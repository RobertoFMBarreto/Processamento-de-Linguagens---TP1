
\begin{itemize}
\item {Proveniência: }
\end{itemize}\documentclass{article}
\usepackage[portuguese]{babel}
\title{S}
\begin{document}
Gênero de plantas australianas.
\section{Secretamente}
\begin{itemize}
\item {Grp. gram.:adv.}
\end{itemize}
De modo secreto; em segrêdo; ás escondidas; mysteriosamente.
\section{Secretaría}
\begin{itemize}
\item {Grp. gram.:f.}
\end{itemize}
Casa, onde se faz o expediente, relativo a uma associação, e especialmente a qualquer serviço público.
Repartição pública.
Ministério.
(De \textunderscore secreto\textunderscore ).
\section{Secretária}
\begin{itemize}
\item {Grp. gram.:f.}
\end{itemize}
Mulhér, que exerce as funcções de secretário.
Mulhér, que guarda segredos de outrem.
Mesa, própria para se escrever sobre ella, e ordináriamente com gavetas para se guardarem papéis ou documentos de importância.
(Lat. \textunderscore secretaria\textunderscore ).
\section{Secretariado}
\begin{itemize}
\item {Grp. gram.:m.}
\end{itemize}
Cargo ou dignidade de secretário.
Lugar, onde o secretário exerce as suas funcções.
Tempo, que dura o exercício dessas funcções.
(De \textunderscore secretário\textunderscore ).
\section{Secretariamente}
\begin{itemize}
\item {Grp. gram.:adv.}
\end{itemize}
\begin{itemize}
\item {Utilização:Ant.}
\end{itemize}
Em segrêdo, secretamente.
(De \textunderscore secretário\textunderscore ).
\section{Secretariar}
\begin{itemize}
\item {Grp. gram.:v. t.}
\end{itemize}
\begin{itemize}
\item {Grp. gram.:V. t.}
\end{itemize}
Exercer as funcções de secretário.
Sêr secretário de, ou exercer as funcções de secretário, junto de (um presidente de assembleia): \textunderscore abriu a sessão o presidente, secretariado por F.\textunderscore 
\section{Secretário}
\begin{itemize}
\item {Grp. gram.:m.}
\end{itemize}
\begin{itemize}
\item {Utilização:Bras}
\end{itemize}
\begin{itemize}
\item {Grp. gram.:Adj.}
\end{itemize}
\begin{itemize}
\item {Utilização:Ant.}
\end{itemize}
Aquelle que escreve as actas de uma assembleia.
Indivíduo, que escreve a correspondência de qualquer pessôa ou corporação, especialmente de personagens elevadas ou de funccionários superiores.
Aquelle que guarda segrêdos de alguém.
Livro, que contém modelos de cartas.
Ministro.
O mesmo que \textunderscore secretária\textunderscore , móvel. Cf. Filinto, XXI, 169.
Indivíduo que o cocheiro trazia na boleia e que era encarregado de offerecer o coche ao transeunte, que delle precisava, em-quanto a polícia fluminense não prohibiu esses agentes. Cf. \textunderscore Jorn.-do-Comm.\textunderscore , do Rio, de 23-VII-900.
O mesmo que \textunderscore secreto\textunderscore .
(Lat. \textunderscore secretarius\textunderscore ).
\section{Secreto}
\begin{itemize}
\item {Grp. gram.:adj.}
\end{itemize}
\begin{itemize}
\item {Grp. gram.:Adv.}
\end{itemize}
\begin{itemize}
\item {Grp. gram.:M.}
\end{itemize}
Afastado, solitário.
Desviado do conhecimento público.
Ignorado.
Encoberto; occulto.
Que tem discricção.
Que se disfarça.
Que se não revela.
Que está em segrêdo.
Íntimo.
O mesmo que \textunderscore secretamente\textunderscore .
O mesmo que \textunderscore segrêdo\textunderscore .
(Lat. \textunderscore secretus\textunderscore ).
\section{Secretor}
\begin{itemize}
\item {Grp. gram.:adj.}
\end{itemize}
Que segrega; em que se dão secreções.
(De \textunderscore secreto\textunderscore ).
\section{Secretório}
\begin{itemize}
\item {Grp. gram.:adj.}
\end{itemize}
\begin{itemize}
\item {Utilização:Des.}
\end{itemize}
Que segrega; em que se dão secreções.
(De \textunderscore secreto\textunderscore ).
\section{Secricri}
\begin{itemize}
\item {Grp. gram.:m.}
\end{itemize}
Ave brasileira, cujo canto imita o seu nome. Cf. \textunderscore Jorn.-do-Comm.\textunderscore , do Rio, de 24-X-901.
\section{Secta}
\begin{itemize}
\item {Grp. gram.:f.}
\end{itemize}
(Fórma ant. de seita).
Cf. Barros, \textunderscore Déc.\textunderscore  II, l. X, c. 6.
\section{Sectário}
\begin{itemize}
\item {Grp. gram.:adj.}
\end{itemize}
\begin{itemize}
\item {Grp. gram.:M.}
\end{itemize}
\begin{itemize}
\item {Utilização:Fig.}
\end{itemize}
Relativo a seita.
Membro de uma seita.
Prosélyto, partidário.
Partidário ferrenho; amouco.
(Lat. \textunderscore sectarius\textunderscore ).
\section{Sectarismo}
\begin{itemize}
\item {Grp. gram.:m.}
\end{itemize}
O mesmo que \textunderscore partidarismo\textunderscore .
(De \textunderscore sectário\textunderscore ).
\section{Séctil}
\begin{itemize}
\item {Grp. gram.:adj.}
\end{itemize}
Que se póde cortar.
(Lat. \textunderscore sectilis\textunderscore ).
\section{Sector}
\begin{itemize}
\item {Grp. gram.:m.}
\end{itemize}
\begin{itemize}
\item {Utilização:Mathem.}
\end{itemize}
Superfície de um círculo, comprehendida entre um arco e os dois raios que tocam nas extremidades dêsse arco.
Porção de superfície plana, entre duas rectas que se cortam e um arco de curva.
Instrumento astronómico, que conta de um arco de 20 a 30^o e um óculo.
Parte de um recinto fortificado, posta sob o commando de um official.
(Lat. \textunderscore sector\textunderscore ).
\section{Sectório}
\begin{itemize}
\item {Grp. gram.:adj.}
\end{itemize}
O mesmo que \textunderscore incisivo\textunderscore , (falando-se dos dentes).
(Lat. \textunderscore sectorius\textunderscore ).
\section{Sectura}
\begin{itemize}
\item {Grp. gram.:f.}
\end{itemize}
Acto de retalhar substâncias medicinaes, com instrumentos cortantes.
(Lat. \textunderscore sectura\textunderscore ).
\section{Secular}
\begin{itemize}
\item {Grp. gram.:adj.}
\end{itemize}
\begin{itemize}
\item {Grp. gram.:M.}
\end{itemize}
\begin{itemize}
\item {Proveniência:(Lat. \textunderscore saecularis\textunderscore )}
\end{itemize}
Que se faz do século a século.
Que existe há séculos.
Relativo a século.
Que é muito antigo.
Relativo aos leigos; que vive no século ou sem votos monásticos.
Mundano.
Temporal.
Civil.
Leigo.
Aquelle que não faz parte de Ordens religiosas.
\section{S}
\begin{itemize}
\item {fónica:ésse}
\end{itemize}
\begin{itemize}
\item {Grp. gram.:m.}
\end{itemize}
Décima nona letra do alphabeto português.
Abreviatura de \textunderscore Sul\textunderscore , \textunderscore Santo\textunderscore , \textunderscore sua\textunderscore , etc.
Como letra numeral, valia antigamente 7 ou 70 e, com um til, 70$000.
Como sinal musical, valia \textunderscore sursum\textunderscore  e indicava que o canto devia subir.
\section{S. A.}
Abrev. de \textunderscore Sua Alteza\textunderscore .
\section{Sa}
\begin{itemize}
\item {fónica:sâ}
\end{itemize}
\begin{itemize}
\item {Grp. gram.:pron. f.}
\end{itemize}
\begin{itemize}
\item {Utilização:Ant.}
\end{itemize}
O mesmo que \textunderscore sua\textunderscore .
\section{Saa}
\begin{itemize}
\item {Grp. gram.:f.}
\end{itemize}
Antiga medida oriental, correspondente ao módio romano.
\section{Saamona}
\begin{itemize}
\item {Grp. gram.:f.}
\end{itemize}
Árvore americana, cujos frutos têm a apparência de ervilhas vermelhas.
\section{Saar}
\begin{itemize}
\item {Grp. gram.:v. t.  e  i.}
\end{itemize}
\begin{itemize}
\item {Utilização:Ant.}
\end{itemize}
\begin{itemize}
\item {Proveniência:(Do lat. \textunderscore sanare\textunderscore )}
\end{itemize}
O mesmo que \textunderscore sarar\textunderscore . Cf. Frei Fortun., \textunderscore Inéditos\textunderscore , 313.
\section{Saba}
\begin{itemize}
\item {Grp. gram.:f.}
\end{itemize}
Bilha para malufo, na Lunda.
\section{Sabácia}
\begin{itemize}
\item {Grp. gram.:f.}
\end{itemize}
Gênero de plantas gencianáceas.
O mesmo que \textunderscore sabázia\textunderscore ?
\section{Sabacu}
\begin{itemize}
\item {Grp. gram.:m.}
\end{itemize}
\begin{itemize}
\item {Utilização:Bras}
\end{itemize}
Ave que vive nos paúes.
\section{Sabacuim}
\begin{itemize}
\item {Grp. gram.:m.}
\end{itemize}
\begin{itemize}
\item {Utilização:Bras}
\end{itemize}
Ave; o mesmo que \textunderscore sabacu\textunderscore ?
\section{Sabaio}
\begin{itemize}
\item {Grp. gram.:m.}
\end{itemize}
Designação do governador de Gôa, antes do domínio português.
\section{Sabaísmo}
\begin{itemize}
\item {Grp. gram.:m.}
\end{itemize}
O mesmo que \textunderscore sabeísmo\textunderscore .
\section{Sabajo}
\begin{itemize}
\item {Grp. gram.:m.}
\end{itemize}
Coisa endemoninhada; diabrura; artes do diabo.
\section{Sabajóia}
\begin{itemize}
\item {Grp. gram.:f.}
\end{itemize}
\begin{itemize}
\item {Utilização:T. de Melgaço}
\end{itemize}
Coisa endemoninhada; diabrura; artes do diabo.
\section{Sabal}
\begin{itemize}
\item {Grp. gram.:m.}
\end{itemize}
Espécie de palmeira.
\section{Sabalíneas}
\begin{itemize}
\item {Grp. gram.:f. pl.}
\end{itemize}
\begin{itemize}
\item {Utilização:Bot.}
\end{itemize}
\begin{itemize}
\item {Proveniência:(De \textunderscore sabal\textunderscore )}
\end{itemize}
Secção da fam. das palmeiras.
\section{Sabandijo}
\begin{itemize}
\item {Grp. gram.:m.}
\end{itemize}
\begin{itemize}
\item {Utilização:Prov.}
\end{itemize}
\begin{itemize}
\item {Utilização:trasm.}
\end{itemize}
O gatilho da espingarda.
(Cp. cast. \textunderscore sabandija\textunderscore )
\section{Sabanilha}
\begin{itemize}
\item {Grp. gram.:f.}
\end{itemize}
\begin{itemize}
\item {Utilização:Prov.}
\end{itemize}
\begin{itemize}
\item {Utilização:trasm.}
\end{itemize}
Espécie de toalha, sôbre que se peneira o pão, de cima das varilhas. (Colhido em Bragança)
(Cast. \textunderscore sabanilla\textunderscore )
\section{Sabão}
\begin{itemize}
\item {Grp. gram.:m.}
\end{itemize}
\begin{itemize}
\item {Utilização:Fam.}
\end{itemize}
\begin{itemize}
\item {Proveniência:(Do lat. \textunderscore sapo\textunderscore , \textunderscore saponis\textunderscore )}
\end{itemize}
Composição, que resulta de acção da potassa e da soda em corpos gordurosos e serve para lavagem.
Censura, remoque; lembrete.
Árvore brasileira e de San-Thomé, (\textunderscore sapindus saponaria\textunderscore ).
O mesmo que \textunderscore mandrião\textunderscore , ave.
\section{Sabão}
\begin{itemize}
\item {Grp. gram.:m.}
\end{itemize}
\begin{itemize}
\item {Utilização:Burl.}
\end{itemize}
Grande sábio. Cf. Castilho, \textunderscore Sabichonas\textunderscore , 284.
\section{Sabasto}
\begin{itemize}
\item {Grp. gram.:m.}
\end{itemize}
\begin{itemize}
\item {Utilização:Ant.}
\end{itemize}
Tira de pano, sôbre uma peça de vestuário, de côr differente.
\section{Sabastro}
\begin{itemize}
\item {Grp. gram.:m.}
\end{itemize}
O mesmo que \textunderscore sabasto\textunderscore . Cf. Sousa, \textunderscore Vida do Arceb.\textunderscore , III, 257.
\section{Sabatados}
\begin{itemize}
\item {Grp. gram.:m. pl.}
\end{itemize}
Herejes espanhóes, que usavam uma corôa no calçado. Cf. S. R. Viterbo, \textunderscore Elucidário\textunderscore .
(Provavelmente, corrupção de \textunderscore sapatados\textunderscore , de \textunderscore sapato\textunderscore )
\section{Sabável}
\begin{itemize}
\item {Grp. gram.:adj.}
\end{itemize}
\begin{itemize}
\item {Utilização:Bras}
\end{itemize}
Agradável ao paladar; saboroso; gostoso.
(Cp. \textunderscore sabor\textunderscore )
\section{Sabázia}
\begin{itemize}
\item {Grp. gram.:f.}
\end{itemize}
\begin{itemize}
\item {Grp. gram.:Pl.}
\end{itemize}
\begin{itemize}
\item {Proveniência:(Do gr. \textunderscore Sabazios\textunderscore , sobrenome de Baccho)}
\end{itemize}
Planta herbácea da América tropical, da fam. das compostas.
Antigas festas, em honra de Baccho.
\section{Sabadeador}
\begin{itemize}
\item {Grp. gram.:m.  e  adj.}
\end{itemize}
O que sabadeia.
\section{Sabadear}
\begin{itemize}
\item {Grp. gram.:v. i.}
\end{itemize}
Não trabalhar ao Sábado, guardar o Sábado como os Judeus.
\section{Sábado}
\begin{itemize}
\item {Grp. gram.:m.}
\end{itemize}
\begin{itemize}
\item {Proveniência:(Do lat. \textunderscore sabbatum\textunderscore )}
\end{itemize}
Sétimo e último dia da semana entre os Cristãos.
Dia de descanso ou descanso religioso, entre os Judeus.
\section{Sabático}
\begin{itemize}
\item {Grp. gram.:adj.}
\end{itemize}
\begin{itemize}
\item {Proveniência:(Lat. \textunderscore sabbaticus\textunderscore )}
\end{itemize}
Relativo ao Sábado.
\section{Sabatina}
\begin{itemize}
\item {Grp. gram.:f.}
\end{itemize}
\begin{itemize}
\item {Proveniência:(Do lat. \textunderscore sabbatum\textunderscore )}
\end{itemize}
Repetição, feita no Sábado, das matérias escolares, prelecionadas durante a semana.
Recapitulação de lições.
Reza, própria de Sábado.
Tese, que os estudantes de Filosofia sustentavam no fim do primeiro ano do seu curso.
\section{Sabatinar}
\begin{itemize}
\item {Grp. gram.:v. t.  e  i.}
\end{itemize}
\begin{itemize}
\item {Proveniência:(De \textunderscore sabatina\textunderscore )}
\end{itemize}
Discutir miudamente, cavilosamente.
\section{Sabatineiro}
\begin{itemize}
\item {Grp. gram.:adj.}
\end{itemize}
Relativo a sabatina; próprio de sabatina:«\textunderscore ...as disputas sabatineiras...\textunderscore »Sousa Martins, \textunderscore Nosographia\textunderscore .
\section{Sabatino}
\begin{itemize}
\item {Grp. gram.:adj.}
\end{itemize}
\begin{itemize}
\item {Proveniência:(Do lat. \textunderscore sabbatum\textunderscore )}
\end{itemize}
Relativo á sabatina; sabático.
\section{Sabatismo}
\begin{itemize}
\item {Grp. gram.:m.}
\end{itemize}
Prática de sabadear.
(Cp. \textunderscore sabatizar\textunderscore )
\section{Sabatizar}
\begin{itemize}
\item {Grp. gram.:v. i.}
\end{itemize}
\begin{itemize}
\item {Proveniência:(Lat. \textunderscore sabbatizare\textunderscore )}
\end{itemize}
O mesmo que \textunderscore sabadear\textunderscore .
\section{Sabbadeador}
\begin{itemize}
\item {Grp. gram.:m.  e  adj.}
\end{itemize}
O que sabbadeia.
\section{Sabbadear}
\begin{itemize}
\item {Grp. gram.:v. i.}
\end{itemize}
Não trabalhar ao Sábbado, guardar o Sábbado como os Judeus.
\section{Sábbado}
\begin{itemize}
\item {Grp. gram.:m.}
\end{itemize}
\begin{itemize}
\item {Proveniência:(Do lat. \textunderscore sabbatum\textunderscore )}
\end{itemize}
Sétimo e último dia da semana entre os Christãos.
Dia de descanso ou descanso religioso, entre os Judeus.
\section{Sabbático}
\begin{itemize}
\item {Grp. gram.:adj.}
\end{itemize}
\begin{itemize}
\item {Proveniência:(Lat. \textunderscore sabbaticus\textunderscore )}
\end{itemize}
Relativo ao Sábbado.
\section{Sabbatina}
\begin{itemize}
\item {Grp. gram.:f.}
\end{itemize}
\begin{itemize}
\item {Proveniência:(Do lat. \textunderscore sabbatum\textunderscore )}
\end{itemize}
Repetição, feita no Sábbado, das matérias escolares, preleccionadas durante a semana.
Recapitulação de lições.
Reza, própria de Sábbado.
These, que os estudantes de Philosophia sustentavam no fim do primeiro anno do seu curso.
\section{Sabbatinar}
\begin{itemize}
\item {Grp. gram.:v. t.  e  i.}
\end{itemize}
\begin{itemize}
\item {Proveniência:(De \textunderscore sabbatina\textunderscore )}
\end{itemize}
Discutir miudamente, cavilosamente.
\section{Sabbatineiro}
\begin{itemize}
\item {Grp. gram.:adj.}
\end{itemize}
Relativo a sabbatina; próprio de sabbatina:«\textunderscore ...as disputas sabbatineiras...\textunderscore »Sousa Martins, \textunderscore Nosographia\textunderscore .
\section{Sabbatino}
\begin{itemize}
\item {Grp. gram.:adj.}
\end{itemize}
\begin{itemize}
\item {Proveniência:(Do lat. \textunderscore sabbatum\textunderscore )}
\end{itemize}
Relativo á sabbatina; sabbático.
\section{Sabbatismo}
\begin{itemize}
\item {Grp. gram.:m.}
\end{itemize}
Prática de sabbadear.
(Cp. \textunderscore sabbatizar\textunderscore )
\section{Sabbatizar}
\begin{itemize}
\item {Grp. gram.:v. i.}
\end{itemize}
\begin{itemize}
\item {Proveniência:(Lat. \textunderscore sabbatizare\textunderscore )}
\end{itemize}
O mesmo que \textunderscore sabbadear\textunderscore .
\section{Sabedor}
\begin{itemize}
\item {Grp. gram.:m.  e  adj.}
\end{itemize}
O que sabe; erudito; sábio.
\section{Sabedoramente}
\begin{itemize}
\item {Grp. gram.:adv.}
\end{itemize}
\begin{itemize}
\item {Proveniência:(De \textunderscore sabedor\textunderscore )}
\end{itemize}
Com sabedoria; com conhecimento de causa.
\section{Sabedoria}
\begin{itemize}
\item {Grp. gram.:f.}
\end{itemize}
\begin{itemize}
\item {Proveniência:(De \textunderscore sabedor\textunderscore )}
\end{itemize}
Grande cópia de conhecimentos.
Qualidade de quem é sabedor.
Conhecimento da verdade.
Sciência, saber.
Prudência.
Rectidão; razão.
\section{Sabedormente}
\begin{itemize}
\item {Grp. gram.:adv.}
\end{itemize}
\begin{itemize}
\item {Utilização:Ant.}
\end{itemize}
O mesmo que \textunderscore sabedoramente\textunderscore . Cf. S. R. Viterbo, \textunderscore Elucidário\textunderscore .
\section{Sabeísmo}
\begin{itemize}
\item {Grp. gram.:m.}
\end{itemize}
Religião dos que adoravam os astros.
Seita christan, eivada de magia, e derivada dos Gnósticos.
(Talvez de \textunderscore Çabi\textunderscore  ou \textunderscore Zabi\textunderscore , n. p. de uma personagem bíblica)
\section{Sabeísta}
\begin{itemize}
\item {Grp. gram.:m.}
\end{itemize}
O mesmo que \textunderscore sabeíta\textunderscore .
\section{Sabeíta}
\begin{itemize}
\item {Grp. gram.:m.}
\end{itemize}
Sectário do sabeísmo.
\section{Sabelianismo}
\begin{itemize}
\item {Grp. gram.:m.}
\end{itemize}
Doutrina herética dos Sabelianos.
\section{Sabelianos}
\begin{itemize}
\item {Grp. gram.:m. pl.}
\end{itemize}
\begin{itemize}
\item {Proveniência:(De \textunderscore Sabéllio\textunderscore , n. p.)}
\end{itemize}
Herejes do século III, que negavam a Trindade e sustentavam que em Deus há só uma pessôa, o Padre, de quem o Filho e o Espírito Santo são meros atributos, emanações ou operações.
\section{Sabélico}
\begin{itemize}
\item {Grp. gram.:adj.}
\end{itemize}
\begin{itemize}
\item {Proveniência:(Lat. \textunderscore sabellicus\textunderscore )}
\end{itemize}
Diz-se de um antigo alphabeto itálico, o alphabeto dos Sabellos, hoje escassamente conhecido, e de que só restam dois monumentos. Cf. Daremberg, \textunderscore Diction. des Antiq.\textunderscore , vb. \textunderscore alphabet\textunderscore .
\section{Sabellianismo}
\begin{itemize}
\item {Grp. gram.:m.}
\end{itemize}
Doutrina herética dos Sabellianos.
\section{Sabellianos}
\begin{itemize}
\item {Grp. gram.:m. pl.}
\end{itemize}
\begin{itemize}
\item {Proveniência:(De \textunderscore Sabéllio\textunderscore , n. p.)}
\end{itemize}
Herejes do século III, que negavam a Trindade e sustentavam que em Deus há só uma pessôa, o Padre, de quem o Filho e o Espírito Santo são meros atributos, emanações ou operações.
\section{Sabéllico}
\begin{itemize}
\item {Grp. gram.:adj.}
\end{itemize}
\begin{itemize}
\item {Proveniência:(Lat. \textunderscore sabellicus\textunderscore )}
\end{itemize}
Diz-se de um antigo alphabeto itálico, o alphabeto dos Sabellos, hoje escassamente conhecido, e de que só restam dois monumentos. Cf. Daremberg, \textunderscore Diction. des Antiq.\textunderscore , vb. \textunderscore alphabet\textunderscore .
\section{Sabellos}
\begin{itemize}
\item {Grp. gram.:m. pl.}
\end{itemize}
\begin{itemize}
\item {Proveniência:(Lat. \textunderscore sabelli\textunderscore )}
\end{itemize}
Designação antiga dos montanheses italianos, que depois se chamaram Sabinos.
\section{Sabelos}
\begin{itemize}
\item {Grp. gram.:m. pl.}
\end{itemize}
\begin{itemize}
\item {Proveniência:(Lat. \textunderscore sabelli\textunderscore )}
\end{itemize}
Designação antiga dos montanheses italianos, que depois se chamaram Sabinos.
\section{Sabena}
\begin{itemize}
\item {Grp. gram.:f.}
\end{itemize}
\begin{itemize}
\item {Utilização:Ant.}
\end{itemize}
\begin{itemize}
\item {Proveniência:(Lat. \textunderscore sabanum\textunderscore )}
\end{itemize}
Coberta da cama.
Lençol.
\section{Sabença}
\begin{itemize}
\item {Grp. gram.:f.}
\end{itemize}
\begin{itemize}
\item {Utilização:Pop.}
\end{itemize}
\begin{itemize}
\item {Proveniência:(De \textunderscore saber\textunderscore )}
\end{itemize}
O mesmo que \textunderscore sabedoria\textunderscore ; erudição. Cf. Camillo, \textunderscore Quéda\textunderscore , 209.
\section{Sabendas}
\begin{itemize}
\item {Grp. gram.:f. pl.}
\end{itemize}
\begin{itemize}
\item {Utilização:Ant.}
\end{itemize}
\begin{itemize}
\item {Proveniência:(De \textunderscore saber\textunderscore )}
\end{itemize}
Caso pensado.
Sabença.
\section{Sabente}
\begin{itemize}
\item {Grp. gram.:adj.}
\end{itemize}
\begin{itemize}
\item {Utilização:Ant.}
\end{itemize}
O mesmo que \textunderscore sabedor\textunderscore . Cf. S. Monteiro, \textunderscore Auto dos Esquecidos\textunderscore .
\section{Saber}
\begin{itemize}
\item {Grp. gram.:v. t.}
\end{itemize}
\begin{itemize}
\item {Grp. gram.:V. i.}
\end{itemize}
\begin{itemize}
\item {Grp. gram.:Loc.}
\end{itemize}
\begin{itemize}
\item {Utilização:fam.}
\end{itemize}
\begin{itemize}
\item {Grp. gram.:Loc.}
\end{itemize}
\begin{itemize}
\item {Utilização:fam.}
\end{itemize}
\begin{itemize}
\item {Grp. gram.:Loc.}
\end{itemize}
\begin{itemize}
\item {Utilização:fam.}
\end{itemize}
\begin{itemize}
\item {Grp. gram.:Loc.}
\end{itemize}
\begin{itemize}
\item {Utilização:fam.}
\end{itemize}
\begin{itemize}
\item {Grp. gram.:M.}
\end{itemize}
\begin{itemize}
\item {Proveniência:(Lat. \textunderscore sapere\textunderscore )}
\end{itemize}
Têr conhecimento de: \textunderscore sabes história\textunderscore .
Comprehender, perceber.
Estar convencido de: \textunderscore sei que me enganas\textunderscore .
Têr meios para.
Reter na memória: \textunderscore saber um conto\textunderscore .
Têr a certeza de.
Têr capacidade ou conhecimentos necessários para: \textunderscore sei nadar\textunderscore .
Têr grande cópia de conhecimentos.
Sêr erudito.
Têr as necessárias habilitações para certo fim.
Estar informado.
Têr sabor; impressionar o sentido do gôsto: \textunderscore há iguarias que sabem bem e outras que sabem mal\textunderscore .
\textunderscore Sabê-la toda\textunderscore , têr conhecimentos ou habilidade para certas coisas.
Tratar dos seus interesses com manha.
\textunderscore Saber as linhas com que se cose\textunderscore , saber o que lhe convém, o que lhe é preciso.
\textunderscore Saber da poda\textunderscore , têr certos conhecimentos especiaes.
\textunderscore Saber o nome aos bois\textunderscore , sêr perito ou prático em certos assumptos.
Sciência.
Erudição.
Experiência.
Sensatez.
\section{Saberecar}
\begin{itemize}
\item {Grp. gram.:v. t.}
\end{itemize}
\begin{itemize}
\item {Utilização:Bras}
\end{itemize}
O mesmo que \textunderscore sapecar\textunderscore ^1.
\section{Saberete}
\begin{itemize}
\item {fónica:berê}
\end{itemize}
\begin{itemize}
\item {Grp. gram.:m.}
\end{itemize}
\begin{itemize}
\item {Utilização:Fam.}
\end{itemize}
\begin{itemize}
\item {Proveniência:(De \textunderscore saber\textunderscore )}
\end{itemize}
Pouco saber.
Conhecimento imperfeito.
Ronha, manha.
\section{Sabeu}
\begin{itemize}
\item {Grp. gram.:adj.}
\end{itemize}
\begin{itemize}
\item {Utilização:Poét.}
\end{itemize}
\begin{itemize}
\item {Grp. gram.:Pl.}
\end{itemize}
\begin{itemize}
\item {Proveniência:(Lat. \textunderscore sabaeus\textunderscore )}
\end{itemize}
Relativo a Sabá.
Povos de Sabá, na Arábia.
\section{Sábia}
\begin{itemize}
\item {Grp. gram.:f.}
\end{itemize}
Gênero de plantas anacardiáceas.
\section{Sabiá}
\begin{itemize}
\item {Grp. gram.:m.}
\end{itemize}
Pássaro dentirostro do Brasil, de canto mui suave.
\section{Sabiabe}
\begin{itemize}
\item {Grp. gram.:m.}
\end{itemize}
\begin{itemize}
\item {Utilização:Ant.}
\end{itemize}
Moéda de cobre em Cambaia.
\section{Sabiàci}
\begin{itemize}
\item {Grp. gram.:m.}
\end{itemize}
Ave brasileira, de côr verde e bico redondo, espécie de papagaio.
\section{Sabiá-cica}
\begin{itemize}
\item {Grp. gram.:m.}
\end{itemize}
O mesmo que \textunderscore sabiàci\textunderscore .
\section{Sabiamente}
\begin{itemize}
\item {Grp. gram.:adv.}
\end{itemize}
\begin{itemize}
\item {Proveniência:(De \textunderscore sábio\textunderscore )}
\end{itemize}
Com sabedoria; com circumspecção; discretamente; prudentemente.
\section{Sabichã}
\begin{itemize}
\item {Grp. gram.:f.}
\end{itemize}
O mesmo que \textunderscore sabichona\textunderscore .
\section{Sabichan}
\begin{itemize}
\item {Grp. gram.:f.}
\end{itemize}
O mesmo que \textunderscore sabichona\textunderscore .
\section{Sabichão}
\begin{itemize}
\item {Grp. gram.:m.  e  adj.}
\end{itemize}
\begin{itemize}
\item {Utilização:irón.}
\end{itemize}
\begin{itemize}
\item {Utilização:Fam.}
\end{itemize}
Grande sábio.
Aquelle que alardeia sabedoria.
\section{Sabichar}
\begin{itemize}
\item {Grp. gram.:v. t.}
\end{itemize}
\begin{itemize}
\item {Utilização:Prov.}
\end{itemize}
\begin{itemize}
\item {Utilização:beir.}
\end{itemize}
\begin{itemize}
\item {Proveniência:(De \textunderscore saber\textunderscore )}
\end{itemize}
Indagar, investigar, procurar saber.
\section{Sabichona}
\begin{itemize}
\item {Grp. gram.:f.  e  adj.}
\end{itemize}
\begin{itemize}
\item {Utilização:irón.}
\end{itemize}
\begin{itemize}
\item {Utilização:Fam.}
\end{itemize}
\begin{itemize}
\item {Proveniência:(De \textunderscore sabichão\textunderscore )}
\end{itemize}
Mulher, que presume de muito sabedora; literata.
\section{Sabichoso}
\begin{itemize}
\item {Grp. gram.:m.  e  adj.}
\end{itemize}
\begin{itemize}
\item {Proveniência:(De \textunderscore sabichão\textunderscore )}
\end{itemize}
Diz-se de quem emprega mal o seu saber.
\section{Sabidamente}
\begin{itemize}
\item {Grp. gram.:adv.}
\end{itemize}
De modo sabido; claramente; notoriamente.
\section{Sabidas}
\begin{itemize}
\item {Grp. gram.:f. pl. Loc. adv.}
\end{itemize}
\begin{itemize}
\item {Grp. gram.:Loc. adv.}
\end{itemize}
\begin{itemize}
\item {Proveniência:(De \textunderscore sabido\textunderscore )}
\end{itemize}
\textunderscore Ás sabidas\textunderscore , ás claras, publicamente.
\textunderscore Ás não sabidas\textunderscore , ás occultas; em segredo:«\textunderscore ...e já este, ás não sabidas daquelle, requeria...\textunderscore »Filinto, \textunderscore D. Man.\textunderscore , II, 9.
\section{Sabido}
\begin{itemize}
\item {Grp. gram.:adj.}
\end{itemize}
\begin{itemize}
\item {Grp. gram.:M. pl.}
\end{itemize}
Que se sabe; que é conhecido: \textunderscore histórias sabidas\textunderscore .
Erudito; sabedor.
Circunspecto.
Astuto; finório.
Emolumentos; ordenados.
\section{Sabidório}
\begin{itemize}
\item {Grp. gram.:m.}
\end{itemize}
\begin{itemize}
\item {Utilização:Fam.}
\end{itemize}
\begin{itemize}
\item {Proveniência:(De \textunderscore sabido\textunderscore )}
\end{itemize}
Homem, que tem presumpção de sabedor ou letrado.
\section{Sabina}
\begin{itemize}
\item {Grp. gram.:f.}
\end{itemize}
Planta cucurbitácea da Índia Portuguesa, (\textunderscore trichosanthes cucumerina\textunderscore , Lin.).
\section{Sabina}
\begin{itemize}
\item {Grp. gram.:f.}
\end{itemize}
\begin{itemize}
\item {Proveniência:(Lat. \textunderscore sabina\textunderscore )}
\end{itemize}
Arbusto conífero, (\textunderscore juniperus sabina\textunderscore ).
\section{Sabina-da-praia}
\begin{itemize}
\item {Grp. gram.:f.}
\end{itemize}
\begin{itemize}
\item {Utilização:Bot.}
\end{itemize}
O mesmo que \textunderscore sabina\textunderscore ^2.
\section{Sabinense}
\begin{itemize}
\item {Grp. gram.:adj.}
\end{itemize}
Diz-se de Cardeal de Santa-Sabina, que em Portugal foi legado do Papa em tempo de Sancho II:«\textunderscore o Cardeal sabinense...\textunderscore »Herculano, \textunderscore Hist. de Port.\textunderscore , II, 302.
\section{Sabino}
\begin{itemize}
\item {Grp. gram.:adj.}
\end{itemize}
Que tem pêlo branco, mesclado de vermelho e preto, (falando-se do cavallo).
\section{Sabino}
\begin{itemize}
\item {Grp. gram.:m.}
\end{itemize}
\begin{itemize}
\item {Grp. gram.:Pl.}
\end{itemize}
\begin{itemize}
\item {Proveniência:(Lat. \textunderscore sabinus\textunderscore )}
\end{itemize}
Dialecto itálico, que se misturou com a língua dos Romanos.
Antigo povo da Itália, que se reuniu com os Latinos para formar um só povo com o nome de Quirites.
\section{Sábio}
\begin{itemize}
\item {Grp. gram.:adj.}
\end{itemize}
\begin{itemize}
\item {Utilização:Fig.}
\end{itemize}
\begin{itemize}
\item {Grp. gram.:M.}
\end{itemize}
\begin{itemize}
\item {Utilização:Prov.}
\end{itemize}
\begin{itemize}
\item {Utilização:trasm.}
\end{itemize}
Que sabe muito.
Que é versado nas sciências; erudito.
Destro, perito.
Prudente.
Homem, que sabe muito; philóshopho; erudito.
O mesmo que \textunderscore feiticeiro\textunderscore .
(Cp. lat. \textunderscore sapidus\textunderscore  e \textunderscore nesapius\textunderscore )
\section{Sabitu}
\begin{itemize}
\item {Grp. gram.:m.}
\end{itemize}
\begin{itemize}
\item {Utilização:Bras. de San-Paulo}
\end{itemize}
O mesmo que \textunderscore saúba\textunderscore .
\section{Sabível}
\begin{itemize}
\item {Grp. gram.:adj.}
\end{itemize}
\begin{itemize}
\item {Utilização:P. us.}
\end{itemize}
\begin{itemize}
\item {Proveniência:(De \textunderscore saber\textunderscore )}
\end{itemize}
Que se póde saber. Cf. Castilho, \textunderscore Livr. Cliss.\textunderscore , VII, 120.
\section{Sable}
\begin{itemize}
\item {Grp. gram.:m.}
\end{itemize}
\begin{itemize}
\item {Utilização:Heráld.}
\end{itemize}
\begin{itemize}
\item {Proveniência:(Fr. \textunderscore sable\textunderscore )}
\end{itemize}
Côr verde, nos brasões.
\section{Saboaria}
\begin{itemize}
\item {Grp. gram.:f.}
\end{itemize}
Lugar, onde se fabríca, se vende ou se guarda sabão.
\section{Saboeira}
\begin{itemize}
\item {Grp. gram.:f.}
\end{itemize}
\begin{itemize}
\item {Utilização:Bot.}
\end{itemize}
\begin{itemize}
\item {Proveniência:(De \textunderscore saboeiro\textunderscore )}
\end{itemize}
Vendedora de sabão.
Saboneteira.
Planta sapindácea, (\textunderscore sapindus saponaria\textunderscore ); saponária.
\section{Saboeiro}
\begin{itemize}
\item {Grp. gram.:m.}
\end{itemize}
\begin{itemize}
\item {Utilização:Prov.}
\end{itemize}
\begin{itemize}
\item {Utilização:trasm.}
\end{itemize}
\begin{itemize}
\item {Utilização:Bot.}
\end{itemize}
\begin{itemize}
\item {Proveniência:(Lat. \textunderscore saponarius\textunderscore )}
\end{itemize}
Fabricante ou vendedor de sabão.
O mesmo que \textunderscore saboneteira\textunderscore .
O mesmo que \textunderscore saponária\textunderscore .
Homem pouco asseado no vestir.
Designação genérica de várias plantas sapindáceas do Brasil.
\section{Saboga}
\begin{itemize}
\item {Grp. gram.:f.}
\end{itemize}
\begin{itemize}
\item {Proveniência:(Do ár. \textunderscore çaboga\textunderscore )}
\end{itemize}
O mesmo que \textunderscore sàvelha\textunderscore .
\section{Saboiana}
\begin{itemize}
\item {Grp. gram.:f.}
\end{itemize}
\begin{itemize}
\item {Proveniência:(De \textunderscore Sabóia\textunderscore , n. p.)}
\end{itemize}
Antigo vestuário de mulhér, espécie de vasquinha.
\section{Saboiano}
\begin{itemize}
\item {Grp. gram.:adj.}
\end{itemize}
\begin{itemize}
\item {Grp. gram.:M.}
\end{itemize}
Relativo a Sabóia.
Habitante da Sabóia.
\section{Saboneira}
\begin{itemize}
\item {Grp. gram.:f.}
\end{itemize}
\begin{itemize}
\item {Utilização:Bras}
\end{itemize}
O mesmo que \textunderscore saboneteira\textunderscore .
\section{Sabonetada}
\begin{itemize}
\item {Grp. gram.:f.}
\end{itemize}
\begin{itemize}
\item {Utilização:Prov.}
\end{itemize}
\begin{itemize}
\item {Proveniência:(De \textunderscore sabonete\textunderscore )}
\end{itemize}
Descompostura.
\section{Sabonete}
\begin{itemize}
\item {fónica:nê}
\end{itemize}
\begin{itemize}
\item {Grp. gram.:m.}
\end{itemize}
\begin{itemize}
\item {Utilização:Fam.}
\end{itemize}
\begin{itemize}
\item {Utilização:Bot.}
\end{itemize}
\begin{itemize}
\item {Utilização:Pop.}
\end{itemize}
\begin{itemize}
\item {Proveniência:(De \textunderscore sabão\textunderscore )}
\end{itemize}
Pedaço de sabão fino, mais ou menos aromatizado e em fórma de pastilha ou bola.
Relógio pequeno de algibeira.
Saponária.
Reprehensão, lembrete.
\section{Saboneteira}
\begin{itemize}
\item {Grp. gram.:f.}
\end{itemize}
Lugar ou caixinha, para conter o sabonete.
\section{Sabongo}
\begin{itemize}
\item {Grp. gram.:m.}
\end{itemize}
\begin{itemize}
\item {Utilização:Bras}
\end{itemize}
Doce de côco, ou de bagaço de côco, com mel de cana.
\section{Sabor}
\begin{itemize}
\item {Grp. gram.:m.}
\end{itemize}
\begin{itemize}
\item {Utilização:Fig.}
\end{itemize}
\begin{itemize}
\item {Proveniência:(Lat. \textunderscore sapor\textunderscore )}
\end{itemize}
Gôsto, impressão produzida no paladar por certas substâncias.
Propriedade, que tem certas substâncias, de impressionar o órgão do gôsto; saibo.
Qualidade; índole: \textunderscore estilo, de sabor clássico\textunderscore .
Jovialidade.
Natureza.
Fórma.
Vontade, capricho, talante: \textunderscore proceder a seu sabor\textunderscore .
\section{Saborá}
\begin{itemize}
\item {Grp. gram.:m.}
\end{itemize}
\begin{itemize}
\item {Utilização:Bras. do N}
\end{itemize}
Substância amarela, que se encontra misturada com o mel de abelhas.
(Cp. \textunderscore borá\textunderscore )
\section{Saborear}
\begin{itemize}
\item {Grp. gram.:v. t.}
\end{itemize}
\begin{itemize}
\item {Utilização:Fig.}
\end{itemize}
\begin{itemize}
\item {Utilização:Irón.}
\end{itemize}
\begin{itemize}
\item {Grp. gram.:V. p.}
\end{itemize}
\begin{itemize}
\item {Utilização:Fig.}
\end{itemize}
\begin{itemize}
\item {Proveniência:(De \textunderscore sabor\textunderscore )}
\end{itemize}
Tornar saboroso, dar sabor ou bom sabor a.
Comer de vagar, com gôsto; provar com prazer, comendo ou bebendo.
Apreciar o sabor de.
Gozar lentamente, vuluptuosamente.
Comprazer-se em: \textunderscore saborear uma leitura\textunderscore .
Soffrer lentamente.
Sentir prazer.
Deleitar-se, comendo ou bebendo.
Deleitar-se.
Ficar gostando de alguma coisa, appetecendo-a sempre.
\section{Saborido}
\begin{itemize}
\item {Grp. gram.:adj.}
\end{itemize}
O mesmo que \textunderscore saboroso\textunderscore .
\section{Saborosamente}
\begin{itemize}
\item {Grp. gram.:adv.}
\end{itemize}
De modo saboroso.
Agradavelmente; com deleite.
\section{Saboroso}
\begin{itemize}
\item {Grp. gram.:adj.}
\end{itemize}
\begin{itemize}
\item {Utilização:Fig.}
\end{itemize}
\begin{itemize}
\item {Utilização:Ant.}
\end{itemize}
Que tem sabor ou bom saibo.
Deleitoso, agradável.
Affável, amável. Cf. \textunderscore Port. Mon. Hist.\textunderscore , \textunderscore Script.\textunderscore , 272.
\section{Saborra}
\begin{itemize}
\item {Grp. gram.:f.}
\end{itemize}
O mesmo que \textunderscore saburra\textunderscore .
\section{Saborreiro}
\begin{itemize}
\item {Grp. gram.:m.}
\end{itemize}
\begin{itemize}
\item {Utilização:Prov.}
\end{itemize}
\begin{itemize}
\item {Utilização:trasm.}
\end{itemize}
Calor abafado de um dia ennevoado de verão.
(Cp. \textunderscore saborra\textunderscore )
\section{Sabotagem}
\begin{itemize}
\item {Grp. gram.:f.}
\end{itemize}
\begin{itemize}
\item {Utilização:Ext.}
\end{itemize}
Acto ou effeito de sabotar.
Destroços, feitos em carris de ferro por grevistas ou anarchistas.
\section{Sabotar}
\begin{itemize}
\item {Grp. gram.:v. t.}
\end{itemize}
\begin{itemize}
\item {Proveniência:(Fr. \textunderscore saboter\textunderscore )}
\end{itemize}
Abrir entalhe em (travessas de linha ferrea), para que o carril fique um tanto inclinado.
\section{Sabra}
\begin{itemize}
\item {Grp. gram.:f.}
\end{itemize}
Variedade de uva branca, temporan.
\section{Sabraínho}
\begin{itemize}
\item {Grp. gram.:m.}
\end{itemize}
\begin{itemize}
\item {Proveniência:(De \textunderscore sabra\textunderscore )}
\end{itemize}
Uva tinta, de inferior qualidade.
\section{Sabra-molle}
\begin{itemize}
\item {Grp. gram.:f.}
\end{itemize}
Uva da Estremadura, tinta-sobreirinha.
\section{Sabras}
\begin{itemize}
\item {Grp. gram.:f. pl.}
\end{itemize}
Espécie de uva branca, provavelmente o mesmo que \textunderscore sabra\textunderscore .
\section{Sabre}
\begin{itemize}
\item {Grp. gram.:m.}
\end{itemize}
\begin{itemize}
\item {Proveniência:(Fr. \textunderscore sabre\textunderscore )}
\end{itemize}
Espada curta, terçado.
\section{Sabre-baioneta}
\begin{itemize}
\item {Grp. gram.:m.}
\end{itemize}
Pequeno sabre, que se adapta á bôca das espingardas, como as baionetas.
\section{Sabrecar}
\begin{itemize}
\item {Grp. gram.:v. t.}
\end{itemize}
\begin{itemize}
\item {Utilização:Bras}
\end{itemize}
O mesmo que \textunderscore sapecar\textunderscore ^2.
\section{Sàbrista}
\begin{itemize}
\item {Grp. gram.:m.}
\end{itemize}
Aquelle que sabe esgrimir com o sabre.
\section{Sabroso}
\begin{itemize}
\item {Grp. gram.:adj.}
\end{itemize}
\begin{itemize}
\item {Utilização:Ant.}
\end{itemize}
O mesmo que \textunderscore saboroso\textunderscore . Cf. \textunderscore Eufrosina\textunderscore , 169.
\section{Sabudo}
\begin{itemize}
\item {Grp. gram.:adj.}
\end{itemize}
\begin{itemize}
\item {Utilização:Ant.}
\end{itemize}
O mesmo que \textunderscore sabido\textunderscore .
\section{Sabugal}
\begin{itemize}
\item {Grp. gram.:m.}
\end{itemize}
\begin{itemize}
\item {Grp. gram.:F.  e  adj.}
\end{itemize}
Terreno, onde crescem sabugueiros.
Espécie de uva, uva de cão.
\section{Sabugar}
\begin{itemize}
\item {Grp. gram.:v. t.}
\end{itemize}
\begin{itemize}
\item {Utilização:Bras. do N}
\end{itemize}
Bater em; sovar; espancar.
\section{Sabugo}
\begin{itemize}
\item {Grp. gram.:m.}
\end{itemize}
\begin{itemize}
\item {Utilização:Prov.}
\end{itemize}
\begin{itemize}
\item {Utilização:trasm.}
\end{itemize}
\begin{itemize}
\item {Proveniência:(Do lat. \textunderscore sambucus\textunderscore )}
\end{itemize}
Miolo do sabugueiro.
Sabugueiro.
Parte interior e pouco dura dos cornos.
Parte da cauda dos animaes, donde nascem as sedas.
Parte do dedo a que adhere a unha.
Parte interna da espiga de milho.
O mesmo que \textunderscore chifre\textunderscore .
\section{Sabugueirinho}
\begin{itemize}
\item {Grp. gram.:m.}
\end{itemize}
Planta caprifoliácea, o mesmo que \textunderscore engos\textunderscore . Cf. P. Coutinho, \textunderscore Flora\textunderscore , 586.
\section{Sabugueiro}
\begin{itemize}
\item {Grp. gram.:m.}
\end{itemize}
\begin{itemize}
\item {Proveniência:(De \textunderscore sabugo\textunderscore )}
\end{itemize}
Arbusto caprifoliáceo.
\textunderscore Vinho de sabugueiro\textunderscore , infusão de entrecasca de sabugueiro em vinho branco, usada contra a hydropisia.
\section{Sabugueiro-de-água}
\begin{itemize}
\item {Grp. gram.:m.}
\end{itemize}
O mesmo que [[novelos|novello]].
\section{Sabujar}
\begin{itemize}
\item {Grp. gram.:v. t.}
\end{itemize}
\begin{itemize}
\item {Proveniência:(De \textunderscore sabujo\textunderscore )}
\end{itemize}
Adular, bajular. Cf. Eça. \textunderscore P. Basílio\textunderscore , 95.
\section{Sabujice}
\begin{itemize}
\item {Grp. gram.:f.}
\end{itemize}
Qualidade do que é sabujo ou servil.
\section{Sabujo}
\begin{itemize}
\item {Grp. gram.:m.}
\end{itemize}
\begin{itemize}
\item {Utilização:Fig.}
\end{itemize}
Cão de caça grossa.
Homem servil, bajulador; sevandija.
(B. lat. \textunderscore segusius\textunderscore )
\section{Sabuloso}
\begin{itemize}
\item {Grp. gram.:adj.}
\end{itemize}
\begin{itemize}
\item {Proveniência:(Lat. \textunderscore sabulosus\textunderscore )}
\end{itemize}
Que tem areias, areento.
\section{Saburra}
\begin{itemize}
\item {Grp. gram.:f.}
\end{itemize}
\begin{itemize}
\item {Proveniência:(Lat. \textunderscore saburra\textunderscore )}
\end{itemize}
Matérias, que se suppunha accumularem-se no estômago, em consequência das más digestões.
Crôsta, geralmente esbranquiçada, que cobre a parte superior da língua durante certas doenças.
\section{Saburrar}
\begin{itemize}
\item {Grp. gram.:v. t.}
\end{itemize}
\begin{itemize}
\item {Proveniência:(De \textunderscore saburra\textunderscore )}
\end{itemize}
Lastrar (um navio).
\section{Saburrento}
\begin{itemize}
\item {Grp. gram.:adj.}
\end{itemize}
O mesmo que \textunderscore saburroso\textunderscore .
\section{Saburrinha}
\begin{itemize}
\item {Grp. gram.:f.}
\end{itemize}
\begin{itemize}
\item {Proveniência:(De \textunderscore saburra\textunderscore )}
\end{itemize}
Uma das espécies de limo, que apparecem nas salinas.
\section{Saburrosidade}
\begin{itemize}
\item {Grp. gram.:f.}
\end{itemize}
Estado de saburroso.
\section{Saburroso}
\begin{itemize}
\item {Grp. gram.:adj.}
\end{itemize}
Que tem saburra.
\section{Saca}
\begin{itemize}
\item {Grp. gram.:f.}
\end{itemize}
\begin{itemize}
\item {Utilização:Ant.}
\end{itemize}
Acto de sacar; sacadela.
Movimento da onda para a praia.
Saída; licença para exportar ou para tirar de um lugar.
\section{Saca}
\begin{itemize}
\item {Grp. gram.:f.}
\end{itemize}
\begin{itemize}
\item {Proveniência:(De \textunderscore saco\textunderscore ^2)}
\end{itemize}
Grande saco.
\section{Saca}
\begin{itemize}
\item {Grp. gram.:f.}
\end{itemize}
Gato selvagem de Madagáscar.
\section{Saca...}
\begin{itemize}
\item {Grp. gram.:pref.}
\end{itemize}
\begin{itemize}
\item {Proveniência:(De \textunderscore sacar\textunderscore )}
\end{itemize}
(designativo de \textunderscore sacar\textunderscore  ou \textunderscore tirar\textunderscore )
\section{Saca-balas}
\begin{itemize}
\item {Grp. gram.:m.}
\end{itemize}
Instrumento, para extrahir balas.
\section{Saca-bocado}
\begin{itemize}
\item {Grp. gram.:m.}
\end{itemize}
\begin{itemize}
\item {Utilização:Serralh.}
\end{itemize}
\begin{itemize}
\item {Grp. gram.:Adj.}
\end{itemize}
\begin{itemize}
\item {Utilização:Ant.}
\end{itemize}
Instrumento, para fazer buracos em pano ou coiro.
Instrumento, para desbastar.
Ferramenta, para cortar ferro.
Máquina, para cortar as lâminas, de que se fazem os discos das moédas.
Dizia-se do pano com orifícios feitos a saca-bocado.
\section{Saca-buxa}
\begin{itemize}
\item {Grp. gram.:m.}
\end{itemize}
O mesmo que \textunderscore saca-trapo\textunderscore .
\section{Saca-buxa}
\begin{itemize}
\item {Grp. gram.:f.}
\end{itemize}
\begin{itemize}
\item {Utilização:Náut.}
\end{itemize}
\begin{itemize}
\item {Proveniência:(Do fr. \textunderscore saquebute\textunderscore )}
\end{itemize}
Instrumento antigo, semelhante a uma trompa.
Nome ant. do trombone.
Registo nos órgãos antigos.
Espécie de bomba.
\section{Sacacas}
\begin{itemize}
\item {Grp. gram.:m. pl.}
\end{itemize}
\begin{itemize}
\item {Utilização:Bras}
\end{itemize}
Tríbo de aborígenes do Pará.
\section{Sacada}
\begin{itemize}
\item {Grp. gram.:f.}
\end{itemize}
\begin{itemize}
\item {Proveniência:(De \textunderscore sacar\textunderscore )}
\end{itemize}
O mesmo que \textunderscore sacadela\textunderscore .
Exportação.
Imposto, que pagavam antigamente os exportadores.
Construcção, que resalta da face de uma parede ou do nível de outra construcção.
Balcão de uma janela, que resái da parede.
Sacão, galão.
\section{Sacada}
\begin{itemize}
\item {Grp. gram.:f.}
\end{itemize}
\begin{itemize}
\item {Proveniência:(De \textunderscore saco\textunderscore )}
\end{itemize}
Aquillo que um saco póde conter.
Rêde de arrastar para terra, usada no Minho.
Rêde de suspensão, empregada nas armações redondas de Peniche.
Apparelho de pesca, usado no mar alto pelos pescadores de Peniche.
\section{Sacadela}
\begin{itemize}
\item {Grp. gram.:f.}
\end{itemize}
Acto ou effeito de sacar; puxão.
Acto de tirar rapidamente da água o anzol, quando o peixe morde a isca.
\section{Sacado}
\begin{itemize}
\item {Grp. gram.:m.}
\end{itemize}
\begin{itemize}
\item {Utilização:Jur.}
\end{itemize}
\begin{itemize}
\item {Proveniência:(De \textunderscore sacar\textunderscore )}
\end{itemize}
Indivíduo, contra quem se passou uma letra de câmbio.
\section{Sacador}
\begin{itemize}
\item {Grp. gram.:m.  e  adj.}
\end{itemize}
\begin{itemize}
\item {Utilização:Jur.}
\end{itemize}
\begin{itemize}
\item {Utilização:Ant.}
\end{itemize}
O que saca.
O que passa uma letra de câmbio.
Cobrador de impostos.
\section{Saca-estrepe-da-mata}
\begin{itemize}
\item {Grp. gram.:m.}
\end{itemize}
Planta herbácea do Brasil.
\section{Saca-estrepe-de-campinas}
\begin{itemize}
\item {Grp. gram.:m.}
\end{itemize}
Planta brasileira, da fam. das compostas.
\section{Saca-filaça}
\begin{itemize}
\item {Grp. gram.:f.}
\end{itemize}
Agulha de artilheiro.
\section{Saca-fundo}
\begin{itemize}
\item {Grp. gram.:m.}
\end{itemize}
O mesmo ou melhor que \textunderscore tira-fundo\textunderscore .
\section{Saca-gaxetas}
\begin{itemize}
\item {Grp. gram.:m.}
\end{itemize}
Instrumento, com que a bordo se arrancam as gaxetas usadas.
\section{Sacaí}
\begin{itemize}
\item {Grp. gram.:m.}
\end{itemize}
\begin{itemize}
\item {Utilização:Bras. do N}
\end{itemize}
\begin{itemize}
\item {Proveniência:(Do guar. \textunderscore içacaí\textunderscore )}
\end{itemize}
Graveto.
Galho sêco de árvore.
Accendalha.
\section{Sacal}
\begin{itemize}
\item {Grp. gram.:m.}
\end{itemize}
Casta de uva.
\section{Saca-la-mana}
\begin{itemize}
\item {Grp. gram.:m.}
\end{itemize}
Espécie de jôgo popular.
\section{Sacalão}
\begin{itemize}
\item {Grp. gram.:m.}
\end{itemize}
\begin{itemize}
\item {Utilização:Pop.}
\end{itemize}
O mesmo que \textunderscore sacadela\textunderscore .
\section{Sacalhos}
\begin{itemize}
\item {Grp. gram.:m.}
\end{itemize}
\begin{itemize}
\item {Utilização:Prov.}
\end{itemize}
\begin{itemize}
\item {Utilização:trasm.}
\end{itemize}
Tamancos velhos.
(Por \textunderscore socalhos\textunderscore , de \textunderscore sóco\textunderscore , tamanco)
\section{Sacalinha}
\begin{itemize}
\item {Grp. gram.:f.}
\end{itemize}
(V.sancadilha)
\section{Sacamalo}
\begin{itemize}
\item {Grp. gram.:m.}
\end{itemize}
Planta escrofularínea.
\section{Sacamecrans}
\begin{itemize}
\item {Grp. gram.:m. pl.}
\end{itemize}
\begin{itemize}
\item {Utilização:Bras}
\end{itemize}
Tríbo de aborígenes do Maranhão.
\section{Saçamelo}
\begin{itemize}
\item {Grp. gram.:m.}
\end{itemize}
\begin{itemize}
\item {Utilização:Prov.}
\end{itemize}
\begin{itemize}
\item {Utilização:trasm.}
\end{itemize}
Aquelle que pronuncia defeituosamente o \textunderscore ç\textunderscore , metendo a língua entre os dentes.
Melhor se escreverá talvez \textunderscore çaçamelo\textunderscore .
\section{Saca-metal}
\begin{itemize}
\item {Grp. gram.:m.}
\end{itemize}
Agulha grossa, com que se remendam velas de navio.
\section{Saca-molas}
\begin{itemize}
\item {Grp. gram.:m.}
\end{itemize}
\begin{itemize}
\item {Utilização:Deprec.}
\end{itemize}
Instrumento, para arrancar dentes.
Mau dentista.
\section{Saca-nabo}
\begin{itemize}
\item {Grp. gram.:m.}
\end{itemize}
Gancho ou haste de ferro, com que se move o êmbolo da bomba, nos navios.
\section{Sacanga}
\begin{itemize}
\item {Grp. gram.:f.}
\end{itemize}
\begin{itemize}
\item {Utilização:Bras. do Rio}
\end{itemize}
O mesmo que \textunderscore sacaí\textunderscore .
\section{Sacão}
\begin{itemize}
\item {Grp. gram.:m.}
\end{itemize}
\begin{itemize}
\item {Proveniência:(De \textunderscore sacar\textunderscore )}
\end{itemize}
Salto de uma cavalgadura, para sacudir o cavalleiro; galão.
Safanão.
\section{Saca-peloiro}
\begin{itemize}
\item {Grp. gram.:m.}
\end{itemize}
Saca-trapo de artilharia.
\section{Saca-projéctil}
\begin{itemize}
\item {Grp. gram.:m.}
\end{itemize}
O mesmo que \textunderscore saca-peloiro\textunderscore .
\section{Sacar}
\begin{itemize}
\item {Grp. gram.:v. t.}
\end{itemize}
\begin{itemize}
\item {Utilização:Jur.}
\end{itemize}
\begin{itemize}
\item {Grp. gram.:V. i.}
\end{itemize}
\begin{itemize}
\item {Proveniência:(Do ant. fr. \textunderscore saquer\textunderscore )}
\end{itemize}
Tirar á fôrça, arrancar.
Tirar.
Fazer saír.
Lucrar, auferir.
Sêr sacador de (letra de câmbio).
Puxar; tirar com violência alguma coisa.
\section{Saca-rabo}
\begin{itemize}
\item {Grp. gram.:m.}
\end{itemize}
Animal, semelhante ao furão, mas maior e de longo rabo; o mesmo que \textunderscore mangusto\textunderscore . Cf. P. Moraes, \textunderscore Zool. Elem.\textunderscore , 188.
\section{Sacarato}
\begin{itemize}
\item {Grp. gram.:m.}
\end{itemize}
\begin{itemize}
\item {Proveniência:(Do lat. \textunderscore saccharum\textunderscore )}
\end{itemize}
Sal, produzido pela combinação do ácido sacárico como uma base.
\section{Sacaria}
\begin{itemize}
\item {Grp. gram.:f.}
\end{itemize}
\begin{itemize}
\item {Utilização:Ant.}
\end{itemize}
\begin{itemize}
\item {Proveniência:(De \textunderscore sacar\textunderscore )}
\end{itemize}
Rebate falso, para revista de tropas antes de combate.
\section{Sacaria}
\begin{itemize}
\item {Grp. gram.:f.}
\end{itemize}
\begin{itemize}
\item {Proveniência:(De \textunderscore saco\textunderscore ^2 ou \textunderscore saca\textunderscore ^2)}
\end{itemize}
Grande porção de sacos ou sacas.
\section{Sacárico}
\begin{itemize}
\item {Grp. gram.:adj.}
\end{itemize}
\begin{itemize}
\item {Proveniência:(Do lat. \textunderscore saccharum\textunderscore )}
\end{itemize}
Diz-se de um ácido, obtido pela oxidação de várias espécies de açúcar e do amido.
\section{Sacarídeo}
\begin{itemize}
\item {Grp. gram.:adj.}
\end{itemize}
\begin{itemize}
\item {Proveniência:(Do gr. \textunderscore sakkharon\textunderscore  + \textunderscore eidos\textunderscore )}
\end{itemize}
Semelhante ao açúcar.
\section{Sacarífero}
\begin{itemize}
\item {Grp. gram.:adj.}
\end{itemize}
\begin{itemize}
\item {Proveniência:(Do lat. \textunderscore saccharum\textunderscore  + \textunderscore ferre\textunderscore )}
\end{itemize}
Que produz açúcar: \textunderscore plantas sacaríferas\textunderscore .
\section{Sacarificação}
\begin{itemize}
\item {Grp. gram.:f.}
\end{itemize}
Acto ou efeito de sacarificar.
\section{Sacarificador}
\begin{itemize}
\item {Grp. gram.:adj.}
\end{itemize}
Que sacarifica.
\section{Sacarificante}
\begin{itemize}
\item {Grp. gram.:adj.}
\end{itemize}
Que sacarifica.
\section{Sacarificar}
\begin{itemize}
\item {Grp. gram.:v. t.}
\end{itemize}
\begin{itemize}
\item {Proveniência:(Do lat. \textunderscore saccharum\textunderscore  + \textunderscore facere\textunderscore )}
\end{itemize}
Converter em açúcar.
\section{Sacarificável}
\begin{itemize}
\item {Grp. gram.:adj.}
\end{itemize}
Que se póde sacarificar. Cf. \textunderscore Decreto\textunderscore  de 22-VII-905.
\section{Sacarimetria}
\begin{itemize}
\item {Grp. gram.:f.}
\end{itemize}
Aplicação do sacarímetro.
\section{Sacarimétrico}
\begin{itemize}
\item {Grp. gram.:adj.}
\end{itemize}
Relativo á sacarimetria.
\section{Sacarímetro}
\begin{itemize}
\item {Grp. gram.:m.}
\end{itemize}
\begin{itemize}
\item {Proveniência:(Do gr. \textunderscore sakkharon\textunderscore  + \textunderscore metron\textunderscore )}
\end{itemize}
Instrumento, para avaliar a porção de substância sacarina, contida noutra substância.
\section{Sacarina}
\begin{itemize}
\item {Grp. gram.:f.}
\end{itemize}
\begin{itemize}
\item {Proveniência:(De \textunderscore sacarino\textunderscore )}
\end{itemize}
Pó branco, muito fino, pouco solúvel na água, de um sabor pronunciadamente açucarado, e que se extrai do alcatrão de hulha, por um processo muito complexo.
\section{Sacaríneas}
\begin{itemize}
\item {Grp. gram.:f. pl.}
\end{itemize}
\begin{itemize}
\item {Proveniência:(De \textunderscore sácaro\textunderscore )}
\end{itemize}
Tríbo de plantas gramíneas.
\section{Sacarino}
\begin{itemize}
\item {Grp. gram.:adj.}
\end{itemize}
\begin{itemize}
\item {Proveniência:(Do lat. \textunderscore saccharum\textunderscore )}
\end{itemize}
Relativo ao açúcar.
Em que há açúcar.
Doce como açúcar.
Que se alimenta de açúcar, (falando-se de certos animaes).
\section{Sacário}
\begin{itemize}
\item {Grp. gram.:m.}
\end{itemize}
\begin{itemize}
\item {Proveniência:(Lat. \textunderscore saccarius\textunderscore )}
\end{itemize}
Carregador que, entre os Romanos, tinha o privilégio de conduzir fardos, do pôrto para os armazéns.
\section{Sacarito}
\begin{itemize}
\item {Grp. gram.:m.}
\end{itemize}
Silicato alcalino de alumina e de cal.
\section{Sacarívoro}
\begin{itemize}
\item {Grp. gram.:adj.}
\end{itemize}
\begin{itemize}
\item {Proveniência:(Do lat. \textunderscore saccharum\textunderscore  + \textunderscore vorare\textunderscore )}
\end{itemize}
Que se alimenta de açúcar.
\section{Sácaro}
\begin{itemize}
\item {Grp. gram.:m.}
\end{itemize}
\begin{itemize}
\item {Proveniência:(Lat. \textunderscore saccharum\textunderscore )}
\end{itemize}
Gênero de gramineas, que compreende a cana de açúcar.
\section{Sacaróide}
\begin{itemize}
\item {Grp. gram.:adj.}
\end{itemize}
\begin{itemize}
\item {Utilização:Miner.}
\end{itemize}
\begin{itemize}
\item {Proveniência:(Do gr. \textunderscore sakkaron\textunderscore  + \textunderscore eidos\textunderscore )}
\end{itemize}
Que tem estructura granulosa como o açúcar.
\section{Sacarol}
\begin{itemize}
\item {Grp. gram.:m.}
\end{itemize}
\begin{itemize}
\item {Utilização:Pharm.}
\end{itemize}
O açúcar, considerado como excipiente.
\section{Sacarolado}
\begin{itemize}
\item {Grp. gram.:adj.}
\end{itemize}
\begin{itemize}
\item {Utilização:Pharm.}
\end{itemize}
\begin{itemize}
\item {Proveniência:(De \textunderscore sacarol\textunderscore )}
\end{itemize}
Que tem o açúcar ou o mel como excipiente.
\section{Sacaróleo}
\begin{itemize}
\item {Grp. gram.:m.}
\end{itemize}
\begin{itemize}
\item {Proveniência:(Do lat. \textunderscore saccharum\textunderscore  + \textunderscore oleum\textunderscore )}
\end{itemize}
Preparado farmacêutico de açúcar e óleo volátil.
\section{Saca-rolha}
\begin{itemize}
\item {Grp. gram.:f.}
\end{itemize}
\begin{itemize}
\item {Utilização:Bras. de Minas}
\end{itemize}
Nome de uma planta medicinal.
\section{Saca-rolhas}
\begin{itemize}
\item {Grp. gram.:m.}
\end{itemize}
\begin{itemize}
\item {Utilização:Bot.}
\end{itemize}
Instrumento, com que se tiram rolhas de garrafas ou de outros vasos.
Nome de diversas esterculiáceas do Brasil.
\section{Sacarolinita}
\begin{itemize}
\item {Grp. gram.:f.}
\end{itemize}
Medicamento açucarado, de pequeno volume.
\section{Sacaromicose}
\begin{itemize}
\item {Grp. gram.:f.}
\end{itemize}
\begin{itemize}
\item {Utilização:Med.}
\end{itemize}
\begin{itemize}
\item {Proveniência:(Do gr. \textunderscore sakkharon\textunderscore  + \textunderscore mukes\textunderscore )}
\end{itemize}
Aftas das crianças de mama.
\section{Sacarose}
\begin{itemize}
\item {Grp. gram.:f.}
\end{itemize}
\begin{itemize}
\item {Proveniência:(Do lat. \textunderscore saccharum\textunderscore )}
\end{itemize}
O açúcar comum.
\section{Sacaroso}
\begin{itemize}
\item {Grp. gram.:adj.}
\end{itemize}
\begin{itemize}
\item {Proveniência:(Do lat. \textunderscore saccharum\textunderscore )}
\end{itemize}
Que é da natureza do açúcar.
\section{Sacarrão}
\begin{itemize}
\item {Grp. gram.:m.}
\end{itemize}
\begin{itemize}
\item {Utilização:Prov.}
\end{itemize}
Saco grande.
\section{Saca-soca}
\begin{itemize}
\item {Grp. gram.:f.}
\end{itemize}
Ave, do Cabo da Bôa-Esperança.
\section{Saca-trapo}
\begin{itemize}
\item {Grp. gram.:m.}
\end{itemize}
\begin{itemize}
\item {Utilização:pop.}
\end{itemize}
\begin{itemize}
\item {Utilização:Fig.}
\end{itemize}
Instrumento, com que se tira a buxa das armas de fogo.
Manha, para conseguir alguma coisa.
\section{Saca-tutano}
\begin{itemize}
\item {Grp. gram.:m.}
\end{itemize}
Utensílio de prata, com que, á mesa, se tira o tutano dos ossos.
\section{Sacaubarana}
\begin{itemize}
\item {Grp. gram.:f.}
\end{itemize}
Planta do Brasil.
\section{Sacaveno}
\begin{itemize}
\item {Grp. gram.:m.}
\end{itemize}
Habitante de Sacavém.
\section{Saccharato}
\begin{itemize}
\item {fónica:ca}
\end{itemize}
\begin{itemize}
\item {Grp. gram.:m.}
\end{itemize}
\begin{itemize}
\item {Proveniência:(Do lat. \textunderscore saccharum\textunderscore )}
\end{itemize}
Sal, produzido pela combinação do ácido sacchárico como uma base.
\section{Sacchárico}
\begin{itemize}
\item {fónica:cá}
\end{itemize}
\begin{itemize}
\item {Grp. gram.:adj.}
\end{itemize}
\begin{itemize}
\item {Proveniência:(Do lat. \textunderscore saccharum\textunderscore )}
\end{itemize}
Diz-se de um ácido, obtido pela oxydação de várias espécies de açúcar e do amido.
\section{Saccharídeo}
\begin{itemize}
\item {fónica:ca}
\end{itemize}
\begin{itemize}
\item {Grp. gram.:adj.}
\end{itemize}
\begin{itemize}
\item {Proveniência:(Do gr. \textunderscore sakkharon\textunderscore  + \textunderscore eidos\textunderscore )}
\end{itemize}
Semelhante ao açúcar.
\section{Saccharífero}
\begin{itemize}
\item {fónica:ca}
\end{itemize}
\begin{itemize}
\item {Grp. gram.:adj.}
\end{itemize}
\begin{itemize}
\item {Proveniência:(Do lat. \textunderscore saccharum\textunderscore  + \textunderscore ferre\textunderscore )}
\end{itemize}
Que produz açúcar: \textunderscore plantas saccharíferas\textunderscore .
\section{Saccharificação}
\begin{itemize}
\item {fónica:ca}
\end{itemize}
\begin{itemize}
\item {Grp. gram.:f.}
\end{itemize}
Acto ou effeito de saccharificar.
\section{Saccharificador}
\begin{itemize}
\item {fónica:ca}
\end{itemize}
\begin{itemize}
\item {Grp. gram.:adj.}
\end{itemize}
Que saccharifica.
\section{Saccharificante}
\begin{itemize}
\item {fónica:ca}
\end{itemize}
\begin{itemize}
\item {Grp. gram.:adj.}
\end{itemize}
Que saccharifica.
\section{Saccharificar}
\begin{itemize}
\item {fónica:ca}
\end{itemize}
\begin{itemize}
\item {Grp. gram.:v. t.}
\end{itemize}
\begin{itemize}
\item {Proveniência:(Do lat. \textunderscore saccharum\textunderscore  + \textunderscore facere\textunderscore )}
\end{itemize}
Converter em açúcar.
\section{Saccharificável}
\begin{itemize}
\item {fónica:ca}
\end{itemize}
\begin{itemize}
\item {Grp. gram.:adj.}
\end{itemize}
Que se póde saccharificar. Cf. \textunderscore Decreto\textunderscore  de 22-VII-905.
\section{Saccharimetria}
\begin{itemize}
\item {fónica:ca}
\end{itemize}
\begin{itemize}
\item {Grp. gram.:f.}
\end{itemize}
Applicação do saccharímetro.
\section{Saccharimétrico}
\begin{itemize}
\item {fónica:ca}
\end{itemize}
\begin{itemize}
\item {Grp. gram.:adj.}
\end{itemize}
Relativo á saccharimetria.
\section{Saccharímetro}
\begin{itemize}
\item {fónica:ca}
\end{itemize}
\begin{itemize}
\item {Grp. gram.:m.}
\end{itemize}
\begin{itemize}
\item {Proveniência:(Do gr. \textunderscore sakkharon\textunderscore  + \textunderscore metron\textunderscore )}
\end{itemize}
Instrumento, para avaliar a porção de substância saccharina, contida noutra substância.
\section{Saccharina}
\begin{itemize}
\item {fónica:ca}
\end{itemize}
\begin{itemize}
\item {Grp. gram.:f.}
\end{itemize}
\begin{itemize}
\item {Proveniência:(De \textunderscore saccharino\textunderscore )}
\end{itemize}
Pó branco, muito fino, pouco solúvel na água, de um sabor pronunciadamente açucarado, e que se extrai do alcatrão de hulha, por um processo muito complexo.
\section{Saccharíneas}
\begin{itemize}
\item {fónica:ca}
\end{itemize}
\begin{itemize}
\item {Grp. gram.:f. pl.}
\end{itemize}
\begin{itemize}
\item {Proveniência:(De \textunderscore sáccharo\textunderscore )}
\end{itemize}
Tríbo de plantas gramíneas.
\section{Saccharino}
\begin{itemize}
\item {fónica:ca}
\end{itemize}
\begin{itemize}
\item {Grp. gram.:adj.}
\end{itemize}
\begin{itemize}
\item {Proveniência:(Do lat. \textunderscore saccharum\textunderscore )}
\end{itemize}
Relativo ao açúcar.
Em que há açúcar.
Doce como açúcar.
Que se alimenta de açúcar, (falando-se de certos animaes).
\section{Saccharito}
\begin{itemize}
\item {fónica:ca}
\end{itemize}
\textunderscore m.\textunderscore 
Silicato alcalino de alumina e de cal.
\section{Saccharívoro}
\begin{itemize}
\item {fónica:ca}
\end{itemize}
\begin{itemize}
\item {Grp. gram.:adj.}
\end{itemize}
\begin{itemize}
\item {Proveniência:(Do lat. \textunderscore saccharum\textunderscore  + \textunderscore vorare\textunderscore )}
\end{itemize}
Que se alimenta de açúcar.
\section{Sáccharo}
\begin{itemize}
\item {fónica:ca}
\end{itemize}
\begin{itemize}
\item {Grp. gram.:m.}
\end{itemize}
\begin{itemize}
\item {Proveniência:(Lat. \textunderscore saccharum\textunderscore )}
\end{itemize}
Gênero de gramineas, que comprehende a cana de açúcar.
\section{Saccharóide}
\begin{itemize}
\item {fónica:ca}
\end{itemize}
\begin{itemize}
\item {Grp. gram.:adj.}
\end{itemize}
\begin{itemize}
\item {Utilização:Miner.}
\end{itemize}
\begin{itemize}
\item {Proveniência:(Do gr. \textunderscore sakkaron\textunderscore  + \textunderscore eidos\textunderscore )}
\end{itemize}
Que tem estructura granulosa como o açúcar.
\section{Saccharol}
\begin{itemize}
\item {fónica:ca}
\end{itemize}
\begin{itemize}
\item {Grp. gram.:m.}
\end{itemize}
\begin{itemize}
\item {Utilização:Pharm.}
\end{itemize}
O açúcar, considerado como excipiente.
\section{Saccharolado}
\begin{itemize}
\item {fónica:ca}
\end{itemize}
\begin{itemize}
\item {Grp. gram.:adj.}
\end{itemize}
\begin{itemize}
\item {Utilização:Pharm.}
\end{itemize}
\begin{itemize}
\item {Proveniência:(De \textunderscore saccharol\textunderscore )}
\end{itemize}
Que tem o açúcar ou o mel como excipiente.
\section{Saccharóleo}
\begin{itemize}
\item {fónica:ca}
\end{itemize}
\begin{itemize}
\item {Grp. gram.:m.}
\end{itemize}
\begin{itemize}
\item {Proveniência:(Do lat. \textunderscore saccharum\textunderscore  + \textunderscore oleum\textunderscore )}
\end{itemize}
Preparado pharmacêutico de açúcar e óleo volátil.
\section{Saccharolinita}
\begin{itemize}
\item {fónica:ca}
\end{itemize}
\begin{itemize}
\item {Grp. gram.:f.}
\end{itemize}
Medicamento açucarado, de pequeno volume.
\section{Saccharomycose}
\begin{itemize}
\item {fónica:ca}
\end{itemize}
\begin{itemize}
\item {Grp. gram.:f.}
\end{itemize}
\begin{itemize}
\item {Utilização:Med.}
\end{itemize}
\begin{itemize}
\item {Proveniência:(Do gr. \textunderscore sakkharon\textunderscore  + \textunderscore mukes\textunderscore )}
\end{itemize}
Aphtas das crianças de mama.
\section{Saccharose}
\begin{itemize}
\item {fónica:ca}
\end{itemize}
\begin{itemize}
\item {Grp. gram.:f.}
\end{itemize}
\begin{itemize}
\item {Proveniência:(Do lat. \textunderscore saccharum\textunderscore )}
\end{itemize}
O açúcar commum.
\section{Saccharoso}
\begin{itemize}
\item {fónica:ca}
\end{itemize}
\begin{itemize}
\item {Grp. gram.:adj.}
\end{itemize}
\begin{itemize}
\item {Proveniência:(Do lat. \textunderscore saccharum\textunderscore )}
\end{itemize}
Que é da natureza do açúcar.
\section{Sacchogomita}
\begin{itemize}
\item {fónica:co}
\end{itemize}
\begin{itemize}
\item {Grp. gram.:f.}
\end{itemize}
Princípio açucarado do alcaçuz.
\section{Saccholactato}
\begin{itemize}
\item {fónica:co}
\end{itemize}
\begin{itemize}
\item {Grp. gram.:m.}
\end{itemize}
\begin{itemize}
\item {Utilização:Chím.}
\end{itemize}
Gênero de sal, produzido pelo ácido saccholáctico com uma base.
\section{Saccholáctico}
\begin{itemize}
\item {fónica:co}
\end{itemize}
\begin{itemize}
\item {Grp. gram.:adj.}
\end{itemize}
\begin{itemize}
\item {Utilização:Pharm.}
\end{itemize}
Que se obtém por meio de leite e açúcar.
\section{Saceliforme}
\begin{itemize}
\item {Grp. gram.:adj.}
\end{itemize}
\begin{itemize}
\item {Utilização:Bot.}
\end{itemize}
\begin{itemize}
\item {Proveniência:(De \textunderscore sacello\textunderscore ^2 + \textunderscore fórma\textunderscore )}
\end{itemize}
Que tem a fórma de pequeno saco.
\section{Sacelinho}
\begin{itemize}
\item {Grp. gram.:m.}
\end{itemize}
\begin{itemize}
\item {Proveniência:(De \textunderscore sacello\textunderscore ^1)}
\end{itemize}
Capella pequena. Cf. Castilho, \textunderscore Fastos\textunderscore , III, 95.
\section{Sacelliforme}
\begin{itemize}
\item {Grp. gram.:adj.}
\end{itemize}
\begin{itemize}
\item {Utilização:Bot.}
\end{itemize}
\begin{itemize}
\item {Proveniência:(De \textunderscore sacello\textunderscore ^2 + \textunderscore fórma\textunderscore )}
\end{itemize}
Que tem a fórma de pequeno saco.
\section{Sacellinho}
\begin{itemize}
\item {Grp. gram.:m.}
\end{itemize}
\begin{itemize}
\item {Proveniência:(De \textunderscore sacello\textunderscore ^1)}
\end{itemize}
Capella pequena. Cf. Castilho, \textunderscore Fastos\textunderscore , III, 95.
\section{Sacello}
\begin{itemize}
\item {Grp. gram.:m.}
\end{itemize}
\begin{itemize}
\item {Utilização:Ant.}
\end{itemize}
\begin{itemize}
\item {Proveniência:(Lat. \textunderscore sacellum\textunderscore )}
\end{itemize}
Pequeno templo, santuário.
\section{Sacello}
\begin{itemize}
\item {Grp. gram.:m.}
\end{itemize}
\begin{itemize}
\item {Utilização:Bot.}
\end{itemize}
\begin{itemize}
\item {Proveniência:(Lat. \textunderscore saccellus\textunderscore )}
\end{itemize}
Fruto monospérmico, de invólucro membranoso.
\section{Sacelo}
\begin{itemize}
\item {Grp. gram.:m.}
\end{itemize}
\begin{itemize}
\item {Utilização:Ant.}
\end{itemize}
\begin{itemize}
\item {Proveniência:(Lat. \textunderscore sacellum\textunderscore )}
\end{itemize}
Pequeno templo, santuário.
\section{Sacelo}
\begin{itemize}
\item {Grp. gram.:m.}
\end{itemize}
\begin{itemize}
\item {Utilização:Bot.}
\end{itemize}
\begin{itemize}
\item {Proveniência:(Lat. \textunderscore saccellus\textunderscore )}
\end{itemize}
Fruto monospérmico, de invólucro membranoso.
\section{Sacerdócio}
\begin{itemize}
\item {Grp. gram.:m.}
\end{itemize}
\begin{itemize}
\item {Utilização:Fig.}
\end{itemize}
\begin{itemize}
\item {Proveniência:(Lat. \textunderscore sacerdotium\textunderscore )}
\end{itemize}
Ministério ou funcções daquelles que, entre os Judeus, tinham o poder de offerecer víctimas á divindade.
Dignidade ou offício de sacerdote, entre os Christãos.
Poder sacerdotal.
Qualidade daquillo que é venerável, nobre, superior.
Profissão honrosa: \textunderscore o sacerdócio da imprensa\textunderscore .
\section{Sacerdotal}
\begin{itemize}
\item {Grp. gram.:adj.}
\end{itemize}
\begin{itemize}
\item {Proveniência:(Lat. \textunderscore sacerdotalis\textunderscore )}
\end{itemize}
Relativo a sacerdote ou a sacerdócio: \textunderscore traje sacerdotal\textunderscore .
\section{Sacerdotalismo}
\begin{itemize}
\item {Grp. gram.:m.}
\end{itemize}
Predomínio sacerdotal; clericalismo; theocracia.
\section{Sacerdote}
\begin{itemize}
\item {Grp. gram.:m.}
\end{itemize}
\begin{itemize}
\item {Utilização:Fig.}
\end{itemize}
\begin{itemize}
\item {Proveniência:(Lat. \textunderscore sacerdos\textunderscore , \textunderscore sacerdotis\textunderscore )}
\end{itemize}
Ministro dos sacrifícios religiosos, entre os antigos.
O mesmo que \textunderscore padre\textunderscore .
O que exerce profissão muito honrosa ou cumpre missão elevada.
\section{Sacerdotisa}
\begin{itemize}
\item {Grp. gram.:f.}
\end{itemize}
\begin{itemize}
\item {Proveniência:(Lat. \textunderscore sacerdotissa\textunderscore )}
\end{itemize}
Mulhér, que exercia as funcções de sacerdote, entre os Pagãos.
\section{Sacha}
\begin{itemize}
\item {Grp. gram.:f.}
\end{itemize}
O mesmo que \textunderscore sachadura\textunderscore .
\section{Sachada}
\begin{itemize}
\item {Grp. gram.:f.}
\end{itemize}
\begin{itemize}
\item {Utilização:Prov.}
\end{itemize}
O mesmo que \textunderscore sachadura\textunderscore . Cf. Júl. Dinís, \textunderscore Morgadinha\textunderscore , 85.
\section{Sachador}
\begin{itemize}
\item {Grp. gram.:adj.}
\end{itemize}
\begin{itemize}
\item {Grp. gram.:M.}
\end{itemize}
Que sacha.
Aquelle que sacha.
O mesmo que \textunderscore sachola\textunderscore .
Apparelho agrícola, constituido por uma armação triangular, que descansa, á frente, numa roda, e é munida de dentes, com que se limpas das ervas ruíns os intervallos das linhas das plantas sachadas, (milho, batata, etc.).
\section{Sachadura}
\begin{itemize}
\item {Grp. gram.:f.}
\end{itemize}
Acto ou effeito de sachar.
\section{Sachar}
\begin{itemize}
\item {Grp. gram.:v. t.}
\end{itemize}
Mondar com o sacho.
Escavar com o sacho.
Tirar com o sacho as ervas damninhas a: \textunderscore sachar o meloal\textunderscore .
\section{Sacho}
\begin{itemize}
\item {Grp. gram.:m.}
\end{itemize}
\begin{itemize}
\item {Utilização:Pesc.}
\end{itemize}
\begin{itemize}
\item {Proveniência:(Do lat. \textunderscore sarculum\textunderscore )}
\end{itemize}
Espécie de pequena sachola, que tem na parte superior do ólho uma pêta ponteaguda ou bifurcada.
Estribo de madeira da poita.
\section{Sachola}
\begin{itemize}
\item {Grp. gram.:f.}
\end{itemize}
\begin{itemize}
\item {Proveniência:(De \textunderscore sacho\textunderscore )}
\end{itemize}
Pequena enxada.
\section{Sacholada}
\begin{itemize}
\item {Grp. gram.:f.}
\end{itemize}
Pancada ou ferimento com a sachola.
\section{Sacholar}
\begin{itemize}
\item {Grp. gram.:v. t.}
\end{itemize}
Cavar com a sachola.
Escavar.
Cavar superficialmente.
Espancar com sacho ou sachola.
\section{Sacholo}
\begin{itemize}
\item {fónica:chô}
\end{itemize}
\begin{itemize}
\item {Grp. gram.:m.}
\end{itemize}
\begin{itemize}
\item {Utilização:Prov.}
\end{itemize}
\begin{itemize}
\item {Utilização:beir.}
\end{itemize}
Pequena sachola; sacho grande.
(Cf. \textunderscore sachola\textunderscore )
\section{Saci}
\begin{itemize}
\item {Grp. gram.:m.}
\end{itemize}
\begin{itemize}
\item {Utilização:Bras}
\end{itemize}
Ente phantástico, repreentado por um negrinho que usa barrete vermelho e frequenta á noite os brejos.
Passarinho, cujo canto imita a pronúncia do seu nome.
\section{Saciar}
\begin{itemize}
\item {Grp. gram.:v. t.}
\end{itemize}
\begin{itemize}
\item {Proveniência:(Lat. \textunderscore satiare\textunderscore )}
\end{itemize}
Encher, fartar.
Satisfazer: \textunderscore saciar a fome\textunderscore ; \textunderscore saciar ódios\textunderscore .
\section{Saciável}
\begin{itemize}
\item {Grp. gram.:adj.}
\end{itemize}
\begin{itemize}
\item {Proveniência:(Lat. \textunderscore satiabilis\textunderscore )}
\end{itemize}
Que se póde saciar.
\section{Saciedade}
\begin{itemize}
\item {Grp. gram.:f.}
\end{itemize}
\begin{itemize}
\item {Grp. gram.:Loc. adv.}
\end{itemize}
\begin{itemize}
\item {Proveniência:(Do lat. \textunderscore satietas\textunderscore )}
\end{itemize}
Estado de quem se saciou.
Satisfação do appetite.
Fartura.
Aborrecimento, fastio.
\textunderscore Até á saciedade\textunderscore , plenamente, completamente.
Até se satisfazer de todo.
\section{Saco}
\begin{itemize}
\item {Grp. gram.:m.}
\end{itemize}
\begin{itemize}
\item {Utilização:Pop.}
\end{itemize}
\begin{itemize}
\item {Utilização:Pesc.}
\end{itemize}
\begin{itemize}
\item {Utilização:T. de Angola}
\end{itemize}
\begin{itemize}
\item {Utilização:T. da Bairrada}
\end{itemize}
\begin{itemize}
\item {Utilização:Prov.}
\end{itemize}
\begin{itemize}
\item {Proveniência:(Lat. \textunderscore saccus\textunderscore )}
\end{itemize}
Receptáculo de tecido ou coiro, aberto em cima e cosido por baixo e dos lados, ou cosido por baixo e de um só lado, quando do outro há continuidade da peça.
Quantidade, que um saco póde conter.
Tufo ou coisa, que dá a apparência de saco.
Pequena mala.
Antigo hábito de penitente.
Pássaro conirostro de Angola, o mesmo que \textunderscore quissengo\textunderscore .
Pessôa gorda ou desajeitada.
Rêde, de fórma cónica que, nos apparelhos de arrastar e nos de cêrco, serve para recolher a pescaria.
Peça central da rêde de pescar sardinha.
Trinta mil réis em moéda de cobre; o pêso dêsse dinheiro.
Sujidade, que se accumula no fundo do cano das espingardas, que se carregam pela bôca.
Medida de cereaes, correspondente a 6 alqueires.
\textunderscore Saco de terra\textunderscore , terreno, que póde levar 6 alqueires de semeadura de cereaes.
\section{Saco}
\begin{itemize}
\item {Grp. gram.:m.}
\end{itemize}
\begin{itemize}
\item {Utilização:Ant.}
\end{itemize}
O mesmo que \textunderscore saque\textunderscore ^2:«\textunderscore quem defende vossa casa de um saco?\textunderscore »\textunderscore Eufrosina\textunderscore , 54.«\textunderscore ...toda a mais cidade deo a saco.\textunderscore »Filinto, \textunderscore D. Man.\textunderscore , II, 252.
\section{Saço}
\begin{itemize}
\item {Grp. gram.:m.}
\end{itemize}
\begin{itemize}
\item {Utilização:Prov.}
\end{itemize}
\begin{itemize}
\item {Utilização:trasm.}
\end{itemize}
O mesmo que \textunderscore saçamelo\textunderscore .
\section{Saco-de-areia}
\begin{itemize}
\item {Grp. gram.:m.}
\end{itemize}
\begin{itemize}
\item {Utilização:Bras}
\end{itemize}
Dança, acompanhada de canto, e usada nas roças.
\section{Sacóforo}
\begin{itemize}
\item {Grp. gram.:adj.}
\end{itemize}
\begin{itemize}
\item {Grp. gram.:M.}
\end{itemize}
\begin{itemize}
\item {Grp. gram.:Pl.}
\end{itemize}
\begin{itemize}
\item {Proveniência:(Do gr. \textunderscore sakos\textunderscore  + \textunderscore phoros\textunderscore )}
\end{itemize}
Que tem órgão saculiforme.
Penitente, que se cobria de saco.
O mesmo que \textunderscore tunicários\textunderscore .
\section{Sacogomita}
\begin{itemize}
\item {Grp. gram.:f.}
\end{itemize}
Princípio açucarado do alcaçuz.
\section{Sacola}
\begin{itemize}
\item {Grp. gram.:f.}
\end{itemize}
\begin{itemize}
\item {Utilização:Ext.}
\end{itemize}
\begin{itemize}
\item {Proveniência:(De \textunderscore saco\textunderscore )}
\end{itemize}
Reunião de dois sacos, ou saco de dois fundos.
Alforge.
Algibeira; bornal.
\section{Sacoláctico}
\begin{itemize}
\item {Grp. gram.:adj.}
\end{itemize}
\begin{itemize}
\item {Utilização:Pharm.}
\end{itemize}
Que se obtém por meio de leite e açúcar.
\section{Sacolactato}
\begin{itemize}
\item {Grp. gram.:m.}
\end{itemize}
\begin{itemize}
\item {Utilização:Chím.}
\end{itemize}
Gênero de sal, produzido pelo ácido sacoláctico com uma base.
\section{Sacolejar}
\begin{itemize}
\item {Grp. gram.:v. t.}
\end{itemize}
\begin{itemize}
\item {Proveniência:(De \textunderscore sacola\textunderscore )}
\end{itemize}
Agitar muitas vezes; vascolejar.
\section{Sacolejo}
\begin{itemize}
\item {Grp. gram.:m.}
\end{itemize}
Acto de sacolejar.
\section{Sacomano}
\begin{itemize}
\item {Grp. gram.:m.}
\end{itemize}
\begin{itemize}
\item {Utilização:Ant.}
\end{itemize}
\begin{itemize}
\item {Proveniência:(De \textunderscore sacar\textunderscore  + lat. \textunderscore manus\textunderscore ?)}
\end{itemize}
O mesmo que \textunderscore saque\textunderscore ^1.
Sacomão.
\section{Sacomão}
\begin{itemize}
\item {Grp. gram.:m.}
\end{itemize}
\begin{itemize}
\item {Utilização:Ant.}
\end{itemize}
\begin{itemize}
\item {Proveniência:(De \textunderscore sacar\textunderscore  + \textunderscore mão\textunderscore ?)}
\end{itemize}
Salteador.
Mendigo; infeliz.
\section{Sacomardo}
\begin{itemize}
\item {Grp. gram.:m.}
\end{itemize}
\begin{itemize}
\item {Utilização:Ant.}
\end{itemize}
O mesmo que \textunderscore sacomão\textunderscore .
\section{Sacôndio}
\begin{itemize}
\item {Grp. gram.:m.}
\end{itemize}
\begin{itemize}
\item {Proveniência:(Lat. \textunderscore sacondios\textunderscore )}
\end{itemize}
Variedade de amethysta.
\section{Sacopétalo}
\begin{itemize}
\item {Grp. gram.:m.}
\end{itemize}
\begin{itemize}
\item {Proveniência:(Do gr. \textunderscore sakos\textunderscore  + \textunderscore petalon\textunderscore )}
\end{itemize}
Gênero de plantas anonáceas.
\section{Sacóphoro}
\begin{itemize}
\item {Grp. gram.:adj.}
\end{itemize}
\begin{itemize}
\item {Grp. gram.:M.}
\end{itemize}
\begin{itemize}
\item {Grp. gram.:Pl.}
\end{itemize}
\begin{itemize}
\item {Proveniência:(Do gr. \textunderscore sakos\textunderscore  + \textunderscore phoros\textunderscore )}
\end{itemize}
Que tem órgão saculiforme.
Penitente, que se cobria de saco.
O mesmo que \textunderscore tunicários\textunderscore .
\section{Saco-roto}
\begin{itemize}
\item {Grp. gram.:m.}
\end{itemize}
\begin{itemize}
\item {Utilização:Fam.}
\end{itemize}
Aquelle que não sabe guardar segredos, que vai repetir quanto ouve.
\section{Sacoto}
\begin{itemize}
\item {fónica:cô}
\end{itemize}
\begin{itemize}
\item {Grp. gram.:m.}
\end{itemize}
\begin{itemize}
\item {Utilização:T. da Bairrada}
\end{itemize}
O mesmo que \textunderscore saracoto\textunderscore .
\section{Sacra}
\begin{itemize}
\item {Grp. gram.:f.}
\end{itemize}
\begin{itemize}
\item {Proveniência:(Lat. \textunderscore sacra\textunderscore )}
\end{itemize}
Quadro pequeno, que contém várias orações e outras fórmulas, e que se encosta á banqueta do altar, para auxiliar a memória de quem celebra a Missa.
\section{Sacrafineiro}
\begin{itemize}
\item {Grp. gram.:m.}
\end{itemize}
\begin{itemize}
\item {Utilização:Prov.}
\end{itemize}
\begin{itemize}
\item {Utilização:trasm.}
\end{itemize}
Indivíduo muito atarefado com bagatelas; maricas.
\section{Sacramentado}
\begin{itemize}
\item {Grp. gram.:m.}
\end{itemize}
\begin{itemize}
\item {Proveniência:(De \textunderscore sacramentar\textunderscore )}
\end{itemize}
Aquelle que se sacramentou.
\section{Sacramental}
\begin{itemize}
\item {Grp. gram.:adj.}
\end{itemize}
\begin{itemize}
\item {Utilização:Fig.}
\end{itemize}
Relativo ao sacramento.
Consuetudinário; obrigatório: \textunderscore palavras sacramentaes\textunderscore .
\section{Sacramentalmente}
\begin{itemize}
\item {Grp. gram.:adv.}
\end{itemize}
De modo sacramental.
\section{Sacramentar}
\begin{itemize}
\item {Grp. gram.:v. t.}
\end{itemize}
\begin{itemize}
\item {Grp. gram.:V. p.}
\end{itemize}
Ministrar os sacramentos a, especialmente os sacramentos da confissão e communhão.
Sagrar.
Confessar.
Receber os sacramentos.
\section{Sacramentário}
\begin{itemize}
\item {Grp. gram.:m.}
\end{itemize}
\begin{itemize}
\item {Proveniência:(Lat. \textunderscore sacramentarium\textunderscore )}
\end{itemize}
Livro antigo, em que se descreviam certas ceremónias religiosas, especialmente as relativas aos sacramentos.
Sectário do Protestantismo.
\section{Sacramento}
\begin{itemize}
\item {Grp. gram.:m.}
\end{itemize}
\begin{itemize}
\item {Grp. gram.:Pl.}
\end{itemize}
\begin{itemize}
\item {Proveniência:(Lat. \textunderscore sacramentum\textunderscore )}
\end{itemize}
Juramento.
Acto religioso, instituído por Deus, para santificação da alma.
Ceremónia christan, destinada a consagrar diversas phases da vida dos fiéis.
Acto de consagrar.
Custódia, que encerra a hóstia consagrada.
Eucharistia.
Últimos sacramentos, isto é, a confissão, a communhão e a extrema-uncção.
\section{Sacrário}
\begin{itemize}
\item {Grp. gram.:m.}
\end{itemize}
\begin{itemize}
\item {Utilização:Fig.}
\end{itemize}
\begin{itemize}
\item {Proveniência:(Lat. \textunderscore sacrarium\textunderscore )}
\end{itemize}
Lugar, onde se guardam coisas sagradas, especialmente hóstias ou relíquias.
Partículas da communhão.
Intimidade.
Lugar reservado e respeitável: \textunderscore o sacrário do teu coração\textunderscore .
\section{Sacratíssimo}
\begin{itemize}
\item {Grp. gram.:adj.}
\end{itemize}
\begin{itemize}
\item {Proveniência:(Lat. \textunderscore sacratissimus\textunderscore )}
\end{itemize}
Muito sagrado; santíssimo.
\section{Sacre}
\begin{itemize}
\item {Grp. gram.:m.}
\end{itemize}
\begin{itemize}
\item {Proveniência:(Do ár. \textunderscore çaqre\textunderscore )}
\end{itemize}
Espécie de falcão.
Antigo e grande canhão.
\section{Sácri}
\begin{itemize}
\item {Grp. gram.:m.}
\end{itemize}
O mesmo que \textunderscore sacre\textunderscore . Cf. Usque, 7.
\section{Sacrífero}
\begin{itemize}
\item {Grp. gram.:adj.}
\end{itemize}
\begin{itemize}
\item {Proveniência:(Lat. \textunderscore sacrifer\textunderscore )}
\end{itemize}
Que transporta as coisas sagradas. Cf. Castilho, \textunderscore Fastos\textunderscore , II, 135; Filinto, VI, 245.
\section{Sacrificador}
\begin{itemize}
\item {Grp. gram.:m.  e  adj.}
\end{itemize}
\begin{itemize}
\item {Proveniência:(Lat. \textunderscore sacrificator\textunderscore )}
\end{itemize}
O que sacrifica.
\section{Sacrifical}
\begin{itemize}
\item {Grp. gram.:adj.}
\end{itemize}
\begin{itemize}
\item {Proveniência:(Lat. \textunderscore sacrificalis\textunderscore )}
\end{itemize}
Relativo ao sacrifício.
\section{Sacrificante}
\begin{itemize}
\item {Grp. gram.:m.  e  adj.}
\end{itemize}
\begin{itemize}
\item {Proveniência:(Lat. \textunderscore sacrificans\textunderscore )}
\end{itemize}
O mesmo que \textunderscore sacrificador\textunderscore .
Aquelle que celebra Missa.
\section{Sacrificar}
\begin{itemize}
\item {Grp. gram.:v. t.}
\end{itemize}
\begin{itemize}
\item {Grp. gram.:V. i.}
\end{itemize}
\begin{itemize}
\item {Grp. gram.:V. p.}
\end{itemize}
\begin{itemize}
\item {Proveniência:(Lat. \textunderscore sacrificare\textunderscore )}
\end{itemize}
Offerecer á divindade em sacrifício; immolar.
Renunciar, abandonar: \textunderscore sacrificar prazeres\textunderscore .
Sujeitar a perigos: \textunderscore sacrificar o património dos filhos\textunderscore .
Victimar.
Fazer sacrifícios em honra da divindade.
Offerecer-se em sacrifício.
Tornar-se dedicado a alguém.
Sujeitar-se; tornar-se víctima.
\section{Sacrificativo}
\begin{itemize}
\item {Grp. gram.:adj.}
\end{itemize}
\begin{itemize}
\item {Proveniência:(De \textunderscore sacrificar\textunderscore )}
\end{itemize}
Próprio para o sacrifício.
\section{Sacrificatório}
\begin{itemize}
\item {Grp. gram.:adj.}
\end{itemize}
O mesmo que \textunderscore sacrifical\textunderscore .
\section{Sacrificável}
\begin{itemize}
\item {Grp. gram.:adj.}
\end{itemize}
Que se póde sacrificar.
\section{Sacrífice}
\begin{itemize}
\item {Grp. gram.:m.}
\end{itemize}
\begin{itemize}
\item {Proveniência:(Do lat. \textunderscore sacrum\textunderscore  + \textunderscore facere\textunderscore )}
\end{itemize}
O mesmo que \textunderscore sacrificador\textunderscore . Cf. Latino, \textunderscore Elogios\textunderscore , 204.
\section{Sacrificial}
\begin{itemize}
\item {Grp. gram.:adj.}
\end{itemize}
Relativo ao sacrifício ou á offerta que se fazia aos deuses.
\section{Sacrifício}
\begin{itemize}
\item {Grp. gram.:m.}
\end{itemize}
\begin{itemize}
\item {Proveniência:(Lat. \textunderscore sacrificium\textunderscore )}
\end{itemize}
Offerta de víctimas ou de donativos á divindade, entre os Hebreus.
Aquillo que se offerecia aos deuses, no Paganismo.
A morte de Christo.
A Missa.
Acto de alguém se sacrificar por uma coisa ou pessôa.
Privações, a que alguém se sujeita, em benefício de outrem.
Renúncia; isenção; abnegação.
\section{Sacrífico}
\begin{itemize}
\item {Grp. gram.:m.  e  adj.}
\end{itemize}
\begin{itemize}
\item {Utilização:Poét.}
\end{itemize}
\begin{itemize}
\item {Proveniência:(Lat. \textunderscore sacrificus\textunderscore )}
\end{itemize}
O mesmo que \textunderscore sacrificador\textunderscore .
\section{Sacrifículo}
\begin{itemize}
\item {Grp. gram.:m.}
\end{itemize}
\begin{itemize}
\item {Proveniência:(Lat. \textunderscore sacrificulus\textunderscore )}
\end{itemize}
Aquelle que ajudava o sacrificador de víctimas.
Acólyto.
\section{Sacrilegamente}
\begin{itemize}
\item {Grp. gram.:adv.}
\end{itemize}
De modo sacrílego; com sacrilégio; com profanação.
\section{Sacrilégio}
\begin{itemize}
\item {Grp. gram.:m.}
\end{itemize}
\begin{itemize}
\item {Utilização:Ext.}
\end{itemize}
\begin{itemize}
\item {Utilização:Ant.}
\end{itemize}
\begin{itemize}
\item {Proveniência:(Lat. \textunderscore sacrilegium\textunderscore )}
\end{itemize}
Acto do impiedade, com que se profanam as coisas sagradas.
Acção, com que se ultraja pessôa sagrada ou venerável.
Acto irreligioso.
Acção condemnável.
Multa pecuniária, que os excommungados pagavam.
\section{Sacrílego}
\begin{itemize}
\item {Grp. gram.:adj.}
\end{itemize}
\begin{itemize}
\item {Utilização:Jur.}
\end{itemize}
\begin{itemize}
\item {Proveniência:(Lat. \textunderscore sacrilegus\textunderscore )}
\end{itemize}
Relativo a sacrilégio.
Que commete sacrilégio.
Diz-se do filho do padre, ou de outras pessôas que tenham feito voto de castidade.
\section{Sacripanta}
\begin{itemize}
\item {Grp. gram.:m. ,  f.  e  adj.}
\end{itemize}
\begin{itemize}
\item {Grp. gram.:M.}
\end{itemize}
\begin{itemize}
\item {Utilização:Fam.}
\end{itemize}
\begin{itemize}
\item {Proveniência:(De \textunderscore Sacripante\textunderscore , n. p.)}
\end{itemize}
Pessôa desprezível; bigorrilha; sevandija.
Beato fingido.
\section{Sacripante}
\begin{itemize}
\item {Grp. gram.:m. ,  f.  e  adj.}
\end{itemize}
\begin{itemize}
\item {Utilização:Burl.}
\end{itemize}
\begin{itemize}
\item {Grp. gram.:M.}
\end{itemize}
\begin{itemize}
\item {Utilização:Fam.}
\end{itemize}
\begin{itemize}
\item {Proveniência:(De \textunderscore Sacripante\textunderscore , n. p.)}
\end{itemize}
Pessôa desprezível; bigorrilha; sevandija.
Beato fingido.
\section{Sacrismocho}
\begin{itemize}
\item {fónica:mô}
\end{itemize}
\begin{itemize}
\item {Grp. gram.:m.}
\end{itemize}
\begin{itemize}
\item {Utilização:Ant.}
\end{itemize}
\begin{itemize}
\item {Utilização:Fam.}
\end{itemize}
O mesmo que \textunderscore sacristão\textunderscore .
\section{Sacrista}
\begin{itemize}
\item {Grp. gram.:m.}
\end{itemize}
\begin{itemize}
\item {Utilização:Fam.}
\end{itemize}
O mesmo que \textunderscore sacristão\textunderscore .
(B. lat. \textunderscore sacrista\textunderscore )
\section{Sacristã}
\begin{itemize}
\item {Grp. gram.:f.}
\end{itemize}
Mulhér de sacristão.
Mulhér, incumbida da limpeza e arranjos da sacristia.
(Fem. de \textunderscore sacristão\textunderscore )
\section{Sacristan}
\begin{itemize}
\item {Grp. gram.:f.}
\end{itemize}
Mulhér de sacristão.
Mulhér, incumbida da limpeza e arranjos da sacristia.
(Fem. de \textunderscore sacristão\textunderscore )
\section{Sacristania}
\begin{itemize}
\item {Grp. gram.:f.}
\end{itemize}
\begin{itemize}
\item {Proveniência:(De \textunderscore sacristão\textunderscore )}
\end{itemize}
Offício de sacristão ou sacristan.
\section{Sacristão}
\begin{itemize}
\item {Grp. gram.:m.}
\end{itemize}
\begin{itemize}
\item {Proveniência:(Do b. lat. \textunderscore sacristanus\textunderscore )}
\end{itemize}
Homem, que tem a seu cargo o arranjo e guarda da sacristia.
Aquelle que se emprega habitualmente nos arranjos de uma igreja, em ajudar a Missa, etc.
\section{Sacristia}
\begin{itemize}
\item {Grp. gram.:f.}
\end{itemize}
Casa, annexa a uma igreja ou que faz parte della, e em que se guardam os paramentos e outros objectos do culto.
(B. lat. \textunderscore sacristia\textunderscore )
\section{Sacro}
\begin{itemize}
\item {Grp. gram.:adj.}
\end{itemize}
\begin{itemize}
\item {Utilização:Fig.}
\end{itemize}
\begin{itemize}
\item {Utilização:Anat.}
\end{itemize}
\begin{itemize}
\item {Grp. gram.:M.}
\end{itemize}
\begin{itemize}
\item {Proveniência:(Lat. \textunderscore sacer\textunderscore )}
\end{itemize}
O mesmo que \textunderscore sagrado\textunderscore .
Venerável.
Diz-se do osso triangular, que está na parte inferior da columna vertebral.
Relativo a êsse osso.
Osso sacro.
\section{Sacro-coccýgio}
\begin{itemize}
\item {Grp. gram.:adj.}
\end{itemize}
\begin{itemize}
\item {Utilização:Anat.}
\end{itemize}
Que ao mesmo tempo diz respeito ao sacro e ao cóccyx.
\section{Sacro-femural}
\begin{itemize}
\item {Grp. gram.:adj.}
\end{itemize}
\begin{itemize}
\item {Utilização:Anat.}
\end{itemize}
Commum ao sacro e ao fêmur.
\section{Sacro-ilíaco}
\begin{itemize}
\item {Grp. gram.:adj.}
\end{itemize}
Commum ao sacro e ao ôsso ilíaco.
\section{Sacro-lombar}
\begin{itemize}
\item {Grp. gram.:adj.}
\end{itemize}
Relativo ao sacro e ao lombo.
\section{Sacrosanto}
\begin{itemize}
\item {fónica:san}
\end{itemize}
\begin{itemize}
\item {Grp. gram.:adj.}
\end{itemize}
\begin{itemize}
\item {Proveniência:(Lat. \textunderscore sacrosanctus\textunderscore )}
\end{itemize}
Inviolável.
Reconhecido como sagrado.
Sagrado e santo.
\section{Sacro-spinal}
\begin{itemize}
\item {Grp. gram.:adj.}
\end{itemize}
\begin{itemize}
\item {Utilização:Anat.}
\end{itemize}
\begin{itemize}
\item {Proveniência:(Do lat. \textunderscore sacer\textunderscore  + \textunderscore spina\textunderscore )}
\end{itemize}
Commum ao sacro e á espinha dorsal.
\section{Sacrossanto}
\begin{itemize}
\item {Grp. gram.:adj.}
\end{itemize}
\begin{itemize}
\item {Proveniência:(Lat. \textunderscore sacrosanctus\textunderscore )}
\end{itemize}
Inviolável.
Reconhecido como sagrado.
Sagrado e santo.
\section{Sacubaré}
\begin{itemize}
\item {Grp. gram.:m.}
\end{itemize}
Planta do Brasil, espécie de musgo.
\section{Sacro-vertebral}
\begin{itemize}
\item {Grp. gram.:adj.}
\end{itemize}
\begin{itemize}
\item {Utilização:Anat.}
\end{itemize}
Commum ao sacro e ás vértebras.
\section{Sacudida}
\begin{itemize}
\item {Grp. gram.:f.}
\end{itemize}
O mesmo que \textunderscore sacudidura\textunderscore .
\section{Sacudidamente}
\begin{itemize}
\item {Grp. gram.:adv.}
\end{itemize}
De modo sacudido.
Desembaraçadamente.
De súbito.
Violentamente.
\section{Sacudidela}
\begin{itemize}
\item {Grp. gram.:f.}
\end{itemize}
\begin{itemize}
\item {Utilização:Fam.}
\end{itemize}
\begin{itemize}
\item {Proveniência:(De \textunderscore sacudir\textunderscore )}
\end{itemize}
O mesmo que \textunderscore sacudidura\textunderscore .
Pequena tunda ou sova.
\section{Sacudido}
\begin{itemize}
\item {Grp. gram.:adj.}
\end{itemize}
\begin{itemize}
\item {Utilização:Bras. de Minas}
\end{itemize}
\begin{itemize}
\item {Proveniência:(De \textunderscore sacudir\textunderscore )}
\end{itemize}
Agitado, movido em direcções oppostas.
Desembaraçado: \textunderscore a pequena tem modos sacudidos\textunderscore .
Galhardo, esbelto, formoso.
\section{Sacudidor}
\begin{itemize}
\item {Grp. gram.:m.  e  adj.}
\end{itemize}
O que sacode.
\section{Sacudidura}
\begin{itemize}
\item {Grp. gram.:f.}
\end{itemize}
Acto ou effeito de sacudir.
\section{Sacudimento}
\begin{itemize}
\item {Grp. gram.:m.}
\end{itemize}
O mesmo que \textunderscore sacudidura\textunderscore .
\section{Sacudir}
\begin{itemize}
\item {Grp. gram.:v. t.}
\end{itemize}
\begin{itemize}
\item {Proveniência:(Do lat. \textunderscore succutere\textunderscore )}
\end{itemize}
Agitar de novo.
Agitar repetidas vezes.
Mover muitas vezes em direcções oppostas; abanar: \textunderscore o vento sacode as árvores\textunderscore .
Abalar: \textunderscore o terremoto sacudiu o prédio\textunderscore .
Repellir.
Atirar.
Limpar, agitando: \textunderscore sacudir o tapête\textunderscore .
\section{Sacular}
\begin{itemize}
\item {Grp. gram.:adj.}
\end{itemize}
Relativo a sáculo.
\section{Saculiforme}
\begin{itemize}
\item {Grp. gram.:adj.}
\end{itemize}
\begin{itemize}
\item {Proveniência:(De \textunderscore sáculo\textunderscore  + \textunderscore fórma\textunderscore )}
\end{itemize}
Que tem fórma de sáculo.
\section{Sáculo}
\begin{itemize}
\item {Grp. gram.:m.}
\end{itemize}
\begin{itemize}
\item {Utilização:Bot.}
\end{itemize}
\begin{itemize}
\item {Proveniência:(De \textunderscore saco\textunderscore )}
\end{itemize}
Pequeno saco ou bolsa, que cobre a radícula de certos embryões.
\section{Sacupema}
\begin{itemize}
\item {Grp. gram.:f.}
\end{itemize}
Ave gallinácea da América.
\section{Sacureu}
\begin{itemize}
\item {Grp. gram.:m.}
\end{itemize}
Antigo dignitário chinês.
\section{Sacuubarana}
\begin{itemize}
\item {Grp. gram.:f.}
\end{itemize}
O mesmo que \textunderscore sacaubarana\textunderscore .
\section{Sada}
\begin{itemize}
\item {Grp. gram.:f.}
\end{itemize}
\begin{itemize}
\item {Utilização:Prov.}
\end{itemize}
\begin{itemize}
\item {Utilização:trasm.}
\end{itemize}
O mesmo que \textunderscore enxada\textunderscore .
\section{Sadão}
\begin{itemize}
\item {Grp. gram.:m.}
\end{itemize}
\begin{itemize}
\item {Utilização:Prov.}
\end{itemize}
\begin{itemize}
\item {Utilização:trasm.}
\end{itemize}
O mesmo que \textunderscore enxadão\textunderscore .
\section{Sàdiamente}
\begin{itemize}
\item {Grp. gram.:adv.}
\end{itemize}
De modo sàdio; com saúde.
Vigorosamente.
\section{Sádico}
\begin{itemize}
\item {Grp. gram.:adj.}
\end{itemize}
Que soffre sàdismo.
\section{Sàdio}
\begin{itemize}
\item {Grp. gram.:adj.}
\end{itemize}
\begin{itemize}
\item {Utilização:T. de Aveiro}
\end{itemize}
\begin{itemize}
\item {Proveniência:(Do lat. \textunderscore sanativus\textunderscore )}
\end{itemize}
Que dá saúde ou a auxilia.
Hygiênico.
Que tem bôa saúde: \textunderscore homem sàdio\textunderscore .
\textunderscore Mal sadio\textunderscore , o mesmo que \textunderscore doente\textunderscore .
\section{Sàdismo}
\begin{itemize}
\item {Grp. gram.:m.}
\end{itemize}
\begin{itemize}
\item {Proveniência:(De \textunderscore Sade\textunderscore , n. p. de um vicioso Marquês, que escandalizou a França no séc. XVIII)}
\end{itemize}
Perversão de quem sente volúpia genital, acompanhada de excessos, violências ou sevícias, exercidas em pessôas de qualquer sexo, animaes ou objectos.
\section{Sàdista}
\begin{itemize}
\item {Grp. gram.:m.}
\end{itemize}
Aquelle que tem a perversão do sàdismo.
\section{Sadrá}
\begin{itemize}
\item {Grp. gram.:f.}
\end{itemize}
Grande árvore, (\textunderscore pentaptera glabra\textunderscore ), com cuja casca os pescadores indianos pintam as suas redes, e cujo tronco, reduzido a cinza, serve para curtimento de pelles.
O mesmo que \textunderscore sandará\textunderscore ?
\section{Sadrá}
\begin{itemize}
\item {Grp. gram.:m.}
\end{itemize}
Espécie de camisa finíssima, que os Parses usam sôbre a pelle.
\section{Saduceísmo}
\begin{itemize}
\item {Grp. gram.:m.}
\end{itemize}
Seita dos Saduceus.
\section{Saduceu}
\begin{itemize}
\item {Grp. gram.:m.}
\end{itemize}
\begin{itemize}
\item {Proveniência:(Do hebr. \textunderscore çaddig\textunderscore )}
\end{itemize}
Membro de uma seita judaica, que negava a immortalidade da alma.
\section{Saêta}
\begin{itemize}
\item {Grp. gram.:f.}
\end{itemize}
Tecido antigo.
\section{Safa!}
\begin{itemize}
\item {Grp. gram.:interj.}
\end{itemize}
(designativa de repugnância ou admiração) Cf. Rebello, \textunderscore Contos e Lendas\textunderscore , 91.
\section{Safa-cabos!}
\begin{itemize}
\item {Grp. gram.:interj.}
\end{itemize}
Voz, com que, a bordo, se manda colher os cabos, amarras, etc., depois da manobra.
\section{Safado}
\begin{itemize}
\item {Grp. gram.:adj.}
\end{itemize}
\begin{itemize}
\item {Utilização:Pop.}
\end{itemize}
\begin{itemize}
\item {Grp. gram.:M.}
\end{itemize}
Gasto ou deteriorado pelo uso; cotiado: \textunderscore casaco safado\textunderscore .
Desavergonhado.
Homem vil, desprezível.
\section{Safanão}
\begin{itemize}
\item {Grp. gram.:m.}
\end{itemize}
\begin{itemize}
\item {Utilização:Pop.}
\end{itemize}
Acto de safar.
Bofetada; empurrão; sacudidura.
\section{Safão}
\begin{itemize}
\item {Grp. gram.:m.}
\end{itemize}
\begin{itemize}
\item {Utilização:Ant.}
\end{itemize}
Espécie de tecido.
\section{Safar}
\begin{itemize}
\item {Grp. gram.:v. t.}
\end{itemize}
\begin{itemize}
\item {Grp. gram.:V. p.}
\end{itemize}
\begin{itemize}
\item {Utilização:Fam.}
\end{itemize}
Tirar, puxando.
Tirar.
Extrahir.
Subtrahir; furtar.
Livrar, desembaraçar.
Gastar pelo attrito ou uso.
Desgastar.
Escapar, fugir.
(Cp. cast. \textunderscore zafar\textunderscore )
\section{Sáfara}
\begin{itemize}
\item {Grp. gram.:f.}
\end{itemize}
\begin{itemize}
\item {Proveniência:(De \textunderscore sáfaro\textunderscore )}
\end{itemize}
Terreno sáfaro; penhasco.
\section{Safardana}
\begin{itemize}
\item {Grp. gram.:m.}
\end{itemize}
\begin{itemize}
\item {Utilização:Burl.}
\end{itemize}
\begin{itemize}
\item {Proveniência:(De \textunderscore safado\textunderscore ?)}
\end{itemize}
Bisbórria; biltre, pulha.
\section{Safaria}
\begin{itemize}
\item {Grp. gram.:adj. f.}
\end{itemize}
\begin{itemize}
\item {Proveniência:(Do ár. \textunderscore safari\textunderscore ?)}
\end{itemize}
Diz-se de uma casta de roman, de bagos grandes e quadrados.
\section{Sáfaro}
\begin{itemize}
\item {Grp. gram.:adj.}
\end{itemize}
\begin{itemize}
\item {Utilização:Des.}
\end{itemize}
\begin{itemize}
\item {Proveniência:(Do ár. \textunderscore çahra\textunderscore , deserto)}
\end{itemize}
Inculto, agreste.
Estéril.
Bravo.
Estranho, alheio, distante.
Mal morigerado. Cf. Sousa, \textunderscore Vida do Arceb.\textunderscore , 121.
\section{Safa-safa}
\begin{itemize}
\item {Grp. gram.:f.}
\end{itemize}
\begin{itemize}
\item {Proveniência:(De \textunderscore safar\textunderscore )}
\end{itemize}
Arranjo e accommodação das coisas contidas num navio, para pôr a artilharia em estado de combate.
\section{Safata}
\begin{itemize}
\item {Grp. gram.:f.}
\end{itemize}
Peixe marítimo da costa de Portugal.
(Corr. de \textunderscore açafate\textunderscore ?)
\section{Sáfeo}
\begin{itemize}
\item {Grp. gram.:adj.}
\end{itemize}
\begin{itemize}
\item {Utilização:Ant.}
\end{itemize}
Reles, desprezível, o mesmo que \textunderscore sáfio\textunderscore :«\textunderscore ...se não fosse casar ante co mais sáfeo bargante que come pão e cebola.\textunderscore »G. Vicente, \textunderscore Inês Pereira\textunderscore .
\section{Sáfio}
\begin{itemize}
\item {Grp. gram.:adj.}
\end{itemize}
\begin{itemize}
\item {Proveniência:(Do ár. \textunderscore jafi\textunderscore )}
\end{itemize}
O mesmo que \textunderscore sáfaro\textunderscore .
Grosseiro; rude.
Ordinário; vil.
\section{Safio}
\begin{itemize}
\item {Grp. gram.:m.}
\end{itemize}
\begin{itemize}
\item {Proveniência:(Do ár. \textunderscore safio\textunderscore , de \textunderscore seflon\textunderscore , lugar fundo? Cf. Sousa, \textunderscore Vest. da Ling. Aráb.\textunderscore )}
\end{itemize}
Pequeno congro.
\section{Safio}
\begin{itemize}
\item {Grp. gram.:adj.}
\end{itemize}
\begin{itemize}
\item {Utilização:Prov.}
\end{itemize}
Diz-se dos chibos e cabras de pêlo curto.
\section{Safo}
\begin{itemize}
\item {Grp. gram.:adj.}
\end{itemize}
\begin{itemize}
\item {Grp. gram.:M.}
\end{itemize}
\begin{itemize}
\item {Utilização:Gír.}
\end{itemize}
\begin{itemize}
\item {Proveniência:(De \textunderscore safar\textunderscore )}
\end{itemize}
Que se safou.
Que escapou; livre.
Lenço.
\section{Safões}
\begin{itemize}
\item {Grp. gram.:m. pl.}
\end{itemize}
(V.çafões)
\section{Saforil}
\begin{itemize}
\item {Grp. gram.:m.}
\end{itemize}
\begin{itemize}
\item {Utilização:Prov.}
\end{itemize}
\begin{itemize}
\item {Utilização:trasm.}
\end{itemize}
Pessôa ordinária mas espevitada.
Animal reles.
(Relaciona-se com \textunderscore sáfaro\textunderscore ?)
\section{Safra}
\begin{itemize}
\item {Grp. gram.:f.}
\end{itemize}
Espécie de bigorna grande, com uma ponta só.
\section{Safra}
\begin{itemize}
\item {Grp. gram.:f.}
\end{itemize}
Colheita.
Bôa nascença de frutos.
\section{Safra}
\begin{itemize}
\item {Grp. gram.:f.}
\end{itemize}
\begin{itemize}
\item {Proveniência:(Do ár. \textunderscore çafr\textunderscore )}
\end{itemize}
Pó de um ácido de cobalto, próprio para a fabricação do vidro azul.
\section{Safradeira}
\begin{itemize}
\item {Grp. gram.:f.}
\end{itemize}
\begin{itemize}
\item {Proveniência:(De \textunderscore safra\textunderscore ^1)}
\end{itemize}
Instrumento, sôbre que se furam as ferraduras e se abre o ôlho das enxadas e de outros utensílios, na falta de bigorna com furos. Cf. Leon., \textunderscore Arte de ferrar\textunderscore , 68.
\section{Safranina}
\begin{itemize}
\item {Grp. gram.:f.}
\end{itemize}
Um dos productos da hulha.
(Cp. \textunderscore açafrão\textunderscore )
\section{Safrão}
\begin{itemize}
\item {Grp. gram.:m.}
\end{itemize}
\begin{itemize}
\item {Utilização:Náut.}
\end{itemize}
Peça, que se ajusta ao leme, para o reforçar ou para lhe facilitar o movimento.
(Relaciona-se com \textunderscore safra\textunderscore ^1?)
\section{Safre}
\begin{itemize}
\item {Grp. gram.:m.}
\end{itemize}
\begin{itemize}
\item {Utilização:Miner.}
\end{itemize}
Óxydo de cobalto; o mesmo que \textunderscore safra\textunderscore ^3.
\section{Safré}
\begin{itemize}
\item {Grp. gram.:m.}
\end{itemize}
\begin{itemize}
\item {Utilização:T. santomense}
\end{itemize}
Fruto do safueiro e que em Angola se diz \textunderscore mubafo\textunderscore .
(Não haveria êrro, por \textunderscore safué\textunderscore ?)
\section{Safu}
\begin{itemize}
\item {Grp. gram.:m.}
\end{itemize}
Nome santhomense do mubafo.
\section{Safucala}
\begin{itemize}
\item {Grp. gram.:f.}
\end{itemize}
Árvore do Congo.
\section{Safu-de-obó}
\begin{itemize}
\item {Grp. gram.:m.}
\end{itemize}
Árvore medicinal da ilha de San-Thomé, o mesmo que \textunderscore safueiro\textunderscore .
\section{Safueiro}
\begin{itemize}
\item {Grp. gram.:m.}
\end{itemize}
Árvore da ilha do San-Thomé, (\textunderscore canarium edule\textunderscore ), o mesmo que \textunderscore mubafo\textunderscore  em Angola.
\section{Safuta}
\begin{itemize}
\item {Grp. gram.:f.}
\end{itemize}
Árvore de Moçambique.
\section{Saga}
\begin{itemize}
\item {Grp. gram.:f.}
\end{itemize}
\begin{itemize}
\item {Proveniência:(Al. \textunderscore sage\textunderscore )}
\end{itemize}
Qualquer lenda escandinava.
Canção baseada em alguma dessas lendas.
\section{Saga}
\begin{itemize}
\item {Grp. gram.:f.}
\end{itemize}
\begin{itemize}
\item {Proveniência:(Lat. \textunderscore saga\textunderscore )}
\end{itemize}
Bruxa ou feiticeira, entre os Romanos.
\section{Sagaçaria}
\begin{itemize}
\item {Grp. gram.:f.}
\end{itemize}
\begin{itemize}
\item {Utilização:Ant.}
\end{itemize}
O mesmo que \textunderscore sagacidade\textunderscore .
(Cp. \textunderscore sagaz\textunderscore )
\section{Sagaceza}
\begin{itemize}
\item {Grp. gram.:f.}
\end{itemize}
\begin{itemize}
\item {Utilização:Ant.}
\end{itemize}
O mesmo que \textunderscore sagacidade\textunderscore .
\section{Sagácia}
\begin{itemize}
\item {Grp. gram.:f.}
\end{itemize}
O mesmo que \textunderscore sagacidade\textunderscore .
\section{Sagacidade}
\begin{itemize}
\item {Grp. gram.:f.}
\end{itemize}
\begin{itemize}
\item {Proveniência:(Do lat. \textunderscore sagacitas\textunderscore )}
\end{itemize}
Qualidade do que é sagaz.
Agudeza de espirito ou de intelligência.
Intelligência perspicaz.
Perspicácia; finura.
\section{Saganha}
\begin{itemize}
\item {Grp. gram.:f.}
\end{itemize}
O mesmo que \textunderscore saganho\textunderscore .
\section{Saganho}
\begin{itemize}
\item {Grp. gram.:m.}
\end{itemize}
\begin{itemize}
\item {Utilização:Prov.}
\end{itemize}
\begin{itemize}
\item {Utilização:minh.}
\end{itemize}
Planta vulgar nos montados, (\textunderscore cistus hirsutus\textunderscore , Lin.).
\section{Saganho-oiro}
\begin{itemize}
\item {Grp. gram.:m.}
\end{itemize}
\begin{itemize}
\item {Utilização:Prov.}
\end{itemize}
\begin{itemize}
\item {Utilização:minh.}
\end{itemize}
Espécie de planta, (\textunderscore helianthemum olyssoides\textunderscore ).
\section{Sagão}
\begin{itemize}
\item {Grp. gram.:m.}
\end{itemize}
O mesmo que \textunderscore saguão\textunderscore . Cf. Filinto, XIV, 49.
\section{Sagapejo}
\begin{itemize}
\item {Grp. gram.:m.}
\end{itemize}
O mesmo que \textunderscore sagapeno\textunderscore .
\section{Sagapeno}
\begin{itemize}
\item {Grp. gram.:m.}
\end{itemize}
\begin{itemize}
\item {Proveniência:(Lat. \textunderscore sagapenon\textunderscore )}
\end{itemize}
Espécie de resina, que entra na composição de alguns emplastros.
\section{Sagaz}
\begin{itemize}
\item {Grp. gram.:adj.}
\end{itemize}
\begin{itemize}
\item {Utilização:Prov.}
\end{itemize}
\begin{itemize}
\item {Proveniência:(Lat. \textunderscore sagax\textunderscore )}
\end{itemize}
Que tem agudeza ou penetração de espírito.
Perspicaz; fino.
Astuto; solerte.
Reservado, capaz de guardar segrêdo.
\section{Sagazmente}
\begin{itemize}
\item {Grp. gram.:adv.}
\end{itemize}
De modo sagaz; com sagacidade; com finura.
\section{Sage}
\begin{itemize}
\item {Grp. gram.:adj.}
\end{itemize}
O mesmo que \textunderscore sages\textunderscore . Cf. Arn. Gama, \textunderscore Última Dona\textunderscore , 56.
\section{Sageira}
\begin{itemize}
\item {Grp. gram.:f.}
\end{itemize}
\begin{itemize}
\item {Utilização:Ant.}
\end{itemize}
O mesmo que \textunderscore sajaria\textunderscore .
\section{Ságena}
\begin{itemize}
\item {Grp. gram.:f.}
\end{itemize}
\begin{itemize}
\item {Proveniência:(Do ár. \textunderscore sijn\textunderscore )}
\end{itemize}
Prisão de Christãos cativos, entre Moiros.
\section{Sagenária}
\begin{itemize}
\item {Grp. gram.:f.}
\end{itemize}
Gênero de plantas fósseis.
\section{Sagenito}
\begin{itemize}
\item {Grp. gram.:m.}
\end{itemize}
\begin{itemize}
\item {Utilização:Miner.}
\end{itemize}
Variedade reticulada de óxydo de titano.
\section{Sagerécia}
\begin{itemize}
\item {Grp. gram.:f.}
\end{itemize}
Gênero de plantas rhamnáceas.
\section{Sages}
\begin{itemize}
\item {Grp. gram.:adj.}
\end{itemize}
\begin{itemize}
\item {Utilização:Ant.}
\end{itemize}
Discreto, prudente.
(Cp. fr. \textunderscore sage\textunderscore )
\section{Sagesmente}
\begin{itemize}
\item {Grp. gram.:adv.}
\end{itemize}
\begin{itemize}
\item {Utilização:Ant.}
\end{itemize}
\begin{itemize}
\item {Proveniência:(De \textunderscore sages\textunderscore )}
\end{itemize}
Com sabedoria, com prudência, com juízo.
\section{Sagez}
\begin{itemize}
\item {Grp. gram.:adj.}
\end{itemize}
\begin{itemize}
\item {Utilização:Ant.}
\end{itemize}
Que tem sageza; o mesmo que \textunderscore sages\textunderscore .
\section{Sageza}
\begin{itemize}
\item {Grp. gram.:f.}
\end{itemize}
\begin{itemize}
\item {Utilização:Ant.}
\end{itemize}
O mesmo que \textunderscore sajaria\textunderscore .
\section{Sagina}
\begin{itemize}
\item {Grp. gram.:f.}
\end{itemize}
\begin{itemize}
\item {Utilização:Ant.}
\end{itemize}
Acto de saginar.
\section{Saginar}
\begin{itemize}
\item {Grp. gram.:v. t.}
\end{itemize}
\begin{itemize}
\item {Utilização:Des.}
\end{itemize}
\begin{itemize}
\item {Proveniência:(Lat. \textunderscore saginare\textunderscore )}
\end{itemize}
Tornar gordo, cevar.
\section{Sagionia}
\begin{itemize}
\item {Grp. gram.:f.}
\end{itemize}
\begin{itemize}
\item {Utilização:Ant.}
\end{itemize}
O mesmo que \textunderscore saionaria\textunderscore .
\section{Sagitado}
\begin{itemize}
\item {Grp. gram.:adj.}
\end{itemize}
O mesmo que \textunderscore sagital\textunderscore .
\section{Sagital}
\begin{itemize}
\item {Grp. gram.:adj.}
\end{itemize}
\begin{itemize}
\item {Utilização:Anat.}
\end{itemize}
\begin{itemize}
\item {Proveniência:(Do lat. \textunderscore sagitta\textunderscore )}
\end{itemize}
Que tem fórma de seta.
Diz-se da sutura, correspondente á linha média da abóbada craniana.
Diz-se do plano imaginário, que separa os antímeros.
\section{Sagitária}
\begin{itemize}
\item {Grp. gram.:f.}
\end{itemize}
\begin{itemize}
\item {Proveniência:(Lat. \textunderscore sagittaria\textunderscore )}
\end{itemize}
Planta alismácea.
\section{Sagitário}
\begin{itemize}
\item {Grp. gram.:adj.}
\end{itemize}
\begin{itemize}
\item {Grp. gram.:M.}
\end{itemize}
\begin{itemize}
\item {Proveniência:(Lat. \textunderscore saggitarius\textunderscore )}
\end{itemize}
O mesmo que \textunderscore sagitífero\textunderscore .
Constelação representada sob a figura de um centauro, que segura um arco retesado e armado de frexa.
\section{Sagitela}
\begin{itemize}
\item {Grp. gram.:f.}
\end{itemize}
\begin{itemize}
\item {Proveniência:(Do lat. \textunderscore sagitta\textunderscore )}
\end{itemize}
Gênero de moluscos nadadores dos mares da América.
\section{Sagitífero}
\begin{itemize}
\item {Grp. gram.:adj.}
\end{itemize}
\begin{itemize}
\item {Utilização:Poét.}
\end{itemize}
\begin{itemize}
\item {Proveniência:(Lat. \textunderscore sagittífer\textunderscore )}
\end{itemize}
Armado de seta.
\section{Sagitifoliado}
\begin{itemize}
\item {Grp. gram.:adj.}
\end{itemize}
\begin{itemize}
\item {Utilização:Bot.}
\end{itemize}
\begin{itemize}
\item {Proveniência:(Do lat. \textunderscore sagitta\textunderscore  + \textunderscore folium\textunderscore )}
\end{itemize}
Que tem fôlhas em fórma de seta.
\section{Sagittado}
\begin{itemize}
\item {Grp. gram.:adj.}
\end{itemize}
O mesmo que \textunderscore sagittal\textunderscore .
\section{Sagittal}
\begin{itemize}
\item {Grp. gram.:adj.}
\end{itemize}
\begin{itemize}
\item {Utilização:Anat.}
\end{itemize}
\begin{itemize}
\item {Proveniência:(Do lat. \textunderscore sagitta\textunderscore )}
\end{itemize}
Que tem fórma de seta.
Diz-se da sutura, correspondente á linha média da abóbada craniana.
Diz-se do plano imaginário, que separa os antímeros.
\section{Sagittária}
\begin{itemize}
\item {Grp. gram.:f.}
\end{itemize}
\begin{itemize}
\item {Proveniência:(Lat. \textunderscore sagittaria\textunderscore )}
\end{itemize}
Planta alismácea.
\section{Sagittario}
\begin{itemize}
\item {Grp. gram.:adj.}
\end{itemize}
\begin{itemize}
\item {Grp. gram.:M.}
\end{itemize}
\begin{itemize}
\item {Proveniência:(Lat. \textunderscore saggitarius\textunderscore )}
\end{itemize}
O mesmo que \textunderscore sagittífero\textunderscore .
Constellação representada sob a figura de um centauro, que segura um arco retesado e armado de frexa.
\section{Sagitella}
\begin{itemize}
\item {Grp. gram.:f.}
\end{itemize}
\begin{itemize}
\item {Proveniência:(Do lat. \textunderscore sagitta\textunderscore )}
\end{itemize}
Gênero de molluscos nadadores dos mares da América.
\section{Sagittífero}
\begin{itemize}
\item {Grp. gram.:adj.}
\end{itemize}
\begin{itemize}
\item {Utilização:Poét.}
\end{itemize}
\begin{itemize}
\item {Proveniência:(Lat. \textunderscore sagittífer\textunderscore )}
\end{itemize}
Armado de seta.
\section{Sagittifoliado}
\begin{itemize}
\item {Grp. gram.:adj.}
\end{itemize}
\begin{itemize}
\item {Utilização:Bot.}
\end{itemize}
\begin{itemize}
\item {Proveniência:(Do lat. \textunderscore sagitta\textunderscore  + \textunderscore folium\textunderscore )}
\end{itemize}
Que tem fôlhas em fórma de seta.
\section{Sago}
\begin{itemize}
\item {Grp. gram.:m.}
\end{itemize}
\begin{itemize}
\item {Utilização:Ant.}
\end{itemize}
\begin{itemize}
\item {Proveniência:(Lat. \textunderscore sagum\textunderscore )}
\end{itemize}
Espécie de saio, que se usava por cima da armadura até o joelho. Cf. Herculano, \textunderscore Hist. de Port.\textunderscore , I.
\section{Sagotar}
\begin{itemize}
\item {Grp. gram.:v. t.}
\end{itemize}
\begin{itemize}
\item {Utilização:Prov.}
\end{itemize}
Cortar com sagote.
\section{Sagote}
\begin{itemize}
\item {Grp. gram.:m.}
\end{itemize}
O mesmo que \textunderscore corta-palha\textunderscore .
(Por \textunderscore segote\textunderscore , de \textunderscore segar\textunderscore )
\section{Sagra}
\begin{itemize}
\item {Grp. gram.:f.}
\end{itemize}
\begin{itemize}
\item {Proveniência:(Do lat. \textunderscore sacra\textunderscore )}
\end{itemize}
Festa de San-Domingos, em Cascaes.
\section{Sagração}
\begin{itemize}
\item {Grp. gram.:f.}
\end{itemize}
Acto ou effeito de sagrar; consagração.
(Oo lat. \textunderscore sacratio\textunderscore )
\section{Sagradamente}
\begin{itemize}
\item {Grp. gram.:adv.}
\end{itemize}
De modo sagrado.
Segundo o rito ou fórmulas religiosas.
\section{Sagrado}
\begin{itemize}
\item {Grp. gram.:adj.}
\end{itemize}
\begin{itemize}
\item {Grp. gram.:M.}
\end{itemize}
\begin{itemize}
\item {Utilização:Prov.}
\end{itemize}
\begin{itemize}
\item {Utilização:trasm.}
\end{itemize}
\begin{itemize}
\item {Proveniência:(De \textunderscore sagrar\textunderscore )}
\end{itemize}
Relativo aos ritos ou ao culto religioso.
Profundamente venerável.
Inviolável.
A que se deve o maior respeito.
Puro, santo.
Aquillo que é sagrado.
Adro da igreja.
\section{Sagral}
\begin{itemize}
\item {Grp. gram.:adj.}
\end{itemize}
\begin{itemize}
\item {Utilização:Ant.}
\end{itemize}
O mesmo que \textunderscore secular\textunderscore .
(Por \textunderscore segral\textunderscore , de \textunderscore segre\textunderscore )
\section{Sagrar}
\begin{itemize}
\item {Grp. gram.:v. t.}
\end{itemize}
\begin{itemize}
\item {Proveniência:(Do lat. \textunderscore sacrare\textunderscore )}
\end{itemize}
Dedicar a Deus ou aos deuses.
Dedicar ao serviço divino.
Consagrar.
Tornar bento, benzer.
Santificar.
Tornar venerado, venerar.
Conferir uma dignidade a, por meio do ceremónias religiosas.
Dedicar: \textunderscore sagrar muito tempo ao estudo\textunderscore .
\section{Sagre}
\begin{itemize}
\item {Grp. gram.:m.}
\end{itemize}
O mesmo que \textunderscore sacre\textunderscore . Cf. Garção, II, 40.
\section{Ságrea}
\begin{itemize}
\item {Grp. gram.:f.}
\end{itemize}
Gênero de plantas melastomáceas.
\section{Sagres}
\begin{itemize}
\item {Grp. gram.:m.}
\end{itemize}
\begin{itemize}
\item {Utilização:Ant.}
\end{itemize}
\begin{itemize}
\item {Proveniência:(Do ár. \textunderscore sacron\textunderscore ?)}
\end{itemize}
Variedade de peça de artilharia.
\section{Sagro}
\begin{itemize}
\item {Grp. gram.:m.}
\end{itemize}
\begin{itemize}
\item {Utilização:Prov.}
\end{itemize}
\begin{itemize}
\item {Utilização:dur.}
\end{itemize}
Fundo chato dos barcos rabelos, formado de tabuões de pinho, sendo de castanheiro quási toda a outra madeira dêsses barcos.
\section{Sagu}
\begin{itemize}
\item {Grp. gram.:m.}
\end{itemize}
Substância amylácea, extrahida da parte central das hastes de algumas palmeiras.
Espécie de licor, destillado pelos ramos de palmeira, e usado na Índia.
Substância farinácea, extrahida de algumas plantas e de que fazem pão os Japoneses.
(Mal. \textunderscore sagu\textunderscore )
\section{Saguão}
\begin{itemize}
\item {Grp. gram.:m.}
\end{itemize}
Pátio estreito e descoberto, no interior de um edifício.
Espécie de alpendre, á entrada dos conventos.
(Cast. \textunderscore zaguan\textunderscore )
\section{Saguaragi}
\begin{itemize}
\item {Grp. gram.:m.}
\end{itemize}
Árvore brasileira, própria para construcções.
\section{Saguate}
\begin{itemize}
\item {Grp. gram.:m.}
\end{itemize}
\begin{itemize}
\item {Utilização:Des.}
\end{itemize}
\begin{itemize}
\item {Proveniência:(Do conc. \textunderscore saguvat\textunderscore )}
\end{itemize}
Presente, donativo.
\section{Sagucho}
\begin{itemize}
\item {Grp. gram.:m.}
\end{itemize}
\begin{itemize}
\item {Utilização:Prov.}
\end{itemize}
\begin{itemize}
\item {Utilização:dur.}
\end{itemize}
O mesmo que \textunderscore saguncho\textunderscore .
\section{Sagueiro}
\begin{itemize}
\item {fónica:gu-ei}
\end{itemize}
\begin{itemize}
\item {Grp. gram.:m.}
\end{itemize}
Espécie de palmeira, que produz sagu.
\section{Sagueza}
\begin{itemize}
\item {Grp. gram.:f.}
\end{itemize}
\begin{itemize}
\item {Utilização:Ant.}
\end{itemize}
O mesmo que \textunderscore sagacidade\textunderscore .
\section{Saguí}
\begin{itemize}
\item {fónica:gu-i}
\end{itemize}
\begin{itemize}
\item {Grp. gram.:m.}
\end{itemize}
Pequeno macaco, de cauda felpuda e comprida, (\textunderscore geopithecus\textunderscore ).
\section{Saguim}
\begin{itemize}
\item {fónica:gu-im}
\end{itemize}
\begin{itemize}
\item {Grp. gram.:m.}
\end{itemize}
Pequeno macaco, de cauda felpuda e comprida, (\textunderscore geopithecus\textunderscore ).
\section{Sagum}
\begin{itemize}
\item {Grp. gram.:m.}
\end{itemize}
O mesmo que \textunderscore sagu\textunderscore . Cf. Barros, \textunderscore Déc.\textunderscore . III, l. III, c. 5.
\section{Saguncho}
\begin{itemize}
\item {Grp. gram.:m.}
\end{itemize}
O mesmo que \textunderscore peixe-pau\textunderscore .
\section{Saguntinos}
\begin{itemize}
\item {Grp. gram.:m. pl.}
\end{itemize}
\begin{itemize}
\item {Proveniência:(Lat. \textunderscore Saguntini\textunderscore )}
\end{itemize}
Habitantes da antiga cidade de Sagunto, hoje Murviedro, em Espanha.
\section{Saguvate}
\begin{itemize}
\item {Grp. gram.:m.}
\end{itemize}
\begin{itemize}
\item {Utilização:T. de Ceilão}
\end{itemize}
O mesmo que \textunderscore saguate\textunderscore .
\section{Saí}
\begin{itemize}
\item {Grp. gram.:m.}
\end{itemize}
Espécie de macaco, o mesmo que \textunderscore saitaia-chorão\textunderscore .
Gênero de pássaros brasileiros, nocivos aos frutos, e de que há várias espécies, como o \textunderscore saí-bicudo\textunderscore , o \textunderscore saí-de-colleira\textunderscore , o \textunderscore saí-da-sécia\textunderscore , o \textunderscore saí-papagaio\textunderscore , o \textunderscore sai-xê\textunderscore , etc.
O mesmo que \textunderscore saíco\textunderscore .
\section{Sai}
\begin{itemize}
\item {Grp. gram.:m.}
\end{itemize}
O mesmo que \textunderscore bonzo\textunderscore .
\section{Saia}
\begin{itemize}
\item {Grp. gram.:f.}
\end{itemize}
\begin{itemize}
\item {Utilização:Náut.}
\end{itemize}
\begin{itemize}
\item {Utilização:Pop.}
\end{itemize}
\begin{itemize}
\item {Utilização:Ant.}
\end{itemize}
\begin{itemize}
\item {Proveniência:(Do lat. \textunderscore sagum\textunderscore )}
\end{itemize}
Vestuário da mulher, apertado na cintura e pendente até os pés ou quási.
Supplemento ás velas latinas.
A mulhér: \textunderscore o rapaz não gosta de saias\textunderscore .
Saio.
\section{Saiaguês}
\begin{itemize}
\item {Grp. gram.:m.}
\end{itemize}
\begin{itemize}
\item {Proveniência:(De \textunderscore saio\textunderscore )}
\end{itemize}
Homem, que vestia saial.
Campónio.
\section{Saial}
\begin{itemize}
\item {Grp. gram.:m.}
\end{itemize}
\begin{itemize}
\item {Proveniência:(De \textunderscore sáio\textunderscore )}
\end{itemize}
Antiga e grosseira vestidura, para homem ou mulhér.
\section{Saião}
\begin{itemize}
\item {Grp. gram.:m.}
\end{itemize}
Nome de duas plantas crassuláceas.
\section{Saião}
\begin{itemize}
\item {Grp. gram.:m.}
\end{itemize}
\begin{itemize}
\item {Utilização:Ant.}
\end{itemize}
\begin{itemize}
\item {Grp. gram.:Adj.}
\end{itemize}
\begin{itemize}
\item {Proveniência:(De \textunderscore saio\textunderscore )}
\end{itemize}
Verdugo, algoz.
Insolente, petulante.
\section{Saias}
\begin{itemize}
\item {Grp. gram.:f. pl.}
\end{itemize}
\begin{itemize}
\item {Utilização:Prov.}
\end{itemize}
\begin{itemize}
\item {Utilização:alent.}
\end{itemize}
Dança popular em Elvas e Campo-Maior.
\section{Saibo}
\begin{itemize}
\item {Grp. gram.:m.}
\end{itemize}
\begin{itemize}
\item {Utilização:Pop.}
\end{itemize}
O mesmo que \textunderscore sabor\textunderscore .
\section{Saibo}
\begin{itemize}
\item {Grp. gram.:m.}
\end{itemize}
\begin{itemize}
\item {Utilização:Ant.}
\end{itemize}
O mesmo que \textunderscore sábio\textunderscore .
\section{Saibramento}
\begin{itemize}
\item {Grp. gram.:m.}
\end{itemize}
Acto de saibrar.
\section{Saibrão}
\begin{itemize}
\item {Grp. gram.:m.}
\end{itemize}
\begin{itemize}
\item {Proveniência:(De \textunderscore saibro\textunderscore )}
\end{itemize}
Terreno argilloso e areento, próprio para plantações de açúcar.
\section{Saibrar}
\begin{itemize}
\item {Grp. gram.:v. t.}
\end{itemize}
Surribar, para plantação de bacellos. Cf. Júl. Moreira, \textunderscore Estudos da Língua Port.\textunderscore , I, 187.
Cobrir de saibro; balaustrar.
\section{Saibreira}
\begin{itemize}
\item {Grp. gram.:f.}
\end{itemize}
Lugar, donde se extrái saibro.
Terreno saibroso.
\section{Saibro}
\begin{itemize}
\item {Grp. gram.:m.}
\end{itemize}
\begin{itemize}
\item {Proveniência:(Do lat. \textunderscore sabulum\textunderscore )}
\end{itemize}
Argilla, misturada com areia e pedras.
Sable.
Uva da Golegan. Cf. \textunderscore Rev. Agron.\textunderscore , I, 18.
\section{Saibro}
\begin{itemize}
\item {Grp. gram.:m.}
\end{itemize}
Uva, o mesmo que \textunderscore sabra\textunderscore . Cf. B. Pereira, \textunderscore Prosódia\textunderscore , vb. \textunderscore sobria\textunderscore .
\section{Saibro-branco}
\begin{itemize}
\item {Grp. gram.:m.}
\end{itemize}
Casta de uva do districto de Leiria.
\section{Saibroso}
\begin{itemize}
\item {Grp. gram.:adj.}
\end{itemize}
Que tem saibro; em que há saibro.
\section{Saíco}
\begin{itemize}
\item {Grp. gram.:m.}
\end{itemize}
Ave canora do Brasil.
\section{Saída}
\begin{itemize}
\item {Grp. gram.:f.}
\end{itemize}
\begin{itemize}
\item {Utilização:Ant.}
\end{itemize}
Acto ou effeito de sair.
Exportação.
Venda, extracção: \textunderscore o polvo tem pouca saída\textunderscore .
Lugar, por onde se sái.
Expediente, recurso: \textunderscore um desastre sem saída\textunderscore .
Sortida.
\section{Saído}
\begin{itemize}
\item {Grp. gram.:adj.}
\end{itemize}
\begin{itemize}
\item {Proveniência:(De \textunderscore sair\textunderscore )}
\end{itemize}
Que anda fóra de casa.
Que anda com cio, (falando-se de animaes).
\section{Saide}
\begin{itemize}
\item {Grp. gram.:m.}
\end{itemize}
Árvore indiana, de fibras têxteis.
\section{Saidor}
\begin{itemize}
\item {fónica:sa-i}
\end{itemize}
\begin{itemize}
\item {Grp. gram.:m.}
\end{itemize}
\begin{itemize}
\item {Utilização:Bras. do S}
\end{itemize}
\begin{itemize}
\item {Proveniência:(De \textunderscore sair\textunderscore )}
\end{itemize}
Cavalleiro, que sai ou fica de pé, caindo o cavallo.
\section{Saieira}
\begin{itemize}
\item {Grp. gram.:f.}
\end{itemize}
\begin{itemize}
\item {Utilização:Bras}
\end{itemize}
Costureira de saias.
\section{Saieta}
\begin{itemize}
\item {fónica:ê}
\end{itemize}
\begin{itemize}
\item {Grp. gram.:f.}
\end{itemize}
\begin{itemize}
\item {Proveniência:(De \textunderscore saia\textunderscore )}
\end{itemize}
Tecido de lan, próprio para forros.
\section{Saieza}
\begin{itemize}
\item {Grp. gram.:f.}
\end{itemize}
(Corr. de \textunderscore sagueza\textunderscore )
\section{Saiga}
\begin{itemize}
\item {Grp. gram.:f.}
\end{itemize}
Antílope da Rússia.
\section{Sail}
\begin{itemize}
\item {Grp. gram.:m.}
\end{itemize}
Azeite de peixe; o mesmo que \textunderscore saim\textunderscore .
\section{Saim}
\begin{itemize}
\item {Grp. gram.:m.}
\end{itemize}
\begin{itemize}
\item {Utilização:Prov.}
\end{itemize}
\begin{itemize}
\item {Utilização:minh.}
\end{itemize}
Óleo de sardinha. (Colhido em Viana)
\section{Saimel}
\begin{itemize}
\item {Grp. gram.:m.}
\end{itemize}
Primeira pedra que, formando a volta de um arco, assenta sôbre um capitel ou uma ombreira.
(Cp. \textunderscore enxaimel\textunderscore )
\section{Saimento}
\begin{itemize}
\item {fónica:sa-i}
\end{itemize}
\begin{itemize}
\item {Grp. gram.:m.}
\end{itemize}
\begin{itemize}
\item {Proveniência:(De \textunderscore sair\textunderscore )}
\end{itemize}
Saída.
Funeral; cortejo fúnebre.
\section{Saindarus}
\begin{itemize}
\item {Grp. gram.:m. pl.}
\end{itemize}
Indígenas do norte do Brasil.
\section{Sainete}
\begin{itemize}
\item {fónica:nê}
\end{itemize}
\begin{itemize}
\item {Grp. gram.:m.}
\end{itemize}
Isca, que se dá aos falcões, para os amansar.
Aquillo que suaviza uma impressão desagradável.
Coisa desagradável.
Gôsto especial; graça: \textunderscore aquelle folhetinista tem sainete\textunderscore .
Remoque, picuínha:«\textunderscore ...assazoando tudo com apodos, e malévolos sainetes.\textunderscore »Filinto, \textunderscore D. Man.\textunderscore , II, 131.
(Cast. \textunderscore sainete\textunderscore )
\section{Saínha}
\begin{itemize}
\item {Grp. gram.:f.}
\end{itemize}
(Corr. de \textunderscore salina\textunderscore )
\section{Saínha}
\begin{itemize}
\item {Grp. gram.:f.}
\end{itemize}
\begin{itemize}
\item {Utilização:T. da Bairrada}
\end{itemize}
Insecto amarelo e comprido, que ataca os milhaes.
\section{Saínho}
\begin{itemize}
\item {Grp. gram.:m.}
\end{itemize}
\begin{itemize}
\item {Utilização:Ant.}
\end{itemize}
Pequeno sáio.
Gibão redondo e sem abas.
\section{Sainte}
\begin{itemize}
\item {Grp. gram.:adj.}
\end{itemize}
\begin{itemize}
\item {Grp. gram.:F.}
\end{itemize}
\begin{itemize}
\item {Utilização:Des.}
\end{itemize}
\begin{itemize}
\item {Proveniência:(De \textunderscore sair\textunderscore )}
\end{itemize}
Que sai.
Que vai acabar.
Hora da saída. Cf. Fern. Lopes, \textunderscore Chrón. de D. Fern.\textunderscore , XXXIX.
\section{Saio}
\begin{itemize}
\item {Grp. gram.:m.}
\end{itemize}
\begin{itemize}
\item {Utilização:Prov.}
\end{itemize}
\begin{itemize}
\item {Utilização:minh.}
\end{itemize}
\begin{itemize}
\item {Proveniência:(Do lat. \textunderscore sagum\textunderscore )}
\end{itemize}
Antigo e largo vestuário, com fraldão e abas.
Antigo casacão de militares.
O mesmo que \textunderscore véstia\textunderscore ^1.
\section{Saioado}
\begin{itemize}
\item {Grp. gram.:m.}
\end{itemize}
\begin{itemize}
\item {Utilização:Ant.}
\end{itemize}
O mesmo que \textunderscore saionaria\textunderscore .
\section{Saioaria}
\begin{itemize}
\item {Grp. gram.:f.}
\end{itemize}
\begin{itemize}
\item {Utilização:Ant.}
\end{itemize}
O mesmo que \textunderscore saionaria\textunderscore .
\section{Saiodino}
\begin{itemize}
\item {Grp. gram.:m.}
\end{itemize}
\begin{itemize}
\item {Utilização:Pharm.}
\end{itemize}
Um dos succedâneos do iodeto de potássio.
\section{Saiola}
\begin{itemize}
\item {Grp. gram.:f.}
\end{itemize}
\begin{itemize}
\item {Utilização:T. de Barroso}
\end{itemize}
\begin{itemize}
\item {Proveniência:(De \textunderscore saia\textunderscore )}
\end{itemize}
O mesmo que \textunderscore anágua\textunderscore .
\section{Saionaria}
\begin{itemize}
\item {Grp. gram.:f.}
\end{itemize}
\begin{itemize}
\item {Utilização:Ant.}
\end{itemize}
Offício de saião.
Procedimento injurioso; insolência.
\section{Saionia}
\begin{itemize}
\item {Grp. gram.:f.}
\end{itemize}
\begin{itemize}
\item {Utilização:Ant.}
\end{itemize}
O mesmo que \textunderscore saionaria\textunderscore .
\section{Saionício}
\begin{itemize}
\item {Grp. gram.:m.}
\end{itemize}
\begin{itemize}
\item {Utilização:Ant.}
\end{itemize}
O mesmo que \textunderscore saionízio\textunderscore .
\section{Saiozínio}
\begin{itemize}
\item {Grp. gram.:m.}
\end{itemize}
\begin{itemize}
\item {Utilização:Ant.}
\end{itemize}
\begin{itemize}
\item {Proveniência:(De \textunderscore saião\textunderscore ^2)}
\end{itemize}
Estipêndio de esbirro e de algoz.
\section{Saioria}
\begin{itemize}
\item {Grp. gram.:f.}
\end{itemize}
\begin{itemize}
\item {Utilização:Ant.}
\end{itemize}
O mesmo que \textunderscore saionaria\textunderscore .
\section{Saiote}
\begin{itemize}
\item {Grp. gram.:m.}
\end{itemize}
\begin{itemize}
\item {Proveniência:(De \textunderscore saio\textunderscore )}
\end{itemize}
Saia curta e de tecido grosso, que as mulhéres usam por baixo de outra saia ou saias.
\section{Saioto}
\begin{itemize}
\item {fónica:ô}
\end{itemize}
\begin{itemize}
\item {Grp. gram.:m.}
\end{itemize}
\begin{itemize}
\item {Utilização:Prov.}
\end{itemize}
O mesmo que \textunderscore saiote\textunderscore . Cf. Camillo, \textunderscore Brasileira\textunderscore , 156.
\section{Sair}
\begin{itemize}
\item {Grp. gram.:v. i.}
\end{itemize}
\begin{itemize}
\item {Utilização:Bras. do S}
\end{itemize}
\begin{itemize}
\item {Grp. gram.:V. t.}
\end{itemize}
\begin{itemize}
\item {Grp. gram.:V. p.}
\end{itemize}
\begin{itemize}
\item {Proveniência:(Do lat. \textunderscore salire\textunderscore )}
\end{itemize}
Ir ou passar para fóra.
Passar os limites, a raia.
Afastar-se.
Tirar-se donde estava.
Resaltar, formar saliência.
Distinguir-se.
Separar-se de um grêmio ou corporação.
Separar-se.
Libertar-se.
Apparecer.
Investir, arremeter.
Publicar-se, estampar-se, vir a público: \textunderscore saiu ontem mais um jornal\textunderscore .
Derivar, dimanar; brotar.
Resultar.
Tornar-se, transformar-se.
Caber em sorte.
Ficar em pé (o cavalleiro), quando o cavallo cai.
Ir para fóra de, passar àlém de.
Libertar-se.
Desembaraçar-se.
Afastar-se.
Atrever-se.
Deixar de sêr tímido.
Obter êxito, bom ou mau.
\section{Saira}
\begin{itemize}
\item {fónica:sa-i}
\end{itemize}
\begin{itemize}
\item {Grp. gram.:m.}
\end{itemize}
Espécie de \textunderscore saí\textunderscore , pássaro.
\section{Sairá}
\begin{itemize}
\item {fónica:sa-i}
\end{itemize}
\begin{itemize}
\item {Grp. gram.:m.}
\end{itemize}
Espécie de cotinga brasileira.
\section{Sairé}
\begin{itemize}
\item {fónica:sa-i}
\end{itemize}
\begin{itemize}
\item {Grp. gram.:m.}
\end{itemize}
\begin{itemize}
\item {Utilização:Bras}
\end{itemize}
Espécie de dança popular.
Apparelho de cipó, feito em semi-círculo e levado por mulhéres índias, ao som da música, em certas festas religiosas.
\section{Sairro}
\begin{itemize}
\item {Grp. gram.:m.}
\end{itemize}
\begin{itemize}
\item {Utilização:Prov.}
\end{itemize}
\begin{itemize}
\item {Utilização:beir.}
\end{itemize}
O mesmo que \textunderscore sarro\textunderscore .
\section{Saitaia}
\begin{itemize}
\item {fónica:sa-i}
\end{itemize}
\begin{itemize}
\item {Grp. gram.:m.}
\end{itemize}
Gênero de macacos americanos, a que pertence o \textunderscore saitaia-chorão\textunderscore , e outros.
\section{Saitaia-chorão}
\begin{itemize}
\item {Grp. gram.:m.}
\end{itemize}
Espécie de saitaia.
\section{Sajaria}
\begin{itemize}
\item {Grp. gram.:f.}
\end{itemize}
\begin{itemize}
\item {Utilização:Ant.}
\end{itemize}
\begin{itemize}
\item {Proveniência:(De \textunderscore sages\textunderscore )}
\end{itemize}
Sabedoria; prudência, discrição.
\section{Saju}
\begin{itemize}
\item {Grp. gram.:m.}
\end{itemize}
\begin{itemize}
\item {Proveniência:(Do guar. \textunderscore çay-guazu\textunderscore )}
\end{itemize}
Pequeno macaco do Brasil.
\section{Sal}
\begin{itemize}
\item {Grp. gram.:m.}
\end{itemize}
\begin{itemize}
\item {Utilização:Fig.}
\end{itemize}
\begin{itemize}
\item {Proveniência:(Lat. \textunderscore sal\textunderscore )}
\end{itemize}
Substância dura e friável, de um sabor picante, e que, solvendo-se na água, serve habitualmente de tempêro.
Substância, que resulta da combinação de um ácido com uma base chímica.
Bom gôsto; graça.
Finura de espírito.
Malícia.
Vivacidade; chiste.
\textunderscore Sal de embate\textunderscore , sal fino, produzido nas bordas das salinas pelo embate da água, agitada por ventos propícios.
\section{Sala}
\begin{itemize}
\item {Grp. gram.:f.}
\end{itemize}
\begin{itemize}
\item {Utilização:Ant.}
\end{itemize}
\begin{itemize}
\item {Utilização:Ant.}
\end{itemize}
\begin{itemize}
\item {Utilização:Bras. do N}
\end{itemize}
Um dos principaes compartimentos de uma casa, ou o principal, destinado ordinariamente á recepção de visitas.
Qualquer compartimento de um edifício, com mais ou menos amplidão.
Cadeira, em que se sentava o juiz da aldeia, ao ar livre.
Muralha, que entesta com o baluarte.
O primeiro dos compartimentos de um curral-de-peixe.
\textunderscore Fazer sala\textunderscore  ou \textunderscore salas\textunderscore , fazer côrte a, lisonjear:«\textunderscore V. R. está havido, na opinião dos que mais salas lhe fazem e se lhe mais submetem...\textunderscore »Jerón. Osório, \textunderscore Carta\textunderscore .
(B. lat. \textunderscore sala\textunderscore )
\section{Sala}
\begin{itemize}
\item {Grp. gram.:f.}
\end{itemize}
\begin{itemize}
\item {Utilização:Ant.}
\end{itemize}
Salva ou bandeja de metal.
\section{Salabórdia}
\begin{itemize}
\item {Grp. gram.:f.}
\end{itemize}
\begin{itemize}
\item {Utilização:Chul.}
\end{itemize}
Sensaboria; conversa insípida.
\section{Salácia}
\begin{itemize}
\item {Grp. gram.:f.}
\end{itemize}
\begin{itemize}
\item {Proveniência:(Lat. \textunderscore salacia\textunderscore )}
\end{itemize}
Gênero de crustáceos decápodes.
Gênero de polypeiros.
Gênero de arbustos, de ramos angulosos, nas regiões tropicaes.
\section{Salácia}
\begin{itemize}
\item {Grp. gram.:f.}
\end{itemize}
O mesmo que \textunderscore salacidade\textunderscore :«\textunderscore a salácia dos costumes.\textunderscore »Rui Barb., \textunderscore Réplica\textunderscore , 177.
\section{Salacidade}
\begin{itemize}
\item {Grp. gram.:f.}
\end{itemize}
\begin{itemize}
\item {Proveniência:(Lat. \textunderscore salacitas\textunderscore )}
\end{itemize}
Qualidade de salaz.
Libertinagem; devassidão.
\section{Salacorta}
\begin{itemize}
\item {Grp. gram.:f.}
\end{itemize}
Planta medicinal da ilha de San-Thomé.
\section{Salactol}
\begin{itemize}
\item {Grp. gram.:m.}
\end{itemize}
\begin{itemize}
\item {Utilização:Chím.}
\end{itemize}
Dissolução de salicylato e de lactato de sódio em água.
\section{Salada}
\begin{itemize}
\item {Grp. gram.:f.}
\end{itemize}
\begin{itemize}
\item {Utilização:Fig.}
\end{itemize}
\begin{itemize}
\item {Utilização:Chul.}
\end{itemize}
Planta ou plantas hortenses, que, depois de migadas, se temperam com sal e outras especiarias e se comem cruas.
Iguaria, temperada com môlhos diversos, sem ir ao fogo.
Estado do que está moído, pisado, sovado.
Salgalhada.
(Talvez do cast. \textunderscore ensalada\textunderscore )
\section{Saladeira}
\begin{itemize}
\item {Grp. gram.:f.}
\end{itemize}
Prato grande e fundo, ou espécie de travessa, que leva a salada á mesa.
\section{Saladino}
\begin{itemize}
\item {Grp. gram.:adj.}
\end{itemize}
\begin{itemize}
\item {Proveniência:(De \textunderscore Saladino\textunderscore , n. p.)}
\end{itemize}
Dizia-se do tributo, que abrangia o dízimo de todos os bens, e que eram obrigados a pagar os que em França e em Inglaterra não se cruzassem para a conquista da Palestina.
\section{Salado}
\begin{itemize}
\item {Grp. gram.:adj.}
\end{itemize}
\begin{itemize}
\item {Utilização:Ant.}
\end{itemize}
\begin{itemize}
\item {Proveniência:(Do lat. \textunderscore salum\textunderscore )}
\end{itemize}
O mesmo que \textunderscore salgado\textunderscore . Cf. \textunderscore Eufrosina\textunderscore , 181.
\section{Salafrário}
\begin{itemize}
\item {Grp. gram.:m.}
\end{itemize}
\begin{itemize}
\item {Utilização:Pop.}
\end{itemize}
Homem ordinário.
Bisbórria.
Patife. Cf. Junqueiro, \textunderscore D. João\textunderscore , 248.
(Relaciona-se com \textunderscore ceroferário\textunderscore ?)
\section{Salagre}
\begin{itemize}
\item {Grp. gram.:adj.}
\end{itemize}
\begin{itemize}
\item {Utilização:Pop.}
\end{itemize}
\begin{itemize}
\item {Utilização:Fig.}
\end{itemize}
Quebradiço.
Que chora com qualquer coisa.
\section{Salalé}
\begin{itemize}
\item {Grp. gram.:m.}
\end{itemize}
Formiga branca, (\textunderscore termes bellicosus\textunderscore ), do gênero das térmites, e de origem africana.
\section{Salama}
\begin{itemize}
\item {Grp. gram.:f.}
\end{itemize}
O mesmo que \textunderscore salamaleque\textunderscore . Cf. Dom. Vieira, \textunderscore Diccion.\textunderscore 
\section{Salamaleque}
\begin{itemize}
\item {Grp. gram.:m.}
\end{itemize}
\begin{itemize}
\item {Utilização:pop.}
\end{itemize}
\begin{itemize}
\item {Utilização:Fig.}
\end{itemize}
\begin{itemize}
\item {Proveniência:(Do ár. \textunderscore salam\textunderscore  + \textunderscore haleik\textunderscore )}
\end{itemize}
Saudação, entre os Turcos.
Mesura exaggerada.
Grande reverência; cumprimentos affectados.
\section{Salamandra}
\begin{itemize}
\item {Grp. gram.:f.}
\end{itemize}
\begin{itemize}
\item {Proveniência:(Lat. \textunderscore salamandra\textunderscore )}
\end{itemize}
Gênero de batrácios, semelhantes aos lagartos.
Gênio, que governa o fogo e nelle vive:«\textunderscore ...os poderosos gênios, que a seu sabor os elementos movem, salamandras, ondins, silfos e gnomos...\textunderscore »Castilho, \textunderscore Primavera\textunderscore , 233.
\section{Salamânia}
\begin{itemize}
\item {Grp. gram.:f.}
\end{itemize}
Espécie de frauta, entre os Turcos.
\section{Salamanquino}
\begin{itemize}
\item {Grp. gram.:adj.}
\end{itemize}
\begin{itemize}
\item {Grp. gram.:M.}
\end{itemize}
Relativo a Salamanca.
Habitante de Salamanca.
\section{Salamântega}
\begin{itemize}
\item {Grp. gram.:f.}
\end{itemize}
O mesmo que \textunderscore salamântiga\textunderscore .
\section{Salamanticense}
\begin{itemize}
\item {Grp. gram.:adj.}
\end{itemize}
\begin{itemize}
\item {Grp. gram.:M.}
\end{itemize}
Relativo á Salamanca.
Habitante de Salamanca.
\section{Salamântico}
\begin{itemize}
\item {Grp. gram.:adj.}
\end{itemize}
O mesmo que \textunderscore salamanticense\textunderscore .
\section{Salamântiga}
\begin{itemize}
\item {Grp. gram.:f.}
\end{itemize}
\begin{itemize}
\item {Utilização:Pop.}
\end{itemize}
O mesmo que \textunderscore salamandra\textunderscore .--Na Extremadura, dizem \textunderscore salamantiga\textunderscore , contra o uso geral do país, e contra a analogia fonética de \textunderscore salamandra\textunderscore .
\section{Salamar}
\begin{itemize}
\item {Grp. gram.:f.}
\end{itemize}
Variedade de pêra portuguesa.
\section{Salamba}
\begin{itemize}
\item {Grp. gram.:f.}
\end{itemize}
Árvore africana, cujo fruto tem uma polpa ácida e agradável.
\section{Salame}
\begin{itemize}
\item {Proveniência:(It. \textunderscore salame\textunderscore )}
\end{itemize}
Espécie de paio.
\section{Salame}
\begin{itemize}
\item {Grp. gram.:m.}
\end{itemize}
O mesmo que \textunderscore salamaleque\textunderscore . Cf. Th. Ribeiro, \textunderscore Jornadas\textunderscore , II, 101.
\section{Salamé}
\begin{itemize}
\item {Grp. gram.:m.}
\end{itemize}
O mesmo que \textunderscore salamaleque\textunderscore . Cf. Dom. Vieira, \textunderscore Diccion.\textunderscore 
\section{Salamim}
\begin{itemize}
\item {Grp. gram.:m.}
\end{itemize}
\begin{itemize}
\item {Utilização:Ant.}
\end{itemize}
\begin{itemize}
\item {Proveniência:(T. ind.)}
\end{itemize}
Direito de corretagem, que se pagava em Dio.
\section{Salamorda}
\begin{itemize}
\item {fónica:môr}
\end{itemize}
\begin{itemize}
\item {Grp. gram.:m.  e  f.}
\end{itemize}
\begin{itemize}
\item {Utilização:Prov.}
\end{itemize}
\begin{itemize}
\item {Utilização:trasm.}
\end{itemize}
O mesmo que \textunderscore salamurdo\textunderscore . (Colhido em Sabrosa)
\section{Salamurdo}
\begin{itemize}
\item {Grp. gram.:m.}
\end{itemize}
\begin{itemize}
\item {Utilização:Prov.}
\end{itemize}
\begin{itemize}
\item {Utilização:trasm.}
\end{itemize}
Indivíduo, que fala pouco, mas que é sonso e morde pela calada.
\section{Salandra}
\begin{itemize}
\item {Grp. gram.:f.}
\end{itemize}
\begin{itemize}
\item {Utilização:Veter.}
\end{itemize}
Arestim, na junta do curvilhão dos equídeos.
\section{Salangana}
\begin{itemize}
\item {Grp. gram.:f.}
\end{itemize}
\begin{itemize}
\item {Proveniência:(Do mal. \textunderscore salangang\textunderscore )}
\end{itemize}
Andorinha oriental, de cujos ninhos se faz uma sopa muito apreciada no Oriente.
\section{Salango}
\begin{itemize}
\item {Grp. gram.:m.}
\end{itemize}
Gênero de peixes acanthopterýgios.
\section{Salão}
\begin{itemize}
\item {Grp. gram.:m.}
\end{itemize}
Grande sala.
\section{Salão}
\begin{itemize}
\item {Grp. gram.:m.}
\end{itemize}
\begin{itemize}
\item {Utilização:Bras}
\end{itemize}
O mesmo que \textunderscore solão\textunderscore ^1.
Plano de argilla e pedregulho, numa escarpa, talhada a pique sôbre os rios Amazónicos.
\section{Salariar}
\begin{itemize}
\item {Grp. gram.:v. t.}
\end{itemize}
O mesmo que \textunderscore assalariar\textunderscore .
\section{Salário}
\begin{itemize}
\item {Grp. gram.:m.}
\end{itemize}
\begin{itemize}
\item {Utilização:Restrict.}
\end{itemize}
\begin{itemize}
\item {Proveniência:(Lat. \textunderscore salarius\textunderscore )}
\end{itemize}
Retribuição de serviço.
Paga.
Retribuição de serviço, feito aos dias ou ás horas.
\section{Salá-salá}
\begin{itemize}
\item {Grp. gram.:m.}
\end{itemize}
Árvore medicinal da ilha de San-Thomé.
\section{Salatinos}
\begin{itemize}
\item {Grp. gram.:m. pl.}
\end{itemize}
\begin{itemize}
\item {Utilização:Ant.}
\end{itemize}
Moiros ou corsários de Salé.
Nome depreciativo, que o rapazío de Coímbra dá aos seus adversários.
Lavradores dos arredores de Lisbôa, procedentes daquelles Moiros.
Saloios.
(Por \textunderscore saletinos\textunderscore , de \textunderscore Salé\textunderscore , n. p.)
\section{Salaz}
\begin{itemize}
\item {Grp. gram.:adj.}
\end{itemize}
\begin{itemize}
\item {Proveniência:(Lat. \textunderscore salax\textunderscore )}
\end{itemize}
Impudico; propenso á luxúria; libertino.
\section{Salça-prôa}
\begin{itemize}
\item {Grp. gram.:f.}
\end{itemize}
\begin{itemize}
\item {Utilização:Náut.}
\end{itemize}
Prôa, que não tem beque, mas uma curva a que se prende a trinca.
\section{Salçarana}
\begin{itemize}
\item {Grp. gram.:f.}
\end{itemize}
Planta aromática das regiões do Amazonas.
\section{Saldanhista}
\begin{itemize}
\item {Grp. gram.:m.}
\end{itemize}
Aquelle que seguia a política do Duque de Saldanha, em Portugal.
\section{Saldanita}
\begin{itemize}
\item {Grp. gram.:f.}
\end{itemize}
Sulfato de alumina, muito abundante em Nova-Galles.
\section{Saldar}
\begin{itemize}
\item {Grp. gram.:v. t.}
\end{itemize}
\begin{itemize}
\item {Utilização:Fig.}
\end{itemize}
\begin{itemize}
\item {Proveniência:(Do lat. \textunderscore solidare\textunderscore )}
\end{itemize}
Pagar o saldo de.
Liquidar ou ajustar (contas).
\textunderscore Saldar contas\textunderscore , desforrar-se; tomar ou exigir satisfações.
\section{Saldínia}
\begin{itemize}
\item {Grp. gram.:f.}
\end{itemize}
Gênero de plantas rubiáceas.
\section{Saldo}
\begin{itemize}
\item {Grp. gram.:m.}
\end{itemize}
\begin{itemize}
\item {Utilização:Fig.}
\end{itemize}
\begin{itemize}
\item {Grp. gram.:Adj.}
\end{itemize}
\begin{itemize}
\item {Proveniência:(De \textunderscore saldar\textunderscore )}
\end{itemize}
Differença entre o débito e o crédito, nas contas de devedores com crèdores.
Resto.
Quantia, necessária para equilibrar determinada receita com determinada despesa.
Desforra, vingança.
Liquidação de aggravos.
Liquidado, quite: \textunderscore ficaram saldas as minhas contas\textunderscore .
\section{Saldunes}
\begin{itemize}
\item {Grp. gram.:m. pl.}
\end{itemize}
Aquelles que, entre os Gállios, faziam juramento de eterna amizade, marchando para o combate, ligados por uma corrente, porque nem a morte os devia separar. Cf. C. Neto, \textunderscore Saldunes\textunderscore .
(Cp. \textunderscore soldúrios\textunderscore )
\section{Salé}
\begin{itemize}
\item {Grp. gram.:f.}
\end{itemize}
\begin{itemize}
\item {Utilização:Pop.}
\end{itemize}
\begin{itemize}
\item {Proveniência:(Fr. \textunderscore salé\textunderscore )}
\end{itemize}
Carne salgada.
\section{Saleira}
\begin{itemize}
\item {Grp. gram.:f.}
\end{itemize}
Barco chato, usado no Vouga, para transporte de sal.
\section{Saleiro}
\begin{itemize}
\item {Grp. gram.:m.}
\end{itemize}
\begin{itemize}
\item {Grp. gram.:Adj.}
\end{itemize}
Pequeno vaso, em que se guarda o sal ou se leva á mesa.
Vendedor de sal.
Fabricante de sal.
Relativo a sal: \textunderscore negociantes saleiros\textunderscore .
\section{Saleiro}
\begin{itemize}
\item {Grp. gram.:m.}
\end{itemize}
Ponta dos galhos do veado, quando rebentam.
\section{Salém}
\begin{itemize}
\item {Grp. gram.:m.}
\end{itemize}
\begin{itemize}
\item {Utilização:Des.}
\end{itemize}
O mesmo que \textunderscore ramalhete\textunderscore . Cf. Filinto, VII, 251.
(Do ár.?)
\section{Salema}
\begin{itemize}
\item {Grp. gram.:f.}
\end{itemize}
\begin{itemize}
\item {Utilização:Ant.}
\end{itemize}
\begin{itemize}
\item {Proveniência:(Do ár. \textunderscore çalam\textunderscore )}
\end{itemize}
Peixe esparoide.
Saudação, cumprimentos, o mesmo que \textunderscore salama\textunderscore . Cf. Latino, \textunderscore Elog. Acad.\textunderscore 
\section{Salênia}
\begin{itemize}
\item {Grp. gram.:f.}
\end{itemize}
Gênero de echinodermes fósseis dos terrenos cretáceos.
\section{Salepeira-maior}
\begin{itemize}
\item {Grp. gram.:f.}
\end{itemize}
Planta orchidácea, também conhecida por \textunderscore satirião macho\textunderscore .
\section{Salepo}
\begin{itemize}
\item {Grp. gram.:m.}
\end{itemize}
\begin{itemize}
\item {Proveniência:(Do fr. \textunderscore salep\textunderscore )}
\end{itemize}
Espécie de orchídea.
Substância alimentícia, que se extrái dos tubérculos das orchídeas.
Araruta.
\section{Salepo-maior}
\begin{itemize}
\item {Grp. gram.:m.}
\end{itemize}
Variedade de orchídea portuguesa, (\textunderscore orchis mascula\textunderscore , Lin.).
\section{Salernitano}
\begin{itemize}
\item {Grp. gram.:adj.}
\end{itemize}
\begin{itemize}
\item {Grp. gram.:M.}
\end{itemize}
\begin{itemize}
\item {Proveniência:(Lat. \textunderscore salernitanus\textunderscore )}
\end{itemize}
Relativo a Salerno.
Habitante de Salerno.
\section{Salésia}
\begin{itemize}
\item {Grp. gram.:f.}
\end{itemize}
\begin{itemize}
\item {Proveniência:(De \textunderscore Sales\textunderscore , n. p.)}
\end{itemize}
Freira, da Ordem de Nossa Senhora da Visitação, instituida por San Francisco de Sales.
\section{Salesiano}
\begin{itemize}
\item {Grp. gram.:adj.}
\end{itemize}
\begin{itemize}
\item {Proveniência:(De \textunderscore Sales\textunderscore , n. p.)}
\end{itemize}
Diz-se da Ordem ou Congregação religiosa, instituida por San Francisco de Sales.
\section{Saleta}
\begin{itemize}
\item {fónica:lê}
\end{itemize}
\begin{itemize}
\item {Grp. gram.:f.}
\end{itemize}
Pequena sala.
\section{Salga}
\begin{itemize}
\item {Grp. gram.:f.}
\end{itemize}
Acto de salgar.
\section{Salga}
\begin{itemize}
\item {Grp. gram.:f.}
\end{itemize}
\begin{itemize}
\item {Utilização:Ant.}
\end{itemize}
O mesmo que \textunderscore acelga\textunderscore .
\section{Salgação}
\begin{itemize}
\item {Grp. gram.:f.}
\end{itemize}
O mesmo que \textunderscore salga\textunderscore ^1.
Feitiço, bruxaria.
\section{Salgadamente}
\begin{itemize}
\item {Grp. gram.:adv.}
\end{itemize}
\begin{itemize}
\item {Proveniência:(De \textunderscore salgado\textunderscore )}
\end{itemize}
Com muito sal.
\section{Salgadeira}
\begin{itemize}
\item {Grp. gram.:f.}
\end{itemize}
\begin{itemize}
\item {Utilização:Pop.}
\end{itemize}
\begin{itemize}
\item {Utilização:Bot.}
\end{itemize}
\begin{itemize}
\item {Proveniência:(De \textunderscore salgar\textunderscore )}
\end{itemize}
Vasilha ou lugar, onde se salga peixe, carne, etc.
Caixão mortuário.
Planta herbácea.
\section{Salgadiço}
\begin{itemize}
\item {Grp. gram.:adj.}
\end{itemize}
\begin{itemize}
\item {Grp. gram.:M.}
\end{itemize}
\begin{itemize}
\item {Proveniência:(De \textunderscore salgar\textunderscore )}
\end{itemize}
Que tem qualidades salinas, pela vizinhança do mar: \textunderscore terreno salgadiço\textunderscore .
Terreno, que tem essas qualidades.
\section{Salgadinha}
\begin{itemize}
\item {Grp. gram.:f.}
\end{itemize}
\begin{itemize}
\item {Utilização:Bot.}
\end{itemize}
Planta, o mesmo que \textunderscore salgueirinha\textunderscore . Cf. P. Moraes, \textunderscore Zool. Elem.\textunderscore , 649.
\section{Salgadio}
\begin{itemize}
\item {Grp. gram.:adj.}
\end{itemize}
\begin{itemize}
\item {Grp. gram.:M.}
\end{itemize}
\begin{itemize}
\item {Proveniência:(De \textunderscore salgado\textunderscore )}
\end{itemize}
Que, pela sua vizinhança do mar, soffre a acção do sal marinho; salgadiço.
Terreno, banhado por água salgada ou muito exposto á aragem marinha. Cf. B. Pato, \textunderscore Livro do Monte\textunderscore .
\section{Salgado}
\begin{itemize}
\item {Grp. gram.:adj.}
\end{itemize}
\begin{itemize}
\item {Utilização:Fig.}
\end{itemize}
\begin{itemize}
\item {Utilização:Pop.}
\end{itemize}
\begin{itemize}
\item {Utilização:Taur.}
\end{itemize}
\begin{itemize}
\item {Grp. gram.:M. pl.}
\end{itemize}
Impregnado de sal.
Que se temperou com sal demasiado: \textunderscore iguaria salgada\textunderscore .
Picante.
Que tem graça.
Que se adquiriu por alto preço.
Caro.
O mesmo que \textunderscore roseiro\textunderscore .
Terrenos, pouco productivos, na vizinhança do mar.
\section{Salgadura}
\begin{itemize}
\item {Grp. gram.:f.}
\end{itemize}
Acto ou effeito de salgar.
\section{Salgalhada}
\begin{itemize}
\item {Grp. gram.:f.}
\end{itemize}
\begin{itemize}
\item {Utilização:Pop.}
\end{itemize}
\begin{itemize}
\item {Proveniência:(De \textunderscore salgar\textunderscore )}
\end{itemize}
Confusão; mixórdia; trapalhada.
\section{Salgante}
\begin{itemize}
\item {Grp. gram.:adj.}
\end{itemize}
Que salga. Cf. \textunderscore Museu Techn.\textunderscore , 75 e 84.
\section{Salgar}
\begin{itemize}
\item {Grp. gram.:v. t.}
\end{itemize}
\begin{itemize}
\item {Utilização:Ant.}
\end{itemize}
\begin{itemize}
\item {Proveniência:(Do lat. hyp. \textunderscore salicare\textunderscore )}
\end{itemize}
Temperar com sal.
Impregnar de sal; deitar muito sal em.
Fazer feitiços, espalhando sal á porta de.
Espalhar sal em (terreno em que se commeteu crime ou profanação), para que fique maldito e estéril.
\section{Sal-gemma}
\begin{itemize}
\item {Grp. gram.:m.}
\end{itemize}
Sal commum fossilizado.
\section{Salgueira}
\begin{itemize}
\item {Grp. gram.:f.}
\end{itemize}
\begin{itemize}
\item {Grp. gram.:Adj.}
\end{itemize}
Variedade de uva preta minhota.
Planta myrsínea da Índia Portuguesa, (\textunderscore aegiceras majus\textunderscore , Gaertn).
Diz-se de uma variedade de azeitona, também chamada \textunderscore lentisca\textunderscore  e \textunderscore durázia\textunderscore .
\section{Salgueira-branca}
\begin{itemize}
\item {Grp. gram.:f.}
\end{itemize}
Planta verbenácea da Índia Portuguesa, (\textunderscore avicennia officinalis\textunderscore , Lin.).
\section{Salgueiral}
\begin{itemize}
\item {Grp. gram.:m.}
\end{itemize}
Terreno, onde crescem salgueiros.
\section{Salgueirinha}
\begin{itemize}
\item {Grp. gram.:f.}
\end{itemize}
Planta salicínea, (\textunderscore lithrum salicaria\textunderscore , Lin.).
\section{Salgueiro}
\begin{itemize}
\item {Grp. gram.:m.}
\end{itemize}
\begin{itemize}
\item {Proveniência:(Do lat. hypoth. \textunderscore salicarius\textunderscore )}
\end{itemize}
Árvore, que cresce habitualmente nos campos e á beira dos rios.
Chorão.
Árvore borragínea.
Espécie de uva do districto de Aveiro.
\section{Salgueiro-falso}
\begin{itemize}
\item {Grp. gram.:m.}
\end{itemize}
Planta combretácea da Índia Portuguesa, (\textunderscore lumnitzera racemosa\textunderscore , Wild).
\section{Salgueiro-mainato}
\begin{itemize}
\item {Grp. gram.:m.}
\end{itemize}
Planta rizophórea da Índia Portugesa, (\textunderscore rhizophora mucronata\textunderscore , Lamk.).
\section{Salhar}
\begin{itemize}
\item {Grp. gram.:v. t.}
\end{itemize}
\begin{itemize}
\item {Utilização:Ant.}
\end{itemize}
Arrastar.
Puxar para cima.
Assestar (artilharia)
\section{Salicáceas}
\begin{itemize}
\item {Grp. gram.:f. pl.}
\end{itemize}
O mesmo ou melhor que \textunderscore salicíneas\textunderscore .
\section{Salicariáceas}
\begin{itemize}
\item {Grp. gram.:f. pl.}
\end{itemize}
(V.lythrariadas)
\section{Salicícola}
\begin{itemize}
\item {Grp. gram.:adj.}
\end{itemize}
\begin{itemize}
\item {Utilização:Hist. Nat.}
\end{itemize}
\begin{itemize}
\item {Proveniência:(Do lat. \textunderscore salix\textunderscore  + \textunderscore colere\textunderscore )}
\end{itemize}
Que vive nos salgueiros.
\section{Sàlicifoliado}
\begin{itemize}
\item {Grp. gram.:adj.}
\end{itemize}
\begin{itemize}
\item {Utilização:Bot.}
\end{itemize}
\begin{itemize}
\item {Proveniência:(Do lat. \textunderscore salix\textunderscore  + \textunderscore folium\textunderscore )}
\end{itemize}
Que tem fôlhas parecidas ás do salgueiro.
\section{Salicilagem}
\begin{itemize}
\item {Grp. gram.:f.}
\end{itemize}
Acto de \textunderscore salicilar\textunderscore .
\section{Salicilar}
\begin{itemize}
\item {Grp. gram.:v.}
\end{itemize}
\begin{itemize}
\item {Utilização:t. Chím.}
\end{itemize}
Misturar com ácido salicílico. Cf. Ferr. da Silva, \textunderscore Questão de Vinhos\textunderscore .
\section{Salicilato}
\begin{itemize}
\item {Grp. gram.:m.}
\end{itemize}
\begin{itemize}
\item {Utilização:Chím.}
\end{itemize}
\begin{itemize}
\item {Proveniência:(De \textunderscore salicílico\textunderscore )}
\end{itemize}
Sal, produzido pela combinação do ácido salicílico com uma base.
\section{Saliciloso}
\begin{itemize}
\item {Grp. gram.:adj.}
\end{itemize}
\begin{itemize}
\item {Proveniência:(Do lat. \textunderscore salix\textunderscore  + gr. \textunderscore ule\textunderscore )}
\end{itemize}
Diz-se de um ácido, que se extrái da ulmária, por meio de destilação com água.
\section{Salicina}
\begin{itemize}
\item {Grp. gram.:f.}
\end{itemize}
\begin{itemize}
\item {Proveniência:(Do lat. \textunderscore salix\textunderscore )}
\end{itemize}
Substância, que se encontra na casca do salgueiro.
\section{Salicíneas}
\begin{itemize}
\item {Grp. gram.:f. pl.}
\end{itemize}
Família de plantas, que tem por typo o salgueiro.
(Fem. pl. de \textunderscore salicíneo\textunderscore )
\section{Salicíneo}
\begin{itemize}
\item {Grp. gram.:adj.}
\end{itemize}
\begin{itemize}
\item {Proveniência:(Do lat. \textunderscore salix\textunderscore )}
\end{itemize}
Relativo ou semelhante ao salgueiro.
\section{Salicional}
\begin{itemize}
\item {Grp. gram.:f.}
\end{itemize}
\begin{itemize}
\item {Utilização:Mús.}
\end{itemize}
\begin{itemize}
\item {Proveniência:(Do lat. \textunderscore salix\textunderscore , \textunderscore salicis\textunderscore )}
\end{itemize}
Registo de órgão, afrautado e suave.
\section{Salicívoro}
\begin{itemize}
\item {Grp. gram.:adj.}
\end{itemize}
\begin{itemize}
\item {Proveniência:(Do lat. \textunderscore salix\textunderscore  + \textunderscore vorare\textunderscore )}
\end{itemize}
Diz-se do animal, que come flôres ou fôlhas de salgueiro.
\section{Sálico}
\begin{itemize}
\item {Grp. gram.:adj.}
\end{itemize}
\begin{itemize}
\item {Proveniência:(De \textunderscore Sálios\textunderscore )}
\end{itemize}
Relativo aos Francos sálios.
Diz-se especialmente da lei dos Francos, que excluía do throno as mulheres.
\section{Salícola}
\begin{itemize}
\item {Grp. gram.:adj.}
\end{itemize}
\begin{itemize}
\item {Proveniência:(Do lat. \textunderscore sal\textunderscore  + \textunderscore colere\textunderscore )}
\end{itemize}
Que trata das culturas das salinas.
Que produz sal.
\section{Salicóquio}
\begin{itemize}
\item {Grp. gram.:m.}
\end{itemize}
O mesmo que \textunderscore lagostim\textunderscore .
\section{Salicórnia}
\begin{itemize}
\item {Grp. gram.:f.}
\end{itemize}
\begin{itemize}
\item {Proveniência:(Do lat. \textunderscore sal\textunderscore  + \textunderscore cornu\textunderscore )}
\end{itemize}
Gênero de plantas chenopodiáceas.
\section{Salicultura}
\begin{itemize}
\item {Grp. gram.:f.}
\end{itemize}
\begin{itemize}
\item {Proveniência:(Do lat. \textunderscore sal\textunderscore  + \textunderscore cultura\textunderscore )}
\end{itemize}
Cultura das salinas.
Producção artificial do sal.
\section{Salicylagem}
\begin{itemize}
\item {Grp. gram.:f.}
\end{itemize}
Acto de \textunderscore salicylar\textunderscore .
\section{Salicylar}
\begin{itemize}
\item {Grp. gram.:v.}
\end{itemize}
\begin{itemize}
\item {Utilização:t. Chím.}
\end{itemize}
Misturar com ácido salicýlico. Cf. Ferr. da Silva, \textunderscore Questão de Vinhos\textunderscore .
\section{Salicylato}
\begin{itemize}
\item {Grp. gram.:m.}
\end{itemize}
\begin{itemize}
\item {Utilização:Chím.}
\end{itemize}
\begin{itemize}
\item {Proveniência:(De \textunderscore salicýlico\textunderscore )}
\end{itemize}
Sal, produzido pela combinação do ácido salicýlico com uma base.
\section{Salicýlico}
\begin{itemize}
\item {Grp. gram.:adj.}
\end{itemize}
\begin{itemize}
\item {Proveniência:(Do lat. \textunderscore salix\textunderscore  + gr. \textunderscore ule\textunderscore )}
\end{itemize}
Diz-se de um ácido, que se obtém, aquecendo ácido salicyloso com um excesso de hydrato de potassa.
\section{Salicyloso}
\begin{itemize}
\item {Grp. gram.:adj.}
\end{itemize}
\begin{itemize}
\item {Proveniência:(Do lat. \textunderscore salix\textunderscore  + gr. \textunderscore ule\textunderscore )}
\end{itemize}
Diz-se de um ácido, que se extrái da ulmária, por meio de destillação com água.
\section{Saliência}
\begin{itemize}
\item {Grp. gram.:f.}
\end{itemize}
Qualidade do que é saliente.
Proeminência; resalto.
\section{Salientar}
\begin{itemize}
\item {Grp. gram.:v. t.}
\end{itemize}
\begin{itemize}
\item {Utilização:Neol.}
\end{itemize}
\begin{itemize}
\item {Grp. gram.:V. p.}
\end{itemize}
Tornar saliente.
Tornar bem visível ou distinto.
Tornar-se saliente ou notável.
Evidenciar-se; distinguir-se.
\section{Saliente}
\begin{itemize}
\item {Grp. gram.:adj.}
\end{itemize}
\begin{itemize}
\item {Utilização:Fig.}
\end{itemize}
\begin{itemize}
\item {Proveniência:(Lat. \textunderscore saliens\textunderscore )}
\end{itemize}
Que sái para fóra do plano a que está unido.
Que resái ou resalta ou sobresái.
Evidente; notável.
Que dá na vista; que é objecto de reparo.
\section{Saliferiano}
\begin{itemize}
\item {Grp. gram.:adj.}
\end{itemize}
\begin{itemize}
\item {Utilização:Geol.}
\end{itemize}
\begin{itemize}
\item {Proveniência:(De \textunderscore salífero\textunderscore )}
\end{itemize}
Diz-se do período geológico, em que mais abundam os bancos de sal-gemma. Cf. \textunderscore Museu Techn.\textunderscore , 36, 38 e 124.
\section{Salífero}
\begin{itemize}
\item {Grp. gram.:adj.}
\end{itemize}
\begin{itemize}
\item {Utilização:Geol.}
\end{itemize}
\begin{itemize}
\item {Proveniência:(Do lat. \textunderscore sal\textunderscore  + \textunderscore ferre\textunderscore )}
\end{itemize}
Que tem ou produz sal. Cf. \textunderscore Museu Techn.\textunderscore , 33, 39 e 106.
Diz-se do terreno triásico.
\section{Salificar}
\begin{itemize}
\item {Grp. gram.:v. t.}
\end{itemize}
\begin{itemize}
\item {Proveniência:(Do lat. \textunderscore sal\textunderscore  + \textunderscore ferre\textunderscore )}
\end{itemize}
Converter em sal.
\section{Salificável}
\begin{itemize}
\item {Grp. gram.:adj.}
\end{itemize}
Que se póde salificar.
\section{Saligas}
\begin{itemize}
\item {Grp. gram.:m.}
\end{itemize}
Arma, o mesmo que \textunderscore saliques\textunderscore . Cf. Fern. Mendes, \textunderscore Peregrin.\textunderscore , CXXXVIII.
\section{Saligues}
\begin{itemize}
\item {Grp. gram.:m.}
\end{itemize}
\begin{itemize}
\item {Utilização:Ant.}
\end{itemize}
Arma, o mesmo que \textunderscore saliques\textunderscore . Cf. Fern. Mendes, \textunderscore Peregrin.\textunderscore , CXXXVIII.
\section{Salina}
\begin{itemize}
\item {Grp. gram.:f.}
\end{itemize}
\begin{itemize}
\item {Proveniência:(Lat. \textunderscore salinae\textunderscore )}
\end{itemize}
Porção de terreno plano, exposto ao vento Norte ou Nordeste, e preparada para nelle se produzir o sal pela evaporação da água que esse terreno recebe no mar.
Monte de sal.
\section{Salinação}
\begin{itemize}
\item {Grp. gram.:f.}
\end{itemize}
\begin{itemize}
\item {Proveniência:(De \textunderscore salina\textunderscore )}
\end{itemize}
Crystallização do sal.
Formação natural do sal.
\section{Salinagem}
\begin{itemize}
\item {Grp. gram.:f.}
\end{itemize}
O mesmo que \textunderscore salinação\textunderscore .
\section{Salinar}
\begin{itemize}
\item {Grp. gram.:v. t.}
\end{itemize}
\begin{itemize}
\item {Proveniência:(De \textunderscore salina\textunderscore )}
\end{itemize}
Crystallizar (a safra do sal).
\section{Salinável}
\begin{itemize}
\item {Grp. gram.:adj.}
\end{itemize}
\begin{itemize}
\item {Proveniência:(De \textunderscore salinar\textunderscore )}
\end{itemize}
O mesmo que \textunderscore salificável\textunderscore .
\section{Salineira}
\begin{itemize}
\item {Grp. gram.:f.}
\end{itemize}
\begin{itemize}
\item {Proveniência:(De \textunderscore salineiro\textunderscore )}
\end{itemize}
Mulhér, que trabalha nas salinas.
\section{Salineiro}
\begin{itemize}
\item {Grp. gram.:m.}
\end{itemize}
\begin{itemize}
\item {Grp. gram.:Adj.}
\end{itemize}
\begin{itemize}
\item {Proveniência:(Do lat. \textunderscore salinarius\textunderscore )}
\end{itemize}
Aquelle que fabríca sal ou o empilha.
Vendedor de sal.
Relativo a salinas: \textunderscore indústria salineira\textunderscore .
\section{Salinidade}
\begin{itemize}
\item {Grp. gram.:f.}
\end{itemize}
Qualidade de salino.
Grau da densidade do sal em um líquido.
\section{Saliniense}
\begin{itemize}
\item {Grp. gram.:m.}
\end{itemize}
\begin{itemize}
\item {Utilização:Des.}
\end{itemize}
Negociante de sal. Cf. Herculano, \textunderscore Hist. de Port.\textunderscore , I, 254.
\section{Salino}
\begin{itemize}
\item {Grp. gram.:adj.}
\end{itemize}
\begin{itemize}
\item {Utilização:Bras. do S}
\end{itemize}
\begin{itemize}
\item {Proveniência:(Lat. \textunderscore salinus\textunderscore )}
\end{itemize}
Que tem sal ou é da natureza delle.
Nascido á beiramar.
Que tem o pêlo salpicado de pintas brancas.
\section{Salinómetro}
\begin{itemize}
\item {Grp. gram.:m.}
\end{itemize}
\begin{itemize}
\item {Proveniência:(Do lat. \textunderscore salinus\textunderscore  + gr. \textunderscore metron\textunderscore )}
\end{itemize}
Instrumento, que mostra a densidade de uma solução salina.
\section{Sálio}
\begin{itemize}
\item {Grp. gram.:m.}
\end{itemize}
Gênero de insectos hymenópteros.
\section{Sálio}
\begin{itemize}
\item {Grp. gram.:adj.}
\end{itemize}
\begin{itemize}
\item {Grp. gram.:M. pl.}
\end{itemize}
\begin{itemize}
\item {Proveniência:(Lat. \textunderscore salii\textunderscore )}
\end{itemize}
Relativo aos sacerdotes sálios.
Sacerdotes de Marte, entre os Romanos.
\section{Sálios}
\begin{itemize}
\item {Grp. gram.:m. pl.}
\end{itemize}
\begin{itemize}
\item {Proveniência:(De \textunderscore Sala\textunderscore , n. p.)}
\end{itemize}
Uma das tríbos dos Francos.
\section{Salipirina}
\begin{itemize}
\item {Grp. gram.:f.}
\end{itemize}
Medicamento antireumático e antitérmico.
\section{Salipyrina}
\begin{itemize}
\item {Grp. gram.:f.}
\end{itemize}
Medicamento antirheumático e antithérmico.
\section{Saliques}
\begin{itemize}
\item {Grp. gram.:m.}
\end{itemize}
\begin{itemize}
\item {Utilização:Ant.}
\end{itemize}
Arma de arremêsso.
\section{Salir}
\begin{itemize}
\item {Grp. gram.:v. i.}
\end{itemize}
\begin{itemize}
\item {Utilização:Ant.}
\end{itemize}
O mesmo que \textunderscore sair\textunderscore .
(Cast. \textunderscore salir\textunderscore )
\section{Salisbúria}
\begin{itemize}
\item {Grp. gram.:f.}
\end{itemize}
\begin{itemize}
\item {Proveniência:(De \textunderscore Salisbury\textunderscore , n. p.)}
\end{itemize}
Gênero de plantas taxíneas.
\section{Salitração}
\begin{itemize}
\item {Grp. gram.:f.}
\end{itemize}
Acto ou effeito de salitrar.
\section{Salitral}
\begin{itemize}
\item {Grp. gram.:m.}
\end{itemize}
\begin{itemize}
\item {Proveniência:(De \textunderscore salitre\textunderscore )}
\end{itemize}
O mesmo que \textunderscore nitreira\textunderscore .
\section{Salfeno}
\begin{itemize}
\item {Grp. gram.:m.}
\end{itemize}
Medicamento, de aplicação semelhante á do salol.
\section{Salitrar}
\begin{itemize}
\item {Grp. gram.:v. t.}
\end{itemize}
Converter em salitre.
Misturar ou preparar com salitre.
\section{Salitraria}
\begin{itemize}
\item {Grp. gram.:f.}
\end{itemize}
Fábrica de refinação de salitre.
\section{Salitre}
\begin{itemize}
\item {Grp. gram.:m.}
\end{itemize}
\begin{itemize}
\item {Proveniência:(Do lat. \textunderscore salnitre\textunderscore )}
\end{itemize}
Designação vulgar do nitro.
\section{Salitreiro}
\begin{itemize}
\item {Grp. gram.:m.  e  adj.}
\end{itemize}
O que fabríca salitre.
\section{Salitrização}
\begin{itemize}
\item {Grp. gram.:f.}
\end{itemize}
Acto ou effeito de salitrizar.
\section{Salitrizar}
\begin{itemize}
\item {Grp. gram.:v. t.}
\end{itemize}
O mesmo que \textunderscore salitrar\textunderscore .
\section{Salitroso}
\begin{itemize}
\item {Grp. gram.:adj.}
\end{itemize}
Que contém salitre ou é da natureza delle.
\section{Saliva}
\begin{itemize}
\item {Grp. gram.:f.}
\end{itemize}
\begin{itemize}
\item {Proveniência:(Lat. \textunderscore saliva\textunderscore )}
\end{itemize}
Humor transparente e insípido, segregado pelas glândulas buccaes, e que actua na digestão dos alimentos.
\section{Salivação}
\begin{itemize}
\item {Grp. gram.:f.}
\end{itemize}
\begin{itemize}
\item {Proveniência:(Do lat. \textunderscore salivatio\textunderscore )}
\end{itemize}
Acto ou effeito de salivar^2.
\section{Salival}
\begin{itemize}
\item {Grp. gram.:adj.}
\end{itemize}
O mesmo que \textunderscore salivante\textunderscore .
Relativo a saliva.
\section{Salivante}
\begin{itemize}
\item {Grp. gram.:adj.}
\end{itemize}
\begin{itemize}
\item {Proveniência:(Lat. \textunderscore salivans\textunderscore )}
\end{itemize}
Que produz saliva.
\section{Salivar}
\begin{itemize}
\item {Grp. gram.:adj.}
\end{itemize}
Relativo á saliva; salivante.
\section{Salivar}
\begin{itemize}
\item {Grp. gram.:v. i.}
\end{itemize}
\begin{itemize}
\item {Grp. gram.:V. t.}
\end{itemize}
\begin{itemize}
\item {Proveniência:(Lat. \textunderscore salivare\textunderscore )}
\end{itemize}
Expellir saliva, cuspir.
Expellir á maneira de saliva.
\section{Salivária}
\begin{itemize}
\item {Grp. gram.:f.}
\end{itemize}
\begin{itemize}
\item {Proveniência:(Lat. \textunderscore salivaria\textunderscore )}
\end{itemize}
O mesmo que \textunderscore pýrethro\textunderscore .
\section{Salivoso}
\begin{itemize}
\item {Grp. gram.:adj.}
\end{itemize}
\begin{itemize}
\item {Proveniência:(Lat. \textunderscore salivosus\textunderscore )}
\end{itemize}
Que tem saliva ou as propriedades della.
Semelhante á saliva.
\section{Salmácide}
\begin{itemize}
\item {Grp. gram.:m.}
\end{itemize}
Gênero de echinodermes do Mar-Vermelho e do Oceano Índico.
\section{Salmaço}
\begin{itemize}
\item {Grp. gram.:adj.}
\end{itemize}
\begin{itemize}
\item {Utilização:Ant.}
\end{itemize}
O mesmo que \textunderscore salobro\textunderscore .
\section{Salmanticense}
\begin{itemize}
\item {Grp. gram.:m.  e  adj.}
\end{itemize}
\begin{itemize}
\item {Proveniência:(Lat. \textunderscore salmanticensis\textunderscore )}
\end{itemize}
O mesmo que \textunderscore salamanquino\textunderscore .
\section{Salmantino}
\begin{itemize}
\item {Grp. gram.:m.  e  adj.}
\end{itemize}
O mesmo que \textunderscore salamanquino\textunderscore .
(Cp. lat. \textunderscore Salmantica\textunderscore , n. p.)
\section{Salmão}
\begin{itemize}
\item {Grp. gram.:m.}
\end{itemize}
\begin{itemize}
\item {Proveniência:(Lat. \textunderscore salmo\textunderscore )}
\end{itemize}
Gênero de peixes malacopterýgios.
\section{Salmear}
\begin{itemize}
\item {Grp. gram.:v. t.}
\end{itemize}
\begin{itemize}
\item {Grp. gram.:V. i.}
\end{itemize}
Cantar em fórma de salmo.
Entoar tristemente.
Entoar salmos, sem mudar de tom.
Cantar, lêr ou recitar monotonamente.
Têr estilo monótono.
\section{Salmeira}
\begin{itemize}
\item {Grp. gram.:f.}
\end{itemize}
Gênero de molluscos pectiniformes.
\section{Salmejar}
\begin{itemize}
\item {Grp. gram.:v. t.}
\end{itemize}
\begin{itemize}
\item {Utilização:Des.}
\end{itemize}
\begin{itemize}
\item {Proveniência:(Do lat. \textunderscore sagma\textunderscore )}
\end{itemize}
Levar para a eira (os cereaes).
\section{Sálmico}
\begin{itemize}
\item {Grp. gram.:adj.}
\end{itemize}
Relativo a salmo.
Semelhante a salmo.
\section{Salmilhado}
\begin{itemize}
\item {Grp. gram.:adj.}
\end{itemize}
\begin{itemize}
\item {Utilização:Bras}
\end{itemize}
\begin{itemize}
\item {Proveniência:(De \textunderscore sal\textunderscore  + \textunderscore milho\textunderscore ?)}
\end{itemize}
Salpicado de branco e amarello.
Mosqueado; pintalgado:«\textunderscore ...as pennas salmilhadas do peito...\textunderscore »Júl. Ribeiro, \textunderscore A Carne\textunderscore . Cf. \textunderscore idem\textunderscore , \textunderscore Padre Belch.\textunderscore , 165.
\section{Salmista}
\begin{itemize}
\item {Grp. gram.:m.  e  f.}
\end{itemize}
\begin{itemize}
\item {Utilização:Restrict.}
\end{itemize}
\begin{itemize}
\item {Proveniência:(Lat. \textunderscore psalmista\textunderscore )}
\end{itemize}
Pessôa, que faz salmos.
O rei David, autor dos psalmos bíblicos.
\section{Salmo}
\begin{itemize}
\item {Grp. gram.:m.}
\end{itemize}
\begin{itemize}
\item {Proveniência:(Lat. \textunderscore psalmus\textunderscore )}
\end{itemize}
\textunderscore m.\textunderscore  (e der.)
O mesmo ou melhor que \textunderscore psalmo\textunderscore , etc.
Cada um dos cânticos, attribuídos a David.
Cântico de louvor a Deus.
\section{Salmodejar}
\begin{itemize}
\item {Grp. gram.:v. t.}
\end{itemize}
O mesmo que \textunderscore salmodiar\textunderscore . Cf. Alv. Mendes, \textunderscore Discursos\textunderscore , 39.
\section{Salmodia}
\begin{itemize}
\item {Grp. gram.:f.}
\end{itemize}
\begin{itemize}
\item {Utilização:Fig.}
\end{itemize}
\begin{itemize}
\item {Proveniência:(Lat. \textunderscore psalmodia\textunderscore )}
\end{itemize}
Maneira de cantar ou recitar salmos.
Monotonia em declamar, recitar, lêr ou escrever.
(A pron. exacta é \textunderscore salmódia\textunderscore , mas não se usa.)
\section{Salmodiar}
\begin{itemize}
\item {Grp. gram.:v. t.  e  i.}
\end{itemize}
O mesmo que \textunderscore salmear\textunderscore .
\section{Salmoeira}
\textunderscore f.\textunderscore  (e der.)
O mesmo que \textunderscore salmoira\textunderscore , etc.
\section{Salmoeiro}
\begin{itemize}
\item {Grp. gram.:m.}
\end{itemize}
Vasilha para a salmoira.
(Cp. \textunderscore salmoeira\textunderscore )
\section{Salmoira}
\begin{itemize}
\item {Grp. gram.:f.}
\end{itemize}
\begin{itemize}
\item {Proveniência:(Do lat. \textunderscore sal\textunderscore  + \textunderscore muria\textunderscore )}
\end{itemize}
Porção de água, saturada de sal marinho, e applicada geralmente á conservação de substâncias orgânicas.
Vasilha, em que se conservam essas substâncias, com a respectiva água salgada.
Humidade, que escorre do peixe ou carne salgada.
\section{Salmoirar}
\begin{itemize}
\item {Grp. gram.:v. t.}
\end{itemize}
Pôr em salmoira; salgar.
\section{Salmonada}
\begin{itemize}
\item {Grp. gram.:f.}
\end{itemize}
Peixe, do gênero salmão.
(Cp. \textunderscore salmonado\textunderscore )
\section{Salmonado}
\begin{itemize}
\item {Grp. gram.:adj.}
\end{itemize}
\begin{itemize}
\item {Utilização:Zool.}
\end{itemize}
\begin{itemize}
\item {Proveniência:(Do lat. \textunderscore salmo\textunderscore )}
\end{itemize}
Que tem carne vermelha como a do salmão.
\section{Salmonejo}
\begin{itemize}
\item {Grp. gram.:m.}
\end{itemize}
\begin{itemize}
\item {Grp. gram.:Adj.}
\end{itemize}
\begin{itemize}
\item {Proveniência:(Do lat. \textunderscore salmo\textunderscore , \textunderscore salmonis\textunderscore )}
\end{itemize}
O mesmo que \textunderscore salmonete\textunderscore ^1.
Que se parece com o salmão.
\section{Salmonete}
\begin{itemize}
\item {fónica:nê}
\end{itemize}
\begin{itemize}
\item {Grp. gram.:m.}
\end{itemize}
\begin{itemize}
\item {Proveniência:(Do lat. \textunderscore salmo\textunderscore , \textunderscore salmonis\textunderscore )}
\end{itemize}
Peixe pércida, (\textunderscore mullus barbatus\textunderscore ).
Peixe gádida, (\textunderscore mora mediterranea\textunderscore ).
\section{Salmonete}
\begin{itemize}
\item {fónica:nê}
\end{itemize}
\begin{itemize}
\item {Grp. gram.:m.}
\end{itemize}
\begin{itemize}
\item {Utilização:Prov.}
\end{itemize}
\begin{itemize}
\item {Utilização:trasm.}
\end{itemize}
Descompostura.
(Cp. \textunderscore sabonete\textunderscore )
\section{Salmonídeo}
\begin{itemize}
\item {Grp. gram.:adj.}
\end{itemize}
\begin{itemize}
\item {Grp. gram.:M. pl.}
\end{itemize}
\begin{itemize}
\item {Proveniência:(Do lat. \textunderscore salmo\textunderscore , \textunderscore salmonis\textunderscore  + gr. \textunderscore eidos\textunderscore )}
\end{itemize}
Relativo ou semelhante ao salmão.
Família de peixes, que tem por typo o salmão.
\section{Salmoura}
\begin{itemize}
\item {Grp. gram.:f.}
\end{itemize}
\begin{itemize}
\item {Proveniência:(Do lat. \textunderscore sal\textunderscore  + \textunderscore muria\textunderscore )}
\end{itemize}
Porção de água, saturada de sal marinho, e applicada geralmente á conservação de substâncias orgânicas.
Vasilha, em que se conservam essas substâncias, com a respectiva água salgada.
Humidade, que escorre do peixe ou carne salgada.
\section{Salmourar}
\begin{itemize}
\item {Grp. gram.:v. t.}
\end{itemize}
Pôr em salmoura; salgar.
\section{Salobre}
\begin{itemize}
\item {fónica:lô}
\end{itemize}
\begin{itemize}
\item {Grp. gram.:adj.}
\end{itemize}
Que sabe um pouco a sal.
Diz-se da água, que têm em dissolução alguns saes ou substâncias, que a tornam desagradável.
(Cp. cast. \textunderscore salobre\textunderscore )
\section{Salobro}
\begin{itemize}
\item {fónica:lô}
\end{itemize}
\begin{itemize}
\item {Grp. gram.:adj.}
\end{itemize}
Que sabe um pouco a sal.
Diz-se da água, que têm em dissolução alguns saes ou substâncias, que a tornam desagradável.
(Cp. cast. \textunderscore salobre\textunderscore )
\section{Saloia}
\begin{itemize}
\item {Grp. gram.:f.}
\end{itemize}
Flexão fem. de saloio.
\section{Saloio}
\begin{itemize}
\item {Grp. gram.:m.  e  adj.}
\end{itemize}
\begin{itemize}
\item {Utilização:Fig.}
\end{itemize}
\begin{itemize}
\item {Utilização:Ant.}
\end{itemize}
\begin{itemize}
\item {Proveniência:(De \textunderscore çaloio\textunderscore , nome ár. de um tributo que em Lisbôa pagavam os padeiros moiros)}
\end{itemize}
Camponês ou aldeão dos arrabaldes de Lisbôa.
Aldeão.
Grosseiro.
Rústico.
Finório, velhaco.
Diz-se de um pão, feito de uma variedade de trigo durázio, que se cultiva perto de Lisbôa.
Moiro, originário de Salé.
\section{Salol}
\begin{itemize}
\item {Grp. gram.:m.}
\end{itemize}
\begin{itemize}
\item {Utilização:Pharm.}
\end{itemize}
Salicylato de phenol, antipyrético e antineurálgico.
\section{Salomónia}
\begin{itemize}
\item {Grp. gram.:f.}
\end{itemize}
\begin{itemize}
\item {Proveniência:(De \textunderscore Salomão\textunderscore , n. p.)}
\end{itemize}
Gênero de plantas polygaláceas, cujo typo cresce na China.
\section{Salomónico}
\begin{itemize}
\item {Grp. gram.:adj.}
\end{itemize}
\begin{itemize}
\item {Utilização:Archit.}
\end{itemize}
\begin{itemize}
\item {Proveniência:(De \textunderscore Salomão\textunderscore , n. p.)}
\end{itemize}
Relativo a Salomão.
Diz-se da columna, lavrada em espiral.
\section{Saloquinina}
\begin{itemize}
\item {Grp. gram.:f.}
\end{itemize}
\begin{itemize}
\item {Utilização:Pharm.}
\end{itemize}
Salicylato de quinina, contra neuralgias, febre typhóide, etc.
\section{Salosândalo}
\begin{itemize}
\item {fónica:sân}
\end{itemize}
\begin{itemize}
\item {Grp. gram.:m.}
\end{itemize}
\begin{itemize}
\item {Utilização:Chím.}
\end{itemize}
Producto da dissolução do salol em essência de sândalo.
\section{Salossândalo}
\begin{itemize}
\item {Grp. gram.:m.}
\end{itemize}
\begin{itemize}
\item {Utilização:Chím.}
\end{itemize}
Producto da dissolução do salol em essência de sândalo.
\section{Salpa}
\begin{itemize}
\item {Grp. gram.:f.}
\end{itemize}
\begin{itemize}
\item {Proveniência:(Lat. \textunderscore salpa\textunderscore )}
\end{itemize}
Gênero de animálculos phosphorescentes e gelatinosos.
\section{Salpheno}
\begin{itemize}
\item {Grp. gram.:m.}
\end{itemize}
Medicamento, de applicação semelhante á do salol.
\section{Salpiantho}
\begin{itemize}
\item {Grp. gram.:m.}
\end{itemize}
Gênero de plantas nyctagineas.
\section{Salpianto}
\begin{itemize}
\item {Grp. gram.:m.}
\end{itemize}
Gênero de plantas nictagineas.
\section{Salpica}
\begin{itemize}
\item {Grp. gram.:f.}
\end{itemize}
O mesmo que \textunderscore salpico\textunderscore . Cf. \textunderscore Luz e Calor\textunderscore , 259.
\section{Salpicador}
\begin{itemize}
\item {Grp. gram.:m.  e  adj.}
\end{itemize}
O que salpica.
\section{Salpicadura}
\begin{itemize}
\item {Grp. gram.:f.}
\end{itemize}
Acto ou effeito de salpicar.
\section{Salpicamento}
\begin{itemize}
\item {Grp. gram.:m.}
\end{itemize}
O mesmo que \textunderscore salpicadura\textunderscore .
\section{Salpicão}
\begin{itemize}
\item {Grp. gram.:m.}
\end{itemize}
Chouriço grosso, feito de presunto e ás vezes com vinho e alho.
O mesmo que \textunderscore salsichão\textunderscore .
(Cast. \textunderscore salpicón\textunderscore )
\section{Salopicar}
\begin{itemize}
\item {Grp. gram.:v. t.}
\end{itemize}
\begin{itemize}
\item {Utilização:Fig.}
\end{itemize}
\begin{itemize}
\item {Proveniência:(De \textunderscore sal\textunderscore  + \textunderscore picar\textunderscore )}
\end{itemize}
Salgar, espalhando gotas salgadas ou pedras de sal.
Sarapintar.
Espalhar manchas em.
Deitar pingos ou salpicos em.
Polvilhar; espalhar.
Desacreditar, manchar, infamar.
\section{Salpico}
\begin{itemize}
\item {Grp. gram.:m.}
\end{itemize}
\begin{itemize}
\item {Proveniência:(De \textunderscore salpicar\textunderscore )}
\end{itemize}
O mesmo que \textunderscore salpicadura\textunderscore .
Cada uma das pedras de sal, com que se salga o peixe ou a carne.
\section{Salpícola}
\begin{itemize}
\item {Grp. gram.:f.}
\end{itemize}
Planta escrofularínea, de flôres azues.
\section{Salpiglossa}
\begin{itemize}
\item {Grp. gram.:f.}
\end{itemize}
\begin{itemize}
\item {Proveniência:(Do lat. \textunderscore salpinx\textunderscore  + gr. \textunderscore glossa\textunderscore )}
\end{itemize}
Gênero de plantas, typo das salpiglósseas.
\section{Salpiglósseas}
\begin{itemize}
\item {Grp. gram.:f. pl.}
\end{itemize}
Tríbo de plantas escrofularíneas, cuja planta-typo cresce no Chile.
\section{Salpim}
\begin{itemize}
\item {Grp. gram.:m.}
\end{itemize}
\begin{itemize}
\item {Utilização:Açor}
\end{itemize}
Capote de grande cabeção, na ilha de San-Miguel.
\section{Salpimenta}
\begin{itemize}
\item {Grp. gram.:f.}
\end{itemize}
\begin{itemize}
\item {Grp. gram.:Adj.}
\end{itemize}
Mistura de sal e pimenta.
Branco e cinzento.
Grisalho.
\section{Salpimentar}
\begin{itemize}
\item {Grp. gram.:v. t.}
\end{itemize}
\begin{itemize}
\item {Utilização:Fig.}
\end{itemize}
\begin{itemize}
\item {Proveniência:(De \textunderscore salpimenta\textunderscore )}
\end{itemize}
Temperar com sal e pimenta.
Maltratar com palavras azedas. Cf. Moraes, \textunderscore Diccion.\textunderscore 
\section{Salpina}
\begin{itemize}
\item {Grp. gram.:f.}
\end{itemize}
\begin{itemize}
\item {Proveniência:(De \textunderscore salpa\textunderscore )}
\end{itemize}
Gênero de infusórios microscópicos.
\section{Salpinga}
\begin{itemize}
\item {Grp. gram.:f.}
\end{itemize}
\begin{itemize}
\item {Proveniência:(Do lat. \textunderscore salpinx\textunderscore , \textunderscore salpingis\textunderscore )}
\end{itemize}
Gênero de plantas melastomáceas.
\section{Salpinge}
\begin{itemize}
\item {Grp. gram.:m.}
\end{itemize}
\begin{itemize}
\item {Utilização:Anat.}
\end{itemize}
\begin{itemize}
\item {Proveniência:(Do lat. \textunderscore salpinx\textunderscore , \textunderscore salpingis\textunderscore )}
\end{itemize}
A trompa uterina.
\section{Salpingite}
\begin{itemize}
\item {Grp. gram.:f.}
\end{itemize}
\begin{itemize}
\item {Utilização:Med.}
\end{itemize}
Inflammação da salpinge.
\section{Salpingo}
\begin{itemize}
\item {Grp. gram.:m.}
\end{itemize}
\begin{itemize}
\item {Proveniência:(Do lat. \textunderscore salpinx\textunderscore , \textunderscore salpingis\textunderscore )}
\end{itemize}
Gênero de insectos coleópteros.
\section{Salpingorrafia}
\begin{itemize}
\item {Grp. gram.:f.}
\end{itemize}
\begin{itemize}
\item {Utilização:Med.}
\end{itemize}
\begin{itemize}
\item {Proveniência:(De gr. \textunderscore salpinx\textunderscore  + \textunderscore raphe\textunderscore )}
\end{itemize}
Sutura da trompa uterina.
\section{Salpingorrhaphia}
\begin{itemize}
\item {Grp. gram.:f.}
\end{itemize}
\begin{itemize}
\item {Utilização:Med.}
\end{itemize}
\begin{itemize}
\item {Proveniência:(De gr. \textunderscore salpinx\textunderscore  + \textunderscore raphe\textunderscore )}
\end{itemize}
Sutura da trompa uterina.
\section{Salpingotomia}
\begin{itemize}
\item {Grp. gram.:f.}
\end{itemize}
Incisão da salpinge.
\section{Sálpios}
\begin{itemize}
\item {Grp. gram.:m. pl.}
\end{itemize}
Família de animálculos microscópicos, que tem por typo a salpa.
\section{Salpór}
\begin{itemize}
\item {Grp. gram.:m.}
\end{itemize}
\begin{itemize}
\item {Utilização:Prov.}
\end{itemize}
\begin{itemize}
\item {Utilização:beir.}
\end{itemize}
O mesmo que \textunderscore serpol\textunderscore .
(Por \textunderscore selpór\textunderscore , metáth. de \textunderscore serpol\textunderscore )
\section{Salporinha}
\begin{itemize}
\item {Grp. gram.:f.}
\end{itemize}
\begin{itemize}
\item {Utilização:Prov.}
\end{itemize}
\begin{itemize}
\item {Utilização:trasm.}
\end{itemize}
\begin{itemize}
\item {Proveniência:(De \textunderscore salpór\textunderscore )}
\end{itemize}
Erva, de cheiro intenso, talvez o mesmo que \textunderscore salpór\textunderscore , e usada em curar azeitonas.
\section{Salpresar}
\begin{itemize}
\item {Grp. gram.:v. t.}
\end{itemize}
\begin{itemize}
\item {Proveniência:(De \textunderscore salprêso\textunderscore )}
\end{itemize}
Salgar um tanto.
\section{Salprêso}
\begin{itemize}
\item {Grp. gram.:adj.}
\end{itemize}
Que se salpresou; um tanto salgado.
\section{Salretas}
\begin{itemize}
\item {fónica:rê}
\end{itemize}
\begin{itemize}
\item {Grp. gram.:f. pl.}
\end{itemize}
\begin{itemize}
\item {Utilização:Prov.}
\end{itemize}
\begin{itemize}
\item {Utilização:alg.}
\end{itemize}
Diz-se que vai cheio até ás \textunderscore salretas\textunderscore  o barco, cuja carregação se eleva acima dos bancos.
\section{Salsa}
\begin{itemize}
\item {Grp. gram.:f.}
\end{itemize}
Planta umbellífera, cuja principal espécie é a salsa vulgar, (\textunderscore petroselium sativum\textunderscore ), muito usada em temperos culinários.
Môlho, para estimular o apetite e dar melhor sabor á carne e ao peixe. C. Moraes, \textunderscore Diccion.\textunderscore 
Espécie de uva branca.
Vulcão, que expelle líquidos lodosos, em que há matérias salinas.
Vulcão de lama.
Casta de uva da Arruda. Cf. \textunderscore Rev. Agron.\textunderscore , I, 18.
(Relaciona-se com o lat. \textunderscore salsus\textunderscore , como querem alguns diccion.? Parece-me duvidoso, pelo menos na primeira e segunda accepção. Entretanto, cp. fr. \textunderscore sauce\textunderscore , do lat. \textunderscore salsa\textunderscore )
\section{Salsa}
\begin{itemize}
\item {Grp. gram.:m.}
\end{itemize}
\begin{itemize}
\item {Utilização:Gír.}
\end{itemize}
\begin{itemize}
\item {Utilização:Ext.}
\end{itemize}
Indivíduo mascarado que, pelo Carnaval, percorre as ruas de Lisbôa, procurando têr ditos graciosos.
Peralta, homem presumido.
\section{Salsada}
\begin{itemize}
\item {Grp. gram.:f.}
\end{itemize}
\begin{itemize}
\item {Proveniência:(De \textunderscore salsa\textunderscore ?)}
\end{itemize}
Confusão, mistura, amálgama; embrulhada.
\section{Salsa-da-praia}
\begin{itemize}
\item {Grp. gram.:f.}
\end{itemize}
\begin{itemize}
\item {Utilização:Bras}
\end{itemize}
Planta medicinal.
\section{Salsa-do-monte}
\begin{itemize}
\item {Grp. gram.:f.}
\end{itemize}
Planta, o mesmo que \textunderscore aipo\textunderscore . Cf. B. Pereira, \textunderscore Prosódia\textunderscore , vb. \textunderscore oreosclinum\textunderscore .
\section{Salsa-parrilha}
\begin{itemize}
\item {Grp. gram.:f.}
\end{itemize}
\begin{itemize}
\item {Proveniência:(Do cast. \textunderscore zarza\textunderscore  + \textunderscore Parillo\textunderscore , n. p.)}
\end{itemize}
Planta, de origem americana, cuja raíz é depurativa e sudorífica.
Planta indígena, também conhecida por \textunderscore legação\textunderscore .
O mesmo que \textunderscore japecanga\textunderscore .
\section{Salseira}
\begin{itemize}
\item {Grp. gram.:f.}
\end{itemize}
\begin{itemize}
\item {Proveniência:(De \textunderscore salsa\textunderscore . Cp. fr. \textunderscore saucière\textunderscore )}
\end{itemize}
Vasilha, em que se servem môlhos, á mesa.
\section{Salseirada}
\begin{itemize}
\item {Grp. gram.:f.}
\end{itemize}
\begin{itemize}
\item {Proveniência:(De \textunderscore salseiro\textunderscore )}
\end{itemize}
Aguaceiro.
\section{Salseirinha}
\begin{itemize}
\item {Grp. gram.:f.}
\end{itemize}
\begin{itemize}
\item {Proveniência:(De \textunderscore salseira\textunderscore . Cp. cast. \textunderscore salsarilla\textunderscore )}
\end{itemize}
Pequena tigela, com tintas ou outros ingredientes, a qual os pintores precisam geralmente têr ao alcance da mão.
\section{Salseiro}
\begin{itemize}
\item {Grp. gram.:m.}
\end{itemize}
\begin{itemize}
\item {Utilização:T. de Setubal}
\end{itemize}
\begin{itemize}
\item {Proveniência:(De \textunderscore salso\textunderscore )}
\end{itemize}
Pancada de água, aguaceiro. Cf. Filinto, XVIII, 94.
Vento baixo e violento. Cp. Rev. \textunderscore Tradição\textunderscore , V, 11.
\section{Salsetano}
\begin{itemize}
\item {Grp. gram.:adj.}
\end{itemize}
\begin{itemize}
\item {Grp. gram.:M.}
\end{itemize}
Relativo a Salsete.
Habitante de Salsete.
\section{Salsicha}
\begin{itemize}
\item {Grp. gram.:f.}
\end{itemize}
\begin{itemize}
\item {Proveniência:(It. \textunderscore salsiccia\textunderscore , do lat. \textunderscore salsicia\textunderscore )}
\end{itemize}
Chouriço, linguíça.
Rastilho, com que antigamente se communicava fôgo ás minas.
\section{Salsichão}
\begin{itemize}
\item {Grp. gram.:m.}
\end{itemize}
Grande salsicha.
Paio.
Mólho de paus, que serve de faxina em as fortificações.
\section{Salsicharia}
\begin{itemize}
\item {Grp. gram.:f.}
\end{itemize}
\begin{itemize}
\item {Proveniência:(De \textunderscore salsicha\textunderscore )}
\end{itemize}
Estabelecimento ou arte de salsicheiro.
\section{Salsicheira}
\begin{itemize}
\item {Grp. gram.:f.}
\end{itemize}
Flexão fem. de \textunderscore salsicheiro\textunderscore .
\section{Salsicheiro}
\begin{itemize}
\item {Grp. gram.:m.}
\end{itemize}
Aquelle que faz salsichas ou salsichões.
Aquelle que vende artigos de salsicharia.
\section{Salsifré}
\begin{itemize}
\item {Grp. gram.:m.}
\end{itemize}
\begin{itemize}
\item {Utilização:Gír.}
\end{itemize}
Bailarico; sarau desenvolto.
(De \textunderscore salsa\textunderscore ^2)?
\section{Salsinha}
\begin{itemize}
\item {Grp. gram.:m.}
\end{itemize}
\begin{itemize}
\item {Utilização:Pop.}
\end{itemize}
Homem effeminado; maricas.
(Cp. \textunderscore salsa\textunderscore ^2)
\section{Salso}
\begin{itemize}
\item {Grp. gram.:adj.}
\end{itemize}
\begin{itemize}
\item {Utilização:Poét.}
\end{itemize}
\begin{itemize}
\item {Proveniência:(Lat. \textunderscore salsus\textunderscore )}
\end{itemize}
Salgado.
Diz-se especialmente do mar ou das águas do mar.
\section{Sálsola}
\begin{itemize}
\item {Grp. gram.:f.}
\end{itemize}
\begin{itemize}
\item {Proveniência:(Do lat. \textunderscore salsus\textunderscore )}
\end{itemize}
Nome scientífico da soda, planta.
\section{Salsoláceas}
\begin{itemize}
\item {Grp. gram.:f. pl.}
\end{itemize}
Família de plantas, que tem por typo a \textunderscore sálsola\textunderscore .
\section{Salsóleas}
\begin{itemize}
\item {Grp. gram.:f. pl.}
\end{itemize}
Tríbo de plantas salsoláceas.
\section{Salsugem}
\begin{itemize}
\item {Grp. gram.:f.}
\end{itemize}
\begin{itemize}
\item {Utilização:Med.}
\end{itemize}
\begin{itemize}
\item {Proveniência:(Lat. \textunderscore salsugo\textunderscore )}
\end{itemize}
Lodo, em que há substâncias salíferas.
Qualidade do que é salso.
Propriedade inherente ás águas do mar.
Affecção cutânea, mais conhecida por \textunderscore impetigem\textunderscore .
\section{Salsuginoso}
\begin{itemize}
\item {Grp. gram.:adj.}
\end{itemize}
\begin{itemize}
\item {Proveniência:(Do lat. \textunderscore salsugo\textunderscore )}
\end{itemize}
Que tem salsugem.
\section{Salta-caroço}
\begin{itemize}
\item {Grp. gram.:m.}
\end{itemize}
Variedade de pêssego, em que o caroço não adhere ao mesocarpo.
\section{Saltachão}
\begin{itemize}
\item {Grp. gram.:m.}
\end{itemize}
\begin{itemize}
\item {Utilização:Bras. do Maranhão}
\end{itemize}
Designação vulgar de um pássaro, nocivo aos frutos.
\section{Saltada}
\begin{itemize}
\item {Grp. gram.:f.}
\end{itemize}
\begin{itemize}
\item {Proveniência:(De \textunderscore saltar\textunderscore )}
\end{itemize}
Salto; grande salto.
Investida.
Incursão.
Ímpeto, no acto de saltar.
Entrada imprevista, feita por quem vai fazer pesquisas.
\section{Saltado}
\begin{itemize}
\item {Grp. gram.:adj.}
\end{itemize}
\begin{itemize}
\item {Proveniência:(De \textunderscore saltar\textunderscore )}
\end{itemize}
Saliente; que resái acima de um nível ou para fóra de um plano.
\section{Saltadoiro}
\begin{itemize}
\item {Grp. gram.:m.}
\end{itemize}
Rêde para a pesca de taínhas.
\section{Saltador}
\begin{itemize}
\item {Grp. gram.:m.  e  adj.}
\end{itemize}
\begin{itemize}
\item {Grp. gram.:M. pl.}
\end{itemize}
\begin{itemize}
\item {Proveniência:(Lat. \textunderscore saltador\textunderscore )}
\end{itemize}
O que salta.
Família de insectos orthópteros, cujas pernas posteriores são mais compridas que as anteriores e próprias para o salto.
\section{Saltadouro}
\begin{itemize}
\item {Grp. gram.:m.}
\end{itemize}
Rêde para a pesca de taínhas.
\section{Salta-marquês}
\begin{itemize}
\item {Grp. gram.:m.}
\end{itemize}
\begin{itemize}
\item {Utilização:Pop.}
\end{itemize}
O mesmo que \textunderscore gafanhoto\textunderscore . (Colhido nas Caldas da Raínha)
\section{Salta-montes}
\begin{itemize}
\item {Grp. gram.:m.}
\end{itemize}
Pequena ave canora do Oriente.
\section{Saltanga}
\begin{itemize}
\item {Grp. gram.:f.}
\end{itemize}
\begin{itemize}
\item {Utilização:Prov.}
\end{itemize}
\begin{itemize}
\item {Utilização:trasm.}
\end{itemize}
Fogueira, que se acende em lugar público, nas noites de San-João e de San-Pedro, e por cima da qual os rapazes e as raparigas saltam, folgando.
(Cp. \textunderscore salto\textunderscore ^1)
\section{Saltante}
\begin{itemize}
\item {Grp. gram.:adj.}
\end{itemize}
\begin{itemize}
\item {Utilização:Heráld.}
\end{itemize}
\begin{itemize}
\item {Proveniência:(Lat. \textunderscore saltans\textunderscore )}
\end{itemize}
Que salta; saltador.
Diz-se do animal, que, no campo do escudo, se representa formando salto.
\section{Saltão}
\begin{itemize}
\item {Grp. gram.:m.  e  adj.}
\end{itemize}
\begin{itemize}
\item {Grp. gram.:M.}
\end{itemize}
\begin{itemize}
\item {Utilização:Pop.}
\end{itemize}
O que salta muito ou dá saltos grandes.
Gafanhoto.
Mosquito, antes de terminar a sua metamorphose.
\section{Salta-paredes}
\begin{itemize}
\item {Grp. gram.:m.}
\end{itemize}
Ave trepadora mexicana, que se abriga principalmente nas igrejas.
\section{Salta-pocinhas}
\begin{itemize}
\item {Grp. gram.:m.}
\end{itemize}
\begin{itemize}
\item {Utilização:Pop.}
\end{itemize}
Indivíduo affectado, que não caminha desembaraçadamente, mas com passo vagaroso e adamado.
\section{Saltar}
\begin{itemize}
\item {Grp. gram.:v. i.}
\end{itemize}
\begin{itemize}
\item {Proveniência:(Lat. \textunderscore saltare\textunderscore )}
\end{itemize}
Dar salto ou saltos.
Pular, brincando.
Brotar.
Surgir com ímpeto.
Mudar rapidamente de posição ou de direcção.
Investir, assaltar.
Apear-se: \textunderscore saltar do cavallo\textunderscore .
Mudar de lugar, transpondo um fôsso, um muro, etc.
Passar por cima de; galgar: \textunderscore saltar uma parede\textunderscore .
Passar em claro; omittir.
\section{Salta-regra}
\begin{itemize}
\item {Grp. gram.:m.}
\end{itemize}
Instrumento, para medir ângulos, acuta.
\section{Saltarelar}
\begin{itemize}
\item {Grp. gram.:v. i.}
\end{itemize}
O mesmo que \textunderscore saltarilhar\textunderscore . Cf. Filinto, XIII, 284.
\section{Saltarellar}
\begin{itemize}
\item {Grp. gram.:v. i.}
\end{itemize}
O mesmo que \textunderscore saltarilhar\textunderscore . Cf. Filinto, XIII, 284.
\section{Saltarello}
\begin{itemize}
\item {Grp. gram.:adj.}
\end{itemize}
\begin{itemize}
\item {Grp. gram.:M.}
\end{itemize}
\begin{itemize}
\item {Proveniência:(It. \textunderscore saltarello\textunderscore )}
\end{itemize}
Que salta.
Espécie de dança popular.
\section{Saltarelo}
\begin{itemize}
\item {Grp. gram.:adj.}
\end{itemize}
\begin{itemize}
\item {Grp. gram.:M.}
\end{itemize}
\begin{itemize}
\item {Proveniência:(It. \textunderscore saltarello\textunderscore )}
\end{itemize}
Que salta.
Espécie de dança popular.
\section{Saltarico}
\begin{itemize}
\item {Grp. gram.:m.}
\end{itemize}
\begin{itemize}
\item {Utilização:Prov.}
\end{itemize}
\begin{itemize}
\item {Utilização:trasm.}
\end{itemize}
O mesmo que \textunderscore salta-marquês\textunderscore .
\section{Saltarilhar}
\begin{itemize}
\item {Grp. gram.:v. i.}
\end{itemize}
Andar aos saltos; saltitar.
\section{Saltarilho}
\begin{itemize}
\item {Grp. gram.:m.}
\end{itemize}
Aquelle que saltarilha.
\section{Saltarinhar}
\begin{itemize}
\item {Grp. gram.:v. i.}
\end{itemize}
O mesmo que \textunderscore saltarilhar\textunderscore .
\section{Salta-sebes}
\begin{itemize}
\item {Grp. gram.:m.}
\end{itemize}
Espécie de planta, (\textunderscore fumaria muralis\textunderscore , Sond.).
\section{Saltatrice}
\begin{itemize}
\item {Grp. gram.:f.}
\end{itemize}
O mesmo que \textunderscore dançarina\textunderscore , ou \textunderscore saltatriz\textunderscore .
\section{Saltratiz}
\begin{itemize}
\item {Grp. gram.:f.  e  adj.}
\end{itemize}
\begin{itemize}
\item {Utilização:Zool.}
\end{itemize}
\begin{itemize}
\item {Proveniência:(Lat. \textunderscore saltatrix\textunderscore )}
\end{itemize}
Mulhér, que salta; dançarina.
Espécie de aranha.
\section{Salta-vallados}
\begin{itemize}
\item {Grp. gram.:m.}
\end{itemize}
\begin{itemize}
\item {Utilização:Pop.}
\end{itemize}
O que salta muito, grande saltador.
\section{Salteada}
\begin{itemize}
\item {Grp. gram.:f.}
\end{itemize}
O mesmo que \textunderscore salteamento\textunderscore .
\section{Salteado}
\begin{itemize}
\item {Grp. gram.:adj.}
\end{itemize}
\begin{itemize}
\item {Proveniência:(De \textunderscore saltear\textunderscore )}
\end{itemize}
Exposto interpoladamente, sem seguimento, fora da ordem vulgar: \textunderscore sabe a lição de cór e salteada\textunderscore .
\section{Salteador}
\begin{itemize}
\item {Grp. gram.:m.  e  adj.}
\end{itemize}
O que salteia.
\section{Salteagem}
\begin{itemize}
\item {Grp. gram.:f.}
\end{itemize}
\begin{itemize}
\item {Utilização:Neol.}
\end{itemize}
O mesmo que \textunderscore salteamento\textunderscore .
\section{Salteamento}
\begin{itemize}
\item {Grp. gram.:m.}
\end{itemize}
Acto ou effeito de saltear.
\section{Saltear}
\begin{itemize}
\item {Grp. gram.:v. t.}
\end{itemize}
\begin{itemize}
\item {Grp. gram.:V. i.}
\end{itemize}
\begin{itemize}
\item {Grp. gram.:V. p.}
\end{itemize}
\begin{itemize}
\item {Proveniência:(De \textunderscore salto\textunderscore ^1)}
\end{itemize}
Assaltar.
Atacar de súbito, para matar ou roubar.
Roubar.
Surprehender.
Tomar de súbito.
Sêr salteador.
Viver do roubo.
Apavorar-se com uma notícia má.
Assustar-se; sobresaltar-se.
\section{Salteio}
\begin{itemize}
\item {Grp. gram.:m.}
\end{itemize}
Acto de saltear. Cf. Ed. Prado, \textunderscore Illusão Amer.\textunderscore 
\section{Salteiro}
\begin{itemize}
\item {Grp. gram.:m.}
\end{itemize}
O que faz saltos^1 de madeira para o calçado.
\section{Salteiro}
\begin{itemize}
\item {Grp. gram.:m.}
\end{itemize}
\begin{itemize}
\item {Utilização:Ant.}
\end{itemize}
O mesmo que \textunderscore psaltério\textunderscore .
Os 150 psalmos do David.
Os 7 psalmos penitenciaes.
Rosário de 150 ave-marias.
Residência parochial, presbytério.
\section{Saltério}
\begin{itemize}
\item {Grp. gram.:m.}
\end{itemize}
\begin{itemize}
\item {Utilização:Veter.}
\end{itemize}
\begin{itemize}
\item {Proveniência:(Lat. \textunderscore psalterium\textunderscore )}
\end{itemize}
O mesmo que \textunderscore psaltério\textunderscore .
Instrumento musical de cordas, que se dedilhavam ou se tocavam com o plectro.
Instrumento triangular moderno, com treze ordens de cordas, que se ferem com uma palheta.
O mesmo que \textunderscore folhoso\textunderscore , terceiro estômago dos ruminantes. Cf. Mac. Pinto, \textunderscore Comp. de Veter.\textunderscore , I, 468.
\section{Sáltria}
\begin{itemize}
\item {Grp. gram.:f.}
\end{itemize}
\begin{itemize}
\item {Proveniência:(Lat. \textunderscore psaltria\textunderscore )}
\end{itemize}
Mulhér que tocava cítara. Cf. Camillo, \textunderscore Maria da Fonte\textunderscore , 337.
\section{Saltígrado}
\begin{itemize}
\item {Grp. gram.:adj.}
\end{itemize}
\begin{itemize}
\item {Proveniência:(Do lat. \textunderscore saltus\textunderscore  + \textunderscore gradi\textunderscore )}
\end{itemize}
Que caminha, dando saltos.
\section{Saltimbanco}
\begin{itemize}
\item {Grp. gram.:m.}
\end{itemize}
\begin{itemize}
\item {Proveniência:(Do it. \textunderscore saltare\textunderscore  + \textunderscore in\textunderscore  + \textunderscore banco\textunderscore )}
\end{itemize}
Charlatão de feira ou de circo.
Histrião; pelotiqueiro.
\section{Saltimbarca}
\begin{itemize}
\item {Grp. gram.:f.}
\end{itemize}
Antigo vestuário rústico, aberto aos lados.
Espécie de balandrau, ou hábito dos condemnados a auto de fé. Cf. Filinto, XIII, 285; Camillo, \textunderscore Caveira\textunderscore , 663.
\section{Saltinvão}
\begin{itemize}
\item {Grp. gram.:m.}
\end{itemize}
\begin{itemize}
\item {Proveniência:(De \textunderscore salto\textunderscore  + \textunderscore em\textunderscore  + \textunderscore vão\textunderscore )}
\end{itemize}
Jôgo de rapazes.
\section{Saltitante}
\begin{itemize}
\item {Grp. gram.:adj.}
\end{itemize}
\begin{itemize}
\item {Proveniência:(Lat. \textunderscore saltitans\textunderscore )}
\end{itemize}
Que saltita.
\section{Saltitar}
\begin{itemize}
\item {Grp. gram.:v. i.}
\end{itemize}
\begin{itemize}
\item {Utilização:Ext.}
\end{itemize}
\begin{itemize}
\item {Proveniência:(Lat. \textunderscore saltitare\textunderscore )}
\end{itemize}
Dar saltos pequenos e frequentes.
Divagar de um assumpto para outro.
Mostrar inconstância.
\section{Salto}
\begin{itemize}
\item {Grp. gram.:m.}
\end{itemize}
\begin{itemize}
\item {Utilização:Pesc.}
\end{itemize}
\begin{itemize}
\item {Grp. gram.:Loc. adv.}
\end{itemize}
\begin{itemize}
\item {Grp. gram.:Pl.}
\end{itemize}
\begin{itemize}
\item {Utilização:Prov.}
\end{itemize}
\begin{itemize}
\item {Utilização:beir.}
\end{itemize}
\begin{itemize}
\item {Proveniência:(Lat. \textunderscore saltus\textunderscore )}
\end{itemize}
Movimento brusco, com que um corpo vivo se eleva do solo, lançando-se de um para outro lugar.
Movimento rápido de um corpo que, por effeito de quéda ou reflexão, se eleva acima de uma superfície.
Catadupa, cataracta.
Transição rápida de um estado para outro: \textunderscore a natureza não procede por saltos\textunderscore .
Pequena quantidade de um cabo náutico.
Assalto, saque, roubo na estrada.
Peça de madeira ou de coiro, para altear o calçado, na parte correspondente ao calcanhar.
Parte do tamanco, que corresponde ao salto das botas.
Jôgo de parada em três cartas contra uma.
Rêde, também chamada \textunderscore parreira\textunderscore , para apanhar os peixes que saltam para fóra da água, especialmente o robalo e a taínha. Cf. Rev. \textunderscore Tradição\textunderscore , V, 66.
\textunderscore De salto\textunderscore , de improviso; de repente; num pulo.
O mesmo que \textunderscore alpondras\textunderscore .
\section{Salto}
\begin{itemize}
\item {Grp. gram.:m.}
\end{itemize}
\begin{itemize}
\item {Utilização:Des.}
\end{itemize}
\begin{itemize}
\item {Proveniência:(Lat. \textunderscore saltus\textunderscore )}
\end{itemize}
O mesmo que \textunderscore bosque\textunderscore , brenha.
Cêrro, oiteiro.
Lugar eminente.
\section{Salto-beduíno}
\begin{itemize}
\item {Grp. gram.:m.}
\end{itemize}
\begin{itemize}
\item {Utilização:Gymn.}
\end{itemize}
Salto mortal, de costas ou para trás, com torção no ar e quéda de frente.
\section{Salto-mortal}
\begin{itemize}
\item {Grp. gram.:m.}
\end{itemize}
\begin{itemize}
\item {Utilização:Gymn.}
\end{itemize}
Salto para trás, em que as mãos tocam no chão, em quanto o corpo executa uma volta.
\section{Saltuário}
\begin{itemize}
\item {Grp. gram.:adj.}
\end{itemize}
\begin{itemize}
\item {Utilização:Ant.}
\end{itemize}
\begin{itemize}
\item {Grp. gram.:M.}
\end{itemize}
\begin{itemize}
\item {Proveniência:(Lat. \textunderscore saltuarius\textunderscore )}
\end{itemize}
Relativo a bosque.
Guarda de um bosque, guarda florestal ou guarda campestre, entre os antigos.
\section{Salubérrimo}
\begin{itemize}
\item {Grp. gram.:adj.}
\end{itemize}
\begin{itemize}
\item {Proveniência:(Lat. \textunderscore saluberrimus\textunderscore )}
\end{itemize}
Muito salubre; muito saudável.
\section{Salubre}
\begin{itemize}
\item {Grp. gram.:adj.}
\end{itemize}
\begin{itemize}
\item {Proveniência:(Lat. \textunderscore saluber\textunderscore )}
\end{itemize}
Saudável; propício á saúde.
Hygiênico.
Que se póde curar facilmente.
\section{Salubre}
\begin{itemize}
\item {Grp. gram.:m.}
\end{itemize}
Apparelho das officinas de cardagem nas fábricas de fiação, apparelho em que o algodão se converte em mecha.
(Não parece relacionar-se com \textunderscore salubre\textunderscore ^1)
\section{Salubridade}
\begin{itemize}
\item {Grp. gram.:f.}
\end{itemize}
\begin{itemize}
\item {Proveniência:(Lat. \textunderscore salubritas\textunderscore )}
\end{itemize}
Qualidade do que é salubre.
Conjunto das condições favoráveis á saúde.
\section{Salubrificar}
\begin{itemize}
\item {Grp. gram.:v. t.}
\end{itemize}
\begin{itemize}
\item {Proveniência:(Do lat. \textunderscore saluber\textunderscore  + \textunderscore facere\textunderscore )}
\end{itemize}
Tornar salubre, sanear.
\section{Salubrol}
\begin{itemize}
\item {Grp. gram.:m.}
\end{itemize}
\begin{itemize}
\item {Utilização:Pharm.}
\end{itemize}
Pó medicinal, succedâneo de iodofórmio e empregado em cicatrizes.
\section{Saluço}
\begin{itemize}
\item {Grp. gram.:m.}
\end{itemize}
\begin{itemize}
\item {Utilização:ant.}
\end{itemize}
\begin{itemize}
\item {Utilização:Pop.}
\end{itemize}
O mesmo que \textunderscore soluço\textunderscore . Cf. \textunderscore Lusíadas\textunderscore , II, 43.
\section{Saludador}
\begin{itemize}
\item {Grp. gram.:m.  e  adj.}
\end{itemize}
O que saluda.
\section{Saludar}
\begin{itemize}
\item {Grp. gram.:v. t.}
\end{itemize}
\begin{itemize}
\item {Proveniência:(Lat. \textunderscore salutare\textunderscore )}
\end{itemize}
Curar por meio de rezas; benzer para curar.
\section{Saluga}
\begin{itemize}
\item {Grp. gram.:f.}
\end{itemize}
\begin{itemize}
\item {Utilização:Prov.}
\end{itemize}
\begin{itemize}
\item {Utilização:trasm.}
\end{itemize}
Diz-se \textunderscore pão-de-saluga\textunderscore , o que não tem mistura.
\section{Salutante}
\begin{itemize}
\item {Grp. gram.:adj.}
\end{itemize}
\begin{itemize}
\item {Utilização:Des.}
\end{itemize}
\begin{itemize}
\item {Proveniência:(Do lat. \textunderscore salutans\textunderscore )}
\end{itemize}
Que saúda, ou cumprimenta.
\section{Salutar}
\begin{itemize}
\item {Grp. gram.:adj.}
\end{itemize}
\begin{itemize}
\item {Utilização:Fig.}
\end{itemize}
\begin{itemize}
\item {Proveniência:(Lat. \textunderscore salutaris\textunderscore )}
\end{itemize}
Conveniente para a conservação da saúde.
Hygiênico.
Fortificante.
Que faz bem ao espírito e ao coração.
Edificante; moralizador: \textunderscore lições salutares\textunderscore .
\section{Salutarmente}
\begin{itemize}
\item {Grp. gram.:adv.}
\end{itemize}
De modo salutar.
\section{Salutífero}
\begin{itemize}
\item {Grp. gram.:adj.}
\end{itemize}
\begin{itemize}
\item {Utilização:Poét.}
\end{itemize}
\begin{itemize}
\item {Utilização:Fig.}
\end{itemize}
\begin{itemize}
\item {Proveniência:(Lat. \textunderscore salutifer\textunderscore )}
\end{itemize}
Que dá saúde; saudável.
Favorável, útil.
\section{Salva}
\begin{itemize}
\item {Grp. gram.:f.}
\end{itemize}
\begin{itemize}
\item {Utilização:Ant.}
\end{itemize}
\begin{itemize}
\item {Utilização:Ant.}
\end{itemize}
\begin{itemize}
\item {Proveniência:(De \textunderscore salvar\textunderscore )}
\end{itemize}
Descarga de armas de fogo, em honra de alguém ou em sinal de regozijo.
Saudação official, manifestada por tiros de artilharia.
Saudação.
Resalva; subterfúgio.
Purificação canónica, juízo de Deus, ou prova terminante, com que alguém se mostrava livre de um crime.
O mesmo que \textunderscore segurança\textunderscore .
\section{Salva}
\begin{itemize}
\item {Grp. gram.:f.}
\end{itemize}
\begin{itemize}
\item {Proveniência:(Lat. \textunderscore salvia\textunderscore )}
\end{itemize}
Nome de várias plantas labiadas, asparagíneas, verbenáceas e compostas.
\section{Salva}
\begin{itemize}
\item {Grp. gram.:f.}
\end{itemize}
Espécie de bandeja.
\section{Salva}
\begin{itemize}
\item {Grp. gram.:f.}
\end{itemize}
\begin{itemize}
\item {Utilização:Ant.}
\end{itemize}
\begin{itemize}
\item {Proveniência:(De \textunderscore salvè\textunderscore )}
\end{itemize}
A oração christan \textunderscore Salvè, Raínha\textunderscore . Cf. Fern. Lopes, \textunderscore Chrón. de D. Fernando.\textunderscore 
\section{Salvabilidade}
\begin{itemize}
\item {Grp. gram.:f.}
\end{itemize}
Qualidade de salvável.
\section{Salvação}
\begin{itemize}
\item {Grp. gram.:f.}
\end{itemize}
\begin{itemize}
\item {Proveniência:(Lat. \textunderscore salvatio\textunderscore )}
\end{itemize}
Acto ou effeito de salvar, de saudar, de remir.
\section{Salvádego}
\begin{itemize}
\item {Grp. gram.:m.}
\end{itemize}
\begin{itemize}
\item {Grp. gram.:M.  e  adj.}
\end{itemize}
\begin{itemize}
\item {Proveniência:(Do lat. hyp. \textunderscore salvaticum\textunderscore )}
\end{itemize}
Gratificação extraordinária, que se dá aos marinheiros, pelos esforços que empregaram na salvação de navio ou carga, naufragados ou perseguidos por inimigos.
Diz-se o navio, empregado no salvamento dos despojos de naufrágios.--Ferreira Borges, que criou o voc., (\textunderscore Diccion. Jur.\textunderscore ), deu-lhe a pronúncia de \textunderscore salvadêgo\textunderscore , que é errónea.
\section{Salvado}
\begin{itemize}
\item {Grp. gram.:m.}
\end{itemize}
\begin{itemize}
\item {Utilização:Ant.}
\end{itemize}
\begin{itemize}
\item {Proveniência:(De \textunderscore salvar\textunderscore )}
\end{itemize}
Aquelle que provava terminantemente estar isento de culpa ou crime.
\section{Salvador}
\begin{itemize}
\item {Grp. gram.:m.  e  adj.}
\end{itemize}
\begin{itemize}
\item {Utilização:Restrict.}
\end{itemize}
\begin{itemize}
\item {Proveniência:(Lat. \textunderscore salvator\textunderscore )}
\end{itemize}
O que salva.
Christo.
Designação obsoleta de um dos empregados ou artífices da Casa da Moéda.
\section{Salvadora}
\begin{itemize}
\item {Grp. gram.:f.}
\end{itemize}
Gênero de plantas, que serve de typo ás salvadoráceas.
\section{Salvadoráceas}
\begin{itemize}
\item {Grp. gram.:f. pl.}
\end{itemize}
Família de plantas plumbagíneas, de fôlhas oppostas e fruto carnoso.
\section{Salvados}
\begin{itemize}
\item {Grp. gram.:m. pl.}
\end{itemize}
\begin{itemize}
\item {Proveniência:(De \textunderscore salvar\textunderscore )}
\end{itemize}
Tudo aquillo que escapou de uma catástrophe, especialmente de um incêndio ou de um naufrágio: \textunderscore um leilão de salvados\textunderscore .
\section{Salvage}
\begin{itemize}
\item {Grp. gram.:f.}
\end{itemize}
O mesmo que \textunderscore salvagem\textunderscore ^1. Cf. Herculano, \textunderscore Cistèr\textunderscore , II, 88.
\section{Salvagem}
\begin{itemize}
\item {Grp. gram.:f.}
\end{itemize}
\begin{itemize}
\item {Proveniência:(De \textunderscore salvar\textunderscore )}
\end{itemize}
Direito sôbre o que se salvou de um navio naufragado.
\section{Salvagem}
\begin{itemize}
\item {Grp. gram.:f.}
\end{itemize}
\begin{itemize}
\item {Proveniência:(De \textunderscore salvar\textunderscore ?)}
\end{itemize}
Antiga peça de artilharia.
\section{Salvaguarda}
\begin{itemize}
\item {Grp. gram.:f.}
\end{itemize}
\begin{itemize}
\item {Utilização:Fig.}
\end{itemize}
\begin{itemize}
\item {Proveniência:(De \textunderscore salvar\textunderscore  + \textunderscore guardar\textunderscore )}
\end{itemize}
Salvo conducto.
Cautela, resalva.
Pessôa, que proteje ou serve de defesa.
Coisa, que resguarda de um perigo.
\section{Salvaguardar}
\begin{itemize}
\item {Grp. gram.:v. t.}
\end{itemize}
\begin{itemize}
\item {Proveniência:(De \textunderscore salvaguarda\textunderscore )}
\end{itemize}
Livrar de perigo; proteger; acautelar; resalvar.
\section{Salvamento}
\begin{itemize}
\item {Grp. gram.:m.}
\end{itemize}
\begin{itemize}
\item {Proveniência:(De \textunderscore salvar\textunderscore )}
\end{itemize}
O mesmo que \textunderscore salvação\textunderscore .
Lugar, em que alguma coisa ou pessôa está isenta de perigo; segurança.
Bom êxito.
\section{Salvanda}
\begin{itemize}
\item {Grp. gram.:f.}
\end{itemize}
\begin{itemize}
\item {Proveniência:(De \textunderscore salvar\textunderscore ?)}
\end{itemize}
Ligeira camada de barro, entre o filão das minas e o terreno adjacente.
\section{Salvanor}
\begin{itemize}
\item {Grp. gram.:m.}
\end{itemize}
\begin{itemize}
\item {Utilização:Ant.}
\end{itemize}
\begin{itemize}
\item {Proveniência:(De \textunderscore salvo\textunderscore  + \textunderscore honor\textunderscore )}
\end{itemize}
O devido respeito:«\textunderscore falando com salvanor, tu diabo me pareces.\textunderscore »G. Vicente, I, 158.
\section{Salvante}
\begin{itemize}
\item {Grp. gram.:adj.}
\end{itemize}
\begin{itemize}
\item {Grp. gram.:Prep.}
\end{itemize}
Que salva.
Excepto; salvo; tirante:«\textunderscore ...salvante lembraruos que os fauoreçaes...\textunderscore »\textunderscore Eufrosina\textunderscore , 13.
\section{Salvantes}
\begin{itemize}
\item {Grp. gram.:prep.}
\end{itemize}
\begin{itemize}
\item {Utilização:Ant.}
\end{itemize}
Á excepção de; excepto; salvante.
(Cp. \textunderscore salvante\textunderscore )
\section{Salvar}
\begin{itemize}
\item {Grp. gram.:v. t.}
\end{itemize}
\begin{itemize}
\item {Grp. gram.:V. i.}
\end{itemize}
\begin{itemize}
\item {Grp. gram.:V. p.}
\end{itemize}
\begin{itemize}
\item {Proveniência:(Lat. \textunderscore salvare\textunderscore )}
\end{itemize}
Defender contra um perigo ou catástrophe.
Defender.
Livrar; preservar.
Pôr como condição.
Conservar intacto.
Livrar da morte.
Passar por címa de, saltanto.
Cumprimentar, saudar.
Livrar das penas do inferno.
Trazer ao bom caminho.
Dar salvas de artilharia.
Disparar canhão a bordo, para sinal de tempestade próxima.
Fazer saudações ou cumprimentos.
Livrar-se; escapar.
Fugir; refugiar-se.
Obter a bem-aventurança.
\section{Salvarana}
\begin{itemize}
\item {Grp. gram.:f.}
\end{itemize}
Árvore brasileira, própria para construcções.
\section{Salvatela}
\begin{itemize}
\item {Grp. gram.:adj. f.}
\end{itemize}
\begin{itemize}
\item {Utilização:Anat.}
\end{itemize}
Diz-se da veia, que vai das costas da mão até á parte interna do ante-braço.
(B. lat. \textunderscore salvatella\textunderscore )
\section{Salvatella}
\begin{itemize}
\item {Grp. gram.:adj. f.}
\end{itemize}
\begin{itemize}
\item {Utilização:Anat.}
\end{itemize}
Diz-se da veia, que vai das costas da mão até á parte interna do ante-braço.
(B. lat. \textunderscore salvatella\textunderscore )
\section{Salvatério}
\begin{itemize}
\item {Grp. gram.:m.}
\end{itemize}
\begin{itemize}
\item {Utilização:Pop.}
\end{itemize}
\begin{itemize}
\item {Proveniência:(De \textunderscore salvar\textunderscore )}
\end{itemize}
Salvamento; desculpa.
Expediente para escapar.
\section{Salvaterra}
\begin{itemize}
\item {Grp. gram.:f.}
\end{itemize}
\begin{itemize}
\item {Utilização:Obsol.}
\end{itemize}
Espada curta, semelhante ao punhal dos Romanos; cimitarra.
(B. lat. \textunderscore salvaterra\textunderscore )
\section{Salvatoriano}
\begin{itemize}
\item {Grp. gram.:adj.}
\end{itemize}
\begin{itemize}
\item {Grp. gram.:M.}
\end{itemize}
\begin{itemize}
\item {Proveniência:(Do lat. \textunderscore Salvator\textunderscore , n. p.)}
\end{itemize}
Relativo á república de San-Salvador.
Habitante de San-Salvador.
\section{Salvável}
\begin{itemize}
\item {Grp. gram.:adj.}
\end{itemize}
Que se póde salvar.
\section{Salva-vida}
\begin{itemize}
\item {Grp. gram.:f.}
\end{itemize}
\begin{itemize}
\item {Utilização:Bras}
\end{itemize}
O mesmo que \textunderscore arruda-dos-muros\textunderscore .
\section{Salva-vidas}
\begin{itemize}
\item {Grp. gram.:m.}
\end{itemize}
Qualquer apparelho, próprio para salvar náufragos, para evitar que alguém se submirja na água, para evitar o perigo pessoal num incêndio, etc.
\section{Salvè!}
\begin{itemize}
\item {Grp. gram.:interj.}
\end{itemize}
\begin{itemize}
\item {Proveniência:(Lat. \textunderscore salve\textunderscore , de \textunderscore salvere\textunderscore )}
\end{itemize}
(designativa de \textunderscore saudação\textunderscore  ou \textunderscore cumprimento\textunderscore )
Deus te salve; Deus vos salve.
\section{Salvè-raínha}
\begin{itemize}
\item {Grp. gram.:f.}
\end{itemize}
Oração christan, dirigida á Virgem Maria, e que começa pelas palavras \textunderscore salvè, raínha\textunderscore .
\section{Salveta}
\begin{itemize}
\item {fónica:vê}
\end{itemize}
\begin{itemize}
\item {Grp. gram.:f.}
\end{itemize}
\begin{itemize}
\item {Proveniência:(De \textunderscore salva\textunderscore ^3)}
\end{itemize}
Salva ou prato, em que se assentam os candeeiros de mesa.
\section{Salveta}
\begin{itemize}
\item {fónica:vê}
\end{itemize}
\begin{itemize}
\item {Grp. gram.:f.}
\end{itemize}
\begin{itemize}
\item {Proveniência:(De \textunderscore salva\textunderscore ^2)}
\end{itemize}
Planta, espécie de salva^2, (\textunderscore salvia officinalis\textunderscore ).
\section{Sálvia}
\begin{itemize}
\item {Grp. gram.:f.}
\end{itemize}
\begin{itemize}
\item {Proveniência:(Lat. \textunderscore salvia\textunderscore )}
\end{itemize}
Designação especial de salveta^2.
\section{Salvidade}
\begin{itemize}
\item {Grp. gram.:f.}
\end{itemize}
\begin{itemize}
\item {Utilização:Ant.}
\end{itemize}
\begin{itemize}
\item {Proveniência:(De \textunderscore salvo\textunderscore )}
\end{itemize}
O mesmo que \textunderscore salvação\textunderscore .
\section{Salvíneas}
\begin{itemize}
\item {Grp. gram.:f. pl.}
\end{itemize}
O mesmo que \textunderscore salviniáceas\textunderscore .
\section{Salvínia}
\begin{itemize}
\item {Grp. gram.:f.}
\end{itemize}
\begin{itemize}
\item {Proveniência:(De \textunderscore Salvini\textunderscore , n. p.)}
\end{itemize}
Planta cryptogâmica aquática.
\section{Salviniáceas}
\begin{itemize}
\item {Grp. gram.:f. pl.}
\end{itemize}
Família de plantas, que tem por typo a salvínia.
(Fem. pl. de \textunderscore salviniáceo\textunderscore )
\section{Salviniáceo}
\begin{itemize}
\item {Grp. gram.:adj.}
\end{itemize}
Relativo ou semelhante á salvínia.
\section{Salvo}
\begin{itemize}
\item {Grp. gram.:adj.}
\end{itemize}
\begin{itemize}
\item {Utilização:Ant.}
\end{itemize}
\begin{itemize}
\item {Grp. gram.:Prep.}
\end{itemize}
\begin{itemize}
\item {Grp. gram.:Loc. adv.}
\end{itemize}
\begin{itemize}
\item {Grp. gram.:Loc. adv.}
\end{itemize}
\begin{itemize}
\item {Proveniência:(Lat. \textunderscore salvus\textunderscore )}
\end{itemize}
Livre de um perigo, de uma doença, de uma difficuldade, de uma contrariedade ou desgôsto.
Animador, propício.
Resalvado.
Exceptuado, omittido.
O mesmo que [[permittido|permittir]].
Á excepção de, excepto.
\textunderscore A salvo\textunderscore , sem perigo; com segurança; livre de risco; em lugar seguro.
\textunderscore Em salvo\textunderscore , em lugar seguro.
\section{Salvo-conducto}
\begin{itemize}
\item {Grp. gram.:m.}
\end{itemize}
\begin{itemize}
\item {Utilização:Fig.}
\end{itemize}
Licença por escrito, para alguém viajar ou transitar livremente.
Segurança; privilégio.
\section{Sama}
\begin{itemize}
\item {Grp. gram.:f.}
\end{itemize}
\begin{itemize}
\item {Utilização:Pop.}
\end{itemize}
O mesmo que \textunderscore caruma\textunderscore . (Colhido em Tôrres-Vedras)
\section{Samabumbulo}
\begin{itemize}
\item {Grp. gram.:m.}
\end{itemize}
Insecto africano, que constrói a sua habitação de terra junto ao varejo da cobertura das cubatas.
\section{Samagaio}
\begin{itemize}
\item {Grp. gram.:m.}
\end{itemize}
\begin{itemize}
\item {Utilização:Prov.}
\end{itemize}
\begin{itemize}
\item {Utilização:minh.}
\end{itemize}
Pão, usado nas festas de baptizado, em Guimarães, e que, no dia do baptismo, a madrinha do neóphyto deve dar a quem o peça.
\section{Samambaia}
\begin{itemize}
\item {Grp. gram.:f.}
\end{itemize}
Planta polypodiácea do Brasil, espécie de fêto.
\section{Samanco}
\begin{itemize}
\item {Grp. gram.:m.}
\end{itemize}
\begin{itemize}
\item {Utilização:Prov.}
\end{itemize}
O mesmo que \textunderscore tamanco\textunderscore .
\section{Samango}
\begin{itemize}
\item {Grp. gram.:m.}
\end{itemize}
\begin{itemize}
\item {Utilização:Bras}
\end{itemize}
Homem preguiçoso.
Maltrapilho.
\section{Samanguaiá}
\begin{itemize}
\item {Grp. gram.:m.}
\end{itemize}
\begin{itemize}
\item {Utilização:Bras. do S}
\end{itemize}
Mollusco acéphalo.
\section{Sâmara}
\begin{itemize}
\item {Grp. gram.:f.}
\end{itemize}
\begin{itemize}
\item {Utilização:Bot.}
\end{itemize}
\begin{itemize}
\item {Proveniência:(Lat. \textunderscore samara\textunderscore )}
\end{itemize}
Fruto dehiscente, cujo pericarpo offerece uma ou mais dobras membranosas ou asas.
\section{Samarídeo}
\begin{itemize}
\item {Grp. gram.:adj.}
\end{itemize}
\begin{itemize}
\item {Utilização:Bot.}
\end{itemize}
\begin{itemize}
\item {Proveniência:(Do lat. \textunderscore samara\textunderscore  + gr. \textunderscore eidos\textunderscore )}
\end{itemize}
Diz-se do fruto, composto de muitas sâmaras, ligadas pela base.
\section{Samaritana}
\begin{itemize}
\item {Grp. gram.:f.}
\end{itemize}
\begin{itemize}
\item {Proveniência:(De \textunderscore samaritano\textunderscore )}
\end{itemize}
Mulhér de Samaria, com quem Jesus teve uma conversação edificante, referida no \textunderscore Evangelho\textunderscore  de San-João.
\section{Samaritano}
\begin{itemize}
\item {Grp. gram.:adj.}
\end{itemize}
\begin{itemize}
\item {Grp. gram.:M.}
\end{itemize}
Relativo a Samaria.
Habitante de Samaria.
Língua dos samaritanos.
\section{Samarra}
\begin{itemize}
\item {Grp. gram.:f.}
\end{itemize}
\begin{itemize}
\item {Utilização:Prov.}
\end{itemize}
\begin{itemize}
\item {Utilização:trasm.}
\end{itemize}
\begin{itemize}
\item {Utilização:Prov.}
\end{itemize}
\begin{itemize}
\item {Utilização:minh.}
\end{itemize}
\begin{itemize}
\item {Grp. gram.:M.}
\end{itemize}
\begin{itemize}
\item {Utilização:Deprec.}
\end{itemize}
Chimarra.
Vestuário antigo e rústico de pelles de ovelha.
Pello de ovelha ou carneiro, em quanto conserva a lan.
Traje ecclesiástico, chimarra.
Homem carcunda; a giba do carcunda.
Costado, costas: \textunderscore foi-lhe á samarra\textunderscore .
Padre.
(Cast. \textunderscore zamarra\textunderscore )
\section{Samarrão}
\begin{itemize}
\item {Grp. gram.:m.}
\end{itemize}
\begin{itemize}
\item {Utilização:Prov.}
\end{itemize}
\begin{itemize}
\item {Utilização:beir.}
\end{itemize}
\begin{itemize}
\item {Utilização:T. de Turquel}
\end{itemize}
Grande samarra.
Mulhér pública, coiro.
Homem muito gordo.
\section{Samarreiro}
\begin{itemize}
\item {Grp. gram.:m.}
\end{itemize}
\begin{itemize}
\item {Proveniência:(De \textunderscore samarra\textunderscore )}
\end{itemize}
Negociante de pelles de ovelha e carneiro.
\section{Samarrinho}
\begin{itemize}
\item {Grp. gram.:m.}
\end{itemize}
Variedade de uva preta.
\section{Samarro}
\begin{itemize}
\item {Grp. gram.:m.}
\end{itemize}
O mesmo que \textunderscore samarra\textunderscore . Cf. G. Vicente, I.
\section{Samatra}
\begin{itemize}
\item {Grp. gram.:f.}
\end{itemize}
\begin{itemize}
\item {Utilização:Gír.}
\end{itemize}
Pênis.
Bebedeira.
\section{Samaúma}
\begin{itemize}
\item {Grp. gram.:f.}
\end{itemize}
\begin{itemize}
\item {Utilização:Ext.}
\end{itemize}
Grande árvore bombácea, procedente da América.
Algodão, produzido por esta planta.
Pêlo, que cobre as sementes de algumas plantas, e que, como o algodão da samaúma, serve para encher colchões, almofadas, etc.
\section{Samba}
\begin{itemize}
\item {Grp. gram.:m.}
\end{itemize}
\begin{itemize}
\item {Utilização:Bras}
\end{itemize}
Bailado popular.
\section{Sambacaeté}
\begin{itemize}
\item {Grp. gram.:m.}
\end{itemize}
Arbusto labiado do Brasil.
\section{Sambacuim}
\begin{itemize}
\item {Grp. gram.:m.}
\end{itemize}
Árvore urticácea do Brasil.
\section{Sambaíba}
\begin{itemize}
\item {Grp. gram.:f.}
\end{itemize}
Nome de várias plantas do Brasil.
\section{Sambaibinha}
\begin{itemize}
\item {fónica:ba-i}
\end{itemize}
\begin{itemize}
\item {Grp. gram.:f.}
\end{itemize}
Nome de algumas plantas dilleniáceas do Brasil.
\section{Sambambaia}
\begin{itemize}
\item {Grp. gram.:f.}
\end{itemize}
O mesmo que \textunderscore samambaia\textunderscore .
\section{Sambaqui}
\begin{itemize}
\item {Grp. gram.:m.}
\end{itemize}
\begin{itemize}
\item {Utilização:Bras}
\end{itemize}
Depósito antigo de cascas de ostras e outras conchas.
\section{Sambar}
\begin{itemize}
\item {Grp. gram.:v. i.}
\end{itemize}
\begin{itemize}
\item {Utilização:Bras}
\end{itemize}
Frequentar sambas.
\section{Sambarca}
\begin{itemize}
\item {Grp. gram.:f.}
\end{itemize}
\begin{itemize}
\item {Proveniência:(De \textunderscore assambarcar\textunderscore ?)}
\end{itemize}
Faixa, com que se rodeia o peito das cavalgaduras, para que os tirantes as não magôem.
Fáixa, com que as mulheres do povo cingiam o peito.
Travessa, que se pregava nas portas das casas penhoradas.
\section{Sambarcar}
\begin{itemize}
\item {Grp. gram.:v. i.}
\end{itemize}
\begin{itemize}
\item {Utilização:Ant.}
\end{itemize}
Fechar com sambarca (as portas dos fallidos ou de casas penhoradas).
\section{Sambarco}
\begin{itemize}
\item {Grp. gram.:m.}
\end{itemize}
\begin{itemize}
\item {Utilização:Ant.}
\end{itemize}
Cinto largo, que as mulheres usavam por baixo dos peitos.
Sapato, chinelo.
(Cp. \textunderscore sambarca\textunderscore )
\section{Sambear}
\begin{itemize}
\item {Grp. gram.:v. t.}
\end{itemize}
\begin{itemize}
\item {Utilização:Bras. do N}
\end{itemize}
Dançar em samba.
Frequentar sambas.
\section{Sambenitar}
\begin{itemize}
\item {Grp. gram.:v. t.}
\end{itemize}
O mesmo que \textunderscore ensambenitar\textunderscore .
\section{Sambenito}
\begin{itemize}
\item {Grp. gram.:m.}
\end{itemize}
Hábito, que se enfiava pela cabeça em fórma de saco, e que se vestia aos condemnados que iam nos autos de fé.
(Cast. \textunderscore sambenito\textunderscore )
\section{Sambento}
\begin{itemize}
\item {Grp. gram.:f.}
\end{itemize}
Variedade de pêra portuguesa.
\section{Sambereba}
\begin{itemize}
\item {Grp. gram.:f.}
\end{itemize}
(V.sebereba)
\section{Sambernardo}
\begin{itemize}
\item {Grp. gram.:f.}
\end{itemize}
Variedade de pêra portuguesa, de qualidade inferior.
Casta de cães alpinos, muito estimada.
\section{Sambista}
\begin{itemize}
\item {Grp. gram.:m.  e  f.}
\end{itemize}
\begin{itemize}
\item {Utilização:Bras}
\end{itemize}
Pessôa, que frequenta sambas.
\section{Samblage}
\begin{itemize}
\item {Grp. gram.:f.}
\end{itemize}
\begin{itemize}
\item {Utilização:Ant.}
\end{itemize}
\begin{itemize}
\item {Proveniência:(Fr. \textunderscore assemblage\textunderscore )}
\end{itemize}
Reunião de muitas peças de carpintaria. Cp. \textunderscore ensamblar\textunderscore .
\section{Samblagem}
\begin{itemize}
\item {Grp. gram.:f.}
\end{itemize}
\begin{itemize}
\item {Utilização:Ant.}
\end{itemize}
\begin{itemize}
\item {Proveniência:(Fr. \textunderscore assemblage\textunderscore )}
\end{itemize}
Reunião de muitas peças de carpintaria. Cp. \textunderscore ensamblar\textunderscore .
\section{Samblar}
\textunderscore v. t.\textunderscore  (e der.)
(V. \textunderscore ensamblar\textunderscore , etc.)
\section{Sambo}
\begin{itemize}
\item {Grp. gram.:m.}
\end{itemize}
Árvore angolense de Caconda.
\section{Sambongo}
\begin{itemize}
\item {Grp. gram.:m.}
\end{itemize}
\begin{itemize}
\item {Utilização:Bras}
\end{itemize}
Doce de côco e mel.
\section{Samborá}
\begin{itemize}
\item {Grp. gram.:m.}
\end{itemize}
\begin{itemize}
\item {Utilização:Bras. de Piauí}
\end{itemize}
O póllen das flôres.
Substância amarela e ácida, que se encontra no interior das abelheiras.
\section{Sambuca}
\begin{itemize}
\item {Grp. gram.:f.}
\end{itemize}
\begin{itemize}
\item {Proveniência:(Lat. \textunderscore sambuca\textunderscore )}
\end{itemize}
Antigo instrumento de cordas, de fórma triangular.
Antiga máquina de guerra, espécie de ponte de assalto para o ataque de fortalezas.
\section{Sambucáceas}
\begin{itemize}
\item {Grp. gram.:f. pl.}
\end{itemize}
\begin{itemize}
\item {Proveniência:(Do lat. \textunderscore sambucus\textunderscore )}
\end{itemize}
Família de plantas, que tem por typo o sabugueiro.
\section{Sambúceas}
\begin{itemize}
\item {Grp. gram.:f. pl.}
\end{itemize}
(V.sambucáceas)
\section{Sambucina}
\begin{itemize}
\item {Grp. gram.:f.}
\end{itemize}
\begin{itemize}
\item {Proveniência:(Lat. \textunderscore sambucina\textunderscore )}
\end{itemize}
Tocadora de sambuca.
\section{Sambucina}
\begin{itemize}
\item {Grp. gram.:f.}
\end{itemize}
\begin{itemize}
\item {Proveniência:(Do lat. \textunderscore sambucus\textunderscore )}
\end{itemize}
Substância particular, que existe na flôr do sabugueiro.
\section{Sambucístria}
\begin{itemize}
\item {Grp. gram.:f.}
\end{itemize}
\begin{itemize}
\item {Proveniência:(Lat. \textunderscore sambucistria\textunderscore )}
\end{itemize}
O mesmo que \textunderscore sambucina\textunderscore ^1.
\section{Sambuco}
\begin{itemize}
\item {Grp. gram.:m.}
\end{itemize}
Pequena embarcação indiana.
\section{Sâmbula}
\begin{itemize}
\item {Grp. gram.:f.}
\end{itemize}
\begin{itemize}
\item {Utilização:Bot.}
\end{itemize}
Planta umbellífera, medicinal. Cf. \textunderscore Pharmacopeia Port.\textunderscore 
\section{Sambúmbia}
\begin{itemize}
\item {Grp. gram.:f.}
\end{itemize}
Bebida fermentada, feita de água com suco de cana ou outro ingrediente, e usada pelos Cubanos.
\section{Samburá}
\begin{itemize}
\item {Grp. gram.:m.}
\end{itemize}
\begin{itemize}
\item {Utilização:Bras}
\end{itemize}
Espécie de cesto, em que os pescadores levam a isca.
\section{Samburro}
\begin{itemize}
\item {Grp. gram.:f.}
\end{itemize}
\begin{itemize}
\item {Utilização:Prov.}
\end{itemize}
\begin{itemize}
\item {Utilização:minh.}
\end{itemize}
O mesmo que \textunderscore zaburro\textunderscore .
\section{Samear}
\begin{itemize}
\item {Grp. gram.:v. t.}
\end{itemize}
\begin{itemize}
\item {Utilização:Pop.}
\end{itemize}
O mesmo que \textunderscore semear\textunderscore .
\section{Samelo}
\begin{itemize}
\item {fónica:mê}
\end{itemize}
\begin{itemize}
\item {Grp. gram.:m.}
\end{itemize}
\begin{itemize}
\item {Utilização:Prov.}
\end{itemize}
Pequena pedra: \textunderscore o garoto atirou-lhe um samelo\textunderscore .
\section{Sàmenina}
\begin{itemize}
\item {Utilização:T. de Buarcos}
\end{itemize}
Bóia grande.
\section{Samessuga}
\begin{itemize}
\item {Grp. gram.:f.}
\end{itemize}
\begin{itemize}
\item {Utilização:Prov.}
\end{itemize}
\begin{itemize}
\item {Utilização:Prov.}
\end{itemize}
\begin{itemize}
\item {Utilização:minh.}
\end{itemize}
O mesmo que \textunderscore sanguesuga\textunderscore .
O mesmo que \textunderscore sinapismo\textunderscore .
\section{Samiana}
\begin{itemize}
\item {Grp. gram.:f.  e  adj.}
\end{itemize}
Diz-se de uma espécie de terra branca e medicinal, que vem da ilha de Samos.
\section{Samica}
\begin{itemize}
\item {Grp. gram.:adv.}
\end{itemize}
\begin{itemize}
\item {Utilização:Ant.}
\end{itemize}
O mesmo que \textunderscore samicas\textunderscore ^2.
\section{Samicas}
\begin{itemize}
\item {Grp. gram.:m.}
\end{itemize}
\begin{itemize}
\item {Utilização:Pop.}
\end{itemize}
O mesmo que \textunderscore maricas\textunderscore .
\section{Samicas}
\begin{itemize}
\item {Grp. gram.:adv.}
\end{itemize}
\begin{itemize}
\item {Utilização:Ant.}
\end{itemize}
Talvez; porventura; também; quiçá.
\section{Samiel}
\begin{itemize}
\item {Grp. gram.:m.}
\end{itemize}
Vento forte e perigoso, que sopra ao sul da Pérsia.
\section{Samintar}
\begin{itemize}
\item {Grp. gram.:v. t.}
\end{itemize}
\begin{itemize}
\item {Utilização:T. da Bairrada}
\end{itemize}
Diffundir, espalhar; desperdiçar.
(Alter. de \textunderscore sementar\textunderscore )
\section{Sã}
\begin{itemize}
\item {Grp. gram.:adj. f.}
\end{itemize}
(Fem. de \textunderscore são\textunderscore ^1)
\section{Samidáceas}
\begin{itemize}
\item {Grp. gram.:f. pl.}
\end{itemize}
Família de plantas indianas, a que pertence a satagana.
\section{Sâmio}
\begin{itemize}
\item {Grp. gram.:adj.}
\end{itemize}
\begin{itemize}
\item {Grp. gram.:M.}
\end{itemize}
\begin{itemize}
\item {Grp. gram.:Pl.}
\end{itemize}
\begin{itemize}
\item {Proveniência:(Lat. \textunderscore samius\textunderscore )}
\end{itemize}
Relativo á ilha de Samos.
Vaso frágil, feito de terra sâmia.
Habitantes de Samos.
\section{Samnita}
\begin{itemize}
\item {Grp. gram.:m.}
\end{itemize}
O mesmo que \textunderscore samnite\textunderscore .
\section{Samnite}
\begin{itemize}
\item {Grp. gram.:m.}
\end{itemize}
\begin{itemize}
\item {Grp. gram.:Pl.}
\end{itemize}
\begin{itemize}
\item {Proveniência:(Lat. \textunderscore samnis\textunderscore , \textunderscore samnitis\textunderscore )}
\end{itemize}
Gênero de gladiadores romanos.
Povo da Itália antiga.
\section{Samo}
\begin{itemize}
\item {Grp. gram.:m.}
\end{itemize}
O mesmo que \textunderscore alburno\textunderscore .
\section{Samo}
\begin{itemize}
\item {Grp. gram.:m.}
\end{itemize}
\begin{itemize}
\item {Utilização:Prov.}
\end{itemize}
O mesmo que \textunderscore capatão\textunderscore .
\section{Samoanos}
\begin{itemize}
\item {Grp. gram.:m. pl.}
\end{itemize}
Habitantes do archipélago de Samôa, na Oceânia. Cf. \textunderscore Jornal-de-Viagens\textunderscore , VI, 274.
\section{Samóbia}
\begin{itemize}
\item {Grp. gram.:f.}
\end{itemize}
Gênero de molluscos.
\section{Samoco}
\begin{itemize}
\item {fónica:mô}
\end{itemize}
\begin{itemize}
\item {Grp. gram.:m.}
\end{itemize}
O mesmo que \textunderscore samouco\textunderscore , planta.
\section{Samoieda}
\begin{itemize}
\item {Grp. gram.:m.}
\end{itemize}
Grupo de línguas, o mesmo que \textunderscore samoiedo\textunderscore .
\section{Samoiédico}
\begin{itemize}
\item {Grp. gram.:adj.}
\end{itemize}
Relativo aos Samoiedos; o mesmo que \textunderscore samoiedo\textunderscore .
\section{Samoiedo}
\begin{itemize}
\item {Grp. gram.:adj.}
\end{itemize}
\begin{itemize}
\item {Grp. gram.:M.}
\end{itemize}
\begin{itemize}
\item {Grp. gram.:Pl.}
\end{itemize}
Relativo aos Samoiedos.
Grupo de línguas uralo-altaicas.
Habitantes da parte setentrional do império russo.
(Do russo \textunderscore samoied\textunderscore )
\section{Samóleas}
\begin{itemize}
\item {Grp. gram.:f. pl.}
\end{itemize}
Tríbo de plantas, que tem por typo o sâmolo.
\section{Sâmolo}
\begin{itemize}
\item {Grp. gram.:m.}
\end{itemize}
Gênero de plantas primuláceas.
\section{Samorais}
\begin{itemize}
\item {Grp. gram.:m. pl.}
\end{itemize}
A classe dos serviçaes, no Japão. Cf. V. Moraes, \textunderscore Dai-Nippon\textunderscore , 26.
\section{Samorense}
\begin{itemize}
\item {Grp. gram.:adj.}
\end{itemize}
Relativo a Samora. Cf. Herculano, \textunderscore Hist. de Port.\textunderscore , I, 170.
\section{Samouco}
\begin{itemize}
\item {Grp. gram.:m.}
\end{itemize}
Planta myriácea, (\textunderscore myrica faia\textunderscore ).
Crôsta, que a pedra traz, separando-se da pedreira.
\section{Sampana}
\begin{itemize}
\item {Grp. gram.:f.}
\end{itemize}
Barco, em que as cortesans da Indo-China rodeiam os navios ancorados, para tentar os marinheiros.
\section{Sampar}
\begin{itemize}
\item {Grp. gram.:v. t.}
\end{itemize}
\begin{itemize}
\item {Utilização:Bras. do S}
\end{itemize}
Atirar; arremessar.
\section{Samponha}
\begin{itemize}
\item {Grp. gram.:f.}
\end{itemize}
\begin{itemize}
\item {Utilização:Des.}
\end{itemize}
O mesmo que \textunderscore sanfona\textunderscore . Cf. Filinto, VIII, 264.
\section{Samudo}
\begin{itemize}
\item {Grp. gram.:adj.}
\end{itemize}
\begin{itemize}
\item {Utilização:Prov.}
\end{itemize}
Diz-se da vara ou tronco que tem muita rama.
\section{Samydáceas}
\begin{itemize}
\item {Grp. gram.:f. pl.}
\end{itemize}
Família de plantas indianas, a que pertence a satagana.
\section{Samýdeas}
\begin{itemize}
\item {Grp. gram.:f. pl.}
\end{itemize}
(V.samydáceas)
\section{San}
Abrev. de \textunderscore santo\textunderscore .
O mesmo ou melhor que \textunderscore são\textunderscore ^2.
\section{San}
\begin{itemize}
\item {Grp. gram.:adj. f.}
\end{itemize}
(Fem. de \textunderscore são\textunderscore ^1)
\section{San}
\begin{itemize}
\item {Grp. gram.:f.}
\end{itemize}
\begin{itemize}
\item {Utilização:Prov.}
\end{itemize}
Verme da carne de porco mal curtida; vareja.
\section{Sanação}
\begin{itemize}
\item {Grp. gram.:f.}
\end{itemize}
\begin{itemize}
\item {Proveniência:(Do lat. \textunderscore sanatio\textunderscore )}
\end{itemize}
Acto ou effeito de sanar. Cf. Herculano, \textunderscore Hist. de Port.\textunderscore , III, 401.
\section{Sanador}
\begin{itemize}
\item {Grp. gram.:adj.}
\end{itemize}
\begin{itemize}
\item {Proveniência:(Do lat. \textunderscore sanator\textunderscore )}
\end{itemize}
O mesmo que \textunderscore sanativo\textunderscore .
\section{Sanambaia}
\begin{itemize}
\item {Grp. gram.:f.}
\end{itemize}
\begin{itemize}
\item {Utilização:Bras}
\end{itemize}
Planta aquática, (\textunderscore ceratopteris thalictroides\textunderscore ).
\section{Sanamunda}
\begin{itemize}
\item {Grp. gram.:f.}
\end{itemize}
O mesmo que \textunderscore erva-benta\textunderscore .
\section{Sanan}
\begin{itemize}
\item {Grp. gram.:m.}
\end{itemize}
Ave brasileira, espécie de saracura pequena.
\section{Sanar}
\begin{itemize}
\item {Grp. gram.:v. t.}
\end{itemize}
\begin{itemize}
\item {Utilização:Fig.}
\end{itemize}
\begin{itemize}
\item {Proveniência:(Lat. \textunderscore sanare\textunderscore )}
\end{itemize}
Tornar são; curar.
Remediar.
Obstar a (um mal, uma difficuldade).
\section{Sanativo}
\begin{itemize}
\item {Grp. gram.:adj.}
\end{itemize}
\begin{itemize}
\item {Proveniência:(Lat. \textunderscore sanativus\textunderscore )}
\end{itemize}
Que sana; próprio para sanar.
\section{Sanatogênio}
\begin{itemize}
\item {Grp. gram.:m.}
\end{itemize}
Preparação alimentícia de caseína e soda.
\section{Sanatório}
\begin{itemize}
\item {Grp. gram.:m.}
\end{itemize}
\begin{itemize}
\item {Proveniência:(Do lat. \textunderscore sanare\textunderscore )}
\end{itemize}
Estabelecimento ou residência, apropriada para doentes ou convalescentes.
\section{Sanável}
\begin{itemize}
\item {Grp. gram.:adj.}
\end{itemize}
\begin{itemize}
\item {Proveniência:(Lat. \textunderscore sanabilis\textunderscore )}
\end{itemize}
Que se póde sanar.
\section{San-bento}
\begin{itemize}
\item {Grp. gram.:f.}
\end{itemize}
Variedade de pêra portuguesa.
\section{San-bernardo}
\begin{itemize}
\item {Grp. gram.:f.}
\end{itemize}
Variedade de pêra portuguesa, de qualidade inferior.
Casta de cães alpinos, muito estimada.
\section{Sanca}
\begin{itemize}
\item {Grp. gram.:f.}
\end{itemize}
\begin{itemize}
\item {Utilização:Prov.}
\end{itemize}
\begin{itemize}
\item {Utilização:trasm.}
\end{itemize}
\begin{itemize}
\item {Utilização:Náut.}
\end{itemize}
Cimalha convexa, que liga uma parede a um tecto.
Parte do telhado, assente sôbre a espessura da parede.
O mesmo que \textunderscore chanca\textunderscore .
Parte do corrimão, que sai fóra do talabardão.
(Cast. \textunderscore zanca\textunderscore )
\section{Sancadilha}
\begin{itemize}
\item {Grp. gram.:f.}
\end{itemize}
\begin{itemize}
\item {Utilização:Prov.}
\end{itemize}
\begin{itemize}
\item {Utilização:trasm.}
\end{itemize}
Cambapé.
Cunha, para calçar pontões.
Acaso; bambúrrio.
(Cast. \textunderscore zancadilla\textunderscore )
\section{San-caetano}
\begin{itemize}
\item {Grp. gram.:m.}
\end{itemize}
\begin{itemize}
\item {Utilização:Bras}
\end{itemize}
Erva medicinal.
\section{Sancan}
\begin{itemize}
\item {Grp. gram.:m.}
\end{itemize}
\begin{itemize}
\item {Utilização:Bras}
\end{itemize}
O mesmo que \textunderscore sacanga\textunderscore .
\section{Sancarrão}
\begin{itemize}
\item {Grp. gram.:m.}
\end{itemize}
\begin{itemize}
\item {Grp. gram.:Adj.}
\end{itemize}
Sanco grande.
Desajeitado; estrambótico.
Feio.
Ignorante, lerdo.
(Cast. \textunderscore zancarrón\textunderscore )
\section{Sancção}
\begin{itemize}
\item {Grp. gram.:f.}
\end{itemize}
\begin{itemize}
\item {Proveniência:(Do lat. \textunderscore sanctio\textunderscore )}
\end{itemize}
Approvação, que o chefe do Estado dá a uma lei.
Parte da lei, em que se indicam as penas contra os que a transgridem.
Cláusula, em que se assegura a execução de uma lei.
Confirmação, approvação.
\section{Sanccionador}
\begin{itemize}
\item {Grp. gram.:m.  e  adj.}
\end{itemize}
O que sancciona.
\section{Sanccionar}
\begin{itemize}
\item {Grp. gram.:v. t.}
\end{itemize}
\begin{itemize}
\item {Proveniência:(Do lat. \textunderscore sanctio\textunderscore )}
\end{itemize}
Dar sancção a.
Ratificar; confirmar.
\section{Sanceno}
\begin{itemize}
\item {Grp. gram.:m.}
\end{itemize}
\begin{itemize}
\item {Utilização:Prov.}
\end{itemize}
\begin{itemize}
\item {Utilização:trasm.}
\end{itemize}
O mesmo que \textunderscore sincelo\textunderscore ^1.
\section{Sancha}
\begin{itemize}
\item {Grp. gram.:f.}
\end{itemize}
\begin{itemize}
\item {Utilização:Prov.}
\end{itemize}
Variedade de cogumelo.
\section{Sancheira}
\begin{itemize}
\item {Grp. gram.:f.}
\end{itemize}
\begin{itemize}
\item {Utilização:Prov.}
\end{itemize}
Lugar, onde há muitas sanchas.
\section{Sanchete}
\begin{itemize}
\item {fónica:chê}
\end{itemize}
\begin{itemize}
\item {Grp. gram.:m.}
\end{itemize}
Moéda de prata, mandada cunhar pelo rei Sancho, o \textunderscore Sábio\textunderscore , de Navarra.
\section{Sanchézia}
\begin{itemize}
\item {Grp. gram.:f.}
\end{itemize}
\begin{itemize}
\item {Proveniência:(De \textunderscore Sanchez\textunderscore , n. p.)}
\end{itemize}
Gênero de plantas escrofularíneas do Peru.
\section{Sanco}
\begin{itemize}
\item {Grp. gram.:m.}
\end{itemize}
\begin{itemize}
\item {Utilização:Fig.}
\end{itemize}
\begin{itemize}
\item {Utilização:T. de Viana}
\end{itemize}
Perna da ave, desde a garra até á junta da coxa.
Perna delgada.
Perna de qualquer animal de açougue.
(Cast. \textunderscore zanco\textunderscore )
\section{Sancristão}
\begin{itemize}
\item {Grp. gram.:m.}
\end{itemize}
(Corr. \textunderscore pop.\textunderscore  e \textunderscore ant.\textunderscore  de \textunderscore sacristão\textunderscore ) Cf. Filinto, X, 123.
\section{Sancristia}
\begin{itemize}
\item {Grp. gram.:f.}
\end{itemize}
(Corr. de \textunderscore sacristia\textunderscore ) Cf. \textunderscore Peregrinação\textunderscore , LXIX.
\section{Sandala}
\begin{itemize}
\item {Grp. gram.:f.}
\end{itemize}
Ponte, usada entre os Levantinos, para descarga de navios.
\section{Sandalado}
\begin{itemize}
\item {Grp. gram.:adj.}
\end{itemize}
Perfumado de sândalo.
\section{Sandalha}
\begin{itemize}
\item {Grp. gram.:f.}
\end{itemize}
\begin{itemize}
\item {Utilização:P. us.}
\end{itemize}
O mesmo que \textunderscore sandália\textunderscore .
\section{Sandália}
\begin{itemize}
\item {Grp. gram.:f.}
\end{itemize}
\begin{itemize}
\item {Proveniência:(Lat. \textunderscore sandalia\textunderscore )}
\end{itemize}
Espécie de calçado, formado de uma sola ligada ao pé por correias.
Abarca; chinela antiga.
\section{Sandalino}
\begin{itemize}
\item {Grp. gram.:adj.}
\end{itemize}
Relativo a sândalo.
Feito de sândalo.
Que tem aroma parecido ao do sândalo.
\section{Sandaliólitho}
\begin{itemize}
\item {Grp. gram.:m.}
\end{itemize}
\begin{itemize}
\item {Proveniência:(Do lat. \textunderscore sandalium\textunderscore  + gr. \textunderscore lithos\textunderscore )}
\end{itemize}
Madrépora fóssil, que imita a fórma de um pé humano.
\section{Sandaliólito}
\begin{itemize}
\item {Grp. gram.:m.}
\end{itemize}
\begin{itemize}
\item {Proveniência:(Do lat. \textunderscore sandalium\textunderscore  + gr. \textunderscore lithos\textunderscore )}
\end{itemize}
Madrépora fóssil, que imita a fórma de um pé humano.
\section{Sandálitho}
\begin{itemize}
\item {Grp. gram.:m.}
\end{itemize}
\begin{itemize}
\item {Proveniência:(De \textunderscore sândalo\textunderscore  + gr. \textunderscore lithos\textunderscore )}
\end{itemize}
Madeira de sândalo petrificada.
\section{Sandálito}
\begin{itemize}
\item {Grp. gram.:m.}
\end{itemize}
\begin{itemize}
\item {Proveniência:(De \textunderscore sândalo\textunderscore  + gr. \textunderscore lithos\textunderscore )}
\end{itemize}
Madeira de sândalo petrificada.
\section{Sândalo}
\begin{itemize}
\item {Grp. gram.:m.}
\end{itemize}
Gênero de árvores indianas, typo das santaláceas, (\textunderscore santalus album\textunderscore , Lin.), também conhecido por \textunderscore sândalo branco\textunderscore , \textunderscore sândalo citrino\textunderscore  e \textunderscore sândalo amarelo\textunderscore .--Não se confunde com o sândalo vermelho, árvore leguminosa, (\textunderscore pterocarpus santalinus\textunderscore , Lin.).
(Ár. \textunderscore çandal\textunderscore )
\section{Sandambungi}
\begin{itemize}
\item {Grp. gram.:m.}
\end{itemize}
Ave africana, espécie de tordo.
\section{Sandápila}
\begin{itemize}
\item {Grp. gram.:f.}
\end{itemize}
\begin{itemize}
\item {Proveniência:(Lat. \textunderscore sandapila\textunderscore )}
\end{itemize}
Espécie de maca ou tumba, em que os defuntos pobres eram levados á cova, entre os Romanos.
\section{Sandapilário}
\begin{itemize}
\item {Grp. gram.:m.}
\end{itemize}
\begin{itemize}
\item {Proveniência:(Lat. \textunderscore sandapilarius\textunderscore )}
\end{itemize}
Cada um dos indivíduos que conduziam a sandápila.
\section{Sandar}
\begin{itemize}
\item {Grp. gram.:v. t.}
\end{itemize}
\begin{itemize}
\item {Utilização:Prov.}
\end{itemize}
\begin{itemize}
\item {Utilização:minh.}
\end{itemize}
O mesmo que \textunderscore sarar\textunderscore .
(Cp. \textunderscore sanidade\textunderscore )
\section{Sandará}
\begin{itemize}
\item {Grp. gram.:m.}
\end{itemize}
Árvore de Damão, (\textunderscore terminalia glabra\textunderscore ).
\section{Sandáraca}
\begin{itemize}
\item {Grp. gram.:f.}
\end{itemize}
\begin{itemize}
\item {Utilização:Miner.}
\end{itemize}
\begin{itemize}
\item {Proveniência:(Lat. \textunderscore sandaraca\textunderscore )}
\end{itemize}
Resina aromática de algumas árvores.
Arsênico rubro.
\section{Sandareso}
\begin{itemize}
\item {Grp. gram.:m.}
\end{itemize}
\begin{itemize}
\item {Proveniência:(Lat. \textunderscore sandaresus\textunderscore )}
\end{itemize}
Variedade de pedra preciosa do Oriente, descrita por Plinio.
\section{Sandasiro}
\begin{itemize}
\item {Grp. gram.:m.}
\end{itemize}
\begin{itemize}
\item {Proveniência:(Lat. \textunderscore sandasirus\textunderscore )}
\end{itemize}
Variedade de pedra preciosa, talvez o mesmo que \textunderscore sandareso\textunderscore .--Valdez, \textunderscore Diccion. Esp. Port.\textunderscore , menciona \textunderscore sandastro\textunderscore , que supponho êrro, procedente da fácil troca de um \textunderscore i\textunderscore  por um \textunderscore t\textunderscore .
\section{Sandejar}
\begin{itemize}
\item {Grp. gram.:v. i.}
\end{itemize}
Dizer sandices.
Proceder como sandeu.
\section{Sandeu}
\begin{itemize}
\item {Grp. gram.:m.  e  adj.}
\end{itemize}
\begin{itemize}
\item {Grp. gram.:M.}
\end{itemize}
\begin{itemize}
\item {Utilização:Ant.}
\end{itemize}
Idiota; mentecapto; pateta.
O mesmo que \textunderscore bobo\textunderscore . Cf. Rev. \textunderscore Tradição\textunderscore , II, 17.
(Cast. \textunderscore sandeo\textunderscore )
\section{Sandia}
\begin{itemize}
\item {Grp. gram.:f.  e  adj.}
\end{itemize}
(Fem. de \textunderscore sandeu\textunderscore )
\section{Sandiamente}
\begin{itemize}
\item {Grp. gram.:adv.}
\end{itemize}
De modo sandio; tolamente; á maneira de pateta.
\section{Sandiçal}
\begin{itemize}
\item {Grp. gram.:adj.}
\end{itemize}
Relativo a sandice; em que há sandice. Cf. Macedo, \textunderscore Burros\textunderscore , 284.
\section{Sandicino}
\begin{itemize}
\item {Grp. gram.:adj.}
\end{itemize}
\begin{itemize}
\item {Utilização:Des.}
\end{itemize}
\begin{itemize}
\item {Proveniência:(De \textunderscore sândalo\textunderscore  vermelho?)}
\end{itemize}
O mesmo que \textunderscore vermelho\textunderscore .
\section{Sandim}
\begin{itemize}
\item {Grp. gram.:m.}
\end{itemize}
Planta rhamnácea, (\textunderscore rhamnus alaternus\textunderscore ).
\section{Sandimento}
\begin{itemize}
\item {Grp. gram.:m.}
\end{itemize}
\begin{itemize}
\item {Utilização:Prov.}
\end{itemize}
\begin{itemize}
\item {Utilização:trasm.}
\end{itemize}
O mesmo que \textunderscore descimento\textunderscore .
(Corr. de \textunderscore descendimento\textunderscore )
\section{Sandio}
\begin{itemize}
\item {Grp. gram.:adj.}
\end{itemize}
Próprio de sandeu; insensato; disparatado.
(Cast. \textunderscore sandio\textunderscore )
\section{Sandiz}
\begin{itemize}
\item {Grp. gram.:m.}
\end{itemize}
\begin{itemize}
\item {Proveniência:(Do gr. \textunderscore sandux\textunderscore )}
\end{itemize}
Alvaiade calcinado.
Espécie de mínio ou côr semelhante á do mínio.
Erva, de flôr escarlate.
\section{Sandomingos}
\begin{itemize}
\item {Grp. gram.:m.}
\end{itemize}
Variedade de tabaco. Cf. \textunderscore Inquér. Indust.\textunderscore , 2.^a p., vol. I, 320 e 334.
\section{Sanduíche}
\begin{itemize}
\item {Grp. gram.:f.}
\end{itemize}
\begin{itemize}
\item {Proveniência:(Do ingl. \textunderscore sandwich\textunderscore )}
\end{itemize}
Conjunto de duas fatias de pão, tendo entre si uma tira de fiambre, ou salame ou queijo, etc.
\section{Saneamento}
\begin{itemize}
\item {Grp. gram.:m.}
\end{itemize}
Acto ou effeito de sanear.
\section{Sanear}
\begin{itemize}
\item {Grp. gram.:v. t.}
\end{itemize}
\begin{itemize}
\item {Utilização:Fig.}
\end{itemize}
\begin{itemize}
\item {Grp. gram.:V. i.}
\end{itemize}
\begin{itemize}
\item {Utilização:P. us.}
\end{itemize}
\begin{itemize}
\item {Proveniência:(Do lat. \textunderscore sanus\textunderscore )}
\end{itemize}
O mesmo que \textunderscore sanar\textunderscore .
Tornar apto para se respirar: \textunderscore sanear o ambiente\textunderscore .
Tornar habitável.
Tornar apto para a cultura.
Restituir ao estado normal.
Remediar.
Pôr côbro a.
Congraçar-se; captar a amizade.
\section{Saneável}
\begin{itemize}
\item {Grp. gram.:adj.}
\end{itemize}
Que se póde sanear.
\section{Sanedivão}
\begin{itemize}
\item {Grp. gram.:m.}
\end{itemize}
\begin{itemize}
\item {Utilização:Ant.}
\end{itemize}
Multa que, na Índia Portuguesa, pagavam os cobradores de rendas, quando prevaricavam.
\section{Sanefa}
\begin{itemize}
\item {Grp. gram.:f.}
\end{itemize}
\begin{itemize}
\item {Proveniência:(Do ár. \textunderscore aç-çanifa\textunderscore )}
\end{itemize}
Larga tira de fazenda, que se atravessa como ornato na extremidade superior de uma cortina, nas vêrgas das janelas, etc.
Tábua atravessada, a que se segura uma série de outras, que são verticaes áquella.
\section{Sanfeno}
\begin{itemize}
\item {Grp. gram.:m.}
\end{itemize}
\begin{itemize}
\item {Proveniência:(De \textunderscore são\textunderscore ^1 + \textunderscore feno\textunderscore )}
\end{itemize}
Planta leguminosa, própria para pasto de gados.
\section{Sanfona}
\begin{itemize}
\item {Grp. gram.:f.}
\end{itemize}
\begin{itemize}
\item {Grp. gram.:M.}
\end{itemize}
\begin{itemize}
\item {Utilização:Fam.}
\end{itemize}
\begin{itemize}
\item {Proveniência:(De \textunderscore symphónia\textunderscore , por \textunderscore symphonia\textunderscore )}
\end{itemize}
Instrumento músico, com cordas de tripa muito tensas, que são friccionadas por uma roda, posta em movimento por uma manivela.
Utensílio de ferreiro, o mesmo que \textunderscore rabeca\textunderscore .
Homem insignificante, bisbórria.
\section{Sanfonha}
\begin{itemize}
\item {Grp. gram.:m.}
\end{itemize}
\begin{itemize}
\item {Utilização:Pop.}
\end{itemize}
O mesmo que \textunderscore sanfona\textunderscore .
\section{Sanfonina}
\begin{itemize}
\item {Grp. gram.:f.}
\end{itemize}
\begin{itemize}
\item {Utilização:Chul.}
\end{itemize}
\begin{itemize}
\item {Grp. gram.:M.}
\end{itemize}
Sanfona pequena.
Cantilena desentoada.
Tocador de sanfona.
\section{Sanfoninar}
\begin{itemize}
\item {Grp. gram.:v. i.}
\end{itemize}
\begin{itemize}
\item {Utilização:Pop.}
\end{itemize}
\begin{itemize}
\item {Proveniência:(De \textunderscore sanfonina\textunderscore )}
\end{itemize}
Tocar sanfona.
Tocar mal qualquer instrumento de corda.
Falar importunamente; serranizar.
\section{Sanfonineiro}
\begin{itemize}
\item {Grp. gram.:m.}
\end{itemize}
Aquelle que sanfonina.
\section{Sanforinheiro}
\begin{itemize}
\item {Grp. gram.:m.}
\end{itemize}
\begin{itemize}
\item {Utilização:Prov.}
\end{itemize}
\begin{itemize}
\item {Utilização:beir.}
\end{itemize}
O mesmo que \textunderscore sanfonineiro\textunderscore .
Sujeito inquieto ou metediço.
\section{Sanga}
\begin{itemize}
\item {Grp. gram.:f.}
\end{itemize}
\begin{itemize}
\item {Utilização:Bras. do S}
\end{itemize}
\begin{itemize}
\item {Proveniência:(Do cast. \textunderscore zanja\textunderscore )}
\end{itemize}
Escavação funda, produzida num terreno pela chuva ou por correntes subterrâneas.
\section{Sanga}
\begin{itemize}
\item {Grp. gram.:f.}
\end{itemize}
\begin{itemize}
\item {Utilização:Bras}
\end{itemize}
O mesmo que \textunderscore algirão\textunderscore .
\section{Sangado}
\begin{itemize}
\item {Grp. gram.:adj.}
\end{itemize}
\begin{itemize}
\item {Utilização:Bras}
\end{itemize}
Apanhado na sanga^2.
\section{Sangage}
\begin{itemize}
\item {Grp. gram.:m.}
\end{itemize}
Homem nobre de Ternate. Cf. Barros, \textunderscore Déc.\textunderscore  IV, l. IX. c. 20.
\section{Sangalha}
\begin{itemize}
\item {Grp. gram.:f.}
\end{itemize}
Antiga medida de sólidos e líquidos.
(Cp. \textunderscore sangalho\textunderscore )
\section{Sangalheiro}
\begin{itemize}
\item {Grp. gram.:m.}
\end{itemize}
\begin{itemize}
\item {Grp. gram.:Adj.}
\end{itemize}
Habitante de Sangalhos.
Relativo a Sangalhos.
Diz-se de uma variedade de pêso da Bairrada.
Dizia-se das medidas do estalão do extinto concelho de Sangalhos.
\section{Sanedrim}
\begin{itemize}
\item {Grp. gram.:m.}
\end{itemize}
O mesmo que \textunderscore sanédrio\textunderscore .
\section{Sanédrio}
\begin{itemize}
\item {Grp. gram.:m.}
\end{itemize}
(V.synhedrim)
\section{Sangalhera}
\begin{itemize}
\item {fónica:lhê}
\end{itemize}
\begin{itemize}
\item {Grp. gram.:adj. f.}
\end{itemize}
Dizia-se das medidas sangalheiras.--Us. em vários foraes de D. Manuel.
\section{Sangalho}
\begin{itemize}
\item {Grp. gram.:m.}
\end{itemize}
Antiga medida de cinco celamins.
\section{Sangangu}
\begin{itemize}
\item {Grp. gram.:m.}
\end{itemize}
\begin{itemize}
\item {Utilização:Bras}
\end{itemize}
\begin{itemize}
\item {Utilização:chul.}
\end{itemize}
Zaragata, desordem com pancadaria.
\section{Sangeaco}
\begin{itemize}
\item {Grp. gram.:m.}
\end{itemize}
\begin{itemize}
\item {Utilização:Ant.}
\end{itemize}
Governador de território, no Egypto e em terra de Turcos:«\textunderscore nesta batalha morreo o Baxá dos Turcos, e elegerão outro, que era um Sangeaco, chamado...\textunderscore »Couto, \textunderscore Dec.\textunderscore , VII, c. 10.--Moraes e Silva, \textunderscore Diccion.\textunderscore , escreveu \textunderscore sangíaco\textunderscore , pronúncia que tenho por inexacta.
(Cp. \textunderscore sanjaco\textunderscore . Do turco \textunderscore sangak\textunderscore )
\section{Sangês}
\begin{itemize}
\item {Grp. gram.:m.}
\end{itemize}
\begin{itemize}
\item {Utilização:Prov.}
\end{itemize}
\begin{itemize}
\item {Utilização:minh.}
\end{itemize}
Comediante; palhaço.
Pessôa ridícula.
(Talvez de \textunderscore San-Gens\textunderscore  ou \textunderscore San-Genésio\textunderscore , de quem diz a lenda que se convertera, quando parodiava no theatro o baptismo dos Christãos.)
\section{Sangoé}
\begin{itemize}
\item {Grp. gram.:m.}
\end{itemize}
Réptil angolense, (\textunderscore monitor saurus\textunderscore , Lin.).
\section{Sangoim}
\begin{itemize}
\item {Grp. gram.:m.}
\end{itemize}
Outra fórma de \textunderscore sagui\textunderscore .
\section{Sangra}
\begin{itemize}
\item {Grp. gram.:f.}
\end{itemize}
\begin{itemize}
\item {Proveniência:(De \textunderscore sangrar\textunderscore )}
\end{itemize}
Líquido arroxado, que mana da azeitona comprimida ou em pilha.
\section{Sangradeira}
\begin{itemize}
\item {Grp. gram.:f.}
\end{itemize}
\begin{itemize}
\item {Proveniência:(De \textunderscore sangrar\textunderscore )}
\end{itemize}
Portal que, nas salinas, põe em communicação os crystallizadores com o entraval.
\section{Sangradoiro}
\begin{itemize}
\item {Grp. gram.:m.}
\end{itemize}
\begin{itemize}
\item {Utilização:Bras}
\end{itemize}
\begin{itemize}
\item {Proveniência:(De \textunderscore sangrar\textunderscore )}
\end{itemize}
Parte do braço, opposta ao cotovelo e preferida para a sangria.
Sulco ou lugar, por onde se desvia parte da água de um rio ou fonte.
Lugar, no pescoço dos animaes, onde se dá o golpe, para os matar.
\section{Sangrador}
\begin{itemize}
\item {Grp. gram.:m.  e  adj.}
\end{itemize}
O que sangra.
\section{Sangradouro}
\begin{itemize}
\item {Grp. gram.:m.}
\end{itemize}
\begin{itemize}
\item {Utilização:Bras}
\end{itemize}
\begin{itemize}
\item {Proveniência:(De \textunderscore sangrar\textunderscore )}
\end{itemize}
Parte do braço, opposta ao cotovelo e preferida para a sangria.
Sulco ou lugar, por onde se desvia parte da água de um rio ou fonte.
Lugar, no pescoço dos animaes, onde se dá o golpe, para os matar.
\section{Sangradura}
\begin{itemize}
\item {Grp. gram.:f.}
\end{itemize}
Acto ou effeito de sangrar.
\section{Sangralinga}
\begin{itemize}
\item {Grp. gram.:f.}
\end{itemize}
\begin{itemize}
\item {Proveniência:(De \textunderscore sangrar\textunderscore  + \textunderscore língua\textunderscore )}
\end{itemize}
Erva, de fôlhas longas e ásperas.
\section{Sangralíngua}
\begin{itemize}
\item {Grp. gram.:f.}
\end{itemize}
O mesmo que \textunderscore sangralinga\textunderscore .
\section{Sangra-mocho}
\begin{itemize}
\item {Grp. gram.:m.}
\end{itemize}
\begin{itemize}
\item {Utilização:Prov.}
\end{itemize}
\begin{itemize}
\item {Utilização:trasm.}
\end{itemize}
Armadilha, para caçar pássaros.
\section{Sangranho}
\begin{itemize}
\item {Grp. gram.:m.}
\end{itemize}
\begin{itemize}
\item {Utilização:Prov.}
\end{itemize}
\begin{itemize}
\item {Utilização:trasm.}
\end{itemize}
Sargaço escuro, espécie de giesta.
\section{Sangrar}
\begin{itemize}
\item {Grp. gram.:v. t.}
\end{itemize}
\begin{itemize}
\item {Grp. gram.:V. i.}
\end{itemize}
\begin{itemize}
\item {Utilização:Fig.}
\end{itemize}
\begin{itemize}
\item {Grp. gram.:V. p.}
\end{itemize}
\begin{itemize}
\item {Proveniência:(Do cast. \textunderscore sangre\textunderscore , sangue)}
\end{itemize}
Ferir ou picar para tirar sangue.
Tirar algum líquido a.
Extrahir.
Ferir.
Esvaziar: \textunderscore sangrar uma poça\textunderscore .
Privar.
Tirar a fôrça a.
Atormentar.
Deitar sangue.
Cair em gotas, gotejar.
\textunderscore Sangrar-se em saúde\textunderscore , precaver-se; desviar previamente a responsabilidade de um acto.
\section{Sangre}
\begin{itemize}
\item {Grp. gram.:m.}
\end{itemize}
\begin{itemize}
\item {Utilização:Ant.}
\end{itemize}
O mesmo que \textunderscore sangue\textunderscore .
(Cast. \textunderscore sangre\textunderscore )
\section{Sangrento}
\begin{itemize}
\item {Grp. gram.:adj.}
\end{itemize}
\begin{itemize}
\item {Proveniência:(Do cast. \textunderscore sangre\textunderscore )}
\end{itemize}
Que derrama sangue.
Coberto de sangue.
Que produz derramamento de sangue; cruento: \textunderscore combates sangrentos\textunderscore .
\section{Sangria}
\begin{itemize}
\item {Grp. gram.:f.}
\end{itemize}
\begin{itemize}
\item {Utilização:Fig.}
\end{itemize}
\begin{itemize}
\item {Utilização:Pop.}
\end{itemize}
Acto ou effeito de sangrar.
Sangue, extrahido ou derramado.
Sanja.
Extorsão ardilosa ou fraudulenta.
Bebida refrigerante, composta de vinho, água e açúcar.
(Cast. \textunderscore sangria\textunderscore )
\section{Sangrinheiro}
\begin{itemize}
\item {Grp. gram.:m.}
\end{itemize}
O mesmo que \textunderscore sangrinho\textunderscore .
\section{Sangrinho}
\begin{itemize}
\item {Grp. gram.:m.}
\end{itemize}
\begin{itemize}
\item {Utilização:Prov.}
\end{itemize}
\begin{itemize}
\item {Utilização:trasm.}
\end{itemize}
Árvore, o mesmo que \textunderscore sanguinho\textunderscore . Cf. Dom. Vieira, \textunderscore Diccion.\textunderscore , vb. \textunderscore sanguinho\textunderscore ^3.
\section{Sangue}
\begin{itemize}
\item {Grp. gram.:m.}
\end{itemize}
\begin{itemize}
\item {Utilização:Fig.}
\end{itemize}
\begin{itemize}
\item {Utilização:Fig.}
\end{itemize}
\begin{itemize}
\item {Proveniência:(Lat. \textunderscore sanguis\textunderscore )}
\end{itemize}
Líquido espêsso, geralmente vermelho, que enche as veias e os vasos arteriaes.
A vida: \textunderscore dar o sangue por alguém\textunderscore .
Prole; família.
Geração: \textunderscore sangue nobre\textunderscore .
Sumo.
Mênstruo: \textunderscore a rapariga está com o sangue\textunderscore .
Natureza, opposição á graça, em theologia.
O mesmo que \textunderscore pau-sangue\textunderscore .
\textunderscore Sangue azul\textunderscore , o mesmo que \textunderscore fidalguia\textunderscore , nobreza de sangue.
\section{Sanguechuva}
\begin{itemize}
\item {Grp. gram.:f.}
\end{itemize}
\begin{itemize}
\item {Proveniência:(De \textunderscore sangue\textunderscore  + \textunderscore chuva\textunderscore )}
\end{itemize}
Fluxo de sangue.
\section{Sangue-de-boi}
\begin{itemize}
\item {Grp. gram.:m.}
\end{itemize}
Passarinho, (\textunderscore ramphocelus brasilens\textunderscore ), de corpo vermelho e asas pretas.
\section{Sangue-de-cristo}
\begin{itemize}
\item {Grp. gram.:m.}
\end{itemize}
\begin{itemize}
\item {Utilização:Bras. do N}
\end{itemize}
Espécie de rôla.
\section{Sangue-de-drago}
\begin{itemize}
\item {Grp. gram.:m.}
\end{itemize}
Substância, extrahida do dragoeiro.
\section{Sangueira}
\begin{itemize}
\item {Grp. gram.:f.}
\end{itemize}
Grande porção de sangue derramado.
Sangue, que escorre das reses mortas.
\section{Sanguentado}
\begin{itemize}
\item {Grp. gram.:adj.}
\end{itemize}
O mesmo que [[ensanguentado|ensanguentar]].
Coberto de sangue:«\textunderscore ...os braços hirtos e sanguentados\textunderscore ». Camillo, \textunderscore Suicida\textunderscore , 31.
\section{Sanguento}
\begin{itemize}
\item {Grp. gram.:adj.}
\end{itemize}
Sangrento; sanguinolento.
\section{Sanguessuga}
\begin{itemize}
\item {Grp. gram.:f.}
\end{itemize}
\begin{itemize}
\item {Utilização:pop.}
\end{itemize}
\begin{itemize}
\item {Utilização:Fig.}
\end{itemize}
\begin{itemize}
\item {Proveniência:(Lat. \textunderscore sanguisuga\textunderscore )}
\end{itemize}
Animal aquático, anélido, que suga o sangue e que se emprega em Medicina, quando se quer fazer a sangria capilar.
Beberrão.
Aquele que ardilosamente tira dinheiro a outrem.
Aquele que explora outrem.
\section{Sanguesuga}
\begin{itemize}
\item {fónica:su}
\end{itemize}
\begin{itemize}
\item {Grp. gram.:f.}
\end{itemize}
\begin{itemize}
\item {Utilização:pop.}
\end{itemize}
\begin{itemize}
\item {Utilização:Fig.}
\end{itemize}
\begin{itemize}
\item {Proveniência:(Lat. \textunderscore sanguisuga\textunderscore )}
\end{itemize}
Animal aquático, anélido, que suga o sangue e que se emprega em Medicina, quando se quere fazer a sangria capilar.
Beberrão.
Aquelle que ardilosamente tira dinheiro a outrem.
Aquelle que explora outrem.
\section{Sanguexuga}
\begin{itemize}
\item {Grp. gram.:f.}
\end{itemize}
\begin{itemize}
\item {Utilização:ant.}
\end{itemize}
\begin{itemize}
\item {Utilização:Pop.}
\end{itemize}
O mesmo que \textunderscore sanguesuga\textunderscore . Cf. \textunderscore Peregrinação\textunderscore , LXXXII.
\section{Sanguicel}
\begin{itemize}
\item {fónica:gu-i}
\end{itemize}
\begin{itemize}
\item {Grp. gram.:m.}
\end{itemize}
Pequena embarcação asiática.
\section{Sanguidário}
\begin{itemize}
\item {Grp. gram.:m.}
\end{itemize}
\begin{itemize}
\item {Utilização:Prov.}
\end{itemize}
\begin{itemize}
\item {Utilização:minh.}
\end{itemize}
Espécie de avental.
\section{Sanguífero}
\begin{itemize}
\item {Grp. gram.:adj.}
\end{itemize}
\begin{itemize}
\item {Utilização:Poét.}
\end{itemize}
\begin{itemize}
\item {Proveniência:(Do lat. \textunderscore sanguis\textunderscore  + \textunderscore ferre\textunderscore )}
\end{itemize}
Que tem ou produz sangue.
\section{Sanguificação}
\begin{itemize}
\item {Grp. gram.:f.}
\end{itemize}
Acto ou effeito de sanguificar.
Formação do sangue.
Conversão do sangue venoso em arterial.
\section{Sanguificar}
\begin{itemize}
\item {Grp. gram.:v. t.}
\end{itemize}
\begin{itemize}
\item {Proveniência:(Do lat. \textunderscore sanguis\textunderscore  + \textunderscore facere\textunderscore )}
\end{itemize}
Converter em sangue.
\section{Sanguificativo}
\begin{itemize}
\item {Grp. gram.:adj.}
\end{itemize}
Que sanguifica.
\section{Sanguífico}
\begin{itemize}
\item {Grp. gram.:adj.}
\end{itemize}
Que sanguifica.
\section{Sanguileixado}
\begin{itemize}
\item {Grp. gram.:adj.}
\end{itemize}
\begin{itemize}
\item {Utilização:Ant.}
\end{itemize}
\begin{itemize}
\item {Proveniência:(De \textunderscore sangue\textunderscore  + \textunderscore leixar\textunderscore )}
\end{itemize}
O mesmo que [[sangrado|sangrar]].
\section{Sanguileixador}
\begin{itemize}
\item {Grp. gram.:m.}
\end{itemize}
\begin{itemize}
\item {Utilização:Ant.}
\end{itemize}
O mesmo que \textunderscore sangrador\textunderscore .
\section{Sanguileixia}
\begin{itemize}
\item {Grp. gram.:f.}
\end{itemize}
\begin{itemize}
\item {Utilização:Ant.}
\end{itemize}
\begin{itemize}
\item {Proveniência:(De \textunderscore sangue\textunderscore  + \textunderscore leixar\textunderscore )}
\end{itemize}
O mesmo que \textunderscore sangria\textunderscore .
\section{Sanguim}
\begin{itemize}
\item {Grp. gram.:m.}
\end{itemize}
Outra fórma de \textunderscore sagui\textunderscore .
\section{Sanguimisto}
\begin{itemize}
\item {Grp. gram.:m.}
\end{itemize}
\begin{itemize}
\item {Utilização:Ant.}
\end{itemize}
Homem lascivo; estuprador.
\section{Sanguimixto}
\begin{itemize}
\item {Grp. gram.:m.}
\end{itemize}
\begin{itemize}
\item {Utilização:Ant.}
\end{itemize}
Homem lascivo; estuprador.
\section{Sanguina}
\begin{itemize}
\item {Grp. gram.:f.}
\end{itemize}
\begin{itemize}
\item {Proveniência:(De \textunderscore sangue\textunderscore )}
\end{itemize}
Peróxydo de ferro, empregado no fabríco de lápis encarnados e em polir certos metaes.
\section{Sanguinação}
\begin{itemize}
\item {Grp. gram.:f.}
\end{itemize}
\begin{itemize}
\item {Proveniência:(Lat. \textunderscore sanguinatio\textunderscore )}
\end{itemize}
Formação do sangue.
Erupção sanguínea.
\section{Sanguinária}
\begin{itemize}
\item {Grp. gram.:f.}
\end{itemize}
Planta polygónea, também chamada \textunderscore corriola-bastarda\textunderscore .
\section{Sanguinariamente}
\begin{itemize}
\item {Grp. gram.:adv.}
\end{itemize}
De modo sanguinário; ferozmente.
\section{Sanguinarina}
\begin{itemize}
\item {Grp. gram.:f.}
\end{itemize}
\begin{itemize}
\item {Proveniência:(De \textunderscore sanguinária\textunderscore )}
\end{itemize}
Medicamento tónico, estimulante do systema nervoso.
\section{Sanguinário}
\begin{itemize}
\item {Grp. gram.:adj.}
\end{itemize}
\begin{itemize}
\item {Utilização:Ext.}
\end{itemize}
\begin{itemize}
\item {Proveniência:(Lat. \textunderscore sanguinarius\textunderscore )}
\end{itemize}
Que se compraz em derramar sangue.
Feroz.
Que serve para fazer derramar sangue.
\section{Sanguínea}
\begin{itemize}
\item {Grp. gram.:f.}
\end{itemize}
Planta, o mesmo que \textunderscore sanguinária\textunderscore .
Variedade de pêra.
O mesmo que \textunderscore sanguina\textunderscore .
Variedade de maçan. Cf. \textunderscore Port. au point de vue agr.\textunderscore , 149.
\section{Sanguíneo}
\begin{itemize}
\item {Grp. gram.:adj.}
\end{itemize}
\begin{itemize}
\item {Grp. gram.:M.}
\end{itemize}
\begin{itemize}
\item {Proveniência:(Lat. \textunderscore sanguineus\textunderscore )}
\end{itemize}
Relativo ao sangue.
Que tem côr de sangue.
Em cujo temperamento predomina o sangue.
Sanguinolento; sanguinário.
Indivíduo, em cujo temperamento predomina o sangue.
\section{Sanguinha}
\begin{itemize}
\item {Grp. gram.:f.}
\end{itemize}
\begin{itemize}
\item {Utilização:Prov.}
\end{itemize}
\begin{itemize}
\item {Utilização:minh.}
\end{itemize}
(V.sanguinária)
Variedade de chouriço, feito de carne, gordura e sangue de porco.
\section{Sanguinhal}
\begin{itemize}
\item {Grp. gram.:m.}
\end{itemize}
Mata de sanguinhos.
\section{Sanguinheiro}
\begin{itemize}
\item {Grp. gram.:m.}
\end{itemize}
\begin{itemize}
\item {Proveniência:(De \textunderscore sanguinho\textunderscore )}
\end{itemize}
Plânta rhamnácea, (\textunderscore rhamnus frangula\textunderscore ).
\section{Sanguinho}
\begin{itemize}
\item {Grp. gram.:m.}
\end{itemize}
\begin{itemize}
\item {Grp. gram.:Adj.}
\end{itemize}
\begin{itemize}
\item {Utilização:Des.}
\end{itemize}
\begin{itemize}
\item {Proveniência:(Do lat. \textunderscore sanguineus\textunderscore )}
\end{itemize}
Pequeno pano, com que o sacerdote enxuga o cálix, depois de consumir o vinho consagrado.
Sandim.
Planta caprifoliácea, (\textunderscore cornus sanguinea\textunderscore ).
Árvore, de madeira amarelada e sabor amargo, (\textunderscore rhamnus latifolius\textunderscore ).
\textunderscore Sanguinho das sebes\textunderscore , o mesmo que \textunderscore aderno\textunderscore .
Sanguíneo; encarnado.
\section{Sanguinidade}
\begin{itemize}
\item {Grp. gram.:f.}
\end{itemize}
(V.consanguinidade)
\section{Sanguino}
\begin{itemize}
\item {Grp. gram.:adj.}
\end{itemize}
\begin{itemize}
\item {Grp. gram.:M.}
\end{itemize}
\begin{itemize}
\item {Utilização:Phot.}
\end{itemize}
Sanguíneo; que produz morte ou derramamento de sangue.
Côr avermelhada.
\textunderscore Viragens em sanguino\textunderscore , provas tiradas em vermelho carregado, como um desenho a sanguino.
\section{Sanguinolária}
\begin{itemize}
\item {Grp. gram.:f.}
\end{itemize}
Gênero de molluscos.
\section{Sanguinolência}
\begin{itemize}
\item {Grp. gram.:f.}
\end{itemize}
\begin{itemize}
\item {Proveniência:(Lat. \textunderscore sanguinolentia\textunderscore )}
\end{itemize}
Qualidade do que é sanguinolento.
\section{Sanguinolentamente}
\begin{itemize}
\item {Grp. gram.:adv.}
\end{itemize}
De modo sanguinolento; cruelmente, ferozmente.
\section{Sanguinolento}
\begin{itemize}
\item {Grp. gram.:adj.}
\end{itemize}
\begin{itemize}
\item {Proveniência:(Lat. \textunderscore sanguinolentus\textunderscore )}
\end{itemize}
Coberto de sangue.
Misturado ou tinto de sangue.
Sanguinário.
\section{Sanguinoso}
\begin{itemize}
\item {Grp. gram.:adj.}
\end{itemize}
\begin{itemize}
\item {Proveniência:(Lat. \textunderscore sanguinosus\textunderscore )}
\end{itemize}
O mesmo que \textunderscore sanguinolento\textunderscore .
\section{Sanguisedento}
\begin{itemize}
\item {fónica:se}
\end{itemize}
\begin{itemize}
\item {Grp. gram.:adj.}
\end{itemize}
\begin{itemize}
\item {Utilização:Poét.}
\end{itemize}
\begin{itemize}
\item {Proveniência:(De \textunderscore sangue\textunderscore  + \textunderscore sedento\textunderscore )}
\end{itemize}
Que tem sêde de sangue; que é sanguinário.
\section{Sanguisorba}
\begin{itemize}
\item {fónica:sor}
\end{itemize}
\begin{itemize}
\item {Grp. gram.:f.}
\end{itemize}
\begin{itemize}
\item {Proveniência:(Do lat. \textunderscore sanguis\textunderscore  + \textunderscore sorbere\textunderscore )}
\end{itemize}
Planta rosácea, o mesmo que \textunderscore pimpinela\textunderscore .
\section{Sanguisorbeáceas}
\begin{itemize}
\item {fónica:sor}
\end{itemize}
\begin{itemize}
\item {Grp. gram.:f. pl.}
\end{itemize}
Família de plantas, separada das rosáceas, e que tem por typo a sanguisorba.
\section{Sanguissedento}
\begin{itemize}
\item {Grp. gram.:adj.}
\end{itemize}
\begin{itemize}
\item {Utilização:Poét.}
\end{itemize}
\begin{itemize}
\item {Proveniência:(De \textunderscore sangue\textunderscore  + \textunderscore sedento\textunderscore )}
\end{itemize}
Que tem sêde de sangue; que é sanguinário.
\section{Sanguissorba}
\begin{itemize}
\item {Grp. gram.:f.}
\end{itemize}
\begin{itemize}
\item {Proveniência:(Do lat. \textunderscore sanguis\textunderscore  + \textunderscore sorbere\textunderscore )}
\end{itemize}
Planta rosácea, o mesmo que \textunderscore pimpinela\textunderscore .
\section{Sanguissorbeáceas}
\begin{itemize}
\item {Grp. gram.:f. pl.}
\end{itemize}
Família de plantas, separada das rosáceas, e que tem por typo a sanguissorba.
\section{Sangurinheiro}
\begin{itemize}
\item {Grp. gram.:m.}
\end{itemize}
O mesmo que \textunderscore sanguinheiro\textunderscore .
\section{Sanha}
\begin{itemize}
\item {Grp. gram.:f.}
\end{itemize}
\begin{itemize}
\item {Proveniência:(Do lat. \textunderscore insania\textunderscore , seg. Cornu)}
\end{itemize}
Rancor; fúria; ira.
\section{Sanha}
\begin{itemize}
\item {Grp. gram.:f.}
\end{itemize}
Madeira de Cabinda, muito applicada em construcções.
\section{Sanhaço}
\begin{itemize}
\item {Grp. gram.:m.}
\end{itemize}
\begin{itemize}
\item {Utilização:Bras}
\end{itemize}
Passarinho de peito azul.
\section{Sanhaçu}
\begin{itemize}
\item {Grp. gram.:m.}
\end{itemize}
\begin{itemize}
\item {Utilização:Bras}
\end{itemize}
Passarinho de peito azul.
\section{Sanharó}
\begin{itemize}
\item {Grp. gram.:m.}
\end{itemize}
\begin{itemize}
\item {Utilização:Bras}
\end{itemize}
Espécie de abelha preta.
\section{Sanhedrim}
\begin{itemize}
\item {fónica:ne}
\end{itemize}
\begin{itemize}
\item {Grp. gram.:m.}
\end{itemize}
O mesmo que \textunderscore sanhédrio\textunderscore .
\section{Sanhédrio}
\begin{itemize}
\item {fónica:né}
\end{itemize}
\begin{itemize}
\item {Grp. gram.:m.}
\end{itemize}
(V.synhedrim)
\section{Sanheiro}
\begin{itemize}
\item {Grp. gram.:m.}
\end{itemize}
Nome, que se dá ao marnoto, nas margens do Guadiana. Cf. \textunderscore Museu Techn.\textunderscore , 108.
(Contr. de \textunderscore salinheiro\textunderscore , por \textunderscore salineiro\textunderscore )
\section{Sanhoaneiro}
\begin{itemize}
\item {Grp. gram.:m.}
\end{itemize}
\begin{itemize}
\item {Utilização:Ant.}
\end{itemize}
O mesmo que \textunderscore sanjoaneiro\textunderscore .
\section{Sanhoso}
\begin{itemize}
\item {Grp. gram.:adj.}
\end{itemize}
Que tem sanha; irascível.
\section{Sanhudamente}
\begin{itemize}
\item {Grp. gram.:adv.}
\end{itemize}
De modo sanhudo; com sanha; iradamente.
\section{Sanhudo}
\begin{itemize}
\item {Grp. gram.:adj.}
\end{itemize}
\begin{itemize}
\item {Utilização:Fig.}
\end{itemize}
\begin{itemize}
\item {Proveniência:(De \textunderscore sanha\textunderscore )}
\end{itemize}
O mesmo que \textunderscore sanhoso\textunderscore .
Terrível; que põe medo.
\section{Sanicar}
\begin{itemize}
\item {Grp. gram.:v. t.}
\end{itemize}
\begin{itemize}
\item {Utilização:Prov.}
\end{itemize}
\begin{itemize}
\item {Utilização:minh.}
\end{itemize}
\begin{itemize}
\item {Grp. gram.:V. i.}
\end{itemize}
\begin{itemize}
\item {Utilização:Prov.}
\end{itemize}
\begin{itemize}
\item {Utilização:beir.}
\end{itemize}
Mexer, agitar, sacudir.
Tactear alguma coisa, procurando abertura que se não acha.
Mover-se com hesitação deante de outrem, embaraçando-lhe ou demorando-lhe a passagem.
(Por \textunderscore acenicar\textunderscore , de \textunderscore aceno\textunderscore ?)
\section{Sanícula}
\begin{itemize}
\item {Grp. gram.:f.}
\end{itemize}
Planta umbellífera.
\textunderscore Sanícula dos montes\textunderscore , saxífraga branca, (\textunderscore saxífraga granulata\textunderscore , Lin.).
(Dem. do lat. \textunderscore sana\textunderscore )
\section{Sanidade}
\begin{itemize}
\item {Grp. gram.:f.}
\end{itemize}
\begin{itemize}
\item {Proveniência:(Lat. \textunderscore sanitas\textunderscore )}
\end{itemize}
Qualidade do que é são; hygiene; salubridade.
\section{Sanidina}
\begin{itemize}
\item {Grp. gram.:f.}
\end{itemize}
\begin{itemize}
\item {Utilização:Miner.}
\end{itemize}
Espécie de orthose, de contextura vitrosa, nos terrenos eruptivos modernos.
\section{Sânie}
\begin{itemize}
\item {Grp. gram.:f.}
\end{itemize}
\begin{itemize}
\item {Proveniência:(Lat. \textunderscore sanies\textunderscore )}
\end{itemize}
Pus ou matéria purulenta, que as úlceras produzem; podridão.
\section{Sanificação}
\begin{itemize}
\item {Grp. gram.:f.}
\end{itemize}
Acto ou effeito de sanificar.
\section{Sanificante}
\begin{itemize}
\item {Grp. gram.:adj.}
\end{itemize}
Que sanifica.
\section{Sanificar}
\begin{itemize}
\item {Grp. gram.:v. t.}
\end{itemize}
\begin{itemize}
\item {Proveniência:(Do lat. \textunderscore sanus\textunderscore  + \textunderscore facere\textunderscore )}
\end{itemize}
Tornar são ou salubre; desinfectar.
\section{Sanioso}
\begin{itemize}
\item {Grp. gram.:adj.}
\end{itemize}
\begin{itemize}
\item {Proveniência:(Lat. \textunderscore saniosus\textunderscore )}
\end{itemize}
Em que há sânie.
\section{Sanisca}
\begin{itemize}
\item {Grp. gram.:f.}
\end{itemize}
\begin{itemize}
\item {Utilização:Prov.}
\end{itemize}
\begin{itemize}
\item {Utilização:trasm.}
\end{itemize}
Fragmento; estilhaço.
\section{Sanitário}
\begin{itemize}
\item {Grp. gram.:adj.}
\end{itemize}
\begin{itemize}
\item {Proveniência:(Do lat. \textunderscore sanitas\textunderscore )}
\end{itemize}
Relativo á saúde; relativo á hygiene: \textunderscore policia sanitária\textunderscore .
\section{Sanitarista}
\begin{itemize}
\item {Grp. gram.:m.}
\end{itemize}
\begin{itemize}
\item {Utilização:Neol.}
\end{itemize}
Aquelle que é perito em assumptos sanitários. Cf. R. Jorge, \textunderscore Boletím dos Serviços Sanitários\textunderscore , 7.
\section{Sanja}
\begin{itemize}
\item {Grp. gram.:f.}
\end{itemize}
Abertura, feita para escoamento de água.
Sargeta, valeta.
Rêgo entre os bacellos. Cp. \textunderscore sanga\textunderscore ^2.
(Cast. \textunderscore zanja\textunderscore )
\section{Sanjaco}
\begin{itemize}
\item {Grp. gram.:m.}
\end{itemize}
O mesmo que \textunderscore sangeaco\textunderscore .
\section{Sanjaque}
\begin{itemize}
\item {Grp. gram.:m.}
\end{itemize}
O mesmo ou melhor que \textunderscore sangeaco\textunderscore .
\section{Sanjar}
\begin{itemize}
\item {Grp. gram.:v. t.  e  i.}
\end{itemize}
Fazer ou abrir sanjas em.
\section{Sanjiaco}
\begin{itemize}
\item {Grp. gram.:m.}
\end{itemize}
\begin{itemize}
\item {Utilização:Ant.}
\end{itemize}
Governador de território, no Egypto e em terra de Turcos:«\textunderscore nesta batalha morreo o Baxá dos Turcos, e elegerão outro, que era um Sanjiaco, chamado...\textunderscore »Couto, \textunderscore Dec.\textunderscore , VII, c. 10.--Moraes e Silva, \textunderscore Diccion.\textunderscore , escreveu \textunderscore sangíaco\textunderscore , pronúncia que tenho por inexacta.
(Cp. \textunderscore sanjaco\textunderscore . Do turco \textunderscore sangak\textunderscore )
\section{Sanjoaneira}
\begin{itemize}
\item {Grp. gram.:f.}
\end{itemize}
\begin{itemize}
\item {Proveniência:(De \textunderscore San Joane\textunderscore , fórma ant. de \textunderscore San-João\textunderscore )}
\end{itemize}
Tributo, que se pagava na época do San-João.
Espécie de pêra.
Mulhér, que toma parte em descantes, próprios das festas de San-João.
\section{Sanjoaneiro}
\begin{itemize}
\item {Grp. gram.:adj.}
\end{itemize}
\begin{itemize}
\item {Utilização:Ant.}
\end{itemize}
\begin{itemize}
\item {Utilização:Prov.}
\end{itemize}
\begin{itemize}
\item {Utilização:Prov.}
\end{itemize}
\begin{itemize}
\item {Utilização:dur.}
\end{itemize}
\begin{itemize}
\item {Grp. gram.:Adj.}
\end{itemize}
Cobrador da renda ou tributo chamado \textunderscore sanjoaneira\textunderscore .
Cantador das festas populares de San-João.
Indivíduo, natural de São-João-da-Foz.
Que se colhe na época de San-João, (falando-se de frutos).
\section{Sanjoão}
\begin{itemize}
\item {Grp. gram.:f.}
\end{itemize}
Variedade de pêra temporan, provavelmente o mesmo que \textunderscore sanjoaneira\textunderscore .
Dança e música populares do Norte. Cf. Alb. Pimentel, \textunderscore Alegres Canções\textunderscore , 73.
\section{Sanluqueno}
\begin{itemize}
\item {Grp. gram.:adj.}
\end{itemize}
\begin{itemize}
\item {Grp. gram.:M.}
\end{itemize}
Relativo a Sanlúcar.
Habitante de Sanlúcar, em Espanha.
\section{Sanmente}
\begin{itemize}
\item {Grp. gram.:adv.}
\end{itemize}
De modo são; com saúde.
De acôrdo com os preceitos hygiênicos.
\section{Sanoca}
\begin{itemize}
\item {Grp. gram.:f.}
\end{itemize}
\begin{itemize}
\item {Utilização:Prov.}
\end{itemize}
\begin{itemize}
\item {Utilização:trasm.}
\end{itemize}
Bolo de sêmeas.
\section{Sanoco}
\begin{itemize}
\item {fónica:nô}
\end{itemize}
\begin{itemize}
\item {Grp. gram.:m.}
\end{itemize}
\begin{itemize}
\item {Utilização:Prov.}
\end{itemize}
\begin{itemize}
\item {Utilização:trasm.}
\end{itemize}
Pedaço: \textunderscore um sanoco de pão\textunderscore .
\section{Sanofórmio}
\begin{itemize}
\item {Grp. gram.:m.}
\end{itemize}
\begin{itemize}
\item {Utilização:Pharm.}
\end{itemize}
Medicamento, que é um dos succedâneos do iodofórmio.
\section{Sanona}
\begin{itemize}
\item {Grp. gram.:f.}
\end{itemize}
Árvore africana, da fam. das cucurbitáceas, de frutos pequenos, amarelos e espinhosos.
\section{Sànona}
\begin{itemize}
\item {Grp. gram.:m.}
\end{itemize}
\begin{itemize}
\item {Utilização:Fam.}
\end{itemize}
Pateta.
\section{San-pedro-branco}
\begin{itemize}
\item {Grp. gram.:m.}
\end{itemize}
\begin{itemize}
\item {Utilização:Bras}
\end{itemize}
Espécie de mandioca, de talo branco e grandes raízes.
\section{San-pedro-molle}
\begin{itemize}
\item {Grp. gram.:m.}
\end{itemize}
\begin{itemize}
\item {Utilização:Bras}
\end{itemize}
Espécie de mandioca, de talo muito sucoso.
\section{San-pedro-pequeno}
\begin{itemize}
\item {Grp. gram.:m.}
\end{itemize}
\begin{itemize}
\item {Utilização:Bras}
\end{itemize}
Espécie de mandioca.
\section{San-pedro-vermelho}
\begin{itemize}
\item {Grp. gram.:m.}
\end{itemize}
\begin{itemize}
\item {Utilização:Bras}
\end{itemize}
Espécie de mandioca, de talo côr de rosa.
\section{Sanquitar}
\begin{itemize}
\item {Grp. gram.:v. t.}
\end{itemize}
Voltear com farinha dentro do alguidar ou bacia (a massa, de que se há de fazer brôa).
\section{Sansama}
\begin{itemize}
\item {Grp. gram.:f.}
\end{itemize}
Árvore do Congo.
\section{Sansão}
\begin{itemize}
\item {Grp. gram.:m.}
\end{itemize}
Espécie de guindaste, usado em certas construções.
\section{Sansão-barandão}
\begin{itemize}
\item {Grp. gram.:m.}
\end{itemize}
Planta trepadeira da Guiné, de qualidades purgativas.
\section{Sansão-brandão}
\begin{itemize}
\item {Grp. gram.:m.}
\end{itemize}
Planta trepadeira da Guiné, de qualidades purgativas.
\section{Sansardoninho}
\begin{itemize}
\item {Grp. gram.:m.  e  adj.}
\end{itemize}
\begin{itemize}
\item {Utilização:Pop.}
\end{itemize}
\begin{itemize}
\item {Proveniência:(De \textunderscore San-Saturnino\textunderscore ? Ou do rad. de \textunderscore sonso\textunderscore ?)}
\end{itemize}
Indivíduo sonso, dissimulado, velhaco.
\section{Sanscrítico}
\begin{itemize}
\item {Grp. gram.:adj.}
\end{itemize}
Relativo ao sânscrito.
\section{Sanscritismo}
\begin{itemize}
\item {Grp. gram.:m.}
\end{itemize}
Estudo do sânscrito.
Doutrinas, derivadas dêsse estudo.
\section{Sanscritista}
\begin{itemize}
\item {Grp. gram.:m.  e  f.}
\end{itemize}
\begin{itemize}
\item {Proveniência:(De \textunderscore sânscrito\textunderscore )}
\end{itemize}
Pessôa, que é versada no sânscrito ou em coisas da Índia antiga; indianista.
\section{Sânscrito}
\begin{itemize}
\item {Grp. gram.:m.}
\end{itemize}
\begin{itemize}
\item {Grp. gram.:Adj.}
\end{itemize}
Antiga língua dos Brâhmanes.
Língua sagrada do Indostão.
O mesmo que \textunderscore sanscrítico\textunderscore .
(Sanscr. \textunderscore sanskrita\textunderscore )
\section{Sanscritóide}
\begin{itemize}
\item {Grp. gram.:adj.}
\end{itemize}
\begin{itemize}
\item {Proveniência:(De \textunderscore sânscrito\textunderscore  + gr. \textunderscore eidos\textunderscore )}
\end{itemize}
Diz-se, talvez com pouca propriedade, das línguas procedentes do sânscrito.
\section{Sanseviera}
\begin{itemize}
\item {Grp. gram.:f.}
\end{itemize}
\begin{itemize}
\item {Proveniência:(De \textunderscore Sansever\textunderscore , n. p.)}
\end{itemize}
Gênero de plantas liliáceas, (\textunderscore sansevieria guineensis\textunderscore , Wild.).
\section{Sanseviéria}
\begin{itemize}
\item {Grp. gram.:f.}
\end{itemize}
\begin{itemize}
\item {Proveniência:(De \textunderscore Sansever\textunderscore , n. p.)}
\end{itemize}
Gênero de plantas liliáceas, (\textunderscore sansevieria guineensis\textunderscore , Wild.).
\section{Sansimonismo}
\begin{itemize}
\item {Grp. gram.:m.}
\end{itemize}
Systema de philosophia social, preconizado por Saint-Simon.
\section{Sansimonista}
\begin{itemize}
\item {Grp. gram.:m.}
\end{itemize}
Sectário do sansimonismo.
\section{Santa}
\begin{itemize}
\item {Grp. gram.:f.}
\end{itemize}
\begin{itemize}
\item {Utilização:Bras}
\end{itemize}
(Fem. de \textunderscore santo\textunderscore )
Peixe, espécie de arraia.
\section{Santa-ana}
\begin{itemize}
\item {Grp. gram.:f.}
\end{itemize}
O mesmo que \textunderscore santa-batuta\textunderscore .
\section{Santa-bárbara}
\begin{itemize}
\item {Grp. gram.:f.}
\end{itemize}
\begin{itemize}
\item {Utilização:Náut.}
\end{itemize}
\begin{itemize}
\item {Proveniência:(Do fr. \textunderscore sainte-barbe\textunderscore )}
\end{itemize}
Câmara, em que se guarda a pólvora a bordo.
\section{Santa-batuta}
\begin{itemize}
\item {Grp. gram.:f.}
\end{itemize}
Espécie de jôgo popular.
\section{Santa-clara}
\begin{itemize}
\item {Grp. gram.:f.}
\end{itemize}
Planta leguminosa de Cabo-Verde, (\textunderscore abrus precatorius\textunderscore , Lin.).
\section{Santa-cruz}
\begin{itemize}
\item {Grp. gram.:f.}
\end{itemize}
Ave, das regiões do Amazonas, semelhante á pomba.
\section{Santa-fé}
\begin{itemize}
\item {Grp. gram.:f.}
\end{itemize}
\begin{itemize}
\item {Utilização:Bras}
\end{itemize}
Planta gramínea que, depois de sêca, serve para cobertura de casas rústicas.
\section{Santal}
\begin{itemize}
\item {Grp. gram.:m.}
\end{itemize}
\begin{itemize}
\item {Utilização:Ant.}
\end{itemize}
O mesmo que \textunderscore santoral\textunderscore .
\section{Santaláceas}
\begin{itemize}
\item {Grp. gram.:f. pl.}
\end{itemize}
Família de plantas, que tem por typo o sândalo.
(Fem. pl. de \textunderscore santaláceo\textunderscore )
\section{Santaláceo}
\begin{itemize}
\item {Grp. gram.:adj.}
\end{itemize}
\begin{itemize}
\item {Proveniência:(De \textunderscore santalum\textunderscore , nome scientífico do sândalo)}
\end{itemize}
Relativo ou semelhante ao sândalo.
\section{Santalhão}
\begin{itemize}
\item {Grp. gram.:m.}
\end{itemize}
O mesmo que \textunderscore santarrão\textunderscore . Cf. Júl. Dinis, \textunderscore Morgadinha\textunderscore , 86.
\section{Santalina}
\begin{itemize}
\item {Grp. gram.:f.}
\end{itemize}
\begin{itemize}
\item {Proveniência:(De \textunderscore santalum\textunderscore , nome scientífico do sândalo)}
\end{itemize}
Substância còrante do sândalo.
\section{Santa-luzia}
\begin{itemize}
\item {Grp. gram.:f.}
\end{itemize}
\begin{itemize}
\item {Utilização:Bras}
\end{itemize}
\begin{itemize}
\item {Utilização:Fam.}
\end{itemize}
Árvore euphorbiácea, de seiva leitosa e medicinal.
O mesmo que \textunderscore palmatória\textunderscore .
\section{Santa-maria}
\begin{itemize}
\item {Grp. gram.:f.}
\end{itemize}
Designação vulgar de várias plantas herbáceas.
\section{Santa-maria-de-alcântara}
\begin{itemize}
\item {Grp. gram.:f.}
\end{itemize}
Casta de uva espanhola.
\section{Santa-marta}
\begin{itemize}
\item {Grp. gram.:adj.}
\end{itemize}
Diz-se de uma variedade de trigo rijo.
\section{Santamente}
\begin{itemize}
\item {Grp. gram.:adv.}
\end{itemize}
De modo santo; virtuosamente; como os santos.
\section{Santa-morena}
\begin{itemize}
\item {Grp. gram.:f.}
\end{itemize}
Variedade de uva brasileira.
\section{Santanário}
\begin{itemize}
\item {Grp. gram.:m.  e  adj.}
\end{itemize}
O mesmo que \textunderscore santarrão\textunderscore .
\section{Santança}
\begin{itemize}
\item {Grp. gram.:f.}
\end{itemize}
\begin{itemize}
\item {Utilização:Prov.}
\end{itemize}
\begin{itemize}
\item {Utilização:beir.}
\end{itemize}
O mesmo que \textunderscore sentença\textunderscore .
(Cp. b. lat. \textunderscore sanctantia\textunderscore )
\section{Santana}
\begin{itemize}
\item {Grp. gram.:m.}
\end{itemize}
Variedade de pêssegos grandes, de pelle amarelo-rosada e polpa sucosa.
Variedade de pereira.
\section{Santantoninhas}
\begin{itemize}
\item {Grp. gram.:f. pl.}
\end{itemize}
\begin{itemize}
\item {Utilização:Bot.}
\end{itemize}
Planta, o mesmo que \textunderscore alfena\textunderscore . Cf. P. Coutinho, \textunderscore Flora\textunderscore , 480.
\section{Santantoninho}
\begin{itemize}
\item {Grp. gram.:m.}
\end{itemize}
\begin{itemize}
\item {Utilização:Fam.}
\end{itemize}
Pessôa, que se estima muito, que se amima, que se apaparica. Cf. Castilho, \textunderscore Fastos\textunderscore , III, 557.
\section{Santão}
\begin{itemize}
\item {Grp. gram.:m.  e  adj.}
\end{itemize}
O mesmo que \textunderscore santarrão\textunderscore .
\section{Santaomé}
\begin{itemize}
\item {Grp. gram.:m.}
\end{itemize}
\begin{itemize}
\item {Utilização:Ant.}
\end{itemize}
Pano, que se fabricava em Santomér, donde tirava o nome.
\section{Santarém}
\begin{itemize}
\item {Grp. gram.:m.}
\end{itemize}
\begin{itemize}
\item {Proveniência:(De \textunderscore Santarém\textunderscore , n. p.)}
\end{itemize}
Espécie de uva trincadeira, oval.
\section{Santareno}
\begin{itemize}
\item {Grp. gram.:adj.}
\end{itemize}
\begin{itemize}
\item {Grp. gram.:M.}
\end{itemize}
Relativo a Santarém.
Habitante de Santarém. Cf. B. Pato, \textunderscore Ciprestes\textunderscore , 250.
\section{Santaria}
\begin{itemize}
\item {Grp. gram.:f.}
\end{itemize}
\begin{itemize}
\item {Utilização:Deprec.}
\end{itemize}
Porção de santos. Cf. Eça, \textunderscore P. Amaro\textunderscore , 385.
\section{Santa-rita}
\begin{itemize}
\item {Grp. gram.:f.}
\end{itemize}
\begin{itemize}
\item {Utilização:T. do Fundão}
\end{itemize}
Golpe, gilvaz.
\section{Santarrão}
\begin{itemize}
\item {Grp. gram.:m.  e  adj.}
\end{itemize}
\begin{itemize}
\item {Proveniência:(De \textunderscore santo\textunderscore )}
\end{itemize}
Que finge santidade; hypócrita.
Beato falso.
Beato.
\section{Santeiro}
\begin{itemize}
\item {Grp. gram.:adj.}
\end{itemize}
\begin{itemize}
\item {Grp. gram.:M.}
\end{itemize}
Devoto.
Santificado:«\textunderscore dias santeiros\textunderscore ». R. Lobo, \textunderscore Pastor Peregr.\textunderscore 
Aquelle que faz ou vende imagens de santos.
\section{Santelmo}
\begin{itemize}
\item {Grp. gram.:m.}
\end{itemize}
Chama-se \textunderscore fogo de santelmo\textunderscore  á chama azulada que, principalmente em occasião de tempestade, apparece nos mastros dos navios, por effeito da electricidade.--Não se sabe ao certo donde veio o voc. para o português e para o castelhano. Os Espanhóis, entretanto, preferem-lhe \textunderscore helena\textunderscore ; e, entre nós, antigamente, também ao respectivo phenómeno se dava o nome de \textunderscore corpo santo\textunderscore . É certo todavia que os nossos clássicos quinhentistas já conheceram e usaram o voc., o que bastará para rejeitar o conceito de Madureira Feijó, (\textunderscore Orthogr.\textunderscore ), que viu estrangeirismo em \textunderscore Santhelmo\textunderscore  e aconselhara a sua substituição por \textunderscore corpo santo\textunderscore .
\section{Santelo}
\begin{itemize}
\item {fónica:tê}
\end{itemize}
\begin{itemize}
\item {Grp. gram.:m.}
\end{itemize}
Antiga rede de pescar peixe miúdo.
\section{Santhomé}
\begin{itemize}
\item {Grp. gram.:m.}
\end{itemize}
\begin{itemize}
\item {Utilização:Bras}
\end{itemize}
\begin{itemize}
\item {Utilização:Bras}
\end{itemize}
Moéda de oiro, cunhada por Garcia de Sá na Índia.
Árvore silvestre, que produz uma resina, parecida ao beijoim.
Variedade de bananeira, originária da ilha de San-Thomé.
\section{Santhomense}
\begin{itemize}
\item {Grp. gram.:adj.}
\end{itemize}
\begin{itemize}
\item {Grp. gram.:M.}
\end{itemize}
Relativo á ilha ou á província de San-Thomé.
Habitante de San-Thomé.
\section{Sântia}
\begin{itemize}
\item {Grp. gram.:f.}
\end{itemize}
Gênero de plantas rubiáceas da Índia.
\section{Santiago}
\begin{itemize}
\item {Grp. gram.:f.}
\end{itemize}
\begin{itemize}
\item {Grp. gram.:Interj.}
\end{itemize}
\begin{itemize}
\item {Grp. gram.:M.}
\end{itemize}
\begin{itemize}
\item {Utilização:Ant.}
\end{itemize}
Variedade de pêra excellente.
Lençaria fabricada em Santiago de Galliza.
Grito de guerra, com que os Castelhanos invocavam o seu patrono Santo Iago.
Voz de ataque aos Moiros:«\textunderscore não podendo aguardar mais, deram Santiago, de maneira...\textunderscore ». \textunderscore Jorn. de Áfr.\textunderscore , VI.
O mesmo que \textunderscore derrota\textunderscore ^2:«\textunderscore ...com toda a tropa sua, tal Sant-Iago deu aos Mouros...\textunderscore »Filinto, \textunderscore D. Man.\textunderscore , I, 138.
Ímpeto guerreiro. Cf. \textunderscore Idem\textunderscore , \textunderscore ibidem\textunderscore , I, 138.
\section{Santiagueiro}
\begin{itemize}
\item {Grp. gram.:adj.}
\end{itemize}
\begin{itemize}
\item {Grp. gram.:M.}
\end{itemize}
Relativo a Santiago de Compostela.
Habitante de Santiago de Compostela. Cf. Deusdado, \textunderscore Escorços\textunderscore , 64.
\section{Santiaguês}
\begin{itemize}
\item {Grp. gram.:m.  e  adj.}
\end{itemize}
O que é de Santiago de Compostela.
\section{Santiámen}
\begin{itemize}
\item {Grp. gram.:m.}
\end{itemize}
\begin{itemize}
\item {Utilização:Fam.}
\end{itemize}
\begin{itemize}
\item {Proveniência:(Do lat. \textunderscore sanctus\textunderscore  + \textunderscore amen\textunderscore )}
\end{itemize}
Momento, instante: \textunderscore desapparece num sentiámen\textunderscore .
\section{Santico}
\begin{itemize}
\item {Grp. gram.:m.}
\end{itemize}
\begin{itemize}
\item {Utilização:Pop.}
\end{itemize}
Pingente, que tem esmaltada a imagem de um santo.
\section{Santidade}
\begin{itemize}
\item {Grp. gram.:f.}
\end{itemize}
\begin{itemize}
\item {Proveniência:(Lat. \textunderscore sanctitas\textunderscore )}
\end{itemize}
Qualidade ou estado daquelle ou daquillo que é santo.
Título do Papa.
\section{Santificação}
\begin{itemize}
\item {Grp. gram.:f.}
\end{itemize}
\begin{itemize}
\item {Proveniência:(Do lat. \textunderscore sanctificatio\textunderscore )}
\end{itemize}
Acto ou effeito de santificar.
\section{Santificador}
\begin{itemize}
\item {Grp. gram.:m.  e  adj.}
\end{itemize}
\begin{itemize}
\item {Proveniência:(Do lat. \textunderscore sanctificator\textunderscore )}
\end{itemize}
O que santifica.
\section{Santificante}
\begin{itemize}
\item {Grp. gram.:adj.}
\end{itemize}
\begin{itemize}
\item {Proveniência:(Lat. \textunderscore sanctificans\textunderscore )}
\end{itemize}
Que santifica.
\section{Santificar}
\begin{itemize}
\item {Grp. gram.:v. t.}
\end{itemize}
\begin{itemize}
\item {Grp. gram.:V. p.}
\end{itemize}
\begin{itemize}
\item {Utilização:Ant.}
\end{itemize}
\begin{itemize}
\item {Proveniência:(Lat. \textunderscore sanctificare\textunderscore )}
\end{itemize}
Tornar santo.
Sagrar; canonizar.
Tornar venerado, nobilitar.
Educar religiosamente.
Conduzir pelo caminho da salvação eterna.
Moralizar.
O mesmo que [[benzer-se|benzer]]. Cf. \textunderscore Port. Mon. Hist.\textunderscore , \textunderscore Scrip.\textunderscore , 259.
\section{Santificável}
\begin{itemize}
\item {Grp. gram.:adj.}
\end{itemize}
Que se póde ou se deve santificar.
\section{Santigar}
\begin{itemize}
\item {Grp. gram.:v. t.}
\end{itemize}
\begin{itemize}
\item {Utilização:Des.}
\end{itemize}
\begin{itemize}
\item {Grp. gram.:V. p.}
\end{itemize}
Benzer.
Dizer orações sôbre (um enfermo), para o curar.
Persignar-se.
(Corr. de \textunderscore santificar\textunderscore )
\section{Santiguar}
\begin{itemize}
\item {Grp. gram.:v. t.  e  p.}
\end{itemize}
(V.santigar)
\section{San-gonçalo}
\begin{itemize}
\item {Grp. gram.:m.}
\end{itemize}
\begin{itemize}
\item {Utilização:Bras}
\end{itemize}
Espécie de baile, em que os festeiros cantam, dançam e se embriagam, de noite, ao ar livre, em frente da imagem de San-Gonçalo.
\section{San-martinho}
\begin{itemize}
\item {Grp. gram.:m.}
\end{itemize}
Variedade de pêra ordinária.
\section{San-miguel}
\begin{itemize}
\item {Grp. gram.:m.}
\end{itemize}
\begin{itemize}
\item {Utilização:Fig.}
\end{itemize}
Época das colheitas: \textunderscore prometeu pagar pelo San-Miguel\textunderscore .
Colheita.
Fortuna.
Variedade de pêra pardacenta, açucarada e um pouco granulosa.
\section{San-pedro-caá}
\begin{itemize}
\item {Grp. gram.:m.}
\end{itemize}
Planta labiada do Brasil.--As duas variantes deixam-me suppor que \textunderscore san-pedro-caá\textunderscore  é corruptela, devida aos diccionaristas, em vez de \textunderscore san-pedro-çaá\textunderscore .
\section{San-pedro-saá}
\begin{itemize}
\item {Grp. gram.:m.}
\end{itemize}
Planta labiada do Brasil.--As duas variantes deixam-me suppor que \textunderscore san-pedro-saá\textunderscore  é corruptela, devida aos diccionaristas, em vez de \textunderscore san-pedro-çaá\textunderscore .
\section{Santil}
\begin{itemize}
\item {Grp. gram.:m.}
\end{itemize}
\begin{itemize}
\item {Utilização:Chím.}
\end{itemize}
Éter salicilico da essência do sândalo, aplicável contra a blenorragia.
\section{Santilão}
\begin{itemize}
\item {Grp. gram.:m.}
\end{itemize}
\begin{itemize}
\item {Utilização:Pop.}
\end{itemize}
O mesmo que \textunderscore santarrão\textunderscore .
\section{Santimónia}
\begin{itemize}
\item {Grp. gram.:f.}
\end{itemize}
\begin{itemize}
\item {Proveniência:(Lat. \textunderscore sanctimónia\textunderscore )}
\end{itemize}
Santidade.
Modos de santo.
Devoções religiosas.
\section{Santimonial}
\begin{itemize}
\item {Grp. gram.:adj.}
\end{itemize}
\begin{itemize}
\item {Proveniência:(Lat. \textunderscore sanctimonialis\textunderscore )}
\end{itemize}
Relativo a santimónia.
Santanário; devoto.
\section{Santinho}
\begin{itemize}
\item {Grp. gram.:m.}
\end{itemize}
\begin{itemize}
\item {Utilização:Fam.}
\end{itemize}
Pequena imagem de um santo.
\textunderscore Santinho de pau carunchoso\textunderscore , indivíduo sonso, velhaco.
\section{Santíssimo}
\begin{itemize}
\item {Grp. gram.:adj.}
\end{itemize}
\begin{itemize}
\item {Grp. gram.:M.}
\end{itemize}
\begin{itemize}
\item {Utilização:Gír.}
\end{itemize}
\begin{itemize}
\item {Proveniência:(Lat. \textunderscore sanctissimus\textunderscore )}
\end{itemize}
Muito santo; que é mais santo que todos os santos.
Sacramento da Eucharistia.
Hóstia consagrada.
\textunderscore Irmão do Santíssimo\textunderscore , o mesmo que \textunderscore percevejo\textunderscore .
\section{Santista}
\begin{itemize}
\item {Grp. gram.:adj.}
\end{itemize}
\begin{itemize}
\item {Utilização:Bras}
\end{itemize}
\begin{itemize}
\item {Grp. gram.:M.}
\end{itemize}
\begin{itemize}
\item {Utilização:Bras}
\end{itemize}
Relativo á cidade de Santos.
Habitante de Santos.
\section{Santo}
\begin{itemize}
\item {Grp. gram.:adj.}
\end{itemize}
\begin{itemize}
\item {Grp. gram.:M.}
\end{itemize}
\begin{itemize}
\item {Utilização:Fig.}
\end{itemize}
\begin{itemize}
\item {Proveniência:(Lat. \textunderscore sanctus\textunderscore )}
\end{itemize}
Relativo á religião ou ás práticas sagradas.
Que vive segundo a lei divina.
Bem aventurado.
Puro; immaculado.
Innocente.
Venerável.
Inviolável.
Profícuo.
Que cura, que é efficaz: \textunderscore foi remédio santo\textunderscore .
Diz-se dos dias santificados ou em que a Igreja prohibe trabalho.
Indivíduo, que morreu em estado de santidade, ou que foi canonizado.
Imagem de indivíduo que foi canonizado.
Indivíduo de grande austeridade de costumes ou de extraordinária bondade.
\section{Santoanné}
\begin{itemize}
\item {Grp. gram.:m.}
\end{itemize}
\begin{itemize}
\item {Utilização:Ant.}
\end{itemize}
\begin{itemize}
\item {Proveniência:(De \textunderscore Santoanne\textunderscore , n. p.)}
\end{itemize}
Tecido leve, próprio para vestuário em tempo quente ou em tempo de San-João.
\section{Santo-antónio}
\begin{itemize}
\item {Grp. gram.:f.}
\end{itemize}
Espécie de ameixa côr de rosa.
Variedade de pêra ordinária e temporan.
\section{Santo-e-senha}
\begin{itemize}
\item {Grp. gram.:m.}
\end{itemize}
\begin{itemize}
\item {Utilização:Ext.}
\end{itemize}
Bilhete, em que se escreve o nome de um santo e mais um sinal ou outro nome, e que se entrega ás guardas e ás sentinela, para não considerarem inimigo aquelle que mostrar conhecer o nome e o sinal contido nesse bilhete.
Sinal, combinado, para se conhecer sem indiscrição quem é partidário ou adversário.
\section{Santola}
\begin{itemize}
\item {Grp. gram.:f.}
\end{itemize}
Gênero de grandes caranguejos, de que há várias espécies.
\section{Santolina}
\begin{itemize}
\item {Grp. gram.:f.}
\end{itemize}
Planta, da fam. das compostas, (\textunderscore diotis candidissima\textunderscore ).
\section{Santolinha}
\begin{itemize}
\item {Grp. gram.:f.}
\end{itemize}
Espécie de santola.
\section{Santolino}
\begin{itemize}
\item {Grp. gram.:m.}
\end{itemize}
O mesmo que \textunderscore santolina\textunderscore .
\section{Santom}
\begin{itemize}
\item {Grp. gram.:m.}
\end{itemize}
Árvore da Índia Portuguesa.
\section{Santomé}
\begin{itemize}
\item {Grp. gram.:m.}
\end{itemize}
\begin{itemize}
\item {Utilização:Bras}
\end{itemize}
\begin{itemize}
\item {Utilização:Bras}
\end{itemize}
Moéda de oiro, cunhada por Garcia de Sá na Índia.
Árvore silvestre, que produz uma resina, parecida ao beijoim.
Variedade de bananeira, originária da ilha de San-Thomé.
\section{Santomense}
\begin{itemize}
\item {Grp. gram.:adj.}
\end{itemize}
\begin{itemize}
\item {Grp. gram.:M.}
\end{itemize}
Relativo á ilha ou á província de San-Thomé.
Habitante de San-Thomé.
\section{Santoméri}
\begin{itemize}
\item {Grp. gram.:m.}
\end{itemize}
\begin{itemize}
\item {Utilização:Ant.}
\end{itemize}
O mesmo que \textunderscore santaomé\textunderscore .
\section{Santonica}
\begin{itemize}
\item {Grp. gram.:f.}
\end{itemize}
O mesmo que \textunderscore santonina\textunderscore .
\section{Santonico}
\begin{itemize}
\item {Grp. gram.:m.}
\end{itemize}
O mesmo que \textunderscore santonina\textunderscore .
\section{Santonina}
\begin{itemize}
\item {Grp. gram.:f.}
\end{itemize}
\begin{itemize}
\item {Utilização:Chím.}
\end{itemize}
Planta vermífuga, da fam. das compostas, (\textunderscore artemisia santonica\textunderscore , Lin.).
Princípio immediato da santonina, com que se fabricam pastilhas vermífugas.
\section{Santono}
\begin{itemize}
\item {Grp. gram.:m.}
\end{itemize}
Planta aromática da Índia, talvez o mesmo que \textunderscore santom\textunderscore .
\section{Santopeia}
\begin{itemize}
\item {Grp. gram.:f.}
\end{itemize}
\begin{itemize}
\item {Utilização:ant.}
\end{itemize}
\begin{itemize}
\item {Utilização:Pop.}
\end{itemize}
O mesmo que \textunderscore centopeia\textunderscore . Cf. B. Pereira. \textunderscore Prosódia\textunderscore , vb. \textunderscore scolopendra\textunderscore .
\section{Santor}
\begin{itemize}
\item {Grp. gram.:m.}
\end{itemize}
\begin{itemize}
\item {Utilização:Heráld.}
\end{itemize}
\begin{itemize}
\item {Proveniência:(Fr. \textunderscore sautoir\textunderscore )}
\end{itemize}
Figura, composta de dois objectos, dispostos de maneira que imitam um X ou a cruz de Santo André.
Aspa nos brasões.
\section{Santoral}
\begin{itemize}
\item {Grp. gram.:m.}
\end{itemize}
\begin{itemize}
\item {Proveniência:(De \textunderscore santo\textunderscore )}
\end{itemize}
O mesmo que \textunderscore agiológio\textunderscore .
Livro litúrgico, que contém os hymnos dos santos; hymnanário.
\section{Santório}
\begin{itemize}
\item {Grp. gram.:m.}
\end{itemize}
\begin{itemize}
\item {Utilização:Prov.}
\end{itemize}
O mesmo que \textunderscore santoro\textunderscore .
\section{Santoro}
\begin{itemize}
\item {Grp. gram.:m.}
\end{itemize}
\begin{itemize}
\item {Utilização:Prov.}
\end{itemize}
\begin{itemize}
\item {Utilização:T. da Bairrada}
\end{itemize}
\begin{itemize}
\item {Proveniência:(Do lat. \textunderscore sanctorum\textunderscore , genitivo pl. de \textunderscore sanctus\textunderscore )}
\end{itemize}
Espécie de pão bento.
Bolo comprido, que se dá em dia de Finados ou de Todos-os-Santos e que é do feitio de uma tíbia.
Fruta, que no dia de Todos-os-Santos se dá ás crianças que a andam pedindo de porta em porta.
\section{Santórum}
\begin{itemize}
\item {Grp. gram.:m.}
\end{itemize}
\begin{itemize}
\item {Utilização:Prov.}
\end{itemize}
(V.santoro)
\section{Santos}
\begin{itemize}
\item {Grp. gram.:m.}
\end{itemize}
\begin{itemize}
\item {Utilização:Bras}
\end{itemize}
Peixe, o mesmo que \textunderscore santa\textunderscore .
\section{Santuária}
\begin{itemize}
\item {Grp. gram.:f.}
\end{itemize}
Planta da serra de Sintra.
\section{Santuário}
\begin{itemize}
\item {Grp. gram.:m.}
\end{itemize}
\begin{itemize}
\item {Utilização:Fig.}
\end{itemize}
\begin{itemize}
\item {Proveniência:(Lat. \textunderscore sanctuarium\textunderscore )}
\end{itemize}
Lugar reservado e consagrado pela religião.
O lugar mais sagrado do templo de Jerusalém, onde estava a Arca da Alliança.
Templo, capella.
Sacrário; relicário.
A parte mais íntima (do peito, do coração, da alma).
\section{Santulha}
\begin{itemize}
\item {Grp. gram.:adj. f.}
\end{itemize}
O mesmo que \textunderscore santulhana\textunderscore .
\section{Santulhana}
\begin{itemize}
\item {Grp. gram.:adj. f.}
\end{itemize}
\begin{itemize}
\item {Proveniência:(De \textunderscore Santulhão\textunderscore , n. p.)}
\end{itemize}
Diz-se de uma variedade de azeitona trasmontana. Cf. \textunderscore Port. au point de vue agr.\textunderscore , 452 e 453.
\section{Santyl}
\begin{itemize}
\item {Grp. gram.:m.}
\end{itemize}
\begin{itemize}
\item {Utilização:Chím.}
\end{itemize}
Éther salicylico da essência do sândalo, applicável contra a blenorrhagia.
\section{San-vicente}
\begin{itemize}
\item {Grp. gram.:m.}
\end{itemize}
Moéda de oiro, do tempo de D. João III.
\section{Sanvitália}
\begin{itemize}
\item {Grp. gram.:f.}
\end{itemize}
Gênero de plantas herbáceas, originárias do México.
\section{Sanvori}
\begin{itemize}
\item {Grp. gram.:m.}
\end{itemize}
Planta aromática da Índia.
\section{Sanzala}
\begin{itemize}
\item {Grp. gram.:f.}
\end{itemize}
O mesmo que \textunderscore senzala\textunderscore . Cf. Castilho, \textunderscore Fausto\textunderscore , 190.
\section{São}
\begin{itemize}
\item {Grp. gram.:adj.}
\end{itemize}
\begin{itemize}
\item {Grp. gram.:M.}
\end{itemize}
\begin{itemize}
\item {Proveniência:(Do lat. \textunderscore sanus\textunderscore )}
\end{itemize}
Que tem saúde.
Curado.
Incólume.
Que não tem defeito.
Salutar.
Íntegro; recto.
Puro.
Justo; razoável: \textunderscore doutrina san\textunderscore .
Indivíduo, que tem saúde.
Parte san de um objecto ou de um organismo.
Estado perfeito.
\section{São}
Abrev. de \textunderscore santo\textunderscore , a qual se antepõe ao nome dos santos que começa por consoante: \textunderscore São-Pedro\textunderscore , \textunderscore São-Francisco\textunderscore , ...
\section{São}
Fórma ant. da 1.^a pess. do indic. pres. do v. \textunderscore sêr\textunderscore . Cf. G. Vicente.
(Cp. \textunderscore sam\textunderscore )
\section{São-gonçalo}
\begin{itemize}
\item {Grp. gram.:m.}
\end{itemize}
\begin{itemize}
\item {Utilização:Bras}
\end{itemize}
Espécie de baile, em que os festeiros cantam, dançam e se embriagam, de noite, ao ar livre, em frente da imagem de San-Gonçalo.
\section{São-martinho}
\begin{itemize}
\item {Grp. gram.:m.}
\end{itemize}
Variedade de pêra ordinária.
\section{São-miguel}
\begin{itemize}
\item {Grp. gram.:m.}
\end{itemize}
\begin{itemize}
\item {Utilização:Fig.}
\end{itemize}
Época das colheitas: \textunderscore prometeu pagar pelo São-Miguel\textunderscore .
Colheita.
Fortuna.
Variedade de pêra pardacenta, açucarada e um pouco granulosa.
\section{São-pedro-caá}
\begin{itemize}
\item {Grp. gram.:m.}
\end{itemize}
Planta labiada do Brasil.--As duas variantes deixam-me suppor que \textunderscore san-pedro-caá\textunderscore  é corruptela, devida aos diccionaristas, em vez de \textunderscore san-pedro-çaá\textunderscore .
\section{Sapa}
\begin{itemize}
\item {Grp. gram.:f.}
\end{itemize}
\begin{itemize}
\item {Utilização:Prov.}
\end{itemize}
\begin{itemize}
\item {Utilização:trasm.}
\end{itemize}
\begin{itemize}
\item {Utilização:Fig.}
\end{itemize}
Pá, com que se ergue a terra que se escavou.
Abertura de fossos, trincheiras e galerias subterrâneas etc., geralmente para se accommeter uma praça ao abrigo dos sitiados.
Trabalho de sapador.
Alluvião.
Trabalho occulto, ardiloso.
Ardil.
(Cp. cast. \textunderscore zapa\textunderscore  e o b. lat. \textunderscore zappa\textunderscore )
\section{Sapa}
\begin{itemize}
\item {Grp. gram.:f.}
\end{itemize}
\begin{itemize}
\item {Utilização:Prov.}
\end{itemize}
\begin{itemize}
\item {Utilização:beir.}
\end{itemize}
Tampa ou testo de panela ou de outro vaso culinário.
\section{Sapa}
\begin{itemize}
\item {Grp. gram.:f.}
\end{itemize}
\begin{itemize}
\item {Utilização:Prov.}
\end{itemize}
\begin{itemize}
\item {Utilização:trasm.}
\end{itemize}
\begin{itemize}
\item {Proveniência:(De \textunderscore sapo\textunderscore )}
\end{itemize}
Pessôa de baixa estatura.
\section{Sapada}
\begin{itemize}
\item {Grp. gram.:f.}
\end{itemize}
\begin{itemize}
\item {Utilização:Prov.}
\end{itemize}
\begin{itemize}
\item {Proveniência:(De \textunderscore sapar\textunderscore ^1)}
\end{itemize}
Desmoronamento de cômoros ou socalcos nas vinhas.
\section{Sapadoira}
\begin{itemize}
\item {Grp. gram.:f.}
\end{itemize}
\begin{itemize}
\item {Utilização:Prov.}
\end{itemize}
\begin{itemize}
\item {Proveniência:(De \textunderscore sapar\textunderscore ^2)}
\end{itemize}
Tampa, o mesmo que \textunderscore sapa\textunderscore ^2.
\section{Sapador}
\begin{itemize}
\item {Grp. gram.:m.}
\end{itemize}
Soldado, ou qualquer indivíduo, que executa trabalhos de sapa.
(B. lat. \textunderscore zappator\textunderscore )
\section{Sapadoura}
\begin{itemize}
\item {Grp. gram.:f.}
\end{itemize}
\begin{itemize}
\item {Utilização:Prov.}
\end{itemize}
\begin{itemize}
\item {Proveniência:(De \textunderscore sapar\textunderscore ^2)}
\end{itemize}
Tampa, o mesmo que \textunderscore sapa\textunderscore ^2.
\section{Sapajo}
\begin{itemize}
\item {Grp. gram.:m.}
\end{itemize}
O mesmo ou melhor que \textunderscore sapaju-aurora\textunderscore .
\section{Sapaju-aurora}
\begin{itemize}
\item {Grp. gram.:m.}
\end{itemize}
Espécie de macaco, também conhecido por \textunderscore seimiri\textunderscore .
\section{Sapal}
\begin{itemize}
\item {Grp. gram.:m.}
\end{itemize}
\begin{itemize}
\item {Proveniência:(De \textunderscore sapo\textunderscore ?)}
\end{itemize}
Terra alagadiça, ordinariamente á beira dos rios; brejo; paúl.
\section{Sapam}
\begin{itemize}
\item {Grp. gram.:m.}
\end{itemize}
\begin{itemize}
\item {Utilização:Ant.}
\end{itemize}
\begin{itemize}
\item {Proveniência:(Do mal. \textunderscore sápan\textunderscore )}
\end{itemize}
O mesmo que \textunderscore brasil\textunderscore ^1, pau.
\section{Sapanzoba}
\begin{itemize}
\item {Grp. gram.:f.}
\end{itemize}
Pássaro conirostro, (\textunderscore fringillaria flaviventris\textunderscore ).
\section{Sapão}
\begin{itemize}
\item {Grp. gram.:m.}
\end{itemize}
\begin{itemize}
\item {Utilização:Ant.}
\end{itemize}
\begin{itemize}
\item {Proveniência:(Do mal. \textunderscore sápan\textunderscore )}
\end{itemize}
O mesmo que \textunderscore brasil\textunderscore ^1, pau.
\section{Sapar}
\begin{itemize}
\item {Grp. gram.:v. i.}
\end{itemize}
Trabalhar com a sapa.
Executar trabalhos de sapa.
\section{Sapar}
\begin{itemize}
\item {Grp. gram.:v. t.}
\end{itemize}
\begin{itemize}
\item {Utilização:Prov.}
\end{itemize}
\begin{itemize}
\item {Utilização:beir.}
\end{itemize}
Cobrir com testo ou tampa; pôr a sapa em.
\section{Saparás}
\begin{itemize}
\item {Grp. gram.:m. pl.}
\end{itemize}
Indígenas do norte do Brasil.
\section{Saparrão}
\begin{itemize}
\item {Grp. gram.:m.}
\end{itemize}
\begin{itemize}
\item {Utilização:Fig.}
\end{itemize}
Sapo grande.
Homem gordo e desajeitado:«\textunderscore ...uma flôr sêr mulher de um saparrão assim...\textunderscore »Castilho, \textunderscore Tartufo\textunderscore , 40.
\section{Sapata}
\begin{itemize}
\item {Grp. gram.:f.}
\end{itemize}
\begin{itemize}
\item {Utilização:Náut.}
\end{itemize}
Chinela de coiro.
Peça de madeira sôbre um pilar, para reforçar ou equilibrar a trave que assenta nella.
Pequena bigota, com furo no meio, e em fórma de sapato.
Poleame, que se firma no chicote dos cabrestos, estais, etc.
Rodela de camurça, na chave dos instrumentos músicos.
Calço de pedra ou supplemento saliente á base de uma parede, para a fortificar.
O mesmo que \textunderscore berma\textunderscore .
(Cast. \textunderscore zapata\textunderscore )
\section{Sapata-branca}
\begin{itemize}
\item {Grp. gram.:f.}
\end{itemize}
Peixe plagióstomo, cinzento, de focinho largo, chato e muito longo.
\section{Sapatada}
\begin{itemize}
\item {Grp. gram.:f.}
\end{itemize}
\begin{itemize}
\item {Utilização:Pop.}
\end{itemize}
Pancada com o sapato.
Pancada, que o gato dá com a pata.
\section{Sapatadinha}
\begin{itemize}
\item {Grp. gram.:f.}
\end{itemize}
Espécie de jôgo popular.
\section{Sapata-preta}
\begin{itemize}
\item {Grp. gram.:f.}
\end{itemize}
Peixe plagióstomo, de focinho chato e longo, e de côr pardo-escura.
\section{Sapatar}
\begin{itemize}
\item {Grp. gram.:v. t.}
\end{itemize}
\begin{itemize}
\item {Utilização:Prov.}
\end{itemize}
Partir (vagens): \textunderscore feijões de sapatar\textunderscore ; \textunderscore ervilhas de sapatar\textunderscore . (Colhido na Bairrada)
\section{Sapataria}
\begin{itemize}
\item {Grp. gram.:f.}
\end{itemize}
Arte ou estabelecimento de sapateiro.
Arruamento de sapateiros.
\section{Sapaté}
\begin{itemize}
\item {Grp. gram.:m.}
\end{itemize}
Pequeno arbusto, que cresce junto de Bissau, e cujas fôlhas tem propriedades calmantes.
\section{Sapateada}
\begin{itemize}
\item {Grp. gram.:f.}
\end{itemize}
Acto ou effeito de sapatear.
\section{Sapateado}
\begin{itemize}
\item {Grp. gram.:m.}
\end{itemize}
\begin{itemize}
\item {Proveniência:(De \textunderscore sapatear\textunderscore )}
\end{itemize}
Sapateada.
Dança popular, em que se faz grande ruído com os tacões do calçado.
\section{Sapateão}
\begin{itemize}
\item {Grp. gram.:m.}
\end{itemize}
Espécie de embarcação chinesa.
\section{Sapatear}
\begin{itemize}
\item {Grp. gram.:v. i.}
\end{itemize}
\begin{itemize}
\item {Grp. gram.:V. t.}
\end{itemize}
\begin{itemize}
\item {Proveniência:(De \textunderscore sapato\textunderscore )}
\end{itemize}
Bater no chão com o salto do calçado.
Executar (uma dança), fazendo grande ruído com o calçado ou só com os saltos do calçado.
\section{Sapateia}
\begin{itemize}
\item {Grp. gram.:f.}
\end{itemize}
\begin{itemize}
\item {Proveniência:(De \textunderscore sapatear\textunderscore )}
\end{itemize}
Dança popular dos Açores.
\section{Sapateira}
\begin{itemize}
\item {Grp. gram.:f.}
\end{itemize}
Mulhér de sapateiro.
Mulhér, que faz sapatos.
Designação genérica do várias plantas melastomáceas.
Nome de alguns crustáceos decápodes.
\section{Sapateiral}
\begin{itemize}
\item {Grp. gram.:adj.}
\end{itemize}
\begin{itemize}
\item {Utilização:Chul.}
\end{itemize}
Próprio de sapateiro.
Que tem modos de sapateiro.
\section{Sapateiro}
\begin{itemize}
\item {Grp. gram.:m.}
\end{itemize}
\begin{itemize}
\item {Utilização:T. de Penafiel}
\end{itemize}
Aquelle que faz sapatos ou trabalha em calçado.
Vendedor de calçado.
Insecto orthóptero, também chamado \textunderscore fèdevelha\textunderscore .
\section{Sapateta}
\begin{itemize}
\item {Grp. gram.:f.}
\end{itemize}
\begin{itemize}
\item {Proveniência:(De \textunderscore sapata\textunderscore )}
\end{itemize}
Chinela.
Ruído, produzido pelos tacões quando se anda.
\section{Sapatilha}
\begin{itemize}
\item {Grp. gram.:f.}
\end{itemize}
\begin{itemize}
\item {Proveniência:(De \textunderscore sapata\textunderscore )}
\end{itemize}
Sapata dos instrumentos músicos.
*\textunderscore Chapel.\textunderscore 
Peça de ferro, com que os fulistas recalcam os chapéus, para dar unidade e consistência ao pêlo.
\section{Sapatilho}
\begin{itemize}
\item {Grp. gram.:m.}
\end{itemize}
\begin{itemize}
\item {Proveniência:(De \textunderscore sapato\textunderscore )}
\end{itemize}
Arco de ferro, canelado exteriormente, que se firma nos chicotes dos cabos náuticos, etc.
Primeira fôlha sêca, que se tira da cana do açúcar, quando esta se limpa.
\section{Safena}
\begin{itemize}
\item {Grp. gram.:f.}
\end{itemize}
\begin{itemize}
\item {Utilização:Anat.}
\end{itemize}
\begin{itemize}
\item {Proveniência:(De \textunderscore safeno\textunderscore )}
\end{itemize}
Veia safena.
\section{Safeno}
\begin{itemize}
\item {Grp. gram.:adj.}
\end{itemize}
\begin{itemize}
\item {Utilização:Anat.}
\end{itemize}
\begin{itemize}
\item {Proveniência:(Do gr. \textunderscore saphenès\textunderscore )}
\end{itemize}
Diz-se de duas veias da perna e do pé.
Diz-se de alguns feixes nervosos da perna e da côxa.
\section{Sáfico}
\begin{itemize}
\item {Grp. gram.:adj.}
\end{itemize}
\begin{itemize}
\item {Proveniência:(Gr. \textunderscore saphikos\textunderscore )}
\end{itemize}
Relativo a Sapho.
Diz-se de um verso de cinco pés.
Diz-se do verso português decasýlabo, com accentuação tónica na quarta, oitava e décima sýllaba.
Diz-se também de uma estrophe, que tem três versos sáphicos e um adónio.
\section{Safira}
\begin{itemize}
\item {Grp. gram.:f.}
\end{itemize}
\begin{itemize}
\item {Proveniência:(Do lat. \textunderscore saphirus\textunderscore )}
\end{itemize}
Pedra preciosa azul.
A côr azul.
\section{Safírico}
\begin{itemize}
\item {Grp. gram.:adj.}
\end{itemize}
Relativo a safira:«\textunderscore brilho safírico\textunderscore ». Eça.
\section{Safismo}
\begin{itemize}
\item {Grp. gram.:m.}
\end{itemize}
\begin{itemize}
\item {Proveniência:(De \textunderscore Sapho\textunderscore , n. p.)}
\end{itemize}
Amor homosexual, de mulhér para mulhér; amor lésbio.
\section{Safista}
\begin{itemize}
\item {Grp. gram.:f.}
\end{itemize}
Mulhér, que tem safismo.
\section{Sapatinho}
\begin{itemize}
\item {Grp. gram.:m.}
\end{itemize}
Sapato pequeno e delicado.
Espécie de jôgo popular.
\section{Sapatinho-dos-jardins}
\begin{itemize}
\item {Grp. gram.:m.}
\end{itemize}
\begin{itemize}
\item {Utilização:Bras}
\end{itemize}
Planta euphorbiácea medicinal.
\section{Sapato}
\begin{itemize}
\item {Grp. gram.:m.}
\end{itemize}
\begin{itemize}
\item {Grp. gram.:Loc.}
\end{itemize}
\begin{itemize}
\item {Utilização:fam.}
\end{itemize}
Peça de calçado, que cobre só o pé.
\textunderscore Sapatos de defunto\textunderscore , promessas ou esperanças, cuja realização é muito demorada ou muito incerta.
(Cp. \textunderscore sapata\textunderscore )
\section{Sapatola}
\begin{itemize}
\item {Grp. gram.:f.}
\end{itemize}
\begin{itemize}
\item {Grp. gram.:M.}
\end{itemize}
Sapato grande e mal feito.
O mesmo que \textunderscore remendão\textunderscore .
\section{Sapatorra}
\begin{itemize}
\item {fónica:tô}
\end{itemize}
\begin{itemize}
\item {Grp. gram.:f.}
\end{itemize}
O mesmo que \textunderscore sapatorro\textunderscore .
\section{Sapatorro}
\begin{itemize}
\item {fónica:tô}
\end{itemize}
\begin{itemize}
\item {Grp. gram.:m.}
\end{itemize}
Sapato grosseiro e malfeito.
\section{Sapatrancas}
\begin{itemize}
\item {Grp. gram.:f. pl.}
\end{itemize}
\begin{itemize}
\item {Utilização:Bras}
\end{itemize}
Sapatos grossos, sapatorros.
\section{Sape!}
\begin{itemize}
\item {Grp. gram.:interj.}
\end{itemize}
(usada para afugentar os gatos)
\section{Sapé}
\begin{itemize}
\item {Grp. gram.:m.}
\end{itemize}
Nome de várias plantas gramíneas do Brasil.
\section{Sapeca}
\begin{itemize}
\item {Grp. gram.:f.}
\end{itemize}
Ínfima moéda de cobre chinesa, com um orifício no centro.
(Mal. \textunderscore sapeka\textunderscore , enfiada de cem moédas de cobre)
\section{Sapeca}
\begin{itemize}
\item {Grp. gram.:f.}
\end{itemize}
\begin{itemize}
\item {Utilização:Bras}
\end{itemize}
Acto de sapecar.
Chamuscadura.
Sova, tunda.
\section{Sapeca}
\begin{itemize}
\item {Grp. gram.:f.}
\end{itemize}
\begin{itemize}
\item {Utilização:Açor}
\end{itemize}
Descompostura.
\section{Sapeca}
\begin{itemize}
\item {Grp. gram.:f.}
\end{itemize}
\begin{itemize}
\item {Utilização:Prov.}
\end{itemize}
Rapariga vadia ou muito namoradeira.
\section{Sapecagem}
\begin{itemize}
\item {Grp. gram.:f.}
\end{itemize}
Acto de sapecar.
\section{Sapecar}
\begin{itemize}
\item {Grp. gram.:v. i.}
\end{itemize}
\begin{itemize}
\item {Utilização:Bras}
\end{itemize}
Vadiar.
Namorar.
(Cp. \textunderscore sapeca\textunderscore ^4)
\section{Sapecar}
\begin{itemize}
\item {Grp. gram.:v. t.}
\end{itemize}
\begin{itemize}
\item {Utilização:Bras}
\end{itemize}
Chamuscar ou secar (a congonha).
Secar ou moquear (carne ou peças de caça), para se conservarem.
(Do tupi \textunderscore sapec\textunderscore )
\section{Sapécoas}
\begin{itemize}
\item {Grp. gram.:f. pl.}
\end{itemize}
\begin{itemize}
\item {Utilização:T. da Maia}
\end{itemize}
Dinheiro.
(Cp. \textunderscore sapeca\textunderscore ^1)
\section{Sape-gato}
\begin{itemize}
\item {Grp. gram.:m.}
\end{itemize}
Espécie de jôgo infantil.
\section{Sapeira}
\begin{itemize}
\item {Grp. gram.:adj. f.}
\end{itemize}
\begin{itemize}
\item {Utilização:Pesc.}
\end{itemize}
\begin{itemize}
\item {Grp. gram.:F.}
\end{itemize}
\begin{itemize}
\item {Utilização:Prov.}
\end{itemize}
\begin{itemize}
\item {Utilização:trasm.}
\end{itemize}
\begin{itemize}
\item {Utilização:Prov.}
\end{itemize}
\begin{itemize}
\item {Utilização:trasm.}
\end{itemize}
\begin{itemize}
\item {Proveniência:(De \textunderscore sapo\textunderscore )}
\end{itemize}
Diz-se da truta de água doce.
Ódio figadal.
O mesmo que \textunderscore fome\textunderscore .
\section{Sapeira}
\begin{itemize}
\item {Grp. gram.:f.}
\end{itemize}
\begin{itemize}
\item {Utilização:T. de Canaveses}
\end{itemize}
Desmoronamento de sapa, de mina ou de galeria subterrânea?:«\textunderscore um pedreiro trabalhando num cano, foi apanhado e morto por uma sapeira.\textunderscore »(Dos jornaes)
(Cp. \textunderscore sapadoira\textunderscore )
\section{Sapejar}
\begin{itemize}
\item {Grp. gram.:v. i.}
\end{itemize}
\begin{itemize}
\item {Utilização:Prov.}
\end{itemize}
\begin{itemize}
\item {Proveniência:(De \textunderscore sapo\textunderscore )}
\end{itemize}
Caminhar com difficuldade, curvado, quási de rastos, como o sapo.
\section{Sapejar}
\begin{itemize}
\item {Grp. gram.:v. t.}
\end{itemize}
\begin{itemize}
\item {Utilização:T. de Turquel}
\end{itemize}
\begin{itemize}
\item {Proveniência:(De \textunderscore sape\textunderscore )}
\end{itemize}
Enxotar (gatos)
\section{Sape-leve}
\begin{itemize}
\item {Grp. gram.:m.}
\end{itemize}
\begin{itemize}
\item {Utilização:Prov.}
\end{itemize}
O mesmo que \textunderscore falcão\textunderscore , (\textunderscore falco communis\textunderscore , Gmel.). Cf. Ed. Sequeira, \textunderscore Ovos e Ninhos\textunderscore .
(Cp. \textunderscore sapo-leve\textunderscore )
\section{Sapelo}
\begin{itemize}
\item {fónica:pê}
\end{itemize}
\begin{itemize}
\item {Proveniência:(De \textunderscore sapo\textunderscore )}
\end{itemize}
\textunderscore m. Prov. minh.\textunderscore 
Pessôa de baixa estatura.
\section{Sapenos}
\begin{itemize}
\item {Grp. gram.:m.}
\end{itemize}
\begin{itemize}
\item {Proveniência:(Lat. \textunderscore sapenos\textunderscore )}
\end{itemize}
Variedade de amethysta oriental, azul-clara.
\section{Sapequeiro}
\begin{itemize}
\item {Grp. gram.:m.}
\end{itemize}
\begin{itemize}
\item {Utilização:Bras. do N}
\end{itemize}
\begin{itemize}
\item {Proveniência:(De \textunderscore sapeca\textunderscore ^2)}
\end{itemize}
Terreno, em que lavrou fogo.
\section{Saperda}
\begin{itemize}
\item {Grp. gram.:f.}
\end{itemize}
\begin{itemize}
\item {Proveniência:(Lat. \textunderscore saperda\textunderscore )}
\end{itemize}
Gênero de insectos coleópteros, longicórneos.
\section{Saperê}
\begin{itemize}
\item {Grp. gram.:adj.}
\end{itemize}
\begin{itemize}
\item {Utilização:Bras}
\end{itemize}
Diz-se da cana de açúcar sem préstimo para a moagem ou replantação.
\section{Saperecar}
\begin{itemize}
\item {Grp. gram.:v. t.}
\end{itemize}
\begin{itemize}
\item {Utilização:Bras}
\end{itemize}
O mesmo que \textunderscore sapecar\textunderscore ^2.
\section{Sapezal}
\begin{itemize}
\item {Grp. gram.:m.}
\end{itemize}
Terreno, onde crescem sapés.
\section{Saphena}
\begin{itemize}
\item {Grp. gram.:f.}
\end{itemize}
\begin{itemize}
\item {Utilização:Anat.}
\end{itemize}
\begin{itemize}
\item {Proveniência:(De \textunderscore sapheno\textunderscore )}
\end{itemize}
Veia saphena.
\section{Sapheno}
\begin{itemize}
\item {Grp. gram.:adj.}
\end{itemize}
\begin{itemize}
\item {Utilização:Anat.}
\end{itemize}
\begin{itemize}
\item {Proveniência:(Do gr. \textunderscore saphenès\textunderscore )}
\end{itemize}
Diz-se de duas veias da perna e do pé.
Diz-se de alguns feixes nervosos da perna e da côxa.
\section{Sáphico}
\begin{itemize}
\item {Grp. gram.:adj.}
\end{itemize}
\begin{itemize}
\item {Proveniência:(Gr. \textunderscore saphikos\textunderscore )}
\end{itemize}
Relativo a Sapho.
Diz-se de um verso de cinco pés.
Diz-se do verso português decasýlabo, com accentuação tónica na quarta, oitava e décima sýllaba.
Diz-se também de uma estrophe, que tem três versos sáphicos e um adónio.
\section{Saphira}
\begin{itemize}
\item {Grp. gram.:f.}
\end{itemize}
\begin{itemize}
\item {Proveniência:(Do lat. \textunderscore saphirus\textunderscore )}
\end{itemize}
Pedra preciosa azul.
A côr azul.
\section{Saphírico}
\begin{itemize}
\item {Grp. gram.:adj.}
\end{itemize}
Relativo a saphira:«\textunderscore brilho saphírico\textunderscore ». Eça.
\section{Saphismo}
\begin{itemize}
\item {Grp. gram.:m.}
\end{itemize}
\begin{itemize}
\item {Proveniência:(De \textunderscore Sapho\textunderscore , n. p.)}
\end{itemize}
Amor homosexual, de mulhér para mulhér; amor lésbio.
\section{Saphista}
\begin{itemize}
\item {Grp. gram.:f.}
\end{itemize}
Mulhér, que tem saphismo.
\section{Sápia}
\begin{itemize}
\item {Grp. gram.:f.}
\end{itemize}
Variedade de madeira de pinho.
(Cp. lat. \textunderscore sapinus\textunderscore )
\section{Sápido}
\begin{itemize}
\item {Grp. gram.:adj.}
\end{itemize}
\begin{itemize}
\item {Proveniência:(Lat. \textunderscore sapidus\textunderscore )}
\end{itemize}
Que tem sabor; saboroso.
\section{Sapiência}
\begin{itemize}
\item {Grp. gram.:f.}
\end{itemize}
\begin{itemize}
\item {Proveniência:(Lat. \textunderscore sapientia\textunderscore )}
\end{itemize}
Qualidade do que é sapiente.
Sabedoria divina.
Tratamento irónico:«\textunderscore Vossa Sapiência...\textunderscore »Rebello, \textunderscore Contos e Lendas\textunderscore , 95.
\textunderscore Oração de sapiência\textunderscore , discurso, pronunciado na inauguração de um curso escolar pelo respectivo director ou reitor, ou por um lente.
\section{Sapiencial}
\begin{itemize}
\item {Grp. gram.:adj.}
\end{itemize}
Relativo á sapiência.
\section{Sapiente}
\begin{itemize}
\item {Grp. gram.:adj.}
\end{itemize}
\begin{itemize}
\item {Proveniência:(Lat. \textunderscore sapiens\textunderscore )}
\end{itemize}
Sabedor, sábio.
Que conhece as coisas divinas e humanas.
\section{Sapientemente}
\begin{itemize}
\item {Grp. gram.:adv.}
\end{itemize}
De modo sapiente.
\section{Sapindáceas}
\begin{itemize}
\item {Grp. gram.:f. pl.}
\end{itemize}
Família de plantas, que tem por typo a saponária ou saboeira.
(Fem. pl. de \textunderscore sapindáceo\textunderscore )
\section{Sapindáceo}
\begin{itemize}
\item {Grp. gram.:adj.}
\end{itemize}
\begin{itemize}
\item {Proveniência:(Do lat. \textunderscore sapindus\textunderscore )}
\end{itemize}
Relativo ou semelhante á saboeira.
\section{Sapinhos}
\begin{itemize}
\item {Grp. gram.:m. pl.}
\end{itemize}
\begin{itemize}
\item {Utilização:Gír de rapazes.}
\end{itemize}
\begin{itemize}
\item {Proveniência:(De \textunderscore sapo\textunderscore )}
\end{itemize}
Aphtas na bôca das crianças de leite.
Inflammação aos lados do freio da língua, nos cavallos.
Saliência carnosa, na língua dos cavallos.
Passarinhos recém-nascidos trocando-se-lhes assim o nome, para que as formigas os não vão comer.
\section{Sapiquá}
\begin{itemize}
\item {Grp. gram.:m.}
\end{itemize}
\begin{itemize}
\item {Utilização:Bras}
\end{itemize}
O mesmo que \textunderscore picoá\textunderscore .
\section{Sapiranga}
\begin{itemize}
\item {Grp. gram.:f.}
\end{itemize}
\begin{itemize}
\item {Utilização:Bras}
\end{itemize}
\begin{itemize}
\item {Proveniência:(T. tupi)}
\end{itemize}
Inflammação das pálpebras, produzida pela presença de um parasito que faz cair as pestanas.
\section{Sapiroca}
\begin{itemize}
\item {Grp. gram.:f.}
\end{itemize}
\begin{itemize}
\item {Utilização:Bras. do Rio}
\end{itemize}
O mesmo que \textunderscore sapiranga\textunderscore .
\section{Sapitaca}
\begin{itemize}
\item {Grp. gram.:f.}
\end{itemize}
\begin{itemize}
\item {Utilização:Bras}
\end{itemize}
\begin{itemize}
\item {Utilização:Fig.}
\end{itemize}
O mesmo que \textunderscore ran\textunderscore .
Mulhér, que se saracoteia; sirigaita.
\section{Sapitica}
\begin{itemize}
\item {Grp. gram.:f.}
\end{itemize}
\begin{itemize}
\item {Utilização:Bras}
\end{itemize}
Nome de um pássaro.
\section{Sapo}
\begin{itemize}
\item {Grp. gram.:m.}
\end{itemize}
\begin{itemize}
\item {Utilização:Açor}
\end{itemize}
\begin{itemize}
\item {Utilização:Bras}
\end{itemize}
\begin{itemize}
\item {Utilização:Chul.}
\end{itemize}
\begin{itemize}
\item {Proveniência:(Do lat. \textunderscore seps\textunderscore , \textunderscore sepis\textunderscore )}
\end{itemize}
Espécie de batrácio ranídeo, (\textunderscore rana bufo\textunderscore ).
Pequeno peixe que, soprando-se-lhe na boca, ostenta um grande papo, que o obriga a boiar, de ventre para cima.
Fiscal de bonde.
\section{Sapo-aranzeiro}
\begin{itemize}
\item {Grp. gram.:m.}
\end{itemize}
Espécie de sapo amphíbio, (\textunderscore bufo calamita\textunderscore ). Cf. P. Moraes. \textunderscore Zool. Elem.\textunderscore , 458.
\section{Sapo-cachorro}
\begin{itemize}
\item {Grp. gram.:m.}
\end{itemize}
\begin{itemize}
\item {Utilização:Bras. de Minas}
\end{itemize}
Amphíbio, que coaxa, imitando o latir dos cães.
\section{Sapocado}
\begin{itemize}
\item {Grp. gram.:adj.}
\end{itemize}
\begin{itemize}
\item {Utilização:Bras}
\end{itemize}
Diz-se dos olhos salientes, meio fóra das órbitas.
\section{Sapo-concho}
\begin{itemize}
\item {Grp. gram.:m.}
\end{itemize}
\begin{itemize}
\item {Utilização:Pop.}
\end{itemize}
O mesmo que \textunderscore gyrino\textunderscore ; cágado.
\section{Sapo-porteiro}
\begin{itemize}
\item {Grp. gram.:m.}
\end{itemize}
Espécie de sapo (\textunderscore alytes obstetricans\textunderscore ). Cf. P. Moraes, \textunderscore Zool. Elem.\textunderscore , 450.
\section{Sapoila}
\begin{itemize}
\item {Grp. gram.:adj.}
\end{itemize}
\begin{itemize}
\item {Utilização:Prov.}
\end{itemize}
\begin{itemize}
\item {Utilização:trasm.}
\end{itemize}
\begin{itemize}
\item {Proveniência:(De \textunderscore sapo\textunderscore )}
\end{itemize}
Vagaroso; indolente.
\section{Sapoilar}
\begin{itemize}
\item {Grp. gram.:v. i.}
\end{itemize}
\begin{itemize}
\item {Utilização:Prov.}
\end{itemize}
\begin{itemize}
\item {Utilização:Trasm.}
\end{itemize}
\begin{itemize}
\item {Proveniência:(De \textunderscore sapoila\textunderscore )}
\end{itemize}
Andar devagar.
\section{Sapo-leve}
\begin{itemize}
\item {Grp. gram.:m.}
\end{itemize}
\begin{itemize}
\item {Utilização:Prov.}
\end{itemize}
\begin{itemize}
\item {Utilização:minh.}
\end{itemize}
O mesmo que \textunderscore milhafre\textunderscore , segundo informações seguras.
Entretanto, cp. \textunderscore sape-leve\textunderscore .
\section{Sapolga}
\begin{itemize}
\item {Grp. gram.:adj.}
\end{itemize}
\begin{itemize}
\item {Utilização:Prov.}
\end{itemize}
\begin{itemize}
\item {Utilização:alg.}
\end{itemize}
\begin{itemize}
\item {Proveniência:(De \textunderscore sapo\textunderscore )}
\end{itemize}
O mesmo que \textunderscore obeso\textunderscore .
\section{Sapolina}
\begin{itemize}
\item {Grp. gram.:f.}
\end{itemize}
Nome, que alguns mineralogistas dão ao ácido bórico.
\section{Saponáceas}
\begin{itemize}
\item {Grp. gram.:f. pl.}
\end{itemize}
\begin{itemize}
\item {Utilização:Bot.}
\end{itemize}
\begin{itemize}
\item {Proveniência:(De \textunderscore saponáceo\textunderscore )}
\end{itemize}
Família de plantas, constituída por alguns botânicos á custa das caryophylláceas, e que tem por typo a saponária.
O mesmo que \textunderscore sapindáceas\textunderscore .
\section{Saponáceo}
\begin{itemize}
\item {Grp. gram.:adj.}
\end{itemize}
\begin{itemize}
\item {Proveniência:(Do lat. \textunderscore sapo\textunderscore , \textunderscore saponis\textunderscore )}
\end{itemize}
Que tem a natureza do sabão.
Que se póde empregar como sabão.
\section{Saponária}
\begin{itemize}
\item {Grp. gram.:f.}
\end{itemize}
\begin{itemize}
\item {Proveniência:(De \textunderscore saponário\textunderscore )}
\end{itemize}
Gênero de plantas caryophylláceas, também conhecido por \textunderscore saboeira legítima\textunderscore .
\section{Saponarina}
\begin{itemize}
\item {Grp. gram.:f.}
\end{itemize}
Substância crystallizável, que se achou numa espécie de saponária.
\section{Saponário}
\begin{itemize}
\item {Grp. gram.:adj.}
\end{itemize}
\begin{itemize}
\item {Proveniência:(Lat. \textunderscore saponarius\textunderscore )}
\end{itemize}
Que tem sabão, (falando-se de alguns medicamentos).
\section{Sapónase}
\begin{itemize}
\item {Grp. gram.:f.}
\end{itemize}
\begin{itemize}
\item {Proveniência:(Do lat. \textunderscore sapo\textunderscore , \textunderscore saponis\textunderscore )}
\end{itemize}
O mesmo que \textunderscore lípase\textunderscore .
\section{Saponificação}
\begin{itemize}
\item {Grp. gram.:f.}
\end{itemize}
Acto ou effeito de saponificar.
\section{Saponificar}
\begin{itemize}
\item {Grp. gram.:v. t.}
\end{itemize}
\begin{itemize}
\item {Proveniência:(Do lat. \textunderscore sapo\textunderscore , \textunderscore saponis\textunderscore  + \textunderscore facere\textunderscore )}
\end{itemize}
Transformar em sabão.
\section{Saponificável}
\begin{itemize}
\item {Grp. gram.:adj.}
\end{itemize}
Que se póde saponificar.
\section{Saponiforme}
\begin{itemize}
\item {Grp. gram.:adj.}
\end{itemize}
\begin{itemize}
\item {Proveniência:(Do lat. \textunderscore sapo\textunderscore  + \textunderscore forma\textunderscore )}
\end{itemize}
Que tem o aspecto de sabão.
\section{Saponina}
\begin{itemize}
\item {Grp. gram.:f.}
\end{itemize}
\begin{itemize}
\item {Proveniência:(Do lat. \textunderscore sapo\textunderscore , \textunderscore saponis\textunderscore )}
\end{itemize}
Princípio immediato, extrahido da raíz da saponária.
\section{Saponito}
\begin{itemize}
\item {Grp. gram.:m.}
\end{itemize}
\begin{itemize}
\item {Utilização:Miner.}
\end{itemize}
\begin{itemize}
\item {Proveniência:(Do lat. \textunderscore sapo\textunderscore )}
\end{itemize}
Silicato de alumina e magnésia, de côr pardacenta ou esbranquiçada, untuoso como o sabão.
\section{Sapopema}
\begin{itemize}
\item {Grp. gram.:f.}
\end{itemize}
\begin{itemize}
\item {Utilização:Bras}
\end{itemize}
Raízes que se desenvolvem com o tronco de muitas árvores, formando em volta delle divisões achatadas.
(Cp. \textunderscore sapupema\textunderscore )
\section{Sapopemba}
\begin{itemize}
\item {Grp. gram.:f.}
\end{itemize}
O mesmo que \textunderscore sapopema\textunderscore .
\section{Sapopés}
\begin{itemize}
\item {Grp. gram.:m. pl.}
\end{itemize}
\begin{itemize}
\item {Utilização:Bras}
\end{itemize}
Tribo de indígenas do Pará.
\section{Sapoquena}
\begin{itemize}
\item {Grp. gram.:f.}
\end{itemize}
(V.sapucairana)
\section{Saporífero}
\begin{itemize}
\item {Grp. gram.:adj.}
\end{itemize}
\begin{itemize}
\item {Proveniência:(Do lat. \textunderscore sapor\textunderscore  + \textunderscore ferre\textunderscore )}
\end{itemize}
Que tem sabor.
\section{Saporífico}
\begin{itemize}
\item {Grp. gram.:adj.}
\end{itemize}
\begin{itemize}
\item {Proveniência:(Do lat. \textunderscore sapor\textunderscore  + \textunderscore facere\textunderscore )}
\end{itemize}
O mesmo que \textunderscore saporífero\textunderscore .
\section{Sapota}
\begin{itemize}
\item {Grp. gram.:f.}
\end{itemize}
Gênero de árvores americanas, lactescentes e, algumas vezes, espinhosas.
Árvore chenopodácea.
\section{Sapota-açu}
\begin{itemize}
\item {Grp. gram.:m.}
\end{itemize}
Planta sapotácea do Brasil.
\section{Sapotáceas}
\begin{itemize}
\item {Grp. gram.:f.}
\end{itemize}
Família de plantas, que tem por typo a sapota.
(Fem. pl. de \textunderscore sapotáceo\textunderscore )
\section{Sapotáceo}
\begin{itemize}
\item {Grp. gram.:adj.}
\end{itemize}
Relativo ou semelhante á sapota.
\section{Sapote}
\begin{itemize}
\item {Grp. gram.:m.}
\end{itemize}
O mesmo que \textunderscore sapota\textunderscore .
\section{Sapóteas}
\begin{itemize}
\item {Grp. gram.:f. pl.}
\end{itemize}
(V.sapotáceas)
\section{Sapote-grande}
\begin{itemize}
\item {Grp. gram.:m.}
\end{itemize}
O mesmo que \textunderscore uiqué\textunderscore .
\section{Sapoti}
\begin{itemize}
\item {Grp. gram.:m.}
\end{itemize}
O mesmo que \textunderscore sapota\textunderscore .--Os diccion. dizem \textunderscore sapóti\textunderscore , o que talvez seja êrro. Cf. G. Amorim, \textunderscore Cantos Matut.\textunderscore , 83.
\section{Sapotilha}
\begin{itemize}
\item {Grp. gram.:f.}
\end{itemize}
Árvore sapotácea da Índia Portuguesa, o mesmo que \textunderscore sapoti\textunderscore .
\section{Sapotizeiro}
\begin{itemize}
\item {Grp. gram.:m.}
\end{itemize}
\begin{itemize}
\item {Utilização:Bras}
\end{itemize}
O mesmo que \textunderscore sapota\textunderscore .
\section{Saprecar}
\begin{itemize}
\item {Grp. gram.:v. t.}
\end{itemize}
\begin{itemize}
\item {Utilização:Bras}
\end{itemize}
O mesmo que \textunderscore sapecar\textunderscore ^2.
\section{Saprino}
\begin{itemize}
\item {Grp. gram.:m.}
\end{itemize}
Gênero de insectos coleópteros pentâmeros, clavicórneos.
\section{Saprófago}
\begin{itemize}
\item {Grp. gram.:adj.}
\end{itemize}
\begin{itemize}
\item {Grp. gram.:M. pl.}
\end{itemize}
Que se alimenta de coisas putrefactas.
Gênero de insectos coleópteros pentâmeros, clavicórneos.
\section{Saprolégnia}
\begin{itemize}
\item {Grp. gram.:f.}
\end{itemize}
\begin{itemize}
\item {Proveniência:(Do gr. \textunderscore sapros\textunderscore  + \textunderscore legnon\textunderscore )}
\end{itemize}
Gênero de plantas, que crescem nos corpos animaes e vegetaes, submergidos na água e em princípio de decomposição.
\section{Sapróphago}
\begin{itemize}
\item {Grp. gram.:adj.}
\end{itemize}
\begin{itemize}
\item {Grp. gram.:M. pl.}
\end{itemize}
Que se alimenta de coisas putrefactas.
Gênero de insectos coleópteros pentâmeros, clavicórneos.
\section{Saprófilo}
\begin{itemize}
\item {Grp. gram.:adj.}
\end{itemize}
\begin{itemize}
\item {Proveniência:(Do gr. \textunderscore sapros\textunderscore  + \textunderscore philos\textunderscore )}
\end{itemize}
Amigo da podridão.
\section{Saprófita}
\begin{itemize}
\item {Grp. gram.:f.}
\end{itemize}
O mesmo que \textunderscore saprófito\textunderscore .
\section{Saprófito}
\begin{itemize}
\item {Grp. gram.:m.}
\end{itemize}
\begin{itemize}
\item {Proveniência:(Do gr. \textunderscore sapros\textunderscore  + \textunderscore phuton\textunderscore )}
\end{itemize}
Micróbio, que vive de substâncias mortas ou putrefactas.
\section{Sapróphylo}
\begin{itemize}
\item {Grp. gram.:adj.}
\end{itemize}
\begin{itemize}
\item {Proveniência:(Do gr. \textunderscore sapros\textunderscore  + \textunderscore philos\textunderscore )}
\end{itemize}
Amigo da podridão.
\section{Sapróphyta}
\begin{itemize}
\item {Grp. gram.:f.}
\end{itemize}
O mesmo que \textunderscore sapróphyto\textunderscore .
\section{Sapróphyto}
\begin{itemize}
\item {Grp. gram.:m.}
\end{itemize}
\begin{itemize}
\item {Proveniência:(Do gr. \textunderscore sapros\textunderscore  + \textunderscore phuton\textunderscore )}
\end{itemize}
Micróbio, que vive de substâncias mortas ou putrefactas.
\section{Saprosma}
\begin{itemize}
\item {Grp. gram.:f.}
\end{itemize}
Gênero de plantas rubiáceas.
\section{Sapu}
\begin{itemize}
\item {Grp. gram.:m.}
\end{itemize}
Pássaro conirostro do Brasil.
\section{Sapu}
\begin{itemize}
\item {Grp. gram.:m.}
\end{itemize}
Fruto indiano.
\section{Sapucaeira}
\begin{itemize}
\item {Grp. gram.:f.}
\end{itemize}
O mesmo que \textunderscore sapucaia\textunderscore .
\section{Sapucaeiro}
\begin{itemize}
\item {Grp. gram.:m.}
\end{itemize}
O mesmo que \textunderscore sapucaia-mirim\textunderscore .
\section{Sapucaia}
\begin{itemize}
\item {Grp. gram.:f.}
\end{itemize}
Nome de várias árvores myrtáceas do Brasil.--Alguns confundem a sapucaia com a sapota. Cf. Valdez, \textunderscore Diccion. Esp.\textunderscore , vb. \textunderscore sapote\textunderscore 
\section{Sapucaia-mirim}
\begin{itemize}
\item {Grp. gram.:f.}
\end{itemize}
Árvore myrtácea, (\textunderscore lexythis minor\textunderscore ).
\section{Sapucairana}
\begin{itemize}
\item {Grp. gram.:f.}
\end{itemize}
Árvore myrtácea do Brasil.
\section{Sapudo}
\begin{itemize}
\item {Grp. gram.:adj.}
\end{itemize}
\begin{itemize}
\item {Utilização:Pop.}
\end{itemize}
\begin{itemize}
\item {Proveniência:(De \textunderscore sapo\textunderscore )}
\end{itemize}
Atarracado; grosso e baixo, (falando-se de alguém). Cf. Eça, \textunderscore P. Basílio\textunderscore , 174.
Grosseiro e gordo: \textunderscore mãos sapudas\textunderscore .
\section{Sapujuba}
\begin{itemize}
\item {Grp. gram.:m.}
\end{itemize}
O mesmo que \textunderscore sapu\textunderscore ^2.
\section{Sapupema}
\begin{itemize}
\item {Grp. gram.:m.}
\end{itemize}
\begin{itemize}
\item {Utilização:Bras. do N}
\end{itemize}
\begin{itemize}
\item {Proveniência:(Do guar. \textunderscore sapú\textunderscore  + tupi \textunderscore pema\textunderscore )}
\end{itemize}
O mesmo ou melhor que \textunderscore sapopema\textunderscore .
\section{Saputá}
\begin{itemize}
\item {Grp. gram.:m.}
\end{itemize}
Árvore rhizobolácea do Brasil.
\section{Saputi}
\begin{itemize}
\item {Grp. gram.:m.}
\end{itemize}
\begin{itemize}
\item {Utilização:Bras}
\end{itemize}
Fruta do saputizeiro; o mesmo que \textunderscore saputizeiro\textunderscore .--É o mesmo que \textunderscore sapoti\textunderscore ?
\section{Saputiaba}
\begin{itemize}
\item {Grp. gram.:f.}
\end{itemize}
\begin{itemize}
\item {Utilização:Bras}
\end{itemize}
Árvore silvestre.
\section{Saputizeiro}
\begin{itemize}
\item {Grp. gram.:m.}
\end{itemize}
\begin{itemize}
\item {Utilização:Bras}
\end{itemize}
\begin{itemize}
\item {Proveniência:(De \textunderscore saputi\textunderscore )}
\end{itemize}
Árvore sapotácea.
\section{Saquarema}
\begin{itemize}
\item {Grp. gram.:m.}
\end{itemize}
\begin{itemize}
\item {Utilização:Bras}
\end{itemize}
Designação do sectário do partido conservador:«\textunderscore saquarema é o que êlle gostava de sêr chamado\textunderscore ». M. Assis, \textunderscore Hist. sem Data\textunderscore , 73.
\section{Saqué}
\begin{itemize}
\item {fónica:qu-é}
\end{itemize}
\begin{itemize}
\item {Grp. gram.:f.}
\end{itemize}
\begin{itemize}
\item {Utilização:Bras}
\end{itemize}
Conquém, pintada, gallinha-da-índia.
\section{Saqué}
\begin{itemize}
\item {Grp. gram.:m.}
\end{itemize}
Vinho de arroz, transparente e capitoso, usado no Japão. Cf. V. Moraes, \textunderscore Dai-Nippon\textunderscore , 286.
\section{Saque}
\begin{itemize}
\item {Grp. gram.:m.}
\end{itemize}
\begin{itemize}
\item {Utilização:Prov.}
\end{itemize}
\begin{itemize}
\item {Utilização:trasm.}
\end{itemize}
Acto ou effeito de sacar.
Letra de câmbio, que se sacou.
Partida do jôgo da pela.
Passagem da mão do mesmo jôgo para novos parceiros.
\section{Saque}
\begin{itemize}
\item {Grp. gram.:m.}
\end{itemize}
Acto ou effeito de saquear.
\section{Saqueador}
\begin{itemize}
\item {Grp. gram.:m.  e  adj.}
\end{itemize}
O que saqueia.
\section{Saquear}
\begin{itemize}
\item {Grp. gram.:v. t.}
\end{itemize}
\begin{itemize}
\item {Proveniência:(De \textunderscore saco\textunderscore ^2)}
\end{itemize}
Despojar violentamente; roubar; assolar.
\section{Saqueio}
\begin{itemize}
\item {Grp. gram.:m.}
\end{itemize}
O mesmo que \textunderscore saque\textunderscore ^2.
\section{Saquetaria}
\begin{itemize}
\item {Grp. gram.:f.}
\end{itemize}
\begin{itemize}
\item {Utilização:Ant.}
\end{itemize}
O mesmo que \textunderscore saquitaria\textunderscore .
\section{Saquete}
\begin{itemize}
\item {fónica:quê}
\end{itemize}
\begin{itemize}
\item {Grp. gram.:m.}
\end{itemize}
Pequeno saco.
\section{Saqui}
\begin{itemize}
\item {Grp. gram.:m.}
\end{itemize}
Mammífero quadrumano, de longa cauda, do Brasil e da Guiana.
\section{Saquilada}
\begin{itemize}
\item {Grp. gram.:f.}
\end{itemize}
\begin{itemize}
\item {Utilização:Des.}
\end{itemize}
\begin{itemize}
\item {Proveniência:(De \textunderscore saco\textunderscore ^1?)}
\end{itemize}
Colheita do trigo.
\section{Saquilhão}
\begin{itemize}
\item {Grp. gram.:m.}
\end{itemize}
\begin{itemize}
\item {Proveniência:(De \textunderscore sacar\textunderscore ?)}
\end{itemize}
Ramo, ligado ás aivecas do arado, e que torna mais largo o rêgo, em que se há de plantar bacello.
\section{Saquim}
\begin{itemize}
\item {Grp. gram.:m.}
\end{itemize}
Cutello, com que os Judeus abatem as grandes reses.
\section{Saquinho}
\begin{itemize}
\item {Grp. gram.:m.}
\end{itemize}
Pequeno saco.
Cartuxo de pólvora, com que se carregam as peças de artilharia.
\section{Saquista}
\begin{itemize}
\item {Grp. gram.:m.}
\end{itemize}
\begin{itemize}
\item {Utilização:Bras}
\end{itemize}
Aquelle que faz sacos para café em grão.
\section{Saquitaria}
\begin{itemize}
\item {Grp. gram.:f.}
\end{itemize}
\begin{itemize}
\item {Utilização:Ant.}
\end{itemize}
\begin{itemize}
\item {Proveniência:(De \textunderscore saco\textunderscore )}
\end{itemize}
Depósito de pão cozido, na casa real.
Cargo de saquiteiro.
\section{Saquitário}
\begin{itemize}
\item {Grp. gram.:m.}
\end{itemize}
\begin{itemize}
\item {Utilização:Ant.}
\end{itemize}
O mesmo que \textunderscore saquiteiro\textunderscore .
\section{Saquiteiro}
\begin{itemize}
\item {Grp. gram.:m.}
\end{itemize}
\begin{itemize}
\item {Utilização:Ant.}
\end{itemize}
\begin{itemize}
\item {Proveniência:(De \textunderscore saco\textunderscore ^1)}
\end{itemize}
Aquelle que tinha a seu cargo o pão cozido para a mesa do Rei.
\section{Saquitel}
\begin{itemize}
\item {Grp. gram.:m.}
\end{itemize}
O mesmo que \textunderscore saquinho\textunderscore .
\section{Saquito}
\begin{itemize}
\item {Grp. gram.:m.}
\end{itemize}
Pequeno saco.
\section{Sarabacana}
\begin{itemize}
\item {Grp. gram.:f.}
\end{itemize}
Projéctil africano?:«\textunderscore Roosovelt, atravessando um leão com uma sarabacana...\textunderscore »Rev. \textunderscore Serões\textunderscore , XLVIII.
\section{Sarabádi}
\begin{itemize}
\item {Grp. gram.:m.}
\end{itemize}
Planta medicinal da Guiana inglesa.
\section{Sarabaiara}
\begin{itemize}
\item {Grp. gram.:f.}
\end{itemize}
Nardo silvestre. Cf. B. Pereira, \textunderscore Prosódia\textunderscore , vb. \textunderscore vulgago\textunderscore .
\section{Sarabanda}
\begin{itemize}
\item {Grp. gram.:f.}
\end{itemize}
\begin{itemize}
\item {Utilização:Pop.}
\end{itemize}
Dança antiga, popular e desenvolta.
Censura, reprehensão.
(Cast. \textunderscore zarabanda\textunderscore )
\section{Sarabandear}
\begin{itemize}
\item {Grp. gram.:v. i.}
\end{itemize}
\begin{itemize}
\item {Grp. gram.:V. i.}
\end{itemize}
Dançar a sarabanda.
Dançar.
\section{Sarabaritas}
\begin{itemize}
\item {Grp. gram.:m. pl.}
\end{itemize}
\begin{itemize}
\item {Proveniência:(Do lat. \textunderscore sarabara\textunderscore )}
\end{itemize}
Falsos apóstolos, que appareceram no Egypto, logo depois da morte dos verdadeiros apóstolos, e que, a pretexto de observarem a lei, não obedeciam ás autoridades da Igreja.
\section{Sarabatana}
\begin{itemize}
\item {Grp. gram.:f.}
\end{itemize}
O mesmo que \textunderscore buzina\textunderscore .
Instrumento guerreiro, fabricado por mulheres do alto Amazonas.
\section{Sarabulhento}
\begin{itemize}
\item {Grp. gram.:m.}
\end{itemize}
\begin{itemize}
\item {Utilização:Pop.}
\end{itemize}
Que tem sarabulhos.
Que tem bostelas; ulceroso.
\section{Sarabulho}
\begin{itemize}
\item {Grp. gram.:m.}
\end{itemize}
\begin{itemize}
\item {Utilização:Pop.}
\end{itemize}
Asperezas na superfície da loiça.
Bostela.
\section{Sarabulhoso}
\begin{itemize}
\item {Grp. gram.:adj.}
\end{itemize}
O mesmo que \textunderscore sarabulhento\textunderscore .
\section{Saraça}
\begin{itemize}
\item {Grp. gram.:f.}
\end{itemize}
\begin{itemize}
\item {Utilização:T. da Índia port}
\end{itemize}
Tecido fino de algodão.
O mesmo que \textunderscore cobertor\textunderscore .
(Cast. \textunderscore zaraza\textunderscore )
\section{Saraça}
\begin{itemize}
\item {Grp. gram.:m.}
\end{itemize}
\begin{itemize}
\item {Utilização:Prov.}
\end{itemize}
\begin{itemize}
\item {Utilização:trasm.}
\end{itemize}
Homem trapalhão; homem desajeitado.
(Cp. \textunderscore zaranza\textunderscore )
\section{Saraças}
\begin{itemize}
\item {Grp. gram.:f. pl.}
\end{itemize}
\begin{itemize}
\item {Utilização:Prov.}
\end{itemize}
\begin{itemize}
\item {Utilização:trasm.}
\end{itemize}
\begin{itemize}
\item {Proveniência:(De \textunderscore sarar\textunderscore )}
\end{itemize}
Amavios, mèzinhas, que as mulheres dão aos homens para que as amem.
\section{Saracenária}
\begin{itemize}
\item {Grp. gram.:f.}
\end{itemize}
Gênero de polypeiros fósseis.
\section{Saraço}
\begin{itemize}
\item {Grp. gram.:m.}
\end{itemize}
\begin{itemize}
\item {Utilização:Prov.}
\end{itemize}
\begin{itemize}
\item {Utilização:trasm.}
\end{itemize}
O mesmo que \textunderscore rato\textunderscore ^1.
\section{Saracote}
\begin{itemize}
\item {Grp. gram.:m.}
\end{itemize}
O mesmo que \textunderscore saracoteio\textunderscore .
\section{Saracoteador}
\begin{itemize}
\item {Grp. gram.:m.  e  adj.}
\end{itemize}
O que saracoteia.
\section{Saracotear}
\begin{itemize}
\item {Grp. gram.:v. t.}
\end{itemize}
\begin{itemize}
\item {Grp. gram.:V. i.}
\end{itemize}
\begin{itemize}
\item {Grp. gram.:V. p.}
\end{itemize}
Mover com desenvoltura e graça (o corpo, os braços, os quadris).
Vaguear por um lugar e por outro.
Andar numa roda viva.
Vagabundear.
Fazer meneios graciosos e desenvoltos.
Agitar-se desenvoltamente.
\section{Saracoteio}
\begin{itemize}
\item {Grp. gram.:m.}
\end{itemize}
Acto ou effeito de saracotear.
\section{Saracoto}
\begin{itemize}
\item {fónica:cô}
\end{itemize}
\begin{itemize}
\item {Grp. gram.:m.}
\end{itemize}
\begin{itemize}
\item {Utilização:T. da Bairrada}
\end{itemize}
Rabo curto de animal.
(Outra fórma de \textunderscore seracoto\textunderscore )
\section{Sàracura}
\begin{itemize}
\item {Grp. gram.:f.}
\end{itemize}
Planta bignoniácea do Brasil.
Planta onagrária, (\textunderscore jussioea angulata\textunderscore ).
Ave aquática brasileira, (\textunderscore aramites\textunderscore ); gallinha de água.
\section{Sarafulha}
\begin{itemize}
\item {Grp. gram.:f.}
\end{itemize}
\begin{itemize}
\item {Utilização:Prov.}
\end{itemize}
\begin{itemize}
\item {Utilização:minh.}
\end{itemize}
Rama de pinheiro; caruma.
\section{Saragaça}
\begin{itemize}
\item {Grp. gram.:f.}
\end{itemize}
O mesmo que \textunderscore sargaço\textunderscore . Cf. \textunderscore Pharmacopeia Port.\textunderscore 
\section{Saragata}
\begin{itemize}
\item {Grp. gram.:f.}
\end{itemize}
\begin{itemize}
\item {Utilização:Prov.}
\end{itemize}
O mesmo que \textunderscore zaragata\textunderscore .
\section{Sarago}
\begin{itemize}
\item {Grp. gram.:m.}
\end{itemize}
Gênero de insectos coleópteros heterómeros.
\section{Saragoça}
\begin{itemize}
\item {fónica:gô}
\end{itemize}
\begin{itemize}
\item {Grp. gram.:f.}
\end{itemize}
\begin{itemize}
\item {Proveniência:(De \textunderscore Zaragoza\textunderscore , n. p.)}
\end{itemize}
Tecido grosso de lan escura.
O mesmo que \textunderscore mandrião\textunderscore , ave.
\section{Saragoçana}
\begin{itemize}
\item {Grp. gram.:f.}
\end{itemize}
\begin{itemize}
\item {Proveniência:(De \textunderscore saragoçano\textunderscore )}
\end{itemize}
Espécie de ameixa comprida, escura e saborosa.
\section{Saragoçano}
\begin{itemize}
\item {Grp. gram.:adj.}
\end{itemize}
\begin{itemize}
\item {Grp. gram.:M.}
\end{itemize}
Relativo a Saragoça.
Habitante de Saragoça.
\section{Saraiva}
\begin{itemize}
\item {Grp. gram.:f.}
\end{itemize}
\begin{itemize}
\item {Utilização:Fig.}
\end{itemize}
Chuva de pedra; granizo.
Grande quantidade de coisas, que caem como saraiva ou que se succedem com rapidez: \textunderscore saraiva de peloiros\textunderscore .
\section{Saraivada}
\begin{itemize}
\item {Grp. gram.:f.}
\end{itemize}
\begin{itemize}
\item {Utilização:Fig.}
\end{itemize}
Saraiva.
Chuva abundante de pedra.
Descarga.
\section{Saraivar}
\begin{itemize}
\item {Grp. gram.:v. i.}
\end{itemize}
\begin{itemize}
\item {Grp. gram.:V. t.}
\end{itemize}
Cair saraiva.
Bater ou açoitar com saraiva ou gêlo.
\section{Saraiveiro}
\begin{itemize}
\item {Grp. gram.:m.}
\end{itemize}
\begin{itemize}
\item {Utilização:Prov.}
\end{itemize}
O mesmo que \textunderscore saraivada\textunderscore .
\section{Saraivisco}
\begin{itemize}
\item {Grp. gram.:m.}
\end{itemize}
\begin{itemize}
\item {Utilização:Prov.}
\end{itemize}
Saraivada de pedrisco miúdo e passageiro.
\section{Saramago}
\begin{itemize}
\item {Grp. gram.:m.}
\end{itemize}
\begin{itemize}
\item {Proveniência:(Do lat. \textunderscore siser\textunderscore  + \textunderscore amaricum\textunderscore )}
\end{itemize}
Planta crucífera e rasteira, que é comestível e cresce sem cultura.
\section{Saramântiga}
\begin{itemize}
\item {Grp. gram.:f.}
\end{itemize}
\begin{itemize}
\item {Utilização:Pop.}
\end{itemize}
O mesmo que \textunderscore salamandra\textunderscore .
\section{Saramátulo}
\begin{itemize}
\item {Grp. gram.:m.}
\end{itemize}
Cada um dos chifres ainda tenros do veado.
\section{Saramba}
\begin{itemize}
\item {Grp. gram.:f.}
\end{itemize}
\begin{itemize}
\item {Utilização:Bras. do S}
\end{itemize}
Espécie de fandango.
\section{Sarambeque}
\begin{itemize}
\item {Grp. gram.:m.}
\end{itemize}
\begin{itemize}
\item {Proveniência:(De \textunderscore saramba\textunderscore )}
\end{itemize}
Dança desenvolta de pretos. Cf. Camillo, \textunderscore Caveira\textunderscore , 221 e 401.
\section{Sarâmbia}
\begin{itemize}
\item {Grp. gram.:f.}
\end{itemize}
\begin{itemize}
\item {Utilização:Gír.}
\end{itemize}
Acto de masturbar-se alguém.
\section{Sarambura}
\begin{itemize}
\item {Grp. gram.:f.}
\end{itemize}
Tecido de algodão, fabricado em Bengala.
\section{Saramela}
\begin{itemize}
\item {Grp. gram.:f.}
\end{itemize}
\begin{itemize}
\item {Utilização:Prov.}
\end{itemize}
\begin{itemize}
\item {Utilização:minh.}
\end{itemize}
O mesmo que \textunderscore salamandra\textunderscore .
\section{Saramenheira}
\begin{itemize}
\item {Grp. gram.:f.}
\end{itemize}
\begin{itemize}
\item {Proveniência:(De \textunderscore saramenho\textunderscore )}
\end{itemize}
Espécie de pereira.
\section{Saramenheiro}
\begin{itemize}
\item {Grp. gram.:m.}
\end{itemize}
O mesmo que \textunderscore saramenheira\textunderscore .
\section{Saramenho}
\begin{itemize}
\item {Grp. gram.:m.}
\end{itemize}
Espécie de pêra miúda.
\section{Saramiques}
\begin{itemize}
\item {Grp. gram.:m.}
\end{itemize}
\begin{itemize}
\item {Utilização:Bras}
\end{itemize}
Grande cobra, das regiões do Amazonas.
\section{Saramona}
\begin{itemize}
\item {Grp. gram.:f.}
\end{itemize}
Rêde de pesca, no Doiro.
\section{Sarampão}
\begin{itemize}
\item {Grp. gram.:m.}
\end{itemize}
\begin{itemize}
\item {Utilização:Pop.}
\end{itemize}
Ataque de sarampo.
(Cp. cast. \textunderscore sarampión\textunderscore )
\section{Sarampelo}
\begin{itemize}
\item {fónica:pê}
\end{itemize}
\begin{itemize}
\item {Grp. gram.:m.}
\end{itemize}
\begin{itemize}
\item {Utilização:Pop.}
\end{itemize}
Sarampo benigno.
\section{Sarampo}
\begin{itemize}
\item {Grp. gram.:m.}
\end{itemize}
\begin{itemize}
\item {Proveniência:(De \textunderscore sarampão\textunderscore , como se êste fôsse derivado de \textunderscore sarampo\textunderscore )}
\end{itemize}
Doença febril e contagiosa, caracterizada por pintas vermelhas na pelle.
\section{Saramuga}
\begin{itemize}
\item {Grp. gram.:f.}
\end{itemize}
\begin{itemize}
\item {Utilização:Prov.}
\end{itemize}
\begin{itemize}
\item {Utilização:minh.}
\end{itemize}
O mesmo que \textunderscore faúlha\textunderscore .
\section{Saramugo}
\begin{itemize}
\item {Grp. gram.:m.}
\end{itemize}
Peixe do Tejo.
\section{Saran}
\begin{itemize}
\item {Grp. gram.:m.}
\end{itemize}
\begin{itemize}
\item {Utilização:Bras}
\end{itemize}
Arbusto, que nasce nos terrenos alagados pelas cheias.
\section{Saranda}
\begin{itemize}
\item {Grp. gram.:m.  e  adj.}
\end{itemize}
\begin{itemize}
\item {Utilização:Bras}
\end{itemize}
O mesmo que \textunderscore vadio\textunderscore .
(Por \textunderscore ciranda\textunderscore , de \textunderscore cirandar\textunderscore )
\section{Sarandagem}
\begin{itemize}
\item {Grp. gram.:f.}
\end{itemize}
\begin{itemize}
\item {Utilização:Bras}
\end{itemize}
\begin{itemize}
\item {Proveniência:(De \textunderscore saranda\textunderscore )}
\end{itemize}
Vadiagem.
\section{Sarandalhas}
\begin{itemize}
\item {Grp. gram.:f.}
\end{itemize}
\begin{itemize}
\item {Utilização:Fig.}
\end{itemize}
Maravalhas.
Restos.
Gente ordinária, ralé.
\section{Sarandear}
\begin{itemize}
\item {Grp. gram.:v. i.}
\end{itemize}
\begin{itemize}
\item {Utilização:Bras}
\end{itemize}
Saracotear-se, menear o corpo na dança.
(Cp. \textunderscore cirandar\textunderscore )
\section{Sarangosti}
\begin{itemize}
\item {Grp. gram.:m.}
\end{itemize}
Espécie de goma indiana, que se emprega, em vez de breu, na vedação das fendas dos navios.
\section{Sarangui}
\begin{itemize}
\item {Grp. gram.:m.}
\end{itemize}
Instrumento, com que os desprezados filhos das bailadeiras indianas acompanham a dança de suas mães ou irmans. Cf. Th. Ribeiro, \textunderscore Jornadas\textunderscore , II, 102.
\section{Saranzal}
\begin{itemize}
\item {Grp. gram.:m.}
\end{itemize}
\begin{itemize}
\item {Utilização:Bras}
\end{itemize}
Lugar, onde crescem sarans.
\section{Sarão}
\begin{itemize}
\item {Grp. gram.:m.}
\end{itemize}
Um dos dois panos, com que se cobrem habitualmente as mulheres indígenas de Timor.
\section{Sarapalhento}
\begin{itemize}
\item {Grp. gram.:adj.}
\end{itemize}
\begin{itemize}
\item {Utilização:Ant.}
\end{itemize}
O mesmo que \textunderscore sarabulhento\textunderscore :«\textunderscore ...no beiço sarapalhento, grosso e chato\textunderscore ». \textunderscore Anat. Joc.\textunderscore , I, 253.
\section{Sarapanel}
\begin{itemize}
\item {Grp. gram.:m.}
\end{itemize}
Arco abatido, em architectura.
\section{Sarapantão}
\begin{itemize}
\item {Grp. gram.:adj.}
\end{itemize}
\begin{itemize}
\item {Utilização:Pop.}
\end{itemize}
O mesmo que [[sarapintado|sarapintado]].
\section{Sarapantar}
\begin{itemize}
\item {Grp. gram.:v. t.}
\end{itemize}
O mesmo que \textunderscore assarapantar\textunderscore .
\section{Sarapatel}
\begin{itemize}
\item {Grp. gram.:m.}
\end{itemize}
\begin{itemize}
\item {Utilização:Prov.}
\end{itemize}
\begin{itemize}
\item {Utilização:trasm.}
\end{itemize}
Iguaria, preparada com sangue, fígado, rim, bofe e coração de porco ou carneiro, com caldo.
Confusão, balbúrdia.
(Cast. \textunderscore zarapatel\textunderscore )
\section{Sarapieira}
\begin{itemize}
\item {Grp. gram.:f.}
\end{itemize}
\begin{itemize}
\item {Utilização:Bras. de Goiás}
\end{itemize}
Asneira, tolice.
\section{Sarapinta}
\begin{itemize}
\item {Grp. gram.:f.}
\end{itemize}
\begin{itemize}
\item {Utilização:Prov.}
\end{itemize}
O mesmo que \textunderscore sarapintadela\textunderscore .
\section{Sarapintadela}
\begin{itemize}
\item {Grp. gram.:f.}
\end{itemize}
\begin{itemize}
\item {Utilização:Prov.}
\end{itemize}
Acto ou effeito de sarapintar.
\section{Sarapintar}
\begin{itemize}
\item {Grp. gram.:v. t.}
\end{itemize}
\begin{itemize}
\item {Proveniência:(De um pref. desconhecido e \textunderscore pintar\textunderscore )}
\end{itemize}
Fazer pintas variadas em.
Pintar de várias côres.
Mosquear.
\section{Sarapó}
\begin{itemize}
\item {Grp. gram.:m.}
\end{itemize}
\begin{itemize}
\item {Utilização:Bras}
\end{itemize}
O mesmo que \textunderscore beiju\textunderscore .
\section{Saraquitar}
\begin{itemize}
\item {Grp. gram.:v. i.}
\end{itemize}
\begin{itemize}
\item {Utilização:Pop.}
\end{itemize}
Andar de um lado para outro; traquinar.
(Talvez alter. de \textunderscore sacotear\textunderscore )
\section{Sarar}
\begin{itemize}
\item {Grp. gram.:v. t.}
\end{itemize}
\begin{itemize}
\item {Utilização:Fig.}
\end{itemize}
\begin{itemize}
\item {Grp. gram.:V. i.}
\end{itemize}
Dar saúde a (quem está doente).
Curar.
Sanar; emendar.
Curar-se; restabelecer a saúde.
(Alter. de \textunderscore sanar\textunderscore )
\section{Sarará}
\begin{itemize}
\item {Grp. gram.:m.  e  f.}
\end{itemize}
\begin{itemize}
\item {Utilização:Bras}
\end{itemize}
Mulato, arruivado ou aça.
\section{Sararaca}
\begin{itemize}
\item {Grp. gram.:f.}
\end{itemize}
\begin{itemize}
\item {Utilização:Bras}
\end{itemize}
Espécie de frecha, com que os selvagens matam a tartaruga e outros peixes.
\section{Sarás}
\begin{itemize}
\item {Grp. gram.:m. pl.}
\end{itemize}
Indígenas do norte do Brasil.
\section{Sarasará}
\begin{itemize}
\item {Grp. gram.:m.}
\end{itemize}
\begin{itemize}
\item {Utilização:Bras}
\end{itemize}
Espécie de formiga.
\section{Sarassará}
\begin{itemize}
\item {Grp. gram.:m.}
\end{itemize}
\begin{itemize}
\item {Utilização:Bras}
\end{itemize}
Espécie de formiga.
\section{Sarau}
\begin{itemize}
\item {Grp. gram.:m.}
\end{itemize}
Reunião festiva, de noite, em casa particular, em clube ou theatro.
Concêrto musical, de noite.
(Cast. \textunderscore sarao\textunderscore )
\section{Sarça}
\begin{itemize}
\item {Grp. gram.:f.}
\end{itemize}
O mesmo que \textunderscore silva\textunderscore .
Silvado, matagal.
(Cast. \textunderscore zarza\textunderscore )
\section{Sarça-ardente}
\begin{itemize}
\item {Grp. gram.:f.}
\end{itemize}
\begin{itemize}
\item {Utilização:Bot.}
\end{itemize}
Planta pomácea, (\textunderscore cotoneaster pyracantha\textunderscore , Spach).
\section{Sarça-de-moisés}
\begin{itemize}
\item {Grp. gram.:f.}
\end{itemize}
Designação vulgar de uma árvore pomácea, (\textunderscore crataegus piracantha\textunderscore , Lin.).
\section{Sarça-ideia}
\begin{itemize}
\item {Grp. gram.:f.}
\end{itemize}
Arbusto rosáceo, (\textunderscore rubus idaea\textunderscore , Lin.).
\section{Sarçal}
\begin{itemize}
\item {Grp. gram.:m.}
\end{itemize}
Silvado.
Lugar, onde crescem sarças.
\section{Sarçamora}
\begin{itemize}
\item {Grp. gram.:f.}
\end{itemize}
Fruta de uma espécie de amoreira silvestre do México.
\section{Sarcântemo}
\begin{itemize}
\item {Grp. gram.:m.}
\end{itemize}
\begin{itemize}
\item {Proveniência:(Do gr. \textunderscore sarx\textunderscore  + \textunderscore anthema\textunderscore )}
\end{itemize}
Gênero de plantas, da fam. das compostas.
\section{Sarcânthemo}
\begin{itemize}
\item {Grp. gram.:m.}
\end{itemize}
\begin{itemize}
\item {Proveniência:(Do gr. \textunderscore sarx\textunderscore  + \textunderscore anthema\textunderscore )}
\end{itemize}
Gênero de plantas, da fam. das compostas.
\section{Sarcantho}
\begin{itemize}
\item {Grp. gram.:m.}
\end{itemize}
\begin{itemize}
\item {Proveniência:(Do gr. \textunderscore sarx\textunderscore  + \textunderscore anthos\textunderscore )}
\end{itemize}
Gênero de orchídeas.
\section{Sarcanto}
\begin{itemize}
\item {Grp. gram.:m.}
\end{itemize}
\begin{itemize}
\item {Proveniência:(Do gr. \textunderscore sarx\textunderscore  + \textunderscore anthos\textunderscore )}
\end{itemize}
Gênero de orquídeas.
\section{Sarçaparrilha}
\begin{itemize}
\item {Grp. gram.:f.}
\end{itemize}
O mesmo ou melhor que \textunderscore salsaparrilha\textunderscore . Cp. B. Pereira, \textunderscore Prosódia\textunderscore , vb. \textunderscore smilax\textunderscore  e \textunderscore salsaparrilla\textunderscore .
\section{Sarcásmico}
\begin{itemize}
\item {Grp. gram.:adj.}
\end{itemize}
Em que há sarcasmos.
\section{Sarcasmo}
\begin{itemize}
\item {Grp. gram.:m.}
\end{itemize}
\begin{itemize}
\item {Proveniência:(Lat. \textunderscore sarcasmus\textunderscore )}
\end{itemize}
Zombaria insultante.
Ironia mordaz; escárneo.
\section{Sarcástico}
\begin{itemize}
\item {Grp. gram.:adj.}
\end{itemize}
\begin{itemize}
\item {Proveniência:(Gr. \textunderscore sarkastikos\textunderscore )}
\end{itemize}
Que envolve sarcasmo; escarnecedor.
\section{Sárcina}
\begin{itemize}
\item {Grp. gram.:f.}
\end{itemize}
\begin{itemize}
\item {Proveniência:(Lat. \textunderscore sarcina\textunderscore )}
\end{itemize}
Planta coriácea e transparente que, formada de massas cúbicas ou prismáticas, se encontra no vómito de certos enfermos, em depósitos urinários, etc. Cf. Littré e Rubin, \textunderscore Diccion. de Médecine\textunderscore .
\section{Sarcínula}
\begin{itemize}
\item {Grp. gram.:f.}
\end{itemize}
Gênero de polypeiros livres.
\section{Sarcita}
\begin{itemize}
\item {Grp. gram.:f.}
\end{itemize}
\begin{itemize}
\item {Utilização:Miner.}
\end{itemize}
\begin{itemize}
\item {Proveniência:(Do gr. \textunderscore sarx\textunderscore )}
\end{itemize}
Variedade de pedra, da côr da carne.
\section{Sarcita}
\begin{itemize}
\item {Grp. gram.:f. Loc.}
\end{itemize}
\begin{itemize}
\item {Utilização:trasm}
\end{itemize}
\textunderscore Vir da sarcita\textunderscore , vir esfomeado, trazer fóme canina.
\section{Sarcito}
\begin{itemize}
\item {Grp. gram.:m.}
\end{itemize}
O mesmo ou melhór que \textunderscore sarcita\textunderscore ^1.
\section{Sarco...}
\begin{itemize}
\item {Grp. gram.:pref.}
\end{itemize}
\begin{itemize}
\item {Proveniência:(Do gr. \textunderscore sarx\textunderscore )}
\end{itemize}
(designativo de \textunderscore carne\textunderscore  ou \textunderscore polpa\textunderscore )
\section{Sarcobase}
\begin{itemize}
\item {Grp. gram.:f.}
\end{itemize}
\begin{itemize}
\item {Utilização:Bot.}
\end{itemize}
\begin{itemize}
\item {Proveniência:(De \textunderscore sarco...\textunderscore  + \textunderscore base\textunderscore )}
\end{itemize}
Base carnuda do ovário de certas plantas.
\section{Sarcocálice}
\begin{itemize}
\item {Grp. gram.:f.}
\end{itemize}
\begin{itemize}
\item {Proveniência:(De \textunderscore sarx\textunderscore  gr. + \textunderscore cálice\textunderscore )}
\end{itemize}
Gênero de plantas leguminosas.
\section{Sarcocarpiano}
\begin{itemize}
\item {Grp. gram.:adj.}
\end{itemize}
Relativo a sarcocarpo.
\section{Sarcocarpno}
\begin{itemize}
\item {Grp. gram.:m.}
\end{itemize}
Gênero de plantas papaveráceas.
\section{Sarcocarpo}
\begin{itemize}
\item {Grp. gram.:m.}
\end{itemize}
\begin{itemize}
\item {Utilização:Bot.}
\end{itemize}
\begin{itemize}
\item {Proveniência:(De \textunderscore sarco...\textunderscore  + \textunderscore carpo\textunderscore )}
\end{itemize}
Parte do pericarpo, mais ou menos carnuda, entre a epiderme dos frutos e a membrana que está em contacto com a semente.
\section{Sarcocéfalo}
\begin{itemize}
\item {Grp. gram.:m.}
\end{itemize}
\begin{itemize}
\item {Proveniência:(Do gr. \textunderscore sarx\textunderscore , \textunderscore sarkos\textunderscore  + \textunderscore khephale\textunderscore )}
\end{itemize}
Gênero de plantas rubiáceas da África tropical.
\section{Sarcocele}
\begin{itemize}
\item {Grp. gram.:m.}
\end{itemize}
\begin{itemize}
\item {Proveniência:(Do gr. \textunderscore sarx\textunderscore  + \textunderscore kele\textunderscore )}
\end{itemize}
Tumor cistoso nos testículos.
\section{Sarcocéphalo}
\begin{itemize}
\item {Grp. gram.:m.}
\end{itemize}
\begin{itemize}
\item {Proveniência:(Do gr. \textunderscore sarx\textunderscore , \textunderscore sarkos\textunderscore  + \textunderscore khephale\textunderscore )}
\end{itemize}
Gênero de plantas rubiáceas da África tropical.
\section{Sarcococco}
\begin{itemize}
\item {Grp. gram.:m.}
\end{itemize}
\begin{itemize}
\item {Proveniência:(Do gr. \textunderscore sarx\textunderscore , \textunderscore sarkos\textunderscore  + \textunderscore kokkos\textunderscore )}
\end{itemize}
Gênero de plantas euphorbiáceas de Nepal.
\section{Sarcococo}
\begin{itemize}
\item {Grp. gram.:m.}
\end{itemize}
\begin{itemize}
\item {Proveniência:(Do gr. \textunderscore sarx\textunderscore , \textunderscore sarkos\textunderscore  + \textunderscore kokkos\textunderscore )}
\end{itemize}
Gênero de plantas euforbiáceas de Nepal.
\section{Sarcocola}
\begin{itemize}
\item {Grp. gram.:f.}
\end{itemize}
\begin{itemize}
\item {Proveniência:(De \textunderscore sarco...\textunderscore  + \textunderscore cola\textunderscore )}
\end{itemize}
Resina da sarcocoleira.
\section{Sarcocoleira}
\begin{itemize}
\item {Grp. gram.:f.}
\end{itemize}
\begin{itemize}
\item {Proveniência:(De \textunderscore sarcocola\textunderscore )}
\end{itemize}
Gênero de árvores, (\textunderscore penaca sarcocolla\textunderscore ).
\section{Sarcocolina}
\begin{itemize}
\item {Grp. gram.:f.}
\end{itemize}
Substância, não cristalizável extraida da sarcocola.
\section{Sarcocolla}
\begin{itemize}
\item {Grp. gram.:f.}
\end{itemize}
\begin{itemize}
\item {Proveniência:(De \textunderscore sarco...\textunderscore  + \textunderscore colla\textunderscore )}
\end{itemize}
Resina da sarcocolleira.
\section{Sarcocolleira}
\begin{itemize}
\item {Grp. gram.:f.}
\end{itemize}
\begin{itemize}
\item {Proveniência:(De \textunderscore sarcocolla\textunderscore )}
\end{itemize}
Gênero de árvores, (\textunderscore penaca sarcocolla\textunderscore ).
\section{Sarcocollina}
\begin{itemize}
\item {Grp. gram.:f.}
\end{itemize}
Substância, não crystallizável extrahida da sarcocolla.
\section{Sarcode}
\begin{itemize}
\item {Grp. gram.:m.}
\end{itemize}
\begin{itemize}
\item {Proveniência:(Gr. \textunderscore sarcodes\textunderscore )}
\end{itemize}
Substância animal, sem tegumentos, que constitue os infusórios, na opinião de alguns zoólogos.
\section{Sarcoderme}
\begin{itemize}
\item {Grp. gram.:m.}
\end{itemize}
\begin{itemize}
\item {Proveniência:(Do gr. \textunderscore sarx\textunderscore  + \textunderscore derma\textunderscore )}
\end{itemize}
Parênchyma, comprehendido entre as duas pellículas de uma semente.
\section{Sarcódico}
\begin{itemize}
\item {Grp. gram.:adj.}
\end{itemize}
Relativo ao sarcode ou que é da natureza delle.
\section{Sarcófago}
\begin{itemize}
\item {Grp. gram.:adj.}
\end{itemize}
\begin{itemize}
\item {Grp. gram.:M.}
\end{itemize}
\begin{itemize}
\item {Utilização:Ext.}
\end{itemize}
\begin{itemize}
\item {Proveniência:(Lat. \textunderscore sarcophagus\textunderscore )}
\end{itemize}
Que corrói as carnes.
Túmulo, em que os antigos metiam os cadáveres que não queriam queimar, e que era feito de uma variedade de pedra, a que se atribuía a propriedade de consumir as carnes.
Túmulo.
\section{Sarcofila}
\begin{itemize}
\item {Grp. gram.:f.}
\end{itemize}
\begin{itemize}
\item {Proveniência:(Do gr. \textunderscore sarx\textunderscore  + \textunderscore phullon\textunderscore )}
\end{itemize}
A parte carnuda das fôlhas.
\section{Sarcofilo}
\begin{itemize}
\item {Grp. gram.:m.}
\end{itemize}
\begin{itemize}
\item {Grp. gram.:Pl.}
\end{itemize}
Gênero de plantas leguminosas, cuja espécie típica cresce no Cabo da Bôa-Esperança.
Grupo de mamíferos marzupiaes, no sistema de Cuvier.
\section{Sarcoídeo}
\begin{itemize}
\item {Grp. gram.:adj.}
\end{itemize}
\begin{itemize}
\item {Proveniência:(Do gr. \textunderscore sarxeidos\textunderscore )}
\end{itemize}
Que tem a apparência de carne.
\section{Sarcolema}
\begin{itemize}
\item {Grp. gram.:m.}
\end{itemize}
\begin{itemize}
\item {Proveniência:(Do gr. \textunderscore sarx\textunderscore  + \textunderscore lemma\textunderscore )}
\end{itemize}
Tubo transparente, que contém as fibrilhas musculares.
\section{Sarcolemma}
\begin{itemize}
\item {Grp. gram.:m.}
\end{itemize}
\begin{itemize}
\item {Proveniência:(Do gr. \textunderscore sarx\textunderscore  + \textunderscore lemma\textunderscore )}
\end{itemize}
Tubo transparente, que contém as fibrilhas musculares.
\section{Sarcólita}
\begin{itemize}
\item {Grp. gram.:f.}
\end{itemize}
O mesmo que \textunderscore sarcólito\textunderscore .
\section{Sarcólitha}
\begin{itemize}
\item {Grp. gram.:f.}
\end{itemize}
O mesmo que \textunderscore sarcólitho\textunderscore .
\section{Sarcólitho}
\begin{itemize}
\item {Grp. gram.:m.}
\end{itemize}
\begin{itemize}
\item {Proveniência:(Do gr. \textunderscore sarx\textunderscore  + \textunderscore lithos\textunderscore )}
\end{itemize}
Pedra transparente e da côr da carne.
\section{Sarcólito}
\begin{itemize}
\item {Grp. gram.:m.}
\end{itemize}
\begin{itemize}
\item {Proveniência:(Do gr. \textunderscore sarx\textunderscore  + \textunderscore lithos\textunderscore )}
\end{itemize}
Pedra transparente e da côr da carne.
\section{Sarcólobo}
\begin{itemize}
\item {Grp. gram.:m.}
\end{itemize}
\begin{itemize}
\item {Proveniência:(Do gr. \textunderscore sarx\textunderscore , \textunderscore sarkos\textunderscore  + \textunderscore lobos\textunderscore )}
\end{itemize}
Gênero de plantas asclepiádeas.
\section{Sarcologia}
\begin{itemize}
\item {Grp. gram.:f.}
\end{itemize}
\begin{itemize}
\item {Proveniência:(Do gr. \textunderscore sarx\textunderscore  + \textunderscore logos\textunderscore )}
\end{itemize}
Tratado do tecido muscular ou das partes carnudas do corpo.
\section{Sarcológico}
\begin{itemize}
\item {Grp. gram.:adj.}
\end{itemize}
Relativo á sarcologia.
\section{Sarcoma}
\begin{itemize}
\item {Grp. gram.:m.}
\end{itemize}
\begin{itemize}
\item {Proveniência:(Gr. \textunderscore sarkoma\textunderscore )}
\end{itemize}
Tumor ou excrescência mórbida, com a consistência da carne.
\section{Sarcomatoso}
\begin{itemize}
\item {Grp. gram.:adj.}
\end{itemize}
Relativo á sarcoma; que tem sarcoma.
\section{Sarcômphalo}
\begin{itemize}
\item {Grp. gram.:m.}
\end{itemize}
\begin{itemize}
\item {Proveniência:(Do gr. \textunderscore sarx\textunderscore  + \textunderscore omphalos\textunderscore )}
\end{itemize}
Tumor duro no umbigo.
\section{Sarcônfalo}
\begin{itemize}
\item {Grp. gram.:m.}
\end{itemize}
\begin{itemize}
\item {Proveniência:(Do gr. \textunderscore sarx\textunderscore  + \textunderscore omphalos\textunderscore )}
\end{itemize}
Tumor duro no umbigo.
\section{Sarcóphago}
\begin{itemize}
\item {Grp. gram.:adj.}
\end{itemize}
\begin{itemize}
\item {Grp. gram.:M.}
\end{itemize}
\begin{itemize}
\item {Utilização:Ext.}
\end{itemize}
\begin{itemize}
\item {Proveniência:(Lat. \textunderscore sarcophagus\textunderscore )}
\end{itemize}
Que corrói as carnes.
Túmulo, em que os antigos metiam os cadáveres que não queriam queimar, e que era feito de uma variedade de pedra, a que se atribuía a propriedade de consumir as carnes.
Túmulo.
\section{Sarcophylla}
\begin{itemize}
\item {Grp. gram.:f.}
\end{itemize}
\begin{itemize}
\item {Proveniência:(Do gr. \textunderscore sarx\textunderscore  + \textunderscore phullon\textunderscore )}
\end{itemize}
A parte carnuda das fôlhas.
\section{Sarcophyllo}
\begin{itemize}
\item {Grp. gram.:m.}
\end{itemize}
\begin{itemize}
\item {Grp. gram.:Pl.}
\end{itemize}
Gênero de plantas leguminosas, cuja espécie týpica cresce no Cabo da Bôa-Esperança.
Grupo de mammíferos marzupiaes, no systema de Cuvier.
\section{Sarcopióide}
\begin{itemize}
\item {Grp. gram.:adj.}
\end{itemize}
\begin{itemize}
\item {Proveniência:(Do gr. \textunderscore sarx\textunderscore  + \textunderscore puon\textunderscore  + \textunderscore eidos\textunderscore )}
\end{itemize}
Que tem a aparência de carne e pus.
\section{Sarcopirâmide}
\begin{itemize}
\item {Grp. gram.:f.}
\end{itemize}
\begin{itemize}
\item {Proveniência:(De \textunderscore sarco...\textunderscore  + \textunderscore pirâmide\textunderscore )}
\end{itemize}
Gênero de plantas melastomáceas.
\section{Sarcopsiclo}
\begin{itemize}
\item {Grp. gram.:m.}
\end{itemize}
Animálculo parasito, que na África ataca a pelle da gente, causando doença contagiosa.
\section{Sarcopsyclo}
\begin{itemize}
\item {Grp. gram.:m.}
\end{itemize}
Animálculo parasito, que na África ataca a pelle da gente, causando doença contagiosa.
\section{Sarcopto}
\begin{itemize}
\item {Grp. gram.:m.}
\end{itemize}
Gênero de arachnídeos, da ordem dos ácaros, que tem por typo o ácaro da sarna.
(Contr. de \textunderscore sarcocopto\textunderscore , do gr. \textunderscore sarx\textunderscore  + \textunderscore koptein\textunderscore )
\section{Sarcopyóide}
\begin{itemize}
\item {Grp. gram.:adj.}
\end{itemize}
\begin{itemize}
\item {Proveniência:(Do gr. \textunderscore sarx\textunderscore  + \textunderscore puon\textunderscore  + \textunderscore eidos\textunderscore )}
\end{itemize}
Que tem a apparência de carne e pus.
\section{Sarcopyrâmide}
\begin{itemize}
\item {Grp. gram.:f.}
\end{itemize}
\begin{itemize}
\item {Proveniência:(De \textunderscore sarco...\textunderscore  + \textunderscore pyrâmide\textunderscore )}
\end{itemize}
Gênero de plantas melastomáceas.
\section{Sarçoso}
\begin{itemize}
\item {Grp. gram.:adj.}
\end{itemize}
Que tem sarças, que tem espinhos; que produz sarças.
\section{Sarcospermo}
\begin{itemize}
\item {Grp. gram.:adj.}
\end{itemize}
Que tem sementes carnudas.
(Do \textunderscore sarx\textunderscore  + \textunderscore sperma\textunderscore )
\section{Sarcostema}
\begin{itemize}
\item {Grp. gram.:f.}
\end{itemize}
\begin{itemize}
\item {Proveniência:(Do gr. \textunderscore sarx\textunderscore , \textunderscore sarkos\textunderscore  + \textunderscore stemma\textunderscore )}
\end{itemize}
Gênero de plantas asclepiádeas.
\section{Sarcostemma}
\begin{itemize}
\item {Grp. gram.:f.}
\end{itemize}
\begin{itemize}
\item {Proveniência:(Do gr. \textunderscore sarx\textunderscore , \textunderscore sarkos\textunderscore  + \textunderscore stemma\textunderscore )}
\end{itemize}
Gênero de plantas asclepiádeas.
\section{Sarcóstomo}
\begin{itemize}
\item {Grp. gram.:adj.}
\end{itemize}
\begin{itemize}
\item {Utilização:Zool.}
\end{itemize}
\begin{itemize}
\item {Proveniência:(Do gr. \textunderscore sarx\textunderscore  + \textunderscore stoma\textunderscore )}
\end{itemize}
Que tem a bôca carnuda.
\section{Sarcótico}
\begin{itemize}
\item {Grp. gram.:adj.}
\end{itemize}
\begin{itemize}
\item {Proveniência:(Gr. \textunderscore sarkotikos\textunderscore )}
\end{itemize}
Que facilita a regeneração das carnes.
\section{Sarcotripsia}
\begin{itemize}
\item {Grp. gram.:f.}
\end{itemize}
\begin{itemize}
\item {Proveniência:(Do gr. \textunderscore sarx\textunderscore  + \textunderscore tripsis\textunderscore )}
\end{itemize}
Operação cirúrgica, que consiste no esmagamento linear das carnes.
\section{Sarda}
\begin{itemize}
\item {Grp. gram.:f.}
\end{itemize}
\begin{itemize}
\item {Utilização:Gír.}
\end{itemize}
\begin{itemize}
\item {Proveniência:(Lat. \textunderscore sarda\textunderscore )}
\end{itemize}
Nome vulgar de dois peixes acanthopterýgios.
Faca.
\section{Sarda}
\begin{itemize}
\item {Grp. gram.:f.}
\end{itemize}
(V.sardas)
\section{Sarda-ágatha}
\begin{itemize}
\item {Grp. gram.:f.}
\end{itemize}
Espécie de ágatha, alaranjada e vermelho-clara.
\section{Sardácata}
\begin{itemize}
\item {Grp. gram.:f.}
\end{itemize}
O mesmo que \textunderscore sarda-ágatha\textunderscore .
\section{Sardágata}
\begin{itemize}
\item {Grp. gram.:f.}
\end{itemize}
O mesmo que \textunderscore sarda-ágatha\textunderscore .
\section{Sardágatha}
\begin{itemize}
\item {Grp. gram.:f.}
\end{itemize}
O mesmo que \textunderscore sarda-ágatha\textunderscore .
\section{Sardanapalamente}
\begin{itemize}
\item {Grp. gram.:adv.}
\end{itemize}
De modo sardanapalesco. Cf. Camillo, \textunderscore Esqueleto\textunderscore , 160.
\section{Sardanapalesco}
\begin{itemize}
\item {fónica:lês}
\end{itemize}
\begin{itemize}
\item {Grp. gram.:adj.}
\end{itemize}
Effeminado, licencioso e glotão como Sardanapalo.
Semelhante aos costumes de Sardanapalo.
\section{Sardanapálico}
\begin{itemize}
\item {Grp. gram.:adj.}
\end{itemize}
O mesmo ou melhor que \textunderscore sardanapalesco\textunderscore .
\section{Sardanapalizar}
\begin{itemize}
\item {Grp. gram.:v. i.}
\end{itemize}
Viver á maneira de Sardanapalo.
\section{Sardaneta}
\begin{itemize}
\item {fónica:nê}
\end{itemize}
\begin{itemize}
\item {Grp. gram.:f.}
\end{itemize}
\begin{itemize}
\item {Utilização:T. da Bairrada}
\end{itemize}
O mesmo que \textunderscore sardanisca\textunderscore .
\section{Sardanisca}
\begin{itemize}
\item {Grp. gram.:f.}
\end{itemize}
\begin{itemize}
\item {Proveniência:(De \textunderscore sardão\textunderscore )}
\end{itemize}
O mesmo que \textunderscore lagartixa\textunderscore .
\section{Sardanita}
\begin{itemize}
\item {Grp. gram.:f.}
\end{itemize}
\begin{itemize}
\item {Utilização:Bras}
\end{itemize}
O mesmo que \textunderscore sardanisca\textunderscore .
\section{Sardão}
\begin{itemize}
\item {Grp. gram.:m.}
\end{itemize}
\begin{itemize}
\item {Utilização:T. de Vinhaes}
\end{itemize}
\begin{itemize}
\item {Proveniência:(Do gr. \textunderscore saura\textunderscore ? Ou de \textunderscore sardo\textunderscore ^1? Ou do ár. \textunderscore hardam\textunderscore ? Cf. Sousa, \textunderscore Vestig. da Ling. Ár.\textunderscore )}
\end{itemize}
Espécie de lagarto.
Raiz ou ramo retorcido de carrasco, que vegeta em fendas de penedia.
\section{Sardas}
\begin{itemize}
\item {Grp. gram.:f. pl.}
\end{itemize}
\begin{itemize}
\item {Proveniência:(De \textunderscore sarda\textunderscore ^1?)}
\end{itemize}
Manchas amareladas, que algumas pessôas apresentam no rosto, sobretudo as de cabello ruivo.
\section{Sardento}
\begin{itemize}
\item {Grp. gram.:adj.}
\end{itemize}
O mesmo que \textunderscore sardoso\textunderscore .
\section{Sardessai}
\begin{itemize}
\item {Grp. gram.:m.}
\end{itemize}
Donatário de território, na Índia Portuguesa. Cf. Th. Ribeiro, \textunderscore Jornadas\textunderscore , II, 17 e 331.
\section{Sardíaco}
\begin{itemize}
\item {Grp. gram.:m.  e  adj.}
\end{itemize}
O mesmo que \textunderscore sardo\textunderscore ^2.
\section{Sardinha}
\begin{itemize}
\item {Grp. gram.:f.}
\end{itemize}
\begin{itemize}
\item {Utilização:Gír.}
\end{itemize}
\begin{itemize}
\item {Utilização:Prov.}
\end{itemize}
\begin{itemize}
\item {Utilização:trasm.}
\end{itemize}
\begin{itemize}
\item {Proveniência:(Lat. \textunderscore sardina\textunderscore )}
\end{itemize}
Pequeno peixe clúpeo.
Espécie de jôgo de crianças.
Porco.
O mesmo que \textunderscore bofetada\textunderscore .
\section{Sardinheira}
\begin{itemize}
\item {Grp. gram.:f.}
\end{itemize}
Vendedeira de sardinhas.
Pesca de sardinhas.
Espécie de gerânio.
Rêde de pescar sardinha.
\section{Sardinheiro}
\begin{itemize}
\item {Grp. gram.:adj.}
\end{itemize}
\begin{itemize}
\item {Grp. gram.:M.}
\end{itemize}
\begin{itemize}
\item {Utilização:T. de Penafiel}
\end{itemize}
Relativo a sardinha.
Vendedor de sardinha.
Insecto neuróptero, semelhante ás libéllulas que volitam sôbre as águas, mas um tanto maior, de côr castanha e que dá caça a outros insectos.
\section{Sardinheta}
\begin{itemize}
\item {fónica:nhê}
\end{itemize}
\begin{itemize}
\item {Grp. gram.:f.}
\end{itemize}
\begin{itemize}
\item {Utilização:Ant.}
\end{itemize}
\begin{itemize}
\item {Utilização:Fam.}
\end{itemize}
Pequena sardinha.
Açoite.
Batecu.
\section{Sárdio}
\begin{itemize}
\item {Grp. gram.:m.}
\end{itemize}
\begin{itemize}
\item {Proveniência:(Lat. \textunderscore sardius\textunderscore )}
\end{itemize}
Pedra preciosa, sem brilho.
\section{Sardo}
\begin{itemize}
\item {Grp. gram.:adj.}
\end{itemize}
Que tem sardas, sardoso.
\section{Sardo}
\begin{itemize}
\item {Grp. gram.:adj.}
\end{itemize}
\begin{itemize}
\item {Grp. gram.:M.}
\end{itemize}
\begin{itemize}
\item {Proveniência:(Lat. \textunderscore sardus\textunderscore )}
\end{itemize}
Relativo á Sardenha.
Habitante da Sardenha.
Peixe plagióstomo, de dentes triangulares.
\section{Sardoeira}
\begin{itemize}
\item {Grp. gram.:f.}
\end{itemize}
\begin{itemize}
\item {Utilização:Prov.}
\end{itemize}
\begin{itemize}
\item {Utilização:minh.}
\end{itemize}
\begin{itemize}
\item {Proveniência:(De \textunderscore sardão\textunderscore )}
\end{itemize}
Quintal ou quinta murada.
\section{Sardónia}
\begin{itemize}
\item {Grp. gram.:f.}
\end{itemize}
\begin{itemize}
\item {Proveniência:(Lat. \textunderscore sardonia\textunderscore )}
\end{itemize}
Planta ranunculácea.
O mesmo que \textunderscore sardónica\textunderscore .
\section{Sardónica}
\begin{itemize}
\item {Grp. gram.:f.}
\end{itemize}
\begin{itemize}
\item {Proveniência:(Lat. \textunderscore sardonycha\textunderscore )}
\end{itemize}
Espécie de calcedónia, de côr escuro-alaranjada.
\section{Sardónico}
\begin{itemize}
\item {Grp. gram.:adj.}
\end{itemize}
Relativo á sardónica.
\section{Sardónico}
\begin{itemize}
\item {Grp. gram.:adj.}
\end{itemize}
Diz-se do riso forçado e sarcástico, que, segundo os antigos, podia sêr produzido pela sardónia, planta.
\section{Sardónio}
\begin{itemize}
\item {Grp. gram.:m.  e  adj.}
\end{itemize}
\begin{itemize}
\item {Proveniência:(Lat. \textunderscore sardonius\textunderscore )}
\end{itemize}
O mesmo que \textunderscore sardo\textunderscore ^2.
\section{Sardonisca}
\begin{itemize}
\item {Grp. gram.:f.}
\end{itemize}
\begin{itemize}
\item {Utilização:T. de Penafiel}
\end{itemize}
O mesmo que \textunderscore sardanisca\textunderscore .
\section{Sardoso}
\begin{itemize}
\item {Grp. gram.:adj.}
\end{itemize}
Que tem sardas.
\section{Saresma}
\begin{itemize}
\item {fónica:sarês}
\end{itemize}
\begin{itemize}
\item {Grp. gram.:m.  e  f.}
\end{itemize}
O mesmo que \textunderscore seresma\textunderscore .
\section{Sarga}
\begin{itemize}
\item {Grp. gram.:f.}
\end{itemize}
Espécie de uva.
\section{Sargaça}
\begin{itemize}
\item {Grp. gram.:f.}
\end{itemize}
Planta cistínea, (\textunderscore cistus halimifolius\textunderscore ).
\section{Sargaceiro}
\begin{itemize}
\item {Grp. gram.:m.}
\end{itemize}
Homem, que se emprega na apanha do sargaço.
\section{Sargacinha}
\begin{itemize}
\item {Grp. gram.:f.  e  adj.}
\end{itemize}
\begin{itemize}
\item {Utilização:Prov.}
\end{itemize}
\begin{itemize}
\item {Utilização:trasm.}
\end{itemize}
\begin{itemize}
\item {Utilização:T. da Bairrada}
\end{itemize}
\begin{itemize}
\item {Proveniência:(De \textunderscore sargaço\textunderscore )}
\end{itemize}
Espécie de uva, de bagos miúdos.
Planta medicinal, de flôr azul, mais conhecida por \textunderscore erva-das-sete-sangrias\textunderscore . Cf. \textunderscore Pharmacopeia Port.\textunderscore 
Espécie de ameixa branca e redonda.
\section{Sargaço}
\begin{itemize}
\item {Grp. gram.:m.}
\end{itemize}
\begin{itemize}
\item {Utilização:Prov.}
\end{itemize}
\begin{itemize}
\item {Utilização:beir.}
\end{itemize}
Gênero de algas, da fam. das fucáceas, que cresce sôbre as águas, occupando larga superficie de alguns mares.
Bodelha.
Planta montesinha, de flôr branca e ás vezes amarela:«\textunderscore até nos sargaços me sabem abraços\textunderscore ». (Prolóquio pop.)
(Cast. \textunderscore sargaza\textunderscore )
\section{Sargenta}
\begin{itemize}
\item {Grp. gram.:f.}
\end{itemize}
\begin{itemize}
\item {Utilização:Ant.}
\end{itemize}
\begin{itemize}
\item {Proveniência:(De \textunderscore sargente\textunderscore )}
\end{itemize}
Criada; aia.
\section{Sargenta}
\begin{itemize}
\item {Grp. gram.:f.}
\end{itemize}
(Corr. de \textunderscore sargeta\textunderscore ^1)
\section{Sargente}
\begin{itemize}
\item {Grp. gram.:m.  e  f.}
\end{itemize}
\begin{itemize}
\item {Utilização:Des.}
\end{itemize}
\begin{itemize}
\item {Proveniência:(Do fr. \textunderscore sergent\textunderscore )}
\end{itemize}
Criado; servente.
Pessôa, que auxilia.
Official de justiça.
\section{Sargentear}
\begin{itemize}
\item {Grp. gram.:v. i.}
\end{itemize}
\begin{itemize}
\item {Proveniência:(De \textunderscore sargento\textunderscore ^1)}
\end{itemize}
Exercer funcções de sargento.
Afadigar-se.
Lidar com afan.
Saracotear.
\section{Sargento}
\begin{itemize}
\item {Grp. gram.:m.}
\end{itemize}
Official militar, de que há duas categorias, (1.^o e 2.^o sargento).
(Alter. de \textunderscore sargente\textunderscore )
\section{Sargento}
\begin{itemize}
\item {Grp. gram.:m.}
\end{itemize}
(Corr. de \textunderscore cingento\textunderscore )
\section{Sargentola}
\begin{itemize}
\item {Grp. gram.:m.}
\end{itemize}
\begin{itemize}
\item {Utilização:Deprec.}
\end{itemize}
Sargento rude.
\section{Sargeta}
\begin{itemize}
\item {fónica:jê}
\end{itemize}
\begin{itemize}
\item {Grp. gram.:f.}
\end{itemize}
\begin{itemize}
\item {Proveniência:(De \textunderscore sarja\textunderscore ^1)}
\end{itemize}
Sulco, para escoar águas; valleta.
Abertura nas ruas ou praças, por onde as águas pluviaes se escoam para a canalização geral.
\section{Sargeta}
\begin{itemize}
\item {fónica:jê}
\end{itemize}
\begin{itemize}
\item {Grp. gram.:f.}
\end{itemize}
\begin{itemize}
\item {Proveniência:(De \textunderscore sarja\textunderscore ^2)}
\end{itemize}
Sarja estreita ou delgada.
\section{Sargo}
\begin{itemize}
\item {Grp. gram.:m.}
\end{itemize}
\begin{itemize}
\item {Proveniência:(Lat. \textunderscore sargus\textunderscore )}
\end{itemize}
O mesmo que \textunderscore pargo\textunderscore .
Espécie de labro, peixe.
\section{Sargola}
\begin{itemize}
\item {Grp. gram.:f.}
\end{itemize}
\begin{itemize}
\item {Proveniência:(De \textunderscore sargo\textunderscore )}
\end{itemize}
Peixe de Portugal.
\section{Sarguetet}
\begin{itemize}
\item {fónica:guê}
\end{itemize}
\begin{itemize}
\item {Grp. gram.:m.}
\end{itemize}
Pequeno sargo.
\section{Sari}
\begin{itemize}
\item {Grp. gram.:m.}
\end{itemize}
Espécie de chale, usado pelas parsinas.
\section{Saria}
\begin{itemize}
\item {Grp. gram.:f.}
\end{itemize}
\begin{itemize}
\item {Utilização:Prov.}
\end{itemize}
Casta de uva, provavelmente o mesmo que \textunderscore assario\textunderscore  ou \textunderscore asserio\textunderscore .
\section{Saribanda}
\begin{itemize}
\item {Grp. gram.:f.}
\end{itemize}
O mesmo que \textunderscore sarabanda\textunderscore .
\section{Saribebe}
\begin{itemize}
\item {Grp. gram.:m.}
\end{itemize}
Árvore da Guiana inglesa.
\section{Sariema}
\begin{itemize}
\item {Grp. gram.:f.}
\end{itemize}
\begin{itemize}
\item {Utilização:Bras}
\end{itemize}
Ave, de canto estridente.
\section{Sariga}
\begin{itemize}
\item {Grp. gram.:m.}
\end{itemize}
\begin{itemize}
\item {Utilização:Zool.}
\end{itemize}
Espécie de monotreme da Oceânia.
\section{Sarigué}
\begin{itemize}
\item {Grp. gram.:f.}
\end{itemize}
\begin{itemize}
\item {Proveniência:(Do guar. \textunderscore çarigueija\textunderscore )}
\end{itemize}
Animal mammifero, da ordem dos marsupiaes, e cuja fêmea tem sob o ventre uma espécie de bôlsa, em que conduz os filhos.
\section{Sariguéa}
\begin{itemize}
\item {Grp. gram.:f.}
\end{itemize}
\begin{itemize}
\item {Proveniência:(Do guar. \textunderscore çarigueija\textunderscore )}
\end{itemize}
Animal mammifero, da ordem dos marsupiaes, e cuja fêmea tem sob o ventre uma espécie de bôlsa, em que conduz os filhos.
\section{Sarigueia}
\begin{itemize}
\item {Grp. gram.:f.}
\end{itemize}
\begin{itemize}
\item {Proveniência:(Do guar. \textunderscore çarigueija\textunderscore )}
\end{itemize}
Animal mammifero, da ordem dos marsupiaes, e cuja fêmea tem sob o ventre uma espécie de bôlsa, em que conduz os filhos.
\section{Sarilhar}
\begin{itemize}
\item {Grp. gram.:v. t.}
\end{itemize}
O mesmo que \textunderscore ensarilhar\textunderscore .
O mesmo que \textunderscore traquinar\textunderscore .
\section{Sarjeta}
\begin{itemize}
\item {fónica:jê}
\end{itemize}
\begin{itemize}
\item {Grp. gram.:f.}
\end{itemize}
\begin{itemize}
\item {Proveniência:(De \textunderscore sarja\textunderscore ^1)}
\end{itemize}
Sulco, para escoar águas; valleta.
Abertura nas ruas ou praças, por onde as águas pluviaes se escoam para a canalização geral.
\section{Sarjeta}
\begin{itemize}
\item {fónica:jê}
\end{itemize}
\begin{itemize}
\item {Grp. gram.:f.}
\end{itemize}
\begin{itemize}
\item {Proveniência:(De \textunderscore sarja\textunderscore ^2)}
\end{itemize}
Sarja estreita ou delgada.
\section{Sarilho}
\begin{itemize}
\item {Grp. gram.:m.}
\end{itemize}
\begin{itemize}
\item {Utilização:Bras}
\end{itemize}
\begin{itemize}
\item {Proveniência:(Do lat. hyp. \textunderscore sericulum\textunderscore , de \textunderscore sericum\textunderscore ?)}
\end{itemize}
Instrumento, a que se imprime movimento rotatório, para se lhe enrolarem os fios das maçarocas, e formarem-se as meadas.
Maquinismo, composto de um cylindro, suspenso por barras na extremidade, e em que se enrola a corda que sustenta um pêso que se quere elevar.
Movimento rotatório do corpo em volta de um trapézio.
Movimento rotatório.
Roda viva.
Encostamento de espingardas, em grupos de três.
Haste vertical, formando cruz com outras hastes, e á qual se encostam as espingardas nos acampamentos.
Espécie de jôgo popular.
Engenho, para tirar água, o mesmo que \textunderscore nora\textunderscore .
\section{Sarissa}
\begin{itemize}
\item {Grp. gram.:f.}
\end{itemize}
\begin{itemize}
\item {Proveniência:(Lat. \textunderscore sarrissa\textunderscore )}
\end{itemize}
Espécie de lança muito comprida, usada pelos Macedónios.
\section{Sarissóforo}
\begin{itemize}
\item {Grp. gram.:m.}
\end{itemize}
\begin{itemize}
\item {Proveniência:(Lat. \textunderscore sarissophorus\textunderscore )}
\end{itemize}
Soldado, armado de sarissa.
\section{Sarissóphoro}
\begin{itemize}
\item {Grp. gram.:m.}
\end{itemize}
\begin{itemize}
\item {Proveniência:(Lat. \textunderscore sarissophorus\textunderscore )}
\end{itemize}
Soldado, armado de sarissa.
\section{Sarja}
\begin{itemize}
\item {Grp. gram.:f.}
\end{itemize}
Incisão cirúrgica, feita na pelle, para a extracção de sangue ou de pus.
(Parece relacionar-se com \textunderscore sanja\textunderscore )
\section{Sarja}
\begin{itemize}
\item {Grp. gram.:f.}
\end{itemize}
\begin{itemize}
\item {Proveniência:(Do ár. \textunderscore sarje\textunderscore )}
\end{itemize}
Tecido entrançado de seda ou lan.
\section{Sarjação}
\begin{itemize}
\item {Grp. gram.:f.}
\end{itemize}
Acto ou effeito de sarjar.
\section{Sarjadeira}
\begin{itemize}
\item {Grp. gram.:f.}
\end{itemize}
\begin{itemize}
\item {Utilização:Bras}
\end{itemize}
Instrumento, o mesmo que \textunderscore sarjador\textunderscore .
\section{Sarjado}
\begin{itemize}
\item {Grp. gram.:adj.}
\end{itemize}
\begin{itemize}
\item {Proveniência:(De \textunderscore sarja\textunderscore ^2)}
\end{itemize}
Diz-se do tecido, que tem fórma de sarja; entrançado.
\section{Sarjador}
\begin{itemize}
\item {Grp. gram.:m.  e  adj.}
\end{itemize}
\begin{itemize}
\item {Grp. gram.:M.}
\end{itemize}
O que sarja.
Espécie de lanceta para sarjar.
\section{Sarjadura}
\begin{itemize}
\item {Grp. gram.:f.}
\end{itemize}
O mesmo que \textunderscore sarjação\textunderscore .
\section{Sarjar}
\begin{itemize}
\item {Grp. gram.:v. t.}
\end{itemize}
\begin{itemize}
\item {Proveniência:(De \textunderscore sarja\textunderscore ^1)}
\end{itemize}
Fazer sarjas em.
\section{Sarjel}
\begin{itemize}
\item {Grp. gram.:m.}
\end{itemize}
\begin{itemize}
\item {Proveniência:(De \textunderscore sarja\textunderscore )}
\end{itemize}
Tecido grosseiro de lan.
\section{Sarmão}
\begin{itemize}
\item {Grp. gram.:m.}
\end{itemize}
\begin{itemize}
\item {Utilização:pop.}
\end{itemize}
\begin{itemize}
\item {Utilização:Ant.}
\end{itemize}
O mesmo que \textunderscore salmão\textunderscore . Cf. P. Carvalho, \textunderscore Corogr. Port.\textunderscore 
\section{Sármatas}
\begin{itemize}
\item {Grp. gram.:m. pl.}
\end{itemize}
\begin{itemize}
\item {Proveniência:(Lat. \textunderscore sarmatae\textunderscore )}
\end{itemize}
Antigo povo eslavo, que habitava a região situada entre o Vístula e o Don.
\section{Sarmático}
\begin{itemize}
\item {Grp. gram.:adj.}
\end{itemize}
Relativo aos Sármatas. Cf. \textunderscore Lusíadas\textunderscore , III, 10.
\section{Sarmenho}
\begin{itemize}
\item {Grp. gram.:m.}
\end{itemize}
\begin{itemize}
\item {Utilização:Prov.}
\end{itemize}
\begin{itemize}
\item {Utilização:trasm.}
\end{itemize}
O mesmo que \textunderscore saramenho\textunderscore .
Espécie de pêro.
\section{Sarmentáceas}
\begin{itemize}
\item {Grp. gram.:f. pl.}
\end{itemize}
Família de plantas, que tem por typo a videira.
Ampelídeas.
(Fem. pl. de \textunderscore sarmentáceo\textunderscore )
\section{Sarmentáceo}
\begin{itemize}
\item {Grp. gram.:adj.}
\end{itemize}
\begin{itemize}
\item {Proveniência:(De \textunderscore sarmento\textunderscore )}
\end{itemize}
Relativo ou semelhante á videira.
Sarmentífero.
\section{Sarmentício}
\begin{itemize}
\item {Grp. gram.:adj.}
\end{itemize}
O mesmo que \textunderscore sarmentoso\textunderscore .
\section{Sarmentífero}
\begin{itemize}
\item {Grp. gram.:adj.}
\end{itemize}
\begin{itemize}
\item {Proveniência:(Do lat. \textunderscore sarmentum\textunderscore  + \textunderscore ferre\textunderscore )}
\end{itemize}
Que tem ou produz sarmentos.
\section{Sarmento}
\begin{itemize}
\item {Grp. gram.:m.}
\end{itemize}
\begin{itemize}
\item {Proveniência:(Lat. \textunderscore sarmentum\textunderscore )}
\end{itemize}
Vide; rebento da videira.
Rebento vegetal.
Haste de trepadeira.
Caule nodoso, que lança raizes pelos nós.
Vide sêca, para lenha.
\section{Sarmentoso}
\begin{itemize}
\item {Grp. gram.:adj.}
\end{itemize}
\begin{itemize}
\item {Proveniência:(Lat. \textunderscore sarmentosus\textunderscore )}
\end{itemize}
Relativo a sarmento; que tem sarmentos.
Que tem a natureza de sarmento.
\section{Sarna}
\begin{itemize}
\item {Grp. gram.:f.}
\end{itemize}
\begin{itemize}
\item {Grp. gram.:M.  e  f.}
\end{itemize}
\begin{itemize}
\item {Utilização:Pop.}
\end{itemize}
Doença cutânea e contagiosa produzida pela presença de uns animálculos arachnídeos.
Ronha, nos cavallos.
Doença das oliveiras, que se manifesta por tubérculos irregulares nos ramos novos.
Pessôa impertinente, maçadora: \textunderscore o pequeno é um sarna\textunderscore .
(B. lat. \textunderscore sarna\textunderscore )
\section{Sarnambi}
\begin{itemize}
\item {Grp. gram.:m.}
\end{itemize}
\begin{itemize}
\item {Utilização:Bras}
\end{itemize}
Mollusco comestível, de concha bivalve.
\section{Sarné}
\begin{itemize}
\item {Grp. gram.:m.}
\end{itemize}
\begin{itemize}
\item {Utilização:Bras}
\end{itemize}
Quadrúpede dos sertões.
\section{Sarnento}
\begin{itemize}
\item {Grp. gram.:adj.}
\end{itemize}
\begin{itemize}
\item {Utilização:Fig.}
\end{itemize}
\begin{itemize}
\item {Grp. gram.:M.}
\end{itemize}
\begin{itemize}
\item {Utilização:Prov.}
\end{itemize}
Que tem sarna.
Râncido.
Acanaveado.
Planta herbácea. (Colhido em Turquel)
\section{Sarnícula}
\begin{itemize}
\item {Grp. gram.:m.}
\end{itemize}
\begin{itemize}
\item {Utilização:Pop.}
\end{itemize}
\begin{itemize}
\item {Proveniência:(De \textunderscore sarna\textunderscore )}
\end{itemize}
Homem impertinente e rabugento.
\section{Sarnir}
\begin{itemize}
\item {Grp. gram.:v. i.}
\end{itemize}
\begin{itemize}
\item {Utilização:Pop.}
\end{itemize}
Rabujar, serrazinar.
(Cp. \textunderscore sarna\textunderscore )
\section{Sarnoso}
\begin{itemize}
\item {Grp. gram.:adj.}
\end{itemize}
O mesmo que \textunderscore sarnento\textunderscore .
\section{Saro}
\begin{itemize}
\item {Grp. gram.:m.}
\end{itemize}
Espécie de palmeira africana.
\section{Saroar}
\begin{itemize}
\item {Grp. gram.:v. i.}
\end{itemize}
\begin{itemize}
\item {Utilização:Pop.}
\end{itemize}
O mesmo que \textunderscore seroar\textunderscore .
\section{Saroido}
\begin{itemize}
\item {Grp. gram.:adj.}
\end{itemize}
\begin{itemize}
\item {Utilização:Prov.}
\end{itemize}
\begin{itemize}
\item {Utilização:alent.}
\end{itemize}
O mesmo que \textunderscore serôdio\textunderscore . Cf. Rev. \textunderscore Tradição\textunderscore , 98.
\section{Sarónide}
\begin{itemize}
\item {Grp. gram.:m.}
\end{itemize}
\begin{itemize}
\item {Proveniência:(Do gr. \textunderscore saronis\textunderscore )}
\end{itemize}
O mesmo que \textunderscore druida\textunderscore . Cf. Filinto, XV, 73.
\section{Sarópode}
\begin{itemize}
\item {Grp. gram.:adj.}
\end{itemize}
\begin{itemize}
\item {Utilização:Zool.}
\end{itemize}
\begin{itemize}
\item {Grp. gram.:M.}
\end{itemize}
\begin{itemize}
\item {Proveniência:(Do gr. \textunderscore saros\textunderscore  + \textunderscore pous\textunderscore , \textunderscore podos\textunderscore )}
\end{itemize}
Que tem patas peludas ou parecidas a vassoiras.
Gênero de insectos hymenópteros.
\section{Sarotamno}
\begin{itemize}
\item {Grp. gram.:m.}
\end{itemize}
\begin{itemize}
\item {Proveniência:(Do gr. \textunderscore saros\textunderscore  + \textunderscore tamnos\textunderscore )}
\end{itemize}
Gênero de plantas leguminosas do centro e do sul da Europa.
\section{Sarote}
\begin{itemize}
\item {Grp. gram.:m.}
\end{itemize}
\begin{itemize}
\item {Proveniência:(Do gr. \textunderscore sarotes\textunderscore )}
\end{itemize}
Gênero de plantas bittneriáceas da Nova-Hollanda.
Gênero de arachnídeos.
\section{Saroto}
\begin{itemize}
\item {fónica:sarô}
\end{itemize}
\begin{itemize}
\item {Grp. gram.:adj.}
\end{itemize}
\begin{itemize}
\item {Utilização:Prov.}
\end{itemize}
\begin{itemize}
\item {Utilização:trasm.}
\end{itemize}
Que tem cortada a ponta do rabo.
Que tem um dedo cortado.
(Cp. \textunderscore saracoto\textunderscore )
\section{Sarpão}
\begin{itemize}
\item {Grp. gram.:m.}
\end{itemize}
\begin{itemize}
\item {Utilização:Des.}
\end{itemize}
O mesmo que \textunderscore serpão\textunderscore . Cf. B. Pereira, \textunderscore Prosódia\textunderscore , vb. \textunderscore erraticum\textunderscore .
\section{Sarpar}
\begin{itemize}
\item {Grp. gram.:v. t.}
\end{itemize}
\begin{itemize}
\item {Grp. gram.:V. i.}
\end{itemize}
Erguer (uma âncora).
Levantar ferro, navegar.
(Cast. \textunderscore zarpar\textunderscore )
\section{Sarra}
\begin{itemize}
\item {Grp. gram.:f.}
\end{itemize}
\begin{itemize}
\item {Utilização:Pop.}
\end{itemize}
\begin{itemize}
\item {Proveniência:(De \textunderscore sarrar\textunderscore )}
\end{itemize}
Instrumento para sarrar.
\section{Sarrabal}
\begin{itemize}
\item {Grp. gram.:m.}
\end{itemize}
O mesmo que \textunderscore sarrabalho\textunderscore . Cf. Filinto, I, 131, 146 e 246; III, 217; IV, 87.
\section{Sarrabalho}
\begin{itemize}
\item {Grp. gram.:m.}
\end{itemize}
\begin{itemize}
\item {Utilização:Bras. do S}
\end{itemize}
Baile campestre, espécie de fandango.
\section{Sarrabiscos}
\begin{itemize}
\item {Grp. gram.:f. pl.}
\end{itemize}
\begin{itemize}
\item {Utilização:Prov.}
\end{itemize}
\begin{itemize}
\item {Utilização:minh.}
\end{itemize}
Rabiscos, gatafunhos.
\section{Sarrabulhada}
\begin{itemize}
\item {Grp. gram.:f.}
\end{itemize}
\begin{itemize}
\item {Utilização:Fig.}
\end{itemize}
Grande quantidade de sarrabulho.
Mistifório.
Desordem, confusão.
\section{Sarrabulhento}
\begin{itemize}
\item {Grp. gram.:adj.}
\end{itemize}
\begin{itemize}
\item {Utilização:T. do Fundão}
\end{itemize}
\begin{itemize}
\item {Utilização:Fig.}
\end{itemize}
\begin{itemize}
\item {Proveniência:(De \textunderscore sarrabulho\textunderscore )}
\end{itemize}
Diz-se do feijão já meio sêco.
Desordeiro. Cf. Macedo \textunderscore Burros\textunderscore , 336.
\section{Sarrabulho}
\begin{itemize}
\item {Grp. gram.:m.}
\end{itemize}
\begin{itemize}
\item {Utilização:Prov.}
\end{itemize}
\begin{itemize}
\item {Utilização:Fig.}
\end{itemize}
Sangue coagulado de porco.
Guisado, feito com esse sangue, figado e banha de porco derretida.
Matança dos porcos e actos subsequentes.
Espalhafato; desordem; confusão.
\section{Sarracênia}
\begin{itemize}
\item {Grp. gram.:f.}
\end{itemize}
Gênero de plantas vivazes, de grandes flôres solitárias, procedentes de regiões pantanosas da América do Norte.
\section{Sarraceniáceas}
\begin{itemize}
\item {Grp. gram.:f. pl.}
\end{itemize}
Família de plantas, que tem por typo a sarracênia.
\section{Sarraceno}
\begin{itemize}
\item {Grp. gram.:adj.}
\end{itemize}
\begin{itemize}
\item {Utilização:Bot.}
\end{itemize}
\begin{itemize}
\item {Grp. gram.:M.}
\end{itemize}
\begin{itemize}
\item {Proveniência:(Do lat. \textunderscore sarraceni\textunderscore )}
\end{itemize}
Diz-se dos Árabes que dominaram na Espanha, na Sicília e na África.
Árabe; moirisco.
Diz-se de uma variedade de milho amarelo, de palha curta. Cf. \textunderscore Port. au point de vue agr.\textunderscore , 578.
Indivíduo, pertencente aos Árabes e Muçulmanos que dominaram na Europa e na África.
Moiro.
\section{Sarraco}
\begin{itemize}
\item {Grp. gram.:m.}
\end{itemize}
\begin{itemize}
\item {Proveniência:(Lat. \textunderscore sarracum\textunderscore )}
\end{itemize}
Espécie de carroça, em que os agricultores romanos transportavam os productos agrícolas.
\section{Sarradela}
\begin{itemize}
\item {Grp. gram.:f.}
\end{itemize}
\begin{itemize}
\item {Utilização:Pop.}
\end{itemize}
O mesmo que \textunderscore serradela\textunderscore ^2.
\section{Sarrafaçador}
\begin{itemize}
\item {Grp. gram.:m.  e  adj.}
\end{itemize}
O que sarrafaça.
\section{Sarrafaçadura}
\begin{itemize}
\item {Grp. gram.:f.}
\end{itemize}
Acto ou effeito de sarrafaçar.
\section{Sarrafaçal}
\begin{itemize}
\item {Grp. gram.:m.}
\end{itemize}
\begin{itemize}
\item {Utilização:Pop.}
\end{itemize}
\begin{itemize}
\item {Proveniência:(De \textunderscore sarrafaçar\textunderscore )}
\end{itemize}
Official imperito, que trabalha mal ou abrutadamente.
\section{Sarrafaçar}
\begin{itemize}
\item {Grp. gram.:v. i.}
\end{itemize}
\begin{itemize}
\item {Grp. gram.:V. t.}
\end{itemize}
\begin{itemize}
\item {Utilização:P. us.}
\end{itemize}
\begin{itemize}
\item {Proveniência:(De \textunderscore sarrafar\textunderscore )}
\end{itemize}
Cortar alguma coisa com instrumento mal afiado.
Fazer barulho, cortando ou aguçando.
Trabalhar mal, grosseiramente.
Sarjar.
\section{Sarrafado}
\begin{itemize}
\item {Grp. gram.:adj.}
\end{itemize}
\begin{itemize}
\item {Utilização:Prov.}
\end{itemize}
\begin{itemize}
\item {Utilização:minh.}
\end{itemize}
\begin{itemize}
\item {Proveniência:(De \textunderscore sarrafo\textunderscore )}
\end{itemize}
Que é feito de duas peças; que não é inteiriço.
\section{Sarrafão}
\begin{itemize}
\item {Grp. gram.:m.}
\end{itemize}
\begin{itemize}
\item {Proveniência:(De \textunderscore sarrafo\textunderscore )}
\end{itemize}
O mesmo que \textunderscore vigota\textunderscore .
\section{Sarrafar}
\begin{itemize}
\item {Grp. gram.:v. i.}
\end{itemize}
\begin{itemize}
\item {Proveniência:(De \textunderscore sarrar\textunderscore )}
\end{itemize}
O mesmo que \textunderscore sarrafaçar\textunderscore .
\section{Sarrafear}
\begin{itemize}
\item {Grp. gram.:v. t.}
\end{itemize}
\begin{itemize}
\item {Grp. gram.:V. i.}
\end{itemize}
Cortar em sarrafos.
O mesmo que \textunderscore sarrafar\textunderscore .
\section{Sarrafo}
\begin{itemize}
\item {Grp. gram.:m.}
\end{itemize}
\begin{itemize}
\item {Proveniência:(De \textunderscore sarrafar\textunderscore ?)}
\end{itemize}
Fasquía; tira de madeira; ripa.
\section{Sarrafusca}
\begin{itemize}
\item {Grp. gram.:f.}
\end{itemize}
\begin{itemize}
\item {Utilização:Pop.}
\end{itemize}
Motim; balbúrdia.
\section{Sarragão}
\begin{itemize}
\item {Grp. gram.:m.}
\end{itemize}
Peixe, o mesmo que \textunderscore bonito\textunderscore .
\section{Sarraipa}
\begin{itemize}
\item {Grp. gram.:f.}
\end{itemize}
O mesmo que \textunderscore surraipa\textunderscore .
\section{Sarrajão}
\begin{itemize}
\item {Grp. gram.:m.}
\end{itemize}
\begin{itemize}
\item {Utilização:Zool.}
\end{itemize}
O mesmo que \textunderscore sarragão\textunderscore . Cf. P. Moraes, \textunderscore Zool. Elem.\textunderscore , 521.
\section{Sarralha}
\begin{itemize}
\item {Grp. gram.:f.}
\end{itemize}
\begin{itemize}
\item {Utilização:Bot.}
\end{itemize}
O mesmo ou melhor que \textunderscore serralha\textunderscore , (se a etym. é o lat. \textunderscore sarralia\textunderscore , como parece).
\section{Sarranho}
\begin{itemize}
\item {Grp. gram.:m.}
\end{itemize}
\begin{itemize}
\item {Utilização:Prov.}
\end{itemize}
\begin{itemize}
\item {Utilização:minh.}
\end{itemize}
\begin{itemize}
\item {Proveniência:(De \textunderscore sarro\textunderscore )}
\end{itemize}
Mascarra, nódoa.
\section{Sarrão}
\begin{itemize}
\item {Grp. gram.:m.}
\end{itemize}
\begin{itemize}
\item {Utilização:Prov.}
\end{itemize}
O mesmo que \textunderscore surrão\textunderscore .
Saco ou taleiga de coiro, em que se levam cereaes ao moínho.
\section{Sarrar}
\begin{itemize}
\item {Grp. gram.:v. t.  e  i.}
\end{itemize}
(Fórma pop. de \textunderscore serrar\textunderscore )
\section{Sarreiro}
\begin{itemize}
\item {Grp. gram.:m.}
\end{itemize}
Homem, que tira e compra o sarro das vasilhas do vinho.
\section{Sarrento}
\begin{itemize}
\item {Grp. gram.:adj.}
\end{itemize}
Que tem sarro; coberto de sarro.
\section{Sarreta}
\begin{itemize}
\item {fónica:rê}
\end{itemize}
\begin{itemize}
\item {Grp. gram.:f.}
\end{itemize}
Cada uma das peças que aguentam os paneiros á ré de um navio.
\section{Sarrico}
\begin{itemize}
\item {Grp. gram.:m.}
\end{itemize}
\begin{itemize}
\item {Utilização:Pesc.}
\end{itemize}
Pequeno xalavar, em que as mulheres e os rapazes deitam o peixe, que saltou da rêde e que podem apanhar.
\section{Sarrido}
\begin{itemize}
\item {Grp. gram.:m.}
\end{itemize}
\begin{itemize}
\item {Utilização:Pop.}
\end{itemize}
\begin{itemize}
\item {Proveniência:(De \textunderscore sarrar\textunderscore ?)}
\end{itemize}
Rouquido de moribundo.
\section{Sarrim}
\begin{itemize}
\item {Grp. gram.:m.}
\end{itemize}
\begin{itemize}
\item {Utilização:Prov.}
\end{itemize}
\begin{itemize}
\item {Utilização:trasm.}
\end{itemize}
\begin{itemize}
\item {Proveniência:(De \textunderscore sarrar\textunderscore )}
\end{itemize}
O mesmo que \textunderscore serradura\textunderscore .
\section{Sarro}
\begin{itemize}
\item {Grp. gram.:m.}
\end{itemize}
Fezes, principalmente depois de sêcas, que o vinho e outros líquidos deixam adherentes ao fundo das vasilhas.
Saburra.
Crosta, formada sôbre os dentes que se não limpam.
Fuligem, que a pólvora queimada deixa nas armas em que ella ardeu.
(Cast. \textunderscore sarro\textunderscore )
\section{Sarroada}
\begin{itemize}
\item {Grp. gram.:f.}
\end{itemize}
\begin{itemize}
\item {Utilização:Prov.}
\end{itemize}
\begin{itemize}
\item {Utilização:beir.}
\end{itemize}
\begin{itemize}
\item {Proveniência:(De \textunderscore sarrão\textunderscore )}
\end{itemize}
Quéda de alguém.
\section{Sarronca}
\begin{itemize}
\item {Grp. gram.:f.}
\end{itemize}
\begin{itemize}
\item {Utilização:Prov.}
\end{itemize}
O mesmo que \textunderscore papão\textunderscore .
\section{Sarronca}
\begin{itemize}
\item {Grp. gram.:f.}
\end{itemize}
Instrumento músico, espanhol.
\section{Sarsará}
\begin{itemize}
\item {Grp. gram.:m.}
\end{itemize}
\begin{itemize}
\item {Utilização:Bras}
\end{itemize}
O mesmo que \textunderscore sarasará\textunderscore .
\section{Sarsório}
\begin{itemize}
\item {Grp. gram.:m.}
\end{itemize}
\begin{itemize}
\item {Proveniência:(Do lat. \textunderscore sarsorium\textunderscore )}
\end{itemize}
Mosaico antigo, feito de mármore variegado.
\section{Sarta}
\begin{itemize}
\item {Grp. gram.:f.}
\end{itemize}
\begin{itemize}
\item {Utilização:Ant.}
\end{itemize}
\begin{itemize}
\item {Proveniência:(Lat. \textunderscore sarta\textunderscore )}
\end{itemize}
Cordame, que se fixa nas antennas do navio.
Enxárcia.
Enfiada.
O mesmo que \textunderscore sartal\textunderscore .
\section{Sartã}
\begin{itemize}
\item {Grp. gram.:f.}
\end{itemize}
\begin{itemize}
\item {Proveniência:(Lat. \textunderscore sartago\textunderscore , \textunderscore sartaginis\textunderscore )}
\end{itemize}
Vaso largo, redondo e pouco fundo, que serve para frigir ovos, peixes, etc.; frigideira.
\section{Sartal}
\begin{itemize}
\item {Grp. gram.:m.}
\end{itemize}
\begin{itemize}
\item {Utilização:Ant.}
\end{itemize}
\begin{itemize}
\item {Proveniência:(De \textunderscore sarta\textunderscore )}
\end{itemize}
Cordões preciosos, para enfeite; fio de pérolas.
\section{Sartan}
\begin{itemize}
\item {Grp. gram.:f.}
\end{itemize}
\begin{itemize}
\item {Proveniência:(Lat. \textunderscore sartago\textunderscore , \textunderscore sartaginis\textunderscore )}
\end{itemize}
Vaso largo, redondo e pouco fundo, que serve para frigir ovos, peixes, etc.; frigideira.
\section{Sartigalho}
\begin{itemize}
\item {Grp. gram.:m.}
\end{itemize}
\begin{itemize}
\item {Utilização:Prov.}
\end{itemize}
\begin{itemize}
\item {Utilização:trasm.}
\end{itemize}
O mesmo que \textunderscore gafanhoto\textunderscore .
(Talvez por \textunderscore saltarigalho\textunderscore , de \textunderscore saltar\textunderscore ; cf. \textunderscore saltão\textunderscore )
\section{Sartório}
\begin{itemize}
\item {Grp. gram.:m.  e  adj.}
\end{itemize}
\begin{itemize}
\item {Utilização:Anat.}
\end{itemize}
\begin{itemize}
\item {Proveniência:(Do lat. \textunderscore sartor\textunderscore )}
\end{itemize}
O mesmo que \textunderscore costureiro\textunderscore .
\section{Sarú}
\begin{itemize}
\item {Grp. gram.:adj.}
\end{itemize}
\begin{itemize}
\item {Utilização:Bras. do N}
\end{itemize}
Diz-se do lago, que está tranquillo, porque não há nelle agitação de peixes, e é infructífera a pescaria.
(Do guar)
\section{Saruê}
\begin{itemize}
\item {Grp. gram.:m.}
\end{itemize}
\begin{itemize}
\item {Utilização:Bras}
\end{itemize}
O mesmo que \textunderscore sarigueia\textunderscore .
\section{Saruga}
\begin{itemize}
\item {Grp. gram.:f.}
\end{itemize}
O mesmo que \textunderscore pragana\textunderscore .
O mesmo que \textunderscore saluga\textunderscore .
\section{Saruma}
\begin{itemize}
\item {Grp. gram.:f.}
\end{itemize}
Planta medicinal da Guiana inglesa.
\section{Sarumás}
\begin{itemize}
\item {Grp. gram.:m. pl.}
\end{itemize}
\begin{itemize}
\item {Utilização:Bras}
\end{itemize}
Tríbo de aborígenes de Mato-Grosso.
\section{Sassa}
\begin{itemize}
\item {Grp. gram.:f.}
\end{itemize}
Árvore corpulenta da Núbia, espécie de mimosa.
\section{Sassafraz}
\begin{itemize}
\item {Grp. gram.:m.}
\end{itemize}
Nome de duas árvores lauráceas da América.
\section{Sassânidas}
\begin{itemize}
\item {Grp. gram.:m. pl.}
\end{itemize}
Antiga dynastia persa. Cf. V. Abreu, \textunderscore Contos da Índia\textunderscore .
\section{Sasse}
\begin{itemize}
\item {Grp. gram.:m.}
\end{itemize}
Arbusto trepador de Angola, (\textunderscore mesoneurum angolense\textunderscore , Welw.).
\section{Sassi}
\begin{itemize}
\item {Grp. gram.:m.}
\end{itemize}
\begin{itemize}
\item {Utilização:Bras}
\end{itemize}
Ave, o mesmo que \textunderscore alma-de-gato\textunderscore .
\section{Sassi-pererê}
\begin{itemize}
\item {Grp. gram.:m.}
\end{itemize}
\begin{itemize}
\item {Utilização:Bras. do Rio}
\end{itemize}
O diabo.
\section{Sassor}
\begin{itemize}
\item {Grp. gram.:m.}
\end{itemize}
\begin{itemize}
\item {Utilização:Gal}
\end{itemize}
\begin{itemize}
\item {Proveniência:(Do fr. \textunderscore sasseur\textunderscore )}
\end{itemize}
Espécie de peneira. Cf. \textunderscore Inquér. Industr.\textunderscore , P. I, l. 1.^o, 198, e P. II, l. 3.^o, 41.
\section{Sastre}
\begin{itemize}
\item {Grp. gram.:m.}
\end{itemize}
\begin{itemize}
\item {Utilização:Ant.}
\end{itemize}
O mesmo que \textunderscore alfaiate\textunderscore .
(Cp. \textunderscore chastre\textunderscore )
\section{Satagana}
\begin{itemize}
\item {Grp. gram.:f.}
\end{itemize}
Planta da Índia portuguesa, (\textunderscore casearia esculenta\textunderscore , Roxb.).
\section{Satan}
\begin{itemize}
\item {Grp. gram.:m.}
\end{itemize}
\begin{itemize}
\item {Proveniência:(Do hebr. \textunderscore Satan\textunderscore , n. p.)}
\end{itemize}
O mesmo que \textunderscore Satanás\textunderscore .
\section{Satanás}
\begin{itemize}
\item {Grp. gram.:m.}
\end{itemize}
\begin{itemize}
\item {Utilização:Ext.}
\end{itemize}
\begin{itemize}
\item {Proveniência:(Lat. \textunderscore Satanas\textunderscore )}
\end{itemize}
Nome, que a \textunderscore Biblia\textunderscore  dá ao chefe dos anjos rebeldes.
Diabo.
\section{Satanicamente}
\begin{itemize}
\item {Grp. gram.:adv.}
\end{itemize}
De modo satânico; diabolicamente.
\section{Satânico}
\begin{itemize}
\item {Grp. gram.:adj.}
\end{itemize}
Relativo a Satan; diabólico.
\section{Satanismo}
\begin{itemize}
\item {Grp. gram.:m.}
\end{itemize}
\begin{itemize}
\item {Proveniência:(De \textunderscore Satan\textunderscore )}
\end{itemize}
Qualidade do que é satânico. Cf. Pato, \textunderscore Ciprestes\textunderscore , 320.
\section{Satanizar}
\begin{itemize}
\item {Grp. gram.:v. t.}
\end{itemize}
Dar modos ou aspecto de Satanás a.
\section{Satão}
\begin{itemize}
\item {Grp. gram.:m.}
\end{itemize}
O mesmo que \textunderscore satanás\textunderscore . Cf. G. Vicente, I, 259.
\section{Satária}
\begin{itemize}
\item {Grp. gram.:f.}
\end{itemize}
Planta da serra de Sintra.
\section{Satelício}
\begin{itemize}
\item {Grp. gram.:m.}
\end{itemize}
\begin{itemize}
\item {Utilização:P. us.}
\end{itemize}
\begin{itemize}
\item {Proveniência:(Lat. \textunderscore satellitium\textunderscore )}
\end{itemize}
Guarda; escolta.
\section{Satélite}
\begin{itemize}
\item {Grp. gram.:m.}
\end{itemize}
\begin{itemize}
\item {Grp. gram.:Adj.}
\end{itemize}
\begin{itemize}
\item {Utilização:Anat.}
\end{itemize}
\begin{itemize}
\item {Proveniência:(Lat. \textunderscore satelles\textunderscore , \textunderscore satellitis\textunderscore )}
\end{itemize}
Indivíduo assalariado, que acompanha ou auxilía outro em más obras.
Guarda-costas.
Companheiro inseparável.
Planeta, que gira em volta de outro maior.
Quási parallelo ás artérias, (falando-se de certos nervos e veias).
\section{Satellício}
\begin{itemize}
\item {Grp. gram.:m.}
\end{itemize}
\begin{itemize}
\item {Utilização:P. us.}
\end{itemize}
\begin{itemize}
\item {Proveniência:(Lat. \textunderscore satellitium\textunderscore )}
\end{itemize}
Guarda; escolta.
\section{Satéllite}
\begin{itemize}
\item {Grp. gram.:m.}
\end{itemize}
\begin{itemize}
\item {Grp. gram.:Adj.}
\end{itemize}
\begin{itemize}
\item {Utilização:Anat.}
\end{itemize}
\begin{itemize}
\item {Proveniência:(Lat. \textunderscore satelles\textunderscore , \textunderscore satellitis\textunderscore )}
\end{itemize}
Indivíduo assalariado, que acompanha ou auxilía outro em más obras.
Guarda-costas.
Companheiro inseparável.
Planeta, que gira em volta de outro maior.
Quási parallelo ás artérias, (falando-se de certos nervos e veias).
\section{Sateposa}
\begin{itemize}
\item {Grp. gram.:f.}
\end{itemize}
Antigo pano de algodão, fabricado em Bengala.
\section{Satilhas}
\begin{itemize}
\item {Grp. gram.:f.}
\end{itemize}
Planta solânea, (\textunderscore withania somnifera\textunderscore , Dunal).
\section{Sátira}
\begin{itemize}
\item {Grp. gram.:f.}
\end{itemize}
\begin{itemize}
\item {Proveniência:(Lat. \textunderscore satira\textunderscore )}
\end{itemize}
Composição poética, que tem por fim censurar ou ridiculizar defeitos ou vícios.
Censura jocosa.
\section{Satirão}
\begin{itemize}
\item {Grp. gram.:m.}
\end{itemize}
Planta esterculiácea da Índia Portuguesa, (\textunderscore sterculia foetida\textunderscore , Lin.).
Planta verbenácea da Índia Portuguesa, (\textunderscore vitex leucoxylon\textunderscore , Lin.).
\section{Satirião}
\begin{itemize}
\item {Grp. gram.:m.}
\end{itemize}
\begin{itemize}
\item {Proveniência:(Lat. \textunderscore satyrion\textunderscore )}
\end{itemize}
Nome de três plantas orquidáceas, o \textunderscore satirião macho\textunderscore , o \textunderscore satirião menor\textunderscore  e o \textunderscore satirião bastardo\textunderscore .
\section{Satiríase}
\begin{itemize}
\item {Grp. gram.:f.}
\end{itemize}
\begin{itemize}
\item {Proveniência:(Lat. \textunderscore satyriasis\textunderscore )}
\end{itemize}
Irritação dos órgãos genitaes, acompanhada de desejos libidinosos.
\section{Satiricamente}
\begin{itemize}
\item {Grp. gram.:adv.}
\end{itemize}
De modo satírico.
Á maneira de sátira; ironicamente.
\section{Satírico}
\begin{itemize}
\item {Grp. gram.:adj.}
\end{itemize}
\begin{itemize}
\item {Utilização:Ext.}
\end{itemize}
Relativo a sátira.
Que satiriza ou que envolve sátira.
Mordaz; picante.
\section{Satirídeos}
\begin{itemize}
\item {Grp. gram.:m. pl.}
\end{itemize}
\begin{itemize}
\item {Proveniência:(Do gr. \textunderscore saturos\textunderscore  + \textunderscore eidos\textunderscore )}
\end{itemize}
Família de lepidópteros, a que pertence a borboleta chamada sátiro.
\section{Satirista}
\begin{itemize}
\item {Grp. gram.:m.}
\end{itemize}
\begin{itemize}
\item {Utilização:Des.}
\end{itemize}
Aquelle que faz sátiras; aquelle que é maledicente. Cf. Fern. Oliveira, \textunderscore Livro da Fábr. das Naus\textunderscore , 87.
\section{Satirizador}
\begin{itemize}
\item {Grp. gram.:adj.}
\end{itemize}
Que satiriza. Cf. Filinto, VVIII, 186.
\section{Satirizar}
\begin{itemize}
\item {Grp. gram.:v. t.}
\end{itemize}
Fazer sátira contra.
Criticar com sátira.
Dirigir motejos picantes a; ridiculizar.
\section{Satisdação}
\begin{itemize}
\item {Grp. gram.:f.}
\end{itemize}
\begin{itemize}
\item {Utilização:Ant.}
\end{itemize}
\begin{itemize}
\item {Proveniência:(Lat. \textunderscore satisdatio\textunderscore )}
\end{itemize}
Caução; fiança.
\section{Satisdar}
\begin{itemize}
\item {Grp. gram.:v. i.}
\end{itemize}
\begin{itemize}
\item {Proveniência:(Lat. \textunderscore satisdare\textunderscore )}
\end{itemize}
Dar fiança, prestar caução.
\section{Satisfação}
\begin{itemize}
\item {Grp. gram.:f.}
\end{itemize}
\begin{itemize}
\item {Proveniência:(Lat. \textunderscore satisfactio\textunderscore )}
\end{itemize}
Acto ou effeito de satisfazer.
Aprazimento; contentamento.
Sentimento de approvação.
Pagamento: \textunderscore satisfação de dívidas\textunderscore .
Desempenho: \textunderscore satisfação de deveres\textunderscore .
Indemnização.
Expiação; castigo.
Desculpa.
Retractação: \textunderscore exigir satisfações\textunderscore .
\section{Satisfacção}
\begin{itemize}
\item {Grp. gram.:f.}
\end{itemize}
\begin{itemize}
\item {Proveniência:(Lat. \textunderscore satisfactio\textunderscore )}
\end{itemize}
Acto ou effeito de satisfazer.
Aprazimento; contentamento.
Sentimento de approvação.
Pagamento: \textunderscore satisfação de dívidas\textunderscore .
Desempenho: \textunderscore satisfação de deveres\textunderscore .
Indemnização.
Expiação; castigo.
Desculpa.
Retractação: \textunderscore exigir satisfações\textunderscore .
\section{Satisfactoriamente}
\begin{itemize}
\item {Grp. gram.:adv.}
\end{itemize}
De modo satisfactório.
Com satisfacção; de modo acceitável, regular: \textunderscore no exame, respondeu satisfactoriamente\textunderscore .
\section{Satisfactório}
\begin{itemize}
\item {Grp. gram.:adj.}
\end{itemize}
\begin{itemize}
\item {Proveniência:(Lat. \textunderscore satisfactorius\textunderscore )}
\end{itemize}
Que póde satisfazer.
Acceitável.
Soffrível; regular.
\section{Satisfazer}
\begin{itemize}
\item {Grp. gram.:v. i.}
\end{itemize}
\begin{itemize}
\item {Grp. gram.:V. t.}
\end{itemize}
\begin{itemize}
\item {Grp. gram.:V. p.}
\end{itemize}
\begin{itemize}
\item {Proveniência:(Lat. \textunderscore satisfacere\textunderscore )}
\end{itemize}
Sêr bastante ou sufficiente.
Corresponder ao que se deseja: \textunderscore é um estudante que satisfaz\textunderscore .
Chegar a certa medida ou limites.
Dar contentamento ou prazer.
Fazer o que se deve em relação a alguém ou a alguma coisa.
Adaptar-se.
Corresponder.
Obviar.
Pagar: \textunderscore satisfazer dividas\textunderscore .
Realizar.
Saciar; mitigar: \textunderscore satisfazer a fome\textunderscore .
Indemnizar.
Convencer.
Dar boa solução a.
Saciar-se.
Vingar-se.
Contentar-se: \textunderscore satisfaz-se com uma só refeição diária\textunderscore .
\section{Satisfazimento}
\begin{itemize}
\item {Grp. gram.:m.}
\end{itemize}
\begin{itemize}
\item {Utilização:Ant.}
\end{itemize}
O mesmo que \textunderscore satisfacção\textunderscore .
\section{Satisfeito}
\begin{itemize}
\item {Grp. gram.:adj.}
\end{itemize}
\begin{itemize}
\item {Proveniência:(Lat. \textunderscore satisfactus\textunderscore )}
\end{itemize}
Saciado; repleto.
Que se satisfez.
Contente: \textunderscore hoje estou satisfeito\textunderscore .
Executado, realizado: \textunderscore o teu pedido está satisfeito\textunderscore .
\section{Sativo}
\begin{itemize}
\item {Grp. gram.:adj.}
\end{itemize}
\begin{itemize}
\item {Proveniência:(Lat. \textunderscore sativus\textunderscore )}
\end{itemize}
Que se semeia ou se cultiva.
\section{Sátrapa}
\begin{itemize}
\item {Grp. gram.:m.}
\end{itemize}
\begin{itemize}
\item {Utilização:Fig.}
\end{itemize}
Título dos antigos governadores de províncias, entre os Persas.
Homem poderoso.
Grande dignitário.
Sybarita.
(Cp. \textunderscore sátrape\textunderscore )
\section{Sátrape}
\begin{itemize}
\item {Grp. gram.:m.}
\end{itemize}
\begin{itemize}
\item {Proveniência:(Lat. \textunderscore satrapes\textunderscore )}
\end{itemize}
O mesmo ou melhor que \textunderscore sátrapa\textunderscore .
\section{Satrapear}
\begin{itemize}
\item {Grp. gram.:v. i.}
\end{itemize}
\begin{itemize}
\item {Utilização:Neol.}
\end{itemize}
\begin{itemize}
\item {Proveniência:(De \textunderscore sátrapa\textunderscore )}
\end{itemize}
Alardear de grande senhor.
Governar despoticamente.
\section{Satrapia}
\begin{itemize}
\item {Grp. gram.:f.}
\end{itemize}
\begin{itemize}
\item {Proveniência:(Lat. \textunderscore satrapea\textunderscore )}
\end{itemize}
Cargo ou governo de um sátrapa.
\section{Saturabilidade}
\begin{itemize}
\item {Grp. gram.:f.}
\end{itemize}
Qualidade daquillo que é saturável.
\section{Saturação}
\begin{itemize}
\item {Grp. gram.:f.}
\end{itemize}
\begin{itemize}
\item {Proveniência:(Do lat. \textunderscore saturatio\textunderscore )}
\end{itemize}
Acto ou effeito de saturar.
\section{Saturador}
\begin{itemize}
\item {Grp. gram.:adj.}
\end{itemize}
\begin{itemize}
\item {Grp. gram.:M.}
\end{itemize}
\begin{itemize}
\item {Proveniência:(Do lat. \textunderscore saturator\textunderscore )}
\end{itemize}
Que satura.
Apparelho, para saturar líquidos.
\section{Saturagem}
\begin{itemize}
\item {Grp. gram.:f.}
\end{itemize}
Planta, o mesmo que \textunderscore segurelha\textunderscore ^2.
\section{Saturamento}
\begin{itemize}
\item {Grp. gram.:m.}
\end{itemize}
O mesmo que \textunderscore saturação\textunderscore .
\section{Saturante}
\begin{itemize}
\item {Grp. gram.:adj.}
\end{itemize}
\begin{itemize}
\item {Proveniência:(Lat. \textunderscore satturans\textunderscore )}
\end{itemize}
Que satura.
\section{Saturar}
\begin{itemize}
\item {Grp. gram.:v. t.}
\end{itemize}
\begin{itemize}
\item {Proveniência:(Lat. \textunderscore saturare\textunderscore )}
\end{itemize}
Encher completamente.
Impregnar.
Saciar, satisfazer.
Fazer impar.
Dissolver em (um líquido ou um gás) a máxima quantidade de uma substância.
Combinar com (um corpo) a máxima quantidade de uma substância.
\section{Saturável}
\begin{itemize}
\item {Grp. gram.:adj.}
\end{itemize}
\begin{itemize}
\item {Proveniência:(Do lat. \textunderscore saturabilis\textunderscore )}
\end{itemize}
Que se póde saturar.
\section{Saturnal}
\begin{itemize}
\item {Grp. gram.:adj.}
\end{itemize}
\begin{itemize}
\item {Grp. gram.:F.}
\end{itemize}
\begin{itemize}
\item {Utilização:Fig.}
\end{itemize}
\begin{itemize}
\item {Grp. gram.:Pl.}
\end{itemize}
\begin{itemize}
\item {Proveniência:(Lat. \textunderscore saturnalis\textunderscore )}
\end{itemize}
Relativo a Saturno ou ás festas em honra de Saturno.
Orgia.
Festim de libertinos.
Devassidão.
Antigas festas, em honra de Saturno.
\section{Saturnianos}
\begin{itemize}
\item {Grp. gram.:m. pl.}
\end{itemize}
Herejes, que negavam ao homem o livre arbítrio e, portanto, o mérito e demérito das acções humanas.
\section{Saturnino}
\begin{itemize}
\item {Grp. gram.:adj.}
\end{itemize}
\begin{itemize}
\item {Proveniência:(De \textunderscore Saturno\textunderscore )}
\end{itemize}
O mesmo que \textunderscore saturnal\textunderscore .
Relativo ao chumbo e aos seus compostos.
Causado pelo chumbo, (falando-se de certas doenças).
\section{Satúrnio}
\begin{itemize}
\item {Grp. gram.:adj.}
\end{itemize}
O mesmo que \textunderscore saturnino\textunderscore .
Influenciado por Saturno:«\textunderscore ...os saturníos fados meus.\textunderscore »Filinto, III, 102.
\section{Saturnismo}
\begin{itemize}
\item {Grp. gram.:m.}
\end{itemize}
\begin{itemize}
\item {Proveniência:(De \textunderscore saturno\textunderscore )}
\end{itemize}
Doença ou envenenamento de pessôas que lidam com o chumbo em certas indústrias, e das que abusam do rapé contido em invólucros de chumbo.
\section{Saturno}
\begin{itemize}
\item {Grp. gram.:m.}
\end{itemize}
\begin{itemize}
\item {Utilização:Pop.}
\end{itemize}
\begin{itemize}
\item {Grp. gram.:Adj.}
\end{itemize}
\begin{itemize}
\item {Proveniência:(Lat. \textunderscore Saturnus\textunderscore , n. p.)}
\end{itemize}
Um dos planetas do systema solar, cuja revolução se faz em 29 annos e meio.
O tempo.
O chumbo, (porque se suppõe que êste era o metal mais antigo e pai de todos).
Grande calor, sem aragem.
Tempo quente e abafadiço.
Saturnal, orgíaco. Cf. Filinto, IX, 158.
\section{Satyrião}
\begin{itemize}
\item {Grp. gram.:m.}
\end{itemize}
\begin{itemize}
\item {Proveniência:(Lat. \textunderscore satyrion\textunderscore )}
\end{itemize}
Nome de três plantas orchidáceas, o \textunderscore satyrião macho\textunderscore , o \textunderscore satyrião menor\textunderscore  e o \textunderscore satyrião bastardo\textunderscore .
\section{Satyríase}
\begin{itemize}
\item {Grp. gram.:f.}
\end{itemize}
\begin{itemize}
\item {Proveniência:(Lat. \textunderscore satyriasis\textunderscore )}
\end{itemize}
Irritação dos órgãos genitaes, acompanhada de desejos libidinosos.
\section{Satyrídeos}
\begin{itemize}
\item {Grp. gram.:m. pl.}
\end{itemize}
\begin{itemize}
\item {Proveniência:(Do gr. \textunderscore saturos\textunderscore  + \textunderscore eidos\textunderscore )}
\end{itemize}
Família de lepidópteros, a que pertence a borboleta chamada sátyro.
\section{Satírio}
\begin{itemize}
\item {Grp. gram.:m.}
\end{itemize}
\begin{itemize}
\item {Utilização:Bot.}
\end{itemize}
O mesmo que \textunderscore satirião\textunderscore . Cf. R. Pereira, \textunderscore Prosódia\textunderscore , vb. \textunderscore secacul\textunderscore .
\section{Sátiro}
\begin{itemize}
\item {Grp. gram.:m.}
\end{itemize}
\begin{itemize}
\item {Utilização:Fig.}
\end{itemize}
\begin{itemize}
\item {Proveniência:(Lat. \textunderscore satyrus\textunderscore )}
\end{itemize}
Semi-deus, que, segundo os Pagãos, tinha pés e pernas de bode e habitava nas florestas.
Gênero de borboletas diurnas.
Homem libidinoso, devasso.
\section{Satýrio}
\begin{itemize}
\item {Grp. gram.:m.}
\end{itemize}
\begin{itemize}
\item {Utilização:Bot.}
\end{itemize}
O mesmo que \textunderscore satyrião\textunderscore . Cf. R. Pereira, \textunderscore Prosódia\textunderscore , vb. \textunderscore secacul\textunderscore .
\section{Sátyro}
\begin{itemize}
\item {Grp. gram.:m.}
\end{itemize}
\begin{itemize}
\item {Utilização:Fig.}
\end{itemize}
\begin{itemize}
\item {Proveniência:(Lat. \textunderscore satyrus\textunderscore )}
\end{itemize}
Semi-deus, que, segundo os Pagãos, tinha pés e pernas de bode e habitava nas florestas.
Gênero de borboletas diurnas.
Homem libidinoso, devasso.
\section{Sauaçu}
\begin{itemize}
\item {Grp. gram.:m.}
\end{itemize}
\begin{itemize}
\item {Utilização:Bras}
\end{itemize}
Espécie de macaco.
\section{Saúba}
\begin{itemize}
\item {Grp. gram.:f.}
\end{itemize}
\begin{itemize}
\item {Utilização:Bras}
\end{itemize}
Espécie de formiga (\textunderscore oecodoma cephalotes\textunderscore ), muito nociva aos pomares e a outras plantações.
\section{Sauco}
\begin{itemize}
\item {Grp. gram.:m.}
\end{itemize}
\begin{itemize}
\item {Utilização:Veter.}
\end{itemize}
Parte do casco das bêstas, entre a tapa e a palma.
(Cast. \textunderscore sauco\textunderscore , sabugo)
\section{Saudação}
\begin{itemize}
\item {fónica:sa-u}
\end{itemize}
\begin{itemize}
\item {Grp. gram.:f.}
\end{itemize}
\begin{itemize}
\item {Proveniência:(Do lat. \textunderscore salutatio\textunderscore )}
\end{itemize}
Acto ou effeito de saudar; cumprimentos.
Homenagens de respeito ou de admiração.
\section{Saudade}
\begin{itemize}
\item {fónica:sa-u}
\end{itemize}
\begin{itemize}
\item {Grp. gram.:f.}
\end{itemize}
\begin{itemize}
\item {Utilização:Bot.}
\end{itemize}
\begin{itemize}
\item {Utilização:Bras. de Pernambuco}
\end{itemize}
\begin{itemize}
\item {Grp. gram.:Pl.}
\end{itemize}
Lembrança triste e suave de pessôas ou coisas distantes ou extintas, acompanhada do desejo de as tornar a possuir ou vêr presentes.
Pesar, pela ausência de alguém que nos é querido.
Nostalgia.
Nome de várias plantas ou da sua flôr.
Planta, o mesmo que \textunderscore official-de-sala\textunderscore .
Cumprimentos, lembranças affectuosas, dirigidas a pessôas ausentes: \textunderscore dê-lhe saudades\textunderscore .
(Alter, de \textunderscore soidade\textunderscore , de \textunderscore soledade\textunderscore )
\section{Saudador}
\begin{itemize}
\item {fónica:sa-u}
\end{itemize}
\begin{itemize}
\item {Grp. gram.:m.  e  adj.}
\end{itemize}
\begin{itemize}
\item {Proveniência:(Do lat. \textunderscore salutator\textunderscore )}
\end{itemize}
O que saúda.
\section{Saudante}
\begin{itemize}
\item {fónica:sa-u}
\end{itemize}
\begin{itemize}
\item {Grp. gram.:adj.}
\end{itemize}
\begin{itemize}
\item {Proveniência:(Do lat. \textunderscore salutans\textunderscore )}
\end{itemize}
Que saúda.
\section{Saudar}
\begin{itemize}
\item {fónica:sa-u}
\end{itemize}
\begin{itemize}
\item {Grp. gram.:v. t.}
\end{itemize}
\begin{itemize}
\item {Grp. gram.:M.}
\end{itemize}
\begin{itemize}
\item {Proveniência:(Do lat. \textunderscore salutare\textunderscore )}
\end{itemize}
Desejar saúde a.
Cumprimentar.
Felicitar.
Salvar.
Acclamar.
Sentir júbilo á vista de.
Louvar.
O mesmo que \textunderscore saudação\textunderscore .
\section{Saudável}
\begin{itemize}
\item {fónica:sa-u}
\end{itemize}
\begin{itemize}
\item {Grp. gram.:adj.}
\end{itemize}
\begin{itemize}
\item {Utilização:Ext.}
\end{itemize}
\begin{itemize}
\item {Proveniência:(De \textunderscore saúde\textunderscore )}
\end{itemize}
Conveniente para a saúde; salutar; hygiênico: \textunderscore uma casa saudável\textunderscore .
Benéfico; útil.
\section{Saudavelmente}
\begin{itemize}
\item {fónica:sa-u}
\end{itemize}
\begin{itemize}
\item {Grp. gram.:adv.}
\end{itemize}
De modo saudável.
\section{Saúde}
\begin{itemize}
\item {Grp. gram.:f.}
\end{itemize}
\begin{itemize}
\item {Proveniência:(Do lat. \textunderscore salus\textunderscore , \textunderscore salutis\textunderscore )}
\end{itemize}
Estado do que é são, ou de quem tem as funcções orgânicas no seu estado normal.
Vigor.
Bôa ou má disposição no organismo de um indivíduo.
Saudação, brinde, acto de se beber á mesa em homenagem ou lembrança affectuosa de alguém: \textunderscore ao jantar, houve muitas saúdes\textunderscore .
\section{Saudosamente}
\begin{itemize}
\item {fónica:sa-u}
\end{itemize}
\begin{itemize}
\item {Grp. gram.:adv.}
\end{itemize}
De modo saudoso.
Com saudade.
\section{Saudoso}
\begin{itemize}
\item {fónica:sa-u}
\end{itemize}
\begin{itemize}
\item {Grp. gram.:adj.}
\end{itemize}
Que produz saudades: \textunderscore nos saudosos tempos da meninice\textunderscore .
Que sente saudades.
\section{Saúga}
\begin{itemize}
\item {Grp. gram.:f.}
\end{itemize}
(Por. \textunderscore saúba\textunderscore , no \textunderscore Vocab. Bras.\textunderscore  de B. C. Rubim)
\section{Sauguate}
\begin{itemize}
\item {Grp. gram.:m.}
\end{itemize}
Doce de fruta, na China.
O mesmo que \textunderscore saguate\textunderscore . Cf. \textunderscore Ásia Sínica\textunderscore , 60.
\section{Sauí}
\begin{itemize}
\item {Grp. gram.:m.}
\end{itemize}
O mesmo que \textunderscore saguim\textunderscore .
\section{Sauiá}
\begin{itemize}
\item {Grp. gram.:m.}
\end{itemize}
\begin{itemize}
\item {Utilização:Bras. do N}
\end{itemize}
Pequena cotia, com cauda.
\section{Saupé}
\begin{itemize}
\item {Grp. gram.:m.}
\end{itemize}
\begin{itemize}
\item {Utilização:Bras}
\end{itemize}
Peixe fluvial.
\section{Saurim}
\begin{itemize}
\item {Grp. gram.:m.}
\end{itemize}
Espécie de pano antigo, que vinha da Índia.
\section{Sáurios}
\begin{itemize}
\item {Grp. gram.:m. pl.}
\end{itemize}
\begin{itemize}
\item {Proveniência:(Do lat. \textunderscore sauros\textunderscore )}
\end{itemize}
Ordem de reptis, que tem por typo o lagarto.
\section{Saurite}
\begin{itemize}
\item {Grp. gram.:f.}
\end{itemize}
\begin{itemize}
\item {Proveniência:(Lat. \textunderscore sauritis\textunderscore )}
\end{itemize}
Variedade de pedra que, segundo a crença dos antigos, se encontrava no ventre de um lagarto.
\section{Saurófago}
\begin{itemize}
\item {Grp. gram.:adj.}
\end{itemize}
\begin{itemize}
\item {Utilização:Zool.}
\end{itemize}
\begin{itemize}
\item {Proveniência:(Do gr. \textunderscore saura\textunderscore  + \textunderscore phagein\textunderscore )}
\end{itemize}
Diz-se do animal, que come sáurios ou lagartos.
\section{Sauroglosso}
\begin{itemize}
\item {Grp. gram.:m.}
\end{itemize}
\begin{itemize}
\item {Proveniência:(Do gr. \textunderscore saura\textunderscore  + \textunderscore glossa\textunderscore )}
\end{itemize}
Gênero de orchídeas.
\section{Saurografia}
\begin{itemize}
\item {Grp. gram.:f.}
\end{itemize}
Descripção ou tratado dos reptis sáurios.
(Cp. \textunderscore saurógrafo\textunderscore )
\section{Saurográfico}
\begin{itemize}
\item {Grp. gram.:adj.}
\end{itemize}
Relativo á saurografia.
\section{Saurógrafo}
\begin{itemize}
\item {Grp. gram.:m.}
\end{itemize}
\begin{itemize}
\item {Proveniência:(Do gr. \textunderscore saura\textunderscore  + \textunderscore graphein\textunderscore )}
\end{itemize}
Aquele que escreve de saurografia.
\section{Saurographia}
\begin{itemize}
\item {Grp. gram.:f.}
\end{itemize}
Descripção ou tratado dos reptis sáurios.
(Cp. \textunderscore saurógrapho\textunderscore )
\section{Saurográphico}
\begin{itemize}
\item {Grp. gram.:adj.}
\end{itemize}
Relativo á saurographia.
\section{Saurógrapho}
\begin{itemize}
\item {Grp. gram.:m.}
\end{itemize}
\begin{itemize}
\item {Proveniência:(Do gr. \textunderscore saura\textunderscore  + \textunderscore graphein\textunderscore )}
\end{itemize}
Aquelle que escreve de saurographia.
\section{Saurologia}
\begin{itemize}
\item {Grp. gram.:f.}
\end{itemize}
Parte da Zoologia, que trata dos reptis sáurios.
(Cp. \textunderscore saurólogo\textunderscore )
\section{Saurológico}
\begin{itemize}
\item {Grp. gram.:adj.}
\end{itemize}
Relativo á saurologia.
\section{Saurólogo}
\begin{itemize}
\item {Grp. gram.:m.}
\end{itemize}
\begin{itemize}
\item {Proveniência:(Do gr. \textunderscore saura\textunderscore  + \textunderscore logos\textunderscore )}
\end{itemize}
Aquelle que é períto em saurologia.
\section{Saurómato}
\begin{itemize}
\item {Grp. gram.:m.}
\end{itemize}
Gênero de plantas aráceas da Índia.
\section{Sauromorfo}
\begin{itemize}
\item {Grp. gram.:m.}
\end{itemize}
\begin{itemize}
\item {Proveniência:(Do gr. \textunderscore saura\textunderscore  + \textunderscore morphe\textunderscore )}
\end{itemize}
Gênero de insectos coleópteros pentâmeros.
\section{Sauromorpho}
\begin{itemize}
\item {Grp. gram.:m.}
\end{itemize}
\begin{itemize}
\item {Proveniência:(Do gr. \textunderscore saura\textunderscore  + \textunderscore morphe\textunderscore )}
\end{itemize}
Gênero de insectos coleópteros pentâmeros.
\section{Sauróphago}
\begin{itemize}
\item {Grp. gram.:adj.}
\end{itemize}
\begin{itemize}
\item {Utilização:Zool.}
\end{itemize}
\begin{itemize}
\item {Proveniência:(Do gr. \textunderscore saura\textunderscore  + \textunderscore phagein\textunderscore )}
\end{itemize}
Diz-se do animal, que come sáurios ou lagartos.
\section{Sauropo}
\begin{itemize}
\item {Grp. gram.:m.}
\end{itemize}
\begin{itemize}
\item {Proveniência:(Do gr. \textunderscore saura\textunderscore  + \textunderscore pous\textunderscore , \textunderscore podos\textunderscore )}
\end{itemize}
Gênero de plantas euphorbiácia de Java.
\section{Saurópside}
\begin{itemize}
\item {Grp. gram.:m.}
\end{itemize}
\begin{itemize}
\item {Proveniência:(Do gr. \textunderscore saura\textunderscore  + \textunderscore ops\textunderscore )}
\end{itemize}
Gênero de peixes fósseis dos terrenos jurásicos.
\section{Sauropterígios}
\begin{itemize}
\item {Grp. gram.:m. pl.}
\end{itemize}
\begin{itemize}
\item {Proveniência:(Do gr. \textunderscore saura\textunderscore  + \textunderscore pterux\textunderscore )}
\end{itemize}
Espécie de reptis marinhos, fósseis, do período mesozoico.
\section{Sauropterýgios}
\begin{itemize}
\item {Grp. gram.:m. pl.}
\end{itemize}
\begin{itemize}
\item {Proveniência:(Do gr. \textunderscore saura\textunderscore  + \textunderscore pterux\textunderscore )}
\end{itemize}
Espécie de reptis marinhos, fósseis, do período mesozoico.
\section{Saurúreas}
\begin{itemize}
\item {Grp. gram.:f. pl.}
\end{itemize}
\begin{itemize}
\item {Proveniência:(De \textunderscore saururo\textunderscore )}
\end{itemize}
Família de plantas monocotyledóneas da América do Norte e da Ásia oriental.
\section{Saururo}
\begin{itemize}
\item {Grp. gram.:m.}
\end{itemize}
\begin{itemize}
\item {Proveniência:(Do gr. \textunderscore saura\textunderscore  + \textunderscore oura\textunderscore )}
\end{itemize}
Gênero de plantas, que serve de typo ás saurúreas.
\section{Saussurite}
\begin{itemize}
\item {Grp. gram.:f.}
\end{itemize}
\begin{itemize}
\item {Utilização:Bras}
\end{itemize}
\begin{itemize}
\item {Proveniência:(De \textunderscore Saussure\textunderscore , n. p.)}
\end{itemize}
Silicato alcalino de alumina e cal.
Pedra que, talhada pelas amazonas, ellas lançam ao pescoço dos Guacarés, quando êstes as visitam, e que serve para que ellas os reconheçam quando elles voltem a procurá-las. Cf. \textunderscore Cartas da América\textunderscore , por Lopes Mendes, no \textunderscore Boletim da Socied. de Geogr. de Lisbôa\textunderscore .
\section{Sautéria}
\begin{itemize}
\item {Grp. gram.:f.}
\end{itemize}
Gênero de plantas hepáticas, que crescem nas altas montanhas.
\section{Sautór}
\begin{itemize}
\item {Grp. gram.:m.}
\end{itemize}
\begin{itemize}
\item {Utilização:Heráld.}
\end{itemize}
\begin{itemize}
\item {Proveniência:(Do fr. \textunderscore sautoir\textunderscore )}
\end{itemize}
O mesmo que \textunderscore aspa\textunderscore . Cf. Camillo, \textunderscore Narcót.\textunderscore , II, 100.--Os diccion. portugueses dizem \textunderscore santor\textunderscore , o que é êrro evidente.
\section{Savacu}
\begin{itemize}
\item {Grp. gram.:m.}
\end{itemize}
Gênero de aves.
\section{Sàval}
\begin{itemize}
\item {Grp. gram.:m.}
\end{itemize}
\begin{itemize}
\item {Utilização:Pesc.}
\end{itemize}
\begin{itemize}
\item {Proveniência:(De \textunderscore sável\textunderscore )}
\end{itemize}
Rêde de emmalhar.
\section{Savana}
\begin{itemize}
\item {Grp. gram.:f.}
\end{itemize}
\begin{itemize}
\item {Proveniência:(Do fr. \textunderscore savanne\textunderscore )}
\end{itemize}
Larga planície, que só produz pastagens, nas Antilhas, Guiana, etc.
Floresta de árvores resinosas, no Canadá.
\section{Savandija}
\begin{itemize}
\item {Grp. gram.:m.  e  f.}
\end{itemize}
O mesmo ou melhor que \textunderscore sevandija\textunderscore . Cf. \textunderscore Luz e Calor\textunderscore , 539.
(Cast. \textunderscore sabandija\textunderscore )
\section{Savará}
\begin{itemize}
\item {Grp. gram.:m.}
\end{itemize}
Língua da província de Madrasta e pertencente ao grupo decânico.
\section{Savarim}
\begin{itemize}
\item {Grp. gram.:m.}
\end{itemize}
\begin{itemize}
\item {Proveniência:(De \textunderscore Savarin\textunderscore , n. p.)}
\end{itemize}
Espécie de pudim.
\section{Savasto}
\begin{itemize}
\item {Grp. gram.:m.}
\end{itemize}
O mesmo que \textunderscore sebasto\textunderscore .
\section{Savastro}
\begin{itemize}
\item {Grp. gram.:m.}
\end{itemize}
\begin{itemize}
\item {Utilização:Ant.}
\end{itemize}
O mesmo que \textunderscore sebasto\textunderscore .
\section{Sàveira}
\begin{itemize}
\item {Grp. gram.:f.}
\end{itemize}
\begin{itemize}
\item {Proveniência:(De \textunderscore sàveiro\textunderscore )}
\end{itemize}
Mulhér, que dirige um sàveiro.
O mesmo que \textunderscore sàveiro\textunderscore .
\section{Sàveiro}
\begin{itemize}
\item {Grp. gram.:m.}
\end{itemize}
\begin{itemize}
\item {Utilização:Bras. do Rio}
\end{itemize}
Barco estreito e comprido, empregado na travessia dos grandes rios, como o Sado, e na pesca á linha.
Homem, que dirige êsse barco.
Embarcação de forte construcção, que se emprega na carga ou descarga de gêneros alimentícios.
(Por \textunderscore sàveleiro\textunderscore , de \textunderscore sável\textunderscore )
\section{Sável}
\begin{itemize}
\item {Grp. gram.:m.}
\end{itemize}
Peixe clúpeo, (\textunderscore clupea alosa\textunderscore ).
\section{Sàvelha}
\begin{itemize}
\item {fónica:vê}
\end{itemize}
\begin{itemize}
\item {Grp. gram.:f.}
\end{itemize}
Espécie de sável; saboga.
\section{Saviá}
\begin{itemize}
\item {Grp. gram.:m.}
\end{itemize}
O mesmo que \textunderscore sauiá\textunderscore .
\section{Savica}
\begin{itemize}
\item {Grp. gram.:f.}
\end{itemize}
Peça da carruagem, que se mete nas pontas dos eixos, para pegar na chaveta das rodas.
\section{Savicão}
\begin{itemize}
\item {Grp. gram.:m.}
\end{itemize}
\begin{itemize}
\item {Utilização:Prov.}
\end{itemize}
\begin{itemize}
\item {Utilização:alent.}
\end{itemize}
\begin{itemize}
\item {Proveniência:(De \textunderscore savica\textunderscore )}
\end{itemize}
Peça de ferro, que acompanha todo o comprimento do eixo dos carros, quando êste é de madeira.
\section{Savígnia}
\begin{itemize}
\item {Grp. gram.:f.}
\end{itemize}
\begin{itemize}
\item {Proveniência:(De \textunderscore Savigny\textunderscore , n. p.)}
\end{itemize}
Gênero de plantas crucíferas, cuja espécie típica é um arbusto que cresce no Egipto.
\section{Savígnya}
\begin{itemize}
\item {Grp. gram.:f.}
\end{itemize}
\begin{itemize}
\item {Proveniência:(De \textunderscore Savigny\textunderscore , n. p.)}
\end{itemize}
Gênero de plantas crucíferas, cuja espécie týpica é um arbusto que cresce no Egypto.
\section{Savitu}
\begin{itemize}
\item {Grp. gram.:m.}
\end{itemize}
\begin{itemize}
\item {Utilização:Bras}
\end{itemize}
O mesmo que \textunderscore saúba\textunderscore .
\section{Savodinskito}
\begin{itemize}
\item {Grp. gram.:m.}
\end{itemize}
\begin{itemize}
\item {Utilização:Miner.}
\end{itemize}
\begin{itemize}
\item {Proveniência:(De \textunderscore Savodinsk\textunderscore , n. p.)}
\end{itemize}
Variedade de tellureto de prata, que se encontra nas minas do Altai.
\section{Savónulo}
\begin{itemize}
\item {Grp. gram.:m.}
\end{itemize}
\begin{itemize}
\item {Utilização:Chím.}
\end{itemize}
\begin{itemize}
\item {Proveniência:(Fr. \textunderscore savonule\textunderscore )}
\end{itemize}
Nome genérico de algumas combinações, formadas por certas essências, ao contacto dos álcalis.
\section{Saxão}
\begin{itemize}
\item {fónica:csão}
\end{itemize}
\begin{itemize}
\item {Grp. gram.:adj.}
\end{itemize}
\begin{itemize}
\item {Proveniência:(De \textunderscore Saxe\textunderscore , n. p.)}
\end{itemize}
Relativo aos Saxões.
\section{Saxátil}
\begin{itemize}
\item {fónica:csá}
\end{itemize}
\begin{itemize}
\item {Grp. gram.:adj.}
\end{itemize}
\begin{itemize}
\item {Proveniência:(Lat. \textunderscore saxatilis\textunderscore )}
\end{itemize}
Que habita entre pedras.
Adherente a rochedos.
\section{Sáxeo}
\begin{itemize}
\item {fónica:cse}
\end{itemize}
\begin{itemize}
\item {Grp. gram.:adj.}
\end{itemize}
\begin{itemize}
\item {Utilização:Poét.}
\end{itemize}
\begin{itemize}
\item {Proveniência:(Lat. \textunderscore saxeus\textunderscore )}
\end{itemize}
Que é de pedra; pedregoso.
\section{Saxícava}
\begin{itemize}
\item {fónica:csi}
\end{itemize}
\begin{itemize}
\item {Grp. gram.:f.}
\end{itemize}
Gênero de molluscos.
(Do \textunderscore saxum\textunderscore  + \textunderscore cavus\textunderscore )
\section{Saxícola}
\begin{itemize}
\item {fónica:csi}
\end{itemize}
\begin{itemize}
\item {Grp. gram.:adj.}
\end{itemize}
\begin{itemize}
\item {Proveniência:(Lat. \textunderscore saxicola\textunderscore )}
\end{itemize}
O mesmo que \textunderscore saxátil\textunderscore .
\section{Saxicolídeas}
\begin{itemize}
\item {fónica:csi}
\end{itemize}
\begin{itemize}
\item {Grp. gram.:f. pl.}
\end{itemize}
\begin{itemize}
\item {Proveniência:(De \textunderscore saxícola\textunderscore )}
\end{itemize}
Família de aves, da ordem dos pássaros, que comprehende espécies que vivem em terrenos secos, e pedregosos.
\section{Saxífraga}
\begin{itemize}
\item {fónica:csi}
\end{itemize}
\begin{itemize}
\item {Grp. gram.:f.}
\end{itemize}
\begin{itemize}
\item {Proveniência:(Lat. \textunderscore saxifraga\textunderscore )}
\end{itemize}
Gênero de plantas, uma de cujas espécies se empregava em dissolver os cálculos da bexiga.
\textunderscore Saxífraga branca\textunderscore , o mesmo que \textunderscore sanícula-dos-montes\textunderscore .
\section{Saxifragáceas}
\begin{itemize}
\item {fónica:csi}
\end{itemize}
\begin{itemize}
\item {Grp. gram.:f. pl.}
\end{itemize}
Família de plantas, que tem por typo a saxífraga.
(Fem. pl. de \textunderscore saxifragáceo\textunderscore )
\section{Saxifragáceo}
\begin{itemize}
\item {fónica:csi}
\end{itemize}
\begin{itemize}
\item {Grp. gram.:adj.}
\end{itemize}
Relativo ou semelhante á saxífraga.
\section{Saxifrageáceas}
\begin{itemize}
\item {fónica:csi}
\end{itemize}
\begin{itemize}
\item {Grp. gram.:f. pl.}
\end{itemize}
O mesmo que \textunderscore saxifragáceas\textunderscore .
\section{Saxifrágia}
\begin{itemize}
\item {fónica:csi}
\end{itemize}
\begin{itemize}
\item {Grp. gram.:f.}
\end{itemize}
O mesmo que \textunderscore saxífraga\textunderscore .
\section{Saxífrago}
\begin{itemize}
\item {fónica:csi}
\end{itemize}
\begin{itemize}
\item {Grp. gram.:adj.}
\end{itemize}
\begin{itemize}
\item {Proveniência:(Do lat. \textunderscore saxum\textunderscore  + \textunderscore frangere\textunderscore )}
\end{itemize}
Que quebra ou dissolve pedras.
\section{Saxina}
\begin{itemize}
\item {fónica:csi}
\end{itemize}
\begin{itemize}
\item {Grp. gram.:f.}
\end{itemize}
Gênero de plantas caryophylláceas.
\section{Saxões}
\begin{itemize}
\item {fónica:csões}
\end{itemize}
\begin{itemize}
\item {Grp. gram.:m. pl.}
\end{itemize}
Antigo povo germânico, entre o Rheno e o Báltico.
(Pl. de \textunderscore saxão\textunderscore )
\section{Saxofone}
\begin{itemize}
\item {fónica:csó}
\end{itemize}
\begin{itemize}
\item {Grp. gram.:m.}
\end{itemize}
\begin{itemize}
\item {Proveniência:(De \textunderscore Sax\textunderscore , n. p. + gr. \textunderscore phone\textunderscore )}
\end{itemize}
Instrumento de metal, com chaves, e com embocadura semelhante á do clarinete.
\section{Saxofónio}
\begin{itemize}
\item {fónica:csó}
\end{itemize}
\begin{itemize}
\item {Grp. gram.:m.}
\end{itemize}
\begin{itemize}
\item {Proveniência:(De \textunderscore Sax\textunderscore , n. p. + gr. \textunderscore phone\textunderscore )}
\end{itemize}
Instrumento de metal, com chaves, e com embocadura semelhante á do clarinete.
\section{Saxones}
\begin{itemize}
\item {fónica:csó}
\end{itemize}
\begin{itemize}
\item {Grp. gram.:m. pl.}
\end{itemize}
O mesmo ou melhor que \textunderscore saxões.\textunderscore  Cf. \textunderscore Lusíadas\textunderscore , III, 11.
\section{Saxónio}
\begin{itemize}
\item {fónica:csó}
\end{itemize}
\begin{itemize}
\item {Grp. gram.:adj.}
\end{itemize}
\begin{itemize}
\item {Grp. gram.:M.}
\end{itemize}
Relativo á Saxónia.
Relativo aos saxões.
Habitante da Saxónia.
\section{Saxophone}
\begin{itemize}
\item {fónica:csó}
\end{itemize}
\begin{itemize}
\item {Grp. gram.:m.}
\end{itemize}
\begin{itemize}
\item {Proveniência:(De \textunderscore Sax\textunderscore , n. p. + gr. \textunderscore phone\textunderscore )}
\end{itemize}
Instrumento de metal, com chaves, e com embocadura semelhante á do clarinete.
\section{Saxophónio}
\begin{itemize}
\item {fónica:csó}
\end{itemize}
\begin{itemize}
\item {Grp. gram.:m.}
\end{itemize}
\begin{itemize}
\item {Proveniência:(De \textunderscore Sax\textunderscore , n. p. + gr. \textunderscore phone\textunderscore )}
\end{itemize}
Instrumento de metal, com chaves, e com embocadura semelhante á do clarinete.
\section{Saxoso}
\begin{itemize}
\item {fónica:csó}
\end{itemize}
\begin{itemize}
\item {Grp. gram.:adj.}
\end{itemize}
\begin{itemize}
\item {Proveniência:(Lat. \textunderscore saxosus\textunderscore )}
\end{itemize}
O mesmo que \textunderscore pedregoso\textunderscore .
\section{Saxotromba}
\begin{itemize}
\item {fónica:csó}
\end{itemize}
\begin{itemize}
\item {Grp. gram.:m.}
\end{itemize}
O mesmo que \textunderscore saxotrompa\textunderscore . Cf. \textunderscore Diccion. Mus.\textunderscore 
\section{Saxotrompa}
\begin{itemize}
\item {fónica:csó}
\end{itemize}
\begin{itemize}
\item {Grp. gram.:m.}
\end{itemize}
\begin{itemize}
\item {Proveniência:(De \textunderscore Sax\textunderscore , n. p. + \textunderscore trompa\textunderscore )}
\end{itemize}
Instrumento de metal, com três, quatro ou cinco cylindros.
\section{Sazão}
\begin{itemize}
\item {Grp. gram.:f.}
\end{itemize}
\begin{itemize}
\item {Utilização:Fig.}
\end{itemize}
\begin{itemize}
\item {Proveniência:(Do lat. \textunderscore satio\textunderscore )}
\end{itemize}
Estação do anno.
Opportunidade, ensejo.
\section{Sàzeiro}
\begin{itemize}
\item {Grp. gram.:m.}
\end{itemize}
\begin{itemize}
\item {Utilização:Prov.}
\end{itemize}
\begin{itemize}
\item {Utilização:minh.}
\end{itemize}
Planta, (\textunderscore salix salvifolia\textunderscore , Brot.).
\section{Sazo}
\begin{itemize}
\item {Grp. gram.:m.}
\end{itemize}
Sacerdote, de classe inferior, no reino de Camboja.
\section{Sazoamento}
\begin{itemize}
\item {Grp. gram.:m.}
\end{itemize}
O mesmo que \textunderscore sazonamento\textunderscore . Cf. Castilho, \textunderscore Felic. pela Agr.\textunderscore , 175.
\section{Sazoar}
\textunderscore v. t.\textunderscore  (e der.)
O mesmo que \textunderscore sazonar\textunderscore , etc.
\section{Sazonação}
\begin{itemize}
\item {Grp. gram.:f.}
\end{itemize}
Acto de sazonar.
\section{Sazonadamente}
\begin{itemize}
\item {Grp. gram.:adv.}
\end{itemize}
De modo sazonado.
\section{Sazonado}
\begin{itemize}
\item {Grp. gram.:adj.}
\end{itemize}
\begin{itemize}
\item {Utilização:Fig.}
\end{itemize}
\begin{itemize}
\item {Proveniência:(De \textunderscore sazonar\textunderscore )}
\end{itemize}
Maduro.
Experiente.
\section{Sazonar}
\begin{itemize}
\item {Grp. gram.:v. t.}
\end{itemize}
\begin{itemize}
\item {Utilização:Fig.}
\end{itemize}
\begin{itemize}
\item {Grp. gram.:V. i.  e  p.}
\end{itemize}
\begin{itemize}
\item {Utilização:Fig.}
\end{itemize}
\begin{itemize}
\item {Proveniência:(De \textunderscore sazão\textunderscore )}
\end{itemize}
Tornar maduro.
Dar bom sabor a.
Temperar.
Condimentar.
Tornar experimentado.
Tornar-se maduro.
Melhorar-se; tornar-se perfeito.
\section{Sazonável}
\begin{itemize}
\item {Grp. gram.:adj.}
\end{itemize}
\begin{itemize}
\item {Proveniência:(De \textunderscore sazonar\textunderscore )}
\end{itemize}
Que está em condições de amadurecer.
Productivo.
\section{Sazu}
\begin{itemize}
\item {Grp. gram.:m.}
\end{itemize}
Pássaro da Sofala, do tamanho de um pardal.
\section{Scala}
\begin{itemize}
\item {Grp. gram.:f.}
\end{itemize}
\begin{itemize}
\item {Utilização:Ant.}
\end{itemize}
Taça ou copo lavrado a buril.
\section{Scaldo}
\begin{itemize}
\item {Grp. gram.:m.}
\end{itemize}
Cantor medieval; bardo. Cf. Garrett, \textunderscore Romanceiro\textunderscore , I, p. XVIII.
\section{Scalído}
\begin{itemize}
\item {Grp. gram.:m.}
\end{itemize}
\begin{itemize}
\item {Utilização:Ant.}
\end{itemize}
Cavouco do moínho, ou lugar, onde lança as águas a cale do moínho.
\section{Scapo}
\begin{itemize}
\item {Grp. gram.:m.}
\end{itemize}
\begin{itemize}
\item {Proveniência:(Lat. \textunderscore scapus\textunderscore )}
\end{itemize}
Aste: tronco. Cf. Castilho, \textunderscore Fastos\textunderscore , I, 312, 319 e 320.
\section{Sceleradamente}
\begin{itemize}
\item {Grp. gram.:adv.}
\end{itemize}
De modo scelerado; com perversidade.
\section{Scelerado}
\begin{itemize}
\item {Grp. gram.:adj.}
\end{itemize}
\begin{itemize}
\item {Grp. gram.:M.}
\end{itemize}
\begin{itemize}
\item {Proveniência:(Lat. \textunderscore sceleratus\textunderscore )}
\end{itemize}
Que praticou grande crime.
Que é capaz de grandes crimes.
Perverso.
Que revela grande perversidade.
Indivíduo scelerado.
\section{Scena}
\begin{itemize}
\item {Grp. gram.:f.}
\end{itemize}
\begin{itemize}
\item {Utilização:Fig.}
\end{itemize}
\begin{itemize}
\item {Proveniência:(Lat. \textunderscore scena\textunderscore )}
\end{itemize}
Parte do theatro, em que os actores representam os seus papéis.
Palco.
Decoração theatral.
Parte de um acto de uma peça theatral, durante a qual as vistas do palco são as mesmas e os mesmos os actores que representam.
Lugar, onde se realiza algum facto.
Acontecimento dramático ou susceptível de representação theatral.
Acto mais ou menos censurável.
Perspectiva; coisa ou coisas, que se abrangem com a vista; panorama.
Arte dramatica: \textunderscore dedicar-se á scena\textunderscore .
\section{Scenário}
\begin{itemize}
\item {Grp. gram.:m.}
\end{itemize}
\begin{itemize}
\item {Proveniência:(Lat. \textunderscore scenarium\textunderscore )}
\end{itemize}
Decoração theatral.
Conjunto de bastidores e vistas, apropriadas aos factos que se representam.
\section{Scenedesmo}
\begin{itemize}
\item {Grp. gram.:m.}
\end{itemize}
\begin{itemize}
\item {Proveniência:(Do gr. \textunderscore skene\textunderscore  + \textunderscore desmos\textunderscore )}
\end{itemize}
Gênero de algas microscópicas.
\section{Scênico}
\begin{itemize}
\item {Grp. gram.:adj.}
\end{itemize}
Relativo á scena; theatral.
\section{Scênio}
\begin{itemize}
\item {Grp. gram.:m.}
\end{itemize}
\begin{itemize}
\item {Proveniência:(Do gr. \textunderscore skenion\textunderscore )}
\end{itemize}
Fachada nos theatros antigos.
\section{Scenographia}
\begin{itemize}
\item {Grp. gram.:f.}
\end{itemize}
\begin{itemize}
\item {Utilização:Restrict.}
\end{itemize}
\begin{itemize}
\item {Proveniência:(Lat. \textunderscore scenographia\textunderscore )}
\end{itemize}
Arte de desenhar os lugares, os edifícios, etc., alargando-os ou estreitando-os, segundo as regras da perspectiva.
Arte de pintar as decorações de um theatro.
Conjunto dos objectos representados.
\section{Scenographicamente}
\begin{itemize}
\item {Grp. gram.:adv.}
\end{itemize}
De modo scenográphico, ou segundo as regras da scenographia.
\section{Scenográphico}
\begin{itemize}
\item {Grp. gram.:adj.}
\end{itemize}
Relativo á scenographia.
\section{Scenógrapho}
\begin{itemize}
\item {Grp. gram.:m.}
\end{itemize}
\begin{itemize}
\item {Proveniência:(Do gr. \textunderscore skene\textunderscore  + \textunderscore graphein\textunderscore )}
\end{itemize}
Aquelle que pratíca a scenographia.
Aquelle que pinta o scenario.
\section{Scenopégia}
\begin{itemize}
\item {Grp. gram.:f.}
\end{itemize}
\begin{itemize}
\item {Proveniência:(Lat. \textunderscore scenopegia\textunderscore )}
\end{itemize}
A festa dos tabernáculos, entre os Judeus, que com ella celebravam a sua estada de quarenta annos no deserto.
\section{Scepa}
\begin{itemize}
\item {Grp. gram.:f.}
\end{itemize}
\begin{itemize}
\item {Proveniência:(Do gr. \textunderscore skepe\textunderscore )}
\end{itemize}
Gênero de árvores indianas.
\section{Scepticamente}
\begin{itemize}
\item {Grp. gram.:adv.}
\end{itemize}
De modo scéptico; com scepticismo; de modo pyrrhónico.
\section{Scepticismo}
\begin{itemize}
\item {Grp. gram.:m.}
\end{itemize}
\begin{itemize}
\item {Proveniência:(De \textunderscore scéptico\textunderscore )}
\end{itemize}
Doutrina philosóphica dos que duvidam, examinando.
Estado de quem duvida de tudo; pyrrhonismo.
\section{Scéptico}
\begin{itemize}
\item {Grp. gram.:adj.}
\end{itemize}
\begin{itemize}
\item {Grp. gram.:M.}
\end{itemize}
\begin{itemize}
\item {Proveniência:(Gr. \textunderscore skeptikos\textunderscore )}
\end{itemize}
Diz-se dos philósophos, cujo dogma principal era duvidar de tudo; descrente.
Sectário do scepticismo.
Indivíduo descrente ou que duvída de tudo.
\section{Sceptrígero}
\begin{itemize}
\item {Grp. gram.:adj.}
\end{itemize}
\begin{itemize}
\item {Proveniência:(Lat. \textunderscore sceptriger\textunderscore )}
\end{itemize}
Que usa sceptro.
\section{Sceptro}
\begin{itemize}
\item {Grp. gram.:m.}
\end{itemize}
\begin{itemize}
\item {Utilização:Fig.}
\end{itemize}
\begin{itemize}
\item {Proveniência:(Lat. \textunderscore sceptrum\textunderscore )}
\end{itemize}
Bastão, que antigamente designava autoridade real.
Pequeno bastão, encimado por uma flôr, uma esphera ou outro qualquer ornato, usado antigamente pelos Consules e Imperadores romanos e modernamente pelos Soberanos da Europa.
Autoridade soberana.
Poder real.
O rei.
Preeminência.
Despotismo.
\section{Schauéria}
\begin{itemize}
\item {Grp. gram.:f.}
\end{itemize}
\begin{itemize}
\item {Proveniência:(De \textunderscore Schauer\textunderscore , n. p.)}
\end{itemize}
Gênero de plantas acantháceas.
\section{Scheelito}
\begin{itemize}
\item {Grp. gram.:m.}
\end{itemize}
\begin{itemize}
\item {Proveniência:(De \textunderscore Sceel\textunderscore , n. p.)}
\end{itemize}
Variedade de mineral alvacento.
\section{Schefféria}
\begin{itemize}
\item {Grp. gram.:f.}
\end{itemize}
\begin{itemize}
\item {Proveniência:(De \textunderscore Scheffer\textunderscore , n. p.)}
\end{itemize}
Gênero de plantas rhamnáceas.
\section{Schema}
\begin{itemize}
\item {fónica:quê}
\end{itemize}
\textunderscore m.\textunderscore  (e der.)
(V. \textunderscore eschema\textunderscore , etc.)
\section{Schepéria}
\begin{itemize}
\item {Grp. gram.:f.}
\end{itemize}
\begin{itemize}
\item {Proveniência:(De \textunderscore Scheper\textunderscore , n. p.)}
\end{itemize}
Gênero de plantas capparáceas.
\section{Scheuchzéria}
\begin{itemize}
\item {Grp. gram.:f.}
\end{itemize}
\begin{itemize}
\item {Proveniência:(De \textunderscore Scheuchzer\textunderscore , n. p.)}
\end{itemize}
Gênero de plantas alismáceas.
\section{Schisto}
\textunderscore m.\textunderscore  (e der.)
(V. \textunderscore xisto\textunderscore ^1, etc.)
\section{Schizocéphalo}
\begin{itemize}
\item {fónica:qui}
\end{itemize}
\begin{itemize}
\item {Grp. gram.:adj.}
\end{itemize}
\begin{itemize}
\item {Proveniência:(Do gr. \textunderscore skhizein\textunderscore  + \textunderscore kephale\textunderscore )}
\end{itemize}
Que tem a cabeça dividida longitudinalmente, (falando-se de certos monstros).
\section{Schizólitho}
\begin{itemize}
\item {fónica:qui}
\end{itemize}
\begin{itemize}
\item {Grp. gram.:m.}
\end{itemize}
\begin{itemize}
\item {Proveniência:(Do gr. \textunderscore skizein\textunderscore  + \textunderscore lithos\textunderscore )}
\end{itemize}
Gênero do mineraes, que comprehende a mica e outros.
\section{Schizomycetes}
\begin{itemize}
\item {fónica:qui}
\end{itemize}
\begin{itemize}
\item {Grp. gram.:m. pl.}
\end{itemize}
\begin{itemize}
\item {Proveniência:(Do gr. \textunderscore skizein\textunderscore  + \textunderscore muke\textunderscore )}
\end{itemize}
Nome, dado ás bactérias pelos autores que as classificam como cogumelos.
\section{Schizóphyto}
\begin{itemize}
\item {fónica:qui}
\end{itemize}
\begin{itemize}
\item {Grp. gram.:adj.}
\end{itemize}
\begin{itemize}
\item {Utilização:Bot.}
\end{itemize}
\begin{itemize}
\item {Proveniência:(Do gr. \textunderscore skizein\textunderscore  + \textunderscore phuton\textunderscore )}
\end{itemize}
Diz-se dos vegetaes, que se reproduzem por fissiparidade.
\section{Schizópode}
\begin{itemize}
\item {fónica:qui}
\end{itemize}
\begin{itemize}
\item {Grp. gram.:adj.}
\end{itemize}
\begin{itemize}
\item {Grp. gram.:m. pl.}
\end{itemize}
\begin{itemize}
\item {Proveniência:(Gr. \textunderscore skizopous\textunderscore )}
\end{itemize}
Que tem os pés fendidos.
Gênero de crustáceos.
\section{Schizóptero}
\begin{itemize}
\item {fónica:qui}
\end{itemize}
\begin{itemize}
\item {Grp. gram.:adj.}
\end{itemize}
\begin{itemize}
\item {Proveniência:(Gr. \textunderscore skhizopteros\textunderscore )}
\end{itemize}
Que tem asas fendidas.
\section{Schizothórax}
\begin{itemize}
\item {fónica:qui}
\end{itemize}
\begin{itemize}
\item {Grp. gram.:m.}
\end{itemize}
\begin{itemize}
\item {Proveniência:(Do gr. \textunderscore skhizein\textunderscore  + \textunderscore thorax\textunderscore )}
\end{itemize}
Monstruosidade, caracterizada pela divisão do esterno ou das paredes thorácicas.
\section{Schizotrichia}
\begin{itemize}
\item {fónica:qui}
\end{itemize}
\begin{itemize}
\item {Grp. gram.:f.}
\end{itemize}
\begin{itemize}
\item {Proveniência:(Do gr. \textunderscore skhizein\textunderscore  + \textunderscore strix\textunderscore , \textunderscore strikos\textunderscore )}
\end{itemize}
Qualidade de certos cabellos, que são fendidos na extremidade.
\section{Schkúhria}
\begin{itemize}
\item {Grp. gram.:f.}
\end{itemize}
\begin{itemize}
\item {Proveniência:(De \textunderscore Schkuhr\textunderscore , n. p.)}
\end{itemize}
Gênero de plantas, da fam. das compostas.
\section{Schleichera}
\begin{itemize}
\item {Grp. gram.:f.}
\end{itemize}
\begin{itemize}
\item {Proveniência:(De \textunderscore Schleicher\textunderscore , n. p.)}
\end{itemize}
Gênero de plantas sapindáceas.
\section{Schleidênia}
\begin{itemize}
\item {Grp. gram.:f.}
\end{itemize}
\begin{itemize}
\item {Proveniência:(De \textunderscore Schleiden\textunderscore , n. p.)}
\end{itemize}
Gênero de plantas asperifoliáceas.
\section{Schmidélia}
\begin{itemize}
\item {Grp. gram.:f.}
\end{itemize}
\begin{itemize}
\item {Proveniência:(De \textunderscore Schmidel\textunderscore )}
\end{itemize}
Gênero de plantas sapindáceas.
\section{Schnebelite}
\begin{itemize}
\item {Grp. gram.:f.}
\end{itemize}
\begin{itemize}
\item {Proveniência:(De \textunderscore Schenebelin\textunderscore , n. p.)}
\end{itemize}
Explosivo, de invenção recente (1894), que resiste á fricção e á mais alta temperatura.
\section{Sciascopia}
\begin{itemize}
\item {Grp. gram.:f.}
\end{itemize}
\begin{itemize}
\item {Proveniência:(Do gr. \textunderscore skia\textunderscore  + \textunderscore skopein\textunderscore )}
\end{itemize}
Determinação da refracção do ôlho pelo estudo das sombras que se observam no campo pupillar com o auxilio do ophthalmoscópio.
\section{Sciática}
\begin{itemize}
\item {Grp. gram.:f.}
\end{itemize}
\begin{itemize}
\item {Proveniência:(De \textunderscore sciático\textunderscore )}
\end{itemize}
Dôr sciática.
\section{Sciático}
\begin{itemize}
\item {Grp. gram.:adj.}
\end{itemize}
\begin{itemize}
\item {Proveniência:(Lat. \textunderscore sciaticus\textunderscore )}
\end{itemize}
Relativo ás ancas ou á parte superior da coxa.
Diz-se do nervo mais grosso de todo o organismo animal.
Diz-se da dôr, que se fixa nesse nervo, occupando a parte posterior da coxa e da perna.
\section{Sciência}
\begin{itemize}
\item {Grp. gram.:f.}
\end{itemize}
\begin{itemize}
\item {Proveniência:(Lat. \textunderscore scientia\textunderscore )}
\end{itemize}
Conhecimento de qualquer coisa.
Instrucção.
Conjunto ou systema de certos conhecimentos.
Saber, que se adquire pela leitura e pela meditação.
Conjunto systemático de princípios ou leis, que dizem respeito a objectos correlacionados.
Tudo que é susceptível de formar preceitos ou regras.
\section{Sciente}
\begin{itemize}
\item {Grp. gram.:adj.}
\end{itemize}
\begin{itemize}
\item {Proveniência:(Lat. \textunderscore sciens\textunderscore )}
\end{itemize}
Que tem conhecimento de alguma coisa; que sabe: \textunderscore ficou sciente da recommendação\textunderscore .
Que tem sciência.
\section{Scientemente}
\begin{itemize}
\item {Grp. gram.:adv.}
\end{itemize}
\begin{itemize}
\item {Proveniência:(De \textunderscore sciente\textunderscore )}
\end{itemize}
Com sciência, com conhecimento.
Adrede, de caso pensado.
\section{Scientificamente}
\begin{itemize}
\item {Grp. gram.:adv.}
\end{itemize}
De modo scientifico.
Segundo os processos ou preceitos da sciência; com sciência.
\section{Scientificar}
\begin{itemize}
\item {Grp. gram.:v. t.}
\end{itemize}
\begin{itemize}
\item {Utilização:Neol.}
\end{itemize}
\begin{itemize}
\item {Proveniência:(Do lat. \textunderscore sciens\textunderscore  + \textunderscore facere\textunderscore )}
\end{itemize}
Tornar sciente.
\section{Scientífico}
\begin{itemize}
\item {Grp. gram.:adj.}
\end{itemize}
\begin{itemize}
\item {Proveniência:(Do lat. \textunderscore scientia\textunderscore  + \textunderscore facere\textunderscore )}
\end{itemize}
Relativo á sciência: \textunderscore progressos scientíficos\textunderscore .
Que mostra sciência.
\section{Scientista}
\begin{itemize}
\item {Grp. gram.:m.}
\end{itemize}
\begin{itemize}
\item {Utilização:Neol.}
\end{itemize}
\begin{itemize}
\item {Proveniência:(Do lat. \textunderscore scientia\textunderscore )}
\end{itemize}
Aquelle que se occupa de sciências ou de uma sciência.
\section{Scieropia}
\begin{itemize}
\item {Grp. gram.:f.}
\end{itemize}
\begin{itemize}
\item {Proveniência:(Do gr. \textunderscore skieros\textunderscore  + \textunderscore ops\textunderscore )}
\end{itemize}
Doença da vista, que faz vêr os objectos com uma côr mais carregada que a real.
\section{Scilla}
\begin{itemize}
\item {Grp. gram.:f.}
\end{itemize}
\begin{itemize}
\item {Proveniência:(Lat. \textunderscore scilla\textunderscore )}
\end{itemize}
Gênero de plantas liliáceas.
Espécie de narciso, (\textunderscore pancratium guyanensis\textunderscore ).
\section{Scillítico}
\begin{itemize}
\item {Grp. gram.:adj.}
\end{itemize}
Que contém suco de scilla ou que é feito com suco de scilla.
\section{Scillitina}
\begin{itemize}
\item {Grp. gram.:f.}
\end{itemize}
Princípio acre, que se encontra no suco da scilla.
\section{Scillito}
\begin{itemize}
\item {Grp. gram.:m.}
\end{itemize}
Vinho de scilla, preparado nas boticas.
\section{Scinco}
\begin{itemize}
\item {Grp. gram.:m.}
\end{itemize}
\begin{itemize}
\item {Proveniência:(Do gr. \textunderscore skinkos\textunderscore )}
\end{itemize}
Gênero do reptis sáurios.
\section{Scindir}
\begin{itemize}
\item {Grp. gram.:v. t.}
\end{itemize}
O mesmo que \textunderscore escindir\textunderscore .
\section{Scintilla}
\begin{itemize}
\item {Grp. gram.:f.}
\end{itemize}
\begin{itemize}
\item {Proveniência:(Lat. \textunderscore scintilla\textunderscore )}
\end{itemize}
O mesmo que \textunderscore centelha\textunderscore :«\textunderscore ...uma scintilla resistente de instincto feminil.\textunderscore »Camillo, \textunderscore Brasileira\textunderscore , 388.
\section{Scintillação}
\begin{itemize}
\item {Grp. gram.:f.}
\end{itemize}
\begin{itemize}
\item {Proveniência:(Do lat. \textunderscore scintillatio\textunderscore )}
\end{itemize}
Acto ou effeito de scintillar; brilho intenso.
\section{Scintillante}
\begin{itemize}
\item {Grp. gram.:adj.}
\end{itemize}
\begin{itemize}
\item {Proveniência:(Lat. \textunderscore scintillans\textunderscore )}
\end{itemize}
Que scintilla; muito brilhante; deslumbrante.
\section{Scintillar}
\begin{itemize}
\item {Grp. gram.:v. i.}
\end{itemize}
\begin{itemize}
\item {Grp. gram.:V. t.}
\end{itemize}
\begin{itemize}
\item {Proveniência:(Lat. \textunderscore scintillare\textunderscore )}
\end{itemize}
Têr brilho, semelhante ao das scentelhas.
Brilhar, tremendo; tremeluzir: \textunderscore scintillam estrêllas\textunderscore .
Faiscar.
Brilhar muito.
Resplandecer.
Irradiar, diffundir luminosamente:«\textunderscore está scintillando differentes brilhos.\textunderscore »\textunderscore Luz e Calor\textunderscore , 327.«\textunderscore Scintillaram áscuas de júbilo os olhos da menina.\textunderscore »Camillo, \textunderscore Caveira\textunderscore , 162.
\section{Scintillómetro}
\begin{itemize}
\item {Grp. gram.:m.}
\end{itemize}
\begin{itemize}
\item {Proveniência:(Do lat. \textunderscore scintilla\textunderscore  + gr. \textunderscore metron\textunderscore )}
\end{itemize}
Instrumento, inventado por Arago, para apreciar a intensidade da scintillação dos astros.
\section{Scintura}
\begin{itemize}
\item {Grp. gram.:f.}
\end{itemize}
\begin{itemize}
\item {Utilização:Ant.}
\end{itemize}
Acto ou effeito de scindir:«\textunderscore com fome seram consumidos, e será por scintura desfallecimento dos membros.\textunderscore »Usque, 19 v.^o
(Por \textunderscore scindura\textunderscore , de \textunderscore scindir\textunderscore )
\section{Sciographia}
\begin{itemize}
\item {Grp. gram.:f.}
\end{itemize}
\begin{itemize}
\item {Proveniência:(Lat. \textunderscore sciographia\textunderscore )}
\end{itemize}
Desenho de um edifício, que se representa cortado longitudinalmente ou transversalmente, para se lhe poder observar a disposição interior.
Arte de conhecer as horas pela sombra dos astros.
\section{Sciográphico}
\begin{itemize}
\item {Grp. gram.:adj.}
\end{itemize}
Relativo á sciographia.
\section{Sciógrapho}
\begin{itemize}
\item {Grp. gram.:m.}
\end{itemize}
\begin{itemize}
\item {Proveniência:(Do gr. \textunderscore skia\textunderscore  + \textunderscore graphein\textunderscore )}
\end{itemize}
Aquelle que conhece ou pratica a sciographia.
\section{Scióptico}
\begin{itemize}
\item {Grp. gram.:adj.}
\end{itemize}
\begin{itemize}
\item {Proveniência:(Do gr. \textunderscore skia\textunderscore  + \textunderscore opteskein\textunderscore )}
\end{itemize}
Relativo á visão na sombra.
\section{Sciras}
\begin{itemize}
\item {Grp. gram.:f. pl.}
\end{itemize}
O mesmo que \textunderscore sciophórias\textunderscore .
\section{Scírias}
\begin{itemize}
\item {Grp. gram.:f. pl.}
\end{itemize}
(V.sciophórias)
\section{Sciophórias}
\begin{itemize}
\item {Grp. gram.:f. pl.}
\end{itemize}
\begin{itemize}
\item {Proveniência:(Do gr. \textunderscore skiros\textunderscore  + \textunderscore phoros\textunderscore )}
\end{itemize}
Festas, que os Atheneienses celebrava em honra de Minerva, durante as quaes se enramavam cabanas, e em que os mancebos, nos jogos respectivos, sustinham nas mãos cepas carregadas de uvas.
\section{Scirophório}
\begin{itemize}
\item {Grp. gram.:m.}
\end{itemize}
Mês, em que os Athenienses celebravam as scirophórias e que era o duodécimo do anno áttico.
\section{Scirrho}
\begin{itemize}
\item {Proveniência:(Gr. \textunderscore skirros\textunderscore )}
\end{itemize}
\textunderscore m.\textunderscore  (e der.)
O mesmo que \textunderscore cirro\textunderscore ^1, etc.
\section{Scisma}
\begin{itemize}
\item {Grp. gram.:m.}
\end{itemize}
\begin{itemize}
\item {Proveniência:(Do lat. \textunderscore schisma\textunderscore )}
\end{itemize}
Separação, que um indivíduo ou uma collectividade faz, de uma religião ou de indivíduos que obedecem a um chefe religioso, não reconhecido por aquelles.
Separação do povo judeu em dois reinos.
Separação de crenças políticas ou literárias.
\section{Scisma}
\begin{itemize}
\item {Grp. gram.:f.}
\end{itemize}
Acto de scismar.
Mania.
Devaneio.
\section{Scismar}
\begin{itemize}
\item {Grp. gram.:v. t.}
\end{itemize}
\begin{itemize}
\item {Grp. gram.:V. i.}
\end{itemize}
\begin{itemize}
\item {Grp. gram.:M.}
\end{itemize}
Pensar muito em.
Meditar, preoccupar-se.
Andar apprehensivo.
Ideia fixa, scisma.
(Or. duvidosa. Relacionar-se-á com o cast. \textunderscore ensimismar-se\textunderscore ? Neste caso, deveríamos escrever \textunderscore sismar\textunderscore )
\section{Scismaticamente}
\begin{itemize}
\item {Grp. gram.:adv.}
\end{itemize}
\begin{itemize}
\item {Proveniência:(De \textunderscore scismático\textunderscore ^2)}
\end{itemize}
De modo scismático.
Com scisma^2; como quem devaneia.
Á maneira de quem é aprehensivo.
\section{Scismático}
\begin{itemize}
\item {Grp. gram.:adj.}
\end{itemize}
\begin{itemize}
\item {Grp. gram.:M.}
\end{itemize}
\begin{itemize}
\item {Utilização:Ant.}
\end{itemize}
\begin{itemize}
\item {Proveniência:(Do lat. \textunderscore schismaticus\textunderscore )}
\end{itemize}
Que segue um scisma.
Relativo a scisma^2.
O mesmo que \textunderscore castelhano\textunderscore .
\section{Scismático}
\begin{itemize}
\item {Grp. gram.:adj.}
\end{itemize}
\begin{itemize}
\item {Proveniência:(De \textunderscore scismar\textunderscore )}
\end{itemize}
Que anda aprehensivo.
Que medita, sem objecto determinado.
Que devaneia.
\section{Scissão}
\begin{itemize}
\item {Grp. gram.:f.}
\end{itemize}
\begin{itemize}
\item {Proveniência:(Do lat. \textunderscore scissio\textunderscore )}
\end{itemize}
Acto ou effeito de scindir; scisma^1; desharmonia.
\section{Scissiparidade}
\begin{itemize}
\item {Grp. gram.:f.}
\end{itemize}
Qualidade do que é scissíparo.
\section{Scissíparo}
\begin{itemize}
\item {Grp. gram.:adj.}
\end{itemize}
\begin{itemize}
\item {Proveniência:(Do lat. \textunderscore scissus\textunderscore  + \textunderscore parere\textunderscore )}
\end{itemize}
O mesmo que \textunderscore fissíparo\textunderscore .
\section{Scissura}
\begin{itemize}
\item {Grp. gram.:f.}
\end{itemize}
\begin{itemize}
\item {Utilização:Fig.}
\end{itemize}
\begin{itemize}
\item {Proveniência:(Lat. \textunderscore scissura\textunderscore )}
\end{itemize}
Fenda, fissura.
Quebra de relações de amizade.
\section{Scítalo}
\begin{itemize}
\item {Grp. gram.:m.}
\end{itemize}
\begin{itemize}
\item {Proveniência:(Do gr. \textunderscore skitalos\textunderscore )}
\end{itemize}
Gênero de insectos coleópteros da Austrália.
\section{Scitosamente}
\begin{itemize}
\item {Grp. gram.:adv.}
\end{itemize}
\begin{itemize}
\item {Utilização:Ant.}
\end{itemize}
\begin{itemize}
\item {Proveniência:(Do rad. do lat. \textunderscore scitus\textunderscore , de \textunderscore scire\textunderscore )}
\end{itemize}
O mesmo que \textunderscore scientemente\textunderscore .
\section{Scopelismo}
\begin{itemize}
\item {Grp. gram.:m.}
\end{itemize}
\begin{itemize}
\item {Proveniência:(Do gr. \textunderscore skopelon\textunderscore )}
\end{itemize}
Acto de dispor pedras por certa fórma numa propriedade alheia, o que na antiguidade romana equivalia a uma ameaça de morte.
\section{Scýbalo}
\begin{itemize}
\item {Grp. gram.:m.}
\end{itemize}
\begin{itemize}
\item {Proveniência:(Do gr. \textunderscore skubalos\textunderscore )}
\end{itemize}
Excremento duro e arredondado.
\section{Scyros}
\begin{itemize}
\item {Grp. gram.:m. pl.}
\end{itemize}
\begin{itemize}
\item {Proveniência:(Lat. \textunderscore scyri\textunderscore )}
\end{itemize}
Tríbo combatente nas guerras góticas da Espanha. Cf. Herculano, \textunderscore Eurico\textunderscore , c. IV.
\section{Scythas}
\begin{itemize}
\item {Grp. gram.:m. Pl.}
\end{itemize}
\begin{itemize}
\item {Proveniência:(Lat. \textunderscore Scythae\textunderscore )}
\end{itemize}
Designação genérica dos povos nómades do norte da Europa e da Ásia.
\section{Scýthico}
\begin{itemize}
\item {Grp. gram.:adj.}
\end{itemize}
Relativo aos Scythas.
\section{Scythissa}
\begin{itemize}
\item {Grp. gram.:f.}
\end{itemize}
\begin{itemize}
\item {Proveniência:(Lat. \textunderscore scythissa\textunderscore )}
\end{itemize}
Mulhér, natural da Scýthia.
\section{Se}
\begin{itemize}
\item {Grp. gram.:conj.}
\end{itemize}
\begin{itemize}
\item {Proveniência:(Do lat. \textunderscore si\textunderscore )}
\end{itemize}
Dado que.
No caso de: \textunderscore se hoje chover...\textunderscore 
\section{Se}
\begin{itemize}
\item {Grp. gram.:pron.}
\end{itemize}
\begin{itemize}
\item {Proveniência:(Lat. \textunderscore se\textunderscore )}
\end{itemize}
A si.--é um dos casos do pron. \textunderscore êlle\textunderscore , e emprega-se geralmente como complemento directo e algumas vezes como terminativo.
É também partícula, que apassiva os verbos, como em: \textunderscore acabou-se a obra\textunderscore .
\section{Sé}
\begin{itemize}
\item {Grp. gram.:f.}
\end{itemize}
\begin{itemize}
\item {Proveniência:(Do lat. \textunderscore sedes\textunderscore )}
\end{itemize}
Igreja episcopal, archiepiscopal ou patriarchal.
Jurisdicção episcopal.
A \textunderscore santa sé\textunderscore , o poder pontifício, a Igreja romana.
\section{Sê}
\begin{itemize}
\item {Grp. gram.:m.}
\end{itemize}
(Contr. pop. de \textunderscore senhor\textunderscore ):«\textunderscore amigo e sê Zé!\textunderscore »Camillo, \textunderscore Brasileira\textunderscore , 235.
\section{S. E.}
\textunderscore abrev.\textunderscore  de \textunderscore Sueste\textunderscore .
\section{Seara}
\begin{itemize}
\item {Grp. gram.:f.}
\end{itemize}
\begin{itemize}
\item {Utilização:Fig.}
\end{itemize}
\begin{itemize}
\item {Proveniência:(Do b. lat. \textunderscore senara\textunderscore )}
\end{itemize}
Terreno onde crescem cereaes.
Terreno semeado; messe; campo cultivado.
Qualquer campo, coberto de vegetação.
Aggremiação, partido.
Conjunto numeroso: \textunderscore mondar a seara do vocabulário português\textunderscore .
\section{Seareiro}
\begin{itemize}
\item {Grp. gram.:m.}
\end{itemize}
\begin{itemize}
\item {Utilização:Prov.}
\end{itemize}
\begin{itemize}
\item {Utilização:alent.}
\end{itemize}
Cultivador de searas.
Aquelle que traz de arrendamento ou cultiva uma pequena terra de semeadura, pagando ao proprietário determinada parte dos frutos colhidos.
\section{Seba}
\begin{itemize}
\item {Grp. gram.:f.}
\end{itemize}
Estrume de plantas marinhas, destinado especialmente a adubar as vinhas.
(Por \textunderscore ceva\textunderscore ?)
\section{Sebáceo}
\begin{itemize}
\item {Grp. gram.:adj.}
\end{itemize}
\begin{itemize}
\item {Proveniência:(Lat. \textunderscore sebaceus\textunderscore )}
\end{itemize}
Que é da natureza do sebo.
Que tem sebo; gorduroso.
Sujo; sebento.
\section{Sebácico}
\begin{itemize}
\item {Grp. gram.:adj.}
\end{itemize}
\begin{itemize}
\item {Proveniência:(De \textunderscore sebáceo\textunderscore )}
\end{itemize}
Diz-se de um ácido, que se obtém, decompondo as gorduras pelo calor.
\section{Sebacina}
\begin{itemize}
\item {Grp. gram.:f.}
\end{itemize}
O mesmo que [[ácido sebácico|sebácico]].
\section{Sebada}
\begin{itemize}
\item {Grp. gram.:f.}
\end{itemize}
Sebe; reunião de sebes.
\section{Sebastianista}
\begin{itemize}
\item {Grp. gram.:m. f.  e  adj.}
\end{itemize}
\begin{itemize}
\item {Utilização:Ext.}
\end{itemize}
\begin{itemize}
\item {Proveniência:(De \textunderscore Sebastião\textunderscore , n. p.)}
\end{itemize}
Pessôa, que ainda espera que o Rei D. Sebastião volte da África.
Retrógrado; caturra.
\section{Sebastião}
\begin{itemize}
\item {Grp. gram.:m.}
\end{itemize}
\begin{itemize}
\item {Utilização:Gír.}
\end{itemize}
Tolo.
Pateta.
\section{Sebastião-da-arruda}
\begin{itemize}
\item {Grp. gram.:m.}
\end{itemize}
Variedade de salgueiro, (\textunderscore physocalymna florida\textunderscore ).
\section{Sebasto}
\begin{itemize}
\item {Grp. gram.:m.}
\end{itemize}
(V.sebastro)
\section{Sebastocrator}
\begin{itemize}
\item {Grp. gram.:m.}
\end{itemize}
Antigo dignitário da côrte de Constantinopla.
\section{Sebastro}
\begin{itemize}
\item {Grp. gram.:m.}
\end{itemize}
\begin{itemize}
\item {Utilização:Ant.}
\end{itemize}
O mesmo que \textunderscore sabasto\textunderscore .
\section{Sebato}
\begin{itemize}
\item {Grp. gram.:m.}
\end{itemize}
\begin{itemize}
\item {Proveniência:(De \textunderscore sebo\textunderscore )}
\end{itemize}
Sal, resultante da combinação do ácido sebácico com uma base.
\section{Sebe}
\begin{itemize}
\item {Grp. gram.:f.}
\end{itemize}
\begin{itemize}
\item {Proveniência:(Do lat. \textunderscore sepes\textunderscore )}
\end{itemize}
Tapume de ramos ou de varas entretecidas, para vedar terrenos.
Taipa.
Tabique.
Tapume de varas delgadas, com que se cerca o tabuleiro dos carros, para amprarar a carga.--Êste tapume de carro, na região da Bairrada, é designado exclusivamente pelo plural, \textunderscore sebes\textunderscore .
\section{Sebeiro}
\begin{itemize}
\item {Grp. gram.:m.}
\end{itemize}
Pedaço de madeira, com que os calafates põem sebo nas brocas, verrumões, etc.
Aquelle que prepara ou vende sebo.
\section{Sebel}
\begin{itemize}
\item {Grp. gram.:adj.}
\end{itemize}
\begin{itemize}
\item {Utilização:Anat.}
\end{itemize}
\begin{itemize}
\item {Utilização:Ant.}
\end{itemize}
\begin{itemize}
\item {Proveniência:(T. ár.)}
\end{itemize}
Dizia-se de uma veia ocular, que os médicos também chamavam \textunderscore dilatativa\textunderscore . Cf. Sousa, \textunderscore Vestig. da Ling. Aráb.\textunderscore 
\section{Sebenta}
\begin{itemize}
\item {Grp. gram.:f.}
\end{itemize}
\begin{itemize}
\item {Proveniência:(De \textunderscore sebo\textunderscore )}
\end{itemize}
Lição ou explicação lithographada, para uso de estudantes de Coímbra.
\section{Sebentão}
\begin{itemize}
\item {Grp. gram.:m.  e  adj.}
\end{itemize}
\begin{itemize}
\item {Utilização:Fam.}
\end{itemize}
Indivíduo muito sebento, muito sujo, desmazelado.
\section{Sebentaria}
\begin{itemize}
\item {Grp. gram.:f.}
\end{itemize}
\begin{itemize}
\item {Utilização:T. de Coímbra}
\end{itemize}
Officina, onde se lithographam sebentas.
\section{Sebenteiro}
\begin{itemize}
\item {Grp. gram.:m.  e  adj.}
\end{itemize}
Estudante, que redíge ou escreve a sebenta.
O que só estuda a sebenta.
Aquelle que lithographa sebentas.
\section{Sebentice}
\begin{itemize}
\item {Grp. gram.:f.}
\end{itemize}
Qualidade do que é sebento.
Sujidade ou falta de limpeza no vestuario.
\section{Sebento}
\begin{itemize}
\item {Grp. gram.:adj.}
\end{itemize}
\begin{itemize}
\item {Grp. gram.:M.}
\end{itemize}
Que é da natureza do sebo.
Que usa fato muito sujo; immundo.
Indivíduo sebento.
\section{Sebereba}
\begin{itemize}
\item {Grp. gram.:f.}
\end{itemize}
\begin{itemize}
\item {Utilização:Bras. do N}
\end{itemize}
Bebida refrigerante, o mesmo que \textunderscore tiquara\textunderscore .
\section{Sebesta}
\begin{itemize}
\item {Grp. gram.:f.}
\end{itemize}
\begin{itemize}
\item {Proveniência:(Do ár. \textunderscore sebesten\textunderscore )}
\end{itemize}
Espécie de ameixa escura do Oriente.
\section{Sebesteira}
\begin{itemize}
\item {Grp. gram.:f.}
\end{itemize}
O mesmo que \textunderscore sebesteiro\textunderscore .
\section{Sebesteiro}
\begin{itemize}
\item {Grp. gram.:m.}
\end{itemize}
\begin{itemize}
\item {Proveniência:(De \textunderscore sebesta\textunderscore )}
\end{itemize}
Árvore borragínea, que produz a sebesta, (\textunderscore cordia myxa\textunderscore , Lin.).
\section{Sebina}
\begin{itemize}
\item {Grp. gram.:f.}
\end{itemize}
\begin{itemize}
\item {Utilização:Prov.}
\end{itemize}
\begin{itemize}
\item {Utilização:trasm.}
\end{itemize}
Prego da ferragem das rodas do carro; prego de trilho.
(Provavelmente, alter. de \textunderscore sovina\textunderscore . Cp. \textunderscore sovina\textunderscore )
\section{Sebipira}
\begin{itemize}
\item {Grp. gram.:f.}
\end{itemize}
Árvore leguminosa do Brasil, o mesmo que \textunderscore sicupira\textunderscore .
\section{Sebo}
\begin{itemize}
\item {fónica:sê}
\end{itemize}
\begin{itemize}
\item {Grp. gram.:m.}
\end{itemize}
\begin{itemize}
\item {Utilização:Bras}
\end{itemize}
\begin{itemize}
\item {Grp. gram.:Interj.}
\end{itemize}
\begin{itemize}
\item {Utilização:Chul.}
\end{itemize}
\begin{itemize}
\item {Proveniência:(Lat. \textunderscore sebum\textunderscore )}
\end{itemize}
Substância gorda e consistente, extrahida das vísceras abdominaes de alguns quadrúpedes.
Casa de alfarrabista.
(indicativa de \textunderscore desagrado\textunderscore  ou \textunderscore desprêzo\textunderscore ); bolas! cebolório!
\section{Sebório}
\begin{itemize}
\item {Grp. gram.:m.}
\end{itemize}
\begin{itemize}
\item {Utilização:Ant.}
\end{itemize}
\begin{itemize}
\item {Proveniência:(De \textunderscore sebo\textunderscore )}
\end{itemize}
Homem emporcalhado, cheio de nódoas, nojento, sebento.
\section{Seborreia}
\begin{itemize}
\item {Grp. gram.:f.}
\end{itemize}
\begin{itemize}
\item {Proveniência:(Do lat. \textunderscore sebum\textunderscore  + gr. \textunderscore rhein\textunderscore )}
\end{itemize}
Erupção de pelle na base dos cabellos.
\section{Seborreico}
\begin{itemize}
\item {Grp. gram.:adj.}
\end{itemize}
Relativo á seborreia.
\section{Seborrheia}
\begin{itemize}
\item {Grp. gram.:f.}
\end{itemize}
\begin{itemize}
\item {Proveniência:(Do lat. \textunderscore sebum\textunderscore  + gr. \textunderscore rhein\textunderscore )}
\end{itemize}
Erupção de pelle na base dos cabellos.
\section{Seborrheico}
\begin{itemize}
\item {Grp. gram.:adj.}
\end{itemize}
Relativo á seborrheia.
\section{Seboso}
\begin{itemize}
\item {Grp. gram.:adj.}
\end{itemize}
\begin{itemize}
\item {Proveniência:(Lat. \textunderscore sebosus\textunderscore )}
\end{itemize}
Coberto ou sujo de sebo; sebáceo.
\section{Sebraju}
\begin{itemize}
\item {Grp. gram.:m.}
\end{itemize}
\begin{itemize}
\item {Utilização:Bras}
\end{itemize}
Árvore silvestre, cuja madeira vermelha é bôa para construcções.
\section{Sebruno}
\begin{itemize}
\item {Grp. gram.:adj.}
\end{itemize}
\begin{itemize}
\item {Utilização:Bras}
\end{itemize}
Diz-se do cavallo meio escuro.
(Alter. de \textunderscore zebruno\textunderscore ?)
\section{Sebuu-uva}
\begin{itemize}
\item {Grp. gram.:f.}
\end{itemize}
\begin{itemize}
\item {Utilização:Bras}
\end{itemize}
Planta apocýnea.
\section{Séca}
\begin{itemize}
\item {Grp. gram.:f.}
\end{itemize}
\begin{itemize}
\item {Utilização:Pop.}
\end{itemize}
\begin{itemize}
\item {Utilização:Bras. de Minas}
\end{itemize}
\begin{itemize}
\item {Grp. gram.:M.}
\end{itemize}
\begin{itemize}
\item {Proveniência:(De \textunderscore secar\textunderscore )}
\end{itemize}
Maçada.
Impertinência; estopada.
Luxo; ceremónia: \textunderscore é um homem sem séca\textunderscore .
Indivíduo maçador, importuno.
\section{Séca}
\begin{itemize}
\item {Grp. gram.:f.}
\end{itemize}
Acto de pôr a secar ou a enxugar.
\section{Sêca}
\begin{itemize}
\item {Grp. gram.:f.}
\end{itemize}
\begin{itemize}
\item {Utilização:T. de Alcanena}
\end{itemize}
Acto ou effeito de secar; estiagem.
Bofetada ou pancada sonora.
\section{Séca-bofes}
\begin{itemize}
\item {Grp. gram.:m.}
\end{itemize}
\begin{itemize}
\item {Utilização:Ant.}
\end{itemize}
\begin{itemize}
\item {Utilização:Fam.}
\end{itemize}
O mesmo que \textunderscore breviário\textunderscore . Cf. Filinto, IX, 28.
\section{Secação}
\begin{itemize}
\item {Grp. gram.:f.}
\end{itemize}
O mesmo que \textunderscore sêca\textunderscore .
\section{Secace}
\begin{itemize}
\item {Grp. gram.:m.  e  adj.}
\end{itemize}
\begin{itemize}
\item {Utilização:Ant.}
\end{itemize}
O mesmo que \textunderscore sequaz\textunderscore . Cf. \textunderscore Eufrosina\textunderscore , 182.
\section{Secadal}
\begin{itemize}
\item {Grp. gram.:m.}
\end{itemize}
\begin{itemize}
\item {Utilização:Prov.}
\end{itemize}
\begin{itemize}
\item {Utilização:trasm.}
\end{itemize}
Terra cultivada, que não é regadia; sequeiro.
(Cp. \textunderscore sêco\textunderscore )
\section{Secadeira}
\begin{itemize}
\item {Grp. gram.:f.}
\end{itemize}
\begin{itemize}
\item {Proveniência:(De \textunderscore secar\textunderscore )}
\end{itemize}
Um dos compartimentos da chocadeira, onde se collocam os pintaínhos, logo depois de nascerem, para se lhes secar a humidade da pennugem.
\section{Secadoiro}
\begin{itemize}
\item {Grp. gram.:m.}
\end{itemize}
Lugar, onde seca alguma coisa.
\section{Secadouro}
\begin{itemize}
\item {Grp. gram.:m.}
\end{itemize}
Lugar, onde seca alguma coisa.
\section{Séca-e-meca}
\begin{itemize}
\item {Grp. gram.:f.}
\end{itemize}
\begin{itemize}
\item {Utilização:Fam.}
\end{itemize}
Usado na loc. \textunderscore andar por séca-e-meca\textunderscore , andar por várias terras, por aqui e por ali; vaguear; fazer digressões ao acaso.
\section{Secagem}
\begin{itemize}
\item {Grp. gram.:f.}
\end{itemize}
\begin{itemize}
\item {Proveniência:(De \textunderscore secar\textunderscore )}
\end{itemize}
Operação, feita aos grãos da cevada, tornando-os loiros e amargos, para se tornarem aptos para o fabrico da cerveja.
\section{Secalhal}
\begin{itemize}
\item {Grp. gram.:adj.}
\end{itemize}
\begin{itemize}
\item {Utilização:Prov.}
\end{itemize}
\begin{itemize}
\item {Utilização:trasm.}
\end{itemize}
O mesmo que \textunderscore sêco\textunderscore .
(Cp. \textunderscore secadal\textunderscore )
\section{Secamente}
\begin{itemize}
\item {Grp. gram.:adv.}
\end{itemize}
De modo sêco; com frieza ou descortesia.
\section{Secância}
\begin{itemize}
\item {Grp. gram.:f.}
\end{itemize}
Qualidade de secante^1. Cf. Garrett, \textunderscore Helena\textunderscore .
\section{Secante}
\begin{itemize}
\item {Grp. gram.:m. ,  f.  e  adj.}
\end{itemize}
\begin{itemize}
\item {Proveniência:(De \textunderscore secar\textunderscore ^1)}
\end{itemize}
Pessôa que séca, pessôa importuna.
\section{Secante}
\begin{itemize}
\item {Grp. gram.:m.  e  adj.}
\end{itemize}
\begin{itemize}
\item {Utilização:Geom.}
\end{itemize}
\begin{itemize}
\item {Proveniência:(Do lat. \textunderscore secans\textunderscore , \textunderscore secantis\textunderscore )}
\end{itemize}
Diz-se da linha ou superfície, que corta outra.
\section{Secante}
\begin{itemize}
\item {Grp. gram.:adj.}
\end{itemize}
\begin{itemize}
\item {Grp. gram.:M.}
\end{itemize}
\begin{itemize}
\item {Proveniência:(Do lat. \textunderscore siccans\textunderscore )}
\end{itemize}
Que seca.
Substância usada pelos pintores, para fazer secar facilmente as tintas.
\section{Secar}
\begin{itemize}
\item {Grp. gram.:v. t.}
\end{itemize}
Importunar, maçar; ensecar.
(Alter. de \textunderscore ensecar\textunderscore ^2)
\section{Secar}
\begin{itemize}
\item {Grp. gram.:v. t.}
\end{itemize}
\begin{itemize}
\item {Utilização:Náut.}
\end{itemize}
\begin{itemize}
\item {Grp. gram.:V. i.}
\end{itemize}
\begin{itemize}
\item {Utilização:pop.}
\end{itemize}
\begin{itemize}
\item {Proveniência:(Do lat. \textunderscore siccare\textunderscore )}
\end{itemize}
Tirar a humidade a.
Tornar enxuto: \textunderscore secar a roupa\textunderscore .
Estancar.
Tornar murcho: \textunderscore o calor secou as plantas\textunderscore .
Fazer cessar.
Ferrar (a vela do navio).
Deixar de ser húmido.
Deixar de correr (falando-se de líquidos ou de humores).
Sumir-se (a voz).
Evaporar-se.
Estancar-se.
Cessar.
Ammadurecer muito.
Enfraquecer-se; mirrar-se; murchar.
\section{Secarrão}
\begin{itemize}
\item {Grp. gram.:adj.}
\end{itemize}
\begin{itemize}
\item {Utilização:Pop.}
\end{itemize}
Muito sêco.
\section{Secativo}
\begin{itemize}
\item {Grp. gram.:m.  e  adj.}
\end{itemize}
\begin{itemize}
\item {Proveniência:(Do lat. \textunderscore siccativus\textunderscore )}
\end{itemize}
Diz-se da preparação pharmacêutica, que tem acção adstringente nos tecidos vivos.
\section{Secatória}
\begin{itemize}
\item {Grp. gram.:f.}
\end{itemize}
\begin{itemize}
\item {Proveniência:(Do lat. \textunderscore secatus\textunderscore )}
\end{itemize}
Tesoira de jardineiro e enxertador.
\section{Secatura}
\begin{itemize}
\item {Grp. gram.:f.}
\end{itemize}
O mesmo que \textunderscore séca\textunderscore ^1.
\section{Secção}
\begin{itemize}
\item {Grp. gram.:f.}
\end{itemize}
\begin{itemize}
\item {Proveniência:(Do lat. \textunderscore sectio\textunderscore )}
\end{itemize}
Acto ou effeito de cortar.
Parte de um todo.
Divisão de um gênero, em História Natural.
Divisão ou subdivisão de uma obra, de um tratado, etc.
Córte de um edifício ou de outro corpo pelo centro, para se lhe observar a disposição interior.
Cada uma das divisões de uma Repartição pública.
Linha ou superfície, segundo a qual se cortam duas superfícies, dois sólidos ou uma superfície e um sólido.
Córte vertical.
\section{Seccional}
\begin{itemize}
\item {Grp. gram.:adj.}
\end{itemize}
\begin{itemize}
\item {Proveniência:(Do lat. \textunderscore sectio\textunderscore )}
\end{itemize}
Relativo a secção.
\section{Seccionar}
\begin{itemize}
\item {Grp. gram.:v. t.}
\end{itemize}
Dividir em secções.
\section{Secéspita}
\begin{itemize}
\item {Grp. gram.:f.}
\end{itemize}
\begin{itemize}
\item {Proveniência:(Lat. \textunderscore secespita\textunderscore )}
\end{itemize}
Cutello, que se usava nos antigos sacrifícios.
\section{Secessão}
\begin{itemize}
\item {Grp. gram.:f.}
\end{itemize}
\begin{itemize}
\item {Proveniência:(Lat. \textunderscore secessio\textunderscore )}
\end{itemize}
Separação; retirada. Cf. Latino, \textunderscore Or. da Corôa\textunderscore , CCXXX.
\section{Secesso}
\begin{itemize}
\item {Grp. gram.:m.}
\end{itemize}
\begin{itemize}
\item {Proveniência:(Lat. \textunderscore secessus\textunderscore )}
\end{itemize}
Lugar afastado.
Retiro; esconso; recesso.
\section{Sechuana}
\begin{itemize}
\item {Grp. gram.:m.}
\end{itemize}
Língua banta, do centro da África meridional.
\section{Sécia}
\begin{itemize}
\item {Grp. gram.:f.}
\end{itemize}
\begin{itemize}
\item {Utilização:Bot.}
\end{itemize}
\begin{itemize}
\item {Proveniência:(De \textunderscore sécio\textunderscore )}
\end{itemize}
Mulhér casquilha e presumida.
Espécie de roupão, para senhora ou menina.
Veneta; balda.
Prenda.
Planta, da fam. das compostas, (\textunderscore callistephus chinensis\textunderscore , Nees).
O mesmo que \textunderscore enfeite\textunderscore . Cf. Castilho, \textunderscore Fausto\textunderscore , 369.
Última moda. Cf. Filinto, VI, 6.
\section{Sécio}
\begin{itemize}
\item {Grp. gram.:m.  e  adj.}
\end{itemize}
Indivíduo, que traja com affectação; peralvilho; casquilho.
Aquelle que se saracoteia muito.
\section{Sêco}
\begin{itemize}
\item {Grp. gram.:adj.}
\end{itemize}
\begin{itemize}
\item {Utilização:Fig.}
\end{itemize}
\begin{itemize}
\item {Utilização:Pop.}
\end{itemize}
\begin{itemize}
\item {Utilização:T. da Bairrada}
\end{itemize}
\begin{itemize}
\item {Grp. gram.:M.}
\end{itemize}
\begin{itemize}
\item {Grp. gram.:Pl.}
\end{itemize}
\begin{itemize}
\item {Utilização:Bras}
\end{itemize}
\begin{itemize}
\item {Proveniência:(Do lat. \textunderscore siccus\textunderscore )}
\end{itemize}
Diz-se do tempo, em que não há chuva ou humidade.
Áspero; árido; que não tem vegetação: \textunderscore terra sêca\textunderscore .
Insensível; que se não commove.
Que não commove.
Severo.
Descortês; rude.
Que não tem suavidade, (falando-se de obras de arte).
Despejado, esgotado: \textunderscore cisterna sêca\textunderscore .
Diz-se do indivíduo, a quem a morpheia fez desapparecer os dedos.
Baixo de areia, que a vasante deixa a descoberto.
Mantimentos sólidos ou secos, por opposição a bebidas e outros líquidos: \textunderscore o Soisa tem uma loja de secos e molhados na rua do Ouvidor\textunderscore .
\section{Secreção}
\begin{itemize}
\item {Grp. gram.:f.}
\end{itemize}
\begin{itemize}
\item {Utilização:Ext.}
\end{itemize}
\begin{itemize}
\item {Proveniência:(Lat. \textunderscore secretio\textunderscore )}
\end{itemize}
Propriedade dos tecidos orgânicos, em virtude da qual saem da substância delles as moléculas interiores que, segundo a sua natureza, são expellidas ou reabsorvidas ou fixadas nas cavidades do organismo.
Substância segregada.
Quaesquer matérias, que saem do corpo; excreção.
\section{Secresto}
\textunderscore m.\textunderscore  (e der.)
Fórma pop. de \textunderscore sequestro\textunderscore , etc.
(Cp. toscano \textunderscore secresto\textunderscore )
\section{Secreta}
\begin{itemize}
\item {Grp. gram.:f.}
\end{itemize}
\begin{itemize}
\item {Utilização:Pop.}
\end{itemize}
\begin{itemize}
\item {Utilização:T. de Lisbôa}
\end{itemize}
\begin{itemize}
\item {Proveniência:(Lat. \textunderscore secreta\textunderscore )}
\end{itemize}
These que, nalgumas Universidades, é defendida só na presença dos lentes.
Oração que o celebrante da Missa diz em voz baixa, antes do prefácio.
O mesmo que \textunderscore latrina\textunderscore .
Polícia secreta: \textunderscore o Silva pertence á secreta\textunderscore .
\section{Secularidade}
\begin{itemize}
\item {Grp. gram.:f.}
\end{itemize}
Qualidade do que é secular.
Acto ou dito, que é próprio de leigos ou de pessôas seculares.
\section{Secularização}
\begin{itemize}
\item {Grp. gram.:f.}
\end{itemize}
Acto ou effeito de secularizar.
\section{Secularizar}
\begin{itemize}
\item {Grp. gram.:v. t.}
\end{itemize}
\begin{itemize}
\item {Grp. gram.:V. p.}
\end{itemize}
Tornar secular ou leigo (o que era ecclesiástico).
Sujeitar á lei commum, á lei civil.
Dispensar dos votos monásticos.
Deixar de pertencer a uma Ordem religiosa.
\section{Secularmente}
\begin{itemize}
\item {Grp. gram.:adv.}
\end{itemize}
De modo secular; de séculos em séculos.
\section{Século}
\begin{itemize}
\item {Grp. gram.:m.}
\end{itemize}
\begin{itemize}
\item {Proveniência:(Lat. \textunderscore saeculum\textunderscore )}
\end{itemize}
Espaço de cem annos.
Longo tempo; grande espaço de tempo.
Época.
Duração de alguma coisa notável.
O tempo presente.
Vida secular.
O mundo, considerado pelos mýsticos sob ponto de vista das tentações e vaidades.
\section{Secúlo}
\begin{itemize}
\item {Grp. gram.:m.}
\end{itemize}
Cada um dos indivíduos que, com os Macotas, constituem, por assim dizer, o estado-maior do soba. Cf. Capello e Ivens, I, 173.
\section{Secunda}
\begin{itemize}
\item {Grp. gram.:f.}
\end{itemize}
\begin{itemize}
\item {Utilização:Ant.}
\end{itemize}
\begin{itemize}
\item {Proveniência:(T. lat.)}
\end{itemize}
Os cereaes de segunda ordem, milho miúdo e paínço, em quanto o trigo, o centeio e a cevada se consideravam cereaes de primeira ordem.
\section{Secundar}
\begin{itemize}
\item {Grp. gram.:v. t.}
\end{itemize}
\begin{itemize}
\item {Grp. gram.:V. i.}
\end{itemize}
\begin{itemize}
\item {Proveniência:(Lat. \textunderscore secundare\textunderscore )}
\end{itemize}
Reforçar, ajudar, coadjuvar.
Fazer pela segunda vez, repetir.
Repetir um acto.--Há quem duvide da vernaculidade do termo, porque suppõe-se, não foi usado por clássicos; parece-me porém lidimamente português, de bôa derivação, expressivo e vulgarizável, embora os Franceses, antes de nós termos \textunderscore secundar\textunderscore , tivessem \textunderscore seconder\textunderscore .
\section{Secundariamente}
\begin{itemize}
\item {Grp. gram.:adv.}
\end{itemize}
De modo secundário.
Em segundo lugar; inferiormente.
\section{Secundário}
\begin{itemize}
\item {Grp. gram.:adj.}
\end{itemize}
\begin{itemize}
\item {Utilização:Geol.}
\end{itemize}
\begin{itemize}
\item {Proveniência:(Lat. \textunderscore secundarius\textunderscore )}
\end{itemize}
Que está em segundo lugar.
Que está em segunda ordem.
Que é menos importante que outro ou outrem.
Inferior.
Insignificante.
Diz-se o ensino ou a instrucção, que é intermédia á primária e á superior, e especialmente a que é ministrada nos lyceus.
Diz-se o segundo período geológico, os terrenos dêsse período e os respectivos seres.
\section{Secundianos}
\begin{itemize}
\item {Grp. gram.:m. pl.}
\end{itemize}
\begin{itemize}
\item {Proveniência:(Lat. \textunderscore secundianus\textunderscore )}
\end{itemize}
Soldados da segunda legião, entre os antigos Romanos.
\section{Secundifalange}
\begin{itemize}
\item {Grp. gram.:f.}
\end{itemize}
\begin{itemize}
\item {Utilização:Anat.}
\end{itemize}
A segunda falange do pé.
\section{Secundifalangeta}
\begin{itemize}
\item {fónica:gê}
\end{itemize}
\begin{itemize}
\item {Grp. gram.:f.}
\end{itemize}
A segunda falangeta do pé.
\section{Secundifalanginha}
\begin{itemize}
\item {Grp. gram.:f.}
\end{itemize}
\begin{itemize}
\item {Utilização:Anat.}
\end{itemize}
A segunda falanginha do pé.
\section{Secundimetatársico}
\begin{itemize}
\item {Grp. gram.:adj.}
\end{itemize}
Diz-se do segundo ôsso metatársico.
\section{Secundinas}
\begin{itemize}
\item {Grp. gram.:f. pl.}
\end{itemize}
\begin{itemize}
\item {Proveniência:(Do lat. \textunderscore secundus\textunderscore )}
\end{itemize}
Placenta e membranas, que ficam na madre, depois do parto.
\section{Secundinistas}
\begin{itemize}
\item {Grp. gram.:m. pl.}
\end{itemize}
Herejes, que negavam a immutabilidade e a immortalidade de Deus.
\section{Secundípara}
\begin{itemize}
\item {Grp. gram.:adj.}
\end{itemize}
\begin{itemize}
\item {Proveniência:(Do lat. \textunderscore secundus\textunderscore  + \textunderscore parere\textunderscore )}
\end{itemize}
Diz-se da mulhér, que pariu pela segunda vez. Cp. \textunderscore primípara\textunderscore .
\section{Secundiphalange}
\begin{itemize}
\item {Grp. gram.:f.}
\end{itemize}
\begin{itemize}
\item {Utilização:Anat.}
\end{itemize}
A segunda phalange do pé.
\section{Secundiphalangeta}
\begin{itemize}
\item {fónica:gê}
\end{itemize}
\begin{itemize}
\item {Grp. gram.:f.}
\end{itemize}
A segunda phalangeta do pé.
\section{Secundiphalanginha}
\begin{itemize}
\item {Grp. gram.:f.}
\end{itemize}
\begin{itemize}
\item {Utilização:Anat.}
\end{itemize}
A segunda phalanginha do pé.
\section{Secundo-gênito}
\begin{itemize}
\item {Grp. gram.:m.  e  adj.}
\end{itemize}
(V.segundo-gênito)
\section{Secura}
\begin{itemize}
\item {Grp. gram.:f.}
\end{itemize}
Qualidade do que é sêco ou enxuto.
O mesmo que \textunderscore sêde\textunderscore .
\section{Secure}
\begin{itemize}
\item {Grp. gram.:f.}
\end{itemize}
O mesmo que \textunderscore segure\textunderscore . Cf. Herculano, \textunderscore Eurico\textunderscore , 251.
\section{Securidaca}
\begin{itemize}
\item {Grp. gram.:f.}
\end{itemize}
Gênero de plantas papilionáceas.
\section{Securiforme}
\begin{itemize}
\item {Grp. gram.:adj.}
\end{itemize}
\begin{itemize}
\item {Proveniência:(Do lat. \textunderscore securis\textunderscore  + \textunderscore forma\textunderscore )}
\end{itemize}
Que tem fórma de machadinha.
\section{Securígera}
\begin{itemize}
\item {Grp. gram.:f.}
\end{itemize}
\begin{itemize}
\item {Proveniência:(Do lat. \textunderscore securis\textunderscore  + \textunderscore gerere\textunderscore )}
\end{itemize}
Gênero de plantas leguminosas.
\section{Securígero}
\begin{itemize}
\item {Grp. gram.:adj.}
\end{itemize}
\begin{itemize}
\item {Utilização:Bot.}
\end{itemize}
\begin{itemize}
\item {Proveniência:(Do lat. \textunderscore securis\textunderscore  + \textunderscore gerere\textunderscore )}
\end{itemize}
Que tem órgão ou appêndice em fórma de machadinha.
\section{Securinega}
\begin{itemize}
\item {Grp. gram.:f.}
\end{itemize}
Gênero de plantas euphorbiáceas, indígenas de França.
\section{Securipalpo}
\begin{itemize}
\item {Grp. gram.:adj.}
\end{itemize}
\begin{itemize}
\item {Utilização:Zool.}
\end{itemize}
\begin{itemize}
\item {Proveniência:(Do lat. \textunderscore securis\textunderscore  + \textunderscore palpus\textunderscore )}
\end{itemize}
Que tem palpos em fórma de machadinha.
\section{Secussão}
\begin{itemize}
\item {Grp. gram.:f.}
\end{itemize}
Abalo.
Grande perturbação. Cf. Latino, \textunderscore Humboldt\textunderscore , 216.
(Certamente êrro gráphico, em vez de \textunderscore succussão\textunderscore , do lat. \textunderscore succussio\textunderscore )
\section{Secutor}
\begin{itemize}
\item {Grp. gram.:m.}
\end{itemize}
\begin{itemize}
\item {Proveniência:(Lat. \textunderscore sequutor\textunderscore )}
\end{itemize}
Gladiador, que suppria o lugar de outro, que havia sido morto.
\section{Séda}
\begin{itemize}
\item {Grp. gram.:f.}
\end{itemize}
\begin{itemize}
\item {Utilização:Ant.}
\end{itemize}
O mesmo que \textunderscore sêde\textunderscore .
\section{Sêda}
\begin{itemize}
\item {Grp. gram.:f.}
\end{itemize}
\begin{itemize}
\item {Utilização:Bot.}
\end{itemize}
\begin{itemize}
\item {Grp. gram.:Pl.}
\end{itemize}
\begin{itemize}
\item {Utilização:Pop.}
\end{itemize}
\begin{itemize}
\item {Proveniência:(Do lat. \textunderscore seta\textunderscore )}
\end{itemize}
Substancia filamentosa, produzida pela larva de um insecto, chamado vulgarmente bicho da sêda.
Tecido, feito com essa substância.
Pequena fenda em instrumentos mecânicos, pela qual ordinariamente se partem.
Pêlo áspero, nos invólucros floraes das gramíneas.
Pêlos ásperos e compridos de certos animaes.
Trajes de seda.
\section{Sedação}
\begin{itemize}
\item {Grp. gram.:f.}
\end{itemize}
\begin{itemize}
\item {Proveniência:(Lat. \textunderscore sedatio\textunderscore )}
\end{itemize}
Acto de sedar.
\section{Sedaceiro}
\begin{itemize}
\item {Grp. gram.:m.}
\end{itemize}
Aquelle que trabalha em sedaços.
\section{Sedaço}
\begin{itemize}
\item {Grp. gram.:m.}
\end{itemize}
\begin{itemize}
\item {Proveniência:(Do lat. \textunderscore setaceus\textunderscore )}
\end{itemize}
Sêda rala para peneiras.
Instrumento, com que se côa o leite.
\section{Sedadeiro}
\begin{itemize}
\item {Grp. gram.:m.}
\end{itemize}
O mesmo que \textunderscore sedeiro\textunderscore . Cf. \textunderscore Inquér. Industr.\textunderscore , 2.^a p., l. III, 217.
\section{Sedal}
\begin{itemize}
\item {Grp. gram.:adj.}
\end{itemize}
\begin{itemize}
\item {Proveniência:(De \textunderscore séde\textunderscore )}
\end{itemize}
Relativo ao ânus.
\section{Sedalha}
\begin{itemize}
\item {Grp. gram.:f.}
\end{itemize}
O mesmo que \textunderscore sedela\textunderscore .
\section{Sedante}
\begin{itemize}
\item {Grp. gram.:adj.}
\end{itemize}
\begin{itemize}
\item {Proveniência:(De \textunderscore sedar\textunderscore ^1)}
\end{itemize}
O mesmo que \textunderscore sedativo\textunderscore . Cf. Ortigão, \textunderscore Praias\textunderscore , 121; Queirós, \textunderscore Comédia do Campo\textunderscore , II, 96.
\section{Sedão}
\begin{itemize}
\item {Grp. gram.:m.}
\end{itemize}
\begin{itemize}
\item {Proveniência:(De \textunderscore sêda\textunderscore )}
\end{itemize}
Anomalia ou doença do gado suíno, a qual consiste numa fístula ao pé das parótidas, com um feixe de sedas ou cordas, profundamente encravadas, do que resulta inflammação e ás vezes gangrena. Cf. \textunderscore Gaz.-das-Aldeias\textunderscore , de 20-V-906.
\section{Sedar}
\begin{itemize}
\item {Grp. gram.:v. t.}
\end{itemize}
\begin{itemize}
\item {Proveniência:(Lat. \textunderscore sedare\textunderscore )}
\end{itemize}
Acalmar (aquelle ou aquillo que estava excitado ou perturbado).
Moderar a acção excessiva de (um órgão ou systema de órgãos).
\section{Sedar}
\begin{itemize}
\item {Grp. gram.:v. t.}
\end{itemize}
\begin{itemize}
\item {Proveniência:(De \textunderscore sêda\textunderscore )}
\end{itemize}
O mesmo que \textunderscore assedar\textunderscore . Cf. \textunderscore Inquér. Industr.\textunderscore , 2.^a p., l. III, 68.
\section{Sedativo}
\begin{itemize}
\item {Grp. gram.:m.  e  adj.}
\end{itemize}
\begin{itemize}
\item {Proveniência:(De \textunderscore sedar\textunderscore )}
\end{itemize}
O que séda ou acalma, (falando-se especialmente de medicamentos).
Calmante.
\section{Séde}
\begin{itemize}
\item {Grp. gram.:f.}
\end{itemize}
\begin{itemize}
\item {Proveniência:(Lat. \textunderscore sedes\textunderscore )}
\end{itemize}
Lugar em que alguém se póde assentar.
Capital de uma diocese; diocese; jurisdicção exercida numa diocese.
Ponto, onde se concentram certos factos ou phenómenos; lugar onde se realiza um acontecimento.
Assento de pedra, lixado na parede, junto á janela, usado especialmente em construcções antigas.
\section{Sêde}
\begin{itemize}
\item {Grp. gram.:f.}
\end{itemize}
\begin{itemize}
\item {Utilização:Pop.}
\end{itemize}
\begin{itemize}
\item {Utilização:Fig.}
\end{itemize}
\begin{itemize}
\item {Proveniência:(Do lat. \textunderscore sitis\textunderscore )}
\end{itemize}
Sensação da necessidade de beber ou introduzir um líquido na bôca ou no estômago.
Pequena porção, um gole:«\textunderscore uma sêde, uma só, de água, uma só, por compaixão.\textunderscore »Garrett, \textunderscore Adozinda\textunderscore .
Desejo de vingança: \textunderscore tenho-lhe uma sêde...\textunderscore 
Grande desejo; cubiça; avidez.
Impaciência.
Falta de humidade, secura.
\section{Sedear}
\begin{itemize}
\item {Grp. gram.:v. t.}
\end{itemize}
\begin{itemize}
\item {Proveniência:(De \textunderscore sêda\textunderscore )}
\end{itemize}
Escovar com sêdas (objectos de ourivezaria).
\section{Sedeiro}
\begin{itemize}
\item {Grp. gram.:m.}
\end{itemize}
\begin{itemize}
\item {Proveniência:(De \textunderscore sêda\textunderscore )}
\end{itemize}
O mesmo que \textunderscore rastello\textunderscore .
\section{Sedela}
\begin{itemize}
\item {Grp. gram.:f.}
\end{itemize}
\begin{itemize}
\item {Proveniência:(De \textunderscore sêda\textunderscore )}
\end{itemize}
Cordel de sêdas, que sustenta o anzol, na pesca á linha.
\section{Sedém}
\begin{itemize}
\item {Grp. gram.:m.}
\end{itemize}
\begin{itemize}
\item {Utilização:Bras}
\end{itemize}
\begin{itemize}
\item {Utilização:Ant.}
\end{itemize}
\begin{itemize}
\item {Proveniência:(De \textunderscore sedenho\textunderscore ?)}
\end{itemize}
Cauda dos animaes.
\section{Sedenho}
\begin{itemize}
\item {Grp. gram.:m.}
\end{itemize}
\begin{itemize}
\item {Utilização:Prov.}
\end{itemize}
\begin{itemize}
\item {Utilização:beir.}
\end{itemize}
\begin{itemize}
\item {Utilização:Ant.}
\end{itemize}
\begin{itemize}
\item {Utilização:Prov.}
\end{itemize}
\begin{itemize}
\item {Utilização:alg.}
\end{itemize}
\begin{itemize}
\item {Proveniência:(De \textunderscore sêda\textunderscore )}
\end{itemize}
Mecha de fios, que se mete nos tecidos orgânicos, para extrahir delles os humores nocivos.
Fontanella.
Cordão de crina, com que se retesam as testeiras de uma serra de carpinteiro ou marceneiro.
Trança de sedas ou de pêlos da cauda do boi ou do cavallo, com que se prende ao assento o chanco das pescócias.
Cilício de sêdas ásperas e mortificantes.
Lato, ou cordão de pita, grosso.
\section{Sedentariamente}
\begin{itemize}
\item {Grp. gram.:adv.}
\end{itemize}
De modo sedentário.
\section{Sedentariedade}
\begin{itemize}
\item {Grp. gram.:f.}
\end{itemize}
Qualidade ou vida de sedentário.
\section{Sedentário}
\begin{itemize}
\item {Grp. gram.:adj.}
\end{itemize}
\begin{itemize}
\item {Grp. gram.:M.}
\end{itemize}
\begin{itemize}
\item {Proveniência:(Lat. \textunderscore sedentarius\textunderscore )}
\end{itemize}
Que está quási sempre sentado.
Que anda pouco ou faz pouco exercício.
Inactivo; relativo á inactividade: \textunderscore hábitos sedentários\textunderscore .
Que tem habitação fixa.
Aquelle que vive sedentariamente.
\section{Sedentarismo}
\begin{itemize}
\item {Grp. gram.:m.}
\end{itemize}
Hábitos sedentários.
\section{Sedente}
\begin{itemize}
\item {Grp. gram.:adj.}
\end{itemize}
\begin{itemize}
\item {Utilização:Poét.}
\end{itemize}
\begin{itemize}
\item {Proveniência:(Do lat. \textunderscore sitiens\textunderscore )}
\end{itemize}
O mesmo que \textunderscore sedento\textunderscore .
\section{Sedento}
\begin{itemize}
\item {Grp. gram.:adj.}
\end{itemize}
\begin{itemize}
\item {Utilização:Fig.}
\end{itemize}
\begin{itemize}
\item {Proveniência:(De \textunderscore sêde\textunderscore , se não é alter. de \textunderscore sedente\textunderscore )}
\end{itemize}
Que tem sêde; sequioso.
Que tem grande desejo ou avidez.
\section{Sederento}
\begin{itemize}
\item {Grp. gram.:adj.}
\end{itemize}
\begin{itemize}
\item {Utilização:Ant.}
\end{itemize}
O mesmo que \textunderscore sedento\textunderscore .
\section{Sedeúdo}
\begin{itemize}
\item {Grp. gram.:adj.}
\end{itemize}
\begin{itemize}
\item {Proveniência:(De \textunderscore seda\textunderscore )}
\end{itemize}
Sedoso; cabelludo:«\textunderscore ...põe á vela o sedeúdo rabo...\textunderscore »Filinto, II, 203.
\section{Sedição}
\begin{itemize}
\item {Grp. gram.:f.}
\end{itemize}
\begin{itemize}
\item {Proveniência:(Do lat. \textunderscore seditio\textunderscore )}
\end{itemize}
Perturbação da ordem pública.
Sublevação contra a autoridade legal.
Revólta; motim.
\section{Sediciosamente}
\begin{itemize}
\item {Grp. gram.:adv.}
\end{itemize}
De modo sedicioso; por meio de sedição.
\section{Sedicioso}
\begin{itemize}
\item {Grp. gram.:adj.}
\end{itemize}
\begin{itemize}
\item {Grp. gram.:M.}
\end{itemize}
\begin{itemize}
\item {Proveniência:(Lat. \textunderscore seditiosus\textunderscore )}
\end{itemize}
Que faz sedição ou toma parte nella.
Que tem o carácter de sedição.
Indócil, indisciplinado.
Indivíduo sedicioso.
\section{Sediço}
\begin{itemize}
\item {Grp. gram.:adj.}
\end{itemize}
\begin{itemize}
\item {Utilização:Fig.}
\end{itemize}
\begin{itemize}
\item {Proveniência:(Do lat. hyp. \textunderscore sedititius\textunderscore )}
\end{itemize}
Diz-se da água assente, estagnada, corrupta.
Que está fóra da moda.
Antigo.
Rotineiro; corriqueiro.
\section{Sediela}
\begin{itemize}
\item {Grp. gram.:f.}
\end{itemize}
\begin{itemize}
\item {Utilização:Pop.}
\end{itemize}
O mesmo que \textunderscore sedela\textunderscore .
\section{Sedígero}
\begin{itemize}
\item {Grp. gram.:adj.}
\end{itemize}
\begin{itemize}
\item {Utilização:Neol.}
\end{itemize}
\begin{itemize}
\item {Proveniência:(Do lat. \textunderscore seta\textunderscore  + \textunderscore gerere\textunderscore )}
\end{itemize}
Que produz sêda.
\section{Sedilúvio}
\begin{itemize}
\item {Grp. gram.:m.}
\end{itemize}
\begin{itemize}
\item {Proveniência:(Do lat. \textunderscore sedes\textunderscore  + \textunderscore luere\textunderscore )}
\end{itemize}
O mesmo que \textunderscore semicúpio\textunderscore .
\section{Sedimentação}
\begin{itemize}
\item {Grp. gram.:f.}
\end{itemize}
Formação de sedimentos.
\section{Sedimentar}
\begin{itemize}
\item {Grp. gram.:v. i.}
\end{itemize}
Formar sedimento.
\section{Sedimentar}
\begin{itemize}
\item {Grp. gram.:adj.}
\end{itemize}
Que tem o carácter de sedimento.
Produzido por sedimento.
\section{Sedimentário}
\begin{itemize}
\item {Grp. gram.:adj.}
\end{itemize}
O mesmo que \textunderscore sedimentar\textunderscore ^2.
\section{Sedimento}
\begin{itemize}
\item {Grp. gram.:m.}
\end{itemize}
\begin{itemize}
\item {Proveniência:(Lat. \textunderscore sedimentum\textunderscore )}
\end{itemize}
Depósito, resultante da precipitação de substâncias dissolvidas ou suspensas num líquido.
Fezes.
Camada, formada pelas matérias que as águas deixaram ao retirar-se.
\section{Sedimentoso}
\begin{itemize}
\item {Grp. gram.:adj.}
\end{itemize}
Sedimentar; que tem muitos sedimentos.
\section{Sedlitz}
\begin{itemize}
\item {Grp. gram.:m.}
\end{itemize}
\begin{itemize}
\item {Proveniência:(De \textunderscore Sedlitz\textunderscore , n. p.)}
\end{itemize}
Diz-se de uma água mineral, procedente de uma cidade bohêmia daquelle nome.
Diz-se de um sal, que é o sulfato de magnésia.
Diz-se de vários medicamentos, em que entra aquella água ou êste sulfato.
\section{Sedonho}
\begin{itemize}
\item {Grp. gram.:m.}
\end{itemize}
\begin{itemize}
\item {Proveniência:(De \textunderscore sêda\textunderscore )}
\end{itemize}
Doença dos porcos, caracterizada pelo nascimento de pêlos nas guelas.
\section{Sedorento}
\begin{itemize}
\item {Grp. gram.:adj.}
\end{itemize}
\begin{itemize}
\item {Utilização:Ant.}
\end{itemize}
O mesmo que \textunderscore sedento\textunderscore .
(Cp. \textunderscore sederento\textunderscore )
\section{Sedoso}
\begin{itemize}
\item {Grp. gram.:adj.}
\end{itemize}
\begin{itemize}
\item {Proveniência:(Do lat. \textunderscore setosus\textunderscore )}
\end{itemize}
Que tem sedas.
Semelhante á sêda.
Peludo.
\section{Sedução}
\begin{itemize}
\item {Grp. gram.:f.}
\end{itemize}
\begin{itemize}
\item {Proveniência:(Lat. \textunderscore seductio\textunderscore )}
\end{itemize}
Acto ou efeito de seduzir ou de sêr seduzido.
Qualidade do que seduz.
Atracção, encanto.
\section{Seducção}
\begin{itemize}
\item {Grp. gram.:f.}
\end{itemize}
\begin{itemize}
\item {Proveniência:(Lat. \textunderscore seductio\textunderscore )}
\end{itemize}
Acto ou effeito de seduzir ou de sêr seduzido.
Qualidade do que seduz.
Attracção, encanto.
\section{Seductor}
\begin{itemize}
\item {Grp. gram.:adj.}
\end{itemize}
\begin{itemize}
\item {Grp. gram.:M.}
\end{itemize}
\begin{itemize}
\item {Proveniência:(Lat. \textunderscore seductor\textunderscore )}
\end{itemize}
Que seduz.
Que attrái, que encanta.
Aquelle que seduz.
Homem, que deshonra uma mulhér por seducção.
\section{Sédulo}
\begin{itemize}
\item {Grp. gram.:adj.}
\end{itemize}
\begin{itemize}
\item {Proveniência:(Lat. \textunderscore sedulus\textunderscore )}
\end{itemize}
Activo; cuidadoso.
\section{Sedutor}
\begin{itemize}
\item {Grp. gram.:adj.}
\end{itemize}
\begin{itemize}
\item {Grp. gram.:M.}
\end{itemize}
\begin{itemize}
\item {Proveniência:(Lat. \textunderscore seductor\textunderscore )}
\end{itemize}
Que seduz.
Que attrái, que encanta.
Aquelle que seduz.
Homem, que deshonra uma mulhér por seducção.
\section{Seduzimento}
\begin{itemize}
\item {Grp. gram.:m.}
\end{itemize}
(V.seducção)
\section{Seduzir}
\begin{itemize}
\item {Grp. gram.:v. t.}
\end{itemize}
\begin{itemize}
\item {Utilização:Fig.}
\end{itemize}
\begin{itemize}
\item {Proveniência:(Lat. \textunderscore seducere\textunderscore )}
\end{itemize}
Desviar do caminho da verdade.
Fazer caír em êrro ou culpa.
Enganar ardilosamente.
Persuadir á prática do mal ou ao desvio dos bons costumes.
Deshonrar.
Subornar para fins illícitos.
Attrahir, encantar, fascinar, dominar a vontade de.
\section{Seduzível}
\begin{itemize}
\item {Grp. gram.:adj.}
\end{itemize}
\begin{itemize}
\item {Proveniência:(Lat. \textunderscore seducibilis\textunderscore )}
\end{itemize}
Que se póde seduzir.
\section{See}
\begin{itemize}
\item {Grp. gram.:f.}
\end{itemize}
\begin{itemize}
\item {Utilização:Ant.}
\end{itemize}
O mesmo que \textunderscore sé\textunderscore .
\section{Seeda}
\begin{itemize}
\item {Grp. gram.:f.}
\end{itemize}
\begin{itemize}
\item {Utilização:Ant.}
\end{itemize}
O mesmo que \textunderscore séde\textunderscore .
\section{Seelo}
\begin{itemize}
\item {Grp. gram.:m.}
\end{itemize}
\begin{itemize}
\item {Utilização:Ant.}
\end{itemize}
O mesmo que \textunderscore sêllo\textunderscore .
\section{Seenda}
\begin{itemize}
\item {Grp. gram.:f.}
\end{itemize}
\begin{itemize}
\item {Utilização:Ant.}
\end{itemize}
O mesmo que \textunderscore assento\textunderscore  ou \textunderscore morada\textunderscore .
(Por \textunderscore seêda\textunderscore )
\section{Seente}
\begin{itemize}
\item {Grp. gram.:adj.}
\end{itemize}
\begin{itemize}
\item {Utilização:Ant.}
\end{itemize}
\begin{itemize}
\item {Proveniência:(De \textunderscore seer\textunderscore )}
\end{itemize}
O mesmo que [[sentado|sentar]]. Cf. Frei Fortun., \textunderscore Inéd.\textunderscore , 314.
\section{Seer}
\begin{itemize}
\item {Grp. gram.:v. i.}
\end{itemize}
\begin{itemize}
\item {Utilização:Ant.}
\end{itemize}
\begin{itemize}
\item {Proveniência:(Do lat. \textunderscore sedere\textunderscore )}
\end{itemize}
Estar sentado.
O mesmo que \textunderscore sêr\textunderscore .
\section{Seetzênia}
\begin{itemize}
\item {Grp. gram.:f.}
\end{itemize}
\begin{itemize}
\item {Proveniência:(De \textunderscore Seetzen\textunderscore , n. p.)}
\end{itemize}
Gênero de plantas, de organização anómala.
\section{Sefelpa}
\begin{itemize}
\item {Grp. gram.:f.}
\end{itemize}
\begin{itemize}
\item {Utilização:Prov.}
\end{itemize}
\begin{itemize}
\item {Utilização:trasm.}
\end{itemize}
Casaco.
Tosa, tunda.
\section{Sefia}
\begin{itemize}
\item {Grp. gram.:f.}
\end{itemize}
Peixe esparoide, (\textunderscore sargus vulgaris\textunderscore ).
\section{Seflosa}
\begin{itemize}
\item {Grp. gram.:f.}
\end{itemize}
\begin{itemize}
\item {Utilização:Chapel.}
\end{itemize}
Máquina, para abrir e dividir o pêlo dos chapéus.--Registo o t. como o oiço, mas é evidentemente corruptela de \textunderscore sufflosa\textunderscore , que seria a fórma aportuguesada do fr. \textunderscore suffleuse\textunderscore , de \textunderscore souffler\textunderscore .
\section{Sega}
\begin{itemize}
\item {Grp. gram.:f.}
\end{itemize}
Acto ou effeito de segar.
Ceifa.
Duração da ceifa.
Ferro, que se põe adeante da relha do arado, para facilitar a lavra e cortar as raízes.
\section{Segada}
\begin{itemize}
\item {Grp. gram.:f.}
\end{itemize}
O mesmo que \textunderscore sega\textunderscore .
\section{Segadeira}
\begin{itemize}
\item {Grp. gram.:f.}
\end{itemize}
\begin{itemize}
\item {Proveniência:(De \textunderscore segar\textunderscore )}
\end{itemize}
Espécie de foice grande.
\section{Segadoiro}
\begin{itemize}
\item {Grp. gram.:adj.}
\end{itemize}
Que serve para ceifar ou segar: \textunderscore foice segadoira\textunderscore .
Que está em condições de se ceifar: \textunderscore trigo segadoiro\textunderscore .
\section{Segador}
\begin{itemize}
\item {Grp. gram.:m.  e  adj.}
\end{itemize}
O que sega; ceifeiro.
\section{Segadouro}
\begin{itemize}
\item {Grp. gram.:adj.}
\end{itemize}
Que serve para ceifar ou segar: \textunderscore foice segadoura\textunderscore .
Que está em condições de se ceifar: \textunderscore trigo segadouro\textunderscore .
\section{Segadura}
\begin{itemize}
\item {Grp. gram.:f.}
\end{itemize}
O mesmo que \textunderscore sega\textunderscore .
\section{Segão}
\begin{itemize}
\item {Grp. gram.:m.}
\end{itemize}
\begin{itemize}
\item {Proveniência:(De \textunderscore segar\textunderscore )}
\end{itemize}
O mesmo que \textunderscore sega\textunderscore , ferro que se addiciona ao arado.
\section{Segar}
\begin{itemize}
\item {Grp. gram.:v. t.}
\end{itemize}
\begin{itemize}
\item {Utilização:Fig.}
\end{itemize}
\begin{itemize}
\item {Proveniência:(Do lat. \textunderscore secare\textunderscore )}
\end{itemize}
O mesmo que \textunderscore ceifar\textunderscore .
Cortar, pôr fim a: \textunderscore segar os dias da vida\textunderscore .
\section{Sega-vidas}
\begin{itemize}
\item {Grp. gram.:m.  e  adj.}
\end{itemize}
\begin{itemize}
\item {Utilização:Poét.}
\end{itemize}
O que tira muitas vidas; homicida.
\section{Sege}
\begin{itemize}
\item {Grp. gram.:f.}
\end{itemize}
\begin{itemize}
\item {Utilização:Ext.}
\end{itemize}
\begin{itemize}
\item {Proveniência:(Do fr. \textunderscore siége\textunderscore )}
\end{itemize}
Coche desusado, com duas rodas e um só assento, fechado com cortinas na frente.
Carruagem.
\section{Segécia}
\begin{itemize}
\item {Grp. gram.:f.}
\end{itemize}
Gênero de insectos lepidópteros nocturnos.
\section{Segeiro}
\begin{itemize}
\item {Grp. gram.:m.}
\end{itemize}
\begin{itemize}
\item {Utilização:Ext.}
\end{itemize}
\begin{itemize}
\item {Proveniência:(De \textunderscore sege\textunderscore )}
\end{itemize}
Fabricante de seges.
Fabricante de carruagens.
\section{Segéstria}
\begin{itemize}
\item {Grp. gram.:f.}
\end{itemize}
\begin{itemize}
\item {Proveniência:(Lat. \textunderscore segestria\textunderscore )}
\end{itemize}
Gênero de aranhas.
\section{Segetal}
\begin{itemize}
\item {Grp. gram.:adj.}
\end{itemize}
\begin{itemize}
\item {Proveniência:(Lat. \textunderscore segetalis\textunderscore )}
\end{itemize}
Relativo a searas; que cresce entre as searas.
\section{Segitório}
\begin{itemize}
\item {Grp. gram.:m.}
\end{itemize}
\begin{itemize}
\item {Utilização:Ant.}
\end{itemize}
Andor ou charola, em que se levava a imagem de San-Sebastião.
(Cp. \textunderscore sege\textunderscore )
\section{Segmentação}
\begin{itemize}
\item {Grp. gram.:f.}
\end{itemize}
Acto de segmentar.
\section{Segmentar}
\begin{itemize}
\item {Grp. gram.:v. t.}
\end{itemize}
Reduzir a segmentos; tirar segmento a.
\section{Segmentar}
\begin{itemize}
\item {Grp. gram.:adj.}
\end{itemize}
\begin{itemize}
\item {Proveniência:(De \textunderscore segmento\textunderscore )}
\end{itemize}
Formado de segmentos.
\section{Segmentário}
\begin{itemize}
\item {Grp. gram.:adj.}
\end{itemize}
\begin{itemize}
\item {Proveniência:(Lat. \textunderscore segmentarius\textunderscore )}
\end{itemize}
O mesmo que \textunderscore segmentar\textunderscore ^2.
\section{Segmento}
\begin{itemize}
\item {Grp. gram.:m.}
\end{itemize}
\begin{itemize}
\item {Utilização:Geom.}
\end{itemize}
\begin{itemize}
\item {Proveniência:(Lat. \textunderscore segmentum\textunderscore )}
\end{itemize}
Parte de um todo.
Secção.
Porção determinada de um objecto.
Superfície, comprehendida entre a corda de um círculo e o respectivo arco.
\section{Segnícia}
\begin{itemize}
\item {Grp. gram.:f.}
\end{itemize}
\begin{itemize}
\item {Proveniência:(Lat. \textunderscore segnitia\textunderscore )}
\end{itemize}
Indolência.
Preguiça; lentidão.
\section{Segnície}
\begin{itemize}
\item {Grp. gram.:f.}
\end{itemize}
\begin{itemize}
\item {Proveniência:(Lat. \textunderscore segnities\textunderscore )}
\end{itemize}
O mesmo que \textunderscore segnícia\textunderscore .
\section{Segontíacos}
\begin{itemize}
\item {Grp. gram.:m. pl.}
\end{itemize}
\begin{itemize}
\item {Proveniência:(Lat. \textunderscore Segontiaci\textunderscore )}
\end{itemize}
Antigo povo da Gran-Bretanha, que vivia talvez na região chamada hoje país de Galles.
\section{Segóvia}
\begin{itemize}
\item {Grp. gram.:f.}
\end{itemize}
\begin{itemize}
\item {Utilização:Gír.}
\end{itemize}
O mesmo que \textunderscore sarâmbia\textunderscore .
O mesmo que \textunderscore salada\textunderscore .
\section{Segral}
\begin{itemize}
\item {Grp. gram.:adj.}
\end{itemize}
\begin{itemize}
\item {Utilização:Ant.}
\end{itemize}
\begin{itemize}
\item {Proveniência:(De \textunderscore segre\textunderscore )}
\end{itemize}
O mesmo que \textunderscore secular\textunderscore .
\section{Segre}
\begin{itemize}
\item {Grp. gram.:m.}
\end{itemize}
\begin{itemize}
\item {Utilização:Ant.}
\end{itemize}
O mesmo que \textunderscore século\textunderscore . Cf. G. Vicente \textunderscore Carta a D. João III\textunderscore ; \textunderscore Port. Mon. Hist. Script.\textunderscore , 246.
(Cp. \textunderscore sigro\textunderscore )
\section{Segredamento}
\begin{itemize}
\item {Grp. gram.:m.}
\end{itemize}
Acto de segredar.
\section{Segredar}
\begin{itemize}
\item {Grp. gram.:v. t.}
\end{itemize}
\begin{itemize}
\item {Grp. gram.:V. i.}
\end{itemize}
Dizer em segrêdo.
Dizer em voz baixa; cochichar.
Dizer segredos.
\section{Segredeiro}
\begin{itemize}
\item {Grp. gram.:adj.}
\end{itemize}
Que segreda, que diz segredos.
\section{Segredista}
\begin{itemize}
\item {Grp. gram.:m. ,  f.  e  adj.}
\end{itemize}
\begin{itemize}
\item {Proveniência:(De \textunderscore segrêdo\textunderscore )}
\end{itemize}
Pessôa, que guarda segredos ou fala em segrêdo, ou cochicha.
\section{Segrêdo}
\begin{itemize}
\item {Grp. gram.:m.}
\end{itemize}
\begin{itemize}
\item {Grp. gram.:Pl.}
\end{itemize}
\begin{itemize}
\item {Utilização:Prov.}
\end{itemize}
\begin{itemize}
\item {Utilização:alg.}
\end{itemize}
\begin{itemize}
\item {Proveniência:(Lat. \textunderscore secretum\textunderscore )}
\end{itemize}
Aquillo que não está divulgado.
O que se occulta de outrem.
Aquillo que se não deve dizer a ninguém.
Mystério.
O que se diz ao ouvido de alguém.
Lugar occulto; recesso.
Prisão rigorosa, em que o preso não tem communicação com outros.
Esconderijo.
Discrição ou reserva, que se guarda á cêrca do que nos foi communicado particularmente.
Confidência.
Meio ou processo industrial, artístico ou scientífico, apenas conhecido de um indivíduo \textunderscore ou\textunderscore de poucos.
A parte mais diffícil de uma arte, sciência, etc.
Meio particular, para se obter certo resultado.
Mola occulta: \textunderscore fechadura de segrêdo\textunderscore .
Espécie de jôgo popular.
Alforges.
\section{Segregação}
\begin{itemize}
\item {Grp. gram.:f.}
\end{itemize}
\begin{itemize}
\item {Proveniência:(Do lat. \textunderscore segregatio\textunderscore )}
\end{itemize}
Acto ou effeito de segregar.
\section{Segregadamente}
\begin{itemize}
\item {Grp. gram.:adv.}
\end{itemize}
Com segregação.
Insuladamente; em separado; á parte.
\section{Segregar}
\begin{itemize}
\item {Grp. gram.:v. t.}
\end{itemize}
\begin{itemize}
\item {Proveniência:(Lat. \textunderscore segregare\textunderscore )}
\end{itemize}
Pôr de lado; separar.
Desligar.
Expellir: \textunderscore segregar bílis\textunderscore .
\section{Segregatício}
\begin{itemize}
\item {Grp. gram.:adj.}
\end{itemize}
\begin{itemize}
\item {Proveniência:(De \textunderscore segregar\textunderscore )}
\end{itemize}
Relativo á segregação; próprio para segregar.
\section{Segregativo}
\begin{itemize}
\item {Grp. gram.:adj.}
\end{itemize}
\begin{itemize}
\item {Utilização:Gram.}
\end{itemize}
\begin{itemize}
\item {Proveniência:(Lat. \textunderscore segregativus\textunderscore )}
\end{itemize}
Que segrega.
O mesmo que \textunderscore partitivo\textunderscore .
\section{Segrel}
\begin{itemize}
\item {Grp. gram.:m.}
\end{itemize}
\begin{itemize}
\item {Utilização:Ant.}
\end{itemize}
\begin{itemize}
\item {Proveniência:(De \textunderscore segre\textunderscore )}
\end{itemize}
Cavalleiro trovador.
\section{Seguida}
\begin{itemize}
\item {Grp. gram.:f.}
\end{itemize}
\begin{itemize}
\item {Grp. gram.:Loc. adv.}
\end{itemize}
\begin{itemize}
\item {Proveniência:(De \textunderscore seguir\textunderscore )}
\end{itemize}
Seguimento.
\textunderscore Em seguida\textunderscore , seguidamente, logo depois.
\section{Seguidamente}
\begin{itemize}
\item {Grp. gram.:adv.}
\end{itemize}
De modo seguido ou contínuo.
Logo; após; immediatamente.
\section{Seguidilha}
\begin{itemize}
\item {Grp. gram.:f.}
\end{itemize}
Gênero de canções espanholas, alegres e mais ou menos lascivas.
Ária e dança animada, a três tempos.
(Cast. \textunderscore seguidilla\textunderscore )
\section{Seguidilheira}
\begin{itemize}
\item {Grp. gram.:f.}
\end{itemize}
Flexão fem. de \textunderscore seguidilheiro\textunderscore .
\section{Seguidilheiro}
\begin{itemize}
\item {Grp. gram.:m}
\end{itemize}
Aquelle que canta ou dança seguidilhas.
\section{Seguido}
\begin{itemize}
\item {Grp. gram.:adj.}
\end{itemize}
\begin{itemize}
\item {Proveniência:(De \textunderscore seguir\textunderscore )}
\end{itemize}
Immediato.
Contínuo: \textunderscore phrases seguidas\textunderscore .
Que está logo depois.
\section{Seguidor}
\begin{itemize}
\item {Grp. gram.:m.  e  adj.}
\end{itemize}
O que segue; partidário; sectário.
\section{Seguiéria}
\begin{itemize}
\item {Grp. gram.:f.}
\end{itemize}
\begin{itemize}
\item {Proveniência:(De \textunderscore Séguier\textunderscore , n. p.)}
\end{itemize}
Gênero de plantas phytoláceas da América do Nórte.
\section{Seguilhote}
\begin{itemize}
\item {Grp. gram.:m.}
\end{itemize}
\begin{itemize}
\item {Utilização:Bras}
\end{itemize}
Filho de baleia, de mais de seis meses mas ainda mamão.
\section{Seguimento}
\begin{itemize}
\item {Grp. gram.:m.}
\end{itemize}
Acto ou effeito de seguir ou de andar.
Consequência; resultado.
\section{Seguinte}
\begin{itemize}
\item {Grp. gram.:adj.}
\end{itemize}
\begin{itemize}
\item {Grp. gram.:M.}
\end{itemize}
\begin{itemize}
\item {Grp. gram.:Pl.}
\end{itemize}
Que segue, ou que se segue.
Seguido, immediato; continuado.
Aquelle ou aquillo que segue outrem ou outra coisa.
Ângulos de alvenaria.
Peças lateraes das gelosias.
\section{Seguintemente}
\begin{itemize}
\item {Grp. gram.:adv.}
\end{itemize}
\begin{itemize}
\item {Proveniência:(De \textunderscore seguinte\textunderscore )}
\end{itemize}
Seguidamente; por conseguinte.
\section{Seguir}
\begin{itemize}
\item {Grp. gram.:v. t.}
\end{itemize}
\begin{itemize}
\item {Grp. gram.:V. i.}
\end{itemize}
\begin{itemize}
\item {Grp. gram.:V. p.}
\end{itemize}
\begin{itemize}
\item {Proveniência:(Do lat. hyp. \textunderscore seguere\textunderscore )}
\end{itemize}
Ir atrás de.
Acompanhar.
Sêr consequência de.
Acontecer depois de.
Estar próximo a.
Acompanhar com a attenção ou com a observação.
Acompanhar em espírito.
Percorrer, andar por (estrada, caminho, etc.).
Sêr sectário de: \textunderscore seguir o Protestantismo\textunderscore .
Tomar o partido de.
Proseguir.
Destinar-se á profissão de.
Continuar, proseguir.
Tomar certa direcção.
Acontecer depois; sobrevir.
Succeder; vir depois; resultar: \textunderscore donde se segue que mal andaste\textunderscore .
\section{Segunda}
\begin{itemize}
\item {Grp. gram.:f.}
\end{itemize}
\begin{itemize}
\item {Utilização:Ant.}
\end{itemize}
Prova typográphica de uma fôlha, que já se corrigiu.
Intervallo musical, de um tom a outro, immediato.
Os cereaes de segunda ordem, como o painço ou o milho miúdo.
Cp. \textunderscore secunda\textunderscore .
(Fem. de \textunderscore segundo\textunderscore ^1)
\section{Segunda-feira}
\begin{itemize}
\item {Grp. gram.:f.}
\end{itemize}
Segundo dia da semana.
\section{Segundamente}
\begin{itemize}
\item {Grp. gram.:adv.}
\end{itemize}
Em segundo lugar.
\section{Segundanista}
\begin{itemize}
\item {Grp. gram.:m.}
\end{itemize}
Estudante, que frequenta o segundo anno de qualquer faculdade universitária, ou de outra escola superior.
\section{Segundannista}
\begin{itemize}
\item {Grp. gram.:m.}
\end{itemize}
Estudante, que frequenta o segundo anno de qualquer faculdade universitária, ou de outra escola superior.
\section{Segundar}
\begin{itemize}
\item {Grp. gram.:v. t.}
\end{itemize}
\begin{itemize}
\item {Proveniência:(De \textunderscore segundo\textunderscore ^1)}
\end{itemize}
O mesmo que \textunderscore secundar\textunderscore .
\section{Segundariamente}
\begin{itemize}
\item {Grp. gram.:adv.}
\end{itemize}
O mesmo que \textunderscore secundariamente\textunderscore . Cf. \textunderscore Eufrosina\textunderscore , 98.
\section{Segundeira}
\begin{itemize}
\item {Grp. gram.:f.}
\end{itemize}
Segunda porção de vinho, que se dava aos frades em dias festivos.
A segunda camada de cortiça, nos sobreiros.
(Fem. de \textunderscore segundeiro\textunderscore )
\section{Segundeiro}
\begin{itemize}
\item {Grp. gram.:adj.}
\end{itemize}
\begin{itemize}
\item {Proveniência:(De \textunderscore segundo\textunderscore ^1)}
\end{itemize}
O mesmo que \textunderscore secundário\textunderscore .
Diz-se do moínho para milho miúdo e painço.
\section{Segundo}
\begin{itemize}
\item {Grp. gram.:adj.}
\end{itemize}
\begin{itemize}
\item {Utilização:Fig.}
\end{itemize}
\begin{itemize}
\item {Grp. gram.:M.}
\end{itemize}
\begin{itemize}
\item {Proveniência:(Do lat. \textunderscore secundus\textunderscore )}
\end{itemize}
Que está logo depois do primeiro: \textunderscore o segundo filho\textunderscore .
Que, numa série de dois, occupa o último lugar.
Secundário.
Indirecto.
Outro.
Semelhante, rival: \textunderscore formosa sem segunda\textunderscore .
Pessôa ou coisa, que está em segundo lugar.
Sexagésima parte de um minuto: \textunderscore o minuto tem 60 segundos\textunderscore .
\section{Segundo}
\begin{itemize}
\item {Grp. gram.:prep.}
\end{itemize}
\begin{itemize}
\item {Proveniência:(Lat. \textunderscore secundum\textunderscore )}
\end{itemize}
Em harmonia com; conforme: \textunderscore proceder segundo a lei\textunderscore .
Semelhantemente a.
Ao passo que.
\section{Segundo}
\begin{itemize}
\item {Grp. gram.:adv.}
\end{itemize}
\begin{itemize}
\item {Proveniência:(Lat. \textunderscore secundo\textunderscore )}
\end{itemize}
Em segundo lugar.
\section{Segundo-gênito}
\begin{itemize}
\item {Grp. gram.:m.  e  adj.}
\end{itemize}
Diz-se do filho segundo.
\section{Segundo-genitura}
\begin{itemize}
\item {Grp. gram.:f.}
\end{itemize}
Estado de quem é segundo-gênito.
\section{Segur}
\begin{itemize}
\item {Grp. gram.:f.}
\end{itemize}
\begin{itemize}
\item {Proveniência:(Lat. \textunderscore securis\textunderscore )}
\end{itemize}
Machadinha.
\section{Segura}
\begin{itemize}
\item {Grp. gram.:f.}
\end{itemize}
\begin{itemize}
\item {Proveniência:(Do lat. \textunderscore securis\textunderscore )}
\end{itemize}
Espécie de enxó de tanoeiro.
\section{Seguração}
\begin{itemize}
\item {Grp. gram.:f.}
\end{itemize}
\begin{itemize}
\item {Proveniência:(De \textunderscore segurar\textunderscore )}
\end{itemize}
O mesmo que \textunderscore segurança\textunderscore ; seguro.
\section{Segurado}
\begin{itemize}
\item {Grp. gram.:m.}
\end{itemize}
\begin{itemize}
\item {Proveniência:(De \textunderscore segurar\textunderscore )}
\end{itemize}
Aquelle que paga o prêmio num contrato de seguro.
\section{Segurador}
\begin{itemize}
\item {Grp. gram.:m.  e  adj.}
\end{itemize}
O que segura.
O que se obriga, num contrato de seguro, a indemnizar prejuizos eventuaes.
\section{Seguramente}
\begin{itemize}
\item {Grp. gram.:adv.}
\end{itemize}
De modo seguro; com segurança; com certeza; evidentemente.
\section{Segurança}
\begin{itemize}
\item {Grp. gram.:f.}
\end{itemize}
\begin{itemize}
\item {Proveniência:(Do b. lat. \textunderscore securancia\textunderscore )}
\end{itemize}
Acto ou effeito de segurar.
Condição daquelle ou daquillo que está seguro.
Confiança.
Affirmação.
Firmeza.
Caução.
Pessôa ou coisa, que serve de esteio ou amparo a outrem.
Afoiteza.
Prenhez das fêmeas dos quadrúpedes.
\section{Segurar}
\begin{itemize}
\item {Grp. gram.:v. t.}
\end{itemize}
\begin{itemize}
\item {Proveniência:(Do b. lat. \textunderscore segurare\textunderscore )}
\end{itemize}
Tornar seguro, firme, fixo.
Escorar.
Amparar.
Agarrar: \textunderscore segurou-lhe o braço\textunderscore .
Caucionar.
Livrar de perigo.
Certificar, assegurar.
Tranquillizar.
Pôr no seguro, fazer contracto de seguro á cêrca de: \textunderscore segurar a mobília\textunderscore .
\section{Segure}
\begin{itemize}
\item {Grp. gram.:f.}
\end{itemize}
\begin{itemize}
\item {Proveniência:(Lat. \textunderscore securis\textunderscore )}
\end{itemize}
O mesmo ou melhor que \textunderscore segur\textunderscore .
O mesmo que \textunderscore segura\textunderscore .
Machado grande.
\section{Segurelha}
\begin{itemize}
\item {fónica:gurê}
\end{itemize}
\begin{itemize}
\item {Grp. gram.:f.}
\end{itemize}
\begin{itemize}
\item {Proveniência:(Do lat. \textunderscore securicula\textunderscore )}
\end{itemize}
Peça de ferro, em que entra o ferro que segura a mó inferior das atafonas.
Peça de madeira, enfiada no espigão da mó inferior, para tornar uniforme o movimento da peça superior.
\section{Segurelha}
\begin{itemize}
\item {fónica:gurê}
\end{itemize}
\begin{itemize}
\item {Grp. gram.:f.}
\end{itemize}
Nome de várias plantas labiadas.
Mangericão de Ceilão.
\section{Segurelhal}
\begin{itemize}
\item {Grp. gram.:m.}
\end{itemize}
Cavidade, em que entra a \textunderscore segurelha\textunderscore ^1.
\section{Segureza}
\begin{itemize}
\item {Grp. gram.:f.}
\end{itemize}
O mesmo que \textunderscore segurança\textunderscore . Cf. Filinto, IX, 283; XVI, 39.
\section{Seguridade}
\begin{itemize}
\item {Grp. gram.:f.}
\end{itemize}
\begin{itemize}
\item {Proveniência:(Do lat. \textunderscore securitas\textunderscore )}
\end{itemize}
O mesmo que \textunderscore segurança\textunderscore .
\section{Seguro}
\begin{itemize}
\item {Grp. gram.:adj.}
\end{itemize}
\begin{itemize}
\item {Utilização:Fam.}
\end{itemize}
\begin{itemize}
\item {Grp. gram.:M.}
\end{itemize}
\begin{itemize}
\item {Proveniência:(Do lat. \textunderscore securus\textunderscore )}
\end{itemize}
Que não tem receios.
Livre de perigo.
Afoito.
Acautelado; garantido.
Inhabalável.
Firme.
Encarcerado.
Infallível.
Efficaz: \textunderscore remédio seguro\textunderscore .
Constante.
Avarento.
Certeza.
Garantia.
Amparo.
Salva-guarda.
Salvo-conducto.
Contrato aleatório, em que uma das partes se obriga a indemnizar outra, de um perigo ou prejuízo eventual.
\section{Seiada}
\begin{itemize}
\item {Grp. gram.:f.}
\end{itemize}
\begin{itemize}
\item {Utilização:Bras}
\end{itemize}
\begin{itemize}
\item {Proveniência:(De \textunderscore seio\textunderscore )}
\end{itemize}
Série de recôncavos na montanha. Cf. Pacheco, \textunderscore Promptuário\textunderscore .
\section{Seide}
\begin{itemize}
\item {Grp. gram.:m.}
\end{itemize}
\begin{itemize}
\item {Proveniência:(T. ár.)}
\end{itemize}
Título honorífico dos descendentes de Maoma.
\section{Seidiço}
\begin{itemize}
\item {Grp. gram.:adj.}
\end{itemize}
\begin{itemize}
\item {Utilização:Ant.}
\end{itemize}
O mesmo que \textunderscore sediço\textunderscore . Cf. B. Pereira, \textunderscore Prosodia\textunderscore , vb. \textunderscore requietus\textunderscore .
\section{Seifia}
\begin{itemize}
\item {Grp. gram.:f.}
\end{itemize}
Peixe labroide, (\textunderscore scarus denticulatus\textunderscore ).
\section{Seima}
\begin{itemize}
\item {Grp. gram.:f.}
\end{itemize}
Peixe esparoide, (\textunderscore pargus auriga\textunderscore ).
\section{Seimiri}
\begin{itemize}
\item {Grp. gram.:m.}
\end{itemize}
Espécie de macaco americano.
\section{Seio}
\begin{itemize}
\item {Grp. gram.:m.}
\end{itemize}
\begin{itemize}
\item {Utilização:Náut.}
\end{itemize}
\begin{itemize}
\item {Grp. gram.:Pl.}
\end{itemize}
\begin{itemize}
\item {Proveniência:(Do lat. \textunderscore sinus\textunderscore )}
\end{itemize}
Curvatura.
Sinuosidade.
Bojo da vela, quando enfunada.
Parte do corpo humano, em que há as glândulas mamaes.
Pomas.
Recesso.
Parte íntima.
Útero.
Centro.
Aquillo que occulta.
Parte média de um cabo náutico.
Coração, alma, âmago.
Cúmulo, auge.
Grêmio: \textunderscore no seio do Parlamento\textunderscore .
Ambiente.
Intimidade.
Enseada, golfo.
Glândulas mamaes.
\section{Seira}
\begin{itemize}
\item {Grp. gram.:f.}
\end{itemize}
Cesto, cabaz ou saco, tecido de junco, esparto ou vimes.
(Or. germ.)
\section{Seirão}
\begin{itemize}
\item {Grp. gram.:m.}
\end{itemize}
Seira grande.
\section{Seires}
\begin{itemize}
\item {Grp. gram.:m.}
\end{itemize}
Pequena árvore madeirense, (\textunderscore salix canariensis\textunderscore , Smith), cuja madeira se emprega em embutidos. Cf. \textunderscore Bol. da Socied. de Geogr.\textunderscore , XXX, 620.
\section{Seis}
\begin{itemize}
\item {Grp. gram.:adj.}
\end{itemize}
\begin{itemize}
\item {Grp. gram.:M.}
\end{itemize}
\begin{itemize}
\item {Proveniência:(Do lat. \textunderscore sex\textunderscore )}
\end{itemize}
Diz-se do número cardinal, formado de cinco e mais um.
Sexto.
O algarismo representativo do número seis.
Carta de jogar ou peça de dominó, que tem seis pontos.
Aquelle ou aquillo que numa série de seis occupa o último lugar.
\section{Seisavo}
\begin{itemize}
\item {Grp. gram.:m.}
\end{itemize}
\begin{itemize}
\item {Proveniência:(De \textunderscore seis\textunderscore  + \textunderscore avo\textunderscore )}
\end{itemize}
A sexta parte de um número.
\section{Seiscentismo}
\begin{itemize}
\item {Grp. gram.:m.}
\end{itemize}
Estilo ou escola literária dos seiscentistas.
\section{Seiscentista}
\begin{itemize}
\item {Grp. gram.:m.}
\end{itemize}
\begin{itemize}
\item {Grp. gram.:Adj.}
\end{itemize}
\begin{itemize}
\item {Proveniência:(De \textunderscore seiscentos\textunderscore )}
\end{itemize}
Escritor do século, que principiou em 1600.
Diz-se do estilo ou da escola literária do mesmo século.
\section{Seiscentos}
\begin{itemize}
\item {Grp. gram.:adj.}
\end{itemize}
\begin{itemize}
\item {Proveniência:(De \textunderscore seis\textunderscore  + \textunderscore cento\textunderscore )}
\end{itemize}
Seis vezes cem.
\section{Seisdobro}
\begin{itemize}
\item {fónica:dô}
\end{itemize}
\begin{itemize}
\item {Grp. gram.:m.  e  adj.}
\end{itemize}
O mesmo que \textunderscore sêxtuplo\textunderscore .
(Do \textunderscore seis\textunderscore  + \textunderscore dôbro\textunderscore )
\section{Seis-filetes}
\begin{itemize}
\item {Grp. gram.:m.}
\end{itemize}
Gênero de aves, (\textunderscore paradisoea aurea\textunderscore , Lin.), que têm a cabeça guarnecida de seis filetes, três a cada lado.
\section{Seisto}
\begin{itemize}
\item {Grp. gram.:adj.}
\end{itemize}
\begin{itemize}
\item {Utilização:Ant.}
\end{itemize}
O mesmo que \textunderscore sexto\textunderscore :«\textunderscore canto seisto\textunderscore ». \textunderscore Lusíadas\textunderscore .
\section{Seita}
\begin{itemize}
\item {Grp. gram.:f.}
\end{itemize}
\begin{itemize}
\item {Proveniência:(Do lat. \textunderscore secta\textunderscore )}
\end{itemize}
Conjunto de indivíduos, que professam doutrina que se afasta da geralmente seguida.
Conjunto de indivíduos, que seguem a mesma doutrina ou systema.
Systema religioso, philosóphico, político ou literário.
Theoria, proclamada por homem illustre, e seguida por muita gente.
Partido.
\section{Seita}
\begin{itemize}
\item {Grp. gram.:f.}
\end{itemize}
\begin{itemize}
\item {Utilização:Prov.}
\end{itemize}
\begin{itemize}
\item {Utilização:minh.}
\end{itemize}
\begin{itemize}
\item {Utilização:Prov.}
\end{itemize}
\begin{itemize}
\item {Proveniência:(Do lat. \textunderscore secta\textunderscore , de \textunderscore secare\textunderscore )}
\end{itemize}
Céspede ou leira, que o ferro do vessadoiro levanta e deita aos lados.
O mesmo que \textunderscore sega\textunderscore , ferro.
\section{Seitador}
\begin{itemize}
\item {Grp. gram.:m.}
\end{itemize}
\begin{itemize}
\item {Utilização:Prov.}
\end{itemize}
\begin{itemize}
\item {Utilização:minh.}
\end{itemize}
O mesmo que \textunderscore seitante\textunderscore .
\section{Seitante}
\begin{itemize}
\item {Grp. gram.:m.}
\end{itemize}
\begin{itemize}
\item {Utilização:Prov.}
\end{itemize}
\begin{itemize}
\item {Utilização:minh.}
\end{itemize}
\begin{itemize}
\item {Proveniência:(De \textunderscore seita\textunderscore ^2)}
\end{itemize}
Aquelle que guia a junta de gado que puxa o seitoril.
\section{Seiteira}
\begin{itemize}
\item {Grp. gram.:f.}
\end{itemize}
\begin{itemize}
\item {Utilização:Prov.}
\end{itemize}
\begin{itemize}
\item {Utilização:trasm.}
\end{itemize}
O mesmo que \textunderscore seitoira\textunderscore .
\section{Seitoira}
\begin{itemize}
\item {Grp. gram.:f.}
\end{itemize}
\begin{itemize}
\item {Utilização:Prov.}
\end{itemize}
\begin{itemize}
\item {Utilização:trasm.}
\end{itemize}
\begin{itemize}
\item {Utilização:beir.}
\end{itemize}
\begin{itemize}
\item {Utilização:Prov.}
\end{itemize}
\begin{itemize}
\item {Utilização:minh.}
\end{itemize}
Foice, para ceifar pão.
Espécie de temão, que tem numa extremidade a sega de ferro, e que pela outra se prende ao jugo; o mesmo que \textunderscore seitoril\textunderscore .
(Cp. \textunderscore sètoira\textunderscore )
\section{Seitoril}
\begin{itemize}
\item {Grp. gram.:m.}
\end{itemize}
\begin{itemize}
\item {Utilização:Prov.}
\end{itemize}
\begin{itemize}
\item {Utilização:minh.}
\end{itemize}
\begin{itemize}
\item {Proveniência:(De \textunderscore seita\textunderscore ^2)}
\end{itemize}
Temão, que tem cravada a sega, com que se corta a leiva deixada pelo vessadoiro.
\section{Seitoso}
\begin{itemize}
\item {Grp. gram.:adj.}
\end{itemize}
\begin{itemize}
\item {Utilização:Des.}
\end{itemize}
\begin{itemize}
\item {Proveniência:(De \textunderscore seita\textunderscore ^1)}
\end{itemize}
Traiçoeiro; faccioso.
\section{Seiva}
\begin{itemize}
\item {Grp. gram.:f.}
\end{itemize}
\begin{itemize}
\item {Utilização:Ext.}
\end{itemize}
\begin{itemize}
\item {Utilização:Ant.}
\end{itemize}
\begin{itemize}
\item {Proveniência:(Do lat. hyp. \textunderscore sapia\textunderscore )}
\end{itemize}
Líquido, que as raízes absorvem da terra e de que se nutre o respectivo vegetal.
Elementos vitaes; sangue.
Alento, vigor.
O mesmo que \textunderscore saliva\textunderscore .
\section{Seivar}
\begin{itemize}
\item {Grp. gram.:v. t.}
\end{itemize}
O mesmo ou melhor que \textunderscore ceivar\textunderscore . Cf. B. Pereira, vb. \textunderscore dejugo\textunderscore .
\section{Seive}
\begin{itemize}
\item {Grp. gram.:m.}
\end{itemize}
\begin{itemize}
\item {Utilização:Ant.}
\end{itemize}
\begin{itemize}
\item {Proveniência:(De \textunderscore seivar\textunderscore )}
\end{itemize}
Campo aberto, sem vallo nem tapume.
\section{Seivo}
\begin{itemize}
\item {Grp. gram.:m.}
\end{itemize}
\begin{itemize}
\item {Utilização:Ant.}
\end{itemize}
\begin{itemize}
\item {Proveniência:(De \textunderscore seivar\textunderscore )}
\end{itemize}
Campo aberto, sem vallo nem tapume.
\section{Seivoeira}
\begin{itemize}
\item {Grp. gram.:f.}
\end{itemize}
\begin{itemize}
\item {Utilização:Prov.}
\end{itemize}
\begin{itemize}
\item {Utilização:dur.}
\end{itemize}
O mesmo que \textunderscore maçarico\textunderscore .
\section{Seivoso}
\begin{itemize}
\item {Grp. gram.:adj.}
\end{itemize}
Que tem seiva.
Que facilita a circulação da seiva.
\section{Seixa}
\begin{itemize}
\item {Grp. gram.:f.}
\end{itemize}
\begin{itemize}
\item {Utilização:Prov.}
\end{itemize}
\begin{itemize}
\item {Utilização:alent.}
\end{itemize}
Representação de um ádem, nos brasões dos Seixas.
Espécie de pombo bravo, o mesmo que \textunderscore sousa\textunderscore .
Variedade de caranguejo, de casco amarelo e azulado.
Pequeno antílope africano.
\section{Seixada}
\begin{itemize}
\item {Grp. gram.:f.}
\end{itemize}
Pancada com seixo.
\section{Seixal}
\begin{itemize}
\item {Grp. gram.:m.}
\end{itemize}
Lugar, onde há muitos seixos.
\section{Seixas}
\begin{itemize}
\item {Grp. gram.:f. pl.}
\end{itemize}
Parte das capas dos livros, que sobresái ás fôlhas.
\section{Seixebra}
\begin{itemize}
\item {Grp. gram.:f.}
\end{itemize}
\begin{itemize}
\item {Utilização:Prov.}
\end{itemize}
\begin{itemize}
\item {Utilização:minh.}
\end{itemize}
Planta medicinal, (\textunderscore teucrium scorodonia\textunderscore , Lin.).
\section{Seixeira}
\begin{itemize}
\item {Grp. gram.:f.}
\end{itemize}
\begin{itemize}
\item {Utilização:Prov.}
\end{itemize}
\begin{itemize}
\item {Utilização:minh.}
\end{itemize}
Lugar, onde há muitos seixos; seixal.
\section{Seixo}
\begin{itemize}
\item {Grp. gram.:m.}
\end{itemize}
\begin{itemize}
\item {Proveniência:(Do lat. \textunderscore saxum\textunderscore )}
\end{itemize}
Pedra dura; calhau; rebo.
\section{Seixoeira}
\begin{itemize}
\item {Grp. gram.:f.}
\end{itemize}
\begin{itemize}
\item {Proveniência:(De \textunderscore seixa\textunderscore , pombo?)}
\end{itemize}
Espécie de rôla, de papo vermelho, (\textunderscore tringa canutus\textunderscore , Lin.).
\section{Seixoso}
\begin{itemize}
\item {Grp. gram.:adj.}
\end{itemize}
\begin{itemize}
\item {Proveniência:(Do lat. \textunderscore saxosus\textunderscore )}
\end{itemize}
Abundante de seixos.
\section{Séjana}
\begin{itemize}
\item {Grp. gram.:f.}
\end{itemize}
\begin{itemize}
\item {Utilização:Ant.}
\end{itemize}
\begin{itemize}
\item {Proveniência:(Do ár. \textunderscore sijn\textunderscore )}
\end{itemize}
O mesmo que \textunderscore cadeia\textunderscore  ou \textunderscore prisão\textunderscore .
Cadeia de Christãos, entre os Moiros. Cf. Camillo, \textunderscore Caveira\textunderscore , 195.
\section{Sela}
\begin{itemize}
\item {Grp. gram.:f.}
\end{itemize}
\begin{itemize}
\item {Utilização:Des.}
\end{itemize}
\begin{itemize}
\item {Utilização:Anat.}
\end{itemize}
\begin{itemize}
\item {Proveniência:(Lat. \textunderscore sella\textunderscore )}
\end{itemize}
Aparelho, que se coloca sôbre a cavalgadura, e que é próprio para se sentar o cavaleiro.
Assento.
Poltrona.
\textunderscore Sela túrcica\textunderscore , cavidade, onde se forma a glândula pituritária.
\textunderscore Sela polaca\textunderscore , espécie de ostra do mar das Indias.
\section{Seláceo}
\begin{itemize}
\item {Grp. gram.:adj.}
\end{itemize}
\begin{itemize}
\item {Grp. gram.:M. pl.}
\end{itemize}
\begin{itemize}
\item {Utilização:Zool.}
\end{itemize}
\begin{itemize}
\item {Proveniência:(Do gr. \textunderscore selakhos\textunderscore )}
\end{itemize}
Cartilaginoso.
Ordem de peixes cartilaginosos, que comprehende as raias e os esqualos.
\section{Selada}
\begin{itemize}
\item {Grp. gram.:f.}
\end{itemize}
\begin{itemize}
\item {Proveniência:(De \textunderscore sela\textunderscore )}
\end{itemize}
Depressão na lombada de um monte.
Cavidade oblonga numa montanha.
\section{Seladerma}
\begin{itemize}
\item {Grp. gram.:f.}
\end{itemize}
\begin{itemize}
\item {Proveniência:(Do gr. \textunderscore sela\textunderscore  + \textunderscore derma\textunderscore )}
\end{itemize}
Gênero de insectos hymenópteros.
\section{Selado}
\begin{itemize}
\item {Grp. gram.:m.}
\end{itemize}
\begin{itemize}
\item {Grp. gram.:Adj.}
\end{itemize}
\begin{itemize}
\item {Utilização:Prov.}
\end{itemize}
\begin{itemize}
\item {Utilização:Bras. do N}
\end{itemize}
\begin{itemize}
\item {Utilização:minh.}
\end{itemize}
\begin{itemize}
\item {Proveniência:(De \textunderscore selar\textunderscore ^1)}
\end{itemize}
Curvatura das ilhargas.
Curvatura da parte lateral do pé.
Diz-se da pessôa ou animal, que tem o dorso curvado.
\section{Selado}
\begin{itemize}
\item {Grp. gram.:adj.}
\end{itemize}
\begin{itemize}
\item {Proveniência:(De \textunderscore selar\textunderscore ^2)}
\end{itemize}
Que tem sêlo: \textunderscore papel selado\textunderscore .
\section{Selador}
\begin{itemize}
\item {Grp. gram.:m.  e  adj.}
\end{itemize}
\begin{itemize}
\item {Proveniência:(De \textunderscore selar\textunderscore ^2)}
\end{itemize}
O que sela.
\section{Seladouro}
\begin{itemize}
\item {Grp. gram.:m.}
\end{itemize}
\begin{itemize}
\item {Utilização:Fig.}
\end{itemize}
\begin{itemize}
\item {Proveniência:(De \textunderscore selar\textunderscore ^1)}
\end{itemize}
Parte do corpo do animal, em que se coloca a sela.
Talhe do fato, correspondente ás ilhargas.
\section{Seladura}
\begin{itemize}
\item {Grp. gram.:f.}
\end{itemize}
Acto ou efeito de selar^1.
Seladoiro.
\section{Selagão}
\begin{itemize}
\item {Grp. gram.:m.}
\end{itemize}
\begin{itemize}
\item {Proveniência:(De \textunderscore selar\textunderscore )}
\end{itemize}
Sela de pequeno arção anterior, e sem arção posterior.
\section{Selagem}
\begin{itemize}
\item {Grp. gram.:f.}
\end{itemize}
Acto ou efeito de selar^2.
\section{Selagina}
\begin{itemize}
\item {Grp. gram.:f.}
\end{itemize}
\begin{itemize}
\item {Proveniência:(Do lat. \textunderscore selago\textunderscore )}
\end{itemize}
Gênero de plantas herbáceas, do Cabo da Bôa-Esperança.
O mesmo que \textunderscore selago\textunderscore .
\section{Selagíneas}
\begin{itemize}
\item {Grp. gram.:f. pl.}
\end{itemize}
Família de plantas dicotyledóneas, que tem por typo a selagina.
(Fem. pl. de \textunderscore selagíneo\textunderscore )
\section{Selagíneo}
\begin{itemize}
\item {Grp. gram.:adj.}
\end{itemize}
Relativo ou semelhante á selagina.
\section{Selaginito}
\begin{itemize}
\item {Grp. gram.:m.}
\end{itemize}
\begin{itemize}
\item {Proveniência:(De \textunderscore selagina\textunderscore )}
\end{itemize}
Gênero de plantas fósseis, que parecem pertencer ás lycopodiáceas.
\section{Selago}
\begin{itemize}
\item {Grp. gram.:m.}
\end{itemize}
\begin{itemize}
\item {Proveniência:(Lat. \textunderscore selago\textunderscore )}
\end{itemize}
Planta lycopodiácea, que os Druidas colhiam com práticas religiosas.--Melhór sería \textunderscore selagem\textunderscore .
\section{Selamim}
\begin{itemize}
\item {Grp. gram.:m.}
\end{itemize}
(V.celamim)
\section{Selamorda}
\begin{itemize}
\item {fónica:môr}
\end{itemize}
\begin{itemize}
\item {Grp. gram.:m.  e  f.}
\end{itemize}
(V.salamorda)
\section{Selândra}
\begin{itemize}
\item {Grp. gram.:f.}
\end{itemize}
Espécie de embarcação, usada na Idade-Média. Cf. Castilho, \textunderscore Fastos\textunderscore , II, 482.
\section{Selândria}
\begin{itemize}
\item {Grp. gram.:f.}
\end{itemize}
\begin{itemize}
\item {Proveniência:(Do gr. \textunderscore selas\textunderscore  + \textunderscore aner\textunderscore , \textunderscore andros\textunderscore )}
\end{itemize}
Gênero de insectos hymenópteros.
\section{Selar}
\begin{itemize}
\item {Grp. gram.:v. t.}
\end{itemize}
Pôr sela em: \textunderscore selar um cavalo\textunderscore .
\section{Selar}
\begin{itemize}
\item {Grp. gram.:v. t.}
\end{itemize}
\begin{itemize}
\item {Grp. gram.:V. p.}
\end{itemize}
\begin{itemize}
\item {Utilização:Fig.}
\end{itemize}
\begin{itemize}
\item {Proveniência:(Do lat. \textunderscore sigillare\textunderscore )}
\end{itemize}
Pôr sêlo em.
Carimbar.
Estampilhar: \textunderscore selar uma carta\textunderscore .
Pôr marca em.
Pôr fim a.
Fechar.
Tornar válido.
Sujeitar-se.
Manchar-se.
\section{Selaria}
\begin{itemize}
\item {Grp. gram.:f.}
\end{itemize}
\begin{itemize}
\item {Proveniência:(De \textunderscore sela\textunderscore )}
\end{itemize}
Arte de seleiro.
Estabelecimento ou arruamento de seleiros.
Porção de selas e outros arreios.
\section{Selário}
\begin{itemize}
\item {Grp. gram.:m.}
\end{itemize}
Antigo imposto, que se pagava em Lisbôa, antes de D. João I.
\section{Selásia}
\begin{itemize}
\item {Grp. gram.:f.}
\end{itemize}
\begin{itemize}
\item {Proveniência:(Do gr. \textunderscore selas\textunderscore )}
\end{itemize}
Gênero de insectos coleópteros.
\section{Selecção}
\begin{itemize}
\item {Grp. gram.:f.}
\end{itemize}
\begin{itemize}
\item {Proveniência:(Lat. \textunderscore selectio\textunderscore )}
\end{itemize}
Acto ou effeito de escolher.
Escolha fundamentada.
Qualidade natural de uma espécie ou de uma variedade, pela qual certos typos zoológicos ou vegetaes tendem a reproduzir-se ou a modificar-se, desapparecendo as espécies ou variedades que não podem lutar.
\section{Selecta}
\begin{itemize}
\item {Grp. gram.:f.}
\end{itemize}
\begin{itemize}
\item {Proveniência:(De \textunderscore selecto\textunderscore )}
\end{itemize}
Collecção de trechos literários, extrahidos de várias obras.
Variedade de pêra sucosa e aromática.
\section{Selectar}
\begin{itemize}
\item {Grp. gram.:v. t.}
\end{itemize}
\begin{itemize}
\item {Utilização:Neol.}
\end{itemize}
\begin{itemize}
\item {Proveniência:(De \textunderscore selecto\textunderscore )}
\end{itemize}
Fazer selecção de.
Escolher entre muitas ou várias coisas.
\section{Selectivo}
\begin{itemize}
\item {Grp. gram.:adj.}
\end{itemize}
\begin{itemize}
\item {Proveniência:(De \textunderscore selecto\textunderscore )}
\end{itemize}
Relativo a selecção.
\section{Selecto}
\begin{itemize}
\item {Grp. gram.:adj.}
\end{itemize}
\begin{itemize}
\item {Utilização:Ext.}
\end{itemize}
\begin{itemize}
\item {Proveniência:(Lat. \textunderscore selectus\textunderscore )}
\end{itemize}
Escolhido.
Especial; excellente.
\section{Selector}
\begin{itemize}
\item {Grp. gram.:m.}
\end{itemize}
\begin{itemize}
\item {Proveniência:(Lat. \textunderscore selector\textunderscore )}
\end{itemize}
Apparelho telegráphico, inventado pela Padre Cerobotani, para escolher a natureza da corrente eléctrica, que convém recolher, e para escolher a intensidade dessa corrente.
\section{Seleiro}
\begin{itemize}
\item {Grp. gram.:m.}
\end{itemize}
\begin{itemize}
\item {Grp. gram.:Adj.}
\end{itemize}
Fabricante ou vendedor de selas.
Que é bom cavaleiro ou que se sustenta bem na sela.
Diz-se do cavalo, que já experimentou a sela.
\section{Selim}
\begin{itemize}
\item {Grp. gram.:m.}
\end{itemize}
\begin{itemize}
\item {Proveniência:(De \textunderscore sela\textunderscore )}
\end{itemize}
Pequena sela, sem arção.
Molusco bivalve, pertencente, segundo Cuvier, á fam. das ostras.
\section{Selenato}
\begin{itemize}
\item {Grp. gram.:m.}
\end{itemize}
\begin{itemize}
\item {Utilização:Miner.}
\end{itemize}
Espécie de mineraes sulfutinos.
O mesmo que \textunderscore seleniato\textunderscore .
\section{Selenhydrato}
\begin{itemize}
\item {Grp. gram.:m.}
\end{itemize}
Sal, formado pelo hydrogênio seleniado.
\section{Selênia}
\begin{itemize}
\item {Grp. gram.:f.}
\end{itemize}
\begin{itemize}
\item {Proveniência:(Do gr. \textunderscore selene\textunderscore )}
\end{itemize}
Gênero de plantas crucíferas da América do Norte.
\section{Seleniado}
\begin{itemize}
\item {Grp. gram.:adj.}
\end{itemize}
Que tem selênio.
\section{Seleniato}
\begin{itemize}
\item {Grp. gram.:m.}
\end{itemize}
\begin{itemize}
\item {Proveniência:(De \textunderscore selênio\textunderscore )}
\end{itemize}
Sal, resultante da combinação do ácido selênico com uma base.
\section{Selenibase}
\begin{itemize}
\item {Grp. gram.:f.}
\end{itemize}
\begin{itemize}
\item {Utilização:Chím.}
\end{itemize}
Combinação do selênio, figurando êste de base.
\section{Selênico}
\begin{itemize}
\item {Grp. gram.:adj.}
\end{itemize}
Relativo á Lua.
Relativo ao selênio.
\section{Selênidos}
\begin{itemize}
\item {Grp. gram.:m. pl.}
\end{itemize}
Família de mineraes, que tem por base o selênio.
\section{Selenidrato}
\begin{itemize}
\item {Grp. gram.:m.}
\end{itemize}
Sal, formado pelo hidrogênio seleniado.
\section{Selenífero}
\begin{itemize}
\item {Grp. gram.:adj.}
\end{itemize}
\begin{itemize}
\item {Proveniência:(Do gr. \textunderscore selene\textunderscore  + lat. \textunderscore ferre\textunderscore )}
\end{itemize}
O mesmo que \textunderscore seleniado\textunderscore .
\section{Selênio}
\begin{itemize}
\item {Grp. gram.:m.}
\end{itemize}
\begin{itemize}
\item {Proveniência:(Do gr. \textunderscore selene\textunderscore )}
\end{itemize}
Metallóide sólido e friável, descoberto por Berzélio em 1817.
\section{Selenioso}
\begin{itemize}
\item {Grp. gram.:adj.}
\end{itemize}
Diz-se de um dos ácidos do selênio.
\section{Selenita}
\begin{itemize}
\item {Grp. gram.:f.}
\end{itemize}
\begin{itemize}
\item {Grp. gram.:F.}
\end{itemize}
Supposto habitante da Lua.
Designação antiga do sulfato de cobre.
(Cp. \textunderscore selenite\textunderscore )
\section{Selenite}
\begin{itemize}
\item {Grp. gram.:f.}
\end{itemize}
\begin{itemize}
\item {Proveniência:(Gr. \textunderscore selenites\textunderscore )}
\end{itemize}
Bello crystal transparente, sob cuja fórma se encontra em a natureza o gypso ou gêsso de Paris.
Gêsso crystallizado.
\section{Selenito}
\begin{itemize}
\item {Grp. gram.:m.}
\end{itemize}
O mesmo ou melhor que \textunderscore selenite\textunderscore .
\section{Selenitoso}
\begin{itemize}
\item {Grp. gram.:adj.}
\end{itemize}
Que contém selenito.
\section{Selenocéfalo}
\begin{itemize}
\item {Grp. gram.:m.}
\end{itemize}
\begin{itemize}
\item {Proveniência:(Do gr. \textunderscore selene\textunderscore  + \textunderscore kephale\textunderscore )}
\end{itemize}
Gênero de insectos hemípteros.
\section{Selenocêntrico}
\begin{itemize}
\item {Grp. gram.:adj.}
\end{itemize}
\begin{itemize}
\item {Proveniência:(De \textunderscore selene\textunderscore  gr. + \textunderscore centro\textunderscore )}
\end{itemize}
Relativo ao centro da Lua.
\section{Selenocéphalo}
\begin{itemize}
\item {Grp. gram.:m.}
\end{itemize}
\begin{itemize}
\item {Proveniência:(Do gr. \textunderscore selene\textunderscore  + \textunderscore kephale\textunderscore )}
\end{itemize}
Gênero de insectos hemípteros.
\section{Selenódero}
\begin{itemize}
\item {Grp. gram.:m.}
\end{itemize}
Gênero de insectos coleópteros pentâmeros.
\section{Selenodonte}
\begin{itemize}
\item {Grp. gram.:m.}
\end{itemize}
\begin{itemize}
\item {Proveniência:(Do gr. \textunderscore selene\textunderscore  + \textunderscore odous\textunderscore )}
\end{itemize}
Gênero de insectos coleópteros pentâmeros.
\section{Selenognóstica}
\begin{itemize}
\item {Grp. gram.:f.}
\end{itemize}
\begin{itemize}
\item {Proveniência:(Do gr. \textunderscore selene\textunderscore  + \textunderscore gnosis\textunderscore )}
\end{itemize}
Reunião de todos os factos conhecidos á cêrca da constituição phýsica da Lua.
\section{Selenografia}
\begin{itemize}
\item {Grp. gram.:f.}
\end{itemize}
Descripção da Lua.
(Cp. \textunderscore selenógrafo\textunderscore )
\section{Selenográfico}
\begin{itemize}
\item {Grp. gram.:adj.}
\end{itemize}
Relativo á selenografia.
\section{Selenógrafo}
\begin{itemize}
\item {Grp. gram.:m.}
\end{itemize}
\begin{itemize}
\item {Proveniência:(Do gr. \textunderscore selene\textunderscore  + \textunderscore graphein\textunderscore )}
\end{itemize}
Tratadista de selenografia.
\section{Selenographia}
\begin{itemize}
\item {Grp. gram.:f.}
\end{itemize}
Descripção da Lua.
(Cp. \textunderscore selenógrapho\textunderscore )
\section{Selenográphico}
\begin{itemize}
\item {Grp. gram.:adj.}
\end{itemize}
Relativo á selenographia.
\section{Selenógrapho}
\begin{itemize}
\item {Grp. gram.:m.}
\end{itemize}
\begin{itemize}
\item {Proveniência:(Do gr. \textunderscore selene\textunderscore  + \textunderscore graphein\textunderscore )}
\end{itemize}
Tratadista de selenographia.
\section{Selenóide}
\begin{itemize}
\item {Grp. gram.:m.}
\end{itemize}
\begin{itemize}
\item {Proveniência:(Do gr. \textunderscore selene\textunderscore  + \textunderscore eidos\textunderscore )}
\end{itemize}
Apparelho de inducção eléctrica.
\section{Selenologia}
\begin{itemize}
\item {Grp. gram.:f.}
\end{itemize}
\begin{itemize}
\item {Proveniência:(Do gr. \textunderscore selene\textunderscore  + \textunderscore logos\textunderscore )}
\end{itemize}
Tratado á cêrca da Lua e seus phenómenos.
\section{Selenopalpo}
\begin{itemize}
\item {Grp. gram.:m.}
\end{itemize}
Gênero de insectos coleópteros heterómeros.
\section{Selenope}
\begin{itemize}
\item {Grp. gram.:f.}
\end{itemize}
Gênero de aranhas.
\section{Selenose}
\begin{itemize}
\item {Grp. gram.:f.}
\end{itemize}
\begin{itemize}
\item {Proveniência:(Do gr. \textunderscore selene\textunderscore )}
\end{itemize}
Mancha esbranquiçada nas unhas.
\section{Selenóstato}
\begin{itemize}
\item {Grp. gram.:m.}
\end{itemize}
\begin{itemize}
\item {Proveniência:(Do gr. \textunderscore selene\textunderscore  + \textunderscore states\textunderscore )}
\end{itemize}
Instrumento fixo, com que se observam os movimentos da Lua.
\section{Selenotopografia}
\begin{itemize}
\item {Grp. gram.:f.}
\end{itemize}
\begin{itemize}
\item {Proveniência:(Do gr. \textunderscore selene\textunderscore  + \textunderscore topos\textunderscore  + \textunderscore graphein\textunderscore )}
\end{itemize}
Descripção da superfície da Lua.
\section{Selenotopográfico}
\begin{itemize}
\item {Grp. gram.:adj.}
\end{itemize}
Relativo á selenotopografia.
\section{Selenotopographia}
\begin{itemize}
\item {Grp. gram.:f.}
\end{itemize}
\begin{itemize}
\item {Proveniência:(Do gr. \textunderscore selene\textunderscore  + \textunderscore topos\textunderscore  + \textunderscore graphein\textunderscore )}
\end{itemize}
Descripção da superfície da Lua.
\section{Selenotopográphico}
\begin{itemize}
\item {Grp. gram.:adj.}
\end{itemize}
Relativo á selenotopographia.
\section{Selêucida}
\begin{itemize}
\item {Grp. gram.:m.}
\end{itemize}
\begin{itemize}
\item {Proveniência:(De \textunderscore Seleuco\textunderscore , n. p.)}
\end{itemize}
Membro de uma dynastia de Reis gregos da Sýria, fundada por Seleuco, general de Alexandre Magno.
\section{Selha}
\begin{itemize}
\item {fónica:sê}
\end{itemize}
\begin{itemize}
\item {Grp. gram.:f.}
\end{itemize}
\begin{itemize}
\item {Proveniência:(Do lat. \textunderscore situla\textunderscore )}
\end{itemize}
Vaso redondo de madeira, ou tabuleiro, com bordas baixas.
\section{Selho}
\begin{itemize}
\item {fónica:sé}
\end{itemize}
\begin{itemize}
\item {Grp. gram.:adj.}
\end{itemize}
\begin{itemize}
\item {Utilização:Ant.}
\end{itemize}
O mesmo que \textunderscore senho\textunderscore  ou \textunderscore sendo\textunderscore .
\section{Seliera}
\begin{itemize}
\item {Grp. gram.:f.}
\end{itemize}
Gênero de plantas goodeniáceas.
\section{Selina}
\begin{itemize}
\item {Grp. gram.:f.}
\end{itemize}
\begin{itemize}
\item {Utilização:Med.}
\end{itemize}
\begin{itemize}
\item {Proveniência:(Fr. \textunderscore seline\textunderscore , má der. do gr. \textunderscore selene\textunderscore , Lua)}
\end{itemize}
Estado das unhas, caracterizado por manchas brancas, devidas á falta de pigmento.
\section{Selíneas}
\begin{itemize}
\item {Grp. gram.:f. pl.}
\end{itemize}
\begin{itemize}
\item {Proveniência:(De \textunderscore selino\textunderscore )}
\end{itemize}
Tríbo de plantas umbellíferas.
\section{Selino}
\begin{itemize}
\item {Grp. gram.:m.}
\end{itemize}
\begin{itemize}
\item {Proveniência:(Do gr. \textunderscore selinon\textunderscore )}
\end{itemize}
O mesmo que \textunderscore selino-palustre\textunderscore .
\section{Selino-palustre}
\begin{itemize}
\item {Grp. gram.:m.}
\end{itemize}
Planta umbellífera.
\section{Sélio}
\begin{itemize}
\item {Grp. gram.:m.}
\end{itemize}
Gênero de crustáceos.
\section{Selisca}
\begin{itemize}
\item {Grp. gram.:f.}
\end{itemize}
\begin{itemize}
\item {Utilização:Prov.}
\end{itemize}
\begin{itemize}
\item {Utilização:minh.}
\end{itemize}
Bocadinho; pouca coisa.
\section{Selistérnias}
\begin{itemize}
\item {Grp. gram.:f. pl.}
\end{itemize}
\begin{itemize}
\item {Proveniência:(Lat. \textunderscore sellisternia\textunderscore )}
\end{itemize}
Banquete, que os Romanos celebravam em honra de uma deusa, sentando-se as mulheres em cadeiras, emquanto os homens se sentavam em leitos próprios dos festins romanos.
\section{Sella}
\begin{itemize}
\item {Grp. gram.:f.}
\end{itemize}
\begin{itemize}
\item {Utilização:Des.}
\end{itemize}
\begin{itemize}
\item {Utilização:Anat.}
\end{itemize}
\begin{itemize}
\item {Proveniência:(Lat. \textunderscore sella\textunderscore )}
\end{itemize}
Apparelho, que se colloca sôbre a cavalgadura, e que é próprio para se sentar o cavalleiro.
Assento.
Poltrona.
\textunderscore Sella túrcica\textunderscore , cavidade, onde se forma a glândula pituritária.
\textunderscore Sella polaca\textunderscore , espécie de ostra do mar das Indias.
\section{Sellada}
\begin{itemize}
\item {Grp. gram.:f.}
\end{itemize}
\begin{itemize}
\item {Proveniência:(De \textunderscore sella\textunderscore )}
\end{itemize}
Depressão na lombada de um monte.
Cavidade oblonga numa montanha.
\section{Sellado}
\begin{itemize}
\item {Grp. gram.:m.}
\end{itemize}
\begin{itemize}
\item {Grp. gram.:Adj.}
\end{itemize}
\begin{itemize}
\item {Utilização:Prov.}
\end{itemize}
\begin{itemize}
\item {Utilização:Bras. do N}
\end{itemize}
\begin{itemize}
\item {Utilização:minh.}
\end{itemize}
\begin{itemize}
\item {Proveniência:(De \textunderscore sellar\textunderscore ^1)}
\end{itemize}
Curvatura das ilhargas.
Curvatura da parte lateral do pé.
Diz-se da pessôa ou animal, que tem o dorso curvado.
\section{Sellado}
\begin{itemize}
\item {Grp. gram.:adj.}
\end{itemize}
\begin{itemize}
\item {Proveniência:(De \textunderscore sellar\textunderscore ^2)}
\end{itemize}
Que tem sêllo: \textunderscore papel sellado\textunderscore .
\section{Selladoiro}
\begin{itemize}
\item {Grp. gram.:m.}
\end{itemize}
\begin{itemize}
\item {Utilização:Fig.}
\end{itemize}
\begin{itemize}
\item {Proveniência:(De \textunderscore sellar\textunderscore ^1)}
\end{itemize}
Parte do corpo do animal, em que se colloca a sella.
Talhe do fato, correspondente ás ilhargas.
\section{Sellador}
\begin{itemize}
\item {Grp. gram.:m.  e  adj.}
\end{itemize}
\begin{itemize}
\item {Proveniência:(De \textunderscore sellar\textunderscore ^2)}
\end{itemize}
O que sella.
\section{Selladura}
\begin{itemize}
\item {Grp. gram.:f.}
\end{itemize}
Acto ou effeito de sellar^1.
Selladoiro.
\section{Sellagão}
\begin{itemize}
\item {Grp. gram.:m.}
\end{itemize}
\begin{itemize}
\item {Proveniência:(De \textunderscore sellar\textunderscore )}
\end{itemize}
Sella de pequeno arção anterior, e sem arção posterior.
\section{Sellagem}
\begin{itemize}
\item {Grp. gram.:f.}
\end{itemize}
Acto ou effeito de sellar^2.
\section{Sellar}
\begin{itemize}
\item {Grp. gram.:v. t.}
\end{itemize}
Pôr sella em: \textunderscore sellar um cavallo\textunderscore .
\section{Sellar}
\begin{itemize}
\item {Grp. gram.:v. t.}
\end{itemize}
\begin{itemize}
\item {Grp. gram.:V. p.}
\end{itemize}
\begin{itemize}
\item {Utilização:Fig.}
\end{itemize}
\begin{itemize}
\item {Proveniência:(Do lat. \textunderscore sigillare\textunderscore )}
\end{itemize}
Pôr sêllo em.
Carimbar.
Estampilhar: \textunderscore sellar uma carta\textunderscore .
Pôr marca em.
Pôr fim a.
Fechar.
Tornar válido.
Sujeitar-se.
Manchar-se.
\section{Sellaria}
\begin{itemize}
\item {Grp. gram.:f.}
\end{itemize}
\begin{itemize}
\item {Proveniência:(De \textunderscore sella\textunderscore )}
\end{itemize}
Arte de selleiro.
Estabelecimento ou arruamento de selleiros.
Porção de sellas e outros arreios.
\section{Selleiro}
\begin{itemize}
\item {Grp. gram.:m.}
\end{itemize}
\begin{itemize}
\item {Grp. gram.:Adj.}
\end{itemize}
Fabricante ou vendedor de sellas.
Que é bom cavalleiro ou que se sustenta bem na sella.
Diz-se do cavallo, que já experimentou a sella
\section{Sellim}
\begin{itemize}
\item {Grp. gram.:m.}
\end{itemize}
\begin{itemize}
\item {Proveniência:(De \textunderscore sella\textunderscore )}
\end{itemize}
Pequena sella, sem arção.
Mollusco bivalve, pertencente, segundo Cuvier, á fam. das ostras.
\section{Sellistérnias}
\begin{itemize}
\item {Grp. gram.:f. pl.}
\end{itemize}
\begin{itemize}
\item {Proveniência:(Lat. \textunderscore sellisternia\textunderscore )}
\end{itemize}
Banquete, que os Romanos celebravam em honra de uma deusa, sentando-se as mulheres em cadeiras, em-quanto os homens se sentavam em leitos próprios dos festins romanos.
\section{Sêllo}
\begin{itemize}
\item {Grp. gram.:m.}
\end{itemize}
\begin{itemize}
\item {Utilização:Fig.}
\end{itemize}
\begin{itemize}
\item {Utilização:Bras. do N}
\end{itemize}
\begin{itemize}
\item {Proveniência:(Do lat. \textunderscore sigillum\textunderscore )}
\end{itemize}
Peça, geralmente metállica, em que estão gravadas armas, divisa ou assinatura, e que serve para imprimir sôbre certos papéis, para os tornar válidos ou authênticos.
Carimbo, sinete.
Chancella; a marca estampada pelo sinete ou chancella.
Lugar, onde se carimbam ou chancellam officialmente documentos.
Estampilha.
Fecho.
Tudo que serve para fechar.
Sinal; cunho; distinctivo.
Quantia de dinheiro, igual a 480 reis.
\section{Sélloa}
\begin{itemize}
\item {Grp. gram.:f.}
\end{itemize}
Gênero das plantas, da fam. das compostas.
\section{Sêllo-de-salomão}
\begin{itemize}
\item {Grp. gram.:m.}
\end{itemize}
Planta asparagínea, medicinal, (\textunderscore polygonatum officinale\textunderscore , All.). Cf. \textunderscore Desengano da Med.\textunderscore , 59; P. Coutinho, \textunderscore Flora\textunderscore , 138.
\section{Sellote}
\begin{itemize}
\item {Grp. gram.:m.}
\end{itemize}
O mesmo que \textunderscore sellim\textunderscore .
\section{Sêlo}
\begin{itemize}
\item {Grp. gram.:m.}
\end{itemize}
\begin{itemize}
\item {Utilização:Fig.}
\end{itemize}
\begin{itemize}
\item {Utilização:Bras. do N}
\end{itemize}
\begin{itemize}
\item {Proveniência:(Do lat. \textunderscore sigillum\textunderscore )}
\end{itemize}
Peça, geralmente metállica, em que estão gravadas armas, divisa ou assinatura, e que serve para imprimir sôbre certos papéis, para os tornar válidos ou authênticos.
Carimbo, sinete.
Chancella; a marca estampada pelo sinete ou chancella.
Lugar, onde se carimbam ou chancellam officialmente documentos.
Estampilha.
Fecho.
Tudo que serve para fechar.
Sinal; cunho; distinctivo.
Quantia de dinheiro, igual a 480 reis.
\section{Séloa}
\begin{itemize}
\item {Grp. gram.:f.}
\end{itemize}
Gênero das plantas, da fam. das compostas.
\section{Selote}
\begin{itemize}
\item {Grp. gram.:m.}
\end{itemize}
O mesmo que \textunderscore selim\textunderscore .
\section{Selva}
\begin{itemize}
\item {Grp. gram.:f.}
\end{itemize}
\begin{itemize}
\item {Utilização:Fig.}
\end{itemize}
\begin{itemize}
\item {Proveniência:(Do lat. \textunderscore silva\textunderscore )}
\end{itemize}
Lugar, arborizado por natureza.
Matatagal; bosque.
Grande porção de coisas, especialmente de coisas emmaranhadas.
\section{Selvageira}
\begin{itemize}
\item {Grp. gram.:f.}
\end{itemize}
\begin{itemize}
\item {Utilização:T. da Madeira}
\end{itemize}
Arbusto, da fam. das labiadas, (\textunderscore sideritís massoniana\textunderscore , Benth.). Cf. \textunderscore Bol. da Socied. de Geogr.\textunderscore , XXX, 618.
\section{Selvagem}
\begin{itemize}
\item {Grp. gram.:adj.}
\end{itemize}
\begin{itemize}
\item {Grp. gram.:M.  e  f.}
\end{itemize}
Relativo a selvas ou próprio dellas: \textunderscore vida selvagem\textunderscore .
Inculto; agreste.
Que vive nas selvas ou fóra do convívio humano: \textunderscore homem selvagem\textunderscore .
Bravio, não domesticado: \textunderscore gato selvagem\textunderscore .
Despovoado.
Silvestre.
Que nasce sem cultura: \textunderscore planta selvagem\textunderscore .
Bárbaro; rude; bruto; ignorante: \textunderscore índole selvagem\textunderscore .
Pessôa grosseira.
Pessôa que não convive com os seus semelhantes, ou que vive nas selvas.
\section{Selvagem}
\begin{itemize}
\item {Grp. gram.:f.}
\end{itemize}
Antiga peça de artilharia, o mesmo que \textunderscore salvagem\textunderscore ^2:«\textunderscore ...vendo que em a casa da pólvora não havia nenhuma mandou descarregar um espalhafato e uma selvagem.\textunderscore »Lopo Coutinho, \textunderscore Hist. do Cêrco de Dio\textunderscore , l. 2.^o, c. 19, 222.
(Cp. \textunderscore salvagem\textunderscore ^2)
\section{Selvagíneo}
\begin{itemize}
\item {Grp. gram.:adj.}
\end{itemize}
\begin{itemize}
\item {Proveniência:(De \textunderscore selvagem\textunderscore )}
\end{itemize}
Selvático.
Relativo aos animaes selvagens.
\section{Selvagino}
\begin{itemize}
\item {Grp. gram.:adj.}
\end{itemize}
O mesmo que \textunderscore selvagíneo\textunderscore .
\section{Selvagismo}
\begin{itemize}
\item {Grp. gram.:m.}
\end{itemize}
Qualidade do que é selvagem.
Acto ou dito próprio de selvagem; modos de selvagem.
\section{Selvajaria}
\begin{itemize}
\item {Grp. gram.:f.}
\end{itemize}
Qualidade do que é selvagem.
Acto ou dito próprio de selvagem; modos de selvagem.
\section{Selvaticamente}
\begin{itemize}
\item {Grp. gram.:adv.}
\end{itemize}
De modo selvático; rudemente; á maneira de selvagem.
\section{Selvático}
\begin{itemize}
\item {Grp. gram.:adj.}
\end{itemize}
\begin{itemize}
\item {Proveniência:(Do lat. \textunderscore silvaticus\textunderscore )}
\end{itemize}
Que nasce ou se cria nas selvas; selvagem.
\section{Selvatiqueza}
\begin{itemize}
\item {Grp. gram.:f.}
\end{itemize}
\begin{itemize}
\item {Utilização:Des.}
\end{itemize}
\begin{itemize}
\item {Proveniência:(De \textunderscore selvátivo\textunderscore )}
\end{itemize}
O mesmo que \textunderscore selvajaria\textunderscore .
\section{Selvela}
\begin{itemize}
\item {Grp. gram.:f.}
\end{itemize}
\begin{itemize}
\item {Proveniência:(De \textunderscore selva\textunderscore ?)}
\end{itemize}
Espécie de ameixa das Antilhas.
\section{Selvoso}
\begin{itemize}
\item {Grp. gram.:adj.}
\end{itemize}
\begin{itemize}
\item {Proveniência:(Do lat. \textunderscore silvosus\textunderscore )}
\end{itemize}
Em que há selvas.
\section{Sem}
\begin{itemize}
\item {Grp. gram.:prep.}
\end{itemize}
\begin{itemize}
\item {Grp. gram.:Adv.}
\end{itemize}
\begin{itemize}
\item {Utilização:Ant.}
\end{itemize}
\begin{itemize}
\item {Proveniência:(Do lat. \textunderscore sine\textunderscore )}
\end{itemize}
(designativa de \textunderscore falta\textunderscore , \textunderscore exclusão\textunderscore , \textunderscore ausência\textunderscore , \textunderscore condição\textunderscore , \textunderscore excepção\textunderscore , etc.)
O mesmo que \textunderscore não\textunderscore ^1.
\section{Semafórico}
\begin{itemize}
\item {Grp. gram.:adj.}
\end{itemize}
\begin{itemize}
\item {Grp. gram.:M.}
\end{itemize}
\begin{itemize}
\item {Proveniência:(De \textunderscore semáforo\textunderscore )}
\end{itemize}
Diz-se do telégrafo, que nas costas marítimas dá sinal da chegada ou manobras dos navios, que navegam ou cruzam á vista das costas ou deante dos portos.
Telegrafista, encarregado dêsse telégrafo.
\section{Semáforo}
\begin{itemize}
\item {Grp. gram.:m.}
\end{itemize}
\begin{itemize}
\item {Proveniência:(Do gr. \textunderscore sema\textunderscore  + \textunderscore phoros\textunderscore )}
\end{itemize}
Telégrafo aéreo, estabelecido junto dos portos ou em pontos elevados da costa, para noticiar a chegada ou a passagem de navios de guerra ou mercantes.
Poste de sinaes nas linhas férreas, com farol e hastes móveis.
\section{Semana}
\begin{itemize}
\item {Grp. gram.:f.}
\end{itemize}
\begin{itemize}
\item {Grp. gram.:Loc.}
\end{itemize}
\begin{itemize}
\item {Utilização:pop.}
\end{itemize}
\begin{itemize}
\item {Utilização:T. escol. de Coimbra.}
\end{itemize}
\begin{itemize}
\item {Utilização:Coimbra}
\end{itemize}
\begin{itemize}
\item {Proveniência:(Do lat. \textunderscore septimana\textunderscore )}
\end{itemize}
Espaço de sete dias, desde o Domingo de Sábbado inclusivamente.
Espaço de sete dias consecutivos.
Os seis dias immediatos ao Domingo.
Trabalho, que dura uma semana: \textunderscore pagar a semana ao operário\textunderscore .
Retribuição dêsse trabalho: \textunderscore já recebeu a semana\textunderscore .
Espécie de jôgo popular.
\textunderscore Na semana dos nove dias\textunderscore , nunca.
\textunderscore Semana macha\textunderscore , a semana, em que não há o feriado de quinta-feira.
\section{Semanal}
\begin{itemize}
\item {Grp. gram.:adj.}
\end{itemize}
Relativo á semana.
Que se faz ou acontece de semana a semana: \textunderscore publicação semanal\textunderscore .
\section{Semanalmente}
\begin{itemize}
\item {Grp. gram.:adv.}
\end{itemize}
De modo semanal; de semana a semana; de sete em sete dias.
\section{Semanário}
\begin{itemize}
\item {Grp. gram.:adj.}
\end{itemize}
\begin{itemize}
\item {Grp. gram.:M.}
\end{itemize}
\begin{itemize}
\item {Utilização:Bras}
\end{itemize}
O mesmo que \textunderscore semanal\textunderscore .
Periódico, que se publíca uma vez em cada semana.
Cada camarista que, em cada semana, estava ao serviço do Imperador.
\section{Semana-solteira}
\begin{itemize}
\item {Grp. gram.:f.}
\end{itemize}
\begin{itemize}
\item {Utilização:Bras}
\end{itemize}
Semana, que não tem dia santo.
\section{Semaneiro}
\begin{itemize}
\item {Grp. gram.:adj.}
\end{itemize}
\begin{itemize}
\item {Utilização:Ant.}
\end{itemize}
O mesmo que \textunderscore semanal\textunderscore .
\section{Semântica}
\begin{itemize}
\item {Grp. gram.:f.}
\end{itemize}
\begin{itemize}
\item {Proveniência:(De \textunderscore semântico\textunderscore )}
\end{itemize}
O mesmo que \textunderscore semiologia\textunderscore , em Philologia.
\section{Semântico}
\begin{itemize}
\item {Grp. gram.:adj.}
\end{itemize}
\begin{itemize}
\item {Proveniência:(Gr. \textunderscore semântikos\textunderscore )}
\end{itemize}
Relativo a significação; significativo.
\section{Semaphórico}
\begin{itemize}
\item {Grp. gram.:adj.}
\end{itemize}
\begin{itemize}
\item {Grp. gram.:M.}
\end{itemize}
\begin{itemize}
\item {Proveniência:(De \textunderscore semáphoro\textunderscore )}
\end{itemize}
Diz-se do telégrapho, que nas costas marítimas dá sinal da chegada ou manobras dos navios, que navegam ou cruzam á vista das costas ou deante dos portos.
Telegraphista, encarregado dêsse telégrapho.
\section{Semáphoro}
\begin{itemize}
\item {Grp. gram.:m.}
\end{itemize}
\begin{itemize}
\item {Proveniência:(Do gr. \textunderscore sema\textunderscore  + \textunderscore phoros\textunderscore )}
\end{itemize}
Telégrapho aéreo, estabelecido junto dos portos ou em pontos elevados da costa, para noticiar a chegada ou a passagem de navios de guerra ou mercantes.
Poste de sinaes nas linhas férreas, com farol e hastes móveis.
\section{Sematologia}
\begin{itemize}
\item {Grp. gram.:f.}
\end{itemize}
\begin{itemize}
\item {Utilização:Espir.}
\end{itemize}
\begin{itemize}
\item {Proveniência:(Do gr. \textunderscore sema\textunderscore , \textunderscore sematos\textunderscore  + \textunderscore logos\textunderscore )}
\end{itemize}
Tratado da significação e modificações das palavras.
Linguagem dos sinaes, ou communicação dos espíritos pelo movimento dos corpos inertes. Cf. Lachatre, \textunderscore Diction. Univ.\textunderscore 
\section{Sembagulho}
\begin{itemize}
\item {Grp. gram.:m.}
\end{itemize}
Casta de uva. Cf. Lapa, \textunderscore Proc. de Vin.\textunderscore , 26.
\section{Sembarba}
\begin{itemize}
\item {Grp. gram.:m.}
\end{itemize}
O mesmo que \textunderscore ôlho-rapado\textunderscore .
\section{Semblante}
\begin{itemize}
\item {Grp. gram.:m.}
\end{itemize}
\begin{itemize}
\item {Utilização:Fig.}
\end{itemize}
O mesmo que \textunderscore rosto\textunderscore ^1.
Apparência; physionomia.
(Cast. \textunderscore semblante\textunderscore )
\section{Sembereba}
\begin{itemize}
\item {Grp. gram.:f.}
\end{itemize}
(V.sebereba)
\section{Semble}
\begin{itemize}
\item {Grp. gram.:m.}
\end{itemize}
Gênero de insectos neurópteros.
\section{Sêmblide}
\begin{itemize}
\item {Grp. gram.:m.}
\end{itemize}
O mesmo que \textunderscore semble\textunderscore .
\section{Sêmblidos}
\begin{itemize}
\item {Grp. gram.:m. pl.}
\end{itemize}
\begin{itemize}
\item {Proveniência:(De \textunderscore sêmblide\textunderscore )}
\end{itemize}
Família de insectos nevrópteros.
\section{Sembra}
\begin{itemize}
\item {Grp. gram.:f.}
\end{itemize}
\begin{itemize}
\item {Utilização:T. de Miranda}
\end{itemize}
Monte de palha.
(Cp. \textunderscore ensembra\textunderscore )
\section{Sembrante}
\begin{itemize}
\item {Grp. gram.:m.}
\end{itemize}
O mesmo que \textunderscore semblante\textunderscore :«\textunderscore só o Arcebispo não recebeo alteração nem mudou sembrante.\textunderscore »Sousa, \textunderscore Vida do Arceb.\textunderscore 
\section{Sembrar}
\begin{itemize}
\item {Grp. gram.:v. i.}
\end{itemize}
\begin{itemize}
\item {Utilização:Des.}
\end{itemize}
O mesmo que \textunderscore parecer\textunderscore .
(Cast. \textunderscore semblar\textunderscore )
\section{Sem-ceremónia}
\begin{itemize}
\item {Grp. gram.:f.}
\end{itemize}
\begin{itemize}
\item {Proveniência:(De \textunderscore sem\textunderscore  + \textunderscore ceremónia\textunderscore )}
\end{itemize}
Falta de ceremónia.
Desprêzo das convenções sociaes.
Falta de polidez.
\section{Sem-ceremonioso}
\begin{itemize}
\item {Grp. gram.:adj.}
\end{itemize}
Em que há sem-ceremónia; que não tem ceremónia.
\section{Sem-dita}
\begin{itemize}
\item {Grp. gram.:m.  e  f.}
\end{itemize}
Pessôa infeliz. Cf. \textunderscore Viriato Trág.\textunderscore , XIV, 47.
\section{Sêmea}
\begin{itemize}
\item {Grp. gram.:f.}
\end{itemize}
\begin{itemize}
\item {Utilização:Prov.}
\end{itemize}
\begin{itemize}
\item {Proveniência:(Do lat. \textunderscore simila\textunderscore )}
\end{itemize}
Flôr da farinha de trigo.
Parte da farinha do trigo, que fica depois que a mesma farinha é peneirada e separada do rolão.
Farelo miúdo.
Pão de sêmea.
\section{Semeação}
\begin{itemize}
\item {Grp. gram.:f.}
\end{itemize}
Acto ou effeito de semear.
\section{Semeada}
\begin{itemize}
\item {Grp. gram.:f.}
\end{itemize}
\begin{itemize}
\item {Proveniência:(De \textunderscore semeado\textunderscore )}
\end{itemize}
Terreno semeado; sementeira.
\section{Semeadiço}
\begin{itemize}
\item {Grp. gram.:adj.}
\end{itemize}
\begin{itemize}
\item {Utilização:Mad}
\end{itemize}
\begin{itemize}
\item {Proveniência:(De \textunderscore semear\textunderscore )}
\end{itemize}
Diz-se dos terrenos de semeadura.
\section{Semeado}
\begin{itemize}
\item {Grp. gram.:adj.}
\end{itemize}
\begin{itemize}
\item {Grp. gram.:M.}
\end{itemize}
\begin{itemize}
\item {Proveniência:(De \textunderscore semear\textunderscore )}
\end{itemize}
Em que se fez sementeira: \textunderscore terreno semeado\textunderscore .
Que se semeou: \textunderscore trigo semeado\textunderscore .
O mesmo que \textunderscore semeada\textunderscore .
\section{Semeadoiro}
\begin{itemize}
\item {Grp. gram.:m.  e  adj.}
\end{itemize}
\begin{itemize}
\item {Proveniência:(De \textunderscore semear\textunderscore )}
\end{itemize}
Diz-se do terreno disposto, para receber a sementeira.
\section{Semeador}
\begin{itemize}
\item {Grp. gram.:m.  e  adj.}
\end{itemize}
\begin{itemize}
\item {Grp. gram.:M.}
\end{itemize}
\begin{itemize}
\item {Proveniência:(Do lat. \textunderscore seminator\textunderscore )}
\end{itemize}
O que semeia.
Máquina, para semear cereaes.
\section{Semeadouro}
\begin{itemize}
\item {Grp. gram.:m.  e  adj.}
\end{itemize}
\begin{itemize}
\item {Proveniência:(De \textunderscore semear\textunderscore )}
\end{itemize}
Diz-se do terreno disposto, para receber a sementeira.
\section{Semeadura}
\begin{itemize}
\item {Grp. gram.:f.}
\end{itemize}
\begin{itemize}
\item {Proveniência:(Do lat. \textunderscore seminatura\textunderscore )}
\end{itemize}
Semeação; semeada.
Porção de cereaes, bastantes para se semear um terreno.
\section{Semear}
\begin{itemize}
\item {Grp. gram.:v. t.}
\end{itemize}
\begin{itemize}
\item {Utilização:Fig.}
\end{itemize}
\begin{itemize}
\item {Proveniência:(Do lat. \textunderscore seminare\textunderscore )}
\end{itemize}
Deitar ou espalhar sementes de, para germinar: \textunderscore semear melancías\textunderscore .
Espalhar ou deitar sementes em: \textunderscore semear uma belga\textunderscore .
Espalhar.
Diffundir.
Propalar.
Publicar: \textunderscore semear doutrinas sans\textunderscore .
Produzir, occasionar: \textunderscore semear discórdias\textunderscore .
Collocar àquém e àlém, sem ordem: \textunderscore semear donativos\textunderscore .
\section{Semeável}
\begin{itemize}
\item {Grp. gram.:adj.}
\end{itemize}
Que se póde semear.
\section{Semeável}
\begin{itemize}
\item {Grp. gram.:adj.}
\end{itemize}
\begin{itemize}
\item {Utilização:Ant.}
\end{itemize}
O mesmo que \textunderscore semelhante\textunderscore .
(Contr. de \textunderscore semelhável\textunderscore )
\section{Semedeiro}
\begin{itemize}
\item {Grp. gram.:m.}
\end{itemize}
\begin{itemize}
\item {Utilização:Ant.}
\end{itemize}
O mesmo que \textunderscore semideiro\textunderscore .
\section{Semeia-linho}
\begin{itemize}
\item {Grp. gram.:m.}
\end{itemize}
\begin{itemize}
\item {Utilização:Prov.}
\end{itemize}
O mesmo que \textunderscore megengra\textunderscore .
O mesmo que \textunderscore ferreirinho\textunderscore .
\section{Semeia-milho}
\begin{itemize}
\item {Grp. gram.:m.}
\end{itemize}
\begin{itemize}
\item {Utilização:T. da Bairrada}
\end{itemize}
O mesmo que \textunderscore cedovém\textunderscore ; semeia-linho.
\section{Semeiologia}
\begin{itemize}
\item {Grp. gram.:f.}
\end{itemize}
(V.semiologia)
\section{Semeiótica}
\begin{itemize}
\item {Grp. gram.:f.}
\end{itemize}
(V.semiótica)
\section{Semel}
\begin{itemize}
\item {Grp. gram.:m.}
\end{itemize}
\begin{itemize}
\item {Utilização:Ant.}
\end{itemize}
\begin{itemize}
\item {Proveniência:(Do lat. \textunderscore semen\textunderscore )}
\end{itemize}
Geração; descendência; descendente.
\section{Semelé}
\begin{itemize}
\item {Grp. gram.:adj. f.}
\end{itemize}
\begin{itemize}
\item {Utilização:Prov.}
\end{itemize}
\begin{itemize}
\item {Utilização:trasm.}
\end{itemize}
Diz-se da mulhér engegada, mas maliciosa.
\section{Semelhança}
\begin{itemize}
\item {Grp. gram.:f.}
\end{itemize}
Qualidade do que é semelhante.
Identidade.
Analogia.
Aspecto.
Conformidade entre uma cópia e seu modêlo.
Imitação.
Confronto.
\section{Semelhante}
\begin{itemize}
\item {Grp. gram.:adj.}
\end{itemize}
\begin{itemize}
\item {Grp. gram.:M.}
\end{itemize}
\begin{itemize}
\item {Grp. gram.:Loc. adv.}
\end{itemize}
\begin{itemize}
\item {Proveniência:(Do lat. \textunderscore similans\textunderscore )}
\end{itemize}
Igual.
Conforme.
Parecido.
Que tem a mesma natureza que outro.
Tal.
Êste.
Aquelle.
Pessôa ou coisa da mesma natureza que outra ou parecida com ella: \textunderscore os nossos semelhantes\textunderscore .
\textunderscore De semelhante\textunderscore , assim desta maneira.
\section{Semelhantemente}
\begin{itemize}
\item {Grp. gram.:adv.}
\end{itemize}
De modo semelhante.
\section{Semelhar}
\begin{itemize}
\item {Grp. gram.:v. t.}
\end{itemize}
\begin{itemize}
\item {Proveniência:(Do lat. \textunderscore similare\textunderscore )}
\end{itemize}
Sêr semelhante a.
Têr a apparência de.
Confrontar, comparar.
\section{Semelhável}
\begin{itemize}
\item {Grp. gram.:adj.}
\end{itemize}
Que se póde semelhar.
Semelhante.
\section{Semelhavelmente}
\begin{itemize}
\item {Grp. gram.:adv.}
\end{itemize}
De modo semelhável.
\section{Sêmen}
\begin{itemize}
\item {Grp. gram.:m.}
\end{itemize}
\begin{itemize}
\item {Proveniência:(Lat. \textunderscore semen\textunderscore )}
\end{itemize}
Semente.
Esperma.
\section{Sêmen-contra}
\begin{itemize}
\item {Grp. gram.:m.}
\end{itemize}
Medicamente, contra os vermes intestinaes, constituído por uns pós vegetaes, amargos e aromáticos.
Gênero de plantas synanthéreas medicinaes, (\textunderscore artemisia contra\textunderscore , Lin.).
\section{Semental}
\begin{itemize}
\item {Grp. gram.:adj.}
\end{itemize}
Relativo a semente.
Que é bom reproductor.
\section{Sementão}
\begin{itemize}
\item {Grp. gram.:m.}
\end{itemize}
\begin{itemize}
\item {Utilização:Prov.}
\end{itemize}
\begin{itemize}
\item {Proveniência:(De \textunderscore semente\textunderscore )}
\end{itemize}
Bode, para cobrição de cabras.
\section{Sementar}
\begin{itemize}
\item {Grp. gram.:v. t.}
\end{itemize}
\begin{itemize}
\item {Utilização:Prov.}
\end{itemize}
\begin{itemize}
\item {Utilização:trasm.}
\end{itemize}
\begin{itemize}
\item {Utilização:Des.}
\end{itemize}
\begin{itemize}
\item {Utilização:Bras}
\end{itemize}
\begin{itemize}
\item {Proveniência:(Lat. \textunderscore sementare\textunderscore )}
\end{itemize}
Semear.
Emprestar ou dar sementes a.
Fornecer canas de açucar a, para plantações.
\section{Semente}
\begin{itemize}
\item {Grp. gram.:f.}
\end{itemize}
\begin{itemize}
\item {Utilização:Fig.}
\end{itemize}
\begin{itemize}
\item {Utilização:Bras}
\end{itemize}
\begin{itemize}
\item {Proveniência:(Lat. \textunderscore sementis\textunderscore )}
\end{itemize}
Grão de cereaes, que se lança na terra, para germinar.
Qualquer grão ou substância, que se semeia ou que se póde semear.
Esperma.
Origem.
Aquillo que, comparado a uma semente, deve germinar no espírito ou no coração.
Aquillo que com o tempo deve produzir certo effeito.
Pedaços de cana para plantação.
\section{Sementeira}
\begin{itemize}
\item {Grp. gram.:f.}
\end{itemize}
\begin{itemize}
\item {Utilização:Fig.}
\end{itemize}
\begin{itemize}
\item {Proveniência:(Do b. lat. \textunderscore samentaria\textunderscore )}
\end{itemize}
Porção de semente, que se lança á terra para germinar.
Terreno semeado.
Viveiro de plantas.
Derramamento.
Origem.
\section{Sementeiro}
\begin{itemize}
\item {Grp. gram.:m.  e  adj.}
\end{itemize}
\begin{itemize}
\item {Proveniência:(De \textunderscore semente\textunderscore )}
\end{itemize}
Semeador.
Diz-se do saco, em que se levam as sementes para o terreno a que se destinam.
\section{Sementilhas}
\begin{itemize}
\item {Grp. gram.:f. pl.}
\end{itemize}
\begin{itemize}
\item {Proveniência:(De \textunderscore semente\textunderscore )}
\end{itemize}
Sementes da saponária.
\section{Sementinas}
\begin{itemize}
\item {Grp. gram.:f. pl.}
\end{itemize}
\begin{itemize}
\item {Proveniência:(Do lat. \textunderscore sementis\textunderscore )}
\end{itemize}
Festas antigas, que os lavradores celebravam depois das sementeiras, para obter de Ceres colheita abundante. Cf. \textunderscore Diction. Abregé d'Antiq.\textunderscore 
\section{Semestral}
\begin{itemize}
\item {Grp. gram.:adj.}
\end{itemize}
Relativo a semestre; que succede ou se realiza, de seis em seis meses: \textunderscore pagamento semestral\textunderscore .
\section{Semestre}
\begin{itemize}
\item {Grp. gram.:m.}
\end{itemize}
\begin{itemize}
\item {Grp. gram.:Adj.}
\end{itemize}
\begin{itemize}
\item {Utilização:Des.}
\end{itemize}
\begin{itemize}
\item {Proveniência:(Lat. \textunderscore semestris\textunderscore )}
\end{itemize}
Espaço de seis meses consecutivos.
O que se paga ou se recebe pelo ordenado ou pela renda, relativa a seis meses: \textunderscore pagar um semestre\textunderscore .
Semestral.
Que dura seis meses consecutivos ou que se faz de seis em seis meses.
\section{Semestreiro}
\begin{itemize}
\item {Grp. gram.:adj.}
\end{itemize}
O mesmo que \textunderscore semestral\textunderscore .
\section{Semetidinho}
\begin{itemize}
\item {Grp. gram.:adj.}
\end{itemize}
\begin{itemize}
\item {Utilização:Prov.}
\end{itemize}
\begin{itemize}
\item {Utilização:minh.}
\end{itemize}
Tímido, acanhado.
(Por \textunderscore sometidinho\textunderscore , de \textunderscore someter\textunderscore  = \textunderscore submeter\textunderscore )
\section{Sem-fim}
\begin{itemize}
\item {Grp. gram.:adj.}
\end{itemize}
\begin{itemize}
\item {Grp. gram.:M.}
\end{itemize}
Indefinido; que não tem número.
Quantidade ou número indeterminado.
Espaço indefinido, illimitado.
\section{Semi...}
\begin{itemize}
\item {Grp. gram.:pref.}
\end{itemize}
\begin{itemize}
\item {Proveniência:(Do lat. \textunderscore semis\textunderscore )}
\end{itemize}
(designativo de \textunderscore meio\textunderscore  ou \textunderscore metade\textunderscore )
\section{Semiabarcante}
\begin{itemize}
\item {Grp. gram.:adj.}
\end{itemize}
\begin{itemize}
\item {Utilização:Bot.}
\end{itemize}
Diz-se da fôlha dos vegetaes, que abraça metade da haste.
(Do \textunderscore semi...\textunderscore  + \textunderscore abarcar\textunderscore )
\section{Semiaberto}
\begin{itemize}
\item {Grp. gram.:adj.}
\end{itemize}
\begin{itemize}
\item {Proveniência:(De \textunderscore semi...\textunderscore  + \textunderscore aberto\textunderscore )}
\end{itemize}
Meio aberto; entreaberto. Cf. Herculano, \textunderscore Eurico\textunderscore , XIV.
\section{Semiacerbo}
\begin{itemize}
\item {Grp. gram.:adj.}
\end{itemize}
\begin{itemize}
\item {Proveniência:(De \textunderscore semi...\textunderscore  + \textunderscore acerbo\textunderscore )}
\end{itemize}
Um tanto azêdo.
\section{Semiaderente}
\begin{itemize}
\item {Grp. gram.:adj.}
\end{itemize}
\begin{itemize}
\item {Utilização:Bot.}
\end{itemize}
\begin{itemize}
\item {Proveniência:(De \textunderscore semi...\textunderscore  + \textunderscore adherente\textunderscore )}
\end{itemize}
Que adere, em parte do seu comprimento.
\section{Semiadherente}
\begin{itemize}
\item {Grp. gram.:adj.}
\end{itemize}
\begin{itemize}
\item {Utilização:Bot.}
\end{itemize}
\begin{itemize}
\item {Proveniência:(De \textunderscore semi...\textunderscore  + \textunderscore adherente\textunderscore )}
\end{itemize}
Que adhere, em parte do seu comprimento.
\section{Semialegórico}
\begin{itemize}
\item {Grp. gram.:adj.}
\end{itemize}
\begin{itemize}
\item {Proveniência:(De \textunderscore semi...\textunderscore  + \textunderscore alegórico\textunderscore )}
\end{itemize}
Que participa do carácter alegórico. Cf. Garrett, \textunderscore Romanceiro\textunderscore , II, 299.
\section{Semiallegórico}
\begin{itemize}
\item {Grp. gram.:adj.}
\end{itemize}
\begin{itemize}
\item {Proveniência:(De \textunderscore semi...\textunderscore  + \textunderscore allegórico\textunderscore )}
\end{itemize}
Que participa do carácter allegórico. Cf. Garrett, \textunderscore Romanceiro\textunderscore , II, 299.
\section{Semialma}
\begin{itemize}
\item {Grp. gram.:f.}
\end{itemize}
\begin{itemize}
\item {Utilização:Fig.}
\end{itemize}
\begin{itemize}
\item {Proveniência:(De \textunderscore semi...\textunderscore  + \textunderscore alma\textunderscore )}
\end{itemize}
Espírito imperfeito ou boçal. Cf. Macedo, \textunderscore Burros\textunderscore , 8.
\section{Semiamplèxicaule}
\begin{itemize}
\item {fónica:csi}
\end{itemize}
\begin{itemize}
\item {Grp. gram.:adj.}
\end{itemize}
\begin{itemize}
\item {Utilização:Bot.}
\end{itemize}
\begin{itemize}
\item {Proveniência:(De \textunderscore semi...\textunderscore  + \textunderscore amplexo\textunderscore  + \textunderscore caule\textunderscore )}
\end{itemize}
Que abraça o tronco em parte.
\section{Semianão}
\begin{itemize}
\item {Grp. gram.:adj.}
\end{itemize}
\begin{itemize}
\item {Proveniência:(De \textunderscore semi...\textunderscore  + \textunderscore anão\textunderscore )}
\end{itemize}
Que tem pequena estatura, um tanto parecida á dos anões. Cf. Júl. Dinís, \textunderscore Morgadinha\textunderscore , 229.
\section{Semiânime}
\begin{itemize}
\item {Grp. gram.:adj.}
\end{itemize}
\begin{itemize}
\item {Proveniência:(Lat. \textunderscore semianimis\textunderscore )}
\end{itemize}
Quasi morto; exânime.
\section{Semiannual}
\begin{itemize}
\item {Grp. gram.:adj.}
\end{itemize}
O mesmo que \textunderscore semiânnuo\textunderscore .
\section{Semiânnuo}
\begin{itemize}
\item {Grp. gram.:adj.}
\end{itemize}
\begin{itemize}
\item {Proveniência:(De \textunderscore semi...\textunderscore  + \textunderscore ânnuo\textunderscore )}
\end{itemize}
O mesmo que \textunderscore semestral\textunderscore .
Que tem meio anno.
\section{Semianual}
\begin{itemize}
\item {Grp. gram.:adj.}
\end{itemize}
O mesmo que \textunderscore semiânuo\textunderscore .
\section{Semiânuo}
\begin{itemize}
\item {Grp. gram.:adj.}
\end{itemize}
\begin{itemize}
\item {Proveniência:(De \textunderscore semi...\textunderscore  + \textunderscore ânnuo\textunderscore )}
\end{itemize}
O mesmo que \textunderscore semestral\textunderscore .
Que tem meio ano.
\section{Semianular}
\begin{itemize}
\item {Grp. gram.:adj.}
\end{itemize}
\begin{itemize}
\item {Proveniência:(De \textunderscore semi...\textunderscore  + \textunderscore anular\textunderscore )}
\end{itemize}
Que tem fórma de meio anel.
\section{Semiarianismo}
\begin{itemize}
\item {Grp. gram.:m.}
\end{itemize}
Seita herética dos semiarianos.
\section{Semiarianos}
\begin{itemize}
\item {Grp. gram.:m. pl.}
\end{itemize}
Partidários do arianismo modificado.
\section{Semibárbaro}
\begin{itemize}
\item {Grp. gram.:adj.}
\end{itemize}
\begin{itemize}
\item {Proveniência:(Do lat. \textunderscore semibarbarus\textunderscore )}
\end{itemize}
Meio bárbaro.
Que não tem quási nada de civilização; quási selvagem. Cf. Herculano, \textunderscore Hist. de Portugal\textunderscore , 93, 252.
\section{Semibíblico}
\begin{itemize}
\item {Grp. gram.:adj.}
\end{itemize}
\begin{itemize}
\item {Proveniência:(De \textunderscore semi...\textunderscore  + \textunderscore bíblico\textunderscore )}
\end{itemize}
Que participa da feição bíblica: \textunderscore estilo semibíblico\textunderscore . Cf. Rui Barb., \textunderscore Cartas de Ingl.\textunderscore , 55.
\section{Semibreve}
\begin{itemize}
\item {Grp. gram.:f.}
\end{itemize}
\begin{itemize}
\item {Proveniência:(De \textunderscore semi...\textunderscore  + \textunderscore breve\textunderscore )}
\end{itemize}
Nota musical, do valor de duas mínimas ou de metade da breve.
\section{Semicadáver}
\begin{itemize}
\item {Grp. gram.:m.}
\end{itemize}
\begin{itemize}
\item {Proveniência:(De \textunderscore semi...\textunderscore  + \textunderscore cadáver\textunderscore )}
\end{itemize}
Pessôa semimorta.
\section{Semicapro}
\begin{itemize}
\item {Grp. gram.:m.  e  adj.}
\end{itemize}
\begin{itemize}
\item {Proveniência:(Do lat. \textunderscore semicaper\textunderscore )}
\end{itemize}
Diz-se dos seres fabulosos, cujo corpo é homem numa das metades e bode na outra.
\section{Semichas}
\begin{itemize}
\item {Grp. gram.:f. pl.}
\end{itemize}
\begin{itemize}
\item {Utilização:Pop.}
\end{itemize}
Aquillo que sobeja ou se entorna, quando se medem cereaes ou líquidos.
\section{Semichristão}
\begin{itemize}
\item {Grp. gram.:m.}
\end{itemize}
\begin{itemize}
\item {Proveniência:(De \textunderscore semi...\textunderscore  + \textunderscore christão\textunderscore )}
\end{itemize}
Indivíduo que é um tanto christão. Cf. Castilho, \textunderscore Fastos\textunderscore , I, 529; II, 490.
\section{Semicilíndrico}
\begin{itemize}
\item {Grp. gram.:adj.}
\end{itemize}
\begin{itemize}
\item {Proveniência:(De \textunderscore semi...\textunderscore  + \textunderscore cilíndrico\textunderscore )}
\end{itemize}
Que tem fórma de meio cilindro.
\section{Semicilindro}
\begin{itemize}
\item {Grp. gram.:m.}
\end{itemize}
Meio cilindro. Cf. Rui Barb., \textunderscore Réplica\textunderscore , 158.
\section{Semicircular}
\begin{itemize}
\item {Grp. gram.:adj.}
\end{itemize}
Relativo ou semelhante ao semicírculo.
\section{Semicírculo}
\begin{itemize}
\item {Grp. gram.:m.}
\end{itemize}
\begin{itemize}
\item {Grp. gram.:Adj.}
\end{itemize}
\begin{itemize}
\item {Proveniência:(De \textunderscore semi...\textunderscore  + \textunderscore círculo\textunderscore )}
\end{itemize}
Metade de um círculo.
Instrumento mathemático, em fórma de semicírculo, e dividido em 180°.
O mesmo que \textunderscore semicircular\textunderscore . Cf. Filinto, XIV, 66 e 100.
\section{Semicivilizado}
\begin{itemize}
\item {Grp. gram.:adj.}
\end{itemize}
\begin{itemize}
\item {Proveniência:(De \textunderscore semi...\textunderscore  + \textunderscore civilizado\textunderscore )}
\end{itemize}
Um tanto civilizado. Cf. Rui Rarb., \textunderscore Réplica\textunderscore , 158.
\section{Semiclausura}
\begin{itemize}
\item {Grp. gram.:f.}
\end{itemize}
Encêrro, pouco menos apertado que uma clausura. Cf. Castilho, \textunderscore Metam.\textunderscore , XL.
\section{Semicolcheia}
\begin{itemize}
\item {Grp. gram.:f.}
\end{itemize}
\begin{itemize}
\item {Proveniência:(De \textunderscore semi...\textunderscore  + \textunderscore colcheia\textunderscore )}
\end{itemize}
Nota musical do valor de metade da colcheia.
\section{Semicolosso}
\begin{itemize}
\item {Grp. gram.:m.}
\end{itemize}
Pessôa ou coisa um tanto colossal. Cf. Rui Barb., \textunderscore Réplica\textunderscore , 158.
\section{Semicómico}
\begin{itemize}
\item {Grp. gram.:adj.}
\end{itemize}
\begin{itemize}
\item {Proveniência:(De \textunderscore semi...\textunderscore  + \textunderscore cómico\textunderscore )}
\end{itemize}
Um tanto cómico. Cf. Arn. Gama, \textunderscore Motim\textunderscore , 473.
\section{Semi-complemento}
\begin{itemize}
\item {Grp. gram.:m.}
\end{itemize}
Meio complemento, em Mathemática.
\section{Semicristão}
\begin{itemize}
\item {Grp. gram.:m.}
\end{itemize}
\begin{itemize}
\item {Proveniência:(De \textunderscore semi...\textunderscore  + \textunderscore cristão\textunderscore )}
\end{itemize}
Indivíduo que é um tanto cristão. Cf. Castilho, \textunderscore Fastos\textunderscore , I, 529; II, 490.
\section{Semicubital}
\begin{itemize}
\item {Grp. gram.:adj.}
\end{itemize}
\begin{itemize}
\item {Utilização:Des.}
\end{itemize}
\begin{itemize}
\item {Proveniência:(Do lat. \textunderscore semicubitalis\textunderscore )}
\end{itemize}
Que tem meio côvado de comprimento.
\section{Semicúpio}
\begin{itemize}
\item {Grp. gram.:m.}
\end{itemize}
\begin{itemize}
\item {Proveniência:(Do lat. \textunderscore semicupium\textunderscore )}
\end{itemize}
Banho, em que se immerge o corpo desde as coxas á cintura.
Banho de assento.
\section{Semicúpula}
\begin{itemize}
\item {Grp. gram.:f.}
\end{itemize}
\begin{itemize}
\item {Proveniência:(De \textunderscore semi...\textunderscore  + \textunderscore cúpula\textunderscore )}
\end{itemize}
Abóbada espheroidal de volta inteira.
\section{Semicylíndrico}
\begin{itemize}
\item {Grp. gram.:adj.}
\end{itemize}
\begin{itemize}
\item {Proveniência:(De \textunderscore semi...\textunderscore  + \textunderscore cylíndrico\textunderscore )}
\end{itemize}
Que tem fórma de meio cylindro.
\section{Semicylindro}
\begin{itemize}
\item {Grp. gram.:m.}
\end{itemize}
Meio cylindro. Cf. Rui Barb., \textunderscore Réplica\textunderscore , 158.
\section{Semidéa}
\begin{itemize}
\item {Grp. gram.:f.}
\end{itemize}
\begin{itemize}
\item {Proveniência:(Lat. \textunderscore semidea\textunderscore )}
\end{itemize}
O mesmo que \textunderscore semideusa\textunderscore .
\section{Semidefunto}
\begin{itemize}
\item {Grp. gram.:adj.}
\end{itemize}
O mesmo que \textunderscore semimorto\textunderscore .
\section{Semideia}
\begin{itemize}
\item {Grp. gram.:f.}
\end{itemize}
\begin{itemize}
\item {Proveniência:(Lat. \textunderscore semidea\textunderscore )}
\end{itemize}
O mesmo que \textunderscore semideusa\textunderscore .
\section{Semideiro}
\begin{itemize}
\item {Grp. gram.:m.}
\end{itemize}
\begin{itemize}
\item {Proveniência:(Do lat. \textunderscore semitarius\textunderscore )}
\end{itemize}
Caminho estreito, atalho.
\section{Semideus}
\begin{itemize}
\item {Grp. gram.:m.}
\end{itemize}
\begin{itemize}
\item {Proveniência:(Lat. \textunderscore semideus\textunderscore )}
\end{itemize}
Homem mythológico, de natureza superior á dos homens e inferior á dos deuses.
Heróe divinizado.
\section{Semideusa}
\begin{itemize}
\item {Grp. gram.:f.}
\end{itemize}
\begin{itemize}
\item {Proveniência:(De \textunderscore semi...\textunderscore  + \textunderscore deusa\textunderscore )}
\end{itemize}
Mulhér fabulosa, de natureza superior á das mulheres e inferior á das deusas.
Heroína divinizada.
\section{Semidiâmetro}
\begin{itemize}
\item {Grp. gram.:m.}
\end{itemize}
\begin{itemize}
\item {Proveniência:(De \textunderscore semi...\textunderscore  + \textunderscore diâmetro\textunderscore )}
\end{itemize}
Metade do diâmetro.
\section{Semidiapasão}
\begin{itemize}
\item {Grp. gram.:m.}
\end{itemize}
\begin{itemize}
\item {Utilização:Mús.}
\end{itemize}
\begin{itemize}
\item {Proveniência:(De \textunderscore semi...\textunderscore  + \textunderscore diapasão\textunderscore )}
\end{itemize}
Intervallo dissonante de oito vozes, quatro tons e três semitons maiores.
\section{Semidiáfano}
\begin{itemize}
\item {Grp. gram.:adj.}
\end{itemize}
\begin{itemize}
\item {Proveniência:(De \textunderscore semi...\textunderscore  + \textunderscore diáfano\textunderscore )}
\end{itemize}
Um tanto diáfano.
\section{Semidiapente}
\begin{itemize}
\item {Grp. gram.:m.}
\end{itemize}
\begin{itemize}
\item {Utilização:Mús.}
\end{itemize}
Intervallo de dois tons e dois semitons maiores.
\section{Semidiáphano}
\begin{itemize}
\item {Grp. gram.:adj.}
\end{itemize}
\begin{itemize}
\item {Proveniência:(De \textunderscore semi...\textunderscore  + \textunderscore diáphano\textunderscore )}
\end{itemize}
Um tanto diáphano.
\section{Semidiatesarão}
\begin{itemize}
\item {Grp. gram.:m.}
\end{itemize}
\begin{itemize}
\item {Utilização:Mús.}
\end{itemize}
Intervalo dissonante de quatro vozes, um tom e dois semitons.
\section{Semidiathesarão}
\begin{itemize}
\item {Grp. gram.:m.}
\end{itemize}
\begin{itemize}
\item {Utilização:Mús.}
\end{itemize}
Intervallo dissonante de quatro vozes, um tom e dois semitons.
\section{Semidisco}
\begin{itemize}
\item {Grp. gram.:m.}
\end{itemize}
\begin{itemize}
\item {Proveniência:(De \textunderscore semi...\textunderscore  + \textunderscore disco\textunderscore )}
\end{itemize}
Metade de um disco.
\section{Semidigital}
\begin{itemize}
\item {Grp. gram.:adj.}
\end{itemize}
\begin{itemize}
\item {Proveniência:(De \textunderscore semi...\textunderscore  + \textunderscore digital\textunderscore )}
\end{itemize}
Que tem o comprimento de meio dedo.
\section{Semiditongo}
\begin{itemize}
\item {Grp. gram.:m.}
\end{itemize}
\begin{itemize}
\item {Utilização:Gram.}
\end{itemize}
\begin{itemize}
\item {Proveniência:(De \textunderscore semi...\textunderscore  + \textunderscore ditongo\textunderscore )}
\end{itemize}
Grupo vocálico, em que o som de cada vogal sôa distintamente, sem que se possa separar do som da outra, como em \textunderscore glória\textunderscore , \textunderscore láctea\textunderscore , \textunderscore quando\textunderscore , etc.
\section{Semidítono}
\begin{itemize}
\item {Grp. gram.:m.}
\end{itemize}
\begin{itemize}
\item {Proveniência:(De \textunderscore semi...\textunderscore  + \textunderscore dítono\textunderscore )}
\end{itemize}
Intervallo musical, constante de um tom, um semitom, e uma terceira menor.
\section{Semidivindade}
\begin{itemize}
\item {Grp. gram.:f.}
\end{itemize}
\begin{itemize}
\item {Proveniência:(De \textunderscore semi...\textunderscore  + \textunderscore divindade\textunderscore )}
\end{itemize}
Carácter ou qualidade de semideus.
Um semideus ou uma semideusa.
\section{Semidivino}
\begin{itemize}
\item {Grp. gram.:adj.}
\end{itemize}
\begin{itemize}
\item {Proveniência:(De \textunderscore semi...\textunderscore  + \textunderscore divino\textunderscore )}
\end{itemize}
Quási divino.
\section{Semidobrado}
\begin{itemize}
\item {Grp. gram.:adj.}
\end{itemize}
\begin{itemize}
\item {Proveniência:(De \textunderscore semi...\textunderscore  + \textunderscore dobrado\textunderscore )}
\end{itemize}
Meio dobrado.
\section{Semidobrez}
\begin{itemize}
\item {Grp. gram.:f.}
\end{itemize}
\begin{itemize}
\item {Utilização:Bot.}
\end{itemize}
\begin{itemize}
\item {Proveniência:(De \textunderscore semi...\textunderscore  + \textunderscore dobrez\textunderscore )}
\end{itemize}
Qualidade das flôres semi-dobradas.
\section{Semidouto}
\begin{itemize}
\item {Grp. gram.:m.  e  adj.}
\end{itemize}
\begin{itemize}
\item {Proveniência:(Do lat. \textunderscore semidoctus\textunderscore )}
\end{itemize}
Indivíduo meio douto, ou que tem mediana instrucção.
\section{Semidúplex}
\begin{itemize}
\item {Grp. gram.:adj.}
\end{itemize}
\begin{itemize}
\item {Proveniência:(De \textunderscore semi...\textunderscore  + \textunderscore dúplex\textunderscore )}
\end{itemize}
Diz-se do offício ou da festa ecclesiástica, em que se observa parte do rito duplex.
\section{Semienterrado}
\begin{itemize}
\item {Grp. gram.:adj.}
\end{itemize}
\begin{itemize}
\item {Proveniência:(De \textunderscore semi...\textunderscore  + \textunderscore enterrado\textunderscore )}
\end{itemize}
Meio enterrado.
\section{Semiesfera}
\begin{itemize}
\item {Grp. gram.:f.}
\end{itemize}
Metade de uma esfera.
\section{Semiesférico}
\begin{itemize}
\item {Grp. gram.:adj.}
\end{itemize}
Que tem a fórma de semiesfera.
\section{Semiesferoidal}
\begin{itemize}
\item {Grp. gram.:adj.}
\end{itemize}
Que tem a fórma de um semiesferóide.
\section{Semiesferóide}
\begin{itemize}
\item {Grp. gram.:m.}
\end{itemize}
Meio esferóide.
\section{Semiesphera}
\begin{itemize}
\item {Grp. gram.:f.}
\end{itemize}
Metade de uma esphera.
\section{Semiesphérico}
\begin{itemize}
\item {Grp. gram.:adj.}
\end{itemize}
Que tem a fórma de semiesphera.
\section{Semiespheroidal}
\begin{itemize}
\item {Grp. gram.:adj.}
\end{itemize}
Que tem a fórma de um semiespheróide.
\section{Semiespheróide}
\begin{itemize}
\item {Grp. gram.:m.}
\end{itemize}
Meio espheróide.
\section{Semiestragado}
\begin{itemize}
\item {Grp. gram.:adj.}
\end{itemize}
\begin{itemize}
\item {Proveniência:(De \textunderscore semi...\textunderscore  + \textunderscore estragado\textunderscore )}
\end{itemize}
Um tanto estragado.
\section{Semifavor}
\begin{itemize}
\item {Grp. gram.:m.}
\end{itemize}
Meio favor.
Pequeno favor. Cf. Rui Barb., \textunderscore Réplica\textunderscore , 158.
\section{Semifendido}
\begin{itemize}
\item {Grp. gram.:adj.}
\end{itemize}
\begin{itemize}
\item {Proveniência:(De \textunderscore semi...\textunderscore  + \textunderscore fendido\textunderscore )}
\end{itemize}
Meio fendido; dividido em dois segmentos.
\section{Semífero}
\begin{itemize}
\item {Grp. gram.:adj.}
\end{itemize}
\begin{itemize}
\item {Proveniência:(Lat. \textunderscore semifer\textunderscore )}
\end{itemize}
Que é metade homem e metade animal. Cf. Castilho, \textunderscore Fastos\textunderscore , III, 45.
\section{Semifigurado}
\begin{itemize}
\item {Grp. gram.:adj.}
\end{itemize}
\begin{itemize}
\item {Utilização:Mús.}
\end{itemize}
\begin{itemize}
\item {Proveniência:(De \textunderscore semi...\textunderscore  + \textunderscore figurado\textunderscore )}
\end{itemize}
Diz-se do canto religioso, extremamente singelo, quási cantochão.
\section{Semiflósculo}
\begin{itemize}
\item {Grp. gram.:m.}
\end{itemize}
\begin{itemize}
\item {Utilização:Bot.}
\end{itemize}
\begin{itemize}
\item {Proveniência:(De \textunderscore semi...\textunderscore  + \textunderscore flósculo\textunderscore )}
\end{itemize}
Flósculo liguloso.
\section{Semiflosculoso}
\begin{itemize}
\item {Grp. gram.:adj.}
\end{itemize}
\begin{itemize}
\item {Utilização:Bot.}
\end{itemize}
\begin{itemize}
\item {Proveniência:(De \textunderscore semi...\textunderscore  + \textunderscore flosculoso\textunderscore )}
\end{itemize}
Que tem semiflósculos.
\section{Semifluido}
\begin{itemize}
\item {Grp. gram.:adj.}
\end{itemize}
\begin{itemize}
\item {Proveniência:(De \textunderscore semi...\textunderscore  + \textunderscore fluido\textunderscore )}
\end{itemize}
Que não está inteiramente fluido; xaroposo.
\section{Semíforo}
\begin{itemize}
\item {Grp. gram.:m.}
\end{itemize}
\begin{itemize}
\item {Utilização:Des.}
\end{itemize}
\begin{itemize}
\item {Proveniência:(Do gr. \textunderscore semeion\textunderscore  + \textunderscore phoros\textunderscore )}
\end{itemize}
O mesmo que \textunderscore porta-bandeira\textunderscore .
\section{Semifusa}
\begin{itemize}
\item {Grp. gram.:f.}
\end{itemize}
Nota musical, do valor de metade de uma fusa.
\section{Semigasto}
\begin{itemize}
\item {Grp. gram.:adj.}
\end{itemize}
Meio gasto, um tanto gasto. Cf. Castilho, \textunderscore Fastos\textunderscore , II, 123.
\section{Semiglobuloso}
\begin{itemize}
\item {Grp. gram.:adj.}
\end{itemize}
\begin{itemize}
\item {Utilização:Bot.}
\end{itemize}
\begin{itemize}
\item {Proveniência:(De \textunderscore semi...\textunderscore  + \textunderscore globuloso\textunderscore )}
\end{itemize}
Que tem fórma semiesphérica ou de meio globo.
\section{Semigola}
\begin{itemize}
\item {Grp. gram.:f.}
\end{itemize}
\begin{itemize}
\item {Proveniência:(De \textunderscore semi...\textunderscore  + \textunderscore gola\textunderscore )}
\end{itemize}
Linha, tirada do ângulo da cortina de uma fortaleza para o flanco.
\section{Semigolla}
\begin{itemize}
\item {Grp. gram.:f.}
\end{itemize}
\begin{itemize}
\item {Proveniência:(De \textunderscore semi...\textunderscore  + \textunderscore golla\textunderscore )}
\end{itemize}
Linha, tirada do ângulo da cortina de uma fortaleza para o flanco.
\section{Semigótico}
\begin{itemize}
\item {Grp. gram.:adj.}
\end{itemize}
\begin{itemize}
\item {Proveniência:(De \textunderscore semi...\textunderscore  + \textunderscore gótico\textunderscore )}
\end{itemize}
Que participa do gótico e de outro elemento, especialmente do romano.
\section{Semihistórico}
\begin{itemize}
\item {Grp. gram.:adj.}
\end{itemize}
\begin{itemize}
\item {Proveniência:(De \textunderscore semi...\textunderscore  + \textunderscore histórico\textunderscore )}
\end{itemize}
Que contém factos históricos, entremeados de factos imaginários ou romanescos.
\section{Semihomem}
\begin{itemize}
\item {Grp. gram.:m.}
\end{itemize}
\begin{itemize}
\item {Proveniência:(Do lat. \textunderscore semihomo\textunderscore )}
\end{itemize}
Sêr imaginário, cuja metade é de homem.
\section{Semiistórico}
\begin{itemize}
\item {Grp. gram.:adj.}
\end{itemize}
\begin{itemize}
\item {Proveniência:(De \textunderscore semi...\textunderscore  + \textunderscore histórico\textunderscore )}
\end{itemize}
Que contém factos históricos, entremeados de factos imaginários ou romanescos.
\section{Semiomem}
\begin{itemize}
\item {Grp. gram.:m.}
\end{itemize}
\begin{itemize}
\item {Proveniência:(Do lat. \textunderscore semihomo\textunderscore )}
\end{itemize}
Sêr imaginário, cuja metade é de homem.
\section{Semiimproviso}
\begin{itemize}
\item {Grp. gram.:m.}
\end{itemize}
\begin{itemize}
\item {Proveniência:(De \textunderscore semi...\textunderscore  + \textunderscore improviso\textunderscore )}
\end{itemize}
Discurso ou composição quási improvisada. Cf. Castilho, \textunderscore Fausto\textunderscore , p. X.
\section{Semiinternato}
\begin{itemize}
\item {Grp. gram.:m.}
\end{itemize}
Estado do que é semi-interno.
Estabelecimento escolar, cujos alumnos são semiinternos.
\section{Semiinterno}
\begin{itemize}
\item {Grp. gram.:m.  e  adj.}
\end{itemize}
Diz-se do alumno, que está no collégio somente durante o dia, tomando aí alguma ou algumas refeições.
\section{Semilha}
\begin{itemize}
\item {Grp. gram.:f.}
\end{itemize}
Nome, que na Madeira se dá á batata.
(Cast. \textunderscore semilla\textunderscore )
\section{Semilhos}
\begin{itemize}
\item {Grp. gram.:m. pl.}
\end{itemize}
\begin{itemize}
\item {Utilização:T. de Miranda}
\end{itemize}
O mesmo que \textunderscore alcacel\textunderscore ^1.
\section{Semiloiro}
\begin{itemize}
\item {Grp. gram.:adj.}
\end{itemize}
\begin{itemize}
\item {Proveniência:(De \textunderscore semi...\textunderscore  + \textunderscore loiro\textunderscore )}
\end{itemize}
Um tanto loiro. Cf. Lapa, \textunderscore Phýsica\textunderscore , 63.
\section{Semilouco}
\begin{itemize}
\item {Grp. gram.:adj.}
\end{itemize}
\begin{itemize}
\item {Proveniência:(De \textunderscore semi...\textunderscore  + \textunderscore louco\textunderscore )}
\end{itemize}
Meio louco.
\section{Semilouro}
\begin{itemize}
\item {Grp. gram.:adj.}
\end{itemize}
\begin{itemize}
\item {Proveniência:(De \textunderscore semi...\textunderscore  + \textunderscore loiro\textunderscore )}
\end{itemize}
Um tanto loiro. Cf. Lapa, \textunderscore Phýsica\textunderscore , 63.
\section{Semilunar}
\begin{itemize}
\item {Grp. gram.:adj.}
\end{itemize}
\begin{itemize}
\item {Grp. gram.:M.}
\end{itemize}
\begin{itemize}
\item {Utilização:Anat.}
\end{itemize}
\begin{itemize}
\item {Proveniência:(De \textunderscore semi...\textunderscore  + \textunderscore lunar\textunderscore )}
\end{itemize}
Que tem fórma de meia lua.
Um dos ossos do corpo.
\section{Semilunático}
\begin{itemize}
\item {Grp. gram.:adj.}
\end{itemize}
\begin{itemize}
\item {Proveniência:(Do lat. \textunderscore semilunaticus\textunderscore )}
\end{itemize}
Meio lunático; quási tolo.
\section{Semilúnio}
\begin{itemize}
\item {Grp. gram.:m.}
\end{itemize}
\begin{itemize}
\item {Proveniência:(Do lat. \textunderscore semis\textunderscore  + \textunderscore luna\textunderscore )}
\end{itemize}
Metade de uma revolução da Lua.
\section{Semimédico}
\begin{itemize}
\item {Grp. gram.:m.}
\end{itemize}
\begin{itemize}
\item {Proveniência:(De \textunderscore semi...\textunderscore  + \textunderscore médico\textunderscore )}
\end{itemize}
Meio médico.
Homem, que exerce a Medicina, sem graduação scientífica.
\section{Semimembranoso}
\begin{itemize}
\item {Grp. gram.:adj.}
\end{itemize}
\begin{itemize}
\item {Utilização:Anat.}
\end{itemize}
Diz-se de um músculo, situado na parte posterior da coxa.
\section{Semimetal}
\begin{itemize}
\item {Grp. gram.:m.}
\end{itemize}
\begin{itemize}
\item {Proveniência:(De \textunderscore semi...\textunderscore  + \textunderscore metal\textunderscore )}
\end{itemize}
Substância mineral, menos pesada e menos sólida que o metal.
\section{Semimorte}
\begin{itemize}
\item {Grp. gram.:m.}
\end{itemize}
\begin{itemize}
\item {Proveniência:(De \textunderscore semi...\textunderscore  + \textunderscore morte\textunderscore )}
\end{itemize}
Estado de moribundo.
\section{Semimorto}
\begin{itemize}
\item {Grp. gram.:adj.}
\end{itemize}
\begin{itemize}
\item {Utilização:Fig.}
\end{itemize}
\begin{itemize}
\item {Proveniência:(Do lat. \textunderscore simimortuns\textunderscore )}
\end{itemize}
Quási morto.
Amortecido: \textunderscore luz semimorta\textunderscore .
\section{Seminação}
\begin{itemize}
\item {Grp. gram.:f.}
\end{itemize}
\begin{itemize}
\item {Utilização:Bot.}
\end{itemize}
\begin{itemize}
\item {Utilização:Med.}
\end{itemize}
\begin{itemize}
\item {Proveniência:(Do lat. \textunderscore seminatio\textunderscore )}
\end{itemize}
Dispersão natural das sementes de uma planta.
O mesmo que \textunderscore cóito\textunderscore .
\section{Seminal}
\begin{itemize}
\item {Grp. gram.:adj.}
\end{itemize}
\begin{itemize}
\item {Utilização:Fig.}
\end{itemize}
\begin{itemize}
\item {Proveniência:(Lat. \textunderscore seminalis\textunderscore )}
\end{itemize}
Relativo a semente ou a sêmen.
Productivo.
\section{Seminário}
\begin{itemize}
\item {Grp. gram.:m.}
\end{itemize}
\begin{itemize}
\item {Utilização:Fig.}
\end{itemize}
\begin{itemize}
\item {Proveniência:(Lat. \textunderscore seminarius\textunderscore )}
\end{itemize}
Viveiro vegetal.
Centro de criação ou producção.
Estabelecimento escolar, que habilita para o estado ecclesiástico.
\section{Seminarista}
\begin{itemize}
\item {Grp. gram.:m.}
\end{itemize}
Alumno interno de um seminário.
\section{Seminarístico}
\begin{itemize}
\item {Grp. gram.:adj.}
\end{itemize}
Relativo a seminarista.
\section{Seminata}
\begin{itemize}
\item {Grp. gram.:f.}
\end{itemize}
\begin{itemize}
\item {Utilização:T. de Portalegre}
\end{itemize}
O mesmo que \textunderscore sarrabulho\textunderscore .
\section{Seminífero}
\begin{itemize}
\item {Grp. gram.:adj.}
\end{itemize}
\begin{itemize}
\item {Utilização:Bot.}
\end{itemize}
\begin{itemize}
\item {Proveniência:(Do lat. \textunderscore semen\textunderscore  + \textunderscore ferre\textunderscore )}
\end{itemize}
Que tem sementes, ou que produz sêmen.
Diz-se dos septos das válvulas, quando os grãos adherem a êstes, como succede nas nympheáceas, gencianáceas, etc.
\section{Semínima}
\begin{itemize}
\item {Grp. gram.:f.}
\end{itemize}
\begin{itemize}
\item {Proveniência:(De \textunderscore semi...\textunderscore  + \textunderscore mínima\textunderscore )}
\end{itemize}
Nota músical, de valor de metade de uma mínima.
\section{Semininfa}
\begin{itemize}
\item {Grp. gram.:f.}
\end{itemize}
\begin{itemize}
\item {Utilização:Zool.}
\end{itemize}
Ninfa, que difere pouco do insecto perfeito.
\section{Seminivérbio}
\begin{itemize}
\item {Grp. gram.:m.}
\end{itemize}
\begin{itemize}
\item {Proveniência:(Lat. \textunderscore seminiverbius\textunderscore )}
\end{itemize}
Semeador de palavras; discursador. Cf. Filinto, XIV, 281.
\section{Semino}
\begin{itemize}
\item {Grp. gram.:m.}
\end{itemize}
Cada uma das bóias, que sustêm certas rêdes de pesca.
\section{Seminota}
\begin{itemize}
\item {Grp. gram.:f.}
\end{itemize}
Gênero de insectos hymenópteros.
\section{Seminu}
\begin{itemize}
\item {Grp. gram.:adj.}
\end{itemize}
\begin{itemize}
\item {Proveniência:(Do lat. \textunderscore seminudus\textunderscore )}
\end{itemize}
Meio nu; andrajoso.
\section{Seminudez}
\begin{itemize}
\item {Grp. gram.:f.}
\end{itemize}
\begin{itemize}
\item {Proveniência:(De \textunderscore semi...\textunderscore  + \textunderscore nudez\textunderscore )}
\end{itemize}
Estado de seminu. Cf. Benalcanfor, \textunderscore Cartas de Lisb.\textunderscore , 111.
\section{Semínula}
\begin{itemize}
\item {Grp. gram.:f.}
\end{itemize}
\begin{itemize}
\item {Utilização:Bot.}
\end{itemize}
O mesmo ou melhor que \textunderscore semínulo\textunderscore . Cf. A. A. F. Benevides, \textunderscore Gloss. Bot.\textunderscore 
\section{Seminulífero}
\begin{itemize}
\item {Grp. gram.:adj.}
\end{itemize}
\begin{itemize}
\item {Proveniência:(De \textunderscore semínulo\textunderscore  + lat. \textunderscore ferre\textunderscore )}
\end{itemize}
Que tem semínulos.
\section{Semínulo}
\begin{itemize}
\item {Grp. gram.:m.}
\end{itemize}
Pequena semente; espóro.
(Dem. de \textunderscore sêmen\textunderscore )
\section{Seminume}
\begin{itemize}
\item {Grp. gram.:m.}
\end{itemize}
\begin{itemize}
\item {Proveniência:(De \textunderscore semi...\textunderscore  + \textunderscore nume\textunderscore )}
\end{itemize}
O mesmo que \textunderscore semideus\textunderscore . Cf. Castilho, \textunderscore Fastos\textunderscore , I, p. XI.
\section{Seminympha}
\begin{itemize}
\item {Grp. gram.:f.}
\end{itemize}
\begin{itemize}
\item {Utilização:Zool.}
\end{itemize}
Nympha, que differe pouco do insecto perfeito.
\section{Semiofficial}
\begin{itemize}
\item {Grp. gram.:adj.}
\end{itemize}
\begin{itemize}
\item {Proveniência:(De \textunderscore semi...\textunderscore  + \textunderscore official\textunderscore )}
\end{itemize}
Quási official.
\section{Semioficial}
\begin{itemize}
\item {Grp. gram.:adj.}
\end{itemize}
\begin{itemize}
\item {Proveniência:(De \textunderscore semi...\textunderscore  + \textunderscore oficial\textunderscore )}
\end{itemize}
Quási oficial.
\section{Semiófora}
\begin{itemize}
\item {Grp. gram.:f.}
\end{itemize}
Gênero de insectos lepidópteros.
(Cp. \textunderscore semióforo\textunderscore )
\section{Semióforo}
\begin{itemize}
\item {Grp. gram.:m.}
\end{itemize}
\begin{itemize}
\item {Proveniência:(Do gr. \textunderscore semeion\textunderscore  + \textunderscore phoros\textunderscore )}
\end{itemize}
Oficial inferior, no antigo exército grego.
Gênero de peixes.
\section{Semiografia}
\begin{itemize}
\item {Grp. gram.:f.}
\end{itemize}
\begin{itemize}
\item {Proveniência:(Do gr. \textunderscore semeion\textunderscore  + \textunderscore graphein\textunderscore )}
\end{itemize}
Representação por meio de sinaes; notação.
\section{Semiographia}
\begin{itemize}
\item {Grp. gram.:f.}
\end{itemize}
\begin{itemize}
\item {Proveniência:(Do gr. \textunderscore semeion\textunderscore  + \textunderscore graphein\textunderscore )}
\end{itemize}
Representação por meio de sinaes; notação.
\section{Semioitava}
\begin{itemize}
\item {Grp. gram.:f.}
\end{itemize}
\begin{itemize}
\item {Utilização:Des.}
\end{itemize}
\begin{itemize}
\item {Proveniência:(De \textunderscore semi...\textunderscore  + \textunderscore oitava\textunderscore )}
\end{itemize}
Quatro versos consonantes, que constituem a primeira metade de uma oitava.
\section{Semiologia}
\begin{itemize}
\item {Grp. gram.:f.}
\end{itemize}
\begin{itemize}
\item {Utilização:Philol.}
\end{itemize}
\begin{itemize}
\item {Proveniência:(Do gr. \textunderscore semeion\textunderscore  + \textunderscore logos\textunderscore )}
\end{itemize}
Tratado dos symptomas das doenças.
Estudo das trasladações ou mudanças que no espaço e no tempo experimenta a significação das palavras, consideradas como sinaes das ideias.
\section{Semiológico}
\begin{itemize}
\item {Grp. gram.:adj.}
\end{itemize}
Relativo á semiologia.
\section{Semióphora}
\begin{itemize}
\item {Grp. gram.:f.}
\end{itemize}
Gênero de insectos lepidópteros.
(Cp. \textunderscore semióphoro\textunderscore )
\section{Semióphoro}
\begin{itemize}
\item {Grp. gram.:m.}
\end{itemize}
\begin{itemize}
\item {Proveniência:(Do gr. \textunderscore semeion\textunderscore  + \textunderscore phoros\textunderscore )}
\end{itemize}
Official inferior, no antigo exército grego.
Gênero de peixes.
\section{Semiorbe}
\begin{itemize}
\item {Grp. gram.:m.}
\end{itemize}
\begin{itemize}
\item {Utilização:Ant.}
\end{itemize}
\begin{itemize}
\item {Proveniência:(Lat. \textunderscore semi-orbis\textunderscore )}
\end{itemize}
O mesmo que \textunderscore semicírculo\textunderscore .
\section{Semiotechnia}
\begin{itemize}
\item {Grp. gram.:f.}
\end{itemize}
\begin{itemize}
\item {Proveniência:(Do gr. \textunderscore semeion\textunderscore  + \textunderscore tekhne\textunderscore )}
\end{itemize}
Tratado de sinaes gráphicos.
\textunderscore Semiotechnia músical\textunderscore , conhecimento dos sinaes gráphicos da música.
\section{Semiotecnia}
\begin{itemize}
\item {Grp. gram.:f.}
\end{itemize}
\begin{itemize}
\item {Proveniência:(Do gr. \textunderscore semeion\textunderscore  + \textunderscore tekhne\textunderscore )}
\end{itemize}
Tratado de sinaes gráficos.
\textunderscore Semiotecnia músical\textunderscore , conhecimento dos sinaes gráficos da música.
\section{Semiótica}
\begin{itemize}
\item {Grp. gram.:f.}
\end{itemize}
\begin{itemize}
\item {Proveniência:(Gr. \textunderscore semeiotike\textunderscore )}
\end{itemize}
O mesmo que \textunderscore semiologia\textunderscore .
Arte de dirigir manobras militares por meio de sinaes, em vez da voz.
\section{Semioto}
\begin{itemize}
\item {Grp. gram.:m.}
\end{itemize}
\begin{itemize}
\item {Proveniência:(Do gr. \textunderscore semeiotos\textunderscore )}
\end{itemize}
Gênero de insectos coleópteros.
\section{Semioval}
\begin{itemize}
\item {Grp. gram.:adj.}
\end{itemize}
\begin{itemize}
\item {Utilização:Bot.}
\end{itemize}
\begin{itemize}
\item {Proveniência:(De \textunderscore semi...\textunderscore  + \textunderscore oval\textunderscore )}
\end{itemize}
Diz-se das estípulas, que são ovaes na sua metade.
\section{Semipagão}
\begin{itemize}
\item {Grp. gram.:m.}
\end{itemize}
\begin{itemize}
\item {Utilização:Poét.}
\end{itemize}
\begin{itemize}
\item {Proveniência:(Do lat. \textunderscore semipaganus\textunderscore )}
\end{itemize}
Indivíduo meio rústico e meio civilizado.
Um tanto profano.
\section{Semiparente}
\begin{itemize}
\item {Grp. gram.:adj.}
\end{itemize}
\begin{itemize}
\item {Proveniência:(De \textunderscore semi...\textunderscore  + \textunderscore parente\textunderscore )}
\end{itemize}
Diz-se de quem tem algum parentesco com outrem.
\section{Semipedal}
\begin{itemize}
\item {Grp. gram.:adj.}
\end{itemize}
\begin{itemize}
\item {Proveniência:(Lat. \textunderscore semipedalis\textunderscore )}
\end{itemize}
Que tem meio pé de comprimento.
\section{Semipelagianismo}
\begin{itemize}
\item {Grp. gram.:m.}
\end{itemize}
Doutrina dos Semipelagianos.
\section{Semipelagianos}
\begin{itemize}
\item {Grp. gram.:m. pl.}
\end{itemize}
Herejes que, no século V, procuraram conciliar o pelagianismo com a orthodoxia.
\section{Semiperiferia}
\begin{itemize}
\item {Grp. gram.:f.}
\end{itemize}
\begin{itemize}
\item {Proveniência:(De \textunderscore semi...\textunderscore  + \textunderscore periferia\textunderscore )}
\end{itemize}
Metade de uma periferia.
\section{Semiperipheria}
\begin{itemize}
\item {Grp. gram.:f.}
\end{itemize}
\begin{itemize}
\item {Proveniência:(De \textunderscore semi...\textunderscore  + \textunderscore peripheria\textunderscore )}
\end{itemize}
Metade de uma peripheria.
\section{Semipermeável}
\begin{itemize}
\item {Grp. gram.:adj.}
\end{itemize}
\begin{itemize}
\item {Proveniência:(De \textunderscore semi...\textunderscore  + \textunderscore permeável\textunderscore )}
\end{itemize}
Um tanto permeável.--É t. pouco correcto, usado por alguns chímicos. Cf. Ach. Machado, \textunderscore Propr. Colligativas\textunderscore .
\section{Semíphoro}
\begin{itemize}
\item {Grp. gram.:m.}
\end{itemize}
\begin{itemize}
\item {Utilização:Des.}
\end{itemize}
\begin{itemize}
\item {Proveniência:(Do gr. \textunderscore semeion\textunderscore  + \textunderscore phoros\textunderscore )}
\end{itemize}
O mesmo que \textunderscore porta-bandeira\textunderscore .
\section{Semiplenamente}
\begin{itemize}
\item {Grp. gram.:adv.}
\end{itemize}
De modo semipleno; incompletamente.
\section{Semipleno}
\begin{itemize}
\item {Grp. gram.:adj.}
\end{itemize}
\begin{itemize}
\item {Utilização:Fig.}
\end{itemize}
\begin{itemize}
\item {Proveniência:(Lat. \textunderscore semiplenus\textunderscore )}
\end{itemize}
Cheio até ao meio.
Incompleto.
\section{Semipoéta}
\begin{itemize}
\item {Grp. gram.:m.}
\end{itemize}
\begin{itemize}
\item {Utilização:Deprec.}
\end{itemize}
\begin{itemize}
\item {Proveniência:(De \textunderscore semi...\textunderscore  + \textunderscore poéta\textunderscore )}
\end{itemize}
Mau poéta; poéta medíocre.
\section{Semiprova}
\begin{itemize}
\item {Grp. gram.:f.}
\end{itemize}
\begin{itemize}
\item {Proveniência:(De \textunderscore semi...\textunderscore  + \textunderscore prova\textunderscore )}
\end{itemize}
Prova incompleta.
\section{Semiputo}
\begin{itemize}
\item {Grp. gram.:m.}
\end{itemize}
\begin{itemize}
\item {Utilização:T. de Coímbra}
\end{itemize}
Estudante do segundo anno de qualquer faculdade da Universidade; segundannista.
\section{Semipútrido}
\begin{itemize}
\item {Grp. gram.:adj.}
\end{itemize}
\begin{itemize}
\item {Proveniência:(De \textunderscore semi...\textunderscore  + \textunderscore pútrido\textunderscore )}
\end{itemize}
Meio podre; que começa a corromper-se.
\section{Semiquadrado}
\begin{itemize}
\item {Grp. gram.:adj.}
\end{itemize}
\begin{itemize}
\item {Proveniência:(De \textunderscore semi...\textunderscore  + \textunderscore quadrado\textunderscore )}
\end{itemize}
Diz-se do aspecto de dois planetas separados entre si 45°.
\section{Semiracional}
\begin{itemize}
\item {fónica:ra}
\end{itemize}
\begin{itemize}
\item {Grp. gram.:adj.}
\end{itemize}
\begin{itemize}
\item {Proveniência:(De \textunderscore semi...\textunderscore  + \textunderscore racional\textunderscore )}
\end{itemize}
Muito estúpido.
\section{Semiradiado}
\begin{itemize}
\item {fónica:ra}
\end{itemize}
\begin{itemize}
\item {Grp. gram.:adj.}
\end{itemize}
\begin{itemize}
\item {Utilização:Bot.}
\end{itemize}
\begin{itemize}
\item {Proveniência:(De \textunderscore semi...\textunderscore  + \textunderscore radiar\textunderscore )}
\end{itemize}
Que tem meia corôa radiante.
\section{Semirâmio}
\begin{itemize}
\item {Grp. gram.:adj.}
\end{itemize}
\begin{itemize}
\item {Proveniência:(Lat. \textunderscore semiramius\textunderscore )}
\end{itemize}
Relativo a Semiramis.
\section{Semirecto}
\begin{itemize}
\item {fónica:ré}
\end{itemize}
\begin{itemize}
\item {Grp. gram.:adj.}
\end{itemize}
\begin{itemize}
\item {Proveniência:(De \textunderscore semi...\textunderscore  + \textunderscore recto\textunderscore )}
\end{itemize}
Meio recto.
Igual a metade do que é recto, (falando-se de ângulos).
\section{Semi-Rei}
\begin{itemize}
\item {Grp. gram.:m.}
\end{itemize}
Indivíduo, que tem quási os poderes de Rei. Cf. Th. Ribeiro, \textunderscore Jornadas\textunderscore , I, 149.
\section{Semiroliço}
\begin{itemize}
\item {fónica:ro}
\end{itemize}
\begin{itemize}
\item {Grp. gram.:adj.}
\end{itemize}
\begin{itemize}
\item {Utilização:Bot.}
\end{itemize}
O mesmo que \textunderscore semicylíndrico\textunderscore .
\section{Semiroto}
\begin{itemize}
\item {fónica:rô}
\end{itemize}
\begin{itemize}
\item {Grp. gram.:adj.}
\end{itemize}
\begin{itemize}
\item {Proveniência:(De \textunderscore semi...\textunderscore  + \textunderscore roto\textunderscore )}
\end{itemize}
Meio roto; meio partido.
\section{Semirracional}
\begin{itemize}
\item {Grp. gram.:adj.}
\end{itemize}
\begin{itemize}
\item {Proveniência:(De \textunderscore semi...\textunderscore  + \textunderscore racional\textunderscore )}
\end{itemize}
Muito estúpido.
\section{Semirradiado}
\begin{itemize}
\item {Grp. gram.:adj.}
\end{itemize}
\begin{itemize}
\item {Utilização:Bot.}
\end{itemize}
\begin{itemize}
\item {Proveniência:(De \textunderscore semi...\textunderscore  + \textunderscore radiar\textunderscore )}
\end{itemize}
Que tem meia corôa radiante.
\section{Semirrecto}
\begin{itemize}
\item {Grp. gram.:adj.}
\end{itemize}
\begin{itemize}
\item {Proveniência:(De \textunderscore semi...\textunderscore  + \textunderscore recto\textunderscore )}
\end{itemize}
Meio recto.
Igual a metade do que é recto, (falando-se de ângulos).
\section{Semirroliço}
\begin{itemize}
\item {Grp. gram.:adj.}
\end{itemize}
\begin{itemize}
\item {Utilização:Bot.}
\end{itemize}
O mesmo que \textunderscore semicilíndrico\textunderscore .
\section{Semirroto}
\begin{itemize}
\item {Grp. gram.:adj.}
\end{itemize}
\begin{itemize}
\item {Proveniência:(De \textunderscore semi...\textunderscore  + \textunderscore roto\textunderscore )}
\end{itemize}
Meio roto; meio partido.
\section{Sêmis}
\begin{itemize}
\item {Grp. gram.:f.}
\end{itemize}
\begin{itemize}
\item {Proveniência:(Lat. \textunderscore semis\textunderscore )}
\end{itemize}
O mesmo que \textunderscore metade\textunderscore . Cf. Castilho, \textunderscore Fastos\textunderscore , I, 387 e 389.
\section{Semisábio}
\begin{itemize}
\item {fónica:sá}
\end{itemize}
\begin{itemize}
\item {Grp. gram.:adj.}
\end{itemize}
\begin{itemize}
\item {Proveniência:(De \textunderscore semi...\textunderscore  + \textunderscore sábio\textunderscore )}
\end{itemize}
Que tem sciência incompleta.
Que fala de tudo e sabe pouco.
\section{Semiscarúnfio}
\begin{itemize}
\item {Grp. gram.:adj.}
\end{itemize}
\begin{itemize}
\item {Utilização:Pop.}
\end{itemize}
\begin{itemize}
\item {Utilização:Prov.}
\end{itemize}
Esquisito.
Feio.
Reles.
Mal disposto.
Tristonho.
\section{Semisecular}
\begin{itemize}
\item {fónica:se}
\end{itemize}
\begin{itemize}
\item {Grp. gram.:adj.}
\end{itemize}
\begin{itemize}
\item {Proveniência:(De \textunderscore semi...\textunderscore  + \textunderscore secular\textunderscore )}
\end{itemize}
Que tem meio século.
\section{Semisegredo}
\begin{itemize}
\item {fónica:se,grê}
\end{itemize}
\begin{itemize}
\item {Grp. gram.:m.}
\end{itemize}
\begin{itemize}
\item {Proveniência:(De \textunderscore semi...\textunderscore  + \textunderscore segrêdo\textunderscore )}
\end{itemize}
Meio segrêdo. Cf. Castilho, \textunderscore Metam.\textunderscore , I.
\section{Semiselvagem}
\begin{itemize}
\item {fónica:sel}
\end{itemize}
\begin{itemize}
\item {Grp. gram.:adj.}
\end{itemize}
Quási selvagem; brutal; muito rude.
\section{Semisfera}
\begin{itemize}
\item {Grp. gram.:f.}
\end{itemize}
O mesmo que \textunderscore semiesfera\textunderscore .
\section{Semisfério}
\begin{itemize}
\item {Grp. gram.:m.}
\end{itemize}
\begin{itemize}
\item {Utilização:Des.}
\end{itemize}
\begin{itemize}
\item {Proveniência:(Lat. \textunderscore semisphaerium\textunderscore )}
\end{itemize}
O mesmo que \textunderscore hemisfério\textunderscore .
Semiesfera.
\section{Semisilvestre}
\begin{itemize}
\item {fónica:sil}
\end{itemize}
\begin{itemize}
\item {Grp. gram.:adj.}
\end{itemize}
\begin{itemize}
\item {Proveniência:(De \textunderscore semi...\textunderscore  + \textunderscore silvestre\textunderscore )}
\end{itemize}
Um tanto silvestre. Cf. Castilho, \textunderscore Fastos\textunderscore , II, 477.
\section{Semispata}
\begin{itemize}
\item {Grp. gram.:f.}
\end{itemize}
Espécie de espada curta, usada pelos Godos. Cf. C. Aires, \textunderscore Hist. do Exérc. Port.\textunderscore 
(B. lat. \textunderscore semispatha\textunderscore )
\section{Semispatha}
\begin{itemize}
\item {Grp. gram.:f.}
\end{itemize}
Espécie de espada curta, usada pelos Godos. Cf. C. Aires, \textunderscore Hist. do Exérc. Port.\textunderscore 
(B. lat. \textunderscore semispatha\textunderscore )
\section{Semisphera}
\begin{itemize}
\item {Grp. gram.:f.}
\end{itemize}
O mesmo que \textunderscore semiesphera\textunderscore .
\section{Semisphério}
\begin{itemize}
\item {Grp. gram.:m.}
\end{itemize}
\begin{itemize}
\item {Utilização:Des.}
\end{itemize}
\begin{itemize}
\item {Proveniência:(Lat. \textunderscore semisphaerium\textunderscore )}
\end{itemize}
O mesmo que \textunderscore hemisphério\textunderscore .
Semiesphera.
\section{Semissábio}
\begin{itemize}
\item {Grp. gram.:adj.}
\end{itemize}
\begin{itemize}
\item {Proveniência:(De \textunderscore semi...\textunderscore  + \textunderscore sábio\textunderscore )}
\end{itemize}
Que tem sciência incompleta.
Que fala de tudo e sabe pouco.
\section{Semisse}
\begin{itemize}
\item {Grp. gram.:m.}
\end{itemize}
\begin{itemize}
\item {Proveniência:(Lat. \textunderscore semissis\textunderscore )}
\end{itemize}
Metade de um asse ou 5 onças, entre os antigos Romanos.
\section{Semissecular}
\begin{itemize}
\item {Grp. gram.:adj.}
\end{itemize}
\begin{itemize}
\item {Proveniência:(De \textunderscore semi...\textunderscore  + \textunderscore secular\textunderscore )}
\end{itemize}
Que tem meio século.
\section{Semissegredo}
\begin{itemize}
\item {fónica:grê}
\end{itemize}
\begin{itemize}
\item {Grp. gram.:m.}
\end{itemize}
\begin{itemize}
\item {Proveniência:(De \textunderscore semi...\textunderscore  + \textunderscore segrêdo\textunderscore )}
\end{itemize}
Meio segrêdo. Cf. Castilho, \textunderscore Metam.\textunderscore , I.
\section{Semisselvagem}
\begin{itemize}
\item {Grp. gram.:adj.}
\end{itemize}
Quási selvagem; brutal; muito rude.
\section{Semissilvestre}
\begin{itemize}
\item {Grp. gram.:adj.}
\end{itemize}
\begin{itemize}
\item {Proveniência:(De \textunderscore semi...\textunderscore  + \textunderscore silvestre\textunderscore )}
\end{itemize}
Um tanto silvestre. Cf. Castilho, \textunderscore Fastos\textunderscore , II, 477.
\section{Semistaminar}
\begin{itemize}
\item {Grp. gram.:adj.}
\end{itemize}
O mesmo que \textunderscore semistaminário\textunderscore .
\section{Semistaminário}
\begin{itemize}
\item {Grp. gram.:adj.}
\end{itemize}
\begin{itemize}
\item {Utilização:Bot.}
\end{itemize}
\begin{itemize}
\item {Proveniência:(De \textunderscore semi...\textunderscore  + \textunderscore estame\textunderscore )}
\end{itemize}
Diz-se das flôres dobradas, em que só uma porção dos estames se transforma em pétalas.
\section{Semita}
\begin{itemize}
\item {Grp. gram.:m.}
\end{itemize}
Homem, pertencente a uma raça, que se diz oriunda de Sem.
\section{Sêmita}
\begin{itemize}
\item {Grp. gram.:f.}
\end{itemize}
\begin{itemize}
\item {Proveniência:(Lat. \textunderscore semita\textunderscore )}
\end{itemize}
O mesmo que \textunderscore senda\textunderscore .
\section{Semiterçan}
\begin{itemize}
\item {Grp. gram.:adj.}
\end{itemize}
Diz-se da febre quotidiana, com um accesso mais intenso em dias alternados.
\section{Semítico}
\begin{itemize}
\item {Grp. gram.:adj.}
\end{itemize}
Relativo aos Semitas.
\section{Semitismo}
\begin{itemize}
\item {Grp. gram.:m.}
\end{itemize}
\begin{itemize}
\item {Proveniência:(De \textunderscore semita\textunderscore )}
\end{itemize}
Carácter do que é semítico.
Civilização semítica ou sua influência.
\section{Semitom}
\begin{itemize}
\item {Grp. gram.:m.}
\end{itemize}
\begin{itemize}
\item {Proveniência:(De \textunderscore semi...\textunderscore  + \textunderscore tom\textunderscore )}
\end{itemize}
Meio tom, na música.
\section{Semitonado}
\begin{itemize}
\item {Grp. gram.:adj.}
\end{itemize}
\begin{itemize}
\item {Utilização:Mús.}
\end{itemize}
\begin{itemize}
\item {Proveniência:(De \textunderscore semitom\textunderscore )}
\end{itemize}
Que procede por meios tons ou que pertence ao gênero chromático.
\section{Semítono}
\begin{itemize}
\item {Grp. gram.:m.}
\end{itemize}
O mesmo que \textunderscore semitom\textunderscore .
\section{Semitransparente}
\begin{itemize}
\item {Grp. gram.:adj.}
\end{itemize}
\begin{itemize}
\item {Proveniência:(De \textunderscore semi...\textunderscore  + \textunderscore transparente\textunderscore )}
\end{itemize}
Um tanto transparente.
\section{Semiústo}
\begin{itemize}
\item {Grp. gram.:adj.}
\end{itemize}
\begin{itemize}
\item {Utilização:Poét.}
\end{itemize}
\begin{itemize}
\item {Proveniência:(Lat. \textunderscore semiustus\textunderscore )}
\end{itemize}
Um tanto queimado.
\section{Semiverdade}
\begin{itemize}
\item {Grp. gram.:f.}
\end{itemize}
\begin{itemize}
\item {Proveniência:(De \textunderscore semi...\textunderscore  + \textunderscore verdade\textunderscore )}
\end{itemize}
Um tanto de verdade. Cf. Castilho, \textunderscore Metam.\textunderscore , 283.
\section{Semíviro}
\begin{itemize}
\item {Grp. gram.:m.}
\end{itemize}
\begin{itemize}
\item {Proveniência:(Do lat. \textunderscore semi-vir\textunderscore )}
\end{itemize}
Homem imperfeito; eunuco. Cf. Camões, \textunderscore ode\textunderscore  VII.
\section{Semivítreo}
\begin{itemize}
\item {Grp. gram.:adj.}
\end{itemize}
Meio vítreo.
\section{Semi-viver}
\begin{itemize}
\item {Grp. gram.:v. i.}
\end{itemize}
\begin{itemize}
\item {Proveniência:(De \textunderscore semi...\textunderscore  + \textunderscore viver\textunderscore )}
\end{itemize}
Viver incompletamente:«\textunderscore convem volver o espírito saudoso e semiviver de imagens.\textunderscore »Castilho, \textunderscore Escav. Poét.\textunderscore , 12.
\section{Semivivo}
\begin{itemize}
\item {Grp. gram.:adj.}
\end{itemize}
\begin{itemize}
\item {Proveniência:(Lat. \textunderscore semivivus\textunderscore )}
\end{itemize}
Quási sem vida; semimorto.
\section{Semivogal}
\begin{itemize}
\item {Grp. gram.:adj.}
\end{itemize}
\begin{itemize}
\item {Utilização:Gram.}
\end{itemize}
\begin{itemize}
\item {Proveniência:(De \textunderscore semi...\textunderscore  + \textunderscore vogal\textunderscore )}
\end{itemize}
Designação de algumas vogaes, como o \textunderscore i\textunderscore  e o \textunderscore u\textunderscore  em \textunderscore maior\textunderscore  e \textunderscore água\textunderscore .
\section{Sem-justiça}
\begin{itemize}
\item {Grp. gram.:f.}
\end{itemize}
Acto injusto; iniquidade.
\section{Sem-luz}
\begin{itemize}
\item {Grp. gram.:m.}
\end{itemize}
Aquelle que \textunderscore não\textunderscore  vê; aquelle que vive nas trevas:«\textunderscore ...as sombras dos sem-luz.\textunderscore »Castilho, \textunderscore Geórg.\textunderscore 
\section{Sémnio}
\begin{itemize}
\item {Grp. gram.:m.}
\end{itemize}
\begin{itemize}
\item {Proveniência:(Do gr. \textunderscore semnion\textunderscore )}
\end{itemize}
Planta odorífera, espécie de junco.
\section{Sem-nome}
\begin{itemize}
\item {Grp. gram.:m. ,  f.  e  adj.}
\end{itemize}
\begin{itemize}
\item {Grp. gram.:F.}
\end{itemize}
Pessôa anónyma.
Variedade de uva, também chamada \textunderscore janeanes\textunderscore .
\section{Semnopiteco}
\begin{itemize}
\item {Grp. gram.:m.}
\end{itemize}
\begin{itemize}
\item {Proveniência:(Do gr. \textunderscore semno\textunderscore  + \textunderscore pithekos\textunderscore )}
\end{itemize}
Gênero de mammíferos quadrúmanos.
\section{Semnopitheco}
\begin{itemize}
\item {Grp. gram.:m.}
\end{itemize}
\begin{itemize}
\item {Proveniência:(Do gr. \textunderscore semno\textunderscore  + \textunderscore pithekos\textunderscore )}
\end{itemize}
Gênero de mammíferos quadrúmanos.
\section{Sem-número}
\begin{itemize}
\item {Grp. gram.:adj.}
\end{itemize}
\begin{itemize}
\item {Grp. gram.:M.}
\end{itemize}
Que se não póde numerar; innumerável.
Grande número.
\section{Sêmola}
\begin{itemize}
\item {Grp. gram.:f.}
\end{itemize}
\begin{itemize}
\item {Proveniência:(It. \textunderscore semola\textunderscore )}
\end{itemize}
Fécula da farinha de arroz.
\section{Semólido}
\begin{itemize}
\item {Grp. gram.:m.}
\end{itemize}
\begin{itemize}
\item {Utilização:Zool.}
\end{itemize}
Gênero de peixes abdomianes.
\section{Semones}
\begin{itemize}
\item {Grp. gram.:m. pl.}
\end{itemize}
\begin{itemize}
\item {Proveniência:(Lat. \textunderscore semones\textunderscore )}
\end{itemize}
Deuses de categoria inferior, na antiga Itália. Cf. Castilho, \textunderscore Fastos\textunderscore , III, 559 e 568.
\section{Semoto}
\begin{itemize}
\item {Grp. gram.:adj.}
\end{itemize}
\begin{itemize}
\item {Utilização:Poét.}
\end{itemize}
\begin{itemize}
\item {Proveniência:(Lat. \textunderscore semotus\textunderscore )}
\end{itemize}
Afastado, distante.
\section{Semovente}
\begin{itemize}
\item {Grp. gram.:adj.}
\end{itemize}
\begin{itemize}
\item {Proveniência:(De \textunderscore se\textunderscore ^2 + \textunderscore movente\textunderscore )}
\end{itemize}
Que anda ou se move por si próprio.
\section{Sempar}
\begin{itemize}
\item {Grp. gram.:adj.}
\end{itemize}
\begin{itemize}
\item {Proveniência:(De \textunderscore sem\textunderscore  + \textunderscore par\textunderscore )}
\end{itemize}
Que é único, que não tem igual.
\section{Sempiternamente}
\begin{itemize}
\item {Grp. gram.:adv.}
\end{itemize}
De modo sempiterno; perpetuamente; sem princípio nem fim.
\section{Sempiterno}
\begin{itemize}
\item {Grp. gram.:adj.}
\end{itemize}
\begin{itemize}
\item {Proveniência:(Lat. \textunderscore sempiternus\textunderscore )}
\end{itemize}
Que dura sempre.
Perpétuo; que não teve princípio nem há de têr fim.
Antiquíssimo.
\section{Sempre}
\begin{itemize}
\item {Grp. gram.:adv.}
\end{itemize}
\begin{itemize}
\item {Grp. gram.:M.}
\end{itemize}
\begin{itemize}
\item {Proveniência:(Do lat. \textunderscore semper\textunderscore )}
\end{itemize}
Em todo o tempo.
Em toda a vida.
Em todo o momento; constantemente; sem interrupção.
Todavia.
Effectivamente, realmente, na verdade: \textunderscore sempre há cada pateta\textunderscore !
Todo tempo, passado ou futuro: \textunderscore desde todo o sempre...\textunderscore , \textunderscore para todo o sempre...\textunderscore 
\section{Sempre-noiva}
\begin{itemize}
\item {Grp. gram.:f.}
\end{itemize}
\begin{itemize}
\item {Utilização:Prov.}
\end{itemize}
\begin{itemize}
\item {Utilização:alg.}
\end{itemize}
O mesmo que \textunderscore sempre-viva\textunderscore .
Ornato de parede, formado por ladrilhos, na parte inferior da chaminé, em frente da lareira.
\section{Sempre-verde}
\begin{itemize}
\item {Grp. gram.:f.}
\end{itemize}
\begin{itemize}
\item {Utilização:Prov.}
\end{itemize}
\begin{itemize}
\item {Utilização:minh.}
\end{itemize}
O mesmo que \textunderscore sempre-viva\textunderscore .
O mesmo que \textunderscore sabugueiro\textunderscore .
\section{Sempre-viva}
\begin{itemize}
\item {Grp. gram.:f.}
\end{itemize}
Planta polygónea, o mesmo que \textunderscore sanguinária\textunderscore .
\section{Sem-razão}
\begin{itemize}
\item {Grp. gram.:f.}
\end{itemize}
\begin{itemize}
\item {Proveniência:(De \textunderscore sem\textunderscore  + \textunderscore razão\textunderscore )}
\end{itemize}
Acto ou conceito infundado; injustiça; afronta.
\section{Sem-sabor}
\begin{itemize}
\item {Grp. gram.:adj.}
\end{itemize}
\begin{itemize}
\item {Grp. gram.:M.  e  f.}
\end{itemize}
\begin{itemize}
\item {Proveniência:(De \textunderscore sem\textunderscore  + \textunderscore sabor\textunderscore )}
\end{itemize}
Que não tem sabor.
Insípido; monótono; desengraçado.
Pessôa sem-sabor.
\section{Sem-saborão}
\begin{itemize}
\item {Grp. gram.:m.  e  adj.}
\end{itemize}
\begin{itemize}
\item {Proveniência:(De \textunderscore sensabor\textunderscore )}
\end{itemize}
Indivíduo muito sem-sabor.
\section{Sem-saboria}
\begin{itemize}
\item {Grp. gram.:f.}
\end{itemize}
\begin{itemize}
\item {Utilização:Fam.}
\end{itemize}
Qualidade daquelle ou daquillo que é sem-sabor.
Acto ou acontecimento desagradável ou que causa ou póde causar desgôstos.
\section{Sem-sal}
\begin{itemize}
\item {Grp. gram.:adj.}
\end{itemize}
\begin{itemize}
\item {Utilização:Fig.}
\end{itemize}
Insulso.
Sem-saborão.
\section{Sem-segundo}
\begin{itemize}
\item {Grp. gram.:adj.}
\end{itemize}
Sem par; que não tem igual; único.
\section{Sem-termo}
\begin{itemize}
\item {Grp. gram.:m.}
\end{itemize}
O mesmo que \textunderscore sem-fim\textunderscore . Cf. Sous. Monteiro, \textunderscore Elog. de Lat.\textunderscore 
\section{Sem-tirte-nem-guarte}
\begin{itemize}
\item {Grp. gram.:loc. adv.}
\end{itemize}
De repente; sem aviso.
(Contr. de \textunderscore sem\textunderscore  + \textunderscore tira-te\textunderscore , \textunderscore nem\textunderscore  + \textunderscore guarda-te\textunderscore )
\section{Sena}
\begin{itemize}
\item {Grp. gram.:f.}
\end{itemize}
\begin{itemize}
\item {Grp. gram.:Pl.}
\end{itemize}
\begin{itemize}
\item {Proveniência:(Do lat. \textunderscore seni\textunderscore )}
\end{itemize}
Carta ou dado com seis pintas.
Peça do dominó, que apresenta duas senas.
\section{Sena}
\begin{itemize}
\item {Grp. gram.:f.}
\end{itemize}
Nome, com que alguns botânicos designam o sene. Cf. Dalgado, \textunderscore Flora\textunderscore , 61.
\section{Senábria}
\begin{itemize}
\item {Grp. gram.:f.}
\end{itemize}
\begin{itemize}
\item {Utilização:Prov.}
\end{itemize}
Trave que, no tecto, corre parallelamente entre a cumeeira e o frechal.
(Colhido em Vouzela)
\section{Senáculo}
\begin{itemize}
\item {Grp. gram.:m.}
\end{itemize}
\begin{itemize}
\item {Proveniência:(Lat. \textunderscore senaculum\textunderscore )}
\end{itemize}
Lugar ou praça, onde o senado romano celebrava as suas sessões.
\section{Senado}
\begin{itemize}
\item {Grp. gram.:m.}
\end{itemize}
\begin{itemize}
\item {Proveniência:(Do lat. \textunderscore senatus\textunderscore )}
\end{itemize}
Antiga magistratura romana, constituida por patrícios ou nobres.
Lugar, onde essa magistratura funcionava.
Câmara legislativa e hereditária ou de livre nomeação do chefe do Estado, em alguns países.
Câmara municipal.
Casa onde se reunem ou funccionam essas corporações.
Nalguns países, tribunal de última instância.
\section{Senador}
\begin{itemize}
\item {Grp. gram.:m.}
\end{itemize}
\begin{itemize}
\item {Proveniência:(Lat. \textunderscore senator\textunderscore )}
\end{itemize}
Membro do senado.
\section{Senadora}
\begin{itemize}
\item {Grp. gram.:f.}
\end{itemize}
\begin{itemize}
\item {Proveniência:(Do lat. \textunderscore senator\textunderscore )}
\end{itemize}
Designação de cada uma das mulheres de Heliogábalo. Cf. Rui Barb., \textunderscore Réplica\textunderscore , 158.
\section{Senal}
\begin{itemize}
\item {Grp. gram.:adj.}
\end{itemize}
\begin{itemize}
\item {Proveniência:(Do lat. \textunderscore seni\textunderscore ?)}
\end{itemize}
Diz-se do diamante inlapidado e muito pequeno.
\section{Senão}
\begin{itemize}
\item {Grp. gram.:conj.}
\end{itemize}
\begin{itemize}
\item {Grp. gram.:Prep.}
\end{itemize}
\begin{itemize}
\item {Grp. gram.:M.}
\end{itemize}
\begin{itemize}
\item {Proveniência:(De \textunderscore se\textunderscore ^1 + \textunderscore não\textunderscore )}
\end{itemize}
Aliás, quando não: \textunderscore toma juízo; senão, estás perdido\textunderscore .
Excepto: \textunderscore nenhum escapou, senão o mais velho\textunderscore .
Mácula, defeito: \textunderscore formosa sem senão\textunderscore .
\section{Senário}
\begin{itemize}
\item {Grp. gram.:adj.}
\end{itemize}
\begin{itemize}
\item {Proveniência:(Lat. \textunderscore senarius\textunderscore )}
\end{itemize}
Que consta de seis unidades.
Que tem seis pés, (falando-se de versos latinos).
\section{Senásqua}
\begin{itemize}
\item {Grp. gram.:f.}
\end{itemize}
Variedade de videira americana.
\section{Senatoria}
\begin{itemize}
\item {Grp. gram.:f.}
\end{itemize}
\begin{itemize}
\item {Utilização:bras}
\end{itemize}
\begin{itemize}
\item {Utilização:Neol.}
\end{itemize}
\begin{itemize}
\item {Proveniência:(Do lat. \textunderscore senator\textunderscore )}
\end{itemize}
Cargo de senador.
\section{Senatorial}
\begin{itemize}
\item {Grp. gram.:adj.}
\end{itemize}
\begin{itemize}
\item {Proveniência:(Lat. \textunderscore senatorius\textunderscore )}
\end{itemize}
Relativo ao senado.
\section{Senatório}
\begin{itemize}
\item {Grp. gram.:adj.}
\end{itemize}
\begin{itemize}
\item {Proveniência:(Lat. \textunderscore senatorius\textunderscore )}
\end{itemize}
Relativo ao senado.
\section{Senatriz}
\begin{itemize}
\item {Grp. gram.:f.}
\end{itemize}
\begin{itemize}
\item {Proveniência:(Lat. \textunderscore senatrix\textunderscore )}
\end{itemize}
Mulhér de senador.
\section{Senatus-consulto}
\begin{itemize}
\item {Grp. gram.:m.}
\end{itemize}
\begin{itemize}
\item {Proveniência:(Lat. \textunderscore senatusconsultum\textunderscore )}
\end{itemize}
Decreto, com fôrça de lei, do antigo senado romano.
Decisão do senado conservador, no regime do primeiro e do segundo império francês.
\section{Senceno}
\begin{itemize}
\item {Grp. gram.:m.}
\end{itemize}
\begin{itemize}
\item {Utilização:Prov.}
\end{itemize}
\begin{itemize}
\item {Utilização:trasm.}
\end{itemize}
O mesmo que \textunderscore neblina\textunderscore .
(Relaciona-se com \textunderscore sincelo\textunderscore ?)
\section{Senciente}
\begin{itemize}
\item {Grp. gram.:adj.}
\end{itemize}
\begin{itemize}
\item {Proveniência:(Lat. \textunderscore sentiens\textunderscore )}
\end{itemize}
Que sente; que tem sensações.
\section{Senda}
\begin{itemize}
\item {Grp. gram.:f.}
\end{itemize}
\begin{itemize}
\item {Utilização:Fig.}
\end{itemize}
\begin{itemize}
\item {Proveniência:(Do lat. \textunderscore semita\textunderscore )}
\end{itemize}
Caminho estreito; vereda; atalho.
Praxe; rotina; hábito.
\section{Sendal}
\begin{itemize}
\item {Grp. gram.:m.}
\end{itemize}
(V.cendal)
\section{Sendeira}
\begin{itemize}
\item {Grp. gram.:f.}
\end{itemize}
\begin{itemize}
\item {Utilização:Fam.}
\end{itemize}
\begin{itemize}
\item {Proveniência:(De \textunderscore sendeiro\textunderscore )}
\end{itemize}
Parvoíce, dislate.
\section{Sendeirada}
\begin{itemize}
\item {Grp. gram.:f.}
\end{itemize}
O mesmo que \textunderscore sendeirice\textunderscore .
\section{Sendeirice}
\begin{itemize}
\item {Grp. gram.:f.}
\end{itemize}
O mesmo que \textunderscore sendeira\textunderscore .
\section{Sendeiro}
\begin{itemize}
\item {Grp. gram.:m.  e  adj.}
\end{itemize}
\begin{itemize}
\item {Utilização:Chul.}
\end{itemize}
\begin{itemize}
\item {Proveniência:(De \textunderscore senda\textunderscore )}
\end{itemize}
Diz-se do cavallo e do burro, velho ou ruím.
Diz-se, no Brasil, do cavallo de carga, robusto mas pouco encorpado.
Indivíduo desprezível, sevandija.
\section{Sendinês}
\begin{itemize}
\item {Grp. gram.:m.}
\end{itemize}
Dialecto português de Sendim.
\section{Sendos}
\begin{itemize}
\item {Grp. gram.:adj. pl.}
\end{itemize}
\begin{itemize}
\item {Utilização:Ant.}
\end{itemize}
Dizia-se de dois objectos da mesma natureza, que se referem ou pertencem a duas pessôas, levando ou tendo cada um o seu: \textunderscore iam os dois amigos em sendos cavallos\textunderscore .
(Cp. \textunderscore senhos\textunderscore  e cast. \textunderscore sendos\textunderscore )
\section{Sendtenera}
\begin{itemize}
\item {Grp. gram.:f.}
\end{itemize}
\begin{itemize}
\item {Proveniência:(De \textunderscore Sendtener\textunderscore , n. p.)}
\end{itemize}
Gênero de plantas hepáticas.
\section{Sene}
\begin{itemize}
\item {Grp. gram.:m.}
\end{itemize}
\begin{itemize}
\item {Proveniência:(Do ár. \textunderscore sena\textunderscore )}
\end{itemize}
Nome de várias plantas cesalpíneas, do gênero da cássia.
\section{Sene}
\begin{itemize}
\item {Grp. gram.:m.}
\end{itemize}
\begin{itemize}
\item {Utilização:Ant.}
\end{itemize}
\begin{itemize}
\item {Proveniência:(Do lat. \textunderscore senex\textunderscore )}
\end{itemize}
Homem velho, idoso.
\section{Senebiera}
\begin{itemize}
\item {Grp. gram.:f.}
\end{itemize}
\begin{itemize}
\item {Proveniência:(De \textunderscore Senebier\textunderscore , n. p.)}
\end{itemize}
Gênero de plantas crucíferas.
\section{Sêneca}
\begin{itemize}
\item {Grp. gram.:f.}
\end{itemize}
Planta polygalácea, o mesmo que \textunderscore sênega\textunderscore .
\section{Sêneca}
\begin{itemize}
\item {Grp. gram.:f.}
\end{itemize}
(V.sênica)
\section{Senécio}
\begin{itemize}
\item {Grp. gram.:m.}
\end{itemize}
\begin{itemize}
\item {Utilização:Bot.}
\end{itemize}
\begin{itemize}
\item {Proveniência:(Lat. \textunderscore senecio\textunderscore )}
\end{itemize}
Designação scientífica da tasneirinha ou do \textunderscore não-me-deixes\textunderscore .
\section{Senecióneas}
\begin{itemize}
\item {Grp. gram.:f. pl.}
\end{itemize}
\begin{itemize}
\item {Utilização:Bot.}
\end{itemize}
\begin{itemize}
\item {Proveniência:(De \textunderscore senecio\textunderscore )}
\end{itemize}
Grupo de synanthéreas, estabelecido por Cassini.
\section{Senecionídeas}
\begin{itemize}
\item {Grp. gram.:f. pl.}
\end{itemize}
Família de plantas compostas que tem por typo a tasneirinha ou senécio.
(Fem. pl. de \textunderscore senecionídeo\textunderscore )
\section{Senecionídeo}
\begin{itemize}
\item {Grp. gram.:adj.}
\end{itemize}
\begin{itemize}
\item {Proveniência:(Do lat. \textunderscore senecio\textunderscore  + gr. \textunderscore eidos\textunderscore )}
\end{itemize}
Relativo ou semelhante á tasneirinha.
\section{Senecto}
\begin{itemize}
\item {Grp. gram.:adj.}
\end{itemize}
\begin{itemize}
\item {Utilização:Ant.}
\end{itemize}
\begin{itemize}
\item {Proveniência:(Lat. \textunderscore senectus\textunderscore )}
\end{itemize}
Velho, antigo.
\section{Senectude}
\begin{itemize}
\item {Grp. gram.:f.}
\end{itemize}
\begin{itemize}
\item {Proveniência:(Do lat. \textunderscore senectus\textunderscore , \textunderscore senectutis\textunderscore )}
\end{itemize}
Idade senil; decrepitude.
\section{Sénega}
\begin{itemize}
\item {Grp. gram.:f.}
\end{itemize}
Planta polygalácea, de raíz medicinal, (\textunderscore polygala senega\textunderscore , Lin.). Cf. \textunderscore Pharmacopeia Port.\textunderscore 
\section{Senegalês}
\begin{itemize}
\item {Grp. gram.:adj.}
\end{itemize}
\begin{itemize}
\item {Grp. gram.:M.}
\end{itemize}
Relativo ao Senegal.
Habitante do Senegal.
\section{Senegali}
\begin{itemize}
\item {Grp. gram.:m.}
\end{itemize}
Ave do Senegal, pertencente á órdem de pássaros.
\section{Senegaliano}
\begin{itemize}
\item {Grp. gram.:m.  e  adj.}
\end{itemize}
O mesmo que \textunderscore senegalês\textunderscore .
\section{Senegálico}
\begin{itemize}
\item {Grp. gram.:m.  e  adj.}
\end{itemize}
O mesmo que \textunderscore senegalês\textunderscore .
\section{Senembi}
\begin{itemize}
\item {Utilização:bras}
\end{itemize}
O mesmo que \textunderscore iguano\textunderscore .
\section{Senembu}
\begin{itemize}
\item {Grp. gram.:m.}
\end{itemize}
\begin{itemize}
\item {Utilização:Bras}
\end{itemize}
O mesmo que \textunderscore iguano\textunderscore .
\section{Senescal}
\begin{itemize}
\item {Grp. gram.:m.}
\end{itemize}
\begin{itemize}
\item {Proveniência:(Do b. lat. \textunderscore senescalcus\textunderscore )}
\end{itemize}
Antigo mordomo-mór ou vedor, em certas casas reaes.
Magistrado judicial, em alguns países.
\section{Senescalia}
\begin{itemize}
\item {Grp. gram.:f.}
\end{itemize}
Cargo ou dignidade de senescal.
\section{Senga}
\begin{itemize}
\item {Grp. gram.:f.}
\end{itemize}
\begin{itemize}
\item {Utilização:Bras. do Rio}
\end{itemize}
Conjunto de fragmentos.
\section{Sengar}
\begin{itemize}
\item {Grp. gram.:v. t.}
\end{itemize}
\begin{itemize}
\item {Utilização:Bras. do Rio}
\end{itemize}
\begin{itemize}
\item {Proveniência:(De \textunderscore senga\textunderscore )}
\end{itemize}
Separar por meio de peneira (diversas substâncias), ficando de um lado os corpos mais pesados e do outro os mais leves.
\section{Sengas}
\begin{itemize}
\item {Grp. gram.:m. pl.}
\end{itemize}
Uma das tríbos cafreaes de Tete e Zumbo.
\section{Sengo}
\begin{itemize}
\item {Grp. gram.:adj.}
\end{itemize}
\begin{itemize}
\item {Utilização:Ant.}
\end{itemize}
\begin{itemize}
\item {Utilização:Prov.}
\end{itemize}
\begin{itemize}
\item {Utilização:beir.}
\end{itemize}
\begin{itemize}
\item {Proveniência:(Do lat. \textunderscore senicus\textunderscore , por \textunderscore senex\textunderscore )}
\end{itemize}
Intelligente; atilado.
Sonso. Cf. \textunderscore Eufrosina\textunderscore , 39.
\section{Senha}
\begin{itemize}
\item {Grp. gram.:f.}
\end{itemize}
Sinal.
Gesto, combinado entre duas ou mais pessôas.
Recibo.
Papel ou bilhete, que autoriza a admissão ou readmissão numa assembleia ou num espectáculo.
Documento, que mostra têr pago as respectivas propinas quem pretende fazer certos exames ou actos.
(Cast. \textunderscore seña\textunderscore )
\section{Senha}
\begin{itemize}
\item {Grp. gram.:f.}
\end{itemize}
Gênero de árvores do Congo.--Há senha de água, senha amarela e senha rosa.
\section{Senheiro}
\begin{itemize}
\item {Grp. gram.:adj.}
\end{itemize}
\begin{itemize}
\item {Utilização:Ant.}
\end{itemize}
\begin{itemize}
\item {Proveniência:(Do lat. hyp. \textunderscore singularius\textunderscore )}
\end{itemize}
Solitário. Cf. Car. Michaëlis, \textunderscore Estatinga\textunderscore .
\section{Senho}
\begin{itemize}
\item {Grp. gram.:m.}
\end{itemize}
O mesmo que \textunderscore cenho\textunderscore  e orthogr. mais correcta.
\section{Senhor}
\begin{itemize}
\item {Grp. gram.:m.}
\end{itemize}
\begin{itemize}
\item {Utilização:Fam.}
\end{itemize}
\begin{itemize}
\item {Grp. gram.:F.}
\end{itemize}
\begin{itemize}
\item {Utilização:Ant.}
\end{itemize}
\begin{itemize}
\item {Proveniência:(Do lat. \textunderscore senior\textunderscore )}
\end{itemize}
Aquelle que tinha autoridade feudal sôbre certas pessôas ou propriedades.
Possuidor, dominador.
Dono.
Título, que por deferência se dá a certos homens, distintos pela sua posição ou dignidade.
Indivíduo distinto.
Título nobiliário.
Tratamento ceremonioso.
O supremo dominador, Deus.
Dono da casa: \textunderscore ó Joanna, chama o senhor para a mesa\textunderscore .
O mesmo que \textunderscore senhora\textunderscore :«\textunderscore senhor fremosa...\textunderscore »Conde de Barcellos, 67.
\section{Senhora}
\begin{itemize}
\item {Grp. gram.:f.}
\end{itemize}
\begin{itemize}
\item {Utilização:Fam.}
\end{itemize}
\begin{itemize}
\item {Proveniência:(De \textunderscore senhor\textunderscore )}
\end{itemize}
Mulhér, que tem autoridade sôbre certas pessôas ou coisas.
Dona.
Dona da casa.
Possuidora.
Título de cortesia, dado a mulheres.
Nação ou qualquer entidade, que influe sôbre outra ou a domina.
A Virgem Maria: \textunderscore rezar á Senhora\textunderscore .
O mesmo que \textunderscore espôsa\textunderscore : \textunderscore como está, meu amigo? E sua senhora como passa\textunderscore ?
\section{Senhoraça}
\begin{itemize}
\item {Grp. gram.:f.}
\end{itemize}
\begin{itemize}
\item {Utilização:Fam.}
\end{itemize}
Mulhér de baixa estirpe, que procura parecer senhora, trajando com luxo ou garridice.
Senhora encorpada e mais ou menos esbelta.
\section{Senhoraço}
\begin{itemize}
\item {Grp. gram.:m.}
\end{itemize}
\begin{itemize}
\item {Utilização:Burl.}
\end{itemize}
\begin{itemize}
\item {Proveniência:(De \textunderscore senhor\textunderscore )}
\end{itemize}
Homem de inferior condição, que se inculca como pertencente a categoria superior.
\section{Senhorama}
\begin{itemize}
\item {Grp. gram.:f.}
\end{itemize}
\begin{itemize}
\item {Utilização:Fam.}
\end{itemize}
\begin{itemize}
\item {Proveniência:(De \textunderscore senhora\textunderscore )}
\end{itemize}
Conjunto de senhoras:«\textunderscore toda essa senhorama dos arredores...\textunderscore »Eça.
\section{Senhoreador}
\begin{itemize}
\item {Grp. gram.:m.  e  adj.}
\end{itemize}
O que senhoreia.
\section{Senhorear}
\begin{itemize}
\item {Grp. gram.:v. t.}
\end{itemize}
\begin{itemize}
\item {Grp. gram.:V. i.}
\end{itemize}
Tornar-se senhor de; conquistar; dominar.
Influír moralmente sôbre.
Captar o ânimo de.
Reduzir.
Exercer domínio.
\section{Senhoria}
\begin{itemize}
\item {Grp. gram.:f.}
\end{itemize}
\begin{itemize}
\item {Proveniência:(De \textunderscore senhor\textunderscore )}
\end{itemize}
Qualidade de senhor ou de senhora.
Senhorio.
Proprietária de um prédio que se tomou de arrendamento.
Tratamento, que se dá hoje a pessôas de posição decente mas não elevada, e que dantes se dava ás pessôas da primeira nobreza.
\section{Senhoriagem}
\begin{itemize}
\item {Grp. gram.:f.}
\end{itemize}
\begin{itemize}
\item {Proveniência:(De \textunderscore senhorio\textunderscore )}
\end{itemize}
Contribuição, que se pagava, como reconhecimento de um senhorio.
Direito, que se pagava ao Rei pela cunhagem da moéda.
Differença entre o valor real e o nominal da moéda. Cf. F. de Mendonça, \textunderscore Vocab. Techn.\textunderscore 
\section{Senhorial}
\begin{itemize}
\item {Grp. gram.:adj.}
\end{itemize}
Relativo a senhorio.
\section{Senhoril}
\begin{itemize}
\item {Grp. gram.:adj.}
\end{itemize}
\begin{itemize}
\item {Utilização:Ext.}
\end{itemize}
\begin{itemize}
\item {Proveniência:(De \textunderscore senhor\textunderscore  ou \textunderscore senhora\textunderscore )}
\end{itemize}
Próprio de senhor ou senhora.
Distinto, majestoso; elegante.
\section{Senhorilidade}
\begin{itemize}
\item {Grp. gram.:f.}
\end{itemize}
Qualidade de senhoril.
\section{Senhorilmente}
\begin{itemize}
\item {Grp. gram.:adv.}
\end{itemize}
De modo senhoril.
\section{Senhorio}
\begin{itemize}
\item {Grp. gram.:m.}
\end{itemize}
\begin{itemize}
\item {Proveniência:(De \textunderscore senhor\textunderscore )}
\end{itemize}
Direito, que o senhor tem, relativamente a certas pessôas ou coisas.
Domínio; autoridade.
Propriedade ou quaesquer coisas, em que recái o direito de um senhor.
Proprietário de um prédio, que se tomou de arrendamento: \textunderscore o inquilino procurou o senhorio\textunderscore .
\textunderscore Senhorio direito\textunderscore , a entidade que recebe o foro de um prazo.
\textunderscore Senhorio útil\textunderscore , a entidade que possue prédio emphytêutico e paga o respectivo foro.
\section{Senhorita}
\begin{itemize}
\item {Grp. gram.:f.}
\end{itemize}
\begin{itemize}
\item {Utilização:Pop.}
\end{itemize}
\begin{itemize}
\item {Utilização:Bras}
\end{itemize}
Pequena senhora.
Mulher, de pequena estatura.
Senhoraça.
Senhora solteira.
(Cp. cast. \textunderscore señorita\textunderscore )
\section{Senhorizar}
\begin{itemize}
\item {Grp. gram.:v. t.}
\end{itemize}
\begin{itemize}
\item {Utilização:Ant.}
\end{itemize}
Exercer jurisdição em.
Tornar senhor; dar govêrno ou jurisdicção a.
\section{Senhos}
\begin{itemize}
\item {Grp. gram.:adj. pl.}
\end{itemize}
\begin{itemize}
\item {Utilização:Ant.}
\end{itemize}
\begin{itemize}
\item {Proveniência:(Do lat. \textunderscore singuli\textunderscore )}
\end{itemize}
Cada um com o seu.
O mesmo que \textunderscore sendos\textunderscore . Cf. \textunderscore Port. Mon. Hist.\textunderscore , \textunderscore Script.\textunderscore , 255.
\section{Sênica}
\begin{itemize}
\item {Grp. gram.:f.}
\end{itemize}
Designação popular do arsênico: \textunderscore envenenou-se com sênica\textunderscore .
\section{Senil}
\begin{itemize}
\item {Grp. gram.:adj.}
\end{itemize}
\begin{itemize}
\item {Proveniência:(Lat. \textunderscore senilis\textunderscore )}
\end{itemize}
Relativo á velhice; velho; decrépito; próprio da velhice.
\section{Senilidade}
\begin{itemize}
\item {Grp. gram.:f.}
\end{itemize}
Qualidade ou estado do que é senil.
Enfraquecimento intellectual, determinado pela velhice.
\section{Sefela}
\begin{itemize}
\item {Grp. gram.:f.}
\end{itemize}
Gênero de insectos hemípteros.
\section{Sefelo}
\begin{itemize}
\item {Grp. gram.:m.}
\end{itemize}
Gênero de plantas sinantéreas.
\section{Sefina}
\begin{itemize}
\item {Grp. gram.:f.}
\end{itemize}
Gênero de insectos hemípteros.
\section{Senilização}
\begin{itemize}
\item {Grp. gram.:f.}
\end{itemize}
\begin{itemize}
\item {Utilização:bras}
\end{itemize}
\begin{itemize}
\item {Utilização:Neol.}
\end{itemize}
\begin{itemize}
\item {Proveniência:(De \textunderscore senil\textunderscore )}
\end{itemize}
Acto de envelhecer.
\section{Sênio}
\begin{itemize}
\item {Grp. gram.:m.}
\end{itemize}
\begin{itemize}
\item {Utilização:Des.}
\end{itemize}
\begin{itemize}
\item {Proveniência:(Lat. \textunderscore senium\textunderscore )}
\end{itemize}
O mesmo que \textunderscore velhice\textunderscore .
\section{Senior}
\begin{itemize}
\item {fónica:sêniòr}
\end{itemize}
\begin{itemize}
\item {Grp. gram.:adj.}
\end{itemize}
\begin{itemize}
\item {Grp. gram.:M.}
\end{itemize}
\begin{itemize}
\item {Utilização:Veloc.}
\end{itemize}
\begin{itemize}
\item {Proveniência:(T. lat.)}
\end{itemize}
Mais velho.
Velocipedista, que já obteve primeiros prêmios.
Chama-se \textunderscore senior fraco\textunderscore , aquelle que, pela estatura ou por outras circunstâncias não póde bater-se com os \textunderscore fortes\textunderscore ; \textunderscore senior forte\textunderscore  é o melhor velocipedista, o mestre, o profissional, o qual, depois de têr ganho prêmio pecuniário, já não póde funccionar como corredor.
\section{Senisga}
\begin{itemize}
\item {Grp. gram.:f.}
\end{itemize}
\begin{itemize}
\item {Utilização:T. da Bairrada}
\end{itemize}
\begin{itemize}
\item {Utilização:Prov.}
\end{itemize}
\begin{itemize}
\item {Utilização:trasm.}
\end{itemize}
Pequena porca, leitôa.
Rapariga espevitada e mexeriqueira.
O mesmo que \textunderscore pencha\textunderscore .
\section{Sennefeldera}
\begin{itemize}
\item {Grp. gram.:f.}
\end{itemize}
\begin{itemize}
\item {Proveniência:(De \textunderscore Sennefelder\textunderscore , n. p.)}
\end{itemize}
Gênero de plantas euphorbiáceas.
\section{Seno}
\begin{itemize}
\item {Grp. gram.:m.}
\end{itemize}
\begin{itemize}
\item {Utilização:Geom.}
\end{itemize}
\begin{itemize}
\item {Proveniência:(Do lat. \textunderscore sinus\textunderscore )}
\end{itemize}
Linha perpendicular, que vai da extremidade de um arco ao raio que passa sôbre a outra extremidade.
Relação entre êste raio e aquella perpendicular.
\section{Senodónia}
\begin{itemize}
\item {Grp. gram.:f.}
\end{itemize}
Gênero de insectos coleópteros pentâmeros.
\section{Senoga}
\begin{itemize}
\item {Grp. gram.:f.}
\end{itemize}
\begin{itemize}
\item {Utilização:Ant.}
\end{itemize}
O mesmo que \textunderscore synagoga\textunderscore .
\section{Senogastros}
\begin{itemize}
\item {Grp. gram.:m. pl.}
\end{itemize}
\begin{itemize}
\item {Proveniência:(Do gr. \textunderscore senos\textunderscore  + \textunderscore gaster\textunderscore )}
\end{itemize}
Gênero de insectos dípteros.
\section{Senoite}
\begin{itemize}
\item {fónica:á}
\end{itemize}
\begin{itemize}
\item {Grp. gram.:loc. adv.}
\end{itemize}
\begin{itemize}
\item {Utilização:T. da Feira}
\end{itemize}
Á noitinha; ao cair da noite.
(Por \textunderscore sonoite\textunderscore , de \textunderscore sub\textunderscore  + \textunderscore nocte\textunderscore )
\section{Senometópia}
\begin{itemize}
\item {Grp. gram.:f.}
\end{itemize}
\begin{itemize}
\item {Proveniência:(Do gr. \textunderscore senos\textunderscore  + \textunderscore metope\textunderscore )}
\end{itemize}
Gênero de insectos dípteros.
\section{Senones}
\begin{itemize}
\item {Grp. gram.:m. pl.}
\end{itemize}
\begin{itemize}
\item {Proveniência:(Lat. \textunderscore senones\textunderscore )}
\end{itemize}
Antigos habitantes da Gállia-lugdunense.
\section{Senoniano}
\begin{itemize}
\item {Grp. gram.:adj.}
\end{itemize}
\begin{itemize}
\item {Utilização:Geol.}
\end{itemize}
Diz-se de uma das espécies do terreno cretáceo.
\section{Senopterina}
\begin{itemize}
\item {Grp. gram.:f.}
\end{itemize}
Gênero de insectos dípteros.
\section{Senouro}
\begin{itemize}
\item {Grp. gram.:adj.}
\end{itemize}
\begin{itemize}
\item {Utilização:Prov.}
\end{itemize}
Murcho, sêco, (falando-se de plantas).
\section{Senoute}
\begin{itemize}
\item {fónica:á}
\end{itemize}
\begin{itemize}
\item {Grp. gram.:loc. adv.}
\end{itemize}
\begin{itemize}
\item {Utilização:T. da Feira}
\end{itemize}
Á noitinha; ao cair da noite.
(Por \textunderscore sonoite\textunderscore , de \textunderscore sub\textunderscore  + \textunderscore nocte\textunderscore )
\section{Senra}
\begin{itemize}
\item {Grp. gram.:f.}
\end{itemize}
\begin{itemize}
\item {Utilização:Ant.}
\end{itemize}
O mesmo que \textunderscore seara\textunderscore .
(Contr. do b. lat. \textunderscore senara\textunderscore )
\section{Senrada}
\begin{itemize}
\item {Grp. gram.:f.}
\end{itemize}
Porção de senras; senra extensa.
\section{Senradela}
\begin{itemize}
\item {Grp. gram.:f.}
\end{itemize}
\begin{itemize}
\item {Utilização:T. de Penafiel}
\end{itemize}
O mesmo que \textunderscore serradela\textunderscore ^1.
\section{Sensabor}
\begin{itemize}
\item {Grp. gram.:m. ,  f.  e  adj.}
\end{itemize}
\begin{itemize}
\item {Grp. gram.:adj.}
\end{itemize}
\begin{itemize}
\item {Grp. gram.:M.  e  f.}
\end{itemize}
\begin{itemize}
\item {Proveniência:(De \textunderscore sem\textunderscore  + \textunderscore sabor\textunderscore )}
\end{itemize}
O mesmo ou melhor que \textunderscore sem-sabor\textunderscore :«\textunderscore ó vós sois dhuns sensabores...\textunderscore », \textunderscore Anfitriões\textunderscore , act. I, sc. I.
Que não tem sabor.
Insípido; monótono; desengraçado.
Pessôa sem-sabor.
\section{Sensaborão}
\begin{itemize}
\item {Grp. gram.:m.  e  adj.}
\end{itemize}
\begin{itemize}
\item {Proveniência:(De \textunderscore sensabor\textunderscore )}
\end{itemize}
Indivíduo muito sem-sabor.
\section{Sensaboria}
\begin{itemize}
\item {Grp. gram.:f.}
\end{itemize}
\begin{itemize}
\item {Utilização:Fam.}
\end{itemize}
Qualidade daquelle ou daquillo que é sem-sabor.
Acto ou acontecimento desagradável ou que causa ou póde causar desgôstos.
\section{Sensação}
\begin{itemize}
\item {Grp. gram.:f.}
\end{itemize}
\begin{itemize}
\item {Utilização:Fig.}
\end{itemize}
\begin{itemize}
\item {Proveniência:(Do lat. \textunderscore sensatio\textunderscore )}
\end{itemize}
Impressão, produzido num órgão dos sentidos pelos objectos exteriores, transmittida ao cérebro pelos nervos e determinante de um juízo ou conceito.
Surpresa ou grande impressão, produzido por successo extraordinário: \textunderscore novidades de sensação\textunderscore .
Sensibilidade.
Commoção moral.
\section{Sensacional}
\begin{itemize}
\item {Grp. gram.:adj.}
\end{itemize}
\begin{itemize}
\item {Proveniência:(Do lat. \textunderscore sensatio\textunderscore )}
\end{itemize}
Relativo a sensação; que produz grande sensação.
\section{Sensacionalmente}
\begin{itemize}
\item {Grp. gram.:adv.}
\end{itemize}
De modo sensacional.
\section{Sensatamente}
\begin{itemize}
\item {Grp. gram.:adv.}
\end{itemize}
De modo sensato; prudentemente, com tino, com siso.
\section{Sensatez}
\begin{itemize}
\item {Grp. gram.:f.}
\end{itemize}
Qualidade do que é sensato; prudência; discrição, bom senso moral.
\section{Sensato}
\begin{itemize}
\item {Grp. gram.:adj.}
\end{itemize}
\begin{itemize}
\item {Proveniência:(Lat. \textunderscore sensatus\textunderscore )}
\end{itemize}
Que tem bom senso; prudente; circunspecto.
\section{Sensibilidade}
\begin{itemize}
\item {Grp. gram.:f.}
\end{itemize}
\begin{itemize}
\item {Proveniência:(Do lat. \textunderscore sensibilitas\textunderscore )}
\end{itemize}
Qualidade do que é sensível.
Propriedade de certas partes do systema nervoso, pela qual os homens e os animaes percebem as impressões causadas por objectos exteriores, ou produzidas interiormente.
Impressionabilidade; susceptibilidade.
Precisão, com que certos instrumenttos ou apparelhos indicam a menor differença ou alteração.
\section{Sensibilizador}
\begin{itemize}
\item {Grp. gram.:adj.}
\end{itemize}
\begin{itemize}
\item {Utilização:Phot.}
\end{itemize}
Que sensibiliza.
Diz-se do banho, em que se lançam as chapas, para se tornarem sensíveis á acção da luz.
\section{Sensibilizante}
\begin{itemize}
\item {Grp. gram.:adj.}
\end{itemize}
Que sensibiliza.
\section{Sensibilizar}
\begin{itemize}
\item {Grp. gram.:v. t.}
\end{itemize}
\begin{itemize}
\item {Proveniência:(Do lat. \textunderscore sensibilis\textunderscore )}
\end{itemize}
Tornar sensível.
Causar abalo a; commover.
\section{Sensificar}
\begin{itemize}
\item {Grp. gram.:v. t.}
\end{itemize}
\begin{itemize}
\item {Proveniência:(Do lat. \textunderscore sensus\textunderscore  + \textunderscore facere\textunderscore )}
\end{itemize}
Tornar sensível.
Restabelecer a sensibilidade em.
\section{Sensitiva}
\begin{itemize}
\item {Grp. gram.:f.}
\end{itemize}
\begin{itemize}
\item {Utilização:Fig.}
\end{itemize}
\begin{itemize}
\item {Proveniência:(De \textunderscore sensitivo\textunderscore )}
\end{itemize}
Planta mimósea, cujas fôlhas têm a propriedade de se retrahir, quando se lhes toca.
Pessôa, que se melindra facilmente, ou que é de grande susceptibilidade.
\section{Sensitivo}
\begin{itemize}
\item {Grp. gram.:adj.}
\end{itemize}
\begin{itemize}
\item {Proveniência:(Do lat. \textunderscore sensus\textunderscore )}
\end{itemize}
Relativo aos sentidos; que tem a faculdade do sentir; que produz sensação.
\section{Sensível}
\begin{itemize}
\item {Grp. gram.:adj.}
\end{itemize}
\begin{itemize}
\item {Utilização:Fig.}
\end{itemize}
\begin{itemize}
\item {Proveniência:(Do lat. \textunderscore sensibilis\textunderscore )}
\end{itemize}
Que sente.
Que tem sentidos.
Que se impressiona facilmente.
Que recebe com facilidade as sensações externas ou as impressões moraes. Sujeito á acção dos sentidos: \textunderscore o mundo sensível\textunderscore .
Que impressiona os sentidos.
Que impressiona moralmente.
Que sente as dores ou males alheios.
Compassivo.
Doloroso.
Evidente.
Que se póde apreciar.
Que indica a menor differença ou alteração, (falando-se de certos apparelhos).
Que se retrai ao contacto de alguém ou de alguma coisa, (falando-se de vegetaes).
Diz-se da nota musical, que está meio tom abaixo da tónica.
\section{Sensivelmente}
\begin{itemize}
\item {Grp. gram.:adv.}
\end{itemize}
De modo sensível.
\section{Sensivo}
\begin{itemize}
\item {Grp. gram.:adj.}
\end{itemize}
(V.sensível)
\section{Senso}
\begin{itemize}
\item {Grp. gram.:m.}
\end{itemize}
\begin{itemize}
\item {Proveniência:(Lat. \textunderscore sensus\textunderscore )}
\end{itemize}
Juízo claro.
Acto de raciocinar.
Sisudez; circunspecção; siso.
Sentido, direcção:«\textunderscore ...e para ir o mesmo navio em senso contrário...\textunderscore »Filinto, \textunderscore D. Man.\textunderscore , I, 193.
\section{Sensorial}
\begin{itemize}
\item {Grp. gram.:adj.}
\end{itemize}
Relativo ao cérebro, ou á parte do cérebro chamada \textunderscore sensório\textunderscore . Cf. Dom. Vieira \textunderscore Diccion.\textunderscore  vb. \textunderscore anevrismo\textunderscore .
\section{Sensório}
\begin{itemize}
\item {Grp. gram.:adj.}
\end{itemize}
\begin{itemize}
\item {Grp. gram.:M.}
\end{itemize}
\begin{itemize}
\item {Proveniência:(Lat. \textunderscore sensorium\textunderscore )}
\end{itemize}
Relativo á sensibilidade.
Próprio para a transmissão das sensações.
O cérebro ou parte do cérebro, que alguns philósophos consíderam como centro das sensações.
\section{Sensual}
\begin{itemize}
\item {Grp. gram.:adj.}
\end{itemize}
\begin{itemize}
\item {Grp. gram.:M.}
\end{itemize}
\begin{itemize}
\item {Proveniência:(Lat. \textunderscore sensualis\textunderscore )}
\end{itemize}
Relativo aos sentidos.
Voluptuoso; lúbrico; lascivo.
Indivíduo sensual ou libertino.
\section{Sensualidade}
\begin{itemize}
\item {Grp. gram.:f.}
\end{itemize}
\begin{itemize}
\item {Proveniência:(Lat. \textunderscore sensualitas\textunderscore )}
\end{itemize}
Qualidade de que é sensual.
Lubricidade; volúpia.
Amor aos prazeres materiaes.
\section{Sensualismo}
\begin{itemize}
\item {Grp. gram.:m.}
\end{itemize}
\begin{itemize}
\item {Proveniência:(De \textunderscore sensual\textunderscore )}
\end{itemize}
Doutrina, que attribue tudo á acção dos sentidos externos, na formação das ideias.
O mesmo que \textunderscore sensualidade\textunderscore .
\section{Sensualista}
\begin{itemize}
\item {Grp. gram.:adj.}
\end{itemize}
\begin{itemize}
\item {Grp. gram.:M.  e  f.}
\end{itemize}
\begin{itemize}
\item {Proveniência:(De \textunderscore sensual\textunderscore )}
\end{itemize}
Relativo ao sensualismo.
Pessôa que segue o sensualismo.
\section{Sensualizar}
\begin{itemize}
\item {Grp. gram.:v. t.}
\end{itemize}
\begin{itemize}
\item {Proveniência:(De \textunderscore sensual\textunderscore )}
\end{itemize}
Excitar aos prazeres dos sentidos.
\section{Sensualmente}
\begin{itemize}
\item {Grp. gram.:adv.}
\end{itemize}
De modo sensual.
\section{Sentada}
\begin{itemize}
\item {Grp. gram.:f.}
\end{itemize}
\begin{itemize}
\item {Utilização:Bras. do S}
\end{itemize}
\begin{itemize}
\item {Proveniência:(De \textunderscore sentar\textunderscore ?)}
\end{itemize}
Parada súbita do cavallo que galopa.
\section{Sentador}
\begin{itemize}
\item {Grp. gram.:adj.}
\end{itemize}
\begin{itemize}
\item {Utilização:Bras. do S}
\end{itemize}
\begin{itemize}
\item {Proveniência:(De \textunderscore sentar\textunderscore ?)}
\end{itemize}
Diz-se do cavallo que, preso á estaca, se atira para trás, quebrando o laço e ficando em liberdade.
\section{Sentar}
\begin{itemize}
\item {Grp. gram.:v. t.}
\end{itemize}
\begin{itemize}
\item {Grp. gram.:V. p.}
\end{itemize}
\begin{itemize}
\item {Proveniência:(Do lat. \textunderscore sedens\textunderscore , \textunderscore sedentis\textunderscore )}
\end{itemize}
O mesmo que \textunderscore assentar\textunderscore .
Tomar assento; fixar-se.
\section{Sente}
\begin{itemize}
\item {Utilização:port}
\end{itemize}
\begin{itemize}
\item {Utilização:Ant.}
\end{itemize}
\begin{itemize}
\item {Proveniência:(De \textunderscore sêr\textunderscore )}
\end{itemize}
Sendo, estando.
\section{Sentença}
\begin{itemize}
\item {Grp. gram.:f.}
\end{itemize}
\begin{itemize}
\item {Proveniência:(Lat. \textunderscore sententia\textunderscore )}
\end{itemize}
Locução que contém um princípio ou pensamento moral.
Provérbio.
Julgamento pronunciado por um juíz, por um tribunal ou por árbitros.
Despacho ou decisão.
Resolução inabalável.
Julgamento divino á cêrca das acções humanas.
\section{Sentenciador}
\begin{itemize}
\item {Grp. gram.:m.  e  adj.}
\end{itemize}
O que sentencia.
\section{Sentenciar}
\begin{itemize}
\item {Grp. gram.:v. t.}
\end{itemize}
\begin{itemize}
\item {Grp. gram.:V. i.}
\end{itemize}
\begin{itemize}
\item {Proveniência:(Do lat. \textunderscore sententia\textunderscore )}
\end{itemize}
Decidir por meio de sentença.
Julgar.
Condemnar por meio de sentença.
Pronunciar sentença.
Dar ou manifestar voto.
\section{Sentenciosamente}
\begin{itemize}
\item {Grp. gram.:adv.}
\end{itemize}
De modo sentencioso.
Gravemente; em tom autoritário.
\section{Sentencioso}
\begin{itemize}
\item {Grp. gram.:adj.}
\end{itemize}
\begin{itemize}
\item {Proveniência:(Lat. \textunderscore sententiosus\textunderscore )}
\end{itemize}
Que tem fórma de sentença.
Em que há sentença ou provérbio.
Que se exprime gravemente e com laconismo, formulando decisões.
Grave como um juiz.
\section{Sentenciúncula}
\begin{itemize}
\item {Grp. gram.:f.}
\end{itemize}
Pequena sentença, adágio, prolóquio:«\textunderscore boa parte dessas taes sentenciúnculas tem um quid que geralmente agrada.\textunderscore »Castilho.
\section{Sentidamente}
\begin{itemize}
\item {Grp. gram.:adv.}
\end{itemize}
De modo sentido; com sentimento.
\section{Sentido}
\begin{itemize}
\item {Grp. gram.:adj.}
\end{itemize}
\begin{itemize}
\item {Grp. gram.:M.}
\end{itemize}
\begin{itemize}
\item {Grp. gram.:Interj.}
\end{itemize}
\begin{itemize}
\item {Grp. gram.:Pl.}
\end{itemize}
Sensível.
Que revela pesar, triste; plangente.
Cada uma das fórmas de receber sensações, segundo os órgãos destas.
Faculdade de sentir ou de apreciar.
Bom senso.
Juízo.
Ideia: \textunderscore ninguém sabe o que elle tem no sentido\textunderscore .
Escopo; intento: \textunderscore procurou-me, com o sentido de me convencer\textunderscore .
Significação: \textunderscore o sentido de uma phrase\textunderscore .
Cautela.
Attenção: \textunderscore tome sentido\textunderscore .
Asserção.
Modo de considerar um facto ou uma coisa.
Direcção que um acto segue.
Ponto de vista.
Cautela! cuidado!
Sensualidade.
Faculdade de sentir o prazer material ou carnal.
Faculdades intellectuaes; raciocínio.
\section{Sentieiro}
\begin{itemize}
\item {Grp. gram.:m.}
\end{itemize}
\begin{itemize}
\item {Utilização:Prov.}
\end{itemize}
\begin{itemize}
\item {Utilização:minh.}
\end{itemize}
O mesmo que \textunderscore míscaro\textunderscore .
\section{Sentimental}
\begin{itemize}
\item {Grp. gram.:adj.}
\end{itemize}
Relativo ao sentimento.
Que alardeia de affectuoso.
Impressionável; compassivo.
\section{Sentimentalidade}
\begin{itemize}
\item {Grp. gram.:f.}
\end{itemize}
Qualidade do que é sentimental.
\section{Sentimentalismo}
\begin{itemize}
\item {Grp. gram.:m.}
\end{itemize}
\begin{itemize}
\item {Proveniência:(De \textunderscore sentimental\textunderscore )}
\end{itemize}
O mesmo que \textunderscore sentimentalidade\textunderscore .
Affectação de quem procura mostrar-se muito sentimental ou muito sensível.
Gênero literário ou artístico, em que predomina o sentimento.
\section{Sentimentalista}
\begin{itemize}
\item {Grp. gram.:adj.}
\end{itemize}
\begin{itemize}
\item {Grp. gram.:M.  e  f.}
\end{itemize}
\begin{itemize}
\item {Proveniência:(De \textunderscore sentimental\textunderscore )}
\end{itemize}
Relativo ao sentimentalismo.
Pessôa, dada ao sentimentalismo.
\section{Sentimentalizar}
\begin{itemize}
\item {Grp. gram.:v. t.}
\end{itemize}
Tornar sentimental.
\section{Sentimentalmente}
\begin{itemize}
\item {Grp. gram.:adv.}
\end{itemize}
De modo sentimental.
\section{Sentimento}
\begin{itemize}
\item {Grp. gram.:m.}
\end{itemize}
\begin{itemize}
\item {Grp. gram.:Pl.}
\end{itemize}
Acto ou effeito de sentir.
Aptidão para sentir.
Sensação.
Sensibilidade.
Comprehensão; percepção.
Paixão.
Pesar.
Desgôsto.
Presentimento.
Bôa índole, bôas qualidades moraes.
Qualidades moraes.
Pêsames, ou manifestação de pesar pelas dores ou desgraças de outrem.
\section{Sentina}
\begin{itemize}
\item {Grp. gram.:f.}
\end{itemize}
\begin{itemize}
\item {Utilização:Fig.}
\end{itemize}
\begin{itemize}
\item {Proveniência:(Lat. \textunderscore sentina\textunderscore )}
\end{itemize}
A parte mais baixa do interior de um navio, onde as águas se juntam e se corrompem.
O mesmo que \textunderscore latrina\textunderscore .
Lugar immundo.
Pessôa viciosa.
\section{Sentinela}
\begin{itemize}
\item {Grp. gram.:f.}
\end{itemize}
\begin{itemize}
\item {Utilização:Fig.}
\end{itemize}
\begin{itemize}
\item {Proveniência:(It. \textunderscore sentinella\textunderscore . Mas o cast. \textunderscore centinela\textunderscore  aconselharia a mesma fórma em português)}
\end{itemize}
Soldado, que está de vigia ou guardando um pôsto, um acampamento, um monumento, etc.
Indivíduo, que guarda alguma coisa ou vela sôbre alguma coisa.
Acto de guardar, vigiar ou espiar.
Árvore, castelo ou coisa elevada, em sitio descampado ou ermo.
\section{Sentinella}
\begin{itemize}
\item {Grp. gram.:f.}
\end{itemize}
\begin{itemize}
\item {Utilização:Fig.}
\end{itemize}
\begin{itemize}
\item {Proveniência:(It. \textunderscore sentinella\textunderscore . Mas o cast. \textunderscore centinela\textunderscore  aconselharia a mesma fórma em português)}
\end{itemize}
Soldado, que está de vigia ou guardando um pôsto, um acampamento, um monumento, etc.
Indivíduo, que guarda alguma coisa ou vela sôbre alguma coisa.
Acto de guardar, vigiar ou espiar.
Árvore, castello ou coisa elevada, em sitio descampado ou ermo.
\section{Sentir}
\begin{itemize}
\item {Grp. gram.:v. t.}
\end{itemize}
\begin{itemize}
\item {Grp. gram.:V. i.}
\end{itemize}
\begin{itemize}
\item {Grp. gram.:V. p.}
\end{itemize}
\begin{itemize}
\item {Grp. gram.:M.}
\end{itemize}
\begin{itemize}
\item {Proveniência:(Lat. \textunderscore sentire\textunderscore )}
\end{itemize}
Perceber, por meio de qualquer órgão dos sentidos.
Perceber.
Experimentar.
Soffrer.
Têr (impressão moral).
Ouvir vagamente: \textunderscore sentir passos\textunderscore .
Impressionar-se por.
Estar persuadido de.
Reconhecer.
Prever.
Lastimar; mostrar sentimento por: \textunderscore sinto os seus desgostos\textunderscore .
Resentir-se com.
Julgar.
Estranhar, levar a mal: \textunderscore sinto que me não prevenissem\textunderscore .
Deixar-se impressionar por.
Entrever.
Receber impressões por meio dos sentidos.
Ser sensível.
Soffrer.
Têr pesar.
Têr a faculdade de receber sensações.
Conhecer-se intimamente.
Saber o que se passa no seu interior.
Mostrar-se resentido ou melindrado.
Molestar-se, maguar-se.
Modo de vêr.
Opinião.
Sentimento.
\section{Sento}
\begin{itemize}
\item {Grp. gram.:m.}
\end{itemize}
\begin{itemize}
\item {Utilização:Ant.}
\end{itemize}
Herdade cultivada.
\section{Senzala}
\begin{itemize}
\item {Grp. gram.:f.}
\end{itemize}
\begin{itemize}
\item {Utilização:Fig.}
\end{itemize}
\begin{itemize}
\item {Proveniência:(T. afr.)}
\end{itemize}
Habitação de Negros.
Povoação de Negros.
Residência de um soba.
Barulho, vozearia.
Lugar, em que há barulho.
\section{Sépala}
\begin{itemize}
\item {Grp. gram.:f.}
\end{itemize}
\begin{itemize}
\item {Proveniência:(Do lat. \textunderscore separ\textunderscore )}
\end{itemize}
Cada um dos folíolos dos cálices das flôres.
\section{Sepalóide}
\begin{itemize}
\item {Grp. gram.:adj.}
\end{itemize}
\begin{itemize}
\item {Proveniência:(De \textunderscore sépala\textunderscore  + gr. \textunderscore eidos\textunderscore )}
\end{itemize}
Que tem fórma de sépala.
\section{Separação}
\begin{itemize}
\item {Grp. gram.:f.}
\end{itemize}
\begin{itemize}
\item {Proveniência:(Lat. \textunderscore separatio\textunderscore )}
\end{itemize}
Acto ou effeito de separar.
Aquillo que separa ou veda.
Afastamento.
Quebra da união matrimonial.
\section{Separadamente}
\begin{itemize}
\item {Grp. gram.:adv.}
\end{itemize}
De modo separado.
Insuladamente; á parte.
\section{Separado}
\begin{itemize}
\item {Grp. gram.:adj.}
\end{itemize}
\begin{itemize}
\item {Grp. gram.:Loc. adv.}
\end{itemize}
\begin{itemize}
\item {Proveniência:(De \textunderscore separar\textunderscore )}
\end{itemize}
Desligado.
Pôsto de lado.
\textunderscore Em separado\textunderscore , separadamente.
\section{Separador}
\begin{itemize}
\item {Grp. gram.:m.  e  adj.}
\end{itemize}
\begin{itemize}
\item {Grp. gram.:M.}
\end{itemize}
\begin{itemize}
\item {Proveniência:(Lat. \textunderscore separator\textunderscore )}
\end{itemize}
O que separa.
Apparelho, que é uma desnatadeira, e que se chama assim, porque a fôrça centrífuga, applicada a uma mistura de líquidos, de differentes densidades, tem por effeito separar êsses líquidos, classificando-os por densidades decrescentes, do centro para a peripheria.
\section{Separar}
\begin{itemize}
\item {Grp. gram.:v. t.}
\end{itemize}
\begin{itemize}
\item {Proveniência:(Lat. \textunderscore separare\textunderscore )}
\end{itemize}
Desligar, desunir; afastar.
Interromper.
Pôr de lado.
Pôr á parte.
Distinguir.
Estremar.
Permittir ou decretar a ruptura da vida conjugal entre.
Dividir.
Estabelecer discórdia entre.
Impedir a união de.
\section{Separata}
\begin{itemize}
\item {Grp. gram.:f.}
\end{itemize}
\begin{itemize}
\item {Proveniência:(Lat. \textunderscore separata\textunderscore )}
\end{itemize}
Republicação, em volume ou opúsculo, de artigos, publicados num jornal ou revista.
\section{Separatismo}
\begin{itemize}
\item {Grp. gram.:m.}
\end{itemize}
Systema ou partido de separatistas.
\section{Separatista}
\begin{itemize}
\item {Grp. gram.:adj.}
\end{itemize}
\begin{itemize}
\item {Grp. gram.:M.  e  f.}
\end{itemize}
\begin{itemize}
\item {Proveniência:(Do lat. \textunderscore seperatus\textunderscore )}
\end{itemize}
Relativo á separação de um Estado, de uma província, de um indivíduo, etc.
Que tende a tornar-se independente.
Pessôa, que tem ideias separatistas.
\section{Separativo}
\begin{itemize}
\item {Grp. gram.:adj.}
\end{itemize}
\begin{itemize}
\item {Proveniência:(Lat. \textunderscore separativus\textunderscore )}
\end{itemize}
Que póde separar.
\section{Separatório}
\begin{itemize}
\item {Grp. gram.:adj.}
\end{itemize}
\begin{itemize}
\item {Grp. gram.:M.}
\end{itemize}
\begin{itemize}
\item {Proveniência:(De \textunderscore separar\textunderscore )}
\end{itemize}
O mesmo que \textunderscore separativo\textunderscore .
Vaso, com que se faz a separação de substâncias líquidas.
\section{Separável}
\begin{itemize}
\item {Grp. gram.:adj.}
\end{itemize}
\begin{itemize}
\item {Proveniência:(Lat. \textunderscore separabilis\textunderscore )}
\end{itemize}
Que se póde separar.
\section{Sepedónio}
\begin{itemize}
\item {Grp. gram.:m.}
\end{itemize}
\begin{itemize}
\item {Proveniência:(Do gr. \textunderscore sepedon\textunderscore )}
\end{itemize}
Gênero de cogumelos.
\section{Sepegar}
\begin{itemize}
\item {Grp. gram.:v. t.}
\end{itemize}
\begin{itemize}
\item {Utilização:Prov.}
\end{itemize}
\begin{itemize}
\item {Utilização:trasm.}
\end{itemize}
Açular (cães)
\section{Sepepira}
\begin{itemize}
\item {Grp. gram.:f.}
\end{itemize}
\begin{itemize}
\item {Utilização:Bras}
\end{itemize}
O mesmo que \textunderscore sicupira\textunderscore .
\section{Sephela}
\begin{itemize}
\item {Grp. gram.:f.}
\end{itemize}
Gênero de insectos hemípteros.
\section{Sephelo}
\begin{itemize}
\item {Grp. gram.:m.}
\end{itemize}
Gênero de plantas synanthéreas.
\section{Sephina}
\begin{itemize}
\item {Grp. gram.:f.}
\end{itemize}
Gênero de insectos hemípteros.
\section{Sépia}
\begin{itemize}
\item {Grp. gram.:f.}
\end{itemize}
\begin{itemize}
\item {Proveniência:(Do lat. \textunderscore sepia\textunderscore )}
\end{itemize}
Substância escura, que se extrai das sibas e é muito applicada em pintura.
\section{Sepícola}
\begin{itemize}
\item {Grp. gram.:adj.}
\end{itemize}
Que vive nas sebes.
\section{Sepídio}
\begin{itemize}
\item {Grp. gram.:m.}
\end{itemize}
\begin{itemize}
\item {Proveniência:(Do gr. \textunderscore sepedon\textunderscore )}
\end{itemize}
Gênero de insectos lepidópteros heterómeros.
\section{Sepio}
\begin{itemize}
\item {Grp. gram.:m.}
\end{itemize}
\begin{itemize}
\item {Utilização:Mad}
\end{itemize}
Chapéu alto.
\section{Sepíola}
\begin{itemize}
\item {Grp. gram.:f.}
\end{itemize}
Gênero de molluscos cephalópodes.
\section{Sepiostário}
\begin{itemize}
\item {Grp. gram.:m.}
\end{itemize}
Osso do chóco, concreção calcária da siba, (\textunderscore sepia officinales\textunderscore , Lin.). Cf. \textunderscore Pharmacopeia Port.\textunderscore 
\section{Sepiotanto}
\begin{itemize}
\item {Grp. gram.:m.}
\end{itemize}
Gênero de molluscos cefalópodes.
\section{Sepiothantho}
\begin{itemize}
\item {Grp. gram.:m.}
\end{itemize}
Gênero de molluscos cephalópodes.
\section{Seposição}
\begin{itemize}
\item {Grp. gram.:f.}
\end{itemize}
\begin{itemize}
\item {Utilização:Des.}
\end{itemize}
\begin{itemize}
\item {Proveniência:(Lat. \textunderscore sepositio\textunderscore )}
\end{itemize}
Pedido instante; rogos.
\section{Sepse}
\begin{itemize}
\item {Grp. gram.:f.}
\end{itemize}
\begin{itemize}
\item {Proveniência:(Do gr. \textunderscore sepsis\textunderscore )}
\end{itemize}
Corrupção ou putrefacção de tecidos ou substâncias orgânicas.
\section{Sepsia}
\begin{itemize}
\item {Grp. gram.:f.}
\end{itemize}
\begin{itemize}
\item {Proveniência:(Do gr. \textunderscore sepsis\textunderscore )}
\end{itemize}
Corrupção ou putrefacção de tecidos ou substâncias orgânicas.
\section{Sepsichimía}
\begin{itemize}
\item {fónica:qui}
\end{itemize}
\begin{itemize}
\item {Grp. gram.:f.}
\end{itemize}
\begin{itemize}
\item {Utilização:Med.}
\end{itemize}
Tendência dos humores para a putrefacção.
\section{Sépside}
\begin{itemize}
\item {Grp. gram.:m.}
\end{itemize}
Gênero de insectos dípteros.
\section{Sèpsina}
\begin{itemize}
\item {Grp. gram.:f.}
\end{itemize}
\begin{itemize}
\item {Utilização:Med.}
\end{itemize}
\begin{itemize}
\item {Proveniência:(De \textunderscore sepsia\textunderscore )}
\end{itemize}
Vírus que se desenvolve á superfície das chagas, produzindo a septicemía.
\section{Sepsiquimia}
\begin{itemize}
\item {Grp. gram.:f.}
\end{itemize}
\begin{itemize}
\item {Utilização:Med.}
\end{itemize}
Tendência dos humores para a putrefacção.
\section{Sépsis}
\begin{itemize}
\item {Grp. gram.:f.}
\end{itemize}
O mesmo que \textunderscore sepsia\textunderscore .
\section{Septa}
\begin{itemize}
\item {Grp. gram.:f.}
\end{itemize}
Gênero de plantas crassuláceas.
\section{Septe}
\begin{itemize}
\item {Grp. gram.:adj.}
\end{itemize}
(V.sete)
\section{Septem...}
\begin{itemize}
\item {Grp. gram.:pref.}
\end{itemize}
\begin{itemize}
\item {Proveniência:(Lat. \textunderscore septem\textunderscore )}
\end{itemize}
(designativo de \textunderscore sete\textunderscore )
\section{Septembro}
\begin{itemize}
\item {Grp. gram.:m.}
\end{itemize}
(V.setembro)
\section{Septem-angulado}
\begin{itemize}
\item {Grp. gram.:adj.}
\end{itemize}
\begin{itemize}
\item {Utilização:Bot.}
\end{itemize}
Que tem sete ângulos.
\section{Septêmfluo}
\begin{itemize}
\item {Grp. gram.:adj.}
\end{itemize}
\begin{itemize}
\item {Utilização:Poét.}
\end{itemize}
\begin{itemize}
\item {Proveniência:(Do lat. \textunderscore septem\textunderscore  + \textunderscore fluere\textunderscore )}
\end{itemize}
Que deriva de sete fontes.
\section{Septemfoliolado}
\begin{itemize}
\item {Grp. gram.:adj.}
\end{itemize}
\begin{itemize}
\item {Utilização:Bot.}
\end{itemize}
Que tem sete folíolos.
\section{Septem-lobado}
\begin{itemize}
\item {Grp. gram.:adj.}
\end{itemize}
\begin{itemize}
\item {Utilização:Bot.}
\end{itemize}
Que tem sete lóbulos.
\section{Septêmplice}
\begin{itemize}
\item {Grp. gram.:adj.}
\end{itemize}
\begin{itemize}
\item {Utilização:Poét.}
\end{itemize}
\begin{itemize}
\item {Proveniência:(Lat. \textunderscore septemplex\textunderscore )}
\end{itemize}
Dobrado em sete.
Que tem sete lâminas.
\section{Septemvirado}
\begin{itemize}
\item {Grp. gram.:m.}
\end{itemize}
\begin{itemize}
\item {Proveniência:(Do lat. \textunderscore septemviratus\textunderscore )}
\end{itemize}
Cargo ou dignidade de septêmviro.
Assembleia ou tribunal dos septêmviros.
\section{Septemviral}
\begin{itemize}
\item {Grp. gram.:adj.}
\end{itemize}
\begin{itemize}
\item {Proveniência:(Lat. \textunderscore septemviralis\textunderscore )}
\end{itemize}
Relativo aos septêmviros.
\section{Septemvirato}
\begin{itemize}
\item {Grp. gram.:m.}
\end{itemize}
(V.septemvirado)
\section{Septêmviro}
\begin{itemize}
\item {Grp. gram.:m.}
\end{itemize}
\begin{itemize}
\item {Proveniência:(Lat. \textunderscore septemvir\textunderscore )}
\end{itemize}
Cada um dos sete sacerdotes e magistrados romanos, que fiscalizavam os banquetes em honra dos deuses e os que se celebravam depois dos jogos públicos.
\section{Septena}
\begin{itemize}
\item {Grp. gram.:f.}
\end{itemize}
\begin{itemize}
\item {Grp. gram.:Adj. f.}
\end{itemize}
\begin{itemize}
\item {Proveniência:(Lat. \textunderscore septena\textunderscore )}
\end{itemize}
Estrophe de sete versos.
Diz-se de uma febre, cujos accessos se repetem de sete em sete dias.
\section{Septenário}
\begin{itemize}
\item {Grp. gram.:adj.}
\end{itemize}
\begin{itemize}
\item {Grp. gram.:M.}
\end{itemize}
\begin{itemize}
\item {Proveniência:(Lat. \textunderscore septenarius\textunderscore )}
\end{itemize}
Que vale ou contém sete.
Espaço de sete dias ou sete annos.
Festa religiosa, que dura sete dias.
\section{Septennado}
\begin{itemize}
\item {Grp. gram.:m.}
\end{itemize}
\begin{itemize}
\item {Proveniência:(Do lat. \textunderscore septennis\textunderscore )}
\end{itemize}
Govêrno, que se formou em França em 1873, para durar sete annos.
\section{Septennal}
\begin{itemize}
\item {Grp. gram.:adj.}
\end{itemize}
\begin{itemize}
\item {Proveniência:(Do lat. \textunderscore septennis\textunderscore )}
\end{itemize}
Que se realiza de sete em sete annos.
\section{Septennalidade}
\begin{itemize}
\item {Grp. gram.:f.}
\end{itemize}
Qualidade do que é septennal. Cf. Garrett, \textunderscore Port. na Balança\textunderscore , 89 e 99.
\section{Septennato}
\begin{itemize}
\item {Grp. gram.:m.}
\end{itemize}
O mesmo que \textunderscore septennado\textunderscore .
\section{Septennial}
\begin{itemize}
\item {Grp. gram.:adj.}
\end{itemize}
\begin{itemize}
\item {Proveniência:(De \textunderscore septênnio\textunderscore )}
\end{itemize}
Que dura sete annos.
\section{Septênnio}
\begin{itemize}
\item {Grp. gram.:m.}
\end{itemize}
\begin{itemize}
\item {Proveniência:(Lat. \textunderscore septennium\textunderscore )}
\end{itemize}
Espaço de sete annos.
\section{Septicemía}
\begin{itemize}
\item {Grp. gram.:f.}
\end{itemize}
\begin{itemize}
\item {Proveniência:(Do gr. \textunderscore spetikos\textunderscore  + \textunderscore haima\textunderscore )}
\end{itemize}
Alteração do sangue por substâncias pútridas.
\section{Septicêmico}
\begin{itemize}
\item {Grp. gram.:adj.}
\end{itemize}
Relativo a septicemía.
\section{Septiciana}
\begin{itemize}
\item {Grp. gram.:f.}
\end{itemize}
\begin{itemize}
\item {Proveniência:(Lat. \textunderscore septiciana\textunderscore )}
\end{itemize}
Libra romana que, durante a segunda guerra púnica, foi reduzida de doze onças a oito e meia.
\section{Septicida}
\begin{itemize}
\item {Grp. gram.:adj.}
\end{itemize}
\begin{itemize}
\item {Utilização:Bot.}
\end{itemize}
\begin{itemize}
\item {Proveniência:(Do lat. \textunderscore septum\textunderscore  + \textunderscore caedere\textunderscore )}
\end{itemize}
Diz-se da dehiscência dos frutos, quando se faz entre as duas fôlhas dos septos.
\section{Séptico}
\begin{itemize}
\item {Grp. gram.:adj.}
\end{itemize}
\begin{itemize}
\item {Proveniência:(Gr. \textunderscore septikos\textunderscore )}
\end{itemize}
Que causa putrefacção.
\section{Septicolle}
\begin{itemize}
\item {Grp. gram.:adj.}
\end{itemize}
\begin{itemize}
\item {Utilização:Poét.}
\end{itemize}
\begin{itemize}
\item {Proveniência:(Lat. \textunderscore septicollis\textunderscore )}
\end{itemize}
Que tem sete oiteiros.
\section{Septicolor}
\begin{itemize}
\item {Grp. gram.:m.}
\end{itemize}
\begin{itemize}
\item {Proveniência:(Do lat. \textunderscore septem\textunderscore  + \textunderscore color\textunderscore )}
\end{itemize}
Espécie de tangará, de plumagem variegada.
\section{Septicorde}
\begin{itemize}
\item {Grp. gram.:adj.}
\end{itemize}
\begin{itemize}
\item {Utilização:Poét.}
\end{itemize}
\begin{itemize}
\item {Proveniência:(Lat. \textunderscore septemchordis\textunderscore )}
\end{itemize}
Que tem sete cordas.
\section{Septífero}
\begin{itemize}
\item {Grp. gram.:adj.}
\end{itemize}
\begin{itemize}
\item {Utilização:Bot.}
\end{itemize}
\begin{itemize}
\item {Proveniência:(Do lat. \textunderscore septum\textunderscore  + \textunderscore ferre\textunderscore )}
\end{itemize}
Que tem septos.
\section{Septiforme}
\begin{itemize}
\item {Grp. gram.:adj.}
\end{itemize}
\begin{itemize}
\item {Proveniência:(Lat. \textunderscore septiformis\textunderscore )}
\end{itemize}
Que tem sete fórmas.
\section{Septiforme}
\begin{itemize}
\item {Grp. gram.:adj.}
\end{itemize}
\begin{itemize}
\item {Proveniência:(Do lat. \textunderscore septum\textunderscore  + \textunderscore forma\textunderscore )}
\end{itemize}
Que tem fórma de parede.
\section{Septífrago}
\begin{itemize}
\item {Grp. gram.:adj.}
\end{itemize}
\begin{itemize}
\item {Utilização:Bot.}
\end{itemize}
\begin{itemize}
\item {Proveniência:(Do lat. \textunderscore septum\textunderscore  + \textunderscore frangerere\textunderscore )}
\end{itemize}
Diz-se da dehiscência de um pericarpo, quando a ruptura se dá no septo, o qual fica livre e inteiro.
\section{Séptil}
\begin{itemize}
\item {Grp. gram.:adj.}
\end{itemize}
\begin{itemize}
\item {Utilização:Bot.}
\end{itemize}
\begin{itemize}
\item {Proveniência:(De \textunderscore septo\textunderscore )}
\end{itemize}
Diz-se dos grãos e da placenta, quando esta é unida ao septo.
\section{Septillião}
\begin{itemize}
\item {Grp. gram.:m.}
\end{itemize}
\begin{itemize}
\item {Proveniência:(Do lat. \textunderscore septem\textunderscore )}
\end{itemize}
Mil sextilliões.
\section{Septímanos}
\begin{itemize}
\item {Grp. gram.:m. pl.}
\end{itemize}
\begin{itemize}
\item {Proveniência:(Lat. \textunderscore septimani\textunderscore )}
\end{itemize}
Soldados da sétima legião romana.
\section{Septimátrias}
\begin{itemize}
\item {Grp. gram.:f. pl.}
\end{itemize}
\begin{itemize}
\item {Proveniência:(Do lat. \textunderscore septimatrus\textunderscore )}
\end{itemize}
Antigas festas romanas, que se celebravam em honra de Minerva, no sétimo dia depois dos idos.
\section{Septimestre}
\begin{itemize}
\item {Grp. gram.:adj.}
\end{itemize}
Que tem sete meses. Cf. \textunderscore Reino da Estupidez\textunderscore , 190.
\section{Septimôncio}
\begin{itemize}
\item {Grp. gram.:m.}
\end{itemize}
\begin{itemize}
\item {Proveniência:(Lat. \textunderscore septimontium\textunderscore )}
\end{itemize}
Festas, que os Romanos celebravam, em commemoração da encorporação das sete collinas no recinto de Roma.
\section{Septigentésimo}
\begin{itemize}
\item {Grp. gram.:adj.}
\end{itemize}
\begin{itemize}
\item {Proveniência:(Lat. \textunderscore septigentesimus\textunderscore )}
\end{itemize}
Que numa série de 700 occupa o último lugar.
\section{Septísono}
\begin{itemize}
\item {fónica:so}
\end{itemize}
\begin{itemize}
\item {Grp. gram.:adj.}
\end{itemize}
\begin{itemize}
\item {Proveniência:(Do lat. \textunderscore septem\textunderscore  + \textunderscore sonus\textunderscore )}
\end{itemize}
Que tem sete sons:«\textunderscore ...lyra septísona...\textunderscore »A. Dinís, \textunderscore Odes Pind.\textunderscore 
\section{Septisýllabo}
\begin{itemize}
\item {fónica:si}
\end{itemize}
\begin{itemize}
\item {Grp. gram.:adj.}
\end{itemize}
\begin{itemize}
\item {Grp. gram.:M.}
\end{itemize}
Que tem sete sýllabas.
Verso de sete sýllabas.
\section{Septívoco}
\begin{itemize}
\item {Grp. gram.:adj.}
\end{itemize}
\begin{itemize}
\item {Utilização:Poét.}
\end{itemize}
\begin{itemize}
\item {Proveniência:(Do lat. \textunderscore septem\textunderscore  + \textunderscore vox\textunderscore )}
\end{itemize}
Que tem sete vozes.
\section{Septizónio}
\begin{itemize}
\item {Grp. gram.:m.}
\end{itemize}
\begin{itemize}
\item {Proveniência:(Lat. \textunderscore septizonium\textunderscore )}
\end{itemize}
Nome de alguns edificios romanos que, segundo uns, eram rodeados de sete ordens de columnas \textunderscore ou\textunderscore , segundo outros, formados de sete andares.
Os sete planetas da astronomia antiga.
\section{Septo}
\begin{itemize}
\item {Grp. gram.:m.}
\end{itemize}
\begin{itemize}
\item {Proveniência:(Lat. \textunderscore septum\textunderscore )}
\end{itemize}
Membrana entre duas cavidades.
\section{Septómetro}
\begin{itemize}
\item {Grp. gram.:m.}
\end{itemize}
\begin{itemize}
\item {Proveniência:(Do gr. \textunderscore septos\textunderscore  + \textunderscore metron\textunderscore )}
\end{itemize}
Instrumento, para recolher e avaliar as substâncias orgânicas, que viciam a atmosphera.
\section{Septuagenário}
\begin{itemize}
\item {Grp. gram.:m.  e  adj.}
\end{itemize}
\begin{itemize}
\item {Proveniência:(Lat. \textunderscore septuagenarius\textunderscore )}
\end{itemize}
O que tem setenta annos, ou pouco mais ou menos.
\section{Septuagésima}
\begin{itemize}
\item {Grp. gram.:f.}
\end{itemize}
Terceiro Domingo antes do primeiro da Quaresma.
(Fem. de \textunderscore septuagésimo\textunderscore )
\section{Septuagésimo}
\begin{itemize}
\item {Grp. gram.:adj.}
\end{itemize}
\begin{itemize}
\item {Proveniência:(Lat. \textunderscore septuagesimus\textunderscore )}
\end{itemize}
Relativo a setenta.
Que numa série de setenta occupa o último lugar.
\section{Séptula}
\begin{itemize}
\item {Grp. gram.:f.}
\end{itemize}
\begin{itemize}
\item {Utilização:Bot.}
\end{itemize}
\begin{itemize}
\item {Proveniência:(De \textunderscore septo\textunderscore )}
\end{itemize}
Repartimento, que divide em céllulas a anthera das orchídeas.
\section{Séptunx}
\begin{itemize}
\item {Grp. gram.:m.}
\end{itemize}
\begin{itemize}
\item {Proveniência:(Do lat. \textunderscore septunx\textunderscore , \textunderscore septuncis\textunderscore )}
\end{itemize}
Pêso ou medida de sete onças, entre os Romanos. Cf. Castilho, \textunderscore Fastos\textunderscore , I, 387.--A rigorosa firma portuguesa seria \textunderscore septunce\textunderscore .
\section{Séptuor}
\begin{itemize}
\item {Grp. gram.:m.}
\end{itemize}
\begin{itemize}
\item {Proveniência:(Do lat. \textunderscore septem\textunderscore )}
\end{itemize}
Trêcho musical, para sêr executado por sete vozes ou sete instrumentos.
\section{Septupleta}
\begin{itemize}
\item {Grp. gram.:f.}
\end{itemize}
\begin{itemize}
\item {Utilização:Veloc.}
\end{itemize}
\begin{itemize}
\item {Proveniência:(De \textunderscore séptuplo\textunderscore )}
\end{itemize}
Velocípede com duas rodas, para sete pessoas.
\section{Septuplicar}
\begin{itemize}
\item {Grp. gram.:v. t.}
\end{itemize}
\begin{itemize}
\item {Proveniência:(De \textunderscore séptuplo\textunderscore )}
\end{itemize}
Tornar sete vezes maior.
\section{Séptuplo}
\begin{itemize}
\item {Grp. gram.:adj.}
\end{itemize}
\begin{itemize}
\item {Proveniência:(Lat. \textunderscore septuplus\textunderscore )}
\end{itemize}
Que vale sete vezes outro, ou que é sete vezes maior que outro.
\section{Septusse}
\begin{itemize}
\item {Grp. gram.:m.}
\end{itemize}
\begin{itemize}
\item {Proveniência:(Lat. \textunderscore septussis\textunderscore )}
\end{itemize}
Antiga moéda romana, de sete asses. Cf. Castilho, \textunderscore Fastos\textunderscore , I, 354.
\section{Sepulcral}
\begin{itemize}
\item {Grp. gram.:adj.}
\end{itemize}
\begin{itemize}
\item {Utilização:Fig.}
\end{itemize}
\begin{itemize}
\item {Proveniência:(Lat. \textunderscore sepulcralis\textunderscore )}
\end{itemize}
Relativo a sepulcro.
Que tem apparencia de sepulcro.
Fúnebre.
Sombrio; extremamente pállido.
\section{Sepulcrário}
\begin{itemize}
\item {Grp. gram.:m.}
\end{itemize}
\begin{itemize}
\item {Proveniência:(De \textunderscore sepulcro\textunderscore )}
\end{itemize}
Terreno, próprio para enterramentos.
\section{Sepulcro}
\begin{itemize}
\item {Grp. gram.:m.}
\end{itemize}
\begin{itemize}
\item {Utilização:Fig.}
\end{itemize}
\begin{itemize}
\item {Proveniência:(Lat. \textunderscore sepulcrum\textunderscore )}
\end{itemize}
Lugar, onde se enterram cadáveres.
Sepultura; túmulo.
Lugar, onde morre muita gente.
Aquillo que esconde como um túmulo.
\section{Sepultado}
\begin{itemize}
\item {Grp. gram.:adj.}
\end{itemize}
\begin{itemize}
\item {Proveniência:(De \textunderscore sepultar\textunderscore )}
\end{itemize}
Que se sepultou.
Enterrado.
Submergido.
\section{Sepultador}
\begin{itemize}
\item {Grp. gram.:m.  e  adj.}
\end{itemize}
O que sepulta.
\section{Sepultadora}
\begin{itemize}
\item {Grp. gram.:adj.}
\end{itemize}
\begin{itemize}
\item {Proveniência:(De \textunderscore sepultador\textunderscore )}
\end{itemize}
Diz-se de uma espécie de broqueleira.
\section{Sepultamento}
\begin{itemize}
\item {Grp. gram.:m.}
\end{itemize}
\begin{itemize}
\item {Utilização:Neol.}
\end{itemize}
Acto de sepultar.
\section{Sepultante}
\begin{itemize}
\item {Grp. gram.:adj.}
\end{itemize}
\begin{itemize}
\item {Proveniência:(Lat. \textunderscore sepultans\textunderscore )}
\end{itemize}
Que sepulta.
\section{Sepultar}
\begin{itemize}
\item {Grp. gram.:v. t.}
\end{itemize}
\begin{itemize}
\item {Utilização:Fig.}
\end{itemize}
\begin{itemize}
\item {Proveniência:(Lat. \textunderscore sepultare\textunderscore )}
\end{itemize}
Meter em sepultura.
Enterrar.
Soterrar.
Esconder.
Tornar occulto.
Afogar; mergulhar.
\section{Sepulto}
\begin{itemize}
\item {Grp. gram.:adj.}
\end{itemize}
\begin{itemize}
\item {Proveniência:(Lat. \textunderscore sepultus\textunderscore )}
\end{itemize}
O mesmo que \textunderscore sepultado\textunderscore .
\section{Sepultura}
\begin{itemize}
\item {Grp. gram.:f.}
\end{itemize}
\begin{itemize}
\item {Utilização:Fig.}
\end{itemize}
\begin{itemize}
\item {Proveniência:(Lat. \textunderscore sepultura\textunderscore )}
\end{itemize}
Acto ou effeito de sepultar.
Cova, em que se enterram cadáveres; sepulcro.
Morte.
Sítio, onde morre muita gente.
\section{Sepultureiro}
\begin{itemize}
\item {Grp. gram.:m.}
\end{itemize}
\begin{itemize}
\item {Proveniência:(De \textunderscore sepultura\textunderscore )}
\end{itemize}
O mesmo que \textunderscore coveiro\textunderscore .
\section{Sequaes}
\begin{itemize}
\item {Utilização:ant.}
\end{itemize}
O mesmo que \textunderscore siquaes\textunderscore .
\section{Sequaz}
\begin{itemize}
\item {Grp. gram.:m.  e  adj.}
\end{itemize}
\begin{itemize}
\item {Proveniência:(Lat. \textunderscore sequax\textunderscore )}
\end{itemize}
O que segue ou acompanha assiduamente; prosélyto; partidário.
\section{Sequeira}
\begin{itemize}
\item {Grp. gram.:f.}
\end{itemize}
\begin{itemize}
\item {Utilização:Fam.}
\end{itemize}
\begin{itemize}
\item {Proveniência:(De \textunderscore secar\textunderscore )}
\end{itemize}
O mesmo que \textunderscore séca\textunderscore ^1, maçada. Cf. Júl. Dinis, \textunderscore Fidalgos\textunderscore , I, 3.
\section{Sequeiro}
\begin{itemize}
\item {Grp. gram.:adj.}
\end{itemize}
\begin{itemize}
\item {Grp. gram.:M.}
\end{itemize}
\begin{itemize}
\item {Utilização:Prov.}
\end{itemize}
\begin{itemize}
\item {Utilização:trasm.}
\end{itemize}
\begin{itemize}
\item {Utilização:Prov.}
\end{itemize}
\begin{itemize}
\item {Utilização:trasm.}
\end{itemize}
\begin{itemize}
\item {Utilização:Prov.}
\end{itemize}
\begin{itemize}
\item {Proveniência:(Do b. lat. \textunderscore siccarius\textunderscore )}
\end{itemize}
Em que não há água.
Que não é regado: \textunderscore terreno sequeiro\textunderscore .
Lugar, que não é regadio.
Lugar, onde se estende roupa ou artefactos cerâmicos, para enxugar.
Monte de lenha, geralmente de carvalho, á porta do lavrador. (Colhido em Caçarelhos)
Conjunto dos tabuleiros, em que se seca fruta.
Espécie de espigueiro, armado no ar, para nele se secarem espigas de milho.
\section{Sequela}
\begin{itemize}
\item {fónica:qu-é}
\end{itemize}
\begin{itemize}
\item {Grp. gram.:f.}
\end{itemize}
\begin{itemize}
\item {Utilização:Pop.}
\end{itemize}
\begin{itemize}
\item {Proveniência:(Lat. \textunderscore sequela\textunderscore )}
\end{itemize}
Acto de seguir.
Consequência.
Bando.
Súcia.
Série de coisas.
\section{Sequência}
\begin{itemize}
\item {fónica:cu-en}
\end{itemize}
\begin{itemize}
\item {Grp. gram.:f.}
\end{itemize}
\begin{itemize}
\item {Proveniência:(Lat. \textunderscore sequentia\textunderscore )}
\end{itemize}
Seguimento, continuação; série.
Parte de um escrito, começado em outro livro ou em outro lugar.
Trecho em verso, que se reza na Missa depois da Epístola.
Série de cartas de jogar, sendo do mesmo naipe.
\section{Sequenha}
\begin{itemize}
\item {Grp. gram.:m.}
\end{itemize}
Árvore do Congo.
\section{Sequente}
\begin{itemize}
\item {fónica:cu-en}
\end{itemize}
\begin{itemize}
\item {Grp. gram.:adj.}
\end{itemize}
\begin{itemize}
\item {Proveniência:(Lat. \textunderscore sequens\textunderscore )}
\end{itemize}
O mesmo que \textunderscore seguinte\textunderscore .
\section{Sequer}
\begin{itemize}
\item {Grp. gram.:adv.}
\end{itemize}
\begin{itemize}
\item {Utilização:Ant.}
\end{itemize}
\begin{itemize}
\item {Proveniência:(De \textunderscore se\textunderscore ^1 + \textunderscore querer\textunderscore )}
\end{itemize}
Pelo menos; ao menos.
O mesmo que \textunderscore ainda\textunderscore .
\section{Sequestração}
\begin{itemize}
\item {Grp. gram.:f.}
\end{itemize}
\begin{itemize}
\item {Proveniência:(Do lat. \textunderscore sequestratio\textunderscore )}
\end{itemize}
Acto ou effeito de sequestrar.
\section{Sequestrador}
\begin{itemize}
\item {Grp. gram.:m.  e  adj.}
\end{itemize}
O que sequestra.
\section{Sequestrar}
\begin{itemize}
\item {Grp. gram.:v. t.}
\end{itemize}
\begin{itemize}
\item {Proveniência:(Lat. \textunderscore sequestrare\textunderscore )}
\end{itemize}
Fazer sequestro de.
Encarcerar illegalmente.
Tomar violentamente.
Esbulhar.
Afastar de lugares ou coisas perniciosas; insular.
\section{Sequestrável}
\begin{itemize}
\item {Grp. gram.:adj.}
\end{itemize}
Que se póde sequestrar.
\section{Sequestre}
\begin{itemize}
\item {Grp. gram.:m.}
\end{itemize}
\begin{itemize}
\item {Proveniência:(Lat. \textunderscore sequester\textunderscore )}
\end{itemize}
Medianeiro de certos negócios, entre os Romanos.
Depositário das sommas, com que nas eleições de cargos públicos, entre os Romanos, se compravam os votos do povo.
\section{Sequestro}
\begin{itemize}
\item {Grp. gram.:m.}
\end{itemize}
\begin{itemize}
\item {Proveniência:(Lat. \textunderscore sequestrum\textunderscore )}
\end{itemize}
Depósito de alguma coisa em poder de terceira pessôa, por convenção de interessados ou por ordem judicial.
Pessôa, a quem se confia aquelle depósito.
Arresto, penhóra.
Objecto ou objectos depositados.
Retenção illegal.
Parte inflammada de um osso, que se separa da parte san.
Sequestração.
\section{Séqui}
\begin{itemize}
\item {Grp. gram.:m.}
\end{itemize}
Espécie de cuco africano.
\section{Sequidão}
\begin{itemize}
\item {Grp. gram.:f.}
\end{itemize}
\begin{itemize}
\item {Utilização:T. de Alcanena}
\end{itemize}
O mesmo que \textunderscore secura\textunderscore .
\textunderscore Má sequidão\textunderscore , doença vegetal, que faz secar os bagos das uvas, ainda verdes.
\section{Sequilho}
\begin{itemize}
\item {Grp. gram.:m.}
\end{itemize}
\begin{itemize}
\item {Utilização:T. de Turquel}
\end{itemize}
Bôlo sêco e simples.
Fruta sêca.
\section{Sequilo}
\begin{itemize}
\item {Grp. gram.:m.}
\end{itemize}
\begin{itemize}
\item {Utilização:Bras. de San-Paulo}
\end{itemize}
O mesmo que \textunderscore sequilho\textunderscore .
\section{Sequim}
\begin{itemize}
\item {Grp. gram.:m.}
\end{itemize}
(V. \textunderscore cequim\textunderscore , que é a fórma exacta)
\section{Sequinhoso}
\begin{itemize}
\item {Grp. gram.:adj.}
\end{itemize}
\begin{itemize}
\item {Utilização:Ant.}
\end{itemize}
\begin{itemize}
\item {Proveniência:(De \textunderscore sêco\textunderscore )}
\end{itemize}
Sequioso.
Sêco:«\textunderscore ...sequinhosa areia...\textunderscore »R. Lobo, \textunderscore Côrte na Aldeia\textunderscore , I, 99.
\section{Sequiosamente}
\begin{itemize}
\item {Grp. gram.:adv.}
\end{itemize}
De modo sequioso; anciosamente; avidamente.
\section{Sequioso}
\begin{itemize}
\item {Grp. gram.:adj.}
\end{itemize}
\begin{itemize}
\item {Utilização:Fig.}
\end{itemize}
\begin{itemize}
\item {Proveniência:(De \textunderscore sêco\textunderscore )}
\end{itemize}
Que tem sêde ou grande desejo de água.
Que mostra falta de água; muito sêco: \textunderscore searas sequiosas\textunderscore .
Ávido; cubiçoso.
\section{Sequista}
\begin{itemize}
\item {Grp. gram.:m.  e  adj.}
\end{itemize}
\begin{itemize}
\item {Utilização:Bras}
\end{itemize}
\begin{itemize}
\item {Utilização:fam.}
\end{itemize}
\begin{itemize}
\item {Proveniência:(De \textunderscore secar\textunderscore )}
\end{itemize}
Indivíduo secante, maçador.
\section{Sequititi}
\begin{itemize}
\item {Grp. gram.:f.}
\end{itemize}
Formiga do Peru, que destrói qualquer classe de insectos.
\section{Séquito}
\begin{itemize}
\item {Grp. gram.:m.}
\end{itemize}
\begin{itemize}
\item {Utilização:Fig.}
\end{itemize}
\begin{itemize}
\item {Utilização:Ant.}
\end{itemize}
\begin{itemize}
\item {Proveniência:(Do lat. \textunderscore sequi\textunderscore )}
\end{itemize}
Conjunto do pessôas, que acompanham outra ou outras, por dever ou cortesia.
Acompanhamento, cortejo.
Seguimento.
Popularidade.
Acto de perseguir o inimigo.
\section{Sêr}
\begin{itemize}
\item {Grp. gram.:v. i.}
\end{itemize}
\begin{itemize}
\item {Grp. gram.:V. p.}
\end{itemize}
\begin{itemize}
\item {Grp. gram.:M.}
\end{itemize}
\begin{itemize}
\item {Grp. gram.:Loc. adj.}
\end{itemize}
\begin{itemize}
\item {Grp. gram.:Pl.}
\end{itemize}
\begin{itemize}
\item {Proveniência:(Do lat. \textunderscore sedere\textunderscore )}
\end{itemize}
Têr uma qualidade ou um modo de existir, indicado pelo adjectivo que segue mediata ou immediatamente o verbo: \textunderscore sêr talentoso\textunderscore .
Estar.
Existir.
Têr existência real, (falando se de Deus)
Pertencer: \textunderscore aquillo é do tio\textunderscore .
Acontecer: \textunderscore choras? então que foi\textunderscore ?
Têr a natureza de; ser formado de; consistir: \textunderscore esta mesa é de pedra\textunderscore .
(O mesmo sign.). Cf. Castilho, \textunderscore Tartufo\textunderscore , 65; Camillo, \textunderscore Brasileira\textunderscore , 84.
Aquelle ou aquillo que é.
Ente; existência.
Qualidade do que é.
Consciência de si próprio.
Realidade.--Em muitos casos, palavra expletiva.
\textunderscore Em sêr\textunderscore , que ainda não foi vendido, que está disponivel: \textunderscore as fazendas em sêr, naquelle estabelecimento, cobrem o passivo\textunderscore .
Na própria substância ou espécie:«\textunderscore pagavam o foro de cevada em sêr\textunderscore ». P. Carvalho, \textunderscore Corogr. Port.\textunderscore , I, 456.
Tudo que existe; tudo que foi criado.
\section{Seracoto}
\begin{itemize}
\item {fónica:cô}
\end{itemize}
\begin{itemize}
\item {Grp. gram.:adj.}
\end{itemize}
\begin{itemize}
\item {Utilização:T. do Fundão}
\end{itemize}
Diz-se do animal, que tem o rabo cortado.
\section{Serafanada}
\begin{itemize}
\item {Grp. gram.:f.}
\end{itemize}
O mesmo que \textunderscore piadeira\textunderscore .
\section{Serafina}
\begin{itemize}
\item {Grp. gram.:f.}
\end{itemize}
Tecido de lan, próprio para forros.
Espécie de baêta encorpada, geralmente com desenhos ou debuxos.
\section{Seral}
\begin{itemize}
\item {Grp. gram.:adj.}
\end{itemize}
\begin{itemize}
\item {Utilização:Neol.}
\end{itemize}
\begin{itemize}
\item {Proveniência:(Do lat. \textunderscore serus\textunderscore )}
\end{itemize}
Relativo á noite.
Que se faz ou succede durante a noite. (Us. especialmente nos theatros)
\textunderscore Despesas seraes\textunderscore , as que se fazem com as luzes e com os serviços inherentes a uma representação theatral de noite.
\section{Seráfica}
\begin{itemize}
\item {Grp. gram.:f.}
\end{itemize}
\begin{itemize}
\item {Proveniência:(De \textunderscore seráfico\textunderscore )}
\end{itemize}
A Ordem dos frades franciscanos. Cf. Filinto, XIII, 7.
\section{Seraficamente}
\begin{itemize}
\item {Grp. gram.:adv.}
\end{itemize}
De modo seráfico.
Beatificamente.
Á maneira de pessôa muito devota.
\section{Seráfico}
\begin{itemize}
\item {Grp. gram.:adj.}
\end{itemize}
\begin{itemize}
\item {Utilização:Fig.}
\end{itemize}
\begin{itemize}
\item {Proveniência:(Do lat. \textunderscore seraph\textunderscore )}
\end{itemize}
Relativo aos serafins.
Beatífico; que tem modos de devoto.
\section{Serafim}
\begin{itemize}
\item {Grp. gram.:m.}
\end{itemize}
\begin{itemize}
\item {Proveniência:(Lat. \textunderscore seraphim\textunderscore , pl. de \textunderscore seraph\textunderscore )}
\end{itemize}
Anjo da primeira gerarquia.
\section{Serafita}
\begin{itemize}
\item {Grp. gram.:f.}
\end{itemize}
\begin{itemize}
\item {Proveniência:(Do gr. \textunderscore seira\textunderscore  + \textunderscore phuton\textunderscore )}
\end{itemize}
Gênero de orquídeas.
\section{Seral}
\begin{itemize}
\item {Grp. gram.:f.}
\end{itemize}
\begin{itemize}
\item {Utilização:Prov.}
\end{itemize}
\begin{itemize}
\item {Utilização:alent.}
\end{itemize}
(ou antes \textunderscore ceral\textunderscore ?)
Variedade de azeitona, semelhante á lentisqueira. Cf. \textunderscore Port. au point de vue agr.\textunderscore , 481.
\section{Seramangar}
\begin{itemize}
\item {Grp. gram.:v. i.}
\end{itemize}
\begin{itemize}
\item {Utilização:Prov.}
\end{itemize}
\begin{itemize}
\item {Utilização:trasm.}
\end{itemize}
Andar devagar, arrastando os sócos.
\section{Serampo}
\begin{itemize}
\item {Grp. gram.:m.}
\end{itemize}
\begin{itemize}
\item {Utilização:ant.}
\end{itemize}
\begin{itemize}
\item {Utilização:Pop.}
\end{itemize}
O mesmo que \textunderscore sarampo\textunderscore . Cf. B. Pereira, \textunderscore Prosódia\textunderscore , vb. \textunderscore exanthemata\textunderscore .
\section{Serancolino}
\begin{itemize}
\item {Grp. gram.:m.}
\end{itemize}
\begin{itemize}
\item {Proveniência:(De \textunderscore Sarancolino\textunderscore , n. p.)}
\end{itemize}
Mármore dos Pyrenéus, da côr da ágata.
\section{Serandar}
\begin{itemize}
\item {Grp. gram.:v. i.}
\end{itemize}
\begin{itemize}
\item {Utilização:Prov.}
\end{itemize}
O mesmo que \textunderscore seroar\textunderscore . Cf. Júl. Dinis, \textunderscore Serões\textunderscore , 154.
\section{Serandeiro}
\begin{itemize}
\item {Grp. gram.:m.}
\end{itemize}
\begin{itemize}
\item {Utilização:Prov.}
\end{itemize}
\begin{itemize}
\item {Proveniência:(De \textunderscore serandar\textunderscore )}
\end{itemize}
Aquelle que faz serão. Cf. Júl. Dinis, \textunderscore Pupillas\textunderscore , 176 e 143; \textunderscore Fidalgos\textunderscore , II, 89.
\section{Seranzar}
\begin{itemize}
\item {Grp. gram.:v. i.}
\end{itemize}
\begin{itemize}
\item {Utilização:Prov.}
\end{itemize}
\begin{itemize}
\item {Utilização:minh.}
\end{itemize}
O mesmo que \textunderscore seroar\textunderscore .
\section{Serão}
\begin{itemize}
\item {Grp. gram.:m.}
\end{itemize}
\begin{itemize}
\item {Proveniência:(Do lat. \textunderscore serum\textunderscore )}
\end{itemize}
Trabalho, feito de noite.
Duração dêsse trabalho.
Retribuição delle.
O mesmo que \textunderscore sarau\textunderscore .
Recreio ou passatempo, nas primeiras horas da noite, dentro de casa, muitas vezes á lareira, entre pessôas de família ou gente amiga.
\section{Seráphica}
\begin{itemize}
\item {Grp. gram.:f.}
\end{itemize}
\begin{itemize}
\item {Proveniência:(De \textunderscore seráphico\textunderscore )}
\end{itemize}
A Ordem dos frades franciscanos. Cf. Filinto, XIII, 7.
\section{Seraphicamente}
\begin{itemize}
\item {Grp. gram.:adv.}
\end{itemize}
De modo seráphico.
Beatificamente.
Á maneira de pessôa muito devota.
\section{Seráphico}
\begin{itemize}
\item {Grp. gram.:adj.}
\end{itemize}
\begin{itemize}
\item {Utilização:Fig.}
\end{itemize}
\begin{itemize}
\item {Proveniência:(Do lat. \textunderscore seraph\textunderscore )}
\end{itemize}
Relativo aos seraphins.
Beatífico; que tem modos de devoto.
\section{Seraphim}
\begin{itemize}
\item {Grp. gram.:m.}
\end{itemize}
\begin{itemize}
\item {Proveniência:(Lat. \textunderscore seraphim\textunderscore , pl. de \textunderscore seraph\textunderscore )}
\end{itemize}
Anjo da primeira gerarchia.
\section{Seraphyta}
\begin{itemize}
\item {Grp. gram.:f.}
\end{itemize}
\begin{itemize}
\item {Proveniência:(Do gr. \textunderscore seira\textunderscore  + \textunderscore phuton\textunderscore )}
\end{itemize}
Gênero de orchídeas.
\section{Serápia}
\begin{itemize}
\item {Grp. gram.:f.}
\end{itemize}
\begin{itemize}
\item {Proveniência:(Do gr. \textunderscore serapias\textunderscore )}
\end{itemize}
Gênero de orchídeas.
\section{Serapilheira}
\begin{itemize}
\item {Grp. gram.:f.}
\end{itemize}
\begin{itemize}
\item {Utilização:Bras}
\end{itemize}
\begin{itemize}
\item {Proveniência:(Do b. lat. \textunderscore sarapilleria\textunderscore )}
\end{itemize}
Tecido grosseiro, para envolver fardos.
Pano grosso, para limpeza ou lavagem de casas.
Tecido grosso, applicado por camponeses em seus vestidos.
Planta, que nasce em terrenos de inferior qualidade.
\section{Serapino}
\begin{itemize}
\item {Grp. gram.:m.}
\end{itemize}
\begin{itemize}
\item {Utilização:Des.}
\end{itemize}
\begin{itemize}
\item {Proveniência:(De \textunderscore serápia\textunderscore )}
\end{itemize}
Espécie de goma medicinal, extrahida de uma orchidea.
\section{Seracomás}
\begin{itemize}
\item {Grp. gram.:m. pl.}
\end{itemize}
Indígenas do norte do Brasil.
\section{Sereia}
\begin{itemize}
\item {Grp. gram.:f.}
\end{itemize}
\begin{itemize}
\item {Utilização:Fig.}
\end{itemize}
\begin{itemize}
\item {Proveniência:(Do lat. \textunderscore sirena\textunderscore )}
\end{itemize}
Sêr mythológico, metade mulhér e metade peixe, que, pela doçura do seu canto, attrahia os navegantes para os escolhos do mar da Sicilia.
Mulhér, que canta suavissimamente, Mulhér encantadora ou de voz melodiosa.
Reptil, semelhante á salamandra.
Instrumento, com que se determina o número das vibrações de um som.
Espécie de golfinho dos rios da Guiana inglesa.
Apparelho, que, nos automóveis e nos navios, produz sons mais ou menos estridentes, para avisar da aproximação do vehículo. Cf. Benevides, \textunderscore Automóveis\textunderscore .
\section{Sereíba}
\begin{itemize}
\item {Grp. gram.:f.}
\end{itemize}
Variedade de mangue, (\textunderscore laguneularia ramosa\textunderscore ).
\section{Sereibuno}
\begin{itemize}
\item {Grp. gram.:m.}
\end{itemize}
\begin{itemize}
\item {Utilização:Bras}
\end{itemize}
Espécie de mangue bravo.
\section{Serelepe}
\begin{itemize}
\item {fónica:lê}
\end{itemize}
\begin{itemize}
\item {Grp. gram.:m.}
\end{itemize}
\begin{itemize}
\item {Utilização:Bras}
\end{itemize}
O mesmo que \textunderscore caxinguelê\textunderscore .
\section{Serena}
\begin{itemize}
\item {Grp. gram.:f.}
\end{itemize}
Espécie de batedeira, de movimento sereno.
(Fem. de \textunderscore sereno\textunderscore ^1)
\section{Serenada}
\begin{itemize}
\item {Grp. gram.:f.}
\end{itemize}
O mesmo que \textunderscore serenata\textunderscore .
\section{Serenagem}
\begin{itemize}
\item {Grp. gram.:f.}
\end{itemize}
\begin{itemize}
\item {Proveniência:(De \textunderscore serenar\textunderscore )}
\end{itemize}
Acto de pôr sereno.
\section{Serenamente}
\begin{itemize}
\item {Grp. gram.:adv.}
\end{itemize}
De modo sereno^1; tranquillamente; em paz.
\section{Serenar}
\begin{itemize}
\item {Grp. gram.:v. t.}
\end{itemize}
\begin{itemize}
\item {Grp. gram.:V. i.}
\end{itemize}
\begin{itemize}
\item {Utilização:Bras. do N}
\end{itemize}
\begin{itemize}
\item {Proveniência:(Lat. \textunderscore serenare\textunderscore )}
\end{itemize}
Tornar sereno.
Applacar; acalmar; pacificar.
Tornar-se sereno; tranquillizar-se.
Chuviscar.
\section{Serenata}
\begin{itemize}
\item {Grp. gram.:f.}
\end{itemize}
\begin{itemize}
\item {Proveniência:(It. \textunderscore serenata\textunderscore )}
\end{itemize}
Concêrto musical, de noite e ao ar livre.
Composição musical, melodiosa, graciosa e simples, mais ou menos análoga ás trovas dos cantores ambulantes.
\section{Serenatista}
\begin{itemize}
\item {Utilização:bras}
\end{itemize}
\begin{itemize}
\item {Utilização:Neol.}
\end{itemize}
Aquelle que toca ou descanta em serenatas.
\section{Serenidade}
\begin{itemize}
\item {Grp. gram.:f.}
\end{itemize}
\begin{itemize}
\item {Proveniência:(Do lat. \textunderscore serenitas\textunderscore )}
\end{itemize}
Qualidade ou estado do que é sereno; tranquillidade; paz; suavidade.
\section{Serenim}
\begin{itemize}
\item {Grp. gram.:m.}
\end{itemize}
\begin{itemize}
\item {Utilização:Ant.}
\end{itemize}
Antigo vestuário de senhora.
Antiga canção portuguesa.
O mesmo que \textunderscore serenata\textunderscore .
Sarau, que cantavam pessôas reaes: \textunderscore nos serenins de Queluz...\textunderscore  Cf. \textunderscore Panorama\textunderscore , XII, 79.
\section{Sereníssimo}
\begin{itemize}
\item {Grp. gram.:adj.}
\end{itemize}
\begin{itemize}
\item {Proveniência:(Lat. \textunderscore serenissimus\textunderscore )}
\end{itemize}
Muito sereno.
Titulo honorífico dos Infantes.
Titulo honorífico da Casa de Bragança.
\section{Sereno}
\begin{itemize}
\item {Grp. gram.:adj.}
\end{itemize}
\begin{itemize}
\item {Grp. gram.:M.}
\end{itemize}
\begin{itemize}
\item {Utilização:Bras. do S}
\end{itemize}
\begin{itemize}
\item {Proveniência:(Lat. \textunderscore serenus\textunderscore )}
\end{itemize}
Limpo de nuvens; calmo, tranquillo.
Claro; puro.
Que mostra serenidade de espírito.
Vapor atmosphérico, ligeiro ou pouco espêsso, que se resolve em chuva finíssima.
Humidade atmosphérica, peculiar a algumas noites claras do verão; o mesmo que \textunderscore relento\textunderscore .
Chuva miúda.
\section{Sereno}
\begin{itemize}
\item {Grp. gram.:m.}
\end{itemize}
Indivíduo que, em Espanha, á semelhança dos nossos guardas-nocturnos, ronda de noite as ruas, para annunciar incêndios, evitar desordens ou roubos e dizer em voz alta, de quando em quando, o estado do tempo e as horas que são.
Cocheiro que, em Lisbôa, faz serviço de noite.
Serviço noturno de cocheiro.
(Cast. \textunderscore sereno\textunderscore )
\section{Sereno}
\begin{itemize}
\item {Grp. gram.:m.}
\end{itemize}
\begin{itemize}
\item {Utilização:Prov.}
\end{itemize}
O mesmo que \textunderscore milheira\textunderscore , ave.
\section{Séres}
\begin{itemize}
\item {Grp. gram.:m. pl.}
\end{itemize}
\begin{itemize}
\item {Proveniência:(Lat. \textunderscore Seres\textunderscore )}
\end{itemize}
Nome, que os Romanos deram aos habitantes da Ásia oriental, hoje China, célebres pela fabricação de estofos de seda.
\section{Seresma}
\begin{itemize}
\item {fónica:serês}
\end{itemize}
\begin{itemize}
\item {Grp. gram.:f.}
\end{itemize}
\begin{itemize}
\item {Grp. gram.:M.}
\end{itemize}
Mulhér fraca ou indolente e inútil.
Mulhér velha e feia.
Qualquer coisa nojenta.
O mesmo que \textunderscore paspalhão\textunderscore . Cf. Castilho, \textunderscore Fausto\textunderscore , 340; Rebelllo, \textunderscore Mocidade\textunderscore , I, 182; II, 159.
\section{Serezino}
\begin{itemize}
\item {Grp. gram.:m.}
\end{itemize}
\begin{itemize}
\item {Utilização:Prov.}
\end{itemize}
\begin{itemize}
\item {Utilização:minh.}
\end{itemize}
Pequeno pássaro, de canto monótono.
Espécie de milheira.
(Cp. \textunderscore sereno\textunderscore ^3)
\section{Sergantana}
\begin{itemize}
\item {Grp. gram.:f.}
\end{itemize}
(V.lagartixa)
\section{Sergenta}
\begin{itemize}
\item {Grp. gram.:f.}
\end{itemize}
\begin{itemize}
\item {Utilização:Ant.}
\end{itemize}
(Fem. de \textunderscore sergente\textunderscore )
\section{Sergente}
\textunderscore m.\textunderscore  e \textunderscore f. Ant\textunderscore .
O mesmo ou melhor que \textunderscore sargente\textunderscore .
\section{Sergesto}
\begin{itemize}
\item {Grp. gram.:m.}
\end{itemize}
Gênero de crustáceos decápodes.
\section{Sergeta}
\begin{itemize}
\item {fónica:gê}
\end{itemize}
\begin{itemize}
\item {Grp. gram.:f.}
\end{itemize}
\begin{itemize}
\item {Utilização:Pop.}
\end{itemize}
Rapariga buliçosa, inquieta, sirigaita.
(Cp. \textunderscore sergenta\textunderscore )
\section{Sérgia}
\begin{itemize}
\item {Grp. gram.:f.}
\end{itemize}
\begin{itemize}
\item {Proveniência:(Lat. \textunderscore sergia\textunderscore )}
\end{itemize}
Variedade de azeitona, conhecida dos antigos, a mesma que chamamos \textunderscore negral\textunderscore .
\section{Sergipano}
\begin{itemize}
\item {Grp. gram.:adj.}
\end{itemize}
\begin{itemize}
\item {Grp. gram.:M.}
\end{itemize}
Relativo ao Estado de Sergipe, no Brasil.
Habitante dêsse Estado.
\section{Sergir}
\begin{itemize}
\item {Grp. gram.:v. t.}
\end{itemize}
\begin{itemize}
\item {Utilização:pop.}
\end{itemize}
\begin{itemize}
\item {Utilização:Ant.}
\end{itemize}
O mesmo que \textunderscore serzir\textunderscore . Cf. P. Pereira, \textunderscore Prosódia\textunderscore , vb. \textunderscore sonsarcino\textunderscore .
\section{Serguia}
\begin{itemize}
\item {Grp. gram.:f.}
\end{itemize}
\begin{itemize}
\item {Utilização:Mad}
\end{itemize}
O mesmo que \textunderscore seriguilha\textunderscore :«\textunderscore ó bruto do campo..., trajando serguia\textunderscore ». Alv. Azevedo, \textunderscore Cancion da Madeira\textunderscore , 36.
\section{Serguilha}
\begin{itemize}
\item {Grp. gram.:f.}
\end{itemize}
\begin{itemize}
\item {Utilização:Prov.}
\end{itemize}
\begin{itemize}
\item {Utilização:trasm.}
\end{itemize}
Fórma pop. de \textunderscore seriguilha\textunderscore .
O mesmo que \textunderscore rodilho\textunderscore .
(Cp. cast. \textunderscore jerguilla\textunderscore )
\section{Sergulhal}
\begin{itemize}
\item {Grp. gram.:m.}
\end{itemize}
\begin{itemize}
\item {Utilização:Prov.}
\end{itemize}
\begin{itemize}
\item {Utilização:minh.}
\end{itemize}
Encaixe inferior da mó do moínho.
\section{Seriação}
\begin{itemize}
\item {Grp. gram.:f.}
\end{itemize}
\begin{itemize}
\item {Utilização:Neol.}
\end{itemize}
\begin{itemize}
\item {Proveniência:(De \textunderscore série\textunderscore )}
\end{itemize}
Acto de collocar ou dispor certas coisas em série. Cf. Fred. Laranjo, \textunderscore Direito Polit.\textunderscore , 3.
\section{Serial}
\begin{itemize}
\item {Grp. gram.:adj.}
\end{itemize}
Relativo a série.
Disposto em série.
\section{Serialária}
\begin{itemize}
\item {Grp. gram.:f.}
\end{itemize}
Gênero de polypeiros.
\section{Seriamente}
\begin{itemize}
\item {Grp. gram.:adv.}
\end{itemize}
De modo sério; a valer; realmente.
Com gravidade.
Com sisudez.
\section{Seriar}
\begin{itemize}
\item {Grp. gram.:v. t.}
\end{itemize}
\begin{itemize}
\item {Utilização:Neol.}
\end{itemize}
Dispor em séries.
Fazer a classificação de.
\section{Seriário}
\begin{itemize}
\item {Grp. gram.:adj.}
\end{itemize}
Relativo a uma série; que se faz por séries.
\section{Seriatóporo}
\begin{itemize}
\item {Grp. gram.:m.}
\end{itemize}
Gênero de polypeiros.
\section{Seribolo}
\begin{itemize}
\item {fónica:bô}
\end{itemize}
\begin{itemize}
\item {Grp. gram.:m.}
\end{itemize}
\begin{itemize}
\item {Utilização:Bras. do N}
\end{itemize}
Barulho.
Desordem.
\section{Sericaia}
\begin{itemize}
\item {Grp. gram.:f.}
\end{itemize}
Iguaria fina de Malaca.
\section{Sericatos}
\begin{itemize}
\item {Grp. gram.:m. pl.}
\end{itemize}
\begin{itemize}
\item {Proveniência:(Lat. \textunderscore sericatus\textunderscore )}
\end{itemize}
Família de pássaros, que comprehende os que têm o bico curto, fendido e deprimido na base.
\section{Seríceo}
\begin{itemize}
\item {Grp. gram.:adj.}
\end{itemize}
\begin{itemize}
\item {Utilização:Poét.}
\end{itemize}
\begin{itemize}
\item {Proveniência:(Lat. \textunderscore sericeus\textunderscore )}
\end{itemize}
Relativo a sêda.
Feito de sêda.
Que tem a apparência de sêda.
\section{Sericícola}
\begin{itemize}
\item {Grp. gram.:adj.}
\end{itemize}
\begin{itemize}
\item {Grp. gram.:M.  e  f.}
\end{itemize}
\begin{itemize}
\item {Proveniência:(Do lat. \textunderscore sericum\textunderscore  + \textunderscore colere\textunderscore )}
\end{itemize}
Relativo á producção da sêda.
Pessôa, que trata da criação dos bichos de sêda.
Pessôa, que trata da preparação da sêda.
\section{Sericicultor}
\begin{itemize}
\item {Grp. gram.:m.  e  adj.}
\end{itemize}
\begin{itemize}
\item {Proveniência:(Do lat. \textunderscore sericum\textunderscore  + \textunderscore cultor\textunderscore )}
\end{itemize}
O que exerce a sericicultura.
O que promove a indústria da sêda.
\section{Sericicultura}
\begin{itemize}
\item {Grp. gram.:f.}
\end{itemize}
\begin{itemize}
\item {Proveniência:(Do lat. \textunderscore sericum\textunderscore  + \textunderscore cultura\textunderscore )}
\end{itemize}
Acto de preparar a sêda.
Fabricação da sêda.
\section{Sericígeno}
\begin{itemize}
\item {Grp. gram.:adj.}
\end{itemize}
\begin{itemize}
\item {Proveniência:(Do gr. \textunderscore serikos\textunderscore  + \textunderscore genos\textunderscore )}
\end{itemize}
Que produz sêda, (falando-se das espécies de bichos de sêda). Cf. \textunderscore Portugal Agrícola\textunderscore , XI, 356.
\section{Sericite}
\begin{itemize}
\item {Grp. gram.:f.}
\end{itemize}
\begin{itemize}
\item {Utilização:Miner.}
\end{itemize}
Variedade de moscovite hydratada e fluorífera.
\section{Sericito}
\begin{itemize}
\item {Grp. gram.:m.}
\end{itemize}
O mesmo ou melhor que \textunderscore sericite\textunderscore .
\section{Sérico}
\begin{itemize}
\item {Grp. gram.:adj.}
\end{itemize}
\begin{itemize}
\item {Proveniência:(Lat. \textunderscore sericus\textunderscore )}
\end{itemize}
O mesmo que \textunderscore seríceo\textunderscore .
\section{Sericocarpo}
\begin{itemize}
\item {Grp. gram.:m.}
\end{itemize}
\begin{itemize}
\item {Proveniência:(Do gr. \textunderscore serikos\textunderscore  + \textunderscore karpos\textunderscore )}
\end{itemize}
Gênero de plantas, da fam. das compostas.
\section{Sericogastro}
\begin{itemize}
\item {Grp. gram.:m.}
\end{itemize}
\begin{itemize}
\item {Proveniência:(Do gr. \textunderscore serikos\textunderscore  + \textunderscore gaster\textunderscore )}
\end{itemize}
Gênero de insectos hymenópteros.
\section{Serícola}
\begin{itemize}
\item {Grp. gram.:m.  e  adj.}
\end{itemize}
O mesmo que \textunderscore sericultor\textunderscore . Cf. Júl. Dinís, \textunderscore Morgadinha\textunderscore , 219.
(Contr. \textunderscore sericíola\textunderscore )
\section{Sericólitho}
\begin{itemize}
\item {Grp. gram.:m.}
\end{itemize}
\begin{itemize}
\item {Utilização:Miner.}
\end{itemize}
\begin{itemize}
\item {Proveniência:(Do gr. \textunderscore serikos\textunderscore  + \textunderscore lithos\textunderscore )}
\end{itemize}
Variedade de calcário.
\section{Sericólito}
\begin{itemize}
\item {Grp. gram.:m.}
\end{itemize}
\begin{itemize}
\item {Utilização:Miner.}
\end{itemize}
\begin{itemize}
\item {Proveniência:(Do gr. \textunderscore serikos\textunderscore  + \textunderscore lithos\textunderscore )}
\end{itemize}
Variedade de calcário.
\section{Sericóstoma}
\begin{itemize}
\item {Grp. gram.:f.}
\end{itemize}
\begin{itemize}
\item {Proveniência:(Do gr. \textunderscore serikos\textunderscore  + \textunderscore stoma\textunderscore )}
\end{itemize}
Gênero de insectos neurópteros.
\section{Sericultor}
\begin{itemize}
\item {Grp. gram.:m.  e  adj.}
\end{itemize}
(Contr. de \textunderscore sericicultor\textunderscore )
\section{Sericultura}
\begin{itemize}
\item {Grp. gram.:f.}
\end{itemize}
(Contr. de \textunderscore sericicultura\textunderscore ) Cf. Júl. Dinís, \textunderscore Morgadinha\textunderscore , 422.
\section{Sericura}
\begin{itemize}
\item {Grp. gram.:f.}
\end{itemize}
\begin{itemize}
\item {Utilização:Bras}
\end{itemize}
Ave ribeirinha.
\section{Séride}
\begin{itemize}
\item {Grp. gram.:f.}
\end{itemize}
\begin{itemize}
\item {Proveniência:(Do lat. \textunderscore seris\textunderscore , \textunderscore seridis\textunderscore )}
\end{itemize}
Gênero de plantas, da fam. das compostas.
\section{Serídea}
\begin{itemize}
\item {Grp. gram.:f.}
\end{itemize}
\begin{itemize}
\item {Proveniência:(De \textunderscore séride\textunderscore )}
\end{itemize}
Gênero das plantas, da fam. das compostas.
\section{Série}
\begin{itemize}
\item {Grp. gram.:f.}
\end{itemize}
\begin{itemize}
\item {Proveniência:(Lat. \textunderscore series\textunderscore )}
\end{itemize}
Em Mathemática, successão de grandezas, que crescem ou deminuem, segundo uma lei.
Seguimento.
Ordem de factos ou coisas da mesma natureza, classificadas pela mesma fórma e segundo a mesma lei.
Conjunto de corpos homólogos, (em Chímica).
Disposição zoológica, em que se passa successivamente de um grupo de animaes com organização menos complicada para um grupo, que tem organização mais complicada.
Reunião de objectos, que servem para sinaes marítimos.
Na Philosophia positiva, classificação ou disposição das sciências fundamentaes e abstractas, occupando lugar superior as mais complicadas.
No fourierismo, conjunto de trabalhadores, classificados por grupos e applicados a determinada ordem de funcções.
\section{Seriedade}
\begin{itemize}
\item {Grp. gram.:f.}
\end{itemize}
\begin{itemize}
\item {Proveniência:(Do lat. \textunderscore serietas\textunderscore )}
\end{itemize}
Qualidade daquelle ou daquillo que é sério.
Gravidade de porte.
Integridade de carácter.
\section{Seriema}
\begin{itemize}
\item {Grp. gram.:f.}
\end{itemize}
\begin{itemize}
\item {Utilização:Bras}
\end{itemize}
Espécie de pequena ema.
\section{Serífio}
\begin{itemize}
\item {Grp. gram.:m.}
\end{itemize}
Gênero de plantas, da fam. das compostas.
\section{Seriga}
\begin{itemize}
\item {Grp. gram.:f.}
\end{itemize}
(Parece êrro gráphico, cometido no \textunderscore Elucidário\textunderscore  de Viterbo, e reproduzido algures, em vez de \textunderscore sésiga\textunderscore . V. \textunderscore sésiga\textunderscore ). Cf. S. R. Viterbo, \textunderscore Elucidário\textunderscore .
\section{Serigado}
\begin{itemize}
\item {Grp. gram.:m.}
\end{itemize}
\begin{itemize}
\item {Utilização:Bras. do N}
\end{itemize}
Peixe de água salgada.
\section{Serigaita}
\begin{itemize}
\item {Grp. gram.:f.}
\end{itemize}
(V.sirigaita). Cf. Castilho, \textunderscore Fausto\textunderscore , 370.
\section{Serigaria}
\begin{itemize}
\item {Grp. gram.:f.}
\end{itemize}
Fábrica ou estabelecimento de serigueiro.
\section{Serigola}
\begin{itemize}
\item {Grp. gram.:f.}
\end{itemize}
\begin{itemize}
\item {Utilização:Bras. do N}
\end{itemize}
Correia das cabeçadas, que passa pelo pescoço das cavalgaduras.
(Cp. \textunderscore serigote\textunderscore )
\section{Serigolla}
\begin{itemize}
\item {Grp. gram.:f.}
\end{itemize}
\begin{itemize}
\item {Utilização:Bras. do N}
\end{itemize}
Correia das cabeçadas, que passa pelo pescoço das cavalgaduras.
(Cp. \textunderscore serigote\textunderscore )
\section{Serigote}
\begin{itemize}
\item {Grp. gram.:m.}
\end{itemize}
\begin{itemize}
\item {Utilização:Bras. do S}
\end{itemize}
Lombilho curto.
\section{Serigueiro}
\begin{itemize}
\item {Grp. gram.:m.}
\end{itemize}
\begin{itemize}
\item {Proveniência:(Do lat. \textunderscore sericarius\textunderscore )}
\end{itemize}
Aquelle que faz obras de sêda; sirgueiro.
\section{Seriguilha}
\begin{itemize}
\item {Grp. gram.:f.}
\end{itemize}
\begin{itemize}
\item {Proveniência:(Do lat. \textunderscore sericum\textunderscore )}
\end{itemize}
Pano grosso de lan, sem pêlo.
\section{Serina}
\begin{itemize}
\item {Grp. gram.:f.}
\end{itemize}
\begin{itemize}
\item {Proveniência:(Do lat. \textunderscore serum\textunderscore )}
\end{itemize}
O mesmo que \textunderscore pyína\textunderscore .
Albumina do sôro.
\section{Serineta}
\begin{itemize}
\item {Grp. gram.:f.}
\end{itemize}
Gênero de insectos hemípteros.
\section{Serinetha}
\begin{itemize}
\item {Grp. gram.:f.}
\end{itemize}
Gênero de insectos hemípteros.
\section{Seringa}
\begin{itemize}
\item {Grp. gram.:f.}
\end{itemize}
\begin{itemize}
\item {Grp. gram.:M.  e  f.}
\end{itemize}
\begin{itemize}
\item {Utilização:Pop.}
\end{itemize}
\begin{itemize}
\item {Grp. gram.:Adj.}
\end{itemize}
\begin{itemize}
\item {Proveniência:(Do lat. \textunderscore syringa\textunderscore )}
\end{itemize}
Bomba portátil, que attrai e expelle o ar e os líquidos, e que serve especialmente para introduzir líquidos nas cavidades interiores do corpo.
Leite da seringueira, (\textunderscore hevea brasiliensis\textunderscore ), ainda não coagulado.
Pessôa importuna ou esquisita.
Diz-se do pau de uma árvore que produz borracha.
\section{Seringação}
\begin{itemize}
\item {Grp. gram.:f.}
\end{itemize}
Acto ou effeito de seringar.
\section{Seringada}
\begin{itemize}
\item {Grp. gram.:f.}
\end{itemize}
\begin{itemize}
\item {Proveniência:(De \textunderscore seringar\textunderscore )}
\end{itemize}
Expulsão do líquido, contido na seringa.
Seringação.
\section{Seringadela}
\begin{itemize}
\item {Grp. gram.:f.}
\end{itemize}
O mesmo que \textunderscore seringação\textunderscore .
\section{Seringal}
\begin{itemize}
\item {Grp. gram.:m.}
\end{itemize}
\begin{itemize}
\item {Utilização:Bras}
\end{itemize}
\begin{itemize}
\item {Proveniência:(De \textunderscore seringa\textunderscore )}
\end{itemize}
Mata de seringueiras.
\section{Seringar}
\begin{itemize}
\item {Grp. gram.:v. t.}
\end{itemize}
\begin{itemize}
\item {Utilização:Pop.}
\end{itemize}
Injectar o líquido da seringa em.
Molhar com o líquido da seringa.
Importunar; maçar.
\section{Seringatório}
\begin{itemize}
\item {Grp. gram.:adj.}
\end{itemize}
\begin{itemize}
\item {Grp. gram.:M.}
\end{itemize}
\begin{itemize}
\item {Proveniência:(De \textunderscore seringar\textunderscore )}
\end{itemize}
Relativo á seringa.
Medicamento, injectado com seringa.
\section{Seringueira}
\begin{itemize}
\item {Grp. gram.:f.}
\end{itemize}
\begin{itemize}
\item {Proveniência:(De \textunderscore seringa\textunderscore )}
\end{itemize}
Árvore euphorbiácea, de que se extrai borracha.
\section{Seringueiro}
\begin{itemize}
\item {Grp. gram.:m.}
\end{itemize}
\begin{itemize}
\item {Utilização:Bras}
\end{itemize}
\begin{itemize}
\item {Proveniência:(De \textunderscore seringa\textunderscore )}
\end{itemize}
Aquelle que extrai o leite da seringueira e o prepara para se tornar em borracha.
\section{Sério}
\begin{itemize}
\item {Grp. gram.:adj.}
\end{itemize}
\begin{itemize}
\item {Utilização:Fam.}
\end{itemize}
\begin{itemize}
\item {Grp. gram.:M.}
\end{itemize}
\begin{itemize}
\item {Grp. gram.:Adv.}
\end{itemize}
\begin{itemize}
\item {Proveniência:(Lat. \textunderscore serius\textunderscore )}
\end{itemize}
Que tem gravidade ou sisudez.
Sensato.
Que procede com pontualidade.
Que cumpre aquillo a que se obriga.
Methódico.
Que revela sisudez ou é própria de pessôa sisuda.
Circunspecto.
Sincero; real.
Importante: \textunderscore com coisas sérias não se brinca\textunderscore .
Que se não ri: \textunderscore não te rias, põe-te sério\textunderscore .
Gravidade:«\textunderscore uma vez por outra saia do seu sério.\textunderscore »Camillo, \textunderscore Viuva do Enforc.\textunderscore , III, 29 e 30.
Seriamente; realmente: \textunderscore mas, sério, vocês não jantam\textunderscore ?
\section{Seríphio}
\begin{itemize}
\item {Grp. gram.:m.}
\end{itemize}
Gênero de plantas, da fam. das compostas.
\section{Serísico}
\begin{itemize}
\item {Grp. gram.:m.}
\end{itemize}
Gênero de insectos coleópteros.
\section{Serissa}
\begin{itemize}
\item {Grp. gram.:f.}
\end{itemize}
O mesmo que \textunderscore serisse\textunderscore .
\section{Serisse}
\begin{itemize}
\item {Grp. gram.:m.}
\end{itemize}
Árvore indiana, (\textunderscore bombax malabaricum\textunderscore ).
\section{Serjânia}
\begin{itemize}
\item {Grp. gram.:f.}
\end{itemize}
Gênero de plantas sapindáceas.
\section{Sermão}
\begin{itemize}
\item {Grp. gram.:m.}
\end{itemize}
\begin{itemize}
\item {Utilização:Fig.}
\end{itemize}
\begin{itemize}
\item {Proveniência:(Lat. \textunderscore sermo\textunderscore )}
\end{itemize}
Discurso christão, que se pronuncia no púlpito.
Prédica.
Admoestação, com o fim de moralizar ou tornar virtuoso.
Reprehensão, censura.
\section{Sermôa}
\begin{itemize}
\item {Grp. gram.:f.}
\end{itemize}
\begin{itemize}
\item {Utilização:Fam.}
\end{itemize}
\begin{itemize}
\item {Proveniência:(De \textunderscore sermão\textunderscore )}
\end{itemize}
Sermão de pouco valor.
Prédica.
\section{Sermoar}
\begin{itemize}
\item {Grp. gram.:v. t.}
\end{itemize}
\begin{itemize}
\item {Utilização:Ant.}
\end{itemize}
Exhortar.
Animar com bons conselhos.
(Cp. \textunderscore sermonar\textunderscore )
\section{Serfo}
\begin{itemize}
\item {Grp. gram.:m.}
\end{itemize}
Gênero de insectos himenópteros.
\section{Sermonar}
\begin{itemize}
\item {Grp. gram.:v. t.}
\end{itemize}
\begin{itemize}
\item {Proveniência:(Do lat. \textunderscore sermo\textunderscore , \textunderscore sermonis\textunderscore )}
\end{itemize}
Expor em fórma de sermão. Cf. Filinto, IV, 240.
\section{Sermonário}
\begin{itemize}
\item {Grp. gram.:m.}
\end{itemize}
\begin{itemize}
\item {Proveniência:(Do lat. \textunderscore sermo\textunderscore , \textunderscore sermonis\textunderscore )}
\end{itemize}
Collecção de sermões.
\section{Sermontésio}
\begin{itemize}
\item {Grp. gram.:adj.}
\end{itemize}
\begin{itemize}
\item {Utilização:Des.}
\end{itemize}
\begin{itemize}
\item {Proveniência:(Do lat. \textunderscore sermo\textunderscore  + \textunderscore mons\textunderscore ?)}
\end{itemize}
Feito em linguagem rústica, (falando-se de versos).
\section{Serna}
\begin{itemize}
\item {Grp. gram.:f.}
\end{itemize}
O mesmo que \textunderscore senra\textunderscore .
\section{Sernada}
\begin{itemize}
\item {Grp. gram.:f.}
\end{itemize}
O mesmo que \textunderscore senrada\textunderscore .
\section{Sernambi}
\begin{itemize}
\item {Grp. gram.:m.}
\end{itemize}
\begin{itemize}
\item {Utilização:Bras}
\end{itemize}
\begin{itemize}
\item {Proveniência:(T. tupi)}
\end{itemize}
Espécie de mollusco; amêijoa.
\section{Sernambi}
\begin{itemize}
\item {Grp. gram.:m.}
\end{itemize}
\begin{itemize}
\item {Utilização:Bras. do N}
\end{itemize}
O mesmo que \textunderscore sambaqui\textunderscore .
\section{Sernambi}
\begin{itemize}
\item {Grp. gram.:m.}
\end{itemize}
\begin{itemize}
\item {Utilização:Bras. do N}
\end{itemize}
Goma elástica, de inferior qualidade.
\section{Seró}
\begin{itemize}
\item {Grp. gram.:m.}
\end{itemize}
Pequena embarcação asiática. Cf. \textunderscore Peregrinação\textunderscore .
\section{Seroada}
\begin{itemize}
\item {Grp. gram.:f.}
\end{itemize}
\begin{itemize}
\item {Proveniência:(De \textunderscore seroar\textunderscore )}
\end{itemize}
Grande serão; serão.
\section{Seroar}
\begin{itemize}
\item {Grp. gram.:v. i.}
\end{itemize}
Fazer serão; trabalhar de noite.
\section{Serôdio}
\begin{itemize}
\item {Grp. gram.:adj.}
\end{itemize}
Que vem tarde ou a deshoras.
Tardio.
Que apparece no fim da estação própria, (falando-se de frutos).
(Cp. lat. \textunderscore serotinus\textunderscore )
\section{Seroeiro}
\begin{itemize}
\item {Grp. gram.:m.}
\end{itemize}
Aquelle que faz serão.
\section{Sérola}
\begin{itemize}
\item {Grp. gram.:f.}
\end{itemize}
Gênero de crustáceos isópodes.
\section{Seros}
\begin{itemize}
\item {Grp. gram.:m. pl.}
\end{itemize}
O mesmo que \textunderscore Séres\textunderscore .
\section{Serosa}
\begin{itemize}
\item {Grp. gram.:f.}
\end{itemize}
\begin{itemize}
\item {Utilização:T. de Viseu}
\end{itemize}
\begin{itemize}
\item {Proveniência:(De \textunderscore seroso\textunderscore )}
\end{itemize}
Membrana serosa.
Variedade de azeitona. Cp. \textunderscore seral\textunderscore ^2.
\section{Serosidade}
\begin{itemize}
\item {Grp. gram.:f.}
\end{itemize}
Qualidade do que é seroso.
Humor, segregado por certas membranas.
Humor, contido nas bolhas, produzidas pelos vesicatórios.
Parte aquosa dos humores animaes.
\section{Seroso}
\begin{itemize}
\item {Grp. gram.:adj.}
\end{itemize}
\begin{itemize}
\item {Proveniência:(Do lat. \textunderscore serum\textunderscore )}
\end{itemize}
Relativo a soro.
Que tem soro.
Aquoso.
\section{Seroterapêutica}
\begin{itemize}
\item {Grp. gram.:f.}
\end{itemize}
O mesmo que \textunderscore seroterapia\textunderscore .
\section{Seroterapia}
\begin{itemize}
\item {Grp. gram.:f.}
\end{itemize}
\begin{itemize}
\item {Proveniência:(Do lat. \textunderscore serum\textunderscore  + gr. \textunderscore therapeia\textunderscore )}
\end{itemize}
Sistema medicinal, que tem por base o soro.
\section{Seroterápico}
\begin{itemize}
\item {Grp. gram.:adj.}
\end{itemize}
Relativo á seroterapia.
\section{Serotherapêutica}
\begin{itemize}
\item {Grp. gram.:f.}
\end{itemize}
O mesmo que \textunderscore serotherapia\textunderscore .
\section{Serotherapia}
\begin{itemize}
\item {Grp. gram.:f.}
\end{itemize}
\begin{itemize}
\item {Proveniência:(Do lat. \textunderscore serum\textunderscore  + gr. \textunderscore therapeia\textunderscore )}
\end{itemize}
Systema medicinal, que tem por base o soro.
\section{Serotherápico}
\begin{itemize}
\item {Grp. gram.:adj.}
\end{itemize}
Relativo á serotherapia.
\section{Serpão}
\begin{itemize}
\item {Grp. gram.:m.}
\end{itemize}
Planta labiada, muito aromática.
(Cp. \textunderscore serpol\textunderscore )
\section{Serpe}
\begin{itemize}
\item {Grp. gram.:f.}
\end{itemize}
\begin{itemize}
\item {Utilização:Poét.}
\end{itemize}
O mesmo que \textunderscore serpente\textunderscore .
Antiga peça de artilharia.
*\textunderscore Archit.\textunderscore 
Linha de ornato, em fórma de serpente.
(B. lat. \textunderscore serpe\textunderscore )
\section{Serpeante}
\begin{itemize}
\item {Grp. gram.:adj.}
\end{itemize}
Que serpeia.
\section{Serpear}
\begin{itemize}
\item {Grp. gram.:v. i.}
\end{itemize}
\begin{itemize}
\item {Proveniência:(Do lat. \textunderscore serpere\textunderscore )}
\end{itemize}
Andar, arrastando-se pelo chão, como a serpente.
Mover-se sinuosamente, ondular: \textunderscore serpeiam regatos\textunderscore .
Sêr turtuoso: \textunderscore erguiam-se carvalhos serpeando\textunderscore .
\section{Serpejante}
\begin{itemize}
\item {Grp. gram.:adj.}
\end{itemize}
Que serpeja.
\section{Serpejar}
\begin{itemize}
\item {Grp. gram.:v. i.}
\end{itemize}
O mesmo que \textunderscore serpear\textunderscore . Cf. Camillo, \textunderscore Perfil do Marquês\textunderscore , 289.
\section{Serpejinoso}
\begin{itemize}
\item {Grp. gram.:adj.}
\end{itemize}
\begin{itemize}
\item {Utilização:Med.}
\end{itemize}
\begin{itemize}
\item {Proveniência:(De \textunderscore serpejar\textunderscore )}
\end{itemize}
\begin{itemize}
\item {Proveniência:(Cp. fr. \textunderscore serpigineux\textunderscore )
}
\end{itemize}
Diz-se das úlceras e das erysipelas, que se curam de um lado, aumentando do outro, parecendo que se desdobram, serpejando.
\section{Serpentante}
\begin{itemize}
\item {Grp. gram.:adj.}
\end{itemize}
Que serpenha.
\section{Serpentão}
\begin{itemize}
\item {Grp. gram.:m.}
\end{itemize}
\begin{itemize}
\item {Proveniência:(De \textunderscore serpente\textunderscore )}
\end{itemize}
Espécie de trombone antigo, cuja parte superior imitava o corpo e a cabeça de uma serpente.
\section{Serpentar}
\begin{itemize}
\item {Grp. gram.:v. i.}
\end{itemize}
O mesmo que \textunderscore serpear\textunderscore .
\section{Serpentária}
\begin{itemize}
\item {Grp. gram.:f.}
\end{itemize}
\begin{itemize}
\item {Proveniência:(Lat. \textunderscore serpentaria\textunderscore )}
\end{itemize}
Nome de algumas plantas aráceas, que se applicavam contra a mordedura das víboras.
Constellação, o mesmo que \textunderscore ophiúcho\textunderscore .
\section{Serpentáridas}
\begin{itemize}
\item {Grp. gram.:f. pl.}
\end{itemize}
Família de aves, que tem por typo o \textunderscore serpentário\textunderscore .
\section{Serpentário}
\begin{itemize}
\item {Grp. gram.:m.}
\end{itemize}
\begin{itemize}
\item {Proveniência:(De \textunderscore serpente\textunderscore )}
\end{itemize}
Ave de rapina, que se alimenta especialmente de serpentes.
Constellação boreal, também conhecida por \textunderscore serpentária\textunderscore .
\section{Serpente}
\begin{itemize}
\item {Grp. gram.:f.}
\end{itemize}
\begin{itemize}
\item {Proveniência:(Lat. \textunderscore serpens\textunderscore )}
\end{itemize}
Gênero de reptis, desprovidos de membros ou que têm apenas membros rudimentares.
Cobra.
Mulhér feia ou velha e feia.
Pessôa má ou traiçoeira.
Coisa nociva.
\section{Serpenteante}
\begin{itemize}
\item {Grp. gram.:adj.}
\end{itemize}
Que serpenteia.
\section{Serpentear}
\begin{itemize}
\item {Grp. gram.:v. i.}
\end{itemize}
O mesmo que \textunderscore serpear\textunderscore .
\section{Serpentes}
\begin{itemize}
\item {Grp. gram.:m. pl.}
\end{itemize}
Tríbo indígena da América do Norte.
\section{Serpentezona}
\begin{itemize}
\item {Grp. gram.:f.}
\end{itemize}
\begin{itemize}
\item {Utilização:Fam.}
\end{itemize}
\begin{itemize}
\item {Utilização:Fig.}
\end{itemize}
Serpente grande.
Mulhér ruím:«\textunderscore sois muito más, minhas serpentezonas.\textunderscore »Garrett, \textunderscore Fábulas\textunderscore , 66.
\section{Serpenticida}
\begin{itemize}
\item {Grp. gram.:m.  e  adj.}
\end{itemize}
\begin{itemize}
\item {Proveniência:(Do lat. \textunderscore serpens\textunderscore , \textunderscore serpentis\textunderscore  + \textunderscore caedere\textunderscore )}
\end{itemize}
Matador de serpentes.
\section{Serpentífero}
\begin{itemize}
\item {Grp. gram.:adj.}
\end{itemize}
\begin{itemize}
\item {Utilização:Poét.}
\end{itemize}
\begin{itemize}
\item {Proveniência:(Do lat. \textunderscore serpens\textunderscore , \textunderscore serpentis\textunderscore  + \textunderscore ferre\textunderscore )}
\end{itemize}
Em que há serpentes, que produz serpentes.
\section{Serpentiforme}
\begin{itemize}
\item {Grp. gram.:adj.}
\end{itemize}
\begin{itemize}
\item {Proveniência:(Do lat. \textunderscore serpens\textunderscore , \textunderscore serpentis\textunderscore  + \textunderscore forma\textunderscore )}
\end{itemize}
Que tem fórma de serpente.
\section{Serpentígeno}
\begin{itemize}
\item {Grp. gram.:adj.}
\end{itemize}
\begin{itemize}
\item {Utilização:Poét.}
\end{itemize}
\begin{itemize}
\item {Proveniência:(Lat. \textunderscore serpentigena\textunderscore )}
\end{itemize}
Gerado por serpente.
\section{Serpentígero}
\begin{itemize}
\item {Grp. gram.:adj.}
\end{itemize}
\begin{itemize}
\item {Proveniência:(Lat. \textunderscore serpentiger\textunderscore )}
\end{itemize}
Cujos pés são serpentes. Cf. Castilho, \textunderscore Metam.\textunderscore , 242.
\section{Serpentina}
\begin{itemize}
\item {Grp. gram.:f.}
\end{itemize}
\begin{itemize}
\item {Utilização:Bras}
\end{itemize}
\begin{itemize}
\item {Utilização:Miner.}
\end{itemize}
\begin{itemize}
\item {Proveniência:(De \textunderscore serpentino\textunderscore )}
\end{itemize}
Planta, o mesmo que \textunderscore serpentária\textunderscore , (\textunderscore dracunculus vulgaris\textunderscore , Schott.).
Vela de três lumes, que se costuma accender no Sábbado de Alleluia.
Castiçal de dois ou mais braços tortuosos, na extremidade dos quaes se fixam velas.
Trepadeira das regiões do Amazonas.
Antiga peça de artilharia.
Tubo espiral do alambique.
Palanquim com cortinados, que contém um leito de rêde.
Variedade de pedra fina que, depois de polida, apresenta manchas semelhantes ás da pelle da serpente.
Espécie de silicato hydratado de magnésio natural.
\section{Serpentino}
\begin{itemize}
\item {Grp. gram.:adj.}
\end{itemize}
\begin{itemize}
\item {Utilização:Miner.}
\end{itemize}
\begin{itemize}
\item {Proveniência:(Lat. \textunderscore serpentinus\textunderscore )}
\end{itemize}
Relativo a serpente.
Serpentiforme.
Que tem listas tortuosas, (falando-se de certos mármores).
Diz-se da rocha, em que abunda a serpentina, (silicato).
\section{Serpentinoso}
\begin{itemize}
\item {Grp. gram.:adj.}
\end{itemize}
\begin{itemize}
\item {Proveniência:(De \textunderscore serpentino\textunderscore )}
\end{itemize}
Relativo ao mármore serpentino. Cf. Castilho, \textunderscore Fastos\textunderscore , III, 411.
\section{Serpete}
\begin{itemize}
\item {Grp. gram.:f.}
\end{itemize}
\begin{itemize}
\item {Proveniência:(Fr. \textunderscore serpette\textunderscore )}
\end{itemize}
Instrumento de jardineiro, de lâmina curva e cabo que termina em saliência, para que se não desprenda da mão.
\section{Serpho}
\begin{itemize}
\item {Grp. gram.:m.}
\end{itemize}
Gênero de insectos hymenópteros.
\section{Serpilho}
\begin{itemize}
\item {Grp. gram.:m.}
\end{itemize}
\begin{itemize}
\item {Proveniência:(Do lat. \textunderscore serpyllum\textunderscore )}
\end{itemize}
O mesmo que \textunderscore serpão\textunderscore .
\section{Serpol}
\begin{itemize}
\item {Grp. gram.:m.}
\end{itemize}
\begin{itemize}
\item {Proveniência:(Do lat. \textunderscore serpyllum\textunderscore )}
\end{itemize}
O mesmo que \textunderscore serpão\textunderscore .
\section{Sérpula}
\begin{itemize}
\item {Grp. gram.:f.}
\end{itemize}
\begin{itemize}
\item {Proveniência:(Lat. \textunderscore serpula\textunderscore )}
\end{itemize}
Gênero de anelídeos marinhos.
\section{Serpulária}
\begin{itemize}
\item {Grp. gram.:f.}
\end{itemize}
Gênero de anelídeos fósseis, análogos ás sérpulas.
\section{Serra}
\begin{itemize}
\item {Grp. gram.:f.}
\end{itemize}
\begin{itemize}
\item {Utilização:Fig.}
\end{itemize}
\begin{itemize}
\item {Utilização:Prov.}
\end{itemize}
\begin{itemize}
\item {Utilização:trasm.}
\end{itemize}
\begin{itemize}
\item {Grp. gram.:M.}
\end{itemize}
\begin{itemize}
\item {Grp. gram.:Loc.}
\end{itemize}
\begin{itemize}
\item {Utilização:fam.}
\end{itemize}
\begin{itemize}
\item {Proveniência:(Lat. \textunderscore serra\textunderscore )}
\end{itemize}
Instrumento cortante, que tem por peça principal uma lâmina de aço dentado.
Montanha, cujo cume tem muitos accidentes ou anfractuosidades.
Montanha.
Elevação, que dá a apparência de uma serra.
Peixe escômbrido; espadarte.
Acto de serrar, serração: \textunderscore hoje é a serra da velha\textunderscore .
Modinha popular da Beiramar e do Doiro. Cf. \textunderscore Port. Ant. e Mod.\textunderscore , XII, 1541.
\textunderscore Ir á serra\textunderscore , dar o cavaco, melindrar-se, amuar-se.
\section{Serração}
\begin{itemize}
\item {Grp. gram.:f.}
\end{itemize}
Acto ou effeiro de serrar.
\textunderscore Serração da velha\textunderscore , costumeira, que ainda há nalgumas aldeias, de irem grupos populares á porta de mulheres idosas, em meio da Quaresma, e fingirem que as serram, folgando e motejando-as.
\section{Serradela}
\begin{itemize}
\item {Grp. gram.:f.}
\end{itemize}
O mesmo que \textunderscore serração\textunderscore .
\section{Serradela}
\begin{itemize}
\item {Grp. gram.:f.}
\end{itemize}
\begin{itemize}
\item {Proveniência:(Do lat. \textunderscore serratula\textunderscore )}
\end{itemize}
Planta leguminosa.
\section{Serradiço}
\begin{itemize}
\item {Grp. gram.:adj.}
\end{itemize}
\begin{itemize}
\item {Proveniência:(De \textunderscore serrar\textunderscore )}
\end{itemize}
Diz-se da madeira serrada e aparada.
O mesmo que \textunderscore serradinho\textunderscore .
\section{Serradinho}
\begin{itemize}
\item {Grp. gram.:adj.}
\end{itemize}
Diz-se de uma variedade de linho, também chamado \textunderscore serrano\textunderscore . Cf. \textunderscore Port. au point de vue agr.\textunderscore , 670.
\section{Serrador}
\begin{itemize}
\item {Grp. gram.:adj.}
\end{itemize}
\begin{itemize}
\item {Grp. gram.:M.}
\end{itemize}
\begin{itemize}
\item {Utilização:Prov.}
\end{itemize}
\begin{itemize}
\item {Utilização:beir.}
\end{itemize}
\begin{itemize}
\item {Utilização:Bras}
\end{itemize}
Que serra.
Homem, que serra madeira.
Espécie de grande foice dentada ou serrote recurvado, suspenso de uma aselha de ferro, cravada em parede, e no qual se corta em miúdos a palha para os animaes. (Colhido na Guarda)
Pássaro escuro, que se levanta e poisa, com pequenos intervallos, dando ideia dos movimentos de um homem que serra.
\section{Serradura}
\begin{itemize}
\item {Grp. gram.:f.}
\end{itemize}
\begin{itemize}
\item {Proveniência:(Do lat. \textunderscore serratura\textunderscore )}
\end{itemize}
O mesmo que \textunderscore serração\textunderscore .
Partículas de madeira, que caem quando se serra.
\section{Serrafaçar}
\textunderscore v. i.\textunderscore  (e der.)
O mesmo que \textunderscore sarrafaçar\textunderscore , etc.
\section{Serrafanada}
\begin{itemize}
\item {Grp. gram.:f.}
\end{itemize}
Ave, o mesmo que \textunderscore serafanada\textunderscore , (\textunderscore anas penelope\textunderscore , Lin.). Cf. P. Moraes, \textunderscore Zool. Elem.\textunderscore , 401.
\section{Serrafo}
\textunderscore m.\textunderscore  (e der.)
O mesmo que \textunderscore sarrafo\textunderscore , etc.
\section{Serragem}
\begin{itemize}
\item {Grp. gram.:f.}
\end{itemize}
\begin{itemize}
\item {Proveniência:(Do lat. \textunderscore serrago\textunderscore )}
\end{itemize}
Acto ou effeito de serrar; serradura.
\section{Serrajão}
\begin{itemize}
\item {Grp. gram.:m.}
\end{itemize}
Peixe, o mesmo que \textunderscore sarrajão\textunderscore .
\section{Serralha}
\begin{itemize}
\item {Grp. gram.:f.}
\end{itemize}
Planta, da fam. das compostas, (\textunderscore sonchus dearaceus\textunderscore , Lin.).
(Cp. \textunderscore sarralha\textunderscore )
\section{Serralha}
\begin{itemize}
\item {Grp. gram.:f.}
\end{itemize}
\begin{itemize}
\item {Utilização:Des.}
\end{itemize}
\begin{itemize}
\item {Utilização:Gír.}
\end{itemize}
Peça de 7$500 reis em oiro.
\section{Serralhar}
\begin{itemize}
\item {Grp. gram.:v. t.}
\end{itemize}
\begin{itemize}
\item {Grp. gram.:V. i.}
\end{itemize}
\begin{itemize}
\item {Proveniência:(De \textunderscore serralheiro\textunderscore )}
\end{itemize}
Lavrar ou limar como serralheiro.
Fazer barulho como os serralheiros.
\section{Serralharia}
\begin{itemize}
\item {Grp. gram.:f.}
\end{itemize}
Officina ou arte de serralheiro.
(Cp. \textunderscore serralheiro\textunderscore )
\section{Serralheiro}
\begin{itemize}
\item {Grp. gram.:m.}
\end{itemize}
\begin{itemize}
\item {Proveniência:(Do provn. \textunderscore serralh\textunderscore )}
\end{itemize}
Artista ou operário, que faz fechaduras e outras obras de ferro.
\section{Serralhinha}
\begin{itemize}
\item {Grp. gram.:f.}
\end{itemize}
\begin{itemize}
\item {Proveniência:(De \textunderscore serralha\textunderscore ^1)}
\end{itemize}
Planta, da fam. das compostas.
\section{Serralho}
\begin{itemize}
\item {Grp. gram.:m.}
\end{itemize}
\begin{itemize}
\item {Utilização:Fig.}
\end{itemize}
\begin{itemize}
\item {Proveniência:(Do it. \textunderscore serraglio\textunderscore )}
\end{itemize}
Palácio imperial, na Turquia.
Parte dêsse palácio, habitado pelas mulhéres do Sultão.
Harém.
Mulhéres, que habitam o harém.
Lupanar.
Prostíbulo.
Casa, onde vivem concubinas várias de um só homem.
\section{Serra-madeira}
\begin{itemize}
\item {Grp. gram.:f.}
\end{itemize}
Espécie de jôgo de crianças.
\section{Serramancar}
\begin{itemize}
\item {Grp. gram.:v. i.}
\end{itemize}
\begin{itemize}
\item {Utilização:Prov.}
\end{itemize}
\begin{itemize}
\item {Utilização:trasm.}
\end{itemize}
O mesmo que \textunderscore seramangar\textunderscore .
\section{Serra-metaes}
\begin{itemize}
\item {Grp. gram.:m.}
\end{itemize}
\begin{itemize}
\item {Utilização:T. de Lamego}
\end{itemize}
O mesmo que \textunderscore serralheiro\textunderscore .
\section{Serrana}
\begin{itemize}
\item {Grp. gram.:f.}
\end{itemize}
\begin{itemize}
\item {Utilização:Bras}
\end{itemize}
\begin{itemize}
\item {Proveniência:(De \textunderscore serrano\textunderscore )}
\end{itemize}
Mulhér, que vive nas serras.
Mulhér rústica.
Espécie de canção popular na Beira.
Bailado campestre, espécie de fandango.
\section{Serrania}
\begin{itemize}
\item {Grp. gram.:f.}
\end{itemize}
\begin{itemize}
\item {Utilização:Fig.}
\end{itemize}
\begin{itemize}
\item {Proveniência:(De \textunderscore serrano\textunderscore )}
\end{itemize}
Agglomeração de serras; cordilheira.
Ondas encapelladas.
\section{Serranice}
\begin{itemize}
\item {Grp. gram.:f.}
\end{itemize}
Modos de serrano.
\section{Serranilha}
\begin{itemize}
\item {Grp. gram.:f.}
\end{itemize}
\begin{itemize}
\item {Proveniência:(De \textunderscore serrano\textunderscore )}
\end{itemize}
Canção pastoril, que era uma das fórmas lýricas dos antigos trovadores portugueses.
\section{Serrano}
\begin{itemize}
\item {Grp. gram.:adj.}
\end{itemize}
\begin{itemize}
\item {Grp. gram.:M.}
\end{itemize}
\begin{itemize}
\item {Utilização:Ext.}
\end{itemize}
\begin{itemize}
\item {Proveniência:(Lat. \textunderscore serranus\textunderscore )}
\end{itemize}
Relativo a serras.
Que habita nas serras.
Montesino.
Diz-se de uma variedade de linho.
Habitante das serras.
Montanhês.
Camponês.
Espécie de barco, usado no Mondego.
\section{Serrão}
\begin{itemize}
\item {Grp. gram.:m.}
\end{itemize}
\begin{itemize}
\item {Utilização:Prov.}
\end{itemize}
\begin{itemize}
\item {Utilização:minh.}
\end{itemize}
\begin{itemize}
\item {Utilização:Bras}
\end{itemize}
O mesmo que \textunderscore serrano\textunderscore .
Serra grande, com dois pegadoiros, para serrar toros de madeira.
Ave, o mesmo que \textunderscore serrador\textunderscore .
\section{Serrar}
\begin{itemize}
\item {Grp. gram.:v. t.}
\end{itemize}
\begin{itemize}
\item {Grp. gram.:V. i.}
\end{itemize}
Cortar com serra ou serrote.
Trabalhar com serra.
(B. lat. \textunderscore serrare\textunderscore )
\section{Serraria}
\begin{itemize}
\item {Grp. gram.:f.}
\end{itemize}
\begin{itemize}
\item {Proveniência:(De \textunderscore serrar\textunderscore )}
\end{itemize}
Armação de madeira, em que se apoia a trave ou o pau, que se quere serrar com o auxílio de uma serra grande, movida por dois serradores.
Fábrica de serrar madeira, a vapor.
\section{Serraria}
\begin{itemize}
\item {Grp. gram.:f.}
\end{itemize}
\begin{itemize}
\item {Utilização:P. us.}
\end{itemize}
\begin{itemize}
\item {Proveniência:(De \textunderscore serra\textunderscore )}
\end{itemize}
Conjunto de serras ou de montes.
Montes successivos; serrania.
\section{Serra-serra}
\begin{itemize}
\item {Grp. gram.:m.}
\end{itemize}
Ave canora do Brasil, que canta, dando uma volta no ar.
\section{Serrátil}
\begin{itemize}
\item {Grp. gram.:adj.}
\end{itemize}
\begin{itemize}
\item {Utilização:Med.}
\end{itemize}
\begin{itemize}
\item {Utilização:Mathem.}
\end{itemize}
\begin{itemize}
\item {Proveniência:(Do b. lat. \textunderscore serratus\textunderscore )}
\end{itemize}
Que tem fórma de serra.
Diz-se do pulso, que, ao mesmo tempo, apresenta pulsações em vários pontos.
Que tem cinco superfícies, três das quaes são parallelogrammos.
\section{Serratura}
\begin{itemize}
\item {Grp. gram.:f.}
\end{itemize}
\begin{itemize}
\item {Utilização:Bot.}
\end{itemize}
O mesmo que \textunderscore dente\textunderscore , segundo a terminologia de Desvaux.
(Cp. \textunderscore serrear\textunderscore )
\section{Serrazina}
\begin{itemize}
\item {Grp. gram.:f.}
\end{itemize}
\begin{itemize}
\item {Utilização:T. da Bairrada}
\end{itemize}
\begin{itemize}
\item {Grp. gram.:M. ,  f.  e  adj.}
\end{itemize}
Acto de serrazinar.
O mesmo que \textunderscore pintarroxo\textunderscore .
Pessôa, que serrazina.
\section{Serrazinar}
\begin{itemize}
\item {Grp. gram.:v. i.}
\end{itemize}
Sêr maçador ou importuno, insistindo no mesmo assumpto ou no mesmo pedido.
(Relaciona-se com \textunderscore serra\textunderscore )
\section{Serrear}
\begin{itemize}
\item {Grp. gram.:v. t.}
\end{itemize}
Dar fórma de serra a; recortar ou dentear em fórma de serra.
\section{Serreado}
\begin{itemize}
\item {Grp. gram.:adj.}
\end{itemize}
\begin{itemize}
\item {Proveniência:(De \textunderscore serrear\textunderscore )}
\end{itemize}
Que tem fórma de serra ou dentes, semelhantes aos da serra.
\section{Serredo}
\begin{itemize}
\item {fónica:rê}
\end{itemize}
\begin{itemize}
\item {Grp. gram.:m.}
\end{itemize}
\begin{itemize}
\item {Utilização:T. de Turquel}
\end{itemize}
\begin{itemize}
\item {Proveniência:(De \textunderscore serra\textunderscore )}
\end{itemize}
Fraguedo; penedia.
\section{Serrenho}
\begin{itemize}
\item {Grp. gram.:m.  e  adj.}
\end{itemize}
\begin{itemize}
\item {Utilização:Prov.}
\end{itemize}
\begin{itemize}
\item {Utilização:alg.}
\end{itemize}
O mesmo que \textunderscore serrano\textunderscore .
\section{Sérreo}
\begin{itemize}
\item {Grp. gram.:adj.}
\end{itemize}
Relativo a serra; que tem fórma de serra.
\section{Serricórneos}
\begin{itemize}
\item {Grp. gram.:m. pl.}
\end{itemize}
\begin{itemize}
\item {Proveniência:(De \textunderscore serra\textunderscore  + \textunderscore corno\textunderscore )}
\end{itemize}
Família de insectos coleópteros pentâmeros, que tem as antennas em fórma de serra.
\section{Serridênteo}
\begin{itemize}
\item {Grp. gram.:adj.}
\end{itemize}
\begin{itemize}
\item {Utilização:Zool.}
\end{itemize}
\begin{itemize}
\item {Proveniência:(De \textunderscore serra\textunderscore  + \textunderscore dente\textunderscore )}
\end{itemize}
Que tem dentes como a serra; serreado.
\section{Serril}
\begin{itemize}
\item {Grp. gram.:adj.}
\end{itemize}
\begin{itemize}
\item {Proveniência:(De \textunderscore serra\textunderscore )}
\end{itemize}
Sérreo; serrano.
\section{Serrilha}
\begin{itemize}
\item {Grp. gram.:f.}
\end{itemize}
\begin{itemize}
\item {Utilização:Açor}
\end{itemize}
\begin{itemize}
\item {Proveniência:(De \textunderscore serra\textunderscore )}
\end{itemize}
Lavor em fórma de dentes de serra, para ornato.
Lavor denteado na circunferência de certas moédas.
Peça dos arreios, guarnecida de pontas de ferro, para refrear a cavalgadura, cuja barbada guarnece.
Bordo denteado de qualquer objecto.
Moéda espanhola, correspondente a 190 reis.
Moéda de 200 reis.
O mesmo que \textunderscore serrilhador\textunderscore .
\section{Serrilhador}
\begin{itemize}
\item {Grp. gram.:m.}
\end{itemize}
Máquina de serrilhar moéda.
\section{Serrilhar}
\begin{itemize}
\item {Grp. gram.:v. t.}
\end{itemize}
\begin{itemize}
\item {Grp. gram.:V. i.}
\end{itemize}
Fazer serrilha em.
Puxar em sentido opposto as duas rédeas do cavallo, quando êste toma freio nos dentes.
\section{Serrilho}
\begin{itemize}
\item {Grp. gram.:m.}
\end{itemize}
Grande eixo, a que está presa a roda grande dos engenhos de açúcar.
(Cp. \textunderscore serrilha\textunderscore )
\section{Serrim}
\begin{itemize}
\item {Grp. gram.:m.}
\end{itemize}
Espécie de forragem; o mesmo que \textunderscore serradela\textunderscore ^2.
(Cast. \textunderscore serrin\textunderscore )
\section{Serrim}
\begin{itemize}
\item {Grp. gram.:m.}
\end{itemize}
\begin{itemize}
\item {Utilização:T. do Pôrto}
\end{itemize}
O mesmo que \textunderscore serradura\textunderscore .
\section{Serrino}
\begin{itemize}
\item {Grp. gram.:adj.}
\end{itemize}
O mesmo que \textunderscore serrátil\textunderscore .
\section{Serrípede}
\begin{itemize}
\item {Grp. gram.:adj.}
\end{itemize}
\begin{itemize}
\item {Utilização:Zool.}
\end{itemize}
\begin{itemize}
\item {Proveniência:(Do lat. \textunderscore serra\textunderscore  + \textunderscore pes\textunderscore )}
\end{itemize}
Que tem pés serreados.
\section{Serrirostro}
\begin{itemize}
\item {fónica:rós}
\end{itemize}
\begin{itemize}
\item {Grp. gram.:adj.}
\end{itemize}
\begin{itemize}
\item {Utilização:Zool.}
\end{itemize}
\begin{itemize}
\item {Proveniência:(Do lat. \textunderscore serra\textunderscore  + \textunderscore rostrum\textunderscore )}
\end{itemize}
Que tem o bico em fórma de serra.
\section{Serrirrostro}
\begin{itemize}
\item {Grp. gram.:adj.}
\end{itemize}
\begin{itemize}
\item {Utilização:Zool.}
\end{itemize}
\begin{itemize}
\item {Proveniência:(Do lat. \textunderscore serra\textunderscore  + \textunderscore rostrum\textunderscore )}
\end{itemize}
Que tem o bico em fórma de serra.
\section{Serrócero}
\begin{itemize}
\item {Grp. gram.:m.}
\end{itemize}
\begin{itemize}
\item {Proveniência:(Do lat. \textunderscore serra\textunderscore  + gr. \textunderscore keras\textunderscore )}
\end{itemize}
Gênero de insectos, da fam. dos serricórneos.
\section{Serropalpo}
\begin{itemize}
\item {Grp. gram.:m.}
\end{itemize}
\begin{itemize}
\item {Proveniência:(Do lat. \textunderscore serra\textunderscore  + \textunderscore palpus\textunderscore )}
\end{itemize}
Gênero de insectos coleópteros heterómeros.
\section{Serropião}
\begin{itemize}
\item {Grp. gram.:m.}
\end{itemize}
\begin{itemize}
\item {Utilização:Prov.}
\end{itemize}
\begin{itemize}
\item {Utilização:dur.}
\end{itemize}
\begin{itemize}
\item {Proveniência:(De \textunderscore escorpião\textunderscore ?)}
\end{itemize}
Verme peçonhento, semelhante á lombriga.
\section{Serrotar}
\begin{itemize}
\item {Grp. gram.:v. t.}
\end{itemize}
Cortar com serrote.
\section{Serrote}
\begin{itemize}
\item {Grp. gram.:m.}
\end{itemize}
\begin{itemize}
\item {Utilização:Bras. do N}
\end{itemize}
\begin{itemize}
\item {Proveniência:(De \textunderscore serra\textunderscore )}
\end{itemize}
Lâmina denteada como a da serra, sem outra armação que um cabo por onde se lhe pega.
Espécie de peixe da ria de Aveiro.
Montanha pequena.
\section{Sertã}
\begin{itemize}
\item {Grp. gram.:f.}
\end{itemize}
(V.sartan)
\section{Sertan}
\begin{itemize}
\item {Grp. gram.:f.}
\end{itemize}
(V.sartan)
\section{Sertanejo}
\begin{itemize}
\item {Grp. gram.:adj.}
\end{itemize}
\begin{itemize}
\item {Grp. gram.:M.}
\end{itemize}
Relativo ao sertão.
Que vive no sertão.
Silvestre.
Rude.
Indivíduo sertanejo.
\section{Sertanista}
\begin{itemize}
\item {Grp. gram.:m.}
\end{itemize}
\begin{itemize}
\item {Utilização:Bras}
\end{itemize}
Aquelle que conhece ou frequenta o sertão.
\section{Sertão}
\begin{itemize}
\item {Grp. gram.:m.}
\end{itemize}
Lugar inculto, distante de povoações ou de terrenos cultivados.
Floresta, no interior de um continente, ou longe da costa.
\section{Sertela}
\begin{itemize}
\item {Grp. gram.:f.}
\end{itemize}
\begin{itemize}
\item {Utilização:Prov.}
\end{itemize}
Pesca das enguias.
O mesmo que \textunderscore sertelha\textunderscore .
\section{Sertelha}
\begin{itemize}
\item {fónica:tê}
\end{itemize}
\begin{itemize}
\item {Grp. gram.:f.}
\end{itemize}
\begin{itemize}
\item {Utilização:Pesc.}
\end{itemize}
Apparelho, usado na pesca das enguias.
\section{Sertém}
\begin{itemize}
\item {Grp. gram.:f.}
\end{itemize}
\begin{itemize}
\item {Utilização:Prov.}
\end{itemize}
(V.sartan)
\section{Sertulária}
\begin{itemize}
\item {Grp. gram.:f.}
\end{itemize}
Gênero de pólypos.
\section{Sertulários}
\begin{itemize}
\item {Grp. gram.:m. pl.}
\end{itemize}
Ordem de pólypos, a que pertence a sertulária.
\section{Sertum}
\begin{itemize}
\item {Grp. gram.:m.}
\end{itemize}
\begin{itemize}
\item {Utilização:Prov.}
\end{itemize}
\begin{itemize}
\item {Utilização:beir.}
\end{itemize}
Collete de mulhér.
Collete de homem.
(Talvez do fr. \textunderscore surtout\textunderscore . Cp. \textunderscore surtum\textunderscore )
\section{Serubuna}
\begin{itemize}
\item {Grp. gram.:f.}
\end{itemize}
O mesmo que \textunderscore serutinga\textunderscore .
\section{Seruda}
\begin{itemize}
\item {Grp. gram.:f.}
\end{itemize}
\begin{itemize}
\item {Utilização:Prov.}
\end{itemize}
\begin{itemize}
\item {Utilização:trasm.}
\end{itemize}
\begin{itemize}
\item {Utilização:minh.}
\end{itemize}
Espécie de planta, (\textunderscore chelidonium majus\textunderscore , Lin.), com cujo latex os camponeses combatem as verrugas.
\section{Sérum}
\begin{itemize}
\item {Grp. gram.:m.}
\end{itemize}
(V.sôro)
\section{Serumtherapia}
\textunderscore f.\textunderscore  (e der.)
(V. \textunderscore serotherapia\textunderscore , etc.)
\section{Serunterapia}
\textunderscore f.\textunderscore  (e der.)
(V. \textunderscore seroterapia\textunderscore , etc.)
\section{Serutinga}
\begin{itemize}
\item {Grp. gram.:f.}
\end{itemize}
Variedade de mangue, (\textunderscore avicennia nitida\textunderscore ).
\section{Serva}
\begin{itemize}
\item {Grp. gram.:f.}
\end{itemize}
(Flexão fem. de \textunderscore servo\textunderscore )
\section{Servador}
\begin{itemize}
\item {Grp. gram.:adj.}
\end{itemize}
\begin{itemize}
\item {Utilização:Poét.}
\end{itemize}
\begin{itemize}
\item {Proveniência:(Do lat. \textunderscore servator\textunderscore )}
\end{itemize}
Conservador.
Que salva; libertador.
\section{Servência}
\begin{itemize}
\item {Grp. gram.:f.}
\end{itemize}
\begin{itemize}
\item {Utilização:Ant.}
\end{itemize}
\begin{itemize}
\item {Proveniência:(De \textunderscore servente\textunderscore )}
\end{itemize}
Qualidade do que serve; serventia.
\section{Servente}
\begin{itemize}
\item {Grp. gram.:adj.}
\end{itemize}
\begin{itemize}
\item {Grp. gram.:M.  e  f.}
\end{itemize}
\begin{itemize}
\item {Proveniência:(Lat. \textunderscore servens\textunderscore )}
\end{itemize}
Que serve.
Pessôa, que serve; criado ou criada.
\section{Serventia}
\begin{itemize}
\item {Grp. gram.:f.}
\end{itemize}
\begin{itemize}
\item {Utilização:Des.}
\end{itemize}
\begin{itemize}
\item {Proveniência:(De \textunderscore servente\textunderscore )}
\end{itemize}
Qualidade do que serve ou do que tem utilidade ou préstimo.
Servidão.
Passagem.
Passadiço.
Serviço provisório ou feito em nome de outrem.
Trabalho de servente.
Escravidão.
\section{Serventuário}
\begin{itemize}
\item {Grp. gram.:m.}
\end{itemize}
\begin{itemize}
\item {Proveniência:(De \textunderscore servente\textunderscore )}
\end{itemize}
Aquelle que desempenha provisoriamente um encargo, na falta ou substituição do empregado proprietário.
\section{Serviçal}
\begin{itemize}
\item {Grp. gram.:adj.}
\end{itemize}
\begin{itemize}
\item {Grp. gram.:M.}
\end{itemize}
\begin{itemize}
\item {Grp. gram.:F.}
\end{itemize}
\begin{itemize}
\item {Proveniência:(De \textunderscore serviço\textunderscore )}
\end{itemize}
Obsequiador.
Prestadio.
Relativo a servos ou criados.
Criado.
Trabalhador assalariado.
O mesmo que \textunderscore criada\textunderscore  ou mulhér assalariada.
\section{Serviçalmente}
\begin{itemize}
\item {Grp. gram.:adv.}
\end{itemize}
De modo serviçal; obsequiosamente.
\section{Serviçaria}
\begin{itemize}
\item {Grp. gram.:f.}
\end{itemize}
\begin{itemize}
\item {Utilização:Ant.}
\end{itemize}
\begin{itemize}
\item {Proveniência:(De \textunderscore serviço\textunderscore )}
\end{itemize}
Lavoira.
\section{Servicial}
\begin{itemize}
\item {Grp. gram.:m. ,  f.  e  adj.}
\end{itemize}
O mesmo que \textunderscore serviçal\textunderscore .
\section{Serviço}
\begin{itemize}
\item {Grp. gram.:m.}
\end{itemize}
\begin{itemize}
\item {Utilização:Ant.}
\end{itemize}
\begin{itemize}
\item {Utilização:Ant.}
\end{itemize}
\begin{itemize}
\item {Utilização:Ant.}
\end{itemize}
\begin{itemize}
\item {Utilização:Ant.}
\end{itemize}
\begin{itemize}
\item {Utilização:Ant.}
\end{itemize}
\begin{itemize}
\item {Utilização:Ant.}
\end{itemize}
\begin{itemize}
\item {Proveniência:(Do lat. \textunderscore servitium\textunderscore )}
\end{itemize}
Acto ou effeito de servir.
Exercício de funcções obrigatórias.
Duração dêsse exercício.
Desempenho de qualquer trabalho.
Estado de quem serve outrem por salário: \textunderscore ficas a meu serviço\textunderscore .
Proveito: \textunderscore trabalhar em serviço das letras\textunderscore .
Utilidade ou préstimo de alguma coisa.
Obséquio: \textunderscore prestei-lhe serviços\textunderscore .
Serventia.
Baixella; conjunto das peças de loiça, que servem para um jantar ou para outra refeição: \textunderscore um serviço de prata\textunderscore .
Loiça e talheres próprios para uma refeição.
O último parceiro no jôgo da pela.
Vaso para excrementos.
Acto de celebrar officio divino.
Celebração de actos religiosos.
Tributo de vassalagem.
Determinada pensão, em dinheiro ou frutos.
Refeição, que os colonos ou emphytheutas davam ao senhorio directo.
Contribuição, que os mosteiros e lugares pios pagavam aos respectivos fundadores ou aos herdeiros dêstes.
Foro, que constava de um alqueire de trigo, um alqueire de cevada e uma gallinha.
O mesmo que \textunderscore presente\textunderscore , dádiva:«\textunderscore ordenaram levar-lhe á dita Villa das Caldas um serviço, que eram quatrocentas gallinhas e patos e cento e cincoenta alqueires de cevada.\textunderscore »(Doc. do séc. XV, na Bibl. de Evora)
\section{Servidão}
\begin{itemize}
\item {Grp. gram.:f.}
\end{itemize}
\begin{itemize}
\item {Utilização:Jur.}
\end{itemize}
\begin{itemize}
\item {Proveniência:(Do lat. \textunderscore servitudo\textunderscore )}
\end{itemize}
Condição do que é servo ou escravo.
Escravidão.
Dependência.
Encargo, imposto num prédio, em benefício de outro que pertence a dono differente.
Passagem, para uso do publico, por um terreno, que é propriedade particular.
\section{Servidiço}
\begin{itemize}
\item {Grp. gram.:adj.}
\end{itemize}
\begin{itemize}
\item {Proveniência:(De \textunderscore servido\textunderscore )}
\end{itemize}
Que serviu muitas vezes; usado, gasto. Cf. \textunderscore Techn. Rur.\textunderscore , 216.
\section{Servido}
\begin{itemize}
\item {Grp. gram.:adj.}
\end{itemize}
\begin{itemize}
\item {Utilização:Gír.}
\end{itemize}
\begin{itemize}
\item {Proveniência:(De \textunderscore servir\textunderscore )}
\end{itemize}
Usado; gasto.
Provído, fornecido.
Preso.
\textunderscore Sêr servido\textunderscore , haver por bem, dignar-se.
\section{Servidor}
\begin{itemize}
\item {Grp. gram.:m.  e  adj.}
\end{itemize}
\begin{itemize}
\item {Grp. gram.:M.}
\end{itemize}
\begin{itemize}
\item {Utilização:Prov.}
\end{itemize}
\begin{itemize}
\item {Proveniência:(Do lat. \textunderscore servitor\textunderscore )}
\end{itemize}
Servente.
Obsequiador.
Pontual no serviço.
O mesmo que \textunderscore penico\textunderscore .
\section{Serviente}
\begin{itemize}
\item {Grp. gram.:adj.}
\end{itemize}
\begin{itemize}
\item {Proveniência:(Lat. \textunderscore serviens\textunderscore )}
\end{itemize}
Sujeito á servidão, (falando-se de um prédio).
\section{Servil}
\begin{itemize}
\item {Grp. gram.:adj.}
\end{itemize}
\begin{itemize}
\item {Proveniência:(Lat. \textunderscore servilis\textunderscore )}
\end{itemize}
Relativo a servo ou próprio delle.
Vil.
Tôrpe.
Bajulador.
Que segue rigorosamente um modêlo ou um original: \textunderscore traducção servil\textunderscore .
\section{Servilão}
\begin{itemize}
\item {Grp. gram.:m.}
\end{itemize}
Homem muito servil. Cf. B. Pato, \textunderscore Cantos e Sát.\textunderscore , 216.
\section{Servilha}
\begin{itemize}
\item {Grp. gram.:f.}
\end{itemize}
\begin{itemize}
\item {Utilização:Ant.}
\end{itemize}
\begin{itemize}
\item {Utilização:Bras}
\end{itemize}
\begin{itemize}
\item {Proveniência:(Do cast. \textunderscore servilla\textunderscore )}
\end{itemize}
Barco para pesca de sardinha.
Sapato de coiro.
Sapato de ourelo.
\section{Servilheiro}
\begin{itemize}
\item {Grp. gram.:m.}
\end{itemize}
Tripulante de servilha; sardinheiro.
\section{Servilheta}
\begin{itemize}
\item {fónica:lhê}
\end{itemize}
\begin{itemize}
\item {Grp. gram.:f.}
\end{itemize}
\begin{itemize}
\item {Proveniência:(Do cast. \textunderscore servilleta\textunderscore )}
\end{itemize}
Criada, serva.
\section{Servilheteiro}
\begin{itemize}
\item {Grp. gram.:m.}
\end{itemize}
Aquelle que namora ou galanteia servilhetas.
\section{Servilismo}
\begin{itemize}
\item {Grp. gram.:m.}
\end{itemize}
Qualidade do que é servil.
Qualidade de bajulador.
\section{Servilmente}
\begin{itemize}
\item {Grp. gram.:adv.}
\end{itemize}
De modo servil; com bajulação; com adulação.
\section{Sérvio}
\begin{itemize}
\item {Grp. gram.:adj.}
\end{itemize}
\begin{itemize}
\item {Grp. gram.:M.}
\end{itemize}
Relativo á Sérvia.
Habitante da Sérvia.
\section{Serviola}
\begin{itemize}
\item {Grp. gram.:f.}
\end{itemize}
\begin{itemize}
\item {Utilização:Náut.}
\end{itemize}
Turco de lambareiro, quando êste é de madeira.
Pedaço grosso de madeira, hoje pouco usado, que protegia o costado contra a âncora e a amarra.
(Cast. \textunderscore serviola\textunderscore )
\section{Servir}
\begin{itemize}
\item {Grp. gram.:v. t.}
\end{itemize}
\begin{itemize}
\item {Grp. gram.:Loc.}
\end{itemize}
\begin{itemize}
\item {Utilização:pop.}
\end{itemize}
\begin{itemize}
\item {Grp. gram.:V. i.}
\end{itemize}
\begin{itemize}
\item {Grp. gram.:V. p.}
\end{itemize}
\begin{itemize}
\item {Proveniência:(Lat. \textunderscore servire\textunderscore )}
\end{itemize}
Prestar utilidade a.
Sêr criado de.
Sêr útil a.
Estar ás ordens de.
Cumprir.
Pôr na mesa qualquer refeição ou tempêro.
Auxiliar.
Cuidar de.
\textunderscore Servir o rei\textunderscore , estar no exército, assentar praça.
Viver ou trabalhar como servo.
Viver na dependência de alguém.
Exercer quaesquer funcções.
Obedecer.
Prestar serviços.
Sêr prestadio, útil, vantajoso.
Sêr opportuno, vir em bôa occasião.
Fazer as vezes de alguma coisa.
Sêr causa.
Adaptar-se.
Dar serventia.
Aproveitar-se.
Dignar-se, haver por bem: \textunderscore sirva-se V. Ex.^a de dar as suas ordens\textunderscore .[**. adicionado]
Utilizar-se de uma iguaria, á mesa: \textunderscore sirva-se, minha senhora\textunderscore .
\section{Servitas}
\begin{itemize}
\item {Grp. gram.:m.  e  f. pl.}
\end{itemize}
\begin{itemize}
\item {Proveniência:(De \textunderscore servo\textunderscore )}
\end{itemize}
Congregação religiosa, que se estabeleceu no convento do Rêgo em Lisbôa.
\section{Servitude}
\begin{itemize}
\item {Grp. gram.:f.}
\end{itemize}
\begin{itemize}
\item {Utilização:Neol.}
\end{itemize}
O mesmo que \textunderscore servidão\textunderscore .
\section{Servível}
\begin{itemize}
\item {Grp. gram.:adj.}
\end{itemize}
\begin{itemize}
\item {Utilização:Neol.}
\end{itemize}
Que serve, que presta utilidade ou serviço.
\section{Servo}
\begin{itemize}
\item {Grp. gram.:m.}
\end{itemize}
\begin{itemize}
\item {Grp. gram.:Adj.}
\end{itemize}
\begin{itemize}
\item {Utilização:T. de Turquel}
\end{itemize}
\begin{itemize}
\item {Proveniência:(Lat. \textunderscore servus\textunderscore )}
\end{itemize}
Aquelle que não exerce direitos.
Aquelle que não dispõe da sua pessôa e bens.
Homem adstricto a uma terra ou sujeito a um senhor.
Servente; criado.
Que não é livre.
Que presta serviços a alguém.
Que tem a condição de criado.
Que é escravo.
Jornaleiro agrícola.
\section{Servo-croata}
\begin{itemize}
\item {Grp. gram.:m.}
\end{itemize}
Língua do grupo eslávico.
\section{Serzeta}
\begin{itemize}
\item {fónica:zê}
\end{itemize}
\begin{itemize}
\item {Grp. gram.:f.}
\end{itemize}
Espécie de narceja, (\textunderscore gallinago gallinula\textunderscore , Lin.).
\section{Serzete}
\begin{itemize}
\item {fónica:zê}
\end{itemize}
\begin{itemize}
\item {Grp. gram.:m.}
\end{itemize}
\begin{itemize}
\item {Utilização:Prov.}
\end{itemize}
O mesmo que \textunderscore merganso\textunderscore .
(Cp. \textunderscore serzeta\textunderscore )
\section{Serzideira}
\begin{itemize}
\item {Grp. gram.:f.}
\end{itemize}
\begin{itemize}
\item {Utilização:Náut.}
\end{itemize}
Mulher que sirze.
Cabo das testas da gávea.
\section{Serzidor}
\begin{itemize}
\item {Grp. gram.:m.  e  adj.}
\end{itemize}
\begin{itemize}
\item {Utilização:Deprec.}
\end{itemize}
O que sirze.
Publicista, cujos trabalhos são, em grande parte, compilação de trechos de outros escritores.
\section{Serzidura}
\begin{itemize}
\item {Grp. gram.:f.}
\end{itemize}
Acto ou effeito de serzir.
\section{Serzilho}
\begin{itemize}
\item {Grp. gram.:m.}
\end{itemize}
\begin{itemize}
\item {Utilização:Ant.}
\end{itemize}
O mesmo que \textunderscore arrecada\textunderscore ?«\textunderscore ...hũs serzilhos de orelhas...\textunderscore »(\textunderscore Doc.\textunderscore  do séc. XVI)
\section{Serzino}
\begin{itemize}
\item {Grp. gram.:m.}
\end{itemize}
Ave, o mesmo que \textunderscore milheira\textunderscore .
(Cp. \textunderscore serezino\textunderscore )
\section{Serzir}
\begin{itemize}
\item {Grp. gram.:v. t.}
\end{itemize}
\begin{itemize}
\item {Utilização:Ext.}
\end{itemize}
\begin{itemize}
\item {Proveniência:(Do lat. \textunderscore sarcire\textunderscore )}
\end{itemize}
Coser, dando pontos tão miúdos, que seja imperceptível a costura.
Unir; intercalar.
\section{Sesâmeas}
\begin{itemize}
\item {Grp. gram.:f. pl.}
\end{itemize}
\begin{itemize}
\item {Proveniência:(De \textunderscore sêsamo\textunderscore )}
\end{itemize}
Família de plantas, estabelecida por Brown, á custa das bignoniáceas de Jusieu.
O mesmo que \textunderscore pedalíneas\textunderscore , designação anterior a \textunderscore sesâmeas\textunderscore .
\section{Sêsamo}
\begin{itemize}
\item {Grp. gram.:m.}
\end{itemize}
\begin{itemize}
\item {Proveniência:(Lat. \textunderscore sesamum\textunderscore )}
\end{itemize}
O mesmo que \textunderscore gergelim\textunderscore .
\section{Sesamóide-menór}
\begin{itemize}
\item {Grp. gram.:f.}
\end{itemize}
Planta chicoriácea, (\textunderscore catamanche coerulea\textunderscore , Lin.).
\section{Sesamoídeo}
\begin{itemize}
\item {Grp. gram.:adj.}
\end{itemize}
\begin{itemize}
\item {Grp. gram.:M.}
\end{itemize}
\begin{itemize}
\item {Proveniência:(Do gr. \textunderscore sesamon\textunderscore  + \textunderscore eidos\textunderscore )}
\end{itemize}
Semelhante á semente do sêsamo.
Nome de pequenos ossos, que há em certas articulações.
\section{Sesana}
\begin{itemize}
\item {Grp. gram.:f.}
\end{itemize}
Gênero de crustáceos decápodes.
\section{Sesanna}
\begin{itemize}
\item {Grp. gram.:f.}
\end{itemize}
Gênero de crustáceos decápodes.
\section{Sesbânia}
\begin{itemize}
\item {Grp. gram.:f.}
\end{itemize}
\begin{itemize}
\item {Proveniência:(Do ár. \textunderscore sesban\textunderscore )}
\end{itemize}
Gênero de plantas leguminosas.
\section{Sésea}
\begin{itemize}
\item {Grp. gram.:f.}
\end{itemize}
Gênero de plantas solâneas.
\section{Séseli}
\begin{itemize}
\item {Grp. gram.:m.}
\end{itemize}
\begin{itemize}
\item {Proveniência:(Lat. \textunderscore seseli\textunderscore )}
\end{itemize}
Planta umbellífera.
\section{Sesélio}
\begin{itemize}
\item {Grp. gram.:m.}
\end{itemize}
\begin{itemize}
\item {Proveniência:(Lat. \textunderscore seselium\textunderscore )}
\end{itemize}
O mesmo que \textunderscore séseli\textunderscore .
\section{Seserigo}
\begin{itemize}
\item {Grp. gram.:m.}
\end{itemize}
\begin{itemize}
\item {Utilização:Ant.}
\end{itemize}
Planície.
Chapada.
O mesmo que \textunderscore sésiga\textunderscore .
\section{Seserino}
\begin{itemize}
\item {Grp. gram.:m.}
\end{itemize}
Gênero de peixes acanthopterýgeos.
\section{Sesgo}
\begin{itemize}
\item {fónica:sês}
\end{itemize}
\begin{itemize}
\item {Grp. gram.:adj.}
\end{itemize}
\begin{itemize}
\item {Grp. gram.:M.}
\end{itemize}
\begin{itemize}
\item {Utilização:Taur.}
\end{itemize}
Oblíquo, dirigido de lado.
Torcido.
Sorte de bandarilhas, que se collocam de ambos os lados do toiro, estando êste oblíquo com a trincheira, do modo que o toireiro só lhe póde sair pela frente, quarteando-se para prender as bandarilhas.
(Cast. \textunderscore sesgo\textunderscore )
\section{Sésia}
\begin{itemize}
\item {Grp. gram.:f.}
\end{itemize}
\begin{itemize}
\item {Proveniência:(Do gr. \textunderscore ses\textunderscore )}
\end{itemize}
Gênero de insectos lepidópteros.
\section{Sésica}
\begin{itemize}
\item {Grp. gram.:f.}
\end{itemize}
\begin{itemize}
\item {Utilização:Ant.}
\end{itemize}
O mesmo que \textunderscore sésiga\textunderscore .
\section{Sésicas}
\begin{itemize}
\item {Grp. gram.:f. pl.}
\end{itemize}
\begin{itemize}
\item {Utilização:Ant.}
\end{itemize}
Direito de renovar, em terreno alheio, a plantação de uma árvore ou a construcção de um moínho, direito pertencente a quem nesse terreno tivera árvore ou moínho que deixára de existir.
(Cp. \textunderscore sésiga\textunderscore )
\section{Sésiga}
\begin{itemize}
\item {Grp. gram.:f.}
\end{itemize}
\begin{itemize}
\item {Utilização:Ant.}
\end{itemize}
\begin{itemize}
\item {Proveniência:(Do b. lat. \textunderscore sesica\textunderscore )}
\end{itemize}
Lugar, onde alguma coisa está, onde alguma coisa tem assento.
\section{Sesléria}
\begin{itemize}
\item {Grp. gram.:f.}
\end{itemize}
\begin{itemize}
\item {Proveniência:(De \textunderscore Sesler\textunderscore , n. p.)}
\end{itemize}
Gênero de plantas gramíneas, rasteiras, que crescem nas montanhas.
\section{Sesma}
\begin{itemize}
\item {fónica:sês}
\end{itemize}
\begin{itemize}
\item {Grp. gram.:f.}
\end{itemize}
\begin{itemize}
\item {Utilização:Ant.}
\end{itemize}
\begin{itemize}
\item {Proveniência:(Do lat. \textunderscore sex\textunderscore )}
\end{itemize}
A sexta parte.
Medida antiga, a terça parte do côvado.
\section{Sesmar}
\begin{itemize}
\item {Grp. gram.:v. t.}
\end{itemize}
\begin{itemize}
\item {Utilização:Ant.}
\end{itemize}
\begin{itemize}
\item {Proveniência:(De \textunderscore sesma\textunderscore )}
\end{itemize}
Dividir em sesmarias.
\section{Sesmaria}
\begin{itemize}
\item {Grp. gram.:f.}
\end{itemize}
\begin{itemize}
\item {Proveniência:(De \textunderscore sesmar\textunderscore )}
\end{itemize}
Terreno inculto ou abandonado; maninho.
\section{Sesmeiro}
\begin{itemize}
\item {Grp. gram.:m.}
\end{itemize}
\begin{itemize}
\item {Proveniência:(De \textunderscore sesma\textunderscore )}
\end{itemize}
Aquelle que dividia as sesmarias.
Aquelle, a quem se deu uma sesmaria para a cultivar. Cf. Herculano, \textunderscore Hist. de Port.\textunderscore , IV, 241 e 242.
\section{Sesmo}
\begin{itemize}
\item {Grp. gram.:m.}
\end{itemize}
\begin{itemize}
\item {Utilização:Prov.}
\end{itemize}
\begin{itemize}
\item {Utilização:alent.}
\end{itemize}
\begin{itemize}
\item {Utilização:Ant.}
\end{itemize}
\begin{itemize}
\item {Utilização:Ant.}
\end{itemize}
\begin{itemize}
\item {Utilização:Ant.}
\end{itemize}
\begin{itemize}
\item {Proveniência:(De \textunderscore sesma\textunderscore )}
\end{itemize}
Terreno sesmado.
Lugar, onde há sesmarias.
Espaço arroteado, entre os matos ou sesmarias, para servir de caminho e evitar a propagação de incêndios.
Limite, extremo.
Quinhão, partilha.
A sexta parte, sesma:«\textunderscore ...aqui neste ponto tomei o sol em 15 graus e um sesmo\textunderscore ». Pero Lopes, \textunderscore Diário da Naveg.\textunderscore , 7.
\section{Sésqui...}
\begin{itemize}
\item {Grp. gram.:pref.}
\end{itemize}
\begin{itemize}
\item {Proveniência:(Lat. \textunderscore sesqui\textunderscore )}
\end{itemize}
(designativo de \textunderscore um e meio\textunderscore )
\section{Sèsquiáltera}
\begin{itemize}
\item {Grp. gram.:f.}
\end{itemize}
\begin{itemize}
\item {Proveniência:(De \textunderscore sèsquiáltero\textunderscore )}
\end{itemize}
Grupo de seis figuras musicaes, que se executam ao mesmo tempo, em que se deveriam executar quatro da mesma espécie.
\section{Sèsquiáltero}
\begin{itemize}
\item {Grp. gram.:adj.}
\end{itemize}
\begin{itemize}
\item {Proveniência:(Lat. \textunderscore sesquialter\textunderscore )}
\end{itemize}
Diz-se de duas quantidades matemáticas, uma das quaes contém a outra uma vez e meia.
\section{Sèsquioitava}
\begin{itemize}
\item {Grp. gram.:f.}
\end{itemize}
\begin{itemize}
\item {Utilização:Mús.}
\end{itemize}
\begin{itemize}
\item {Utilização:ant.}
\end{itemize}
Proporção de 9/8.
\section{Sesqui-óxydo}
\begin{itemize}
\item {Grp. gram.:m.}
\end{itemize}
\begin{itemize}
\item {Proveniência:(De \textunderscore sésqui...\textunderscore  + \textunderscore óxydo\textunderscore )}
\end{itemize}
Oxydo, que contém uma vez e meia a quantidade de oxygenio, contida no protóxydo.
\section{Sèsquipedal}
\begin{itemize}
\item {Grp. gram.:adj.}
\end{itemize}
\begin{itemize}
\item {Utilização:Burl.}
\end{itemize}
\begin{itemize}
\item {Proveniência:(Lat. \textunderscore sesquipedalis\textunderscore )}
\end{itemize}
Que tem pé e meio de comprimento.
Que é muito grande, (falando-se de certos versos ou palavras).
\section{Sèsquiquarta}
\begin{itemize}
\item {Grp. gram.:f.}
\end{itemize}
\begin{itemize}
\item {Utilização:Mús.}
\end{itemize}
\begin{itemize}
\item {Utilização:Ant.}
\end{itemize}
Proporção de 5/4.
\section{Sèsquiquinta}
\begin{itemize}
\item {Grp. gram.:f.}
\end{itemize}
\begin{itemize}
\item {Utilização:Mús.}
\end{itemize}
\begin{itemize}
\item {Utilização:Ant.}
\end{itemize}
Proporção de 5/6.
\section{Sèsquisal}
\begin{itemize}
\item {fónica:sal}
\end{itemize}
\begin{itemize}
\item {Grp. gram.:m.}
\end{itemize}
\begin{itemize}
\item {Utilização:Chím.}
\end{itemize}
Sal, cuja base ou cujo ácido equivale a uma vez e meia da base ou do acido do sal neutro correspondente.
\section{Sèsquisétima}
\begin{itemize}
\item {fónica:sé}
\end{itemize}
\begin{itemize}
\item {Grp. gram.:f.}
\end{itemize}
\begin{itemize}
\item {Utilização:Mús.}
\end{itemize}
\begin{itemize}
\item {Utilização:Ant.}
\end{itemize}
Proporção de 8/7.
\section{Sèsquisexta}
\begin{itemize}
\item {fónica:seis}
\end{itemize}
\begin{itemize}
\item {Grp. gram.:f.}
\end{itemize}
\begin{itemize}
\item {Utilização:Mús.}
\end{itemize}
\begin{itemize}
\item {Utilização:Ant.}
\end{itemize}
Proporção de 7/6.
\section{Sèsquissal}
\begin{itemize}
\item {Grp. gram.:m.}
\end{itemize}
\begin{itemize}
\item {Utilização:Chím.}
\end{itemize}
Sal, cuja base ou cujo ácido equivale a uma vez e meia da base ou do acido do sal neutro correspondente.
\section{Sèsquissétima}
\begin{itemize}
\item {Grp. gram.:f.}
\end{itemize}
\begin{itemize}
\item {Utilização:Mús.}
\end{itemize}
\begin{itemize}
\item {Utilização:Ant.}
\end{itemize}
Proporção de 8/7.
\section{Sèsquissexta}
\begin{itemize}
\item {Grp. gram.:f.}
\end{itemize}
\begin{itemize}
\item {Utilização:Mús.}
\end{itemize}
\begin{itemize}
\item {Utilização:Ant.}
\end{itemize}
Proporção de 7/6.
\section{Sèsquissulfuretos}
\begin{itemize}
\item {Grp. gram.:m. pl.}
\end{itemize}
Ordem de sulfuretos, a que pertence o antimonito.
\section{Sèsquisulfuretos}
\begin{itemize}
\item {fónica:sul}
\end{itemize}
\begin{itemize}
\item {Grp. gram.:m. pl.}
\end{itemize}
Ordem de sulfuretos, a que pertence o antimonito.
\section{Sèsquitércia}
\begin{itemize}
\item {Grp. gram.:f.}
\end{itemize}
\begin{itemize}
\item {Utilização:Mús.}
\end{itemize}
\begin{itemize}
\item {Utilização:Ant.}
\end{itemize}
Proporção de 4/3.
\section{Sessão}
\begin{itemize}
\item {Grp. gram.:f.}
\end{itemize}
\begin{itemize}
\item {Proveniência:(Lat. \textunderscore sessio\textunderscore )}
\end{itemize}
O mesmo que \textunderscore assentada\textunderscore .
Tempo, durante o qual está reunida uma corporação deliberativa.
Tempo, durante o qual funcciona um congresso ou uma junta.
Tempo, que decorre, desde a abertura até ao encerramento das Côrtes em cada anno.
\section{Sessão}
\begin{itemize}
\item {Grp. gram.:f.}
\end{itemize}
\begin{itemize}
\item {Utilização:Prov.}
\end{itemize}
\begin{itemize}
\item {Utilização:minh.}
\end{itemize}
Humidade ou frescura da terra: \textunderscore o meu quintal tem muita sessão\textunderscore .
\section{Sessar}
\begin{itemize}
\item {Grp. gram.:v. t.}
\end{itemize}
\begin{itemize}
\item {Utilização:Bras}
\end{itemize}
Joeirar com urupema.
\section{Séssega}
\begin{itemize}
\item {Grp. gram.:f.}
\end{itemize}
\begin{itemize}
\item {Utilização:Ant.}
\end{itemize}
Lugar, onde se faz uma construcção.
Sésiga.
(Cp. \textunderscore sêsso\textunderscore )
\section{Sessegar}
\textunderscore v. t.\textunderscore  e \textunderscore i.\textunderscore  (e der.) \textunderscore Ant.\textunderscore 
O mesmo que \textunderscore sossegar\textunderscore , etc. B. Pereira, \textunderscore Prosodia\textunderscore , vb. \textunderscore sedamen\textunderscore .
\section{Sessene}
\begin{itemize}
\item {Grp. gram.:m.}
\end{itemize}
Antiga moéda castelhana.
\section{Sessenta}
\begin{itemize}
\item {Grp. gram.:adj.}
\end{itemize}
\begin{itemize}
\item {Proveniência:(Do lat. \textunderscore sexaginta\textunderscore )}
\end{itemize}
Seis vezes déz.
\section{Sessenta-e-um}
\begin{itemize}
\item {Grp. gram.:m.}
\end{itemize}
\begin{itemize}
\item {Utilização:Bras. de Minas}
\end{itemize}
Espécie de jôgo de cartas.
\section{Séssil}
\begin{itemize}
\item {Grp. gram.:adj.}
\end{itemize}
\begin{itemize}
\item {Utilização:Bot.}
\end{itemize}
\begin{itemize}
\item {Proveniência:(Lat. \textunderscore sessilis\textunderscore )}
\end{itemize}
Que não tem supporte ou pedúnculo.
\section{Sessiliflôr}
\begin{itemize}
\item {Grp. gram.:adj.}
\end{itemize}
\begin{itemize}
\item {Proveniência:(De \textunderscore séssil\textunderscore  + \textunderscore flôr\textunderscore )}
\end{itemize}
Que tem flôres sésseis.
\section{Sessilifloro}
\begin{itemize}
\item {Grp. gram.:adj.}
\end{itemize}
\begin{itemize}
\item {Utilização:Bot.}
\end{itemize}
\begin{itemize}
\item {Proveniência:(De \textunderscore séssil\textunderscore  + \textunderscore flôr\textunderscore )}
\end{itemize}
Que tem flôres sésseis.
\section{Sessilifoliado}
\begin{itemize}
\item {Grp. gram.:adj.}
\end{itemize}
\begin{itemize}
\item {Utilização:Bot.}
\end{itemize}
\begin{itemize}
\item {Proveniência:(De \textunderscore séssil\textunderscore  + \textunderscore foliado\textunderscore )}
\end{itemize}
Que tem fôlhas sésseis.
\section{Sêsso}
\begin{itemize}
\item {Grp. gram.:m.}
\end{itemize}
\begin{itemize}
\item {Utilização:Pleb.}
\end{itemize}
\begin{itemize}
\item {Proveniência:(Lat. \textunderscore sessus\textunderscore )}
\end{itemize}
Assento, nádegas. Cf. \textunderscore Peregrinação\textunderscore , CLXXVII.
\section{Sessoeira}
\begin{itemize}
\item {Grp. gram.:f.}
\end{itemize}
\begin{itemize}
\item {Utilização:T. da Maia}
\end{itemize}
\begin{itemize}
\item {Proveniência:(De \textunderscore sessão\textunderscore )}
\end{itemize}
Sala das sessões de uma assembleia.
\section{Sessor}
\begin{itemize}
\item {Grp. gram.:m.}
\end{itemize}
\begin{itemize}
\item {Utilização:Des.}
\end{itemize}
Aquelle que está sentado. Cf. Macedo, \textunderscore Burros\textunderscore , 269.
\section{Sesta}
\begin{itemize}
\item {Grp. gram.:f.}
\end{itemize}
Hora de descanso, depois de jantar.
O pino da calma, a hora de mais calor.
(Cp. cast. \textunderscore siesta\textunderscore )
\section{Sestear}
\begin{itemize}
\item {Grp. gram.:v. t.}
\end{itemize}
\begin{itemize}
\item {Grp. gram.:V. i.}
\end{itemize}
Abrigar do calor (o gado).
Dormir a sesta.
\section{Sesteiro}
\begin{itemize}
\item {Grp. gram.:m.}
\end{itemize}
\begin{itemize}
\item {Utilização:Prov.}
\end{itemize}
\begin{itemize}
\item {Utilização:Ant.}
\end{itemize}
\begin{itemize}
\item {Utilização:Ant.}
\end{itemize}
Medida de cereaes, equivalente a três ou quatro alqueires.
A sexta parte de um moio.
Medida, equivalente a um alqueire.
(Por \textunderscore sexteiro\textunderscore , de \textunderscore sexto\textunderscore )
\section{Sesterciário}
\begin{itemize}
\item {Grp. gram.:m.}
\end{itemize}
\begin{itemize}
\item {Utilização:Poét.}
\end{itemize}
\begin{itemize}
\item {Proveniência:(Lat. \textunderscore sestertiarius\textunderscore )}
\end{itemize}
Homem muito pobre, que só terá um sestércio.
\section{Sestércio}
\begin{itemize}
\item {Grp. gram.:m.}
\end{itemize}
\begin{itemize}
\item {Proveniência:(Lat. \textunderscore sestertius\textunderscore )}
\end{itemize}
Moéda romana, de cobre.
\section{Sestercíolo}
\begin{itemize}
\item {Grp. gram.:m.}
\end{itemize}
\begin{itemize}
\item {Proveniência:(Lat. \textunderscore sestertiolus\textunderscore )}
\end{itemize}
Pequeno sestércio.
\section{Sestro}
\begin{itemize}
\item {Grp. gram.:adj.}
\end{itemize}
\begin{itemize}
\item {Utilização:Fig.}
\end{itemize}
\begin{itemize}
\item {Grp. gram.:M.}
\end{itemize}
\begin{itemize}
\item {Proveniência:(Do lat. \textunderscore sinister\textunderscore )}
\end{itemize}
O mesmo que \textunderscore esquerdo\textunderscore .
O mesmo que \textunderscore sinistro\textunderscore .
Destino; sorte.
Predicado.
Manha.
\section{Sestro}
\begin{itemize}
\item {Grp. gram.:m.}
\end{itemize}
O mesmo que \textunderscore sistro\textunderscore .
\section{Sestroso}
\begin{itemize}
\item {Grp. gram.:adj.}
\end{itemize}
Que tem sestro^1.
\section{Sesúvio}
\begin{itemize}
\item {Grp. gram.:m.}
\end{itemize}
Gênero de plantas portuláceas.
\section{Seta}
\begin{itemize}
\item {Grp. gram.:f.}
\end{itemize}
\begin{itemize}
\item {Utilização:Fig.}
\end{itemize}
\begin{itemize}
\item {Utilização:Veter.}
\end{itemize}
Haste de madeira, armada de um ferro, a qual se atira por meio de um arco ou bésta; frecha.
Ponteiro, que indica as horas nos relógios.
Signal, em fórma de seta, mostrando a direcção em que se podem mover os ponteiros do relógio, quando é preciso alterar-lhes a posição.
Planta alismácea.
Constellação, junto da via láctea.
Violência do um acto ou de um sentimento.
Dito satírico.
Rodopelo, junto da base da cauda dos cavallos.
(Cast. \textunderscore saeta\textunderscore , do lat. \textunderscore saggita\textunderscore )
\section{Seta}
\begin{itemize}
\item {Grp. gram.:f.}
\end{itemize}
\begin{itemize}
\item {Utilização:Prov.}
\end{itemize}
\begin{itemize}
\item {Utilização:trasm.}
\end{itemize}
Variedade de cogumelo comestível.
\section{Setáceo}
\begin{itemize}
\item {Grp. gram.:adj.}
\end{itemize}
\begin{itemize}
\item {Proveniência:(Do lat. \textunderscore seta\textunderscore )}
\end{itemize}
Que é da natureza das sedas ou pêlos do porco; cerdoso.
\section{Setada}
\begin{itemize}
\item {Grp. gram.:f.}
\end{itemize}
Golpe ou ferimento, feito com seta.
\section{Setária}
\begin{itemize}
\item {Grp. gram.:f.}
\end{itemize}
\begin{itemize}
\item {Proveniência:(Lat. bot. \textunderscore setaria\textunderscore )}
\end{itemize}
Gênero de plantas gramíneas, que cresce principalmente nos montes e serve para pastagem.
\section{Sete}
\begin{itemize}
\item {Grp. gram.:adj.}
\end{itemize}
\begin{itemize}
\item {Grp. gram.:M.}
\end{itemize}
\begin{itemize}
\item {Proveniência:(Do lat. \textunderscore septem\textunderscore )}
\end{itemize}
Diz se do número cardinal, formado de seis e mais um.
Sétimo.
O algarismo, que representa o número sete.
Carta de jogar, que tem sete pontos.
O que numa série de sete occupa o último lugar.
\section{Setear}
\begin{itemize}
\item {Grp. gram.:v. t.}
\end{itemize}
Ferir com seta; assetear.
\section{Sete-casacas}
\begin{itemize}
\item {Grp. gram.:f.}
\end{itemize}
Planta myrtácea do Brasil.
\section{Sete-casas}
\begin{itemize}
\item {Grp. gram.:f. pl.}
\end{itemize}
\begin{itemize}
\item {Utilização:Des.}
\end{itemize}
Edifício lisbonense, onde se recebiam os impostos sôbre gêneros de consumo.
\section{Sete-cascos}
\begin{itemize}
\item {Grp. gram.:m.}
\end{itemize}
Planta monimiácea do Brasil.
\section{Sètecentos}
\begin{itemize}
\item {Grp. gram.:adj.}
\end{itemize}
\begin{itemize}
\item {Proveniência:(De \textunderscore sete\textunderscore  + \textunderscore cento\textunderscore )}
\end{itemize}
Sete vezes cem.
\section{Sete-coiros}
\begin{itemize}
\item {Grp. gram.:m.}
\end{itemize}
\begin{itemize}
\item {Utilização:Bras}
\end{itemize}
Árvore silvestre.
O mesmo que \textunderscore sete-cascos\textunderscore ?
\section{Sete-cotovelos}
\begin{itemize}
\item {Grp. gram.:m.}
\end{itemize}
Espécie de pêra, caracterizada especialmente por protuberâncias na peripheria.
\section{Sete-e-meio}
\begin{itemize}
\item {Grp. gram.:m.}
\end{itemize}
Jôgo de carta, análogo ao trinta-e-um, e em que, distribuída uma carta a cada um dos parceiros, êstes pedem as que julgam precisas para se aproximarem de sete pontos e meio, sem excederem êste número.
\section{Sete-em-rama}
\begin{itemize}
\item {Grp. gram.:m.}
\end{itemize}
Planta rosácea.
\section{Sete-espigas}
\begin{itemize}
\item {Grp. gram.:f.}
\end{itemize}
Casta de uva branca algarvia.
\section{Sete-estrêllo}
\begin{itemize}
\item {Grp. gram.:m.}
\end{itemize}
\begin{itemize}
\item {Utilização:Pop.}
\end{itemize}
Constellação, o mesmo que \textunderscore pléiades\textunderscore .
\section{Seteira}
\begin{itemize}
\item {Grp. gram.:f.}
\end{itemize}
\begin{itemize}
\item {Proveniência:(De \textunderscore setta\textunderscore )}
\end{itemize}
Pequena abertura nas muralhas, pela qual se atiravam setas contra os inimigos ou sitiantes.
Qualquer fresta, nas paredes de um edifício, para dar luz ao interior.
\section{Seteiro}
\begin{itemize}
\item {Grp. gram.:m.  e  adj.}
\end{itemize}
O que atira setas.
\section{Setembrismo}
\begin{itemize}
\item {Grp. gram.:m.}
\end{itemize}
Partido setembrista. Cf. Herculano, \textunderscore Voz do Propheta\textunderscore , (introd.).
\section{Setembrista}
\begin{itemize}
\item {Grp. gram.:adj.}
\end{itemize}
\begin{itemize}
\item {Grp. gram.:M.}
\end{itemize}
Relativo á revolução de Setembro de 1836.
Sectário da política, que determinou essa revolução.
\section{Setembro}
\begin{itemize}
\item {Grp. gram.:m.}
\end{itemize}
\begin{itemize}
\item {Proveniência:(Do lat. \textunderscore september\textunderscore )}
\end{itemize}
Nono mês do anno romano.
\section{Sètemesinho}
\begin{itemize}
\item {Grp. gram.:adj.}
\end{itemize}
\begin{itemize}
\item {Utilização:Fam.}
\end{itemize}
\begin{itemize}
\item {Proveniência:(De \textunderscore sete\textunderscore  + \textunderscore mês\textunderscore )}
\end{itemize}
Diz-se da criança que nasceu, tendo só sete meses de gestação.
\section{Sètena}
\begin{itemize}
\item {Grp. gram.:f.}
\end{itemize}
\begin{itemize}
\item {Grp. gram.:Adj. f.}
\end{itemize}
\begin{itemize}
\item {Proveniência:(Lat. \textunderscore septena\textunderscore )}
\end{itemize}
Estrophe de sete versos.
Diz-se de uma febre, cujos acessos se repetem de sete em sete dias.
\section{Sètenado}
\begin{itemize}
\item {Grp. gram.:adj.}
\end{itemize}
\begin{itemize}
\item {Utilização:Bot.}
\end{itemize}
\begin{itemize}
\item {Proveniência:(Do lat. \textunderscore septeni\textunderscore )}
\end{itemize}
Que tem sete folíolos num pecíolo commum, (falando-se das fôlhas do sete-em-rama).
\section{Sètenado}
\begin{itemize}
\item {Grp. gram.:m.}
\end{itemize}
\begin{itemize}
\item {Proveniência:(Do lat. \textunderscore septennis\textunderscore )}
\end{itemize}
Govêrno, que se formou em França em 1873, para durar sete anos.
\section{Sètenal}
\begin{itemize}
\item {Grp. gram.:adj.}
\end{itemize}
\begin{itemize}
\item {Proveniência:(Do lat. \textunderscore septennis\textunderscore )}
\end{itemize}
Que se realiza de sete em sete anos.
\section{Sètenalidade}
\begin{itemize}
\item {Grp. gram.:f.}
\end{itemize}
Qualidade do que é septenal. Cf. Garrett, \textunderscore Port. na Balança\textunderscore , 89 e 99.
\section{Sètenário}
\begin{itemize}
\item {Grp. gram.:adj.}
\end{itemize}
\begin{itemize}
\item {Grp. gram.:M.}
\end{itemize}
\begin{itemize}
\item {Proveniência:(Lat. \textunderscore septenarius\textunderscore )}
\end{itemize}
Que vale ou contém sete.
Espaço de sete dias ou sete annos.
Festa religiosa, que dura sete dias.
\section{Sètenário}
\begin{itemize}
\item {Grp. gram.:m.  e  adj.}
\end{itemize}
O mesmo ou melhor que \textunderscore septenário\textunderscore .
\section{Sètenato}
\begin{itemize}
\item {Grp. gram.:m.}
\end{itemize}
O mesmo que \textunderscore septennado\textunderscore .
\section{Setênfluo}
\begin{itemize}
\item {Grp. gram.:adj.}
\end{itemize}
\begin{itemize}
\item {Utilização:Poét.}
\end{itemize}
\begin{itemize}
\item {Proveniência:(Do lat. \textunderscore septem\textunderscore  + \textunderscore fluere\textunderscore )}
\end{itemize}
Que deriva de sete fontes.
\section{Sètenial}
\begin{itemize}
\item {Grp. gram.:adj.}
\end{itemize}
\begin{itemize}
\item {Proveniência:(De \textunderscore septênnio\textunderscore )}
\end{itemize}
Que dura sete anos.
\section{Sètênio}
\begin{itemize}
\item {Grp. gram.:m.}
\end{itemize}
\begin{itemize}
\item {Proveniência:(Lat. \textunderscore septennium\textunderscore )}
\end{itemize}
Espaço de sete anos.
\section{Seteno}
\begin{itemize}
\item {Grp. gram.:m.}
\end{itemize}
\begin{itemize}
\item {Proveniência:(Do lat. \textunderscore septeni\textunderscore )}
\end{itemize}
O mesmo que \textunderscore septênnio\textunderscore :«\textunderscore entrei nos oitenta e sete annos; foi para mi tão crítico este seteno...\textunderscore »A. Vieira, \textunderscore Cartas\textunderscore .
Tributo ou foragem, em que se pagava 1 por 7, quanto a frutos. Cf. L. Cardoso, \textunderscore Diccion. Geogr. Med.\textunderscore 
O sétimo dia, em que certas doenças fazem crise.
\section{Setênplice}
\begin{itemize}
\item {Grp. gram.:adj.}
\end{itemize}
\begin{itemize}
\item {Utilização:Poét.}
\end{itemize}
\begin{itemize}
\item {Proveniência:(Lat. \textunderscore septemplex\textunderscore )}
\end{itemize}
Dobrado em sete.
Que tem sete lâminas.
\section{Setenta}
\begin{itemize}
\item {Grp. gram.:adj.}
\end{itemize}
\begin{itemize}
\item {Proveniência:(Do lat. \textunderscore septuaginta\textunderscore )}
\end{itemize}
Sete vezes déz.
\section{Setentrião}
\begin{itemize}
\item {Grp. gram.:m.}
\end{itemize}
\begin{itemize}
\item {Utilização:Poét.}
\end{itemize}
\begin{itemize}
\item {Proveniência:(Do lat. \textunderscore septemtrio\textunderscore )}
\end{itemize}
O pólo norte.
As regiões do Norte.
Vento do Norte.
\section{Setentrional}
\begin{itemize}
\item {Grp. gram.:adj.}
\end{itemize}
\begin{itemize}
\item {Grp. gram.:M.  e  f.}
\end{itemize}
\begin{itemize}
\item {Proveniência:(Do lat. \textunderscore septemtrionalis\textunderscore )}
\end{itemize}
Relativo a setentrião.
Que habita do lado do Norte.
Situado ao norte.
Pessôa setentrional.
\section{Setenvirado}
\begin{itemize}
\item {Grp. gram.:m.}
\end{itemize}
\begin{itemize}
\item {Proveniência:(Do lat. \textunderscore septemviratus\textunderscore )}
\end{itemize}
Cargo ou dignidade de septênviro.
Assembleia ou tribunal dos septênviros.
\section{Setenviral}
\begin{itemize}
\item {Grp. gram.:adj.}
\end{itemize}
\begin{itemize}
\item {Proveniência:(Lat. \textunderscore septemviralis\textunderscore )}
\end{itemize}
Relativo aos septênviros.
\section{Setenvirato}
\begin{itemize}
\item {Grp. gram.:m.}
\end{itemize}
(V.septemvirado)
\section{Setênviro}
\begin{itemize}
\item {Grp. gram.:m.}
\end{itemize}
\begin{itemize}
\item {Proveniência:(Lat. \textunderscore septemvir\textunderscore )}
\end{itemize}
Cada um dos sete sacerdotes e magistrados romanos, que fiscalizavam os banquetes em honra dos deuses e os que se celebravam depois dos jogos públicos.
\section{Sete-sangrias}
\begin{itemize}
\item {Grp. gram.:f.}
\end{itemize}
Planta salicínea.
Nome de duas plantas brasileiras.
\section{Setia}
\begin{itemize}
\item {Grp. gram.:f.}
\end{itemize}
Pequena embarcação asiática.
Cale do moínho.
Cano de madeira, geralmente aberto na parte superior, e que leva a água que faz mover os engenhos hydráulicos.
Valla, que conduz do rio ou do mar a água salgada para as loiças da salina, e é fechada pela comporta.
\section{Setia}
\begin{itemize}
\item {Grp. gram.:f.}
\end{itemize}
Espécie de prego, de arame. Cf. \textunderscore Inquér. Industr.\textunderscore , p. II, l. I, 282.
\section{Setial}
\begin{itemize}
\item {Grp. gram.:m.}
\end{itemize}
Banco ou assento ornamentado, nas igrejas.
Escabello; assento.
Cômoro ou elevação de terra, em que alguém se póde sentar como em banco.
(Por \textunderscore sedial\textunderscore , de \textunderscore séde\textunderscore )
\section{Seticauda}
\begin{itemize}
\item {Grp. gram.:m.}
\end{itemize}
\begin{itemize}
\item {Proveniência:(Do lat. \textunderscore seta\textunderscore  + \textunderscore cauda\textunderscore )}
\end{itemize}
Designação antiga de um gênero de insectos, que tem seis pés e abdome terminado em cerdas.
\section{Seticole}
\begin{itemize}
\item {Grp. gram.:adj.}
\end{itemize}
\begin{itemize}
\item {Utilização:Poét.}
\end{itemize}
\begin{itemize}
\item {Proveniência:(Lat. \textunderscore septicollis\textunderscore )}
\end{itemize}
Que tem sete oiteiros.
\section{Sèticolor}
\begin{itemize}
\item {Grp. gram.:m.}
\end{itemize}
\begin{itemize}
\item {Proveniência:(Do lat. \textunderscore septem\textunderscore  + \textunderscore color\textunderscore )}
\end{itemize}
Espécie de tangará, de plumagem variegada.
\section{Sèticorde}
\begin{itemize}
\item {Grp. gram.:adj.}
\end{itemize}
\begin{itemize}
\item {Utilização:Poét.}
\end{itemize}
\begin{itemize}
\item {Proveniência:(Lat. \textunderscore septemchordis\textunderscore )}
\end{itemize}
Que tem sete cordas.
\section{Seticórneo}
\begin{itemize}
\item {Grp. gram.:adj.}
\end{itemize}
\begin{itemize}
\item {Utilização:Zool.}
\end{itemize}
\begin{itemize}
\item {Grp. gram.:Pl.}
\end{itemize}
\begin{itemize}
\item {Proveniência:(Do gr. \textunderscore seta\textunderscore  + \textunderscore cornu\textunderscore )}
\end{itemize}
Que tem antennas em fórma de sêdas.
Familia de insectos seticórneos.
\section{Setífero}
\begin{itemize}
\item {Grp. gram.:adj.}
\end{itemize}
\begin{itemize}
\item {Proveniência:(Do lat. \textunderscore seta\textunderscore  + \textunderscore ferre\textunderscore )}
\end{itemize}
Relativo á sêda; que produz sêda.
\section{Setiforme}
\begin{itemize}
\item {Grp. gram.:adj.}
\end{itemize}
\begin{itemize}
\item {Proveniência:(Do lat. \textunderscore seta\textunderscore  + \textunderscore forma\textunderscore )}
\end{itemize}
Que tem o aspecto de cerdas.
\section{Sètiforme}
\begin{itemize}
\item {Grp. gram.:adj.}
\end{itemize}
\begin{itemize}
\item {Proveniência:(Lat. \textunderscore septiformis\textunderscore )}
\end{itemize}
Que tem sete fórmas.
\section{Setígero}
\begin{itemize}
\item {Grp. gram.:adj.}
\end{itemize}
\begin{itemize}
\item {Utilização:Bot.}
\end{itemize}
\begin{itemize}
\item {Proveniência:(Do lat. \textunderscore setiger\textunderscore )}
\end{itemize}
O mesmo que \textunderscore setífero\textunderscore .
Que tem uma ou muitas sedas.
\section{Setilha}
\begin{itemize}
\item {Grp. gram.:f.}
\end{itemize}
\begin{itemize}
\item {Proveniência:(De \textunderscore sete\textunderscore )}
\end{itemize}
Estrophe de sete versos.
\section{Sètilião}
\begin{itemize}
\item {Grp. gram.:m.}
\end{itemize}
\begin{itemize}
\item {Proveniência:(Do lat. \textunderscore septem\textunderscore )}
\end{itemize}
Mil sextiliões.
\section{Setim}
\begin{itemize}
\item {Grp. gram.:m.}
\end{itemize}
\begin{itemize}
\item {Utilização:Fig.}
\end{itemize}
\begin{itemize}
\item {Proveniência:(Do ár. \textunderscore zeituni\textunderscore )}
\end{itemize}
Pano lustroso e fino de sêda ou lan.
Coisa macia ou suave.
O mesmo que \textunderscore pau-setim\textunderscore .
\section{Sétima}
\begin{itemize}
\item {Grp. gram.:f.}
\end{itemize}
\begin{itemize}
\item {Proveniência:(De \textunderscore sétimo\textunderscore )}
\end{itemize}
Intervallo musical entre dois tons, que distam reciprocamente sete graus.
Sete cartas do mesmo naipe, no jôgo dos centos.
\section{Sètimátrias}
\begin{itemize}
\item {Grp. gram.:f. pl.}
\end{itemize}
\begin{itemize}
\item {Proveniência:(Do lat. \textunderscore septimatrus\textunderscore )}
\end{itemize}
Antigas festas romanas, que se celebravam em honra de Minerva, no sétimo dia depois dos idos.
\section{Sètimestre}
\begin{itemize}
\item {Grp. gram.:adj.}
\end{itemize}
Que tem sete meses. Cf. \textunderscore Reino da Estupidez\textunderscore , 190.
\section{Sétimo}
\begin{itemize}
\item {Grp. gram.:m.  e  adj.}
\end{itemize}
\begin{itemize}
\item {Grp. gram.:M.}
\end{itemize}
\begin{itemize}
\item {Proveniência:(Lat. \textunderscore septimus\textunderscore )}
\end{itemize}
O que numa série de sete occupa o último lugar.
Sétima parte.
\section{Sètimôncio}
\begin{itemize}
\item {Grp. gram.:m.}
\end{itemize}
\begin{itemize}
\item {Proveniência:(Lat. \textunderscore septimontium\textunderscore )}
\end{itemize}
Festas, que os Romanos celebravam, em commemoração da encorporação das sete colinas no recinto de Roma.
\section{Setineta}
\begin{itemize}
\item {fónica:nê}
\end{itemize}
\begin{itemize}
\item {Grp. gram.:f.}
\end{itemize}
\begin{itemize}
\item {Proveniência:(Do fr. \textunderscore satinette\textunderscore )}
\end{itemize}
Tecido fino de sêda e algodão, imitando setim.
\section{Sètingentésimo}
\begin{itemize}
\item {Grp. gram.:adj.}
\end{itemize}
\begin{itemize}
\item {Proveniência:(Lat. \textunderscore septigentesimus\textunderscore )}
\end{itemize}
Que numa série de 700 ocupa o último lugar.
\section{Setinoso}
\begin{itemize}
\item {Grp. gram.:adj.}
\end{itemize}
O mesmo que \textunderscore assetinado\textunderscore .
Macio ao tacto; avelludado.
\section{Sètissílabo}
\begin{itemize}
\item {Grp. gram.:adj.}
\end{itemize}
\begin{itemize}
\item {Grp. gram.:M.}
\end{itemize}
Que tem sete sílabas.
Verso de sete sílabas.
\section{Sètíssono}
\begin{itemize}
\item {Grp. gram.:adj.}
\end{itemize}
\begin{itemize}
\item {Proveniência:(Do lat. \textunderscore septem\textunderscore  + \textunderscore sonus\textunderscore )}
\end{itemize}
Que tem sete sons:«\textunderscore ...lyra setíssona...\textunderscore »A. Dinís, \textunderscore Odes Pind.\textunderscore 
\section{Sètívoco}
\begin{itemize}
\item {Grp. gram.:adj.}
\end{itemize}
\begin{itemize}
\item {Utilização:Poét.}
\end{itemize}
\begin{itemize}
\item {Proveniência:(Do lat. \textunderscore septem\textunderscore  + \textunderscore vox\textunderscore )}
\end{itemize}
Que tem sete vozes.
\section{Sètizónio}
\begin{itemize}
\item {Grp. gram.:m.}
\end{itemize}
\begin{itemize}
\item {Proveniência:(Lat. \textunderscore septizonium\textunderscore )}
\end{itemize}
Nome de alguns edificios romanos que, segundo uns, eram rodeados de sete ordens de colunas \textunderscore ou\textunderscore , segundo outros, formados de sete andares.
Os sete planetas da astronomia antiga.
\section{Seto}
\begin{itemize}
\item {Grp. gram.:m.}
\end{itemize}
\begin{itemize}
\item {Utilização:Prov.}
\end{itemize}
\begin{itemize}
\item {Utilização:trasm.}
\end{itemize}
\begin{itemize}
\item {Utilização:Ant.}
\end{itemize}
\begin{itemize}
\item {Proveniência:(Do lat. \textunderscore septum\textunderscore )}
\end{itemize}
O mesmo que \textunderscore sebe\textunderscore  ou \textunderscore estacada\textunderscore .
Caniço, á frente ou na retaguarda do carro de bois.
\section{Setófaga}
\begin{itemize}
\item {Grp. gram.:f.}
\end{itemize}
Gênero de aves da ordem dos pássaros.
\section{Sètoira}
\begin{itemize}
\item {Grp. gram.:f.}
\end{itemize}
\begin{itemize}
\item {Proveniência:(Do lat. \textunderscore sectoria\textunderscore )}
\end{itemize}
Foice para ceifar.
O mesmo que \textunderscore seitoira\textunderscore .
\section{Setóphaga}
\begin{itemize}
\item {Grp. gram.:f.}
\end{itemize}
Gênero de aves da ordem dos pássaros.
\section{Sétodo}
\begin{itemize}
\item {Grp. gram.:m.}
\end{itemize}
Gênero de insectos neurópteros.
\section{Setrossos}
\begin{itemize}
\item {Grp. gram.:m. Pl.}
\end{itemize}
Cavilhas nas carrêtas das peças de artilharia.
\section{Sètuagenário}
\begin{itemize}
\item {Grp. gram.:m.  e  adj.}
\end{itemize}
\begin{itemize}
\item {Proveniência:(Lat. \textunderscore septuagenarius\textunderscore )}
\end{itemize}
O que tem setenta anos, ou pouco mais ou menos.
\section{Sètuagésima}
\begin{itemize}
\item {Grp. gram.:f.}
\end{itemize}
Terceiro Domingo antes do primeiro da Quaresma.
(Fem. de \textunderscore sètuagésimo\textunderscore )
\section{Sètuagésimo}
\begin{itemize}
\item {Grp. gram.:adj.}
\end{itemize}
\begin{itemize}
\item {Proveniência:(Lat. \textunderscore septuagesimus\textunderscore )}
\end{itemize}
Relativo a setenta.
Que numa série de setenta occupa o último lugar.
\section{Setual}
\begin{itemize}
\item {Grp. gram.:m.}
\end{itemize}
\begin{itemize}
\item {Utilização:Ant.}
\end{itemize}
O mesmo que \textunderscore setial\textunderscore .
\section{Setúbal}
\begin{itemize}
\item {Grp. gram.:m.}
\end{itemize}
Vinho de Setúbal, provavelmente moscatel:«\textunderscore viva o setúbal que a tristeza afunda...\textunderscore »Filinto, IV, 75.
\section{Setubalense}
\begin{itemize}
\item {Grp. gram.:adj.}
\end{itemize}
\begin{itemize}
\item {Grp. gram.:M.}
\end{itemize}
Relativo a Setúbal.
Habitante de Setúbal.
\section{Sètupleta}
\begin{itemize}
\item {fónica:plê}
\end{itemize}
\begin{itemize}
\item {Grp. gram.:f.}
\end{itemize}
\begin{itemize}
\item {Utilização:Veloc.}
\end{itemize}
\begin{itemize}
\item {Proveniência:(De \textunderscore sètuplo\textunderscore )}
\end{itemize}
Velocípede com duas rodas, para sete pessoas.
\section{Sètuplicar}
\begin{itemize}
\item {Grp. gram.:v. t.}
\end{itemize}
\begin{itemize}
\item {Proveniência:(De \textunderscore sètuplo\textunderscore )}
\end{itemize}
Tornar sete vezes maior.
\section{Sètuplo}
\begin{itemize}
\item {Grp. gram.:adj.}
\end{itemize}
\begin{itemize}
\item {Proveniência:(Lat. \textunderscore septuplus\textunderscore )}
\end{itemize}
Que vale sete vezes outro, ou que é sete vezes maior que outro.
\section{Seu}
\begin{itemize}
\item {Grp. gram.:pron. adj.}
\end{itemize}
\begin{itemize}
\item {Grp. gram.:M.}
\end{itemize}
\begin{itemize}
\item {Proveniência:(Do lat. \textunderscore suus\textunderscore )}
\end{itemize}
(designativo da \textunderscore posse\textunderscore  que tem a pessôa, de quem se fala)
Relativo a êlle.
Próprio dêlle.
E usa-se na accepção de \textunderscore vosso\textunderscore , quando falamos com alguém a quem não damos o tratamento de tu: \textunderscore creia que sou amigo de seu pai\textunderscore .
Aquillo que pertence á pessôa ou pessôas, de quem se fala: \textunderscore as pessoas ricas dão do seu aos pobres\textunderscore .
\section{Seu}
\begin{itemize}
\item {Grp. gram.:m.}
\end{itemize}
O mesmo que \textunderscore sê\textunderscore .
\section{Seutera}
\begin{itemize}
\item {Grp. gram.:f.}
\end{itemize}
\begin{itemize}
\item {Proveniência:(De \textunderscore Seuter\textunderscore , n. p.)}
\end{itemize}
Gênero de plantas asclepiadáceas da América.
\section{Seutil}
\begin{itemize}
\item {Grp. gram.:m.}
\end{itemize}
Árvore americana, parecida com o limoeiro.
\section{Seu-vizinho}
\begin{itemize}
\item {Grp. gram.:m.}
\end{itemize}
\begin{itemize}
\item {Utilização:Chul.}
\end{itemize}
Dedo anular.
\section{Seva}
\begin{itemize}
\item {Grp. gram.:f.}
\end{itemize}
\begin{itemize}
\item {Utilização:Bras}
\end{itemize}
Acto de sevar.
\section{Seva}
\begin{itemize}
\item {Grp. gram.:f.}
\end{itemize}
\begin{itemize}
\item {Utilização:Bras}
\end{itemize}
Cipó ou corda horizontal, em que se penduram as fôlhas verdes do tabaco para secarem.
\section{Sevadeira}
\begin{itemize}
\item {Grp. gram.:f.}
\end{itemize}
\begin{itemize}
\item {Utilização:Bras}
\end{itemize}
\begin{itemize}
\item {Proveniência:(De \textunderscore sevar\textunderscore )}
\end{itemize}
Mulhér, empregada na seva^1.
\section{Sevamente}
\begin{itemize}
\item {Grp. gram.:adj.}
\end{itemize}
De modo sevo.
Deshumanamente. Cf. Camillo, \textunderscore Brasileira\textunderscore , 339.
\section{Sevandija}
\begin{itemize}
\item {Grp. gram.:f.}
\end{itemize}
\begin{itemize}
\item {Grp. gram.:M.  e  f.}
\end{itemize}
Insecto parasito e immundo.
Pessôa, vergonhosamente servil; parasito.
(Cast. \textunderscore sabandija\textunderscore )
\section{Sevandijar-se}
\begin{itemize}
\item {Grp. gram.:v. p.}
\end{itemize}
Tornar-se sevandija.
Aviltar-se, rebaixar-se indecorosamente.
\section{Sevar}
\begin{itemize}
\item {Grp. gram.:v. t.}
\end{itemize}
\begin{itemize}
\item {Utilização:Bras}
\end{itemize}
Ralar, reduzindo a farinha, (a mandioca).
(Talvez corr. de \textunderscore sovar\textunderscore )
\section{Sevas}
\begin{itemize}
\item {Grp. gram.:f. pl.}
\end{itemize}
\begin{itemize}
\item {Utilização:Bot.}
\end{itemize}
O mesmo que \textunderscore chalota\textunderscore .
\section{Sevasto}
\begin{itemize}
\item {Grp. gram.:m.}
\end{itemize}
O mesmo que \textunderscore sabasto\textunderscore .
\section{Sevastro}
\begin{itemize}
\item {Grp. gram.:m.}
\end{itemize}
\begin{itemize}
\item {Utilização:Ant.}
\end{itemize}
O mesmo que \textunderscore sabasto\textunderscore .
\section{Severamente}
\begin{itemize}
\item {Grp. gram.:adv.}
\end{itemize}
De modo severo.
Com severidade; com modos rudes ou ásperos.
\section{Severianos}
\begin{itemize}
\item {Grp. gram.:m. pl.}
\end{itemize}
Herejes, que negavam a canonicidade dos livros que a Igreja reputa canónicos.
\section{Severidade}
\begin{itemize}
\item {Grp. gram.:f.}
\end{itemize}
\begin{itemize}
\item {Proveniência:(Lat. \textunderscore severitas\textunderscore )}
\end{itemize}
Qualidade daquelle ou daquillo que é severo.
Inflexibilidade de carácter.
\section{Severino}
\begin{itemize}
\item {Grp. gram.:m.}
\end{itemize}
Espécie de peixe da Póvoa de Varzim, (\textunderscore heptanchus cinereus\textunderscore , Mull.).
\section{Severita}
\begin{itemize}
\item {Grp. gram.:f.}
\end{itemize}
\begin{itemize}
\item {Utilização:Miner.}
\end{itemize}
Variedade de hydro-silicato de alumina.
\section{Severite}
\begin{itemize}
\item {Grp. gram.:f.}
\end{itemize}
\begin{itemize}
\item {Utilização:Miner.}
\end{itemize}
Variedade de hydro-silicato de alumina.
\section{Severito}
\begin{itemize}
\item {Grp. gram.:m.}
\end{itemize}
O mesmo ou melhor que \textunderscore severita\textunderscore .
\section{Seixa}
\begin{itemize}
\item {Grp. gram.:f.}
\end{itemize}
Cobertura da cabeça, entre os Turcos. Cf. Pant. de Aveiro, \textunderscore Itiner.\textunderscore , 50 v.^o (2.^a ed.).
\section{Severizar}
\begin{itemize}
\item {Grp. gram.:v. t.}
\end{itemize}
Tornar severo:«\textunderscore severizando o semblante...\textunderscore »Camillo, \textunderscore Caveira\textunderscore , 145.
\section{Severo}
\begin{itemize}
\item {Grp. gram.:adj.}
\end{itemize}
\begin{itemize}
\item {Utilização:Fig.}
\end{itemize}
\begin{itemize}
\item {Proveniência:(Lat. \textunderscore severus\textunderscore )}
\end{itemize}
Rígido, austero.
Sério, grave.
Vehemente.
Áspero: \textunderscore descompostura severa\textunderscore .
Inflexível.
Exacto, pontual.
Simples e elegante, (falando-se do estilo).
Correcto, bem definido, accentuado.
\section{Sevícia}
\begin{itemize}
\item {Grp. gram.:f.}
\end{itemize}
(V.sevícias)
\section{Seviciar}
\begin{itemize}
\item {Grp. gram.:v. t.}
\end{itemize}
Causar sevícias a.
\section{Sevícias}
\begin{itemize}
\item {Grp. gram.:f. pl.}
\end{itemize}
\begin{itemize}
\item {Proveniência:(Lat. \textunderscore saevitia\textunderscore )}
\end{itemize}
Maus tratos; deshumanidade; actos de crueldade.
\section{Sevilhana}
\begin{itemize}
\item {Grp. gram.:f.}
\end{itemize}
\begin{itemize}
\item {Proveniência:(De \textunderscore sevilhano\textunderscore )}
\end{itemize}
Grande navalha, de fôlha curva e estreita.
Variedade de azeitona grande, o mesmo que \textunderscore redondil\textunderscore .
Canto popular de Sevilha.
Nome de uma ave gallinácea.
\section{Sevilhano}
\begin{itemize}
\item {Grp. gram.:adj.}
\end{itemize}
\begin{itemize}
\item {Grp. gram.:M.}
\end{itemize}
Relativo a Sevilha.
Habitante de Sevilha.
\section{Sevilhão}
\begin{itemize}
\item {Grp. gram.:m.  e  adj.}
\end{itemize}
\begin{itemize}
\item {Utilização:Ant.}
\end{itemize}
O mesmo que \textunderscore sevilhano\textunderscore . Cf. \textunderscore Chancell. de Aff. V\textunderscore , l. II, f. 7.
\section{Sevo}
\begin{itemize}
\item {Grp. gram.:adj.}
\end{itemize}
\begin{itemize}
\item {Utilização:Poét.}
\end{itemize}
\begin{itemize}
\item {Proveniência:(Lat. \textunderscore saevus\textunderscore )}
\end{itemize}
Deshumano.
Severo; cruel.
\section{Sevres}
\begin{itemize}
\item {Grp. gram.:m.}
\end{itemize}
Porcelana fabricada em Sevres: \textunderscore uma baixella de Sevres\textunderscore .
\section{Sex...}
\begin{itemize}
\item {Grp. gram.:pref.}
\end{itemize}
\begin{itemize}
\item {Proveniência:(Lat. \textunderscore sex\textunderscore )}
\end{itemize}
(designativo de \textunderscore seis\textunderscore )
\section{Sexa}
\begin{itemize}
\item {Grp. gram.:f.}
\end{itemize}
\begin{itemize}
\item {Proveniência:(Do quimb. \textunderscore séxi\textunderscore )}
\end{itemize}
Pequeno ruminante de Angola, o mesmo que \textunderscore seixa\textunderscore .
\section{Sexa}
\begin{itemize}
\item {Grp. gram.:f.}
\end{itemize}
Cobertura da cabeça, entre os Turcos. Cf. Pant. de Aveiro, \textunderscore Itiner.\textunderscore , 50 v.^o (2.^a ed.).
\section{Sexagenário}
\begin{itemize}
\item {fónica:csa}
\end{itemize}
\begin{itemize}
\item {Grp. gram.:m.  e  adj.}
\end{itemize}
\begin{itemize}
\item {Proveniência:(Lat. \textunderscore sexagenarius\textunderscore )}
\end{itemize}
Indivíduo, que tem sessenta annos.
\section{Sexagésima}
\begin{itemize}
\item {fónica:csa}
\end{itemize}
\begin{itemize}
\item {Grp. gram.:f.}
\end{itemize}
\begin{itemize}
\item {Proveniência:(De \textunderscore sexagésimo\textunderscore )}
\end{itemize}
Cada uma das sessenta partes que constituem um todo.
O Domingo que está quinze dias antes do primeiro Domingo da Quaresma.
\section{Sexagesimal}
\begin{itemize}
\item {fónica:csa}
\end{itemize}
\begin{itemize}
\item {Grp. gram.:adj.}
\end{itemize}
\begin{itemize}
\item {Proveniência:(De \textunderscore sexagésimo\textunderscore )}
\end{itemize}
Relativo a sessenta.
\section{Sexagésimo}
\begin{itemize}
\item {fónica:csa}
\end{itemize}
\begin{itemize}
\item {Grp. gram.:adj.}
\end{itemize}
\begin{itemize}
\item {Grp. gram.:M.}
\end{itemize}
\begin{itemize}
\item {Proveniência:(Lat. \textunderscore sexagesimus\textunderscore )}
\end{itemize}
Que numa série de sessenta occupa o último lugar.
Cada uma das sessenta partes de um todo.
\section{Sexangulado}
\begin{itemize}
\item {fónica:csan}
\end{itemize}
\begin{itemize}
\item {Grp. gram.:adj.}
\end{itemize}
\begin{itemize}
\item {Proveniência:(Lat. \textunderscore sexangulatus\textunderscore )}
\end{itemize}
Que tem seis ângulos.
\section{Sexangular}
\begin{itemize}
\item {fónica:csan}
\end{itemize}
\begin{itemize}
\item {Grp. gram.:adj.}
\end{itemize}
O mesmo que \textunderscore sexangulado\textunderscore .
\section{Sexangulo}
\begin{itemize}
\item {fónica:csan}
\end{itemize}
\begin{itemize}
\item {Grp. gram.:adj.}
\end{itemize}
O mesmo que \textunderscore sexangulado\textunderscore .
\section{Sexatria}
\begin{itemize}
\item {fónica:csa}
\end{itemize}
\begin{itemize}
\item {Grp. gram.:f.}
\end{itemize}
\begin{itemize}
\item {Proveniência:(Do lat. \textunderscore sexatrus\textunderscore )}
\end{itemize}
Festa, que os Romanos celebravam no sexto dia depois dos idos.
\section{Sexavô}
\begin{itemize}
\item {fónica:csa}
\end{itemize}
\begin{itemize}
\item {Grp. gram.:m.}
\end{itemize}
\begin{itemize}
\item {Utilização:Des.}
\end{itemize}
\begin{itemize}
\item {Proveniência:(De \textunderscore sex...\textunderscore  + \textunderscore avô\textunderscore )}
\end{itemize}
Sexto avô.
\section{Sexcellular}
\begin{itemize}
\item {fónica:secs}
\end{itemize}
\begin{itemize}
\item {Grp. gram.:adj.}
\end{itemize}
\begin{itemize}
\item {Utilização:Bot.}
\end{itemize}
Que apresenta seis céllulas.
\section{Sexcelular}
\begin{itemize}
\item {fónica:secs}
\end{itemize}
\begin{itemize}
\item {Grp. gram.:adj.}
\end{itemize}
\begin{itemize}
\item {Utilização:Bot.}
\end{itemize}
Que apresenta seis células.
\section{Sexcentésimo}
\begin{itemize}
\item {fónica:secs}
\end{itemize}
\begin{itemize}
\item {Grp. gram.:adj.}
\end{itemize}
\begin{itemize}
\item {Proveniência:(Lat. \textunderscore sexcentesimus\textunderscore )}
\end{itemize}
Último ou um de seiscentos.
\section{Sexcúncia}
\begin{itemize}
\item {Grp. gram.:f.}
\end{itemize}
Antiga moéda espanhola, que pesava onça e meia.
\section{Sexdigital}
\begin{itemize}
\item {fónica:secs}
\end{itemize}
\begin{itemize}
\item {Grp. gram.:adj.}
\end{itemize}
\begin{itemize}
\item {Proveniência:(Do lat. \textunderscore sex\textunderscore  + \textunderscore digitalis\textunderscore )}
\end{itemize}
Diz-se da mão ou do pé que tem seis dedos.
\section{Sexdigitário}
\begin{itemize}
\item {fónica:secs}
\end{itemize}
\begin{itemize}
\item {Grp. gram.:m.  e  adj.}
\end{itemize}
\begin{itemize}
\item {Proveniência:(De \textunderscore sex\textunderscore  + \textunderscore digitus\textunderscore , lat.)}
\end{itemize}
Indivíduo, que tem pé ou mão com seis dedos.
\section{Sexênio}
\begin{itemize}
\item {fónica:csê}
\end{itemize}
\begin{itemize}
\item {Grp. gram.:m.}
\end{itemize}
\begin{itemize}
\item {Proveniência:(Lat. \textunderscore sexennium\textunderscore )}
\end{itemize}
Espaço de seis annos.
\section{Sexennal}
\begin{itemize}
\item {fónica:cse}
\end{itemize}
\begin{itemize}
\item {Grp. gram.:adj.}
\end{itemize}
Relativo ao sexennio.
Que se realiza de seis em seis annos.
\section{Sexênnio}
\begin{itemize}
\item {fónica:csê}
\end{itemize}
\begin{itemize}
\item {Grp. gram.:m.}
\end{itemize}
\begin{itemize}
\item {Proveniência:(Lat. \textunderscore sexennium\textunderscore )}
\end{itemize}
Espaço de seis annos.
\section{Sexífero}
\begin{itemize}
\item {fónica:csi}
\end{itemize}
\begin{itemize}
\item {Grp. gram.:adj.}
\end{itemize}
\begin{itemize}
\item {Proveniência:(Do lat. \textunderscore sexum\textunderscore  + \textunderscore ferre\textunderscore )}
\end{itemize}
Que tem sexo.
\section{Sexjugado}
\begin{itemize}
\item {fónica:secs}
\end{itemize}
\begin{itemize}
\item {Grp. gram.:adj.}
\end{itemize}
\begin{itemize}
\item {Utilização:Bot.}
\end{itemize}
\begin{itemize}
\item {Proveniência:(Do lat. \textunderscore sex\textunderscore  + \textunderscore jugum\textunderscore )}
\end{itemize}
Diz-se das fôlhas, compostas pela reunião de seis pares de folíolos.
\section{Sexo}
\begin{itemize}
\item {fónica:cso}
\end{itemize}
\begin{itemize}
\item {Grp. gram.:m.}
\end{itemize}
\begin{itemize}
\item {Proveniência:(Lat. \textunderscore sexus\textunderscore )}
\end{itemize}
Conformação especial, que distingue do macho a fêmea, nos animaes e nos vegetaes.
Conjunto das pessôas, que têm a mesma conformação phýsica, consideradas sob o ponto de vista da geração.
\section{Sexta}
\begin{itemize}
\item {fónica:seis}
\end{itemize}
\begin{itemize}
\item {Grp. gram.:f.}
\end{itemize}
\begin{itemize}
\item {Utilização:Prov.}
\end{itemize}
\begin{itemize}
\item {Utilização:alg.}
\end{itemize}
\begin{itemize}
\item {Proveniência:(Lat. \textunderscore sexta\textunderscore )}
\end{itemize}
A terceira das quatro partes do dia, entre os Romanos.
Uma das horas canónicas, na liturgia cathólica.
Intervallo musical de seis notas.
O mesmo que \textunderscore sesta\textunderscore .
\section{Sextadecimanos}
\begin{itemize}
\item {fónica:seis}
\end{itemize}
\begin{itemize}
\item {Grp. gram.:m. pl.}
\end{itemize}
\begin{itemize}
\item {Proveniência:(Lat. \textunderscore sextadecimani\textunderscore )}
\end{itemize}
Soldados da 16.^a legião, entre os Romanos.
\section{Sexta-feira}
\begin{itemize}
\item {fónica:seis}
\end{itemize}
\begin{itemize}
\item {Grp. gram.:f.}
\end{itemize}
O sexto dia da semana.
\section{Sextanário}
\begin{itemize}
\item {fónica:seis}
\end{itemize}
\begin{itemize}
\item {Grp. gram.:m.}
\end{itemize}
\begin{itemize}
\item {Utilização:Ant.}
\end{itemize}
\begin{itemize}
\item {Proveniência:(De \textunderscore sexto\textunderscore )}
\end{itemize}
Sacerdote, que recebia a sexta parte da côngrua de um cónego.
\section{Sextanista}
\begin{itemize}
\item {fónica:seis}
\end{itemize}
\begin{itemize}
\item {Grp. gram.:m.}
\end{itemize}
Alumno do sexto anno de um curso.
\section{Sextannista}
\begin{itemize}
\item {fónica:seis}
\end{itemize}
\begin{itemize}
\item {Grp. gram.:m.}
\end{itemize}
Alumno do sexto anno de um curso.
\section{Sextante}
\begin{itemize}
\item {fónica:seis}
\end{itemize}
\begin{itemize}
\item {Grp. gram.:m.}
\end{itemize}
\begin{itemize}
\item {Proveniência:(Lat. \textunderscore sextans\textunderscore )}
\end{itemize}
Instrumento mathemático, para a medição dos ângulos.
Sexta parte de um círculo, ou arco de 60°.
Pequena constellação boreal.
\section{Sextário}
\begin{itemize}
\item {fónica:seis}
\end{itemize}
\begin{itemize}
\item {Grp. gram.:m.}
\end{itemize}
\begin{itemize}
\item {Proveniência:(Lat. \textunderscore sextarius\textunderscore )}
\end{itemize}
Medida romana, para líquidos e sêcos, correspondente á sexta parte do côngio. Cf. \textunderscore Museu Techn.\textunderscore , 34.
\section{Sextavar}
\begin{itemize}
\item {fónica:seis}
\end{itemize}
\begin{itemize}
\item {Grp. gram.:v. t.}
\end{itemize}
\begin{itemize}
\item {Proveniência:(De \textunderscore sexto\textunderscore , sob infl. da fórma \textunderscore oitavar\textunderscore )}
\end{itemize}
Talhar em fórma sexangular.
Dar seis faces a.
\section{Sexteiro}
\begin{itemize}
\item {fónica:seis}
\end{itemize}
\begin{itemize}
\item {Grp. gram.:m.}
\end{itemize}
\begin{itemize}
\item {Utilização:Ant.}
\end{itemize}
O mesmo ou melhor que \textunderscore sesteiro.\textunderscore 
\section{Sexteto}
\begin{itemize}
\item {fónica:seis,tê}
\end{itemize}
\begin{itemize}
\item {Grp. gram.:m.}
\end{itemize}
\begin{itemize}
\item {Proveniência:(It. \textunderscore sestetto\textunderscore )}
\end{itemize}
Composição musical para seis vozes ou seis instrumentos.
Conjunto dos músicos, que executam essa composição.
\section{Sextil}
\begin{itemize}
\item {Grp. gram.:adj.}
\end{itemize}
\begin{itemize}
\item {Proveniência:(Lat. \textunderscore sextilis\textunderscore )}
\end{itemize}
Diz-se do aspecto de dois astros, para que distam entre si 60°.
No calendário da primeira república francesa, dizia-se do anno que tinha um sexto dia complementar, e dizia-se dêsse mesmo dia.
\section{Sextilha}
\begin{itemize}
\item {Grp. gram.:f.}
\end{itemize}
\begin{itemize}
\item {Proveniência:(Do cast. \textunderscore sextilla\textunderscore )}
\end{itemize}
Estância de seis versos.
Composição poética, que abrange seis dessas estâncias.
\section{Sextilhão}
\begin{itemize}
\item {fónica:seis}
\end{itemize}
\begin{itemize}
\item {Grp. gram.:m.}
\end{itemize}
O mesmo ou melhor que \textunderscore sextillião\textunderscore .
\section{Sextilhão}
\begin{itemize}
\item {fónica:seis}
\end{itemize}
\begin{itemize}
\item {Grp. gram.:m.}
\end{itemize}
\begin{itemize}
\item {Proveniência:(De \textunderscore sexto\textunderscore )}
\end{itemize}
Mil quintilliões, segundo o systema francês; um milhão de quintilliões, segundo o systema inglês.
\section{Sextilião}
\begin{itemize}
\item {fónica:seis}
\end{itemize}
\begin{itemize}
\item {Grp. gram.:m.}
\end{itemize}
\begin{itemize}
\item {Proveniência:(De \textunderscore sexto\textunderscore )}
\end{itemize}
Mil quintiliões, segundo o sistema francês; um milhão de quintiliões, segundo o sistema inglês.
\section{Sextillião}
\begin{itemize}
\item {fónica:seis}
\end{itemize}
\begin{itemize}
\item {Grp. gram.:m.}
\end{itemize}
\begin{itemize}
\item {Proveniência:(De \textunderscore sexto\textunderscore )}
\end{itemize}
Mil quintilliões, segundo o systema francês; um milhão de quintilliões, segundo o systema inglês.
\section{Sextina}
\begin{itemize}
\item {fónica:seis}
\end{itemize}
\begin{itemize}
\item {Grp. gram.:f.}
\end{itemize}
O mesmo que \textunderscore sextilha\textunderscore .
\section{Sexto}
\begin{itemize}
\item {fónica:seis}
\end{itemize}
\begin{itemize}
\item {Grp. gram.:adj.}
\end{itemize}
\begin{itemize}
\item {Grp. gram.:M.}
\end{itemize}
\begin{itemize}
\item {Proveniência:(Lat. \textunderscore sextus\textunderscore )}
\end{itemize}
Que numa série de seis occupa o último lugar.
Sexta parte.
\section{Séxtulo}
\begin{itemize}
\item {fónica:seis}
\end{itemize}
\begin{itemize}
\item {Grp. gram.:m.}
\end{itemize}
\begin{itemize}
\item {Proveniência:(Lat. \textunderscore sextula\textunderscore )}
\end{itemize}
Pêso de quatro escrópulos, a sexta parte da onça.
\section{Séxtuor}
\begin{itemize}
\item {fónica:secs}
\end{itemize}
\begin{itemize}
\item {Grp. gram.:m.}
\end{itemize}
\begin{itemize}
\item {Proveniência:(Do lat. \textunderscore sex\textunderscore , influênciado por \textunderscore quatuor\textunderscore )}
\end{itemize}
Trecho musical, para sêr executado por seis vozes ou seis instrumentos.
\section{Sextupleta}
\begin{itemize}
\item {fónica:seis,plê}
\end{itemize}
\begin{itemize}
\item {Grp. gram.:f.}
\end{itemize}
\begin{itemize}
\item {Utilização:Veloc.}
\end{itemize}
\begin{itemize}
\item {Proveniência:(De \textunderscore sêxtuplo\textunderscore )}
\end{itemize}
Velocípede de duas rodas, para seis pessôas.
\section{Sêxtuplo}
\begin{itemize}
\item {fónica:seis}
\end{itemize}
\begin{itemize}
\item {Grp. gram.:adj.}
\end{itemize}
\begin{itemize}
\item {Grp. gram.:M.}
\end{itemize}
\begin{itemize}
\item {Proveniência:(Lat. \textunderscore sextuplus\textunderscore )}
\end{itemize}
Que é seis vezes maior que outro.
Aquillo que é seis vezes maior que outra coisa.
\section{Sexual}
\begin{itemize}
\item {fónica:csu}
\end{itemize}
\begin{itemize}
\item {Grp. gram.:adj.}
\end{itemize}
\begin{itemize}
\item {Proveniência:(Lat. \textunderscore sexualis\textunderscore )}
\end{itemize}
Relativo ao sexo; que tem sexo; que caracteriza o sexo.
\section{Sexualidade}
\begin{itemize}
\item {fónica:csu}
\end{itemize}
\begin{itemize}
\item {Grp. gram.:f.}
\end{itemize}
Qualidade do que é sexual.
\section{Sexualismo}
\begin{itemize}
\item {fónica:csu}
\end{itemize}
\begin{itemize}
\item {Grp. gram.:m.}
\end{itemize}
\begin{itemize}
\item {Proveniência:(De \textunderscore sexual\textunderscore )}
\end{itemize}
Estado do que tem sexo.
\section{Sezão}
\begin{itemize}
\item {Grp. gram.:f.}
\end{itemize}
Febre intermittente ou periódica.
(Por \textunderscore sazão\textunderscore , do lat. \textunderscore satio\textunderscore )
\section{Sezeno}
\begin{itemize}
\item {Grp. gram.:adj.}
\end{itemize}
\begin{itemize}
\item {Proveniência:(Fr. \textunderscore seizain\textunderscore )}
\end{itemize}
Dizia-se de uma espécie de pano, que tinha mil e seiscentos fios de urdidura.
\section{Sezonado}
\begin{itemize}
\item {Grp. gram.:adj.}
\end{itemize}
Que tem sezões.
\section{Sezonal}
\begin{itemize}
\item {Grp. gram.:adj.}
\end{itemize}
Relativo a sezão: \textunderscore febre sezonal\textunderscore .
\section{Sezonático}
\begin{itemize}
\item {Grp. gram.:adj.}
\end{itemize}
\begin{itemize}
\item {Proveniência:(De \textunderscore sezão\textunderscore )}
\end{itemize}
Que produz sezões.
Em que costuma haver sezões, (falando-se de certas localidades ou regiões).
Que soffre sezões.
\section{Sezónico}
\begin{itemize}
\item {Grp. gram.:adj.}
\end{itemize}
O mesmo que \textunderscore sezonal\textunderscore .
\section{Sezonígeno}
\begin{itemize}
\item {Grp. gram.:adj.}
\end{itemize}
\begin{itemize}
\item {Proveniência:(De \textunderscore sezão\textunderscore  + gr. \textunderscore genos\textunderscore )}
\end{itemize}
Que produz sezões.
\section{Sezonismo}
\begin{itemize}
\item {Grp. gram.:m.}
\end{itemize}
\begin{itemize}
\item {Proveniência:(De \textunderscore sezão\textunderscore )}
\end{itemize}
Infecção, determinada pela mordedura de certos insectos palustres, e manifestada em febres intermittentes.--É expressão modernamente preferida a \textunderscore impaludismo\textunderscore  e \textunderscore malaria\textunderscore .
\section{Sezonologia}
\begin{itemize}
\item {Grp. gram.:f.}
\end{itemize}
\begin{itemize}
\item {Proveniência:(De \textunderscore sezão\textunderscore  + gr. \textunderscore logos\textunderscore )}
\end{itemize}
Tratado á cêrca do sezonismo.
\section{Sezuto}
\begin{itemize}
\item {Grp. gram.:m.}
\end{itemize}
Uma das três linguas, que se falam no Baroce, em África.
\section{Shakespeareano}
\begin{itemize}
\item {Grp. gram.:adj.}
\end{itemize}
Relativo a Shakespeare ou ás suas obras. Cf. B. Pato, \textunderscore Ciprestes\textunderscore , 242.
\section{Si}
\begin{itemize}
\item {Proveniência:(Do lat. \textunderscore sibi\textunderscore )}
\end{itemize}
(flexão do pron. \textunderscore êlle\textunderscore , quando é precedido de prep.: \textunderscore tratar só de si\textunderscore )
Vulgarmente, mas incorrectamente, applica-se á pessôa a quem se fala: \textunderscore tenho pena de si\textunderscore .
\section{Si}
\begin{itemize}
\item {Grp. gram.:conj.}
\end{itemize}
\begin{itemize}
\item {Utilização:Bras}
\end{itemize}
\begin{itemize}
\item {Utilização:ant.}
\end{itemize}
O mesmo que \textunderscore se\textunderscore ^1:«\textunderscore ...nem é cheo de area, si se faz com diligencia...\textunderscore »Garcia Orta, \textunderscore Coll.\textunderscore  II.
\section{Si}
\begin{itemize}
\item {Grp. gram.:m.}
\end{itemize}
Sétima nota da escala musical.
Sinal, representativo dessa nota.
(Sýllaba, inventada para aquelle effeito por um flamengo)
\section{Si}
\begin{itemize}
\item {Grp. gram.:adv.}
\end{itemize}
O mesmo que \textunderscore sim\textunderscore :«\textunderscore digo-te que si\textunderscore ». G. Vicente, I, 226.
\section{Siá}
\begin{itemize}
\item {Grp. gram.:f.}
\end{itemize}
\begin{itemize}
\item {Utilização:Bras}
\end{itemize}
O mesmo que \textunderscore sinhá\textunderscore .
\section{Sia}
Fórma ant. da 3.^a pess. do indic. pres. do v. \textunderscore sêr\textunderscore . Cf. G. Vicente.
\section{Siagonagra}
\begin{itemize}
\item {Grp. gram.:f.}
\end{itemize}
\begin{itemize}
\item {Utilização:Med.}
\end{itemize}
\begin{itemize}
\item {Proveniência:(Do gr. \textunderscore siagon\textunderscore  + \textunderscore agra\textunderscore )}
\end{itemize}
Rheumatismo, na articulação da maxilla inferior.
\section{Siagónia}
\begin{itemize}
\item {Grp. gram.:f.}
\end{itemize}
\begin{itemize}
\item {Proveniência:(Do gr. \textunderscore siagon\textunderscore )}
\end{itemize}
Gênero de insectos coleópteros pentâmeros.
\section{Siagro}
\begin{itemize}
\item {Grp. gram.:m.}
\end{itemize}
\begin{itemize}
\item {Proveniência:(Do lat. \textunderscore syagrus\textunderscore )}
\end{itemize}
Gênero de palmeiras da região do Amazonas.
\section{Sialadenite}
\begin{itemize}
\item {Grp. gram.:f.}
\end{itemize}
\begin{itemize}
\item {Utilização:Med.}
\end{itemize}
\begin{itemize}
\item {Proveniência:(Do gr. \textunderscore sialon\textunderscore  + \textunderscore aden\textunderscore )}
\end{itemize}
Inflammação das glândulas salivares.
\section{Sialagogo}
\begin{itemize}
\item {Grp. gram.:m.  e  adj.}
\end{itemize}
\begin{itemize}
\item {Proveniência:(Do gr. \textunderscore sialon\textunderscore  + \textunderscore agein\textunderscore )}
\end{itemize}
Diz-se do medicamento, que provoca a salivação.
\section{Sialismo}
\begin{itemize}
\item {Grp. gram.:m.}
\end{itemize}
\begin{itemize}
\item {Proveniência:(Do gr. \textunderscore sialon\textunderscore )}
\end{itemize}
Abundância de salivação.
\section{Sialologia}
\begin{itemize}
\item {Grp. gram.:f.}
\end{itemize}
\begin{itemize}
\item {Proveniência:(Do gr. \textunderscore sialon\textunderscore  + \textunderscore logos\textunderscore )}
\end{itemize}
Parte da Anatomia e da Physiologia, que se occupa da saliva.
\section{Sialológico}
\begin{itemize}
\item {Grp. gram.:adj.}
\end{itemize}
Relativo á sialologia.
\section{Sialorreia}
\begin{itemize}
\item {Grp. gram.:f.}
\end{itemize}
\begin{itemize}
\item {Proveniência:(Do gr. \textunderscore sialon\textunderscore  + \textunderscore rhein\textunderscore )}
\end{itemize}
O mesmo que \textunderscore sialismo\textunderscore .
\section{Sialorrheia}
\begin{itemize}
\item {Grp. gram.:f.}
\end{itemize}
\begin{itemize}
\item {Proveniência:(Do gr. \textunderscore sialon\textunderscore  + \textunderscore rhein\textunderscore )}
\end{itemize}
O mesmo que \textunderscore sialismo\textunderscore .
\section{Siám}
\begin{itemize}
\item {Grp. gram.:m.}
\end{itemize}
Antiga peça de artilharia. Cf. \textunderscore Livro das Monções\textunderscore , 13.
\section{Siame}
\begin{itemize}
\item {Grp. gram.:m.  e  adj.}
\end{itemize}
O mesmo ou melhor que \textunderscore siamês\textunderscore . Cf. \textunderscore Peregrinação\textunderscore , I, XLVIII.
\section{Siamense}
\begin{itemize}
\item {Grp. gram.:m.  e  adj.}
\end{itemize}
O mesmo que \textunderscore siamês\textunderscore . Cf. Garret, \textunderscore Helena\textunderscore , 39.
\section{Siamês}
\begin{itemize}
\item {Grp. gram.:adj.}
\end{itemize}
\begin{itemize}
\item {Grp. gram.:M.}
\end{itemize}
Relativo a Sião.
Habitante de Sião.
Língua falada em Sião.
\section{Siampão}
\begin{itemize}
\item {Grp. gram.:m.}
\end{itemize}
Pequeno navio chinês, de vela e remos.
\section{Siar}
\begin{itemize}
\item {Grp. gram.:v. t.}
\end{itemize}
Fechar (as asas), para descer mais rapidamente.
\section{Siba}
\begin{itemize}
\item {Grp. gram.:f.}
\end{itemize}
\begin{itemize}
\item {Proveniência:(Do lat. \textunderscore sepia\textunderscore )}
\end{itemize}
Gênero de molluscos, que têm por typo o chôco vulgar.
\section{Siba}
\begin{itemize}
\item {Grp. gram.:f.}
\end{itemize}
Árvore de Timor, de madeira amarelada e setinosa. Cf. \textunderscore Século\textunderscore , de 30-VII-911.
\section{Sibadani}
\begin{itemize}
\item {Grp. gram.:m.}
\end{itemize}
Árvore da Guiana inglesa.
\section{Sibana}
\begin{itemize}
\item {Grp. gram.:f.}
\end{itemize}
\begin{itemize}
\item {Utilização:Ant.}
\end{itemize}
Cabana, palhoça.
\section{Sibar}
\begin{itemize}
\item {Grp. gram.:m.}
\end{itemize}
Barco asiático.
\section{Sibbáldia}
\begin{itemize}
\item {Grp. gram.:f.}
\end{itemize}
\begin{itemize}
\item {Proveniência:(De \textunderscore Sibbald\textunderscore , n. p.)}
\end{itemize}
Gênero de plantas rosáceas.
\section{Sibe}
\begin{itemize}
\item {Grp. gram.:m.}
\end{itemize}
Árvore da Índia portuguesa.
\section{Siberiano}
\begin{itemize}
\item {Grp. gram.:adj.}
\end{itemize}
\begin{itemize}
\item {Grp. gram.:M.}
\end{itemize}
Relativo á Sibéria.
Habitante da Sibéria.
\section{Siberite}
\begin{itemize}
\item {Grp. gram.:f.}
\end{itemize}
\begin{itemize}
\item {Utilização:Miner.}
\end{itemize}
Variedade de turmalina avermelhada.
\section{Siberito}
\begin{itemize}
\item {Grp. gram.:m.}
\end{itemize}
O mesmo ou melhor que \textunderscore siberite\textunderscore .
\section{Sibila}
\begin{itemize}
\item {Grp. gram.:f.}
\end{itemize}
\begin{itemize}
\item {Utilização:Fam.}
\end{itemize}
\begin{itemize}
\item {Proveniência:(Lat. \textunderscore sibylla\textunderscore )}
\end{itemize}
Prophetisa, entre os antigos.
Bruxa.
\section{Sibilação}
\begin{itemize}
\item {Grp. gram.:f.}
\end{itemize}
\begin{itemize}
\item {Utilização:Med.}
\end{itemize}
\begin{itemize}
\item {Proveniência:(Do lat. \textunderscore sibilatio\textunderscore )}
\end{itemize}
Acto ou effeito de sibilar.
Silvo.
Ruído nos órgãos respiratórios, semelhante a um silvo.
\section{Sibilamento}
\begin{itemize}
\item {Grp. gram.:m.}
\end{itemize}
O mesmo que \textunderscore sibilação\textunderscore .
\section{Sibilante}
\begin{itemize}
\item {Grp. gram.:adj.}
\end{itemize}
\begin{itemize}
\item {Proveniência:(Lat. \textunderscore sibilans\textunderscore )}
\end{itemize}
Que sibila.
\section{Sibilar}
\begin{itemize}
\item {Grp. gram.:v. i.}
\end{itemize}
\begin{itemize}
\item {Proveniência:(Lat. \textunderscore sibilare\textunderscore )}
\end{itemize}
Assobiar, silvar.
Produzir som agudo e prolongado, assoprando.
\section{Sibilino}
\begin{itemize}
\item {Grp. gram.:adj.}
\end{itemize}
\begin{itemize}
\item {Utilização:Fig.}
\end{itemize}
\begin{itemize}
\item {Proveniência:(Lat. \textunderscore sibyllinus\textunderscore )}
\end{itemize}
Relativo a sibila.
Enigmático; diffícil de compreender.
\section{Sibilismo}
\begin{itemize}
\item {Grp. gram.:m.}
\end{itemize}
Doutrina das sibilas.
As predicções sibilinas.
Crença nos livros das sibilas.
\section{Sibilista}
\begin{itemize}
\item {Grp. gram.:adj.}
\end{itemize}
Diz-se dos Cristãos, que nos livros sibilinos viam prophecias á cêrca de Cristo.
\section{Sibilítico}
\begin{itemize}
\item {Grp. gram.:adj.}
\end{itemize}
O mesmo que \textunderscore sibilino\textunderscore .
\section{Síbilo}
\begin{itemize}
\item {Grp. gram.:m.}
\end{itemize}
\begin{itemize}
\item {Proveniência:(Lat. \textunderscore sibilus\textunderscore )}
\end{itemize}
O mesmo que \textunderscore sibilação\textunderscore .
\section{Síbina}
\begin{itemize}
\item {Grp. gram.:f.}
\end{itemize}
\begin{itemize}
\item {Proveniência:(Lat. \textunderscore sibina\textunderscore )}
\end{itemize}
Espécie de lança ou chuço, com que os caçadores feriam os javalis.
\section{Sibipira}
\begin{itemize}
\item {Grp. gram.:f.}
\end{itemize}
O mesmo que \textunderscore sicupira\textunderscore .
\section{Sibitar}
\begin{itemize}
\item {Grp. gram.:v. i.}
\end{itemize}
\begin{itemize}
\item {Utilização:Náut.}
\end{itemize}
O mesmo que \textunderscore sibilar\textunderscore .
\section{Sibylla}
\begin{itemize}
\item {Grp. gram.:f.}
\end{itemize}
\begin{itemize}
\item {Utilização:Fam.}
\end{itemize}
\begin{itemize}
\item {Proveniência:(Lat. \textunderscore sibylla\textunderscore )}
\end{itemize}
Prophetisa, entre os antigos.
Bruxa.
\section{Sibyllino}
\begin{itemize}
\item {Grp. gram.:adj.}
\end{itemize}
\begin{itemize}
\item {Utilização:Fig.}
\end{itemize}
\begin{itemize}
\item {Proveniência:(Lat. \textunderscore sibyllinus\textunderscore )}
\end{itemize}
Relativo a sibylla.
Enigmático; diffícil de comprehender.
\section{Sibyllismo}
\begin{itemize}
\item {Grp. gram.:m.}
\end{itemize}
Doutrina das sibyllas.
As predicções sibyllinas.
Crença nos livros das sibyllas.
\section{Sibylista}
\begin{itemize}
\item {Grp. gram.:adj.}
\end{itemize}
Diz-se dos Christãos, que nos livros sibyllinos viam prophecias á cêrca de Christo.
\section{Sibyllitico}
\begin{itemize}
\item {Grp. gram.:adj.}
\end{itemize}
O mesmo que \textunderscore sibyllino\textunderscore .
\section{Sica}
\begin{itemize}
\item {Grp. gram.:f.}
\end{itemize}
\begin{itemize}
\item {Proveniência:(Lat. \textunderscore sica\textunderscore )}
\end{itemize}
Punhal dos antigos Romanos.
\section{Sicá}
\begin{itemize}
\item {Grp. gram.:m.}
\end{itemize}
Árvore santhomense, de fibras têxteis.
\section{Sicambros}
\begin{itemize}
\item {Grp. gram.:m. pl.}
\end{itemize}
\begin{itemize}
\item {Proveniência:(Lat. \textunderscore Sicambri\textunderscore )}
\end{itemize}
Antigo povo, que habitou a Germânia setentrional e que ao depois se misturou com os Francos.
\section{Sicano}
\begin{itemize}
\item {Grp. gram.:m.}
\end{itemize}
Gênero de insectos hemípteros.
\section{Sicanos}
\begin{itemize}
\item {Grp. gram.:m. pl.}
\end{itemize}
\begin{itemize}
\item {Proveniência:(Lat. \textunderscore sicani\textunderscore )}
\end{itemize}
Antigo povo da Sicília. Cf. \textunderscore Viriato Trág.\textunderscore , I, 68.
\section{Sicariato}
\begin{itemize}
\item {Grp. gram.:m.}
\end{itemize}
\begin{itemize}
\item {Utilização:P. us.}
\end{itemize}
Assassínio, commetido por um sicário.
\section{Sicário}
\begin{itemize}
\item {Grp. gram.:m.}
\end{itemize}
\begin{itemize}
\item {Proveniência:(Lat. \textunderscore sicarius\textunderscore )}
\end{itemize}
Assassino; faccinora; faquista.
\section{Sicatividade}
\begin{itemize}
\item {Grp. gram.:f.}
\end{itemize}
Qualidade de sicativo.
\section{Sicativo}
\begin{itemize}
\item {Grp. gram.:adj.}
\end{itemize}
\begin{itemize}
\item {Grp. gram.:M.}
\end{itemize}
\begin{itemize}
\item {Proveniência:(Lat. \textunderscore siccativus\textunderscore )}
\end{itemize}
O mesmo que \textunderscore secante\textunderscore ^3.
Medicamento, que faz secar ou cicatrizar feridas.
\section{Siccatividade}
\begin{itemize}
\item {Grp. gram.:f.}
\end{itemize}
Qualidade de siccativo.
\section{Siccativo}
\begin{itemize}
\item {Grp. gram.:adj.}
\end{itemize}
\begin{itemize}
\item {Grp. gram.:M.}
\end{itemize}
\begin{itemize}
\item {Proveniência:(Lat. \textunderscore siccativus\textunderscore )}
\end{itemize}
O mesmo que \textunderscore secante\textunderscore ^3.
Medicamento, que faz secar ou cicatrizar feridas.
\section{Sícera}
\begin{itemize}
\item {Grp. gram.:f.}
\end{itemize}
\begin{itemize}
\item {Proveniência:(Lat. \textunderscore sicera\textunderscore )}
\end{itemize}
Qualquer bebida enebriante, excepto o vinho.
Cerveja. Cf. Herculano, \textunderscore Eurico\textunderscore , 237.
\section{Sicia}
\begin{itemize}
\item {Grp. gram.:f.}
\end{itemize}
O mesmo que \textunderscore cia\textunderscore . Cf. P. Moraes, \textunderscore Zool. Elem.\textunderscore , 307.--Parece fórma incorrecta.
Cp. \textunderscore cicia\textunderscore .
\section{Siciliana}
\begin{itemize}
\item {Grp. gram.:f.}
\end{itemize}
\begin{itemize}
\item {Proveniência:(De \textunderscore siciliano\textunderscore )}
\end{itemize}
Ária e dança, originárias da Sicília.
\section{Siciliano}
\begin{itemize}
\item {Grp. gram.:adj.}
\end{itemize}
\begin{itemize}
\item {Grp. gram.:M.}
\end{itemize}
Relativo á Sicília.
Habitante da Sicília.
\section{Sicílico}
\begin{itemize}
\item {Grp. gram.:m.}
\end{itemize}
\begin{itemize}
\item {Proveniência:(Lat. \textunderscore sicilicus\textunderscore )}
\end{itemize}
A quarta parte da onça ou a quadragésima oitava parte de um asse, entre os antigos Romanos.
\section{Sicílio}
\begin{itemize}
\item {Grp. gram.:adj.}
\end{itemize}
Diz-se de uma variedade de trigo rijo.
\section{Sicorda}
\begin{itemize}
\item {Grp. gram.:f.}
\end{itemize}
\begin{itemize}
\item {Utilização:Náut.}
\end{itemize}
Cada uma das tábuas grossas, que servem de lados ás escotilhas.
\section{Sicrano}
\begin{itemize}
\item {Grp. gram.:m.}
\end{itemize}
Designação vulgar da segunda de duas pessôas indeterminadas, dando-se á primeira o nome de \textunderscore fulano\textunderscore .
\section{Sículo}
\begin{itemize}
\item {Grp. gram.:adj.}
\end{itemize}
\begin{itemize}
\item {Grp. gram.:M.}
\end{itemize}
\begin{itemize}
\item {Grp. gram.:Pl.}
\end{itemize}
\begin{itemize}
\item {Proveniência:(Lat. \textunderscore siculus\textunderscore )}
\end{itemize}
Relativo á Sicília.
Habitante da Sicília, siciliano.
Antigos povos das margens do Tibre, os quaes, repellidos para o Sul, se refugiaram na Trinácria, que dêlles recebeu o nome de \textunderscore Sicília\textunderscore .
\section{Sicupira}
\begin{itemize}
\item {Grp. gram.:f.}
\end{itemize}
Nome de duas árvores leguminosas do Brasil e da África, (\textunderscore pentaclethra macrophylla\textunderscore )--A sicupira é conhecida também por outros nomes, como \textunderscore sucupira\textunderscore  ou \textunderscore sucopira\textunderscore , \textunderscore sibipira\textunderscore , \textunderscore sepepira\textunderscore  ou \textunderscore sipipira\textunderscore , etc.
\section{Sida}
\begin{itemize}
\item {Grp. gram.:f.}
\end{itemize}
Gênero de plantas malváceas.
\section{Sídeas}
\begin{itemize}
\item {Grp. gram.:f. pl.}
\end{itemize}
Tríbo de plantas, que tem por typo a \textunderscore sida\textunderscore .
\section{Sideração}
\begin{itemize}
\item {Grp. gram.:f.}
\end{itemize}
\begin{itemize}
\item {Utilização:Ant.}
\end{itemize}
\begin{itemize}
\item {Proveniência:(Do lat. \textunderscore sideratio\textunderscore )}
\end{itemize}
Influência, attribuída a um astro, sôbre a vida ou saúde de alguém.
Horóscopo.
Acto ou effeito de fulminar.
Aniquilamento repentino.
Constituição dos astros.
\section{Sideral}
\begin{itemize}
\item {Grp. gram.:adj.}
\end{itemize}
\begin{itemize}
\item {Proveniência:(Lat. \textunderscore sideralis\textunderscore )}
\end{itemize}
Relativo aos astros; celeste.
\section{Sidéreo}
\begin{itemize}
\item {Grp. gram.:adj.}
\end{itemize}
\begin{itemize}
\item {Utilização:Poét.}
\end{itemize}
\begin{itemize}
\item {Proveniência:(Lat. \textunderscore sidereus\textunderscore )}
\end{itemize}
O mesmo que \textunderscore sideral\textunderscore .
\section{Sidérico}
\begin{itemize}
\item {Grp. gram.:adj.}
\end{itemize}
\begin{itemize}
\item {Proveniência:(De \textunderscore sidéreo\textunderscore )}
\end{itemize}
O mesmo que \textunderscore sideral\textunderscore .
Que provém dos astros.
\section{Sidérico}
\begin{itemize}
\item {Grp. gram.:adj.}
\end{itemize}
\begin{itemize}
\item {Proveniência:(Do gr. \textunderscore sideros\textunderscore )}
\end{itemize}
Relativo ao ferro.
\section{Siderismo}
\begin{itemize}
\item {Grp. gram.:m.}
\end{itemize}
\begin{itemize}
\item {Proveniência:(Do lat. \textunderscore sidus\textunderscore , \textunderscore sideris\textunderscore )}
\end{itemize}
Adoração dos astros.
\section{Siderita}
\begin{itemize}
\item {Grp. gram.:f.}
\end{itemize}
Planta, da fam. das labiadas.
O mesmo que \textunderscore siderite\textunderscore .
\section{Siderite}
\begin{itemize}
\item {Grp. gram.:f.}
\end{itemize}
\begin{itemize}
\item {Proveniência:(Lat. \textunderscore siderites\textunderscore )}
\end{itemize}
Substância metállica, em que há mistura de ferro.
\section{Sideritina}
\begin{itemize}
\item {Grp. gram.:f.}
\end{itemize}
\begin{itemize}
\item {Proveniência:(De \textunderscore siderite\textunderscore )}
\end{itemize}
Variedade de sulfato de ferro.
\section{Siderito}
\begin{itemize}
\item {Grp. gram.:m.}
\end{itemize}
O mesmo que \textunderscore siderite\textunderscore .
\section{Sídero-betão}
\begin{itemize}
\item {Grp. gram.:m.}
\end{itemize}
Combinação de ferro e betão, como material de construcção.
\section{Sideròcalcito}
\begin{itemize}
\item {Grp. gram.:m.}
\end{itemize}
Carbonato de cal e de magnésia.
\section{Siderochromo}
\begin{itemize}
\item {Grp. gram.:m.}
\end{itemize}
Chromato de ferro.
\section{Siderocimento}
\begin{itemize}
\item {Grp. gram.:m.}
\end{itemize}
Moderno material de construcção, em que entra ferro e cimento.
\section{Siderocromo}
\begin{itemize}
\item {Grp. gram.:m.}
\end{itemize}
Chromato de ferro.
\section{Siderodendro}
\begin{itemize}
\item {Grp. gram.:m.}
\end{itemize}
\begin{itemize}
\item {Proveniência:(Do gr. \textunderscore sideros\textunderscore  + \textunderscore dendron\textunderscore )}
\end{itemize}
Gênero de arvores rubiáceas da América tropical.
\section{Siderodromofobia}
\begin{itemize}
\item {Grp. gram.:f.}
\end{itemize}
\begin{itemize}
\item {Proveniência:(Do gr. \textunderscore sideros\textunderscore , ferro + \textunderscore dromos\textunderscore , carreira + \textunderscore phobein\textunderscore , temer)}
\end{itemize}
Mêdo mórbido dos caminhos de ferro.
\section{Siderodromófobo}
\begin{itemize}
\item {Grp. gram.:m.}
\end{itemize}
Aquele que sofre siderodromofobia.
\section{Siderodromophobia}
\begin{itemize}
\item {Grp. gram.:f.}
\end{itemize}
\begin{itemize}
\item {Proveniência:(Do gr. \textunderscore sideros\textunderscore , ferro + \textunderscore dromos\textunderscore , carreira + \textunderscore phobein\textunderscore , temer)}
\end{itemize}
Mêdo mórbido dos caminhos de ferro.
\section{Siderodromóphobo}
\begin{itemize}
\item {Grp. gram.:m.}
\end{itemize}
Aquelle que soffre siderodromophobia.
\section{Siderogastro}
\begin{itemize}
\item {Grp. gram.:m. adj.}
\end{itemize}
\begin{itemize}
\item {Utilização:Zool.}
\end{itemize}
\begin{itemize}
\item {Proveniência:(Do gr. \textunderscore sideros\textunderscore  + \textunderscore gaster\textunderscore )}
\end{itemize}
Que tem o ventre ferruginoso ou da côr da ferrugem.
\section{Siderografia}
\begin{itemize}
\item {Grp. gram.:f.}
\end{itemize}
\begin{itemize}
\item {Proveniência:(Do gr. \textunderscore sideros\textunderscore  + \textunderscore graphein\textunderscore )}
\end{itemize}
Arte de gravar em aço.
\section{Siderográfico}
\begin{itemize}
\item {Grp. gram.:adj.}
\end{itemize}
Relativo á siderografia.
\section{Siderógrafo}
\begin{itemize}
\item {Grp. gram.:m.}
\end{itemize}
Gravador em aço.
(Cp. \textunderscore siderographia\textunderscore )
\section{Siderographia}
\begin{itemize}
\item {Grp. gram.:f.}
\end{itemize}
\begin{itemize}
\item {Proveniência:(Do gr. \textunderscore sideros\textunderscore  + \textunderscore graphein\textunderscore )}
\end{itemize}
Arte de gravar em aço.
\section{Siderográphico}
\begin{itemize}
\item {Grp. gram.:adj.}
\end{itemize}
Relativo á siderographia.
\section{Siderógrapho}
\begin{itemize}
\item {Grp. gram.:m.}
\end{itemize}
Gravador em aço.
(Cp. \textunderscore siderographia\textunderscore )
\section{Siderólita}
\begin{itemize}
\item {Grp. gram.:f.}
\end{itemize}
O mesmo que \textunderscore siderólito\textunderscore .
\section{Siderólitha}
\begin{itemize}
\item {Grp. gram.:f.}
\end{itemize}
O mesmo que \textunderscore siderólitho\textunderscore .
\section{Siderolíthico}
\begin{itemize}
\item {Grp. gram.:adj.}
\end{itemize}
\begin{itemize}
\item {Proveniência:(De \textunderscore siderólitho\textunderscore )}
\end{itemize}
Que tem rochas ferruginosas.
\section{Siderólitho}
\begin{itemize}
\item {Grp. gram.:m.}
\end{itemize}
\begin{itemize}
\item {Proveniência:(Do gr. \textunderscore sideros\textunderscore  + \textunderscore lithos\textunderscore )}
\end{itemize}
Gênero de conchas fósseis, que se encontram abaixo do terreno cretáceo.
\section{Siderolítico}
\begin{itemize}
\item {Grp. gram.:adj.}
\end{itemize}
\begin{itemize}
\item {Proveniência:(De \textunderscore siderólito\textunderscore )}
\end{itemize}
Que tem rochas ferruginosas.
\section{Siderólito}
\begin{itemize}
\item {Grp. gram.:m.}
\end{itemize}
\begin{itemize}
\item {Proveniência:(Do gr. \textunderscore sideros\textunderscore  + \textunderscore lithos\textunderscore )}
\end{itemize}
Gênero de conchas fósseis, que se encontram abaixo do terreno cretáceo.
\section{Sideromancia}
\begin{itemize}
\item {Grp. gram.:f.}
\end{itemize}
\begin{itemize}
\item {Proveniência:(Do gr. \textunderscore sideros\textunderscore  + \textunderscore manteia\textunderscore )}
\end{itemize}
Supposta arte de adivinhar, por meio de uma barra de ferro candente, sôbre a qual se lançavam troços de palha, para se observar como ardiam e que direcção o fumo tomava.
\section{Sideromântico}
\begin{itemize}
\item {Grp. gram.:adj.}
\end{itemize}
Relativo a sideromancia.
Que praticava a sideromancia.
\section{Sideróporo}
\begin{itemize}
\item {Grp. gram.:m.}
\end{itemize}
\begin{itemize}
\item {Proveniência:(Do lat. \textunderscore sidus\textunderscore , \textunderscore sideris\textunderscore  + \textunderscore porus\textunderscore )}
\end{itemize}
Gênero de pólypos.
\section{Sideroscópio}
\begin{itemize}
\item {Grp. gram.:m.}
\end{itemize}
\begin{itemize}
\item {Proveniência:(Do gr. \textunderscore sideros\textunderscore  + \textunderscore skopein\textunderscore )}
\end{itemize}
Instrumento, para estudar a influência dos magnetes.
\section{Siderose}
\begin{itemize}
\item {Grp. gram.:f.}
\end{itemize}
\begin{itemize}
\item {Proveniência:(Do gr. \textunderscore sideros\textunderscore )}
\end{itemize}
Côr ferruginosa de qualquer parte do corpo.
Carbonato de ferro, que nalgumas regiões constitue o minério explorado para a extracção do ferro.
\section{Sideróstato}
\begin{itemize}
\item {Grp. gram.:m.}
\end{itemize}
\begin{itemize}
\item {Proveniência:(Do lat. \textunderscore sidus\textunderscore , \textunderscore sideris\textunderscore  + gr. \textunderscore statos\textunderscore )}
\end{itemize}
Apparelho, para estudar a luz dos astros.
\section{Siderotechnia}
\begin{itemize}
\item {Grp. gram.:f.}
\end{itemize}
\begin{itemize}
\item {Proveniência:(Do gr. \textunderscore sideros\textunderscore  + \textunderscore tekhne\textunderscore )}
\end{itemize}
Arte de trabalhar em ferro.
Arte de ferrador.
\section{Siderotéchnico}
\begin{itemize}
\item {Grp. gram.:adj.}
\end{itemize}
Relativo á siderotechnia.
\section{Siderotecnia}
\begin{itemize}
\item {Grp. gram.:f.}
\end{itemize}
\begin{itemize}
\item {Proveniência:(Do gr. \textunderscore sideros\textunderscore  + \textunderscore tekhne\textunderscore )}
\end{itemize}
Arte de trabalhar em ferro.
Arte de ferrador.
\section{Siderotécnico}
\begin{itemize}
\item {Grp. gram.:adj.}
\end{itemize}
Relativo á siderotecnia.
\section{Siderotério}
\begin{itemize}
\item {Grp. gram.:m.}
\end{itemize}
\begin{itemize}
\item {Proveniência:(Do gr. \textunderscore sideros\textunderscore  + \textunderscore therion\textunderscore )}
\end{itemize}
Gênero do paquidermes fósseis.
\section{Siderothério}
\begin{itemize}
\item {Grp. gram.:m.}
\end{itemize}
\begin{itemize}
\item {Proveniência:(Do gr. \textunderscore sideros\textunderscore  + \textunderscore therion\textunderscore )}
\end{itemize}
Gênero do pachydermes fósseis.
\section{Siderotina}
\begin{itemize}
\item {Grp. gram.:f.}
\end{itemize}
\begin{itemize}
\item {Utilização:Miner.}
\end{itemize}
Substância traslúcida, frágil e de aspecto resinoso, que consta de ácido arsenioso, ácido sulfúrico, peróxydo de ferro e água.
\section{Sideróxido}
\begin{itemize}
\item {fónica:csi}
\end{itemize}
\begin{itemize}
\item {Grp. gram.:m.}
\end{itemize}
\begin{itemize}
\item {Proveniência:(Do gr. \textunderscore sideros\textunderscore  + \textunderscore oxus\textunderscore )}
\end{itemize}
Qualquer ácido de ferro.
\section{Sideróxilo}
\begin{itemize}
\item {fónica:csi}
\end{itemize}
\begin{itemize}
\item {Grp. gram.:m.}
\end{itemize}
\begin{itemize}
\item {Proveniência:(Do gr. \textunderscore sideros\textunderscore  + \textunderscore xule\textunderscore )}
\end{itemize}
Gênero de plantas sapotáceas.
\section{Sideróxydo}
\begin{itemize}
\item {Grp. gram.:m.}
\end{itemize}
\begin{itemize}
\item {Proveniência:(Do gr. \textunderscore sideros\textunderscore  + \textunderscore oxus\textunderscore )}
\end{itemize}
Qualquer ácido de ferro.
\section{Sideróxylo}
\begin{itemize}
\item {Grp. gram.:m.}
\end{itemize}
\begin{itemize}
\item {Proveniência:(Do gr. \textunderscore sideros\textunderscore  + \textunderscore xule\textunderscore )}
\end{itemize}
Gênero de plantas sapotáceas.
\section{Siderurgia}
\begin{itemize}
\item {Grp. gram.:f.}
\end{itemize}
\begin{itemize}
\item {Proveniência:(Do gr. \textunderscore sideros\textunderscore  + \textunderscore ergon\textunderscore )}
\end{itemize}
O mesmo que \textunderscore siderotechnia\textunderscore .
\section{Siderúrgico}
\begin{itemize}
\item {Grp. gram.:adj.}
\end{itemize}
Relativo á siderurgia.
\section{Sidonal}
\begin{itemize}
\item {Grp. gram.:m.}
\end{itemize}
\begin{itemize}
\item {Utilização:Pharm.}
\end{itemize}
Quinato de piperazina, applicado contra a gota.
\section{Sidoniano}
\begin{itemize}
\item {Grp. gram.:m.  e  adj.}
\end{itemize}
O mesmo que \textunderscore sidónio\textunderscore .
\section{Sidónio}
\begin{itemize}
\item {Grp. gram.:adj.}
\end{itemize}
\begin{itemize}
\item {Grp. gram.:M.}
\end{itemize}
\begin{itemize}
\item {Proveniência:(Lat. \textunderscore sidonius\textunderscore )}
\end{itemize}
Relativo a Sidon, célebre cidade da Phenícia.
Habitante de Sidon:«\textunderscore ...sidonia origem...\textunderscore »Castilho, \textunderscore Fastos\textunderscore , III, 73.
\section{Sidra}
\begin{itemize}
\item {Grp. gram.:f.}
\end{itemize}
Espécie de vinho, feito do sumo de maçans.
(Fórma preferível a \textunderscore cidra\textunderscore , do fr. \textunderscore cidre\textunderscore . Cast. \textunderscore sidra\textunderscore , do lat. \textunderscore sicera\textunderscore . Cp. \textunderscore sícera\textunderscore )
\section{Sieda}
\begin{itemize}
\item {Grp. gram.:f.}
\end{itemize}
\begin{itemize}
\item {Utilização:Ant.}
\end{itemize}
O mesmo que \textunderscore séde\textunderscore .
\section{Siegesbéckia}
\begin{itemize}
\item {Grp. gram.:f.}
\end{itemize}
Gênero de plantas, da fam. das compostas.
\section{Sífano}
\begin{itemize}
\item {Grp. gram.:m.}
\end{itemize}
\begin{itemize}
\item {Utilização:Prov.}
\end{itemize}
\begin{itemize}
\item {Utilização:beir.}
\end{itemize}
Espécie de mosquito.
\section{Siga}
\begin{itemize}
\item {Grp. gram.:f.}
\end{itemize}
\begin{itemize}
\item {Utilização:Prov.}
\end{itemize}
Lâmina de ferro cortante, cravada no seitoril.
(Corr. de \textunderscore sega\textunderscore ?)
\section{Sigilação}
\begin{itemize}
\item {Grp. gram.:f.}
\end{itemize}
Acto ou effeito de sigilar.
\section{Sigilado}
\begin{itemize}
\item {Grp. gram.:adj.}
\end{itemize}
\begin{itemize}
\item {Proveniência:(De \textunderscore sigilar\textunderscore )}
\end{itemize}
Dizia-se de uma espécie de argila, a que se atribuiam propriedades medicinaes.
\section{Sigilária}
\begin{itemize}
\item {Grp. gram.:f.}
\end{itemize}
Gênero de plantas fósseis, cujas espécies são troncos de fêtos arborescentes.
\section{Sigilarias}
\begin{itemize}
\item {Grp. gram.:f. pl.}
\end{itemize}
\begin{itemize}
\item {Proveniência:(Lat. \textunderscore sigillaria\textunderscore )}
\end{itemize}
Ultimos dias das saturnaes, nos quaes se enviavam estatuetas ou pequenas figuras, como presente.
\section{Sigilismo}
\begin{itemize}
\item {Grp. gram.:m.}
\end{itemize}
\begin{itemize}
\item {Proveniência:(De \textunderscore sigilo\textunderscore )}
\end{itemize}
Cisma religioso, que appareceu em Coimbra no segundo quartel do séc. XVIII, e cujo êrro capital era a violação do sigilo da confissão.
Também se chamou \textunderscore jacobeia\textunderscore . Cf. Latino, \textunderscore Hist. Pol.\textunderscore , I, 100.
\section{Sigilista}
\begin{itemize}
\item {Grp. gram.:m.}
\end{itemize}
\begin{itemize}
\item {Grp. gram.:Adj.}
\end{itemize}
Séctário do sigilismo.
Em que há sigilo:«\textunderscore documentos sigilistas.\textunderscore »Herculano, \textunderscore Reacção Ultram.\textunderscore 
\section{Sigillação}
\begin{itemize}
\item {Grp. gram.:f.}
\end{itemize}
Acto ou effeito de sigillar.
\section{Sigillado}
\begin{itemize}
\item {Grp. gram.:adj.}
\end{itemize}
\begin{itemize}
\item {Proveniência:(De \textunderscore sigilar\textunderscore )}
\end{itemize}
Dizia-se de uma espécie de argilla, a que se attribuiam propriedades medicinaes.
\section{Sigillar}
\begin{itemize}
\item {Grp. gram.:v. t.}
\end{itemize}
\begin{itemize}
\item {Proveniência:(Lat. \textunderscore sigilare\textunderscore )}
\end{itemize}
O mesmo que \textunderscore sellar\textunderscore ^2
\section{Sigillária}
\begin{itemize}
\item {Grp. gram.:f.}
\end{itemize}
Gênero de plantas fósseis, cujas espécies são troncos de fêtos arborescentes.
\section{Sigillarias}
\begin{itemize}
\item {Grp. gram.:f. pl.}
\end{itemize}
\begin{itemize}
\item {Proveniência:(Lat. \textunderscore sigillaria\textunderscore )}
\end{itemize}
Ultimos dias das saturnaes, nos quaes se enviavam estatuetas ou pequenas figuras, como presente.
\section{Sigillismo}
\begin{itemize}
\item {Grp. gram.:m.}
\end{itemize}
\begin{itemize}
\item {Proveniência:(De \textunderscore sigillo\textunderscore )}
\end{itemize}
Scisma religioso, que appareceu em Coimbra no segundo quartel do séc. XVIII, e cujo êrro capital era a violação do sigillo da confissão.
Também se chamou \textunderscore jacobeia\textunderscore . Cf. Latino, \textunderscore Hist. Pol.\textunderscore , I, 100.
\section{Sigillista}
\begin{itemize}
\item {Grp. gram.:m.}
\end{itemize}
\begin{itemize}
\item {Grp. gram.:Adj.}
\end{itemize}
Séctário do sigillismo.
Em que há sigillo:«\textunderscore documentos sigillistas.\textunderscore »Herculano, \textunderscore Reacção Ultram.\textunderscore 
\section{Sigillo}
\begin{itemize}
\item {Grp. gram.:m.}
\end{itemize}
\begin{itemize}
\item {Utilização:Des.}
\end{itemize}
\begin{itemize}
\item {Proveniência:(Lat. \textunderscore sigillum\textunderscore )}
\end{itemize}
O mesmo que \textunderscore segrêdo\textunderscore .
O mesmo que \textunderscore sêllo\textunderscore .
\section{Sigilo}
\begin{itemize}
\item {Grp. gram.:m.}
\end{itemize}
\begin{itemize}
\item {Utilização:Des.}
\end{itemize}
\begin{itemize}
\item {Proveniência:(Lat. \textunderscore sigillum\textunderscore )}
\end{itemize}
O mesmo que \textunderscore segrêdo\textunderscore .
O mesmo que \textunderscore sêlo\textunderscore .
\section{Sigla}
\begin{itemize}
\item {Grp. gram.:f.}
\end{itemize}
\begin{itemize}
\item {Proveniência:(Lat. \textunderscore sigla\textunderscore )}
\end{itemize}
Letra inicial, empregada como abreviatura nos manuscritos, medalhas e monumentos antigos.
Monogramma.
\section{Sigma}
\begin{itemize}
\item {Grp. gram.:m.}
\end{itemize}
Letra do alphabeto grego, correspondente ao nosso \textunderscore s\textunderscore .
\section{Sigmatela}
\begin{itemize}
\item {Grp. gram.:f.}
\end{itemize}
\begin{itemize}
\item {Proveniência:(De \textunderscore sigma\textunderscore )}
\end{itemize}
Gênero de algas da Europa central.
\section{Sigmático}
\begin{itemize}
\item {Grp. gram.:adj.}
\end{itemize}
\begin{itemize}
\item {Utilização:Gram.}
\end{itemize}
\begin{itemize}
\item {Proveniência:(De \textunderscore sigma\textunderscore )}
\end{itemize}
Em que há a letra \textunderscore s\textunderscore .
Que tem \textunderscore s\textunderscore .
Que mantém o \textunderscore s\textunderscore .
\section{Sigmatismo}
\begin{itemize}
\item {Grp. gram.:m.}
\end{itemize}
\begin{itemize}
\item {Proveniência:(De \textunderscore sigma\textunderscore )}
\end{itemize}
Repetição viciosa da letra \textunderscore s\textunderscore  ou de outras sibilantes.
\section{Sigmodonte}
\begin{itemize}
\item {Grp. gram.:m.}
\end{itemize}
\begin{itemize}
\item {Proveniência:(Do gr. \textunderscore sigma\textunderscore  + \textunderscore odous\textunderscore )}
\end{itemize}
Gênero de mammíferos roedores.
\section{Sigmodostilo}
\begin{itemize}
\item {Grp. gram.:m.}
\end{itemize}
Gênero de plantas leguminosas.
\section{Sigmodostylo}
\begin{itemize}
\item {Grp. gram.:m.}
\end{itemize}
Gênero de plantas leguminosas.
\section{Sigmoídeo}
\begin{itemize}
\item {Grp. gram.:adj.}
\end{itemize}
\begin{itemize}
\item {Utilização:Anat.}
\end{itemize}
\begin{itemize}
\item {Proveniência:(Do gr. \textunderscore sigma\textunderscore  + \textunderscore eidos\textunderscore )}
\end{itemize}
Que tem fórma de sigma, (falando-se de certas cavidades e válvulas do corpo humano).
\section{Sigmoidite}
\begin{itemize}
\item {Grp. gram.:f.}
\end{itemize}
\begin{itemize}
\item {Utilização:Med.}
\end{itemize}
\begin{itemize}
\item {Proveniência:(Do gr. \textunderscore sigma\textunderscore  + \textunderscore eidos\textunderscore )}
\end{itemize}
Inflammação da quarta porção do cólon.
\section{Signa}
\begin{itemize}
\item {Grp. gram.:f.}
\end{itemize}
\begin{itemize}
\item {Proveniência:(Lat. \textunderscore signa\textunderscore )}
\end{itemize}
Bandeira; estandarte.
\section{Signáculo}
\begin{itemize}
\item {Grp. gram.:m.}
\end{itemize}
\begin{itemize}
\item {Utilização:Des.}
\end{itemize}
\begin{itemize}
\item {Proveniência:(Lat. \textunderscore signaculum\textunderscore )}
\end{itemize}
O mesmo que \textunderscore sêllo\textunderscore .
Cunho, vestigio. Cf. Bernárdez, \textunderscore Luz e Calor\textunderscore , 161.
\section{Signal}
\begin{itemize}
\item {Grp. gram.:m.}
\end{itemize}
\begin{itemize}
\item {Grp. gram.:Pl.}
\end{itemize}
\begin{itemize}
\item {Utilização:Ant.}
\end{itemize}
\begin{itemize}
\item {Proveniência:(Do lat. \textunderscore signalis\textunderscore )}
\end{itemize}
\textunderscore m.\textunderscore  (e seus der.)
(V. \textunderscore sinal\textunderscore , etc.)
Coisa, que serve de advertência.
Meio de transmittir, para longe ou para certa distância, mas á vista, ordens, notícias, etc.
Indícios.
Manifestação externa: \textunderscore êsse acto é signal de prudência\textunderscore .
Gesto.
Marca.
Letreiro.
Mancha na pelle: \textunderscore tem um signal na face\textunderscore .
Dinheiro ou objecto, que um dos contratantes deixa em poder do outro, para segurança da sua obrigação: \textunderscore ajustou a compra e deixou signal\textunderscore .
Preságio: \textunderscore aquella nuvem é signal de trovoada\textunderscore .
Firma de tabellião.
Firma de um signatário: \textunderscore o tabellião reconheceu o meu signal\textunderscore .
Traço ou traços de sentido convencional.
Feições do corpo humano.
Dobre de sinos, por finados.
Pedacinhos de tafetá, que as mulheres collavam ao rôsto para enfeite.
\section{Signar-se}
\begin{itemize}
\item {Grp. gram.:v. p.}
\end{itemize}
\begin{itemize}
\item {Utilização:Ant.}
\end{itemize}
O mesmo que \textunderscore persignar-se\textunderscore . Cf. \textunderscore Port. Mon. Hist.\textunderscore , \textunderscore Script.\textunderscore , 299.
\section{Signatário}
\begin{itemize}
\item {Grp. gram.:m.  e  adj.}
\end{itemize}
\begin{itemize}
\item {Proveniência:(Do lat. \textunderscore signatus\textunderscore )}
\end{itemize}
O que assina ou subscreve um documento.
\section{Signatura}
\begin{itemize}
\item {Grp. gram.:f.}
\end{itemize}
\begin{itemize}
\item {Utilização:Des.}
\end{itemize}
O mesmo que \textunderscore assinatura\textunderscore . Cf. \textunderscore Luz e Calor\textunderscore , 553.
\section{Signífero}
\begin{itemize}
\item {Grp. gram.:m.}
\end{itemize}
\begin{itemize}
\item {Utilização:Des.}
\end{itemize}
\begin{itemize}
\item {Proveniência:(Lat. \textunderscore signifer\textunderscore )}
\end{itemize}
O mesmo que \textunderscore porta-bandeira\textunderscore . Cf. C. da Ericeira, \textunderscore Henriqueida\textunderscore , III, 230.
\section{Significação}
\begin{itemize}
\item {Grp. gram.:f.}
\end{itemize}
\begin{itemize}
\item {Proveniência:(Do lat. \textunderscore significatio\textunderscore )}
\end{itemize}
Acto ou effeito de significar.
Aquillo que significa alguma coisa.
\section{Significado}
\begin{itemize}
\item {Grp. gram.:m.}
\end{itemize}
O mesmo que \textunderscore significação\textunderscore .
\section{Significador}
\begin{itemize}
\item {Grp. gram.:m.  e  adj.}
\end{itemize}
O que significa.
\section{Significança}
\begin{itemize}
\item {Grp. gram.:f.}
\end{itemize}
\begin{itemize}
\item {Utilização:Ant.}
\end{itemize}
O mesmo que \textunderscore significância\textunderscore .
\section{Significância}
\begin{itemize}
\item {Grp. gram.:f.}
\end{itemize}
O mesmo que \textunderscore significação\textunderscore . Cf. Camillo, \textunderscore Esqueleto\textunderscore , 271.
\section{Significante}
\begin{itemize}
\item {Grp. gram.:adj.}
\end{itemize}
\begin{itemize}
\item {Proveniência:(Lat. \textunderscore significans\textunderscore )}
\end{itemize}
O mesmo que \textunderscore significativo\textunderscore .
\section{Significar}
\begin{itemize}
\item {Grp. gram.:v. t.}
\end{itemize}
\begin{itemize}
\item {Proveniência:(Lat. \textunderscore significare\textunderscore )}
\end{itemize}
Dar sinal de.
Têr o sentido de: \textunderscore «prolfaças»significa parabens\textunderscore .
Denotar.
Exprimir: \textunderscore palavras, que significam ódio\textunderscore .
Manifestar-se por.
Notificar: \textunderscore significou-lhe o maior desprêzo\textunderscore .
\section{Significativamente}
\begin{itemize}
\item {Grp. gram.:adv.}
\end{itemize}
De modo significativo.
De modo expressivo.
\section{Significativo}
\begin{itemize}
\item {Grp. gram.:adj.}
\end{itemize}
\begin{itemize}
\item {Proveniência:(Lat. \textunderscore significativus\textunderscore )}
\end{itemize}
Que significa; que exprime claramente.
Que contém revelação interessante.
\section{Signo}
\begin{itemize}
\item {Grp. gram.:m.}
\end{itemize}
\begin{itemize}
\item {Utilização:Mús.}
\end{itemize}
\begin{itemize}
\item {Utilização:Ant.}
\end{itemize}
\begin{itemize}
\item {Proveniência:(Lat. \textunderscore signum\textunderscore )}
\end{itemize}
Cada uma das dôze partes, em que se divide o Zodiaco.
Constellação, correspondente a cada uma dessas dôze partes.
O mesmo que \textunderscore nota\textunderscore  (de música).
\section{Sigo}
\begin{itemize}
\item {Grp. gram.:pron.}
\end{itemize}
\begin{itemize}
\item {Utilização:Ant.}
\end{itemize}
\begin{itemize}
\item {Proveniência:(Do lat. \textunderscore secum\textunderscore )}
\end{itemize}
O mesmo que \textunderscore consigo\textunderscore .
\section{Sigro}
\begin{itemize}
\item {Grp. gram.:m.}
\end{itemize}
\begin{itemize}
\item {Utilização:Ant.}
\end{itemize}
O mesmo que \textunderscore século\textunderscore . Cf. G. Vicente, I, 175.
(Cast. \textunderscore siglo\textunderscore )
\section{Silagem}
\begin{itemize}
\item {Grp. gram.:f.}
\end{itemize}
O mesmo que \textunderscore ensilagem\textunderscore .
Farragem, que se tira dos silos, para alimento dos animaes.
\section{Silena}
\begin{itemize}
\item {Grp. gram.:f.}
\end{itemize}
\begin{itemize}
\item {Proveniência:(De \textunderscore sileno\textunderscore )}
\end{itemize}
Gênero de plantas, da fam. dos cravos.
\section{Silêncio}
\begin{itemize}
\item {Grp. gram.:m.}
\end{itemize}
\begin{itemize}
\item {Grp. gram.:Interj.}
\end{itemize}
\begin{itemize}
\item {Proveniência:(Lat. \textunderscore silentium\textunderscore )}
\end{itemize}
Estado de quem se abstém de falar.
Privação de falar.
Taciturnidade.
Interrupção de correspondência epistolar: \textunderscore não me escreve há muito, e estranho êste silêncio\textunderscore .
Interrupção de qualquer ruído.
Sossêgo.
Segrêdo.
(para mandar calar ou impor sossêgo)
\section{Silenciosa}
\begin{itemize}
\item {Grp. gram.:f.}
\end{itemize}
\begin{itemize}
\item {Proveniência:(De \textunderscore silencioso\textunderscore )}
\end{itemize}
Máquina de costura, que faz pouco barulho.
\section{Silenciosamente}
\begin{itemize}
\item {Grp. gram.:adv.}
\end{itemize}
De modo silencioso; em silêncio.
\section{Silencioso}
\begin{itemize}
\item {Grp. gram.:adj.}
\end{itemize}
\begin{itemize}
\item {Grp. gram.:M.}
\end{itemize}
\begin{itemize}
\item {Proveniência:(Do lat. \textunderscore silentiosus\textunderscore )}
\end{itemize}
Que não fala.
Que guarda silêncio.
Em que não há ruído: \textunderscore casa silenciosa\textunderscore .
Que não faz barulho.
Indivíduo silencioso.
Pássaro brasileiro, que se nutre das sementes que busca nos campos.
\section{Silêneas}
\begin{itemize}
\item {Grp. gram.:f. pl.}
\end{itemize}
\begin{itemize}
\item {Proveniência:(De \textunderscore silena\textunderscore )}
\end{itemize}
Família de plantas caryophylláceas.
\section{Silene-aurora}
\begin{itemize}
\item {Grp. gram.:f.}
\end{itemize}
Planta, também conhecida por \textunderscore dama-dos-jardins\textunderscore .
\section{Sileno}
\begin{itemize}
\item {Grp. gram.:m.}
\end{itemize}
\begin{itemize}
\item {Proveniência:(De \textunderscore Sileno\textunderscore , n. p.)}
\end{itemize}
Insecto lepidóptero diurno.
Quadrúpede de Ceilão.
\section{Silente}
\begin{itemize}
\item {Grp. gram.:adj.}
\end{itemize}
\begin{itemize}
\item {Utilização:Poét.}
\end{itemize}
\begin{itemize}
\item {Proveniência:(Lat. \textunderscore silens\textunderscore )}
\end{itemize}
O mesmo que \textunderscore silencioso\textunderscore .
\section{Síler}
\begin{itemize}
\item {Grp. gram.:m.}
\end{itemize}
\begin{itemize}
\item {Proveniência:(Lat. \textunderscore siler\textunderscore )}
\end{itemize}
Planta umbellífera.
\section{Siléria}
\begin{itemize}
\item {Grp. gram.:f.}
\end{itemize}
Espécie de tecido antigo.
\section{Sílex}
\begin{itemize}
\item {Grp. gram.:m.}
\end{itemize}
\begin{itemize}
\item {Proveniência:(Lat. \textunderscore silex\textunderscore )}
\end{itemize}
Gênero de pedras, que contém as duas espécies de quartzo e opala, constituídas pelo ácido sílico.
Pederneira.
Pl. \textunderscore sílices\textunderscore .
\section{Silha}
\begin{itemize}
\item {Grp. gram.:f.}
\end{itemize}
\begin{itemize}
\item {Utilização:Prov.}
\end{itemize}
\begin{itemize}
\item {Utilização:trasm.}
\end{itemize}
Cadeira, (des. nêste sentido).
Pedra, em que assenta o cortiço das abelhas.
Série de cortiços de abelhas, dispostos em lugares próprios, para a procriação dellas e fabricação do mel.
Um dos muros, que separam os compartimentos das marinhas.
Lugar certo ou próprio de qualquer coisa; paradeiro, estância.
(Cast. \textunderscore silla\textunderscore )
\section{Silhal}
\begin{itemize}
\item {Grp. gram.:m.}
\end{itemize}
Silha numerosa de abelhas.
Lugar, onde há silhas ou colmeias.
\section{Silhão}
\begin{itemize}
\item {Grp. gram.:m.}
\end{itemize}
Construcção ao meio de um fôsso, em redor de uma praça.
(Por \textunderscore cilhão\textunderscore , de \textunderscore cilha\textunderscore ?)
\section{Silhão}
\begin{itemize}
\item {Grp. gram.:m.}
\end{itemize}
\begin{itemize}
\item {Proveniência:(De \textunderscore silha\textunderscore )}
\end{itemize}
Sella grande, em que montam mulheres.
\section{Silhar}
\begin{itemize}
\item {Grp. gram.:m.}
\end{itemize}
\begin{itemize}
\item {Proveniência:(De \textunderscore silha\textunderscore )}
\end{itemize}
Pedra, lavrada em quadrado, para formação ou revestimento de paredes.
Pedra, que vai de uma face até meio da parede.
Pedra, em que assenta o cortiço das abelhas.
\section{Silharia}
\begin{itemize}
\item {Grp. gram.:f.}
\end{itemize}
\begin{itemize}
\item {Proveniência:(De \textunderscore silhar\textunderscore )}
\end{itemize}
Obra, em que há silhares.
\section{Silho}
\begin{itemize}
\item {Grp. gram.:m.}
\end{itemize}
\begin{itemize}
\item {Utilização:Prov.}
\end{itemize}
\begin{itemize}
\item {Utilização:trasm.}
\end{itemize}
Cântaro ou vaso antigo, com duas asas symétricas, que partem rentes da bôca e descansam no bojo; asado.
\section{Silhuêta}
\begin{itemize}
\item {Grp. gram.:f.}
\end{itemize}
\begin{itemize}
\item {Utilização:Gal}
\end{itemize}
\begin{itemize}
\item {Proveniência:(Do fr. \textunderscore Silhouette\textunderscore , n. p.)}
\end{itemize}
Desenho, que representa o perfil de uma pessôa, segundo os contornos que a sombra della projecta.
\section{Sílica}
\begin{itemize}
\item {Grp. gram.:f.}
\end{itemize}
(V.síliqua)
\section{Sílica}
\begin{itemize}
\item {Grp. gram.:f.}
\end{itemize}
\begin{itemize}
\item {Utilização:Miner.}
\end{itemize}
\begin{itemize}
\item {Proveniência:(De \textunderscore silex\textunderscore )}
\end{itemize}
Substância branca e sólida, o mesmo que óxydo de silício.
\section{Silicatado}
\begin{itemize}
\item {Grp. gram.:adj.}
\end{itemize}
\begin{itemize}
\item {Utilização:Geol.}
\end{itemize}
Diz-se das rochas, cujos elementos são silicatos.
\section{Silicatização}
\begin{itemize}
\item {Grp. gram.:f.}
\end{itemize}
Formação de silicato. Cf. \textunderscore Museu Techn.\textunderscore , 21.
\section{Silicato}
\begin{itemize}
\item {Grp. gram.:m.}
\end{itemize}
\begin{itemize}
\item {Proveniência:(De \textunderscore silício\textunderscore )}
\end{itemize}
Sal, produzido pela combinação do ácido silícico com uma base.
\section{Sílice}
\begin{itemize}
\item {Grp. gram.:m.}
\end{itemize}
O mesmo ou melhor que \textunderscore sílex\textunderscore . Cf. Garrett, \textunderscore Camões\textunderscore , 147.
\section{Silicérnio}
\begin{itemize}
\item {Grp. gram.:m.}
\end{itemize}
\begin{itemize}
\item {Proveniência:(Lat. \textunderscore silicernium\textunderscore )}
\end{itemize}
Ceremónia fúnebre, em que os Romanos distribuíam pelo povo carne crua, depois das exéquias de cidadão rico ou nobre.
\section{Silícico}
\begin{itemize}
\item {Grp. gram.:adj.}
\end{itemize}
Diz-se de todos os corpos compostos, que contém silício.
\section{Silicícola}
\begin{itemize}
\item {Grp. gram.:adj.}
\end{itemize}
\begin{itemize}
\item {Proveniência:(Do lat. \textunderscore silex\textunderscore  + \textunderscore colere\textunderscore )}
\end{itemize}
Diz-se das plantas que crescem de preferência nos terrenos silicosos.
\section{Silicífero}
\begin{itemize}
\item {Grp. gram.:adj.}
\end{itemize}
\begin{itemize}
\item {Proveniência:(Do lat. \textunderscore silex\textunderscore  + \textunderscore ferre\textunderscore )}
\end{itemize}
Que contém silício.
\section{Silicinos}
\begin{itemize}
\item {Grp. gram.:m. pl.}
\end{itemize}
\begin{itemize}
\item {Utilização:Miner.}
\end{itemize}
\begin{itemize}
\item {Proveniência:(Do lat. \textunderscore silex\textunderscore , \textunderscore silicis\textunderscore )}
\end{itemize}
Uma das quatro ordens em que se divide a classe dos oxysaes.
\section{Silício}
\begin{itemize}
\item {Grp. gram.:m.}
\end{itemize}
\begin{itemize}
\item {Utilização:Miner.}
\end{itemize}
\begin{itemize}
\item {Proveniência:(De \textunderscore silex\textunderscore )}
\end{itemize}
Corpo simples, que produz a sílica, combinando-se com o oxygênio.
\section{Sila}
\begin{itemize}
\item {Grp. gram.:f.}
\end{itemize}
\begin{itemize}
\item {Utilização:Ant.}
\end{itemize}
Mancha? arte má? Cf. G. Vicente, II, 351.
\section{Silfa}
\begin{itemize}
\item {Grp. gram.:f.}
\end{itemize}
\begin{itemize}
\item {Proveniência:(Do gr. \textunderscore silphe\textunderscore )}
\end{itemize}
Gênero de insectos coleópteros.
Broqueleira.
\section{Silicioso}
\begin{itemize}
\item {Grp. gram.:adj.}
\end{itemize}
\begin{itemize}
\item {Proveniência:(De \textunderscore silício\textunderscore )}
\end{itemize}
Que tem a natureza do sílex; que contém sílica.
\section{Silicita}
\begin{itemize}
\item {Grp. gram.:f.}
\end{itemize}
\begin{itemize}
\item {Proveniência:(De \textunderscore sílica\textunderscore )}
\end{itemize}
Substância mineral, muito rica de sílica e que se encontra especialmente na Irlanda.
\section{Silicite}
\begin{itemize}
\item {Grp. gram.:f.}
\end{itemize}
O mesmo que \textunderscore silicito\textunderscore .
\section{Silicito}
\begin{itemize}
\item {Grp. gram.:m.}
\end{itemize}
\begin{itemize}
\item {Utilização:Miner.}
\end{itemize}
O mesmo ou melhor que \textunderscore silicita\textunderscore .
\section{Silícula}
\begin{itemize}
\item {Grp. gram.:f.}
\end{itemize}
\begin{itemize}
\item {Proveniência:(Lat. \textunderscore silicula\textunderscore )}
\end{itemize}
Pequena síliqua.
Espécie de pericarpo.
\section{Siliculiforme}
\begin{itemize}
\item {Grp. gram.:adj.}
\end{itemize}
\begin{itemize}
\item {Utilização:Bot.}
\end{itemize}
Que tem fórma de silícula.
\section{Siliculoso}
\begin{itemize}
\item {Grp. gram.:adj.}
\end{itemize}
Que tem silícula.
\section{Siligem}
\begin{itemize}
\item {Grp. gram.:f.}
\end{itemize}
O mesmo ou melhor que \textunderscore siligo\textunderscore .
\section{Siliginário}
\begin{itemize}
\item {Grp. gram.:m.}
\end{itemize}
\begin{itemize}
\item {Proveniência:(Lat. \textunderscore siliginarius\textunderscore )}
\end{itemize}
O mesmo que \textunderscore pasteleiro\textunderscore . Cf. Castilho, \textunderscore Fastos\textunderscore , III, 484.
\section{Siligo}
\begin{itemize}
\item {Grp. gram.:m.}
\end{itemize}
\begin{itemize}
\item {Proveniência:(Lat. \textunderscore siligo\textunderscore )}
\end{itemize}
Pão de primeira qualidade. Cf. Castilho, \textunderscore Fastos\textunderscore , III, 479 e 484.
\section{Silindra}
\begin{itemize}
\item {Grp. gram.:f.}
\end{itemize}
Designação vulgar de uma planta, cujas flôres brancas têm aroma penetrante e suavíssimo, (\textunderscore philadelphus coronarius\textunderscore , Lin.).
(Por \textunderscore syringa\textunderscore , do lat. \textunderscore syrinx\textunderscore , frauta de pau. Cp. \textunderscore seringar\textunderscore )
\section{Silingórnio}
\begin{itemize}
\item {Grp. gram.:adj.}
\end{itemize}
\begin{itemize}
\item {Utilização:Chul.}
\end{itemize}
Sonso, que fala mansamente para enganar.
\section{Silingos}
\begin{itemize}
\item {Grp. gram.:m. pl.}
\end{itemize}
Ramo de Vândalos, que se estabeleceu na Andaluzia. Cf. Herculano, \textunderscore Hist. de Port.\textunderscore , I, 28.
\section{Silingues}
\begin{itemize}
\item {Grp. gram.:m. pl.}
\end{itemize}
(V.Silingos)
\section{Síliqua}
\begin{itemize}
\item {Grp. gram.:f.}
\end{itemize}
\begin{itemize}
\item {Utilização:Bot.}
\end{itemize}
\begin{itemize}
\item {Proveniência:(Lat. \textunderscore silíqua\textunderscore )}
\end{itemize}
Fruto sêco, alongado e bivalve, cujos grãos adherem alternadamente a duas suturas longitudinaes e oppostas.
\section{Siliquiforme}
\begin{itemize}
\item {fónica:cu-i}
\end{itemize}
\begin{itemize}
\item {Grp. gram.:adj.}
\end{itemize}
\begin{itemize}
\item {Utilização:Bot.}
\end{itemize}
\begin{itemize}
\item {Proveniência:(De \textunderscore silíqua\textunderscore  + \textunderscore fórma\textunderscore )}
\end{itemize}
Que tem fórma de silíqua.
\section{Siliquoso}
\begin{itemize}
\item {fónica:cu-ô}
\end{itemize}
\begin{itemize}
\item {Grp. gram.:adj.}
\end{itemize}
Que tem silíqua ou que é da natureza della.
\section{Silla}
Mancha? arte má? Cf. G. Vicente, II, 351.
\section{Sillo}
\begin{itemize}
\item {Grp. gram.:m.}
\end{itemize}
\begin{itemize}
\item {Proveniência:(Do gr. \textunderscore sillos\textunderscore )}
\end{itemize}
Poema satírico, entre os Gregos, em fórma de paródia.
\section{Sillographia}
\begin{itemize}
\item {Grp. gram.:f.}
\end{itemize}
\begin{itemize}
\item {Proveniência:(De \textunderscore sillógrapho\textunderscore )}
\end{itemize}
Arte de compor sillos.
\section{Sillográphico}
\begin{itemize}
\item {Grp. gram.:adj.}
\end{itemize}
Relativo á sillographia.
\section{Sillógrapho}
\begin{itemize}
\item {Grp. gram.:m.}
\end{itemize}
\begin{itemize}
\item {Proveniência:(Do gr. \textunderscore sillos\textunderscore  + \textunderscore graphein\textunderscore )}
\end{itemize}
Autor de sillos.
\section{Silo}
\begin{itemize}
\item {Grp. gram.:m.}
\end{itemize}
Tulha subterrânea.
Construcção impermeável, onde se conservam forragens verdes e suculentas, para alimento de animaes.
(Cast. \textunderscore silo\textunderscore , do gr. \textunderscore siros\textunderscore )
\section{Silo}
\begin{itemize}
\item {Grp. gram.:m.}
\end{itemize}
\begin{itemize}
\item {Proveniência:(Do gr. \textunderscore sillos\textunderscore )}
\end{itemize}
Poema satírico, entre os Gregos, em fórma de paródia.
\section{Silografia}
\begin{itemize}
\item {Grp. gram.:f.}
\end{itemize}
\begin{itemize}
\item {Proveniência:(De \textunderscore silógrafo\textunderscore )}
\end{itemize}
Arte de compor silos.
\section{Silográfico}
\begin{itemize}
\item {Grp. gram.:adj.}
\end{itemize}
Relativo á silografia.
\section{Silógrafo}
\begin{itemize}
\item {Grp. gram.:m.}
\end{itemize}
\begin{itemize}
\item {Proveniência:(Do gr. \textunderscore sillos\textunderscore  + \textunderscore graphein\textunderscore )}
\end{itemize}
Autor de silos.
\section{Silpha}
\begin{itemize}
\item {Grp. gram.:f.}
\end{itemize}
\begin{itemize}
\item {Proveniência:(Do gr. \textunderscore silphe\textunderscore )}
\end{itemize}
Gênero de insectos coleópteros.
Broqueleira.
\section{Siluriano}
\begin{itemize}
\item {Grp. gram.:adj.}
\end{itemize}
\begin{itemize}
\item {Utilização:Geol.}
\end{itemize}
\begin{itemize}
\item {Proveniência:(De \textunderscore siluro\textunderscore ^1)}
\end{itemize}
Diz-se do terreno, em que há camadas fossilíferas, dispostas debaixo do antigo grés vermelho, e que pertence á terceira camada ou período geológico.
\section{Silúrico}
\begin{itemize}
\item {Grp. gram.:adj.}
\end{itemize}
\begin{itemize}
\item {Proveniência:(De \textunderscore siluro\textunderscore ^1)}
\end{itemize}
Relativo aos Siluros.
\section{Silúridas}
\begin{itemize}
\item {Grp. gram.:m. pl.}
\end{itemize}
Família de peixes, que tem por typo o \textunderscore siluro\textunderscore ^2.
\section{Siluro}
\begin{itemize}
\item {Grp. gram.:m.}
\end{itemize}
Antigo habitante do país de Galles.
\section{Siluro}
\begin{itemize}
\item {Grp. gram.:m.}
\end{itemize}
\begin{itemize}
\item {Proveniência:(Lat. \textunderscore silurus\textunderscore )}
\end{itemize}
Gênero de peixes abdominaes.
\section{Siluróides}
\begin{itemize}
\item {Grp. gram.:m. pl.}
\end{itemize}
\begin{itemize}
\item {Proveniência:(Do gr. \textunderscore silouros\textunderscore  + \textunderscore eidos\textunderscore )}
\end{itemize}
Peixes, comprehendidos nos ganóides, segundo Agassiz.
\section{Silva}
\begin{itemize}
\item {Grp. gram.:f.}
\end{itemize}
\begin{itemize}
\item {Proveniência:(Lat. \textunderscore silva\textunderscore )}
\end{itemize}
Nome de várias plantas rosáceas, especialmente da silva das amoras.
Composição poética, em que se alternam versos de déz e de seis sýllabas.
Miscellânea literária.
Cilício de arame.
Mancha alongada, ao lado das ventas do cavallo.
\section{Silva-da-praia}
\begin{itemize}
\item {Grp. gram.:f.}
\end{itemize}
Planta cesalpínea da Índia Portuguesa, (\textunderscore caesalpinea bondusella\textunderscore ; Fleming).
\section{Silvado}
\begin{itemize}
\item {Grp. gram.:m.}
\end{itemize}
Moita de silvas.
Tapume, feito de silvas.
Nome de várias plantas rosáceas da Ilha da Madeira, (\textunderscore rubus Bollei\textunderscore , Focke; \textunderscore rubus ulmifolius\textunderscore , Schottt.; \textunderscore rubus grandifolius\textunderscore , Lowe). Cf. \textunderscore Bol.\textunderscore  da Socied. de Geogr., XXX, 613.
\section{Silvado}
\begin{itemize}
\item {Grp. gram.:adj.}
\end{itemize}
\begin{itemize}
\item {Proveniência:(De \textunderscore silva\textunderscore ?)}
\end{itemize}
Diz-se do toiro, que tem pequenas manchas brancas na frente, sendo escuro o resto da cabeça.
\section{Silva-macha}
\begin{itemize}
\item {Grp. gram.:f.}
\end{itemize}
O mesmo que \textunderscore roseira-canina\textunderscore .
\section{Silvandra}
\begin{itemize}
\item {Grp. gram.:f.}
\end{itemize}
Insecto lepidóptero.
\section{Silvano}
\begin{itemize}
\item {Grp. gram.:m.  e  adj.}
\end{itemize}
\begin{itemize}
\item {Utilização:Des.}
\end{itemize}
\begin{itemize}
\item {Proveniência:(De \textunderscore silva\textunderscore )}
\end{itemize}
Homem rústico; lapúrdio.
\section{Silvão}
\begin{itemize}
\item {Grp. gram.:m.}
\end{itemize}
Espécie de silva, (\textunderscore rubus canina\textunderscore ).
\section{Silvar}
\begin{itemize}
\item {Grp. gram.:v. i.}
\end{itemize}
O mesmo que \textunderscore sibilar\textunderscore .
\section{Silvas}
\begin{itemize}
\item {Grp. gram.:f. pl.}
\end{itemize}
\begin{itemize}
\item {Utilização:Bras}
\end{itemize}
Campos geraes da Guiana.
\section{Silvático}
\begin{itemize}
\item {Grp. gram.:adj.}
\end{itemize}
\begin{itemize}
\item {Proveniência:(Lat. \textunderscore silvaticus\textunderscore )}
\end{itemize}
O mesmo que \textunderscore selvático\textunderscore .
\section{Silvedo}
\begin{itemize}
\item {fónica:vê}
\end{itemize}
\begin{itemize}
\item {Grp. gram.:m.}
\end{itemize}
O mesmo que \textunderscore silvado\textunderscore ^1.
\section{Silveira}
\begin{itemize}
\item {Grp. gram.:f.}
\end{itemize}
Moita de silvas, silvado.
Silva.
Árvore de Angola.
\section{Silveiral}
\begin{itemize}
\item {Grp. gram.:adj.}
\end{itemize}
Relativo a silveira. Cf. Camillo, \textunderscore Estrêll. Prop.\textunderscore , 130.
\section{Silveiro}
\begin{itemize}
\item {Grp. gram.:adj.}
\end{itemize}
Diz-se do toiro, que tem uma malha branca na testa, sendo escura a cabeça.
(Cp. \textunderscore silvado\textunderscore ^2)
\section{Silvestre}
\begin{itemize}
\item {Grp. gram.:adj.}
\end{itemize}
\begin{itemize}
\item {Proveniência:(Lat. \textunderscore silvestris\textunderscore )}
\end{itemize}
O mesmo que \textunderscore selvático\textunderscore .
Que dá flôres ou frutos, sem necessidade de cultura: \textunderscore figueira silvestre\textunderscore .
Que não dá frutos; bravio, sáfaro.
\section{Sílvico}
\begin{itemize}
\item {Grp. gram.:adj.}
\end{itemize}
\begin{itemize}
\item {Proveniência:(Do lat. \textunderscore silva\textunderscore )}
\end{itemize}
Diz-se de um dos ácidos, contidos na resina do pinheiro. Cf. \textunderscore Techn. Rur.\textunderscore , 335.
\section{Silvícola}
\begin{itemize}
\item {Grp. gram.:m.  e  adj.}
\end{itemize}
\begin{itemize}
\item {Proveniência:(Lat. \textunderscore silvicola\textunderscore )}
\end{itemize}
Pessôa, que nasce ou vive nas selvas ou matas.
\section{Silvicultor}
\begin{itemize}
\item {Grp. gram.:m.}
\end{itemize}
Aquelle que se dedica á silvicultura ou que trata da cultura ou desenvolvimento de floresta.
(Cp. \textunderscore silvicultura\textunderscore )
\section{Silvicultura}
\begin{itemize}
\item {Grp. gram.:f.}
\end{itemize}
\begin{itemize}
\item {Proveniência:(Do lat. \textunderscore silva\textunderscore  + \textunderscore cultura\textunderscore )}
\end{itemize}
Tratado ou estudo da cultura das florestas.
Cultura das florestas.
\section{Silviense}
\begin{itemize}
\item {Grp. gram.:adj.}
\end{itemize}
Relativo á cidade de Silves:«\textunderscore o Prelado silviense...\textunderscore »Herculano, \textunderscore Hist. de Port.\textunderscore , III, 28.
\section{Silvina}
\begin{itemize}
\item {Grp. gram.:f.}
\end{itemize}
\begin{itemize}
\item {Utilização:Miner.}
\end{itemize}
Chloreto de potássio.
\section{Silvo}
\begin{itemize}
\item {Grp. gram.:m.}
\end{itemize}
\begin{itemize}
\item {Proveniência:(Do lat. \textunderscore sibilus\textunderscore  &lt; \textunderscore silibus\textunderscore  &lt; \textunderscore sílibo\textunderscore  &lt; \textunderscore sílbo\textunderscore  &lt; \textunderscore silvo\textunderscore )}
\end{itemize}
O mesmo que \textunderscore síbilo\textunderscore .
Som agudo, que as serpentes soltam.
\section{Silvoso}
\begin{itemize}
\item {Grp. gram.:adj.}
\end{itemize}
\begin{itemize}
\item {Proveniência:(Lat. \textunderscore silvosus\textunderscore )}
\end{itemize}
Em que há silvas; vedado com silvas.
\section{Sim}
\begin{itemize}
\item {Grp. gram.:adv.}
\end{itemize}
\begin{itemize}
\item {Utilização:Ant.}
\end{itemize}
\begin{itemize}
\item {Grp. gram.:M.}
\end{itemize}
\begin{itemize}
\item {Proveniência:(Do lat. \textunderscore sic.\textunderscore )}
\end{itemize}
(designativo de \textunderscore affirmação\textunderscore , \textunderscore acôrdo\textunderscore  ou \textunderscore permissão\textunderscore ).
O mesmo que \textunderscore assim\textunderscore . Cf. Pant. de Aveiro, \textunderscore Itiner.\textunderscore , 198 v.^o, (2.^a ed.).
Acto de consentir, expresso pela palavra \textunderscore sim\textunderscore .
\section{Simão}
\begin{itemize}
\item {Grp. gram.:m.}
\end{itemize}
\begin{itemize}
\item {Utilização:Gír.}
\end{itemize}
Macaco.
(Infl. de \textunderscore símio\textunderscore )
\section{Simaruba}
\begin{itemize}
\item {Grp. gram.:f.}
\end{itemize}
\begin{itemize}
\item {Proveniência:(T. guianês)}
\end{itemize}
Gênero de árvores rutáceas, cujas raízes e cascas amargas têm applicação medicinal.
\section{Simarubáceas}
\begin{itemize}
\item {Grp. gram.:f. pl.}
\end{itemize}
Família de plantas, que tem por typo a simaruba.
(Fam. pl. de \textunderscore simarubáceo\textunderscore )
\section{Simarubáceo}
\begin{itemize}
\item {Grp. gram.:adj.}
\end{itemize}
Relativo ou semelhante á simaruba.
\section{Simbaíba}
\begin{itemize}
\item {Grp. gram.:f.}
\end{itemize}
Planta do Maranhão.
\section{Simbereba}
\begin{itemize}
\item {Grp. gram.:f.}
\end{itemize}
(V.sebereba)
\section{Simbor}
\begin{itemize}
\item {Grp. gram.:m.}
\end{itemize}
Planta, mal conhecida, da Índia e de Java.
\section{Símea}
\begin{itemize}
\item {Grp. gram.:f.}
\end{itemize}
Cima?«\textunderscore ...e o cirne pôs ho couce nas símeas da grade...\textunderscore »\textunderscore Cartas\textunderscore  de Affonso de Albuquerque, 120.
\section{Símel}
\begin{itemize}
\item {Grp. gram.:m.}
\end{itemize}
\begin{itemize}
\item {Utilização:Ant.}
\end{itemize}
O mesmo que \textunderscore símile\textunderscore .
\section{Símia}
\begin{itemize}
\item {Grp. gram.:f.}
\end{itemize}
O mesmo que \textunderscore símio\textunderscore . Cf. Herculano, \textunderscore Quest. Públ.\textunderscore , II, 9.
\section{Simiano}
\begin{itemize}
\item {Grp. gram.:adj.}
\end{itemize}
Relativo ou semelhante ao símio.
\section{Simiesco}
\begin{itemize}
\item {fónica:ês}
\end{itemize}
\begin{itemize}
\item {Grp. gram.:adj.}
\end{itemize}
\begin{itemize}
\item {Proveniência:(De \textunderscore símio\textunderscore )}
\end{itemize}
Relativo ou semelhante ao Macaco. Cf. E. do Prado, \textunderscore Ill. Amer.\textunderscore 
\section{Símil}
\begin{itemize}
\item {Grp. gram.:adj.}
\end{itemize}
\begin{itemize}
\item {Utilização:Poét.}
\end{itemize}
\begin{itemize}
\item {Grp. gram.:M.}
\end{itemize}
\begin{itemize}
\item {Proveniência:(Lat. \textunderscore similis\textunderscore )}
\end{itemize}
Semelhante.
O mesmo ou melhór que \textunderscore símile\textunderscore :«\textunderscore isto se explica com o símil do arco...\textunderscore »Bernárdez, \textunderscore Luz e Calor\textunderscore , 2.
\section{Similar}
\begin{itemize}
\item {Grp. gram.:adj.}
\end{itemize}
\begin{itemize}
\item {Grp. gram.:M.}
\end{itemize}
\begin{itemize}
\item {Proveniência:(De \textunderscore símil\textunderscore )}
\end{itemize}
Que é da mesma natureza; homogêneo.
Objecto similar.
\section{Similaridade}
\begin{itemize}
\item {Grp. gram.:f.}
\end{itemize}
Qualidade do que é similar.
\section{Simildão}
\begin{itemize}
\item {Grp. gram.:f.}
\end{itemize}
\begin{itemize}
\item {Utilização:Ant.}
\end{itemize}
O mesmo que \textunderscore similidão\textunderscore .
\section{Símile}
\begin{itemize}
\item {Grp. gram.:m.}
\end{itemize}
\begin{itemize}
\item {Proveniência:(Lat. \textunderscore similis\textunderscore )}
\end{itemize}
Qualidade do que é semelhante.
Analogia.
Comparação de coisas semelhantes.
\section{Similhar}
\textunderscore v. t.\textunderscore  (e der.)
(V. \textunderscore semelhar\textunderscore , etc.)
\section{Similidade}
\begin{itemize}
\item {Grp. gram.:f.}
\end{itemize}
O mesmo que \textunderscore semelhança\textunderscore . Cf. Palmeirim, \textunderscore Portugal e seus Detract.\textunderscore , 28.
\section{Similidão}
\begin{itemize}
\item {Grp. gram.:f.}
\end{itemize}
\begin{itemize}
\item {Utilização:Ant.}
\end{itemize}
\begin{itemize}
\item {Proveniência:(Do lat. \textunderscore similitudo\textunderscore )}
\end{itemize}
O mesmo que \textunderscore semelhança\textunderscore .
\section{Similiflôr}
\begin{itemize}
\item {Grp. gram.:adj.}
\end{itemize}
\begin{itemize}
\item {Proveniência:(Do lat. \textunderscore similis\textunderscore  + \textunderscore flos\textunderscore , \textunderscore floris\textunderscore )}
\end{itemize}
Que tem semelhantes todas as flôres.
\section{Similifloro}
\begin{itemize}
\item {Grp. gram.:adj.}
\end{itemize}
\begin{itemize}
\item {Utilização:Bot.}
\end{itemize}
\begin{itemize}
\item {Proveniência:(Do lat. \textunderscore similis\textunderscore  + \textunderscore flos\textunderscore , \textunderscore floris\textunderscore )}
\end{itemize}
Que tem semelhantes todas as flôres.
\section{Simílimo}
\begin{itemize}
\item {Grp. gram.:adj.}
\end{itemize}
\begin{itemize}
\item {Proveniência:(Do lat. \textunderscore similis\textunderscore )}
\end{itemize}
Muito semelhante. Cf. Filinto, VII, 203.
\section{Similitude}
\begin{itemize}
\item {Grp. gram.:f.}
\end{itemize}
\begin{itemize}
\item {Proveniência:(Lat. \textunderscore similitudo\textunderscore )}
\end{itemize}
O mesmo que \textunderscore semelhança\textunderscore .
\section{Similitudinário}
\begin{itemize}
\item {Grp. gram.:adj.}
\end{itemize}
\begin{itemize}
\item {Proveniência:(Do lat. \textunderscore similitudo\textunderscore , \textunderscore similitudinis\textunderscore )}
\end{itemize}
Em que há semelhança.
\section{Símio}
\begin{itemize}
\item {Grp. gram.:m.}
\end{itemize}
\begin{itemize}
\item {Grp. gram.:Adj.}
\end{itemize}
\begin{itemize}
\item {Proveniência:(Lat. \textunderscore simius\textunderscore )}
\end{itemize}
Macaco, bugio.
Relativo ou semelhante ao macaco.
\section{Simira}
\begin{itemize}
\item {Grp. gram.:f.}
\end{itemize}
Planta rubiácea do Brasil.
\section{Simiri}
\begin{itemize}
\item {Grp. gram.:m.}
\end{itemize}
O mesmo que \textunderscore locusta\textunderscore ^1.
\section{Simonia}
\begin{itemize}
\item {Grp. gram.:f.}
\end{itemize}
\begin{itemize}
\item {Proveniência:(De \textunderscore Simão\textunderscore , n. p.)}
\end{itemize}
Retribuição illicita ou criminosa, que se recebe a trôco de coisas santas ou espirituaes, como os sacramentos, benefícios ecclesiásticos, etc.
\section{Simoníaco}
\begin{itemize}
\item {Grp. gram.:adj.}
\end{itemize}
\begin{itemize}
\item {Grp. gram.:M.}
\end{itemize}
\begin{itemize}
\item {Grp. gram.:Pl.}
\end{itemize}
\begin{itemize}
\item {Proveniência:(Lat. \textunderscore simoniacus\textunderscore )}
\end{itemize}
Relativo a simonia.
Aquelle que commeteu simonia.
O mesmo que \textunderscore simonianos\textunderscore .
\section{Simonianos}
\begin{itemize}
\item {Grp. gram.:m. pl.}
\end{itemize}
\begin{itemize}
\item {Proveniência:(De \textunderscore Simão\textunderscore , n. p.)}
\end{itemize}
Membros de uma seita, fundada no século I por Simão o Mágico, e segundo a qual podia adquirir se por dinheiro a fé, as graças espirituaes, os cargos ecclesiásticos, etc.
\section{Simonte}
\begin{itemize}
\item {Grp. gram.:m.  e  adj.}
\end{itemize}
Diz-se do tabaco da primeira fôlha.
\section{Simplalhão}
\begin{itemize}
\item {Grp. gram.:m.}
\end{itemize}
\begin{itemize}
\item {Utilização:Pop.}
\end{itemize}
Indivíduo simplório, atoleimado.
\section{Simplacheirão}
\begin{itemize}
\item {Grp. gram.:m.  e  adj.}
\end{itemize}
\begin{itemize}
\item {Proveniência:(De \textunderscore simples\textunderscore , com o mesmo suff. de \textunderscore bonacheirão\textunderscore )}
\end{itemize}
O mesmo que \textunderscore simplório\textunderscore .
\section{Simples}
\begin{itemize}
\item {Grp. gram.:adj.}
\end{itemize}
\begin{itemize}
\item {Utilização:Fig.}
\end{itemize}
\begin{itemize}
\item {Grp. gram.:M.  e  f.}
\end{itemize}
\begin{itemize}
\item {Proveniência:(Lat. \textunderscore simplex\textunderscore )}
\end{itemize}
Que não tem composição.
Que não é duplo ou múltiplo.
Que não tem ornatos.
Singelo: \textunderscore uma descripção simples\textunderscore .
Exclusivo.
Em que não há mistura: \textunderscore água simples\textunderscore .
Que se comprehende facilmente.
Evidente.
Natural.
Fácil: \textunderscore meios simples\textunderscore .
Mero: \textunderscore um simples acaso\textunderscore .
Unico.
Vulgar, ordinário.
Ingênuo: \textunderscore uma criatura muito simples\textunderscore .
Papalvo.
Que não gosta de ostentações.
Que traja ou vive modestamente, sem luxo.
Pessôa simples.
\section{Simples}
\begin{itemize}
\item {Grp. gram.:m. pl.}
\end{itemize}
(Alter. errónea de \textunderscore cimbres\textunderscore .)(V.cimbre)
\section{Simplesa}
\begin{itemize}
\item {Grp. gram.:adj.}
\end{itemize}
Flexão fem. ant. de \textunderscore simples\textunderscore ^1. Cf. Fern. Lopes, \textunderscore Chrón. de D. Fern.\textunderscore , I, 7.
\section{Simplesmente}
\begin{itemize}
\item {Grp. gram.:adv.}
\end{itemize}
De modo simples.
Com simplicidade.
Unicamente; meramente.
\section{Símplez}
\begin{itemize}
\item {Grp. gram.:adj.}
\end{itemize}
\begin{itemize}
\item {Utilização:Fig.}
\end{itemize}
\begin{itemize}
\item {Grp. gram.:M.  e  f.}
\end{itemize}
\begin{itemize}
\item {Proveniência:(Lat. \textunderscore simplex\textunderscore )}
\end{itemize}
Que não tem composição.
Que não é duplo ou múltiplo.
Que não tem ornatos.
Singelo: \textunderscore uma descripção símplez\textunderscore .
Exclusivo.
Em que não há mistura: \textunderscore água símplez\textunderscore .
Que se comprehende facilmente.
Evidente.
Natural.
Fácil: \textunderscore meios símplez\textunderscore .
Mero: \textunderscore um símplez acaso\textunderscore .
Unico.
Vulgar, ordinário.
Ingênuo: \textunderscore uma criatura muito símplez\textunderscore .
Papalvo.
Que não gosta de ostentações.
Que traja ou vive modestamente, sem luxo.
Pessôa simples.
\section{Simpleza}
\begin{itemize}
\item {Grp. gram.:f.}
\end{itemize}
O mesmo que \textunderscore simplicidade\textunderscore .
\section{Simplicela}
\begin{itemize}
\item {Grp. gram.:f.}
\end{itemize}
Variedade de esponja do Mar-Branco.
\section{Simplicella}
\begin{itemize}
\item {Grp. gram.:f.}
\end{itemize}
Variedade de esponja do Mar-Branco.
\section{Símplices}
\begin{itemize}
\item {Grp. gram.:m. pl.}
\end{itemize}
\begin{itemize}
\item {Proveniência:(Lat. \textunderscore simplices\textunderscore )}
\end{itemize}
Quaesquer drogas, que entram na composição dos medicamentos.
Ingredientes, que entram nas composições das tintas.
Elementos da composição dos corpos.
\section{Simplicidade}
\begin{itemize}
\item {Grp. gram.:f.}
\end{itemize}
\begin{itemize}
\item {Proveniência:(Lat. \textunderscore simplicitas\textunderscore )}
\end{itemize}
Qualidade do que é simples.
Qualidade daquillo que é fácil de comprehender ou observar.
Carácter originário.
Fórma simples e natural do dizer ou de escrever.
Singeleza.
Ingenuidade, candura.
Sinceridade, franqueza.
Extrema ingenuidade; parvoíce.
\section{Simplicissimamente}
\begin{itemize}
\item {Grp. gram.:adv.}
\end{itemize}
De modo simplicíssimo.
Com a maior simplicidade.
\section{Simplicíssimo}
\begin{itemize}
\item {Grp. gram.:adj.}
\end{itemize}
\begin{itemize}
\item {Proveniência:(Lat. \textunderscore simplicissimus\textunderscore )}
\end{itemize}
Muito simples.
\section{Simplicista}
\begin{itemize}
\item {Grp. gram.:m.  e  adj.}
\end{itemize}
\begin{itemize}
\item {Utilização:Des.}
\end{itemize}
\begin{itemize}
\item {Proveniência:(De \textunderscore símplices\textunderscore )}
\end{itemize}
O que curava doenças por meio de símplices.
O que escrevia á cêrca de símplices.
\section{Simplificação}
\begin{itemize}
\item {Grp. gram.:f.}
\end{itemize}
Acto ou effeito de simplificar.
\section{Simplificacionista}
\begin{itemize}
\item {Grp. gram.:m.}
\end{itemize}
\begin{itemize}
\item {Utilização:Neol.}
\end{itemize}
Partidário da simplificação orthográphica.
\section{Simplificador}
\begin{itemize}
\item {Grp. gram.:m.  e  adj.}
\end{itemize}
O que simplifica.
\section{Simplificar}
\begin{itemize}
\item {Grp. gram.:v. t.}
\end{itemize}
\begin{itemize}
\item {Proveniência:(Do lat. \textunderscore simplex\textunderscore  + \textunderscore facere\textunderscore )}
\end{itemize}
Tornar simples ou mais simples.
Tornar claro ou fácil.
Reduzir a termos mais precisos.
\section{Simplificativo}
\begin{itemize}
\item {Grp. gram.:adj.}
\end{itemize}
Que serve para simplificar.
\section{Simplificável}
\begin{itemize}
\item {Grp. gram.:adj.}
\end{itemize}
Que se póde simplificar.
\section{Simplório}
\begin{itemize}
\item {Grp. gram.:m.  e  adj.}
\end{itemize}
\begin{itemize}
\item {Proveniência:(De \textunderscore simples\textunderscore ^1)}
\end{itemize}
Ingênuo; papalvo.
Indivíduo muito crédulo.
\section{Simplote}
\begin{itemize}
\item {Grp. gram.:m.}
\end{itemize}
\begin{itemize}
\item {Utilização:Des.}
\end{itemize}
Homem simplório. Cf. \textunderscore Anat. Joc.\textunderscore , I, 374.
\section{Simpralhão}
\begin{itemize}
\item {Grp. gram.:m.}
\end{itemize}
O mesmo que \textunderscore simplalhão\textunderscore .
\section{Símprez}
\begin{itemize}
\item {Grp. gram.:adj.}
\end{itemize}
(Fórma ant. de \textunderscore simples\textunderscore ^1)
\section{Simproso}
\begin{itemize}
\item {Grp. gram.:adj.}
\end{itemize}
\begin{itemize}
\item {Utilização:Ant.}
\end{itemize}
\begin{itemize}
\item {Proveniência:(De \textunderscore símprez\textunderscore )}
\end{itemize}
O mesmo que \textunderscore simplório\textunderscore .
\section{Símpulo}
\begin{itemize}
\item {Grp. gram.:m.}
\end{itemize}
\begin{itemize}
\item {Proveniência:(Lat. \textunderscore simpulum\textunderscore )}
\end{itemize}
Pequena taça para libações, que se usava nos sacrifícios romanos. Cf. Castilho, \textunderscore Fastos\textunderscore , I, 474.
\section{Simpúvio}
\begin{itemize}
\item {Grp. gram.:m.}
\end{itemize}
\begin{itemize}
\item {Proveniência:(Lat. \textunderscore simpuvium\textunderscore )}
\end{itemize}
Vaso sagrado, em que os Romanos faziam libações nos sacrifícios.
\section{Sim-senhor}
\begin{itemize}
\item {Grp. gram.:m.}
\end{itemize}
\begin{itemize}
\item {Utilização:Pop.}
\end{itemize}
Nádegas.
\section{Sim-sim}
\begin{itemize}
\item {Grp. gram.:m.}
\end{itemize}
Animal comestível da Guiné Portuguesa.
\section{Simulação}
\begin{itemize}
\item {Grp. gram.:f.}
\end{itemize}
\begin{itemize}
\item {Proveniência:(Lat. \textunderscore simulatio\textunderscore )}
\end{itemize}
Acto ou effeito de simular; fingimento; disfarce.
\section{Simulacro}
\begin{itemize}
\item {Grp. gram.:m.}
\end{itemize}
\begin{itemize}
\item {Proveniência:(Lat. \textunderscore simulacrum\textunderscore )}
\end{itemize}
Effígie.
Imitação.
Semelhança.
Apparência.
Fantasma.
Fingimento.
Cópia ou reproducção imperfeita.
Aquillo com que, imperfeita e ridiculamente, se procura imitar outra coisa.
\section{Simuladamente}
\begin{itemize}
\item {Grp. gram.:adv.}
\end{itemize}
De modo simulado; com fingimento; com disfarce.
\section{Simulado}
\begin{itemize}
\item {Grp. gram.:adj.}
\end{itemize}
\begin{itemize}
\item {Proveniência:(De \textunderscore simular\textunderscore )}
\end{itemize}
Fingido; supposto: \textunderscore casamento simulado\textunderscore .
\section{Simulador}
\begin{itemize}
\item {Grp. gram.:m.  e  adj.}
\end{itemize}
\begin{itemize}
\item {Proveniência:(Do lat. \textunderscore simulator\textunderscore )}
\end{itemize}
O que simula.
\section{Simulamento}
\begin{itemize}
\item {Grp. gram.:m.}
\end{itemize}
O mesmo que \textunderscore simulação\textunderscore .
\section{Simular}
\begin{itemize}
\item {Grp. gram.:v. i.}
\end{itemize}
\begin{itemize}
\item {Proveniência:(Lat. \textunderscore simulare\textunderscore )}
\end{itemize}
Representar com semelhança.
Imitar.
Fingir; apparentar: \textunderscore simular honradez\textunderscore .
Disfarçar.
\section{Simulatório}
\begin{itemize}
\item {Grp. gram.:adj.}
\end{itemize}
\begin{itemize}
\item {Proveniência:(Lat. \textunderscore simulatorius\textunderscore )}
\end{itemize}
Em que há simulação.
\section{Simulcadência}
\begin{itemize}
\item {Grp. gram.:f.}
\end{itemize}
\begin{itemize}
\item {Proveniência:(Do lat. \textunderscore simul\textunderscore  + \textunderscore cadens\textunderscore )}
\end{itemize}
Terminação de phrases ou períodos por meio de palavras iguaes.
\section{Simuldesinência}
\begin{itemize}
\item {Grp. gram.:f.}
\end{itemize}
\begin{itemize}
\item {Proveniência:(De \textunderscore simul\textunderscore  lat. + \textunderscore desinência\textunderscore )}
\end{itemize}
O mesmo que \textunderscore simulcadência\textunderscore .
\section{Sifão}
\begin{itemize}
\item {Grp. gram.:m.}
\end{itemize}
\begin{itemize}
\item {Proveniência:(Lat. \textunderscore sipho\textunderscore )}
\end{itemize}
Tubo recurvado, com ramos desiguaes, que serve geralmente para fazer passar líquidos de um vaso para outro, ou para os extrair de um vaso sem o inclinar.
\section{Sifona}
\begin{itemize}
\item {Grp. gram.:f.}
\end{itemize}
\begin{itemize}
\item {Proveniência:(Do gr. \textunderscore siphon\textunderscore )}
\end{itemize}
Gênero de insectos dípteros.
\section{Sifonacanto}
\begin{itemize}
\item {Grp. gram.:m.}
\end{itemize}
Gênero de plantas acantáceas.
\section{Sifonápteros}
\begin{itemize}
\item {Grp. gram.:m. pl.}
\end{itemize}
\begin{itemize}
\item {Proveniência:(Do gr. \textunderscore siphon\textunderscore  + \textunderscore apteros\textunderscore )}
\end{itemize}
Ordem de insectos ápteros.
\section{Sifonária}
\begin{itemize}
\item {Grp. gram.:f.}
\end{itemize}
\begin{itemize}
\item {Proveniência:(Do gr. \textunderscore siphon\textunderscore )}
\end{itemize}
Gênero de moluscos gasterópodes.
\section{Sifónia}
\begin{itemize}
\item {Grp. gram.:f.}
\end{itemize}
Gênero de plantas euforbiáceas da Guiana, (\textunderscore siphonia elástica\textunderscore , Persoon).
\section{Sifonóide}
\begin{itemize}
\item {Grp. gram.:adj.}
\end{itemize}
\begin{itemize}
\item {Proveniência:(Do gr. \textunderscore siphon\textunderscore  + \textunderscore eidos\textunderscore )}
\end{itemize}
Que tem fórma de sifão.
\section{Simultaneamente}
\begin{itemize}
\item {Grp. gram.:adv.}
\end{itemize}
De modo simultâneo; ao mesmo tempo; conjuntamente.
\section{Simultaneidade}
\begin{itemize}
\item {Grp. gram.:f.}
\end{itemize}
Qualidade do que é simultâneo; coincidência.
\section{Simultâneo}
\begin{itemize}
\item {Grp. gram.:adj.}
\end{itemize}
\begin{itemize}
\item {Proveniência:(Lat. \textunderscore simultaneus\textunderscore )}
\end{itemize}
Que se realiza ou acontece ao mesmo tempo que outra coisa: \textunderscore duas victórias simultâneas\textunderscore .
\section{Simum}
\begin{itemize}
\item {Grp. gram.:m.}
\end{itemize}
Vento muito quente, que sopra do centro da África para o Norte.
(Ár. \textunderscore semom\textunderscore )
\section{Sina}
\begin{itemize}
\item {Grp. gram.:f.}
\end{itemize}
\begin{itemize}
\item {Utilização:Fam.}
\end{itemize}
\begin{itemize}
\item {Proveniência:(Do lat. \textunderscore signa\textunderscore )}
\end{itemize}
O mesmo que \textunderscore signa\textunderscore .
Fado, destino, sorte: \textunderscore a cigana leu a minha sina\textunderscore .
\section{Sinaíta}
\begin{itemize}
\item {Grp. gram.:adj.}
\end{itemize}
\begin{itemize}
\item {Grp. gram.:F.}
\end{itemize}
\begin{itemize}
\item {Utilização:Miner.}
\end{itemize}
Relativo ao Sinai.
Espécie de syenita, que se encontra no monte Sinai.
\section{Sinal}
\begin{itemize}
\item {Grp. gram.:m.}
\end{itemize}
\begin{itemize}
\item {Grp. gram.:Pl.}
\end{itemize}
\begin{itemize}
\item {Utilização:Ant.}
\end{itemize}
\begin{itemize}
\item {Proveniência:(Do lat. \textunderscore signalis\textunderscore )}
\end{itemize}
Coisa, que serve de advertência.
Meio de transmittir, para longe ou para certa distância, mas á vista, ordens, notícias, etc.
Indícios.
Manifestação externa: \textunderscore êsse acto é sinal de prudência\textunderscore .
Gesto.
Marca.
Letreiro.
Mancha na pelle: \textunderscore tem um sinal na face\textunderscore .
Dinheiro ou objecto, que um dos contratantes deixa em poder do outro, para segurança da sua obrigação: \textunderscore ajustou a compra e deixou sinal\textunderscore .
Preságio: \textunderscore aquella nuvem é sinal de trovoada\textunderscore .
Firma de tabellião.
Firma de um signatário: \textunderscore o tabellião reconheceu o meu sinal\textunderscore .
Traço ou traços de sentido convencional.
Feições do corpo humano.
Dobre de sinos, por finados.
Pedacinhos de tafetá, que as mulheres collavam ao rôsto para enfeite.
\section{Sinalar}
\begin{itemize}
\item {Grp. gram.:v. t.}
\end{itemize}
O mesmo que \textunderscore assinalar\textunderscore .
\section{Sinaleiro}
\begin{itemize}
\item {Grp. gram.:m.}
\end{itemize}
Indivíduo, encarregado de dar sinaes, a bordo.
Aquelle que, nas estações do Caminhos de Ferro, dá sinal aos comboios, que chegam, de que a linha está desembaraçada.
\section{Sinalética}
\begin{itemize}
\item {Grp. gram.:f.}
\end{itemize}
\begin{itemize}
\item {Proveniência:(De \textunderscore sinal\textunderscore )}
\end{itemize}
Processo de observar e registar os sinaes ou marcas ou cicatrizes, para a identificação dos criminosos.
\section{Sinalização}
\begin{itemize}
\item {Grp. gram.:f.}
\end{itemize}
Acto de sinalizar.
\section{Sinalizar}
\begin{itemize}
\item {Grp. gram.:v. i.}
\end{itemize}
\begin{itemize}
\item {Utilização:Neol.}
\end{itemize}
\begin{itemize}
\item {Proveniência:(De \textunderscore sinal\textunderscore )}
\end{itemize}
Exercer as funcções de sinaleiro.
\section{Sinalpende}
\begin{itemize}
\item {Grp. gram.:m.}
\end{itemize}
Antiga medida agrária, de 120 pés quadrados.
(Refl. do lat. \textunderscore arpentum\textunderscore ?)
\section{Sinanduba}
\begin{itemize}
\item {Grp. gram.:f.}
\end{itemize}
\begin{itemize}
\item {Utilização:Bras}
\end{itemize}
Planta espinhosa, usada em cêrcas ou cerrados, para que os animaes não lesem as plantações.
\section{Sinapato}
\begin{itemize}
\item {Grp. gram.:m.}
\end{itemize}
\begin{itemize}
\item {Proveniência:(Do lat. \textunderscore sinapi\textunderscore )}
\end{itemize}
Sal, resultante da combinação do ácido sinápico com uma base.
\section{Sinápico}
\begin{itemize}
\item {Grp. gram.:adj.}
\end{itemize}
\begin{itemize}
\item {Proveniência:(Do lat. \textunderscore sinapi\textunderscore )}
\end{itemize}
Relativo á mostarda.
\section{Sinapidendro}
\begin{itemize}
\item {Grp. gram.:m.}
\end{itemize}
\begin{itemize}
\item {Proveniência:(Do lat. \textunderscore sinapi\textunderscore  + gr. \textunderscore dendron\textunderscore )}
\end{itemize}
Gênero de plantas crucíferas.
\section{Sinapina}
\begin{itemize}
\item {Grp. gram.:f.}
\end{itemize}
\begin{itemize}
\item {Utilização:Chím.}
\end{itemize}
\begin{itemize}
\item {Proveniência:(Do lat. \textunderscore sinapi\textunderscore )}
\end{itemize}
Base orgânica, que existe nos grãos da mostarda branca.
\section{Sinapisina}
\begin{itemize}
\item {Grp. gram.:f.}
\end{itemize}
\begin{itemize}
\item {Utilização:Chím.}
\end{itemize}
\begin{itemize}
\item {Proveniência:(Do lat. \textunderscore sinapi\textunderscore )}
\end{itemize}
Substância branca sulfurada, extrahida da mostarda.
\section{Sinapismo}
\begin{itemize}
\item {Grp. gram.:m.}
\end{itemize}
\begin{itemize}
\item {Proveniência:(Lat. \textunderscore sinapismus\textunderscore )}
\end{itemize}
Cataplasma do mostarda, applicada geralmente como revulsivo.
\section{Sinapizar}
\begin{itemize}
\item {Grp. gram.:v. t.}
\end{itemize}
\begin{itemize}
\item {Proveniência:(Lat. \textunderscore sinapizare\textunderscore )}
\end{itemize}
Temperar ou polvilhar com mostarda em pó.
\section{Sinapolina}
\begin{itemize}
\item {Grp. gram.:f.}
\end{itemize}
\begin{itemize}
\item {Utilização:Chím.}
\end{itemize}
\begin{itemize}
\item {Proveniência:(Do lat. \textunderscore sinapi\textunderscore )}
\end{itemize}
Substância crystallizada, obtida pela acção do óxydo de chumbo hydratado sôbre a essência de mostarda.
\section{Sinar}
\begin{itemize}
\item {Grp. gram.:v. t.}
\end{itemize}
\begin{itemize}
\item {Utilização:Ant.}
\end{itemize}
O mesmo que \textunderscore assinar\textunderscore :«\textunderscore ...esta nossa carta por nós sinada\textunderscore ». \textunderscore Chancell. de Aff. V\textunderscore , l. X, f. 15.
\section{Sinarada}
\begin{itemize}
\item {Grp. gram.:f.}
\end{itemize}
Toque de sinos. Cf. S. Dias, \textunderscore Peninsulares\textunderscore , 385.
\section{Sinceiral}
\begin{itemize}
\item {Grp. gram.:m.}
\end{itemize}
\begin{itemize}
\item {Proveniência:(De \textunderscore sinceiro\textunderscore )}
\end{itemize}
O mesmo que \textunderscore salgueiral\textunderscore .
\section{Sinceiro}
\begin{itemize}
\item {Grp. gram.:m.}
\end{itemize}
O mesmo que \textunderscore salgueiro\textunderscore . Cf. Herculano, \textunderscore Harpa do Crente\textunderscore .
\section{Sincelada}
\begin{itemize}
\item {Grp. gram.:f.}
\end{itemize}
\begin{itemize}
\item {Utilização:Prov.}
\end{itemize}
\begin{itemize}
\item {Utilização:trasm.}
\end{itemize}
Porção de sincelo.
\section{Sincelar}
\begin{itemize}
\item {Grp. gram.:v. i.}
\end{itemize}
\begin{itemize}
\item {Utilização:Prov.}
\end{itemize}
\begin{itemize}
\item {Utilização:trasm.}
\end{itemize}
Formar sincelo.
\section{Sincelo}
\begin{itemize}
\item {Grp. gram.:m.}
\end{itemize}
Pedaços de caramelo, suspensos das árvores ou dos beiraes dos telhados e resultantes da congelação da chuva ou do orvalho.
\section{Sinceloso}
\begin{itemize}
\item {Grp. gram.:adj.}
\end{itemize}
Coberto ou cheio de sincelo: \textunderscore um castanheiro sinceloso\textunderscore .
\section{Sincenada}
\begin{itemize}
\item {Grp. gram.:f.}
\end{itemize}
\begin{itemize}
\item {Utilização:Prov.}
\end{itemize}
\begin{itemize}
\item {Utilização:trasm.}
\end{itemize}
\begin{itemize}
\item {Proveniência:(De \textunderscore sinceno\textunderscore )}
\end{itemize}
O mesmo que \textunderscore sincelada\textunderscore .
\section{Sincenho}
\begin{itemize}
\item {Grp. gram.:m.}
\end{itemize}
\begin{itemize}
\item {Utilização:T. de Miranda}
\end{itemize}
O mesmo que \textunderscore sincelo\textunderscore ^1.
Nevoeiro pouco espêsso.
\section{Sinceno}
\begin{itemize}
\item {Grp. gram.:m.}
\end{itemize}
\begin{itemize}
\item {Utilização:Prov.}
\end{itemize}
\begin{itemize}
\item {Utilização:trasm.}
\end{itemize}
(V. \textunderscore sincelo\textunderscore ^1)
\section{Sinceramente}
\begin{itemize}
\item {Grp. gram.:adv.}
\end{itemize}
De modo sincero; francamente, com lisura.
\section{Sinceridade}
\begin{itemize}
\item {Grp. gram.:f.}
\end{itemize}
\begin{itemize}
\item {Proveniência:(Lat. \textunderscore sinceritas\textunderscore )}
\end{itemize}
Qualidade do que é sincero.
Franqueza; lisura de carácter.
\section{Sincero}
\begin{itemize}
\item {Grp. gram.:adj.}
\end{itemize}
\begin{itemize}
\item {Proveniência:(Lat. \textunderscore sincerus\textunderscore )}
\end{itemize}
Que diz com franqueza o que sente.
Verdadeiro.
Simples, franco.
Em que não há disfarce ou dissimulação; em que não há malícia: \textunderscore palavras sinceras\textunderscore .
\section{Sincipital}
\begin{itemize}
\item {Grp. gram.:adj.}
\end{itemize}
Relativo ao sincipúcio.
\section{Sincipúcio}
\begin{itemize}
\item {Grp. gram.:m.}
\end{itemize}
\begin{itemize}
\item {Utilização:Anat.}
\end{itemize}
\begin{itemize}
\item {Proveniência:(Do lat. \textunderscore sinciput\textunderscore )}
\end{itemize}
A parte superior da cabeça.
\section{Sincláiria}
\begin{itemize}
\item {Grp. gram.:f.}
\end{itemize}
\begin{itemize}
\item {Proveniência:(De \textunderscore Sinclair\textunderscore , n. p.)}
\end{itemize}
Gênero de plantas, da fam. das compostas.
\section{Sindi}
\begin{itemize}
\item {Grp. gram.:m.}
\end{itemize}
Idioma, falado ao norte do Penjab, em Sindi, na Índia.
\section{Sindiba}
\begin{itemize}
\item {Grp. gram.:f.}
\end{itemize}
\begin{itemize}
\item {Utilização:Bras}
\end{itemize}
O mesmo que \textunderscore milho-cozido\textunderscore , árvore.
\section{Sindo}
\begin{itemize}
\item {Grp. gram.:m.}
\end{itemize}
\begin{itemize}
\item {Utilização:Des.}
\end{itemize}
Nome, que se dá ao mandarim em o norte da Índia.
(Cp. \textunderscore sindi\textunderscore )
\section{Sinecura}
\begin{itemize}
\item {Grp. gram.:f.}
\end{itemize}
\begin{itemize}
\item {Proveniência:(Do lat. \textunderscore sine\textunderscore  + \textunderscore cura\textunderscore )}
\end{itemize}
Emprêgo rendoso, que não obriga a trabalho.
\section{Sinecurismo}
\begin{itemize}
\item {Grp. gram.:m.}
\end{itemize}
Systema governamental, que se apoia nas sinecuras que promove.
\section{Sinecurista}
\begin{itemize}
\item {Grp. gram.:m.  e  f.}
\end{itemize}
Pessôa, que tem sinecura ou sinecuras, ou que gosta dellas.
\section{Sineira}
\begin{itemize}
\item {Grp. gram.:f.}
\end{itemize}
\begin{itemize}
\item {Utilização:Pesc.}
\end{itemize}
Mulhér, que toca sinos ou sinetas.
Abertura na parte superior da torre, onde está o sino.
Mulhér do sineiro.
Cada uma das pequenas bóias de cortiça, ligadas ao chicote de um cabo delgado, que se amarra, de espaço a espaço, na tralha superior dos tresmalhos.
\section{Sineiro}
\begin{itemize}
\item {Grp. gram.:m.}
\end{itemize}
\begin{itemize}
\item {Grp. gram.:Adj.}
\end{itemize}
Fabricante de sinos.
Indivíduo que, por offício ou por obrigação, toca os sinos.
Ave americana, espécie do papa-formigas, cuja voz imita o rebate dos sinos, (\textunderscore turdus tinniens\textunderscore , Gmel.).
Que tem sino, (falando-se de tôrres, etc.).
\section{Sineta}
\begin{itemize}
\item {fónica:nê}
\end{itemize}
\begin{itemize}
\item {Grp. gram.:f.}
\end{itemize}
Pequeno sino.
\section{Sinete}
\begin{itemize}
\item {fónica:nê}
\end{itemize}
\begin{itemize}
\item {Grp. gram.:m.}
\end{itemize}
\begin{itemize}
\item {Proveniência:(Do fr. \textunderscore signet\textunderscore )}
\end{itemize}
Utensílio, com assinatura ou divisa gravada, e que serve para imprimir em papel, lacre, etc.
Carimbo; chancella.
\section{Singalês}
\begin{itemize}
\item {Grp. gram.:m.  e  adj.}
\end{itemize}
(V.cingalês)
\section{Singela}
\begin{itemize}
\item {Grp. gram.:f.}
\end{itemize}
Fila longitudinal de pequenos compartimentos, de cada lado do corredor, nas marinhas do Sado.
(Fem. de \textunderscore singelo\textunderscore )
\section{Singelamente}
\begin{itemize}
\item {Grp. gram.:adv.}
\end{itemize}
Do modo singelo; simplesmente; ingenuamente.
\section{Singeleira}
\begin{itemize}
\item {Grp. gram.:f.}
\end{itemize}
\begin{itemize}
\item {Proveniência:(De \textunderscore singelo\textunderscore )}
\end{itemize}
Espécie de rêde, para pesca de peixe miúdo.
\section{Singelez}
\begin{itemize}
\item {Grp. gram.:f.}
\end{itemize}
Qualidade do que é singelo; ingenuidade; simplicidade.
\section{Singeleza}
\begin{itemize}
\item {Grp. gram.:f.}
\end{itemize}
Qualidade do que é singelo; ingenuidade; simplicidade.
\section{Singelo}
\begin{itemize}
\item {Grp. gram.:m.}
\end{itemize}
\begin{itemize}
\item {Proveniência:(Lat. hyp. \textunderscore singellus\textunderscore , de \textunderscore singulus\textunderscore )}
\end{itemize}
Simples.
Sincero.
Innocente; inoffensivo.
\section{Singradura}
\begin{itemize}
\item {Grp. gram.:f.}
\end{itemize}
Acto ou effeito de singrar.
Rumo, por onde se singra.
Espaço, que se percorreu num dia, singrando.
\section{Singrante}
\begin{itemize}
\item {Grp. gram.:adj.}
\end{itemize}
\begin{itemize}
\item {Proveniência:(De \textunderscore singrar\textunderscore )}
\end{itemize}
Preparado para singrar, (falando-se do navio).
\section{Singrar}
\begin{itemize}
\item {Grp. gram.:v. i.}
\end{itemize}
\begin{itemize}
\item {Proveniência:(Do ant. alt. al. \textunderscore segelen\textunderscore )}
\end{itemize}
Navegar á vela, velejar.
\section{Singular}
\begin{itemize}
\item {Grp. gram.:adj.}
\end{itemize}
\begin{itemize}
\item {Utilização:Gram.}
\end{itemize}
\begin{itemize}
\item {Grp. gram.:M.}
\end{itemize}
\begin{itemize}
\item {Utilização:Gram.}
\end{itemize}
\begin{itemize}
\item {Proveniência:(Lat. \textunderscore singularis\textunderscore )}
\end{itemize}
Individual.
Que pertence a um só.
Único.
Extraordinário.
Distinto.
Privativo, especial.
Excêntrico.
Diz-se do número, que, nos nomes e nos verbos, se refere a uma só pessôa.
O número singular dos nomes e dos verbos: \textunderscore «deve»é terceira pessôa do singular do tempo presente do indicativo do verbo«dever»\textunderscore .
\section{Singularidade}
\begin{itemize}
\item {Grp. gram.:f.}
\end{itemize}
\begin{itemize}
\item {Proveniência:(Lat. \textunderscore singularitas\textunderscore )}
\end{itemize}
Qualidade do que é singular.
Acto ou dito singular.
\section{Singularizar}
\begin{itemize}
\item {Grp. gram.:v. t.}
\end{itemize}
Tornar singular; privilegiar.
Fazer excepção de.
Especificar.
\section{Singularmente}
\begin{itemize}
\item {Grp. gram.:adv.}
\end{itemize}
De modo singular; em particular; com especialidade.
Extraordináriamente; fóra do vulgar.
\section{Singulto}
\begin{itemize}
\item {Grp. gram.:m.}
\end{itemize}
\begin{itemize}
\item {Utilização:Poét.}
\end{itemize}
\begin{itemize}
\item {Proveniência:(Lat. \textunderscore singultus\textunderscore )}
\end{itemize}
O mesmo que \textunderscore soluço\textunderscore . Cf. Galhegos, \textunderscore Templo da Memória\textunderscore , II, 107.
\section{Singultoso}
\begin{itemize}
\item {Grp. gram.:adj.}
\end{itemize}
Que tem singultos.
\section{Sinhá}
\begin{itemize}
\item {Grp. gram.:f.}
\end{itemize}
\begin{itemize}
\item {Utilização:Bras}
\end{itemize}
\begin{itemize}
\item {Utilização:pop.}
\end{itemize}
O mesmo que \textunderscore senhora\textunderscore .
\section{Sinhama}
\begin{itemize}
\item {Grp. gram.:f.}
\end{itemize}
\begin{itemize}
\item {Utilização:Gír.}
\end{itemize}
Senhora.
(Cp. \textunderscore sinhá\textunderscore )
\section{Sinhaninha}
\begin{itemize}
\item {Grp. gram.:f.}
\end{itemize}
\begin{itemize}
\item {Utilização:Bras}
\end{itemize}
Espiguilha, em fórma de ziguezague.
\section{Sinhara}
\begin{itemize}
\item {Grp. gram.:f.}
\end{itemize}
\begin{itemize}
\item {Utilização:Bras}
\end{itemize}
\begin{itemize}
\item {Utilização:pop.}
\end{itemize}
O mesmo que \textunderscore sinhá\textunderscore .
\section{Sinhàzinha}
\begin{itemize}
\item {Grp. gram.:f.}
\end{itemize}
(Dem. de \textunderscore sinhá\textunderscore )
\section{Sinhô}
\begin{itemize}
\item {Grp. gram.:m.}
\end{itemize}
\begin{itemize}
\item {Utilização:Bras}
\end{itemize}
\begin{itemize}
\item {Utilização:pop.}
\end{itemize}
Senhor.
\section{Sinhôzinho}
\begin{itemize}
\item {Grp. gram.:m.}
\end{itemize}
(Dem. de \textunderscore sinhô\textunderscore )
\section{Sínico}
\begin{itemize}
\item {Grp. gram.:adj.}
\end{itemize}
\begin{itemize}
\item {Proveniência:(Do lat. mod. \textunderscore Sina\textunderscore , China)}
\end{itemize}
Relativo á China.
Concernente aos Chineses, em território portuguez: \textunderscore o Seabra foi agora nomeado procurador dos negócios sínicos em Macau\textunderscore .
\section{Sinimbu}
\begin{itemize}
\item {Grp. gram.:m.}
\end{itemize}
\begin{itemize}
\item {Utilização:Bras}
\end{itemize}
Espécie de reptil verde, cuja carne é comestível.
\section{Sinistramente}
\begin{itemize}
\item {Grp. gram.:adv.}
\end{itemize}
De modo sinistro; desastradamente; pavorosamente.
\section{Sinistrar}
\begin{itemize}
\item {Grp. gram.:v. i.}
\end{itemize}
Soffrer sinistro (um objecto de contrato de seguro).
\section{Sinistrizar}
\begin{itemize}
\item {Grp. gram.:v. t.}
\end{itemize}
\begin{itemize}
\item {Utilização:Neol.}
\end{itemize}
Tornar sinistro.
\section{Sinistro}
\begin{itemize}
\item {Grp. gram.:adj.}
\end{itemize}
\begin{itemize}
\item {Grp. gram.:M.}
\end{itemize}
\begin{itemize}
\item {Proveniência:(Do lat. \textunderscore sinister\textunderscore )}
\end{itemize}
Esquerdo.
Que faz temer desgraças.
Que é de mau preságio; funesto.
Que indíca má índole.
Mau.
Desastre; grande prejuízo material; ruína.
\section{Sino}
\begin{itemize}
\item {Grp. gram.:m.}
\end{itemize}
\begin{itemize}
\item {Utilização:Gír.}
\end{itemize}
\begin{itemize}
\item {Proveniência:(Do lat. \textunderscore signum\textunderscore )}
\end{itemize}
Instrumento, geralmente de bronze e de fórma cónica, suspenso em eixos lateraes, e que produz sons mais ou menos fortes, pela percussão de uma peça interior chamada badalo ou de um martelo exterior.
Apparelho, em fórma de pyrâmide truncada, para serviço de mergulhadores.
Copo de vinho.
\section{Sino}
\begin{itemize}
\item {Grp. gram.:m.}
\end{itemize}
\begin{itemize}
\item {Utilização:Ant.}
\end{itemize}
\begin{itemize}
\item {Proveniência:(Lat. \textunderscore sinus\textunderscore )}
\end{itemize}
O mesmo que \textunderscore golfo\textunderscore . Cf. Esmeraldo, \textunderscore De situ Orbis\textunderscore , 90.
\section{Sino}
\begin{itemize}
\item {Grp. gram.:m.}
\end{itemize}
\begin{itemize}
\item {Utilização:Ant.}
\end{itemize}
O mesmo que \textunderscore signo\textunderscore .
\section{Sinoble}
\begin{itemize}
\item {Grp. gram.:m.}
\end{itemize}
O mesmo que \textunderscore sinople\textunderscore .
\section{Sino-grande}
\begin{itemize}
\item {Grp. gram.:m.}
\end{itemize}
\begin{itemize}
\item {Utilização:Gír.}
\end{itemize}
\begin{itemize}
\item {Utilização:Gír.}
\end{itemize}
Copo grande. Cf. G. Braga, \textunderscore Mal da Delf.\textunderscore , 159.
Pena, immediata á maiór, do \textunderscore Código Penal\textunderscore .
\section{Sinologia}
\begin{itemize}
\item {Grp. gram.:f.}
\end{itemize}
\begin{itemize}
\item {Proveniência:(De \textunderscore sinólogo\textunderscore )}
\end{itemize}
Estudo do que diz respeito á China.
\section{Sinológico}
\begin{itemize}
\item {Grp. gram.:adj.}
\end{itemize}
Relativo a sinologia.
\section{Sinólogo}
\begin{itemize}
\item {Grp. gram.:m.  e  adj.}
\end{itemize}
\begin{itemize}
\item {Proveniência:(Do lat. mod. \textunderscore Sina\textunderscore , n. p. + gr. \textunderscore logos\textunderscore )}
\end{itemize}
O que é perito em sinologia.
\section{Sinopla}
\begin{itemize}
\item {Grp. gram.:f.}
\end{itemize}
O mesmo que \textunderscore sinople\textunderscore .
\section{Sinople}
\begin{itemize}
\item {Grp. gram.:f.}
\end{itemize}
\begin{itemize}
\item {Utilização:Heráld.}
\end{itemize}
\begin{itemize}
\item {Proveniência:(Fr. \textunderscore sinople\textunderscore )}
\end{itemize}
Côr negra nos escudos.
Variedade de quartzo.
\section{Sino-saimão}
\begin{itemize}
\item {Grp. gram.:m.}
\end{itemize}
\begin{itemize}
\item {Proveniência:(Do lat. \textunderscore signum\textunderscore  + \textunderscore Salomon\textunderscore , n. p.)}
\end{itemize}
Espécie de amuleto ou talisman, constituído por dois triângulos de metal, entrelaçados em fórma de estrêlla.
\section{Sino-samão}
\begin{itemize}
\item {Grp. gram.:m.}
\end{itemize}
\begin{itemize}
\item {Utilização:Ant.}
\end{itemize}
O mesmo que \textunderscore sino-saimão\textunderscore . Cf. Filinto, VIII, 166.
\section{Sinto}
\begin{itemize}
\item {Grp. gram.:m.}
\end{itemize}
O mesmo que \textunderscore sintoísmo\textunderscore .
\section{Sintó}
\begin{itemize}
\item {Grp. gram.:m.}
\end{itemize}
\begin{itemize}
\item {Proveniência:(Do jap. \textunderscore sintau\textunderscore )}
\end{itemize}
Religião principal do Japão, anterior ao budismo.
\section{Sintoísmo}
\begin{itemize}
\item {Grp. gram.:m.}
\end{itemize}
\begin{itemize}
\item {Proveniência:(Do jap. \textunderscore sintau\textunderscore )}
\end{itemize}
Religião principal do Japão, anterior ao budismo.
\section{Sintro}
\begin{itemize}
\item {Grp. gram.:m.}
\end{itemize}
\begin{itemize}
\item {Utilização:Bot.}
\end{itemize}
O mesmo que \textunderscore absintho\textunderscore . Cf. P. Coutinho, \textunderscore Flora\textunderscore , 635.
\section{Sinuado}
\begin{itemize}
\item {Grp. gram.:adj.}
\end{itemize}
\begin{itemize}
\item {Utilização:Bot.}
\end{itemize}
\begin{itemize}
\item {Proveniência:(Do lat. \textunderscore sinuatus\textunderscore )}
\end{itemize}
Que tem lóbulos salientes e arredondados, (falando-se de órgãos vegetaes).
\section{Sinuar}
\begin{itemize}
\item {Grp. gram.:v. t.}
\end{itemize}
\begin{itemize}
\item {Utilização:Fig.}
\end{itemize}
\begin{itemize}
\item {Proveniência:(Lat. \textunderscore sinuare\textunderscore )}
\end{itemize}
Dar relêvo a, por meio de rodeios:«\textunderscore o carácter lýrico das minhas canções ovidianas... obrigava-me até a... diffundi-las e sinuá-las, para deixar apparecer os bordados\textunderscore ». Castilho, \textunderscore Arte de Amar\textunderscore , 33.
\section{Sinuêlo}
\begin{itemize}
\item {Grp. gram.:m.}
\end{itemize}
\begin{itemize}
\item {Utilização:Bras. do S}
\end{itemize}
\begin{itemize}
\item {Proveniência:(Do cast. \textunderscore señuelo\textunderscore )}
\end{itemize}
Gado manso, que se junta ao bravo, para lhe servir de guia.
Cabresto.
\section{Sinuosidade}
\begin{itemize}
\item {Grp. gram.:f.}
\end{itemize}
Qualidade ou estado do que é sinuoso.
Tergiversação.
\section{Sinuoso}
\begin{itemize}
\item {Grp. gram.:adj.}
\end{itemize}
\begin{itemize}
\item {Proveniência:(Lat. \textunderscore sinuosus\textunderscore )}
\end{itemize}
Ondulado ou recurvado em varias direcções.
Tortuoso.
Que descreve ou segue uma linha mais ou menos irregular: \textunderscore uma corrente sinuosa\textunderscore .
\section{Sinusite}
\begin{itemize}
\item {Grp. gram.:f.}
\end{itemize}
\begin{itemize}
\item {Utilização:Med.}
\end{itemize}
\begin{itemize}
\item {Proveniência:(Do lat. \textunderscore sinus\textunderscore )}
\end{itemize}
Inflammação das cavidades do rosto.
\section{Sinusoidal}
\begin{itemize}
\item {Grp. gram.:adj.}
\end{itemize}
Relativo á sinusóide.
\section{Sinusóide}
\begin{itemize}
\item {Grp. gram.:f.}
\end{itemize}
\begin{itemize}
\item {Utilização:Mathem.}
\end{itemize}
\begin{itemize}
\item {Proveniência:(Do lat. \textunderscore sinus\textunderscore  + gr. \textunderscore eidos\textunderscore )}
\end{itemize}
Curva, em que a ordenada representa o seno geométrico do arco, tomado sôbre um círculo, cujo raio é igual á abscissa.
\section{Sinzal}
\begin{itemize}
\item {Grp. gram.:f.}
\end{itemize}
Espécie de uva minhota.
(Corr. de \textunderscore cinzal\textunderscore , de \textunderscore cinza\textunderscore ?)
\section{Sio}
\begin{itemize}
\item {Grp. gram.:m.}
\end{itemize}
Gênero de plantas umbellíferas.
\section{Sioba}
\begin{itemize}
\item {Grp. gram.:f.}
\end{itemize}
\begin{itemize}
\item {Utilização:Bras}
\end{itemize}
Nome de um peixe.
\section{Siocho}
\begin{itemize}
\item {fónica:ô}
\end{itemize}
\begin{itemize}
\item {Grp. gram.:m.}
\end{itemize}
Pássaro conirostro, o mesmo que \textunderscore cicia\textunderscore .
\section{Siões}
\begin{itemize}
\item {Grp. gram.:m. pl.}
\end{itemize}
\begin{itemize}
\item {Utilização:Des.}
\end{itemize}
Os habitantes do reino de Siám; Siameses. Cf. \textunderscore Peregrinação\textunderscore , CXLIX.
\section{Sionismo}
\begin{itemize}
\item {Grp. gram.:m.}
\end{itemize}
\begin{itemize}
\item {Proveniência:(De \textunderscore Sião\textunderscore , = \textunderscore Síon\textunderscore , n. p.)}
\end{itemize}
Estudo de coisas relativas a Jerusalém.
\section{Siote}
\begin{itemize}
\item {Grp. gram.:m.}
\end{itemize}
Pequena ave canora de Nova-Granada.
\section{Sipahuba}
\begin{itemize}
\item {Grp. gram.:f.}
\end{itemize}
Arbusto brasileiro, (\textunderscore combretum ascendens\textunderscore ).
\section{Sipais}
\begin{itemize}
\item {Grp. gram.:m. pl.}
\end{itemize}
\begin{itemize}
\item {Proveniência:(Do pers. \textunderscore sipahi\textunderscore )}
\end{itemize}
Outra fórma de \textunderscore cipaios\textunderscore .
Soldados indígenas da Índia, ao serviço de Ingleses.
\section{Sipário}
\begin{itemize}
\item {Grp. gram.:m.}
\end{itemize}
\begin{itemize}
\item {Proveniência:(Lat. \textunderscore siparium\textunderscore )}
\end{itemize}
Pano theatral, entre os Romanos, que servia para decorar o fundo da scena, nas representações cómicas.
\section{Siparuna}
\begin{itemize}
\item {Grp. gram.:f.}
\end{itemize}
Arbusto rutáceo do Brasil.
\section{Sipatão}
\begin{itemize}
\item {Grp. gram.:m.}
\end{itemize}
Designação antiga de cada um de certos funccionários da China. Cf. \textunderscore Peregrinação\textunderscore , CXIV.
\section{Sipaúba}
\begin{itemize}
\item {Grp. gram.:f.}
\end{itemize}
Arbusto brasileiro, (\textunderscore combretum ascendens\textunderscore ).
\section{Siphão}
\begin{itemize}
\item {Grp. gram.:m.}
\end{itemize}
\begin{itemize}
\item {Proveniência:(Lat. \textunderscore sipho\textunderscore )}
\end{itemize}
Tubo recurvado, com ramos desiguaes, que serve geralmente para fazer passar líquidos de um vaso para outro, ou para os extrahir de um vaso sem o inclinar.
\section{Siphona}
\begin{itemize}
\item {Grp. gram.:f.}
\end{itemize}
\begin{itemize}
\item {Proveniência:(Do gr. \textunderscore siphon\textunderscore )}
\end{itemize}
Gênero de insectos dípteros.
\section{Siphonacantho}
\begin{itemize}
\item {Grp. gram.:m.}
\end{itemize}
Gênero de plantas acantháceas.
\section{Siphonápteros}
\begin{itemize}
\item {Grp. gram.:m. pl.}
\end{itemize}
\begin{itemize}
\item {Proveniência:(Do gr. \textunderscore siphon\textunderscore  + \textunderscore apteros\textunderscore )}
\end{itemize}
Ordem de insectos ápteros.
\section{Siphonária}
\begin{itemize}
\item {Grp. gram.:f.}
\end{itemize}
Gênero de molluscos gasterópodes.
\section{Siphónia}
\begin{itemize}
\item {Grp. gram.:f.}
\end{itemize}
Gênero de plantas euphorbiáceas da Guiana, (\textunderscore siphonia elástica\textunderscore , Persoon).
\section{Siphonóide}
\begin{itemize}
\item {Grp. gram.:adj.}
\end{itemize}
\begin{itemize}
\item {Proveniência:(Do gr. \textunderscore siphon\textunderscore  + \textunderscore eidos\textunderscore )}
\end{itemize}
Que tem fórma de siphão.
\section{Sifonoma}
\begin{itemize}
\item {Grp. gram.:m.}
\end{itemize}
\begin{itemize}
\item {Utilização:Med.}
\end{itemize}
\begin{itemize}
\item {Proveniência:(Do gr. \textunderscore siphon\textunderscore , tubo)}
\end{itemize}
Nome, que se deu a um tumor, de aspecto fibroso, mas mole, encontrado no mesentério de um rapaz.
\section{Sifonóforos}
\begin{itemize}
\item {Grp. gram.:m. pl.}
\end{itemize}
\begin{itemize}
\item {Proveniência:(Do gr. \textunderscore siphon\textunderscore  + \textunderscore phoros\textunderscore )}
\end{itemize}
Gênero de protozoários.
\section{Sifonostégia}
\begin{itemize}
\item {Grp. gram.:f.}
\end{itemize}
\begin{itemize}
\item {Proveniência:(Do gr. \textunderscore siphon\textunderscore  + \textunderscore stege\textunderscore )}
\end{itemize}
Gênero de plantas escrofularíneas.
\section{Sifonóstomo}
\begin{itemize}
\item {Grp. gram.:adj.}
\end{itemize}
\begin{itemize}
\item {Utilização:Zool.}
\end{itemize}
\begin{itemize}
\item {Grp. gram.:Pl.}
\end{itemize}
\begin{itemize}
\item {Proveniência:(Do gr. \textunderscore siphon\textunderscore  + \textunderscore stoma\textunderscore )}
\end{itemize}
Que tem a bôca prolongada em fórma de sifão.
Família do peixes acantòpterígios.
\section{Sífula}
\begin{itemize}
\item {Grp. gram.:f.}
\end{itemize}
Gênero do líchens.
\section{Sifunculídeos}
\begin{itemize}
\item {Grp. gram.:m. pl.}
\end{itemize}
\begin{itemize}
\item {Proveniência:(Do lat. \textunderscore siphunculus\textunderscore )}
\end{itemize}
Grupo de animaes em fórma de tubo coriáceo, e que alguns classificam como equinodermes e outros como anelídeos.
\section{Siphonoma}
\begin{itemize}
\item {Grp. gram.:m.}
\end{itemize}
\begin{itemize}
\item {Utilização:Med.}
\end{itemize}
\begin{itemize}
\item {Proveniência:(Do gr. \textunderscore siphon\textunderscore , tubo)}
\end{itemize}
Nome, que se deu a um tumor, de aspecto fibroso, mas molle, encontrado no mesentério de um rapaz.
\section{Siphonóphoros}
\begin{itemize}
\item {Grp. gram.:m. pl.}
\end{itemize}
\begin{itemize}
\item {Proveniência:(Do gr. \textunderscore siphon\textunderscore  + \textunderscore phoros\textunderscore )}
\end{itemize}
Gênero de protozoários.
\section{Siphonostégia}
\begin{itemize}
\item {Grp. gram.:f.}
\end{itemize}
\begin{itemize}
\item {Proveniência:(Do gr. \textunderscore siphon\textunderscore  + \textunderscore stege\textunderscore )}
\end{itemize}
Gênero de plantas escrofularíneas.
\section{Siphonóstomo}
\begin{itemize}
\item {Grp. gram.:adj.}
\end{itemize}
\begin{itemize}
\item {Utilização:Zool.}
\end{itemize}
\begin{itemize}
\item {Grp. gram.:Pl.}
\end{itemize}
\begin{itemize}
\item {Proveniência:(Do gr. \textunderscore siphon\textunderscore  + \textunderscore stoma\textunderscore )}
\end{itemize}
Que tem a bôca prolongada em fórma de siphão.
Família do peixes acanthòpterýgios.
\section{Síphula}
\begin{itemize}
\item {Grp. gram.:f.}
\end{itemize}
Gênero do líchens.
\section{Siphunculídeos}
\begin{itemize}
\item {Grp. gram.:m. pl.}
\end{itemize}
\begin{itemize}
\item {Proveniência:(Do lat. \textunderscore siphunculus\textunderscore )}
\end{itemize}
Grupo de animaes em fórma de tubo coriáceo, e que alguns classificam como echinodermes e outros como anelídeos.
\section{Sipipira}
\begin{itemize}
\item {Grp. gram.:f.}
\end{itemize}
O mesmo que \textunderscore sicupira\textunderscore .
\section{Siquaes}
\begin{itemize}
\item {Grp. gram.:conj.}
\end{itemize}
\begin{itemize}
\item {Utilização:Ant.}
\end{itemize}
Se acaso:«\textunderscore cerrai a porta sobre vós com a vossa candeiazinha: e siquaes sereis vós minha, entonces veremos nós.\textunderscore »G. Vicente, \textunderscore Inês Pereira\textunderscore .
\section{Siquer}
\begin{itemize}
\item {Grp. gram.:adv.}
\end{itemize}
(V.sequer)
\section{Sira}
\begin{itemize}
\item {Grp. gram.:f.}
\end{itemize}
Provavelmente, o mesmo que \textunderscore síria\textunderscore :«\textunderscore de mi parte não sei nem tenho ponta de sira.\textunderscore »G. Vicente, I, 266.
\section{Sirage}
\begin{itemize}
\item {Grp. gram.:m.}
\end{itemize}
Óleo de gergelim.
(Talvez do ár.)
\section{Sire}
\begin{itemize}
\item {Grp. gram.:m.}
\end{itemize}
\begin{itemize}
\item {Proveniência:(Fr. \textunderscore sire\textunderscore )}
\end{itemize}
Tratamento, que se dava aos Reis de França, aos senhores feudaes e a outras personagens.
\section{Sirena}
\begin{itemize}
\item {Grp. gram.:f.}
\end{itemize}
\begin{itemize}
\item {Utilização:Bras}
\end{itemize}
\begin{itemize}
\item {Proveniência:(Lat. \textunderscore sirena\textunderscore )}
\end{itemize}
O mesmo que \textunderscore sereia\textunderscore . Cf. Camões, \textunderscore soneto 120\textunderscore .
O mesmo que \textunderscore sereia\textunderscore  das embarcações. Cf. \textunderscore Jorn.-do-Comm.\textunderscore , de 2-VI-900.
\section{Sirênico}
\begin{itemize}
\item {Grp. gram.:adj.}
\end{itemize}
\begin{itemize}
\item {Proveniência:(Do lat. \textunderscore sirena\textunderscore )}
\end{itemize}
Relativo ás sereias.
\section{Sirga}
\begin{itemize}
\item {Grp. gram.:f.}
\end{itemize}
\begin{itemize}
\item {Proveniência:(De \textunderscore sirgo\textunderscore )}
\end{itemize}
Corda, com que se puxa uma embarcação ao longo da margem.
Acto ou effeito de sirgar.
\section{Sirgagem}
\begin{itemize}
\item {Grp. gram.:f.}
\end{itemize}
Acto de sirgar.
\section{Sirgar}
\begin{itemize}
\item {Grp. gram.:v. t.}
\end{itemize}
\begin{itemize}
\item {Proveniência:(Do lat. hyp. \textunderscore siricare\textunderscore )}
\end{itemize}
Puxar ou conduzir (um barco), por meio de sirga.
\section{Sirgaria}
\begin{itemize}
\item {Grp. gram.:f.}
\end{itemize}
Grande porção de sirgas.
Fabrica de sirgas.
Estabelecimento onde se vendem sirgas.
\section{Sirgaria}
\begin{itemize}
\item {Grp. gram.:f.}
\end{itemize}
\begin{itemize}
\item {Proveniência:(De \textunderscore sirgo\textunderscore )}
\end{itemize}
Estabelecimento de sirgueiro; o mesmo que \textunderscore serigaria\textunderscore .
\section{Sirgideira}
\begin{itemize}
\item {Grp. gram.:f.}
\end{itemize}
\begin{itemize}
\item {Proveniência:(De \textunderscore sirgir\textunderscore )}
\end{itemize}
Corda, própria para enxárcia.
O mesmo que \textunderscore serzideira\textunderscore .
\section{Sirgilim}
\begin{itemize}
\item {Grp. gram.:m.}
\end{itemize}
(V.gergelim)
\section{Sirgir}
\textunderscore v. t.\textunderscore  (e der.)
(V. \textunderscore serzir\textunderscore , etc.)
\section{Sirgo}
\begin{itemize}
\item {Grp. gram.:m.}
\end{itemize}
\begin{itemize}
\item {Utilização:Ant.}
\end{itemize}
\begin{itemize}
\item {Proveniência:(Do b. lat. \textunderscore siricus\textunderscore )}
\end{itemize}
Bicho da sêda.
Seriguilha grossa.
Sêda.
\section{Sirgueiro}
\begin{itemize}
\item {Grp. gram.:m.}
\end{itemize}
\begin{itemize}
\item {Proveniência:(De \textunderscore sirgo\textunderscore )}
\end{itemize}
O mesmo que \textunderscore serigueiro\textunderscore .
\section{Sirguilha}
\begin{itemize}
\item {Grp. gram.:f.}
\end{itemize}
\begin{itemize}
\item {Proveniência:(De \textunderscore sirgo\textunderscore )}
\end{itemize}
O mesmo ou melhor que \textunderscore seriguilha\textunderscore .
\section{Siri}
\begin{itemize}
\item {Grp. gram.:m.}
\end{itemize}
\begin{itemize}
\item {Utilização:Bras}
\end{itemize}
\begin{itemize}
\item {Proveniência:(T. tupi)}
\end{itemize}
Nome de várias espécies de crustáceos decápodes.
\section{Síria}
\begin{itemize}
\item {Grp. gram.:f.}
\end{itemize}
\begin{itemize}
\item {Utilização:Prov.}
\end{itemize}
\begin{itemize}
\item {Utilização:alent.}
\end{itemize}
\begin{itemize}
\item {Utilização:Prov.}
\end{itemize}
\begin{itemize}
\item {Utilização:trasm.}
\end{itemize}
\begin{itemize}
\item {Utilização:T. de Turquel}
\end{itemize}
Compleição, constituição phýsica.
Consistência ou robustez (das pernas).
Animação, vivacidade.
\section{Siríase}
\begin{itemize}
\item {Grp. gram.:f.}
\end{itemize}
\begin{itemize}
\item {Utilização:Med.}
\end{itemize}
\begin{itemize}
\item {Proveniência:(Lat. \textunderscore siriasis\textunderscore )}
\end{itemize}
Inflammação do cérebro ou das suas membranas.
\section{Siricaia}
\begin{itemize}
\item {Grp. gram.:f.}
\end{itemize}
\begin{itemize}
\item {Utilização:Bras}
\end{itemize}
Iguaria, em que entram principalmente ovos, leite e açúcar.
\section{Sirema}
\begin{itemize}
\item {Grp. gram.:f.}
\end{itemize}
\begin{itemize}
\item {Utilização:Bras}
\end{itemize}
Ave pernalta, notável pela guerra que faz a todos os animaes.
\section{Sirifoles}
\begin{itemize}
\item {Grp. gram.:m.}
\end{itemize}
Planta indiana, também conhecida por \textunderscore marmeleira-da-índia\textunderscore . Cf. X. Gracias, \textunderscore Flora\textunderscore .
\section{Sirigaita}
\begin{itemize}
\item {Grp. gram.:f.}
\end{itemize}
\begin{itemize}
\item {Utilização:Fig.}
\end{itemize}
Pequeno pássaro, semelhante á carriça.
Mulhér pretensiosa, que se saracoteia muito.
Mulhér buliçosa, ladina.
\section{Siringa}
\textunderscore f.\textunderscore  (e der.)
O mesmo que \textunderscore seringa\textunderscore , etc.
\section{Sírio}
\begin{itemize}
\item {Grp. gram.:m.}
\end{itemize}
\begin{itemize}
\item {Proveniência:(Lat. \textunderscore sirius\textunderscore )}
\end{itemize}
Grande estrêlla, da constellação do Cão Grande, chamada vulgarmente \textunderscore canícula\textunderscore .
\section{Sírio}
\begin{itemize}
\item {Grp. gram.:m.}
\end{itemize}
\begin{itemize}
\item {Utilização:Bras}
\end{itemize}
Saco, para transporte de mandioca.
\section{Sirito}
\begin{itemize}
\item {Grp. gram.:m.}
\end{itemize}
\begin{itemize}
\item {Utilização:Bras}
\end{itemize}
O mesmo que \textunderscore matame\textunderscore .
\section{Siroco}
\begin{itemize}
\item {Grp. gram.:m.}
\end{itemize}
O mesmo que \textunderscore xaroco\textunderscore .
\section{Sirolico-tico}
\begin{itemize}
\item {Grp. gram.:m.}
\end{itemize}
Espécie de jôgo infantil.
\section{Siro-siro}
\begin{itemize}
\item {Grp. gram.:m.}
\end{itemize}
Árvore de Benguela.
\section{Sirvente}
\begin{itemize}
\item {Grp. gram.:f.}
\end{itemize}
Poesia satírica, da escola trovadoresca.
(Provn. \textunderscore sirventes\textunderscore )
\section{Sirventésio}
\begin{itemize}
\item {Grp. gram.:adj.}
\end{itemize}
Diz-se do verso, próprio de sirventes. Cf. Garrett, \textunderscore Romanceiro\textunderscore , II, 127.
\section{Sisa}
\begin{itemize}
\item {Grp. gram.:f.}
\end{itemize}
\begin{itemize}
\item {Proveniência:(Do b. lat. \textunderscore assisia\textunderscore )}
\end{itemize}
Nome antigo do chamado hoje imposto de transmissão.
\section{Sisal}
\begin{itemize}
\item {Grp. gram.:m.}
\end{itemize}
Planta americana, (\textunderscore agave rigida sisalana\textunderscore ), de fibra têxtil, e originária do México. Cf. \textunderscore Jorn.-do Comm.\textunderscore , do Rio, de 10-XII-908.
\section{Sisalana}
\begin{itemize}
\item {Grp. gram.:f.}
\end{itemize}
Fibra têxtil do sisal, succedâneo do cânhamo.
\section{Sisamina}
\begin{itemize}
\item {Grp. gram.:f.}
\end{itemize}
\begin{itemize}
\item {Utilização:Anat.}
\end{itemize}
\begin{itemize}
\item {Utilização:Ant.}
\end{itemize}
\begin{itemize}
\item {Proveniência:(Do ar. \textunderscore semsaminat\textunderscore )}
\end{itemize}
Cada um dos ossos miúdos das junturas dos dedos.
\section{Sisão}
\begin{itemize}
\item {Grp. gram.:m.}
\end{itemize}
\begin{itemize}
\item {Utilização:Ant.}
\end{itemize}
Nome de uma ave:«\textunderscore ...e se cevam\textunderscore  (os falcões) \textunderscore em garçotas e meãs, sisões, zambralhos e ganços reaes.\textunderscore »Fernandes, \textunderscore Caça de Altan.\textunderscore , p. III, c. I.
\section{Sisar}
\begin{itemize}
\item {Grp. gram.:v. t.}
\end{itemize}
\begin{itemize}
\item {Utilização:Ant.}
\end{itemize}
\begin{itemize}
\item {Utilização:Fam.}
\end{itemize}
\begin{itemize}
\item {Utilização:Ant.}
\end{itemize}
\begin{itemize}
\item {Grp. gram.:V. i.}
\end{itemize}
\begin{itemize}
\item {Utilização:Prov.}
\end{itemize}
Impor sisa a.
Tributar com sisa.
Furtar nas compras, dando conta superior ás despesas.
Deminuír, cercear. Cf. \textunderscore Luz e Calor\textunderscore , 93.
Pagar sisa ou contribuição de registo por título oneroso. (Colhido em Alcanena)
\section{Siseiro}
\begin{itemize}
\item {Grp. gram.:m.}
\end{itemize}
\begin{itemize}
\item {Utilização:Ant.}
\end{itemize}
Cobrador de sisas. Cf. G. Vicente, I. 344.
\section{Sisgo}
\begin{itemize}
\item {Grp. gram.:m.}
\end{itemize}
\begin{itemize}
\item {Utilização:Prov.}
\end{itemize}
\begin{itemize}
\item {Utilização:Taur.}
\end{itemize}
\begin{itemize}
\item {Utilização:minh.}
\end{itemize}
O mesmo que \textunderscore sesgo\textunderscore .
\section{Sisífio}
\begin{itemize}
\item {Grp. gram.:adj.}
\end{itemize}
\begin{itemize}
\item {Proveniência:(Lat. \textunderscore sisyphius\textunderscore )}
\end{itemize}
Relativo a Corintho.
\section{Sisímbrio}
\begin{itemize}
\item {Grp. gram.:m.}
\end{itemize}
\begin{itemize}
\item {Proveniência:(Lat. \textunderscore sisymbrium\textunderscore )}
\end{itemize}
Gênero de plantas crucíferas, entre cujas espécies se conta o serpol.
\section{Sisma}
\begin{itemize}
\item {Grp. gram.:f.}
\end{itemize}
Acto de sismar.
Mania.
Devaneio.
\section{Sismal}
\begin{itemize}
\item {Grp. gram.:adj.}
\end{itemize}
\begin{itemize}
\item {Proveniência:(Do gr. \textunderscore seismos\textunderscore )}
\end{itemize}
Diz-se da linha que indica a direcção de um terremoto.
\section{Sismar}
\begin{itemize}
\item {Grp. gram.:v. i.}
\end{itemize}
\begin{itemize}
\item {Grp. gram.:v. t.}
\end{itemize}
\begin{itemize}
\item {Grp. gram.:V. i.}
\end{itemize}
\begin{itemize}
\item {Grp. gram.:M.}
\end{itemize}
Talvez melhor que \textunderscore scismar\textunderscore .
(Cp. cast. \textunderscore ensimismarse\textunderscore )
Pensar muito em.
Meditar, preoccupar-se.
Andar apprehensivo.
Ideia fixa, scisma.
(Or. duvidosa. Relacionar-se-á com o cast. \textunderscore ensimismar-se\textunderscore ? Neste caso, deveríamos escrever \textunderscore sismar\textunderscore )
\section{Sísmico}
\begin{itemize}
\item {Grp. gram.:adj.}
\end{itemize}
\begin{itemize}
\item {Proveniência:(De \textunderscore sismo\textunderscore )}
\end{itemize}
Relativo aos terremotos.
\section{Sismo}
\begin{itemize}
\item {Grp. gram.:m.}
\end{itemize}
\begin{itemize}
\item {Proveniência:(Do gr. \textunderscore seismos\textunderscore , abalo)}
\end{itemize}
Designação scientífica do terremoto.
\section{Sismografia}
\begin{itemize}
\item {Grp. gram.:f.}
\end{itemize}
Aplicação do sismógrafo.
\section{Sismográfico}
\begin{itemize}
\item {Grp. gram.:adj.}
\end{itemize}
Relativo á sismografia.
\section{Sismógrafo}
\begin{itemize}
\item {Grp. gram.:m.}
\end{itemize}
\begin{itemize}
\item {Proveniência:(Do gr. \textunderscore seismos\textunderscore  + \textunderscore graphein\textunderscore )}
\end{itemize}
Instrumento, para indicar a intensidade dos tremores de terra.
\section{Sismographia}
\begin{itemize}
\item {Grp. gram.:f.}
\end{itemize}
Applicação do sismógrapho.
\section{Sismográphico}
\begin{itemize}
\item {Grp. gram.:adj.}
\end{itemize}
Relativo á sismographia.
\section{Sismógrapho}
\begin{itemize}
\item {Grp. gram.:m.}
\end{itemize}
\begin{itemize}
\item {Proveniência:(Do gr. \textunderscore seismos\textunderscore  + \textunderscore graphein\textunderscore )}
\end{itemize}
Instrumento, para indicar a intensidade dos tremores de terra.
\section{Sismologia}
\begin{itemize}
\item {Grp. gram.:f.}
\end{itemize}
\begin{itemize}
\item {Proveniência:(Do gr. \textunderscore seismos\textunderscore  + \textunderscore logos\textunderscore )}
\end{itemize}
Tratado dos tremores de terra.
\section{Sismológico}
\begin{itemize}
\item {Grp. gram.:adj.}
\end{itemize}
Relativo á sismologia.
\section{Sismólogo}
\begin{itemize}
\item {Grp. gram.:m.}
\end{itemize}
Aquelle que é perito em sismologia.
\section{Sismómetro}
\begin{itemize}
\item {Grp. gram.:m.}
\end{itemize}
\begin{itemize}
\item {Proveniência:(Do gr. \textunderscore seismos\textunderscore  + \textunderscore metron\textunderscore )}
\end{itemize}
Apparelho para a observação directa dos terremotos.
\section{Sismondina}
\begin{itemize}
\item {Grp. gram.:f.}
\end{itemize}
\begin{itemize}
\item {Proveniência:(De \textunderscore Sismondí\textunderscore , n. p.)}
\end{itemize}
Hydròsilicato de alumina e ferro.
\section{Sismoterapia}
\begin{itemize}
\item {Grp. gram.:f.}
\end{itemize}
\begin{itemize}
\item {Utilização:Med.}
\end{itemize}
\begin{itemize}
\item {Proveniência:(Do gr. \textunderscore seismos\textunderscore  + \textunderscore therapeia\textunderscore )}
\end{itemize}
Tratamento, que consiste em se imprimirem ao corpo, ou a parte dêle, vibrações rápidas, regulares e de pequena amplitude.
\section{Sismotherapia}
\begin{itemize}
\item {Grp. gram.:f.}
\end{itemize}
\begin{itemize}
\item {Utilização:Med.}
\end{itemize}
\begin{itemize}
\item {Proveniência:(Do gr. \textunderscore seismos\textunderscore  + \textunderscore therapeia\textunderscore )}
\end{itemize}
Tratamento, que consiste em se imprimirem ao corpo, ou a parte dêlle, vibrações rápidas, regulares e de pequena amplitude.
\section{Siso}
\begin{itemize}
\item {Grp. gram.:m.}
\end{itemize}
\begin{itemize}
\item {Grp. gram.:Loc. adv.}
\end{itemize}
\begin{itemize}
\item {Proveniência:(Do lat. \textunderscore sensus\textunderscore )}
\end{itemize}
Bom senso; circunspecção; juízo; prudência.
\textunderscore De siso\textunderscore , sensatamente:«\textunderscore discorria tão de siso...\textunderscore »Camillo, \textunderscore Agulha\textunderscore , 16, (2.^a ed., 1865).
\textunderscore Dente do siso\textunderscore , o último dos queixaes, que nasce nos adultos.
\section{Siso}
\begin{itemize}
\item {Grp. gram.:m.}
\end{itemize}
\begin{itemize}
\item {Utilização:Prov.}
\end{itemize}
Rodela de cortiça, que alarga o canal da roca.
\section{Sisório}
\begin{itemize}
\item {Grp. gram.:m.}
\end{itemize}
\begin{itemize}
\item {Utilização:Pop.}
\end{itemize}
Muito siso.
\section{Sissó}
\begin{itemize}
\item {Grp. gram.:m.}
\end{itemize}
(V.ciçó)
\section{Sistrado}
\begin{itemize}
\item {Grp. gram.:adj.}
\end{itemize}
\begin{itemize}
\item {Proveniência:(Lat. \textunderscore sistratus\textunderscore )}
\end{itemize}
Que usa sistro ou conduz o sistro.
\section{Sistro}
\begin{itemize}
\item {Grp. gram.:m.}
\end{itemize}
\begin{itemize}
\item {Proveniência:(Lat. \textunderscore sistrum\textunderscore )}
\end{itemize}
Antigo instrumento músico de Egypto, o qual era um pequeno arco de metal, atravessado por hastes metállicas que, agitadas, produziam um som agudo e prolongado.
Actualmente, dá-se o nome de sistro a uma espécie de marimbas, com lâminas metállicas.
\section{Sisudez}
\begin{itemize}
\item {Grp. gram.:f.}
\end{itemize}
Qualidade do que é sisudo.
Seriedade; gravidade de porte.
\section{Sisudeza}
\begin{itemize}
\item {Grp. gram.:f.}
\end{itemize}
Qualidade do que é sisudo.
Seriedade; gravidade de porte.
\section{Sisudo}
\begin{itemize}
\item {Grp. gram.:adj.}
\end{itemize}
\begin{itemize}
\item {Grp. gram.:M.}
\end{itemize}
\begin{itemize}
\item {Proveniência:(De \textunderscore siso\textunderscore )}
\end{itemize}
Que tem siso; sério.
Prudente.
Indivíduo sisudo.
Espécie de jôgo popular.
\section{Sisýmbrio}
\begin{itemize}
\item {Grp. gram.:m.}
\end{itemize}
\begin{itemize}
\item {Proveniência:(Lat. \textunderscore sisymbrium\textunderscore )}
\end{itemize}
Gênero de plantas crucíferas, entre cujas espécies se conta o serpol.
\section{Sisýphio}
\begin{itemize}
\item {Grp. gram.:adj.}
\end{itemize}
\begin{itemize}
\item {Proveniência:(Lat. \textunderscore sisyphius\textunderscore )}
\end{itemize}
Relativo a Corintho.
\section{Sita}
\begin{itemize}
\item {Grp. gram.:f.}
\end{itemize}
Gênero de aves, a que pertence o picanço.
\section{Sita}
\begin{itemize}
\item {Grp. gram.:f.}
\end{itemize}
\begin{itemize}
\item {Utilização:Ant.}
\end{itemize}
Marca ou cunho da moéda. Cf. \textunderscore Tombo do Estado da Índia\textunderscore , 226.
\section{Sitana}
\begin{itemize}
\item {Grp. gram.:f.}
\end{itemize}
Gênero de reptis sáurios.
\section{Sitárcia}
\begin{itemize}
\item {Grp. gram.:f.}
\end{itemize}
\begin{itemize}
\item {Proveniência:(Lat. \textunderscore sitarcia\textunderscore )}
\end{itemize}
Saco, em que os antigos levavam provisões de viagem.
Os comestíveis, contidos nêsse saco.
\section{Sitarrão}
\begin{itemize}
\item {Grp. gram.:m.}
\end{itemize}
\begin{itemize}
\item {Utilização:Bras. do N}
\end{itemize}
Sítio ou quinta grande.
(Por \textunderscore sitiarrão\textunderscore , de \textunderscore sítio\textunderscore ^1)
\section{Sitela}
\begin{itemize}
\item {Grp. gram.:f.}
\end{itemize}
\begin{itemize}
\item {Proveniência:(De \textunderscore sita\textunderscore ^1)}
\end{itemize}
Gênero de aves trepadeiras.
\section{Sitela}
\begin{itemize}
\item {Grp. gram.:f.}
\end{itemize}
\begin{itemize}
\item {Proveniência:(Lat. \textunderscore sitella\textunderscore )}
\end{itemize}
Urna de barro ou de bronze, de que os Romanos se serviam em certas votações e eleições.
\section{Sitella}
\begin{itemize}
\item {Grp. gram.:f.}
\end{itemize}
\begin{itemize}
\item {Proveniência:(Lat. \textunderscore sitella\textunderscore )}
\end{itemize}
Urna de barro ou de bronze, de que os Romanos se serviam em certas votações e eleições.
\section{Sitiado}
\begin{itemize}
\item {Grp. gram.:adj.}
\end{itemize}
\begin{itemize}
\item {Grp. gram.:M.}
\end{itemize}
\begin{itemize}
\item {Proveniência:(De \textunderscore sitiar\textunderscore )}
\end{itemize}
Cercado por fôrças militares; assediado.
Aquelle que está sitiado.
\section{Sitiador}
\begin{itemize}
\item {Grp. gram.:m.  e  adj.}
\end{itemize}
O que sitía.
\section{Sitial}
\begin{itemize}
\item {Grp. gram.:m.}
\end{itemize}
O mesmo que \textunderscore setial\textunderscore . Cf. \textunderscore Viriato Trág.\textunderscore , 9.
\section{Sitiamento}
\begin{itemize}
\item {Grp. gram.:m.}
\end{itemize}
O mesmo que \textunderscore sítio\textunderscore ^2.
\section{Sitiante}
\begin{itemize}
\item {Grp. gram.:m. ,  f.  e  adj.}
\end{itemize}
Pessôa que sitía.
\section{Sitiar}
\begin{itemize}
\item {Grp. gram.:v. t.}
\end{itemize}
\begin{itemize}
\item {Utilização:Ext.}
\end{itemize}
\begin{itemize}
\item {Proveniência:(Do ant. alt. al. \textunderscore sittian\textunderscore )}
\end{itemize}
Pôr sítio ou cêrco a.
Assediar.
Cercar.
\section{Sitibundo}
\begin{itemize}
\item {Grp. gram.:m.  e  adj.}
\end{itemize}
\begin{itemize}
\item {Utilização:Poét.}
\end{itemize}
\begin{itemize}
\item {Proveniência:(Lat. \textunderscore sitibundus\textunderscore )}
\end{itemize}
Que tem sêde.
Sedento. Cf. \textunderscore Lusíadas\textunderscore , IV, 44.
\section{Sítio}
\begin{itemize}
\item {Grp. gram.:m.}
\end{itemize}
\begin{itemize}
\item {Utilização:Pop.}
\end{itemize}
\begin{itemize}
\item {Utilização:Bras}
\end{itemize}
\begin{itemize}
\item {Proveniência:(Do lat. \textunderscore situs\textunderscore )}
\end{itemize}
Lugar ou espaço, occupado por um objecto.
Local.
Chão descoberto.
Lugar, assinalado por acontecimento importante.
Povoação; aldeia.
Roça.
Quinta.
Casa rústica, com granja.
Pequena roça.
\section{Sítio}
\begin{itemize}
\item {Grp. gram.:m.}
\end{itemize}
Acto ou effeito de sitiar.
\section{Sitiofobia}
\begin{itemize}
\item {Grp. gram.:f.}
\end{itemize}
\begin{itemize}
\item {Utilização:Med.}
\end{itemize}
\begin{itemize}
\item {Proveniência:(Do gr. \textunderscore sition\textunderscore  + \textunderscore phobos\textunderscore )}
\end{itemize}
Recusa absoluta de alimento.
\section{Sitiologia}
\begin{itemize}
\item {Grp. gram.:f.}
\end{itemize}
\begin{itemize}
\item {Proveniência:(Do gr. \textunderscore sition\textunderscore  + \textunderscore logos\textunderscore )}
\end{itemize}
Tratado dos alimentos ou da alimentação.
\section{Sitiológico}
\begin{itemize}
\item {Grp. gram.:adj.}
\end{itemize}
Relativo á sitiologia.
\section{Sitiomania}
\begin{itemize}
\item {Grp. gram.:f.}
\end{itemize}
\begin{itemize}
\item {Proveniência:(Do gr. \textunderscore sition\textunderscore  + \textunderscore mania\textunderscore )}
\end{itemize}
Mania ou paixão pela comida.
\section{Sitiophobia}
\begin{itemize}
\item {Grp. gram.:f.}
\end{itemize}
\begin{itemize}
\item {Utilização:Med.}
\end{itemize}
\begin{itemize}
\item {Proveniência:(Do gr. \textunderscore sition\textunderscore  + \textunderscore phobos\textunderscore )}
\end{itemize}
Recusa absoluta de alimento.
\section{Sitite}
\begin{itemize}
\item {Grp. gram.:f.}
\end{itemize}
\begin{itemize}
\item {Proveniência:(Lat. \textunderscore sitites\textunderscore )}
\end{itemize}
Variedade de pedra preciosa, mencionada entre os antigos e hoje desconhecida.
\section{Sito}
\begin{itemize}
\item {Grp. gram.:adj.}
\end{itemize}
\begin{itemize}
\item {Proveniência:(Lat. \textunderscore situs\textunderscore , de \textunderscore sino\textunderscore )}
\end{itemize}
O mesmo que \textunderscore situado\textunderscore .
\section{Sito}
\begin{itemize}
\item {Grp. gram.:m.}
\end{itemize}
\begin{itemize}
\item {Proveniência:(Do lat. \textunderscore situs\textunderscore )}
\end{itemize}
Bafio, bolor.
\section{Sitófago}
\begin{itemize}
\item {Grp. gram.:adj.}
\end{itemize}
\begin{itemize}
\item {Proveniência:(Gr. \textunderscore sitophagos\textunderscore )}
\end{itemize}
Que se alimenta de trigo.
\section{Sitófilo}
\begin{itemize}
\item {Grp. gram.:m.}
\end{itemize}
\begin{itemize}
\item {Proveniência:(Do gr. \textunderscore sitos\textunderscore  + \textunderscore philos\textunderscore )}
\end{itemize}
Gênero de insectos coleópteros tetrâmeros.
\section{Sitofílace}
\begin{itemize}
\item {Grp. gram.:m.}
\end{itemize}
\begin{itemize}
\item {Proveniência:(Do gr. \textunderscore sitophulax\textunderscore )}
\end{itemize}
Cada um dos cinco magistrados athenienses, que vigiavam as provisões do trigo, obstando, para que o gênero não escasseasse no mercado, a que qualquer cidadão comprasse mais do que precisava.
\section{Sitona}
\begin{itemize}
\item {Grp. gram.:m.}
\end{itemize}
\begin{itemize}
\item {Proveniência:(Lat. \textunderscore sitona\textunderscore )}
\end{itemize}
Intendente dos cereaes na Grécia antiga, ou funcionário, que adquiria o trigo para o consumo público.
\section{Sitóphago}
\begin{itemize}
\item {Grp. gram.:adj.}
\end{itemize}
\begin{itemize}
\item {Proveniência:(Gr. \textunderscore sitophagos\textunderscore )}
\end{itemize}
Que se alimenta de trigo.
\section{Sitóphilo}
\begin{itemize}
\item {Grp. gram.:m.}
\end{itemize}
\begin{itemize}
\item {Proveniência:(Do gr. \textunderscore sitos\textunderscore  + \textunderscore philos\textunderscore )}
\end{itemize}
Gênero de insectos coleópteros tetrâmeros.
\section{Sitophýlace}
\begin{itemize}
\item {Grp. gram.:m.}
\end{itemize}
\begin{itemize}
\item {Proveniência:(Do gr. \textunderscore sitophulax\textunderscore )}
\end{itemize}
Cada um dos cinco magistrados athenienses, que vigiavam as provisões do trigo, obstando, para que o gênero não escasseasse no mercado, a que qualquer cidadão comprasse mais do que precisava.
\section{Sitta}
\begin{itemize}
\item {Grp. gram.:f.}
\end{itemize}
Gênero de aves, a que pertence o picanço.
\section{Sittella}
\begin{itemize}
\item {Grp. gram.:f.}
\end{itemize}
\begin{itemize}
\item {Proveniência:(De \textunderscore sitta\textunderscore )}
\end{itemize}
Gênero de aves trepadeiras.
\section{Situação}
\begin{itemize}
\item {Grp. gram.:f.}
\end{itemize}
\begin{itemize}
\item {Utilização:Bras. do N}
\end{itemize}
Acto ou effeito de situar.
Sítio.
Posição.
Disposição recíproca das differentes partes de um todo.
Lugar, onde está uma coisa ou pessôa.
Phase governamental ou ministerial.
O Ministério ou Govêrno, relativamente á actualidade ou a uma dada época.
Posição de um indivíduo, relativamente á sua profissão.
Estado financeiro de um indivíduo ou de uma collectividade.
Estado ou condição.
Lance.
Vicissitude; occorrência.
Pequena fazenda ou quinta, para criação de animaes.
\section{Situacionista}
\begin{itemize}
\item {Grp. gram.:m.}
\end{itemize}
\begin{itemize}
\item {Utilização:bras}
\end{itemize}
\begin{itemize}
\item {Utilização:Neol.}
\end{itemize}
Partidário da situação política.
\section{Situado}
\begin{itemize}
\item {Grp. gram.:adj.}
\end{itemize}
\begin{itemize}
\item {Proveniência:(De \textunderscore situar\textunderscore )}
\end{itemize}
Que occupa determinado lugar: \textunderscore uma casa bem situada\textunderscore .
Estabelecido.
\section{Situar}
\begin{itemize}
\item {Grp. gram.:v. t.}
\end{itemize}
\begin{itemize}
\item {Grp. gram.:V. i.}
\end{itemize}
\begin{itemize}
\item {Utilização:Bras}
\end{itemize}
\begin{itemize}
\item {Proveniência:(Do lat. \textunderscore situs\textunderscore )}
\end{itemize}
Collocar ou estabelecer.
Edificar em lugar próprio ou escolhido.
Assinar lugar a.
Fundar ou estabelecer pequena quinta para criação de (animaes)
\section{Sítula}
\begin{itemize}
\item {Grp. gram.:f.}
\end{itemize}
\begin{itemize}
\item {Proveniência:(Lat. \textunderscore situla\textunderscore )}
\end{itemize}
Taramela do moínho.
\section{Siús}
\begin{itemize}
\item {Grp. gram.:m. pl.}
\end{itemize}
Tríbo de índios da América do Norte.
\section{Siva}
\begin{itemize}
\item {Grp. gram.:f.}
\end{itemize}
Espécie de dança muito variada e agitada, usada pelos Samoanos. Cf. \textunderscore Jorn.-de-Viagens\textunderscore , IV, 275.
\section{Sivaísmo}
\begin{itemize}
\item {Grp. gram.:m.}
\end{itemize}
\begin{itemize}
\item {Proveniência:(De \textunderscore Siva\textunderscore , uma das pessôas da trimurti)}
\end{itemize}
Seita brahmânica.
\section{Sivaísta}
\begin{itemize}
\item {Grp. gram.:m.}
\end{itemize}
Sectário do sivaísmo.
\section{Sivan}
\begin{itemize}
\item {Grp. gram.:m.}
\end{itemize}
Nono mês do anno civil dos Hebreus.
\section{Sivane}
\begin{itemize}
\item {Grp. gram.:f.}
\end{itemize}
Árvore de grandes dimensões, (\textunderscore gmelina arborea\textunderscore ), cuja madeira leve e resistente é, na Índia Portuguesa, muito empregada em coronhas de armas, estatuêtas e outros trabalhos de marcenaria.
\section{Sivatério}
\begin{itemize}
\item {Grp. gram.:m.}
\end{itemize}
\begin{itemize}
\item {Proveniência:(De \textunderscore Siva\textunderscore , contr. \textunderscore Sivalike\textunderscore , n. p. + gr. \textunderscore therion\textunderscore )}
\end{itemize}
Gênero de mamíferos fósseis dos terrenos terciários.
\section{Sivathério}
\begin{itemize}
\item {Grp. gram.:m.}
\end{itemize}
\begin{itemize}
\item {Proveniência:(De \textunderscore Siva\textunderscore , contr. \textunderscore Sivalike\textunderscore , n. p. + gr. \textunderscore therion\textunderscore )}
\end{itemize}
Gênero de mammíferos fósseis dos terrenos terciários.
\section{Sivom}
\begin{itemize}
\item {Grp. gram.:m.}
\end{itemize}
Árvore da Índia Portuguesa.
\section{Sizau}
\begin{itemize}
\item {Grp. gram.:m.}
\end{itemize}
O mesmo que \textunderscore alcaravão\textunderscore .
\section{Slânea}
\begin{itemize}
\item {Grp. gram.:f.}
\end{itemize}
Gênero do plantas tiliáceas.
\section{Slavo}
\begin{itemize}
\item {Grp. gram.:m.  e  adj.}
\end{itemize}
(V.eslavo)
\section{Snr.}
O mesmo que \textunderscore sr.\textunderscore .
\section{Só}
\begin{itemize}
\item {Grp. gram.:adj.}
\end{itemize}
\begin{itemize}
\item {Grp. gram.:Adv.}
\end{itemize}
\begin{itemize}
\item {Grp. gram.:M.}
\end{itemize}
\begin{itemize}
\item {Grp. gram.:Loc. adv.}
\end{itemize}
\begin{itemize}
\item {Grp. gram.:Loc. adv.}
\end{itemize}
\begin{itemize}
\item {Utilização:Prov.}
\end{itemize}
\begin{itemize}
\item {Proveniência:(Do lat. \textunderscore solus\textunderscore )}
\end{itemize}
Que está sem companhia, que não está com outros.
Único.
Afastado da convivência social.
Desamparado.
Solitário; ermo: \textunderscore sítio muito só\textunderscore .
Somente.
Aquelle que vive só.
Aquelle que no voltarete joga somente com as cartas que teve e não compra nenhuma.
\textunderscore A sós\textunderscore , consigo próprio; sem mais companhia.
\textunderscore só por só\textunderscore , um por um; insuladamente. (Colhido na Bairrada)
\section{Só}
\begin{itemize}
\item {Grp. gram.:m.}
\end{itemize}
\begin{itemize}
\item {Utilização:Prov.}
\end{itemize}
\begin{itemize}
\item {Utilização:trasm.}
\end{itemize}
Fundo de vasilha.
Fundo de agulha.
\section{Sô}
\begin{itemize}
\item {Grp. gram.:m.}
\end{itemize}
\begin{itemize}
\item {Utilização:Pop.}
\end{itemize}
(Corr. de \textunderscore senhor\textunderscore )
\section{Sô}
\begin{itemize}
\item {Grp. gram.:prep.}
\end{itemize}
\begin{itemize}
\item {Utilização:Ant.}
\end{itemize}
O mesmo que \textunderscore sob\textunderscore .
\section{So...}
\begin{itemize}
\item {Grp. gram.:pref.}
\end{itemize}
\begin{itemize}
\item {Proveniência:(Do lat. \textunderscore sub\textunderscore )}
\end{itemize}
(designativo de \textunderscore debaixo\textunderscore  ou \textunderscore sob\textunderscore )
\section{S. O.}
Abrev. de \textunderscore sudoéste\textunderscore .
\section{Soabrir}
\begin{itemize}
\item {Grp. gram.:v. t.}
\end{itemize}
\begin{itemize}
\item {Proveniência:(De \textunderscore so...\textunderscore  + \textunderscore abrir\textunderscore )}
\end{itemize}
O mesmo que \textunderscore entreabrir\textunderscore .
\section{Soada}
\begin{itemize}
\item {Grp. gram.:f.}
\end{itemize}
Acto ou effeito de soar.
Toada de cantiga.
Ruído.
Rumor indistinto.
Boato; fama.
\section{Soadeiro}
\begin{itemize}
\item {Grp. gram.:m.}
\end{itemize}
\begin{itemize}
\item {Utilização:Ant.}
\end{itemize}
Lenço de assoar? Cf. G. Vicente.
(Por \textunderscore assoadeiro\textunderscore , de \textunderscore assoar\textunderscore ?)
\section{Soadeiro}
\begin{itemize}
\item {Grp. gram.:adj.}
\end{itemize}
\begin{itemize}
\item {Utilização:Ant.}
\end{itemize}
\begin{itemize}
\item {Proveniência:(De \textunderscore soar\textunderscore )}
\end{itemize}
Afamado; notável.
\section{Soado}
\begin{itemize}
\item {Grp. gram.:adj.}
\end{itemize}
\begin{itemize}
\item {Proveniência:(De \textunderscore soar\textunderscore )}
\end{itemize}
Que soou.
\section{Soaje}
\begin{itemize}
\item {Grp. gram.:f.}
\end{itemize}
\begin{itemize}
\item {Utilização:Ant.}
\end{itemize}
Espécie de ornato, talvez espécie de arabesco:«\textunderscore ...hũa cadeyra... toda chea de suas soajes e de liõis...\textunderscore »\textunderscore Chrón. dos reis de Bisnaga\textunderscore , 102.
\section{Soagem}
\begin{itemize}
\item {Grp. gram.:f.}
\end{itemize}
\begin{itemize}
\item {Proveniência:(Do lat. \textunderscore solaga\textunderscore )}
\end{itemize}
Planta borragínea.
\section{Soala}
\begin{itemize}
\item {Grp. gram.:f.}
\end{itemize}
Gênero de plantas clusiáceas.
\section{Soalha}
\begin{itemize}
\item {Grp. gram.:f.}
\end{itemize}
\begin{itemize}
\item {Utilização:Náut.}
\end{itemize}
\begin{itemize}
\item {Utilização:Ant.}
\end{itemize}
\begin{itemize}
\item {Proveniência:(De \textunderscore soar\textunderscore )}
\end{itemize}
Cada uma das chapas metállicas do pandeiro, que retinem, batendo umas nas outras.
Travéssa, que se adaptava ao virote da balestrilha, e que também se chamava \textunderscore transversário\textunderscore .
\section{Soalhado}
\begin{itemize}
\item {Grp. gram.:m.}
\end{itemize}
\begin{itemize}
\item {Proveniência:(De \textunderscore soalhar\textunderscore ^2)}
\end{itemize}
Soalho; madeiramento para soalhar.
\section{Soalhal}
\begin{itemize}
\item {Grp. gram.:m.}
\end{itemize}
\begin{itemize}
\item {Utilização:Pop.}
\end{itemize}
O mesmo que \textunderscore soalheiro\textunderscore :«\textunderscore o Natal ao soalhal e a Páscoa ao luar.\textunderscore »(Prolóquio popular)
\section{Soalhar}
\begin{itemize}
\item {Grp. gram.:v. t.}
\end{itemize}
Agitar as soalhas de (um pandeiro).
\section{Soalhar}
\begin{itemize}
\item {Grp. gram.:v. t.}
\end{itemize}
O mesmo que \textunderscore assoalhar\textunderscore ^2.
\section{Soalhar}
\begin{itemize}
\item {Grp. gram.:m.}
\end{itemize}
\begin{itemize}
\item {Utilização:Pop.}
\end{itemize}
O mesmo que \textunderscore soalhal\textunderscore .
\section{Soalheira}
\begin{itemize}
\item {Grp. gram.:f.}
\end{itemize}
\begin{itemize}
\item {Proveniência:(De \textunderscore soalheiro\textunderscore )}
\end{itemize}
A hora do calor mais intenso.
Exposição aos raios do sol; calma.
\section{Soalheiro}
\begin{itemize}
\item {Grp. gram.:adj.}
\end{itemize}
\begin{itemize}
\item {Grp. gram.:M.}
\end{itemize}
\begin{itemize}
\item {Utilização:Fam.}
\end{itemize}
\begin{itemize}
\item {Proveniência:(De \textunderscore soalho\textunderscore ^1)}
\end{itemize}
Exposto á acção do sol.
Lugar exposto ao sol.
Reunião de pessôas ociosas, que tratam da vida alheia, ordinariamente sentadas ao sol.
\section{Soalho}
\begin{itemize}
\item {Grp. gram.:m.}
\end{itemize}
\begin{itemize}
\item {Proveniência:(De \textunderscore soalhar\textunderscore ^2)}
\end{itemize}
O mesmo que \textunderscore sòlheiro\textunderscore .
\section{Soalho}
\begin{itemize}
\item {Grp. gram.:m.}
\end{itemize}
(Corr. de \textunderscore sôlho\textunderscore ^1)
\section{Soan-soá}
\begin{itemize}
\item {Grp. gram.:m.}
\end{itemize}
O mesmo que \textunderscore soá-soá\textunderscore .
\section{Soante}
\begin{itemize}
\item {Grp. gram.:adj.}
\end{itemize}
\begin{itemize}
\item {Proveniência:(Do lat. \textunderscore sonans\textunderscore )}
\end{itemize}
Que sôa.
\section{Soão}
\begin{itemize}
\item {Grp. gram.:m.}
\end{itemize}
\begin{itemize}
\item {Utilização:Des.}
\end{itemize}
\begin{itemize}
\item {Proveniência:(Do lat. \textunderscore solanus\textunderscore )}
\end{itemize}
Vento, que sopra do Oriente.
Oriente.
\section{Soar}
\begin{itemize}
\item {Grp. gram.:v. i.}
\end{itemize}
\begin{itemize}
\item {Grp. gram.:V. t.}
\end{itemize}
\begin{itemize}
\item {Utilização:Fig.}
\end{itemize}
\begin{itemize}
\item {Proveniência:(Lat. \textunderscore sonare\textunderscore )}
\end{itemize}
Produzir som.
Retumbar.
Echoar.
Constar, divulgar-se, propalar-se.
Têr semelhança.
Aprazer, convir: \textunderscore é proposta que me não sôa\textunderscore .
Tanger, tocar.
Divulgar; celebrar.
Indicar.
\section{Soar}
\begin{itemize}
\item {Grp. gram.:m.}
\end{itemize}
\begin{itemize}
\item {Utilização:Ant.}
\end{itemize}
Solar, onde algum território ou coito attesta jurisdicção especial, concedida a alguém pelo Rei.
(Contr. de \textunderscore solar\textunderscore ^4)
\section{Soar}
\begin{itemize}
\item {Grp. gram.:m.}
\end{itemize}
\begin{itemize}
\item {Utilização:Prov.}
\end{itemize}
\begin{itemize}
\item {Utilização:minh.}
\end{itemize}
Solar da porta; soleira.
(Contr. de \textunderscore solar\textunderscore ^3)
\section{Soá-soá}
\begin{itemize}
\item {Grp. gram.:m.}
\end{itemize}
Árvore de Angola e San-Thomé, própria para construcções, da fam. das violáceas.
\section{Soassar}
\begin{itemize}
\item {Grp. gram.:v. t.}
\end{itemize}
\begin{itemize}
\item {Proveniência:(De \textunderscore so...\textunderscore  + \textunderscore assar\textunderscore )}
\end{itemize}
Assar ligeiramente.
\section{Sob}
\begin{itemize}
\item {Grp. gram.:prep.}
\end{itemize}
\begin{itemize}
\item {Proveniência:(Do lat. \textunderscore sub\textunderscore )}
\end{itemize}
Debaixo de: \textunderscore sob o telhado\textunderscore ; \textunderscore sob condições\textunderscore .
No tempo ou govêrno de: \textunderscore o Christianismo triumphou sob Constantino\textunderscore .
\section{Soba}
\begin{itemize}
\item {Grp. gram.:m.}
\end{itemize}
Chefe de tríbo africana; régulo.
\section{Sobaco}
\begin{itemize}
\item {Grp. gram.:m.}
\end{itemize}
O mesmo que \textunderscore sovaco\textunderscore .
\section{Sobado}
\begin{itemize}
\item {Grp. gram.:m.}
\end{itemize}
Território, governado por um soba. Cf. \textunderscore Século\textunderscore , de 22-IV-901.
\section{Sobalçar}
\begin{itemize}
\item {Grp. gram.:v. t.}
\end{itemize}
\begin{itemize}
\item {Utilização:Fig.}
\end{itemize}
\begin{itemize}
\item {Proveniência:(De \textunderscore sob...\textunderscore  + \textunderscore alçar\textunderscore )}
\end{itemize}
Alçar muito.
Exaltar.
\section{Sobarba}
\begin{itemize}
\item {Grp. gram.:f.}
\end{itemize}
\begin{itemize}
\item {Utilização:Ant.}
\end{itemize}
\begin{itemize}
\item {Proveniência:(De \textunderscore so...\textunderscore  + \textunderscore barba\textunderscore )}
\end{itemize}
Parte ou accessório de qualquer cobertura da cabeça, destinado a atar-se por debaixo da barba.
\section{Sobarbada}
\begin{itemize}
\item {Grp. gram.:f.}
\end{itemize}
\begin{itemize}
\item {Proveniência:(De \textunderscore sobarba\textunderscore )}
\end{itemize}
Barbella de corda.
Pancada por baixo da barba.
\section{Sobcapa}
\begin{itemize}
\item {Grp. gram.:f.}
\end{itemize}
\begin{itemize}
\item {Proveniência:(De \textunderscore sob\textunderscore  + \textunderscore capa\textunderscore )}
\end{itemize}
O mesmo que \textunderscore socapa\textunderscore .
\section{Sobcavar}
\begin{itemize}
\item {Grp. gram.:v. t.}
\end{itemize}
\begin{itemize}
\item {Proveniência:(De \textunderscore sob\textunderscore  + \textunderscore cavar\textunderscore )}
\end{itemize}
O mesmo que \textunderscore socavar\textunderscore . Cf. Filinto, XV, 29.
\section{Sobceder}
\begin{itemize}
\item {Grp. gram.:v. i.}
\end{itemize}
\begin{itemize}
\item {Utilização:Ant.}
\end{itemize}
O mesmo que \textunderscore succeder\textunderscore . Cf. \textunderscore Inéd. da Hist. de Port.\textunderscore , I, 519.
\section{Sobcedimento}
\begin{itemize}
\item {Grp. gram.:m.}
\end{itemize}
\begin{itemize}
\item {Utilização:Ant.}
\end{itemize}
O mesmo que \textunderscore succedimento\textunderscore . Cf. R. Pina, \textunderscore Chrón. de Aff. V\textunderscore , CLIV.
\section{Sobcessor}
\begin{itemize}
\item {Grp. gram.:m.}
\end{itemize}
\begin{itemize}
\item {Utilização:Ant.}
\end{itemize}
O mesmo que \textunderscore successor\textunderscore . Cf. S. de Frias, \textunderscore Pombeiro\textunderscore , 199.
\section{Sobcolor}
\begin{itemize}
\item {Grp. gram.:loc. prep.}
\end{itemize}
\begin{itemize}
\item {Proveniência:(De \textunderscore sob\textunderscore  + \textunderscore color\textunderscore )}
\end{itemize}
A pretexto.
Com a apparência.
\section{Sobcor}
\begin{itemize}
\item {Grp. gram.:loc. prep.}
\end{itemize}
O mesmo que \textunderscore sobcolor\textunderscore .
\section{Sobdominante}
\begin{itemize}
\item {Grp. gram.:adj.}
\end{itemize}
\begin{itemize}
\item {Utilização:Mús.}
\end{itemize}
A nota, que é quarto grau na escala diatónica.
\section{Sobeira}
\begin{itemize}
\item {Grp. gram.:f.}
\end{itemize}
\begin{itemize}
\item {Proveniência:(De \textunderscore so...\textunderscore  + \textunderscore beira\textunderscore )}
\end{itemize}
Ordem de telhas que, á beira de um telhado, sustentam e reforçam aquellas por onde corre a água.
\section{Sobejadamente}
\begin{itemize}
\item {Grp. gram.:adv.}
\end{itemize}
\begin{itemize}
\item {Proveniência:(De \textunderscore sobejado\textunderscore )}
\end{itemize}
O mesmo que \textunderscore sobejamente\textunderscore .
\section{Sobejamente}
\begin{itemize}
\item {Grp. gram.:adv.}
\end{itemize}
De modo sobejo, com excesso, com demasia.
\section{Sobejar}
\begin{itemize}
\item {Grp. gram.:v. i.}
\end{itemize}
\begin{itemize}
\item {Grp. gram.:V. p.}
\end{itemize}
Sêr sobejo ou demasiado; exceder o que é preciso; sobrar.
Têr com abundância; têr de sobejo.
\section{Sobejidão}
\begin{itemize}
\item {Grp. gram.:f.}
\end{itemize}
Qualidade do que é sobejo; excesso; grande abundância; pujança; immensidade.
\section{Sobejo}
\begin{itemize}
\item {Grp. gram.:adj.}
\end{itemize}
\begin{itemize}
\item {Grp. gram.:Adv.}
\end{itemize}
\begin{itemize}
\item {Grp. gram.:M. pl.}
\end{itemize}
\begin{itemize}
\item {Proveniência:(Do lat. hyp. \textunderscore superculus\textunderscore )}
\end{itemize}
Que sobeja; excessivo, demasiado.
Innumerável.
Enorme; immenso.
Sobejamente.
Aquillo que sobra ou resta; restos de qualquer coisa.
Excesso.
\section{Sob-emenda}
\begin{itemize}
\item {Grp. gram.:loc. adv.}
\end{itemize}
Com dependência de emenda ou correcção.
Salvo qualquer emenda.
\section{Soberana}
\begin{itemize}
\item {Grp. gram.:f.}
\end{itemize}
\begin{itemize}
\item {Utilização:Fig.}
\end{itemize}
\begin{itemize}
\item {Proveniência:(De \textunderscore soberano\textunderscore )}
\end{itemize}
Mulhér, que exerce o supremo govêrno do um Estado.
Raínha.
Imperatriz.
Mulhér, que entre outras occupa o primeiro lugar.
Qualquer entidade feminina que tem a primazia entre outras: \textunderscore a rosa, a soberana das flôres...\textunderscore 
\section{Soberanamente}
\begin{itemize}
\item {Grp. gram.:adv.}
\end{itemize}
De modo soberano.
Com império, com majestade.
\section{Soberania}
\begin{itemize}
\item {Grp. gram.:f.}
\end{itemize}
Qualidade do que é soberano.
Poder supremo.
Autoridade de Soberano ou Príncipe.
Autoridade moral.
\textunderscore Soberania do povo\textunderscore , doutrina política, que attribue ao povo ou á nação o poder soberano.
\section{Soberanizar}
\begin{itemize}
\item {Grp. gram.:v. t.}
\end{itemize}
\begin{itemize}
\item {Utilização:Fig.}
\end{itemize}
Tornar soberano.
Exaltar.
\section{Soberano}
\begin{itemize}
\item {Grp. gram.:adj.}
\end{itemize}
\begin{itemize}
\item {Utilização:Fig.}
\end{itemize}
\begin{itemize}
\item {Grp. gram.:M.}
\end{itemize}
\begin{itemize}
\item {Utilização:Pop.}
\end{itemize}
\begin{itemize}
\item {Proveniência:(Do b. lat. \textunderscore superanus\textunderscore )}
\end{itemize}
Que occupa o primeiro lugar.
Supremo.
Magnífico.
Absoluto; que tem a autoridade superior.
Dominador.
Altivo.
Que tem grande poder.
Arrogante.
Notável.
Aquelle que na direcção de um Estado tem o govêrno supremo.
Imperante.
O que influe poderosamente.
Libra esterlina.
\section{Soberba}
\begin{itemize}
\item {fónica:bêr}
\end{itemize}
\begin{itemize}
\item {Grp. gram.:f.}
\end{itemize}
\begin{itemize}
\item {Proveniência:(Do lat. \textunderscore superbia\textunderscore )}
\end{itemize}
Elevação.
Altivez; sobrançaria.
Arrogância.
Presumpção; orgulho.
\section{Soberbaço}
\begin{itemize}
\item {Grp. gram.:m.  e  adj.}
\end{itemize}
O mesmo que \textunderscore soberbão\textunderscore .
\section{Soberbamente}
\begin{itemize}
\item {Grp. gram.:adv.}
\end{itemize}
De modo soberbo.
Magnificamente.
\section{Soberbão}
\begin{itemize}
\item {Grp. gram.:m.  e  adj.}
\end{itemize}
O que é muito soberbo.
\section{Soberbete}
\begin{itemize}
\item {fónica:bê}
\end{itemize}
\begin{itemize}
\item {Grp. gram.:m.  e  adj.}
\end{itemize}
O que é um tanto soberbo.
\section{Soberbia}
\begin{itemize}
\item {Grp. gram.:f.}
\end{itemize}
Qualidade do que é soberbo.
Grande soberba.
\section{Soberbo}
\begin{itemize}
\item {fónica:bêr}
\end{itemize}
\begin{itemize}
\item {Grp. gram.:adj.}
\end{itemize}
\begin{itemize}
\item {Grp. gram.:M.}
\end{itemize}
\begin{itemize}
\item {Proveniência:(Lat. \textunderscore superbus\textunderscore )}
\end{itemize}
Que tem soberba.
Altivo; orgulhoso.
Grandioso; magnífico; sublime: \textunderscore um poema soberbo\textunderscore .
Aquelle que é soberbo.
Gênero de aves (\textunderscore paradisoea superba\textunderscore , Lin.).
\section{Soberbosamente}
\begin{itemize}
\item {Grp. gram.:adv.}
\end{itemize}
De modo soberboso.
\section{Soberboso}
\begin{itemize}
\item {Grp. gram.:adj.}
\end{itemize}
\begin{itemize}
\item {Utilização:Pop.}
\end{itemize}
O mesmo que \textunderscore soberbo\textunderscore .
\section{Sobernal}
\begin{itemize}
\item {Grp. gram.:m.}
\end{itemize}
\begin{itemize}
\item {Proveniência:(Lat. hyp. \textunderscore supernalis\textunderscore , de \textunderscore supernus\textunderscore )}
\end{itemize}
Trabalho excessivo; (que os Franceses designam \textunderscore surmenage\textunderscore ). Cf. G. Viana, \textunderscore Apostilas\textunderscore .
\section{Sobessa}
\begin{itemize}
\item {Grp. gram.:f.}
\end{itemize}
\begin{itemize}
\item {Utilização:Prov.}
\end{itemize}
\begin{itemize}
\item {Utilização:trasm.}
\end{itemize}
\begin{itemize}
\item {Proveniência:(De \textunderscore sob\textunderscore  + \textunderscore essa\textunderscore , pron.?)}
\end{itemize}
A parte inferior, relativamente a um lugar ou a um objecto.
O lado de baixo.
Posição inferior de uma coisa ou pessôa, relativamente a um ponto qualquer.
\section{Sobestar}
\begin{itemize}
\item {Grp. gram.:v. i.}
\end{itemize}
\begin{itemize}
\item {Proveniência:(De \textunderscore sob\textunderscore  + \textunderscore estar\textunderscore )}
\end{itemize}
Estar abaixo de; sêr inferior a.
\section{Sobeta}
\begin{itemize}
\item {fónica:bê}
\end{itemize}
\begin{itemize}
\item {Grp. gram.:m.}
\end{itemize}
Pequeno soba.
Soba, que tem escassos domínios. Cf. Capello e Ivens, II, 38.
\section{Sobeu}
\begin{itemize}
\item {Grp. gram.:m.}
\end{itemize}
\begin{itemize}
\item {Utilização:Prov.}
\end{itemize}
\begin{itemize}
\item {Utilização:trasm.}
\end{itemize}
Correia forte, com que se prende a cabeçalha do carro ao jugo dos bois.
\section{Sobgrave}
\begin{itemize}
\item {Grp. gram.:adj.}
\end{itemize}
\begin{itemize}
\item {Proveniência:(De \textunderscore sob\textunderscore  + \textunderscore grave\textunderscore )}
\end{itemize}
Inferior ao grave, na música.
\section{Sobiote}
\begin{itemize}
\item {Grp. gram.:m.}
\end{itemize}
\begin{itemize}
\item {Utilização:Prov.}
\end{itemize}
\begin{itemize}
\item {Utilização:trasm.}
\end{itemize}
Assobio pequeno; apito de metal ou madeira.
(Por \textunderscore assobiote\textunderscore , de \textunderscore assobio\textunderscore )
\section{Sob-ligação}
\begin{itemize}
\item {Grp. gram.:loc. adv.}
\end{itemize}
\begin{itemize}
\item {Utilização:Ant.}
\end{itemize}
Debaixo de obrigação; obrigatoriamente.
\section{Sob-mediante}
\begin{itemize}
\item {Grp. gram.:f.}
\end{itemize}
\begin{itemize}
\item {Utilização:Mús.}
\end{itemize}
O mesmo que \textunderscore sobretónica\textunderscore .
\section{Sobnegar}
\begin{itemize}
\item {Grp. gram.:v. t.}
\end{itemize}
O mesmo que \textunderscore sonegar\textunderscore . Cf. Filinto, \textunderscore D. Man.\textunderscore , II, 252.
\section{Sôbola}
\begin{itemize}
\item {Utilização:Ant.}
\end{itemize}
(Contr. de \textunderscore sôbre\textunderscore  + \textunderscore lo\textunderscore , \textunderscore la\textunderscore )
\textunderscore Sôbola tarde\textunderscore , sôbre a tarde; á tardinha.
\section{Sóbole}
\begin{itemize}
\item {Grp. gram.:m.}
\end{itemize}
\begin{itemize}
\item {Proveniência:(Lat. \textunderscore soboles\textunderscore )}
\end{itemize}
Geração; prole.
Raça.
Gomo, rebento.
\section{Sôbolo}
\begin{itemize}
\item {Utilização:Ant.}
\end{itemize}
(Contr. de \textunderscore sôbre\textunderscore  + \textunderscore lo\textunderscore , \textunderscore la\textunderscore )
\textunderscore Sôbolo tarde\textunderscore , sôbre a tarde; á tardinha.
\section{Soborda}
\begin{itemize}
\item {Grp. gram.:f.}
\end{itemize}
\begin{itemize}
\item {Proveniência:(De \textunderscore so...\textunderscore  + \textunderscore borda\textunderscore )}
\end{itemize}
A parte immediatamente inferior á borda, num navio:«\textunderscore cingirão c'hum nastro de bronze o vaso pela soborda...\textunderscore »Filinto, \textunderscore D. Man.\textunderscore , I, 78.
\section{Soborralhadoiro}
\begin{itemize}
\item {Grp. gram.:m.}
\end{itemize}
\begin{itemize}
\item {Proveniência:(De \textunderscore soborralhar\textunderscore )}
\end{itemize}
Vassoiro ou varredoiro de forno.
\section{Soborralhadouro}
\begin{itemize}
\item {Grp. gram.:m.}
\end{itemize}
\begin{itemize}
\item {Proveniência:(De \textunderscore soborralhar\textunderscore )}
\end{itemize}
Vassoiro ou varredouro de forno.
\section{Soborralhar}
\begin{itemize}
\item {Grp. gram.:v. t.}
\end{itemize}
\begin{itemize}
\item {Proveniência:(De \textunderscore soborralho\textunderscore )}
\end{itemize}
Meter no borralho.
\section{Soborralho}
\begin{itemize}
\item {Grp. gram.:m.}
\end{itemize}
\begin{itemize}
\item {Proveniência:(De \textunderscore so...\textunderscore  + \textunderscore borralho\textunderscore )}
\end{itemize}
Calor, mantido pelo borralho.
Brasas ou outra coisa que está sob o borralho.
\section{Sob-pé}
\begin{itemize}
\item {Grp. gram.:m.}
\end{itemize}
(V.sopé)
\section{Sobpear}
\begin{itemize}
\item {Grp. gram.:v. t.}
\end{itemize}
\begin{itemize}
\item {Utilização:Ant.}
\end{itemize}
O mesmo que \textunderscore sopear\textunderscore . Cf. Usque, 46, v.^o.
\section{Sob-pena}
\begin{itemize}
\item {Grp. gram.:loc. adv.}
\end{itemize}
Incorrendo na pena; expondo-se ás consequências.
\section{Sobpor}
\begin{itemize}
\item {Grp. gram.:v. t.}
\end{itemize}
\begin{itemize}
\item {Utilização:Fig.}
\end{itemize}
\begin{itemize}
\item {Proveniência:(De \textunderscore sob\textunderscore  + \textunderscore pôr\textunderscore )}
\end{itemize}
Pôr debaixo.
Têr em menos conta, menosprezar:«\textunderscore sobpondo ao reconhecimento os exemplos do espião...\textunderscore »Camillo, \textunderscore Brasileira\textunderscore , 227.
\section{Sobra}
\begin{itemize}
\item {Grp. gram.:f.}
\end{itemize}
\begin{itemize}
\item {Grp. gram.:Loc. adv.}
\end{itemize}
Acto ou effeito de sobrar.
Resto; sobejos.
\textunderscore De sobra\textunderscore , sobejamente; superabundantemente; com excesso; excessivamente.
\section{Sobraçar}
\begin{itemize}
\item {Grp. gram.:v. t.}
\end{itemize}
\begin{itemize}
\item {Utilização:Fig.}
\end{itemize}
\begin{itemize}
\item {Grp. gram.:V. p.}
\end{itemize}
\begin{itemize}
\item {Proveniência:(De \textunderscore so...\textunderscore  + \textunderscore braço\textunderscore )}
\end{itemize}
Meter debaixo do braço.
Segurar com o braço.
Segurar, amparar.
Levar em braços.
Dar o braço a outrem.
\section{Sobradamente}
\begin{itemize}
\item {Grp. gram.:adv.}
\end{itemize}
\begin{itemize}
\item {Proveniência:(De \textunderscore sobrado\textunderscore ^1)}
\end{itemize}
De sobra.
\section{Sobradar}
\begin{itemize}
\item {Grp. gram.:v. t.}
\end{itemize}
Fazer sobrado^2 em: \textunderscore sobradar uma sala\textunderscore .
\section{Sobradiz}
\begin{itemize}
\item {Grp. gram.:m.}
\end{itemize}
\begin{itemize}
\item {Utilização:Prov.}
\end{itemize}
Cada um dos madeiros, em que se pregam as tábuas do sobrado, e cujas extremidades assentam nas vigas. (Colhido em Arganil)
\section{Sobrado}
\begin{itemize}
\item {Grp. gram.:adj.}
\end{itemize}
\begin{itemize}
\item {Proveniência:(De \textunderscore sobrar\textunderscore )}
\end{itemize}
Que sobrou ou que sobra; demasiado.
Farto; abastado.
\section{Sobrado}
\begin{itemize}
\item {Grp. gram.:m.}
\end{itemize}
Pavimento de madeira.
(Inclino-me a que a palavra se relacione com o lat. \textunderscore superare\textunderscore , estar acima, visto que \textunderscore sobrado\textunderscore  é geralmente um pavimento superior ao pavimento térreo de um edifício)
\section{Sobraínho}
\begin{itemize}
\item {Grp. gram.:m.}
\end{itemize}
Casta de uva extremenha.
\section{Sobral}
\begin{itemize}
\item {Grp. gram.:m.}
\end{itemize}
\begin{itemize}
\item {Proveniência:(De \textunderscore sôbro\textunderscore )}
\end{itemize}
Lugar, onde crescem sobreiros.
\section{Sobrançar}
\begin{itemize}
\item {Grp. gram.:v. t.}
\end{itemize}
O mesmo que \textunderscore sobrancear\textunderscore . Cf. Filinto, I, 372; II, 139.
\section{Sobrançaria}
\begin{itemize}
\item {Grp. gram.:f.}
\end{itemize}
O mesmo ou melhor que \textunderscore sobranceria\textunderscore . Cf. \textunderscore Ethiop. Or.\textunderscore , II, 324.
\section{Sobrancear}
\begin{itemize}
\item {Grp. gram.:v. i.}
\end{itemize}
\begin{itemize}
\item {Proveniência:(Do lat. \textunderscore superans\textunderscore )}
\end{itemize}
Estar sobranceiro a; exceder. Cf. Camillo, \textunderscore Quéda\textunderscore , 74.
Pôr em cima:«\textunderscore ...no brasão soberbo, que sobranceou ao vasto portão.\textunderscore »Camillo, \textunderscore Mulhér Fatal\textunderscore , 141.
\section{Sobranceiramente}
\begin{itemize}
\item {Grp. gram.:adv.}
\end{itemize}
De modo sobranceiro.
Em lugar elevado.
Orgulhosamente, altivamente; com desdém.
\section{Sobranceiro}
\begin{itemize}
\item {Grp. gram.:adj.}
\end{itemize}
\begin{itemize}
\item {Utilização:Fig.}
\end{itemize}
\begin{itemize}
\item {Grp. gram.:Adv.}
\end{itemize}
\begin{itemize}
\item {Proveniência:(De \textunderscore sobrancear\textunderscore )}
\end{itemize}
Que occupa lugar superior.
Que tem situação elevada.
Proeminente.
Superior.
Que vê de alto.
Orgulhoso.
Altivo; arrogante: \textunderscore olhar sobranceiro\textunderscore .
Animoso.
Que sobresai.
Com sobrançaria; em lugar elevado.
\section{Sobrancelha}
\begin{itemize}
\item {fónica:cê}
\end{itemize}
\begin{itemize}
\item {Grp. gram.:f.}
\end{itemize}
\begin{itemize}
\item {Proveniência:(Do lat. \textunderscore supercilium\textunderscore )}
\end{itemize}
Reunião de pêlos, que se arqueia na parte superior das órbitas oculares.
\section{Sobrancelhudo}
\begin{itemize}
\item {Grp. gram.:adj.}
\end{itemize}
Que tem grandes sobrancelhas; carrancudo.
\section{Sobranceria}
\begin{itemize}
\item {Grp. gram.:f.}
\end{itemize}
Qualidade do que é sobranceiro.
Acto ou modos de sobranceiro.
Altivez; orgulho; desdém.
\section{Sobrar}
\begin{itemize}
\item {Grp. gram.:v. i.}
\end{itemize}
\begin{itemize}
\item {Proveniência:(Do lat. \textunderscore superare\textunderscore )}
\end{itemize}
Estar superior a outro.
Estar sobranceiro.
Sobejar.
Sêr mais que sufficiente.
Constituir os restos de alguma coisa.
\section{Sobrasar}
\begin{itemize}
\item {Grp. gram.:v. t.}
\end{itemize}
\begin{itemize}
\item {Proveniência:(De \textunderscore so...\textunderscore  + \textunderscore brasa\textunderscore )}
\end{itemize}
Pôr brasas debaixo de.
\section{Sôbre}
\begin{itemize}
\item {Grp. gram.:prep.}
\end{itemize}
\begin{itemize}
\item {Grp. gram.:M.}
\end{itemize}
\begin{itemize}
\item {Utilização:Náut.}
\end{itemize}
\begin{itemize}
\item {Proveniência:(Do lat. \textunderscore super\textunderscore )}
\end{itemize}
Na parte superior de.
Para cima de.
Em cima de.
Pela superfície de.
Ao pé de, próximo de.
De encontro a.
Para o lado de.
Além de: \textunderscore sôbre sêr patife, é tolo\textunderscore .
Ácêrca de.
Por causa de.
Por detrás de.
Atrás de.
Depois de.
Em consequência de.
Conforme: \textunderscore foi nomeado sôbre proposta do Director\textunderscore .
Em seguida a.
Proporcionalmente.
Em nome de: \textunderscore affirmou sôbre a sua honra...\textunderscore 
Contra.
Entre.
Qualquer das velas mais altas de um navio.
Qualquer das velas, cuja designação começa por \textunderscore sôbre\textunderscore .
\section{Sobreabundar}
\textunderscore v. i.\textunderscore  (e der.)
O mesmo que \textunderscore superabundar\textunderscore , etc.
\section{Sobreagitar}
\begin{itemize}
\item {Grp. gram.:v. t.}
\end{itemize}
\begin{itemize}
\item {Proveniência:(De \textunderscore sôbre\textunderscore  + \textunderscore agitar\textunderscore )}
\end{itemize}
Agitar muito; inquietar profundamente.
\section{Sobreaguado}
\begin{itemize}
\item {Grp. gram.:adj.}
\end{itemize}
\begin{itemize}
\item {Proveniência:(De \textunderscore sôbre\textunderscore  + \textunderscore aguado\textunderscore )}
\end{itemize}
Coberto de água; alagado.
\section{Sobreagudo}
\begin{itemize}
\item {Grp. gram.:adj.}
\end{itemize}
Muito agudo. Cf. \textunderscore Jorn.-do-Comm.\textunderscore , do Rio, de 1-I-905.
\section{Sobrealcunha}
\begin{itemize}
\item {Grp. gram.:f.}
\end{itemize}
\begin{itemize}
\item {Proveniência:(De \textunderscore sôbre\textunderscore  + \textunderscore alcunha\textunderscore )}
\end{itemize}
Segunda alcunha.
Alcunha em seguida a outra.
\section{Sobreanca}
\begin{itemize}
\item {Grp. gram.:f.}
\end{itemize}
\begin{itemize}
\item {Proveniência:(De \textunderscore sôbre\textunderscore  + \textunderscore anca\textunderscore )}
\end{itemize}
O mesmo que \textunderscore xairel\textunderscore .
\section{Sobreapelido}
\begin{itemize}
\item {Grp. gram.:m.}
\end{itemize}
\begin{itemize}
\item {Proveniência:(De \textunderscore sobre\textunderscore  + \textunderscore apelido\textunderscore )}
\end{itemize}
Segundo apelido.
Apelido, junto a outro.
\section{Sobreappellido}
\begin{itemize}
\item {Grp. gram.:m.}
\end{itemize}
\begin{itemize}
\item {Proveniência:(De \textunderscore sobre\textunderscore  + \textunderscore appellido\textunderscore )}
\end{itemize}
Segundo appellido.
Appellido, junto a outro.
\section{Sobrearco}
\begin{itemize}
\item {Grp. gram.:m.}
\end{itemize}
\begin{itemize}
\item {Proveniência:(De \textunderscore sôbre\textunderscore  + \textunderscore arco\textunderscore )}
\end{itemize}
Vêrga da porta.
Tórça.
Padieira.
\section{Sobreaviso}
\begin{itemize}
\item {Grp. gram.:m.}
\end{itemize}
\begin{itemize}
\item {Grp. gram.:Adj.}
\end{itemize}
\begin{itemize}
\item {Proveniência:(De \textunderscore sôbre\textunderscore  + \textunderscore aviso\textunderscore )}
\end{itemize}
Precaução, prevenção.
Prevenido, acautelado:«\textunderscore ...o Gama estava sobreaviso...\textunderscore »Filinto, \textunderscore D. Man.\textunderscore , I, 112.
\section{Sobreaxilar}
\begin{itemize}
\item {Grp. gram.:adj.}
\end{itemize}
\begin{itemize}
\item {Utilização:Bot.}
\end{itemize}
\begin{itemize}
\item {Proveniência:(De \textunderscore sôbre\textunderscore  + \textunderscore axilar\textunderscore )}
\end{itemize}
Que está por cima da axila; sobrefolheáceo.
\section{Sobreaxillar}
\begin{itemize}
\item {Grp. gram.:adj.}
\end{itemize}
\begin{itemize}
\item {Utilização:Bot.}
\end{itemize}
\begin{itemize}
\item {Proveniência:(De \textunderscore sôbre\textunderscore  + \textunderscore axillar\textunderscore )}
\end{itemize}
Que está por cima da axilla; sobrefolheáceo.
\section{Sobrebailéo}
\begin{itemize}
\item {Grp. gram.:m.}
\end{itemize}
\begin{itemize}
\item {Proveniência:(De \textunderscore sôbre\textunderscore  + \textunderscore bailéu\textunderscore )}
\end{itemize}
Bailéu, que fica por cima de outro.
\section{Sobrebailéu}
\begin{itemize}
\item {Grp. gram.:m.}
\end{itemize}
\begin{itemize}
\item {Proveniência:(De \textunderscore sôbre\textunderscore  + \textunderscore bailéu\textunderscore )}
\end{itemize}
Bailéu, que fica por cima de outro.
\section{Sobrebaínha}
\begin{itemize}
\item {Grp. gram.:f.}
\end{itemize}
\begin{itemize}
\item {Proveniência:(De \textunderscore sôbre\textunderscore  + \textunderscore baínha\textunderscore )}
\end{itemize}
Fôrro externo da baínha.
\section{Sobrebico}
\begin{itemize}
\item {Grp. gram.:m.}
\end{itemize}
\begin{itemize}
\item {Utilização:Zool.}
\end{itemize}
\begin{itemize}
\item {Proveniência:(De \textunderscore sôbre\textunderscore  + \textunderscore bico\textunderscore )}
\end{itemize}
Parte superior do bico das aves.
\section{Sobrebrocha}
\begin{itemize}
\item {Grp. gram.:f.}
\end{itemize}
\begin{itemize}
\item {Proveniência:(De \textunderscore sôbre\textunderscore  + \textunderscore brocha\textunderscore )}
\end{itemize}
Grande correia, ligada ás brochas, em carros de bois.
\section{Sobrecabado}
\begin{itemize}
\item {Grp. gram.:adj.}
\end{itemize}
\begin{itemize}
\item {Utilização:P. us.}
\end{itemize}
\begin{itemize}
\item {Proveniência:(De \textunderscore sôbre\textunderscore  + \textunderscore cabo\textunderscore )}
\end{itemize}
Que está em lugar alto; eminente.
\section{Sobrecabar}
\begin{itemize}
\item {Grp. gram.:v. i.}
\end{itemize}
\begin{itemize}
\item {Utilização:Ant.}
\end{itemize}
\begin{itemize}
\item {Proveniência:(Do lat. \textunderscore super\textunderscore  + \textunderscore caput\textunderscore )}
\end{itemize}
Responsabilizar-se moralmente por alguém. Cf. Herculano, \textunderscore Hist. de Port.\textunderscore , IV, 352.
\section{Sobrecabeça}
\begin{itemize}
\item {fónica:bê}
\end{itemize}
\begin{itemize}
\item {Grp. gram.:f.}
\end{itemize}
\begin{itemize}
\item {Proveniência:(De \textunderscore sôbre\textunderscore  + \textunderscore cabeça\textunderscore )}
\end{itemize}
Porção de metal, que resai das bocas de fogo depois de fundidas, e que se corta quando ellas se aperfeiçôam.
\section{Sôbre-cabeceira}
\begin{itemize}
\item {Grp. gram.:f.}
\end{itemize}
Um dos compartimentos das salinas. Cf. \textunderscore Museu Tecnol.\textunderscore , 57.
\section{Sobrecadeia}
\begin{itemize}
\item {Grp. gram.:f.}
\end{itemize}
\begin{itemize}
\item {Proveniência:(De \textunderscore sôbre\textunderscore  + \textunderscore cadeia\textunderscore )}
\end{itemize}
Peça de madeira, pregada transversalmente na extremidade do leito do carro, para reforçar o caixilho do mesmo leito.
\section{Sôbre-cama}
\begin{itemize}
\item {Grp. gram.:f.}
\end{itemize}
\begin{itemize}
\item {Utilização:Ant.}
\end{itemize}
Colcha de cama.
\section{Sôbre-câmara}
\begin{itemize}
\item {Grp. gram.:f.}
\end{itemize}
\begin{itemize}
\item {Utilização:Prov.}
\end{itemize}
\begin{itemize}
\item {Utilização:alent.}
\end{itemize}
\begin{itemize}
\item {Proveniência:(De \textunderscore sôbre\textunderscore  + \textunderscore câmara\textunderscore )}
\end{itemize}
Águas-furtadas, sótão.
\section{Sobrecana}
\begin{itemize}
\item {Grp. gram.:f.}
\end{itemize}
\begin{itemize}
\item {Proveniência:(De \textunderscore sôbre\textunderscore  + \textunderscore cana\textunderscore )}
\end{itemize}
Tumor duro, nos membros anteriores do cavallo.
\section{Sobrecarga}
\begin{itemize}
\item {Grp. gram.:f.}
\end{itemize}
\begin{itemize}
\item {Grp. gram.:M.}
\end{itemize}
\begin{itemize}
\item {Proveniência:(De \textunderscore sôbre\textunderscore  + \textunderscore carga\textunderscore )}
\end{itemize}
Carga demasiada.
Aquillo que se junta á carga.
Aquillo que perturba o equilíbrio da carga.
Espécie de cilha, com que se aperta a carga da bêsta.
Marca, que as repartições postaes põem nas estampilhas da correspondência.
Aquelle que dirige o carregamento de um navio.
\section{Sobrecarregar}
\begin{itemize}
\item {Grp. gram.:v. t.}
\end{itemize}
\begin{itemize}
\item {Proveniência:(De \textunderscore sôbre\textunderscore  + \textunderscore carregar\textunderscore )}
\end{itemize}
Carregar demasiadamente.
Aumentar excessivamente.
Causar vexame a; aumentar encargos a.
\section{Sobrecarta}
\begin{itemize}
\item {Grp. gram.:f.}
\end{itemize}
\begin{itemize}
\item {Proveniência:(De \textunderscore sôbre\textunderscore  + \textunderscore carta\textunderscore )}
\end{itemize}
Segunda carta.
Carta, seguida a outra, com que tem relação.
O mesmo que \textunderscore sobrescrito\textunderscore . Cf. M. Assis, \textunderscore Mão e Luva\textunderscore , c. X.
\section{Sobrecasaca}
\begin{itemize}
\item {Grp. gram.:f.}
\end{itemize}
\begin{itemize}
\item {Proveniência:(De \textunderscore sôbre\textunderscore  + \textunderscore casaca\textunderscore )}
\end{itemize}
Casaco largo que se póde vestir sôbre outro.
Casaco comprido, cujas abas fórmam roda e cáem perpendicularmente na deanteira.
\section{Sobrecelente}
\begin{itemize}
\item {Grp. gram.:adj.}
\end{itemize}
\begin{itemize}
\item {Grp. gram.:M.  e  adj.}
\end{itemize}
\begin{itemize}
\item {Proveniência:(De \textunderscore sôbre\textunderscore  + \textunderscore excelente\textunderscore , se não é corr. de \textunderscore sobresalente\textunderscore )}
\end{itemize}
O mesmo que \textunderscore excedente\textunderscore .
Tudo que sobeja e é próprio para suprir faltas.
\section{Sobreceleste}
\begin{itemize}
\item {Grp. gram.:adj.}
\end{itemize}
\begin{itemize}
\item {Proveniência:(De \textunderscore sôbre\textunderscore  + \textunderscore celeste\textunderscore )}
\end{itemize}
Mais que celeste; divino.
\section{Sobrecelestial}
\begin{itemize}
\item {Grp. gram.:adj.}
\end{itemize}
\begin{itemize}
\item {Proveniência:(De \textunderscore sôbre\textunderscore  + \textunderscore celestial\textunderscore )}
\end{itemize}
O mesmo que \textunderscore sobreceleste\textunderscore .
\section{Sobrecelestialmente}
\begin{itemize}
\item {Grp. gram.:adv.}
\end{itemize}
De modo sobrecelestial.
\section{Sobrecellente}
\begin{itemize}
\item {Grp. gram.:adj.}
\end{itemize}
\begin{itemize}
\item {Grp. gram.:M.  e  adj.}
\end{itemize}
\begin{itemize}
\item {Proveniência:(De \textunderscore sôbre\textunderscore  + \textunderscore excellente\textunderscore , se não é corr. de \textunderscore sobresalente\textunderscore )}
\end{itemize}
O mesmo que \textunderscore excedente\textunderscore .
Tudo que sobeja e é próprio para supprir faltas.
\section{Sobrecenho}
\begin{itemize}
\item {Grp. gram.:m.}
\end{itemize}
\begin{itemize}
\item {Proveniência:(De \textunderscore sôbre\textunderscore  + \textunderscore cenho\textunderscore )}
\end{itemize}
Sobrançaria.
Semblante torvo, carregado; carranca.
\section{Sobrecéo}
\begin{itemize}
\item {Grp. gram.:m.}
\end{itemize}
\begin{itemize}
\item {Proveniência:(De \textunderscore sôbre\textunderscore  + \textunderscore céu\textunderscore )}
\end{itemize}
Cobertura, suspensa por cima de um leito ou de um pavilhão; dossel.
\section{Sobrecéu}
\begin{itemize}
\item {Grp. gram.:m.}
\end{itemize}
\begin{itemize}
\item {Proveniência:(De \textunderscore sôbre\textunderscore  + \textunderscore céu\textunderscore )}
\end{itemize}
Cobertura, suspensa por cima de um leito ou de um pavilhão; dossel.
\section{Sobrecevadeira}
\begin{itemize}
\item {Grp. gram.:f.}
\end{itemize}
\begin{itemize}
\item {Proveniência:(De \textunderscore sôbre\textunderscore  + \textunderscore cevadeira\textunderscore )}
\end{itemize}
Pequena vela de navio, sôbre a cevadeira.
\section{Sobrechegar}
\begin{itemize}
\item {Grp. gram.:v. i.}
\end{itemize}
\begin{itemize}
\item {Proveniência:(De \textunderscore sobre\textunderscore  + \textunderscore chegar\textunderscore )}
\end{itemize}
O mesmo que \textunderscore sobrevir\textunderscore .
\section{Sobrecheio}
\begin{itemize}
\item {Grp. gram.:adj.}
\end{itemize}
\begin{itemize}
\item {Proveniência:(De \textunderscore sôbre\textunderscore  + \textunderscore cheio\textunderscore )}
\end{itemize}
Muito cheio; acogulado.
\section{Sobrecincha}
\begin{itemize}
\item {Grp. gram.:f.}
\end{itemize}
\begin{itemize}
\item {Utilização:Bras. do S}
\end{itemize}
Tira de coiro, que se aperta por cima do coxinilho.
\section{Sobreclaustra}
\begin{itemize}
\item {Grp. gram.:f.}
\end{itemize}
O mesmo que \textunderscore sobreclaustro\textunderscore .
\section{Sobreclaustro}
\begin{itemize}
\item {Grp. gram.:m.}
\end{itemize}
\begin{itemize}
\item {Proveniência:(De \textunderscore sôbre\textunderscore  + \textunderscore claustro\textunderscore )}
\end{itemize}
Claustro superior.
\section{Sobrecoberta}
\begin{itemize}
\item {Grp. gram.:f.}
\end{itemize}
\begin{itemize}
\item {Proveniência:(De \textunderscore sôbre\textunderscore  + \textunderscore coberta\textunderscore )}
\end{itemize}
Coberta construída acima de outra.
\section{Sobrecopa}
\begin{itemize}
\item {Grp. gram.:f.}
\end{itemize}
\begin{itemize}
\item {Proveniência:(De \textunderscore sôbre\textunderscore  + \textunderscore copa\textunderscore )}
\end{itemize}
O mesmo que \textunderscore tampa\textunderscore .
\section{Sobrecommum}
\begin{itemize}
\item {Grp. gram.:adj.}
\end{itemize}
\begin{itemize}
\item {Utilização:Gram.}
\end{itemize}
\begin{itemize}
\item {Proveniência:(De \textunderscore sôbre\textunderscore  + \textunderscore commum\textunderscore )}
\end{itemize}
Diz-se do substantivo, que não tem duas flexões para os dois gêneros: \textunderscore criança\textunderscore , \textunderscore algoz\textunderscore , \textunderscore guia\textunderscore . Cf. João Ribeiro, \textunderscore Diccion. Gramm.\textunderscore 
\section{Sobrecomposto}
\begin{itemize}
\item {Grp. gram.:adj.}
\end{itemize}
\begin{itemize}
\item {Utilização:Bot.}
\end{itemize}
Diz-se das fôlhas, cujo pecíolo commum se divide mais de duas vezes em pecíolos menores, como succede na espírea.
\section{Sobrecomum}
\begin{itemize}
\item {Grp. gram.:adj.}
\end{itemize}
\begin{itemize}
\item {Utilização:Gram.}
\end{itemize}
\begin{itemize}
\item {Proveniência:(De \textunderscore sôbre\textunderscore  + \textunderscore comum\textunderscore )}
\end{itemize}
Diz-se do substantivo, que não tem duas flexões para os dois gêneros: \textunderscore criança\textunderscore , \textunderscore algoz\textunderscore , \textunderscore guia\textunderscore . Cf. João Ribeiro, \textunderscore Diccion. Gramm.\textunderscore 
\section{Sobrecoser}
\begin{itemize}
\item {Grp. gram.:v. t.}
\end{itemize}
\begin{itemize}
\item {Proveniência:(De \textunderscore sôbre\textunderscore  + \textunderscore coser\textunderscore )}
\end{itemize}
Fazer sobrecostura em.
\section{Sobrecostelar}
\begin{itemize}
\item {Grp. gram.:m.}
\end{itemize}
\begin{itemize}
\item {Utilização:Bras}
\end{itemize}
\begin{itemize}
\item {Proveniência:(De \textunderscore sôbre\textunderscore  + \textunderscore costela\textunderscore )}
\end{itemize}
Porção de carne, que se tira de cima das costelas da rês.
\section{Sobrecostellar}
\begin{itemize}
\item {Grp. gram.:m.}
\end{itemize}
\begin{itemize}
\item {Utilização:Bras}
\end{itemize}
\begin{itemize}
\item {Proveniência:(De \textunderscore sôbre\textunderscore  + \textunderscore costella\textunderscore )}
\end{itemize}
Porção de carne, que se tira de cima das costellas da rês.
\section{Sobrecostura}
\begin{itemize}
\item {Grp. gram.:f.}
\end{itemize}
\begin{itemize}
\item {Proveniência:(De \textunderscore sôbre...\textunderscore  + \textunderscore costura\textunderscore )}
\end{itemize}
Costura sôbre duas peças, já cosidas uma á outra.
\section{Sobrecrescer}
\begin{itemize}
\item {Grp. gram.:v. i.}
\end{itemize}
O mesmo que \textunderscore accrescer\textunderscore . Cf. Castilho, \textunderscore Livrar. Cláss.\textunderscore , VII, 93; Latino, \textunderscore Camões\textunderscore , 340.
\section{Sobrecu}
\begin{itemize}
\item {Grp. gram.:m.}
\end{itemize}
\begin{itemize}
\item {Proveniência:(De \textunderscore sôbre\textunderscore  + \textunderscore cu\textunderscore )}
\end{itemize}
O mesmo que \textunderscore uropýgio\textunderscore .
\section{Sobrecurva}
\begin{itemize}
\item {Grp. gram.:f.}
\end{itemize}
\begin{itemize}
\item {Proveniência:(De \textunderscore sôbre\textunderscore  + \textunderscore curva\textunderscore )}
\end{itemize}
Tumor duro, na curva do jarrete da cavalgadura.
\section{Sobredental}
\begin{itemize}
\item {Grp. gram.:adj.}
\end{itemize}
\begin{itemize}
\item {Utilização:Anat.}
\end{itemize}
\begin{itemize}
\item {Proveniência:(De \textunderscore sôbre\textunderscore  + \textunderscore dental\textunderscore )}
\end{itemize}
Que está sôbre os dentes ou acima dêlles.
\section{Sobredente}
\begin{itemize}
\item {Grp. gram.:m.}
\end{itemize}
\begin{itemize}
\item {Proveniência:(De \textunderscore sôbre\textunderscore  + \textunderscore dente\textunderscore )}
\end{itemize}
Dente, que nasceu encostado a outro.
\section{Sobredito}
\begin{itemize}
\item {Grp. gram.:adj.}
\end{itemize}
\begin{itemize}
\item {Proveniência:(De \textunderscore sôbre\textunderscore  + \textunderscore dito\textunderscore )}
\end{itemize}
Dito acima ou antes; já mencionado; alludido.
\section{Sobredivino}
\begin{itemize}
\item {Grp. gram.:adj.}
\end{itemize}
\begin{itemize}
\item {Proveniência:(De \textunderscore sôbre\textunderscore  + \textunderscore divino\textunderscore )}
\end{itemize}
Mais que divino.
\section{Sobredoirado}
\begin{itemize}
\item {Grp. gram.:m.}
\end{itemize}
\begin{itemize}
\item {Proveniência:(De \textunderscore sobredoirar\textunderscore )}
\end{itemize}
Obra doirada.
\section{Sobredoirar}
\begin{itemize}
\item {Grp. gram.:v. t.}
\end{itemize}
\begin{itemize}
\item {Utilização:Fig.}
\end{itemize}
\begin{itemize}
\item {Proveniência:(De \textunderscore sôbre\textunderscore  + \textunderscore doirar\textunderscore )}
\end{itemize}
Doirar superiormente.
Exornar; engrandecer.
Colorir com artifício, para induzir em êrro.
Illuminar as partes mais elevadas de, (falando-se do sol, etc.).
\section{Sôbre-dominante}
\begin{itemize}
\item {Grp. gram.:f.}
\end{itemize}
\begin{itemize}
\item {Utilização:Mús.}
\end{itemize}
Sexto grau da escala diatónica.
\section{Sobredourado}
\begin{itemize}
\item {Grp. gram.:m.}
\end{itemize}
\begin{itemize}
\item {Proveniência:(De \textunderscore sobredourar\textunderscore )}
\end{itemize}
Obra dourada.
\section{Sobredourar}
\begin{itemize}
\item {Grp. gram.:v. t.}
\end{itemize}
\begin{itemize}
\item {Utilização:Fig.}
\end{itemize}
\begin{itemize}
\item {Proveniência:(De \textunderscore sôbre\textunderscore  + \textunderscore dourar\textunderscore )}
\end{itemize}
Dourar superiormente.
Exornar; engrandecer.
Colorir com artifício, para induzir em êrro.
Illuminar as partes mais elevadas de, (falando-se do sol, etc.).
\section{Sobreeminência}
\begin{itemize}
\item {Grp. gram.:f.}
\end{itemize}
\begin{itemize}
\item {Proveniência:(De \textunderscore sôbre\textunderscore  + \textunderscore eminência\textunderscore )}
\end{itemize}
Qualidade do que é sobreeminente.
\section{Sobreeminente}
\begin{itemize}
\item {Grp. gram.:adj.}
\end{itemize}
\begin{itemize}
\item {Proveniência:(De \textunderscore sôbre\textunderscore  + \textunderscore eminente\textunderscore )}
\end{itemize}
Muito elevado; magnifico.
\section{Sobreerguer}
\begin{itemize}
\item {Grp. gram.:v. t.}
\end{itemize}
\begin{itemize}
\item {Proveniência:(De \textunderscore sôbre\textunderscore  + \textunderscore erguer\textunderscore )}
\end{itemize}
Erguer mais alto que outra coisa.
\section{Sobreescritar}
\begin{itemize}
\item {Grp. gram.:v. t.}
\end{itemize}
O mesmo que \textunderscore sobrescritar\textunderscore . Cf. Latino, \textunderscore Or. da Corôa\textunderscore , CLXXII.
\section{Sobreestar}
\begin{itemize}
\item {Grp. gram.:v. i.}
\end{itemize}
O mesmo que \textunderscore sobrestar\textunderscore .
\section{Sobreexaltar}
\begin{itemize}
\item {Grp. gram.:v. i.}
\end{itemize}
\begin{itemize}
\item {Proveniência:(De \textunderscore sôbre\textunderscore  + \textunderscore exaltar\textunderscore )}
\end{itemize}
Exaltar excessivamente.
\section{Sobreexcedente}
\begin{itemize}
\item {Grp. gram.:adj.}
\end{itemize}
\begin{itemize}
\item {Proveniência:(De \textunderscore sôbre\textunderscore  + \textunderscore excedente\textunderscore )}
\end{itemize}
Que sobreexcede.
\section{Sobreexceder}
\begin{itemize}
\item {Grp. gram.:v. t.}
\end{itemize}
\begin{itemize}
\item {Grp. gram.:V. i.}
\end{itemize}
\begin{itemize}
\item {Proveniência:(De \textunderscore sôbre\textunderscore  + \textunderscore exceder\textunderscore )}
\end{itemize}
Exceder muito; ultrapassar.
Ir muito além, avantajar-se.
\section{Sobreexcelência}
\begin{itemize}
\item {Grp. gram.:f.}
\end{itemize}
\begin{itemize}
\item {Proveniência:(De \textunderscore sôbre\textunderscore  + \textunderscore excelência\textunderscore )}
\end{itemize}
Qualidade do que é sobreexcelente.
\section{Sobreexcelente}
\begin{itemize}
\item {Grp. gram.:adj.}
\end{itemize}
\begin{itemize}
\item {Proveniência:(De \textunderscore sobre\textunderscore  + \textunderscore excelente\textunderscore )}
\end{itemize}
Excelentíssimo; sublime.
\section{Sobreexcellência}
\begin{itemize}
\item {Grp. gram.:f.}
\end{itemize}
\begin{itemize}
\item {Proveniência:(De \textunderscore sôbre\textunderscore  + \textunderscore excellência\textunderscore )}
\end{itemize}
Qualidade do que é sobreexcellente.
\section{Sobreexcellente}
\begin{itemize}
\item {Grp. gram.:adj.}
\end{itemize}
\begin{itemize}
\item {Proveniência:(De \textunderscore sobre\textunderscore  + \textunderscore excellente\textunderscore )}
\end{itemize}
Excellentíssimo; sublime.
\section{Sobreexcitação}
\begin{itemize}
\item {Grp. gram.:f.}
\end{itemize}
Acto ou effeito de sobreexcitar.
Grande excitação de ânimo; grande excitação nervosa.
\section{Sobreexcitar}
\begin{itemize}
\item {Grp. gram.:v. t.}
\end{itemize}
\begin{itemize}
\item {Proveniência:(De \textunderscore sôbre\textunderscore  + \textunderscore excitar\textunderscore )}
\end{itemize}
Excitar intensamente.
Agitar ou impressionar vivamente o ânimo de.
\section{Sobreface}
\begin{itemize}
\item {Grp. gram.:f.}
\end{itemize}
\begin{itemize}
\item {Utilização:Ant.}
\end{itemize}
\begin{itemize}
\item {Proveniência:(De \textunderscore sôbre\textunderscore  + \textunderscore face\textunderscore )}
\end{itemize}
Espaço, que separa do ângulo externo de um baluarte o flanco prolongado.
Superfície.
\section{Sobrefolheáceo}
\begin{itemize}
\item {Grp. gram.:adj.}
\end{itemize}
\begin{itemize}
\item {Proveniência:(De \textunderscore sôbre\textunderscore  + \textunderscore foliáceo\textunderscore )}
\end{itemize}
Que está sôbre a fôlha; adherente á fôlha, pela parte superior desta.
\section{Sobrefoliáceo}
\begin{itemize}
\item {Grp. gram.:adj.}
\end{itemize}
\begin{itemize}
\item {Utilização:Bot.}
\end{itemize}
\begin{itemize}
\item {Proveniência:(De \textunderscore sôbre\textunderscore  + \textunderscore foliáceo\textunderscore )}
\end{itemize}
Que está sôbre a fôlha; adherente á fôlha, pela parte superior desta.
\section{Sobregata}
\begin{itemize}
\item {Grp. gram.:f.}
\end{itemize}
\begin{itemize}
\item {Utilização:Náut.}
\end{itemize}
\begin{itemize}
\item {Proveniência:(De \textunderscore sôbre\textunderscore  + \textunderscore gata\textunderscore )}
\end{itemize}
Segunda vela do mastro de mezena.
Vêrga, correspondente a essa vela.
\section{Sobregatinha}
\begin{itemize}
\item {Grp. gram.:f.}
\end{itemize}
\begin{itemize}
\item {Utilização:Náut.}
\end{itemize}
\begin{itemize}
\item {Proveniência:(De \textunderscore sobregata\textunderscore )}
\end{itemize}
Vela redonda, superior a sobregata.
Vêrga, que lhe corresponde.
\section{Sobregávea}
\begin{itemize}
\item {Grp. gram.:f.}
\end{itemize}
\begin{itemize}
\item {Utilização:Náut.}
\end{itemize}
\begin{itemize}
\item {Proveniência:(De \textunderscore sôbre\textunderscore  + \textunderscore gávea\textunderscore )}
\end{itemize}
Peça superior á gávea.
\section{Sobrego}
\begin{itemize}
\item {Grp. gram.:m.}
\end{itemize}
Variedade de pêra, hoje desconhecida. Cf. Rui Fernandes, \textunderscore Inéd. da Hist. Port.\textunderscore 
\section{Sobregovêrno}
\begin{itemize}
\item {Grp. gram.:m.}
\end{itemize}
\begin{itemize}
\item {Proveniência:(De \textunderscore sôbre\textunderscore  + \textunderscore govêrno\textunderscore )}
\end{itemize}
Supremo govêrno; mando superior.
\section{Sobreguisa}
\begin{itemize}
\item {Grp. gram.:adv.}
\end{itemize}
\begin{itemize}
\item {Proveniência:(De \textunderscore sôbre\textunderscore  + \textunderscore guisa\textunderscore )}
\end{itemize}
Excessivamente, sôbremaneira.
\section{Sobrehumano}
\begin{itemize}
\item {Grp. gram.:adj.}
\end{itemize}
\begin{itemize}
\item {Utilização:Fig.}
\end{itemize}
\begin{itemize}
\item {Proveniência:(De \textunderscore sôbre\textunderscore  + \textunderscore humano\textunderscore )}
\end{itemize}
Superior á natureza do homem ou ás fôrças humanas.
Sublime.
\section{Sôbre-infusa}
\begin{itemize}
\item {Grp. gram.:f.}
\end{itemize}
\begin{itemize}
\item {Utilização:Prov.}
\end{itemize}
\begin{itemize}
\item {Utilização:alent.}
\end{itemize}
Espécie de sopa, temperada com toicinho.
\section{Sôbre-intelligivel}
\begin{itemize}
\item {Grp. gram.:adj.}
\end{itemize}
Superior á intelligência humana. Cf. Herculano, \textunderscore Opúsc.\textunderscore , III, p. XIX.
\section{Sobreintender}
\begin{itemize}
\item {Grp. gram.:v. i.}
\end{itemize}
O mesmo que \textunderscore superintender\textunderscore .
\section{Sobreír}
\begin{itemize}
\item {Grp. gram.:v. i.}
\end{itemize}
\begin{itemize}
\item {Proveniência:(De \textunderscore sôbre\textunderscore  + \textunderscore ir\textunderscore )}
\end{itemize}
Ir sôbre ou contra. Cf. Castilho, \textunderscore Fastos\textunderscore , II, 197.
\section{Sobreira}
\begin{itemize}
\item {Grp. gram.:f.}
\end{itemize}
Espécie de sobreiro.
\section{Sobreiral}
\begin{itemize}
\item {Grp. gram.:m.}
\end{itemize}
O mesmo que \textunderscore sobral\textunderscore .
\section{Sobreirinho}
\begin{itemize}
\item {Grp. gram.:m.}
\end{itemize}
Espécie de uva preta, também chamada \textunderscore tinta sobreirinha\textunderscore .
\section{Sobreiro}
\begin{itemize}
\item {Grp. gram.:m.}
\end{itemize}
\begin{itemize}
\item {Grp. gram.:Adj.}
\end{itemize}
\begin{itemize}
\item {Utilização:T. do Fundão}
\end{itemize}
\begin{itemize}
\item {Proveniência:(Do b. lat. \textunderscore suberarius\textunderscore )}
\end{itemize}
Árvore cupulífera, (\textunderscore quercus hispanica\textunderscore ).
Sôbro.
Bruto; estúpido.
\section{Sobreirritar}
\begin{itemize}
\item {fónica:bre-i}
\end{itemize}
\begin{itemize}
\item {Grp. gram.:v. t.}
\end{itemize}
Irritar extraordinariamente. Cf. Rui Barb., \textunderscore Réplica\textunderscore , 158.
\section{Sobrejacente}
\begin{itemize}
\item {Grp. gram.:adj.}
\end{itemize}
\begin{itemize}
\item {Utilização:Geol.}
\end{itemize}
\begin{itemize}
\item {Proveniência:(De \textunderscore sôbre\textunderscore  + \textunderscore jacente\textunderscore )}
\end{itemize}
Diz-se das rochas vulcânicas, por estarem sôbre as graníticas.
\section{Sobrejanela}
\begin{itemize}
\item {Grp. gram.:f.}
\end{itemize}
Pano ornamental, que pende interiormente do alto da janela. Cf. Corvo, \textunderscore Anno na Côrte\textunderscore , IX.
\section{Sobrejoanete}
\begin{itemize}
\item {fónica:nê}
\end{itemize}
\begin{itemize}
\item {Grp. gram.:m.}
\end{itemize}
\begin{itemize}
\item {Utilização:Náut.}
\end{itemize}
\begin{itemize}
\item {Proveniência:(De \textunderscore sôbre\textunderscore  + \textunderscore joanete\textunderscore )}
\end{itemize}
Vela, que fica sôbre o joanete grande.
Vela que se larga sôbre o joanete da prôa.
\section{Sobrejoanetinho}
\begin{itemize}
\item {Grp. gram.:m.}
\end{itemize}
\begin{itemize}
\item {Utilização:Náut.}
\end{itemize}
Cada uma das velas que ficam acima dos sobrejoanetes.
\section{Sobrejuiz}
\begin{itemize}
\item {Grp. gram.:m.}
\end{itemize}
\begin{itemize}
\item {Utilização:Ant.}
\end{itemize}
\begin{itemize}
\item {Proveniência:(De \textunderscore sôbre\textunderscore  + \textunderscore juíz\textunderscore )}
\end{itemize}
Juiz da segunda ou última instáncia.
\section{Sobrelanço}
\begin{itemize}
\item {Grp. gram.:m.}
\end{itemize}
\begin{itemize}
\item {Proveniência:(De \textunderscore sôbre\textunderscore  + \textunderscore lanço\textunderscore )}
\end{itemize}
Lanço, seguido a outro.
Lanço, maior que outros.
\section{Sobrelátego}
\begin{itemize}
\item {Grp. gram.:m.}
\end{itemize}
\begin{itemize}
\item {Utilização:Bras. do S}
\end{itemize}
\begin{itemize}
\item {Proveniência:(De \textunderscore sôbre\textunderscore  + \textunderscore látego\textunderscore )}
\end{itemize}
Tira de coiro cru, nos arreios da cavalgadura.
\section{Sobreleitar}
\begin{itemize}
\item {Grp. gram.:v.}
\end{itemize}
\begin{itemize}
\item {Utilização:t. Constr.}
\end{itemize}
\begin{itemize}
\item {Proveniência:(De \textunderscore sôbre\textunderscore  + \textunderscore leito\textunderscore )}
\end{itemize}
Picar superiormente (uma pedra), para nella assentar outra.
\section{Sobreleite}
\begin{itemize}
\item {Grp. gram.:m.}
\end{itemize}
\begin{itemize}
\item {Proveniência:(De \textunderscore sôbre\textunderscore  + \textunderscore leite\textunderscore )}
\end{itemize}
Massa carnuda, adeante das glândulas mamaes das vacas.
\section{Sobreleito}
\begin{itemize}
\item {Grp. gram.:m.}
\end{itemize}
\begin{itemize}
\item {Utilização:Constr.}
\end{itemize}
\begin{itemize}
\item {Proveniência:(De \textunderscore sôbre\textunderscore  + \textunderscore leito\textunderscore )}
\end{itemize}
Superfície inferior de cada uma das camadas, que constituem a parede.
\section{Sobrelevante}
\begin{itemize}
\item {Grp. gram.:adj.}
\end{itemize}
Que sobreleva.
\section{Sobrelevar}
\begin{itemize}
\item {Grp. gram.:v. t.}
\end{itemize}
\begin{itemize}
\item {Grp. gram.:V. i.}
\end{itemize}
\begin{itemize}
\item {Proveniência:(De \textunderscore sôbre\textunderscore  + \textunderscore elevar\textunderscore )}
\end{itemize}
Sêr mais alto que.
Exceder.
Erguer; levantar.
Supplantar.
Vencer.
Supportar.
Sobresaír.
Levar vantagem.
\section{Sobrelimão}
\begin{itemize}
\item {Grp. gram.:m.}
\end{itemize}
\begin{itemize}
\item {Utilização:Prov.}
\end{itemize}
Peça, que, no carro alentejano, fica ao lado do limão.
\section{Sobreliminar}
\begin{itemize}
\item {Grp. gram.:adj.}
\end{itemize}
\begin{itemize}
\item {Proveniência:(De \textunderscore sôbre\textunderscore  + \textunderscore liminar\textunderscore )}
\end{itemize}
Viga, atravessada sôbre os esteios de ponte levadiça.
\section{Sobreloja}
\begin{itemize}
\item {Grp. gram.:f.}
\end{itemize}
\begin{itemize}
\item {Proveniência:(De \textunderscore sôbre\textunderscore  + \textunderscore loja\textunderscore )}
\end{itemize}
Pavimento de um prédio, a pouca altura da rua, e entre a loja ou rés-do-chão e o primeiro andar.
\section{Sobrelotação}
\begin{itemize}
\item {Grp. gram.:f.}
\end{itemize}
\begin{itemize}
\item {Proveniência:(De \textunderscore sôbre\textunderscore  + \textunderscore lotação\textunderscore )}
\end{itemize}
Aquillo que excede a lotação ordinária de um navio.
\section{Sobremachinho}
\begin{itemize}
\item {Grp. gram.:m.}
\end{itemize}
\begin{itemize}
\item {Proveniência:(De \textunderscore sôbre\textunderscore  + \textunderscore machinho\textunderscore )}
\end{itemize}
Protuberância, resultante da inflammação dos tendões das cavalgaduras.
\section{Sobremaneira}
\begin{itemize}
\item {Grp. gram.:adv.}
\end{itemize}
\begin{itemize}
\item {Proveniência:(De \textunderscore sôbre\textunderscore  + \textunderscore maneira\textunderscore )}
\end{itemize}
Muito; excessivamente; extraordináriamente.
\section{Sobremanhã}
\begin{itemize}
\item {Grp. gram.:f.}
\end{itemize}
\begin{itemize}
\item {Proveniência:(De \textunderscore sôbre\textunderscore  + \textunderscore manhã\textunderscore )}
\end{itemize}
Fim da manhã.
\section{Sobremanhan}
\begin{itemize}
\item {Grp. gram.:f.}
\end{itemize}
\begin{itemize}
\item {Proveniência:(De \textunderscore sôbre\textunderscore  + \textunderscore manhan\textunderscore )}
\end{itemize}
Fim da manhan.
\section{Sobremão}
\begin{itemize}
\item {Grp. gram.:m.}
\end{itemize}
\begin{itemize}
\item {Grp. gram.:Loc. adv.}
\end{itemize}
\begin{itemize}
\item {Proveniência:(De \textunderscore sôbre\textunderscore  + \textunderscore mão\textunderscore )}
\end{itemize}
Tumor, na mão da bêsta.
\textunderscore De sobremão\textunderscore , com empenho, com interesse.
Com esmêro.
Com todo o descanso.
Com fartura.
\section{Sobremaravilhar}
\begin{itemize}
\item {Grp. gram.:v. t.}
\end{itemize}
\begin{itemize}
\item {Proveniência:(De \textunderscore sôbre\textunderscore  + \textunderscore maravilhar\textunderscore )}
\end{itemize}
Maravilhar extremamente.
\section{Sobremesa}
\begin{itemize}
\item {Grp. gram.:f.}
\end{itemize}
\begin{itemize}
\item {Proveniência:(De \textunderscore sôbre\textunderscore  + \textunderscore mesa\textunderscore )}
\end{itemize}
Fruta, doce ou outra iguaria, mais ou menos delicada e leve, com que se termina uma refeição.
\section{Sobremocóvia}
\begin{itemize}
\item {Grp. gram.:f.}
\end{itemize}
\begin{itemize}
\item {Utilização:Gír.}
\end{itemize}
O mesmo que \textunderscore sobremoscóvia\textunderscore .
\section{Sobremodo}
\begin{itemize}
\item {Grp. gram.:adv.}
\end{itemize}
\begin{itemize}
\item {Proveniência:(De \textunderscore sôbre\textunderscore  + \textunderscore modo\textunderscore )}
\end{itemize}
O mesmo que \textunderscore sobremaneira\textunderscore .
\section{Sobremordomo}
\begin{itemize}
\item {Grp. gram.:m.}
\end{itemize}
\begin{itemize}
\item {Utilização:Des.}
\end{itemize}
\begin{itemize}
\item {Utilização:Ant.}
\end{itemize}
O mesmo que [[mordomo-mór|mordomo]]. Cf. \textunderscore Hist. de Port.\textunderscore , III, 300.
\section{Sobremoscóvia}
\begin{itemize}
\item {Grp. gram.:f.}
\end{itemize}
\begin{itemize}
\item {Utilização:Gír.}
\end{itemize}
\begin{itemize}
\item {Proveniência:(De \textunderscore sôbre\textunderscore  + \textunderscore moscóvia\textunderscore )}
\end{itemize}
Sobrecasaca.
\section{Sobremunhoneiras}
\begin{itemize}
\item {Grp. gram.:f. pl.}
\end{itemize}
\begin{itemize}
\item {Utilização:Náut.}
\end{itemize}
\begin{itemize}
\item {Proveniência:(De \textunderscore sôbre\textunderscore  + \textunderscore munhoneira\textunderscore )}
\end{itemize}
Peças, que ficam sobre as munhoneiras, para que dentro destas se segurem os munhões.
\section{Sobrenadação}
\begin{itemize}
\item {Grp. gram.:f.}
\end{itemize}
Acto de sobrenadar.
\section{Sobrenadante}
\begin{itemize}
\item {Grp. gram.:adj.}
\end{itemize}
Que sobrenada.
\section{Sobrenadar}
\begin{itemize}
\item {Grp. gram.:v. i.}
\end{itemize}
\begin{itemize}
\item {Proveniência:(Do lat. \textunderscore supernatare\textunderscore )}
\end{itemize}
Nadar em cima; andar á tona de água; boiar.
\section{Sobrenatural}
\begin{itemize}
\item {Grp. gram.:adj.}
\end{itemize}
\begin{itemize}
\item {Grp. gram.:M.}
\end{itemize}
\begin{itemize}
\item {Proveniência:(De \textunderscore sôbre\textunderscore  + \textunderscore natural\textunderscore )}
\end{itemize}
Superior ao que é natural.
Excessivo.
Sobrehumano.
Aquillo que é superior á natureza ou o que é muito extraordinário.
\section{Sobrenaturalidade}
\begin{itemize}
\item {Grp. gram.:f.}
\end{itemize}
Qualidade do que é sobrenatural.
\section{Sobrenaturalmente}
\begin{itemize}
\item {Grp. gram.:adv.}
\end{itemize}
De modo sobrenatural.
\section{Sobrenervo}
\begin{itemize}
\item {fónica:nêr}
\end{itemize}
\begin{itemize}
\item {Grp. gram.:m.}
\end{itemize}
\begin{itemize}
\item {Proveniência:(De \textunderscore sôbre\textunderscore  + \textunderscore nervo\textunderscore )}
\end{itemize}
Tumor sôbre um nervo.
\section{Sobrenome}
\begin{itemize}
\item {Grp. gram.:m.}
\end{itemize}
\begin{itemize}
\item {Proveniência:(De \textunderscore sôbre\textunderscore  + \textunderscore nome\textunderscore )}
\end{itemize}
Nome, que segue o do baptismo; appellido.
\section{Sobrenomear}
\begin{itemize}
\item {Grp. gram.:v. t.}
\end{itemize}
Dar sobrenome a; appellidar.
\section{Sobrenumerável}
\begin{itemize}
\item {Grp. gram.:adj.}
\end{itemize}
\begin{itemize}
\item {Proveniência:(De \textunderscore sôbre\textunderscore  + \textunderscore numerável\textunderscore )}
\end{itemize}
Innúmero; que não tem conto.
\section{Sobreolhar}
\begin{itemize}
\item {Grp. gram.:v. t.}
\end{itemize}
\begin{itemize}
\item {Proveniência:(De \textunderscore sôbre\textunderscore  + \textunderscore olhar\textunderscore )}
\end{itemize}
Olhar sobranceiramente, com desprêzo.
\section{Sobreôsso}
\begin{itemize}
\item {Grp. gram.:m.}
\end{itemize}
\begin{itemize}
\item {Proveniência:(De \textunderscore sôbre\textunderscore  + \textunderscore osso\textunderscore )}
\end{itemize}
Engrossamento anormal de um ôsso, nas cavalgaduras.
\section{Sobrepaga}
\begin{itemize}
\item {Grp. gram.:f.}
\end{itemize}
\begin{itemize}
\item {Proveniência:(De \textunderscore sôbre\textunderscore  + \textunderscore paga\textunderscore )}
\end{itemize}
O que se paga além daquillo que se estipulou.
Gratificação.
\section{Sobrepairar}
\begin{itemize}
\item {Grp. gram.:v. t.}
\end{itemize}
\begin{itemize}
\item {Proveniência:(De \textunderscore sôbre\textunderscore  + \textunderscore pairar\textunderscore )}
\end{itemize}
Pairar alto. Cf. Castilho, \textunderscore Metam.\textunderscore , 103.
\section{Sobreparto}
\begin{itemize}
\item {Grp. gram.:adv.}
\end{itemize}
\begin{itemize}
\item {Grp. gram.:M.}
\end{itemize}
\begin{itemize}
\item {Proveniência:(De \textunderscore sôbre\textunderscore  + \textunderscore parto\textunderscore )}
\end{itemize}
Depois do parto.
Doença, que póde sobrevir ao parto.
\section{Sobrepasto}
\begin{itemize}
\item {Grp. gram.:m.}
\end{itemize}
O mesmo que \textunderscore sobremesa\textunderscore . Cf. Arn. Gama, \textunderscore Segr. do Abbade\textunderscore , 246.
\section{Sobrepé}
\begin{itemize}
\item {Grp. gram.:m.}
\end{itemize}
\begin{itemize}
\item {Proveniência:(De \textunderscore sôbre\textunderscore  + \textunderscore pé\textunderscore )}
\end{itemize}
Sobreôsso na corôa posterior do pé da bêsta.
\section{Sobrepeliz}
\begin{itemize}
\item {Grp. gram.:f.}
\end{itemize}
\begin{itemize}
\item {Proveniência:(Do b. lat. \textunderscore superpellicium\textunderscore )}
\end{itemize}
Espécie de mantelete branco, com mangas ou sem elas, que os clérigos usam sôbre a batina.
\section{Sobrepelliz}
\begin{itemize}
\item {Grp. gram.:f.}
\end{itemize}
\begin{itemize}
\item {Proveniência:(Do b. lat. \textunderscore superpellicium\textunderscore )}
\end{itemize}
Espécie de mantelete branco, com mangas ou sem ellas, que os clérigos usam sôbre a batina.
\section{Sobrepensado}
\begin{itemize}
\item {Grp. gram.:adv.}
\end{itemize}
\begin{itemize}
\item {Proveniência:(De \textunderscore sobrepensar\textunderscore )}
\end{itemize}
De propósito; de caso pensado; premeditadamente.
\section{Sobrepensar}
\begin{itemize}
\item {Grp. gram.:v. i.}
\end{itemize}
\begin{itemize}
\item {Grp. gram.:V. t.}
\end{itemize}
\begin{itemize}
\item {Proveniência:(De \textunderscore sôbre\textunderscore  + \textunderscore pensar\textunderscore )}
\end{itemize}
Pensar muito, reflectir; meditar.
Pensar muito á cêrca de; premeditar.
\section{Sobrepeso}
\begin{itemize}
\item {fónica:pê}
\end{itemize}
\begin{itemize}
\item {Grp. gram.:m.}
\end{itemize}
\begin{itemize}
\item {Proveniência:(De \textunderscore sôbre\textunderscore  + \textunderscore pêso\textunderscore )}
\end{itemize}
O mesmo que \textunderscore sobrecarga\textunderscore .
\section{Sobrepontas}
\begin{itemize}
\item {Grp. gram.:f. pl.}
\end{itemize}
\begin{itemize}
\item {Utilização:Prov.}
\end{itemize}
\begin{itemize}
\item {Utilização:alent.}
\end{itemize}
\begin{itemize}
\item {Proveniência:(De \textunderscore sôbre\textunderscore  + \textunderscore ponta\textunderscore )}
\end{itemize}
Os caçadores que, formados com outros em linha de atiradores, occupam os lugares immediatos aos das extremidades da linha.
\section{Sobrepor}
\begin{itemize}
\item {Grp. gram.:v. t.}
\end{itemize}
\begin{itemize}
\item {Grp. gram.:V. p.}
\end{itemize}
\begin{itemize}
\item {Proveniência:(Do lat. \textunderscore superponere\textunderscore )}
\end{itemize}
Pôr em cima.
Encostar.
Juxtapor.
Juntar, accrescentar.
Dobrar na parte superior.
Sobrevir.
\section{Sobreporta}
\begin{itemize}
\item {Grp. gram.:f.}
\end{itemize}
\begin{itemize}
\item {Proveniência:(De \textunderscore sôbre\textunderscore  + \textunderscore porta\textunderscore )}
\end{itemize}
Parte superior e fixa de uma porta.
Bandeira da porta.
\section{Sobreposição}
\begin{itemize}
\item {Grp. gram.:f.}
\end{itemize}
\begin{itemize}
\item {Proveniência:(Do lat. \textunderscore superpositio\textunderscore )}
\end{itemize}
Acto ou effeito de sobrepor.
\section{Sobreposse}
\begin{itemize}
\item {Grp. gram.:loc. adv.}
\end{itemize}
\begin{itemize}
\item {Proveniência:(De \textunderscore sôbre\textunderscore  + \textunderscore posse\textunderscore )}
\end{itemize}
Demasiadamente; por demais; contra vontade; sem espontaneidade: \textunderscore comer sobreposse\textunderscore .
\section{Sobreposta}
\begin{itemize}
\item {Grp. gram.:f.}
\end{itemize}
\begin{itemize}
\item {Utilização:Prov.}
\end{itemize}
\begin{itemize}
\item {Utilização:trasm.}
\end{itemize}
Peça de madeira, na extremidade do cabeçalho, onde está o chavelhão.
\section{Sobreposto}
\begin{itemize}
\item {fónica:pôs}
\end{itemize}
\begin{itemize}
\item {Grp. gram.:m. pl.}
\end{itemize}
\begin{itemize}
\item {Proveniência:(Do lat. \textunderscore superpositus\textunderscore )}
\end{itemize}
Aquillo que se põe, como ornato, sôbre vestidos, jaêzes, etc.
\section{Sobrepovoar}
\begin{itemize}
\item {Grp. gram.:v. t.}
\end{itemize}
\begin{itemize}
\item {Proveniência:(De \textunderscore sôbre\textunderscore  + \textunderscore povoar\textunderscore )}
\end{itemize}
Desenvolver a população de.
\section{Sobrepratear}
\begin{itemize}
\item {Grp. gram.:v. t.}
\end{itemize}
\begin{itemize}
\item {Proveniência:(De \textunderscore sôbre\textunderscore  + \textunderscore pratear\textunderscore )}
\end{itemize}
Pratear por cima; cobrir com lâmina de prata.
\section{Sobreprova}
\begin{itemize}
\item {Grp. gram.:f.}
\end{itemize}
\begin{itemize}
\item {Proveniência:(De \textunderscore sôbre\textunderscore  + \textunderscore prova\textunderscore )}
\end{itemize}
Nova prova; confirmação.
\section{Sobrepujamento}
\begin{itemize}
\item {Grp. gram.:m.}
\end{itemize}
Acto ou effeito de sobrepujar.
\section{Sobrepujança}
\begin{itemize}
\item {Grp. gram.:f.}
\end{itemize}
O mesmo que \textunderscore sobrepujamento\textunderscore .
\section{Sobrepujante}
\begin{itemize}
\item {Grp. gram.:adj.}
\end{itemize}
Que sobrepuja.
\section{Sobrepujantemente}
\begin{itemize}
\item {Grp. gram.:adv.}
\end{itemize}
De modo sobrepujante.
\section{Sobrepujar}
\begin{itemize}
\item {Grp. gram.:v. t.}
\end{itemize}
\begin{itemize}
\item {Grp. gram.:V. i.}
\end{itemize}
Pujar muito.
Exceder.
Superar.
Ultrapassar; passar por cima de.
Sobresair; levar vantagem.
(Do \textunderscore sôbre\textunderscore  + \textunderscore pujar\textunderscore )
\section{Sobrequartela}
\begin{itemize}
\item {Grp. gram.:f.}
\end{itemize}
\begin{itemize}
\item {Proveniência:(De \textunderscore sôbre\textunderscore  + \textunderscore quartela\textunderscore )}
\end{itemize}
Protuberância anormal, resultante da dilatação das bolsas synoviaes.
\section{Sobrequilha}
\begin{itemize}
\item {Grp. gram.:f.}
\end{itemize}
\begin{itemize}
\item {Proveniência:(De \textunderscore sôbre\textunderscore  + \textunderscore quilha\textunderscore )}
\end{itemize}
Peça ou peças de madeira, que vão da prôa á popa do navio, para fortalecer as cavernas.
\section{Sobrerelha}
\begin{itemize}
\item {fónica:rê}
\end{itemize}
\begin{itemize}
\item {Grp. gram.:f.}
\end{itemize}
\begin{itemize}
\item {Utilização:Prov.}
\end{itemize}
\begin{itemize}
\item {Utilização:trasm.}
\end{itemize}
\begin{itemize}
\item {Proveniência:(De \textunderscore sôbre\textunderscore  + \textunderscore relha\textunderscore )}
\end{itemize}
Cada uma das peças de ferro, que ligam entre si o mile, os impoltos e as cambas.
\section{Sobrerenal}
\begin{itemize}
\item {fónica:re}
\end{itemize}
\begin{itemize}
\item {Grp. gram.:adj.}
\end{itemize}
\begin{itemize}
\item {Utilização:Anat.}
\end{itemize}
Que está situado sôbre o rim: \textunderscore glande\textunderscore  ou \textunderscore capsula sobrerenal\textunderscore .
\section{Sobrerestar}
\begin{itemize}
\item {fónica:res}
\end{itemize}
\begin{itemize}
\item {Grp. gram.:v. i.}
\end{itemize}
\begin{itemize}
\item {Proveniência:(De \textunderscore sôbre\textunderscore  + \textunderscore restar\textunderscore )}
\end{itemize}
Restar depois; sobreviver.
\section{Sobrerodella}
\begin{itemize}
\item {fónica:ro}
\end{itemize}
\begin{itemize}
\item {Grp. gram.:f.}
\end{itemize}
\begin{itemize}
\item {Proveniência:(De \textunderscore sôbre\textunderscore  + \textunderscore rodella\textunderscore )}
\end{itemize}
Tumor, sôbre o joêlho das cavalgaduras.
\section{Sobrerolda}
\begin{itemize}
\item {fónica:rol}
\end{itemize}
\begin{itemize}
\item {Proveniência:(De \textunderscore sôbre\textunderscore  + \textunderscore rolda\textunderscore )}
\end{itemize}
\textunderscore m.\textunderscore  e \textunderscore f.\textunderscore  (e der.)
O mesmo que \textunderscore sobrerronda\textunderscore , etc.
\section{Sobreronda}
\begin{itemize}
\item {fónica:ron}
\end{itemize}
\begin{itemize}
\item {Grp. gram.:f.}
\end{itemize}
\begin{itemize}
\item {Grp. gram.:M.  e  f.}
\end{itemize}
\begin{itemize}
\item {Proveniência:(De \textunderscore sôbre\textunderscore  + \textunderscore ronda\textunderscore )}
\end{itemize}
Vigia das rondas.
Indivíduo, que fiscaliza ou vigia o serviço das rondas.
\section{Sobrerondar}
\begin{itemize}
\item {fónica:ron}
\end{itemize}
\begin{itemize}
\item {Grp. gram.:v. t.}
\end{itemize}
\begin{itemize}
\item {Grp. gram.:V. i.}
\end{itemize}
\begin{itemize}
\item {Proveniência:(De \textunderscore sôbre\textunderscore  + \textunderscore rondar\textunderscore )}
\end{itemize}
Vigiar como sobreronda.
Fazer sobreronda.
\section{Sobrerosado}
\begin{itemize}
\item {fónica:ro}
\end{itemize}
\begin{itemize}
\item {Grp. gram.:adj.}
\end{itemize}
\begin{itemize}
\item {Proveniência:(De \textunderscore sôbre\textunderscore  + \textunderscore rosado\textunderscore )}
\end{itemize}
Tirante a rosado.
\section{Sobrerrelha}
\begin{itemize}
\item {Grp. gram.:f.}
\end{itemize}
\begin{itemize}
\item {Utilização:Prov.}
\end{itemize}
\begin{itemize}
\item {Utilização:trasm.}
\end{itemize}
\begin{itemize}
\item {Proveniência:(De \textunderscore sôbre\textunderscore  + \textunderscore relha\textunderscore )}
\end{itemize}
Cada uma das peças de ferro, que ligam entre si o mile, os impoltos e as cambas.
\section{Sobrerrenal}
\begin{itemize}
\item {Grp. gram.:adj.}
\end{itemize}
\begin{itemize}
\item {Utilização:Anat.}
\end{itemize}
Que está situado sôbre o rim: \textunderscore glande\textunderscore  ou \textunderscore capsula sobrerrenal\textunderscore .
\section{Sobrerrestar}
\begin{itemize}
\item {Grp. gram.:v. i.}
\end{itemize}
\begin{itemize}
\item {Proveniência:(De \textunderscore sôbre\textunderscore  + \textunderscore restar\textunderscore )}
\end{itemize}
Restar depois; sobreviver.
\section{Sobrerrodela}
\begin{itemize}
\item {Grp. gram.:f.}
\end{itemize}
\begin{itemize}
\item {Proveniência:(De \textunderscore sôbre\textunderscore  + \textunderscore rodella\textunderscore )}
\end{itemize}
Tumor, sôbre o joêlho das cavalgaduras.
\section{Sobrerrolda}
\begin{itemize}
\item {Proveniência:(De \textunderscore sôbre\textunderscore  + \textunderscore rolda\textunderscore )}
\end{itemize}
\textunderscore m.\textunderscore  e \textunderscore f.\textunderscore  (e der.)
O mesmo que \textunderscore sobrerronda\textunderscore , etc.
\section{Sobrerronda}
\begin{itemize}
\item {Grp. gram.:f.}
\end{itemize}
\begin{itemize}
\item {Grp. gram.:M.  e  f.}
\end{itemize}
\begin{itemize}
\item {Proveniência:(De \textunderscore sôbre\textunderscore  + \textunderscore ronda\textunderscore )}
\end{itemize}
Vigia das rondas.
Indivíduo, que fiscaliza ou vigia o serviço das rondas.
\section{Sobrerrondar}
\begin{itemize}
\item {Grp. gram.:v. t.}
\end{itemize}
\begin{itemize}
\item {Grp. gram.:V. i.}
\end{itemize}
\begin{itemize}
\item {Proveniência:(De \textunderscore sôbre\textunderscore  + \textunderscore rondar\textunderscore )}
\end{itemize}
Vigiar como sobreronda.
Fazer sobreronda.
\section{Sobrerrosado}
\begin{itemize}
\item {Grp. gram.:adj.}
\end{itemize}
\begin{itemize}
\item {Proveniência:(De \textunderscore sôbre\textunderscore  + \textunderscore rosado\textunderscore )}
\end{itemize}
Tirante a rosado.
\section{Sobresaia}
\begin{itemize}
\item {fónica:sá}
\end{itemize}
\begin{itemize}
\item {Grp. gram.:f.}
\end{itemize}
\begin{itemize}
\item {Utilização:P. us.}
\end{itemize}
\begin{itemize}
\item {Proveniência:(De \textunderscore sôbre\textunderscore  + \textunderscore sáia\textunderscore )}
\end{itemize}
Vestuário, para se usar sôbre a sáia.
\section{Sobresair}
\begin{itemize}
\item {fónica:sa}
\end{itemize}
\begin{itemize}
\item {Grp. gram.:v. i.}
\end{itemize}
\begin{itemize}
\item {Proveniência:(De \textunderscore sôbre\textunderscore  + \textunderscore sair\textunderscore )}
\end{itemize}
Sêr ou estar saliente; resair.
Avultar.
Tornar-se visível.
Distinguir-se.
Ver-se ou ouvir-se distintamente.
Prender a attenção.
\section{Sobresalente}
\begin{itemize}
\item {fónica:sa}
\end{itemize}
\begin{itemize}
\item {Grp. gram.:adj.}
\end{itemize}
\begin{itemize}
\item {Grp. gram.:M.}
\end{itemize}
Que sobresái.
Aquillo que sobresái, que excede.
Aquillo que sobeja, que é a mais.
O mesmo ou melhor que \textunderscore sobrecellente\textunderscore . Cf. Barros, \textunderscore Dec.\textunderscore  III, l. 4.^o, c. 4.
(Por \textunderscore sobresaliente\textunderscore , de \textunderscore sôbre\textunderscore  + \textunderscore saliente\textunderscore . Cp. cast. \textunderscore sobresaliente\textunderscore )
\section{Sobresaliente}
\begin{itemize}
\item {fónica:sa}
\end{itemize}
\begin{itemize}
\item {Grp. gram.:adj.}
\end{itemize}
\begin{itemize}
\item {Proveniência:(De \textunderscore sôbre\textunderscore  + \textunderscore saliente\textunderscore )}
\end{itemize}
O mesmo ou melhor que \textunderscore sobresalente\textunderscore . Cf. Castilho, \textunderscore Fastos\textunderscore , III, 380.
\section{Sobresaltar}
\begin{itemize}
\item {fónica:sal}
\end{itemize}
\begin{itemize}
\item {Grp. gram.:v. t.}
\end{itemize}
\begin{itemize}
\item {Proveniência:(De \textunderscore sôbre\textunderscore  + \textunderscore saltar\textunderscore )}
\end{itemize}
Saltar sôbre.
Surprehender.
Assaltar.
Assustar.
Passar em claro.
Preterir, omittir.
Passar além de.
Transpor.
\section{Sobresaltear}
\begin{itemize}
\item {fónica:sal}
\end{itemize}
\begin{itemize}
\item {Grp. gram.:v. t.}
\end{itemize}
\begin{itemize}
\item {Proveniência:(De \textunderscore sôbre\textunderscore  + \textunderscore saltear\textunderscore )}
\end{itemize}
Assaltar; atacar imprevistamente ou á traição.
Saltear.
\section{Sobresalto}
\begin{itemize}
\item {fónica:sal}
\end{itemize}
\begin{itemize}
\item {Grp. gram.:m.}
\end{itemize}
Acto ou effeito de sobresaltar.
Acontecimento imprevisto.
Perturbação phýsica ou moral; inquietação.
\section{Sobresano}
\begin{itemize}
\item {fónica:sa}
\end{itemize}
\begin{itemize}
\item {Grp. gram.:m.}
\end{itemize}
\begin{itemize}
\item {Utilização:Náut.}
\end{itemize}
Parte do costado do navio, abaixo do nível da água.
\section{Sobresarar}
\begin{itemize}
\item {fónica:sa}
\end{itemize}
\begin{itemize}
\item {Grp. gram.:v. i.}
\end{itemize}
\begin{itemize}
\item {Grp. gram.:V. t.}
\end{itemize}
\begin{itemize}
\item {Proveniência:(De \textunderscore sôbre\textunderscore  + \textunderscore sarar\textunderscore )}
\end{itemize}
Adquirir melhóras passageiras.
Não sarar completamente.
Curar superficialmente, sem atacar a raíz do mal. Cf. Vieira, VI, 407.
\section{Sobresaturação}
\begin{itemize}
\item {fónica:sa}
\end{itemize}
\begin{itemize}
\item {Grp. gram.:f.}
\end{itemize}
\begin{itemize}
\item {Proveniência:(De \textunderscore sôbre\textunderscore  + \textunderscore saturação\textunderscore )}
\end{itemize}
Acto de fazer dissolver num líquido uma substância, que excede aquella que, em condições normaes, bastaría para saturar o mesmo líquido.
\section{Sobresaturar}
\begin{itemize}
\item {fónica:sa}
\end{itemize}
\begin{itemize}
\item {Grp. gram.:v. t.}
\end{itemize}
\begin{itemize}
\item {Proveniência:(De \textunderscore sôbre\textunderscore  + \textunderscore saturar\textunderscore )}
\end{itemize}
Saturar com excesso.
Reduzir ao estado de sobresaturação.
\section{Sobrescada}
\begin{itemize}
\item {Grp. gram.:f.}
\end{itemize}
\begin{itemize}
\item {Utilização:Des.}
\end{itemize}
\begin{itemize}
\item {Proveniência:(De \textunderscore sôbre\textunderscore  + \textunderscore escada\textunderscore )}
\end{itemize}
Espécie de alpendre ou coberto, ao cimo da escada exterior de uma habitação. Cf. Pant. de Aveiro, \textunderscore Itiner.\textunderscore , 64, (3.^a ed.).
\section{Sobrescrever}
\begin{itemize}
\item {Grp. gram.:v. t.}
\end{itemize}
\begin{itemize}
\item {Proveniência:(De \textunderscore sôbre\textunderscore  + \textunderscore escrever\textunderscore )}
\end{itemize}
Escrever sôbre; sobrescritar.
\section{Sobrescritar}
\begin{itemize}
\item {Grp. gram.:v. t.}
\end{itemize}
\begin{itemize}
\item {Utilização:Fig.}
\end{itemize}
\begin{itemize}
\item {Proveniência:(De \textunderscore sobrescrito\textunderscore )}
\end{itemize}
Fazer o sobrescrito de.
Escrever o enderêço de (uma carta).
Destinar; dirigir.
\section{Sobrescrito}
\begin{itemize}
\item {Grp. gram.:m.}
\end{itemize}
\begin{itemize}
\item {Proveniência:(De \textunderscore sôbre\textunderscore  + \textunderscore escrito\textunderscore )}
\end{itemize}
Invólucro de uma carta ou de um offício, em que se escreve o nome do destinatário e o da sua residência.
Enderêço.
Aquillo que se escreve no invólucro da carta ou offício.
\section{Sobresello}
\begin{itemize}
\item {fónica:sê}
\end{itemize}
\begin{itemize}
\item {Grp. gram.:m.}
\end{itemize}
\begin{itemize}
\item {Proveniência:(De \textunderscore sôbre\textunderscore  + \textunderscore sêllo\textunderscore )}
\end{itemize}
Segundo sêllo.
Sêllo, sobreposto a outro.
\section{Sobresemear}
\begin{itemize}
\item {fónica:se}
\end{itemize}
\begin{itemize}
\item {Grp. gram.:v. t.}
\end{itemize}
\begin{itemize}
\item {Proveniência:(Do lat. \textunderscore superseminare\textunderscore )}
\end{itemize}
Semear sôbre; semear superficialmente.
\section{Sobreser}
\begin{itemize}
\item {fónica:ser}
\end{itemize}
\begin{itemize}
\item {Proveniência:(De \textunderscore sôbre\textunderscore  + \textunderscore sêr\textunderscore )}
\end{itemize}
\textunderscore v. i.\textunderscore  (e der.)
O mesmo que \textunderscore sobrestar\textunderscore .
\section{Sobresi}
\begin{itemize}
\item {fónica:si}
\end{itemize}
\begin{itemize}
\item {Grp. gram.:m.}
\end{itemize}
\begin{itemize}
\item {Utilização:Ant.}
\end{itemize}
\begin{itemize}
\item {Proveniência:(De \textunderscore sôbre\textunderscore  + \textunderscore si\textunderscore )}
\end{itemize}
Vigilância, fiscalização.
\section{Sobresimento}
\begin{itemize}
\item {fónica:si}
\end{itemize}
\begin{itemize}
\item {Grp. gram.:m.}
\end{itemize}
\begin{itemize}
\item {Utilização:Ant.}
\end{itemize}
Acto de sobreser.
Demora; adiamento; espera.
\section{Sobresinal}
\begin{itemize}
\item {fónica:si}
\end{itemize}
\begin{itemize}
\item {Grp. gram.:m.}
\end{itemize}
\begin{itemize}
\item {Proveniência:(De \textunderscore sôbre\textunderscore  + \textunderscore sinal\textunderscore )}
\end{itemize}
Sinal ou insígnia sôbre o vestuário.
\section{Sobresolar}
\begin{itemize}
\item {fónica:so}
\end{itemize}
\begin{itemize}
\item {Grp. gram.:v. t.}
\end{itemize}
\begin{itemize}
\item {Proveniência:(De \textunderscore sôbre\textunderscore  + \textunderscore sola\textunderscore )}
\end{itemize}
Pôr solas novas em (calçado velho ou usado).
\section{Sobresoleira}
\begin{itemize}
\item {fónica:so}
\end{itemize}
\begin{itemize}
\item {Grp. gram.:f.}
\end{itemize}
\begin{itemize}
\item {Proveniência:(De \textunderscore sôbre\textunderscore  + \textunderscore soleira\textunderscore )}
\end{itemize}
Peça sobre a soleira.
\section{Sobresperar}
\begin{itemize}
\item {Grp. gram.:v. t.  e  i.}
\end{itemize}
\begin{itemize}
\item {Proveniência:(De \textunderscore sôbre\textunderscore  + \textunderscore esperar\textunderscore )}
\end{itemize}
Esperar muito; esperar por muito tempo.
\section{Sobressaia}
\begin{itemize}
\item {Grp. gram.:f.}
\end{itemize}
\begin{itemize}
\item {Utilização:P. us.}
\end{itemize}
\begin{itemize}
\item {Proveniência:(De \textunderscore sôbre\textunderscore  + \textunderscore sáia\textunderscore )}
\end{itemize}
Vestuário, para se usar sôbre a sáia.
\section{Sobressair}
\begin{itemize}
\item {Grp. gram.:v. i.}
\end{itemize}
\begin{itemize}
\item {Proveniência:(De \textunderscore sôbre\textunderscore  + \textunderscore sair\textunderscore )}
\end{itemize}
Sêr ou estar saliente; resair.
Avultar.
Tornar-se visível.
Distinguir-se.
Ver-se ou ouvir-se distintamente.
Prender a attenção.
\section{Sobressalente}
\begin{itemize}
\item {Grp. gram.:adj.}
\end{itemize}
\begin{itemize}
\item {Grp. gram.:M.}
\end{itemize}
Que sobressái.
Aquilo que sobressái, que excede.
Aquilo que sobeja, que é a mais.
O mesmo ou melhor que \textunderscore sobrecelente\textunderscore . Cf. Barros, \textunderscore Dec.\textunderscore  III, l. 4.^o, c. 4.
(Por \textunderscore sobresaliente\textunderscore , de \textunderscore sôbre\textunderscore  + \textunderscore saliente\textunderscore . Cp. cast. \textunderscore sobresaliente\textunderscore )
\section{Sobressaliente}
\begin{itemize}
\item {Grp. gram.:adj.}
\end{itemize}
\begin{itemize}
\item {Proveniência:(De \textunderscore sôbre\textunderscore  + \textunderscore saliente\textunderscore )}
\end{itemize}
O mesmo ou melhor que \textunderscore sobresalente\textunderscore . Cf. Castilho, \textunderscore Fastos\textunderscore , III, 380.
\section{Sobressaltar}
\begin{itemize}
\item {Grp. gram.:v. t.}
\end{itemize}
\begin{itemize}
\item {Proveniência:(De \textunderscore sôbre\textunderscore  + \textunderscore saltar\textunderscore )}
\end{itemize}
Saltar sôbre.
Surpreender.
Assaltar.
Assustar.
Passar em claro.
Preterir, omitir.
Passar além de.
Transpor.
\section{Sobressaltear}
\begin{itemize}
\item {Grp. gram.:v. t.}
\end{itemize}
\begin{itemize}
\item {Proveniência:(De \textunderscore sôbre\textunderscore  + \textunderscore saltear\textunderscore )}
\end{itemize}
Assaltar; atacar imprevistamente ou á traição.
Saltear.
\section{Sobressalto}
\begin{itemize}
\item {Grp. gram.:m.}
\end{itemize}
Acto ou efeito de sobressaltar.
Acontecimento imprevisto.
Perturbação física ou moral; inquietação.
\section{Sobressano}
\begin{itemize}
\item {Grp. gram.:m.}
\end{itemize}
\begin{itemize}
\item {Utilização:Náut.}
\end{itemize}
Parte do costado do navio, abaixo do nível da água.
\section{Sobressarar}
\begin{itemize}
\item {Grp. gram.:v. i.}
\end{itemize}
\begin{itemize}
\item {Grp. gram.:V. t.}
\end{itemize}
\begin{itemize}
\item {Proveniência:(De \textunderscore sôbre\textunderscore  + \textunderscore sarar\textunderscore )}
\end{itemize}
Adquirir melhóras passageiras.
Não sarar completamente.
Curar superficialmente, sem atacar a raíz do mal. Cf. Vieira, VI, 407.
\section{Sobressaturação}
\begin{itemize}
\item {Grp. gram.:f.}
\end{itemize}
\begin{itemize}
\item {Proveniência:(De \textunderscore sôbre\textunderscore  + \textunderscore saturação\textunderscore )}
\end{itemize}
Acto de fazer dissolver num líquido uma substância, que excede aquela que, em condições normaes, bastaría para saturar o mesmo líquido.
\section{Sobressaturar}
\begin{itemize}
\item {Grp. gram.:v. t.}
\end{itemize}
\begin{itemize}
\item {Proveniência:(De \textunderscore sôbre\textunderscore  + \textunderscore saturar\textunderscore )}
\end{itemize}
Saturar com excesso.
Reduzir ao estado de sobressaturação.
\section{Sobresselo}
\begin{itemize}
\item {fónica:sê}
\end{itemize}
\begin{itemize}
\item {Grp. gram.:m.}
\end{itemize}
\begin{itemize}
\item {Proveniência:(De \textunderscore sôbre\textunderscore  + \textunderscore sêlo\textunderscore )}
\end{itemize}
Segundo sêlo.
Sêlo, sobreposto a outro.
\section{Sobressemear}
\begin{itemize}
\item {Grp. gram.:v. t.}
\end{itemize}
\begin{itemize}
\item {Proveniência:(Do lat. \textunderscore superseminare\textunderscore )}
\end{itemize}
Semear sôbre; semear superficialmente.
\section{Sobresser}
\begin{itemize}
\item {Proveniência:(De \textunderscore sôbre\textunderscore  + \textunderscore sêr\textunderscore )}
\end{itemize}
\textunderscore v. i.\textunderscore  (e der.)
O mesmo que \textunderscore sobrestar\textunderscore .
\section{Sobressi}
\begin{itemize}
\item {Grp. gram.:m.}
\end{itemize}
\begin{itemize}
\item {Utilização:Ant.}
\end{itemize}
\begin{itemize}
\item {Proveniência:(De \textunderscore sôbre\textunderscore  + \textunderscore si\textunderscore )}
\end{itemize}
Vigilância, fiscalização.
\section{Sobressimento}
\begin{itemize}
\item {Grp. gram.:m.}
\end{itemize}
\begin{itemize}
\item {Utilização:Ant.}
\end{itemize}
Acto de sobresser.
Demora; adiamento; espera.
\section{Sobressinal}
\begin{itemize}
\item {Grp. gram.:m.}
\end{itemize}
\begin{itemize}
\item {Proveniência:(De \textunderscore sôbre\textunderscore  + \textunderscore sinal\textunderscore )}
\end{itemize}
Sinal ou insígnia sôbre o vestuário.
\section{Sobressolar}
\begin{itemize}
\item {Grp. gram.:v. t.}
\end{itemize}
\begin{itemize}
\item {Proveniência:(De \textunderscore sôbre\textunderscore  + \textunderscore sola\textunderscore )}
\end{itemize}
Pôr solas novas em (calçado velho ou usado).
\section{Sobressoleira}
\begin{itemize}
\item {Grp. gram.:f.}
\end{itemize}
\begin{itemize}
\item {Proveniência:(De \textunderscore sôbre\textunderscore  + \textunderscore soleira\textunderscore )}
\end{itemize}
Peça sobre a soleira.
\section{Sobressubstancial}
\begin{itemize}
\item {Grp. gram.:adj.}
\end{itemize}
\begin{itemize}
\item {Proveniência:(Do lat. \textunderscore supersubstantialis\textunderscore )}
\end{itemize}
Muito substancial. Cf. Vieira, IX, 27.
\section{Sobrestante}
\begin{itemize}
\item {Grp. gram.:adj.}
\end{itemize}
\begin{itemize}
\item {Grp. gram.:M.}
\end{itemize}
\begin{itemize}
\item {Proveniência:(De \textunderscore sobrestar\textunderscore )}
\end{itemize}
Que sobrestá.
Sobranceiro, proeminente.
Superintendente; guarda.
\section{Sobrestar}
\begin{itemize}
\item {Grp. gram.:v. i.}
\end{itemize}
\begin{itemize}
\item {Proveniência:(Do lat. \textunderscore superstare\textunderscore )}
\end{itemize}
Parar.
Não proseguir.
Cessar; abster-se.
Deter-se.
Estar sobranceiro.
Estar imminente.
\section{Sobrestimação}
\begin{itemize}
\item {Grp. gram.:f.}
\end{itemize}
\begin{itemize}
\item {Utilização:Des.}
\end{itemize}
Acto ou effeito de sobrestimar.
\section{Sobrestimar}
\begin{itemize}
\item {Grp. gram.:v. t.}
\end{itemize}
\begin{itemize}
\item {Utilização:Des.}
\end{itemize}
\begin{itemize}
\item {Proveniência:(De \textunderscore sôbre\textunderscore  + \textunderscore estimar\textunderscore )}
\end{itemize}
Estimar muito, em alto grau.
\section{Sobresubstancial}
\begin{itemize}
\item {fónica:su}
\end{itemize}
\begin{itemize}
\item {Grp. gram.:adj.}
\end{itemize}
\begin{itemize}
\item {Proveniência:(Do lat. \textunderscore supersubstantialis\textunderscore )}
\end{itemize}
Muito substancial. Cf. Vieira, IX, 27.
\section{Sobretal}
\begin{itemize}
\item {Grp. gram.:adv.}
\end{itemize}
\begin{itemize}
\item {Utilização:Ant.}
\end{itemize}
\begin{itemize}
\item {Proveniência:(De \textunderscore sôbre\textunderscore  + \textunderscore tal\textunderscore )}
\end{itemize}
Finalmente.
Em conclusão. Cf. R. S. Viterbo, \textunderscore Elucidário\textunderscore .
\section{Sobretaleira}
\begin{itemize}
\item {Grp. gram.:f.}
\end{itemize}
\begin{itemize}
\item {Utilização:Prov.}
\end{itemize}
\begin{itemize}
\item {Utilização:alent.}
\end{itemize}
\begin{itemize}
\item {Proveniência:(De \textunderscore sôbre\textunderscore  + \textunderscore taleira\textunderscore )}
\end{itemize}
Cada uma das travessas, pregadas na extremidade anterior e posterior das chedas e sôbre a pírtiga.
\section{Sobretarde}
\begin{itemize}
\item {Grp. gram.:f.}
\end{itemize}
\begin{itemize}
\item {Proveniência:(De \textunderscore sôbre\textunderscore  + \textunderscore tarde\textunderscore )}
\end{itemize}
Fim da tarde.
Lusco-fusco; noitinha.
\section{Sobretaxa}
\begin{itemize}
\item {Grp. gram.:f.}
\end{itemize}
\begin{itemize}
\item {Proveniência:(De \textunderscore sôbre\textunderscore  + \textunderscore taxa\textunderscore )}
\end{itemize}
Quantia que, em serviços de caminhos de ferro, accresce aos preços ou tarifas ordinárias.
\section{Sobretecer}
\begin{itemize}
\item {Grp. gram.:v. t.}
\end{itemize}
\begin{itemize}
\item {Proveniência:(De \textunderscore sôbre\textunderscore  + \textunderscore tecer\textunderscore )}
\end{itemize}
Tecer sôbre tecido; entretecer.
\section{Sobreteima}
\begin{itemize}
\item {Grp. gram.:adv.}
\end{itemize}
\begin{itemize}
\item {Proveniência:(De \textunderscore sôbre\textunderscore  + \textunderscore teima\textunderscore )}
\end{itemize}
Com muita teimosia; com pertinácia.
\section{Sobreterrestre}
\begin{itemize}
\item {Grp. gram.:adj.}
\end{itemize}
\begin{itemize}
\item {Proveniência:(De \textunderscore sôbre\textunderscore  + \textunderscore terrestre\textunderscore )}
\end{itemize}
Que está em cima da terra; terrestre.
\section{Sobretoalha}
\begin{itemize}
\item {Grp. gram.:f.}
\end{itemize}
\begin{itemize}
\item {Proveniência:(De \textunderscore sôbre\textunderscore  + \textunderscore toalha\textunderscore )}
\end{itemize}
Toalha, que está em cima de outra para resguardar.
\section{Sobretónica}
\begin{itemize}
\item {Grp. gram.:f.}
\end{itemize}
\begin{itemize}
\item {Utilização:Mús.}
\end{itemize}
Segundo grau da escala diatónica.
\section{Sobretudo}
\begin{itemize}
\item {Grp. gram.:m.}
\end{itemize}
\begin{itemize}
\item {Grp. gram.:Adv.}
\end{itemize}
\begin{itemize}
\item {Proveniência:(De \textunderscore sôbre\textunderscore  + \textunderscore tudo\textunderscore )}
\end{itemize}
Casaco comprido e largo, próprio para se vestir sôbre outro, como resguardo contra o frio ou a chuva.
Acima de tudo; principalmente.
\section{Sobreumano}
\begin{itemize}
\item {fónica:bre-u}
\end{itemize}
\begin{itemize}
\item {Grp. gram.:adj.}
\end{itemize}
\begin{itemize}
\item {Utilização:Fig.}
\end{itemize}
\begin{itemize}
\item {Proveniência:(De \textunderscore sôbre\textunderscore  + \textunderscore humano\textunderscore )}
\end{itemize}
Superior á natureza do homem ou ás fôrças humanas.
Sublime.
\section{Sobrevela}
\begin{itemize}
\item {Grp. gram.:f.}
\end{itemize}
\begin{itemize}
\item {Utilização:Ant.}
\end{itemize}
\begin{itemize}
\item {Proveniência:(De \textunderscore sôbre\textunderscore  + \textunderscore vela\textunderscore )}
\end{itemize}
Refôrço de sentinellas.
\section{Sobrevença}
\begin{itemize}
\item {Grp. gram.:f.}
\end{itemize}
\begin{itemize}
\item {Utilização:Ant.}
\end{itemize}
Acto de sobrevir.
\section{Sobreveniente}
\begin{itemize}
\item {Grp. gram.:adj.}
\end{itemize}
O mesmo que \textunderscore superveniente\textunderscore . Cf. Filinto, II, 184.
\section{Sobrevento}
\begin{itemize}
\item {Grp. gram.:m.}
\end{itemize}
\begin{itemize}
\item {Utilização:P. us.}
\end{itemize}
\begin{itemize}
\item {Proveniência:(Do lat. \textunderscore superventus\textunderscore )}
\end{itemize}
Tudo que sobrevém inesperadamente, causando transtôrno.
\section{Sobrevento}
\begin{itemize}
\item {Grp. gram.:m.}
\end{itemize}
\begin{itemize}
\item {Proveniência:(De \textunderscore sôbre\textunderscore  + \textunderscore vento\textunderscore )}
\end{itemize}
Rajada súbita de vento, perturbando a marcha de um navio.
\section{Sobreveste}
\begin{itemize}
\item {Grp. gram.:m.  e  f.}
\end{itemize}
\begin{itemize}
\item {Proveniência:(De \textunderscore sôbre\textunderscore  + \textunderscore veste\textunderscore )}
\end{itemize}
Vestuário, que se traz sôbre outro; sobretudo.
\section{Sobrevestir}
\begin{itemize}
\item {Grp. gram.:v. t.}
\end{itemize}
\begin{itemize}
\item {Proveniência:(Do lat. \textunderscore supervestire\textunderscore )}
\end{itemize}
Vestir por cima; revestir.
\section{Sobrevigiar}
\begin{itemize}
\item {Grp. gram.:v. t.}
\end{itemize}
\begin{itemize}
\item {Proveniência:(De \textunderscore sôbre\textunderscore  + \textunderscore vigiar\textunderscore )}
\end{itemize}
Vigiar superiormente ou como chefe.
Superintender.
\section{Sobrevindo}
\begin{itemize}
\item {Grp. gram.:adj.}
\end{itemize}
\begin{itemize}
\item {Grp. gram.:M.}
\end{itemize}
\begin{itemize}
\item {Proveniência:(De \textunderscore sobrevir\textunderscore )}
\end{itemize}
Que sobreveio.
Indivíduo, que sobreveio, que chegou inesperadamente.
\section{Sobrevir}
\begin{itemize}
\item {Grp. gram.:v. i.}
\end{itemize}
\begin{itemize}
\item {Proveniência:(Do lat. \textunderscore supervenire\textunderscore )}
\end{itemize}
Vir sôbre alguma coisa.
Vir em seguida.
Acontecer depois.
Chegar ou succeder imprevistamente.
\section{Sobrevirtude}
\begin{itemize}
\item {Grp. gram.:f.}
\end{itemize}
\begin{itemize}
\item {Proveniência:(De \textunderscore sôbre\textunderscore  + \textunderscore virtude\textunderscore )}
\end{itemize}
Véu, que as freiras usam sôbre a touca.
\section{Sobrevista}
\begin{itemize}
\item {Grp. gram.:f.}
\end{itemize}
\begin{itemize}
\item {Proveniência:(De \textunderscore sobre\textunderscore  + \textunderscore vista\textunderscore )}
\end{itemize}
Peça de ferro, nas bordas dos morriões.
\section{Sobrevivência}
\begin{itemize}
\item {Grp. gram.:f.}
\end{itemize}
Qualidade ou estado de quem é sobrevivente.
\section{Sobrevivente}
\begin{itemize}
\item {Grp. gram.:m. ,  f.  e  adj.}
\end{itemize}
Pessôa que sobrevive.
\section{Sobreviver}
\begin{itemize}
\item {Grp. gram.:v. i.}
\end{itemize}
\begin{itemize}
\item {Utilização:Fig.}
\end{itemize}
\begin{itemize}
\item {Proveniência:(Lat. \textunderscore supervivere\textunderscore )}
\end{itemize}
Continuar a viver depois de outra coisa ou de outra pessôa.
Escapar; resistir.
\section{Sobrevivo}
\begin{itemize}
\item {Grp. gram.:m.  e  adj.}
\end{itemize}
O mesmo que \textunderscore sobrevivente\textunderscore .
\section{Sobrexceder}
\textunderscore v. t.\textunderscore  e \textunderscore i.\textunderscore  (e der.)
O mesmo que \textunderscore sobreexceder\textunderscore , etc.
\section{Sobrexceler}
\begin{itemize}
\item {Grp. gram.:v. i.}
\end{itemize}
Sobrelevar; sobrepujar.
Sêr sobreexcellente. Cf. Filinto, XXII, XXIV.
(Cp. \textunderscore sobreexcellente\textunderscore )
\section{Sobrexceller}
\begin{itemize}
\item {Grp. gram.:v. i.}
\end{itemize}
Sobrelevar; sobrepujar.
Sêr sobreexcellente. Cf. Filinto, XXII, XXIV.
(Cp. \textunderscore sobreexcellente\textunderscore )
\section{Sobriamente}
\begin{itemize}
\item {Grp. gram.:adv.}
\end{itemize}
De modo sóbrio.
\section{Sobriedade}
\begin{itemize}
\item {Grp. gram.:f.}
\end{itemize}
\begin{itemize}
\item {Proveniência:(Lat. \textunderscore sobrietas\textunderscore )}
\end{itemize}
Qualidade do que é sóbrio.
\section{Sobrinha}
\begin{itemize}
\item {Grp. gram.:f.}
\end{itemize}
Flexão fem. de \textunderscore sobrinho\textunderscore ^1.
\section{Sobrinho}
\begin{itemize}
\item {Grp. gram.:m.}
\end{itemize}
\begin{itemize}
\item {Proveniência:(Lat. \textunderscore sobrinus\textunderscore )}
\end{itemize}
Diz-se de um indivíduo, em relação aos irmãos de seus pais.
\section{Sobrinho}
\begin{itemize}
\item {Grp. gram.:m.}
\end{itemize}
\begin{itemize}
\item {Utilização:Náut.}
\end{itemize}
\begin{itemize}
\item {Proveniência:(De \textunderscore sôbre\textunderscore )}
\end{itemize}
Cada uma das últimas velas trapezóides.
\section{Sóbrio}
\begin{itemize}
\item {Grp. gram.:adj.}
\end{itemize}
\begin{itemize}
\item {Utilização:Ext.}
\end{itemize}
\begin{itemize}
\item {Proveniência:(Lat. \textunderscore sobrius\textunderscore )}
\end{itemize}
Moderado no uso de bebidas espirituosas, principalmente vinho.
Moderado na alimentação.
Moderado.
Parco; simples: \textunderscore alimentação sóbria\textunderscore .
\section{Sôbro}
\begin{itemize}
\item {Grp. gram.:m.}
\end{itemize}
\begin{itemize}
\item {Proveniência:(Do lat. \textunderscore suber\textunderscore )}
\end{itemize}
Árvore cupulífera, (\textunderscore quercus suber\textunderscore ).
Madeira ou lenha desta arvore.
\section{Sob-roda}
\begin{itemize}
\item {Grp. gram.:f.}
\end{itemize}
Saliência, pedregulho, depressão de terreno, ou qualquer accidente analogo numa estrada ou rua, capaz de perturbar o andamento de um vehículo.
\section{Sobrolho}
\begin{itemize}
\item {fónica:brô}
\end{itemize}
\begin{itemize}
\item {Grp. gram.:m.}
\end{itemize}
\begin{itemize}
\item {Utilização:Des.}
\end{itemize}
\begin{itemize}
\item {Proveniência:(De \textunderscore sôbre\textunderscore  + \textunderscore ôlho\textunderscore )}
\end{itemize}
O mesmo que \textunderscore sobrancelha\textunderscore .
Attenção, precaução:«\textunderscore ...pôr as mercadorias em vigilante sobrolho...\textunderscore »Filinto, \textunderscore D. Man.\textunderscore , I, 115.
\section{Sobrosso}
\begin{itemize}
\item {fónica:brô}
\end{itemize}
\begin{itemize}
\item {Grp. gram.:m.}
\end{itemize}
\begin{itemize}
\item {Utilização:Des.}
\end{itemize}
Embaraço, impedimento:«\textunderscore ...o pader em sua presença, diga palavras sem sobrosso.\textunderscore »\textunderscore Luz e Calor\textunderscore , 60. Cf. \textunderscore Eufrosina\textunderscore , 124; Filinto, I, 39 e 357.
\section{Sobscrever}
\begin{itemize}
\item {Grp. gram.:v. t.}
\end{itemize}
O mesmo que \textunderscore subscrever\textunderscore .
\section{Sobscrito}
\begin{itemize}
\item {Grp. gram.:adj.}
\end{itemize}
\begin{itemize}
\item {Proveniência:(De \textunderscore sobscrever\textunderscore )}
\end{itemize}
Que se sobscreveu.
\section{Soca}
\begin{itemize}
\item {Grp. gram.:f.}
\end{itemize}
\begin{itemize}
\item {Utilização:Fam.}
\end{itemize}
Cheta; pouco dinheiro.
\section{Soca}
\begin{itemize}
\item {Grp. gram.:f.}
\end{itemize}
\begin{itemize}
\item {Utilização:Bras}
\end{itemize}
Designação vulgar do rhizoma ou caule subterrâneo.
Brotamento, que segue o primeiro córte da cana do açúcar.
\section{Soca}
\begin{itemize}
\item {Grp. gram.:m.}
\end{itemize}
\begin{itemize}
\item {Utilização:Prov.}
\end{itemize}
O mesmo que \textunderscore chinela\textunderscore  ou \textunderscore tamanca\textunderscore .
(Cp. \textunderscore sóco\textunderscore ^2)
\section{Socado}
\begin{itemize}
\item {Grp. gram.:m.}
\end{itemize}
\begin{itemize}
\item {Utilização:Bras. do S}
\end{itemize}
\begin{itemize}
\item {Proveniência:(De \textunderscore socar\textunderscore )}
\end{itemize}
Lombilho de cabeça alta, feita de coiro cru.
\section{Socadura}
\begin{itemize}
\item {Grp. gram.:f.}
\end{itemize}
Acto de socar.
\section{Socairo}
\begin{itemize}
\item {Grp. gram.:m.}
\end{itemize}
\begin{itemize}
\item {Utilização:Prov.}
\end{itemize}
\begin{itemize}
\item {Proveniência:(De \textunderscore so...\textunderscore  + \textunderscore cairo\textunderscore )}
\end{itemize}
Cabo que sobeja, ao fazerem-se certas manobras náuticas.
Correia, corda ou corrente, que passa por uma argola na extremidade do cabeçalho e cujas pontas se prendem á canga.
Laço, que uma corda dá, em volta dos tornos do carro e ata ou subjuga os volumes que o carro transporta. (Colhido em Arganil)
\section{Socairo}
\begin{itemize}
\item {Grp. gram.:m.}
\end{itemize}
\begin{itemize}
\item {Proveniência:(De \textunderscore sócco\textunderscore  + \textunderscore ?\textunderscore )}
\end{itemize}
Abrigo no sopé de um monte.
Lapa; abrigo.
Sopé.
\section{Socalcar}
\begin{itemize}
\item {Grp. gram.:v. t.}
\end{itemize}
\begin{itemize}
\item {Proveniência:(De \textunderscore so...\textunderscore  + \textunderscore calcar\textunderscore )}
\end{itemize}
Calcar bem; fazer socalcos em.
\section{Socalco}
\begin{itemize}
\item {Grp. gram.:m.}
\end{itemize}
\begin{itemize}
\item {Proveniência:(De \textunderscore socalcar\textunderscore )}
\end{itemize}
Porção mais ou menos plana de terreno, num monte ou numa encosta e sustida por muro ou botaréu.
\section{Socancra}
\begin{itemize}
\item {Grp. gram.:m. ,  f.  e  adj.}
\end{itemize}
Pessôa sonsa; pessôa sovina.
\section{Socapa}
\begin{itemize}
\item {Grp. gram.:f.}
\end{itemize}
\begin{itemize}
\item {Grp. gram.:Loc. adv.}
\end{itemize}
\begin{itemize}
\item {Grp. gram.:Adv.}
\end{itemize}
\begin{itemize}
\item {Proveniência:(De \textunderscore so...\textunderscore  + \textunderscore capa\textunderscore )}
\end{itemize}
Disfarce; manha.
\textunderscore Á socapa\textunderscore , ou \textunderscore de socapa\textunderscore , furtivamente, com disfarce.
(a mesma significação da loc. adv.)
\section{Socado}
\begin{itemize}
\item {Grp. gram.:adj.}
\end{itemize}
\begin{itemize}
\item {Utilização:Bras}
\end{itemize}
\begin{itemize}
\item {Proveniência:(De \textunderscore sócco\textunderscore )}
\end{itemize}
Gordo e baixo, atarracado.
\section{Socar}
\begin{itemize}
\item {Grp. gram.:v. t.}
\end{itemize}
\begin{itemize}
\item {Utilização:Prov.}
\end{itemize}
\begin{itemize}
\item {Utilização:minh.}
\end{itemize}
\begin{itemize}
\item {Utilização:Bras}
\end{itemize}
\begin{itemize}
\item {Proveniência:(De \textunderscore sôco\textunderscore ^1)}
\end{itemize}
Dar sova ou tunda em.
Contundir, pisar.
Espalmar a massa, de que se faz o pão com os punhos cerrados.
Apertar ou calcar (a polvora, no canhão).
Apertar muito (a volta ou o nó de um cabo de navio).
Calcar ou apertar terra debaixo de (uma pedra), para que esta se seguro e se firme.
Pisar no gral.
\section{Socar}
\begin{itemize}
\item {Grp. gram.:v. i.}
\end{itemize}
\begin{itemize}
\item {Utilização:Bras. do N}
\end{itemize}
\begin{itemize}
\item {Proveniência:(De \textunderscore soca\textunderscore ^2)}
\end{itemize}
Brotar; renascer.
\section{Socarrão}
\begin{itemize}
\item {Grp. gram.:m.  e  adj.}
\end{itemize}
(Corr. de \textunderscore sancarrão\textunderscore )
\section{Socata}
\begin{itemize}
\item {Grp. gram.:f.}
\end{itemize}
Ferro manipulado e considerado inutil, especialmente o que serviu em caminhos de ferro, e que se aproveita para ser refundido e entregue de novo ao commercio.
(Cast. \textunderscore zocata\textunderscore )
\section{Socata}
\begin{itemize}
\item {Grp. gram.:f.}
\end{itemize}
\begin{itemize}
\item {Utilização:Prov.}
\end{itemize}
\begin{itemize}
\item {Utilização:beir.}
\end{itemize}
O mesmo que \textunderscore socapa\textunderscore .
\section{Socate}
\begin{itemize}
\item {Grp. gram.:m.}
\end{itemize}
Pequeno sôco, soquete, empurrão. Cf. Filinto, V, 13: VIII, 99 e 198.
(Cp. \textunderscore soquete\textunderscore )
\section{Socava}
\begin{itemize}
\item {Grp. gram.:f.}
\end{itemize}
\begin{itemize}
\item {Proveniência:(De \textunderscore so...\textunderscore  + \textunderscore cava\textunderscore )}
\end{itemize}
Cavidade subterrânea; subterrâneo.
\section{Socavado}
\begin{itemize}
\item {Grp. gram.:adj.}
\end{itemize}
\begin{itemize}
\item {Grp. gram.:M.}
\end{itemize}
\begin{itemize}
\item {Proveniência:(De \textunderscore socavar\textunderscore )}
\end{itemize}
Escavado por baixo.
Desentulho.
\section{Socavão}
\begin{itemize}
\item {Grp. gram.:m.}
\end{itemize}
\begin{itemize}
\item {Proveniência:(De \textunderscore socava\textunderscore )}
\end{itemize}
Grande socava; lapa; esconderijo, abrigo:«\textunderscore o gado busca o socavão da serra\textunderscore ». C. Neto, \textunderscore Saldunes\textunderscore .
\section{Socavar}
\begin{itemize}
\item {Grp. gram.:v. t.}
\end{itemize}
\begin{itemize}
\item {Grp. gram.:V. i.}
\end{itemize}
\begin{itemize}
\item {Proveniência:(De \textunderscore so...\textunderscore  + \textunderscore cavar\textunderscore )}
\end{itemize}
Escavar por baixo.
Fazer escavação.
\section{Soccado}
\begin{itemize}
\item {Grp. gram.:adj.}
\end{itemize}
\begin{itemize}
\item {Utilização:Bras}
\end{itemize}
\begin{itemize}
\item {Proveniência:(De \textunderscore sócco\textunderscore )}
\end{itemize}
Gordo e baixo, atarracado.
\section{Socco}
\begin{itemize}
\item {Grp. gram.:m.}
\end{itemize}
\begin{itemize}
\item {Utilização:Náut.}
\end{itemize}
\begin{itemize}
\item {Proveniência:(Lat. \textunderscore soccus\textunderscore )}
\end{itemize}
Calçado grego, usado por comediantes, ao passo que os trágicos usavam coturno.
Base quadrangular de um pedestal.
Suppedâneo.
Base apparente das paredes dos edifícios.
Apoio do enxertário da vêrga no mastaréu.
\section{Soccorredor}
\begin{itemize}
\item {Grp. gram.:m.  e  adj.}
\end{itemize}
O que soccorre.
\section{Soccorrer}
\begin{itemize}
\item {Grp. gram.:v. t.}
\end{itemize}
\begin{itemize}
\item {Grp. gram.:V. p.}
\end{itemize}
\begin{itemize}
\item {Proveniência:(Do lat. \textunderscore succurrere\textunderscore )}
\end{itemize}
Defender; proteger.
Remediar.
Esmolar.
Procurar auxílio; pedir soccorro.
Valer-se.
\section{Soccorrido}
\begin{itemize}
\item {Grp. gram.:m.}
\end{itemize}
\begin{itemize}
\item {Proveniência:(De \textunderscore soccorrer\textunderscore )}
\end{itemize}
Aquelle que recebeu soccorro ou soccorros.
\section{Soccorrimento}
\begin{itemize}
\item {Grp. gram.:m.}
\end{itemize}
(V.soccorro)
\section{Soccorro}
\begin{itemize}
\item {Grp. gram.:m.}
\end{itemize}
Acto ou effeito de soccorrer; auxílio.
Protecção, amparo.
\section{Socegar}
\textunderscore v. t.\textunderscore  e \textunderscore i.\textunderscore  (e der.)
(V. \textunderscore sossegar\textunderscore , etc.)
\section{Socha}
\begin{itemize}
\item {Grp. gram.:f.}
\end{itemize}
\begin{itemize}
\item {Utilização:Prov.}
\end{itemize}
\begin{itemize}
\item {Utilização:alent.}
\end{itemize}
O mesmo que \textunderscore choça\textunderscore ^2; cabana.
(Por \textunderscore çocha\textunderscore , metáth. de \textunderscore choça\textunderscore )
\section{Sochantrado}
\begin{itemize}
\item {Grp. gram.:m.}
\end{itemize}
Cargo de sochantre.
\section{Sochantre}
\begin{itemize}
\item {Grp. gram.:m.}
\end{itemize}
\begin{itemize}
\item {Proveniência:(De \textunderscore so...\textunderscore  + \textunderscore chantre\textunderscore )}
\end{itemize}
Aquelle que faz as vezes de chantre.
\section{Sochantrear}
\begin{itemize}
\item {Grp. gram.:v. i.}
\end{itemize}
Exercer o cargo de sochantre.
\section{Sochão}
\begin{itemize}
\item {Grp. gram.:m.}
\end{itemize}
\begin{itemize}
\item {Utilização:T. do Alto Minho}
\end{itemize}
\begin{itemize}
\item {Proveniência:(De \textunderscore so...\textunderscore  + \textunderscore chão\textunderscore , ou de \textunderscore socha\textunderscore )}
\end{itemize}
Espécie de abrigo, escavado na encosta de um monte.
\section{Sochear}
\begin{itemize}
\item {Grp. gram.:v. t.}
\end{itemize}
\begin{itemize}
\item {Utilização:Prov.}
\end{itemize}
\begin{itemize}
\item {Utilização:trasm.}
\end{itemize}
\begin{itemize}
\item {Utilização:dur.}
\end{itemize}
\begin{itemize}
\item {Proveniência:(De \textunderscore so...\textunderscore  + \textunderscore cheio\textunderscore )}
\end{itemize}
Escavar em roda de (videiras), para fazer a mergulhia de uma ou mais varas.
\section{Socheio}
\begin{itemize}
\item {Grp. gram.:m.}
\end{itemize}
\begin{itemize}
\item {Utilização:Prov.}
\end{itemize}
\begin{itemize}
\item {Utilização:trasm.}
\end{itemize}
\begin{itemize}
\item {Proveniência:(De \textunderscore sochear\textunderscore )}
\end{itemize}
Escavação, ao lado da valla da bacellagem, para se juntar terra que caia sôbre o plantio.
\section{Sociabilidade}
\begin{itemize}
\item {Grp. gram.:f.}
\end{itemize}
\begin{itemize}
\item {Proveniência:(Do lat. \textunderscore sociabilis\textunderscore )}
\end{itemize}
Qualidade do que é sociável.
Tendência para viver em sociedade.
Modos de quem vive em sociedade.
\section{Sociabilizar}
\begin{itemize}
\item {Grp. gram.:v. t.}
\end{itemize}
\begin{itemize}
\item {Proveniência:(Do lat. \textunderscore sociabilis\textunderscore )}
\end{itemize}
Tornar sociável.
Reunir em sociedade.
\section{Social}
\begin{itemize}
\item {Grp. gram.:adj.}
\end{itemize}
\begin{itemize}
\item {Proveniência:(Lat. \textunderscore socialis\textunderscore )}
\end{itemize}
Relativo á sociedade; sociável.
Que convém á sociedade.
\section{Socialismo}
\begin{itemize}
\item {Grp. gram.:m.}
\end{itemize}
\begin{itemize}
\item {Proveniência:(De \textunderscore social\textunderscore )}
\end{itemize}
Cada um dos vários systemas, que tem por base a reforma social.
\section{Socialista}
\begin{itemize}
\item {Grp. gram.:adj.}
\end{itemize}
\begin{itemize}
\item {Grp. gram.:M.}
\end{itemize}
\begin{itemize}
\item {Proveniência:(De \textunderscore social\textunderscore )}
\end{itemize}
Relativo ao socialismo.
Sectário do socialismo.
\section{Socialização}
\begin{itemize}
\item {Grp. gram.:f.}
\end{itemize}
Acto ou effeito de socializar.
\section{Socializar}
\begin{itemize}
\item {Grp. gram.:v. t.}
\end{itemize}
Tornar social; sociabilizar.
\section{Socialmente}
\begin{itemize}
\item {Grp. gram.:adv.}
\end{itemize}
De modo social; em sociedade.
\section{Sociar}
\begin{itemize}
\item {Grp. gram.:v. i.}
\end{itemize}
\begin{itemize}
\item {Utilização:P. us.}
\end{itemize}
\begin{itemize}
\item {Proveniência:(Lat. \textunderscore sociare\textunderscore )}
\end{itemize}
O mesmo que [[associar-se|associar]].
\section{Sociável}
\begin{itemize}
\item {Grp. gram.:adj.}
\end{itemize}
\begin{itemize}
\item {Utilização:Fig.}
\end{itemize}
\begin{itemize}
\item {Proveniência:(Do lat. \textunderscore sociabilis\textunderscore )}
\end{itemize}
Que se póde associar.
Próprio para viver em sociedade.
Que tende para viver em sociedade.
Civilizado.
Urbano; delicado.
\section{Sociedade}
\begin{itemize}
\item {Grp. gram.:f.}
\end{itemize}
\begin{itemize}
\item {Proveniência:(Do lat. \textunderscore societas\textunderscore )}
\end{itemize}
Reunião de homens, que têm a mesma origem, as mesmas leis e os mesmos costumes.
Estado social.
Associação.
Aggremiação.
Reunião de animaes, que têm os mesmos interesses ou são destinados ao mesmo fim.
Parceria, participação: \textunderscore têr sociedade numa empresa\textunderscore .
Relações ou frequência habitual de pessôas.
Casa, em que se reúnem os membros de uma aggremiação qualquer.
\section{Societariado}
\begin{itemize}
\item {Grp. gram.:m.}
\end{itemize}
Qualidade do que é societário.
Conjunto de societários.
\section{Societariamente}
\begin{itemize}
\item {Grp. gram.:adv.}
\end{itemize}
De modo societário; socialmente.
\section{Societário}
\begin{itemize}
\item {Grp. gram.:m.  e  adj.}
\end{itemize}
\begin{itemize}
\item {Proveniência:(Do lat. \textunderscore societas\textunderscore )}
\end{itemize}
Membro de uma sociedade.
Sócio.
Diz-se do animal, que vive em sociedade.
\section{Sócio}
\begin{itemize}
\item {Grp. gram.:m.}
\end{itemize}
\begin{itemize}
\item {Grp. gram.:Adj.}
\end{itemize}
\begin{itemize}
\item {Proveniência:(Lat. \textunderscore socius\textunderscore )}
\end{itemize}
Membro de uma sociedade.
Aquelle que se associa com outro ou outros numa empresa, de que espera auferir lucros.
Parceiro.
Cúmplice.
O mesmo que [[associado|associar]].
\section{Sociocracia}
\begin{itemize}
\item {Grp. gram.:f.}
\end{itemize}
\begin{itemize}
\item {Utilização:bras}
\end{itemize}
\begin{itemize}
\item {Utilização:Neol.}
\end{itemize}
\begin{itemize}
\item {Proveniência:(Do lat. \textunderscore socius\textunderscore  + gr. \textunderscore kratein\textunderscore )}
\end{itemize}
Govêrno social.
\section{Sociocrático}
\begin{itemize}
\item {Grp. gram.:adj.}
\end{itemize}
Relativo á sociocracia.
\section{Sociofilia}
\begin{itemize}
\item {Grp. gram.:f.}
\end{itemize}
Qualidade de sociófilo.
\section{Sociófilo}
\begin{itemize}
\item {Grp. gram.:m.}
\end{itemize}
\begin{itemize}
\item {Proveniência:(Do lat. \textunderscore socius\textunderscore  + gr. \textunderscore logos\textunderscore )}
\end{itemize}
Aquele que é amigo da sociedade.
\section{Sociogenia}
\begin{itemize}
\item {Grp. gram.:f.}
\end{itemize}
\begin{itemize}
\item {Proveniência:(Do lat. \textunderscore socius\textunderscore  + gr. \textunderscore gene\textunderscore )}
\end{itemize}
Estudo sôbre a formação da sociedade.
\section{Sociolatria}
\begin{itemize}
\item {Grp. gram.:f.}
\end{itemize}
\begin{itemize}
\item {Proveniência:(Do lat. \textunderscore socius\textunderscore  + gr. \textunderscore latrein\textunderscore )}
\end{itemize}
Adoração da humanidade, espécie de religião inventada por Comite.
\section{Sociologia}
\begin{itemize}
\item {Grp. gram.:f.}
\end{itemize}
\begin{itemize}
\item {Proveniência:(Do lat. \textunderscore socius\textunderscore  + gr. \textunderscore logos\textunderscore )}
\end{itemize}
Sciência, que trata da constituição e desenvolvimento das sociedades humanas.
\section{Sociologicamente}
\begin{itemize}
\item {Grp. gram.:adv.}
\end{itemize}
De modo sociológico.
Segundo os principios da Sociologia.
\section{Sociológico}
\begin{itemize}
\item {Grp. gram.:adj.}
\end{itemize}
Relativo á Sociologia.
\section{Sociologista}
\begin{itemize}
\item {Grp. gram.:m.}
\end{itemize}
Tratadista de Sociologia.
\section{Sociólogo}
\begin{itemize}
\item {Grp. gram.:m.}
\end{itemize}
Indivíduo perito em Sociologia.
\section{Sociophilia}
\begin{itemize}
\item {Grp. gram.:f.}
\end{itemize}
Qualidade de socióphilo.
\section{Socióphilo}
\begin{itemize}
\item {Grp. gram.:m.}
\end{itemize}
\begin{itemize}
\item {Proveniência:(Do lat. \textunderscore socius\textunderscore  + gr. \textunderscore logos\textunderscore )}
\end{itemize}
Aquelle que é amigo da sociedade.
\section{Sôco}
\textunderscore m.\textunderscore 
Pancada com a mão fechada.
Murro.
Mossa, que um pião faz em outro, num jôgo de rapazes.
\section{Sôco!}
\begin{itemize}
\item {Grp. gram.:interj.}
\end{itemize}
\begin{itemize}
\item {Utilização:Bras}
\end{itemize}
(Designa \textunderscore reprovação\textunderscore )
\section{Sóco}
\begin{itemize}
\item {Grp. gram.:m.}
\end{itemize}
\begin{itemize}
\item {Utilização:Náut.}
\end{itemize}
\begin{itemize}
\item {Proveniência:(Lat. \textunderscore soccus\textunderscore )}
\end{itemize}
Calçado grego, usado por comediantes, ao passo que os trágicos usavam coturno.
Base quadrangular de um pedestal.
Suppedâneo.
Base apparente das paredes dos edifícios.
Apoio do enxertário da vêrga no mastaréu.
\section{Sóco}
\begin{itemize}
\item {Grp. gram.:m.}
\end{itemize}
\begin{itemize}
\item {Utilização:Ant.}
\end{itemize}
O mesmo que \textunderscore tamanco\textunderscore .
O mesmo que \textunderscore chapim\textunderscore ^1.
(Cp. cast. \textunderscore zueco\textunderscore )
\section{Sòcó}
\begin{itemize}
\item {Grp. gram.:m.}
\end{itemize}
\begin{itemize}
\item {Utilização:Bras}
\end{itemize}
Ave pernalta, escura, de pescoço longo.
\section{Socobói}
\begin{itemize}
\item {Grp. gram.:m.}
\end{itemize}
\begin{itemize}
\item {Utilização:Bras}
\end{itemize}
Ave pernalta.
\section{Soçobra}
\begin{itemize}
\item {Grp. gram.:f.}
\end{itemize}
O mesmo que \textunderscore soçôbro\textunderscore .
\section{Soçobrar}
\begin{itemize}
\item {Grp. gram.:v. t.}
\end{itemize}
\begin{itemize}
\item {Utilização:Fig.}
\end{itemize}
\begin{itemize}
\item {Grp. gram.:V. i.}
\end{itemize}
\begin{itemize}
\item {Utilização:Fig.}
\end{itemize}
Revolver, de baixo para cima e de cima para baixo.
Afundar.
Fazer naufragar.
Perturbar.
Afundar-se.
Naufragar.
Perder-se.
Correr perigo.
Desgraçar-se.
Aniquilar-se.
(Cast. \textunderscore zozobrar\textunderscore )
\section{Socobreta}
\begin{itemize}
\item {fónica:brê}
\end{itemize}
\begin{itemize}
\item {Grp. gram.:f.}
\end{itemize}
\begin{itemize}
\item {Utilização:Ant.}
\end{itemize}
\begin{itemize}
\item {Utilização:Fam.}
\end{itemize}
Mau agoiro; enguiço.
Embirração.
\section{Soçôbro}
\begin{itemize}
\item {Grp. gram.:m.}
\end{itemize}
Acto ou effeito de \textunderscore soçobrar\textunderscore .
\section{Socolipé}
\begin{itemize}
\item {Grp. gram.:m.}
\end{itemize}
\begin{itemize}
\item {Utilização:Prov.}
\end{itemize}
\begin{itemize}
\item {Utilização:beir.}
\end{itemize}
O mesmo que \textunderscore pospelo\textunderscore .
\section{Socolor}
\begin{itemize}
\item {Grp. gram.:adv.}
\end{itemize}
O mesmo que [[sob]]-[[color]].
\section{Socórdia}
\begin{itemize}
\item {Grp. gram.:f.}
\end{itemize}
\begin{itemize}
\item {Utilização:Des.}
\end{itemize}
\begin{itemize}
\item {Proveniência:(Lat. \textunderscore socordiae\textunderscore )}
\end{itemize}
Cobardia.
Inércia. Cf. Vieira, XI, 258.
\section{Socornar}
\begin{itemize}
\item {Grp. gram.:v. t.}
\end{itemize}
\begin{itemize}
\item {Utilização:Ant.}
\end{itemize}
\begin{itemize}
\item {Proveniência:(De \textunderscore so...\textunderscore  + \textunderscore corno\textunderscore )}
\end{itemize}
Fazer baixar a cabeça de.
\section{Socorredor}
\begin{itemize}
\item {Grp. gram.:m.  e  adj.}
\end{itemize}
O que socorre.
\section{Socorrer}
\begin{itemize}
\item {Grp. gram.:v. t.}
\end{itemize}
\begin{itemize}
\item {Grp. gram.:V. p.}
\end{itemize}
\begin{itemize}
\item {Proveniência:(Do lat. \textunderscore succurrere\textunderscore )}
\end{itemize}
Defender; proteger.
Remediar.
Esmolar.
Procurar auxílio; pedir socorro.
Valer-se.
\section{Socorrido}
\begin{itemize}
\item {Grp. gram.:m.}
\end{itemize}
\begin{itemize}
\item {Proveniência:(De \textunderscore socorrer\textunderscore )}
\end{itemize}
Aquele que recebeu socorro ou socorros.
\section{Socorro}
\begin{itemize}
\item {Grp. gram.:m.}
\end{itemize}
Acto ou efeito de socorrer; auxílio.
Protecção, amparo.
\section{Sócos-da-raínha}
\begin{itemize}
\item {Grp. gram.:m. pl.}
\end{itemize}
Antigo tributo, que os moradores de Sintra pagavam ás raínhas de Portugal.
\section{Socotorino}
\begin{itemize}
\item {Grp. gram.:adj.}
\end{itemize}
Que é de Socotorá ou veio de Socotorá:«\textunderscore ...o aloes socotorino...\textunderscore »Garcia Orta, \textunderscore Coll.\textunderscore  II.
\section{Socovão}
\begin{itemize}
\item {Grp. gram.:m.}
\end{itemize}
\begin{itemize}
\item {Proveniência:(De \textunderscore so...\textunderscore  + \textunderscore covão\textunderscore )}
\end{itemize}
Subterrâneo, por debaixo de uma casa; socavão.
\section{Socraticamente}
\begin{itemize}
\item {Grp. gram.:adv.}
\end{itemize}
De modo socrático.
Segundo o systema ou os processos de Sócrates.
\section{Socrático}
\begin{itemize}
\item {Grp. gram.:adj.}
\end{itemize}
\begin{itemize}
\item {Proveniência:(Lat. \textunderscore socraticus\textunderscore )}
\end{itemize}
Relativo a Sócrates ou á sua doutrina.
Diz-se do méthodo pedagógico, que, primeiramente, leva o alumno ao conhecimento do próprio êrro e, depois, ao descobrimento e acquisição da verdade.
\section{Socrestar}
\begin{itemize}
\item {Grp. gram.:v. t.}
\end{itemize}
\begin{itemize}
\item {Utilização:Ant.}
\end{itemize}
\begin{itemize}
\item {Proveniência:(De \textunderscore so...\textunderscore  + \textunderscore crestar\textunderscore ^2)}
\end{itemize}
Fazer cresta ligeiramente:«\textunderscore a solicita abelha vai socrestar o suco da flor.\textunderscore »\textunderscore Luz e Calor\textunderscore .
\section{Socresteiro}
\begin{itemize}
\item {Grp. gram.:adj.}
\end{itemize}
\begin{itemize}
\item {Utilização:Prov.}
\end{itemize}
\begin{itemize}
\item {Utilização:minh.}
\end{itemize}
\begin{itemize}
\item {Proveniência:(De \textunderscore socresto\textunderscore )}
\end{itemize}
O mesmo que \textunderscore comilão\textunderscore .
\section{Socresto}
\begin{itemize}
\item {Grp. gram.:m.}
\end{itemize}
\begin{itemize}
\item {Utilização:Prov.}
\end{itemize}
O mesmo que \textunderscore sequestro\textunderscore .
\section{Soda}
\begin{itemize}
\item {Grp. gram.:f.}
\end{itemize}
\begin{itemize}
\item {Utilização:Fam.}
\end{itemize}
\begin{itemize}
\item {Proveniência:(Do b. lat. \textunderscore solda\textunderscore )}
\end{itemize}
Gênero de plantas, da fam. das salsoláceas.
Óxydo de sódio.
Carbonato, que tem por base êsse óxydo.
Carbonato de potassa.
Combinação do ácido tartárico com o bicarbonato de soda, a qual se usa como refrigerante.
\textunderscore Soda cáustica\textunderscore , hydrato de sódio.
\section{Soda}
\begin{itemize}
\item {Grp. gram.:f.}
\end{itemize}
\begin{itemize}
\item {Utilização:Med.}
\end{itemize}
\begin{itemize}
\item {Utilização:Ant.}
\end{itemize}
\begin{itemize}
\item {Proveniência:(T. ar.)}
\end{itemize}
O mesmo que \textunderscore cephalalgia\textunderscore .
\section{Sodalício}
\begin{itemize}
\item {Grp. gram.:m.}
\end{itemize}
\begin{itemize}
\item {Proveniência:(Lat. \textunderscore sodalicium\textunderscore )}
\end{itemize}
Reunião de pessôas que vivem em commum.
\section{Sodálitho}
\begin{itemize}
\item {Grp. gram.:m.}
\end{itemize}
Silicato de alumina e de soda.
\section{Sodálito}
\begin{itemize}
\item {Grp. gram.:m.}
\end{itemize}
Silicato de alumina e de soda.
\section{Sodar}
\begin{itemize}
\item {Grp. gram.:v. t.}
\end{itemize}
Misturar com soda. Cf. \textunderscore Techn. Rur.\textunderscore , 76.
\section{Sódico}
\begin{itemize}
\item {Grp. gram.:adj.}
\end{itemize}
Relativo á soda^1.
\section{Sódio}
\begin{itemize}
\item {Grp. gram.:m.}
\end{itemize}
\begin{itemize}
\item {Proveniência:(De \textunderscore soda\textunderscore )}
\end{itemize}
Corpo metállico, de que a soda é o óxydo.
\section{Sodomia}
\begin{itemize}
\item {Grp. gram.:f.}
\end{itemize}
\begin{itemize}
\item {Proveniência:(De \textunderscore Sodoma\textunderscore , n. p.)}
\end{itemize}
Acto sensual contra a natureza.
\section{Sodómico}
\begin{itemize}
\item {Grp. gram.:adj.}
\end{itemize}
Relativo á sodomia.
\section{Sodomita}
\begin{itemize}
\item {Grp. gram.:m.}
\end{itemize}
\begin{itemize}
\item {Proveniência:(Lat. \textunderscore sodomitae\textunderscore )}
\end{itemize}
Aquelle que pratíca sodomia.
\section{Sodomítico}
\begin{itemize}
\item {Grp. gram.:adj.}
\end{itemize}
Relativo á sodomia ou aos sodomitas.
\section{Sodra}
\begin{itemize}
\item {Grp. gram.:f.}
\end{itemize}
Sulco, nas coxas de alguns cavallos.
\section{Soedade}
\begin{itemize}
\item {Grp. gram.:f.}
\end{itemize}
\begin{itemize}
\item {Utilização:Ant.}
\end{itemize}
O mesmo que \textunderscore soledade\textunderscore .
(Contr. de \textunderscore soledade\textunderscore )
\section{Soedor}
\begin{itemize}
\item {fónica:so-ê}
\end{itemize}
\begin{itemize}
\item {Grp. gram.:adj.}
\end{itemize}
\begin{itemize}
\item {Proveniência:(De \textunderscore soer\textunderscore )}
\end{itemize}
Que sói.
Acostumado.
\section{Soeiras}
\begin{itemize}
\item {Grp. gram.:f. pl.}
\end{itemize}
\begin{itemize}
\item {Utilização:Ant.}
\end{itemize}
\begin{itemize}
\item {Proveniência:(De \textunderscore soer\textunderscore )}
\end{itemize}
Costumeiras, usos.
\section{Soenga}
\begin{itemize}
\item {Grp. gram.:f.}
\end{itemize}
\begin{itemize}
\item {Utilização:Prov.}
\end{itemize}
\begin{itemize}
\item {Utilização:trasm.}
\end{itemize}
Forno para loiça.
\section{Soer}
\begin{itemize}
\item {Grp. gram.:v. i.}
\end{itemize}
\begin{itemize}
\item {Utilização:Des.}
\end{itemize}
\begin{itemize}
\item {Proveniência:(Do lat. \textunderscore solere\textunderscore )}
\end{itemize}
O mesmo que \textunderscore costumar\textunderscore .
\section{Soerguer}
\begin{itemize}
\item {Grp. gram.:v. t.}
\end{itemize}
\begin{itemize}
\item {Proveniência:(De \textunderscore so...\textunderscore  + \textunderscore erguer\textunderscore )}
\end{itemize}
Erguer um tanto; solevar.
\section{Soez}
\begin{itemize}
\item {Grp. gram.:adj.}
\end{itemize}
Vil.
Reles; torpe.
(Da mesma or. que \textunderscore sujo\textunderscore . Cp. \textunderscore sujo\textunderscore )
\section{Sofá}
\begin{itemize}
\item {Grp. gram.:m.}
\end{itemize}
\begin{itemize}
\item {Utilização:Des.}
\end{itemize}
\begin{itemize}
\item {Proveniência:(Fr. \textunderscore sofa\textunderscore )}
\end{itemize}
Canapé estofado.
Estrado alto e com tapete.
\section{Sofenha}
\begin{itemize}
\item {Grp. gram.:f.}
\end{itemize}
Variedade de figueira algarvia.
\section{Sofenho}
\begin{itemize}
\item {Grp. gram.:m.}
\end{itemize}
Casta de figo, muito apreciada.
\section{Sofeno}
\begin{itemize}
\item {Grp. gram.:m.}
\end{itemize}
\begin{itemize}
\item {Utilização:Prov.}
\end{itemize}
\begin{itemize}
\item {Utilização:alg.}
\end{itemize}
\begin{itemize}
\item {Grp. gram.:m.}
\end{itemize}
Casta de figo, muito apreciada.
Variedade de figos doces e branquícentos.
\section{Soffito}
\begin{itemize}
\item {Grp. gram.:m.}
\end{itemize}
\begin{itemize}
\item {Proveniência:(It. \textunderscore soffito\textunderscore )}
\end{itemize}
Face com ornatos, por baixo de uma architrave.
\section{Soffreada}
\begin{itemize}
\item {Grp. gram.:f.}
\end{itemize}
O mesmo que \textunderscore soffreamento\textunderscore .
\section{Soffreadura}
\begin{itemize}
\item {Grp. gram.:f.}
\end{itemize}
Acto ou effeito de soffrear.
\section{Soffreamento}
\begin{itemize}
\item {Grp. gram.:m.}
\end{itemize}
Acto ou effeito de soffrear.
\section{Soffrear}
\begin{itemize}
\item {Grp. gram.:v. t.}
\end{itemize}
\begin{itemize}
\item {Utilização:Fig.}
\end{itemize}
\begin{itemize}
\item {Proveniência:(Do lat. \textunderscore suffrenare\textunderscore )}
\end{itemize}
Sustar ou modificar a andadura de (uma cavalgadura), puxando ou retesando a rédea.
Refrear.
Reprimir; conter: \textunderscore soffrear ímpetos de vingança\textunderscore .
Corrigir.
\section{Soffreável}
\begin{itemize}
\item {Grp. gram.:adj.}
\end{itemize}
Que se póde soffrear.
\section{Soffredor}
\begin{itemize}
\item {Grp. gram.:adj.}
\end{itemize}
\begin{itemize}
\item {Grp. gram.:M.}
\end{itemize}
Que soffre.
Aquelle que soffre.
\section{Soffrença}
\begin{itemize}
\item {Grp. gram.:f.}
\end{itemize}
\begin{itemize}
\item {Utilização:Ant.}
\end{itemize}
O mesmo que \textunderscore soffrimento\textunderscore .
\section{Soffrer}
\begin{itemize}
\item {Grp. gram.:v. t.}
\end{itemize}
\begin{itemize}
\item {Grp. gram.:V. i.}
\end{itemize}
\begin{itemize}
\item {Grp. gram.:M.}
\end{itemize}
\begin{itemize}
\item {Utilização:Bras}
\end{itemize}
\begin{itemize}
\item {Proveniência:(Do lat. hyp. \textunderscore sufferere\textunderscore , de \textunderscore sufferre\textunderscore )}
\end{itemize}
Supportar; tolerar: \textunderscore soffrer afrontas\textunderscore .
Padecer com paciência.
Padecer.
Têr dôres.
Pássaro amarelo, de cabeça, cauda, pescoço e asas pretas, e cujo canto imita a pronúncia do seu nome.
\section{Soffridamente}
\begin{itemize}
\item {Grp. gram.:adv.}
\end{itemize}
De modo soffrido; com resignação; pacientemente.
\section{Soffrido}
\begin{itemize}
\item {Grp. gram.:adj.}
\end{itemize}
\begin{itemize}
\item {Proveniência:(De \textunderscore soffrer\textunderscore )}
\end{itemize}
Que soffre com paciência; paciente.
\section{Soffrimento}
\begin{itemize}
\item {Grp. gram.:m.}
\end{itemize}
Acto ou effeito de soffrer.
Padecimento.
Dôr.
Amargura.
Paciência.
Desastre.
\section{Soffrível}
\begin{itemize}
\item {Grp. gram.:adj.}
\end{itemize}
Que se póde soffrer.
Tolerável.
Quási sufficiente.
Razoável.
Que está acima de medíocre.
\section{Soffrivelmente}
\begin{itemize}
\item {Grp. gram.:adv.}
\end{itemize}
De modo soffrível.
Quási sufficientemente; razoavelmente.
\section{Sofio}
\begin{itemize}
\item {Grp. gram.:m.}
\end{itemize}
(?):«\textunderscore poem pés seguros, como passavante de sofio.\textunderscore »\textunderscore Aulegrafia\textunderscore , 89.
\section{Sofito}
\begin{itemize}
\item {Grp. gram.:m.}
\end{itemize}
\begin{itemize}
\item {Proveniência:(It. \textunderscore soffito\textunderscore )}
\end{itemize}
Face com ornatos, por baixo de uma arquitrave.
\section{Soforar}
\begin{itemize}
\item {Grp. gram.:v. t.}
\end{itemize}
\begin{itemize}
\item {Utilização:Ant.}
\end{itemize}
\begin{itemize}
\item {Proveniência:(Do lat. \textunderscore sub\textunderscore  + \textunderscore forare\textunderscore )}
\end{itemize}
Esporear; fustigar.
\section{Sofragante}
\begin{itemize}
\item {Grp. gram.:m.}
\end{itemize}
\begin{itemize}
\item {Utilização:Prov.}
\end{itemize}
\begin{itemize}
\item {Utilização:beir.}
\end{itemize}
O mesmo que \textunderscore comenos\textunderscore : \textunderscore neste sofragante appareceu a polícia\textunderscore .
(Provavelmente de \textunderscore so...\textunderscore  + \textunderscore flagrante\textunderscore )
\section{Sofraldar}
\begin{itemize}
\item {Grp. gram.:v. t.}
\end{itemize}
\begin{itemize}
\item {Utilização:Fig.}
\end{itemize}
\begin{itemize}
\item {Proveniência:(De \textunderscore so...\textunderscore  + \textunderscore fralda\textunderscore )}
\end{itemize}
Erguer a fralda de.
Solevar qualquer objecto, para descobrir outro que está debaixo daquelle.
\section{Sofreada}
\begin{itemize}
\item {Grp. gram.:f.}
\end{itemize}
O mesmo que \textunderscore sofreamento\textunderscore .
\section{Sofreadura}
\begin{itemize}
\item {Grp. gram.:f.}
\end{itemize}
Acto ou efeito de sofrear.
\section{Sofreamento}
\begin{itemize}
\item {Grp. gram.:m.}
\end{itemize}
Acto ou efeito de sofrear.
\section{Sofrear}
\begin{itemize}
\item {Grp. gram.:v. t.}
\end{itemize}
\begin{itemize}
\item {Utilização:Fig.}
\end{itemize}
\begin{itemize}
\item {Proveniência:(Do lat. \textunderscore suffrenare\textunderscore )}
\end{itemize}
Sustar ou modificar a andadura de (uma cavalgadura), puxando ou retesando a rédea.
Refrear.
Reprimir; conter: \textunderscore sofrear ímpetos de vingança\textunderscore .
Corrigir.
\section{Sofreável}
\begin{itemize}
\item {Grp. gram.:adj.}
\end{itemize}
Que se póde sofrear.
\section{Sofredor}
\begin{itemize}
\item {Grp. gram.:adj.}
\end{itemize}
\begin{itemize}
\item {Grp. gram.:M.}
\end{itemize}
Que sofre.
Aquele que sofre.
\section{Sofregamente}
\begin{itemize}
\item {Grp. gram.:adv.}
\end{itemize}
De modo sôfrego; com sofreguidão.
Ambiciosamente.
\section{Sôfrego}
\begin{itemize}
\item {Grp. gram.:adj.}
\end{itemize}
Apressado em comer ou beber.
Ávido.
Ambicioso.
Impaciente.
\section{Sofreguice}
\begin{itemize}
\item {Grp. gram.:f.}
\end{itemize}
\begin{itemize}
\item {Utilização:Pop.}
\end{itemize}
Acto, modos ou qualidade do que é sôfrego.
Ambição; impaciência.
\section{Sofreguidão}
\begin{itemize}
\item {Grp. gram.:f.}
\end{itemize}
Acto, modos ou qualidade do que é sôfrego.
Ambição; impaciência.
\section{Sofrenaço}
\begin{itemize}
\item {Grp. gram.:m.}
\end{itemize}
\begin{itemize}
\item {Utilização:Bras. do S}
\end{itemize}
Acto de sofrenar.
\section{Sofrenar}
\begin{itemize}
\item {Grp. gram.:v. t.}
\end{itemize}
\begin{itemize}
\item {Utilização:Bras. do S}
\end{itemize}
\begin{itemize}
\item {Proveniência:(De \textunderscore so...\textunderscore  + lat. \textunderscore frenare\textunderscore )}
\end{itemize}
Sofrear (o cavallo), para o fazer parar ou recuar.
\section{Sofrença}
\begin{itemize}
\item {Grp. gram.:f.}
\end{itemize}
\begin{itemize}
\item {Utilização:Ant.}
\end{itemize}
O mesmo que \textunderscore sofrimento\textunderscore .
\section{Sofrer}
\begin{itemize}
\item {Grp. gram.:v. t.}
\end{itemize}
\begin{itemize}
\item {Grp. gram.:V. i.}
\end{itemize}
\begin{itemize}
\item {Grp. gram.:M.}
\end{itemize}
\begin{itemize}
\item {Utilização:Bras}
\end{itemize}
\begin{itemize}
\item {Proveniência:(Do lat. hyp. \textunderscore sufferere\textunderscore , de \textunderscore sufferre\textunderscore )}
\end{itemize}
Suportar; tolerar: \textunderscore sofrer afrontas\textunderscore .
Padecer com paciência.
Padecer.
Têr dôres.
Pássaro amarelo, de cabeça, cauda, pescoço e asas pretas, e cujo canto imita a pronúncia do seu nome.
\section{Sofridamente}
\begin{itemize}
\item {Grp. gram.:adv.}
\end{itemize}
De modo sofrido; com resignação; pacientemente.
\section{Sofrido}
\begin{itemize}
\item {Grp. gram.:adj.}
\end{itemize}
\begin{itemize}
\item {Proveniência:(De \textunderscore sofrer\textunderscore )}
\end{itemize}
Que sofre com paciência; paciente.
\section{Sofrimento}
\begin{itemize}
\item {Grp. gram.:m.}
\end{itemize}
Acto ou efeito de sofrer.
Padecimento.
Dôr.
Amargura.
Paciência.
Desastre.
\section{Sofrível}
\begin{itemize}
\item {Grp. gram.:adj.}
\end{itemize}
Que se póde sofrer.
Tolerável.
Quási suficiente.
Razoável.
Que está acima de medíocre.
\section{Sofrivelmente}
\begin{itemize}
\item {Grp. gram.:adv.}
\end{itemize}
De modo sofrível.
Quási suficientemente; razoavelmente.
\section{Soga}
\begin{itemize}
\item {Grp. gram.:f.}
\end{itemize}
(no Minho, \textunderscore sôga\textunderscore )
Corda de esparto.
Corda grossa.
Tira de coiro, cujas extremidades se prendem ás pontas do boi, e pela qual êlle é puxado ou guiado.
Sulco para conducção de águas.
(Do \textunderscore vasconço\textunderscore ?)
\section{Sogar}
\begin{itemize}
\item {Grp. gram.:v. t.}
\end{itemize}
Prender com soga.
\section{Sogigar}
\begin{itemize}
\item {Grp. gram.:v. t. (e der.)}
\end{itemize}
\begin{itemize}
\item {Utilização:Bras. do N}
\end{itemize}
Fórma ant. de \textunderscore subjugar\textunderscore , etc., ainda us. na Baía:«\textunderscore ...pois amor se sogiga...\textunderscore »\textunderscore Eufrosina\textunderscore , 27.«\textunderscore ...a quem com todo o seu poder não sogigava.\textunderscore »Filinto, \textunderscore D. Man.\textunderscore , I, 196.
Segurar, prender.
\section{Sogra}
\begin{itemize}
\item {Grp. gram.:f.}
\end{itemize}
\begin{itemize}
\item {Proveniência:(Do b. lat. \textunderscore socra\textunderscore )}
\end{itemize}
(\textunderscore Fem.\textunderscore  de \textunderscore sogro\textunderscore )
\section{Sogra}
\begin{itemize}
\item {Grp. gram.:f.}
\end{itemize}
O mesmo que \textunderscore rodilha\textunderscore ^1.
\section{Sogro}
\begin{itemize}
\item {fónica:sô}
\end{itemize}
\begin{itemize}
\item {Grp. gram.:m.}
\end{itemize}
\begin{itemize}
\item {Proveniência:(Do b. lat. \textunderscore socrus\textunderscore )}
\end{itemize}
Diz se de um indivíduo, em relação ao cônjuge de seu filho ou filha.
\section{Soguilha}
\begin{itemize}
\item {Grp. gram.:f.}
\end{itemize}
\begin{itemize}
\item {Utilização:P. us.}
\end{itemize}
Torçal, com que se enfeitam vestidos.
(Cast. \textunderscore soguilla\textunderscore )
\section{Soí}
\begin{itemize}
\item {Grp. gram.:m.}
\end{itemize}
\begin{itemize}
\item {Utilização:Bras. do N}
\end{itemize}
(V.soim)
\section{Sóia}
\begin{itemize}
\item {Grp. gram.:f.}
\end{itemize}
Peixe marítimo de Pernambuco.
\section{Soidade}
\begin{itemize}
\item {Grp. gram.:f.}
\end{itemize}
\begin{itemize}
\item {Utilização:Prov.}
\end{itemize}
\begin{itemize}
\item {Utilização:trasm.}
\end{itemize}
\begin{itemize}
\item {Utilização:Ant.}
\end{itemize}
O mesmo que \textunderscore soedade\textunderscore .
\section{Soidão}
\begin{itemize}
\item {Grp. gram.:f.}
\end{itemize}
\begin{itemize}
\item {Utilização:Ant.}
\end{itemize}
O mesmo que \textunderscore solidão\textunderscore .
(Contr. de \textunderscore solidão\textunderscore )
\section{Soído}
\begin{itemize}
\item {Grp. gram.:m.}
\end{itemize}
O mesmo que \textunderscore sonido\textunderscore .
(Contr. de \textunderscore sonido\textunderscore )
\section{Soidoso}
\begin{itemize}
\item {Grp. gram.:adj.}
\end{itemize}
\begin{itemize}
\item {Utilização:Ant.}
\end{itemize}
O mesmo que \textunderscore saudoso\textunderscore . Cf. Camões, elegia III.
\section{Soieira}
\begin{itemize}
\item {Grp. gram.:f.}
\end{itemize}
\begin{itemize}
\item {Utilização:Ant.}
\end{itemize}
\begin{itemize}
\item {Proveniência:(De \textunderscore só\textunderscore ?)}
\end{itemize}
Caça de coêlhos.
Lugar, onde o caçador espera a passagem do animal que quer caçar.
\section{Soila}
\begin{itemize}
\item {Grp. gram.:f.}
\end{itemize}
\begin{itemize}
\item {Utilização:Prov.}
\end{itemize}
\begin{itemize}
\item {Utilização:minh.}
\end{itemize}
O mesmo que \textunderscore jejum\textunderscore : \textunderscore estar em soila\textunderscore . (Colhido em Barcelos)
\section{Soim}
\begin{itemize}
\item {Grp. gram.:m.}
\end{itemize}
\begin{itemize}
\item {Utilização:Bras. do N}
\end{itemize}
O mesmo que \textunderscore saguim\textunderscore .
\section{Soitaria}
\begin{itemize}
\item {Grp. gram.:f.}
\end{itemize}
\begin{itemize}
\item {Utilização:Prov.}
\end{itemize}
\begin{itemize}
\item {Utilização:trasm.}
\end{itemize}
Soito grande.
Muitos soitos.
\section{Soiteira}
\begin{itemize}
\item {Grp. gram.:f.}
\end{itemize}
\begin{itemize}
\item {Utilização:T. de Águeda}
\end{itemize}
O mesmo que \textunderscore seitoira\textunderscore , foicinha.
\section{Soitenho}
\begin{itemize}
\item {Grp. gram.:adj.}
\end{itemize}
Relativo a soito; que se cria nos soitos.
\section{Soitinha}
\begin{itemize}
\item {Grp. gram.:adj. f.}
\end{itemize}
\begin{itemize}
\item {Utilização:Prov.}
\end{itemize}
\begin{itemize}
\item {Utilização:trasm.}
\end{itemize}
\begin{itemize}
\item {Proveniência:(De \textunderscore soito\textunderscore )}
\end{itemize}
Diz-se da castanha redonda.
\section{Soito}
\begin{itemize}
\item {Grp. gram.:m.}
\end{itemize}
\begin{itemize}
\item {Proveniência:(Do lat. \textunderscore saltus\textunderscore )}
\end{itemize}
Bosque denso.
Mata de castanheiros.
Lugar muito arborizado e próprio para passeio.
Alameda.
\section{Soja}
\begin{itemize}
\item {Grp. gram.:f.}
\end{itemize}
Planta leguminosa, (\textunderscore glycine soja\textunderscore , Sieb).
\section{Soja-híspida}
\begin{itemize}
\item {Grp. gram.:f.}
\end{itemize}
Planta gramínea, muito recommendada para alimento de animaes.
\section{Sojornar}
\begin{itemize}
\item {Grp. gram.:v. i.}
\end{itemize}
\begin{itemize}
\item {Utilização:Ant.}
\end{itemize}
\begin{itemize}
\item {Proveniência:(It. \textunderscore soggiornare\textunderscore )}
\end{itemize}
Residir.
Permanecer.
Ficar.
\section{Sol}
\begin{itemize}
\item {Grp. gram.:m.}
\end{itemize}
\begin{itemize}
\item {Utilização:Fig.}
\end{itemize}
\begin{itemize}
\item {Grp. gram.:Loc. adv.}
\end{itemize}
\begin{itemize}
\item {Proveniência:(Lat. \textunderscore sol\textunderscore )}
\end{itemize}
Astro, que occupa o centro do nosso systema planetário, e que dá calor e luz aos planetas que gravitam em torno delle.
Astro.
Estrêlla.
A luz e o calor do Sol: \textunderscore tem cuidado, não apanhes muito sol\textunderscore .
Luz.
Calor.
Círculo de doze raios, com esmalte de oiro, nos brazões.
Peixe plectógnatho.
O dia.
Grande resplendor; gênio.
\textunderscore Porta do sol\textunderscore , porta do lado do Nascente, nas antigas villas e cidades muradas.
\textunderscore De sol a sol\textunderscore , desde o nascer até o pôr do sol: \textunderscore trabalhar de sol a sol\textunderscore .
\section{Sol}
\begin{itemize}
\item {Grp. gram.:m.}
\end{itemize}
Quinta nota da escala musical.
Sinal representativo dessa nota.
Primeira corda do contra-baixo; terceira corda do violoncello e da violeta; quarta corda do violino.
(Da 1.^a sýllaba do lat. \textunderscore solve\textunderscore , aproveitada por G. de Arezzo, entre as 1.^{as} sýllabas das 1.^{as} palavras de um hymno religioso, para a formação da antiga escala musical)
\section{Sol}
\begin{itemize}
\item {Grp. gram.:adv.}
\end{itemize}
\begin{itemize}
\item {Utilização:Ant.}
\end{itemize}
\begin{itemize}
\item {Proveniência:(Do lat. \textunderscore solum\textunderscore )}
\end{itemize}
O mesmo que \textunderscore sòmente\textunderscore .
\section{Sola}
\begin{itemize}
\item {Grp. gram.:f.}
\end{itemize}
\begin{itemize}
\item {Utilização:Bras. do Rio}
\end{itemize}
\begin{itemize}
\item {Proveniência:(Do lat. \textunderscore solea\textunderscore )}
\end{itemize}
Coiro curtido de boi e próprio para manufacturar calçado.
Cabeçalho, com que se puxa a grade ou a charrua.
A parte mais dura do calçado, correspondente á planta do pé.
Espécie de beiju de tapioca, tostado no forno da farinha de mandioca.
\section{Solação}
\begin{itemize}
\item {Grp. gram.:f.}
\end{itemize}
\begin{itemize}
\item {Utilização:Ant.}
\end{itemize}
O mesmo que \textunderscore consolação\textunderscore . Cf. Usque, 45 v.^o.
(Cp. \textunderscore consolação\textunderscore  e lat. \textunderscore solatium\textunderscore )
\section{Solaçar}
\begin{itemize}
\item {Grp. gram.:v. t.}
\end{itemize}
\begin{itemize}
\item {Utilização:Ant.}
\end{itemize}
\begin{itemize}
\item {Proveniência:(De \textunderscore solaz\textunderscore )}
\end{itemize}
O mesmo que \textunderscore consolar\textunderscore . Cf. Usque, 23.
\section{Solaçoso}
\begin{itemize}
\item {Grp. gram.:adj.}
\end{itemize}
\begin{itemize}
\item {Utilização:P. us.}
\end{itemize}
\begin{itemize}
\item {Proveniência:(De \textunderscore solaz\textunderscore )}
\end{itemize}
Que causa prazer ou deleite; consolador.
\section{Solado}
\begin{itemize}
\item {Grp. gram.:adj.}
\end{itemize}
\begin{itemize}
\item {Proveniência:(De \textunderscore solo\textunderscore ^1)}
\end{itemize}
Alapardado, cosido com o solo, (falando-se do coelho, depois de batido na caça).
\section{Solamente}
\begin{itemize}
\item {Grp. gram.:adv.}
\end{itemize}
\begin{itemize}
\item {Utilização:Ant.}
\end{itemize}
\begin{itemize}
\item {Proveniência:(Do lat. \textunderscore solus\textunderscore )}
\end{itemize}
O mesmo que \textunderscore sòmente\textunderscore . Cf. G. Vicente, I, 324.
\section{Solanáceas}
\begin{itemize}
\item {Grp. gram.:f. pl.}
\end{itemize}
O mesmo ou melhor que \textunderscore solâneas\textunderscore .
\section{Solandre}
\begin{itemize}
\item {Grp. gram.:m.}
\end{itemize}
\begin{itemize}
\item {Utilização:Veter.}
\end{itemize}
Fenda na dobra do curvilhão.
\section{Solâneas}
\begin{itemize}
\item {Grp. gram.:f. pl.}
\end{itemize}
Família de plantas, que tem por typo o solano.
(Fem. pl. de \textunderscore solâneo\textunderscore )
\section{Solâneo}
\begin{itemize}
\item {Grp. gram.:adj.}
\end{itemize}
Relativo ou semelhante ao solano.
\section{Solanina}
\begin{itemize}
\item {Grp. gram.:f.}
\end{itemize}
\begin{itemize}
\item {Proveniência:(De \textunderscore solano\textunderscore )}
\end{itemize}
Alcalóide, descoberto nas hastes e tubérculos de algumas solâneas.
\section{Solano}
\begin{itemize}
\item {Grp. gram.:m.}
\end{itemize}
\begin{itemize}
\item {Proveniência:(Lat. \textunderscore solanum\textunderscore )}
\end{itemize}
Nome scientífico de um gênero de plantas, a que pertence a erva-moira, a batateira, etc.
\section{Solao}
\begin{itemize}
\item {Grp. gram.:m.}
\end{itemize}
Antigo romance em verso, ordinariamente acompanhado por música.
\section{Solão}
\begin{itemize}
\item {Grp. gram.:m.}
\end{itemize}
\begin{itemize}
\item {Proveniência:(De \textunderscore solo\textunderscore ^1)}
\end{itemize}
Terreno arenoso ou barrento.
\section{Solão}
\begin{itemize}
\item {Grp. gram.:m.}
\end{itemize}
\begin{itemize}
\item {Utilização:Ant.}
\end{itemize}
O mesmo que \textunderscore solaz\textunderscore .
\section{Solapa}
\begin{itemize}
\item {Grp. gram.:f.}
\end{itemize}
\begin{itemize}
\item {Utilização:Pop.}
\end{itemize}
\begin{itemize}
\item {Proveniência:(De \textunderscore so...\textunderscore  + \textunderscore lapa\textunderscore )}
\end{itemize}
Escavação encoberta, ou tapada por fórma que não é vista.
Ardil; ronha.
\section{Solapadamente}
\begin{itemize}
\item {Grp. gram.:adv.}
\end{itemize}
De modo solapado; ás escondidas; á surrelfa.
\section{Solapado}
\begin{itemize}
\item {Grp. gram.:adj.}
\end{itemize}
Escavado.
Recondito, occulto; dissimulado.
\section{Solapamento}
\begin{itemize}
\item {Grp. gram.:m.}
\end{itemize}
Acto ou effeito de solapar; escavação.
\section{Solapar}
\begin{itemize}
\item {Grp. gram.:v. t.}
\end{itemize}
\begin{itemize}
\item {Utilização:Fig.}
\end{itemize}
\begin{itemize}
\item {Proveniência:(De \textunderscore solapa\textunderscore )}
\end{itemize}
Formar lapa em.
Escavar.
Minar.
Arruinar.
Disfarçar; occultar.
\section{Solar}
\begin{itemize}
\item {Grp. gram.:adj.}
\end{itemize}
\begin{itemize}
\item {Proveniência:(Lat. \textunderscore solaris\textunderscore )}
\end{itemize}
Relativo ao Sol: \textunderscore eclipse solar\textunderscore .
\section{Solar}
\begin{itemize}
\item {Grp. gram.:v. t.}
\end{itemize}
Pôr solas em (calçado).
\section{Solar}
\begin{itemize}
\item {Grp. gram.:adj.}
\end{itemize}
\begin{itemize}
\item {Grp. gram.:M.}
\end{itemize}
\begin{itemize}
\item {Utilização:Prov.}
\end{itemize}
\begin{itemize}
\item {Utilização:beir.}
\end{itemize}
Relativo a sola.
O mesmo que \textunderscore soleira\textunderscore .
\section{Solar}
\begin{itemize}
\item {Grp. gram.:m.}
\end{itemize}
\begin{itemize}
\item {Proveniência:(De \textunderscore solo\textunderscore ^1)}
\end{itemize}
Herdade ou morada de família nobre e antiga.
\section{Solar}
\begin{itemize}
\item {Grp. gram.:v. i.}
\end{itemize}
\begin{itemize}
\item {Proveniência:(De \textunderscore solo\textunderscore ^2)}
\end{itemize}
Ganhar, no jôgo do solo.
\section{Solar}
\begin{itemize}
\item {Grp. gram.:m.}
\end{itemize}
Padrão monetário no Peru.
\section{Solarego}
\begin{itemize}
\item {fónica:larê}
\end{itemize}
\begin{itemize}
\item {Grp. gram.:adj.}
\end{itemize}
\begin{itemize}
\item {Utilização:Ant.}
\end{itemize}
O mesmo que \textunderscore solarengo\textunderscore .
\section{Solarengo}
\begin{itemize}
\item {Grp. gram.:adj.}
\end{itemize}
\begin{itemize}
\item {Grp. gram.:M.}
\end{itemize}
\begin{itemize}
\item {Utilização:Ant.}
\end{itemize}
Relativo a solar^4.
Aquelle que, como serviçal ou lavrador, vivia no solar ou fazenda de outrem. Cf. Herculano, \textunderscore Hist. de Port.\textunderscore , IV. 339.
Senhor de solar:«\textunderscore Maria afastou-se do solarengo do Tâmega.\textunderscore »Camillo, \textunderscore Estrêll. Fun.\textunderscore 
\section{Solariego}
\begin{itemize}
\item {fónica:ê}
\end{itemize}
\begin{itemize}
\item {Grp. gram.:adj.}
\end{itemize}
O mesmo que \textunderscore solarengo\textunderscore . Cf. P. Carvalho, \textunderscore Corogr. Port.\textunderscore , I, 193 e 289.
\section{Solário}
\begin{itemize}
\item {Grp. gram.:m.}
\end{itemize}
\begin{itemize}
\item {Proveniência:(Lat. \textunderscore solarium\textunderscore , de \textunderscore sol\textunderscore )}
\end{itemize}
O relógio de sol, usado pelos antigos Romanos.
\section{Solário}
\begin{itemize}
\item {Grp. gram.:m.}
\end{itemize}
\begin{itemize}
\item {Utilização:Jur.}
\end{itemize}
\begin{itemize}
\item {Proveniência:(Lat. \textunderscore solarium\textunderscore , de \textunderscore solum\textunderscore )}
\end{itemize}
Tributo que pagavam as propriedades rústicas, no império romano.
\section{Solaroso}
\begin{itemize}
\item {Grp. gram.:adj.}
\end{itemize}
\begin{itemize}
\item {Utilização:Ant.}
\end{itemize}
O mesmo que \textunderscore solaz\textunderscore , adj.
\section{Solau}
\begin{itemize}
\item {Grp. gram.:m.}
\end{itemize}
Antigo romance em verso, ordinariamente acompanhado por música.
\section{Solavancar}
\begin{itemize}
\item {Grp. gram.:v. i.}
\end{itemize}
\begin{itemize}
\item {Utilização:Neol.}
\end{itemize}
Dar solavancos.
\section{Solavanco}
\begin{itemize}
\item {Grp. gram.:m.}
\end{itemize}
Balanço imprevisto ou violento de um vehículo, ou da pessôa que êste transporta.
(Por \textunderscore solevanco\textunderscore , de \textunderscore solevar\textunderscore )
\section{Solaz}
\begin{itemize}
\item {Grp. gram.:m.}
\end{itemize}
\begin{itemize}
\item {Grp. gram.:Adj.}
\end{itemize}
\begin{itemize}
\item {Proveniência:(Do lat. \textunderscore solatium\textunderscore )}
\end{itemize}
Distracção; recreio.
Consolação. Cf. Latino, \textunderscore Camões\textunderscore , 314.
Consolador.
\section{Sol-cris}
\begin{itemize}
\item {Grp. gram.:m.}
\end{itemize}
\begin{itemize}
\item {Utilização:Ant.}
\end{itemize}
Eclipse do Sol.
\section{Solda}
\begin{itemize}
\item {Grp. gram.:f.}
\end{itemize}
\begin{itemize}
\item {Proveniência:(Do lat. \textunderscore solida\textunderscore )}
\end{itemize}
Substância metallica e fusível, para unir peças, também metállicas.
\section{Solda}
\begin{itemize}
\item {Grp. gram.:f.}
\end{itemize}
Planta rubiácea, o mesmo que \textunderscore mollugem\textunderscore .
\section{Soldada}
\begin{itemize}
\item {Grp. gram.:f.}
\end{itemize}
\begin{itemize}
\item {Utilização:Fig.}
\end{itemize}
\begin{itemize}
\item {Utilização:Ant.}
\end{itemize}
\begin{itemize}
\item {Proveniência:(De \textunderscore sôldo\textunderscore )}
\end{itemize}
Pagamento do serviço de criado, operários, etc.
Salário.
Recompensa.
Prêmio.
Foro de um ou mais soldos.
Aquillo que se comprava com certa porção de soldos.
\section{Soldadeira}
\begin{itemize}
\item {Grp. gram.:f.}
\end{itemize}
\begin{itemize}
\item {Utilização:Des.}
\end{itemize}
Criada, mulhér que serve por soldada. Cf. \textunderscore Port. Mon. Hist.\textunderscore , \textunderscore Script.\textunderscore , 321.
\section{Soldadeiro}
\begin{itemize}
\item {Grp. gram.:m.  e  adj.}
\end{itemize}
\begin{itemize}
\item {Proveniência:(De \textunderscore soldada\textunderscore )}
\end{itemize}
O que é assoldadado.
\section{Soldadesca}
\begin{itemize}
\item {fónica:dês}
\end{itemize}
\begin{itemize}
\item {Grp. gram.:f.}
\end{itemize}
\begin{itemize}
\item {Utilização:Deprec.}
\end{itemize}
Tropas.
Classe militar.
Gente de guerra.
(Fem. de \textunderscore soldadesco\textunderscore )
\section{Soldadesco}
\begin{itemize}
\item {fónica:dês}
\end{itemize}
\begin{itemize}
\item {Grp. gram.:adj.}
\end{itemize}
Relativo a soldados ou próprio delles.
\section{Soldado}
\begin{itemize}
\item {Grp. gram.:m.}
\end{itemize}
\begin{itemize}
\item {Utilização:Fig.}
\end{itemize}
\begin{itemize}
\item {Proveniência:(De \textunderscore sôldo\textunderscore )}
\end{itemize}
Militar, de categoria inferior.
Qualquer militar.
Peixe do Brasil.
Ave brasileira, também conhecida por \textunderscore encontro\textunderscore .
Partidário; campeão.
\section{Soldador}
\begin{itemize}
\item {Grp. gram.:m.  e  adj.}
\end{itemize}
O que solda.
\section{Soldadura}
\begin{itemize}
\item {Grp. gram.:f.}
\end{itemize}
Acto ou effeito de soldar.
Tumor subcutâneo nas costellas das cavalgaduras.
\section{Soldagem}
\begin{itemize}
\item {Grp. gram.:f.}
\end{itemize}
O mesmo que \textunderscore soldadura\textunderscore ; acto de soldar (latas de sardinha, especialmente).
\section{Soldanela}
\begin{itemize}
\item {Grp. gram.:f.}
\end{itemize}
Planta convolvulácea, (\textunderscore calystegium soldanella\textunderscore ).
\section{Soldânia}
\begin{itemize}
\item {Grp. gram.:f.}
\end{itemize}
\begin{itemize}
\item {Proveniência:(De \textunderscore Soldani\textunderscore , n. p.)}
\end{itemize}
Gênero de conchas microscópicas.
\section{Soldanella}
\begin{itemize}
\item {Grp. gram.:f.}
\end{itemize}
Planta convolvulácea, (\textunderscore calystegium soldanella\textunderscore ).
\section{Soldão}
\begin{itemize}
\item {Grp. gram.:m.}
\end{itemize}
\begin{itemize}
\item {Utilização:Ant.}
\end{itemize}
O mesmo que \textunderscore sultão\textunderscore . Cf. \textunderscore Roteiro de Vasco da Gama\textunderscore ; Filinto, \textunderscore D. Man.\textunderscore , I, 330.
\section{Soldar}
\begin{itemize}
\item {Grp. gram.:v. t.}
\end{itemize}
\begin{itemize}
\item {Proveniência:(De \textunderscore solda\textunderscore ^1)}
\end{itemize}
Unir ou pegar por meio de solda; fechar.
\section{Soldar}
\begin{itemize}
\item {Grp. gram.:adj.}
\end{itemize}
Diz-se de uma espécie de cereja, vermelha e molle.
\section{Solda-real}
\begin{itemize}
\item {Grp. gram.:f.}
\end{itemize}
Planta da serra de Sintra.
\section{Soldável}
\begin{itemize}
\item {Grp. gram.:adj.}
\end{itemize}
Que se póde soldar.
\section{Sol-de-gata}
\begin{itemize}
\item {Grp. gram.:f.}
\end{itemize}
\begin{itemize}
\item {Utilização:Prov.}
\end{itemize}
\begin{itemize}
\item {Utilização:trasm.}
\end{itemize}
Fieira de pedras salientes, a certa altura de uma parede.
\section{Soldevilha}
\begin{itemize}
\item {Grp. gram.:f.}
\end{itemize}
\begin{itemize}
\item {Proveniência:(De \textunderscore Soldeville\textunderscore , n. p.)}
\end{itemize}
Gênero de plantas chicoriáceas.
\section{Sôldo}
\begin{itemize}
\item {Grp. gram.:m.}
\end{itemize}
\begin{itemize}
\item {Utilização:Fig.}
\end{itemize}
\begin{itemize}
\item {Proveniência:(Do lat. \textunderscore solidus\textunderscore )}
\end{itemize}
Retribuição do serviço dos militares.
Vencimento de militares.
Moéda francesa, correspondente á vigésima parte de um franco.
Antiga moéda de oiro, entre os Romanos.
Nome de várias moédas antigas de Portugal, em oiro, prata e cobre.
Retribuição; salário; recompensa.
\section{Soldra}
\begin{itemize}
\item {Grp. gram.:f.}
\end{itemize}
\begin{itemize}
\item {Utilização:Veter.}
\end{itemize}
Saliência, sôbre a articulação da coxa com a perna, nas cavalgaduras.
\section{Soldúrios}
\begin{itemize}
\item {Grp. gram.:m. pl.}
\end{itemize}
\begin{itemize}
\item {Proveniência:(Lat. \textunderscore soldurii\textunderscore )}
\end{itemize}
Os indivíduos, que constituíam a guarda mais fiel ou mais dedicada de um chefe, entre os Gállios.
\section{Sole}
\begin{itemize}
\item {Grp. gram.:m.}
\end{itemize}
Ave trepadora da África, (\textunderscore indicator minor\textunderscore ).
\section{Sólea}
\begin{itemize}
\item {Grp. gram.:f.}
\end{itemize}
\begin{itemize}
\item {Utilização:Ant.}
\end{itemize}
\begin{itemize}
\item {Proveniência:(Lat. \textunderscore solea\textunderscore )}
\end{itemize}
Alparca; chapim.
\section{Sólea}
\begin{itemize}
\item {Grp. gram.:f.}
\end{itemize}
Gênero de plantas violáceas.
\section{Solecar}
\begin{itemize}
\item {Grp. gram.:v.}
\end{itemize}
\begin{itemize}
\item {Utilização:t. Náut.}
\end{itemize}
Arrear (um cabo), um pouco sob volta.
\section{Solecismo}
\begin{itemize}
\item {Grp. gram.:m.}
\end{itemize}
\begin{itemize}
\item {Utilização:Ext.}
\end{itemize}
\begin{itemize}
\item {Proveniência:(Lat. \textunderscore soloecismus\textunderscore )}
\end{itemize}
Êrro contra as regras da syntaxe.
Êrro, culpa.
\section{Solecista}
\begin{itemize}
\item {Grp. gram.:m.  e  adj.}
\end{itemize}
O que commete solecismos.
(B. lat. \textunderscore soloecista\textunderscore )
\section{Soleçole}
\begin{itemize}
\item {Grp. gram.:m.}
\end{itemize}
Pássaro dentirostro da África.
(Cp. \textunderscore sole\textunderscore )
\section{Solecurta}
\begin{itemize}
\item {Grp. gram.:f.}
\end{itemize}
Gênero de molluscos.
\section{Soledade}
\begin{itemize}
\item {Grp. gram.:f.}
\end{itemize}
\begin{itemize}
\item {Proveniência:(Do lat. \textunderscore solitas\textunderscore )}
\end{itemize}
O mesmo que \textunderscore solidão\textunderscore .
Lugar ermo.
Tristeza de quem está só ou abandonado.
\section{Sol-e-dó}
\begin{itemize}
\item {Grp. gram.:m.}
\end{itemize}
\begin{itemize}
\item {Utilização:Pop.}
\end{itemize}
\begin{itemize}
\item {Proveniência:(De \textunderscore sol\textunderscore ^2 + \textunderscore e\textunderscore  + \textunderscore dó\textunderscore ^2)}
\end{itemize}
Música.
Philarmónica ordinária.
Concêrto de guitarras e violas.
\section{Soleira}
\begin{itemize}
\item {Grp. gram.:f.}
\end{itemize}
\begin{itemize}
\item {Proveniência:(De \textunderscore sola\textunderscore )}
\end{itemize}
Peça quadrilonga de pedra ou madeira, em que assentam os ombraes da porta, ou que se estende entre êlles no chão, parallelamente á vêrga ou tórça.
Limiar.
Ferro, por baixo das tesoiras do coche.
Estribo de carruagem.
Uma das correias da espora, a que passa por baixo do calçado.
Grande peça de madeira, entre a taleira e a parte deanteira da carrêta de uma peça, nos navios.
\section{Soleirólia}
\begin{itemize}
\item {Grp. gram.:f.}
\end{itemize}
\begin{itemize}
\item {Proveniência:(De \textunderscore Soleirol\textunderscore , n. p.)}
\end{itemize}
Gênero de plantas urticáceas.
\section{Solêmia}
\begin{itemize}
\item {Grp. gram.:f.}
\end{itemize}
Gênero de moluscos.
\section{Solêmya}
\begin{itemize}
\item {Grp. gram.:f.}
\end{itemize}
Gênero de molluscos.
\section{Solenáceos}
\begin{itemize}
\item {Grp. gram.:m. pl.}
\end{itemize}
\begin{itemize}
\item {Proveniência:(Do gr. \textunderscore solen\textunderscore )}
\end{itemize}
Família de molluscos.
\section{Solenantho}
\begin{itemize}
\item {Grp. gram.:m.}
\end{itemize}
\begin{itemize}
\item {Proveniência:(Do gr. \textunderscore solen\textunderscore  + \textunderscore anthos\textunderscore )}
\end{itemize}
Gênero de plantas borragíneas.
\section{Solenanto}
\begin{itemize}
\item {Grp. gram.:m.}
\end{itemize}
\begin{itemize}
\item {Proveniência:(Do gr. \textunderscore solen\textunderscore  + \textunderscore anthos\textunderscore )}
\end{itemize}
Gênero de plantas borragíneas.
\section{Solene}
\begin{itemize}
\item {Grp. gram.:adj.}
\end{itemize}
\begin{itemize}
\item {Utilização:Fam.}
\end{itemize}
\begin{itemize}
\item {Proveniência:(Lat. \textunderscore solennis\textunderscore )}
\end{itemize}
Que se celebra todos os anos com ceremónias públicas.
Pomposo.
Grave, majestoso: \textunderscore acto solene\textunderscore .
Público.
Que é feito com as formalidades necessárias ou exigidas.
Enfático: \textunderscore estilo solene\textunderscore .
\section{Solenemente}
\begin{itemize}
\item {Grp. gram.:adv.}
\end{itemize}
De modo solene; com solenidade; com pompa; majestosamente.
\section{Solenidade}
\begin{itemize}
\item {Grp. gram.:f.}
\end{itemize}
\begin{itemize}
\item {Utilização:Fam.}
\end{itemize}
\begin{itemize}
\item {Proveniência:(Do lat. \textunderscore solennitas\textunderscore )}
\end{itemize}
Qualidade do que é solenne.
Acto solene.
Festividade.
Formalidades que devem acompanhar certos actos, para que êstes se tornem autênticos ou válidos.
Enfase, arrogância.
\section{Soleníscia}
\begin{itemize}
\item {Grp. gram.:f.}
\end{itemize}
\begin{itemize}
\item {Proveniência:(Do gr. \textunderscore soleniskos\textunderscore )}
\end{itemize}
Gênero de plantas epacrídeas.
\section{Solenização}
\begin{itemize}
\item {Grp. gram.:f.}
\end{itemize}
Acto ou efeito de solenizar.
\section{Solenizador}
\begin{itemize}
\item {Grp. gram.:m.  e  adj.}
\end{itemize}
O que soleniza.
\section{Solenizar}
\begin{itemize}
\item {Grp. gram.:v. t.}
\end{itemize}
Celebrar com ceremónia ou pompa.
Tornar solene.
\section{Solenne}
\begin{itemize}
\item {Grp. gram.:adj.}
\end{itemize}
\begin{itemize}
\item {Utilização:Fam.}
\end{itemize}
\begin{itemize}
\item {Proveniência:(Lat. \textunderscore solennis\textunderscore )}
\end{itemize}
Que se celebra todos os annos com ceremónias públicas.
Pomposo.
Grave, majestoso: \textunderscore acto solenne\textunderscore .
Público.
Que é feito com as formalidades necessárias ou exigidas.
Emphático: \textunderscore estilo solenne\textunderscore .
\section{Solennemente}
\begin{itemize}
\item {Grp. gram.:adv.}
\end{itemize}
De modo solenne; com solennidade; com pompa; majestosamente.
\section{Solennidade}
\begin{itemize}
\item {Grp. gram.:f.}
\end{itemize}
\begin{itemize}
\item {Utilização:Fam.}
\end{itemize}
\begin{itemize}
\item {Proveniência:(Do lat. \textunderscore solennitas\textunderscore )}
\end{itemize}
Qualidade do que é solenne.
Acto solenne.
Festividade.
Formalidades que devem acompanhar certos actos, para que êstes se tornem authênticos ou válidos.
Emphase, arrogância.
\section{Solennização}
\begin{itemize}
\item {Grp. gram.:f.}
\end{itemize}
Acto ou effeito de solennizar.
\section{Solennizador}
\begin{itemize}
\item {Grp. gram.:m.  e  adj.}
\end{itemize}
O que solenniza.
\section{Solennizar}
\begin{itemize}
\item {Grp. gram.:v. t.}
\end{itemize}
Celebrar com ceremónia ou pompa.
Tornar solenne.
\section{Solenocarpo}
\begin{itemize}
\item {Grp. gram.:m.}
\end{itemize}
\begin{itemize}
\item {Proveniência:(Do gr. \textunderscore solen\textunderscore  + \textunderscore karpos\textunderscore )}
\end{itemize}
Gênero de plantas anacardiáceas.
\section{Solenodonte}
\begin{itemize}
\item {Grp. gram.:m.}
\end{itemize}
\begin{itemize}
\item {Proveniência:(Do gr. \textunderscore solen\textunderscore  + \textunderscore odous\textunderscore )}
\end{itemize}
Gênero de mammíferos insectívoros.
\section{Solenófora}
\begin{itemize}
\item {Grp. gram.:f.}
\end{itemize}
\begin{itemize}
\item {Proveniência:(Do gr. \textunderscore solen\textunderscore  + \textunderscore phoros\textunderscore )}
\end{itemize}
Gênero de plantas gesneriáceas.
\section{Solenóide}
\begin{itemize}
\item {Grp. gram.:m.}
\end{itemize}
\begin{itemize}
\item {Proveniência:(Do gr. \textunderscore solen\textunderscore  + \textunderscore eidos\textunderscore )}
\end{itemize}
Fio eléctrico, enrolado em espiral, e que serve para mostrar a analogia dos phenómenos eléctricos e dos magnéticos.
\section{Solenóphora}
\begin{itemize}
\item {Grp. gram.:f.}
\end{itemize}
\begin{itemize}
\item {Proveniência:(Do gr. \textunderscore solen\textunderscore  + \textunderscore phoros\textunderscore )}
\end{itemize}
Gênero de plantas gesneriáceas.
\section{Solenóptera}
\begin{itemize}
\item {Grp. gram.:f.}
\end{itemize}
\begin{itemize}
\item {Proveniência:(Do gr. \textunderscore solen\textunderscore  + \textunderscore pteron\textunderscore )}
\end{itemize}
Gênero de insectos coleópteros longicórneos.
\section{Solenostema}
\begin{itemize}
\item {Grp. gram.:f.}
\end{itemize}
\begin{itemize}
\item {Proveniência:(Do gr. \textunderscore solen\textunderscore  + \textunderscore stemma\textunderscore )}
\end{itemize}
Gênero de plantas asclepiadáceas.
\section{Solenostemma}
\begin{itemize}
\item {Grp. gram.:f.}
\end{itemize}
\begin{itemize}
\item {Proveniência:(Do gr. \textunderscore solen\textunderscore  + \textunderscore stemma\textunderscore )}
\end{itemize}
Gênero de plantas asclepiadáceas.
\section{Solenóstoma}
\begin{itemize}
\item {Grp. gram.:f.}
\end{itemize}
\begin{itemize}
\item {Proveniência:(Do gr. \textunderscore solen\textunderscore  + \textunderscore stoma\textunderscore )}
\end{itemize}
Gênero de peixes acanthopterýgios.
\section{Solenoteca}
\begin{itemize}
\item {Grp. gram.:f.}
\end{itemize}
\begin{itemize}
\item {Proveniência:(Do gr. \textunderscore solen\textunderscore  + \textunderscore theke\textunderscore )}
\end{itemize}
Gênero de plantas, da fam. das compostas.
\section{Solenotheca}
\begin{itemize}
\item {Grp. gram.:f.}
\end{itemize}
\begin{itemize}
\item {Proveniência:(Do gr. \textunderscore solen\textunderscore  + \textunderscore theke\textunderscore )}
\end{itemize}
Gênero de plantas, da fam. das compostas.
\section{Soleque}
\begin{itemize}
\item {Grp. gram.:m.}
\end{itemize}
\begin{itemize}
\item {Utilização:T. do Fundão}
\end{itemize}
\begin{itemize}
\item {Utilização:Chul.}
\end{itemize}
O mesmo que \textunderscore sondeque\textunderscore .
\section{Solequei}
\begin{itemize}
\item {Grp. gram.:m.}
\end{itemize}
Pássaro dentirostro da Africa occidental.
\section{Solércia}
\begin{itemize}
\item {Grp. gram.:f.}
\end{itemize}
\begin{itemize}
\item {Proveniência:(Lat. \textunderscore solertia\textunderscore )}
\end{itemize}
Qualidade do que é solerte; argúcia, ardil.
\section{Solerte}
\begin{itemize}
\item {Grp. gram.:m.  e  adj.}
\end{itemize}
\begin{itemize}
\item {Proveniência:(Lat. \textunderscore solers\textunderscore )}
\end{itemize}
O que é sagaz, astuto ou manhoso; velhaco.
\section{Soles}
\begin{itemize}
\item {Grp. gram.:m.}
\end{itemize}
Cambão, a que se atrela mais de uma junta de bois.
\section{Soleta}
\begin{itemize}
\item {fónica:lê}
\end{itemize}
\begin{itemize}
\item {Grp. gram.:f.}
\end{itemize}
\begin{itemize}
\item {Proveniência:(De \textunderscore sola\textunderscore )}
\end{itemize}
Peça de sola, cortada para calçado.
Palmilha.
\section{Solentano}
\begin{itemize}
\item {Grp. gram.:m.}
\end{itemize}
Um dos dialectos do vasconço, em França.
\section{Soletração}
\begin{itemize}
\item {Grp. gram.:f.}
\end{itemize}
Acto ou effeito de soletrar.
\section{Soletrador}
\begin{itemize}
\item {Grp. gram.:m.  e  adj.}
\end{itemize}
O que soletra.
\section{Soletrar}
\begin{itemize}
\item {Grp. gram.:v. t.}
\end{itemize}
\begin{itemize}
\item {Utilização:Fig.}
\end{itemize}
\begin{itemize}
\item {Grp. gram.:V. i.}
\end{itemize}
\begin{itemize}
\item {Proveniência:(De \textunderscore so...\textunderscore  + \textunderscore letra\textunderscore )}
\end{itemize}
Lêr, pronunciando separadamente as letras e juntando estas em sýllabas.
Entender, decifrar.
Adivinhar.
Lêr mal.
Separar as letras de cada palavra, juntando-as em sýllabas, para fazer a leitura da mesma palavra.
\section{Soletrear}
\begin{itemize}
\item {Grp. gram.:v. t.}
\end{itemize}
O mesmo que \textunderscore soletrar\textunderscore . Cf. \textunderscore Eufrosina\textunderscore , act. III, sc. II.
\section{Solevamento}
\begin{itemize}
\item {Grp. gram.:m.}
\end{itemize}
Acto ou effeito de solevar.
\section{Solevantar}
\begin{itemize}
\item {Grp. gram.:v. t.}
\end{itemize}
\begin{itemize}
\item {Proveniência:(De \textunderscore so...\textunderscore  + \textunderscore levantar\textunderscore )}
\end{itemize}
Erguer um pouco; levantar a pouca distância.
Levar com difficuldade.
\section{Solevar}
\begin{itemize}
\item {Proveniência:(Do lat. \textunderscore sub\textunderscore  + \textunderscore levare\textunderscore )}
\end{itemize}
\textunderscore v. t.\textunderscore  Levantar; solevantar; soerguer.
\section{Solfa}
\begin{itemize}
\item {Grp. gram.:f.}
\end{itemize}
\begin{itemize}
\item {Utilização:Pop.}
\end{itemize}
\begin{itemize}
\item {Proveniência:(De \textunderscore sol\textunderscore ^2 + \textunderscore fá\textunderscore )}
\end{itemize}
Música; arte de solfejar.
Gritaria.
\section{Solfado}
\begin{itemize}
\item {Grp. gram.:adj.}
\end{itemize}
\begin{itemize}
\item {Proveniência:(De \textunderscore solfar\textunderscore ^1)}
\end{itemize}
Diz-se do papel, pautado á largura da fôlha, em vez de o sêr á altura, como é mais usual.
\section{Solfar}
\begin{itemize}
\item {Grp. gram.:v. t.  e  i.}
\end{itemize}
O mesmo que \textunderscore solfejar\textunderscore .
\section{Solfar}
\begin{itemize}
\item {Grp. gram.:v. t.}
\end{itemize}
\begin{itemize}
\item {Proveniência:(Do it. \textunderscore sodo\textunderscore  + \textunderscore fare\textunderscore )}
\end{itemize}
Consertar as margens de (uma fôlha de livro, rôta ou gasta).
Aumentar as margens de (uma fôlha ou um livro).
\section{Solfatara}
\begin{itemize}
\item {Grp. gram.:f.}
\end{itemize}
\begin{itemize}
\item {Proveniência:(It. \textunderscore solfatara\textunderscore , de \textunderscore solfato\textunderscore )}
\end{itemize}
Terreno, em que se desenvolvem vapores sulfurosos ou em que se deposita enxôfre.
Enxofreira.
\section{Solfejação}
\begin{itemize}
\item {Grp. gram.:f.}
\end{itemize}
\begin{itemize}
\item {Utilização:Des.}
\end{itemize}
O mesmo que \textunderscore solfejo\textunderscore .
\section{Solfejar}
\begin{itemize}
\item {Grp. gram.:v. t.  e  i.}
\end{itemize}
\begin{itemize}
\item {Proveniência:(De \textunderscore solfa\textunderscore )}
\end{itemize}
Lêr ou entoar os nomes das notas de uma peça musical.
\section{Solfejo}
\begin{itemize}
\item {Grp. gram.:m.}
\end{itemize}
\begin{itemize}
\item {Grp. gram.:Pl.}
\end{itemize}
Acto ou effeito de solfejar.
Exercício musical, para se aprender a solfejar.
Compêndio ou caderno de exercícios musicaes, para se aprender a solfejar.
\section{Solfista}
\begin{itemize}
\item {Grp. gram.:m.  e  f.}
\end{itemize}
\begin{itemize}
\item {Grp. gram.:M.}
\end{itemize}
\begin{itemize}
\item {Utilização:Pop.}
\end{itemize}
\begin{itemize}
\item {Proveniência:(De \textunderscore solfa\textunderscore )}
\end{itemize}
Pessôa que solfeja.
Músico.
\section{Sôlha}
\begin{itemize}
\item {Grp. gram.:f.}
\end{itemize}
\begin{itemize}
\item {Utilização:Chul.}
\end{itemize}
\begin{itemize}
\item {Utilização:Prov.}
\end{itemize}
\begin{itemize}
\item {Utilização:minh.}
\end{itemize}
\begin{itemize}
\item {Utilização:Ant.}
\end{itemize}
\begin{itemize}
\item {Proveniência:(Do lat. \textunderscore solea\textunderscore )}
\end{itemize}
Peixe pleuronecto.
Bofetada.
Chinelo velho.
Espécie de cota de armas, guarnecida com lâminas de aço.
\section{Solhado}
\begin{itemize}
\item {Grp. gram.:m.}
\end{itemize}
O mesmo que \textunderscore soalho\textunderscore ^1.
\section{Solhar}
\textunderscore v. t.\textunderscore  (e der.)
O mesmo que \textunderscore assoalhar\textunderscore ^2, etc.
\section{Solhar}
\begin{itemize}
\item {Grp. gram.:adj.}
\end{itemize}
\begin{itemize}
\item {Utilização:Anat.}
\end{itemize}
\begin{itemize}
\item {Proveniência:(Lat. \textunderscore solearis\textunderscore )}
\end{itemize}
Diz-se de um dos músculos da barriga da perna, pela sua semelhança com o sôlho.
\section{Sôlhas}
\begin{itemize}
\item {Grp. gram.:f. pl.}
\end{itemize}
Antiga armadura, guarnecida de lâminas de aço, quasi á feição do peixe sôlha.
\section{Solheira}
\begin{itemize}
\item {Grp. gram.:f.}
\end{itemize}
Rêde para pescar sôlhas.
\section{Sòlheira}
\begin{itemize}
\item {Grp. gram.:f.}
\end{itemize}
(V.soalheira)
\section{Sòlheiro}
\begin{itemize}
\item {Grp. gram.:m.}
\end{itemize}
O mesmo que \textunderscore soalheiro\textunderscore .
\section{Sôlho}
\begin{itemize}
\item {Grp. gram.:m.}
\end{itemize}
\begin{itemize}
\item {Utilização:Prov.}
\end{itemize}
\begin{itemize}
\item {Utilização:beir.}
\end{itemize}
\begin{itemize}
\item {Proveniência:(Do lat. \textunderscore solium\textunderscore )}
\end{itemize}
O mesmo que \textunderscore sobrado\textunderscore ^2.
\textunderscore Prego de sôlho\textunderscore , o mesmo que \textunderscore meia-galeota\textunderscore .
\section{Sôlho}
\begin{itemize}
\item {Grp. gram.:m.}
\end{itemize}
\begin{itemize}
\item {Proveniência:(De \textunderscore sôlha\textunderscore )}
\end{itemize}
Peixe esturónio, (\textunderscore accipenser sturio\textunderscore ).
Esturjão.
\section{Sôlho-rei}
\begin{itemize}
\item {Grp. gram.:m.}
\end{itemize}
\begin{itemize}
\item {Utilização:Pesc.}
\end{itemize}
\begin{itemize}
\item {Utilização:T. de Coimbra}
\end{itemize}
O mesmo que \textunderscore esturjão\textunderscore .
O mesmo que \textunderscore rodovalho\textunderscore .
\section{Solia}
\begin{itemize}
\item {Grp. gram.:f.}
\end{itemize}
Antigo tecido de lan.
Vestuário, feito dêsse tecido.
\section{Solias}
\begin{itemize}
\item {Grp. gram.:f. pl.}
\end{itemize}
\begin{itemize}
\item {Proveniência:(De \textunderscore sola\textunderscore )}
\end{itemize}
Sapatos; qualquer outro calçado.
\section{Solicas}
\begin{itemize}
\item {Grp. gram.:f. pl.}
\end{itemize}
\begin{itemize}
\item {Utilização:Ant.}
\end{itemize}
\begin{itemize}
\item {Proveniência:(De \textunderscore sola\textunderscore )}
\end{itemize}
Sapatos; qualquer outro calçado.
\section{Solicitação}
\begin{itemize}
\item {Grp. gram.:f.}
\end{itemize}
\begin{itemize}
\item {Proveniência:(Do lat. \textunderscore solicitatio\textunderscore )}
\end{itemize}
Acto ou effeito de solicitar.
Pedido.
\section{Solicitador}
\begin{itemize}
\item {Grp. gram.:m.  e  adj.}
\end{itemize}
\begin{itemize}
\item {Grp. gram.:M.}
\end{itemize}
\begin{itemize}
\item {Proveniência:(Do lat. \textunderscore solicitator\textunderscore )}
\end{itemize}
O que solicita.
Procurador, habilitado legalmente para requerer em juízo ou promover o andamento de negôcios forenses.
\section{Solicitamente}
\begin{itemize}
\item {Grp. gram.:adv.}
\end{itemize}
De modo solícito; com cuidado, com diligência.
\section{Solicitante}
\begin{itemize}
\item {Grp. gram.:m.  e  adj.}
\end{itemize}
\begin{itemize}
\item {Proveniência:(Lat. \textunderscore solicitans\textunderscore )}
\end{itemize}
O que solicita.
\section{Solicitar}
\begin{itemize}
\item {Grp. gram.:v. t.}
\end{itemize}
\begin{itemize}
\item {Grp. gram.:V. i.}
\end{itemize}
\begin{itemize}
\item {Proveniência:(Lat. \textunderscore solicitare\textunderscore )}
\end{itemize}
Demover.
Induzir.
Agenciar com empenho.
Pedir instantemente.
Attrahir.
Provocar.
Requestar.
Promover como solicitador judicial.
Requerer.
Fazer requerimentos perante os tribunaes, como procurador, e promover negócios forenses de outrem.
\section{Solicitável}
\begin{itemize}
\item {Grp. gram.:adj.}
\end{itemize}
Que se póde solicitar.
\section{Solícito}
\begin{itemize}
\item {Grp. gram.:adj.}
\end{itemize}
\begin{itemize}
\item {Proveniência:(Lat. \textunderscore solicitus\textunderscore )}
\end{itemize}
Cuidadoso; diligente; activo.
Inquieto.
Apprehensivo.
Delicado; prestadio.
\section{Solicitude}
\begin{itemize}
\item {Grp. gram.:f.}
\end{itemize}
\begin{itemize}
\item {Proveniência:(Lat. \textunderscore solicitudo\textunderscore )}
\end{itemize}
Qualidade do que é solícito.
Diligência.
Actividade.
\section{Solidamente}
\begin{itemize}
\item {Grp. gram.:adv.}
\end{itemize}
De modo sólido.
Com solidez; com firmeza.
\section{Solidão}
\begin{itemize}
\item {Grp. gram.:f.}
\end{itemize}
\begin{itemize}
\item {Proveniência:(Do lat. \textunderscore solitudo\textunderscore )}
\end{itemize}
Estado do que está só.
Lugar despovoado.
Insulamento.
\section{Solidar}
\begin{itemize}
\item {Grp. gram.:v. t.}
\end{itemize}
\begin{itemize}
\item {Proveniência:(Lat. \textunderscore solidare\textunderscore )}
\end{itemize}
Solidificar.
Confirmar, corroborar.
\section{Solidariamente}
\begin{itemize}
\item {Grp. gram.:adv.}
\end{itemize}
De modo solidário.
\section{Solidariedade}
\begin{itemize}
\item {Grp. gram.:f.}
\end{itemize}
Qualidade do que é solidário.
Ligação recíproca de coisas, que são dependentes umas das outras.
Direito, que tem qualquer de vários credores, a exigir só para si o que se deve a todos.
\section{Solidário}
\begin{itemize}
\item {Grp. gram.:adj.}
\end{itemize}
\begin{itemize}
\item {Proveniência:(Do lat. \textunderscore solidus\textunderscore )}
\end{itemize}
Que torna cada um de muitos devedores obrigado ao pagamento total da dívida.
Que dá a cada um de muitos crèdores o direito de receber a totalidade da dívida.
Que tem responsabilidade recíproca ou interesse commum.
\section{Solidarização}
\begin{itemize}
\item {Grp. gram.:f.}
\end{itemize}
Acto ou effeito de solidarizar.
\section{Solidarizar}
\begin{itemize}
\item {Grp. gram.:v. t.}
\end{itemize}
\begin{itemize}
\item {Utilização:Neol.}
\end{itemize}
Tornar solidário.
\section{Solidéo}
\begin{itemize}
\item {Grp. gram.:m.}
\end{itemize}
\begin{itemize}
\item {Proveniência:(De \textunderscore soli Deo\textunderscore , loc. lat. de \textunderscore solus\textunderscore  + \textunderscore Deus\textunderscore )}
\end{itemize}
Pequeno barrete, com que os padres cobrem a corôa ou pouco mais.
Pequeno barrete, usado especialmente por pessôas calvas.
\section{Solidéu}
\begin{itemize}
\item {Grp. gram.:m.}
\end{itemize}
\begin{itemize}
\item {Proveniência:(De \textunderscore soli Deo\textunderscore , loc. lat. de \textunderscore solus\textunderscore  + \textunderscore Deus\textunderscore )}
\end{itemize}
Pequeno barrete, com que os padres cobrem a corôa ou pouco mais.
Pequeno barrete, usado especialmente por pessôas calvas.
\section{Solidez}
\begin{itemize}
\item {Grp. gram.:f.}
\end{itemize}
Qualidade do que é sólido.
Qualidade de resistente ou duradoiro.
Segurança.
Fundamento.
Propriedade das substâncias ou corpos, caracterizados pela immobilidade molecular, permanência de fórma e resistência ás fôrças que tendem a desaggregar lhes as moléculas.
Fôrça de resistência.
\section{Solidificação}
\begin{itemize}
\item {Grp. gram.:f.}
\end{itemize}
Acto ou effeito de solidificar.
Propriedade, que um fluido tem, de passar ao estado de sólido.
\section{Solidificar}
\begin{itemize}
\item {Grp. gram.:v. t.}
\end{itemize}
\begin{itemize}
\item {Proveniência:(Do lat. \textunderscore solidus\textunderscore  + \textunderscore facere\textunderscore )}
\end{itemize}
Tornar sólido.
Robustecer.
Tornar estável.
Congelar.
\section{Solidismo}
\begin{itemize}
\item {Grp. gram.:m.}
\end{itemize}
\begin{itemize}
\item {Proveniência:(De sólido)}
\end{itemize}
Doutrina dos médicos que attribuem todas as doenças a lesões das partes sólidas do organismo.
\section{Solidista}
\begin{itemize}
\item {Grp. gram.:m. ,  f.  e  adj.}
\end{itemize}
\begin{itemize}
\item {Proveniência:(De \textunderscore sólido\textunderscore )}
\end{itemize}
Pessôa, partidária do solidismo.
\section{Sólido}
\begin{itemize}
\item {Grp. gram.:adj.}
\end{itemize}
\begin{itemize}
\item {Grp. gram.:M.}
\end{itemize}
\begin{itemize}
\item {Proveniência:(Lat. \textunderscore solidus\textunderscore )}
\end{itemize}
Que tem consistência.
Íntegro.
Que não tem vácuo.
Massiço.
Duro, durável.
Firme, forte.
Substâncial.
Robusto.
Que tem fundamento real; incontestável: \textunderscore doutrina sólida\textunderscore .
Aquillo que tem comprimento, largura e altura.
Cujas partes adherem, tornando diffícil a sua separação.
Aquillo que é sólido.
\section{Solidónia}
\begin{itemize}
\item {Grp. gram.:f.}
\end{itemize}
Planta brasileira, da fam. das compostas.
\section{Soliférreo}
\begin{itemize}
\item {Grp. gram.:m.}
\end{itemize}
\begin{itemize}
\item {Proveniência:(Lat. \textunderscore soliferreum\textunderscore )}
\end{itemize}
Antiga lança, toda de ferro massiço.
\section{Solífugo}
\begin{itemize}
\item {Grp. gram.:adj.}
\end{itemize}
\begin{itemize}
\item {Utilização:Poét.}
\end{itemize}
\begin{itemize}
\item {Proveniência:(Do lat. \textunderscore sol\textunderscore  + \textunderscore fugere\textunderscore )}
\end{itemize}
Que evita a luz do Sol.
Que gosta das trévas; nocturno.
\section{Solilóquio}
\begin{itemize}
\item {Grp. gram.:m.}
\end{itemize}
\begin{itemize}
\item {Proveniência:(Lat. \textunderscore soliloquium\textunderscore )}
\end{itemize}
O mesmo que \textunderscore monólogo\textunderscore .
\section{Solimão}
\begin{itemize}
\item {Grp. gram.:m.}
\end{itemize}
\begin{itemize}
\item {Utilização:Pop.}
\end{itemize}
Sublimado corrosivo.
(Cast. \textunderscore solimán\textunderscore )
\section{Solina}
\begin{itemize}
\item {Grp. gram.:f.}
\end{itemize}
\begin{itemize}
\item {Utilização:Bras}
\end{itemize}
\begin{itemize}
\item {Proveniência:(De \textunderscore Sol\textunderscore )}
\end{itemize}
Calor do Sol; soalheira.
\section{Solinhadeira}
\begin{itemize}
\item {Grp. gram.:f.}
\end{itemize}
\begin{itemize}
\item {Proveniência:(De \textunderscore solinhar\textunderscore )}
\end{itemize}
Martelo de cavouqueiro.
\section{Solinhado}
\begin{itemize}
\item {Grp. gram.:m.}
\end{itemize}
\begin{itemize}
\item {Utilização:Náut.}
\end{itemize}
Face do madeiro, parallela á xeura.
\section{Solinhar}
\begin{itemize}
\item {Grp. gram.:v. t.  e  i.}
\end{itemize}
\begin{itemize}
\item {Proveniência:(De \textunderscore so...\textunderscore  + \textunderscore linha\textunderscore )}
\end{itemize}
Lavrar pedra ou madeira, seguindo direcção marcada.
Desbastar.
\section{Solinho}
\begin{itemize}
\item {Grp. gram.:m.}
\end{itemize}
Acto de solinhar.
\section{Solinho}
\begin{itemize}
\item {Grp. gram.:m.}
\end{itemize}
\begin{itemize}
\item {Utilização:T. da Bairrada}
\end{itemize}
\begin{itemize}
\item {Proveniência:(De \textunderscore solo\textunderscore ^1)}
\end{itemize}
Terra, escavada ou mechida no fundo da manta, em que se unha o bacello.
\section{Sólio}
\begin{itemize}
\item {Grp. gram.:m.}
\end{itemize}
\begin{itemize}
\item {Utilização:Fig.}
\end{itemize}
\begin{itemize}
\item {Proveniência:(Lat. \textunderscore solium\textunderscore )}
\end{itemize}
Assento real.
Throno.
Cadeira pontifícia.
O poder real.
\section{Soliota}
\begin{itemize}
\item {Grp. gram.:f.}
\end{itemize}
\begin{itemize}
\item {Proveniência:(De \textunderscore solia\textunderscore )}
\end{itemize}
Espécie de tecido antigo. Cf. Garret, \textunderscore Escritos Diversos\textunderscore , 247.
\section{Solipé}
\begin{itemize}
\item {Grp. gram.:m.}
\end{itemize}
\begin{itemize}
\item {Utilização:Prov.}
\end{itemize}
Jôgo de rapazes, o mesmo que \textunderscore pé coxinho\textunderscore .
\section{Solípede}
\begin{itemize}
\item {Grp. gram.:adj.}
\end{itemize}
\begin{itemize}
\item {Grp. gram.:M. pl.}
\end{itemize}
\begin{itemize}
\item {Proveniência:(Lat. \textunderscore solipides\textunderscore )}
\end{itemize}
Que tem só um casco.
Família de mammíferos pachydermes, que têm um só casco.
\section{Solipsismo}
\begin{itemize}
\item {Grp. gram.:m.}
\end{itemize}
\begin{itemize}
\item {Utilização:Neol.}
\end{itemize}
\begin{itemize}
\item {Proveniência:(De \textunderscore solipso\textunderscore )}
\end{itemize}
Vida ou costumes de quem é solitário ou vive retiradamente. Cf. Castilho, \textunderscore Montalverne\textunderscore .
\section{Solipso}
\begin{itemize}
\item {Grp. gram.:m.  e  adj.}
\end{itemize}
\begin{itemize}
\item {Proveniência:(Do lat. \textunderscore solus\textunderscore  + \textunderscore ipse\textunderscore )}
\end{itemize}
Aquelle que vive só para si.
Aquelle que vive solitário.
Celibatário; solteirão.
O que é dado a prazeres solitários.
\section{Solista}
\begin{itemize}
\item {Grp. gram.:m.  e  f.}
\end{itemize}
\begin{itemize}
\item {Proveniência:(De \textunderscore solo\textunderscore ^2)}
\end{itemize}
Pessôa, que executa um solo musical, ou que é perita em música, principalmente nos solos.
\section{Solitária}
\begin{itemize}
\item {Grp. gram.:f.}
\end{itemize}
\begin{itemize}
\item {Proveniência:(De \textunderscore solitário\textunderscore )}
\end{itemize}
Animal intestinal, tênia.
Collar para adorno, semelhante aos anéis de tênia.
\section{Solitariamente}
\begin{itemize}
\item {Grp. gram.:adv.}
\end{itemize}
De modo solitário; a sós; insuladamente.
\section{Solitário}
\begin{itemize}
\item {Grp. gram.:adj.}
\end{itemize}
\begin{itemize}
\item {Grp. gram.:M.}
\end{itemize}
\begin{itemize}
\item {Utilização:Prov.}
\end{itemize}
\begin{itemize}
\item {Utilização:minh.}
\end{itemize}
\begin{itemize}
\item {Utilização:trasm.}
\end{itemize}
\begin{itemize}
\item {Proveniência:(Lat. \textunderscore solitarius\textunderscore )}
\end{itemize}
Só.
Que evita a convivência social.
Que vive no ermo.
Relativo ao ermo.
Que vive longe ou em lugar despovoado.
Abandonado de todos.
Aquelle que vive na solidão.
Aquelle que vive só, ou longe do movimento social.
Monge.
Jóia, em que há engastada uma só pedra preciosa.
Passarinho quási preto, variedade de melro, também conhecido por \textunderscore melro azul\textunderscore , (\textunderscore petrocincla cyanea\textunderscore ), e por \textunderscore melro das rochas\textunderscore  ou \textunderscore macuco\textunderscore , (\textunderscore monticola saxatilis\textunderscore , Lin.).
\section{Sólito}
\begin{itemize}
\item {Grp. gram.:adj.}
\end{itemize}
\begin{itemize}
\item {Proveniência:(Lat. \textunderscore solitus\textunderscore )}
\end{itemize}
Usado; habitual.
\section{Solmização}
\begin{itemize}
\item {Grp. gram.:f.}
\end{itemize}
\begin{itemize}
\item {Utilização:Mús.}
\end{itemize}
\begin{itemize}
\item {Utilização:Des.}
\end{itemize}
\begin{itemize}
\item {Proveniência:(De \textunderscore sol\textunderscore ^2 + \textunderscore mi\textunderscore )}
\end{itemize}
O mesmo que \textunderscore solfejo\textunderscore .
\section{Solo}
\begin{itemize}
\item {Grp. gram.:m.}
\end{itemize}
\begin{itemize}
\item {Proveniência:(Do lat. \textunderscore solum\textunderscore )}
\end{itemize}
Porção de superfície da terra; terra.
Terreno.
Chão; pavimento.
\section{Solo}
\begin{itemize}
\item {Grp. gram.:m.}
\end{itemize}
\begin{itemize}
\item {Proveniência:(Lat. \textunderscore solus\textunderscore )}
\end{itemize}
Trecho musical, para sêr executado por uma só pessôa, cantando ou tocando, com acompanhamento ou sem elle.
Dança inglesa, executada por uma só pessôa.
Jôgo de cartas, de andamento semelhante ao do voltarete e, no valor das cartas, á manilha.
\section{Solobro}
\begin{itemize}
\item {fónica:lô}
\end{itemize}
\begin{itemize}
\item {Grp. gram.:adj.}
\end{itemize}
\begin{itemize}
\item {Utilização:Ant.}
\end{itemize}
O mesmo que \textunderscore salobro\textunderscore . Cf. \textunderscore Rot. do Mar Verm.\textunderscore , 1119.
\section{Sologastro}
\begin{itemize}
\item {Grp. gram.:m.}
\end{itemize}
Gênero de echinodermes.
\section{Solo-inglês}
\begin{itemize}
\item {Grp. gram.:m.}
\end{itemize}
Dança afamada, dos princípios do século XIX.
\section{Solombra}
\begin{itemize}
\item {Grp. gram.:f.}
\end{itemize}
\begin{itemize}
\item {Utilização:Ant.}
\end{itemize}
O mesmo que \textunderscore sombra\textunderscore .
(Mirandês \textunderscore solombra\textunderscore , talvez de \textunderscore so...\textunderscore  + \textunderscore la\textunderscore  + lat. \textunderscore umbra\textunderscore )
\section{Sol-pôsto}
\begin{itemize}
\item {Grp. gram.:m.}
\end{itemize}
A hora em que o Sol desapparece do horizonte.
O pôr do Sol; occaso.
\section{Solsticial}
\begin{itemize}
\item {Grp. gram.:adj.}
\end{itemize}
\begin{itemize}
\item {Proveniência:(Lat. \textunderscore solstitialis\textunderscore )}
\end{itemize}
Relativo ao solstício.
\section{Solstício}
\begin{itemize}
\item {Grp. gram.:m.}
\end{itemize}
\begin{itemize}
\item {Proveniência:(Lat. \textunderscore solstitium\textunderscore )}
\end{itemize}
Tempo, em que o Sol, tendo-se afastado o mais possível do Equador, parece estacionário durante alguns dias, antes de começar a aproximar-se novamente do Equador.
\section{Sôlta}
\begin{itemize}
\item {Grp. gram.:f.}
\end{itemize}
\begin{itemize}
\item {Utilização:Bras. da Baía}
\end{itemize}
\begin{itemize}
\item {Utilização:ant.}
\end{itemize}
\begin{itemize}
\item {Utilização:Fig.}
\end{itemize}
\begin{itemize}
\item {Grp. gram.:Loc. adv.}
\end{itemize}
Acto ou effeito de soltar.
Peia para bêstas.
Criação de gado á solta.
Prisão.
\textunderscore Á sôlta\textunderscore , livremente, sem peias; desenfreadamente; á tuna.
(B. lat. \textunderscore solta\textunderscore )
\section{Soltador}
\begin{itemize}
\item {Grp. gram.:m.  e  adj.}
\end{itemize}
O que solta.
\section{Soltamente}
\begin{itemize}
\item {Grp. gram.:adv.}
\end{itemize}
De modo sôlto; em liberdade; licenciosamente; á sôlta.
\section{Soltamento}
\begin{itemize}
\item {Grp. gram.:m.}
\end{itemize}
Acto ou effeito de soltar. Cf. Rui Barb., \textunderscore Réplica\textunderscore , 159.
\section{Soltar}
\begin{itemize}
\item {Grp. gram.:v. t.}
\end{itemize}
\begin{itemize}
\item {Grp. gram.:V. i.}
\end{itemize}
Tornar livre.
Desembaraçar, desprender.
Desatar.
Desfraldar: \textunderscore soltar as velas\textunderscore .
Arremessar.
Atirar, expellir.
Largar da mão.
Emittir: \textunderscore soltar gemidos\textunderscore .
Pronunciar.
Desmanchar.
Explicar.
Dar quitação a.
Afroixar.
Pôr-se a caminho, partir.
(B. lat. \textunderscore soltare\textunderscore )
\section{Solteira}
\begin{itemize}
\item {Grp. gram.:f.}
\end{itemize}
\begin{itemize}
\item {Grp. gram.:Adj. f.}
\end{itemize}
\begin{itemize}
\item {Utilização:Bras. do N}
\end{itemize}
(Fem. de \textunderscore solteiro\textunderscore )
Diz-se da mulhér pública.
Diz-se das fêmeas, que não têm filhos: \textunderscore vaca solteira\textunderscore .
\section{Solteiramente}
\begin{itemize}
\item {Grp. gram.:adv.}
\end{itemize}
Á maneira de solteiro; livremente.
\section{Solteirão}
\begin{itemize}
\item {Grp. gram.:m.  e  adj.}
\end{itemize}
\begin{itemize}
\item {Proveniência:(De \textunderscore solteiro\textunderscore )}
\end{itemize}
Diz-se do homem de mais de meia idade, que ainda não casou.
\section{Solteiro}
\begin{itemize}
\item {Grp. gram.:m.  e  adj.}
\end{itemize}
\begin{itemize}
\item {Utilização:Náut.}
\end{itemize}
\begin{itemize}
\item {Proveniência:(Do b. lat. \textunderscore soltarius\textunderscore )}
\end{itemize}
O que não casou.
Diz-se dos cabos disponíveis e promptos para serviço.
\section{Solteirona}
\begin{itemize}
\item {Grp. gram.:m.  e  adj.}
\end{itemize}
\begin{itemize}
\item {Proveniência:(De \textunderscore solteirão\textunderscore )}
\end{itemize}
Diz-se da mulhér de mais de meia idade, que ainda não casou.
\section{Sôlto}
\begin{itemize}
\item {Grp. gram.:adj.}
\end{itemize}
\begin{itemize}
\item {Grp. gram.:M.}
\end{itemize}
Cujas partes não são adherentes; desaggregado: \textunderscore terra sôlta\textunderscore .
Libertino, licencioso.
Entrecortado.
Que não é rimado, (falando-se do verso).
Que não têm estação ou ancoradoiro certo, (falando-se de navios).
\textunderscore Fôlha solta\textunderscore , fôlha de papel impressa, que se espalha ou se distribue em público, como meio de propaganda.
Pasquim, que se distribue impresso.
Peixe ordinário das costas do Brasil, (\textunderscore caranx pisquetus\textunderscore ).
(B. lat. \textunderscore soltus\textunderscore )
\section{Soltura}
\begin{itemize}
\item {Grp. gram.:f.}
\end{itemize}
\begin{itemize}
\item {Proveniência:(De \textunderscore sôlto\textunderscore )}
\end{itemize}
Acto ou effeito de soltar.
Atrevimento.
Desvergonha.
Libertinagem.
Solução; interpretação.
Diarreia.
\section{Solubilidade}
\begin{itemize}
\item {Grp. gram.:f.}
\end{itemize}
Qualidade do que é solúvel.
\section{Solubilizar}
\begin{itemize}
\item {Grp. gram.:v. t.}
\end{itemize}
Tornar solúvel.
\section{Soluçante}
\begin{itemize}
\item {Grp. gram.:adj.}
\end{itemize}
Que soluça. Cf. Junqueiro, \textunderscore Musa\textunderscore , 94.
\section{Solução}
\begin{itemize}
\item {Grp. gram.:f.}
\end{itemize}
\begin{itemize}
\item {Proveniência:(Do lat. \textunderscore solutio\textunderscore )}
\end{itemize}
Acto ou effeito de solver.
Desfecho, conclusão.
Decisão.
Dissolução chímica.
Líquido, que resulta de uma dissolução.
Líquido, em que se dissolvem outras substâncias.
Interrupção.
Resolução de uma difficuldade; resolução de um problema.
Determinação da incógnita, num problema mathemático.
O mesmo que \textunderscore pagamento\textunderscore . Cf. Herculano, \textunderscore Hist. de Port.\textunderscore , III, 247 e 248.
\section{Soluçar}
\begin{itemize}
\item {Grp. gram.:v. i.}
\end{itemize}
\begin{itemize}
\item {Utilização:Fig.}
\end{itemize}
\begin{itemize}
\item {Grp. gram.:V. t.}
\end{itemize}
\begin{itemize}
\item {Grp. gram.:M.}
\end{itemize}
Dar soluços.
Sussurrar, (falando-se do mar).
Agitar-se, arfar.
Exprimir entre soluços: \textunderscore soluçar imprecações\textunderscore .
O mesmo que \textunderscore soluço\textunderscore .
\section{Solucionar}
\begin{itemize}
\item {Grp. gram.:v. t.}
\end{itemize}
\begin{itemize}
\item {Utilização:Neol.}
\end{itemize}
Dar solução a.
Resolver.
Decidir.
\section{Solucionista}
\begin{itemize}
\item {Grp. gram.:m.}
\end{itemize}
\begin{itemize}
\item {Utilização:bras}
\end{itemize}
\begin{itemize}
\item {Utilização:Neol.}
\end{itemize}
\begin{itemize}
\item {Proveniência:(De \textunderscore solução\textunderscore )}
\end{itemize}
Jogador, que aprende com facilidade a solução de um problema. Cf. \textunderscore Notícia\textunderscore , do Rio de 3-XI-900.
\section{Soluço}
\begin{itemize}
\item {Grp. gram.:m.}
\end{itemize}
\begin{itemize}
\item {Utilização:Fig.}
\end{itemize}
\begin{itemize}
\item {Proveniência:(Do b. lat. \textunderscore suggultium\textunderscore )}
\end{itemize}
Contracção espasmódica do diaphragma, seguida de distensão, em virtude da qual o pouco ar que entrara no peito é expulso com ruído.
Suspiro.
O arfar do navio.
O arfar das ondas.
Grande ruído; fragor.
\section{Soluçoso}
\begin{itemize}
\item {Grp. gram.:adj.}
\end{itemize}
Que soluça.
Que se exprime entre soluços.
\section{Solutivo}
\begin{itemize}
\item {Grp. gram.:adj.}
\end{itemize}
\begin{itemize}
\item {Proveniência:(De \textunderscore soluto\textunderscore )}
\end{itemize}
Que póde solver ou dissolver.
Laxante.
\section{Soluto}
\begin{itemize}
\item {Grp. gram.:adj.}
\end{itemize}
\begin{itemize}
\item {Grp. gram.:M.}
\end{itemize}
\begin{itemize}
\item {Proveniência:(Lat. \textunderscore solutus\textunderscore )}
\end{itemize}
O mesmo que \textunderscore sôlto\textunderscore .
Dissolvido.
O mesmo que \textunderscore solução\textunderscore .
\section{Solutol}
\begin{itemize}
\item {Grp. gram.:m.}
\end{itemize}
\begin{itemize}
\item {Proveniência:(De \textunderscore soluto\textunderscore  + \textunderscore óleo\textunderscore )}
\end{itemize}
Substância desinfectante, composta de cresilol, tornado solúvel pela addição de cresilato de soda.
\section{Solutreano}
\begin{itemize}
\item {Grp. gram.:adj.}
\end{itemize}
\begin{itemize}
\item {Utilização:Geol.}
\end{itemize}
Diz-se do terreno, que constitue o terceiro andar da série quaternária, segundo Mortillet.
\section{Solúvel}
\begin{itemize}
\item {Grp. gram.:adj.}
\end{itemize}
\begin{itemize}
\item {Proveniência:(Lat. \textunderscore solubilis\textunderscore )}
\end{itemize}
Que se póde solver, resolver ou dissolver.
\section{Solvabilidade}
\begin{itemize}
\item {Grp. gram.:f.}
\end{itemize}
\begin{itemize}
\item {Utilização:Gal}
\end{itemize}
Qualidade do que é solvável.
\section{Solvável}
\begin{itemize}
\item {Grp. gram.:adj.}
\end{itemize}
\begin{itemize}
\item {Utilização:Gal}
\end{itemize}
(V.solvível)
\section{Solvência}
\begin{itemize}
\item {Grp. gram.:f.}
\end{itemize}
Solução.
Qualidade do que é solvente.
Solvibilidade.
\section{Solvente}
\begin{itemize}
\item {Grp. gram.:adj.}
\end{itemize}
\begin{itemize}
\item {Proveniência:(Lat. \textunderscore solvens\textunderscore )}
\end{itemize}
Que solve ou póde solver.
Que paga ou póde pagar o que deve: \textunderscore devedor solvente\textunderscore .
\section{Solveol}
\begin{itemize}
\item {Grp. gram.:m.}
\end{itemize}
Substância desinfectante, composta de cresilol e creosotinato de soda.
\section{Solver}
\begin{itemize}
\item {Grp. gram.:v. t.}
\end{itemize}
\begin{itemize}
\item {Proveniência:(Lat. \textunderscore solvere\textunderscore )}
\end{itemize}
Separar, desligar.
Resolver.
Desatar.
Explicar.
Resolver; dissolver.
Pagar; satisfazer: \textunderscore solver dividas\textunderscore .
\section{Solvibilidade}
\begin{itemize}
\item {Grp. gram.:f.}
\end{itemize}
Qualidade do que é solvível.
\section{Solvível}
\begin{itemize}
\item {Grp. gram.:adj.}
\end{itemize}
\begin{itemize}
\item {Proveniência:(De \textunderscore solver\textunderscore )}
\end{itemize}
Que póde pagar o que deve; solvente.
Que se póde solver; que se póde pagar: \textunderscore divida solvível\textunderscore .
\section{Som}
\begin{itemize}
\item {Grp. gram.:m.}
\end{itemize}
\begin{itemize}
\item {Utilização:Fig.}
\end{itemize}
\begin{itemize}
\item {Proveniência:(Do lat. \textunderscore sonus\textunderscore )}
\end{itemize}
Effeito, produzido no sentido da audição, pelas vibrações dos corpos sonoros.
Aquillo que impressiona o ouvido.
Ruído.
Emissão de voz; voz.
Maneira.
\section{Soma}
\begin{itemize}
\item {Grp. gram.:m.}
\end{itemize}
Bebida que, segundo a crença dos Árias, dá aos justos o privilégio de conservar a immortalidade do corpo.
(Do sanscr.)
\section{Soma}
\begin{itemize}
\item {Grp. gram.:m.}
\end{itemize}
Chefe de tríbo ao sul de Angola.
\section{Soma}
\begin{itemize}
\item {Grp. gram.:f.}
\end{itemize}
\begin{itemize}
\item {Utilização:Fig.}
\end{itemize}
\begin{itemize}
\item {Proveniência:(Do lat. \textunderscore summa\textunderscore )}
\end{itemize}
Resultado de quantidades addicionadas.
Operação matemática, para se achar aquele resultado.
Adição.
Grande porção: \textunderscore soma de dinheiro\textunderscore .
Resumo, síntese.
Quantia de dinheiro: \textunderscore fui pagar-lhe uma soma\textunderscore .
O mesmo que \textunderscore summa\textunderscore . Cf. Sousa, \textunderscore Vida do Arceb.\textunderscore , I, 243.
\section{Soma}
\begin{itemize}
\item {Grp. gram.:adv.}
\end{itemize}
\begin{itemize}
\item {Utilização:Ant.}
\end{itemize}
Em summa; em-fim. Cf. G. Vicente, \textunderscore Inês Pereira\textunderscore .
(Cp. \textunderscore summa\textunderscore )
\section{Somada}
\begin{itemize}
\item {Grp. gram.:f.}
\end{itemize}
(V.assomada)
\section{Somalis}
\begin{itemize}
\item {Grp. gram.:m. pl.}
\end{itemize}
Habitantes de Somal, na África.
\section{Somar}
\begin{itemize}
\item {Grp. gram.:v. t.}
\end{itemize}
\begin{itemize}
\item {Utilização:Fig.}
\end{itemize}
\begin{itemize}
\item {Grp. gram.:V. i.}
\end{itemize}
Fazer ou procurar a soma de.
Adicionar.
Sêr equivalente a.
Resumir.
Fazer a operação da soma.
\section{Somasco}
\begin{itemize}
\item {Grp. gram.:m.}
\end{itemize}
\begin{itemize}
\item {Proveniência:(De \textunderscore Somasco\textunderscore , n. p.)}
\end{itemize}
Membro de uma antiga congregação religiosa em Itália, que se propunha especialmente a educar e instruír crianças.
\section{Somático}
\begin{itemize}
\item {Grp. gram.:adj.}
\end{itemize}
\begin{itemize}
\item {Utilização:Physiol.}
\end{itemize}
\begin{itemize}
\item {Proveniência:(Gr. \textunderscore somatikos\textunderscore )}
\end{itemize}
Relativo ao corpo.
\section{Somatista}
\begin{itemize}
\item {Grp. gram.:m.}
\end{itemize}
Partidário da doutrina que attribue a loucura a lesões materiaes do systema nervoso e não a causas psýchicas.
(Cp. \textunderscore somático\textunderscore )
\section{Somatochromo}
\begin{itemize}
\item {Grp. gram.:adj.}
\end{itemize}
\begin{itemize}
\item {Proveniência:(Do gr. \textunderscore soma\textunderscore  + \textunderscore khroma\textunderscore )}
\end{itemize}
Diz-se das partes do protoplasma cellular, que tomam côr pela acção de certas substâncias còrantes.
\section{Somatocromo}
\begin{itemize}
\item {Grp. gram.:adj.}
\end{itemize}
\begin{itemize}
\item {Proveniência:(Do gr. \textunderscore soma\textunderscore  + \textunderscore khroma\textunderscore )}
\end{itemize}
Diz-se das partes do protoplasma cellular, que tomam côr pela acção de certas substâncias còrantes.
\section{Somatologia}
\begin{itemize}
\item {Grp. gram.:f.}
\end{itemize}
\begin{itemize}
\item {Proveniência:(Do gr. \textunderscore soma\textunderscore  + \textunderscore logos\textunderscore )}
\end{itemize}
Tratado do corpo humano.
\section{Somatológico}
\begin{itemize}
\item {Grp. gram.:adj.}
\end{itemize}
Relativo á somatologia.
\section{Somatólogo}
\begin{itemize}
\item {Grp. gram.:m.}
\end{itemize}
Aquelle que é perito em somatologia.
\section{Somatório}
\begin{itemize}
\item {Grp. gram.:m.}
\end{itemize}
\begin{itemize}
\item {Utilização:Fig.}
\end{itemize}
\begin{itemize}
\item {Grp. gram.:Adj.}
\end{itemize}
\begin{itemize}
\item {Proveniência:(De \textunderscore somar\textunderscore )}
\end{itemize}
Soma geral.
Totalidade.
Indicativo de uma soma.
\section{Somatose}
\begin{itemize}
\item {Grp. gram.:f.}
\end{itemize}
\begin{itemize}
\item {Utilização:Pharm.}
\end{itemize}
\begin{itemize}
\item {Proveniência:(Do gr. \textunderscore soma\textunderscore , \textunderscore somatos\textunderscore )}
\end{itemize}
Nome de uma preparação dietética, de albumina de carne.
\section{Sombo}
\begin{itemize}
\item {Grp. gram.:m.}
\end{itemize}
O mesmo que \textunderscore pau-óleo\textunderscore .
\section{Sombra}
\begin{itemize}
\item {Grp. gram.:f.}
\end{itemize}
\begin{itemize}
\item {Grp. gram.:Loc. adv.}
\end{itemize}
\begin{itemize}
\item {Utilização:pop.}
\end{itemize}
\begin{itemize}
\item {Grp. gram.:Loc.}
\end{itemize}
\begin{itemize}
\item {Utilização:fam.}
\end{itemize}
\begin{itemize}
\item {Proveniência:(Do lat. \textunderscore sub\textunderscore  + \textunderscore umbra\textunderscore )}
\end{itemize}
Espaço, que a interposição de um corpo opaco privou de luz ou tornou menos claro.
Escuridão.
Noite.
Defeito, nódoa: \textunderscore reputação sem sombras\textunderscore .
A parte escura de um quadro ou desenho.
Aspecto; apparência: \textunderscore homem de bôa sombra\textunderscore .
Espírito.
Fantasma.
Figura, que traz á ideia um espectro.
Guarda-costas.
Pessôa, que acompanha outra sempre.
Pessôa impertinente, que não deixa outra.
Noções rudimentares, tintura.
Protecção: \textunderscore viver á sombra de alguém\textunderscore .
Favor.
Pessôa ou coisa decadente: \textunderscore aquillo é uma sombra do que foi\textunderscore .
Imagem imperfeita.
Bandeira de candeeiro ou de vela; pantalha, sombreira.
Solidão: \textunderscore apraz-lhe viver na sombra\textunderscore .
Mystério.
Aspecto, semblante, modos: \textunderscore recebeu-o com bôa sombra\textunderscore .
\textunderscore Á sombra\textunderscore , na cadeia.
\textunderscore Olhar para a sombra\textunderscore , começar a ter pretenções, vaidade; namoriscar.
\textunderscore De má sombra\textunderscore , mal encarado. Cf. Camillo, \textunderscore Enjeitado\textunderscore , 126.
\section{Sombral}
\begin{itemize}
\item {Grp. gram.:m.}
\end{itemize}
\begin{itemize}
\item {Proveniência:(De \textunderscore sombra\textunderscore )}
\end{itemize}
Lugar, sombrio.
Lugar abrigado do sol por latadas ou arvoredo.
\section{Sombrar}
\begin{itemize}
\item {Grp. gram.:v. t.}
\end{itemize}
\begin{itemize}
\item {Utilização:Des.}
\end{itemize}
O mesmo que \textunderscore assombrar\textunderscore . Cf. Barros, \textunderscore Déc.\textunderscore  I.
\section{Sombreado}
\begin{itemize}
\item {Grp. gram.:adj.}
\end{itemize}
\begin{itemize}
\item {Grp. gram.:M.}
\end{itemize}
\begin{itemize}
\item {Proveniência:(De \textunderscore sombrear\textunderscore )}
\end{itemize}
Em que há sombra.
Gradação do escuro, num quadro ou desenho.
\section{Sombrear}
\begin{itemize}
\item {Grp. gram.:v. t.}
\end{itemize}
\begin{itemize}
\item {Grp. gram.:V. i.}
\end{itemize}
Dar sombra a.
Manchar.
Desgostar.
Dar sombreado a uma tela, a um desenho, etc.
\section{Sombreira}
\begin{itemize}
\item {Grp. gram.:f.}
\end{itemize}
\begin{itemize}
\item {Proveniência:(De \textunderscore sombra\textunderscore )}
\end{itemize}
Bandeira de candeeiro ou de vela.
Quebra-luz, pantalha.
Designação antiga de uma planta medicinal, que também se dizia \textunderscore sombreira de lágrima\textunderscore .
\section{Sombreireiro}
\begin{itemize}
\item {Grp. gram.:m.}
\end{itemize}
Fabricante ou vendedor de sombreiros; chapeleiro.
\section{Sobreirinho-dos-telhados}
\begin{itemize}
\item {Grp. gram.:m.}
\end{itemize}
Erva crassulácea, o mesmo que \textunderscore conchelo\textunderscore .
\section{Sombreiro}
\begin{itemize}
\item {Grp. gram.:m.}
\end{itemize}
\begin{itemize}
\item {Utilização:Prov.}
\end{itemize}
\begin{itemize}
\item {Utilização:beir.}
\end{itemize}
\begin{itemize}
\item {Grp. gram.:Adj.}
\end{itemize}
\begin{itemize}
\item {Proveniência:(De \textunderscore sombra\textunderscore )}
\end{itemize}
Aquillo que dá sombra.
Chapéu.
Guarda-sol.
Nome, que os mareantes deram a um peixe extraordinário, do mar das Índias. Cf. Barros, \textunderscore Déc.\textunderscore  III, l. IV, c. 7.
Que dá sombra, que faz sombra.«\textunderscore ...plátanos sombreiros.\textunderscore »Castilho. \textunderscore Geôrgicas\textunderscore .
\section{Sombrejar}
\begin{itemize}
\item {Grp. gram.:v. t.}
\end{itemize}
O mesmo que \textunderscore sombrear\textunderscore .
\section{Sombrela}
\begin{itemize}
\item {Grp. gram.:f.}
\end{itemize}
\begin{itemize}
\item {Proveniência:(De \textunderscore sombra\textunderscore )}
\end{itemize}
Vaso ou campânula, com que se resguardam da intempérie as plantas mimosas.
\section{Sombria}
\begin{itemize}
\item {Grp. gram.:f.}
\end{itemize}
\begin{itemize}
\item {Proveniência:(De \textunderscore sombrio\textunderscore )}
\end{itemize}
Passaro dentirostro, semelhante á cotovia.
\section{Sombrífero}
\begin{itemize}
\item {Grp. gram.:adj.}
\end{itemize}
(V.umbrífero). Cf. Pato Moniz, \textunderscore Apparição\textunderscore , 31.
\section{Sombrinha}
\begin{itemize}
\item {Grp. gram.:f.}
\end{itemize}
\begin{itemize}
\item {Grp. gram.:Pl.}
\end{itemize}
\begin{itemize}
\item {Proveniência:(De \textunderscore sombra\textunderscore )}
\end{itemize}
Pequeno guarda-sol, para senhoras.
Scenas ou paisagens, observadas por apparelho de phantasmagoria; fantoches.
\section{Sombrio}
\begin{itemize}
\item {Grp. gram.:adj.}
\end{itemize}
\begin{itemize}
\item {Grp. gram.:M.}
\end{itemize}
\begin{itemize}
\item {Utilização:Gír.}
\end{itemize}
Que produz sombra; em que há sombra.
Que não é exposto ao sol.
Escuro; triste.
Torvo.
Que desconsola.
Severo; ríspido; despótico: \textunderscore carácter sombrio\textunderscore .
Lugar sombrio.
Iscas de figado de porco.
\section{Sombró}
\begin{itemize}
\item {Grp. gram.:m.}
\end{itemize}
Árvore da Índia Portuguesa.
\section{Sombroso}
\begin{itemize}
\item {Grp. gram.:adj.}
\end{itemize}
Que produz sombra; em que há sombra; sombrio.
\section{Someiro}
\begin{itemize}
\item {Grp. gram.:m.}
\end{itemize}
\begin{itemize}
\item {Proveniência:(Do cast. \textunderscore somero\textunderscore )}
\end{itemize}
Nome de duas peças nos antigos prelos.
Pequena trave, que serve de vêrga ou tórça, em porta ou janela.
Espécie de caixa, a que está ligado o folle dos órgãos.
Pedra, que sustenta outra em que se firma urna platibanda.
Pedra, talhada de fórma que, assente sôbre columna ou pé direito, recebe a primeira aduela de uma abóbada.
\section{Someiro}
\begin{itemize}
\item {Grp. gram.:adj.}
\end{itemize}
\begin{itemize}
\item {Utilização:Açor}
\end{itemize}
O mesmo que \textunderscore somítico\textunderscore .
\section{Somenos}
\begin{itemize}
\item {Grp. gram.:adj.}
\end{itemize}
\begin{itemize}
\item {Proveniência:(De \textunderscore so...\textunderscore  + \textunderscore menos\textunderscore )}
\end{itemize}
Inferior.
Que vale menos que outro.
Ordinário; reles.
\section{Sòmente}
\begin{itemize}
\item {Grp. gram.:adv.}
\end{itemize}
\begin{itemize}
\item {Proveniência:(De \textunderscore só\textunderscore )}
\end{itemize}
Unicamente; exclusivamente; só.
\section{Sòmentes}
\begin{itemize}
\item {Grp. gram.:adj.}
\end{itemize}
\begin{itemize}
\item {Utilização:ant.}
\end{itemize}
\begin{itemize}
\item {Utilização:Pop.}
\end{itemize}
O mesmo que \textunderscore sòmente\textunderscore . Cf. Pant. de Aveiro, \textunderscore Itiner.\textunderscore , 8, (2.^a ed.).
\section{Somergulhar}
\begin{itemize}
\item {Grp. gram.:v. t.}
\end{itemize}
\begin{itemize}
\item {Utilização:Ant.}
\end{itemize}
\begin{itemize}
\item {Proveniência:(De \textunderscore so...\textunderscore  + \textunderscore mergulhar\textunderscore )}
\end{itemize}
O mesmo que \textunderscore submergir\textunderscore . Cf. \textunderscore Rev. Lus.\textunderscore , XVI, 11.
\section{Someter}
\begin{itemize}
\item {Grp. gram.:v. t.}
\end{itemize}
\begin{itemize}
\item {Utilização:Pop.}
\end{itemize}
\begin{itemize}
\item {Utilização:Ant.}
\end{itemize}
\begin{itemize}
\item {Proveniência:(De \textunderscore so...\textunderscore  + \textunderscore meter\textunderscore )}
\end{itemize}
Meter por baixo de alguma coisa.
Aconchegar tanto, que fique quási debaixo.
O mesmo que \textunderscore subjugar\textunderscore  ou \textunderscore submeter\textunderscore :«\textunderscore ...mas Fé, subcujo suave jugo sometereis o mundo.\textunderscore »\textunderscore Eufrosina\textunderscore , 13.
\section{Sometimente}
\begin{itemize}
\item {Grp. gram.:m.}
\end{itemize}
\begin{itemize}
\item {Utilização:Ant.}
\end{itemize}
Acto de someter.
Suggestão ou inspiração diabólica.
Mau conselho.
\section{Somitarro}
\begin{itemize}
\item {Grp. gram.:adj.}
\end{itemize}
\begin{itemize}
\item {Utilização:Açor}
\end{itemize}
Muito somítico. (Colhido em S. Jorge)
\section{Somiticaria}
\begin{itemize}
\item {Grp. gram.:f.}
\end{itemize}
Qualidade ou acção de quem é somítico; sovinice; avareza sórdida.
\section{Somítico}
\begin{itemize}
\item {Grp. gram.:adj.}
\end{itemize}
Avarento; sovina; ridículo.
(Contr. de \textunderscore sodomítico\textunderscore ?)
\section{Somítigo}
\begin{itemize}
\item {Grp. gram.:adj.}
\end{itemize}
\begin{itemize}
\item {Utilização:Pop.}
\end{itemize}
O mesmo que \textunderscore somítico\textunderscore .
\section{Somito}
\begin{itemize}
\item {Grp. gram.:m.}
\end{itemize}
\begin{itemize}
\item {Proveniência:(Do gr. \textunderscore soma\textunderscore , corpo)}
\end{itemize}
O mesmo que \textunderscore porto-vértebra\textunderscore .
\section{Somma}
\begin{itemize}
\item {Grp. gram.:f.}
\end{itemize}
\begin{itemize}
\item {Utilização:Fig.}
\end{itemize}
\begin{itemize}
\item {Proveniência:(Do lat. \textunderscore summa\textunderscore )}
\end{itemize}
Resultado de quantidades addicionadas.
Operação mathemática, para se achar aquelle resultado.
Addição.
Grande porção: \textunderscore somma de dinheiro\textunderscore .
Resumo, sýnthese.
Quantia de dinheiro: \textunderscore fui pagar-lhe uma soma\textunderscore .
O mesmo que \textunderscore summa\textunderscore . Cf. Sousa, \textunderscore Vida do Arceb.\textunderscore , I, 243.
\section{Sommar}
\begin{itemize}
\item {Grp. gram.:v. t.}
\end{itemize}
\begin{itemize}
\item {Utilização:Fig.}
\end{itemize}
\begin{itemize}
\item {Grp. gram.:V. i.}
\end{itemize}
Fazer ou procurar a somma de.
Addicionar.
Sêr equivalente a.
Resumir.
Fazer a operação da somma.
\section{Sommatório}
\begin{itemize}
\item {Grp. gram.:m.}
\end{itemize}
\begin{itemize}
\item {Utilização:Fig.}
\end{itemize}
\begin{itemize}
\item {Grp. gram.:Adj.}
\end{itemize}
\begin{itemize}
\item {Proveniência:(De \textunderscore sommar\textunderscore )}
\end{itemize}
Somma geral.
Totalidade.
Indicativo de uma somma.
\section{Somnambulismo}
\begin{itemize}
\item {Grp. gram.:m.}
\end{itemize}
Estado de quem é somnâmbulo.
\section{Somnambulizar}
\begin{itemize}
\item {Grp. gram.:v. t.}
\end{itemize}
Tornar somnâmbulo.
\section{Somnâmbulo}
\begin{itemize}
\item {Grp. gram.:adj.}
\end{itemize}
\begin{itemize}
\item {Grp. gram.:M.}
\end{itemize}
\begin{itemize}
\item {Proveniência:(Do lat. \textunderscore somnus\textunderscore  + \textunderscore ambulare\textunderscore )}
\end{itemize}
Que, dormindo, fala, levanta-se, anda e realiza outros actos, habituaes em quem está acordado.
Homem somnâmbulo.
\section{Somnata}
\begin{itemize}
\item {Grp. gram.:f.}
\end{itemize}
O mesmo que \textunderscore somneca\textunderscore .
\section{Somneca}
\begin{itemize}
\item {Grp. gram.:f.}
\end{itemize}
O mesmo que \textunderscore somnolência\textunderscore .
Curto espaço de tempo, que se passa dormindo.
\section{Somneira}
\begin{itemize}
\item {Grp. gram.:f.}
\end{itemize}
O mesmo que \textunderscore somnolência\textunderscore .
\section{Somnial}
\begin{itemize}
\item {Grp. gram.:adj.}
\end{itemize}
\begin{itemize}
\item {Proveniência:(Lat. \textunderscore somnialis\textunderscore )}
\end{itemize}
Relativo aos sonhos.
\section{Somniculoso}
\begin{itemize}
\item {Grp. gram.:adj.}
\end{itemize}
\begin{itemize}
\item {Proveniência:(Lat. \textunderscore somniculosus\textunderscore )}
\end{itemize}
Somnolento, dorminhoco.
\section{Somnífero}
\begin{itemize}
\item {Grp. gram.:adj.}
\end{itemize}
\begin{itemize}
\item {Utilização:Poét.}
\end{itemize}
\begin{itemize}
\item {Grp. gram.:M.}
\end{itemize}
\begin{itemize}
\item {Proveniência:(Lat. \textunderscore somnifer\textunderscore )}
\end{itemize}
Que faz somno.
Substância soporífera.
\section{Somnígrapho}
\begin{itemize}
\item {Grp. gram.:m.}
\end{itemize}
\begin{itemize}
\item {Utilização:Neol.}
\end{itemize}
\begin{itemize}
\item {Proveniência:(Do lat. \textunderscore somnum\textunderscore  + gr. \textunderscore graphein\textunderscore )}
\end{itemize}
Aquelle que descreve sonhos. Cf. Garrett, \textunderscore Viagens\textunderscore , I, 30.
\section{Somnílogo}
\begin{itemize}
\item {Grp. gram.:m.}
\end{itemize}
\begin{itemize}
\item {Utilização:Neol.}
\end{itemize}
\begin{itemize}
\item {Proveniência:(Do lat. \textunderscore somnum\textunderscore  + gr. \textunderscore logos\textunderscore )}
\end{itemize}
Aquelle que sabe interpretar sonhos. Cf. Garrett, \textunderscore Viagens\textunderscore , I, 30.
\section{Somníloquo}
\begin{itemize}
\item {Grp. gram.:m.  e  adj.}
\end{itemize}
\begin{itemize}
\item {Proveniência:(Do lat. \textunderscore somnus\textunderscore  + \textunderscore loqui\textunderscore )}
\end{itemize}
O que fala a dormir.
\section{Somno}
\begin{itemize}
\item {Grp. gram.:m.}
\end{itemize}
\begin{itemize}
\item {Utilização:Fig.}
\end{itemize}
\begin{itemize}
\item {Proveniência:(Lat. \textunderscore somnus\textunderscore )}
\end{itemize}
Completo adormecimento dos sentidos.
Cessação momentânea da actividade, própria dos systemas que têm as propriedades da vida animal.
Desejo de dormir ou sentimento da necessidade de dormir: \textunderscore tenho somno\textunderscore .
Estado de quem dorme: \textunderscore caiu no somno\textunderscore .
Inércia: preguiça; indolência.
\textunderscore Doença do somno\textunderscore , doença endêmica, peculiar á raça negra, produzida pela picada de um insecto e caracterizada por um sopor profundo, a que a morte quasi sempre põe termo; trypanosomose.
\section{Somnolência}
\begin{itemize}
\item {Grp. gram.:f.}
\end{itemize}
\begin{itemize}
\item {Utilização:Fig.}
\end{itemize}
\begin{itemize}
\item {Proveniência:(Lat. \textunderscore somnolentia\textunderscore )}
\end{itemize}
Somno imperfeito.
Momento de transição, entre o estado de quem dorme e o acordar.
Disposição para dormir; modorra.
Inércia, entorpecimento.
\section{Somnolento}
\begin{itemize}
\item {Grp. gram.:adj.}
\end{itemize}
\begin{itemize}
\item {Utilização:Fig.}
\end{itemize}
\begin{itemize}
\item {Proveniência:(Lat. \textunderscore somnolentus\textunderscore )}
\end{itemize}
Que tem somnolência.
Que causa somno.
Relativo á somnolência.
Vagaroso; inerte.
\section{Somnurno}
\begin{itemize}
\item {Grp. gram.:adj.}
\end{itemize}
\begin{itemize}
\item {Proveniência:(Lat. \textunderscore somnurnus\textunderscore )}
\end{itemize}
Relativo ao somno.
Que se vê em sonhos; fantástico.
\section{Sona}
\begin{itemize}
\item {Grp. gram.:adj.}
\end{itemize}
\begin{itemize}
\item {Utilização:Gír.}
\end{itemize}
\begin{itemize}
\item {Proveniência:(De \textunderscore somno\textunderscore )}
\end{itemize}
Preguiçoso.
\section{Sonada}
\begin{itemize}
\item {Grp. gram.:f.}
\end{itemize}
\begin{itemize}
\item {Utilização:Ant.}
\end{itemize}
O mesmo que \textunderscore sonata\textunderscore ^2.
\section{Sonador}
\begin{itemize}
\item {Grp. gram.:adj.}
\end{itemize}
\begin{itemize}
\item {Utilização:Bras. do S}
\end{itemize}
\begin{itemize}
\item {Proveniência:(Do lat. \textunderscore sonare\textunderscore )}
\end{itemize}
Diz-se do cavallo que, galopando, faz sair pela bôca e ventas uma espécie de ronco.
\section{Sonaja}
\begin{itemize}
\item {Grp. gram.:f.}
\end{itemize}
\begin{itemize}
\item {Proveniência:(Do cast. \textunderscore sonaja\textunderscore )}
\end{itemize}
Espécie de ferrinhos.
\section{Sonalha}
\begin{itemize}
\item {Grp. gram.:f.}
\end{itemize}
\begin{itemize}
\item {Utilização:Mús.}
\end{itemize}
\begin{itemize}
\item {Proveniência:(Do cast. \textunderscore sonaja\textunderscore )}
\end{itemize}
Espécie de ferrinhos.
\section{Sonambulismo}
\begin{itemize}
\item {Grp. gram.:m.}
\end{itemize}
Estado de quem é sonâmbulo.
\section{Sonambulizar}
\begin{itemize}
\item {Grp. gram.:v. t.}
\end{itemize}
Tornar sonâmbulo.
\section{Sonâmbulo}
\begin{itemize}
\item {Grp. gram.:adj.}
\end{itemize}
\begin{itemize}
\item {Grp. gram.:M.}
\end{itemize}
\begin{itemize}
\item {Proveniência:(Do lat. \textunderscore somnus\textunderscore  + \textunderscore ambulare\textunderscore )}
\end{itemize}
Que, dormindo, fala, levanta-se, anda e realiza outros actos, habituaes em quem está acordado.
Homem sonâmbulo.
\section{Sonância}
\begin{itemize}
\item {Grp. gram.:f.}
\end{itemize}
Qualidade daquillo que é sonante.
Melodia, música:«\textunderscore eu canto em minha viola sonâncias do meu paiz\textunderscore ». Junqueira Freire.
\section{Sonante}
\begin{itemize}
\item {Grp. gram.:adj.}
\end{itemize}
\begin{itemize}
\item {Proveniência:(Lat. \textunderscore sonans\textunderscore )}
\end{itemize}
Que sôa: \textunderscore metal sonante\textunderscore .
\section{Sonar}
\begin{itemize}
\item {Grp. gram.:v. i.}
\end{itemize}
\begin{itemize}
\item {Utilização:Ant.}
\end{itemize}
\begin{itemize}
\item {Utilização:Gír.}
\end{itemize}
\begin{itemize}
\item {Proveniência:(De \textunderscore sono\textunderscore  = \textunderscore somno\textunderscore )}
\end{itemize}
Dormir.
\section{Sonarento}
\begin{itemize}
\item {Grp. gram.:adj.}
\end{itemize}
O mesmo que \textunderscore somnolento\textunderscore .
(Por \textunderscore somnolento\textunderscore )
\section{Sonata}
\begin{itemize}
\item {Grp. gram.:f.}
\end{itemize}
\begin{itemize}
\item {Proveniência:(It. \textunderscore sonata\textunderscore )}
\end{itemize}
Peça musical para instrumentos, divergindo as partes della em carácter e andamento.
\section{Sonata}
\begin{itemize}
\item {Grp. gram.:f.}
\end{itemize}
O mesmo que \textunderscore soneca\textunderscore .
\section{Sonatina}
\begin{itemize}
\item {Grp. gram.:f.}
\end{itemize}
Pequena sonata.
\section{Sonave}
\begin{itemize}
\item {Grp. gram.:f.}
\end{itemize}
\begin{itemize}
\item {Utilização:T. do Fundão}
\end{itemize}
O mesmo que \textunderscore viga\textunderscore .
\section{Sonavota}
\begin{itemize}
\item {Grp. gram.:f.}
\end{itemize}
\begin{itemize}
\item {Utilização:Prov.}
\end{itemize}
\begin{itemize}
\item {Proveniência:(De \textunderscore sonave\textunderscore )}
\end{itemize}
Viga menos grossa que a sonave.
\section{Sonda}
\begin{itemize}
\item {Grp. gram.:f.}
\end{itemize}
\begin{itemize}
\item {Utilização:Fig.}
\end{itemize}
\begin{itemize}
\item {Proveniência:(De \textunderscore so...\textunderscore  + \textunderscore onda\textunderscore )}
\end{itemize}
Prumo ou objecto análogo, com que se avalia ou examina a profundeza das aguas, o interior de um objecto, estado de um órgão ou de um ferimento, etc.
Effeito de uma sondagem.
Investigação; meio de investigação.
Profundidade.
\section{Sondador}
\begin{itemize}
\item {Grp. gram.:m.  e  adj.}
\end{itemize}
\begin{itemize}
\item {Utilização:Gír.}
\end{itemize}
\begin{itemize}
\item {Proveniência:(De \textunderscore sondar\textunderscore )}
\end{itemize}
O que sonda.
Guarda fiscal ou guarda-barreira.
\section{Sondagem}
\begin{itemize}
\item {Grp. gram.:f.}
\end{itemize}
Acto ou effeito de sondar.
\section{Sondar}
\begin{itemize}
\item {Grp. gram.:v. t.}
\end{itemize}
\begin{itemize}
\item {Grp. gram.:V. i.}
\end{itemize}
\begin{itemize}
\item {Utilização:Gír.}
\end{itemize}
\begin{itemize}
\item {Proveniência:(De \textunderscore sonda\textunderscore )}
\end{itemize}
Examinar com a sonda.
Investigar.
Tactear; explorar.
Morrer.
\section{Sondareza}
\begin{itemize}
\item {Grp. gram.:adj. f.}
\end{itemize}
\begin{itemize}
\item {Proveniência:(De \textunderscore sondar\textunderscore )}
\end{itemize}
Corda graduada que, ligada a um pedaço de chumbo, serve para as sondagens marítimas.
\section{Sondável}
\begin{itemize}
\item {Grp. gram.:adj.}
\end{itemize}
Que se póde sondar.
\section{Sondeque}
\begin{itemize}
\item {Grp. gram.:m.}
\end{itemize}
\begin{itemize}
\item {Utilização:Gír.}
\end{itemize}
Bofetada.
\section{Soneca}
\begin{itemize}
\item {Grp. gram.:f.}
\end{itemize}
\begin{itemize}
\item {Utilização:Fam.}
\end{itemize}
\begin{itemize}
\item {Grp. gram.:f.}
\end{itemize}
O mesmo ou melhor que \textunderscore somneca\textunderscore .
O mesmo que \textunderscore sonolência\textunderscore .
Curto espaço de tempo, que se passa dormindo.
\section{Sonega}
\begin{itemize}
\item {Grp. gram.:f.}
\end{itemize}
Acto ou effeito de sonegar.
\section{Sonegação}
\begin{itemize}
\item {Grp. gram.:f.}
\end{itemize}
Acto ou effeito de sonegar.
\section{Sonegadamente}
\begin{itemize}
\item {Grp. gram.:adv.}
\end{itemize}
\begin{itemize}
\item {Proveniência:(De \textunderscore sonegar\textunderscore )}
\end{itemize}
Com sonegação; ás occultas.
\section{Sonegador}
\begin{itemize}
\item {Grp. gram.:m.  e  adj.}
\end{itemize}
O que sonega.
\section{Sonegados}
\begin{itemize}
\item {Grp. gram.:m. pl.}
\end{itemize}
\begin{itemize}
\item {Proveniência:(De \textunderscore sonegar\textunderscore )}
\end{itemize}
Objectos sonegados.
\section{Sonegamento}
\begin{itemize}
\item {Grp. gram.:m.}
\end{itemize}
O mesmo que \textunderscore sonegação\textunderscore .
\section{Sofia}
\begin{itemize}
\item {Grp. gram.:f.}
\end{itemize}
\begin{itemize}
\item {Proveniência:(Gr. \textunderscore sophia\textunderscore )}
\end{itemize}
O mesmo que \textunderscore ciência\textunderscore . Cf. Camões, ode III.
\section{Sofisma}
\begin{itemize}
\item {Grp. gram.:m.}
\end{itemize}
\begin{itemize}
\item {Proveniência:(Lat. \textunderscore sophisma\textunderscore )}
\end{itemize}
Argumento falso, ou falso raciocínio com alguma aparência de verdade.
\section{Sofismar}
\begin{itemize}
\item {Grp. gram.:v. t.}
\end{itemize}
\begin{itemize}
\item {Utilização:Fig.}
\end{itemize}
\begin{itemize}
\item {Grp. gram.:V. i.}
\end{itemize}
Encobrir com sofisma; discutir, sofismando.
Enganar.
Empregar sofisma.
\section{Sofismável}
\begin{itemize}
\item {Grp. gram.:adj.}
\end{itemize}
Que se póde sofismar.
\section{Sofista}
\begin{itemize}
\item {Grp. gram.:m. ,  f.  e  adj.}
\end{itemize}
\begin{itemize}
\item {Proveniência:(Lat. \textunderscore sophista\textunderscore )}
\end{itemize}
Pessôa, que argumenta, sofismando.
\section{Sofistaria}
\begin{itemize}
\item {Grp. gram.:f.}
\end{itemize}
\begin{itemize}
\item {Proveniência:(De \textunderscore sofista\textunderscore )}
\end{itemize}
Discurso sofístico; conjunto de sofismas.
\section{Sofística}
\begin{itemize}
\item {Grp. gram.:f.}
\end{itemize}
\begin{itemize}
\item {Proveniência:(De \textunderscore sofístico\textunderscore )}
\end{itemize}
Parte da Lógica, que ensina a refutar sofismas.
Arte de sofismar.
\section{Sofisticação}
\begin{itemize}
\item {Grp. gram.:f.}
\end{itemize}
Acto ou efeito de sofisticar.
\section{Sofisticador}
\begin{itemize}
\item {Grp. gram.:m.}
\end{itemize}
Aquele que sofistica. Cf. \textunderscore Techn. Rur.\textunderscore , 341.
\section{Sofisticamente}
\begin{itemize}
\item {Grp. gram.:adv.}
\end{itemize}
De modo sofístico.
\section{Sofisticar}
\begin{itemize}
\item {Grp. gram.:v. t.}
\end{itemize}
\begin{itemize}
\item {Grp. gram.:V. i.}
\end{itemize}
\begin{itemize}
\item {Proveniência:(De \textunderscore sofístico\textunderscore )}
\end{itemize}
Sofismar; falsificar.
Tratar com subtileza.
Fazer sofismas.
\section{Sofístico}
\begin{itemize}
\item {Grp. gram.:adj.}
\end{itemize}
\begin{itemize}
\item {Proveniência:(Lat. \textunderscore sophisticus\textunderscore )}
\end{itemize}
Relativo a sofisma; em que há sofisma; que usa sofismas.
\section{Sofomania}
\begin{itemize}
\item {Grp. gram.:f.}
\end{itemize}
\begin{itemize}
\item {Proveniência:(Do gr. \textunderscore sophos\textunderscore  + \textunderscore mania\textunderscore )}
\end{itemize}
Mania de passar por sábio.
\section{Sofomaniaco}
\begin{itemize}
\item {Grp. gram.:m.  e  adj.}
\end{itemize}
O mesmo que sofómano.
\section{Sofómano}
\begin{itemize}
\item {Grp. gram.:m.  e  adj.}
\end{itemize}
O que tem sofomania.
\section{Sofrónia}
\begin{itemize}
\item {Grp. gram.:f.}
\end{itemize}
\begin{itemize}
\item {Proveniência:(Gr. \textunderscore sophronia\textunderscore )}
\end{itemize}
Gênero de insectos lepidópteros nocturnos.
\section{Sofronista}
\begin{itemize}
\item {Grp. gram.:m.}
\end{itemize}
\begin{itemize}
\item {Proveniência:(Do gr. \textunderscore sophronistes\textunderscore )}
\end{itemize}
Aquele que ensinava moral nos ginásios gregos.
\section{Sofronistério}
\begin{itemize}
\item {Grp. gram.:m.}
\end{itemize}
\begin{itemize}
\item {Proveniência:(Gr. \textunderscore sophronisterion\textunderscore )}
\end{itemize}
Casa de correcção, onde os rapazes incorrigíveis eram recolhidos por ordem dos sofronistas.
\section{Sofronita}
\begin{itemize}
\item {Grp. gram.:f.}
\end{itemize}
\begin{itemize}
\item {Proveniência:(Do gr. \textunderscore sophronisis\textunderscore )}
\end{itemize}
Espécie de orquídea.
\section{Sonegar}
\begin{itemize}
\item {Grp. gram.:v. t.}
\end{itemize}
\begin{itemize}
\item {Grp. gram.:V. p.}
\end{itemize}
\begin{itemize}
\item {Proveniência:(Do lat. \textunderscore subnegare\textunderscore )}
\end{itemize}
Occultar, deixando de mencionar ou descrever, nos casos em que a menção ou descripção é exigida por lei.
Occultar fraudulentamente.
Deixar de pagar.
Subtrahir.
Esquivar-se ao cumprimento de uma ordem.
\section{Soneira}
\begin{itemize}
\item {Grp. gram.:f.}
\end{itemize}
O mesmo que \textunderscore sonolência\textunderscore .
\section{Sonetada}
\begin{itemize}
\item {Grp. gram.:f.}
\end{itemize}
\begin{itemize}
\item {Utilização:Prov.}
\end{itemize}
O mesmo que \textunderscore descompostura\textunderscore .
(Cp. \textunderscore soneto\textunderscore )
\section{Sonetar}
\begin{itemize}
\item {Grp. gram.:v. i.}
\end{itemize}
O mesmo que \textunderscore sonetear\textunderscore . Cf. Camillo, \textunderscore Cav. em Ruínas\textunderscore , 48, 49 e 64; \textunderscore Cancion. Al.\textunderscore , 363.
\section{Sonetear}
\begin{itemize}
\item {Grp. gram.:v. i.}
\end{itemize}
Fazer sonetos.
\section{Soneteiro}
\begin{itemize}
\item {Grp. gram.:m.  e  adj.}
\end{itemize}
O que faz sonetos.
\section{Sonetista}
\begin{itemize}
\item {Grp. gram.:m. ,  f.  e  adj.}
\end{itemize}
Pessôa, que faz sonetos.
\section{Soneto}
\begin{itemize}
\item {fónica:nê}
\end{itemize}
\begin{itemize}
\item {Grp. gram.:m.}
\end{itemize}
\begin{itemize}
\item {Utilização:Fam.}
\end{itemize}
\begin{itemize}
\item {Proveniência:(It. \textunderscore sonetto\textunderscore )}
\end{itemize}
Composição poética de quatorze versos, ordinariamente decasýllabos, e dispostos em dois quartetos e dois tercetos.
Remoque, censura; sátira.
\section{Songa}
\begin{itemize}
\item {Grp. gram.:m.  e  f.}
\end{itemize}
\begin{itemize}
\item {Utilização:Prov.}
\end{itemize}
\begin{itemize}
\item {Utilização:minh.}
\end{itemize}
O mesmo que \textunderscore songa-monga\textunderscore .
\section{Songai}
\begin{itemize}
\item {Grp. gram.:m.}
\end{itemize}
Língua do Níger médio.
\section{Songa-monga}
\begin{itemize}
\item {Grp. gram.:m.  e  f.}
\end{itemize}
\begin{itemize}
\item {Utilização:Fam.}
\end{itemize}
Pessôa sonsa.
(Cast. \textunderscore songa\textunderscore )
\section{Songos}
\begin{itemize}
\item {Grp. gram.:m. pl.}
\end{itemize}
Tríbo das margens do Cuanza.
\section{Songue}
\begin{itemize}
\item {Grp. gram.:m.}
\end{itemize}
Espécie de antílope africano.
\section{Songuinha}
\begin{itemize}
\item {Grp. gram.:m.  e  f.}
\end{itemize}
\begin{itemize}
\item {Utilização:Pop.}
\end{itemize}
Pessôa muito songa; o mesmo que \textunderscore songa\textunderscore . Cf. Camillo, \textunderscore Brasileira\textunderscore , 23.
\section{Sonhado}
\begin{itemize}
\item {Grp. gram.:adj.}
\end{itemize}
\begin{itemize}
\item {Utilização:Fig.}
\end{itemize}
\begin{itemize}
\item {Proveniência:(De \textunderscore sonhar\textunderscore )}
\end{itemize}
Fictício; ideal.
\section{Sonhador}
\begin{itemize}
\item {Grp. gram.:m.  e  adj.}
\end{itemize}
O que sonha; devaneador; fantasista.
\section{Sonhar}
\begin{itemize}
\item {Grp. gram.:v. i.}
\end{itemize}
\begin{itemize}
\item {Utilização:Fig.}
\end{itemize}
\begin{itemize}
\item {Utilização:Gír.}
\end{itemize}
\begin{itemize}
\item {Grp. gram.:V. t.}
\end{itemize}
\begin{itemize}
\item {Grp. gram.:M.}
\end{itemize}
\begin{itemize}
\item {Proveniência:(Do lat. \textunderscore somniare\textunderscore )}
\end{itemize}
Têr sonho ou sonhos.
Devanear; fantasiar.
Têr ideia fixa.
\textunderscore Sonhar com o pai\textunderscore , embriagar-se.
Vêr em sonhos.
Sonho.
\section{Sonho}
\begin{itemize}
\item {Grp. gram.:m.}
\end{itemize}
\begin{itemize}
\item {Proveniência:(Lat. \textunderscore somnium\textunderscore )}
\end{itemize}
Operação inconsciente das faculdades intellectuaes, imperfeitamente despertadas em quem dorme.
Fantasia.
Utopia.
Ficção.
Coisa fútil ou transitória.
Visão.
Aspiração: \textunderscore o meu sonho é a ventura dos meus filhos\textunderscore .
Vivo desejo.
Bolo de farinha e ovos, frito em azeite e manteiga, e passado depois por calda de açúcar.
\section{Sonial}
\begin{itemize}
\item {Grp. gram.:adj.}
\end{itemize}
\begin{itemize}
\item {Proveniência:(Lat. \textunderscore somnialis\textunderscore )}
\end{itemize}
Relativo aos sonhos.
\section{Sónica}
\begin{itemize}
\item {Grp. gram.:f.}
\end{itemize}
\begin{itemize}
\item {Utilização:Deprec.}
\end{itemize}
\begin{itemize}
\item {Proveniência:(De \textunderscore sónico\textunderscore )}
\end{itemize}
Systema de graphar as palavras conforme o som dellas, independentemente da etymologia e da história da língua.
\section{Sonicéfalo}
\begin{itemize}
\item {Grp. gram.:m.}
\end{itemize}
\begin{itemize}
\item {Proveniência:(Do lat. \textunderscore sonus\textunderscore  + gr. \textunderscore kephale\textunderscore )}
\end{itemize}
Nome de vários insectos zunidores.
\section{Sonicéphalo}
\begin{itemize}
\item {Grp. gram.:m.}
\end{itemize}
\begin{itemize}
\item {Proveniência:(Do lat. \textunderscore sonus\textunderscore  + gr. \textunderscore kephale\textunderscore )}
\end{itemize}
Nome de vários insectos zunidores.
\section{Sónico}
\begin{itemize}
\item {Grp. gram.:adj.}
\end{itemize}
\begin{itemize}
\item {Grp. gram.:M.}
\end{itemize}
\begin{itemize}
\item {Proveniência:(Do lat. \textunderscore sonus\textunderscore )}
\end{itemize}
Relativo ao som.
Conforme ao som; phonético.
Aquella que pratica a sónica, ou é partidário della: \textunderscore os sónicos andam ás aranhas\textunderscore .
\section{Soniculoso}
\begin{itemize}
\item {Grp. gram.:adj.}
\end{itemize}
\begin{itemize}
\item {Proveniência:(Lat. \textunderscore somniculosus\textunderscore )}
\end{itemize}
Sonolento, dorminhoco.
\section{Sonido}
\begin{itemize}
\item {Grp. gram.:m.}
\end{itemize}
\begin{itemize}
\item {Proveniência:(Lat. \textunderscore sonitus\textunderscore )}
\end{itemize}
Qualquer som.
Rumor; estrondo.
\section{Sonífero}
\begin{itemize}
\item {Grp. gram.:adj.}
\end{itemize}
\begin{itemize}
\item {Utilização:Poét.}
\end{itemize}
\begin{itemize}
\item {Grp. gram.:M.}
\end{itemize}
\begin{itemize}
\item {Proveniência:(Lat. \textunderscore somnifer\textunderscore )}
\end{itemize}
Que faz sono.
Substância soporífera.
\section{Sonígrafo}
\begin{itemize}
\item {Grp. gram.:m.}
\end{itemize}
\begin{itemize}
\item {Utilização:Neol.}
\end{itemize}
\begin{itemize}
\item {Proveniência:(Do lat. \textunderscore somnum\textunderscore  + gr. \textunderscore graphein\textunderscore )}
\end{itemize}
Aquele que descreve sonhos. Cf. Garrett, \textunderscore Viagens\textunderscore , I, 30.
\section{Sonílogo}
\begin{itemize}
\item {Grp. gram.:m.}
\end{itemize}
\begin{itemize}
\item {Utilização:Neol.}
\end{itemize}
\begin{itemize}
\item {Proveniência:(Do lat. \textunderscore somnum\textunderscore  + gr. \textunderscore logos\textunderscore )}
\end{itemize}
Aquele que sabe interpretar sonhos. Cf. Garrett, \textunderscore Viagens\textunderscore , I, 30.
\section{Soníloquo}
\begin{itemize}
\item {Grp. gram.:m.  e  adj.}
\end{itemize}
\begin{itemize}
\item {Proveniência:(Do lat. \textunderscore somnus\textunderscore  + \textunderscore loqui\textunderscore )}
\end{itemize}
O que fala a dormir.
\section{Sonínia}
\begin{itemize}
\item {Grp. gram.:f.}
\end{itemize}
Gênero de plantas asclepiadáceas.
\section{Sonípede}
\begin{itemize}
\item {Grp. gram.:m.  e  adj.}
\end{itemize}
\begin{itemize}
\item {Utilização:Poét.}
\end{itemize}
\begin{itemize}
\item {Proveniência:(Lat. \textunderscore sonipes\textunderscore )}
\end{itemize}
O que faz ruído com os pés.
\section{Sonívio}
\begin{itemize}
\item {Grp. gram.:adv.}
\end{itemize}
\begin{itemize}
\item {Utilização:Poét.}
\end{itemize}
\begin{itemize}
\item {Proveniência:(Lat. \textunderscore sonivius\textunderscore )}
\end{itemize}
Que produz som, caindo no caminho.
\section{Sonnínia}
\begin{itemize}
\item {Grp. gram.:f.}
\end{itemize}
Gênero de plantas asclepiadáceas.
\section{Sono}
\begin{itemize}
\item {Grp. gram.:m.}
\end{itemize}
\begin{itemize}
\item {Utilização:Fig.}
\end{itemize}
\begin{itemize}
\item {Proveniência:(Lat. \textunderscore somnus\textunderscore )}
\end{itemize}
\textunderscore m.\textunderscore  (e der.)
O mesmo ou melhor que \textunderscore somno\textunderscore , etc.
Completo adormecimento dos sentidos.
Cessação momentânea da actividade, própria dos sistemas que têm as propriedades da vida animal.
Desejo de dormir ou sentimento da necessidade de dormir: \textunderscore tenho sono\textunderscore .
Estado de quem dorme: \textunderscore caiu no sono\textunderscore .
Inércia: preguiça; indolência.
\textunderscore Doença do sono\textunderscore , doença endêmica, peculiar á raça negra, produzida pela picada de um insecto e caracterizada por um sopor profundo, a que a morte quasi sempre põe termo; tripanosomose.
\section{Sonoite}
\begin{itemize}
\item {Grp. gram.:f.}
\end{itemize}
\begin{itemize}
\item {Utilização:Des.}
\end{itemize}
O anoitecer, o lusco-fusco. Cf. Sá de Miranda, 932, (ed. \textunderscore Michaëlis\textunderscore )
\section{Sonolência}
\begin{itemize}
\item {Grp. gram.:f.}
\end{itemize}
\begin{itemize}
\item {Utilização:Fig.}
\end{itemize}
\begin{itemize}
\item {Proveniência:(Lat. \textunderscore somnolentia\textunderscore )}
\end{itemize}
Sono imperfeito.
Momento de transição, entre o estado de quem dorme e o acordar.
Disposição para dormir; modorra.
Inércia, entorpecimento.
\section{Sonolento}
\begin{itemize}
\item {Grp. gram.:adj.}
\end{itemize}
\begin{itemize}
\item {Utilização:Fig.}
\end{itemize}
\begin{itemize}
\item {Proveniência:(Lat. \textunderscore somnolentus\textunderscore )}
\end{itemize}
Que tem sonolência.
Que causa sono.
Relativo á sonolência.
Vagaroso; inerte.
\section{Sonometria}
\begin{itemize}
\item {Grp. gram.:f.}
\end{itemize}
Arte de applicar o sonómetro.
\section{Sonométrico}
\begin{itemize}
\item {Grp. gram.:adj.}
\end{itemize}
Relativo á sonometria.
\section{Sonómetro}
\begin{itemize}
\item {Grp. gram.:m.}
\end{itemize}
\begin{itemize}
\item {Proveniência:(Do lat. \textunderscore sonus\textunderscore  + gr. \textunderscore metron\textunderscore )}
\end{itemize}
Instrumento, para medir as vibrações sonoras.
Harmonómetro.
\section{Sonoramente}
\begin{itemize}
\item {Grp. gram.:adv.}
\end{itemize}
De modo sonoro.
\section{Sonoridade}
\begin{itemize}
\item {Grp. gram.:f.}
\end{itemize}
\begin{itemize}
\item {Proveniência:(Do lat. \textunderscore sonoritas\textunderscore )}
\end{itemize}
Qualidade do que é sonoro.
Propriedade de produzir ou de forçar sons.
\section{Sonorizar}
\begin{itemize}
\item {Grp. gram.:v. t.}
\end{itemize}
\begin{itemize}
\item {Utilização:Neol.}
\end{itemize}
Tornar sonoro. Cf. C. Neto, \textunderscore Baladilhas\textunderscore , 140.
\section{Sonoro}
\begin{itemize}
\item {Grp. gram.:adj.}
\end{itemize}
\begin{itemize}
\item {Proveniência:(Lat. \textunderscore sonorus\textunderscore )}
\end{itemize}
Que produz som.
Que emitte som intenso.
Que reforça o som.
Que sôa agradavelmente.
Harmonioso; suave.
\section{Sonoroso}
\begin{itemize}
\item {Grp. gram.:adj.}
\end{itemize}
O mesmo que \textunderscore sonoro\textunderscore .
Que tem som alto e agradavel; melodioso.
\section{Sonsa}
\begin{itemize}
\item {Grp. gram.:f.}
\end{itemize}
Qualidade do que é sonso.
\section{Sonsice}
\begin{itemize}
\item {Grp. gram.:f.}
\end{itemize}
Qualidade do que é sonso.
\section{Sonsinho}
\begin{itemize}
\item {Grp. gram.:adj.}
\end{itemize}
\begin{itemize}
\item {Proveniência:(De \textunderscore sonso\textunderscore )}
\end{itemize}
Velhaco; solerte; que tem muita manha.
\section{Sonso}
\begin{itemize}
\item {Grp. gram.:adj.}
\end{itemize}
\begin{itemize}
\item {Proveniência:(De \textunderscore insonso\textunderscore , por \textunderscore insosso\textunderscore ?)}
\end{itemize}
Solerte; manhoso; dissimulado velhaco.
\section{Sonsonete}
\begin{itemize}
\item {fónica:nê}
\end{itemize}
\begin{itemize}
\item {Grp. gram.:m.}
\end{itemize}
\begin{itemize}
\item {Proveniência:(De \textunderscore sonso\textunderscore )}
\end{itemize}
Inflexão especial, com que se profere uma ironia.
\section{Sontal}
\begin{itemize}
\item {Grp. gram.:m.}
\end{itemize}
Língua decânica da província de Bengala.
\section{Sonthal}
\begin{itemize}
\item {Grp. gram.:m.}
\end{itemize}
Língua decânica da província de Bengala.
\section{Sonto}
\begin{itemize}
\item {Grp. gram.:m.}
\end{itemize}
Variedade de chá.
\section{Sonurno}
\begin{itemize}
\item {Grp. gram.:adj.}
\end{itemize}
\begin{itemize}
\item {Proveniência:(Lat. \textunderscore somnurnus\textunderscore )}
\end{itemize}
Relativo ao sono.
Que se vê em sonhos; fantástico.
\section{Sopa}
\begin{itemize}
\item {fónica:sô}
\end{itemize}
\begin{itemize}
\item {Grp. gram.:f.}
\end{itemize}
\begin{itemize}
\item {Grp. gram.:Pl.}
\end{itemize}
\begin{itemize}
\item {Utilização:Pop.}
\end{itemize}
\begin{itemize}
\item {Proveniência:(Al. \textunderscore suppe\textunderscore )}
\end{itemize}
Caldo, com alguma substância sólida, e que constitue ordinariamente o primeiro prato de um jantar.
Pedaço de pão, embebido num liquido: \textunderscore deita uma sopa ao cão\textunderscore .
Coisa muito molhada: \textunderscore êsse casaco está uma sopa\textunderscore .
Alimentação, sustento: \textunderscore viver ás sopas de algúem\textunderscore , viver a custa delle, por elle alimentado.
\section{Sopada}
\begin{itemize}
\item {Grp. gram.:f.}
\end{itemize}
\begin{itemize}
\item {Utilização:Pop.}
\end{itemize}
Abundância de sopas.
\section{Sopaina}
\begin{itemize}
\item {Grp. gram.:m.}
\end{itemize}
\begin{itemize}
\item {Utilização:Prov.}
\end{itemize}
\begin{itemize}
\item {Utilização:trasm.}
\end{itemize}
Indivíduo cambaio, torto das pernas.
\section{Sopangas}
\begin{itemize}
\item {Grp. gram.:m. pl.}
\end{itemize}
Antigo povo da África Oriental. Cf. Barros, \textunderscore Déc.\textunderscore  IV, l. III, c. 5.
\section{Sopão}
\begin{itemize}
\item {Grp. gram.:m.  e  adj.}
\end{itemize}
\begin{itemize}
\item {Utilização:Chul.}
\end{itemize}
\begin{itemize}
\item {Proveniência:(De \textunderscore sopa\textunderscore )}
\end{itemize}
Beberrão.
\section{Sopapo}
\begin{itemize}
\item {Grp. gram.:m.}
\end{itemize}
\begin{itemize}
\item {Grp. gram.:Loc.}
\end{itemize}
\begin{itemize}
\item {Utilização:fam.}
\end{itemize}
\begin{itemize}
\item {Proveniência:(De \textunderscore so...\textunderscore  + \textunderscore papo\textunderscore )}
\end{itemize}
Murro, dado debaixo do queixo; bofetada.
\textunderscore Dar uma sopapo na gaveta\textunderscore , surripiar dinheiro.
\section{Sopé}
\begin{itemize}
\item {Grp. gram.:m.}
\end{itemize}
\begin{itemize}
\item {Proveniência:(De \textunderscore so...\textunderscore  + \textunderscore pé\textunderscore )}
\end{itemize}
Base; falda.
Parte inferior de uma encosta, de um muro, etc.
\section{Sopeador}
\begin{itemize}
\item {Grp. gram.:m.  e  adj.}
\end{itemize}
O que sopeia.
\section{Sopeamento}
\begin{itemize}
\item {Grp. gram.:m.}
\end{itemize}
Acto ou effeito de sopear.
\section{Sopear}
\begin{itemize}
\item {Grp. gram.:v. t.}
\end{itemize}
\begin{itemize}
\item {Proveniência:(De \textunderscore so...\textunderscore  + \textunderscore pear\textunderscore )}
\end{itemize}
Pôr debaixo dos pés; calcar.
Refrear; reprimir.
Sujeitar; humilhar.
\section{Sopeira}
\begin{itemize}
\item {Grp. gram.:f.}
\end{itemize}
\begin{itemize}
\item {Utilização:Fam.}
\end{itemize}
\begin{itemize}
\item {Utilização:Fam.}
\end{itemize}
Vaso para sôpa.
Criada para serviço de cozinha; cozinheira. Cf. Camillo, \textunderscore Corja\textunderscore , 191.
Titulo de crédito, o mesmo que \textunderscore sopeirinha\textunderscore .
\section{Sopeirame}
\begin{itemize}
\item {Grp. gram.:m.}
\end{itemize}
\begin{itemize}
\item {Utilização:Chul. de Lisbôa.}
\end{itemize}
A classe das sopeiras \textunderscore ou \textunderscore cozinheiras.
\section{Sopeirinha}
\begin{itemize}
\item {Grp. gram.:f.}
\end{itemize}
Designação familiar de um título de crédito, que, pelo seu limitado preço e pelos prêmios que destribue, é adquirido por sopeiras ou criadas e pessôas de limitados haveres.
\section{Sopeiro}
\begin{itemize}
\item {Grp. gram.:adj.}
\end{itemize}
\begin{itemize}
\item {Grp. gram.:M.  e  adj.}
\end{itemize}
\begin{itemize}
\item {Proveniência:(De \textunderscore sopa\textunderscore )}
\end{itemize}
Relativo a sopa.
Que serve para conter sopa: \textunderscore prato sopeiro\textunderscore .
O que gosta de sopas.
O que é alimentado á custa de outro.
\section{Sopelão}
\begin{itemize}
\item {Grp. gram.:m.}
\end{itemize}
\begin{itemize}
\item {Utilização:Prov.}
\end{itemize}
\begin{itemize}
\item {Utilização:trasm.}
\end{itemize}
\begin{itemize}
\item {Proveniência:(De lat. \textunderscore sub\textunderscore  + \textunderscore pellere\textunderscore . Cf. \textunderscore repelão\textunderscore )}
\end{itemize}
Impulso, de baixo para cima; solavanco.
\section{Sopellão}
\begin{itemize}
\item {Grp. gram.:m.}
\end{itemize}
\begin{itemize}
\item {Utilização:Prov.}
\end{itemize}
\begin{itemize}
\item {Utilização:trasm.}
\end{itemize}
\begin{itemize}
\item {Proveniência:(De lat. \textunderscore sub\textunderscore  + \textunderscore pellere\textunderscore . Cf. \textunderscore repellão\textunderscore )}
\end{itemize}
Impulso, de baixo para cima; solavanco.
\section{Sopesar}
\begin{itemize}
\item {Grp. gram.:v. t.}
\end{itemize}
\begin{itemize}
\item {Proveniência:(De \textunderscore so...\textunderscore  + \textunderscore pesar\textunderscore )}
\end{itemize}
Tomar com a mão o pêso a.
Suspender com a mão.
Contrapesar.
Supportar o pêso.
Distribuir methodicamente, parcimoniosamente.
\section{Sopêso}
\begin{itemize}
\item {Grp. gram.:m.}
\end{itemize}
Acto ou effeito de sopesar.
\section{Sopetarra}
\begin{itemize}
\item {Grp. gram.:f.}
\end{itemize}
\begin{itemize}
\item {Utilização:Fam.}
\end{itemize}
Sopa grande.
\section{Sopetear}
\begin{itemize}
\item {Grp. gram.:v. t.}
\end{itemize}
\begin{itemize}
\item {Grp. gram.:V. i.}
\end{itemize}
\begin{itemize}
\item {Proveniência:(De \textunderscore sopa\textunderscore )}
\end{itemize}
Saborear; gozar.
Molhar muitas vezes o pão num líquido.
\section{Sopheno}
\begin{itemize}
\item {Grp. gram.:m.}
\end{itemize}
Variedade de figos doces e branquícentos.
\section{Sophia}
\begin{itemize}
\item {Grp. gram.:f.}
\end{itemize}
\begin{itemize}
\item {Proveniência:(Gr. \textunderscore sophia\textunderscore )}
\end{itemize}
O mesmo que \textunderscore sciência\textunderscore . Cf. Camões, ode III.
\section{Sóphia-dos-cirurgiões}
\begin{itemize}
\item {Grp. gram.:f.}
\end{itemize}
Planta annual ou bisannual, (\textunderscore sisymbrium sophia\textunderscore , Lin.).
\section{Sophisma}
\begin{itemize}
\item {Grp. gram.:m.}
\end{itemize}
\begin{itemize}
\item {Proveniência:(Lat. \textunderscore sophisma\textunderscore )}
\end{itemize}
Argumento falso, ou falso raciocínio com alguma apparência de verdade.
\section{Sophismar}
\begin{itemize}
\item {Grp. gram.:v. t.}
\end{itemize}
\begin{itemize}
\item {Utilização:Fig.}
\end{itemize}
\begin{itemize}
\item {Grp. gram.:V. i.}
\end{itemize}
Encobrir com sophisma; discutir, sophismando.
Enganar.
Empregar sophisma.
\section{Sophismável}
\begin{itemize}
\item {Grp. gram.:adj.}
\end{itemize}
Que se póde sophismar.
\section{Sophista}
\begin{itemize}
\item {Grp. gram.:m. ,  f.  e  adj.}
\end{itemize}
\begin{itemize}
\item {Proveniência:(Lat. \textunderscore sophista\textunderscore )}
\end{itemize}
Pessôa, que argumenta, sophismando.
\section{Sophistaria}
\begin{itemize}
\item {Grp. gram.:f.}
\end{itemize}
\begin{itemize}
\item {Proveniência:(De \textunderscore sophista\textunderscore )}
\end{itemize}
Discurso sophístico; conjunto de sophismas.
\section{Sophística}
\begin{itemize}
\item {Grp. gram.:f.}
\end{itemize}
\begin{itemize}
\item {Proveniência:(De \textunderscore sophístico\textunderscore )}
\end{itemize}
Parte da Lógica, que ensina a refutar sophismas.
Arte de sophismar.
\section{Sophisticação}
\begin{itemize}
\item {Grp. gram.:f.}
\end{itemize}
Acto ou effeito de sophisticar.
\section{Sophisticador}
\begin{itemize}
\item {Grp. gram.:m.}
\end{itemize}
Aquelle que sophistica. Cf. \textunderscore Techn. Rur.\textunderscore , 341.
\section{Sophisticamente}
\begin{itemize}
\item {Grp. gram.:adv.}
\end{itemize}
De modo sophístico.
\section{Sophisticar}
\begin{itemize}
\item {Grp. gram.:v. t.}
\end{itemize}
\begin{itemize}
\item {Grp. gram.:V. i.}
\end{itemize}
\begin{itemize}
\item {Proveniência:(De \textunderscore sophístico\textunderscore )}
\end{itemize}
Sophismar; falsificar.
Tratar com subtileza.
Fazer sophismas.
\section{Sophístico}
\begin{itemize}
\item {Grp. gram.:adj.}
\end{itemize}
\begin{itemize}
\item {Proveniência:(Lat. \textunderscore sophisticus\textunderscore )}
\end{itemize}
Relativo a sophisma; em que há sophisma; que usa sophismas.
\section{Sophomania}
\begin{itemize}
\item {Grp. gram.:f.}
\end{itemize}
\begin{itemize}
\item {Proveniência:(Do gr. \textunderscore sophos\textunderscore  + \textunderscore mania\textunderscore )}
\end{itemize}
Mania de passar por sábio.
\section{Sophomaníaco}
\begin{itemize}
\item {Grp. gram.:m.  e  adj.}
\end{itemize}
O mesmo que sophómano.
\section{Sophómano}
\begin{itemize}
\item {Grp. gram.:m.  e  adj.}
\end{itemize}
O que tem sophomania.
\section{Sophrónia}
\begin{itemize}
\item {Grp. gram.:f.}
\end{itemize}
\begin{itemize}
\item {Proveniência:(Gr. \textunderscore sophronia\textunderscore )}
\end{itemize}
Gênero de insectos lepidópteros nocturnos.
\section{Sophronista}
\begin{itemize}
\item {Grp. gram.:m.}
\end{itemize}
\begin{itemize}
\item {Proveniência:(Do gr. \textunderscore sophronistes\textunderscore )}
\end{itemize}
Aquelle que ensinava moral nos gymnásios gregos.
\section{Sophronistrério}
\begin{itemize}
\item {Grp. gram.:m.}
\end{itemize}
\begin{itemize}
\item {Proveniência:(Gr. \textunderscore sophronisterion\textunderscore )}
\end{itemize}
Casa de correcção, onde os rapazes incorrigíveis eram recolhidos por ordem dos sophronistas.
\section{Sophronita}
\begin{itemize}
\item {Grp. gram.:f.}
\end{itemize}
\begin{itemize}
\item {Proveniência:(Do gr. \textunderscore sophronisis\textunderscore )}
\end{itemize}
Espécie de orchídea.
\section{Sopiado}
\begin{itemize}
\item {Grp. gram.:adj.}
\end{itemize}
\begin{itemize}
\item {Utilização:Prov.}
\end{itemize}
\begin{itemize}
\item {Utilização:trasm.}
\end{itemize}
\begin{itemize}
\item {Utilização:minh.}
\end{itemize}
\begin{itemize}
\item {Proveniência:(De \textunderscore so...\textunderscore  + \textunderscore pia\textunderscore ?)}
\end{itemize}
Diz-se da criança, que foi baptizada em casa.
\section{Sopilho}
\begin{itemize}
\item {Grp. gram.:m.}
\end{itemize}
\begin{itemize}
\item {Utilização:Prov.}
\end{itemize}
\begin{itemize}
\item {Utilização:trasm.}
\end{itemize}
Utensílio quadrilongo de madeira, com orifícios, nos quaes gira o fuso da encanhadeira.
\section{Sopista}
\begin{itemize}
\item {Grp. gram.:m. ,  f.  e  adj.}
\end{itemize}
Pessôa, que gosta de sopas.
\section{Sopitado}
\begin{itemize}
\item {Grp. gram.:adj.}
\end{itemize}
\begin{itemize}
\item {Proveniência:(De \textunderscore sopitar\textunderscore )}
\end{itemize}
Que caiu em somnolência; adormecido.
Effeminado.
\section{Sopitamento}
\begin{itemize}
\item {Grp. gram.:m.}
\end{itemize}
Acto ou effeito de sopitar. Cf. Camillo, \textunderscore Suicida\textunderscore , 119.
\section{Sopitar}
\begin{itemize}
\item {Grp. gram.:v. t.}
\end{itemize}
\begin{itemize}
\item {Proveniência:(De \textunderscore sopito\textunderscore )}
\end{itemize}
Adormentar; acalmar.
Debilitar; alquebrar.
Effeminar.
Fazer nascer esperanças em.
\section{Sopitável}
\begin{itemize}
\item {Grp. gram.:adj.}
\end{itemize}
Que se póde sopitar.
\section{Sopito}
\begin{itemize}
\item {Grp. gram.:adj.}
\end{itemize}
\begin{itemize}
\item {Proveniência:(Lat. \textunderscore sopitus\textunderscore )}
\end{itemize}
O mesmo que \textunderscore sopitado\textunderscore .
\section{Soplo}
\begin{itemize}
\item {fónica:sô}
\end{itemize}
\begin{itemize}
\item {Grp. gram.:m.}
\end{itemize}
\begin{itemize}
\item {Utilização:Ant.}
\end{itemize}
O mesmo que \textunderscore sôpro\textunderscore . Cf. Usque, 10.
\section{Sópo}
\begin{itemize}
\item {Grp. gram.:adj.}
\end{itemize}
\begin{itemize}
\item {Utilização:Prov.}
\end{itemize}
\begin{itemize}
\item {Utilização:trasm.}
\end{itemize}
Diz-se do jumento, cavallo, etc., que tem algum casco recurvado, assentando a parte anterior em vez da planta.
\section{Sopontadura}
\begin{itemize}
\item {Grp. gram.:f.}
\end{itemize}
Acto ou effeito de sopontar.
\section{Sopontar}
\begin{itemize}
\item {Grp. gram.:v. t.}
\end{itemize}
\begin{itemize}
\item {Proveniência:(De \textunderscore so...\textunderscore  + \textunderscore ponto\textunderscore )}
\end{itemize}
Marcar com pontinhos por baixo (palavras), para indicar que ellas são demais.
\section{Sopor}
\begin{itemize}
\item {Grp. gram.:m.}
\end{itemize}
\begin{itemize}
\item {Proveniência:(Lat. \textunderscore sopor\textunderscore )}
\end{itemize}
Somno profundo; estado comatoso.
\section{Soporado}
\begin{itemize}
\item {Grp. gram.:adj.}
\end{itemize}
Que tem sopor.
Que produz sopor. Cf. \textunderscore Ulysseia\textunderscore , IV, 34.
\section{Soporal}
\begin{itemize}
\item {Grp. gram.:adj.}
\end{itemize}
Relativo a sopor. Cf. R. Jorge, na \textunderscore Gaz. Méd.\textunderscore 
\section{Soporativo}
\begin{itemize}
\item {Grp. gram.:adj.}
\end{itemize}
\begin{itemize}
\item {Utilização:Fig.}
\end{itemize}
\begin{itemize}
\item {Grp. gram.:M.}
\end{itemize}
\begin{itemize}
\item {Utilização:Fig.}
\end{itemize}
\begin{itemize}
\item {Proveniência:(Do lat. \textunderscore soporatus\textunderscore )}
\end{itemize}
Que produz sopor.
Enfadonho; fastidioso.
Substância, que faz dormir.
Coisa fastidiosa.
\section{Soporífero}
\begin{itemize}
\item {Grp. gram.:adj.}
\end{itemize}
\begin{itemize}
\item {Proveniência:(Lat. \textunderscore soporifer\textunderscore )}
\end{itemize}
O mesmo que \textunderscore soporífico\textunderscore .
\section{Soporífico}
\begin{itemize}
\item {Grp. gram.:adj.}
\end{itemize}
\begin{itemize}
\item {Utilização:Fig.}
\end{itemize}
\begin{itemize}
\item {Proveniência:(Do lat. \textunderscore sopor\textunderscore  + \textunderscore facere\textunderscore )}
\end{itemize}
Que produz somno ou sopor; soporativo.
Maçador, fastiento.
\section{Soporizar}
\begin{itemize}
\item {Grp. gram.:v. t.}
\end{itemize}
\begin{itemize}
\item {Proveniência:(De \textunderscore sopor\textunderscore )}
\end{itemize}
O mesmo que \textunderscore sopitar\textunderscore .
\section{Soporoso}
\begin{itemize}
\item {Grp. gram.:adj.}
\end{itemize}
Relativo a sopor.
Somnolento.
\section{Soportal}
\begin{itemize}
\item {Grp. gram.:m.}
\end{itemize}
\begin{itemize}
\item {Proveniência:(De \textunderscore so...\textunderscore  + \textunderscore portal\textunderscore )}
\end{itemize}
A parte inferior do portal; soleira.
Átrio.
\section{Soportar}
\textunderscore v. t.\textunderscore  (e der.)
(Fórma antiga de \textunderscore supportar\textunderscore , etc.)
\section{Soprano}
\begin{itemize}
\item {Grp. gram.:m.}
\end{itemize}
\begin{itemize}
\item {Proveniência:(It. \textunderscore soprano\textunderscore )}
\end{itemize}
A mais elevada das vozes.
Tiple.
Cantor ou cantora, que tem voz de soprano.
\section{Soprão}
\begin{itemize}
\item {Grp. gram.:m.}
\end{itemize}
\begin{itemize}
\item {Utilização:Prov.}
\end{itemize}
\begin{itemize}
\item {Utilização:trasm.}
\end{itemize}
Papa-jantares; explorador do próximo.
(Cp. \textunderscore sopão\textunderscore )
\section{Soprar}
\begin{itemize}
\item {Grp. gram.:v. t.}
\end{itemize}
\begin{itemize}
\item {Grp. gram.:V. i.}
\end{itemize}
\begin{itemize}
\item {Proveniência:(Do b. lat. \textunderscore suplare\textunderscore )}
\end{itemize}
Dirigir o sôpro para ou sôbre.
Agitar com o sôpro.
Apagar com o sôpro.
Encher de ar por meio de sôpro.
Bafejar.
Segredar: \textunderscore soprou-lhe ao ouvido uma revelação\textunderscore .
Favonear, favorecer.
Suggerir.
Insinuar.
Estimular ás occultas.
Retirar ou separar (peças), no jôgo das damas ou do xadrez.
Emittir sôpro.
Agitar-se, produzir-se, (falando-se do vento).
\section{Sopremo}
\begin{itemize}
\item {Grp. gram.:m.}
\end{itemize}
\begin{itemize}
\item {Utilização:Prov.}
\end{itemize}
Us. na loc. \textunderscore pôr sopremo\textunderscore , cohibir ou refrear.
\section{Sopresa}
\begin{itemize}
\item {Grp. gram.:f.}
\end{itemize}
Acto de sopresar. Cf. \textunderscore Viriato Trág.\textunderscore , IV, 35.
\section{Sopresar}
\begin{itemize}
\item {Grp. gram.:v. t.}
\end{itemize}
\begin{itemize}
\item {Utilização:Fig.}
\end{itemize}
\begin{itemize}
\item {Proveniência:(De \textunderscore so...\textunderscore  + \textunderscore presa\textunderscore )}
\end{itemize}
Apresar; apanhar de assalto.
Embair com falsas apparências.
\section{Soprilho}
\begin{itemize}
\item {Grp. gram.:m.}
\end{itemize}
\begin{itemize}
\item {Proveniência:(De \textunderscore sôpro\textunderscore ?)}
\end{itemize}
Variedade de sêda muito transparente.
\section{Sôpro}
\begin{itemize}
\item {Grp. gram.:m.}
\end{itemize}
\begin{itemize}
\item {Utilização:Fig.}
\end{itemize}
\begin{itemize}
\item {Proveniência:(De \textunderscore soprar\textunderscore )}
\end{itemize}
Vento, que se produz impellindo o ar com o auxílio da bôca.
Acto de expellir com alguma fôrça o ar que se decompôs e se transformou nos pulmões.
Bafejo.
Hálito.
Aragem.
Agitação do ar.
Exhalação.
Influência; insinuação; instigação.
Som.
\section{Sopúbia}
\begin{itemize}
\item {Grp. gram.:f.}
\end{itemize}
Gênero de plantas escrofularíneas.
\section{Soque}
\begin{itemize}
\item {Grp. gram.:m.}
\end{itemize}
\begin{itemize}
\item {Utilização:Bras}
\end{itemize}
Acto de socar.
\section{Soquear}
\begin{itemize}
\item {Grp. gram.:v. t.}
\end{itemize}
Dar sôcos em; socar. Cf. Castilho, \textunderscore Avarento\textunderscore , III, 5.
\section{Soqueira}
\begin{itemize}
\item {Grp. gram.:f.}
\end{itemize}
\begin{itemize}
\item {Proveniência:(De \textunderscore soca\textunderscore ^2)}
\end{itemize}
Conjunto das raízes das canas, depois de cortadas estas.
\section{Soqueiro}
\begin{itemize}
\item {Grp. gram.:m.}
\end{itemize}
\begin{itemize}
\item {Utilização:Prov.}
\end{itemize}
\begin{itemize}
\item {Proveniência:(De \textunderscore sóco\textunderscore )}
\end{itemize}
O mesmo que \textunderscore tamanqueiro\textunderscore . Júl. Dinís, \textunderscore Fidalgos\textunderscore , I, 144.
\section{Soqueixar}
\begin{itemize}
\item {Grp. gram.:v. t.}
\end{itemize}
\begin{itemize}
\item {Proveniência:(De \textunderscore soqueixo\textunderscore )}
\end{itemize}
Ligar por baixo do queixo:«\textunderscore ...leva beatilha soqueixada.\textunderscore »R. Lobo, \textunderscore Églogas\textunderscore , 110.
\section{Soqueixo}
\begin{itemize}
\item {Grp. gram.:m.}
\end{itemize}
\begin{itemize}
\item {Proveniência:(De \textunderscore so...\textunderscore  + \textunderscore queixo\textunderscore )}
\end{itemize}
Ligadura, por baixo do queixo.
Lenço ou pano, que se ata por baixo do queixo.
\section{Soquete}
\begin{itemize}
\item {fónica:quê}
\end{itemize}
\begin{itemize}
\item {Grp. gram.:m.}
\end{itemize}
\begin{itemize}
\item {Utilização:Bras}
\end{itemize}
\begin{itemize}
\item {Proveniência:(De \textunderscore sôco\textunderscore )}
\end{itemize}
Utensílio, com que se calca a pólvora e a bala, dentro do canhão.
Sôco, dado com pouca fôrça.
Espécie de sopa.
\section{Soquetear}
\begin{itemize}
\item {Grp. gram.:v. t.}
\end{itemize}
Calcar com o soquete; socar.
\section{Soquir}
\begin{itemize}
\item {Grp. gram.:v. i.}
\end{itemize}
\begin{itemize}
\item {Utilização:Ant.}
\end{itemize}
\begin{itemize}
\item {Utilização:Gír.}
\end{itemize}
Comer.
\section{Sôr}
\begin{itemize}
\item {Grp. gram.:m.}
\end{itemize}
\begin{itemize}
\item {Utilização:Pleb.}
\end{itemize}
(Contr. de \textunderscore senhor\textunderscore )
\section{Sór}
\begin{itemize}
\item {Grp. gram.:f.}
\end{itemize}
(Contr. de \textunderscore sóror\textunderscore )
\section{Sôra}
\begin{itemize}
\item {Grp. gram.:f.}
\end{itemize}
\begin{itemize}
\item {Utilização:Pleb.}
\end{itemize}
(Contr. de \textunderscore senhora\textunderscore ). Cf. Arn. Gama, \textunderscore Motim\textunderscore , 371.
\section{Sóra}
\begin{itemize}
\item {Grp. gram.:f.}
\end{itemize}
Bebida, usada pelos Peruanos, e composta de maís, que se deita de môlho até germinar, sendo depois moído e cozido em água, na qual se deixa de infusão.
\section{Sóraco}
\begin{itemize}
\item {Grp. gram.:m.}
\end{itemize}
\begin{itemize}
\item {Proveniência:(Lat. \textunderscore soracum\textunderscore )}
\end{itemize}
Cofre ou arca, em que os comediantes levavam os seus vestuários e adereços.
\section{Sorar}
\begin{itemize}
\item {Grp. gram.:v. t.}
\end{itemize}
Transformar em sôro.
\section{Sorbicadela}
\begin{itemize}
\item {Grp. gram.:f.}
\end{itemize}
\begin{itemize}
\item {Utilização:Prov.}
\end{itemize}
\begin{itemize}
\item {Utilização:trasm.}
\end{itemize}
Acto de sorbicar.
\section{Sorbicar}
\begin{itemize}
\item {Grp. gram.:v. t.}
\end{itemize}
\begin{itemize}
\item {Utilização:Prov.}
\end{itemize}
\begin{itemize}
\item {Utilização:trasm.}
\end{itemize}
O mesmo que \textunderscore beliscar\textunderscore .
\section{Sorbina}
\begin{itemize}
\item {Grp. gram.:f.}
\end{itemize}
\begin{itemize}
\item {Proveniência:(Do lat. \textunderscore sorbum\textunderscore )}
\end{itemize}
Substância, levemente açucarada, que se extrai da sorva.
\section{Sorbona}
\begin{itemize}
\item {Grp. gram.:f.}
\end{itemize}
A Faculdade de Theologia da Universidade de Paris, de que foi fundador, em 1252, Roberto Sorbon.
Hoje, séde dos cursos públicos das Faculdades da Universidade de Paris.--Escreve-se com S maiúsculo.
\section{Sorbonista}
\begin{itemize}
\item {Grp. gram.:m.  e  adj.}
\end{itemize}
Indivíduo, diplomado por uma Faculdade da Sorbona.
\section{Sorça}
\begin{itemize}
\item {fónica:sôr}
\end{itemize}
\begin{itemize}
\item {Grp. gram.:f.}
\end{itemize}
\begin{itemize}
\item {Utilização:Prov.}
\end{itemize}
\begin{itemize}
\item {Utilização:trasm.}
\end{itemize}
\begin{itemize}
\item {Utilização:Ant.}
\end{itemize}
\begin{itemize}
\item {Utilização:Prov.}
\end{itemize}
\begin{itemize}
\item {Utilização:trasm.}
\end{itemize}
Espécie de chouriço:«\textunderscore ...enxeram sorças dellas\textunderscore  (aves)»\textunderscore Lendas da Índia\textunderscore , c. IX.
O mesmo que \textunderscore surça\textunderscore .
(Cp. \textunderscore surça\textunderscore  e \textunderscore çorça\textunderscore )
\section{Sordes}
\begin{itemize}
\item {Grp. gram.:m. e f.}
\end{itemize}
\begin{itemize}
\item {Utilização:Pop.}
\end{itemize}
\begin{itemize}
\item {Proveniência:(Lat. \textunderscore sordes\textunderscore )}
\end{itemize}
O mesmo que \textunderscore pus\textunderscore .
\section{Sordícia}
\begin{itemize}
\item {Grp. gram.:f.}
\end{itemize}
\begin{itemize}
\item {Proveniência:(Lat. \textunderscore sorditia\textunderscore )}
\end{itemize}
Sordidez; sordes:«\textunderscore ...conspurcado na sordícia dos vícios\textunderscore ». Camillo, \textunderscore Caveira\textunderscore , 179.
\section{Sordície}
\begin{itemize}
\item {Grp. gram.:f.}
\end{itemize}
\begin{itemize}
\item {Proveniência:(Lat. \textunderscore sordities\textunderscore )}
\end{itemize}
O mesmo que \textunderscore sordícia\textunderscore .
\section{Sordidamente}
\begin{itemize}
\item {Grp. gram.:adv.}
\end{itemize}
De modo sórdido; indignamente; immundamente.
\section{Sordidez}
\begin{itemize}
\item {Grp. gram.:f.}
\end{itemize}
\begin{itemize}
\item {Utilização:Ext.}
\end{itemize}
Qualidade ou estado do que é sórdido.
Immundície.
Indignidade.
Vileza.
Torpeza.
Avareza repugnante.
\section{Sordideza}
\begin{itemize}
\item {Grp. gram.:f.}
\end{itemize}
O mesmo que \textunderscore sordidez\textunderscore .
\section{Sórdido}
\begin{itemize}
\item {Grp. gram.:adj.}
\end{itemize}
\begin{itemize}
\item {Proveniência:(Lat. \textunderscore sordidus\textunderscore )}
\end{itemize}
Sujo.
Esquálido.
Nojento.
Torpe.
Avarento.
Vil.
Obsceno.
\section{Sordo}
\begin{itemize}
\item {fónica:sôr}
\end{itemize}
\begin{itemize}
\item {Grp. gram.:adj.}
\end{itemize}
\begin{itemize}
\item {Utilização:Prov.}
\end{itemize}
\begin{itemize}
\item {Utilização:beir.}
\end{itemize}
Diz-se de um lugar, em que se ouve mal ou que tem más condições acústicas.
(Cast. \textunderscore sordo\textunderscore  = \textunderscore surdo\textunderscore )
\section{Sorelo}
\begin{itemize}
\item {fónica:sorê}
\end{itemize}
\begin{itemize}
\item {Grp. gram.:m.}
\end{itemize}
\begin{itemize}
\item {Utilização:Prov.}
\end{itemize}
\begin{itemize}
\item {Utilização:minh.}
\end{itemize}
Pequeno peixe, semelhante ao carapau.
\section{Sorema}
\begin{itemize}
\item {Grp. gram.:f.}
\end{itemize}
Gênero de plantas solanáceas.
\section{Sorgo}
\begin{itemize}
\item {Grp. gram.:m.}
\end{itemize}
\begin{itemize}
\item {Proveniência:(Do b. lat. \textunderscore surgus\textunderscore )}
\end{itemize}
Espécie de milho, (\textunderscore sorghum vulgare\textunderscore ).
Corpulenta árvore africana, da fam. das gramíneas.
Milho burro ou milho zaburro vermelho, (\textunderscore andropogon sorghum\textunderscore , Brotero).
\section{Sória}
\begin{itemize}
\item {Grp. gram.:f.}
\end{itemize}
\begin{itemize}
\item {Utilização:Ant.}
\end{itemize}
\begin{itemize}
\item {Proveniência:(De \textunderscore Sória\textunderscore , n. p.)}
\end{itemize}
Espécie de burel.
\section{Soriano}
\begin{itemize}
\item {Grp. gram.:adj.}
\end{itemize}
\begin{itemize}
\item {Grp. gram.:M.}
\end{itemize}
\begin{itemize}
\item {Utilização:Prov.}
\end{itemize}
\begin{itemize}
\item {Utilização:alg.}
\end{itemize}
Relativo a Sória, cidade espanhola.
Habitante de Sória.
Espécie de burel, de várias côres, fabricado no Algarve.
\section{Sorídio}
\begin{itemize}
\item {Grp. gram.:m.}
\end{itemize}
\begin{itemize}
\item {Proveniência:(Do gr. \textunderscore sorex\textunderscore )}
\end{itemize}
Gênero de reptis sáurios.
\section{Sorimões}
\begin{itemize}
\item {Grp. gram.:m. pl.}
\end{itemize}
\begin{itemize}
\item {Utilização:Bras}
\end{itemize}
Tríbo de aborígenes do Pará.
\section{Sorindeia}
\begin{itemize}
\item {Grp. gram.:f.}
\end{itemize}
Gênero de plantas anacardiáceas.
\section{Sorites}
\begin{itemize}
\item {Grp. gram.:m.}
\end{itemize}
\begin{itemize}
\item {Proveniência:(Lat. \textunderscore sorites\textunderscore )}
\end{itemize}
Raciocínio, composto de uma série de preposições, das quaes a segunda deve explicar o attributo da primeira, a terceira o attributo da segunda, e assim por deante, até que se chegue á conclusão que se procura.
\section{Sorítico}
\begin{itemize}
\item {Grp. gram.:adj.}
\end{itemize}
\begin{itemize}
\item {Proveniência:(Lat. \textunderscore soriticus\textunderscore )}
\end{itemize}
Relativo a sorites.
\section{Sormenho}
\begin{itemize}
\item {Grp. gram.:m.}
\end{itemize}
O mesmo que \textunderscore soromenho\textunderscore . Cf. M. Feijó, \textunderscore orthogr.\textunderscore 
\section{Sôrna}
\begin{itemize}
\item {Grp. gram.:f.}
\end{itemize}
\begin{itemize}
\item {Grp. gram.:M. ,  f.  e  adj.}
\end{itemize}
Indolência, inércia; soneca.
Pessôa inerte, indolente, preguiçosa.
(Cast. \textunderscore sorna\textunderscore )
\section{Sôrna}
\begin{itemize}
\item {Grp. gram.:f.}
\end{itemize}
\begin{itemize}
\item {Utilização:Gír.}
\end{itemize}
\begin{itemize}
\item {Grp. gram.:Adj.}
\end{itemize}
\begin{itemize}
\item {Proveniência:(Do provn. \textunderscore sorn\textunderscore , obscuro)}
\end{itemize}
Cama; noite.
Maçador, impertinente. Cf. Corvo, \textunderscore Anno na Côrte\textunderscore , c. II.
\section{Sórna}
\begin{itemize}
\item {Grp. gram.:f.}
\end{itemize}
\begin{itemize}
\item {Grp. gram.:M. ,  f.  e  adj.}
\end{itemize}
Indolência, inércia; soneca.
Pessôa inerte, indolente, preguiçosa.
(Cast. \textunderscore sorna\textunderscore )
\section{Sórna}
\begin{itemize}
\item {Grp. gram.:f.}
\end{itemize}
\begin{itemize}
\item {Utilização:Gír.}
\end{itemize}
\begin{itemize}
\item {Grp. gram.:Adj.}
\end{itemize}
\begin{itemize}
\item {Proveniência:(Do provn. \textunderscore sorn\textunderscore , obscuro)}
\end{itemize}
Cama; noite.
Maçador, impertinente. Cf. Corvo, \textunderscore Anno na Côrte\textunderscore , c. II.
\section{Sornar}
\begin{itemize}
\item {Grp. gram.:v. i.}
\end{itemize}
\begin{itemize}
\item {Utilização:Prov.}
\end{itemize}
\begin{itemize}
\item {Utilização:trasm.}
\end{itemize}
\begin{itemize}
\item {Grp. gram.:Loc.}
\end{itemize}
\begin{itemize}
\item {Utilização:fam.}
\end{itemize}
Dormir, rosnando.
\textunderscore Está sornando\textunderscore , estar nas tintas, não fazer caso.
(Provavelmente, por \textunderscore sorrenar\textunderscore , metáth. de \textunderscore resonar\textunderscore )
\section{Sornar}
\begin{itemize}
\item {Grp. gram.:v. i.}
\end{itemize}
\begin{itemize}
\item {Proveniência:(De \textunderscore sôrna\textunderscore ^1)}
\end{itemize}
Sêr pachorrento.
Proceder com sôrna.
\section{Sorneiro}
\begin{itemize}
\item {Grp. gram.:m.  e  adj.}
\end{itemize}
O que sórna.
\section{Sôro}
\begin{itemize}
\item {Grp. gram.:m.}
\end{itemize}
\begin{itemize}
\item {Proveniência:(Do lat. \textunderscore serum\textunderscore )}
\end{itemize}
Parte aquosa, que se separa do leite, quando êste se coagula ou quando se fórma o queijo.
Líquido, que se separa dos grumos do sangue, depois que êste se coagula.
\section{Soroca}
\begin{itemize}
\item {Grp. gram.:f.}
\end{itemize}
\begin{itemize}
\item {Utilização:Bras}
\end{itemize}
O mesmo que \textunderscore sororoca\textunderscore .
\section{Soromenha}
\begin{itemize}
\item {Grp. gram.:f.  e  adj.}
\end{itemize}
Pereira brava.
Fruto dessa árvore.
\section{Soromenho}
\begin{itemize}
\item {Grp. gram.:m.  e  adj.}
\end{itemize}
O mesmo que \textunderscore soromenha\textunderscore . Cp. \textunderscore sarmenho\textunderscore , e \textunderscore saramenho\textunderscore .
\section{Soror}
\begin{itemize}
\item {Grp. gram.:f.}
\end{itemize}
\begin{itemize}
\item {Proveniência:(Lat. \textunderscore soror\textunderscore )}
\end{itemize}
Tratamento, que se da ás freiras.--Filinto, X, 127, fórma o pl. \textunderscore sórors\textunderscore , mas tal formação é avêssa á índole da língua. O pl. é \textunderscore sorores\textunderscore . Cp. \textunderscore carácter\textunderscore , e \textunderscore caractéres\textunderscore .
\section{Sororal}
\begin{itemize}
\item {Grp. gram.:adj.}
\end{itemize}
\begin{itemize}
\item {Utilização:Neol.}
\end{itemize}
O mesmo que \textunderscore sorório\textunderscore .
\section{Sorório}
\begin{itemize}
\item {Grp. gram.:adj.}
\end{itemize}
\begin{itemize}
\item {Utilização:P. us.}
\end{itemize}
\begin{itemize}
\item {Proveniência:(Lat. \textunderscore sororius\textunderscore )}
\end{itemize}
Relativo a soror: \textunderscore hábito sorório\textunderscore .
\section{Sororoca}
\begin{itemize}
\item {Grp. gram.:f.}
\end{itemize}
\begin{itemize}
\item {Utilização:Bras}
\end{itemize}
Peixe marítimo, semelhante á cavalla, (\textunderscore sybium regale\textunderscore ).
\section{Sorosa}
\begin{itemize}
\item {Grp. gram.:f.}
\end{itemize}
O mesmo que \textunderscore sorose\textunderscore .
\section{Sorose}
\begin{itemize}
\item {Grp. gram.:f.}
\end{itemize}
\begin{itemize}
\item {Utilização:Bot.}
\end{itemize}
\begin{itemize}
\item {Proveniência:(Do gr. \textunderscore soros\textunderscore )}
\end{itemize}
Fruto, formado pela reunião de muitos num só, como a amora, o ananás, etc.
\section{Soroso}
\begin{itemize}
\item {Grp. gram.:adj.}
\end{itemize}
\begin{itemize}
\item {Proveniência:(De \textunderscore sôro\textunderscore )}
\end{itemize}
O mesmo ou melhor que \textunderscore seroso\textunderscore .
\section{Soroterapia}
\begin{itemize}
\item {Grp. gram.:f.}
\end{itemize}
O mesmo que \textunderscore seroterapia\textunderscore .
\section{Sorotherapia}
\begin{itemize}
\item {Grp. gram.:f.}
\end{itemize}
O mesmo que \textunderscore serotherapia\textunderscore .
\section{Sorprender}
\begin{itemize}
\item {Grp. gram.:v. t.}
\end{itemize}
O mesmo ou melhór que \textunderscore surprehender\textunderscore .
\section{Sorrabar}
\begin{itemize}
\item {Grp. gram.:v. t.}
\end{itemize}
\begin{itemize}
\item {Proveniência:(De \textunderscore so...\textunderscore  + \textunderscore rabo\textunderscore )}
\end{itemize}
Andar atrás de; bajular.
\section{Sorrascadoiro}
\begin{itemize}
\item {Grp. gram.:m.}
\end{itemize}
\begin{itemize}
\item {Utilização:Prov.}
\end{itemize}
\begin{itemize}
\item {Utilização:dur.}
\end{itemize}
\begin{itemize}
\item {Utilização:minh.}
\end{itemize}
\begin{itemize}
\item {Proveniência:(De \textunderscore sorrascar\textunderscore )}
\end{itemize}
Vassoiro com que se limpam os fornos, depois de quentes.
\section{Sorrascadouro}
\begin{itemize}
\item {Grp. gram.:m.}
\end{itemize}
\begin{itemize}
\item {Utilização:Prov.}
\end{itemize}
\begin{itemize}
\item {Utilização:dur.}
\end{itemize}
\begin{itemize}
\item {Utilização:minh.}
\end{itemize}
\begin{itemize}
\item {Proveniência:(De \textunderscore sorrascar\textunderscore )}
\end{itemize}
Vassoiro com que se limpam os fornos, depois de quentes.
\section{Sorrascador}
\begin{itemize}
\item {Grp. gram.:m.}
\end{itemize}
O mesmo que \textunderscore sorrasqueiro\textunderscore .
\section{Sorrascar}
\begin{itemize}
\item {Grp. gram.:v. t.  e  i.}
\end{itemize}
\begin{itemize}
\item {Utilização:Prov.}
\end{itemize}
\begin{itemize}
\item {Utilização:dur.}
\end{itemize}
\begin{itemize}
\item {Utilização:minh.}
\end{itemize}
\begin{itemize}
\item {Proveniência:(De \textunderscore so...\textunderscore  + \textunderscore rascar\textunderscore )}
\end{itemize}
Agitar ou varrer a cinza do forno com o sorrascadoiro.
\section{Sorrasco}
\begin{itemize}
\item {Grp. gram.:m.}
\end{itemize}
\begin{itemize}
\item {Utilização:Prov.}
\end{itemize}
\begin{itemize}
\item {Utilização:minh.}
\end{itemize}
Bolo ázymo, cozido nas brasas.
O mesmo que \textunderscore sorrascadoiro\textunderscore .
\section{Sorrasqueiro}
\begin{itemize}
\item {Grp. gram.:m.}
\end{itemize}
\begin{itemize}
\item {Utilização:Prov.}
\end{itemize}
\begin{itemize}
\item {Utilização:minh.}
\end{itemize}
\begin{itemize}
\item {Proveniência:(De \textunderscore sorrascar\textunderscore )}
\end{itemize}
Homem, que trabalha com o sorrascadoiro.
\section{Sorrasquinho}
\begin{itemize}
\item {Grp. gram.:m.}
\end{itemize}
\begin{itemize}
\item {Utilização:Prov.}
\end{itemize}
\begin{itemize}
\item {Utilização:minh.}
\end{itemize}
\begin{itemize}
\item {Proveniência:(De \textunderscore sorrasco\textunderscore )}
\end{itemize}
Pequeno bolo, cozido nas brasas.
\section{Sorrateiro}
\textunderscore adj.\textunderscore  (e der.)
O mesmo que \textunderscore surrateiro\textunderscore , etc.
\section{Sorrelfa}
\begin{itemize}
\item {Grp. gram.:f.}
\end{itemize}
\begin{itemize}
\item {Grp. gram.:M. ,  f.  e  adj.}
\end{itemize}
Sonsice; disfarce para enganar.
Socapa.
Pessôa manhosa.
Pessôa avarenta.
\section{Sorrelfo}
\begin{itemize}
\item {Grp. gram.:adj.}
\end{itemize}
\begin{itemize}
\item {Utilização:Des.}
\end{itemize}
Dissimulado; sonso.
(Cp. \textunderscore sorrelfa\textunderscore )
\section{Sorrenar}
\begin{itemize}
\item {Grp. gram.:v. i.}
\end{itemize}
\begin{itemize}
\item {Utilização:Gír.}
\end{itemize}
Ressonar; dormir.
(Metáth. de \textunderscore ressonar\textunderscore )
\section{Sorridente}
\begin{itemize}
\item {Grp. gram.:adj.}
\end{itemize}
\begin{itemize}
\item {Proveniência:(Do lat. \textunderscore subridens\textunderscore )}
\end{itemize}
Que sorri; prazenteiro; alegre; amável.
\section{Sorrir}
\begin{itemize}
\item {Grp. gram.:v. i.}
\end{itemize}
\begin{itemize}
\item {Grp. gram.:V. i.}
\end{itemize}
\begin{itemize}
\item {Grp. gram.:M.}
\end{itemize}
\begin{itemize}
\item {Proveniência:(Do lat. \textunderscore subridere\textunderscore )}
\end{itemize}
Rir sem ruído, rir levemente, fazendo apenas uma ligeira contracção dos músculos faciaes.
Mostrar-se alegre.
Aprazer: \textunderscore é um plano que me sorri\textunderscore .
Dar esperanças, mostrar-se prometedor.
Significar de um modo risonho.
Exprimir agradavelmente.
Sorriso.
\section{Sorriscar}
\begin{itemize}
\item {Grp. gram.:v. t.}
\end{itemize}
\begin{itemize}
\item {Utilização:T. de Turquel}
\end{itemize}
\begin{itemize}
\item {Proveniência:(De \textunderscore so...\textunderscore  + \textunderscore riscar\textunderscore ?)}
\end{itemize}
Salpicar.
\section{Sorriso}
\begin{itemize}
\item {Grp. gram.:m.}
\end{itemize}
\begin{itemize}
\item {Proveniência:(Do lat. \textunderscore subrisus\textunderscore )}
\end{itemize}
Acto de sorrir.
Manifestação, que se faz, de um sentimento de benevolência, de sympathia ou de ironia, sorrindo.
\section{Sorroda}
\begin{itemize}
\item {Grp. gram.:f.}
\end{itemize}
\begin{itemize}
\item {Utilização:T. de Elvas}
\end{itemize}
\begin{itemize}
\item {Proveniência:(De \textunderscore so...\textunderscore  + \textunderscore roda\textunderscore )}
\end{itemize}
O mesmo que \textunderscore relheira\textunderscore .
\section{Sorrolho}
\begin{itemize}
\item {fónica:rô}
\end{itemize}
\begin{itemize}
\item {Grp. gram.:m.}
\end{itemize}
\begin{itemize}
\item {Utilização:Prov.}
\end{itemize}
\begin{itemize}
\item {Utilização:trasm.}
\end{itemize}
Escuridão.
(Por \textunderscore cerrolho\textunderscore , de \textunderscore cerrar\textunderscore ?)
\section{Sorte}
\begin{itemize}
\item {Grp. gram.:f.}
\end{itemize}
\begin{itemize}
\item {Utilização:Fig.}
\end{itemize}
\begin{itemize}
\item {Utilização:Bras. do N}
\end{itemize}
\begin{itemize}
\item {Utilização:Prov.}
\end{itemize}
\begin{itemize}
\item {Grp. gram.:Loc. adv.}
\end{itemize}
\begin{itemize}
\item {Grp. gram.:Loc.}
\end{itemize}
\begin{itemize}
\item {Utilização:fam.}
\end{itemize}
\begin{itemize}
\item {Grp. gram.:Loc. adv.}
\end{itemize}
\begin{itemize}
\item {Grp. gram.:Pl.}
\end{itemize}
\begin{itemize}
\item {Proveniência:(Lat. \textunderscore sors\textunderscore , \textunderscore sortis\textunderscore )}
\end{itemize}
Destino, considerado como causa dos acontecimentos da vida, segundo a crença dos antigos.
Acaso.
Risco.
Successo casual.
Ventura inesperada.
Fortuna.
Quinhão, aquillo que coube a alguém em partilha.
Estado de alguém, relativamente a riqueza.
Sorteamento militar.
Bilhete ou pequena esphera, nas rifas ou lotarias.
Desgraça.
Qualidade, mancha, modo.
Sortimento: \textunderscore de tal sorte se portaram...\textunderscore 
Lote de fazendas ou tecidos.
Manobra, para farpear o toiro ou para o enganar.
Movimento do toiro, deixando-se farpear.
Ponto de ganhar, no jôgo.
Cada uma das reses, que cabem ao vaqueiro, em pagamento do seu trabalho.
Leira, faixa de terreno lavradio, não murada, mas limitada por marcos.
\textunderscore Á sorte\textunderscore , ao acaso, por meio de sorteio.
\textunderscore Estar com sorte\textunderscore , estar em maré de felicidade.
\textunderscore Dar sorte\textunderscore , dar o cavaco, amuar.
Corresponder a manifestações amorosas. Cf. Camillo, \textunderscore Corja\textunderscore .
\textunderscore De sorte\textunderscore , raramente, só por acaso.
\textunderscore De sorte maravilha\textunderscore  (a mesma sign.).
\textunderscore Cair nas sortes\textunderscore , sêr recrutado ou apurado para o serviço militar.
\textunderscore Entrar nas sortes\textunderscore , sêr recenseado para o serviço militar, mas ainda não sorteado.
\section{Sorteadamente}
\begin{itemize}
\item {Grp. gram.:adv.}
\end{itemize}
De modo sorteado; por sorteio; á sorte.
\section{Sorteado}
\begin{itemize}
\item {Grp. gram.:adj.}
\end{itemize}
Escolhido ou designado por sorte.
Que no sorteio do recrutamento teve um número que o obriga a assentar praça.
Variado, (falando-se de côres, drogas ou fazendas).
Sortido, misturado.
\section{Sorteador}
\begin{itemize}
\item {Grp. gram.:m.  e  adj.}
\end{itemize}
O que sorteia.
\section{Sorteamento}
\begin{itemize}
\item {Grp. gram.:m.}
\end{itemize}
O mesmo que \textunderscore sorteio\textunderscore .
\section{Sortear}
\begin{itemize}
\item {Grp. gram.:v. t.}
\end{itemize}
\begin{itemize}
\item {Proveniência:(De \textunderscore sorte\textunderscore )}
\end{itemize}
Repartir por sortes.
Determinar ou escolher por sorte.
Rifar.
Sortir; variar.
\section{Sortegar}
\begin{itemize}
\item {Grp. gram.:v. t.}
\end{itemize}
\begin{itemize}
\item {Utilização:Ant.}
\end{itemize}
O mesmo que \textunderscore sortear\textunderscore .
\section{Sorteio}
\begin{itemize}
\item {Grp. gram.:m.}
\end{itemize}
Acto ou effeito de sortear.
\section{Sorteiro}
\begin{itemize}
\item {Grp. gram.:m.}
\end{itemize}
\begin{itemize}
\item {Proveniência:(De \textunderscore sorte\textunderscore )}
\end{itemize}
Aquelle que sorteia.
\section{Sortela}
\begin{itemize}
\item {Grp. gram.:f.}
\end{itemize}
O mesmo que \textunderscore sortilha\textunderscore .
\section{Sortelha}
\begin{itemize}
\item {fónica:tê}
\end{itemize}
\begin{itemize}
\item {Grp. gram.:f.}
\end{itemize}
O mesmo que \textunderscore sortilha\textunderscore .
\section{Sortida}
\begin{itemize}
\item {Grp. gram.:f.}
\end{itemize}
(Corr. de \textunderscore surtida\textunderscore )
\section{Sortido}
\begin{itemize}
\item {Grp. gram.:m.}
\end{itemize}
O mesmo que \textunderscore sortimento\textunderscore : \textunderscore um grande sortido de fazendas claras\textunderscore .
\section{Sortilégio}
\begin{itemize}
\item {Grp. gram.:m.}
\end{itemize}
\begin{itemize}
\item {Proveniência:(Do lat. \textunderscore sortiligium\textunderscore )}
\end{itemize}
Malefício de feiticeiro.
Maquinação.
\section{Sortílego}
\begin{itemize}
\item {Grp. gram.:m.  e  adj.}
\end{itemize}
\begin{itemize}
\item {Proveniência:(Lat. \textunderscore sortilegus\textunderscore )}
\end{itemize}
O que faz sortilégios.
\section{Sortilha}
\begin{itemize}
\item {Grp. gram.:f.}
\end{itemize}
\begin{itemize}
\item {Utilização:Ant.}
\end{itemize}
\begin{itemize}
\item {Proveniência:(Do lat. \textunderscore sorticula\textunderscore )}
\end{itemize}
Anel, empregado especialmente em sortilégios ou na magia.
\section{Sortimento}
\begin{itemize}
\item {Grp. gram.:m.}
\end{itemize}
Acto ou effeito de sortir^1.
Provisão de fazendas, drogas, etc.
Mistura de coisas várias.
\section{Sortir}
\begin{itemize}
\item {Grp. gram.:v. t.}
\end{itemize}
\begin{itemize}
\item {Grp. gram.:V. i.}
\end{itemize}
\begin{itemize}
\item {Grp. gram.:V. p.}
\end{itemize}
\begin{itemize}
\item {Proveniência:(Lat. \textunderscore sortire\textunderscore )}
\end{itemize}
Abastecer.
Variar.
Combinar.
Misturar.
Caber em sorte. Cf. Filinto, XXI, 227; XVII, 196.
Fazer provisão de drogas, fazendas, etc.
\section{Sortir}
\begin{itemize}
\item {Grp. gram.:v. t.  e  i.}
\end{itemize}
(Alter. de \textunderscore surtir\textunderscore )
\section{Soruma}
\begin{itemize}
\item {Grp. gram.:f.}
\end{itemize}
\begin{itemize}
\item {Utilização:T. da Áfr. Or}
\end{itemize}
Planta, (\textunderscore cannabis sativa\textunderscore ), cuja fôlha os Negros aproveitam para fumar.
\section{Sorumbático}
\begin{itemize}
\item {Grp. gram.:m.  e  adj.}
\end{itemize}
O que é sombrio, triste, macambúzio.
(Por \textunderscore sombriático\textunderscore , de \textunderscore sombrio\textunderscore ?)
\section{Sôrva}
\begin{itemize}
\item {Grp. gram.:f.}
\end{itemize}
\begin{itemize}
\item {Proveniência:(Do lat. \textunderscore sorbum\textunderscore )}
\end{itemize}
Fruto da sorveira.
\section{Sôrva}
\begin{itemize}
\item {Grp. gram.:f.}
\end{itemize}
Árvore leitosa do Brasil, cuja goma substitue o breu.
O mesmo que \textunderscore sorveira\textunderscore ?
\section{Sorval}
\begin{itemize}
\item {Grp. gram.:adj.}
\end{itemize}
\begin{itemize}
\item {Proveniência:(De \textunderscore sôrva\textunderscore ^1)}
\end{itemize}
Diz-se de uma pêra sumarenta.
\section{Sorvalhada}
\begin{itemize}
\item {Grp. gram.:f.}
\end{itemize}
\begin{itemize}
\item {Proveniência:(De \textunderscore sôrva\textunderscore ^1)}
\end{itemize}
Grande porção de fruta, espalhada em desordem pelo chão.
\section{Sorvar}
\begin{itemize}
\item {Grp. gram.:v. i.  e  p.}
\end{itemize}
\begin{itemize}
\item {Proveniência:(De \textunderscore sôrva\textunderscore ^1)}
\end{itemize}
Começar a apodrecer, (falando-se da fruta).
Estar combalido, ir amollecendo.
\section{Sorvedela}
\begin{itemize}
\item {Grp. gram.:f.}
\end{itemize}
Acto de sorver.
\section{Sorvedoiro}
\begin{itemize}
\item {Grp. gram.:m.}
\end{itemize}
\begin{itemize}
\item {Proveniência:(De \textunderscore sorver\textunderscore )}
\end{itemize}
Remoínho de água, em mar ou rio.
\section{Sorvedouro}
\begin{itemize}
\item {Grp. gram.:m.}
\end{itemize}
\begin{itemize}
\item {Proveniência:(De \textunderscore sorver\textunderscore )}
\end{itemize}
Remoínho de água, em mar ou rio.
\section{Sorvedura}
\begin{itemize}
\item {Grp. gram.:f.}
\end{itemize}
O mesmo que \textunderscore sôrvo\textunderscore .
\section{Sorveira}
\begin{itemize}
\item {Grp. gram.:f.}
\end{itemize}
\begin{itemize}
\item {Proveniência:(De \textunderscore sôrva\textunderscore ^1)}
\end{itemize}
Árvore rosácea, (\textunderscore pyrus sorbus\textunderscore ).
\section{Sorveira-dos-passarinhos}
\begin{itemize}
\item {Grp. gram.:f.}
\end{itemize}
Arbusto, o mesmo que \textunderscore tramazeira\textunderscore  e \textunderscore corno-godinho.\textunderscore 
\section{Sorver}
\begin{itemize}
\item {Grp. gram.:v. t.}
\end{itemize}
\begin{itemize}
\item {Utilização:Fig.}
\end{itemize}
\begin{itemize}
\item {Proveniência:(Lat. \textunderscore sorbere\textunderscore )}
\end{itemize}
Haurir ou beber, aspirando.
Chupar.
Beber a pouco e pouco.
Absorver.
Attrahir para baixo.
Subverter.
Afundar.
Recolher.
Aniquilar.
\section{Sorvete}
\begin{itemize}
\item {fónica:vê}
\end{itemize}
\begin{itemize}
\item {Grp. gram.:m.}
\end{itemize}
\begin{itemize}
\item {Grp. gram.:F.}
\end{itemize}
Confeição gelada de leite, sumo de frutas, etc.
Espécie de limonada, usada pelos Turcos.
Variedade de pêra portuguesa.
(Cast. \textunderscore sorvete\textunderscore )
\section{Sorveteira}
\begin{itemize}
\item {Grp. gram.:f.}
\end{itemize}
Apparelho, para fazer sorvetes e outros gelados.
\section{Sorvível}
\begin{itemize}
\item {Grp. gram.:adj.}
\end{itemize}
Que se póde sorver.
\section{Sôrvo}
\begin{itemize}
\item {Grp. gram.:m.}
\end{itemize}
Acto ou effeito de sorver.
Trago; gole.
\section{Sós, a}
\begin{itemize}
\item {Grp. gram.:loc. adv.}
\end{itemize}
\begin{itemize}
\item {Proveniência:(De \textunderscore só\textunderscore )}
\end{itemize}
Sem companhia ou sem auxílio de outrem.
Solitariamente.
\section{Sosa}
\begin{itemize}
\item {Grp. gram.:f.}
\end{itemize}
Árvore do Congo.
\section{Sosano}
\begin{itemize}
\item {Grp. gram.:m.}
\end{itemize}
\begin{itemize}
\item {Utilização:Ant.}
\end{itemize}
Qualidade de resoluto; desembaraço.
\section{Sósia}
\begin{itemize}
\item {Grp. gram.:m.}
\end{itemize}
\begin{itemize}
\item {Proveniência:(Lat. \textunderscore Sosia\textunderscore , n. p.)}
\end{itemize}
Indivíduo parecido ou semelhante a outro.
\section{Soslaio}
\begin{itemize}
\item {Grp. gram.:m.}
\end{itemize}
Obliquidade, (us. só na loc. adv. \textunderscore de soslaio\textunderscore , de esguelha, de través).
\section{Sosquinar}
\begin{itemize}
\item {Grp. gram.:v. t.}
\end{itemize}
\begin{itemize}
\item {Utilização:Ant.}
\end{itemize}
\begin{itemize}
\item {Proveniência:(Do cast. \textunderscore sosquin\textunderscore )}
\end{itemize}
Fazer inclinar.
\section{Sossega}
\begin{itemize}
\item {Grp. gram.:f.}
\end{itemize}
\begin{itemize}
\item {Utilização:Pop.}
\end{itemize}
\begin{itemize}
\item {Utilização:Fam.}
\end{itemize}
\begin{itemize}
\item {Utilização:Prov.}
\end{itemize}
\begin{itemize}
\item {Utilização:minh.}
\end{itemize}
\begin{itemize}
\item {Proveniência:(De \textunderscore sossegar\textunderscore )}
\end{itemize}
Vinho, que se bebe, para conciliar o somno.
Somno.
Árvore, plantada em terreno municipal e que o dono não póde cortar.
\section{Sossegadamente}
\begin{itemize}
\item {Grp. gram.:adv.}
\end{itemize}
Com sossêgo; em paz.
\section{Sossegador}
\begin{itemize}
\item {Grp. gram.:m.  e  adj.}
\end{itemize}
O que sossega ou tranquilliza.
\section{Sossegamento}
\begin{itemize}
\item {Grp. gram.:m.}
\end{itemize}
Acto de sossegar.
\section{Sossegar}
\begin{itemize}
\item {Grp. gram.:v. t.}
\end{itemize}
\begin{itemize}
\item {Grp. gram.:V. i.}
\end{itemize}
\begin{itemize}
\item {Proveniência:(Do lat. hyp. \textunderscore sessicare\textunderscore ?)}
\end{itemize}
Dar descanso a.
Pôr em descanso.
Tranquillizar.
Descansar.
Tornar-se quieto, acalmar-se.
Deixar de sêr desordeiro.
\section{Sossêgo}
\begin{itemize}
\item {Grp. gram.:m.}
\end{itemize}
Acto ou effeito de sossegar.
\section{Sosso}
\begin{itemize}
\item {fónica:sô}
\end{itemize}
\begin{itemize}
\item {Grp. gram.:adj.}
\end{itemize}
Diz-se da pedra, que entra na construcção do uma parede, sem argamassa.
\section{Sossobrar}
\textunderscore v. t.\textunderscore  e \textunderscore i.\textunderscore  (e der.)
(V. \textunderscore soçobrar\textunderscore . etc.)
\section{Sostinentes}
\begin{itemize}
\item {Grp. gram.:m. pl.}
\end{itemize}
(V.sustinentes)
\section{Sostra}
\begin{itemize}
\item {fónica:sôs}
\end{itemize}
\begin{itemize}
\item {Grp. gram.:f.}
\end{itemize}
\begin{itemize}
\item {Utilização:Ant.}
\end{itemize}
\begin{itemize}
\item {Utilização:Prov.}
\end{itemize}
\begin{itemize}
\item {Utilização:beir.}
\end{itemize}
O mesmo que \textunderscore crosta\textunderscore .
Mulhér suja e enxovalhada.
O mesmo que \textunderscore lôstra\textunderscore .
\section{Sostrice}
\begin{itemize}
\item {Grp. gram.:f.}
\end{itemize}
Qualidade da mulhér sostra.
\section{Soiéria}
\begin{itemize}
\item {Grp. gram.:f.}
\end{itemize}
\begin{itemize}
\item {Proveniência:(De \textunderscore Soyer\textunderscore , n. p.)}
\end{itemize}
Gênero de plantas, da fam. das compostas.
\section{Sota}
\begin{itemize}
\item {Grp. gram.:f.}
\end{itemize}
\begin{itemize}
\item {Grp. gram.:Loc.}
\end{itemize}
\begin{itemize}
\item {Utilização:fam.}
\end{itemize}
\begin{itemize}
\item {Grp. gram.:Pl.}
\end{itemize}
\begin{itemize}
\item {Grp. gram.:M.}
\end{itemize}
\begin{itemize}
\item {Utilização:Prov.}
\end{itemize}
\begin{itemize}
\item {Utilização:alent.}
\end{itemize}
Dama, nas cartas de jogar.
Folga, descanso, intervallo.
\textunderscore Dar sota e ás\textunderscore , replicar com vantagem.
A parelha da frente, num carro puxado por mais de uma.
Bolieiro.
O que monta a cavalgadura da sella.
Rapaz que, em serviços de viação, conduz as cavalgaduras que reforçam nas subidas a parelha que puxa um vehículo.
Capataz de aguadeiros.
O que vai na frente dos que puxam uma bomba de incêndios.
Chefe ou capataz de macobios.
(Cp. \textunderscore sota...\textunderscore )
\section{Sota}
\begin{itemize}
\item {Grp. gram.:f.}
\end{itemize}
\begin{itemize}
\item {Utilização:Prov.}
\end{itemize}
\begin{itemize}
\item {Utilização:alg.}
\end{itemize}
Mulhér manhosa.
\section{Sota}
\begin{itemize}
\item {Grp. gram.:f.}
\end{itemize}
\begin{itemize}
\item {Utilização:Ant.}
\end{itemize}
Loja, o mesmo que \textunderscore sótão\textunderscore . Cf. \textunderscore Livro da Fazenda da Univ. de Coimbra\textunderscore , mss., 15 v.^o
\section{Sôta}
\begin{itemize}
\item {Grp. gram.:f.}
\end{itemize}
\begin{itemize}
\item {Utilização:T. da Guiné}
\end{itemize}
Movimento das águas, propício ao desembarque.
\section{Sota...}
\begin{itemize}
\item {Grp. gram.:pref.}
\end{itemize}
\begin{itemize}
\item {Proveniência:(Do lat. \textunderscore subtus\textunderscore )}
\end{itemize}
(designativo de \textunderscore inferior\textunderscore )
\section{Sotaans}
\begin{itemize}
\item {Grp. gram.:f. pl.}
\end{itemize}
Indígenas do norte do Brasil.
\section{Sòtacapitânea}
\begin{itemize}
\item {Grp. gram.:f.}
\end{itemize}
\begin{itemize}
\item {Proveniência:(De \textunderscore sota...\textunderscore  + \textunderscore capitânea\textunderscore )}
\end{itemize}
Antiga nau, que servia de capitânea.
\section{Sòtacapitão}
\begin{itemize}
\item {Grp. gram.:m.}
\end{itemize}
Commandante de sòtacapitânea. Cf. Goes, \textunderscore Chrón. D. Man.\textunderscore  II, 35.
\section{Sòtacocheiro}
\begin{itemize}
\item {Grp. gram.:m.}
\end{itemize}
\begin{itemize}
\item {Proveniência:(De \textunderscore sota...\textunderscore  + \textunderscore cocheiro\textunderscore )}
\end{itemize}
O segundo cocheiro, na viação antiga.
\section{Sòtacomitre}
\begin{itemize}
\item {Grp. gram.:m.}
\end{itemize}
\begin{itemize}
\item {Proveniência:(De \textunderscore sota...\textunderscore  + \textunderscore comitre\textunderscore )}
\end{itemize}
Segundo comitre.
\section{Sòtaembaixador}
\begin{itemize}
\item {Grp. gram.:m.}
\end{itemize}
\begin{itemize}
\item {Utilização:Des.}
\end{itemize}
Adjunto do embaixador. Cf. Tenreiro, c. II.
\section{Sòtageneral}
\begin{itemize}
\item {Grp. gram.:m.}
\end{itemize}
Adjunto do general. Cf. Filinto, III, 17.
\section{Sotaina}
\begin{itemize}
\item {Grp. gram.:f.}
\end{itemize}
\begin{itemize}
\item {Grp. gram.:M.}
\end{itemize}
\begin{itemize}
\item {Utilização:Pop.}
\end{itemize}
\begin{itemize}
\item {Proveniência:(Do b. lat. \textunderscore subtana\textunderscore )}
\end{itemize}
O mesmo que batina de padre.
Padre.
\section{Sotaina}
\begin{itemize}
\item {Grp. gram.:f.}
\end{itemize}
\begin{itemize}
\item {Utilização:Prov.}
\end{itemize}
\begin{itemize}
\item {Utilização:trasm.}
\end{itemize}
\begin{itemize}
\item {Utilização:pop.}
\end{itemize}
Póla, sova.
\section{Sotal}
\begin{itemize}
\item {Grp. gram.:adv.}
\end{itemize}
\begin{itemize}
\item {Utilização:Ant.}
\end{itemize}
\begin{itemize}
\item {Proveniência:(De \textunderscore so...\textunderscore  + \textunderscore tal\textunderscore )}
\end{itemize}
Sob condição; condicionalmente.
\section{Sotalhar}
\begin{itemize}
\item {Grp. gram.:v. t.}
\end{itemize}
\begin{itemize}
\item {Proveniência:(De \textunderscore so...\textunderscore  + \textunderscore talhar\textunderscore )}
\end{itemize}
Tornar mais largo em baixo, (falando-se dos furos, que os canteiros abrem na pedra em que hão de embeber uma peça chumbada, de maneira que o chumbo forme cabeça inferiormente): \textunderscore êsse furo que fique bem sotalhado\textunderscore .
\section{Sotam}
\begin{itemize}
\item {Grp. gram.:m.}
\end{itemize}
\begin{itemize}
\item {Utilização:Prov.}
\end{itemize}
\begin{itemize}
\item {Proveniência:(Do ár. \textunderscore as sotehia\textunderscore )}
\end{itemize}
Terraço no alto de um edifício.
O pavimento mais alto de um edifício.
Compartimento esconso, na parte superior de um edifício.
Sôbre-camara.
Pavimento inferior de um prédio, ao rés-do-chão; loja:«\textunderscore ...sotam ou logea...\textunderscore »\textunderscore Orden. do Reino\textunderscore , liv. I, tit. I.
\section{Sotana}
\begin{itemize}
\item {Grp. gram.:f.}
\end{itemize}
O mesmo que \textunderscore sotaina\textunderscore ^1.
\section{Sótão}
\begin{itemize}
\item {Grp. gram.:m.}
\end{itemize}
\begin{itemize}
\item {Utilização:Prov.}
\end{itemize}
\begin{itemize}
\item {Proveniência:(Do ár. \textunderscore as sotehia\textunderscore )}
\end{itemize}
Terraço no alto de um edifício.
O pavimento mais alto de um edifício.
Compartimento esconso, na parte superior de um edifício.
Sôbre-camara.
Pavimento inferior de um prédio, ao rés-do-chão; loja:«\textunderscore ...sótão ou logea...\textunderscore »\textunderscore Orden. do Reino\textunderscore , liv. I, tit. I.
\section{Sòtapatrão}
\begin{itemize}
\item {Grp. gram.:m.}
\end{itemize}
\begin{itemize}
\item {Utilização:Náut.}
\end{itemize}
\begin{itemize}
\item {Proveniência:(De \textunderscore sota...\textunderscore  + \textunderscore patrão\textunderscore )}
\end{itemize}
Segundo patrão de galeota.
Indivíduo, que suppria o patrão, nas galeotas. Cf. Azurara, \textunderscore Chrón. do C. D. Pedro\textunderscore , c. LVIII, 401.
\section{Sòtapiloto}
\begin{itemize}
\item {Grp. gram.:m.}
\end{itemize}
\begin{itemize}
\item {Proveniência:(De \textunderscore sota...\textunderscore  + \textunderscore piloto\textunderscore )}
\end{itemize}
Segundo piloto.
Indivíduo, que supre a falta do piloto.
\section{Sotaque}
\begin{itemize}
\item {Grp. gram.:m.}
\end{itemize}
\begin{itemize}
\item {Utilização:Pop.}
\end{itemize}
\begin{itemize}
\item {Grp. gram.:Loc. adv.}
\end{itemize}
Remoque.
Dito picante.
Pronúncia peculiar a um indivíduo, a uma região, etc.
\textunderscore De sotaque\textunderscore , subitamente.
\section{Sotaquear}
\begin{itemize}
\item {Grp. gram.:v. t.}
\end{itemize}
\begin{itemize}
\item {Proveniência:(De \textunderscore sotaque\textunderscore )}
\end{itemize}
Jogar remoque a.
Mofar de, motejar de. Cf. \textunderscore Viriato Trág.\textunderscore , II, 75.
\section{Sotaventear}
\begin{itemize}
\item {Grp. gram.:v. t.}
\end{itemize}
\begin{itemize}
\item {Grp. gram.:V. i.  e  p.}
\end{itemize}
Voltar para sotavento (o navio).
Ir de barlavento para sotavento.
\section{Sotavento}
\begin{itemize}
\item {Grp. gram.:m.}
\end{itemize}
\begin{itemize}
\item {Proveniência:(De \textunderscore sota...\textunderscore  + \textunderscore vento\textunderscore )}
\end{itemize}
Borda do navio, opposta ao lado de onde sopra o vento.
\section{Soteia}
\begin{itemize}
\item {Grp. gram.:f.}
\end{itemize}
\begin{itemize}
\item {Utilização:Prov.}
\end{itemize}
\begin{itemize}
\item {Utilização:alg.}
\end{itemize}
Eirado ou terrado, em substituição do telhado.
O mesmo que \textunderscore assoteia\textunderscore .
(Cp. \textunderscore sótão\textunderscore )
\section{Soteiro}
\begin{itemize}
\item {Grp. gram.:adj.}
\end{itemize}
\begin{itemize}
\item {Proveniência:(De \textunderscore sota\textunderscore ^1)}
\end{itemize}
Diz-se de um dos cabos náuticos, com que se içam e arreiam pesos.
\section{Sotérias}
\begin{itemize}
\item {Grp. gram.:f. pl.}
\end{itemize}
\begin{itemize}
\item {Proveniência:(Lat. \textunderscore soleria\textunderscore )}
\end{itemize}
Festas, que os antigos celebravam em acção de graças aos deuses, por se verem livres de algum perigo ou de alguma epidemia
\section{Soterim}
\begin{itemize}
\item {Grp. gram.:m.}
\end{itemize}
Official de justiça, ou juiz de categoria inferior, entre os Judeus.
\section{Soternocamente}
\begin{itemize}
\item {Grp. gram.:adv.}
\end{itemize}
\begin{itemize}
\item {Utilização:Ant.}
\end{itemize}
Ás occultas; em segrêdo. Cf. S. R. Viterbo, \textunderscore Elucidário\textunderscore .
\section{Soterração}
\begin{itemize}
\item {Grp. gram.:f.}
\end{itemize}
O mesmo que \textunderscore soterramento\textunderscore .
\section{Soterramento}
\begin{itemize}
\item {Grp. gram.:m.}
\end{itemize}
\begin{itemize}
\item {Utilização:Ant.}
\end{itemize}
Acto ou effeito de soterrar^1.
Entêrro, funeral.
\section{Soterrâneo}
\begin{itemize}
\item {Grp. gram.:m.  e  adj.}
\end{itemize}
O mesmo que \textunderscore subterrâneo\textunderscore .
\section{Soterrar}
\begin{itemize}
\item {Grp. gram.:v. t.}
\end{itemize}
\begin{itemize}
\item {Proveniência:(De \textunderscore so...\textunderscore  + \textunderscore terra\textunderscore )}
\end{itemize}
Meter debaixo da terra; enterrar.
\section{Soterrar}
\begin{itemize}
\item {Grp. gram.:v. t.}
\end{itemize}
\begin{itemize}
\item {Utilização:P. us.}
\end{itemize}
\begin{itemize}
\item {Proveniência:(De \textunderscore so...\textunderscore  + \textunderscore aterrar\textunderscore )}
\end{itemize}
Aterrar muito, causar grande terror a.
\section{Sotia}
\begin{itemize}
\item {Grp. gram.:f.}
\end{itemize}
\begin{itemize}
\item {Proveniência:(Fr. \textunderscore sotie\textunderscore )}
\end{itemize}
Antiga peça do theatro francês, espécie de satira dialogada, em que os comediantes representavam personagens de um povo de doidos, com allusão a personagens do mundo real.
\section{Soticapa}
\begin{itemize}
\item {Grp. gram.:f.}
\end{itemize}
\begin{itemize}
\item {Utilização:Ant.}
\end{itemize}
\begin{itemize}
\item {Proveniência:(De \textunderscore soto...\textunderscore  + \textunderscore capa\textunderscore )}
\end{itemize}
O mesmo que \textunderscore socapa\textunderscore . Cf. \textunderscore Aulegrafia\textunderscore , 6.
\section{Sotil}
\textunderscore adj.\textunderscore  (e der.) \textunderscore Ant.\textunderscore 
O mesmo que \textunderscore subtil\textunderscore , etc. Cf. \textunderscore Aulegrafia\textunderscore , 3.
\section{Sotilicário}
\begin{itemize}
\item {Grp. gram.:m.}
\end{itemize}
Ave marítima, cujas asas sem pennas têm o aspecto de cotos, donde veio á mesma ave o nome de cotete, (\textunderscore aptenodyla dermersa\textunderscore , Lin.). Cf. Filinto, \textunderscore D. Man.\textunderscore , I, 71.
\section{Soto}
\begin{itemize}
\item {Grp. gram.:m.}
\end{itemize}
\begin{itemize}
\item {Utilização:pop.}
\end{itemize}
\begin{itemize}
\item {Utilização:Ant.}
\end{itemize}
O mesmo que \textunderscore sótão\textunderscore .
\section{Soto...}
\begin{itemize}
\item {Grp. gram.:pref.}
\end{itemize}
O mesmo que \textunderscore sota...\textunderscore 
\section{Sotoalmirante}
\begin{itemize}
\item {Grp. gram.:m.}
\end{itemize}
\begin{itemize}
\item {Proveniência:(De \textunderscore soto...\textunderscore  + \textunderscore almirante\textunderscore )}
\end{itemize}
Aquelle que substituía o almirante, na falta dêste.
\section{Sotoar}
\begin{itemize}
\item {Grp. gram.:m.}
\end{itemize}
(V.santor)
\section{Sotocapitão}
\begin{itemize}
\item {Grp. gram.:m.}
\end{itemize}
\begin{itemize}
\item {Proveniência:(De \textunderscore soto...\textunderscore  + \textunderscore capitão\textunderscore )}
\end{itemize}
Indivíduo, que substitue o capitão, a bordo; immediato.
\section{Sotoembaixador}
\begin{itemize}
\item {Grp. gram.:m.}
\end{itemize}
\begin{itemize}
\item {Utilização:Ant.}
\end{itemize}
\begin{itemize}
\item {Proveniência:(De \textunderscore soto...\textunderscore  + \textunderscore embaixador\textunderscore )}
\end{itemize}
O que acompanhava o embaixador, para o substituír nas suas faltas.
\section{Sotomestre}
\begin{itemize}
\item {Grp. gram.:m.}
\end{itemize}
\begin{itemize}
\item {Proveniência:(De \textunderscore soto...\textunderscore  + \textunderscore mestre\textunderscore )}
\end{itemize}
Indivíduo, que substitue o mestre, a bordo.
\section{Sotoministro}
\begin{itemize}
\item {Grp. gram.:m.}
\end{itemize}
\begin{itemize}
\item {Proveniência:(De \textunderscore soto...\textunderscore  + \textunderscore ministro\textunderscore )}
\end{itemize}
Jesuíta, que superintende nos confrades encarregados dos negócios da cozinha, refeitório, etc.
\section{Sotopiloto}
\begin{itemize}
\item {Grp. gram.:m.}
\end{itemize}
O mesmo que \textunderscore sòtapiloto\textunderscore .
\section{Sotopor}
\begin{itemize}
\item {Grp. gram.:v. t.}
\end{itemize}
\begin{itemize}
\item {Proveniência:(De \textunderscore soto...\textunderscore  + \textunderscore pôr\textunderscore )}
\end{itemize}
Pôr por baixo.
Omittir, postergar.
\section{Sotrancão}
\begin{itemize}
\item {Grp. gram.:adj.}
\end{itemize}
\begin{itemize}
\item {Proveniência:(De \textunderscore sotrancar\textunderscore )}
\end{itemize}
Dissimulado, sonso.
\section{Sotrancar}
\begin{itemize}
\item {Grp. gram.:v. t.}
\end{itemize}
\begin{itemize}
\item {Proveniência:(De \textunderscore so...\textunderscore  + \textunderscore trancar\textunderscore )}
\end{itemize}
O mesmo que \textunderscore abarcar\textunderscore .
\section{Sotroços}
\begin{itemize}
\item {fónica:trô}
\end{itemize}
\begin{itemize}
\item {Grp. gram.:m. pl.}
\end{itemize}
(V.setrossos)
\section{Soturnez}
\begin{itemize}
\item {Utilização:bras}
\end{itemize}
\begin{itemize}
\item {Utilização:Neol.}
\end{itemize}
O mesmo que \textunderscore soturnidade\textunderscore .
\section{Soturnidade}
\begin{itemize}
\item {Grp. gram.:f.}
\end{itemize}
Qualidade de soturno. Cf. G. Junqueiro, \textunderscore Simples\textunderscore , 37.
\section{Soturno}
\begin{itemize}
\item {Grp. gram.:adj.}
\end{itemize}
\begin{itemize}
\item {Utilização:Pop.}
\end{itemize}
\begin{itemize}
\item {Grp. gram.:M.}
\end{itemize}
\begin{itemize}
\item {Utilização:Pop.}
\end{itemize}
\begin{itemize}
\item {Proveniência:(De \textunderscore Saturno\textunderscore , n. p., por allusão á supposta infl. dêste planeta)}
\end{itemize}
Sombrio.
De apparência torva, tristonha.
Lúgubre; silencioso.
Quente ou abafadiço, (falando-se do tempo).
Aspecto triste ou taciturno.
Tempo quente e abafadiço.
\section{Sousa}
\begin{itemize}
\item {Grp. gram.:m.}
\end{itemize}
\begin{itemize}
\item {Utilização:T. de Leiria}
\end{itemize}
Espécie de pombo bravo, também conhecido por \textunderscore seixá\textunderscore .
Pêro esverdeado.
\section{Sousanas}
\begin{itemize}
\item {Grp. gram.:f. pl.}
\end{itemize}
\begin{itemize}
\item {Utilização:Prov.}
\end{itemize}
\begin{itemize}
\item {Utilização:trasm.}
\end{itemize}
O mesmo que [[soitinhas|soitinha]].
\section{Sousão}
\begin{itemize}
\item {Grp. gram.:m.}
\end{itemize}
\begin{itemize}
\item {Proveniência:(De \textunderscore Sousa\textunderscore , n. p.)}
\end{itemize}
Variedade de uva preta do Minho e Doiro.
\section{Souto}
\begin{itemize}
\item {Grp. gram.:m.}
\end{itemize}
\begin{itemize}
\item {Proveniência:(Do lat. \textunderscore saltus\textunderscore )}
\end{itemize}
Bosque denso.
Mata de castanheiros.
Lugar muito arborizado e próprio para passeio.
Alameda.
\section{Sova}
\begin{itemize}
\item {Grp. gram.:f.}
\end{itemize}
Acto ou effeito de sovar.
Tunda.
(Cp. cast. \textunderscore soba\textunderscore )
\section{Sova}
\textunderscore m.\textunderscore  (e der.)
O mesmo que \textunderscore soba\textunderscore , etc. Cf. Serpa Pinto, I, 141.
\section{Sovaco}
\begin{itemize}
\item {Grp. gram.:m.}
\end{itemize}
Cavidade extero-inferior, na juncção do braço com o ombro.
Axilla.
Peça de estôfo ou borracha, que as senhoras collocam na parte interior do vestuário, correspondente á axilla, para evitar as nódoas da transudação.
(Cp. cast. \textunderscore sobaco\textunderscore )
\section{Sovadura}
\begin{itemize}
\item {Grp. gram.:f.}
\end{itemize}
\begin{itemize}
\item {Utilização:Ant.}
\end{itemize}
O mesmo que \textunderscore sova\textunderscore ^1.
Amassadura de pão. Cf. B. Pereira, \textunderscore Prosódia\textunderscore , vb. \textunderscore exaggies\textunderscore .
\section{Sovaqueiro}
\begin{itemize}
\item {Grp. gram.:m.  e  adj.}
\end{itemize}
\begin{itemize}
\item {Proveniência:(De \textunderscore sovaco\textunderscore )}
\end{itemize}
Diz-se do gatuno, que tem o costume de fugir com os roubos debaixo do braço.
\section{Sovaquete}
\begin{itemize}
\item {fónica:quê}
\end{itemize}
\begin{itemize}
\item {Grp. gram.:m.}
\end{itemize}
\begin{itemize}
\item {Proveniência:(De \textunderscore sovar\textunderscore )}
\end{itemize}
Acto de tirar a péla da respectiva casa, no jôgo da péla.
\section{Sovaquinho}
\begin{itemize}
\item {Grp. gram.:adj.}
\end{itemize}
\begin{itemize}
\item {Utilização:Fam.}
\end{itemize}
\begin{itemize}
\item {Grp. gram.:M.}
\end{itemize}
Diz-se do cheiro desagradável, que emana do suór dos sovacos.
Cheira de sovacos suados.
\section{Sovar}
\begin{itemize}
\item {Grp. gram.:v. t.}
\end{itemize}
Amassar.
Bater a massa de.
Pisar (uvas)
Moêr.
Dar pancadas em.
(Cp. cast. \textunderscore sobar\textunderscore )
\section{Soveio}
\begin{itemize}
\item {Grp. gram.:m.}
\end{itemize}
\begin{itemize}
\item {Utilização:Prov.}
\end{itemize}
\begin{itemize}
\item {Utilização:trasm.}
\end{itemize}
Correia grossa, que prende o carro ou o arado ao jugo.
\section{Soveiro}
\begin{itemize}
\item {Grp. gram.:m.}
\end{itemize}
\begin{itemize}
\item {Utilização:T. de Bragança}
\end{itemize}
O mesmo que \textunderscore soveio\textunderscore .
\section{Sovela}
\begin{itemize}
\item {Grp. gram.:f.}
\end{itemize}
\begin{itemize}
\item {Proveniência:(Do lat. hyp. \textunderscore subella\textunderscore , dem. do lat. \textunderscore subula\textunderscore )}
\end{itemize}
Instrumento, formado de uma espécie de agulha direita ou curva e encabada, com que os sapateiros e correeiros furam o cabedal, para o coser.
Ave pernalta, espécie de pequeno maçarico.
O mesmo que \textunderscore alfaiate\textunderscore , ave.
\section{Sovelada}
\begin{itemize}
\item {Grp. gram.:f.}
\end{itemize}
Acto ou effeito de sovelar.
\section{Sovelão}
\begin{itemize}
\item {Grp. gram.:m.}
\end{itemize}
Grande sovela.
\textunderscore Voz de sovelão\textunderscore , voz aguda e áspera, de homem.
\section{Sovelão}
\begin{itemize}
\item {Grp. gram.:m.}
\end{itemize}
\begin{itemize}
\item {Utilização:Gír.}
\end{itemize}
Avarento.
(Cp. \textunderscore sovina\textunderscore )
\section{Sovelar}
\begin{itemize}
\item {Grp. gram.:v. t.}
\end{itemize}
\begin{itemize}
\item {Utilização:Fig.}
\end{itemize}
Furor com sovela.
Furar.
\section{Soveleiro}
\begin{itemize}
\item {Grp. gram.:m.}
\end{itemize}
Fabricante ou vendedor de sovelas.
\section{Soveral}
\begin{itemize}
\item {Grp. gram.:m.}
\end{itemize}
(V.sobral)
\section{Sovereira}
\begin{itemize}
\item {Grp. gram.:f.}
\end{itemize}
Pequeno sovereiro. Cf. G. Resende, \textunderscore Cancion. Ger.\textunderscore 
\section{Sovereiro}
\begin{itemize}
\item {Grp. gram.:m.}
\end{itemize}
(V.sobreiro)
\section{Sôvero}
\begin{itemize}
\item {Grp. gram.:m.}
\end{itemize}
O mesmo que \textunderscore sôbro\textunderscore . Cf. \textunderscore Eufrosina\textunderscore , 94.
\section{Soverter}
\textunderscore v. t.\textunderscore  (e der.)
Fórma ant. de \textunderscore subverter\textunderscore , etc. Cf. A. Ferreira, \textunderscore Castro\textunderscore , acto I.
\section{Sovessa}
\begin{itemize}
\item {Grp. gram.:f.}
\end{itemize}
\begin{itemize}
\item {Utilização:Prov.}
\end{itemize}
\begin{itemize}
\item {Utilização:trasm.}
\end{itemize}
\textunderscore Tomar alguém á sovessa\textunderscore , entrar e embirrar com êlle, tomá-lo de ponta.
(Cp. \textunderscore sobessa\textunderscore )
\section{Soveu}
\begin{itemize}
\item {Grp. gram.:m.}
\end{itemize}
\begin{itemize}
\item {Utilização:Bras. do S}
\end{itemize}
O mesmo que \textunderscore soveio\textunderscore .
Laço grosseiro, mas forte, com que se peiam os cavallos, e que é de dois ou tres tentos.
\section{Sovina}
\begin{itemize}
\item {Grp. gram.:f.}
\end{itemize}
\begin{itemize}
\item {Utilização:Prov.}
\end{itemize}
\begin{itemize}
\item {Utilização:beir.}
\end{itemize}
\begin{itemize}
\item {Utilização:Prov.}
\end{itemize}
\begin{itemize}
\item {Utilização:beir.}
\end{itemize}
\begin{itemize}
\item {Grp. gram.:M. ,  f.  e  adj.}
\end{itemize}
\begin{itemize}
\item {Grp. gram.:Adj.}
\end{itemize}
Tôrno de madeira.
Cavilha de pau, que retesa as brochas do mangual.
Instrumento perfurante, em fórma de lima.
Pau aguçado numa das pontos, para se picarem bêstas.
Pessôa avara, mesquinha, somítica, miserável.
Mesquinho, miserável, (falando-se de coisas ou actos):«\textunderscore remuneração sovina.\textunderscore »Camillo, \textunderscore Hist. e Sentiment.\textunderscore , 10.
\section{Sovinada}
\begin{itemize}
\item {Grp. gram.:f.}
\end{itemize}
\begin{itemize}
\item {Proveniência:(De \textunderscore sovinar\textunderscore )}
\end{itemize}
Picada ou golpe com sovina ou com outro instrumento perfurante.
Dito picante.
\section{Sovinar}
\begin{itemize}
\item {Grp. gram.:v. t.}
\end{itemize}
\begin{itemize}
\item {Utilização:Fig.}
\end{itemize}
Furor com sovina ou com outro instrumento análogo.
Molestar; magoar.
\section{Sovinaria}
\begin{itemize}
\item {Grp. gram.:f.}
\end{itemize}
O mesmo que \textunderscore sovinice\textunderscore .
\section{Sovinha}
\begin{itemize}
\item {Grp. gram.:f.}
\end{itemize}
\begin{itemize}
\item {Utilização:Prov.}
\end{itemize}
\begin{itemize}
\item {Utilização:trasm.}
\end{itemize}
Cada um de dois pregos de pau, que prendem os atafaes á albarda, quando esta não tem fivelas.
(Cp. \textunderscore sovina\textunderscore )
\section{Sovinice}
\begin{itemize}
\item {Grp. gram.:f.}
\end{itemize}
Qualidade de quem é sovina; avareza, mesquinhez.
\section{Soyéria}
\begin{itemize}
\item {Grp. gram.:f.}
\end{itemize}
\begin{itemize}
\item {Proveniência:(De \textunderscore Soyer\textunderscore , n. p.)}
\end{itemize}
Gênero de plantas, da fam. das compostas.
\section{Sozal}
\begin{itemize}
\item {Grp. gram.:m.}
\end{itemize}
\begin{itemize}
\item {Utilização:Pharm.}
\end{itemize}
Medicamento adstringente e antiséptico.
\section{Sòzinho}
\begin{itemize}
\item {Grp. gram.:adj.}
\end{itemize}
\begin{itemize}
\item {Proveniência:(De \textunderscore só\textunderscore )}
\end{itemize}
Inteiramente só; abandonado; único.
\section{Spadicifloro}
\begin{itemize}
\item {Grp. gram.:adj.}
\end{itemize}
\begin{itemize}
\item {Utilização:Bot.}
\end{itemize}
\begin{itemize}
\item {Proveniência:(Do lat. \textunderscore spatha\textunderscore  + \textunderscore flos\textunderscore , \textunderscore floris\textunderscore )}
\end{itemize}
Que tem as flôres contidas em uma espatha.
\section{Spalanzânia}
\begin{itemize}
\item {Grp. gram.:f.}
\end{itemize}
\begin{itemize}
\item {Proveniência:(De \textunderscore Spallanz\textunderscore , n. p.)}
\end{itemize}
Gênero de plantas rubiáceas.
\section{Spallanzânia}
\begin{itemize}
\item {Grp. gram.:f.}
\end{itemize}
\begin{itemize}
\item {Proveniência:(De \textunderscore Spallanz\textunderscore , n. p.)}
\end{itemize}
Gênero de plantas rubiáceas.
\section{Sparmânia}
\begin{itemize}
\item {Grp. gram.:f.}
\end{itemize}
\begin{itemize}
\item {Proveniência:(De \textunderscore Sparmann\textunderscore , n. p.)}
\end{itemize}
Gênero de plantas lilliáceas.
\section{Speklínia}
\begin{itemize}
\item {Grp. gram.:f.}
\end{itemize}
\begin{itemize}
\item {Proveniência:(De \textunderscore Speklin\textunderscore , n. p.)}
\end{itemize}
Gênero de orchídeas.
\section{Spera}
\textunderscore f.\textunderscore  (e der.)
O mesmo que \textunderscore esphera\textunderscore , etc. Cf. \textunderscore Rot. do Mar Verm.\textunderscore , 198.
\section{Spicanardo}
\begin{itemize}
\item {Grp. gram.:m.}
\end{itemize}
\begin{itemize}
\item {Proveniência:(Do lat. \textunderscore spica\textunderscore  + \textunderscore nardus\textunderscore )}
\end{itemize}
Nardo indiano.
\section{Spielmânia}
\begin{itemize}
\item {Grp. gram.:f.}
\end{itemize}
\begin{itemize}
\item {Proveniência:(De \textunderscore Spielmann\textunderscore , n. p.)}
\end{itemize}
Gênero de plantas verbenáceas.
\section{Spinella}
\begin{itemize}
\item {Grp. gram.:f.}
\end{itemize}
(V.espinela)
\section{Spinosismo}
\begin{itemize}
\item {Grp. gram.:m.}
\end{itemize}
Systema philosóphico de Spinosa, segundo o qual a natureza é activa e passiva.
\section{Spinosista}
\begin{itemize}
\item {Grp. gram.:m.}
\end{itemize}
Sectário de Spinosa.
(Cp. \textunderscore spinosismo\textunderscore )
\section{S. P. Q. R.}
Iniciaes, que designam a legenda \textunderscore senatus populusque romanus\textunderscore , (\textunderscore o Senado e o povo romano\textunderscore )
\section{Sr.}
Abrev., que precede nomes próprios de homens e significa \textunderscore senhor\textunderscore .
\section{Sr.ª}
Abrev., que precede nomes de mulhér e significa \textunderscore senhora\textunderscore .
\section{SS.}
Abrev. de \textunderscore Santissimo\textunderscore .
Abrev. de \textunderscore Sua-Santidade\textunderscore .
\section{S. S. E.}
Abrev. de \textunderscore susueste\textunderscore .
\section{Sta}
\begin{itemize}
\item {Grp. gram.:pron.}
\end{itemize}
\begin{itemize}
\item {Utilização:Ant.}
\end{itemize}
O mesmo que \textunderscore esta\textunderscore . Cf. S. R. Viterbo, \textunderscore Elucidário\textunderscore .
\section{Stahlianismo}
\begin{itemize}
\item {Grp. gram.:m.}
\end{itemize}
Doutrina dos Stahlianos.
\section{Stahlianos}
\begin{itemize}
\item {Grp. gram.:m. pl.}
\end{itemize}
Sectários de médico alemão Stahl, para quem os phenómenos da vida e as doenças estavam sob a direcção da alma, em-quanto o corpo era considerado inerte.
\section{Stallo}
\begin{itemize}
\item {Grp. gram.:m.}
\end{itemize}
\begin{itemize}
\item {Utilização:Ant.}
\end{itemize}
O mesmo que \textunderscore estau\textunderscore , Cf. Herculano, \textunderscore Bobo\textunderscore , 294, 295 e 301.
\section{Stanleya}
\begin{itemize}
\item {Grp. gram.:f.}
\end{itemize}
\begin{itemize}
\item {Proveniência:(De \textunderscore Stanley\textunderscore , n. p.)}
\end{itemize}
Gênero de plantas crucíferas.
\section{Stapélia}
\begin{itemize}
\item {Grp. gram.:f.}
\end{itemize}
\begin{itemize}
\item {Proveniência:(De \textunderscore Stapel\textunderscore , n. p.)}
\end{itemize}
Gênero de plantas asclepiadáceas.
\section{Stevênia}
\begin{itemize}
\item {Grp. gram.:f.}
\end{itemize}
Gênero de plantas crucíferas.
Gênero de insectos dípteros.
(Do \textunderscore Steven\textunderscore , n. p.)
\section{Sto}
\begin{itemize}
\item {Grp. gram.:pron.}
\end{itemize}
\begin{itemize}
\item {Utilização:Ant.}
\end{itemize}
O mesmo que \textunderscore isto\textunderscore .
\section{Stonce}
\begin{itemize}
\item {Grp. gram.:adv.}
\end{itemize}
\begin{itemize}
\item {Utilização:Ant.}
\end{itemize}
Então.
(Cp. cast. \textunderscore entonces\textunderscore )
\section{Strabónia}
\begin{itemize}
\item {Grp. gram.:f.}
\end{itemize}
\begin{itemize}
\item {Proveniência:(De \textunderscore Estrabão\textunderscore , n. p.)}
\end{itemize}
Gênero de plantas, da fam. das compostas.
\section{Strado}
\begin{itemize}
\item {Grp. gram.:adj.}
\end{itemize}
\begin{itemize}
\item {Utilização:Ant.}
\end{itemize}
O mesmo que [[prostrado|prostrar]]. Cf. Frei Fortun., \textunderscore Inéd.\textunderscore , 315.
\section{Stráusia}
\begin{itemize}
\item {Grp. gram.:f.}
\end{itemize}
\begin{itemize}
\item {Proveniência:(De \textunderscore Straus\textunderscore , n. p.)}
\end{itemize}
Gênero de insectos dípteros.
\section{Striga}
\begin{itemize}
\item {Grp. gram.:f.}
\end{itemize}
\begin{itemize}
\item {Proveniência:(Lat. \textunderscore striga\textunderscore )}
\end{itemize}
Bruxa, feiticeira:«\textunderscore acentos sinistros de uma velha striga.\textunderscore »Herculano, \textunderscore Bobo\textunderscore , 72.
\section{Strogonówia}
\begin{itemize}
\item {Grp. gram.:f.}
\end{itemize}
\begin{itemize}
\item {Proveniência:(De \textunderscore Strogonow\textunderscore , n. p.)}
\end{itemize}
Gênero de plantas crucíferas.
\section{Stuarita}
\begin{itemize}
\item {Grp. gram.:f.}
\end{itemize}
\begin{itemize}
\item {Proveniência:(De \textunderscore Stuar\textunderscore , n. p.)}
\end{itemize}
Gênero de plantas ternstremiáceas.
\section{Stúrmia}
\begin{itemize}
\item {Grp. gram.:f.}
\end{itemize}
\begin{itemize}
\item {Proveniência:(De \textunderscore Sturm\textunderscore , n. p.)}
\end{itemize}
Gênero de insectos dípteros.
\section{Sua}
\begin{itemize}
\item {Grp. gram.:pron.}
\end{itemize}
\begin{itemize}
\item {Grp. gram.:F.}
\end{itemize}
(Fem. de \textunderscore seu\textunderscore )
Fórma ellíptica, em vez de \textunderscore sua opinião\textunderscore , \textunderscore sua vontade\textunderscore , etc.: \textunderscore sempre ficou com a sua\textunderscore ; \textunderscore venceu a sua\textunderscore .
\section{Suã}
\begin{itemize}
\item {Grp. gram.:f.}
\end{itemize}
\begin{itemize}
\item {Utilização:Prov.}
\end{itemize}
\begin{itemize}
\item {Proveniência:(Do lat. \textunderscore sus\textunderscore , \textunderscore suis\textunderscore )}
\end{itemize}
Carne de porco, da parte inferior do lombo.
Ossos da espinha dorsal dos porcos.
\section{Suacar}
\begin{itemize}
\item {Grp. gram.:v. i.}
\end{itemize}
\begin{itemize}
\item {Utilização:Ant.}
\end{itemize}
Suar muito? Cf. G. Vicente, I, 264.
\section{Suaçu}
\begin{itemize}
\item {Grp. gram.:m.}
\end{itemize}
\begin{itemize}
\item {Utilização:Bras}
\end{itemize}
O mesmo que \textunderscore veado\textunderscore ^1.
(Do guar.)
\section{Suadela}
\begin{itemize}
\item {Grp. gram.:f.}
\end{itemize}
\begin{itemize}
\item {Proveniência:(Lat. \textunderscore suadela\textunderscore )}
\end{itemize}
O mesmo que \textunderscore persuasão\textunderscore . Cf. Filinto, III, 178.
\section{Suadela}
\begin{itemize}
\item {Grp. gram.:f.}
\end{itemize}
\begin{itemize}
\item {Utilização:Fam.}
\end{itemize}
Acto de suar.
O mesmo que \textunderscore transpiração\textunderscore .
\section{Suadir}
\begin{itemize}
\item {Grp. gram.:v. t.}
\end{itemize}
\begin{itemize}
\item {Utilização:Ant.}
\end{itemize}
\begin{itemize}
\item {Proveniência:(Lat. \textunderscore suadere\textunderscore )}
\end{itemize}
O mesmo que \textunderscore persuadir\textunderscore .
\section{Suado}
\begin{itemize}
\item {Grp. gram.:adj.}
\end{itemize}
\begin{itemize}
\item {Utilização:Fig.}
\end{itemize}
Que tem suor.
Que custou muito trabalho, que se adquiriu com trabalho.
\section{Suadoiro}
\begin{itemize}
\item {Grp. gram.:m.}
\end{itemize}
Acto ou effeito de suar.
Sudorífico.
Lavagem de vasilhas com água, sal e outras substâncias.
Parte do lombo da cavalgadura, correspondente á sella.
Xairel de lan.
\section{Suadouro}
\begin{itemize}
\item {Grp. gram.:m.}
\end{itemize}
Acto ou effeito de suar.
Sudorífico.
Lavagem de vasilhas com água, sal e outras substâncias.
Parte do lombo da cavalgadura, correspondente á sella.
Xairel de lan.
\section{Suan}
\begin{itemize}
\item {Grp. gram.:f.}
\end{itemize}
\begin{itemize}
\item {Utilização:Prov.}
\end{itemize}
\begin{itemize}
\item {Proveniência:(Do lat. \textunderscore sus\textunderscore , \textunderscore suis\textunderscore )}
\end{itemize}
Carne de porco, da parte inferior do lombo.
Ossos da espinha dorsal dos porcos.
\section{Suangue}
\begin{itemize}
\item {Grp. gram.:m.}
\end{itemize}
Nome que, em Timor, se dá ao feiticeiro.
\section{Suão}
\begin{itemize}
\item {Grp. gram.:m.  e  adj.}
\end{itemize}
(V.soão)
\section{Suar}
\begin{itemize}
\item {Grp. gram.:v. i.}
\end{itemize}
\begin{itemize}
\item {Utilização:Fig.}
\end{itemize}
\begin{itemize}
\item {Grp. gram.:V. t.}
\end{itemize}
\begin{itemize}
\item {Proveniência:(Do lat. \textunderscore sudare.\textunderscore  Cp. \textunderscore suór\textunderscore )}
\end{itemize}
Deitar suor pelos poros.
Transudar, transpirar.
Resumar.
Afadigar-se.
Empregar grande esfôrço.
Verter.
Expellir como suor: \textunderscore suar sangue\textunderscore .
Adquirir com grande trabalho.
\section{Suarabácti}
\begin{itemize}
\item {Grp. gram.:m.}
\end{itemize}
\begin{itemize}
\item {Utilização:sanscrit}
\end{itemize}
\begin{itemize}
\item {Utilização:Gram.}
\end{itemize}
Vogal intercalar, que desune consoantes, como em \textunderscore prão\textunderscore  = \textunderscore porão\textunderscore .
\section{Suarda}
\begin{itemize}
\item {Grp. gram.:f.}
\end{itemize}
Matéria gordurosa da lan de ovelha.
Substância oleosa, que os panos deixam no pisão.
Nódoa na lan antes de cardada.
\section{Suarento}
\begin{itemize}
\item {Grp. gram.:adj.}
\end{itemize}
\begin{itemize}
\item {Proveniência:(De \textunderscore suar\textunderscore )}
\end{itemize}
Que tem suór.
Coberto do suór.
\section{Suári}
\begin{itemize}
\item {Grp. gram.:m.}
\end{itemize}
Grande árvore da Guiana inglesa, (\textunderscore caryocar tomentosum\textunderscore ).
\section{Suasão}
\begin{itemize}
\item {Grp. gram.:f.}
\end{itemize}
\begin{itemize}
\item {Utilização:Ant.}
\end{itemize}
\begin{itemize}
\item {Proveniência:(Do lat. \textunderscore suasio\textunderscore )}
\end{itemize}
O mesmo que \textunderscore persuasão\textunderscore .
\section{Suasivo}
\begin{itemize}
\item {Grp. gram.:adj.}
\end{itemize}
O mesmo que \textunderscore suasório\textunderscore .
Cf. Filinto, XV, 23.
\section{Suasório}
\begin{itemize}
\item {Grp. gram.:adj.}
\end{itemize}
\begin{itemize}
\item {Proveniência:(Lat. \textunderscore suasorius\textunderscore )}
\end{itemize}
O mesmo que \textunderscore persuasivo\textunderscore .
\section{Suassureça}
\begin{itemize}
\item {Grp. gram.:m.}
\end{itemize}
\begin{itemize}
\item {Utilização:Bras. do N}
\end{itemize}
Planta medicinal, do frutos saborosos.
O fruto dessa planta.--O cónego Sousa chama-lhe \textunderscore suassureçá\textunderscore ; L. da Fonseca, \textunderscore Flora Bras.\textunderscore , diz \textunderscore suassureça\textunderscore . Serão permittidas as duas fórmas?
\section{Suassureçá}
\begin{itemize}
\item {Grp. gram.:m.}
\end{itemize}
\begin{itemize}
\item {Utilização:Bras. do N}
\end{itemize}
Planta medicinal, do frutos saborosos.
O fruto dessa planta.--O cónego Sousa chama-lhe \textunderscore suassureçá\textunderscore ; L. da Fonseca, \textunderscore Flora Bras.\textunderscore , diz \textunderscore suassureça\textunderscore . Serão permittidas as duas fórmas?
\section{Suástica}
\begin{itemize}
\item {Grp. gram.:m.  ou  f.}
\end{itemize}
Antigo sýmbolo religioso, de origem ariana, em fórma de cruz, tendo as extremidades das hastes recurvadas ou angulares.
(Do sanscr.)
\section{Suave}
\begin{itemize}
\item {Grp. gram.:adj.}
\end{itemize}
\begin{itemize}
\item {Proveniência:(Lat. \textunderscore suavis\textunderscore )}
\end{itemize}
Agradável, aprazível.
Que tem doçura.
Meigo; melodioso: \textunderscore música suave\textunderscore .
Ameno: \textunderscore tempo suave\textunderscore .
Brando.
Que custa pouco ou que se faz sem sacrifício: \textunderscore trabalho suave\textunderscore .
Delicado.
\section{Suavemente}
\begin{itemize}
\item {Grp. gram.:adv.}
\end{itemize}
De modo suave.
Agradavelmente; harmoniosamente.
\section{Suavidade}
\begin{itemize}
\item {Grp. gram.:f.}
\end{itemize}
\begin{itemize}
\item {Proveniência:(Do lat. \textunderscore suavitas\textunderscore )}
\end{itemize}
Qualidade do que é suave.
Impressão agradável, produzida nos sentidos.
Doçura (de um som, de uma voz, etc.)
Graça.
Macieza.
Estado tranquillo e agradável da alma.
Graça divina.
\section{Suaviloquência}
\begin{itemize}
\item {fónica:cu-en}
\end{itemize}
\begin{itemize}
\item {Grp. gram.:f.}
\end{itemize}
\begin{itemize}
\item {Proveniência:(Lat. \textunderscore suaviloquentia\textunderscore )}
\end{itemize}
Suavidade ou doçura nas palavras, na linguagem.
\section{Suaviloquente}
\begin{itemize}
\item {fónica:qu-en}
\end{itemize}
\begin{itemize}
\item {Grp. gram.:adj.}
\end{itemize}
\begin{itemize}
\item {Proveniência:(Lat. \textunderscore suaviloquens\textunderscore )}
\end{itemize}
Que tem doçura ou suavidade nas palavras ou na linguagem. Cf. Filinto, XIII, 86.
\section{Suavíloquo}
\begin{itemize}
\item {Grp. gram.:adj.}
\end{itemize}
\begin{itemize}
\item {Proveniência:(Lat. \textunderscore suaviloquus\textunderscore )}
\end{itemize}
O mesmo que \textunderscore suaviloquente\textunderscore .
\section{Suavização}
\begin{itemize}
\item {Grp. gram.:f.}
\end{itemize}
Acto ou effeito de suavizar.
\section{Suavizador}
\begin{itemize}
\item {Grp. gram.:adj.}
\end{itemize}
Que suaviza.
\section{Suavizar}
\begin{itemize}
\item {Grp. gram.:v. t.}
\end{itemize}
\begin{itemize}
\item {Utilização:Fig.}
\end{itemize}
\begin{itemize}
\item {Proveniência:(De \textunderscore suave\textunderscore )}
\end{itemize}
Tornar suave.
Abrandar, mitigar: \textunderscore suavizar dores\textunderscore .
\section{Sub...}
\begin{itemize}
\item {Grp. gram.:pref.}
\end{itemize}
\begin{itemize}
\item {Proveniência:(Lat. \textunderscore sub\textunderscore )}
\end{itemize}
(designativo de \textunderscore inferioridade\textunderscore , \textunderscore substituição\textunderscore , \textunderscore aproximação\textunderscore , etc.)
\section{Subabdominal}
\begin{itemize}
\item {Grp. gram.:adj.}
\end{itemize}
\begin{itemize}
\item {Utilização:Zool.}
\end{itemize}
\begin{itemize}
\item {Proveniência:(De \textunderscore sub...\textunderscore  + \textunderscore abdominal\textunderscore )}
\end{itemize}
Situado abaixo do abdome.
\section{Subabia}
\begin{itemize}
\item {Grp. gram.:f.}
\end{itemize}
Região, governada por um subabo.
\section{Subabo}
\begin{itemize}
\item {Grp. gram.:m.}
\end{itemize}
Espécie de Vice-rei que, em nome do Grão-Mogol, governava grande divisão do império mongol na Índia, com categoria superior á do nababo.
\section{Subacetato}
\begin{itemize}
\item {Grp. gram.:m.}
\end{itemize}
\begin{itemize}
\item {Utilização:Chím.}
\end{itemize}
\begin{itemize}
\item {Proveniência:(De \textunderscore sub...\textunderscore  + \textunderscore acetato\textunderscore )}
\end{itemize}
Acetato, com excesso de base.
\section{Subacicular}
\begin{itemize}
\item {Grp. gram.:adj.}
\end{itemize}
\begin{itemize}
\item {Utilização:Miner.}
\end{itemize}
\begin{itemize}
\item {Proveniência:(De \textunderscore sub...\textunderscore  + \textunderscore acicular\textunderscore )}
\end{itemize}
Que tem quási a fórma de agulha.
\section{Subácido}
\begin{itemize}
\item {Grp. gram.:adj.}
\end{itemize}
\begin{itemize}
\item {Proveniência:(De \textunderscore sub...\textunderscore  + \textunderscore ácido\textunderscore )}
\end{itemize}
Que tem propriedades quási análogas ás dos ácidos.
\section{Subaéreo}
\begin{itemize}
\item {Grp. gram.:adj.}
\end{itemize}
Que está por baixo da camada inferior da atmosphera.
(Do \textunderscore sub...\textunderscore  + \textunderscore aéreo\textunderscore )
\section{Subagudo}
\begin{itemize}
\item {Grp. gram.:adj.}
\end{itemize}
\begin{itemize}
\item {Proveniência:(De \textunderscore sub...\textunderscore  + \textunderscore agudo\textunderscore )}
\end{itemize}
Levemente agudo.
\section{Subailio}
\begin{itemize}
\item {Grp. gram.:m.}
\end{itemize}
\begin{itemize}
\item {Utilização:Des.}
\end{itemize}
\begin{itemize}
\item {Proveniência:(De \textunderscore sub...\textunderscore  + \textunderscore bailio\textunderscore )}
\end{itemize}
O substituto do bailio.
\section{Subalado}
\begin{itemize}
\item {Grp. gram.:adj.}
\end{itemize}
\begin{itemize}
\item {Utilização:Zool.}
\end{itemize}
\begin{itemize}
\item {Proveniência:(De \textunderscore sub...\textunderscore  + \textunderscore alado\textunderscore )}
\end{itemize}
Que tem appêndices, parecidos com asas.
\section{Subalar}
\begin{itemize}
\item {Grp. gram.:adj.}
\end{itemize}
\begin{itemize}
\item {Utilização:Zool.}
\end{itemize}
\begin{itemize}
\item {Proveniência:(Lat. \textunderscore subalaris\textunderscore )}
\end{itemize}
Que está debaixo das asas.
\section{Subalcaide}
\begin{itemize}
\item {Grp. gram.:m.}
\end{itemize}
\begin{itemize}
\item {Proveniência:(De \textunderscore sub...\textunderscore  + \textunderscore alcaide\textunderscore )}
\end{itemize}
Substituto do alcaide; segundo alcaide.
\section{Subalhitos}
\begin{itemize}
\item {Grp. gram.:m. pl.}
\end{itemize}
\begin{itemize}
\item {Utilização:Prov.}
\end{itemize}
\begin{itemize}
\item {Utilização:trasm.}
\end{itemize}
Restos de comida.
(Talvez por \textunderscore sibalhitos\textunderscore , de \textunderscore sibalho\textunderscore )
\section{Subalpino}
\begin{itemize}
\item {Grp. gram.:adj.}
\end{itemize}
\begin{itemize}
\item {Utilização:Poét.}
\end{itemize}
\begin{itemize}
\item {Proveniência:(Lat. \textunderscore sub-alpinus\textunderscore , de \textunderscore sub...\textunderscore  + \textunderscore Alpes\textunderscore , n. p.)}
\end{itemize}
Situado nas faldas dos Alpes.
Diz-se da zona montanhosa, situada entre 1200 e 2000 metros de altitude.
\section{Subalternação}
\begin{itemize}
\item {Grp. gram.:f.}
\end{itemize}
Acto ou effeito de subalternar.
Qualidade do que é subalterno ou do que subalterna com outro.
\section{Subalternadamente}
\begin{itemize}
\item {Grp. gram.:adv.}
\end{itemize}
De modo subalternado; com subalternação.
\section{Subalternar}
\begin{itemize}
\item {Grp. gram.:v. t.}
\end{itemize}
\begin{itemize}
\item {Grp. gram.:V. i.  e  p.}
\end{itemize}
\begin{itemize}
\item {Proveniência:(De \textunderscore sub...\textunderscore  + \textunderscore alternar\textunderscore )}
\end{itemize}
Tornar subalterno.
Pôr em categoria inferior.
Alternar-se.
\section{Subalternidade}
\begin{itemize}
\item {Grp. gram.:f.}
\end{itemize}
Qualidade do que é subalterno; inferioridade; dependência.
\section{Subalternizamento}
\begin{itemize}
\item {Grp. gram.:m.}
\end{itemize}
\begin{itemize}
\item {Utilização:bras}
\end{itemize}
\begin{itemize}
\item {Utilização:Neol.}
\end{itemize}
Acto ou effeito de subalternizar.
\section{Subalternizar}
\begin{itemize}
\item {Grp. gram.:v. t.}
\end{itemize}
Tornar subalterno.
Dar categoria inferior a.
\section{Subalterno}
\begin{itemize}
\item {Grp. gram.:adj.}
\end{itemize}
\begin{itemize}
\item {Grp. gram.:M.}
\end{itemize}
\begin{itemize}
\item {Proveniência:(Lat. \textunderscore subalternus\textunderscore )}
\end{itemize}
Que é sujeito a outro.
Que tem graduação ou autoridade inferior á de outrem.
Subordinado.
Inferior.
Indivíduo subalterno.
\section{Subalugar}
\begin{itemize}
\item {Grp. gram.:v. t.}
\end{itemize}
O mesmo que \textunderscore sublocar\textunderscore .
\section{Subaluguér}
\begin{itemize}
\item {Grp. gram.:m.}
\end{itemize}
O mesmo que \textunderscore sublocação\textunderscore .
\section{Subáptero}
\begin{itemize}
\item {Grp. gram.:adj.}
\end{itemize}
\begin{itemize}
\item {Utilização:Zool.}
\end{itemize}
\begin{itemize}
\item {Proveniência:(De \textunderscore sub...\textunderscore  + \textunderscore áptero\textunderscore )}
\end{itemize}
Que tem alguma semelhança com os insectos ápteros.
\section{Subaquático}
\begin{itemize}
\item {Grp. gram.:adj.}
\end{itemize}
\begin{itemize}
\item {Proveniência:(De \textunderscore sub...\textunderscore  + \textunderscore aquático\textunderscore )}
\end{itemize}
Que está debaixo de água.
\section{Subarbústeo}
\begin{itemize}
\item {Grp. gram.:adj.}
\end{itemize}
\begin{itemize}
\item {Utilização:Bot.}
\end{itemize}
\begin{itemize}
\item {Proveniência:(De \textunderscore subarbusto\textunderscore )}
\end{itemize}
Diz-se do tronco, cujos ramos secam annualmente.
\section{Subarbusto}
\begin{itemize}
\item {Grp. gram.:m.}
\end{itemize}
\begin{itemize}
\item {Proveniência:(De \textunderscore sub...\textunderscore  + \textunderscore arbusto\textunderscore )}
\end{itemize}
Planta, que occupa o meio termo entre o arbusto e a erva.
\section{Subarmal}
\begin{itemize}
\item {Grp. gram.:m.}
\end{itemize}
\begin{itemize}
\item {Proveniência:(Lat. \textunderscore subarmalis\textunderscore )}
\end{itemize}
Vestuário, que os Romanos usavam por baixo das armas.
\section{Subarqaeado}
\begin{itemize}
\item {Grp. gram.:adj.}
\end{itemize}
\begin{itemize}
\item {Proveniência:(De \textunderscore sub...\textunderscore  + \textunderscore arqueado\textunderscore )}
\end{itemize}
Pouco arqueado.
\section{Subarrendamento}
\begin{itemize}
\item {Grp. gram.:m.}
\end{itemize}
Acto ou effeito de subarrendar.
\section{Subarrendar}
\begin{itemize}
\item {Grp. gram.:v. t.}
\end{itemize}
\begin{itemize}
\item {Proveniência:(De \textunderscore sub...\textunderscore  + \textunderscore arrendar\textunderscore )}
\end{itemize}
Arrendar a outro (aquillo que se tinha tomado de arrendamento).
Sublocar.
\section{Subarrendatário}
\begin{itemize}
\item {Grp. gram.:m.  e  adj.}
\end{itemize}
\begin{itemize}
\item {Proveniência:(De \textunderscore sub...\textunderscore  + \textunderscore arrendatário\textunderscore )}
\end{itemize}
O que tomou de subarrendamento um prédio.
\section{Subarseniato}
\begin{itemize}
\item {Grp. gram.:m.}
\end{itemize}
\begin{itemize}
\item {Utilização:Chím.}
\end{itemize}
\begin{itemize}
\item {Proveniência:(De \textunderscore sub...\textunderscore  + \textunderscore arseniato\textunderscore )}
\end{itemize}
Arseniato, com excesso de base.
\section{Subauriforme}
\begin{itemize}
\item {Grp. gram.:adj.}
\end{itemize}
\begin{itemize}
\item {Proveniência:(De \textunderscore sub...\textunderscore  + \textunderscore auriforme\textunderscore )}
\end{itemize}
Que se parece um pouco com uma orelha.
\section{Subaxilar}
\begin{itemize}
\item {Grp. gram.:adj.}
\end{itemize}
\begin{itemize}
\item {Utilização:Bot.}
\end{itemize}
\begin{itemize}
\item {Proveniência:(De \textunderscore sub...\textunderscore  + \textunderscore axilar\textunderscore )}
\end{itemize}
Que está ou parece estar debaixo do axila.
\section{Subaxillar}
\begin{itemize}
\item {Grp. gram.:adj.}
\end{itemize}
\begin{itemize}
\item {Utilização:Bot.}
\end{itemize}
\begin{itemize}
\item {Proveniência:(De \textunderscore sub...\textunderscore  + \textunderscore axillar\textunderscore )}
\end{itemize}
Que está ou parece estar debaixo do axilla.
\section{Sub-bailio}
\begin{itemize}
\item {Grp. gram.:m.}
\end{itemize}
\begin{itemize}
\item {Utilização:Des.}
\end{itemize}
\begin{itemize}
\item {Proveniência:(De \textunderscore sub...\textunderscore  + \textunderscore bailio\textunderscore )}
\end{itemize}
O substituto do bailio.
\section{Sub-bibliothecário}
\begin{itemize}
\item {Grp. gram.:m.}
\end{itemize}
Funccionário de bibliotheca, immediato ao bibliothecário ou substituto dêlle.
\section{Subcacuminal}
\begin{itemize}
\item {Grp. gram.:adj.}
\end{itemize}
\begin{itemize}
\item {Utilização:Gram.}
\end{itemize}
\begin{itemize}
\item {Proveniência:(Do lat. \textunderscore sub\textunderscore  + \textunderscore cacumen\textunderscore )}
\end{itemize}
O mesmo que \textunderscore reverso\textunderscore .
\section{Subcampanulado}
\begin{itemize}
\item {Grp. gram.:adj.}
\end{itemize}
\begin{itemize}
\item {Utilização:Bot.}
\end{itemize}
\begin{itemize}
\item {Proveniência:(De \textunderscore sub...\textunderscore  + \textunderscore campânula\textunderscore )}
\end{itemize}
Cuja fórma se aproxima da de uma campaínha.
\section{Subcapilar}
\begin{itemize}
\item {Grp. gram.:adj.}
\end{itemize}
\begin{itemize}
\item {Proveniência:(De \textunderscore sub...\textunderscore  + \textunderscore capilar\textunderscore )}
\end{itemize}
Que tem quáse a tenuidade de um cabelo.
\section{Subcapillar}
\begin{itemize}
\item {Grp. gram.:adj.}
\end{itemize}
\begin{itemize}
\item {Proveniência:(De \textunderscore sub...\textunderscore  + \textunderscore capillar\textunderscore )}
\end{itemize}
Que tem quáse a tenuidade de um cabello.
\section{Subcarbonato}
\begin{itemize}
\item {Grp. gram.:m.}
\end{itemize}
\begin{itemize}
\item {Utilização:Chím.}
\end{itemize}
\begin{itemize}
\item {Proveniência:(De \textunderscore sub...\textunderscore  + \textunderscore carbonato\textunderscore )}
\end{itemize}
Designação genérica dos saes, em que o ácido carbónico se encontra com um excesso de base.
\section{Subcaudal}
\begin{itemize}
\item {Grp. gram.:adj.}
\end{itemize}
\begin{itemize}
\item {Proveniência:(De \textunderscore sub...\textunderscore  + \textunderscore cauda\textunderscore )}
\end{itemize}
Que está por baixo da cauda.
\section{Subcensítico}
\begin{itemize}
\item {Grp. gram.:adj.}
\end{itemize}
\begin{itemize}
\item {Utilização:Jur.}
\end{itemize}
\begin{itemize}
\item {Proveniência:(De \textunderscore sub...\textunderscore  + \textunderscore censítico\textunderscore )}
\end{itemize}
O mesmo que \textunderscore subemphytêutico\textunderscore .
\section{Subcessor}
\begin{itemize}
\item {Grp. gram.:m.}
\end{itemize}
\begin{itemize}
\item {Utilização:Ant.}
\end{itemize}
O mesmo que \textunderscore sucessor\textunderscore . Cf. S. de Frias, \textunderscore Pombeiro\textunderscore , 199.
\section{Subchefe}
\begin{itemize}
\item {Grp. gram.:m.}
\end{itemize}
\begin{itemize}
\item {Proveniência:(De \textunderscore sub...\textunderscore  + \textunderscore chefe\textunderscore )}
\end{itemize}
Funccionário, immediato ao chefe ou substituto dêlle.
\section{Subcilíndrico}
\begin{itemize}
\item {Grp. gram.:adj.}
\end{itemize}
\begin{itemize}
\item {Utilização:Bot.}
\end{itemize}
\begin{itemize}
\item {Proveniência:(De \textunderscore sub...\textunderscore  + \textunderscore cilíndrico\textunderscore )}
\end{itemize}
Que se aproxima da fórma cilíndrica.
\section{Subcinerício}
\begin{itemize}
\item {Grp. gram.:adj.}
\end{itemize}
\begin{itemize}
\item {Proveniência:(Lat. \textunderscore subcinericius\textunderscore )}
\end{itemize}
Que está debaixo da cinza ou borralho.
Que se coze debaixo do borralho.
Relativo a cinza.
\section{Subclasse}
\begin{itemize}
\item {Grp. gram.:f.}
\end{itemize}
\begin{itemize}
\item {Proveniência:(De \textunderscore sub...\textunderscore  + \textunderscore classe\textunderscore )}
\end{itemize}
Divisão de classe.
\section{Subclavicular}
\begin{itemize}
\item {Grp. gram.:adj.}
\end{itemize}
\begin{itemize}
\item {Proveniência:(De \textunderscore sub...\textunderscore  + \textunderscore clavicular\textunderscore )}
\end{itemize}
Que está debaixo das clavículas.
\section{Subclávio}
\begin{itemize}
\item {Grp. gram.:adj.}
\end{itemize}
O mesmo que \textunderscore subclavicular\textunderscore .
\section{Subcoleitor}
\begin{itemize}
\item {Grp. gram.:m.}
\end{itemize}
\begin{itemize}
\item {Proveniência:(De \textunderscore sub...\textunderscore  + \textunderscore coleitor\textunderscore )}
\end{itemize}
Aquele que fazia as vezes de coleitor. Cf. Herculano, \textunderscore Hist. de Port.\textunderscore , III, 75.
\section{Subcolleitor}
\begin{itemize}
\item {Grp. gram.:m.}
\end{itemize}
\begin{itemize}
\item {Proveniência:(De \textunderscore sub...\textunderscore  + \textunderscore colleitor\textunderscore )}
\end{itemize}
Aquelle que fazia as vezes de colleitor. Cf. Herculano, \textunderscore Hist. de Port.\textunderscore , III, 75.
\section{Subcomissão}
\begin{itemize}
\item {Grp. gram.:f.}
\end{itemize}
\begin{itemize}
\item {Proveniência:(De \textunderscore sub...\textunderscore  + \textunderscore commissão\textunderscore )}
\end{itemize}
Cada uma das comissões, em que uma comissão se divide.
\section{Subcomissário}
\begin{itemize}
\item {Grp. gram.:m.}
\end{itemize}
\begin{itemize}
\item {Proveniência:(De \textunderscore sub...\textunderscore  + \textunderscore comissário\textunderscore )}
\end{itemize}
Funcionário, imediato ao comissário ou substituto dêle.
\section{Subcommissão}
\begin{itemize}
\item {Grp. gram.:f.}
\end{itemize}
\begin{itemize}
\item {Proveniência:(De \textunderscore sub...\textunderscore  + \textunderscore commissão\textunderscore )}
\end{itemize}
Cada uma das commissões, em que uma commissão se divide.
\section{Subcommissário}
\begin{itemize}
\item {Grp. gram.:m.}
\end{itemize}
\begin{itemize}
\item {Proveniência:(De \textunderscore sub...\textunderscore  + \textunderscore commissário\textunderscore )}
\end{itemize}
Funccionário, immediato ao commissário ou substituto dêlle.
\section{Subconjunctival}
\begin{itemize}
\item {Grp. gram.:adj.}
\end{itemize}
Que está debaixo da conjunctiva.
\section{Subcontrário}
\begin{itemize}
\item {Grp. gram.:adj.}
\end{itemize}
\begin{itemize}
\item {Proveniência:(De \textunderscore sub...\textunderscore  + \textunderscore contrário\textunderscore )}
\end{itemize}
Diz-se das proposições, que têm o mesmo sujeito e o mesmo attributo, mas affirmando uma o que outra nega.
\section{Subcordiforme}
\begin{itemize}
\item {Grp. gram.:adj.}
\end{itemize}
\begin{itemize}
\item {Utilização:Bot.}
\end{itemize}
\begin{itemize}
\item {Proveniência:(De \textunderscore sub...\textunderscore  + \textunderscore cordiforme\textunderscore )}
\end{itemize}
Que se aproxima da fórma de coração.
\section{Subcorrente}
\begin{itemize}
\item {Grp. gram.:f.}
\end{itemize}
\begin{itemize}
\item {Proveniência:(De \textunderscore sub...\textunderscore  + \textunderscore corrente\textunderscore )}
\end{itemize}
Corrente marítima, que passa debaixo de outra.
Corrente marítima.
\section{Subcostal}
\begin{itemize}
\item {Grp. gram.:adj.}
\end{itemize}
\begin{itemize}
\item {Utilização:Anat.}
\end{itemize}
\begin{itemize}
\item {Proveniência:(De \textunderscore sub...\textunderscore  + \textunderscore costal\textunderscore )}
\end{itemize}
Que está debaixo das costellas.
\section{Subcujo}
\begin{itemize}
\item {Grp. gram.:pron.}
\end{itemize}
\begin{itemize}
\item {Utilização:Des.}
\end{itemize}
Debaixo de cujo:«\textunderscore ...mas Fé, subcujo suave jugo sobmetereís o mundo...\textunderscore »\textunderscore Eufrosina\textunderscore , 13.
\section{Subcutâneo}
\begin{itemize}
\item {Grp. gram.:adj.}
\end{itemize}
\begin{itemize}
\item {Utilização:Anat.}
\end{itemize}
\begin{itemize}
\item {Proveniência:(De \textunderscore sub...\textunderscore  + \textunderscore cutâneo\textunderscore )}
\end{itemize}
Que está por baixo da cútis.
\section{Subcylíndrico}
\begin{itemize}
\item {Grp. gram.:adj.}
\end{itemize}
\begin{itemize}
\item {Utilização:Bot.}
\end{itemize}
\begin{itemize}
\item {Proveniência:(De \textunderscore sub...\textunderscore  + \textunderscore cylíndrico\textunderscore )}
\end{itemize}
Que se aproxima da fórma cylíndrica.
\section{Subdecano}
\begin{itemize}
\item {Grp. gram.:m.}
\end{itemize}
\begin{itemize}
\item {Proveniência:(De \textunderscore sub...\textunderscore  + \textunderscore decano\textunderscore )}
\end{itemize}
Segundo decano.
\section{Subdécuplo}
\begin{itemize}
\item {Grp. gram.:adj.}
\end{itemize}
\begin{itemize}
\item {Proveniência:(De \textunderscore sub...\textunderscore  + \textunderscore décuplo\textunderscore )}
\end{itemize}
Que de dez partes contém uma.
\section{Subdelegação}
\begin{itemize}
\item {Grp. gram.:f.}
\end{itemize}
Acto ou effeito de subdelegar.
Qualidade de subdelegado.
Repartição de subdelegado.
Delegação ou succursal de um estabelecimento.
(Do \textunderscore sub...\textunderscore  + \textunderscore delegação\textunderscore )
\section{Subdelegado}
\begin{itemize}
\item {Grp. gram.:m.}
\end{itemize}
Funccionário immediato ao delegado, ou substituto dêste.
\section{Subdelegante}
\begin{itemize}
\item {Grp. gram.:adj.}
\end{itemize}
Que subdelega.
\section{Subdelegar}
\begin{itemize}
\item {Grp. gram.:v. t.}
\end{itemize}
\begin{itemize}
\item {Proveniência:(De \textunderscore sub...\textunderscore  + \textunderscore delegar\textunderscore )}
\end{itemize}
Transmittir (um encargo) quem o tinha como delegado.
\section{Subdelegável}
\begin{itemize}
\item {Grp. gram.:adj.}
\end{itemize}
Que se póde subdelegar.
\section{Subdelírio}
\begin{itemize}
\item {Grp. gram.:m.}
\end{itemize}
\begin{itemize}
\item {Proveniência:(De \textunderscore sub...\textunderscore  + \textunderscore delírio\textunderscore )}
\end{itemize}
Delírio incompleto.
\section{Subderivado}
\begin{itemize}
\item {Grp. gram.:m.}
\end{itemize}
\begin{itemize}
\item {Utilização:Gram.}
\end{itemize}
\begin{itemize}
\item {Proveniência:(De \textunderscore sub...\textunderscore  + \textunderscore derivado\textunderscore )}
\end{itemize}
Palavra, derivada de outra, que é também um derivado: \textunderscore marinheiro\textunderscore  &gt; \textunderscore marinha\textunderscore  &gt; \textunderscore mar\textunderscore ; \textunderscore centralização\textunderscore  &gt; \textunderscore centralizar\textunderscore  &gt; \textunderscore central\textunderscore .
\section{Subdiaconato}
\begin{itemize}
\item {Grp. gram.:m.}
\end{itemize}
\begin{itemize}
\item {Proveniência:(Lat. \textunderscore subdiaconatus\textunderscore )}
\end{itemize}
Estado, dignidade ou ordens de subdiácono.
\section{Subdiaconiza}
\begin{itemize}
\item {Grp. gram.:f.}
\end{itemize}
\begin{itemize}
\item {Proveniência:(Lat. \textunderscore subdiaconissa\textunderscore )}
\end{itemize}
Mulhér do subdiácono, nos tempos antigos da Igreja.
\section{Subdiácono}
\begin{itemize}
\item {Grp. gram.:m.}
\end{itemize}
\begin{itemize}
\item {Proveniência:(Lat. \textunderscore subdiaconus\textunderscore )}
\end{itemize}
Clérigo, que tem a primeira ordem sagrada ou ordem immediatamente inferior á do diácono.
\section{Subdíptero}
\begin{itemize}
\item {Grp. gram.:adj.}
\end{itemize}
\begin{itemize}
\item {Utilização:Zool.}
\end{itemize}
\begin{itemize}
\item {Proveniência:(De \textunderscore sub...\textunderscore  + \textunderscore díptero\textunderscore )}
\end{itemize}
Diz-se dos insectos, cujos elytros são muito curtos e não cobrem as asas.
\section{Subdirecção}
\begin{itemize}
\item {Grp. gram.:f.}
\end{itemize}
\begin{itemize}
\item {Proveniência:(De \textunderscore sub...\textunderscore  + \textunderscore direcção\textunderscore )}
\end{itemize}
Cargo de subdirector.
Repartição, dirigida por um subdirector.
\section{Subdirector}
\begin{itemize}
\item {Grp. gram.:m.}
\end{itemize}
\begin{itemize}
\item {Proveniência:(De \textunderscore sub...\textunderscore  + \textunderscore director\textunderscore )}
\end{itemize}
Funccionário immediato ao director ou que o substitue.
\section{Subdirectora}
\begin{itemize}
\item {Grp. gram.:f.}
\end{itemize}
(Fem. de \textunderscore subdirector\textunderscore )
\section{Subdistinção}
\begin{itemize}
\item {Grp. gram.:f.}
\end{itemize}
\begin{itemize}
\item {Proveniência:(Lat. \textunderscore subdistinctio\textunderscore )}
\end{itemize}
Distinção de outra distinção.
\section{Subdistincção}
\begin{itemize}
\item {Grp. gram.:f.}
\end{itemize}
\begin{itemize}
\item {Proveniência:(Lat. \textunderscore subdistinctio\textunderscore )}
\end{itemize}
Distincção de outra distincção.
\section{Subdistinguir}
\begin{itemize}
\item {Grp. gram.:v. t.}
\end{itemize}
\begin{itemize}
\item {Proveniência:(Lat. \textunderscore subdistinguere\textunderscore )}
\end{itemize}
Fazer subdistincção de.
\section{Súbdito}
\begin{itemize}
\item {Grp. gram.:m.  e  adj.}
\end{itemize}
\begin{itemize}
\item {Proveniência:(Lat. \textunderscore subditus\textunderscore )}
\end{itemize}
O que está sujeito á vontade de outrem.
O que está dependente da jurisdição de um superior; vassallo.
\section{Subdividir}
\begin{itemize}
\item {Grp. gram.:v. t.}
\end{itemize}
\begin{itemize}
\item {Proveniência:(Lat. \textunderscore subdividere\textunderscore )}
\end{itemize}
Dividir novamente; fazer subdivisões de.
\section{Subdivisão}
\begin{itemize}
\item {Grp. gram.:f.}
\end{itemize}
\begin{itemize}
\item {Proveniência:(Lat. \textunderscore subdivisio\textunderscore )}
\end{itemize}
Acto ou effeito de subdividir.
\section{Subdivisionário}
\begin{itemize}
\item {Grp. gram.:adj.}
\end{itemize}
\begin{itemize}
\item {Proveniência:(De \textunderscore sub...\textunderscore  + \textunderscore divisionário\textunderscore )}
\end{itemize}
Relativo a subdivisão.
\section{Subdivisível}
\begin{itemize}
\item {Grp. gram.:adj.}
\end{itemize}
\begin{itemize}
\item {Proveniência:(De \textunderscore sub...\textunderscore  + \textunderscore diviso\textunderscore )}
\end{itemize}
Que se póde subdividir.
\section{Subdoloso}
\begin{itemize}
\item {Grp. gram.:adj.}
\end{itemize}
Astucioso; traiçoeiro. Cf. C. Lobo, \textunderscore Sát. de Juv.\textunderscore , I, 229.
(Cp. lat. \textunderscore subdolus\textunderscore  e \textunderscore subdolositas\textunderscore )
\section{Subdominante}
\begin{itemize}
\item {Grp. gram.:f.}
\end{itemize}
\begin{itemize}
\item {Utilização:Mús.}
\end{itemize}
\begin{itemize}
\item {Proveniência:(De \textunderscore sub...\textunderscore  + \textunderscore dominante\textunderscore )}
\end{itemize}
Quarto grau da escala diatónica.
\section{Subduplo}
\begin{itemize}
\item {Grp. gram.:adj.}
\end{itemize}
\begin{itemize}
\item {Proveniência:(Lat. \textunderscore subduplus\textunderscore )}
\end{itemize}
Diz-se de um pequeno número, que está duas vezes contido noutro ou que é metade de outro.
\section{Subdural}
\begin{itemize}
\item {Grp. gram.:adj.}
\end{itemize}
\begin{itemize}
\item {Utilização:Anat.}
\end{itemize}
Diz-se da região que fica sob a dura-máter.
\section{Subemphyteuse}
\begin{itemize}
\item {Grp. gram.:f.}
\end{itemize}
\begin{itemize}
\item {Proveniência:(De \textunderscore sub...\textunderscore  + \textunderscore emphyteuse\textunderscore )}
\end{itemize}
Acto ou contrato, com que um foreiro passa a outrem o domínio útil do respectivo prédio emphytêutico.
\section{Subemphyteuta}
\begin{itemize}
\item {Grp. gram.:m.  e  f.}
\end{itemize}
\begin{itemize}
\item {Proveniência:(De \textunderscore sub...\textunderscore  + \textunderscore emphyteuta\textunderscore )}
\end{itemize}
Pessôa, que adquiriu um prazo, por subemphyteuse.
\section{Subasta}
\begin{itemize}
\item {Grp. gram.:f.}
\end{itemize}
O mesmo que \textunderscore subastação\textunderscore .
\section{Subastação}
\begin{itemize}
\item {Grp. gram.:f.}
\end{itemize}
\begin{itemize}
\item {Proveniência:(Do lat. \textunderscore subhastatio\textunderscore )}
\end{itemize}
Acto ou efeito de subastar.
\section{Subastar}
\begin{itemize}
\item {Grp. gram.:v. t.}
\end{itemize}
\begin{itemize}
\item {Proveniência:(Lat. \textunderscore subhastare\textunderscore )}
\end{itemize}
Vender em almoéda.
\section{Subemphyteuticar}
\begin{itemize}
\item {Grp. gram.:v. t.}
\end{itemize}
\begin{itemize}
\item {Proveniência:(De \textunderscore sub...\textunderscore  + \textunderscore emphyteuticar\textunderscore )}
\end{itemize}
Transmittir por subemphyteuse; subemprazar.
\section{Subemphytêutico}
\begin{itemize}
\item {Grp. gram.:adj.}
\end{itemize}
\begin{itemize}
\item {Proveniência:(De \textunderscore sub...\textunderscore  + \textunderscore emphytêutico\textunderscore )}
\end{itemize}
Relativo á subemphyteuse.
\section{Subemprazamento}
\begin{itemize}
\item {Grp. gram.:m.}
\end{itemize}
Acto ou effeito de subemprazar.
\section{Subemprazar}
\begin{itemize}
\item {Grp. gram.:v. t.}
\end{itemize}
\begin{itemize}
\item {Proveniência:(De \textunderscore sub...\textunderscore  + \textunderscore emprazar\textunderscore )}
\end{itemize}
Transmittir a outrem por subemphyteuse; subemphyteuticar.
\section{Subenfiteuse}
\begin{itemize}
\item {Grp. gram.:f.}
\end{itemize}
\begin{itemize}
\item {Proveniência:(De \textunderscore sub...\textunderscore  + \textunderscore emnfiteuse\textunderscore )}
\end{itemize}
Acto ou contrato, com que um foreiro passa a outrem o domínio útil do respectivo prédio enfitêutico.
\section{Subenfiteuta}
\begin{itemize}
\item {Grp. gram.:m.  e  f.}
\end{itemize}
\begin{itemize}
\item {Proveniência:(De \textunderscore sub...\textunderscore  + \textunderscore enfiteuta\textunderscore )}
\end{itemize}
Pessôa, que adquiriu um prazo, por subenfiteuse.
\section{Subenfiteuticar}
\begin{itemize}
\item {Grp. gram.:v. t.}
\end{itemize}
\begin{itemize}
\item {Proveniência:(De \textunderscore sub...\textunderscore  + \textunderscore enfiteuticar\textunderscore )}
\end{itemize}
Transmitir por subenfiteuse; subemprazar.
\section{Subenfitêutico}
\begin{itemize}
\item {Grp. gram.:adj.}
\end{itemize}
\begin{itemize}
\item {Proveniência:(De \textunderscore sub...\textunderscore  + \textunderscore enfitêutico\textunderscore )}
\end{itemize}
Relativo á subenfiteuse.
\section{Subente}
\begin{itemize}
\item {Grp. gram.:f.}
\end{itemize}
\begin{itemize}
\item {Utilização:Ant.}
\end{itemize}
\begin{itemize}
\item {Proveniência:(Do lat. \textunderscore sub\textunderscore  + \textunderscore ens\textunderscore , \textunderscore entis\textunderscore )}
\end{itemize}
Peça de coiro, sotoposta á altibaixa, nas tapeçarias antigas.
\section{Subentender}
\begin{itemize}
\item {Grp. gram.:v. t.}
\end{itemize}
\begin{itemize}
\item {Proveniência:(De \textunderscore sub...\textunderscore  + \textunderscore entender\textunderscore )}
\end{itemize}
Entender ou perceber (o que não estava exposto ou bem explicado).
Suppor.
\section{Subentendido}
\begin{itemize}
\item {Grp. gram.:m.}
\end{itemize}
\begin{itemize}
\item {Proveniência:(De \textunderscore subentender\textunderscore )}
\end{itemize}
Aquillo que está no pensamento, mas que se não exprime.
\section{Subenvasamento}
\begin{itemize}
\item {Grp. gram.:m.}
\end{itemize}
\begin{itemize}
\item {Proveniência:(De \textunderscore sub...\textunderscore  + \textunderscore envasamento\textunderscore )}
\end{itemize}
Corpo inferior ao envasamento, em Architectura.
\section{Subepático}
\begin{itemize}
\item {Grp. gram.:adj.}
\end{itemize}
\begin{itemize}
\item {Utilização:Anat.}
\end{itemize}
Que está debaixo do fígado.
\section{Suberato}
\begin{itemize}
\item {Grp. gram.:m.}
\end{itemize}
\begin{itemize}
\item {Proveniência:(Do lat. \textunderscore suber\textunderscore )}
\end{itemize}
Sal, resultante da combinação do ácido subérico com uma base.
\section{Subêrbo}
\textunderscore m.\textunderscore  e \textunderscore adj.\textunderscore  (e der.)
(V. \textunderscore soberbo\textunderscore , etc.)
\section{Subérico}
\begin{itemize}
\item {Grp. gram.:adj.}
\end{itemize}
\begin{itemize}
\item {Proveniência:(Do lat. \textunderscore suber\textunderscore )}
\end{itemize}
Diz-se de um ácido, obtido pela acção do ácido azótico sôbre a cortiça.
\section{Suberina}
\begin{itemize}
\item {Grp. gram.:f.}
\end{itemize}
\begin{itemize}
\item {Proveniência:(Do lat. \textunderscore suber\textunderscore )}
\end{itemize}
Matéria, extrahida da cortiça.
\section{Suberização}
\begin{itemize}
\item {Grp. gram.:f.}
\end{itemize}
\begin{itemize}
\item {Utilização:Neol.}
\end{itemize}
\begin{itemize}
\item {Proveniência:(Do lat. \textunderscore suber\textunderscore )}
\end{itemize}
Formação de cortiça no sobreiro.
\section{Suberoso}
\begin{itemize}
\item {Grp. gram.:adj.}
\end{itemize}
\begin{itemize}
\item {Proveniência:(Do lat. \textunderscore suber\textunderscore )}
\end{itemize}
Que tem a consistência da cortiça.
\section{Subescapular}
\begin{itemize}
\item {Grp. gram.:adj.}
\end{itemize}
\begin{itemize}
\item {Utilização:Anat.}
\end{itemize}
Situado abaixo das espáduas.
\section{Subespécie}
\begin{itemize}
\item {Grp. gram.:f.}
\end{itemize}
Divisão de espécie.
\section{Subespinhal}
\begin{itemize}
\item {Grp. gram.:adj.}
\end{itemize}
\begin{itemize}
\item {Utilização:Anat.}
\end{itemize}
\begin{itemize}
\item {Proveniência:(De \textunderscore sub...\textunderscore  + \textunderscore espinha\textunderscore )}
\end{itemize}
Situado debaixo da espinha dorsal.
\section{Subespinhoso}
\begin{itemize}
\item {Grp. gram.:m.  e  adj.}
\end{itemize}
\begin{itemize}
\item {Utilização:Anat.}
\end{itemize}
\begin{itemize}
\item {Utilização:ant.}
\end{itemize}
O segundo dos nove músculos do braço.
(Cp. \textunderscore subespinhal\textunderscore )
\section{Subestabelecer}
\textunderscore v. t.\textunderscore  (e der.)
O mesmo que \textunderscore substabelecer\textunderscore , etc.
\section{Subface}
\begin{itemize}
\item {Grp. gram.:f.}
\end{itemize}
\begin{itemize}
\item {Proveniência:(De \textunderscore sub...\textunderscore  + \textunderscore face\textunderscore )}
\end{itemize}
A parte inferior da cabeça de um insecto.
\section{Subfeudatário}
\begin{itemize}
\item {Grp. gram.:m.  e  adj.}
\end{itemize}
\begin{itemize}
\item {Proveniência:(De \textunderscore sub...\textunderscore  + \textunderscore feudatário\textunderscore )}
\end{itemize}
O que recebeu encargos de um vassallo feudatário.
\section{Subfeudo}
\begin{itemize}
\item {Grp. gram.:m.}
\end{itemize}
\begin{itemize}
\item {Proveniência:(De \textunderscore sub...\textunderscore  + \textunderscore feudo\textunderscore )}
\end{itemize}
Feudo, dependente de um vassallo feudatário.
\section{Subfixa}
\begin{itemize}
\item {Grp. gram.:f.}
\end{itemize}
\begin{itemize}
\item {Utilização:P. us.}
\end{itemize}
O mesmo que \textunderscore suffixo\textunderscore .
\section{Subfoliáceo}
\begin{itemize}
\item {Grp. gram.:adj.}
\end{itemize}
\begin{itemize}
\item {Proveniência:(De \textunderscore sub...\textunderscore  + \textunderscore foliáceo\textunderscore )}
\end{itemize}
Semelhante a uma fôlha.
\section{Subfretar}
\begin{itemize}
\item {Grp. gram.:v. t.}
\end{itemize}
\begin{itemize}
\item {Proveniência:(De \textunderscore sub...\textunderscore  + \textunderscore fretar\textunderscore )}
\end{itemize}
Fretar (embarcação já fretada).
\section{Subgemíparo}
\begin{itemize}
\item {Grp. gram.:adj.}
\end{itemize}
\begin{itemize}
\item {Proveniência:(De \textunderscore sub...\textunderscore  + \textunderscore gemmíparo\textunderscore )}
\end{itemize}
Que se reproduz por meio de gomos.
\section{Subgemmíparo}
\begin{itemize}
\item {Grp. gram.:adj.}
\end{itemize}
\begin{itemize}
\item {Proveniência:(De \textunderscore sub...\textunderscore  + \textunderscore gemmíparo\textunderscore )}
\end{itemize}
Que se reproduz por meio de gomos.
\section{Subgênero}
\begin{itemize}
\item {Grp. gram.:m.}
\end{itemize}
\begin{itemize}
\item {Utilização:Hist. Nat.}
\end{itemize}
\begin{itemize}
\item {Proveniência:(De \textunderscore sub...\textunderscore  + \textunderscore gênero\textunderscore )}
\end{itemize}
Divisão immediata de um gênero, a qual se admitte naquelles que têm muitas espécies, quando estas se distinguem por alguns caracteres, não bastantes para uma separação genérica.
\section{Subglabro}
\begin{itemize}
\item {Grp. gram.:adj.}
\end{itemize}
\begin{itemize}
\item {Proveniência:(De \textunderscore sub...\textunderscore  + \textunderscore glabro\textunderscore )}
\end{itemize}
Quási glabro.
\section{Subglobuloso}
\begin{itemize}
\item {Grp. gram.:adj.}
\end{itemize}
\begin{itemize}
\item {Utilização:Bot.}
\end{itemize}
\begin{itemize}
\item {Proveniência:(De \textunderscore sub...\textunderscore  + \textunderscore globuloso\textunderscore )}
\end{itemize}
Quási globuloso.
\section{Subgrave}
\begin{itemize}
\item {Grp. gram.:adj.}
\end{itemize}
\begin{itemize}
\item {Proveniência:(De \textunderscore sub...\textunderscore  + \textunderscore grave\textunderscore )}
\end{itemize}
Que está abaixo de grave, na música.
\section{Sub-grupo}
\begin{itemize}
\item {Grp. gram.:m.}
\end{itemize}
Cada um dos grupos, em que se subdivide um grupo.
\section{Subhasta}
\begin{itemize}
\item {Grp. gram.:f.}
\end{itemize}
O mesmo que \textunderscore subhastação\textunderscore .
\section{Subhastação}
\begin{itemize}
\item {Grp. gram.:f.}
\end{itemize}
\begin{itemize}
\item {Proveniência:(Do lat. \textunderscore subhastatio\textunderscore )}
\end{itemize}
Acto ou effeito de subhastar.
\section{Subhastar}
\begin{itemize}
\item {Grp. gram.:v. t.}
\end{itemize}
\begin{itemize}
\item {Proveniência:(Lat. \textunderscore subhastare\textunderscore )}
\end{itemize}
Vender em almoéda.
\section{Subhepático}
\begin{itemize}
\item {Grp. gram.:adj.}
\end{itemize}
\begin{itemize}
\item {Utilização:Anat.}
\end{itemize}
Que está debaixo do fígado.
\section{Subhydrochlorato}
\begin{itemize}
\item {Grp. gram.:m.}
\end{itemize}
\begin{itemize}
\item {Utilização:Chím.}
\end{itemize}
\begin{itemize}
\item {Proveniência:(De \textunderscore sub...\textunderscore  + \textunderscore hydrochlorato\textunderscore )}
\end{itemize}
Hydrochlorato, com excesso de base.
\section{Subicterícia}
\begin{itemize}
\item {Grp. gram.:f.}
\end{itemize}
\begin{itemize}
\item {Utilização:Med.}
\end{itemize}
Icterícia ligeira.
\section{Subida}
\begin{itemize}
\item {Grp. gram.:f.}
\end{itemize}
Acto ou effeito de subir; encosta; declive, ladeira.
\section{Subidamente}
\begin{itemize}
\item {Grp. gram.:adv.}
\end{itemize}
\begin{itemize}
\item {Proveniência:(De \textunderscore subido\textunderscore )}
\end{itemize}
Em alto grau.
\section{Subideira}
\begin{itemize}
\item {Grp. gram.:f.}
\end{itemize}
\begin{itemize}
\item {Utilização:Prov.}
\end{itemize}
\begin{itemize}
\item {Utilização:beir.}
\end{itemize}
\begin{itemize}
\item {Utilização:Gír.}
\end{itemize}
\begin{itemize}
\item {Proveniência:(De \textunderscore subir\textunderscore )}
\end{itemize}
Pequena ave trepadora, semelhante á carriça, (\textunderscore certhia brachydactyla\textunderscore , Lin.).
Escada.
\section{Subideiro}
\begin{itemize}
\item {Grp. gram.:m.}
\end{itemize}
\begin{itemize}
\item {Utilização:Ant.}
\end{itemize}
\begin{itemize}
\item {Proveniência:(De \textunderscore subida\textunderscore )}
\end{itemize}
Atalho íngreme.
Escada estreita.
\section{Subido}
\begin{itemize}
\item {Grp. gram.:adj.}
\end{itemize}
\begin{itemize}
\item {Utilização:Fig.}
\end{itemize}
Alto; sublime; excessivo pomposo.
\section{Subidroclorato}
\begin{itemize}
\item {Grp. gram.:m.}
\end{itemize}
\begin{itemize}
\item {Utilização:Chím.}
\end{itemize}
\begin{itemize}
\item {Proveniência:(De \textunderscore sub...\textunderscore  + \textunderscore hidroclorato\textunderscore )}
\end{itemize}
Hidroclorato, com excesso de base.
\section{Subimento}
\begin{itemize}
\item {Grp. gram.:m.}
\end{itemize}
\begin{itemize}
\item {Proveniência:(De \textunderscore subir\textunderscore )}
\end{itemize}
Subida.
Aumento; demasia.
\section{Subimbrical}
\begin{itemize}
\item {Grp. gram.:adj.}
\end{itemize}
\begin{itemize}
\item {Utilização:Bot.}
\end{itemize}
\begin{itemize}
\item {Proveniência:(De \textunderscore sub...\textunderscore  + \textunderscore imbricar\textunderscore )}
\end{itemize}
Diz-se das fôlhas, que abraçam o caule, applicando-se uma contra a outra pela face inferior, depois de torcerem o seu pecíolo.
\section{Subinflamação}
\begin{itemize}
\item {Grp. gram.:f.}
\end{itemize}
\begin{itemize}
\item {Proveniência:(De \textunderscore sub...\textunderscore  + \textunderscore inflamação\textunderscore )}
\end{itemize}
Ligeira inflamação.
\section{Subinflamatório}
\begin{itemize}
\item {Grp. gram.:adj.}
\end{itemize}
\begin{itemize}
\item {Proveniência:(De \textunderscore sub...\textunderscore  + \textunderscore imflamatório\textunderscore )}
\end{itemize}
Ligeiramente inflamatório.
\section{Subinflammação}
\begin{itemize}
\item {Grp. gram.:f.}
\end{itemize}
\begin{itemize}
\item {Proveniência:(De \textunderscore sub...\textunderscore  + \textunderscore inflammação\textunderscore )}
\end{itemize}
Ligeira inflammação.
\section{Subinflammatório}
\begin{itemize}
\item {Grp. gram.:adj.}
\end{itemize}
\begin{itemize}
\item {Proveniência:(De \textunderscore sub...\textunderscore  + \textunderscore imflammatório\textunderscore )}
\end{itemize}
Ligeiramente inflammatório.
\section{Subinte}
\begin{itemize}
\item {Grp. gram.:adj.}
\end{itemize}
\begin{itemize}
\item {Proveniência:(De \textunderscore subir\textunderscore )}
\end{itemize}
Que sobe, que ascende.
\section{Subintendência}
\begin{itemize}
\item {Grp. gram.:f.}
\end{itemize}
\begin{itemize}
\item {Proveniência:(De \textunderscore sub...\textunderscore  + \textunderscore intendência\textunderscore )}
\end{itemize}
Cargo ou repartição de subintendente.
\section{Subintendente}
\begin{itemize}
\item {Grp. gram.:m.}
\end{itemize}
\begin{itemize}
\item {Proveniência:(De \textunderscore sub...\textunderscore  + \textunderscore intendente\textunderscore )}
\end{itemize}
Funccionário, immediato ao intendente, ou que o substitue.
\section{Subintitular}
\begin{itemize}
\item {Grp. gram.:v. t.}
\end{itemize}
\begin{itemize}
\item {Proveniência:(De \textunderscore sub...\textunderscore  + \textunderscore intitular\textunderscore )}
\end{itemize}
Pôr subtítulo a.
\section{Subintrante}
\begin{itemize}
\item {Grp. gram.:adj.}
\end{itemize}
\begin{itemize}
\item {Utilização:Med.}
\end{itemize}
\begin{itemize}
\item {Proveniência:(Lat. \textunderscore subintrans\textunderscore )}
\end{itemize}
Diz-se dos accessos de febre intermittente, quando são tão próximos que, quando um começa, ainda o precedente não tem terminado.
\section{Subir}
\begin{itemize}
\item {Grp. gram.:v. i.}
\end{itemize}
\begin{itemize}
\item {Grp. gram.:V. t.}
\end{itemize}
\begin{itemize}
\item {Utilização:Fig.}
\end{itemize}
\begin{itemize}
\item {Proveniência:(Lat. \textunderscore subire\textunderscore )}
\end{itemize}
Ir para cima.
Trepar.
Elevar-se.
Chegar a certa altura.
Attingir.
Aumentar.
Percorrer, andando para cima: \textunderscore subir uma ladeira\textunderscore .
Trepar por: \textunderscore subir uma escada\textunderscore .
Puxar para cima:«\textunderscore onde me sobes, Musa?\textunderscore »Filinto.
Engrandecer.
Elevar a posição social de.
\section{Subitamente}
\begin{itemize}
\item {Grp. gram.:adv.}
\end{itemize}
De modo súbito; de repente: inopinadamente; a súbitas.
\section{Subitaneamente}
\begin{itemize}
\item {Grp. gram.:adv.}
\end{itemize}
De modo subitâneo; subitamente.
\section{Subitaneidade}
\begin{itemize}
\item {Grp. gram.:f.}
\end{itemize}
Qualidade de subitâneo.
\section{Subitâneo}
\begin{itemize}
\item {Grp. gram.:adj.}
\end{itemize}
\begin{itemize}
\item {Proveniência:(Lat. \textunderscore subitaneus\textunderscore )}
\end{itemize}
O mesmo que \textunderscore súbito\textunderscore .
\section{Súbitas, a}
\begin{itemize}
\item {Grp. gram.:loc. adv.}
\end{itemize}
\begin{itemize}
\item {Proveniência:(De \textunderscore súbito\textunderscore )}
\end{itemize}
Subitamente; de repente. Cf. Camillo, \textunderscore Enjeitada\textunderscore , 31.
\section{Súbito}
\begin{itemize}
\item {Grp. gram.:adj.}
\end{itemize}
\begin{itemize}
\item {Grp. gram.:M.}
\end{itemize}
\begin{itemize}
\item {Grp. gram.:Adv.}
\end{itemize}
\begin{itemize}
\item {Proveniência:(Lat. \textunderscore subitus\textunderscore )}
\end{itemize}
Que apparece ou succede sem sêr previsto.
Repentino; inesperado.
Prompto.
Successo repentino; repente.
Subitamente.
\section{Subjacente}
\begin{itemize}
\item {Grp. gram.:adj.}
\end{itemize}
\begin{itemize}
\item {Proveniência:(De \textunderscore sub...\textunderscore  + \textunderscore jacente\textunderscore )}
\end{itemize}
Que jaz ou está por baixo.
\section{Subjecção}
\begin{itemize}
\item {Grp. gram.:f.}
\end{itemize}
\begin{itemize}
\item {Utilização:Rhet.}
\end{itemize}
\begin{itemize}
\item {Proveniência:(Do lat. \textunderscore subjectio\textunderscore )}
\end{itemize}
Figura de pensamento, que consisto em interrogar o adversário e suppor a resposta, ou prever o que elle responderia, e dar logo a réplica.
\section{Subjectivação}
\begin{itemize}
\item {Grp. gram.:f.}
\end{itemize}
Acto ou effeito de subjectivar.
\section{Subjectivamente}
\begin{itemize}
\item {Grp. gram.:adv.}
\end{itemize}
De modo subjectivo.
\section{Subjectivar}
\begin{itemize}
\item {Grp. gram.:v. t.}
\end{itemize}
Tornar ou considerar subjectivo.
\section{Subjectividade}
\begin{itemize}
\item {Grp. gram.:f.}
\end{itemize}
Qualidade do que é subjectivo.
\section{Subjectivismo}
\begin{itemize}
\item {Grp. gram.:m.}
\end{itemize}
\begin{itemize}
\item {Proveniência:(De \textunderscore subjectivo\textunderscore )}
\end{itemize}
Tendência viciosa para a subjectividade.
Theoria de que a necessidade de certas fórmas lógicas deriva só da contribuição do nosso espírito.
\section{Subjectivo}
\begin{itemize}
\item {Grp. gram.:adj.}
\end{itemize}
\begin{itemize}
\item {Grp. gram.:M.}
\end{itemize}
\begin{itemize}
\item {Proveniência:(Lat. \textunderscore subjectivus\textunderscore )}
\end{itemize}
Relativo a sujeito.
Que existe no sujeito.
Que se passa exclusivamente no espírito de um indivíduo.
Aquillo que é subjectivo.
\section{Subjugação}
\begin{itemize}
\item {Grp. gram.:f.}
\end{itemize}
Acto ou effeito de subjugar.
\section{Subjugador}
\begin{itemize}
\item {Grp. gram.:m.  e  adj.}
\end{itemize}
\begin{itemize}
\item {Proveniência:(Lat. \textunderscore subjugator\textunderscore )}
\end{itemize}
O que subjuga.
\section{Subjugante}
\begin{itemize}
\item {Grp. gram.:adj.}
\end{itemize}
Que subjuga.
Que domina; que se impõe.
\section{Subjugar}
\begin{itemize}
\item {Grp. gram.:v. t.}
\end{itemize}
\begin{itemize}
\item {Utilização:Fig.}
\end{itemize}
\begin{itemize}
\item {Proveniência:(Lat. \textunderscore subjugare\textunderscore )}
\end{itemize}
Pôr debaixo do jugo, jungir.
Sujeitar.
Conquistar.
Refrear, conter.
Domesticar; dominar.
\section{Subjunção}
\begin{itemize}
\item {Grp. gram.:f.}
\end{itemize}
\begin{itemize}
\item {Proveniência:(De \textunderscore sub...\textunderscore  + \textunderscore junção\textunderscore )}
\end{itemize}
Junção imediata.
\section{Subjuncção}
\begin{itemize}
\item {Grp. gram.:f.}
\end{itemize}
\begin{itemize}
\item {Proveniência:(De \textunderscore sub...\textunderscore  + \textunderscore juncção\textunderscore )}
\end{itemize}
Juncção immediata.
\section{Subjunctivo}
\begin{itemize}
\item {Grp. gram.:adj.}
\end{itemize}
\begin{itemize}
\item {Utilização:Gram.}
\end{itemize}
\begin{itemize}
\item {Grp. gram.:M.}
\end{itemize}
\begin{itemize}
\item {Proveniência:(Lat. \textunderscore subjunctivus\textunderscore )}
\end{itemize}
Subordinado.
Relativo ao modo conjunctivo dos verbos.
O modo conjunctivo dos verbos.
\section{Subjuntivo}
\begin{itemize}
\item {Grp. gram.:adj.}
\end{itemize}
\begin{itemize}
\item {Utilização:Gram.}
\end{itemize}
\begin{itemize}
\item {Grp. gram.:M.}
\end{itemize}
\begin{itemize}
\item {Proveniência:(Lat. \textunderscore subjunctivus\textunderscore )}
\end{itemize}
Subordinado.
Relativo ao modo conjuntivo dos verbos.
O modo conjuntivo dos verbos.
\section{Sublacustre}
\begin{itemize}
\item {Grp. gram.:adj.}
\end{itemize}
\begin{itemize}
\item {Proveniência:(De \textunderscore sub...\textunderscore  + \textunderscore lacustre\textunderscore )}
\end{itemize}
Que está debaixo das águas de um lago.
\section{Sublapsário}
\begin{itemize}
\item {Grp. gram.:m.}
\end{itemize}
\begin{itemize}
\item {Proveniência:(De \textunderscore sub...\textunderscore  + \textunderscore lapso\textunderscore )}
\end{itemize}
Calvinista, para quem os homens não predestinados estão condemnados inevitavelmente, em consequência da quéda de Adão e não obstante o baptismo.
\section{Sublenhoso}
\begin{itemize}
\item {Grp. gram.:adj.}
\end{itemize}
\begin{itemize}
\item {Utilização:Bot.}
\end{itemize}
\begin{itemize}
\item {Proveniência:(De \textunderscore sub...\textunderscore  + \textunderscore lenhoso\textunderscore )}
\end{itemize}
Diz se das plantas, cujo tronco é lenhoso na base e herbáceo no ápice.
\section{Sublevação}
\begin{itemize}
\item {Grp. gram.:f.}
\end{itemize}
\begin{itemize}
\item {Proveniência:(Lat. \textunderscore sublevatio\textunderscore )}
\end{itemize}
Acto ou effeito de sublevar.
Revolta; rebellião.
\section{Sublevador}
\begin{itemize}
\item {Grp. gram.:m.  e  adj.}
\end{itemize}
O que subleva.
\section{Sublevar}
\begin{itemize}
\item {Grp. gram.:v. t.}
\end{itemize}
\begin{itemize}
\item {Proveniência:(Lat. \textunderscore sublevare\textunderscore )}
\end{itemize}
Levantar; amotinar; revoltar.
\section{Sublício}
\begin{itemize}
\item {Grp. gram.:adj.}
\end{itemize}
\begin{itemize}
\item {Utilização:Des.}
\end{itemize}
\begin{itemize}
\item {Proveniência:(Lat. \textunderscore sublicius\textunderscore )}
\end{itemize}
Feito de madeira. Cf. Filinto, XIV, 206.
\section{Subligáculo}
\begin{itemize}
\item {Grp. gram.:m.}
\end{itemize}
\begin{itemize}
\item {Proveniência:(Lat. \textunderscore subligaculum\textunderscore )}
\end{itemize}
O mesmo que \textunderscore subligar\textunderscore .
\section{Subligar}
\begin{itemize}
\item {Grp. gram.:m.}
\end{itemize}
\begin{itemize}
\item {Utilização:Ant.}
\end{itemize}
\begin{itemize}
\item {Proveniência:(Lat. \textunderscore subligar\textunderscore )}
\end{itemize}
Bragas, ou cobertura para as nádegas e partes pudendas.
\section{Sublimação}
\begin{itemize}
\item {Grp. gram.:f.}
\end{itemize}
\begin{itemize}
\item {Proveniência:(Do lat. \textunderscore sublimatio\textunderscore )}
\end{itemize}
Acto ou effeito de sublimar.
\section{Sublimado}
\begin{itemize}
\item {Grp. gram.:adj.}
\end{itemize}
\begin{itemize}
\item {Grp. gram.:M.}
\end{itemize}
\begin{itemize}
\item {Proveniência:(De \textunderscore sublimar\textunderscore )}
\end{itemize}
Elevado á maior altura.
Engrandecido.
Volatilizado chimicamente.
Substância sublimada.
\section{Sublimar}
\begin{itemize}
\item {Grp. gram.:v. t.}
\end{itemize}
\begin{itemize}
\item {Proveniência:(Lat. \textunderscore sublimare\textunderscore )}
\end{itemize}
Erguer muito alto.
Exaltar.
Tornar sublime.
Engrandecer.
Elevar á maior perfeição.
Volatilizar chimicamente pelo calor e reduzir ao estado sólido pelo resfriamento.
Purificar.
\section{Sublimatório}
\begin{itemize}
\item {Grp. gram.:adj.}
\end{itemize}
\begin{itemize}
\item {Grp. gram.:M.}
\end{itemize}
\begin{itemize}
\item {Proveniência:(De \textunderscore sublimar\textunderscore )}
\end{itemize}
Relativo á sublimação.
Vaso, em que se recolhem os productos das sublimações chímicas.
\section{Sublimável}
\begin{itemize}
\item {Grp. gram.:adj.}
\end{itemize}
Que se póde sublimar.
\section{Sublime}
\begin{itemize}
\item {Grp. gram.:adj.}
\end{itemize}
\begin{itemize}
\item {Grp. gram.:M.}
\end{itemize}
\begin{itemize}
\item {Proveniência:(Lat. \textunderscore sublimis\textunderscore )}
\end{itemize}
Muito alto; excelso.
Que fica acima de nós.
Perfeitíssimo.
Majestoso; grandioso.
Poderoso.
Encantador; esplêndido.
O mais alto grau de perfeição.
Aquillo que é sublime.
\section{Sublimeão}
\begin{itemize}
\item {Grp. gram.:adj.}
\end{itemize}
\begin{itemize}
\item {Utilização:Ant.}
\end{itemize}
Muito sublime; muito eminente. Cf. S. R. Viterbo, \textunderscore Elucidário\textunderscore .
\section{Sublimemente}
\begin{itemize}
\item {Grp. gram.:adv.}
\end{itemize}
De modo sublime.
\section{Sublimidade}
\begin{itemize}
\item {Grp. gram.:f.}
\end{itemize}
\begin{itemize}
\item {Proveniência:(Do lat. \textunderscore sublimitas\textunderscore )}
\end{itemize}
Qualidade do que é sublime.
Grande altura.
Perfeição.
Excellência.
A maior grandeza.
\section{Sublinear}
\begin{itemize}
\item {Grp. gram.:adj.}
\end{itemize}
\begin{itemize}
\item {Proveniência:(De \textunderscore sub...\textunderscore  + \textunderscore linear\textunderscore )}
\end{itemize}
Que se escreve por entre linhas ou por baixo de linhas.
\section{Sublingual}
\begin{itemize}
\item {Grp. gram.:adj.}
\end{itemize}
\begin{itemize}
\item {Utilização:Anat.}
\end{itemize}
\begin{itemize}
\item {Proveniência:(De \textunderscore sub...\textunderscore  + \textunderscore lingual\textunderscore )}
\end{itemize}
Que está debaixo da língua.
\section{Sublinha}
\begin{itemize}
\item {Grp. gram.:f.}
\end{itemize}
\begin{itemize}
\item {Proveniência:(De \textunderscore sub...\textunderscore  + \textunderscore linha\textunderscore )}
\end{itemize}
Linha, traçada por debaixo de palavra ou palavras.
\section{Sublinhar}
\begin{itemize}
\item {Grp. gram.:v. t.}
\end{itemize}
\begin{itemize}
\item {Utilização:Fig.}
\end{itemize}
\begin{itemize}
\item {Proveniência:(De \textunderscore sublinha\textunderscore )}
\end{itemize}
Traçar uma linha ou linhas por baixo de.
Tornar sensível, accentuar bem: \textunderscore sublinhar uma affirmação\textunderscore .
\section{Sublobulado}
\begin{itemize}
\item {Grp. gram.:adj.}
\end{itemize}
\begin{itemize}
\item {Proveniência:(De \textunderscore sub...\textunderscore  + \textunderscore lobulado\textunderscore )}
\end{itemize}
Dividido em lóbulos.
\section{Sublocação}
\begin{itemize}
\item {Grp. gram.:f.}
\end{itemize}
Acto ou effeito de sublocar.
\section{Sublocador}
\begin{itemize}
\item {Grp. gram.:m.}
\end{itemize}
Aquelle que subloca.
\section{Sublocar}
\begin{itemize}
\item {Grp. gram.:v. t.}
\end{itemize}
\begin{itemize}
\item {Proveniência:(Do lat. \textunderscore sub\textunderscore  + \textunderscore locare\textunderscore )}
\end{itemize}
Subarrendar.
Transmittir, alugando (o que se tinha tomado por aluguér): \textunderscore sublocar uma loja\textunderscore .
\section{Sublocatário}
\begin{itemize}
\item {Grp. gram.:adj.}
\end{itemize}
\begin{itemize}
\item {Proveniência:(De \textunderscore sublocar\textunderscore )}
\end{itemize}
Aquelle que recebe por sublocação.
\section{Sublunar}
\begin{itemize}
\item {Grp. gram.:adj.}
\end{itemize}
\begin{itemize}
\item {Proveniência:(De \textunderscore sub...\textunderscore  + \textunderscore lunar\textunderscore )}
\end{itemize}
Que está abaixo da Lua ou entre a Terra e a Lua.
\section{Subluxação}
\textunderscore f. Med.\textunderscore 
Luxação incompleta.
\section{Submarinho}
\begin{itemize}
\item {Grp. gram.:adj.}
\end{itemize}
O mesmo que \textunderscore submarino\textunderscore . Cf. Castilho, \textunderscore Metam.\textunderscore , 214.
\section{Submarino}
\begin{itemize}
\item {Grp. gram.:adj.}
\end{itemize}
\begin{itemize}
\item {Grp. gram.:M.}
\end{itemize}
\begin{itemize}
\item {Proveniência:(De \textunderscore sub...\textunderscore  + \textunderscore marino\textunderscore )}
\end{itemize}
Que está debaixo das águas do mar.
Immergido no mar.
Destinado a navegar, por baixo da superfície da água.
Barco submarino.
\section{Submassiço}
\begin{itemize}
\item {Grp. gram.:m.}
\end{itemize}
\begin{itemize}
\item {Utilização:Med.}
\end{itemize}
Som, obtido pela percussão de uma parte do corpo, e caracterizado pela elevação do tom e deminuição da sonoridade.
(Cp. fr. \textunderscore submatité\textunderscore )
\section{Submaxilar}
\begin{itemize}
\item {Grp. gram.:adj.}
\end{itemize}
\begin{itemize}
\item {Utilização:Anat.}
\end{itemize}
\begin{itemize}
\item {Proveniência:(De \textunderscore sub...\textunderscore  + \textunderscore maxilar\textunderscore )}
\end{itemize}
Que está debaixo das maxilas.
\section{Submaxillar}
\begin{itemize}
\item {Grp. gram.:adj.}
\end{itemize}
\begin{itemize}
\item {Utilização:Anat.}
\end{itemize}
\begin{itemize}
\item {Proveniência:(De \textunderscore sub...\textunderscore  + \textunderscore maxillar\textunderscore )}
\end{itemize}
Que está debaixo das maxillas.
\section{Submental}
\begin{itemize}
\item {Grp. gram.:adj.}
\end{itemize}
\begin{itemize}
\item {Proveniência:(De \textunderscore sub...\textunderscore  + \textunderscore mento\textunderscore )}
\end{itemize}
Que está debaixo do mento ou queixo.
\section{Submergido}
\begin{itemize}
\item {Grp. gram.:adj.}
\end{itemize}
\begin{itemize}
\item {Proveniência:(De \textunderscore submergir\textunderscore )}
\end{itemize}
Que se afundou.
Que está debaixo de água ou coberto de água.
\section{Submergir}
\begin{itemize}
\item {Grp. gram.:v. t.}
\end{itemize}
\begin{itemize}
\item {Utilização:Fig.}
\end{itemize}
\begin{itemize}
\item {Proveniência:(Lat. \textunderscore submergere\textunderscore )}
\end{itemize}
Meter debaixo de água.
Immergir.
Afundar.
Inundar: \textunderscore o temporal submergiu o campo\textunderscore .
Envolver; fazer desapparecer.
Destruir.
\section{Submergível}
\begin{itemize}
\item {Grp. gram.:adj.}
\end{itemize}
Que se póde submergir.
\section{Submersão}
\begin{itemize}
\item {Grp. gram.:f.}
\end{itemize}
\begin{itemize}
\item {Proveniência:(Do lat. \textunderscore submersio\textunderscore )}
\end{itemize}
Acto ou effeito de submergir.
Abatimento do casco de uma cavalgadura, em resultado de pancada.
\section{Submersível}
\begin{itemize}
\item {Grp. gram.:adj.}
\end{itemize}
\begin{itemize}
\item {Proveniência:(De \textunderscore submerso\textunderscore )}
\end{itemize}
O mesmo que \textunderscore submergível\textunderscore .
E diz-se da planta que se submerge na água, depois da florescência.
\section{Submerso}
\begin{itemize}
\item {Grp. gram.:adj.}
\end{itemize}
\begin{itemize}
\item {Proveniência:(Lat. \textunderscore submersus\textunderscore )}
\end{itemize}
O mesmo que \textunderscore submergido\textunderscore .
\section{Submeter}
\begin{itemize}
\item {Grp. gram.:v. t.}
\end{itemize}
\begin{itemize}
\item {Proveniência:(Do lat. \textunderscore submittere\textunderscore )}
\end{itemize}
Pôr debaixo de.
Tornar dependente.
Subjugar; obrigar, sujeitar: \textunderscore submeter os rebeldes\textunderscore .
\section{Submetimento}
\begin{itemize}
\item {Grp. gram.:m.}
\end{itemize}
Acto ou effeito de submeter.
\section{Submilhér}
\begin{itemize}
\item {Grp. gram.:m.}
\end{itemize}
O mesmo que \textunderscore sumilhér\textunderscore .
Bandido, aventureiro:«\textunderscore despido por dois daquelles sub-milheres...\textunderscore »F. Manuel, \textunderscore Apólogos\textunderscore .
\section{Subministração}
\begin{itemize}
\item {Grp. gram.:f.}
\end{itemize}
\begin{itemize}
\item {Proveniência:(Do lat. \textunderscore subministratio\textunderscore )}
\end{itemize}
Acto ou effeito de subministrar.
\section{Subministrador}
\begin{itemize}
\item {Grp. gram.:m.  e  adj.}
\end{itemize}
\begin{itemize}
\item {Proveniência:(Do lat. \textunderscore subministrator\textunderscore )}
\end{itemize}
O que subministra.
\section{Subministrar}
\begin{itemize}
\item {Grp. gram.:v. t.}
\end{itemize}
\begin{itemize}
\item {Proveniência:(Lat. \textunderscore subministrare\textunderscore )}
\end{itemize}
Prover do necessário.
Ministrar; fornecer.
\section{Submissão}
\begin{itemize}
\item {Grp. gram.:f.}
\end{itemize}
\begin{itemize}
\item {Proveniência:(Lat. \textunderscore submissio\textunderscore )}
\end{itemize}
Acto ou effeito de submeter.
Humildade.
Sujeição.
Adhesão espontânea da vontade de alguém á de outrem.
\section{Submissivo}
\begin{itemize}
\item {Grp. gram.:adj.}
\end{itemize}
Que mostra submissão:«\textunderscore submissivo respondeu...\textunderscore »\textunderscore Viriato Trág.\textunderscore , XII, 70.
\section{Submisso}
\begin{itemize}
\item {Grp. gram.:adj.}
\end{itemize}
\begin{itemize}
\item {Proveniência:(Lat. \textunderscore submissus\textunderscore )}
\end{itemize}
Obediente.
Que está em posição ou lugar inferior; inferior.
Dócil; respeitoso: \textunderscore é um rapaz submisso\textunderscore .
\section{Submúltiplex}
\begin{itemize}
\item {Grp. gram.:f.}
\end{itemize}
\begin{itemize}
\item {Utilização:Arith.}
\end{itemize}
\begin{itemize}
\item {Proveniência:(Do lat. \textunderscore sub\textunderscore  + \textunderscore multiplex\textunderscore )}
\end{itemize}
Diz-se a razão, quando o antecedente é contido no consequente, algumas vezes exactamente.
\section{Submúltiplo}
\begin{itemize}
\item {Grp. gram.:adj.}
\end{itemize}
\begin{itemize}
\item {Grp. gram.:M.}
\end{itemize}
\begin{itemize}
\item {Proveniência:(De \textunderscore sub...\textunderscore  + \textunderscore múltiplo\textunderscore )}
\end{itemize}
Que se contém exactamente noutro um certo número de vezes: \textunderscore decímetro é submúltiplo de metro\textunderscore .
Número submúltiplo.
\section{Subnasal}
\begin{itemize}
\item {Grp. gram.:adj.}
\end{itemize}
\begin{itemize}
\item {Proveniência:(De \textunderscore sub...\textunderscore  + \textunderscore nasal\textunderscore )}
\end{itemize}
Situado por baixo do nariz.
\section{Subnegar}
\textunderscore v. t.\textunderscore  (e der.)
O mesmo que \textunderscore sonegar\textunderscore , etc. Cf. \textunderscore Peregrinação\textunderscore , XCIX.
\section{Subnitrato}
\begin{itemize}
\item {Grp. gram.:m.}
\end{itemize}
\begin{itemize}
\item {Utilização:Chím.}
\end{itemize}
\begin{itemize}
\item {Proveniência:(De \textunderscore sub...\textunderscore  + \textunderscore nitrato\textunderscore )}
\end{itemize}
Nitrato, que encerra duas, três ou seis vezes tanta quantidade de base como o neutro.
\section{Suboccipital}
\begin{itemize}
\item {Grp. gram.:adj.}
\end{itemize}
\begin{itemize}
\item {Utilização:Anat.}
\end{itemize}
\begin{itemize}
\item {Proveniência:(De \textunderscore sub...\textunderscore  + \textunderscore occipital\textunderscore )}
\end{itemize}
Situado abaixo do occipício.
\section{Subocipital}
\begin{itemize}
\item {Grp. gram.:adj.}
\end{itemize}
\begin{itemize}
\item {Utilização:Anat.}
\end{itemize}
\begin{itemize}
\item {Proveniência:(De \textunderscore sub...\textunderscore  + \textunderscore ocipital\textunderscore )}
\end{itemize}
Situado abaixo do ocipício.
\section{Subocular}
\begin{itemize}
\item {Grp. gram.:adj.}
\end{itemize}
\begin{itemize}
\item {Utilização:Anat.}
\end{itemize}
\begin{itemize}
\item {Proveniência:(De \textunderscore sub...\textunderscore  + \textunderscore ocular\textunderscore )}
\end{itemize}
Situado abaixo dos olhos.
\section{Suboleato}
\begin{itemize}
\item {Grp. gram.:m.}
\end{itemize}
\begin{itemize}
\item {Utilização:Chím.}
\end{itemize}
\begin{itemize}
\item {Proveniência:(De \textunderscore sub...\textunderscore  + \textunderscore oleato\textunderscore )}
\end{itemize}
Oleato, com excesso de base.
\section{Suborbicular}
\begin{itemize}
\item {Grp. gram.:adj.}
\end{itemize}
\begin{itemize}
\item {Utilização:Anat.}
\end{itemize}
\begin{itemize}
\item {Proveniência:(De \textunderscore sub...\textunderscore  + \textunderscore orbicular\textunderscore )}
\end{itemize}
Situado abaixo da órbita dos olhos.
\section{Suborbitário}
\begin{itemize}
\item {Grp. gram.:adj.}
\end{itemize}
O mesmo que \textunderscore suborbicular\textunderscore .
\section{Sub-ordem}
\begin{itemize}
\item {Grp. gram.:f.}
\end{itemize}
Divisão de uma ordem, nas classificações vegetaes e animaes.
\section{Subordem}
\begin{itemize}
\item {Grp. gram.:f.}
\end{itemize}
Divisão de uma ordem, nas classificações vegetaes e animaes.
\section{Subordinação}
\begin{itemize}
\item {Grp. gram.:f.}
\end{itemize}
\begin{itemize}
\item {Utilização:Gram.}
\end{itemize}
\begin{itemize}
\item {Proveniência:(Lat. \textunderscore subordinatio\textunderscore )}
\end{itemize}
Acto ou effeito de subordinar.
Dependência, em que alguém está, de outrem.
Dependência, em que uma coisa está, de outra.
Obediência.
Dependência, em que uma palavra está, de outra, na mesma proposição.
\section{Subordinada}
\begin{itemize}
\item {Grp. gram.:f.}
\end{itemize}
\begin{itemize}
\item {Utilização:Gram.}
\end{itemize}
Oração subordinada.
(Fem. de \textunderscore subordinado\textunderscore )
\section{Subordinadamente}
\begin{itemize}
\item {Grp. gram.:adv.}
\end{itemize}
De modo subordinado; com subordinação; com obediência.
\section{Subordinado}
\begin{itemize}
\item {Grp. gram.:adj.}
\end{itemize}
\begin{itemize}
\item {Utilização:Gram.}
\end{itemize}
\begin{itemize}
\item {Grp. gram.:M.}
\end{itemize}
\begin{itemize}
\item {Proveniência:(De \textunderscore subordinar\textunderscore )}
\end{itemize}
Dependente; subalterno; inferior.
Secundário.
Que por si não faz sentido completo ou independente: \textunderscore proposição subordinada\textunderscore .
Aquelle que está ás ordens de outrem; subalterno.
Serviçal, criado.
\section{Subordinador}
\begin{itemize}
\item {Grp. gram.:m.  e  adj.}
\end{itemize}
O que subordina.
\section{Subordinante}
\begin{itemize}
\item {Grp. gram.:adj.}
\end{itemize}
\begin{itemize}
\item {Grp. gram.:F.  e  adj.}
\end{itemize}
\begin{itemize}
\item {Utilização:Gram.}
\end{itemize}
\begin{itemize}
\item {Proveniência:(De \textunderscore subordinar\textunderscore )}
\end{itemize}
Que subordina.
Diz-se da oração principal, em relação a outra ou outras, dentro do mesmo periodo.
\section{Subordinar}
\begin{itemize}
\item {Grp. gram.:v. t.}
\end{itemize}
\begin{itemize}
\item {Proveniência:(Lat. \textunderscore sub\textunderscore  + \textunderscore ordinare\textunderscore )}
\end{itemize}
Tornar dependente; submeter.
\section{Subordinativo}
\begin{itemize}
\item {Grp. gram.:adj.}
\end{itemize}
Que estabelece subordinação.
\section{Subordinável}
\begin{itemize}
\item {Grp. gram.:adj.}
\end{itemize}
Que se póde subordinar.
\section{Subornação}
\begin{itemize}
\item {Grp. gram.:f.}
\end{itemize}
\begin{itemize}
\item {Proveniência:(Do lat. \textunderscore subornatio\textunderscore )}
\end{itemize}
O mesmo que \textunderscore subôrno\textunderscore .
\section{Subornador}
\begin{itemize}
\item {Grp. gram.:m.  e  adj.}
\end{itemize}
\begin{itemize}
\item {Proveniência:(Do lat. \textunderscore subornator\textunderscore )}
\end{itemize}
O que suborna.
\section{Subornamento}
\begin{itemize}
\item {Grp. gram.:m.}
\end{itemize}
O mesmo que \textunderscore subôrno\textunderscore .
\section{Subornar}
\begin{itemize}
\item {Grp. gram.:v. t.}
\end{itemize}
\begin{itemize}
\item {Proveniência:(Lat. \textunderscore subornare\textunderscore )}
\end{itemize}
Attrahir, enganando.
Alliciar para mau fim.
Peitar.
Dar dinheiro ou quaesquer valores a, para conseguir alguma coisa opposta á justiça, á moral ou ao dever: \textunderscore subornar um juiz\textunderscore .
\section{Subornável}
\begin{itemize}
\item {Grp. gram.:adj.}
\end{itemize}
Que se póde subornar.
\section{Subôrno}
\begin{itemize}
\item {Grp. gram.:m.}
\end{itemize}
Acto ou effeito de subornar.
\section{Suborralho}
\begin{itemize}
\item {Grp. gram.:m.}
\end{itemize}
\begin{itemize}
\item {Proveniência:(De \textunderscore sub...\textunderscore  + \textunderscore borralho\textunderscore )}
\end{itemize}
A parte inferior do borralho: \textunderscore a cozinheira assou uma pêra no suborralho\textunderscore .
\section{Suboxalato}
\begin{itemize}
\item {fónica:csa}
\end{itemize}
\begin{itemize}
\item {Grp. gram.:m.}
\end{itemize}
\begin{itemize}
\item {Utilização:Chím.}
\end{itemize}
\begin{itemize}
\item {Proveniência:(De \textunderscore sub...\textunderscore  + \textunderscore oxalato\textunderscore )}
\end{itemize}
Oxalato, com excesso de base.
\section{Subóxido}
\begin{itemize}
\item {fónica:csi}
\end{itemize}
\begin{itemize}
\item {Grp. gram.:m.}
\end{itemize}
\begin{itemize}
\item {Utilização:Chím.}
\end{itemize}
\begin{itemize}
\item {Proveniência:(De \textunderscore sub...\textunderscore  + \textunderscore óxido\textunderscore )}
\end{itemize}
Óxido, que não encerra o oxigênio bastante para desempenhar o papel de base e combinar-se com os ácidos.
\section{Subóxydo}
\begin{itemize}
\item {Grp. gram.:m.}
\end{itemize}
\begin{itemize}
\item {Utilização:Chím.}
\end{itemize}
\begin{itemize}
\item {Proveniência:(De \textunderscore sub...\textunderscore  + \textunderscore óxydo\textunderscore )}
\end{itemize}
Óxydo, que não encerra o oxygênio bastante para desempenhar o papel de base e combinar-se com os ácidos.
\section{Subpericárdico}
\begin{itemize}
\item {Grp. gram.:adj.}
\end{itemize}
\begin{itemize}
\item {Utilização:Anat.}
\end{itemize}
Situado debaixo do pericardio.
\section{Subperpendicular}
\begin{itemize}
\item {Grp. gram.:adj.}
\end{itemize}
\begin{itemize}
\item {Utilização:Geom.}
\end{itemize}
\begin{itemize}
\item {Proveniência:(De \textunderscore sub...\textunderscore  + \textunderscore perpendicular\textunderscore )}
\end{itemize}
Que está abaixo da perpendicular.
\section{Subpolar}
\begin{itemize}
\item {Grp. gram.:adj.}
\end{itemize}
\begin{itemize}
\item {Proveniência:(De \textunderscore sub...\textunderscore  + \textunderscore polar\textunderscore )}
\end{itemize}
Que está debaixo do pólo.
\section{Subpor}
\begin{itemize}
\item {Grp. gram.:v. t.}
\end{itemize}
\begin{itemize}
\item {Proveniência:(Do lat. \textunderscore sub\textunderscore  + \textunderscore ponere\textunderscore )}
\end{itemize}
Pôr debaixo; sotopor.
\section{Subposto}
\begin{itemize}
\item {fónica:pôs}
\end{itemize}
\begin{itemize}
\item {Grp. gram.:adj.}
\end{itemize}
\begin{itemize}
\item {Proveniência:(Do lat. \textunderscore sub\textunderscore  + \textunderscore positus\textunderscore )}
\end{itemize}
Collocado debaixo; sotoposto.
\section{Subprefeito}
\begin{itemize}
\item {Grp. gram.:m.}
\end{itemize}
\begin{itemize}
\item {Proveniência:(Lat. \textunderscore subpraefectus\textunderscore )}
\end{itemize}
Funccionário immediato a prefeito ou que o substitue.
\section{Subprefeitura}
\begin{itemize}
\item {Grp. gram.:f.}
\end{itemize}
\begin{itemize}
\item {Proveniência:(Lat. \textunderscore subpraefectura\textunderscore )}
\end{itemize}
Cargo ou dignidade de subprefeito.
\section{Subpromotor}
\begin{itemize}
\item {Grp. gram.:m.}
\end{itemize}
\begin{itemize}
\item {Proveniência:(De \textunderscore sub...\textunderscore  + \textunderscore promotor\textunderscore )}
\end{itemize}
Substituto de promotor.
Official que, na canonização dos santos, faz as vezes de promotor.
\section{Subragi}
\begin{itemize}
\item {Grp. gram.:f.}
\end{itemize}
Planta rhamnácea do Brasil.
\section{Subrasil}
\begin{itemize}
\item {Grp. gram.:m.}
\end{itemize}
\begin{itemize}
\item {Utilização:Bras}
\end{itemize}
Árvore silvestre, provavelmente o mesmo que \textunderscore subragi\textunderscore .
\section{Subregano}
\begin{itemize}
\item {fónica:re}
\end{itemize}
\begin{itemize}
\item {Grp. gram.:m.}
\end{itemize}
\begin{itemize}
\item {Utilização:Ant.}
\end{itemize}
Prédio rústico, sôbre que pesava o foro de um leitão.
(Por \textunderscore subrecano\textunderscore , de \textunderscore sub...\textunderscore  + \textunderscore reco\textunderscore )
\section{Subrepção}
\begin{itemize}
\item {fónica:ré}
\end{itemize}
\begin{itemize}
\item {Grp. gram.:f.}
\end{itemize}
\begin{itemize}
\item {Proveniência:(Do lat. \textunderscore subreptio\textunderscore )}
\end{itemize}
Graça, conseguida por meio de uma falsa exposição.
Fraude ou surpresa, feita a um superior.
Subtracção.
\section{Subrepticiamente}
\begin{itemize}
\item {fónica:re}
\end{itemize}
\begin{itemize}
\item {Grp. gram.:adv.}
\end{itemize}
De modo subreptício; fraudulentamente.
\section{Subreptício}
\begin{itemize}
\item {fónica:re}
\end{itemize}
\begin{itemize}
\item {Grp. gram.:adj.}
\end{itemize}
\begin{itemize}
\item {Proveniência:(Lat. \textunderscore subrepticius\textunderscore )}
\end{itemize}
Conseguido por subrepção.
Conseguido illicitamente.
Fraudulento.
\section{Subrício}
\begin{itemize}
\item {fónica:ri}
\end{itemize}
\begin{itemize}
\item {Grp. gram.:m.}
\end{itemize}
\begin{itemize}
\item {Utilização:Ant.}
\end{itemize}
\begin{itemize}
\item {Proveniência:(De \textunderscore sub...\textunderscore  + \textunderscore rico\textunderscore ?)}
\end{itemize}
Fidalgo, de categoria immediatamente inferior á de rico-homem.
\section{Subrogação}
\begin{itemize}
\item {fónica:ro}
\end{itemize}
\begin{itemize}
\item {Grp. gram.:f.}
\end{itemize}
\begin{itemize}
\item {Proveniência:(Do lat. \textunderscore subrogatio\textunderscore )}
\end{itemize}
Acto ou effeito de subrogar.
Substituição judicial de uma pessôa ou coisa por outra.
\section{Subrogado}
\begin{itemize}
\item {fónica:ro}
\end{itemize}
\begin{itemize}
\item {Grp. gram.:adj.}
\end{itemize}
\begin{itemize}
\item {Utilização:Jur.}
\end{itemize}
Investido na qualidade ou direitos de outrem.
Transmittido por successão.
\section{Subrogador}
\begin{itemize}
\item {fónica:ro}
\end{itemize}
\begin{itemize}
\item {Grp. gram.:m.  e  adj.}
\end{itemize}
\begin{itemize}
\item {Proveniência:(De \textunderscore subrogar\textunderscore )}
\end{itemize}
O que subroga.
\section{Subrogante}
\begin{itemize}
\item {fónica:ro}
\end{itemize}
\begin{itemize}
\item {Grp. gram.:adj.}
\end{itemize}
\begin{itemize}
\item {Proveniência:(Lat. \textunderscore subrogans\textunderscore )}
\end{itemize}
Que subroga.
\section{Subrogar}
\begin{itemize}
\item {fónica:ro}
\end{itemize}
\begin{itemize}
\item {Grp. gram.:v. t.}
\end{itemize}
\begin{itemize}
\item {Proveniência:(Lat. \textunderscore subrogare\textunderscore )}
\end{itemize}
Collocar em lugar de alguém; substituir.
Transferir direito ou encargo a.
\section{Subrogatório}
\begin{itemize}
\item {fónica:ro}
\end{itemize}
\begin{itemize}
\item {Grp. gram.:adj.}
\end{itemize}
O mesmo que \textunderscore subrogante\textunderscore .
\section{Sub-rolho}
\begin{itemize}
\item {fónica:rô}
\end{itemize}
\begin{itemize}
\item {Grp. gram.:m.}
\end{itemize}
\begin{itemize}
\item {Utilização:Prov.}
\end{itemize}
\begin{itemize}
\item {Utilização:trasm.}
\end{itemize}
Calor abafadiço, em dias anuveados.
Bochorno.
\section{Subrostrado}
\begin{itemize}
\item {fónica:ros}
\end{itemize}
\begin{itemize}
\item {Grp. gram.:adj.}
\end{itemize}
\begin{itemize}
\item {Proveniência:(De \textunderscore sub...\textunderscore  + \textunderscore rostro\textunderscore )}
\end{itemize}
Que tem a apparência de pequeno bico.
\section{Subscrever}
\begin{itemize}
\item {Grp. gram.:v. t.}
\end{itemize}
\begin{itemize}
\item {Utilização:Ext.}
\end{itemize}
\begin{itemize}
\item {Grp. gram.:V. i.}
\end{itemize}
\begin{itemize}
\item {Grp. gram.:V. p.}
\end{itemize}
\begin{itemize}
\item {Proveniência:(Do lat. \textunderscore subscribere\textunderscore )}
\end{itemize}
Escrever por baixo de.
Assinar.
Approvar ou acceitar (qualquer conceito, opinião, systema, etc.).
Conformar-se.
Annuir.
Acceder.
Obrigar-se a contribuir com certa quantia para dado fim; tomar parte numa subscripção.
Assinar para um periódico ou para uma obra.
Assinar-se.
\section{Subscrição}
\begin{itemize}
\item {Grp. gram.:f.}
\end{itemize}
\begin{itemize}
\item {Proveniência:(Do lat. \textunderscore subscriptio\textunderscore )}
\end{itemize}
Acto ou effeito de subscrever.
Compromisso de concorrer com uma quantia para certos fins.
\section{Subscritor}
\begin{itemize}
\item {Grp. gram.:m.  e  adj.}
\end{itemize}
\begin{itemize}
\item {Proveniência:(Lat. \textunderscore subscriptor\textunderscore )}
\end{itemize}
O que subscreve.
\section{Subscripção}
\begin{itemize}
\item {Grp. gram.:f.}
\end{itemize}
\begin{itemize}
\item {Proveniência:(Do lat. \textunderscore subscriptio\textunderscore )}
\end{itemize}
Acto ou effeito de subscrever.
Compromisso de concorrer com uma quantia para certos fins.
\section{Subscriptor}
\begin{itemize}
\item {Grp. gram.:m.  e  adj.}
\end{itemize}
\begin{itemize}
\item {Proveniência:(Lat. \textunderscore subscriptor\textunderscore )}
\end{itemize}
O que subscreve.
\section{Subsecção}
\begin{itemize}
\item {Grp. gram.:f.}
\end{itemize}
\begin{itemize}
\item {Proveniência:(De \textunderscore sub...\textunderscore  + \textunderscore secção\textunderscore )}
\end{itemize}
Divisão de secção.
\section{Subsecivo}
\begin{itemize}
\item {Grp. gram.:adj.}
\end{itemize}
\begin{itemize}
\item {Proveniência:(Lat. \textunderscore subsecivus\textunderscore )}
\end{itemize}
Que se corta ou se separa por sêr de mais.
Que sobeja; secundário.
\section{Subsecretário}
\begin{itemize}
\item {Grp. gram.:m.}
\end{itemize}
\begin{itemize}
\item {Proveniência:(De \textunderscore sub...\textunderscore  + \textunderscore secretário\textunderscore )}
\end{itemize}
O mesmo que [[vice-secretário|secretário]].
Em Espanha, funccionário superior de um ministério, immediato ao ministro.
\section{Subsecutivamente}
\begin{itemize}
\item {Grp. gram.:adv.}
\end{itemize}
De modo subsecutivo; consecutivamente.
\section{Subsecutivo}
\begin{itemize}
\item {Grp. gram.:adj.}
\end{itemize}
\begin{itemize}
\item {Proveniência:(Do lat. \textunderscore subsecutus\textunderscore )}
\end{itemize}
O mesmo que \textunderscore consecutivo\textunderscore .
\section{Subseguir}
\begin{itemize}
\item {Grp. gram.:v. t.}
\end{itemize}
\begin{itemize}
\item {Utilização:P. us.}
\end{itemize}
\begin{itemize}
\item {Proveniência:(Do lat. \textunderscore subsequi\textunderscore )}
\end{itemize}
Ir após de.
Seguir immediatamente.
\section{Subsentido}
\begin{itemize}
\item {Grp. gram.:m.}
\end{itemize}
\begin{itemize}
\item {Proveniência:(De \textunderscore sub...\textunderscore  + \textunderscore sentido\textunderscore )}
\end{itemize}
Segundo sentido, ideia reservada.
\section{Subsequência}
\begin{itemize}
\item {fónica:cu-en}
\end{itemize}
\begin{itemize}
\item {Grp. gram.:f.}
\end{itemize}
Qualidade do que é subsequente; seguimento; continuação.
\section{Subsequente}
\begin{itemize}
\item {fónica:cu-en}
\end{itemize}
\begin{itemize}
\item {Grp. gram.:adj.}
\end{itemize}
\begin{itemize}
\item {Proveniência:(Lat. \textunderscore subsequens\textunderscore )}
\end{itemize}
Que subsegue; seguinte; immediato.
Ulterior.
\section{Subsequentemente}
\begin{itemize}
\item {fónica:cu-en}
\end{itemize}
\begin{itemize}
\item {Grp. gram.:adv.}
\end{itemize}
De modo subsequente.
\section{Subserviência}
\begin{itemize}
\item {Grp. gram.:f.}
\end{itemize}
Qualidade do que é subserviente.
Bajulação; servilismo.
\section{Subserviente}
\begin{itemize}
\item {Grp. gram.:adj.}
\end{itemize}
\begin{itemize}
\item {Proveniência:(Lat. \textunderscore subserviens\textunderscore )}
\end{itemize}
Que serve ás ordens de outrem.
Servil.
Muito condescendente.
Aquelle que é amouco.
\section{Subséssil}
\begin{itemize}
\item {Grp. gram.:adj.}
\end{itemize}
\begin{itemize}
\item {Utilização:Bot.}
\end{itemize}
\begin{itemize}
\item {Proveniência:(De \textunderscore sub...\textunderscore  + \textunderscore séssil\textunderscore )}
\end{itemize}
Quási séssil.
\section{Subsidiado}
\begin{itemize}
\item {Grp. gram.:adj.}
\end{itemize}
\begin{itemize}
\item {Grp. gram.:M.}
\end{itemize}
\begin{itemize}
\item {Proveniência:(De \textunderscore subsidiar\textunderscore )}
\end{itemize}
Que recebe subsídio.
Que vive de algum subsídio ou auxílio fixado.
Que se realiza por meio de subsídio ou subscripção.
Aquelle que recebe subsídio do Estado ou de alguém, para cursar aulas ou para outro effeito.
\section{Subsidiar}
\begin{itemize}
\item {Grp. gram.:v. t.}
\end{itemize}
\begin{itemize}
\item {Proveniência:(Lat. \textunderscore subsidiari\textunderscore )}
\end{itemize}
Dar subsídio a.
Auxiliar.
Soccorrer.
Constribuir com qualquer subsídio para.
\section{Subsidiariamente}
\begin{itemize}
\item {Grp. gram.:adv.}
\end{itemize}
De modo subsidiário.
Como subsídio.
De refôrço; com auxílio.
Secundariamente.
\section{Subsidiário}
\begin{itemize}
\item {Grp. gram.:adj.}
\end{itemize}
\begin{itemize}
\item {Proveniência:(Lat. \textunderscore subsidiarius\textunderscore )}
\end{itemize}
Que subsidia.
Relativo a subsídio.
Que fortalece.
Que vem em apoio ou refôrço do que se allegou.
\section{Subsídio}
\begin{itemize}
\item {Grp. gram.:m.}
\end{itemize}
\begin{itemize}
\item {Proveniência:(Lat. \textunderscore subsidium\textunderscore )}
\end{itemize}
Auxílio.
Benefício.
Soccorro.
Quantia, com que se subscreve para uma obra de beneficência.
Quantia, que o Estado ou outra corporação concede para obras de interesse público; aquillo que serve de subsídio.
\section{Subsignano}
\begin{itemize}
\item {Grp. gram.:adj.}
\end{itemize}
\begin{itemize}
\item {Grp. gram.:M.}
\end{itemize}
\begin{itemize}
\item {Proveniência:(Lat. \textunderscore subsignanus\textunderscore )}
\end{itemize}
Dizia-se, entre os Romanos, do soldado destinado a reforçar o centro do exército.
O soldado, que militava sob bandeira particular, e não sob a bandeira da águia romana como os legionários.
\section{Subsiles}
\begin{itemize}
\item {Grp. gram.:m. pl.}
\end{itemize}
\begin{itemize}
\item {Proveniência:(Lat. \textunderscore subsilles\textunderscore )}
\end{itemize}
Lâminas, com figuras humanas, usadas nos sacrifícios antigos, e das quaes parece que se serviam os mágicos para despertar amor.
\section{Subsilles}
\begin{itemize}
\item {Grp. gram.:m. pl.}
\end{itemize}
\begin{itemize}
\item {Proveniência:(Lat. \textunderscore subsilles\textunderscore )}
\end{itemize}
Lâminas, com figuras humanas, usadas nos sacrifícios antigos, e das quaes parece que se serviam os mágicos para despertar amor.
\section{Subsino}
\begin{itemize}
\item {Grp. gram.:m.}
\end{itemize}
\begin{itemize}
\item {Utilização:Des.}
\end{itemize}
\begin{itemize}
\item {Proveniência:(De \textunderscore sub...\textunderscore  + \textunderscore sino\textunderscore )}
\end{itemize}
Pequena igreja ou paróchia, sujeita a outra maiór.
\section{Subsinuoso}
\begin{itemize}
\item {Grp. gram.:adj.}
\end{itemize}
\begin{itemize}
\item {Proveniência:(De \textunderscore sub...\textunderscore  + \textunderscore sinuoso\textunderscore )}
\end{itemize}
Um tanto sinuoso.
\section{Subsistência}
\begin{itemize}
\item {Grp. gram.:f.}
\end{itemize}
\begin{itemize}
\item {Proveniência:(Lat. \textunderscore subsistentia\textunderscore )}
\end{itemize}
Estado ou qualidade do que é subsistente.
Estabilidade.
Conjunto das coisa, necessárias para a sustentação da vida; sustento: \textunderscore já ganha para a subsistência da família\textunderscore .
\section{Subsistente}
\begin{itemize}
\item {Grp. gram.:adj.}
\end{itemize}
\begin{itemize}
\item {Proveniência:(Lat. \textunderscore subsistens\textunderscore )}
\end{itemize}
Que subsiste; que continúa a existir.
\section{Subsistir}
\begin{itemize}
\item {Grp. gram.:v. i.}
\end{itemize}
\begin{itemize}
\item {Proveniência:(Lat. \textunderscore subsistere\textunderscore )}
\end{itemize}
Na accepção primitiva, sustar-se ou parar.
Persistir.
Existir.
Sêr.
Continuar a sêr; permanecer: \textunderscore ainda hoje subsistem as razões que dantes se adduziam\textunderscore .
\section{Subsolano}
\begin{itemize}
\item {Grp. gram.:m.}
\end{itemize}
\begin{itemize}
\item {Proveniência:(Lat. \textunderscore subsolanus\textunderscore )}
\end{itemize}
Vento de Levante.
\section{Subsolo}
\begin{itemize}
\item {Grp. gram.:m.}
\end{itemize}
\begin{itemize}
\item {Proveniência:(De \textunderscore sub...\textunderscore  + \textunderscore solo\textunderscore )}
\end{itemize}
Camada de solo, immediatamente inferior á que se vê ou se póde lavrar.
\section{Substabelecer}
\begin{itemize}
\item {Grp. gram.:v. t.}
\end{itemize}
\begin{itemize}
\item {Proveniência:(De \textunderscore sub...\textunderscore  + \textunderscore estabelecer\textunderscore )}
\end{itemize}
Pôr em lugar de outrem ou de outra coisa.
Nomear como substituto.
Transferir para outrem, (encargo ou procuração que se recebeu).
Subrogar.
\section{Substabelecimento}
\begin{itemize}
\item {Grp. gram.:m.}
\end{itemize}
Acto ou effeito de substabelecer.
\section{Substância}
\begin{itemize}
\item {Grp. gram.:f.}
\end{itemize}
\begin{itemize}
\item {Grp. gram.:Loc. adv.}
\end{itemize}
\begin{itemize}
\item {Proveniência:(Lat. \textunderscore substantia\textunderscore )}
\end{itemize}
Aquillo que subsiste por si.
Essência.
Natureza de uma coisa.
Matéria, de que é formado um corpo.
O que há do nutritivo ou suculento numa coisa: \textunderscore a substância da carne\textunderscore ; \textunderscore um caldo de substância\textunderscore .
O que é indispensável para a nutrição.
O que há de essencial em qualquer coisa.
Fôrça, vigor.
Súmmula, sýnthese.
\textunderscore Em substância\textunderscore , em summa; sem entrar em minúcias.
\section{Substanciado}
\begin{itemize}
\item {Grp. gram.:adj.}
\end{itemize}
\begin{itemize}
\item {Proveniência:(De \textunderscore substanciar\textunderscore )}
\end{itemize}
De que se extrahiu a substância; resumido, synthetizado.
\section{Substancial}
\begin{itemize}
\item {Grp. gram.:adj.}
\end{itemize}
\begin{itemize}
\item {Grp. gram.:M.}
\end{itemize}
\begin{itemize}
\item {Proveniência:(Lat. \textunderscore substantialis\textunderscore )}
\end{itemize}
Relativo a substância.
Que tem substância ou é substancioso.
Nutritivo, alimenticio.
Fundamental.
Essencial.
Abundante.
Que encerra muitos ensinamentos ou esclarecimentos: \textunderscore um tratado substancial\textunderscore .
Aquillo que é essencial ou fundamental; substância.
\section{Substancialidade}
\begin{itemize}
\item {Grp. gram.:f.}
\end{itemize}
\begin{itemize}
\item {Proveniência:(Do lat. \textunderscore substantialitas\textunderscore )}
\end{itemize}
Qualidade do que é substancial.
\section{Substancialismo}
\begin{itemize}
\item {Grp. gram.:m.}
\end{itemize}
\begin{itemize}
\item {Proveniência:(De \textunderscore substancial\textunderscore )}
\end{itemize}
Systema philosóphico, que admitte a realidade substancial, e é por isso opposto ao idealismo.
\section{Substancializar}
\begin{itemize}
\item {Grp. gram.:v. t.}
\end{itemize}
\begin{itemize}
\item {Proveniência:(De \textunderscore substancial\textunderscore )}
\end{itemize}
Converter em substância; considerar como substância.
\section{Substancialmente}
\begin{itemize}
\item {Grp. gram.:adv.}
\end{itemize}
De modo substancial.
Relativamente á substância.
Em summa.
\section{Substanciar}
\begin{itemize}
\item {Grp. gram.:v. t.}
\end{itemize}
\begin{itemize}
\item {Utilização:Fig.}
\end{itemize}
\begin{itemize}
\item {Proveniência:(De \textunderscore substância\textunderscore )}
\end{itemize}
Dar comida substancial a.
Nutrir; reforçar.
Expor summariamente.
\section{Substancioso}
\begin{itemize}
\item {Grp. gram.:adj.}
\end{itemize}
\begin{itemize}
\item {Proveniência:(De \textunderscore substância\textunderscore )}
\end{itemize}
Que dá fôrça, que alimenta; que nutre.
\section{Substantificar}
\begin{itemize}
\item {Grp. gram.:v. t.}
\end{itemize}
\begin{itemize}
\item {Proveniência:(Do lat. \textunderscore substantia\textunderscore  + \textunderscore facere\textunderscore )}
\end{itemize}
Dar fórma concreta a.
\section{Substantífico}
\begin{itemize}
\item {Grp. gram.:adj.}
\end{itemize}
\begin{itemize}
\item {Proveniência:(De \textunderscore substantificar\textunderscore )}
\end{itemize}
O mesmo que \textunderscore substancioso\textunderscore .
\section{Substantivação}
\begin{itemize}
\item {Grp. gram.:f.}
\end{itemize}
\begin{itemize}
\item {Utilização:Gram.}
\end{itemize}
Acto do substantivar.
\section{Substantivadamente}
\begin{itemize}
\item {Grp. gram.:adv.}
\end{itemize}
De modo substantivado; á maneira de substantivo.
\section{Substantivado}
\begin{itemize}
\item {Grp. gram.:adj.}
\end{itemize}
Que assumiu a qualidade de substantivo.
\section{Substantivamente}
\begin{itemize}
\item {Grp. gram.:adv.}
\end{itemize}
\begin{itemize}
\item {Proveniência:(De \textunderscore substantivo\textunderscore )}
\end{itemize}
O mesmo que \textunderscore substantivadamente\textunderscore .
\section{Substantivar}
\begin{itemize}
\item {Grp. gram.:v. t.}
\end{itemize}
Empregar como substantivo.
\section{Substantivo}
\begin{itemize}
\item {Grp. gram.:adj.}
\end{itemize}
\begin{itemize}
\item {Grp. gram.:M.}
\end{itemize}
\begin{itemize}
\item {Proveniência:(Lat. \textunderscore substantivus\textunderscore )}
\end{itemize}
Que por si só designa a sua substância.
Que designa uma coisa que subsiste.
Relativo ao substantivo grammatical.
Palavra, que designa um sêr real, pessôa ou coisa; nome.
\section{Substatório}
\begin{itemize}
\item {Grp. gram.:adj.}
\end{itemize}
\begin{itemize}
\item {Proveniência:(Do lat. \textunderscore substare\textunderscore )}
\end{itemize}
Que faz sobrestar em alguma coisa.
Que envolve preceito para se sobrestar.
\section{Substentar}
\begin{itemize}
\item {Grp. gram.:v. t.}
\end{itemize}
\begin{itemize}
\item {Utilização:Ant.}
\end{itemize}
O mesmo que \textunderscore sustentar\textunderscore . Cf. B. Cruz, \textunderscore Chrón. de D. Sebast.\textunderscore , c. II.
\section{Substituição}
\begin{itemize}
\item {fónica:tu-i}
\end{itemize}
\begin{itemize}
\item {Grp. gram.:f.}
\end{itemize}
\begin{itemize}
\item {Utilização:Mathem.}
\end{itemize}
\begin{itemize}
\item {Proveniência:(Lat. \textunderscore substitutio\textunderscore )}
\end{itemize}
Acto ou effeito de substituír.
Disposição testamentária, em que se indica, não só o herdeiro directo, mas também aquelle ou aquelles, que hão de succeder a êste.
Representação de uma quantidade algébrica pelo seu valor.
\section{Substituído}
\begin{itemize}
\item {Grp. gram.:adj.}
\end{itemize}
\begin{itemize}
\item {Grp. gram.:M.}
\end{itemize}
\begin{itemize}
\item {Proveniência:(De \textunderscore substituir\textunderscore )}
\end{itemize}
Que se substituiu.
Indivíduo que foi substituído.
\section{Substituinte}
\begin{itemize}
\item {Grp. gram.:adj.}
\end{itemize}
\begin{itemize}
\item {Proveniência:(Lat. \textunderscore substituens\textunderscore )}
\end{itemize}
Que substitue.
\section{Substituir}
\begin{itemize}
\item {Grp. gram.:v. t.}
\end{itemize}
\begin{itemize}
\item {Proveniência:(Lat. \textunderscore substituere\textunderscore )}
\end{itemize}
Pôr uma pessôa ou coisa em lugar de.
Sêr ou fazer-se em vez de.
Fazer as vezes de.
\section{Substituível}
\begin{itemize}
\item {Grp. gram.:adj.}
\end{itemize}
Que se póde substituir.
\section{Substitutivo}
\begin{itemize}
\item {Grp. gram.:adj.}
\end{itemize}
\begin{itemize}
\item {Utilização:Med.}
\end{itemize}
\begin{itemize}
\item {Proveniência:(Lat. \textunderscore substitutivus\textunderscore )}
\end{itemize}
Diz-se do medicamento irritante, que altera o modo da inflammação, e a torna mais facilmente curável.
\section{Substituto}
\begin{itemize}
\item {Grp. gram.:adj.}
\end{itemize}
\begin{itemize}
\item {Grp. gram.:M.}
\end{itemize}
\begin{itemize}
\item {Proveniência:(Lat. \textunderscore substitutus\textunderscore )}
\end{itemize}
Que substitue.
Indivíduo, que substitue outro ou faz as suas vezes.
\section{Substracção}
\begin{itemize}
\item {Grp. gram.:f.}
\end{itemize}
\begin{itemize}
\item {Proveniência:(Do lat. \textunderscore substractus\textunderscore )}
\end{itemize}
Penitência canónica do terceiro grau, na Igreja antiga.
\section{Substrução}
\begin{itemize}
\item {Grp. gram.:f.}
\end{itemize}
\begin{itemize}
\item {Proveniência:(Lat. \textunderscore substructio\textunderscore )}
\end{itemize}
Alicerce; base de um edifício.
\section{Substrucção}
\begin{itemize}
\item {Grp. gram.:f.}
\end{itemize}
\begin{itemize}
\item {Proveniência:(Lat. \textunderscore substructio\textunderscore )}
\end{itemize}
Alicerce; base de um edifício.
\section{Substructura}
\begin{itemize}
\item {Grp. gram.:f.}
\end{itemize}
Estructura de partes inferiores: \textunderscore trabalhos de substructura\textunderscore .
Antónimo, \textunderscore superstructura\textunderscore .
\section{Substrutura}
\begin{itemize}
\item {Grp. gram.:f.}
\end{itemize}
Estructura de partes inferiores: \textunderscore trabalhos de substrutura\textunderscore .
Antónimo, \textunderscore superstrutura\textunderscore .
\section{Subsulfato}
\begin{itemize}
\item {Grp. gram.:m.}
\end{itemize}
\begin{itemize}
\item {Utilização:Chím.}
\end{itemize}
\begin{itemize}
\item {Proveniência:(De \textunderscore sub...\textunderscore  + \textunderscore sulfato\textunderscore )}
\end{itemize}
Sulfato, com excesso de base.
\section{Subsultar}
\begin{itemize}
\item {Grp. gram.:v. i.}
\end{itemize}
\begin{itemize}
\item {Utilização:Poét.}
\end{itemize}
\begin{itemize}
\item {Proveniência:(Lat. \textunderscore subsultare\textunderscore )}
\end{itemize}
Saltar repetidas vezes, saltitar.
\section{Subtangente}
\begin{itemize}
\item {Grp. gram.:f.}
\end{itemize}
\begin{itemize}
\item {Utilização:Mathem.}
\end{itemize}
\begin{itemize}
\item {Proveniência:(De \textunderscore sub...\textunderscore  + \textunderscore tangente\textunderscore )}
\end{itemize}
Parte do eixo de uma curva, entre a ordenada e a tangente que lhe corresponde.
\section{Subtendente}
\begin{itemize}
\item {Grp. gram.:f.}
\end{itemize}
\begin{itemize}
\item {Utilização:Mathem.}
\end{itemize}
\begin{itemize}
\item {Proveniência:(De \textunderscore subtender\textunderscore )}
\end{itemize}
Linha direita, que vai de uma á outra extremidade de um arco.
\section{Subtender}
\begin{itemize}
\item {Grp. gram.:v. t.}
\end{itemize}
\begin{itemize}
\item {Utilização:Mathem.}
\end{itemize}
\begin{itemize}
\item {Proveniência:(Lat. \textunderscore subtendere\textunderscore )}
\end{itemize}
Estender por baixo.
Formar corda, juntando as extremidades de (um arco).
\section{Subtenso}
\begin{itemize}
\item {Grp. gram.:adj.}
\end{itemize}
\begin{itemize}
\item {Utilização:Geom.}
\end{itemize}
\begin{itemize}
\item {Proveniência:(Do lat. \textunderscore sub...\textunderscore  + \textunderscore tensus\textunderscore )}
\end{itemize}
Diz-se da corda de um arco.
\section{Subterfúrgio}
\begin{itemize}
\item {Grp. gram.:m.}
\end{itemize}
\begin{itemize}
\item {Proveniência:(De \textunderscore subterfugir\textunderscore )}
\end{itemize}
Pretexto; ardil, para evitar difficuldades; evasiva.
\section{Subterfugioso}
\begin{itemize}
\item {Grp. gram.:adj.}
\end{itemize}
\begin{itemize}
\item {Utilização:Neol.}
\end{itemize}
Em que há subterfúgio: \textunderscore procedimento subterfugioso\textunderscore .
\section{Subterfugir}
\begin{itemize}
\item {Grp. gram.:v. t.}
\end{itemize}
\begin{itemize}
\item {Proveniência:(Lat. \textunderscore subterfugere\textunderscore )}
\end{itemize}
Empregar subterfúgio; escapar ardilosamente; esquivar-se com subterfúgio.
\section{Subterrâneo}
\begin{itemize}
\item {Grp. gram.:adj.}
\end{itemize}
\begin{itemize}
\item {Grp. gram.:M.}
\end{itemize}
\begin{itemize}
\item {Proveniência:(Lat. \textunderscore subterraneus\textunderscore )}
\end{itemize}
Que está debaixo da terra.
Que se realiza debaixo da terra.
Que está ou se faz debaixo de ruínas.
Lugar subterrâneo; cavidade subterrânea.
Cova.
Casa ou compartimento de casa, abaixo do nível do solo.
\section{Subterrar}
\begin{itemize}
\item {Grp. gram.:v. t.}
\end{itemize}
O mesmo que \textunderscore soterrar\textunderscore ^1.
\section{Subtérreo}
\begin{itemize}
\item {Grp. gram.:adj.}
\end{itemize}
\begin{itemize}
\item {Proveniência:(Lat. \textunderscore subterreus\textunderscore )}
\end{itemize}
O mesmo que \textunderscore subterrâneo\textunderscore .
\section{Subtil}
\begin{itemize}
\item {Grp. gram.:adj.}
\end{itemize}
\begin{itemize}
\item {Utilização:Fig.}
\end{itemize}
\begin{itemize}
\item {Grp. gram.:M.}
\end{itemize}
\begin{itemize}
\item {Proveniência:(Lat. \textunderscore subtilis\textunderscore )}
\end{itemize}
Tênue.
Agudo.
Fino.
Grácil.
Penetrante.
Muito miúdo, quási impalpável.
Perspicaz.
Hábil.
Engenhoso.
Feito delicadamente.
Que anda, sem fazer ruído; imperceptível.
O mesmo que \textunderscore subtileza\textunderscore .
\section{Subtileza}
\begin{itemize}
\item {Grp. gram.:f.}
\end{itemize}
Qualidade do que é subtil.
Tenuidade.
Delicadeza.
Finura.
Agudeza de espírito, penetração.
Dito ou argumento de alguém, para embaraçar outrem, ou que embaraça outrem.
\section{Subtilidade}
\begin{itemize}
\item {Grp. gram.:f.}
\end{itemize}
\begin{itemize}
\item {Proveniência:(Do lat. \textunderscore subtilitas\textunderscore )}
\end{itemize}
O mesmo que \textunderscore subtileza\textunderscore .
\section{Subtilização}
\begin{itemize}
\item {Grp. gram.:f.}
\end{itemize}
Acto ou effeito de subtilizar.
\section{Subtilizador}
\begin{itemize}
\item {Grp. gram.:m.  e  adj.}
\end{itemize}
O que subtiliza.
\section{Subtilizar}
\begin{itemize}
\item {Grp. gram.:v. t.}
\end{itemize}
\begin{itemize}
\item {Grp. gram.:V. i.}
\end{itemize}
\begin{itemize}
\item {Proveniência:(De \textunderscore subtil\textunderscore )}
\end{itemize}
Tornar subtil.
Raciocinar com subtilezas; disputar arguciosamente.
\section{Subtilmente}
\begin{itemize}
\item {Grp. gram.:adv.}
\end{itemize}
De modo subtil.
Com argúcia.
Com subtileza.
Sem ruído, levemente: \textunderscore andar subtilmente\textunderscore .
\section{Subtítulo}
\begin{itemize}
\item {Grp. gram.:m.}
\end{itemize}
\begin{itemize}
\item {Proveniência:(De \textunderscore sub...\textunderscore  + \textunderscore título\textunderscore )}
\end{itemize}
Segundo título.
Titulo, que segue outro.
\section{Subtracção}
\begin{itemize}
\item {Grp. gram.:f.}
\end{itemize}
\begin{itemize}
\item {Proveniência:(Lat. \textunderscore subtractio\textunderscore )}
\end{itemize}
Acto ou effeito de subtrahir; roubo fraudulento.
\section{Subtractivo}
\begin{itemize}
\item {Grp. gram.:adj.}
\end{itemize}
\begin{itemize}
\item {Grp. gram.:M.}
\end{itemize}
\begin{itemize}
\item {Proveniência:(Do lat. \textunderscore subtractus\textunderscore )}
\end{itemize}
Relativo á subtracção.
Aquillo que se subtrái.
\section{Subtrahir}
\begin{itemize}
\item {Grp. gram.:v. t.}
\end{itemize}
\begin{itemize}
\item {Grp. gram.:V. p.}
\end{itemize}
\begin{itemize}
\item {Proveniência:(Lat. \textunderscore subtrahere\textunderscore )}
\end{itemize}
Tirar ás escondidas ou com fraude.
Furtar.
Tirar.
Afastar.
Separar.
Deminuir, deduzir.
Esquivar-se; fugir.
\section{Subtrair}
\begin{itemize}
\item {Grp. gram.:v. t.}
\end{itemize}
\begin{itemize}
\item {Grp. gram.:V. p.}
\end{itemize}
\begin{itemize}
\item {Proveniência:(Lat. \textunderscore subtrahere\textunderscore )}
\end{itemize}
Tirar ás escondidas ou com fraude.
Furtar.
Tirar.
Afastar.
Separar.
Deminuír, deduzir.
Esquivar-se; fugir.
\section{Subtribo}
\begin{itemize}
\item {Grp. gram.:f.}
\end{itemize}
\begin{itemize}
\item {Utilização:Hist. Nat.}
\end{itemize}
\begin{itemize}
\item {Proveniência:(De \textunderscore sub...\textunderscore  + \textunderscore tribu\textunderscore )}
\end{itemize}
Tríbo secundária, subordinada a uma tríbo primária.
\section{Subtríplice}
\begin{itemize}
\item {Grp. gram.:adj.}
\end{itemize}
\begin{itemize}
\item {Proveniência:(Lat. \textunderscore subtriplus\textunderscore )}
\end{itemize}
Diz-se de um número, contido três vezes noutro.
\section{Subtriplo}
\begin{itemize}
\item {Grp. gram.:adj.}
\end{itemize}
\begin{itemize}
\item {Proveniência:(Lat. \textunderscore subtriplus\textunderscore )}
\end{itemize}
Diz-se de um número, contido três vezes noutro.
\section{Sub-typo}
\begin{itemize}
\item {Grp. gram.:m.}
\end{itemize}
\begin{itemize}
\item {Utilização:Hist. Nat.}
\end{itemize}
\begin{itemize}
\item {Proveniência:(De \textunderscore sub...\textunderscore  + \textunderscore typo\textunderscore )}
\end{itemize}
Typo secundário, subordinado a um typo primário.
\section{Súbula}
\begin{itemize}
\item {Grp. gram.:f.}
\end{itemize}
\begin{itemize}
\item {Proveniência:(Lat. \textunderscore subula\textunderscore )}
\end{itemize}
Gênero de insectos dípteros.
\section{Subulado}
\begin{itemize}
\item {Grp. gram.:adj.}
\end{itemize}
\begin{itemize}
\item {Utilização:Hist. Nat.}
\end{itemize}
\begin{itemize}
\item {Proveniência:(Do lat. \textunderscore subula\textunderscore )}
\end{itemize}
Que termina superiormente em ponta de sovela.
\section{Subulária}
\begin{itemize}
\item {Grp. gram.:f.}
\end{itemize}
\begin{itemize}
\item {Proveniência:(Do lat. \textunderscore subula\textunderscore )}
\end{itemize}
Gênero de plantas crucíferas.
\section{Subulifoliado}
\begin{itemize}
\item {Grp. gram.:adj.}
\end{itemize}
\begin{itemize}
\item {Utilização:Bot.}
\end{itemize}
\begin{itemize}
\item {Proveniência:(Do lat. \textunderscore subula\textunderscore  + \textunderscore folium\textunderscore )}
\end{itemize}
Que tem fôlhas subuladas.
\section{Subulipalpos}
\begin{itemize}
\item {Grp. gram.:m. pl.}
\end{itemize}
\begin{itemize}
\item {Proveniência:(Do lat. \textunderscore subula\textunderscore  + \textunderscore palpus\textunderscore )}
\end{itemize}
Secção de insectos coleópteros.
\section{Subulípede}
\begin{itemize}
\item {Grp. gram.:adj.}
\end{itemize}
\begin{itemize}
\item {Utilização:Zool.}
\end{itemize}
\begin{itemize}
\item {Proveniência:(Do lat. \textunderscore subula\textunderscore  + \textunderscore pes\textunderscore )}
\end{itemize}
Que tem pé comprido e delgado.
\section{Subulirostros}
\begin{itemize}
\item {fónica:rós}
\end{itemize}
\begin{itemize}
\item {Grp. gram.:m. pl.}
\end{itemize}
\begin{itemize}
\item {Proveniência:(Do lat. \textunderscore subula\textunderscore  + \textunderscore rostrum\textunderscore )}
\end{itemize}
Família de aves, estabelecida por Dumeril, e que abrange as que têm bico curto, fraco e flexível, não chanfrado, de base estreita e arredondada.
\section{Subulirrostos}
\begin{itemize}
\item {Grp. gram.:m. pl.}
\end{itemize}
\begin{itemize}
\item {Proveniência:(Do lat. \textunderscore subula\textunderscore  + \textunderscore rostrum\textunderscore )}
\end{itemize}
Família de aves, estabelecida por Dumeril, e que abrange as que têm bico curto, fraco e flexível, não chanfrado, de base estreita e arredondada.
\section{Súbulo}
\begin{itemize}
\item {Grp. gram.:m.}
\end{itemize}
\begin{itemize}
\item {Proveniência:(Lat. \textunderscore subulo\textunderscore )}
\end{itemize}
Tocador de frauta, entre os Romanos.
\section{Suburbano}
\begin{itemize}
\item {Grp. gram.:adj.}
\end{itemize}
\begin{itemize}
\item {Proveniência:(Lat. \textunderscore suburbanus\textunderscore )}
\end{itemize}
Relativo a subúrbios.
Que está próximo da cidade.
\section{Suburbicário}
\begin{itemize}
\item {Grp. gram.:adj.}
\end{itemize}
\begin{itemize}
\item {Proveniência:(Lat. \textunderscore suburbicarius\textunderscore )}
\end{itemize}
Dizia-se das cidades, submetidas ao govêrno do prefeito de Roma; e dizia-se das províncias e igrejas, comprehendidas na diocese de Roma.
\section{Subúrbio}
\begin{itemize}
\item {Grp. gram.:m.}
\end{itemize}
\begin{itemize}
\item {Proveniência:(Lat. \textunderscore suburbium\textunderscore )}
\end{itemize}
Arrabaldes ou vizinhanças de cidade ou de qualquer povoação.
Aros, redondeza, cercanias.
\section{Subúrbios}
\begin{itemize}
\item {Grp. gram.:m. pl.}
\end{itemize}
\begin{itemize}
\item {Proveniência:(Lat. \textunderscore suburbium\textunderscore )}
\end{itemize}
Arrabaldes ou vizinhanças de cidade ou de qualquer povoação.
Aros, redondeza, cercanias.
\section{Subutraquistas}
\begin{itemize}
\item {Grp. gram.:m. pl.}
\end{itemize}
Membros de uma seita, que administrava a communhão sob duas espécies consagradas, pão e vinho.
(Da loc. lat. \textunderscore sub ultraque specie\textunderscore )
\section{Subvassallo}
\begin{itemize}
\item {Grp. gram.:m.}
\end{itemize}
\begin{itemize}
\item {Utilização:Ant.}
\end{itemize}
\begin{itemize}
\item {Proveniência:(De \textunderscore sub...\textunderscore  + \textunderscore vassallo\textunderscore )}
\end{itemize}
Vassallo de vassallo.
\section{Subvassalo}
\begin{itemize}
\item {Grp. gram.:m.}
\end{itemize}
\begin{itemize}
\item {Utilização:Ant.}
\end{itemize}
\begin{itemize}
\item {Proveniência:(De \textunderscore sub...\textunderscore  + \textunderscore vassalo\textunderscore )}
\end{itemize}
Vassalo de vassalo.
\section{Subvenção}
\begin{itemize}
\item {Grp. gram.:f.}
\end{itemize}
\begin{itemize}
\item {Proveniência:(Do lat. \textunderscore subventio\textunderscore )}
\end{itemize}
Subsídio; auxílio pecuniário: \textunderscore teve subvenção do Govêrno para estudar em Berlim\textunderscore .
\section{Subvencional}
\begin{itemize}
\item {Grp. gram.:adj.}
\end{itemize}
\begin{itemize}
\item {Proveniência:(Do lat. \textunderscore subventio\textunderscore )}
\end{itemize}
Relativo a subvenção.
\section{Subvencionar}
\begin{itemize}
\item {Grp. gram.:v. t.}
\end{itemize}
\begin{itemize}
\item {Proveniência:(Do lat. \textunderscore subventio\textunderscore )}
\end{itemize}
Dar subvenção a.
\section{Subventâneo}
\begin{itemize}
\item {Grp. gram.:adj.}
\end{itemize}
\begin{itemize}
\item {Utilização:Des.}
\end{itemize}
\begin{itemize}
\item {Proveniência:(Lat. \textunderscore subventaneus\textunderscore )}
\end{itemize}
Que não teve resultado; que abortou.
\section{Subventral}
\begin{itemize}
\item {Grp. gram.:adj.}
\end{itemize}
\begin{itemize}
\item {Utilização:Zool.}
\end{itemize}
\begin{itemize}
\item {Proveniência:(De \textunderscore sub...\textunderscore  + \textunderscore ventral\textunderscore )}
\end{itemize}
Situado por baixo do ventre.
\section{Subversão}
\begin{itemize}
\item {Grp. gram.:f.}
\end{itemize}
\begin{itemize}
\item {Proveniência:(Do lat. \textunderscore subversio\textunderscore )}
\end{itemize}
Acto ou effeito de subverter.
Revolta, insubordinação.
\section{Subversivo}
\begin{itemize}
\item {Grp. gram.:adj.}
\end{itemize}
\begin{itemize}
\item {Proveniência:(Do lat. \textunderscore subversus\textunderscore )}
\end{itemize}
Que subverte; que póde subverter; revolucionário.
\section{Subversor}
\begin{itemize}
\item {Grp. gram.:m.  e  adj.}
\end{itemize}
\begin{itemize}
\item {Proveniência:(Lat. \textunderscore subversor\textunderscore )}
\end{itemize}
O que subverte.
\section{Subvertedor}
\begin{itemize}
\item {Grp. gram.:m.  e  adj.}
\end{itemize}
O mesmo que \textunderscore subversor\textunderscore .
\section{Subverter}
\begin{itemize}
\item {Grp. gram.:v. t.}
\end{itemize}
\begin{itemize}
\item {Proveniência:(Lat. \textunderscore subvertere\textunderscore )}
\end{itemize}
Revolver, voltar de baixo para cima.
Destruir.
Submergir.
Perverter.
Arruinar.
Revolucionar; pôr em estado de desordem.
\section{Subvertimento}
\begin{itemize}
\item {Grp. gram.:m.}
\end{itemize}
O mesmo que \textunderscore subversão\textunderscore .
\section{Subvéspero}
\begin{itemize}
\item {Grp. gram.:m.}
\end{itemize}
\begin{itemize}
\item {Utilização:Des.}
\end{itemize}
\begin{itemize}
\item {Proveniência:(Lat. \textunderscore sub-vesperus\textunderscore )}
\end{itemize}
Vento de Sudoéste.
\section{Sucado}
\begin{itemize}
\item {Grp. gram.:adj.}
\end{itemize}
\begin{itemize}
\item {Proveniência:(De \textunderscore suco\textunderscore )}
\end{itemize}
Atuchado, endurecido.
\textunderscore Sucado de carnes\textunderscore , que tem carnes rijas, sem flaccidez. Cf. Camillo, \textunderscore Cav. em Ruinas\textunderscore , 70.
\section{Sucanga}
\begin{itemize}
\item {Grp. gram.:f.}
\end{itemize}
Árvore do Congo.
\section{Sucapé}
\begin{itemize}
\item {Grp. gram.:m.}
\end{itemize}
O mesmo que \textunderscore sapé\textunderscore .
\section{Sucar}
\begin{itemize}
\item {Grp. gram.:v. t.}
\end{itemize}
\begin{itemize}
\item {Utilização:Prov.}
\end{itemize}
O mesmo que \textunderscore sugar\textunderscore .
\section{Sucção}
\begin{itemize}
\item {Grp. gram.:f.}
\end{itemize}
\begin{itemize}
\item {Proveniência:(Do lat. \textunderscore suctus\textunderscore )}
\end{itemize}
Acto ou effeito de sugar.
\section{Succedâneo}
\begin{itemize}
\item {Grp. gram.:m.  e  adj.}
\end{itemize}
\begin{itemize}
\item {Proveniência:(Lat. \textunderscore succedaneus\textunderscore )}
\end{itemize}
Diz-se do medicamento, com que se póde substituir outro, por têr proximamente as mesmas propriedades.
Diz-se de uma substância, que póde substituir outra: \textunderscore a sisalana é succedânea do cânhamo\textunderscore .
\section{Succedenho}
\begin{itemize}
\item {Grp. gram.:m.}
\end{itemize}
\begin{itemize}
\item {Utilização:Prov.}
\end{itemize}
\begin{itemize}
\item {Proveniência:(Do lat. \textunderscore succedaneus\textunderscore )}
\end{itemize}
O mesmo que \textunderscore successo\textunderscore , acontecimento.
\section{Succeder}
\begin{itemize}
\item {Grp. gram.:v. i.}
\end{itemize}
\begin{itemize}
\item {Grp. gram.:V. p.}
\end{itemize}
\begin{itemize}
\item {Proveniência:(Lat. \textunderscore succedere\textunderscore )}
\end{itemize}
Acontecer depois.
Vir em seguida.
Tomar o lugar de outrem ou de outra coisa: \textunderscore a República succedeu á Monarchia\textunderscore .
Acontecer.
Realizar-se: \textunderscore é natural: são coisas que succedem\textunderscore .
Sêr chamado por lei ou testamento a uma herança.
Vir depois, ou acontecer successivamente, (falando-se de mais de uma coisa, em relação a outra).
\section{Succedido}
\begin{itemize}
\item {Grp. gram.:adj.}
\end{itemize}
\begin{itemize}
\item {Grp. gram.:M.}
\end{itemize}
\begin{itemize}
\item {Proveniência:(De \textunderscore succeder\textunderscore )}
\end{itemize}
Que aconteceu, que se realizou: \textunderscore um facto, succedido há três annos\textunderscore .
O mesmo que \textunderscore successo\textunderscore : \textunderscore vou contar-lhe o succedido\textunderscore .
\section{Succedimento}
\begin{itemize}
\item {Grp. gram.:m.}
\end{itemize}
\begin{itemize}
\item {Proveniência:(De \textunderscore succeder\textunderscore )}
\end{itemize}
Successão; successo.
\section{Succedo}
\begin{itemize}
\item {Grp. gram.:m.}
\end{itemize}
\begin{itemize}
\item {Utilização:Prov.}
\end{itemize}
\begin{itemize}
\item {Utilização:trasm.}
\end{itemize}
\begin{itemize}
\item {Proveniência:(De \textunderscore succeder\textunderscore )}
\end{itemize}
O mesmo que \textunderscore successo\textunderscore .
\section{Succenturiado}
\begin{itemize}
\item {Grp. gram.:adj.}
\end{itemize}
\begin{itemize}
\item {Proveniência:(Lat. \textunderscore succenturiatus\textunderscore )}
\end{itemize}
Diz-se da dilatação do canal digestivo das aves, entre o papo e a moéla.
\section{Successão}
\begin{itemize}
\item {Grp. gram.:f.}
\end{itemize}
\begin{itemize}
\item {Utilização:Fig.}
\end{itemize}
\begin{itemize}
\item {Proveniência:(Do lat. \textunderscore successio\textunderscore )}
\end{itemize}
Acto ou effeito de succeder.
Herança.
Descendência.
Qualidade, que se transmitte aos descendentes.
\section{Successivamente}
\begin{itemize}
\item {Grp. gram.:adv.}
\end{itemize}
De modo successivo.
Ordenadamente; sem interupção.
Por graus progressivos.
\section{Successível}
\begin{itemize}
\item {Grp. gram.:adv.}
\end{itemize}
\begin{itemize}
\item {Proveniência:(Do lat. \textunderscore successus\textunderscore )}
\end{itemize}
Que póde succeder como herdeiro ou com outro qualquer título.
\section{Successivo}
\begin{itemize}
\item {Grp. gram.:adj.}
\end{itemize}
\begin{itemize}
\item {Proveniência:(Lat. \textunderscore successivus\textunderscore )}
\end{itemize}
Que vem depois, que vem em seguida.
Contínuo; que não tem interrupção.
Que vem depois de outro, com pequeno intervallo.
\section{Successo}
\begin{itemize}
\item {Grp. gram.:m.}
\end{itemize}
\begin{itemize}
\item {Proveniência:(Lat. \textunderscore successus\textunderscore )}
\end{itemize}
Aquillo que succede.
Acontecimento.
Resultado.
Conclusão.
Êxito: \textunderscore a tentativa não teve successo\textunderscore .
O mesmo que \textunderscore parto\textunderscore : \textunderscore a parturiente teve mau successo\textunderscore .
\section{Successor}
\begin{itemize}
\item {Grp. gram.:m.  e  adj.}
\end{itemize}
\begin{itemize}
\item {Proveniência:(Lat. \textunderscore successor\textunderscore )}
\end{itemize}
O que succede a outrem.
Herdeiro.
Aquelle que tem a mesma dignidade ou os mesmos predicados que outrem teve.
\section{Successoral}
\begin{itemize}
\item {Grp. gram.:adj.}
\end{itemize}
\begin{itemize}
\item {Utilização:Neol.}
\end{itemize}
O mesmo que \textunderscore successorial\textunderscore .
\section{Successorial}
\begin{itemize}
\item {Grp. gram.:adj.}
\end{itemize}
\begin{itemize}
\item {Utilização:bras}
\end{itemize}
\begin{itemize}
\item {Utilização:Neol.}
\end{itemize}
Relativo a successor ou a successão.
\section{Successório}
\begin{itemize}
\item {Grp. gram.:adj.}
\end{itemize}
\begin{itemize}
\item {Proveniência:(Lat. \textunderscore successorius\textunderscore )}
\end{itemize}
Relativo á successão.
\section{Succinato}
\begin{itemize}
\item {Grp. gram.:m.}
\end{itemize}
\begin{itemize}
\item {Proveniência:(De \textunderscore succino\textunderscore )}
\end{itemize}
Sal, resultante da combinação do ácido succínico com uma base.
\section{Succíneas}
\begin{itemize}
\item {Grp. gram.:f. pl.}
\end{itemize}
Gênero de molluscos gasterópodes.
(Fem. pl. de \textunderscore succíneo\textunderscore )
\section{Succíneo}
\begin{itemize}
\item {Grp. gram.:adj.}
\end{itemize}
\begin{itemize}
\item {Proveniência:(Lat. \textunderscore succineus\textunderscore )}
\end{itemize}
Que tem a côr de succino.
\section{Succínico}
\begin{itemize}
\item {Grp. gram.:adj.}
\end{itemize}
Relativo ao succino. Cf. \textunderscore Techn. Rur.\textunderscore , 27.
\section{Succino}
\begin{itemize}
\item {Grp. gram.:m.}
\end{itemize}
\begin{itemize}
\item {Proveniência:(Lat. \textunderscore succinum\textunderscore )}
\end{itemize}
Âmbar amarelo.
\section{Succintamente}
\begin{itemize}
\item {Grp. gram.:adv.}
\end{itemize}
De modo succinto.
Summariamente.
Em resumo.
\section{Succinto}
\begin{itemize}
\item {Grp. gram.:adj.}
\end{itemize}
\begin{itemize}
\item {Proveniência:(Lat. \textunderscore succinctus\textunderscore )}
\end{itemize}
Que tem poucas palavras; resumido: \textunderscore exposição succinta\textunderscore .
\section{Succóvia}
\begin{itemize}
\item {Grp. gram.:f.}
\end{itemize}
Planta crucífera.
\section{Súccuba}
\begin{itemize}
\item {Grp. gram.:f.}
\end{itemize}
\begin{itemize}
\item {Utilização:Ant.}
\end{itemize}
\begin{itemize}
\item {Proveniência:(De \textunderscore súccubo\textunderscore )}
\end{itemize}
Concubina.
\section{Succúbico}
\begin{itemize}
\item {Grp. gram.:adj.}
\end{itemize}
Próprio de súccubo.
Relativo a súccubo.
\section{Súccubo}
\begin{itemize}
\item {Grp. gram.:adj.}
\end{itemize}
\begin{itemize}
\item {Grp. gram.:M.  e  adj.}
\end{itemize}
\begin{itemize}
\item {Proveniência:(Do lat. \textunderscore succubare\textunderscore )}
\end{itemize}
Que se põe por baixo.
Dizia-se de um demónio, a cuja influência se attribuiam os sonhos maus.
\section{Succumbido}
\begin{itemize}
\item {Grp. gram.:adj.}
\end{itemize}
Que succumbiu.
\section{Succumbimento}
\begin{itemize}
\item {Grp. gram.:m.}
\end{itemize}
Acto ou effeito de succumbir.
\section{Succumbir}
\begin{itemize}
\item {Grp. gram.:v. i.}
\end{itemize}
\begin{itemize}
\item {Proveniência:(Lat. \textunderscore succumbere\textunderscore )}
\end{itemize}
Cair debaixo.
Vergar.
Ceder aos esforços de outrem.
Sêr dominado.
Não poder resistir.
Desalentar-se.
Acabar; morrer; desapparecer.
\section{Succursal}
\begin{itemize}
\item {Grp. gram.:adj.}
\end{itemize}
\begin{itemize}
\item {Grp. gram.:F.}
\end{itemize}
\begin{itemize}
\item {Proveniência:(Do lat. \textunderscore succursus\textunderscore )}
\end{itemize}
Diz-se de um estabelecimento dependente de outro.
Casa ou estabelecimento succursal.
\section{Succussão}
\begin{itemize}
\item {Grp. gram.:f.}
\end{itemize}
\begin{itemize}
\item {Proveniência:(Do lat. \textunderscore succussio\textunderscore )}
\end{itemize}
Abalo.
\section{Sucedâneo}
\begin{itemize}
\item {Grp. gram.:m.  e  adj.}
\end{itemize}
\begin{itemize}
\item {Proveniência:(Lat. \textunderscore succedaneus\textunderscore )}
\end{itemize}
Diz-se do medicamento, com que se póde substituir outro, por têr proximamente as mesmas propriedades.
Diz-se de uma substância, que póde substituir outra: \textunderscore a sisalana é sucedânea do cânhamo\textunderscore .
\section{Sucedenho}
\begin{itemize}
\item {Grp. gram.:m.}
\end{itemize}
\begin{itemize}
\item {Utilização:Prov.}
\end{itemize}
\begin{itemize}
\item {Proveniência:(Do lat. \textunderscore succedaneus\textunderscore )}
\end{itemize}
O mesmo que \textunderscore sucesso\textunderscore , acontecimento.
\section{Suceder}
\begin{itemize}
\item {Grp. gram.:v. i.}
\end{itemize}
\begin{itemize}
\item {Grp. gram.:V. p.}
\end{itemize}
\begin{itemize}
\item {Proveniência:(Lat. \textunderscore succedere\textunderscore )}
\end{itemize}
Acontecer depois.
Vir em seguida.
Tomar o lugar de outrem ou de outra coisa: \textunderscore a República sucedeu á Monarquia\textunderscore .
Acontecer.
Realizar-se: \textunderscore é natural: são coisas que sucedem\textunderscore .
Sêr chamado por lei ou testamento a uma herança.
Vir depois, ou acontecer sucessivamente, (falando-se de mais de uma coisa, em relação a outra).
\section{Sucedido}
\begin{itemize}
\item {Grp. gram.:adj.}
\end{itemize}
\begin{itemize}
\item {Grp. gram.:M.}
\end{itemize}
\begin{itemize}
\item {Proveniência:(De \textunderscore suceder\textunderscore )}
\end{itemize}
Que aconteceu, que se realizou: \textunderscore um facto, sucedido há três anos\textunderscore .
O mesmo que \textunderscore sucesso\textunderscore : \textunderscore vou contar-lhe o sucedido\textunderscore .
\section{Sucedimento}
\begin{itemize}
\item {Grp. gram.:m.}
\end{itemize}
\begin{itemize}
\item {Proveniência:(De \textunderscore suceder\textunderscore )}
\end{itemize}
Sucessão; sucesso.
\section{Sucedo}
\begin{itemize}
\item {Grp. gram.:m.}
\end{itemize}
\begin{itemize}
\item {Utilização:Prov.}
\end{itemize}
\begin{itemize}
\item {Utilização:trasm.}
\end{itemize}
\begin{itemize}
\item {Proveniência:(De \textunderscore suceder\textunderscore )}
\end{itemize}
O mesmo que \textunderscore sucesso\textunderscore .
\section{Sucenturiado}
\begin{itemize}
\item {Grp. gram.:adj.}
\end{itemize}
\begin{itemize}
\item {Proveniência:(Lat. \textunderscore succenturiatus\textunderscore )}
\end{itemize}
Diz-se da dilatação do canal digestivo das aves, entre o papo e a moéla.
\section{Sucessão}
\begin{itemize}
\item {Grp. gram.:f.}
\end{itemize}
\begin{itemize}
\item {Utilização:Fig.}
\end{itemize}
\begin{itemize}
\item {Proveniência:(Do lat. \textunderscore successio\textunderscore )}
\end{itemize}
Acto ou efeito de suceder.
Herança.
Descendência.
Qualidade, que se transmite aos descendentes.
\section{Sucessivamente}
\begin{itemize}
\item {Grp. gram.:adv.}
\end{itemize}
De modo sucessivo.
Ordenadamente; sem interupção.
Por graus progressivos.
\section{Sucessível}
\begin{itemize}
\item {Grp. gram.:adv.}
\end{itemize}
\begin{itemize}
\item {Proveniência:(Do lat. \textunderscore successus\textunderscore )}
\end{itemize}
Que póde suceder como herdeiro ou com outro qualquer título.
\section{Sucessivo}
\begin{itemize}
\item {Grp. gram.:adj.}
\end{itemize}
\begin{itemize}
\item {Proveniência:(Lat. \textunderscore successivus\textunderscore )}
\end{itemize}
Que vem depois, que vem em seguida.
Contínuo; que não tem interrupção.
Que vem depois de outro, com pequeno intervalo.
\section{Sucesso}
\begin{itemize}
\item {Grp. gram.:m.}
\end{itemize}
\begin{itemize}
\item {Proveniência:(Lat. \textunderscore successus\textunderscore )}
\end{itemize}
Aquilo que sucede.
Acontecimento.
Resultado.
Conclusão.
Êxito: \textunderscore a tentativa não teve sucesso\textunderscore .
O mesmo que \textunderscore parto\textunderscore : \textunderscore a parturiente teve mau sucesso\textunderscore .
\section{Sucessor}
\begin{itemize}
\item {Grp. gram.:m.  e  adj.}
\end{itemize}
\begin{itemize}
\item {Proveniência:(Lat. \textunderscore successor\textunderscore )}
\end{itemize}
O que sucede a outrem.
Herdeiro.
Aquele que tem a mesma dignidade ou os mesmos predicados que outrem teve.
\section{Sucessoral}
\begin{itemize}
\item {Grp. gram.:adj.}
\end{itemize}
\begin{itemize}
\item {Utilização:Neol.}
\end{itemize}
O mesmo que \textunderscore sucessorial\textunderscore .
\section{Sucessorial}
\begin{itemize}
\item {Grp. gram.:adj.}
\end{itemize}
\begin{itemize}
\item {Utilização:bras}
\end{itemize}
\begin{itemize}
\item {Utilização:Neol.}
\end{itemize}
Relativo a sucessor ou a sucessão.
\section{Sucessório}
\begin{itemize}
\item {Grp. gram.:adj.}
\end{itemize}
\begin{itemize}
\item {Proveniência:(Lat. \textunderscore successorius\textunderscore )}
\end{itemize}
Relativo á sucessão.
\section{Suchia}
\begin{itemize}
\item {Grp. gram.:f. Loc. adv.}
\end{itemize}
\begin{itemize}
\item {Utilização:Prov.}
\end{itemize}
\begin{itemize}
\item {Utilização:trasm.}
\end{itemize}
\textunderscore Á suchia\textunderscore , ás escondidas, em segrêdo, pela calada.
\section{Sucho}
\begin{itemize}
\item {Grp. gram.:m.}
\end{itemize}
\begin{itemize}
\item {Utilização:Prov.}
\end{itemize}
\begin{itemize}
\item {Utilização:trasm.}
\end{itemize}
Medo.
\section{Súcia}
\begin{itemize}
\item {Grp. gram.:f.}
\end{itemize}
\begin{itemize}
\item {Utilização:Des.}
\end{itemize}
\begin{itemize}
\item {Utilização:Prov.}
\end{itemize}
\begin{itemize}
\item {Utilização:minh.}
\end{itemize}
\begin{itemize}
\item {Proveniência:(De \textunderscore súcio\textunderscore )}
\end{itemize}
Reunião de pessôas de má índole ou de má fama; matula.
Sociedade, Assembleia.
Qualquer grupo ou rancho: \textunderscore esperava-se a súcia dos mascarados\textunderscore .
\section{Suciar}
\begin{itemize}
\item {Grp. gram.:v. i.}
\end{itemize}
Fazer parte de uma súcia; vadiar.
\section{Suciata}
\begin{itemize}
\item {Grp. gram.:f.}
\end{itemize}
\begin{itemize}
\item {Utilização:Pop.}
\end{itemize}
\begin{itemize}
\item {Proveniência:(De \textunderscore súcia\textunderscore )}
\end{itemize}
Patuscada, pândega.
\section{Sucinato}
\begin{itemize}
\item {Grp. gram.:m.}
\end{itemize}
\begin{itemize}
\item {Proveniência:(De \textunderscore sucino\textunderscore )}
\end{itemize}
Sal, resultante da combinação do ácido sucínico com uma base.
\section{Sucíneas}
\begin{itemize}
\item {Grp. gram.:f. pl.}
\end{itemize}
Gênero de moluscos gasterópodes.
(Fem. pl. de \textunderscore succíneo\textunderscore )
\section{Sucíneo}
\begin{itemize}
\item {Grp. gram.:adj.}
\end{itemize}
\begin{itemize}
\item {Proveniência:(Lat. \textunderscore succineus\textunderscore )}
\end{itemize}
Que tem a côr de sucino.
\section{Sucínico}
\begin{itemize}
\item {Grp. gram.:adj.}
\end{itemize}
Relativo ao sucino. Cf. \textunderscore Techn. Rur.\textunderscore , 27.
\section{Sucino}
\begin{itemize}
\item {Grp. gram.:m.}
\end{itemize}
\begin{itemize}
\item {Proveniência:(Lat. \textunderscore succinum\textunderscore )}
\end{itemize}
Âmbar amarelo.
\section{Sucintamente}
\begin{itemize}
\item {Grp. gram.:adv.}
\end{itemize}
De modo sucinto.
Sumariamente.
Em resumo.
\section{Sucinto}
\begin{itemize}
\item {Grp. gram.:adj.}
\end{itemize}
\begin{itemize}
\item {Proveniência:(Lat. \textunderscore succinctus\textunderscore )}
\end{itemize}
Que tem poucas palavras; resumido: \textunderscore exposição sucinta\textunderscore .
\section{Súcio}
\begin{itemize}
\item {Grp. gram.:m.}
\end{itemize}
\begin{itemize}
\item {Utilização:Deprec.}
\end{itemize}
\begin{itemize}
\item {Proveniência:(Do lat. \textunderscore sucidus\textunderscore ?)}
\end{itemize}
Indivíduo, que faz parte de uma súcia.
Troca-tintas, biltre; vadio.
\section{Suco}
\begin{itemize}
\item {Grp. gram.:m.}
\end{itemize}
\begin{itemize}
\item {Utilização:Pop.}
\end{itemize}
\begin{itemize}
\item {Utilização:Fig.}
\end{itemize}
\begin{itemize}
\item {Proveniência:(Lat. \textunderscore sucus\textunderscore )}
\end{itemize}
Certo líquido, que se encontra nos vegetaes e na carne.
Sumo.
Seiva.
Gordura.
Essência.
O que há de principal ou aproveitável num systema, num escrito, etc.
\section{Suco}
\begin{itemize}
\item {Grp. gram.:m.}
\end{itemize}
Conjunto de aldeias, em Timor.
\section{Suco}
\begin{itemize}
\item {Grp. gram.:m.}
\end{itemize}
\begin{itemize}
\item {Utilização:T. de Miranda}
\end{itemize}
O mesmo que \textunderscore sulco\textunderscore ^1.
\section{Sucoso}
\begin{itemize}
\item {Grp. gram.:adj.}
\end{itemize}
\begin{itemize}
\item {Proveniência:(Lat. \textunderscore sucosus\textunderscore )}
\end{itemize}
O mesmo que \textunderscore suculento\textunderscore .
\section{Sucóvia}
\begin{itemize}
\item {Grp. gram.:f.}
\end{itemize}
Planta crucífera.
\section{Suçuaia}
\begin{itemize}
\item {Grp. gram.:f.}
\end{itemize}
\begin{itemize}
\item {Utilização:Bras}
\end{itemize}
Nome de duas plantas medicinaes.
\section{Suçuapara}
\begin{itemize}
\item {Grp. gram.:m.}
\end{itemize}
\begin{itemize}
\item {Utilização:Bras. do N}
\end{itemize}
Veado dos brejos.
\section{Suçuarana}
\begin{itemize}
\item {Grp. gram.:m.}
\end{itemize}
Animal carnívoro da América do Sul.
\section{Súcuba}
\begin{itemize}
\item {Grp. gram.:f.}
\end{itemize}
\begin{itemize}
\item {Utilização:Ant.}
\end{itemize}
\begin{itemize}
\item {Proveniência:(De \textunderscore súcubo\textunderscore )}
\end{itemize}
Concubina.
\section{Sucúbico}
\begin{itemize}
\item {Grp. gram.:adj.}
\end{itemize}
Próprio de súcubo.
Relativo a súcubo.
\section{Súcubo}
\begin{itemize}
\item {Grp. gram.:adj.}
\end{itemize}
\begin{itemize}
\item {Grp. gram.:M.  e  adj.}
\end{itemize}
\begin{itemize}
\item {Proveniência:(Do lat. \textunderscore succubare\textunderscore )}
\end{itemize}
Que se põe por baixo.
Dizia-se de um demónio, a cuja influência se atribuiam os sonhos maus.
\section{Sucumbido}
\begin{itemize}
\item {Grp. gram.:adj.}
\end{itemize}
Que sucumbiu.
\section{Sucumbimento}
\begin{itemize}
\item {Grp. gram.:m.}
\end{itemize}
Acto ou efeito de sucumbir.
\section{Sucumbir}
\begin{itemize}
\item {Grp. gram.:v. i.}
\end{itemize}
\begin{itemize}
\item {Proveniência:(Lat. \textunderscore succumbere\textunderscore )}
\end{itemize}
Cair debaixo.
Vergar.
Ceder aos esforços de outrem.
Sêr dominado.
Não poder resistir.
Desalentar-se.
Acabar; morrer; desaparecer.
\section{Sucursal}
\begin{itemize}
\item {Grp. gram.:adj.}
\end{itemize}
\begin{itemize}
\item {Grp. gram.:F.}
\end{itemize}
\begin{itemize}
\item {Proveniência:(Do lat. \textunderscore succursus\textunderscore )}
\end{itemize}
Diz-se de um estabelecimento dependente de outro.
Casa ou estabelecimento sucursal.
\section{Sucussão}
\begin{itemize}
\item {Grp. gram.:f.}
\end{itemize}
\begin{itemize}
\item {Proveniência:(Do lat. \textunderscore succussio\textunderscore )}
\end{itemize}
Abalo.
\section{Suçuva}
\begin{itemize}
\item {Grp. gram.:f.}
\end{itemize}
Planta brasileira, também conhecida por \textunderscore erva-grossa\textunderscore .
\section{Sucuba}
\begin{itemize}
\item {Grp. gram.:f.}
\end{itemize}
\begin{itemize}
\item {Utilização:Bras}
\end{itemize}
Árvore silvestre, que destilla, por incisão, um líquido branco, que dizem sêr vermífugo.
\section{Suculência}
\begin{itemize}
\item {Grp. gram.:f.}
\end{itemize}
Abundância de suco.
Qualidade de suculento. Cf. \textunderscore Diccion. Exeg.\textunderscore 
\section{Suculênteas}
\begin{itemize}
\item {Grp. gram.:f. pl.}
\end{itemize}
\begin{itemize}
\item {Proveniência:(De \textunderscore suculento\textunderscore )}
\end{itemize}
Ordem de plantas, que comprehende as ficoídeas, as crassuláceas e outras.
\section{Suculento}
\begin{itemize}
\item {Grp. gram.:adj.}
\end{itemize}
\begin{itemize}
\item {Proveniência:(Lat. \textunderscore suculentus\textunderscore )}
\end{itemize}
Que tem suco.
Que tem polpa.
Gordo.
Substancial.
Esponjoso e consistente, quási como a carne, (falando-se de órgãos vegetaes).
\section{Sucuri}
\begin{itemize}
\item {Grp. gram.:m.}
\end{itemize}
Espécie de cobra grande do Brasil.
\section{Sucuriju}
\begin{itemize}
\item {Grp. gram.:m.}
\end{itemize}
Espécie de cobra grande do Brasil.
\section{Sucurijuba}
\begin{itemize}
\item {Grp. gram.:f.}
\end{itemize}
Espécie de cobra grande do Brasil.
\section{Sucuriú}
\begin{itemize}
\item {Grp. gram.:m.}
\end{itemize}
Espécie de cobra grande do Brasil.
\section{Sucuriúba}
\begin{itemize}
\item {Grp. gram.:f.}
\end{itemize}
O mesmo que \textunderscore sucuriú\textunderscore .
\section{Sucuruju}
\begin{itemize}
\item {Grp. gram.:m.}
\end{itemize}
O mesmo que \textunderscore sucuriú\textunderscore .
\section{Sucurujuba}
\begin{itemize}
\item {Grp. gram.:m.}
\end{itemize}
O mesmo que \textunderscore sucuriú\textunderscore .
\section{Sucuuva}
\begin{itemize}
\item {Grp. gram.:f.}
\end{itemize}
O mesmo que \textunderscore sucuba\textunderscore .
\section{Sudação}
\begin{itemize}
\item {Grp. gram.:f.}
\end{itemize}
\begin{itemize}
\item {Proveniência:(Do lat. \textunderscore sudatio\textunderscore )}
\end{itemize}
O mesmo que \textunderscore suadoiro\textunderscore .
\section{Sudairo}
\begin{itemize}
\item {Grp. gram.:m.}
\end{itemize}
Fórma ant. de \textunderscore sudário\textunderscore .
O mesmo que \textunderscore lenço\textunderscore . Cf. Frei Fortun., \textunderscore Inéd.\textunderscore , 315.
\section{Sudâmina}
\begin{itemize}
\item {Grp. gram.:m.}
\end{itemize}
\begin{itemize}
\item {Utilização:Med.}
\end{itemize}
\begin{itemize}
\item {Proveniência:(Do lat. \textunderscore sudamen\textunderscore )}
\end{itemize}
Formação de pequenas vesículas na pelle, em consequência de transpiração abundante, como succede na febre typhóide.
\section{Sudanês}
\begin{itemize}
\item {Grp. gram.:m.}
\end{itemize}
\begin{itemize}
\item {Proveniência:(De \textunderscore Sudão\textunderscore , n. p.)}
\end{itemize}
Língua da Guiné, o mesmo que \textunderscore pepel\textunderscore .
\section{Sudário}
\begin{itemize}
\item {Grp. gram.:m.}
\end{itemize}
\begin{itemize}
\item {Proveniência:(Lat. \textunderscore sudarium\textunderscore )}
\end{itemize}
Pano, com que antigamente se limpava o suór.
Mortalha.
Pano, que representa o rosto ensanguentado de Christo.
Mortalha de Christo.
Exposição de erros ou de coisas tristes.
\section{Sudatório}
\begin{itemize}
\item {Grp. gram.:adj.}
\end{itemize}
\begin{itemize}
\item {Proveniência:(Lat. \textunderscore sudatorius\textunderscore )}
\end{itemize}
O mesmo que \textunderscore sudorífico\textunderscore .
\section{Sudoestada}
\begin{itemize}
\item {fónica:do-es}
\end{itemize}
\begin{itemize}
\item {Grp. gram.:f.}
\end{itemize}
Vento forte de Sudoéste.
\section{Sudoestar}
\begin{itemize}
\item {fónica:do-es}
\end{itemize}
\begin{itemize}
\item {Grp. gram.:v. i.}
\end{itemize}
\begin{itemize}
\item {Utilização:Náut.}
\end{itemize}
Descair o vento para Sudoéste.
\section{Sudoéste}
\begin{itemize}
\item {Grp. gram.:m.}
\end{itemize}
\begin{itemize}
\item {Grp. gram.:Adj.}
\end{itemize}
\begin{itemize}
\item {Proveniência:(De \textunderscore sud\textunderscore  al. + \textunderscore Oéste\textunderscore )}
\end{itemize}
Ponto do horizonte, a igual distância do Sul e do Oéste.
Vento, que sopra dêsse lado.
Relativo ao Sudoéste.
\section{Sudoral}
\begin{itemize}
\item {Grp. gram.:adj.}
\end{itemize}
\begin{itemize}
\item {Proveniência:(Do lat. \textunderscore sudor\textunderscore )}
\end{itemize}
Relativo a suór.
\section{Sudorífero}
\begin{itemize}
\item {Grp. gram.:adj.}
\end{itemize}
\begin{itemize}
\item {Proveniência:(Do lat. \textunderscore sudor\textunderscore  + \textunderscore ferre\textunderscore )}
\end{itemize}
O mesmo que \textunderscore sudorífico\textunderscore .
\section{Sudorífico}
\begin{itemize}
\item {Grp. gram.:adj.}
\end{itemize}
\begin{itemize}
\item {Grp. gram.:M.}
\end{itemize}
\begin{itemize}
\item {Proveniência:(Do lat. \textunderscore sudor\textunderscore  + \textunderscore facere\textunderscore )}
\end{itemize}
Que faz suar.
Aquillo que faz suar: \textunderscore tomar um sudorífico\textunderscore .
\section{Sudoríparo}
\begin{itemize}
\item {Grp. gram.:adj.}
\end{itemize}
\begin{itemize}
\item {Proveniência:(Do lat. \textunderscore sudor\textunderscore  + \textunderscore p[-a]rere\textunderscore )}
\end{itemize}
Que sua, que emitte suór.
Relativo ao suór.
\section{Sudra}
\begin{itemize}
\item {Grp. gram.:m.}
\end{itemize}
Indivíduo da classe inferior dos Índios, e que se emprega geralmente nos trabalhos mais rudes.
(Do sânscr.)
\section{Sudrar}
\begin{itemize}
\item {Grp. gram.:v. t.}
\end{itemize}
\begin{itemize}
\item {Utilização:Prov.}
\end{itemize}
\begin{itemize}
\item {Utilização:minh.}
\end{itemize}
Manchar, sujar, especialmente com substâncias gordurosas.
(Cp. \textunderscore sudro\textunderscore ^2)
\section{Sudro}
\begin{itemize}
\item {Grp. gram.:m.}
\end{itemize}
O mesmo que \textunderscore sudra\textunderscore .
\section{Sudro}
\begin{itemize}
\item {Grp. gram.:m.}
\end{itemize}
\begin{itemize}
\item {Utilização:Prov.}
\end{itemize}
\begin{itemize}
\item {Utilização:minh.}
\end{itemize}
O mesmo que \textunderscore surro\textunderscore .
\section{Sué}
\begin{itemize}
\item {Grp. gram.:m.}
\end{itemize}
Nome de muitas plantas brasileiras.
\section{Suéca}
\begin{itemize}
\item {Grp. gram.:f.  e  adj.}
\end{itemize}
\begin{itemize}
\item {Utilização:Mús.}
\end{itemize}
\begin{itemize}
\item {Proveniência:(De \textunderscore suéco\textunderscore )}
\end{itemize}
Espécie de bisca, em que cada parceiro joga com três cartas.
Diz-se de uma espécie de quadrilha, de andamento ligeiro.
\section{Suécia}
\begin{itemize}
\item {Grp. gram.:f.}
\end{itemize}
Instrumento de serralheiro.
\section{Suécio}
\begin{itemize}
\item {Grp. gram.:adj.}
\end{itemize}
O mesmo que \textunderscore suéco\textunderscore , (falando-se de uma espécie de ferro, muito malleável).
\section{Suéco}
\begin{itemize}
\item {Grp. gram.:adj.}
\end{itemize}
\begin{itemize}
\item {Grp. gram.:M.}
\end{itemize}
Relativo á Suécia.
Diz-se de uma espécie de ferro, muito malleável.
Habitante da Suécia.
Língua sueca.
\section{Suéda}
\begin{itemize}
\item {Grp. gram.:f.}
\end{itemize}
Gênero de plantas chenopódias.
\section{Sueira}
\begin{itemize}
\item {Grp. gram.:f.}
\end{itemize}
\begin{itemize}
\item {Utilização:Ant.}
\end{itemize}
Variedade de pedra preciosa.
\section{Sueste}
\begin{itemize}
\item {Grp. gram.:m.}
\end{itemize}
\begin{itemize}
\item {Grp. gram.:Adj.}
\end{itemize}
\begin{itemize}
\item {Proveniência:(De \textunderscore Sul\textunderscore  + \textunderscore Éste\textunderscore )}
\end{itemize}
Ponto do horizonte, a igual distância do Sul e do Éste.
Vento, que sopra dêsse lado.
Chapéu desabado, de oleado, próprio de marinheiro; chapéu de pano, de feitio semelhante ao daquelle.
Relativo a Sueste.
\section{Sue-sue}
\begin{itemize}
\item {Grp. gram.:m.}
\end{itemize}
Ave da África occidental.
\section{Suéto}
\begin{itemize}
\item {Grp. gram.:m.}
\end{itemize}
\begin{itemize}
\item {Utilização:Des.}
\end{itemize}
\begin{itemize}
\item {Proveniência:(Lat. \textunderscore suetus\textunderscore )}
\end{itemize}
Descanso; feriado escolar.
Costumeira, usança.
\section{Suévos}
\begin{itemize}
\item {Grp. gram.:m. pl.}
\end{itemize}
\begin{itemize}
\item {Proveniência:(Lat. \textunderscore suevi\textunderscore )}
\end{itemize}
Povo germânico, que no século V se estabeleceu na Espanha, apoderando-se da Galliza e da Lusitânia.
\section{Sufete}
\begin{itemize}
\item {Grp. gram.:m.}
\end{itemize}
\begin{itemize}
\item {Proveniência:(Do lat. \textunderscore suffes\textunderscore , \textunderscore suffetis\textunderscore )}
\end{itemize}
Cada um dos principães magistrados da antiga Cartago.
\section{Suffete}
\begin{itemize}
\item {Grp. gram.:m.}
\end{itemize}
\begin{itemize}
\item {Proveniência:(Do lat. \textunderscore suffes\textunderscore , \textunderscore suffetis\textunderscore )}
\end{itemize}
Cada um dos principães magistrados da antiga Carthago.
\section{Suffíbulo}
\begin{itemize}
\item {Grp. gram.:m.}
\end{itemize}
\begin{itemize}
\item {Proveniência:(Lat. \textunderscore suffibulum\textunderscore )}
\end{itemize}
Véu branco, seguro com um colchete, e com que as vestaes cobriam a cabeça durante os sacrifícios.
\section{Sufficiência}
\begin{itemize}
\item {Grp. gram.:f.}
\end{itemize}
\begin{itemize}
\item {Proveniência:(Lat. \textunderscore sufficientia\textunderscore )}
\end{itemize}
Qualidade do que é sufficiente.
Aptidão; habilidade.
\section{Sufficiente}
\begin{itemize}
\item {Grp. gram.:adj.}
\end{itemize}
\begin{itemize}
\item {Grp. gram.:M.}
\end{itemize}
\begin{itemize}
\item {Proveniência:(Lat. \textunderscore sufficiens\textunderscore )}
\end{itemize}
Que é bastante.
Que satisfaz o que é preciso.
Apto.
Hábil.
Capaz.
Nota, com que se designa a sufficiente applicação ou aproveitamento de um alumno.
\section{Sufficientemente}
\begin{itemize}
\item {Grp. gram.:adv.}
\end{itemize}
De modo sufficiente.
\section{Suffixativo}
\begin{itemize}
\item {fónica:csa}
\end{itemize}
\begin{itemize}
\item {Grp. gram.:adj.}
\end{itemize}
\begin{itemize}
\item {Utilização:Philol.}
\end{itemize}
Diz-se das línguas, em cuja formação entram os suffixos.
\section{Suffixo}
\begin{itemize}
\item {fónica:cso}
\end{itemize}
\begin{itemize}
\item {Grp. gram.:m.}
\end{itemize}
\begin{itemize}
\item {Proveniência:(Lat. \textunderscore suffixus\textunderscore )}
\end{itemize}
Sýllaba ou letras, que se juntam ás raízes das palavras, para lhes determinar a ideia geral ou para lhes modificar o sentido.
Desinência.
\section{Suffocação}
\begin{itemize}
\item {Grp. gram.:f.}
\end{itemize}
\begin{itemize}
\item {Proveniência:(Do lat. \textunderscore suffocatio\textunderscore )}
\end{itemize}
Acto ou effeito de suffocar.
\section{Suffocador}
\begin{itemize}
\item {Grp. gram.:m.  e  adj.}
\end{itemize}
\begin{itemize}
\item {Grp. gram.:M.}
\end{itemize}
O que suffoca.
Vaso de ferro, em que se lança o carvão, depois de sair dos carbonizadores, para que se não inflamme.
\section{Suffocamento}
\begin{itemize}
\item {Grp. gram.:m.}
\end{itemize}
O mesmo que \textunderscore suffocação\textunderscore .
\section{Suffocante}
\begin{itemize}
\item {Grp. gram.:adj.}
\end{itemize}
\begin{itemize}
\item {Proveniência:(Lat. \textunderscore suffocans\textunderscore )}
\end{itemize}
Que suffoca.
Que difficulta a respiração; suffocador: \textunderscore calor suffocante\textunderscore .
\section{Suffocar}
\begin{itemize}
\item {Grp. gram.:v. t.}
\end{itemize}
\begin{itemize}
\item {Grp. gram.:V. i.  e  p.}
\end{itemize}
\begin{itemize}
\item {Proveniência:(Lat. \textunderscore suffocare\textunderscore )}
\end{itemize}
Impedir ou reprimir a respiração de.
Afogar.
Abafar.
Tornar diffícil a respiração de.
Reprimir: \textunderscore suffocar uma revolta\textunderscore .
Deixar de respirar; respirar difficilmente.
\section{Suffocativo}
\begin{itemize}
\item {Grp. gram.:adj.}
\end{itemize}
\begin{itemize}
\item {Proveniência:(De \textunderscore suffocar\textunderscore )}
\end{itemize}
Suffocante.
Próprio para reprimir.
\section{Suffragâneo}
\begin{itemize}
\item {Grp. gram.:m.  e  adj.}
\end{itemize}
\begin{itemize}
\item {Proveniência:(Do lat. \textunderscore suffragium\textunderscore )}
\end{itemize}
O que é dependente de um metropolitano, (falando-se de Bispos e bispados).
\section{Suffragar}
\begin{itemize}
\item {Grp. gram.:v. t.}
\end{itemize}
\begin{itemize}
\item {Proveniência:(Lat. \textunderscore suffragari\textunderscore )}
\end{itemize}
Apoiar com suffrágio ou voto.
Orar pela alma de.
Applicar (esmolas, acções pias, offícios divinos, etc.), em benefício da alma de.
Supplicar.
\section{Suffrágio}
\begin{itemize}
\item {Grp. gram.:m.}
\end{itemize}
\begin{itemize}
\item {Proveniência:(Lat. \textunderscore suffragium\textunderscore )}
\end{itemize}
Voto, votação.
Apoio, adhesão.
Acto de piedade, oração ou prece, pelos mortos.
\section{Suffumigação}
\begin{itemize}
\item {Grp. gram.:f.}
\end{itemize}
\begin{itemize}
\item {Utilização:Med.}
\end{itemize}
\begin{itemize}
\item {Proveniência:(Lat. \textunderscore suffumigatio\textunderscore )}
\end{itemize}
Fumigação, que se dá por baixo de alguma coisa.
Applicação do vapor medicinal a qualquer parte do corpo.
Combustão de substâncias odoríficas, para purificar a atmosphera.
\section{Suffumígio}
\begin{itemize}
\item {Grp. gram.:m.}
\end{itemize}
O mesmo que \textunderscore suffumigação\textunderscore .
\section{Suffusão}
\begin{itemize}
\item {Grp. gram.:f.}
\end{itemize}
\begin{itemize}
\item {Utilização:Med.}
\end{itemize}
\begin{itemize}
\item {Proveniência:(Do lat. \textunderscore suffusio\textunderscore )}
\end{itemize}
Acto, pelo qual um humor, espalhando-se debaixo da pelle, se torna visível pela sua accumulação.
\section{Sufi}
\begin{itemize}
\item {Grp. gram.:m.}
\end{itemize}
Nome que, no Occidente, se dava dantes ao rei da Pérsia.
Sectário de uma escola pantheísta, entre os Muçulmanos.
(Ár. \textunderscore sufi\textunderscore )
\section{Sufíbulo}
\begin{itemize}
\item {Grp. gram.:m.}
\end{itemize}
\begin{itemize}
\item {Proveniência:(Lat. \textunderscore suffibulum\textunderscore )}
\end{itemize}
Véu branco, seguro com um colchete, e com que as vestaes cobriam a cabeça durante os sacrifícios.
\section{Suficiência}
\begin{itemize}
\item {Grp. gram.:f.}
\end{itemize}
\begin{itemize}
\item {Proveniência:(Lat. \textunderscore sufficientia\textunderscore )}
\end{itemize}
Qualidade do que é suficiente.
Aptidão; habilidade.
\section{Suficiente}
\begin{itemize}
\item {Grp. gram.:adj.}
\end{itemize}
\begin{itemize}
\item {Grp. gram.:M.}
\end{itemize}
\begin{itemize}
\item {Proveniência:(Lat. \textunderscore sufficiens\textunderscore )}
\end{itemize}
Que é bastante.
Que satisfaz o que é preciso.
Apto.
Hábil.
Capaz.
Nota, com que se designa a suficiente aplicação ou aproveitamento de um aluno.
\section{Suficientemente}
\begin{itemize}
\item {Grp. gram.:adv.}
\end{itemize}
De modo suficiente.
\section{Sufixativo}
\begin{itemize}
\item {fónica:csa}
\end{itemize}
\begin{itemize}
\item {Grp. gram.:adj.}
\end{itemize}
\begin{itemize}
\item {Utilização:Philol.}
\end{itemize}
Diz-se das línguas, em cuja formação entram os sufixos.
\section{Sufixo}
\begin{itemize}
\item {fónica:cso}
\end{itemize}
\begin{itemize}
\item {Grp. gram.:m.}
\end{itemize}
\begin{itemize}
\item {Proveniência:(Lat. \textunderscore suffixus\textunderscore )}
\end{itemize}
Sílaba ou letras, que se juntam ás raízes das palavras, para lhes determinar a ideia geral ou para lhes modificar o sentido.
Desinência.
\section{Sufocação}
\begin{itemize}
\item {Grp. gram.:f.}
\end{itemize}
\begin{itemize}
\item {Proveniência:(Do lat. \textunderscore suffocatio\textunderscore )}
\end{itemize}
Acto ou efeito de sufocar.
\section{Sufocador}
\begin{itemize}
\item {Grp. gram.:m.  e  adj.}
\end{itemize}
\begin{itemize}
\item {Grp. gram.:M.}
\end{itemize}
O que sufoca.
Vaso de ferro, em que se lança o carvão, depois de sair dos carbonizadores, para que se não inflame.
\section{Sufocamento}
\begin{itemize}
\item {Grp. gram.:m.}
\end{itemize}
O mesmo que \textunderscore sufocação\textunderscore .
\section{Sufocante}
\begin{itemize}
\item {Grp. gram.:adj.}
\end{itemize}
\begin{itemize}
\item {Proveniência:(Lat. \textunderscore suffocans\textunderscore )}
\end{itemize}
Que sufoca.
Que dificulta a respiração; sufocador: \textunderscore calor sufocante\textunderscore .
\section{Sufocar}
\begin{itemize}
\item {Grp. gram.:v. t.}
\end{itemize}
\begin{itemize}
\item {Grp. gram.:V. i.  e  p.}
\end{itemize}
\begin{itemize}
\item {Proveniência:(Lat. \textunderscore suffocare\textunderscore )}
\end{itemize}
Impedir ou reprimir a respiração de.
Afogar.
Abafar.
Tornar difícil a respiração de.
Reprimir: \textunderscore sufocar uma revolta\textunderscore .
Deixar de respirar; respirar dificilmente.
\section{Sufocativo}
\begin{itemize}
\item {Grp. gram.:adj.}
\end{itemize}
\begin{itemize}
\item {Proveniência:(De \textunderscore sufocar\textunderscore )}
\end{itemize}
Sufocante.
Próprio para reprimir.
\section{Sufradeira}
\begin{itemize}
\item {Grp. gram.:f.}
\end{itemize}
Grande argola de ferro, em que os serralheiros ou ferreiros collocam as peças, a que se têm de aperfeiçoar os encabadoiros.
\section{Sufragâneo}
\begin{itemize}
\item {Grp. gram.:m.  e  adj.}
\end{itemize}
\begin{itemize}
\item {Proveniência:(Do lat. \textunderscore suffragium\textunderscore )}
\end{itemize}
O que é dependente de um metropolitano, (falando-se de Bispos e bispados).
\section{Sufragar}
\begin{itemize}
\item {Grp. gram.:v. t.}
\end{itemize}
\begin{itemize}
\item {Proveniência:(Lat. \textunderscore suffragari\textunderscore )}
\end{itemize}
Apoiar com sufrágio ou voto.
Orar pela alma de.
Aplicar (esmolas, acções pias, ofícios divinos, etc.), em benefício da alma de.
Suplicar.
\section{Sufrágio}
\begin{itemize}
\item {Grp. gram.:m.}
\end{itemize}
\begin{itemize}
\item {Proveniência:(Lat. \textunderscore suffragium\textunderscore )}
\end{itemize}
Voto, votação.
Apoio, adesão.
Acto de piedade, oração ou prece, pelos mortos.
\section{Sufumigação}
\begin{itemize}
\item {Grp. gram.:f.}
\end{itemize}
\begin{itemize}
\item {Utilização:Med.}
\end{itemize}
\begin{itemize}
\item {Proveniência:(Lat. \textunderscore suffumigatio\textunderscore )}
\end{itemize}
Fumigação, que se dá por baixo de alguma coisa.
Aplicação do vapor medicinal a qualquer parte do corpo.
Combustão de substâncias odoríficas, para purificar a atmosfera.
\section{Sufumígio}
\begin{itemize}
\item {Grp. gram.:m.}
\end{itemize}
O mesmo que \textunderscore sufumigação\textunderscore .
\section{Sufusão}
\begin{itemize}
\item {Grp. gram.:f.}
\end{itemize}
\begin{itemize}
\item {Utilização:Med.}
\end{itemize}
\begin{itemize}
\item {Proveniência:(Do lat. \textunderscore suffusio\textunderscore )}
\end{itemize}
Acto, pelo qual um humor, espalhando-se debaixo da pele, se torna visível pela sua acumulação.
\section{Sugação}
\begin{itemize}
\item {Grp. gram.:f.}
\end{itemize}
Acto ou effeito de sugar.
\section{Sugado}
\begin{itemize}
\item {Grp. gram.:adj.}
\end{itemize}
\begin{itemize}
\item {Utilização:Prov.}
\end{itemize}
\begin{itemize}
\item {Utilização:minh.}
\end{itemize}
\begin{itemize}
\item {Proveniência:(De \textunderscore sugar\textunderscore )}
\end{itemize}
Que se sugou.
Que se extorquiu.
Decomposto, putrefacto: \textunderscore peixe sugado\textunderscore . (Colhido na Povoa de Varzim)
\section{Sugadoiro}
\begin{itemize}
\item {Grp. gram.:m.}
\end{itemize}
\begin{itemize}
\item {Utilização:Zool.}
\end{itemize}
\begin{itemize}
\item {Proveniência:(De \textunderscore sugar\textunderscore )}
\end{itemize}
Espécie de tromba de alguns insectos, ou bôca em fórma de tromba, com que alguns animálculos sugam o sangue ou outros líquidos.
\section{Sugador}
\begin{itemize}
\item {Grp. gram.:m.  e  adj.}
\end{itemize}
\begin{itemize}
\item {Grp. gram.:M.}
\end{itemize}
O que suga.
O mesmo que \textunderscore sugadoiro\textunderscore .
\section{Sugadouro}
\begin{itemize}
\item {Grp. gram.:m.}
\end{itemize}
\begin{itemize}
\item {Utilização:Zool.}
\end{itemize}
\begin{itemize}
\item {Proveniência:(De \textunderscore sugar\textunderscore )}
\end{itemize}
Espécie de tromba de alguns insectos, ou bôca em fórma de tromba, com que alguns animálculos sugam o sangue ou outros líquidos.
\section{Sugar}
\begin{itemize}
\item {Grp. gram.:v. t.}
\end{itemize}
\begin{itemize}
\item {Proveniência:(Lat. \textunderscore sugare\textunderscore )}
\end{itemize}
O mesmo que \textunderscore chupar\textunderscore .
Extrahir.
Subtrahir com fraude, extorquir.
\section{Sugarda}
\begin{itemize}
\item {Grp. gram.:f.}
\end{itemize}
O mesmo que \textunderscore suarda\textunderscore . Cf. \textunderscore Inquér. Industr.\textunderscore , 2.^a p., l. II, 113.
\section{Sugeridor}
\begin{itemize}
\item {Grp. gram.:adj.}
\end{itemize}
Que sugere.
\section{Sugerir}
\begin{itemize}
\item {Grp. gram.:v. t.}
\end{itemize}
\begin{itemize}
\item {Proveniência:(Do lat. \textunderscore suggere\textunderscore )}
\end{itemize}
Proporcionar, fornecer: \textunderscore sugerir meios de resistência\textunderscore .
Ocasionar, sêr causa moral de.
Fazer nascer no espírito; insinuar; inspirar: \textunderscore sugerir ideias ruins\textunderscore .
Lembrar.
Dizer a meia voz ou em segrêdo.
Promover.
\section{Sugestão}
\begin{itemize}
\item {Grp. gram.:f.}
\end{itemize}
\begin{itemize}
\item {Proveniência:(Do lat. \textunderscore suggestio\textunderscore )}
\end{itemize}
Acto ou efeito sugerir.
Inspiração; estímulo, instigação.
\section{Sugestibilidade}
\begin{itemize}
\item {Grp. gram.:f.}
\end{itemize}
\begin{itemize}
\item {Utilização:Med.}
\end{itemize}
Qualidade de sugestível.
Disposição ou aptidão, que alguém tem, para sêr influenciado, por uma ideia recebida pelo cérebro, e para a realizar.
\section{Sugestionar}
\begin{itemize}
\item {Grp. gram.:v. t.}
\end{itemize}
Produzir sugestão em.
Estimular.
Inspirar.
\section{Sugestionável}
\begin{itemize}
\item {Grp. gram.:adj.}
\end{itemize}
Que se póde sugestionar.
\section{Sugestível}
\begin{itemize}
\item {Grp. gram.:adj.}
\end{itemize}
Que póde sêr sugestionado ou influenciado.
(Cp. \textunderscore sugestivo\textunderscore )
\section{Sugestivo}
\begin{itemize}
\item {Grp. gram.:adj.}
\end{itemize}
\begin{itemize}
\item {Proveniência:(Do lat. \textunderscore suggestus\textunderscore )}
\end{itemize}
Que sugere.
\section{Sugesto}
\begin{itemize}
\item {Grp. gram.:m.}
\end{itemize}
\begin{itemize}
\item {Proveniência:(Lat. \textunderscore suggestus\textunderscore )}
\end{itemize}
Lugar alto ou tribuna, donde os oradores romanos falavam ao povo.
\section{Suggeridor}
\begin{itemize}
\item {Grp. gram.:adj.}
\end{itemize}
Que suggere.
\section{Suggerir}
\begin{itemize}
\item {Grp. gram.:v. t.}
\end{itemize}
\begin{itemize}
\item {Proveniência:(Do lat. \textunderscore suggere\textunderscore )}
\end{itemize}
Proporcionar, fornecer: \textunderscore suggerir meios de resistência\textunderscore .
Occasionar, sêr causa moral de.
Fazer nascer no espírito; insinuar; inspirar: \textunderscore suggerir ideias ruins\textunderscore .
Lembrar.
Dizer a meia voz ou em segrêdo.
Promover.
\section{Suggestão}
\begin{itemize}
\item {Grp. gram.:f.}
\end{itemize}
\begin{itemize}
\item {Proveniência:(Do lat. \textunderscore suggestio\textunderscore )}
\end{itemize}
Acto ou effeito suggerir.
Inspiração; estímulo, instigação.
\section{Suggestibilidade}
\begin{itemize}
\item {Grp. gram.:f.}
\end{itemize}
\begin{itemize}
\item {Utilização:Med.}
\end{itemize}
Qualidade de suggestível.
Disposição ou aptidão, que alguém tem, para sêr influenciado, por uma ideia recebida pelo cérebro, e para a realizar.
\section{Suggestionar}
\begin{itemize}
\item {Grp. gram.:v. t.}
\end{itemize}
Produzir suggestão em.
Estimular.
Inspirar.
\section{Suggestionável}
\begin{itemize}
\item {Grp. gram.:adj.}
\end{itemize}
Que se póde suggestionar.
\section{Suggestível}
\begin{itemize}
\item {Grp. gram.:adj.}
\end{itemize}
Que póde sêr suggestionado ou influenciado.
(Cp. \textunderscore suggestivo\textunderscore )
\section{Suggestivo}
\begin{itemize}
\item {Grp. gram.:adj.}
\end{itemize}
\begin{itemize}
\item {Proveniência:(Do lat. \textunderscore suggestus\textunderscore )}
\end{itemize}
Que suggere.
\section{Suggesto}
\begin{itemize}
\item {Grp. gram.:m.}
\end{itemize}
\begin{itemize}
\item {Proveniência:(Lat. \textunderscore suggestus\textunderscore )}
\end{itemize}
Lugar alto ou tribuna, donde os oradores romanos falavam ao povo.
\section{Sugigola}
\begin{itemize}
\item {Grp. gram.:f.}
\end{itemize}
Correia que, fazendo parte da cabeçada, passa por baixo do queixo do animal. Cf. Andrade, \textunderscore Arte de Cavall.\textunderscore 
\section{Sugilação}
\begin{itemize}
\item {Grp. gram.:f.}
\end{itemize}
\begin{itemize}
\item {Proveniência:(Do lat. \textunderscore sugilatio\textunderscore )}
\end{itemize}
Acto ou effeito de sugilar.
Leve echymose cutânea.
Lividez cadavérica.
\section{Sugilar}
\begin{itemize}
\item {Grp. gram.:v. t.}
\end{itemize}
\begin{itemize}
\item {Utilização:Fig.}
\end{itemize}
\begin{itemize}
\item {Proveniência:(Lat. \textunderscore sugilare\textunderscore )}
\end{itemize}
Produzir echymose em; contundir.
Manchar; infamar.
\section{Sugo}
\begin{itemize}
\item {Grp. gram.:m.}
\end{itemize}
\begin{itemize}
\item {Utilização:T. de Ílhavo}
\end{itemize}
\begin{itemize}
\item {Utilização:Prov.}
\end{itemize}
O mesmo que \textunderscore suarda\textunderscore .
Líquidos ou despejos, que correm nas valetas das ruas.
Líquidos que ficam por baixo das latrinas antigas, depois de tirado o estrume.
Excesso de gordura que, para a fabricação do queijo, se tira ao leite, fazendo-o passar por varias coadeiras.
(Relaciona-se com \textunderscore suco\textunderscore ?)
\section{Sugumburno}
\begin{itemize}
\item {Grp. gram.:m.}
\end{itemize}
Pássaro da África occidental.
\section{Suíça}
\begin{itemize}
\item {Grp. gram.:f.}
\end{itemize}
\begin{itemize}
\item {Proveniência:(De \textunderscore suíço\textunderscore )}
\end{itemize}
Parte da barba, que se deixou crescer nas partes lateraes das faces.
Guarda de espingardeiros, criada por Affonso de Albuquerque na Índia.
\section{Suícero}
\begin{itemize}
\item {Grp. gram.:m.}
\end{itemize}
\begin{itemize}
\item {Utilização:Ant.}
\end{itemize}
\begin{itemize}
\item {Proveniência:(It. \textunderscore suizzero\textunderscore )}
\end{itemize}
O mesmo que \textunderscore suíço\textunderscore . Cf. \textunderscore Viriato Trág.\textunderscore , II, 32.
\section{Suicida}
\begin{itemize}
\item {Grp. gram.:m.  e  f.}
\end{itemize}
\begin{itemize}
\item {Grp. gram.:Adj.}
\end{itemize}
Pessôa, que se mata a si própria.
Que serviu de instrumento de suicídio: \textunderscore a arma suicida\textunderscore .
(Cp. \textunderscore suicidar-se\textunderscore )
\section{Suicidar-se}
\begin{itemize}
\item {fónica:su-i}
\end{itemize}
\begin{itemize}
\item {Grp. gram.:v. p.}
\end{itemize}
\begin{itemize}
\item {Utilização:Fig.}
\end{itemize}
\begin{itemize}
\item {Proveniência:(Do lat. \textunderscore sui\textunderscore  + \textunderscore caedere\textunderscore )}
\end{itemize}
Causar a morte a si próprio.
Arruinar-se por culpa própria.
\section{Suicídio}
\begin{itemize}
\item {fónica:su-i}
\end{itemize}
\begin{itemize}
\item {Grp. gram.:m.}
\end{itemize}
\begin{itemize}
\item {Utilização:Fig.}
\end{itemize}
Acto ou effeito de suicidar-se.
Ruína ou desgraça, que se procura espontaneamente ou por falta de bom juízo.
\section{Suíço}
\begin{itemize}
\item {Grp. gram.:adj.}
\end{itemize}
\begin{itemize}
\item {Grp. gram.:M.}
\end{itemize}
Relativo á Suíça.
Aquelle que é natural da Suíça.
(Cp. port. e cast. ant. \textunderscore suíço\textunderscore , cast. mod. \textunderscore suizo\textunderscore )
\section{Suídeos}
\begin{itemize}
\item {Grp. gram.:m. pl.}
\end{itemize}
\begin{itemize}
\item {Proveniência:(Do lat. \textunderscore sus\textunderscore  + gr. \textunderscore eidos\textunderscore )}
\end{itemize}
Animaes da família do porco.
\section{Suindara}
\begin{itemize}
\item {Grp. gram.:f.}
\end{itemize}
\begin{itemize}
\item {Utilização:Bras}
\end{itemize}
O mesmo que \textunderscore coruja\textunderscore .
\section{Suinicida}
\begin{itemize}
\item {fónica:su-i}
\end{itemize}
\begin{itemize}
\item {Grp. gram.:m.}
\end{itemize}
\begin{itemize}
\item {Grp. gram.:Adj.}
\end{itemize}
\begin{itemize}
\item {Proveniência:(Do lat. \textunderscore suinus\textunderscore  + \textunderscore caedere\textunderscore )}
\end{itemize}
Aquelle que mata porcos.
Próprio de matador de porcos.
Próprio de magarefe:«\textunderscore e fazia gesticulações suinicidas...\textunderscore »Camillo, \textunderscore Volcoens\textunderscore , 56.
\section{Suinicídio}
\begin{itemize}
\item {fónica:su-i}
\end{itemize}
\begin{itemize}
\item {Grp. gram.:m.}
\end{itemize}
Acto de matar um porco.
(Cp. \textunderscore suinicida\textunderscore )
\section{Suíno}
\begin{itemize}
\item {Grp. gram.:adj.}
\end{itemize}
\begin{itemize}
\item {Grp. gram.:M.}
\end{itemize}
\begin{itemize}
\item {Proveniência:(Lat. \textunderscore suinus\textunderscore )}
\end{itemize}
Relativo a porcos.
O porco.
\section{Suinofobia}
\begin{itemize}
\item {fónica:su-i}
\end{itemize}
\begin{itemize}
\item {Grp. gram.:f.}
\end{itemize}
\begin{itemize}
\item {Proveniência:(Do lat. \textunderscore suinus\textunderscore  + gr. \textunderscore phobein\textunderscore )}
\end{itemize}
Aversão aos porcos.
\section{Suinophobia}
\begin{itemize}
\item {Grp. gram.:f.}
\end{itemize}
\begin{itemize}
\item {Proveniência:(Do lat. \textunderscore suinus\textunderscore  + gr. \textunderscore phobein\textunderscore )}
\end{itemize}
Aversão aos porcos.
\section{Suíssa}
\begin{itemize}
\item {Grp. gram.:f.}
\end{itemize}
(V. \textunderscore suíça\textunderscore ^1, que é a graphia exacta)
\section{Suísso}
\begin{itemize}
\item {Grp. gram.:m.  e  adj.}
\end{itemize}
(V. \textunderscore suíço\textunderscore , que é a graphia exacta)
\section{Suízaro}
\begin{itemize}
\item {Grp. gram.:m.}
\end{itemize}
O mesmo que \textunderscore suícero\textunderscore . Cf. M. Bernardez.
\section{Sujamente}
\begin{itemize}
\item {Grp. gram.:adv.}
\end{itemize}
De modo sujo.
Porcamente.
\section{Sujar}
\begin{itemize}
\item {Grp. gram.:v. t.}
\end{itemize}
\begin{itemize}
\item {Grp. gram.:V. i.}
\end{itemize}
\begin{itemize}
\item {Grp. gram.:V. p.}
\end{itemize}
\begin{itemize}
\item {Utilização:Fig.}
\end{itemize}
\begin{itemize}
\item {Proveniência:(De \textunderscore sujo\textunderscore )}
\end{itemize}
Tornar sujo; manchar.
Fazer dejecções.
Praticar actos infamantes.
\section{Sujeição}
\begin{itemize}
\item {Grp. gram.:f.}
\end{itemize}
\begin{itemize}
\item {Proveniência:(Do lat. \textunderscore subjectio\textunderscore )}
\end{itemize}
Acto ou effeito de sujeitar.
Estado do que está sujeito; dependência.
\section{Sujeita}
\begin{itemize}
\item {Grp. gram.:f.}
\end{itemize}
(Fem. de \textunderscore sujeito\textunderscore )
\section{Sujeitador}
\begin{itemize}
\item {Grp. gram.:m.  e  adj.}
\end{itemize}
O que sujeita.
\section{Sujeitar}
\begin{itemize}
\item {Grp. gram.:v. t.}
\end{itemize}
\begin{itemize}
\item {Grp. gram.:V. p.}
\end{itemize}
\begin{itemize}
\item {Proveniência:(Do lat. \textunderscore subjectare\textunderscore )}
\end{itemize}
Pôr debaixo.
Dominar.
Subjugar.
Tornar dependente.
Obrigar.
Arriscar, aventurar.
Tornar firme, immobilizar.
Submeter-se; conformar-se, obedecendo.
\section{Sujeitável}
\begin{itemize}
\item {Grp. gram.:adj.}
\end{itemize}
Que se póde sujeitar.
\section{Sujeito}
\begin{itemize}
\item {Grp. gram.:adj.}
\end{itemize}
\begin{itemize}
\item {Grp. gram.:M.}
\end{itemize}
\begin{itemize}
\item {Utilização:Gram.}
\end{itemize}
\begin{itemize}
\item {Proveniência:(Do lat. \textunderscore subjectus\textunderscore )}
\end{itemize}
Escravizado.
Obediente.
Adstricto.
Que não tem vontade própria.
Disposto naturalmente, habituado.
Exposto, arriscado: \textunderscore sujeito a perder a vida\textunderscore .
Pessôa ou coisa, que produz ou determina a acção expressa por um verbo.
Indivíduo, de quem se omitte o nome; homem: \textunderscore vai alli um sujeito\textunderscore .
Súbdito.
O mesmo que \textunderscore assumpto\textunderscore :«\textunderscore ...escrever quanto o auto sujeito da obra o merece\textunderscore ». Usque. Cf. Sousa, \textunderscore Vida do Arceb.\textunderscore , II, 203.
\section{Sujeitório}
\begin{itemize}
\item {Grp. gram.:m.}
\end{itemize}
\begin{itemize}
\item {Utilização:Deprec.}
\end{itemize}
\begin{itemize}
\item {Proveniência:(De \textunderscore sujeito\textunderscore )}
\end{itemize}
Indivíduo sem importância, muito ordinário, reles. Cf. Camillo, \textunderscore Corja\textunderscore , 256; \textunderscore Quéda\textunderscore , 92.
\section{Sujidade}
\begin{itemize}
\item {Grp. gram.:f.}
\end{itemize}
Qualidade do que é sujo.
Excrementos.
\section{Sujinada}
\begin{itemize}
\item {Grp. gram.:f.}
\end{itemize}
\begin{itemize}
\item {Utilização:Ant.}
\end{itemize}
\begin{itemize}
\item {Proveniência:(De \textunderscore sujo\textunderscore ? Ou alter, de \textunderscore suinada\textunderscore , de \textunderscore suíno\textunderscore ?)}
\end{itemize}
O mesmo que \textunderscore sujidade\textunderscore . Cf. \textunderscore Anat. Joc.\textunderscore , 107.
\section{Sujo}
\begin{itemize}
\item {Grp. gram.:adj.}
\end{itemize}
\begin{itemize}
\item {Utilização:Fig.}
\end{itemize}
\begin{itemize}
\item {Grp. gram.:M.}
\end{itemize}
\begin{itemize}
\item {Utilização:Bras. de Minas}
\end{itemize}
\begin{itemize}
\item {Utilização:Pop.}
\end{itemize}
Emporcalhado; sórdido; que não está limpo.
Indecoroso.
Deshonesto.
Maculado.
O mesmo que \textunderscore satanás\textunderscore .
\textunderscore O porco sujo\textunderscore , o demónio.
\section{Sul}
\begin{itemize}
\item {Grp. gram.:m.}
\end{itemize}
\begin{itemize}
\item {Grp. gram.:Adj.}
\end{itemize}
\begin{itemize}
\item {Proveniência:(Al. \textunderscore sud\textunderscore )}
\end{itemize}
A parte do mundo, opposta ao Norte.
Pólo austral.
Parte de uma região ou de um continente, que, em relação ás outras partes, fica mais perto daquelle pólo.
Vento, que sopra do Sul para o lado do Norte.
Relativo ao Sul.
\section{Sula}
\begin{itemize}
\item {Grp. gram.:f.}
\end{itemize}
\begin{itemize}
\item {Utilização:Bras. do N}
\end{itemize}
Acto em que duas pessôas manejam outras tantas mãos do mesmo gral para activar a trituração de qualquer gênero.
\section{Sula}
\begin{itemize}
\item {Grp. gram.:f.}
\end{itemize}
\begin{itemize}
\item {Utilização:Prov.}
\end{itemize}
\begin{itemize}
\item {Utilização:trasm.}
\end{itemize}
O mesmo que \textunderscore enxó\textunderscore ^1.
\section{Sulamba}
\begin{itemize}
\item {Grp. gram.:m. ,  f.  e  adj.}
\end{itemize}
\begin{itemize}
\item {Utilização:Bras}
\end{itemize}
O mesmo que \textunderscore samango\textunderscore .
\section{Sulano}
\begin{itemize}
\item {Grp. gram.:adj.}
\end{itemize}
Relativo a San-Pedro-do-Sul.
Diz-se especialmente de uma raça bovina da região de Lafões. Cf. \textunderscore Port. au point de vue agr.\textunderscore , 230 e 252.
\section{Sulaventear}
\begin{itemize}
\item {Grp. gram.:v. i.}
\end{itemize}
Navegar para sulavento.
\section{Sulavento}
\begin{itemize}
\item {Grp. gram.:m.}
\end{itemize}
\begin{itemize}
\item {Proveniência:(De \textunderscore Sul\textunderscore  + \textunderscore vento\textunderscore )}
\end{itemize}
O mesmo que \textunderscore sotavento\textunderscore .
\section{Sulcador}
\begin{itemize}
\item {Grp. gram.:adj.}
\end{itemize}
Que sulca ou lavra:«\textunderscore o ferro sulcador\textunderscore ». Castilho, \textunderscore Metam.\textunderscore , 117.
\section{Sulcar}
\begin{itemize}
\item {Grp. gram.:v. t.}
\end{itemize}
\begin{itemize}
\item {Proveniência:(Lat. \textunderscore sulcare\textunderscore )}
\end{itemize}
Fazer sulcos em; cortar as águas de, navegar por.
Enrugar.
\section{Sulco}
\begin{itemize}
\item {Grp. gram.:m.}
\end{itemize}
\begin{itemize}
\item {Proveniência:(Lat. \textunderscore sulcus\textunderscore )}
\end{itemize}
Rêgo, aberto pelo arado.
Depressão, que um navio faz nas águas, cortando-as; ruga.
\section{Sulco}
\begin{itemize}
\item {Grp. gram.:m.}
\end{itemize}
\begin{itemize}
\item {Utilização:Prov.}
\end{itemize}
\begin{itemize}
\item {Utilização:trasm.}
\end{itemize}
O mesmo que \textunderscore suco\textunderscore ^1.
\section{Suleiro}
\begin{itemize}
\item {Grp. gram.:m.  e  adj.}
\end{itemize}
\begin{itemize}
\item {Utilização:Bras}
\end{itemize}
\begin{itemize}
\item {Proveniência:(De \textunderscore sul\textunderscore )}
\end{itemize}
Habitante dos Estados brasileiros do Sul, (por opposição a \textunderscore nortista\textunderscore ).
\section{Sulfácido}
\begin{itemize}
\item {Grp. gram.:m.}
\end{itemize}
\begin{itemize}
\item {Proveniência:(De \textunderscore sulfo...\textunderscore  + \textunderscore ácido\textunderscore )}
\end{itemize}
Sulfureto, que numa combinação chímica serve de ácido.
\section{Sulfantimónico}
\begin{itemize}
\item {Grp. gram.:adj.}
\end{itemize}
\begin{itemize}
\item {Proveniência:(De \textunderscore sulfo...\textunderscore  + \textunderscore antimónio\textunderscore )}
\end{itemize}
Em que há combinação ácida de enxôfre com antimónio.
\section{Sulfarsênico}
\begin{itemize}
\item {Grp. gram.:adj.}
\end{itemize}
\begin{itemize}
\item {Proveniência:(De \textunderscore sulfo...\textunderscore  + \textunderscore arsênico\textunderscore )}
\end{itemize}
Diz-se da combinação ácida do enxôfre com o arsênico.
\section{Sulfatagem}
\begin{itemize}
\item {Grp. gram.:f.}
\end{itemize}
Acto de sulfatar.
\section{Sulfatar}
\begin{itemize}
\item {Grp. gram.:v. t.}
\end{itemize}
Impregnar de sulfato metállico.
Aspergir uma solução de sulfato metállico em (videiras ou outras plantas), contra certas moléstias.
\section{Sulfatização}
\begin{itemize}
\item {Grp. gram.:f.}
\end{itemize}
Acto ou effeito de sulfatizar.
\section{Sulfatizar}
\begin{itemize}
\item {Grp. gram.:v. t.}
\end{itemize}
Converter em sulfato.
\section{Sulfato}
\begin{itemize}
\item {Grp. gram.:m.}
\end{itemize}
\begin{itemize}
\item {Proveniência:(Do lat. \textunderscore sulfur\textunderscore )}
\end{itemize}
Sal, resultante da combinação do ácido sulfúrico com uma base.
\section{Sulfhýdrico}
\begin{itemize}
\item {Grp. gram.:adj.}
\end{itemize}
\begin{itemize}
\item {Grp. gram.:M.}
\end{itemize}
\begin{itemize}
\item {Proveniência:(De \textunderscore sulfo...\textunderscore  + \textunderscore hýdrico\textunderscore )}
\end{itemize}
Diz-se de um ácido, formado de enxôfre e hydrogênio.
Ácido sulfhýdrico.
\section{Sulfhydrometria}
\begin{itemize}
\item {Grp. gram.:f.}
\end{itemize}
\begin{itemize}
\item {Proveniência:(De \textunderscore sulfhydrómetro\textunderscore )}
\end{itemize}
Méthodo de anályse, com que se avalia o enxôfre contido em águas sulfúreas.
\section{Sulfhydrométrico}
\begin{itemize}
\item {Grp. gram.:adj.}
\end{itemize}
Relativo á sulfhydrometria.
\section{Sulfhydrómetro}
\begin{itemize}
\item {Grp. gram.:m.}
\end{itemize}
\begin{itemize}
\item {Proveniência:(De \textunderscore sulfo...\textunderscore  + \textunderscore hydrómetro\textunderscore )}
\end{itemize}
Instrumento, com que se pratica a sulfhydrometria.
\section{Sulfídrico}
\begin{itemize}
\item {Grp. gram.:adj.}
\end{itemize}
\begin{itemize}
\item {Grp. gram.:M.}
\end{itemize}
\begin{itemize}
\item {Proveniência:(De \textunderscore sulfo...\textunderscore  + \textunderscore hídrico\textunderscore )}
\end{itemize}
Diz-se de um ácido, formado de enxôfre e hidrogênio.
Ácido sulfídrico.
\section{Sulfidrometria}
\begin{itemize}
\item {Grp. gram.:f.}
\end{itemize}
\begin{itemize}
\item {Proveniência:(De \textunderscore sulfidrómetro\textunderscore )}
\end{itemize}
Método de análise, com que se avalia o enxôfre contido em águas sulfúreas.
\section{Sulfidrométrico}
\begin{itemize}
\item {Grp. gram.:adj.}
\end{itemize}
Relativo á sulfidrometria.
\section{Sulfidrómetro}
\begin{itemize}
\item {Grp. gram.:m.}
\end{itemize}
\begin{itemize}
\item {Proveniência:(De \textunderscore sulfo...\textunderscore  + \textunderscore hidrómetro\textunderscore )}
\end{itemize}
Instrumento, com que se pratica a sulfidrometria.
\section{Sulfina}
\begin{itemize}
\item {Grp. gram.:f.}
\end{itemize}
\begin{itemize}
\item {Proveniência:(Do lat. \textunderscore sulfur\textunderscore )}
\end{itemize}
Preparação de enxôfre e de outras substâncias, applicável contra os insectos nocivos á vegetação.
\section{Sulfito}
\begin{itemize}
\item {Grp. gram.:m.}
\end{itemize}
\begin{itemize}
\item {Proveniência:(Do lat. \textunderscore sulfur\textunderscore )}
\end{itemize}
Sal, resultante da combinação do ácido sulfúrico com uma base.
\section{Sulfo...}
\begin{itemize}
\item {Grp. gram.:pref.}
\end{itemize}
\begin{itemize}
\item {Proveniência:(Lat. \textunderscore sulfur\textunderscore )}
\end{itemize}
(designativo de \textunderscore enxôfre\textunderscore )
\section{Sulfobase}
\begin{itemize}
\item {Grp. gram.:f.}
\end{itemize}
\begin{itemize}
\item {Proveniência:(De \textunderscore sulfo...\textunderscore  + \textunderscore base\textunderscore )}
\end{itemize}
Sulfureto, que serve de base a uma combinação.
\section{Sulfocarbonato}
\begin{itemize}
\item {Grp. gram.:m.}
\end{itemize}
\begin{itemize}
\item {Proveniência:(De \textunderscore sulfòcarbónico\textunderscore )}
\end{itemize}
Sal, resultante da combinação do ácido sulfocarbónico com uma base.
\section{Sulfòcarbónico}
\begin{itemize}
\item {Grp. gram.:adj.}
\end{itemize}
\begin{itemize}
\item {Proveniência:(De \textunderscore sulfo...\textunderscore  + \textunderscore carbónico\textunderscore )}
\end{itemize}
Relativo a enxôfre e carbóneo.
\section{Sulfonaftaleico}
\begin{itemize}
\item {Grp. gram.:adj.}
\end{itemize}
Diz-se de um ácido, que é um composto orgânico sulfurado de naftalina com propriedades ácidas.
\section{Sulfonal}
\begin{itemize}
\item {Grp. gram.:m.}
\end{itemize}
Medicamento hypnótico.
\section{Sulfonaphtaleico}
\begin{itemize}
\item {Grp. gram.:adj.}
\end{itemize}
Diz-se de um ácido, que é um composto orgânico sulfurado de naphtalina com propriedades ácidas.
\section{Sulfosal}
\begin{itemize}
\item {fónica:sal}
\end{itemize}
\begin{itemize}
\item {Grp. gram.:m.}
\end{itemize}
\begin{itemize}
\item {Proveniência:(De \textunderscore sulfo...\textunderscore  + \textunderscore sal\textunderscore )}
\end{itemize}
Sal, resultante da combinação de um sulfácido com uma sulfòbase.
\section{Sulfossal}
\begin{itemize}
\item {Grp. gram.:m.}
\end{itemize}
\begin{itemize}
\item {Proveniência:(De \textunderscore sulfo...\textunderscore  + \textunderscore sal\textunderscore )}
\end{itemize}
Sal, resultante da combinação de um sulfácido com uma sulfòbase.
\section{Sulfòsteatite}
\begin{itemize}
\item {Grp. gram.:f.}
\end{itemize}
\begin{itemize}
\item {Proveniência:(Do lat. \textunderscore sulfur\textunderscore  + gr. \textunderscore osteon\textunderscore )}
\end{itemize}
Preparação chímica, em que entra enxôfre e ossos pulverizados, e que serve para tratamento das vinhas.
\section{Súlfur}
\begin{itemize}
\item {Grp. gram.:m.}
\end{itemize}
\begin{itemize}
\item {Proveniência:(Lat. \textunderscore sulfur\textunderscore )}
\end{itemize}
Solução medicamentosa de enxôfre, obtida pela homopathia.
\section{Sulfuração}
\begin{itemize}
\item {Grp. gram.:f.}
\end{itemize}
Acto de sulfurar.
\section{Sulfurador}
\begin{itemize}
\item {Grp. gram.:m.}
\end{itemize}
Instrumento, para sulfurar o vinho. Cf. \textunderscore Techn. Rur.\textunderscore , 249.
\section{Sulfurar}
\begin{itemize}
\item {Grp. gram.:v. t.}
\end{itemize}
\begin{itemize}
\item {Proveniência:(Do lat. \textunderscore sulfur\textunderscore )}
\end{itemize}
Combinar ou misturar com enxôfre; enxofrar.
\section{Sulfurária}
\begin{itemize}
\item {Grp. gram.:f.}
\end{itemize}
\begin{itemize}
\item {Proveniência:(De \textunderscore súlfur\textunderscore )}
\end{itemize}
Alga microscópica, peculiar ás águas thermaes sulfurosas.
\section{Sulfurável}
\begin{itemize}
\item {Grp. gram.:adj.}
\end{itemize}
Que se póde sulfurar.
\section{Sulfúreo}
\begin{itemize}
\item {Grp. gram.:adj.}
\end{itemize}
\begin{itemize}
\item {Proveniência:(Lat. \textunderscore sulfureus\textunderscore )}
\end{itemize}
Que tem a natureza de enxôfre.
\section{Sulfuretar}
\begin{itemize}
\item {Grp. gram.:v. t.}
\end{itemize}
Juntar com sulfureto.
\section{Sulfureto}
\begin{itemize}
\item {fónica:furê}
\end{itemize}
\begin{itemize}
\item {Grp. gram.:m.}
\end{itemize}
\begin{itemize}
\item {Utilização:Chím.}
\end{itemize}
\begin{itemize}
\item {Proveniência:(Do lat. \textunderscore sulfur\textunderscore )}
\end{itemize}
Designação genérica dos compostos binários, formados pelo enxôfre com os metaes e alguns metallóides.
\section{Sulfúrico}
\begin{itemize}
\item {Grp. gram.:adj.}
\end{itemize}
\begin{itemize}
\item {Proveniência:(Do lat. \textunderscore sulfur\textunderscore )}
\end{itemize}
Relativo ao enxôfre.
Diz-se do ácido, que resulta da combinação do enxôfre com o oxygênio.
\section{Sulfurino}
\begin{itemize}
\item {Grp. gram.:adj.}
\end{itemize}
\begin{itemize}
\item {Grp. gram.:M. pl.}
\end{itemize}
\begin{itemize}
\item {Utilização:Miner.}
\end{itemize}
\begin{itemize}
\item {Proveniência:(De \textunderscore súlfur\textunderscore )}
\end{itemize}
Que tem côr de enxôfre.
Uma das quatro ordens, em que se divide a classe dos oxysaes, e na qual se inclue o gêsso, a baryte e o alúmen.
\section{Sulfuroso}
\begin{itemize}
\item {Grp. gram.:adj.}
\end{itemize}
\begin{itemize}
\item {Proveniência:(Lat. \textunderscore sulfurosus\textunderscore )}
\end{itemize}
O mesmo que \textunderscore sulfúreo\textunderscore .
Diz-se de um ácido, que resulta da combustão do enxôfre.
\section{Sulifrate}
\begin{itemize}
\item {Grp. gram.:m.}
\end{itemize}
\begin{itemize}
\item {Utilização:Prov.}
\end{itemize}
\begin{itemize}
\item {Utilização:beir.}
\end{itemize}
O mesmo que \textunderscore sulfato\textunderscore .
\section{Sulipa}
\begin{itemize}
\item {Grp. gram.:f.}
\end{itemize}
\begin{itemize}
\item {Utilização:Gír.}
\end{itemize}
O mesmo que \textunderscore chulipa\textunderscore ^2. Cf. Camillo, \textunderscore Sc. da Foz\textunderscore , 87.
\section{Sulista}
\begin{itemize}
\item {Grp. gram.:m.  e  f.}
\end{itemize}
\begin{itemize}
\item {Utilização:Bras}
\end{itemize}
\begin{itemize}
\item {Grp. gram.:Adj.}
\end{itemize}
\begin{itemize}
\item {Proveniência:(De \textunderscore sul\textunderscore )}
\end{itemize}
Indivíduo, natural do sul do Brasil.
Relativo ao sul do Brasil. Cf. Júl. Ribeiro, \textunderscore Diccion. Gram.\textunderscore , 65.
\section{Sulivântia}
\begin{itemize}
\item {Grp. gram.:f.}
\end{itemize}
Gênero de plantas saxifragáceas.
\section{Sullivântia}
\begin{itemize}
\item {Grp. gram.:f.}
\end{itemize}
Gênero de plantas saxifragáceas.
\section{Sulmonense}
\begin{itemize}
\item {Grp. gram.:adj.}
\end{itemize}
Natural de Sulmona:«\textunderscore o sulmonense Ovídio desterrado...\textunderscore »\textunderscore Lusíadas\textunderscore , III, 157.
\section{Sulo}
\begin{itemize}
\item {Grp. gram.:adj.}
\end{itemize}
\begin{itemize}
\item {Utilização:Prov.}
\end{itemize}
\begin{itemize}
\item {Utilização:minh.}
\end{itemize}
O mesmo que \textunderscore suro\textunderscore  ou antes \textunderscore çuro\textunderscore .
\section{Sulpicianos}
\begin{itemize}
\item {Grp. gram.:m. pl.}
\end{itemize}
Congregação religiosa, fundada em França em 1642.
\section{Sultana}
\begin{itemize}
\item {Grp. gram.:f.}
\end{itemize}
\begin{itemize}
\item {Proveniência:(De \textunderscore sultão\textunderscore )}
\end{itemize}
Mulhér ou filha de Sultão.
A odalisca ou amásia do Sultão, que dêlle teve algum filho.
Navio de guerra, entre os Turcos. Cf. Camillo, \textunderscore Caveira\textunderscore , 84.
Fita ou faixa que as mulheres espanholas usavam ao pescoço, como enfeite.
Nome de uma ave gallinácea.
\section{Sultanado}
\begin{itemize}
\item {Grp. gram.:m.}
\end{itemize}
\begin{itemize}
\item {Proveniência:(Do b. lat. \textunderscore sultanus\textunderscore )}
\end{itemize}
Dignidade do Sultão.
País, governado por um Sultão.
\section{Sultanato}
\begin{itemize}
\item {Grp. gram.:m.}
\end{itemize}
O mesmo que \textunderscore sultanado\textunderscore .
\section{Sultanear}
\begin{itemize}
\item {Grp. gram.:v. i.}
\end{itemize}
Viver como Sultão. Cf. Rui Barb., \textunderscore Réplica\textunderscore , 158.
\section{Sultanesco}
\begin{itemize}
\item {fónica:nês}
\end{itemize}
\begin{itemize}
\item {Grp. gram.:adj.}
\end{itemize}
Próprio de Sultão. Cf. C. Lobo, \textunderscore Sát. de Juv.\textunderscore , I, 53.
\section{Sultani}
\begin{itemize}
\item {Grp. gram.:m.}
\end{itemize}
\begin{itemize}
\item {Proveniência:(De \textunderscore Sultão\textunderscore )}
\end{itemize}
Antiga moéda de oiro, em Gôa, equivalente a 429 reis.
Moéda egýpcia.
Moéda tunesina.
Moéda argelina.
\section{Sultania}
\begin{itemize}
\item {Grp. gram.:f.}
\end{itemize}
Província governada por um Sultão.
\section{Sultanim}
\begin{itemize}
\item {Grp. gram.:m.}
\end{itemize}
O mesmo que \textunderscore sultani\textunderscore .
\section{Sultanina}
\begin{itemize}
\item {Grp. gram.:f.}
\end{itemize}
Uva de mesa, originária da Asia-Menor.
\section{Sultanino}
\begin{itemize}
\item {Grp. gram.:m.}
\end{itemize}
O mesmo que \textunderscore sultani\textunderscore .
\section{Sultânico}
\begin{itemize}
\item {Grp. gram.:adj.}
\end{itemize}
Relativo a Sultão; próprio de Sultão.
\section{Sultão}
\begin{itemize}
\item {Grp. gram.:m.}
\end{itemize}
\begin{itemize}
\item {Utilização:Fig.}
\end{itemize}
\begin{itemize}
\item {Proveniência:(Do lat. \textunderscore sultanus\textunderscore )}
\end{itemize}
Título do Imperador dos Turcos, e de outro príncipes mahometanos e tártaros.
Senhor poderoso, príncipe absoluto.
Homem, que tem muitas amantes.
\section{Sulvento}
\begin{itemize}
\item {Grp. gram.:m.}
\end{itemize}
\begin{itemize}
\item {Proveniência:(De \textunderscore sul\textunderscore  + \textunderscore vento\textunderscore )}
\end{itemize}
Vento do Sul.
\section{Suma}
\begin{itemize}
\item {Grp. gram.:f.}
\end{itemize}
\begin{itemize}
\item {Grp. gram.:Loc. adv.}
\end{itemize}
\begin{itemize}
\item {Proveniência:(Lat. \textunderscore summa\textunderscore )}
\end{itemize}
Soma.
Epítome.
Resumo.
Substância.
\textunderscore Em suma\textunderscore , resumidamente; em poucos termos; numa palavra.
\section{Suma}
\begin{itemize}
\item {Grp. gram.:f.}
\end{itemize}
\begin{itemize}
\item {Utilização:Bras}
\end{itemize}
Planta medicinal.
\section{Sumaca}
\begin{itemize}
\item {Grp. gram.:f.}
\end{itemize}
Pequena embarcação de dois mastros, usada especialmente na América do Sul.
\section{Sumagral}
\begin{itemize}
\item {Grp. gram.:m.}
\end{itemize}
Lugar, onde cresce sumagre.
\section{Sumagrar}
\begin{itemize}
\item {Grp. gram.:v. t.}
\end{itemize}
Tingir com sumagre.
\section{Sumagre}
\begin{itemize}
\item {Grp. gram.:m.}
\end{itemize}
Gênero de plantas terebintháceas.
Pó, mais ou menos grosseiro, resultante da trituração das fôlhas, flôres, etc., dêsse gênero de plantas, e empregado em Medicina e tinturaria.
(Cp. cast. \textunderscore zumaque\textunderscore )
\section{Sumagreiro}
\begin{itemize}
\item {Grp. gram.:m.}
\end{itemize}
Aquelle que prepara o sumagre para a tinturaria e Medicina.
\section{Sumalar}
\begin{itemize}
\item {Grp. gram.:adj.}
\end{itemize}
\begin{itemize}
\item {Proveniência:(Lat. \textunderscore summalaris\textunderscore )}
\end{itemize}
O mesmo que \textunderscore sumalário\textunderscore .
\section{Sumalário}
\begin{itemize}
\item {Grp. gram.:adj.}
\end{itemize}
Dizia-se o soldado estranjeiro, que se encorporava no exército romano e que tomava lugar na ala esquerda como cavaleiro.
(Cp. \textunderscore sumalar\textunderscore )
\section{Sumamente}
\begin{itemize}
\item {Grp. gram.:adv.}
\end{itemize}
\begin{itemize}
\item {Utilização:Ant.}
\end{itemize}
\begin{itemize}
\item {Proveniência:(De \textunderscore sumo\textunderscore )}
\end{itemize}
Em suma; extremamente, em alto grau: \textunderscore é sumamente patife\textunderscore .
Muito bem. Cf. Pant. de Aveiro, \textunderscore Itiner.\textunderscore , 11, (2.^a ed.).
\section{Sumanaes}
\begin{itemize}
\item {Grp. gram.:m. pl.}
\end{itemize}
\begin{itemize}
\item {Proveniência:(Do lat. \textunderscore summanalia\textunderscore )}
\end{itemize}
Bolos grandes e redondos, espécie de fogaças, que se usavam nos sacrifícios de Plutão.
\section{Sumaré}
\begin{itemize}
\item {Grp. gram.:m.}
\end{itemize}
Espécie de orchídea.
\section{Sumarento}
\begin{itemize}
\item {Grp. gram.:adj.}
\end{itemize}
Que tem sumo ou muito sumo: \textunderscore laranja sumarenta\textunderscore .
\section{Sumariamente}
\begin{itemize}
\item {Grp. gram.:adv.}
\end{itemize}
De modo sumário; sinteticamente; em resumo.
\section{Sumariar}
\begin{itemize}
\item {Grp. gram.:v. t.}
\end{itemize}
Tornar sumário, resumir; sintetizar.
\section{Sumário}
\begin{itemize}
\item {Grp. gram.:adj.}
\end{itemize}
\begin{itemize}
\item {Grp. gram.:M.}
\end{itemize}
\begin{itemize}
\item {Utilização:Ant.}
\end{itemize}
\begin{itemize}
\item {Proveniência:(Lat. \textunderscore summarius\textunderscore )}
\end{itemize}
Feito resumidamente.
Breve.
Feito sem formalidades: \textunderscore julgamento sumário\textunderscore .
Simples.
Recapitulação.
Suma.
Macho, azêmola, bêsta de carga.
\section{Sumatra}
\begin{itemize}
\item {Grp. gram.:f.}
\end{itemize}
\begin{itemize}
\item {Utilização:Ant.}
\end{itemize}
Variedade de tabaco. Cf. \textunderscore Inquér. Industr.\textunderscore , II, p., V. I, 320 e 324.
O mesmo que \textunderscore vulcão\textunderscore .«\textunderscore Tu, que, se queres furacão violento, || Sumatra feia, tempestade escura, || Desatas e subjugas num momento...\textunderscore »Bocage, soneto CCXXV.
\section{Sumaúma}
\begin{itemize}
\item {Grp. gram.:f.}
\end{itemize}
O mesmo que \textunderscore samaúma\textunderscore .
\section{Sumaumeira}
\begin{itemize}
\item {fónica:ma-u}
\end{itemize}
\begin{itemize}
\item {Grp. gram.:f.}
\end{itemize}
O mesmo que \textunderscore samaúma\textunderscore .
\section{Sumaumeira-de-macaco}
\begin{itemize}
\item {Grp. gram.:f.}
\end{itemize}
Grande árvore cujo fruto encerra uma polpa semelhante ao algodão.
\section{Sumbamba}
\begin{itemize}
\item {Grp. gram.:f.}
\end{itemize}
Nome de uma ave africana.
\section{Sumbo}
\begin{itemize}
\item {Grp. gram.:m.}
\end{itemize}
Nome de várias aves africanas.
\section{Sumbrar}
\begin{itemize}
\item {Grp. gram.:v. t.}
\end{itemize}
\begin{itemize}
\item {Utilização:T. de Miranda}
\end{itemize}
O mesmo que \textunderscore semear\textunderscore .
\section{Súmeas}
\begin{itemize}
\item {Grp. gram.:f. pl.}
\end{itemize}
\begin{itemize}
\item {Utilização:Náut.}
\end{itemize}
Peças de madeira, com que se conserta ou fortifica o leme.
\section{Sumelga}
\begin{itemize}
\item {Grp. gram.:m.}
\end{itemize}
\begin{itemize}
\item {Utilização:Deprec.}
\end{itemize}
Homem insignificante, bisbórria. Cf. Camillo, \textunderscore Filha do Arced.\textunderscore , c. XVII.
\section{Sumergir}
\begin{itemize}
\item {Grp. gram.:v. t.}
\end{itemize}
O mesmo que \textunderscore submergir\textunderscore . Cf. \textunderscore Luz e Calor\textunderscore , 14.
\section{Sumeriano}
\begin{itemize}
\item {Grp. gram.:m.}
\end{itemize}
\begin{itemize}
\item {Proveniência:(Do lat. \textunderscore Sumere\textunderscore , n. p.)}
\end{itemize}
Uma das línguas mortas da Ásia, falada outrora nas margens do Tigre (?). Cf. M. Remédios, \textunderscore Liter.\textunderscore 
\section{Sumição}
\begin{itemize}
\item {Grp. gram.:f.}
\end{itemize}
Acto ou effeito de sumir; desapparecimento.
\section{Sumiço}
\begin{itemize}
\item {Grp. gram.:m.}
\end{itemize}
\begin{itemize}
\item {Utilização:Pop.}
\end{itemize}
Acto ou effeito de sumir; desapparecimento.
\section{Sumidade}
\begin{itemize}
\item {Grp. gram.:f.}
\end{itemize}
\begin{itemize}
\item {Utilização:Fig.}
\end{itemize}
\begin{itemize}
\item {Proveniência:(Lat. \textunderscore summitas\textunderscore )}
\end{itemize}
Qualidade do que é alto, eminente.
O ponto mais alto; cumeeira, cimo.
Pessôa muito importante ou muito distinta.
\section{Sumidiço}
\begin{itemize}
\item {Grp. gram.:adj.}
\end{itemize}
Que se some ou desapparece facilmente.
\section{Sumido}
\begin{itemize}
\item {Grp. gram.:adj.}
\end{itemize}
Que mal se vê; encovado.
Froixo, que mal se póde ouvir.
Distante, que se avista com difficuldade.
Magro.
\section{Sumidoiro}
\begin{itemize}
\item {Grp. gram.:m.}
\end{itemize}
\begin{itemize}
\item {Proveniência:(De \textunderscore sumir\textunderscore )}
\end{itemize}
Abertura, por que se escôa um líquido.
Lugar onde desapparecem muitas coisas.
Sargeta.
Urinol.
Coisa, em que se gasta muito dinheiro.
\section{Sumidouro}
\begin{itemize}
\item {Grp. gram.:m.}
\end{itemize}
\begin{itemize}
\item {Proveniência:(De \textunderscore sumir\textunderscore )}
\end{itemize}
Abertura, por que se escôa um líquido.
Lugar onde desapparecem muitas coisas.
Sargeta.
Urinol.
Coisa, em que se gasta muito dinheiro.
\section{Sumidura}
\begin{itemize}
\item {Grp. gram.:f.}
\end{itemize}
O mesmo que \textunderscore sumiço\textunderscore .
\section{Sumiga}
\begin{itemize}
\item {Grp. gram.:f.}
\end{itemize}
\begin{itemize}
\item {Utilização:Açor}
\end{itemize}
O mesmo que \textunderscore batata\textunderscore . (Colhido em San-Jorge)
\section{Sumilhér}
\begin{itemize}
\item {Grp. gram.:m.}
\end{itemize}
Reposteiro da casa real; reposteiro do paço.
(Cast. \textunderscore sumiller\textunderscore )
\section{Sumir}
\begin{itemize}
\item {Grp. gram.:v. t.}
\end{itemize}
\begin{itemize}
\item {Grp. gram.:V. p.}
\end{itemize}
\begin{itemize}
\item {Proveniência:(Lat. \textunderscore sumere\textunderscore )}
\end{itemize}
Tomar, apanhar.
Fazer desapparecer.
Afundar.
Esconder.
Gastar.
Destruír; eliminar.
Expungir.
Desapparecer.
Extinguir-se.
Fugir.
\section{Sumista}
\begin{itemize}
\item {Grp. gram.:m.  e  f.}
\end{itemize}
Pessôa, que faz sumas, sínteses ou compêndios.
\section{Summa}
\begin{itemize}
\item {Grp. gram.:f.}
\end{itemize}
\begin{itemize}
\item {Grp. gram.:Loc. adv.}
\end{itemize}
\begin{itemize}
\item {Proveniência:(Lat. \textunderscore summa\textunderscore )}
\end{itemize}
Somma.
Epítome.
Resumo.
Substância.
\textunderscore Em summa\textunderscore , resumidamente; em poucos termos; numa palavra.
\section{Summalar}
\begin{itemize}
\item {Grp. gram.:adj.}
\end{itemize}
\begin{itemize}
\item {Proveniência:(Lat. \textunderscore summalaris\textunderscore )}
\end{itemize}
O mesmo que \textunderscore summalário\textunderscore .
\section{Summalário}
\begin{itemize}
\item {Grp. gram.:adj.}
\end{itemize}
Dizia-se o soldado estranjeiro, que se encorporava no exército romano e que tomava lugar na ala esquerda como cavalleiro.
(Cp. \textunderscore summalar\textunderscore )
\section{Summamente}
\begin{itemize}
\item {Grp. gram.:adv.}
\end{itemize}
\begin{itemize}
\item {Utilização:Ant.}
\end{itemize}
\begin{itemize}
\item {Proveniência:(De \textunderscore summo\textunderscore )}
\end{itemize}
Em summa; extremamente, em alto grau: \textunderscore é summamente patife\textunderscore .
Muito bem. Cf. Pant. de Aveiro, \textunderscore Itiner.\textunderscore , 11, (2.^a ed.).
\section{Summanaes}
\begin{itemize}
\item {Grp. gram.:m. pl.}
\end{itemize}
\begin{itemize}
\item {Proveniência:(Do lat. \textunderscore summanalia\textunderscore )}
\end{itemize}
Bolos grandes e redondos, espécie de fogaças, que se usavam nos sacrifícios de Plutão.
\section{Summariamente}
\begin{itemize}
\item {Grp. gram.:adv.}
\end{itemize}
De modo summário; syntheticamente; em resumo.
\section{Summariar}
\begin{itemize}
\item {Grp. gram.:v. t.}
\end{itemize}
Tornar summário, resumir; synthetizar.
\section{Summário}
\begin{itemize}
\item {Grp. gram.:adj.}
\end{itemize}
\begin{itemize}
\item {Grp. gram.:M.}
\end{itemize}
\begin{itemize}
\item {Utilização:Ant.}
\end{itemize}
\begin{itemize}
\item {Proveniência:(Lat. \textunderscore summarius\textunderscore )}
\end{itemize}
Feito resumidamente.
Breve.
Feito sem formalidades: \textunderscore julgamento summário\textunderscore .
Simples.
Recapitulação.
Summa.
Macho, azêmola, bêsta de carga.
\section{Summidade}
\begin{itemize}
\item {Grp. gram.:f.}
\end{itemize}
\begin{itemize}
\item {Utilização:Fig.}
\end{itemize}
\begin{itemize}
\item {Proveniência:(Lat. \textunderscore summitas\textunderscore )}
\end{itemize}
Qualidade do que é alto, eminente.
O ponto mais alto; cumeeira, cimo.
Pessôa muito importante ou muito distinta.
\section{Summista}
\begin{itemize}
\item {Grp. gram.:m.  e  f.}
\end{itemize}
Pessôa, que faz summas, sýntheses ou compêndios.
\section{Summo}
\begin{itemize}
\item {Grp. gram.:adj.}
\end{itemize}
\begin{itemize}
\item {Grp. gram.:M.}
\end{itemize}
\begin{itemize}
\item {Proveniência:(Lat. \textunderscore summus\textunderscore )}
\end{itemize}
Que está no lugar mais alto; muito elevado; supremo; máximo.
Cume.
\section{Súmmula}
\begin{itemize}
\item {Grp. gram.:f.}
\end{itemize}
\begin{itemize}
\item {Proveniência:(Lat. \textunderscore summula\textunderscore )}
\end{itemize}
Pequena summa; epitome.
\section{Summulista}
\begin{itemize}
\item {Grp. gram.:m.}
\end{itemize}
Aquelle que faz súmmulas; autor de uma súmmula.
\section{Sumo}
\begin{itemize}
\item {Grp. gram.:adj.}
\end{itemize}
\begin{itemize}
\item {Grp. gram.:M.}
\end{itemize}
\begin{itemize}
\item {Proveniência:(Lat. \textunderscore summus\textunderscore )}
\end{itemize}
Que está no lugar mais alto; muito elevado; supremo; máximo.
Cume.
\section{Sumo}
\begin{itemize}
\item {Grp. gram.:m.}
\end{itemize}
Suco.
Liquido, extrahido de algumas substâncias vegetaes.
(Cp. cast. \textunderscore zumo\textunderscore )
\section{Sumoso}
\begin{itemize}
\item {Grp. gram.:adj.}
\end{itemize}
Que tem sumo; sumarento.
\section{Sumpção}
\begin{itemize}
\item {Grp. gram.:f.}
\end{itemize}
\begin{itemize}
\item {Proveniência:(Do lat. \textunderscore sumptio\textunderscore )}
\end{itemize}
Acto ou effeito de engulir.
\section{Sumpto}
\begin{itemize}
\item {Grp. gram.:m.}
\end{itemize}
\begin{itemize}
\item {Proveniência:(Lat. \textunderscore sumptus\textunderscore )}
\end{itemize}
Despesa, custo.
\section{Sumptuário}
\begin{itemize}
\item {Grp. gram.:adj.}
\end{itemize}
\begin{itemize}
\item {Proveniência:(Lat. \textunderscore sumptuarius\textunderscore )}
\end{itemize}
Relativo a despesas.
Relativo a luxo: \textunderscore impostos sumptuários\textunderscore .
\section{Sumptuosamente}
\begin{itemize}
\item {Grp. gram.:adv.}
\end{itemize}
De modo sumptuoso; com magnificência; luxuosamente.
\section{Sumptuosidade}
\begin{itemize}
\item {Grp. gram.:f.}
\end{itemize}
\begin{itemize}
\item {Proveniência:(Do lat. \textunderscore sumptuositas\textunderscore )}
\end{itemize}
Qualidade do que é sumptuoso.
Grande luxo; magnificência.
\section{Sumptuoso}
\begin{itemize}
\item {Grp. gram.:adj.}
\end{itemize}
\begin{itemize}
\item {Proveniência:(Lat. \textunderscore sumptuosus\textunderscore )}
\end{itemize}
Com que se fez grande despesa.
Apparatoso.
Magnificente; em que há muito luxo.
\section{Súmula}
\begin{itemize}
\item {Grp. gram.:f.}
\end{itemize}
\begin{itemize}
\item {Proveniência:(Lat. \textunderscore summula\textunderscore )}
\end{itemize}
Pequena suma; epitome.
\section{Sumulista}
\begin{itemize}
\item {Grp. gram.:m.}
\end{itemize}
Aquele que faz súmulas; autor de uma súmula.
\section{Sundanês}
\begin{itemize}
\item {Grp. gram.:m.  e  adj.}
\end{itemize}
Que é do archipélago da Sunda.
\section{Sundeque}
\begin{itemize}
\item {Grp. gram.:m.}
\end{itemize}
\begin{itemize}
\item {Utilização:Gír.}
\end{itemize}
O mesmo que \textunderscore sondeque\textunderscore .
\section{Sunga}
\begin{itemize}
\item {Grp. gram.:m.}
\end{itemize}
\begin{itemize}
\item {Utilização:Bras. do N}
\end{itemize}
Calções de criança.
Espécie de avental, que cobre o ventre, as coxas.
(Cp. \textunderscore sungar\textunderscore )
\section{Sungar}
\begin{itemize}
\item {Grp. gram.:v. t.}
\end{itemize}
\begin{itemize}
\item {Utilização:Bras}
\end{itemize}
\begin{itemize}
\item {Grp. gram.:V. i.}
\end{itemize}
\begin{itemize}
\item {Utilização:Bras. do N}
\end{itemize}
Puxar para cima.
Trepar, subir.
Montar.
(Do lundês \textunderscore cu-sunga\textunderscore )
\section{Sunguiandondo}
\begin{itemize}
\item {Grp. gram.:m.}
\end{itemize}
Nome de várias aves africanas.
\section{Suna}
\begin{itemize}
\item {Grp. gram.:f.}
\end{itemize}
Livro, que alguns consideram suplemento do \textunderscore Alcorão\textunderscore .
\section{Sunha-açu}
\begin{itemize}
\item {Grp. gram.:m.}
\end{itemize}
Gênero de pássaros brasileiros, nocivos aos frutos.
\section{Sunicúleas}
\begin{itemize}
\item {Grp. gram.:f. pl.}
\end{itemize}
Tríbo de plantas, que tem por typo a sanícula.
\section{Sunita}
\begin{itemize}
\item {Grp. gram.:m.}
\end{itemize}
\begin{itemize}
\item {Proveniência:(Do ár. \textunderscore sunni\textunderscore )}
\end{itemize}
Membro de uma seita muçulmana, que considera a \textunderscore Suna\textunderscore  como complemento do \textunderscore Alcorão\textunderscore .
\section{Sun-malé}
\begin{itemize}
\item {Grp. gram.:m.}
\end{itemize}
Grande árvore medicinal da ilha de San-Thomé.
\section{Sunna}
\begin{itemize}
\item {Grp. gram.:f.}
\end{itemize}
Livro, que alguns consideram supplemento do \textunderscore Alcorão\textunderscore .
\section{Sunnita}
\begin{itemize}
\item {Grp. gram.:m.}
\end{itemize}
\begin{itemize}
\item {Proveniência:(Do ár. \textunderscore sunni\textunderscore )}
\end{itemize}
Membro de uma seita muçulmana, que considera a \textunderscore Sunna\textunderscore  como complemento do \textunderscore Alcorão\textunderscore .
\section{Sununga}
\begin{itemize}
\item {Grp. gram.:f.}
\end{itemize}
\begin{itemize}
\item {Utilização:Bras. do N}
\end{itemize}
Plantação de mandioca, feita no verão.
\section{Suómi}
\begin{itemize}
\item {Grp. gram.:m.}
\end{itemize}
Língua uralo-altaica; o finlandês.
\section{Suór}
\begin{itemize}
\item {Grp. gram.:m.}
\end{itemize}
\begin{itemize}
\item {Utilização:Fig.}
\end{itemize}
\begin{itemize}
\item {Proveniência:(Lat. \textunderscore sudor\textunderscore )}
\end{itemize}
Humor aquoso, que vem á superfície da pelle e que se condensa em gotas.
A saída dêsse humor.
Acto de suar.
Estado de quem sua.
Grande trabalho.
Resultado de grandes fadigas; sacrifício.
\section{Supeditar}
\begin{itemize}
\item {Grp. gram.:v. t.}
\end{itemize}
\begin{itemize}
\item {Proveniência:(Lat. \textunderscore suppeditare\textunderscore )}
\end{itemize}
Fornecer, ministrar.
\section{Super...}
\begin{itemize}
\item {Grp. gram.:pref.}
\end{itemize}
\begin{itemize}
\item {Proveniência:(Lat. \textunderscore super\textunderscore )}
\end{itemize}
(designativo de \textunderscore superioridade\textunderscore )
\section{Superabundância}
\begin{itemize}
\item {Grp. gram.:f.}
\end{itemize}
\begin{itemize}
\item {Proveniência:(Lat. \textunderscore superabundantia\textunderscore )}
\end{itemize}
Qualidade do que é superabundante.
\section{Superabundante}
\begin{itemize}
\item {Grp. gram.:adj.}
\end{itemize}
\begin{itemize}
\item {Proveniência:(Lat. \textunderscore superabundans\textunderscore )}
\end{itemize}
Que superabunda.
\section{Superabundantemente}
\begin{itemize}
\item {Grp. gram.:adv.}
\end{itemize}
De modo superabundante; demasiadamente; em excesso.
\section{Superabundar}
\begin{itemize}
\item {Grp. gram.:v. i.}
\end{itemize}
\begin{itemize}
\item {Proveniência:(Lat. \textunderscore superabundare\textunderscore )}
\end{itemize}
Existir com superfluidade.
Abundar excessivamente; sobejar.
\section{Superacidez}
\begin{itemize}
\item {Grp. gram.:f.}
\end{itemize}
Qualidade de superácido.
\section{Superácido}
\begin{itemize}
\item {Grp. gram.:adj.}
\end{itemize}
\begin{itemize}
\item {Proveniência:(De \textunderscore super...\textunderscore  + \textunderscore ácido\textunderscore )}
\end{itemize}
Extraordinariamente ácido.
\section{Superando}
\begin{itemize}
\item {Grp. gram.:adj.}
\end{itemize}
\begin{itemize}
\item {Proveniência:(Lat. \textunderscore superandus\textunderscore )}
\end{itemize}
Que se há de vencer; que se póde vencer:«\textunderscore obstáculos superandos\textunderscore ». Filinto, XV, 145.
\section{Superante}
\begin{itemize}
\item {Grp. gram.:adj.}
\end{itemize}
\begin{itemize}
\item {Proveniência:(Lat. \textunderscore superans\textunderscore )}
\end{itemize}
Que supera, que excede, que se avantaja.
\section{Superar}
\begin{itemize}
\item {Grp. gram.:v. t.}
\end{itemize}
\begin{itemize}
\item {Proveniência:(Lat. \textunderscore superare\textunderscore )}
\end{itemize}
Sêr superior a: \textunderscore a receita superou a despesa\textunderscore .
Passar por cima de; galgar; exceder.
Vencer; subjugar.
Destruir.
\section{Superável}
\begin{itemize}
\item {Grp. gram.:adj.}
\end{itemize}
\begin{itemize}
\item {Proveniência:(Do lat. \textunderscore superabilis\textunderscore )}
\end{itemize}
Que se póde superar.
\section{Supercílio}
\begin{itemize}
\item {Grp. gram.:m.}
\end{itemize}
\begin{itemize}
\item {Utilização:Poét.}
\end{itemize}
\begin{itemize}
\item {Proveniência:(Lat. \textunderscore supercilium\textunderscore )}
\end{itemize}
O mesmo que \textunderscore sobrancelha\textunderscore .
\section{Supercilioso}
\begin{itemize}
\item {Grp. gram.:adj.}
\end{itemize}
\begin{itemize}
\item {Proveniência:(Lat. \textunderscore superciliosus\textunderscore )}
\end{itemize}
Que tem semblante carregado.
Severo; austero; rispido.
Que tem sobrancelhas espessas. Cf. Camillo, \textunderscore Brasileira\textunderscore , 260; Herculano, \textunderscore Quest. Públ.\textunderscore , I. 294.
\section{Superelegante}
\begin{itemize}
\item {Grp. gram.:adj.}
\end{itemize}
\begin{itemize}
\item {Proveniência:(De \textunderscore super...\textunderscore  + \textunderscore elegante\textunderscore )}
\end{itemize}
Muito elegante.
\section{Supereminencia}
\begin{itemize}
\item {Grp. gram.:f.}
\end{itemize}
\begin{itemize}
\item {Proveniência:(Lat. \textunderscore supereminentia\textunderscore )}
\end{itemize}
Elevação extraordinária; preeminência.
\section{Supereminente}
\begin{itemize}
\item {Grp. gram.:adj.}
\end{itemize}
\begin{itemize}
\item {Proveniência:(Lat. \textunderscore supereminens\textunderscore )}
\end{itemize}
Que tem supereminência.
\section{Supererogação}
\begin{itemize}
\item {fónica:ru}
\end{itemize}
\begin{itemize}
\item {Grp. gram.:f.}
\end{itemize}
\begin{itemize}
\item {Proveniência:(Do lat. \textunderscore supererogatio\textunderscore )}
\end{itemize}
Excesso, demasia:«\textunderscore se me alugarem para andar mil passos, vá eu de supererogação mais outros dois mil.\textunderscore »\textunderscore Luz e Calor\textunderscore , 77.
\section{Supererrugação}
\begin{itemize}
\item {Grp. gram.:f.}
\end{itemize}
\begin{itemize}
\item {Proveniência:(Do lat. \textunderscore supererogatio\textunderscore )}
\end{itemize}
Excesso, demasia:«\textunderscore se me alugarem para andar mil passos, vá eu de supererrugação mais outros dois mil.\textunderscore »\textunderscore Luz e Calor\textunderscore , 77.
\section{Superevangélia}
\begin{itemize}
\item {Grp. gram.:f.}
\end{itemize}
\begin{itemize}
\item {Utilização:Ant.}
\end{itemize}
\begin{itemize}
\item {Proveniência:(Do lat. \textunderscore super\textunderscore  + \textunderscore Evangelium\textunderscore )}
\end{itemize}
Pano precioso, com que se cobriam os Evangelhos nos templos.
\section{Superexaltado}
\begin{itemize}
\item {Grp. gram.:adj.}
\end{itemize}
\begin{itemize}
\item {Proveniência:(De \textunderscore super...\textunderscore  + \textunderscore exaltar\textunderscore )}
\end{itemize}
Muito exaltado. Cf. Bernárdez, \textunderscore Luz e Calor\textunderscore , 627.
\section{Superexcitabilidade}
\begin{itemize}
\item {Grp. gram.:f.}
\end{itemize}
Qualidade de superexcitável.
\section{Superexcitação}
\begin{itemize}
\item {Grp. gram.:f.}
\end{itemize}
Acto ou effeito de superexcitar.
\section{Superexcitante}
\begin{itemize}
\item {Grp. gram.:adj.}
\end{itemize}
Que superexcita.
\section{Superexcitar}
\begin{itemize}
\item {Grp. gram.:v. t.}
\end{itemize}
\begin{itemize}
\item {Utilização:Neol.}
\end{itemize}
\begin{itemize}
\item {Proveniência:(De \textunderscore super...\textunderscore  + \textunderscore excitar\textunderscore )}
\end{itemize}
O mesmo que \textunderscore sobreexcitar\textunderscore . Cf. \textunderscore Jorn.-do-Comm.\textunderscore , do Rio, de 16-I-905.
\section{Superexcitável}
\begin{itemize}
\item {Grp. gram.:adj.}
\end{itemize}
Que se póde superexcitar.
\section{Superexcreção}
\begin{itemize}
\item {Grp. gram.:f.}
\end{itemize}
\begin{itemize}
\item {Proveniência:(De \textunderscore super...\textunderscore  + \textunderscore excreção\textunderscore )}
\end{itemize}
Excessiva excreção.
\section{Superfetação}
\begin{itemize}
\item {Grp. gram.:f.}
\end{itemize}
\begin{itemize}
\item {Utilização:Fig.}
\end{itemize}
\begin{itemize}
\item {Proveniência:(Do lat. \textunderscore superfaetatio\textunderscore )}
\end{itemize}
Concepção de um féto, quando há já outro na madre.
Superfluidade.
\section{Superfetar}
\begin{itemize}
\item {Grp. gram.:v. i.}
\end{itemize}
\begin{itemize}
\item {Proveniência:(Do lat. \textunderscore superfaetare\textunderscore )}
\end{itemize}
Conceber um féto, havendo já outro na madre.
\section{Superficial}
\begin{itemize}
\item {Grp. gram.:adj.}
\end{itemize}
\begin{itemize}
\item {Utilização:Fig.}
\end{itemize}
\begin{itemize}
\item {Proveniência:(Lat. \textunderscore superficialis\textunderscore )}
\end{itemize}
Relativo a superfície.
Que está á superfície.
Pouco profundo; pouco sólido: \textunderscore instrucção superficial\textunderscore .
Leviano.
\section{Superficialidade}
\begin{itemize}
\item {Grp. gram.:f.}
\end{itemize}
Qualidade do que é superficial.
\section{Superficialmente}
\begin{itemize}
\item {Grp. gram.:adv.}
\end{itemize}
De modo superficial.
Ligeiramente, de leve.
\section{Superficiário}
\begin{itemize}
\item {Grp. gram.:adj.}
\end{itemize}
\begin{itemize}
\item {Utilização:Jur.}
\end{itemize}
\begin{itemize}
\item {Proveniência:(Lat. \textunderscore superficiarius\textunderscore )}
\end{itemize}
Dizia-se, entre os Romanos, o edifício construido em terreno alheio, e do qual o constructor só tinha o usofruto. Cf. R. de Brito, \textunderscore Philos. do Dir.\textunderscore  257.
\section{Superfície}
\begin{itemize}
\item {Grp. gram.:f.}
\end{itemize}
\begin{itemize}
\item {Utilização:Fig.}
\end{itemize}
\begin{itemize}
\item {Proveniência:(Lat. \textunderscore superficies\textunderscore )}
\end{itemize}
Parte exterior dos corpos.
Extensão de um corpo, quanto ao seu comprimento e largura, independentemente da profundidade.
Face.
Longe, apparência, aspecto.
\section{Superfino}
\begin{itemize}
\item {Grp. gram.:adj.}
\end{itemize}
\begin{itemize}
\item {Proveniência:(De \textunderscore super...\textunderscore  + \textunderscore fino\textunderscore )}
\end{itemize}
Muito fino; que é da melhor qualidade.
\section{Superfluamente}
\begin{itemize}
\item {Grp. gram.:adv.}
\end{itemize}
De modo supérfluo.
\section{Superfluidade}
\begin{itemize}
\item {fónica:flu-i}
\end{itemize}
\begin{itemize}
\item {Grp. gram.:f.}
\end{itemize}
\begin{itemize}
\item {Proveniência:(Do lat. \textunderscore superfluitas\textunderscore )}
\end{itemize}
Qualidade do que é supérfluo; coisa supérflua.
\section{Supérfluo}
\begin{itemize}
\item {Grp. gram.:adj.}
\end{itemize}
\begin{itemize}
\item {Grp. gram.:M.}
\end{itemize}
\begin{itemize}
\item {Proveniência:(Lat. \textunderscore superfluus\textunderscore )}
\end{itemize}
Que é de mais; demasiado.
Inútil; desnecessário.
Aquillo que é supérfluo.
\section{Superfosfato}
\begin{itemize}
\item {Grp. gram.:m.}
\end{itemize}
Producto industrial, empregado como adubo agrícola, e resultante da acção do ácido sulfúrico sôbre o fosfato de cal.
\section{Superfrívolo}
\begin{itemize}
\item {Grp. gram.:adj.}
\end{itemize}
\begin{itemize}
\item {Proveniência:(De \textunderscore super...\textunderscore  + \textunderscore frívolo\textunderscore )}
\end{itemize}
Excessivamente frívolo. Cf. Filinto, XIX, 229.
\section{Superhumano}
\begin{itemize}
\item {Grp. gram.:adj.}
\end{itemize}
O mesmo que \textunderscore sobrehumano\textunderscore .
\section{Superhumeral}
\begin{itemize}
\item {Grp. gram.:m.}
\end{itemize}
\begin{itemize}
\item {Proveniência:(Lat. \textunderscore superhumerole\textunderscore )}
\end{itemize}
Vestuário ecclesiástico, entre os Hebreus.
\section{Superimpregnação}
\begin{itemize}
\item {Grp. gram.:f.}
\end{itemize}
\begin{itemize}
\item {Utilização:Physiol.}
\end{itemize}
\begin{itemize}
\item {Proveniência:(De \textunderscore super...\textunderscore  + \textunderscore impregnar\textunderscore )}
\end{itemize}
Fecundação de dois ou mais óvulos, em muitos cóitos, mais ou menos distanciados.
\section{Superintendência}
\begin{itemize}
\item {Grp. gram.:f.}
\end{itemize}
\begin{itemize}
\item {Proveniência:(De \textunderscore super...\textunderscore  + \textunderscore intendência\textunderscore )}
\end{itemize}
Acto de superintender.
Cargo de superintendente.
Casa, onde funcciona o superintendente.
\section{Superintendente}
\begin{itemize}
\item {Grp. gram.:m.  e  adj.}
\end{itemize}
O mesmo que superintende.
\section{Superintender}
\begin{itemize}
\item {Grp. gram.:v. t.  e  i.}
\end{itemize}
\begin{itemize}
\item {Proveniência:(Do lat. \textunderscore super...\textunderscore  + \textunderscore intendere\textunderscore )}
\end{itemize}
Dirigir superiormente certos trabalhos.
\section{Superior}
\begin{itemize}
\item {Grp. gram.:adj.}
\end{itemize}
\begin{itemize}
\item {Grp. gram.:M.}
\end{itemize}
\begin{itemize}
\item {Proveniência:(Lat. \textunderscore superior\textunderscore )}
\end{itemize}
Que está mais acima de outro.
Mais elevado, que occupa posição ou ordem mais elevada.
Muito elevado: \textunderscore um talento superior\textunderscore .
Que subrepuja outrem.
Que dimana da autoridade: \textunderscore recebi instrucções superiores\textunderscore .
Situado do lado do Norte.
Mais próximo da nascente de um rio, (falando-se de um país que êsse rio atravessa).
Aquelle que exerce autoridade sôbre outro: \textunderscore respeitar os seus superiores\textunderscore .
O que governa um convento.
\section{Superiora}
\begin{itemize}
\item {Grp. gram.:f.}
\end{itemize}
\begin{itemize}
\item {Proveniência:(De \textunderscore superior\textunderscore )}
\end{itemize}
Freira, que dirige um convento; abadessa; prioresa.
\section{Superiorato}
\begin{itemize}
\item {Grp. gram.:m.}
\end{itemize}
Dignidade ou cargo de superior ou superiora.
\section{Superioridade}
\begin{itemize}
\item {Grp. gram.:f.}
\end{itemize}
Qualidade do que é superior.
\section{Superiorizar}
\begin{itemize}
\item {Grp. gram.:v. t.}
\end{itemize}
Tornar superior; distinguir.
\section{Superiormente}
\begin{itemize}
\item {Grp. gram.:adv.}
\end{itemize}
De modo superior.
Em grau elevado.
\section{Superlativação}
\begin{itemize}
\item {Grp. gram.:f.}
\end{itemize}
Acto ou effeito de superlativar.
\section{Superlativamente}
\begin{itemize}
\item {Grp. gram.:adv.}
\end{itemize}
De modo superlativo; em grau muito elevado; no mais alto grau.
\section{Superlativar}
\begin{itemize}
\item {Grp. gram.:v.}
\end{itemize}
\begin{itemize}
\item {Utilização:t. Gram.}
\end{itemize}
\begin{itemize}
\item {Utilização:Neol.}
\end{itemize}
Tornar superlativo; dar fórma de superlativo a.
\section{Superlativo}
\begin{itemize}
\item {Grp. gram.:adj.}
\end{itemize}
\begin{itemize}
\item {Grp. gram.:M.}
\end{itemize}
\begin{itemize}
\item {Proveniência:(Lat. \textunderscore superlativus\textunderscore )}
\end{itemize}
Que exprime uma qualidade bôa ou má, no mais alto grau.
Muito alto.
Óptimo.
Adjectivo, elevado ao grau superlativo; o mais alto grau.
\section{Supernal}
\begin{itemize}
\item {Grp. gram.:adj.}
\end{itemize}
O mesmo que \textunderscore superno\textunderscore .
\section{Supernatural}
\begin{itemize}
\item {Grp. gram.:adj.}
\end{itemize}
O mesmo que \textunderscore sobrenatural\textunderscore .
\section{Supernumerário}
\begin{itemize}
\item {Grp. gram.:adj.}
\end{itemize}
\begin{itemize}
\item {Proveniência:(Lat. \textunderscore supernumerarius\textunderscore )}
\end{itemize}
O mesmo que \textunderscore supranumerário\textunderscore .
\section{Superno}
\begin{itemize}
\item {Grp. gram.:adj.}
\end{itemize}
\begin{itemize}
\item {Utilização:Fig.}
\end{itemize}
\begin{itemize}
\item {Proveniência:(Lat. \textunderscore supernus\textunderscore )}
\end{itemize}
Superior, muito alto.
Óptimo.
\section{Súpero}
\begin{itemize}
\item {Grp. gram.:adj.}
\end{itemize}
\begin{itemize}
\item {Utilização:Bot.}
\end{itemize}
\begin{itemize}
\item {Proveniência:(Lat. \textunderscore superus\textunderscore )}
\end{itemize}
Superior, superno.
Diz-se do ovário, collocado acima de todas as outras partes de uma flôr \textunderscore ou\textunderscore , pelo menos, acima do cálice.
\section{Súpero...}
\begin{itemize}
\item {Grp. gram.:pref.}
\end{itemize}
\begin{itemize}
\item {Proveniência:(Lat. \textunderscore superus\textunderscore )}
\end{itemize}
(designativo do \textunderscore que está superior\textunderscore )
\section{Súpero-anterior}
\begin{itemize}
\item {Grp. gram.:adj.}
\end{itemize}
Situado superiormente e na parte anterior.
\section{Súpero-exterior}
\begin{itemize}
\item {Grp. gram.:adj.}
\end{itemize}
Situado superiormente e na parte exterior.
\section{Súpero-interior}
\begin{itemize}
\item {Grp. gram.:adj.}
\end{itemize}
Situado superiormente e na parte interior.
\section{Súpero-posterior}
\begin{itemize}
\item {Grp. gram.:adj.}
\end{itemize}
Situado superiormente e na parte posterior.
\section{Superordenar}
\begin{itemize}
\item {Grp. gram.:v. t.}
\end{itemize}
\begin{itemize}
\item {Proveniência:(Do lat. \textunderscore superordinare\textunderscore )}
\end{itemize}
Ordenar superiormente.
Acrescentar, ajuntar.
\section{Superovariado}
\begin{itemize}
\item {Grp. gram.:adj.}
\end{itemize}
\begin{itemize}
\item {Utilização:Bot.}
\end{itemize}
\begin{itemize}
\item {Proveniência:(De \textunderscore supero...\textunderscore  + \textunderscore ovário\textunderscore )}
\end{itemize}
Que tem ovário súpero ou livre.
\section{Superoxidação}
\begin{itemize}
\item {fónica:csi}
\end{itemize}
\begin{itemize}
\item {Grp. gram.:f.}
\end{itemize}
\begin{itemize}
\item {Proveniência:(De \textunderscore super...\textunderscore  + \textunderscore oxydação\textunderscore )}
\end{itemize}
Oxidação ou excesso do oxigênio.
\section{Superoxydação}
\begin{itemize}
\item {Grp. gram.:f.}
\end{itemize}
\begin{itemize}
\item {Proveniência:(De \textunderscore super...\textunderscore  + \textunderscore oxydação\textunderscore )}
\end{itemize}
Oxydação ou excesso do oxygênio.
\section{Superphosphato}
\begin{itemize}
\item {Grp. gram.:m.}
\end{itemize}
Producto industrial, empregado como adubo agrícola, e resultante da acção do ácido sulfúrico sôbre o phosphato de cal.
\section{Superpopulação}
\begin{itemize}
\item {Grp. gram.:f.}
\end{itemize}
\begin{itemize}
\item {Utilização:bras}
\end{itemize}
\begin{itemize}
\item {Utilização:Neol.}
\end{itemize}
\begin{itemize}
\item {Proveniência:(De \textunderscore super...\textunderscore  + \textunderscore população\textunderscore )}
\end{itemize}
O mesmo que \textunderscore superpovoamento\textunderscore . Cf. \textunderscore Jorn.-do-Comm.\textunderscore , do Rio, do 9-X-905.
\section{Superposição}
\begin{itemize}
\item {Grp. gram.:f.}
\end{itemize}
O mesmo que \textunderscore sobreposição\textunderscore .
\section{Superpovoamento}
\begin{itemize}
\item {Grp. gram.:m.}
\end{itemize}
\begin{itemize}
\item {Utilização:bras}
\end{itemize}
\begin{itemize}
\item {Utilização:Neol.}
\end{itemize}
Excesso de população. Cf. \textunderscore Jorn.-do-Comm.\textunderscore , do Rio, do 12-VI-901.
\section{Superproducção}
\begin{itemize}
\item {Grp. gram.:f.}
\end{itemize}
\begin{itemize}
\item {Proveniência:(De \textunderscore super...\textunderscore  + \textunderscore producção\textunderscore )}
\end{itemize}
Excesso de producção:«\textunderscore a nossa superproducção vinícola...\textunderscore »M. Prego, \textunderscore Olivaes\textunderscore .
\section{Superprodução}
\begin{itemize}
\item {Grp. gram.:f.}
\end{itemize}
\begin{itemize}
\item {Proveniência:(De \textunderscore super...\textunderscore  + \textunderscore producção\textunderscore )}
\end{itemize}
Excesso de producção:«\textunderscore a nossa superprodução vinícola...\textunderscore »M. Prego, \textunderscore Olivaes\textunderscore .
\section{Superpurgação}
\begin{itemize}
\item {Grp. gram.:f.}
\end{itemize}
\begin{itemize}
\item {Proveniência:(De \textunderscore super...\textunderscore  + \textunderscore purgação\textunderscore )}
\end{itemize}
Purgação excessiva.
\section{Superrequintado}
\begin{itemize}
\item {Grp. gram.:adj.}
\end{itemize}
\begin{itemize}
\item {Proveniência:(De \textunderscore super...\textunderscore  + \textunderscore requintado\textunderscore )}
\end{itemize}
Requintado excessivamente. Cf. Rui Barb., \textunderscore Réplica\textunderscore , 158.
\section{Supérrimo}
\begin{itemize}
\item {Grp. gram.:adj.}
\end{itemize}
\begin{itemize}
\item {Proveniência:(Lat. \textunderscore superrimus\textunderscore )}
\end{itemize}
O mesmo que \textunderscore supremo\textunderscore . Cf. A. Mendes, \textunderscore Plágios\textunderscore .
\section{Supersecreção}
\begin{itemize}
\item {Grp. gram.:f.}
\end{itemize}
\begin{itemize}
\item {Proveniência:(De \textunderscore super...\textunderscore  + \textunderscore secreção\textunderscore )}
\end{itemize}
Secreção abundante.
\section{Supersensível}
\begin{itemize}
\item {Grp. gram.:adj.}
\end{itemize}
\begin{itemize}
\item {Proveniência:(De \textunderscore super...\textunderscore  + \textunderscore sensível\textunderscore )}
\end{itemize}
Superior á acção dos sentidos.
\section{Superstar}
\begin{itemize}
\item {Grp. gram.:v. i.}
\end{itemize}
O mesmo que \textunderscore sobrestar\textunderscore . Cf. Filinto, \textunderscore D. Man.\textunderscore , I, 156.
\section{Superstição}
\begin{itemize}
\item {Grp. gram.:f.}
\end{itemize}
\begin{itemize}
\item {Proveniência:(Do lat. \textunderscore superstitio\textunderscore )}
\end{itemize}
Sentimento religioso, fundado no temor ou na ignorância, e que induz ao cumprimento de falsos deveres, ao receio de coisas fantásticas e á confiança em coisas inefficazes.
Prática supersticiosa.
Excessiva credulidade.
Crendice; preconceito.
\section{Supersticiosamente}
\begin{itemize}
\item {Grp. gram.:adv.}
\end{itemize}
De modo supersticioso.
\section{Supersticiosidade}
\begin{itemize}
\item {Grp. gram.:f.}
\end{itemize}
Qualidade do que é supersticioso.
Superstição.
\section{Supersticioso}
\begin{itemize}
\item {Grp. gram.:adj.}
\end{itemize}
\begin{itemize}
\item {Grp. gram.:M.}
\end{itemize}
\begin{itemize}
\item {Proveniência:(Lat. \textunderscore superstitiosus\textunderscore )}
\end{itemize}
Que tem superstição; em que há superstição.
Indivíduo supersticioso.
\section{Supérstite}
\begin{itemize}
\item {Grp. gram.:adj.}
\end{itemize}
\begin{itemize}
\item {Proveniência:(Lat. \textunderscore superstes\textunderscore )}
\end{itemize}
Que sobrevive; sobrevivente.
\section{Superstructura}
\begin{itemize}
\item {Grp. gram.:f.}
\end{itemize}
Estructura de partes superiores. Cf. Latino, \textunderscore Or. da Corôa\textunderscore , CCVIII.
\section{Superstrutura}
\begin{itemize}
\item {Grp. gram.:f.}
\end{itemize}
Estrutura de partes superiores. Cf. Latino, \textunderscore Or. da Corôa\textunderscore , CCVIII.
\section{Supersubstancial}
\begin{itemize}
\item {Grp. gram.:adj.}
\end{itemize}
\begin{itemize}
\item {Proveniência:(Do lat. \textunderscore supersubstancialis\textunderscore )}
\end{itemize}
Muito substancial.
\section{Supervacâneo}
\begin{itemize}
\item {Grp. gram.:adj.}
\end{itemize}
\begin{itemize}
\item {Proveniência:(Lat. \textunderscore supervacaneus\textunderscore )}
\end{itemize}
Supérfluo, inútil.
\section{Supervácuo}
\begin{itemize}
\item {Grp. gram.:adj.}
\end{itemize}
\begin{itemize}
\item {Proveniência:(Lat. \textunderscore supervacuus\textunderscore )}
\end{itemize}
O mesmo que \textunderscore supervacâneo\textunderscore .
\section{Supervenção}
\begin{itemize}
\item {Grp. gram.:f.}
\end{itemize}
\begin{itemize}
\item {Proveniência:(Do lat. \textunderscore super\textunderscore  + \textunderscore ventio\textunderscore )}
\end{itemize}
Acto ou effeito de sobrevir.
\section{Superveniência}
\begin{itemize}
\item {Grp. gram.:f.}
\end{itemize}
Qualidade do que é superveniente.
Acto de sobrevir.
\section{Superveniente}
\begin{itemize}
\item {Grp. gram.:adj.}
\end{itemize}
\begin{itemize}
\item {Proveniência:(Lat. \textunderscore superveniens\textunderscore )}
\end{itemize}
Que sobrevém; que vem ou apparece depois.
\section{Supervivência}
\begin{itemize}
\item {Grp. gram.:f.}
\end{itemize}
O mesmo que \textunderscore sobrevivência\textunderscore .
\section{Supervivente}
\begin{itemize}
\item {Grp. gram.:m.  e  adj.}
\end{itemize}
\begin{itemize}
\item {Proveniência:(Lat. \textunderscore supervivens\textunderscore )}
\end{itemize}
O mesmo que \textunderscore sobrevivente\textunderscore .
\section{Supetão, de}
\begin{itemize}
\item {Grp. gram.:loc. adv.}
\end{itemize}
\begin{itemize}
\item {Proveniência:(De \textunderscore súpeto\textunderscore )}
\end{itemize}
De súbito; repentinamente; imprevistamente.
\section{Súpeto}
\begin{itemize}
\item {Grp. gram.:adj.}
\end{itemize}
\begin{itemize}
\item {Utilização:Pop.}
\end{itemize}
O mesmo que \textunderscore súbito\textunderscore .
\section{Supi}
\begin{itemize}
\item {Grp. gram.:m.}
\end{itemize}
\begin{itemize}
\item {Utilização:Bras}
\end{itemize}
Ave das regiões do Amazonas.
\section{Supimpa}
\begin{itemize}
\item {Grp. gram.:adj.}
\end{itemize}
\begin{itemize}
\item {Utilização:Bras. de Minas}
\end{itemize}
Muito bom, superior, excellente: \textunderscore isto é um vinho supimpa\textunderscore .
\section{Supinação}
\begin{itemize}
\item {Grp. gram.:f.}
\end{itemize}
\begin{itemize}
\item {Utilização:Anat.}
\end{itemize}
\begin{itemize}
\item {Proveniência:(Lat. \textunderscore supinatio\textunderscore )}
\end{itemize}
Movimento, produzido pelos músculos supinadores no ante-braço e na mão, de fórma que a palma esteja voltada para deante, quando o braço está pendente.
Posição de enfermo, quando deitado de costas.
\section{Supinador}
\begin{itemize}
\item {Grp. gram.:m.  e  adj.}
\end{itemize}
\begin{itemize}
\item {Utilização:Anat.}
\end{itemize}
\begin{itemize}
\item {Proveniência:(Do lat. \textunderscore supinator\textunderscore )}
\end{itemize}
Diz se dos músculos que exercem no ante-braço e na mão uma acção opposta á dos pronadores.
\section{Supinamente}
\begin{itemize}
\item {Grp. gram.:adv.}
\end{itemize}
De modo supino; excessivamente.
\section{Supino}
\begin{itemize}
\item {Grp. gram.:adj.}
\end{itemize}
\begin{itemize}
\item {Utilização:Fig.}
\end{itemize}
\begin{itemize}
\item {Grp. gram.:M.}
\end{itemize}
\begin{itemize}
\item {Utilização:Gram.}
\end{itemize}
\begin{itemize}
\item {Proveniência:(Lat. \textunderscore supinus\textunderscore )}
\end{itemize}
Superior, elevado.
Deitado de costas.
Que está voltado para cima.
Demasiado, completo, (falando se do êrro ou da ignorância).
Parte do infinitivo latino, com que se formam muitos tempos, e que se póde considerar um nome verbal.
\section{Supino}
\begin{itemize}
\item {Grp. gram.:m.}
\end{itemize}
\begin{itemize}
\item {Utilização:Prov.}
\end{itemize}
\begin{itemize}
\item {Utilização:trasm.}
\end{itemize}
Nádegas, assento.
\section{Supitamente}
\begin{itemize}
\item {Grp. gram.:adv.}
\end{itemize}
\begin{itemize}
\item {Utilização:Ant.}
\end{itemize}
O mesmo que \textunderscore subitamente\textunderscore . Cf. G. Vicente, \textunderscore Carta a D. João III\textunderscore .
\section{Súpito}
\begin{itemize}
\item {Grp. gram.:adj.}
\end{itemize}
\begin{itemize}
\item {Utilização:ant.}
\end{itemize}
\begin{itemize}
\item {Utilização:Pop.}
\end{itemize}
O mesmo que \textunderscore súbito\textunderscore .
\section{Suplantação}
\begin{itemize}
\item {Grp. gram.:f.}
\end{itemize}
\begin{itemize}
\item {Proveniência:(Do lat. \textunderscore supplantatio\textunderscore )}
\end{itemize}
Acto ou efeito de suplantar.
\section{Suplantar}
\begin{itemize}
\item {Grp. gram.:v. t.}
\end{itemize}
\begin{itemize}
\item {Utilização:Fig.}
\end{itemize}
\begin{itemize}
\item {Proveniência:(Lat. \textunderscore supplantare\textunderscore )}
\end{itemize}
Calcar com os pés, pisar.
Pôr debaixo dos pés; derribar.
Vencer.
Exceder.
Humilhar.
\section{Suplementar}
\begin{itemize}
\item {Grp. gram.:adj.}
\end{itemize}
Relativo a suplemento.
Que serve de suplemento.
Que amplia; adicional.
\section{Suplementário}
\begin{itemize}
\item {Grp. gram.:adj.}
\end{itemize}
O mesmo que \textunderscore suplementar\textunderscore .
\section{Supo}
\begin{itemize}
\item {Grp. gram.:m.}
\end{itemize}
Pequeno cesto, entre os Negros de Lourenço-Marques.
\section{Suppedâneo}
\begin{itemize}
\item {Grp. gram.:m.}
\end{itemize}
\begin{itemize}
\item {Utilização:Fig.}
\end{itemize}
\begin{itemize}
\item {Proveniência:(Lat. \textunderscore suppedaneum\textunderscore )}
\end{itemize}
Banco, em que se descansam os pés.
Peanha.
Estrado de madeira, em que o sacerdote põe os pés, em-quanto diz Missa.
Base.
\section{Suppeditar}
\begin{itemize}
\item {Grp. gram.:v. t.}
\end{itemize}
\begin{itemize}
\item {Proveniência:(Lat. \textunderscore suppeditare\textunderscore )}
\end{itemize}
Fornecer, ministrar.
\section{Supplantação}
\begin{itemize}
\item {Grp. gram.:f.}
\end{itemize}
\begin{itemize}
\item {Proveniência:(Do lat. \textunderscore supplantatio\textunderscore )}
\end{itemize}
Acto ou effeito de supplantar.
\section{Supplantador}
\begin{itemize}
\item {Grp. gram.:m.  e  adj.}
\end{itemize}
\begin{itemize}
\item {Proveniência:(Do lat. \textunderscore supplantator\textunderscore )}
\end{itemize}
O que supplanta.
\section{Supplantar}
\begin{itemize}
\item {Grp. gram.:v. t.}
\end{itemize}
\begin{itemize}
\item {Utilização:Fig.}
\end{itemize}
\begin{itemize}
\item {Proveniência:(Lat. \textunderscore supplantare\textunderscore )}
\end{itemize}
Calcar com os pés, pisar.
Pôr debaixo dos pés; derribar.
Vencer.
Exceder.
Humilhar.
\section{Supplementar}
\begin{itemize}
\item {Grp. gram.:adj.}
\end{itemize}
Relativo a supplemento.
Que serve de supplemento.
Que amplia; addicional.
\section{Supplementário}
\begin{itemize}
\item {Grp. gram.:adj.}
\end{itemize}
O mesmo que \textunderscore supplementar\textunderscore .
\section{Superumano}
\begin{itemize}
\item {Grp. gram.:adj.}
\end{itemize}
O mesmo que \textunderscore sobrehumano\textunderscore .
\section{Superumeral}
\begin{itemize}
\item {Grp. gram.:m.}
\end{itemize}
\begin{itemize}
\item {Proveniência:(Lat. \textunderscore superhumerole\textunderscore )}
\end{itemize}
Vestuário eclesiástico, entre os Hebreus.
\section{Suplementarmente}
\begin{itemize}
\item {Grp. gram.:adv.}
\end{itemize}
De modo suplementar.
\section{Suplemento}
\begin{itemize}
\item {Grp. gram.:m.}
\end{itemize}
\begin{itemize}
\item {Proveniência:(Lat. \textunderscore supplementum\textunderscore )}
\end{itemize}
Aquilo que serve para suprir.
O que se dá a mais.
Aquilo que se junta a um todo, para o ampliar, aperfeiçoar ou esclarecer.
Fôlha impressa, que accresce a um número, já publicado, de uma gazeta.
Adição, additamento.
Complemento.
Ângulo que, junto a outro, perfaz 180°.
Arco que, junto a outro, perfaz um semicírculo.
\section{Suplência}
\begin{itemize}
\item {Grp. gram.:f.}
\end{itemize}
\begin{itemize}
\item {Utilização:bras}
\end{itemize}
\begin{itemize}
\item {Utilização:Neol.}
\end{itemize}
Qualidade de suplente.
\section{Suplente}
\begin{itemize}
\item {Grp. gram.:m.  e  adj.}
\end{itemize}
\begin{itemize}
\item {Proveniência:(Lat. \textunderscore supplens\textunderscore )}
\end{itemize}
O que supre.
Substituto.
O que exerce certas funções ou deve exercê-las, na falta ou impedimento daquele a quem elas impendem ou impendiam com efectividade.
\section{Supletivo}
\begin{itemize}
\item {Grp. gram.:adj.}
\end{itemize}
\begin{itemize}
\item {Proveniência:(Lat. \textunderscore suppletivus\textunderscore )}
\end{itemize}
Que supre.
\section{Supletório}
\begin{itemize}
\item {Grp. gram.:adj.}
\end{itemize}
O mesmo que \textunderscore supletivo\textunderscore .
\section{Súplica}
\begin{itemize}
\item {Grp. gram.:f.}
\end{itemize}
Acto ou efeito de suplicar.
Rogativa; oração instante e humilde.
\section{Suplicação}
\begin{itemize}
\item {Grp. gram.:f.}
\end{itemize}
\begin{itemize}
\item {Proveniência:(Lat. \textunderscore supplicatio\textunderscore )}
\end{itemize}
O mesmo que \textunderscore súplica\textunderscore .
Antigo tribunal judicial de segunda instância.
\section{Suplicado}
\begin{itemize}
\item {Grp. gram.:m.}
\end{itemize}
\begin{itemize}
\item {Utilização:Jur.}
\end{itemize}
\begin{itemize}
\item {Proveniência:(De \textunderscore suplicar\textunderscore )}
\end{itemize}
Indivíduo, contra quem um suplicante requere em juízo.
\section{Suplicamento}
\begin{itemize}
\item {Grp. gram.:m.}
\end{itemize}
O mesmo que \textunderscore súplica\textunderscore .
\section{Suplicante}
\begin{itemize}
\item {Grp. gram.:m. ,  f.  e  adj.}
\end{itemize}
\begin{itemize}
\item {Proveniência:(Lat. \textunderscore supplicans\textunderscore )}
\end{itemize}
Pessôa, que súplica.
Requerente; o que pede mercê ou despacho.
O que pede humildemente, dobrando os joêlhos.
\section{Suplicar}
\begin{itemize}
\item {Grp. gram.:v. i.}
\end{itemize}
\begin{itemize}
\item {Proveniência:(Lat. \textunderscore supplicare\textunderscore )}
\end{itemize}
Pedir com instância; rogar.
\section{Suplicativo}
\begin{itemize}
\item {Grp. gram.:adj.}
\end{itemize}
\begin{itemize}
\item {Proveniência:(De \textunderscore suplicar\textunderscore )}
\end{itemize}
Que envolve súplica; suplicante. Cf. Castilho, \textunderscore Fastos\textunderscore , II, 569.
\section{Suplicatorio}
\begin{itemize}
\item {Grp. gram.:adj.}
\end{itemize}
\begin{itemize}
\item {Proveniência:(De \textunderscore suplicar\textunderscore )}
\end{itemize}
Que contém súplica: \textunderscore uma carta suplicatória\textunderscore .
\section{Súplice}
\begin{itemize}
\item {Grp. gram.:adj.}
\end{itemize}
\begin{itemize}
\item {Proveniência:(Lat. \textunderscore supplex\textunderscore )}
\end{itemize}
Que súplica; suplicante.
Que se prostra, pedindo.
\section{Supliciado}
\begin{itemize}
\item {Grp. gram.:adj.}
\end{itemize}
\begin{itemize}
\item {Grp. gram.:M.}
\end{itemize}
\begin{itemize}
\item {Proveniência:(De \textunderscore supliciar\textunderscore )}
\end{itemize}
Justiçado.
Aquele que sofreu suplício ou foi justiçado.
\section{Supliciante}
\begin{itemize}
\item {Grp. gram.:adj.}
\end{itemize}
Que suplicia.
\section{Supliciar}
\begin{itemize}
\item {Grp. gram.:v. t.}
\end{itemize}
\begin{itemize}
\item {Utilização:Fig.}
\end{itemize}
Fazer sofrer suplício a.
Fazer sofrer a pena de morte a.
Torturar.
Afligir, molestar.
\section{Suplício}
\begin{itemize}
\item {Grp. gram.:m.}
\end{itemize}
\begin{itemize}
\item {Utilização:Fig.}
\end{itemize}
\begin{itemize}
\item {Grp. gram.:Pl.}
\end{itemize}
\begin{itemize}
\item {Proveniência:(Lat. \textunderscore supplicium\textunderscore )}
\end{itemize}
Grande punição corporal, imposta por justiça.
Pena de morte.
Execução capital.
Pessôa ou coisa, que aflige muito.
Tortura, grande tormento.
Disciplinas ou correias, que servem para açoitar por penitência ou castigo.
\section{Supontar}
\begin{itemize}
\item {Grp. gram.:v. t.}
\end{itemize}
\begin{itemize}
\item {Utilização:Ant.}
\end{itemize}
\begin{itemize}
\item {Proveniência:(De \textunderscore sub...\textunderscore  + \textunderscore ponto\textunderscore )}
\end{itemize}
Inutilizar (uma escrita), com pontos?:«\textunderscore ...notários, que riscavam ou supontavam, em caso de êrro.\textunderscore »Herculano, \textunderscore Hist. de Port.\textunderscore , II, 430.
\section{Supor}
\begin{itemize}
\item {Grp. gram.:v. t.}
\end{itemize}
\begin{itemize}
\item {Proveniência:(Do lat. \textunderscore supponere\textunderscore )}
\end{itemize}
Alegar por hipótese ou têr como admitida (alguma coisa), para daí tirar uma conclusão.
Conjecturar.
Formar hipóteses sôbre.
Presumir; imaginar.
\section{Suporatório}
\begin{itemize}
\item {Grp. gram.:adj.}
\end{itemize}
\begin{itemize}
\item {Proveniência:(De \textunderscore suppuratorius\textunderscore )}
\end{itemize}
O mesmo que \textunderscore supurativo\textunderscore .
\section{Suportação}
\begin{itemize}
\item {Grp. gram.:f.}
\end{itemize}
Acto ou efeito de suportar.
\section{Suportar}
\begin{itemize}
\item {Grp. gram.:v. t.}
\end{itemize}
\begin{itemize}
\item {Proveniência:(Lat. \textunderscore supportare\textunderscore )}
\end{itemize}
Têr sôbre si: \textunderscore o burro suporta a carga\textunderscore .
Estar debaixo de.
Sustentar.
Sofrer, tolerar: \textunderscore suportar dores\textunderscore .
Aguentar.
Transigir com.
\section{Suportável}
\begin{itemize}
\item {Grp. gram.:adj.}
\end{itemize}
Que se póde suportar.
\section{Suporte}
\begin{itemize}
\item {Grp. gram.:m.}
\end{itemize}
\begin{itemize}
\item {Proveniência:(De \textunderscore suportar\textunderscore )}
\end{itemize}
Aquilo que suporta ou sustenta qualquer coisa.
Áquilo em que alguma coisa assenta ou se firma; sustentáculo.
\section{Suposição}
\begin{itemize}
\item {Grp. gram.:f.}
\end{itemize}
\begin{itemize}
\item {Proveniência:(Do lat. \textunderscore suppositio\textunderscore )}
\end{itemize}
Acto ou efeito de supor.
Acto de apresentar como verdadeiro o que se sabe que é falso.
Hipótese.
\section{Supositício}
\begin{itemize}
\item {Grp. gram.:adj.}
\end{itemize}
\begin{itemize}
\item {Proveniência:(Lat. \textunderscore suppositicius\textunderscore )}
\end{itemize}
Suposto.
Fingido.
\section{Supositivo}
\begin{itemize}
\item {Grp. gram.:adj.}
\end{itemize}
\begin{itemize}
\item {Proveniência:(Lat. \textunderscore suppositivus\textunderscore )}
\end{itemize}
O mesmo que \textunderscore supositício\textunderscore .
\section{Supositório}
\begin{itemize}
\item {Grp. gram.:m.}
\end{itemize}
\begin{itemize}
\item {Utilização:Med.}
\end{itemize}
\begin{itemize}
\item {Proveniência:(Lat. \textunderscore suppositorium\textunderscore )}
\end{itemize}
Medicamento sólido, em fórma cónica, que se aplica na parte interior do ânus.
\section{Suposto}
\begin{itemize}
\item {Grp. gram.:m.}
\end{itemize}
\begin{itemize}
\item {Grp. gram.:M.}
\end{itemize}
\begin{itemize}
\item {Grp. gram.:Prep.}
\end{itemize}
\begin{itemize}
\item {Proveniência:(Do lat. \textunderscore suppositus\textunderscore )}
\end{itemize}
Fictício; hipotético.
Aquilo que subsiste por si.
Substância; coisa suposta.
Pôsto; embora: \textunderscore suposto me enganasses...\textunderscore  Cf. Garrett, \textunderscore Catão\textunderscore , 21, 26 e 28.
\section{Supplementarmente}
\begin{itemize}
\item {Grp. gram.:adv.}
\end{itemize}
De modo supplementar.
\section{Supplemento}
\begin{itemize}
\item {Grp. gram.:m.}
\end{itemize}
\begin{itemize}
\item {Proveniência:(Lat. \textunderscore supplementum\textunderscore )}
\end{itemize}
Aquillo que serve para supprir.
O que se dá a mais.
Aquillo que se junta a um todo, para o ampliar, aperfeiçoar ou esclarecer.
Fôlha impressa, que accresce a um número, já publicado, de uma gazeta.
Addição, additamento.
Complemento.
Ângulo que, junto a outro, perfaz 180°.
Arco que, junto a outro, perfaz um semicírculo.
\section{Supplência}
\begin{itemize}
\item {Grp. gram.:f.}
\end{itemize}
\begin{itemize}
\item {Utilização:bras}
\end{itemize}
\begin{itemize}
\item {Utilização:Neol.}
\end{itemize}
Qualidade de supplente.
\section{Supplente}
\begin{itemize}
\item {Grp. gram.:m.  e  adj.}
\end{itemize}
\begin{itemize}
\item {Proveniência:(Lat. \textunderscore supplens\textunderscore )}
\end{itemize}
O que suppre.
Substituto.
O que exerce certas funcções ou deve exercê-las, na falta ou impedimento daquelle a quem ellas impendem ou impendiam com effectividade.
\section{Suppletivo}
\begin{itemize}
\item {Grp. gram.:adj.}
\end{itemize}
\begin{itemize}
\item {Proveniência:(Lat. \textunderscore suppletivus\textunderscore )}
\end{itemize}
Que suppre.
\section{Suppletório}
\begin{itemize}
\item {Grp. gram.:adj.}
\end{itemize}
O mesmo que \textunderscore suppletivo\textunderscore .
\section{Súpplica}
\begin{itemize}
\item {Grp. gram.:f.}
\end{itemize}
Acto ou effeito de supplicar.
Rogativa; oração instante e humilde.
\section{Supplicação}
\begin{itemize}
\item {Grp. gram.:f.}
\end{itemize}
\begin{itemize}
\item {Proveniência:(Lat. \textunderscore supplicatio\textunderscore )}
\end{itemize}
O mesmo que \textunderscore súpplica\textunderscore .
Antigo tribunal judicial de segunda instância.
\section{Supplicado}
\begin{itemize}
\item {Grp. gram.:m.}
\end{itemize}
\begin{itemize}
\item {Utilização:Jur.}
\end{itemize}
\begin{itemize}
\item {Proveniência:(De \textunderscore supplicar\textunderscore )}
\end{itemize}
Indivíduo, contra quem um supplicante requere em juízo.
\section{Supplicamento}
\begin{itemize}
\item {Grp. gram.:m.}
\end{itemize}
O mesmo que \textunderscore súpplica\textunderscore .
\section{Supplicante}
\begin{itemize}
\item {Grp. gram.:m. ,  f.  e  adj.}
\end{itemize}
\begin{itemize}
\item {Proveniência:(Lat. \textunderscore supplicans\textunderscore )}
\end{itemize}
Pessôa, que súpplica.
Requerente; o que pede mercê ou despacho.
O que pede humildemente, dobrando os joêlhos.
\section{Supplicar}
\begin{itemize}
\item {Grp. gram.:v. i.}
\end{itemize}
\begin{itemize}
\item {Proveniência:(Lat. \textunderscore supplicare\textunderscore )}
\end{itemize}
Pedir com instância; rogar.
\section{Supplicativo}
\begin{itemize}
\item {Grp. gram.:adj.}
\end{itemize}
\begin{itemize}
\item {Proveniência:(De \textunderscore supplicar\textunderscore )}
\end{itemize}
Que envolve súpplica; supplicante. Cf. Castilho, \textunderscore Fastos\textunderscore , II, 569.
\section{Supplicatório}
\begin{itemize}
\item {Grp. gram.:adj.}
\end{itemize}
\begin{itemize}
\item {Proveniência:(De \textunderscore supplicar\textunderscore )}
\end{itemize}
Que contém súpplica: \textunderscore uma carta supplicatória\textunderscore .
\section{Súpplice}
\begin{itemize}
\item {Grp. gram.:adj.}
\end{itemize}
\begin{itemize}
\item {Proveniência:(Lat. \textunderscore supplex\textunderscore )}
\end{itemize}
Que súpplica; supplicante.
Que se prostra, pedindo.
\section{Suppliciado}
\begin{itemize}
\item {Grp. gram.:adj.}
\end{itemize}
\begin{itemize}
\item {Grp. gram.:M.}
\end{itemize}
\begin{itemize}
\item {Proveniência:(De \textunderscore suppliciar\textunderscore )}
\end{itemize}
Justiçado.
Aquelle que soffreu supplício ou foi justiçado.
\section{Suppliciante}
\begin{itemize}
\item {Grp. gram.:adj.}
\end{itemize}
Que supplicia.
\section{Suppliciar}
\begin{itemize}
\item {Grp. gram.:v. t.}
\end{itemize}
\begin{itemize}
\item {Utilização:Fig.}
\end{itemize}
Fazer soffrer supplício a.
Fazer soffrer a pena de morte a.
Torturar.
Affligir, molestar.
\section{Supplício}
\begin{itemize}
\item {Grp. gram.:m.}
\end{itemize}
\begin{itemize}
\item {Utilização:Fig.}
\end{itemize}
\begin{itemize}
\item {Grp. gram.:Pl.}
\end{itemize}
\begin{itemize}
\item {Proveniência:(Lat. \textunderscore supplicium\textunderscore )}
\end{itemize}
Grande punição corporal, imposta por justiça.
Pena de morte.
Execução capital.
Pessôa ou coisa, que afflige muito.
Tortura, grande tormento.
Disciplinas ou correias, que servem para açoitar por penitência ou castigo.
\section{Suppontar}
\begin{itemize}
\item {Grp. gram.:v. t.}
\end{itemize}
\begin{itemize}
\item {Utilização:Ant.}
\end{itemize}
\begin{itemize}
\item {Proveniência:(De \textunderscore sub...\textunderscore  + \textunderscore ponto\textunderscore )}
\end{itemize}
Inutilizar (uma escrita), com pontos?:«\textunderscore ...notários, que riscavam ou suppontavam, em caso de êrro.\textunderscore »Herculano, \textunderscore Hist. de Port.\textunderscore , II, 430.
\section{Suppor}
\begin{itemize}
\item {Grp. gram.:v. t.}
\end{itemize}
\begin{itemize}
\item {Proveniência:(Do lat. \textunderscore supponere\textunderscore )}
\end{itemize}
Allegar por hypóthese ou têr como admittida (alguma coisa), para daí tirar uma conclusão.
Conjecturar.
Formar hypótheses sôbre.
Presumir; imaginar.
\section{Supportação}
\begin{itemize}
\item {Grp. gram.:f.}
\end{itemize}
Acto ou effeito de supportar.
\section{Supportar}
\begin{itemize}
\item {Grp. gram.:v. t.}
\end{itemize}
\begin{itemize}
\item {Proveniência:(Lat. \textunderscore supportare\textunderscore )}
\end{itemize}
Têr sôbre si: \textunderscore o burro supporta a carga\textunderscore .
Estar debaixo de.
Sustentar.
Soffrer, tolerar: \textunderscore supportar dores\textunderscore .
Aguentar.
Transigir com.
\section{Supportável}
\begin{itemize}
\item {Grp. gram.:adj.}
\end{itemize}
Que se póde supportar.
\section{Supporte}
\begin{itemize}
\item {Grp. gram.:m.}
\end{itemize}
\begin{itemize}
\item {Proveniência:(De \textunderscore supportar\textunderscore )}
\end{itemize}
Aquillo que supporta ou sustenta qualquer coisa.
Áquillo em que alguma coisa assenta ou se firma; sustentáculo.
\section{Supposição}
\begin{itemize}
\item {Grp. gram.:f.}
\end{itemize}
\begin{itemize}
\item {Proveniência:(Do lat. \textunderscore suppositio\textunderscore )}
\end{itemize}
Acto ou effeito de suppor.
Acto de apresentar como verdadeiro o que se sabe que é falso.
Hypóthese.
\section{Suppositício}
\begin{itemize}
\item {Grp. gram.:adj.}
\end{itemize}
\begin{itemize}
\item {Proveniência:(Lat. \textunderscore suppositicius\textunderscore )}
\end{itemize}
Supposto.
Fingido.
\section{Suppositivo}
\begin{itemize}
\item {Grp. gram.:adj.}
\end{itemize}
\begin{itemize}
\item {Proveniência:(Lat. \textunderscore suppositivus\textunderscore )}
\end{itemize}
O mesmo que \textunderscore suppositício\textunderscore .
\section{Suppositório}
\begin{itemize}
\item {Grp. gram.:m.}
\end{itemize}
\begin{itemize}
\item {Utilização:Med.}
\end{itemize}
\begin{itemize}
\item {Proveniência:(Lat. \textunderscore suppositorium\textunderscore )}
\end{itemize}
Medicamento sólido, em fórma cónica, que se applica na parte interior do ânus.
\section{Supposto}
\begin{itemize}
\item {Grp. gram.:m.}
\end{itemize}
\begin{itemize}
\item {Grp. gram.:M.}
\end{itemize}
\begin{itemize}
\item {Grp. gram.:Prep.}
\end{itemize}
\begin{itemize}
\item {Proveniência:(Do lat. \textunderscore suppositus\textunderscore )}
\end{itemize}
Fictício; hypothético.
Aquillo que subsiste por si.
Substância; coisa supposta.
Pôsto; embora: \textunderscore supposto me enganasses...\textunderscore  Cf. Garrett, \textunderscore Catão\textunderscore , 21, 26 e 28.
\section{Suppressão}
\begin{itemize}
\item {Grp. gram.:f.}
\end{itemize}
\begin{itemize}
\item {Proveniência:(Do lat. \textunderscore suppressio\textunderscore )}
\end{itemize}
Acto ou effeito de supprimir.
\section{Suppressivo}
\begin{itemize}
\item {Grp. gram.:adj.}
\end{itemize}
\begin{itemize}
\item {Proveniência:(Do lat. \textunderscore suppressus\textunderscore )}
\end{itemize}
Que supprime.
\section{Suppresso}
\begin{itemize}
\item {Grp. gram.:adj.}
\end{itemize}
\begin{itemize}
\item {Proveniência:(Lat. \textunderscore suppressus\textunderscore )}
\end{itemize}
O mesmo que [[supprimido|supprimir]]. Cf. Filinto, XVI, 261.
\section{Suppressor}
\begin{itemize}
\item {Grp. gram.:adj.}
\end{itemize}
\begin{itemize}
\item {Proveniência:(Lat. \textunderscore suppressor\textunderscore )}
\end{itemize}
O mesmo que \textunderscore suppressivo\textunderscore .
\section{Suppressório}
\begin{itemize}
\item {Grp. gram.:adj.}
\end{itemize}
O mesmo que \textunderscore suppressivo\textunderscore .
\section{Supprimir}
\begin{itemize}
\item {Grp. gram.:v. t.}
\end{itemize}
\begin{itemize}
\item {Proveniência:(Lat. \textunderscore supprimere\textunderscore )}
\end{itemize}
Impedir que appareça.
Não deixar publicar.
Impedir a vulgarização de.
Não mencionar, omittir: \textunderscore supprimir argumentos\textunderscore .
Eliminar, cortar: \textunderscore supprimir um capítulo\textunderscore .
Invalidar.
Extinguir.
\section{Supprir}
\textunderscore v. t.\textunderscore  (e der.)
(V. \textunderscore suprir\textunderscore , que é a fórma exacta, visto que o termo é de origem evolutiva e já sofreu outras alterações, antes de chegar á fórma actual)
\section{Suppuração}
\begin{itemize}
\item {Grp. gram.:f.}
\end{itemize}
\begin{itemize}
\item {Proveniência:(Do lat. \textunderscore suppuratio\textunderscore )}
\end{itemize}
Acto ou effeito de suppurar.
\section{Suppurante}
\begin{itemize}
\item {Grp. gram.:adj.}
\end{itemize}
\begin{itemize}
\item {Proveniência:(Lat. \textunderscore suppurans\textunderscore )}
\end{itemize}
Que suppura.
\section{Suppurar}
\begin{itemize}
\item {Grp. gram.:v. t.  e  i.}
\end{itemize}
\begin{itemize}
\item {Proveniência:(Lat. \textunderscore suppurare\textunderscore )}
\end{itemize}
Lançar pus.
Transformar-se em pus.
\section{Suppurativo}
\begin{itemize}
\item {Grp. gram.:adj.}
\end{itemize}
\begin{itemize}
\item {Grp. gram.:M.}
\end{itemize}
\begin{itemize}
\item {Proveniência:(De \textunderscore suppurar\textunderscore )}
\end{itemize}
Que produz ou facilita a suppuração.
Medicamente, que facilita a saída do pus.
\section{Suppuratório}
\begin{itemize}
\item {Grp. gram.:adj.}
\end{itemize}
\begin{itemize}
\item {Proveniência:(De \textunderscore suppuratorius\textunderscore )}
\end{itemize}
O mesmo que \textunderscore suppurativo\textunderscore .
\section{Supputação}
\begin{itemize}
\item {Grp. gram.:f.}
\end{itemize}
\begin{itemize}
\item {Proveniência:(Do lat. \textunderscore supputatio\textunderscore )}
\end{itemize}
Acto ou effeito de supputar.
\section{Supputar}
\begin{itemize}
\item {Grp. gram.:v. t.}
\end{itemize}
\begin{itemize}
\item {Proveniência:(Lat. \textunderscore supputare\textunderscore )}
\end{itemize}
O mesmo que \textunderscore calcular\textunderscore .
\section{Supra...}
\begin{itemize}
\item {Grp. gram.:pref.}
\end{itemize}
\begin{itemize}
\item {Proveniência:(Lat. \textunderscore supra\textunderscore )}
\end{itemize}
(designativo de \textunderscore superioridade\textunderscore , \textunderscore excellência\textunderscore , \textunderscore excesso\textunderscore , etc.)
\section{Supra-axilar}
\begin{itemize}
\item {Grp. gram.:adj.}
\end{itemize}
\begin{itemize}
\item {Utilização:Bot.}
\end{itemize}
Que está acima da axilla de uma fôlha.
\section{Supracitado}
\begin{itemize}
\item {Grp. gram.:adj.}
\end{itemize}
\begin{itemize}
\item {Proveniência:(De \textunderscore supra...\textunderscore  + \textunderscore citado\textunderscore )}
\end{itemize}
Citado ou mencionado acima ou anteriormente.
\section{Supradito}
\begin{itemize}
\item {Grp. gram.:adj.}
\end{itemize}
\begin{itemize}
\item {Proveniência:(De \textunderscore supra...\textunderscore  + \textunderscore dito\textunderscore )}
\end{itemize}
O mesmo que \textunderscore sobredito\textunderscore .
\section{Supraexcitar}
\begin{itemize}
\item {Grp. gram.:v. t.}
\end{itemize}
O mesmo que \textunderscore sobreexcitar\textunderscore . Cf. Júl. Dinis, \textunderscore Serões\textunderscore , 102.
\section{Suprajuraico}
\begin{itemize}
\item {Grp. gram.:adj.}
\end{itemize}
O mesmo que \textunderscore suprajurássico\textunderscore .
\section{Suprajurássico}
\begin{itemize}
\item {Grp. gram.:adj.}
\end{itemize}
\begin{itemize}
\item {Utilização:Geol.}
\end{itemize}
\begin{itemize}
\item {Proveniência:(De \textunderscore supra...\textunderscore  + \textunderscore jurássico\textunderscore )}
\end{itemize}
Diz-se do terreno superior ao calcário jurássico.
\section{Supralunar}
\begin{itemize}
\item {Grp. gram.:adj.}
\end{itemize}
Que está superior á Lua, em opposição a \textunderscore sublunar\textunderscore . Cf. Camillo, \textunderscore Doze Casam.\textunderscore , 60.
\section{Supramundano}
\begin{itemize}
\item {Grp. gram.:adj.}
\end{itemize}
\begin{itemize}
\item {Proveniência:(De \textunderscore supra...\textunderscore  + \textunderscore mundano\textunderscore )}
\end{itemize}
Que \textunderscore é\textunderscore  superior ao mundo.
\section{Supranatural}
\begin{itemize}
\item {Grp. gram.:adj.}
\end{itemize}
\begin{itemize}
\item {Proveniência:(De \textunderscore supra...\textunderscore  + \textunderscore natural\textunderscore )}
\end{itemize}
O mesmo que \textunderscore sobrenatural\textunderscore .
\section{Supranaturalismo}
\begin{itemize}
\item {Grp. gram.:m.}
\end{itemize}
\begin{itemize}
\item {Proveniência:(De \textunderscore supra...\textunderscore  + \textunderscore naturalismo\textunderscore )}
\end{itemize}
Qualidade do que é sobrenatural.
Doutrina dos que admittem uma intervenção sobrenatural nas coisas do mundo.
\section{Supranaturalista}
\begin{itemize}
\item {Grp. gram.:m. ,  f.  e  adj.}
\end{itemize}
\begin{itemize}
\item {Proveniência:(De \textunderscore supra...\textunderscore  + \textunderscore naturalista\textunderscore )}
\end{itemize}
Diz-se da pessôa, que é partidaria do supranaturalismo.
\section{Supranaturalmente}
\begin{itemize}
\item {Grp. gram.:adv.}
\end{itemize}
De modo supranatural.
\section{Supranumerado}
\begin{itemize}
\item {Grp. gram.:adj.}
\end{itemize}
\begin{itemize}
\item {Proveniência:(De \textunderscore supra...\textunderscore  + \textunderscore numerado\textunderscore )}
\end{itemize}
Numerado acima ou antes ou atrás.
\section{Supranumerário}
\begin{itemize}
\item {Grp. gram.:adj.}
\end{itemize}
\begin{itemize}
\item {Grp. gram.:M.}
\end{itemize}
\begin{itemize}
\item {Proveniência:(De \textunderscore supra\textunderscore  + \textunderscore número\textunderscore )}
\end{itemize}
Que passa além do número estabelecido: \textunderscore carteiros supranumerários\textunderscore .
O que está a mais; o que é supranumerário.
\section{Suprarenal}
\begin{itemize}
\item {fónica:re}
\end{itemize}
\begin{itemize}
\item {Grp. gram.:adj.}
\end{itemize}
\begin{itemize}
\item {Utilização:Anat.}
\end{itemize}
\begin{itemize}
\item {Proveniência:(De \textunderscore supra...\textunderscore  + \textunderscore renal\textunderscore )}
\end{itemize}
Situado acima dos rins.
\section{Suprarrenal}
\begin{itemize}
\item {Grp. gram.:adj.}
\end{itemize}
\begin{itemize}
\item {Utilização:Anat.}
\end{itemize}
\begin{itemize}
\item {Proveniência:(De \textunderscore supra...\textunderscore  + \textunderscore renal\textunderscore )}
\end{itemize}
Situado acima dos rins.
\section{Suprasensível}
\begin{itemize}
\item {fónica:sen}
\end{itemize}
\begin{itemize}
\item {Grp. gram.:adj.}
\end{itemize}
\begin{itemize}
\item {Proveniência:(De \textunderscore supra...\textunderscore  + \textunderscore sensível\textunderscore )}
\end{itemize}
O mesmo que \textunderscore supersensível\textunderscore .
\section{Suprassensível}
\begin{itemize}
\item {Grp. gram.:adj.}
\end{itemize}
\begin{itemize}
\item {Proveniência:(De \textunderscore supra...\textunderscore  + \textunderscore sensível\textunderscore )}
\end{itemize}
O mesmo que \textunderscore supersensível\textunderscore .
\section{Suprassumo}
\begin{itemize}
\item {Grp. gram.:m.}
\end{itemize}
O mesmo que \textunderscore suprassúmum\textunderscore . Cf. Macedo, \textunderscore Burros\textunderscore , 345.
\section{Suprassúmum}
\begin{itemize}
\item {Grp. gram.:m.}
\end{itemize}
\begin{itemize}
\item {Proveniência:(Do lat. \textunderscore supra...\textunderscore  + \textunderscore summus\textunderscore )}
\end{itemize}
O ponto mais elevado.
Culminância; requinte:«\textunderscore ...o suprassúmum das vantagens conjugaes.\textunderscore »Silv. da Mota, \textunderscore Viag. na Gall.\textunderscore , 190.
\section{Suprasummo}
\begin{itemize}
\item {fónica:su}
\end{itemize}
\begin{itemize}
\item {Grp. gram.:m.}
\end{itemize}
O mesmo que \textunderscore suprasúmmum\textunderscore . Cf. Macedo, \textunderscore Burros\textunderscore , 345.
\section{Suprasúmmum}
\begin{itemize}
\item {fónica:sum}
\end{itemize}
\begin{itemize}
\item {Grp. gram.:m.}
\end{itemize}
\begin{itemize}
\item {Proveniência:(Do lat. \textunderscore supra...\textunderscore  + \textunderscore summus\textunderscore )}
\end{itemize}
O ponto mais elevado.
Culminância; requinte:«\textunderscore ...o suprasúmmum das vantagens conjugaes.\textunderscore »Silv. da Mota, \textunderscore Viag. na Gall.\textunderscore , 190.
\section{Supraterrâneo}
\begin{itemize}
\item {Grp. gram.:adj.}
\end{itemize}
\begin{itemize}
\item {Proveniência:(De \textunderscore supra...\textunderscore  + \textunderscore terra\textunderscore )}
\end{itemize}
Que está ou se realiza sôbre a terra.
Relativo á superfície da terra.
\section{Suprathorácico}
\begin{itemize}
\item {Grp. gram.:adj.}
\end{itemize}
\begin{itemize}
\item {Utilização:Anat.}
\end{itemize}
\begin{itemize}
\item {Proveniência:(De \textunderscore supra...\textunderscore  + \textunderscore thorácico\textunderscore )}
\end{itemize}
Que está acima do thórax.
\section{Supratorácico}
\begin{itemize}
\item {Grp. gram.:adj.}
\end{itemize}
\begin{itemize}
\item {Utilização:Anat.}
\end{itemize}
\begin{itemize}
\item {Proveniência:(De \textunderscore supra...\textunderscore  + \textunderscore torácico\textunderscore )}
\end{itemize}
Que está acima do tórax.
\section{Supratrochlear}
\begin{itemize}
\item {Grp. gram.:adj.}
\end{itemize}
\begin{itemize}
\item {Utilização:Anat.}
\end{itemize}
Situado sôbre a tróchlea.
\section{Supratroclear}
\begin{itemize}
\item {Grp. gram.:adj.}
\end{itemize}
\begin{itemize}
\item {Utilização:Anat.}
\end{itemize}
Situado sôbre a tróclea.
\section{Supremacia}
\begin{itemize}
\item {Grp. gram.:f.}
\end{itemize}
\begin{itemize}
\item {Proveniência:(De \textunderscore supremo\textunderscore )}
\end{itemize}
Superioridade ou grandeza maior que todas as outras.
Poder supremo.
Preponderância; hegemonia.
\section{Supremamente}
\begin{itemize}
\item {Grp. gram.:adv.}
\end{itemize}
De modo supremo.
\section{Supremo}
\begin{itemize}
\item {Grp. gram.:adj.}
\end{itemize}
\begin{itemize}
\item {Grp. gram.:M.}
\end{itemize}
\begin{itemize}
\item {Utilização:Fam.}
\end{itemize}
\begin{itemize}
\item {Proveniência:(Lat. \textunderscore supremus\textunderscore )}
\end{itemize}
Que está acima de tudo, no seu gênero.
Relativo a Deus.
Que está depois de tudo, derradeiro.
Superior.
Supremo Tribunal de Justiça.
\section{Supressão}
\begin{itemize}
\item {Grp. gram.:f.}
\end{itemize}
\begin{itemize}
\item {Proveniência:(Do lat. \textunderscore suppressio\textunderscore )}
\end{itemize}
Acto ou efeito de suprimir.
\section{Supressivo}
\begin{itemize}
\item {Grp. gram.:adj.}
\end{itemize}
\begin{itemize}
\item {Proveniência:(Do lat. \textunderscore suppressus\textunderscore )}
\end{itemize}
Que suprime.
\section{Supresso}
\begin{itemize}
\item {Grp. gram.:adj.}
\end{itemize}
\begin{itemize}
\item {Proveniência:(Lat. \textunderscore suppressus\textunderscore )}
\end{itemize}
O mesmo que [[suprimido|suprimir]]. Cf. Filinto, XVI, 261.
\section{Supressor}
\begin{itemize}
\item {Grp. gram.:adj.}
\end{itemize}
\begin{itemize}
\item {Proveniência:(Lat. \textunderscore suppressor\textunderscore )}
\end{itemize}
O mesmo que \textunderscore supressivo\textunderscore .
\section{Supressório}
\begin{itemize}
\item {Grp. gram.:adj.}
\end{itemize}
O mesmo que \textunderscore supressivo\textunderscore .
\section{Supridor}
\begin{itemize}
\item {Grp. gram.:m.  e  adj.}
\end{itemize}
O que supre.
\section{Suprimento}
\begin{itemize}
\item {Grp. gram.:m.}
\end{itemize}
\begin{itemize}
\item {Utilização:Prov.}
\end{itemize}
\begin{itemize}
\item {Utilização:trasm.}
\end{itemize}
Acto ou effeito de suprir.
Supplemento.
Auxílio.
Empréstimo.
Substância; sustento: \textunderscore comida de pouco suprimento\textunderscore .
\section{Suprimir}
\begin{itemize}
\item {Grp. gram.:v. t.}
\end{itemize}
\begin{itemize}
\item {Proveniência:(Lat. \textunderscore supprimere\textunderscore )}
\end{itemize}
Impedir que apareça.
Não deixar publicar.
Impedir a vulgarização de.
Não mencionar, omitir: \textunderscore suprimir argumentos\textunderscore .
Eliminar, cortar: \textunderscore suprimir um capítulo\textunderscore .
Invalidar.
Extinguir.
\section{Suprir}
\begin{itemize}
\item {Grp. gram.:v. t.}
\end{itemize}
\begin{itemize}
\item {Grp. gram.:V. i.}
\end{itemize}
\begin{itemize}
\item {Proveniência:(Do lat. \textunderscore supplere\textunderscore )}
\end{itemize}
Preencher a falta de; substituir.
Complementar.
Remediar: \textunderscore suprir necessidades\textunderscore .
Prover.
Auxiliar.
Servir de auxílio; acudir.
Sêr substituto ou supplente.
\section{Suprível}
\begin{itemize}
\item {Grp. gram.:adj.}
\end{itemize}
Que se póde suprir.
\section{Supuração}
\begin{itemize}
\item {Grp. gram.:f.}
\end{itemize}
\begin{itemize}
\item {Proveniência:(Do lat. \textunderscore suppuratio\textunderscore )}
\end{itemize}
Acto ou efeito de supurar.
\section{Supurante}
\begin{itemize}
\item {Grp. gram.:adj.}
\end{itemize}
\begin{itemize}
\item {Proveniência:(Lat. \textunderscore suppurans\textunderscore )}
\end{itemize}
Que supura.
\section{Supurar}
\begin{itemize}
\item {Grp. gram.:v. t.  e  i.}
\end{itemize}
\begin{itemize}
\item {Proveniência:(Lat. \textunderscore suppurare\textunderscore )}
\end{itemize}
Lançar pus.
Transformar-se em pus.
\section{Supurativo}
\begin{itemize}
\item {Grp. gram.:adj.}
\end{itemize}
\begin{itemize}
\item {Grp. gram.:M.}
\end{itemize}
\begin{itemize}
\item {Proveniência:(De \textunderscore supurar\textunderscore )}
\end{itemize}
Que produz ou facilita a supuração.
Medicamente, que facilita a saída do pus.
\section{Suputação}
\begin{itemize}
\item {Grp. gram.:f.}
\end{itemize}
\begin{itemize}
\item {Proveniência:(Do lat. \textunderscore supputatio\textunderscore )}
\end{itemize}
Acto ou efeito de suputar.
\section{Suputar}
\begin{itemize}
\item {Grp. gram.:v. t.}
\end{itemize}
\begin{itemize}
\item {Proveniência:(Lat. \textunderscore supputare\textunderscore )}
\end{itemize}
O mesmo que \textunderscore calcular\textunderscore .
\section{Suquidora}
\begin{itemize}
\item {Grp. gram.:f.}
\end{itemize}
\begin{itemize}
\item {Utilização:Gír.}
\end{itemize}
\begin{itemize}
\item {Proveniência:(De \textunderscore suquir\textunderscore )}
\end{itemize}
Bôca.
\section{Suquir}
\begin{itemize}
\item {Grp. gram.:v. i.}
\end{itemize}
\begin{itemize}
\item {Utilização:Gír.}
\end{itemize}
Comer.
\section{Sura}
\begin{itemize}
\item {Grp. gram.:f.}
\end{itemize}
Suco do cacho da palmeira.
\section{Sura}
\begin{itemize}
\item {Grp. gram.:f.}
\end{itemize}
\begin{itemize}
\item {Proveniência:(Lat. \textunderscore sura\textunderscore )}
\end{itemize}
O mesmo que \textunderscore panturrilha\textunderscore . Cf. Macedo, \textunderscore Burros\textunderscore , 290.
\section{Sural}
\begin{itemize}
\item {Grp. gram.:adj.}
\end{itemize}
\begin{itemize}
\item {Utilização:Anat.}
\end{itemize}
\begin{itemize}
\item {Proveniência:(Do lat. \textunderscore sura\textunderscore )}
\end{itemize}
Relativo á barriga da perna.
\section{Surbião}
\begin{itemize}
\item {Grp. gram.:m.}
\end{itemize}
\begin{itemize}
\item {Utilização:Ant.}
\end{itemize}
Movimento annual e desencontrado das águas do mar, na occasião das calmarias. Cf. Pant. de Aveiro, \textunderscore Itiner.\textunderscore , 300 v.^o, (2.^a ed.).
\section{Sorpreendente}
\begin{itemize}
\item {Grp. gram.:adj.}
\end{itemize}
Que sorpreende; admirável, maravilhoso.
\section{Sorpreendentemente}
\begin{itemize}
\item {Grp. gram.:adv.}
\end{itemize}
De modo sorpreendente.
\section{Sorpreender}
\begin{itemize}
\item {Grp. gram.:v. t.}
\end{itemize}
\begin{itemize}
\item {Utilização:Fig.}
\end{itemize}
\begin{itemize}
\item {Grp. gram.:V. i.}
\end{itemize}
\begin{itemize}
\item {Proveniência:(Do lat. \textunderscore super\textunderscore  + \textunderscore prehendere\textunderscore )}
\end{itemize}
Apanhar subitamente.
Surgir ou apparecer de repente a.
Maravilhar, causar sorpresa a.
Fazer cair em êrro.
Causar sorpresa.
\section{Sorpreendido}
\begin{itemize}
\item {Grp. gram.:adj.}
\end{itemize}
Que se sorpreendeu; que foi objecto de sorpresa.
Admirado, por motivo imprevisto.
\section{Sorpresa}
\begin{itemize}
\item {fónica:prê}
\end{itemize}
\begin{itemize}
\item {Grp. gram.:f.}
\end{itemize}
\begin{itemize}
\item {Proveniência:(De \textunderscore sorpreso\textunderscore )}
\end{itemize}
Acto ou effeito de sorpreender.
Aquillo que sorpreende.
Sobresalto.
Prazer inesperado.
Notícia ou coisa, que alguém prepara ou calcula, para sorpreender outrem agradavelmente.
O mesmo ou melhór que \textunderscore surpresa\textunderscore . Cf. Camillo, \textunderscore Cav. em Ruínas\textunderscore , 49.
\section{Sorpresar}
\begin{itemize}
\item {Grp. gram.:v. t.}
\end{itemize}
\begin{itemize}
\item {Proveniência:(De \textunderscore sorpresa\textunderscore )}
\end{itemize}
O mesmo que \textunderscore sorpreender\textunderscore .
\section{Sorpreso}
\begin{itemize}
\item {fónica:prê}
\end{itemize}
\begin{itemize}
\item {Grp. gram.:adj.}
\end{itemize}
\begin{itemize}
\item {Proveniência:(Do lat. \textunderscore super\textunderscore  + \textunderscore prehensus\textunderscore )}
\end{itemize}
Sorpreendido; perplexo.
\section{Surça}
\begin{itemize}
\item {Grp. gram.:f.}
\end{itemize}
\begin{itemize}
\item {Utilização:Prov.}
\end{itemize}
\begin{itemize}
\item {Utilização:trasm.}
\end{itemize}
Môlho de vinho, alho, sal e pimenta, em que se deita carne de porco, para depois a ensacar.
\section{Surcar}
\begin{itemize}
\item {Grp. gram.:v. t.}
\end{itemize}
O mesmo que \textunderscore sulcar\textunderscore . Cf. Camillo, \textunderscore Caveira\textunderscore , 75.
\section{Súrculo}
\begin{itemize}
\item {Grp. gram.:m.}
\end{itemize}
\begin{itemize}
\item {Utilização:Bot.}
\end{itemize}
\begin{itemize}
\item {Utilização:Med.}
\end{itemize}
\begin{itemize}
\item {Proveniência:(Lat. \textunderscore surculus\textunderscore )}
\end{itemize}
Espécie de tronco, próprio dos musgos, guarnecido sempre de folíolos ou escamas persistentes.
Filete, ramúsculo.
\section{Surdamente}
\begin{itemize}
\item {Grp. gram.:adv.}
\end{itemize}
De modo surdo; á socapa; secretamente.
\section{Surdas, ás}
\begin{itemize}
\item {Grp. gram.:loc. adv.}
\end{itemize}
\begin{itemize}
\item {Proveniência:(De \textunderscore surdo\textunderscore )}
\end{itemize}
Sem ruído:«\textunderscore perpassar ás surdas\textunderscore ». Camillo, \textunderscore Mulher Fatal\textunderscore , 53.
\section{Surdear}
\begin{itemize}
\item {Grp. gram.:v. i.}
\end{itemize}
Fingir-se surdo.
\section{Surdescente}
\begin{itemize}
\item {Grp. gram.:adj.}
\end{itemize}
\begin{itemize}
\item {Proveniência:(Lat. \textunderscore surdescens\textunderscore )}
\end{itemize}
Que faz ensurdecer com ruído. Cf. Filinto. VIII, 96.
\section{Surdez}
\begin{itemize}
\item {Grp. gram.:f.}
\end{itemize}
Qualidade do que é surdo.
\section{Surdimutismo}
\begin{itemize}
\item {Grp. gram.:m.}
\end{itemize}
Qualidade de surdo-mudo. Cf. Castilho, \textunderscore Fastos\textunderscore , III, 389.
\section{Surdina}
\begin{itemize}
\item {Grp. gram.:f.}
\end{itemize}
\begin{itemize}
\item {Grp. gram.:Loc. adv.}
\end{itemize}
\begin{itemize}
\item {Proveniência:(It. \textunderscore sordina\textunderscore )}
\end{itemize}
Peça, com que se enfraquecem os sons, nos instrumentos de corda.
Capotasto.
\textunderscore Á surdina\textunderscore , pela calada, á socapa.
\section{Surdinar}
\begin{itemize}
\item {Grp. gram.:v. i.}
\end{itemize}
\begin{itemize}
\item {Utilização:bras}
\end{itemize}
\begin{itemize}
\item {Utilização:Neol.}
\end{itemize}
\begin{itemize}
\item {Proveniência:(De \textunderscore surdina\textunderscore )}
\end{itemize}
Produzir murmúrio suave.
Rumorejar brandamente.
\section{Surdir}
\begin{itemize}
\item {Grp. gram.:v. i.}
\end{itemize}
Sair para fóra.
Surgir.
Sobresaír.
Emergir, vir á tona da água.
Proseguir.
(Por \textunderscore surtir\textunderscore )
\section{Surdissimamente}
\begin{itemize}
\item {Grp. gram.:adv.}
\end{itemize}
\begin{itemize}
\item {Proveniência:(De \textunderscore surdo\textunderscore )}
\end{itemize}
Em tom ou voz muito baixa. Cf. Castilho, \textunderscore Tosquia\textunderscore .
\section{Surdista}
\begin{itemize}
\item {Grp. gram.:m.  e  adj.}
\end{itemize}
\begin{itemize}
\item {Proveniência:(De \textunderscore surdir\textunderscore ?)}
\end{itemize}
O que tripula um salva-vidas, e que tem por dever ir em soccorro de náufragos.
\section{Surdivém}
\begin{itemize}
\item {Grp. gram.:m.}
\end{itemize}
\begin{itemize}
\item {Utilização:Prov.}
\end{itemize}
O mesmo que \textunderscore cedovém\textunderscore .
\section{Surdo}
\begin{itemize}
\item {Grp. gram.:adj.}
\end{itemize}
\begin{itemize}
\item {Utilização:Fig.}
\end{itemize}
\begin{itemize}
\item {Grp. gram.:M.}
\end{itemize}
\begin{itemize}
\item {Proveniência:(Lat. \textunderscore surdus\textunderscore )}
\end{itemize}
Que não ouve.
Que ouve pouco.
Que é pouco sonoro.
Que mal se faz ouvir: \textunderscore falar, com voz surda\textunderscore .
Feito em silêncio, com pouco ou nenhum ruído.
Secreto, esconso.
Que tem pouco brilho, (falando-se de pintura)
Diz-se das vogaes \textunderscore e\textunderscore  e \textunderscore o\textunderscore , quando soam escassamente, como nas partículas \textunderscore de\textunderscore  e \textunderscore do\textunderscore .
Inflexível; implacável: \textunderscore surdo aos meus clamores\textunderscore .
Aquelle que não ouve.
\section{Surdo-mudez}
\begin{itemize}
\item {Grp. gram.:f.}
\end{itemize}
O mesmo que \textunderscore surdimutismo\textunderscore .
(Cp. it. \textunderscore surdo-mutezza\textunderscore )
\section{Surdo-mudo}
\begin{itemize}
\item {Grp. gram.:m.  e  adj.}
\end{itemize}
Indivíduo, que é surdo e mudo ao mesmo tempo, ou que nasceu surdo e por isso não pôde nunca apprender a falar.
\section{Suredo}
\begin{itemize}
\item {fónica:surê}
\end{itemize}
\begin{itemize}
\item {Grp. gram.:m.}
\end{itemize}
\begin{itemize}
\item {Utilização:Des.}
\end{itemize}
O mesmo que \textunderscore carapau\textunderscore .
\section{Surena}
\begin{itemize}
\item {Grp. gram.:m.}
\end{itemize}
\begin{itemize}
\item {Proveniência:(Lat. \textunderscore surena\textunderscore )}
\end{itemize}
Antigo magistrado, immediato ao Rei, entre os Parthos.
\section{Surgente}
\begin{itemize}
\item {Grp. gram.:adj.}
\end{itemize}
\begin{itemize}
\item {Grp. gram.:F.}
\end{itemize}
Que surge.
O mesmo que \textunderscore nascente\textunderscore . Cf. Castilho, \textunderscore Felic. pela Agr.\textunderscore , 175.
\section{Surgião}
\begin{itemize}
\item {Grp. gram.:m.}
\end{itemize}
\begin{itemize}
\item {Utilização:pop.}
\end{itemize}
\begin{itemize}
\item {Utilização:Ant.}
\end{itemize}
O mesmo que \textunderscore cirurgião\textunderscore . Cf. J. A. Serrano, \textunderscore Osteologia\textunderscore , I, p. LXI.
(Cp. ingl. \textunderscore surgeon\textunderscore )
\section{Surgidoiro}
\begin{itemize}
\item {Grp. gram.:m.}
\end{itemize}
\begin{itemize}
\item {Proveniência:(De \textunderscore surgir\textunderscore )}
\end{itemize}
Lugar, onde surgem ou ancoram navios.
\section{Surgidouro}
\begin{itemize}
\item {Grp. gram.:m.}
\end{itemize}
\begin{itemize}
\item {Proveniência:(De \textunderscore surgir\textunderscore )}
\end{itemize}
Lugar, onde surgem ou ancoram navios.
\section{Surgir}
\begin{itemize}
\item {Grp. gram.:v. i.}
\end{itemize}
\begin{itemize}
\item {Proveniência:(Lat. \textunderscore surgere\textunderscore )}
\end{itemize}
Erguer-se; despontar; nascer: \textunderscore surgiu a Lua\textunderscore .
Irromper.
Apparecer; surdir.
Aportar, lançar âncora.
Chegar.
Despertar.
Occorrer.
\section{Suriana}
\begin{itemize}
\item {Grp. gram.:f.}
\end{itemize}
Gênero de plantas.
\section{Suriapana}
\begin{itemize}
\item {Grp. gram.:f.}
\end{itemize}
Espécie de pendão indiano. Cf. Th. Ribeiro, \textunderscore Jornadas\textunderscore , II, 113.
\section{Suricate}
\begin{itemize}
\item {Grp. gram.:m.}
\end{itemize}
Gênero de mammíferos digitígrados, carnívoros.
\section{Súrio}
\begin{itemize}
\item {Grp. gram.:adj.}
\end{itemize}
\begin{itemize}
\item {Utilização:Ant.}
\end{itemize}
O mesmo que \textunderscore suro\textunderscore .
\section{Surjão}
\begin{itemize}
\item {Grp. gram.:m.}
\end{itemize}
\begin{itemize}
\item {Utilização:Ant.}
\end{itemize}
O mesmo que \textunderscore surgião\textunderscore .
\section{Surnalho}
\begin{itemize}
\item {Grp. gram.:m.}
\end{itemize}
\begin{itemize}
\item {Utilização:T. de Miranda}
\end{itemize}
Lenço de assoar.
\section{Surnar}
\begin{itemize}
\item {Grp. gram.:v. t.}
\end{itemize}
\begin{itemize}
\item {Utilização:Gír. ant. de marujos.}
\end{itemize}
Surripiar, larapiar.
\section{Suro}
Que não tem rabo; derrabado: \textunderscore uma gallinha sura\textunderscore .
(Cast. \textunderscore zuro\textunderscore )
\section{Surobeco}
\begin{itemize}
\item {Grp. gram.:m.  e  adj.}
\end{itemize}
O mesmo que \textunderscore surrobeco\textunderscore .
\section{Suropo}
\begin{itemize}
\item {Grp. gram.:m.}
\end{itemize}
Nome, que na Índia Portuguesa se dá á serpente \textunderscore bôa\textunderscore .
\section{Surprehendente}
\begin{itemize}
\item {Grp. gram.:adj.}
\end{itemize}
Que surprehende; admirável, maravilhoso.
\section{Surprehendentemente}
\begin{itemize}
\item {Grp. gram.:adv.}
\end{itemize}
De modo surprehendente.
\section{Surprehender}
\begin{itemize}
\item {Grp. gram.:v. t.}
\end{itemize}
\begin{itemize}
\item {Utilização:Fig.}
\end{itemize}
\begin{itemize}
\item {Grp. gram.:V. i.}
\end{itemize}
\begin{itemize}
\item {Proveniência:(Do lat. \textunderscore super\textunderscore  + \textunderscore prehendere\textunderscore )}
\end{itemize}
Apanhar subitamente.
Surgir ou apparecer de repente a.
Maravilhar, causar surpresa a.
Fazer cair em êrro.
Causar surpresa.
\section{Surprehendido}
\begin{itemize}
\item {Grp. gram.:adj.}
\end{itemize}
Que se surprehendeu; que foi objecto de surpresa.
Admirado, por motivo imprevisto.
\section{Surpresa}
\begin{itemize}
\item {Grp. gram.:f.}
\end{itemize}
\begin{itemize}
\item {Proveniência:(De \textunderscore surpreso\textunderscore )}
\end{itemize}
Acto ou effeito de surprehender.
Aquillo que surprehende.
Sobresalto.
Prazer inesperado.
Notícia ou coisa, que alguém prepara ou calcula, para surprehender outrem agradavelmente.
\section{Surpresar}
\begin{itemize}
\item {Grp. gram.:v. t.}
\end{itemize}
\begin{itemize}
\item {Proveniência:(De \textunderscore surpresa\textunderscore )}
\end{itemize}
O mesmo que \textunderscore surprehender\textunderscore .
\section{Surpreso}
\begin{itemize}
\item {Grp. gram.:adj.}
\end{itemize}
\begin{itemize}
\item {Proveniência:(Do lat. \textunderscore super\textunderscore  + \textunderscore prehensus\textunderscore )}
\end{itemize}
Surprehendido; perplexo.
\section{Surra}
\begin{itemize}
\item {Grp. gram.:f.}
\end{itemize}
\begin{itemize}
\item {Utilização:Pop.}
\end{itemize}
\begin{itemize}
\item {Proveniência:(De \textunderscore surrar\textunderscore )}
\end{itemize}
Pancadaria; tunda.
\section{Surra}
\begin{itemize}
\item {Grp. gram.:f.}
\end{itemize}
(Contr. de \textunderscore súrria\textunderscore )
\section{Surra-burra}
\begin{itemize}
\item {Grp. gram.:f.}
\end{itemize}
\begin{itemize}
\item {Utilização:T. de Serpa e Moura}
\end{itemize}
O mesmo que \textunderscore sarrabulho\textunderscore .
\section{Surrado}
\begin{itemize}
\item {Grp. gram.:adj.}
\end{itemize}
\begin{itemize}
\item {Utilização:Gír.}
\end{itemize}
\begin{itemize}
\item {Proveniência:(De \textunderscore surrar\textunderscore )}
\end{itemize}
Curtido.
Pisado.
Maltratado.
O mesmo que [[furtado|furtar]].
\section{Surrado}
\begin{itemize}
\item {Grp. gram.:adj.}
\end{itemize}
Coberto de surro; sujo. Cf. Filinto, VIII, 37.
\section{Surrador}
\begin{itemize}
\item {Grp. gram.:m.  e  adj.}
\end{itemize}
O que surra.
\section{Surraipa}
\begin{itemize}
\item {Grp. gram.:f.}
\end{itemize}
\begin{itemize}
\item {Utilização:Prov.}
\end{itemize}
Subsolo, constituído por uma camada compacta e dura de saibro.
\section{Surramento}
\begin{itemize}
\item {Grp. gram.:m.}
\end{itemize}
Acto ou effeito de surrar.
\section{Surrão}
\begin{itemize}
\item {Grp. gram.:m.}
\end{itemize}
\begin{itemize}
\item {Utilização:Pleb.}
\end{itemize}
Bolsa ou saco de coiro, destinado especialmente a farnel de pastores.
Fato sujo e gasto.
O mesmo que \textunderscore prostituta\textunderscore  reles.
(Cast. \textunderscore zurrón\textunderscore )
\section{Surrapa}
\begin{itemize}
\item {Grp. gram.:f.}
\end{itemize}
O mesmo que \textunderscore zurrapa\textunderscore . Cf. Filinto, I, 59.
\section{Surrar}
\begin{itemize}
\item {Grp. gram.:v. t.}
\end{itemize}
Curtir, pisar ou machucar (pelles).
Bater.
Fustigar.
Maltratar com pancadas.
\textunderscore V. p.\textunderscore O mesmo que \textunderscore surubi\textunderscore .
Cotear-se, roçar-se, (uma peça de vestuário).
(Cast. \textunderscore zurrar\textunderscore )
\section{Surrasco}
\begin{itemize}
\item {Grp. gram.:m.}
\end{itemize}
\begin{itemize}
\item {Utilização:Prov.}
\end{itemize}
\begin{itemize}
\item {Utilização:beir.}
\end{itemize}
\begin{itemize}
\item {Proveniência:(De \textunderscore surro\textunderscore )}
\end{itemize}
Nódoa ou mascarra no rosto.
\section{Surrate}
\begin{itemize}
\item {Grp. gram.:m.}
\end{itemize}
O mesmo que \textunderscore surrelfa\textunderscore .
(Cp. \textunderscore surrateiro\textunderscore )
\section{Surratear}
\begin{itemize}
\item {Grp. gram.:v. t.}
\end{itemize}
\begin{itemize}
\item {Proveniência:(Do lat. \textunderscore surreptus\textunderscore )}
\end{itemize}
Furtar; surripiar.
\section{Surrateiramente}
\begin{itemize}
\item {Grp. gram.:adv.}
\end{itemize}
De modo surrateiro; á socapa.
\section{Surrateiro}
\begin{itemize}
\item {Grp. gram.:adj.}
\end{itemize}
\begin{itemize}
\item {Utilização:Zool.}
\end{itemize}
\begin{itemize}
\item {Proveniência:(Do lat. \textunderscore surreptus\textunderscore )}
\end{itemize}
Que procede com manha, pela calada.
Matreiro.
Que olha de soslaio.
Insecto coleóptero pentâmero.
\section{Surrelfa}
\begin{itemize}
\item {Grp. gram.:f.}
\end{itemize}
O mesmo que \textunderscore sorrelfa\textunderscore .
\section{Surreição}
\begin{itemize}
\item {Grp. gram.:f.}
\end{itemize}
\begin{itemize}
\item {Utilização:Pop.}
\end{itemize}
\begin{itemize}
\item {Proveniência:(Do lat. \textunderscore surrectio\textunderscore )}
\end{itemize}
O mesmo que \textunderscore resurreição\textunderscore .
\section{Surreira}
\begin{itemize}
\item {Grp. gram.:f.}
\end{itemize}
\begin{itemize}
\item {Utilização:Prov.}
\end{itemize}
\begin{itemize}
\item {Utilização:minh.}
\end{itemize}
\begin{itemize}
\item {Proveniência:(De \textunderscore surro\textunderscore )}
\end{itemize}
O mesmo que \textunderscore enxurreira\textunderscore .
\section{Surrento}
\begin{itemize}
\item {Grp. gram.:adj.}
\end{itemize}
\begin{itemize}
\item {Utilização:Prov.}
\end{itemize}
\begin{itemize}
\item {Utilização:trasm.}
\end{itemize}
Cheio de surro; immundo.
\section{Súrria!}
\begin{itemize}
\item {Grp. gram.:interj.}
\end{itemize}
\begin{itemize}
\item {Utilização:Pop.}
\end{itemize}
O mesmo que \textunderscore surriada\textunderscore !
(Serve para designar troça, motejo, vaias)
\section{Surriada}
\begin{itemize}
\item {Grp. gram.:f.}
\end{itemize}
\begin{itemize}
\item {Utilização:Pop.}
\end{itemize}
\begin{itemize}
\item {Grp. gram.:Interj.}
\end{itemize}
Descarga de artilharia.
Espuma das ondas que se quebram.
Troça, escárneo.
(designativa de \textunderscore troça\textunderscore  ou \textunderscore caçoada\textunderscore )
\section{Surriba}
\begin{itemize}
\item {Grp. gram.:f.}
\end{itemize}
Acto ou effeito de surribar.
\section{Surribar}
\begin{itemize}
\item {Grp. gram.:v. t.}
\end{itemize}
\begin{itemize}
\item {Proveniência:(De \textunderscore sub...\textunderscore  + \textunderscore riba\textunderscore ?)}
\end{itemize}
Escavar para afofar (a terra).
Fazer escavação em volta de (árvores transplantadas), para que melhor brotem.
\section{Surriola}
\begin{itemize}
\item {Grp. gram.:f.}
\end{itemize}
\begin{itemize}
\item {Utilização:Náut.}
\end{itemize}
Vergôntea, também conhecida por \textunderscore pau-de-surriola\textunderscore , e fixa por uma das extremidades ao costado do navio, na altura das mesas do traquete, em cada um dos bordos, e que se póde prolongar em sentido perpendicular á quilha, para nele ancorarem as varredoiras, \textunderscore ou\textunderscore , quando surto o navio, para nelle se ancorarem as embarcações miudas.
(Cp. \textunderscore serviola\textunderscore )
\section{Surripiar}
\begin{itemize}
\item {Grp. gram.:v. t.}
\end{itemize}
\begin{itemize}
\item {Utilização:Pop.}
\end{itemize}
Furtar.
Subtrahir, tirar ás escondidas.
(Relaciona-se com o lat. \textunderscore surripere\textunderscore )
\section{Surripilhar}
\begin{itemize}
\item {Grp. gram.:v. t.}
\end{itemize}
\begin{itemize}
\item {Utilização:Pop.}
\end{itemize}
O mesmo que \textunderscore surripiar\textunderscore .
\section{Surro}
\begin{itemize}
\item {Grp. gram.:m.}
\end{itemize}
\begin{itemize}
\item {Utilização:Prov.}
\end{itemize}
\begin{itemize}
\item {Utilização:trasm.}
\end{itemize}
Sujidade no rosto, nas mãos ou nos pés, especialmente a que provém do suór.
O mesmo que \textunderscore café\textunderscore .
(Cp. \textunderscore churdo\textunderscore  e \textunderscore surrar\textunderscore )
\section{Surrobeco}
\begin{itemize}
\item {Grp. gram.:m.}
\end{itemize}
\begin{itemize}
\item {Grp. gram.:Adj.}
\end{itemize}
Pano grosseiro e duradoiro, semelhante ao burel, mas um pouco mais largo e fabricado na Covilhan. Cf. \textunderscore Inquér. Industr.\textunderscore , II, p., l. III, 103.
Que tem côr de surrobeco. Cf. \textunderscore Ibidem\textunderscore , 93.
\section{Surrobeque}
\begin{itemize}
\item {Grp. gram.:m.}
\end{itemize}
(V.surrobeco)
\section{Surrolho}
\begin{itemize}
\item {fónica:rô}
\end{itemize}
\begin{itemize}
\item {Grp. gram.:m.}
\end{itemize}
\begin{itemize}
\item {Utilização:Prov.}
\end{itemize}
\begin{itemize}
\item {Utilização:trasm.}
\end{itemize}
Escuridão eventual, como a resultante de um nevoeiro.
Atmosphera abafadiça.
(Cp. \textunderscore sobrolho\textunderscore )
\section{Surtida}
\begin{itemize}
\item {Grp. gram.:f.}
\end{itemize}
\begin{itemize}
\item {Proveniência:(De \textunderscore surtir\textunderscore )}
\end{itemize}
Saída de sitiados contra sitiantes.
Lugar próprio, por onde se sái contra o inimigo; investida.
\section{Surtir}
\begin{itemize}
\item {Grp. gram.:v. t.}
\end{itemize}
\begin{itemize}
\item {Grp. gram.:V. i.}
\end{itemize}
\begin{itemize}
\item {Proveniência:(De \textunderscore surto\textunderscore )}
\end{itemize}
Têr como resultado; originar: \textunderscore aquillo surtiu effeito\textunderscore .
Têr consequência, bôa ou má.
\section{Surto}
\begin{itemize}
\item {Grp. gram.:adj.}
\end{itemize}
\begin{itemize}
\item {Grp. gram.:M.}
\end{itemize}
\begin{itemize}
\item {Utilização:Fig.}
\end{itemize}
Ancorado.
Vôo.
Ambição.
\section{Surtum}
\begin{itemize}
\item {Grp. gram.:m.}
\end{itemize}
\begin{itemize}
\item {Utilização:Ant.}
\end{itemize}
\begin{itemize}
\item {Proveniência:(Do fr. \textunderscore surtout\textunderscore ?)}
\end{itemize}
O mesmo que \textunderscore sertum\textunderscore . Cf. M. Feijó, \textunderscore Orthogr.\textunderscore 
\section{Suru}
\begin{itemize}
\item {Grp. gram.:adj.}
\end{itemize}
\begin{itemize}
\item {Utilização:Bras. do N}
\end{itemize}
Diz se das aves que não tem cauda: \textunderscore gallinha suru\textunderscore .
O mesmo que \textunderscore suro\textunderscore .
\section{Surubi}
\begin{itemize}
\item {Grp. gram.:m.}
\end{itemize}
\begin{itemize}
\item {Utilização:Bras}
\end{itemize}
Gênero de peixes, do norte do Brasil.
\section{Surubim}
\begin{itemize}
\item {Grp. gram.:m.}
\end{itemize}
\begin{itemize}
\item {Utilização:Bras}
\end{itemize}
O mesmo que \textunderscore surubi\textunderscore .
Gênero de peixes, do norte do Brasil.
\section{Surucucu}
\begin{itemize}
\item {Grp. gram.:f.}
\end{itemize}
Espécie de víbora do Brasil.
\section{Surucucutinga}
\begin{itemize}
\item {Grp. gram.:f.}
\end{itemize}
Perigosa cobra do Brasil, (\textunderscore lachesis mutus\textunderscore ), também chamada \textunderscore surucucu-pico-de-jaca\textunderscore .
\section{Surucura}
\begin{itemize}
\item {Grp. gram.:f.}
\end{itemize}
Árvore bignoniácea, (\textunderscore bignonia hirtella\textunderscore ).
\section{Surucutinga}
\begin{itemize}
\item {Grp. gram.:f.}
\end{itemize}
O mesmo que \textunderscore surucucutinga\textunderscore .
\section{Surulina}
\begin{itemize}
\item {Grp. gram.:f.}
\end{itemize}
\begin{itemize}
\item {Utilização:Bras. do N}
\end{itemize}
Espécie de pomba, parecida com a perdiz.
\section{Suruquá}
\begin{itemize}
\item {Grp. gram.:m.}
\end{itemize}
\begin{itemize}
\item {Utilização:Bras}
\end{itemize}
\begin{itemize}
\item {Proveniência:(T. tupi)}
\end{itemize}
Nome de várias aves trepadoras.
\section{Sururina}
\begin{itemize}
\item {Grp. gram.:f.}
\end{itemize}
\begin{itemize}
\item {Utilização:Bras}
\end{itemize}
Ave gallinácea das regiões do Amazonas.
\section{Sururu}
\begin{itemize}
\item {Grp. gram.:m.}
\end{itemize}
\begin{itemize}
\item {Utilização:Bras. do N}
\end{itemize}
\begin{itemize}
\item {Proveniência:(T. tupi)}
\end{itemize}
Espécie de mollusco; mexilhão.
\section{Sururuca}
\begin{itemize}
\item {Grp. gram.:f.}
\end{itemize}
Planta passiflórea do Brasil.
\section{Sururuca}
\begin{itemize}
\item {Grp. gram.:f.}
\end{itemize}
\begin{itemize}
\item {Utilização:Bras}
\end{itemize}
Espécie de peneira grossa.
(Do tupi \textunderscore sururu\textunderscore )
\section{Sururucujá}
\begin{itemize}
\item {Grp. gram.:f.}
\end{itemize}
Planta passiflórea.
\section{Sus!}
\begin{itemize}
\item {Grp. gram.:interj.}
\end{itemize}
\begin{itemize}
\item {Proveniência:(Lat. \textunderscore susum\textunderscore )}
\end{itemize}
Eia!
Coragem!
\section{Susã}
\begin{itemize}
\item {Grp. gram.:adj. f.}
\end{itemize}
(Fem. de \textunderscore susão\textunderscore )
\section{Susalpim}
\begin{itemize}
\item {Grp. gram.:m.}
\end{itemize}
\begin{itemize}
\item {Utilização:Açor}
\end{itemize}
Capote, o mesmo que \textunderscore salpim\textunderscore .
\section{Susan}
\begin{itemize}
\item {Grp. gram.:adj. f.}
\end{itemize}
(Fem. de \textunderscore susão\textunderscore )
\section{Susana}
\begin{itemize}
\item {Grp. gram.:f.}
\end{itemize}
\begin{itemize}
\item {Utilização:Mús.}
\end{itemize}
\begin{itemize}
\item {Utilização:Des.}
\end{itemize}
Variação musical. Cf. E. Vieira, \textunderscore Diccion. Mús.\textunderscore 
\section{Susano}
\begin{itemize}
\item {Grp. gram.:adj.}
\end{itemize}
\begin{itemize}
\item {Proveniência:(De \textunderscore suso\textunderscore )}
\end{itemize}
Que está acima.
Dizia-se especialmente de algumas localidades divididas em duas partes, que se distinguiam pelo sua posição: \textunderscore Caria susan\textunderscore  e \textunderscore Caria jusan\textunderscore ; \textunderscore villa susan e villa jusan\textunderscore , etc.
\section{Susão}
\begin{itemize}
\item {Grp. gram.:adj.}
\end{itemize}
\begin{itemize}
\item {Utilização:Ant.}
\end{itemize}
\begin{itemize}
\item {Proveniência:(De \textunderscore suso\textunderscore )}
\end{itemize}
Que está acima.
Dizia-se especialmente de algumas localidades divididas em duas partes, que se distinguiam pelo sua posição: \textunderscore Caria susan\textunderscore  e \textunderscore Caria jusan\textunderscore ; \textunderscore villa susan e villa jusan\textunderscore , etc.
\section{Susceptibilidade}
\begin{itemize}
\item {Grp. gram.:f.}
\end{itemize}
\begin{itemize}
\item {Proveniência:(Do lat. \textunderscore susceptibilis\textunderscore )}
\end{itemize}
Qualidade do que é susceptível.
Disposição para sentir influências ou contrahir enfermidades.
Disposição para se melindrar facilmente; melindre.--Nesta última accepção, o t. é considerado gallicismo inútil.
\section{Susceptibilizar}
\begin{itemize}
\item {Grp. gram.:v. t.}
\end{itemize}
\begin{itemize}
\item {Grp. gram.:V. p.}
\end{itemize}
\begin{itemize}
\item {Proveniência:(Do lat. \textunderscore susceptibilis\textunderscore )}
\end{itemize}
Melindrar ou offender ligeiramente.
Melindrar-se; considerar-se offendido.
\section{Susceptível}
\begin{itemize}
\item {Grp. gram.:adj.}
\end{itemize}
\begin{itemize}
\item {Grp. gram.:M.}
\end{itemize}
\begin{itemize}
\item {Proveniência:(Do lat. \textunderscore susceptibilis\textunderscore )}
\end{itemize}
Que póde receber certas modificações, impressões ou qualidades.
Capaz.
Melindroso; que se offende facilmente.
Indivíduo susceptível ou melindroso.
\section{Susceptor}
\begin{itemize}
\item {Grp. gram.:m.}
\end{itemize}
\begin{itemize}
\item {Proveniência:(Lat. \textunderscore susceptor\textunderscore )}
\end{itemize}
Exactor, arcário, questor municipal no tempo do Império Romano. Cf. Herculano, \textunderscore Hist. de Port.\textunderscore , IV, 27.
\section{Suscitação}
\begin{itemize}
\item {Grp. gram.:f.}
\end{itemize}
\begin{itemize}
\item {Proveniência:(Do lat. \textunderscore suscitatio\textunderscore )}
\end{itemize}
Acto ou effeito de suscitar; suggestão.
\section{Suscitador}
\begin{itemize}
\item {Grp. gram.:m.  e  adj.}
\end{itemize}
\begin{itemize}
\item {Proveniência:(Do lat. \textunderscore suscitator\textunderscore )}
\end{itemize}
O que suscita.
\section{Suscitamento}
\begin{itemize}
\item {Grp. gram.:m.}
\end{itemize}
O mesmo que \textunderscore suscitação\textunderscore .
\section{Suscitar}
\begin{itemize}
\item {Grp. gram.:v. t.}
\end{itemize}
\begin{itemize}
\item {Proveniência:(Lat. \textunderscore suscitare\textunderscore )}
\end{itemize}
Dar origem a.
Fazer apparecer.
Promover: \textunderscore suscitar difficuldades\textunderscore .
Oppor.
Suggerir, lembrar: \textunderscore suscitar um estratagema\textunderscore .
\section{Suserania}
\begin{itemize}
\item {Grp. gram.:f.}
\end{itemize}
Qualidade do que é suserano.
Território, em que o Soberano domina.
\section{Suserano}
\begin{itemize}
\item {Grp. gram.:adj.}
\end{itemize}
\begin{itemize}
\item {Grp. gram.:M.}
\end{itemize}
\begin{itemize}
\item {Proveniência:(Do lat. \textunderscore susum\textunderscore , sob a infl. de \textunderscore soberano\textunderscore )}
\end{itemize}
Que tem um feudo, de que dependem outros feudos.
Relativo aos Soberanos, a quem outros Estados, apparentemente autónomos, rendem vassallagem.
Senhor feudal.
\section{Susfito}
\begin{itemize}
\item {Grp. gram.:m.}
\end{itemize}
Planta brasileira. Cf. \textunderscore Jorn.-do-Comm.\textunderscore , do Rio, de 14-VI-901.
\section{Susino}
\begin{itemize}
\item {Grp. gram.:adj.}
\end{itemize}
\begin{itemize}
\item {Grp. gram.:M.}
\end{itemize}
\begin{itemize}
\item {Proveniência:(Gr. \textunderscore sousinos\textunderscore )}
\end{itemize}
Relativo a lírio ou extrahido do lírio, (falado-se de certo óleo aromático).
Essência aromática do lírio.
\section{Suso}
\begin{itemize}
\item {Grp. gram.:prep.}
\end{itemize}
\begin{itemize}
\item {Utilização:Des.}
\end{itemize}
\begin{itemize}
\item {Proveniência:(Lat. \textunderscore susum\textunderscore )}
\end{itemize}
Acima; atrás; anteriormente.
\section{Susodito}
\begin{itemize}
\item {Grp. gram.:adj.}
\end{itemize}
\begin{itemize}
\item {Utilização:Ant.}
\end{itemize}
O mesmo que \textunderscore sobredito\textunderscore .
\section{Suspeição}
\begin{itemize}
\item {Grp. gram.:f.}
\end{itemize}
\begin{itemize}
\item {Proveniência:(Do lat. \textunderscore suspectio\textunderscore )}
\end{itemize}
O mesmo que \textunderscore suspeita\textunderscore ; desconfiança.
\section{Suspeita}
\begin{itemize}
\item {Grp. gram.:f.}
\end{itemize}
\begin{itemize}
\item {Proveniência:(Lat. \textunderscore suspecta\textunderscore )}
\end{itemize}
Opinião, mais ou menos desfavorável, a respeito de alguém ou de alguma coisa.
Desconfiança; supposição.
\section{Suspeitador}
\begin{itemize}
\item {Grp. gram.:m.  e  adj.}
\end{itemize}
O que suspeita.
\section{Suspeitar}
\begin{itemize}
\item {Grp. gram.:v. t.}
\end{itemize}
\begin{itemize}
\item {Grp. gram.:V. i.}
\end{itemize}
\begin{itemize}
\item {Proveniência:(Do lat. \textunderscore suspectare\textunderscore )}
\end{itemize}
Têr suspeita de.
Conjecturar; suppor, com mais ou menos probabilidade.
Têr desconfiança.
Fazer supposição.
\section{Suspeito}
\begin{itemize}
\item {Grp. gram.:adj.}
\end{itemize}
\begin{itemize}
\item {Proveniência:(Lat. \textunderscore suspectus\textunderscore )}
\end{itemize}
Que infunde suspeitas.
Duvidoso.
Que dá cuidados.
De cuja existência ou de cuja verdade se duvída.
Que parece têr defeito.
Que não inspira confiança: \textunderscore um vizinho suspeito\textunderscore .
\section{Suspeitosamente}
\begin{itemize}
\item {Grp. gram.:adv.}
\end{itemize}
De modo suspeitoso.
Com suspeita.
\section{Suspeitoso}
\begin{itemize}
\item {Grp. gram.:adj.}
\end{itemize}
O mesmo que \textunderscore suspeito\textunderscore .
Que tem suspeitas ou receios.
\section{Suspender}
\begin{itemize}
\item {Grp. gram.:v. t.}
\end{itemize}
\begin{itemize}
\item {Grp. gram.:V. p.}
\end{itemize}
\begin{itemize}
\item {Proveniência:(Lat. \textunderscore suspendere\textunderscore )}
\end{itemize}
Suster no ar, pendurar.
Deixar pender.
Sustar, interromper.
Fazer parar.
Procrastinar; adiar.
Impedir.
Privar de um cargo ou dos respectivos vencimentos, provisoriamente: \textunderscore suspender um amanuense\textunderscore .
Impedir por algum tempo a publicação de: \textunderscore suspender um jornal\textunderscore .
Equilibrar-se no ar.
Pendurar-se: \textunderscore Judas suspendeu-se de uma figueira\textunderscore .
Parar.
Estar em lugar muito alto.
Ficar perplexo.
Enlevar-se.
\section{Sibarismo}
\begin{itemize}
\item {Grp. gram.:m.}
\end{itemize}
Carácter dos Sibaritas.
Desejo imoderado de luxo e prazeres.
O mesmo que \textunderscore sibaritismo\textunderscore .
(Cp. \textunderscore sibarita\textunderscore )
\section{Sibarita}
\begin{itemize}
\item {Grp. gram.:m. ,  f.  e  adj.}
\end{itemize}
\begin{itemize}
\item {Utilização:Fig.}
\end{itemize}
\begin{itemize}
\item {Proveniência:(Lat. \textunderscore sybarita\textunderscore )}
\end{itemize}
Diz-se da pessôa que vive na voluptuosidade ou que é efeminada.
\section{Sibarítico}
\begin{itemize}
\item {Grp. gram.:adj.}
\end{itemize}
Relativo a sibarita ou próprio de sibarita.
\section{Sibaritismo}
\begin{itemize}
\item {Grp. gram.:m.}
\end{itemize}
Vida de sibarita.
Voluptuosidade excessiva.
\section{Sicefalia}
\begin{itemize}
\item {Grp. gram.:f.}
\end{itemize}
Estado de sicéfalo.
\section{Sicéfalo}
\begin{itemize}
\item {Grp. gram.:adj.}
\end{itemize}
\begin{itemize}
\item {Proveniência:(Do gr. \textunderscore sun\textunderscore  + \textunderscore kephale\textunderscore )}
\end{itemize}
Diz-se do monstro, que tem duas cabeças reunidas.
\section{Sicite}
\begin{itemize}
\item {Grp. gram.:f.}
\end{itemize}
\begin{itemize}
\item {Proveniência:(Lat. \textunderscore sícites\textunderscore )}
\end{itemize}
Vinho de figos, usado entre os antigos.
\section{Sicite}
\begin{itemize}
\item {Grp. gram.:f.}
\end{itemize}
\begin{itemize}
\item {Proveniência:(Lat. \textunderscore sycitis\textunderscore )}
\end{itemize}
Pedra preciosa, da côr do figo.
\section{Sicomancia}
\begin{itemize}
\item {Grp. gram.:f.}
\end{itemize}
\begin{itemize}
\item {Proveniência:(Do gr. \textunderscore sucon\textunderscore  + \textunderscore manteia\textunderscore )}
\end{itemize}
Antigo sistema de adivinhação por meio de fôlhas de figueira, nas quaes se escreviam as preguntas, de que se desejava têr a resposta.
\section{Sicomântico}
\begin{itemize}
\item {Grp. gram.:adj.}
\end{itemize}
Relativo a sicomancia.
\section{Sicômoro}
\begin{itemize}
\item {Grp. gram.:m.}
\end{itemize}
\begin{itemize}
\item {Proveniência:(Lat. \textunderscore sicumorus\textunderscore )}
\end{itemize}
Espécie de figueira das margens do Mediterrâneo.
Nome de outras árvores.
\section{Sícone}
\begin{itemize}
\item {Grp. gram.:m.}
\end{itemize}
O mesmo que \textunderscore sícono\textunderscore .
\section{Sícono}
\begin{itemize}
\item {Grp. gram.:m.}
\end{itemize}
\begin{itemize}
\item {Utilização:Bot.}
\end{itemize}
\begin{itemize}
\item {Proveniência:(Do gr. \textunderscore sukon\textunderscore , figo)}
\end{itemize}
Inflorescência em que o receptáculo envolve as flôres, como sucede no figo.
\section{Sincéfalo}
\begin{itemize}
\item {Grp. gram.:adj.}
\end{itemize}
\begin{itemize}
\item {Grp. gram.:m.}
\end{itemize}
\begin{itemize}
\item {Proveniência:(Do gr. \textunderscore sun\textunderscore  + \textunderscore kephale\textunderscore )}
\end{itemize}
Diz-se do monstro, que tem duas cabeças reunidas.
Gênero de plantas, da fam. das compostas.
\section{Suspêndio}
\begin{itemize}
\item {Grp. gram.:m.}
\end{itemize}
\begin{itemize}
\item {Utilização:Ant.}
\end{itemize}
\begin{itemize}
\item {Proveniência:(Lat. \textunderscore suspendium\textunderscore )}
\end{itemize}
O mesmo que \textunderscore fôrca\textunderscore .
\section{Suspensão}
\begin{itemize}
\item {Grp. gram.:f.}
\end{itemize}
\begin{itemize}
\item {Utilização:Gram.}
\end{itemize}
\begin{itemize}
\item {Proveniência:(Do lat. \textunderscore suspensio\textunderscore )}
\end{itemize}
Acto ou effeito de suspender.
Estado das substâncias sólidas, que fluctuam num líquido.
Prolongamento de uma nota ou pausa, na música.
Vaso de flôres ou plantas, pendurado do tecto ou da vêrga de uma janela ou porta, como ornato.
Espécie de miragem, em que os objectos parecem suspensos, sem reflectirem a imagem.
Interrupção ou suspensão do sentido.
\section{Suspensivo}
\begin{itemize}
\item {Grp. gram.:adj.}
\end{itemize}
\begin{itemize}
\item {Utilização:Gram.}
\end{itemize}
\begin{itemize}
\item {Proveniência:(De \textunderscore suspenso\textunderscore )}
\end{itemize}
Que póde suspender.
Que suspende o sentido de uma proposição.
\section{Suspenso}
\begin{itemize}
\item {Grp. gram.:adj.}
\end{itemize}
\begin{itemize}
\item {Utilização:Gram.}
\end{itemize}
\begin{itemize}
\item {Utilização:Heráld.}
\end{itemize}
\begin{itemize}
\item {Proveniência:(Lat. \textunderscore suspensus\textunderscore )}
\end{itemize}
Pendente; pendurado.
Interrompido.
Perplexo; hesitante.
Parado, sustado.
Que faz sentido incompleto.
Diz-se da cruz, cujas hastes não attingem os extremos do campo do escudo.
\section{Suspensor}
\begin{itemize}
\item {Grp. gram.:m.  e  adj.}
\end{itemize}
\begin{itemize}
\item {Proveniência:(De \textunderscore suspenso\textunderscore )}
\end{itemize}
Diz-se o objecto ou utensílio que serve para suspender outro.
\section{Suspensório}
\begin{itemize}
\item {Grp. gram.:adj.}
\end{itemize}
\begin{itemize}
\item {Grp. gram.:M.}
\end{itemize}
\begin{itemize}
\item {Grp. gram.:Pl.}
\end{itemize}
\begin{itemize}
\item {Proveniência:(De \textunderscore suspenso\textunderscore )}
\end{itemize}
Próprio para fazer suspender.
Que suspende.
Ligadura, com que se sustêm o escroto.
Fitas, que, passando por cima, dos ombros, seguram as calças pelo cós; alças.
\section{Suspeso}
\begin{itemize}
\item {fónica:pê}
\end{itemize}
\begin{itemize}
\item {Grp. gram.:adj.}
\end{itemize}
\begin{itemize}
\item {Utilização:Ant.}
\end{itemize}
Suspenso. Cf. Usque, 39.
\section{Suspicácia}
\begin{itemize}
\item {Grp. gram.:f.}
\end{itemize}
Qualidade de suspicaz. Cf. Latino, \textunderscore Hist. Pol.\textunderscore , I, 34.
\section{Suspicaz}
\begin{itemize}
\item {Grp. gram.:adj.}
\end{itemize}
\begin{itemize}
\item {Proveniência:(Lat. \textunderscore suspicax\textunderscore )}
\end{itemize}
Suspeito.
Que tem suspeita, desconfiado.
\section{Suspiráculo}
\begin{itemize}
\item {Grp. gram.:m.}
\end{itemize}
\begin{itemize}
\item {Proveniência:(De \textunderscore suspirar\textunderscore )}
\end{itemize}
Lugar, onde mal se respira.
\section{Suspirado}
\begin{itemize}
\item {Grp. gram.:adj.}
\end{itemize}
\begin{itemize}
\item {Proveniência:(De \textunderscore suspirar\textunderscore )}
\end{itemize}
Acompanhado de suspiros.
Muito desejado.
\section{Suspirador}
\begin{itemize}
\item {Grp. gram.:m.  e  adj.}
\end{itemize}
O que suspira.
\section{Suspirar}
\begin{itemize}
\item {Grp. gram.:v. t.}
\end{itemize}
\begin{itemize}
\item {Grp. gram.:V. i.}
\end{itemize}
\begin{itemize}
\item {Utilização:Poét.}
\end{itemize}
\begin{itemize}
\item {Grp. gram.:M.}
\end{itemize}
\begin{itemize}
\item {Utilização:Poét.}
\end{itemize}
\begin{itemize}
\item {Proveniência:(Lat. \textunderscore suspirare\textunderscore )}
\end{itemize}
Significar por meio de suspiros: \textunderscore suspirar cuidados\textunderscore .
Exprimir com tristeza.
Desejar vehementemente.
Têr saudades de.
Dar suspiros.
Soprar ligeiramente; rumorejar; murmurar: \textunderscore suspira a brisa nos bosques\textunderscore .
Murmúrio.
\section{Suspiro}
\begin{itemize}
\item {Grp. gram.:m.}
\end{itemize}
\begin{itemize}
\item {Utilização:Fig.}
\end{itemize}
\begin{itemize}
\item {Utilização:Prov.}
\end{itemize}
\begin{itemize}
\item {Utilização:minh.}
\end{itemize}
\begin{itemize}
\item {Proveniência:(Do lat. \textunderscore suspirium\textunderscore )}
\end{itemize}
Respiração, mais ou menos prolongada, e produzida por desgôsto ou incômmodo phýsico.
Gemido.
Ânsia.
Som triste e suave.
Murmúrio.
Pequeno orifício, para se extrahir um líquido em pequena quantidade.
Espécie de bolo tenro.
Nome de uma planta, também conhecida por \textunderscore perpétua\textunderscore  e \textunderscore saudade\textunderscore .
Abertura ou poço, por onde se dá luz e ar a uma mina ou túnel.
Respiradoiro.
\section{Suspiroso}
\begin{itemize}
\item {Grp. gram.:adj.}
\end{itemize}
\begin{itemize}
\item {Proveniência:(Do lat. \textunderscore suspiriosus\textunderscore )}
\end{itemize}
Que suspira.
Relativo a suspiro.
Lamentoso; suspirado.
\section{Susque-dono!}
\begin{itemize}
\item {Grp. gram.:interj.}
\end{itemize}
\begin{itemize}
\item {Utilização:Prov.}
\end{itemize}
\begin{itemize}
\item {Utilização:trasm.}
\end{itemize}
\begin{itemize}
\item {Proveniência:(De \textunderscore susquir\textunderscore  + \textunderscore com\textunderscore  + \textunderscore o\textunderscore  + \textunderscore dono\textunderscore )}
\end{itemize}
Ala! rua! gira!
\section{Susquir-se}
\begin{itemize}
\item {Grp. gram.:V. p.}
\end{itemize}
\begin{itemize}
\item {Utilização:Prov.}
\end{itemize}
\begin{itemize}
\item {Utilização:trasm.}
\end{itemize}
Safar-se; retirar-se á pressa.
\section{Sussarra}
\begin{itemize}
\item {Grp. gram.:f.}
\end{itemize}
\begin{itemize}
\item {Utilização:Prov.}
\end{itemize}
\begin{itemize}
\item {Utilização:alg.}
\end{itemize}
O mesmo que \textunderscore molleja\textunderscore .
\section{Sussuarana}
\begin{itemize}
\item {Grp. gram.:f.}
\end{itemize}
\begin{itemize}
\item {Utilização:Bras}
\end{itemize}
Mammífero carnívoro, espécie de onça.
(Do tupi \textunderscore suassu-rana\textunderscore )
\section{Sussudoéste}
\begin{itemize}
\item {Grp. gram.:m.}
\end{itemize}
\begin{itemize}
\item {Proveniência:(De \textunderscore sul\textunderscore  + \textunderscore Sudoéste\textunderscore )}
\end{itemize}
Ponto do horizonte, a igual distância do Sul e do Sudoéste.
Vento, que sopra dêsse lado.
\section{Sussuéste}
\begin{itemize}
\item {Grp. gram.:m.}
\end{itemize}
\begin{itemize}
\item {Proveniência:(De \textunderscore Sul\textunderscore  + \textunderscore Suéste\textunderscore )}
\end{itemize}
Ponto do horizonte, a igual distância do Sul e do Suéste.
Vento, que sopra dêsse lado.
\section{Sussurração}
\begin{itemize}
\item {Grp. gram.:f.}
\end{itemize}
O mesmo que \textunderscore sussurro\textunderscore . Cf. Eça, \textunderscore Mandarim\textunderscore , 127.
\section{Sussurrante}
\begin{itemize}
\item {Grp. gram.:adj.}
\end{itemize}
\begin{itemize}
\item {Proveniência:(Lat. \textunderscore susurrans\textunderscore )}
\end{itemize}
Que sussurra.
Que sôa vagamente.
Que murmura ou rumoreja.
\section{Sussurrar}
\begin{itemize}
\item {Grp. gram.:v. i.}
\end{itemize}
\begin{itemize}
\item {Grp. gram.:V. t.}
\end{itemize}
\begin{itemize}
\item {Proveniência:(Lat. \textunderscore sussurrare\textunderscore )}
\end{itemize}
Fazer sussurro.
Murmurar.
Zumbir.
Dizer em voz baixa, segredar.
\section{Sussurro}
\begin{itemize}
\item {Grp. gram.:m.}
\end{itemize}
\begin{itemize}
\item {Proveniência:(Lat. \textunderscore susurrus\textunderscore )}
\end{itemize}
Som confuso; murmúrio.
Acto de falar em voz baixa.
Zumbido de alguns insectos.
\section{Sustância}
\begin{itemize}
\item {Grp. gram.:f.}
\end{itemize}
O mesmo que \textunderscore substância\textunderscore : \textunderscore um calda de sustância\textunderscore .
(B. lat. \textunderscore sustantia\textunderscore )
\section{Sustar}
\begin{itemize}
\item {Grp. gram.:v. t.}
\end{itemize}
\begin{itemize}
\item {Grp. gram.:V. i.}
\end{itemize}
\begin{itemize}
\item {Proveniência:(Lat. \textunderscore substare\textunderscore )}
\end{itemize}
Fazer parar.
Interromper: \textunderscore sustar a marcha\textunderscore .
Parar.
Suspender-se, interromper-se.
Sobrestar.
\section{Sustatório}
\begin{itemize}
\item {Grp. gram.:adj.}
\end{itemize}
Que serve para sustar.
Que faz sobrestar.
\section{Sustedor}
\begin{itemize}
\item {Grp. gram.:adj.}
\end{itemize}
\begin{itemize}
\item {Proveniência:(De \textunderscore suster\textunderscore )}
\end{itemize}
Que sustém; que apoia:«\textunderscore ...nestes bordões sustedores.\textunderscore »Castilho, \textunderscore Primavera\textunderscore , 184.
\section{Sustenido}
\begin{itemize}
\item {Grp. gram.:m.}
\end{itemize}
\begin{itemize}
\item {Utilização:Prov.}
\end{itemize}
\begin{itemize}
\item {Utilização:trasm.}
\end{itemize}
\begin{itemize}
\item {Proveniência:(Do lat. \textunderscore sustinere\textunderscore )}
\end{itemize}
Sinal musical, que indica que a nota, collocada á direita dêlle, deve subir meio tom.
O mesmo que \textunderscore bofetada\textunderscore .
\section{Sustenizar}
\begin{itemize}
\item {Grp. gram.:v.}
\end{itemize}
\begin{itemize}
\item {Utilização:t. Mús.}
\end{itemize}
Marcar com sustenido ou sustenidos.
\section{Sustentação}
\begin{itemize}
\item {Grp. gram.:f.}
\end{itemize}
\begin{itemize}
\item {Proveniência:(Do lat. \textunderscore sustentatio\textunderscore )}
\end{itemize}
Acto ou effeito de sustentar.
Alimento; conservação; sustentáculo.
\section{Sustentáculo}
\begin{itemize}
\item {Grp. gram.:m.}
\end{itemize}
\begin{itemize}
\item {Proveniência:(Lat. \textunderscore sustentaculum\textunderscore )}
\end{itemize}
Aquillo que sustenta ou sustém.
Base; supporte.
Amparo, apoio.
\section{Sustentador}
\begin{itemize}
\item {Grp. gram.:m.  e  adj.}
\end{itemize}
O que sustenta.
\section{Sustentamento}
\begin{itemize}
\item {Grp. gram.:m.}
\end{itemize}
O mesmo que \textunderscore sustentação\textunderscore .
\section{Sustentante}
\begin{itemize}
\item {Grp. gram.:adj.}
\end{itemize}
\begin{itemize}
\item {Proveniência:(Lat. \textunderscore sustentans\textunderscore )}
\end{itemize}
Que sustenta.
\section{Sustentar}
\begin{itemize}
\item {Grp. gram.:v. t.}
\end{itemize}
\begin{itemize}
\item {Proveniência:(Lat. \textunderscore sustentare\textunderscore )}
\end{itemize}
Segurar por baixo; suster, supportar: \textunderscore aquella peanha sustenta uma estátua\textunderscore .
Amparar; auxiliar.
Impedir que alguma coisa caia.
Conservar; manter: \textunderscore sustentar o seu crédito\textunderscore .
Alimentar, phýsica ou moralmente: \textunderscore o pai sustenta os filhos\textunderscore .
Estimular.
Fortificar.
Perpetuar.
Oppor-se a.
Defender.
Pelejar a favor de.
Defender com argumentos.
Affirmar categoricamente: \textunderscore sustentar a immortalidade da alma\textunderscore .
\section{Sustentável}
\begin{itemize}
\item {Grp. gram.:adj.}
\end{itemize}
Que se póde sustentar.
\section{Sustento}
\begin{itemize}
\item {Grp. gram.:m.}
\end{itemize}
Alimento.
Aquillo que serve de alimentação.
Acto ou effeito de sustentar.
\section{Suster}
\begin{itemize}
\item {Grp. gram.:v. t.}
\end{itemize}
\begin{itemize}
\item {Grp. gram.:V. p.}
\end{itemize}
\begin{itemize}
\item {Proveniência:(Lat. \textunderscore sustinere\textunderscore )}
\end{itemize}
Segurar para que não caia.
Sustentar, alimentar.
Refrear, reprimir: \textunderscore suster a cólera\textunderscore .
Parar.
Sobrestar.
Moderar-se.
Firmar-se.
Equilibrar-se.
Manter-se.
\section{Sustimento}
\begin{itemize}
\item {Grp. gram.:m.}
\end{itemize}
Acto ou effeito de suster.
\section{Sustinência}
\begin{itemize}
\item {Grp. gram.:f.}
\end{itemize}
\begin{itemize}
\item {Proveniência:(Lat. \textunderscore sustinentia\textunderscore )}
\end{itemize}
O mesmo que \textunderscore sustimento\textunderscore .
\section{Sustinente}
\begin{itemize}
\item {Grp. gram.:adj.}
\end{itemize}
\begin{itemize}
\item {Proveniência:(Lat. \textunderscore sustinens\textunderscore )}
\end{itemize}
Que sustém.
\section{Sustinentes}
\begin{itemize}
\item {Grp. gram.:m. pl.}
\end{itemize}
\begin{itemize}
\item {Utilização:Ant.}
\end{itemize}
Peças, de prata ás vezes, na guarnição do arreio de cavallo de brida.
(Pl. de \textunderscore sustinente\textunderscore )
\section{Susto}
\begin{itemize}
\item {Grp. gram.:m.}
\end{itemize}
\begin{itemize}
\item {Utilização:Ext.}
\end{itemize}
\begin{itemize}
\item {Utilização:Gír.}
\end{itemize}
\begin{itemize}
\item {Proveniência:(De \textunderscore sustar\textunderscore )}
\end{itemize}
Mêdo repentino.
Sobresalto.
Temor, causado por notícia ou factos imprevistos.
Mêdo, receio.
Pão.
\section{Susudoéste}
\begin{itemize}
\item {fónica:sussu}
\end{itemize}
\begin{itemize}
\item {Grp. gram.:m.}
\end{itemize}
\begin{itemize}
\item {Proveniência:(De \textunderscore sul\textunderscore  + \textunderscore Sudoéste\textunderscore )}
\end{itemize}
Ponto do horizonte, a igual distância do Sul e do Sudoéste.
Vento, que sopra dêsse lado.
\section{Susuéste}
\begin{itemize}
\item {fónica:sussu}
\end{itemize}
\begin{itemize}
\item {Grp. gram.:m.}
\end{itemize}
\begin{itemize}
\item {Proveniência:(De \textunderscore Sul\textunderscore  + \textunderscore Suéste\textunderscore )}
\end{itemize}
Ponto do horizonte, a igual distância do Sul e do Suéste.
Vento, que sopra dêsse lado.
\section{Susurrar}
\begin{itemize}
\item {fónica:sussu}
\end{itemize}
\begin{itemize}
\item {Grp. gram.:v. i.}
\end{itemize}
(V.sussurrar)
\section{Suta}
\begin{itemize}
\item {Grp. gram.:f.}
\end{itemize}
\begin{itemize}
\item {Proveniência:(De \textunderscore sutar\textunderscore )}
\end{itemize}
Instrumento, com que se marcam ângulos num terreno.
Espécie de esquadro, de peças móveis, para traçar ângulos.
\section{Sutache}
\begin{itemize}
\item {Grp. gram.:m.  e  f.}
\end{itemize}
\begin{itemize}
\item {Proveniência:(Fr. \textunderscore soutache\textunderscore )}
\end{itemize}
Trancinha de sêda, lan ou algodão, com que se enfeitam peças de vestuário.
\section{Sutage}
\begin{itemize}
\item {Grp. gram.:f.}
\end{itemize}
(Alter. de \textunderscore sutache\textunderscore )
\section{Sutagem}
\begin{itemize}
\item {Grp. gram.:f.}
\end{itemize}
(Alter. de \textunderscore sutache\textunderscore )
\section{Sutar}
\begin{itemize}
\item {Grp. gram.:v. t.}
\end{itemize}
Ajustar (uma peça) noutra, servindo-se da suta.
(Talvez contr. de \textunderscore suturar\textunderscore , de \textunderscore sutura\textunderscore )
\section{Sutate}
\begin{itemize}
\item {Grp. gram.:m.}
\end{itemize}
Bebida chinesa, preparada com uma espécie de feijões, anil, casca de laranja, etc. Cf. \textunderscore Ásia Sínica\textunderscore , 59.
\section{Sutera}
\begin{itemize}
\item {Grp. gram.:f.}
\end{itemize}
\begin{itemize}
\item {Proveniência:(De \textunderscore Suter\textunderscore , n. p.)}
\end{itemize}
Gênero de plantas escrofularíneas.
\section{Sutéria}
\begin{itemize}
\item {Grp. gram.:f.}
\end{itemize}
Gênero de plantas leguminosas.
Gênero de plantas rubiáceas.
\section{Sutherlândia}
\begin{itemize}
\item {Grp. gram.:f.}
\end{itemize}
\begin{itemize}
\item {Proveniência:(De \textunderscore Sutherland\textunderscore , n. p.)}
\end{itemize}
Gênero de plantas leguminosas.
\section{Sutil}
\textunderscore adj.\textunderscore  (e der.)
O mesmo que \textunderscore subtil\textunderscore , etc. Cf. Castilho, \textunderscore Tartufo\textunderscore , 10; \textunderscore Eufrosina\textunderscore , 188.
\section{Sútil}
\begin{itemize}
\item {Grp. gram.:adj.}
\end{itemize}
\begin{itemize}
\item {Proveniência:(Lat. \textunderscore sutilis\textunderscore )}
\end{itemize}
Dizia-se da cabana, feita de coiros, em que habitavam os Scythas.
\section{Sutilicairo}
\begin{itemize}
\item {Grp. gram.:m.}
\end{itemize}
O mesmo que \textunderscore sotilicário\textunderscore .
\section{Sutra}
\begin{itemize}
\item {Grp. gram.:f.}
\end{itemize}
\begin{itemize}
\item {Utilização:Prov.}
\end{itemize}
Instrumento de pedreiro, formado de duas hastes de madeira, á semelhança de compasso, e formando esquadria.
(Cp. \textunderscore suta\textunderscore )
\section{Sutura}
\begin{itemize}
\item {Grp. gram.:f.}
\end{itemize}
\begin{itemize}
\item {Utilização:Anat.}
\end{itemize}
\begin{itemize}
\item {Utilização:Bot.}
\end{itemize}
\begin{itemize}
\item {Proveniência:(Lat. \textunderscore sutura\textunderscore )}
\end{itemize}
Operação, que consiste em coser os lábios de uma ferida, para os juntar.
Juntura.
Costura.
União ou articulação de dois ossos, que engranzam um no outro por meio de recorte denteado.
Linhas, pouco salientes, que indicam os pontos onde se há de dar ruptura do fruto ou do invólucro.
\section{Sutural}
\begin{itemize}
\item {Grp. gram.:adj.}
\end{itemize}
Relativo á sutura.
\section{Suturar}
\begin{itemize}
\item {Grp. gram.:v. t.}
\end{itemize}
Fazer a sutura de.
\section{Suxar}
\begin{itemize}
\item {Grp. gram.:v. t.}
\end{itemize}
Tornar froixo; alargar; soltar.
\section{Suxo}
\begin{itemize}
\item {Grp. gram.:adj.}
\end{itemize}
\begin{itemize}
\item {Proveniência:(De \textunderscore suxar\textunderscore )}
\end{itemize}
Que se suxou; froixo, bambo, lasso.
\section{Svitrâmia}
\begin{itemize}
\item {Grp. gram.:f.}
\end{itemize}
\begin{itemize}
\item {Proveniência:(De \textunderscore Svitram\textunderscore , n. p.)}
\end{itemize}
Gênero de plantas melastomáceas.
\section{Swártzia}
\begin{itemize}
\item {fónica:su}
\end{itemize}
\begin{itemize}
\item {Grp. gram.:f.}
\end{itemize}
\begin{itemize}
\item {Proveniência:(De \textunderscore Swartz\textunderscore , n. p.)}
\end{itemize}
Gênero de plantas leguminosas.
\section{Swietênia}
\begin{itemize}
\item {fónica:su}
\end{itemize}
\begin{itemize}
\item {Grp. gram.:f.}
\end{itemize}
\begin{itemize}
\item {Proveniência:(De \textunderscore Swieten\textunderscore , n. p.)}
\end{itemize}
Gênero de plantas meliáceas.
\section{Sybarismo}
\begin{itemize}
\item {Grp. gram.:m.}
\end{itemize}
Carácter dos Sybaritas.
Desejo immoderado de luxo e prazeres.
O mesmo que \textunderscore sybaritismo\textunderscore .
(Cp. \textunderscore sybarita\textunderscore )
\section{Sybarita}
\begin{itemize}
\item {Grp. gram.:m. ,  f.  e  adj.}
\end{itemize}
\begin{itemize}
\item {Utilização:Fig.}
\end{itemize}
\begin{itemize}
\item {Proveniência:(Lat. \textunderscore sybarita\textunderscore )}
\end{itemize}
Diz-se da pessôa que vive na voluptuosidade ou que é effeminada.
\section{Sybarítico}
\begin{itemize}
\item {Grp. gram.:adj.}
\end{itemize}
Relativo a sybarita ou próprio de sybarita.
\section{Sybaritismo}
\begin{itemize}
\item {Grp. gram.:m.}
\end{itemize}
Vida de sybarita.
Voluptuosidade excessiva.
\section{Sycephalia}
\begin{itemize}
\item {Grp. gram.:f.}
\end{itemize}
Estado de sycéphalo.
\section{Sycéphalo}
\begin{itemize}
\item {Grp. gram.:adj.}
\end{itemize}
\begin{itemize}
\item {Proveniência:(Do gr. \textunderscore sun\textunderscore  + \textunderscore kephale\textunderscore )}
\end{itemize}
Diz-se do monstro, que tem duas cabeças reunidas.
\section{Sycite}
\begin{itemize}
\item {Grp. gram.:f.}
\end{itemize}
\begin{itemize}
\item {Proveniência:(Lat. \textunderscore sícites\textunderscore )}
\end{itemize}
Vinho de figos, usado entre os antigos.
\section{Sycite}
\begin{itemize}
\item {Grp. gram.:f.}
\end{itemize}
\begin{itemize}
\item {Proveniência:(Lat. \textunderscore sycitis\textunderscore )}
\end{itemize}
Pedra preciosa, da côr do figo.
\section{Sycomancia}
\begin{itemize}
\item {Grp. gram.:f.}
\end{itemize}
\begin{itemize}
\item {Proveniência:(Do gr. \textunderscore sucon\textunderscore  + \textunderscore manteia\textunderscore )}
\end{itemize}
Antigo systema de adivinhação por meio de fôlhas de figueira, nas quaes se escreviam as preguntas, de que se desejava têr a resposta.
\section{Sycomântico}
\begin{itemize}
\item {Grp. gram.:adj.}
\end{itemize}
Relativo a sycomancia.
\section{Sycômoro}
\begin{itemize}
\item {Grp. gram.:m.}
\end{itemize}
\begin{itemize}
\item {Proveniência:(Lat. \textunderscore sicumorus\textunderscore )}
\end{itemize}
Espécie de figueira das margens do Mediterrâneo.
Nome de outras árvores.
\section{Sýcone}
\begin{itemize}
\item {Grp. gram.:m.}
\end{itemize}
O mesmo que \textunderscore sýcono\textunderscore .
\section{Sýcono}
\begin{itemize}
\item {Grp. gram.:m.}
\end{itemize}
\begin{itemize}
\item {Utilização:Bot.}
\end{itemize}
\begin{itemize}
\item {Proveniência:(Do gr. \textunderscore sukon\textunderscore , figo)}
\end{itemize}
Inflorescência em que o receptáculo envolve as flôres, como succede no figo.
\section{Sicófago}
\begin{itemize}
\item {Grp. gram.:m.  e  adj.}
\end{itemize}
\begin{itemize}
\item {Proveniência:(Do gr. \textunderscore sukon\textunderscore  + \textunderscore phagein\textunderscore )}
\end{itemize}
O que se alimenta de figos.
\section{Sicofanta}
\begin{itemize}
\item {Grp. gram.:m.}
\end{itemize}
\begin{itemize}
\item {Proveniência:(Lat. \textunderscore sycophanta\textunderscore )}
\end{itemize}
Mentiroso; calumniador; velhaco.
\section{Sicofantismo}
\begin{itemize}
\item {Grp. gram.:m.}
\end{itemize}
Carácter de sicofanta.
\section{Sicofilo}
\begin{itemize}
\item {Grp. gram.:m.}
\end{itemize}
\begin{itemize}
\item {Utilização:Bot.}
\end{itemize}
\begin{itemize}
\item {Proveniência:(Lat. \textunderscore sycophyllon\textunderscore )}
\end{itemize}
Designação antiga da alteia.
\section{Sicose}
\begin{itemize}
\item {Grp. gram.:f.}
\end{itemize}
\begin{itemize}
\item {Utilização:Bot.}
\end{itemize}
\begin{itemize}
\item {Proveniência:(Lat. \textunderscore sycosis\textunderscore )}
\end{itemize}
Doença dos folículos pilosos, causada geralmente por um parasito criptogâmico.
\section{Sicótico}
\begin{itemize}
\item {Grp. gram.:adj.}
\end{itemize}
Relativo á sicose.
\section{Sienite}
\begin{itemize}
\item {Grp. gram.:f.}
\end{itemize}
\begin{itemize}
\item {Proveniência:(Lat. \textunderscore syenites\textunderscore )}
\end{itemize}
Espécie de rocha granítica.
\section{Sienítico}
\begin{itemize}
\item {Grp. gram.:adj.}
\end{itemize}
Relativo á sienita.
\section{Sílaba}
\begin{itemize}
\item {Grp. gram.:f.}
\end{itemize}
\begin{itemize}
\item {Utilização:fam.}
\end{itemize}
\begin{itemize}
\item {Utilização:Fig.}
\end{itemize}
\begin{itemize}
\item {Proveniência:(Lat. \textunderscore syllaba\textunderscore )}
\end{itemize}
Som, produzido por uma só emissão de voz.
Letra ou conjunto de letras, que se pronuncíam com uma só emissão de voz.
Som articulado.
\section{Silabação}
\begin{itemize}
\item {Grp. gram.:f.}
\end{itemize}
Acto ou efeito de silabar.
Modo de lêr, dividindo as palavras em sílabas.
\section{Silabada}
\begin{itemize}
\item {Grp. gram.:f.}
\end{itemize}
\begin{itemize}
\item {Utilização:Fam.}
\end{itemize}
\begin{itemize}
\item {Proveniência:(De \textunderscore sílaba\textunderscore )}
\end{itemize}
Êrro de pronúncia ou de acentuação da palavra.
\section{Silabar}
\begin{itemize}
\item {Grp. gram.:v. i.}
\end{itemize}
\begin{itemize}
\item {Proveniência:(De \textunderscore sílaba\textunderscore )}
\end{itemize}
O mesmo que \textunderscore soletrar\textunderscore .
\section{Silabário}
\begin{itemize}
\item {Grp. gram.:m.}
\end{itemize}
\begin{itemize}
\item {Proveniência:(De \textunderscore sílaba\textunderscore )}
\end{itemize}
Pequeno livro, em que as crianças aprendem a lêr.
Cartilha.
Parte da cartilha, em que as sílabas estão mais ou menos metodicamente dispostas.
\section{Silabicamente}
\begin{itemize}
\item {Grp. gram.:adv.}
\end{itemize}
De modo silábico.
Segundo a disposição das sílabas.
\section{Silábico}
\begin{itemize}
\item {Grp. gram.:adj.}
\end{itemize}
\begin{itemize}
\item {Proveniência:(Lat. \textunderscore syllabicus\textunderscore )}
\end{itemize}
Relativo ás sílabas.
\section{Silabismo}
\begin{itemize}
\item {Grp. gram.:m.}
\end{itemize}
\begin{itemize}
\item {Proveniência:(De \textunderscore sílaba\textunderscore )}
\end{itemize}
Sistema do escrita, em que cada sílaba é representada por um sinal próprio.
\section{Silepse}
\begin{itemize}
\item {Grp. gram.:f.}
\end{itemize}
\begin{itemize}
\item {Proveniência:(Lat. \textunderscore syllepsis\textunderscore )}
\end{itemize}
Figura gramatical, em que a regência das palavras segue mais a lógica que as regras gramaticaes.
Emprêgo de uma palavra no sentido próprio e figurado, ao mesmo tempo.
Conhecimento reflexo, em Filosofia.
\section{Sileptico}
\begin{itemize}
\item {Grp. gram.:adj.}
\end{itemize}
Relativo á silepse.
\section{Sílfide}
\begin{itemize}
\item {Grp. gram.:f.}
\end{itemize}
O mesmo que \textunderscore silfo\textunderscore .
\section{Silfídico}
\begin{itemize}
\item {Grp. gram.:adj.}
\end{itemize}
Relativo a sílfide.
\section{Silfo}
\begin{itemize}
\item {Grp. gram.:m.}
\end{itemize}
\begin{itemize}
\item {Utilização:poét.}
\end{itemize}
\begin{itemize}
\item {Utilização:Fig.}
\end{itemize}
\begin{itemize}
\item {Proveniência:(T. inventado por Paracelso, segundo dizem; ou do gr. \textunderscore silphe\textunderscore , segundo Stappers)}
\end{itemize}
Gênio do ar.
Mulhér franzina e delicada, imagem vaporosa.
\section{Silogismo}
\begin{itemize}
\item {Grp. gram.:m.}
\end{itemize}
\begin{itemize}
\item {Proveniência:(Lat. \textunderscore syllogismus\textunderscore )}
\end{itemize}
Argumento, formado de três proposições, estando a conclusão contida numa das duas primeiras, e mostrando a outra que a mesma conclusão ali está contida.
\section{Silogístico}
\begin{itemize}
\item {Grp. gram.:adj.}
\end{itemize}
\begin{itemize}
\item {Proveniência:(Lat. \textunderscore syllogisticus\textunderscore )}
\end{itemize}
Relativo ao silogismo.
\section{Silogizar}
\begin{itemize}
\item {Grp. gram.:v. t.}
\end{itemize}
\begin{itemize}
\item {Grp. gram.:V. i.}
\end{itemize}
\begin{itemize}
\item {Proveniência:(Lat. \textunderscore syllogizare\textunderscore )}
\end{itemize}
Concluir por meio de raciocinio.
Empregar silogismos.
\section{Silvanite}
\begin{itemize}
\item {Grp. gram.:f.}
\end{itemize}
O mesmo que \textunderscore silvanito\textunderscore .
\section{Silvanito}
\begin{itemize}
\item {Grp. gram.:m.}
\end{itemize}
\begin{itemize}
\item {Utilização:Miner.}
\end{itemize}
Tellureto de prata e oiro natural.
\section{Silviano}
\begin{itemize}
\item {Grp. gram.:adj.}
\end{itemize}
\begin{itemize}
\item {Utilização:Anat.}
\end{itemize}
\begin{itemize}
\item {Proveniência:(De \textunderscore Sýlvio\textunderscore , n. p.)}
\end{itemize}
Diz-se dos vasos e outros órgãos, que se acham na depressão cerebral, chamada \textunderscore scissura de Sýlvio\textunderscore , e que separa os lóbulos anterior e médio.
\section{Simbiose}
\begin{itemize}
\item {Grp. gram.:f.}
\end{itemize}
\begin{itemize}
\item {Utilização:Bot.}
\end{itemize}
\begin{itemize}
\item {Utilização:P. us.}
\end{itemize}
\begin{itemize}
\item {Proveniência:(Do gr. \textunderscore sun\textunderscore  + \textunderscore bios\textunderscore )}
\end{itemize}
Associação mútua entre a alga e o cogumelo.
Vida em comum; concubinato.
\section{Simbiota}
\begin{itemize}
\item {Grp. gram.:m.}
\end{itemize}
\begin{itemize}
\item {Proveniência:(Do gr. \textunderscore sun\textunderscore  + \textunderscore bios\textunderscore )}
\end{itemize}
Espécie de ácaro, parecido ao psoropta.
\section{Simbléfaro}
\begin{itemize}
\item {Grp. gram.:m.}
\end{itemize}
\begin{itemize}
\item {Proveniência:(Do gr. \textunderscore suan\textunderscore  + \textunderscore blepharon\textunderscore )}
\end{itemize}
Reunião, mais ou menos completa, da pálpebra com o globo ocular.
\section{Simbolanto}
\begin{itemize}
\item {Grp. gram.:m.}
\end{itemize}
\begin{itemize}
\item {Proveniência:(Do gr. \textunderscore sumbole\textunderscore  + \textunderscore anthos\textunderscore )}
\end{itemize}
Espécie de genciana dos Andes.
\section{Simbolia}
\begin{itemize}
\item {Grp. gram.:f.}
\end{itemize}
\begin{itemize}
\item {Utilização:Neol.}
\end{itemize}
O mesmo que \textunderscore simbólica\textunderscore .
\section{Simbólica}
\begin{itemize}
\item {Grp. gram.:f.}
\end{itemize}
\begin{itemize}
\item {Proveniência:(De \textunderscore simbólico\textunderscore )}
\end{itemize}
Conjunto de símbolos, próprios de uma religião, de uma época ou de um povo.
Ciência ou obra, que se ocupa dêsses símbolos.
Sistema dos que consideram como símbolos os mitos do Politeísmo.
\section{Simbolicamente}
\begin{itemize}
\item {Grp. gram.:adv.}
\end{itemize}
De modo simbólico.
Por meio de símbolo; alegoricamente.
\section{Simbólico}
\begin{itemize}
\item {Grp. gram.:adj.}
\end{itemize}
\begin{itemize}
\item {Proveniência:(Lat. \textunderscore symbolicus\textunderscore )}
\end{itemize}
Relativo a símbolo.
Que tem o carácter de símbolo.
Alegórico.
Relativo aos formulários da fé.
\section{Simbolismo}
\begin{itemize}
\item {Grp. gram.:m.}
\end{itemize}
\begin{itemize}
\item {Proveniência:(De \textunderscore símbolo\textunderscore )}
\end{itemize}
Expressão ou interpretação por meio de símbolos.
Simbólica.
Escola ou processo literário, que prefere as fórmulas enigmáticas e abstrusas.
\section{Simbolista}
\begin{itemize}
\item {Grp. gram.:adj.}
\end{itemize}
\begin{itemize}
\item {Grp. gram.:M.}
\end{itemize}
\begin{itemize}
\item {Proveniência:(De \textunderscore símbolo\textunderscore )}
\end{itemize}
Relativo ao simbolismo.
Sectário do simbolismo.
\section{Simbolístico}
\begin{itemize}
\item {Grp. gram.:adj.}
\end{itemize}
\begin{itemize}
\item {Utilização:Neol.}
\end{itemize}
Relativo aos simbolistas.
\section{Simbolização}
\begin{itemize}
\item {Grp. gram.:f.}
\end{itemize}
Acto ou efeito de simbolizar.
\section{Simbolizador}
\begin{itemize}
\item {Grp. gram.:m.  e  adj.}
\end{itemize}
O que simboliza.
\section{Simbolizar}
\begin{itemize}
\item {Grp. gram.:v. t.}
\end{itemize}
\begin{itemize}
\item {Grp. gram.:V. i.}
\end{itemize}
\begin{itemize}
\item {Proveniência:(De \textunderscore símbolo\textunderscore )}
\end{itemize}
Significar por meio de símbolos.
Exprimir simbolicamente.
Sêr símbolo de: \textunderscore a balança simboliza a justiça\textunderscore .
Falar ou escrever simbolicamente.
\section{Símbolo}
\begin{itemize}
\item {Grp. gram.:m.}
\end{itemize}
\begin{itemize}
\item {Utilização:Fig.}
\end{itemize}
\begin{itemize}
\item {Proveniência:(Lat. \textunderscore symbolum\textunderscore )}
\end{itemize}
Sinal particular, pelo qual se reconheciam os iniciados nos mistérios do culto de algumas divindades gregas.
Imagem, que se emprega como sinal de uma coisa.
Sinal externo de um sacramento.
Conjunto dos principaes artigos de uma religião.
Substituição do nome de uma coisa pelo nome de um sinal.
Sinal, divisa.
\section{Simbologia}
\begin{itemize}
\item {Grp. gram.:f.}
\end{itemize}
Estudo á cêrca dos símbolos.
\section{Simbológico}
\begin{itemize}
\item {Grp. gram.:adj.}
\end{itemize}
Relativo á simbologia.
\section{Simetria}
\begin{itemize}
\item {Grp. gram.:f.}
\end{itemize}
\begin{itemize}
\item {Proveniência:(Lat. \textunderscore symmetria\textunderscore )}
\end{itemize}
Relação de grandeza e de figura entre as partes de um todo ou entre as partes e um todo.
Qualquer disposição de coisas, observando-se certa ordem ou proporção.
\section{Simetríaco}
\begin{itemize}
\item {Grp. gram.:adj.}
\end{itemize}
Em que há simetria:«\textunderscore simetríacas regras\textunderscore ». Macedo, \textunderscore Motim\textunderscore , I, 202.
\section{Simetricamente}
\begin{itemize}
\item {Grp. gram.:adv.}
\end{itemize}
De modo simétrico.
Com simetria; com proporção.
\section{Simétrico}
\begin{itemize}
\item {Grp. gram.:adj.}
\end{itemize}
\begin{itemize}
\item {Proveniência:(Gr. \textunderscore sumetrikos\textunderscore )}
\end{itemize}
Que tem simetria; relativo á simetria.
\section{Simetrizar}
\begin{itemize}
\item {Grp. gram.:v. t.}
\end{itemize}
\begin{itemize}
\item {Grp. gram.:V. i.}
\end{itemize}
\begin{itemize}
\item {Proveniência:(De \textunderscore simetria\textunderscore )}
\end{itemize}
Tornar simétrico, dispor simetricamente.
Têr simetria, em relação a outra coisa.
\section{Sycóphago}
\begin{itemize}
\item {Grp. gram.:m.  e  adj.}
\end{itemize}
\begin{itemize}
\item {Proveniência:(Do gr. \textunderscore sukon\textunderscore  + \textunderscore phagein\textunderscore )}
\end{itemize}
O que se alimenta de figos.
\section{Sycophanta}
\begin{itemize}
\item {Grp. gram.:m.}
\end{itemize}
\begin{itemize}
\item {Proveniência:(Lat. \textunderscore sycophanta\textunderscore )}
\end{itemize}
Mentiroso; calumniador; velhaco.
\section{Sycophantismo}
\begin{itemize}
\item {Grp. gram.:m.}
\end{itemize}
Carácter de sycophanta.
\section{Sycophyllo}
\begin{itemize}
\item {Grp. gram.:m.}
\end{itemize}
\begin{itemize}
\item {Utilização:Bot.}
\end{itemize}
\begin{itemize}
\item {Proveniência:(Lat. \textunderscore sycophyllon\textunderscore )}
\end{itemize}
Designação antiga da altheia.
\section{Sycose}
\begin{itemize}
\item {Grp. gram.:f.}
\end{itemize}
\begin{itemize}
\item {Utilização:Bot.}
\end{itemize}
\begin{itemize}
\item {Proveniência:(Lat. \textunderscore sycosis\textunderscore )}
\end{itemize}
Doença dos follículos pilosos, causada geralmente por um parasito cryptogâmico.
\section{Sycótico}
\begin{itemize}
\item {Grp. gram.:adj.}
\end{itemize}
Relativo á sycose.
\section{Syenite}
\begin{itemize}
\item {Grp. gram.:f.}
\end{itemize}
\begin{itemize}
\item {Proveniência:(Lat. \textunderscore syenites\textunderscore )}
\end{itemize}
Espécie de rocha granítica.
\section{Syenítico}
\begin{itemize}
\item {Grp. gram.:adj.}
\end{itemize}
Relativo á syenita.
\section{Sylimba}
\begin{itemize}
\item {Grp. gram.:f.}
\end{itemize}
Instrumento músico, do Norte do Zambeze.
\section{Syllaba}
\begin{itemize}
\item {Grp. gram.:f.}
\end{itemize}
\begin{itemize}
\item {Utilização:fam.}
\end{itemize}
\begin{itemize}
\item {Utilização:Fig.}
\end{itemize}
\begin{itemize}
\item {Proveniência:(Lat. \textunderscore syllaba\textunderscore )}
\end{itemize}
Som, produzido por uma só emissão de voz.
Letra ou conjunto de letras, que se pronuncíam com uma só emissão de voz.
Som articulado.
\section{Syllabação}
\begin{itemize}
\item {Grp. gram.:f.}
\end{itemize}
Acto ou effeito de syllabar.
Modo de lêr, dividindo as palavras em sýllabas.
\section{Syllabada}
\begin{itemize}
\item {Grp. gram.:f.}
\end{itemize}
\begin{itemize}
\item {Utilização:Fam.}
\end{itemize}
\begin{itemize}
\item {Proveniência:(De \textunderscore sýllaba\textunderscore )}
\end{itemize}
Êrro de pronúncia ou de accentuação da palavra.
\section{Syllabar}
\begin{itemize}
\item {Grp. gram.:v. i.}
\end{itemize}
\begin{itemize}
\item {Proveniência:(De \textunderscore sýllaba\textunderscore )}
\end{itemize}
O mesmo que \textunderscore soletrar\textunderscore .
\section{Syllabário}
\begin{itemize}
\item {Grp. gram.:m.}
\end{itemize}
\begin{itemize}
\item {Proveniência:(De \textunderscore sýllaba\textunderscore )}
\end{itemize}
Pequeno livro, em que as crianças aprendem a lêr.
Cartilha.
Parte da cartilha, em que as sýllabas estão mais ou menos methodicamente dispostas.
\section{Syllabicamente}
\begin{itemize}
\item {Grp. gram.:adv.}
\end{itemize}
De modo syllábico.
Segundo a disposição das sýllabas.
\section{Syllábico}
\begin{itemize}
\item {Grp. gram.:adj.}
\end{itemize}
\begin{itemize}
\item {Proveniência:(Lat. \textunderscore syllabicus\textunderscore )}
\end{itemize}
Relativo ás sýllabas.
\section{Syllabismo}
\begin{itemize}
\item {Grp. gram.:m.}
\end{itemize}
\begin{itemize}
\item {Proveniência:(De \textunderscore sýllaba\textunderscore )}
\end{itemize}
Systema do escrita, em que cada sýllaba é representada por um sinal próprio.
\section{Syllepse}
\begin{itemize}
\item {Grp. gram.:f.}
\end{itemize}
\begin{itemize}
\item {Proveniência:(Lat. \textunderscore syllepsis\textunderscore )}
\end{itemize}
Figura grammatical, em que a regência das palavras segue mais a lógica que as regras grammaticaes.
Emprêgo de uma palavra no sentido próprio e figurado, ao mesmo tempo.
Conhecimento reflexo, em Philosophia.
\section{Sylleptico}
\begin{itemize}
\item {Grp. gram.:adj.}
\end{itemize}
Relativo á syllepse.
\section{Syllogismo}
\begin{itemize}
\item {Grp. gram.:m.}
\end{itemize}
\begin{itemize}
\item {Proveniência:(Lat. \textunderscore syllogismus\textunderscore )}
\end{itemize}
Argumento, formado de três proposições, estando a conclusão contida numa das duas primeiras, e mostrando a outra que a mesma conclusão alli está contida.
\section{Syllogístico}
\begin{itemize}
\item {Grp. gram.:adj.}
\end{itemize}
\begin{itemize}
\item {Proveniência:(Lat. \textunderscore syllogisticus\textunderscore )}
\end{itemize}
Relativo ao syllogismo.
\section{Syllogizar}
\begin{itemize}
\item {Grp. gram.:v. t.}
\end{itemize}
\begin{itemize}
\item {Grp. gram.:V. i.}
\end{itemize}
\begin{itemize}
\item {Proveniência:(Lat. \textunderscore syllogizare\textunderscore )}
\end{itemize}
Concluir por meio de raciocinio.
Empregar syllogismos.
\section{Sýlphide}
\begin{itemize}
\item {Grp. gram.:f.}
\end{itemize}
O mesmo que \textunderscore sylpho\textunderscore .
\section{Sylphídico}
\begin{itemize}
\item {Grp. gram.:adj.}
\end{itemize}
Relativo a sýlphide.
\section{Sylpho}
\begin{itemize}
\item {Grp. gram.:m.}
\end{itemize}
\begin{itemize}
\item {Utilização:poét.}
\end{itemize}
\begin{itemize}
\item {Utilização:Fig.}
\end{itemize}
\begin{itemize}
\item {Proveniência:(T. inventado por Paracelso, segundo dizem; ou do gr. \textunderscore silphe\textunderscore , segundo Stappers)}
\end{itemize}
Gênio do ar.
Mulhér franzina e delicada, imagem vaporosa.
\section{Sylvanite}
\begin{itemize}
\item {Grp. gram.:f.}
\end{itemize}
O mesmo que \textunderscore sylvanito\textunderscore .
\section{Sylvanito}
\begin{itemize}
\item {Grp. gram.:m.}
\end{itemize}
\begin{itemize}
\item {Utilização:Miner.}
\end{itemize}
Tellureto de prata e oiro natural.
\section{Sylviano}
\begin{itemize}
\item {Grp. gram.:adj.}
\end{itemize}
\begin{itemize}
\item {Utilização:Anat.}
\end{itemize}
\begin{itemize}
\item {Proveniência:(De \textunderscore Sýlvio\textunderscore , n. p.)}
\end{itemize}
Diz-se dos vasos e outros órgãos, que se acham na depressão cerebral, chamada \textunderscore scissura de Sýlvio\textunderscore , e que separa os lóbulos anterior e médio.
\section{Sylvina}
\begin{itemize}
\item {Grp. gram.:f.}
\end{itemize}
\begin{itemize}
\item {Utilização:Miner.}
\end{itemize}
Chloreto de potássio natural.
\section{Symbiose}
\begin{itemize}
\item {Grp. gram.:f.}
\end{itemize}
\begin{itemize}
\item {Utilização:Bot.}
\end{itemize}
\begin{itemize}
\item {Utilização:P. us.}
\end{itemize}
\begin{itemize}
\item {Proveniência:(Do gr. \textunderscore sun\textunderscore  + \textunderscore bios\textunderscore )}
\end{itemize}
Associação mútua entre a alga e o cogumelo.
Vida em commum; concubinato.
\section{Symbiota}
\begin{itemize}
\item {Grp. gram.:m.}
\end{itemize}
\begin{itemize}
\item {Proveniência:(Do gr. \textunderscore sun\textunderscore  + \textunderscore bios\textunderscore )}
\end{itemize}
Espécie de ácaro, parecido ao psoropta.
\section{Symblépharo}
\begin{itemize}
\item {Grp. gram.:m.}
\end{itemize}
\begin{itemize}
\item {Proveniência:(Do gr. \textunderscore suan\textunderscore  + \textunderscore blepharon\textunderscore )}
\end{itemize}
Reunião, mais ou menos completa, da pálpebra com o globo ocular.
\section{Symbolantho}
\begin{itemize}
\item {Grp. gram.:m.}
\end{itemize}
\begin{itemize}
\item {Proveniência:(Do gr. \textunderscore sumbole\textunderscore  + \textunderscore anthos\textunderscore )}
\end{itemize}
Espécie de genciana dos Andes.
\section{Symbolia}
\begin{itemize}
\item {Grp. gram.:f.}
\end{itemize}
\begin{itemize}
\item {Utilização:Neol.}
\end{itemize}
O mesmo que \textunderscore symbólica\textunderscore .
\section{Symbólica}
\begin{itemize}
\item {Grp. gram.:f.}
\end{itemize}
\begin{itemize}
\item {Proveniência:(De \textunderscore symbólico\textunderscore )}
\end{itemize}
Conjunto de sýmbolos, próprios de uma religião, de uma época ou de um povo.
Sciência ou obra, que se occupa dêsses sýmbolos.
Systema dos que consideram como sýmbolos os mythos do Polytheísmo.
\section{Symbolicamente}
\begin{itemize}
\item {Grp. gram.:adv.}
\end{itemize}
De modo symbólico.
Por meio de sýmbolo; allegoricamente.
\section{Symbólico}
\begin{itemize}
\item {Grp. gram.:adj.}
\end{itemize}
\begin{itemize}
\item {Proveniência:(Lat. \textunderscore symbolicus\textunderscore )}
\end{itemize}
Relativo a sýmbolo.
Que tem o carácter de sýmbolo.
Allegórico.
Relativo aos formulários da fé.
\section{Symbolismo}
\begin{itemize}
\item {Grp. gram.:m.}
\end{itemize}
\begin{itemize}
\item {Proveniência:(De \textunderscore sýmbolo\textunderscore )}
\end{itemize}
Expressão ou interpretação por meio de sýmbolos.
Symbólica.
Escola ou processo literário, que prefere as fórmulas enigmáticas e abstrusas.
\section{Symbolista}
\begin{itemize}
\item {Grp. gram.:adj.}
\end{itemize}
\begin{itemize}
\item {Grp. gram.:M.}
\end{itemize}
\begin{itemize}
\item {Proveniência:(De \textunderscore sýmbolo\textunderscore )}
\end{itemize}
Relativo ao symbolismo.
Sectário do symbolismo.
\section{Symbolístico}
\begin{itemize}
\item {Grp. gram.:adj.}
\end{itemize}
\begin{itemize}
\item {Utilização:Neol.}
\end{itemize}
Relativo aos symbolistas.
\section{Symbolização}
\begin{itemize}
\item {Grp. gram.:f.}
\end{itemize}
Acto ou effeito de symbolizar.
\section{Symbolizador}
\begin{itemize}
\item {Grp. gram.:m.  e  adj.}
\end{itemize}
O que symboliza.
\section{Symbolizar}
\begin{itemize}
\item {Grp. gram.:v. t.}
\end{itemize}
\begin{itemize}
\item {Grp. gram.:V. i.}
\end{itemize}
\begin{itemize}
\item {Proveniência:(De \textunderscore sýmbolo\textunderscore )}
\end{itemize}
Significar por meio de sýmbolos.
Exprimir symbolicamente.
Sêr sýmbolo de: \textunderscore a balança symboliza a justiça\textunderscore .
Falar ou escrever symbolicamente.
\section{Sýmbolo}
\begin{itemize}
\item {Grp. gram.:m.}
\end{itemize}
\begin{itemize}
\item {Utilização:Fig.}
\end{itemize}
\begin{itemize}
\item {Proveniência:(Lat. \textunderscore symbolum\textunderscore )}
\end{itemize}
Sinal particular, pelo qual se reconheciam os iniciados nos mystérios do culto de algumas divindades gregas.
Imagem, que se emprega como sinal de uma coisa.
Sinal externo de um sacramento.
Conjunto dos principaes artigos de uma religião.
Substituição do nome de uma coisa pelo nome de um sinal.
Sinal, divisa.
\section{Symbologia}
\begin{itemize}
\item {Grp. gram.:f.}
\end{itemize}
Estudo á cêrca dos sýmbolos.
\section{Symbológico}
\begin{itemize}
\item {Grp. gram.:adj.}
\end{itemize}
Relativo á symbologia.
\section{Symetria}
\begin{itemize}
\item {Grp. gram.:f.}
\end{itemize}
\begin{itemize}
\item {Proveniência:(Lat. \textunderscore symmetria\textunderscore )}
\end{itemize}
Relação de grandeza e de figura entre as partes de um todo ou entre as partes e um todo.
Qualquer disposição de coisas, observando-se certa ordem ou proporção.
\section{Symetríaco}
\begin{itemize}
\item {Grp. gram.:adj.}
\end{itemize}
Em que há symetria:«\textunderscore symetríacas regras\textunderscore ». Macedo, \textunderscore Motim\textunderscore , I, 202.
\section{Symetricamente}
\begin{itemize}
\item {Grp. gram.:adv.}
\end{itemize}
De modo symétrico.
Com symetria; com proporção.
\section{Symétrico}
\begin{itemize}
\item {Grp. gram.:adj.}
\end{itemize}
\begin{itemize}
\item {Proveniência:(Gr. \textunderscore sumetrikos\textunderscore )}
\end{itemize}
Que tem symetria; relativo á symetria.
\section{Symetrizar}
\begin{itemize}
\item {Grp. gram.:v. t.}
\end{itemize}
\begin{itemize}
\item {Grp. gram.:V. i.}
\end{itemize}
\begin{itemize}
\item {Proveniência:(De \textunderscore symetria\textunderscore )}
\end{itemize}
Tornar symétrico, dispor symetricamente.
Têr symetria, em relação a outra coisa.
\section{Simélio}
\begin{itemize}
\item {Grp. gram.:m.}
\end{itemize}
\begin{itemize}
\item {Utilização:Terat.}
\end{itemize}
\begin{itemize}
\item {Proveniência:(Do gr. \textunderscore sun\textunderscore  + \textunderscore melos\textunderscore )}
\end{itemize}
Monstro, caracterizado pela união dos dois membros inferiores, que terminam num pé duplo.
\section{Simpatia}
\begin{itemize}
\item {Grp. gram.:f.}
\end{itemize}
\begin{itemize}
\item {Proveniência:(Lat. \textunderscore sympathia\textunderscore )}
\end{itemize}
Relação fisiológica entre dois órgãos, mais ou menos afastados.
Tendência instinctiva para uma pessôa ou para uma coisa.
Inclinação mútua de duas pessôas ou entre duas coisas.
\section{Simpaticamente}
\begin{itemize}
\item {Grp. gram.:adv.}
\end{itemize}
De modo simpático; com simpatia.
\section{Simpático}
\begin{itemize}
\item {Grp. gram.:adj.}
\end{itemize}
Relativo á simpatia.
Que dimana da simpatia.
Que inspira simpatia.
\section{Simpatismo}
\begin{itemize}
\item {Grp. gram.:m.}
\end{itemize}
\begin{itemize}
\item {Utilização:Neol.}
\end{itemize}
O mesmo que \textunderscore simpatia\textunderscore .
\section{Simpatista}
\begin{itemize}
\item {Grp. gram.:m.  e  f.}
\end{itemize}
\begin{itemize}
\item {Proveniência:(De \textunderscore simpatia\textunderscore )}
\end{itemize}
Pessôa, que sustenta que a causa dos sentimentos, que alguém nos inspira, são as emanações dêste.
\section{Simpatizante}
\begin{itemize}
\item {Grp. gram.:adj.}
\end{itemize}
Que simpatiza.
\section{Simpatizar}
\begin{itemize}
\item {Grp. gram.:v. i.}
\end{itemize}
Têr simpatia; sentir inclinação, afeição ou tendência.
\section{Simpetálico}
\begin{itemize}
\item {Grp. gram.:adj.}
\end{itemize}
\begin{itemize}
\item {Utilização:Bot.}
\end{itemize}
\begin{itemize}
\item {Proveniência:(Do gr. \textunderscore sun\textunderscore  + \textunderscore petalon\textunderscore )}
\end{itemize}
Diz-se dos estames que, reunindo as pétalas, dão a uma corola polipétala a aparência de monopétala.
\section{Sinfiandra}
\begin{itemize}
\item {Grp. gram.:f.}
\end{itemize}
\begin{itemize}
\item {Proveniência:(Do gr. \textunderscore sumphuo\textunderscore  + \textunderscore aner\textunderscore )}
\end{itemize}
Gênero de plantas campanuláceas.
\section{Sinfisandria}
\begin{itemize}
\item {Grp. gram.:f.}
\end{itemize}
\begin{itemize}
\item {Utilização:Bot.}
\end{itemize}
\begin{itemize}
\item {Proveniência:(Do gr. \textunderscore sun\textunderscore  + \textunderscore phusis\textunderscore  + \textunderscore aner\textunderscore )}
\end{itemize}
Classe de plantas, formada por A. Richard no systema de Linneu, e que compreende as plantas de ovário polispérmico e anteras reunidas.
\section{Sinfisantéreas}
\begin{itemize}
\item {Grp. gram.:f. pl.}
\end{itemize}
\begin{itemize}
\item {Utilização:Bot.}
\end{itemize}
Plantas, cujos estames são reunidos pelas anteras.
\section{Sínfise}
\begin{itemize}
\item {Grp. gram.:f.}
\end{itemize}
\begin{itemize}
\item {Utilização:Anat.}
\end{itemize}
\begin{itemize}
\item {Utilização:Med.}
\end{itemize}
\begin{itemize}
\item {Proveniência:(Do gr. \textunderscore sun\textunderscore  + \textunderscore phusis\textunderscore )}
\end{itemize}
Articulação imóvel de dois ossos.
Aderência de dois folhetos de uma serose.
\section{Sinfisiano}
\begin{itemize}
\item {Grp. gram.:adj.}
\end{itemize}
Relativo á sínfise.
\section{Sinfisiário}
\begin{itemize}
\item {Grp. gram.:adj.}
\end{itemize}
O mesmo que \textunderscore sinfisiano\textunderscore .
\section{Sinfísio}
\begin{itemize}
\item {Grp. gram.:adj.}
\end{itemize}
O mesmo que \textunderscore sinfisiano\textunderscore .
\section{Sinfisiógino}
\begin{itemize}
\item {Grp. gram.:adj.}
\end{itemize}
\begin{itemize}
\item {Utilização:Bot.}
\end{itemize}
\begin{itemize}
\item {Proveniência:(Do gr. \textunderscore sun\textunderscore  + \textunderscore phusis\textunderscore  + \textunderscore gune\textunderscore )}
\end{itemize}
Diz-se das plantas, em que os órgãos femininos estão soldados.
\section{Sinfisiotomia}
\begin{itemize}
\item {Grp. gram.:f.}
\end{itemize}
\begin{itemize}
\item {Utilização:Cir.}
\end{itemize}
\begin{itemize}
\item {Proveniência:(Do gr. \textunderscore sumphusis\textunderscore  + \textunderscore tome\textunderscore )}
\end{itemize}
Incisão da substância fibro-cartilaginosa, que liga os ossos púbicos.
\section{Sinfisiotómico}
\begin{itemize}
\item {Grp. gram.:adj.}
\end{itemize}
Relativo á sinfisiotomia.
\section{Sinfonia}
\begin{itemize}
\item {Grp. gram.:f.}
\end{itemize}
\begin{itemize}
\item {Proveniência:(Lat. \textunderscore symphonia\textunderscore )}
\end{itemize}
Reunião de vozes ou conjunto de sons.
Harmonia.
Música, executada só por uma orquestra.
Composição musical, em fórma de sonata.
Conjunto de sinfonistas.
Trecho instrumental, que precede uma ópera, um concêrto, etc.
\section{Sinfonista}
\begin{itemize}
\item {Grp. gram.:m.  e  adj.}
\end{itemize}
Pessôa, que compõe sinfonias.
Instrumentista de sinfonias.
\section{Sinfonizar}
\begin{itemize}
\item {Grp. gram.:v.}
\end{itemize}
\begin{itemize}
\item {Utilização:t. Mús.}
\end{itemize}
\begin{itemize}
\item {Utilização:Des.}
\end{itemize}
\begin{itemize}
\item {Proveniência:(De \textunderscore sinfonia\textunderscore )}
\end{itemize}
Cantar em oitavas.
\section{Sinforina}
\begin{itemize}
\item {Grp. gram.:f.}
\end{itemize}
Arbusto caprifoliáceo, originário da Virgínia, (\textunderscore lonicera symphoricarpus\textunderscore , Lin.).
\section{Simpiezómetro}
\begin{itemize}
\item {Grp. gram.:m.}
\end{itemize}
\begin{itemize}
\item {Proveniência:(Do gr. \textunderscore sun\textunderscore  + \textunderscore piezein\textunderscore  + \textunderscore metron\textunderscore )}
\end{itemize}
Barómetro com reservatório de ar.
\section{Simpléctico}
\begin{itemize}
\item {Grp. gram.:adj.}
\end{itemize}
\begin{itemize}
\item {Utilização:Hist. Nat.}
\end{itemize}
\begin{itemize}
\item {Grp. gram.:M.}
\end{itemize}
\begin{itemize}
\item {Proveniência:(Gr. \textunderscore sunplektikos\textunderscore )}
\end{itemize}
Que está entrelaçado com outro corpo.
Uma das peças ósseas da cabeça dos peixes.
\section{Simplocarpo}
\begin{itemize}
\item {Grp. gram.:m.}
\end{itemize}
\begin{itemize}
\item {Proveniência:(Do gr. \textunderscore sunplokos\textunderscore  + \textunderscore karpos\textunderscore )}
\end{itemize}
Gênero de plantas aráceas.
\section{Símploce}
\begin{itemize}
\item {Grp. gram.:f.}
\end{itemize}
\begin{itemize}
\item {Utilização:Rhet.}
\end{itemize}
\begin{itemize}
\item {Proveniência:(Gr. \textunderscore sumploke\textunderscore )}
\end{itemize}
Figura de palavras ou repetição, que consiste em começar ou acabar frases pelas mesmas palavras.
\section{Simposiarca}
\begin{itemize}
\item {Grp. gram.:m.}
\end{itemize}
\begin{itemize}
\item {Proveniência:(Do gr. \textunderscore sumposion\textunderscore  + \textunderscore arkhe\textunderscore )}
\end{itemize}
Aquele que, entre os Gregos, era escolhido, á sorte, rei de um festim.
\section{Simprítane}
\begin{itemize}
\item {Grp. gram.:m.}
\end{itemize}
\begin{itemize}
\item {Proveniência:(Do gr. \textunderscore sun\textunderscore  + \textunderscore prutanis\textunderscore )}
\end{itemize}
Cada um dos colegas do primeiro magistrado, em algumas antigas républicas gregas.
\section{Simptose}
\begin{itemize}
\item {Grp. gram.:f.}
\end{itemize}
\begin{itemize}
\item {Utilização:Des.}
\end{itemize}
\begin{itemize}
\item {Proveniência:(Gr. \textunderscore sumptosis\textunderscore )}
\end{itemize}
Debilidade ou enfraquecimento dos órgãos.
Magreza.
\section{Sinadelfo}
\begin{itemize}
\item {Grp. gram.:adj.}
\end{itemize}
\begin{itemize}
\item {Proveniência:(Do gr. \textunderscore sun\textunderscore  + \textunderscore adelphos\textunderscore )}
\end{itemize}
Diz-se do monstro, que tem um só tronco e oito membros.
\section{Sinagelástico}
\begin{itemize}
\item {Grp. gram.:adj.}
\end{itemize}
\begin{itemize}
\item {Proveniência:(Do gr. \textunderscore sun\textunderscore  + \textunderscore agelazein\textunderscore )}
\end{itemize}
Que vive em grupos ou bandos.
\section{Sinagoga}
\begin{itemize}
\item {Grp. gram.:f.}
\end{itemize}
\begin{itemize}
\item {Utilização:Bras. de Minas}
\end{itemize}
\begin{itemize}
\item {Proveniência:(Do gr. \textunderscore sunagoge\textunderscore )}
\end{itemize}
Assembleia de fiéis, entre os Hebreus.
Templo judaico.
Desarranjo, desordem.
\section{Sinalefa}
\begin{itemize}
\item {Grp. gram.:f.}
\end{itemize}
\begin{itemize}
\item {Proveniência:(Gr. \textunderscore sunaloiphe\textunderscore )}
\end{itemize}
Reunião de duas sílabas numa só, por sinérese, crase ou elisão.
Instrumento de encadernador, para doirar filetes nas capas dos livros.
\section{Sinalefista}
\begin{itemize}
\item {Grp. gram.:m.}
\end{itemize}
\begin{itemize}
\item {Utilização:Neol.}
\end{itemize}
\begin{itemize}
\item {Proveniência:(De \textunderscore sinalefa\textunderscore )}
\end{itemize}
Aquelle que faz sinalefas ou que gosta de escrever as palavras, fazendo sinalefas inúteis:«\textunderscore p'ra nós\textunderscore »;«\textunderscore a entrada d'avenida\textunderscore »;« \textunderscore José d'Almeida\textunderscore ». Cf. Júl. Ribeiro, \textunderscore Padre Belch.\textunderscore , XII.
\section{Sinalagmático}
\begin{itemize}
\item {Grp. gram.:adj.}
\end{itemize}
\begin{itemize}
\item {Utilização:Jur.}
\end{itemize}
\begin{itemize}
\item {Proveniência:(Gr. \textunderscore sunallagmátikos\textunderscore )}
\end{itemize}
Diz-se de um contrato bilateral.
\section{Sínfito}
\begin{itemize}
\item {Grp. gram.:m.}
\end{itemize}
\begin{itemize}
\item {Proveniência:(Gr. \textunderscore sumphutos\textunderscore )}
\end{itemize}
Nome científico da consolda.
\section{Sintoma}
\begin{itemize}
\item {Grp. gram.:m.}
\end{itemize}
\begin{itemize}
\item {Utilização:Fig.}
\end{itemize}
\begin{itemize}
\item {Proveniência:(Lat. \textunderscore symptoma\textunderscore )}
\end{itemize}
Fenómeno das funções ou da constituição material dos órgãos, próprio para indicar a natureza, existência ou séde de uma enfermidade.
Indício; preságio.
\section{Sintomáticamente}
\begin{itemize}
\item {Grp. gram.:adv.}
\end{itemize}
De modo sintomático; como sintoma.
\section{Sintomático}
\begin{itemize}
\item {Grp. gram.:adj.}
\end{itemize}
\begin{itemize}
\item {Proveniência:(Gr. \textunderscore sumptomatikos\textunderscore )}
\end{itemize}
Relativo a sintoma; que é sintoma.
\section{Sintomatismo}
\begin{itemize}
\item {Grp. gram.:m.}
\end{itemize}
\begin{itemize}
\item {Proveniência:(Do gr. \textunderscore sumptoma\textunderscore )}
\end{itemize}
Sistema medicinal, que consiste em atacar os sintomas de uma doença e não a própria doença.
\section{Sintomatista}
\begin{itemize}
\item {Grp. gram.:m.  e  adj.}
\end{itemize}
\begin{itemize}
\item {Proveniência:(Do gr. \textunderscore sumptoma\textunderscore )}
\end{itemize}
O que segue o sintomatismo.
\section{Sintomatologia}
\begin{itemize}
\item {Grp. gram.:f.}
\end{itemize}
\begin{itemize}
\item {Proveniência:(Do gr. \textunderscore sumptoma\textunderscore , \textunderscore sumptomatos\textunderscore  + \textunderscore logos\textunderscore )}
\end{itemize}
Parte da Medicina que trata dos sintomas das doenças.
\section{Sintomatológico}
\begin{itemize}
\item {Grp. gram.:adj.}
\end{itemize}
Relativo á sintomatologia.
\section{Sintomatologista}
\begin{itemize}
\item {Grp. gram.:m.  e  adj.}
\end{itemize}
O que se ocupa ou que escreve á cêrca de sintomatologia.
\section{Sintomologia}
\textunderscore f.\textunderscore  (e der.)
O mesmo que \textunderscore sintomatologia\textunderscore , etc. Cf. Sousa Martins, \textunderscore Nosographia\textunderscore .
\section{Symmélio}
\begin{itemize}
\item {Grp. gram.:m.}
\end{itemize}
\begin{itemize}
\item {Utilização:Terat.}
\end{itemize}
\begin{itemize}
\item {Proveniência:(Do gr. \textunderscore sun\textunderscore  + \textunderscore melos\textunderscore )}
\end{itemize}
Monstro, caracterizado pela união dos dois membros inferiores, que terminam num pé duplo.
\section{Sympathia}
\begin{itemize}
\item {Grp. gram.:f.}
\end{itemize}
\begin{itemize}
\item {Proveniência:(Lat. \textunderscore sympathia\textunderscore )}
\end{itemize}
Relação physiológica entre dois órgãos, mais ou menos afastados.
Tendência instinctiva para uma pessôa ou para uma coisa.
Inclinação mútua de duas pessôas ou entre duas coisas.
\section{Sympathicamente}
\begin{itemize}
\item {Grp. gram.:adv.}
\end{itemize}
De modo sympáthico; com sympathia.
\section{Sympáthico}
\begin{itemize}
\item {Grp. gram.:adj.}
\end{itemize}
Relativo á sympathia.
Que dimana da sympathia.
Que inspira sympathia.
\section{Sympathismo}
\begin{itemize}
\item {Grp. gram.:m.}
\end{itemize}
\begin{itemize}
\item {Utilização:Neol.}
\end{itemize}
O mesmo que \textunderscore sympathia\textunderscore .
\section{Sympathista}
\begin{itemize}
\item {Grp. gram.:m.  e  f.}
\end{itemize}
\begin{itemize}
\item {Proveniência:(De \textunderscore sympathia\textunderscore )}
\end{itemize}
Pessôa, que sustenta que a causa dos sentimentos, que alguém nos inspira, são as emanações dêste.
\section{Sympathizante}
\begin{itemize}
\item {Grp. gram.:adj.}
\end{itemize}
Que sympathiza.
\section{Sympathizar}
\begin{itemize}
\item {Grp. gram.:v. i.}
\end{itemize}
Têr sympathia; sentir inclinação, affeição ou tendência.
\section{Sympetálico}
\begin{itemize}
\item {Grp. gram.:adj.}
\end{itemize}
\begin{itemize}
\item {Utilização:Bot.}
\end{itemize}
\begin{itemize}
\item {Proveniência:(Do gr. \textunderscore sun\textunderscore  + \textunderscore petalon\textunderscore )}
\end{itemize}
Diz-se dos estames que, reunindo as pétalas, dão a uma corolla polypétala a apparência de monopétala.
\section{Symphonia}
\begin{itemize}
\item {Grp. gram.:f.}
\end{itemize}
\begin{itemize}
\item {Proveniência:(Lat. \textunderscore symphonia\textunderscore )}
\end{itemize}
Reunião de vozes ou conjunto de sons.
Harmonia.
Música, executada só por uma orchestra.
Composição musical, em fórma de sonata.
Conjunto de symphonistas.
Trecho instrumental, que precede uma ópera, um concêrto, etc.
\section{Symphonista}
\begin{itemize}
\item {Grp. gram.:m.  e  adj.}
\end{itemize}
Pessôa, que compõe symphonias.
Instrumentista de symphonias.
\section{Symphonizar}
\begin{itemize}
\item {Grp. gram.:v.}
\end{itemize}
\begin{itemize}
\item {Utilização:t. Mús.}
\end{itemize}
\begin{itemize}
\item {Utilização:Des.}
\end{itemize}
\begin{itemize}
\item {Proveniência:(De \textunderscore symphonia\textunderscore )}
\end{itemize}
Cantar em oitavas.
\section{Symphorina}
\begin{itemize}
\item {Grp. gram.:f.}
\end{itemize}
Arbusto caprifoliáceo, originário da Virgínia, (\textunderscore lonicera symphoricarpus\textunderscore , Lin.).
\section{Symphyandra}
\begin{itemize}
\item {Grp. gram.:f.}
\end{itemize}
\begin{itemize}
\item {Proveniência:(Do gr. \textunderscore sumphuo\textunderscore  + \textunderscore aner\textunderscore )}
\end{itemize}
Gênero de plantas campanuláceas.
\section{Symphysandria}
\begin{itemize}
\item {Grp. gram.:f.}
\end{itemize}
\begin{itemize}
\item {Utilização:Bot.}
\end{itemize}
\begin{itemize}
\item {Proveniência:(Do gr. \textunderscore sun\textunderscore  + \textunderscore phusis\textunderscore  + \textunderscore aner\textunderscore )}
\end{itemize}
Classe de plantas, formada por A. Richard no systema de Linneu, e que comprehende as plantas de ovário polyspérmico e antheras reunidas.
\section{Symphysanthéreas}
\begin{itemize}
\item {Grp. gram.:f. pl.}
\end{itemize}
\begin{itemize}
\item {Utilização:Bot.}
\end{itemize}
Plantas, cujos estames são reunidos pelas antheras.
\section{Sýmphyse}
\begin{itemize}
\item {Grp. gram.:f.}
\end{itemize}
\begin{itemize}
\item {Utilização:Anat.}
\end{itemize}
\begin{itemize}
\item {Utilização:Med.}
\end{itemize}
\begin{itemize}
\item {Proveniência:(Do gr. \textunderscore sun\textunderscore  + \textunderscore phusis\textunderscore )}
\end{itemize}
Articulação immóvel de dois ossos.
Adherência de dois folhetos de uma serose.
\section{Symphysiano}
\begin{itemize}
\item {Grp. gram.:adj.}
\end{itemize}
Relativo á sýmphyse.
\section{Symphysiário}
\begin{itemize}
\item {Grp. gram.:adj.}
\end{itemize}
O mesmo que \textunderscore symphysiano\textunderscore .
\section{Symphýsio}
\begin{itemize}
\item {Grp. gram.:adj.}
\end{itemize}
O mesmo que \textunderscore symphysiano\textunderscore .
\section{Symphysiógyno}
\begin{itemize}
\item {Grp. gram.:adj.}
\end{itemize}
\begin{itemize}
\item {Utilização:Bot.}
\end{itemize}
\begin{itemize}
\item {Proveniência:(Do gr. \textunderscore sun\textunderscore  + \textunderscore phusis\textunderscore  + \textunderscore gune\textunderscore )}
\end{itemize}
Diz-se das plantas, em que os órgãos femininos estão soldados.
\section{Symphysiotomia}
\begin{itemize}
\item {Grp. gram.:f.}
\end{itemize}
\begin{itemize}
\item {Utilização:Cir.}
\end{itemize}
\begin{itemize}
\item {Proveniência:(Do gr. \textunderscore sumphusis\textunderscore  + \textunderscore tome\textunderscore )}
\end{itemize}
Incisão da substância fibro-cartilaginosa, que liga os ossos púbicos.
\section{Symphysiotómico}
\begin{itemize}
\item {Grp. gram.:adj.}
\end{itemize}
Relativo á symphysiotomia.
\section{Sýmphyto}
\begin{itemize}
\item {Grp. gram.:m.}
\end{itemize}
\begin{itemize}
\item {Proveniência:(Gr. \textunderscore sumphutos\textunderscore )}
\end{itemize}
Nome scientífico da consolda.
\section{Sympiezómetro}
\begin{itemize}
\item {Grp. gram.:m.}
\end{itemize}
\begin{itemize}
\item {Proveniência:(Do gr. \textunderscore sun\textunderscore  + \textunderscore piezein\textunderscore  + \textunderscore metron\textunderscore )}
\end{itemize}
Barómetro com reservatório de ar.
\section{Sympléctico}
\begin{itemize}
\item {Grp. gram.:adj.}
\end{itemize}
\begin{itemize}
\item {Utilização:Hist. Nat.}
\end{itemize}
\begin{itemize}
\item {Grp. gram.:M.}
\end{itemize}
\begin{itemize}
\item {Proveniência:(Gr. \textunderscore sunplektikos\textunderscore )}
\end{itemize}
Que está entrelaçado com outro corpo.
Uma das peças ósseas da cabeça dos peixes.
\section{Symplocarpo}
\begin{itemize}
\item {Grp. gram.:m.}
\end{itemize}
\begin{itemize}
\item {Proveniência:(Do gr. \textunderscore sunplokos\textunderscore  + \textunderscore karpos\textunderscore )}
\end{itemize}
Gênero de plantas aráceas.
\section{Sýmploce}
\begin{itemize}
\item {Grp. gram.:f.}
\end{itemize}
\begin{itemize}
\item {Utilização:Rhet.}
\end{itemize}
\begin{itemize}
\item {Proveniência:(Gr. \textunderscore sumploke\textunderscore )}
\end{itemize}
Figura de palavras ou repetição, que consiste em começar ou acabar phrases pelas mesmas palavras.
\section{Symposiarcha}
\begin{itemize}
\item {fónica:ca}
\end{itemize}
\begin{itemize}
\item {Grp. gram.:m.}
\end{itemize}
\begin{itemize}
\item {Proveniência:(Do gr. \textunderscore sumposion\textunderscore  + \textunderscore arkhe\textunderscore )}
\end{itemize}
Aquelle que, entre os Gregos, era escolhido, á sorte, rei de um festim.
\section{Symprýtane}
\begin{itemize}
\item {Grp. gram.:m.}
\end{itemize}
\begin{itemize}
\item {Proveniência:(Do gr. \textunderscore sun\textunderscore  + \textunderscore prutanis\textunderscore )}
\end{itemize}
Cada um dos collegas do primeiro magistrado, em algumas antigas républicas gregas.
\section{Symptoma}
\begin{itemize}
\item {Grp. gram.:m.}
\end{itemize}
\begin{itemize}
\item {Utilização:Fig.}
\end{itemize}
\begin{itemize}
\item {Proveniência:(Lat. \textunderscore symptoma\textunderscore )}
\end{itemize}
Phenómeno das funcções ou da constituição material dos órgãos, próprio para indicar a natureza, existência ou séde de uma enfermidade.
Indício; preságio.
\section{Symptomáticamente}
\begin{itemize}
\item {Grp. gram.:adv.}
\end{itemize}
De modo symptomático; como symptoma.
\section{Symptomático}
\begin{itemize}
\item {Grp. gram.:adj.}
\end{itemize}
\begin{itemize}
\item {Proveniência:(Gr. \textunderscore sumptomatikos\textunderscore )}
\end{itemize}
Relativo a symptoma; que é symptoma.
\section{Symptomatismo}
\begin{itemize}
\item {Grp. gram.:m.}
\end{itemize}
\begin{itemize}
\item {Proveniência:(Do gr. \textunderscore sumptoma\textunderscore )}
\end{itemize}
Systema medicinal, que consiste em atacar os symptomas de uma doença e não a própria doença.
\section{Symptomatista}
\begin{itemize}
\item {Grp. gram.:m.  e  adj.}
\end{itemize}
\begin{itemize}
\item {Proveniência:(Do gr. \textunderscore sumptoma\textunderscore )}
\end{itemize}
O que segue o symptomatismo.
\section{Symptomatologia}
\begin{itemize}
\item {Grp. gram.:f.}
\end{itemize}
\begin{itemize}
\item {Proveniência:(Do gr. \textunderscore sumptoma\textunderscore , \textunderscore sumptomatos\textunderscore  + \textunderscore logos\textunderscore )}
\end{itemize}
Parte da Medicina que trata dos symptomas das doenças.
\section{Symptomatológico}
\begin{itemize}
\item {Grp. gram.:adj.}
\end{itemize}
Relativo á symptomatologia.
\section{Symptomatologista}
\begin{itemize}
\item {Grp. gram.:m.  e  adj.}
\end{itemize}
O que se occupa ou que escreve á cêrca de symptomatologia.
\section{Symptomologia}
\textunderscore f.\textunderscore  (e der.)
O mesmo que \textunderscore symptomatologia\textunderscore , etc. Cf. Sousa Martins, \textunderscore Nosographia\textunderscore .
\section{Symptose}
\begin{itemize}
\item {Grp. gram.:f.}
\end{itemize}
\begin{itemize}
\item {Utilização:Des.}
\end{itemize}
\begin{itemize}
\item {Proveniência:(Gr. \textunderscore sumptosis\textunderscore )}
\end{itemize}
Debilidade ou enfraquecimento dos órgãos.
Magreza.
\section{Synadelpho}
\begin{itemize}
\item {Grp. gram.:adj.}
\end{itemize}
\begin{itemize}
\item {Proveniência:(Do gr. \textunderscore sun\textunderscore  + \textunderscore adelphos\textunderscore )}
\end{itemize}
Diz-se do monstro, que tem um só tronco e oito membros.
\section{Synagelástico}
\begin{itemize}
\item {Grp. gram.:adj.}
\end{itemize}
\begin{itemize}
\item {Proveniência:(Do gr. \textunderscore sun\textunderscore  + \textunderscore agelazein\textunderscore )}
\end{itemize}
Que vive em grupos ou bandos.
\section{Synagoga}
\begin{itemize}
\item {Grp. gram.:f.}
\end{itemize}
\begin{itemize}
\item {Utilização:Bras. de Minas}
\end{itemize}
\begin{itemize}
\item {Proveniência:(Do gr. \textunderscore sunagoge\textunderscore )}
\end{itemize}
Assembleia de fiéis, entre os Hebreus.
Templo judaico.
Desarranjo, desordem.
\section{Synalepha}
\begin{itemize}
\item {Grp. gram.:f.}
\end{itemize}
\begin{itemize}
\item {Proveniência:(Gr. \textunderscore sunaloiphe\textunderscore )}
\end{itemize}
Reunião de duas sýllabas numa só, por synérese, crase ou elisão.
Instrumento de encadernador, para doirar filetes nas capas dos livros.
\section{Synalephista}
\begin{itemize}
\item {Grp. gram.:m.}
\end{itemize}
\begin{itemize}
\item {Utilização:Neol.}
\end{itemize}
\begin{itemize}
\item {Proveniência:(De \textunderscore synalepha\textunderscore )}
\end{itemize}
Aquelle que faz synalephas ou que gosta de escrever as palavras, fazendo synalephas inúteis:«\textunderscore p'ra nós\textunderscore »;«\textunderscore a entrada d'avenida\textunderscore »;« \textunderscore José d'Almeida\textunderscore ». Cf. Júl. Ribeiro, \textunderscore Padre Belch.\textunderscore , XII.
\section{Synallagmático}
\begin{itemize}
\item {Grp. gram.:adj.}
\end{itemize}
\begin{itemize}
\item {Utilização:Jur.}
\end{itemize}
\begin{itemize}
\item {Proveniência:(Gr. \textunderscore sunallagmátikos\textunderscore )}
\end{itemize}
Diz-se de um contrato billateral.
\section{Sinandra}
\begin{itemize}
\item {Grp. gram.:f.}
\end{itemize}
\begin{itemize}
\item {Proveniência:(Do gr. \textunderscore sun\textunderscore  + \textunderscore aner\textunderscore )}
\end{itemize}
Gênero de plantas labiadas.
\section{Sinânteas}
\begin{itemize}
\item {Grp. gram.:f. pl.}
\end{itemize}
\begin{itemize}
\item {Utilização:Bot.}
\end{itemize}
\begin{itemize}
\item {Proveniência:(Do gr. \textunderscore sun\textunderscore  + \textunderscore anthos\textunderscore )}
\end{itemize}
Plantas, cujas flôres nascem juntamente com as fôlhas.
\section{Sinantéreas}
\begin{itemize}
\item {Grp. gram.:f. pl.}
\end{itemize}
Família de plantas, o mesmo que \textunderscore compostas\textunderscore .
(Fem. pl. de \textunderscore sinantéreo\textunderscore )
\section{Sinantéreo}
\begin{itemize}
\item {Grp. gram.:adj.}
\end{itemize}
\begin{itemize}
\item {Proveniência:(Do gr. \textunderscore sun\textunderscore  + \textunderscore antheros\textunderscore )}
\end{itemize}
Diz-se das flôres, cujos estames são soldados pelas anteras.
E diz-se dos estames soldados pelas anteras.
\section{Sinantérico}
\begin{itemize}
\item {Grp. gram.:adj.}
\end{itemize}
\begin{itemize}
\item {Utilização:Bot.}
\end{itemize}
Que tem anteras reunidas.
O mesmo que \textunderscore sinantéreo\textunderscore .
\section{Sinanterografia}
\begin{itemize}
\item {Grp. gram.:f.}
\end{itemize}
\begin{itemize}
\item {Proveniência:(Do gr. \textunderscore sun\textunderscore  + \textunderscore antheros\textunderscore  + \textunderscore graphein\textunderscore )}
\end{itemize}
Parte da sinanterologia, na qual se descrevem os gêneros e espécies das sinantéreas.
\section{Sinanterográfico}
\begin{itemize}
\item {Grp. gram.:adj.}
\end{itemize}
Relativo á sinanterografia.
\section{Sinanterógrafo}
\begin{itemize}
\item {Grp. gram.:m.}
\end{itemize}
Botânico, que se dedica ao estudo das sinantéreas.
(Cp. \textunderscore sinanterografia\textunderscore )
\section{Sinanterologia}
\begin{itemize}
\item {Grp. gram.:f.}
\end{itemize}
\begin{itemize}
\item {Proveniência:(Do gr. \textunderscore sun\textunderscore  + \textunderscore antheros\textunderscore  + \textunderscore logos\textunderscore )}
\end{itemize}
Tratado científico das sinantéreas, no qual se faz a analise crítica dos botânicos que se tem ocupado das sinantéreas.
\section{Sinanterológico}
\begin{itemize}
\item {Grp. gram.:adj.}
\end{itemize}
Relativo á sinanterologia.
\section{Sinanteronomia}
\begin{itemize}
\item {Grp. gram.:f.}
\end{itemize}
\begin{itemize}
\item {Proveniência:(Do gr. \textunderscore sun\textunderscore  + \textunderscore antheros\textunderscore  + \textunderscore nomos\textunderscore )}
\end{itemize}
Parte da sinanterologia, em que se trata dos caracteres, organização e funcções das sinantéreas.
\section{Sinanteronómico}
\begin{itemize}
\item {Grp. gram.:adj.}
\end{itemize}
Relativo á sinanteronomia.
\section{Sinanterotécnia}
\begin{itemize}
\item {Grp. gram.:f.}
\end{itemize}
\begin{itemize}
\item {Proveniência:(Do gr. \textunderscore sun\textunderscore  + \textunderscore antheros\textunderscore  + \textunderscore tekhne\textunderscore )}
\end{itemize}
Estudo prático das sinantéreas.
\section{Sinanterotécnico}
\begin{itemize}
\item {Grp. gram.:adj.}
\end{itemize}
Relativo á sinanterotecnia.
\section{Sinantia}
\begin{itemize}
\item {Grp. gram.:f.}
\end{itemize}
\begin{itemize}
\item {Proveniência:(Do gr. \textunderscore sun\textunderscore  + \textunderscore anthos\textunderscore )}
\end{itemize}
Monstruosidade botânica, que consiste na soldadura anómala de duas flôres vizinhas, ou pelos invólucros ou pelos pecíolos.
\section{Sinantocarpado}
\begin{itemize}
\item {Grp. gram.:adj.}
\end{itemize}
\begin{itemize}
\item {Proveniência:(Do gr. \textunderscore sun\textunderscore  + \textunderscore anthos\textunderscore  + \textunderscore karpos\textunderscore )}
\end{itemize}
Diz-se de qualquer fruto, formado pela união de muitos ovários, pertencentes a flôres, primitivamente distintas.
\section{Sinapta}
\begin{itemize}
\item {Grp. gram.:f.}
\end{itemize}
Gênero de equinodermes.
\section{Sináptase}
\begin{itemize}
\item {Grp. gram.:f.}
\end{itemize}
\begin{itemize}
\item {Utilização:Chím.}
\end{itemize}
\begin{itemize}
\item {Proveniência:(Do gr. \textunderscore sun\textunderscore  + \textunderscore aptein\textunderscore )}
\end{itemize}
Fermento, que se desenvolve nas amêndoas amargas, sob a acção da água e do calor brando, e que produz o ácido cianídrico.
\section{Sinapto}
\begin{itemize}
\item {Grp. gram.:m.}
\end{itemize}
\begin{itemize}
\item {Proveniência:(Do gr. \textunderscore sunaptos\textunderscore )}
\end{itemize}
Gênero de insectos coleópteros pentâmeros.
\section{Sinartrose}
\begin{itemize}
\item {Grp. gram.:f.}
\end{itemize}
\begin{itemize}
\item {Utilização:Anat.}
\end{itemize}
\begin{itemize}
\item {Proveniência:(Gr. \textunderscore sunarthrosis\textunderscore )}
\end{itemize}
Articulação, que não permite o movimento dos ossos que liga.
\section{Sinaspisma}
\begin{itemize}
\item {Grp. gram.:f.}
\end{itemize}
\begin{itemize}
\item {Proveniência:(Do gr. \textunderscore sun\textunderscore  + \textunderscore aspis\textunderscore )}
\end{itemize}
Gênero de plantas euforbiáceas.
\section{Sinaspismo}
\begin{itemize}
\item {Grp. gram.:m.}
\end{itemize}
\begin{itemize}
\item {Proveniência:(Do gr. \textunderscore sun\textunderscore  + \textunderscore aspis\textunderscore )}
\end{itemize}
Formatura defensiva da falange grega.
\section{Sinatroísmo}
\begin{itemize}
\item {Grp. gram.:m.}
\end{itemize}
\begin{itemize}
\item {Proveniência:(Gr. \textunderscore sunathroismos\textunderscore )}
\end{itemize}
Figura de Rètórica, com que se acumulam numa frase muitos termos de significação correlativa, isto é, muitos adjectivos, muitos verbos, etc.
\section{Sinaulia}
\begin{itemize}
\item {Grp. gram.:f.}
\end{itemize}
\begin{itemize}
\item {Proveniência:(Gr. \textunderscore sunaulia\textunderscore )}
\end{itemize}
Reunião de instrumentos de sôpro, na música antiga.
\section{Sinaxário}
\begin{itemize}
\item {fónica:csa}
\end{itemize}
\begin{itemize}
\item {Grp. gram.:m.}
\end{itemize}
\begin{itemize}
\item {Utilização:P. us.}
\end{itemize}
\begin{itemize}
\item {Proveniência:(De \textunderscore sinaxe\textunderscore )}
\end{itemize}
Relação dos nomes dos santos.
Calendário.
\section{Sinaxe}
\begin{itemize}
\item {fónica:cse}
\end{itemize}
\begin{itemize}
\item {Grp. gram.:f.}
\end{itemize}
\begin{itemize}
\item {Proveniência:(Lat. \textunderscore synaxis\textunderscore )}
\end{itemize}
Assembleia de Cristãos, nos primeiros tempos do Cristianismo.
\section{Sincalipta}
\begin{itemize}
\item {Grp. gram.:f.}
\end{itemize}
\begin{itemize}
\item {Proveniência:(Do gr. \textunderscore sun\textunderscore  + \textunderscore kalupto\textunderscore )}
\end{itemize}
Gênero de insectos coleópteros pentâmeros.
\section{Sincarpado}
\begin{itemize}
\item {Grp. gram.:adj.}
\end{itemize}
\begin{itemize}
\item {Utilização:Bot.}
\end{itemize}
\begin{itemize}
\item {Proveniência:(De \textunderscore sincarpo\textunderscore )}
\end{itemize}
Diz-se do fruto, que tem muitas carpelas soldadas.
\section{Sincárpia}
\begin{itemize}
\item {Grp. gram.:f.}
\end{itemize}
\begin{itemize}
\item {Proveniência:(Do gr. \textunderscore sun\textunderscore  + \textunderscore karpos\textunderscore )}
\end{itemize}
Gênero de plantas mirtáceas.
\section{Sincarpo}
\begin{itemize}
\item {Grp. gram.:m.}
\end{itemize}
\begin{itemize}
\item {Utilização:Bot.}
\end{itemize}
\begin{itemize}
\item {Proveniência:(Do gr. \textunderscore sun\textunderscore  + \textunderscore karpos\textunderscore )}
\end{itemize}
Fruto, que tem muitos utrículos reunidos.
\section{Sincategorema}
\begin{itemize}
\item {Grp. gram.:m.}
\end{itemize}
\begin{itemize}
\item {Utilização:Gram.}
\end{itemize}
\begin{itemize}
\item {Proveniência:(Lat. \textunderscore syncategorema\textunderscore )}
\end{itemize}
Palavra, que só por si nada significa, como os pronomes indefinidos.
\section{Sincategoremático}
\begin{itemize}
\item {Grp. gram.:adj.}
\end{itemize}
\begin{itemize}
\item {Proveniência:(Do gr. \textunderscore sun\textunderscore  + \textunderscore categorema\textunderscore )}
\end{itemize}
Relativo aos acessórios das categorias, na Lógica.
\section{Sincefalanto}
\begin{itemize}
\item {Grp. gram.:m.}
\end{itemize}
\begin{itemize}
\item {Proveniência:(Do gr. \textunderscore sun\textunderscore  + \textunderscore kephale\textunderscore  + \textunderscore anthos\textunderscore )}
\end{itemize}
Gênero de plantas, da fam. das compostas.
\section{Sincelo}
\begin{itemize}
\item {fónica:cê}
\end{itemize}
\begin{itemize}
\item {Grp. gram.:m.}
\end{itemize}
\begin{itemize}
\item {Proveniência:(Gr. \textunderscore suncellos\textunderscore )}
\end{itemize}
Funcionário que, na antiga Igreja grega, era encarregado de vigiar o procedimento dos Bispos, Patriarcas, etc.
\section{Sincodendro}
\begin{itemize}
\item {Grp. gram.:m.}
\end{itemize}
Gênero de plantas, da fam. das compostas.
\section{Sincolóstomo}
\begin{itemize}
\item {Grp. gram.:m.}
\end{itemize}
Gênero de plantas labiadas.
\section{Sincondrose}
\begin{itemize}
\item {Grp. gram.:f.}
\end{itemize}
\begin{itemize}
\item {Utilização:Anat.}
\end{itemize}
\begin{itemize}
\item {Proveniência:(Do gr. \textunderscore sun\textunderscore  + \textunderscore khondros\textunderscore )}
\end{itemize}
União de dois ossos por meio de cartilagem; articulação.
\section{Sincondrotomia}
\begin{itemize}
\item {Grp. gram.:f.}
\end{itemize}
\begin{itemize}
\item {Utilização:Cir.}
\end{itemize}
\begin{itemize}
\item {Proveniência:(Do gr. \textunderscore sun\textunderscore  + \textunderscore khondros\textunderscore  + \textunderscore tome\textunderscore )}
\end{itemize}
Secção da símphise pubiana.
\section{Sincondrotómico}
\begin{itemize}
\item {Grp. gram.:adj.}
\end{itemize}
Relativo á sincondrotomia.
\section{Sincronicamente}
\begin{itemize}
\item {Grp. gram.:adv.}
\end{itemize}
De modo sincronico; ao mesmo tempo.
\section{Sincrónico}
\begin{itemize}
\item {Grp. gram.:adj.}
\end{itemize}
O mesmo que \textunderscore síncrono\textunderscore .
\section{Sincronismo}
\begin{itemize}
\item {Grp. gram.:m.}
\end{itemize}
\begin{itemize}
\item {Proveniência:(Gr. \textunderscore sunckronismos\textunderscore )}
\end{itemize}
Relação entre factos síncronos.
\section{Sincronista}
\begin{itemize}
\item {Grp. gram.:m. ,  f.  e  adj.}
\end{itemize}
Diz-se de quem emprega método síncrono na exposição dos factos.
\section{Sincronizar}
\begin{itemize}
\item {Grp. gram.:v. t.}
\end{itemize}
\begin{itemize}
\item {Proveniência:(De \textunderscore síncrono\textunderscore )}
\end{itemize}
Descrever sincronicamente.
\section{Síncrono}
\begin{itemize}
\item {Grp. gram.:adj.}
\end{itemize}
\begin{itemize}
\item {Proveniência:(Lat. \textunderscore synchronus\textunderscore )}
\end{itemize}
Que se realiza ou se faz ao mesmo tempo.
Relativo aos factos que succedem ao mesmo tempo ou na mesma época; contemporâneo.
\section{Sincronologia}
\begin{itemize}
\item {Grp. gram.:f.}
\end{itemize}
\begin{itemize}
\item {Proveniência:(Do gr. \textunderscore sunckhronos\textunderscore  + \textunderscore logos\textunderscore )}
\end{itemize}
Tratado de sincronismos.
\section{Sinquilia}
\begin{itemize}
\item {Grp. gram.:f.}
\end{itemize}
\begin{itemize}
\item {Utilização:Med.}
\end{itemize}
\begin{itemize}
\item {Proveniência:(Do gr. \textunderscore sun\textunderscore  + \textunderscore kheilos\textunderscore )}
\end{itemize}
Atresia do orifício buccal, com perda da substância dos lábios, etc.
\section{Synandra}
\begin{itemize}
\item {Grp. gram.:f.}
\end{itemize}
\begin{itemize}
\item {Proveniência:(Do gr. \textunderscore sun\textunderscore  + \textunderscore aner\textunderscore )}
\end{itemize}
Gênero de plantas labiadas.
\section{Synântheas}
\begin{itemize}
\item {Grp. gram.:f. pl.}
\end{itemize}
\begin{itemize}
\item {Utilização:Bot.}
\end{itemize}
\begin{itemize}
\item {Proveniência:(Do gr. \textunderscore sun\textunderscore  + \textunderscore anthos\textunderscore )}
\end{itemize}
Plantas, cujas flôres nascem juntamente com as fôlhas.
\section{Synanthéreas}
\begin{itemize}
\item {Grp. gram.:f. pl.}
\end{itemize}
Família de plantas, o mesmo que \textunderscore compostas\textunderscore .
(Fem. pl. de \textunderscore synanthéreo\textunderscore )
\section{Synanthéreo}
\begin{itemize}
\item {Grp. gram.:adj.}
\end{itemize}
\begin{itemize}
\item {Proveniência:(Do gr. \textunderscore sun\textunderscore  + \textunderscore antheros\textunderscore )}
\end{itemize}
Diz-se das flôres, cujos estames são soldados pelas antheras.
E diz-se dos estames soldados pelas antheras.
\section{Synanthérico}
\begin{itemize}
\item {Grp. gram.:adj.}
\end{itemize}
\begin{itemize}
\item {Utilização:Bot.}
\end{itemize}
Que tem antheras reunidas.
O mesmo que \textunderscore synanthéreo\textunderscore .
\section{Synantherographia}
\begin{itemize}
\item {Grp. gram.:f.}
\end{itemize}
\begin{itemize}
\item {Proveniência:(Do gr. \textunderscore sun\textunderscore  + \textunderscore antheros\textunderscore  + \textunderscore graphein\textunderscore )}
\end{itemize}
Parte da synantherologia, na qual se descrevem os gêneros e espécies das synanthéreas.
\section{Synantherográphico}
\begin{itemize}
\item {Grp. gram.:adj.}
\end{itemize}
Relativo á synantherographia.
\section{Synantherógrapho}
\begin{itemize}
\item {Grp. gram.:m.}
\end{itemize}
Botânico, que se dedica ao estudo das synanthéreas.
(Cp. \textunderscore synantherographia\textunderscore )
\section{Synantherologia}
\begin{itemize}
\item {Grp. gram.:f.}
\end{itemize}
\begin{itemize}
\item {Proveniência:(Do gr. \textunderscore sun\textunderscore  + \textunderscore antheros\textunderscore  + \textunderscore logos\textunderscore )}
\end{itemize}
Tratado scientífico das synanthéreas, no qual se faz a analyse crítica dos botânicos que se tem occupado das synanthéreas.
\section{Synantherológico}
\begin{itemize}
\item {Grp. gram.:adj.}
\end{itemize}
Relativo á synantherologia.
\section{Synantheronomia}
\begin{itemize}
\item {Grp. gram.:f.}
\end{itemize}
\begin{itemize}
\item {Proveniência:(Do gr. \textunderscore sun\textunderscore  + \textunderscore antheros\textunderscore  + \textunderscore nomos\textunderscore )}
\end{itemize}
Parte da synantherologia, em que se trata dos caracteres, organização e funcções das synanthéreas.
\section{Synantheronómico}
\begin{itemize}
\item {Grp. gram.:adj.}
\end{itemize}
Relativo á synantheronomia.
\section{Synantherotéchnia}
\begin{itemize}
\item {Grp. gram.:f.}
\end{itemize}
\begin{itemize}
\item {Proveniência:(Do gr. \textunderscore sun\textunderscore  + \textunderscore antheros\textunderscore  + \textunderscore tekhne\textunderscore )}
\end{itemize}
Estudo prático das synanthéreas.
\section{Synantherotéchnico}
\begin{itemize}
\item {Grp. gram.:adj.}
\end{itemize}
Relativo á synantherotechnia.
\section{Synanthia}
\begin{itemize}
\item {Grp. gram.:f.}
\end{itemize}
\begin{itemize}
\item {Proveniência:(Do gr. \textunderscore sun\textunderscore  + \textunderscore anthos\textunderscore )}
\end{itemize}
Monstruosidade botânica, que consiste na soldadura anómala de duas flôres vizinhas, ou pelos invólucros ou pelos pecíolos.
\section{Synanthocarpado}
\begin{itemize}
\item {Grp. gram.:adj.}
\end{itemize}
\begin{itemize}
\item {Proveniência:(Do gr. \textunderscore sun\textunderscore  + \textunderscore anthos\textunderscore  + \textunderscore karpos\textunderscore )}
\end{itemize}
Diz-se de qualquer fruto, formado pela união de muitos ovários, pertencentes a flôres, primitivamente distintas.
\section{Synapta}
\begin{itemize}
\item {Grp. gram.:f.}
\end{itemize}
Gênero de echinodermes.
\section{Synáptase}
\begin{itemize}
\item {Grp. gram.:f.}
\end{itemize}
\begin{itemize}
\item {Utilização:Chím.}
\end{itemize}
\begin{itemize}
\item {Proveniência:(Do gr. \textunderscore sun\textunderscore  + \textunderscore aptein\textunderscore )}
\end{itemize}
Fermento, que se desenvolve nas amêndoas amargas, sob a acção da água e do calor brando, e que produz o ácido cyanhýdrico.
\section{Synapto}
\begin{itemize}
\item {Grp. gram.:m.}
\end{itemize}
\begin{itemize}
\item {Proveniência:(Do gr. \textunderscore sunaptos\textunderscore )}
\end{itemize}
Gênero de insectos coleópteros pentâmeros.
\section{Synarthrose}
\begin{itemize}
\item {Grp. gram.:f.}
\end{itemize}
\begin{itemize}
\item {Utilização:Anat.}
\end{itemize}
\begin{itemize}
\item {Proveniência:(Gr. \textunderscore sunarthrosis\textunderscore )}
\end{itemize}
Articulação, que não permitte o movimento dos ossos que liga.
\section{Synaspisma}
\begin{itemize}
\item {Grp. gram.:f.}
\end{itemize}
\begin{itemize}
\item {Proveniência:(Do gr. \textunderscore sun\textunderscore  + \textunderscore aspis\textunderscore )}
\end{itemize}
Gênero de plantas euphorbiáceas.
\section{Synaspismo}
\begin{itemize}
\item {Grp. gram.:m.}
\end{itemize}
\begin{itemize}
\item {Proveniência:(Do gr. \textunderscore sun\textunderscore  + \textunderscore aspis\textunderscore )}
\end{itemize}
Formatura defensiva da phalange grega.
\section{Synathroísmo}
\begin{itemize}
\item {Grp. gram.:m.}
\end{itemize}
\begin{itemize}
\item {Proveniência:(Gr. \textunderscore sunathroismos\textunderscore )}
\end{itemize}
Figura de Rhètórica, com que se accumulam numa phrase muitos termos de significação correlativa, isto é, muitos adjectivos, muitos verbos, etc.
\section{Synaulia}
\begin{itemize}
\item {Grp. gram.:f.}
\end{itemize}
\begin{itemize}
\item {Proveniência:(Gr. \textunderscore sunaulia\textunderscore )}
\end{itemize}
Reunião de instrumentos de sôpro, na música antiga.
\section{Synaxário}
\begin{itemize}
\item {fónica:csa}
\end{itemize}
\begin{itemize}
\item {Grp. gram.:m.}
\end{itemize}
\begin{itemize}
\item {Utilização:P. us.}
\end{itemize}
\begin{itemize}
\item {Proveniência:(De \textunderscore synaxe\textunderscore )}
\end{itemize}
Relação dos nomes dos santos.
Calendário.
\section{Synaxe}
\begin{itemize}
\item {fónica:cse}
\end{itemize}
\begin{itemize}
\item {Grp. gram.:f.}
\end{itemize}
\begin{itemize}
\item {Proveniência:(Lat. \textunderscore synaxis\textunderscore )}
\end{itemize}
Assembleia de Christãos, nos primeiros tempos do Christianismo.
\section{Syncalypta}
\begin{itemize}
\item {Grp. gram.:f.}
\end{itemize}
\begin{itemize}
\item {Proveniência:(Do gr. \textunderscore sun\textunderscore  + \textunderscore kalupto\textunderscore )}
\end{itemize}
Gênero de insectos coleópteros pentâmeros.
\section{Syncarpado}
\begin{itemize}
\item {Grp. gram.:adj.}
\end{itemize}
\begin{itemize}
\item {Utilização:Bot.}
\end{itemize}
\begin{itemize}
\item {Proveniência:(De \textunderscore syncarpo\textunderscore )}
\end{itemize}
Diz-se do fruto, que tem muitas carpellas soldadas.
\section{Syncárpia}
\begin{itemize}
\item {Grp. gram.:f.}
\end{itemize}
\begin{itemize}
\item {Proveniência:(Do gr. \textunderscore sun\textunderscore  + \textunderscore karpos\textunderscore )}
\end{itemize}
Gênero de plantas myrtáceas.
\section{Syncarpo}
\begin{itemize}
\item {Grp. gram.:m.}
\end{itemize}
\begin{itemize}
\item {Utilização:Bot.}
\end{itemize}
\begin{itemize}
\item {Proveniência:(Do gr. \textunderscore sun\textunderscore  + \textunderscore karpos\textunderscore )}
\end{itemize}
Fruto, que tem muitos utrículos reunidos.
\section{Syncategorema}
\begin{itemize}
\item {Grp. gram.:m.}
\end{itemize}
\begin{itemize}
\item {Utilização:Gram.}
\end{itemize}
\begin{itemize}
\item {Proveniência:(Lat. \textunderscore syncategorema\textunderscore )}
\end{itemize}
Palavra, que só por si nada significa, como os pronomes indefinidos.
\section{Syncategoremático}
\begin{itemize}
\item {Grp. gram.:adj.}
\end{itemize}
\begin{itemize}
\item {Proveniência:(Do gr. \textunderscore sun\textunderscore  + \textunderscore categorema\textunderscore )}
\end{itemize}
Relativo aos accessórios das categorias, na Lógica.
\section{Syncello}
\begin{itemize}
\item {fónica:cê}
\end{itemize}
\begin{itemize}
\item {Grp. gram.:m.}
\end{itemize}
\begin{itemize}
\item {Proveniência:(Gr. \textunderscore suncellos\textunderscore )}
\end{itemize}
Funccionário que, na antiga Igreja grega, era encarregado de vigiar o procedimento dos Bispos, Patriarchas, etc.
\section{Syncephalantho}
\begin{itemize}
\item {Grp. gram.:m.}
\end{itemize}
\begin{itemize}
\item {Proveniência:(Do gr. \textunderscore sun\textunderscore  + \textunderscore kephale\textunderscore  + \textunderscore anthos\textunderscore )}
\end{itemize}
Gênero de plantas, da fam. das compostas.
\section{Syncéphalo}
\begin{itemize}
\item {Grp. gram.:m.}
\end{itemize}
\begin{itemize}
\item {Proveniência:(Do gr. \textunderscore sun\textunderscore  + \textunderscore kephale\textunderscore )}
\end{itemize}
Gênero de plantas, da fam. das compostas.
\section{Synchilia}
\begin{itemize}
\item {fónica:qui}
\end{itemize}
\begin{itemize}
\item {Grp. gram.:f.}
\end{itemize}
\begin{itemize}
\item {Utilização:Med.}
\end{itemize}
\begin{itemize}
\item {Proveniência:(Do gr. \textunderscore sun\textunderscore  + \textunderscore kheilos\textunderscore )}
\end{itemize}
Atresia do orifício buccal, com perda da substância dos lábios, etc.
\section{Synchodendro}
\begin{itemize}
\item {fónica:co}
\end{itemize}
\begin{itemize}
\item {Grp. gram.:m.}
\end{itemize}
Gênero de plantas, da fam. das compostas.
\section{Syncholóstomo}
\begin{itemize}
\item {fónica:co}
\end{itemize}
\begin{itemize}
\item {Grp. gram.:m.}
\end{itemize}
Gênero de plantas labiadas.
\section{Synchondrose}
\begin{itemize}
\item {fónica:con}
\end{itemize}
\begin{itemize}
\item {Grp. gram.:f.}
\end{itemize}
\begin{itemize}
\item {Utilização:Anat.}
\end{itemize}
\begin{itemize}
\item {Proveniência:(Do gr. \textunderscore sun\textunderscore  + \textunderscore khondros\textunderscore )}
\end{itemize}
União de dois ossos por meio de cartillagem; articulação.
\section{Synchondrotomia}
\begin{itemize}
\item {fónica:con}
\end{itemize}
\begin{itemize}
\item {Grp. gram.:f.}
\end{itemize}
\begin{itemize}
\item {Utilização:Cir.}
\end{itemize}
\begin{itemize}
\item {Proveniência:(Do gr. \textunderscore sun\textunderscore  + \textunderscore khondros\textunderscore  + \textunderscore tome\textunderscore )}
\end{itemize}
Secção da sýmphise pubiana.
\section{Synchondrotómico}
\begin{itemize}
\item {fónica:con}
\end{itemize}
\begin{itemize}
\item {Grp. gram.:adj.}
\end{itemize}
Relativo á synchondrotomia.
\section{Synchronicamente}
\begin{itemize}
\item {Grp. gram.:adv.}
\end{itemize}
De modo synchronico; ao mesmo tempo.
\section{Synchrónico}
\begin{itemize}
\item {Grp. gram.:adj.}
\end{itemize}
O mesmo que \textunderscore sýnchrono\textunderscore .
\section{Synchronismo}
\begin{itemize}
\item {Grp. gram.:m.}
\end{itemize}
\begin{itemize}
\item {Proveniência:(Gr. \textunderscore sunckronismos\textunderscore )}
\end{itemize}
Relação entre factos sýnchronos.
\section{Synchronista}
\begin{itemize}
\item {Grp. gram.:m. ,  f.  e  adj.}
\end{itemize}
Diz-se de quem emprega méthodo sýnchrono na exposição dos factos.
\section{Synchronizar}
\begin{itemize}
\item {Grp. gram.:v. t.}
\end{itemize}
\begin{itemize}
\item {Proveniência:(De \textunderscore sýnchrono\textunderscore )}
\end{itemize}
Descrever synchronicamente.
\section{Sýnchrono}
\begin{itemize}
\item {Grp. gram.:adj.}
\end{itemize}
\begin{itemize}
\item {Proveniência:(Lat. \textunderscore synchronus\textunderscore )}
\end{itemize}
Que se realiza ou se faz ao mesmo tempo.
Relativo aos factos que succedem ao mesmo tempo ou na mesma época; contemporâneo.
\section{Synchronologia}
\begin{itemize}
\item {Grp. gram.:f.}
\end{itemize}
\begin{itemize}
\item {Proveniência:(Do gr. \textunderscore sunckhronos\textunderscore  + \textunderscore logos\textunderscore )}
\end{itemize}
Tratado de synchronismos.
\section{Sincraniano}
\begin{itemize}
\item {Grp. gram.:adj.}
\end{itemize}
\begin{itemize}
\item {Proveniência:(De \textunderscore sin\textunderscore  + \textunderscore craniano\textunderscore )}
\end{itemize}
Diz-se da maxila superior, por estar ligada ao crânio.
\section{Sincrético}
\begin{itemize}
\item {Grp. gram.:adj.}
\end{itemize}
Relativo ao sincretismo.
\section{Sincretismo}
\begin{itemize}
\item {Grp. gram.:m.}
\end{itemize}
\begin{itemize}
\item {Proveniência:(Gr. \textunderscore sugkretismos\textunderscore )}
\end{itemize}
Sistema filosófico, que combinava os princípios de diversos sistemas.
Eclectismo.
Amálgama de concepções heterogêneas.
\section{Sincretista}
\begin{itemize}
\item {Grp. gram.:adj.}
\end{itemize}
\begin{itemize}
\item {Grp. gram.:M. ,  f.  e  adj.}
\end{itemize}
O mesmo que \textunderscore sincrético\textunderscore .
Diz-se da pessôa partidária do sincretismo.
(Cp. \textunderscore sincretismo\textunderscore )
\section{Síncrise}
\begin{itemize}
\item {Grp. gram.:f.}
\end{itemize}
\begin{itemize}
\item {Utilização:Chím.}
\end{itemize}
\begin{itemize}
\item {Utilização:Ant.}
\end{itemize}
\begin{itemize}
\item {Proveniência:(Lat. \textunderscore syncrisis\textunderscore )}
\end{itemize}
Opposição, o mesmo que \textunderscore antitese\textunderscore .
Reunião de duas vogaes num ditongo.
Coagulação de líquidos misturados.
\section{Sincrita}
\begin{itemize}
\item {Grp. gram.:f.}
\end{itemize}
\begin{itemize}
\item {Utilização:Zool.}
\end{itemize}
Gênero de infusórios.
\section{Sincrítico}
\begin{itemize}
\item {Grp. gram.:adj.}
\end{itemize}
\begin{itemize}
\item {Utilização:Med.}
\end{itemize}
Relativo á \textunderscore síncrise\textunderscore .
O mesmo que \textunderscore adstringente\textunderscore .
\section{Sincronológico}
\begin{itemize}
\item {Grp. gram.:adj.}
\end{itemize}
Relativo á sincronologia.
\section{Sindáctilo}
\begin{itemize}
\item {Grp. gram.:adj.}
\end{itemize}
\begin{itemize}
\item {Grp. gram.:M. pl.}
\end{itemize}
\begin{itemize}
\item {Proveniência:(Do gr. \textunderscore sun\textunderscore  + \textunderscore daktulus\textunderscore )}
\end{itemize}
Que tem os dedos reunidos.
Família de aves sindáctilas.
Família de mamíferos sindáctilos.
\section{Sínquise}
\begin{itemize}
\item {Grp. gram.:f.}
\end{itemize}
\begin{itemize}
\item {Utilização:Gram.}
\end{itemize}
\begin{itemize}
\item {Proveniência:(Lat. \textunderscore synchysis\textunderscore )}
\end{itemize}
Inversão da ordem natural das palavras, tornando a frase obscura.
Hipérbato exagerado.
\section{Sindapso}
\begin{itemize}
\item {Grp. gram.:m.}
\end{itemize}
(?):«\textunderscore ...do letargo em que a (língua) puseram balofos bíltris, mazorraes sindapsos\textunderscore ». Filinto, I, 50.--Invenção enigmática, talvez relacionada com o gr. \textunderscore sun\textunderscore  e o lat. \textunderscore daps\textunderscore , designando quem se banqueteia lautamente com outros. O enigma é confessado pelo autor, pois diz, em nota: \textunderscore quis potest capere capiat\textunderscore .
Diversão mal empregada.
\section{Sindectomia}
\begin{itemize}
\item {Grp. gram.:f.}
\end{itemize}
\begin{itemize}
\item {Utilização:Cir.}
\end{itemize}
\begin{itemize}
\item {Proveniência:(Do gr. \textunderscore sundes\textunderscore  + \textunderscore ektome\textunderscore )}
\end{itemize}
Excisão da conjunctiva.
\section{Sindesmografia}
\begin{itemize}
\item {Grp. gram.:f.}
\end{itemize}
\begin{itemize}
\item {Proveniência:(De \textunderscore sindesmógrafo\textunderscore )}
\end{itemize}
Descripção dos ligamentos, em Anatomia.
\section{Sindesmográfico}
\begin{itemize}
\item {Grp. gram.:adj.}
\end{itemize}
Relativo á sindesmografia.
\section{Sindesmógrafo}
\begin{itemize}
\item {Grp. gram.:m.}
\end{itemize}
\begin{itemize}
\item {Proveniência:(Do gr. \textunderscore sundesmos\textunderscore  + \textunderscore graphein\textunderscore )}
\end{itemize}
Aquele que se ocupa de sindesmografia.
\section{Sindesmologia}
\begin{itemize}
\item {Grp. gram.:f.}
\end{itemize}
O mesmo que \textunderscore sindesmografia\textunderscore .
\section{Sindesmólogo}
\begin{itemize}
\item {Grp. gram.:m.}
\end{itemize}
Tratadista da sindesmologia.
\section{Sindesmose}
\begin{itemize}
\item {Grp. gram.:f.}
\end{itemize}
\begin{itemize}
\item {Utilização:Anat.}
\end{itemize}
\begin{itemize}
\item {Proveniência:(Do gr. \textunderscore sundesmos\textunderscore )}
\end{itemize}
Reunião de ossos por meio de ligamentos.
\section{Sindesmotomia}
\begin{itemize}
\item {Grp. gram.:f.}
\end{itemize}
\begin{itemize}
\item {Utilização:Anat.}
\end{itemize}
\begin{itemize}
\item {Proveniência:(Do gr. \textunderscore sundesmos\textunderscore  + \textunderscore tome\textunderscore )}
\end{itemize}
Dissecação de ligamentos.
\section{Sindesmotómico}
\begin{itemize}
\item {Grp. gram.:adj.}
\end{itemize}
Relativo á sindesmotomia.
\section{Sindicação}
\begin{itemize}
\item {Grp. gram.:f.}
\end{itemize}
Acto ou efeito de sindicar.
\section{Sindicado}
\begin{itemize}
\item {Grp. gram.:adj.}
\end{itemize}
\begin{itemize}
\item {Grp. gram.:M.}
\end{itemize}
\begin{itemize}
\item {Proveniência:(De \textunderscore sindicar\textunderscore )}
\end{itemize}
Que foi objecto de sindicância.
Que se inquiriu ou se investigou.
Indivíduo sindicado.
Cargo ou funções de síndico.
\section{Sindicador}
\begin{itemize}
\item {Grp. gram.:m.}
\end{itemize}
Aquele que sindica; sindicante.
\section{Sindical}
\begin{itemize}
\item {Grp. gram.:adj.}
\end{itemize}
\begin{itemize}
\item {Utilização:Neol.}
\end{itemize}
Relativo a sindicato: \textunderscore casa sindical\textunderscore .
\section{Sindicalismo}
\begin{itemize}
\item {Grp. gram.:m.}
\end{itemize}
\begin{itemize}
\item {Proveniência:(De \textunderscore sindical\textunderscore )}
\end{itemize}
Teoria das doutrinas sôbre sindicatos.
Defesa dos sindicatos.
\section{Sindicalista}
\begin{itemize}
\item {Grp. gram.:adj.}
\end{itemize}
\begin{itemize}
\item {Grp. gram.:M.}
\end{itemize}
Relativo a sindicato: \textunderscore reivindicações sindicalistas\textunderscore .
Partidário ou defensor do sindicalismo ou de certo sindicato.
\section{Sindicância}
\begin{itemize}
\item {Grp. gram.:f.}
\end{itemize}
\begin{itemize}
\item {Proveniência:(De \textunderscore sindicar\textunderscore )}
\end{itemize}
O mesmo que \textunderscore sindicação\textunderscore .
Inquérito.
\section{Sindicante}
\begin{itemize}
\item {Grp. gram.:m.  e  adj.}
\end{itemize}
O que sindica.
\section{Sindicar}
\begin{itemize}
\item {Grp. gram.:v. t.  e  i.}
\end{itemize}
\begin{itemize}
\item {Proveniência:(De \textunderscore síndico\textunderscore )}
\end{itemize}
Tomar informações de alguma coisa; inquirir.
\section{Sindicatal}
\begin{itemize}
\item {Grp. gram.:adj.}
\end{itemize}
\begin{itemize}
\item {Utilização:bras}
\end{itemize}
\begin{itemize}
\item {Utilização:Neol.}
\end{itemize}
Relativo a sindicato.
O mesmo ou melhor que \textunderscore sindical\textunderscore .
\section{Sindicateiro}
\begin{itemize}
\item {Grp. gram.:m.  e  adj.}
\end{itemize}
\begin{itemize}
\item {Utilização:Deprec.}
\end{itemize}
Indivíduo, que faz parte, ou gosta de fazer parte, de sindicatos financeiros.
\section{Sindicato}
\begin{itemize}
\item {Grp. gram.:m.}
\end{itemize}
\begin{itemize}
\item {Grp. gram.:M.}
\end{itemize}
\begin{itemize}
\item {Utilização:Deprec.}
\end{itemize}
\begin{itemize}
\item {Proveniência:(De \textunderscore sindicar\textunderscore )}
\end{itemize}
O mesmo que \textunderscore sindicado\textunderscore .
Companhia ou associação de capitalistas, interessados na mesma empresa e pondo em commum os seus títulos, para que na venda dêstes não haja alteração de preço.
Especulação financeira pouco lícita.
Associação dos indivíduos de uma classe, para defesa dos seus interesses económicos: \textunderscore sindicato agrícola\textunderscore ; \textunderscore sindicato operário\textunderscore .
\section{Sindicatório}
\begin{itemize}
\item {Grp. gram.:adj.}
\end{itemize}
\begin{itemize}
\item {Grp. gram.:M.}
\end{itemize}
Relativo a sindicato.
Membro de um sindicato financeiro.
\section{Sindicatura}
\begin{itemize}
\item {Grp. gram.:f.}
\end{itemize}
\begin{itemize}
\item {Proveniência:(De \textunderscore sindicato\textunderscore )}
\end{itemize}
Emprêgo ou funções de síndico.
\section{Síndico}
\begin{itemize}
\item {Grp. gram.:m.}
\end{itemize}
\begin{itemize}
\item {Proveniência:(Gr. \textunderscore sundikos\textunderscore )}
\end{itemize}
Antigo procurador de uma comunidade, côrtes, etc.
Advogado de corporação administrativa: \textunderscore fui síndico da Câmara Municipal de Alcácer\textunderscore .
Aquele que é encarregado de uma sindicância.
O que é escolhido para zelar ou defender os interesses de uma associação ou de uma classe.
\section{Síndroma}
\begin{itemize}
\item {Grp. gram.:m.}
\end{itemize}
\begin{itemize}
\item {Proveniência:(Gr. \textunderscore sundrome\textunderscore )}
\end{itemize}
Designação antiga dos sintomas mórbidos, sem referência a determinada doença.
\section{Síndrome}
\begin{itemize}
\item {Grp. gram.:m.}
\end{itemize}
O mesmo ou melhor que \textunderscore síndroma\textunderscore .
\section{Sinédoque}
\begin{itemize}
\item {Grp. gram.:f.}
\end{itemize}
\begin{itemize}
\item {Utilização:Gram.}
\end{itemize}
\begin{itemize}
\item {Proveniência:(Lat. \textunderscore synedoche\textunderscore )}
\end{itemize}
Figura, em que se toma o gênero pela espécie, a espécie pelo genero, o todo pela parte, a parte pelo todo.
\section{Sinedrela}
\begin{itemize}
\item {Grp. gram.:f.}
\end{itemize}
Gênero de plantas, da fam. das compostas.
\section{Sinedrim}
\begin{itemize}
\item {Grp. gram.:m.}
\end{itemize}
O mesmo que \textunderscore sinédrio\textunderscore .
\section{Sinédrio}
\begin{itemize}
\item {Grp. gram.:m.}
\end{itemize}
\begin{itemize}
\item {Utilização:Ext.}
\end{itemize}
\begin{itemize}
\item {Proveniência:(Gr. \textunderscore sunedrion\textunderscore )}
\end{itemize}
Supremo Conselho, entre os Judeus.
Assembleia.
\section{Sinelcosciádio}
\begin{itemize}
\item {Grp. gram.:m.}
\end{itemize}
Gênero de plantas umbelíferas.
\section{Sinema}
\begin{itemize}
\item {Grp. gram.:m.}
\end{itemize}
\begin{itemize}
\item {Utilização:Bot.}
\end{itemize}
\begin{itemize}
\item {Proveniência:(Do gr. \textunderscore sun\textunderscore  + \textunderscore nema\textunderscore )}
\end{itemize}
Parte da columna das orquídeas, que representa os filetes dos estames.
\section{Sinemático}
\begin{itemize}
\item {Grp. gram.:adj.}
\end{itemize}
\begin{itemize}
\item {Utilização:Bot.}
\end{itemize}
Relativo aos estames.
Que fórma, ou concorre para formar, os estames.
\section{Sinemúria}
\begin{itemize}
\item {Grp. gram.:f.}
\end{itemize}
Gênero de moluscos acéfalos.
\section{Sinequia}
\begin{itemize}
\item {Grp. gram.:f.}
\end{itemize}
\begin{itemize}
\item {Utilização:Med.}
\end{itemize}
\begin{itemize}
\item {Proveniência:(Do gr. \textunderscore sunekheia\textunderscore )}
\end{itemize}
Aderência da íris.
\section{Sinérese}
\begin{itemize}
\item {Grp. gram.:f.}
\end{itemize}
\begin{itemize}
\item {Utilização:Gram.}
\end{itemize}
\begin{itemize}
\item {Proveniência:(Lat. \textunderscore synaeresis\textunderscore )}
\end{itemize}
Contracção de duas sílabas numa, mas sem alteração de letras nem de sons.
\section{Sinergia}
\begin{itemize}
\item {Grp. gram.:f.}
\end{itemize}
\begin{itemize}
\item {Proveniência:(Gr. \textunderscore sunergeia\textunderscore )}
\end{itemize}
Acto ou esfôrço simultâneo de diversos órgãos ou músculos.
\section{Sinérgico}
\begin{itemize}
\item {Grp. gram.:adj.}
\end{itemize}
Relativo á sinergia.
\section{Sinérgide}
\begin{itemize}
\item {Grp. gram.:f.}
\end{itemize}
\begin{itemize}
\item {Utilização:Bot.}
\end{itemize}
\begin{itemize}
\item {Proveniência:(Do gr. \textunderscore sunergos\textunderscore )}
\end{itemize}
Cada uma das duas massas protoplásmicas, que se agrupam em tôrno de dois núcleos lateraes no óvulo.
\section{Sínese}
\begin{itemize}
\item {Grp. gram.:f.}
\end{itemize}
\begin{itemize}
\item {Utilização:Gram.}
\end{itemize}
\begin{itemize}
\item {Proveniência:(Lat. \textunderscore synesis\textunderscore )}
\end{itemize}
Construcção sintáctica, em que se attende mais ao sentido do que ao rigor da fórma.
\section{Sinestesia}
\begin{itemize}
\item {Grp. gram.:f.}
\end{itemize}
\begin{itemize}
\item {Utilização:Med.}
\end{itemize}
\begin{itemize}
\item {Proveniência:(Do gr. \textunderscore sun\textunderscore  + \textunderscore asthesia\textunderscore )}
\end{itemize}
Certa perturbação na percepção das sensações.
\section{Sineta}
\begin{itemize}
\item {Grp. gram.:f.}
\end{itemize}
\begin{itemize}
\item {Proveniência:(Do gr. \textunderscore sunethes\textunderscore )}
\end{itemize}
Gênero de insectos coleópteros.
\section{Singênese}
\begin{itemize}
\item {Grp. gram.:adj.}
\end{itemize}
\begin{itemize}
\item {Grp. gram.:F.}
\end{itemize}
\begin{itemize}
\item {Proveniência:(Do gr. \textunderscore sun\textunderscore  + \textunderscore genesis\textunderscore )}
\end{itemize}
O mesmo que \textunderscore sinantéreo\textunderscore .
Hipótese dos que admitem a criação simultânea de todos os seres vivos.
\section{Singenesia}
\begin{itemize}
\item {Grp. gram.:f.}
\end{itemize}
\begin{itemize}
\item {Proveniência:(De \textunderscore singênese\textunderscore )}
\end{itemize}
Conjunto das plantas, cujas flôres ou estames são ligados pelas anteras, e que constituem uma classe no sistema de Linneu.
\section{Singenésico}
\begin{itemize}
\item {Grp. gram.:adj.}
\end{itemize}
Relativo á singênese e á singenesia.
\section{Singenesista}
\begin{itemize}
\item {Grp. gram.:m. ,  f.  e  adj.}
\end{itemize}
Diz-se da pessôa partidária da singênese.
\section{Sineurose}
\begin{itemize}
\item {Grp. gram.:f.}
\end{itemize}
\begin{itemize}
\item {Proveniência:(Gr. \textunderscore sunneurosis\textunderscore )}
\end{itemize}
Ligação de dois ossos.
\section{Singenista}
\begin{itemize}
\item {Grp. gram.:m. ,  f.  e  adj.}
\end{itemize}
(V.singenesista)
\section{Singnátidas}
\begin{itemize}
\item {Grp. gram.:f. pl.}
\end{itemize}
Gênero de peixes, o mesmo que \textunderscore síngnatos\textunderscore .
\section{Síngnatos}
\begin{itemize}
\item {Grp. gram.:m. pl.}
\end{itemize}
\begin{itemize}
\item {Proveniência:(Do gr. \textunderscore sun\textunderscore  + \textunderscore gnathos\textunderscore )}
\end{itemize}
Gênero de peixes, de barbatanas cinzentas, sem língua nem dentes.
\section{Síngrafa}
\begin{itemize}
\item {Grp. gram.:f.}
\end{itemize}
\begin{itemize}
\item {Proveniência:(Lat. \textunderscore syngrapha\textunderscore )}
\end{itemize}
O mesmo ou melhór que \textunderscore síngrafo\textunderscore .
\section{Singráfico}
\begin{itemize}
\item {Grp. gram.:adj.}
\end{itemize}
Relativo ao síngrafo.
\section{Síngrafo}
\begin{itemize}
\item {Grp. gram.:m.}
\end{itemize}
\begin{itemize}
\item {Proveniência:(Do gr. \textunderscore sun\textunderscore  + \textunderscore graphein\textunderscore )}
\end{itemize}
Documento de dívida, assignado pelo credor e pelo devedor.
\section{Sinhedrim}
\begin{itemize}
\item {fónica:ne}
\end{itemize}
\begin{itemize}
\item {Grp. gram.:m.}
\end{itemize}
(V.sinedrim)
\section{Sinhédrio}
\begin{itemize}
\item {fónica:né}
\end{itemize}
\begin{itemize}
\item {Grp. gram.:m.}
\end{itemize}
(V.sinédrio)
\section{Sinistrato}
\begin{itemize}
\item {Grp. gram.:adj.}
\end{itemize}
\begin{itemize}
\item {Utilização:Zool.}
\end{itemize}
\begin{itemize}
\item {Grp. gram.:M. pl.}
\end{itemize}
Diz-se do insecto, cujas queixadas são reunidas pela base ao lábio inferior.
Classe de insectos, que compreende a maior parte dos neurópteros e alguns ápteros.
\section{Sinizese}
\begin{itemize}
\item {Grp. gram.:f.}
\end{itemize}
\begin{itemize}
\item {Utilização:Gram.}
\end{itemize}
\begin{itemize}
\item {Utilização:Cir.}
\end{itemize}
\begin{itemize}
\item {Proveniência:(Lat. \textunderscore synizesis\textunderscore )}
\end{itemize}
Pronúncia de duas vogaes distintas em um só tempo prosódico, sem formar ditongo.
Oclusão da pupila, em consequência de uma inflamação.
\section{Sinoquite}
\begin{itemize}
\item {Grp. gram.:f.}
\end{itemize}
\begin{itemize}
\item {Proveniência:(Lat. \textunderscore synochitis\textunderscore )}
\end{itemize}
Pedra preciosa, mencionada por Plínio e hoje desconhecida.
\section{Sínoco}
\begin{itemize}
\item {Grp. gram.:adj.}
\end{itemize}
\begin{itemize}
\item {Utilização:Med.}
\end{itemize}
\begin{itemize}
\item {Proveniência:(Gr. \textunderscore sunokhos\textunderscore )}
\end{itemize}
Inflamatório.
\section{Sinodal}
\begin{itemize}
\item {Grp. gram.:adj.}
\end{itemize}
\begin{itemize}
\item {Proveniência:(Lat. \textunderscore synodalis\textunderscore )}
\end{itemize}
Relativo ao sínodo.
\section{Sinodático}
\begin{itemize}
\item {Grp. gram.:adj.}
\end{itemize}
Que se realiza num sínodo.
\section{Sinodendro}
\begin{itemize}
\item {Grp. gram.:m.}
\end{itemize}
Insecto coleóptero, que vive nas águas.
\section{Sinodicamente}
\begin{itemize}
\item {Grp. gram.:adv.}
\end{itemize}
De modo sinótico; em sínodo.
\section{Sinódico}
\begin{itemize}
\item {Grp. gram.:adj.}
\end{itemize}
\begin{itemize}
\item {Utilização:Astron.}
\end{itemize}
\begin{itemize}
\item {Grp. gram.:M.}
\end{itemize}
\begin{itemize}
\item {Proveniência:(Lat. \textunderscore synodicus\textunderscore )}
\end{itemize}
O mesmo que \textunderscore sinodal\textunderscore .
Relativo á revolução dos planetas.
Colecção de resoluções sinodaes.
\section{Sínodo}
\begin{itemize}
\item {Grp. gram.:m.}
\end{itemize}
\begin{itemize}
\item {Proveniência:(Lat. \textunderscore synodus\textunderscore )}
\end{itemize}
Assembleia de párochos e de outros padres, convocada por ordem do seu prelado ou de outro superior.
\section{Sinodonte}
\begin{itemize}
\item {Grp. gram.:m.}
\end{itemize}
\begin{itemize}
\item {Proveniência:(Do gr. \textunderscore sun\textunderscore  + \textunderscore odous\textunderscore )}
\end{itemize}
Gênero de peixes malacopterígios.
\section{Sinóico}
\begin{itemize}
\item {Grp. gram.:m.}
\end{itemize}
\begin{itemize}
\item {Proveniência:(Do gr. \textunderscore sunoikes\textunderscore )}
\end{itemize}
Gênero de moluscos de Spitzberg.
\section{Sinonímia}
\begin{itemize}
\item {Grp. gram.:f.}
\end{itemize}
\begin{itemize}
\item {Proveniência:(Lat. \textunderscore synonymia\textunderscore )}
\end{itemize}
Qualidade do que é sinónimo.
Acto de exprimir a mesma coisa por palavras sinónimas.
\section{Sinonímica}
\begin{itemize}
\item {Grp. gram.:f.}
\end{itemize}
\begin{itemize}
\item {Proveniência:(De \textunderscore sinonímico\textunderscore )}
\end{itemize}
Arte ou estudo dos sinónimos e sua distincção.
\section{Sinonimicamente}
\begin{itemize}
\item {Grp. gram.:adv.}
\end{itemize}
De modo sinonímico; por meio de sinónimos.
\section{Sinonímico}
\begin{itemize}
\item {Grp. gram.:adj.}
\end{itemize}
Relativo á sinonímia ou aos sinónimos.
\section{Sinonimista}
\begin{itemize}
\item {Grp. gram.:m. ,  f.  e  adj.}
\end{itemize}
Diz-se da pessôa, que se ocupa de sinónimos.
\section{Sinonimizar}
\begin{itemize}
\item {Grp. gram.:v. t.}
\end{itemize}
Tornar sinónimo.
\section{Sinónimo}
\begin{itemize}
\item {Grp. gram.:adj.}
\end{itemize}
\begin{itemize}
\item {Grp. gram.:M.}
\end{itemize}
\begin{itemize}
\item {Proveniência:(Lat. \textunderscore synonymus\textunderscore )}
\end{itemize}
Diz-se da palavra que tem proximamente o mesmo sentido que outra.
Palavra sinónima.
\section{Synchronológico}
\begin{itemize}
\item {Grp. gram.:adj.}
\end{itemize}
Relativo á synchronologia.
\section{Sýnchyse}
\begin{itemize}
\item {fónica:qui}
\end{itemize}
\begin{itemize}
\item {Grp. gram.:f.}
\end{itemize}
\begin{itemize}
\item {Utilização:Gram.}
\end{itemize}
\begin{itemize}
\item {Proveniência:(Lat. \textunderscore synchysis\textunderscore )}
\end{itemize}
Inversão da ordem natural das palavras, tornando a phrase obscura.
Hypérbato exaggerado.
\section{Synclinal}
\begin{itemize}
\item {Grp. gram.:adj.}
\end{itemize}
\begin{itemize}
\item {Utilização:Geol.}
\end{itemize}
\begin{itemize}
\item {Proveniência:(Do gr. \textunderscore sun\textunderscore  + \textunderscore klinein\textunderscore )}
\end{itemize}
Diz-se da linha, seguida pelas camadas de terreno, que, curvando-se em direcções oppostas, tendem a reunir-se.
\section{Synclínico}
\begin{itemize}
\item {Grp. gram.:adj.}
\end{itemize}
O mesmo ou melhor que \textunderscore synclinal\textunderscore .
\section{Sýnclise}
\begin{itemize}
\item {Grp. gram.:f.}
\end{itemize}
\begin{itemize}
\item {Proveniência:(Do gr. \textunderscore sun\textunderscore  + \textunderscore klinein\textunderscore )}
\end{itemize}
Emprêgo de pronome synclítico.
\section{Synclítica}
\begin{itemize}
\item {Grp. gram.:f.}
\end{itemize}
\begin{itemize}
\item {Utilização:Gram.}
\end{itemize}
\begin{itemize}
\item {Proveniência:(De \textunderscore synclítico\textunderscore )}
\end{itemize}
Palavra, que se intercala noutra, perdendo o accento próprio.
\section{Synclítico}
\begin{itemize}
\item {Grp. gram.:adj.}
\end{itemize}
\begin{itemize}
\item {Utilização:Gram.}
\end{itemize}
\begin{itemize}
\item {Proveniência:(Do gr. \textunderscore sun\textunderscore  + \textunderscore klinein\textunderscore )}
\end{itemize}
Diz-se do pronome, que se intercala numa palavra:«far-\textunderscore se\textunderscore -á».
\section{Synclitismo}
\begin{itemize}
\item {Grp. gram.:m.}
\end{itemize}
\begin{itemize}
\item {Utilização:Med.}
\end{itemize}
\begin{itemize}
\item {Utilização:Gram.}
\end{itemize}
\begin{itemize}
\item {Proveniência:(De \textunderscore synclítico\textunderscore )}
\end{itemize}
Descida da cabeça do feto pela bacia, por fórma que o seu diâmetro bi-parietal é parallelo ao plano do estreito superior.
Theoria da collocação dos pronomes complementares.
\section{Sýncopa}
\begin{itemize}
\item {Grp. gram.:f.}
\end{itemize}
\begin{itemize}
\item {Utilização:Des.}
\end{itemize}
\begin{itemize}
\item {Proveniência:(Lat. \textunderscore syncopa\textunderscore )}
\end{itemize}
O mesmo que \textunderscore sýncope\textunderscore .
\section{Syncopal}
\begin{itemize}
\item {Grp. gram.:adj.}
\end{itemize}
Relativo a sýncope; que tem o carácter de sýncope.
\section{Syncopar}
\begin{itemize}
\item {Grp. gram.:v.}
\end{itemize}
\begin{itemize}
\item {Utilização:t. Gram.}
\end{itemize}
\begin{itemize}
\item {Grp. gram.:V. i.}
\end{itemize}
\begin{itemize}
\item {Proveniência:(Lat. \textunderscore syncopare\textunderscore )}
\end{itemize}
Tirar letra ou sýllaba, por meio de sýncope, a.
Fazer sýncope em música.
\section{Sýncope}
\begin{itemize}
\item {Grp. gram.:f.}
\end{itemize}
\begin{itemize}
\item {Utilização:Med.}
\end{itemize}
\begin{itemize}
\item {Utilização:Gram.}
\end{itemize}
\begin{itemize}
\item {Utilização:Mús.}
\end{itemize}
\begin{itemize}
\item {Proveniência:(Lat. \textunderscore syncope\textunderscore )}
\end{itemize}
Deminuição repentina e passageira da acção do coração, interrompendo-se a respiração, as sensações e os movimentos voluntários.
Suppressão de uma letra ou sýllaba no meio de uma palavra.
Ligação da última nota de um compasso musical com a primeira do seguinte.
\section{Syncopizante}
\begin{itemize}
\item {Grp. gram.:adj.}
\end{itemize}
\begin{itemize}
\item {Utilização:Med.}
\end{itemize}
\begin{itemize}
\item {Proveniência:(De \textunderscore syncopizar\textunderscore )}
\end{itemize}
Que tem sýncope.
\section{Syncopizar}
\begin{itemize}
\item {Grp. gram.:v. t.}
\end{itemize}
\begin{itemize}
\item {Grp. gram.:V. i.  e  p.}
\end{itemize}
\begin{itemize}
\item {Utilização:Med.}
\end{itemize}
O mesmo que \textunderscore syncopar\textunderscore .
Têr sýncope.
\section{Syncotyledóneo}
\begin{itemize}
\item {Grp. gram.:adj.}
\end{itemize}
\begin{itemize}
\item {Proveniência:(De \textunderscore syn...\textunderscore  + \textunderscore colyledóneo\textunderscore )}
\end{itemize}
Diz-se do vegetal, que tem os cotylédones reunidos num só corpo.
\section{Syncraniano}
\begin{itemize}
\item {Grp. gram.:adj.}
\end{itemize}
\begin{itemize}
\item {Proveniência:(De \textunderscore syn\textunderscore  + \textunderscore craniano\textunderscore )}
\end{itemize}
Diz-se da maxilla superior, por estar ligada ao crânio.
\section{Syncrético}
\begin{itemize}
\item {Grp. gram.:adj.}
\end{itemize}
Relativo ao syncretismo.
\section{Syncretismo}
\begin{itemize}
\item {Grp. gram.:m.}
\end{itemize}
\begin{itemize}
\item {Proveniência:(Gr. \textunderscore sugkretismos\textunderscore )}
\end{itemize}
Systema philosóphico, que combinava os princípios de diversos systemas.
Eclectismo.
Amálgama de concepções heterogêneas.
\section{Syncretista}
\begin{itemize}
\item {Grp. gram.:adj.}
\end{itemize}
\begin{itemize}
\item {Grp. gram.:M. ,  f.  e  adj.}
\end{itemize}
O mesmo que \textunderscore syncrético\textunderscore .
Diz-se da pessôa partidária do syncretismo.
(Cp. \textunderscore syncretismo\textunderscore )
\section{Syncripta}
\begin{itemize}
\item {Grp. gram.:f.}
\end{itemize}
\begin{itemize}
\item {Utilização:Zool.}
\end{itemize}
Gênero de infusórios
\section{Sýncrise}
\begin{itemize}
\item {Grp. gram.:f.}
\end{itemize}
\begin{itemize}
\item {Utilização:Chím.}
\end{itemize}
\begin{itemize}
\item {Utilização:Ant.}
\end{itemize}
\begin{itemize}
\item {Proveniência:(Lat. \textunderscore syncrisis\textunderscore )}
\end{itemize}
Opposição, o mesmo que \textunderscore anthitese\textunderscore .
Reunião de duas vogaes num ditongo.
Coagulação de líquidos misturados.
\section{Syncrítico}
\begin{itemize}
\item {Grp. gram.:adj.}
\end{itemize}
\begin{itemize}
\item {Utilização:Med.}
\end{itemize}
Relativo á \textunderscore sýncrise\textunderscore .
O mesmo que \textunderscore adstringente\textunderscore .
\section{Syncýclia}
\begin{itemize}
\item {Grp. gram.:f.}
\end{itemize}
\begin{itemize}
\item {Proveniência:(Do gr. \textunderscore sun\textunderscore  + \textunderscore kuklos\textunderscore )}
\end{itemize}
Gênero de plantas phýceas.
\section{Syndáctylo}
\begin{itemize}
\item {Grp. gram.:adj.}
\end{itemize}
\begin{itemize}
\item {Grp. gram.:M. pl.}
\end{itemize}
\begin{itemize}
\item {Proveniência:(Do gr. \textunderscore sun\textunderscore  + \textunderscore daktulus\textunderscore )}
\end{itemize}
Que tem os dedos reunidos.
Família de aves syndáctylas.
Família de mammíferos syndáctylos.
\section{Syndapso}
\begin{itemize}
\item {Grp. gram.:m.}
\end{itemize}
(?):«\textunderscore ...do lethargo em que a (língua) puseram balofos bíltris, mazorraes syndapsos\textunderscore ». Filinto, I, 50.--Invenção enigmática, talvez relacionada com o gr. \textunderscore sun\textunderscore  e o lat. \textunderscore daps\textunderscore , designando quem se banqueteia lautamente com outros. O enigma é confessado pelo autor, pois diz, em nota: \textunderscore quis potest capere capiat\textunderscore .
Diversão mal empregada.
\section{Syndectomia}
\begin{itemize}
\item {Grp. gram.:f.}
\end{itemize}
\begin{itemize}
\item {Utilização:Cir.}
\end{itemize}
\begin{itemize}
\item {Proveniência:(Do gr. \textunderscore sundes\textunderscore  + \textunderscore ektome\textunderscore )}
\end{itemize}
Excisão da conjunctiva.
\section{Syndesmographia}
\begin{itemize}
\item {Grp. gram.:f.}
\end{itemize}
\begin{itemize}
\item {Proveniência:(De \textunderscore syndesmógrapho\textunderscore )}
\end{itemize}
Descripção dos ligamentos, em Anatomia.
\section{Syndesmográphico}
\begin{itemize}
\item {Grp. gram.:adj.}
\end{itemize}
Relativo á syndesmographia.
\section{Syndesmógrapho}
\begin{itemize}
\item {Grp. gram.:m.}
\end{itemize}
\begin{itemize}
\item {Proveniência:(Do gr. \textunderscore sundesmos\textunderscore  + \textunderscore graphein\textunderscore )}
\end{itemize}
Aquelle que se occupa de syndesmographia.
\section{Syndesmologia}
\begin{itemize}
\item {Grp. gram.:f.}
\end{itemize}
O mesmo que \textunderscore syndesmographia\textunderscore .
\section{Syndesmólogo}
\begin{itemize}
\item {Grp. gram.:m.}
\end{itemize}
Tratadista da syndesmologia.
\section{Syndesmose}
\begin{itemize}
\item {Grp. gram.:f.}
\end{itemize}
\begin{itemize}
\item {Utilização:Anat.}
\end{itemize}
\begin{itemize}
\item {Proveniência:(Do gr. \textunderscore sundesmos\textunderscore )}
\end{itemize}
Reunião de ossos por meio de ligamentos.
\section{Syndesmotomia}
\begin{itemize}
\item {Grp. gram.:f.}
\end{itemize}
\begin{itemize}
\item {Utilização:Anat.}
\end{itemize}
\begin{itemize}
\item {Proveniência:(Do gr. \textunderscore sundesmos\textunderscore  + \textunderscore tome\textunderscore )}
\end{itemize}
Dissecação de ligamentos.
\section{Syndesmotómico}
\begin{itemize}
\item {Grp. gram.:adj.}
\end{itemize}
Relativo á syndesmotomia
\section{Syndicação}
\begin{itemize}
\item {Grp. gram.:f.}
\end{itemize}
Acto ou effeito de syndicar.
\section{Syndicado}
\begin{itemize}
\item {Grp. gram.:adj.}
\end{itemize}
\begin{itemize}
\item {Grp. gram.:M.}
\end{itemize}
\begin{itemize}
\item {Proveniência:(De \textunderscore syndicar\textunderscore )}
\end{itemize}
Que foi objecto de syndicância.
Que se inquiriu ou se investigou.
Indivíduo syndicado.
Cargo ou funcções de sýndico.
\section{Syndicador}
\begin{itemize}
\item {Grp. gram.:m.}
\end{itemize}
Aquelle que syndica; syndicante.
\section{Syndical}
\begin{itemize}
\item {Grp. gram.:adj.}
\end{itemize}
\begin{itemize}
\item {Utilização:Neol.}
\end{itemize}
Relativo a syndicato: \textunderscore casa syndical\textunderscore .
\section{Syndicalismo}
\begin{itemize}
\item {Grp. gram.:m.}
\end{itemize}
\begin{itemize}
\item {Proveniência:(De \textunderscore syndical\textunderscore )}
\end{itemize}
Theoria das doutrinas sôbre syndicatos.
Defesa dos syndicatos.
\section{Syndicalista}
\begin{itemize}
\item {Grp. gram.:adj.}
\end{itemize}
\begin{itemize}
\item {Grp. gram.:M.}
\end{itemize}
Relativo a syndicato: \textunderscore reivindicações syndicalistas\textunderscore .
Partidário ou defensor do syndicalismo ou de certo syndicato.
\section{Syndicância}
\begin{itemize}
\item {Grp. gram.:f.}
\end{itemize}
\begin{itemize}
\item {Proveniência:(De \textunderscore syndicar\textunderscore )}
\end{itemize}
O mesmo que \textunderscore syndicação\textunderscore .
Inquérito.
\section{Syndicante}
\begin{itemize}
\item {Grp. gram.:m.  e  adj.}
\end{itemize}
O que syndica.
\section{Syndicar}
\begin{itemize}
\item {Grp. gram.:v. t.  e  i.}
\end{itemize}
\begin{itemize}
\item {Proveniência:(De \textunderscore sýndico\textunderscore )}
\end{itemize}
Tomar informações de alguma coisa; inquirir.
\section{Syndicatal}
\begin{itemize}
\item {Grp. gram.:adj.}
\end{itemize}
\begin{itemize}
\item {Utilização:bras}
\end{itemize}
\begin{itemize}
\item {Utilização:Neol.}
\end{itemize}
Relativo a syndicato.
O mesmo ou melhor que \textunderscore syndical\textunderscore .
\section{Syndicateiro}
\begin{itemize}
\item {Grp. gram.:m.  e  adj.}
\end{itemize}
\begin{itemize}
\item {Utilização:Deprec.}
\end{itemize}
Indivíduo, que faz parte, ou gosta de fazer parte, de syndicatos financeiros.
\section{Syndicato}
\begin{itemize}
\item {Grp. gram.:m.}
\end{itemize}
\begin{itemize}
\item {Grp. gram.:M.}
\end{itemize}
\begin{itemize}
\item {Utilização:Deprec.}
\end{itemize}
\begin{itemize}
\item {Proveniência:(De \textunderscore syndicar\textunderscore )}
\end{itemize}
O mesmo que \textunderscore syndicado\textunderscore .
Companhia ou associação de capitalistas, interessados na mesma empresa e pondo em commum os seus títulos, para que na venda dêstes não haja alteração de preço.
Especulação financeira pouco lícita.
Associação dos indivíduos de uma classe, para defesa dos seus interesses económicos: \textunderscore syndicato agrícola\textunderscore ; \textunderscore syndicato operário\textunderscore .
\section{Syndicatório}
\begin{itemize}
\item {Grp. gram.:adj.}
\end{itemize}
\begin{itemize}
\item {Grp. gram.:M.}
\end{itemize}
Relativo a syndicato.
Membro de um syndicato financeiro.
\section{Syndicatura}
\begin{itemize}
\item {Grp. gram.:f.}
\end{itemize}
\begin{itemize}
\item {Proveniência:(De \textunderscore syndicato\textunderscore )}
\end{itemize}
Emprêgo ou funcções de sýndico.
\section{Sýndico}
\begin{itemize}
\item {Grp. gram.:m.}
\end{itemize}
\begin{itemize}
\item {Proveniência:(Gr. \textunderscore sundikos\textunderscore )}
\end{itemize}
Antigo procurador de uma communidade, côrtes, etc.
Advogado de corporação administrativa: \textunderscore fui sýndico da Câmara Municipal de Alcácer\textunderscore .
Aquelle que é encarregado de uma syndicância.
O que é escolhido para zelar ou defender os interesses de uma associação ou de uma classe.
\section{Sýndroma}
\begin{itemize}
\item {Grp. gram.:m.}
\end{itemize}
\begin{itemize}
\item {Proveniência:(Gr. \textunderscore sundrome\textunderscore )}
\end{itemize}
Designação antiga dos symptomas mórbidos, sem referência a determinada doença.
\section{Sýndrome}
\begin{itemize}
\item {Grp. gram.:m.}
\end{itemize}
O mesmo ou melhor que \textunderscore sýndroma\textunderscore .
\section{Synechia}
\begin{itemize}
\item {fónica:qui}
\end{itemize}
\begin{itemize}
\item {Grp. gram.:f.}
\end{itemize}
\begin{itemize}
\item {Utilização:Med.}
\end{itemize}
\begin{itemize}
\item {Proveniência:(Do gr. \textunderscore sunekheia\textunderscore )}
\end{itemize}
Adherência da íris.
\section{Synédoche}
\begin{itemize}
\item {fónica:que}
\end{itemize}
\begin{itemize}
\item {Grp. gram.:f.}
\end{itemize}
\begin{itemize}
\item {Utilização:Gram.}
\end{itemize}
\begin{itemize}
\item {Proveniência:(Lat. \textunderscore synedoche\textunderscore )}
\end{itemize}
Figura, em que se toma o gênero pela espécie, a espécie pelo genero, o todo pela parte, a parte pelo todo.
\section{Synedrella}
\begin{itemize}
\item {Grp. gram.:f.}
\end{itemize}
Gênero de plantas, da fam. das compostas.
\section{Synedrim}
\begin{itemize}
\item {Grp. gram.:m.}
\end{itemize}
O mesmo que \textunderscore synédrio\textunderscore .
\section{Synédrio}
\begin{itemize}
\item {Grp. gram.:m.}
\end{itemize}
\begin{itemize}
\item {Utilização:Ext.}
\end{itemize}
\begin{itemize}
\item {Proveniência:(Gr. \textunderscore sunedrion\textunderscore )}
\end{itemize}
Supremo Conselho, entre os Judeus.
Assembleia.
\section{Synelcosciádio}
\begin{itemize}
\item {Grp. gram.:m.}
\end{itemize}
Gênero de plantas umbelíferas.
\section{Synema}
\begin{itemize}
\item {Grp. gram.:m.}
\end{itemize}
\begin{itemize}
\item {Utilização:Bot.}
\end{itemize}
\begin{itemize}
\item {Proveniência:(Do gr. \textunderscore sun\textunderscore  + \textunderscore nema\textunderscore )}
\end{itemize}
Parte da columna das orchídeas, que representa os filetes dos estames.
\section{Synemático}
\begin{itemize}
\item {Grp. gram.:adj.}
\end{itemize}
\begin{itemize}
\item {Utilização:Bot.}
\end{itemize}
Relativo aos estames.
Que fórma, ou concorre para formar, os estames.
\section{Synemúria}
\begin{itemize}
\item {Grp. gram.:f.}
\end{itemize}
Gênero de molluscos acéphalos.
\section{Synérese}
\begin{itemize}
\item {Grp. gram.:f.}
\end{itemize}
\begin{itemize}
\item {Utilização:Gram.}
\end{itemize}
\begin{itemize}
\item {Proveniência:(Lat. \textunderscore synaeresis\textunderscore )}
\end{itemize}
Contracção de duas sýllabas numa, mas sem alteração de letras nem de sons.
\section{Synergia}
\begin{itemize}
\item {Grp. gram.:f.}
\end{itemize}
\begin{itemize}
\item {Proveniência:(Gr. \textunderscore sunergeia\textunderscore )}
\end{itemize}
Acto ou esfôrço simultâneo de diversos órgãos ou músculos.
\section{Synérgico}
\begin{itemize}
\item {Grp. gram.:adj.}
\end{itemize}
Relativo á synergia.
\section{Synérgide}
\begin{itemize}
\item {Grp. gram.:f.}
\end{itemize}
\begin{itemize}
\item {Utilização:Bot.}
\end{itemize}
\begin{itemize}
\item {Proveniência:(Do gr. \textunderscore sunergos\textunderscore )}
\end{itemize}
Cada uma das duas massas protoplásmicas, que se agrupam em tôrno de dois núcleos lateraes no óvulo.
\section{Sýnese}
\begin{itemize}
\item {Grp. gram.:f.}
\end{itemize}
\begin{itemize}
\item {Utilização:Gram.}
\end{itemize}
\begin{itemize}
\item {Proveniência:(Lat. \textunderscore synesis\textunderscore )}
\end{itemize}
Construcção syntáctica, em que se attende mais ao sentido do que ao rigor da fórma.
\section{Synesthesia}
\begin{itemize}
\item {Grp. gram.:f.}
\end{itemize}
\begin{itemize}
\item {Utilização:Med.}
\end{itemize}
\begin{itemize}
\item {Proveniência:(Do gr. \textunderscore sun\textunderscore  + \textunderscore asthesia\textunderscore )}
\end{itemize}
Certa perturbação na percepção das sensações.
\section{Syneta}
\begin{itemize}
\item {Grp. gram.:f.}
\end{itemize}
\begin{itemize}
\item {Proveniência:(Do gr. \textunderscore sunethes\textunderscore )}
\end{itemize}
Gênero de insectos coleópteros.
\section{Syngênese}
\begin{itemize}
\item {Grp. gram.:adj.}
\end{itemize}
\begin{itemize}
\item {Grp. gram.:F.}
\end{itemize}
\begin{itemize}
\item {Proveniência:(Do gr. \textunderscore sun\textunderscore  + \textunderscore genesis\textunderscore )}
\end{itemize}
O mesmo que \textunderscore synanthéreo\textunderscore .
Hypóthese dos que admittem a criação simultânea de todos os seres vivos.
\section{Syngenesia}
\begin{itemize}
\item {Grp. gram.:f.}
\end{itemize}
\begin{itemize}
\item {Proveniência:(De \textunderscore syngênese\textunderscore )}
\end{itemize}
Conjunto das plantas, cujas flôres ou estames são ligados pelas antheras, e que constituem uma classe no systema de Linneu.
\section{Syngenésico}
\begin{itemize}
\item {Grp. gram.:adj.}
\end{itemize}
Relativo á syngênese e á syngenesia.
\section{Syngenesista}
\begin{itemize}
\item {Grp. gram.:m. ,  f.  e  adj.}
\end{itemize}
Diz-se da pessôa partidária da syngênese.
\section{Syngenista}
\begin{itemize}
\item {Grp. gram.:m. ,  f.  e  adj.}
\end{itemize}
(V.syngenesista)
\section{Syngnáthidas}
\begin{itemize}
\item {Grp. gram.:f. pl.}
\end{itemize}
Gênero de peixes, o mesmo que \textunderscore sýngnathos\textunderscore .
\section{Sýngnathos}
\begin{itemize}
\item {Grp. gram.:m. pl.}
\end{itemize}
\begin{itemize}
\item {Proveniência:(Do gr. \textunderscore sun\textunderscore  + \textunderscore gnathos\textunderscore )}
\end{itemize}
Gênero de peixes, de barbatanas cinzentas, sem língua nem dentes.
\section{Sýngrapha}
\begin{itemize}
\item {Grp. gram.:f.}
\end{itemize}
\begin{itemize}
\item {Proveniência:(Lat. \textunderscore syngrapha\textunderscore )}
\end{itemize}
O mesmo ou melhór que \textunderscore sýngrapho\textunderscore .
\section{Syngráphico}
\begin{itemize}
\item {Grp. gram.:adj.}
\end{itemize}
Relativo ao sýngrapho.
\section{Sýngrapho}
\begin{itemize}
\item {Grp. gram.:m.}
\end{itemize}
\begin{itemize}
\item {Proveniência:(Do gr. \textunderscore sun\textunderscore  + \textunderscore graphein\textunderscore )}
\end{itemize}
Documento de dívida, assignado pelo credor e pelo devedor.
\section{Synhedrim}
\begin{itemize}
\item {fónica:ne}
\end{itemize}
\begin{itemize}
\item {Grp. gram.:m.}
\end{itemize}
(V.synedrim)
\section{Synhédrio}
\begin{itemize}
\item {fónica:né}
\end{itemize}
\begin{itemize}
\item {Grp. gram.:m.}
\end{itemize}
(V.synédrio)
\section{Synistrato}
\begin{itemize}
\item {Grp. gram.:adj.}
\end{itemize}
\begin{itemize}
\item {Utilização:Zool.}
\end{itemize}
\begin{itemize}
\item {Grp. gram.:M. pl.}
\end{itemize}
Diz-se do insecto, cujas queixadas são reunidas pela base ao lábio inferior.
Classe de insectos, que comprehende a maior parte dos neurópteros e alguns ápteros.
\section{Synizese}
\begin{itemize}
\item {Grp. gram.:f.}
\end{itemize}
\begin{itemize}
\item {Utilização:Gram.}
\end{itemize}
\begin{itemize}
\item {Utilização:Cir.}
\end{itemize}
\begin{itemize}
\item {Proveniência:(Lat. \textunderscore synizesis\textunderscore )}
\end{itemize}
Pronúncia de duas vogaes distintas em um só tempo prosódico, sem formar ditongo.
Occlusão da pupilla, em consequência de uma inflammação.
\section{Synnema}
\begin{itemize}
\item {Grp. gram.:m.}
\end{itemize}
\begin{itemize}
\item {Utilização:Bot.}
\end{itemize}
\begin{itemize}
\item {Proveniência:(Do gr. \textunderscore sun\textunderscore  + \textunderscore nema\textunderscore )}
\end{itemize}
Parte da columna das orchídeas, que representa os filetes dos estames.
\section{Synneurose}
\begin{itemize}
\item {Grp. gram.:f.}
\end{itemize}
\begin{itemize}
\item {Proveniência:(Gr. \textunderscore sunneurosis\textunderscore )}
\end{itemize}
Ligação de dois ossos.
\section{Synochite}
\begin{itemize}
\item {fónica:qui}
\end{itemize}
\begin{itemize}
\item {Grp. gram.:f.}
\end{itemize}
\begin{itemize}
\item {Proveniência:(Lat. \textunderscore synochitis\textunderscore )}
\end{itemize}
Pedra preciosa, mencionada por Plínio e hoje desconhecida.
\section{Sýnocho}
\begin{itemize}
\item {fónica:co}
\end{itemize}
\begin{itemize}
\item {Grp. gram.:adj.}
\end{itemize}
\begin{itemize}
\item {Utilização:Med.}
\end{itemize}
\begin{itemize}
\item {Proveniência:(Gr. \textunderscore sunokhos\textunderscore )}
\end{itemize}
Inflammatório.
\section{Synodal}
\begin{itemize}
\item {Grp. gram.:adj.}
\end{itemize}
\begin{itemize}
\item {Proveniência:(Lat. \textunderscore synodalis\textunderscore )}
\end{itemize}
Relativo ao sýnodo.
\section{Synodático}
\begin{itemize}
\item {Grp. gram.:adj.}
\end{itemize}
Que se realiza num sýnodo.
\section{Synodendro}
\begin{itemize}
\item {Grp. gram.:m.}
\end{itemize}
Insecto coleóptero, que vive nas águas.
\section{Synodicamente}
\begin{itemize}
\item {Grp. gram.:adv.}
\end{itemize}
De modo synótico; em sýnodo.
\section{Synódico}
\begin{itemize}
\item {Grp. gram.:adj.}
\end{itemize}
\begin{itemize}
\item {Utilização:Astron.}
\end{itemize}
\begin{itemize}
\item {Grp. gram.:M.}
\end{itemize}
\begin{itemize}
\item {Proveniência:(Lat. \textunderscore synodicus\textunderscore )}
\end{itemize}
O mesmo que \textunderscore synodal\textunderscore .
Relativo á revolução dos planetas.
Collecção de resoluções synodaes.
\section{Sýnodo}
\begin{itemize}
\item {Grp. gram.:m.}
\end{itemize}
\begin{itemize}
\item {Proveniência:(Lat. \textunderscore synodus\textunderscore )}
\end{itemize}
Assembleia de párochos e de outros padres, convocada por ordem do seu prelado ou de outro superior.
\section{Synodonte}
\begin{itemize}
\item {Grp. gram.:m.}
\end{itemize}
\begin{itemize}
\item {Proveniência:(Do gr. \textunderscore sun\textunderscore  + \textunderscore odous\textunderscore )}
\end{itemize}
Gênero de peixes malacopterýgios.
\section{Synóico}
\begin{itemize}
\item {Grp. gram.:m.}
\end{itemize}
\begin{itemize}
\item {Proveniência:(Do gr. \textunderscore sunoikes\textunderscore )}
\end{itemize}
Gênero de molluscos de Spitzberg.
\section{Synonýmia}
\begin{itemize}
\item {Grp. gram.:f.}
\end{itemize}
\begin{itemize}
\item {Proveniência:(Lat. \textunderscore synonymia\textunderscore )}
\end{itemize}
Qualidade do que é synónymo.
Acto de exprimir a mesma coisa por palavras synónymas.
\section{Synonýmica}
\begin{itemize}
\item {Grp. gram.:f.}
\end{itemize}
\begin{itemize}
\item {Proveniência:(De \textunderscore synonýmico\textunderscore )}
\end{itemize}
Arte ou estudo dos synónymos e sua distincção.
\section{Synonymicamente}
\begin{itemize}
\item {Grp. gram.:adv.}
\end{itemize}
De modo synonýmico; por meio de synónymos.
\section{Synonýmico}
\begin{itemize}
\item {Grp. gram.:adj.}
\end{itemize}
Relativo á synonýmia ou aos synónymos.
\section{Synonymista}
\begin{itemize}
\item {Grp. gram.:m. ,  f.  e  adj.}
\end{itemize}
Diz-se da pessôa, que se occupa de synónymos.
\section{Synonymizar}
\begin{itemize}
\item {Grp. gram.:v. t.}
\end{itemize}
Tornar synónymo.
\section{Synónymo}
\begin{itemize}
\item {Grp. gram.:adj.}
\end{itemize}
\begin{itemize}
\item {Grp. gram.:M.}
\end{itemize}
\begin{itemize}
\item {Proveniência:(Lat. \textunderscore synonymus\textunderscore )}
\end{itemize}
Diz-se da palavra que tem proximamente o mesmo sentido que outra.
Palavra synónyma.
\section{Sincíclia}
\begin{itemize}
\item {Grp. gram.:f.}
\end{itemize}
\begin{itemize}
\item {Proveniência:(Do gr. \textunderscore sun\textunderscore  + \textunderscore kuklos\textunderscore )}
\end{itemize}
Gênero de plantas fíceas.
\section{Sinclinal}
\begin{itemize}
\item {Grp. gram.:adj.}
\end{itemize}
\begin{itemize}
\item {Utilização:Geol.}
\end{itemize}
\begin{itemize}
\item {Proveniência:(Do gr. \textunderscore sun\textunderscore  + \textunderscore klinein\textunderscore )}
\end{itemize}
Diz-se da linha, seguida pelas camadas de terreno, que, curvando-se em direcções opostas, tendem a reunir-se.
\section{Sinclínico}
\begin{itemize}
\item {Grp. gram.:adj.}
\end{itemize}
O mesmo ou melhor que \textunderscore sinclinal\textunderscore .
\section{Sinclise}
\begin{itemize}
\item {Grp. gram.:f.}
\end{itemize}
\begin{itemize}
\item {Proveniência:(Do gr. \textunderscore sun\textunderscore  + \textunderscore klinein\textunderscore )}
\end{itemize}
Emprêgo de pronome sinclítico.
\section{Sinclítica}
\begin{itemize}
\item {Grp. gram.:f.}
\end{itemize}
\begin{itemize}
\item {Utilização:Gram.}
\end{itemize}
\begin{itemize}
\item {Proveniência:(De \textunderscore sinclítico\textunderscore )}
\end{itemize}
Palavra, que se intercala noutra, perdendo o accento próprio.
\section{Sinclítico}
\begin{itemize}
\item {Grp. gram.:adj.}
\end{itemize}
\begin{itemize}
\item {Utilização:Gram.}
\end{itemize}
\begin{itemize}
\item {Proveniência:(Do gr. \textunderscore sun\textunderscore  + \textunderscore klinein\textunderscore )}
\end{itemize}
Diz-se do pronome, que se intercala numa palavra:«far-\textunderscore se\textunderscore -á».
\section{Sinclitismo}
\begin{itemize}
\item {Grp. gram.:m.}
\end{itemize}
\begin{itemize}
\item {Utilização:Med.}
\end{itemize}
\begin{itemize}
\item {Utilização:Gram.}
\end{itemize}
\begin{itemize}
\item {Proveniência:(De \textunderscore sinclítico\textunderscore )}
\end{itemize}
Descida da cabeça do feto pela bacia, por fórma que o seu diâmetro bi-parietal é paralelo ao plano do estreito superior.
Teoria da colocação dos pronomes complementares.
\section{Síncopa}
\begin{itemize}
\item {Grp. gram.:f.}
\end{itemize}
\begin{itemize}
\item {Utilização:Des.}
\end{itemize}
\begin{itemize}
\item {Proveniência:(Lat. \textunderscore syncopa\textunderscore )}
\end{itemize}
O mesmo que \textunderscore síncope\textunderscore .
\section{Sincopal}
\begin{itemize}
\item {Grp. gram.:adj.}
\end{itemize}
Relativo a síncope; que tem o carácter de síncope.
\section{Sincopar}
\begin{itemize}
\item {Grp. gram.:v.}
\end{itemize}
\begin{itemize}
\item {Utilização:t. Gram.}
\end{itemize}
\begin{itemize}
\item {Grp. gram.:V. i.}
\end{itemize}
\begin{itemize}
\item {Proveniência:(Lat. \textunderscore syncopare\textunderscore )}
\end{itemize}
Tirar letra ou sílaba, por meio de síncope, a.
Fazer síncope em música.
\section{Síncope}
\begin{itemize}
\item {Grp. gram.:f.}
\end{itemize}
\begin{itemize}
\item {Utilização:Med.}
\end{itemize}
\begin{itemize}
\item {Utilização:Gram.}
\end{itemize}
\begin{itemize}
\item {Utilização:Mús.}
\end{itemize}
\begin{itemize}
\item {Proveniência:(Lat. \textunderscore syncope\textunderscore )}
\end{itemize}
Deminuição repentina e passageira da acção do coração, interrompendo-se a respiração, as sensações e os movimentos voluntários.
Supressão de uma letra ou sílaba no meio de uma palavra.
Ligação da última nota de um compasso musical com a primeira do seguinte.
\section{Sincopizante}
\begin{itemize}
\item {Grp. gram.:adj.}
\end{itemize}
\begin{itemize}
\item {Utilização:Med.}
\end{itemize}
\begin{itemize}
\item {Proveniência:(De \textunderscore sincopizar\textunderscore )}
\end{itemize}
Que tem síncope.
\section{Sincopizar}
\begin{itemize}
\item {Grp. gram.:v. t.}
\end{itemize}
\begin{itemize}
\item {Grp. gram.:V. i.  e  p.}
\end{itemize}
\begin{itemize}
\item {Utilização:Med.}
\end{itemize}
O mesmo que \textunderscore sincopar\textunderscore .
Têr síncope.
\section{Sincotiledóneo}
\begin{itemize}
\item {Grp. gram.:adj.}
\end{itemize}
\begin{itemize}
\item {Proveniência:(De \textunderscore sin...\textunderscore  + \textunderscore coliledóneo\textunderscore )}
\end{itemize}
Diz-se do vegetal, que tem os cotilédones reunidos num só corpo.
\section{Sifilofobia}
\begin{itemize}
\item {Grp. gram.:f.}
\end{itemize}
\begin{itemize}
\item {Utilização:Med.}
\end{itemize}
\begin{itemize}
\item {Proveniência:(De \textunderscore sífilis\textunderscore  + gr. \textunderscore phobein\textunderscore )}
\end{itemize}
Terror mórbido das doenças venéreas.
\section{Siracusa}
\begin{itemize}
\item {Grp. gram.:m.}
\end{itemize}
Vinho célebre, produzido pelos vinhedos de Siracusa.
\section{Siracusano}
\begin{itemize}
\item {Grp. gram.:adj.}
\end{itemize}
\begin{itemize}
\item {Grp. gram.:M.}
\end{itemize}
\begin{itemize}
\item {Proveniência:(Lat. \textunderscore syracusanus\textunderscore )}
\end{itemize}
Relativo a Siracusa.
Habitante de Siracusa.
\section{Sirenópside}
\begin{itemize}
\item {Grp. gram.:m.}
\end{itemize}
Gênero de plantas crucíferas.
\section{Sinopse}
\begin{itemize}
\item {Grp. gram.:f.}
\end{itemize}
\begin{itemize}
\item {Proveniência:(Lat. \textunderscore synopsis\textunderscore )}
\end{itemize}
Obra ou tratado, que apresenta em síntese o conjunto de uma ciência.
Síntese; resumo; sumário; compêndio.
\section{Sinopsia}
\begin{itemize}
\item {Grp. gram.:f.}
\end{itemize}
\begin{itemize}
\item {Utilização:Med.}
\end{itemize}
\begin{itemize}
\item {Proveniência:(Do gr. \textunderscore sun\textunderscore  + \textunderscore opsis\textunderscore )}
\end{itemize}
Associação de fenómenos visuaes ás sensações percebidas por outros sentidos.
\section{Sinóptico}
\begin{itemize}
\item {Grp. gram.:adj.}
\end{itemize}
\begin{itemize}
\item {Proveniência:(Lat. \textunderscore synopticus\textunderscore )}
\end{itemize}
Relativo á sinopse; que tem fórma de sinopse; resumido.
\section{Sinorquia}
\begin{itemize}
\item {Grp. gram.:f.}
\end{itemize}
\begin{itemize}
\item {Utilização:Med.}
\end{itemize}
\begin{itemize}
\item {Proveniência:(Do gr. \textunderscore sun\textunderscore  + \textunderscore orkhis\textunderscore )}
\end{itemize}
Reunião intra-abdominal de dois testículos, na linha média do corpo.
\section{Sinorquidia}
\begin{itemize}
\item {Grp. gram.:f.}
\end{itemize}
(V.sinorquia)
\section{Sinorrizo}
\begin{itemize}
\item {Grp. gram.:adj.}
\end{itemize}
\begin{itemize}
\item {Utilização:Bot.}
\end{itemize}
Diz-se do embrião, quando a radícula está um pouco soldada ao perisperma. Cf. Richard, \textunderscore Bot.\textunderscore 
\section{Sinosteografia}
\begin{itemize}
\item {Grp. gram.:f.}
\end{itemize}
\begin{itemize}
\item {Proveniência:(Do gr. \textunderscore sun\textunderscore  + \textunderscore osteon\textunderscore  + \textunderscore graphein\textunderscore )}
\end{itemize}
Parte da Anatomia, em que se descrevem as articulações.
\section{Sinosteográfico}
\begin{itemize}
\item {Grp. gram.:adj.}
\end{itemize}
Relativo á sinosteografia.
\section{Sinosteologia}
\begin{itemize}
\item {Grp. gram.:f.}
\end{itemize}
\begin{itemize}
\item {Utilização:Anat.}
\end{itemize}
\begin{itemize}
\item {Proveniência:(De \textunderscore sin...\textunderscore  + \textunderscore osteologia\textunderscore )}
\end{itemize}
Tratado das articulações.
\section{Sinosteológico}
\begin{itemize}
\item {Grp. gram.:adj.}
\end{itemize}
Relativo á sinosteologia.
\section{Sinosteotomia}
\begin{itemize}
\item {Grp. gram.:f.}
\end{itemize}
\begin{itemize}
\item {Utilização:Cir.}
\end{itemize}
\begin{itemize}
\item {Proveniência:(Do gr. \textunderscore sun\textunderscore  + \textunderscore osteon\textunderscore  + \textunderscore tome\textunderscore )}
\end{itemize}
Dissecção das articulações.
\section{Sinosteotómico}
\begin{itemize}
\item {Grp. gram.:adj.}
\end{itemize}
Relativo á sinosteotomia.
\section{Sinostose}
\begin{itemize}
\item {Grp. gram.:f.}
\end{itemize}
\begin{itemize}
\item {Proveniência:(Do gr. \textunderscore sun\textunderscore  + \textunderscore osteon\textunderscore )}
\end{itemize}
Sutura dos ossos.
\section{Sinotia}
\begin{itemize}
\item {Grp. gram.:f.}
\end{itemize}
Estado de sinoto.
\section{Sinoto}
\begin{itemize}
\item {Grp. gram.:adj.}
\end{itemize}
\begin{itemize}
\item {Proveniência:(Do gr. \textunderscore sun\textunderscore  + \textunderscore ous\textunderscore , \textunderscore otos\textunderscore )}
\end{itemize}
Diz-se do monstro, cujas orelhas estão reunidas.
\section{Sinóvia}
\begin{itemize}
\item {Grp. gram.:f.}
\end{itemize}
Humor, segregado pelas membranas que revestem a superfície das cavidades articulares.
(Palavra inventada por Paracelso. Do gr. \textunderscore sun\textunderscore  + lat. \textunderscore ovum\textunderscore ?)
\section{Sinovial}
\begin{itemize}
\item {Grp. gram.:adj.}
\end{itemize}
Relativo á sinóvia.
\section{Sinovite}
\begin{itemize}
\item {Grp. gram.:f.}
\end{itemize}
\begin{itemize}
\item {Proveniência:(De \textunderscore sinóvia\textunderscore )}
\end{itemize}
Inflamação das membranas sinoviaes.
\section{Sinovina}
\begin{itemize}
\item {Grp. gram.:f.}
\end{itemize}
\begin{itemize}
\item {Proveniência:(De \textunderscore sinóvia\textunderscore )}
\end{itemize}
Substância especial, extraida da sinóvia.
\section{Sintático}
\begin{itemize}
\item {Grp. gram.:adj.}
\end{itemize}
Relativo a sintaxe.
Conforme ás regras da sintaxe.
\section{Sintagma}
\begin{itemize}
\item {Grp. gram.:m.}
\end{itemize}
\begin{itemize}
\item {Grp. gram.:F.}
\end{itemize}
\begin{itemize}
\item {Proveniência:(Gr. \textunderscore suntagma\textunderscore )}
\end{itemize}
Qualquer tratado, cujo assumpto é metodicamente dividido em classes, números, etc.
Propriamente, era uma divisão de 256 homens, que forma quadrado na falange grega.
\section{Sintagmarca}
\begin{itemize}
\item {Grp. gram.:m.}
\end{itemize}
\begin{itemize}
\item {Proveniência:(Do gr. \textunderscore suntagma\textunderscore  + \textunderscore arkhein\textunderscore )}
\end{itemize}
Commandante de uma sintagma, no antigo exército grego.
\section{Sintaxe}
\begin{itemize}
\item {fónica:ce}
\end{itemize}
\begin{itemize}
\item {Grp. gram.:f.}
\end{itemize}
\begin{itemize}
\item {Proveniência:(Lat. \textunderscore syntaxis\textunderscore )}
\end{itemize}
Parte da Gramática que trata da disposição das palavras, da construcção das proposições, da relação lógica das frases entre si e dos preceitos geraes e particulares que se devam observar para tornar correcto, puro e elegante o estilo e a linguagem.
Livro, em que se expõem as regras da sintaxe.
\section{Sintáxico}
\begin{itemize}
\item {fónica:ci}
\end{itemize}
\begin{itemize}
\item {Grp. gram.:adj.}
\end{itemize}
(V.sintático)
\section{Sintaxístico}
\begin{itemize}
\item {fónica:cis}
\end{itemize}
\begin{itemize}
\item {Grp. gram.:adj.}
\end{itemize}
O mesmo que \textunderscore sintático\textunderscore . Cf. Filinto, I, 146.
\section{Síntese}
\begin{itemize}
\item {Grp. gram.:f.}
\end{itemize}
\begin{itemize}
\item {Proveniência:(Lat. \textunderscore synthesis\textunderscore )}
\end{itemize}
Operação química, com que se reúnem corpos simples, para formar corpos compostos, ou com que se reúnem corpos compostos para formar outros de composição mais complexa.
Processo filosófico, com que se desce dos princípios ás consequências, e das causas aos efeitos.
Quadro, que expõe o conjunto de uma ciência.
Resenha literária ou científica.
Demonstração matemática das proposições, pela simples dedução das que estão já provadas.
Operação cirúrgica, com que se reúnem ou se restituem ao estado primitivo as partes deslocadas ou separadas.
Organização mental de um sistema.
Figura gramatical, o mesmo que \textunderscore silepse\textunderscore .
Entre os antigos, era uma espécie de clâmide ou rocló que, nos banquetes, se vestia sôbre o vestuário comum, para que êste se não enodoasse.
\section{Sifilicómio}
\begin{itemize}
\item {Grp. gram.:m.}
\end{itemize}
\begin{itemize}
\item {Proveniência:(De \textunderscore sífilis\textunderscore  + gr. \textunderscore komein\textunderscore )}
\end{itemize}
Hospital ou dispensário, destinado especialmente ao tratamento da sífilis.
\section{Sifílide}
\begin{itemize}
\item {Grp. gram.:f.}
\end{itemize}
Afecção cutânea, derivada da sífilis.
\section{Sifilidofobia}
\begin{itemize}
\item {Grp. gram.:f.}
\end{itemize}
(V.sifilofobia). Cf. \textunderscore Jorn.-do-Comm.\textunderscore , do Rio, de 4-V-901.
\section{Sifiligrafia}
\begin{itemize}
\item {Grp. gram.:f.}
\end{itemize}
\begin{itemize}
\item {Proveniência:(De \textunderscore sifiligrafo\textunderscore )}
\end{itemize}
Tratado ou descripção da sífilis.
\section{Sifiligráfico}
\begin{itemize}
\item {Grp. gram.:adj.}
\end{itemize}
Relativo á sifiligrafia.
\section{Sifilígrafo}
\begin{itemize}
\item {Grp. gram.:m.}
\end{itemize}
Tratadista de sifiligrafia.
(Do \textunderscore sífilis\textunderscore  + gr. \textunderscore graphein\textunderscore )
\section{Sífilis}
\begin{itemize}
\item {Grp. gram.:f.}
\end{itemize}
Doença específica, de natureza venérea, transmitida por contacto ou por herança, e cujos efeitos afectam a constituição do organismo.
(Palavra inventada por Fracastor, num poema latino sôbre doenças venéreas)
\section{Sifilismo}
\begin{itemize}
\item {Grp. gram.:m.}
\end{itemize}
\begin{itemize}
\item {Proveniência:(De \textunderscore sífilis\textunderscore )}
\end{itemize}
Disposição natural para a sifilização.
\section{Sifilítico}
\begin{itemize}
\item {Grp. gram.:adj.}
\end{itemize}
\begin{itemize}
\item {Grp. gram.:M.}
\end{itemize}
Relativo á sífilis.
O doente de sífilis.
\section{Sifilização}
\begin{itemize}
\item {Grp. gram.:f.}
\end{itemize}
Acto ou efeito de sifilizar.
\section{Sifilizar}
\begin{itemize}
\item {Grp. gram.:v. t.}
\end{itemize}
Comunicar sífilis a.
\section{Sifilografia}
\textunderscore f.\textunderscore  (e der.)
O mesmo ou melhór que \textunderscore sifiligrafia\textunderscore , etc.
\section{Sifilóide}
\begin{itemize}
\item {Grp. gram.:f.}
\end{itemize}
\begin{itemize}
\item {Utilização:Med.}
\end{itemize}
\begin{itemize}
\item {Proveniência:(De \textunderscore sífilis\textunderscore  + gr. \textunderscore eidos\textunderscore )}
\end{itemize}
Erupção cutânea, parecida com a erupção sifilítica.
\section{Sifiloma}
\begin{itemize}
\item {Grp. gram.:m.}
\end{itemize}
\begin{itemize}
\item {Proveniência:(Do gr. \textunderscore suphilis\textunderscore )}
\end{itemize}
Producção neoplástica, de origem sifilítica.
\section{Sinteticamente}
\begin{itemize}
\item {Grp. gram.:adv.}
\end{itemize}
De modo sintético.
Em síntese; resumidamente.
\section{Sintético}
\begin{itemize}
\item {Grp. gram.:adj.}
\end{itemize}
\begin{itemize}
\item {Proveniência:(Do gr. \textunderscore sunthetikos\textunderscore )}
\end{itemize}
Relativo á síntese.
Feito em síntese; compendiado, resumido.
\section{Sintetismo}
\begin{itemize}
\item {Grp. gram.:m.}
\end{itemize}
\begin{itemize}
\item {Proveniência:(De \textunderscore síntese\textunderscore )}
\end{itemize}
Conjunto das operações necessárias para a síntese cirúrgica, isto é, para a redução de uma fractura, e para a conservar reduzida.
\section{Sintetização}
\begin{itemize}
\item {Grp. gram.:f.}
\end{itemize}
Acto ou efeito de sintetizar.
\section{Sintetizador}
\begin{itemize}
\item {Grp. gram.:adj.}
\end{itemize}
Que sintetiza.
\section{Sintetizar}
\begin{itemize}
\item {Grp. gram.:v. t.}
\end{itemize}
\begin{itemize}
\item {Proveniência:(De \textunderscore sintético\textunderscore )}
\end{itemize}
Tornar sintético, compendiar, resumir.
\section{Sintomia}
\begin{itemize}
\item {Grp. gram.:f.}
\end{itemize}
\begin{itemize}
\item {Proveniência:(Do gr. \textunderscore suntomos\textunderscore )}
\end{itemize}
Exposição abreviada, bosquejo.
Gênero de insectos coleópteros pentâmeros.
\section{Sintonina}
\begin{itemize}
\item {Grp. gram.:f.}
\end{itemize}
\begin{itemize}
\item {Proveniência:(Do gr. \textunderscore sun\textunderscore  + \textunderscore tonos\textunderscore )}
\end{itemize}
Fibrina muscular.
Musculina.
\section{Síntono}
\begin{itemize}
\item {Grp. gram.:m.}
\end{itemize}
\begin{itemize}
\item {Proveniência:(Lat. \textunderscore syntonum\textunderscore )}
\end{itemize}
Antigo instrumento músico.
\section{Synopse}
\begin{itemize}
\item {Grp. gram.:f.}
\end{itemize}
\begin{itemize}
\item {Proveniência:(Lat. \textunderscore synopsis\textunderscore )}
\end{itemize}
Obra ou tratado, que apresenta em sýnthese o conjunto de uma sciência.
Sýnthese; resumo; summário; compêndio.
\section{Synopsia}
\begin{itemize}
\item {Grp. gram.:f.}
\end{itemize}
\begin{itemize}
\item {Utilização:Med.}
\end{itemize}
\begin{itemize}
\item {Proveniência:(Do gr. \textunderscore sun\textunderscore  + \textunderscore opsis\textunderscore )}
\end{itemize}
Associação de phenómenos visuaes ás sensações percebidas por outros sentidos.
\section{Synóptico}
\begin{itemize}
\item {Grp. gram.:adj.}
\end{itemize}
\begin{itemize}
\item {Proveniência:(Lat. \textunderscore synopticus\textunderscore )}
\end{itemize}
Relativo á synopse; que tem fórma de synopse; resumido.
\section{Synorchia}
\begin{itemize}
\item {fónica:qui}
\end{itemize}
\begin{itemize}
\item {Grp. gram.:f.}
\end{itemize}
\begin{itemize}
\item {Utilização:Med.}
\end{itemize}
\begin{itemize}
\item {Proveniência:(Do gr. \textunderscore sun\textunderscore  + \textunderscore orkhis\textunderscore )}
\end{itemize}
Reunião intra-abdominal de dois testículos, na linha média do corpo.
\section{Synorchidia}
\begin{itemize}
\item {fónica:qui}
\end{itemize}
\begin{itemize}
\item {Grp. gram.:f.}
\end{itemize}
(V.synorchia)
\section{Synorrhizo}
\begin{itemize}
\item {Grp. gram.:adj.}
\end{itemize}
\begin{itemize}
\item {Utilização:Bot.}
\end{itemize}
Diz-se do embryão, quando a radícula está um pouco soldada ao perisperma. Cf. Richard, \textunderscore Bot.\textunderscore 
\section{Synosteographia}
\begin{itemize}
\item {Grp. gram.:f.}
\end{itemize}
\begin{itemize}
\item {Proveniência:(Do gr. \textunderscore sun\textunderscore  + \textunderscore osteon\textunderscore  + \textunderscore graphein\textunderscore )}
\end{itemize}
Parte da Anatomia, em que se descrevem as articulações.
\section{Synosteográphico}
\begin{itemize}
\item {Grp. gram.:adj.}
\end{itemize}
Relativo á synosteographia.
\section{Synosteologia}
\begin{itemize}
\item {Grp. gram.:f.}
\end{itemize}
\begin{itemize}
\item {Utilização:Anat.}
\end{itemize}
\begin{itemize}
\item {Proveniência:(De \textunderscore syn...\textunderscore  + \textunderscore osteologia\textunderscore )}
\end{itemize}
Tratado das articulações.
\section{Synosteológico}
\begin{itemize}
\item {Grp. gram.:adj.}
\end{itemize}
Relativo á synosteologia.
\section{Synosteotomia}
\begin{itemize}
\item {Grp. gram.:f.}
\end{itemize}
\begin{itemize}
\item {Utilização:Cir.}
\end{itemize}
\begin{itemize}
\item {Proveniência:(Do gr. \textunderscore sun\textunderscore  + \textunderscore osteon\textunderscore  + \textunderscore tome\textunderscore )}
\end{itemize}
Dissecção das articulações.
\section{Synosteotómico}
\begin{itemize}
\item {Grp. gram.:adj.}
\end{itemize}
Relativo á synosteotomia.
\section{Synostose}
\begin{itemize}
\item {Grp. gram.:f.}
\end{itemize}
\begin{itemize}
\item {Proveniência:(Do gr. \textunderscore sun\textunderscore  + \textunderscore osteon\textunderscore )}
\end{itemize}
Sutura dos ossos.
\section{Synotia}
\begin{itemize}
\item {Grp. gram.:f.}
\end{itemize}
Estado de synoto.
\section{Synoto}
\begin{itemize}
\item {Grp. gram.:adj.}
\end{itemize}
\begin{itemize}
\item {Proveniência:(Do gr. \textunderscore sun\textunderscore  + \textunderscore ous\textunderscore , \textunderscore otos\textunderscore )}
\end{itemize}
Diz-se do monstro, cujas orelhas estão reunidas.
\section{Synóvia}
\begin{itemize}
\item {Grp. gram.:f.}
\end{itemize}
Humor, segregado pelas membranas que revestem a superfície das cavidades articulares.
(Palavra inventada por Paracelso. Do gr. \textunderscore sun\textunderscore  + lat. \textunderscore ovum\textunderscore ?)
\section{Synovial}
\begin{itemize}
\item {Grp. gram.:adj.}
\end{itemize}
Relativo á synóvia.
\section{Synovite}
\begin{itemize}
\item {Grp. gram.:f.}
\end{itemize}
\begin{itemize}
\item {Proveniência:(De \textunderscore synóvia\textunderscore )}
\end{itemize}
Inflammação das membranas synoviaes.
\section{Synovina}
\begin{itemize}
\item {Grp. gram.:f.}
\end{itemize}
\begin{itemize}
\item {Proveniência:(De \textunderscore synóvia\textunderscore )}
\end{itemize}
Substância especial, extrahida da synóvia.
\section{Syntáctico}
\begin{itemize}
\item {Grp. gram.:adj.}
\end{itemize}
Relativo a syntaxe.
Conforme ás regras da syntaxe.
\section{Syntagma}
\begin{itemize}
\item {Grp. gram.:m.}
\end{itemize}
\begin{itemize}
\item {Grp. gram.:F.}
\end{itemize}
\begin{itemize}
\item {Proveniência:(Gr. \textunderscore suntagma\textunderscore )}
\end{itemize}
Qualquer tratado, cujo assumpto é methodicamente dividido em classes, números, etc.
Propriamente, era uma divisão de 256 homens, que forma quadrado na phalange grega.
\section{Syntagmarcha}
\begin{itemize}
\item {fónica:ca}
\end{itemize}
\begin{itemize}
\item {Grp. gram.:m.}
\end{itemize}
\begin{itemize}
\item {Proveniência:(Do gr. \textunderscore suntagma\textunderscore  + \textunderscore arkhein\textunderscore )}
\end{itemize}
Commandante de uma syntagma, no antigo exército grego.
\section{Syntaxe}
\begin{itemize}
\item {fónica:ce}
\end{itemize}
\begin{itemize}
\item {Grp. gram.:f.}
\end{itemize}
\begin{itemize}
\item {Proveniência:(Lat. \textunderscore syntaxis\textunderscore )}
\end{itemize}
Parte da Grammática que trata da disposição das palavras, da construcção das proposições, da relação lógica das phrases entre si e dos preceitos geraes e particulares que se devam observar para tornar correcto, puro e elegante o estilo e a linguagem.
Livro, em que se expõem as regras da syntaxe.
\section{Syntáxico}
\begin{itemize}
\item {fónica:ci}
\end{itemize}
\begin{itemize}
\item {Grp. gram.:adj.}
\end{itemize}
(V.syntáctico)
\section{Syntaxístico}
\begin{itemize}
\item {fónica:cis}
\end{itemize}
\begin{itemize}
\item {Grp. gram.:adj.}
\end{itemize}
O mesmo que \textunderscore syntáctico\textunderscore . Cf. Filinto, I, 146.
\section{Sýnthese}
\begin{itemize}
\item {Grp. gram.:f.}
\end{itemize}
\begin{itemize}
\item {Proveniência:(Lat. \textunderscore synthesis\textunderscore )}
\end{itemize}
Operação chímica, com que se reúnem corpos simples, para formar corpos compostos, ou com que se reúnem corpos compostos para formar outros de composição mais complexa.
Processo philosóphico, com que se desce dos princípios ás consequências, e das causas aos effeitos.
Quadro, que expõe o conjunto de uma sciência.
Resenha literária ou scientífica.
Demonstração mathemática das proposições, pela simples deducção das que estão já provadas.
Operação cirúrgica, com que se reúnem ou se restituem ao estado primitivo as partes deslocadas ou separadas.
Organização mental de um systema.
Figura grammatical, o mesmo que \textunderscore syllepse\textunderscore .
Entre os antigos, era uma espécie de chlâmide ou rocló que, nos banquetes, se vestia sôbre o vestuário commum, para que êste se não ennodoasse.
\section{Syntheticamente}
\begin{itemize}
\item {Grp. gram.:adv.}
\end{itemize}
De modo synthético.
Em sýnthese; resumidamente.
\section{Synthético}
\begin{itemize}
\item {Grp. gram.:adj.}
\end{itemize}
\begin{itemize}
\item {Proveniência:(Do gr. \textunderscore sunthetikos\textunderscore )}
\end{itemize}
Relativo á sýnthese.
Feito em sýnthese; compendiado, resumido.
\section{Synthetismo}
\begin{itemize}
\item {Grp. gram.:m.}
\end{itemize}
\begin{itemize}
\item {Proveniência:(De \textunderscore sýnthese\textunderscore )}
\end{itemize}
Conjunto das operações necessárias para a sýnthese cirúrgica, isto é, para a reducção de uma fractura, e para a conservar reduzida.
\section{Synthetização}
\begin{itemize}
\item {Grp. gram.:f.}
\end{itemize}
Acto ou effeito de synthetizar.
\section{Synthetizador}
\begin{itemize}
\item {Grp. gram.:adj.}
\end{itemize}
Que synthetiza.
\section{Synthetizar}
\begin{itemize}
\item {Grp. gram.:v. t.}
\end{itemize}
\begin{itemize}
\item {Proveniência:(De \textunderscore synthético\textunderscore )}
\end{itemize}
Tornar synthético, compendiar, resumir.
\section{Syntomia}
\begin{itemize}
\item {Grp. gram.:f.}
\end{itemize}
\begin{itemize}
\item {Proveniência:(Do gr. \textunderscore suntomos\textunderscore )}
\end{itemize}
Exposição abreviada, bosquejo.
Gênero de insectos coleópteros pentâmeros.
\section{Syntonina}
\begin{itemize}
\item {Grp. gram.:f.}
\end{itemize}
\begin{itemize}
\item {Proveniência:(Do gr. \textunderscore sun\textunderscore  + \textunderscore tonos\textunderscore )}
\end{itemize}
Fibrina muscular.
Musculina.
\section{Sýntono}
\begin{itemize}
\item {Grp. gram.:m.}
\end{itemize}
\begin{itemize}
\item {Proveniência:(Lat. \textunderscore syntonum\textunderscore )}
\end{itemize}
Antigo instrumento músico.
\section{Syphilicómio}
\begin{itemize}
\item {Grp. gram.:m.}
\end{itemize}
\begin{itemize}
\item {Proveniência:(De \textunderscore sýphilis\textunderscore  + gr. \textunderscore komein\textunderscore )}
\end{itemize}
Hospital ou dispensário, destinado especialmente ao tratamento da sýphilis.
\section{Syphílide}
\begin{itemize}
\item {Grp. gram.:f.}
\end{itemize}
Affecção cutânea, derivada da sýphilis.
\section{Syphilidophobia}
\begin{itemize}
\item {Grp. gram.:f.}
\end{itemize}
(V.syphilophobia). Cf. \textunderscore Jorn.-do-Comm.\textunderscore , do Rio, de 4-V-901.
\section{Syphiligraphia}
\begin{itemize}
\item {Grp. gram.:f.}
\end{itemize}
\begin{itemize}
\item {Proveniência:(De \textunderscore syphiligrapho\textunderscore )}
\end{itemize}
Tratado ou descripção da sýphilis.
\section{Syphiligráphico}
\begin{itemize}
\item {Grp. gram.:adj.}
\end{itemize}
Relativo á syphiligraphia.
\section{Syphillígrapho}
\begin{itemize}
\item {Grp. gram.:m.}
\end{itemize}
Tratadista de syphiligraphia.
(Do \textunderscore sýphilis\textunderscore  + gr. \textunderscore graphein\textunderscore )
\section{Sýphilis}
\begin{itemize}
\item {Grp. gram.:f.}
\end{itemize}
Doença específica, de natureza venérea, transmittida por contacto ou por herança, e cujos effeitos afectam a constituição do organismo.
(Palavra inventada por Fracastor, num poema latino sôbre doenças venéreas)
\section{Syphilismo}
\begin{itemize}
\item {Grp. gram.:m.}
\end{itemize}
\begin{itemize}
\item {Proveniência:(De \textunderscore sýphilis\textunderscore )}
\end{itemize}
Disposição natural para a syphilização.
\section{Syphilítico}
\begin{itemize}
\item {Grp. gram.:adj.}
\end{itemize}
\begin{itemize}
\item {Grp. gram.:M.}
\end{itemize}
Relativo á sýphilis.
O doente de sýphilis.
\section{Syphilização}
\begin{itemize}
\item {Grp. gram.:f.}
\end{itemize}
Acto ou effeito de syphilizar.
\section{Syphilizar}
\begin{itemize}
\item {Grp. gram.:v. t.}
\end{itemize}
Communicar sýphilis a.
\section{Syphilographia}
\textunderscore f.\textunderscore  (e der.)
O mesmo ou melhór que \textunderscore syphiligraphia\textunderscore , etc.
\section{Syphilóide}
\begin{itemize}
\item {Grp. gram.:f.}
\end{itemize}
\begin{itemize}
\item {Utilização:Med.}
\end{itemize}
\begin{itemize}
\item {Proveniência:(De \textunderscore sýphilis\textunderscore  + gr. \textunderscore eidos\textunderscore )}
\end{itemize}
Erupção cutânea, parecida com a erupção syphilítica.
\section{Syphiloma}
\begin{itemize}
\item {Grp. gram.:m.}
\end{itemize}
\begin{itemize}
\item {Proveniência:(Do gr. \textunderscore suphilis\textunderscore )}
\end{itemize}
Producção neoplástica, de origem syphilítica.
\section{Siríaco}
\begin{itemize}
\item {Grp. gram.:adj.}
\end{itemize}
\begin{itemize}
\item {Grp. gram.:M.}
\end{itemize}
\begin{itemize}
\item {Proveniência:(Lat. \textunderscore syriacus\textunderscore )}
\end{itemize}
Relativo aos Sírios.
Língua antiga, falada pelos Sírios.
\section{Siriarca}
\begin{itemize}
\item {Grp. gram.:m.}
\end{itemize}
\begin{itemize}
\item {Proveniência:(Lat. \textunderscore syriarcha\textunderscore )}
\end{itemize}
Supremo sacerdote da religião dos Sírios.
\section{Siringite}
\begin{itemize}
\item {Grp. gram.:f.}
\end{itemize}
\begin{itemize}
\item {Proveniência:(Lat. \textunderscore syringitis\textunderscore )}
\end{itemize}
Pedra preciosa, mencionada entre os antigos e hoje desconhecida.
\section{Siringodendro}
\begin{itemize}
\item {Grp. gram.:m.}
\end{itemize}
\begin{itemize}
\item {Proveniência:(Do gr. \textunderscore surinx\textunderscore  + \textunderscore dendron\textunderscore )}
\end{itemize}
Planta fóssil dos terrenos anteriores ao cretáceo.
\section{Siringomielia}
\begin{itemize}
\item {Grp. gram.:f.}
\end{itemize}
\begin{itemize}
\item {Utilização:Med.}
\end{itemize}
\begin{itemize}
\item {Proveniência:(Do gr. \textunderscore surinx\textunderscore , \textunderscore suringos\textunderscore  + \textunderscore muelos\textunderscore )}
\end{itemize}
Molestia, caracterizada pela existência de lacunas na substância cinzenta da medula.
\section{Siringotomia}
\begin{itemize}
\item {Grp. gram.:f.}
\end{itemize}
\begin{itemize}
\item {Proveniência:(De \textunderscore siringótomo\textunderscore )}
\end{itemize}
Incisão de uma fístula.
\section{Siringotómio}
\begin{itemize}
\item {Grp. gram.:m.}
\end{itemize}
\begin{itemize}
\item {Proveniência:(Lat. \textunderscore syringotomium\textunderscore )}
\end{itemize}
O mesmo \textunderscore  ou\textunderscore  melhór, mas menos usado, que \textunderscore siringótomo\textunderscore .
\section{Siringótomo}
\begin{itemize}
\item {Grp. gram.:m.}
\end{itemize}
\begin{itemize}
\item {Proveniência:(Do gr. \textunderscore surinx\textunderscore  + \textunderscore tome\textunderscore )}
\end{itemize}
Instrumento cirúrgico, que servia antigamente para a operacão da fístula do ânus.
\section{Sírio}
\begin{itemize}
\item {Grp. gram.:adj.}
\end{itemize}
\begin{itemize}
\item {Grp. gram.:M.}
\end{itemize}
\begin{itemize}
\item {Proveniência:(Lat. \textunderscore syrius\textunderscore )}
\end{itemize}
Relativo á Síria.
Habitante da Síria.
\section{Sirma}
\begin{itemize}
\item {Grp. gram.:m.}
\end{itemize}
\begin{itemize}
\item {Proveniência:(Lat. \textunderscore syrma\textunderscore )}
\end{itemize}
Manto de longa cauda, entre os Gregos.
\section{Siro}
\begin{itemize}
\item {Grp. gram.:m.  e  adj.}
\end{itemize}
\begin{itemize}
\item {Proveniência:(Lat. \textunderscore syrus\textunderscore )}
\end{itemize}
O mesmo que \textunderscore sírio\textunderscore ^3.
\section{Sirfo}
\begin{itemize}
\item {Grp. gram.:m.}
\end{itemize}
\begin{itemize}
\item {Proveniência:(Gr. \textunderscore surphos\textunderscore )}
\end{itemize}
Gênero de insectos dípteros, de corpo cónico e alongado.
\section{Sirtes}
\begin{itemize}
\item {Grp. gram.:m.  ou  f. pl.}
\end{itemize}
\begin{itemize}
\item {Utilização:Fig.}
\end{itemize}
\begin{itemize}
\item {Proveniência:(Lat. \textunderscore syrtes\textunderscore )}
\end{itemize}
Recífes ou bancos de areia.
Perigos.
\section{Síspono}
\begin{itemize}
\item {Grp. gram.:m.}
\end{itemize}
Gênero de plantas leguminosas.
\section{Sissarcose}
\begin{itemize}
\item {Grp. gram.:f.}
\end{itemize}
\begin{itemize}
\item {Utilização:Anat.}
\end{itemize}
\begin{itemize}
\item {Proveniência:(Gr. \textunderscore sussarkosis\textunderscore )}
\end{itemize}
Connexão dos ossos, por meio da carne ou dos músculos.
\section{Sissítia}
\begin{itemize}
\item {Grp. gram.:f.}
\end{itemize}
\begin{itemize}
\item {Proveniência:(Gr. \textunderscore sussitia\textunderscore )}
\end{itemize}
Caldo ou refeição pública, entre os Espartanos.
\section{Sissomático}
\begin{itemize}
\item {Grp. gram.:adj.}
\end{itemize}
Relativo á sissomia.
\section{Sissomia}
\begin{itemize}
\item {Grp. gram.:f.}
\end{itemize}
\begin{itemize}
\item {Proveniência:(Do gr. \textunderscore sun\textunderscore  + \textunderscore soma\textunderscore )}
\end{itemize}
Monstruosidade, caracterizada pela juncção de dois corpos ou dois indivíduos.
\section{Sissomiano}
\begin{itemize}
\item {Grp. gram.:adj.}
\end{itemize}
O mesmo que \textunderscore sissomático\textunderscore .
\section{Sissomo}
\begin{itemize}
\item {Grp. gram.:m.}
\end{itemize}
Monstro, constituído por dois corpos confundidos e como entrelaçados.
(Cp. \textunderscore sissomia\textunderscore )
\section{Sistáltico}
\begin{itemize}
\item {Grp. gram.:adj.}
\end{itemize}
\begin{itemize}
\item {Proveniência:(Gr. \textunderscore sustaltikos\textunderscore )}
\end{itemize}
Relativo á sístole.
\section{Sístase}
\begin{itemize}
\item {Grp. gram.:f.}
\end{itemize}
Uma das subdivisões da antiga milicia grega.
\section{Sistema}
\begin{itemize}
\item {Grp. gram.:m.}
\end{itemize}
\begin{itemize}
\item {Proveniência:(Lat. \textunderscore systema\textunderscore )}
\end{itemize}
Conjunto de partes, coordenadas entre si.
Conjunto de partes similares.
Fórma de governo ou constituição política ou social de um Estado: \textunderscore sistema republicano\textunderscore .
Combinação de partes, por fórma que concorram para certo resultado.
Plano.
Modo de coordenar as noções particulares de uma arte, ciência, etc.
Modo, hábito, uso: \textunderscore o meu sistema de vida\textunderscore .
Método.
Conjunto de leis ou de princípios, que regulam certa ordem de fenómenos: \textunderscore o nosso systema planetário\textunderscore .
Conjunto de intervalos musicaes elementares, compreendidos entre os dois limites sonoros extremos, apreciáveis ao ouvido.
\section{Sistemar}
\begin{itemize}
\item {Grp. gram.:v. t.}
\end{itemize}
(V.sistematizar)
\section{Sistematicamente}
\begin{itemize}
\item {Grp. gram.:adv.}
\end{itemize}
De modo sistemático.
Com sistema; segundo certas regras ou preceitos.
\section{Sistemático}
\begin{itemize}
\item {Grp. gram.:adj.}
\end{itemize}
\begin{itemize}
\item {Proveniência:(Lat. \textunderscore systematicus\textunderscore )}
\end{itemize}
Relativo a sistema ou conforme a um sistema.
Metódico, ordenado.
Que observa um sistema.
\section{Sistematização}
\begin{itemize}
\item {Grp. gram.:f.}
\end{itemize}
Acto ou efeito de sistematizar.
\section{Sistematizar}
\begin{itemize}
\item {Grp. gram.:v. t.}
\end{itemize}
\begin{itemize}
\item {Proveniência:(Do gr. \textunderscore sustema\textunderscore )}
\end{itemize}
Reunir num corpo de doutrina; reduzir a sistema.
\section{Sistematologia}
\begin{itemize}
\item {Grp. gram.:f.}
\end{itemize}
\begin{itemize}
\item {Proveniência:(Do gr. \textunderscore sustema\textunderscore )}
\end{itemize}
História ou tratado dos sistemas.
\section{Sistematológico}
\begin{itemize}
\item {Grp. gram.:adj.}
\end{itemize}
Relativo a sistematologia.
\section{Sistena}
\begin{itemize}
\item {Grp. gram.:f.}
\end{itemize}
Gênero de insectos coleópteros.
\section{Sistilo}
\begin{itemize}
\item {Grp. gram.:m.}
\end{itemize}
\begin{itemize}
\item {Proveniência:(Gr. \textunderscore sustulos\textunderscore )}
\end{itemize}
Construcção arquitectónica em que os intercolúnios são de dois diâmetros ou quatro módulos.
\section{Sistolar}
\begin{itemize}
\item {Grp. gram.:adj.}
\end{itemize}
Relativo á sístole; o mesmo que \textunderscore sistólico\textunderscore .
\section{Sístole}
\begin{itemize}
\item {Grp. gram.:f.}
\end{itemize}
\begin{itemize}
\item {Utilização:Med.}
\end{itemize}
\begin{itemize}
\item {Utilização:Gram.}
\end{itemize}
\begin{itemize}
\item {Proveniência:(Lat. \textunderscore systole\textunderscore )}
\end{itemize}
Estado do coração, em que as fibras musculares dêste órgão estão contraidas.
Figura gramatical, com que uma sílaba longa sôa como breve.
\section{Sistólico}
\begin{itemize}
\item {Grp. gram.:adj.}
\end{itemize}
Relativo á sístole.
\section{Sistrefa}
\begin{itemize}
\item {Grp. gram.:f.}
\end{itemize}
Gênero de plantas apocíneas.
\section{Sistrema}
\begin{itemize}
\item {Grp. gram.:f.}
\end{itemize}
\begin{itemize}
\item {Proveniência:(Gr. \textunderscore sustremma\textunderscore )}
\end{itemize}
Uma das subdivisões da falange, na antiga milicia grega.
\section{Sistrematarca}
\begin{itemize}
\item {Grp. gram.:m.}
\end{itemize}
Comandante de uma sistrema.
\section{Sizetese}
\begin{itemize}
\item {Grp. gram.:f.}
\end{itemize}
\begin{itemize}
\item {Proveniência:(Do gr. \textunderscore sun\textunderscore  + \textunderscore zetein\textunderscore )}
\end{itemize}
Figura de Rètórica, pela qual se começa ou se estabelece uma discussão.
\section{Sizígia}
\begin{itemize}
\item {Grp. gram.:f.}
\end{itemize}
\begin{itemize}
\item {Proveniência:(Lat. \textunderscore syzygia\textunderscore )}
\end{itemize}
Posição do Sol e da Lua, quando êstes astros estão em conjuncção ou em oposição, isto é, no novilúnio e no plenilúnio.
\section{Sizígio}
\begin{itemize}
\item {Grp. gram.:m.}
\end{itemize}
O mesmo que \textunderscore sizígia\textunderscore .
Gênero de plantas mirtáceas.
(Cp. \textunderscore sizígia\textunderscore )
\section{Syphilophobia}
\begin{itemize}
\item {Grp. gram.:f.}
\end{itemize}
\begin{itemize}
\item {Utilização:Med.}
\end{itemize}
\begin{itemize}
\item {Proveniência:(De \textunderscore sýphilis\textunderscore  + gr. \textunderscore phobein\textunderscore )}
\end{itemize}
Terror mórbido das doenças venéreas.
\section{Syracusa}
\begin{itemize}
\item {Grp. gram.:m.}
\end{itemize}
Vinho célebre, produzido pelos vinhedos de Syracusa.
\section{Syracusano}
\begin{itemize}
\item {Grp. gram.:adj.}
\end{itemize}
\begin{itemize}
\item {Grp. gram.:M.}
\end{itemize}
\begin{itemize}
\item {Proveniência:(Lat. \textunderscore syracusanus\textunderscore )}
\end{itemize}
Relativo a Syracusa.
Habitante de Syracusa.
\section{Syrenópside}
\begin{itemize}
\item {Grp. gram.:m.}
\end{itemize}
Gênero de plantas crucíferas.
\section{Syríaco}
\begin{itemize}
\item {Grp. gram.:adj.}
\end{itemize}
\begin{itemize}
\item {Grp. gram.:M.}
\end{itemize}
\begin{itemize}
\item {Proveniência:(Lat. \textunderscore syriacus\textunderscore )}
\end{itemize}
Relativo aos Sýrios.
Língua antiga, falada pelos Sýrios.
\section{Syriarcha}
\begin{itemize}
\item {fónica:ca}
\end{itemize}
\begin{itemize}
\item {Grp. gram.:m.}
\end{itemize}
\begin{itemize}
\item {Proveniência:(Lat. \textunderscore syriarcha\textunderscore )}
\end{itemize}
Supremo sacerdote da religião dos Sýrios.
\section{Syringite}
\begin{itemize}
\item {Grp. gram.:f.}
\end{itemize}
\begin{itemize}
\item {Proveniência:(Lat. \textunderscore syringitis\textunderscore )}
\end{itemize}
Pedra preciosa, mencionada entre os antigos e hoje desconhecida.
\section{Syringodendro}
\begin{itemize}
\item {Grp. gram.:m.}
\end{itemize}
\begin{itemize}
\item {Proveniência:(Do gr. \textunderscore surinx\textunderscore  + \textunderscore dendron\textunderscore )}
\end{itemize}
Planta fóssil dos terrenos anteriores ao cretáceo.
\section{Syringomyelia}
\begin{itemize}
\item {Grp. gram.:f.}
\end{itemize}
\begin{itemize}
\item {Utilização:Med.}
\end{itemize}
\begin{itemize}
\item {Proveniência:(Do gr. \textunderscore surinx\textunderscore , \textunderscore suringos\textunderscore  + \textunderscore muelos\textunderscore )}
\end{itemize}
Molestia, caracterizada pela existência de lacunas na substância cinzenta da medulla.
\section{Syringotomia}
\begin{itemize}
\item {Grp. gram.:f.}
\end{itemize}
\begin{itemize}
\item {Proveniência:(De \textunderscore syringótomo\textunderscore )}
\end{itemize}
Incisão de uma fístula.
\section{Syringotómio}
\begin{itemize}
\item {Grp. gram.:m.}
\end{itemize}
\begin{itemize}
\item {Proveniência:(Lat. \textunderscore syringotomium\textunderscore )}
\end{itemize}
O mesmo ou melhór, mas menos usado, que \textunderscore syringótomo\textunderscore .
\section{Syringótomo}
\begin{itemize}
\item {Grp. gram.:m.}
\end{itemize}
\begin{itemize}
\item {Proveniência:(Do gr. \textunderscore surinx\textunderscore  + \textunderscore tome\textunderscore )}
\end{itemize}
Instrumento cirúrgico, que servia antigamente para a operacão da fístula do ânus.
\section{Sýrio}
\begin{itemize}
\item {Grp. gram.:adj.}
\end{itemize}
\begin{itemize}
\item {Grp. gram.:M.}
\end{itemize}
\begin{itemize}
\item {Proveniência:(Lat. \textunderscore syrius\textunderscore )}
\end{itemize}
Relativo á Sýria.
Habitante da Sýria.
\section{Syrma}
\begin{itemize}
\item {Grp. gram.:m.}
\end{itemize}
\begin{itemize}
\item {Proveniência:(Lat. \textunderscore syrma\textunderscore )}
\end{itemize}
Manto de longa cauda, entre os Gregos.
\section{Syro}
\begin{itemize}
\item {Grp. gram.:m.  e  adj.}
\end{itemize}
\begin{itemize}
\item {Proveniência:(Lat. \textunderscore syrus\textunderscore )}
\end{itemize}
O mesmo que \textunderscore sýrio\textunderscore .
\section{Syro-árabe}
\begin{itemize}
\item {Grp. gram.:adj.}
\end{itemize}
Relativo á Sýria e á Arábia.
Diz-se especialmente das línguas semíticas.
\section{Syro-arábico}
\begin{itemize}
\item {Grp. gram.:adj.}
\end{itemize}
O mesmo que \textunderscore syro-árabe\textunderscore . Cf. Latino, \textunderscore Elogios\textunderscore , 57.
\section{Syro-caldaico}
\begin{itemize}
\item {Grp. gram.:m.}
\end{itemize}
Um dos dialectos armênios.
\section{Syro-macedónio}
\begin{itemize}
\item {Grp. gram.:adj.}
\end{itemize}
Relativo ao império grego da Sýria.
\section{Syrpho}
\begin{itemize}
\item {Grp. gram.:m.}
\end{itemize}
\begin{itemize}
\item {Proveniência:(Gr. \textunderscore surphos\textunderscore )}
\end{itemize}
Gênero de insectos dípteros, de corpo cónico e alongado.
\section{Syrtes}
\begin{itemize}
\item {Utilização:Fig.}
\end{itemize}
\begin{itemize}
\item {Proveniência:(Lat. \textunderscore syrtes\textunderscore )}
\end{itemize}
Recífes ou bancos de areia.
Perigos.
\section{Sýspono}
\begin{itemize}
\item {Grp. gram.:m.}
\end{itemize}
Gênero de plantas leguminosas.
\section{Syssarcose}
\begin{itemize}
\item {Grp. gram.:f.}
\end{itemize}
\begin{itemize}
\item {Utilização:Anat.}
\end{itemize}
\begin{itemize}
\item {Proveniência:(Gr. \textunderscore sussarkosis\textunderscore )}
\end{itemize}
Connexão dos ossos, por meio da carne ou dos músculos.
\section{Syssítia}
\begin{itemize}
\item {Grp. gram.:f.}
\end{itemize}
\begin{itemize}
\item {Proveniência:(Gr. \textunderscore sussitia\textunderscore )}
\end{itemize}
Caldo ou refeição pública, entre os Espartanos.
\section{Syssomático}
\begin{itemize}
\item {Grp. gram.:adj.}
\end{itemize}
Relativo á syssomia.
\section{Syssomia}
\begin{itemize}
\item {Grp. gram.:f.}
\end{itemize}
\begin{itemize}
\item {Proveniência:(Do gr. \textunderscore sun\textunderscore  + \textunderscore soma\textunderscore )}
\end{itemize}
Monstruosidade, caracterizada pela juncção de dois corpos ou dois indivíduos.
\section{Syssomiano}
\begin{itemize}
\item {Grp. gram.:adj.}
\end{itemize}
O mesmo que \textunderscore syssomático\textunderscore .
\section{Syssomo}
\begin{itemize}
\item {Grp. gram.:m.}
\end{itemize}
Monstro, constituído por dois corpos confundidos e como entrelaçados.
(Cp. \textunderscore syssomia\textunderscore )
\section{Systáltico}
\begin{itemize}
\item {Grp. gram.:adj.}
\end{itemize}
\begin{itemize}
\item {Proveniência:(Gr. \textunderscore sustaltikos\textunderscore )}
\end{itemize}
Relativo á sýstole.
\section{Sýstase}
\begin{itemize}
\item {Grp. gram.:f.}
\end{itemize}
Uma das subdivisões da antiga milicia grega.
\section{Systema}
\begin{itemize}
\item {Grp. gram.:m.}
\end{itemize}
\begin{itemize}
\item {Proveniência:(Lat. \textunderscore systema\textunderscore )}
\end{itemize}
Conjunto de partes, coordenadas entre si.
Conjunto de partes similares.
Fórma de governo ou constituição política ou social de um Estado: \textunderscore systema republicano\textunderscore .
Combinação de partes, por fórma que concorram para certo resultado.
Plano.
Modo de coordenar as noções particulares de uma arte, sciência, etc.
Modo, hábito, uso: \textunderscore o meu systema de vida\textunderscore .
Méthodo.
Conjunto de leis ou de princípios, que regulam certa ordem de phenómenos: \textunderscore o nosso systema planetário\textunderscore .
Conjunto de intervallos musicaes elementares, comprehendidos entre os dois limites sonoros extremos, apreciáveis ao ouvido.
\section{Systemar}
\begin{itemize}
\item {Grp. gram.:v. t.}
\end{itemize}
(V.systematizar)
\section{Systematicamente}
\begin{itemize}
\item {Grp. gram.:adv.}
\end{itemize}
De modo systemático.
Com systema; segundo certas regras ou preceitos.
\section{Systemático}
\begin{itemize}
\item {Grp. gram.:adj.}
\end{itemize}
\begin{itemize}
\item {Proveniência:(Lat. \textunderscore systematicus\textunderscore )}
\end{itemize}
Relativo a systema ou conforme a um systema.
Methódico, ordenado.
Que observa um systema.
\section{Systematização}
\begin{itemize}
\item {Grp. gram.:f.}
\end{itemize}
Acto ou effeito de systematizar.
\section{Systematizar}
\begin{itemize}
\item {Grp. gram.:v. t.}
\end{itemize}
\begin{itemize}
\item {Proveniência:(Do gr. \textunderscore sustema\textunderscore )}
\end{itemize}
Reunir num corpo de doutrina; reduzir a systema.
\section{Systematologia}
\begin{itemize}
\item {Grp. gram.:f.}
\end{itemize}
\begin{itemize}
\item {Proveniência:(Do gr. \textunderscore sustema\textunderscore )}
\end{itemize}
História ou tratado dos systemas.
\section{Systematológico}
\begin{itemize}
\item {Grp. gram.:adj.}
\end{itemize}
Relativo a systematologia.
\section{Systena}
\begin{itemize}
\item {Grp. gram.:f.}
\end{itemize}
Gênero de insectos coleópteros.
\section{Systolar}
\begin{itemize}
\item {Grp. gram.:adj.}
\end{itemize}
Relativo á sýstole; o mesmo que \textunderscore systólico\textunderscore .
\section{Sýstole}
\begin{itemize}
\item {Grp. gram.:f.}
\end{itemize}
\begin{itemize}
\item {Utilização:Med.}
\end{itemize}
\begin{itemize}
\item {Utilização:Gram.}
\end{itemize}
\begin{itemize}
\item {Proveniência:(Lat. \textunderscore systole\textunderscore )}
\end{itemize}
Estado do coração, em que as fibras musculares dêste órgão estão contrahidas.
Figura grammatical, com que uma sýllaba longa sôa como breve.
\section{Systólico}
\begin{itemize}
\item {Grp. gram.:adj.}
\end{itemize}
Relativo á sýstole.
\section{Systrepha}
\begin{itemize}
\item {Grp. gram.:f.}
\end{itemize}
Gênero de plantas apocýneas.
\section{Systrema}
\begin{itemize}
\item {Grp. gram.:f.}
\end{itemize}
\begin{itemize}
\item {Proveniência:(Gr. \textunderscore sustremma\textunderscore )}
\end{itemize}
Uma das subdivisões da phalange, na antiga milicia grega.
\section{Systrematarcha}
\begin{itemize}
\item {fónica:ca}
\end{itemize}
\begin{itemize}
\item {Grp. gram.:m.}
\end{itemize}
Commandante de uma systrema.
\section{Systylo}
\begin{itemize}
\item {Grp. gram.:m.}
\end{itemize}
\begin{itemize}
\item {Proveniência:(Gr. \textunderscore sustulos\textunderscore )}
\end{itemize}
Construcção architectónica em que os intercolúmnios são de dois diâmetros ou quatro módulos.
\section{Syzetese}
\begin{itemize}
\item {Grp. gram.:f.}
\end{itemize}
\begin{itemize}
\item {Proveniência:(Do gr. \textunderscore sun\textunderscore  + \textunderscore zetein\textunderscore )}
\end{itemize}
Figura de Rhètórica, pela qual se começa ou se estabelece uma discussão.
\section{Syzýgia}
\begin{itemize}
\item {Grp. gram.:f.}
\end{itemize}
\begin{itemize}
\item {Proveniência:(Lat. \textunderscore syzygia\textunderscore )}
\end{itemize}
Posição do Sol e da Lua, quando êstes astros estão em conjuncção ou em opposição, isto é, no novilúnio e no plenilúnio.
\section{Syzýgio}
\begin{itemize}
\item {Grp. gram.:m.}
\end{itemize}
\end{document}