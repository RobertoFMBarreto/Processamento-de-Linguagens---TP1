
\begin{itemize}
\item {Proveniência: }
\end{itemize}\documentclass{article}
\usepackage[portuguese]{babel}
\title{F}
\begin{document}
Introduzir nas vísceras; entranhar.
\section{Faca}
\begin{itemize}
\item {Grp. gram.:f.}
\end{itemize}
\begin{itemize}
\item {Proveniência:(Do gr. \textunderscore phake\textunderscore )}
\end{itemize}
Gênero de plantas leguminosas.
\section{Facélia}
\begin{itemize}
\item {Grp. gram.:f.}
\end{itemize}
\begin{itemize}
\item {Proveniência:(Do gr. \textunderscore phakellos\textunderscore )}
\end{itemize}
Gênero de plantas da América do Norte.
\section{Facocele}
\begin{itemize}
\item {Grp. gram.:f.}
\end{itemize}
\begin{itemize}
\item {Proveniência:(Do gr. \textunderscore phakos\textunderscore  + \textunderscore kele\textunderscore )}
\end{itemize}
Hérnia do cristalino do ôlho.
\section{Facohydropisia}
\begin{itemize}
\item {Grp. gram.:f.}
\end{itemize}
\begin{itemize}
\item {Utilização:Med.}
\end{itemize}
Hydropisia da cápsula do crystallino.
\section{Facóide}
\begin{itemize}
\item {Grp. gram.:adj.}
\end{itemize}
\begin{itemize}
\item {Proveniência:(Lat. \textunderscore phacoides\textunderscore )}
\end{itemize}
Que tem fórma de lentilha.
\section{Facólito}
\begin{itemize}
\item {Grp. gram.:m.}
\end{itemize}
\begin{itemize}
\item {Utilização:Miner.}
\end{itemize}
\begin{itemize}
\item {Proveniência:(Do gr. \textunderscore phakos\textunderscore  + \textunderscore lithos\textunderscore )}
\end{itemize}
Zeólito, de base de potassa, soda e cal.
\section{Facomalacia}
\begin{itemize}
\item {Grp. gram.:f.}
\end{itemize}
\begin{itemize}
\item {Utilização:Med.}
\end{itemize}
\begin{itemize}
\item {Proveniência:(Do gr. \textunderscore phakos\textunderscore  + \textunderscore malakia\textunderscore )}
\end{itemize}
Amolecimento do cristalino.
\section{Facómetro}
\begin{itemize}
\item {Grp. gram.:m.}
\end{itemize}
\begin{itemize}
\item {Proveniência:(Do gr. \textunderscore phakos\textunderscore  + \textunderscore metron\textunderscore )}
\end{itemize}
Instrumento, para medir as lentes ou determinar-lhes o foco.
\section{Faconina}
\begin{itemize}
\item {Grp. gram.:f.}
\end{itemize}
\begin{itemize}
\item {Proveniência:(Do gr. \textunderscore phakos\textunderscore )}
\end{itemize}
Substância particular, que se acha no crystallino do ôlho.
\section{Facosclerose}
\begin{itemize}
\item {Grp. gram.:f.}
\end{itemize}
\begin{itemize}
\item {Utilização:Med.}
\end{itemize}
\begin{itemize}
\item {Proveniência:(Do gr. \textunderscore phakos\textunderscore  + \textunderscore skleros\textunderscore )}
\end{itemize}
Endurecimento do cristalino.
\section{Facoscopia}
\begin{itemize}
\item {Grp. gram.:f.}
\end{itemize}
\begin{itemize}
\item {Utilização:Med.}
\end{itemize}
\begin{itemize}
\item {Proveniência:(Do gr. \textunderscore phakos\textunderscore  + \textunderscore skopein\textunderscore )}
\end{itemize}
Exploração subjectiva ou pessoal dos meios do globo ocular.
\section{Faetonte}
\begin{itemize}
\item {Grp. gram.:m.}
\end{itemize}
\begin{itemize}
\item {Proveniência:(De \textunderscore Phaetonte\textunderscore , n. p. myth. de um filho e cocheiro de Júpiter)}
\end{itemize}
Pequena carruagem de quatro rodas, ligeira e descoberta:«\textunderscore ...parar-lhe á porta um faetonte aéreo.\textunderscore »Filinto, XI, 173.
\section{Fagedênico}
\begin{itemize}
\item {Grp. gram.:adj.}
\end{itemize}
\begin{itemize}
\item {Utilização:Med.}
\end{itemize}
\begin{itemize}
\item {Proveniência:(Gr. \textunderscore phagedainikos\textunderscore )}
\end{itemize}
Diz-se da substância, que corrói a carne morta.
Diz-se da água, que é uma solução de deutocloreto de mercúrio em água de cal.
Diz-se das úlceras corrosivas.
\section{Fagedenismo}
\begin{itemize}
\item {Grp. gram.:m.}
\end{itemize}
Estado ou qualidade de fagedênico.
\section{Fagocitos}
\begin{itemize}
\item {Grp. gram.:m. pl.}
\end{itemize}
\begin{itemize}
\item {Proveniência:(Do gr. \textunderscore phagein\textunderscore  + \textunderscore kutos\textunderscore )}
\end{itemize}
Micróbios benéficos, que destroem ou absorvem os prejudiciaes.
\section{Fagocitose}
\begin{itemize}
\item {Grp. gram.:f.}
\end{itemize}
\begin{itemize}
\item {Proveniência:(Do gr. \textunderscore phagein\textunderscore  + \textunderscore kutos\textunderscore )}
\end{itemize}
Destruição dos micróbios por meio de certas células vivas do organismo, que absorvem as bactérias e as digerem.
\section{Falacrose}
\begin{itemize}
\item {Grp. gram.:f.}
\end{itemize}
\begin{itemize}
\item {Utilização:Med.}
\end{itemize}
\begin{itemize}
\item {Proveniência:(Gr. \textunderscore phalakrosis\textunderscore )}
\end{itemize}
Quéda dos cabelos; calvície.
\section{Falangarquia}
\begin{itemize}
\item {Grp. gram.:f.}
\end{itemize}
Falange elementar, que, entre os gregos antigos, devia teoricamente sêr composta de 256 homens de frente, por 16 de fundo.
\section{Falange}
\begin{itemize}
\item {Grp. gram.:f.}
\end{itemize}
\begin{itemize}
\item {Utilização:Anat.}
\end{itemize}
\begin{itemize}
\item {Utilização:Fig.}
\end{itemize}
\begin{itemize}
\item {Proveniência:(Lat. \textunderscore phalanx\textunderscore , \textunderscore phalangis\textunderscore )}
\end{itemize}
Nome, que os Gregos davam á sua ínfantaria.
Corpo de tropas.
Comuna societária, no sistema de Fourier.
Cada um dos ossos dos dedos, especialmente o ôsso que se articula com o metacarpo.
Multidão.
\section{Falangeal}
\begin{itemize}
\item {Grp. gram.:adj.}
\end{itemize}
Relativo ás falanges dos dedos.
\section{Falangeano}
\begin{itemize}
\item {Grp. gram.:adj.}
\end{itemize}
O mesmo que \textunderscore falangeal\textunderscore .
\section{Falangeta}
\begin{itemize}
\item {fónica:gê}
\end{itemize}
\begin{itemize}
\item {Grp. gram.:f.}
\end{itemize}
\begin{itemize}
\item {Utilização:Anat.}
\end{itemize}
Cada uma das últimas falanges dos dedos ou cada uma das falanges que têm as unhas.
\section{Falanginha}
\begin{itemize}
\item {Grp. gram.:f.}
\end{itemize}
\begin{itemize}
\item {Utilização:Anat.}
\end{itemize}
Cada uma das falanges médias dos dedos em que há três.
\section{Falângio}
\begin{itemize}
\item {Grp. gram.:m.}
\end{itemize}
Planta ornamental.
O mesmo que \textunderscore falangita\textunderscore .
\section{Falangita}
\begin{itemize}
\item {Grp. gram.:m.}
\end{itemize}
\begin{itemize}
\item {Proveniência:(Lat. \textunderscore phalangita\textunderscore )}
\end{itemize}
Soldado de uma falange, nos exércitos gregos.
\section{Falansterianismo}
\begin{itemize}
\item {Grp. gram.:m.}
\end{itemize}
O mesmo que \textunderscore falansterismo\textunderscore .
\section{Falansteriano}
\begin{itemize}
\item {Grp. gram.:m.  e  adj.}
\end{itemize}
\begin{itemize}
\item {Proveniência:(De \textunderscore falanstério\textunderscore )}
\end{itemize}
O que habita num falanstério; sectário de Fourier.
\section{Falanstério}
\begin{itemize}
\item {Grp. gram.:m.}
\end{itemize}
\begin{itemize}
\item {Proveniência:(De \textunderscore falange\textunderscore )}
\end{itemize}
Povoação societária, regida pelo sistema de Fourier.
\section{Falansterismo}
\begin{itemize}
\item {Grp. gram.:m.}
\end{itemize}
Conjunto das doutrinas que devem praticar-se no falanstério.
\section{Falarídeas}
\begin{itemize}
\item {Grp. gram.:f. pl.}
\end{itemize}
\begin{itemize}
\item {Utilização:Bot.}
\end{itemize}
\begin{itemize}
\item {Proveniência:(Do gr. \textunderscore phalaris\textunderscore , painço)}
\end{itemize}
Tríbo de gramíneas.
\section{Falécio}
\begin{itemize}
\item {Grp. gram.:m.}
\end{itemize}
\begin{itemize}
\item {Proveniência:(Lat. \textunderscore phalaecium\textunderscore )}
\end{itemize}
O mesmo ou melhor que \textunderscore falêucio\textunderscore .
\section{Falena}
\begin{itemize}
\item {Grp. gram.:f.}
\end{itemize}
\begin{itemize}
\item {Proveniência:(Do gr. \textunderscore phalaina\textunderscore )}
\end{itemize}
Espécie de borboleta nocturna.
\section{Falagogia}
\begin{itemize}
\item {Grp. gram.:f.}
\end{itemize}
\begin{itemize}
\item {Proveniência:(Do gr. \textunderscore phallos\textunderscore  + \textunderscore agein\textunderscore )}
\end{itemize}
Festa grega, em que o falo era conduzido em procissão.
\section{Falagónias}
\begin{itemize}
\item {Grp. gram.:f. pl.}
\end{itemize}
Antigas festas gregas em honra de Priapo.
(Cp. \textunderscore falofórias\textunderscore )
\section{Fálera}
\begin{itemize}
\item {Grp. gram.:f.}
\end{itemize}
\begin{itemize}
\item {Proveniência:(Lat. \textunderscore phalerae\textunderscore )}
\end{itemize}
Colar de oiro e prata, usado por patrícios e guerreiros, entre os antigos Romanos.
\section{Faleríneas}
\begin{itemize}
\item {Grp. gram.:f. pl.}
\end{itemize}
\begin{itemize}
\item {Utilização:Bot.}
\end{itemize}
\begin{itemize}
\item {Proveniência:(Do gr. \textunderscore phaleros\textunderscore )}
\end{itemize}
Tríbo de timeliáceas.
\section{Falêucio}
\begin{itemize}
\item {Grp. gram.:m.  e  adj.}
\end{itemize}
\begin{itemize}
\item {Proveniência:(Fr. \textunderscore phaleuce\textunderscore )}
\end{itemize}
Verso de cinco pés, entre os Gregos e Romanos.
\section{Falisco}
\begin{itemize}
\item {Grp. gram.:m.}
\end{itemize}
\begin{itemize}
\item {Proveniência:(Lat. \textunderscore phaliscus\textunderscore )}
\end{itemize}
Verso latino de quatro pés, sendo dáctilos os três primeiros e espondeu o último.
\section{Fálicas}
\begin{itemize}
\item {Grp. gram.:f. pl.}
\end{itemize}
\begin{itemize}
\item {Proveniência:(De \textunderscore fálico\textunderscore )}
\end{itemize}
O mesmo que \textunderscore falofórias\textunderscore .
\section{Falicismo}
\begin{itemize}
\item {Grp. gram.:m.}
\end{itemize}
Culto fálico.
(Cp. \textunderscore fálico\textunderscore )
\section{Fálico}
\begin{itemize}
\item {Grp. gram.:adj.}
\end{itemize}
Relativo ao falo ou ao seu culto.
\section{Falite}
\begin{itemize}
\item {Grp. gram.:f.}
\end{itemize}
\begin{itemize}
\item {Utilização:Med.}
\end{itemize}
\begin{itemize}
\item {Proveniência:(Do gr. \textunderscore phallos\textunderscore )}
\end{itemize}
Inflamação do pênis.
\section{Falo}
\begin{itemize}
\item {Grp. gram.:m.}
\end{itemize}
\begin{itemize}
\item {Proveniência:(Do gr. \textunderscore phallos\textunderscore )}
\end{itemize}
Representação do membro viril, adorada entre os antigos, como símbolo da fecundidade da natureza, e correspondente ao linga dos Índios.
\section{Falodinia}
\begin{itemize}
\item {Grp. gram.:f.}
\end{itemize}
\begin{itemize}
\item {Utilização:Med.}
\end{itemize}
\begin{itemize}
\item {Proveniência:(Do gr. \textunderscore phallos\textunderscore  + \textunderscore odune\textunderscore )}
\end{itemize}
Dôr no pênis.
\section{Falofórias}
\begin{itemize}
\item {Grp. gram.:f. pl.}
\end{itemize}
\begin{itemize}
\item {Proveniência:(De \textunderscore falóforo\textunderscore )}
\end{itemize}
Festas pagans em honra do falo.
\section{Falóforo}
\begin{itemize}
\item {Grp. gram.:m.}
\end{itemize}
\begin{itemize}
\item {Proveniência:(Do gr. \textunderscore phallos\textunderscore  + \textunderscore phorein\textunderscore )}
\end{itemize}
Sacerdote grego, que em procissões ou dias de festas transportava o falo.
\section{Falóides}
\begin{itemize}
\item {Grp. gram.:m. pl.}
\end{itemize}
\begin{itemize}
\item {Utilização:Bot.}
\end{itemize}
\begin{itemize}
\item {Proveniência:(Do gr. \textunderscore phallos\textunderscore  + \textunderscore eidos\textunderscore )}
\end{itemize}
Cogumelos, que formam uma secção no sistema de Broguiart.
\section{Falorragia}
\begin{itemize}
\item {Grp. gram.:f.}
\end{itemize}
\begin{itemize}
\item {Utilização:Med.}
\end{itemize}
\begin{itemize}
\item {Proveniência:(Do gr. \textunderscore phallos\textunderscore  + \textunderscore rhagein\textunderscore )}
\end{itemize}
Hemorragía á superficie do pênis.
\section{Faneranto}
\begin{itemize}
\item {Grp. gram.:adj.}
\end{itemize}
\begin{itemize}
\item {Utilização:Bot.}
\end{itemize}
\begin{itemize}
\item {Proveniência:(Do gr. \textunderscore phaneros\textunderscore  + \textunderscore anthos\textunderscore )}
\end{itemize}
Que tem flôres aparentes.
\section{Fânero}
\begin{itemize}
\item {Grp. gram.:m.}
\end{itemize}
\begin{itemize}
\item {Utilização:Anat.}
\end{itemize}
\begin{itemize}
\item {Proveniência:(Gr. \textunderscore phaneros\textunderscore )}
\end{itemize}
Qualquer produção visível e persistente á superfície da pele, como os pêlos, cornos, etc.
\section{Fanero...}
\begin{itemize}
\item {Grp. gram.:pref.}
\end{itemize}
\begin{itemize}
\item {Proveniência:(Gr. \textunderscore phaneros\textunderscore )}
\end{itemize}
(designativo de \textunderscore manifesto\textunderscore , \textunderscore apparente\textunderscore )
\section{Fanerocarpo}
\begin{itemize}
\item {Grp. gram.:adj.}
\end{itemize}
\begin{itemize}
\item {Utilização:Bot.}
\end{itemize}
\begin{itemize}
\item {Proveniência:(Do gr. \textunderscore phaneros\textunderscore  + \textunderscore karpos\textunderscore )}
\end{itemize}
Que tem aparentes os frutos ou os corpúsculos reproductores.
\section{Fanerocotyledóneas}
\begin{itemize}
\item {Grp. gram.:f. pl.}
\end{itemize}
Plantas, cujos cotylédones são apparentes ou fáceis de distinguir.
O mesmo que \textunderscore dicotyledóneas\textunderscore .
\section{Fanerogamia}
\begin{itemize}
\item {Grp. gram.:f.}
\end{itemize}
\begin{itemize}
\item {Utilização:Bot.}
\end{itemize}
\begin{itemize}
\item {Proveniência:(Do gr. \textunderscore phaneros\textunderscore  + \textunderscore gamos\textunderscore )}
\end{itemize}
Estado de uma planta ou animal, que tem os órgãos sexuaes aparentes.
\section{Fanerogâmicas}
\begin{itemize}
\item {Grp. gram.:f. pl.}
\end{itemize}
\begin{itemize}
\item {Utilização:Bot.}
\end{itemize}
Grande divisão do reino vegetal, que abrange todas as espécies que têm órgãos sexuaes aparentes.
(Fem. pl. de \textunderscore fanerogâmico\textunderscore )
\section{Fanerogâmico}
\begin{itemize}
\item {Grp. gram.:adj.}
\end{itemize}
\begin{itemize}
\item {Proveniência:(De \textunderscore fanerogamia\textunderscore )}
\end{itemize}
Que tem órgãos sexuaes aparentes, (falando-se de plantas).
\section{Fanerógamo}
\begin{itemize}
\item {Grp. gram.:adj.}
\end{itemize}
O mesmo que \textunderscore fanerogâmico\textunderscore .
\section{Faneróforo}
\begin{itemize}
\item {Grp. gram.:adj.}
\end{itemize}
\begin{itemize}
\item {Utilização:Anat.}
\end{itemize}
\begin{itemize}
\item {Proveniência:(Do gr. \textunderscore phaneros\textunderscore  + \textunderscore phoros\textunderscore )}
\end{itemize}
Que tem fâneros.
\section{Faraó}
\begin{itemize}
\item {Grp. gram.:m.}
\end{itemize}
\begin{itemize}
\item {Proveniência:(Lat. \textunderscore pharao\textunderscore , do hebr.)}
\end{itemize}
Título comum a soberanos do antigo Egipto.
\section{Faraónico}
\begin{itemize}
\item {Grp. gram.:adj.}
\end{itemize}
Relativo aos faraós ou ao seu tempo.
\section{Farisaico}
\begin{itemize}
\item {Grp. gram.:adj.}
\end{itemize}
\begin{itemize}
\item {Utilização:Fig.}
\end{itemize}
\begin{itemize}
\item {Proveniência:(Lat. \textunderscore pharisaicus\textunderscore )}
\end{itemize}
Relativo a fariseu; próprio de fariseu.
Hipócrita.
\section{Farisaísmo}
\begin{itemize}
\item {Grp. gram.:m.}
\end{itemize}
Carácter dos fariseus; hipocrisia.
\section{Fariseu}
\begin{itemize}
\item {Grp. gram.:m.}
\end{itemize}
\begin{itemize}
\item {Utilização:Fig.}
\end{itemize}
\begin{itemize}
\item {Utilização:Pop.}
\end{itemize}
\begin{itemize}
\item {Proveniência:(Lat. \textunderscore pharisaeus\textunderscore )}
\end{itemize}
Membro de uma seita judaica, caracterizada por ostentar grande santidade exterior.
Aquele que aparenta santidade, não a tendo.
Indivíduo, cujo aspecto repelente denota má índole.
\section{Farmaceuta}
\begin{itemize}
\item {Grp. gram.:m.}
\end{itemize}
O mesmo que \textunderscore farmacêutico\textunderscore .
\section{Farmacêutico}
\begin{itemize}
\item {Grp. gram.:adj.}
\end{itemize}
\begin{itemize}
\item {Grp. gram.:M.}
\end{itemize}
\begin{itemize}
\item {Proveniência:(Lat. \textunderscore pharmaceuticus\textunderscore )}
\end{itemize}
Relativo a farmácia.
Aquele que exerce a farmácia; boticário.
\section{Farmácia}
\begin{itemize}
\item {Grp. gram.:f.}
\end{itemize}
\begin{itemize}
\item {Proveniência:(Gr. \textunderscore pharmakeia\textunderscore )}
\end{itemize}
Arte de preparar medicamentos e de conhecer e conservar as drogas simples.
Estabelecimento, em que se preparam ou vendem medicamentos.
Botica.
Profissão de farmacêutico.
Colecção de medicamentos.--A pronúncia exacta seria \textunderscore farmacía\textunderscore , que se não usa.
\section{Farmacoco}
\begin{itemize}
\item {fónica:cô}
\end{itemize}
\begin{itemize}
\item {Grp. gram.:m.}
\end{itemize}
\begin{itemize}
\item {Utilização:T. de Coímbra}
\end{itemize}
Estudante de farmácia; farmacopola.
(Cp. \textunderscore farmácia\textunderscore )
\section{Farmacodinamia}
\begin{itemize}
\item {Grp. gram.:f.}
\end{itemize}
\begin{itemize}
\item {Proveniência:(Do gr. \textunderscore pharmakon\textunderscore  + \textunderscore dunamis\textunderscore )}
\end{itemize}
Estudo dos efeitos fisiológicos dos medicamentos, ou da acção destes no organismo em estado de saúde. Cf. \textunderscore Pharmacopeia Port.\textunderscore 
\section{Farmacognosia}
\begin{itemize}
\item {Grp. gram.:f.}
\end{itemize}
Estudo das drogas, antes de preparadas na farmácia. Cf. \textunderscore Pharmacopeia Port.\textunderscore 
\section{Farmacografia}
\begin{itemize}
\item {Grp. gram.:f.}
\end{itemize}
\begin{itemize}
\item {Proveniência:(Do gr. \textunderscore pharmakon\textunderscore  + \textunderscore graphein\textunderscore )}
\end{itemize}
Tratado das substâncias medicinaes.
\section{Farmacográfico}
\begin{itemize}
\item {Grp. gram.:adj.}
\end{itemize}
Relativo á farmacografia.
\section{Farmacólito}
\begin{itemize}
\item {Grp. gram.:m.}
\end{itemize}
\begin{itemize}
\item {Utilização:Chím.}
\end{itemize}
\begin{itemize}
\item {Proveniência:(Do gr. \textunderscore pharmakon\textunderscore  + \textunderscore lithos\textunderscore )}
\end{itemize}
Cal arseniatada da Alemanha.
\section{Farmacologia}
\begin{itemize}
\item {Grp. gram.:f.}
\end{itemize}
\begin{itemize}
\item {Proveniência:(Do gr. \textunderscore pharmakon\textunderscore  + \textunderscore logos\textunderscore )}
\end{itemize}
Parte da matéria médica, que tem por objecto fazer conhecer os medicamentos e ensinar a aplicação deles.
\section{Farmacológico}
\begin{itemize}
\item {Grp. gram.:adj.}
\end{itemize}
Relativo á farmacologia.
\section{Farmacopéa}
\begin{itemize}
\item {Grp. gram.:f.}
\end{itemize}
\begin{itemize}
\item {Proveniência:(Gr. \textunderscore pharmacopeia\textunderscore )}
\end{itemize}
Livro, que ensina a prepararar e compôr medicamentos; tratado á cêrca de medicamentos.
\section{Farmacopeia}
\begin{itemize}
\item {Grp. gram.:f.}
\end{itemize}
\begin{itemize}
\item {Proveniência:(Gr. \textunderscore pharmacopeia\textunderscore )}
\end{itemize}
Livro, que ensina a prepararar e compôr medicamentos; tratado á cêrca de medicamentos.
\section{Farmacopola}
\begin{itemize}
\item {Grp. gram.:m.}
\end{itemize}
\begin{itemize}
\item {Utilização:Burl.}
\end{itemize}
\begin{itemize}
\item {Proveniência:(Lat. \textunderscore pharmacopola\textunderscore )}
\end{itemize}
Boticário.
Charlatão.
\section{Farmacopolia}
\begin{itemize}
\item {Grp. gram.:f.}
\end{itemize}
\begin{itemize}
\item {Utilização:Des.}
\end{itemize}
\begin{itemize}
\item {Proveniência:(De \textunderscore farmacopola\textunderscore )}
\end{itemize}
Farmácia.
Colécção de medicamentos.
\section{Farmacopólio}
\begin{itemize}
\item {Grp. gram.:adj.}
\end{itemize}
Relativo a farmacopola.
\section{Farmacoposia}
\begin{itemize}
\item {Grp. gram.:f.}
\end{itemize}
\begin{itemize}
\item {Proveniência:(Gr. \textunderscore pharmacoposia\textunderscore )}
\end{itemize}
Acto de tomar um medicamento, especialmente um purgante.
\section{Fantasmático}
\begin{itemize}
\item {Grp. gram.:adj.}
\end{itemize}
Relativo a fantasma:«\textunderscore a obsessão fantasmática...\textunderscore »R. Jorge, \textunderscore El Greco\textunderscore , 33.
\section{Faquear}
\begin{itemize}
\item {Grp. gram.:v. t.}
\end{itemize}
O mesmo que \textunderscore esfaquear\textunderscore . Cf. Goes, \textunderscore João III\textunderscore , 152, (ed. de Coímbra).
\section{Febre-amarela}
\begin{itemize}
\item {Grp. gram.:f.}
\end{itemize}
Moléstia epidêmica, também conhecida por \textunderscore vómito-negro\textunderscore .
\section{Fèdevelha}
\begin{itemize}
\item {Grp. gram.:f.}
\end{itemize}
\begin{itemize}
\item {Utilização:T. de Penafiel}
\end{itemize}
Insecto orthóptero, também conhecido por \textunderscore sapateiro\textunderscore .
\section{Ferradurina}
\begin{itemize}
\item {Grp. gram.:f.}
\end{itemize}
\begin{itemize}
\item {Utilização:Bot.}
\end{itemize}
Espécie de erva-ferradura, (\textunderscore hippocrepis unisiliquosa\textunderscore , Lin.).
\section{Fibromioma}
\begin{itemize}
\item {Grp. gram.:m.}
\end{itemize}
\begin{itemize}
\item {Utilização:Med.}
\end{itemize}
\begin{itemize}
\item {Proveniência:(De \textunderscore fibroso\textunderscore  + \textunderscore mioma\textunderscore )}
\end{itemize}
Tumor uterino, também chamado \textunderscore fibroma\textunderscore  e \textunderscore tumor fibroso\textunderscore .
\section{Fibromyoma}
\begin{itemize}
\item {Grp. gram.:m.}
\end{itemize}
\begin{itemize}
\item {Utilização:Med.}
\end{itemize}
\begin{itemize}
\item {Proveniência:(De \textunderscore fibroso\textunderscore  + \textunderscore myoma\textunderscore )}
\end{itemize}
Tumor uterino, também chamado \textunderscore fibroma\textunderscore  e \textunderscore tumor fibroso\textunderscore .
\section{Flôr-de-viúva}
\begin{itemize}
\item {Grp. gram.:f.}
\end{itemize}
\begin{itemize}
\item {Utilização:Bot.}
\end{itemize}
Planta, o mesmo que \textunderscore viúvas\textunderscore .
\section{Folegar}
\begin{itemize}
\item {Grp. gram.:v. i.}
\end{itemize}
\begin{itemize}
\item {Utilização:Ant.}
\end{itemize}
Tomar fôlego; respirar. Cf. \textunderscore Rev. Lus.\textunderscore , XVI, 7.
\section{Folhetázio}
\begin{itemize}
\item {Grp. gram.:m.}
\end{itemize}
\begin{itemize}
\item {Utilização:Deprec.}
\end{itemize}
Folheto reles, mas vistoso. Cf. Macedo, \textunderscore Motim\textunderscore , I, 200.
\section{Folheteiro}
\begin{itemize}
\item {Grp. gram.:adj.}
\end{itemize}
\begin{itemize}
\item {Utilização:Burl.}
\end{itemize}
Relativo a folheto:«\textunderscore a praga folheteira\textunderscore ». Macedo, \textunderscore Motim\textunderscore , IV, 59.
\section{Foliculite}
\begin{itemize}
\item {Grp. gram.:f.}
\end{itemize}
\begin{itemize}
\item {Utilização:Med.}
\end{itemize}
Inflammação dos folículos pilosos.
\section{Forcipressão}
\begin{itemize}
\item {Grp. gram.:f.}
\end{itemize}
\begin{itemize}
\item {Utilização:Med.}
\end{itemize}
Méthodo de hemóstase provisória, que consiste em segurar um vaso seccionado com uma pinça, cujas hastes se mantêm fechadas.
(Cp. fr. \textunderscore forcipressure\textunderscore )
\section{Frenação}
\begin{itemize}
\item {Grp. gram.:f.}
\end{itemize}
\begin{itemize}
\item {Proveniência:(Do lat. \textunderscore frenatio\textunderscore )}
\end{itemize}
Acto ou effeito de frenar.
Repressão.
\section{Frenar}
\begin{itemize}
\item {Grp. gram.:v. t.}
\end{itemize}
\begin{itemize}
\item {Utilização:Fig.}
\end{itemize}
\begin{itemize}
\item {Proveniência:(Lat. \textunderscore frenare\textunderscore )}
\end{itemize}
O mesmo que \textunderscore enfrear\textunderscore .
Moderar; reprimir. Cf. R. Jorge, \textunderscore El Greco\textunderscore , 51.
\section{Frigoterapia}
\begin{itemize}
\item {Grp. gram.:f.}
\end{itemize}
\begin{itemize}
\item {Proveniência:(Do lat. \textunderscore frigor\textunderscore  + gr. \textunderscore therapeia\textunderscore )}
\end{itemize}
Tratamento terapêutico pelo frio.
\section{Frigotherapia}
\begin{itemize}
\item {Grp. gram.:f.}
\end{itemize}
\begin{itemize}
\item {Proveniência:(Do lat. \textunderscore frigor\textunderscore  + gr. \textunderscore therapeia\textunderscore )}
\end{itemize}
Tratamento therapêutico pelo frio.
\section{Faringe}
\begin{itemize}
\item {Grp. gram.:f.}
\end{itemize}
\begin{itemize}
\item {Proveniência:(Do gr. \textunderscore pharunge\textunderscore )}
\end{itemize}
Cavidade músculo-membranosa, entre a bôca e a parte superior do esófago.
\section{Faríngeo}
\begin{itemize}
\item {Grp. gram.:adj.}
\end{itemize}
Relativo á faringe.
\section{Faringite}
\begin{itemize}
\item {Grp. gram.:f.}
\end{itemize}
Inflamação da faringe.
\section{Faringocele}
\begin{itemize}
\item {Grp. gram.:m.}
\end{itemize}
\begin{itemize}
\item {Proveniência:(Do gr. \textunderscore pharunge\textunderscore  + \textunderscore kele\textunderscore )}
\end{itemize}
Espécie de tumor, resultante de uma dilatação anormal da faringe.
\section{Faringografia}
\begin{itemize}
\item {Grp. gram.:f.}
\end{itemize}
\begin{itemize}
\item {Proveniência:(Do gr. \textunderscore pharunge\textunderscore  + \textunderscore graphein\textunderscore )}
\end{itemize}
Descripção da faringe.
\section{Faringográfico}
\begin{itemize}
\item {Grp. gram.:adj.}
\end{itemize}
Relativo á faringografia.
\section{Faringologia}
\begin{itemize}
\item {Grp. gram.:f.}
\end{itemize}
\begin{itemize}
\item {Proveniência:(Do gr. \textunderscore pharunge\textunderscore  + \textunderscore logos\textunderscore )}
\end{itemize}
Tratado á cêrca da faringe.
\section{Faringológico}
\begin{itemize}
\item {Grp. gram.:adj.}
\end{itemize}
Relativo á faringologia.
\section{Faringoplegia}
\begin{itemize}
\item {Grp. gram.:f.}
\end{itemize}
\begin{itemize}
\item {Proveniência:(Do gr. \textunderscore pharunge\textunderscore  + \textunderscore plessein\textunderscore )}
\end{itemize}
Paralisia da faringe.
\section{Faringoplégico}
\begin{itemize}
\item {Grp. gram.:adj.}
\end{itemize}
Relativo á faringoplegia.
\section{Faringoscópio}
\begin{itemize}
\item {Grp. gram.:m.}
\end{itemize}
\begin{itemize}
\item {Utilização:Med.}
\end{itemize}
\begin{itemize}
\item {Proveniência:(Do gr. \textunderscore pharunge\textunderscore  + \textunderscore skopein\textunderscore )}
\end{itemize}
Instrumento, para observar a faringe.
\section{Farmacossiderite}
\begin{itemize}
\item {Grp. gram.:f.}
\end{itemize}
Ferro arseniatado, associado a filões de estanho.
\section{Farmacossiderito}
\begin{itemize}
\item {Grp. gram.:m.}
\end{itemize}
O mesmo ou melhor que \textunderscore farmacosiderite\textunderscore .
\section{Farmacotecnia}
\begin{itemize}
\item {Grp. gram.:f.}
\end{itemize}
\begin{itemize}
\item {Proveniência:(Do gr. \textunderscore pharmakon\textunderscore  + \textunderscore tekhne\textunderscore )}
\end{itemize}
Tratado das preparações farmacêuticas.
\section{Farmacotécnico}
\begin{itemize}
\item {Grp. gram.:adj.}
\end{itemize}
Relativo á farmacotecnia.
\section{Farmacoterapia}
\begin{itemize}
\item {Grp. gram.:f.}
\end{itemize}
Estudo dos efeitos dos medicamentos no organismo enfermo.
Conhecimento das fórmulas farmacêuticas.
Modo de dosar e ministrar medicamentos. Cf. \textunderscore Pharmacopeia Port.\textunderscore 
\section{Faro}
\begin{itemize}
\item {Grp. gram.:m.}
\end{itemize}
Terra ou lugar, em que há farol, para guia de navegantes.
O mesmo que \textunderscore farol\textunderscore .
\section{Farol}
\begin{itemize}
\item {Grp. gram.:m.}
\end{itemize}
\begin{itemize}
\item {Utilização:Náut.}
\end{itemize}
\begin{itemize}
\item {Utilização:Fig.}
\end{itemize}
\begin{itemize}
\item {Utilização:Taur.}
\end{itemize}
\begin{itemize}
\item {Proveniência:(Do gr. \textunderscore Pharos\textunderscore , n. p.)}
\end{itemize}
Candeeiro volante, usado a bordo e nos postos semafóricos, para communicação de sinaes ou para cumprimento do regulamento dos portos e das regras internacionaes sôbre a maneira de evitar abalroamentos.
\textunderscore Farol de eclipses\textunderscore , farol para iluminação de costas e baixios, disposto de modo que os diferentes sectores do horizonte são alumiados e ficam em trevas, alternada e sucessivamente.
Lampião na popa da embarcação ou na gávea do mastaréu da gata.
Coisa que alumia.
Direcção.
Director.
Guia; norte.
Nome de uma sorte de bandarilheiro.
\section{Faroleiro}
\begin{itemize}
\item {Grp. gram.:m.}
\end{itemize}
Indivíduo, encarregado de guardar ou tratar de um farol.
\section{Farolete}
\begin{itemize}
\item {fónica:lê}
\end{itemize}
\begin{itemize}
\item {Grp. gram.:m.}
\end{itemize}
\begin{itemize}
\item {Utilização:Bras}
\end{itemize}
Pequeno farol.
\section{Farolim}
\begin{itemize}
\item {Grp. gram.:m.}
\end{itemize}
Pequeno farol.
\section{Farolização}
\begin{itemize}
\item {Grp. gram.:f.}
\end{itemize}
\begin{itemize}
\item {Utilização:Neol.}
\end{itemize}
Acto de farolizar ou estabelecer faroes.
\section{Farolizar}
\begin{itemize}
\item {Grp. gram.:v. t.}
\end{itemize}
\begin{itemize}
\item {Utilização:Neol.}
\end{itemize}
\begin{itemize}
\item {Utilização:Fig.}
\end{itemize}
\begin{itemize}
\item {Proveniência:(De \textunderscore farol\textunderscore )}
\end{itemize}
Estabelecer faroes em: \textunderscore farolizar um pôrto de mar\textunderscore .
Espalhar luz sôbre; iluminar; esclarecêr.
\section{Faringóstomo}
\begin{itemize}
\item {Grp. gram.:adj.}
\end{itemize}
\begin{itemize}
\item {Proveniência:(Do gr. \textunderscore pharungè\textunderscore  + \textunderscore stoma\textunderscore )}
\end{itemize}
Diz-se dos animaes, cuja bôca é constituida pelos bordos do esófago.
\section{Faringotomia}
\begin{itemize}
\item {Grp. gram.:f.}
\end{itemize}
\begin{itemize}
\item {Proveniência:(De \textunderscore faringótomo\textunderscore )}
\end{itemize}
Incisão na faringe.
\section{Faringótomo}
\begin{itemize}
\item {Grp. gram.:m.}
\end{itemize}
\begin{itemize}
\item {Proveniência:(Do gr. \textunderscore pharunge\textunderscore  + \textunderscore tome\textunderscore )}
\end{itemize}
Instrumento, com que se pratica a faringotomia.
\section{Fascáceas}
\begin{itemize}
\item {Grp. gram.:pl.}
\end{itemize}
\begin{itemize}
\item {Utilização:Bot.}
\end{itemize}
\begin{itemize}
\item {Proveniência:(Do gr. \textunderscore phaskos\textunderscore )}
\end{itemize}
Ordem de musgos.
\section{Fascólomo}
\begin{itemize}
\item {Grp. gram.:m.}
\end{itemize}
\begin{itemize}
\item {Proveniência:(Do gr. \textunderscore phaskolon\textunderscore  + \textunderscore mus\textunderscore )}
\end{itemize}
Gênero de mamíferos australianos.
\section{Fase}
\begin{itemize}
\item {Grp. gram.:f.}
\end{itemize}
\begin{itemize}
\item {Proveniência:(Gr. \textunderscore phasis\textunderscore )}
\end{itemize}
Cada um dos aspectos diversos da Lua e de outros planetas, segundo a maneira, com que recebem a luz do Sol.
Cada uma das modificações sucessivas que se notam em certas coisas: \textunderscore as fases de uma questão\textunderscore .
Cada um dos diferentes aspectos que uma coisa apresenta sucessivamente: \textunderscore as fases da vida\textunderscore .
\section{Faseolar}
\begin{itemize}
\item {Grp. gram.:adj.}
\end{itemize}
\begin{itemize}
\item {Proveniência:(Do lat. \textunderscore phaseolum\textunderscore )}
\end{itemize}
Que tem forma de feijão, (o rim, por exemplo).
\section{Faseóleas}
\begin{itemize}
\item {Grp. gram.:f. pl.}
\end{itemize}
\begin{itemize}
\item {Proveniência:(De \textunderscore faséolo\textunderscore )}
\end{itemize}
Tríbo de plantas leguminosas, no sistema de De-Condolle.
\section{Faseólico}
\begin{itemize}
\item {Grp. gram.:adj.}
\end{itemize}
\begin{itemize}
\item {Proveniência:(De \textunderscore faséolo\textunderscore )}
\end{itemize}
Diz-se de um ácido, que existe em certas qualidades de feijão.
\section{Faseolina}
\begin{itemize}
\item {Grp. gram.:f.}
\end{itemize}
\begin{itemize}
\item {Proveniência:(De \textunderscore faséolo\textunderscore )}
\end{itemize}
Substância cristalina, extraida de uma espécie de feijão.
\section{Faséolo}
\begin{itemize}
\item {Grp. gram.:m.}
\end{itemize}
\begin{itemize}
\item {Proveniência:(Lat. \textunderscore phaseolus\textunderscore )}
\end{itemize}
Nome científico do feijão.
\section{Fateusim}
\begin{itemize}
\item {Grp. gram.:m.  e  adj.}
\end{itemize}
(V.enfiteuta)
\section{Fatniorragia}
\begin{itemize}
\item {Grp. gram.:f.}
\end{itemize}
\begin{itemize}
\item {Proveniência:(Do gr. \textunderscore phatnia\textunderscore  + \textunderscore rhagein\textunderscore )}
\end{itemize}
Hemorragía pelo alvéolo de um dente.
\section{Febeu}
\begin{itemize}
\item {Grp. gram.:adj.}
\end{itemize}
\begin{itemize}
\item {Proveniência:(Lat. \textunderscore phoebeus\textunderscore )}
\end{itemize}
Relativo ao Sol.
\section{Febo}
\begin{itemize}
\item {Grp. gram.:m.}
\end{itemize}
\begin{itemize}
\item {Utilização:Poét.}
\end{itemize}
\begin{itemize}
\item {Proveniência:(Do gr. \textunderscore Phoibos\textunderscore , n. p.)}
\end{itemize}
O Sol.
\section{Felândrio}
\begin{itemize}
\item {Grp. gram.:m.}
\end{itemize}
\begin{itemize}
\item {Proveniência:(Lat. \textunderscore phellandrion\textunderscore )}
\end{itemize}
Planta umbelífera e venenosa dos terrenos pantanosos.
\section{Felocarpo}
\begin{itemize}
\item {Grp. gram.:m.}
\end{itemize}
\begin{itemize}
\item {Proveniência:(Do gr. \textunderscore phellos\textunderscore  + \textunderscore karpos\textunderscore )}
\end{itemize}
Gênero de plantas leguminosas da América tropical.
\section{Feloderme}
\begin{itemize}
\item {Grp. gram.:f.}
\end{itemize}
\begin{itemize}
\item {Utilização:Bot.}
\end{itemize}
\begin{itemize}
\item {Proveniência:(Do gr. \textunderscore phellos\textunderscore  + \textunderscore derma\textunderscore )}
\end{itemize}
Zona, que reveste a endoderme ou parte interna do sistema tegumentar, por baixo da capa suberosa.
\section{Felogênio}
\begin{itemize}
\item {Grp. gram.:m.}
\end{itemize}
\begin{itemize}
\item {Utilização:Bot.}
\end{itemize}
\begin{itemize}
\item {Proveniência:(Do gr. \textunderscore phellos\textunderscore  + \textunderscore genea\textunderscore )}
\end{itemize}
Parte do sistema tegumentar que dá origem á feloderme.
\section{Feloplástica}
\begin{itemize}
\item {Grp. gram.:f.}
\end{itemize}
\begin{itemize}
\item {Utilização:Restrict.}
\end{itemize}
\begin{itemize}
\item {Proveniência:(Do gr. \textunderscore phellos\textunderscore  + \textunderscore plassein\textunderscore )}
\end{itemize}
Arte de esculpir em cortiça.
Arte de representar em cortiça monumentos de arquitectura.
\section{Felose}
\begin{itemize}
\item {Grp. gram.:f.}
\end{itemize}
\begin{itemize}
\item {Proveniência:(Do gr. \textunderscore phellos\textunderscore )}
\end{itemize}
Producção acidental de uma espécie de cortiça em alguns vegetaes.
\section{Fena}
\begin{itemize}
\item {Grp. gram.:f.}
\end{itemize}
Espécie de abutre.
\section{Fenacetina}
\begin{itemize}
\item {Grp. gram.:f.}
\end{itemize}
Composto químico, empregado como febrífugo ou antiséptico, e analgésico ou antineurálgico.
\section{Fenato}
\begin{itemize}
\item {Grp. gram.:m.}
\end{itemize}
\begin{itemize}
\item {Utilização:Chím.}
\end{itemize}
Gênero de sáes, formados pelo ácido fênico.
\section{Fene}
\begin{itemize}
\item {Grp. gram.:m.}
\end{itemize}
O mesmo que \textunderscore benzina\textunderscore .
\section{Fenedina}
\begin{itemize}
\item {Grp. gram.:f.}
\end{itemize}
O mesmo que \textunderscore fenacetina\textunderscore .
\section{Fenetol}
\begin{itemize}
\item {Grp. gram.:m.}
\end{itemize}
\begin{itemize}
\item {Utilização:Chím.}
\end{itemize}
Fenato de etilo.
\section{Fênices}
\begin{itemize}
\item {Grp. gram.:m. pl.}
\end{itemize}
\begin{itemize}
\item {Proveniência:(Lat. \textunderscore phoeníces\textunderscore )}
\end{itemize}
O mesmo que [[fenícios|fenício]].
\section{Fenício}
\begin{itemize}
\item {Grp. gram.:adj.}
\end{itemize}
\begin{itemize}
\item {Grp. gram.:M.}
\end{itemize}
Relativo á Fenícia ou aos seus habitantes.
Habitante da Fenícia.
Língua semítica, falada pelos Fenícios.
\section{Fenicito}
\begin{itemize}
\item {Grp. gram.:m.}
\end{itemize}
\begin{itemize}
\item {Utilização:Miner.}
\end{itemize}
\begin{itemize}
\item {Proveniência:(Do gr. \textunderscore phoinix\textunderscore , vermelho)}
\end{itemize}
Variedade de chumbo cromatado.
\section{Fênico}
\begin{itemize}
\item {Grp. gram.:adj.}
\end{itemize}
\begin{itemize}
\item {Proveniência:(Do gr. \textunderscore phainos\textunderscore )}
\end{itemize}
Diz-se de um ácido, extraido do alcatrão da hulha.
Relativo ao fenol.
\section{Fenicóptero}
\begin{itemize}
\item {Grp. gram.:m.}
\end{itemize}
\begin{itemize}
\item {Proveniência:(Gr. \textunderscore phoinikopteros\textunderscore )}
\end{itemize}
Ave pernalta.
\section{Fenigma}
\begin{itemize}
\item {Grp. gram.:m.}
\end{itemize}
\begin{itemize}
\item {Proveniência:(Do gr. \textunderscore phoinix\textunderscore )}
\end{itemize}
Rubefacção da pele, produzida por sinapismos.
\section{Fênix}
\begin{itemize}
\item {fónica:nis}
\end{itemize}
\begin{itemize}
\item {Grp. gram.:f.}
\end{itemize}
\begin{itemize}
\item {Proveniência:(Lat. \textunderscore phoenix\textunderscore )}
\end{itemize}
Ave fabulosa que, segundo a Mitologia, vivia muitos séculos, e que, queimada, renascia da sua cinza.
Constelação austral.
Pessôa ou coisa única no seu gênero.
\section{Fenocarpo}
\begin{itemize}
\item {Grp. gram.:m.}
\end{itemize}
\begin{itemize}
\item {Utilização:Bot.}
\end{itemize}
Dizia-se o fruto que, não aderindo ás partes vizinhas, é por isso muito aparente.
\section{Fenocola}
\begin{itemize}
\item {Grp. gram.:f.}
\end{itemize}
Producto farmacêutico, com propriedades analgésicas e antitérmicas.
\section{Fenodina}
\begin{itemize}
\item {Grp. gram.:f.}
\end{itemize}
O mesmo que \textunderscore hematosina\textunderscore .
\section{Fenogamia}
\begin{itemize}
\item {Grp. gram.:f.}
\end{itemize}
\begin{itemize}
\item {Proveniência:(Do gr. \textunderscore phoinos\textunderscore  + \textunderscore gamos\textunderscore )}
\end{itemize}
Estado do que é fenogâmico.
\section{Fenogâmico}
\begin{itemize}
\item {Grp. gram.:adj.}
\end{itemize}
\begin{itemize}
\item {Proveniência:(De \textunderscore fenogamia\textunderscore )}
\end{itemize}
Diz-se do vegetal ou do animal, que tem aparentes os órgãos sexuaes.
\section{Fenol}
\begin{itemize}
\item {Grp. gram.:m.}
\end{itemize}
\begin{itemize}
\item {Grp. gram.:Pl.}
\end{itemize}
\begin{itemize}
\item {Proveniência:(Do gr. \textunderscore phainein\textunderscore )}
\end{itemize}
Substância, extraida dos óleos, que fornecem o alcatrão do gás.
O mesmo que ácido fênico.
Corpos ternários, compostos de carbone, hidrogênio e oxygênio, e provenientes de um carbureto pela substituição de um átomo de hidrogênio por um oxidrilo.
\section{Fenolite}
\begin{itemize}
\item {Grp. gram.:f.}
\end{itemize}
\begin{itemize}
\item {Utilização:Geol.}
\end{itemize}
Espécie de rocha eruptiva da época posterciária.
\section{Fenolito}
\begin{itemize}
\item {Grp. gram.:m.}
\end{itemize}
O mesmo ou melhor que \textunderscore fenolite\textunderscore .
\section{Fenolftateína}
\begin{itemize}
\item {Grp. gram.:f.}
\end{itemize}
Composição farmacêutica de fenol e neftalina.
\section{Fenomenal}
\begin{itemize}
\item {Grp. gram.:adj.}
\end{itemize}
Que é da natureza do fenómeno.
Admirável; espantoso; singular.
\section{Fenomenalidade}
\begin{itemize}
\item {Grp. gram.:f.}
\end{itemize}
Qualidade do que é fenomenal.
\section{Fenomenização}
\begin{itemize}
\item {Grp. gram.:f.}
\end{itemize}
\begin{itemize}
\item {Utilização:Espir.}
\end{itemize}
Acto ou efeito de fenomenizar-se.
Produção de fenómenos.
\section{Fenomenizar-se}
\begin{itemize}
\item {Grp. gram.:v. p.}
\end{itemize}
\begin{itemize}
\item {Utilização:Neol.}
\end{itemize}
\begin{itemize}
\item {Proveniência:(De \textunderscore fenómeno\textunderscore )}
\end{itemize}
Realizar-se, manifestar-se:«\textunderscore ...talento que se fenomeniza cá em baixo por actos de pequeno alcance.\textunderscore »Tobias Barreto.
\section{Fenómeno}
\begin{itemize}
\item {Grp. gram.:m.}
\end{itemize}
\begin{itemize}
\item {Proveniência:(Gr. \textunderscore phainomenon\textunderscore )}
\end{itemize}
Tudo aquilo em que se exerce a acção dos sentidos ou que póde impressionar a nossa sensibilidade, física ou moralmente.
Facto.
Tudo que se observa de extraordinário no ar ou no céu.
Maravilha.
O que é raro e surpreendente.
\section{Fenomenologia}
\begin{itemize}
\item {Grp. gram.:f.}
\end{itemize}
\begin{itemize}
\item {Proveniência:(Do gr. \textunderscore phainomenon\textunderscore  + \textunderscore logos\textunderscore )}
\end{itemize}
Tratado á cêrca dos fenómenos.
\section{Fenomenoso}
\begin{itemize}
\item {Grp. gram.:adj.}
\end{itemize}
\begin{itemize}
\item {Utilização:P. us.}
\end{itemize}
\begin{itemize}
\item {Proveniência:(De \textunderscore fenómeno\textunderscore )}
\end{itemize}
Extraordinário, admirável.
\section{Fenossalil}
\begin{itemize}
\item {Grp. gram.:m.}
\end{itemize}
Producto farmacêutico, com propriedades antisépticas.
\section{Fenilacetileno}
\begin{itemize}
\item {Grp. gram.:m.}
\end{itemize}
\begin{itemize}
\item {Utilização:Chím.}
\end{itemize}
Um dos carbonetos do grupo pirogenado.
\section{Fenilidroquinazolina}
\begin{itemize}
\item {Grp. gram.:f.}
\end{itemize}
Producto farmacêutico, amargo e excitante do estômago.
\section{Fenilmetânio}
\begin{itemize}
\item {Grp. gram.:m.}
\end{itemize}
Producto farmacêutico, analgésico e antitérmico.
\section{Fenilo}
\begin{itemize}
\item {Grp. gram.:m.}
\end{itemize}
\begin{itemize}
\item {Utilização:Chím.}
\end{itemize}
Radical hipotético do grupo fênico.
\section{Feofícias}
\begin{itemize}
\item {fónica:fe-o}
\end{itemize}
\begin{itemize}
\item {Grp. gram.:f. pl.}
\end{itemize}
\begin{itemize}
\item {Utilização:Geol.}
\end{itemize}
\begin{itemize}
\item {Proveniência:(Do gr. \textunderscore phaios\textunderscore , pardo, e \textunderscore phukos\textunderscore , alga)}
\end{itemize}
Ordem de algas fósseis.
\section{Fíala}
\begin{itemize}
\item {Grp. gram.:f.}
\end{itemize}
\begin{itemize}
\item {Proveniência:(Lat. \textunderscore phiala\textunderscore )}
\end{itemize}
Espécie de taça, usada pelos antigos, e que muitas vezes se oferecia como brinde.
\section{Filadelfo}
\begin{itemize}
\item {Grp. gram.:m.}
\end{itemize}
\begin{itemize}
\item {Proveniência:(Gr. \textunderscore philadelphos\textunderscore )}
\end{itemize}
Membro de uma seita religiosa da Inglaterra, no século XVII; membro de uma sociedade secreta da França, no tempo do primeiro império.
\section{Filadelfos}
\begin{itemize}
\item {Grp. gram.:m. pl.}
\end{itemize}
\begin{itemize}
\item {Proveniência:(Do gr. \textunderscore philos\textunderscore  + \textunderscore adelphos\textunderscore )}
\end{itemize}
Família de pólipos.
\section{Filagónia}
\begin{itemize}
\item {Grp. gram.:f.}
\end{itemize}
Árvore das florestas de Java.
\section{Filandra}
\begin{itemize}
\item {Grp. gram.:f.}
\end{itemize}
\begin{itemize}
\item {Proveniência:(Do gr. \textunderscore philos\textunderscore  + \textunderscore aner\textunderscore , \textunderscore andros\textunderscore )}
\end{itemize}
Nome de duas espécies de sarigueias e de um canguru da Índia.
\section{Filanto}
\begin{itemize}
\item {Grp. gram.:m.}
\end{itemize}
\begin{itemize}
\item {Proveniência:(Do gr. \textunderscore philos\textunderscore  + \textunderscore anthos\textunderscore )}
\end{itemize}
Pássaro de Bengala.
\section{Filantropia}
\begin{itemize}
\item {Grp. gram.:f.}
\end{itemize}
\begin{itemize}
\item {Proveniência:(Lat. \textunderscore philantropia\textunderscore )}
\end{itemize}
Amor á humanidade; caridade.
\section{Filantropicamente}
\begin{itemize}
\item {Grp. gram.:adv.}
\end{itemize}
De modo filantrópico.
\section{Filantrópico}
\begin{itemize}
\item {Grp. gram.:adj.}
\end{itemize}
Relativo á filantropia.
Inspirado pela filantropia.
\section{Filantropismo}
\begin{itemize}
\item {Grp. gram.:m.}
\end{itemize}
Afectação de filantropia.
\section{Filantropo}
\begin{itemize}
\item {Grp. gram.:m.  e  adj.}
\end{itemize}
\begin{itemize}
\item {Proveniência:(Lat. \textunderscore philanthropos\textunderscore )}
\end{itemize}
O que é dotado de filanthropia.
O que ama os seus semelhantes; humanitário.
\section{Filantropomania}
\begin{itemize}
\item {Grp. gram.:f.}
\end{itemize}
\begin{itemize}
\item {Proveniência:(De \textunderscore filantropia\textunderscore  + \textunderscore mania\textunderscore )}
\end{itemize}
Filanthropia ridícula ou pouco sincera.
\section{Filargíria}
\begin{itemize}
\item {Grp. gram.:f.}
\end{itemize}
\begin{itemize}
\item {Proveniência:(Lat. \textunderscore philargyria\textunderscore )}
\end{itemize}
O mesmo que \textunderscore avareza\textunderscore .
\section{Filarmónica}
\begin{itemize}
\item {Grp. gram.:f.}
\end{itemize}
\begin{itemize}
\item {Utilização:Gír.}
\end{itemize}
Sociedade musical.
Banda de música.
A policia, apitando.
(Fem. de \textunderscore philarmónico\textunderscore )
\section{Filarmónico}
\begin{itemize}
\item {Grp. gram.:adj.}
\end{itemize}
\begin{itemize}
\item {Proveniência:(De \textunderscore philo...\textunderscore  + \textunderscore harmónico\textunderscore )}
\end{itemize}
Que é amigo da harmonia ou da música.
Diz-se especialmente de certas sociedades musicaes.
\section{Filatelia}
\begin{itemize}
\item {Grp. gram.:f.}
\end{itemize}
\begin{itemize}
\item {Proveniência:(Do gr. \textunderscore philos\textunderscore  + \textunderscore ateleia\textunderscore )}
\end{itemize}
Estudo dos selos do correio, usados nas diversas nações, e metodicamente coleccionados.--R. Galvão, \textunderscore Vocab.\textunderscore , entende que a forma exacta é \textunderscore filotelia\textunderscore , por a suppor derivada do gr. \textunderscore philos\textunderscore  + \textunderscore telos\textunderscore .
\section{Filatélico}
\begin{itemize}
\item {Grp. gram.:adj.}
\end{itemize}
Relativo á filatelia.
\section{Filatelismo}
\begin{itemize}
\item {Grp. gram.:m.}
\end{itemize}
Gôsto pela filatelia; prática dêsse gôsto.
\section{Filatelista}
\begin{itemize}
\item {Grp. gram.:m.}
\end{itemize}
\begin{itemize}
\item {Proveniência:(De \textunderscore filatelia\textunderscore )}
\end{itemize}
Coleccionador de selos do correio.
\section{Filáucia}
\begin{itemize}
\item {Grp. gram.:f.}
\end{itemize}
\begin{itemize}
\item {Proveniência:(Gr. \textunderscore philautia\textunderscore )}
\end{itemize}
Amor próprio; egoismo; presumpção, vaidade.
\section{Filaucioso}
\begin{itemize}
\item {Grp. gram.:adj.}
\end{itemize}
Que tem filáucia.
\section{Filédono}
\begin{itemize}
\item {Grp. gram.:m.}
\end{itemize}
Gênero de pássaros, estabelecido por Cuvier na fam. dos dentirostros.
\section{Filípica}
\begin{itemize}
\item {Grp. gram.:f.}
\end{itemize}
\begin{itemize}
\item {Utilização:Ext.}
\end{itemize}
\begin{itemize}
\item {Proveniência:(De \textunderscore Philippe\textunderscore , n. p.)}
\end{itemize}
Oração de Demóstenes contra Filipe de Macedónia.
Cada uma das orações de Cícero contra Marco-António.
Discurso violento e injurioso; sátira acerba.
\section{Filipista}
\begin{itemize}
\item {Grp. gram.:m.}
\end{itemize}
Sectário de Luís Filipe, em França.
\section{Filipsita}
\begin{itemize}
\item {Grp. gram.:f.}
\end{itemize}
\begin{itemize}
\item {Proveniência:(De \textunderscore Philips\textunderscore , n. p.)}
\end{itemize}
Sulfureto de cobre e de ferro.
\section{Filipsito}
\begin{itemize}
\item {Grp. gram.:m.}
\end{itemize}
O mesmo ou melhor que \textunderscore filipsita\textunderscore .
\section{Filireia}
\begin{itemize}
\item {Grp. gram.:f.}
\end{itemize}
\begin{itemize}
\item {Proveniência:(Do gr. \textunderscore phillurea\textunderscore )}
\end{itemize}
Árvore oleagínea do sul da Europa.
\section{Filirena}
\begin{itemize}
\item {Grp. gram.:f.}
\end{itemize}
\begin{itemize}
\item {Utilização:Chím.}
\end{itemize}
Princípio, que se extrai da casca da filireia.
\section{Filisteu}
\begin{itemize}
\item {Grp. gram.:m.}
\end{itemize}
\begin{itemize}
\item {Utilização:Pop.}
\end{itemize}
\begin{itemize}
\item {Proveniência:(De \textunderscore filisteus\textunderscore )}
\end{itemize}
Homem corpulento e desajeitado; brutamontes.
\section{Filisteus}
\begin{itemize}
\item {Grp. gram.:m. pl.}
\end{itemize}
\begin{itemize}
\item {Proveniência:(Lat. \textunderscore philistaei\textunderscore )}
\end{itemize}
Um dos povos que habitavam a Palestina, antes da conquista dêste país pelos Hebreus.
\section{Filistinos}
\begin{itemize}
\item {Grp. gram.:m. pl.}
\end{itemize}
O mesmo que \textunderscore Filisteus\textunderscore .
Nome que, entre os estudantes alemães, se dá aos indivíduos estranhos ás universidades, mormente os negociantes.
\section{Filocínico}
\begin{itemize}
\item {Grp. gram.:adj.}
\end{itemize}
\begin{itemize}
\item {Utilização:Neol.}
\end{itemize}
\begin{itemize}
\item {Proveniência:(Do gr. \textunderscore philos\textunderscore  + \textunderscore kuon\textunderscore )}
\end{itemize}
Que gosta de cães; amigo dos cães.
\section{Filodendro}
\begin{itemize}
\item {Grp. gram.:m.}
\end{itemize}
\begin{itemize}
\item {Proveniência:(Do gr. \textunderscore philos\textunderscore  + \textunderscore dendron\textunderscore )}
\end{itemize}
Planta ornamental, de grandes e formosas fôlhas, originária da América do Sul.
\section{Filodérmico}
\begin{itemize}
\item {Grp. gram.:adj.}
\end{itemize}
\begin{itemize}
\item {Proveniência:(Do gr. \textunderscore philos\textunderscore  + \textunderscore derma\textunderscore )}
\end{itemize}
Diz-se dos preparados, que conservam a maciez e frescura da pele.
\section{Filodinasta}
\begin{itemize}
\item {Grp. gram.:adj.}
\end{itemize}
Afeiçoado a uma dinastia. Cf. Camillo, \textunderscore Noites de Insómn.\textunderscore , III, 80.
\section{Filogenitura}
\begin{itemize}
\item {Grp. gram.:f.}
\end{itemize}
\begin{itemize}
\item {Proveniência:(T. hybr., do gr. \textunderscore philos\textunderscore  + lat. \textunderscore genitura\textunderscore )}
\end{itemize}
Amor, que conduz á procriação de filhos.
\section{Filoginia}
\begin{itemize}
\item {Grp. gram.:f.}
\end{itemize}
\begin{itemize}
\item {Utilização:Bras}
\end{itemize}
Amor ás mulheres.
Teoria de igualdade intelectual do homem e da mulher.
(Cp. \textunderscore filógino\textunderscore )
\section{Filogínio}
\begin{itemize}
\item {Grp. gram.:adj.}
\end{itemize}
O mesmo que \textunderscore filógino\textunderscore . Cf. Camillo, \textunderscore Brasileira\textunderscore , 326.
\section{Filógino}
\begin{itemize}
\item {Grp. gram.:adj.}
\end{itemize}
\begin{itemize}
\item {Proveniência:(Do gr. \textunderscore philos\textunderscore  + \textunderscore gune\textunderscore )}
\end{itemize}
Que tem inclinação para as mulheres; apaixonado por mulheres; femeeiro.
\section{Filologia}
\begin{itemize}
\item {Grp. gram.:f.}
\end{itemize}
\begin{itemize}
\item {Proveniência:(De \textunderscore filólogo\textunderscore )}
\end{itemize}
Estudo e conhecimento de uma língua, como instrumento e meio de uma literatura.
Conhecimento geral das belas-letras, línguas, crítica, etc.
\section{Fílica}
\begin{itemize}
\item {Grp. gram.:f.}
\end{itemize}
\begin{itemize}
\item {Proveniência:(Do gr. \textunderscore philuke\textunderscore )}
\end{itemize}
Gênero de plantas ramnáceas do Cabo da Bôa-Esperança.
\section{Filológico}
\begin{itemize}
\item {Grp. gram.:adj.}
\end{itemize}
Relativo á Filologia.
\section{Filologista}
\begin{itemize}
\item {Grp. gram.:m. ,  f.  e  adj.}
\end{itemize}
Pessôa, que se dedica á Filologia.
\section{Filólogo}
\begin{itemize}
\item {Grp. gram.:m.}
\end{itemize}
\begin{itemize}
\item {Proveniência:(Gr. \textunderscore philologos\textunderscore )}
\end{itemize}
Aquele que é versado ou perito em Filologia.
\section{Filomático}
\begin{itemize}
\item {Grp. gram.:adj.}
\end{itemize}
\begin{itemize}
\item {Proveniência:(Do gr. \textunderscore philos\textunderscore  + \textunderscore mathein\textunderscore )}
\end{itemize}
Que ama as ciências.
\section{Filomela}
\begin{itemize}
\item {Grp. gram.:f.}
\end{itemize}
\begin{itemize}
\item {Utilização:Poét.}
\end{itemize}
\begin{itemize}
\item {Proveniência:(Do gr. \textunderscore Philomela\textunderscore , n. p.)}
\end{itemize}
O mesmo que \textunderscore rouxinol\textunderscore .
\section{Filonegro}
\begin{itemize}
\item {fónica:nê}
\end{itemize}
\begin{itemize}
\item {Grp. gram.:m.  e  adj.}
\end{itemize}
\begin{itemize}
\item {Proveniência:(De \textunderscore philo...\textunderscore  + \textunderscore negro\textunderscore )}
\end{itemize}
Indivíduo, que gosta dos Negros; protector ou defensor dos Negros. Cf. Garrett, \textunderscore Helena\textunderscore , 94.
\section{Filónio}
\begin{itemize}
\item {Grp. gram.:m.}
\end{itemize}
Electuário, de composição muito complexa.
\section{Filosofal}
\begin{itemize}
\item {Grp. gram.:adj.}
\end{itemize}
\begin{itemize}
\item {Utilização:Fig.}
\end{itemize}
O mesmo que \textunderscore filosófico\textunderscore .
\textunderscore Pedra filosofal\textunderscore , segrêdo imaginário, que os alquimistas procuravam penetrar, para converter metaes ordinários em metaes preciosos.
Coisa difícil de se descobrir ou de se realizar.
\section{Filosofante}
\begin{itemize}
\item {Grp. gram.:m.  e  adj.}
\end{itemize}
\begin{itemize}
\item {Utilização:Deprec.}
\end{itemize}
\begin{itemize}
\item {Proveniência:(Lat. \textunderscore philosophans\textunderscore )}
\end{itemize}
Filósofo.
O que discorre disparatadamente, com pretensões a erudito.
\section{Filosofar}
\begin{itemize}
\item {Grp. gram.:v. i.}
\end{itemize}
\begin{itemize}
\item {Proveniência:(Lat. \textunderscore philosophari\textunderscore )}
\end{itemize}
Raciocinar sôbre assuntos filosóficos.
Discutir ou discorrer sôbre qualquer matéria científica; raciocinar.
\section{Filosofastro}
\begin{itemize}
\item {Grp. gram.:m.}
\end{itemize}
\begin{itemize}
\item {Proveniência:(Lat. \textunderscore philosophaster\textunderscore )}
\end{itemize}
Indivíduo, que se supõe filósofo, e que discorre sem acêrto.
\section{Filosofear}
\begin{itemize}
\item {Grp. gram.:v. t.}
\end{itemize}
O mesmo que \textunderscore filosofar\textunderscore . Cf. Camillo, \textunderscore Noites de Insómn.\textunderscore , VI, 74.
\section{Filosofia}
\begin{itemize}
\item {Grp. gram.:f.}
\end{itemize}
\begin{itemize}
\item {Proveniência:(Lat. \textunderscore philosophia\textunderscore )}
\end{itemize}
Ciência geral dos princípios e causas, ou sistema de noções geraes sobre o conjunto das coisas.
Cada um dos sistemas particulares de Filosofia.
Firmeza ou elevação de espírito, com que o homem se coloca acima dos sucessos e preconceitos: \textunderscore proceder com filosofia\textunderscore .
Sabedoria.
\section{Filosoficamente}
\begin{itemize}
\item {Grp. gram.:adv.}
\end{itemize}
De modo filosófico; segundo a Filosofia; á maneira de filósofo.
\section{Filosofice}
\begin{itemize}
\item {Grp. gram.:f.}
\end{itemize}
\begin{itemize}
\item {Utilização:Deprec.}
\end{itemize}
Qualidade de quem filosofa ridiculamente.
\section{Filosófico}
\begin{itemize}
\item {Grp. gram.:adj.}
\end{itemize}
\begin{itemize}
\item {Proveniência:(Lat. \textunderscore philosophicus\textunderscore )}
\end{itemize}
Relativo á Filosofia ou aos filósofos; peculiar aos filósofos.
\section{Filosofismo}
\begin{itemize}
\item {Grp. gram.:m.}
\end{itemize}
\begin{itemize}
\item {Proveniência:(De \textunderscore filosofia\textunderscore )}
\end{itemize}
Mania filosófica; falsa Filosofia.
\section{Filósofo}
\begin{itemize}
\item {Grp. gram.:m.  e  adj.}
\end{itemize}
\begin{itemize}
\item {Proveniência:(Lat. \textunderscore philosophus\textunderscore )}
\end{itemize}
Amigo da sabedoria ou o que se aplica ao estudo dos princípios e causas.
Sábio.
Livre pensador.
O que tem um viver sereno e tranquilo, indiferente ás coisas e preconceitos ou convenções do mundo.
\section{Filotécnico}
\begin{itemize}
\item {Grp. gram.:adj.}
\end{itemize}
Que ama as artes.
(Cp. lat. \textunderscore philotechnus\textunderscore )
\section{Filotimia}
\begin{itemize}
\item {Grp. gram.:f.}
\end{itemize}
\begin{itemize}
\item {Proveniência:(Gr. \textunderscore philotimia\textunderscore )}
\end{itemize}
Amor da honra ou das honras.
\section{Fimose}
\begin{itemize}
\item {Grp. gram.:f.}
\end{itemize}
\begin{itemize}
\item {Utilização:Med.}
\end{itemize}
\begin{itemize}
\item {Proveniência:(Gr. \textunderscore phimosis\textunderscore )}
\end{itemize}
Apêrto natural ou acidental, que não deixa que o prepúcio se retire para trás.
\section{Flebectasía}
\begin{itemize}
\item {Grp. gram.:f.}
\end{itemize}
\begin{itemize}
\item {Utilização:Med.}
\end{itemize}
\begin{itemize}
\item {Proveniência:(Do gr. \textunderscore phleps\textunderscore  + \textunderscore ektasis\textunderscore )}
\end{itemize}
Dilatação de uma veia.
\section{Flebenterismo}
\begin{itemize}
\item {Grp. gram.:m.}
\end{itemize}
\begin{itemize}
\item {Proveniência:(Do gr. \textunderscore phleps\textunderscore  + \textunderscore enteron\textunderscore )}
\end{itemize}
Teoria anatómica, segundo a qual se supõe que em certos seres o sistema circulatório desaparece e é substituido pelo tubo digestivo.
\section{Flébico}
\begin{itemize}
\item {Grp. gram.:adj.}
\end{itemize}
\begin{itemize}
\item {Utilização:Med.}
\end{itemize}
\begin{itemize}
\item {Proveniência:(Do gr. \textunderscore phleps\textunderscore )}
\end{itemize}
Relativo ás veias.
\section{Flebite}
\begin{itemize}
\item {Grp. gram.:f.}
\end{itemize}
\begin{itemize}
\item {Utilização:Med.}
\end{itemize}
\begin{itemize}
\item {Proveniência:(Do gr. \textunderscore phleps\textunderscore )}
\end{itemize}
Inflamação da membrana interna das veias.
\section{Flebografia}
\begin{itemize}
\item {Grp. gram.:f.}
\end{itemize}
\begin{itemize}
\item {Proveniência:(De \textunderscore flebógrafo\textunderscore )}
\end{itemize}
Descripção das veias.
\section{Flebográfico}
\begin{itemize}
\item {Grp. gram.:adj.}
\end{itemize}
Relativo á flebografia.
\section{Flebógrafo}
\begin{itemize}
\item {Grp. gram.:m.}
\end{itemize}
\begin{itemize}
\item {Proveniência:(Do gr. \textunderscore phleps\textunderscore , \textunderscore phlebos\textunderscore  + \textunderscore graphein\textunderscore )}
\end{itemize}
Anatomista, que descreve as veias.
\section{Flebólito}
\begin{itemize}
\item {Grp. gram.:f.}
\end{itemize}
\begin{itemize}
\item {Proveniência:(Do gr. \textunderscore phleps\textunderscore , \textunderscore phlebos\textunderscore  + \textunderscore lithos\textunderscore )}
\end{itemize}
Concreção calcária, que se fórma numa veia varicosa.
\section{Flebomalacia}
\begin{itemize}
\item {Grp. gram.:f.}
\end{itemize}
\begin{itemize}
\item {Utilização:Med.}
\end{itemize}
\begin{itemize}
\item {Proveniência:(Do gr. \textunderscore phleps\textunderscore  + \textunderscore malakia\textunderscore )}
\end{itemize}
Amolecimento mòrbido das veias.
\section{Flebopalia}
\begin{itemize}
\item {Grp. gram.:f.}
\end{itemize}
\begin{itemize}
\item {Utilização:Med.}
\end{itemize}
\begin{itemize}
\item {Proveniência:(Gr. \textunderscore phlebopalia\textunderscore )}
\end{itemize}
Pulsação das veias.
\section{Flebóptero}
\begin{itemize}
\item {Grp. gram.:adj.}
\end{itemize}
\begin{itemize}
\item {Utilização:Zool.}
\end{itemize}
\begin{itemize}
\item {Proveniência:(Do gr. \textunderscore phleps\textunderscore  + \textunderscore pteron\textunderscore )}
\end{itemize}
Diz-se dos insectos, que têm asas venosas.
\section{Fleborragia}
\begin{itemize}
\item {Grp. gram.:f.}
\end{itemize}
\begin{itemize}
\item {Proveniência:(Do gr. \textunderscore phleps\textunderscore , \textunderscore phlebos\textunderscore  + \textunderscore rhagein\textunderscore )}
\end{itemize}
Ruptura de uma veia; hemorragia das veias.
\section{Flebotomia}
\begin{itemize}
\item {Grp. gram.:f.}
\end{itemize}
\begin{itemize}
\item {Proveniência:(Lat. \textunderscore phlebotomia\textunderscore )}
\end{itemize}
Sangria; arte de sangrar.
\section{Flebotómico}
\begin{itemize}
\item {Grp. gram.:adj.}
\end{itemize}
Relativo á flebotomia.
\section{Flebótomo}
\begin{itemize}
\item {Grp. gram.:m.}
\end{itemize}
\begin{itemize}
\item {Proveniência:(Lat. \textunderscore phlebotomus\textunderscore )}
\end{itemize}
Instrumento, usado principalmente na Alemanha, para fazer sangrias.
\section{Flegetonte}
\begin{itemize}
\item {Grp. gram.:m.}
\end{itemize}
\begin{itemize}
\item {Proveniência:(Lat. \textunderscore phlegethon\textunderscore , \textunderscore phlegethontis\textunderscore )}
\end{itemize}
Rio infernal; rio escuro:«\textunderscore desagôam tenebrosos flegetontes.\textunderscore »\textunderscore Viriato Trág.\textunderscore , X, 34.
\section{Flegmão}
\begin{itemize}
\item {Grp. gram.:m.}
\end{itemize}
(V.fleimão)
\section{Flegmasia}
\begin{itemize}
\item {Grp. gram.:f.}
\end{itemize}
\begin{itemize}
\item {Utilização:Med.}
\end{itemize}
\begin{itemize}
\item {Proveniência:(Gr. \textunderscore phlegmasia\textunderscore )}
\end{itemize}
O mesmo que \textunderscore inflamação\textunderscore .
\section{Flegmásico}
\begin{itemize}
\item {Grp. gram.:adj.}
\end{itemize}
Relativo á flegmasia.
\section{Flegmatorragia}
\begin{itemize}
\item {Grp. gram.:f.}
\end{itemize}
\begin{itemize}
\item {Utilização:Med.}
\end{itemize}
\begin{itemize}
\item {Proveniência:(Do gr. \textunderscore phlegma\textunderscore  + \textunderscore rhagein\textunderscore )}
\end{itemize}
Excreção abundante, pelas narinas, de uma mucosidade límpida, não acompanhada de inflamação.
\section{Flegmonosa}
\begin{itemize}
\item {Grp. gram.:adj. f.}
\end{itemize}
\begin{itemize}
\item {Utilização:Med.}
\end{itemize}
\begin{itemize}
\item {Proveniência:(Do lat. \textunderscore phlegmone\textunderscore )}
\end{itemize}
Diz-se de uma variedade de angina, determinada por tumor ou fleimão.
Diz-se da gastrite do cão, e da faringite do cão e do cavalo. Cf. A. Torgo, \textunderscore Carteira de um Veterinário\textunderscore .
\section{Fléolo}
\begin{itemize}
\item {Grp. gram.:m.}
\end{itemize}
\begin{itemize}
\item {Proveniência:(Do gr. \textunderscore phleos\textunderscore )}
\end{itemize}
Gênero de plantas gramíneas, que se dá bem nos terrenos áridos e colinas saibrosas.
\section{Flogístico}
\begin{itemize}
\item {Grp. gram.:adj.}
\end{itemize}
\begin{itemize}
\item {Grp. gram.:M.}
\end{itemize}
\begin{itemize}
\item {Proveniência:(Gr. \textunderscore phlogistikos\textunderscore )}
\end{itemize}
Que desenvolve calor interno.
Produzido por inflamação.
Fluido particular, que se supunha inherente aos corpos, para explicar a combustão.
\section{Flogisto}
\begin{itemize}
\item {Grp. gram.:m.}
\end{itemize}
\begin{itemize}
\item {Proveniência:(Gr. \textunderscore phlogistos\textunderscore )}
\end{itemize}
Fluido, de grande movimento vibratório, manifestado pela sensação do calor e luz, e que, junto a uma substância, explicava, para os antigos químicos, o fenómeno da combustão.
\section{Flogistologia}
\begin{itemize}
\item {Grp. gram.:f.}
\end{itemize}
\begin{itemize}
\item {Proveniência:(Do gr. \textunderscore phlogistos\textunderscore  + \textunderscore logos\textunderscore )}
\end{itemize}
Tratado á cêrca das substâncias combustíveis.
\section{Flogistológico}
\begin{itemize}
\item {Grp. gram.:adj.}
\end{itemize}
Relativo á flogistologia.
\section{Flogogênico}
\begin{itemize}
\item {Grp. gram.:adj.}
\end{itemize}
\begin{itemize}
\item {Proveniência:(Do gr. \textunderscore phlogos\textunderscore  + \textunderscore genes\textunderscore )}
\end{itemize}
Que produz inflamação.
\section{Flogógeno}
\begin{itemize}
\item {Grp. gram.:adj.}
\end{itemize}
O mesmo que \textunderscore flogogênico\textunderscore .
\section{Flogose}
\begin{itemize}
\item {Grp. gram.:f.}
\end{itemize}
\begin{itemize}
\item {Utilização:Med.}
\end{itemize}
\begin{itemize}
\item {Proveniência:(Gr. \textunderscore phlogosis\textunderscore )}
\end{itemize}
O mesmo que \textunderscore flegmasia\textunderscore .
Inflamação ligeira ou superfícial.
\section{Flómide}
\begin{itemize}
\item {Grp. gram.:f.}
\end{itemize}
\begin{itemize}
\item {Proveniência:(Lat. \textunderscore phlomis\textunderscore )}
\end{itemize}
Gênero de plantas labiadas.
\section{Flooplastía}
\begin{itemize}
\item {Grp. gram.:f.}
\end{itemize}
\begin{itemize}
\item {Utilização:Bot.}
\end{itemize}
\begin{itemize}
\item {Proveniência:(Do gr. \textunderscore phloos\textunderscore  + \textunderscore plassein\textunderscore )}
\end{itemize}
Reparação ou renovação da casca das árvores.
\section{Floorrizina}
\begin{itemize}
\item {Grp. gram.:f.}
\end{itemize}
\begin{itemize}
\item {Utilização:Chím.}
\end{itemize}
\begin{itemize}
\item {Proveniência:(Do gr. \textunderscore phloos\textunderscore  + \textunderscore rhiza\textunderscore )}
\end{itemize}
Substância cristalizável, que se extrai das raízes de algumas árvores pomíferas.
\section{Floretato}
\begin{itemize}
\item {Grp. gram.:m.}
\end{itemize}
\begin{itemize}
\item {Proveniência:(De \textunderscore florético\textunderscore )}
\end{itemize}
Combinação do ácido florético com uma base.
\section{Florético}
\begin{itemize}
\item {Grp. gram.:adj.}
\end{itemize}
\begin{itemize}
\item {Proveniência:(Do gr. \textunderscore phloios\textunderscore )}
\end{itemize}
Diz-se de um ácido, resultante da acção da potassa cáustica sôbre a floretina.
\section{Floretina}
\begin{itemize}
\item {Grp. gram.:f.}
\end{itemize}
\begin{itemize}
\item {Proveniência:(Do gr. \textunderscore phloios\textunderscore  + \textunderscore retine\textunderscore )}
\end{itemize}
Matéria orgânica neutra, formada sob a influência dos ácidos mineraes.
\section{Floroglusina}
\begin{itemize}
\item {Grp. gram.:f.}
\end{itemize}
\begin{itemize}
\item {Utilização:Chím.}
\end{itemize}
Fenol triatómico isómero do ácido pirogálico.
\section{Flox}
\begin{itemize}
\item {Grp. gram.:m.}
\end{itemize}
\begin{itemize}
\item {Proveniência:(Lat. \textunderscore flox\textunderscore , \textunderscore flogis\textunderscore )}
\end{itemize}
Gênero de plantas polemoniáceas, cultivadas em jardins e notáveis pelo aroma e beleza das suas flôres.--A fórma exacta seria \textunderscore phloge\textunderscore  ou \textunderscore floge\textunderscore .
\section{Flictena}
\begin{itemize}
\item {Grp. gram.:f.}
\end{itemize}
\begin{itemize}
\item {Utilização:Med.}
\end{itemize}
\begin{itemize}
\item {Proveniência:(Gr. \textunderscore phluktaina\textunderscore )}
\end{itemize}
Pequena empôla vesicular e transparente.
Pústula, de natureza linfática, na conjuntiva ou na córnea.
\section{Flictenóide}
\begin{itemize}
\item {Grp. gram.:adj.}
\end{itemize}
\begin{itemize}
\item {Utilização:Med.}
\end{itemize}
\begin{itemize}
\item {Proveniência:(Do gr. \textunderscore phluktaina\textunderscore  + \textunderscore eidos\textunderscore )}
\end{itemize}
Semelhante á flictena.
\section{Flictenular}
\begin{itemize}
\item {Grp. gram.:adj.}
\end{itemize}
\begin{itemize}
\item {Utilização:Med.}
\end{itemize}
\begin{itemize}
\item {Proveniência:(Do hypoth. \textunderscore phlytênula\textunderscore , dem. de \textunderscore phlyctena\textunderscore )}
\end{itemize}
Que apresenta pequenas flictenas.
\section{Fobia}
\begin{itemize}
\item {Grp. gram.:f.}
\end{itemize}
\begin{itemize}
\item {Proveniência:(Do gr. \textunderscore phobein\textunderscore , temer)}
\end{itemize}
Designação genérica das diferentes espécies de medo mórbido, como agorafobía, a talassofobía, etc.
\section{Fobofobía}
\begin{itemize}
\item {Grp. gram.:f.}
\end{itemize}
\begin{itemize}
\item {Proveniência:(Do gr. \textunderscore phobein\textunderscore  + \textunderscore phobein\textunderscore )}
\end{itemize}
O mesmo que \textunderscore nosofobia\textunderscore .
\section{Fobófobo}
\begin{itemize}
\item {Grp. gram.:m.}
\end{itemize}
O mesmo que \textunderscore nosófobo\textunderscore .
\section{Focáceos}
\begin{itemize}
\item {Grp. gram.:m. pl.}
\end{itemize}
Família de mamíferos, que tem por tipo a foca.
\section{Foceia}
\begin{itemize}
\item {Grp. gram.:f.}
\end{itemize}
\begin{itemize}
\item {Proveniência:(Do lat. \textunderscore Phocaea\textunderscore , n. p.)}
\end{itemize}
Pequeno planeta, descoberto em 1853.
\section{Focena}
\begin{itemize}
\item {Grp. gram.:f.}
\end{itemize}
\begin{itemize}
\item {Proveniência:(Gr. \textunderscore phokaina\textunderscore )}
\end{itemize}
Gênero de cetáceos, a que pertence o porco marinho.
\section{Focenato}
\begin{itemize}
\item {Grp. gram.:m.}
\end{itemize}
Sal, resultante da combinação do ácido focênico com uma base.
\section{Focênico}
\begin{itemize}
\item {Grp. gram.:adj.}
\end{itemize}
\begin{itemize}
\item {Proveniência:(De \textunderscore focenina\textunderscore )}
\end{itemize}
Diz-se de um ácido, que se descobriu nos óleos dos mamíferos marinhos.
\section{Focenina}
\begin{itemize}
\item {Grp. gram.:f.}
\end{itemize}
\begin{itemize}
\item {Proveniência:(Do gr. \textunderscore phokaína\textunderscore )}
\end{itemize}
Princípio gordo dos óleos dos mamíferos marinhos.
\section{Fócio}
\begin{itemize}
\item {Grp. gram.:adj.}
\end{itemize}
\begin{itemize}
\item {Proveniência:(Lat. \textunderscore phocius\textunderscore )}
\end{itemize}
Relativo á Fócida, região da Grécia, onde está o monte Parnaso e a fonte de Castália. Cf. Latino, \textunderscore Or. da Corôa\textunderscore , p. XC.
\section{Focomelia}
\begin{itemize}
\item {Grp. gram.:f.}
\end{itemize}
Estado ou qualidade de quem é focómelo.
\section{Focómelo}
\begin{itemize}
\item {Grp. gram.:m.}
\end{itemize}
\begin{itemize}
\item {Proveniência:(Do gr. \textunderscore phoke\textunderscore  + \textunderscore melos\textunderscore )}
\end{itemize}
Monstro que, sem braços nem pernas, parece têr as mãos e os pés insertos imediatamente no tronco, como sucede com as focas.
\section{Fólada}
\begin{itemize}
\item {Grp. gram.:f.}
\end{itemize}
\begin{itemize}
\item {Proveniência:(Gr. \textunderscore pholas\textunderscore )}
\end{itemize}
Molusco acéfalo.
\section{Foladite}
\begin{itemize}
\item {Grp. gram.:f.}
\end{itemize}
\begin{itemize}
\item {Utilização:Miner.}
\end{itemize}
Fólada fóssil.
\section{Foladito}
\begin{itemize}
\item {Grp. gram.:m.}
\end{itemize}
O mesmo ou melhor que \textunderscore foladite\textunderscore .
\section{Foleosânteas}
\begin{itemize}
\item {fónica:le-o}
\end{itemize}
\begin{itemize}
\item {Grp. gram.:f. pl.}
\end{itemize}
Secção de plantas urticáceas, no sistema de Blume.
\section{Folerite}
\begin{itemize}
\item {Grp. gram.:f.}
\end{itemize}
\begin{itemize}
\item {Utilização:Geol.}
\end{itemize}
O mesmo que \textunderscore folidite\textunderscore .
\section{Folidite}
\begin{itemize}
\item {Grp. gram.:f.}
\end{itemize}
\begin{itemize}
\item {Utilização:Geol.}
\end{itemize}
\begin{itemize}
\item {Proveniência:(Do gr. \textunderscore pholis\textunderscore , escama)}
\end{itemize}
Uma das mais importantes variedades de caulim, a qual apparece em lâminas ou escamas.
\section{Folidito}
\begin{itemize}
\item {Grp. gram.:m.}
\end{itemize}
O mesmo ou melhor que \textunderscore folidite\textunderscore .
\section{Folídoto}
\begin{itemize}
\item {Grp. gram.:adj.}
\end{itemize}
\begin{itemize}
\item {Utilização:Hist. Nat.}
\end{itemize}
\begin{itemize}
\item {Proveniência:(Gr. \textunderscore pholidotos\textunderscore )}
\end{itemize}
Coberto de escamas.
\section{Foma}
\begin{itemize}
\item {Grp. gram.:m.}
\end{itemize}
Gênero de cogumelos.
\section{Fonação}
\begin{itemize}
\item {Grp. gram.:f.}
\end{itemize}
\begin{itemize}
\item {Proveniência:(Do gr. \textunderscore phone\textunderscore )}
\end{itemize}
Producção fisiológica da voz.
\section{Fonador}
\begin{itemize}
\item {Grp. gram.:adj.}
\end{itemize}
\begin{itemize}
\item {Proveniência:(Do gr. \textunderscore phone\textunderscore )}
\end{itemize}
Que produz voz.
Diz-se especialmente do aparelho, formado pelos órgãos da voz.
\section{Fonalidade}
\begin{itemize}
\item {Grp. gram.:f.}
\end{itemize}
\begin{itemize}
\item {Proveniência:(Do gr. \textunderscore phone\textunderscore )}
\end{itemize}
Carácter dos sons de uma língua.
\section{Fonascia}
\begin{itemize}
\item {Grp. gram.:f.}
\end{itemize}
\begin{itemize}
\item {Proveniência:(Gr. \textunderscore phonaskia\textunderscore )}
\end{itemize}
Arte de exercitar a voz.
\section{Fonasco}
\begin{itemize}
\item {Grp. gram.:m.}
\end{itemize}
\begin{itemize}
\item {Proveniência:(Lat. \textunderscore phonascus\textunderscore )}
\end{itemize}
Professor de declamação, entre os antigos.
\section{Fonautógrafo}
\begin{itemize}
\item {Grp. gram.:m.}
\end{itemize}
\begin{itemize}
\item {Proveniência:(Do gr. \textunderscore phone\textunderscore  + \textunderscore autos\textunderscore  + \textunderscore graphein\textunderscore )}
\end{itemize}
Aparelho de acústica, para reproduzir graficamente os sons articulados ou as vibrações sonoras.
\section{Fonema}
\begin{itemize}
\item {Grp. gram.:m.}
\end{itemize}
\begin{itemize}
\item {Proveniência:(Gr. \textunderscore phonema\textunderscore )}
\end{itemize}
Qualquer som articulado.
\section{Fonendoscopia}
\begin{itemize}
\item {Grp. gram.:f.}
\end{itemize}
Aplicação do fonendoscópio.
\section{Fonendoscópico}
\begin{itemize}
\item {Grp. gram.:adj.}
\end{itemize}
Relativo á fonendoscopía.
\section{Fonendoscópio}
\begin{itemize}
\item {Grp. gram.:m.}
\end{itemize}
\begin{itemize}
\item {Proveniência:(Do gr. \textunderscore phone\textunderscore  + \textunderscore endos\textunderscore  + \textunderscore skopein\textunderscore )}
\end{itemize}
Aparelho, inventado recentemente, (1898), pelo professor Bianchi, e que, posto em comunicação com os ouvidos de um médico, póde determinar a situação, fórma e volume das vísceras.
\section{Fonética}
\begin{itemize}
\item {Grp. gram.:f.}
\end{itemize}
\begin{itemize}
\item {Utilização:Philol.}
\end{itemize}
\begin{itemize}
\item {Proveniência:(De \textunderscore fonético\textunderscore )}
\end{itemize}
Estudo dos sons articulados, considerados como elementos dos vocábulos.
\section{Foneticamente}
\begin{itemize}
\item {Grp. gram.:adv.}
\end{itemize}
De modo fonético; segundo a fonética.
\section{Foneticismo}
\begin{itemize}
\item {Grp. gram.:m.}
\end{itemize}
O mesmo que \textunderscore fonetismo\textunderscore .
\section{Foneticista}
\begin{itemize}
\item {Grp. gram.:m.}
\end{itemize}
Filólogo, que trata especialmente de fonética.
\section{Fonético}
\begin{itemize}
\item {Grp. gram.:adj.}
\end{itemize}
\begin{itemize}
\item {Proveniência:(Gr. \textunderscore phonetikos\textunderscore )}
\end{itemize}
Relativo á voz ou ao som das palavras.
\section{Fonetismo}
\begin{itemize}
\item {Grp. gram.:m.}
\end{itemize}
\begin{itemize}
\item {Proveniência:(De \textunderscore fonético\textunderscore )}
\end{itemize}
Maneira de representar as ideias, representando os sons.
\section{Fonetista}
\begin{itemize}
\item {Grp. gram.:m.}
\end{itemize}
O mesmo ou melhor que \textunderscore foneticista\textunderscore .
\section{Fónica}
\begin{itemize}
\item {Grp. gram.:f.}
\end{itemize}
\begin{itemize}
\item {Utilização:Phýs.}
\end{itemize}
Arte de combinar os sons, segundo as leis da acústica.
(Fem. de \textunderscore fónico\textunderscore )
\section{Fónico}
\begin{itemize}
\item {Grp. gram.:adj.}
\end{itemize}
\begin{itemize}
\item {Proveniência:(Do gr. \textunderscore phone\textunderscore )}
\end{itemize}
Relativo á voz ou ao som.
\section{Fono...}
\begin{itemize}
\item {Proveniência:(Do gr. \textunderscore phone\textunderscore )}
\end{itemize}
Elemento, que entra na formação de várias palavras, significando \textunderscore som\textunderscore  ou \textunderscore voz\textunderscore .
\section{Fonocântico}
\begin{itemize}
\item {Grp. gram.:adj.}
\end{itemize}
\begin{itemize}
\item {Utilização:Phýs.}
\end{itemize}
\begin{itemize}
\item {Proveniência:(Do gr. \textunderscore phone\textunderscore  + \textunderscore kamptein\textunderscore )}
\end{itemize}
Relativo á reflexão do som.
\section{Fonofobia}
\begin{itemize}
\item {Grp. gram.:f.}
\end{itemize}
\begin{itemize}
\item {Utilização:Med.}
\end{itemize}
\begin{itemize}
\item {Proveniência:(Do gr. \textunderscore phone\textunderscore  + \textunderscore phobein\textunderscore )}
\end{itemize}
Mêdo de falar em voz alta. Cf. Sousa Martins, \textunderscore Nosograph.\textunderscore 
\section{Fonófobo}
\begin{itemize}
\item {Grp. gram.:adj.}
\end{itemize}
\begin{itemize}
\item {Utilização:Med.}
\end{itemize}
Que padece fonofobia.
\section{Fonograma}
\begin{itemize}
\item {Grp. gram.:m.}
\end{itemize}
\begin{itemize}
\item {Proveniência:(Do gr. \textunderscore phone\textunderscore  + \textunderscore gramma\textunderscore )}
\end{itemize}
Figura, obtida pelos processos da fonografia.
\section{Fonografia}
\begin{itemize}
\item {Grp. gram.:f.}
\end{itemize}
\begin{itemize}
\item {Proveniência:(De \textunderscore fonógrafo\textunderscore )}
\end{itemize}
Modo de representar os sons por meio de palavras.
Representação gráfica das vibrações dos corpos sonoros.
\section{Fonográfico}
\begin{itemize}
\item {Grp. gram.:adj.}
\end{itemize}
Relativo á fonografia.
\section{Fonógrafo}
\begin{itemize}
\item {Grp. gram.:m.}
\end{itemize}
\begin{itemize}
\item {Proveniência:(Do gr. \textunderscore phone\textunderscore  + \textunderscore graphein\textunderscore )}
\end{itemize}
Instrumento, que conserva e reproduz os sons ou vibrações sonoras.
\section{Fonólita}
\begin{itemize}
\item {Grp. gram.:f.}
\end{itemize}
O mesmo que \textunderscore fonólito\textunderscore .
\section{Fonolítico}
\begin{itemize}
\item {Grp. gram.:adj.}
\end{itemize}
Relativo ao fonólito.
\section{Fonólito}
\begin{itemize}
\item {Grp. gram.:f.}
\end{itemize}
\begin{itemize}
\item {Proveniência:(Do gr. \textunderscore phone\textunderscore  + \textunderscore lithos\textunderscore )}
\end{itemize}
Gênero de rochas vulcânicas, que emitem um som especial, quando percutidas por um corpo duro.
\section{Fonologia}
\begin{itemize}
\item {Grp. gram.:f.}
\end{itemize}
\begin{itemize}
\item {Utilização:Philol.}
\end{itemize}
\begin{itemize}
\item {Proveniência:(Do gr. \textunderscore phone\textunderscore  + \textunderscore logos\textunderscore )}
\end{itemize}
Tratado dos sons elementares e fundamentaes das línguas, das modificações dêsses sons representados por vocábulos, e da correcta pronúncia dêstes.
\section{Fonológico}
\begin{itemize}
\item {Grp. gram.:adj.}
\end{itemize}
Relativo á fonologia.
\section{Fonometria}
\begin{itemize}
\item {Grp. gram.:f.}
\end{itemize}
Aplicação do fonómetro.
\section{Fonómetro}
\begin{itemize}
\item {Grp. gram.:m.}
\end{itemize}
\begin{itemize}
\item {Proveniência:(Do gr. \textunderscore phone\textunderscore  + \textunderscore metron\textunderscore )}
\end{itemize}
Instrumento, com que se mede a intensidade do som ou da voz.
\section{Fonospasmo}
\begin{itemize}
\item {Grp. gram.:m.}
\end{itemize}
\begin{itemize}
\item {Utilização:Med.}
\end{itemize}
\begin{itemize}
\item {Proveniência:(De \textunderscore phono...\textunderscore  + \textunderscore espasmo\textunderscore )}
\end{itemize}
Espasmo ou convulsão, que acompanha a emissão da voz.
\section{Foranto}
\begin{itemize}
\item {Grp. gram.:m.}
\end{itemize}
\begin{itemize}
\item {Proveniência:(Do gr. \textunderscore phoros\textunderscore  + \textunderscore anthos\textunderscore )}
\end{itemize}
Nome, dado por alguns botânicos ao receptáculo das flôres compostas.
\section{Fórmio}
\begin{itemize}
\item {Grp. gram.:m.}
\end{itemize}
\begin{itemize}
\item {Proveniência:(Do gr. \textunderscore phormion\textunderscore , fio)}
\end{itemize}
Gênero de plantas liliáceas, (\textunderscore phormium temax\textunderscore ), também conhecido por \textunderscore linho da Nova-Zelândia\textunderscore .
\section{Foronomia}
\begin{itemize}
\item {Grp. gram.:f.}
\end{itemize}
\begin{itemize}
\item {Proveniência:(Do gr. \textunderscore phora\textunderscore  + \textunderscore nomos\textunderscore )}
\end{itemize}
Ciência das leis do equilíbrio e do movimento dos corpos.
Mecânica.
\section{Fosfatado}
\begin{itemize}
\item {Grp. gram.:adj.}
\end{itemize}
Que se acha em estado de fosfato; que tem fosfato.
\section{Fosfático}
\begin{itemize}
\item {Grp. gram.:adj.}
\end{itemize}
Formado de fosfato; relativo a fosfato.
\section{Fosfatina}
\begin{itemize}
\item {Grp. gram.:f.}
\end{itemize}
\begin{itemize}
\item {Proveniência:(De \textunderscore fosfato\textunderscore )}
\end{itemize}
Preparação, em que entra farinha de arroz e de tapioca, fécula de batata, araruta, cacau e fosfato de cal, para alimentação de crianças. Cf. \textunderscore Jorn.-do-Comm.\textunderscore , do Rio, de 6-VII-902.
\section{Fosfato}
\begin{itemize}
\item {Grp. gram.:m.}
\end{itemize}
\begin{itemize}
\item {Proveniência:(De \textunderscore fósforo\textunderscore )}
\end{itemize}
Sal, que resulta da combinação do ácido fosfórico com uma base.
\section{Fosfaturia}
\begin{itemize}
\item {Grp. gram.:f.}
\end{itemize}
\begin{itemize}
\item {Utilização:Med.}
\end{itemize}
\begin{itemize}
\item {Proveniência:(De \textunderscore phosphato\textunderscore  + gr. \textunderscore ourein\textunderscore )}
\end{itemize}
Perda de fosfato pela urina.
\section{Fosfena}
\begin{itemize}
\item {Grp. gram.:f.}
\end{itemize}
O mesmo que \textunderscore fosfeno\textunderscore .
\section{Fosfeno}
\begin{itemize}
\item {Grp. gram.:m.}
\end{itemize}
\begin{itemize}
\item {Proveniência:(Do gr. \textunderscore phos\textunderscore  + \textunderscore phainos\textunderscore )}
\end{itemize}
Impressão luminosa, que resulta da compressão do ôlho, estando as pálpebras fechadas.
\section{Fosfito}
\begin{itemize}
\item {Grp. gram.:m.}
\end{itemize}
Gênero de saes, produzidos pela combinação do ácido fosforoso com as bases.
\section{Fosfoglicerato}
\begin{itemize}
\item {Grp. gram.:m.}
\end{itemize}
Sal, resultante do ácido fosfoglicérico com uma base.
\section{Fosfoglicérico}
\begin{itemize}
\item {Grp. gram.:adj.}
\end{itemize}
Diz-se de um ácido, resultante do desdobramento do protagão, sob a acção da água de barita concentrada, ou que se fórma pela mistura de glicerina com ácido fosfórico.
\section{Fosforar}
\begin{itemize}
\item {Grp. gram.:v. t.}
\end{itemize}
Combinar \textunderscore ou\textunderscore misturar com fósforo.
\section{Fosforear}
\begin{itemize}
\item {Grp. gram.:v. i.}
\end{itemize}
Brilhar como o fósforo.
\section{Fosforeira}
\begin{itemize}
\item {Grp. gram.:f.}
\end{itemize}
Caixinha ou utensílio, para guardar fósforos.
\section{Fosforeiro}
\begin{itemize}
\item {Grp. gram.:m.}
\end{itemize}
Aquele que trabalha no fabríco de fósforos.
\section{Fosforejante}
\begin{itemize}
\item {Grp. gram.:adj.}
\end{itemize}
Que fosforeja.
\section{Fosforejar}
\begin{itemize}
\item {Grp. gram.:v. i.}
\end{itemize}
\begin{itemize}
\item {Utilização:Neol.}
\end{itemize}
Brilhar com fósforo inflamado; chamejar.
\section{Fosfóreo}
\begin{itemize}
\item {Grp. gram.:adj.}
\end{itemize}
\begin{itemize}
\item {Proveniência:(Lat. \textunderscore phosphoreus\textunderscore )}
\end{itemize}
O mesmo que \textunderscore fosfórico\textunderscore ; que tem fósforo.
\section{Fosforescência}
\begin{itemize}
\item {Grp. gram.:f.}
\end{itemize}
Propriedade dos corpos fosforescentes.
\section{Fosforescente}
\begin{itemize}
\item {Grp. gram.:adj.}
\end{itemize}
\begin{itemize}
\item {Proveniência:(De \textunderscore fósforo\textunderscore )}
\end{itemize}
Que brilha na obscuridade, sem calor nem combustão.
Que, sendo friccionado, se torna luminoso ou se sujeita a uma descarga electrica.
\section{Fosforescer}
\begin{itemize}
\item {Grp. gram.:v. i.}
\end{itemize}
\begin{itemize}
\item {Proveniência:(De \textunderscore fósforo\textunderscore )}
\end{itemize}
Lançar brilho fosforescente. Cf. Latino, \textunderscore Camões\textunderscore , 119.
\section{Fosforeto}
\begin{itemize}
\item {fónica:forê}
\end{itemize}
\begin{itemize}
\item {Grp. gram.:m.}
\end{itemize}
\begin{itemize}
\item {Proveniência:(De \textunderscore fósforo\textunderscore )}
\end{itemize}
Combinação mineral ou orgânica, que contém fósforo como elemento electro-negativo.
\section{Fosfórico}
\begin{itemize}
\item {Grp. gram.:adj.}
\end{itemize}
\begin{itemize}
\item {Utilização:Pop.}
\end{itemize}
Relativo a fósforo.
Que brilha como fósforo.
Diz-se de um ácido, formado pela combustão do fósforo.
Embaraçado; difícil.
Irascível.
\section{Fosforíforo}
\begin{itemize}
\item {Grp. gram.:adj.}
\end{itemize}
\begin{itemize}
\item {Proveniência:(Do gr. \textunderscore phosphoros\textunderscore  + \textunderscore phoros\textunderscore )}
\end{itemize}
Diz-se dos animaes, em que uma parte do corpo é fosforescente.
\section{Fosforinos}
\begin{itemize}
\item {Grp. gram.:m. pl.}
\end{itemize}
Uma das quatro ordens dos oxisaes, a qual comprehende a turquesa, o nitro, etc.
(Cp. \textunderscore fósforo\textunderscore ^1)
\section{Fosforista}
\begin{itemize}
\item {Grp. gram.:m.}
\end{itemize}
Manipulador de fósforos.
\section{Fosforita}
\begin{itemize}
\item {Grp. gram.:f.}
\end{itemize}
O mesmo que \textunderscore fosforito\textunderscore .
\section{Fosforite}
\begin{itemize}
\item {Grp. gram.:f.}
\end{itemize}
O mesmo que \textunderscore fosforito\textunderscore .
\section{Fosforito}
\begin{itemize}
\item {Grp. gram.:m.}
\end{itemize}
\begin{itemize}
\item {Proveniência:(De \textunderscore fósforo\textunderscore )}
\end{itemize}
Fosfato de ferro natural.
\section{Fosforização}
\begin{itemize}
\item {Grp. gram.:f.}
\end{itemize}
Acto ou efeito de fosforizar.
Influência ou formação de fosfato calcário na economia animal.
\section{Fosforizar}
\begin{itemize}
\item {Grp. gram.:v. t.}
\end{itemize}
\begin{itemize}
\item {Proveniência:(De \textunderscore fósforo\textunderscore )}
\end{itemize}
Tornar fosfórico.
Converter em fosfato.
\section{Fósforo}
\begin{itemize}
\item {Grp. gram.:m.}
\end{itemize}
\begin{itemize}
\item {Grp. gram.:Pl.}
\end{itemize}
\begin{itemize}
\item {Utilização:Chul.}
\end{itemize}
\begin{itemize}
\item {Proveniência:(Gr. \textunderscore phósphoros\textunderscore )}
\end{itemize}
Corpo simples, combustível, luminoso na obscuridade e ardendo ao contacto do ar.
Palito ou pavio, em cuja extremidade há uma substância, que se inflama com a fricção.
\textunderscore Fósforos de espera galego\textunderscore , fósforos de enxôfre.
\section{Fosforoscópio}
\begin{itemize}
\item {Grp. gram.:m.}
\end{itemize}
\begin{itemize}
\item {Proveniência:(Do gr. \textunderscore phosphoros\textunderscore  + \textunderscore skopein\textunderscore )}
\end{itemize}
Instrumento, para observar a fosforescência dos corpos.
\section{Fosforoso}
\begin{itemize}
\item {Grp. gram.:adj.}
\end{itemize}
\begin{itemize}
\item {Proveniência:(De \textunderscore fósforo\textunderscore )}
\end{itemize}
Fosfóreo.
Diz-se do ácido, também chamado fosfórico.
\section{Fosfosiderito}
\begin{itemize}
\item {Grp. gram.:m.}
\end{itemize}
\begin{itemize}
\item {Utilização:Miner.}
\end{itemize}
\begin{itemize}
\item {Proveniência:(De \textunderscore fósforo\textunderscore  + gr. \textunderscore sideros\textunderscore )}
\end{itemize}
Fosfato hidratado de ferro.
\section{Fosfovinato}
\begin{itemize}
\item {Grp. gram.:m.}
\end{itemize}
Combinação do ácido fosfovínico com uma base.
\section{Fosfovínico}
\begin{itemize}
\item {Grp. gram.:adj.}
\end{itemize}
\begin{itemize}
\item {Proveniência:(De \textunderscore fosfórico\textunderscore  + \textunderscore vínico\textunderscore )}
\end{itemize}
Diz-se de um ácido, composto do ácido fosfórico e elementos de álcool.
\section{Fosgênio}
\begin{itemize}
\item {Grp. gram.:m.}
\end{itemize}
\begin{itemize}
\item {Utilização:Chím.}
\end{itemize}
\begin{itemize}
\item {Proveniência:(Do gr. \textunderscore phos\textunderscore  + \textunderscore genos\textunderscore )}
\end{itemize}
Gás, resultante da acção dos raios solares numa mistura, em partes iguaes, de gás chloro e de gás óxido de carbóne.
\section{Fotínia}
\begin{itemize}
\item {Grp. gram.:f.}
\end{itemize}
\begin{itemize}
\item {Proveniência:(Do gr. \textunderscore photeinos\textunderscore )}
\end{itemize}
Gênero de árvores rosáceas da Califórnia e da Ásia tropical.
\section{Fotinianos}
\begin{itemize}
\item {Grp. gram.:m. pl.}
\end{itemize}
Herejes do século IV, que negavam ao Espírito-Santo a personalidade divina e sustentavam que Jesus era filho de José.
\section{Fotismo}
\begin{itemize}
\item {Grp. gram.:m.}
\end{itemize}
\begin{itemize}
\item {Proveniência:(Do gr. \textunderscore phos\textunderscore , \textunderscore photos\textunderscore )}
\end{itemize}
Sensação visual secundária. Cf. R. Galvão, \textunderscore Vocab.\textunderscore 
\section{Foto...}
\begin{itemize}
\item {Grp. gram.:pref.}
\end{itemize}
\begin{itemize}
\item {Proveniência:(Do gr. \textunderscore phos\textunderscore , \textunderscore photos\textunderscore )}
\end{itemize}
(designativo de \textunderscore luz\textunderscore )
\section{Fotocalco}
\begin{itemize}
\item {Grp. gram.:m.}
\end{itemize}
\begin{itemize}
\item {Proveniência:(Do gr. \textunderscore phos\textunderscore , \textunderscore photos\textunderscore  + \textunderscore khalkos\textunderscore )}
\end{itemize}
Pequeno aparelho, espécie de câmara escura simplificada, para facilitar o desenho de uma paisagem, de um monumento, etc.
\section{Fotocartografia}
\begin{itemize}
\item {Grp. gram.:f.}
\end{itemize}
\begin{itemize}
\item {Proveniência:(De \textunderscore foto...\textunderscore  + \textunderscore cartografia\textunderscore )}
\end{itemize}
Aplicação da fotografia a reproduções cartográficas.
\section{Fotocerâmica}
\begin{itemize}
\item {Grp. gram.:f.}
\end{itemize}
\begin{itemize}
\item {Proveniência:(De \textunderscore foto...\textunderscore  + \textunderscore cerâmica\textunderscore )}
\end{itemize}
Aplicação da fotografia á reproducção de desenhos em loiça.
\section{Fotocolografia}
\begin{itemize}
\item {Grp. gram.:f.}
\end{itemize}
Reprodução fotográfica, que tem por base a gelatina.
(Do gr.\textunderscore  phos\textunderscore , \textunderscore photos\textunderscore  + \textunderscore kolla\textunderscore  + \textunderscore graphein\textunderscore )
\section{Fotocópia}
\begin{itemize}
\item {Grp. gram.:f.}
\end{itemize}
\begin{itemize}
\item {Proveniência:(De \textunderscore foto...\textunderscore  + \textunderscore cópia\textunderscore )}
\end{itemize}
Reprodução de uma imagem em papel químico, impressionável pela luz e mediante uma matriz transparente.
\section{Fotocromaticamente}
\begin{itemize}
\item {Grp. gram.:adv.}
\end{itemize}
De modo fotocromático.
\section{Fotocromático}
\begin{itemize}
\item {Grp. gram.:adj.}
\end{itemize}
\begin{itemize}
\item {Proveniência:(De \textunderscore foto...\textunderscore  + \textunderscore cromático\textunderscore )}
\end{itemize}
Relativo á reprodução das côres pela fotografia.
\section{Fotocromografia}
\begin{itemize}
\item {Grp. gram.:f.}
\end{itemize}
\begin{itemize}
\item {Proveniência:(De \textunderscore foto...\textunderscore  + \textunderscore cromografia\textunderscore )}
\end{itemize}
Processo fotográfico, com que se obtém a imagem colorida.
\section{Fotodoscópio}
\begin{itemize}
\item {Grp. gram.:m.}
\end{itemize}
\begin{itemize}
\item {Utilização:Phýs.}
\end{itemize}
\begin{itemize}
\item {Proveniência:(Do gr. \textunderscore photodes\textunderscore  + \textunderscore skopein\textunderscore )}
\end{itemize}
Aparelho para observar a luz.
\section{Fotofulgural}
\begin{itemize}
\item {Grp. gram.:adj.}
\end{itemize}
\begin{itemize}
\item {Utilização:Neol.}
\end{itemize}
\begin{itemize}
\item {Proveniência:(De \textunderscore foto...\textunderscore  + \textunderscore fulgural\textunderscore )}
\end{itemize}
Relativo aos processos fotográficos, que se exercem através dos corpos opacos.
\section{Fotogênico}
\begin{itemize}
\item {Grp. gram.:adj.}
\end{itemize}
Que produz imagens por meio da luz.
Que se representa bem pela fotografia.
(Cp. \textunderscore fotogênio\textunderscore )
\section{Fotoquímica}
\begin{itemize}
\item {Grp. gram.:f.}
\end{itemize}
\begin{itemize}
\item {Proveniência:(De \textunderscore foto...\textunderscore  + \textunderscore química\textunderscore )}
\end{itemize}
Teoria das acções químicas da luz.
\section{Fíceas}
\begin{itemize}
\item {Grp. gram.:f. pl.}
\end{itemize}
\begin{itemize}
\item {Proveniência:(Do gr. \textunderscore phukos\textunderscore , alga)}
\end{itemize}
Plantas aquáticas, de organização simples e fórma variada.
\section{Ficite}
\begin{itemize}
\item {Grp. gram.:f.}
\end{itemize}
\begin{itemize}
\item {Utilização:Chím.}
\end{itemize}
\begin{itemize}
\item {Proveniência:(Do gr. \textunderscore phukos\textunderscore )}
\end{itemize}
Substância cristalina, que se acha numa alga, e é o \textunderscore protococcus vulgaris\textunderscore .
\section{Ficociano}
\begin{itemize}
\item {Grp. gram.:m.}
\end{itemize}
\begin{itemize}
\item {Proveniência:(Do gr. \textunderscore phukos\textunderscore  + \textunderscore kuanon\textunderscore )}
\end{itemize}
Substância còrante, azulada, extraida de certas algas.
\section{Ficóide}
\begin{itemize}
\item {Grp. gram.:adj.}
\end{itemize}
\begin{itemize}
\item {Proveniência:(Do gr. \textunderscore phukos\textunderscore  + \textunderscore eidos\textunderscore )}
\end{itemize}
Semelhante ás algas.
\section{Ficolíchens}
\begin{itemize}
\item {Grp. gram.:m. pl.}
\end{itemize}
\begin{itemize}
\item {Proveniência:(Do gr. \textunderscore phukon\textunderscore  + \textunderscore likhen\textunderscore )}
\end{itemize}
Líchens que, pela sua configuração, se aproximam das algas.
\section{Ficologia}
\begin{itemize}
\item {Grp. gram.:f.}
\end{itemize}
\begin{itemize}
\item {Proveniência:(Do gr. \textunderscore phukos\textunderscore  + \textunderscore logos\textunderscore )}
\end{itemize}
Parte da Botânica, que trata das algas.
\section{Ficológico}
\begin{itemize}
\item {Grp. gram.:adj.}
\end{itemize}
Relativo á ficologia.
\section{Ficologista}
\begin{itemize}
\item {Grp. gram.:m.}
\end{itemize}
Naturalista, que trata de ficologia.
\section{Filactera}
\begin{itemize}
\item {Grp. gram.:f.}
\end{itemize}
Espécie de banda ou bandeirola que, por cima dos escudos ou insuladamente, exibe uma divisa ou legenda.
(Cp. \textunderscore filactério\textunderscore )
\section{Filactério}
\begin{itemize}
\item {Grp. gram.:m.}
\end{itemize}
\begin{itemize}
\item {Proveniência:(Lat. \textunderscore phylacterium\textunderscore )}
\end{itemize}
Nome, que os antigos davam a amuletos.
Pedaço de pele ou pergaminho, em que estavam escritos os mandamentos de Deus, e que os Judeus traziam consigo.
\section{Filânteas}
\begin{itemize}
\item {Grp. gram.:f.}
\end{itemize}
\begin{itemize}
\item {Utilização:Bot.}
\end{itemize}
\begin{itemize}
\item {Proveniência:(De \textunderscore filanto\textunderscore )}
\end{itemize}
Tríbo de euforbiáceas.
\section{Filanto}
\begin{itemize}
\item {Grp. gram.:adj.}
\end{itemize}
\begin{itemize}
\item {Utilização:Bot.}
\end{itemize}
\begin{itemize}
\item {Grp. gram.:M.}
\end{itemize}
\begin{itemize}
\item {Proveniência:(Lat. \textunderscore phyllantes\textunderscore )}
\end{itemize}
Cujas flôres repoisam sôbre as fôlhas.
Gênero de plantas euforbiáceas.
\section{Filarco}
\begin{itemize}
\item {Grp. gram.:m.}
\end{itemize}
\begin{itemize}
\item {Proveniência:(Lat. \textunderscore phylarchus\textunderscore )}
\end{itemize}
Chefe de tríbo, nos primeiros tempos da república ateniense.
\section{Filídea}
\begin{itemize}
\item {Grp. gram.:f.}
\end{itemize}
\begin{itemize}
\item {Proveniência:(Do gr. \textunderscore phullon\textunderscore  + \textunderscore eidos\textunderscore )}
\end{itemize}
Gênero de moluscos gasterópodes.
\section{Filite}
\begin{itemize}
\item {Grp. gram.:f.}
\end{itemize}
O mesmo que \textunderscore filito\textunderscore .
\section{Filito}
\begin{itemize}
\item {Grp. gram.:m.}
\end{itemize}
\begin{itemize}
\item {Utilização:Miner.}
\end{itemize}
\begin{itemize}
\item {Proveniência:(Do gr. \textunderscore phullon\textunderscore )}
\end{itemize}
Variedade de cloritóide.
\section{Filo}
\begin{itemize}
\item {Grp. gram.:m.}
\end{itemize}
\begin{itemize}
\item {Utilização:Bot.}
\end{itemize}
\begin{itemize}
\item {Proveniência:(Lat. \textunderscore phyllon\textunderscore )}
\end{itemize}
O mesmo que \textunderscore sépala\textunderscore , segundo Linck.
\section{Filode}
\begin{itemize}
\item {Grp. gram.:f.}
\end{itemize}
\begin{itemize}
\item {Utilização:Bot.}
\end{itemize}
\begin{itemize}
\item {Proveniência:(Do gr. \textunderscore phullodes\textunderscore )}
\end{itemize}
Pecíolo muito largo, que tomou a aparência de uma fôlha, mas não chegou a formá-la.
\section{Filódio}
\begin{itemize}
\item {Grp. gram.:m.}
\end{itemize}
O mesmo que \textunderscore filode\textunderscore .
\section{Filófago}
\begin{itemize}
\item {Grp. gram.:adj.}
\end{itemize}
\begin{itemize}
\item {Grp. gram.:M. pl.}
\end{itemize}
\begin{itemize}
\item {Proveniência:(Do gr. \textunderscore phullon\textunderscore  + \textunderscore phagein\textunderscore )}
\end{itemize}
Que se alimenta de fôlhas.
Insectos filóphagos.
\section{Filóide}
\begin{itemize}
\item {Grp. gram.:adj.}
\end{itemize}
\begin{itemize}
\item {Grp. gram.:M.}
\end{itemize}
\begin{itemize}
\item {Proveniência:(Do gr. \textunderscore phullon\textunderscore  + \textunderscore eidos\textunderscore )}
\end{itemize}
Que tem a fórma de uma fôlha.
O mesmo que \textunderscore filode\textunderscore .
\section{Filoidinação}
\begin{itemize}
\item {Grp. gram.:f.}
\end{itemize}
\begin{itemize}
\item {Utilização:Bot.}
\end{itemize}
Transformação lenta das fôlhas em filóides.
\section{Filolóbeas}
\begin{itemize}
\item {Grp. gram.:f. pl.}
\end{itemize}
\begin{itemize}
\item {Utilização:Bot.}
\end{itemize}
Um dos dois grupos, em que De-Candolle dividiu a fam. das leguminosas.
\section{Filoma}
\begin{itemize}
\item {Grp. gram.:m.}
\end{itemize}
\begin{itemize}
\item {Utilização:Bot.}
\end{itemize}
\begin{itemize}
\item {Proveniência:(Do gr. \textunderscore phullon\textunderscore )}
\end{itemize}
Conjunto de germes, destinados á produção das fôlhas.
\section{Filópodes}
\begin{itemize}
\item {Grp. gram.:m. pl.}
\end{itemize}
\begin{itemize}
\item {Utilização:Zool.}
\end{itemize}
\begin{itemize}
\item {Proveniência:(Do gr. \textunderscore phullon\textunderscore  + \textunderscore pous\textunderscore , \textunderscore podos\textunderscore )}
\end{itemize}
Ordem de crustáceos, de patas dilatadas, em fórma de lâminas delgadas.
\section{Filoptosia}
\begin{itemize}
\item {Grp. gram.:f.}
\end{itemize}
\begin{itemize}
\item {Proveniência:(Do gr. \textunderscore phullon\textunderscore  + \textunderscore ptosis\textunderscore )}
\end{itemize}
Moléstia vegetal, caracterizada pela quéda das fôlhas, fóra do tempo próprio.
\section{Fotofobia}
\begin{itemize}
\item {Grp. gram.:f.}
\end{itemize}
\begin{itemize}
\item {Proveniência:(De \textunderscore foto...\textunderscore  + \textunderscore fobia\textunderscore )}
\end{itemize}
Aversão á luz.
\section{Fotóforo}
\begin{itemize}
\item {Grp. gram.:adj.}
\end{itemize}
\begin{itemize}
\item {Proveniência:(Do gr. \textunderscore phos\textunderscore , \textunderscore photos\textunderscore  + \textunderscore phoros\textunderscore )}
\end{itemize}
Diz-se de todo aparelho, que permitte obter um feixe luminoso, dirigido sôbre um ponto dado.
\section{Fotografar}
\begin{itemize}
\item {Grp. gram.:v. t.}
\end{itemize}
\begin{itemize}
\item {Utilização:Fig.}
\end{itemize}
\begin{itemize}
\item {Proveniência:(De \textunderscore fotógrafo\textunderscore )}
\end{itemize}
Reproduzir pela fotografia a imagem de; retratar.
Descrever exactamente.
\section{Fotografia}
\begin{itemize}
\item {Grp. gram.:f.}
\end{itemize}
\begin{itemize}
\item {Utilização:Fig.}
\end{itemize}
\begin{itemize}
\item {Proveniência:(De \textunderscore fotógrafo\textunderscore )}
\end{itemize}
Processo ou arte de fixar, numa chapa sensível, pelo auxilio da luz, a imagem dos objectos, que estão deante de uma câmara escura.
Cópia fiel, reprodução exacta.
\section{Fotograficamente}
\begin{itemize}
\item {Grp. gram.:adv.}
\end{itemize}
\begin{itemize}
\item {Proveniência:(De \textunderscore fotográfico\textunderscore )}
\end{itemize}
Por meio de fotografia.
\section{Fotográfico}
\begin{itemize}
\item {Grp. gram.:adj.}
\end{itemize}
Relativo á fotografia.
\section{Fotógrafo}
\begin{itemize}
\item {Grp. gram.:m.}
\end{itemize}
\begin{itemize}
\item {Proveniência:(Do gr. \textunderscore phos\textunderscore , \textunderscore photos\textunderscore  + \textunderscore graphein\textunderscore )}
\end{itemize}
Aquele que se ocupa de fotografia, ou que exerce a fotografia.
\section{Fotograma}
\begin{itemize}
\item {Grp. gram.:m.}
\end{itemize}
\begin{itemize}
\item {Proveniência:(De \textunderscore foto...\textunderscore  + \textunderscore grama\textunderscore )}
\end{itemize}
Qualquer reprodução fotográfica.
\section{Fotogravura}
\begin{itemize}
\item {Grp. gram.:f.}
\end{itemize}
\begin{itemize}
\item {Proveniência:(De \textunderscore foto...\textunderscore  + \textunderscore gravura\textunderscore )}
\end{itemize}
Conjuncto dos processos fotográficos, por meio dos quaes se produzem pranchas gravadas.
\section{Fotolitografia}
\begin{itemize}
\item {Grp. gram.:f.}
\end{itemize}
\begin{itemize}
\item {Proveniência:(De \textunderscore foto...\textunderscore  + \textunderscore litografia\textunderscore )}
\end{itemize}
Processo, com que se transporta para a pedra litográfica uma prova fotográfica.
\section{Fotologia}
\begin{itemize}
\item {Grp. gram.:f.}
\end{itemize}
\begin{itemize}
\item {Proveniência:(Do gr. \textunderscore phos\textunderscore , \textunderscore photos\textunderscore  + \textunderscore logos\textunderscore )}
\end{itemize}
Tratado á cêrca da luz.
\section{Fotológico}
\begin{itemize}
\item {Grp. gram.:adj.}
\end{itemize}
Relativo á fotologia.
\section{Fotomagnético}
\begin{itemize}
\item {Grp. gram.:adj.}
\end{itemize}
\begin{itemize}
\item {Proveniência:(De \textunderscore foto...\textunderscore  + \textunderscore magnético\textunderscore )}
\end{itemize}
Relativo aos fenómenos magnéticos, devidos á acção da luz.
\section{Fotometria}
\begin{itemize}
\item {Grp. gram.:f.}
\end{itemize}
Aplicação do fotómetro.
\section{Fotométrico}
\begin{itemize}
\item {Grp. gram.:adj.}
\end{itemize}
Relativo á fotometria.
\section{Fotómetro}
\begin{itemize}
\item {Grp. gram.:m.}
\end{itemize}
\begin{itemize}
\item {Proveniência:(Do gr. \textunderscore phos\textunderscore , \textunderscore photos\textunderscore  + \textunderscore metron\textunderscore )}
\end{itemize}
Instrumento, com que se avalia a intensidade da luz.
\section{Fotomicrografia}
\begin{itemize}
\item {Grp. gram.:f.}
\end{itemize}
\begin{itemize}
\item {Proveniência:(Do gr. \textunderscore phos\textunderscore , \textunderscore photos\textunderscore  + \textunderscore mikos\textunderscore  + \textunderscore graphein\textunderscore )}
\end{itemize}
Reprodução fotográfica de objectos muito pequenos ou microscópicos.
\section{Fotomicrográfico}
\begin{itemize}
\item {Grp. gram.:adj.}
\end{itemize}
Relativo á fotomicrografia.
\section{Fotominiatura}
\begin{itemize}
\item {Grp. gram.:f.}
\end{itemize}
Processo para reduzir a pequenas dimensões, com o auxílio da fotografia, quadros, paisagens, desenhos, etc.
\section{Fotominiaturista}
\begin{itemize}
\item {Grp. gram.:m.  e  f.}
\end{itemize}
Pessôa, que exerce a fotominiatura.
\section{Fotopsia}
\begin{itemize}
\item {Grp. gram.:f.}
\end{itemize}
\begin{itemize}
\item {Proveniência:(Do gr. \textunderscore phos\textunderscore , \textunderscore photos\textunderscore  + \textunderscore ops\textunderscore )}
\end{itemize}
Visão de traços luminosos que não existem.
\section{Fotosculptura}
\begin{itemize}
\item {Grp. gram.:f.}
\end{itemize}
\begin{itemize}
\item {Proveniência:(De \textunderscore foto...\textunderscore  + \textunderscore esculptura\textunderscore )}
\end{itemize}
Processo fotográfico, com que, reunindo os perfis de uma pessôa, se obtem uma espécie de estatueta.
\section{Fotosfera}
\begin{itemize}
\item {Grp. gram.:f.}
\end{itemize}
\begin{itemize}
\item {Proveniência:(De \textunderscore foto...\textunderscore  + \textunderscore esfera\textunderscore )}
\end{itemize}
Atmosfera luminosa do Sol.
\section{Fototaxia}
\begin{itemize}
\item {fónica:csi}
\end{itemize}
\begin{itemize}
\item {Grp. gram.:f.}
\end{itemize}
\begin{itemize}
\item {Proveniência:(Do gr. \textunderscore phos\textunderscore , \textunderscore photos\textunderscore  + \textunderscore taxis\textunderscore )}
\end{itemize}
Propriedade, que o protoplasma tem, de reagir á luz.
\section{Fototelegrafia}
\begin{itemize}
\item {Grp. gram.:f.}
\end{itemize}
\begin{itemize}
\item {Proveniência:(De \textunderscore foto\textunderscore  + \textunderscore telegrafia\textunderscore )}
\end{itemize}
Reprodução de uma imagem a distância, por meio do fio eléctrico.
\section{Fototerapia}
\begin{itemize}
\item {Grp. gram.:f.}
\end{itemize}
\begin{itemize}
\item {Proveniência:(De \textunderscore foto...\textunderscore  + \textunderscore terapia\textunderscore )}
\end{itemize}
Tratamento médico pela acção da luz.
\section{Fototerápico}
\begin{itemize}
\item {Grp. gram.:adj.}
\end{itemize}
Relativo á fototerapia.
\section{Fototipar}
\begin{itemize}
\item {Grp. gram.:v. t.}
\end{itemize}
O mesmo que \textunderscore fototipiar\textunderscore .
\section{Fototipia}
\begin{itemize}
\item {Grp. gram.:f.}
\end{itemize}
O mesmo que \textunderscore fototipografia\textunderscore .
\section{Fototipiar}
\begin{itemize}
\item {Grp. gram.:v. t.}
\end{itemize}
\begin{itemize}
\item {Utilização:Neol.}
\end{itemize}
\begin{itemize}
\item {Proveniência:(De \textunderscore fototipia\textunderscore )}
\end{itemize}
Reproduzir (desenho, figura ou paisagem), pelo processo fototipográfico. Cf. Alv. Mendes, \textunderscore Herculano\textunderscore , 44.
\section{Fototipografia}
\begin{itemize}
\item {Grp. gram.:f.}
\end{itemize}
\begin{itemize}
\item {Proveniência:(De \textunderscore foto...\textunderscore  + \textunderscore tipografia\textunderscore )}
\end{itemize}
Processo, com que se reproduzem pela fotografia trabalhos tipográficos.
\section{Fototipográfico}
\begin{itemize}
\item {Grp. gram.:adj.}
\end{itemize}
Relativo á fototipografia.
\section{Fototipogravura}
\begin{itemize}
\item {Grp. gram.:f.}
\end{itemize}
Processo fotográfico, próprio para tiragens tipográficas, dando-se meias tintas.
\section{Fototopografia}
\begin{itemize}
\item {Grp. gram.:f.}
\end{itemize}
\begin{itemize}
\item {Proveniência:(De \textunderscore foto...\textunderscore  + \textunderscore topografia\textunderscore )}
\end{itemize}
Arte de fotografar lugares ou paisagens.
\section{Fotozincografia}
\begin{itemize}
\item {Grp. gram.:f.}
\end{itemize}
\begin{itemize}
\item {Proveniência:(De \textunderscore foto...\textunderscore  + \textunderscore zincografia\textunderscore )}
\end{itemize}
Processo de heliogravura, sôbre zinco.
\section{Fotozincográfico}
\begin{itemize}
\item {Grp. gram.:adj.}
\end{itemize}
Relativo á fotozincografia.
\section{Fragma}
\begin{itemize}
\item {Grp. gram.:m.}
\end{itemize}
\begin{itemize}
\item {Utilização:Bot.}
\end{itemize}
\begin{itemize}
\item {Proveniência:(Gr. \textunderscore phragma\textunderscore )}
\end{itemize}
Nome, dado por Linck aos septos transversaes dos frutos.
\section{Frase}
\begin{itemize}
\item {Grp. gram.:f.}
\end{itemize}
\begin{itemize}
\item {Proveniência:(Lat. \textunderscore phrasis\textunderscore )}
\end{itemize}
Reunião de palavras, que formam sentido completo.
Locução; expressão.
Conjunto de sons musicaes, com uma pausa depois do último.
\section{Fraseado}
\begin{itemize}
\item {Grp. gram.:adj.}
\end{itemize}
\begin{itemize}
\item {Grp. gram.:M.}
\end{itemize}
\begin{itemize}
\item {Proveniência:(De \textunderscore frasear\textunderscore )}
\end{itemize}
Que está disposto em frases.
Modo de dizer ou de escever.
Conjunto de palavras.
\section{Fraseador}
\begin{itemize}
\item {Grp. gram.:m.  e  adj.}
\end{itemize}
O que fraseia.
\section{Frasear}
\begin{itemize}
\item {Grp. gram.:v. i.}
\end{itemize}
Fazer frases.
\section{Fraseologia}
\begin{itemize}
\item {Grp. gram.:f}
\end{itemize}
\begin{itemize}
\item {Proveniência:(Do gr. \textunderscore phrasis\textunderscore  + \textunderscore logos\textunderscore )}
\end{itemize}
Parte da Gramática, em que se estuda a construção da frase.
Construção de frase.
\section{Fraseologicamente}
\begin{itemize}
\item {Grp. gram.:adv.}
\end{itemize}
\begin{itemize}
\item {Proveniência:(De \textunderscore fraseológico\textunderscore )}
\end{itemize}
Segundo as regras da fraseologia.
\section{Fraseológico}
\begin{itemize}
\item {Grp. gram.:adj.}
\end{itemize}
Relativo á fraseologia.
\section{Frásis}
\begin{itemize}
\item {Grp. gram.:m.}
\end{itemize}
\begin{itemize}
\item {Utilização:Ant.}
\end{itemize}
O mesmo que \textunderscore frase\textunderscore ; discurso. Cf. \textunderscore Eufrosina\textunderscore , 189.
\section{Fratria}
\begin{itemize}
\item {Grp. gram.:f.}
\end{itemize}
\begin{itemize}
\item {Proveniência:(Gr. \textunderscore phratria\textunderscore )}
\end{itemize}
Cada uma das três divisões de cada tríbo ateniense.
\section{Frenalgia}
\begin{itemize}
\item {Grp. gram.:f.}
\end{itemize}
\begin{itemize}
\item {Utilização:Med.}
\end{itemize}
\begin{itemize}
\item {Proveniência:(Do gr. \textunderscore phren\textunderscore  + \textunderscore algos\textunderscore )}
\end{itemize}
Dôr reumática na cabeça.
\section{Frênico}
\begin{itemize}
\item {Grp. gram.:adj.}
\end{itemize}
\begin{itemize}
\item {Proveniência:(Do gr. \textunderscore phren\textunderscore )}
\end{itemize}
Relativo ao diafragma.
\section{Frenite}
\begin{itemize}
\item {Grp. gram.:f.}
\end{itemize}
\begin{itemize}
\item {Proveniência:(Do gr. \textunderscore phren\textunderscore )}
\end{itemize}
Inflamação do diafragma.
\section{Frenogástrico}
\begin{itemize}
\item {Grp. gram.:adj.}
\end{itemize}
\begin{itemize}
\item {Utilização:Anat.}
\end{itemize}
\begin{itemize}
\item {Proveniência:(Do gr. \textunderscore phren\textunderscore  + \textunderscore gaster\textunderscore )}
\end{itemize}
Relativo ao estômago e ao diafragma.
\section{Frenoglotismo}
\begin{itemize}
\item {Grp. gram.:m.}
\end{itemize}
\begin{itemize}
\item {Utilização:Med.}
\end{itemize}
\begin{itemize}
\item {Proveniência:(Do gr. \textunderscore phren\textunderscore  + \textunderscore glotta\textunderscore )}
\end{itemize}
Espasmo da glote e do diafragma.
\section{Frenologia}
\begin{itemize}
\item {Grp. gram.:f.}
\end{itemize}
Sistema fisiológico, que considera a conformação e as protuberâncias do cérebro como indicativas das diversas faculdades ou disposições innatas do indivíduo.
(Cp. \textunderscore frenólogo\textunderscore )
\section{Frenologicamente}
\begin{itemize}
\item {Grp. gram.:adv.}
\end{itemize}
De modo frenológico.
Segundo a frenologia.
\section{Frenológico}
\begin{itemize}
\item {Grp. gram.:adj.}
\end{itemize}
Relativo á frenologia.
\section{Frenologismo}
\begin{itemize}
\item {Grp. gram.:m.}
\end{itemize}
\begin{itemize}
\item {Proveniência:(De \textunderscore frenologia\textunderscore )}
\end{itemize}
Teoria dos frenólogos; frenologia. Cf. Ed. Burnay, \textunderscore Craniologia\textunderscore , 106.
\section{Frenologista}
\begin{itemize}
\item {Grp. gram.:m.  e  f.}
\end{itemize}
Pessôa, que trata de frenologia; pessôa, partidária da frenologia.
\section{Frenólogo}
\begin{itemize}
\item {Grp. gram.:m.}
\end{itemize}
\begin{itemize}
\item {Proveniência:(Do gr. \textunderscore phren\textunderscore , intelligência + \textunderscore logos\textunderscore , tratado)}
\end{itemize}
Aquele que é versado em frenologia.
\section{Frenopata}
\begin{itemize}
\item {Grp. gram.:m.}
\end{itemize}
\begin{itemize}
\item {Proveniência:(Do gr. \textunderscore phren\textunderscore , \textunderscore phenos\textunderscore  + \textunderscore pathos\textunderscore )}
\end{itemize}
O que padece frenopatia.
\section{Frenopatia}
\begin{itemize}
\item {Grp. gram.:f.}
\end{itemize}
Doença mental.
(Cp. \textunderscore frenopata\textunderscore )
\section{Frenopático}
\begin{itemize}
\item {Grp. gram.:adj.}
\end{itemize}
Relativo á frenopatia.
\section{Frigânio}
\begin{itemize}
\item {Grp. gram.:f.}
\end{itemize}
\begin{itemize}
\item {Proveniência:(Lat. \textunderscore phryganius\textunderscore )}
\end{itemize}
Gênero de insectos hemípteros, cujas espécies são quási todas europeias.
\section{Frígio}
\begin{itemize}
\item {Grp. gram.:adj.}
\end{itemize}
\begin{itemize}
\item {Grp. gram.:M.}
\end{itemize}
\begin{itemize}
\item {Proveniência:(Lat. \textunderscore phrygius\textunderscore )}
\end{itemize}
Relativo á Frigia ou aos seus habitantes.
Diz-se de um barrete encarnado, adoptado em França, no tempo da primeira República e semelhante ao que usavam os Frígios.
Habitante da Frígia.
Um dos idiomas mais antigos do Oriente.
\section{Fronema}
\begin{itemize}
\item {Grp. gram.:m.}
\end{itemize}
\begin{itemize}
\item {Utilização:Philos.}
\end{itemize}
\begin{itemize}
\item {Proveniência:(Gr. \textunderscore phronema\textunderscore )}
\end{itemize}
Foco do pensamento, no cérebro, onde se executa o trabalho da razão pura e que é distinto dos focos sensórios.
\section{Fronetas}
\begin{itemize}
\item {Grp. gram.:m. pl.}
\end{itemize}
\begin{itemize}
\item {Utilização:Philos.}
\end{itemize}
Centros de associação, para a formação do pensamento.
\section{Ftalâmico}
\begin{itemize}
\item {Grp. gram.:adj.}
\end{itemize}
Diz-se de um ácido, resultante da dissolução de ácido ftálico em o amoníaco.
\section{Ftalato}
\begin{itemize}
\item {Grp. gram.:m.}
\end{itemize}
\begin{itemize}
\item {Utilização:Chím.}
\end{itemize}
Sal, formado pelo ácido ftálico com uma base.
\section{Ftálico}
\begin{itemize}
\item {Grp. gram.:adj.}
\end{itemize}
Diz-se de um ácido, produzido pela acção do ácido azótico sôbre o bicloreto de naftalina.
\section{Ftanite}
\begin{itemize}
\item {Grp. gram.:f.}
\end{itemize}
O mesmo que \textunderscore ftanito\textunderscore .
\section{Ftanito}
\begin{itemize}
\item {Grp. gram.:m.}
\end{itemize}
\begin{itemize}
\item {Utilização:Miner.}
\end{itemize}
\begin{itemize}
\item {Proveniência:(Do gr. \textunderscore phthanein\textunderscore )}
\end{itemize}
Silex preto.
\section{Ftiríase}
\begin{itemize}
\item {Grp. gram.:f.}
\end{itemize}
\begin{itemize}
\item {Utilização:Med.}
\end{itemize}
\begin{itemize}
\item {Utilização:Bot.}
\end{itemize}
\begin{itemize}
\item {Proveniência:(Lat. \textunderscore phthiriasis\textunderscore )}
\end{itemize}
Doença, que consiste em uma excessiva multiplicação de piolhos.
Doença de vegetaes, em que êles se cobrem de pequeníssimos parasitos.
\section{Ftisiógeno}
\begin{itemize}
\item {Grp. gram.:adj.}
\end{itemize}
\begin{itemize}
\item {Proveniência:(Do gr. \textunderscore phthisis\textunderscore  + \textunderscore genes\textunderscore )}
\end{itemize}
Que produz tísica.
\section{Ftisiologia}
\begin{itemize}
\item {Grp. gram.:f.}
\end{itemize}
\begin{itemize}
\item {Proveniência:(Do gr. \textunderscore phthisis\textunderscore  + \textunderscore logos\textunderscore )}
\end{itemize}
Tratado médico á cêrca da tísica.
\section{Ftisiologista}
\begin{itemize}
\item {Grp. gram.:m.}
\end{itemize}
Aquele que é perito em ftisiologia.
\section{Ftisuria}
\begin{itemize}
\item {Grp. gram.:f.}
\end{itemize}
\begin{itemize}
\item {Utilização:Med.}
\end{itemize}
\begin{itemize}
\item {Proveniência:(Do gr. \textunderscore phthisis\textunderscore  + \textunderscore ouron\textunderscore )}
\end{itemize}
Consumpção física, produzida por uma excessiva secreção de urina, especialmente de urina açucarada.
\section{Ftórico}
\begin{itemize}
\item {Grp. gram.:adj.}
\end{itemize}
Relativo ao ftório.
\section{Ftório}
\begin{itemize}
\item {Grp. gram.:m.}
\end{itemize}
\begin{itemize}
\item {Utilização:Chím.}
\end{itemize}
\begin{itemize}
\item {Proveniência:(Lat. \textunderscore phthorius\textunderscore )}
\end{itemize}
Nome, dado por Ampére ao fluor, porque êste corrói os vasos em que se contém.
\section{Filogenia}
\begin{itemize}
\item {Grp. gram.:f.}
\end{itemize}
\begin{itemize}
\item {Utilização:Biol.}
\end{itemize}
\begin{itemize}
\item {Proveniência:(Do gr. \textunderscore phule\textunderscore , espécie + \textunderscore genos\textunderscore , geração)}
\end{itemize}
Sucessão genética das espécies orgânicas.
\section{Filogênico}
\begin{itemize}
\item {Grp. gram.:adj.}
\end{itemize}
Relativo á filogenia.
\section{Filossomo}
\begin{itemize}
\item {Grp. gram.:m.}
\end{itemize}
\begin{itemize}
\item {Proveniência:(Do gr. \textunderscore phullon\textunderscore  + \textunderscore soma\textunderscore )}
\end{itemize}
Larva da lagosta.
\section{Filotaxia}
\begin{itemize}
\item {fónica:csi}
\end{itemize}
\begin{itemize}
\item {Grp. gram.:f.}
\end{itemize}
\begin{itemize}
\item {Utilização:Bot.}
\end{itemize}
\begin{itemize}
\item {Proveniência:(Do gr. \textunderscore phullon\textunderscore  + \textunderscore taxis\textunderscore )}
\end{itemize}
Estudo das leis, que presidem á disposição das fôlhas na haste.
\section{Filoxantina}
\begin{itemize}
\item {Grp. gram.:f.}
\end{itemize}
\begin{itemize}
\item {Utilização:Chím.}
\end{itemize}
\begin{itemize}
\item {Proveniência:(Do gr. \textunderscore phullon\textunderscore  + \textunderscore xanthos\textunderscore )}
\end{itemize}
Princípio còrante amarelo, que existe na clorofila.
\section{Filoxera}
\begin{itemize}
\item {fónica:cse}
\end{itemize}
\begin{itemize}
\item {Grp. gram.:f.}
\end{itemize}
\begin{itemize}
\item {Proveniência:(Do gr. \textunderscore phullon\textunderscore  + \textunderscore xeros\textunderscore )}
\end{itemize}
Gênero de insectos hemípteros.
Doença das videiras, causada por um insecto dêsse gênero, \textunderscore phylloxera vastatrix\textunderscore .
\section{Filoxerado}
\begin{itemize}
\item {fónica:cse}
\end{itemize}
\begin{itemize}
\item {Grp. gram.:adj.}
\end{itemize}
Atacado de filoxera.
\section{Filoxericida}
\begin{itemize}
\item {Grp. gram.:adj.}
\end{itemize}
\begin{itemize}
\item {Proveniência:(De \textunderscore phylloxera\textunderscore  + lat. \textunderscore caedere\textunderscore )}
\end{itemize}
Que destrói a filoxera; que se applica contra a filoxera.
\section{Filoxérico}
\begin{itemize}
\item {fónica:csé}
\end{itemize}
\begin{itemize}
\item {Grp. gram.:adj.}
\end{itemize}
Relativo á filoxera.
\section{Fílula}
\begin{itemize}
\item {Grp. gram.:f.}
\end{itemize}
\begin{itemize}
\item {Utilização:Bot.}
\end{itemize}
\begin{itemize}
\item {Proveniência:(Do gr. \textunderscore phullon\textunderscore )}
\end{itemize}
Cicatriz, que a quéda da fôlha deixa no ramo.
\section{Fima}
\begin{itemize}
\item {Grp. gram.:m.}
\end{itemize}
\begin{itemize}
\item {Proveniência:(Lat. \textunderscore phyma\textunderscore )}
\end{itemize}
Tumor inflamatório, que se eleva sôbre a pele.
\section{Fimatina}
\begin{itemize}
\item {Grp. gram.:f.}
\end{itemize}
\begin{itemize}
\item {Utilização:Chím.}
\end{itemize}
\begin{itemize}
\item {Proveniência:(Do gr. \textunderscore phuma\textunderscore )}
\end{itemize}
Substância orgânica, própria dos tubérculos.
\section{Fimatóide}
\begin{itemize}
\item {Grp. gram.:adj.}
\end{itemize}
\begin{itemize}
\item {Utilização:Anat.}
\end{itemize}
\begin{itemize}
\item {Proveniência:(Do gr. \textunderscore phuma\textunderscore  + \textunderscore eidos\textunderscore )}
\end{itemize}
Diz-se do tecido mórbido de côr amarelada.
\section{Fimatose}
\begin{itemize}
\item {Grp. gram.:f.}
\end{itemize}
\begin{itemize}
\item {Utilização:Med.}
\end{itemize}
\begin{itemize}
\item {Proveniência:(Do gr. \textunderscore phuma\textunderscore )}
\end{itemize}
Doença tuberculosa.
\section{Fimatoso}
\begin{itemize}
\item {Grp. gram.:m.  e  adj.}
\end{itemize}
\begin{itemize}
\item {Proveniência:(Do gr. \textunderscore phuma\textunderscore )}
\end{itemize}
O mesmo que \textunderscore tuberculoso\textunderscore .
\section{Fímico}
\begin{itemize}
\item {Grp. gram.:adj.}
\end{itemize}
\begin{itemize}
\item {Proveniência:(Do gr. \textunderscore phuma\textunderscore )}
\end{itemize}
Relativo a tubérculos ou á tuberculose: \textunderscore o mal fímico\textunderscore . Cf. \textunderscore Jorn.-do-Comm.\textunderscore , do Rio, de 4-XI-902.
\section{Fisália}
\begin{itemize}
\item {Grp. gram.:f.}
\end{itemize}
\begin{itemize}
\item {Proveniência:(Do gr. \textunderscore phusalis\textunderscore )}
\end{itemize}
Animal marinho, que tem o aspecto de uma simples vesícula membranosa, donde pendem longos tentáculos, providos de uma espécie de ventosas, com que o animal fixa os peixes para os devorar.
\section{Fisálide}
\begin{itemize}
\item {Grp. gram.:f.}
\end{itemize}
\begin{itemize}
\item {Proveniência:(Do gr. \textunderscore phusalis\textunderscore )}
\end{itemize}
Gênero de plantas solâneas.
\section{Fisalina}
\begin{itemize}
\item {Grp. gram.:f.}
\end{itemize}
\begin{itemize}
\item {Proveniência:(De \textunderscore fisálide\textunderscore )}
\end{itemize}
Substância amarga, extraida de uma espécie de fisálide e considerada como succedâneo da quina.
\section{Fisális}
\begin{itemize}
\item {Grp. gram.:f.}
\end{itemize}
O mesmo que \textunderscore fisálide\textunderscore .
\section{Fisalite}
\begin{itemize}
\item {Grp. gram.:f.}
\end{itemize}
\begin{itemize}
\item {Proveniência:(Do gr. \textunderscore phusalis\textunderscore )}
\end{itemize}
Variedade de topázio.
\section{Fisalito}
\begin{itemize}
\item {Grp. gram.:m.}
\end{itemize}
O mesmo ou melhor que \textunderscore fisalite\textunderscore .
\section{Fisarmónica}
\begin{itemize}
\item {Grp. gram.:f.}
\end{itemize}
Antigo instrumento, em que se utilizavam lâminas metállicas, postas em vibração por um fole.
\section{Fisconia}
\begin{itemize}
\item {Grp. gram.:f.}
\end{itemize}
\begin{itemize}
\item {Utilização:Med.}
\end{itemize}
\begin{itemize}
\item {Proveniência:(Do gr. \textunderscore phuskon\textunderscore )}
\end{itemize}
Tumefacção de uma parte do abdome, sem timpanite nem fluctuação.
\section{Fisema}
\begin{itemize}
\item {Grp. gram.:m.}
\end{itemize}
\begin{itemize}
\item {Utilização:Bot.}
\end{itemize}
\begin{itemize}
\item {Proveniência:(Lat. \textunderscore physema\textunderscore )}
\end{itemize}
Parte das algas, também chamada fôlha.
\section{Fisetérios}
\begin{itemize}
\item {Grp. gram.:m. pl.}
\end{itemize}
Gênero de baleotes.
\section{Fisianto}
\begin{itemize}
\item {Grp. gram.:m.}
\end{itemize}
\begin{itemize}
\item {Proveniência:(Do gr. \textunderscore phusis\textunderscore  + \textunderscore anthos\textunderscore )}
\end{itemize}
Espécie de trepadeira vivaz.
\section{Física}
\begin{itemize}
\item {Grp. gram.:f.}
\end{itemize}
\begin{itemize}
\item {Utilização:Ant.}
\end{itemize}
\begin{itemize}
\item {Utilização:Des.}
\end{itemize}
\begin{itemize}
\item {Proveniência:(Lat. \textunderscore Physica\textunderscore )}
\end{itemize}
Ciência do movimento e das acções recíprocas dos corpos, considerando êstes sob o ponto de vista da sua composição e decomposição, como na Química.
Conhecimento de toda a natureza material.
Medicina.
\section{Fisicamente}
\begin{itemize}
\item {Grp. gram.:adv.}
\end{itemize}
De modo fsico.
Segundo a Física.
De facto, realmente; materialmente: \textunderscore isso é fisicamente impossível\textunderscore .
\section{Fisicismo}
\begin{itemize}
\item {Grp. gram.:m.}
\end{itemize}
\begin{itemize}
\item {Proveniência:(De \textunderscore físico\textunderscore )}
\end{itemize}
Sistema dos que explicam o universo pela relação das fôrças físicas.
\section{Fisicista}
\begin{itemize}
\item {Grp. gram.:m.}
\end{itemize}
\begin{itemize}
\item {Utilização:Neol.}
\end{itemize}
Aquele que se dedica aos problemas da Física.
Partidário do fisicismo.
\section{Físico}
\begin{itemize}
\item {Grp. gram.:adj.}
\end{itemize}
\begin{itemize}
\item {Grp. gram.:M.}
\end{itemize}
\begin{itemize}
\item {Utilização:Des.}
\end{itemize}
\begin{itemize}
\item {Proveniência:(Lat. \textunderscore physicus\textunderscore )}
\end{itemize}
Relativo ás condições e leis da natureza.
Material.
Corpóreo.
Natural.
Conjunto das qualidades exteriores do homem.
Aspecto; configuração.
Conjunto das funcções fisiológicas.
Aquele que estuda Física ou é perito nela.
Médico.
\section{Fisiocracia}
\begin{itemize}
\item {Grp. gram.:f.}
\end{itemize}
\begin{itemize}
\item {Proveniência:(De \textunderscore fisiócrata\textunderscore )}
\end{itemize}
Doutrina dos fisiocratas.
\section{Fisiocrata}
\begin{itemize}
\item {Grp. gram.:m.}
\end{itemize}
\begin{itemize}
\item {Proveniência:(Do gr. \textunderscore phusis\textunderscore  + \textunderscore kratos\textunderscore )}
\end{itemize}
Economista, que considera as fôrças da natureza, e especialmente as da terra, como fonte principal da riqueza pública.
\section{Fisiócrata}
\begin{itemize}
\item {Grp. gram.:m.}
\end{itemize}
\begin{itemize}
\item {Proveniência:(Do gr. \textunderscore phusis\textunderscore  + \textunderscore kratos\textunderscore )}
\end{itemize}
Economista, que considera as fôrças da natureza, e especialmente as da terra, como fonte principal da riqueza pública.
\section{Fisiocrático}
\begin{itemize}
\item {Grp. gram.:adj.}
\end{itemize}
Relativo á fisiocracia.
\section{Fisiogenia}
\begin{itemize}
\item {Grp. gram.:f.}
\end{itemize}
\begin{itemize}
\item {Proveniência:(Do gr. \textunderscore phusis\textunderscore  + \textunderscore genos\textunderscore )}
\end{itemize}
Desenvolvimento natural do organismo.
\section{Fisiognomonia}
\begin{itemize}
\item {Grp. gram.:f.}
\end{itemize}
\begin{itemize}
\item {Proveniência:(Do gr. \textunderscore phusis\textunderscore  + \textunderscore gnomon\textunderscore )}
\end{itemize}
Suposta ciência, que determina as qualidades e inclinações do homem pelas feições do rosto.
\section{Fisiognomónico}
\begin{itemize}
\item {Grp. gram.:adj.}
\end{itemize}
Relativo á fisiognomonia.
\section{Fisiognomonista}
\begin{itemize}
\item {Grp. gram.:m.}
\end{itemize}
Aquele que se dedica a estudos fisiognomónicos, ou que escreve á cêrca da fisiognomonia.
\section{Fisiografia}
\begin{itemize}
\item {Grp. gram.:f.}
\end{itemize}
\begin{itemize}
\item {Proveniência:(Do gr. \textunderscore phusis\textunderscore  + \textunderscore graphein\textunderscore )}
\end{itemize}
Descripção da natureza ou dos seus productos.
\section{Fisiográfico}
\begin{itemize}
\item {Grp. gram.:adj.}
\end{itemize}
Relativo á fisiografia.
\section{Fisiologia}
\begin{itemize}
\item {Grp. gram.:f.}
\end{itemize}
\begin{itemize}
\item {Proveniência:(Lat. \textunderscore physiologia\textunderscore )}
\end{itemize}
Ciência, que trata das funções dos órgãos nos seres vivos, vegetaes e animaes.
Tratado de Fisiologia.
\section{Fisiologicamente}
\begin{itemize}
\item {Grp. gram.:adv.}
\end{itemize}
De modo fisiológico; segundo a Fisiologia.
\section{Fisiológico}
\begin{itemize}
\item {Grp. gram.:adj.}
\end{itemize}
\begin{itemize}
\item {Proveniência:(Lat. \textunderscore physiologicus\textunderscore )}
\end{itemize}
Relativo á Fisiologia.
\section{Fisiologista}
\begin{itemize}
\item {Grp. gram.:m.  e  f.}
\end{itemize}
Pessôa, que trata de Fisiologia.
\section{Fisiólogo}
\begin{itemize}
\item {Grp. gram.:m.}
\end{itemize}
\begin{itemize}
\item {Proveniência:(Lat. \textunderscore physiologus\textunderscore )}
\end{itemize}
Aquele que é versado em Fisiologia.
\section{Fisionomia}
\begin{itemize}
\item {Grp. gram.:f.}
\end{itemize}
Conjunto das feições do rosto.
Aspecto, ar.
Cara, rosto.
Conjunto de caracteres especiaes.
(Contr. de \textunderscore physiognomonia\textunderscore )
\section{Fisionómico}
\begin{itemize}
\item {Grp. gram.:adj.}
\end{itemize}
Relativo á fisionomia.
\section{Fisionomismo}
\begin{itemize}
\item {Grp. gram.:m.}
\end{itemize}
\begin{itemize}
\item {Proveniência:(De \textunderscore fisionomia\textunderscore )}
\end{itemize}
Teoria ou sistema dos fisionomistas.
\section{Fisionomista}
\begin{itemize}
\item {Grp. gram.:m.  e  f.}
\end{itemize}
Pessôa, que conhece a índole de outra pela observação da sua fisionomia.
\section{Fisiopata}
\begin{itemize}
\item {Grp. gram.:m.}
\end{itemize}
Aquele que exerce a fisiopatia.
\section{Fisiopatia}
\begin{itemize}
\item {Grp. gram.:f.}
\end{itemize}
\begin{itemize}
\item {Proveniência:(Do gr. \textunderscore phusis\textunderscore  + \textunderscore pathos\textunderscore )}
\end{itemize}
Sistema terapêutico, que emprega exclusivamente os recursos da natureza.
\section{Fisiopático}
\begin{itemize}
\item {Grp. gram.:adj.}
\end{itemize}
Relativo á fisiopatia.
\section{Fisioterapia}
\begin{itemize}
\item {Grp. gram.:f.}
\end{itemize}
\begin{itemize}
\item {Proveniência:(Do gr. \textunderscore phusis\textunderscore  + \textunderscore therapeia\textunderscore )}
\end{itemize}
Emprêgo dos agentes naturaes, (água, ar, etc.), como meios terapêuticos.
\section{Fisocarpo}
\begin{itemize}
\item {Grp. gram.:adj.}
\end{itemize}
\begin{itemize}
\item {Utilização:Bot.}
\end{itemize}
\begin{itemize}
\item {Proveniência:(Do gr. \textunderscore phusa\textunderscore  + \textunderscore karpos\textunderscore )}
\end{itemize}
Que tem frutos vesiculosos.
\section{Fisocele}
\begin{itemize}
\item {Grp. gram.:f.}
\end{itemize}
\begin{itemize}
\item {Proveniência:(Do gr. \textunderscore phusa\textunderscore  + \textunderscore kele\textunderscore )}
\end{itemize}
Hérnia intestinal, distendida pelos gases até o escroto.
\section{Fisóforos}
\begin{itemize}
\item {Grp. gram.:m. pl.}
\end{itemize}
\begin{itemize}
\item {Utilização:Zool.}
\end{itemize}
\begin{itemize}
\item {Proveniência:(Do gr. \textunderscore phusa\textunderscore  + \textunderscore phoros\textunderscore )}
\end{itemize}
Celenterados, com bôlsas cheias de ar, que lhes permitem fluctuar na água.
\section{Fisóide}
\begin{itemize}
\item {Grp. gram.:adj.}
\end{itemize}
\begin{itemize}
\item {Proveniência:(Do gr. \textunderscore phusa\textunderscore  + \textunderscore eidos\textunderscore )}
\end{itemize}
Que tem fórma de bexiga.
\section{Fisometria}
\begin{itemize}
\item {Grp. gram.:f.}
\end{itemize}
\begin{itemize}
\item {Utilização:Med.}
\end{itemize}
\begin{itemize}
\item {Proveniência:(Do gr. \textunderscore phusa\textunderscore  + \textunderscore metra\textunderscore )}
\end{itemize}
Distensão do útero, causada por gazes.
\section{Fisospermo}
\begin{itemize}
\item {Grp. gram.:m.}
\end{itemize}
\begin{itemize}
\item {Proveniência:(Do gr. \textunderscore phusa\textunderscore  + \textunderscore sperma\textunderscore )}
\end{itemize}
Gênero de plantas umbelíferas.
\section{Fisostigma}
\begin{itemize}
\item {Grp. gram.:f.}
\end{itemize}
Espécie de fava medicinal.
\section{Fisostigmina}
\begin{itemize}
\item {Grp. gram.:f.}
\end{itemize}
Alcalóide da fisostigma, usado em terapêutica ocular e mais conhecido por \textunderscore iserina\textunderscore .
\section{Fisotórax}
\begin{itemize}
\item {Grp. gram.:m.}
\end{itemize}
\begin{itemize}
\item {Utilização:Med.}
\end{itemize}
\begin{itemize}
\item {Proveniência:(Do gr. \textunderscore phusa\textunderscore  + \textunderscore thorax\textunderscore )}
\end{itemize}
Acumulações de gases na cavidade da pleura.
\section{Fitina}
\begin{itemize}
\item {Grp. gram.:f.}
\end{itemize}
\begin{itemize}
\item {Proveniência:(Do gr. \textunderscore phuton\textunderscore )}
\end{itemize}
Combinação fosfórica, extraida de sementes e que se mistura com o alimento das crianças.
\section{Fito...}
\begin{itemize}
\item {Grp. gram.:pref.}
\end{itemize}
\begin{itemize}
\item {Proveniência:(Do gr. \textunderscore phuton\textunderscore )}
\end{itemize}
(designativo de \textunderscore vegetal\textunderscore )
\section{Fitofagia}
\begin{itemize}
\item {Grp. gram.:f.}
\end{itemize}
Qualidade de fitófago.
\section{Fitófago}
\begin{itemize}
\item {Grp. gram.:adj.}
\end{itemize}
\begin{itemize}
\item {Proveniência:(Do gr. \textunderscore phuton\textunderscore  + \textunderscore phagein\textunderscore )}
\end{itemize}
Que se alimenta de vegetaes.
\section{Fitogêneo}
\begin{itemize}
\item {Grp. gram.:adj.}
\end{itemize}
\begin{itemize}
\item {Proveniência:(Do gr. \textunderscore phuton\textunderscore  + \textunderscore genes\textunderscore )}
\end{itemize}
Produzido por vegetaes.
\section{Fitogênese}
\begin{itemize}
\item {Grp. gram.:f.}
\end{itemize}
O mesmo que \textunderscore fitogenia\textunderscore .
\section{Fitogenia}
\begin{itemize}
\item {Grp. gram.:f.}
\end{itemize}
Designação científica da vegetação ou da produção vegetal.
(Cp. \textunderscore fitogêneo\textunderscore )
\section{Fitogênico}
\begin{itemize}
\item {Grp. gram.:adj.}
\end{itemize}
Relativo á fitogenia.
\section{Fitogeografia}
\begin{itemize}
\item {Grp. gram.:f.}
\end{itemize}
\begin{itemize}
\item {Proveniência:(De \textunderscore fito...\textunderscore  + \textunderscore geografia\textunderscore )}
\end{itemize}
Descripção da distribuição de plantas no globo.
\section{Fitogeográfico}
\begin{itemize}
\item {Grp. gram.:adj.}
\end{itemize}
Relativo á fitogeografia.
\section{Fitognomia}
\begin{itemize}
\item {Grp. gram.:f.}
\end{itemize}
\begin{itemize}
\item {Utilização:Bot.}
\end{itemize}
\begin{itemize}
\item {Proveniência:(Do gr. \textunderscore phuton\textunderscore  + \textunderscore gnomon\textunderscore )}
\end{itemize}
Conhecimento das partes, que constituem os vegetaes.
\section{Fitognomónica}
\begin{itemize}
\item {Grp. gram.:f.}
\end{itemize}
\begin{itemize}
\item {Proveniência:(Do gr. \textunderscore phuton\textunderscore  + \textunderscore gnomon\textunderscore )}
\end{itemize}
Nome, dado por Porta ao sistema de determinar a aplicação medicinal das plantas pela sua conformação ou coloração.
\section{Fitografia}
\begin{itemize}
\item {Grp. gram.:f.}
\end{itemize}
\begin{itemize}
\item {Utilização:Bot.}
\end{itemize}
Descripção metódica e natural dos diferentes tipos vegetaes, sob o ponto de vista da sua classificação.
(Cp. \textunderscore fitógrafo\textunderscore )
\section{Fitográfico}
\begin{itemize}
\item {Grp. gram.:adj.}
\end{itemize}
Relativo á fitografia.
\section{Fitógrafo}
\begin{itemize}
\item {Grp. gram.:m.}
\end{itemize}
\begin{itemize}
\item {Proveniência:(Do gr. \textunderscore phuton\textunderscore  + \textunderscore graphein\textunderscore )}
\end{itemize}
Aquele que se dedica á fitografia.
\section{Fitóide}
\begin{itemize}
\item {Grp. gram.:adj.}
\end{itemize}
\begin{itemize}
\item {Proveniência:(Do gr. \textunderscore phuton\textunderscore  + \textunderscore eidos\textunderscore )}
\end{itemize}
Relativo ou semelhante a planta.
\section{Fitolaca}
\begin{itemize}
\item {Grp. gram.:f.}
\end{itemize}
\begin{itemize}
\item {Proveniência:(De \textunderscore phuton\textunderscore  gr. + \textunderscore laca\textunderscore )}
\end{itemize}
Gênero de plantas tinctórias das regiões quentes.
\section{Fitolaceáceas}
\begin{itemize}
\item {Grp. gram.:f. pl.}
\end{itemize}
O mesmo que \textunderscore fitoláceas\textunderscore .
\section{Fitoláceas}
\begin{itemize}
\item {Grp. gram.:f. pl.}
\end{itemize}
Tríbo de plantas, que tem por tipo a fitolaca.
(Fem. pl. de \textunderscore fitolaceo\textunderscore )
\section{Fitoláceo}
\begin{itemize}
\item {Grp. gram.:adj.}
\end{itemize}
Relativo ou semelhante á fitolaca.
\section{Fitólito}
\begin{itemize}
\item {Grp. gram.:m.}
\end{itemize}
\begin{itemize}
\item {Proveniência:(Do gr. \textunderscore phuton\textunderscore  + \textunderscore lithos\textunderscore )}
\end{itemize}
Vegetal fóssil.
Pedra, que apresenta o vestígio de uma planta.
Concreção pedregosa, que se encontra em algumas plantas, como nos bambus.
\section{Fitologia}
\begin{itemize}
\item {Grp. gram.:f.}
\end{itemize}
\begin{itemize}
\item {Proveniência:(Do gr. \textunderscore phuton\textunderscore  + \textunderscore logos\textunderscore )}
\end{itemize}
Tratado ou classificação das plantas; Botânica.
\section{Fitológico}
\begin{itemize}
\item {Grp. gram.:adj.}
\end{itemize}
Relativo á fitologia.
\section{Fitonícia}
\begin{itemize}
\item {Grp. gram.:f.}
\end{itemize}
Uma das supostas artes de adivinhar, usadas pelos antigos.
\section{Fitonimia}
\begin{itemize}
\item {Grp. gram.:f.}
\end{itemize}
Qualidade de fitónimo.
Nomenclatura vegetal.
\section{Fitónimo}
\begin{itemize}
\item {Grp. gram.:adj.}
\end{itemize}
\begin{itemize}
\item {Utilização:Neol.}
\end{itemize}
\begin{itemize}
\item {Proveniência:(Do gr. \textunderscore phuton\textunderscore  + \textunderscore onuma\textunderscore )}
\end{itemize}
Diz-se do indivíduo, cujo nome ou apelido é tirado de uma planta: \textunderscore Carvalho\textunderscore , \textunderscore Figueira\textunderscore , \textunderscore Oliveira\textunderscore , \textunderscore Pinheiro\textunderscore , etc.
\section{Fitonomia}
\begin{itemize}
\item {Grp. gram.:f.}
\end{itemize}
\begin{itemize}
\item {Proveniência:(Do gr. \textunderscore phuton\textunderscore  + \textunderscore nomos\textunderscore )}
\end{itemize}
Parte da Botânica, que trata das leis da vegetação.
\section{Fitonómico}
\begin{itemize}
\item {Grp. gram.:adj.}
\end{itemize}
Relativo a fitonomia.
\section{Fitonose}
\begin{itemize}
\item {Grp. gram.:f.}
\end{itemize}
\begin{itemize}
\item {Proveniência:(Do gr. \textunderscore phuton\textunderscore  + \textunderscore nosos\textunderscore )}
\end{itemize}
Nome genérico das doenças dos vegetaes.
\section{Fitoquímica}
\begin{itemize}
\item {Grp. gram.:f.}
\end{itemize}
\begin{itemize}
\item {Utilização:P. us.}
\end{itemize}
\begin{itemize}
\item {Proveniência:(De \textunderscore fito...\textunderscore  + \textunderscore Química\textunderscore )}
\end{itemize}
Química vegetal.
\section{Fitoquímico}
\begin{itemize}
\item {Grp. gram.:adj.}
\end{itemize}
Relativo á Fitoquímica.
\section{Fitotecnia}
\begin{itemize}
\item {Grp. gram.:f.}
\end{itemize}
\begin{itemize}
\item {Proveniência:(Do gr. \textunderscore phuton\textunderscore  + \textunderscore tekne\textunderscore )}
\end{itemize}
Parte da Botânica, que tem por objecto a classificação e nomenclatura das plantas, bem como a utilidade que delas se póde auferir.
\section{Fitotécnico}
\begin{itemize}
\item {Grp. gram.:adj.}
\end{itemize}
Relativo á fitotecnia.
\section{Fitoterosia}
\begin{itemize}
\item {Grp. gram.:f.}
\end{itemize}
Parte da Botânica, que trata das alterações mórbidas dos vegetaes.
Patologia vegetal.
\section{Fitotipólito}
\begin{itemize}
\item {Grp. gram.:m.}
\end{itemize}
\begin{itemize}
\item {Proveniência:(Do gr. \textunderscore phuton\textunderscore  + \textunderscore tupos\textunderscore  + \textunderscore lithos\textunderscore )}
\end{itemize}
Substância mineral, que contém o vestígio de um vegetal.
\section{Fitotomia}
\begin{itemize}
\item {Grp. gram.:f.}
\end{itemize}
\begin{itemize}
\item {Proveniência:(Do gr. \textunderscore phuton\textunderscore  + \textunderscore tome\textunderscore )}
\end{itemize}
Anatomia vegetal.
\section{Fitotómico}
\begin{itemize}
\item {Grp. gram.:adj.}
\end{itemize}
Relativo á fitotomia.
\section{Fitozoário}
\begin{itemize}
\item {Grp. gram.:adj.}
\end{itemize}
\begin{itemize}
\item {Grp. gram.:M. pl.}
\end{itemize}
\begin{itemize}
\item {Utilização:Zool.}
\end{itemize}
\begin{itemize}
\item {Proveniência:(Do gr. \textunderscore phuton\textunderscore  + \textunderscore zoon\textunderscore )}
\end{itemize}
Diz-se dos seres, que se suppõem intermediários ás plantas e aos animaes.
Seres fitozoários.
Animaes, que tem configuração radiada e que, em geral, formam colónias arborescentes.
\section{Faringismo}
\begin{itemize}
\item {Grp. gram.:m.}
\end{itemize}
\begin{itemize}
\item {Utilização:Med.}
\end{itemize}
Contracção espasmódica dos músculos da faringe.
\section{Faringoscopia}
\begin{itemize}
\item {Grp. gram.:f.}
\end{itemize}
\begin{itemize}
\item {Utilização:Med.}
\end{itemize}
Exame da cavidade faríngea, por meio do faringoscópio.
\section{Fosfatose}
\begin{itemize}
\item {Grp. gram.:f.}
\end{itemize}
Preparação alimentícia para o gado, na qual entram fosfatos.
\section{Fosforismo}
\begin{itemize}
\item {Grp. gram.:m.}
\end{itemize}
\begin{itemize}
\item {Utilização:Med.}
\end{itemize}
Entoxicação pelo fósforo.
\section{F}
\begin{itemize}
\item {fónica:efe}
\end{itemize}
\begin{itemize}
\item {Grp. gram.:m.}
\end{itemize}
\begin{itemize}
\item {Grp. gram.:Adj.}
\end{itemize}
Sexta letra do alphabeto português.
Quando maiúsculo e seguido de um ponto, é abrev. de \textunderscore Fulano\textunderscore  ou \textunderscore Fuão\textunderscore .
Us. na loc. fam. \textunderscore com todos os ff\textunderscore  e \textunderscore rr\textunderscore , que significa \textunderscore apuradamente\textunderscore , com \textunderscore exactidão\textunderscore , \textunderscore completamente\textunderscore .
Em Música, indica o tom de \textunderscore fá\textunderscore  e, collocado acima ou abaixo de uma nota, quer dizer \textunderscore forte\textunderscore  ou, se está duplicado, \textunderscore fortissimo\textunderscore .
Sexto grau da escala, na antiga notação gregoriana.
Diz-se daquillo que occupa o sexto lugar numa série indicada pelas letras do alphabeto, como em \textunderscore Livro A\textunderscore , \textunderscore Livro B\textunderscore , etc.
\section{Fá}
\begin{itemize}
\item {Grp. gram.:m.}
\end{itemize}
Quarta nota da moderna escala musical.
Sinal representativo desta nota.
(Da 1.^a sýllaba do lat. \textunderscore famuli\textunderscore , aproveitado por G. de Arrezo, entre algumas palavras de um hymno religioso, para a formação da antiga escala musical)
\section{Fabagela}
\begin{itemize}
\item {Grp. gram.:f.}
\end{itemize}
Planta medicinal, (\textunderscore zygophyllum fabago\textunderscore ).
\section{Fabagella}
\begin{itemize}
\item {Grp. gram.:f.}
\end{itemize}
Planta medicinal, (\textunderscore zygophyllum fabago\textunderscore ).
\section{Fabela}
\begin{itemize}
\item {Grp. gram.:f.}
\end{itemize}
\begin{itemize}
\item {Proveniência:(Lat. \textunderscore fabella\textunderscore )}
\end{itemize}
Pequena fábula.
\section{Fabella}
\begin{itemize}
\item {Grp. gram.:f.}
\end{itemize}
\begin{itemize}
\item {Proveniência:(Lat. \textunderscore fabella\textunderscore )}
\end{itemize}
Pequena fábula.
\section{Fabiana}
\begin{itemize}
\item {Grp. gram.:f.}
\end{itemize}
\begin{itemize}
\item {Proveniência:(De \textunderscore Fabiano\textunderscore , n. p.)}
\end{itemize}
Gênero de plantas solanáceas.
\section{Fàbordão}
\begin{itemize}
\item {Grp. gram.:m.}
\end{itemize}
\begin{itemize}
\item {Utilização:Fig.}
\end{itemize}
\begin{itemize}
\item {Proveniência:(Do fr. \textunderscore faux-bourdon\textunderscore )}
\end{itemize}
Antigo canto da Igreja, em contraponto simples de primeira espécie (nota contra nota), a três ou quatro vozes.
Música desentoada, sem pausas, a muitas vozes, e de harmonia syllábica.
Sensaboria.
Desentoação.
\section{Fábrica}
\begin{itemize}
\item {Grp. gram.:f.}
\end{itemize}
\begin{itemize}
\item {Utilização:Prov.}
\end{itemize}
\begin{itemize}
\item {Utilização:minh.}
\end{itemize}
\begin{itemize}
\item {Utilização:Fig.}
\end{itemize}
\begin{itemize}
\item {Proveniência:(Lat. \textunderscore fabrica\textunderscore )}
\end{itemize}
Acto ou effeito de fabricar.
Lugar ou estabelecimento em que se fabríca.
Pessoal desse estabelecimento.
Construcção de edifício.
Maquinismo engenhoso.
Rendimento, applicado ao culto religioso numa igreja.
Casa, onde diversas companhias guardam o velame e outros aprestos marítimos.
Aquillo que determina algum acontecimento; origem: \textunderscore aquelle pântano é uma fábrica de sezões\textunderscore .
\section{Fabricação}
\begin{itemize}
\item {Grp. gram.:f.}
\end{itemize}
\begin{itemize}
\item {Proveniência:(Lat. \textunderscore fabricatio\textunderscore )}
\end{itemize}
Acto, effeito, modo ou meio de fabricar.
\section{Fabricador}
\begin{itemize}
\item {Grp. gram.:adj.}
\end{itemize}
\begin{itemize}
\item {Grp. gram.:M.}
\end{itemize}
\begin{itemize}
\item {Proveniência:(Lat. \textunderscore fabricator\textunderscore )}
\end{itemize}
Que fabríca.
Aquelle que fabríca.
\section{Fabricando}
\begin{itemize}
\item {Grp. gram.:adj.}
\end{itemize}
\begin{itemize}
\item {Proveniência:(De \textunderscore fabricar\textunderscore )}
\end{itemize}
Que há de sêr fabricado. Cf. Filinto, IV, 281.
\section{Fabricante}
\begin{itemize}
\item {Grp. gram.:m.}
\end{itemize}
\begin{itemize}
\item {Proveniência:(Lat. \textunderscore fabricans\textunderscore )}
\end{itemize}
Aquelle que fabríca ou dirige a fabricação.
Dono de fábrica.
Aquelle que arranja, organiza ou inventa.
Operário de fábrica.
\section{Fabricar}
\begin{itemize}
\item {Grp. gram.:v. t.}
\end{itemize}
\begin{itemize}
\item {Utilização:Fig.}
\end{itemize}
\begin{itemize}
\item {Proveniência:(Lat. \textunderscore fabricari\textunderscore )}
\end{itemize}
Manufacturar: \textunderscore fabricar chapéus\textunderscore .
Construir.
Produzir mechanicamente.
Cultivar: \textunderscore fabricar herdades\textunderscore .
Fazer consertos a (um navio) num pôrto qualquer.
Inventar; engendrar: \textunderscore fabricar intrigas\textunderscore .
\section{Fabricário}
\begin{itemize}
\item {Grp. gram.:m.  e  adj.}
\end{itemize}
\begin{itemize}
\item {Proveniência:(Lat. \textunderscore fabricarius\textunderscore )}
\end{itemize}
O mesmo que \textunderscore fabriqueiro\textunderscore .
\section{Fabricável}
\begin{itemize}
\item {Grp. gram.:adj.}
\end{itemize}
Que se póde fabricar.
\section{Fabrícia}
\begin{itemize}
\item {Grp. gram.:f.}
\end{itemize}
\begin{itemize}
\item {Proveniência:(De \textunderscore Fabrício\textunderscore , n. p.)}
\end{itemize}
Gênero de plantas myrtáceas.
\section{Fabrico}
\begin{itemize}
\item {Grp. gram.:m.}
\end{itemize}
Acto ou arte de fabricar.
Producto de uma fábrica.
Consêrto, que a um navio se faz num pôrto qualquer.
\section{Fabril}
\begin{itemize}
\item {Grp. gram.:adj.}
\end{itemize}
\begin{itemize}
\item {Proveniência:(Lat. \textunderscore fabrilis\textunderscore )}
\end{itemize}
Relativo a fábrica ou a trabalho de fabricante: \textunderscore indústria fabril\textunderscore .
\section{Fabriqueiro}
\begin{itemize}
\item {Grp. gram.:m.  e  adj.}
\end{itemize}
\begin{itemize}
\item {Proveniência:(Do lat. \textunderscore fabricarius\textunderscore )}
\end{itemize}
Cobrador de rendimento de uma igreja.
Encarregado da guarda dos paramentos e alfaias de uma igreja.
\section{Fabro}
\begin{itemize}
\item {Grp. gram.:m.}
\end{itemize}
\begin{itemize}
\item {Utilização:Des.}
\end{itemize}
\begin{itemize}
\item {Proveniência:(Do lat. \textunderscore faber\textunderscore )}
\end{itemize}
Artífice.
\section{Fábula}
\begin{itemize}
\item {Grp. gram.:f.}
\end{itemize}
\begin{itemize}
\item {Utilização:Des.}
\end{itemize}
\begin{itemize}
\item {Proveniência:(Lat. \textunderscore fabula\textunderscore )}
\end{itemize}
Narração de coisas imaginárias.
Mythologia; narrativa mythológica.
Ficção.
Mentira: \textunderscore isso é fábula\textunderscore .
Composição literária, allegoria, que esconde uma verdade moral: \textunderscore as fábulas de Esopo\textunderscore .
Entrecho ou urdidura de romance, poêma ou drama.
Aquillo que se conta.
Coisa ou pessôa em que se fala.
\section{Fabulação}
\begin{itemize}
\item {Grp. gram.:f.}
\end{itemize}
\begin{itemize}
\item {Proveniência:(Lat. \textunderscore fabulatio\textunderscore )}
\end{itemize}
Narração fabulosa.
Novella.
Mentira.
Moral, contida numa fábula.
\section{Fabulador}
\begin{itemize}
\item {Grp. gram.:m.  e  adj.}
\end{itemize}
\begin{itemize}
\item {Proveniência:(Lat. \textunderscore fabulator\textunderscore )}
\end{itemize}
Aquelle que fabúla.
\section{Fabular}
\begin{itemize}
\item {Grp. gram.:v. t.}
\end{itemize}
\begin{itemize}
\item {Grp. gram.:V. i.}
\end{itemize}
\begin{itemize}
\item {Proveniência:(Lat. \textunderscore fabulari\textunderscore )}
\end{itemize}
Contar em fórma de fábula.
Inventar.
Contar fábulas.
Mentir.
Fazer história levianamente, sem critério.
\section{Fabulário}
\begin{itemize}
\item {Grp. gram.:m.}
\end{itemize}
\begin{itemize}
\item {Proveniência:(Lat. \textunderscore fabularium\textunderscore )}
\end{itemize}
Collecção de fábulas. Cf. Camillo, \textunderscore Quéda de Um Anjo\textunderscore , 86.
\section{Fabuleira}
\begin{itemize}
\item {Grp. gram.:f.}
\end{itemize}
Supposta árvore, que produziria fábulas. Cf. Filinto, XI, 281.
\section{Fabulista}
\begin{itemize}
\item {Grp. gram.:m.}
\end{itemize}
\begin{itemize}
\item {Utilização:Fig.}
\end{itemize}
Compositor de fábulas.
Aquelle que mente; trapaceiro.
\section{Fabulizar}
\begin{itemize}
\item {Grp. gram.:v. t.}
\end{itemize}
O mesmo que \textunderscore fabular\textunderscore .
\section{Fabulosamente}
\begin{itemize}
\item {Grp. gram.:adv.}
\end{itemize}
De modo fabuloso.
\section{Fabuloso}
\begin{itemize}
\item {Grp. gram.:adj.}
\end{itemize}
\begin{itemize}
\item {Utilização:Ext.}
\end{itemize}
\begin{itemize}
\item {Proveniência:(Lat. \textunderscore fabulosus\textunderscore )}
\end{itemize}
Que não tem existência real: \textunderscore o dragão é animal fabuloso\textunderscore .
Inventado.
Relativo á Mythologia.
Incrível.
Obscuro.
Admiravel.
Grandioso; enorme: \textunderscore riquezas fabulosas\textunderscore .
\section{Faca}
\begin{itemize}
\item {Grp. gram.:f.}
\end{itemize}
\begin{itemize}
\item {Utilização:Gír.}
\end{itemize}
\begin{itemize}
\item {Proveniência:(Do lat. \textunderscore falcula\textunderscore )}
\end{itemize}
Instrumento cortante, de lâmina encabada.
Utensílio de madeira, ôsso, etc., para cortar papel, dobrando-o.
Cinta.
\section{Faca}
\begin{itemize}
\item {Grp. gram.:f.}
\end{itemize}
O mesmo que \textunderscore hacaneia\textunderscore .
\section{Facada}
\begin{itemize}
\item {Grp. gram.:f.}
\end{itemize}
\begin{itemize}
\item {Utilização:Fig.}
\end{itemize}
\begin{itemize}
\item {Utilização:Fig.}
\end{itemize}
Golpe de faca.
Surpresa dolorosa.
Offensa, aggressão cobarde.
\section{Faca-de-mato}
\begin{itemize}
\item {Grp. gram.:f.}
\end{itemize}
Espécie de punhal comprido, de que usam caçadores.
\section{Faca-de-rasto}
\begin{itemize}
\item {Grp. gram.:f.}
\end{itemize}
\begin{itemize}
\item {Utilização:Bras. do S}
\end{itemize}
Grande faca, com que se abre caminho no mato.
\section{Facadista}
\begin{itemize}
\item {Grp. gram.:m.}
\end{itemize}
\begin{itemize}
\item {Utilização:Bras}
\end{itemize}
Indivíduo, vezeiro em dar facadas.
\section{Facaia}
\begin{itemize}
\item {Grp. gram.:f. Loc. adv.}
\end{itemize}
\begin{itemize}
\item {Utilização:chul.}
\end{itemize}
\textunderscore Á facaia\textunderscore , á maneira de fadista; gingando.
\section{Faca-inglesa}
\begin{itemize}
\item {Grp. gram.:f.}
\end{itemize}
O mesmo que \textunderscore corta-chefe\textunderscore .
\section{Facalhão}
\begin{itemize}
\item {Grp. gram.:m.}
\end{itemize}
Grande faca.
\section{Facalhaz}
\begin{itemize}
\item {Grp. gram.:m.}
\end{itemize}
O mesmo que \textunderscore facalhão\textunderscore .
\section{Façalvo}
\begin{itemize}
\item {Grp. gram.:adj.}
\end{itemize}
\begin{itemize}
\item {Proveniência:(De \textunderscore face\textunderscore  + \textunderscore alvo\textunderscore )}
\end{itemize}
Que tem um grande sinal branco no focinho, (falando-se de cavallos).
\section{Facané}
\begin{itemize}
\item {Grp. gram.:m.}
\end{itemize}
\begin{itemize}
\item {Utilização:Ant.}
\end{itemize}
Cavallo pequeno ou cavallo menor que o da marca.
(Cp. \textunderscore hacaneia\textunderscore )
\section{Facaneia}
\begin{itemize}
\item {Grp. gram.:f.}
\end{itemize}
\begin{itemize}
\item {Utilização:Des.}
\end{itemize}
(V.hacaneia)
\section{Façanha}
\begin{itemize}
\item {Grp. gram.:f.}
\end{itemize}
\begin{itemize}
\item {Utilização:Irón.}
\end{itemize}
\begin{itemize}
\item {Utilização:Ant.}
\end{itemize}
\begin{itemize}
\item {Proveniência:(Do lat. \textunderscore facinus\textunderscore )}
\end{itemize}
Proêza; acto heroico.
Coisa admirável, extraordinaria.
Acção perversa.
Acto deshonroso.
Aresto; sentença; caso julgado.
\section{Façanheiro}
\begin{itemize}
\item {Grp. gram.:adj.}
\end{itemize}
\begin{itemize}
\item {Grp. gram.:M.}
\end{itemize}
Que alardeia façanhas; gabarola.
Valentão.
\section{Façanhice}
\begin{itemize}
\item {Grp. gram.:f.}
\end{itemize}
O mesmo que \textunderscore façanha\textunderscore , em sentido depreciativo. Cf. Camillo, \textunderscore Coisas Leves e Pesadas\textunderscore , 107.
\section{Façanhosamente}
\begin{itemize}
\item {Grp. gram.:adv.}
\end{itemize}
De modo façanhoso.
\section{Façanhoso}
\begin{itemize}
\item {Grp. gram.:adj.}
\end{itemize}
Admirável.
Que pratíca façanhas.
\section{Façanhudo}
\begin{itemize}
\item {Grp. gram.:adj.}
\end{itemize}
\begin{itemize}
\item {Utilização:Pop.}
\end{itemize}
\begin{itemize}
\item {Proveniência:(De \textunderscore façanha\textunderscore )}
\end{itemize}
Façanhoso.
Desordeiro.
Que pratíca perversidades.
Carrancudo.
\section{Facão}
\begin{itemize}
\item {Grp. gram.:m.}
\end{itemize}
\begin{itemize}
\item {Utilização:Bras. do N}
\end{itemize}
\begin{itemize}
\item {Proveniência:(De \textunderscore faca\textunderscore )}
\end{itemize}
Facalhão.
Utensílio, com que os bombeiros acunham a terra, em volta da bomba.
O mesmo que \textunderscore mascato\textunderscore .
Biscoito grande e mal feito.
\section{Faca-sola}
\begin{itemize}
\item {Grp. gram.:f.}
\end{itemize}
Us. na loc. \textunderscore andar\textunderscore  ou \textunderscore ir á faca-sola\textunderscore , por \textunderscore andar sòzinho, a pé\textunderscore :«\textunderscore ...que o tem visto á faca-sola, sòzinho de rua em rua\textunderscore ». Castilho, \textunderscore Avarento\textunderscore , III, 5.
(Talvez de \textunderscore faca\textunderscore ^2 + \textunderscore sola\textunderscore , como quem diz \textunderscore andar a cavallo nas botas\textunderscore , ou, como diz o povo, \textunderscore andar nos machinhos pretos\textunderscore )
\section{Facção}
\begin{itemize}
\item {Grp. gram.:f.}
\end{itemize}
\begin{itemize}
\item {Proveniência:(Lat. \textunderscore factio\textunderscore )}
\end{itemize}
Empresa militar.
Proêza, insigne feito de armas.
Bando sedicioso.
Parcialidade, partido.
\section{Faccionar}
\begin{itemize}
\item {Grp. gram.:v. t.}
\end{itemize}
\begin{itemize}
\item {Proveniência:(Do lat. \textunderscore factio\textunderscore )}
\end{itemize}
Dividir em facções.
Amotinar; sublevar.
\section{Faccionário}
\begin{itemize}
\item {Grp. gram.:m.}
\end{itemize}
\begin{itemize}
\item {Grp. gram.:Adj.}
\end{itemize}
\begin{itemize}
\item {Proveniência:(Lat. \textunderscore factionarius\textunderscore )}
\end{itemize}
Membro de uma facção.
Pertencente ou relativo a uma facção.
\section{Facciosidade}
\begin{itemize}
\item {Grp. gram.:f.}
\end{itemize}
O mesmo que \textunderscore facciosismo\textunderscore . Cf. Castilho, \textunderscore Tartufo\textunderscore , 216.
\section{Facciosismo}
\begin{itemize}
\item {Grp. gram.:m.}
\end{itemize}
Qualidade de faccioso.
Paixão partidária.
\section{Faccioso}
\begin{itemize}
\item {Grp. gram.:adj.}
\end{itemize}
\begin{itemize}
\item {Proveniência:(Lat. \textunderscore facciosus\textunderscore )}
\end{itemize}
Perturbador da ordem.
Sedicioso.
Sectário apaixonado de uma facção.
Parcial.
\section{Face}
\begin{itemize}
\item {Grp. gram.:f.}
\end{itemize}
\begin{itemize}
\item {Grp. gram.:Loc. prep.}
\end{itemize}
\begin{itemize}
\item {Grp. gram.:Loc. adv.}
\end{itemize}
\begin{itemize}
\item {Proveniência:(Do lat. \textunderscore facies\textunderscore )}
\end{itemize}
Parte anterior da cabeça humana, comprehendendo testa, olhos, nariz, maçans do rosto, lábios e queixo.
Semblante.
Cara; rosto.
Cada uma das partes lateraes da cara, desde os olhos á maxilla inferior: \textunderscore um beijo na face\textunderscore .
Maçan do rosto.
Parte anterior de uma pedra apparelhada.
Parte anterior; frente.
Cada uma das superfícies planas de um sólido: \textunderscore a face do predio\textunderscore .
Superfície: \textunderscore a face do mar\textunderscore .
Lado das medalhas ou moédas, em que está o busto.
Lado de um estôfo, opposto ao avesso.
Situação ou estado de certos assumptos ou de certas ideias.
\textunderscore Á face de\textunderscore , perante, em presença de.
\textunderscore Fazer face\textunderscore , oppor-se.
Estar voltado para algum ponto.
Obviar a alguma coisa.
\textunderscore Face a face\textunderscore , em frente um do outro, (falando-se de duas pessoas).
\section{Facear}
\begin{itemize}
\item {Grp. gram.:v. t.}
\end{itemize}
Fazer faces ou lados de; esquadriar.
\section{Facécia}
\begin{itemize}
\item {Grp. gram.:f.}
\end{itemize}
\begin{itemize}
\item {Proveniência:(Lat. \textunderscore facetia\textunderscore )}
\end{itemize}
Qualidade de facêto.
Phrase ou modos daquelle que é facêto.
\section{Facecioso}
\begin{itemize}
\item {Grp. gram.:adj.}
\end{itemize}
\begin{itemize}
\item {Proveniência:(De \textunderscore facécia\textunderscore )}
\end{itemize}
O mesmo que \textunderscore facêto\textunderscore .
\section{Faceira}
\begin{itemize}
\item {Grp. gram.:f.}
\end{itemize}
\begin{itemize}
\item {Utilização:Fam.}
\end{itemize}
\begin{itemize}
\item {Utilização:Bras}
\end{itemize}
\begin{itemize}
\item {Grp. gram.:M.  e  f.}
\end{itemize}
\begin{itemize}
\item {Grp. gram.:F. pl.}
\end{itemize}
\begin{itemize}
\item {Utilização:Prov.}
\end{itemize}
\begin{itemize}
\item {Utilização:trasm.}
\end{itemize}
\begin{itemize}
\item {Proveniência:(De \textunderscore face\textunderscore )}
\end{itemize}
Carne das partes lateraes do focinho dos bois.
Faces gordas.
Mulher affectada.
Pessôa alegre, galhofeira.
Peralta, petimetre.
Correias, que fazem parte da cabeçada e se ligam á extremidade do freio.
Veiga; terra plana de lavoira, perto de povoação.
Seara: \textunderscore as faceiras de centeio\textunderscore .
\section{Faceiramente}
\begin{itemize}
\item {Grp. gram.:adv.}
\end{itemize}
De modo faceiro.
\section{Faceirar}
\begin{itemize}
\item {Grp. gram.:v. i.}
\end{itemize}
\begin{itemize}
\item {Utilização:Bras}
\end{itemize}
\begin{itemize}
\item {Proveniência:(De \textunderscore faceiro\textunderscore )}
\end{itemize}
Têr maneiras elegantes; vestir com elegância.
\section{Faceirice}
\begin{itemize}
\item {Grp. gram.:f.}
\end{itemize}
\begin{itemize}
\item {Utilização:Bras}
\end{itemize}
\begin{itemize}
\item {Proveniência:(De \textunderscore faceiro\textunderscore )}
\end{itemize}
Tafularia; elegância.
Ar pretensioso.
Aspecto risonho.
\section{Faceiro}
\begin{itemize}
\item {Grp. gram.:adj.}
\end{itemize}
\begin{itemize}
\item {Grp. gram.:M.}
\end{itemize}
\begin{itemize}
\item {Utilização:Pop.}
\end{itemize}
Garrido; vistoso.
Bonacheirão.
Indivíduo simplório; patarata.
(Cp. \textunderscore faceira\textunderscore )
\section{Faceiró}
\begin{itemize}
\item {Grp. gram.:m.}
\end{itemize}
\begin{itemize}
\item {Utilização:Ant.}
\end{itemize}
O mesmo que \textunderscore faceirôa\textunderscore .
\section{Faceirôa}
\begin{itemize}
\item {Grp. gram.:f.}
\end{itemize}
\begin{itemize}
\item {Utilização:Ant.}
\end{itemize}
Travesseiro para repoisar a face.
\section{Facejar}
\begin{itemize}
\item {Grp. gram.:v. t.}
\end{itemize}
(V.facear)
\section{Facer}
\begin{itemize}
\item {Grp. gram.:v. t.  e  i.}
\end{itemize}
\begin{itemize}
\item {Utilização:Ant.}
\end{itemize}
O mesmo que \textunderscore fazer\textunderscore .
\section{Facergenes}
\begin{itemize}
\item {Grp. gram.:m.}
\end{itemize}
\begin{itemize}
\item {Utilização:Ant.}
\end{itemize}
\begin{itemize}
\item {Proveniência:(Do lat. \textunderscore facere\textunderscore  + \textunderscore genu\textunderscore )}
\end{itemize}
Talvez genuflexório deante do altar.
\section{Facêta}
\begin{itemize}
\item {Grp. gram.:f.}
\end{itemize}
\begin{itemize}
\item {Proveniência:(Fr. \textunderscore facette\textunderscore )}
\end{itemize}
Pequena face.
Cada uma das superfícies regulares de uma pedra preciosa.
Porção circunscripta da superfície de um osso.
\section{Facetamente}
\begin{itemize}
\item {Grp. gram.:adv.}
\end{itemize}
De modo facêto.
\section{Facetar}
\begin{itemize}
\item {Grp. gram.:v. t.}
\end{itemize}
Fazer facêtas em; lapidar.
\section{Facetear}
\begin{itemize}
\item {Grp. gram.:v. i.}
\end{itemize}
\begin{itemize}
\item {Proveniência:(Lat. \textunderscore faceteari\textunderscore )}
\end{itemize}
Fazer ou dizer facécias.
\section{Facetear}
\begin{itemize}
\item {Grp. gram.:v. t.}
\end{itemize}
O mesmo que \textunderscore facetar\textunderscore .
\section{Facêto}
\begin{itemize}
\item {Grp. gram.:adj.}
\end{itemize}
\begin{itemize}
\item {Proveniência:(Lat. \textunderscore facetus\textunderscore )}
\end{itemize}
Gracioso; chistoso; alegre; brincalhão.
\section{Facha}
\begin{itemize}
\item {Grp. gram.:f.}
\end{itemize}
\begin{itemize}
\item {Utilização:Des.}
\end{itemize}
O mesmo que \textunderscore facho\textunderscore ^1.
O mesmo que \textunderscore candeio\textunderscore .
\section{Facha}
\begin{itemize}
\item {Grp. gram.:f.}
\end{itemize}
\begin{itemize}
\item {Utilização:Ant.}
\end{itemize}
O mesmo que \textunderscore acha\textunderscore ^2, arma.
\section{Facha}
\begin{itemize}
\item {Grp. gram.:m.}
\end{itemize}
\begin{itemize}
\item {Utilização:Ant.}
\end{itemize}
\begin{itemize}
\item {Proveniência:(It. \textunderscore faccia\textunderscore )}
\end{itemize}
O mesmo que \textunderscore face\textunderscore .
\section{Fachada}
\begin{itemize}
\item {Grp. gram.:f.}
\end{itemize}
\begin{itemize}
\item {Utilização:Fam.}
\end{itemize}
\begin{itemize}
\item {Proveniência:(De \textunderscore facha\textunderscore ^3)}
\end{itemize}
Lado principal ou frontaria de um edifício.
Frontispício de um livro.
Semblante, apparência de uma pessôa.
\section{Fachear}
\begin{itemize}
\item {Grp. gram.:v. i.}
\end{itemize}
\begin{itemize}
\item {Utilização:Bras. de Piaui}
\end{itemize}
Quebrar? partir-se?: \textunderscore vara, que verga mas não facheia\textunderscore .
\section{Facheiro}
\begin{itemize}
\item {Grp. gram.:m.}
\end{itemize}
\begin{itemize}
\item {Utilização:Ant.}
\end{itemize}
Aquelle que conduz o facho.
Lugar ou coisa, em que se apoia o facho.
O mesmo que \textunderscore pharoleiro\textunderscore . Cf. \textunderscore Hist. Insul.\textunderscore , II, 39.
\section{Facheiro}
\begin{itemize}
\item {Grp. gram.:m.}
\end{itemize}
\begin{itemize}
\item {Utilização:Ant.}
\end{itemize}
\begin{itemize}
\item {Proveniência:(De \textunderscore facha\textunderscore ^2)}
\end{itemize}
Homem armado de acha.
\section{Fachis}
\begin{itemize}
\item {Grp. gram.:m. pl.}
\end{itemize}
\begin{itemize}
\item {Utilização:T. de Macau}
\end{itemize}
Os dois pauzinhos, com que os Chineses comem, servindo-se delles como de garfo.
(Chin. \textunderscore fa-chi\textunderscore )
\section{Facho}
\begin{itemize}
\item {Grp. gram.:m.}
\end{itemize}
\begin{itemize}
\item {Utilização:Ext.}
\end{itemize}
\begin{itemize}
\item {Utilização:Fig.}
\end{itemize}
\begin{itemize}
\item {Proveniência:(Do lat. \textunderscore fax\textunderscore , \textunderscore facis\textunderscore ?)}
\end{itemize}
Archote.
Luzeiro.
Pharol.
Matéria inflammada que se acende de noite para qualquer fim.
Tudo que esclarece a intelligência.
Guia, norte.
Aquillo que excita ou desenvolve uma paixão, uma calamidade, etc.
\section{Facho}
\begin{itemize}
\item {Grp. gram.:m.}
\end{itemize}
Nome de um peixe acanthopterýgio, (\textunderscore apsilus fuscus\textunderscore ).
\section{Fachoca}
\begin{itemize}
\item {Grp. gram.:f.}
\end{itemize}
\begin{itemize}
\item {Proveniência:(De \textunderscore facho\textunderscore ^1)}
\end{itemize}
Feixe de palha, toscamente apertado de espaço a espaço, e que serve de archote, para alumiar o caminho, em algumas aldeias.
\section{Fachoqueira}
\begin{itemize}
\item {Grp. gram.:f.}
\end{itemize}
\begin{itemize}
\item {Utilização:Prov.}
\end{itemize}
\begin{itemize}
\item {Utilização:beir.}
\end{itemize}
O mesmo que \textunderscore fachoqueiro\textunderscore .
\section{Fachoqueiro}
\begin{itemize}
\item {Grp. gram.:m.}
\end{itemize}
\begin{itemize}
\item {Utilização:Prov.}
\end{itemize}
\begin{itemize}
\item {Utilização:trasm.}
\end{itemize}
\begin{itemize}
\item {Proveniência:(De \textunderscore fachoca\textunderscore )}
\end{itemize}
Palha ou carqueja acesa, com que se chamusca o porco, depois de morto.
Facho grande; facho tosco. Cf. Camillo, \textunderscore Brasileira\textunderscore , 159.
\section{Fachudaço}
\begin{itemize}
\item {Grp. gram.:adj.}
\end{itemize}
\begin{itemize}
\item {Utilização:Bras. do S}
\end{itemize}
\begin{itemize}
\item {Proveniência:(De \textunderscore fachudo\textunderscore )}
\end{itemize}
Muito lindo.
\section{Fachudo}
\begin{itemize}
\item {Grp. gram.:adj.}
\end{itemize}
\begin{itemize}
\item {Utilização:Bras. do S}
\end{itemize}
\begin{itemize}
\item {Proveniência:(De \textunderscore facha\textunderscore ^3)}
\end{itemize}
Lindo.
\section{Facial}
\begin{itemize}
\item {Grp. gram.:adj.}
\end{itemize}
\begin{itemize}
\item {Proveniência:(Lat. \textunderscore facialis\textunderscore )}
\end{itemize}
Relativo á face.
Que diz respeito aos nervos da face.
\section{Facienda}
\begin{itemize}
\item {Grp. gram.:f.}
\end{itemize}
\begin{itemize}
\item {Utilização:Ant.}
\end{itemize}
\begin{itemize}
\item {Proveniência:(Lat. \textunderscore facienda\textunderscore )}
\end{itemize}
Tudo aquillo que se tem de fazer; o que há para fazer.
Agenda.
\section{Facies}
\begin{itemize}
\item {fónica:fáciès}
\end{itemize}
\begin{itemize}
\item {Grp. gram.:f.}
\end{itemize}
\begin{itemize}
\item {Utilização:Geol.}
\end{itemize}
\begin{itemize}
\item {Proveniência:(T. lat.)}
\end{itemize}
Complexo de caracteres exteriores, que distinguem um grupo de indivíduos.
Alteração physionómica de um doente.
Primeiro aspecto; apparência.
Aspecto de terreno.
\section{Fácil}
\begin{itemize}
\item {Grp. gram.:adj.}
\end{itemize}
\begin{itemize}
\item {Grp. gram.:Adv.}
\end{itemize}
\begin{itemize}
\item {Grp. gram.:Loc. adv.}
\end{itemize}
\begin{itemize}
\item {Proveniência:(Lat. \textunderscore facilís\textunderscore )}
\end{itemize}
Que se faz ou que se consegue sem trabalho ou sem custo.
Simples; vulgar; natural.
Claro.
Muito susceptível.
Que tem facilidade em alguma coisa.
Brando.
Dócil; complacente.
Accessivel.
Ilhano, franco.
Precipitado, que reflecte pouco.
Provável.
Supportável.
O mesmo que \textunderscore facilmente\textunderscore :«\textunderscore ...e para mais fácil o conseguir...\textunderscore »Filinto, \textunderscore D. Man.\textunderscore  I, 25.
\textunderscore Ao fácil\textunderscore , o mesmo que \textunderscore facilmente\textunderscore :«\textunderscore ...se apoderassem mais ao fácil da cidade.\textunderscore »\textunderscore Idem, ibidem\textunderscore , II, 8.
\section{Facilidade}
\begin{itemize}
\item {Grp. gram.:f.}
\end{itemize}
\begin{itemize}
\item {Grp. gram.:Pl.}
\end{itemize}
\begin{itemize}
\item {Proveniência:(Lat. \textunderscore facilitas\textunderscore )}
\end{itemize}
Qualidade daquillo que é fácil.
Promptidão; destreza.
Commodidade.
Sem-ceremónia.
Falta de escrúpulos.
Rapidez na execução de um trabalho ou de obra de arte.
Condescendência.
Meios promptos ou fáceis para a consecução de um fim: \textunderscore encontrar facilidades\textunderscore .
\section{Facilimamente}
\begin{itemize}
\item {Grp. gram.:adv.}
\end{itemize}
De modo facílimo.
\section{Facílimo}
\begin{itemize}
\item {Grp. gram.:adj.}
\end{itemize}
\begin{itemize}
\item {Proveniência:(Lat. \textunderscore facillimus\textunderscore )}
\end{itemize}
Muito fácil.
\section{Facilitação}
\begin{itemize}
\item {Grp. gram.:f.}
\end{itemize}
Acto ou effeito de facilitar.
\section{Facilitador}
\begin{itemize}
\item {Grp. gram.:m.  e  adj.}
\end{itemize}
O que facilita.
\section{Facilitar}
\begin{itemize}
\item {Grp. gram.:v. t.}
\end{itemize}
\begin{itemize}
\item {Grp. gram.:V. p.}
\end{itemize}
\begin{itemize}
\item {Proveniência:(Do lat. \textunderscore facilitas\textunderscore )}
\end{itemize}
Tornar fácil.
Pôr á disposição ou ao alcance de alguém.
Auxiliar: \textunderscore facilitar uma empresa\textunderscore .
Estar disposto; promptificar-se.
Tornar-se destro.
\section{Facillimamente}
\begin{itemize}
\item {Grp. gram.:adv.}
\end{itemize}
De modo facíllimo.
\section{Facíllimo}
\begin{itemize}
\item {Grp. gram.:adj.}
\end{itemize}
\begin{itemize}
\item {Proveniência:(Lat. \textunderscore facillimus\textunderscore )}
\end{itemize}
Muito fácil.
\section{Facilmente}
\begin{itemize}
\item {fónica:fá}
\end{itemize}
\begin{itemize}
\item {Grp. gram.:adv.}
\end{itemize}
De modo fácil.
Irrefletidamente, precipitadamente.
Indubitavelmente.
\section{Facínora}
\begin{itemize}
\item {Grp. gram.:m.}
\end{itemize}
\begin{itemize}
\item {Grp. gram.:Adj.}
\end{itemize}
Homem perverso.
Aquelle que commeteu grande crime.
Facinoroso.
(Talvez abstraindo do lat. \textunderscore facinorosus\textunderscore )
\section{Facinoroso}
\begin{itemize}
\item {Grp. gram.:adj.}
\end{itemize}
\begin{itemize}
\item {Grp. gram.:M.}
\end{itemize}
\begin{itemize}
\item {Proveniência:(Lat. \textunderscore facinorusus\textunderscore )}
\end{itemize}
Perverso.
Que é autor de grande crime.
O mesmo que \textunderscore facínora\textunderscore :«\textunderscore ...os grandes facinorosos\textunderscore »Camillo, \textunderscore Caveira\textunderscore , 414.
\section{Facistol}
\begin{itemize}
\item {Grp. gram.:m.}
\end{itemize}
Grande estante, no côro das igrejas, para livros de canto ou livros litúrgicos; faldistório.
(Cast. \textunderscore facistol\textunderscore )
\section{Faco}
\begin{itemize}
\item {Grp. gram.:m.}
\end{itemize}
\begin{itemize}
\item {Utilização:Prov.}
\end{itemize}
\begin{itemize}
\item {Utilização:alent.}
\end{itemize}
\begin{itemize}
\item {Proveniência:(De \textunderscore faca\textunderscore ^1)}
\end{itemize}
Fôlha de navalha, a que, se põe um cabo rústico.
\section{Façoila}
\begin{itemize}
\item {Grp. gram.:f.}
\end{itemize}
\begin{itemize}
\item {Utilização:Chul.}
\end{itemize}
Faceira; face grande.
\section{Facote}
\begin{itemize}
\item {Grp. gram.:m.}
\end{itemize}
\begin{itemize}
\item {Proveniência:(De \textunderscore faca\textunderscore )}
\end{itemize}
Instrumento cirúrgico, para raspar ossos ou alargar certas fracturas.
\section{Fac-similar}
\begin{itemize}
\item {Grp. gram.:adj.}
\end{itemize}
\begin{itemize}
\item {Utilização:Bras}
\end{itemize}
\begin{itemize}
\item {Utilização:Neol.}
\end{itemize}
Relativo a \textunderscore fac-simile\textunderscore .
\section{Fac-simile}
\begin{itemize}
\item {fónica:fak-similè}
\end{itemize}
\begin{itemize}
\item {Grp. gram.:m.}
\end{itemize}
Reproducção exacta de uma assinatura, ou de outra escrita, ou de uma estampa, por gravuras ou por outro processo.
(Loc. lat., que significa \textunderscore fazer igual\textunderscore )
\section{Facteiro}
\begin{itemize}
\item {Grp. gram.:adj.}
\end{itemize}
\begin{itemize}
\item {Utilização:Prov.}
\end{itemize}
\begin{itemize}
\item {Proveniência:(De \textunderscore facto\textunderscore )}
\end{itemize}
Que conta muitos factos, que diz histórias.
Conversador.
\section{Factício}
\begin{itemize}
\item {Grp. gram.:adj.}
\end{itemize}
\begin{itemize}
\item {Proveniência:(Lat. \textunderscore facticius\textunderscore )}
\end{itemize}
Que não é natural, mas produzido por arte; artificial.
Caprichoso.
Convencional.
\section{Factitivo}
\begin{itemize}
\item {Grp. gram.:adj.}
\end{itemize}
\begin{itemize}
\item {Utilização:Gram.}
\end{itemize}
\begin{itemize}
\item {Proveniência:(Do lat. \textunderscore factitare\textunderscore )}
\end{itemize}
Diz-se dos verbos, que são duplamente activos, como \textunderscore branquear\textunderscore , (fazer que seja branco), \textunderscore adormecer\textunderscore , (fazer que alguém durma), etc.
\section{Factível}
\begin{itemize}
\item {Grp. gram.:adj.}
\end{itemize}
\begin{itemize}
\item {Proveniência:(De \textunderscore facto\textunderscore )}
\end{itemize}
Que se pôde fazer; possível:«\textunderscore ...o que na verdade era cousa mudo factível\textunderscore ». \textunderscore Jornada de Áfr.\textunderscore , c. v.
\section{Facto}
\begin{itemize}
\item {Grp. gram.:m.}
\end{itemize}
\begin{itemize}
\item {Grp. gram.:Loc. adv.}
\end{itemize}
\begin{itemize}
\item {Proveniência:(Lat. \textunderscore factum\textunderscore )}
\end{itemize}
Aquillo que se fez.
Successo.
Acção.
Caso.
\textunderscore De facto\textunderscore , com effeito, effectivamente.
\textunderscore Estar ao facto\textunderscore , sêr sabedor, têr conhecimento, estar sciente.
\section{Factor}
\begin{itemize}
\item {Grp. gram.:m.}
\end{itemize}
\begin{itemize}
\item {Proveniência:(Lat. \textunderscore factor\textunderscore )}
\end{itemize}
Aquelle que faz ou executa uma coisa.
Cada um dos termos de uma multiplicação, em Arithmética.
Aquillo que concorre, para certo resultado.
Empregado de estação de linha férrea, encarregado da escrituração relativa a recepção, expedição e entrega de bagagens e mercadorias.
\section{Factotum}
\begin{itemize}
\item {fónica:faktótum}
\end{itemize}
\begin{itemize}
\item {Grp. gram.:m.}
\end{itemize}
\begin{itemize}
\item {Proveniência:(Do lat. \textunderscore facere\textunderscore  + \textunderscore totum\textunderscore , fazer tudo)}
\end{itemize}
Aquelle que se inculca como apto para resolver todos os negócios.
Aquelle que trata de todos os negócios de alguém.
\section{Factura}
\begin{itemize}
\item {Grp. gram.:f.}
\end{itemize}
\begin{itemize}
\item {Proveniência:(Lat. \textunderscore factura\textunderscore )}
\end{itemize}
Acto, effeito ou maneira de fazer.
Relação de mercadorias expedidas ou vendidas e acompanhadas da indicação do preço respectivo.
\section{Facturar}
\begin{itemize}
\item {Grp. gram.:v. t.}
\end{itemize}
Meter em factura (mercadorias).
\section{Façudo}
\begin{itemize}
\item {Grp. gram.:adj.}
\end{itemize}
\begin{itemize}
\item {Utilização:Chul.}
\end{itemize}
\begin{itemize}
\item {Proveniência:(De \textunderscore face\textunderscore )}
\end{itemize}
Que tem faceira; gorducho.
\section{Fácula}
\begin{itemize}
\item {Grp. gram.:f.}
\end{itemize}
\begin{itemize}
\item {Proveniência:(Lat. \textunderscore facula\textunderscore )}
\end{itemize}
Mancha luminosa no disco do Sol ou da Lua.
\section{Façula}
\begin{itemize}
\item {Grp. gram.:f.}
\end{itemize}
\begin{itemize}
\item {Utilização:Fam.}
\end{itemize}
Face grande; façoila.
\section{Faculdade}
\begin{itemize}
\item {Grp. gram.:f.}
\end{itemize}
\begin{itemize}
\item {Proveniência:(Lat. \textunderscore facultas\textunderscore )}
\end{itemize}
Poder de fazer.
Direito.
Potência moral.
Capacidade.
Ensejo.
Permissão.
Sciência, que se ensina nas Universidades: \textunderscore Faculdade de Direito\textunderscore .
Corporação dos professores dessa sciência.
\section{Facultar}
\begin{itemize}
\item {Grp. gram.:v. t.}
\end{itemize}
\begin{itemize}
\item {Proveniência:(Do lat. \textunderscore facultas\textunderscore )}
\end{itemize}
Permittir.
Tornar fácil.
Pôr á disposição de alguém.
\section{Facultativo}
\begin{itemize}
\item {Grp. gram.:adj.}
\end{itemize}
\begin{itemize}
\item {Grp. gram.:M.}
\end{itemize}
\begin{itemize}
\item {Proveniência:(De \textunderscore facultar\textunderscore )}
\end{itemize}
Que dá faculdade ou poder.
Que permitte que se faça ou se não faça uma coisa; dependente da vontade, arbitrário.
Aquelle que exerce a Medicina.
\section{Facultoso}
\begin{itemize}
\item {Grp. gram.:adj.}
\end{itemize}
\begin{itemize}
\item {Proveniência:(Do lat. \textunderscore facultas\textunderscore )}
\end{itemize}
Que dispõe de muitos recursos; rico.
\section{Facúndia}
\begin{itemize}
\item {Grp. gram.:f.}
\end{itemize}
\begin{itemize}
\item {Proveniência:(Lat. \textunderscore facundia\textunderscore )}
\end{itemize}
Facilidade de discursar.
\section{Facundidade}
\begin{itemize}
\item {Grp. gram.:f.}
\end{itemize}
\begin{itemize}
\item {Proveniência:(Lat. \textunderscore facunditas\textunderscore )}
\end{itemize}
O mesmo que \textunderscore facúndia\textunderscore .
\section{Facundo}
\begin{itemize}
\item {Grp. gram.:adj.}
\end{itemize}
\begin{itemize}
\item {Proveniência:(Lat. \textunderscore facundus\textunderscore )}
\end{itemize}
Que tem facúndia, que revela facúndia; eloquente.
\section{Fada}
\begin{itemize}
\item {Grp. gram.:f.}
\end{itemize}
\begin{itemize}
\item {Utilização:Fig.}
\end{itemize}
\begin{itemize}
\item {Proveniência:(Do lat. \textunderscore fata\textunderscore , pl. de \textunderscore fatum\textunderscore )}
\end{itemize}
Personagem imaginária, a que se attribuía a faculdade sobrenatural de prever o futuro, determinar venturas ou desgraças, produzir encantamentos, etc.
Auspício.
Mulher muito formosa; mulher que seduz ou encanta.
\section{Fadar}
\begin{itemize}
\item {Grp. gram.:v. t.}
\end{itemize}
\begin{itemize}
\item {Grp. gram.:M.}
\end{itemize}
\begin{itemize}
\item {Proveniência:(De \textunderscore fado\textunderscore )}
\end{itemize}
Predestinar.
Auspiciar.
Favorecer.
Conceder dons excepcionaes a: \textunderscore bôas fadas te fadaram\textunderscore .
O mesmo que \textunderscore fadário\textunderscore . Cf. Garrett, \textunderscore Camões\textunderscore .
\section{Fadário}
\begin{itemize}
\item {Grp. gram.:m.}
\end{itemize}
\begin{itemize}
\item {Proveniência:(De \textunderscore fado\textunderscore )}
\end{itemize}
Sorte; fado.
Destino, imposto por um poder sobrenatural.
Vida trabalhosa; desgostos.
\section{Fadejar}
\begin{itemize}
\item {Grp. gram.:v. i.}
\end{itemize}
\begin{itemize}
\item {Grp. gram.:V. t.}
\end{itemize}
\begin{itemize}
\item {Utilização:Neol.}
\end{itemize}
\begin{itemize}
\item {Proveniência:(De \textunderscore fado\textunderscore )}
\end{itemize}
Cumprir o fado ou o destino.
Tocar ou cantar, á maneira de fado.
\section{Fádico}
\begin{itemize}
\item {Grp. gram.:adj.}
\end{itemize}
\begin{itemize}
\item {Utilização:bras}
\end{itemize}
\begin{itemize}
\item {Utilização:Neol.}
\end{itemize}
Próprio de fada; encantador.
\section{Fadiga}
\begin{itemize}
\item {Grp. gram.:f.}
\end{itemize}
\begin{itemize}
\item {Proveniência:(De \textunderscore fadigar\textunderscore )}
\end{itemize}
Cansaço, resultante de trabalho excessivo.
Faina; lida; trabalho.
\section{Fadigar}
\textunderscore v. t.\textunderscore  (e der.)
O mesmo que \textunderscore fatigar\textunderscore ^1, etc.
\section{Fadigosamente}
\begin{itemize}
\item {Grp. gram.:adv.}
\end{itemize}
De modo fadigoso.
\section{Fadigoso}
\begin{itemize}
\item {Grp. gram.:adj.}
\end{itemize}
Em que há fadiga; feito com fadiga.
Fatigante.
\section{Fadista}
\begin{itemize}
\item {Grp. gram.:m.}
\end{itemize}
\begin{itemize}
\item {Grp. gram.:F.}
\end{itemize}
\begin{itemize}
\item {Proveniência:(De \textunderscore fado\textunderscore )}
\end{itemize}
Aquelle que canta ou bate o fado.
Aquelle que frequenta bordeis e convive com a escória das meretrizes.
Aquelle que, pelo traje, se assemelha ao frequentador de alcoices.
Prostituta.
\section{Fadistagem}
\begin{itemize}
\item {Grp. gram.:f.}
\end{itemize}
Classe dos fadistas.
Vida de fadista.
\section{Fadistal}
\begin{itemize}
\item {Grp. gram.:adj.}
\end{itemize}
\begin{itemize}
\item {Utilização:Neol.}
\end{itemize}
Relativo a fadista.
\section{Fadistice}
\begin{itemize}
\item {Grp. gram.:f.}
\end{itemize}
Modos ou acto de fadista.
\section{Fado}
\begin{itemize}
\item {Grp. gram.:m.}
\end{itemize}
\begin{itemize}
\item {Utilização:Pop.}
\end{itemize}
\begin{itemize}
\item {Grp. gram.:Pl.}
\end{itemize}
\begin{itemize}
\item {Proveniência:(Lat. \textunderscore fatum\textunderscore )}
\end{itemize}
Destino, a ordem das coisas.
Agoiro.
Aquillo que tem de acontecer.
Aquillo que se considera destinado irrevogavelmente.
Canção popular, geralmente allusiva aos trabalhos da vida, ao fadário.
Música e dança dessa canção.
A vida do lupanar.
Últimos fins do homem.
A morte.
A fatalidade.
A Providência: \textunderscore se os fados o permittirem\textunderscore .
\section{Faenga}
\begin{itemize}
\item {Grp. gram.:f.}
\end{itemize}
\begin{itemize}
\item {Utilização:Ant.}
\end{itemize}
Casa pública, onde mulheres vendiam pão.
(Cp. \textunderscore fânega\textunderscore  e \textunderscore fanga\textunderscore )
\section{Fagar}
\textunderscore v. t.\textunderscore  (e der.)
O mesmo que \textunderscore afagar\textunderscore , etc. Cf. Filinto, VI, 189.
\section{Fagícola}
\begin{itemize}
\item {Grp. gram.:adj.}
\end{itemize}
\begin{itemize}
\item {Proveniência:(Do lat. \textunderscore fagus\textunderscore  + \textunderscore colere\textunderscore )}
\end{itemize}
Que cresce ou vive sôbre as faias.
\section{Fagónia}
\begin{itemize}
\item {Grp. gram.:f.}
\end{itemize}
\begin{itemize}
\item {Proveniência:(De \textunderscore Fagon\textunderscore , n. p.)}
\end{itemize}
Gênero de plantas zygophýlleas.
\section{Fagopiro}
\begin{itemize}
\item {Grp. gram.:m.}
\end{itemize}
Planta anual e vivaz, (\textunderscore polygoneum fagopyro\textunderscore ).
\section{Fagopyro}
\begin{itemize}
\item {Grp. gram.:m.}
\end{itemize}
Planta annual e vivaz, (\textunderscore polygoneum fagopyro\textunderscore ).
\section{Fagote}
\begin{itemize}
\item {Grp. gram.:m.}
\end{itemize}
\begin{itemize}
\item {Grp. gram.:Loc.}
\end{itemize}
\begin{itemize}
\item {Utilização:fam.}
\end{itemize}
\begin{itemize}
\item {Proveniência:(It. \textunderscore fagotto\textunderscore )}
\end{itemize}
Espécie de clarinete, o mais grave dos instrumentos de madeira, empregados na orchestra e na banda militar.
\textunderscore Ir ao fagote\textunderscore , ou \textunderscore aos fagotes\textunderscore , ir ao costado, bater em alguém.
\section{Fagotista}
\begin{itemize}
\item {Grp. gram.:m.}
\end{itemize}
Tocador de fagote.
\section{Fagueiro}
\begin{itemize}
\item {fónica:fá}
\end{itemize}
\begin{itemize}
\item {Grp. gram.:adj.}
\end{itemize}
\begin{itemize}
\item {Utilização:T. de Turquel}
\end{itemize}
\begin{itemize}
\item {Utilização:T. de San-Jorge}
\end{itemize}
\begin{itemize}
\item {Proveniência:(De \textunderscore fagar\textunderscore )}
\end{itemize}
Que afaga; que é meigo.
Suave; agradável: \textunderscore aragem fagueira\textunderscore .
Diz-se da terra leve, solta, mais ou menos pulverulenta.
Diz-se do espírito que fala de dentro de um corpo humano.
\section{Faguice}
\begin{itemize}
\item {fónica:fá}
\end{itemize}
\begin{itemize}
\item {Grp. gram.:f.}
\end{itemize}
Qualidade de fagueiro; meiguice.
Afago. Cf. Alex. Lobo, II, 135.
(Cp. \textunderscore fagueiro\textunderscore )
\section{Fagulha}
\begin{itemize}
\item {Grp. gram.:f.}
\end{itemize}
\begin{itemize}
\item {Grp. gram.:M.  e  f.}
\end{itemize}
\begin{itemize}
\item {Utilização:Fam.}
\end{itemize}
\begin{itemize}
\item {Utilização:Prov.}
\end{itemize}
\begin{itemize}
\item {Utilização:minh.}
\end{itemize}
\begin{itemize}
\item {Proveniência:(Do lat. \textunderscore favilla\textunderscore )}
\end{itemize}
Centelha; chispa; faísca.
Pessôa irrequieta, que em tudo se intromete.
Caruma sêca.
\section{Fagulhação}
\begin{itemize}
\item {Grp. gram.:f.}
\end{itemize}
Acto de fagulhar.
\section{Fagulhar}
\begin{itemize}
\item {Grp. gram.:v. i.}
\end{itemize}
Expedir fagulhas.
Scintillar.
\section{Fagulharia}
\begin{itemize}
\item {Grp. gram.:f.}
\end{itemize}
Grande porção de fagulhas. Cf. Castilho, \textunderscore Fausto\textunderscore , 344.
\section{Fagulheiro}
\begin{itemize}
\item {Grp. gram.:m.}
\end{itemize}
Orifício nas máquinas de debulhar. Cf. \textunderscore Gaz. dos Lavr.\textunderscore , I, 17.
\section{Fagulhento}
\begin{itemize}
\item {Grp. gram.:adj.}
\end{itemize}
\begin{itemize}
\item {Utilização:Fig.}
\end{itemize}
Que expelle fagulhas.
Irrequieto.
Que é grazina ou fagulha.
\section{Fagundes}
\begin{itemize}
\item {Grp. gram.:m.}
\end{itemize}
\begin{itemize}
\item {Utilização:Prov.}
\end{itemize}
\begin{itemize}
\item {Utilização:beir.}
\end{itemize}
\begin{itemize}
\item {Utilização:fam.}
\end{itemize}
Homem desprezivel, pulha, biltre.
\section{Faia}
\begin{itemize}
\item {Grp. gram.:f.}
\end{itemize}
\begin{itemize}
\item {Proveniência:(Do lat. \textunderscore fagus\textunderscore )}
\end{itemize}
Árvore amentácea, elevada e frondosa.
\section{Faia}
\begin{itemize}
\item {Grp. gram.:f.}
\end{itemize}
Entrelinha typográphica.
\section{Faia}
\begin{itemize}
\item {Grp. gram.:m.}
\end{itemize}
\begin{itemize}
\item {Utilização:Pop.}
\end{itemize}
O mesmo que \textunderscore fadista\textunderscore .
\section{Faial}
\begin{itemize}
\item {Grp. gram.:m.}
\end{itemize}
Bosque de faias.
\section{Faial}
\begin{itemize}
\item {Grp. gram.:m.}
\end{itemize}
\begin{itemize}
\item {Utilização:Prov.}
\end{itemize}
\begin{itemize}
\item {Utilização:trasm.}
\end{itemize}
Despenhadeiro; alcantil. (Colhido em Lagoaça)
\section{Faialite}
\begin{itemize}
\item {Grp. gram.:f.}
\end{itemize}
Mineral do grupo da olivina, reconhecido pela primeira vez nas escórias vulcânicas da ilha do Faial, (Açores). Cf. G. Guimarães, \textunderscore Geologia\textunderscore , 95.
\section{Faianca}
\begin{itemize}
\item {Grp. gram.:f.}
\end{itemize}
\begin{itemize}
\item {Utilização:Des.}
\end{itemize}
O mesmo que \textunderscore fancaria\textunderscore .
\section{Faianca}
\begin{itemize}
\item {Grp. gram.:f.}
\end{itemize}
\begin{itemize}
\item {Grp. gram.:Adj.}
\end{itemize}
\begin{itemize}
\item {Utilização:Prov.}
\end{itemize}
\begin{itemize}
\item {Utilização:trasm.}
\end{itemize}
O mesmo que \textunderscore facaia\textunderscore .
Torto, cambado.
\section{Faiança}
\begin{itemize}
\item {Grp. gram.:f.}
\end{itemize}
\begin{itemize}
\item {Proveniência:(Fr. \textunderscore faience\textunderscore )}
\end{itemize}
Loiça de barro, vidrada ou esmaltada.
Loiça de pó de pedra.
\section{Faianqueiro}
\begin{itemize}
\item {Grp. gram.:m.}
\end{itemize}
\begin{itemize}
\item {Utilização:Ant.}
\end{itemize}
\begin{itemize}
\item {Proveniência:(De \textunderscore faianca\textunderscore ^1)}
\end{itemize}
O mesmo que \textunderscore fanqueiro\textunderscore .
\section{Faiante}
\begin{itemize}
\item {Grp. gram.:m.}
\end{itemize}
\begin{itemize}
\item {Utilização:Gír.}
\end{itemize}
\begin{itemize}
\item {Proveniência:(De \textunderscore faia\textunderscore ^3)}
\end{itemize}
O mesmo que \textunderscore fadista\textunderscore .
Farsola, impostor:«\textunderscore é um faiante, que o está aqui a comer e mais aos patolas da sua laia\textunderscore ». Camillo.
\section{Faião}
\begin{itemize}
\item {Grp. gram.:m.}
\end{itemize}
\begin{itemize}
\item {Utilização:Ant.}
\end{itemize}
Alguidar.
Escudela.
\section{Faiar}
\begin{itemize}
\item {Grp. gram.:v. t.}
\end{itemize}
\begin{itemize}
\item {Utilização:Gír.}
\end{itemize}
\begin{itemize}
\item {Proveniência:(De \textunderscore faia\textunderscore ^2)}
\end{itemize}
Pôr entrelinhas typográphicas em; entrelinhar; espacejar.
Furtar, empalmar.
\section{Faída}
\begin{itemize}
\item {Grp. gram.:f.}
\end{itemize}
\begin{itemize}
\item {Utilização:Ant.}
\end{itemize}
\begin{itemize}
\item {Proveniência:(Do germ. \textunderscore vehida\textunderscore )}
\end{itemize}
Direito de revindicta ou de vingança pessoal, que, por influência germânica, vigorou na península hispânica. Cf. Herculano, \textunderscore Hist. de Port.\textunderscore , IV, 385.
\section{Faido}
\begin{itemize}
\item {Grp. gram.:m.}
\end{itemize}
\begin{itemize}
\item {Utilização:Des.}
\end{itemize}
\begin{itemize}
\item {Proveniência:(T. da Índia port.)}
\end{itemize}
Aquillo que sobeja; resto.
\section{Faim}
\begin{itemize}
\item {Grp. gram.:m.}
\end{itemize}
\begin{itemize}
\item {Utilização:Des.}
\end{itemize}
Espadim.
Ferro agudo de lança e de outras armas.
\section{Faina}
\begin{itemize}
\item {Grp. gram.:f.}
\end{itemize}
\begin{itemize}
\item {Utilização:Ext.}
\end{itemize}
\begin{itemize}
\item {Utilização:Ant.}
\end{itemize}
Qualquer serviço, a bordo de um navio.
Trabalho aturado; lida.
Cortesia.
(Catalão \textunderscore fahena\textunderscore , do lat. \textunderscore facienda\textunderscore )
\section{Faio}
\begin{itemize}
\item {Grp. gram.:m.}
\end{itemize}
\begin{itemize}
\item {Utilização:Prov.}
\end{itemize}
\begin{itemize}
\item {Utilização:Ant.}
\end{itemize}
Trigo degenerado.
\section{Faisão}
\begin{itemize}
\item {Grp. gram.:m.}
\end{itemize}
\begin{itemize}
\item {Proveniência:(Do lat. \textunderscore phasianus\textunderscore )}
\end{itemize}
Ave gallinácea, notável pela sua bella plumagem.
\section{Faísca}
\begin{itemize}
\item {Grp. gram.:f.}
\end{itemize}
\begin{itemize}
\item {Utilização:Prov.}
\end{itemize}
\begin{itemize}
\item {Utilização:trasm.}
\end{itemize}
\begin{itemize}
\item {Utilização:Prov.}
\end{itemize}
\begin{itemize}
\item {Utilização:trasm.}
\end{itemize}
\begin{itemize}
\item {Utilização:Fig.}
\end{itemize}
\begin{itemize}
\item {Proveniência:(Do lat. hyp. \textunderscore favisca\textunderscore , por \textunderscore favilla\textunderscore ?)}
\end{itemize}
O mesmo que \textunderscore centelha\textunderscore .
Phenómeno luminoso, resultante da combinação de electricidades oppostas.
Raio; corisco.
Palheta de oiro, que se apanha na terra ou areia de mina lavrada.
Pessôa bem posta.
O mesmo que \textunderscore caspa\textunderscore .
Aquillo que scintilla ou brilha muito.
Aquillo que póde communicar fogo, produzir calamidade, grande alarma, etc.
\section{Faiscação}
\begin{itemize}
\item {fónica:fa-is}
\end{itemize}
\begin{itemize}
\item {Grp. gram.:f.}
\end{itemize}
Acto de faiscar. Cf. Eça, \textunderscore P. Basílio\textunderscore , 66.
\section{Faiscador}
\begin{itemize}
\item {fónica:fa-is}
\end{itemize}
\begin{itemize}
\item {Grp. gram.:m.}
\end{itemize}
O mesmo que \textunderscore faisqueiro\textunderscore .
\section{Faiscante}
\begin{itemize}
\item {fónica:fa-is}
\end{itemize}
\begin{itemize}
\item {Grp. gram.:adj.}
\end{itemize}
Que faísca.
\section{Faiscar}
\begin{itemize}
\item {fónica:fa-is}
\end{itemize}
\begin{itemize}
\item {Grp. gram.:v. t.}
\end{itemize}
\begin{itemize}
\item {Grp. gram.:V. i.}
\end{itemize}
\begin{itemize}
\item {Proveniência:(De \textunderscore faísca\textunderscore )}
\end{itemize}
Emittir, lançar de si (centelhas, clarões, etc.); expellir como faíscas.
Lançar faíscas.
Scintillar; brilhar muito; deslumbrar: \textunderscore os olhos do Ruben faíscam\textunderscore .
Procurar faíscas de oiro, ou palhetas, em terra de minas lavradas.
\section{Faísca-velha}
\begin{itemize}
\item {Grp. gram.:f.}
\end{itemize}
\begin{itemize}
\item {Utilização:Prov.}
\end{itemize}
A mãe do diabo. (Colhido em Turquel)
\section{Faísco}
\begin{itemize}
\item {Grp. gram.:m.}
\end{itemize}
Variedade de prego, cuja cabeça tem proximamente a fórma de asa de mosca.
\section{Faisqueira}
\begin{itemize}
\item {fónica:fa-is}
\end{itemize}
\begin{itemize}
\item {Grp. gram.:f.}
\end{itemize}
\begin{itemize}
\item {Utilização:Bras. da Baía}
\end{itemize}
Mina ou lugar, donde se extrahem faíscas de oiro.
Cópula carnal.
\section{Faisqueiro}
\begin{itemize}
\item {fónica:fa-is}
\end{itemize}
\begin{itemize}
\item {Grp. gram.:m.}
\end{itemize}
\begin{itemize}
\item {Proveniência:(De \textunderscore faísca\textunderscore )}
\end{itemize}
Aquelle que procura nas minas faíscas de oiro.
\section{Faixa}
\begin{itemize}
\item {Grp. gram.:f.}
\end{itemize}
\begin{itemize}
\item {Utilização:Prov.}
\end{itemize}
\begin{itemize}
\item {Utilização:beir.}
\end{itemize}
\begin{itemize}
\item {Proveniência:(Do lat. \textunderscore fascia\textunderscore )}
\end{itemize}
Banda, cinta, cinto, correia, atadura, listra, tira.
Qualquer objecto em fórma de tira.
Friso chato, entre a architrave e a cornija.
Zona, em volta de um planeta.
Banda transversal num brasão.
Porção de terra estreita e longa; coirela.
Pequeno mólho (de palha de milho)
\section{Faixado}
\begin{itemize}
\item {Grp. gram.:m.}
\end{itemize}
\begin{itemize}
\item {Utilização:Heráld.}
\end{itemize}
Campo coberto de faixas de metal e côr. Cf. L. Ribeiro, \textunderscore Trat. de Armaria\textunderscore .
\section{Faixar}
\begin{itemize}
\item {Grp. gram.:v. t.}
\end{itemize}
(V.enfaixar)
\section{Faixeação}
\begin{itemize}
\item {Grp. gram.:f.}
\end{itemize}
\begin{itemize}
\item {Proveniência:(De \textunderscore faixear\textunderscore )}
\end{itemize}
O mesmo que \textunderscore fasciação\textunderscore .
\section{Faixear}
\begin{itemize}
\item {Grp. gram.:v.}
\end{itemize}
\begin{itemize}
\item {Utilização:t. Carp.}
\end{itemize}
Rodear com faixa de madeira.
\section{Faixeiro}
\begin{itemize}
\item {Grp. gram.:m.}
\end{itemize}
\begin{itemize}
\item {Utilização:Prov.}
\end{itemize}
\begin{itemize}
\item {Utilização:Prov.}
\end{itemize}
\begin{itemize}
\item {Utilização:trasm.}
\end{itemize}
Cueiro.
Tira de malha, que as meninas fazem com duas agulhas de meia, como aprendizado para outros trabalhos de malha, e que também se chama \textunderscore liga\textunderscore .
\section{Fajan}
\begin{itemize}
\item {Grp. gram.:f.}
\end{itemize}
\begin{itemize}
\item {Utilização:açor}
\end{itemize}
\begin{itemize}
\item {Utilização:Ant.}
\end{itemize}
Terra baixa e chan:«\textunderscore fajans virentes.\textunderscore »B. Pato, \textunderscore Paquita\textunderscore .
\section{Fajarda}
\begin{itemize}
\item {Grp. gram.:f.}
\end{itemize}
\begin{itemize}
\item {Utilização:Prov.}
\end{itemize}
\begin{itemize}
\item {Utilização:beir.}
\end{itemize}
Pequena propriedade rústica.
\section{Fajardice}
\begin{itemize}
\item {Grp. gram.:f.}
\end{itemize}
\begin{itemize}
\item {Utilização:Pop.}
\end{itemize}
Acto de fajardo.
Escamoteação; empalmação.
Lôgro, para furtar.
\section{Fajardo}
\begin{itemize}
\item {Grp. gram.:m.}
\end{itemize}
\begin{itemize}
\item {Utilização:Pop.}
\end{itemize}
\begin{itemize}
\item {Proveniência:(De \textunderscore Fajardo\textunderscore , appellido de um aventureiro portuense)}
\end{itemize}
Gatuno hábil.
Traficante; troca-tintas.
\section{Fajau}
\begin{itemize}
\item {Grp. gram.:m.}
\end{itemize}
Casta de uva minhota.
\section{Fajeca}
\begin{itemize}
\item {Grp. gram.:f.}
\end{itemize}
\begin{itemize}
\item {Utilização:Gír.}
\end{itemize}
Mêdo, cobardia.
\section{Fala}
\begin{itemize}
\item {Grp. gram.:f.}
\end{itemize}
Acção de falar.
Aquillo que se fala ou se exprime por palavras.
Voz.
Allocução; discurso.
Diálogo.
Estilo.
Timbre da voz.
Qualquer modo de exprimir uma ideia, não falando.
\section{Falação}
\begin{itemize}
\item {Grp. gram.:f.}
\end{itemize}
\begin{itemize}
\item {Utilização:Ant.}
\end{itemize}
O mesmo que \textunderscore falamento\textunderscore .
\section{Falace}
\begin{itemize}
\item {Grp. gram.:adj.}
\end{itemize}
(V.falaz)
\section{Falacha}
\begin{itemize}
\item {Grp. gram.:f.}
\end{itemize}
\begin{itemize}
\item {Utilização:Prov.}
\end{itemize}
\begin{itemize}
\item {Utilização:minh.}
\end{itemize}
Bolo de massa de castanhas piladas.
\section{Falácia}
\begin{itemize}
\item {Grp. gram.:f.}
\end{itemize}
\begin{itemize}
\item {Proveniência:(De \textunderscore falar\textunderscore )}
\end{itemize}
Ruído de vozes; falatório.
\section{Falácia}
\begin{itemize}
\item {Grp. gram.:f.}
\end{itemize}
\begin{itemize}
\item {Proveniência:(Lat. \textunderscore fallacia\textunderscore )}
\end{itemize}
Qualidade daquele ou daquilo que é falaz.
\section{Falaciloquência}
\begin{itemize}
\item {fónica:cu-en}
\end{itemize}
\begin{itemize}
\item {Grp. gram.:f.}
\end{itemize}
\begin{itemize}
\item {Proveniência:(Lat. \textunderscore fallaciloquentia\textunderscore )}
\end{itemize}
Linguagem cheia de falsidade.
\section{Falaciloquente}
\begin{itemize}
\item {fónica:cu-en}
\end{itemize}
\begin{itemize}
\item {Grp. gram.:adj.}
\end{itemize}
Que tem falaciloquência.
\section{Falacioso}
\begin{itemize}
\item {Grp. gram.:adj.}
\end{itemize}
Que tem falácia; palrador. Cf. Camillo, \textunderscore Quéda de um Anjo\textunderscore , 75.
Que usa falácia.
Burlão. Cf. Camillo, \textunderscore Quéda\textunderscore , 75.
\section{Falada}
\begin{itemize}
\item {Grp. gram.:f.}
\end{itemize}
\begin{itemize}
\item {Utilização:Des.}
\end{itemize}
\begin{itemize}
\item {Proveniência:(De \textunderscore falado\textunderscore )}
\end{itemize}
Falácia; murmuração.
Aquillo que produz escândalo.
Fala, discurso. Cf. Filinto, VIII, 145.
\section{Faladar}
\begin{itemize}
\item {Grp. gram.:v. i.}
\end{itemize}
\begin{itemize}
\item {Utilização:Burl.}
\end{itemize}
Falar muito, á tôa.
\section{Faladeira}
\begin{itemize}
\item {Grp. gram.:f.}
\end{itemize}
\begin{itemize}
\item {Proveniência:(De \textunderscore falar\textunderscore )}
\end{itemize}
Mulher faladora.
\section{Falador}
\begin{itemize}
\item {Grp. gram.:m.}
\end{itemize}
Aquelle que fala muito.
\section{Falagueiro}
\begin{itemize}
\item {Grp. gram.:adj.}
\end{itemize}
\begin{itemize}
\item {Utilização:Ant.}
\end{itemize}
O mesmo que \textunderscore fagueiro\textunderscore . Cf. \textunderscore Cancioneiro da Vaticana\textunderscore .
\section{Falamento}
\begin{itemize}
\item {Grp. gram.:m.}
\end{itemize}
\begin{itemize}
\item {Utilização:Des.}
\end{itemize}
\begin{itemize}
\item {Proveniência:(De \textunderscore falar\textunderscore )}
\end{itemize}
Fala; discurso.
\section{Falança}
\begin{itemize}
\item {Grp. gram.:f.}
\end{itemize}
\begin{itemize}
\item {Utilização:Ant.}
\end{itemize}
O mesmo que \textunderscore falamento\textunderscore .
\section{Falante}
\begin{itemize}
\item {Grp. gram.:adj.}
\end{itemize}
Que fala ou que imita a voz humana: \textunderscore no tempo dos animaes falantes...\textunderscore 
\section{Falar}
\begin{itemize}
\item {Grp. gram.:v. t.}
\end{itemize}
\begin{itemize}
\item {Grp. gram.:V. i.}
\end{itemize}
\begin{itemize}
\item {Utilização:Pop.}
\end{itemize}
\begin{itemize}
\item {Grp. gram.:Loc.}
\end{itemize}
\begin{itemize}
\item {Utilização:fam.}
\end{itemize}
\begin{itemize}
\item {Proveniência:(Do lat. \textunderscore fabulari\textunderscore )}
\end{itemize}
Significar por palavras; dizer; proferir: \textunderscore falar verdades\textunderscore .
Combinar.
Articular palavras.
Conversar.
Fazer discurso.
Referir-se: \textunderscore falar de coisas antigas\textunderscore .
Exprimir alguma idéia, sem articular palavras: \textunderscore os seus olhos falam\textunderscore .
Sêr expressivo, não falando.
Advogar.
Conversar mentalmente.
Têr relações amorosas; namorar: \textunderscore a criada fala com o padeiro\textunderscore .
\textunderscore Falar de papo\textunderscore , falar com entono, com prosápia.
\section{Falárica}
\begin{itemize}
\item {Grp. gram.:f.}
\end{itemize}
\begin{itemize}
\item {Utilização:Ant.}
\end{itemize}
\begin{itemize}
\item {Proveniência:(Lat. \textunderscore falarica\textunderscore )}
\end{itemize}
Espécie de lança, que tinha na ponta estôpa inflammável.
\section{Falario}
\begin{itemize}
\item {Grp. gram.:m.}
\end{itemize}
\begin{itemize}
\item {Proveniência:(De \textunderscore falar\textunderscore )}
\end{itemize}
Falácia, falatório:«\textunderscore ...a ouvir o falario das praças.\textunderscore »Camillo, \textunderscore Filha do Regic.\textunderscore , 193.
\section{Falatório}
\begin{itemize}
\item {Grp. gram.:m.}
\end{itemize}
\begin{itemize}
\item {Proveniência:(De \textunderscore falar\textunderscore )}
\end{itemize}
Ruído de muitas vozes; sussurro de muitas pessôas, que falam simultaneamente.
Cavaqueira.
Murmuração.
Locutório.
\section{Falaz}
\begin{itemize}
\item {Grp. gram.:adj.}
\end{itemize}
\begin{itemize}
\item {Proveniência:(Lat. \textunderscore fallax\textunderscore )}
\end{itemize}
Ardiloso; fraudulento; enganador; enganoso.
\section{Falazar}
\begin{itemize}
\item {Grp. gram.:v. i.}
\end{itemize}
\begin{itemize}
\item {Utilização:Fam.}
\end{itemize}
O mesmo que \textunderscore faladar\textunderscore .
\section{Falbalás}
\begin{itemize}
\item {Grp. gram.:m. pl.}
\end{itemize}
\begin{itemize}
\item {Utilização:Des.}
\end{itemize}
O mesmo que \textunderscore falvalás\textunderscore .
\section{Falca}
\begin{itemize}
\item {Grp. gram.:f.}
\end{itemize}
Tôro de madeira falquejado, com quatro faces rectangulares.
Espécie de porta, no bordo da embarcação.
Tabuões de reparo, unidos pelas taleiras, em artilharia.
(B. lat. \textunderscore falca\textunderscore )
\section{Falca}
\begin{itemize}
\item {Grp. gram.:f.}
\end{itemize}
\begin{itemize}
\item {Utilização:Prov.}
\end{itemize}
\begin{itemize}
\item {Utilização:alent.}
\end{itemize}
Acto de pedir esmola.
\section{Falcada}
\begin{itemize}
\item {Grp. gram.:f.}
\end{itemize}
\begin{itemize}
\item {Utilização:Prov.}
\end{itemize}
\begin{itemize}
\item {Utilização:trasm.}
\end{itemize}
Cada uma das duas phases da Lua, entre os quartos e o plenilúnio.
\section{Falcado}
\begin{itemize}
\item {Grp. gram.:adj.}
\end{itemize}
\begin{itemize}
\item {Proveniência:(Do lat. \textunderscore falcatus\textunderscore )}
\end{itemize}
O mesmo que \textunderscore foiciforme\textunderscore .
\section{Falcão}
\begin{itemize}
\item {Grp. gram.:m.}
\end{itemize}
\begin{itemize}
\item {Proveniência:(Lat. \textunderscore falco\textunderscore )}
\end{itemize}
Ave de rapina.
Antiga peça de artilharia.
\section{Falcar}
\begin{itemize}
\item {Grp. gram.:v. t.}
\end{itemize}
O mesmo que \textunderscore falquear\textunderscore . Cf. \textunderscore Vir. Trág.\textunderscore , XI, 25.
\section{Falcassa}
\begin{itemize}
\item {Grp. gram.:f.}
\end{itemize}
\begin{itemize}
\item {Utilização:Náut.}
\end{itemize}
Ligadura, que se faz na ponta de um cabo, para que êste se não desfie.
O mesmo que \textunderscore falcassadura\textunderscore .
\section{Falcassadura}
\begin{itemize}
\item {Grp. gram.:f.}
\end{itemize}
Acto de falcassar.
\section{Falcassar}
\begin{itemize}
\item {Grp. gram.:v.}
\end{itemize}
\begin{itemize}
\item {Utilização:t. Náut.}
\end{itemize}
Fazer botões com o fio de vela nos chicotes de (cabos náuticos), para que estes se não descochem ou desmanchem.
\section{Falcata}
\begin{itemize}
\item {Grp. gram.:f.}
\end{itemize}
Arma antiga, formada de uma haste, encimada por uma espécie de foice.
(Cp. \textunderscore falcato\textunderscore )
\section{Falcato}
\begin{itemize}
\item {Grp. gram.:adj.}
\end{itemize}
\begin{itemize}
\item {Utilização:Ant.}
\end{itemize}
\begin{itemize}
\item {Proveniência:(Lat. \textunderscore falcatus\textunderscore )}
\end{itemize}
Armado de foice ou de foices.
Que tem fórma de foice; falcular.
\section{Falcatrua}
\begin{itemize}
\item {Grp. gram.:f.}
\end{itemize}
Ardil; artifício para enganar; fraude.
\section{Falcatruar}
\begin{itemize}
\item {Grp. gram.:v. t.}
\end{itemize}
Fazer falcatrua a.
\section{Falcídia}
\begin{itemize}
\item {Grp. gram.:f.  e  adj. f.}
\end{itemize}
\begin{itemize}
\item {Utilização:Jur.}
\end{itemize}
Diz-se de uma lei romana sôbre successões, devida ao tribuno P. Falcídio.
\section{Falcífero}
\begin{itemize}
\item {Grp. gram.:adj.}
\end{itemize}
\begin{itemize}
\item {Proveniência:(Lat. \textunderscore falcifer\textunderscore )}
\end{itemize}
Armado de foice.
\section{Falcifoliado}
\begin{itemize}
\item {Grp. gram.:adj.}
\end{itemize}
\begin{itemize}
\item {Utilização:Bot.}
\end{itemize}
\begin{itemize}
\item {Proveniência:(Do lat. \textunderscore falx\textunderscore  + \textunderscore folium\textunderscore )}
\end{itemize}
Que tem fôlhas em fórma de foice.
\section{Falciforme}
\begin{itemize}
\item {Grp. gram.:adj.}
\end{itemize}
\begin{itemize}
\item {Proveniência:(Do lat. \textunderscore falx\textunderscore  + \textunderscore forma\textunderscore )}
\end{itemize}
Que tem fórma de foice.
\section{Falcípede}
\begin{itemize}
\item {Grp. gram.:adj.}
\end{itemize}
\begin{itemize}
\item {Utilização:Zool.}
\end{itemize}
\begin{itemize}
\item {Proveniência:(Do lat. \textunderscore falcipedius\textunderscore )}
\end{itemize}
Que tem pés curvos, em fórma de foice.
\section{Falcirostros}
\begin{itemize}
\item {fónica:cirrós}
\end{itemize}
\begin{itemize}
\item {Grp. gram.:m. pl.}
\end{itemize}
\begin{itemize}
\item {Proveniência:(Do lat. \textunderscore falx\textunderscore  + \textunderscore rostrum\textunderscore )}
\end{itemize}
Família de aves pernaltas, também conhecidas por \textunderscore cultirostros\textunderscore .
\section{Falcirrostros}
\begin{itemize}
\item {Grp. gram.:m. pl.}
\end{itemize}
\begin{itemize}
\item {Proveniência:(Do lat. \textunderscore falx\textunderscore  + \textunderscore rostrum\textunderscore )}
\end{itemize}
Família de aves pernaltas, também conhecidas por \textunderscore cultirrostros\textunderscore .
\section{Falcoada}
\begin{itemize}
\item {Grp. gram.:f.}
\end{itemize}
\begin{itemize}
\item {Proveniência:(De \textunderscore falcão\textunderscore )}
\end{itemize}
Bando de falcões.
Tiro da peça que se chamava falcão.
\section{Falcoado}
\begin{itemize}
\item {Grp. gram.:adj.}
\end{itemize}
Perseguido por falcão.
\section{Falcoaria}
\begin{itemize}
\item {Grp. gram.:f.}
\end{itemize}
Arte de preparar falcões para a caça.
Caçada com falcões.
Lugar, em que se criam falcões.
\section{Falcoeira}
\begin{itemize}
\item {Grp. gram.:f.}
\end{itemize}
Espécie de gaivota, (\textunderscore larus argentatus\textunderscore , Brehm).
\section{Falcoeiro}
\begin{itemize}
\item {Grp. gram.:m.}
\end{itemize}
\begin{itemize}
\item {Proveniência:(Do b. lat. \textunderscore falconarius\textunderscore )}
\end{itemize}
Aquelle que trata de falcões.
\section{Falconete}
\begin{itemize}
\item {fónica:nê}
\end{itemize}
\begin{itemize}
\item {Grp. gram.:m.}
\end{itemize}
\begin{itemize}
\item {Utilização:Ant.}
\end{itemize}
\begin{itemize}
\item {Proveniência:(De \textunderscore falcão\textunderscore )}
\end{itemize}
Pequena peça de artilharia.
\section{Falconídeo}
\begin{itemize}
\item {Grp. gram.:adj.}
\end{itemize}
\begin{itemize}
\item {Grp. gram.:M. pl.}
\end{itemize}
\begin{itemize}
\item {Proveniência:(Do gr. \textunderscore phalkon\textunderscore  + \textunderscore eidos\textunderscore )}
\end{itemize}
Relativo a falcão.
Família de aves, que têm por typo o falcão.
\section{Falcular}
\begin{itemize}
\item {Grp. gram.:adj.}
\end{itemize}
\begin{itemize}
\item {Proveniência:(Do lat. \textunderscore falcula\textunderscore )}
\end{itemize}
Semelhante a foice.
\section{Falda}
\begin{itemize}
\item {Grp. gram.:f.}
\end{itemize}
O mesmo que \textunderscore fralda\textunderscore .
Sopé, abas (de um monte, de uma serra, etc.).
(B. lat. \textunderscore falda\textunderscore )
\section{Faldão}
\begin{itemize}
\item {Grp. gram.:m.}
\end{itemize}
\begin{itemize}
\item {Utilização:Ant.}
\end{itemize}
O mesmo que \textunderscore fraldão\textunderscore . Cf. Fern. Lopes, \textunderscore Chrón. de D. João I\textunderscore .
\section{Faldistório}
\begin{itemize}
\item {Grp. gram.:m.}
\end{itemize}
\begin{itemize}
\item {Proveniência:(It. \textunderscore faldistorio\textunderscore )}
\end{itemize}
Cadeira episcopal, sem espaldar, ao lado do altar-mór.
\section{Faldra}
\begin{itemize}
\item {Grp. gram.:f.}
\end{itemize}
\begin{itemize}
\item {Utilização:Prov.}
\end{itemize}
\begin{itemize}
\item {Utilização:trasm.}
\end{itemize}
\begin{itemize}
\item {Utilização:Ant.}
\end{itemize}
O mesmo que \textunderscore fralda\textunderscore .
\section{Falecer}
\begin{itemize}
\item {Grp. gram.:v. i.}
\end{itemize}
\begin{itemize}
\item {Proveniência:(Do lat. \textunderscore fallere\textunderscore )}
\end{itemize}
Falhar; escassear: \textunderscore falecem-lhe meios de vida\textunderscore .
Sêr insuficiente.
Morrer: \textunderscore faleceu-lhe o pai\textunderscore .
Têr carência.
\section{Falecido}
\begin{itemize}
\item {Grp. gram.:adj.}
\end{itemize}
\begin{itemize}
\item {Proveniência:(De \textunderscore falecer\textunderscore )}
\end{itemize}
Falho.
Que carece de alguma coisa.
Que faleceu; morto.
\section{Falecimento}
\begin{itemize}
\item {Grp. gram.:m.}
\end{itemize}
Acto de falecer.
Morte.
Falha.
Carência.
Privação.
Incapacidade.
\section{Falência}
\begin{itemize}
\item {Grp. gram.:f.}
\end{itemize}
\begin{itemize}
\item {Proveniência:(Do lat. \textunderscore fallere\textunderscore )}
\end{itemize}
Acto ou efeito de falir; quebra.
Falha.
Omissão; carência.
\section{Falênia}
\begin{itemize}
\item {Grp. gram.:f.}
\end{itemize}
\begin{itemize}
\item {Proveniência:(De \textunderscore Fallen\textunderscore , n. p.)}
\end{itemize}
Gênero de insectos dípteros.
\section{Falerno}
\begin{itemize}
\item {Grp. gram.:m.}
\end{itemize}
\begin{itemize}
\item {Utilização:Ext.}
\end{itemize}
\begin{itemize}
\item {Proveniência:(De \textunderscore Falerno\textunderscore , n. p.)}
\end{itemize}
Antigo vinho da Campânia.
Vinho bom, generoso.
\section{Falgoseiro}
\begin{itemize}
\item {Grp. gram.:adj.}
\end{itemize}
\begin{itemize}
\item {Utilização:Prov.}
\end{itemize}
\begin{itemize}
\item {Utilização:trasm.}
\end{itemize}
Meigo, carinhoso e alegre, (principalmente falando-se de crianças).
(Talvez por \textunderscore fagoseiro\textunderscore , que se relaciona com \textunderscore fagar\textunderscore )
\section{Falguer}
\begin{itemize}
\item {Grp. gram.:v. t.}
\end{itemize}
\begin{itemize}
\item {Utilização:Ant.}
\end{itemize}
Fazer trabalhar.
\section{Falha}
\begin{itemize}
\item {Grp. gram.:f.}
\end{itemize}
\begin{itemize}
\item {Utilização:Fam.}
\end{itemize}
\begin{itemize}
\item {Proveniência:(De \textunderscore falhar\textunderscore )}
\end{itemize}
Fenda.
Defeito.
Falta; omissão.
Lasca.
Fragmento.
Pequena parte, que se separa de um todo.
Aquillo que falta em alguma coisa.
Mania, bolha, pancada.
\section{Falhado}
\begin{itemize}
\item {Grp. gram.:adj.}
\end{itemize}
\begin{itemize}
\item {Proveniência:(De \textunderscore falhar\textunderscore )}
\end{itemize}
Rachado; fendido.
\section{Falhadura}
\begin{itemize}
\item {Grp. gram.:f.}
\end{itemize}
Falha ou bôca de vasilha esboicelada. Cf. \textunderscore Techn. Rur.\textunderscore , I, 279.
\section{Falhão}
\begin{itemize}
\item {Grp. gram.:m.}
\end{itemize}
\begin{itemize}
\item {Proveniência:(De \textunderscore falha\textunderscore )}
\end{itemize}
Tabuão; cada uma das poucas tábuas grossas, em que se póde serrar um tronco de madeira; pranchão.
Facheiro grosso ou reforçado.
\section{Falhar}
\begin{itemize}
\item {Grp. gram.:v. t.}
\end{itemize}
\begin{itemize}
\item {Grp. gram.:V. i.}
\end{itemize}
\begin{itemize}
\item {Proveniência:(Do lat. \textunderscore fallere\textunderscore )}
\end{itemize}
Fazer falhas em; lascar.
Fender.
Têr deminuição no pêso.
Faltar.
Negar fogo, não acertar, (falando-se da espingarda e do tiro).
Dar em falso.
Não se realizar: \textunderscore a conspiração falhou\textunderscore .
\section{Falhas}
\begin{itemize}
\item {Grp. gram.:f. pl.}
\end{itemize}
\begin{itemize}
\item {Utilização:Gír.}
\end{itemize}
\begin{itemize}
\item {Proveniência:(De \textunderscore falhar\textunderscore )}
\end{itemize}
Cartas de jogar.
Percentagem, que se abona a alguém que tem responsabilidades pecuniárias, para o indemnizar dos pequenos prejuízos ou quebras, inherentes a múltiplos e constantes pagamentos.
\section{Falheiro}
\begin{itemize}
\item {Grp. gram.:m.}
\end{itemize}
\begin{itemize}
\item {Proveniência:(De \textunderscore falho\textunderscore )}
\end{itemize}
Primeira tábua, que se separa de um toro ou tronco, quando êste se serra longitudinalmente em várias tábuas, e que é sempre falho na face externa.
\section{Falhipo}
\begin{itemize}
\item {Grp. gram.:m.}
\end{itemize}
\begin{itemize}
\item {Utilização:Prov.}
\end{itemize}
\begin{itemize}
\item {Utilização:trasm.}
\end{itemize}
Benairo, farrapo.
\section{Falho}
\begin{itemize}
\item {Grp. gram.:adj.}
\end{itemize}
\begin{itemize}
\item {Proveniência:(De \textunderscore falhar\textunderscore )}
\end{itemize}
Que tem falha.
Que não tem o pêso devido.
Que não teve effeito.
Que tem poucas cartas de um naipe.
Que tem falta de alguma coisa.
\section{Falhudo}
\begin{itemize}
\item {Grp. gram.:adj.}
\end{itemize}
\begin{itemize}
\item {Utilização:Prov.}
\end{itemize}
\begin{itemize}
\item {Utilização:alg.}
\end{itemize}
\begin{itemize}
\item {Proveniência:(De \textunderscore falha\textunderscore )}
\end{itemize}
Mal cheio; chocho: \textunderscore noz falhuda\textunderscore ; \textunderscore cabeça falhuda\textunderscore .
\section{Falibilidade}
\begin{itemize}
\item {Grp. gram.:f.}
\end{itemize}
Qualidade daquele ou daquilo que é falível.
\section{Falida}
\begin{itemize}
\item {Grp. gram.:f.}
\end{itemize}
\begin{itemize}
\item {Utilização:Des.}
\end{itemize}
Acto de falir.
Falência. Cf. Filinto, VIII, 74.
\section{Falido}
\begin{itemize}
\item {Grp. gram.:adj.}
\end{itemize}
\begin{itemize}
\item {Utilização:Prov.}
\end{itemize}
\begin{itemize}
\item {Grp. gram.:M.}
\end{itemize}
\begin{itemize}
\item {Proveniência:(De \textunderscore falir\textunderscore )}
\end{itemize}
Que faliu.
Falho.
Diz-se dos bens do negociante falido.
Chocho, (falando-se da castanha, da noz, etc.).
Aquele que faliu.
\section{Falija}
\begin{itemize}
\item {Grp. gram.:f.}
\end{itemize}
\begin{itemize}
\item {Utilização:Ant.}
\end{itemize}
Espécie de espada antiga, estreita e leve.
\section{Falimento}
\begin{itemize}
\item {Grp. gram.:m.}
\end{itemize}
\begin{itemize}
\item {Utilização:Des.}
\end{itemize}
\begin{itemize}
\item {Proveniência:(De \textunderscore falir\textunderscore )}
\end{itemize}
Falha.
Êrro.
Omissão.
Falência.
Desmérito.
\section{Falinha}
\begin{itemize}
\item {Grp. gram.:f.}
\end{itemize}
\begin{itemize}
\item {Proveniência:(De \textunderscore fala\textunderscore )}
\end{itemize}
Voz aguda, pouco intensa e desagradável.
\section{Falipa}
\begin{itemize}
\item {Grp. gram.:f.}
\end{itemize}
\begin{itemize}
\item {Utilização:Ant.}
\end{itemize}
(V.pellica)
\section{Falir}
\begin{itemize}
\item {Grp. gram.:v. i.}
\end{itemize}
Faltar.
Minguar.
Desfalecer.
Suspender pagamentos e faltar aos compromissos comerciaes.
Ter falta ou míngua:«\textunderscore Bárbara sentiu-se falir de coragem\textunderscore ». Camillo, \textunderscore Freira no Subterr.\textunderscore , 93.
\section{Falisco}
\begin{itemize}
\item {Grp. gram.:m.}
\end{itemize}
Espécie de paio, inventado pelos Faliscos.
Dialecto dos Faliscos.
\section{Faliscos}
\begin{itemize}
\item {Grp. gram.:m. pl.}
\end{itemize}
\begin{itemize}
\item {Proveniência:(Lat. \textunderscore falisci\textunderscore )}
\end{itemize}
Antigos povos da Etrúria.
\section{Falível}
\begin{itemize}
\item {Grp. gram.:adj.}
\end{itemize}
\begin{itemize}
\item {Proveniência:(De \textunderscore falir\textunderscore )}
\end{itemize}
Que se póde enganar.
Em que póde haver êrro.
Que póde falhar.
\section{Falivelmente}
\begin{itemize}
\item {Grp. gram.:adv.}
\end{itemize}
De modo falível.
\section{Fallace}
\begin{itemize}
\item {Grp. gram.:adj.}
\end{itemize}
(V.fallaz)
\section{Fallácia}
\begin{itemize}
\item {Grp. gram.:f.}
\end{itemize}
\begin{itemize}
\item {Proveniência:(Lat. \textunderscore fallacia\textunderscore )}
\end{itemize}
Qualidade daquelle ou daquillo que é fallaz.
\section{Fallaciloquência}
\begin{itemize}
\item {fónica:cu-en}
\end{itemize}
\begin{itemize}
\item {Grp. gram.:f.}
\end{itemize}
\begin{itemize}
\item {Proveniência:(Lat. \textunderscore fallaciloquentia\textunderscore )}
\end{itemize}
Linguagem cheia de falsidade.
\section{Fallaciloquente}
\begin{itemize}
\item {fónica:cu-en}
\end{itemize}
\begin{itemize}
\item {Grp. gram.:adj.}
\end{itemize}
Que tem fallaciloquência.
\section{Fallacioso}
\begin{itemize}
\item {Grp. gram.:adj.}
\end{itemize}
Que usa fallácia.
Burlão. Cf. Camillo, \textunderscore Quéda\textunderscore , 75.
\section{Fallaz}
\begin{itemize}
\item {Grp. gram.:adj.}
\end{itemize}
\begin{itemize}
\item {Proveniência:(Lat. \textunderscore fallax\textunderscore )}
\end{itemize}
Ardiloso; fraudulento; enganador; enganoso.
\section{Fallecer}
\begin{itemize}
\item {Grp. gram.:v. i.}
\end{itemize}
\begin{itemize}
\item {Proveniência:(Do lat. \textunderscore fallere\textunderscore )}
\end{itemize}
Falhar; escassear: \textunderscore fallecem-lhe meios de vida\textunderscore .
Sêr insufficiente.
Morrer: \textunderscore falleceu-lhe o pai\textunderscore .
Têr carência.
\section{Fallecido}
\begin{itemize}
\item {Grp. gram.:adj.}
\end{itemize}
\begin{itemize}
\item {Proveniência:(De \textunderscore fallecer\textunderscore )}
\end{itemize}
Falho.
Que carece de alguma coisa.
Que falleceu; morto.
\section{Fallecimento}
\begin{itemize}
\item {Grp. gram.:m.}
\end{itemize}
Acto de fallecer.
Morte.
Falha.
Carência.
Privação.
Incapacidade.
\section{Fallência}
\begin{itemize}
\item {Grp. gram.:f.}
\end{itemize}
\begin{itemize}
\item {Proveniência:(Do lat. \textunderscore fallere\textunderscore )}
\end{itemize}
Acto ou effeito de fallir; quebra.
Falha.
Omissão; carência.
\section{Fallênia}
\begin{itemize}
\item {Grp. gram.:f.}
\end{itemize}
\begin{itemize}
\item {Proveniência:(De \textunderscore Fallen\textunderscore , n. p.)}
\end{itemize}
Gênero de insectos dípteros.
\section{Fallibilidade}
\begin{itemize}
\item {Grp. gram.:f.}
\end{itemize}
Qualidade daquelle ou daquillo que é fallível.
\section{Fallida}
\begin{itemize}
\item {Grp. gram.:f.}
\end{itemize}
\begin{itemize}
\item {Utilização:Des.}
\end{itemize}
Acto de fallir.
Fallência. Cf. Filinto, VIII, 74.
\section{Fallido}
\begin{itemize}
\item {Grp. gram.:adj.}
\end{itemize}
\begin{itemize}
\item {Utilização:Prov.}
\end{itemize}
\begin{itemize}
\item {Grp. gram.:M.}
\end{itemize}
\begin{itemize}
\item {Proveniência:(De \textunderscore fallir\textunderscore )}
\end{itemize}
Que falliu.
Falho.
Diz-se dos bens do negociante fallido.
Chocho, (falando-se da castanha, da noz, etc.).
Aquelle que falliu.
\section{Fallimento}
\begin{itemize}
\item {Grp. gram.:m.}
\end{itemize}
\begin{itemize}
\item {Utilização:Des.}
\end{itemize}
\begin{itemize}
\item {Proveniência:(De \textunderscore fallir\textunderscore )}
\end{itemize}
Falha.
Êrro.
Omissão.
Fallência.
Desmérito.
\section{Fallir}
\begin{itemize}
\item {Grp. gram.:v. i.}
\end{itemize}
Faltar.
Minguar.
Desfallecer.
Suspender pagamentos e faltar aos compromissos commerciaes.
Ter falta ou míngua:«\textunderscore Bárbara sentiu-se fallir de coragem\textunderscore ». Camillo, \textunderscore Freira no Subterr.\textunderscore , 93.
\section{Fallível}
\begin{itemize}
\item {Grp. gram.:adj.}
\end{itemize}
\begin{itemize}
\item {Proveniência:(De \textunderscore fallir\textunderscore )}
\end{itemize}
Que se póde enganar.
Em que póde haver êrro.
Que póde falhar.
\section{Fallivelmente}
\begin{itemize}
\item {Grp. gram.:adv.}
\end{itemize}
De modo fallível.
\section{Falmega}
\begin{itemize}
\item {Grp. gram.:f.}
\end{itemize}
\begin{itemize}
\item {Utilização:Prov.}
\end{itemize}
\begin{itemize}
\item {Utilização:minh.}
\end{itemize}
O mesmo que \textunderscore fagulha\textunderscore .
\section{Falocar}
\begin{itemize}
\item {Grp. gram.:v. i.}
\end{itemize}
\begin{itemize}
\item {Utilização:Prov.}
\end{itemize}
\begin{itemize}
\item {Utilização:alg.}
\end{itemize}
\begin{itemize}
\item {Proveniência:(De \textunderscore falar\textunderscore )}
\end{itemize}
Falar muito.
\section{Falperrista}
\begin{itemize}
\item {Grp. gram.:m.}
\end{itemize}
\begin{itemize}
\item {Proveniência:(De \textunderscore Falperra\textunderscore , n. p.)}
\end{itemize}
Ladrão de estrada. Cf. Arn. Gama, \textunderscore Segr. do Abb.\textunderscore , 332.
\section{Falpórria}
\begin{itemize}
\item {Grp. gram.:m.}
\end{itemize}
\begin{itemize}
\item {Utilização:Pop.}
\end{itemize}
Farsola; birbante.
\section{Falpórrias}
\begin{itemize}
\item {Grp. gram.:m.}
\end{itemize}
\begin{itemize}
\item {Utilização:Pop.}
\end{itemize}
Farsola; birbante.
\section{Falporrice}
\begin{itemize}
\item {Grp. gram.:f.}
\end{itemize}
Acção ou dito de falpórria.
\section{Falqueador}
\begin{itemize}
\item {Grp. gram.:m.}
\end{itemize}
Aquelle que falqueia.
\section{Falqueadura}
\begin{itemize}
\item {Grp. gram.:f.}
\end{itemize}
Acto de falquear.
\section{Falquear}
\begin{itemize}
\item {Grp. gram.:v. t.}
\end{itemize}
\begin{itemize}
\item {Proveniência:(De \textunderscore falca\textunderscore )}
\end{itemize}
Desbastar (um tronco de madeira).
Tornar quadrado, esquadriar, com machada ou machado.
Acunhar.
\section{Falquejamento}
\begin{itemize}
\item {Grp. gram.:m.}
\end{itemize}
Acto de falquejar.
\section{Falquejar}
\textunderscore v. t.\textunderscore  (e der.)
O mesmo que \textunderscore falquear\textunderscore , etc.
\section{Falqueta}
\begin{itemize}
\item {fónica:quê}
\end{itemize}
\begin{itemize}
\item {Grp. gram.:f.}
\end{itemize}
Acto de impellir uma bola por cima de outra, no jôgo do bilhar.
(Relaciona-se com \textunderscore falca\textunderscore ?)
\section{Falquito}
\begin{itemize}
\item {Grp. gram.:m.}
\end{itemize}
\begin{itemize}
\item {Utilização:Náut.}
\end{itemize}
Tábua volante, que se põe sôbre a falca, ou se sobrepõe á borda da embarcação.
(Cp. \textunderscore falca\textunderscore ^1)
\section{Falripas}
\begin{itemize}
\item {Grp. gram.:f. pl.}
\end{itemize}
(V.farripas)
\section{Falsa}
\begin{itemize}
\item {Grp. gram.:f.}
\end{itemize}
\begin{itemize}
\item {Utilização:Mús.}
\end{itemize}
\begin{itemize}
\item {Proveniência:(De \textunderscore falso\textunderscore )}
\end{itemize}
Consonância froixa de um semi-tom.
Desafinação.
\section{Falsabraga}
\begin{itemize}
\item {Grp. gram.:f.}
\end{itemize}
Barbacan; parte inferior de muralha.
(Cast. \textunderscore falsabraga\textunderscore )
\section{Falsador}
\begin{itemize}
\item {Grp. gram.:m.}
\end{itemize}
Aquelle que falsa.
\section{Falsamente}
\begin{itemize}
\item {Grp. gram.:adv.}
\end{itemize}
\begin{itemize}
\item {Proveniência:(De \textunderscore falso\textunderscore )}
\end{itemize}
Com falsidade.
\section{Falsaquilha}
\begin{itemize}
\item {Grp. gram.:f.}
\end{itemize}
\begin{itemize}
\item {Proveniência:(De \textunderscore falso\textunderscore  + \textunderscore quilha\textunderscore )}
\end{itemize}
Resguardo de madeira na quilha do navio.
\section{Falsar}
\begin{itemize}
\item {Grp. gram.:v. t.}
\end{itemize}
\begin{itemize}
\item {Grp. gram.:V. i.}
\end{itemize}
\begin{itemize}
\item {Proveniência:(Lat. \textunderscore falsare\textunderscore )}
\end{itemize}
O mesmo que \textunderscore falsificar\textunderscore .
Enganar alguém em (pesos ou medidas).
Fender, abrir.
Baldar.
Mentir ou dizer falsidades.
Desafinar.
Começar a aluír-se, fender-se.
Falhar.
\section{Falsa-rédea}
\begin{itemize}
\item {Grp. gram.:f.}
\end{itemize}
Correia, que prende a cabeçada ao peitoral do cavallo.
\section{Falsário}
\begin{itemize}
\item {Grp. gram.:m.}
\end{itemize}
\begin{itemize}
\item {Proveniência:(Lat. \textunderscore falsarius\textunderscore )}
\end{itemize}
Falsificador.
Aquelle que falta a promessa ou juramentos.
\section{Falsário}
\begin{itemize}
\item {Grp. gram.:m.}
\end{itemize}
\begin{itemize}
\item {Utilização:Des.}
\end{itemize}
O mesmo que \textunderscore falsete\textunderscore .
\section{Falsa-verónica}
\begin{itemize}
\item {Grp. gram.:f.}
\end{itemize}
Planta escrofularínea, (\textunderscore linaria spuria\textunderscore ).
\section{Falseamento}
\begin{itemize}
\item {Grp. gram.:m.}
\end{itemize}
Acto ou effeito de falsear.
\section{Falsear}
\begin{itemize}
\item {Grp. gram.:v. t.}
\end{itemize}
\begin{itemize}
\item {Grp. gram.:V. i.}
\end{itemize}
Sêr falso para com; atraiçoar.
Baldar.
Desafinar.
\section{Falsete}
\begin{itemize}
\item {fónica:sê}
\end{itemize}
\begin{itemize}
\item {Grp. gram.:m.}
\end{itemize}
\begin{itemize}
\item {Utilização:Pop.}
\end{itemize}
\begin{itemize}
\item {Proveniência:(It. \textunderscore falsetto\textunderscore )}
\end{itemize}
Voz, com que se procura imitar o tiple.
Falinha; voz esganiçada.
\section{Falsetear}
\begin{itemize}
\item {Grp. gram.:v. t.}
\end{itemize}
Falar ou cantar em falsete.
\section{Falsia}
\begin{itemize}
\item {Grp. gram.:f.}
\end{itemize}
\begin{itemize}
\item {Utilização:Ant.}
\end{itemize}
\begin{itemize}
\item {Proveniência:(Lat. \textunderscore falsitas\textunderscore )}
\end{itemize}
Qualidade daquelle ou daquillo que é falso.
\section{Falsidade}
\begin{itemize}
\item {Grp. gram.:f.}
\end{itemize}
\begin{itemize}
\item {Proveniência:(Lat. \textunderscore falsitas\textunderscore )}
\end{itemize}
Qualidade daquelle ou daquillo que é falso.
\section{Falsídia}
\begin{itemize}
\item {Grp. gram.:f.}
\end{itemize}
\begin{itemize}
\item {Utilização:Pop.}
\end{itemize}
O mesmo que \textunderscore falsidade\textunderscore .
\section{Falsídico}
\begin{itemize}
\item {Grp. gram.:adj.}
\end{itemize}
\begin{itemize}
\item {Proveniência:(Lat. \textunderscore falsidicus\textunderscore )}
\end{itemize}
Mentiroso; que diz falsidades.
\section{Falsificação}
\begin{itemize}
\item {Grp. gram.:f.}
\end{itemize}
Acto ou effeito de falsificar.
\section{Falsificador}
\begin{itemize}
\item {Grp. gram.:m.  e  adj.}
\end{itemize}
O que falsifica.
\section{Falsificar}
\begin{itemize}
\item {Grp. gram.:v. t.}
\end{itemize}
\begin{itemize}
\item {Proveniência:(Lat. \textunderscore falsificare\textunderscore )}
\end{itemize}
Imitar ou alterar ardilosamente, fraudulentamente.
Contrafazer.
Adulterar (substâncias alimentícias, factos, documentos, etc.).
\section{Falsificável}
\begin{itemize}
\item {Grp. gram.:adj.}
\end{itemize}
Que se póde falsificar.
\section{Falsífico}
\begin{itemize}
\item {Grp. gram.:adj.}
\end{itemize}
\begin{itemize}
\item {Utilização:Des.}
\end{itemize}
\begin{itemize}
\item {Proveniência:(Lat. \textunderscore falsificus\textunderscore )}
\end{itemize}
Que comete falsidades.
\section{Falsinérveo}
\begin{itemize}
\item {Grp. gram.:adj.}
\end{itemize}
\begin{itemize}
\item {Utilização:Bot.}
\end{itemize}
\begin{itemize}
\item {Proveniência:(De \textunderscore falso\textunderscore  + \textunderscore nérveo\textunderscore )}
\end{itemize}
Diz-se das fôlhas de certos vegetaes, nas quaes se observam falsas nervuras.
\section{Falso}
\begin{itemize}
\item {Grp. gram.:adj.}
\end{itemize}
\begin{itemize}
\item {Grp. gram.:M.}
\end{itemize}
\begin{itemize}
\item {Utilização:Pop.}
\end{itemize}
\begin{itemize}
\item {Utilização:Gír.}
\end{itemize}
\begin{itemize}
\item {Utilização:Ant.}
\end{itemize}
\begin{itemize}
\item {Utilização:Gír.}
\end{itemize}
\begin{itemize}
\item {Grp. gram.:Loc. adv.}
\end{itemize}
\begin{itemize}
\item {Proveniência:(Lat. \textunderscore falsus\textunderscore )}
\end{itemize}
Opposto á verdade ou á realidade.
Em que há mentira, fingimento, dissimulação, traição ou deslealdade: \textunderscore juramento falso\textunderscore .
Que segue mau caminho, má direcção.
Infundado; errado; enganoso.
Inexacto.
Falsificado: \textunderscore documento falso\textunderscore .
Adulterado.
\textunderscore Fundo falso\textunderscore , fundo apparente ou dissimulado de uma caixa ou mala, que encobre um escaninho ou compartimento para esconderijo, e debaixo do qual está o verdadeiro fundo do móvel.
Sítio recôndito, numa casa ou móvel, para servir de esconderijo a pessôas ou coisas.
Buraco de fechadura.
Lenço.
\textunderscore Em falso\textunderscore , errando o passo, o movimento ou a pancada: \textunderscore bater em falso\textunderscore .
\section{Falso-açafrão}
\begin{itemize}
\item {Grp. gram.:m.}
\end{itemize}
\begin{itemize}
\item {Utilização:Bras}
\end{itemize}
O mesmo que \textunderscore cólchico\textunderscore .
\section{Falsura}
\begin{itemize}
\item {Grp. gram.:f.}
\end{itemize}
\begin{itemize}
\item {Utilização:Ant.}
\end{itemize}
Falsidade.
Ardil, trapaça.
\section{Falta}
\begin{itemize}
\item {Grp. gram.:f.}
\end{itemize}
Acto ou effeito de faltar.
Ausência; privação: \textunderscore sentir a falta de amigos\textunderscore .
Culpa, peccado: \textunderscore commeter faltas\textunderscore .
Imperfeição; falha.
\section{Faltar}
\begin{itemize}
\item {Grp. gram.:v. i.}
\end{itemize}
Não haver.
Não existir.
Não comparecer: \textunderscore faltar á assembleia\textunderscore .
Falhar.
Não auxiliar.
Sêr indispensável para completar um número ou uma coisa: \textunderscore faltam-me seis centavos para perfazer quatro escudos\textunderscore .
Fallecer.
Desapparecer.
Morrer.
Commeter faltas, delinquir. Cf. \textunderscore Luz e Calor\textunderscore , I.
(Relaciona-se com o lat. \textunderscore faltere\textunderscore , por um hypoth. \textunderscore faltus = falsus\textunderscore )
\section{Falto}
\begin{itemize}
\item {Grp. gram.:adj.}
\end{itemize}
\begin{itemize}
\item {Proveniência:(De \textunderscore faltar\textunderscore )}
\end{itemize}
Que carece de alguma coisa; necessitado.
Falho.
\section{Falua}
\begin{itemize}
\item {Grp. gram.:f.}
\end{itemize}
Embarcação do Tejo, semelhante ao bote, mas maior que êlle.
Fragata.
(Alter. de \textunderscore faluca\textunderscore )
\section{Faluca}
\begin{itemize}
\item {Grp. gram.:f.}
\end{itemize}
Embarcação costeira dos marroquinos.
(Do \textunderscore ár.\textunderscore )
\section{Falucho}
\begin{itemize}
\item {Grp. gram.:m.}
\end{itemize}
Embarcação costeira de vela latina, em uso no Mediterrâneo.
(Cast. \textunderscore falucho\textunderscore . Cp. \textunderscore faluca\textunderscore )
\section{Falueiro}
\begin{itemize}
\item {Grp. gram.:m.}
\end{itemize}
\begin{itemize}
\item {Grp. gram.:Adj.}
\end{itemize}
Aquelle que dirige uma falua.
Relativo a falua.
\section{Fálum}
\begin{itemize}
\item {Grp. gram.:m.}
\end{itemize}
\begin{itemize}
\item {Utilização:Geol.}
\end{itemize}
Um dos andares ou camadas, que constituem o terreno mioceno, e que é formado de destroços de conchas.
\section{Faluneira}
\begin{itemize}
\item {Grp. gram.:f.}
\end{itemize}
Mina de fálum.
\section{Falustria}
\begin{itemize}
\item {Grp. gram.:f.}
\end{itemize}
\begin{itemize}
\item {Utilização:Prov.}
\end{itemize}
\begin{itemize}
\item {Utilização:trasm.}
\end{itemize}
O mesmo que \textunderscore flostria\textunderscore .
\section{Faluz}
\begin{itemize}
\item {Grp. gram.:m.}
\end{itemize}
Antiga e pequena moéda de Ormuz.
\section{Falvalá}
\begin{itemize}
\item {Grp. gram.:m.}
\end{itemize}
Fôlho de sáia.
(Cast. \textunderscore falbalá\textunderscore )
\section{Falvalás}
\begin{itemize}
\item {Grp. gram.:m. pl.}
\end{itemize}
\begin{itemize}
\item {Utilização:Ant.}
\end{itemize}
Fôlho de sáia.
(Cast. \textunderscore falbalá\textunderscore )
\section{Fama}
\begin{itemize}
\item {Grp. gram.:f.}
\end{itemize}
\begin{itemize}
\item {Utilização:Ant.}
\end{itemize}
\begin{itemize}
\item {Grp. gram.:M.  e  adj.}
\end{itemize}
\begin{itemize}
\item {Utilização:Bras}
\end{itemize}
\begin{itemize}
\item {Proveniência:(Lat. \textunderscore fama\textunderscore )}
\end{itemize}
Opinião pública.
Voz geral.
Qualidade daquillo que é notório.
Reputação: \textunderscore homem de má fama\textunderscore .
Glória: \textunderscore a fama dos Albuquerques e Castros\textunderscore .
Notícia.
Homem famoso.
\section{Famaco}
\begin{itemize}
\item {Grp. gram.:adj.}
\end{itemize}
\begin{itemize}
\item {Utilização:Des.}
\end{itemize}
(V.faminto)
\section{Famacósio}
\begin{itemize}
\item {Grp. gram.:m.}
\end{itemize}
Espécie de gato bravo do Paraguai.
(Cast. \textunderscore famacosio\textunderscore )
\section{Famanaz}
\begin{itemize}
\item {Grp. gram.:adj.}
\end{itemize}
\begin{itemize}
\item {Utilização:Bras}
\end{itemize}
\begin{itemize}
\item {Proveniência:(De \textunderscore fama\textunderscore )}
\end{itemize}
Afamado por valor, proêzas ou influência.
\section{Fame}
\begin{itemize}
\item {Grp. gram.:f.}
\end{itemize}
\begin{itemize}
\item {Utilização:Ant.}
\end{itemize}
\begin{itemize}
\item {Proveniência:(Lat. \textunderscore fames\textunderscore )}
\end{itemize}
Fome.
\section{Famelga}
\begin{itemize}
\item {Grp. gram.:m.  e  f.}
\end{itemize}
\begin{itemize}
\item {Utilização:Pop.}
\end{itemize}
\begin{itemize}
\item {Proveniência:(De \textunderscore famélico\textunderscore )}
\end{itemize}
Pessôa franzina, com cara de fome.
\section{Famelgo}
\begin{itemize}
\item {Grp. gram.:m.}
\end{itemize}
\begin{itemize}
\item {Utilização:T. da Bairrada}
\end{itemize}
\begin{itemize}
\item {Utilização:fam.}
\end{itemize}
Sujeito finório, astuto.
\section{Famelguita}
\begin{itemize}
\item {Grp. gram.:m.  e  f.}
\end{itemize}
\begin{itemize}
\item {Utilização:Pop.}
\end{itemize}
\begin{itemize}
\item {Proveniência:(De \textunderscore famelga\textunderscore )}
\end{itemize}
Criança franzina, com cara de fome.
\section{Famélico}
\begin{itemize}
\item {Grp. gram.:adj.}
\end{itemize}
\begin{itemize}
\item {Proveniência:(Lat. \textunderscore famelicus\textunderscore )}
\end{itemize}
O mesmo que \textunderscore faminto\textunderscore .
\section{Famigerado}
\begin{itemize}
\item {Grp. gram.:adj.}
\end{itemize}
\begin{itemize}
\item {Proveniência:(Lat. \textunderscore famigeratus\textunderscore )}
\end{itemize}
Que tem fama.
Celebrado; célebre.
\section{Famigerador}
\begin{itemize}
\item {Grp. gram.:m.  e  adj.}
\end{itemize}
\begin{itemize}
\item {Proveniência:(Lat. \textunderscore famigerator\textunderscore )}
\end{itemize}
O que espalha fama.
\section{Famígero}
\begin{itemize}
\item {Grp. gram.:adj.}
\end{itemize}
\begin{itemize}
\item {Proveniência:(Do lat. \textunderscore fama\textunderscore  + \textunderscore gerere\textunderscore )}
\end{itemize}
O mesmo que \textunderscore famigerado\textunderscore .
\section{Família}
\begin{itemize}
\item {Grp. gram.:f.}
\end{itemize}
\begin{itemize}
\item {Utilização:Ant.}
\end{itemize}
\begin{itemize}
\item {Utilização:Prov.}
\end{itemize}
\begin{itemize}
\item {Utilização:alent.}
\end{itemize}
\begin{itemize}
\item {Utilização:Bras. de Minas}
\end{itemize}
\begin{itemize}
\item {Proveniência:(Lat. \textunderscore família\textunderscore )}
\end{itemize}
Pessôas, que vivem na mesma casa.
Pessôas do mesmo sangue, que vivem ou não em commum.
Descendência, linhagem.
Sectários de um systema.
Membros de uma corporação.
Agrupamento de gêneros ou tríbos de vegetaes ou animaes, ligados por caracteres communs.
Conjunto de vocábulos, que têm a mesma raíz.
Conjunto de escravos, pertencentes a um só indivíduo.
Multidão de gente.
Filho ou filha: \textunderscore tenho quatro famílias\textunderscore .
Filha: \textunderscore tenho cinco filhos, isto é, três moços e duas famílias\textunderscore .
\section{Familiairo}
\begin{itemize}
\item {Grp. gram.:m.}
\end{itemize}
\begin{itemize}
\item {Utilização:Ant.}
\end{itemize}
O mesmo que \textunderscore familiar\textunderscore .
\section{Familiar}
\begin{itemize}
\item {Grp. gram.:adj.}
\end{itemize}
\begin{itemize}
\item {Grp. gram.:M.}
\end{itemize}
\begin{itemize}
\item {Proveniência:(Lat. \textunderscore familiaris\textunderscore )}
\end{itemize}
Relativo á família.
Doméstico: \textunderscore despesas familiares\textunderscore .
Habitual.
Domesticado.
Simples; vulgar.
Pessôa de família.
Pessôa íntima.
Criado.
Confrade de communidade religiosa.
Funccionário da Inquisição.
\section{Familiária}
\begin{itemize}
\item {Grp. gram.:f.}
\end{itemize}
\begin{itemize}
\item {Utilização:Ant.}
\end{itemize}
\begin{itemize}
\item {Proveniência:(Do b. lat. \textunderscore familiarius\textunderscore )}
\end{itemize}
Criada, serva.
\section{Familiaridade}
\begin{itemize}
\item {Grp. gram.:f.}
\end{itemize}
\begin{itemize}
\item {Proveniência:(Lat. \textunderscore familiaritas\textunderscore )}
\end{itemize}
Qualidade daquelle ou daquillo que é familiar.
Confiança, franqueza: \textunderscore não tenho familiaridade com êlle\textunderscore .
\section{Familiarizar}
\begin{itemize}
\item {Grp. gram.:v. t.}
\end{itemize}
\begin{itemize}
\item {Grp. gram.:V. p.}
\end{itemize}
\begin{itemize}
\item {Proveniência:(De \textunderscore familiar\textunderscore )}
\end{itemize}
Tornar familiar.
Vulgarizar.
Habituar.
Relacionar-se.
Entrar no conhecimento corrente de alguma coisa: \textunderscore familiarizar-se com algumas línguas\textunderscore .
Acostumar-se; perder o receio.
\section{Familiarmente}
\begin{itemize}
\item {Grp. gram.:adv.}
\end{itemize}
De modo familiar.
\section{Familiatura}
\begin{itemize}
\item {Grp. gram.:f.}
\end{itemize}
\begin{itemize}
\item {Proveniência:(De \textunderscore familiar\textunderscore )}
\end{itemize}
Cargo ou título de familiar da Inquisição.
\section{Familismo}
\begin{itemize}
\item {Grp. gram.:m.}
\end{itemize}
\begin{itemize}
\item {Utilização:Bras}
\end{itemize}
Tudo que diz respeito á organização da família. Cf. Bevilaqua, \textunderscore Direito de Fam.\textunderscore 
\section{Familistério}
\begin{itemize}
\item {Grp. gram.:m.}
\end{itemize}
\begin{itemize}
\item {Proveniência:(De \textunderscore família\textunderscore . Cp. \textunderscore phalanstério\textunderscore )}
\end{itemize}
Instituição social ou estabelecimento de muitas famílias, segundo o systema de Fourier: \textunderscore haja vista o familistério, fundado em Guise e que abriga mil e seiscentas pessôas\textunderscore . Cf. Rev. \textunderscore Serões\textunderscore , III.
\section{Faminto}
\begin{itemize}
\item {Grp. gram.:adj.}
\end{itemize}
\begin{itemize}
\item {Utilização:Fig.}
\end{itemize}
\begin{itemize}
\item {Proveniência:(Do rad. do lat. \textunderscore fames\textunderscore )}
\end{itemize}
Que tem fome.
Esfomeado.
Que tem avidez, ansiedade, desejo ardente:«\textunderscore oh que famintos beijos na floresta!\textunderscore »\textunderscore Lusiadas\textunderscore , IX.
\section{Famosamente}
\begin{itemize}
\item {Grp. gram.:adv.}
\end{itemize}
De modo famoso.
\section{Famoso}
\begin{itemize}
\item {Grp. gram.:adj.}
\end{itemize}
\begin{itemize}
\item {Proveniência:(Lat. \textunderscore famosus\textunderscore )}
\end{itemize}
Que tem fama.
Que é muito conhecido.
Famigerado.
Célebre.
Extraordinário; excellente: \textunderscore um jantar famoso\textunderscore .
\section{Fâmula}
(\textunderscore fem.\textunderscore  de \textunderscore fâmulo\textunderscore )
\section{Famulagem}
\begin{itemize}
\item {Grp. gram.:f.}
\end{itemize}
\begin{itemize}
\item {Utilização:P. us.}
\end{itemize}
Conjunto de fâmulos.
Os fâmulos.
\section{Famular}
\begin{itemize}
\item {Grp. gram.:v. t.}
\end{itemize}
\begin{itemize}
\item {Utilização:P. us.}
\end{itemize}
\begin{itemize}
\item {Proveniência:(Lat. \textunderscore famulari\textunderscore )}
\end{itemize}
Servir como fâmulo.
\section{Famulatício}
\begin{itemize}
\item {Grp. gram.:adj.}
\end{itemize}
Famulatório.
Que desempenha as funcções de fâmulo.
\section{Famulato}
\begin{itemize}
\item {Grp. gram.:m.}
\end{itemize}
Qualidade ou funcções de fâmulo. Cf. Camillo, \textunderscore N. de Insômnia\textunderscore , I, 55.
\section{Famulatório}
\begin{itemize}
\item {Grp. gram.:adj.}
\end{itemize}
\begin{itemize}
\item {Proveniência:(Lat. \textunderscore famulatorius\textunderscore )}
\end{itemize}
Relativo a fâmulo.
\section{Famulento}
\begin{itemize}
\item {Grp. gram.:adj.}
\end{itemize}
\begin{itemize}
\item {Proveniência:(Do rad. de \textunderscore fame\textunderscore )}
\end{itemize}
O mesmo que \textunderscore faminto\textunderscore .
\section{Famulício}
\begin{itemize}
\item {Grp. gram.:m.}
\end{itemize}
\begin{itemize}
\item {Proveniência:(Lat. \textunderscore famulitium\textunderscore )}
\end{itemize}
Serviço de fâmulo; famulagem.
\section{Fâmulo}
\begin{itemize}
\item {Grp. gram.:m.}
\end{itemize}
\begin{itemize}
\item {Proveniência:(Lat. \textunderscore famulus\textunderscore )}
\end{itemize}
Servidor; criado.
Funccionário subalterno de algumas communidades religiosas.
Caudatário.
Pessôa, que acompanha os Prelados e desempenha certos serviços nos seminários ou na residência dos Bispos.
\section{Fanadoiro}
\begin{itemize}
\item {Grp. gram.:m.}
\end{itemize}
\begin{itemize}
\item {Utilização:Prov.}
\end{itemize}
Espátula grosseira, com que os oleiros alisam as superfícies dos seus artefactos.
\section{Fanadouro}
\begin{itemize}
\item {Grp. gram.:m.}
\end{itemize}
\begin{itemize}
\item {Utilização:Prov.}
\end{itemize}
Espátula grosseira, com que os oleiros alisam as superfícies dos seus artefactos.
\section{Fanaite}
\begin{itemize}
\item {Grp. gram.:m.}
\end{itemize}
\begin{itemize}
\item {Utilização:T. da Bairrada}
\end{itemize}
Momento, instante: \textunderscore muito diligente, a pequena varre a casa toda num fanate\textunderscore .
\section{Fanal}
\begin{itemize}
\item {Grp. gram.:m.}
\end{itemize}
\begin{itemize}
\item {Utilização:Fig.}
\end{itemize}
Facho.
Pharol.
Guia; norte: \textunderscore foi-lhe fanal a esperança\textunderscore .
(Cp. cast. \textunderscore fanal\textunderscore )
\section{Fanane}
\begin{itemize}
\item {Grp. gram.:m.}
\end{itemize}
\begin{itemize}
\item {Utilização:Ant.}
\end{itemize}
O mesmo que \textunderscore fanão\textunderscore .
\section{Fanão}
\begin{itemize}
\item {Grp. gram.:m.}
\end{itemize}
\begin{itemize}
\item {Proveniência:(Do ár. \textunderscore faunon\textunderscore ?)}
\end{itemize}
Antiga moéda indiana, anterior á conquista portuguesa.
\section{Fanar}
\begin{itemize}
\item {Grp. gram.:v. t.}
\end{itemize}
\begin{itemize}
\item {Utilização:Gal}
\end{itemize}
\begin{itemize}
\item {Utilização:Des.}
\end{itemize}
\begin{itemize}
\item {Utilização:Prov.}
\end{itemize}
\begin{itemize}
\item {Proveniência:(Fr. \textunderscore faner\textunderscore )}
\end{itemize}
Murchar.
Circuncidar.
Amputar.
Tirar um bocado de; encetar.
\section{Fanate}
\begin{itemize}
\item {Grp. gram.:m.}
\end{itemize}
\begin{itemize}
\item {Utilização:T. da Bairrada}
\end{itemize}
Momento, instante: \textunderscore muito diligente, a pequena varre a casa toda num fanate\textunderscore .
\section{Fanático}
\begin{itemize}
\item {Grp. gram.:m.  e  adj.}
\end{itemize}
\begin{itemize}
\item {Proveniência:(Lat. \textunderscore fanaticus\textunderscore )}
\end{itemize}
O que se julga inspirado por uma divindade qualquer.
O que tem fanatismo.
Que se apaixona demasiadamente por uma coisa ou pessôa.
Que tem extraordinário zêlo religioso.
Supersticiosamente religioso.
\section{Fanatismo}
\begin{itemize}
\item {Grp. gram.:m.}
\end{itemize}
Excessivo zêlo religioso.
Facciosismo partidário.
Adhesão cega a um systema ou doutrina.
Dedicação excessiva a alguém ou a alguma coisa; paixão.
(Cp. \textunderscore fanático\textunderscore )
\section{Fanatizador}
\begin{itemize}
\item {Grp. gram.:m.  e  adj.}
\end{itemize}
O que fanatiza.
\section{Fanatizar}
\begin{itemize}
\item {Grp. gram.:v. t.}
\end{itemize}
\begin{itemize}
\item {Proveniência:(De \textunderscore fanático\textunderscore )}
\end{itemize}
Tornar fanático.
Inspirar fanatismo ou extrema sympathia a.
\section{Fanca}
\begin{itemize}
\item {Grp. gram.:f.}
\end{itemize}
\begin{itemize}
\item {Utilização:Bras}
\end{itemize}
Conjunto de fazendas para vender.
Objectos de fancaria. Cf. B. C. Rubim, \textunderscore Vocab. Bras.\textunderscore , vb. \textunderscore mascate\textunderscore .
(Der. regressiva de \textunderscore fanqueiro\textunderscore )
\section{Fancaia}
\begin{itemize}
\item {Grp. gram.:f. Loc. adv.}
\end{itemize}
\begin{itemize}
\item {Utilização:Prov.}
\end{itemize}
\begin{itemize}
\item {Utilização:trasm.}
\end{itemize}
\textunderscore Á fancaia\textunderscore , desageitadamente, ás três pancadas, (especialmente alludindo-se a um chapéu, que se põe na cabeça, inclinando-o para o lado, ou pondo-o á banda).
(Cp. \textunderscore facaia\textunderscore )
\section{Fancaria}
\begin{itemize}
\item {Grp. gram.:f.}
\end{itemize}
Commércio de fanqueiros.
\textunderscore Obra de fancaria\textunderscore , trabalho grosseiro, feito á pressa, tendo-se apenas em vista o lucro.
(Cp. \textunderscore fanqueiro\textunderscore )
\section{Fanchona}
\begin{itemize}
\item {Grp. gram.:f.}
\end{itemize}
\begin{itemize}
\item {Utilização:Pop.}
\end{itemize}
Mulher robusta, de aspecto viril e de hábitos ou predilecções próprias do sexo masculino.
\section{Fanchonaça}
\begin{itemize}
\item {Grp. gram.:f.}
\end{itemize}
O mesmo que \textunderscore fanchona\textunderscore . Cf. Camillo, \textunderscore Myst. de Lisb.\textunderscore , II, 29.
\section{Fanchonice}
\begin{itemize}
\item {Grp. gram.:f.}
\end{itemize}
\begin{itemize}
\item {Utilização:Pop.}
\end{itemize}
\begin{itemize}
\item {Proveniência:(De \textunderscore fanchona\textunderscore )}
\end{itemize}
Qualidade de mulher robusta e airosa.
Qualidade de fanchonaça.
\section{Fanchonismo}
\begin{itemize}
\item {Grp. gram.:m.}
\end{itemize}
Qualidade ou hábito de fanchono.
\section{Fanchono}
\begin{itemize}
\item {Grp. gram.:m.}
\end{itemize}
\begin{itemize}
\item {Utilização:Pop.}
\end{itemize}
Homem lúbrico, que procura prazeres nos indivíduos do próprio sexo.
Aquelle que se presta aos prazeres sensuaes de indivíduo do seu sexo.
(Provavelmente, do it. \textunderscore fanciullo\textunderscore )
\section{Fandango}
\begin{itemize}
\item {Grp. gram.:m.}
\end{itemize}
\begin{itemize}
\item {Grp. gram.:Adj.}
\end{itemize}
Dança popular e mais ou menos licenciosa, em Espanha, Portugal e Brasil.
Música, que acompanha essa dança.
Ridículo, ordinário: \textunderscore tropa fandanga\textunderscore . \textunderscore A procissão ia muito desordenada, muito fandanga.\textunderscore 
(Cast. \textunderscore fandango\textunderscore )
\section{Fandangueiro}
\begin{itemize}
\item {Grp. gram.:m.}
\end{itemize}
\begin{itemize}
\item {Grp. gram.:Adj.}
\end{itemize}
Aquelle que dança o fandango.
Que gosta do fandango e de outras danças populares.
\section{Fandinga}
\begin{itemize}
\item {Grp. gram.:m.}
\end{itemize}
\begin{itemize}
\item {Utilização:Prov.}
\end{itemize}
\begin{itemize}
\item {Utilização:trasm.}
\end{itemize}
Maltrapilho.
Garoto.
Sujeito miserável.
\section{Fandingar}
\begin{itemize}
\item {Grp. gram.:v. t.}
\end{itemize}
\begin{itemize}
\item {Utilização:Prov.}
\end{itemize}
\begin{itemize}
\item {Utilização:trasm.}
\end{itemize}
Galantear.
Deshonestar.
(Cp. \textunderscore fandinga\textunderscore )
\section{Faneca}
\begin{itemize}
\item {Grp. gram.:f.}
\end{itemize}
\begin{itemize}
\item {Utilização:Pop.}
\end{itemize}
\begin{itemize}
\item {Grp. gram.:Adj.}
\end{itemize}
Pequeno peixe, da fam. dos gádidos, (\textunderscore gadus luscus\textunderscore ).
Castanha chocha.
Pedaço de pão, faneco.
Magro, sêco.
(Cast. \textunderscore faneca\textunderscore )
\section{Faneco}
\begin{itemize}
\item {Grp. gram.:adj.}
\end{itemize}
Fanado; murcho; chocho.
\section{Faneco}
\begin{itemize}
\item {Grp. gram.:m.}
\end{itemize}
\begin{itemize}
\item {Utilização:Bras}
\end{itemize}
\begin{itemize}
\item {Utilização:Ant.}
\end{itemize}
Pedaço, bocado.
Pedaço de pão.
\section{Fânega}
\begin{itemize}
\item {Grp. gram.:f.}
\end{itemize}
\begin{itemize}
\item {Utilização:P. us.}
\end{itemize}
O mesmo que \textunderscore fanga\textunderscore .
\section{Fanfa}
\begin{itemize}
\item {Grp. gram.:m.}
\end{itemize}
\begin{itemize}
\item {Utilização:Pop.}
\end{itemize}
O mesmo que \textunderscore fanfarrão\textunderscore .
(Ant. cast. \textunderscore fanfa\textunderscore , t. onom., segundo Diez)
\section{Fanfar}
\begin{itemize}
\item {Grp. gram.:v. i.}
\end{itemize}
O mesmo que \textunderscore fanfarrear\textunderscore .
Retrucar com insolência:«\textunderscore não me esteja ahi a fanfar, que já o não enxergo.\textunderscore »Camillo, \textunderscore Brasileira\textunderscore , 257.
\section{Fanfarra}
\begin{itemize}
\item {Grp. gram.:f.}
\end{itemize}
\begin{itemize}
\item {Utilização:Gír.}
\end{itemize}
\begin{itemize}
\item {Proveniência:(Fr. \textunderscore fanfare\textunderscore )}
\end{itemize}
Banda de músicos, com instrumentos de metal; charanga.
Fanfarrice.
Língua.
\section{Fanfarrão}
\begin{itemize}
\item {Grp. gram.:m.  e  adj.}
\end{itemize}
\begin{itemize}
\item {Proveniência:(De \textunderscore fanfarra\textunderscore )}
\end{itemize}
Aquelle que alardeia valentia, não a tendo; impostor.
\section{Fanfarrar}
\begin{itemize}
\item {Grp. gram.:v. i.}
\end{itemize}
\begin{itemize}
\item {Utilização:bras}
\end{itemize}
\begin{itemize}
\item {Utilização:Neol.}
\end{itemize}
Vibrar, soar, (falando-se de instrumentos de metal:«\textunderscore raramente fanfarram as notas do clarim\textunderscore ». Cf. Artagão, \textunderscore Mús. Sacra\textunderscore , 9.)
\section{Fanfarraria}
\begin{itemize}
\item {Grp. gram.:f.}
\end{itemize}
Acto ou qualidade de fanfarrão. Cf. \textunderscore Eufrosina\textunderscore , act. I, sc. 2.
\section{Fanfarrear}
\begin{itemize}
\item {Grp. gram.:v. i.}
\end{itemize}
\begin{itemize}
\item {Proveniência:(De \textunderscore fanfarra\textunderscore )}
\end{itemize}
Têr fanfarrice.
Bazofiar.
\section{Fanfarria}
\begin{itemize}
\item {Grp. gram.:f.}
\end{itemize}
\begin{itemize}
\item {Utilização:Des.}
\end{itemize}
O mesmo que \textunderscore fanfarraria\textunderscore .
\section{Fanfarrice}
\begin{itemize}
\item {Grp. gram.:f.}
\end{itemize}
\begin{itemize}
\item {Proveniência:(De \textunderscore fanfarra\textunderscore )}
\end{itemize}
Qualidade, dito ou acto de fanfarrão.
\section{Fanfarronada}
\begin{itemize}
\item {Grp. gram.:f.}
\end{itemize}
\begin{itemize}
\item {Proveniência:(De \textunderscore fanfarrão\textunderscore )}
\end{itemize}
Fanfarrice; bravata.
\section{Fanfarronal}
\begin{itemize}
\item {Grp. gram.:adj.}
\end{itemize}
Relativo a fanfarrão.
Próprio de fanfarrão.
\section{Fanfarronar}
\begin{itemize}
\item {Grp. gram.:v. i.}
\end{itemize}
\begin{itemize}
\item {Proveniência:(De \textunderscore fanfarrão\textunderscore )}
\end{itemize}
O mesmo que \textunderscore fanfarrear\textunderscore .
\section{Fanfarronice}
\begin{itemize}
\item {Grp. gram.:f.}
\end{itemize}
O mesmo que \textunderscore fanfarrice\textunderscore .
\section{Fanfúrria}
\begin{itemize}
\item {Grp. gram.:f.}
\end{itemize}
\begin{itemize}
\item {Utilização:Pop.}
\end{itemize}
O mesmo que \textunderscore fanfarrice\textunderscore .
\section{Fanfurrice}
\begin{itemize}
\item {Grp. gram.:f.}
\end{itemize}
O mesmo que \textunderscore fanfarrice\textunderscore .
\section{Fanga}
\begin{itemize}
\item {Grp. gram.:f.}
\end{itemize}
Antiga medida de cereaes, de sal, de carvão de pedra.
Casa ou lugar, em que se vendiam cereaes por estiva.
(Cast. \textunderscore fánega\textunderscore )
\section{Fangapena}
\begin{itemize}
\item {Grp. gram.:f.}
\end{itemize}
Palavra, erradamente introduzida nos diccionários portugueses, como significando instrumento, com que os indígenas do Maranhão canteiam a pedra.
Provavelmente, e até porque o tupi não tem a lêtra \textunderscore F\textunderscore , é substituição errónea de \textunderscore tangapema\textunderscore  ou, antes, \textunderscore itangapema\textunderscore , termo tupi, que quer dizer \textunderscore espada de ferro\textunderscore .
\section{Fangina}
\begin{itemize}
\item {Grp. gram.:f.}
\end{itemize}
\begin{itemize}
\item {Utilização:Ant.}
\end{itemize}
O mesmo que \textunderscore faxina\textunderscore .
\section{Fangueirada}
\begin{itemize}
\item {fónica:gu-ei}
\end{itemize}
\begin{itemize}
\item {Grp. gram.:f.}
\end{itemize}
\begin{itemize}
\item {Utilização:Prov.}
\end{itemize}
\begin{itemize}
\item {Utilização:alent.}
\end{itemize}
Pancada com fangueiro ou com bordão.
\section{Fangueiro}
\begin{itemize}
\item {fónica:gu-ei}
\end{itemize}
\begin{itemize}
\item {Grp. gram.:m.}
\end{itemize}
\begin{itemize}
\item {Utilização:Prov.}
\end{itemize}
\begin{itemize}
\item {Utilização:T. da Bairrada}
\end{itemize}
O mesmo que \textunderscore fragueiro\textunderscore .
Fueiro.
\section{Fanha}
\begin{itemize}
\item {Grp. gram.:m.  e  f.}
\end{itemize}
\begin{itemize}
\item {Utilização:Des.}
\end{itemize}
\begin{itemize}
\item {Proveniência:(T. onom.)}
\end{itemize}
Pessôa fanhosa.
\section{Fanhosamente}
\begin{itemize}
\item {Grp. gram.:adv.}
\end{itemize}
De modo fanhoso.
\section{Fanhosear}
\begin{itemize}
\item {Grp. gram.:v. i.}
\end{itemize}
\begin{itemize}
\item {Utilização:Neol.}
\end{itemize}
\begin{itemize}
\item {Grp. gram.:V. t.}
\end{itemize}
\begin{itemize}
\item {Proveniência:(De \textunderscore fanhoso\textunderscore )}
\end{itemize}
Falar fanhosamente.
Pronunciar fanhosamente. Cf. Camillo, \textunderscore Corja\textunderscore , 205.
\section{Fanhoso}
\begin{itemize}
\item {Grp. gram.:adj.}
\end{itemize}
\begin{itemize}
\item {Proveniência:(De \textunderscore fanha\textunderscore )}
\end{itemize}
Que tem a pronúncia defeituosa, como de quem fala pelo nariz.
\section{Fânia}
\begin{itemize}
\item {Grp. gram.:adj. f.}
\end{itemize}
Dizia-se de uma lei, promulgada entre os Romanos pelo cônsul Fânnio, contra o luxo.
\section{Fanicar}
\begin{itemize}
\item {Grp. gram.:v. i.}
\end{itemize}
\begin{itemize}
\item {Utilização:Pop.}
\end{itemize}
Andar ao fanico, á cata de pequenos lucros.
\section{Fanichar}
\begin{itemize}
\item {Grp. gram.:v. t.}
\end{itemize}
\begin{itemize}
\item {Utilização:Prov.}
\end{itemize}
\begin{itemize}
\item {Utilização:minh.}
\end{itemize}
Rachar.
Partir; tirar bocados de.
(Por \textunderscore fanicar\textunderscore , de \textunderscore fanico\textunderscore )
\section{Fanico}
\begin{itemize}
\item {Grp. gram.:m.}
\end{itemize}
\begin{itemize}
\item {Utilização:Fam.}
\end{itemize}
Desmaio; chilique.
Accidente nervoso ou histérico.
\section{Fanico}
\begin{itemize}
\item {Grp. gram.:m.}
\end{itemize}
Fragmento.
Migalha.
\textunderscore Partir em fanicos\textunderscore , despedaçar.
Pequenos lucros.
Cigalho.
* \textunderscore T. de Lisbôa\textunderscore  e \textunderscore do Minho\textunderscore .
\textunderscore Carro do fanico\textunderscore , designação deprec. das viaturas, que recebem passageiros em qualquer ponto, por preços minimos.
(Deve relacionar-se com o cast. \textunderscore anicos\textunderscore , pedaços)
\section{Fânio}
\begin{itemize}
\item {Grp. gram.:m.}
\end{itemize}
Fabricante de papiro, entre os antigos. Cf. Castilho, \textunderscore Fastos\textunderscore , I, 315.
\section{Faniqueira}
\begin{itemize}
\item {Grp. gram.:f.}
\end{itemize}
\begin{itemize}
\item {Utilização:Prov.}
\end{itemize}
Linha de pesca, usada pelos pescadores do Doiro.
Baraça, para jogar o pião.
(Cp. \textunderscore faniqueiro\textunderscore )
\section{Faniqueiro}
\begin{itemize}
\item {Grp. gram.:adj.}
\end{itemize}
\begin{itemize}
\item {Proveniência:(De \textunderscore fanico\textunderscore )}
\end{itemize}
Que anda ao fanico, procurando aquém e além pequenos ganhos.
\section{Faniquito}
\begin{itemize}
\item {Grp. gram.:m.}
\end{itemize}
\begin{itemize}
\item {Utilização:Fam.}
\end{itemize}
\begin{itemize}
\item {Proveniência:(De \textunderscore fanico\textunderscore )}
\end{itemize}
Ataque nervoso, de pouca monta.
\section{Fânnia}
\begin{itemize}
\item {Grp. gram.:adj. f.}
\end{itemize}
Dizia-se de uma lei, promulgada entre os Romanos pelo cônsul Fânnio, contra o luxo.
\section{Fannio}
\begin{itemize}
\item {Grp. gram.:m.}
\end{itemize}
Fabricante de papyro, entre os antigos. Cf. Castilho, \textunderscore Fastos\textunderscore , I, 315.
\section{Fano}
\begin{itemize}
\item {Grp. gram.:m.}
\end{itemize}
\begin{itemize}
\item {Utilização:Des.}
\end{itemize}
\begin{itemize}
\item {Proveniência:(Lat. \textunderscore fanum\textunderscore )}
\end{itemize}
Pequeno templo antigo.
Santuário. Cf. Vieira, VIII, 462.
\section{Fanqueiro}
\begin{itemize}
\item {Grp. gram.:m.}
\end{itemize}
Commerciante de fazendas de algodão, linho, lan, etc.
(Contr. de \textunderscore faniqueiro\textunderscore ?)
\section{Fans}
\begin{itemize}
\item {Grp. gram.:m. pl.}
\end{itemize}
Povo cannibal da África equatorial.
\section{Fantan}
\begin{itemize}
\item {Grp. gram.:m.}
\end{itemize}
\begin{itemize}
\item {Utilização:T. de Macau}
\end{itemize}
Espécie de jôgo de asar, em que, com o auxilio de sapecas, se joga sôbre quatro números inscritos numa loisa.
(Do chin.)
\section{Fantarelo}
\begin{itemize}
\item {Grp. gram.:m.}
\end{itemize}
\begin{itemize}
\item {Utilização:Prov.}
\end{itemize}
\begin{itemize}
\item {Utilização:trasm.}
\end{itemize}
Indivíduo, que tudo julga fácil e se considera destinado para grandes feitos.
Janota presumido e fanfarrão.
\section{Fantascópio}
\begin{itemize}
\item {Grp. gram.:m.}
\end{itemize}
Espécie de lanterna mágica, empregada em fantasmagoria.
(Por \textunderscore fantasmatoscópio\textunderscore , do gr. \textunderscore phantasma\textunderscore  + \textunderscore skopein\textunderscore )
\section{Fantasia}
\begin{itemize}
\item {Grp. gram.:f.}
\end{itemize}
\begin{itemize}
\item {Proveniência:(Lat. \textunderscore fantasia\textunderscore )}
\end{itemize}
Imaginação.
Obra de imaginação.
Concepção.
Ideia.
Vontade passageira.
Gôsto particular.
Capricho.
Extravagância.
Variação musical, segundo o capricho do artista.
Quadro, em que o pintor, seguindo a sua imaginação, se desviou das regras estabelecidas.
Carácter de letra typográphica, que não é simples ou commum.
\section{Fantasiador}
\begin{itemize}
\item {Grp. gram.:m.  e  adj.}
\end{itemize}
O que fantasia.
\section{Fantasiar}
\begin{itemize}
\item {Grp. gram.:v. t.}
\end{itemize}
Pôr na fantasia; imaginar.
\section{Fantasiosamente}
\begin{itemize}
\item {Grp. gram.:adv.}
\end{itemize}
De modo fantasioso.
\section{Fantasioso}
\begin{itemize}
\item {Grp. gram.:adj.}
\end{itemize}
Em que há fantasia.
Que revela imaginação.
Que existe na imaginação.
Imaginoso.
Fantástico.
\section{Fantasista}
\begin{itemize}
\item {Grp. gram.:adj.}
\end{itemize}
\begin{itemize}
\item {Grp. gram.:M.  e  f.}
\end{itemize}
Que tem fantasia.
Pessôa que fantasia.
\section{Fantasma}
\begin{itemize}
\item {Grp. gram.:m.}
\end{itemize}
\begin{itemize}
\item {Grp. gram.:F.}
\end{itemize}
\begin{itemize}
\item {Proveniência:(Lat. \textunderscore fantasma\textunderscore )}
\end{itemize}
Imagem illusória.
Espectro.
Visão aterradora.
Supposta apparição de pessôa defunta.
Avejão.
Simulacro.
Chimera.
Sombra.
Coisa medonha.
Pessôa muito magra e débil.
?:«\textunderscore as falsas fantasmas\textunderscore ». Usque, \textunderscore Tribulações\textunderscore , 26.
\section{Fantasmagoria}
\begin{itemize}
\item {Grp. gram.:f.}
\end{itemize}
\begin{itemize}
\item {Proveniência:(Do lat. \textunderscore fantasma\textunderscore  + gr. \textunderscore agoreuein\textunderscore )}
\end{itemize}
Arte de fazer vêr fantasmas ou figuras luminosas, em meio de grande escuridão.
Fantasma.
Falsa apparência.
\section{Fantasmagoricamente}
\begin{itemize}
\item {Grp. gram.:adv.}
\end{itemize}
De modo fantasmagórico.
Á maneira de fantasmagoria.
\section{Fantasmagórico}
\begin{itemize}
\item {Grp. gram.:adj.}
\end{itemize}
Relativo á fantasmagoria.
Fantástico.
Illusório.
\section{Fantasmagorizar}
\begin{itemize}
\item {Grp. gram.:v. t.}
\end{itemize}
\begin{itemize}
\item {Utilização:Neol.}
\end{itemize}
Tornar fantasmagórico. Cf. Alves Mendes, \textunderscore Pátria\textunderscore , 11.
\section{Fantasmal}
\begin{itemize}
\item {Grp. gram.:adj.}
\end{itemize}
O mesmo que \textunderscore fantástico\textunderscore . Cf. A. J. Silva, \textunderscore Guerras do Alecrim e da Mangerona\textunderscore .
\section{Fantasticamente}
\begin{itemize}
\item {Grp. gram.:adv.}
\end{itemize}
De modo fantástico.
Extraordinariamente.
De modo incrível.
\section{Fantástico}
\begin{itemize}
\item {Grp. gram.:adj.}
\end{itemize}
\begin{itemize}
\item {Grp. gram.:M.}
\end{itemize}
\begin{itemize}
\item {Proveniência:(Lat. \textunderscore fantasticus\textunderscore )}
\end{itemize}
Que só existe na fantasia.
Imaginário.
Caprichoso.
Extraordinário.
Incrível.
Aquillo que só existe na imaginação.
\section{Fantastiquice}
\begin{itemize}
\item {Grp. gram.:f.}
\end{itemize}
\begin{itemize}
\item {Proveniência:(De \textunderscore fantástico\textunderscore )}
\end{itemize}
Extravagância de gostos ou de appetites.
Jactância.
Bazófia.
\section{Fantesia}
\begin{itemize}
\item {Grp. gram.:f.}
\end{itemize}
\begin{itemize}
\item {Utilização:ant.}
\end{itemize}
\begin{itemize}
\item {Utilização:Pop.}
\end{itemize}
O mesmo que \textunderscore fantasia\textunderscore . Cf. Usque, \textunderscore Tribulações\textunderscore .
\section{Fantil}
\begin{itemize}
\item {Grp. gram.:adj.}
\end{itemize}
\begin{itemize}
\item {Utilização:Prov.}
\end{itemize}
\begin{itemize}
\item {Utilização:alent.}
\end{itemize}
Diz-se do cavallo de bôa raça e de bôa altura.
Diz-se da égua, que não trabalha e produz bôas crias.
\section{Fantochada}
\begin{itemize}
\item {Grp. gram.:f.}
\end{itemize}
\begin{itemize}
\item {Utilização:Fam.}
\end{itemize}
\begin{itemize}
\item {Utilização:Fig.}
\end{itemize}
Porção de fantoches.
Scena de fantoches.
Acção ridícula, caricata.
\section{Fantoche}
\begin{itemize}
\item {Grp. gram.:m.}
\end{itemize}
\begin{itemize}
\item {Utilização:Fig.}
\end{itemize}
\begin{itemize}
\item {Proveniência:(Fr. \textunderscore fantoche\textunderscore )}
\end{itemize}
Autómato, ou boneco, que se faz mover por meio de arames ou cordéis, ou com a mão.
Pessôa, que procede e fala, á vontade de outra.
\section{Fanucar}
\begin{itemize}
\item {Grp. gram.:v. i.}
\end{itemize}
\begin{itemize}
\item {Utilização:Prov.}
\end{itemize}
\begin{itemize}
\item {Utilização:dur.}
\end{itemize}
\begin{itemize}
\item {Proveniência:(De \textunderscore fanuco\textunderscore )}
\end{itemize}
Cair neve, nevar.
\section{Fanucho}
\begin{itemize}
\item {Grp. gram.:adj.}
\end{itemize}
\begin{itemize}
\item {Utilização:Prov.}
\end{itemize}
\begin{itemize}
\item {Utilização:minh.}
\end{itemize}
Apertado, estreito: \textunderscore manga fanucha\textunderscore .
\section{Fanuco}
\begin{itemize}
\item {Grp. gram.:m.}
\end{itemize}
\begin{itemize}
\item {Utilização:Prov.}
\end{itemize}
\begin{itemize}
\item {Utilização:dur.}
\end{itemize}
Froco de neve.
\section{Fão}
\begin{itemize}
\item {Grp. gram.:m.}
\end{itemize}
Antigo pêso da China.
\section{Faqueiro}
\begin{itemize}
\item {Grp. gram.:m.}
\end{itemize}
\begin{itemize}
\item {Proveniência:(De \textunderscore faca\textunderscore )}
\end{itemize}
Estôjo para talheres, especialmente para facas.
Caixa ou lugar, em que se guardam talheres.
Fabricante de facas.
Cutileiro.
\section{Faquetão}
\begin{itemize}
\item {Grp. gram.:m.}
\end{itemize}
O mesmo que \textunderscore faquete\textunderscore .
\section{Faquete}
\begin{itemize}
\item {fónica:quê}
\end{itemize}
\begin{itemize}
\item {Grp. gram.:m.}
\end{itemize}
\begin{itemize}
\item {Utilização:Zool.}
\end{itemize}
Espécie de esqualo cinzento-amarelado, de focinho rombo e bôca guarnecida de dentes recurvos e finos.
\section{Faqui}
\begin{itemize}
\item {Grp. gram.:m.}
\end{itemize}
Jurisconsulto muçulmano. Cf. Herculano, \textunderscore Hist. de Port.\textunderscore , I, 129.
(Ar. \textunderscore faqie\textunderscore )
\section{Faquineta}
\begin{itemize}
\item {fónica:nê}
\end{itemize}
\begin{itemize}
\item {Grp. gram.:m.  e  f.}
\end{itemize}
\begin{itemize}
\item {Utilização:Pop.}
\end{itemize}
Pessôa, que usa faca ou dá facadas.
(Por \textunderscore faquinheta\textunderscore , de \textunderscore faquinha\textunderscore , dem. de \textunderscore faca\textunderscore )
\section{Faquino}
\begin{itemize}
\item {Grp. gram.:m.}
\end{itemize}
\begin{itemize}
\item {Utilização:Ant.}
\end{itemize}
\begin{itemize}
\item {Proveniência:(It. \textunderscore facchino\textunderscore )}
\end{itemize}
Varredor da igreja patriarchal de Lisbôa.
O mesmo que \textunderscore mariola\textunderscore ^1.
\section{Faquino}
\begin{itemize}
\item {Grp. gram.:m.}
\end{itemize}
\begin{itemize}
\item {Utilização:Des.}
\end{itemize}
O mesmo que \textunderscore faquir\textunderscore . Cp. \textunderscore Vieira\textunderscore , XI, 240.
\section{Faquir}
\begin{itemize}
\item {Grp. gram.:m.}
\end{itemize}
\begin{itemize}
\item {Proveniência:(T. ár)}
\end{itemize}
Monge muçulmano do Oriente, que vive de esmolas e em rigoroso ascetismo.
\section{Faquista}
\begin{itemize}
\item {Grp. gram.:m.}
\end{itemize}
\begin{itemize}
\item {Proveniência:(De \textunderscore faca\textunderscore )}
\end{itemize}
Aquelle que usa de faca, como arma offensiva.
\section{Far}
\begin{itemize}
\item {Grp. gram.:v. t.}
\end{itemize}
\begin{itemize}
\item {Utilização:Ant.}
\end{itemize}
O mesmo que \textunderscore fazer\textunderscore .
\section{Faraçola}
\begin{itemize}
\item {Grp. gram.:f.}
\end{itemize}
Antigo pêso indiano. Cf. \textunderscore Livro dos Pesos da Yndia e assi medidas e mohêdas\textunderscore , 9, ed. 1868.
\section{Farádico}
\begin{itemize}
\item {Grp. gram.:adj.}
\end{itemize}
Relativo á faradização.
\section{Farádio}
\begin{itemize}
\item {Grp. gram.:m.}
\end{itemize}
\begin{itemize}
\item {Utilização:Phýs.}
\end{itemize}
\begin{itemize}
\item {Proveniência:(De \textunderscore Faraday\textunderscore , n. p.)}
\end{itemize}
Unidade electro-magnética de capacidade eléctrica.
\section{Faradismo}
\begin{itemize}
\item {Grp. gram.:m.}
\end{itemize}
O mesmo que \textunderscore faradização\textunderscore .
\section{Faradização}
\begin{itemize}
\item {Grp. gram.:f.}
\end{itemize}
\begin{itemize}
\item {Proveniência:(De \textunderscore faradizar\textunderscore )}
\end{itemize}
Therapêutica da electricidade de inducção.
\section{Faradizar}
\begin{itemize}
\item {Grp. gram.:v. t.}
\end{itemize}
\begin{itemize}
\item {Proveniência:(De \textunderscore farádio\textunderscore )}
\end{itemize}
Medicar com a electricidade inductiva.
\section{Farafalha}
\begin{itemize}
\item {Grp. gram.:f.}
\end{itemize}
O mesmo que \textunderscore farfalha\textunderscore . Cf. Usque, \textunderscore Tribulações\textunderscore , 16, v.^o
\section{Faramalha}
\begin{itemize}
\item {Grp. gram.:f.}
\end{itemize}
\begin{itemize}
\item {Utilização:Prov.}
\end{itemize}
\begin{itemize}
\item {Utilização:beir.}
\end{itemize}
\begin{itemize}
\item {Utilização:trasm.}
\end{itemize}
Prosápia sem fundamento.
Palavreado ôco.
Grandeza van.
\section{Faramalheiro}
\begin{itemize}
\item {Grp. gram.:m.  e  adj.}
\end{itemize}
Aquelle ou aquillo que tem faramalha.
\section{Faramalhice}
\begin{itemize}
\item {Grp. gram.:f.}
\end{itemize}
Qualidade, acto ou dito de faramalheiro.
\section{Farândula}
\begin{itemize}
\item {Grp. gram.:f.}
\end{itemize}
\begin{itemize}
\item {Utilização:Pop.}
\end{itemize}
Dança de cadeia, na Provença.
Bando de maltrapilhos; súcia.
(Cast. \textunderscore farandula\textunderscore )
\section{Farandulagem}
\begin{itemize}
\item {Grp. gram.:f.}
\end{itemize}
\begin{itemize}
\item {Utilização:Pop.}
\end{itemize}
\begin{itemize}
\item {Proveniência:(De \textunderscore farândula\textunderscore )}
\end{itemize}
Súcia de maltrapilhos.
Farraparia.
\section{Farante}
\begin{itemize}
\item {Grp. gram.:m.}
\end{itemize}
\begin{itemize}
\item {Utilização:Gír.}
\end{itemize}
\begin{itemize}
\item {Proveniência:(De \textunderscore farar\textunderscore )}
\end{itemize}
Aquelle que fareja ou procura alguma coisa.
Alcovista.
\section{Farar}
\begin{itemize}
\item {Grp. gram.:v. t.}
\end{itemize}
\begin{itemize}
\item {Utilização:Gír.}
\end{itemize}
\begin{itemize}
\item {Proveniência:(De \textunderscore faro\textunderscore )}
\end{itemize}
Procurar; apanhar.
\section{Farauta}
\begin{itemize}
\item {Grp. gram.:f.}
\end{itemize}
\begin{itemize}
\item {Utilização:Prov.}
\end{itemize}
\begin{itemize}
\item {Utilização:minh.}
\end{itemize}
O mesmo que \textunderscore farota\textunderscore .
\section{Faraute}
\begin{itemize}
\item {Grp. gram.:m.}
\end{itemize}
\begin{itemize}
\item {Utilização:Ant.}
\end{itemize}
\begin{itemize}
\item {Proveniência:(Do ant. alt. al. \textunderscore hariwasto\textunderscore )}
\end{itemize}
Arauto.
Intérprete, guia ou língua de uma empresa ou expedição.
\section{Faraz}
\begin{itemize}
\item {Grp. gram.:m.}
\end{itemize}
\begin{itemize}
\item {Utilização:Ant.}
\end{itemize}
\begin{itemize}
\item {Proveniência:(Do ár. \textunderscore faras\textunderscore )}
\end{itemize}
Moço de estrebaria, na Índia portuguesa.
\section{Farazala}
\begin{itemize}
\item {Grp. gram.:f.}
\end{itemize}
(Encontra-se a palavra no \textunderscore Rosteiro de Vasco de Gama\textunderscore , cujo autor provavelmente escreveu \textunderscore farazola\textunderscore . V. \textunderscore faraçola\textunderscore )
\section{Farça}
\textunderscore f.\textunderscore  (e der.)
(V. \textunderscore farsa\textunderscore , etc.)
\section{Farcino}
\begin{itemize}
\item {Grp. gram.:m.}
\end{itemize}
\begin{itemize}
\item {Proveniência:(Do fr. \textunderscore farcin\textunderscore )}
\end{itemize}
Fórma clínica do mormo, sem manifestações nasaes.
Moléstia dos bois, sem relação com aquella, e causada por um micróbio, \textunderscore streptotrix\textunderscore .
\section{Farda}
\begin{itemize}
\item {Grp. gram.:f.}
\end{itemize}
\begin{itemize}
\item {Proveniência:(Do ár. \textunderscore fard\textunderscore , vestido)}
\end{itemize}
Traje uniforme para uma categoria de indivíduos; uniforme.
Fardamento.
Libré.
\section{Fardagem}
\begin{itemize}
\item {Grp. gram.:f.}
\end{itemize}
Porção de fardos.
Roupagem; trapagem.
\section{Fardalhão}
\begin{itemize}
\item {Grp. gram.:m.}
\end{itemize}
Farda vistosa ou apparatosa. Cf. Castilho, \textunderscore Fausto\textunderscore , 164.
\section{Fardamenta}
\begin{itemize}
\item {Grp. gram.:f.}
\end{itemize}
\begin{itemize}
\item {Utilização:Bras}
\end{itemize}
O mesmo que \textunderscore fardamento\textunderscore . Cf. J. Ribeiro, \textunderscore Diccion. Gram.\textunderscore , 67.
\section{Fardamento}
\begin{itemize}
\item {Grp. gram.:m.}
\end{itemize}
\begin{itemize}
\item {Proveniência:(De \textunderscore fardar\textunderscore )}
\end{itemize}
Farda.
Conjunto de fardas; typo de fardas.
\section{Fardar}
\begin{itemize}
\item {Grp. gram.:v. t.}
\end{itemize}
Vestir com farda.
Prover de farda ou de fardas.
\section{Fardel}
\begin{itemize}
\item {Grp. gram.:m.}
\end{itemize}
\begin{itemize}
\item {Utilização:Prov.}
\end{itemize}
\begin{itemize}
\item {Proveniência:(De \textunderscore fardo\textunderscore )}
\end{itemize}
Provisões alimentícias para pequena viagem.
Saco de provisões para jornada.
Mólho de roupas, atadas confusamente.
\section{Fardelagem}
\begin{itemize}
\item {Grp. gram.:f.}
\end{itemize}
\begin{itemize}
\item {Utilização:Ant.}
\end{itemize}
\begin{itemize}
\item {Proveniência:(De \textunderscore fardel\textunderscore )}
\end{itemize}
O mesmo que \textunderscore fardagem\textunderscore .
O mesmo que \textunderscore bagagem\textunderscore . Cf. \textunderscore Peregrinação\textunderscore , CXVII.
\section{Fardeta}
\begin{itemize}
\item {fónica:dê}
\end{itemize}
\begin{itemize}
\item {Grp. gram.:f.}
\end{itemize}
Pequena farda, ou farda que os soldados vestem em serviço de faxina.
\section{Fardete}
\begin{itemize}
\item {fónica:dê}
\end{itemize}
\begin{itemize}
\item {Grp. gram.:m.}
\end{itemize}
Pequeno fardo.
\section{Fardo}
\begin{itemize}
\item {Grp. gram.:m.}
\end{itemize}
\begin{itemize}
\item {Utilização:Fig.}
\end{itemize}
\begin{itemize}
\item {Proveniência:(Do ár. \textunderscore fard\textunderscore )}
\end{itemize}
Coisa ou conjunto de coisas, mais ou menos volumosas e pesadas, que se destinam a transporte.
Pacote; embrulho.
Carga.
Aquillo que custa soffrer.
Aquillo que impõe responsabilidade.
\section{Farejar}
\begin{itemize}
\item {Grp. gram.:v. t.}
\end{itemize}
\begin{itemize}
\item {Grp. gram.:V. i.}
\end{itemize}
Cheirar, seguir ou acompanhar, levado pelo faro: \textunderscore farejar caça\textunderscore .
Aspirar o cheiro de.
Procurar, servindo-se do olfato.
Buscar; esquadrinhar.
Tomar o faro.
\section{Farejo}
\begin{itemize}
\item {Grp. gram.:m.}
\end{itemize}
Acto de farejar.
\section{Farel}
\begin{itemize}
\item {Grp. gram.:m.  e  adj.}
\end{itemize}
\begin{itemize}
\item {Utilização:Prov.}
\end{itemize}
\begin{itemize}
\item {Utilização:trasm.}
\end{itemize}
Espécie de muchão ou trombeteiro.
\section{Fareláceo}
\begin{itemize}
\item {Grp. gram.:adj.}
\end{itemize}
Que se desfaz em farelo.
\section{Farelada}
\begin{itemize}
\item {Grp. gram.:f.}
\end{itemize}
\begin{itemize}
\item {Utilização:Ant.}
\end{itemize}
Farelagem.
Água com farelos, para os porcos. Cf. \textunderscore Bibl. da G. do Campo\textunderscore , 333.
Acto de empoar alguém com farelos, em folganças de entrudo. Cf. \textunderscore Alvará de D. Sebast.\textunderscore , in \textunderscore Rev. Lus.\textunderscore , XV, 120.
\section{Farelagem}
\begin{itemize}
\item {Grp. gram.:f.}
\end{itemize}
\begin{itemize}
\item {Utilização:Fig.}
\end{itemize}
Porção de farelos.
Insignificância.
\section{Farelão}
\begin{itemize}
\item {Grp. gram.:m.}
\end{itemize}
\begin{itemize}
\item {Utilização:Prov.}
\end{itemize}
\begin{itemize}
\item {Utilização:minh.}
\end{itemize}
Homem, que se gaba muito; fanfarrão.
\section{Farelar}
\begin{itemize}
\item {Grp. gram.:v. i.}
\end{itemize}
\begin{itemize}
\item {Utilização:Prov.}
\end{itemize}
\begin{itemize}
\item {Utilização:minh.}
\end{itemize}
\begin{itemize}
\item {Proveniência:(De \textunderscore farêlo\textunderscore )}
\end{itemize}
Gabar-se.
Sêr fanfarrão.
\section{Fareleiro}
\begin{itemize}
\item {Grp. gram.:m.}
\end{itemize}
\begin{itemize}
\item {Utilização:Des.}
\end{itemize}
Aquelle, que se gaba muito, que se jacta de saber o que ignora ou de têr o que não tem.
(Cp. \textunderscore farelório\textunderscore )
\section{Farelento}
\begin{itemize}
\item {Grp. gram.:adj.}
\end{itemize}
Abundante de farelos.
Que dá muito farelo.
\section{Farelhão}
\begin{itemize}
\item {Grp. gram.:m.}
\end{itemize}
Pequeno promontório; ilhota escarpada.
(Cast. \textunderscore farellon\textunderscore )
\section{Farelice}
\begin{itemize}
\item {Grp. gram.:f.}
\end{itemize}
\begin{itemize}
\item {Utilização:Des.}
\end{itemize}
Qualidade de quem é fareleiro; bazófia, fanfarrico.
(Cp. \textunderscore fareleiro\textunderscore )
\section{Farêlo}
\begin{itemize}
\item {Grp. gram.:m.}
\end{itemize}
\begin{itemize}
\item {Utilização:Fig.}
\end{itemize}
\begin{itemize}
\item {Utilização:Prov.}
\end{itemize}
\begin{itemize}
\item {Utilização:minh.}
\end{itemize}
\begin{itemize}
\item {Proveniência:(Do lat. \textunderscore far\textunderscore )}
\end{itemize}
Aquillo que fica na peneira, depois de peneirada a farinha.
Resíduos grosseiros dos cereaes moídos.
Insignificância.
Jactância, bazófia, vaidade.
\section{Farelório}
\begin{itemize}
\item {Grp. gram.:m.}
\end{itemize}
\begin{itemize}
\item {Utilização:Fam.}
\end{itemize}
\begin{itemize}
\item {Utilização:Prov.}
\end{itemize}
\begin{itemize}
\item {Utilização:alent.}
\end{itemize}
\begin{itemize}
\item {Proveniência:(De \textunderscore farêlo\textunderscore )}
\end{itemize}
Coisa de pouca monta, insignificante.
Coisa digna de menosprêzo.
Jactância, bravata.
Bolo caseiro, bolo simples.
\section{Faretrado}
\begin{itemize}
\item {Grp. gram.:adj.}
\end{itemize}
\begin{itemize}
\item {Grp. gram.:adj.}
\end{itemize}
\begin{itemize}
\item {Utilização:Poét.}
\end{itemize}
\begin{itemize}
\item {Proveniência:(Lat. \textunderscore pharetratus\textunderscore )}
\end{itemize}
Ferido por seta.
Armado de setas.
Que usa ou leva aljava. Cf. Castilho, \textunderscore Geórg.\textunderscore , 259.
\section{Farfalha}
\begin{itemize}
\item {Grp. gram.:f.}
\end{itemize}
\begin{itemize}
\item {Grp. gram.:F. pl.}
\end{itemize}
\begin{itemize}
\item {Utilização:Fig.}
\end{itemize}
\begin{itemize}
\item {Utilização:Prov.}
\end{itemize}
\begin{itemize}
\item {Utilização:beir.}
\end{itemize}
\begin{itemize}
\item {Proveniência:(De \textunderscore farfalhar\textunderscore )}
\end{itemize}
O mesmo que \textunderscore farfalheira\textunderscore .
Limalha.
Aparas.
Bagatelas.
Frocos de neve.
\section{Farfalhada}
\begin{itemize}
\item {Grp. gram.:f.}
\end{itemize}
\begin{itemize}
\item {Utilização:Fig.}
\end{itemize}
\begin{itemize}
\item {Proveniência:(De \textunderscore farfalhar\textunderscore )}
\end{itemize}
Rumor de maravalhas ou farfalhas.
Ruído, semelhante a êsse rumor.
Insignificância.
Bazófia; falácia.
\section{Farfalhador}
\begin{itemize}
\item {Grp. gram.:m.}
\end{itemize}
Aquelle que farfalha.
\section{Farfalhante}
\begin{itemize}
\item {Grp. gram.:adj.}
\end{itemize}
Que farfalha.
Que é farfalhão.
\section{Farfalhão}
\begin{itemize}
\item {Grp. gram.:m.}
\end{itemize}
\begin{itemize}
\item {Proveniência:(De \textunderscore farfalhar\textunderscore )}
\end{itemize}
O mesmo que \textunderscore farfalhador\textunderscore .
\section{Farfalhar}
\begin{itemize}
\item {Grp. gram.:v. i.}
\end{itemize}
Falar sem tino.
Parolar.
Fazer ostentação.
Fazer barulho.
(Cast. \textunderscore farfallar\textunderscore )
\section{Farfalharia}
\begin{itemize}
\item {Grp. gram.:f.}
\end{itemize}
\begin{itemize}
\item {Grp. gram.:Pl.}
\end{itemize}
O mesmo que \textunderscore farfalhada\textunderscore .
Conjunto de fitas ou laços muito vistosos, ou de outros ornatos exaggerados, sobretudo nos ombros e no pescoço das mulheres.
O mesmo que \textunderscore farfalho\textunderscore .
\section{Farfalheira}
\begin{itemize}
\item {Grp. gram.:f.}
\end{itemize}
\begin{itemize}
\item {Grp. gram.:Pl.}
\end{itemize}
O mesmo que \textunderscore farfalhada\textunderscore .
Conjunto de fitas ou laços muito vistosos, ou de outros ornatos exaggerados, sobretudo nos ombros e no pescoço das mulheres.
O mesmo que \textunderscore farfalho\textunderscore .
\section{Farfalheiro}
\begin{itemize}
\item {Grp. gram.:adj.}
\end{itemize}
O mesmo que \textunderscore farfalhento\textunderscore ; que usa ornatos vistosos, exaggerados.
\section{Farfalhento}
\begin{itemize}
\item {Grp. gram.:adj.}
\end{itemize}
Que farfalha.
\section{Farfalhice}
\begin{itemize}
\item {Grp. gram.:f.}
\end{itemize}
\begin{itemize}
\item {Proveniência:(De \textunderscore farfalho\textunderscore )}
\end{itemize}
Ostentação.
Bazófia.
\section{Farfalho}
\begin{itemize}
\item {Grp. gram.:m.}
\end{itemize}
\begin{itemize}
\item {Proveniência:(De \textunderscore farfalhar\textunderscore )}
\end{itemize}
Acto de farfalhar.
Doença das crianças, caracterizada por vermelhidão da mucosa da bôca, com pequenas elevações de um branco opaco.
Rouquidão, pieira, fervores.
\section{Farfalhoso}
\begin{itemize}
\item {Grp. gram.:adj.}
\end{itemize}
O mesmo que \textunderscore farfalhudo\textunderscore . Cf. Camillo, \textunderscore Quéda\textunderscore , 239.
\section{Farfalhuda}
\begin{itemize}
\item {Grp. gram.:f.}
\end{itemize}
\begin{itemize}
\item {Utilização:Prov.}
\end{itemize}
\begin{itemize}
\item {Utilização:alent.}
\end{itemize}
Variedade de alface.
(Cp. \textunderscore farfalhudo\textunderscore )
\section{Farfalhudo}
\begin{itemize}
\item {Grp. gram.:adj.}
\end{itemize}
\begin{itemize}
\item {Proveniência:(De \textunderscore farfalha\textunderscore )}
\end{itemize}
Vistoso; garrido.
Emproado.
Bombástico; campanudo: \textunderscore estilo farfalhudo\textunderscore .
\section{Farfan}
\begin{itemize}
\item {Grp. gram.:m.}
\end{itemize}
Cada um dos soldados christãos, que, depois da conquista da Espanha pelos Árabes, viveram em Marrocos, ao serviço de Príncipes muçulmanos.
(Cast. \textunderscore farfan\textunderscore )
\section{Farfância}
\begin{itemize}
\item {Grp. gram.:f.}
\end{itemize}
Farfalhice.
\section{Farfantão}
\begin{itemize}
\item {Grp. gram.:m.}
\end{itemize}
\begin{itemize}
\item {Utilização:Prov.}
\end{itemize}
\begin{itemize}
\item {Utilização:trasm.}
\end{itemize}
\begin{itemize}
\item {Utilização:minh.}
\end{itemize}
\begin{itemize}
\item {Proveniência:(De \textunderscore farfante\textunderscore )}
\end{itemize}
Valentão.
Paparrotão.
\section{Farfante}
\begin{itemize}
\item {Grp. gram.:m.  e  adj.}
\end{itemize}
Fanfarrão; valentão.
Farfalhudo.
(Cast. \textunderscore farfante\textunderscore )
\section{Farfantear}
\begin{itemize}
\item {Grp. gram.:v. i.}
\end{itemize}
\begin{itemize}
\item {Proveniência:(De \textunderscore farfante\textunderscore )}
\end{itemize}
O mesmo que \textunderscore fanfarrear\textunderscore . Cf. Arn. Gama, \textunderscore Últ. Dona\textunderscore , 56.
\section{Farfária}
\begin{itemize}
\item {Grp. gram.:f.}
\end{itemize}
Gênero de plantas medicinaes, o mesmo que \textunderscore tussilagem\textunderscore . Cf. S. Costa, \textunderscore Hist. das Plantas Med.\textunderscore 
\section{Farfúncia}
\begin{itemize}
\item {Grp. gram.:f.}
\end{itemize}
\begin{itemize}
\item {Utilização:Prov.}
\end{itemize}
\begin{itemize}
\item {Utilização:alg.}
\end{itemize}
\begin{itemize}
\item {Utilização:Des.}
\end{itemize}
Lide, azáfama.
O mesmo que \textunderscore farfalhice\textunderscore . Cf. Filinto, VIII, 121.
(Cp. \textunderscore farfância\textunderscore )
\section{Fari}
\begin{itemize}
\item {Grp. gram.:m.}
\end{itemize}
\begin{itemize}
\item {Utilização:Ant.}
\end{itemize}
Espécie de incenso, na Índia portuguesa.
\section{Faricoco}
\begin{itemize}
\item {fónica:cô}
\end{itemize}
\begin{itemize}
\item {Grp. gram.:m.}
\end{itemize}
Fórma bras. de \textunderscore farricoco\textunderscore . Cf. Pacheco, \textunderscore Promptuário\textunderscore .
\section{Farilhão}
\begin{itemize}
\item {Grp. gram.:m.}
\end{itemize}
(V.farelhão)
\section{Farinação}
\begin{itemize}
\item {Grp. gram.:f.}
\end{itemize}
Acto de farinar.
\section{Farináceo}
\begin{itemize}
\item {Grp. gram.:adj.}
\end{itemize}
\begin{itemize}
\item {Proveniência:(Lat. \textunderscore farinaceus\textunderscore )}
\end{itemize}
Relativo a farinha.
Que tem a natureza da farinha: \textunderscore alimentos farináceos\textunderscore .
Que contém farinha ou fécula; que tem apparência de farinha.
\section{Farinar}
\begin{itemize}
\item {Grp. gram.:v. t.}
\end{itemize}
\begin{itemize}
\item {Proveniência:(Do lat. \textunderscore farina\textunderscore )}
\end{itemize}
Reduzir a farinha.
\section{Farinha}
\begin{itemize}
\item {Grp. gram.:f.}
\end{itemize}
\begin{itemize}
\item {Proveniência:(Lat. \textunderscore farina\textunderscore )}
\end{itemize}
Pó, a que se reduzem os grãos dos cereaes, depois de moídos.
Pó, em que se convertem certas sementes ou raízes, depois de trituradas.
\section{Farinháceo}
\begin{itemize}
\item {Grp. gram.:adj.}
\end{itemize}
(V.farináceo)Cf. Camillo, \textunderscore Vinho do Porto\textunderscore , 85.
\section{Farinhada}
\begin{itemize}
\item {Grp. gram.:f.}
\end{itemize}
\begin{itemize}
\item {Utilização:Bras}
\end{itemize}
Fabríco da farinha da mandioca.
\section{Farinha-farêlo}
\begin{itemize}
\item {Grp. gram.:f.}
\end{itemize}
Espécie de jôgo popular.
\section{Farinha-queimada}
\begin{itemize}
\item {Grp. gram.:f.}
\end{itemize}
\begin{itemize}
\item {Utilização:Bras}
\end{itemize}
Espécie de bailado popular.
\section{Farinhar}
\begin{itemize}
\item {Grp. gram.:v. i.}
\end{itemize}
\begin{itemize}
\item {Utilização:T. de Aveiro}
\end{itemize}
\begin{itemize}
\item {Proveniência:(De \textunderscore farinha\textunderscore )}
\end{itemize}
Diz-se do tabuleiro das marinhas, em que o sal começa a alvejar.
\section{Farinhata}
\begin{itemize}
\item {Grp. gram.:f.}
\end{itemize}
O mesmo que \textunderscore farinhato\textunderscore .
\section{Farinhato}
\begin{itemize}
\item {Grp. gram.:m.}
\end{itemize}
\begin{itemize}
\item {Utilização:Prov.}
\end{itemize}
\begin{itemize}
\item {Utilização:trasm.}
\end{itemize}
\begin{itemize}
\item {Utilização:Prov.}
\end{itemize}
\begin{itemize}
\item {Utilização:dur.}
\end{itemize}
\begin{itemize}
\item {Proveniência:(De \textunderscore farinha\textunderscore )}
\end{itemize}
Doença das videiras, o mesmo que \textunderscore oídio\textunderscore .
Chouriço, o mesmo que \textunderscore farinheira\textunderscore .
\section{Farinheira}
\begin{itemize}
\item {Grp. gram.:f.}
\end{itemize}
\begin{itemize}
\item {Utilização:Bras}
\end{itemize}
\begin{itemize}
\item {Utilização:Bras}
\end{itemize}
Chouriço de gordura de porco, com farinha ou miolo de pão e vários temperos.
Mulher, que vende farinha.
Árvore silvestre.
Vasilha, para guardar farinha.
Casta de uva tinta.
\section{Farinheiro}
\begin{itemize}
\item {Grp. gram.:m.}
\end{itemize}
\begin{itemize}
\item {Utilização:Prov.}
\end{itemize}
\begin{itemize}
\item {Utilização:beir.}
\end{itemize}
\begin{itemize}
\item {Proveniência:(Do lat. \textunderscore farinarius\textunderscore )}
\end{itemize}
Aquelle que negocia em farinha.
Doença das vinhas, o mesmo que \textunderscore míldio\textunderscore . (Colhido em Fozcôa)
\section{Farinhento}
\begin{itemize}
\item {Grp. gram.:adj.}
\end{itemize}
Que tem semelhança de farinha.
Coberto de farinha.
Farinhudo.
\section{Farinhoso}
\begin{itemize}
\item {Grp. gram.:adj.}
\end{itemize}
\begin{itemize}
\item {Proveniência:(Lat. \textunderscore farinosus\textunderscore )}
\end{itemize}
Farinhento.
\section{Farinhota}
\begin{itemize}
\item {Grp. gram.:f.}
\end{itemize}
\begin{itemize}
\item {Proveniência:(Do rad. de \textunderscore farinha\textunderscore )}
\end{itemize}
Espécie de uva preta.
\section{Farinhoto}
\begin{itemize}
\item {fónica:nhô}
\end{itemize}
\begin{itemize}
\item {Grp. gram.:m.}
\end{itemize}
\begin{itemize}
\item {Utilização:Prov.}
\end{itemize}
\begin{itemize}
\item {Utilização:minh.}
\end{itemize}
\begin{itemize}
\item {Proveniência:(De \textunderscore farinha\textunderscore )}
\end{itemize}
Molhinho de tripas de porco, ligadas por cordel, envoltas em farinha e fritas em banha.
\section{Farinhudo}
\begin{itemize}
\item {Grp. gram.:adj.}
\end{itemize}
\begin{itemize}
\item {Proveniência:(De \textunderscore farinha\textunderscore )}
\end{itemize}
Farinhento.
Diz-se principalmente dos frutos de polpa branda, que se desfaz como em grânulos farináceos: \textunderscore maçan farinhuda\textunderscore .
\section{Fariscador}
\begin{itemize}
\item {Grp. gram.:m.  e  adj.}
\end{itemize}
O que farisca.
\section{Fariscar}
\begin{itemize}
\item {Grp. gram.:v. t.  e  i.}
\end{itemize}
O mesmo que \textunderscore farejar\textunderscore .
\section{Farisco}
\begin{itemize}
\item {Grp. gram.:m.}
\end{itemize}
Acto de fariscar. Cf. Ortigão, \textunderscore Praias\textunderscore , 78.
\section{Fariseu}
\begin{itemize}
\item {Grp. gram.:m.}
\end{itemize}
\begin{itemize}
\item {Utilização:Prov.}
\end{itemize}
\begin{itemize}
\item {Utilização:trasm.}
\end{itemize}
\begin{itemize}
\item {Grp. gram.:m.}
\end{itemize}
\begin{itemize}
\item {Utilização:Ant.}
\end{itemize}
\begin{itemize}
\item {Utilização:Pop.}
\end{itemize}
O mesmo que \textunderscore enxergão\textunderscore .
Enxergão de palha.
\section{Farnel}
\begin{itemize}
\item {Grp. gram.:m.}
\end{itemize}
O mesmo que \textunderscore fardel\textunderscore .
\section{Farnento}
\begin{itemize}
\item {Grp. gram.:m.}
\end{itemize}
Variedade de uva preta de Azeitão.
(Contr. de \textunderscore farinhento\textunderscore ?)
\section{Farnesia}
\begin{itemize}
\item {Grp. gram.:f.}
\end{itemize}
\begin{itemize}
\item {Utilização:Ant.}
\end{itemize}
O mesmo que \textunderscore frenesi\textunderscore .
\section{Farnicoques}
\begin{itemize}
\item {Grp. gram.:m. pl.}
\end{itemize}
O mesmo que \textunderscore fornicoques\textunderscore . Cf. Castilho, \textunderscore Fausto\textunderscore , 274.
\section{Faro}
\begin{itemize}
\item {Grp. gram.:m.}
\end{itemize}
\begin{itemize}
\item {Utilização:Ext.}
\end{itemize}
\begin{itemize}
\item {Utilização:Fig.}
\end{itemize}
Olfato dos animaes, especialmente dos cães.
Cheiro.
Indício; peugada.
\section{Farofa}
\begin{itemize}
\item {Grp. gram.:f.}
\end{itemize}
\begin{itemize}
\item {Utilização:Bras}
\end{itemize}
Comida, feita de farinha cozida em toicinho ou manteiga.
Bravata.
Jactância.
(Cp. \textunderscore farófia\textunderscore )
\section{Farofeiro}
\begin{itemize}
\item {Grp. gram.:m.  e  adj.}
\end{itemize}
\begin{itemize}
\item {Proveniência:(De \textunderscore farofa\textunderscore )}
\end{itemize}
O que tem farófia ou jactância.
\section{Farófia}
\begin{itemize}
\item {Grp. gram.:f.}
\end{itemize}
\begin{itemize}
\item {Utilização:Fig.}
\end{itemize}
Doce fofo de claras de ovos, açúcar e canela.
Iguaria brasileira, em que entra principalmente a farinha de pau ou farinha de raíz de mandioca.
Farofa.
Jactância, bazófia; fanfarronada.
\section{Farola}
\begin{itemize}
\item {Grp. gram.:f.}
\end{itemize}
\begin{itemize}
\item {Utilização:Prov.}
\end{itemize}
\begin{itemize}
\item {Utilização:minh.}
\end{itemize}
Palavreado chocho; dito sem importância.
(Por \textunderscore parola\textunderscore ?)
\section{Faroleiro}
\begin{itemize}
\item {Grp. gram.:m.}
\end{itemize}
\begin{itemize}
\item {Utilização:Prov.}
\end{itemize}
\begin{itemize}
\item {Utilização:minh.}
\end{itemize}
\begin{itemize}
\item {Proveniência:(De \textunderscore farola\textunderscore )}
\end{itemize}
Palrador sem senso.
Aquelle que é vezeiro em farolice.
\section{Farolice}
\begin{itemize}
\item {Grp. gram.:f.}
\end{itemize}
\begin{itemize}
\item {Utilização:Prov.}
\end{itemize}
\begin{itemize}
\item {Utilização:minh.}
\end{itemize}
O mesmo que \textunderscore farola\textunderscore .
\section{Farota}
\begin{itemize}
\item {Grp. gram.:f.}
\end{itemize}
\begin{itemize}
\item {Utilização:Prov.}
\end{itemize}
\begin{itemize}
\item {Utilização:alent.}
\end{itemize}
Ovelha velha, que se vende nas feiras ou se dá em pagamento de pastagens ou como renda de certas terras, ou que se mata para alimentação das quadrilhas de trabalhadores. Cf. Rev. \textunderscore Tradição\textunderscore , I, 84.
\section{Farpa}
\begin{itemize}
\item {Grp. gram.:f.}
\end{itemize}
\begin{itemize}
\item {Proveniência:(Do rad. do germ. \textunderscore harpjan\textunderscore )}
\end{itemize}
Ponta de metal, penetrante, em fórma de ângulo agudo.
Hastil, armado com essa ponta, para ferir toiros em corridas.
Rasgão.
Pequena lasca de madeira, que accidentalmente se introduz na pelle.
\section{Farpante}
\begin{itemize}
\item {Grp. gram.:adj.}
\end{itemize}
Que farpa.
\section{Farpão}
\begin{itemize}
\item {Grp. gram.:m.}
\end{itemize}
\begin{itemize}
\item {Utilização:T. de Vianna}
\end{itemize}
\begin{itemize}
\item {Utilização:Fig.}
\end{itemize}
\begin{itemize}
\item {Proveniência:(De \textunderscore farpa\textunderscore )}
\end{itemize}
Antiga arma guerreira, que terminava em farpa.
Arpão; fateixa.
O mesmo que \textunderscore rasgão\textunderscore .
Golpe doloroso.
Aggressão.
\section{Farpar}
\begin{itemize}
\item {Grp. gram.:v. t.}
\end{itemize}
\begin{itemize}
\item {Proveniência:(De \textunderscore farpa\textunderscore )}
\end{itemize}
Pôr farpas em.
Farpear.
Recortar em fórma de farpa.
Romper, esfarrapar.
\section{Farpear}
\begin{itemize}
\item {Grp. gram.:v. t.}
\end{itemize}
Meter farpas em: \textunderscore farpear toiros\textunderscore .
Ferir com farpas.
\section{Farpela}
\begin{itemize}
\item {Grp. gram.:f.}
\end{itemize}
\begin{itemize}
\item {Utilização:Fam.}
\end{itemize}
Vestuário; fato.
(Contr. de \textunderscore farrapela\textunderscore , de \textunderscore farrapo\textunderscore )
\section{Farpela}
\begin{itemize}
\item {Grp. gram.:f.}
\end{itemize}
\begin{itemize}
\item {Proveniência:(De \textunderscore farpa\textunderscore )}
\end{itemize}
Espécie de gancho agudo, em que terminam de um lado as agulhas de meia ou de croché.
\section{Farra}
\begin{itemize}
\item {Grp. gram.:f.}
\end{itemize}
\begin{itemize}
\item {Utilização:Bras. de San-Paulo}
\end{itemize}
O mesmo que \textunderscore lupanar\textunderscore .
\section{Farra}
\begin{itemize}
\item {Grp. gram.:f.}
\end{itemize}
Espécie de salmão, (\textunderscore salmo lavaretus\textunderscore ).
\section{Farracho}
\begin{itemize}
\item {Grp. gram.:m.}
\end{itemize}
\begin{itemize}
\item {Utilização:Bras}
\end{itemize}
Espécie de terçado, com que se mata o peixe que se pesca á noite, ao candeio.
(Por \textunderscore ferracho\textunderscore , de \textunderscore ferro\textunderscore )
\section{Farrageal}
\begin{itemize}
\item {Grp. gram.:m.}
\end{itemize}
\begin{itemize}
\item {Proveniência:(Do lat. \textunderscore farago\textunderscore )}
\end{itemize}
O mesmo ou melhor que \textunderscore ferregial\textunderscore . Cf. Castilho, \textunderscore Geórgicas\textunderscore , 165.
\section{Farragem}
\begin{itemize}
\item {Grp. gram.:f.}
\end{itemize}
\begin{itemize}
\item {Proveniência:(Do lat. \textunderscore farrago\textunderscore )}
\end{itemize}
Conjunto de coisas mal ordenadas.
\section{Farragoulo}
\begin{itemize}
\item {Grp. gram.:m.}
\end{itemize}
\begin{itemize}
\item {Utilização:Ant.}
\end{itemize}
\begin{itemize}
\item {Proveniência:(It. \textunderscore ferraiuolo\textunderscore )}
\end{itemize}
Gabão de mangas curtas.
\section{Farragulha}
\begin{itemize}
\item {Grp. gram.:m.}
\end{itemize}
\begin{itemize}
\item {Utilização:T. da Bairrada}
\end{itemize}
Homem muito diligente, fura-vidas.
(Cp. \textunderscore fagulha\textunderscore )
\section{Farrambamba}
\begin{itemize}
\item {Grp. gram.:f.}
\end{itemize}
\begin{itemize}
\item {Utilização:Bras. do N}
\end{itemize}
Fanfarronada.
\section{Farrancha}
\begin{itemize}
\item {Grp. gram.:f.}
\end{itemize}
\begin{itemize}
\item {Utilização:Prov.}
\end{itemize}
\begin{itemize}
\item {Utilização:beir.}
\end{itemize}
Espada velha, chanfalho.
(Por \textunderscore ferrancha\textunderscore , de \textunderscore ferro\textunderscore )
\section{Farrancho}
\begin{itemize}
\item {Grp. gram.:m.}
\end{itemize}
\begin{itemize}
\item {Utilização:Chul.}
\end{itemize}
Ranchada, que vai para romagem ou que vai divertir-se.
(Por \textunderscore faz-rancho\textunderscore , de \textunderscore fazer\textunderscore  + \textunderscore rancho\textunderscore )
\section{Farrancho}
\begin{itemize}
\item {Grp. gram.:m.}
\end{itemize}
\begin{itemize}
\item {Utilização:Prov.}
\end{itemize}
\begin{itemize}
\item {Utilização:beir.}
\end{itemize}
O mesmo que \textunderscore farrancha\textunderscore .
(Cp. \textunderscore farracho\textunderscore )
\section{Farrão}
\begin{itemize}
\item {Grp. gram.:m.}
\end{itemize}
(V.farragem)
\section{Farrapa}
\begin{itemize}
\item {Grp. gram.:f.}
\end{itemize}
\begin{itemize}
\item {Utilização:Prov.}
\end{itemize}
\begin{itemize}
\item {Utilização:minh.}
\end{itemize}
O mesmo que \textunderscore farrapagem\textunderscore .
\section{Farrapagem}
\begin{itemize}
\item {Grp. gram.:f.}
\end{itemize}
\begin{itemize}
\item {Proveniência:(De \textunderscore farrapo\textunderscore )}
\end{itemize}
Fardagem; farraparia.
\section{Farrapão}
\begin{itemize}
\item {Grp. gram.:m.}
\end{itemize}
Indivíduo andrajoso, vestido de farrapos, indigente.
\section{Farrapar}
\begin{itemize}
\item {Grp. gram.:v. t.}
\end{itemize}
(V.esfarrapar)
\section{Farraparia}
\begin{itemize}
\item {Grp. gram.:f.}
\end{itemize}
Grande porção de farrapos.
\section{Farrapeira}
\begin{itemize}
\item {Grp. gram.:f.}
\end{itemize}
Nome de uma música pop. da Bairrada.
Espécie de dança de roda.
Mulher que compra e vende farrapos, para fabríco de papel.
\section{Farrapeiro}
\begin{itemize}
\item {Grp. gram.:m.}
\end{itemize}
\begin{itemize}
\item {Utilização:Prov.}
\end{itemize}
\begin{itemize}
\item {Utilização:minh.}
\end{itemize}
\begin{itemize}
\item {Utilização:Prov.}
\end{itemize}
\begin{itemize}
\item {Utilização:minh.}
\end{itemize}
\begin{itemize}
\item {Proveniência:(De \textunderscore farrapo\textunderscore )}
\end{itemize}
O mesmo que \textunderscore trapeiro\textunderscore ^1.
Homem esfarrapado.
O diabo.
\section{Farrapilha}
\begin{itemize}
\item {Grp. gram.:m.  e  f.}
\end{itemize}
\begin{itemize}
\item {Proveniência:(De \textunderscore farrapo\textunderscore )}
\end{itemize}
O mesmo ou melhor que \textunderscore farroupilha\textunderscore .
\section{Farrapo}
\begin{itemize}
\item {Grp. gram.:m.}
\end{itemize}
\begin{itemize}
\item {Grp. gram.:Adj.}
\end{itemize}
\begin{itemize}
\item {Utilização:Des.}
\end{itemize}
Pedaço de pano rasgado ou muito usado.
Peça de vestuário, muito rôta e esfarrapada.
O mesmo que [[esfarrapado|esfarrapar]].
(Cast. \textunderscore harrapo\textunderscore )
\section{Farreguelo}
\begin{itemize}
\item {fónica:guê}
\end{itemize}
\begin{itemize}
\item {Grp. gram.:m.}
\end{itemize}
O mesmo que \textunderscore farragoulo\textunderscore . Cf. D. Bernardez, \textunderscore Lima\textunderscore , 265.
\section{Farrejal}
\begin{itemize}
\item {Grp. gram.:m.}
\end{itemize}
O mesmo que \textunderscore farrageal\textunderscore . Cf. Ficalho, \textunderscore Contos\textunderscore , 14 e 145.
\section{Farrejial}
\begin{itemize}
\item {Grp. gram.:m.}
\end{itemize}
\begin{itemize}
\item {Proveniência:(De \textunderscore farro\textunderscore )}
\end{itemize}
O mesmo que \textunderscore farrageal\textunderscore . Cf. Herculano, \textunderscore Hist. de Port.\textunderscore , IV, 239.
\section{Fárreo}
\begin{itemize}
\item {Grp. gram.:adj.}
\end{itemize}
\begin{itemize}
\item {Grp. gram.:M.}
\end{itemize}
\begin{itemize}
\item {Proveniência:(Lat. \textunderscore farreus\textunderscore )}
\end{itemize}
Relativo ao farro.
O mesmo que \textunderscore farro\textunderscore .
\section{Farricoco}
\begin{itemize}
\item {fónica:cô}
\end{itemize}
\begin{itemize}
\item {Grp. gram.:m.}
\end{itemize}
\begin{itemize}
\item {Utilização:Pop.}
\end{itemize}
Cada um dos que conduziam a tumba da Misericórdia.
Gato-pingado.
Indivíduo encapuzado, que acompanhava as procissões de penitência, tocando trombeta de espaço a espaço.
\section{Farrimónia}
\begin{itemize}
\item {Grp. gram.:f.}
\end{itemize}
\begin{itemize}
\item {Utilização:T. do Fundão}
\end{itemize}
O mesmo que \textunderscore ferramenta\textunderscore .
\section{Farripas}
\begin{itemize}
\item {Grp. gram.:f. pl.}
\end{itemize}
Cabelladura rala; grenha.
\section{Farro}
\begin{itemize}
\item {Grp. gram.:m.}
\end{itemize}
\begin{itemize}
\item {Proveniência:(Do lat. \textunderscore farreus\textunderscore . Cp. \textunderscore fárreo\textunderscore )}
\end{itemize}
Bolo para os sacrifícios, entre os Romanos.
Caldo de cevada.
Certo bolo de farinha.
Trigo candeal.
\section{Farroba}
\begin{itemize}
\item {fónica:rô}
\end{itemize}
\begin{itemize}
\item {Grp. gram.:f.}
\end{itemize}
O mesmo que \textunderscore alfarroba\textunderscore .
\section{Farrobeira}
\begin{itemize}
\item {Grp. gram.:f.}
\end{itemize}
O mesmo que \textunderscore alfarrobeira\textunderscore .
\section{Farromba}
\begin{itemize}
\item {Grp. gram.:m.}
\end{itemize}
\begin{itemize}
\item {Utilização:Prov.}
\end{itemize}
\begin{itemize}
\item {Utilização:bras}
\end{itemize}
O mesmo que \textunderscore farronca\textunderscore .
Aquelle que tem farronca. (Colhido no Fundão e em Minas Geraes)
\section{Farronca}
\begin{itemize}
\item {Grp. gram.:f.}
\end{itemize}
\begin{itemize}
\item {Grp. gram.:M.}
\end{itemize}
Voz grossa.
Jactância.
Ostentação.
Bazófia.
Aquelle que tem farronca ou bazófia.
\section{Farronfa}
\begin{itemize}
\item {Grp. gram.:f.}
\end{itemize}
\begin{itemize}
\item {Utilização:Pop.}
\end{itemize}
O mesmo que \textunderscore farronca\textunderscore .
\section{Farronfear}
\begin{itemize}
\item {Grp. gram.:v. i.}
\end{itemize}
\begin{itemize}
\item {Utilização:Pop.}
\end{itemize}
Usar farronfa; bravatear.
\section{Farronqueiro}
\begin{itemize}
\item {Grp. gram.:adj.}
\end{itemize}
Que tem \textunderscore farronca\textunderscore .
\section{Farroupa}
\begin{itemize}
\item {Grp. gram.:m.  e  f.}
\end{itemize}
O mesmo que \textunderscore farroupilha\textunderscore .
(Parece relacionar-se com \textunderscore farroupilha\textunderscore )
\section{Farroupilha}
\begin{itemize}
\item {Grp. gram.:m.  e  f.}
\end{itemize}
Indivíduo mal trajado, esfarrapado; miserável.
(Por \textunderscore farrapilha\textunderscore , de \textunderscore farrapo\textunderscore )
\section{Farroupinho}
\begin{itemize}
\item {Grp. gram.:m.}
\end{itemize}
\begin{itemize}
\item {Proveniência:(De \textunderscore farroupo\textunderscore )}
\end{itemize}
Pequeno porco.
\section{Farroupo}
\begin{itemize}
\item {Grp. gram.:m.}
\end{itemize}
\begin{itemize}
\item {Utilização:Ant.}
\end{itemize}
\begin{itemize}
\item {Utilização:Ant.}
\end{itemize}
Porco, que não tem mais de um anno.
Porco grande e castrado.
Carneiro velho e castrado.
\section{Farruco}
\begin{itemize}
\item {Grp. gram.:m.}
\end{itemize}
\begin{itemize}
\item {Utilização:Prov.}
\end{itemize}
\begin{itemize}
\item {Utilização:trasm.}
\end{itemize}
Feixe pequeno (de lenha ou de qualquer outra coisa).
\section{Farrumpeu}
\begin{itemize}
\item {Grp. gram.:m.}
\end{itemize}
\begin{itemize}
\item {Utilização:Chul.}
\end{itemize}
Chanfalho; o mesmo que \textunderscore farrusca\textunderscore .
\section{Farrupa}
\begin{itemize}
\item {Grp. gram.:f.}
\end{itemize}
Variedade de uva preta minhota.
\section{Farrusca}
\begin{itemize}
\item {Grp. gram.:f.}
\end{itemize}
\begin{itemize}
\item {Proveniência:(De \textunderscore farrusco\textunderscore )}
\end{itemize}
Chanfalho, espada ferrugenta.
Mascarra.
Nódoa de carvão ou de outra substância escura.
\section{Farrusco}
\begin{itemize}
\item {Grp. gram.:adj.}
\end{itemize}
Sujo de carvão ou de fulígem.
Escuro, negro.
(Por \textunderscore ferrusco\textunderscore , de \textunderscore ferro\textunderscore  ou de \textunderscore ferrugem\textunderscore )
\section{Farsa}
\begin{itemize}
\item {Grp. gram.:f.}
\end{itemize}
Peça theatral burlesca.
Acto ridículo, próprio de farsas.
Coisa burlesca.
Pantomima.
(Cp. cast. \textunderscore farsa\textunderscore , it. \textunderscore farsa\textunderscore )
\section{Farsada}
\begin{itemize}
\item {Grp. gram.:f.}
\end{itemize}
\begin{itemize}
\item {Proveniência:(De \textunderscore farsa\textunderscore )}
\end{itemize}
Acção burlesca.
Palhaçada.
\section{Farsalhão}
\begin{itemize}
\item {Grp. gram.:m.}
\end{itemize}
Farsa grande e de pouco mérito.
\section{Farsanta}
\begin{itemize}
\item {Grp. gram.:f.}
\end{itemize}
\begin{itemize}
\item {Utilização:Ant.}
\end{itemize}
Actriz, que representava farsas. Cp. J. Castilho, \textunderscore Manuelinas\textunderscore , 145.
\section{Farsante}
\begin{itemize}
\item {Grp. gram.:m.  e  f.}
\end{itemize}
Pessôa, que representa farsas.
Pessôa, que pratica actos ridículos ou burlescos, ou que graceja de modo chulo.
Pessôa sem seriedade.
\section{Farsantear}
\begin{itemize}
\item {Grp. gram.:v. i.}
\end{itemize}
Praticar actos ou dizer coisas próprias de farsante.
\section{Farsista}
\begin{itemize}
\item {Grp. gram.:m.}
\end{itemize}
\begin{itemize}
\item {Grp. gram.:Adj.}
\end{itemize}
Farsante.
Pantomimeiro.
Que graceja de modo chulo.
Pessôa sem seriedade.
\section{Farsola}
\begin{itemize}
\item {Grp. gram.:m.}
\end{itemize}
\begin{itemize}
\item {Proveniência:(De \textunderscore farsa\textunderscore )}
\end{itemize}
Fanfarrão.
Pessôa galhofeira, farsista.
\section{Farsolar}
\begin{itemize}
\item {Grp. gram.:v. i.}
\end{itemize}
Praticar actos de farsola.
Jactar-se.
\section{Farsolice}
\begin{itemize}
\item {Grp. gram.:f.}
\end{itemize}
Acto ou dito de farsola.
\section{Farsanga}
\begin{itemize}
\item {Grp. gram.:f.}
\end{itemize}
\begin{itemize}
\item {Proveniência:(Do pers. \textunderscore farsang\textunderscore )}
\end{itemize}
Medida itinerária da Pérsia.
\section{Farsilhão}
\begin{itemize}
\item {Grp. gram.:m.}
\end{itemize}
Parte da fivela, em que se introduz a extremidade de uma correia ou fita afivelada.
\section{Fartação}
\begin{itemize}
\item {Grp. gram.:f.}
\end{itemize}
\begin{itemize}
\item {Proveniência:(De \textunderscore fartar\textunderscore )}
\end{itemize}
O mesmo que \textunderscore enfartamento\textunderscore .
\section{Fartacaz}
\begin{itemize}
\item {Grp. gram.:f.}
\end{itemize}
O mesmo ou melhor que \textunderscore fatacaz\textunderscore .
\section{Fartadela}
\begin{itemize}
\item {Grp. gram.:f.}
\end{itemize}
\begin{itemize}
\item {Utilização:Fam.}
\end{itemize}
Acto ou effeito de fartar.
\section{Fartalejo}
\begin{itemize}
\item {Grp. gram.:m.}
\end{itemize}
O mesmo que \textunderscore farte\textunderscore .
\section{Fartamente}
\begin{itemize}
\item {Grp. gram.:adv.}
\end{itemize}
\begin{itemize}
\item {Proveniência:(De \textunderscore farto\textunderscore )}
\end{itemize}
Com fartura.
\section{Fartança}
\begin{itemize}
\item {Grp. gram.:f.}
\end{itemize}
\begin{itemize}
\item {Utilização:Des.}
\end{itemize}
O mesmo que \textunderscore fartura\textunderscore .
\section{Fartão}
\begin{itemize}
\item {Grp. gram.:m.}
\end{itemize}
\begin{itemize}
\item {Utilização:Fam.}
\end{itemize}
O mesmo que \textunderscore fartadela\textunderscore .
\section{Fartar}
\begin{itemize}
\item {Grp. gram.:v. t.}
\end{itemize}
\begin{itemize}
\item {Grp. gram.:Loc. adv.}
\end{itemize}
Saciar a fome ou a sêde de.
Encher.
Abarrotar.
Satisfazer.
Causar grande aborrecimento a; cansar: \textunderscore aquelle maçador fartou-me\textunderscore .
\textunderscore Que farte\textunderscore , em abundância, em barda.
Copiosamente. Cf. Camillo, \textunderscore Olho de Vidro\textunderscore , 121.
\section{Fartável}
\begin{itemize}
\item {Grp. gram.:adj.}
\end{itemize}
Que se póde fartar.
\section{Farta-velhaco}
\begin{itemize}
\item {Grp. gram.:m.}
\end{itemize}
\begin{itemize}
\item {Utilização:Chul.}
\end{itemize}
Variedade de ameixas.
\textunderscore Coisa de farta-velhaco\textunderscore , coisa grosseira e abundante.
\section{Farte}
\begin{itemize}
\item {Grp. gram.:m.}
\end{itemize}
\begin{itemize}
\item {Proveniência:(Do lat. \textunderscore fartum\textunderscore ?)}
\end{itemize}
Bolo de açúcar e amêndoas, envolto em farinha.
Nome de outros bolos, que contêm creme.
\section{Fártel}
\begin{itemize}
\item {Grp. gram.:m.}
\end{itemize}
\begin{itemize}
\item {Utilização:Ant.}
\end{itemize}
O mesmo que \textunderscore farte\textunderscore .
\section{Fártem}
\begin{itemize}
\item {Grp. gram.:m.}
\end{itemize}
(V.farte)
\section{Farteza}
\begin{itemize}
\item {Grp. gram.:f.}
\end{itemize}
(V.fartura)
\section{Farto}
\begin{itemize}
\item {Grp. gram.:adj.}
\end{itemize}
Saciado; cheio; empanturrado.
Nédio; nutrido.
Abundante: \textunderscore colheita farta\textunderscore .
Que tem alguma coisa em grande quantidade:«\textunderscore sou já tão farto de fome como os outros de comer.\textunderscore »G. Vicente, \textunderscore Quem tem farelos\textunderscore .
Enfastiado, aborrecido: \textunderscore estou farto de te aturar\textunderscore .
\section{Fartote}
\begin{itemize}
\item {Grp. gram.:m.}
\end{itemize}
\begin{itemize}
\item {Utilização:Pop.}
\end{itemize}
\begin{itemize}
\item {Proveniência:(De \textunderscore fartar\textunderscore )}
\end{itemize}
Enchimento de barriga; barrigada.
Grande porção.
\section{Fartum}
\begin{itemize}
\item {Grp. gram.:m.}
\end{itemize}
Cheiro, resultante de ranço.
Bafio.
Cheiro nauseabundo.
\section{Fartura}
\begin{itemize}
\item {Grp. gram.:f.}
\end{itemize}
\begin{itemize}
\item {Grp. gram.:Pl.}
\end{itemize}
\begin{itemize}
\item {Proveniência:(Lat. \textunderscore fartura\textunderscore )}
\end{itemize}
Estado de farto.
Abundância.
Rolos, feitos de farinha e azeite, que se vendem ordinariamente em barracas de feira.
\section{Farum}
\begin{itemize}
\item {Grp. gram.:m.}
\end{itemize}
\begin{itemize}
\item {Utilização:Prov.}
\end{itemize}
\begin{itemize}
\item {Utilização:alg.}
\end{itemize}
\begin{itemize}
\item {Utilização:Prov.}
\end{itemize}
\begin{itemize}
\item {Utilização:minh.}
\end{itemize}
\begin{itemize}
\item {Proveniência:(De \textunderscore faro\textunderscore )}
\end{itemize}
Mau cheiro.
Cheiro, que o mosto exhala nos lagares.
\section{Fascal}
\begin{itemize}
\item {Grp. gram.:m.}
\end{itemize}
\begin{itemize}
\item {Utilização:Prov.}
\end{itemize}
\begin{itemize}
\item {Utilização:minh.}
\end{itemize}
Monte de espigas.
Monte de feixes de palha.
Terreiro, em que se junta a lenha para o lume.
(Cast. \textunderscore fascal\textunderscore )
\section{Fasces}
\begin{itemize}
\item {Grp. gram.:m. pl.}
\end{itemize}
\begin{itemize}
\item {Utilização:Ant.}
\end{itemize}
\begin{itemize}
\item {Proveniência:(Do lat. \textunderscore fascis\textunderscore )}
\end{itemize}
Feixe de varas, com que os lictores acompanhavam os cônsules, como insígnia do direito, que estes tinham, de punir.
\section{Fasciação}
\begin{itemize}
\item {Grp. gram.:f.}
\end{itemize}
\begin{itemize}
\item {Utilização:Bot.}
\end{itemize}
\begin{itemize}
\item {Proveniência:(Do lat. \textunderscore fascia\textunderscore )}
\end{itemize}
Dilatação anormal do caule e das suas divisões, quando tomam a fórma de lâminas ou faixas.
\section{Fasciculado}
\begin{itemize}
\item {Grp. gram.:adj.}
\end{itemize}
Disposto em fascículos ou em feixes; que tem fórma de feixe.
\section{Fascicular}
\begin{itemize}
\item {Grp. gram.:adj.}
\end{itemize}
Que tem a fórma de fascículo.
\section{Fascículo}
\begin{itemize}
\item {Grp. gram.:m.}
\end{itemize}
\begin{itemize}
\item {Proveniência:(Lat. \textunderscore fasciculus\textunderscore )}
\end{itemize}
Pequeno feixe.
Porção de ervas ou varas, que podem transportar-se debaixo do braço.
Gavela; paveia.
Folheto de uma obra que se publica por partes; caderneta.
Porção de estames, ligados pelos filetes.
Fórma de inflorescência, semelhante ao corymbo.
Conjunto de pelos ou cabellos, que nascem fóra do lugar próprio.
\section{Fascinação}
\begin{itemize}
\item {Grp. gram.:f.}
\end{itemize}
\begin{itemize}
\item {Proveniência:(Lat. \textunderscore fascinatio\textunderscore )}
\end{itemize}
Acto ou effeito de fascinar.
\section{Fascinador}
\begin{itemize}
\item {Grp. gram.:adj.}
\end{itemize}
\begin{itemize}
\item {Grp. gram.:M.}
\end{itemize}
\begin{itemize}
\item {Proveniência:(Lat. \textunderscore fascinator\textunderscore )}
\end{itemize}
Que fascina.
Aquelle que fascina.
\section{Fascinante}
\begin{itemize}
\item {Grp. gram.:adj.}
\end{itemize}
\begin{itemize}
\item {Proveniência:(Lat. \textunderscore fascinans\textunderscore )}
\end{itemize}
Que fascina.
\section{Fascinar}
\begin{itemize}
\item {Grp. gram.:v. t.}
\end{itemize}
\begin{itemize}
\item {Utilização:Fig.}
\end{itemize}
\begin{itemize}
\item {Proveniência:(Lat. \textunderscore fascinare\textunderscore )}
\end{itemize}
Dominar por encantamento.
Prender com feitiços.
Dar quebranto ou mau olhado a.
Encantar.
Deslumbrar.
Illudir.
Attrahir irresistivelmente: \textunderscore belleza que fascina\textunderscore .
\section{Fascínio}
\begin{itemize}
\item {Grp. gram.:m.}
\end{itemize}
\begin{itemize}
\item {Utilização:Des.}
\end{itemize}
O mesmo que \textunderscore fascinação\textunderscore .
Mau olhado; encantamento. Cf. Azevedo, \textunderscore Correcção de Abusos\textunderscore .
(Cp. lat. \textunderscore fascinum\textunderscore , de \textunderscore fascinare\textunderscore )
\section{Fascíola}
\begin{itemize}
\item {Grp. gram.:f.}
\end{itemize}
\begin{itemize}
\item {Proveniência:(Lat. \textunderscore fasciola\textunderscore )}
\end{itemize}
Verme intestinal, achatado, que se encontra principalmente nos canaes biliares do carneiro.
Espécie de plantas cryptogâmicas.
\section{Fasciolar}
\begin{itemize}
\item {Grp. gram.:f.}
\end{itemize}
\begin{itemize}
\item {Proveniência:(De \textunderscore fascíola\textunderscore )}
\end{itemize}
Gênero de conchas univalves, fusiformes.
\section{Fasciolária}
\begin{itemize}
\item {Grp. gram.:f.}
\end{itemize}
\begin{itemize}
\item {Proveniência:(De \textunderscore fascíola\textunderscore )}
\end{itemize}
Gênero de conchas univalves, fusiformes.
\section{Fasco}
\begin{itemize}
\item {Grp. gram.:m.}
\end{itemize}
\begin{itemize}
\item {Utilização:Prov.}
\end{itemize}
O mesmo que \textunderscore caruma\textunderscore .
\section{Fasquia}
\begin{itemize}
\item {Grp. gram.:f.}
\end{itemize}
Tira de madeira.
Parte estreita e alongada que se separou de um tronco de madeira, serrando; ripa.
(Do ár.)
\section{Fasquiado}
\begin{itemize}
\item {Grp. gram.:m.}
\end{itemize}
\begin{itemize}
\item {Proveniência:(De \textunderscore fasquiar\textunderscore )}
\end{itemize}
Obra de fasquia.
\section{Fasquiar}
\begin{itemize}
\item {Grp. gram.:v. t.}
\end{itemize}
Serrar em fasquias.
Guarnecer de fasquias.
\section{Fasquio}
\begin{itemize}
\item {Grp. gram.:m.}
\end{itemize}
\begin{itemize}
\item {Utilização:Pop.}
\end{itemize}
Porção de fasquias.
\section{Fasta}
\begin{itemize}
\item {Grp. gram.:interj.}
\end{itemize}
Voz, que os carreiros dirigem aos bois, para os fazer recuar ou desviar. Cf. \textunderscore Eufrosina\textunderscore , 332.
(Por \textunderscore afasta\textunderscore , de \textunderscore afastar\textunderscore )
\section{Fasta}
\begin{itemize}
\item {Grp. gram.:adv.}
\end{itemize}
\begin{itemize}
\item {Utilização:Ant.}
\end{itemize}
O mesmo que \textunderscore até\textunderscore .
(Cast. ant. \textunderscore fasta\textunderscore , mod. \textunderscore hasta\textunderscore )
\section{Fastidiosamente}
\begin{itemize}
\item {Grp. gram.:adv.}
\end{itemize}
De modo fastidioso.
\section{Fastidioso}
\begin{itemize}
\item {Grp. gram.:adj.}
\end{itemize}
\begin{itemize}
\item {Proveniência:(Lat. \textunderscore fastidiosus\textunderscore )}
\end{itemize}
Que produz fastio.
Que causa tédio; enfadonho.
Impertinente.
\section{Fastiento}
\begin{itemize}
\item {Grp. gram.:adj.}
\end{itemize}
Fastidioso; que tem fastio.
Rabugento.
\section{Fastigiado}
\begin{itemize}
\item {Grp. gram.:adj.}
\end{itemize}
\begin{itemize}
\item {Proveniência:(Lat. \textunderscore fastigiatus\textunderscore )}
\end{itemize}
Diz-se das árvores altas e frondosas.
\section{Fastigiária}
\begin{itemize}
\item {Grp. gram.:f.}
\end{itemize}
\begin{itemize}
\item {Proveniência:(De \textunderscore fastígio\textunderscore )}
\end{itemize}
Gênero de algas marinhas.
\section{Fastígio}
\begin{itemize}
\item {Grp. gram.:m.}
\end{itemize}
\begin{itemize}
\item {Utilização:Ant.}
\end{itemize}
\begin{itemize}
\item {Proveniência:(Lat. \textunderscore fastigium\textunderscore )}
\end{itemize}
O ponto mais elevado; cume; eminência.
Disposição dos ramos de uma planta, que, elevando-se de um ponto commum á mesma altura, formam um plano horizontal.
Ornato, que se collocava no alto dos templos romanos.
\section{Fastigioso}
\begin{itemize}
\item {Grp. gram.:adj.}
\end{itemize}
Que está no fastígio ou em posição evidente.
\section{Fastio}
\begin{itemize}
\item {Grp. gram.:m.}
\end{itemize}
\begin{itemize}
\item {Proveniência:(Do lat. \textunderscore fastidium\textunderscore )}
\end{itemize}
Repugnância; aversão.
Tédio; aborrecimento.
Falta de appetite.
\section{Fastioso}
\begin{itemize}
\item {Grp. gram.:adj.}
\end{itemize}
(V.fastidioso)
\section{Fasto}
\begin{itemize}
\item {Grp. gram.:adj.}
\end{itemize}
\begin{itemize}
\item {Utilização:Ant.}
\end{itemize}
\begin{itemize}
\item {Grp. gram.:M.}
\end{itemize}
\begin{itemize}
\item {Proveniência:(Lat. \textunderscore fastus\textunderscore )}
\end{itemize}
Dizia-se dos dias, em que era permittido exercer certas jurisdicções, entre os Romanos.
Não feriado.
Fausto; ostentação; pompa.
\section{Fastos}
\begin{itemize}
\item {Grp. gram.:m. pl.}
\end{itemize}
\begin{itemize}
\item {Proveniência:(Lat. \textunderscore fasti\textunderscore )}
\end{itemize}
Livros, que, entre os Romanos, indicavam as festas públicas, e em que se registavam os factos memoráveis, que iam occorrendo.
Annaes.
Calendário.
História: \textunderscore os fastos da Igreja\textunderscore .
\section{Fastosamente}
\begin{itemize}
\item {Grp. gram.:adv.}
\end{itemize}
De modo fastoso.
\section{Fastoso}
\begin{itemize}
\item {Grp. gram.:adj.}
\end{itemize}
\begin{itemize}
\item {Grp. gram.:M.}
\end{itemize}
Que tem fasto, ostentação.
Pomposo.
Arrogante.
Aquelle que gosta de fasto, de luxo.
\section{Fastuoso}
\begin{itemize}
\item {Grp. gram.:adj.}
\end{itemize}
\begin{itemize}
\item {Grp. gram.:M.}
\end{itemize}
Que tem fasto, ostentação.
Pomposo.
Arrogante.
Aquelle que gosta de fasto, de luxo.
\section{Fata}
\begin{itemize}
\item {Grp. gram.:f.}
\end{itemize}
Árvore do Congo.
\section{Fataça}
\begin{itemize}
\item {Grp. gram.:f.}
\end{itemize}
Taínha grande.
\section{Fatacaz}
\begin{itemize}
\item {Grp. gram.:f.}
\end{itemize}
\begin{itemize}
\item {Utilização:Fam.}
\end{itemize}
Porção grande; naco; tracanaz.
(Contr. de \textunderscore fartacaz\textunderscore ?)
\section{Fatagear}
\begin{itemize}
\item {Grp. gram.:v. i.}
\end{itemize}
\begin{itemize}
\item {Proveniência:(De \textunderscore fatagem\textunderscore )}
\end{itemize}
Mexer em fato; revolver roupas.
\section{Fatagem}
\begin{itemize}
\item {Grp. gram.:f.}
\end{itemize}
\begin{itemize}
\item {Utilização:Prov.}
\end{itemize}
\begin{itemize}
\item {Utilização:minh.}
\end{itemize}
\begin{itemize}
\item {Proveniência:(De \textunderscore fato\textunderscore )}
\end{itemize}
Acto de fatagear.
Intestinos de animaes.
\section{Fatal}
\begin{itemize}
\item {Grp. gram.:adj.}
\end{itemize}
\begin{itemize}
\item {Proveniência:(Lat. \textunderscore fatalis\textunderscore )}
\end{itemize}
Determinado pelo fado.
Irrevogável; infallível.
Que produz desgraças, funesto, nocivo: \textunderscore mulher fatal\textunderscore .
\section{Fatalidade}
\begin{itemize}
\item {Grp. gram.:f.}
\end{itemize}
\begin{itemize}
\item {Proveniência:(Lat. \textunderscore fatalitas\textunderscore )}
\end{itemize}
Qualidade daquillo que é fatal.
Destino.
Acontecimento funesto, nocivo.
\section{Fatalismo}
\begin{itemize}
\item {Grp. gram.:m.}
\end{itemize}
\begin{itemize}
\item {Proveniência:(De \textunderscore fatal\textunderscore )}
\end{itemize}
Systema dos que tudo attribuem ao destino ou á fatalidade, negando o livre arbítrio.
\section{Fatalista}
\begin{itemize}
\item {Grp. gram.:adj.}
\end{itemize}
\begin{itemize}
\item {Grp. gram.:M.  e  f.}
\end{itemize}
\begin{itemize}
\item {Proveniência:(De \textunderscore fatal\textunderscore )}
\end{itemize}
Relativo a fatalismo.
Pessôa partidaria do fatalismo.
\section{Fatalmente}
\begin{itemize}
\item {Grp. gram.:adv.}
\end{itemize}
De modo fatal.
Necessariamente.
\section{Fatana}
\begin{itemize}
\item {Grp. gram.:f.}
\end{itemize}
\begin{itemize}
\item {Utilização:Prov.}
\end{itemize}
\begin{itemize}
\item {Utilização:alg.}
\end{itemize}
\begin{itemize}
\item {Proveniência:(De \textunderscore fato\textunderscore )}
\end{itemize}
Invólucro total da maçaroca do milho.
\section{Fatanar}
\begin{itemize}
\item {Grp. gram.:v. t.}
\end{itemize}
\begin{itemize}
\item {Utilização:Prov.}
\end{itemize}
\begin{itemize}
\item {Utilização:alg.}
\end{itemize}
Tirar a fatana a (o milho); descamisar.
\section{Fatanisca}
\begin{itemize}
\item {Grp. gram.:f.}
\end{itemize}
\begin{itemize}
\item {Utilização:Prov.}
\end{itemize}
\begin{itemize}
\item {Utilização:trasm.}
\end{itemize}
\begin{itemize}
\item {Utilização:Prov.}
\end{itemize}
\begin{itemize}
\item {Utilização:minh.}
\end{itemize}
Isca de bacalhau, envolvida em ovos e farinha e frita depois.
\textunderscore Pôr em fatanisca\textunderscore , espedaçar.
Reduzir a farrapos.
\section{Fatão}
\begin{itemize}
\item {Grp. gram.:m.}
\end{itemize}
\begin{itemize}
\item {Utilização:Prov.}
\end{itemize}
\begin{itemize}
\item {Utilização:minh.}
\end{itemize}
Ameixa grande, sôbre o comprido.
\section{Fate}
\begin{itemize}
\item {Grp. gram.:m.}
\end{itemize}
\begin{itemize}
\item {Utilização:Bras. do N}
\end{itemize}
O mesmo que \textunderscore diabo\textunderscore .
\section{Fateco}
\begin{itemize}
\item {Grp. gram.:m.}
\end{itemize}
\begin{itemize}
\item {Utilização:Prov.}
\end{itemize}
\begin{itemize}
\item {Utilização:alg.}
\end{itemize}
\begin{itemize}
\item {Utilização:Deprec.}
\end{itemize}
Fato, fatiota.
\section{Fateiro}
\begin{itemize}
\item {Grp. gram.:adj.}
\end{itemize}
\begin{itemize}
\item {Utilização:Prov.}
\end{itemize}
\begin{itemize}
\item {Utilização:trasm.}
\end{itemize}
\begin{itemize}
\item {Grp. gram.:M.}
\end{itemize}
\begin{itemize}
\item {Utilização:Bras. do N}
\end{itemize}
Relativo a fato.
Próprio para guardar fato.
\textunderscore Arca fateira\textunderscore , arca para guardar fato.
Aquelle que vende vísceras de gado.
\section{Fateixa}
\begin{itemize}
\item {Grp. gram.:f.}
\end{itemize}
\begin{itemize}
\item {Utilização:Prov.}
\end{itemize}
\begin{itemize}
\item {Utilização:trasm.}
\end{itemize}
\begin{itemize}
\item {Utilização:Fig.}
\end{itemize}
\begin{itemize}
\item {Grp. gram.:Pl.}
\end{itemize}
\begin{itemize}
\item {Utilização:Prov.}
\end{itemize}
\begin{itemize}
\item {Utilização:trasm.}
\end{itemize}
Espécie de âncora, para fundear pequenos barcos.
Gancho, arpão, com que se tiram objectos do fundo da água.
Utensílio de ferro, em que se penduram carnes.
Feixe de colmo ou palha, que cabe numa mão.
O mesmo que \textunderscore mão\textunderscore .
Farrapos velhos.
\section{Fateixar}
\begin{itemize}
\item {Grp. gram.:v. t.}
\end{itemize}
Agarrar com fateixa.
\section{Fateosim}
\begin{itemize}
\item {Grp. gram.:adj.}
\end{itemize}
\begin{itemize}
\item {Grp. gram.:F.}
\end{itemize}
O mesmo que \textunderscore emphytêutico\textunderscore .
O mesmo que \textunderscore emphyteuse\textunderscore .
\section{Fateusim}
\begin{itemize}
\item {Grp. gram.:adj.}
\end{itemize}
\begin{itemize}
\item {Grp. gram.:F.}
\end{itemize}
O mesmo que \textunderscore emphytêutico\textunderscore .
O mesmo que \textunderscore emphyteuse\textunderscore .
\section{Fatia}
\begin{itemize}
\item {Grp. gram.:f.}
\end{itemize}
\begin{itemize}
\item {Utilização:Pop.}
\end{itemize}
\begin{itemize}
\item {Proveniência:(Do ár. \textunderscore fatita\textunderscore ?)}
\end{itemize}
Pedaço chato, delgado e mais ou menos comprido.
Talhada.
Pedaço de pão, de presunto, etc., cortado com faca, ficando em fórma de lâmina, mais ou menos espêssa.
Bom lucro; grande quinhão.
Vantagem.
\section{Fatiar}
\begin{itemize}
\item {Grp. gram.:v. t.}
\end{itemize}
Cortar em fatias.
Fazer em pedaços.
\section{Faticano}
\begin{itemize}
\item {Grp. gram.:adj.}
\end{itemize}
\begin{itemize}
\item {Utilização:Poét.}
\end{itemize}
\begin{itemize}
\item {Proveniência:(Lat. \textunderscore faticanus\textunderscore )}
\end{itemize}
Que annuncia o futuro, que prophetiza.
\section{Faticeira}
\begin{itemize}
\item {Grp. gram.:f.}
\end{itemize}
Peixe marítimo, sarapintado e em fórma de cação.
Rêde, empregada pelos pescadores do Doiro na pesca da solha, linguado, etc.
(Por \textunderscore fataceira\textunderscore , de \textunderscore fataça\textunderscore ?)
\section{Fatidicamente}
\begin{itemize}
\item {Grp. gram.:adv.}
\end{itemize}
De modo fatídico.
\section{Fatídico}
\begin{itemize}
\item {Grp. gram.:adj.}
\end{itemize}
\begin{itemize}
\item {Proveniência:(Lat. \textunderscore fatidicus\textunderscore )}
\end{itemize}
Que prediz o futuro.
Fatal; trágico.
\section{Fatífero}
\begin{itemize}
\item {Grp. gram.:adj.}
\end{itemize}
\begin{itemize}
\item {Proveniência:(Lat. \textunderscore fatifer\textunderscore )}
\end{itemize}
Mortífero.
\section{Fatiga}
\begin{itemize}
\item {Grp. gram.:f.}
\end{itemize}
\begin{itemize}
\item {Utilização:Prov.}
\end{itemize}
\begin{itemize}
\item {Utilização:beir.}
\end{itemize}
\begin{itemize}
\item {Utilização:trasm.}
\end{itemize}
O mesmo que \textunderscore fatia\textunderscore .
Pequena refeição dos trabalhadores ruraes, entre o almôço e o jantar.
\section{Fatigador}
\begin{itemize}
\item {Grp. gram.:adj.}
\end{itemize}
\begin{itemize}
\item {Grp. gram.:M.}
\end{itemize}
Que fatiga.
Aquelle que fatiga.
\section{Fatigamento}
\begin{itemize}
\item {Grp. gram.:m.}
\end{itemize}
(V.fadiga)
\section{Fatigante}
\begin{itemize}
\item {Grp. gram.:adj.}
\end{itemize}
\begin{itemize}
\item {Proveniência:(Lat. \textunderscore fatigans\textunderscore )}
\end{itemize}
Que fatiga.
\section{Fatigar}
\begin{itemize}
\item {Grp. gram.:v. t.}
\end{itemize}
\begin{itemize}
\item {Proveniência:(Lat. \textunderscore fatigare\textunderscore )}
\end{itemize}
Causar fadiga a.
Enfastiar, importunar.
Cansar.
\section{Fatigar}
\begin{itemize}
\item {Grp. gram.:v. i.}
\end{itemize}
\begin{itemize}
\item {Utilização:Prov.}
\end{itemize}
\begin{itemize}
\item {Utilização:beir.}
\end{itemize}
Comer a fatiga. (Colhido em Seia)
\section{Fatigoso}
\begin{itemize}
\item {Grp. gram.:adj.}
\end{itemize}
(V.fatigante)
\section{Fatiloquente}
\begin{itemize}
\item {fónica:cu-en}
\end{itemize}
\begin{itemize}
\item {Grp. gram.:adj.}
\end{itemize}
Que prediz o futuro; fatídico.
\section{Fatíloquo}
\begin{itemize}
\item {Grp. gram.:adj.}
\end{itemize}
\begin{itemize}
\item {Proveniência:(Lat. \textunderscore fatitoquus\textunderscore )}
\end{itemize}
Que prediz o futuro; fatídico.
\section{Fatimitas}
\begin{itemize}
\item {Grp. gram.:m. pl.}
\end{itemize}
Tríbo árabe, que na Idade-Média constituia um império no Magreb. Cf. Herculano, \textunderscore Hist. de Port.\textunderscore , I, 94.
\section{Fatiota}
\begin{itemize}
\item {Grp. gram.:f.}
\end{itemize}
\begin{itemize}
\item {Utilização:Ext.}
\end{itemize}
\begin{itemize}
\item {Proveniência:(Do rad. de \textunderscore fato\textunderscore )}
\end{itemize}
Farpela; fato.
Farraparia.
\section{Fato}
\begin{itemize}
\item {Grp. gram.:m.}
\end{itemize}
\begin{itemize}
\item {Utilização:Prov.}
\end{itemize}
\begin{itemize}
\item {Utilização:Prov.}
\end{itemize}
\begin{itemize}
\item {Utilização:trasm.}
\end{itemize}
\begin{itemize}
\item {Utilização:Des.}
\end{itemize}
\begin{itemize}
\item {Utilização:Bras. do N}
\end{itemize}
Rebanho.
Roupa; vestuário.
Intestinos.
Bando, quadrilha.
Móveis.
Visceras de gado.
(Cp. cast. \textunderscore hato\textunderscore , do germ.)
\section{Fatoco}
\begin{itemize}
\item {fónica:tô}
\end{itemize}
\begin{itemize}
\item {Grp. gram.:m.}
\end{itemize}
\begin{itemize}
\item {Utilização:Prov.}
\end{itemize}
\begin{itemize}
\item {Utilização:minh.}
\end{itemize}
Junção ou empastamento de coisas, que deviam estar desunidas: \textunderscore fatoco de lan\textunderscore , porção de lan empastada.
\section{Fatracaz}
\begin{itemize}
\item {Grp. gram.:m.}
\end{itemize}
\begin{itemize}
\item {Utilização:Prov.}
\end{itemize}
\begin{itemize}
\item {Utilização:beir.}
\end{itemize}
O mesmo que \textunderscore fatacaz\textunderscore . (Colhido em San-Pedro do Sul)
(Cp. \textunderscore fartacaz\textunderscore )
\section{Fátsia}
\begin{itemize}
\item {Grp. gram.:f.}
\end{itemize}
Planta araliácea.
\section{Fatuamente}
\begin{itemize}
\item {Grp. gram.:adv.}
\end{itemize}
De modo fátuo.
\section{Fatuidade}
\begin{itemize}
\item {Grp. gram.:f.}
\end{itemize}
\begin{itemize}
\item {Proveniência:(Lat. \textunderscore fatuitas\textunderscore )}
\end{itemize}
Qualidade de quem é fátuo.
\section{Fátuo}
\begin{itemize}
\item {Grp. gram.:adj.}
\end{itemize}
\begin{itemize}
\item {Utilização:Fig.}
\end{itemize}
\begin{itemize}
\item {Proveniência:(Lat. \textunderscore fatuus\textunderscore )}
\end{itemize}
Muito estulto.
Néscio.
Petulante.
Presumido.
\textunderscore Fogo fátuo\textunderscore , exhalações phosphóricas, luzes que duram pouco.
Brilho transitório.
Prazer ou glória, que se extingue breve.
\section{Faucal}
\begin{itemize}
\item {Grp. gram.:adj.}
\end{itemize}
Relativo a fauce.
\section{Fauce}
\begin{itemize}
\item {Grp. gram.:f.}
\end{itemize}
\begin{itemize}
\item {Utilização:Bot.}
\end{itemize}
\begin{itemize}
\item {Proveniência:(Do lat. \textunderscore fauces\textunderscore )}
\end{itemize}
A parte superior e interior da goéla, junto á raíz da língua, onde o alimento começa a descer ao estômago.
Garganta.
Goéla de animal.
Extremidade do tubo da corolla.
\section{Faúla}
\begin{itemize}
\item {Grp. gram.:f.}
\end{itemize}
\begin{itemize}
\item {Utilização:Prov.}
\end{itemize}
\begin{itemize}
\item {Utilização:minh.}
\end{itemize}
O mesmo que \textunderscore fagulha\textunderscore .
Caruma sêca.
\section{Faulante}
\begin{itemize}
\item {fónica:fa-u}
\end{itemize}
\begin{itemize}
\item {Grp. gram.:adj.}
\end{itemize}
Que faúla. Cf. Camillo, \textunderscore Volcões\textunderscore , 212.
\section{Faular}
\begin{itemize}
\item {fónica:fa-u}
\end{itemize}
\begin{itemize}
\item {Grp. gram.:v. t.}
\end{itemize}
\begin{itemize}
\item {Grp. gram.:V. i.}
\end{itemize}
Lançar em fórma de faúlas.
Deitar faúlas, ardendo:«\textunderscore o perfumador, em que faulavam as cinzas das outras\textunderscore  (cartas)». Camillo, \textunderscore Caveira\textunderscore , 322.
\section{Faúlha}
\begin{itemize}
\item {Grp. gram.:f.}
\end{itemize}
\begin{itemize}
\item {Utilização:Prov.}
\end{itemize}
\begin{itemize}
\item {Utilização:minh.}
\end{itemize}
\begin{itemize}
\item {Proveniência:(Do lat. \textunderscore favilla\textunderscore )}
\end{itemize}
O mesmo que \textunderscore fagulha\textunderscore .
A parte mais subtil da farinha, que se evola na peneiração.
Caruma sêca, faúla.
\section{Faulhento}
\begin{itemize}
\item {fónica:fa-u}
\end{itemize}
\begin{itemize}
\item {Grp. gram.:adj.}
\end{itemize}
Que lança faúlhas.
Que expede pó subtil.
\section{Fáuna}
\begin{itemize}
\item {Grp. gram.:f.}
\end{itemize}
\begin{itemize}
\item {Proveniência:(De \textunderscore Fauna\textunderscore , n. p.)}
\end{itemize}
Conjunto de animaes, próprios de uma região ou de um período geológico.
\section{Fauniano}
\begin{itemize}
\item {Grp. gram.:adj.}
\end{itemize}
Relativo á fauna.
\section{Faunígena}
\begin{itemize}
\item {Grp. gram.:m.}
\end{itemize}
\begin{itemize}
\item {Utilização:Poét.}
\end{itemize}
\begin{itemize}
\item {Proveniência:(Lat. \textunderscore faunigena\textunderscore )}
\end{itemize}
Descendente de Fauno.
Italiano.
\section{Fauno}
\begin{itemize}
\item {Grp. gram.:m.}
\end{itemize}
\begin{itemize}
\item {Proveniência:(De \textunderscore Fauno\textunderscore , n. p.)}
\end{itemize}
Lepidóptero diurno.
\section{Fausel}
\begin{itemize}
\item {Grp. gram.:m.}
\end{itemize}
Aveleira da Índia.
\section{Faustiano}
\begin{itemize}
\item {Grp. gram.:adj.}
\end{itemize}
O mesmo ou melhor que \textunderscore faustino\textunderscore .
\section{Faustino}
\begin{itemize}
\item {Grp. gram.:adj.}
\end{itemize}
Relativo ao \textunderscore Fausto\textunderscore  de Goethe:«\textunderscore ...a propósito dos contendores da questão faustina...\textunderscore »Th. Braga.
\section{Fausto}
\begin{itemize}
\item {Grp. gram.:adj.}
\end{itemize}
\begin{itemize}
\item {Grp. gram.:M.}
\end{itemize}
\begin{itemize}
\item {Proveniência:(Lat. \textunderscore faustus\textunderscore )}
\end{itemize}
Venturoso, ditoso.
Próspero.
Agradável.
(No sentido de ostentação ou grande luxo, é talvez corruptela de \textunderscore fasto\textunderscore )(V.fasto)
\section{Faustoso}
\begin{itemize}
\item {Grp. gram.:adj.}
\end{itemize}
\begin{itemize}
\item {Proveniência:(De \textunderscore fausto\textunderscore )}
\end{itemize}
(V.fastuoso)
\section{Faustuoso}
\begin{itemize}
\item {Grp. gram.:adj.}
\end{itemize}
O mesmo que \textunderscore fastoso\textunderscore .
\section{Fauta}
\begin{itemize}
\item {Grp. gram.:f.}
\end{itemize}
\begin{itemize}
\item {Utilização:Des.}
\end{itemize}
Expressão, us. no jôgo da péla, para designar falta ou falha de partido.
(Alter. de \textunderscore falta\textunderscore )
\section{Fautor}
\begin{itemize}
\item {Grp. gram.:adj.}
\end{itemize}
\begin{itemize}
\item {Grp. gram.:M.}
\end{itemize}
\begin{itemize}
\item {Proveniência:(Lat. \textunderscore fautor\textunderscore )}
\end{itemize}
Que favorece, promove ou determina.
Aquelle que promove, ou fomenta, ou é causa, ou auxilia.
\section{Fautoria}
\begin{itemize}
\item {Grp. gram.:f.}
\end{itemize}
\begin{itemize}
\item {Proveniência:(De \textunderscore fautor\textunderscore )}
\end{itemize}
Acto de favorecer, promover ou auxiliar.
\section{Fautorizar}
\begin{itemize}
\item {Grp. gram.:v. t.}
\end{itemize}
Sêr fautor de.
Auxiliar; proteger.
\section{Fautriz}
(\textunderscore fem.\textunderscore  de \textunderscore fautor\textunderscore )
\section{Fava}
\begin{itemize}
\item {Grp. gram.:f.}
\end{itemize}
\begin{itemize}
\item {Utilização:Bras}
\end{itemize}
\begin{itemize}
\item {Grp. gram.:Loc.}
\end{itemize}
\begin{itemize}
\item {Utilização:fam.}
\end{itemize}
\begin{itemize}
\item {Grp. gram.:Loc.}
\end{itemize}
\begin{itemize}
\item {Utilização:fam.}
\end{itemize}
\begin{itemize}
\item {Grp. gram.:Pl. Loc.}
\end{itemize}
\begin{itemize}
\item {Utilização:fam.}
\end{itemize}
\begin{itemize}
\item {Proveniência:(Lat. \textunderscore faba\textunderscore )}
\end{itemize}
Planta leguminosa, hortense.
Vagem desta planta.
Designação de várias plantas, mais ou menos semelhantes àquella.
Nome de vários objectos, de configuração semelhante á da semente da fava.
Árvore silvestre, que dá bôa madeira.
Doença, caraterizada por inchação, no céu da bôca dos equídeos.
\textunderscore Mandar á fava\textunderscore , repellir com enfado.
\textunderscore Vá á fava\textunderscore , vá bugiar, vá pentear macacos, deixe-me.
\textunderscore Fava preta\textunderscore , voto de reprovação.
\textunderscore Favas contadas\textunderscore , coisa certa, infallivel, inevitável.
\section{Fava-assaria}
\begin{itemize}
\item {Grp. gram.:f.}
\end{itemize}
Variedade de fava, (\textunderscore faba vulgaris major\textunderscore ).
\section{Favaceira}
\begin{itemize}
\item {Grp. gram.:f.}
\end{itemize}
\begin{itemize}
\item {Utilização:Prov.}
\end{itemize}
\begin{itemize}
\item {Utilização:trasm.}
\end{itemize}
\begin{itemize}
\item {Proveniência:(De \textunderscore fava\textunderscore , por allusão ao tempêro della)}
\end{itemize}
Vendedeira de azeite por miúdo.
\section{Fava-da-índia}
\begin{itemize}
\item {Grp. gram.:f.}
\end{itemize}
O mesmo que \textunderscore fava-de-tonca\textunderscore .
\section{Fava-de-santo-inácio}
\begin{itemize}
\item {Grp. gram.:f.}
\end{itemize}
Planta euphorbiácea, (\textunderscore ignacia amara\textunderscore ), medicinal.
\section{Fava-de-tonca}
\begin{itemize}
\item {Grp. gram.:f.}
\end{itemize}
Semente aromática de cumarim.
\section{Favado}
\begin{itemize}
\item {Grp. gram.:adj.}
\end{itemize}
\begin{itemize}
\item {Utilização:Bras. do N}
\end{itemize}
\begin{itemize}
\item {Proveniência:(De \textunderscore favar\textunderscore )}
\end{itemize}
Que se gorou; que teve resultado negativo.
\section{Faval}
\begin{itemize}
\item {Grp. gram.:m.}
\end{itemize}
\begin{itemize}
\item {Proveniência:(Lat. \textunderscore fabalis\textunderscore )}
\end{itemize}
Lugar ou terreno, onde crescem favas.
\section{Favaleiro}
\begin{itemize}
\item {Grp. gram.:m.}
\end{itemize}
\begin{itemize}
\item {Utilização:Prov.}
\end{itemize}
\begin{itemize}
\item {Utilização:trasm.}
\end{itemize}
Vendedor ambulante de peixe.
\section{Fava-oliá}
\begin{itemize}
\item {Grp. gram.:f.}
\end{itemize}
Planta leguminosa de Dio.
\section{Fava-pichurim}
\begin{itemize}
\item {Grp. gram.:f.}
\end{itemize}
\begin{itemize}
\item {Utilização:Bras}
\end{itemize}
O mesmo que \textunderscore pichurim\textunderscore .
\section{Favar}
\begin{itemize}
\item {Grp. gram.:v. i.}
\end{itemize}
\begin{itemize}
\item {Utilização:Bras. do N}
\end{itemize}
Gorar-se; têr resultado negativo.
\section{Fava-rana}
\begin{itemize}
\item {Grp. gram.:f.}
\end{itemize}
\begin{itemize}
\item {Utilização:Bras}
\end{itemize}
Árvore do Pará, própria para construcções.
\section{Favaria}
\begin{itemize}
\item {Grp. gram.:f.}
\end{itemize}
Porção de favas; faval. Cf. Castilho, \textunderscore Geórgicas\textunderscore , 251.
\section{Favária}
\begin{itemize}
\item {Grp. gram.:f.}
\end{itemize}
O mesmo que \textunderscore favária-maior\textunderscore .
\section{Favária-maior}
\begin{itemize}
\item {Grp. gram.:f.}
\end{itemize}
Planta crassulácea, conhecida vulgarmente por \textunderscore erva dos callos\textunderscore .
\section{Favária-vulgar}
\begin{itemize}
\item {Grp. gram.:f.}
\end{itemize}
O mesmo que \textunderscore favária-maior\textunderscore .
\section{Fava-rica}
\begin{itemize}
\item {Grp. gram.:f.}
\end{itemize}
Fava sêca, que, depois de cozida, se come temperada com azeite, alhos e pimenta.
\section{Favas-de-chapas}
\begin{itemize}
\item {Grp. gram.:f. pl.}
\end{itemize}
Planta leguminosa da Índia portuguesa, (\textunderscore pongamia glabra\textunderscore , Vent.).
\section{Favas-de-engenho}
\begin{itemize}
\item {Grp. gram.:f. pl.}
\end{itemize}
Planta leguminosa da Índia portuguesa, (\textunderscore bulea frondosa\textunderscore , Roxb.).
\section{Favas-de-lázaro}
\begin{itemize}
\item {Grp. gram.:f. pl.}
\end{itemize}
Planta leguminosa da Índia, (\textunderscore albizia-odoratissima\textunderscore , Benth.).
\section{Fava-sêca}
\begin{itemize}
\item {Grp. gram.:f.}
\end{itemize}
\begin{itemize}
\item {Utilização:T. de Turquel}
\end{itemize}
Bofetada, que faz bater os dentes.
\section{Fava-tonca}
\begin{itemize}
\item {Grp. gram.:f.}
\end{itemize}
Planta leguminosa, medicinal, (\textunderscore dipterix tetraphylla\textunderscore ).
\section{Faveca}
\begin{itemize}
\item {Grp. gram.:f.}
\end{itemize}
\begin{itemize}
\item {Utilização:Prov.}
\end{itemize}
\begin{itemize}
\item {Utilização:dur.}
\end{itemize}
\begin{itemize}
\item {Utilização:trasm.}
\end{itemize}
Vagem sêca de qualquer planta leguminosa.
\section{Faveco}
\begin{itemize}
\item {Grp. gram.:m.}
\end{itemize}
\begin{itemize}
\item {Utilização:Gír.}
\end{itemize}
\begin{itemize}
\item {Proveniência:(De \textunderscore fava\textunderscore )}
\end{itemize}
Feijão.
\section{Faveira}
\begin{itemize}
\item {Grp. gram.:f.}
\end{itemize}
\begin{itemize}
\item {Proveniência:(Do lat. \textunderscore fabaria\textunderscore )}
\end{itemize}
Fava (planta).
Arvore mimósea do Pará, própria para construcções.
\section{Faveira-branca}
\begin{itemize}
\item {Grp. gram.:f.}
\end{itemize}
Nome de uma árvore leguminosa.
\section{Faveira-de-bolota}
\begin{itemize}
\item {Grp. gram.:f.}
\end{itemize}
\begin{itemize}
\item {Utilização:Bras. do N}
\end{itemize}
Nome de uma árvore leguminosa.
\section{Faviforme}
\begin{itemize}
\item {Grp. gram.:adj.}
\end{itemize}
\begin{itemize}
\item {Proveniência:(Do lat. \textunderscore favus\textunderscore  + \textunderscore forma\textunderscore )}
\end{itemize}
Que tem fórma de alvéolo, ou de cada uma das céllulas que constituem o favo.
\section{Favila}
\begin{itemize}
\item {Grp. gram.:f.}
\end{itemize}
\begin{itemize}
\item {Proveniência:(Lat. \textunderscore favilla\textunderscore )}
\end{itemize}
Cinza.
Lume, coberto ou misturado com cinza.
Centelha.
O mesmo ou melhor que \textunderscore fovila\textunderscore .
\section{Favilla}
\begin{itemize}
\item {Grp. gram.:f.}
\end{itemize}
\begin{itemize}
\item {Proveniência:(Lat. \textunderscore favilla\textunderscore )}
\end{itemize}
Cinza.
Lume, coberto ou misturado com cinza.
Centelha.
O mesmo ou melhor que \textunderscore fovilla\textunderscore .
\section{Favinha}
\begin{itemize}
\item {Grp. gram.:f.}
\end{itemize}
\begin{itemize}
\item {Proveniência:(De \textunderscore fava\textunderscore )}
\end{itemize}
Planta leguminosa de Pernambuco.
\section{Favissas}
\begin{itemize}
\item {Grp. gram.:f. pl.}
\end{itemize}
\begin{itemize}
\item {Proveniência:(Lat. \textunderscore favissae\textunderscore )}
\end{itemize}
Subterrâneo ou lugar esconso, nos antigos templos romanos, para guarda dos vasos sagrados.
\section{Favo}
\begin{itemize}
\item {Grp. gram.:m.}
\end{itemize}
\begin{itemize}
\item {Utilização:Fig.}
\end{itemize}
\begin{itemize}
\item {Proveniência:(Lat. \textunderscore favus\textunderscore )}
\end{itemize}
Alvéolo ou conjunto de alvéolos, em que a abelha deposita o mel.
Aquillo que tem semelhança com o alvéolo.
Coisa doce, agradável.
\section{Favona}
\begin{itemize}
\item {Grp. gram.:f.}
\end{itemize}
\begin{itemize}
\item {Utilização:T. de Cabo-Verde}
\end{itemize}
Espécie de feijoeiro, (\textunderscore phaseolus vulg.\textunderscore , Lin.).
\section{Favonear}
\begin{itemize}
\item {Grp. gram.:v. t.}
\end{itemize}
\begin{itemize}
\item {Proveniência:(De \textunderscore favónio\textunderscore )}
\end{itemize}
O mesmo que \textunderscore favorecer\textunderscore .--Melhor escrita seria \textunderscore favoniar\textunderscore .
\section{Favónio}
\begin{itemize}
\item {Grp. gram.:m.}
\end{itemize}
\begin{itemize}
\item {Grp. gram.:Adj.}
\end{itemize}
\begin{itemize}
\item {Proveniência:(Lat. \textunderscore favonius\textunderscore )}
\end{itemize}
Vento brando do Poente.
Vento propício.
Propício, próspero.
\section{Favor}
\begin{itemize}
\item {Grp. gram.:m.}
\end{itemize}
\begin{itemize}
\item {Utilização:Fam.}
\end{itemize}
\begin{itemize}
\item {Proveniência:(Lat. \textunderscore favor\textunderscore )}
\end{itemize}
Interesse.
Protecção.
Benevolência: \textunderscore tratar com favor\textunderscore .
Parcialidade.
Graça; benefício: \textunderscore receber favores\textunderscore .
Sympathia.
Indulgência.
Condição favorável: \textunderscore o vento soprava a favor da embarcação\textunderscore .
Defensão.
Carta, missiva.
\section{Favorado}
\begin{itemize}
\item {Grp. gram.:adj.}
\end{itemize}
\begin{itemize}
\item {Utilização:Ant.}
\end{itemize}
O mesmo que \textunderscore favorecido\textunderscore .
\section{Favorança}
\begin{itemize}
\item {Grp. gram.:f.}
\end{itemize}
\begin{itemize}
\item {Utilização:Ant.}
\end{itemize}
(V.favor)
\section{Favorável}
\begin{itemize}
\item {Grp. gram.:adj.}
\end{itemize}
\begin{itemize}
\item {Proveniência:(Lat. \textunderscore favorabilis\textunderscore )}
\end{itemize}
Que favorece.
Que dá vantagens.
Que auxilia; propício.
\section{Favoravelmente}
\begin{itemize}
\item {Grp. gram.:adv.}
\end{itemize}
De modo favorável.
\section{Favorecedor}
\begin{itemize}
\item {Grp. gram.:adj.}
\end{itemize}
\begin{itemize}
\item {Grp. gram.:M.}
\end{itemize}
Que favorece.
Aquelle que favorece.
\section{Favorecer}
\begin{itemize}
\item {Grp. gram.:v. t.}
\end{itemize}
Fazer favor a.
Dar protecção a.
Auxiliar.
Promover; fomentar: \textunderscore Dom Dinis favoreceu a agricultura\textunderscore .
Encarecer, realçar.
\section{Favorecido}
\begin{itemize}
\item {Grp. gram.:adj.}
\end{itemize}
\begin{itemize}
\item {Proveniência:(De \textunderscore favorecer\textunderscore )}
\end{itemize}
Protegido.
Obsequiado.
Diz-se do retrato, que representa melhor figura que o original.
\section{Favoreza}
\begin{itemize}
\item {Grp. gram.:f.}
\end{itemize}
\begin{itemize}
\item {Utilização:Ant.}
\end{itemize}
(V.favor)
\section{Favorita}
\begin{itemize}
\item {Grp. gram.:f.}
\end{itemize}
\begin{itemize}
\item {Grp. gram.:Pl.}
\end{itemize}
\begin{itemize}
\item {Utilização:Ant.}
\end{itemize}
\begin{itemize}
\item {Proveniência:(De \textunderscore favorito\textunderscore )}
\end{itemize}
A mais favorecida, a mais querida.
Rolos de cabello, que caíam sôbre a testa.
\section{Favoritismo}
\begin{itemize}
\item {Grp. gram.:m.}
\end{itemize}
\begin{itemize}
\item {Proveniência:(De \textunderscore favorito\textunderscore )}
\end{itemize}
Patronato; protecção sem motivo justo.
Preferência dada a quem é favorito.
\section{Favorito}
\begin{itemize}
\item {Grp. gram.:adj.}
\end{itemize}
\begin{itemize}
\item {Grp. gram.:M.}
\end{itemize}
\begin{itemize}
\item {Proveniência:(It. \textunderscore favorito\textunderscore )}
\end{itemize}
Muito favorecido; preferído; amado com preferência.
Indivíduo predilecto: \textunderscore os favoritos do Rei\textunderscore .
\section{Favorizador}
\begin{itemize}
\item {Grp. gram.:m.}
\end{itemize}
\begin{itemize}
\item {Utilização:Ant.}
\end{itemize}
\begin{itemize}
\item {Proveniência:(De \textunderscore favorizar\textunderscore )}
\end{itemize}
Aquelle que favorece.
\section{Favorizar}
\begin{itemize}
\item {Grp. gram.:v. t.}
\end{itemize}
\begin{itemize}
\item {Utilização:Ant.}
\end{itemize}
(V.favorecer)
\section{Favosa}
\begin{itemize}
\item {Grp. gram.:f.}
\end{itemize}
\begin{itemize}
\item {Utilização:Med.}
\end{itemize}
\begin{itemize}
\item {Proveniência:(De \textunderscore favoso\textunderscore )}
\end{itemize}
Espécie de tinha.
\section{Favoso}
\begin{itemize}
\item {Grp. gram.:adj.}
\end{itemize}
\begin{itemize}
\item {Utilização:Bot.}
\end{itemize}
\begin{itemize}
\item {Utilização:Med.}
\end{itemize}
\begin{itemize}
\item {Proveniência:(De \textunderscore favo\textunderscore )}
\end{itemize}
Que tem pequenas cavidades na superfície, (falando-se de órgãos vegetaes).
Diz-se da tinha verdadeira e contagiosa.
\section{Faxa}
\begin{itemize}
\item {Grp. gram.:f.}
\end{itemize}
(V.faixa)
\section{Faxa}
\begin{itemize}
\item {Grp. gram.:f.}
\end{itemize}
\begin{itemize}
\item {Utilização:Prov.}
\end{itemize}
\begin{itemize}
\item {Utilização:beir.}
\end{itemize}
\begin{itemize}
\item {Proveniência:(Do lat. \textunderscore fascía\textunderscore )}
\end{itemize}
O mesmo que \textunderscore feixe\textunderscore : \textunderscore uma faxa de palha para os bois\textunderscore .
\section{Faxar}
\begin{itemize}
\item {Grp. gram.:v. t.}
\end{itemize}
\begin{itemize}
\item {Utilização:Gír.}
\end{itemize}
Abrir.
\section{Faxeque}
\begin{itemize}
\item {Grp. gram.:m.}
\end{itemize}
\begin{itemize}
\item {Utilização:Ant.}
\end{itemize}
Ministro da justiça, no Japão.
\section{Faxina}
\begin{itemize}
\item {Grp. gram.:f.}
\end{itemize}
\begin{itemize}
\item {Utilização:Prov.}
\end{itemize}
\begin{itemize}
\item {Utilização:alent.}
\end{itemize}
\begin{itemize}
\item {Utilização:minh.}
\end{itemize}
\begin{itemize}
\item {Utilização:Fig.}
\end{itemize}
\begin{itemize}
\item {Grp. gram.:M.}
\end{itemize}
\begin{itemize}
\item {Proveniência:(Lat. \textunderscore fascina\textunderscore )}
\end{itemize}
Feixe de armas ou de paus curtos.
Mólho de lenha.
Unidade de pêso para lenha em achas, espécie de gaiola, em que as achas se metem, até se encher, regulando cada faxina por 60 kilogrammas.
Toro de pinheiro, em toda a sua espessura, e do comprimento aproximado de 1 braça.
Fortificação, feita com mólhos de ramos.
Planta rubiácea do Brasil.
Serviço de limpeza ou de conducção de rancho, nas casernas.
Estrago; desfalque.
Soldado, encarregado da faxina, nas casernas.
\section{Faxinal}
\begin{itemize}
\item {Grp. gram.:m.}
\end{itemize}
\begin{itemize}
\item {Utilização:Bras}
\end{itemize}
\begin{itemize}
\item {Proveniência:(De \textunderscore faxina\textunderscore )}
\end{itemize}
Lugar, em que se corta lenha ou mato curto.
\section{Faxinar}
\begin{itemize}
\item {Grp. gram.:v. t.}
\end{itemize}
\begin{itemize}
\item {Proveniência:(De \textunderscore faxina\textunderscore )}
\end{itemize}
Formar feixes de; enfeixar.
Entupir com faxinas (fossos, pântanos).
\section{Faxineiro}
\begin{itemize}
\item {Grp. gram.:m.}
\end{itemize}
Aquelle que nos quartéis tem serviço de faxina.
\section{Faxo}
\begin{itemize}
\item {Grp. gram.:M.}
\end{itemize}
\begin{itemize}
\item {Utilização:Ant.}
\end{itemize}
\begin{itemize}
\item {Utilização:Gír.}
\end{itemize}
Pau.
Cara.
(Cp. \textunderscore facha\textunderscore ^3)
\section{Fazedoiro}
\begin{itemize}
\item {Grp. gram.:adj.}
\end{itemize}
\begin{itemize}
\item {Utilização:Ant.}
\end{itemize}
\begin{itemize}
\item {Proveniência:(De \textunderscore fazer\textunderscore )}
\end{itemize}
Que se póde ou se deve fazer.
\section{Fazedor}
\begin{itemize}
\item {Grp. gram.:m.}
\end{itemize}
Aquelle que faz.
Aquelle que cumpre ou executa.
Factotum.
\section{Fazedouro}
\begin{itemize}
\item {Grp. gram.:adj.}
\end{itemize}
\begin{itemize}
\item {Utilização:Ant.}
\end{itemize}
\begin{itemize}
\item {Proveniência:(De \textunderscore fazer\textunderscore )}
\end{itemize}
Que se póde ou se deve fazer.
\section{Fazedura}
\begin{itemize}
\item {Grp. gram.:f.}
\end{itemize}
\begin{itemize}
\item {Utilização:Des.}
\end{itemize}
\begin{itemize}
\item {Utilização:Ant.}
\end{itemize}
Acto de fazer.
Bica ou pão de manteiga.
\section{Fazenda}
\begin{itemize}
\item {Grp. gram.:f.}
\end{itemize}
\begin{itemize}
\item {Utilização:Fig.}
\end{itemize}
\begin{itemize}
\item {Utilização:Prov.}
\end{itemize}
\begin{itemize}
\item {Utilização:Ant.}
\end{itemize}
\begin{itemize}
\item {Utilização:Ant.}
\end{itemize}
\begin{itemize}
\item {Utilização:Ant.}
\end{itemize}
\begin{itemize}
\item {Grp. gram.:Loc.}
\end{itemize}
\begin{itemize}
\item {Utilização:Ant.}
\end{itemize}
\begin{itemize}
\item {Proveniência:(Do lat. \textunderscore facienda\textunderscore )}
\end{itemize}
Terreno cultivado.
Prédio rústico.
Bens.
Rendimentos públicos.
Finanças: \textunderscore os problemas da Fazenda\textunderscore .
Pano, de que se hão de fazer peças de vestuário, etc. Pano: \textunderscore fazenda clara\textunderscore .
Mercadoria.
Aquillo que se expõe á venda.
Qualidade.
Rebanho de gado macho.
Qualquer rebanho.
O mesmo que \textunderscore facienda\textunderscore ; acto, procedimento.
Peleja, duello.
Reputação de mulher honesta.
\textunderscore Fazer fazenda\textunderscore , negociar, traficar. Cf. \textunderscore Peregrinação\textunderscore , XLVIII.
\section{Fazendal}
\begin{itemize}
\item {Grp. gram.:adj.}
\end{itemize}
O mesmo que \textunderscore fazendário\textunderscore .
\section{Fazendário}
\begin{itemize}
\item {Grp. gram.:adj.}
\end{itemize}
\begin{itemize}
\item {Proveniência:(De \textunderscore fazenda\textunderscore )}
\end{itemize}
Relativo á fazenda pública; financeiro.
\section{Fazendeira}
\begin{itemize}
\item {Grp. gram.:f.}
\end{itemize}
\begin{itemize}
\item {Proveniência:(De \textunderscore fazenda\textunderscore )}
\end{itemize}
Espécie de tributo antigo.
\section{Fazendeiro}
\begin{itemize}
\item {Grp. gram.:m.  e  adj.}
\end{itemize}
\begin{itemize}
\item {Grp. gram.:M.}
\end{itemize}
O que tem ou cultiva fazendas.
Casta de uva preta.
\section{Fazendista}
\begin{itemize}
\item {Grp. gram.:m.}
\end{itemize}
Aquelle, que é versado em assumptos da Fazenda Pública.
\section{Fazendoiro}
\begin{itemize}
\item {Grp. gram.:adj.}
\end{itemize}
\begin{itemize}
\item {Utilização:Ant.}
\end{itemize}
O mesmo que \textunderscore fazedoiro\textunderscore .
\section{Fazendola}
\begin{itemize}
\item {Grp. gram.:f.}
\end{itemize}
Pequena fazenda.
\section{Fazendouro}
\begin{itemize}
\item {Grp. gram.:adj.}
\end{itemize}
\begin{itemize}
\item {Utilização:Ant.}
\end{itemize}
O mesmo que \textunderscore fazedouro\textunderscore .
\section{Fazente}
\begin{itemize}
\item {Grp. gram.:adj.}
\end{itemize}
\begin{itemize}
\item {Utilização:Ant.}
\end{itemize}
Que faz; fazendo. Cf. D. Bernárdez, \textunderscore Lima\textunderscore , 2.
\section{Fazer}
\begin{itemize}
\item {Grp. gram.:v. t.}
\end{itemize}
\begin{itemize}
\item {Grp. gram.:Loc. inf.}
\end{itemize}
\begin{itemize}
\item {Grp. gram.:Loc.}
\end{itemize}
\begin{itemize}
\item {Utilização:pop.}
\end{itemize}
\begin{itemize}
\item {Grp. gram.:Loc.}
\end{itemize}
\begin{itemize}
\item {Utilização:pop.}
\end{itemize}
\begin{itemize}
\item {Grp. gram.:V. i.}
\end{itemize}
\begin{itemize}
\item {Utilização:Ven.}
\end{itemize}
\begin{itemize}
\item {Proveniência:(Do lat. \textunderscore facere\textunderscore )}
\end{itemize}
Dar existência ou fórma a: \textunderscore fazer uma capa\textunderscore .
Criar, produzir.
Realizar: \textunderscore fazer empréstimos\textunderscore .
Construir: \textunderscore fazer um prédio\textunderscore .
Inventar.
Pintar: \textunderscore fazer um quadro\textunderscore .
Escrever: \textunderscore fazer um poêma\textunderscore .
Converter, reduzir a.
Suppor.
Completar: \textunderscore faço hoje 39 annos\textunderscore .
Perfazer.
Pronunciar, dizer.
\textunderscore Fazer que\textunderscore , causar.
\textunderscore Fazer biquinho\textunderscore , amuar.
\textunderscore Fazer cera\textunderscore , mandriar.
\textunderscore Fazer chichi\textunderscore , urinar.
\textunderscore Fazer cruzes na bôca\textunderscore , resignar-se com a falta de qualquer coisa.
\textunderscore Fazer espécie\textunderscore , fazer desconfiar.
\textunderscore Fazer minga\textunderscore , fazer falta, ser necessário.
\textunderscore Fazer o ninho atrás da orelha\textunderscore , fazer partida com astúcia; embaír, lograr.
\textunderscore Fazer tijolo\textunderscore , estar sepultado.
\textunderscore Fazer conta\textunderscore , convir, fazer arranjo.
\textunderscore Fazer conta que\textunderscore  ou \textunderscore fazer de conta que\textunderscore , suppor, calcular.
\textunderscore Fazer que\textunderscore , simular: \textunderscore faz que anda, mas desanda\textunderscore .
\textunderscore Fazer caminho\textunderscore , caminhar (em determinada direcção).
\textunderscore Fazer escola\textunderscore , assentar princípios ou organizar processos, que depois são seguidos por muita gente.
Diligenciar: \textunderscore fazer por acertar\textunderscore .
Trabalhar.
Proceder.
Desempenhar o lugar, exercer as vezes: \textunderscore fazer de chefe\textunderscore .
Haver.
Perseguir (falando-se do cão que segue a caça).
\section{Fazimento}
\begin{itemize}
\item {Grp. gram.:m.}
\end{itemize}
Acto ou effeito de fazer.
\section{Fazível}
\begin{itemize}
\item {Grp. gram.:adj.}
\end{itemize}
\begin{itemize}
\item {Proveniência:(De \textunderscore fazer\textunderscore )}
\end{itemize}
O mesmo que \textunderscore factível\textunderscore .
\section{Faz-tudo}
\begin{itemize}
\item {Grp. gram.:m.}
\end{itemize}
\begin{itemize}
\item {Utilização:Fam.}
\end{itemize}
Aquelle que se encarrega de pequenos e variados negócios.
Aquelle que exerce variadas indústrias.
Aquelle que se occupa em variados serviços.
\section{Fé}
\begin{itemize}
\item {Grp. gram.:f.}
\end{itemize}
\begin{itemize}
\item {Utilização:Eccles.}
\end{itemize}
Firmeza ou pontualidade na execução de uma promessa ou de um compromisso.
Lealdade.
Bôa reputação, crédito.
Convicção.
Confiança.
Veracidade.
Crença relígiosa, religião.
Uma das virtudes theologaes.
Confirmação de um testemunho.
Testemunho de certos funccionários, que faz fôrça nos tribunaes.
\textunderscore Dar fé\textunderscore , ter notícia, tomar nota.
\section{Fealdade}
\begin{itemize}
\item {Grp. gram.:f.}
\end{itemize}
\begin{itemize}
\item {Utilização:Fig.}
\end{itemize}
\begin{itemize}
\item {Proveniência:(Lat. \textunderscore foeditas\textunderscore )}
\end{itemize}
Qualidade daquelle ou daquillo que é feio.
Desdoiro; indignidade.
\section{Feanchão}
\begin{itemize}
\item {Grp. gram.:m.  e  adj.}
\end{itemize}
\begin{itemize}
\item {Utilização:Des.}
\end{itemize}
Muito feio.
\section{Febra}
\begin{itemize}
\item {fónica:fê}
\end{itemize}
\begin{itemize}
\item {Grp. gram.:f.}
\end{itemize}
\begin{itemize}
\item {Utilização:Fig.}
\end{itemize}
\begin{itemize}
\item {Proveniência:(Do ár. \textunderscore habra\textunderscore )}
\end{itemize}
A parte musculosa dos animaes vertebrados, que são comestíveis.
Carne sem gordura nem osso.
Fôrça, energia.
\section{Febrão}
\begin{itemize}
\item {Grp. gram.:m.}
\end{itemize}
\begin{itemize}
\item {Proveniência:(De \textunderscore febre\textunderscore ^1)}
\end{itemize}
Febre muito intensa; grande accesso de febre.
\section{Febre}
\begin{itemize}
\item {Grp. gram.:f.}
\end{itemize}
\begin{itemize}
\item {Utilização:Fig.}
\end{itemize}
\begin{itemize}
\item {Utilização:Gír.}
\end{itemize}
\begin{itemize}
\item {Proveniência:(Lat. \textunderscore febris\textunderscore , provávelmente metháth. de \textunderscore ferbis\textunderscore , de \textunderscore fervere\textunderscore )}
\end{itemize}
Estado mórbido, carecterizado pela acceleração do pulso e aumento de calor.
Grande perturbação de espírito.
Exaltação.
Ansia de possuir.
\textunderscore Febre cerebral\textunderscore , condemnação á morte.--Na linguagem pop., é as vezes voc. masc.: \textunderscore tem hoje muito febre\textunderscore .
\section{Febre}
\begin{itemize}
\item {Grp. gram.:adj.}
\end{itemize}
\begin{itemize}
\item {Grp. gram.:M.}
\end{itemize}
\begin{itemize}
\item {Proveniência:(Do fr. \textunderscore faible\textunderscore )}
\end{itemize}
Que não tem o pêso legal.
Falta de pêso legal (falando-se de moédas).
\section{Febrento}
\begin{itemize}
\item {Grp. gram.:adj.}
\end{itemize}
\begin{itemize}
\item {Utilização:Bras. do N}
\end{itemize}
O mesmo que \textunderscore febril\textunderscore : \textunderscore corpo febrento\textunderscore .
Em que grassam febres: \textunderscore lugar febrento\textunderscore .
\section{Febricitação}
\begin{itemize}
\item {Grp. gram.:f.}
\end{itemize}
\begin{itemize}
\item {Proveniência:(De \textunderscore febricitar\textunderscore )}
\end{itemize}
Estado de quem febricita.
\section{Febricitante}
\begin{itemize}
\item {Grp. gram.:adj.}
\end{itemize}
\begin{itemize}
\item {Utilização:Fig.}
\end{itemize}
\begin{itemize}
\item {Proveniência:(Lat. \textunderscore febricitans\textunderscore )}
\end{itemize}
Que tem febre, que febricita.
Exaltado, apaixonado.
\section{Febricitar}
\begin{itemize}
\item {Grp. gram.:v. i.}
\end{itemize}
\begin{itemize}
\item {Proveniência:(Lat. \textunderscore febricitare\textunderscore )}
\end{itemize}
Têr febre.
\section{Febrícula}
\begin{itemize}
\item {Grp. gram.:f.}
\end{itemize}
\begin{itemize}
\item {Proveniência:(Lat. \textunderscore febricula\textunderscore )}
\end{itemize}
Febre ligeira.
\section{Febriculoso}
\begin{itemize}
\item {Grp. gram.:adj.}
\end{itemize}
\begin{itemize}
\item {Proveniência:(Lat. \textunderscore febriculosus\textunderscore )}
\end{itemize}
Que tem tendência para febres; que tem febre muitas vezes.
\section{Febrífugo}
\begin{itemize}
\item {Grp. gram.:adj.}
\end{itemize}
\begin{itemize}
\item {Grp. gram.:M.}
\end{itemize}
\begin{itemize}
\item {Proveniência:(Lat. \textunderscore febrifugus\textunderscore )}
\end{itemize}
Diz-se dos medicamentos contra a febre.
Medicamento contra a febre.
\section{Febril}
\begin{itemize}
\item {Grp. gram.:adj.}
\end{itemize}
\begin{itemize}
\item {Utilização:Fig.}
\end{itemize}
\begin{itemize}
\item {Proveniência:(Lat. \textunderscore febrilis\textunderscore )}
\end{itemize}
Relativo a febre, que tem febre.
Exaltado, violento.
\section{Febriologia}
\begin{itemize}
\item {Grp. gram.:f.}
\end{itemize}
\begin{itemize}
\item {Proveniência:(De \textunderscore febriólogo\textunderscore )}
\end{itemize}
Tratado á cêrca das febres.
\section{Febriólogo}
\begin{itemize}
\item {Grp. gram.:m.}
\end{itemize}
Aquelle que é versado em febriologia.
(Palavra mal formada, do lat. \textunderscore febris\textunderscore  + gr. \textunderscore logos\textunderscore )
\section{Febroso}
\begin{itemize}
\item {Grp. gram.:adj.}
\end{itemize}
\begin{itemize}
\item {Utilização:Prov.}
\end{itemize}
\begin{itemize}
\item {Utilização:trasm.}
\end{itemize}
Que tem febre.
Que produz febre.
\section{Fébrua}
\begin{itemize}
\item {Grp. gram.:f.}
\end{itemize}
\begin{itemize}
\item {Grp. gram.:Pl.}
\end{itemize}
\begin{itemize}
\item {Proveniência:(Lat. \textunderscore februa\textunderscore )}
\end{itemize}
Expiação ou purificação, entre os antigos Romanos.
Bolo, que se offerecia ao lictor.
Ramo com que os flâmines adornavam a cabeça. Cf. Castilho, \textunderscore Fastos\textunderscore , I, 81.
Festas da fébrua, que se celebravam a 15 do mês que dellas tirou o nome de Fevereiro, (\textunderscore februarius\textunderscore )
\section{Februaes}
\begin{itemize}
\item {Grp. gram.:f. pl.}
\end{itemize}
Festas, o mesmo que \textunderscore fébrua\textunderscore . Cf. Castilho, \textunderscore Fastos\textunderscore , III, 49.
\section{Februais}
\begin{itemize}
\item {Grp. gram.:f. pl.}
\end{itemize}
Festas, o mesmo que \textunderscore fébrua\textunderscore . Cf. Castilho, \textunderscore Fastos\textunderscore , III, 49.
\section{Fecal}
\begin{itemize}
\item {Grp. gram.:adj.}
\end{itemize}
\begin{itemize}
\item {Proveniência:(Do lat. \textunderscore faex\textunderscore )}
\end{itemize}
Relativo a fezes; excrementício.
\section{Fecaloide}
\begin{itemize}
\item {Grp. gram.:adj.}
\end{itemize}
\begin{itemize}
\item {Proveniência:(De \textunderscore fecal\textunderscore  + gr. \textunderscore eidos\textunderscore )}
\end{itemize}
Que cheira a matérias fecaes, (falando-se do vómito).
\section{Fecer}
\begin{itemize}
\item {Grp. gram.:v. t.}
\end{itemize}
\begin{itemize}
\item {Utilização:Ant.}
\end{itemize}
O mesmo que \textunderscore fender\textunderscore . Cf. Sim. Machado, fl. 2, v.^o
\section{Fecha}
\begin{itemize}
\item {fónica:fê}
\end{itemize}
\begin{itemize}
\item {Grp. gram.:f.}
\end{itemize}
\begin{itemize}
\item {Utilização:Des.}
\end{itemize}
Final de carta.
Data.
(Cast. \textunderscore fecha\textunderscore )
\section{Fechado}
\begin{itemize}
\item {Grp. gram.:adj.}
\end{itemize}
\begin{itemize}
\item {Grp. gram.:m.}
\end{itemize}
\begin{itemize}
\item {Utilização:Bras. do N}
\end{itemize}
\begin{itemize}
\item {Proveniência:(De \textunderscore fechar\textunderscore )}
\end{itemize}
Cerrado.
Acabado.
Acabamento de meia ou peúga.
Parte fechada, em trabalhos de croché.
Mato cerrado.
\section{Fechadora}
\begin{itemize}
\item {Grp. gram.:f.}
\end{itemize}
Mulher, que, nas fábricas de tabaco, é encarregada de fechar as caixas ou pacotes.
\section{Fechadura}
\begin{itemize}
\item {Grp. gram.:f.}
\end{itemize}
\begin{itemize}
\item {Proveniência:(De \textunderscore fechar\textunderscore )}
\end{itemize}
Peça de metal, que, por meio de uma ou mais linguetas, e com auxílio de chave, fecha portas, gavetas, etc.
\section{Fechamento}
\begin{itemize}
\item {Grp. gram.:m.}
\end{itemize}
Acto de fechar.
Fecho de abóbada ou de arco.
\section{Fechar}
\begin{itemize}
\item {Grp. gram.:v. t.}
\end{itemize}
\begin{itemize}
\item {Grp. gram.:V. i.}
\end{itemize}
\begin{itemize}
\item {Grp. gram.:V. p.}
\end{itemize}
\begin{itemize}
\item {Proveniência:(Do gall. \textunderscore pechar\textunderscore )}
\end{itemize}
Apertar ou ajustar (um objecto a outro).
Cerrar.
Tornar fixo por meio de chave, aldrava, tranca, etc. (uma porta, uma gaveta, etc.).
Unir os lábios ou bordas de: \textunderscore fechar uma ferida\textunderscore .
Tapar, cercar, encerrar: \textunderscore fechar o gado\textunderscore .
Concluir: \textunderscore fechar um capitulo\textunderscore .
Limitar.
Unir-se.
Tapar-se.
Cicatrizar.
Findar, concluir-se.
(A mesma significação)
\textunderscore Fechar-se em copas\textunderscore , amuar; acautelar-se.
\section{Fecharia}
\begin{itemize}
\item {Grp. gram.:f.}
\end{itemize}
\begin{itemize}
\item {Proveniência:(De \textunderscore fecho\textunderscore )}
\end{itemize}
Conjunto de peças, que determinam a explosão, em armas de fogo.
\section{Fecho}
\begin{itemize}
\item {fónica:fê}
\end{itemize}
\begin{itemize}
\item {Grp. gram.:m.}
\end{itemize}
\begin{itemize}
\item {Utilização:Prov.}
\end{itemize}
\begin{itemize}
\item {Utilização:Prov.}
\end{itemize}
\begin{itemize}
\item {Utilização:trasm.}
\end{itemize}
\begin{itemize}
\item {Grp. gram.:Pl.}
\end{itemize}
\begin{itemize}
\item {Proveniência:(Do lat. hyp. \textunderscore festulum\textunderscore )}
\end{itemize}
Ferrolho ou aldrava de porta.
Qualquer objecto, com que se fecha ou cerra alguma coisa.
Remate, fim.
Sobrescrito.
Parede forte, á beira de rio ou ribeira, com as pedras dispostas em cunha, para que a água as não leve.
Fecharia.
\section{Fecial}
\begin{itemize}
\item {Grp. gram.:m.}
\end{itemize}
\begin{itemize}
\item {Utilização:Ant.}
\end{itemize}
\begin{itemize}
\item {Proveniência:(Lat. \textunderscore fecialis\textunderscore )}
\end{itemize}
Sacerdote romano, núncio de paz ou de guerra.
\section{Fécula}
\begin{itemize}
\item {Grp. gram.:f.}
\end{itemize}
\begin{itemize}
\item {Utilização:Ant.}
\end{itemize}
\begin{itemize}
\item {Proveniência:(Lat. \textunderscore faecula\textunderscore )}
\end{itemize}
Amido da batata.
Amido.
Matérias, que se precipitam dos sucos vegetaes por meio de espremedura.
\section{Feculência}
\begin{itemize}
\item {Grp. gram.:f.}
\end{itemize}
\begin{itemize}
\item {Proveniência:(Lat. \textunderscore faeculentia\textunderscore )}
\end{itemize}
Qualidade daquillo que é feculento.
Sedimento dos líquidos.
\section{Feculento}
\begin{itemize}
\item {Grp. gram.:adj.}
\end{itemize}
\begin{itemize}
\item {Proveniência:(Lat. \textunderscore faeculentus\textunderscore )}
\end{itemize}
Que contém fécula.
Que tem sedimento ou fezes.
\section{Feculista}
\begin{itemize}
\item {Grp. gram.:m.}
\end{itemize}
Vendedor ou fabricante de fécula.
\section{Feculoídeo}
\begin{itemize}
\item {Grp. gram.:adj.}
\end{itemize}
\begin{itemize}
\item {Proveniência:(De \textunderscore fécula\textunderscore  + gr. \textunderscore eidos\textunderscore )}
\end{itemize}
Que tem apparência de fécula.
\section{Feculómetro}
\begin{itemize}
\item {Grp. gram.:m.}
\end{itemize}
Apparelho, recentemente inventado (1897), para indicar a quantidade, que as batatas contém, de fécula.
\section{Feculoso}
\begin{itemize}
\item {Grp. gram.:adj.}
\end{itemize}
O mesmo que \textunderscore feculento\textunderscore .
\section{Fecundação}
\begin{itemize}
\item {Grp. gram.:f.}
\end{itemize}
\begin{itemize}
\item {Proveniência:(Lat. \textunderscore fecundatio\textunderscore )}
\end{itemize}
Acto ou effeito de fecundar.
\section{Fecundador}
\begin{itemize}
\item {Grp. gram.:adj.}
\end{itemize}
Que fecunda: \textunderscore o sol fecundador\textunderscore .
\section{Fecundamente}
\begin{itemize}
\item {Grp. gram.:adv.}
\end{itemize}
\begin{itemize}
\item {Proveniência:(De \textunderscore fecundo\textunderscore )}
\end{itemize}
Com fecundidade.
\section{Fecundante}
\begin{itemize}
\item {Grp. gram.:adj.}
\end{itemize}
\begin{itemize}
\item {Proveniência:(Lat. \textunderscore fecundans\textunderscore )}
\end{itemize}
O mesmo que \textunderscore fecundador\textunderscore .
\section{Fecundar}
\begin{itemize}
\item {Grp. gram.:v. t.}
\end{itemize}
\begin{itemize}
\item {Grp. gram.:V. i.}
\end{itemize}
\begin{itemize}
\item {Proveniência:(Lat. \textunderscore fecundare\textunderscore )}
\end{itemize}
Tornar fecundo.
Transmittir a causa immediata da germinação a.
Tornar abundante, fertilizar.
Desenvolver.
Fomentar.
Tornar-se fecundo.
Conceber.
\section{Fecundativo}
\begin{itemize}
\item {Grp. gram.:adj.}
\end{itemize}
O mesmo que \textunderscore fecundante\textunderscore . Cf. Chagas, \textunderscore Côrte de D. João V\textunderscore , 128.
\section{Fecundez}
\begin{itemize}
\item {Grp. gram.:f.}
\end{itemize}
O mesmo que \textunderscore fecúndia\textunderscore .
\section{Fecúndia}
\begin{itemize}
\item {Grp. gram.:f.}
\end{itemize}
O mesmo que \textunderscore fecundidade\textunderscore .
\section{Fecundidade}
\begin{itemize}
\item {Grp. gram.:f.}
\end{itemize}
\begin{itemize}
\item {Utilização:Fig.}
\end{itemize}
\begin{itemize}
\item {Proveniência:(Lat. \textunderscore fecunditas\textunderscore )}
\end{itemize}
Qualidade daquelle ou daquillo que é fecundo, que se reproduz.
Abundância.
Facilidade de produzir obras de arte: \textunderscore a fecundidade de Camillo\textunderscore .
\section{Fecundizante}
\begin{itemize}
\item {Grp. gram.:adj.}
\end{itemize}
Que fecundiza.
\section{Fecundizar}
\begin{itemize}
\item {Grp. gram.:v. t.}
\end{itemize}
O mesmo que \textunderscore fecundar\textunderscore .
\section{Fecundo}
\begin{itemize}
\item {Grp. gram.:adj.}
\end{itemize}
\begin{itemize}
\item {Proveniência:(Lat. \textunderscore fecundus\textunderscore )}
\end{itemize}
Capaz de produzir ou de reproduzir.
Que produz muito; fértil: \textunderscore campos fecundos\textunderscore .
Que dá muitos e grandes resultados.
Inventivo, criador.
Que promove ou facilita a produção.
Que dá fruto.
Que dispõe de artifícios ou recursos.
\section{Fèdavelha}
\begin{itemize}
\item {Grp. gram.:f.}
\end{itemize}
\begin{itemize}
\item {Utilização:Prov.}
\end{itemize}
Insecto orthóptero, de côr verde e cheiro repugnante.
(Contr. da loc. \textunderscore fede\textunderscore  + \textunderscore a\textunderscore  + \textunderscore velha\textunderscore )
\section{Fédea}
\begin{itemize}
\item {Grp. gram.:f.}
\end{itemize}
Moéda de Cambaia e Baçaim, correspondente a 15 reis.
(Ar. \textunderscore fidia\textunderscore , do marata)
\section{Fedegosa}
\begin{itemize}
\item {Grp. gram.:f.}
\end{itemize}
\begin{itemize}
\item {Proveniência:(De \textunderscore fedegoso\textunderscore )}
\end{itemize}
Planta leguminosa, cujas sementes torradas parecem café.
\section{Fedegoso}
\begin{itemize}
\item {Grp. gram.:adj.}
\end{itemize}
\begin{itemize}
\item {Grp. gram.:M.}
\end{itemize}
\begin{itemize}
\item {Proveniência:(De \textunderscore feder\textunderscore )}
\end{itemize}
Que fede.
Nome de várias plantas brasileiras, da fam. das borragíneas.
\section{Fedela}
\begin{itemize}
\item {Grp. gram.:f.}
\end{itemize}
\begin{itemize}
\item {Utilização:Ant.}
\end{itemize}
Espécie de mandil.
\section{Fedelhice}
\begin{itemize}
\item {Grp. gram.:f.}
\end{itemize}
\begin{itemize}
\item {Utilização:Fam.}
\end{itemize}
Acção de fedelho ou de criança immunda.
\section{Fedelho}
\begin{itemize}
\item {fónica:dê}
\end{itemize}
\begin{itemize}
\item {Grp. gram.:m.}
\end{itemize}
\begin{itemize}
\item {Utilização:Prov.}
\end{itemize}
\begin{itemize}
\item {Utilização:Prov.}
\end{itemize}
\begin{itemize}
\item {Proveniência:(De \textunderscore feder\textunderscore )}
\end{itemize}
Criança, que cheira a cueiros.
Rapazito, crianço.
Thuríbulo.
Designação vulgar de um coleóptero, que fede muito.
\section{Fedelhota}
\begin{itemize}
\item {Grp. gram.:f.}
\end{itemize}
\begin{itemize}
\item {Utilização:Prov.}
\end{itemize}
\begin{itemize}
\item {Utilização:trasm.}
\end{itemize}
\textunderscore Á fedelhota\textunderscore , á maneira de peralta.
\section{Fedella}
\begin{itemize}
\item {Grp. gram.:f.}
\end{itemize}
\begin{itemize}
\item {Utilização:Ant.}
\end{itemize}
Espécie de mandil.
\section{Fedentina}
\begin{itemize}
\item {Grp. gram.:f.}
\end{itemize}
\begin{itemize}
\item {Proveniência:(Do rad. de \textunderscore feder\textunderscore )}
\end{itemize}
Cheiro repugnante.
\section{Fedentinha}
\begin{itemize}
\item {Grp. gram.:f.}
\end{itemize}
\begin{itemize}
\item {Utilização:Prov.}
\end{itemize}
\begin{itemize}
\item {Utilização:minh.}
\end{itemize}
O mesmo que \textunderscore fedentina\textunderscore .
\section{Feder}
\begin{itemize}
\item {Grp. gram.:v. i.}
\end{itemize}
\begin{itemize}
\item {Utilização:Fig.}
\end{itemize}
\begin{itemize}
\item {Proveniência:(Do lat. \textunderscore foetere\textunderscore )}
\end{itemize}
Lançar mau cheiro.
Causar enfado, sêr importuno.
\section{Federação}
\begin{itemize}
\item {Grp. gram.:f.}
\end{itemize}
\begin{itemize}
\item {Proveniência:(Lat. \textunderscore foederatio\textunderscore )}
\end{itemize}
União política entre nações ou estados.
Alliança.
Associação.
\section{Federado}
\begin{itemize}
\item {Grp. gram.:adj.}
\end{itemize}
\begin{itemize}
\item {Grp. gram.:M.}
\end{itemize}
\begin{itemize}
\item {Proveniência:(De \textunderscore federar\textunderscore )}
\end{itemize}
Que se federou; unido em confederação.

Aquelle que faz parte de uma federação.
\section{Federal}
\begin{itemize}
\item {Utilização:ant.}
\end{itemize}
\begin{itemize}
\item {Grp. gram.:M.}
\end{itemize}
\begin{itemize}
\item {Proveniência:(Do lat. \textunderscore foedus\textunderscore )}
\end{itemize}
Relativo a federação: \textunderscore a capital federal\textunderscore .
Planta do Ceará, medicinal, da fam. das compostas.
\section{Federalismo}
\begin{itemize}
\item {Grp. gram.:m.}
\end{itemize}
\begin{itemize}
\item {Proveniência:(De \textunderscore federal\textunderscore )}
\end{itemize}
Fórma de govêrno, que consiste na reunião de vários Estados numa só nação, conservando elles autonomia, fóra dos negócios de interesse commum.
\section{Federalista}
\begin{itemize}
\item {Grp. gram.:adj.}
\end{itemize}
\begin{itemize}
\item {Grp. gram.:M.}
\end{itemize}
\begin{itemize}
\item {Proveniência:(De \textunderscore federal\textunderscore )}
\end{itemize}
Relativo ao federalismo.
Partidário do federalismo.
\section{Federalizar}
\begin{itemize}
\item {Grp. gram.:v. t.}
\end{itemize}
\begin{itemize}
\item {Utilização:bras}
\end{itemize}
\begin{itemize}
\item {Utilização:Neol.}
\end{itemize}
Tornar federal.
\section{Federar}
\begin{itemize}
\item {Grp. gram.:v. t.}
\end{itemize}
\begin{itemize}
\item {Proveniência:(Lat. \textunderscore foederare\textunderscore )}
\end{itemize}
Reunir em federação; confederar.
\section{Federativo}
\begin{itemize}
\item {Grp. gram.:adj.}
\end{itemize}
\begin{itemize}
\item {Proveniência:(De \textunderscore federar\textunderscore )}
\end{itemize}
Relativo a uma federação ou a uma confederação.
\section{Fédia}
\begin{itemize}
\item {Grp. gram.:f.}
\end{itemize}
Moéda de Cambaia e Baçaim, correspondente a 15 reis.
(Ar. \textunderscore fidia\textunderscore , do marata)
\section{Fédia}
\begin{itemize}
\item {Grp. gram.:f.}
\end{itemize}
\begin{itemize}
\item {Proveniência:(Do lat. \textunderscore foedia\textunderscore )}
\end{itemize}
Gênero de plantas valerianáceas.
\section{Fedífrago}
\begin{itemize}
\item {Grp. gram.:adj.}
\end{itemize}
\begin{itemize}
\item {Proveniência:(Lat. \textunderscore foedrifagus\textunderscore )}
\end{itemize}
Que quebra a fé dos tratados.
Que não executa um compromisso, que quebra uma alliança.
Desleal.
\section{Fedigueira}
\begin{itemize}
\item {Grp. gram.:f.}
\end{itemize}
\begin{itemize}
\item {Utilização:Prov.}
\end{itemize}
\begin{itemize}
\item {Utilização:trasm.}
\end{itemize}
O mesmo que \textunderscore cornalheira\textunderscore .
(Por \textunderscore fedegueira\textunderscore . Cp. \textunderscore fedegosa\textunderscore )
\section{Fedinchar}
\begin{itemize}
\item {Grp. gram.:v. i.}
\end{itemize}
\begin{itemize}
\item {Utilização:Açor}
\end{itemize}
Choramingar (a criança)
\section{Fédito}
\begin{itemize}
\item {Grp. gram.:m.}
\end{itemize}
\begin{itemize}
\item {Utilização:Pop.}
\end{itemize}
O mesmo que \textunderscore fétido\textunderscore .
(Metáth. de \textunderscore fétido\textunderscore )
\section{Fedo}
\begin{itemize}
\item {Grp. gram.:adj.}
\end{itemize}
(V.feio)
\section{Fedoca}
\begin{itemize}
\item {Grp. gram.:f.}
\end{itemize}
\begin{itemize}
\item {Utilização:Prov.}
\end{itemize}
\begin{itemize}
\item {Utilização:alg.}
\end{itemize}
\begin{itemize}
\item {Proveniência:(De \textunderscore fedo\textunderscore )}
\end{itemize}
Us. na \textunderscore loc. adv. á fedoca\textunderscore , desajeitadamente, feiamente.
\section{Fedonhar}
\begin{itemize}
\item {Grp. gram.:v. i.}
\end{itemize}
\begin{itemize}
\item {Utilização:Prov.}
\end{itemize}
\begin{itemize}
\item {Utilização:trasm.}
\end{itemize}
\begin{itemize}
\item {Proveniência:(De \textunderscore fedonho\textunderscore )}
\end{itemize}
Sêr enfadonho, importunar.
\section{Fedonho}
\begin{itemize}
\item {Grp. gram.:adj.}
\end{itemize}
\begin{itemize}
\item {Utilização:Prov.}
\end{itemize}
\begin{itemize}
\item {Utilização:trasm.}
\end{itemize}
Enfadonho, importuno.
Fétido.
\section{Fedor}
\begin{itemize}
\item {Grp. gram.:m.}
\end{itemize}
\begin{itemize}
\item {Proveniência:(Lat. \textunderscore foetor\textunderscore )}
\end{itemize}
Cheiro nauseabundo; mau cheiro.
\section{Fedorenta}
\begin{itemize}
\item {Grp. gram.:f.}
\end{itemize}
\begin{itemize}
\item {Proveniência:(De \textunderscore fedorento\textunderscore )}
\end{itemize}
O mesmo que \textunderscore cainca\textunderscore .
Planta crucífera, (\textunderscore eruca sativa\textunderscore , Lin.).
\section{Fedorentamente}
\begin{itemize}
\item {Grp. gram.:adv.}
\end{itemize}
\begin{itemize}
\item {Proveniência:(De \textunderscore fedorento\textunderscore )}
\end{itemize}
Com fedor.
\section{Fedorentina}
\begin{itemize}
\item {Grp. gram.:f.}
\end{itemize}
(V.fedentina)
\section{Fedorento}
\begin{itemize}
\item {Grp. gram.:adj.}
\end{itemize}
\begin{itemize}
\item {Utilização:Pop.}
\end{itemize}
\begin{itemize}
\item {Proveniência:(De \textunderscore fedor\textunderscore )}
\end{itemize}
Que lança mau cheiro.
Fétido.
Rabugento.
\section{Fedorina}
\begin{itemize}
\item {Grp. gram.:f.}
\end{itemize}
\begin{itemize}
\item {Utilização:T. da Bairrada}
\end{itemize}
\begin{itemize}
\item {Proveniência:(De \textunderscore fedor\textunderscore )}
\end{itemize}
O mesmo que \textunderscore fedentina\textunderscore .
\section{Feduçada}
\begin{itemize}
\item {Grp. gram.:f.}
\end{itemize}
\begin{itemize}
\item {Utilização:Prov.}
\end{itemize}
\begin{itemize}
\item {Utilização:trasm.}
\end{itemize}
Maçada, impertinência, séca.
(Cp. \textunderscore fedúcia\textunderscore )
\section{Fedúcia}
\begin{itemize}
\item {Grp. gram.:m.  e  f.}
\end{itemize}
O mesmo que \textunderscore fedúncia\textunderscore .
\section{Feduço}
\begin{itemize}
\item {Grp. gram.:adj.}
\end{itemize}
\begin{itemize}
\item {Utilização:Prov.}
\end{itemize}
\begin{itemize}
\item {Utilização:trasm.}
\end{itemize}
Enfadonho, aborrecido, impertinente.
(Cp. \textunderscore fedúcia\textunderscore )
\section{Fedúncia}
\begin{itemize}
\item {Grp. gram.:m.  e  f.}
\end{itemize}
\begin{itemize}
\item {Utilização:Pop.}
\end{itemize}
\begin{itemize}
\item {Proveniência:(De \textunderscore feder\textunderscore )}
\end{itemize}
Pessôa pretensiosa, que de tudo desdenha, a quem tudo fede.
\section{Fedúncio}
\begin{itemize}
\item {Grp. gram.:adj.}
\end{itemize}
\begin{itemize}
\item {Utilização:Pop.}
\end{itemize}
O mesmo que \textunderscore feduço\textunderscore .
\section{Fegarite}
\begin{itemize}
\item {Grp. gram.:f.}
\end{itemize}
Espécie de estomatite, que é endêmica nalguns pontos de Espanha.
(Cast. \textunderscore fegarites\textunderscore )
\section{Feiamente}
\begin{itemize}
\item {Grp. gram.:adv.}
\end{itemize}
\begin{itemize}
\item {Proveniência:(De \textunderscore feio\textunderscore )}
\end{itemize}
Com fealdade.
\section{Feição}
\begin{itemize}
\item {Grp. gram.:f.}
\end{itemize}
\begin{itemize}
\item {Grp. gram.:Pl.}
\end{itemize}
\begin{itemize}
\item {Proveniência:(Lat. \textunderscore factio\textunderscore )}
\end{itemize}
Feitio, fórma.
Aspecto.
Maneira.
Jeito.
Índole, carácter.
Bôa disposição: \textunderscore navegar á feição do vento\textunderscore .
Delineamento do rosto humano: \textunderscore feições agradáveis\textunderscore .
Lados da coronha.
\section{Feijão}
\begin{itemize}
\item {Grp. gram.:m.}
\end{itemize}
\begin{itemize}
\item {Proveniência:(Lat. \textunderscore faseolus\textunderscore )}
\end{itemize}
Semente do feijoeiro.
Vagem, que contém essa semente.
Planta, que produz essa vágem.
Feijoeiro.
Variedade de uva minhota, também conhecida por \textunderscore feijôa\textunderscore .
\section{Feijão-do-mato}
\begin{itemize}
\item {Grp. gram.:m.}
\end{itemize}
Arbusto africano, sarmentoso, de folhas alternas, e flores papilionáceas, sôbre um longo pecíolo côr de rosa.
\section{Feijão-molle}
\begin{itemize}
\item {Grp. gram.:m.}
\end{itemize}
Variedade de uva feijôa.
\section{Feijão-pical}
\begin{itemize}
\item {Grp. gram.:m.}
\end{itemize}
Variedade de uva feijôa.
\section{Feijôa}
\begin{itemize}
\item {Grp. gram.:f.}
\end{itemize}
Casta de uva preta do Minho.
\section{Feijoada}
\begin{itemize}
\item {Grp. gram.:f.}
\end{itemize}
Grande porção de feijões.
Preparação culinâria de feijões.
\section{Feijoal}
\begin{itemize}
\item {Grp. gram.:m.}
\end{itemize}
\begin{itemize}
\item {Proveniência:(De \textunderscore feijão\textunderscore )}
\end{itemize}
Terreno, onde crescem feijoeiros.
\section{Feijoca}
\begin{itemize}
\item {Grp. gram.:f.}
\end{itemize}
Variedade de feijão grande.
\section{Feijoco}
\begin{itemize}
\item {fónica:jô}
\end{itemize}
\begin{itemize}
\item {Grp. gram.:m.}
\end{itemize}
Producto vulcânico.
(Cp. \textunderscore feijoca\textunderscore )
\section{Feijoeiro}
\begin{itemize}
\item {Grp. gram.:m.}
\end{itemize}
Planta leguminosa, que produz o feijão.
\section{Feijões-celos}
\begin{itemize}
\item {Grp. gram.:m. pl.}
\end{itemize}
\begin{itemize}
\item {Utilização:Prov.}
\end{itemize}
\begin{itemize}
\item {Utilização:trasm.}
\end{itemize}
Tubérculos, que nascem nos soitos.
\section{Feila}
\begin{itemize}
\item {Grp. gram.:m.}
\end{itemize}
\begin{itemize}
\item {Proveniência:(Do lat. \textunderscore faecula\textunderscore  &lt; \textunderscore fecla\textunderscore  &lt; \textunderscore fegla\textunderscore  &lt; \textunderscore feila\textunderscore )}
\end{itemize}
Pó subtil de farinha, faúlha.
\section{Feio}
\begin{itemize}
\item {Grp. gram.:adj.}
\end{itemize}
\begin{itemize}
\item {Utilização:Fig.}
\end{itemize}
\begin{itemize}
\item {Utilização:Des.}
\end{itemize}
\begin{itemize}
\item {Proveniência:(Do lat. \textunderscore foedus\textunderscore )}
\end{itemize}
Que tem aspecto desagradável.
Desproporcionado, disforme.
Opposto á belleza moral; indecoroso; torpe.
Nocivo.
Insupportável.
Trabalhoso.
\section{Feira}
\begin{itemize}
\item {Grp. gram.:f.}
\end{itemize}
\begin{itemize}
\item {Utilização:Fig.}
\end{itemize}
\begin{itemize}
\item {Grp. gram.:Pl.}
\end{itemize}
\begin{itemize}
\item {Utilização:Prov.}
\end{itemize}
\begin{itemize}
\item {Utilização:minh.}
\end{itemize}
\begin{itemize}
\item {Proveniência:(Do lat. \textunderscore feria\textunderscore )}
\end{itemize}
Lugar público e descoberto, em que se expõem e vendem mercadorias.
Designação complementar da maior parte dos dias da semana: \textunderscore quarta-feira\textunderscore , \textunderscore quinta-feira...\textunderscore 
Balbúrdia.
Falario.
O mesmo que compras.
\section{Feiral}
\begin{itemize}
\item {Grp. gram.:adj.}
\end{itemize}
Relativo a feira: \textunderscore theatros feiraes\textunderscore .
\section{Feirante}
\begin{itemize}
\item {Grp. gram.:m.}
\end{itemize}
\begin{itemize}
\item {Proveniência:(De \textunderscore feirar\textunderscore )}
\end{itemize}
Aquelle que vende na feira.
Aquelle que vai á feira.
\section{Feirão}
\begin{itemize}
\item {Grp. gram.:m.}
\end{itemize}
\begin{itemize}
\item {Utilização:Prov.}
\end{itemize}
\begin{itemize}
\item {Utilização:Prov.}
\end{itemize}
\begin{itemize}
\item {Utilização:minh.}
\end{itemize}
O mesmo que \textunderscore feirante\textunderscore . (Colhido em Vallongo)
Feira pequena.
\section{Feirar}
\begin{itemize}
\item {Grp. gram.:v. t.}
\end{itemize}
\begin{itemize}
\item {Utilização:Prov.}
\end{itemize}
\begin{itemize}
\item {Utilização:minh.}
\end{itemize}
\begin{itemize}
\item {Proveniência:(Do lat. \textunderscore feriari\textunderscore )}
\end{itemize}
O mesmo que \textunderscore enfeirar\textunderscore .
Fazer compras, na feira ou fóra della:«\textunderscore nas feiras já feiro\textunderscore ». Castilho, \textunderscore Escav. Poét.\textunderscore 
\section{Feireira}
\begin{itemize}
\item {Grp. gram.:adj. f.}
\end{itemize}
\begin{itemize}
\item {Utilização:Prov.}
\end{itemize}
\begin{itemize}
\item {Proveniência:(De \textunderscore feira\textunderscore )}
\end{itemize}
Diz-se da rapariga vistosa e garrida.
\section{Feiroto}
\begin{itemize}
\item {fónica:feirô}
\end{itemize}
\begin{itemize}
\item {Grp. gram.:m.}
\end{itemize}
\begin{itemize}
\item {Utilização:Prov.}
\end{itemize}
\begin{itemize}
\item {Utilização:minh.}
\end{itemize}
Pequena feira; feira de pouca importância.
\section{Feita}
\begin{itemize}
\item {Grp. gram.:f.}
\end{itemize}
\begin{itemize}
\item {Proveniência:(De \textunderscore feito\textunderscore ^2)}
\end{itemize}
Acto; occasião: \textunderscore desta feita, esganou-se\textunderscore .
\section{Feital}
\begin{itemize}
\item {Grp. gram.:m.}
\end{itemize}
\begin{itemize}
\item {Utilização:Bras}
\end{itemize}
(V. \textunderscore fetal\textunderscore ^2)
Terra cansada.
\section{Feitão}
\begin{itemize}
\item {Grp. gram.:m.}
\end{itemize}
\begin{itemize}
\item {Utilização:Prov.}
\end{itemize}
\begin{itemize}
\item {Utilização:beir.}
\end{itemize}
O mesmo que \textunderscore féto\textunderscore . (Colhido na Guarda)
\section{Feitar}
\begin{itemize}
\item {Grp. gram.:v. t.}
\end{itemize}
\begin{itemize}
\item {Utilização:Bras. da Baía}
\end{itemize}
\begin{itemize}
\item {Proveniência:(De \textunderscore feito\textunderscore )}
\end{itemize}
O mesmo que \textunderscore fazer\textunderscore .
\section{Feiteira}
\begin{itemize}
\item {Grp. gram.:f.}
\end{itemize}
\begin{itemize}
\item {Utilização:Prov.}
\end{itemize}
\begin{itemize}
\item {Proveniência:(De \textunderscore feito\textunderscore )}
\end{itemize}
Successo, negócio.
\textunderscore Bôa feiteira\textunderscore , bom lanço de rede.
Bom negócio.
Bom resultado.
\section{Feiteira}
\begin{itemize}
\item {Grp. gram.:f.}
\end{itemize}
\begin{itemize}
\item {Utilização:Prov.}
\end{itemize}
\begin{itemize}
\item {Utilização:dur.}
\end{itemize}
\begin{itemize}
\item {Proveniência:(De \textunderscore feito\textunderscore ^3)}
\end{itemize}
Fêto pequeno do mato.
\section{Feitelha}
\begin{itemize}
\item {fónica:tê}
\end{itemize}
\begin{itemize}
\item {Grp. gram.:f.}
\end{itemize}
\begin{itemize}
\item {Utilização:Prov.}
\end{itemize}
\begin{itemize}
\item {Utilização:minh.}
\end{itemize}
\begin{itemize}
\item {Proveniência:(De \textunderscore feito\textunderscore ^3)}
\end{itemize}
Fêto, de fôlhas miúdas.
\section{Feitêm}
\begin{itemize}
\item {Grp. gram.:m.}
\end{itemize}
\begin{itemize}
\item {Utilização:Prov.}
\end{itemize}
\begin{itemize}
\item {Utilização:beir.}
\end{itemize}
O mesmo que \textunderscore féto\textunderscore . (Colhido na Guarda)
\section{Feitiar}
\begin{itemize}
\item {Grp. gram.:v. t.}
\end{itemize}
Dar feitio a.
\section{Feitiçamente}
\begin{itemize}
\item {Grp. gram.:adv.}
\end{itemize}
\begin{itemize}
\item {Proveniência:(De \textunderscore feitiço\textunderscore )}
\end{itemize}
De modo fictício; artificialmente.
\section{Feitiçaria}
\begin{itemize}
\item {Grp. gram.:f.}
\end{itemize}
\begin{itemize}
\item {Utilização:Fig.}
\end{itemize}
\begin{itemize}
\item {Proveniência:(De \textunderscore feitiço\textunderscore )}
\end{itemize}
Emprêgo de feitiços.
Encantamento.
Sortilégio.
Enlêvo, seducção.
\section{Feiticeira}
\begin{itemize}
\item {Grp. gram.:f.}
\end{itemize}
\begin{itemize}
\item {Utilização:Bras}
\end{itemize}
\begin{itemize}
\item {Proveniência:(De \textunderscore feiticeiro\textunderscore )}
\end{itemize}
Mulher, que faz feitiços.
Mulher que seduz ou encanta.
Espécie de abelha preta.
\section{Feiticeiresco}
\begin{itemize}
\item {Grp. gram.:adj.}
\end{itemize}
Relativo a feiticeira.
Próprio de feiticeira. Cf. Castilho, \textunderscore D. Quixote\textunderscore , II, 304.
\section{Feiticeiro}
\begin{itemize}
\item {Grp. gram.:m.}
\end{itemize}
\begin{itemize}
\item {Grp. gram.:Adj.}
\end{itemize}
\begin{itemize}
\item {Proveniência:(De \textunderscore feitiço\textunderscore )}
\end{itemize}
Aquelle que faz feitiços.
Aquelle que encanta, que atrái.
Agradável.
Encantador; seductor.
\section{Feiticismo}
\begin{itemize}
\item {Grp. gram.:m.}
\end{itemize}
Culto e prática de feitiços, entre os indígenas africanos.
\section{Feitiço}
\begin{itemize}
\item {Grp. gram.:adj.}
\end{itemize}
\begin{itemize}
\item {Grp. gram.:M.}
\end{itemize}
\begin{itemize}
\item {Utilização:Fig.}
\end{itemize}
\begin{itemize}
\item {Proveniência:(Do lat. \textunderscore ficticius\textunderscore )}
\end{itemize}
Artificial; postiço; fictício.
Malefício de feiticeiro ou feiticeira.
Objecto, a que se attribuem qualidades sobrenaturaes; amuleto.
Coisa que encanta.
Encanto; fascinação.
Armazem ou taberna, em que se fazem pagamentos aos indígenas, na região do Zaire.
\section{Feitio}
\begin{itemize}
\item {Grp. gram.:m.}
\end{itemize}
\begin{itemize}
\item {Utilização:Des.}
\end{itemize}
\begin{itemize}
\item {Grp. gram.:Loc.}
\end{itemize}
\begin{itemize}
\item {Utilização:fam.}
\end{itemize}
\begin{itemize}
\item {Grp. gram.:Pl.}
\end{itemize}
\begin{itemize}
\item {Utilização:Des.}
\end{itemize}
\begin{itemize}
\item {Proveniência:(De \textunderscore feito\textunderscore )}
\end{itemize}
Feição; configuração.
Maneira.
Qualidade.
Disposição de espírito.
Carácter: \textunderscore tem mau feitio, o João\textunderscore .
Trabalho de um artista, relativamente a um artefacto.
Ornato.
O mesmo que \textunderscore feitura\textunderscore :«\textunderscore ...se ordenou para a censura dos livros e feitio de hum index delles...\textunderscore »Sousa, \textunderscore Vida do Arceb.\textunderscore , I, 236.
\textunderscore Perder o tempo e o feitio\textunderscore , têr-se esforçado debalde.
Excremento de coelho, raposa, etc.
\section{Feito}
\begin{itemize}
\item {Grp. gram.:adj.}
\end{itemize}
\begin{itemize}
\item {Grp. gram.:Loc.}
\end{itemize}
\begin{itemize}
\item {Utilização:fam.}
\end{itemize}
\begin{itemize}
\item {Proveniência:(Do lat. \textunderscore factus\textunderscore )}
\end{itemize}
Acostumado, afeito:«\textunderscore braço ás armas feito.\textunderscore »\textunderscore Lusíadas\textunderscore .
Exercitado.
Crescido, adulto: \textunderscore homem feito\textunderscore .
Amadurecido.
Resolvido, assente: \textunderscore é contrato feito\textunderscore .
Conformado.
Constituido: \textunderscore casa feita de tijolo\textunderscore .
Preparado para ferir ou lutar, (falando-se de armas):«\textunderscore ...com as espadas feitas ferozmente lhe sobrevinha\textunderscore ». Filinto, \textunderscore D. Man.\textunderscore , II, 5;«\textunderscore correm com as lanças feitas\textunderscore ». \textunderscore Idem\textunderscore , \textunderscore Ibidem\textunderscore , I, 27.
\textunderscore Está feito\textunderscore , na realidade; mesmo assim: \textunderscore está feito, não te saíste tão mal, como eu suppunha\textunderscore .
\section{Feito}
\begin{itemize}
\item {Grp. gram.:m.}
\end{itemize}
\begin{itemize}
\item {Grp. gram.:Loc. adv.}
\end{itemize}
\begin{itemize}
\item {Grp. gram.:Pl.}
\end{itemize}
\begin{itemize}
\item {Proveniência:(Lat. \textunderscore factum\textunderscore )}
\end{itemize}
Facto.
Acto.
Empresa.
Façanha.
O parceiro que, no voltarete e em outros jogos, declara têr jôgo, e joga contra os outros parceiros.
\textunderscore De feito\textunderscore , effectivamente, com effeito.
Processos judiciaes: \textunderscore mandou chamar o fiel dos feitos\textunderscore .
\section{Feito}
\begin{itemize}
\item {Grp. gram.:m.}
\end{itemize}
\begin{itemize}
\item {Utilização:Prov.}
\end{itemize}
\begin{itemize}
\item {Utilização:minh.}
\end{itemize}
O mesmo que \textunderscore fêto\textunderscore .
\section{Feitor}
\begin{itemize}
\item {Grp. gram.:m.}
\end{itemize}
\begin{itemize}
\item {Utilização:Prov.}
\end{itemize}
\begin{itemize}
\item {Grp. gram.:Adj.}
\end{itemize}
\begin{itemize}
\item {Proveniência:(Do lat. \textunderscore factor\textunderscore )}
\end{itemize}
Gestor.
Administrador de bens alheios.
Rendeiro.
Capataz.
Aquelle que superintende nos trabalhadores.
Fabricante.
Fazedor.
\section{Feitorar}
\begin{itemize}
\item {Grp. gram.:v. t.}
\end{itemize}
\begin{itemize}
\item {Utilização:Bras}
\end{itemize}
O mesmo que \textunderscore feitorizar\textunderscore .
\section{Feitoria}
\begin{itemize}
\item {Grp. gram.:f.}
\end{itemize}
\begin{itemize}
\item {Utilização:Ant.}
\end{itemize}
\begin{itemize}
\item {Proveniência:(De \textunderscore feitor\textunderscore )}
\end{itemize}
Administração, exercida por feitor.
Estabelecimento commercial.
Impostos de colónias, geridos por feitores da Fazenda Pública.
Um dos processos de fabricar vinho.
Fabríco de vinho.
Fazenda, prédio rústico.
\section{Feitorização}
\begin{itemize}
\item {Grp. gram.:f.}
\end{itemize}
Acto de feitorizar. Cf. Camillo, \textunderscore Noites de Insómn.\textunderscore , IV, 8.
\section{Feitorizar}
\begin{itemize}
\item {Grp. gram.:v. t.}
\end{itemize}
\begin{itemize}
\item {Proveniência:(De \textunderscore feitoria\textunderscore )}
\end{itemize}
Gerir como feitor.
Superintender em.
Fabricar (vinho).
\section{Feitura}
\begin{itemize}
\item {Grp. gram.:f.}
\end{itemize}
\begin{itemize}
\item {Proveniência:(Do lat. \textunderscore factura\textunderscore )}
\end{itemize}
Acto, effeito ou modo de fazer.
Effeito.
Obra, trabalho.
\section{Feitureira}
\begin{itemize}
\item {Grp. gram.:f.}
\end{itemize}
\begin{itemize}
\item {Proveniência:(De \textunderscore feitura\textunderscore )}
\end{itemize}
Mulher, que faz o carapim para os sapatos de liga.
\section{Feiume}
\begin{itemize}
\item {Grp. gram.:m.}
\end{itemize}
\begin{itemize}
\item {Utilização:Bras. de Minas}
\end{itemize}
O mesmo que \textunderscore fealdade\textunderscore .
Coisa feia: \textunderscore não chores, menina; que feiume\textunderscore !
\section{Feixe}
\begin{itemize}
\item {Grp. gram.:m.}
\end{itemize}
\begin{itemize}
\item {Utilização:Prov.}
\end{itemize}
\begin{itemize}
\item {Utilização:trasm.}
\end{itemize}
\begin{itemize}
\item {Utilização:Fig.}
\end{itemize}
\begin{itemize}
\item {Proveniência:(Do lat. \textunderscore fascis\textunderscore )}
\end{itemize}
Gavela; braçado.
Mólho.
Vara de lagar.
Acervo, porção: \textunderscore um feixe de novidades\textunderscore .
\section{Feixota}
\begin{itemize}
\item {Grp. gram.:f.}
\end{itemize}
\begin{itemize}
\item {Utilização:Prov.}
\end{itemize}
\begin{itemize}
\item {Utilização:alent.}
\end{itemize}
\begin{itemize}
\item {Proveniência:(De \textunderscore feixe\textunderscore )}
\end{itemize}
Mólho grande.
\section{Fel}
\begin{itemize}
\item {Grp. gram.:m.}
\end{itemize}
\begin{itemize}
\item {Utilização:Fig.}
\end{itemize}
\begin{itemize}
\item {Proveniência:(Lat. \textunderscore fel\textunderscore )}
\end{itemize}
Matéria animal, líquida, amarelada ou esverdeada, que se gera e se acumula numa vesícula adherente ao fígado.
Bílis.
Vesícula, que contém essa matéria.
Mau humor.
Azedume; ódio.
\textunderscore Fel da terra\textunderscore , nome vulgar da centáurea menor.
\section{Fela}
\begin{itemize}
\item {Grp. gram.:f.}
\end{itemize}
\begin{itemize}
\item {Utilização:Gír.}
\end{itemize}
Cara.
\section{Felá}
\begin{itemize}
\item {Grp. gram.:m.}
\end{itemize}
\begin{itemize}
\item {Proveniência:(Do ár. \textunderscore felah\textunderscore )}
\end{itemize}
Homem de casta inferior, entre os Egýpcios, dedicado aos trabalhos mais rudes.
\section{Felan}
\begin{itemize}
\item {Grp. gram.:f.}
\end{itemize}
Concha bivalve do Senegal.
\section{Felatas}
\begin{itemize}
\item {Grp. gram.:m. pl.}
\end{itemize}
Um dos povos mais importantes da África central.
\section{Feldspáthico}
\begin{itemize}
\item {Grp. gram.:adj.}
\end{itemize}
Que tem feldspatho.
\section{Feldspatho}
\begin{itemize}
\item {Grp. gram.:m.}
\end{itemize}
\begin{itemize}
\item {Proveniência:(Do al. \textunderscore feld\textunderscore  + \textunderscore spath\textunderscore )}
\end{itemize}
Mineral duro e laminoso, composto de sílica, alumina e potassa, e de crystallização semelhante á do crystal de pedra.
\section{Feldspathoide}
\begin{itemize}
\item {Grp. gram.:m.}
\end{itemize}
\begin{itemize}
\item {Proveniência:(De \textunderscore feldspatho\textunderscore  + gr. \textunderscore eidos\textunderscore )}
\end{itemize}
Mineral, que, pela sua composição chímica e pela sua associação, representa papel análogo ao dos feldspathos, como a leucite e a nephelite.
\section{Feldspático}
\begin{itemize}
\item {Grp. gram.:adj.}
\end{itemize}
Que tem feldspato.
\section{Feldspato}
\begin{itemize}
\item {Grp. gram.:m.}
\end{itemize}
\begin{itemize}
\item {Proveniência:(Do al. \textunderscore feld\textunderscore  + \textunderscore spath\textunderscore )}
\end{itemize}
Mineral duro e laminoso, composto de sílica, alumina e potassa, e de cristalização semelhante á do cristal de pedra.
\section{Feldspatoide}
\begin{itemize}
\item {Grp. gram.:m.}
\end{itemize}
\begin{itemize}
\item {Proveniência:(De \textunderscore feldspatho\textunderscore  + gr. \textunderscore eidos\textunderscore )}
\end{itemize}
Mineral, que, pela sua composição química e pela sua associação, representa papel análogo ao dos feldspatos, como a leucite e a nefelite.
\section{Féleo}
\begin{itemize}
\item {Grp. gram.:adj.}
\end{itemize}
\begin{itemize}
\item {Proveniência:(Lat. \textunderscore felleus\textunderscore )}
\end{itemize}
Relativo ao fel.
\section{Felga}
\begin{itemize}
\item {Grp. gram.:f.}
\end{itemize}
\begin{itemize}
\item {Utilização:Prov.}
\end{itemize}
Pequeno torrão ou torrão desfeito.
Raízes, que se avistam acima dos torrões de terra lavrada.
(Por \textunderscore filga\textunderscore , do lat. hypoth. \textunderscore filica\textunderscore . Cp. \textunderscore filicatus\textunderscore , de \textunderscore filix\textunderscore . Esta minha affirmação é corroborada por se dar o nome de \textunderscore felgueira\textunderscore  a uma espécie de fêto e por termos o n. p. \textunderscore Filgueiras\textunderscore  = \textunderscore Felgueiras\textunderscore )
\section{Felgudo}
\begin{itemize}
\item {Grp. gram.:adj.}
\end{itemize}
Coberto de felga; que tem muita felga.
\section{Felgueira}
\begin{itemize}
\item {Grp. gram.:f.}
\end{itemize}
O mesmo que \textunderscore dentebrum\textunderscore .
Terreno, onde há felgas ou onde crescem fêtos.
\section{Felice}
\begin{itemize}
\item {Grp. gram.:adj.}
\end{itemize}
\begin{itemize}
\item {Utilização:Des.}
\end{itemize}
O mesmo que \textunderscore feliz\textunderscore .
\section{Felícia}
\begin{itemize}
\item {Grp. gram.:f.}
\end{itemize}
\begin{itemize}
\item {Utilização:Fam.}
\end{itemize}
\begin{itemize}
\item {Proveniência:(De \textunderscore felice\textunderscore )}
\end{itemize}
Felicidade; ventura. Cf. Camillo, \textunderscore Corja\textunderscore , 280.
\section{Felícia}
\begin{itemize}
\item {Grp. gram.:f.}
\end{itemize}
Gênero de plantas compostas.
\section{Felicidade}
\begin{itemize}
\item {Grp. gram.:f.}
\end{itemize}
\begin{itemize}
\item {Proveniência:(Lat. \textunderscore felicitas\textunderscore )}
\end{itemize}
Qualidade ou estado de quem é feliz.
Bem-estar; contentamento; ventura.
Bom resultado, bom êxito.
\section{Felicíssimo}
\begin{itemize}
\item {Grp. gram.:adj.}
\end{itemize}
\begin{itemize}
\item {Proveniência:(De \textunderscore felice\textunderscore )}
\end{itemize}
Muito feliz.
\section{Felicitação}
\begin{itemize}
\item {Grp. gram.:f.}
\end{itemize}
Acto de felicitar.
\section{Felicitador}
\begin{itemize}
\item {Grp. gram.:adj.}
\end{itemize}
Que felicita.
\section{Felicitar}
\begin{itemize}
\item {Grp. gram.:v. t.}
\end{itemize}
\begin{itemize}
\item {Proveniência:(Lat. \textunderscore felicitare\textunderscore )}
\end{itemize}
Tornar feliz.
Congratular-se com.
Dirigir parabens ou cumprimentos a.--Nesta última accepção, é gallicismo, geralmente admittido.
\section{Felídeos}
\begin{itemize}
\item {Grp. gram.:m. Pl.}
\end{itemize}
\begin{itemize}
\item {Utilização:Zool.}
\end{itemize}
\begin{itemize}
\item {Proveniência:(Do lat. \textunderscore felis\textunderscore  + gr. \textunderscore eidos\textunderscore )}
\end{itemize}
Família de mammíferos, que tem por typo o gato.
\section{Felino}
\begin{itemize}
\item {Grp. gram.:adj.}
\end{itemize}
\begin{itemize}
\item {Utilização:Fig.}
\end{itemize}
\begin{itemize}
\item {Grp. gram.:M. pl.}
\end{itemize}
\begin{itemize}
\item {Proveniência:(Lat. \textunderscore felinus\textunderscore )}
\end{itemize}
Relativo ao gato.
Semelhante ao gato.
Fingido; traiçoeiro.
Família de animaes mammíferos, que tem por typo o gato.
\section{Felipe}
\begin{itemize}
\item {Grp. gram.:m.}
\end{itemize}
\begin{itemize}
\item {Utilização:Bras. do N}
\end{itemize}
Saco de coiro, para guardar comida.
\section{Félis}
\begin{itemize}
\item {Grp. gram.:m.}
\end{itemize}
\begin{itemize}
\item {Utilização:Prov.}
\end{itemize}
\begin{itemize}
\item {Utilização:trasm.}
\end{itemize}
O mesmo que \textunderscore gato\textunderscore .
\section{Felistreca}
\begin{itemize}
\item {Grp. gram.:f.}
\end{itemize}
\begin{itemize}
\item {Utilização:Prov.}
\end{itemize}
\begin{itemize}
\item {Utilização:minh.}
\end{itemize}
Mulher feia e mal vestida.
\section{Felistria}
\begin{itemize}
\item {Grp. gram.:f.}
\end{itemize}
O mesmo que \textunderscore flostria\textunderscore . Cf. Abb. Jazente, II, 22.
\section{Feliz}
\begin{itemize}
\item {Grp. gram.:adj.}
\end{itemize}
\begin{itemize}
\item {Proveniência:(Lat. \textunderscore felix\textunderscore )}
\end{itemize}
Próspero; afortunado: \textunderscore um anno feliz\textunderscore .
Satisfeito: \textunderscore sentir-se feliz\textunderscore .
Abençoado.
Bem imaginado; bem combinado: \textunderscore uma ideia feliz\textunderscore .
Bem executado.
Que teve bom êxito: \textunderscore empresa feliz\textunderscore .
\section{Felizão}
\begin{itemize}
\item {Grp. gram.:m.}
\end{itemize}
\begin{itemize}
\item {Utilização:Fam.}
\end{itemize}
Homem feliz, que tem bôa sorte.
\section{Felizardo}
\begin{itemize}
\item {Grp. gram.:m.}
\end{itemize}
\begin{itemize}
\item {Utilização:Chul.}
\end{itemize}
O mesmo que \textunderscore felizão\textunderscore .
\section{Felizmente}
\begin{itemize}
\item {Grp. gram.:adv.}
\end{itemize}
De modo feliz; venturosamente; com bôa fortuna.
\section{Feliz-meu-bem}
\begin{itemize}
\item {Grp. gram.:m.}
\end{itemize}
\begin{itemize}
\item {Utilização:Bras. do S}
\end{itemize}
Bailado popular, espécie de fandango.
\section{Fellatas}
\begin{itemize}
\item {Grp. gram.:m. pl.}
\end{itemize}
Um dos povos mais importantes da África central.
\section{Félleo}
\begin{itemize}
\item {Grp. gram.:adj.}
\end{itemize}
\begin{itemize}
\item {Proveniência:(Lat. \textunderscore felleus\textunderscore )}
\end{itemize}
Relativo ao fel.
\section{Felonia}
\begin{itemize}
\item {Grp. gram.:f.}
\end{itemize}
\begin{itemize}
\item {Proveniência:(Do b. lat. \textunderscore felo\textunderscore )}
\end{itemize}
Revolta de um vassallo contra seu senhor.
Traição.
Crueldade.
\section{Felosa}
\begin{itemize}
\item {Grp. gram.:f.}
\end{itemize}
\begin{itemize}
\item {Utilização:Pop.}
\end{itemize}
O mesmo que \textunderscore folosa\textunderscore . Cf. M. Paulino, \textunderscore Aves da Península\textunderscore .
Mulher muito magra e fraca.
\section{Felpa}
\begin{itemize}
\item {Grp. gram.:f.}
\end{itemize}
\begin{itemize}
\item {Proveniência:(It. \textunderscore felpa\textunderscore )}
\end{itemize}
Pêlo saliente nos estofos.
Pennugem dos animaes.
Lanugem de fôlhas ou frutos; carepa.
\section{Felpado}
\begin{itemize}
\item {Grp. gram.:adj.}
\end{itemize}
(V.felpudo)
\section{Félpo}
\begin{itemize}
\item {Grp. gram.:m.}
\end{itemize}
\begin{itemize}
\item {Grp. gram.:Adj.}
\end{itemize}
O mesmo que \textunderscore felpa\textunderscore .
O mesmo que \textunderscore felpudo\textunderscore .
\section{Fêlpo}
\begin{itemize}
\item {Grp. gram.:m.}
\end{itemize}
\begin{itemize}
\item {Utilização:Prov.}
\end{itemize}
\begin{itemize}
\item {Utilização:alg.}
\end{itemize}
Acto de enfelpar.
Conjunto de indivíduos agarrados uns aos outros.
Guerreia.
\section{Felposo}
\begin{itemize}
\item {Grp. gram.:adj.}
\end{itemize}
\begin{itemize}
\item {Proveniência:(De \textunderscore felpa\textunderscore )}
\end{itemize}
Diz-se do tecido orgânico da face plantar do casco do cavallo.
\section{Felpudo}
\begin{itemize}
\item {Grp. gram.:adj.}
\end{itemize}
Que tem felpa.
\section{Felsítico}
\begin{itemize}
\item {Grp. gram.:adj.}
\end{itemize}
Relativo ao felsito.
\section{Felsito}
\begin{itemize}
\item {Grp. gram.:m.}
\end{itemize}
\begin{itemize}
\item {Proveniência:(Do al. \textunderscore fels\textunderscore , penedo)}
\end{itemize}
Mineral, semelhante ao silex.
\section{Feltradeira}
\begin{itemize}
\item {Grp. gram.:f.}
\end{itemize}
\begin{itemize}
\item {Proveniência:(De \textunderscore feltrar\textunderscore )}
\end{itemize}
Mulher, que feltra ou apara o pêlo das pelles para chapelaria.
Máquina para o mesmo effeito.
\section{Feltragem}
\begin{itemize}
\item {Grp. gram.:f.}
\end{itemize}
Acto de feltrar.
\section{Feltrar}
\begin{itemize}
\item {Grp. gram.:v. t.}
\end{itemize}
\begin{itemize}
\item {Grp. gram.:V. i.}
\end{itemize}
Estofar.
Fabricar feltro.
\section{Feltreiro}
\begin{itemize}
\item {Grp. gram.:m.}
\end{itemize}
\begin{itemize}
\item {Proveniência:(De \textunderscore feltro\textunderscore )}
\end{itemize}
Carneiro português, de casta ordinaria, de pouca lan, grossa e sêca.
\section{Feltro}
\begin{itemize}
\item {fónica:fêl}
\end{itemize}
\begin{itemize}
\item {Grp. gram.:m.}
\end{itemize}
\begin{itemize}
\item {Grp. gram.:Pl.}
\end{itemize}
\begin{itemize}
\item {Proveniência:(Do ant. al. \textunderscore filt\textunderscore ?)}
\end{itemize}
Espécie de estôfo de lan ou de pêlo, feito por empastamento e applicado principalmente no fabrico de chapéus, pantufos, etc.
Forros de metal nas caldeiras de vapor, para obstar á irradiação do calórico.
\section{Feltroso}
\begin{itemize}
\item {Grp. gram.:adj.}
\end{itemize}
\begin{itemize}
\item {Proveniência:(De \textunderscore feltro\textunderscore )}
\end{itemize}
Diz-se do velo de lan ordinária, grossa e sêca.
\section{Feltrudo}
\begin{itemize}
\item {Grp. gram.:adj.}
\end{itemize}
Feito de feltro. Cf. Pant. de Aveiro, \textunderscore Itiner.\textunderscore , 25, (2.^a ed.)
\section{Felugem}
\begin{itemize}
\item {Grp. gram.:f.}
\end{itemize}
(V.fuligem)
\section{Felugento}
\begin{itemize}
\item {Grp. gram.:adj.}
\end{itemize}
Que tem felugem. Cf. Camillo, \textunderscore M. da Fonte\textunderscore , 142.
\section{Felupes}
\begin{itemize}
\item {Grp. gram.:m. pl.}
\end{itemize}
Gentios da Guiné portuguesa.
\section{Felupo}
\begin{itemize}
\item {Grp. gram.:m.}
\end{itemize}
Grupo de línguas africanas, a que pertence o \textunderscore biafada\textunderscore  e o \textunderscore papel\textunderscore .
\section{Fema}
\begin{itemize}
\item {Grp. gram.:f.}
\end{itemize}
\begin{itemize}
\item {Utilização:T. de Serpa}
\end{itemize}
O mesmo que \textunderscore fêmea\textunderscore .
\section{Fêmea}
\begin{itemize}
\item {Grp. gram.:f.}
\end{itemize}
\begin{itemize}
\item {Utilização:Ext.}
\end{itemize}
\begin{itemize}
\item {Utilização:Gír.}
\end{itemize}
\begin{itemize}
\item {Proveniência:(Do lat. \textunderscore femina\textunderscore )}
\end{itemize}
Mulher.
Qualquer animal do sexo feminino.
Barregan.
Círculo, em que se engatam os machos que seguram o leme.
Fio circular de metal, em que se engancha o colchete.
Fechadura.
\section{Femeaço}
\begin{itemize}
\item {Grp. gram.:m.}
\end{itemize}
\begin{itemize}
\item {Utilização:Pop.}
\end{itemize}
\begin{itemize}
\item {Proveniência:(De \textunderscore fêmea\textunderscore )}
\end{itemize}
Mulherio.
Reunião de mulheres irrequietas ou dissolutas.
As mulheres dissolutas.
\section{Femeal}
\begin{itemize}
\item {Grp. gram.:adj.}
\end{itemize}
\begin{itemize}
\item {Proveniência:(De \textunderscore fêmea\textunderscore )}
\end{itemize}
O mesmo que \textunderscore feminil\textunderscore .
\section{Femeeiro}
\begin{itemize}
\item {Grp. gram.:m.  e  adj.}
\end{itemize}
O mesmo que \textunderscore femieiro\textunderscore .
\section{Femença}
\begin{itemize}
\item {Grp. gram.:f.}
\end{itemize}
\begin{itemize}
\item {Utilização:Ant.}
\end{itemize}
Attenção; actividade; cuidado.
(Corr. de \textunderscore vehemencia\textunderscore )
\section{Femençar}
\begin{itemize}
\item {Grp. gram.:v. t.}
\end{itemize}
\begin{itemize}
\item {Proveniência:(De \textunderscore femença\textunderscore )}
\end{itemize}
Tratar diligentemente de.
Solicitar; cuidar de.
\section{Fementido}
\begin{itemize}
\item {Grp. gram.:adj.}
\end{itemize}
\begin{itemize}
\item {Proveniência:(De \textunderscore fé\textunderscore  + \textunderscore mentido\textunderscore )}
\end{itemize}
Ardiloso; pérfido; perjuro.
\section{Fêmeo}
\begin{itemize}
\item {Grp. gram.:adj.}
\end{itemize}
\begin{itemize}
\item {Utilização:Bot.}
\end{itemize}
\begin{itemize}
\item {Proveniência:(De \textunderscore fêmea\textunderscore )}
\end{itemize}
Relativo a mulheres ou ao sexo feminino:«\textunderscore Areópago fêmeo\textunderscore ». Castilho, 127.
Relativo a qualquer fêmea:«\textunderscore gado fêmeo...\textunderscore »\textunderscore Tradição\textunderscore , n.^o 7.
Diz-se dos vegetaes que não têm estames.
\section{Fêmia}
\begin{itemize}
\item {Proveniência:(Do lat. \textunderscore femina\textunderscore )}
\end{itemize}
\textunderscore f.\textunderscore  (e der.)
O mesmo ou melhor que \textunderscore fêmea\textunderscore , etc. Cf. \textunderscore Filodemo\textunderscore , V, 4; Castilho, \textunderscore Sabichonas\textunderscore , 121.
\section{Femialmente}
\begin{itemize}
\item {Grp. gram.:adv.}
\end{itemize}
O mesmo que \textunderscore effeminadamente\textunderscore :«\textunderscore viveu femialmente como pachá.\textunderscore »Camillo, \textunderscore Caveira\textunderscore , 214.
\section{Femieiro}
\begin{itemize}
\item {Grp. gram.:m.  e  adj.}
\end{itemize}
\begin{itemize}
\item {Proveniência:(De \textunderscore fêmia\textunderscore )}
\end{itemize}
Homem bordeleiro, apaixonado por fêmeas.
\section{Feminação}
\begin{itemize}
\item {Grp. gram.:f.}
\end{itemize}
O mesmo que \textunderscore effeminação\textunderscore . Cf. F. Manuel, \textunderscore Carta de Guia\textunderscore , c. XXXIV.
\section{Feminal}
\begin{itemize}
\item {Grp. gram.:adj.}
\end{itemize}
\begin{itemize}
\item {Proveniência:(Lat. \textunderscore feminalis\textunderscore )}
\end{itemize}
O mesmo que \textunderscore feminil\textunderscore .
\section{Feminela}
\begin{itemize}
\item {Grp. gram.:f.}
\end{itemize}
\begin{itemize}
\item {Proveniência:(Do lat. \textunderscore femina\textunderscore )}
\end{itemize}
Cylindro do soquete, com que se calca a bala e a pólvora dentro da peça de artilharia.
\section{Femíneo}
\begin{itemize}
\item {Grp. gram.:adj.}
\end{itemize}
\begin{itemize}
\item {Proveniência:(Lat. \textunderscore femineus\textunderscore )}
\end{itemize}
O mesmo que \textunderscore feminil\textunderscore .
\section{Feminidade}
\begin{itemize}
\item {Grp. gram.:f.}
\end{itemize}
\begin{itemize}
\item {Proveniência:(De \textunderscore femíneo\textunderscore )}
\end{itemize}
Qualidade de quem é mulher ou fêmea.
\section{Feminifloro}
\begin{itemize}
\item {Grp. gram.:adj.}
\end{itemize}
\begin{itemize}
\item {Utilização:Bot.}
\end{itemize}
\begin{itemize}
\item {Proveniência:(De \textunderscore femíneo\textunderscore  + \textunderscore flôr\textunderscore )}
\end{itemize}
Diz-se da corôa das plantas, quando formada de flôres femininas.
\section{Feminil}
\begin{itemize}
\item {Grp. gram.:adj.}
\end{itemize}
\begin{itemize}
\item {Utilização:Fig.}
\end{itemize}
\begin{itemize}
\item {Proveniência:(Do lat. \textunderscore femina\textunderscore )}
\end{itemize}
Relativo a mulheres, próprio do sexo feminino; feminino.
Mulherengo.
\section{Feminilidade}
\begin{itemize}
\item {Grp. gram.:f.}
\end{itemize}
\begin{itemize}
\item {Proveniência:(De \textunderscore femínil\textunderscore )}
\end{itemize}
Carácter próprio da mulher.
\section{Feminilmente}
\begin{itemize}
\item {Grp. gram.:adv.}
\end{itemize}
De modo feminil.
\section{Feminino}
\begin{itemize}
\item {Grp. gram.:adj.}
\end{itemize}
\begin{itemize}
\item {Proveniência:(Lat. \textunderscore femininus\textunderscore )}
\end{itemize}
Relativo ao sexo, caracterizado pelo ovário nos animaes e nas plantas.
Próprio de fêmea.
Relativo ás mulheres: \textunderscore a fraqueza feminina\textunderscore .
\section{Feminismo}
\begin{itemize}
\item {Grp. gram.:m.}
\end{itemize}
\begin{itemize}
\item {Utilização:Neol.}
\end{itemize}
\begin{itemize}
\item {Proveniência:(Do lat. \textunderscore femina\textunderscore )}
\end{itemize}
Systema dos que preconizam a ampliação legal dos direitos politicos e civis da mulher, ou a igualdade dos direitos dellas aos do homem.
\section{Feminista}
\begin{itemize}
\item {Grp. gram.:adj.}
\end{itemize}
\begin{itemize}
\item {Utilização:Neol.}
\end{itemize}
\begin{itemize}
\item {Utilização:Bras. de Minas}
\end{itemize}
\begin{itemize}
\item {Grp. gram.:M.}
\end{itemize}
Relativo ao feminismo.
Diz-se do cavallo, cuja reproducção são sempre éguas.
Partidário do feminismo.
(Cp. \textunderscore feminismo\textunderscore )
\section{Feminizar}
\begin{itemize}
\item {Grp. gram.:v. t.}
\end{itemize}
\begin{itemize}
\item {Grp. gram.:V. p.}
\end{itemize}
\begin{itemize}
\item {Proveniência:(Do lat. \textunderscore femina\textunderscore )}
\end{itemize}
Dar feição ou carácter feminino a.
Attribuir gênero feminino a: \textunderscore os antigos feminizaram os vocábulos«fim»,«planeta»,«cometa»\textunderscore , etc.
Assumir os caracteres da fêmea, têr qualidades femininas. Cf. Camillo, \textunderscore Vinho do Porto\textunderscore , 42.
\section{Femoral}
\begin{itemize}
\item {Grp. gram.:adj.}
\end{itemize}
\begin{itemize}
\item {Proveniência:(Do lat. \textunderscore femur\textunderscore )}
\end{itemize}
Relativo ao fêmur.
\section{Fêmur}
\begin{itemize}
\item {Grp. gram.:m.}
\end{itemize}
\begin{itemize}
\item {Proveniência:(Lat. \textunderscore femur\textunderscore  ou \textunderscore femor\textunderscore )}
\end{itemize}
Osso, que constitue a parte sólida da coxa da perna.
Coxa.
Nó superior das patas dos insectos.
\section{Fenação}
\begin{itemize}
\item {Grp. gram.:f.}
\end{itemize}
\begin{itemize}
\item {Utilização:Bras}
\end{itemize}
\begin{itemize}
\item {Proveniência:(De \textunderscore feno\textunderscore )}
\end{itemize}
Processo de conservação da forragem.
\section{Fenasco}
\begin{itemize}
\item {Grp. gram.:m.}
\end{itemize}
\begin{itemize}
\item {Utilização:Prov.}
\end{itemize}
\begin{itemize}
\item {Utilização:trasm.}
\end{itemize}
\begin{itemize}
\item {Proveniência:(De \textunderscore feno\textunderscore )}
\end{itemize}
Restolho alto de searas, entremeado de ervas, o qual se recolhe para alimento de crias no inverno.
\section{Fenasco}
\begin{itemize}
\item {Grp. gram.:m.}
\end{itemize}
\begin{itemize}
\item {Utilização:T. da Ind. Port}
\end{itemize}
\begin{itemize}
\item {Proveniência:(Do conc. \textunderscore feni\textunderscore )}
\end{itemize}
O mesmo que \textunderscore aguardente\textunderscore .
\section{Fenda}
\begin{itemize}
\item {Grp. gram.:f.}
\end{itemize}
\begin{itemize}
\item {Proveniência:(De \textunderscore fender\textunderscore )}
\end{itemize}
Abertura de objecto fendido; fisga; racha.
\section{Fendedor}
\begin{itemize}
\item {Grp. gram.:adj.}
\end{itemize}
\begin{itemize}
\item {Grp. gram.:M.}
\end{itemize}
Que fende.
Aquelle que fende.
\section{Fendeleira}
\begin{itemize}
\item {Grp. gram.:f.}
\end{itemize}
\begin{itemize}
\item {Proveniência:(De \textunderscore fender\textunderscore )}
\end{itemize}
Utensílio, geralmente de ferro, para rachar ou fender; cunha.
\section{Fendente}
\begin{itemize}
\item {Grp. gram.:adj.}
\end{itemize}
\begin{itemize}
\item {Grp. gram.:M.}
\end{itemize}
\begin{itemize}
\item {Utilização:Des.}
\end{itemize}
\begin{itemize}
\item {Proveniência:(Lat. \textunderscore findens\textunderscore )}
\end{itemize}
Que fende.
Golpe com que se fende.
\section{Fender}
\begin{itemize}
\item {Grp. gram.:v. t.}
\end{itemize}
\begin{itemize}
\item {Utilização:Fig.}
\end{itemize}
\begin{itemize}
\item {Proveniência:(Lat. \textunderscore findere\textunderscore )}
\end{itemize}
Fazer abertura, mais ou menos longa e estreita, em: \textunderscore o terremoto fendeu aquella parede\textunderscore .
Rasgar.
Rachar.
Separar.
Commover; atravessar.
\section{Fendimento}
\begin{itemize}
\item {Grp. gram.:m.}
\end{itemize}
Acto ou effeito de fender.
Fenda.
\section{Fendrelho}
\begin{itemize}
\item {fónica:drê}
\end{itemize}
\begin{itemize}
\item {Grp. gram.:m.}
\end{itemize}
\begin{itemize}
\item {Utilização:Prov.}
\end{itemize}
\begin{itemize}
\item {Utilização:minh.}
\end{itemize}
Pedaço.
Farrapo.
\section{Fendrelhar}
\begin{itemize}
\item {Grp. gram.:v. t.}
\end{itemize}
O mesmo que \textunderscore fendrilhar\textunderscore .
\section{Fendrilhar}
\begin{itemize}
\item {Grp. gram.:v. t.}
\end{itemize}
\begin{itemize}
\item {Utilização:Prov.}
\end{itemize}
\begin{itemize}
\item {Utilização:minh.}
\end{itemize}
Reduzir a farrapos.
(Cp. cast. \textunderscore hendrija\textunderscore )
\section{Fendrilheira}
\begin{itemize}
\item {Grp. gram.:f.}
\end{itemize}
\begin{itemize}
\item {Utilização:Prov.}
\end{itemize}
\begin{itemize}
\item {Utilização:minh.}
\end{itemize}
Mulher mal feita, desajeitada.
(Cp. \textunderscore fendrelho\textunderscore )
\section{Fenecer}
\begin{itemize}
\item {Grp. gram.:v. i.}
\end{itemize}
Findar.
Extinguir-se; morrer.
(Por \textunderscore finecer\textunderscore , do lat. \textunderscore finis\textunderscore )
\section{Fenecimento}
\begin{itemize}
\item {Grp. gram.:m.}
\end{itemize}
Acto ou effeito de fenecer.
\section{Feneiro}
\begin{itemize}
\item {Grp. gram.:m.}
\end{itemize}
Casa ou abrigo, em que se recolhe o feno.
\section{Feneratício}
\begin{itemize}
\item {Grp. gram.:adj.}
\end{itemize}
\begin{itemize}
\item {Utilização:P. us.}
\end{itemize}
\begin{itemize}
\item {Proveniência:(Lat. \textunderscore feneraticius\textunderscore )}
\end{itemize}
Emprestado com usura.
\section{Fenestrado}
\begin{itemize}
\item {Grp. gram.:adj.}
\end{itemize}
Que tem janelas.
Diz-se das fôlhas vegetaes, compostas só das nervuras ramificadas e anastomosadas, formando uma espécie de caixilho.
\section{Fenestral}
\begin{itemize}
\item {Grp. gram.:adj.}
\end{itemize}
\begin{itemize}
\item {Grp. gram.:M.}
\end{itemize}
\begin{itemize}
\item {Proveniência:(Lat. \textunderscore fenestralis\textunderscore )}
\end{itemize}
Relativo a janela.
Abertura, por onde entra o ar e a luz, como por janela. Cf. Deusdado, \textunderscore Escorços\textunderscore , 93.
\section{Fenestrar}
\begin{itemize}
\item {Grp. gram.:v. t.}
\end{itemize}
\begin{itemize}
\item {Utilização:Des.}
\end{itemize}
\begin{itemize}
\item {Proveniência:(Lat. \textunderscore fenestrare\textunderscore )}
\end{itemize}
Abrir janelas em.
\section{Fengir}
\begin{itemize}
\item {Grp. gram.:v. i.}
\end{itemize}
\begin{itemize}
\item {Utilização:Prov.}
\end{itemize}
\begin{itemize}
\item {Utilização:trasm.}
\end{itemize}
Tender a massa do pão.
\section{Fenianismo}
\begin{itemize}
\item {Grp. gram.:m.}
\end{itemize}
Associação revolucionária, irlandesa, fundada em 1861, para tirar a Irlanda da dominação inglesa.
\section{Feniano}
\begin{itemize}
\item {Grp. gram.:m.}
\end{itemize}
Membro do fenianismo.
Partidário da doutrina do fenianismo.
\section{Fenígeno}
\begin{itemize}
\item {Grp. gram.:adj.}
\end{itemize}
\begin{itemize}
\item {Proveniência:(Do lat. \textunderscore fenum\textunderscore  + \textunderscore genus\textunderscore )}
\end{itemize}
Que tem a natureza do feno.
\section{Feno}
\begin{itemize}
\item {Grp. gram.:m.}
\end{itemize}
\begin{itemize}
\item {Utilização:Ant.}
\end{itemize}
\begin{itemize}
\item {Proveniência:(Lat. \textunderscore fenum\textunderscore )}
\end{itemize}
Erva ou palha, que cresce sem cultura, e que serve para alimento de gados.
Planta gramínea, (\textunderscore anthoxanthum\textunderscore ).
O mesmo que \textunderscore caruma\textunderscore .
\section{Fenó}
\begin{itemize}
\item {Grp. gram.:m.}
\end{itemize}
\begin{itemize}
\item {Utilização:T. da Índia Port}
\end{itemize}
Cacho de bananas.
\section{Feno-grego}
\begin{itemize}
\item {Grp. gram.:m.}
\end{itemize}
\begin{itemize}
\item {Proveniência:(Do lat. \textunderscore fenum\textunderscore  + \textunderscore graecum\textunderscore )}
\end{itemize}
O mesmo que \textunderscore alforva\textunderscore .
\section{Fentam}
\begin{itemize}
\item {Grp. gram.:m.}
\end{itemize}
\begin{itemize}
\item {Utilização:Pop.}
\end{itemize}
O mesmo que \textunderscore fêtam\textunderscore .
\section{Fêntão}
\begin{itemize}
\item {Grp. gram.:m.}
\end{itemize}
\begin{itemize}
\item {Utilização:Pop.}
\end{itemize}
O mesmo que \textunderscore fêtam\textunderscore .
\section{Fentelha}
\begin{itemize}
\item {fónica:tê}
\end{itemize}
\begin{itemize}
\item {Grp. gram.:f.}
\end{itemize}
\begin{itemize}
\item {Utilização:Prov.}
\end{itemize}
\begin{itemize}
\item {Utilização:minh.}
\end{itemize}
\begin{itemize}
\item {Proveniência:(De \textunderscore fento\textunderscore )}
\end{itemize}
Espécie de fêto, que nasce principalmente nos telhados e fendas das paredes.
\section{Fento}
\begin{itemize}
\item {Grp. gram.:m.}
\end{itemize}
\begin{itemize}
\item {Utilização:Pop.}
\end{itemize}
O mesmo que \textunderscore fêto\textunderscore .
(Gall. \textunderscore fento\textunderscore )
\section{Fêo}
\begin{itemize}
\item {Grp. gram.:adj.}
\end{itemize}
\begin{itemize}
\item {Utilização:Des.}
\end{itemize}
O mesmo que \textunderscore feio\textunderscore .
\section{Fèperjuro}
\begin{itemize}
\item {Grp. gram.:adj.}
\end{itemize}
\begin{itemize}
\item {Proveniência:(De \textunderscore fé\textunderscore  + \textunderscore perjuro\textunderscore )}
\end{itemize}
Fementido.
\section{Fér}
\begin{itemize}
\item {Grp. gram.:v. t.}
\end{itemize}
\begin{itemize}
\item {Utilização:Ant.}
\end{itemize}
O mesmo que \textunderscore fazer\textunderscore . Cf. \textunderscore Cancion. da Vaticana\textunderscore .
\section{Fera}
\begin{itemize}
\item {Grp. gram.:f.}
\end{itemize}
\begin{itemize}
\item {Utilização:Fig.}
\end{itemize}
\begin{itemize}
\item {Proveniência:(Lat. \textunderscore fera\textunderscore )}
\end{itemize}
Animal bravio e carniceiro.
Constellação do hemisphério austral.
Pessôa bárbara, cruel.
\section{Feracidade}
\begin{itemize}
\item {Grp. gram.:f.}
\end{itemize}
\begin{itemize}
\item {Proveniência:(Lat. \textunderscore feracitas\textunderscore )}
\end{itemize}
Qualidade daquillo que é feraz.
Fecundidade, fertilidade.
\section{Feraes}
\begin{itemize}
\item {Grp. gram.:f. pl.}
\end{itemize}
\begin{itemize}
\item {Proveniência:(Lat. \textunderscore feralia\textunderscore )}
\end{itemize}
Festas fúnebres, que os Romanos celebravam em honra dos mortos.
\section{Ferais}
\begin{itemize}
\item {Grp. gram.:f. pl.}
\end{itemize}
\begin{itemize}
\item {Proveniência:(Lat. \textunderscore feralia\textunderscore )}
\end{itemize}
Festas fúnebres, que os Romanos celebravam em honra dos mortos.
\section{Feral}
\begin{itemize}
\item {Grp. gram.:adj.}
\end{itemize}
\begin{itemize}
\item {Proveniência:(Lat. \textunderscore feralis\textunderscore )}
\end{itemize}
Lúgubre; fúnebre.
\section{Feramente}
\begin{itemize}
\item {Grp. gram.:adj.}
\end{itemize}
\begin{itemize}
\item {Proveniência:(De \textunderscore fero\textunderscore )}
\end{itemize}
Com fereza.
\section{Feramina}
\begin{itemize}
\item {Grp. gram.:f.}
\end{itemize}
\begin{itemize}
\item {Proveniência:(Do fr. \textunderscore fer-à-mine\textunderscore )}
\end{itemize}
Pyrite commum.
\section{Feraz}
\begin{itemize}
\item {Grp. gram.:adj.}
\end{itemize}
\begin{itemize}
\item {Proveniência:(Lat. \textunderscore ferax\textunderscore )}
\end{itemize}
Que produz muito.
Fértil; fecundo.
\section{Férculo}
\begin{itemize}
\item {Grp. gram.:m.}
\end{itemize}
\begin{itemize}
\item {Utilização:Ant.}
\end{itemize}
\begin{itemize}
\item {Proveniência:(Lat. \textunderscore ferculum\textunderscore )}
\end{itemize}
Andor ou palanquim, em certas solemnidades pagans. Cf. Castilho, \textunderscore Fastos\textunderscore , I, 135.
\section{Ferdinanda}
\begin{itemize}
\item {Grp. gram.:f.}
\end{itemize}
Gênero de plantas compostas.
\section{Fèrefolha}
\begin{itemize}
\item {Grp. gram.:m.  e  f.}
\end{itemize}
\begin{itemize}
\item {Proveniência:(De \textunderscore ferir\textunderscore  + \textunderscore fôlha\textunderscore )}
\end{itemize}
Pessôa irrequieta, metediça, buliçosa.
\section{Ferentário}
\begin{itemize}
\item {Grp. gram.:m.}
\end{itemize}
\begin{itemize}
\item {Proveniência:(Lat. \textunderscore ferentarius\textunderscore )}
\end{itemize}
Antigo soldado romano, armado á ligeira ou fazendo parte de tropas ligeiras.
\section{Féretro}
\begin{itemize}
\item {Grp. gram.:m.}
\end{itemize}
\begin{itemize}
\item {Proveniência:(Lat. \textunderscore feretrum\textunderscore )}
\end{itemize}
Tumba; caixão mortuário; ataúde.
\section{Fereza}
\begin{itemize}
\item {Grp. gram.:f.}
\end{itemize}
Qualidade daquelle ou daquillo que é fero, cruel.
\section{Ferga}
\begin{itemize}
\item {Grp. gram.:f.}
\end{itemize}
\begin{itemize}
\item {Utilização:Prov.}
\end{itemize}
\begin{itemize}
\item {Utilização:minh.}
\end{itemize}
\begin{itemize}
\item {Utilização:Fig.}
\end{itemize}
Felga.
Desordem, confusão.
\section{Fergusonite}
\begin{itemize}
\item {Grp. gram.:f.}
\end{itemize}
\begin{itemize}
\item {Proveniência:(De \textunderscore Ferguson\textunderscore , n. p.)}
\end{itemize}
Substancia mineral escura.
\section{Féria}
\begin{itemize}
\item {Grp. gram.:f.}
\end{itemize}
\begin{itemize}
\item {Utilização:Bras}
\end{itemize}
\begin{itemize}
\item {Grp. gram.:Pl.}
\end{itemize}
\begin{itemize}
\item {Proveniência:(Lat. \textunderscore feria\textunderscore )}
\end{itemize}
Dia de semana.
Jornal ou salário de operário.
Somma dos salários de uma semana.
Rol de salários.
Folga, descanso.
Apuramento diário das vendas de um estabelecimento.
Dias em que se suspendem trabalhos officiaes.
Dias santificados.
\section{Feriado}
\begin{itemize}
\item {Grp. gram.:m.  e  adj.}
\end{itemize}
\begin{itemize}
\item {Proveniência:(Lat. \textunderscore feriatus\textunderscore )}
\end{itemize}
Dia ou tempo em que se suspende o trabalho para descanso, por prescripção civil ou religiosa.
\section{Ferial}
\begin{itemize}
\item {Grp. gram.:adj.}
\end{itemize}
\begin{itemize}
\item {Proveniência:(De \textunderscore féria\textunderscore . Cp. \textunderscore feira\textunderscore )}
\end{itemize}
Relativo a féria ou a férias.
Relativo aos dias da semana, aos dias não festivos.
\section{Feriar}
\begin{itemize}
\item {Grp. gram.:v. i.}
\end{itemize}
\begin{itemize}
\item {Grp. gram.:V. t.}
\end{itemize}
\begin{itemize}
\item {Proveniência:(Lat. \textunderscore feriari\textunderscore )}
\end{itemize}
Têr férias; estar em férias.
Não trabalhar.
Dar férias ou descanso a.
\section{Feriável}
\begin{itemize}
\item {Grp. gram.:adj.}
\end{itemize}
\begin{itemize}
\item {Proveniência:(De \textunderscore feriar\textunderscore )}
\end{itemize}
Que póde sêr feriado.
\section{Feriba}
\begin{itemize}
\item {Grp. gram.:f.}
\end{itemize}
\begin{itemize}
\item {Utilização:Bras}
\end{itemize}
Palmeira, o mesmo que \textunderscore geribá\textunderscore .
\section{Feribá}
\begin{itemize}
\item {Grp. gram.:f.}
\end{itemize}
\begin{itemize}
\item {Utilização:Bras}
\end{itemize}
Palmeira, o mesmo que \textunderscore geribá\textunderscore .
\section{Ferida}
\begin{itemize}
\item {Grp. gram.:f.}
\end{itemize}
\begin{itemize}
\item {Utilização:Fig.}
\end{itemize}
\begin{itemize}
\item {Utilização:Prov.}
\end{itemize}
\begin{itemize}
\item {Proveniência:(De \textunderscore ferido\textunderscore )}
\end{itemize}
Chaga; golpe; úlcera.
Offensa; aggravo; injúria.
Mágoa, dor.
O correr da água em declive.
\section{Feridade}
\begin{itemize}
\item {Grp. gram.:f.}
\end{itemize}
\begin{itemize}
\item {Proveniência:(Lat. \textunderscore feritas\textunderscore )}
\end{itemize}
O mesmo que \textunderscore fereza\textunderscore .
\section{Ferido}
\begin{itemize}
\item {Grp. gram.:adj.}
\end{itemize}
\begin{itemize}
\item {Grp. gram.:M.}
\end{itemize}
\begin{itemize}
\item {Grp. gram.:Loc.}
\end{itemize}
\begin{itemize}
\item {Utilização:ven.}
\end{itemize}
\begin{itemize}
\item {Proveniência:(De \textunderscore ferir\textunderscore )}
\end{itemize}
Que recebeu ferimento.
Aquelle que está ferido.
\textunderscore Dar de ferido\textunderscore , açular (cães), mandando-os buscar caça ferida.
\section{Feridor}
\begin{itemize}
\item {Grp. gram.:adj.}
\end{itemize}
\begin{itemize}
\item {Grp. gram.:M.}
\end{itemize}
\begin{itemize}
\item {Utilização:Bras}
\end{itemize}
\begin{itemize}
\item {Proveniência:(De \textunderscore ferir\textunderscore )}
\end{itemize}
Que fere.
Aquelle que fere.
Extremidade do cálix que fica por cima dos cubos de roda, nos engenhos de açúcar.
\section{Ferifoga}
\begin{itemize}
\item {Grp. gram.:f.}
\end{itemize}
\begin{itemize}
\item {Utilização:Prov.}
\end{itemize}
\begin{itemize}
\item {Utilização:beir.}
\end{itemize}
\begin{itemize}
\item {Grp. gram.:M.  e  f.}
\end{itemize}
\begin{itemize}
\item {Proveniência:(De \textunderscore ferir\textunderscore  + \textunderscore fogo\textunderscore , ou corr. de \textunderscore fèrefolha\textunderscore )}
\end{itemize}
Azáfama; lufalufa.
O mesmo que \textunderscore fèrefolha\textunderscore .
\section{Ferifolha}
\begin{itemize}
\item {Grp. gram.:f.}
\end{itemize}
O mesmo que \textunderscore ferifolho\textunderscore .
\section{Ferifolho}
\begin{itemize}
\item {Grp. gram.:m.}
\end{itemize}
(V.fèrefolha)
\section{Ferimento}
\begin{itemize}
\item {Grp. gram.:m.}
\end{itemize}
Acto ou effeito de ferir.
Ferida.
\section{Ferino}
\begin{itemize}
\item {Grp. gram.:adj.}
\end{itemize}
\begin{itemize}
\item {Proveniência:(Lat. \textunderscore ferinus\textunderscore )}
\end{itemize}
Semelhante a uma fera; feroz; sanguinário; cruel.
\section{Ferir}
\begin{itemize}
\item {Grp. gram.:v. t.}
\end{itemize}
\begin{itemize}
\item {Proveniência:(Lat. \textunderscore ferire\textunderscore )}
\end{itemize}
Fazer feridas em.
Golpear.
Bater.
Travar (combate).
Rasgar.
Tanger, tocar: \textunderscore ferir as cordas do bandolim\textunderscore .
Causar sensação a.
Offender: \textunderscore ferir com injúrias\textunderscore .
Punir.
Articular.
Causar dôr ou mágoa a.
\textunderscore Fazer fogo\textunderscore  ou \textunderscore lume\textunderscore , produzir chispas ou centelhas, pelo attrito de metal ou pedra.
Correr a toda a brida.
Exasperar-se.
\section{Feríssimo}
\begin{itemize}
\item {Grp. gram.:adj.}
\end{itemize}
\begin{itemize}
\item {Proveniência:(De \textunderscore fero\textunderscore )}
\end{itemize}
Muito fero; crudelíssimo.
\section{Fermença}
\begin{itemize}
\item {Grp. gram.:f.}
\end{itemize}
\begin{itemize}
\item {Utilização:Ant.}
\end{itemize}
Firmeza; fé.
(Corr. de \textunderscore firmeza\textunderscore )
\section{Fermentação}
\begin{itemize}
\item {Grp. gram.:f.}
\end{itemize}
\begin{itemize}
\item {Utilização:Fig.}
\end{itemize}
\begin{itemize}
\item {Proveniência:(De \textunderscore fermentar\textunderscore )}
\end{itemize}
Reacção espontânea de um corpo orgânico, pela presença de um fermento que o decompõe.
Agitação; effervescência moral.
\section{Fermentáceo}
\begin{itemize}
\item {Grp. gram.:adj.}
\end{itemize}
O mesmo que \textunderscore fermentante\textunderscore .
\section{Fermentante}
\begin{itemize}
\item {Grp. gram.:adj.}
\end{itemize}
\begin{itemize}
\item {Proveniência:(Lat. \textunderscore fermentans\textunderscore )}
\end{itemize}
Que causa fermentação; que está em fermentação.
\section{Fermentar}
\begin{itemize}
\item {Grp. gram.:v. t.}
\end{itemize}
\begin{itemize}
\item {Utilização:Fig.}
\end{itemize}
\begin{itemize}
\item {Grp. gram.:V. i.}
\end{itemize}
\begin{itemize}
\item {Utilização:Fig.}
\end{itemize}
\begin{itemize}
\item {Proveniência:(Lat. \textunderscore fermentare\textunderscore )}
\end{itemize}
Causar fermentação em.
Agitar.
Fomentar.
Excitar: \textunderscore fermentar sublevações\textunderscore .
Estar em fermentação.
Decompôr-se pela fermentação.
Agitar-se, entrar em movimento.
\section{Fermentário}
\begin{itemize}
\item {Grp. gram.:m.}
\end{itemize}
\begin{itemize}
\item {Proveniência:(Lat. \textunderscore fermentarius\textunderscore )}
\end{itemize}
Sacerdote ou christão grego, que, na consagração, usa pão fermentado, em vez de pão ázimo.
\section{Fermentativo}
\begin{itemize}
\item {Grp. gram.:adj.}
\end{itemize}
Que faz fermentar.
\section{Fermentável}
\begin{itemize}
\item {Grp. gram.:adj.}
\end{itemize}
Que se póde fermentar.
\section{Fermentescente}
\begin{itemize}
\item {Grp. gram.:adj.}
\end{itemize}
\begin{itemize}
\item {Proveniência:(Lat. \textunderscore fermentescens\textunderscore )}
\end{itemize}
Preparado para fermentação.
Que começa a fermentar.
\section{Fermentescibilidade}
\begin{itemize}
\item {Grp. gram.:f.}
\end{itemize}
Qualidade daquillo que é fermentescível.
\section{Fermentescível}
\begin{itemize}
\item {Grp. gram.:adj.}
\end{itemize}
O mesmo que \textunderscore fermentescente\textunderscore .
\section{Fermento}
\begin{itemize}
\item {Grp. gram.:m.}
\end{itemize}
\begin{itemize}
\item {Utilização:Fig.}
\end{itemize}
\begin{itemize}
\item {Utilização:Prov.}
\end{itemize}
\begin{itemize}
\item {Utilização:minh.}
\end{itemize}
\begin{itemize}
\item {Proveniência:(Lat. \textunderscore fermentum\textunderscore )}
\end{itemize}
Massa de farinha sem sal, que, azedando-se, e misturando-se com a massa do pão, excita a fermentação nesta.
Qualquer substância, que, em determinado estado, póde decompôr as matérias orgânicas, com que se põe em contacto, obtendo-se vários productos, como álcool, ácido acético, etc.
Aquillo que excita gradualmente o espírito.
Presente, que os noivos offerecem ás pessôas das suas relações, pouco antes do consórcio.
\section{Fermentoso}
\begin{itemize}
\item {Grp. gram.:adj.}
\end{itemize}
\begin{itemize}
\item {Utilização:Fig.}
\end{itemize}
\begin{itemize}
\item {Proveniência:(De \textunderscore fermento\textunderscore )}
\end{itemize}
Que agita, que excita.
Que dá vida, (falando-se da seiva). Cf. Filinto, VII, 202.
\section{Fermoso}
\textunderscore adj.\textunderscore  (e der.)
O mesmo que \textunderscore formoso\textunderscore , etc.
\section{Fernambucano}
\begin{itemize}
\item {Grp. gram.:m.  e  adj.}
\end{itemize}
\begin{itemize}
\item {Proveniência:(De \textunderscore Fernambuco\textunderscore  = \textunderscore Pernambuco\textunderscore , n. p. Cf. Freund, vb. \textunderscore Fernambocum\textunderscore )}
\end{itemize}
(Fórma pop. de \textunderscore pernambucano\textunderscore )
\section{Fernampires}
\begin{itemize}
\item {Grp. gram.:m.}
\end{itemize}
Casta de uva branca do Doiro, Beira, Extremadura e Alentejo.
\section{Fernampires-do-bêco}
\begin{itemize}
\item {Grp. gram.:m.}
\end{itemize}
Casta de uva de Abrantes. Cf. \textunderscore Rev. Agron.\textunderscore , I, 18.
\section{Fernandézia}
\begin{itemize}
\item {Grp. gram.:f.}
\end{itemize}
\begin{itemize}
\item {Proveniência:(De \textunderscore Fernandez\textunderscore , n. p.)}
\end{itemize}
Variedade de orchídea da América tropical.
\section{Fernandina}
\begin{itemize}
\item {Grp. gram.:f.}
\end{itemize}
Tecido de lan ou algodão.
(Cast. \textunderscore fernandina\textunderscore )
\section{Fernão-pires}
\begin{itemize}
\item {Grp. gram.:m.}
\end{itemize}
(V.fernampires)
\section{Fernão-queimado}
\begin{itemize}
\item {Grp. gram.:m.}
\end{itemize}
\begin{itemize}
\item {Utilização:Prov.}
\end{itemize}
\begin{itemize}
\item {Utilização:alent.}
\end{itemize}
Jôgo de rapazes.
\section{Fernélia}
\begin{itemize}
\item {Grp. gram.:f.}
\end{itemize}
\begin{itemize}
\item {Proveniência:(De \textunderscore Fernel\textunderscore , n. p.)}
\end{itemize}
Árvore rubiácea da Ilha-de-França.
\section{Fero}
\begin{itemize}
\item {Grp. gram.:adj.}
\end{itemize}
\begin{itemize}
\item {Grp. gram.:M.}
\end{itemize}
\begin{itemize}
\item {Utilização:Ant.}
\end{itemize}
\begin{itemize}
\item {Grp. gram.:Pl.}
\end{itemize}
\begin{itemize}
\item {Proveniência:(Lat. \textunderscore ferus\textunderscore )}
\end{itemize}
Feroz; selvagem.
Bravio.
Inculto.
Indómito.
Rústico.
Violento.
Aspero.
Encarniçado: \textunderscore combates feros\textunderscore .
Forte; vigoroso; são: \textunderscore você está fero\textunderscore !
O mesmo que \textunderscore ameaça\textunderscore . Cf. Jac. Freire, \textunderscore Castro\textunderscore , 85.
Fanfarronada.
\section{...fero}
\begin{itemize}
\item {Grp. gram.:suf. adj.}
\end{itemize}
\begin{itemize}
\item {Proveniência:(Do lat. \textunderscore ...fer\textunderscore , de \textunderscore ferre\textunderscore )}
\end{itemize}
(designativo de producção, conteúdo, etc.: \textunderscore mortifero\textunderscore , \textunderscore alífero\textunderscore , etc.)
\section{Ferócia}
\begin{itemize}
\item {Grp. gram.:f.}
\end{itemize}
\begin{itemize}
\item {Proveniência:(Lat. \textunderscore ferocia\textunderscore )}
\end{itemize}
O mesmo que \textunderscore ferocidade\textunderscore .
\section{Ferocidade}
\begin{itemize}
\item {Grp. gram.:f.}
\end{itemize}
\begin{itemize}
\item {Proveniência:(Lat. \textunderscore ferocitas\textunderscore )}
\end{itemize}
Qualidade ou carácter de feroz: \textunderscore a ferocidade do tigre\textunderscore .
\section{Feróico}
\begin{itemize}
\item {Grp. gram.:m.}
\end{itemize}
Dialecto das ilhas de Féroe.
\section{Ferólia}
\begin{itemize}
\item {Grp. gram.:f.}
\end{itemize}
Árvore rosácea da Guiana.
\section{Feroz}
\begin{itemize}
\item {Grp. gram.:adj.}
\end{itemize}
\begin{itemize}
\item {Proveniência:(Lat. \textunderscore ferox\textunderscore )}
\end{itemize}
Que tem índole ou natureza de fera: \textunderscore animal feroz\textunderscore .
Perverso.
Que nada teme.
Insolente.
Arrogante, ameaçador.
\section{Ferozmente}
\begin{itemize}
\item {Grp. gram.:adv.}
\end{itemize}
De modo feroz.
\section{Ferpa}
\begin{itemize}
\item {Grp. gram.:f.}
\end{itemize}
\begin{itemize}
\item {Utilização:Bras. do N}
\end{itemize}
Fragmento de madeira, em fórma de palito ou agulha.
(Por \textunderscore farpa\textunderscore ?)
\section{Ferra}
\begin{itemize}
\item {Grp. gram.:f.}
\end{itemize}
\begin{itemize}
\item {Utilização:Prov.}
\end{itemize}
\begin{itemize}
\item {Utilização:trasm.}
\end{itemize}
\begin{itemize}
\item {Utilização:Prov.}
\end{itemize}
\begin{itemize}
\item {Utilização:trasm.}
\end{itemize}
Pá de ferro, para mexer ou tirar brasas.
Acto de ferrar (gado).
Acto de gravar com ferro em brasa na espádua direita da rês as iniciaes ou marca do lavrador.
Balde, para tirar água de poços.
Instrumento de ferro, para tirar a massa da masseira.
\section{Ferrã}
\begin{itemize}
\item {Grp. gram.:f.}
\end{itemize}
\begin{itemize}
\item {Proveniência:(Do b. lat. \textunderscore ferrago\textunderscore )}
\end{itemize}
Cevada, que se ceifa antes de espigada, para alimento de gado.
Quaesquer plantas ou ervas, cortadas á foice, para alimento de gado.
\section{Ferrabrás}
\begin{itemize}
\item {Grp. gram.:m.}
\end{itemize}
\begin{itemize}
\item {Proveniência:(Fr. \textunderscore fier-à-bras\textunderscore )}
\end{itemize}
Valentão; fanfarrão.
\section{Ferraça}
\begin{itemize}
\item {Grp. gram.:f.}
\end{itemize}
\begin{itemize}
\item {Proveniência:(De \textunderscore ferro\textunderscore )}
\end{itemize}
Chapa redonda de ferro, com um buraco ao centro, por onde se deita fogo ao forno que se quer aquecer.
\section{Ferrada}
\begin{itemize}
\item {Grp. gram.:f.}
\end{itemize}
\begin{itemize}
\item {Utilização:Prov.}
\end{itemize}
Vaso, ou caldeiro, para onde se munge o leite das cabras, ovelhas e vacas.
(Cp. \textunderscore ferrado\textunderscore ^2)
\section{Ferradela}
\begin{itemize}
\item {Grp. gram.:f.}
\end{itemize}
\begin{itemize}
\item {Utilização:Pop.}
\end{itemize}
\begin{itemize}
\item {Proveniência:(De \textunderscore ferrar\textunderscore )}
\end{itemize}
O mesmo que \textunderscore dentada\textunderscore .
\section{Ferrado}
\begin{itemize}
\item {Grp. gram.:m.}
\end{itemize}
Acto de ferrar.
\section{Ferrado}
\begin{itemize}
\item {Grp. gram.:m.}
\end{itemize}
\begin{itemize}
\item {Proveniência:(De \textunderscore ferro\textunderscore )}
\end{itemize}
Humor, segregado pelos chocos.
Fezes negras de recém-nascido.
Vaso para ordenhar; balde.
\section{Ferradoria}
\begin{itemize}
\item {Grp. gram.:f.}
\end{itemize}
\begin{itemize}
\item {Utilização:T. de Lisbôa}
\end{itemize}
Officina de ferrador.
\section{Ferrador}
\begin{itemize}
\item {Grp. gram.:m.}
\end{itemize}
\begin{itemize}
\item {Utilização:Bras}
\end{itemize}
\begin{itemize}
\item {Proveniência:(De \textunderscore ferrar\textunderscore )}
\end{itemize}
Aquelle que ferra.
Aquelle que tem por officio fazer ou pregar ferraduras em bêstas.
O mesmo que \textunderscore araponga\textunderscore .
\section{Ferradura}
\begin{itemize}
\item {Grp. gram.:f.}
\end{itemize}
\begin{itemize}
\item {Grp. gram.:Pl.}
\end{itemize}
\begin{itemize}
\item {Utilização:Ant.}
\end{itemize}
\begin{itemize}
\item {Proveniência:(De \textunderscore ferrar\textunderscore )}
\end{itemize}
Peça de ferro, que se prega na face inferior das patas das bêstas e, ás vezes, dos bois.
Pequeno círculo de ferro, com que se reforça inferiormente o salto do tamanco ou de outro calçado grosseiro.
Objecto disposto ou cortado em fórma de ferradura: \textunderscore usar na gravata uma pequena ferradura de oiro\textunderscore .
Qualquer construcção em fórma de meia lua.
Certos bolos, feitos de ovos e açúcar.
Certa porção de ferro ou o seu valor, que os emphyteutas e vassalos pagavam annualmente aos senhorios, para que êstes mandassem fazer as ferraduras que os seus cavallos precisassem.
\section{Ferrageiro}
\begin{itemize}
\item {Grp. gram.:m.}
\end{itemize}
\begin{itemize}
\item {Proveniência:(De \textunderscore ferragem\textunderscore ^1)}
\end{itemize}
Negociante de ferragens.
\section{Ferragem}
\begin{itemize}
\item {Grp. gram.:f.}
\end{itemize}
\begin{itemize}
\item {Proveniência:(De \textunderscore ferro\textunderscore )}
\end{itemize}
Conjunto ou porção de peças de ferro, necessárias para edificações, para artefactos, etc.
Ferraduras.
Obras de ferro; ferro que guarnece um objecto ou faz parte delle.
\section{Ferragem}
\begin{itemize}
\item {Grp. gram.:f.}
\end{itemize}
\begin{itemize}
\item {Utilização:Ant.}
\end{itemize}
O mesmo que \textunderscore forragem\textunderscore .
(B. lat. \textunderscore ferrago\textunderscore )
\section{Ferragial}
\begin{itemize}
\item {Grp. gram.:m.}
\end{itemize}
\begin{itemize}
\item {Proveniência:(De \textunderscore ferragem\textunderscore ^2)}
\end{itemize}
O mesmo que \textunderscore ferrejial\textunderscore .
\section{Ferragista}
\begin{itemize}
\item {Grp. gram.:m.}
\end{itemize}
\begin{itemize}
\item {Utilização:Bras}
\end{itemize}
O mesmo que \textunderscore ferrageiro\textunderscore .
\section{Ferragosa}
\begin{itemize}
\item {Grp. gram.:f.}
\end{itemize}
\begin{itemize}
\item {Utilização:Ant.}
\end{itemize}
\begin{itemize}
\item {Utilização:Gír.}
\end{itemize}
O mesmo que \textunderscore gravata\textunderscore .
\section{Ferragoulo}
\begin{itemize}
\item {Grp. gram.:m.}
\end{itemize}
\begin{itemize}
\item {Utilização:Ant.}
\end{itemize}
(V.farragoulo)
\section{Ferrajaria}
\begin{itemize}
\item {Grp. gram.:f.}
\end{itemize}
\begin{itemize}
\item {Proveniência:(De \textunderscore ferragem\textunderscore ^1)}
\end{itemize}
Fábrica de ferragens.
Indústria das ferragens. Cf. \textunderscore Inquér. Industr.\textunderscore , 2.^a p., l. II, 246.
\section{Ferral}
\begin{itemize}
\item {Grp. gram.:adj.}
\end{itemize}
\begin{itemize}
\item {Proveniência:(De \textunderscore ferro\textunderscore )}
\end{itemize}
Que tem côr de ferro.
Diz-se de algumas variedades de uva arroxada e resistente, própria de latadas.
\section{Ferral-branco}
\begin{itemize}
\item {Grp. gram.:m.}
\end{itemize}
Variedade de uva ferral.
\section{Ferral-de-borba}
\begin{itemize}
\item {Grp. gram.:m.}
\end{itemize}
Variedade de uva ferral.
\section{Ferral-de-olivença}
\begin{itemize}
\item {Grp. gram.:f.}
\end{itemize}
Variedade de uva ferral.
\section{Ferral-tâmara}
\begin{itemize}
\item {Grp. gram.:f.}
\end{itemize}
Variedade de uva ferral.
\section{Ferramenta}
\begin{itemize}
\item {Grp. gram.:f.}
\end{itemize}
\begin{itemize}
\item {Proveniência:(Lat. \textunderscore ferramenta\textunderscore , pl. de \textunderscore ferramentum\textunderscore )}
\end{itemize}
Utensílio de ferro para artes e offícios.
Conjunto de utensílios, para o exercício de um offício ou arte.
\section{Ferramental}
\begin{itemize}
\item {Grp. gram.:m.}
\end{itemize}
\begin{itemize}
\item {Utilização:Ant.}
\end{itemize}
\begin{itemize}
\item {Utilização:Carp.}
\end{itemize}
\begin{itemize}
\item {Utilização:Gír.}
\end{itemize}
\begin{itemize}
\item {Proveniência:(De \textunderscore ferramenta\textunderscore )}
\end{itemize}
O mesmo que \textunderscore ferramenta\textunderscore .
Peça de madeira, em que se dispõem e seguram as ferramentas, ao alcance de quem tenha de trabalhar com ellas.
Ferros para arrombamentos.
\section{Ferramenteiro}
\begin{itemize}
\item {Grp. gram.:m.}
\end{itemize}
\begin{itemize}
\item {Proveniência:(De \textunderscore ferramenta\textunderscore )}
\end{itemize}
Guarda ou inspector das ferramentas, que se empregam em certas obras do Estado ou das câmaras municipaes.
\section{Ferran}
\begin{itemize}
\item {Grp. gram.:f.}
\end{itemize}
\begin{itemize}
\item {Proveniência:(Do b. lat. \textunderscore ferrago\textunderscore )}
\end{itemize}
Cevada, que se ceifa antes de espigada, para alimento de gado.
Quaesquer plantas ou ervas, cortadas á foice, para alimento de gado.
\section{Ferranchão}
\begin{itemize}
\item {Grp. gram.:m.}
\end{itemize}
\begin{itemize}
\item {Utilização:Prov.}
\end{itemize}
\begin{itemize}
\item {Utilização:alent.}
\end{itemize}
\begin{itemize}
\item {Proveniência:(De \textunderscore ferro\textunderscore )}
\end{itemize}
Pau com ponta de ferro, com que se abrem os buracos, em que se fixam os prumos da rede, que cerca a malhada das ovelhas.
\section{Ferranha}
\begin{itemize}
\item {Grp. gram.:f.}
\end{itemize}
\begin{itemize}
\item {Utilização:Prov.}
\end{itemize}
\begin{itemize}
\item {Utilização:minh.}
\end{itemize}
Frutificação das gramíneas.
(Cp. \textunderscore ferran\textunderscore )
\section{Ferrão}
\begin{itemize}
\item {Grp. gram.:m.}
\end{itemize}
\begin{itemize}
\item {Utilização:Bras}
\end{itemize}
\begin{itemize}
\item {Utilização:Prov.}
\end{itemize}
\begin{itemize}
\item {Proveniência:(De \textunderscore ferro\textunderscore )}
\end{itemize}
Aguilhão.
Ponta de ferro.
Órgão retráctil do abdome de alguns insectos e de outros animaes, com o qual ferem ou se defendem.
Nome de uma avezinha preta.
Casta de uva preta.
O bico de ferro dos piões; ferreta.
\section{Ferrar}
\begin{itemize}
\item {Grp. gram.:v. t.}
\end{itemize}
\begin{itemize}
\item {Utilização:Ant.}
\end{itemize}
\begin{itemize}
\item {Utilização:Pop.}
\end{itemize}
\begin{itemize}
\item {Grp. gram.:V. i.}
\end{itemize}
\begin{itemize}
\item {Utilização:Prov.}
\end{itemize}
Pôr ferro em.
Ornar ou guarnecer de ferro: \textunderscore ferrar um cajado\textunderscore .
Pôr ferraduras em: \textunderscore ferrar um cavallo\textunderscore .
Pôr ferros ou algemas a.
Marcar com ferro quente.
Tornar ferruginosa (a água), embebendo nella ferro em brasa.
Pregar, impingir: \textunderscore ferrar mentirolas\textunderscore .
Dar; applicar: \textunderscore ferrar uma tosa\textunderscore .
Arremessar, introduzir violentamente: \textunderscore ferrar com alguém no calaboiço\textunderscore .
Fixar-se.
Morder: \textunderscore aquelle cão ferra\textunderscore .
Começar o trabalho, (falando-se de cavadores): \textunderscore os servos ferraram hoje ás 5 da manhan\textunderscore . (Colhido em Turquel)
Apoderar-se:«\textunderscore saira um crocodilo e ferrara della\textunderscore ». Couto, \textunderscore Déc.\textunderscore  I, l. VIII, c. 12.
\section{Ferraría}
\begin{itemize}
\item {Grp. gram.:f.}
\end{itemize}
Fábrica de ferragens.
Loja de ferreiro.
Arruamento de ferreiros.
Grande porção de ferro.
\section{Ferrária}
\begin{itemize}
\item {Grp. gram.:f.}
\end{itemize}
\begin{itemize}
\item {Proveniência:(De \textunderscore Ferrari\textunderscore , n. p.)}
\end{itemize}
Gênero de plantas irídeas.
\section{Ferrato}
\begin{itemize}
\item {Grp. gram.:m.}
\end{itemize}
\begin{itemize}
\item {Proveniência:(De \textunderscore ferro\textunderscore )}
\end{itemize}
Sal, resultante da combinação do ácido férrico com uma base.
\section{Ferrato}
\begin{itemize}
\item {Grp. gram.:m.}
\end{itemize}
\begin{itemize}
\item {Proveniência:(Lat. \textunderscore ferratus\textunderscore )}
\end{itemize}
Designação antiga do soldado, que tinha armas brancas ou armadura de ferro.
\section{Férrea}
\begin{itemize}
\item {Grp. gram.:f.}
\end{itemize}
\begin{itemize}
\item {Utilização:Prov.}
\end{itemize}
\begin{itemize}
\item {Utilização:beir.}
\end{itemize}
\begin{itemize}
\item {Proveniência:(Lat. \textunderscore ferrea\textunderscore )}
\end{itemize}
Pá, o mesmo que \textunderscore ferra\textunderscore .
\section{Ferreira}
\begin{itemize}
\item {Grp. gram.:f.}
\end{itemize}
Nome de um peixe.
\section{Ferreirinho}
\begin{itemize}
\item {Grp. gram.:m.}
\end{itemize}
Espécie de gaivina, (\textunderscore hydrochelidonnigra\textunderscore , Gray).
\section{Ferreiro}
\begin{itemize}
\item {Grp. gram.:m.}
\end{itemize}
\begin{itemize}
\item {Grp. gram.:Adj.}
\end{itemize}
\begin{itemize}
\item {Utilização:Bras}
\end{itemize}
\begin{itemize}
\item {Proveniência:(De \textunderscore ferro\textunderscore )}
\end{itemize}
Aquelle que trabalha em obras de ferro.
Ferrageiro.
Pequena ave, semelhante ao pardal.
Gaivão.
Megengra.
Peixe esparoide.
Diz-se dos animaes, que têm o pêlo escuro.
\section{Ferrejar}
\begin{itemize}
\item {Grp. gram.:v. i.}
\end{itemize}
\begin{itemize}
\item {Utilização:Fig.}
\end{itemize}
Segar ferrejo.
Têr actividade, negociar.
\section{Ferrejial}
\begin{itemize}
\item {Grp. gram.:m.}
\end{itemize}
Campo de ferrejo.
Campo de cereais; pastagem.
\section{Ferrejo}
\begin{itemize}
\item {Grp. gram.:m.}
\end{itemize}
\begin{itemize}
\item {Utilização:T. do Ribatejo}
\end{itemize}
O mesmo que \textunderscore ferran\textunderscore .
Milho verde, ainda não sachado.
\section{Ferrelha}
\begin{itemize}
\item {fónica:rê}
\end{itemize}
\begin{itemize}
\item {Grp. gram.:f.}
\end{itemize}
\begin{itemize}
\item {Utilização:Prov.}
\end{itemize}
\begin{itemize}
\item {Utilização:minh.}
\end{itemize}
\begin{itemize}
\item {Proveniência:(De \textunderscore ferra\textunderscore )}
\end{itemize}
Pequena pá de ferro, com que se tiram brasas do forno.
\section{Ferrencheiro}
\begin{itemize}
\item {Grp. gram.:m.}
\end{itemize}
\begin{itemize}
\item {Utilização:Prov.}
\end{itemize}
\begin{itemize}
\item {Utilização:trasm.}
\end{itemize}
O mesmo que \textunderscore ferrageiro\textunderscore .
(Cp. gall. \textunderscore ferrancheiro\textunderscore , negociante de ferros velhos)
\section{Ferrenho}
\begin{itemize}
\item {Grp. gram.:adj.}
\end{itemize}
\begin{itemize}
\item {Utilização:Fig.}
\end{itemize}
\begin{itemize}
\item {Proveniência:(De \textunderscore ferro\textunderscore )}
\end{itemize}
Semelhante ao ferro, na côr ou na dureza.
Intransigente: \textunderscore absolutista ferrenho\textunderscore .
Pertinaz; despótico.
Durázio.
\section{Férreo}
\begin{itemize}
\item {Grp. gram.:adj.}
\end{itemize}
\begin{itemize}
\item {Utilização:Fig.}
\end{itemize}
\begin{itemize}
\item {Proveniência:(Lat. \textunderscore ferreus\textunderscore )}
\end{itemize}
Feito de ferro: \textunderscore uma porta férrea\textunderscore .
Que contém ferro, ou saes do ferro: \textunderscore águas férreas\textunderscore .
Inflexível, ferrenho.
Duro; cruel.
\section{Ferreta}
\begin{itemize}
\item {fónica:rê}
\end{itemize}
\begin{itemize}
\item {Grp. gram.:f.}
\end{itemize}
\begin{itemize}
\item {Proveniência:(De \textunderscore ferro\textunderscore )}
\end{itemize}
Ferrão pequeno ou bico de metal, que constitue a parte inferior do pião.
\section{Ferrete}
\begin{itemize}
\item {fónica:rê}
\end{itemize}
\begin{itemize}
\item {Grp. gram.:m.}
\end{itemize}
\begin{itemize}
\item {Utilização:Fig.}
\end{itemize}
\begin{itemize}
\item {Utilização:Prov.}
\end{itemize}
\begin{itemize}
\item {Utilização:beir.}
\end{itemize}
\begin{itemize}
\item {Grp. gram.:Adj.}
\end{itemize}
\begin{itemize}
\item {Proveniência:(De \textunderscore ferro\textunderscore )}
\end{itemize}
Instrumento, com que se marcavam escravos e criminosos e com que se marca o gado.
Sinal de ignomínia.
Condemnação; labéu.
Nódoa ou mascarra no rosto.
Escuro: \textunderscore pano azul ferrete\textunderscore .
\section{Ferreteamento}
\begin{itemize}
\item {Grp. gram.:m.}
\end{itemize}
Acto ou effeito de ferretear.
\section{Ferreteante}
\begin{itemize}
\item {Grp. gram.:adj.}
\end{itemize}
\begin{itemize}
\item {Utilização:Fig.}
\end{itemize}
\begin{itemize}
\item {Proveniência:(De \textunderscore ferretear\textunderscore )}
\end{itemize}
Que ferreteia.
Pungitivo.
\section{Ferretear}
\begin{itemize}
\item {Grp. gram.:v. t.}
\end{itemize}
\begin{itemize}
\item {Utilização:Fig.}
\end{itemize}
Marcar com ferrete.
Affligir; pungir.
\section{Ferretoada}
\begin{itemize}
\item {Grp. gram.:f.}
\end{itemize}
(V.ferroada)
\section{Ferretoar}
\begin{itemize}
\item {Grp. gram.:v. t.}
\end{itemize}
\begin{itemize}
\item {Utilização:Fig.}
\end{itemize}
\begin{itemize}
\item {Proveniência:(Do rad. de \textunderscore ferrão\textunderscore )}
\end{itemize}
Dar ferroadas em.
Aguilhoar.
Censurar.
\section{Férrico}
\begin{itemize}
\item {Grp. gram.:adj.}
\end{itemize}
\begin{itemize}
\item {Utilização:Chím.}
\end{itemize}
Relativo ao ferro ou aos seus compostos.
\section{Ferrídeos}
\begin{itemize}
\item {Grp. gram.:m. pl.}
\end{itemize}
\begin{itemize}
\item {Utilização:Chím.}
\end{itemize}
\begin{itemize}
\item {Proveniência:(De \textunderscore ferro\textunderscore  + gr. \textunderscore eidos\textunderscore )}
\end{itemize}
Família de corpos simples que têm por typo o ferro.
\section{Ferrífero}
\begin{itemize}
\item {Grp. gram.:adj.}
\end{itemize}
\begin{itemize}
\item {Proveniência:(Do lat. \textunderscore ferrum\textunderscore  + \textunderscore ferre\textunderscore )}
\end{itemize}
Composto de ferro; que tem ferro ou saes de ferro.
\section{Ferrificação}
\begin{itemize}
\item {Grp. gram.:f.}
\end{itemize}
\begin{itemize}
\item {Proveniência:(Do lat. \textunderscore ferrum\textunderscore  + \textunderscore facere\textunderscore )}
\end{itemize}
Formação do ferro.
\section{Ferrinhos}
\begin{itemize}
\item {Grp. gram.:m. pl.}
\end{itemize}
\begin{itemize}
\item {Proveniência:(De \textunderscore ferro\textunderscore )}
\end{itemize}
Instrumento de ferro ou aço, em fórma de triângulo.
Instrumento, com que os soldados desaparafusam as peças da espingarda para as limpar.
\section{Ferripirina}
\begin{itemize}
\item {Grp. gram.:f.}
\end{itemize}
\begin{itemize}
\item {Utilização:Pharm.}
\end{itemize}
Combinação dupla do percloreto de ferro e da antipirina.
\section{Ferripyrina}
\begin{itemize}
\item {Grp. gram.:f.}
\end{itemize}
\begin{itemize}
\item {Utilização:Pharm.}
\end{itemize}
Combinação dupla do perchloreto de ferro e da antipyrina.
\section{Ferro}
\begin{itemize}
\item {Grp. gram.:m.}
\end{itemize}
\begin{itemize}
\item {Utilização:Agr.}
\end{itemize}
\begin{itemize}
\item {Utilização:Pop.}
\end{itemize}
\begin{itemize}
\item {Utilização:Gír.}
\end{itemize}
\begin{itemize}
\item {Utilização:Ant.}
\end{itemize}
\begin{itemize}
\item {Grp. gram.:Pl.}
\end{itemize}
\begin{itemize}
\item {Proveniência:(Lat. \textunderscore ferrum\textunderscore )}
\end{itemize}
Metal malleável, muito conhecido e de numerosas applicações em arte.
Instrumento cortante ou perfurante, fabricado com êsse metal: \textunderscore levar tudo a ferro e fogo\textunderscore .
Ferramenta de arte ou offício.
Farpa: \textunderscore meter dois ferros num toiro\textunderscore .
Nome de vários utensílios, em cuja fabricação entra principalmente o ferro: \textunderscore ferro de engomar\textunderscore .
Marca, feita com ferro quente na face externa da coxa do cavallo.
Ferrão.
Dente de ferro que, na ponta da rabiça, anda debaixo do arado.
Arrelia, zanga: \textunderscore faz-me ferro, aquillo\textunderscore .
Dinheiro.
O mesmo que \textunderscore ferradura\textunderscore .
Grilhetas.
Cárcere.
Âncora.
Tenazes.
\section{Ferroada}
\begin{itemize}
\item {Grp. gram.:f.}
\end{itemize}
\begin{itemize}
\item {Utilização:Fig.}
\end{itemize}
Picada com ferrão.
Censura picante.
\section{Ferroadela}
\begin{itemize}
\item {Grp. gram.:f.}
\end{itemize}
\begin{itemize}
\item {Proveniência:(De \textunderscore ferroar\textunderscore )}
\end{itemize}
O mesmo que \textunderscore ferroada\textunderscore .
\section{Ferroar}
\begin{itemize}
\item {Grp. gram.:v. i.}
\end{itemize}
Picar com o ferrão; dar ferroada em.
\section{Ferroba}
\begin{itemize}
\item {fónica:rô}
\end{itemize}
\begin{itemize}
\item {Grp. gram.:f.}
\end{itemize}
\begin{itemize}
\item {Utilização:Des.}
\end{itemize}
O mesmo que \textunderscore alfarroba\textunderscore .
\section{Ferrolhar}
\begin{itemize}
\item {Grp. gram.:v. t.}
\end{itemize}
O mesmo que \textunderscore aferrolhar\textunderscore .
\section{Ferrolho}
\begin{itemize}
\item {fónica:rô}
\end{itemize}
\begin{itemize}
\item {Grp. gram.:m.}
\end{itemize}
\begin{itemize}
\item {Utilização:Ext.}
\end{itemize}
\begin{itemize}
\item {Proveniência:(Do rad. de \textunderscore ferro\textunderscore )}
\end{itemize}
Tranqueta de ferro corrediça, com que se fecham portas ou janelas.
Aldrava.
Contribuição directa municipal, no concelho da Feira.
\section{Ferrolho-queimado}
\begin{itemize}
\item {Grp. gram.:m.}
\end{itemize}
\begin{itemize}
\item {Utilização:Prov.}
\end{itemize}
\begin{itemize}
\item {Utilização:alent.}
\end{itemize}
Jôgo de rapazes.
\section{Ferro-pau}
\begin{itemize}
\item {Grp. gram.:m.}
\end{itemize}
Casta de uva algarvia.
\section{Ferropear}
\begin{itemize}
\item {Grp. gram.:v. t.}
\end{itemize}
Agrilhoar ou prender com ferropeias.
\section{Ferropeias}
\begin{itemize}
\item {Grp. gram.:f. pl.}
\end{itemize}
Algemas, grilhões.
(Cast. \textunderscore ferropea\textunderscore )
\section{Ferroso}
\begin{itemize}
\item {Grp. gram.:adj.}
\end{itemize}
\begin{itemize}
\item {Proveniência:(De \textunderscore ferro\textunderscore )}
\end{itemize}
Diz-se de um óxydo de ferro.
Que contém ferro.
\section{Ferro-velho}
\begin{itemize}
\item {Grp. gram.:m.}
\end{itemize}
\begin{itemize}
\item {Grp. gram.:Pl.}
\end{itemize}
Homem, que compra e vende objectos usados, como botas, chapéus, etc.
Objectos domésticos, usados e de pouco valor.
\section{Ferrovia}
\begin{itemize}
\item {Grp. gram.:f.}
\end{itemize}
\begin{itemize}
\item {Utilização:Neol.}
\end{itemize}
Estrada de ferro, caminho de ferro.
(T., mal formado, de \textunderscore ferro\textunderscore  + \textunderscore via\textunderscore . Melhor seria \textunderscore ferrivia\textunderscore )
\section{Ferrovial}
\begin{itemize}
\item {Grp. gram.:adj.}
\end{itemize}
\begin{itemize}
\item {Proveniência:(De \textunderscore ferrovia\textunderscore )}
\end{itemize}
Relativo a caminho de ferro.
\section{Ferroviário}
\begin{itemize}
\item {Grp. gram.:adj.}
\end{itemize}
\begin{itemize}
\item {Utilização:Neol.}
\end{itemize}
\begin{itemize}
\item {Proveniência:(De \textunderscore ferrovia\textunderscore )}
\end{itemize}
Relativo a caminho de ferro.
\section{Ferrucho}
\begin{itemize}
\item {Grp. gram.:m.}
\end{itemize}
\begin{itemize}
\item {Utilização:Prov.}
\end{itemize}
\begin{itemize}
\item {Utilização:minh.}
\end{itemize}
Pequeno púcaro, em que se aquece agua na lareira.
\section{Ferrugem}
\begin{itemize}
\item {Grp. gram.:f.}
\end{itemize}
\begin{itemize}
\item {Utilização:Ext.}
\end{itemize}
\begin{itemize}
\item {Utilização:Pop.}
\end{itemize}
\begin{itemize}
\item {Utilização:Fig.}
\end{itemize}
\begin{itemize}
\item {Proveniência:(Lat. \textunderscore ferrugo\textunderscore )}
\end{itemize}
Óxydo, que se forma na superfície de ferro exposto á humidade.
Óxydo, formado sobre outros metaes.
Fuligem.
Alforra.
Falta de instrucção.
Sombras da ignorância.
Ócio, descanso.
\section{Ferrugento}
\begin{itemize}
\item {Grp. gram.:adj.}
\end{itemize}
\begin{itemize}
\item {Utilização:Fig.}
\end{itemize}
\begin{itemize}
\item {Proveniência:(De \textunderscore ferrugem\textunderscore )}
\end{itemize}
Que tem ferrugem.
Antigo; desusado.
\section{Ferrugíneo}
\begin{itemize}
\item {Grp. gram.:adj.}
\end{itemize}
\begin{itemize}
\item {Proveniência:(Lat. \textunderscore ferrugineus\textunderscore )}
\end{itemize}
Escuro, da côr da ferrugem.
\section{Ferruginosidade}
\begin{itemize}
\item {Grp. gram.:f.}
\end{itemize}
Qualidade daquillo que é ferruginoso.
\section{Ferruginoso}
\begin{itemize}
\item {Grp. gram.:adj.}
\end{itemize}
\begin{itemize}
\item {Proveniência:(Do lat. \textunderscore ferrugo\textunderscore )}
\end{itemize}
Que é da natureza do ferro ou do seu óxydo.
Que contém ferro: \textunderscore nascentes ferruginosas\textunderscore .
Que é da côr do ferro.
\section{Ferrujão}
\begin{itemize}
\item {Grp. gram.:m.}
\end{itemize}
Doença dos bois, caracterizada pela incontinência de urinas. Cf. \textunderscore Bibl. da G. do Campo\textunderscore , 567.
\section{Ferruncho}
\begin{itemize}
\item {Grp. gram.:m.}
\end{itemize}
\begin{itemize}
\item {Utilização:Pop.}
\end{itemize}
\begin{itemize}
\item {Proveniência:(De \textunderscore ferro\textunderscore )}
\end{itemize}
Despeito, ciume.
\section{Ferruncho}
\begin{itemize}
\item {Grp. gram.:m.}
\end{itemize}
\begin{itemize}
\item {Utilização:Prov.}
\end{itemize}
\begin{itemize}
\item {Utilização:trasm.}
\end{itemize}
França de giesta.
Vergôntea de colmo ou de outra planta flexível, com que se aperta a vassoira ou o escovalho.
(Por \textunderscore farruncho\textunderscore , de \textunderscore farro\textunderscore ?)
\section{Fértil}
\begin{itemize}
\item {Grp. gram.:adj.}
\end{itemize}
\begin{itemize}
\item {Grp. gram.:Pl.}
\end{itemize}
\begin{itemize}
\item {Utilização:Des.}
\end{itemize}
\begin{itemize}
\item {Proveniência:(Lat. \textunderscore fertilis\textunderscore )}
\end{itemize}
Que produz, que é fecundo.
Que produz muito.
Férteis.
Fértiles:«\textunderscore ...que são terras chans e muito fértiles...\textunderscore »Filinto, \textunderscore D. Man.\textunderscore , III, 156.
\section{Fertilidade}
\begin{itemize}
\item {Grp. gram.:f.}
\end{itemize}
\begin{itemize}
\item {Utilização:Fig.}
\end{itemize}
\begin{itemize}
\item {Proveniência:(Lat. \textunderscore fertilitas\textunderscore )}
\end{itemize}
Qualidade do que é fértil.
Fecundidade.
Disposição para a fecundação.
Opulência, abundância.
\section{Fertilização}
\begin{itemize}
\item {Grp. gram.:f.}
\end{itemize}
Acto ou effeito de fertilizar.
\section{Fertilizador}
\begin{itemize}
\item {Grp. gram.:adj.}
\end{itemize}
\begin{itemize}
\item {Grp. gram.:M.}
\end{itemize}
Que fertiliza.
Aquelle que fertiliza.
\section{Fertilizante}
\begin{itemize}
\item {Grp. gram.:adj.}
\end{itemize}
Que é próprio para fertilizar.
\section{Fertilizar}
\begin{itemize}
\item {Grp. gram.:v. t.}
\end{itemize}
\begin{itemize}
\item {Utilização:Fig.}
\end{itemize}
\begin{itemize}
\item {Proveniência:(De \textunderscore fértil\textunderscore )}
\end{itemize}
Tornar fértil; fecundar.
Desenvolver (aptidões, trabalho, etc.).
\section{Fertilizável}
\begin{itemize}
\item {Grp. gram.:adj.}
\end{itemize}
Que se póde fertilizar.
\section{Fertilmente}
\begin{itemize}
\item {Grp. gram.:adv.}
\end{itemize}
\begin{itemize}
\item {Proveniência:(De \textunderscore fértil\textunderscore )}
\end{itemize}
Com fertilidade, com abundância.
\section{Ferto}
\begin{itemize}
\item {Grp. gram.:m.}
\end{itemize}
\begin{itemize}
\item {Proveniência:(Lat. \textunderscore fertum\textunderscore )}
\end{itemize}
Bolo de farinha, para os sacrifícios, entre os antigos Romanos.
\section{Fertor}
\begin{itemize}
\item {Grp. gram.:m.}
\end{itemize}
\begin{itemize}
\item {Proveniência:(Lat. \textunderscore fertor\textunderscore )}
\end{itemize}
Aquelle que, entre os antigos Romanos, levava nos sacrifícios o bolo de farinha, o mel e o vinho.
\section{Ferucua}
\begin{itemize}
\item {Grp. gram.:m.}
\end{itemize}
Espécie do chanceler, nos antigos tribunaes de Nanquim. Cf. \textunderscore Peregrinação\textunderscore , LXXXV.
\section{Férula}
\begin{itemize}
\item {Grp. gram.:f.}
\end{itemize}
\begin{itemize}
\item {Utilização:Fig.}
\end{itemize}
\begin{itemize}
\item {Proveniência:(Lat. \textunderscore ferula\textunderscore )}
\end{itemize}
Palmatória, instrumento para castigo de crianças e, ás vezes, de adultos.
Planta umbellífera, canafrecha.
Antigo bastão episcopal.
Autoridade; severidade.
Gênero de molluscos, cuja concha termina em fórma de corôa.
\section{Feruláceo}
\begin{itemize}
\item {Grp. gram.:adj.}
\end{itemize}
\begin{itemize}
\item {Proveniência:(Lat. \textunderscore ferulaceus\textunderscore )}
\end{itemize}
Relativo ou semelhante á canafrecha.
\section{Fervedoiro}
\begin{itemize}
\item {Grp. gram.:m.}
\end{itemize}
\begin{itemize}
\item {Proveniência:(De \textunderscore ferver\textunderscore )}
\end{itemize}
Movimento, como o de um líquido que ferve.
Agitação.
Grande ajuntamento.
\section{Fervedouro}
\begin{itemize}
\item {Grp. gram.:m.}
\end{itemize}
\begin{itemize}
\item {Proveniência:(De \textunderscore ferver\textunderscore )}
\end{itemize}
Movimento, como o de um líquido que ferve.
Agitação.
Grande ajuntamento.
\section{Fervedura}
\begin{itemize}
\item {Grp. gram.:f.}
\end{itemize}
(V.fervura)
\section{Fervelhar}
\begin{itemize}
\item {Grp. gram.:v. i.}
\end{itemize}
\begin{itemize}
\item {Proveniência:(De \textunderscore ferver\textunderscore )}
\end{itemize}
O mesmo que \textunderscore fervilhar\textunderscore .
Sêr inquieto ou traquina.
\section{Fervelho}
\begin{itemize}
\item {fónica:vê}
\end{itemize}
\begin{itemize}
\item {Grp. gram.:m.}
\end{itemize}
\begin{itemize}
\item {Utilização:Fam.}
\end{itemize}
\begin{itemize}
\item {Proveniência:(De \textunderscore fervelhar\textunderscore )}
\end{itemize}
Indivíduo, que fervelha, que não tem paro.
Criança traquina.
\section{Fervença}
\begin{itemize}
\item {Grp. gram.:f.}
\end{itemize}
O mesmo que \textunderscore fervura\textunderscore , ardor, vivacidade:«\textunderscore fervenças do corpo, que se não governa pela razão...\textunderscore »M. Bernárdez.
\section{Fervência}
\begin{itemize}
\item {Grp. gram.:f.}
\end{itemize}
\begin{itemize}
\item {Utilização:Ant.}
\end{itemize}
O mesmo que \textunderscore fervura\textunderscore . Cf. \textunderscore Roteiro\textunderscore  de D. João de Castro.
\section{Ferventar}
\textunderscore v. t.\textunderscore  (e der.)
O mesmo que \textunderscore aferventar\textunderscore , etc.
\section{Fervente}
\begin{itemize}
\item {Grp. gram.:adj.}
\end{itemize}
\begin{itemize}
\item {Proveniência:(Lat. \textunderscore fervens\textunderscore )}
\end{itemize}
Que ferve.
Tempestuoso.
Ardente; vehemente.
Fervoroso.
\section{Ferver}
\begin{itemize}
\item {Grp. gram.:v. t.}
\end{itemize}
\begin{itemize}
\item {Grp. gram.:V. i.}
\end{itemize}
\begin{itemize}
\item {Proveniência:(Lat. \textunderscore fervere\textunderscore )}
\end{itemize}
Cozer em água fervente ou noutro líquido em ebullição.
Produzir ebullição em.
Estar em ebullição.
Queimar, como a água que ferve.
Agitar-se.
Saír em cachão.
Apparecer em grande número.
Arder, excitar-se: \textunderscore ferver com raiva\textunderscore .
Sentir paixão.
\textunderscore Ferver o sangue a alguém\textunderscore , impacientar-se.
\section{Fervescente}
\begin{itemize}
\item {Grp. gram.:adj.}
\end{itemize}
\begin{itemize}
\item {Proveniência:(Lat. \textunderscore fervescens\textunderscore )}
\end{itemize}
O mesmo que \textunderscore fervente\textunderscore .
\section{Fervida}
\begin{itemize}
\item {Grp. gram.:f.}
\end{itemize}
\begin{itemize}
\item {Utilização:Des.}
\end{itemize}
\begin{itemize}
\item {Utilização:Prov.}
\end{itemize}
\begin{itemize}
\item {Utilização:alent.}
\end{itemize}
O mesmo que \textunderscore fervura\textunderscore .
\textunderscore Fervida de terras\textunderscore , porção de terra, arrastada por enchente ou alluvião.
\section{Férvido}
\begin{itemize}
\item {Grp. gram.:adj.}
\end{itemize}
\begin{itemize}
\item {Proveniência:(Lat. \textunderscore fervidus\textunderscore )}
\end{itemize}
Quente; muito quente; abrasador.
Apaixonado.
Fervoroso.
Rápido, arrebatado.
Impaciente.
\section{Fervilha}
\begin{itemize}
\item {Grp. gram.:m.}
\end{itemize}
\begin{itemize}
\item {Utilização:Pop.}
\end{itemize}
\begin{itemize}
\item {Proveniência:(De \textunderscore fervilhar\textunderscore )}
\end{itemize}
Pessôa activa, inquieta, que trata de muitos negócios.
\section{Fervilhar}
\begin{itemize}
\item {Grp. gram.:v. i.}
\end{itemize}
\begin{itemize}
\item {Utilização:Fig.}
\end{itemize}
\begin{itemize}
\item {Proveniência:(De \textunderscore ferver\textunderscore )}
\end{itemize}
Ferver frequentemente.
Agitar-se com frequência.
Apparecer ou concorrer em grande número.
Mexer-se para um e outro lado, andar numa roda viva, tratar de muitas coisas.
\section{Fervor}
\begin{itemize}
\item {Grp. gram.:m.}
\end{itemize}
\begin{itemize}
\item {Utilização:Fig.}
\end{itemize}
\begin{itemize}
\item {Proveniência:(Lat. \textunderscore fervor\textunderscore )}
\end{itemize}
Acto de ferver, estado daquillo que ferve.
Ardência.
Zêlo, grande dedicação: \textunderscore amar com fervor\textunderscore .
Grande desejo.
Piedade.
Ardor; ímpeto: \textunderscore pelejar com fervor\textunderscore .
Diligência.
Estertor.
\section{Fervorar}
\begin{itemize}
\item {Grp. gram.:v. t.}
\end{itemize}
(V.afervorar)
\section{Fervorosamente}
\begin{itemize}
\item {Grp. gram.:adv.}
\end{itemize}
De modo fervoroso.
\section{Fervoroso}
\begin{itemize}
\item {Grp. gram.:adj.}
\end{itemize}
Fervente.
Que tem fervor, que é diligente, dedicado, zeloso; vehemente.
\section{Fervura}
\begin{itemize}
\item {Grp. gram.:f.}
\end{itemize}
\begin{itemize}
\item {Utilização:Fig.}
\end{itemize}
\begin{itemize}
\item {Proveniência:(De \textunderscore ferver\textunderscore )}
\end{itemize}
Estado de um líquido que ferve.
Ebullição.
Alvorôço.
Excitação.
\section{Fesceninas}
\begin{itemize}
\item {Grp. gram.:f. pl.}
\end{itemize}
\begin{itemize}
\item {Proveniência:(De \textunderscore fescenino\textunderscore )}
\end{itemize}
Poesias ou composições dramáticas, grosseiras e licenciosas, usadas em Fescênia, na Etrúria, e introduzidas na antiga Roma.
\section{Fescenino}
\begin{itemize}
\item {Grp. gram.:adj.}
\end{itemize}
\begin{itemize}
\item {Proveniência:(Lat. \textunderscore fescenninus\textunderscore )}
\end{itemize}
Burlesco; obsceno.
Diz-se dos escritos licenciosos e obscenos e especialmente de umas composições literárias da antiga Roma.
\section{Fescenninas}
\begin{itemize}
\item {Grp. gram.:f. pl.}
\end{itemize}
\begin{itemize}
\item {Proveniência:(De \textunderscore fescennino\textunderscore )}
\end{itemize}
Poesias ou composições dramáticas, grosseiras e licenciosas, usadas em Fescênnia, na Etrúria, e introduzidas na antiga Roma.
\section{Fescennino}
\begin{itemize}
\item {Grp. gram.:adj.}
\end{itemize}
\begin{itemize}
\item {Proveniência:(Lat. \textunderscore fescenninus\textunderscore )}
\end{itemize}
Burlesco; obsceno.
Diz-se dos escritos licenciosos e obscenos e especialmente de umas composições literárias da antiga Roma.
\section{Fescoço}
\begin{itemize}
\item {fónica:cô}
\end{itemize}
\begin{itemize}
\item {Grp. gram.:m.}
\end{itemize}
\begin{itemize}
\item {Utilização:Prov.}
\end{itemize}
\begin{itemize}
\item {Utilização:alent.}
\end{itemize}
O mesmo que \textunderscore pescoço\textunderscore .
\section{Feso}
\begin{itemize}
\item {fónica:fê}
\end{itemize}
\begin{itemize}
\item {Grp. gram.:adj.}
\end{itemize}
(us. em Melgaço, por \textunderscore feito\textunderscore , de \textunderscore fazer\textunderscore )
\section{Festa}
\begin{itemize}
\item {Grp. gram.:f.}
\end{itemize}
\begin{itemize}
\item {Utilização:Fam.}
\end{itemize}
\begin{itemize}
\item {Grp. gram.:Pl.}
\end{itemize}
\begin{itemize}
\item {Proveniência:(Do lat. \textunderscore festum\textunderscore )}
\end{itemize}
Propriamente, o dia feriado, dia de descanso, dia santificado, dia de regosijo.
Regosijo.
Solennidade, ceremónias, com que se celebra um facto.
Commemoração.
Trabalheira, cuidados.
Facto extraordinário.
Carícias.
\textunderscore Bôas festas\textunderscore , cumprimentos ou felicitações, por occasião do Natal ou da Páschoa.
\section{Festada}
\begin{itemize}
\item {Grp. gram.:f.}
\end{itemize}
\begin{itemize}
\item {Utilização:Prov.}
\end{itemize}
\begin{itemize}
\item {Utilização:dur.}
\end{itemize}
O mesmo que \textunderscore tocata\textunderscore .
\section{Festança}
\begin{itemize}
\item {Grp. gram.:f.}
\end{itemize}
\begin{itemize}
\item {Proveniência:(De \textunderscore festa\textunderscore )}
\end{itemize}
Festa ruidosa; grande divertimento.
\section{Festanga}
\begin{itemize}
\item {Grp. gram.:f.}
\end{itemize}
\begin{itemize}
\item {Utilização:Pop.}
\end{itemize}
Festa reles.
\section{Festão}
\begin{itemize}
\item {Grp. gram.:m.}
\end{itemize}
\begin{itemize}
\item {Proveniência:(Do b. lat. \textunderscore festum\textunderscore ?)}
\end{itemize}
Grinalda; ramalhete.
Ornato, em fórma de grinalda.
\section{Festarola}
\begin{itemize}
\item {Grp. gram.:f.}
\end{itemize}
\begin{itemize}
\item {Utilização:Fam.}
\end{itemize}
\begin{itemize}
\item {Proveniência:(De \textunderscore festa\textunderscore )}
\end{itemize}
Festança; folguedo; salsifré.
\section{Festeiro}
\begin{itemize}
\item {Grp. gram.:m.}
\end{itemize}
\begin{itemize}
\item {Grp. gram.:Adj.}
\end{itemize}
\begin{itemize}
\item {Proveniência:(De \textunderscore festa\textunderscore )}
\end{itemize}
Aquelle, que faz ou dirige uma festa.
Frequentador de festas.
Que frequenta festas.
Que faz carícias.
\section{Festejador}
\begin{itemize}
\item {Grp. gram.:adj.}
\end{itemize}
\begin{itemize}
\item {Grp. gram.:M.}
\end{itemize}
Que festeja.
Aquelle que festeja.
\section{Festejar}
\begin{itemize}
\item {Grp. gram.:v. t.}
\end{itemize}
Fazer festa em honra de; celebrar: \textunderscore festejar uma victória\textunderscore .
Applaudir: \textunderscore festejar um actor\textunderscore .
Fazer festas a.
Saudar.
Acariciar.
\section{Festejável}
\begin{itemize}
\item {Grp. gram.:adj.}
\end{itemize}
Digno de sêr festejado; que se deve festejar. Cf. Castilho, \textunderscore Fausto\textunderscore , X.
\section{Festejo}
\begin{itemize}
\item {Grp. gram.:m.}
\end{itemize}
Acto ou effeito de festejar.
Festividade.
Carícias; galanteio.
\section{Festim}
\begin{itemize}
\item {Grp. gram.:m.}
\end{itemize}
\begin{itemize}
\item {Proveniência:(De \textunderscore festa\textunderscore )}
\end{itemize}
Pequena festa.
Festa em família.
Festa particular.
Banquete.
\section{Festinho}
\begin{itemize}
\item {Grp. gram.:adj.}
\end{itemize}
\begin{itemize}
\item {Utilização:Ant.}
\end{itemize}
\begin{itemize}
\item {Proveniência:(Do lat. \textunderscore festinus\textunderscore )}
\end{itemize}
Apressadamente.
\section{Festival}
\begin{itemize}
\item {Grp. gram.:adj.}
\end{itemize}
\begin{itemize}
\item {Grp. gram.:M.}
\end{itemize}
\begin{itemize}
\item {Utilização:Angl}
\end{itemize}
\begin{itemize}
\item {Proveniência:(De \textunderscore festivo\textunderscore )}
\end{itemize}
O mesmo que \textunderscore festivo\textunderscore .
Grande festa.
Cortejo cívico.
\section{Festivalmente}
\begin{itemize}
\item {Grp. gram.:adv.}
\end{itemize}
De modo festivo.
\section{Festivamente}
\begin{itemize}
\item {Grp. gram.:adv.}
\end{itemize}
De modo festivo.
\section{Festividade}
\begin{itemize}
\item {Grp. gram.:f.}
\end{itemize}
\begin{itemize}
\item {Proveniência:(Lat. \textunderscore festivitas\textunderscore )}
\end{itemize}
Festa religiosa, festa de igreja.
Regozijo.
\section{Festivo}
\begin{itemize}
\item {Grp. gram.:adj.}
\end{itemize}
\begin{itemize}
\item {Proveniência:(Lat. \textunderscore festivus\textunderscore )}
\end{itemize}
Relativo a festa.
Alegre; divertido.
\section{Fêsto}
\begin{itemize}
\item {Grp. gram.:m.}
\end{itemize}
\begin{itemize}
\item {Utilização:Prov.}
\end{itemize}
\begin{itemize}
\item {Utilização:Prov.}
\end{itemize}
\begin{itemize}
\item {Utilização:minh.}
\end{itemize}
Largura de um tecido qualquer.
Dobra de uma peça de pano, em toda a sua extensão.
Refêgo de vestuário.
\textunderscore A fêsto\textunderscore , de costa acima, de subida diffícil.
(Relaciona-se com \textunderscore enfesta\textunderscore ?)
\section{Fésto}
\begin{itemize}
\item {Grp. gram.:adj.}
\end{itemize}
\begin{itemize}
\item {Proveniência:(Lat. \textunderscore festus\textunderscore )}
\end{itemize}
O mesmo que \textunderscore festivo\textunderscore .
\section{Festoar}
\begin{itemize}
\item {Grp. gram.:v. t.}
\end{itemize}
Ornar com festões; engrinaldar.
\section{Festonadas}
\begin{itemize}
\item {Grp. gram.:f. pl.}
\end{itemize}
Grandes festões, em pintura ou esculptura.
\section{Festonar}
\begin{itemize}
\item {Grp. gram.:v. t.}
\end{itemize}
Adornar com festões.
\section{Festoso}
\begin{itemize}
\item {Grp. gram.:adj.}
\end{itemize}
Relativo a festa; festivo.
\section{Festuca}
\begin{itemize}
\item {Grp. gram.:f.}
\end{itemize}
\begin{itemize}
\item {Proveniência:(Lat. \textunderscore festuca\textunderscore )}
\end{itemize}
Gênero de plantas gramíneas, vivazes, no hemisphério austral.
Vara, com que os pretores tocavam no escravo, para lhe dar a liberdade.
\section{Festucário}
\begin{itemize}
\item {Grp. gram.:m.}
\end{itemize}
\begin{itemize}
\item {Proveniência:(Do lat. \textunderscore festuca\textunderscore )}
\end{itemize}
Gênero de vermes intestinaes.
\section{Fetação}
\begin{itemize}
\item {Grp. gram.:f.}
\end{itemize}
Formação do féto.
\section{Fetáceas}
\begin{itemize}
\item {Grp. gram.:f. pl.}
\end{itemize}
\begin{itemize}
\item {Proveniência:(De \textunderscore fetáceo\textunderscore )}
\end{itemize}
Grupo de plantas, que tem por typo o fêto.
\section{Fetáceo}
\begin{itemize}
\item {Grp. gram.:adj.}
\end{itemize}
Relativo ou semelhante ao fêto.
\section{Fetal}
\begin{itemize}
\item {Grp. gram.:adj.}
\end{itemize}
Relativo ao féto.
\section{Fetal}
\begin{itemize}
\item {Grp. gram.:m.}
\end{itemize}
Terreno em que crescem fêtos.
\section{Fêtam}
\begin{itemize}
\item {Grp. gram.:m.}
\end{itemize}
\begin{itemize}
\item {Utilização:Pop.}
\end{itemize}
O mesmo que \textunderscore fêto\textunderscore .--Os diccionaristas, que nunca ouviram a palavra ao povo, fazem-na aguda e escrevem \textunderscore fetão\textunderscore . É êrro: a sýllaba tónica é a 1.^a.
Também se diz \textunderscore fentam\textunderscore .
\section{Fêtão}
\begin{itemize}
\item {Grp. gram.:m.}
\end{itemize}
\begin{itemize}
\item {Utilização:Pop.}
\end{itemize}
O mesmo que \textunderscore fêto\textunderscore .--Os diccionaristas, que nunca ouviram a palavra ao povo, fazem-na aguda e escrevem \textunderscore fetão\textunderscore . É erro: a sýllaba tónica é a 1.^a
Também se diz \textunderscore fentam\textunderscore .
\section{Feteira}
\begin{itemize}
\item {Grp. gram.:f.}
\end{itemize}
Lugar, onde crescem fêtos.
Conjunto de espécies de fêtos.
Fetal^2. Cf. Castilho, \textunderscore Geórgicas\textunderscore , 91.
\section{Fetena}
\begin{itemize}
\item {Grp. gram.:f.}
\end{itemize}
\begin{itemize}
\item {Utilização:Ant.}
\end{itemize}
\begin{itemize}
\item {Proveniência:(Do ár. \textunderscore fitna\textunderscore )}
\end{itemize}
Revolta; rebellião.
O mesmo que \textunderscore alferena\textunderscore .
\section{Fetiche}
\begin{itemize}
\item {Grp. gram.:f.}
\end{itemize}
\begin{itemize}
\item {Utilização:Gal}
\end{itemize}
\begin{itemize}
\item {Utilização:Fig.}
\end{itemize}
\begin{itemize}
\item {Proveniência:(Fr. \textunderscore fetiche\textunderscore , talvez do port. \textunderscore feitiço\textunderscore )}
\end{itemize}
Tudo que é objecto de adoração, em alguns povos selvagens da África.
Pessôa, que se venera e a quem se obedece cegamente.
\section{Fetichismo}
\begin{itemize}
\item {Grp. gram.:m.}
\end{itemize}
\begin{itemize}
\item {Utilização:Fig.}
\end{itemize}
Adoração de fetiches.
Partidarismo faccioso.
Subserviência absoluta.
\section{Fetichista}
\begin{itemize}
\item {Grp. gram.:m.}
\end{itemize}
\begin{itemize}
\item {Utilização:Fig.}
\end{itemize}
\begin{itemize}
\item {Proveniência:(De \textunderscore fetiche\textunderscore )}
\end{itemize}
Aquelle que adora os fetiches.
Sectário faccioso.
Amouco.
\section{Feticida}
\begin{itemize}
\item {Grp. gram.:m.  e  f.}
\end{itemize}
Pessôa que determina a morte do féto.
(Cp. \textunderscore feticídio\textunderscore )
\section{Feticídio}
\begin{itemize}
\item {Grp. gram.:m.}
\end{itemize}
\begin{itemize}
\item {Proveniência:(Do lat. \textunderscore foetum\textunderscore  + \textunderscore caedere\textunderscore )}
\end{itemize}
Morte de um féto, causada voluntariamente; abôrto.
\section{Fetidez}
\begin{itemize}
\item {Grp. gram.:f.}
\end{itemize}
Qualidade daquillo que é fétido; fedor.
\section{Fétido}
\begin{itemize}
\item {Grp. gram.:adj.}
\end{itemize}
\begin{itemize}
\item {Grp. gram.:M.}
\end{itemize}
\begin{itemize}
\item {Utilização:Pop.}
\end{itemize}
\begin{itemize}
\item {Proveniência:(Lat. \textunderscore faetidus\textunderscore )}
\end{itemize}
Que lança mau cheiro.
Podre.
Fedor.
\section{Féto}
\begin{itemize}
\item {Grp. gram.:m.}
\end{itemize}
\begin{itemize}
\item {Proveniência:(Lat. \textunderscore fetus\textunderscore )}
\end{itemize}
Criatura animada, em-quanto não sái do ventre materno.
Embryão; germe.
\section{Fêto}
\begin{itemize}
\item {Grp. gram.:m.}
\end{itemize}
\begin{itemize}
\item {Proveniência:(Do lat. \textunderscore filictum\textunderscore )}
\end{itemize}
Família de plantas cryptogâmicas, ramosas e de haste avermelhada.
\section{Fêto-macho}
\begin{itemize}
\item {Grp. gram.:m.}
\end{itemize}
Planta officinal (\textunderscore aspidium filixmas\textunderscore ).
\section{Feu}
\begin{itemize}
\item {Grp. gram.:m.}
\end{itemize}
\begin{itemize}
\item {Utilização:Ant.}
\end{itemize}
O mesmo que \textunderscore feudo\textunderscore . Cf. \textunderscore Port. Mon. Hist.\textunderscore , \textunderscore Script.\textunderscore , 277.
\section{Feudal}
\begin{itemize}
\item {Grp. gram.:adj.}
\end{itemize}
Relativo a feudo ou a feudalismo.
(B. lat. \textunderscore feudalis\textunderscore )
\section{Feudalidade}
\begin{itemize}
\item {Grp. gram.:f.}
\end{itemize}
(V.feudalismo)
\section{Feudalismo}
\begin{itemize}
\item {Grp. gram.:m.}
\end{itemize}
\begin{itemize}
\item {Proveniência:(De \textunderscore feudal\textunderscore )}
\end{itemize}
O regime ou instituição feudal.
\section{Feudalista}
\begin{itemize}
\item {Grp. gram.:m.}
\end{itemize}
\begin{itemize}
\item {Grp. gram.:Adj.}
\end{itemize}
\begin{itemize}
\item {Proveniência:(De \textunderscore feudal\textunderscore )}
\end{itemize}
Sectário do feudalismo.
Relativo ao feudalismo: \textunderscore o regime feudalista\textunderscore .
\section{Feudatário}
\begin{itemize}
\item {Grp. gram.:adj.}
\end{itemize}
\begin{itemize}
\item {Grp. gram.:M.}
\end{itemize}
Feudal.
Que paga feudo.
Vassallo.
(B. lat. \textunderscore feudatarius\textunderscore )
\section{Feudista}
\begin{itemize}
\item {Grp. gram.:m.}
\end{itemize}
Aquelle que é versado em matéria de feudos.
\section{Feudo}
\begin{itemize}
\item {Grp. gram.:m.}
\end{itemize}
\begin{itemize}
\item {Utilização:Ext.}
\end{itemize}
Propriedade nobre ou bens rústicos, concedidos pelo senhor de certos domínios, sob a condição de vassallagem e prestação de certos serviços ou rendas.
Vassallagem feudal.
Os moradores de terras feudaes.
(B. lat. \textunderscore feudum\textunderscore )
\section{Fêvera}
\begin{itemize}
\item {Grp. gram.:f.}
\end{itemize}
\begin{itemize}
\item {Proveniência:(Do lat. \textunderscore fibra\textunderscore )}
\end{itemize}
Filamento vegetal.
Veio mineral.
\section{Fevereiro}
\begin{itemize}
\item {Grp. gram.:m.}
\end{itemize}
\begin{itemize}
\item {Proveniência:(Do lat. \textunderscore februarius\textunderscore )}
\end{itemize}
Segundo mês do anno romano.
\section{Fevra}
\begin{itemize}
\item {fónica:fé}
\end{itemize}
\begin{itemize}
\item {Grp. gram.:f.}
\end{itemize}
O mesmo que \textunderscore fêvera\textunderscore .
\section{Fevroso}
\begin{itemize}
\item {Grp. gram.:adj.}
\end{itemize}
Que tem fevras.
\section{Féz}
\begin{itemize}
\item {Grp. gram.:f.}
\end{itemize}
(V.fezes)
\section{Fêz}
\begin{itemize}
\item {Grp. gram.:m.}
\end{itemize}
Barrete moirisco.
\section{Fezes}
\begin{itemize}
\item {Grp. gram.:f. pl.}
\end{itemize}
\begin{itemize}
\item {Utilização:Fig.}
\end{itemize}
\begin{itemize}
\item {Utilização:Pop.}
\end{itemize}
\begin{itemize}
\item {Proveniência:(Do lat. \textunderscore faex\textunderscore )}
\end{itemize}
Sedimento de um líquido.
Bôrra.
Matérias fecaes.
Escória dos metaes.
Ralé social.
Arraia miúda.
Escória, o que há de mais vil ou de mais desprezível.
Impaciência.
Enfado; mau humor.
\textunderscore Fezes de oiro\textunderscore , o mesmo que \textunderscore lithargýrio\textunderscore .
\section{Fiação}
\begin{itemize}
\item {Grp. gram.:f.}
\end{itemize}
Acto, effeito ou modo de fiar^1.
Lugar em que se fia.
\section{Fiacre}
\begin{itemize}
\item {Grp. gram.:m.}
\end{itemize}
\begin{itemize}
\item {Proveniência:(De \textunderscore Fiacre\textunderscore , n. p.)}
\end{itemize}
Carruagem de praça, em França.
\section{Fiada}
\begin{itemize}
\item {Grp. gram.:f.}
\end{itemize}
\begin{itemize}
\item {Proveniência:(De \textunderscore fiado\textunderscore )}
\end{itemize}
Correnteza, fileira de pedras ou tejolos.
Fila, enfiada.
\section{Fiadeira}
\textunderscore fem.\textunderscore  de \textunderscore fiadeiro\textunderscore .
\section{Fiadeiro}
\begin{itemize}
\item {Grp. gram.:m.}
\end{itemize}
\begin{itemize}
\item {Utilização:Prov.}
\end{itemize}
\begin{itemize}
\item {Utilização:trasm.}
\end{itemize}
\begin{itemize}
\item {Proveniência:(De \textunderscore fiar\textunderscore ^1)}
\end{itemize}
Aquelle que se emprega em fiar.
Fogueira, que se acende num terreiro, e em volta da qual se reúnem as mulheres do lugar, nas noites de Setembro a Novembro, para fiar e cantar ou rezar. Cf. Deusdado, \textunderscore Escorços Trasm.\textunderscore , 316.
\section{Fiadilho}
\begin{itemize}
\item {Grp. gram.:m.}
\end{itemize}
\begin{itemize}
\item {Proveniência:(De \textunderscore fiado\textunderscore ^1)}
\end{itemize}
Cadarço, a parte, que se não fia, dos casulos da seda.
\section{Fiado}
\begin{itemize}
\item {Grp. gram.:m.}
\end{itemize}
\begin{itemize}
\item {Proveniência:(De \textunderscore fiar\textunderscore ^1)}
\end{itemize}
Substância filamentosa, reduzida a fio.
\section{Fiado}
\begin{itemize}
\item {Grp. gram.:adj.}
\end{itemize}
\begin{itemize}
\item {Grp. gram.:Adv.}
\end{itemize}
\begin{itemize}
\item {Proveniência:(De \textunderscore fiar\textunderscore ^2)}
\end{itemize}
Que tem fé ou confiança em alguém ou nalguma coisa: \textunderscore fiado na palavra do amigo\textunderscore .
Vendido a crédito: \textunderscore mercadorias fiadas\textunderscore .
A crédito: \textunderscore vender fiado\textunderscore .
\section{Fiadoiro}
\begin{itemize}
\item {Grp. gram.:m.}
\end{itemize}
\begin{itemize}
\item {Utilização:Prov.}
\end{itemize}
\begin{itemize}
\item {Utilização:trasm.}
\end{itemize}
\begin{itemize}
\item {Proveniência:(De \textunderscore fiar\textunderscore ^1)}
\end{itemize}
Compartimento nos baixos de uma casa, onde se deita palha nova e para onde, em noites de inverno, vão fiar e tagarelar as mulheres da vizinhança, e que também se chama \textunderscore fiadeiro\textunderscore .
\section{Fiador}
\begin{itemize}
\item {Grp. gram.:m.}
\end{itemize}
\begin{itemize}
\item {Utilização:Prov.}
\end{itemize}
\begin{itemize}
\item {Utilização:minh.}
\end{itemize}
\begin{itemize}
\item {Utilização:Bras}
\end{itemize}
\begin{itemize}
\item {Grp. gram.:Pl.}
\end{itemize}
\begin{itemize}
\item {Proveniência:(De \textunderscore fiar\textunderscore ^2)}
\end{itemize}
Aquelle que abona alguém, responsabilizando-se pelo cumprimento de uma obrigação delle.
Abonação.
Pedaço de amarra, que dá volta á abita e se prende no anilho.
Descanso de espingarda.
Árvore, que se planta junto de outra, para a substituir, quando esta seque ou apodreça.
Cordão dos copos da espada.
Correia do freio dos animaes.
Boçal sem focinheira.
Correias de ferro, que, ligando máquinas ou carruagens de combóio, servem de segurança, para o caso de se partir o engate.
\section{Fiadoria}
\begin{itemize}
\item {Grp. gram.:f.}
\end{itemize}
\begin{itemize}
\item {Proveniência:(De \textunderscore fiador\textunderscore )}
\end{itemize}
Abonação, fiança.
\section{Fiadouro}
\begin{itemize}
\item {Grp. gram.:m.}
\end{itemize}
\begin{itemize}
\item {Utilização:Prov.}
\end{itemize}
\begin{itemize}
\item {Utilização:trasm.}
\end{itemize}
\begin{itemize}
\item {Proveniência:(De \textunderscore fiar\textunderscore ^1)}
\end{itemize}
Compartimento nos baixos de uma casa, onde se deita palha nova e para onde, em noites de inverno, vão fiar e tagarelar as mulheres da vizinhança, e que também se chama \textunderscore fiadeiro\textunderscore .
\section{Fiadura}
\begin{itemize}
\item {Grp. gram.:f.}
\end{itemize}
O mesmo que \textunderscore fiação\textunderscore .
\section{Fiadura}
\begin{itemize}
\item {Grp. gram.:f.}
\end{itemize}
\begin{itemize}
\item {Utilização:Des.}
\end{itemize}
O mesmo que \textunderscore fiadoria\textunderscore .
\section{Fialho}
\begin{itemize}
\item {Grp. gram.:m.}
\end{itemize}
\begin{itemize}
\item {Utilização:Prov.}
\end{itemize}
\begin{itemize}
\item {Utilização:beir.}
\end{itemize}
O mesmo que \textunderscore fiapo\textunderscore .
\section{Fiambre}
\begin{itemize}
\item {Grp. gram.:m.}
\end{itemize}
Carne, especialmente presunto preparada para se comer fria.
(Cast. \textunderscore fiambre\textunderscore )
\section{Fiambreiro}
\begin{itemize}
\item {Grp. gram.:f.}
\end{itemize}
Caixa, para guardar fiambre.
\section{Fian}
\begin{itemize}
\item {Grp. gram.:m.}
\end{itemize}
\begin{itemize}
\item {Proveniência:(De \textunderscore fio\textunderscore )}
\end{itemize}
Antiga medida de capacidade.
Fiada.
\section{Fiança}
\begin{itemize}
\item {Grp. gram.:f.}
\end{itemize}
\begin{itemize}
\item {Utilização:Ant.}
\end{itemize}
Abonação.
Acto de fiar^2 ou caucionar uma obrigação alheia.
Quantia, com que essa obrigação é caucionada.
Responsabilidade.
Confiança.
\section{Fiançado}
\begin{itemize}
\item {Grp. gram.:adj.}
\end{itemize}
\begin{itemize}
\item {Utilização:Gal}
\end{itemize}
\begin{itemize}
\item {Proveniência:(Fr. \textunderscore fiancé\textunderscore )}
\end{itemize}
Prometido como noivo. Cf. C. Michaelis, \textunderscore Inf. D. Maria\textunderscore , 22.
\section{Fiandapira}
\begin{itemize}
\item {Grp. gram.:f.}
\end{itemize}
\begin{itemize}
\item {Utilização:Bras}
\end{itemize}
Um dos apparelhos de fiação. Cf. \textunderscore Jorn. do Com.\textunderscore , do Rio, de 22-III-901.
\section{Fiandeira}
\begin{itemize}
\item {Grp. gram.:f.}
\end{itemize}
O mesmo que \textunderscore fiadeira\textunderscore .
\section{Fiandeiro}
\begin{itemize}
\item {Grp. gram.:m.}
\end{itemize}
Planta medicinal do Brasil.
O mesmo que \textunderscore fiadeiro\textunderscore .
\section{Fiango}
\begin{itemize}
\item {Grp. gram.:m.}
\end{itemize}
\begin{itemize}
\item {Utilização:Bras. do N}
\end{itemize}
\begin{itemize}
\item {Proveniência:(De \textunderscore fio\textunderscore  + \textunderscore ?\textunderscore )}
\end{itemize}
Rêde velha.
\section{Fiapagem}
\begin{itemize}
\item {Grp. gram.:f.}
\end{itemize}
Porção de fiapos.
\section{Fiá-piquina}
\begin{itemize}
\item {Grp. gram.:f.}
\end{itemize}
Árvore medicinal da ilha de San-Thomé.
(Corr. de \textunderscore fôlha pequena\textunderscore , no dialecto santhomense)
\section{Fiapo}
\begin{itemize}
\item {Grp. gram.:m.}
\end{itemize}
\begin{itemize}
\item {Proveniência:(De \textunderscore fio\textunderscore )}
\end{itemize}
Fio tênue, fiozinho.
\section{Fiar}
\begin{itemize}
\item {Grp. gram.:v. t.}
\end{itemize}
\begin{itemize}
\item {Utilização:Fig.}
\end{itemize}
\begin{itemize}
\item {Grp. gram.:V. i.}
\end{itemize}
\begin{itemize}
\item {Proveniência:(Lat. raro \textunderscore filare\textunderscore , de \textunderscore filum\textunderscore )}
\end{itemize}
Reduzir a fio (matérias filamentosas).
Puxar á fieira.
Serrar, (tábua, caibro, etc.) pelo meio, longitudinalmente.
Tramar.
Torcer, reduzir a fio, qualquer matéria filamentosa.
\section{Fiar}
\begin{itemize}
\item {Grp. gram.:v. t.}
\end{itemize}
\begin{itemize}
\item {Grp. gram.:V. i.}
\end{itemize}
\begin{itemize}
\item {Proveniência:(It. \textunderscore fidare\textunderscore , do lat. \textunderscore fides\textunderscore )}
\end{itemize}
Entregar confiadamente.
Confiar á fé de alguém.
Sêr fiador de.
Vender a crédito.
Têr confiança: \textunderscore fio que te portarás bem\textunderscore .
Vender, sem receber logo a importância do que vende.
\section{Fiasco}
\begin{itemize}
\item {Grp. gram.:m.}
\end{itemize}
\begin{itemize}
\item {Proveniência:(It. \textunderscore fiasco\textunderscore )}
\end{itemize}
Êxito desfavorável, vexatório, ridículo: \textunderscore o drama fez fiasco\textunderscore .
Estenderete.
\section{Fiável}
\begin{itemize}
\item {Grp. gram.:adj.}
\end{itemize}
\begin{itemize}
\item {Proveniência:(De \textunderscore fiar\textunderscore ^1)}
\end{itemize}
Que póde sêr fiado.
\section{Fiá-zaiá}
\begin{itemize}
\item {Grp. gram.:f.}
\end{itemize}
Planta medicinal da ilha de San-Thomé.
(Corr. de \textunderscore fôlha\textunderscore  + (?), no dialecto santhomense)
\section{Fibra}
\begin{itemize}
\item {Grp. gram.:f.}
\end{itemize}
\begin{itemize}
\item {Proveniência:(Lat. \textunderscore fibra\textunderscore )}
\end{itemize}
O mesmo que \textunderscore fêvera\textunderscore .
Elemento anatómico, longo e frágil.
\section{Fibrila}
\begin{itemize}
\item {Grp. gram.:f.}
\end{itemize}
(V.fibrilha)
\section{Fibrilar}
\begin{itemize}
\item {Grp. gram.:adj.}
\end{itemize}
\begin{itemize}
\item {Proveniência:(De \textunderscore fibrila\textunderscore )}
\end{itemize}
Disposto em pequenas fibras.
\section{Fibrilha}
\begin{itemize}
\item {Grp. gram.:f.}
\end{itemize}
Pequena fibra.
Cada uma das últimas ramificações das raízes.
\section{Fibrília}
\begin{itemize}
\item {Grp. gram.:f.}
\end{itemize}
\begin{itemize}
\item {Proveniência:(De \textunderscore fibra\textunderscore )}
\end{itemize}
Matéria têxtil, a que se reduz o linho e o cânhamo, para lhe dar as propriedades do algodão.--T. criado em 1861 nos Estados-Unidos.
\section{Fibrilífero}
\begin{itemize}
\item {Grp. gram.:adj.}
\end{itemize}
\begin{itemize}
\item {Proveniência:(De \textunderscore fibrila\textunderscore  + lat. \textunderscore ferre\textunderscore )}
\end{itemize}
Que tem muitos filamentos ou fibras.
\section{Fibrilla}
\begin{itemize}
\item {Grp. gram.:f.}
\end{itemize}
(V.fibrilha)
\section{Fibrillar}
\begin{itemize}
\item {Grp. gram.:adj.}
\end{itemize}
\begin{itemize}
\item {Proveniência:(De \textunderscore fibrilla\textunderscore )}
\end{itemize}
Disposto em pequenas fibras.
\section{Fibrillífero}
\begin{itemize}
\item {Grp. gram.:adj.}
\end{itemize}
\begin{itemize}
\item {Proveniência:(De \textunderscore fibrilla\textunderscore  + lat. \textunderscore ferre\textunderscore )}
\end{itemize}
Que tem muitos filamentos ou fibras.
\section{Fibrilloso}
\begin{itemize}
\item {Grp. gram.:adj.}
\end{itemize}
\begin{itemize}
\item {Proveniência:(De \textunderscore fibrilla\textunderscore )}
\end{itemize}
Que resulta de uma reunião de fibrilhas.
\section{Fibriloso}
\begin{itemize}
\item {Grp. gram.:adj.}
\end{itemize}
\begin{itemize}
\item {Proveniência:(De \textunderscore fibrila\textunderscore )}
\end{itemize}
Que resulta de uma reunião de fibrilhas.
\section{Fibrina}
\begin{itemize}
\item {Grp. gram.:f.}
\end{itemize}
\begin{itemize}
\item {Proveniência:(De \textunderscore fibrino\textunderscore )}
\end{itemize}
Substância orgânica, um pouco elástica, que se encontra na lympha, no chylo, no sangue e em outros líquidos emanados do sangue.
\section{Fibrino}
\begin{itemize}
\item {Grp. gram.:adj.}
\end{itemize}
\begin{itemize}
\item {Proveniência:(Lat. \textunderscore fibrinus\textunderscore )}
\end{itemize}
Relativo ás fibras.
\section{Fibrinoso}
\begin{itemize}
\item {Grp. gram.:adj.}
\end{itemize}
Relativo a fibrina.
\section{Fibro-cartilagem}
\begin{itemize}
\item {Grp. gram.:f.}
\end{itemize}
Tecido cartilaginoso, de trama fibroide.
\section{Fibro-cellular}
\begin{itemize}
\item {Grp. gram.:adj.}
\end{itemize}
\begin{itemize}
\item {Proveniência:(De \textunderscore fibra\textunderscore  + \textunderscore cellular\textunderscore )}
\end{itemize}
Que participa do tecido fibroso e do cellular.
\section{Fibroferrite}
\begin{itemize}
\item {Grp. gram.:f.}
\end{itemize}
Sub-sulfato de ferro, em massas fibrosas.
\section{Fibro-granular}
\begin{itemize}
\item {Grp. gram.:adj.}
\end{itemize}
\begin{itemize}
\item {Proveniência:(De \textunderscore fibra\textunderscore  + \textunderscore granular\textunderscore )}
\end{itemize}
Que apresenta tecido granuloso, entremeado de fibras.
\section{Fibroide}
\begin{itemize}
\item {Grp. gram.:adj.}
\end{itemize}
\begin{itemize}
\item {Proveniência:(De \textunderscore fibra\textunderscore  + gr. \textunderscore eidos\textunderscore )}
\end{itemize}
Semelhante a fibras.
\section{Fibrólitho}
\begin{itemize}
\item {Grp. gram.:m.}
\end{itemize}
\begin{itemize}
\item {Proveniência:(De \textunderscore fibra\textunderscore  + gr. \textunderscore lithos\textunderscore )}
\end{itemize}
Mineral de textura fibrosa.
\section{Fibrólito}
\begin{itemize}
\item {Grp. gram.:m.}
\end{itemize}
\begin{itemize}
\item {Proveniência:(De \textunderscore fibra\textunderscore  + gr. \textunderscore lithos\textunderscore )}
\end{itemize}
Mineral de textura fibrosa.
\section{Fibroma}
\begin{itemize}
\item {Grp. gram.:m.}
\end{itemize}
\begin{itemize}
\item {Proveniência:(De \textunderscore fibra\textunderscore )}
\end{itemize}
Tumor fibroso.
\section{Fibro-mucoso}
\begin{itemize}
\item {Grp. gram.:adj.}
\end{itemize}
Formado de membrana mucosa, sobreposta a uma membrana fibrosa.
\section{Fibro-plástico}
\begin{itemize}
\item {Grp. gram.:adj.}
\end{itemize}
\begin{itemize}
\item {Utilização:Anat.}
\end{itemize}
Diz-se do tecido, que se representa sob a fórma de tumores, compostos principalmente de corpos fusiformes e de matéria amorpha.
\section{Fibro-sedoso}
\begin{itemize}
\item {Grp. gram.:adj.}
\end{itemize}
\begin{itemize}
\item {Utilização:Geol.}
\end{itemize}
Composto de filamentos, que têm o brilho da seda.
\section{Fibro-seroso}
\begin{itemize}
\item {Grp. gram.:adj.}
\end{itemize}
Composto de uma membrana serosa, sobreposta a uma membrana fibrosa.
\section{Fibroso}
\begin{itemize}
\item {Grp. gram.:adj.}
\end{itemize}
Que contém fibras.
Composto de fibras.
Semelhante a fibras.
Relativo a fibras.
\section{Fibro-vascular}
\begin{itemize}
\item {Grp. gram.:adj.}
\end{itemize}
\begin{itemize}
\item {Utilização:Anat.}
\end{itemize}
\begin{itemize}
\item {Proveniência:(De \textunderscore fibra\textunderscore  + \textunderscore vascular\textunderscore )}
\end{itemize}
Composto de vasos e fibras.
\section{Fíbula}
\begin{itemize}
\item {Grp. gram.:f.}
\end{itemize}
\begin{itemize}
\item {Utilização:Des.}
\end{itemize}
\begin{itemize}
\item {Proveniência:(Lat. \textunderscore fibula\textunderscore )}
\end{itemize}
O mesmo que \textunderscore fivela\textunderscore .
\section{Fibulação}
\begin{itemize}
\item {Grp. gram.:f.}
\end{itemize}
\begin{itemize}
\item {Proveniência:(De \textunderscore fíbula\textunderscore )}
\end{itemize}
Desusada operação cirúrgica, que consistia em unir os bordos de uma ferida por meio de colchetes ou ganchos.
\section{Ficáceas}
\begin{itemize}
\item {Grp. gram.:f. pl.}
\end{itemize}
\begin{itemize}
\item {Proveniência:(De \textunderscore ficáceo\textunderscore )}
\end{itemize}
Família de plantas, que têm por typo a figueira.
\section{Ficáceo}
\begin{itemize}
\item {Grp. gram.:adj.}
\end{itemize}
\begin{itemize}
\item {Proveniência:(Do lat. \textunderscore ficus\textunderscore )}
\end{itemize}
Relativo ou semelhante á figueira.
\section{Ficada}
\begin{itemize}
\item {Grp. gram.:f.}
\end{itemize}
\begin{itemize}
\item {Utilização:P. us.}
\end{itemize}
Acto de ficar.
\section{Ficar}
\begin{itemize}
\item {Grp. gram.:v. i.}
\end{itemize}
\begin{itemize}
\item {Grp. gram.:V. t.}
\end{itemize}
\begin{itemize}
\item {Utilização:Prov.}
\end{itemize}
\begin{itemize}
\item {Grp. gram.:V. p.}
\end{itemize}
\begin{itemize}
\item {Proveniência:(Do lat. hyp. \textunderscore figicare\textunderscore , de \textunderscore figere\textunderscore )}
\end{itemize}
Conservar-se num lugar; não saír delle.
Parar de repente.
Quedar-se, morrer.
Servir (de fiador)
Subsistir.
Caber em herança, ou por sorte.
Restar.
\textunderscore Ficar a olhar ao sinal\textunderscore , ficar a vêr navios.
\textunderscore Ficar em águas de bacalhau\textunderscore , ficar em nada, não têr resultado.
\textunderscore Ficar para tia\textunderscore , ficar solteira.
\textunderscore Ficar sem pinga de sangue\textunderscore , ficar muito assustado.
O mesmo que \textunderscore deixar\textunderscore : \textunderscore coitada! morreu e ficou três filhinhos\textunderscore .
Parar de repente.
Quedar-se; permanecer.
Não comprar mais cartas, (no jogo do \textunderscore trinta e um\textunderscore  e em outros).
\section{Ficar}
\begin{itemize}
\item {Grp. gram.:v. t.}
\end{itemize}
\begin{itemize}
\item {Utilização:Ant.}
\end{itemize}
O mesmo que \textunderscore fincar\textunderscore .
\section{...ficar}
\begin{itemize}
\item {Grp. gram.:suf. verb.}
\end{itemize}
\begin{itemize}
\item {Proveniência:(Do lat. \textunderscore facere\textunderscore )}
\end{itemize}
(designativo de \textunderscore fazer\textunderscore , \textunderscore causar\textunderscore , etc.)
\section{Ficária}
\begin{itemize}
\item {Grp. gram.:f.}
\end{itemize}
\begin{itemize}
\item {Proveniência:(Lat. \textunderscore ficaria\textunderscore )}
\end{itemize}
Gênero de plantas ranunculáceas.
\section{Ficário}
\begin{itemize}
\item {Grp. gram.:adj.}
\end{itemize}
\begin{itemize}
\item {Proveniência:(Lat. \textunderscore ficarius\textunderscore )}
\end{itemize}
Relativo a figo.
Dizia-se especialmente dos faunos, em attenção á sua lascívia.
\section{Ficção}
\begin{itemize}
\item {Grp. gram.:f.}
\end{itemize}
\begin{itemize}
\item {Proveniência:(Lat. \textunderscore fictio\textunderscore )}
\end{itemize}
Acto ou effeito de fingir.
Simulação.
Coisa imaginária.
\section{Ficha}
\begin{itemize}
\item {Grp. gram.:f.}
\end{itemize}
\begin{itemize}
\item {Proveniência:(Do ingl. \textunderscore fish\textunderscore , ou do fr. \textunderscore fiche\textunderscore ?)}
\end{itemize}
Tento para o jôgo, em fórma de peixe.
Tento, usado no jôgo da roleta, de fórma redonda, e que varia de côr, segundo a quantia que representa.
\section{Ficiforme}
\begin{itemize}
\item {Grp. gram.:adj.}
\end{itemize}
\begin{itemize}
\item {Proveniência:(Do lat. \textunderscore ficus\textunderscore  + \textunderscore forma\textunderscore )}
\end{itemize}
Que tem fórma de figo.
\section{Ficínia}
\begin{itemize}
\item {Grp. gram.:f.}
\end{itemize}
\begin{itemize}
\item {Proveniência:(De \textunderscore Ficinus\textunderscore , n. p.)}
\end{itemize}
Variedade de junça, do Cabo da Bôa-Esperança.
\section{...fico}
\begin{itemize}
\item {Grp. gram.:suf. adj.}
\end{itemize}
\begin{itemize}
\item {Proveniência:(Lat. \textunderscore ...ficus\textunderscore )}
\end{itemize}
Que faz, que produz, que causa: \textunderscore frigorífico\textunderscore ; \textunderscore benéfico\textunderscore ; \textunderscore prolífico\textunderscore .
\section{Ficoide}
\begin{itemize}
\item {Grp. gram.:f.}
\end{itemize}
\begin{itemize}
\item {Proveniência:(Do lat. \textunderscore ficus\textunderscore  + gr. \textunderscore eidos\textunderscore )}
\end{itemize}
Gênero de plantas exóticas, parecidas ao cacto, e originárias do Cabo da Bôa-Esperança.
\section{Ficoídeas}
\begin{itemize}
\item {Grp. gram.:f. pl.}
\end{itemize}
Família de plantas dicotyledóneas, que comprehende apenas o gênero ficoide.
(Cp. \textunderscore ficoide\textunderscore )
\section{Ficoíta}
\begin{itemize}
\item {Grp. gram.:f.}
\end{itemize}
\begin{itemize}
\item {Proveniência:(Do rad. do lat. \textunderscore ficus\textunderscore )}
\end{itemize}
Espécie de figueira fóssil.
\section{Ficotico}
\begin{itemize}
\item {Grp. gram.:m.}
\end{itemize}
Ave do Brasil.
\section{Ficticiamente}
\begin{itemize}
\item {Grp. gram.:adv.}
\end{itemize}
De modo fictício.
Simuladamente.
\section{Fictício}
\begin{itemize}
\item {Grp. gram.:adj.}
\end{itemize}
\begin{itemize}
\item {Proveniência:(Lat. \textunderscore ficticius\textunderscore )}
\end{itemize}
Imaginário.
Fabuloso.
Simulado; illusório.
\section{Ficticioso}
\begin{itemize}
\item {Grp. gram.:adj.}
\end{itemize}
\begin{itemize}
\item {Utilização:T. de Ceilão}
\end{itemize}
O mesmo que \textunderscore fictício\textunderscore .
\section{Ficto}
\begin{itemize}
\item {Grp. gram.:adj.}
\end{itemize}
\begin{itemize}
\item {Proveniência:(Lat. \textunderscore fictus\textunderscore )}
\end{itemize}
Fingido.
Illusório.
\section{Fictor}
\begin{itemize}
\item {Grp. gram.:m.}
\end{itemize}
\begin{itemize}
\item {Proveniência:(Lat. \textunderscore fictor\textunderscore )}
\end{itemize}
Aquelle que, entre os pagãos, fabricava figuras de cera, de pão, etc., para preencher a falta de animaes para os sacrifícios.
\section{Fidagal}
\begin{itemize}
\item {Grp. gram.:adj.}
\end{itemize}
(Metáth. pop. de \textunderscore figadal\textunderscore , e contr. de \textunderscore fidalgal\textunderscore ). Cf. Camillo, \textunderscore Perfil do Marquês\textunderscore , 173.
\section{Fidalga}
\begin{itemize}
\item {Grp. gram.:f.}
\end{itemize}
\begin{itemize}
\item {Proveniência:(De \textunderscore fidalgo\textunderscore )}
\end{itemize}
Mulher de fidalgo.
Mulher nobre.
Variedade de pêra muito apreciada.
Cp. \textunderscore atequipera\textunderscore .
\section{Fidalgaço}
\begin{itemize}
\item {Grp. gram.:m.}
\end{itemize}
O mesmo que \textunderscore fidalgarrão\textunderscore :«\textunderscore ...eram as línguas dos fidalgaços\textunderscore ». Camillo, \textunderscore Caveira\textunderscore , 47.
\section{Fidalgal}
\begin{itemize}
\item {Grp. gram.:adj.}
\end{itemize}
Relativo a fidalgo; próprio de fidalgo.
\section{Fidalgamente}
\begin{itemize}
\item {Grp. gram.:adv.}
\end{itemize}
Com bizarria; de modo fidalgo.
\section{Fidalgaria}
\begin{itemize}
\item {Grp. gram.:f.}
\end{itemize}
Classe dos fidalgos; chusma de fidalgos.
Modos de fidalgo.
\section{Fidalgarrão}
\begin{itemize}
\item {Grp. gram.:m.}
\end{itemize}
\begin{itemize}
\item {Utilização:Deprec.}
\end{itemize}
\begin{itemize}
\item {Proveniência:(De \textunderscore fidalgo\textunderscore )}
\end{itemize}
Aquelle que alardeia fidalguia. Cf. Castilho, \textunderscore Fausto\textunderscore , 195.
\section{Fidalgo}
\begin{itemize}
\item {Grp. gram.:adj.}
\end{itemize}
\begin{itemize}
\item {Grp. gram.:M.}
\end{itemize}
\begin{itemize}
\item {Utilização:Pop.}
\end{itemize}
Relativo a fidalgaria.
Nobre.
Bizarro; generoso: \textunderscore procedimento fidalgo\textunderscore .
Indivíduo, que tem títulos de nobreza.
Indivíduo bem vestido; aquelle que vive dos seus rendimentos, sem trabalhar.
Peixe do norte do Brasil.
(Cont. de \textunderscore filho\textunderscore  + \textunderscore de\textunderscore  + \textunderscore algo\textunderscore )
\section{Fidalgoso}
\begin{itemize}
\item {Grp. gram.:adj.}
\end{itemize}
\begin{itemize}
\item {Proveniência:(De \textunderscore fidalgo\textunderscore )}
\end{itemize}
Em que há fidalguia, nobreza ou distincção. Cp. P. Caldas, \textunderscore Alv. de Braga\textunderscore .
\section{Fidalgote}
\begin{itemize}
\item {Grp. gram.:m.}
\end{itemize}
\begin{itemize}
\item {Proveniência:(De \textunderscore fidalgo\textunderscore )}
\end{itemize}
Indivíduo afidalgado, que vive fidalgamente, tendo poucos ou duvidosos títulos de nobreza.
\section{Fidalgueiro}
\begin{itemize}
\item {Grp. gram.:m.  e  adj.}
\end{itemize}
\begin{itemize}
\item {Proveniência:(De \textunderscore fidalgo\textunderscore )}
\end{itemize}
Aquelle que procura e frequenta a convivência dos fidalgos.
\section{Fidalguelho}
\begin{itemize}
\item {fónica:guê}
\end{itemize}
\begin{itemize}
\item {Grp. gram.:m.}
\end{itemize}
\begin{itemize}
\item {Utilização:Deprec.}
\end{itemize}
Fidalgo de pouca nobreza.
Fidalgo ridículo. Cf. Arn. Gama, \textunderscore Segr. do Abb.\textunderscore , 122.
\section{Fidalguesco}
\begin{itemize}
\item {fónica:guês}
\end{itemize}
\begin{itemize}
\item {Grp. gram.:adj.}
\end{itemize}
Relativo a fidalgos ou a fidalguia. Cf. Camillo, \textunderscore Cav. em Ruínas\textunderscore , 49.
\section{Fidalguete}
\begin{itemize}
\item {fónica:guê}
\end{itemize}
\begin{itemize}
\item {Utilização:Deprec.}
\end{itemize}
O mesmo que \textunderscore fidalgote\textunderscore . Cf. Filinto, XXI, 103.
\section{Fidalguia}
\begin{itemize}
\item {Grp. gram.:f.}
\end{itemize}
Qualidade de quem é fidalgo.
Classe dos fidalgos.
Acção própria de fidalgo; bizarria, generosidade.
\section{Fidalguice}
\begin{itemize}
\item {Grp. gram.:f.}
\end{itemize}
\begin{itemize}
\item {Proveniência:(De \textunderscore fidalgo\textunderscore )}
\end{itemize}
Qualidade de fidalgote.
Ostentação balofa; bazófia.
\section{Fidalguinho}
\begin{itemize}
\item {Grp. gram.:m.}
\end{itemize}
Nome de uma planta annual, o mesmo que \textunderscore lóio\textunderscore ^1.
Designação vulgar, que em Lisbôa se dá ao \textunderscore cébus\textunderscore .
\section{Fidalguito}
\begin{itemize}
\item {Grp. gram.:m.}
\end{itemize}
Planta da serra de Cintra, o mesmo que \textunderscore fidalguinho\textunderscore .
\section{Fidèdignidade}
\begin{itemize}
\item {Grp. gram.:f.}
\end{itemize}
Qualidade daquelle ou daquillo que é fidèdigno. Cf. Filinto, \textunderscore D. Man.\textunderscore , II, 258.
\section{Fidèdigno}
\begin{itemize}
\item {Grp. gram.:adj.}
\end{itemize}
\begin{itemize}
\item {Proveniência:(Do lat. \textunderscore fides\textunderscore  + \textunderscore dignus\textunderscore )}
\end{itemize}
Digno de fé.
Merecedor de crédito: \textunderscore testemunho fidèdigno\textunderscore .
\section{Fideicomissário}
\begin{itemize}
\item {Grp. gram.:adj.}
\end{itemize}
\begin{itemize}
\item {Grp. gram.:M.}
\end{itemize}
\begin{itemize}
\item {Proveniência:(Lat. \textunderscore fideicommissarius\textunderscore )}
\end{itemize}
Relativo a fideicomisso.
Aquele que recebe do fiduciário a herança ou legado respectivo.
\section{Fideicomisso}
\begin{itemize}
\item {Grp. gram.:m.}
\end{itemize}
\begin{itemize}
\item {Proveniência:(Lat. \textunderscore fideicommissum\textunderscore )}
\end{itemize}
Disposição testamentária, em que um herdeiro ou legatário é encarregado de conservar, e, por sua morte, transmitir a outrem a sua herança ou o seu legado.
\section{Fideicomissório}
\begin{itemize}
\item {Grp. gram.:adj.}
\end{itemize}
Que envolve fideicomiso; relativo a fideicomisso.
\section{Fideicommissario}
\begin{itemize}
\item {Grp. gram.:adj.}
\end{itemize}
\begin{itemize}
\item {Grp. gram.:M.}
\end{itemize}
\begin{itemize}
\item {Proveniência:(Lat. \textunderscore fideicommissarius\textunderscore )}
\end{itemize}
Relativo a fideicommisso.
Aquelle que recebe do fiduciário a herança ou legado respectivo.
\section{Fideicommisso}
\begin{itemize}
\item {Grp. gram.:m.}
\end{itemize}
\begin{itemize}
\item {Proveniência:(Lat. \textunderscore fideicommissum\textunderscore )}
\end{itemize}
Disposição testamentária, em que um herdeiro ou legatário é encarregado de conservar, e, por sua morte, transmittir a outrem a sua herança ou o seu legado.
\section{Fideicommissório}
\begin{itemize}
\item {Grp. gram.:adj.}
\end{itemize}
Que envolve fideicommiso; relativo a fideicommisso.
\section{Fidèjussória}
\begin{itemize}
\item {Grp. gram.:f.}
\end{itemize}
\begin{itemize}
\item {Utilização:Jur.}
\end{itemize}
\begin{itemize}
\item {Proveniência:(De \textunderscore fidèjussório\textunderscore )}
\end{itemize}
Espécie de caução; fiança.
\section{Fidèjussório}
\begin{itemize}
\item {Grp. gram.:adj.}
\end{itemize}
\begin{itemize}
\item {Proveniência:(Lat. \textunderscore fidejussorius\textunderscore )}
\end{itemize}
Relativo a fiança.
\section{Fidelidade}
\begin{itemize}
\item {Grp. gram.:f.}
\end{itemize}
\begin{itemize}
\item {Proveniência:(Lat. \textunderscore fidelitas\textunderscore )}
\end{itemize}
Qualidade de quem é fiel.
Lealdade, firmeza.
Exactidão, verdade: \textunderscore fidelidade de uma cópia\textunderscore .
Probidade.
\section{Fidelissimamente}
\begin{itemize}
\item {Grp. gram.:adv.}
\end{itemize}
De modo fidelíssimo.
\section{Fidelíssimo}
\begin{itemize}
\item {Grp. gram.:adj.}
\end{itemize}
\begin{itemize}
\item {Proveniência:(Do lat. \textunderscore fidelis\textunderscore )}
\end{itemize}
Muito fiel.
\section{Fideputa}
\begin{itemize}
\item {Grp. gram.:m.}
\end{itemize}
\begin{itemize}
\item {Utilização:ant.}
\end{itemize}
\begin{itemize}
\item {Utilização:Pleb.}
\end{itemize}
Homem de má raça e de má índole:«\textunderscore oh fideputa bargante!\textunderscore »Camões, \textunderscore Selenco\textunderscore , prol. Cf. G. Vicente, I, 254.
\section{Fidéus}
\begin{itemize}
\item {Grp. gram.:m. pl.}
\end{itemize}
\begin{itemize}
\item {Proveniência:(Do cast. \textunderscore fideos\textunderscore )}
\end{itemize}
Aletria; massa em fios.
\section{Fídia}
\begin{itemize}
\item {Grp. gram.:f.}
\end{itemize}
\begin{itemize}
\item {Proveniência:(De \textunderscore Fidia\textunderscore , n. p. myth.)}
\end{itemize}
Gênero de insectos coleópteros tetrâmeros da América.
\section{Fido}
\begin{itemize}
\item {Grp. gram.:adj.}
\end{itemize}
\begin{itemize}
\item {Utilização:Poét.}
\end{itemize}
\begin{itemize}
\item {Proveniência:(Lat. \textunderscore fidus\textunderscore )}
\end{itemize}
Fiel.
\section{Fidónia}
\begin{itemize}
\item {Grp. gram.:f.}
\end{itemize}
\begin{itemize}
\item {Proveniência:(De \textunderscore Fidónia\textunderscore , n. p. myth.)}
\end{itemize}
Gênero de insectos lepidópteros nocturnos.
\section{Fidúcia}
\begin{itemize}
\item {Grp. gram.:f.}
\end{itemize}
\begin{itemize}
\item {Utilização:Pop.}
\end{itemize}
\begin{itemize}
\item {Proveniência:(Lat. \textunderscore fiducia\textunderscore )}
\end{itemize}
Confiança.
Atrevimento; prosápia.
\section{Fiducial}
\begin{itemize}
\item {Grp. gram.:adj.}
\end{itemize}
Relativo a fidúcia.
\section{Fiduciário}
\begin{itemize}
\item {Grp. gram.:adj.}
\end{itemize}
\begin{itemize}
\item {Grp. gram.:M.}
\end{itemize}
\begin{itemize}
\item {Proveniência:(Lat. \textunderscore fiduciarius\textunderscore )}
\end{itemize}
Fiducial.
Que revela confiança.
Que depende de confiança.
O encarregado de transmittir a outrem uma herança ou legado, recebido com a condição dessa transmissão.
\section{Fieira}
\begin{itemize}
\item {Grp. gram.:f.}
\end{itemize}
\begin{itemize}
\item {Utilização:Fig.}
\end{itemize}
\begin{itemize}
\item {Utilização:Açor}
\end{itemize}
\begin{itemize}
\item {Proveniência:(De \textunderscore fio\textunderscore )}
\end{itemize}
Apparelho, para reduzir a fio qualquer metal.
Fileira, alinhamento.
Veio mineral.
Viga, em que assentam as asnas do telhado.
Barbante, guita.
Meio de prova; experiência.
Cordel, que prende superiormente o arame á ponta da cana da pesca. (Colhido em San-Jorge)
\section{Fieiro}
\begin{itemize}
\item {Grp. gram.:m.}
\end{itemize}
\begin{itemize}
\item {Utilização:Prov.}
\end{itemize}
\begin{itemize}
\item {Utilização:alent.}
\end{itemize}
\begin{itemize}
\item {Proveniência:(De \textunderscore fio\textunderscore )}
\end{itemize}
Enfiada de bolotas, que se põe ao fumeiro para curar ou avellar.
\section{Fiel}
\begin{itemize}
\item {Grp. gram.:adj.}
\end{itemize}
\begin{itemize}
\item {Utilização:Pop.}
\end{itemize}
\begin{itemize}
\item {Grp. gram.:M.}
\end{itemize}
\begin{itemize}
\item {Utilização:Prov.}
\end{itemize}
\begin{itemize}
\item {Grp. gram.:M. pl.}
\end{itemize}
\begin{itemize}
\item {Utilização:Ant.}
\end{itemize}
\begin{itemize}
\item {Proveniência:(Do lat. \textunderscore fidelis\textunderscore )}
\end{itemize}
Que cumpre aquillo a que se obriga; probo.
Pontual.
Exacto: \textunderscore copia fiel\textunderscore .
Verídico.
Firme, leal: \textunderscore espôsa fiel\textunderscore .
Que não furta: \textunderscore tenho uma criada muito fiel\textunderscore .
Empregado, que tem a seu cargo a guarda de gêneros, papéis ou dinheiro.
Ajudante de thesoireiro.
Talicão, que se deixa ordinariamente abaixo da vara fructífera das videiras, destinado a garantir poda e rebentação futura, se a vara não dér o resultado que se deseja.
Cathólicos.
Sectários de uma religião.
Árbitros, louvados.
Fio de metal, pôsto a prumo no centro da gravidade da balança, para se conhecer a igualdade ou a desigualdade dos pesos:«\textunderscore ficou a oscillar... como balança no fiel\textunderscore ». Herculano, \textunderscore Cister\textunderscore , I, 156.
\section{Fieldade}
\begin{itemize}
\item {Grp. gram.:f.}
\end{itemize}
\begin{itemize}
\item {Utilização:Des.}
\end{itemize}
O mesmo que \textunderscore fidelidade\textunderscore .
\section{Fielmente}
\begin{itemize}
\item {Grp. gram.:adv.}
\end{itemize}
De modo fiel.
\section{Fieto}
\begin{itemize}
\item {Grp. gram.:m.}
\end{itemize}
\begin{itemize}
\item {Utilização:Prov.}
\end{itemize}
\begin{itemize}
\item {Utilização:trasm.}
\end{itemize}
O mesmo que \textunderscore fêto\textunderscore .
(Contr. de \textunderscore figueito\textunderscore )
\section{Fieza}
\begin{itemize}
\item {Grp. gram.:f.}
\end{itemize}
\begin{itemize}
\item {Utilização:P. us.}
\end{itemize}
\begin{itemize}
\item {Proveniência:(De \textunderscore fiar\textunderscore )}
\end{itemize}
Confiança. Cf. Filinto, VI, 182.
\section{Fífia}
\begin{itemize}
\item {Grp. gram.:f.}
\end{itemize}
\begin{itemize}
\item {Proveniência:(T. onom.?)}
\end{itemize}
Voz ou som desafinado.
\section{Figa}
\begin{itemize}
\item {Grp. gram.:f.}
\end{itemize}
\begin{itemize}
\item {Utilização:Fig.}
\end{itemize}
\begin{itemize}
\item {Grp. gram.:Pl. Interj.}
\end{itemize}
\begin{itemize}
\item {Proveniência:(Do it. \textunderscore fica\textunderscore )}
\end{itemize}
Pequeno objecto, da fórma de uma mão fechada, com o dedo pollegar entre o indicador e o maior, o que se usa supersticiosamente como preservativo de malefícios, doenças, etc.; amuleto.
Acto de fechar a mão, metendo o dedo pollegar entre o indicador e o maior, para esconjurar ou repellir.
Esconjuro.
Coisa ruim, desprezível.
Redemoinho de pêlo, na barriga do cavallo, onde a espora fere.
Abrenúncio!
Eu te arrenego!
Some-te!
\textunderscore Figas, canhoto\textunderscore , vai-te, demónio! deixa-me.
\section{Figadal}
\begin{itemize}
\item {Grp. gram.:adj.}
\end{itemize}
\begin{itemize}
\item {Utilização:Fig.}
\end{itemize}
\begin{itemize}
\item {Utilização:T. de Turquel}
\end{itemize}
Relativo ao fígado.
Íntimo.
Profundo; intenso: \textunderscore ódio figadal\textunderscore .
Que tem o rosto muito còrado.
\section{Figadalmente}
\begin{itemize}
\item {Grp. gram.:adv.}
\end{itemize}
De modo figadal.
\section{Figadeira}
\begin{itemize}
\item {Grp. gram.:f.}
\end{itemize}
\begin{itemize}
\item {Utilização:Fam.}
\end{itemize}
Doença no fígado dos animaes.
O mesmo que \textunderscore hepatite\textunderscore .
\section{Fígado}
\begin{itemize}
\item {Grp. gram.:m.}
\end{itemize}
\begin{itemize}
\item {Utilização:Fig.}
\end{itemize}
\begin{itemize}
\item {Proveniência:(Do lat. \textunderscore ficatum\textunderscore )}
\end{itemize}
Víscera volumosa, parte da qual produz a bílis, em-quanto a outra parte produz a substância açucarada que passa aos pulmões, servindo aos actos respiratórios.
Coragem.
Índole, carácter.--Em Trás-os-Montes, ouve-se \textunderscore figádo\textunderscore , prosódia análoga á do lat. \textunderscore ficatum\textunderscore .
\section{Fígaro}
\begin{itemize}
\item {Grp. gram.:m.}
\end{itemize}
\begin{itemize}
\item {Utilização:Fam.}
\end{itemize}
\begin{itemize}
\item {Proveniência:(De \textunderscore Fígaro\textunderscore , n. p.)}
\end{itemize}
O mesmo que \textunderscore barbeiro\textunderscore .
\section{Figle}
\begin{itemize}
\item {Grp. gram.:m.}
\end{itemize}
Instrumento músico de latão.
(Corr. do fr. \textunderscore opbicleide\textunderscore )
\section{Figo}
\begin{itemize}
\item {Grp. gram.:m.}
\end{itemize}
\begin{itemize}
\item {Utilização:Fig.}
\end{itemize}
\begin{itemize}
\item {Proveniência:(Lat. \textunderscore ficus\textunderscore )}
\end{itemize}
Fruto da figueira.
Nome vulgar do fruto de algumas piteiras e palmeiras.
Coisa amachuchada, sêca, magra: \textunderscore reduziu-lhe o chapéu a um figo\textunderscore .
\section{Figo}
\begin{itemize}
\item {Grp. gram.:m.}
\end{itemize}
\begin{itemize}
\item {Utilização:Ant.}
\end{itemize}
O mesmo que \textunderscore filho\textunderscore .
\section{Figo-de-tordo}
\begin{itemize}
\item {Grp. gram.:m.}
\end{itemize}
Árvore medicinal da ilha de San-Thome, (\textunderscore figu-tudô\textunderscore , no dialeto santhomense).
\section{Figo-loiro}
\begin{itemize}
\item {Grp. gram.:m.}
\end{itemize}
O mesmo que \textunderscore pàpafigo\textunderscore , ave.
\section{Figo-porco}
\begin{itemize}
\item {Grp. gram.:m.}
\end{itemize}
Árvore medicinal da ilha do San-Thomé, (\textunderscore figu-plôcu\textunderscore , no dialecto santhomense).
\section{Figueira}
\begin{itemize}
\item {Grp. gram.:f.}
\end{itemize}
\begin{itemize}
\item {Grp. gram.:Loc.}
\end{itemize}
\begin{itemize}
\item {Utilização:fam.}
\end{itemize}
\begin{itemize}
\item {Proveniência:(Lat. \textunderscore ficaria\textunderscore )}
\end{itemize}
Árvore fructífera, da fam. das urticáceas.
Árvore silvestre do Brasil.
\textunderscore Plantar uma figueira\textunderscore , dar uma quéda.
\section{Figueira-bispo}
\begin{itemize}
\item {Grp. gram.:f.}
\end{itemize}
Variedade de figueira, o mesmo que \textunderscore cachopo\textunderscore ^3.
\section{Figueira-brava}
\begin{itemize}
\item {Grp. gram.:f.}
\end{itemize}
Árvore africana, de bôa madeira para construcções, (\textunderscore apodythes dimidiata\textunderscore , E. Mev.), e que não tem nada com o gênero das figueiras.
\section{Figueira-da-berberia}
\begin{itemize}
\item {Grp. gram.:f.}
\end{itemize}
Árvore cactácea, (\textunderscore cactus opuntia\textunderscore ).
\section{Figueira-da-índia}
\begin{itemize}
\item {Grp. gram.:f.}
\end{itemize}
(V.cumbeba)
\section{Figueira-da-ponte}
\begin{itemize}
\item {Grp. gram.:f.}
\end{itemize}
Variedade da figueira algarvia.
\section{Figueira-de-adão}
\begin{itemize}
\item {Grp. gram.:f.}
\end{itemize}
O mesmo que \textunderscore bananeira\textunderscore .
\section{Figueira-do-inferno}
\begin{itemize}
\item {Grp. gram.:f.}
\end{itemize}
O mesmo que \textunderscore estramónio\textunderscore .
\section{Figueira-do-mato}
\begin{itemize}
\item {Grp. gram.:f.}
\end{itemize}
Grande árvore do Brasil, de que se fazem bateiras, gamelas, etc.
\section{Figueira-dos-pagodes}
\begin{itemize}
\item {Grp. gram.:f.}
\end{itemize}
\begin{itemize}
\item {Utilização:T. da Índia port}
\end{itemize}
O mesmo que \textunderscore pimpôl\textunderscore .
\section{Figueiral}
\begin{itemize}
\item {Grp. gram.:m.}
\end{itemize}
Lugar ou terreno, onde crescem figueiras.
\section{Figueira-maldita}
\begin{itemize}
\item {Grp. gram.:f.}
\end{itemize}
Planta clusiácea, (\textunderscore clusia rosea\textunderscore ), da ilha de San-Domingos, e typo das clusiáceas.
\section{Figueiras}
\begin{itemize}
\item {Grp. gram.:f. Pl.}
\end{itemize}
\begin{itemize}
\item {Utilização:Prov.}
\end{itemize}
Espécie de verruga nas bêstas.
\section{Figueiredo}
\begin{itemize}
\item {fónica:gueirê}
\end{itemize}
\begin{itemize}
\item {Grp. gram.:m.}
\end{itemize}
\begin{itemize}
\item {Proveniência:(Do b. lat. \textunderscore ficarietum\textunderscore )}
\end{itemize}
O mesmo que \textunderscore figueiral\textunderscore .
\section{Figueiró}
\begin{itemize}
\item {Grp. gram.:f.}
\end{itemize}
\begin{itemize}
\item {Utilização:Ant.}
\end{itemize}
Figueira pequena.
(Por \textunderscore figueirola\textunderscore , de \textunderscore figueira\textunderscore )
\section{Figueirôa}
\begin{itemize}
\item {Grp. gram.:f.}
\end{itemize}
Variedade de pêra muito apreciada, originária talvez das vizinhanças do Porto.
\section{Figueital}
\begin{itemize}
\item {Grp. gram.:m.}
\end{itemize}
Campo, onde crescem figueitos.
\section{Figueiteira}
\begin{itemize}
\item {Grp. gram.:m.}
\end{itemize}
\begin{itemize}
\item {Utilização:Prov.}
\end{itemize}
\begin{itemize}
\item {Utilização:trasm.}
\end{itemize}
O mesmo que \textunderscore figueital\textunderscore .
\section{Figueito}
\begin{itemize}
\item {Grp. gram.:m.}
\end{itemize}
\begin{itemize}
\item {Utilização:Prov.}
\end{itemize}
\begin{itemize}
\item {Utilização:trasm.}
\end{itemize}
\begin{itemize}
\item {Proveniência:(Do lat. \textunderscore filictum\textunderscore )}
\end{itemize}
O mesmo que \textunderscore fêto\textunderscore .
\section{Figulina}
\begin{itemize}
\item {Grp. gram.:f.}
\end{itemize}
\begin{itemize}
\item {Utilização:Ant.}
\end{itemize}
\begin{itemize}
\item {Proveniência:(De \textunderscore figulino\textunderscore )}
\end{itemize}
Vaso de barro.
\section{Figulino}
\begin{itemize}
\item {Grp. gram.:adj.}
\end{itemize}
\begin{itemize}
\item {Utilização:Fig.}
\end{itemize}
\begin{itemize}
\item {Proveniência:(Lat. figulinus)}
\end{itemize}
Feito de barro.
Que se póde amassar como o barro.
Domesticável; dócil.
\section{Fígulo}
\begin{itemize}
\item {Grp. gram.:m.}
\end{itemize}
\begin{itemize}
\item {Proveniência:(Lat. \textunderscore figulus\textunderscore )}
\end{itemize}
Gênero de insectos coleópteros pentâmeros, da fam. dos lamellicórneos.
\section{Figura}
\begin{itemize}
\item {Grp. gram.:f.}
\end{itemize}
\begin{itemize}
\item {Utilização:Chul.}
\end{itemize}
\begin{itemize}
\item {Grp. gram.:Pl.}
\end{itemize}
\begin{itemize}
\item {Utilização:Heráld.}
\end{itemize}
\begin{itemize}
\item {Proveniência:(Lat. \textunderscore figura\textunderscore )}
\end{itemize}
Aspecto.
Fórma exterior; exterioridade.
Representação.
Importância social.
Pessôa, vulto.
Espaço, terminado por linhas ou superfícies.
Plano de uma construcção.
Carta de jôgo, que tem figura.
Busto ou corpo de pessôa, esculpido, estampado ou desenhado.
Imagem.
Sýmbolo.
Fórma de expressão, em que há permittidas alterações phonéticas, morphológicas ou syntácticas.
Maneira de dizer, que, pela graça ou vivacidade, se afasta da linguagem commum.
Rosto, cara: \textunderscore foi-lhe á figura\textunderscore .
Os objectos que se collocam no campo do escudo e que também se dizem \textunderscore peças\textunderscore  ou \textunderscore móveis\textunderscore .
\section{Figurabilidade}
\begin{itemize}
\item {Grp. gram.:f.}
\end{itemize}
Qualidade de figurável.
\section{Figuração}
\begin{itemize}
\item {Grp. gram.:f.}
\end{itemize}
Figura.
Acto de figurar.
\section{Figuraço}
\begin{itemize}
\item {Grp. gram.:m.}
\end{itemize}
\begin{itemize}
\item {Utilização:Prov.}
\end{itemize}
\begin{itemize}
\item {Utilização:beir.}
\end{itemize}
O mesmo que \textunderscore figurão\textunderscore .
(Na língua oc, \textunderscore figurasso\textunderscore )
\section{Figuradamente}
\begin{itemize}
\item {Grp. gram.:adv.}
\end{itemize}
De modo figurado.
\section{Figurado}
\begin{itemize}
\item {Grp. gram.:adj.}
\end{itemize}
\begin{itemize}
\item {Proveniência:(De \textunderscore figurar\textunderscore )}
\end{itemize}
Em que há figura ou allegoria.
\section{Figural}
\begin{itemize}
\item {Grp. gram.:adj.}
\end{itemize}
Que serve de figura ou de typo.
\section{Figuralidade}
\begin{itemize}
\item {Grp. gram.:f.}
\end{itemize}
\begin{itemize}
\item {Proveniência:(Lat. \textunderscore figuralitas\textunderscore )}
\end{itemize}
Propriedade, que os corpos têm, de tomar tal ou tal figura.
\section{Figuranta}
\begin{itemize}
\item {Grp. gram.:f.}
\end{itemize}
Mulher, que, sem falar, entra numa representação theatral. Cf. Benalcanfor, \textunderscore Cartas de Lisb.\textunderscore , 11.
(Cp. \textunderscore figurante\textunderscore )
\section{Figurante}
\begin{itemize}
\item {Grp. gram.:m.}
\end{itemize}
\begin{itemize}
\item {Proveniência:(Lat. \textunderscore figurans\textunderscore )}
\end{itemize}
Personagem, que entra, sem falar, em representações theatraes.
\section{Figurão}
\begin{itemize}
\item {Grp. gram.:m.}
\end{itemize}
\begin{itemize}
\item {Utilização:Fam.}
\end{itemize}
\begin{itemize}
\item {Utilização:Deprec.}
\end{itemize}
\begin{itemize}
\item {Proveniência:(De \textunderscore figura\textunderscore )}
\end{itemize}
Personagem importante.
Ostentação, acto que dá na vista.
Pessôa extravagante.
Homem manhoso, finório.
\section{Figurar}
\begin{itemize}
\item {Grp. gram.:v. t.}
\end{itemize}
\begin{itemize}
\item {Grp. gram.:V. i.}
\end{itemize}
\begin{itemize}
\item {Proveniência:(Lat. \textunderscore figurare\textunderscore )}
\end{itemize}
Fazer a figura de.
Representar.
Imaginar.
Symbolizar.
Expor allegoricamente.
Suppor.
Têr importância, sêr notável.
Apparecer em scena.
Fazer parte de um conjunto: \textunderscore apparecer numa lista\textunderscore .
Têr apparência do que não é.
\section{Figurárias}
\begin{itemize}
\item {Grp. gram.:f. pl.}
\end{itemize}
\begin{itemize}
\item {Utilização:Des.}
\end{itemize}
\begin{itemize}
\item {Proveniência:(De \textunderscore figura\textunderscore )}
\end{itemize}
Mímica, para divertimento de crianças.
\section{Figurativa}
\begin{itemize}
\item {Grp. gram.:f.}
\end{itemize}
\begin{itemize}
\item {Utilização:Gram.}
\end{itemize}
\begin{itemize}
\item {Proveniência:(Lat. \textunderscore figurativa\textunderscore )}
\end{itemize}
Primeiro suffixo, que designa a classe a que pertencem os vocábulos.
\section{Figurativamente}
\begin{itemize}
\item {Grp. gram.:adv.}
\end{itemize}
De modo figurativo.
\section{Figurativo}
\begin{itemize}
\item {Grp. gram.:adj.}
\end{itemize}
\begin{itemize}
\item {Proveniência:(Lat. \textunderscore figurativus\textunderscore )}
\end{itemize}
Que figura, que representa.
\section{Figurável}
\begin{itemize}
\item {Grp. gram.:adj.}
\end{itemize}
\begin{itemize}
\item {Proveniência:(De \textunderscore figuras\textunderscore )}
\end{itemize}
Que se póde figurar.
\section{Figurelha}
\begin{itemize}
\item {fónica:gurê}
\end{itemize}
\begin{itemize}
\item {Grp. gram.:f.}
\end{itemize}
\begin{itemize}
\item {Proveniência:(De \textunderscore figura\textunderscore )}
\end{itemize}
Figurilha.
Desenho tôsco.
\section{Figurilha}
\begin{itemize}
\item {Grp. gram.:m.  e  f.}
\end{itemize}
Pequena figura.
Fraca figura; bigorrilha.
\section{Figurino}
\begin{itemize}
\item {Grp. gram.:m.}
\end{itemize}
\begin{itemize}
\item {Utilização:Fig.}
\end{itemize}
\begin{itemize}
\item {Proveniência:(De \textunderscore figura\textunderscore )}
\end{itemize}
Figura ou estampa, que representa o traje da moda.
Indivíduo, que traja com affectação, segundo o rigor da moda ou exaggerando a moda.
Modêlo, exemplo.
\section{Figurismo}
\begin{itemize}
\item {Grp. gram.:m.}
\end{itemize}
\begin{itemize}
\item {Proveniência:(De \textunderscore figura\textunderscore )}
\end{itemize}
Systema dos que interpretam allegoricamente os factos referidos na \textunderscore Biblia\textunderscore .
\section{Figurista}
\begin{itemize}
\item {Grp. gram.:m.}
\end{itemize}
\begin{itemize}
\item {Proveniência:(De \textunderscore figura\textunderscore )}
\end{itemize}
Sectário do figurismo.
\section{Figuro}
\begin{itemize}
\item {Grp. gram.:m.}
\end{itemize}
\begin{itemize}
\item {Utilização:Fam.}
\end{itemize}
Sujeito de reputação duvidosa.
Súcio: ratão. Cf. Castilho, \textunderscore Fausto\textunderscore , 173.
\section{Fĩir}
\begin{itemize}
\item {Grp. gram.:v. i.}
\end{itemize}
\begin{itemize}
\item {Utilização:Ant.}
\end{itemize}
\begin{itemize}
\item {Proveniência:(Do lat. \textunderscore finire\textunderscore )}
\end{itemize}
Acabar, findar.
\section{Fila}
\begin{itemize}
\item {Grp. gram.:f.}
\end{itemize}
\begin{itemize}
\item {Proveniência:(Do lat. \textunderscore filum\textunderscore )}
\end{itemize}
Série de coisas, animaes ou pessôas, dispostas em linha recta.
Enfiada; fileira: \textunderscore uma fila de cadeiras\textunderscore .
\section{Fila}
\begin{itemize}
\item {Grp. gram.:f.}
\end{itemize}
\begin{itemize}
\item {Grp. gram.:M.}
\end{itemize}
\begin{itemize}
\item {Utilização:Gír.}
\end{itemize}
Acto de filar.
Official de justiça.
\section{Fila}
\begin{itemize}
\item {Grp. gram.:f.}
\end{itemize}
\begin{itemize}
\item {Utilização:Gír.}
\end{itemize}
Cara, fela.
\section{Filaça}
\begin{itemize}
\item {Grp. gram.:f.}
\end{itemize}
\begin{itemize}
\item {Proveniência:(Do lat. \textunderscore filum\textunderscore )}
\end{itemize}
Filamento de substância têxtil.
\section{Fila-fila}
\begin{itemize}
\item {Grp. gram.:f.}
\end{itemize}
Variedade de gallináceas, (\textunderscore streptopelia semitorquata\textunderscore ).
\section{Filagrana}
\begin{itemize}
\item {Grp. gram.:f.}
\end{itemize}
(V.filigrana)
\section{Filali}
\begin{itemize}
\item {Grp. gram.:m.}
\end{itemize}
O mesmo que \textunderscore filele\textunderscore .
\section{Filame}
\begin{itemize}
\item {Grp. gram.:m.}
\end{itemize}
\begin{itemize}
\item {Utilização:Náut.}
\end{itemize}
\begin{itemize}
\item {Proveniência:(Lat. \textunderscore filamen\textunderscore )}
\end{itemize}
Espaço da amarra, entre o anete da âncora e o travessão da abita.
\section{Filamentar}
\begin{itemize}
\item {Grp. gram.:adj.}
\end{itemize}
Constituído por filamentos.
\section{Filamento}
\begin{itemize}
\item {Grp. gram.:m.}
\end{itemize}
\begin{itemize}
\item {Proveniência:(Lat. \textunderscore filamentum\textunderscore )}
\end{itemize}
Fios tênues da raiz das plantas.
Fibra.
Fio, que alguns mineraes apresentam na sua textura.
\section{Filamentoso}
\begin{itemize}
\item {Grp. gram.:adj.}
\end{itemize}
\begin{itemize}
\item {Proveniência:(De \textunderscore filamento\textunderscore )}
\end{itemize}
O mesmo que \textunderscore filamentar\textunderscore .
\section{Filandras}
\begin{itemize}
\item {Grp. gram.:f. pl.}
\end{itemize}
\begin{itemize}
\item {Proveniência:(Do rad. do lat. \textunderscore filum\textunderscore )}
\end{itemize}
Fios.
Vermes intestinaes de algumas aves.
Fios, que apparecem nas chagas do gado cavallar.
Ervas marítimas, que adherem á quilha do navio.
\section{Filandroso}
\begin{itemize}
\item {Grp. gram.:adj.}
\end{itemize}
Que tem filandras ou nervuras.
Fibroso.
\section{Filanete}
\begin{itemize}
\item {fónica:nê}
\end{itemize}
\begin{itemize}
\item {Grp. gram.:m.}
\end{itemize}
Pequeno filão, filete:«\textunderscore ...uma tampa lavrada e dourada com filanetes.\textunderscore »\textunderscore Ms.\textunderscore  do sec. XVI.
\section{Filante}
\begin{itemize}
\item {Grp. gram.:m.  e  f.}
\end{itemize}
\begin{itemize}
\item {Utilização:Bras}
\end{itemize}
\begin{itemize}
\item {Utilização:Gír.}
\end{itemize}
\begin{itemize}
\item {Proveniência:(De \textunderscore filar\textunderscore )}
\end{itemize}
Pessôa, que procura obter as coisas sem gastar dinheiro.
Agente de polícia, guarda de segurança.
\section{Filante}
\begin{itemize}
\item {Grp. gram.:adj.}
\end{itemize}
\begin{itemize}
\item {Proveniência:(Lat. \textunderscore filans\textunderscore )}
\end{itemize}
Diz-se do vinho deteriorado, quando toma certa espessura como a do mel.
\section{Filantro}
\begin{itemize}
\item {Grp. gram.:m.}
\end{itemize}
\begin{itemize}
\item {Utilização:Bras. do Rio}
\end{itemize}
Espécie de flôr dos jardins.
\section{Filão}
\begin{itemize}
\item {Grp. gram.:m.}
\end{itemize}
\begin{itemize}
\item {Utilização:Neol.}
\end{itemize}
\begin{itemize}
\item {Proveniência:(Fr. \textunderscore filon\textunderscore , do lat. \textunderscore filum\textunderscore )}
\end{itemize}
Veio de metal nas minas.--É t. de duvidosa vernaculidade.
\section{Filar}
\begin{itemize}
\item {Grp. gram.:v. t.}
\end{itemize}
\begin{itemize}
\item {Utilização:Bras. do N}
\end{itemize}
\begin{itemize}
\item {Proveniência:(Lat. des. \textunderscore filare\textunderscore )}
\end{itemize}
Prender.
Agarrar á fôrça.
Segurar com os dentes.
Aproar (a embarcação) ao vento.
Obter gratuitamente; pedir.
\section{Filária}
\begin{itemize}
\item {Grp. gram.:f.}
\end{itemize}
\begin{itemize}
\item {Proveniência:(Lat. \textunderscore filaria\textunderscore )}
\end{itemize}
Gênero de vermes nematoides, filiformes e alongados, que têm a sua séde no crystallino, nos brônchios e noutros órgãos, ignorando-se se entram pela pelle no organismo dos indivíduos que andam descalços, ou se êsses vermes são ingeridos com a água potável.
\section{Filariose}
\begin{itemize}
\item {Grp. gram.:f.}
\end{itemize}
\begin{itemize}
\item {Utilização:Neol.}
\end{itemize}
\begin{itemize}
\item {Proveniência:(T. inventado em 1875 pelo médico bras. Silva Araújo e der. de \textunderscore filária\textunderscore )}
\end{itemize}
Elephantíase, produzida pelo parasitismo da filária.
\section{Filariósico}
\begin{itemize}
\item {Grp. gram.:adj.}
\end{itemize}
Relativo á filariose.
\section{Filástica}
\begin{itemize}
\item {Grp. gram.:f.}
\end{itemize}
\begin{itemize}
\item {Proveniência:(Do rad. do lat. \textunderscore filum\textunderscore )}
\end{itemize}
Filamento dos cabos destorcidos.
\section{Filateria}
\begin{itemize}
\item {Grp. gram.:f.}
\end{itemize}
\begin{itemize}
\item {Utilização:Ant.}
\end{itemize}
Bazófia, jactância. Cf. Pant. de Aveiro, \textunderscore Itiner.\textunderscore , 160 v.^o, (2.^a ed.).
\section{Filatório}
\begin{itemize}
\item {Grp. gram.:m.}
\end{itemize}
\begin{itemize}
\item {Proveniência:(Do rad. do lat. \textunderscore filum\textunderscore )}
\end{itemize}
Apparelho para fiação.
\section{Filé}
\begin{itemize}
\item {Grp. gram.:m.}
\end{itemize}
\begin{itemize}
\item {Utilização:Gír.}
\end{itemize}
Grande desejo ou empenho.
Palpite, esperança, em certo ganho.
O mesmo que \textunderscore filete\textunderscore .
\section{Fileira}
\begin{itemize}
\item {Grp. gram.:f.}
\end{itemize}
\begin{itemize}
\item {Proveniência:(De \textunderscore fila\textunderscore ^1)}
\end{itemize}
Série de coisas, animaes ou pessôas, em linha recta.
Linha; ala.
\textunderscore Pau de fileira\textunderscore , cumeeira, a parte mais alta de um edifício, na qual se apoia a extremidade superior dos caibros.
\section{Filele}
\begin{itemize}
\item {Grp. gram.:m.}
\end{itemize}
Tecido especial de várias côres, próprio para fabrico e consêrto de bandeiras e galhardetes.
(Cp. cast. \textunderscore fileli\textunderscore )
\section{Filerete}
\begin{itemize}
\item {fónica:lerê}
\end{itemize}
\begin{itemize}
\item {Grp. gram.:m.}
\end{itemize}
\begin{itemize}
\item {Grp. gram.:Pl.}
\end{itemize}
\begin{itemize}
\item {Proveniência:(Do rad. do lat. \textunderscore filum\textunderscore )}
\end{itemize}
Espécie de junteira.
Rêdes, em que se metem sacos de algodão, cortiça, etc., e com que se guarnecem os bordos do navio contra as balas dos inimigos.
\section{Filete}
\begin{itemize}
\item {Grp. gram.:m.}
\end{itemize}
\begin{itemize}
\item {Utilização:Typ.}
\end{itemize}
\begin{itemize}
\item {Utilização:T. de jôgo}
\end{itemize}
\begin{itemize}
\item {Grp. gram.:Pl.}
\end{itemize}
\begin{itemize}
\item {Utilização:Prov.}
\end{itemize}
\begin{itemize}
\item {Utilização:trasm.}
\end{itemize}
\begin{itemize}
\item {Utilização:Heráld.}
\end{itemize}
\begin{itemize}
\item {Proveniência:(Fr. \textunderscore filet\textunderscore )}
\end{itemize}
Fiozinho.
Guarnição estreita.
Parte do estame, em que se apoia a anthera, quando esta não é rente.
Espiral de parafuso.
Cada uma das ramificações mais tênues dos nervos.
Linha grossa e escura, nos brasões de bastardia.
Posta delgada e frita de carne ou peixe.
Linha de ornato.
\textunderscore Fazer filete\textunderscore , não puxar carta superior, ou não entrar com ella, esperando fazer depois melhor vasa.
Esgares, fósquinhas, pantomimas.
Peças honrosas, reduzidas a um terço ou menos da sua largura.
\section{Filha}
(\textunderscore fem.\textunderscore  de \textunderscore filho\textunderscore )
\section{Filha}
\begin{itemize}
\item {Grp. gram.:f.}
\end{itemize}
O mesmo que \textunderscore filhada\textunderscore .
\section{Filha}
\begin{itemize}
\item {Grp. gram.:f.}
\end{itemize}
\begin{itemize}
\item {Utilização:Ant.}
\end{itemize}
O mesmo que \textunderscore fila\textunderscore ^1 ou \textunderscore fileira\textunderscore .
\section{Filhação}
\begin{itemize}
\item {Grp. gram.:f.}
\end{itemize}
(V.filiação)
\section{Filhada}
\begin{itemize}
\item {Grp. gram.:f.}
\end{itemize}
\begin{itemize}
\item {Utilização:Ant.}
\end{itemize}
Acto de \textunderscore filhar\textunderscore ^2.
Tomadia de terras maninhas ou incultas.
\section{Filhadoiro}
\begin{itemize}
\item {Grp. gram.:adj.}
\end{itemize}
\begin{itemize}
\item {Utilização:Ant.}
\end{itemize}
\begin{itemize}
\item {Proveniência:(De \textunderscore filhar\textunderscore ^2)}
\end{itemize}
Que se póde apanhar, que ja está sazonado, (falando-se de frutos).
\section{Filhador}
\begin{itemize}
\item {Grp. gram.:m.}
\end{itemize}
\begin{itemize}
\item {Proveniência:(De \textunderscore filhar\textunderscore )}
\end{itemize}
Aquelle que perfilha.
\section{Filhadouro}
\begin{itemize}
\item {Grp. gram.:adj.}
\end{itemize}
\begin{itemize}
\item {Utilização:Ant.}
\end{itemize}
\begin{itemize}
\item {Proveniência:(De \textunderscore filhar\textunderscore ^2)}
\end{itemize}
Que se póde apanhar, que ja está sazonado, (falando-se de frutos).
\section{Filhamento}
\begin{itemize}
\item {Grp. gram.:m.}
\end{itemize}
Acto de filhar^1.
\section{Filhamento}
\begin{itemize}
\item {Grp. gram.:m.}
\end{itemize}
O mesmo que \textunderscore filhada\textunderscore .
\section{Filhar}
\begin{itemize}
\item {Grp. gram.:v. t.}
\end{itemize}
\begin{itemize}
\item {Utilização:Des.}
\end{itemize}
\begin{itemize}
\item {Grp. gram.:V. i.}
\end{itemize}
\begin{itemize}
\item {Proveniência:(De \textunderscore filho\textunderscore )}
\end{itemize}
O mesmo que \textunderscore perfilhar\textunderscore .
Tomar em fôro de fidalgo.
Deitar rebentos, brotar, (falando-se de plantas).
\section{Filhar}
\begin{itemize}
\item {Grp. gram.:v. t.}
\end{itemize}
\begin{itemize}
\item {Utilização:Ant.}
\end{itemize}
\begin{itemize}
\item {Utilização:Ant.}
\end{itemize}
\begin{itemize}
\item {Proveniência:(Do lat. \textunderscore filare\textunderscore )}
\end{itemize}
Agarrar á fôrça, filar, violentar.
Receber, tomar conta de (terrenos maninhos).
Apanhar, colher.
\section{Filharada}
\begin{itemize}
\item {Grp. gram.:f.}
\end{itemize}
\begin{itemize}
\item {Proveniência:(De \textunderscore filharar\textunderscore )}
\end{itemize}
Conjunto de muitos filhos.
\section{Filharar}
\begin{itemize}
\item {Grp. gram.:v. i.}
\end{itemize}
\begin{itemize}
\item {Utilização:Bot.}
\end{itemize}
\begin{itemize}
\item {Proveniência:(De \textunderscore filho\textunderscore )}
\end{itemize}
Deitar filhos ou rebentos (a planta).
\section{Filharasco}
\begin{itemize}
\item {Grp. gram.:m.}
\end{itemize}
\begin{itemize}
\item {Utilização:Prov.}
\end{itemize}
O mesmo que \textunderscore filhastro\textunderscore .
\section{Filhastrar}
\begin{itemize}
\item {Grp. gram.:v. i.}
\end{itemize}
\begin{itemize}
\item {Utilização:Prov.}
\end{itemize}
\begin{itemize}
\item {Utilização:trasm.}
\end{itemize}
Comprehender, perceber.
(Relaciona-se com o cast. \textunderscore hijastro\textunderscore ?)
\section{Filhastro}
\begin{itemize}
\item {Grp. gram.:m.}
\end{itemize}
\begin{itemize}
\item {Utilização:Prov.}
\end{itemize}
\begin{itemize}
\item {Proveniência:(Do lat. \textunderscore filiaster\textunderscore )}
\end{itemize}
O mesmo que \textunderscore enteado\textunderscore .
\section{Filheiro}
\begin{itemize}
\item {Grp. gram.:adj.}
\end{itemize}
\begin{itemize}
\item {Utilização:Prov.}
\end{itemize}
\begin{itemize}
\item {Utilização:trasm.}
\end{itemize}
Que gera muitos filhos.
Muito amigo dos filhos.
\section{Filhento}
\begin{itemize}
\item {Grp. gram.:adj.}
\end{itemize}
Que gera muitos filhos; fecundo.
\section{Filho}
\begin{itemize}
\item {Grp. gram.:m.}
\end{itemize}
\begin{itemize}
\item {Grp. gram.:Adj.}
\end{itemize}
\begin{itemize}
\item {Proveniência:(Lat. \textunderscore filius\textunderscore )}
\end{itemize}
Indivíduo do sexo masculino, em relação ao pai e á mãi.
Prole masculina.
Descendente.
Oriundo: \textunderscore filho do Oriente\textunderscore .
Natural: \textunderscore filho de Portugal\textunderscore .
Indivíduo, em relação ao estabelecimento em que foi educado, ao indivíduo que o educou ou á communidade de que fez parte.
Rebento da planta.
O homem, em relação a Deus.
Expressão de carinho.
Procedente; resultante.
\section{Filhó}
\begin{itemize}
\item {Grp. gram.:m.}
\end{itemize}
\begin{itemize}
\item {Proveniência:(Do lat. hyp. \textunderscore folleola\textunderscore , de \textunderscore follis\textunderscore ?)}
\end{itemize}
Bolo de farinha e ovos, frito em azeite, e ordinariamente passado por calda de açúcar.
\section{Filho-família}
\begin{itemize}
\item {Grp. gram.:m.}
\end{itemize}
\begin{itemize}
\item {Utilização:Pop.}
\end{itemize}
Indivíduo de menor idade, ainda sujeito ao pátrio poder: \textunderscore as más companhias são a perdição dos filhos-famílias\textunderscore .
\section{Filhós}
\begin{itemize}
\item {Grp. gram.:f.}
\end{itemize}
\begin{itemize}
\item {Utilização:Gír.}
\end{itemize}
Nota de banco.
(Cp. \textunderscore filhó\textunderscore )
\section{Filhota}
\begin{itemize}
\item {Grp. gram.:f.}
\end{itemize}
Antiga dança e música campestre, em compasso ternário, e semelhante ao fandango.
(Talvez corr. do it. \textunderscore villotta\textunderscore , aldean, camponesa)
\section{Filhote}
\begin{itemize}
\item {Grp. gram.:m.}
\end{itemize}
\begin{itemize}
\item {Utilização:Bras}
\end{itemize}
\begin{itemize}
\item {Proveniência:(De \textunderscore filho\textunderscore )}
\end{itemize}
Aquelle que é natural de (uma localidade): \textunderscore os filhotes de Coimbra\textunderscore .
Peixe grande do Brasil.
Indivíduo, altamente protegido, com mais ou menos escândalo.
\section{Filhotinho}
\begin{itemize}
\item {Grp. gram.:m.}
\end{itemize}
\begin{itemize}
\item {Utilização:Bras}
\end{itemize}
Peixe grande, o mesmo que \textunderscore filhote\textunderscore .
\section{Filhotismo}
\begin{itemize}
\item {Grp. gram.:m.}
\end{itemize}
\begin{itemize}
\item {Utilização:Bras}
\end{itemize}
\begin{itemize}
\item {Utilização:Bras}
\end{itemize}
Qualidade de filhote, enthusiasmo de filhote.
Apadrinhamento ou patronato, mais ou menos escandaloso.
\section{Filiação}
\begin{itemize}
\item {Grp. gram.:f.}
\end{itemize}
\begin{itemize}
\item {Proveniência:(Lat. \textunderscore filiatio\textunderscore )}
\end{itemize}
Acto de perfilhar.
Descendência de pais para filhos.
Designação dos pais de alguém: \textunderscore diga a sua filiação\textunderscore .
Admissão numa communidade.
Connexão, dependência.
\section{Filial}
\begin{itemize}
\item {Grp. gram.:adj.}
\end{itemize}
\begin{itemize}
\item {Grp. gram.:F.}
\end{itemize}
\begin{itemize}
\item {Proveniência:(Lat. \textunderscore filialis\textunderscore )}
\end{itemize}
Próprio de filho: \textunderscore amor filial\textunderscore .
Relativo a filho.
Que tem filiação.
Succursal: \textunderscore estabelecimento filial\textunderscore .
Estabelecimento succursal ou dependente de outro.
\section{Filialmente}
\begin{itemize}
\item {Grp. gram.:adv.}
\end{itemize}
De modo filial.
\section{Filiar}
\begin{itemize}
\item {Grp. gram.:v. t.}
\end{itemize}
\begin{itemize}
\item {Grp. gram.:V. p.}
\end{itemize}
\begin{itemize}
\item {Proveniência:(Do b. lat. \textunderscore filiare\textunderscore )}
\end{itemize}
Adoptar como filho.
Entroncar.
Admittir numa communidade.
Proceder, originar-se.
Entrar num agrupamento ou numa corporação: \textunderscore filiar-se num partido\textunderscore .
\section{Filíceas}
\begin{itemize}
\item {Grp. gram.:f. pl.}
\end{itemize}
\begin{itemize}
\item {Proveniência:(Do lat. \textunderscore filix\textunderscore )}
\end{itemize}
O mesmo que \textunderscore fissidentadas\textunderscore .
\section{Filicida}
\begin{itemize}
\item {Grp. gram.:m.  e  f.}
\end{itemize}
Pessôa, que mata o próprio filho.
(Cp. \textunderscore filicídio\textunderscore )
\section{Filicídio}
\begin{itemize}
\item {Grp. gram.:m.}
\end{itemize}
\begin{itemize}
\item {Proveniência:(Do lat. \textunderscore filius\textunderscore  + \textunderscore caedere\textunderscore )}
\end{itemize}
Acto de matar o próprio filho.
\section{Filicífero}
\begin{itemize}
\item {Grp. gram.:adj.}
\end{itemize}
\begin{itemize}
\item {Proveniência:(Do lat. \textunderscore filix\textunderscore  + \textunderscore ferre\textunderscore )}
\end{itemize}
Que contém fêtos fósseis ou vestígios de fêtos.
\section{Filicite}
\begin{itemize}
\item {Grp. gram.:f.}
\end{itemize}
\begin{itemize}
\item {Proveniência:(Do lat. \textunderscore filix\textunderscore )}
\end{itemize}
Fêto fóssil.
\section{Filicorne}
\begin{itemize}
\item {Grp. gram.:adj.}
\end{itemize}
\begin{itemize}
\item {Grp. gram.:M. pl.}
\end{itemize}
\begin{itemize}
\item {Proveniência:(Do lat. \textunderscore filium\textunderscore  + \textunderscore cornu\textunderscore )}
\end{itemize}
Que tem antennas semelhante a cornos.
Nome de três famílias de insectos.
\section{Filicórneo}
\begin{itemize}
\item {Grp. gram.:adj.}
\end{itemize}
\begin{itemize}
\item {Utilização:Zool.}
\end{itemize}
\begin{itemize}
\item {Grp. gram.:M. pl.}
\end{itemize}
\begin{itemize}
\item {Proveniência:(Do lat. \textunderscore filium\textunderscore  + \textunderscore cornu\textunderscore )}
\end{itemize}
Que tem antennas semelhante a cornos.
Nome de três famílias de insectos.
\section{Filífero}
\begin{itemize}
\item {Grp. gram.:adj.}
\end{itemize}
\begin{itemize}
\item {Proveniência:(Do lat. \textunderscore filium\textunderscore  + \textunderscore ferre\textunderscore )}
\end{itemize}
Que tem filamentos.
\section{Filífero}
\begin{itemize}
\item {Grp. gram.:adj.}
\end{itemize}
O mesmo que \textunderscore filicífero\textunderscore .
\section{Filifolha}
\begin{itemize}
\item {Grp. gram.:f.}
\end{itemize}
\begin{itemize}
\item {Proveniência:(Do lat. \textunderscore filix\textunderscore  + \textunderscore filium\textunderscore )}
\end{itemize}
O mesmo que \textunderscore fêto\textunderscore .
\section{Filiforme}
\begin{itemize}
\item {Grp. gram.:adj.}
\end{itemize}
\begin{itemize}
\item {Proveniência:(Do lat. \textunderscore filium\textunderscore  + \textunderscore forma\textunderscore )}
\end{itemize}
Delgado como um fio.
Débil, fraco, (falando-se do pulso, em que mal se distinguem as pulsações).
\section{Filigrana}
\begin{itemize}
\item {Grp. gram.:f.}
\end{itemize}
\begin{itemize}
\item {Proveniência:(It. \textunderscore filigrana\textunderscore )}
\end{itemize}
Obra de ourivezaria, formada de fios de oiro ou de prata, delicadamente entrelaçados e soldados.
\section{Filigranar}
\begin{itemize}
\item {Grp. gram.:v. t.}
\end{itemize}
\begin{itemize}
\item {Utilização:Fig.}
\end{itemize}
\begin{itemize}
\item {Grp. gram.:V. i.}
\end{itemize}
\begin{itemize}
\item {Proveniência:(De \textunderscore filigrana\textunderscore )}
\end{itemize}
Fazer (qualquer trabalho delicado ou delicadamente artístico).
Fazer filigrana.
Fazer trabalhos delicados. Cf. A. Mendes, \textunderscore Plágios\textunderscore , 60.
\section{Filintino}
\begin{itemize}
\item {Grp. gram.:adj.}
\end{itemize}
Relativo ao poéta Filinto Elýsio.
\section{Filintista}
\begin{itemize}
\item {Grp. gram.:m.}
\end{itemize}
Admirador de Filinto ou imitador dos seus processos literários.
\section{Filipêndula}
\begin{itemize}
\item {Grp. gram.:f.}
\end{itemize}
\begin{itemize}
\item {Proveniência:(Lat. \textunderscore filipendula\textunderscore )}
\end{itemize}
Planta medicinal, da fam. das rosáceas.
\section{Filipendulado}
\begin{itemize}
\item {Grp. gram.:adj.}
\end{itemize}
Suspenso ou ligado por fios, como a filipêndula.
\section{Filipina}
\begin{itemize}
\item {Grp. gram.:f.}
\end{itemize}
\begin{itemize}
\item {Utilização:Des.}
\end{itemize}
Bebida, feita de água, açúcar e aguardente.
(Talvez de \textunderscore Filipe\textunderscore , n. p. de um botequineiro, no Largo do Pelourinho, em Lisbôa)
\section{Filipino}
\begin{itemize}
\item {Grp. gram.:adj.}
\end{itemize}
\begin{itemize}
\item {Grp. gram.:M.}
\end{itemize}
Relativo ás ilhas Filipinas.
Indivíduo, natural das Filipinas.
\section{Filipino}
\begin{itemize}
\item {Grp. gram.:adj.}
\end{itemize}
Diz-se do govêrno e da dinastia dos Filipes, em Portugal.
\section{Filipista}
\begin{itemize}
\item {Grp. gram.:m.  e  adj.}
\end{itemize}
Partidário dos reis Filipes, em Portugal.
\section{Filipluma}
\begin{itemize}
\item {Grp. gram.:f.}
\end{itemize}
\begin{itemize}
\item {Utilização:Hist. Nat.}
\end{itemize}
\begin{itemize}
\item {Proveniência:(Do lat. \textunderscore filum\textunderscore  + \textunderscore pluma\textunderscore )}
\end{itemize}
Penna de ave, formada por haste delgada, piliforme, de barbas atrophiadas ou sem ellas.
\section{Filippina}
\begin{itemize}
\item {Grp. gram.:f.}
\end{itemize}
\begin{itemize}
\item {Utilização:Des.}
\end{itemize}
Bebida, feita de água, açúcar e aguardente.
(Talvez de \textunderscore Filippe\textunderscore , n. p. de um botequineiro, no Largo do Pelourinho, em Lisbôa)
\section{Filippino}
\begin{itemize}
\item {Grp. gram.:adj.}
\end{itemize}
\begin{itemize}
\item {Grp. gram.:M.}
\end{itemize}
Relativo ás ilhas Filippinas.
Indivíduo, natural das Filippinas.
\section{Filippino}
\begin{itemize}
\item {Grp. gram.:adj.}
\end{itemize}
Diz-se do govêrno e da dynastia dos Filippes, em Portugal.
\section{Filippista}
\begin{itemize}
\item {Grp. gram.:m.  e  adj.}
\end{itemize}
Partidário dos reis Filippes, em Portugal.
\section{Filirostro}
\begin{itemize}
\item {fónica:rós}
\end{itemize}
\begin{itemize}
\item {Grp. gram.:adj.}
\end{itemize}
\begin{itemize}
\item {Proveniência:(Do lat. \textunderscore filum\textunderscore  + \textunderscore rostrum\textunderscore )}
\end{itemize}
Diz-se das aves, que têm o bico adelgaçado.
\section{Filirrostro}
\begin{itemize}
\item {Grp. gram.:adj.}
\end{itemize}
\begin{itemize}
\item {Proveniência:(Do lat. \textunderscore filum\textunderscore  + \textunderscore rostrum\textunderscore )}
\end{itemize}
Diz-se das aves, que têm o bico adelgaçado.
\section{Filistria}
\begin{itemize}
\item {Grp. gram.:f.}
\end{itemize}
\begin{itemize}
\item {Utilização:Pop.}
\end{itemize}
(V.flostria)
\section{Filite}
\begin{itemize}
\item {Grp. gram.:m.}
\end{itemize}
Ornato, que cinge várias bôcas de fogo. Cf. Leoni, \textunderscore Diccion. de Artilh.\textunderscore , inédito.
(Alter. de \textunderscore filete\textunderscore )
\section{Filmogênio}
\begin{itemize}
\item {Grp. gram.:m.}
\end{itemize}
Solução de nitro-cellulose em acetona.
\section{Filó}
\begin{itemize}
\item {Grp. gram.:m.}
\end{itemize}
\begin{itemize}
\item {Proveniência:(Do rad. do lat. \textunderscore filum\textunderscore )}
\end{itemize}
Tecido aberto e fino; espécie de cassa.
\section{Filopichim}
\begin{itemize}
\item {Grp. gram.:m.}
\end{itemize}
Espécie de tecido antigo:«\textunderscore deixo uma saia de filopichim\textunderscore ». (De um testamento de 1691)
\section{Filopluma}
\begin{itemize}
\item {Grp. gram.:f.}
\end{itemize}
O mesmo que \textunderscore filipluma\textunderscore .
\section{Filosela}
\begin{itemize}
\item {Grp. gram.:f.}
\end{itemize}
\begin{itemize}
\item {Proveniência:(It. \textunderscore filosella\textunderscore )}
\end{itemize}
Filaça de seda.
Fio tenuíssimo de seda, pouco torcido.
\section{Filosella}
\begin{itemize}
\item {Grp. gram.:f.}
\end{itemize}
\begin{itemize}
\item {Proveniência:(It. \textunderscore filosella\textunderscore )}
\end{itemize}
Filaça de seda.
Fio tenuíssimo de seda, pouco torcido.
\section{Filtração}
\begin{itemize}
\item {Grp. gram.:f.}
\end{itemize}
Acto de filtrar.
\section{Filtrador}
\begin{itemize}
\item {Grp. gram.:m.  e  adj.}
\end{itemize}
Aquelle ou aquillo que filtra.
\section{Filtramento}
\begin{itemize}
\item {Grp. gram.:m.}
\end{itemize}
O mesmo que \textunderscore filtração\textunderscore . Cf. Camillo, \textunderscore Noites de Insómn.\textunderscore , I, 19.
\section{Filtrar}
\begin{itemize}
\item {Grp. gram.:v. t.}
\end{itemize}
\begin{itemize}
\item {Utilização:Fig.}
\end{itemize}
\begin{itemize}
\item {Proveniência:(De \textunderscore filtro\textunderscore ^1)}
\end{itemize}
Fazer passar por filtro.
Coar; escoar.
Inocular ou instillar no ânimo de alguém.
\section{Filtreiro}
\begin{itemize}
\item {Grp. gram.:m.}
\end{itemize}
\begin{itemize}
\item {Proveniência:(De \textunderscore filtrar\textunderscore )}
\end{itemize}
O mesmo que \textunderscore filtro\textunderscore .
\section{Filtro}
\begin{itemize}
\item {Grp. gram.:m.}
\end{itemize}
\begin{itemize}
\item {Utilização:Fig.}
\end{itemize}
\begin{itemize}
\item {Proveniência:(Lat. \textunderscore philtrum\textunderscore )}
\end{itemize}
Matéria porosa, ou apparelho que contém uma substância porosa, para se depurar ou clarificar o líquido que por ella se côa.
Órgãos secretores dos humores do sangue.
Beberagem, que se suppunha despertar o amor em quem a tomava.
\section{Fim}
\begin{itemize}
\item {Grp. gram.:m.}
\end{itemize}
\begin{itemize}
\item {Proveniência:(Do lat. \textunderscore finis\textunderscore )}
\end{itemize}
Remate.
Termo: \textunderscore no fim da cidade\textunderscore .
Intenção.
Escopo, alvo: \textunderscore com o fim de o enganar\textunderscore .
Motivo.
Morte.--Era t. fem. entre os clássicos, e ainda o é entre o povo do Doiro: \textunderscore pensar na fim do mundo\textunderscore .
\section{Fimbo}
\begin{itemize}
\item {Grp. gram.:m.}
\end{itemize}
Pau tostado, usado como arma entre os Cafres.
\section{Fimbrado}
\begin{itemize}
\item {Grp. gram.:adj.}
\end{itemize}
(V.fimbriado)
\section{Fímbria}
\begin{itemize}
\item {Grp. gram.:f.}
\end{itemize}
\begin{itemize}
\item {Proveniência:(Lat. \textunderscore fimbria\textunderscore )}
\end{itemize}
Franja; orla; guarnição.
\section{Fimbriado}
\begin{itemize}
\item {Grp. gram.:adj.}
\end{itemize}
\begin{itemize}
\item {Proveniência:(Lat. \textunderscore fimbriatus\textunderscore )}
\end{itemize}
Que tem fímbria.
Franjado.
\section{Fimbrilas}
\begin{itemize}
\item {Grp. gram.:f. pl.}
\end{itemize}
\begin{itemize}
\item {Utilização:Bot.}
\end{itemize}
\begin{itemize}
\item {Proveniência:(De \textunderscore fímbria\textunderscore , se não é corr. de \textunderscore fibrilas\textunderscore )}
\end{itemize}
Apêndices do clinanto, que são filetes membranosos. Cf. Benevides, \textunderscore Glossologia\textunderscore .
\section{Fimbrillas}
\begin{itemize}
\item {Grp. gram.:f. pl.}
\end{itemize}
\begin{itemize}
\item {Utilização:Bot.}
\end{itemize}
\begin{itemize}
\item {Proveniência:(De \textunderscore fímbria\textunderscore , se não é corr. de \textunderscore fibrillas\textunderscore )}
\end{itemize}
Appêndices do clinantho, que são filetes membranosos. Cf. Benevides, \textunderscore Glossologia\textunderscore .
\section{Fimento}
\begin{itemize}
\item {Grp. gram.:m.}
\end{itemize}
\begin{itemize}
\item {Utilização:Ant.}
\end{itemize}
O mesmo que \textunderscore affimento\textunderscore .
\section{Fimícola}
\begin{itemize}
\item {Grp. gram.:adj.}
\end{itemize}
\begin{itemize}
\item {Proveniência:(Do lat. \textunderscore fimus\textunderscore  + \textunderscore colere\textunderscore )}
\end{itemize}
Que vive no estêrco.
\section{Fina}
\begin{itemize}
\item {Grp. gram.:f.}
\end{itemize}
\begin{itemize}
\item {Utilização:Gír.}
\end{itemize}
\begin{itemize}
\item {Grp. gram.:Loc.}
\end{itemize}
\begin{itemize}
\item {Utilização:fam.}
\end{itemize}
\begin{itemize}
\item {Utilização:Prov.}
\end{itemize}
\begin{itemize}
\item {Utilização:trasm.}
\end{itemize}
\begin{itemize}
\item {Proveniência:(De \textunderscore fino\textunderscore )}
\end{itemize}
Astúcia, finura.
\textunderscore Dar na fina\textunderscore , acertar, têr sorte. Cf. O'Neill, \textunderscore Fabulário\textunderscore , 190.
Precaução.
\section{Finado}
\begin{itemize}
\item {Grp. gram.:m.}
\end{itemize}
\begin{itemize}
\item {Proveniência:(De \textunderscore finar-se\textunderscore )}
\end{itemize}
Indivíduo que falleceu; defunto.
\section{Finadoiro}
\begin{itemize}
\item {Grp. gram.:m.}
\end{itemize}
\begin{itemize}
\item {Utilização:Prov.}
\end{itemize}
\begin{itemize}
\item {Proveniência:(De \textunderscore finar\textunderscore )}
\end{itemize}
Sensação de debilidade, por falta de alimento.
\section{Finadouro}
\begin{itemize}
\item {Grp. gram.:m.}
\end{itemize}
\begin{itemize}
\item {Utilização:Prov.}
\end{itemize}
\begin{itemize}
\item {Proveniência:(De \textunderscore finar\textunderscore )}
\end{itemize}
Sensação de debilidade, por falta de alimento.
\section{Final}
\begin{itemize}
\item {Grp. gram.:adj.}
\end{itemize}
\begin{itemize}
\item {Grp. gram.:M.}
\end{itemize}
\begin{itemize}
\item {Proveniência:(Lat. \textunderscore finalis\textunderscore )}
\end{itemize}
Relativo ao fim; derradeiro.
Que põe termo.
Fim: \textunderscore no final das contas\textunderscore .
\section{Finalidade}
\begin{itemize}
\item {Grp. gram.:f.}
\end{itemize}
\begin{itemize}
\item {Proveniência:(Lat. \textunderscore finalitas\textunderscore )}
\end{itemize}
Systema philosóphico, que a tudo attribue um fim determinado.
\section{Finalismo}
\begin{itemize}
\item {Grp. gram.:m.}
\end{itemize}
Systema dos finalistas.
\section{Finalista}
\begin{itemize}
\item {Grp. gram.:m.}
\end{itemize}
\begin{itemize}
\item {Proveniência:(De \textunderscore final\textunderscore )}
\end{itemize}
Sectário da finalidade.
\section{Finalização}
\begin{itemize}
\item {Grp. gram.:f.}
\end{itemize}
Acto ou effeito de finalizar.
\section{Finalizar}
\begin{itemize}
\item {Grp. gram.:v. t.}
\end{itemize}
\begin{itemize}
\item {Grp. gram.:V. i.}
\end{itemize}
\begin{itemize}
\item {Proveniência:(De \textunderscore final\textunderscore )}
\end{itemize}
Pôr fim a.
Terminar, concluir.
Acabar, têr fim.
\section{Finalmente}
\begin{itemize}
\item {Grp. gram.:adv.}
\end{itemize}
\begin{itemize}
\item {Proveniência:(De \textunderscore final\textunderscore )}
\end{itemize}
Afinal.
Por fim; em conclusão.
\section{Finamente}
\begin{itemize}
\item {Grp. gram.:adv.}
\end{itemize}
De modo fino.
Com elegância; delicadamente.
\section{Finamento}
\begin{itemize}
\item {Grp. gram.:m.}
\end{itemize}
Acto ou effeito de finar-se.
\section{Finança}
\begin{itemize}
\item {Grp. gram.:f.}
\end{itemize}
\begin{itemize}
\item {Proveniência:(Fr. \textunderscore finance\textunderscore )}
\end{itemize}
Fazenda pública.
Erário.
Estado financeiro.
\section{Finanças}
\begin{itemize}
\item {Grp. gram.:f. pl.}
\end{itemize}
\begin{itemize}
\item {Proveniência:(Fr. \textunderscore finance\textunderscore )}
\end{itemize}
Fazenda pública.
Erário.
Estado financeiro.
\section{Financeiro}
\begin{itemize}
\item {Grp. gram.:adj.}
\end{itemize}
\begin{itemize}
\item {Grp. gram.:M.}
\end{itemize}
Relativo ás finanças.
Aquelle que é versado em assumptos de finanças.
\section{Financial}
\begin{itemize}
\item {Grp. gram.:adj.}
\end{itemize}
\begin{itemize}
\item {Proveniência:(Do fr. \textunderscore finance\textunderscore )}
\end{itemize}
Relativo a finanças; financeiro.
\section{Finanga}
\begin{itemize}
\item {Grp. gram.:f.}
\end{itemize}
Gênero de palmeiras.
\section{Finar}
\begin{itemize}
\item {Grp. gram.:v. i.}
\end{itemize}
Acabar, findar.
O mesmo que \textunderscore finar-se\textunderscore .
\section{Finar-se}
\begin{itemize}
\item {Grp. gram.:v. p.}
\end{itemize}
\begin{itemize}
\item {Proveniência:(Do lat. \textunderscore finis\textunderscore )}
\end{itemize}
Definhar; consumir-se.
Fallecer, morrer.
Sentir imperiosa necessidade de alimento.
\section{Finca}
\begin{itemize}
\item {Grp. gram.:f.}
\end{itemize}
\begin{itemize}
\item {Grp. gram.:Loc. adv.}
\end{itemize}
\begin{itemize}
\item {Proveniência:(De \textunderscore fincar\textunderscore )}
\end{itemize}
O mesmo que \textunderscore escora\textunderscore .
\textunderscore Ás fincas\textunderscore , com afinco, com empenho. Cf. Filinto, XII, 91.
\section{Fincão}
\begin{itemize}
\item {Grp. gram.:m.}
\end{itemize}
\begin{itemize}
\item {Utilização:Prov.}
\end{itemize}
\begin{itemize}
\item {Utilização:trasm.}
\end{itemize}
\begin{itemize}
\item {Proveniência:(De \textunderscore finca\textunderscore )}
\end{itemize}
Pau vertical, que sustenta a loisa de uma armadilha.
Pedra a pino, a servir de marco ou a constituir e adeantar parede ligeira.
\section{Finca-pé}
\begin{itemize}
\item {Grp. gram.:m.}
\end{itemize}
\begin{itemize}
\item {Utilização:Fig.}
\end{itemize}
Acto de fincar o pé com fôrça.
Porfia; empenho.
Amparo.
\section{Fincar}
\begin{itemize}
\item {Grp. gram.:v. t.}
\end{itemize}
Cravar; estribar, apoiar.
Enraizar.
(Cp. \textunderscore ficar\textunderscore ^1)
\section{Finco}
\begin{itemize}
\item {Grp. gram.:m.}
\end{itemize}
\begin{itemize}
\item {Utilização:Ant.}
\end{itemize}
\begin{itemize}
\item {Proveniência:(De \textunderscore fincar\textunderscore )}
\end{itemize}
Contrato por escritura.
\section{Finda}
\begin{itemize}
\item {Grp. gram.:f.}
\end{itemize}
\begin{itemize}
\item {Utilização:Ant.}
\end{itemize}
Acto de findar.
\section{Findador}
\begin{itemize}
\item {Grp. gram.:m.}
\end{itemize}
\begin{itemize}
\item {Proveniência:(De \textunderscore findar\textunderscore )}
\end{itemize}
Aquelle que põe fim.
\section{Findar}
\begin{itemize}
\item {Grp. gram.:v. t.}
\end{itemize}
\begin{itemize}
\item {Grp. gram.:V. i.}
\end{itemize}
\begin{itemize}
\item {Proveniência:(De \textunderscore findo\textunderscore )}
\end{itemize}
Pôr fim a.
Finalizar.
Têr fim; acabar.
\section{Findável}
\begin{itemize}
\item {Grp. gram.:adj.}
\end{itemize}
\begin{itemize}
\item {Proveniência:(De \textunderscore findar\textunderscore )}
\end{itemize}
Que há de têr fim.
Contingente.
Transitório.
\section{Findo}
\begin{itemize}
\item {Grp. gram.:adj.}
\end{itemize}
\begin{itemize}
\item {Proveniência:(Do lat. \textunderscore finitus\textunderscore )}
\end{itemize}
Que findou.
Concluido: \textunderscore está findo aquelle trabalho\textunderscore .
\section{Finês}
\begin{itemize}
\item {Grp. gram.:m.}
\end{itemize}
Língua dos Fineses.
\section{Fineses}
\begin{itemize}
\item {Grp. gram.:m. pl.}
\end{itemize}
\begin{itemize}
\item {Proveniência:(Do lat. \textunderscore fenni\textunderscore )}
\end{itemize}
Povos setentrionaes do antigo continente.
\section{Fineza}
\begin{itemize}
\item {Grp. gram.:f.}
\end{itemize}
Qualidade daquillo que é fino.
Delicadeza.
Obséquio.
Primor.
Amabilidade.
\section{Finfar}
\begin{itemize}
\item {Grp. gram.:v. t.  e  i.}
\end{itemize}
\begin{itemize}
\item {Utilização:Gír.}
\end{itemize}
O mesmo que \textunderscore afinfar\textunderscore .
\section{Fingidamente}
\begin{itemize}
\item {Grp. gram.:adv.}
\end{itemize}
De modo fingido.
\section{Fingidiçamente}
\begin{itemize}
\item {Grp. gram.:adv.}
\end{itemize}
De modo fingidiço.
\section{Fingidiço}
\begin{itemize}
\item {Grp. gram.:adj.}
\end{itemize}
\begin{itemize}
\item {Utilização:Ant.}
\end{itemize}
\begin{itemize}
\item {Proveniência:(De \textunderscore fingir\textunderscore )}
\end{itemize}
O mesmo que \textunderscore fictício\textunderscore .
\section{Fingidor}
\begin{itemize}
\item {Grp. gram.:m.}
\end{itemize}
\begin{itemize}
\item {Proveniência:(De \textunderscore fingir\textunderscore )}
\end{itemize}
Aquelle que finge.
Pintor, que imita os trabalhos alheios.
Pintor de brocha, que com tinta, applicada com uma escôva especial, imita madeiras finas sôbre madeira ordinária.
\section{Fingimento}
\begin{itemize}
\item {Grp. gram.:m.}
\end{itemize}
Acto ou effeito de fingir.
\section{Fingir}
\begin{itemize}
\item {Grp. gram.:v. t.}
\end{itemize}
\begin{itemize}
\item {Grp. gram.:V. i.}
\end{itemize}
\begin{itemize}
\item {Proveniência:(Lat. \textunderscore fingere\textunderscore )}
\end{itemize}
Inventar.
Fantasiar.
Simular: \textunderscore fingir sinceridade\textunderscore .
Arremedar: \textunderscore fingir latidos\textunderscore .
Sêr dissimulado, apparentar o que não é.
\section{Fingir}
\begin{itemize}
\item {Grp. gram.:v. t.}
\end{itemize}
\begin{itemize}
\item {Utilização:Prov.}
\end{itemize}
\begin{itemize}
\item {Utilização:trasm.}
\end{itemize}
Remexer e trabalhar novamente com as mãos (a massa do pão, depois de levedada).
\section{Fini}
\begin{itemize}
\item {Grp. gram.:m.}
\end{itemize}
Bebida alcoólica, usada na Índia e Moçambique, e que é sura, depois de fermentada e destillada.
\section{Fínico}
\begin{itemize}
\item {Grp. gram.:m.}
\end{itemize}
A língua dos Fineses.
O finês.
\section{Finidade}
\begin{itemize}
\item {Grp. gram.:f.}
\end{itemize}
\begin{itemize}
\item {Proveniência:(Do lat. \textunderscore finitus\textunderscore )}
\end{itemize}
Qualidade do que é finito.
\section{Finisterra}
\begin{itemize}
\item {Grp. gram.:f.}
\end{itemize}
\begin{itemize}
\item {Utilização:Geogr.}
\end{itemize}
\begin{itemize}
\item {Proveniência:(Do lat. \textunderscore finis\textunderscore  + \textunderscore terra\textunderscore )}
\end{itemize}
Cabo, que termina uma região ou a parte conhecida de uma região.
\section{Finítimo}
\begin{itemize}
\item {Grp. gram.:adj.}
\end{itemize}
\begin{itemize}
\item {Proveniência:(Lat. \textunderscore finitimus\textunderscore )}
\end{itemize}
Vizinho; confinante.
\section{Finito}
\begin{itemize}
\item {Grp. gram.:adj.}
\end{itemize}
\begin{itemize}
\item {Utilização:Gram.}
\end{itemize}
\begin{itemize}
\item {Proveniência:(Lat. \textunderscore finitus\textunderscore )}
\end{itemize}
Que tem fim.
Transitório, contingente.
Determinado, (falando-se de alguns modos dos verbos).
\section{Finlandês}
\begin{itemize}
\item {Grp. gram.:adj.}
\end{itemize}
\begin{itemize}
\item {Grp. gram.:M.}
\end{itemize}
Relativo á Finlândia.
Habitante da Finlândia.
Língua uralo-altaica do grupo do fínnico.
\section{Finnês}
\begin{itemize}
\item {Grp. gram.:m.}
\end{itemize}
Língua dos Finneses.
\section{Finneses}
\begin{itemize}
\item {Grp. gram.:m. pl.}
\end{itemize}
\begin{itemize}
\item {Proveniência:(Do lat. \textunderscore fenni\textunderscore )}
\end{itemize}
Povos setentrionaes do antigo continente.
\section{Fínnico}
\begin{itemize}
\item {Grp. gram.:m.}
\end{itemize}
A língua dos Finneses.
O finnês.
\section{Fino}
\begin{itemize}
\item {Grp. gram.:adj.}
\end{itemize}
\begin{itemize}
\item {Utilização:Fam.}
\end{itemize}
\begin{itemize}
\item {Grp. gram.:Loc. adv.}
\end{itemize}
\begin{itemize}
\item {Proveniência:(Do lat. \textunderscore finitus\textunderscore , segundo Diez)}
\end{itemize}
Delgado: \textunderscore linhas finas\textunderscore .
Pequenino.
Perfeito.
Delicado; amável.
Superior, excellente: \textunderscore vinho fino\textunderscore .
Suave.
Elegante; bem trajado.
Vibrante.
Afiado.
Penetrante.
Sagaz: \textunderscore homem fino\textunderscore .
Que tem vivacidade: \textunderscore cavallo fino\textunderscore .
Desvelado.
\textunderscore Fazer-se fino\textunderscore , mostrar-se gracioso, atrevido, requestador.
\textunderscore Á fina fôrça\textunderscore , a todo transe; custe o que custar.
\section{Finoriamente}
\begin{itemize}
\item {Grp. gram.:adv.}
\end{itemize}
Com astúcia, com modos de finório.
\section{Finório}
\begin{itemize}
\item {Grp. gram.:m.  e  adj.}
\end{itemize}
\begin{itemize}
\item {Proveniência:(De \textunderscore fion\textunderscore )}
\end{itemize}
Indivíduo sagaz, manhoso.
\section{Finta}
\begin{itemize}
\item {Grp. gram.:f.}
\end{itemize}
Contribuição extraordinária ou encargo pecuniário, proporcional aos rendimentos de quem é fintado.
Derrama parochial.
(Contr. de \textunderscore finita\textunderscore , do lat. \textunderscore finitus\textunderscore )
\section{Fintar}
\begin{itemize}
\item {Grp. gram.:v. t.}
\end{itemize}
\begin{itemize}
\item {Grp. gram.:V. p.}
\end{itemize}
Lançar finta sôbre.
Quotizar-se, pagar por escote.
\section{Fintar}
\begin{itemize}
\item {Grp. gram.:v. t.}
\end{itemize}
\begin{itemize}
\item {Utilização:Prov.}
\end{itemize}
\begin{itemize}
\item {Grp. gram.:V. i.}
\end{itemize}
\begin{itemize}
\item {Proveniência:(De \textunderscore finto\textunderscore )}
\end{itemize}
Levedar, fazer fermentar: \textunderscore fintar a massa do pão\textunderscore .
Tornar-se lêvedo.
\section{Fintar}
\begin{itemize}
\item {Grp. gram.:v. t.}
\end{itemize}
\begin{itemize}
\item {Utilização:Bras}
\end{itemize}
Enganar.
(Relaciona-se com \textunderscore fintar\textunderscore ^1?)
\section{Fintar}
\begin{itemize}
\item {Grp. gram.:v. i.}
\end{itemize}
\begin{itemize}
\item {Utilização:Prov.}
\end{itemize}
\begin{itemize}
\item {Utilização:trasm.}
\end{itemize}
\begin{itemize}
\item {Grp. gram.:V. p.}
\end{itemize}
Acreditar, confiar: \textunderscore não fintes nelle\textunderscore .
(A mesma significação)
\section{Finto}
\begin{itemize}
\item {Grp. gram.:m.}
\end{itemize}
\begin{itemize}
\item {Grp. gram.:Adj.}
\end{itemize}
\begin{itemize}
\item {Utilização:Ant.}
\end{itemize}
\begin{itemize}
\item {Proveniência:(Do lat. \textunderscore finitus\textunderscore )}
\end{itemize}
Antiga contribuição, que se pagava na ilha da Madeira.
Levedado, fermentado.
O mesmo que \textunderscore findo\textunderscore . Cf. G. Vicente, I, 250.
\section{Finura}
\begin{itemize}
\item {Grp. gram.:f.}
\end{itemize}
Qualidade daquelle ou daquillo que é fino.
Malícia, astúcia.
\section{Fio}
\begin{itemize}
\item {Grp. gram.:m.}
\end{itemize}
\begin{itemize}
\item {Utilização:Constr.}
\end{itemize}
\begin{itemize}
\item {Grp. gram.:Loc. adv.}
\end{itemize}
\begin{itemize}
\item {Grp. gram.:Pl.}
\end{itemize}
\begin{itemize}
\item {Proveniência:(Lat. \textunderscore filum\textunderscore )}
\end{itemize}
Fibra, que se extrai de plantas têxteis.
Aquillo que tem semelhança com essa fibra.
Linha, que se fiou ou se torceu.
Fieira.
Gume de um instrumento.
Encadeamento entre as partes de um todo: \textunderscore perder o fio da conversa\textunderscore .
Serragem numa peça de madeira: \textunderscore serrar-se um toro a dois fios, a três\textunderscore , etc.
\textunderscore Estar no fio\textunderscore , diz-se do fato já muito usado ou cotiado.
\textunderscore Levar tudo a fio de espada\textunderscore , acutilar para todos os lados.
\textunderscore Meio fio\textunderscore , certo modo de ligar tábuas de um solho, tecto, etc.
\textunderscore A fio\textunderscore , a eito, seguidamente.
Meios, processos. Cf. Bernárdez, \textunderscore Luz e Calor\textunderscore , 373.
\section{Fiolhal}
\begin{itemize}
\item {Grp. gram.:m.}
\end{itemize}
\begin{itemize}
\item {Proveniência:(De \textunderscore fiolho\textunderscore )}
\end{itemize}
(V.funchal)
\section{Fiolho}
\begin{itemize}
\item {fónica:ô}
\end{itemize}
\begin{itemize}
\item {Grp. gram.:m.}
\end{itemize}
\begin{itemize}
\item {Utilização:Prov.}
\end{itemize}
\begin{itemize}
\item {Utilização:trasm.}
\end{itemize}
\begin{itemize}
\item {Proveniência:(Do lat. \textunderscore foenunculum\textunderscore )}
\end{itemize}
O mesmo que \textunderscore funcho\textunderscore .
\section{Fiomel}
\begin{itemize}
\item {Grp. gram.:m.}
\end{itemize}
\begin{itemize}
\item {Utilização:Prov.}
\end{itemize}
\begin{itemize}
\item {Utilização:beir.}
\end{itemize}
Pessôa muito fraca e adoentada. (Colhido no Fundão)
\section{Fiorita}
\begin{itemize}
\item {Grp. gram.:f.}
\end{itemize}
Espécie de opala nacarada.
\section{Fioses}
\begin{itemize}
\item {Grp. gram.:m. pl.}
\end{itemize}
\begin{itemize}
\item {Utilização:Prov.}
\end{itemize}
\begin{itemize}
\item {Utilização:trasm.}
\end{itemize}
\begin{itemize}
\item {Proveniência:(De \textunderscore fio\textunderscore )}
\end{itemize}
Enredos interesseiros.
\section{Fiota}
\begin{itemize}
\item {Grp. gram.:m.  e  adj.}
\end{itemize}
\begin{itemize}
\item {Utilização:Bras. do N}
\end{itemize}
Janota; peralvilho.
\section{Fiote}
\begin{itemize}
\item {Grp. gram.:m.}
\end{itemize}
\begin{itemize}
\item {Grp. gram.:Pl.}
\end{itemize}
Língua das margens do Zaire.
Uma das tríbos da raça dos Bântus.
\section{Firafolha}
\begin{itemize}
\item {Grp. gram.:f.}
\end{itemize}
O mesmo que \textunderscore felosa\textunderscore .
\section{Firma}
\begin{itemize}
\item {Grp. gram.:f.}
\end{itemize}
\begin{itemize}
\item {Utilização:Deprec.}
\end{itemize}
\begin{itemize}
\item {Utilização:Comm.}
\end{itemize}
\begin{itemize}
\item {Utilização:Ant.}
\end{itemize}
\begin{itemize}
\item {Utilização:T. de Resende}
\end{itemize}
\begin{itemize}
\item {Proveniência:(De \textunderscore firmar\textunderscore )}
\end{itemize}
Assignatura por extenso ou em breve, manuscrita ou gravada.
Ponto de apoio.
\textunderscore Má firma\textunderscore , pessôa que não merece confiança.
Assignatura commercial, que representa uma Sociedade ou Companhia.
O mesmo que \textunderscore testemunha\textunderscore .
Espécie de pano verde.
\section{Firmação}
\begin{itemize}
\item {Grp. gram.:f.}
\end{itemize}
Acto ou effeito de firmar.
\section{Firmado}
\begin{itemize}
\item {Grp. gram.:adj.}
\end{itemize}
\begin{itemize}
\item {Utilização:Heráld.}
\end{itemize}
\begin{itemize}
\item {Proveniência:(De \textunderscore firmar\textunderscore )}
\end{itemize}
Diz-se da cruz, cujas hastes tocam todos os extremos do escudo.
\section{Firmador}
\begin{itemize}
\item {Grp. gram.:m.}
\end{itemize}
Aquelle que firma.
\section{Firmal}
\begin{itemize}
\item {Grp. gram.:m.}
\end{itemize}
\begin{itemize}
\item {Grp. gram.:Pl.}
\end{itemize}
\begin{itemize}
\item {Proveniência:(Do rad. de \textunderscore firme\textunderscore )}
\end{itemize}
Peça antiga de metal.
Espécie de broche, com que se prendiam os vestidos.
Sinete com firma.
Relicário.
Pontas do cabresto, que se prendem nas argolas das ilhargas.
\section{Firmamental}
\begin{itemize}
\item {Grp. gram.:adj.}
\end{itemize}
\begin{itemize}
\item {Utilização:Neol.}
\end{itemize}
Relativo ao firmamento.
\section{Firmamento}
\begin{itemize}
\item {Grp. gram.:m.}
\end{itemize}
\begin{itemize}
\item {Proveniência:(Lat. \textunderscore firmamentum\textunderscore )}
\end{itemize}
Fundamento.
Alicerce.
Sustentáculo.
Abóbada celeste; céu.
\section{Firmão}
\begin{itemize}
\item {Grp. gram.:m.}
\end{itemize}
\begin{itemize}
\item {Proveniência:(Do pers. \textunderscore farmen\textunderscore )}
\end{itemize}
Ordem, emanada de um soberano ou autoridade muçulmana e por ella assinada. Cf. \textunderscore Peregrinação\textunderscore , CXIX.
\section{Firmar}
\begin{itemize}
\item {Grp. gram.:v. t.}
\end{itemize}
\begin{itemize}
\item {Grp. gram.:V. p.}
\end{itemize}
\begin{itemize}
\item {Proveniência:(Lat. \textunderscore firmare\textunderscore )}
\end{itemize}
Tornar firme.
Assegurar.
Confirmar.
Sanccionar.
Fincar.
Subscrever com o seu nome.
Ratificar.
Authenticar.
Gravar.
Estribar.
Apoiar: \textunderscore firmar os pés numa viga\textunderscore .
Fixar (attenção).
Escrever a sua firma.
Fixar a attenção, compenetrar-se.
Basear-se, apoiar-se: \textunderscore firmar-se na opinião pública\textunderscore .
\section{Firme}
\begin{itemize}
\item {Grp. gram.:adj.}
\end{itemize}
\begin{itemize}
\item {Proveniência:(Lat. \textunderscore firmus\textunderscore )}
\end{itemize}
Fixo.
Estável.
Forte.
Robusto.
Inabalável.
Constante.
Resoluto.
Contumaz.
Obstinado.
Seguro, que não treme, que não vacilla: \textunderscore pedra firme\textunderscore .
Que não muda de opinião.
Que tem prazo fixo, (falando-se de empréstimos ou de compra e venda de fundos públicos).
\textunderscore Terra firme\textunderscore , a parte sólida do globo.
\section{Firmemente}
\begin{itemize}
\item {Grp. gram.:adv.}
\end{itemize}
De modo firme.
Solidamente.
Convictamente.
Com firmeza.
\section{Firmeza}
\begin{itemize}
\item {Grp. gram.:f.}
\end{itemize}
Qualidade daquelle ou daquillo que é firme.
Fixidez; estabilidade; solidez.
Constância.
\section{Firmidão}
\begin{itemize}
\item {Grp. gram.:f.}
\end{itemize}
\begin{itemize}
\item {Utilização:Des.}
\end{itemize}
\begin{itemize}
\item {Proveniência:(Do lat. \textunderscore firmitudo\textunderscore )}
\end{itemize}
(V.firmeza)
\section{Firmideu}
\begin{itemize}
\item {Grp. gram.:m.}
\end{itemize}
\begin{itemize}
\item {Utilização:Des.}
\end{itemize}
O mesmo que \textunderscore firmidão\textunderscore .
Valor de um documento público.
Authenticidade.
\section{Firo}
\begin{itemize}
\item {Grp. gram.:m.}
\end{itemize}
\begin{itemize}
\item {Utilização:Ant.}
\end{itemize}
\begin{itemize}
\item {Proveniência:(De \textunderscore ferir\textunderscore ?)}
\end{itemize}
Jôgo de pedras, semelhante ao alguergue.
\section{Fisberta}
\begin{itemize}
\item {Grp. gram.:f.}
\end{itemize}
\begin{itemize}
\item {Utilização:ant.}
\end{itemize}
\begin{itemize}
\item {Utilização:Chul.}
\end{itemize}
Espada; durindana.
(Cast. \textunderscore fisberta\textunderscore )
\section{Fiscal}
\begin{itemize}
\item {Grp. gram.:adj.}
\end{itemize}
\begin{itemize}
\item {Grp. gram.:M.}
\end{itemize}
\begin{itemize}
\item {Proveniência:(Lat. \textunderscore fiscais\textunderscore )}
\end{itemize}
Relativo ao fisco: \textunderscore autoridade fiscal\textunderscore .
Empregado alfandegário.
Indivíduo, encarregado de fiscalizar certos actos ou a execução de certas disposições: \textunderscore os fiscaes do sêllo\textunderscore .
\section{Fiscalização}
\begin{itemize}
\item {Grp. gram.:f.}
\end{itemize}
Acto ou effeito de fiscalizar.
\section{Fiscalizador}
\begin{itemize}
\item {Grp. gram.:m.  e  adj.}
\end{itemize}
O que fiscaliza.
\section{Fiscalizar}
\begin{itemize}
\item {Grp. gram.:v. t.}
\end{itemize}
\begin{itemize}
\item {Proveniência:(De \textunderscore fiscal\textunderscore )}
\end{itemize}
Velar por.
Vigiar; examinar; observar.
Syndicar.
Censurar.
\section{Fiscário}
\begin{itemize}
\item {Grp. gram.:m.}
\end{itemize}
\begin{itemize}
\item {Utilização:Des.}
\end{itemize}
\begin{itemize}
\item {Proveniência:(Lat. \textunderscore fiscarius\textunderscore )}
\end{itemize}
Aquelle que tem o fisco a seu cargo.
\section{Fiscela}
\begin{itemize}
\item {Grp. gram.:f.}
\end{itemize}
\begin{itemize}
\item {Proveniência:(Lat. \textunderscore fiscella\textunderscore )}
\end{itemize}
O mesmo que \textunderscore açamo\textunderscore .
\section{Fiscella}
\begin{itemize}
\item {Grp. gram.:f.}
\end{itemize}
\begin{itemize}
\item {Proveniência:(Lat. \textunderscore fiscella\textunderscore )}
\end{itemize}
O mesmo que \textunderscore açamo\textunderscore .
\section{Fisco}
\begin{itemize}
\item {Grp. gram.:m.}
\end{itemize}
\begin{itemize}
\item {Proveniência:(Lat. \textunderscore fiscus\textunderscore )}
\end{itemize}
Fazenda pública.
Erário.
Parte da Administração pública, encarregada da cobrança dos impostos.
\section{Fisga}
\begin{itemize}
\item {Grp. gram.:f.}
\end{itemize}
Arpão para pesca.
Pequena abertura ou fenda.
(Cast. \textunderscore fisga\textunderscore )
\section{Fisgada}
\begin{itemize}
\item {Utilização:Prov.}
\end{itemize}
\begin{itemize}
\item {Utilização:dur.}
\end{itemize}
\begin{itemize}
\item {Proveniência:(De \textunderscore fisga\textunderscore )}
\end{itemize}
Dôr violenta, com intervallos de descanso.
\section{Fisgado}
\begin{itemize}
\item {Grp. gram.:adj.}
\end{itemize}
\begin{itemize}
\item {Grp. gram.:Loc.}
\end{itemize}
\begin{itemize}
\item {Utilização:pop.}
\end{itemize}
\begin{itemize}
\item {Proveniência:(De \textunderscore fisgar\textunderscore )}
\end{itemize}
Apanhado com fisga.
Agarrado.
\textunderscore Leva a sua fisgada\textunderscore , é arteiro, tem intuitos maliciosos, leva água no bico.
\section{Fisgador}
\begin{itemize}
\item {Grp. gram.:m.  e  adj.}
\end{itemize}
O que fisga.
\section{Fisgar}
\begin{itemize}
\item {Grp. gram.:v. t.}
\end{itemize}
Pescar com fisga.
Prender.
Apanhar rapidamente.
Perceber logo.
\section{Fisqueiro}
\begin{itemize}
\item {Grp. gram.:m.}
\end{itemize}
\begin{itemize}
\item {Utilização:Ant.}
\end{itemize}
\begin{itemize}
\item {Proveniência:(De \textunderscore fisco\textunderscore )}
\end{itemize}
Aquelle que arrecadava as rendas do fisco.
\section{Fissidentadas}
\begin{itemize}
\item {Grp. gram.:f. pl.}
\end{itemize}
\begin{itemize}
\item {Proveniência:(Do lat. \textunderscore fissus\textunderscore  + \textunderscore dentatus\textunderscore )}
\end{itemize}
Pequena tribo de musgos.
\section{Fissifloro}
\begin{itemize}
\item {Grp. gram.:adj.}
\end{itemize}
\begin{itemize}
\item {Utilização:Bot.}
\end{itemize}
\begin{itemize}
\item {Proveniência:(Do lat. \textunderscore fissus\textunderscore  + \textunderscore flos\textunderscore )}
\end{itemize}
Que tem a corolla fendida.
\section{Físsil}
\begin{itemize}
\item {Grp. gram.:adj.}
\end{itemize}
\begin{itemize}
\item {Proveniência:(Lat. \textunderscore fissilis\textunderscore )}
\end{itemize}
Que se póde fender.
\section{Fissiparidade}
\begin{itemize}
\item {Grp. gram.:f.}
\end{itemize}
\begin{itemize}
\item {Proveniência:(De \textunderscore fissíparo\textunderscore )}
\end{itemize}
Reproducção, por fragmentação, de organismos monocellulares.
\section{Fissíparo}
\begin{itemize}
\item {Grp. gram.:adj.}
\end{itemize}
\begin{itemize}
\item {Utilização:Zool.}
\end{itemize}
\begin{itemize}
\item {Proveniência:(Do lat. \textunderscore fissus\textunderscore  + \textunderscore parere\textunderscore )}
\end{itemize}
Que se produz pela divisão do seu próprio corpo.
\section{Fissípede}
\begin{itemize}
\item {Grp. gram.:adj.}
\end{itemize}
\begin{itemize}
\item {Utilização:Zool.}
\end{itemize}
\begin{itemize}
\item {Proveniência:(Lat. \textunderscore fissipes\textunderscore )}
\end{itemize}
Que tem pés ou unhas fendidas, ou os dedos unidos por membranas.
\section{Fissipene}
\begin{itemize}
\item {Grp. gram.:adj.}
\end{itemize}
\begin{itemize}
\item {Utilização:Zool.}
\end{itemize}
\begin{itemize}
\item {Proveniência:(Do lat. \textunderscore fissus\textunderscore  + \textunderscore penna\textunderscore )}
\end{itemize}
Cujas asas estão divididas em partes.
\section{Fissipenne}
\begin{itemize}
\item {Grp. gram.:adj.}
\end{itemize}
\begin{itemize}
\item {Utilização:Zool.}
\end{itemize}
\begin{itemize}
\item {Proveniência:(Do lat. \textunderscore fissus\textunderscore  + \textunderscore penna\textunderscore )}
\end{itemize}
Cujas asas estão divididas em partes.
\section{Fissirostros}
\begin{itemize}
\item {fónica:sirrós}
\end{itemize}
\begin{itemize}
\item {Grp. gram.:m. pl.}
\end{itemize}
\begin{itemize}
\item {Proveniência:(Do lat. \textunderscore fissus\textunderscore  + \textunderscore rostrum\textunderscore )}
\end{itemize}
Aves, que têm o bico fendido.
\section{Fissirrostros}
\begin{itemize}
\item {Grp. gram.:m. pl.}
\end{itemize}
\begin{itemize}
\item {Proveniência:(Do lat. \textunderscore fissus\textunderscore  + \textunderscore rostrum\textunderscore )}
\end{itemize}
Aves, que têm o bico fendido.
\section{Fissura}
\begin{itemize}
\item {Grp. gram.:f.}
\end{itemize}
\begin{itemize}
\item {Proveniência:(Lat. \textunderscore fissura\textunderscore )}
\end{itemize}
Fenda.
Úlcera na mucosa do ânus.
\section{Fissuração}
\begin{itemize}
\item {Grp. gram.:f.}
\end{itemize}
\begin{itemize}
\item {Proveniência:(Do rad. de \textunderscore fissura\textunderscore )}
\end{itemize}
Estado daquillo que está fendido.
Divisão das vísceras em lóbulos.
\section{Fistor}
\begin{itemize}
\item {Grp. gram.:m.}
\end{itemize}
\begin{itemize}
\item {Utilização:Prov.}
\end{itemize}
\begin{itemize}
\item {Proveniência:(Do cast. \textunderscore fistol\textunderscore )}
\end{itemize}
Farçola.
Homem, que presume saber de tudo, e de tudo julga com alarde.
Velhaquete; homem finório.
\section{Fistorice}
\begin{itemize}
\item {Grp. gram.:f.}
\end{itemize}
\begin{itemize}
\item {Utilização:Prov.}
\end{itemize}
\begin{itemize}
\item {Utilização:trasm.}
\end{itemize}
Dito ou acto de fistor.
\section{Fístula}
\begin{itemize}
\item {Grp. gram.:f.}
\end{itemize}
\begin{itemize}
\item {Utilização:Poét.}
\end{itemize}
\begin{itemize}
\item {Utilização:Prov.}
\end{itemize}
\begin{itemize}
\item {Utilização:trasm.}
\end{itemize}
\begin{itemize}
\item {Proveniência:(Lat. \textunderscore fistula\textunderscore )}
\end{itemize}
Úlcera, de abertura estreita, e communicando com uma cavidade natural.
Frauta pastoril.
Sinal, marca.
\section{Fistulado}
\begin{itemize}
\item {Grp. gram.:adj.}
\end{itemize}
\begin{itemize}
\item {Proveniência:(De \textunderscore fístula\textunderscore )}
\end{itemize}
Semelhante a fístula.
Atravessado por tubo.
Que tem canal interior; fístuloso.
\section{Fistular}
\begin{itemize}
\item {Grp. gram.:adj.}
\end{itemize}
\begin{itemize}
\item {Proveniência:(Lat. \textunderscore fistularis\textunderscore )}
\end{itemize}
Fistulado.
Atravessado por um tubo em todo o comprimento; tubular.
\section{Fistular}
\begin{itemize}
\item {Grp. gram.:v. t.}
\end{itemize}
\begin{itemize}
\item {Proveniência:(Lat. \textunderscore fistulare\textunderscore )}
\end{itemize}
Tornar-se fístula.
\section{Fistulivalve}
\begin{itemize}
\item {Grp. gram.:adj.}
\end{itemize}
\begin{itemize}
\item {Proveniência:(De \textunderscore fistula\textunderscore  + \textunderscore valva\textunderscore )}
\end{itemize}
Que tem conchas com as valvas em fórma de tubo.
\section{Fistuloso}
\begin{itemize}
\item {Grp. gram.:adj.}
\end{itemize}
\begin{itemize}
\item {Proveniência:(Lat. \textunderscore fistulosus\textunderscore )}
\end{itemize}
Que tem fístulas; ulcerado; tubular.
\section{Fita}
\begin{itemize}
\item {Grp. gram.:f.}
\end{itemize}
\begin{itemize}
\item {Proveniência:(Do lat. \textunderscore vitta\textunderscore )}
\end{itemize}
Tecido, mais ou menos estreito comprido, que serve para ligar, ornar, etc.
Faixa; tira.
Insígnia honorifica ou nobiliária.
Aquillo que tem fórma de faixa.
O mesmo que \textunderscore maravalha\textunderscore .
\section{Fita}
\begin{itemize}
\item {Grp. gram.:f.}
\end{itemize}
Acto de fitar. Cf. \textunderscore Viriato Trág.\textunderscore , I, 35.
\section{Fitáceo}
\begin{itemize}
\item {Grp. gram.:adj.}
\end{itemize}
Que tem fôlhas em fórma de fita.
\section{Fitaléria}
\begin{itemize}
\item {Grp. gram.:f.}
\end{itemize}
\begin{itemize}
\item {Utilização:Ant.}
\end{itemize}
Espécie de armadilha? Cf. \textunderscore Aulegrafia\textunderscore , 94.
\section{Fitar}
\begin{itemize}
\item {Grp. gram.:v. t.}
\end{itemize}
\begin{itemize}
\item {Utilização:Neol.}
\end{itemize}
\begin{itemize}
\item {Proveniência:(Do lat. \textunderscore fictare\textunderscore )}
\end{itemize}
Fixar (a vista).
Fixar (a attenção, o pensamento).
Endireitar (as orelhas, falando-se de alguns animaes).
Olhar fixamente para. Cf. Camillo, \textunderscore Caveira\textunderscore , 49 e 164.--Nesta última accepção, é linguagem suspeita de incorrecta.
\section{Fitaria}
\begin{itemize}
\item {Grp. gram.:f.}
\end{itemize}
Porção de fitas. Cf. Garrett, \textunderscore Romanceiro\textunderscore , II, 34.
\section{Fiteira}
\begin{itemize}
\item {Grp. gram.:f.}
\end{itemize}
\begin{itemize}
\item {Utilização:Prov.}
\end{itemize}
\begin{itemize}
\item {Utilização:trasm.}
\end{itemize}
\begin{itemize}
\item {Utilização:Bras. do S}
\end{itemize}
\begin{itemize}
\item {Proveniência:(De \textunderscore fita\textunderscore )}
\end{itemize}
Mulher, que faz fitas.
Tábua, assente em gume sôbre outra, e que serve para a ella se encostar o linho, que há de sêr espadelado.
Mostrador; armário envidraçado.
\section{Fiteiro}
\begin{itemize}
\item {Grp. gram.:m.}
\end{itemize}
\begin{itemize}
\item {Utilização:Des.}
\end{itemize}
\begin{itemize}
\item {Grp. gram.:Adj.}
\end{itemize}
\begin{itemize}
\item {Utilização:Prov.}
\end{itemize}
\begin{itemize}
\item {Utilização:trasm.}
\end{itemize}
Aquelle que faz fitas.
Diz-se do vento brando.
\section{Fitilho}
\begin{itemize}
\item {Grp. gram.:m.}
\end{itemize}
Fita muito estreita; nastro.
\section{Fitinha}
\begin{itemize}
\item {Grp. gram.:f.}
\end{itemize}
\begin{itemize}
\item {Utilização:Fam.}
\end{itemize}
\begin{itemize}
\item {Proveniência:(De \textunderscore fita\textunderscore )}
\end{itemize}
Condecoração.
\section{Fito}
\begin{itemize}
\item {Grp. gram.:m.}
\end{itemize}
\begin{itemize}
\item {Grp. gram.:Loc. adv.}
\end{itemize}
\begin{itemize}
\item {Proveniência:(De \textunderscore fitar\textunderscore )}
\end{itemize}
Alvo.
Mira.
Intuito.
Fim.
Jôgo popular, em que se atira bóla ou malha a um pau cravado ou collocado verticalmente no chão, e que também se chama fito.
\textunderscore A fito\textunderscore , fixamente, de olhos fitos:«\textunderscore quando Balbina o encarou muito a fito...\textunderscore »Camillo, \textunderscore Volcões\textunderscore , 139.
\section{Fiunça}
\begin{itemize}
\item {Grp. gram.:f.}
\end{itemize}
\begin{itemize}
\item {Utilização:Prov.}
\end{itemize}
\begin{itemize}
\item {Utilização:trasm.}
\end{itemize}
\begin{itemize}
\item {Proveniência:(De \textunderscore fio\textunderscore )}
\end{itemize}
\textunderscore Ir de fiunça\textunderscore , ir directamente, rapidamente, de propósito.
\section{Fiunco}
\begin{itemize}
\item {Grp. gram.:m.}
\end{itemize}
\begin{itemize}
\item {Utilização:Prov.}
\end{itemize}
\begin{itemize}
\item {Utilização:trasm.}
\end{itemize}
\begin{itemize}
\item {Proveniência:(De \textunderscore fio\textunderscore )}
\end{itemize}
Caulículo.
Uma palha.
\section{Fiúsa}
\begin{itemize}
\item {Grp. gram.:adj.}
\end{itemize}
\begin{itemize}
\item {Utilização:Bras}
\end{itemize}
Desusado, que está fóra da moda: \textunderscore um chapéu fiúsa\textunderscore .
\section{Fiúza}
\begin{itemize}
\item {Grp. gram.:f.}
\end{itemize}
\begin{itemize}
\item {Utilização:Ant.}
\end{itemize}
\begin{itemize}
\item {Proveniência:(Do lat. \textunderscore fiducia\textunderscore )}
\end{itemize}
Confiança.
\section{Fivela}
\begin{itemize}
\item {Grp. gram.:f.}
\end{itemize}
\begin{itemize}
\item {Proveniência:(Lat. hypoth. \textunderscore fibella\textunderscore , de \textunderscore fibula\textunderscore )}
\end{itemize}
Peça de metal, que tem uma parte dentada, em que se prende a presilha de certos vestuários, uma fita, uma correia, etc.
\section{Fiveleta}
\begin{itemize}
\item {fónica:lê}
\end{itemize}
\begin{itemize}
\item {Grp. gram.:f.}
\end{itemize}
\begin{itemize}
\item {Proveniência:(De \textunderscore fivela\textunderscore )}
\end{itemize}
Pequena fivela.
Espécie de dança antiga.
\section{Fixa}
\begin{itemize}
\item {Grp. gram.:f.}
\end{itemize}
\begin{itemize}
\item {Proveniência:(De \textunderscore fixo\textunderscore )}
\end{itemize}
Pau, terminado superiormente por uma argola, e que se usa na medição dos terrenos.
Parte de uma dobradiça, que se embute na madeira.
Espécie de colhér dentada, com que os pedreiros introduzem argamassa nas juntas das pedras, em muros de cantaria.
Haste de madeira ou ferro, que se introduz em orifícios praticados no crânio, para estudos de topografia crânio-cerebral.
\section{Fixação}
\begin{itemize}
\item {fónica:csa}
\end{itemize}
\begin{itemize}
\item {Grp. gram.:f.}
\end{itemize}
Acto de fixar.
Operação chímica, com que se torna fixo um corpo volátil.
\section{Fixador}
\begin{itemize}
\item {fónica:csa}
\end{itemize}
\begin{itemize}
\item {Grp. gram.:m.}
\end{itemize}
\begin{itemize}
\item {Proveniência:(De \textunderscore fixar\textunderscore )}
\end{itemize}
Banho, em que se dissolvem as substâncias, que não foram impressionadas pela luz nas matrizes photográphicas.
\section{Fixamente}
\begin{itemize}
\item {fónica:csa}
\end{itemize}
\begin{itemize}
\item {Grp. gram.:adv.}
\end{itemize}
De modo fixo.
\section{Fixante}
\begin{itemize}
\item {fónica:csan}
\end{itemize}
\begin{itemize}
\item {Grp. gram.:adj.}
\end{itemize}
\begin{itemize}
\item {Proveniência:(De \textunderscore fixar\textunderscore )}
\end{itemize}
Fixo.
Que está embebido em outro.
\section{Fixar}
\begin{itemize}
\item {fónica:csar}
\end{itemize}
\begin{itemize}
\item {Grp. gram.:v. t.}
\end{itemize}
\begin{itemize}
\item {Proveniência:(De \textunderscore fixo\textunderscore )}
\end{itemize}
Fazer fixo.
Cravar: \textunderscore fixar uma estaca\textunderscore .
Determinar: \textunderscore fixar o prazo de um pagamento\textunderscore .
Tornar estável, firme.
Reter na memória: \textunderscore fixar uma data\textunderscore .
Fitar: \textunderscore fixar a attenção\textunderscore .
\section{Fixativo}
\begin{itemize}
\item {fónica:csa}
\end{itemize}
\begin{itemize}
\item {Grp. gram.:adj.}
\end{itemize}
Que fixa.
\section{Fixe}
\begin{itemize}
\item {Grp. gram.:m.}
\end{itemize}
\begin{itemize}
\item {Grp. gram.:Adj.}
\end{itemize}
\begin{itemize}
\item {Utilização:Pop.}
\end{itemize}
\begin{itemize}
\item {Utilização:Bras. do N}
\end{itemize}
\begin{itemize}
\item {Proveniência:(De \textunderscore fixar\textunderscore )}
\end{itemize}
Rectângulo de madeira ou ferro, sôbre rodas, para sustentar a máquina do combóio.
O mesmo que \textunderscore fixo\textunderscore .
Compacto; inteiriço.
\section{Fixidade}
\begin{itemize}
\item {fónica:csi}
\end{itemize}
\begin{itemize}
\item {Grp. gram.:f.}
\end{itemize}
Qualidade daquillo que é ou está fixo.
\section{Fixidez}
\begin{itemize}
\item {fónica:csi}
\end{itemize}
\begin{itemize}
\item {Grp. gram.:f.}
\end{itemize}
(V. \textunderscore fixidade\textunderscore , que é fórma preferível)
\section{Fixismo}
\begin{itemize}
\item {fónica:csis}
\end{itemize}
\begin{itemize}
\item {Grp. gram.:m.}
\end{itemize}
\begin{itemize}
\item {Utilização:Neol.}
\end{itemize}
\begin{itemize}
\item {Proveniência:(De \textunderscore fixo\textunderscore )}
\end{itemize}
Apicultura por quadros fixos, (o contrário de mobilismo).
\section{Fixista}
\begin{itemize}
\item {fónica:csis}
\end{itemize}
\begin{itemize}
\item {Grp. gram.:adj.}
\end{itemize}
\begin{itemize}
\item {Grp. gram.:M.}
\end{itemize}
Relativo a fixismo.
Apicultor pelo fixismo.
\section{Fixo}
\begin{itemize}
\item {fónica:cso}
\end{itemize}
\begin{itemize}
\item {Grp. gram.:adj.}
\end{itemize}
\begin{itemize}
\item {Grp. gram.:M.}
\end{itemize}
\begin{itemize}
\item {Proveniência:(Lat. \textunderscore fixus\textunderscore )}
\end{itemize}
Cravado, estável, firme.
Immóvel.
Dominante.
Determinado; aprazado: \textunderscore em dia fixo\textunderscore .
Que não desbota, que não perde a côr.--Nesta accepção, a pronúncia pop. de \textunderscore fixo\textunderscore  dá ao \textunderscore x\textunderscore  o valor natural, e não \textunderscore cs\textunderscore .
Peça, que se não move.
\section{Fixura}
\begin{itemize}
\item {fónica:csu}
\end{itemize}
\begin{itemize}
\item {Grp. gram.:f.}
\end{itemize}
\begin{itemize}
\item {Proveniência:(Lat. \textunderscore fixura\textunderscore )}
\end{itemize}
Fixidez.
\section{Fízar}
\begin{itemize}
\item {Grp. gram.:v. t.}
\end{itemize}
\begin{itemize}
\item {Utilização:Ant.}
\end{itemize}
\begin{itemize}
\item {Grp. gram.:V. i.}
\end{itemize}
Pôr fim a.
Acabar.
(Por \textunderscore finzar\textunderscore , contr. de \textunderscore finalizar\textunderscore )
\section{Flabelação}
\begin{itemize}
\item {Grp. gram.:f.}
\end{itemize}
Acto de flabelar.
\section{Flabelado}
\begin{itemize}
\item {Grp. gram.:adj.}
\end{itemize}
\begin{itemize}
\item {Proveniência:(Lat. \textunderscore flabelatus\textunderscore )}
\end{itemize}
Que tem fórma de leque.
\section{Flabelar}
\begin{itemize}
\item {Grp. gram.:adj.}
\end{itemize}
\begin{itemize}
\item {Proveniência:(Do lat. \textunderscore flabellum\textunderscore )}
\end{itemize}
O mesmo que \textunderscore flabelado\textunderscore .
\section{Flabelar}
\begin{itemize}
\item {Grp. gram.:v. t.}
\end{itemize}
\begin{itemize}
\item {Grp. gram.:V. i.}
\end{itemize}
\begin{itemize}
\item {Proveniência:(Lat. \textunderscore flabellare\textunderscore )}
\end{itemize}
Agitar com o leque (o ar).
Fazer vento com o leque.
\section{Flabelária}
\begin{itemize}
\item {Grp. gram.:f.}
\end{itemize}
\begin{itemize}
\item {Proveniência:(Do lat. \textunderscore flabellum\textunderscore )}
\end{itemize}
Gênero de algas marinhas do Mediterrâneo.
\section{Flabelífero}
\begin{itemize}
\item {Grp. gram.:adj.}
\end{itemize}
\begin{itemize}
\item {Proveniência:(Do lat. \textunderscore flabellum\textunderscore  + \textunderscore ferre\textunderscore )}
\end{itemize}
Que tem leque ou fórma de leque.
Que tem uma parte em fórma de leque.
\section{Flabelifoliado}
\begin{itemize}
\item {Grp. gram.:adj.}
\end{itemize}
\begin{itemize}
\item {Utilização:Bot.}
\end{itemize}
\begin{itemize}
\item {Proveniência:(Do lat. \textunderscore flabellum\textunderscore  + \textunderscore folium\textunderscore )}
\end{itemize}
Que tem fôlhas em fórma de leque.
\section{Flabeliforme}
\begin{itemize}
\item {Grp. gram.:adj.}
\end{itemize}
\begin{itemize}
\item {Proveniência:(Do lat. \textunderscore flabellum\textunderscore  + \textunderscore forma\textunderscore )}
\end{itemize}
Que tem fórma de leque.
\section{Flabelina}
\begin{itemize}
\item {Grp. gram.:f.}
\end{itemize}
Espécie de madrépora.
\section{Flabelípede}
\begin{itemize}
\item {Grp. gram.:adj.}
\end{itemize}
\begin{itemize}
\item {Utilização:Zool.}
\end{itemize}
\begin{itemize}
\item {Proveniência:(Do lat. \textunderscore flabellum\textunderscore  + \textunderscore pes\textunderscore )}
\end{itemize}
Que tem pés em fórma de leque.
\section{Flabellação}
\begin{itemize}
\item {Grp. gram.:f.}
\end{itemize}
Acto de flabellar.
\section{Flabellado}
\begin{itemize}
\item {Grp. gram.:adj.}
\end{itemize}
\begin{itemize}
\item {Proveniência:(Lat. \textunderscore flabelatus\textunderscore )}
\end{itemize}
Que tem fórma de leque.
\section{Flabellar}
\begin{itemize}
\item {Grp. gram.:adj.}
\end{itemize}
\begin{itemize}
\item {Proveniência:(Do lat. \textunderscore flabellum\textunderscore )}
\end{itemize}
O mesmo que \textunderscore flabellado\textunderscore .
\section{Flabellar}
\begin{itemize}
\item {Grp. gram.:v. t.}
\end{itemize}
\begin{itemize}
\item {Grp. gram.:V. i.}
\end{itemize}
\begin{itemize}
\item {Proveniência:(Lat. \textunderscore flabellare\textunderscore )}
\end{itemize}
Agitar com o leque (o ar).
Fazer vento com o leque.
\section{Flabellária}
\begin{itemize}
\item {Grp. gram.:f.}
\end{itemize}
\begin{itemize}
\item {Proveniência:(Do lat. \textunderscore flabellum\textunderscore )}
\end{itemize}
Gênero de algas marinhas do Mediterrâneo.
\section{Flabellífero}
\begin{itemize}
\item {Grp. gram.:adj.}
\end{itemize}
\begin{itemize}
\item {Proveniência:(Do lat. \textunderscore flabellum\textunderscore  + \textunderscore ferre\textunderscore )}
\end{itemize}
Que tem leque ou fórma de leque.
Que tem uma parte em fórma de leque.
\section{Flabellifoliado}
\begin{itemize}
\item {Grp. gram.:adj.}
\end{itemize}
\begin{itemize}
\item {Utilização:Bot.}
\end{itemize}
\begin{itemize}
\item {Proveniência:(Do lat. \textunderscore flabellum\textunderscore  + \textunderscore folium\textunderscore )}
\end{itemize}
Que tem fôlhas em fórma de leque.
\section{Flabelliforme}
\begin{itemize}
\item {Grp. gram.:adj.}
\end{itemize}
\begin{itemize}
\item {Proveniência:(Do lat. \textunderscore flabellum\textunderscore  + \textunderscore forma\textunderscore )}
\end{itemize}
Que tem fórma de leque.
\section{Flabellina}
\begin{itemize}
\item {Grp. gram.:f.}
\end{itemize}
Espécie de madrépora.
\section{Flabellípede}
\begin{itemize}
\item {Grp. gram.:adj.}
\end{itemize}
\begin{itemize}
\item {Utilização:Zool.}
\end{itemize}
\begin{itemize}
\item {Proveniência:(Do lat. \textunderscore flabellum\textunderscore  + \textunderscore pes\textunderscore )}
\end{itemize}
Que tem pés em fórma de leque.
\section{Flabello}
\begin{itemize}
\item {Grp. gram.:m.}
\end{itemize}
\begin{itemize}
\item {Proveniência:(Lat. \textunderscore flabellum\textunderscore )}
\end{itemize}
O mesmo que \textunderscore leque\textunderscore ^1 ou \textunderscore ventarola\textunderscore .
Alfaia ecclesiástica, com que o diácono enxotava as moscas ao celebrante. Cf. Herculano, \textunderscore Opúsc.\textunderscore , VII, 116.
\section{Flabelo}
\begin{itemize}
\item {Grp. gram.:m.}
\end{itemize}
\begin{itemize}
\item {Proveniência:(Lat. \textunderscore flabellum\textunderscore )}
\end{itemize}
O mesmo que \textunderscore leque\textunderscore ^1 ou \textunderscore ventarola\textunderscore .
Alfaia eclesiástica, com que o diácono enxotava as moscas ao celebrante. Cf. Herculano, \textunderscore Opúsc.\textunderscore , VII, 116.
\section{Flaccidez}
\begin{itemize}
\item {Grp. gram.:f.}
\end{itemize}
Estado ou qualidade daquillo que é fláccido.
Doença epidêmica do bicho da seda. Cf. \textunderscore Inquér. Industr.\textunderscore , p. II, l. III, 73.
\section{Fláccido}
\begin{itemize}
\item {Grp. gram.:adj.}
\end{itemize}
\begin{itemize}
\item {Proveniência:(Lat. \textunderscore flaccidus\textunderscore )}
\end{itemize}
Brando, lânguido.
Adiposo.
\section{Flacidez}
\begin{itemize}
\item {Grp. gram.:f.}
\end{itemize}
Estado ou qualidade daquilo que é flácido.
Doença epidêmica do bicho da seda. Cf. \textunderscore Inquér. Industr.\textunderscore , p. II, l. III, 73.
\section{Flácido}
\begin{itemize}
\item {Grp. gram.:adj.}
\end{itemize}
\begin{itemize}
\item {Proveniência:(Lat. \textunderscore flaccidus\textunderscore )}
\end{itemize}
Brando, lânguido.
Adiposo.
\section{Flacúrtia}
\begin{itemize}
\item {Grp. gram.:f.}
\end{itemize}
\begin{itemize}
\item {Proveniência:(De \textunderscore Flacourte\textunderscore , n. p.)}
\end{itemize}
Gênero de árvores tropicaes.
\section{Flacurtiáceas}
\begin{itemize}
\item {Grp. gram.:f. pl.}
\end{itemize}
\begin{itemize}
\item {Proveniência:(De \textunderscore flacurtiáceo\textunderscore )}
\end{itemize}
Família de plantas, que têm por typo a flacúrtia.
\section{Flacurtiáceo}
\begin{itemize}
\item {Grp. gram.:adj.}
\end{itemize}
Relativo ou semelhante á flacúrtia.
\section{Flacurtianas}
\begin{itemize}
\item {Grp. gram.:f. pl.}
\end{itemize}
O mesmo que \textunderscore flacurtiáceas\textunderscore .
\section{Flagávil}
\begin{itemize}
\item {Grp. gram.:m.}
\end{itemize}
\begin{itemize}
\item {Utilização:Ant.}
\end{itemize}
Imagem esculpida.
\section{Flagelação}
\begin{itemize}
\item {Grp. gram.:f.}
\end{itemize}
\begin{itemize}
\item {Proveniência:(Lat. \textunderscore flagellatio\textunderscore )}
\end{itemize}
Acto ou efeito de flagelar.
Sofrimento.
Suplício.
\section{Flagelador}
\begin{itemize}
\item {Grp. gram.:adj.}
\end{itemize}
\begin{itemize}
\item {Grp. gram.:M.}
\end{itemize}
Que flagela.
Aquele que flagela.
\section{Flagelante}
\begin{itemize}
\item {Grp. gram.:adj.}
\end{itemize}
Que flagela. Cp. Camillo, \textunderscore Pombal\textunderscore , 192.
\section{Flagelar}
\begin{itemize}
\item {Grp. gram.:v. t.}
\end{itemize}
\begin{itemize}
\item {Proveniência:(Lat. \textunderscore flagellare\textunderscore )}
\end{itemize}
Bater com flagelo.
Castigar.
Torturar.
Incomodar, enfadar: \textunderscore não me flageles com mais pedidos\textunderscore .
\section{Flagelativo}
\begin{itemize}
\item {Grp. gram.:adj.}
\end{itemize}
Que flagela.
Próprio para flagelar.
\section{Flagelífero}
\begin{itemize}
\item {Grp. gram.:adj.}
\end{itemize}
\begin{itemize}
\item {Utilização:Bot.}
\end{itemize}
\begin{itemize}
\item {Proveniência:(Do lat. \textunderscore flagellum\textunderscore  + \textunderscore ferre\textunderscore )}
\end{itemize}
Que tem filamentos compridos e muito delgados.
\section{Flageliforme}
\begin{itemize}
\item {Grp. gram.:adj.}
\end{itemize}
\begin{itemize}
\item {Utilização:Bot.}
\end{itemize}
\begin{itemize}
\item {Proveniência:(Do lat. \textunderscore flagellum\textunderscore  + \textunderscore forma\textunderscore )}
\end{itemize}
Diz-se das raízes, troncos, etc., que são compridos, delicados.
\section{Flagellação}
\begin{itemize}
\item {Grp. gram.:f.}
\end{itemize}
\begin{itemize}
\item {Proveniência:(Lat. \textunderscore flagellatio\textunderscore )}
\end{itemize}
Acto ou effeito de flagellar.
Soffrimento.
Supplício.
\section{Flagellador}
\begin{itemize}
\item {Grp. gram.:adj.}
\end{itemize}
\begin{itemize}
\item {Grp. gram.:M.}
\end{itemize}
Que flagella.
Aquelle que flagella.
\section{Flagellante}
\begin{itemize}
\item {Grp. gram.:adj.}
\end{itemize}
Que flagella. Cp. Camillo, \textunderscore Pombal\textunderscore , 192.
\section{Flagellar}
\begin{itemize}
\item {Grp. gram.:v. t.}
\end{itemize}
\begin{itemize}
\item {Proveniência:(Lat. \textunderscore flagellare\textunderscore )}
\end{itemize}
Bater com flagello.
Castigar.
Torturar.
Incommodar, enfadar: \textunderscore não me flagelles com mais pedidos\textunderscore .
\section{Flagellativo}
\begin{itemize}
\item {Grp. gram.:adj.}
\end{itemize}
Que flagella.
Próprio para flagellar.
\section{Flagellífero}
\begin{itemize}
\item {Grp. gram.:adj.}
\end{itemize}
\begin{itemize}
\item {Utilização:Bot.}
\end{itemize}
\begin{itemize}
\item {Proveniência:(Do lat. \textunderscore flagellum\textunderscore  + \textunderscore ferre\textunderscore )}
\end{itemize}
Que tem filamentos compridos e muito delgados.
\section{Flagelliforme}
\begin{itemize}
\item {Grp. gram.:adj.}
\end{itemize}
\begin{itemize}
\item {Utilização:Bot.}
\end{itemize}
\begin{itemize}
\item {Proveniência:(Do lat. \textunderscore flagellum\textunderscore  + \textunderscore forma\textunderscore )}
\end{itemize}
Diz-se das raízes, troncos, etc., que são compridos, delicados.
\section{Flagello}
\begin{itemize}
\item {Grp. gram.:m.}
\end{itemize}
\begin{itemize}
\item {Utilização:Fig.}
\end{itemize}
\begin{itemize}
\item {Proveniência:(Lat. \textunderscore flagellum\textunderscore )}
\end{itemize}
Azorrague para açoitar.
Chicote.
Calamidade.
Castigo.
Peste.
Tortura; coisa ou pessôa que incommoda ou apoquenta: \textunderscore aquelle credor é o teu flagello\textunderscore .
\section{Flagelo}
\begin{itemize}
\item {Grp. gram.:m.}
\end{itemize}
\begin{itemize}
\item {Utilização:Fig.}
\end{itemize}
\begin{itemize}
\item {Proveniência:(Lat. \textunderscore flagellum\textunderscore )}
\end{itemize}
Azorrague para açoitar.
Chicote.
Calamidade.
Castigo.
Peste.
Tortura; coisa ou pessôa que incomoda ou apoquenta: \textunderscore aquele credor é o teu flagelo\textunderscore .
\section{Flagício}
\begin{itemize}
\item {Grp. gram.:m.}
\end{itemize}
\begin{itemize}
\item {Proveniência:(Lat. \textunderscore flagitium\textunderscore )}
\end{itemize}
Acção criminosa ou infame.
Ignomínia.
\section{Flagiciosamente}
\begin{itemize}
\item {Grp. gram.:adv.}
\end{itemize}
De modo flagicioso.
\section{Flagicioso}
\begin{itemize}
\item {Grp. gram.:adj.}
\end{itemize}
\begin{itemize}
\item {Proveniência:(Lat. \textunderscore flagitiosus\textunderscore )}
\end{itemize}
Facinoroso.
Que commeteu grande delicto ou infâmia.
\section{Flagrância}
\begin{itemize}
\item {Grp. gram.:f.}
\end{itemize}
\begin{itemize}
\item {Proveniência:(Lat. \textunderscore flagrantia\textunderscore )}
\end{itemize}
Estado daquillo que é flagrante.
Momento, em que se dá um acto flagrante.
\section{Flagrante}
\begin{itemize}
\item {Grp. gram.:adj.}
\end{itemize}
\begin{itemize}
\item {Grp. gram.:M.}
\end{itemize}
\begin{itemize}
\item {Utilização:Pop.}
\end{itemize}
\begin{itemize}
\item {Grp. gram.:Loc. adv.}
\end{itemize}
\begin{itemize}
\item {Proveniência:(Lat. \textunderscore flagrans\textunderscore )}
\end{itemize}
Ardente.
Acalorado.
Evidente: \textunderscore verdade flagrante\textunderscore .
Que se dá ou se pratíca no próprio momento: \textunderscore crime flagrante\textunderscore .
Em que é observado ou surprehendido.
Ensejo, momento: \textunderscore neste flagrante, começou a trovejar\textunderscore .
\textunderscore Em flagrante\textunderscore , na própria occasião em que se praticou um acto: \textunderscore o criminoso foi apanhado em flagrante\textunderscore .
\section{Flagrar}
\begin{itemize}
\item {Grp. gram.:v. i.}
\end{itemize}
\begin{itemize}
\item {Utilização:Bras}
\end{itemize}
\begin{itemize}
\item {Utilização:Neol.}
\end{itemize}
\begin{itemize}
\item {Proveniência:(Lat. \textunderscore flagrare\textunderscore )}
\end{itemize}
Arder, inflamar-se.
\section{Flaino}
\begin{itemize}
\item {Grp. gram.:m.}
\end{itemize}
\begin{itemize}
\item {Utilização:P. us.}
\end{itemize}
Acto de passear sem destino, ao acaso, por mera diversão:«\textunderscore ...deu-me doze vintens e mandou-me a flaino até á noite\textunderscore ». Bocage, \textunderscore Gil Brás\textunderscore , trad., I, 236, (ed. de 1836) Cf. Arn. Gama, \textunderscore Últ. Dona\textunderscore , 21.
(Cp. fr. \textunderscore flaner\textunderscore )
\section{Flama}
\begin{itemize}
\item {Grp. gram.:f.}
\end{itemize}
\begin{itemize}
\item {Proveniência:(Lat. \textunderscore flamma\textunderscore )}
\end{itemize}
Chama.
Ardor.
Vivacidade.
\section{Flamão}
\begin{itemize}
\item {Grp. gram.:m.}
\end{itemize}
Feltro de pêlo comprido, empregado em chapéus, especialmente em chapéus de ecclesiásticos.
(Provavelmente, contr. de \textunderscore fulamão\textunderscore , de \textunderscore fula\textunderscore )
\section{Flame}
\begin{itemize}
\item {Grp. gram.:m.}
\end{itemize}
Instrumento, para sangrar cavallos.--Assim dizem os nossos diccion., citando como exemplo o lat. \textunderscore flamen\textunderscore , em que não vejo analogia com o significado português, inclinando-me a que a fórma exacta é \textunderscore fleme\textunderscore .(V.fleme)
\section{Flamenga}
\begin{itemize}
\item {Grp. gram.:f.}
\end{itemize}
Variedade de pêra portuguesa.
\section{Flamengo}
\begin{itemize}
\item {Grp. gram.:m.}
\end{itemize}
\begin{itemize}
\item {Grp. gram.:Loc.}
\end{itemize}
\begin{itemize}
\item {Utilização:pop.}
\end{itemize}
\begin{itemize}
\item {Grp. gram.:Adj.}
\end{itemize}
\begin{itemize}
\item {Proveniência:(Do lat. hyp. \textunderscore flamenicus\textunderscore )}
\end{itemize}
Aquelle que é natural de Flandres.
Idioma dêsse país.
\textunderscore Vêr flamengos á meia-noite\textunderscore , vêr-se embaraçado ou perdido, vêr-se grego.
Relativo a Flandres.
\section{Flamengo}
\begin{itemize}
\item {Grp. gram.:m.}
\end{itemize}
Espécie de falcão.
O mesmo que \textunderscore flamingo\textunderscore .
\section{Flaminato}
\begin{itemize}
\item {Grp. gram.:m.}
\end{itemize}
O mesmo que \textunderscore flamínio\textunderscore . Cf. Castilho, \textunderscore Fastos\textunderscore , II, 608.
\section{Flâmine}
\begin{itemize}
\item {Grp. gram.:m.}
\end{itemize}
\begin{itemize}
\item {Proveniência:(Lat. \textunderscore flamen\textunderscore )}
\end{itemize}
Antigo sacerdote romano.
\section{Flamingo}
\begin{itemize}
\item {Grp. gram.:m.}
\end{itemize}
\begin{itemize}
\item {Proveniência:(Do cast. \textunderscore flamenco\textunderscore )}
\end{itemize}
Ave pernalta (\textunderscore phaenicopterus roseus\textunderscore , Pall.).
\section{Flamínia}
\begin{itemize}
\item {Grp. gram.:f.}
\end{itemize}
\begin{itemize}
\item {Proveniência:(Lat. \textunderscore flamínia\textunderscore )}
\end{itemize}
Mulher do flâmine.
Sacerdotisa, que acompanhava e auxiliava o flâmine nos sacrifícios.
\section{Flamínica}
\begin{itemize}
\item {Grp. gram.:f.}
\end{itemize}
\begin{itemize}
\item {Proveniência:(Lat. \textunderscore flaminica\textunderscore )}
\end{itemize}
Mulher do flâmine, a qual tinha sôbre as demais mulheres a vantagem de que seu marido se não podia divorciar della.
\section{Flamínio}
\begin{itemize}
\item {Grp. gram.:m.}
\end{itemize}
\begin{itemize}
\item {Proveniência:(Lat. \textunderscore flaminium\textunderscore )}
\end{itemize}
Dignidade de flâmine.
\section{Flamma}
\begin{itemize}
\item {Grp. gram.:f.}
\end{itemize}
\begin{itemize}
\item {Proveniência:(Lat. \textunderscore flamma\textunderscore )}
\end{itemize}
Chamma.
Ardor.
Vivacidade.
\section{Flamância}
\begin{itemize}
\item {Grp. gram.:f.}
\end{itemize}
Qualidade daquilo que é flamante.
\section{Flamante}
\begin{itemize}
\item {Grp. gram.:adj.}
\end{itemize}
\begin{itemize}
\item {Utilização:Fig.}
\end{itemize}
\begin{itemize}
\item {Proveniência:(Lat. \textunderscore flammans\textunderscore )}
\end{itemize}
Chamejante.
Brilhante.
Ostentoso; esplêndido: \textunderscore festa flamante\textunderscore .
\section{Flamear}
\begin{itemize}
\item {Grp. gram.:v. i.}
\end{itemize}
O mesmo que \textunderscore flamejar\textunderscore .
\section{Flamejado}
\begin{itemize}
\item {Grp. gram.:adj.}
\end{itemize}
\begin{itemize}
\item {Utilização:Bot.}
\end{itemize}
\begin{itemize}
\item {Proveniência:(De \textunderscore flamejar\textunderscore )}
\end{itemize}
Variegado, em fórma de chama.
\section{Flamejamento}
\begin{itemize}
\item {Grp. gram.:m.}
\end{itemize}
Acto de flamejar.
\section{Flamejante}
\begin{itemize}
\item {Grp. gram.:adj.}
\end{itemize}
Que flameja.
Ostentoso; vistoso; flamante.
\section{Flamejar}
\begin{itemize}
\item {Grp. gram.:v. i.}
\end{itemize}
\begin{itemize}
\item {Proveniência:(De \textunderscore flama\textunderscore )}
\end{itemize}
Lançar chamas.
Arder.
Resplandecer.
\section{Flâmeo}
\begin{itemize}
\item {Grp. gram.:adj.}
\end{itemize}
\begin{itemize}
\item {Grp. gram.:M.}
\end{itemize}
\begin{itemize}
\item {Proveniência:(Lat. \textunderscore flammens\textunderscore )}
\end{itemize}
Flamejante.
Antigo véu de noivas, entre os Romanos.
Véu, com que as damas romanas velavam o rosto, quando saíam de casa.
\section{Flamífero}
\begin{itemize}
\item {Grp. gram.:adj.}
\end{itemize}
\begin{itemize}
\item {Proveniência:(Do lat. \textunderscore flamma\textunderscore  + \textunderscore ferre\textunderscore )}
\end{itemize}
Que apresenta chamas.
\section{Flamifervente}
\begin{itemize}
\item {Grp. gram.:adj.}
\end{itemize}
\begin{itemize}
\item {Utilização:Poét.}
\end{itemize}
\begin{itemize}
\item {Proveniência:(Do lat. \textunderscore flamma\textunderscore  + \textunderscore ferrens\textunderscore )}
\end{itemize}
Que ferve, erguendo chamas.
Que chameja, fervendo. Cf. \textunderscore Agostinheida\textunderscore , 56.
\section{Flamígero}
\begin{itemize}
\item {Grp. gram.:adj.}
\end{itemize}
\begin{itemize}
\item {Proveniência:(Lat. \textunderscore flammiger\textunderscore )}
\end{itemize}
O mesmo que \textunderscore flamífero\textunderscore .
\section{Flamipotente}
\begin{itemize}
\item {Grp. gram.:adj.}
\end{itemize}
\begin{itemize}
\item {Proveniência:(Lat. \textunderscore flammipotens\textunderscore )}
\end{itemize}
Poderoso em chamas, (epíteto de Vulcano).
\section{Flamispirante}
\begin{itemize}
\item {Grp. gram.:adj.}
\end{itemize}
\begin{itemize}
\item {Proveniência:(Do lat. \textunderscore flamma\textunderscore  + \textunderscore spirans\textunderscore )}
\end{itemize}
Que respira chamas.
\section{Flamívolo}
\begin{itemize}
\item {Grp. gram.:adj.}
\end{itemize}
\begin{itemize}
\item {Utilização:Poét.}
\end{itemize}
\begin{itemize}
\item {Proveniência:(Lat. \textunderscore flammivolus\textunderscore )}
\end{itemize}
Que vôa, lançando chamas.
\section{Flamívomo}
\begin{itemize}
\item {Grp. gram.:adj.}
\end{itemize}
\begin{itemize}
\item {Proveniência:(Lat. \textunderscore flammivomus\textunderscore )}
\end{itemize}
Que vomita chamas.
\section{Flammância}
\begin{itemize}
\item {Grp. gram.:f.}
\end{itemize}
Qualidade daquillo que é flammante.
\section{Flammante}
\begin{itemize}
\item {Grp. gram.:adj.}
\end{itemize}
\begin{itemize}
\item {Utilização:Fig.}
\end{itemize}
\begin{itemize}
\item {Proveniência:(Lat. \textunderscore flammans\textunderscore )}
\end{itemize}
Chammejante.
Brilhante.
Ostentoso; esplêndido: \textunderscore festa flammante\textunderscore .
\section{Flammear}
\begin{itemize}
\item {Grp. gram.:v. i.}
\end{itemize}
O mesmo que \textunderscore flammejar\textunderscore .
\section{Flammejado}
\begin{itemize}
\item {Grp. gram.:adj.}
\end{itemize}
\begin{itemize}
\item {Utilização:Bot.}
\end{itemize}
\begin{itemize}
\item {Proveniência:(De \textunderscore flammejar\textunderscore )}
\end{itemize}
Variegado, em fórma de chamma.
\section{Flammejamento}
\begin{itemize}
\item {Grp. gram.:m.}
\end{itemize}
Acto de flammejar.
\section{Flammejante}
\begin{itemize}
\item {Grp. gram.:adj.}
\end{itemize}
Que flammeja.
Ostentoso; vistoso; flammante.
\section{Flammejar}
\begin{itemize}
\item {Grp. gram.:v. i.}
\end{itemize}
\begin{itemize}
\item {Proveniência:(De \textunderscore flamma\textunderscore )}
\end{itemize}
Lançar chammas.
Arder.
Resplandecer.
\section{Flâmmeo}
\begin{itemize}
\item {Grp. gram.:adj.}
\end{itemize}
\begin{itemize}
\item {Grp. gram.:M.}
\end{itemize}
\begin{itemize}
\item {Proveniência:(Lat. \textunderscore flammens\textunderscore )}
\end{itemize}
Flammejante.
Antigo véu de noivas, entre os Romanos.
Véu, com que as damas romanas velavam o rosto, quando saíam de casa.
\section{Flammífero}
\begin{itemize}
\item {Grp. gram.:adj.}
\end{itemize}
\begin{itemize}
\item {Proveniência:(Do lat. \textunderscore flamma\textunderscore  + \textunderscore ferre\textunderscore )}
\end{itemize}
Que apresenta chammas.
\section{Flammifervente}
\begin{itemize}
\item {Grp. gram.:adj.}
\end{itemize}
\begin{itemize}
\item {Utilização:Poét.}
\end{itemize}
\begin{itemize}
\item {Proveniência:(Do lat. \textunderscore flamma\textunderscore  + \textunderscore ferrens\textunderscore )}
\end{itemize}
Que ferve, erguendo chammas.
Que chammeja, fervendo. Cf. \textunderscore Agostinheida\textunderscore , 56.
\section{Flammígero}
\begin{itemize}
\item {Grp. gram.:adj.}
\end{itemize}
\begin{itemize}
\item {Proveniência:(Lat. \textunderscore flammiger\textunderscore )}
\end{itemize}
O mesmo que \textunderscore flammífero\textunderscore .
\section{Flammipotente}
\begin{itemize}
\item {Grp. gram.:adj.}
\end{itemize}
\begin{itemize}
\item {Proveniência:(Lat. \textunderscore flammipotens\textunderscore )}
\end{itemize}
Poderoso em chammas, (epítheto de Vulcano).
\section{Flammispirante}
\begin{itemize}
\item {Grp. gram.:adj.}
\end{itemize}
\begin{itemize}
\item {Proveniência:(Do lat. \textunderscore flamma\textunderscore  + \textunderscore spirans\textunderscore )}
\end{itemize}
Que respira chammas.
\section{Flammívolo}
\begin{itemize}
\item {Grp. gram.:adj.}
\end{itemize}
\begin{itemize}
\item {Utilização:Poét.}
\end{itemize}
\begin{itemize}
\item {Proveniência:(Lat. \textunderscore flammivolus\textunderscore )}
\end{itemize}
Que vôa, lançando chammas.
\section{Flammívomo}
\begin{itemize}
\item {Grp. gram.:adj.}
\end{itemize}
\begin{itemize}
\item {Proveniência:(Lat. \textunderscore flammivomus\textunderscore )}
\end{itemize}
Que vomita chammas.
\section{Flammula}
\begin{itemize}
\item {Grp. gram.:f.}
\end{itemize}
\begin{itemize}
\item {Utilização:Náut.}
\end{itemize}
\begin{itemize}
\item {Proveniência:(Lat. \textunderscore flammula\textunderscore )}
\end{itemize}
Pequena chamma.
Faixa ou tira, com a extremidade geralmente farpada, e que no topo dos mastros serve para sinaes e, mais vezes, para ornato.
\section{Flamula}
\begin{itemize}
\item {Grp. gram.:f.}
\end{itemize}
\begin{itemize}
\item {Utilização:Náut.}
\end{itemize}
\begin{itemize}
\item {Proveniência:(Lat. \textunderscore flammula\textunderscore )}
\end{itemize}
Pequena chama.
Faixa ou tira, com a extremidade geralmente farpada, e que no topo dos mastros serve para sinaes e, mais vezes, para ornato.
\section{Flanador}
\begin{itemize}
\item {Grp. gram.:m.}
\end{itemize}
\begin{itemize}
\item {Utilização:Gal}
\end{itemize}
Aquelle que flana.
\section{Flanar}
\begin{itemize}
\item {Grp. gram.:v. i.}
\end{itemize}
\begin{itemize}
\item {Utilização:Gal}
\end{itemize}
\begin{itemize}
\item {Proveniência:(Fr. \textunderscore flaner\textunderscore )}
\end{itemize}
Larear; passear ociosamente.
\section{Flanco}
\begin{itemize}
\item {Grp. gram.:m.}
\end{itemize}
\begin{itemize}
\item {Utilização:Gal}
\end{itemize}
\begin{itemize}
\item {Proveniência:(Fr. \textunderscore flanc\textunderscore )}
\end{itemize}
Espaço entre o baluarte e a cortina, em fortificações.
Lado de um exército ou de um corpo de tropas.
Ponto accessível, expugnável.
Ilharga.
Costado de embarcação.
Lado.
\section{Flandeiro}
\begin{itemize}
\item {Grp. gram.:m.}
\end{itemize}
\begin{itemize}
\item {Utilização:Bras. do N}
\end{itemize}
Aquelle que faz obras de fôlhas de Flandres; funileiro.
(Por \textunderscore flandreiro\textunderscore , de \textunderscore Flandres\textunderscore ^1)
\section{Flandres}
\begin{itemize}
\item {Grp. gram.:m.}
\end{itemize}
\begin{itemize}
\item {Utilização:Bras. do N}
\end{itemize}
\begin{itemize}
\item {Utilização:Fig.}
\end{itemize}
\begin{itemize}
\item {Proveniência:(De \textunderscore Flandres\textunderscore , n. p.)}
\end{itemize}
Fôlha de Flandres; lata; fôlha de ferro estanhado.
Sabre de polícia.
\section{Flandres}
\begin{itemize}
\item {Grp. gram.:m.}
\end{itemize}
(?):«\textunderscore ...Não há melhor flandres! vida alegre e dissipada para uns, e quem trabalha que pague os desvarios dos outros.\textunderscore »H. O'Neill, \textunderscore Fabulário\textunderscore , 878.
\section{Flandrisco}
\begin{itemize}
\item {Grp. gram.:adj.}
\end{itemize}
\begin{itemize}
\item {Grp. gram.:M.}
\end{itemize}
Relativo a Flandres.
Habitante de Flandres; flamengo.
\section{Flanela}
\begin{itemize}
\item {Grp. gram.:f.}
\end{itemize}
\begin{itemize}
\item {Proveniência:(It. \textunderscore flanella\textunderscore )}
\end{itemize}
Tecido de lan, menos encorpado que a baetilha.
Tecido de algodão que imita aquelle.
\section{Flanquear}
\begin{itemize}
\item {Grp. gram.:v. t.}
\end{itemize}
\begin{itemize}
\item {Proveniência:(De \textunderscore flanco\textunderscore )}
\end{itemize}
Atacar de lado.
Marchar ao lado de, parallelamente.
Defender.
Tornar defensável.
\section{Flato}
\begin{itemize}
\item {Grp. gram.:m.}
\end{itemize}
\begin{itemize}
\item {Proveniência:(Lat. \textunderscore flatus\textunderscore )}
\end{itemize}
Flatulência.
Ventosidade.
Hysterismo.
\section{Flatoso}
\begin{itemize}
\item {Grp. gram.:adj.}
\end{itemize}
Que causa flatos.
\section{Flatulência}
\begin{itemize}
\item {Grp. gram.:f.}
\end{itemize}
\begin{itemize}
\item {Proveniência:(Lat. \textunderscore flactulentia\textunderscore . Cf. \textunderscore flactus\textunderscore )}
\end{itemize}
Ar, introduzido no estômago ou nos conductores do sangue.
Ventosidade.
Hysterismo.
\section{Flatulento}
\begin{itemize}
\item {Grp. gram.:adj.}
\end{itemize}
\begin{itemize}
\item {Proveniência:(Do rad. de \textunderscore flutulência\textunderscore )}
\end{itemize}
Que produz flatulência.
Relativo a flatulência: \textunderscore incommodos flatulentos\textunderscore .
\section{Flatuloso}
\begin{itemize}
\item {Grp. gram.:adj.}
\end{itemize}
Que tem flatos.
(Cp. \textunderscore flatulência\textunderscore )
\section{Flatuosidade}
\begin{itemize}
\item {Grp. gram.:f.}
\end{itemize}
\begin{itemize}
\item {Proveniência:(De \textunderscore flatuoso\textunderscore )}
\end{itemize}
O mesmo que \textunderscore flatulência\textunderscore .
\section{Flatuoso}
\begin{itemize}
\item {Grp. gram.:adj.}
\end{itemize}
\begin{itemize}
\item {Proveniência:(De \textunderscore flato\textunderscore )}
\end{itemize}
O mesmo que \textunderscore flatuloso\textunderscore .
\section{Flauta}
\begin{itemize}
\item {Grp. gram.:f.}
\end{itemize}
\begin{itemize}
\item {Grp. gram.:Pl.}
\end{itemize}
\begin{itemize}
\item {Utilização:Fam.}
\end{itemize}
\begin{itemize}
\item {Grp. gram.:M.}
\end{itemize}
\begin{itemize}
\item {Proveniência:(Do lat. \textunderscore flata\textunderscore )}
\end{itemize}
Instrumento músico de sopro, cylíndrico e sem palheta.
Pífaro.
Utensílio de ferreiro, mais ou menos boleado, sôbre o qual se encurvam e se alisam certas peças.
Pernas delgadas.
Aquelle, que, num concêrto, toca flauta.
\section{Flautar}
\begin{itemize}
\item {Grp. gram.:v. t.}
\end{itemize}
\begin{itemize}
\item {Proveniência:(De \textunderscore flauta\textunderscore )}
\end{itemize}
O mesmo que \textunderscore aflautar\textunderscore .
Assobiar.
\section{Flautear}
\begin{itemize}
\item {Grp. gram.:v. i.}
\end{itemize}
Tocar flauta.
\section{Flauteira}
\begin{itemize}
\item {Grp. gram.:f.}
\end{itemize}
\begin{itemize}
\item {Proveniência:(De \textunderscore flauteiro\textunderscore )}
\end{itemize}
Mulher, que toca flauta.
\section{Flauteiro}
\begin{itemize}
\item {Grp. gram.:m.}
\end{itemize}
Tocador de flauta. Cf. Castilho, \textunderscore Geórgicas\textunderscore , 91.
\section{Flautim}
\begin{itemize}
\item {Grp. gram.:m.}
\end{itemize}
Pequena flauta.
\section{Flautista}
\begin{itemize}
\item {Grp. gram.:m.}
\end{itemize}
Tocador ou fabricante de flautas.
\section{Flavéria}
\begin{itemize}
\item {Grp. gram.:f.}
\end{itemize}
\begin{itemize}
\item {Proveniência:(Do lat. \textunderscore flavus\textunderscore , amarello)}
\end{itemize}
Gênero de plantas compostas.
\section{Flavescente}
\begin{itemize}
\item {Grp. gram.:adj.}
\end{itemize}
\begin{itemize}
\item {Proveniência:(Lat. \textunderscore flavescens\textunderscore )}
\end{itemize}
Que amarelece.
Que enloirece.
Que se torna flavo.
\section{Flavescer}
\begin{itemize}
\item {Grp. gram.:v. i.}
\end{itemize}
\begin{itemize}
\item {Proveniência:(Lat. \textunderscore flavescere\textunderscore )}
\end{itemize}
Tornar-se flavo.
\section{Flavibico}
\begin{itemize}
\item {Grp. gram.:adj.}
\end{itemize}
\begin{itemize}
\item {Utilização:Des.}
\end{itemize}
\begin{itemize}
\item {Proveniência:(De \textunderscore flavo\textunderscore  + \textunderscore bico\textunderscore )}
\end{itemize}
Que tem bico amarelo. Cf. Filinto, IX, 55.
\section{Flaviense}
\begin{itemize}
\item {Grp. gram.:adj.}
\end{itemize}
\begin{itemize}
\item {Grp. gram.:M.}
\end{itemize}
\begin{itemize}
\item {Proveniência:(Lat. \textunderscore flaviensis\textunderscore )}
\end{itemize}
Relativo a Chaves.
Indivíduo, natural de Chaves.
\section{Flavífluo}
\begin{itemize}
\item {Grp. gram.:adj.}
\end{itemize}
\begin{itemize}
\item {Proveniência:(Do lat. \textunderscore flavus\textunderscore  + \textunderscore fluere\textunderscore )}
\end{itemize}
Diz-se dos rios que correm sobre areias doiradas ou amareladas. Cf. Filinto, XVI, 303.
\section{Flávio}
\begin{itemize}
\item {Grp. gram.:adj.}
\end{itemize}
\begin{itemize}
\item {Proveniência:(Lat. \textunderscore flavius\textunderscore )}
\end{itemize}
Relativo á família dos Flávios, na Roma antiga.
Diz-se especialmente de uma dynastia imperial, começada em Vespasiano.
\section{Flavípede}
\begin{itemize}
\item {Grp. gram.:adj.}
\end{itemize}
\begin{itemize}
\item {Utilização:Zool.}
\end{itemize}
\begin{itemize}
\item {Proveniência:(Do lat. \textunderscore flavus\textunderscore  + \textunderscore pes\textunderscore , \textunderscore pedis\textunderscore )}
\end{itemize}
Que tem pés amarelos ou amarelados. Cf. Filinto, XVI, 231.
\section{Flavo}
\begin{itemize}
\item {Grp. gram.:adj.}
\end{itemize}
\begin{itemize}
\item {Proveniência:(Lat. \textunderscore flavus\textunderscore )}
\end{itemize}
Loiro.
Fulvo.
Que tem a côr das espigas maduras.
Que tem a côr do oiro.
\section{Flavor}
\begin{itemize}
\item {Grp. gram.:m.}
\end{itemize}
\begin{itemize}
\item {Utilização:Enol.}
\end{itemize}
\begin{itemize}
\item {Proveniência:(De \textunderscore flavo\textunderscore )}
\end{itemize}
Qualidade do vinho, que tem côr amarelada.
\section{Flébil}
\begin{itemize}
\item {Grp. gram.:adj.}
\end{itemize}
\begin{itemize}
\item {Proveniência:(Lat. \textunderscore flebilis\textunderscore )}
\end{itemize}
Lacrimoso; plangente: \textunderscore súpplicas flébeis\textunderscore .
\section{Flecha}
\begin{itemize}
\item {Grp. gram.:f.}
\end{itemize}
Arma offensiva, composta de uma haste, terminada em ferro triangular.
Seta.
Objecto em fórma de seta.
Parte do raio perpendicular á corda, entre esta e o arco, em Geometria.
Haste ou peça pyramidal, que termina superior e exteriormente alguns edifícios.
(Cast. \textunderscore flecha\textunderscore )
\section{Flectir}
\begin{itemize}
\item {Grp. gram.:v. t.}
\end{itemize}
\begin{itemize}
\item {Proveniência:(Lat. \textunderscore flectere\textunderscore )}
\end{itemize}
Dobrar, fazer a flexão de: \textunderscore depois de flectir o tronco...\textunderscore 
\section{Fleima}
\begin{itemize}
\item {Grp. gram.:f.}
\end{itemize}
Impassibilidade, paciência, pachorra.
(Cp. \textunderscore fleuma\textunderscore )
\section{Fleimão}
\begin{itemize}
\item {Grp. gram.:m.}
\end{itemize}
\begin{itemize}
\item {Proveniência:(Do lat. \textunderscore phlegmone\textunderscore )}
\end{itemize}
Inflammação do tecido cellular.
\section{Fleimoso}
\begin{itemize}
\item {Grp. gram.:adj.}
\end{itemize}
Que tem carácter de fleimão.
\section{Flema}
\begin{itemize}
\item {Grp. gram.:f.}
\end{itemize}
\begin{itemize}
\item {Utilização:Ant.}
\end{itemize}
O mesmo que \textunderscore fleuma\textunderscore .
\section{Fleme}
\begin{itemize}
\item {Grp. gram.:m.}
\end{itemize}
\begin{itemize}
\item {Utilização:Bras. da Baía}
\end{itemize}
O mesmo ou melhor que \textunderscore flame\textunderscore .
(Cp. cast. \textunderscore fleme\textunderscore  e fr. ant. \textunderscore fleme\textunderscore )
\section{Flente}
\begin{itemize}
\item {Grp. gram.:adj.}
\end{itemize}
Que chora; lastimoso. Cf. Herculano, \textunderscore Cister\textunderscore , II, 207.
\section{Fléoto}
\begin{itemize}
\item {Grp. gram.:m.}
\end{itemize}
\begin{itemize}
\item {Proveniência:(Do gr. \textunderscore phleos\textunderscore )}
\end{itemize}
Gênero de plantas gramíneas.
\section{Fletaço}
\begin{itemize}
\item {Grp. gram.:m.}
\end{itemize}
Flete grande.
\section{Flete}
\begin{itemize}
\item {Grp. gram.:m.}
\end{itemize}
\begin{itemize}
\item {Utilização:Bras. do S}
\end{itemize}
Cavallo formoso e arreado com luxo.
\section{Fléu}
\begin{itemize}
\item {Grp. gram.:m.}
\end{itemize}
(Alter. vulgar de \textunderscore pheélu\textunderscore )
\section{Fleuma}
\begin{itemize}
\item {Grp. gram.:m.  e  f.}
\end{itemize}
\begin{itemize}
\item {Utilização:Fig.}
\end{itemize}
\begin{itemize}
\item {Proveniência:(Do lat. \textunderscore flegma\textunderscore )}
\end{itemize}
Um dos quatro humores do organismo humano, segundo a Medicina antiga.
Pachorra; serenidade.
Impassibilidade.
\section{Fleumão}
\begin{itemize}
\item {Grp. gram.:m.}
\end{itemize}
O mesmo que \textunderscore fleimão\textunderscore . Cf. B. Pereira, \textunderscore Prosódia\textunderscore , vb. \textunderscore flegmon\textunderscore .
\section{Fleumagogo}
\begin{itemize}
\item {Grp. gram.:adj.}
\end{itemize}
\begin{itemize}
\item {Proveniência:(Do gr. \textunderscore phlegma\textunderscore  + \textunderscore agogos\textunderscore )}
\end{itemize}
Que faz sair do organismo a fleuma, (tratando-se de medicamentos).
\section{Fleumático}
\begin{itemize}
\item {Grp. gram.:adj.}
\end{itemize}
\begin{itemize}
\item {Proveniência:(Do gr. \textunderscore phlegmatikos\textunderscore )}
\end{itemize}
Relativo a fleuma.
Pachorrento, impassível.
\section{Flexão}
\begin{itemize}
\item {fónica:csão}
\end{itemize}
\begin{itemize}
\item {Grp. gram.:f.}
\end{itemize}
\begin{itemize}
\item {Utilização:Gram.}
\end{itemize}
\begin{itemize}
\item {Proveniência:(Lat. \textunderscore flexio\textunderscore )}
\end{itemize}
Acto de dobrar-se, de curvar-se; curvatura.
Variante das desinências nas palavras declináveis e conjugáveis.
\section{Flexibilidade}
\begin{itemize}
\item {fónica:csi}
\end{itemize}
\begin{itemize}
\item {Grp. gram.:f.}
\end{itemize}
\begin{itemize}
\item {Proveniência:(Lat. \textunderscore flexibilitas\textunderscore )}
\end{itemize}
Qualidade daquillo que é flexível.
Aptidão para variadas coisas ou applicações.
Submissão, docilidade.
\section{Flexibilizar}
\begin{itemize}
\item {fónica:csi}
\end{itemize}
\begin{itemize}
\item {Grp. gram.:v. t.}
\end{itemize}
Tornar flexível. Cf. A. Celso, \textunderscore Lupa\textunderscore .
\section{Fléxil}
\begin{itemize}
\item {fónica:csil}
\end{itemize}
\begin{itemize}
\item {Grp. gram.:adj.}
\end{itemize}
\begin{itemize}
\item {Proveniência:(Lat. \textunderscore flexilis\textunderscore )}
\end{itemize}
O mesmo que \textunderscore flexível\textunderscore .
\section{Flexíloquo}
\begin{itemize}
\item {fónica:csi}
\end{itemize}
\begin{itemize}
\item {Grp. gram.:adj.}
\end{itemize}
\begin{itemize}
\item {Proveniência:(Lat. \textunderscore flexiloquus\textunderscore )}
\end{itemize}
Ambíguo ou obscuro na linguagem.
\section{Flexiologia}
\begin{itemize}
\item {fónica:csi}
\end{itemize}
\begin{itemize}
\item {Grp. gram.:f.}
\end{itemize}
Parte da Grammática, que trata das flexões.
\section{Flexional}
\begin{itemize}
\item {fónica:csi}
\end{itemize}
\begin{itemize}
\item {Grp. gram.:adj.}
\end{itemize}
\begin{itemize}
\item {Utilização:Gram.}
\end{itemize}
\begin{itemize}
\item {Proveniência:(Do lat. \textunderscore flexio\textunderscore )}
\end{itemize}
Relativo a flexão.
\section{Flexionar}
\begin{itemize}
\item {fónica:csi}
\end{itemize}
\begin{itemize}
\item {Grp. gram.:v.}
\end{itemize}
\begin{itemize}
\item {Utilização:t. Gram.}
\end{itemize}
Fazer a flexão de. Cf. Pacheco e Lameira, \textunderscore Gram.\textunderscore , 143.
\section{Flexionismo}
\begin{itemize}
\item {fónica:csi}
\end{itemize}
\begin{itemize}
\item {Grp. gram.:m.}
\end{itemize}
\begin{itemize}
\item {Utilização:Gram.}
\end{itemize}
Doutrina da flexão das palavras.
(Cp. \textunderscore flexional\textunderscore )
\section{Flexípede}
\begin{itemize}
\item {fónica:csi}
\end{itemize}
\begin{itemize}
\item {Grp. gram.:adj.}
\end{itemize}
\begin{itemize}
\item {Proveniência:(Lat. \textunderscore flexipes\textunderscore )}
\end{itemize}
Que tem pés tortos.
\section{Flexível}
\begin{itemize}
\item {fónica:csi}
\end{itemize}
\begin{itemize}
\item {Grp. gram.:adj.}
\end{itemize}
\begin{itemize}
\item {Proveniência:(Lat. \textunderscore flexibilis\textunderscore )}
\end{itemize}
Que se póde dobrar, que se póde curvar: \textunderscore vara flexível\textunderscore .
Malleável.
Suave.
Dócil.
Complacente; submisso.
\section{Flexivo}
\begin{itemize}
\item {fónica:csi}
\end{itemize}
\begin{itemize}
\item {Grp. gram.:adj.}
\end{itemize}
\begin{itemize}
\item {Proveniência:(Do lat. \textunderscore flexus\textunderscore )}
\end{itemize}
Diz-se do grupo das línguas, em que as modificações accessórias do sentido dos vocábulos são determinadas por modificações na fórma dêsses vocábulos.
\section{Flexor}
\begin{itemize}
\item {fónica:csôr}
\end{itemize}
\begin{itemize}
\item {Grp. gram.:adj.}
\end{itemize}
\begin{itemize}
\item {Grp. gram.:M.}
\end{itemize}
\begin{itemize}
\item {Utilização:Anat.}
\end{itemize}
\begin{itemize}
\item {Proveniência:(Lat. \textunderscore flexor\textunderscore )}
\end{itemize}
Que faz dobrar.
Músculo, que faz dobrar.
\section{Flexório}
\begin{itemize}
\item {fónica:csó}
\end{itemize}
\begin{itemize}
\item {Grp. gram.:m.}
\end{itemize}
\begin{itemize}
\item {Utilização:Anat.}
\end{itemize}
O músculo flexor.
\section{Flexuosa}
\begin{itemize}
\item {fónica:csu}
\end{itemize}
\begin{itemize}
\item {Grp. gram.:f.}
\end{itemize}
\begin{itemize}
\item {Proveniência:(Lat. \textunderscore flexuosa\textunderscore )}
\end{itemize}
Espécie de videira asiática.
\section{Flexuosidade}
\begin{itemize}
\item {fónica:csu}
\end{itemize}
\begin{itemize}
\item {Grp. gram.:f.}
\end{itemize}
Qualidade daquillo que é flexuoso.
\section{Flexuoso}
\begin{itemize}
\item {fónica:csu}
\end{itemize}
\begin{itemize}
\item {Grp. gram.:adj.}
\end{itemize}
\begin{itemize}
\item {Proveniência:(Lat. \textunderscore flexuosus\textunderscore )}
\end{itemize}
Torto; sinuoso.
\section{Flexura}
\begin{itemize}
\item {fónica:csu}
\end{itemize}
\begin{itemize}
\item {Grp. gram.:f.}
\end{itemize}
\begin{itemize}
\item {Utilização:Anat.}
\end{itemize}
\begin{itemize}
\item {Proveniência:(Lat. \textunderscore flexura\textunderscore )}
\end{itemize}
Junta dos ossos.
Lugar, onde êlles jogam para dobrar.
Meneio.
Flexibilidade.
Indolência, froixidão.
\section{Flibusteiro}
\begin{itemize}
\item {Grp. gram.:m.  e  adj.}
\end{itemize}
\begin{itemize}
\item {Utilização:Fig.}
\end{itemize}
\begin{itemize}
\item {Proveniência:(Fr. \textunderscore flibustier\textunderscore )}
\end{itemize}
Pirata americano.
Aventureiro; ladrão.
\section{Flocado}
\begin{itemize}
\item {Grp. gram.:adj.}
\end{itemize}
Semelhante a flocos; disposto em flocos.
\section{Floco}
\begin{itemize}
\item {Grp. gram.:m.}
\end{itemize}
\begin{itemize}
\item {Proveniência:(Lat. \textunderscore floccus\textunderscore )}
\end{itemize}
Conjunto de filamentos subtis, que esvoaçam ao simples impulso da aragem.
Felpa.
Tufo de pelos, na cauda de alguns animaes.
Vaporização.
Farfalha ou partícula de neve, que cái lentamente, esvoaçando como felpa branca.
\section{Flocoso}
\begin{itemize}
\item {Grp. gram.:adj.}
\end{itemize}
\begin{itemize}
\item {Proveniência:(Lat. \textunderscore floccosus\textunderscore )}
\end{itemize}
Que tem ou produz flocos.
Feito de flocos.
Disposto em flocos.
\section{Flóculo}
\begin{itemize}
\item {Grp. gram.:m.}
\end{itemize}
\begin{itemize}
\item {Proveniência:(Lat. \textunderscore flocculus\textunderscore )}
\end{itemize}
Pequeno floco.
\section{Flôr}
\begin{itemize}
\item {Grp. gram.:f.}
\end{itemize}
\begin{itemize}
\item {Proveniência:(Do lat. \textunderscore flos\textunderscore , \textunderscore floris\textunderscore )}
\end{itemize}
Corolla de algumas plantas, geralmente odorífera e de côres vivas.
Conjunto da corolla, estames, pistillo e ovário, nas plantas.
Substância, produzida á superfície de um corpo, pela decomposição dêste.
Superfície exterior do coiro.
A parte mais nobre, mais distinta, mais fina, de um conjunto ou de uma collectividade: \textunderscore a flôr da rapaziada\textunderscore .
O desabrochar (da vida).
Estado daquillo que é viçoso, fresco.
Belleza.
Coisa ou pessôa bella, agradável: \textunderscore a Maria é uma flôr\textunderscore .
Superfície: \textunderscore á flôr da água\textunderscore .
Elemento da designação de várias plantas.
Objecto ou ornato, que representa uma flôr.
\section{Flora}
\begin{itemize}
\item {Grp. gram.:f.}
\end{itemize}
\begin{itemize}
\item {Proveniência:(De \textunderscore Flora\textunderscore , n. p.)}
\end{itemize}
Conjunto das plantas, que crescem em determinada região.
Tratado, à cêrca dessas plantas.
Pequeno planeta, entre Marte e Clio.
\section{Floração}
\begin{itemize}
\item {Grp. gram.:f.}
\end{itemize}
\begin{itemize}
\item {Proveniência:(De \textunderscore flôr\textunderscore )}
\end{itemize}
O mesmo que \textunderscore inflorescência\textunderscore .
Desenvolvimento da flôr.
Estado das plantas em flôr.
\section{Florada}
\begin{itemize}
\item {Grp. gram.:f.}
\end{itemize}
\begin{itemize}
\item {Proveniência:(De \textunderscore flôr\textunderscore )}
\end{itemize}
Doce de flôres de laranjeira.
Doce de ovos, com a fórma de flôres.
\section{Floraes}
\begin{itemize}
\item {Grp. gram.:m. pl.}
\end{itemize}
\begin{itemize}
\item {Utilização:Mod.}
\end{itemize}
\begin{itemize}
\item {Proveniência:(Do lat. \textunderscore floralia\textunderscore )}
\end{itemize}
Antigos jogos e festas, que se celebravam em honra de Flora.
Concurso poético.
\section{Florais}
\begin{itemize}
\item {Grp. gram.:m. pl.}
\end{itemize}
\begin{itemize}
\item {Utilização:Mod.}
\end{itemize}
\begin{itemize}
\item {Proveniência:(Do lat. \textunderscore floralia\textunderscore )}
\end{itemize}
Antigos jogos e festas, que se celebravam em honra de Flora.
Concurso poético.
\section{Floral}
\begin{itemize}
\item {Grp. gram.:adj.}
\end{itemize}
\begin{itemize}
\item {Proveniência:(Lat. \textunderscore floralis\textunderscore )}
\end{itemize}
Que contêm só flôres.
Relativo a flôres.
\section{Florão}
\begin{itemize}
\item {Grp. gram.:m.}
\end{itemize}
\begin{itemize}
\item {Proveniência:(De \textunderscore flôr\textunderscore )}
\end{itemize}
Inflorescência, composta de muitas flores sésseis, reunidas sôbre um receptáculo commum.
Ornato circular, no centro de um tecto, de uma abóbada, etc.
Espécie de jôgo popular.
Espécie de pequena carruagem antiga.
\section{Flôr-boreal}
\begin{itemize}
\item {Grp. gram.:f.}
\end{itemize}
\begin{itemize}
\item {Utilização:Bras}
\end{itemize}
O mesmo que \textunderscore collinsónia\textunderscore .
\section{Flôr-da-noite}
\begin{itemize}
\item {Grp. gram.:f.}
\end{itemize}
O mesmo que \textunderscore flôr-de-baile\textunderscore .
\section{Flôr-da-paixão}
\begin{itemize}
\item {Grp. gram.:f.}
\end{itemize}
\begin{itemize}
\item {Utilização:Bras}
\end{itemize}
Planta, o mesmo que \textunderscore martýrio\textunderscore  ou \textunderscore passiflora\textunderscore .
\section{Flôr-da-verdade}
\begin{itemize}
\item {Grp. gram.:f.}
\end{itemize}
\begin{itemize}
\item {Utilização:Bras}
\end{itemize}
O mesmo que \textunderscore veratro\textunderscore .
\section{Flôr-de-água}
\begin{itemize}
\item {Grp. gram.:f.}
\end{itemize}
Planta medicinal brasileira, cujo rhizoma se emprega contra a hematuria.
\section{Flôr-de-baile}
\begin{itemize}
\item {Grp. gram.:f.}
\end{itemize}
\begin{itemize}
\item {Utilização:Bras}
\end{itemize}
Espécie de cacto, (\textunderscore cactus grandiflorus\textunderscore ).
\section{Flôr-de-camal}
\begin{itemize}
\item {Grp. gram.:f.}
\end{itemize}
(V.camalassana)
\section{Flôr-de-cera}
\begin{itemize}
\item {Grp. gram.:f.}
\end{itemize}
Planta asclepiádea, (\textunderscore hoya carnosa\textunderscore , R. Br.).
\section{Flôr-de-coral}
\begin{itemize}
\item {Grp. gram.:f.}
\end{itemize}
\begin{itemize}
\item {Utilização:Bras}
\end{itemize}
Planta euphorbiácea medicinal.
\section{Flôr-de-diana}
\begin{itemize}
\item {Grp. gram.:f.}
\end{itemize}
\begin{itemize}
\item {Utilização:Bras}
\end{itemize}
O mesmo que \textunderscore artemísia\textunderscore .
\section{Flôr-de-gêlo}
\begin{itemize}
\item {Grp. gram.:f.}
\end{itemize}
\begin{itemize}
\item {Utilização:Bras}
\end{itemize}
Variedade de mesembriânthemo.
\section{Flôr-de-hércules}
\begin{itemize}
\item {Grp. gram.:f.}
\end{itemize}
\begin{itemize}
\item {Utilização:Bras}
\end{itemize}
Planta vivaz, (\textunderscore heracleum\textunderscore )
\section{Flôr-de-lis}
\begin{itemize}
\item {Grp. gram.:f.}
\end{itemize}
\begin{itemize}
\item {Utilização:Heráld.}
\end{itemize}
Espécie de lírio, (\textunderscore lilium\textunderscore )
Antigo emblema da realeza, em França.
\section{Flôr-de-lisada}
\begin{itemize}
\item {Grp. gram.:adj.}
\end{itemize}
\begin{itemize}
\item {Utilização:Heráld.}
\end{itemize}
Diz-se da cruz, cujas hastes são rematadas por flôres-de-lis. Cf. L. Ribeiro, \textunderscore Trat. de Armaria\textunderscore .
\section{Flôr-de-noiva}
\begin{itemize}
\item {Grp. gram.:f.}
\end{itemize}
Nome de uma árvore brasileira.
\section{Flôr-de-pavão}
\begin{itemize}
\item {Grp. gram.:f.}
\end{itemize}
Arvore leguminosa da Índia, (\textunderscore poinciana regia\textunderscore , Boger.).
\section{Flôr-de-san-benedito}
\begin{itemize}
\item {Grp. gram.:f.}
\end{itemize}
\begin{itemize}
\item {Utilização:Bras}
\end{itemize}
Planta vivaz, (\textunderscore geum atrosanguineum\textunderscore ).
\section{Flôr-de-santantónio}
\begin{itemize}
\item {Grp. gram.:f.}
\end{itemize}
\begin{itemize}
\item {Utilização:Bras}
\end{itemize}
O mesmo que \textunderscore prunela\textunderscore .
\section{Flôr-de-setim}
\begin{itemize}
\item {Grp. gram.:f.}
\end{itemize}
\begin{itemize}
\item {Utilização:Bras}
\end{itemize}
O mesmo que \textunderscore lunária\textunderscore .
\section{Flôr-do-imperador}
\begin{itemize}
\item {Grp. gram.:f.}
\end{itemize}
Planta muito aromática do Brasil, (\textunderscore olea fragrans\textunderscore ).
\section{Flôr-dos-amores}
\begin{itemize}
\item {Grp. gram.:f.}
\end{itemize}
\begin{itemize}
\item {Utilização:Bras}
\end{itemize}
Variedade de amaranto, (\textunderscore amarantus melancholicus\textunderscore ).
\section{Flôr-do-vento}
\begin{itemize}
\item {Grp. gram.:f.}
\end{itemize}
\begin{itemize}
\item {Utilização:Bras}
\end{itemize}
O mesmo que \textunderscore anêmona\textunderscore .
\section{Floreado}
\begin{itemize}
\item {Grp. gram.:m.}
\end{itemize}
\begin{itemize}
\item {Proveniência:(De \textunderscore florear\textunderscore )}
\end{itemize}
Ornato.
Variação fantasiosa, em música.
\section{Floreal}
\begin{itemize}
\item {Grp. gram.:m.}
\end{itemize}
\begin{itemize}
\item {Proveniência:(Do lat. \textunderscore flos\textunderscore , \textunderscore floris\textunderscore )}
\end{itemize}
Oitavo mês do calendário republicano, em França, (20 de Abril a 20 de Maio).
\section{Florear}
\begin{itemize}
\item {Grp. gram.:v. t.}
\end{itemize}
\begin{itemize}
\item {Grp. gram.:V. i.}
\end{itemize}
\begin{itemize}
\item {Utilização:Fig.}
\end{itemize}
Fazer produzir flôres.
Cobrir ou adornar de flôres.
Adornar.
Brandir ou manejar destramente (uma arma branca): \textunderscore florear a espada\textunderscore .
Produzir flôres.
Tornar-se distinto, brilhar.
\section{Florecer}
\begin{itemize}
\item {Grp. gram.:v. t.  e  i.}
\end{itemize}
(V.florescer)
\section{Floreio}
\begin{itemize}
\item {Grp. gram.:m.}
\end{itemize}
Acto de florear.
\section{Floreira}
\begin{itemize}
\item {Grp. gram.:f.}
\end{itemize}
Vaso ou jarra de flôres para mesa de jantar.
Vendedora de flôres.
Florista.
\section{Floreiro}
\begin{itemize}
\item {Grp. gram.:m.}
\end{itemize}
Commerciante de flores.
\section{Florejante}
\begin{itemize}
\item {Grp. gram.:adj.}
\end{itemize}
Que floreja.
\section{Florejar}
\begin{itemize}
\item {Grp. gram.:v. i.}
\end{itemize}
\begin{itemize}
\item {Grp. gram.:V. i.}
\end{itemize}
Fazer brotar flôres em.
Ornar de flôres.
Florear.
Florescer; cobrir-se de flôres.
\section{Florença}
\begin{itemize}
\item {Grp. gram.:f.}
\end{itemize}
\begin{itemize}
\item {Proveniência:(De \textunderscore Florença\textunderscore , n. p.)}
\end{itemize}
Espécie de tecido de algodão, que imitava seda.
\section{Florença}
\begin{itemize}
\item {Grp. gram.:f.}
\end{itemize}
\begin{itemize}
\item {Utilização:Ant.}
\end{itemize}
O mesmo que \textunderscore florim\textunderscore .
\section{Florência}
\begin{itemize}
\item {Grp. gram.:f.}
\end{itemize}
Qualidade de florente. Cf. Camillo, \textunderscore Sc. da Foz\textunderscore , 116.
\section{Florenciado}
\begin{itemize}
\item {Grp. gram.:adj.}
\end{itemize}
\begin{itemize}
\item {Utilização:Heráld.}
\end{itemize}
\begin{itemize}
\item {Proveniência:(Do rad. do lat. \textunderscore florens\textunderscore )}
\end{itemize}
Diz-se da cruz, cujos braços terminam em flôr de lis num brasão.
\section{Florense}
\begin{itemize}
\item {Grp. gram.:adj.}
\end{itemize}
\begin{itemize}
\item {Grp. gram.:M.}
\end{itemize}
Relativo á ilha das Flôres.
Aquelle que é natural da ilha das Flôres.
\section{Florente}
\begin{itemize}
\item {Grp. gram.:adj.}
\end{itemize}
\begin{itemize}
\item {Utilização:Heráld.}
\end{itemize}
\begin{itemize}
\item {Proveniência:(Lat. \textunderscore florens\textunderscore , \textunderscore florentis\textunderscore )}
\end{itemize}
O mesmo que \textunderscore florescente\textunderscore .
O mesmo que \textunderscore flôr-de-lisada\textunderscore .
\section{Florente}
\begin{itemize}
\item {Grp. gram.:m.}
\end{itemize}
\begin{itemize}
\item {Utilização:Ant.}
\end{itemize}
\begin{itemize}
\item {Proveniência:(De \textunderscore Florença\textunderscore , onde era fabricado)}
\end{itemize}
Pano, geralmente encarnado.
\section{Florentino}
\begin{itemize}
\item {Grp. gram.:m.}
\end{itemize}
\begin{itemize}
\item {Grp. gram.:Adj.}
\end{itemize}
\begin{itemize}
\item {Proveniência:(Lat. \textunderscore florentinus\textunderscore )}
\end{itemize}
Aquelle que é natural de Florença.
Relativo a esta cidade.
\section{Florentino}
\begin{itemize}
\item {Grp. gram.:m.}
\end{itemize}
\begin{itemize}
\item {Utilização:Açor}
\end{itemize}
Habitante da ilha das Flôres; florense.
\section{Flóreo}
\begin{itemize}
\item {Grp. gram.:adj.}
\end{itemize}
\begin{itemize}
\item {Proveniência:(Lat. \textunderscore floreus\textunderscore )}
\end{itemize}
Florescente; ornato de flôres.
\section{Florescência}
\begin{itemize}
\item {Grp. gram.:f.}
\end{itemize}
\begin{itemize}
\item {Proveniência:(Lat. \textunderscore florescentia\textunderscore )}
\end{itemize}
Acto de florescer.
Inflorescência.
\section{Florescente}
\begin{itemize}
\item {Grp. gram.:adj.}
\end{itemize}
\begin{itemize}
\item {Proveniência:(Lat. \textunderscore florescens\textunderscore )}
\end{itemize}
Que floresce.
\section{Florescer}
\begin{itemize}
\item {Grp. gram.:v. t.}
\end{itemize}
\begin{itemize}
\item {Grp. gram.:V. i.}
\end{itemize}
\begin{itemize}
\item {Utilização:Fig.}
\end{itemize}
\begin{itemize}
\item {Proveniência:(Lat. \textunderscore florescere\textunderscore )}
\end{itemize}
Enflorar.
Fazer produzir flôres: \textunderscore a primavera floresce os jardins\textunderscore .
Produzir flôres: \textunderscore florescem cedo as amendoeiras\textunderscore .
Prosperar: \textunderscore aquella empresa vai florescendo\textunderscore .
Tornar-se distinto.
Têr fama.
Existir com renome.
\section{Florescimento}
\begin{itemize}
\item {Grp. gram.:m.}
\end{itemize}
Acto ou effeito de florescer.
\section{Floresta}
\begin{itemize}
\item {Grp. gram.:f.}
\end{itemize}
\begin{itemize}
\item {Utilização:Fig.}
\end{itemize}
Mata grande.
Sítio umbroso.
Retiro campestre.
Confusão, labyrinto.
Collecção variada (de máximas, narrativas, etc.).
Conjunto das lanças ou espadas de um exército, disposto para o combate.
(B. lat. \textunderscore foresta\textunderscore , com infl. do voc. \textunderscore flôr\textunderscore )
\section{Florestal}
\begin{itemize}
\item {Grp. gram.:adj.}
\end{itemize}
Relativo a floresta.
Que trata de florestas: \textunderscore engenheiro florestal\textunderscore .
\section{Floreta}
\begin{itemize}
\item {fónica:florê}
\end{itemize}
\begin{itemize}
\item {Grp. gram.:f.}
\end{itemize}
\begin{itemize}
\item {Utilização:Ant.}
\end{itemize}
\begin{itemize}
\item {Proveniência:(De \textunderscore flôr\textunderscore )}
\end{itemize}
Ornato, que imita flôr.
Passo de dança.
\section{Floretado}
\begin{itemize}
\item {Grp. gram.:adj.}
\end{itemize}
Diz-se do vidro, que tem certos relevos, que lhe impedem a transparência.
\section{Florete}
\begin{itemize}
\item {fónica:florê}
\end{itemize}
\begin{itemize}
\item {Grp. gram.:m.}
\end{itemize}
Arma branca, composta, além do cabo, de uma haste de metal, prismática e ponteaguda.
(Cast. \textunderscore florete\textunderscore )
\section{Floreteado}
\begin{itemize}
\item {Grp. gram.:adj.}
\end{itemize}
\begin{itemize}
\item {Proveniência:(De \textunderscore floretear\textunderscore )}
\end{itemize}
Ornado de flôres.
Que tem ponta aguda, como o florete. Cf. Hercul., \textunderscore Opúsc.\textunderscore , III, 172.
\section{Floretear}
\begin{itemize}
\item {Grp. gram.:v. t.}
\end{itemize}
\begin{itemize}
\item {Grp. gram.:V. i.}
\end{itemize}
\begin{itemize}
\item {Proveniência:(De \textunderscore florete\textunderscore )}
\end{itemize}
Florear.
Esgrimir.
\section{Floretista}
\begin{itemize}
\item {Grp. gram.:m.}
\end{itemize}
Jogador de florete.
\section{Florianesco}
\begin{itemize}
\item {fónica:nês}
\end{itemize}
\begin{itemize}
\item {Grp. gram.:adj.}
\end{itemize}
Relativo ao poéta Florian.
Escrito no estilo de Florian.
\section{Florianista}
\begin{itemize}
\item {Grp. gram.:m.}
\end{itemize}
Admirador ou sectário da feição literâria de Florian. Cf. Rui Barb., \textunderscore Cartas de Ingl.\textunderscore , 38.
\section{Floricoroado}
\begin{itemize}
\item {Grp. gram.:adj.}
\end{itemize}
\begin{itemize}
\item {Proveniência:(De \textunderscore flôr\textunderscore  + \textunderscore coroado\textunderscore )}
\end{itemize}
Coroado de flôres. Cf. Castilho, \textunderscore Fastos\textunderscore , II, 33.
\section{Florículo}
\begin{itemize}
\item {Grp. gram.:m.}
\end{itemize}
O mesmo que \textunderscore flósculo\textunderscore .
\section{Floricultura}
\begin{itemize}
\item {Grp. gram.:f.}
\end{itemize}
\begin{itemize}
\item {Proveniência:(Do lat. \textunderscore flos\textunderscore  + \textunderscore cultura\textunderscore )}
\end{itemize}
Arte de cultivar flôres.
Cultura das flôres.
\section{Floridamente}
\begin{itemize}
\item {Grp. gram.:adv.}
\end{itemize}
De modo florido.
\section{Florído}
\begin{itemize}
\item {Grp. gram.:adj.}
\end{itemize}
\begin{itemize}
\item {Proveniência:(De \textunderscore florir\textunderscore )}
\end{itemize}
Florescente; que tem flôres: \textunderscore campos floridos\textunderscore .
Adornado; elegante.
\section{Flórido}
\begin{itemize}
\item {Grp. gram.:adj.}
\end{itemize}
\begin{itemize}
\item {Proveniência:(Lat. \textunderscore floridus\textunderscore )}
\end{itemize}
Flóreo; florescente.
Brilhante.
\section{Florífago}
\begin{itemize}
\item {Grp. gram.:adj.}
\end{itemize}
\begin{itemize}
\item {Proveniência:(Do lat. \textunderscore flos\textunderscore  + gr. \textunderscore phagein\textunderscore )}
\end{itemize}
Que se sustenta de flôres.
\section{Florífero}
\begin{itemize}
\item {Grp. gram.:adj.}
\end{itemize}
\begin{itemize}
\item {Proveniência:(Lat. \textunderscore florifer\textunderscore )}
\end{itemize}
Que tem ou produz flôres.
\section{Floriferto}
\begin{itemize}
\item {Grp. gram.:m.}
\end{itemize}
\begin{itemize}
\item {Proveniência:(Lat. \textunderscore florifertum\textunderscore )}
\end{itemize}
Festa, em que os Romanos iam offerecer as primeiras espigas dos cereaes á deusa Ceres.
\section{Floriforme}
\begin{itemize}
\item {Grp. gram.:adj.}
\end{itemize}
\begin{itemize}
\item {Proveniência:(Do lat. \textunderscore flos\textunderscore  + \textunderscore forma\textunderscore )}
\end{itemize}
Semelhante a flôres.
\section{Florígero}
\begin{itemize}
\item {Grp. gram.:adj.}
\end{itemize}
\begin{itemize}
\item {Proveniência:(Lat. \textunderscore floriger\textunderscore )}
\end{itemize}
O mesmo que \textunderscore florífero\textunderscore .
\section{Florilégio}
\begin{itemize}
\item {Grp. gram.:m.}
\end{itemize}
\begin{itemize}
\item {Utilização:Fig.}
\end{itemize}
\begin{itemize}
\item {Proveniência:(Do lat. \textunderscore flos\textunderscore  + \textunderscore legere\textunderscore )}
\end{itemize}
Collecção de flôres.
Compilação de coisas várias, em literatura; anthologia.
\section{Florim}
\begin{itemize}
\item {Grp. gram.:m.}
\end{itemize}
Moéda de prata ou de oiro, em vários países.
Unidade monetária na Austria-Hungria e nos Países-Baixos.
(Cast. \textunderscore florin\textunderscore )
\section{Floríparo}
\begin{itemize}
\item {Grp. gram.:adj.}
\end{itemize}
\begin{itemize}
\item {Utilização:Bot.}
\end{itemize}
\begin{itemize}
\item {Proveniência:(Do lat. \textunderscore flos\textunderscore  + \textunderscore parere\textunderscore )}
\end{itemize}
Diz-se do botão, que só contém flôres.
\section{Floríphago}
\begin{itemize}
\item {Grp. gram.:adj.}
\end{itemize}
\begin{itemize}
\item {Proveniência:(Do lat. \textunderscore flos\textunderscore  + gr. \textunderscore phagein\textunderscore )}
\end{itemize}
Que se sustenta de flôres.
\section{Floripôndio}
\begin{itemize}
\item {Grp. gram.:m.}
\end{itemize}
\begin{itemize}
\item {Utilização:Bot.}
\end{itemize}
\begin{itemize}
\item {Proveniência:(Do lat. \textunderscore flos\textunderscore  + \textunderscore pondus\textunderscore )}
\end{itemize}
Espécie de estramónio.
\section{Florir}
\begin{itemize}
\item {Grp. gram.:v. i.}
\end{itemize}
\begin{itemize}
\item {Utilização:Fig.}
\end{itemize}
\begin{itemize}
\item {Proveniência:(Lat. \textunderscore florire\textunderscore , por \textunderscore florere\textunderscore )}
\end{itemize}
Florescer, cobrir-se de flôres.
Desabrochar; desenvolver-se.
\section{Florista}
\begin{itemize}
\item {Grp. gram.:m.  e  f.}
\end{itemize}
\begin{itemize}
\item {Proveniência:(De \textunderscore flôr\textunderscore )}
\end{itemize}
Pessôa, que vende flôres.
Fabricante de flôres artificiaes.
\section{Floromania}
\begin{itemize}
\item {Grp. gram.:f.}
\end{itemize}
\begin{itemize}
\item {Proveniência:(De \textunderscore flôr\textunderscore  + \textunderscore mania\textunderscore )}
\end{itemize}
Paixão pelas flôres.
\section{Floromaníaco}
\begin{itemize}
\item {Grp. gram.:adj.}
\end{itemize}
Que tem flòròmania.
\section{Florosa}
\begin{itemize}
\item {Grp. gram.:f.}
\end{itemize}
\begin{itemize}
\item {Utilização:Mad}
\end{itemize}
Ave, o mesmo que \textunderscore papo-roixo\textunderscore .
\section{Flôr-seráfica}
\begin{itemize}
\item {Grp. gram.:f.}
\end{itemize}
\begin{itemize}
\item {Utilização:Bot.}
\end{itemize}
O mesmo que \textunderscore amor-perfeito\textunderscore . Cf. \textunderscore Ann. Scient. da Acad. Polyt. do Porto\textunderscore , IV, 120.
\section{Flórula}
\begin{itemize}
\item {Grp. gram.:f.}
\end{itemize}
\begin{itemize}
\item {Utilização:Bot.}
\end{itemize}
\begin{itemize}
\item {Proveniência:(De \textunderscore flôr\textunderscore )}
\end{itemize}
Pequena flora; flora de uma pequena região.
Flôr insulada de uma espiga, de uma calathide, etc.
\section{Flosa}
\begin{itemize}
\item {Grp. gram.:f.}
\end{itemize}
(V.folosa)Cf. J. Dinís, \textunderscore Pupillas\textunderscore , 108.
\section{Flosculários}
\begin{itemize}
\item {Grp. gram.:m. pl.}
\end{itemize}
\begin{itemize}
\item {Proveniência:(De \textunderscore flósculo\textunderscore )}
\end{itemize}
Família de zoóphitos, cuja cabeça é semelhante a uma flôr de quatro pétalas.
\section{Flósculo}
\begin{itemize}
\item {Grp. gram.:m.}
\end{itemize}
\begin{itemize}
\item {Proveniência:(Lat. \textunderscore flosculus\textunderscore )}
\end{itemize}
Florinha.
Cada uma das flôres, que constituem uma flor composta.
\section{Flosculoso}
\begin{itemize}
\item {Grp. gram.:adj.}
\end{itemize}
Composto de flósculos.
\section{Flostria}
\begin{itemize}
\item {Grp. gram.:f.}
\end{itemize}
\begin{itemize}
\item {Utilização:Pop.}
\end{itemize}
\begin{itemize}
\item {Proveniência:(Fr. ant. \textunderscore folastrie\textunderscore )}
\end{itemize}
Folgança.
Fanfarronada.
\section{Flostriar}
\begin{itemize}
\item {Grp. gram.:v. i.}
\end{itemize}
\begin{itemize}
\item {Utilização:Pop.}
\end{itemize}
\begin{itemize}
\item {Proveniência:(De \textunderscore flostria\textunderscore )}
\end{itemize}
Foliar muito.
Saltar, brincando.
Patuscar.
\section{Flotilha}
\begin{itemize}
\item {Grp. gram.:f.}
\end{itemize}
Pequena frota.
(Cast. \textunderscore flotilla\textunderscore )
\section{Fluantimoniato}
\begin{itemize}
\item {Grp. gram.:m.}
\end{itemize}
\begin{itemize}
\item {Proveniência:(De \textunderscore flúor\textunderscore  + \textunderscore antimoniato\textunderscore )}
\end{itemize}
Fluoreto duplo de antimónio e de diversos metaes.
\section{Fluarseniato}
\begin{itemize}
\item {Grp. gram.:m.}
\end{itemize}
\begin{itemize}
\item {Proveniência:(De \textunderscore flúor\textunderscore  + \textunderscore arseniato\textunderscore )}
\end{itemize}
Sal, resultante da combinação do perfluoreto de arsênico com os fluoretos alcalinos.
\section{Fluato}
\begin{itemize}
\item {Grp. gram.:m.}
\end{itemize}
\begin{itemize}
\item {Proveniência:(De \textunderscore fluor\textunderscore )}
\end{itemize}
Combinação do ácido fluórico com uma base.
\section{Fluctícola}
\begin{itemize}
\item {Grp. gram.:adj.}
\end{itemize}
\begin{itemize}
\item {Proveniência:(Lat. \textunderscore flucticola\textunderscore )}
\end{itemize}
Que habita no mar.
\section{Flucticolor}
\begin{itemize}
\item {Grp. gram.:adj.}
\end{itemize}
\begin{itemize}
\item {Proveniência:(Do lat. \textunderscore fluctus\textunderscore  + \textunderscore color\textunderscore )}
\end{itemize}
Que é da côr do mar.
\section{Fluctígeno}
\begin{itemize}
\item {Grp. gram.:adj.}
\end{itemize}
\begin{itemize}
\item {Proveniência:(Lat. \textunderscore fluctigena\textunderscore )}
\end{itemize}
Que nasce no mar.
\section{Fluctígero}
\begin{itemize}
\item {Grp. gram.:adj.}
\end{itemize}
\begin{itemize}
\item {Proveniência:(Lat. \textunderscore fluctiger\textunderscore )}
\end{itemize}
Batido pelas vagas. Cf. Castilho, \textunderscore Fastos\textunderscore , II, 137.
\section{Fluctisonante}
\begin{itemize}
\item {fónica:tisso}
\end{itemize}
\begin{itemize}
\item {Grp. gram.:adj.}
\end{itemize}
\begin{itemize}
\item {Proveniência:(Do lat. \textunderscore fluctus\textunderscore  + \textunderscore sonans\textunderscore )}
\end{itemize}
Que sôa como as ondas.
\section{Fluctísono}
\begin{itemize}
\item {fónica:tisso}
\end{itemize}
\begin{itemize}
\item {Grp. gram.:adj.}
\end{itemize}
\begin{itemize}
\item {Proveniência:(Lat. \textunderscore fluctisonus\textunderscore )}
\end{itemize}
O mesmo que \textunderscore fluctisonante\textunderscore .
\section{Fluctissonante}
\begin{itemize}
\item {Grp. gram.:adj.}
\end{itemize}
\begin{itemize}
\item {Proveniência:(Do lat. \textunderscore fluctus\textunderscore  + \textunderscore sonans\textunderscore )}
\end{itemize}
Que sôa como as ondas.
\section{Fluctíssono}
\begin{itemize}
\item {Grp. gram.:adj.}
\end{itemize}
\begin{itemize}
\item {Proveniência:(Lat. \textunderscore fluctisonus\textunderscore )}
\end{itemize}
O mesmo que \textunderscore fluctissonante\textunderscore .
\section{Fluctívago}
\begin{itemize}
\item {Grp. gram.:adj.}
\end{itemize}
\begin{itemize}
\item {Proveniência:(Lat. \textunderscore fluctivagus\textunderscore )}
\end{itemize}
Que anda sôbre o mar.
\section{Fluctuabilidade}
\begin{itemize}
\item {Grp. gram.:f.}
\end{itemize}
Qualidade daquillo que é fluctuável.
\section{Fluctuação}
\begin{itemize}
\item {Grp. gram.:f.}
\end{itemize}
\begin{itemize}
\item {Utilização:Fig.}
\end{itemize}
\begin{itemize}
\item {Proveniência:(Lat. \textunderscore fluctuatio\textunderscore )}
\end{itemize}
Acto ou effeito de fluctuar.
Hesitação.
\section{Fluctuador}
\begin{itemize}
\item {Grp. gram.:m.}
\end{itemize}
Instrumento ou apparelho que fluctua.
\section{Fluctuante}
\begin{itemize}
\item {Grp. gram.:adj.}
\end{itemize}
\begin{itemize}
\item {Proveniência:(Lat. \textunderscore fluctuans\textunderscore )}
\end{itemize}
Que fluctua: \textunderscore fôlhas fluctuantes\textunderscore .
\section{Fluctuar}
\begin{itemize}
\item {Grp. gram.:v. i.}
\end{itemize}
\begin{itemize}
\item {Utilização:Fig.}
\end{itemize}
\begin{itemize}
\item {Proveniência:(Lat. \textunderscore fluctuare\textunderscore )}
\end{itemize}
Andar sôbre as ondas.
Sobrenadar.
Boiar.
Ondular.
Agitar-se ao sôpro do vento.
Tumultuar.
Hesitar; estar indeciso.
\section{Fluctuável}
\begin{itemize}
\item {Grp. gram.:adj.}
\end{itemize}
Que póde fluctuar.
Em que se póde fluctuar ou navegar.
Navegável.
\section{Fluctuosidade}
\begin{itemize}
\item {Grp. gram.:f.}
\end{itemize}
Qualidade daquillo que é fluctuoso.
\section{Fluctuoso}
\begin{itemize}
\item {Grp. gram.:adj.}
\end{itemize}
\begin{itemize}
\item {Proveniência:(Lat. \textunderscore fluctuosus\textunderscore )}
\end{itemize}
O mesmo que \textunderscore fluctuante\textunderscore .
\section{Fluência}
\begin{itemize}
\item {Grp. gram.:f.}
\end{itemize}
\begin{itemize}
\item {Proveniência:(Lat. \textunderscore fluentia\textunderscore )}
\end{itemize}
Qualidade daquillo que é fluente.
\section{Fluente}
\begin{itemize}
\item {Grp. gram.:adj.}
\end{itemize}
\begin{itemize}
\item {Utilização:Fig.}
\end{itemize}
\begin{itemize}
\item {Proveniência:(Lat. \textunderscore fluens\textunderscore )}
\end{itemize}
Fluido.
Que corre facilmente.
Natural, espontâneo: \textunderscore estilo fluente\textunderscore .
Fácil.
\section{Fluidez}
\begin{itemize}
\item {fónica:flu-i}
\end{itemize}
\begin{itemize}
\item {Grp. gram.:f.}
\end{itemize}
Qualidade daquillo que é flúido.
\section{Fluídico}
\begin{itemize}
\item {Grp. gram.:adj.}
\end{itemize}
\begin{itemize}
\item {Utilização:Neol.}
\end{itemize}
\begin{itemize}
\item {Proveniência:(De \textunderscore fluido\textunderscore )}
\end{itemize}
Diz-se, em espiritismo, de certos corpos ou sombras, que se não palpam, mas que a photographia reproduz.
\section{Fluididade}
\begin{itemize}
\item {Grp. gram.:f.}
\end{itemize}
\begin{itemize}
\item {Utilização:Des.}
\end{itemize}
O mesmo que \textunderscore fluidez\textunderscore .
\section{Fluidificação}
\begin{itemize}
\item {Grp. gram.:f.}
\end{itemize}
Acto de fluidificar.
\section{Fluidificante}
\begin{itemize}
\item {Grp. gram.:adj.}
\end{itemize}
Que fluidifica.
\section{Fluidificar}
\begin{itemize}
\item {Grp. gram.:v. t.}
\end{itemize}
\begin{itemize}
\item {Proveniência:(Do lat. \textunderscore fluidus\textunderscore  + \textunderscore facere\textunderscore )}
\end{itemize}
Tornar fluido.
\section{Fluidificável}
\begin{itemize}
\item {Grp. gram.:adj.}
\end{itemize}
Que se póde fluidificar.
\section{Fluido}
\begin{itemize}
\item {Grp. gram.:adj.}
\end{itemize}
\begin{itemize}
\item {Utilização:Fig.}
\end{itemize}
\begin{itemize}
\item {Grp. gram.:M.}
\end{itemize}
\begin{itemize}
\item {Proveniência:(Lat. \textunderscore fluidus\textunderscore )}
\end{itemize}
Fluente.
Que corre como um líquido.
Brando, froixo.
Corpo, cujas moléculas cedem facilmente ao tacto, separando-se e movendo-se entre si mesmas, por fôrças que lhes são próprias.
Qualquer líquido.
\section{Fluir}
\begin{itemize}
\item {Grp. gram.:v. i.}
\end{itemize}
\begin{itemize}
\item {Proveniência:(Lat. \textunderscore fluere\textunderscore )}
\end{itemize}
Correr em estado líquido.
Manar.
Derivar; proceder.
\section{Flume}
\begin{itemize}
\item {Grp. gram.:m.}
\end{itemize}
\begin{itemize}
\item {Proveniência:(Lat. \textunderscore flumen\textunderscore )}
\end{itemize}
O mesmo que \textunderscore rio\textunderscore .
\section{Flúmen}
\begin{itemize}
\item {Grp. gram.:m.}
\end{itemize}
\begin{itemize}
\item {Utilização:Poét.}
\end{itemize}
\begin{itemize}
\item {Proveniência:(Lat. \textunderscore flumen\textunderscore )}
\end{itemize}
O mesmo que \textunderscore rio\textunderscore .
\section{Fluminar}
\begin{itemize}
\item {Grp. gram.:m.  e  adj.}
\end{itemize}
O que é do Rio de Janeiro; fluminense. Cf. Camillo, \textunderscore Cancion. Alegre\textunderscore .
\section{Fluminense}
\begin{itemize}
\item {Grp. gram.:adj.}
\end{itemize}
\begin{itemize}
\item {Grp. gram.:M.}
\end{itemize}
\begin{itemize}
\item {Proveniência:(De \textunderscore flumen\textunderscore )}
\end{itemize}
Fluvial.
Relativo ao Rio de Janeiro.
Aquelle que é natural do Rio de Janeiro.
\section{Flumíneo}
\begin{itemize}
\item {Grp. gram.:adj.}
\end{itemize}
\begin{itemize}
\item {Proveniência:(Lat. \textunderscore flumineus\textunderscore )}
\end{itemize}
O mesmo que \textunderscore fluvial\textunderscore .
\section{Fluoborato}
\begin{itemize}
\item {Grp. gram.:m.}
\end{itemize}
\begin{itemize}
\item {Proveniência:(De \textunderscore flúor\textunderscore  + \textunderscore borato\textunderscore )}
\end{itemize}
Sal, resultante da combinação do ácido fluobórico com uma base salificável.
\section{Fluoboreto}
\begin{itemize}
\item {fónica:borê}
\end{itemize}
\begin{itemize}
\item {Grp. gram.:m.}
\end{itemize}
\begin{itemize}
\item {Proveniência:(De \textunderscore flúor\textunderscore )}
\end{itemize}
Nome, que se dá a um composto de fluor, boro e um terceiro corpo.
\section{Fluobórico}
\begin{itemize}
\item {Grp. gram.:adj.}
\end{itemize}
\begin{itemize}
\item {Proveniência:(De \textunderscore flúor\textunderscore  + \textunderscore boro\textunderscore )}
\end{itemize}
Produzido pela combinação do flúor e do boro.
\section{Flúor}
\begin{itemize}
\item {Grp. gram.:m.}
\end{itemize}
\begin{itemize}
\item {Proveniência:(Lat. \textunderscore fluor\textunderscore )}
\end{itemize}
Corpo simples, ainda não insulado.
Mineral incombustível e fusível.
Antigamente, ácido mineral, que é sempre flúido.
\section{Fluorado}
\begin{itemize}
\item {Grp. gram.:adj.}
\end{itemize}
Que contém flúor.
\section{Fluorescência}
\begin{itemize}
\item {Grp. gram.:f.}
\end{itemize}
\begin{itemize}
\item {Proveniência:(De \textunderscore flúor\textunderscore )}
\end{itemize}
Illuminação especial, que apresentam certas substâncias, quando expostas á acção dos raios chímicos.
\section{Fluorescente}
\begin{itemize}
\item {Grp. gram.:adj.}
\end{itemize}
Que tem a propriedade da fluorescência.
\section{Fluoreto}
\begin{itemize}
\item {fónica:orê}
\end{itemize}
\begin{itemize}
\item {Grp. gram.:m.}
\end{itemize}
\begin{itemize}
\item {Proveniência:(De \textunderscore flúor\textunderscore )}
\end{itemize}
Combinação do flúor com outro corpo simples.
\section{Fluorhýdrico}
\begin{itemize}
\item {Grp. gram.:adj.}
\end{itemize}
\begin{itemize}
\item {Proveniência:(De \textunderscore flúor\textunderscore  + gr. \textunderscore hudor\textunderscore )}
\end{itemize}
Diz-se de um ácido, formado pela combinação do hydrogênio com uma base.
\section{Fluórico}
\begin{itemize}
\item {Grp. gram.:adj.}
\end{itemize}
(V.fluorhýdrico)
\section{Fluórido}
\begin{itemize}
\item {Grp. gram.:m.}
\end{itemize}
Combinação do flúor com um ácido.
\section{Fluorídrico}
\begin{itemize}
\item {Grp. gram.:adj.}
\end{itemize}
\begin{itemize}
\item {Proveniência:(De \textunderscore flúor\textunderscore  + gr. \textunderscore hudor\textunderscore )}
\end{itemize}
Diz-se de um ácido, formado pela combinação do hydrogênio com uma base.
\section{Fluorina}
\begin{itemize}
\item {Grp. gram.:f.}
\end{itemize}
Mineral, resultante da combinação do flúor com o cálcio.
\section{Fluorino}
\begin{itemize}
\item {Grp. gram.:m.}
\end{itemize}
O mesmo ou melhor que \textunderscore fluorina\textunderscore .
\section{Fluorite}
\begin{itemize}
\item {Grp. gram.:f.}
\end{itemize}
Mineral, o mesmo que \textunderscore flúor\textunderscore .
\section{Fluorítico}
\begin{itemize}
\item {Grp. gram.:adj.}
\end{itemize}
Em que há flúor.
\section{Fluoroscópico}
\begin{itemize}
\item {Grp. gram.:adj.}
\end{itemize}
Relativo ao fluoroscópio.
\section{Fluoroscópio}
\begin{itemize}
\item {Grp. gram.:m.}
\end{itemize}
\begin{itemize}
\item {Proveniência:(De \textunderscore flúor\textunderscore  + gr. \textunderscore skopein\textunderscore )}
\end{itemize}
Instrumento, inventado há pouco por Edison, e que, como o cryptoscópio, permitte vêr os objectos contidos numa caixa fechada de papelão ou de alumínio.
\section{Fluorureto}
\begin{itemize}
\item {Grp. gram.:m.}
\end{itemize}
(V.fluoreto)
\section{Fluoscopia}
\begin{itemize}
\item {Grp. gram.:f.}
\end{itemize}
O mesmo que \textunderscore radioscopia\textunderscore . Cf. Verg. Machado, \textunderscore Raios X\textunderscore , 9.
\section{Fluosilicato}
\begin{itemize}
\item {fónica:si}
\end{itemize}
\begin{itemize}
\item {Grp. gram.:m.}
\end{itemize}
\begin{itemize}
\item {Proveniência:(De \textunderscore flúor\textunderscore  + \textunderscore silicato\textunderscore )}
\end{itemize}
Sal, resultante da combinação do ácido fluosilícico com uma base.
\section{Fluosilícico}
\begin{itemize}
\item {fónica:si}
\end{itemize}
\begin{itemize}
\item {Grp. gram.:adj.}
\end{itemize}
\begin{itemize}
\item {Proveniência:(De \textunderscore flúor\textunderscore  + \textunderscore silícico\textunderscore )}
\end{itemize}
Diz-se de um ácido, resultante da combinação do silício e do flúor.
\section{Fluossilicato}
\begin{itemize}
\item {Grp. gram.:m.}
\end{itemize}
\begin{itemize}
\item {Proveniência:(De \textunderscore flúor\textunderscore  + \textunderscore silicato\textunderscore )}
\end{itemize}
Sal, resultante da combinação do ácido fluosilícico com uma base.
\section{Fluossilícico}
\begin{itemize}
\item {Grp. gram.:adj.}
\end{itemize}
\begin{itemize}
\item {Proveniência:(De \textunderscore flúor\textunderscore  + \textunderscore silícico\textunderscore )}
\end{itemize}
Diz-se de um ácido, resultante da combinação do silício e do flúor.
\section{Fluotantalato}
\begin{itemize}
\item {Grp. gram.:m.}
\end{itemize}
\begin{itemize}
\item {Proveniência:(De \textunderscore flúor\textunderscore  + \textunderscore tantalato\textunderscore )}
\end{itemize}
Sal, resultante da combinação do ácido fluotantálico com uma base.
\section{Fluotantálico}
\begin{itemize}
\item {Grp. gram.:adj.}
\end{itemize}
Diz-se de um ácido, formado pela combinação do flúor com o tantálico.
\section{Fluotitanato}
\begin{itemize}
\item {Grp. gram.:m.}
\end{itemize}
\begin{itemize}
\item {Proveniência:(De \textunderscore flúor\textunderscore  + \textunderscore titanato\textunderscore )}
\end{itemize}
Sal, resultante da combinação do ácido fluotitânico com uma base.
\section{Fluotitânico}
\begin{itemize}
\item {Grp. gram.:adj.}
\end{itemize}
\begin{itemize}
\item {Proveniência:(De \textunderscore flúor\textunderscore  + \textunderscore titânico\textunderscore )}
\end{itemize}
Diz-se de um ácido, formado pela combinação do flúor com o titânico.
\section{Fluotungstato}
\begin{itemize}
\item {Grp. gram.:m.}
\end{itemize}
\begin{itemize}
\item {Proveniência:(De \textunderscore flúor\textunderscore  + \textunderscore tungstênio\textunderscore )}
\end{itemize}
Combinação do ácido fluotungstênico com uma base.
\section{Fluotungstênico}
\begin{itemize}
\item {Grp. gram.:adj.}
\end{itemize}
\begin{itemize}
\item {Proveniência:(De \textunderscore flúor\textunderscore  + \textunderscore tungstênio\textunderscore )}
\end{itemize}
Diz-se de um ácido, formado pela combinação do flúor com o tungstênio.
\section{Flustra}
\begin{itemize}
\item {Grp. gram.:f.}
\end{itemize}
\begin{itemize}
\item {Proveniência:(Lat. \textunderscore flustra\textunderscore ?)}
\end{itemize}
Designação scientífica de várias fórmas da neurose mental, como hyperesthesia, somnambulismo, etc.
\section{Flustria}
\begin{itemize}
\item {Grp. gram.:f.}
\end{itemize}
(V.flostria)
\section{Fluta}
\begin{itemize}
\item {Grp. gram.:f.}
\end{itemize}
Peixe do Mediterrâneo, semelhante á enguia.
\section{Fluvial}
\begin{itemize}
\item {Grp. gram.:adj.}
\end{itemize}
\begin{itemize}
\item {Proveniência:(Lat. \textunderscore fluvialis\textunderscore )}
\end{itemize}
Relativo a rio: \textunderscore navegação fluvial\textunderscore .
Próprio dos rios.
Que vive nos rios: \textunderscore peixe fluvial\textunderscore .
\section{Fluviátil}
\begin{itemize}
\item {Grp. gram.:adj.}
\end{itemize}
\begin{itemize}
\item {Proveniência:(Lat. \textunderscore fluviatilis\textunderscore )}
\end{itemize}
O mesmo que \textunderscore fluvial\textunderscore .
\section{Fluviométrico}
\begin{itemize}
\item {Grp. gram.:adj.}
\end{itemize}
Relativo ao fluviómetro.
\section{Fluviómetro}
\begin{itemize}
\item {Grp. gram.:m.}
\end{itemize}
\begin{itemize}
\item {Proveniência:(Do lat. \textunderscore fluvius\textunderscore  + gr. \textunderscore metron\textunderscore )}
\end{itemize}
Instrumento, com que se mede a altura das enchentes fluviaes.
\section{Flux}
\begin{itemize}
\item {Grp. gram.:m.}
\end{itemize}
\begin{itemize}
\item {Grp. gram.:Loc. adv.}
\end{itemize}
\begin{itemize}
\item {Proveniência:(Do lat. \textunderscore fluxus\textunderscore )}
\end{itemize}
O mesmo que \textunderscore fluxo\textunderscore .
\textunderscore A flux\textunderscore , em abundância.
\section{Fluxão}
\begin{itemize}
\item {Grp. gram.:f.}
\end{itemize}
\begin{itemize}
\item {Proveniência:(Lat. \textunderscore fluxio\textunderscore )}
\end{itemize}
Fluxo; affluxo de um líquido, determinado por uma causa excitante em uma parte do corpo.
Defluxão.
\section{Fluxibilidade}
\begin{itemize}
\item {fónica:csi}
\end{itemize}
\begin{itemize}
\item {Grp. gram.:f.}
\end{itemize}
Qualidade daquillo que é fluxível.
\section{Fluxionário}
\begin{itemize}
\item {fónica:csi}
\end{itemize}
\begin{itemize}
\item {Grp. gram.:adj.}
\end{itemize}
\begin{itemize}
\item {Utilização:Med.}
\end{itemize}
Relativo a fluxão; que produz fluxão.
\section{Fluxível}
\begin{itemize}
\item {fónica:csi}
\end{itemize}
\begin{itemize}
\item {Grp. gram.:adj.}
\end{itemize}
\begin{itemize}
\item {Proveniência:(De \textunderscore fluxo\textunderscore )}
\end{itemize}
Que é susceptível de fluxão.
Instável, transitório.
\section{Fluxo}
\begin{itemize}
\item {fónica:cso}
\end{itemize}
\begin{itemize}
\item {Grp. gram.:m.}
\end{itemize}
\begin{itemize}
\item {Utilização:Fig.}
\end{itemize}
\begin{itemize}
\item {Grp. gram.:Adj.}
\end{itemize}
\begin{itemize}
\item {Proveniência:(Lat. \textunderscore fluxus\textunderscore )}
\end{itemize}
Preamar, enchente fluvial.
O espraiar (das ondas): \textunderscore há fluxo e refluxo\textunderscore .
Fluxão, corrimento de humores.
Vicissitude dos acontecimentos.
Abundância.
Substância muito fusível, com que se auxilia a fusão de outras.
Flúido.
Passageiro.
\section{Fó!}
\begin{itemize}
\item {Grp. gram.:interj.}
\end{itemize}
\begin{itemize}
\item {Utilização:Mad}
\end{itemize}
(Exprime repugnância)
\section{Foba}
\textunderscore m.\textunderscore  e \textunderscore adj. Bras. da Baía\textunderscore .
Indivíduo medroso.
Preguiçoso.
Palerma.
\section{Foca}
\begin{itemize}
\item {Grp. gram.:m.}
\end{itemize}
\begin{itemize}
\item {Utilização:Fam.}
\end{itemize}
Homem avarento, sovina, unhas de fome.
\section{Foca}
\begin{itemize}
\item {Grp. gram.:f.}
\end{itemize}
\begin{itemize}
\item {Utilização:Prov.}
\end{itemize}
\begin{itemize}
\item {Utilização:minh.}
\end{itemize}
O mesmo que \textunderscore buraco\textunderscore .
\section{Foca}
\begin{itemize}
\item {Grp. gram.:f.}
\end{itemize}
\begin{itemize}
\item {Grp. gram.:M.  e  f.}
\end{itemize}
\begin{itemize}
\item {Utilização:Pop.}
\end{itemize}
\begin{itemize}
\item {Proveniência:(Lat. \textunderscore phoca\textunderscore )}
\end{itemize}
O mesmo que \textunderscore phoca\textunderscore .
Gênero de mamíferos anfíbios.--Em Filinto e Camões, é do gênero masculino. Cf. \textunderscore D. Man.\textunderscore , I, 67:«\textunderscore ...matárão hum grande foca\textunderscore ».
Pessôa avarenta, sovina.
\section{Focal}
\begin{itemize}
\item {Grp. gram.:adj.}
\end{itemize}
Relativo ao foco.
\section{Focar}
\begin{itemize}
\item {Grp. gram.:v. t.}
\end{itemize}
\begin{itemize}
\item {Utilização:Neol.}
\end{itemize}
Pôr em foco.
Tomar por foco.
\section{Focinhada}
\begin{itemize}
\item {Grp. gram.:f.}
\end{itemize}
Pancada com o focinho; trombada.
\section{Focinhar}
\begin{itemize}
\item {Grp. gram.:v. t.  e  i.}
\end{itemize}
O mesmo que \textunderscore afocinhar\textunderscore .
\section{Focinheira}
\begin{itemize}
\item {Grp. gram.:f.}
\end{itemize}
\begin{itemize}
\item {Utilização:Chul.}
\end{itemize}
Focinho de porco.
Focinho.
Tromba.
Correia, que faz parte da cabeçada, por cima das ventas do animal.
Má cara.
Rosto carrancudo.
Cara.
Embarcação de pesca, na costa da Ericeira.
\section{Focinho}
\begin{itemize}
\item {Grp. gram.:m.}
\end{itemize}
\begin{itemize}
\item {Utilização:Chul.}
\end{itemize}
\begin{itemize}
\item {Utilização:Constr.}
\end{itemize}
Rosto de animal.
Rosto humano.
Saliência, quási sempre boleada, do piso de um degrau.
(Fórma dem. do lat. \textunderscore faux\textunderscore , \textunderscore faucis\textunderscore , foz, boca)
\section{Focinho-de-burro}
\begin{itemize}
\item {Grp. gram.:m.}
\end{itemize}
Espécie de antirrhino, (\textunderscore antirrhinium majus\textunderscore ).
\section{Focinhudo}
\begin{itemize}
\item {Grp. gram.:adj.}
\end{itemize}
\begin{itemize}
\item {Utilização:Fig.}
\end{itemize}
Que tem grande focinho.
Macambúzio; trombudo; casmurro:«\textunderscore ...muito focinhudo com as reflexões do infeliz\textunderscore ». Camillo, \textunderscore Corja\textunderscore , 22. Cf. \textunderscore Eufrosina\textunderscore , 209; Filinto, III, 227.
\section{Foco}
\begin{itemize}
\item {Grp. gram.:m.}
\end{itemize}
\begin{itemize}
\item {Utilização:Fig.}
\end{itemize}
\begin{itemize}
\item {Proveniência:(Lat. \textunderscore focus\textunderscore )}
\end{itemize}
Lugar, em que se põe a matéria combustível, no forno.
Ponto, donde sáem os raios sectores para certas curvas.
Ponto, para onde convergem os raios luminosos, depois de reflectidos por um espelho ou de refractados por uma lente.
Centro, ponto de convergência.
Séde.
Ponto, donde irradiam ou sáem emanações ou outras coisas: \textunderscore um foco de infecção\textunderscore .
\section{Fofa}
\begin{itemize}
\item {fónica:fô}
\end{itemize}
\begin{itemize}
\item {Grp. gram.:f.}
\end{itemize}
\begin{itemize}
\item {Utilização:Prov.}
\end{itemize}
\begin{itemize}
\item {Utilização:trasm.}
\end{itemize}
\begin{itemize}
\item {Utilização:Des.}
\end{itemize}
\begin{itemize}
\item {Grp. gram.:Pl.}
\end{itemize}
\begin{itemize}
\item {Utilização:Fig.}
\end{itemize}
\begin{itemize}
\item {Utilização:Gír.}
\end{itemize}
Antiga dança lasciva.
Espécie de bolo.
Fatanisca.
Espécie de vestuário de criança.
Difficuldades, trabalhos: \textunderscore meter-se em fofas\textunderscore .
Mentira.
\section{Fofar}
\begin{itemize}
\item {Grp. gram.:v. t.}
\end{itemize}
Pôr fofos em.
Ornar de fofos; afofar.
\section{Fofe}
\begin{itemize}
\item {Grp. gram.:m.}
\end{itemize}
\begin{itemize}
\item {Utilização:Pleb.}
\end{itemize}
O mesmo que \textunderscore phósphoro\textunderscore .
\section{Fofice}
\begin{itemize}
\item {Grp. gram.:f.}
\end{itemize}
Qualidade daquelle ou daquillo que é fofo.
Prosápia infundada.
\section{Fofo}
\begin{itemize}
\item {fónica:fô}
\end{itemize}
\begin{itemize}
\item {Grp. gram.:adj.}
\end{itemize}
\begin{itemize}
\item {Utilização:Fig.}
\end{itemize}
\begin{itemize}
\item {Grp. gram.:M.}
\end{itemize}
Molle; brando.
Macio.
Elástico.
Que cede facilmente ao tacto, á pressão: \textunderscore pão fofo\textunderscore .
Impostor, patarata.
Que alardeia fidalguia ou parentesco de fidalgos, deante de quem o não conhece.
Ornato fofo ou relevado, para vestuário.
\section{Fogaça}
\begin{itemize}
\item {Grp. gram.:f.}
\end{itemize}
\begin{itemize}
\item {Utilização:Ant.}
\end{itemize}
Grande bolo.
Bolo ou presente, que se offerece á capella ou á igreja, em festas populares e que é depois vendido em leilão.
Rapariga garrida, que conduz êsse bolo em açafate ou tabuleiro.
Presente, offerta de comestíveis, para baptizado ou boda de casamento.
(B. lat. hyp. \textunderscore focacia\textunderscore , de \textunderscore focus\textunderscore )
\section{Fogaceira}
\begin{itemize}
\item {Grp. gram.:f.}
\end{itemize}
Rapariga, que transporta a fogaça, em certas festas populares.
\section{Fogacho}
\begin{itemize}
\item {Grp. gram.:m.}
\end{itemize}
\begin{itemize}
\item {Utilização:Fig.}
\end{itemize}
\begin{itemize}
\item {Utilização:Bras}
\end{itemize}
\begin{itemize}
\item {Proveniência:(Do lat. hyp. \textunderscore focaculum\textunderscore )}
\end{itemize}
Pequena labareda.
Arrebatamento, manifestação de mau gênio.
Chamma súbita.
Calor no rosto, com vermelhidão.
\section{Fogagem}
\begin{itemize}
\item {Grp. gram.:f.}
\end{itemize}
\begin{itemize}
\item {Utilização:Fig.}
\end{itemize}
\begin{itemize}
\item {Proveniência:(De \textunderscore fogo\textunderscore )}
\end{itemize}
Calor do sangue, que se manifesta por erupção de pelle.
Borbulhagem.
Doença dos vegetaes, manifestada por cárie nas sementes.
Arrebatamento de gênio.
\section{Fogal}
\begin{itemize}
\item {Grp. gram.:m.}
\end{itemize}
Antigo imposto, que se pagava por cada fogo ou casa.
\section{Fogaleira}
\begin{itemize}
\item {Grp. gram.:f.}
\end{itemize}
\begin{itemize}
\item {Proveniência:(Do rad. de \textunderscore fogo\textunderscore )}
\end{itemize}
Pá de ferro, com que se tiram brasas do forno.
\section{Fogalha}
\begin{itemize}
\item {Grp. gram.:f.}
\end{itemize}
\begin{itemize}
\item {Proveniência:(De \textunderscore fogo\textunderscore )}
\end{itemize}
Lugar, onde arde o combustível, nos fornos de fundir artilharia. Cf. \textunderscore Leoni\textunderscore , \textunderscore Diccion. de Artilh.\textunderscore , inédito.
\section{Fogão}
\begin{itemize}
\item {Grp. gram.:m.}
\end{itemize}
\begin{itemize}
\item {Proveniência:(Do b. lat. \textunderscore foco\textunderscore , \textunderscore foconis\textunderscore )}
\end{itemize}
Lareira.
Lugar, em que se accende lume para cozinhar.
Espécie de caixa de ferro, com fornalha e chaminé, para cozinhar.
Apparelho de ferro, embebido em parede e em que se accende lume para aquecer salas ou aposentos.
Parte da peça de artilharia, em que está o ouvido.
\section{Fogar}
\begin{itemize}
\item {Grp. gram.:m.}
\end{itemize}
\begin{itemize}
\item {Utilização:Ant.}
\end{itemize}
Casa habitada.
Fogo; casal.
(Cp. cast. \textunderscore hogar\textunderscore , do lat. \textunderscore focale\textunderscore  ou \textunderscore focarium\textunderscore )
\section{Fogaracho}
\begin{itemize}
\item {Grp. gram.:m.}
\end{itemize}
\begin{itemize}
\item {Utilização:P. us.}
\end{itemize}
Fogacho; fogaréu.
Lumeeira. Cf. Eça, \textunderscore Padre Amaro\textunderscore , 423.
\section{Fogareiro}
\begin{itemize}
\item {Grp. gram.:m.}
\end{itemize}
\begin{itemize}
\item {Proveniência:(Do lat. hyp. \textunderscore focararium\textunderscore )}
\end{itemize}
Utensílio de barro ou ferro, com fornalha, para cozinhar.
\section{Fogaréo}
\begin{itemize}
\item {Grp. gram.:m.}
\end{itemize}
Fogueira.
Fogacho.
Tigela, em que se accendem matérias inflammáveis.
Ornato de pedra, que termina em labaredas.
(Cp. lat. \textunderscore focarius\textunderscore )
\section{Fogaréu}
\begin{itemize}
\item {Grp. gram.:m.}
\end{itemize}
Fogueira.
Fogacho.
Tigela, em que se accendem matérias inflammáveis.
Ornato de pedra, que termina em labaredas.
(Cp. lat. \textunderscore focarius\textunderscore )
\section{Fogir}
\begin{itemize}
\item {Grp. gram.:v. i.}
\end{itemize}
\begin{itemize}
\item {Utilização:Ant.}
\end{itemize}
O mesmo que \textunderscore fugir\textunderscore . Cf. \textunderscore Jorn. de África\textunderscore , c. VI, e \textunderscore passim\textunderscore .
\section{Fogo}
\begin{itemize}
\item {fónica:fô}
\end{itemize}
\begin{itemize}
\item {Grp. gram.:m.}
\end{itemize}
\begin{itemize}
\item {Utilização:Fig.}
\end{itemize}
\begin{itemize}
\item {Grp. gram.:Interj.}
\end{itemize}
\begin{itemize}
\item {Grp. gram.:Loc.}
\end{itemize}
\begin{itemize}
\item {Utilização:pop.}
\end{itemize}
\begin{itemize}
\item {Grp. gram.:Pl.}
\end{itemize}
\begin{itemize}
\item {Proveniência:(Do lat. \textunderscore focus\textunderscore )}
\end{itemize}
Desenvolvimento de luz e calor.
Calórico.
Lume.
Incêndio.
Lareira.
Residência de uma família.
Família: \textunderscore aldeia de 50 fogos\textunderscore .
Supplício da fogueira: \textunderscore o herege foi condemnado ao fogo\textunderscore .
Aquillo que serve para alumiar.
Cauterização com ferro candente.
Ardor.
Energia.
Paixão.
Fuzilaria.
Agitação.
(designativa de ordem para disparar armas)
\textunderscore Fogo viste, linguíça\textunderscore . Diz-se de uma coisa que desappareceu breve.
Abertura, por onde sái o fumo das chaminés.
\section{Fogo}
\begin{itemize}
\item {fónica:fô}
\end{itemize}
\begin{itemize}
\item {Grp. gram.:m.}
\end{itemize}
\begin{itemize}
\item {Utilização:Ant.}
\end{itemize}
O mesmo que \textunderscore martinega\textunderscore .
\section{Fogo-apagou}
\begin{itemize}
\item {Grp. gram.:f.}
\end{itemize}
\begin{itemize}
\item {Utilização:Bras}
\end{itemize}
Pequena pomba, cujo arrulho imita o seu nome.
\section{Fogo-de-santo-antão}
\begin{itemize}
\item {Grp. gram.:m.}
\end{itemize}
Nome, que se deu na Idade-Média a uma espécie de erysipela epidêmica. Cf. Deusdado, \textunderscore Escorços Trasm.\textunderscore , 17.
\section{Fogosamente}
\begin{itemize}
\item {Grp. gram.:adv.}
\end{itemize}
De modo fogoso.
\section{Fogosidade}
\begin{itemize}
\item {Grp. gram.:f.}
\end{itemize}
Qualidade daquelle ou daquillo que é fogoso.
\section{Fogoso}
\begin{itemize}
\item {Grp. gram.:adj.}
\end{itemize}
\begin{itemize}
\item {Utilização:Fig.}
\end{itemize}
\begin{itemize}
\item {Proveniência:(De \textunderscore fogo\textunderscore )}
\end{itemize}
Que tem fogo ou calor.
Ardente; esbraseado.
Caloroso.
Impetuoso.
Irrequieto.
Violento; irascível.
\section{Foguear}
\begin{itemize}
\item {Grp. gram.:v. t.}
\end{itemize}
\begin{itemize}
\item {Utilização:P. us.}
\end{itemize}
\begin{itemize}
\item {Grp. gram.:V. i.}
\end{itemize}
\begin{itemize}
\item {Utilização:Prov.}
\end{itemize}
\begin{itemize}
\item {Proveniência:(De \textunderscore fogo\textunderscore )}
\end{itemize}
Queimar, afoguear.
Residir em; habitar.
Acender lume: \textunderscore naquella casa ninguém fogueava desde há muito\textunderscore .
\section{Fogueira}
\begin{itemize}
\item {Grp. gram.:f.}
\end{itemize}
\begin{itemize}
\item {Utilização:Fig.}
\end{itemize}
\begin{itemize}
\item {Utilização:Ant.}
\end{itemize}
\begin{itemize}
\item {Proveniência:(Do lat. \textunderscore focaria\textunderscore )}
\end{itemize}
Labareda.
Matéria combustível em labaredas.
Lume da lareira.
Ardor; exaltação.
Séde de uma família, casa, fogo.
\section{Fogueiro}
\begin{itemize}
\item {Grp. gram.:m.}
\end{itemize}
\begin{itemize}
\item {Proveniência:(Do lat. \textunderscore focarius\textunderscore )}
\end{itemize}
Aquelle, que trata das fornalhas, nas máquinas de vapor.
\section{Foguetada}
\begin{itemize}
\item {Grp. gram.:f.}
\end{itemize}
\begin{itemize}
\item {Utilização:Fig.}
\end{itemize}
\begin{itemize}
\item {Proveniência:(De \textunderscore foguete\textunderscore )}
\end{itemize}
Estampido de muitos foguetes, que estoiram ao mesmo tempo.
Girândola.
Descompostura.
\section{Foguetão}
\begin{itemize}
\item {Grp. gram.:m.}
\end{itemize}
\begin{itemize}
\item {Proveniência:(De \textunderscore foguete\textunderscore )}
\end{itemize}
Espécie de foguete, com que de algumas praias se atiram cabos a náufragos.
\section{Foguetaria}
\begin{itemize}
\item {Grp. gram.:f.}
\end{itemize}
Conjunto de foguetes.
O mesmo que \textunderscore foguetada\textunderscore . Cf. Castilho, \textunderscore Tartufo\textunderscore , 148.
\section{Foguetário}
\begin{itemize}
\item {Grp. gram.:m.}
\end{itemize}
\begin{itemize}
\item {Utilização:Ant.}
\end{itemize}
O mesmo que \textunderscore fogueteiro\textunderscore .
\section{Foguete}
\begin{itemize}
\item {fónica:guê}
\end{itemize}
\begin{itemize}
\item {Grp. gram.:m.}
\end{itemize}
\begin{itemize}
\item {Utilização:Fig.}
\end{itemize}
\begin{itemize}
\item {Utilização:Pop.}
\end{itemize}
\begin{itemize}
\item {Utilização:Mad}
\end{itemize}
\begin{itemize}
\item {Proveniência:(De \textunderscore fogo\textunderscore )}
\end{itemize}
Peça de fogo de artifício, formada por uma cana, que tem na extremidade uma espécie de busca-pé em communicação com um cartucho cheio de bombas ou de matérias inflammáveis.
Descompostura.
Cartuchinho longo e estreito, cheio de confeitos.
Lanço de estrada ou caminho íngreme, com calcetamento de seixos, em fórma de degraus, para facilitar a subida ou a descida.
\section{Foguetear}
\begin{itemize}
\item {Grp. gram.:v. i.}
\end{itemize}
\begin{itemize}
\item {Utilização:Prov.}
\end{itemize}
\begin{itemize}
\item {Utilização:beir.}
\end{itemize}
Deitar ou queimar foguetes.
Fugir depressa, como um foguete. (Colhido na Guarda)
\section{Fogueteiro}
\begin{itemize}
\item {Grp. gram.:m.}
\end{itemize}
Fabricante de foguetes e de outras peças de fogo.
\section{Foguetório}
\begin{itemize}
\item {Grp. gram.:m.}
\end{itemize}
O mesmo que \textunderscore foguetada\textunderscore .
Festa, em que se deitam foguetes.
\section{Foguinho}
\begin{itemize}
\item {Grp. gram.:m.}
\end{itemize}
\begin{itemize}
\item {Utilização:Prov.}
\end{itemize}
\begin{itemize}
\item {Utilização:alent.}
\end{itemize}
Terceira das divisões, que se traçam no chão para o jôgo da calha.
\section{Foguista}
\begin{itemize}
\item {Grp. gram.:m.}
\end{itemize}
\begin{itemize}
\item {Utilização:Bras}
\end{itemize}
\begin{itemize}
\item {Proveniência:(De \textunderscore fogo\textunderscore )}
\end{itemize}
O mesmo que \textunderscore fogueiro\textunderscore .
\section{Fóia}
\begin{itemize}
\item {Grp. gram.:f.}
\end{itemize}
\begin{itemize}
\item {Utilização:Prov.}
\end{itemize}
\begin{itemize}
\item {Utilização:trasm.}
\end{itemize}
\begin{itemize}
\item {Proveniência:(Do lat. \textunderscore fovea\textunderscore )}
\end{itemize}
Buraco em a terra.
Cavidade circular, para onde se atira a castanha no jôgo do fóio.
\section{Foiaíte}
\begin{itemize}
\item {Grp. gram.:f.}
\end{itemize}
Espécie de rocha, que contém orthoclase, oleólitho, mica, etc., e que se encontra principalmente em Fóia de Monchique.
\section{Foiaíto}
\begin{itemize}
\item {Grp. gram.:m.}
\end{itemize}
Espécie de rocha, que contém orthoclase, oleólitho, mica, etc., e que se encontra principalmente em Fóia de Monchique.
\section{Foiçada}
\begin{itemize}
\item {Grp. gram.:f.}
\end{itemize}
Golpe com foice.
\section{Foiçar}
\begin{itemize}
\item {Grp. gram.:v. t.}
\end{itemize}
Cortar com foice.
Ceifar; segar.
\section{Foice}
\begin{itemize}
\item {Grp. gram.:f.}
\end{itemize}
\begin{itemize}
\item {Utilização:Fig.}
\end{itemize}
\begin{itemize}
\item {Proveniência:(Do lat. \textunderscore falx\textunderscore , \textunderscore falcis\textunderscore )}
\end{itemize}
Instrumento curvo para ceifar ou segar.
Membrana, que tem a configuração do peritonéu.
Instrumento symbólico, com que é representado o tempo.
\textunderscore Foice roçadoira\textunderscore , o mesmo que \textunderscore roçadoira\textunderscore .
\section{Foicear}
\begin{itemize}
\item {Grp. gram.:v. i.}
\end{itemize}
\begin{itemize}
\item {Utilização:Neol.}
\end{itemize}
Meter a foice; fazer golpes com a foice.
\section{Foiciforme}
\begin{itemize}
\item {Grp. gram.:adj.}
\end{itemize}
\begin{itemize}
\item {Proveniência:(De \textunderscore foice\textunderscore  + \textunderscore forma\textunderscore )}
\end{itemize}
Que tem fórma de foice.
\section{Foicinha}
\begin{itemize}
\item {Grp. gram.:f.}
\end{itemize}
O mesmo que \textunderscore foicinho\textunderscore .
\section{Foicinhão}
\begin{itemize}
\item {Grp. gram.:m.}
\end{itemize}
Foice grande, com que se corta palha em miúdos, e que, para isso, está fixa em parede.
\section{Foicinho}
\begin{itemize}
\item {Grp. gram.:m.}
\end{itemize}
Pequena foice.
\section{Foicisca}
\begin{itemize}
\item {Grp. gram.:m.}
\end{itemize}
\begin{itemize}
\item {Utilização:Prov.}
\end{itemize}
\begin{itemize}
\item {Utilização:dur.}
\end{itemize}
Foice pequena.
\section{Foila}
\begin{itemize}
\item {Grp. gram.:f.}
\end{itemize}
\begin{itemize}
\item {Utilização:Prov.}
\end{itemize}
\begin{itemize}
\item {Utilização:trasm.}
\end{itemize}
O mesmo que \textunderscore foina\textunderscore .
\section{Foina}
\begin{itemize}
\item {Grp. gram.:f.}
\end{itemize}
\begin{itemize}
\item {Utilização:Prov.}
\end{itemize}
\begin{itemize}
\item {Utilização:trasm.}
\end{itemize}
\begin{itemize}
\item {Grp. gram.:Pl.}
\end{itemize}
O mesmo que \textunderscore fona\textunderscore , centelha.
Farinha fina, que se levanta da moedura e poisa nas paredes do moínho.
\section{Fóio}
\begin{itemize}
\item {Grp. gram.:m.}
\end{itemize}
\begin{itemize}
\item {Utilização:Prov.}
\end{itemize}
\begin{itemize}
\item {Utilização:trasm.}
\end{itemize}
Jôgo de rapazes, por meio de castanhas, ganhando aquelle que mete uma castanha na fóia.
(Cp. \textunderscore fóia\textunderscore ^1)
\section{Foiteza}
\begin{itemize}
\item {Grp. gram.:f.}
\end{itemize}
Qualidade de quem é foito.
Afoiteza.
\section{Foito}
\begin{itemize}
\item {Grp. gram.:adj.}
\end{itemize}
\begin{itemize}
\item {Proveniência:(Do lat. \textunderscore fultus\textunderscore )}
\end{itemize}
O mesmo que \textunderscore afoito\textunderscore .
\section{Fojo}
\begin{itemize}
\item {fónica:fô}
\end{itemize}
\begin{itemize}
\item {Grp. gram.:m.}
\end{itemize}
\begin{itemize}
\item {Utilização:Bras. do N}
\end{itemize}
\begin{itemize}
\item {Proveniência:(Do b. lat. \textunderscore fogium\textunderscore )}
\end{itemize}
Cova funda, cuja abertura se tapa ou se disfarça com ramos, para nella se apanharem, vivos, animaes ferozes.
Cova semelhante, que se faz durante a guerra, para colher inimigos.
Sorvedoiro para águas.
Caverna.
Armadilha, para apanhar ratos ou caça miúda.
\section{Folão}
\begin{itemize}
\item {Grp. gram.:adj.}
\end{itemize}
\begin{itemize}
\item {Utilização:Des.}
\end{itemize}
Fogoso:«\textunderscore asno que me leve quero, e não cavallo folão.\textunderscore »G. Vicente, \textunderscore Inês Pereira\textunderscore .
(Por \textunderscore folião\textunderscore ?)
\section{Folar}
\begin{itemize}
\item {Grp. gram.:m.}
\end{itemize}
\begin{itemize}
\item {Proveniência:(Do lat. \textunderscore floralis\textunderscore ?)}
\end{itemize}
O mesmo que \textunderscore fogaça\textunderscore .
Bolo ou presente, que os padrinhos dão aos afilhados, ou os parochianos ao párocho, pela Páschoa.
Direito parochial de receber êsse bolo ou êsse presente.
\section{Folchlore}
\begin{itemize}
\item {Grp. gram.:m.}
\end{itemize}
\begin{itemize}
\item {Proveniência:(Do ingl. \textunderscore folklore\textunderscore )}
\end{itemize}
Conjunto das tradições, conhecimentos ou crenças populares, expressas em provérbios, contos ou canções.
Conjunto das canções populares de uma época ou região.
\section{Folclorista}
\begin{itemize}
\item {Grp. gram.:m.}
\end{itemize}
\begin{itemize}
\item {Proveniência:(De \textunderscore folclore\textunderscore )}
\end{itemize}
Investigador ou colleccionador de tradições ou canções populares.
Aquelle que faz a história crítica dessas tradições e canções.
\section{Fole}
\begin{itemize}
\item {Grp. gram.:m.}
\end{itemize}
Pequena árvore da Guiné, de frutos redondos e ácidos.
\section{Foleca}
\begin{itemize}
\item {Grp. gram.:f.}
\end{itemize}
\begin{itemize}
\item {Utilização:Prov.}
\end{itemize}
O mesmo que \textunderscore folheca\textunderscore .
\section{Foleco}
\begin{itemize}
\item {Grp. gram.:m.}
\end{itemize}
\begin{itemize}
\item {Utilização:Prov.}
\end{itemize}
\begin{itemize}
\item {Utilização:trasm.}
\end{itemize}
Nome de um pássaro.
\section{Fole-de-elephante}
\begin{itemize}
\item {Grp. gram.:m.}
\end{itemize}
Grande árvore da Guiné, de frutos comestíveis.
\section{Fôlego}
\begin{itemize}
\item {Grp. gram.:m.}
\end{itemize}
\begin{itemize}
\item {Proveniência:(Do hypoth. \textunderscore folegar\textunderscore  &lt; \textunderscore folgar\textunderscore )}
\end{itemize}
Respiração.
Acto de soprar.
Folga.
\section{Folengar}
\begin{itemize}
\item {Grp. gram.:v. t.}
\end{itemize}
\begin{itemize}
\item {Utilização:Ant.}
\end{itemize}
O mesmo que \textunderscore folgar\textunderscore .
\section{Folestria}
\textunderscore f.\textunderscore  (e der.)
O mesmo que \textunderscore flostria\textunderscore , etc.
\section{Folga}
\begin{itemize}
\item {Grp. gram.:f.}
\end{itemize}
\begin{itemize}
\item {Utilização:Prov.}
\end{itemize}
\begin{itemize}
\item {Utilização:alent.}
\end{itemize}
\begin{itemize}
\item {Utilização:Fig.}
\end{itemize}
\begin{itemize}
\item {Utilização:Açor. do Pico}
\end{itemize}
Acto de folgar.
Ócio.
Descanso; tempo de descanso.
Recreio.
Saliência no bôrdo da ferradura.
O mesmo que \textunderscore sesta\textunderscore : \textunderscore o João foi dormir a folga\textunderscore .
Desafôgo.
Bailado popular.
\section{Folgadamente}
\begin{itemize}
\item {Grp. gram.:adv.}
\end{itemize}
De modo folgado.
\section{Folgado}
\begin{itemize}
\item {Grp. gram.:adj.}
\end{itemize}
\begin{itemize}
\item {Proveniência:(De \textunderscore folgar\textunderscore )}
\end{itemize}
Que tem folga.
Que não está cansado ou que tem andado pouco: \textunderscore cavallo folgado\textunderscore .
Amplo, bem medido ou que tem maiores dimensões que as necessárias: \textunderscore casaco folgado\textunderscore .
\section{Folgador}
\begin{itemize}
\item {Grp. gram.:adj.}
\end{itemize}
O mesmo que \textunderscore folgazão\textunderscore ^1.
\section{Folgança}
\begin{itemize}
\item {Grp. gram.:f.}
\end{itemize}
Acto de folgar.
Folga; folguedo.
Brincadeira ruidosa.
\section{Folgante}
\begin{itemize}
\item {Grp. gram.:m.}
\end{itemize}
Aquelle que folga.
\section{Folgão}
\begin{itemize}
\item {Grp. gram.:adj.}
\end{itemize}
\begin{itemize}
\item {Utilização:Des.}
\end{itemize}
\begin{itemize}
\item {Proveniência:(De \textunderscore folgar\textunderscore )}
\end{itemize}
O mesmo que \textunderscore folgazão\textunderscore ^1. Cf. \textunderscore Agostinheida\textunderscore , 2 e 108.
\section{Folgar}
\begin{itemize}
\item {Grp. gram.:v. t.}
\end{itemize}
\begin{itemize}
\item {Grp. gram.:V. i.}
\end{itemize}
Dar folga a.
Desapertar.
Pôr á vontade.
Tornar largo: \textunderscore folgar uma peça de vestuário\textunderscore .
Têr descanso.
Têr prazer, alegrar-se.
Estar desafogado; divertir-se.
(Contr. de \textunderscore folegar\textunderscore , do lat. hyp. \textunderscore follicare\textunderscore )
\section{Folgativo}
\begin{itemize}
\item {Grp. gram.:adj.}
\end{itemize}
Que tem folga; folião.
\section{Folgaz}
\begin{itemize}
\item {Utilização:Ant.}
\end{itemize}
\begin{itemize}
\item {Proveniência:(De \textunderscore folgar\textunderscore )}
\end{itemize}
\textunderscore adj.\textunderscore  (des.)
Que gosta de folgar.
Alegre; brincalhão; galhofeiro.
Casta de uva branca.
O mesmo que \textunderscore mandrião\textunderscore .
\section{Folgazão}
\begin{itemize}
\item {Grp. gram.:adj.}
\end{itemize}
\begin{itemize}
\item {Utilização:Ant.}
\end{itemize}
\begin{itemize}
\item {Proveniência:(De \textunderscore folgar\textunderscore )}
\end{itemize}
Que gosta de folgar.
Alegre; brincalhão; galhofeiro.
Casta de uva branca.
O mesmo que \textunderscore mandrião\textunderscore .
\section{Folgazão}
\begin{itemize}
\item {Grp. gram.:m.}
\end{itemize}
(Corr. de \textunderscore folgosão\textunderscore )
\section{Folgazar}
\begin{itemize}
\item {Grp. gram.:v. i.}
\end{itemize}
\begin{itemize}
\item {Proveniência:(De \textunderscore folgaz\textunderscore )}
\end{itemize}
O mesmo que \textunderscore folgar\textunderscore .
\section{Folgo}
\begin{itemize}
\item {fónica:fôl}
\end{itemize}
\begin{itemize}
\item {Grp. gram.:m.}
\end{itemize}
\begin{itemize}
\item {Utilização:T. do Ribatejo}
\end{itemize}
\begin{itemize}
\item {Utilização:Ant.}
\end{itemize}
O mesmo que \textunderscore fôlego\textunderscore .
Litro de vinho.
Folgança, divertimento. Cf. \textunderscore Alvará\textunderscore  de D. Sebast., in \textunderscore Rev. Lus.\textunderscore , XV, 116.
\section{Fola}
\begin{itemize}
\item {Grp. gram.:f.}
\end{itemize}
\begin{itemize}
\item {Proveniência:(It. \textunderscore folla\textunderscore )}
\end{itemize}
Marulho de ondas.
\section{Folacho}
\begin{itemize}
\item {Grp. gram.:m.  e  f.}
\end{itemize}
\begin{itemize}
\item {Utilização:Pop.}
\end{itemize}
\begin{itemize}
\item {Proveniência:(De \textunderscore fole\textunderscore )}
\end{itemize}
Pessôa branda, doente, fraca.
\section{Fole}
\begin{itemize}
\item {Grp. gram.:m.}
\end{itemize}
\begin{itemize}
\item {Utilização:Fam.}
\end{itemize}
\begin{itemize}
\item {Grp. gram.:Pl.}
\end{itemize}
\begin{itemize}
\item {Grp. gram.:Adj.}
\end{itemize}
\begin{itemize}
\item {Utilização:Prov.}
\end{itemize}
\begin{itemize}
\item {Utilização:beir.}
\end{itemize}
\begin{itemize}
\item {Proveniência:(Lat. \textunderscore follis\textunderscore )}
\end{itemize}
Utensílio, que, pôsto em movimento, produz vento, para activar uma combustão ou para limpar certas cavidades.
Taleiga de coiro.
Papo, estômago: \textunderscore já enchi o fole\textunderscore .
Passadeira de coiro, nos arreios dos muares de diligências.
Parte de um instrumento músico, vulgarmente conhecido por \textunderscore gaita de foles\textunderscore  ou \textunderscore gaita galega\textunderscore .
Diz-se do fruto mole ou sorvado: \textunderscore um pero fole\textunderscore .
\section{Folecha}
\begin{itemize}
\item {fónica:lê}
\end{itemize}
\begin{itemize}
\item {Grp. gram.:f.}
\end{itemize}
\begin{itemize}
\item {Utilização:Prov.}
\end{itemize}
\begin{itemize}
\item {Utilização:beir.}
\end{itemize}
Espécie de passarinho cinzento.
O mesmo que \textunderscore folosa\textunderscore ?
Empôla, o mesmo que \textunderscore folecho\textunderscore .
\section{Folecho}
\begin{itemize}
\item {fónica:lê}
\end{itemize}
\begin{itemize}
\item {Grp. gram.:m.}
\end{itemize}
\begin{itemize}
\item {Utilização:Pop.}
\end{itemize}
\begin{itemize}
\item {Proveniência:(De \textunderscore fole\textunderscore )}
\end{itemize}
Empôla.
Bôlha na pele, contendo aguadilha.
\section{Folega}
\begin{itemize}
\item {Grp. gram.:f.}
\end{itemize}
\begin{itemize}
\item {Utilização:Prov.}
\end{itemize}
\begin{itemize}
\item {Utilização:minh.}
\end{itemize}
Sova, tareia, tunda.
\section{Foleiro}
\begin{itemize}
\item {Grp. gram.:m.}
\end{itemize}
\begin{itemize}
\item {Utilização:Prov.}
\end{itemize}
Fabricante, vendedor ou tocador de foles.
Burro de moleiro.
\section{Folepo}
\begin{itemize}
\item {fónica:lê}
\end{itemize}
\begin{itemize}
\item {Grp. gram.:m.}
\end{itemize}
O mesmo que \textunderscore folecho\textunderscore .
Espécie de fole, que se fórma no fato mal costurado; folipo.
\section{Folgosão}
\begin{itemize}
\item {Grp. gram.:m.}
\end{itemize}
\begin{itemize}
\item {Proveniência:(De \textunderscore Folgosa\textunderscore , n. p.)}
\end{itemize}
Variedade de uva branca.
\section{Folgosinho}
\begin{itemize}
\item {Grp. gram.:m.}
\end{itemize}
\begin{itemize}
\item {Proveniência:(De \textunderscore Folgosa\textunderscore , n. p.)}
\end{itemize}
Casta de uva preta do Minho.
\section{Folguedo}
\begin{itemize}
\item {fónica:guê}
\end{itemize}
\begin{itemize}
\item {Grp. gram.:m.}
\end{itemize}
Acto de folgar.
Divertimento.
Brincadeira; pândega.
\section{Folgura}
\begin{itemize}
\item {Grp. gram.:f.}
\end{itemize}
\begin{itemize}
\item {Utilização:Ant.}
\end{itemize}
Folgança.
Recreio.
Satisfação:«\textunderscore não há prazer nem folgura.\textunderscore »G. Vicente, \textunderscore Auto da Índia\textunderscore .
\section{Fôlha}
\begin{itemize}
\item {Grp. gram.:f.}
\end{itemize}
\begin{itemize}
\item {Utilização:Prov.}
\end{itemize}
\begin{itemize}
\item {Utilização:alent.}
\end{itemize}
\begin{itemize}
\item {Utilização:Prov.}
\end{itemize}
\begin{itemize}
\item {Utilização:trasm.}
\end{itemize}
\begin{itemize}
\item {Proveniência:(Do lat. \textunderscore folia\textunderscore , pl. de \textunderscore folium\textunderscore )}
\end{itemize}
Parte dos vegetaes, membranosa, chata e geralmente verde, que nasce da haste ou nos ramos.
Pétala.
Lâmina de metal.
Ornato, que representa folhagem.
Parte cortante de alguns instrumentos.
Pedaço quadrilongo de papel.
Papel, que se imprime de uma vez, produzindo certo número de páginas.
Cada um dos papéis, dobrados pelo meio em duas partes, os quaes constituem um caderno ou uma resma.
Relação, rol.
Periódico.
Cada uma das palhetas, lascas ou partes delgadas, em que se divide um todo.
Seara.
\textunderscore Folha de caça\textunderscore , pista ou rasto de caça.
\textunderscore Novo em folha\textunderscore , que ainda não foi usado.
\section{Folhada}
\begin{itemize}
\item {Grp. gram.:f.}
\end{itemize}
\begin{itemize}
\item {Utilização:T. de Aveiro}
\end{itemize}
\begin{itemize}
\item {Proveniência:(Lat. \textunderscore foliata\textunderscore )}
\end{itemize}
Porção de folhas caídas.
Folhagem.
Planta lonicéria.
Alga marinha, (\textunderscore ulva latissima\textunderscore , Rg.).
\section{Fôlha-da-fortuna}
\begin{itemize}
\item {Grp. gram.:f.}
\end{itemize}
\begin{itemize}
\item {Utilização:Bras}
\end{itemize}
Planta singular, que grela na margem das folhas, ainda depois de estas se despegarem da planta. Cf. Rubim, \textunderscore Vocabulário Bras.\textunderscore 
\section{Fôlha-de-cunorga}
\begin{itemize}
\item {Grp. gram.:f.}
\end{itemize}
Planta trepadeira, cujas raízes tuberculosas são applicadas pelos indígenas da Guiné, contra as escrófulas.
\section{Fôlha-de-figueira}
\begin{itemize}
\item {Grp. gram.:f.}
\end{itemize}
Casta de uva do districto de Lisbôa.
\section{Fôlha-de-fogo}
\begin{itemize}
\item {Grp. gram.:f.}
\end{itemize}
\begin{itemize}
\item {Utilização:Bras}
\end{itemize}
Arbusto, de fôlhas medicinaes.
\section{Fôlha-de-sabão}
\begin{itemize}
\item {Grp. gram.:f.}
\end{itemize}
Arbusto da Guiné, cujas folhas e raizes se applicam contra a sarna.
\section{Fôlha-de-tara}
\begin{itemize}
\item {Grp. gram.:f.}
\end{itemize}
Arvore medicinal da Guiné.
\section{Folhado}
\begin{itemize}
\item {Grp. gram.:m.}
\end{itemize}
\begin{itemize}
\item {Utilização:Prov.}
\end{itemize}
\begin{itemize}
\item {Utilização:beir.}
\end{itemize}
\begin{itemize}
\item {Utilização:Fig.}
\end{itemize}
\begin{itemize}
\item {Utilização:Bras}
\end{itemize}
Massa estendida para pastéis, empadas, etc.
Folhagem.
Acto ou effeito de folhar.
Fôlhas, caídas do castanheiro.
Palavras vans.
Funileiro, latoeiro.
\section{Folhadura}
\begin{itemize}
\item {Grp. gram.:f.}
\end{itemize}
O mesmo que \textunderscore folheatura\textunderscore .
\section{Fôlha-flôr}
\begin{itemize}
\item {Grp. gram.:f.}
\end{itemize}
Peça muito delgada de madeira, de recorte caprichoso, que os marceneiros applicam aos móveis.
\section{Fôlha-formiga}
\begin{itemize}
\item {Grp. gram.:f.}
\end{itemize}
Planta medicinal da ilha de San-Thomé.
\section{Folhagem}
\begin{itemize}
\item {Grp. gram.:f.}
\end{itemize}
Fôlhas de uma planta.
Porção de fôlhas.
Ornato, que imita flôres ou fôlhas.
Ramaria dos arvoredos.
\section{Folhal}
\begin{itemize}
\item {Grp. gram.:m.}
\end{itemize}
Variedade de uva preta do Minho.
\section{Fôlha-larga}
\begin{itemize}
\item {Grp. gram.:m.}
\end{itemize}
\begin{itemize}
\item {Utilização:Bras}
\end{itemize}
O mesmo que \textunderscore andrequicé\textunderscore .
\section{Folhame}
\begin{itemize}
\item {Grp. gram.:m.}
\end{itemize}
(V.folhagem)
\section{Folhão}
\begin{itemize}
\item {Grp. gram.:m.}
\end{itemize}
\begin{itemize}
\item {Proveniência:(De \textunderscore folho\textunderscore )}
\end{itemize}
Cavallo, que tem excrescência no casco.
\section{Folhão}
\begin{itemize}
\item {Grp. gram.:adj.}
\end{itemize}
O mesmo que \textunderscore folão\textunderscore .
\section{Folhão}
\begin{itemize}
\item {Grp. gram.:m.}
\end{itemize}
Fôlho grande. Cf. Camillo, \textunderscore Caveira\textunderscore , 451.
\section{Fôlha-pequena}
\begin{itemize}
\item {Grp. gram.:f.}
\end{itemize}
Árvore da ilha de San-Thomé.
\section{Folhar}
\begin{itemize}
\item {Grp. gram.:v. t.}
\end{itemize}
\begin{itemize}
\item {Grp. gram.:V. i.}
\end{itemize}
Fazer produzir fôlhas a.
Tornar semelhante a fôlhas: \textunderscore folhar um pastel\textunderscore .
Ornar com folhagem.
Lavrar ou pintar folhagem em.
Revestir de lâminas.
Cobrir-se de fôlhas.
\section{Fôlha-rachada}
\begin{itemize}
\item {Grp. gram.:f.}
\end{itemize}
Variedade de videira da Bairrada.
\section{Folharia}
\begin{itemize}
\item {Grp. gram.:f.}
\end{itemize}
(V.folhagem)
\section{Fôlha-santa}
\begin{itemize}
\item {Grp. gram.:f.}
\end{itemize}
Planta medicinal da Guiné.
\section{Fôlhas-da-trindade}
\begin{itemize}
\item {Grp. gram.:f. pl.}
\end{itemize}
Leguminosa indiana, também conhecida por \textunderscore pongueró\textunderscore .
\section{Folhato}
\begin{itemize}
\item {Grp. gram.:m.}
\end{itemize}
\begin{itemize}
\item {Utilização:Bras}
\end{itemize}
O mesmo que \textunderscore folhelho\textunderscore .
\section{Folheação}
\begin{itemize}
\item {Grp. gram.:f.}
\end{itemize}
\begin{itemize}
\item {Utilização:Bot.}
\end{itemize}
\begin{itemize}
\item {Proveniência:(De \textunderscore folhear\textunderscore )}
\end{itemize}
Acto de se revestir de fôlhas (a planta).
Folheatura.
\section{Folheáceo}
\begin{itemize}
\item {Grp. gram.:adj.}
\end{itemize}
(V.foliáceo)
\section{Folheado}
\begin{itemize}
\item {Grp. gram.:m.}
\end{itemize}
\begin{itemize}
\item {Proveniência:(De \textunderscore folhear\textunderscore )}
\end{itemize}
Lamina de madeira ou metal, com que se revestem móveis.
\section{Folheador}
\begin{itemize}
\item {Grp. gram.:adj.}
\end{itemize}
\begin{itemize}
\item {Proveniência:(De \textunderscore folhear\textunderscore ^1)}
\end{itemize}
Que volve as fôlhas dos livros, lendo-as por alto.
\section{Folhear}
\begin{itemize}
\item {Grp. gram.:v. t.}
\end{itemize}
\begin{itemize}
\item {Proveniência:(De \textunderscore folha\textunderscore )}
\end{itemize}
Percorrer, lêr perfunctoriamente, as fôlhas de (um livro, folheto, revista, etc.)
Consultar, lêr.
Dividir em folhas.
Cobrir com laminas de madeira ou metal: \textunderscore folhear uma cómmoda\textunderscore .
Prover de fôlhas, folhar.
\section{Folhear}
\begin{itemize}
\item {Grp. gram.:adj.}
\end{itemize}
Composto de fôlhas.
Relativo a fôlhas.
(Por \textunderscore folhiar\textunderscore , do lat. hyp. \textunderscore foliaris\textunderscore , de \textunderscore folium\textunderscore )
\section{Folheatura}
\begin{itemize}
\item {Grp. gram.:f.}
\end{itemize}
\begin{itemize}
\item {Proveniência:(Do lat. \textunderscore foliatura\textunderscore )}
\end{itemize}
Acto de folhear.
Época em que rebentam as fôlhas.
Vernação.
\section{Folheca}
\begin{itemize}
\item {Grp. gram.:f.}
\end{itemize}
\begin{itemize}
\item {Proveniência:(De \textunderscore fôlha\textunderscore )}
\end{itemize}
Frocos de neve.
\section{Folhedo}
\begin{itemize}
\item {fónica:lhê}
\end{itemize}
\begin{itemize}
\item {Grp. gram.:m.}
\end{itemize}
\begin{itemize}
\item {Proveniência:(De \textunderscore fôlha\textunderscore )}
\end{itemize}
Conjunto de fôlhas desprendidas de arvore.
Folhagem. Cf. Castilho, \textunderscore Geórgicas\textunderscore , 103 e 107.
\section{Folheio}
\begin{itemize}
\item {Grp. gram.:m.}
\end{itemize}
Acto de folhear (livros). Cf. Filinto, VIII, 80.
\section{Folheira}
\begin{itemize}
\item {Grp. gram.:adj. f.}
\end{itemize}
\begin{itemize}
\item {Utilização:Prov.}
\end{itemize}
\begin{itemize}
\item {Utilização:trasm.}
\end{itemize}
\begin{itemize}
\item {Proveniência:(De \textunderscore fôlha\textunderscore )}
\end{itemize}
Diz-se da farinha não remoída.
\section{Folheiro}
\begin{itemize}
\item {Grp. gram.:adj.}
\end{itemize}
\begin{itemize}
\item {Utilização:Bras. do S}
\end{itemize}
\begin{itemize}
\item {Proveniência:(De \textunderscore fôlho\textunderscore )}
\end{itemize}
Vistoso; airoso; farfalhudo.
\section{Folheiro}
\begin{itemize}
\item {Grp. gram.:adj.}
\end{itemize}
\begin{itemize}
\item {Proveniência:(De \textunderscore fôlha\textunderscore )}
\end{itemize}
Que junta as fôlhas sêcas das árvores. Cf. \textunderscore Vir. Trág.\textunderscore , VIII, 43.
\section{Folhelho}
\begin{itemize}
\item {fónica:lhê}
\end{itemize}
\begin{itemize}
\item {Grp. gram.:m.}
\end{itemize}
\begin{itemize}
\item {Proveniência:(Do lat. \textunderscore folliculus\textunderscore )}
\end{itemize}
Pellícula, que reveste a espiga do milho, o bago da uva, legumes, etc.
O mesmo que \textunderscore folhedo\textunderscore  e \textunderscore folhado\textunderscore .
\section{Folhento}
\begin{itemize}
\item {Grp. gram.:adj.}
\end{itemize}
Que tem muitas fôlhas; folhudo.
Copado.
\section{Folhepo}
\begin{itemize}
\item {fónica:lhê}
\end{itemize}
\begin{itemize}
\item {Grp. gram.:m.}
\end{itemize}
(V.folheca)
\section{Folheta}
\begin{itemize}
\item {fónica:lhê}
\end{itemize}
\begin{itemize}
\item {Grp. gram.:f.}
\end{itemize}
\begin{itemize}
\item {Utilização:Prov.}
\end{itemize}
Pequena fôlha.
O mesmo que \textunderscore lata\textunderscore ^1.
\section{Folhetaria}
\begin{itemize}
\item {Grp. gram.:f.}
\end{itemize}
Folhagem desenhada ou pintada.
Collecção de folhetos.
\section{Folhetear}
\begin{itemize}
\item {Grp. gram.:v. t.}
\end{itemize}
Pôr folhetas em (pedras).
Engastar.
Folhear (móveis).
\section{Folheteira}
\begin{itemize}
\item {Grp. gram.:f.}
\end{itemize}
\begin{itemize}
\item {Utilização:Prov.}
\end{itemize}
\begin{itemize}
\item {Utilização:minh.}
\end{itemize}
\begin{itemize}
\item {Proveniência:(De \textunderscore folheta\textunderscore )}
\end{itemize}
Pescaria nos pegos marginaes dos rios.
\section{Folhetim}
\begin{itemize}
\item {Grp. gram.:m.}
\end{itemize}
\begin{itemize}
\item {Proveniência:(Do cast. \textunderscore folletin\textunderscore )}
\end{itemize}
Secção literária de um periódico, occupando geralmente toda ou quási toda a parte inferior de uma ou mais páginas.
\section{Folhetinista}
\begin{itemize}
\item {Grp. gram.:m.}
\end{itemize}
Aquelle que escreve folhetins.
\section{Folhetinístico}
\begin{itemize}
\item {Grp. gram.:adj.}
\end{itemize}
Relativo a folhetinista.
\section{Folhetinizar}
\begin{itemize}
\item {Grp. gram.:v. t.}
\end{itemize}
\begin{itemize}
\item {Utilização:Neol.}
\end{itemize}
Descrever em folhetins.
\section{Folheto}
\begin{itemize}
\item {fónica:lhê}
\end{itemize}
\begin{itemize}
\item {Grp. gram.:m.}
\end{itemize}
\begin{itemize}
\item {Proveniência:(De \textunderscore fôlha\textunderscore )}
\end{itemize}
Livro de poucas fôlhas, brochado.
Laminas, que constituem a parte inferior da cabeça dos agáricos.
Cada uma das partes de um corpo lamelloso.
\section{Folhido}
\begin{itemize}
\item {Grp. gram.:m.}
\end{itemize}
\begin{itemize}
\item {Utilização:Prov.}
\end{itemize}
Acervo de fôlhas caídas.
O mesmo que \textunderscore folhada\textunderscore .
O mesmo que \textunderscore folhelho\textunderscore . Cf. J. Dinís, \textunderscore Pupillas\textunderscore , 173 e 174.
\section{Folhinha}
\begin{itemize}
\item {Grp. gram.:f.}
\end{itemize}
\begin{itemize}
\item {Proveniência:(De \textunderscore fôlha\textunderscore )}
\end{itemize}
Pequena fôlha impressa ou livrinho, que contém o calendário.
Calendário.
Directório da reza obrigatória dos padres.
\section{Folho}
\begin{itemize}
\item {fónica:fô}
\end{itemize}
\begin{itemize}
\item {Grp. gram.:m.}
\end{itemize}
\begin{itemize}
\item {Proveniência:(Do lat. \textunderscore folium\textunderscore )}
\end{itemize}
Guarnição de pregas, para vestuário, toalhas de altar, etc.
Excrescência no casco dos animaes.
Terceiro estômago dos ruminantes.
\section{Folhos-de-sinhá}
\begin{itemize}
\item {Grp. gram.:m. pl.}
\end{itemize}
\begin{itemize}
\item {Utilização:Bras}
\end{itemize}
Espécie de doce.
\section{Folhoso}
\begin{itemize}
\item {Grp. gram.:adj.}
\end{itemize}
\begin{itemize}
\item {Grp. gram.:M.}
\end{itemize}
O mesmo que \textunderscore folhudo\textunderscore .
Terceiro estômago dos ruminantes, também conhecido por \textunderscore fôlho\textunderscore .
\section{Folhudo}
\begin{itemize}
\item {Grp. gram.:adj.}
\end{itemize}
Que tem muitas fôlhas; còpado.
\section{Folia}
\begin{itemize}
\item {Grp. gram.:f.}
\end{itemize}
\begin{itemize}
\item {Utilização:Prov.}
\end{itemize}
\begin{itemize}
\item {Utilização:beir.}
\end{itemize}
\begin{itemize}
\item {Proveniência:(Do fr. \textunderscore folie\textunderscore )}
\end{itemize}
Dança rápida, ao som do pandeiro.
O mesmo que \textunderscore fòliá\textunderscore .
\section{Fòliá}
\begin{itemize}
\item {Grp. gram.:f.}
\end{itemize}
\begin{itemize}
\item {Utilização:Gír. do Algarve.}
\end{itemize}
Nome de certos divertimentos, por occasião das festas do Espírito-Santo.
(Alter. de \textunderscore folia\textunderscore )
\section{Foliação}
\begin{itemize}
\item {Grp. gram.:f.}
\end{itemize}
(V.folheação)
\section{Foliáceo}
\begin{itemize}
\item {Grp. gram.:adj.}
\end{itemize}
\begin{itemize}
\item {Proveniência:(Lat. \textunderscore foliaceus\textunderscore )}
\end{itemize}
Relativo a fôlhas.
Semelhante a fôlhas.
Feito de fôlhas.
\section{Foliado}
\begin{itemize}
\item {Grp. gram.:adj.}
\end{itemize}
\begin{itemize}
\item {Proveniência:(Do lat. \textunderscore folium\textunderscore )}
\end{itemize}
Que tem fôlhas.
Foliáceo.
Revestido de laminas de madeira ou metal; folheado: \textunderscore um móvel foliado\textunderscore .
\section{Foliagudo}
\begin{itemize}
\item {Grp. gram.:adj.}
\end{itemize}
\begin{itemize}
\item {Proveniência:(T. hybr., do lat. \textunderscore folium\textunderscore  e do port. \textunderscore agudo\textunderscore )}
\end{itemize}
Que tem fôlhas agudas.
\section{Folião}
\begin{itemize}
\item {Grp. gram.:m.}
\end{itemize}
\begin{itemize}
\item {Proveniência:(De \textunderscore folia\textunderscore )}
\end{itemize}
Histrião.
Farsante.
Indivíduo folgazão.
Peixe dos Açores.
\section{Fòlião}
\begin{itemize}
\item {Grp. gram.:m.}
\end{itemize}
\begin{itemize}
\item {Utilização:Prov.}
\end{itemize}
\begin{itemize}
\item {Utilização:alg.}
\end{itemize}
\begin{itemize}
\item {Proveniência:(De \textunderscore fòliá\textunderscore )}
\end{itemize}
Aquelle que pede esmolas para a fòliá.
Membro da commissão que promove a fòliá.
\section{Foliar}
\begin{itemize}
\item {Grp. gram.:v. t.}
\end{itemize}
Entrar em folias.
Pular; divertir-se.
\section{Foliar}
\begin{itemize}
\item {Grp. gram.:adj.}
\end{itemize}
\begin{itemize}
\item {Proveniência:(Do lat. \textunderscore folium\textunderscore )}
\end{itemize}
Relativo a fôlhas.
\section{Folicular}
\begin{itemize}
\item {Grp. gram.:adj.}
\end{itemize}
\begin{itemize}
\item {Proveniência:(Lat. \textunderscore follicularis\textunderscore )}
\end{itemize}
Relativo a folículo.
\section{Foliculário}
\begin{itemize}
\item {Grp. gram.:m.}
\end{itemize}
\begin{itemize}
\item {Utilização:Deprec.}
\end{itemize}
\begin{itemize}
\item {Proveniência:(Fr. \textunderscore folliculaire\textunderscore )}
\end{itemize}
Escritor de folhetos.
Periodiqueiro.
\section{Folículo}
\begin{itemize}
\item {Grp. gram.:m.}
\end{itemize}
\begin{itemize}
\item {Proveniência:(Lat. \textunderscore folliculus\textunderscore )}
\end{itemize}
Pequeno fole.
Vesícula.
Vagem de uma só sutura longitudinal.
Pequena cavidade nas glândulas, entre as extremidades das artérias, veias e canaes secretores.
\section{Folículo}
\begin{itemize}
\item {Grp. gram.:m.}
\end{itemize}
\begin{itemize}
\item {Proveniência:(Do rad. do lat. \textunderscore folium\textunderscore )}
\end{itemize}
Folhelho.
Folheto.
Pequena fôlha ou lâmina.
Casca.
\section{Foliculoso}
\begin{itemize}
\item {Grp. gram.:adj.}
\end{itemize}
\begin{itemize}
\item {Proveniência:(Lat. \textunderscore folliculosus\textunderscore )}
\end{itemize}
Que tem folículos ou natureza de folículos.
\section{Foliento}
\begin{itemize}
\item {Grp. gram.:adj.}
\end{itemize}
\begin{itemize}
\item {Proveniência:(De \textunderscore folia\textunderscore )}
\end{itemize}
O mesmo que \textunderscore folgazão\textunderscore ^1.
\section{Folífago}
\begin{itemize}
\item {Grp. gram.:adj.}
\end{itemize}
\begin{itemize}
\item {Proveniência:(Do lat. \textunderscore folium\textunderscore  + gr. \textunderscore phagein\textunderscore )}
\end{itemize}
Diz-se dos animaes, que se alimentam de fôlhas ou de substâncias vegetaes.
\section{Folífero}
\begin{itemize}
\item {Grp. gram.:adj.}
\end{itemize}
\begin{itemize}
\item {Proveniência:(Do lat. \textunderscore folium\textunderscore  + \textunderscore ferre\textunderscore )}
\end{itemize}
Que tem ou produz fôlhas.
\section{Foliforme}
\begin{itemize}
\item {Grp. gram.:adj.}
\end{itemize}
\begin{itemize}
\item {Proveniência:(Do lat. \textunderscore folium\textunderscore  + \textunderscore forma\textunderscore )}
\end{itemize}
Que tem fórma de fôlha.
\section{Foliforme}
\begin{itemize}
\item {Grp. gram.:adj.}
\end{itemize}
\begin{itemize}
\item {Proveniência:(Do lat. \textunderscore follis\textunderscore  + \textunderscore forma\textunderscore )}
\end{itemize}
Que tem fórma de fole.
\section{Folilho}
\begin{itemize}
\item {Grp. gram.:m.}
\end{itemize}
\begin{itemize}
\item {Proveniência:(Do lat. \textunderscore folliculus\textunderscore )}
\end{itemize}
Espécie de pericarpo côncavo.
\section{Fólio}
\begin{itemize}
\item {Grp. gram.:m.}
\end{itemize}
\begin{itemize}
\item {Proveniência:(Do lat. \textunderscore folium\textunderscore )}
\end{itemize}
Livro commercial, numerado por fôlhas.
Livro in-fólio.
As duas páginas de uma fôlha.
\section{Fólio-cheiroso}
\begin{itemize}
\item {Grp. gram.:m.}
\end{itemize}
Droga medicinal, que se prepara na China.
\section{Foliolado}
\begin{itemize}
\item {Grp. gram.:adj.}
\end{itemize}
Que tem folíolos.
\section{Folíolo}
\begin{itemize}
\item {Grp. gram.:m.}
\end{itemize}
\begin{itemize}
\item {Grp. gram.:Pl.}
\end{itemize}
\begin{itemize}
\item {Utilização:Bot.}
\end{itemize}
\begin{itemize}
\item {Proveniência:(Lat. \textunderscore foliolum\textunderscore )}
\end{itemize}
Folhinha, ao lado do peciolo.
Sépalas do cálice.
\section{Foliona}
\begin{itemize}
\item {Grp. gram.:f.  e  adj.}
\end{itemize}
\begin{itemize}
\item {Proveniência:(De \textunderscore folião\textunderscore )}
\end{itemize}
Diz-se da mulher que gosta de foliar.
\section{Folioso}
\begin{itemize}
\item {Grp. gram.:adj.}
\end{itemize}
\begin{itemize}
\item {Utilização:Prov.}
\end{itemize}
Em que há folia: \textunderscore um dia folioso\textunderscore .
\section{Folipa}
\begin{itemize}
\item {Grp. gram.:f.}
\end{itemize}
\begin{itemize}
\item {Utilização:Prov.}
\end{itemize}
\begin{itemize}
\item {Utilização:trasm.}
\end{itemize}
Empôla.
Bôlha.
Folheca.
Espécie de folle, que se fórma no vestido mal costurado; folepo.
Folle pequeno.
(Cp. \textunderscore folículo\textunderscore ^1)
\section{Folíparo}
\begin{itemize}
\item {Grp. gram.:adj.}
\end{itemize}
\begin{itemize}
\item {Proveniência:(Do lat. \textunderscore folium\textunderscore  + \textunderscore parere\textunderscore . Cp. \textunderscore follíparo\textunderscore )}
\end{itemize}
Diz-se das plantas que só produzem fôlhas.
\section{Folíphago}
\begin{itemize}
\item {Grp. gram.:adj.}
\end{itemize}
\begin{itemize}
\item {Proveniência:(Do lat. \textunderscore folium\textunderscore  + gr. \textunderscore phagein\textunderscore )}
\end{itemize}
Diz-se dos animaes, que se alimentam de fôlhas ou de substâncias vegetaes.
\section{Folipo}
\begin{itemize}
\item {Grp. gram.:m.}
\end{itemize}
\begin{itemize}
\item {Utilização:Prov.}
\end{itemize}
\begin{itemize}
\item {Utilização:trasm.}
\end{itemize}
Empôla.
Bôlha.
Folheca.
Espécie de fole, que se fórma no vestido mal costurado; folepo.
Fole pequeno.
(Cp. \textunderscore folículo\textunderscore ^1)
\section{Fólis}
\begin{itemize}
\item {Grp. gram.:m.}
\end{itemize}
\begin{itemize}
\item {Proveniência:(Lat. \textunderscore follis\textunderscore )}
\end{itemize}
Moéda de cobre, do pêso de uma onça, entre os antigos Romanos. Cf. Castilho, \textunderscore Fastos\textunderscore , I, 374.
\section{Folla}
\begin{itemize}
\item {fónica:fô}
\end{itemize}
\begin{itemize}
\item {Grp. gram.:f.}
\end{itemize}
\begin{itemize}
\item {Proveniência:(It. \textunderscore folla\textunderscore )}
\end{itemize}
Marulho de ondas.
\section{Follacho}
\begin{itemize}
\item {Grp. gram.:m.  e  f.}
\end{itemize}
\begin{itemize}
\item {Utilização:Pop.}
\end{itemize}
\begin{itemize}
\item {Proveniência:(De \textunderscore folle\textunderscore )}
\end{itemize}
Pessôa branda, doente, fraca.
\section{Folle}
\begin{itemize}
\item {Grp. gram.:m.}
\end{itemize}
\begin{itemize}
\item {Utilização:Fam.}
\end{itemize}
\begin{itemize}
\item {Grp. gram.:Pl.}
\end{itemize}
\begin{itemize}
\item {Grp. gram.:Adj.}
\end{itemize}
\begin{itemize}
\item {Utilização:Prov.}
\end{itemize}
\begin{itemize}
\item {Utilização:beir.}
\end{itemize}
\begin{itemize}
\item {Proveniência:(Lat. \textunderscore follis\textunderscore )}
\end{itemize}
Utensílio, que, pôsto em movimento, produz vento, para activar uma combustão ou para limpar certas cavidades.
Taleiga de coiro.
Papo, estômago: \textunderscore já enchi o folle\textunderscore .
Passadeira de coiro, nos arreios dos muares de diligências.
Parte de um instrumento músico, vulgarmente conhecido por \textunderscore gaita de folles\textunderscore  ou \textunderscore gaita gallega\textunderscore .
Diz-se do fruto molle ou sorvado: \textunderscore um pero folle\textunderscore .
\section{Follecha}
\begin{itemize}
\item {fónica:lê}
\end{itemize}
\begin{itemize}
\item {Grp. gram.:f.}
\end{itemize}
\begin{itemize}
\item {Utilização:Prov.}
\end{itemize}
\begin{itemize}
\item {Utilização:beir.}
\end{itemize}
Espécie de passarinho cinzento.
O mesmo que \textunderscore folosa\textunderscore ?
Empôla, o mesmo que \textunderscore follecho\textunderscore .
\section{Follecho}
\begin{itemize}
\item {fónica:lê}
\end{itemize}
\begin{itemize}
\item {Grp. gram.:m.}
\end{itemize}
\begin{itemize}
\item {Utilização:Pop.}
\end{itemize}
\begin{itemize}
\item {Proveniência:(De \textunderscore folle\textunderscore )}
\end{itemize}
Empôla.
Bôlha na pelle, contendo aguadilha.
\section{Folle-das-migas}
\begin{itemize}
\item {Grp. gram.:m.}
\end{itemize}
\begin{itemize}
\item {Utilização:Gír.}
\end{itemize}
Barriga.
\section{Follega}
\begin{itemize}
\item {Grp. gram.:f.}
\end{itemize}
\begin{itemize}
\item {Utilização:Prov.}
\end{itemize}
\begin{itemize}
\item {Utilização:minh.}
\end{itemize}
Sova, tareia, tunda.
\section{Folleiro}
\begin{itemize}
\item {Grp. gram.:m.}
\end{itemize}
\begin{itemize}
\item {Utilização:Prov.}
\end{itemize}
Fabricante, vendedor ou tocador de folles.
Burro de moleiro.
\section{Follepo}
\begin{itemize}
\item {fónica:lê}
\end{itemize}
\begin{itemize}
\item {Grp. gram.:m.}
\end{itemize}
O mesmo que \textunderscore follecho\textunderscore .
Espécie de folle, que se fórma no fato mal costurado; follipo.
\section{Follicular}
\begin{itemize}
\item {Grp. gram.:adj.}
\end{itemize}
\begin{itemize}
\item {Proveniência:(Lat. \textunderscore follicularis\textunderscore )}
\end{itemize}
Relativo a follículo.
\section{Follículo}
\begin{itemize}
\item {Grp. gram.:m.}
\end{itemize}
\begin{itemize}
\item {Proveniência:(Lat. \textunderscore folliculus\textunderscore )}
\end{itemize}
Pequeno folle.
Vesícula.
Vagem de uma só sutura longitudinal.
Pequena cavidade nas glândulas, entre as extremidades das artérias, veias e canaes secretores.
\section{Folliculoso}
\begin{itemize}
\item {Grp. gram.:adj.}
\end{itemize}
\begin{itemize}
\item {Proveniência:(Lat. \textunderscore folliculosus\textunderscore )}
\end{itemize}
Que tem follículos ou natureza de follículos.
\section{Follífero}
\begin{itemize}
\item {Grp. gram.:adj.}
\end{itemize}
\begin{itemize}
\item {Proveniência:(Do lat. \textunderscore flos\textunderscore  + \textunderscore ferre\textunderscore )}
\end{itemize}
Diz-se do botão, que só produz fôlhas.
\section{Folliforme}
\begin{itemize}
\item {Grp. gram.:adj.}
\end{itemize}
\begin{itemize}
\item {Proveniência:(Do lat. \textunderscore follis\textunderscore  + \textunderscore forma\textunderscore )}
\end{itemize}
Que tem fórma de folle.
\section{Follilho}
\begin{itemize}
\item {Grp. gram.:m.}
\end{itemize}
\begin{itemize}
\item {Proveniência:(Do lat. \textunderscore folliculus\textunderscore )}
\end{itemize}
Espécie de pericarpo côncavo.
\section{Follipa}
\begin{itemize}
\item {Grp. gram.:f.}
\end{itemize}
\begin{itemize}
\item {Utilização:Prov.}
\end{itemize}
\begin{itemize}
\item {Utilização:trasm.}
\end{itemize}
Empôla.
Bôlha.
Folheca.
Espécie de folle, que se fórma no vestido mal costurado; follepo.
Folle pequeno.
(Cp. \textunderscore follículo\textunderscore )
\section{Follíparo}
\begin{itemize}
\item {Grp. gram.:adj.}
\end{itemize}
\begin{itemize}
\item {Proveniência:(Do lat. \textunderscore folium\textunderscore  + \textunderscore parere\textunderscore )}
\end{itemize}
O mesmo que \textunderscore follífero\textunderscore .
\section{Follipo}
\begin{itemize}
\item {Grp. gram.:m.}
\end{itemize}
\begin{itemize}
\item {Utilização:Prov.}
\end{itemize}
\begin{itemize}
\item {Utilização:trasm.}
\end{itemize}
Empôla.
Bôlha.
Folheca.
Espécie de folle, que se fórma no vestido mal costurado; follepo.
Folle pequeno.
(Cp. \textunderscore follículo\textunderscore )
\section{Fóllis}
\begin{itemize}
\item {Grp. gram.:m.}
\end{itemize}
\begin{itemize}
\item {Proveniência:(Lat. \textunderscore follis\textunderscore )}
\end{itemize}
Moéda de cobre, do pêso de uma onça, entre os antigos Romanos. Cf. Castilho, \textunderscore Fastos\textunderscore , I, 374.
\section{Folosa}
\begin{itemize}
\item {Grp. gram.:f.}
\end{itemize}
\begin{itemize}
\item {Utilização:T. de Turquel}
\end{itemize}
Pequeno pássaro dentirostro.
Rapariga fraca, magra, sem côr e sem animação.
(Por \textunderscore follosa\textunderscore , de \textunderscore folle\textunderscore ? Fernão Lopes escreveu \textunderscore follosa\textunderscore )
\section{Folosinho}
\begin{itemize}
\item {Grp. gram.:m.}
\end{itemize}
Variedade de uva preta do Minho.
(Talvez corr. de \textunderscore folgosinho\textunderscore )
\section{Fome}
\begin{itemize}
\item {Grp. gram.:f.}
\end{itemize}
\begin{itemize}
\item {Proveniência:(Do lat. \textunderscore fames\textunderscore )}
\end{itemize}
Grande appetite de comer.
Urgência de alimento.
Miséria.
Falta, escassez.
Soffreguidão.
\section{Fòmenica}
\begin{itemize}
\item {Grp. gram.:m.}
\end{itemize}
\begin{itemize}
\item {Utilização:Pop.}
\end{itemize}
\begin{itemize}
\item {Utilização:T. da Bairrada}
\end{itemize}
Indivíduo avarento, sovina.
Indivíduo, que come muito pouco, que é biqueiro.
\section{Fomentação}
\begin{itemize}
\item {Grp. gram.:f.}
\end{itemize}
\begin{itemize}
\item {Proveniência:(Lat. \textunderscore fomentatio\textunderscore )}
\end{itemize}
Acto ou effeito de fomentar.
Fricção medicamentosa da pelle.
\section{Fomentador}
\begin{itemize}
\item {Grp. gram.:adj.}
\end{itemize}
\begin{itemize}
\item {Grp. gram.:M.}
\end{itemize}
Que fomenta.
Aquelle que fomenta, promove ou causa.
\section{Fomentar}
\begin{itemize}
\item {Grp. gram.:v. t.}
\end{itemize}
\begin{itemize}
\item {Proveniência:(Lat. \textunderscore fomentare\textunderscore )}
\end{itemize}
Excitar, facilitar.
Promover o desenvolvimento de: \textunderscore fomentar as indústrias\textunderscore .
Estimular.
Esfregar (a pelle), applicando um medicamento líquido.
\section{Fomentativo}
\begin{itemize}
\item {Grp. gram.:adj.}
\end{itemize}
Que fomenta.
\section{Fomentista}
\begin{itemize}
\item {Grp. gram.:m.}
\end{itemize}
Aquelle que fomenta.
Promotor do progresso material. Cf. Camillo, \textunderscore Maria da Fonte\textunderscore , 103.
\section{Fomento}
\begin{itemize}
\item {Grp. gram.:m.}
\end{itemize}
\begin{itemize}
\item {Utilização:Fig.}
\end{itemize}
\begin{itemize}
\item {Proveniência:(Lat. \textunderscore fomentum\textunderscore )}
\end{itemize}
Acto de fomentar.
Medicamento, que se applica na pelle, friccionando-a.
Allívio.
Refrigério.
Protecção, auxílio.
\section{Fomes}
\begin{itemize}
\item {Grp. gram.:m.}
\end{itemize}
\begin{itemize}
\item {Utilização:Ant.}
\end{itemize}
\begin{itemize}
\item {Proveniência:(Lat. \textunderscore fomes\textunderscore )}
\end{itemize}
Concupiscência.
Grande appetite.
Aquillo que estimula.
Incentivo.
\section{Fomo}
\begin{itemize}
\item {Grp. gram.:m.}
\end{itemize}
(Palavra erradamente introduzida em diccionários modernos, como termo brasileiro que significa bacia em que se séca ao fogo a massa da mandioca. O termo é \textunderscore forno\textunderscore )
\section{Fona}
\begin{itemize}
\item {Grp. gram.:f.}
\end{itemize}
\begin{itemize}
\item {Utilização:Bras}
\end{itemize}
\begin{itemize}
\item {Utilização:Pop.}
\end{itemize}
\begin{itemize}
\item {Grp. gram.:M.}
\end{itemize}
\begin{itemize}
\item {Utilização:T. da Bairrada}
\end{itemize}
\begin{itemize}
\item {Grp. gram.:M.  e  f.}
\end{itemize}
Centelha, que se extingue no ar; faúlha.
Prisma de madeira, com que se joga, atirando-o ao ar, e indicando a sua face superior, depois da quéda, se o jogador perdeu ou ganhou.
Azáfama, lufa-lufa: \textunderscore andar numa fona\textunderscore .
Aquelle que, em certo jôgo de rapazes, é o último a jogar.
Aquelle que vai no coice de um grupo de gente.
Indivíduo effeminado, mulherengo; fraco.
Pessôa avarenta.
Sovina.
\section{Fonção}
\begin{itemize}
\item {Grp. gram.:f.}
\end{itemize}
\begin{itemize}
\item {Utilização:Bras. do N}
\end{itemize}
O mesmo que \textunderscore funcção\textunderscore .
Divertimento popular, geralmente com danças e descantes.
\section{Foneca}
\begin{itemize}
\item {Grp. gram.:f.}
\end{itemize}
\begin{itemize}
\item {Utilização:T. do Fundão}
\end{itemize}
Castanha choca, faneca.
\section{Fonice}
\begin{itemize}
\item {Grp. gram.:f.}
\end{itemize}
\begin{itemize}
\item {Proveniência:(De \textunderscore fona\textunderscore )}
\end{itemize}
Avareza, somiticaria.
\section{Fonisca}
\begin{itemize}
\item {Grp. gram.:m.}
\end{itemize}
\begin{itemize}
\item {Utilização:Prov.}
\end{itemize}
\begin{itemize}
\item {Utilização:trasm.}
\end{itemize}
\begin{itemize}
\item {Proveniência:(De \textunderscore fona\textunderscore )}
\end{itemize}
O mesmo que \textunderscore faúlha\textunderscore , ou faúlha muito pequena.
\section{Fonjo}
\begin{itemize}
\item {Grp. gram.:adj.}
\end{itemize}
\begin{itemize}
\item {Utilização:Prov.}
\end{itemize}
\begin{itemize}
\item {Utilização:trasm.}
\end{itemize}
Diz-se do pano ou tecido fraco ou pouco consistente, em opposição ao encorpado e forte.
\section{Fontaínha}
\begin{itemize}
\item {Grp. gram.:f.}
\end{itemize}
Pequena fonte.
(Contr. de \textunderscore fontaninha\textunderscore , de \textunderscore fontana\textunderscore )
\section{Fontal}
\begin{itemize}
\item {Grp. gram.:adj.}
\end{itemize}
Relativo a fonte; originário.
\section{Fontana}
\begin{itemize}
\item {Grp. gram.:f.}
\end{itemize}
\begin{itemize}
\item {Utilização:Ant.}
\end{itemize}
\begin{itemize}
\item {Proveniência:(T. cast.)}
\end{itemize}
O mesmo que \textunderscore fonte\textunderscore . Cf. \textunderscore Cancioneiro da Vaticana\textunderscore .
\section{Fontanal}
\begin{itemize}
\item {Grp. gram.:adj.}
\end{itemize}
\begin{itemize}
\item {Proveniência:(Lat. \textunderscore fontanalis\textunderscore )}
\end{itemize}
O mesmo que \textunderscore fontal\textunderscore .
\section{Fontanário}
\begin{itemize}
\item {Grp. gram.:adj.}
\end{itemize}
\begin{itemize}
\item {Proveniência:(De \textunderscore fontana\textunderscore )}
\end{itemize}
O mesmo que \textunderscore fontal\textunderscore .
\section{Fontanela}
\begin{itemize}
\item {Grp. gram.:f.}
\end{itemize}
\begin{itemize}
\item {Proveniência:(De \textunderscore fontana\textunderscore )}
\end{itemize}
Parte membranosa do crânio das crianças.
Fonte aberta por operação cirúrgica, como revulsivo.
Fontículo.
\section{Fontanésia}
\begin{itemize}
\item {Grp. gram.:f.}
\end{itemize}
\begin{itemize}
\item {Proveniência:(De \textunderscore Desfontaines\textunderscore , n. p.)}
\end{itemize}
Gênero de plantas oleáceas.
\section{Fontange}
\begin{itemize}
\item {Grp. gram.:m.}
\end{itemize}
\begin{itemize}
\item {Utilização:Ant.}
\end{itemize}
\begin{itemize}
\item {Proveniência:(Fr. \textunderscore fontange\textunderscore , do nome de uma amante de Luis XIV)}
\end{itemize}
Jóia de pedraria.
Laço de fitas no toucado.
\section{Fontano}
\begin{itemize}
\item {Grp. gram.:adj.}
\end{itemize}
\begin{itemize}
\item {Proveniência:(Lat. \textunderscore fontanus\textunderscore )}
\end{itemize}
Relativo a fonte.
\section{Fonte}
\begin{itemize}
\item {Grp. gram.:f.}
\end{itemize}
\begin{itemize}
\item {Utilização:Fig.}
\end{itemize}
\begin{itemize}
\item {Grp. gram.:Pl.}
\end{itemize}
\begin{itemize}
\item {Proveniência:(Lat. \textunderscore fons\textunderscore , \textunderscore fontis\textunderscore )}
\end{itemize}
Lugar, onde nasce água perennemente.
Água nascente, água que irrompe perennemente do solo.
Chafariz.
Bica artificial, por onde corre e donde se recebe a água destinada ao consumo doméstico.
Fontanela, sedenho.
Origem; princípio.
Causa.
Texto originário de uma obra.
Lados da cabeça, que formam a região temporal de cada lado.
\section{Fonteca}
\begin{itemize}
\item {Grp. gram.:f.}
\end{itemize}
\begin{itemize}
\item {Utilização:T. de Villa-Viçosa}
\end{itemize}
Pequena fonte.
\section{Fonte-cal}
\begin{itemize}
\item {Grp. gram.:f.}
\end{itemize}
O mesmo que \textunderscore fonte-canal\textunderscore .
\section{Fonte-canal}
\begin{itemize}
\item {Grp. gram.:f.}
\end{itemize}
Variedade de uva.
\section{Fonteira}
\begin{itemize}
\item {Grp. gram.:f.}
\end{itemize}
\begin{itemize}
\item {Utilização:Prov.}
\end{itemize}
\begin{itemize}
\item {Utilização:beir.}
\end{itemize}
\begin{itemize}
\item {Proveniência:(De \textunderscore fonte\textunderscore )}
\end{itemize}
Mulher, que fornece água da fonte para os domicílios.
\section{Fontela}
\begin{itemize}
\item {Grp. gram.:f.}
\end{itemize}
\begin{itemize}
\item {Grp. gram.:Pl.}
\end{itemize}
Pequena fonte ou nascente.
Poros, ou imperceptíveis orifícios, por onde as vasilhas de barro deixam passar água. Cf. \textunderscore Portugalia\textunderscore , II, 76.
\section{Fontenário}
\begin{itemize}
\item {Grp. gram.:adj.}
\end{itemize}
(V.fontanário)
\section{Fontícola}
\begin{itemize}
\item {Grp. gram.:adj.}
\end{itemize}
\begin{itemize}
\item {Proveniência:(Do lat. \textunderscore fons\textunderscore  + \textunderscore colere\textunderscore )}
\end{itemize}
Que vive ou cresce nas fontes ou junto dellas.
\section{Fontículo}
\begin{itemize}
\item {Grp. gram.:m.}
\end{itemize}
\begin{itemize}
\item {Proveniência:(Lat. \textunderscore fonticulus\textunderscore )}
\end{itemize}
Pequena fonte; fontanela.
\section{Fontinal}
\begin{itemize}
\item {Grp. gram.:adj.}
\end{itemize}
\begin{itemize}
\item {Proveniência:(Lat. \textunderscore fontinalis\textunderscore )}
\end{itemize}
O mesmo que \textunderscore fontanal\textunderscore .
\section{Fonzadar}
\begin{itemize}
\item {Grp. gram.:m.}
\end{itemize}
Cobrador de impostos, em algumas aldeias da Índia portuguesa.
\section{Fopa}
\begin{itemize}
\item {Grp. gram.:f.}
\end{itemize}
\begin{itemize}
\item {Utilização:Prov.}
\end{itemize}
\begin{itemize}
\item {Utilização:trasm.}
\end{itemize}
\begin{itemize}
\item {Utilização:Prov.}
\end{itemize}
\begin{itemize}
\item {Utilização:beir.}
\end{itemize}
Faúlha, que se levanta da cinza.
Froco volante de cinza.
Centelha.
\section{Fôr}
\begin{itemize}
\item {Grp. gram.:m.}
\end{itemize}
\begin{itemize}
\item {Utilização:Ant.}
\end{itemize}
\begin{itemize}
\item {Grp. gram.:Loc. adv.}
\end{itemize}
O mesmo que fôro.
Costume; fórma. Cf. G. Vicente.
\textunderscore A fôr\textunderscore , á moda, segundo o costume.
\section{Fóra}
\begin{itemize}
\item {Grp. gram.:adv.}
\end{itemize}
\begin{itemize}
\item {Grp. gram.:Prep.}
\end{itemize}
\begin{itemize}
\item {Grp. gram.:Interj.}
\end{itemize}
\begin{itemize}
\item {Grp. gram.:M. pl.}
\end{itemize}
\begin{itemize}
\item {Utilização:Des.}
\end{itemize}
\begin{itemize}
\item {Grp. gram.:Loc.}
\end{itemize}
\begin{itemize}
\item {Utilização:pop.}
\end{itemize}
\begin{itemize}
\item {Proveniência:(Do lat. \textunderscore foras\textunderscore )}
\end{itemize}
Exteriormente, na face externa: \textunderscore esta casa, por fóra, não é feia\textunderscore .
Em país estranho: \textunderscore lá fóra, estuda-se mais\textunderscore .
Para distância, para longe: \textunderscore deitar fóra\textunderscore .
Em lugar diverso do da residência habitual.
Excepto, com exclusão: \textunderscore sairam todos de casa, fóra os criados\textunderscore .
Além de.
Longe de.
Afastado de.
Alheiamente.
Arreda.
Sáia para fóra.
A parte exterior.
\textunderscore Estar\textunderscore  ou \textunderscore andar fóra dos eixos\textunderscore , não têr tino.
Andar adoentado.
\section{Foradiço}
\begin{itemize}
\item {Grp. gram.:adj.}
\end{itemize}
\begin{itemize}
\item {Utilização:Ant.}
\end{itemize}
Dizia-se da água de presa, com que se regam terras que della precisam.
(Provavelmente do lat. \textunderscore forare\textunderscore , por allusão ao furo por onde sai a água da presa)
\section{Foragem}
\begin{itemize}
\item {Grp. gram.:f.}
\end{itemize}
Pequeno foro.
\section{Foragido}
\begin{itemize}
\item {Grp. gram.:adj.}
\end{itemize}
\begin{itemize}
\item {Proveniência:(De \textunderscore foragir-se\textunderscore )}
\end{itemize}
Que anda fóra da sua terra.
Emigrado.
Escondido e errante, para escapar á justiça.
Perseguido.
\section{Foragir-se}
\begin{itemize}
\item {Grp. gram.:v. p.}
\end{itemize}
\begin{itemize}
\item {Utilização:bras}
\end{itemize}
\begin{itemize}
\item {Utilização:Neol.}
\end{itemize}
\begin{itemize}
\item {Proveniência:(Do rad. do lat. \textunderscore foras\textunderscore )}
\end{itemize}
Expatriar-se; emigrar.
Homisiar-se.
\section{Foral}
\begin{itemize}
\item {Grp. gram.:m.}
\end{itemize}
\begin{itemize}
\item {Proveniência:(De \textunderscore fôro\textunderscore )}
\end{itemize}
Carta de lei, que regulava a administração de uma localidade, ou que concedia previlégios a indivíduos ou corporações.
Título de aforamento de terras.
Antigo regulamento de Repartições públicas.
\section{Foraleiro}
\begin{itemize}
\item {Grp. gram.:adj.}
\end{itemize}
Relativo a foral.
\section{Forame}
\begin{itemize}
\item {Grp. gram.:m.}
\end{itemize}
\begin{itemize}
\item {Proveniência:(Lat. \textunderscore foramen\textunderscore )}
\end{itemize}
Abertura, cova.
O mesmo que \textunderscore micrópylo\textunderscore .
\section{Forâmen}
\begin{itemize}
\item {Grp. gram.:m.}
\end{itemize}
O mesmo que \textunderscore forame\textunderscore .
\section{Foraminíferos}
\begin{itemize}
\item {Grp. gram.:m. pl.}
\end{itemize}
\begin{itemize}
\item {Proveniência:(Do lat. \textunderscore foramen\textunderscore  + \textunderscore ferre\textunderscore )}
\end{itemize}
Classe de infusórios.
\section{Foraminoso}
\begin{itemize}
\item {Grp. gram.:adj.}
\end{itemize}
\begin{itemize}
\item {Proveniência:(Lat. \textunderscore foraminosus\textunderscore )}
\end{itemize}
Que tem forames.
\section{Foramontão}
\begin{itemize}
\item {Grp. gram.:adj.}
\end{itemize}
\begin{itemize}
\item {Grp. gram.:M.}
\end{itemize}
\begin{itemize}
\item {Proveniência:(De \textunderscore foro\textunderscore  + \textunderscore monte\textunderscore )}
\end{itemize}
Dizia-se do lugar ou povoado, que pagava foro da montaria.
Casa ou lugar, que pagava o foro imposto sôbre as casas de prostituição.
\section{Forâneo}
\begin{itemize}
\item {Grp. gram.:adj.}
\end{itemize}
\begin{itemize}
\item {Proveniência:(Do rad. de \textunderscore fóra\textunderscore )}
\end{itemize}
Estranho, que é de fóra.
\section{Foraria}
\begin{itemize}
\item {Grp. gram.:f.}
\end{itemize}
\begin{itemize}
\item {Utilização:Ant.}
\end{itemize}
O mesmo que \textunderscore foragem\textunderscore .
\section{Forasteiro}
\begin{itemize}
\item {Grp. gram.:m.  e  adj.}
\end{itemize}
\begin{itemize}
\item {Proveniência:(It. \textunderscore forestiere\textunderscore )}
\end{itemize}
Estrangeiro: \textunderscore encontram-se muitos forasteiros em Lisbôa\textunderscore .
Peregrino.
Alheio.
\section{Forata}
\begin{itemize}
\item {Grp. gram.:f.}
\end{itemize}
\begin{itemize}
\item {Proveniência:(It. \textunderscore forata\textunderscore )}
\end{itemize}
Apparelho moderno, formado de uma caixa circular e de cylindros concêntricos, e destinado a substituir as seiras na espremedura do bagaço da azeitona.
\section{Fôrca}
\begin{itemize}
\item {Grp. gram.:f.}
\end{itemize}
\begin{itemize}
\item {Utilização:Fig.}
\end{itemize}
\begin{itemize}
\item {Proveniência:(Do lat. \textunderscore furca\textunderscore )}
\end{itemize}
Apparelho, formado geralmente por três espeques e corda, e que servia para supplício de estrangulação; patíbulo.
Madeiro ou objecto, de que alguém se pendurou, estrangulando-se.
Corda, com que se enforca alguém.
Forquilha.
Laço, cilada.
\section{Fôrça}
\begin{itemize}
\item {Grp. gram.:f.}
\end{itemize}
\begin{itemize}
\item {Grp. gram.:Loc. adv.}
\end{itemize}
\begin{itemize}
\item {Grp. gram.:Loc. adv.}
\end{itemize}
\begin{itemize}
\item {Proveniência:(Do b. lat. \textunderscore fortia\textunderscore )}
\end{itemize}
Faculdade de operar, de mover ou de mover-se.
Poder.
Energia; vigor.
Robustez.
Valentia: \textunderscore homem de muita fôrça\textunderscore .
Motivo, causa: \textunderscore a fôrça das circumstâncias é que o levou ao crime\textunderscore .
Necessidade.
Grande porção, abundância.
A parte principal de um conjunto.
Trôço de militares, destacamento.
Resistência. Cf. R. Lobo, \textunderscore Côrte na Ald.\textunderscore , II, 87.
\textunderscore Á fôrça\textunderscore , com violência.
\textunderscore Por fôrça\textunderscore , forçosamente, necessariamente.
\textunderscore É fôrça que\textunderscore , é necessárario que.
\textunderscore Desta fôrça\textunderscore , desta grandeza, dêste tamanho.
\section{Forcacha}
\begin{itemize}
\item {Grp. gram.:f.}
\end{itemize}
\begin{itemize}
\item {Utilização:Pop.}
\end{itemize}
\begin{itemize}
\item {Proveniência:(De \textunderscore fôrca\textunderscore )}
\end{itemize}
Hastes de madeira, que formam ângulo.
Ramalho bifurcado, que se colloca no pescoço da bêsta, ficando uma haste de cada lado, para evitar a mordedura das môscas.
\section{Forcada}
\begin{itemize}
\item {Grp. gram.:f.}
\end{itemize}
\begin{itemize}
\item {Utilização:Prov.}
\end{itemize}
\begin{itemize}
\item {Utilização:Prov.}
\end{itemize}
\begin{itemize}
\item {Utilização:trasm.}
\end{itemize}
\begin{itemize}
\item {Proveniência:(De \textunderscore fôrca\textunderscore )}
\end{itemize}
Ponto de bifurcação.
Parte do tronco humano, em que se bifurcam as pernas.
Instrumento de lavoira, o mesmo que \textunderscore forcado\textunderscore . (Colhido em Valpaços)
\section{Forçadamente}
\begin{itemize}
\item {Grp. gram.:adv.}
\end{itemize}
De modo forçado.
Violentamente.
\section{Forcadela}
\begin{itemize}
\item {Grp. gram.:f.}
\end{itemize}
\begin{itemize}
\item {Proveniência:(De \textunderscore Forcadela\textunderscore , n. p.)}
\end{itemize}
Pequeno barco, usado no alto Minho.
\section{Forcado}
\begin{itemize}
\item {Grp. gram.:m.}
\end{itemize}
\begin{itemize}
\item {Grp. gram.:Adj.}
\end{itemize}
\begin{itemize}
\item {Utilização:Ant.}
\end{itemize}
\begin{itemize}
\item {Proveniência:(Do b. lat. \textunderscore furcatus\textunderscore )}
\end{itemize}
Utensílio de lavoira, formado de uma haste de pau, terminada em duas ou três pontas do mesmo pau ou de ferro.
Quantidade de palha, de estrume ou de erva, que um forcado levanta de uma vez.
Tejolo largo e delgado.
\textunderscore Moço de forcado\textunderscore , aquelle que nos círcos agarra toiros e os conduz, depois de corridos, com as chocas.
Dizia-se do pé fendido, como o de alguns animaes. Cf. \textunderscore Port. Mon. Hist.\textunderscore , \textunderscore Script.\textunderscore , 259.
\section{Forçado}
\begin{itemize}
\item {Grp. gram.:m.}
\end{itemize}
\begin{itemize}
\item {Proveniência:(De \textunderscore forçar\textunderscore )}
\end{itemize}
Grilheta; o condemnado a trabalhos públicos.
\section{Forçador}
\begin{itemize}
\item {Grp. gram.:m.}
\end{itemize}
Aquelle que força.
\section{Forcadura}
\begin{itemize}
\item {Grp. gram.:f.}
\end{itemize}
\begin{itemize}
\item {Proveniência:(De \textunderscore forcado\textunderscore )}
\end{itemize}
Espaço entre as pontas do forcado.
Ornato de palmas, encruzado ou em fórma de forcado.
\section{Forcalha}
\begin{itemize}
\item {Grp. gram.:f.}
\end{itemize}
\begin{itemize}
\item {Utilização:Prov.}
\end{itemize}
\begin{itemize}
\item {Utilização:minh.}
\end{itemize}
\begin{itemize}
\item {Utilização:Pop.}
\end{itemize}
\begin{itemize}
\item {Proveniência:(De \textunderscore fôrca\textunderscore )}
\end{itemize}
Parte da cabeçalha, onde entra o jugo.
Pau, com uma das extremidades aberta propositadamente, para formar uma espécie de forcado.
\section{Forçamento}
\begin{itemize}
\item {Grp. gram.:m.}
\end{itemize}
Acto de forçar.
Acto de violentar uma mulher.
\section{Forcanha}
\begin{itemize}
\item {Grp. gram.:f.}
\end{itemize}
\begin{itemize}
\item {Utilização:Prov.}
\end{itemize}
Galho de árvore, que fórma fôrca ou ângulo.
\section{Forçante}
\begin{itemize}
\item {Grp. gram.:adj.}
\end{itemize}
Que fórça.
\section{Forcão}
\begin{itemize}
\item {Grp. gram.:m.}
\end{itemize}
\begin{itemize}
\item {Utilização:Prov.}
\end{itemize}
\begin{itemize}
\item {Proveniência:(De \textunderscore fôrca\textunderscore )}
\end{itemize}
O mesmo que \textunderscore forcado\textunderscore .
\section{Forcar}
\begin{itemize}
\item {Grp. gram.:v. t.}
\end{itemize}
\begin{itemize}
\item {Utilização:Ant.}
\end{itemize}
\begin{itemize}
\item {Proveniência:(De \textunderscore fôrca\textunderscore )}
\end{itemize}
Revolver com forcado.
O mesmo que \textunderscore enforcar\textunderscore . Cf. \textunderscore Eufrosina\textunderscore , 93.
\section{Forçar}
\begin{itemize}
\item {Grp. gram.:v. t.}
\end{itemize}
\begin{itemize}
\item {Proveniência:(Do b. lat. \textunderscore fortiare\textunderscore )}
\end{itemize}
Sujeitar pela fôrça.
Obrigar, constranger.
Violentar.
Conseguir violentamente, á fôrça.
Desviar de uma posição, de um rumo, de um plano.
Desvirtuar, interpretar mal: \textunderscore forçar o sentido de uma lei\textunderscore .
Estuprar.
Romper, desbaratar.
Entrar á fôrça em, arrombar: \textunderscore forçar uma porta\textunderscore .
\section{Forcarete}
\begin{itemize}
\item {fónica:carê}
\end{itemize}
\begin{itemize}
\item {Grp. gram.:m.}
\end{itemize}
\begin{itemize}
\item {Utilização:Ant.}
\end{itemize}
Parece que era uma espécie de cobertor, pelo que se póde inferír das \textunderscore Provas da Hist. Geneal.\textunderscore , vol. II, na descripção do enxoval de D. Beatriz.
\section{Forcaz}
\begin{itemize}
\item {Grp. gram.:m.}
\end{itemize}
\begin{itemize}
\item {Utilização:Prov.}
\end{itemize}
\begin{itemize}
\item {Utilização:alent.}
\end{itemize}
\begin{itemize}
\item {Proveniência:(De \textunderscore fôrca\textunderscore )}
\end{itemize}
Peça da charrua, aberta adeante em fórma de fôrca e na qual entra uma cavilha chamada rebate.
\section{Forcejar}
\begin{itemize}
\item {Grp. gram.:v. i.}
\end{itemize}
Empregar fôrça.
Fazer diligência.
Esforçar-se; lutar.
\section{Forcejo}
\begin{itemize}
\item {Grp. gram.:m.}
\end{itemize}
Acto ou effeito de forcejar.
\section{Fórceps}
\begin{itemize}
\item {Grp. gram.:m.}
\end{itemize}
\begin{itemize}
\item {Proveniência:(T. lat. A fórma portuguesa seria \textunderscore fórcipe\textunderscore )}
\end{itemize}
Tenaz ou pinça cirúrgica, para extrahir de um corpo corpos estranhos.
Instrumento, com que se extrái do útero a criança.
\section{Forçosamente}
\begin{itemize}
\item {Grp. gram.:adv.}
\end{itemize}
\begin{itemize}
\item {Proveniência:(De \textunderscore forçoso\textunderscore )}
\end{itemize}
Á fôrça.
Necessariamente; fatalmente.
\section{Forçoso}
\begin{itemize}
\item {Grp. gram.:adj.}
\end{itemize}
Que tem fôrça, vigor.
Violento.
Necessário, inevitável:«\textunderscore é forçoso deixar-te e partir\textunderscore ». S. de Passos, \textunderscore Poesias\textunderscore .
\section{Forçura}
\begin{itemize}
\item {Grp. gram.:f.}
\end{itemize}
\begin{itemize}
\item {Proveniência:(De \textunderscore fôrça\textunderscore )}
\end{itemize}
Escora, esteio.
Pequeno camarote, pouco acima do nível do chão, e cujos lados como que sustentam o camarote superior.
\section{Fordicídio}
\begin{itemize}
\item {Grp. gram.:m.}
\end{itemize}
\begin{itemize}
\item {Proveniência:(Lat. \textunderscore fordicidia\textunderscore )}
\end{itemize}
Sacrifício de uma vaca prenhe, o qual se fazia entre os Romanos a 20 de Abril.
\section{Fordo}
\begin{itemize}
\item {fónica:fôr}
\end{itemize}
\begin{itemize}
\item {Grp. gram.:adj.}
\end{itemize}
\begin{itemize}
\item {Proveniência:(Lat. \textunderscore fordus\textunderscore )}
\end{itemize}
Pejado, prenhe:«\textunderscore litae co'a fôrda vacca os sacrifícios, ó pontífices.\textunderscore »Castilho, \textunderscore Fastos\textunderscore , II, 175.
\section{Foreca}
\begin{itemize}
\item {Grp. gram.:f.}
\end{itemize}
\begin{itemize}
\item {Utilização:Ant.}
\end{itemize}
Livro de lembranças.
\section{Foreira}
\begin{itemize}
\item {Grp. gram.:f.}
\end{itemize}
Mulher, que paga fôro.
Mulher do foreiro.
\section{Foreiro}
\begin{itemize}
\item {Grp. gram.:m.}
\end{itemize}
\begin{itemize}
\item {Grp. gram.:Adj.}
\end{itemize}
\begin{itemize}
\item {Utilização:Ant.}
\end{itemize}
\begin{itemize}
\item {Utilização:T. da Bairrada}
\end{itemize}
\begin{itemize}
\item {Proveniência:(Do b. lat. \textunderscore forarius\textunderscore )}
\end{itemize}
Aquelle que tem o domínio útil de algum prédio, pagando fôro ao directo senhorio.
Relativo a fôro.
Que paga fôro.
Inevitável.
Sujeito, obrigado.
Diz-se do rêgo ou regueira, sujeita á correição, consignada nas posturas municipaes.
\section{Forense}
\begin{itemize}
\item {Grp. gram.:adj.}
\end{itemize}
\begin{itemize}
\item {Proveniência:(Lat. \textunderscore forensis\textunderscore )}
\end{itemize}
Relativo ao fôro judicial.
Relativo aos tribunaes: \textunderscore a prática forense\textunderscore .
\section{Forfalha}
\begin{itemize}
\item {Grp. gram.:f.}
\end{itemize}
\begin{itemize}
\item {Utilização:Prov.}
\end{itemize}
\begin{itemize}
\item {Utilização:trasm.}
\end{itemize}
Migalha de pão.
\section{Forfete}
\begin{itemize}
\item {fónica:fê}
\end{itemize}
\begin{itemize}
\item {Grp. gram.:m.}
\end{itemize}
Espécie de vinho, que se vende no Cairo.
\section{Fórfex}
\begin{itemize}
\item {Grp. gram.:m.}
\end{itemize}
\begin{itemize}
\item {Proveniência:(Lat. \textunderscore forfex\textunderscore . A fórma portuguesa sería \textunderscore fórfice\textunderscore )}
\end{itemize}
Espécie de tesoira cirúrgica.
\section{Forfícula}
\begin{itemize}
\item {Grp. gram.:f.}
\end{itemize}
\begin{itemize}
\item {Proveniência:(Lat. \textunderscore forficula\textunderscore )}
\end{itemize}
Gênero de insectos orthópteros.
\section{Forficulários}
\begin{itemize}
\item {Grp. gram.:m. pl.}
\end{itemize}
\begin{itemize}
\item {Proveniência:(De \textunderscore forfícula\textunderscore )}
\end{itemize}
Família de insectos orthópteros.
\section{Forfolha}
\begin{itemize}
\item {fónica:fô}
\end{itemize}
\begin{itemize}
\item {Grp. gram.:f.}
\end{itemize}
Pássaro dos Açores, conhecido em San Miguel por \textunderscore estrellinha\textunderscore , e talvez o mesmo que \textunderscore folosa\textunderscore .
\section{Forgalha}
\begin{itemize}
\item {Grp. gram.:f.}
\end{itemize}
\begin{itemize}
\item {Utilização:Prov.}
\end{itemize}
\begin{itemize}
\item {Utilização:trasm.}
\end{itemize}
O mesmo que \textunderscore forfalha\textunderscore .
\section{Forgul}
\begin{itemize}
\item {Grp. gram.:m.}
\end{itemize}
Árvore da Índia portuguesa.
\section{Forja}
\begin{itemize}
\item {Grp. gram.:f.}
\end{itemize}
\begin{itemize}
\item {Utilização:Fig.}
\end{itemize}
\begin{itemize}
\item {Proveniência:(Fr. \textunderscore forge\textunderscore )}
\end{itemize}
Conjunto da fornalha, folle e bigorna, de que em seu offício se servem os ferreiros e outros artifices que trabalham em metal.
Officina de ferreiro.
\textunderscore Estar na forja\textunderscore , achar-se em preparação, quási prompto.
\section{Forjador}
\begin{itemize}
\item {Grp. gram.:adj.}
\end{itemize}
\begin{itemize}
\item {Grp. gram.:M.}
\end{itemize}
Que forja.
Aquelle que forja.
\section{Forjadura}
\begin{itemize}
\item {Grp. gram.:f.}
\end{itemize}
Acto de forjar.
\section{Forjamento}
\begin{itemize}
\item {Grp. gram.:m.}
\end{itemize}
O mesmo que \textunderscore forjadura\textunderscore .
\section{Forjar}
\begin{itemize}
\item {Grp. gram.:v. t.}
\end{itemize}
\begin{itemize}
\item {Utilização:Fig.}
\end{itemize}
\begin{itemize}
\item {Grp. gram.:V. i.}
\end{itemize}
\begin{itemize}
\item {Proveniência:(De \textunderscore forja\textunderscore )}
\end{itemize}
Aquecer, caldear, bater e operar na forja.
Fabricar.
Inventar; maquinar: \textunderscore forjar intrigas\textunderscore .
Dizem os ferradores, falando do cavallo, que, quando trota, dá com as ferraduras das mãos nas dos pés. Cf. Leon, \textunderscore Arte de Ferrar\textunderscore , 190.
\section{Forjicador}
\begin{itemize}
\item {Grp. gram.:m.}
\end{itemize}
Aquelle que forjica.
\section{Forjicar}
\begin{itemize}
\item {Grp. gram.:v. t.}
\end{itemize}
\begin{itemize}
\item {Utilização:Deprec.}
\end{itemize}
\begin{itemize}
\item {Utilização:Ant.}
\end{itemize}
Forjar mal.
Arranjar defeituosamente.
Inventar, fabricar com apuro, burilar. Cf. \textunderscore Eufrosina\textunderscore , 188; \textunderscore Aulegrafia\textunderscore , 141.
\section{Forjoco}
\begin{itemize}
\item {fónica:jô}
\end{itemize}
\begin{itemize}
\item {Grp. gram.:m.}
\end{itemize}
\begin{itemize}
\item {Utilização:Prov.}
\end{itemize}
\begin{itemize}
\item {Utilização:trasm.}
\end{itemize}
Buraco.
Cova; barranco.
\section{Forlim}
\begin{itemize}
\item {Grp. gram.:m.}
\end{itemize}
\begin{itemize}
\item {Utilização:Ant.}
\end{itemize}
O mesmo que \textunderscore florim\textunderscore .
(Metáth. de \textunderscore florim\textunderscore )
\section{Fórma}
\begin{itemize}
\item {Grp. gram.:f.}
\end{itemize}
\begin{itemize}
\item {Grp. gram.:Loc. adv.}
\end{itemize}
\begin{itemize}
\item {Utilização:Mil.}
\end{itemize}
\begin{itemize}
\item {Proveniência:(Lat. \textunderscore forma\textunderscore )}
\end{itemize}
Disposição exterior das partes de um corpo.
Apparência; configuração; feitio.
Manifestação.
Estado.
Modo.
Modêlo.
Carácter de estilo.
Alinhamento de tropas.
\textunderscore Em fórma\textunderscore , em termos convenientes, devidamente.
\textunderscore Debaixo de fórma\textunderscore , em fileira, por ordem, recebendo ordens ou para receber ordens.
\section{Fôrma}
\begin{itemize}
\item {Grp. gram.:f.}
\end{itemize}
\begin{itemize}
\item {Utilização:Açor}
\end{itemize}
\begin{itemize}
\item {Utilização:Prov.}
\end{itemize}
\begin{itemize}
\item {Utilização:minh.}
\end{itemize}
\begin{itemize}
\item {Proveniência:(Lat. \textunderscore forma\textunderscore )}
\end{itemize}
Peça de madeira, sôbre que o sapateiro cose as peças do calçado.
Molde, sôbre que o chapeleiro dispõe a copa do chapéu.
Cestinho ou vaso de metal, em que se fazem queijos; cincho.
Molde, sôbre que, ou dentro de que, se fórma qualquer coisa que toma o feitio dêsse molde.
Caixilho, em que estão dispostos por sua ordem os caracteres typográphicos.
Peça de madeira, com que se enchem as meias, para as alisar ou para as expor á venda.
Armação de chapéu de senhora; carcassa.
Vaso, em que se coalha e apura o açúcar.
Botão das calças.
Botão de osso em roupa branca.
\section{Formá}
\begin{itemize}
\item {Grp. gram.:m.}
\end{itemize}
Antigo imposto, nas communidades indianas.
\section{Formação}
\begin{itemize}
\item {Grp. gram.:f.}
\end{itemize}
\begin{itemize}
\item {Proveniência:(Lat. \textunderscore formatio\textunderscore )}
\end{itemize}
Acto, effeito ou modo de formar.
\section{Formador}
\begin{itemize}
\item {Grp. gram.:adj.}
\end{itemize}
\begin{itemize}
\item {Grp. gram.:M.}
\end{itemize}
\begin{itemize}
\item {Proveniência:(Lat. \textunderscore formator\textunderscore )}
\end{itemize}
Que fórma.
Aquelle que fórma.
\section{Formadura}
\begin{itemize}
\item {Grp. gram.:f.}
\end{itemize}
\begin{itemize}
\item {Proveniência:(Lat. \textunderscore formatura\textunderscore )}
\end{itemize}
Acto ou effeito de formar.
\section{Formal}
\begin{itemize}
\item {Grp. gram.:adj.}
\end{itemize}
\begin{itemize}
\item {Grp. gram.:M.}
\end{itemize}
\begin{itemize}
\item {Utilização:Jur.}
\end{itemize}
\begin{itemize}
\item {Proveniência:(Lat. \textunderscore formalis\textunderscore )}
\end{itemize}
Relativo á forma.
Evidente, positivo.
Genuíno.
Carta judicial de partilhas.
Casa ou residência, dentro de propriedade emphytêutica.
\section{Formal}
\begin{itemize}
\item {Grp. gram.:m.}
\end{itemize}
\begin{itemize}
\item {Utilização:Prov.}
\end{itemize}
Campo ou região, em que domina certa cultura: \textunderscore grande parte da bacia do Dão é um formal de vinhas\textunderscore .
(Relaciona-se com o fr. \textunderscore fermage\textunderscore ?)
\section{Formaldehydo}
\begin{itemize}
\item {Grp. gram.:m.}
\end{itemize}
Substância antiséptica, o mesmo que \textunderscore formol\textunderscore .
\section{Formaldeído}
\begin{itemize}
\item {Grp. gram.:m.}
\end{itemize}
Substância antiséptica, o mesmo que \textunderscore formol\textunderscore .
\section{Formalidade}
\begin{itemize}
\item {Grp. gram.:f.}
\end{itemize}
\begin{itemize}
\item {Utilização:Prov.}
\end{itemize}
\begin{itemize}
\item {Utilização:minh.}
\end{itemize}
\begin{itemize}
\item {Proveniência:(Lat. \textunderscore formalitas\textunderscore )}
\end{itemize}
Modo de proceder publicamente.
Praxe.
Fórmula.
Ceremónia.
Substancialidade.
Quinhão de terra, em partilha.
\section{Formalina}
\begin{itemize}
\item {Grp. gram.:f.}
\end{itemize}
Composto chímico, de natureza orgânica, empregado como desinfectante.
(Cp. \textunderscore fórmico\textunderscore )
\section{Formalismo}
\begin{itemize}
\item {Grp. gram.:m.}
\end{itemize}
\begin{itemize}
\item {Proveniência:(De \textunderscore formal\textunderscore )}
\end{itemize}
Systema philosóphico, que, negando a existência da matéria, só admitte a fórma.
Systema dos que se prendem muito com formalidades ou ceremónias.
\section{Formalista}
\begin{itemize}
\item {Grp. gram.:m.  e  adj.}
\end{itemize}
\begin{itemize}
\item {Proveniência:(De \textunderscore formal\textunderscore )}
\end{itemize}
Sectário do formalismo.
Amigo de formalidades.
\section{Formalizar}
\begin{itemize}
\item {Grp. gram.:v. t.}
\end{itemize}
\begin{itemize}
\item {Grp. gram.:V. p.}
\end{itemize}
\begin{itemize}
\item {Utilização:Fam.}
\end{itemize}
\begin{itemize}
\item {Proveniência:(De \textunderscore formal\textunderscore )}
\end{itemize}
Realizar, segundo as fórmulas ou segundo as formalidades.
Executar, conforme as regras ou cláusulas.
Dar-se por offendido; escandalizar-se.
\section{Formalmente}
\begin{itemize}
\item {Grp. gram.:adv.}
\end{itemize}
De modo formal.
\section{Formalote}
\begin{itemize}
\item {Grp. gram.:m.}
\end{itemize}
Arco saliente ou nervura de uma abóbada góthica.
(Cp. fr. \textunderscore formeret\textunderscore )
\section{Formão}
\begin{itemize}
\item {Grp. gram.:m.}
\end{itemize}
\begin{itemize}
\item {Proveniência:(Do rad. de \textunderscore formar\textunderscore )}
\end{itemize}
Utensílio de ferro, que tem uma extremidade chata e cortante e a outra embebida num cabo de madeira.
\section{Formão}
\begin{itemize}
\item {Grp. gram.:m.}
\end{itemize}
O mesmo que \textunderscore firmão\textunderscore .
\section{Formar}
\begin{itemize}
\item {Grp. gram.:v. t.}
\end{itemize}
\begin{itemize}
\item {Grp. gram.:V. i.}
\end{itemize}
\begin{itemize}
\item {Grp. gram.:V. p.}
\end{itemize}
\begin{itemize}
\item {Proveniência:(Lat. \textunderscore formare\textunderscore )}
\end{itemize}
Dar fórma a.
Produzir, criar; fabricar.
Conceber: \textunderscore formar um plano\textunderscore .
Pôr em ordem, em linha.
Instruir.
Estabelecer.
Promover a formatura universitária a: \textunderscore meu pai formou o filho mais velho\textunderscore .
Entrar na fórma, em linha.
Tomar fórma.
Educar-se.
Adquirir a formatura universitária.
\section{Formaria}
\begin{itemize}
\item {Grp. gram.:f.}
\end{itemize}
Conjunto de fôrmas de chapeleiro, etc.
\section{Formativo}
\begin{itemize}
\item {Grp. gram.:adj.}
\end{itemize}
\begin{itemize}
\item {Proveniência:(Do lat. \textunderscore formatus\textunderscore )}
\end{itemize}
Que dá fórma.
\section{Formato}
\begin{itemize}
\item {Grp. gram.:m.}
\end{itemize}
\begin{itemize}
\item {Proveniência:(Lat. \textunderscore formatus\textunderscore )}
\end{itemize}
Feitio, dimensão: \textunderscore o formato de um livro\textunderscore .
\section{Formatura}
\begin{itemize}
\item {Grp. gram.:f.}
\end{itemize}
\begin{itemize}
\item {Proveniência:(Lat. \textunderscore formatura\textunderscore )}
\end{itemize}
Acto ou effeito de formar.
Ordenada disposição de tropas.
Approvação no último exame de uma faculdade universitária.
\section{...forme}
\begin{itemize}
\item {Grp. gram.:suf.}
\end{itemize}
\begin{itemize}
\item {Proveniência:(Do lat. \textunderscore forma\textunderscore )}
\end{itemize}
(designativo de fórma, feitio, semelhança)
\section{Formeiro}
\begin{itemize}
\item {Grp. gram.:m.}
\end{itemize}
Aquelle que faz fôrmas de calçado.
\section{Formena}
\begin{itemize}
\item {Grp. gram.:f.}
\end{itemize}
Corpo chímico, composto de carbone e hydrogênio e que predomina no grisu; gás dos pântanos.
\section{Formênico}
\begin{itemize}
\item {Grp. gram.:adj.}
\end{itemize}
\begin{itemize}
\item {Utilização:Chím.}
\end{itemize}
\begin{itemize}
\item {Proveniência:(De \textunderscore formena\textunderscore )}
\end{itemize}
Diz-se de um grupo de carbonetos.
\section{Formenofone}
\begin{itemize}
\item {Grp. gram.:m.}
\end{itemize}
\begin{itemize}
\item {Proveniência:(De \textunderscore formena\textunderscore  + gr. \textunderscore phone\textunderscore )}
\end{itemize}
Aparelho, inventado em 1895, para se conhecer a porção de grisu, que há na atmosfera de uma mina, e poder evitar-se a explosão.
\section{Formenofónio}
\begin{itemize}
\item {Grp. gram.:m.}
\end{itemize}
\begin{itemize}
\item {Proveniência:(De \textunderscore formena\textunderscore  + gr. \textunderscore phone\textunderscore )}
\end{itemize}
Aparelho, inventado em 1895, para se conhecer a porção de grisu, que há na atmosfera de uma mina, e poder evitar-se a explosão.
\section{Formenophone}
\begin{itemize}
\item {Grp. gram.:m.}
\end{itemize}
\begin{itemize}
\item {Proveniência:(De \textunderscore formena\textunderscore  + gr. \textunderscore phone\textunderscore )}
\end{itemize}
Apparelho, inventado em 1895, para se conhecer a porção de grisu, que há na atmosphera de uma mina, e poder evitar-se a explosão.
\section{Formi}
\begin{itemize}
\item {Grp. gram.:m.}
\end{itemize}
Doença, que ataca o bico dos falcões.
\section{Formiato}
\begin{itemize}
\item {Grp. gram.:m.}
\end{itemize}
\begin{itemize}
\item {Proveniência:(De \textunderscore fórmico\textunderscore )}
\end{itemize}
Sal, produzido pela combinação do ácido fórmico com uma base.
\section{Formica}
\begin{itemize}
\item {Grp. gram.:f.}
\end{itemize}
\begin{itemize}
\item {Proveniência:(Lat. \textunderscore formica\textunderscore . Cp. \textunderscore formiga\textunderscore ^1)}
\end{itemize}
Doença herpética.
\section{Formicação}
\begin{itemize}
\item {Grp. gram.:f.}
\end{itemize}
\begin{itemize}
\item {Proveniência:(Lat. \textunderscore formicatio\textunderscore )}
\end{itemize}
Formigueiro.
\section{Formicante}
\begin{itemize}
\item {Grp. gram.:adj.}
\end{itemize}
\begin{itemize}
\item {Proveniência:(Lat. \textunderscore formicans\textunderscore )}
\end{itemize}
Diz-se do pulso fraco.
\section{Formicaríneas}
\begin{itemize}
\item {Grp. gram.:f. pl.}
\end{itemize}
Familia de aves, que têm por typo o formicário.
\section{Formicário}
\begin{itemize}
\item {Grp. gram.:adj.}
\end{itemize}
\begin{itemize}
\item {Grp. gram.:M.}
\end{itemize}
\begin{itemize}
\item {Grp. gram.:Pl.}
\end{itemize}
\begin{itemize}
\item {Proveniência:(Do lat. \textunderscore formica\textunderscore )}
\end{itemize}
Semelhante ou relativo á formiga.
Gênero de aves, que se alimentam de formigas e de outros insectos.
Família de insectos hymenópteros, que têm por typo a formiga.
\section{Formicida}
\begin{itemize}
\item {Grp. gram.:m.}
\end{itemize}
Preparação chímica, para destruição de formigas.
(Por \textunderscore formicicida\textunderscore , do lat. \textunderscore formica\textunderscore  + \textunderscore caedere\textunderscore )
\section{Formicídio}
\begin{itemize}
\item {Grp. gram.:m.}
\end{itemize}
\begin{itemize}
\item {Utilização:bras}
\end{itemize}
\begin{itemize}
\item {Utilização:Neol.}
\end{itemize}
Destruição de formigas. Cf. \textunderscore Jorn. do Commércio\textunderscore , do Rio, de 14-XII-904.
(Cp. \textunderscore formicida\textunderscore )
\section{Formicívoro}
\begin{itemize}
\item {Grp. gram.:adj.}
\end{itemize}
\begin{itemize}
\item {Proveniência:(Do lat. \textunderscore formica\textunderscore  + \textunderscore vorare\textunderscore )}
\end{itemize}
Que se alimenta de formigas.
\section{Fórmico}
\begin{itemize}
\item {Grp. gram.:adj.}
\end{itemize}
\begin{itemize}
\item {Proveniência:(Do rad. do lat. \textunderscore formica\textunderscore )}
\end{itemize}
Diz-se de um ácido, que se extrái das formigas.
\section{Formicular}
\begin{itemize}
\item {Grp. gram.:adj.}
\end{itemize}
\begin{itemize}
\item {Proveniência:(Do lat. \textunderscore formicula\textunderscore )}
\end{itemize}
Relativo ou semelhante a formigas.
\section{Formidando}
\begin{itemize}
\item {Grp. gram.:adj.}
\end{itemize}
\begin{itemize}
\item {Proveniência:(Lat. \textunderscore formidandus\textunderscore )}
\end{itemize}
Que mete mêdo, terrível; formidável. Cf. Castilho, \textunderscore Fastos\textunderscore , I, 22.
\section{Formidável}
\begin{itemize}
\item {Grp. gram.:adj.}
\end{itemize}
\begin{itemize}
\item {Proveniência:(Lat. \textunderscore formidabilis\textunderscore )}
\end{itemize}
Medonhamente grande; terrível.
Pavoroso.
Temeroso: \textunderscore incêndio formidável\textunderscore .
\section{Formidavelmente}
\begin{itemize}
\item {Grp. gram.:adv.}
\end{itemize}
De modo formidável.
\section{Formidolosamente}
\begin{itemize}
\item {Grp. gram.:adv.}
\end{itemize}
De modo formidoloso.
\section{Formidoloso}
\begin{itemize}
\item {Grp. gram.:adj.}
\end{itemize}
\begin{itemize}
\item {Proveniência:(Lat. \textunderscore formidolosus\textunderscore )}
\end{itemize}
Que causa mêdo.
Que tem mêdo.
\section{Formiga}
\begin{itemize}
\item {Grp. gram.:f.}
\end{itemize}
\begin{itemize}
\item {Utilização:Fig.}
\end{itemize}
\begin{itemize}
\item {Grp. gram.:Loc. adv.}
\end{itemize}
\begin{itemize}
\item {Proveniência:(Lat. \textunderscore formica\textunderscore )}
\end{itemize}
Pequeno insecto hymenóptero, que vive debaixo da terra.
Pessôa económica.
Rochedo, coberto de água.
Baixio.
\textunderscore Á formiga\textunderscore , á socapa, sorrateiramente.
\section{Formiga}
\begin{itemize}
\item {Grp. gram.:f.}
\end{itemize}
\begin{itemize}
\item {Proveniência:(De \textunderscore Formiga\textunderscore , n. p. de uma quinta, próximo de Lamego)}
\end{itemize}
Variedade de pêra.
\section{Formiga-branca}
\begin{itemize}
\item {Grp. gram.:f.}
\end{itemize}
Pequena formiga, que corrói a madeira e que no Brasil se chama \textunderscore cupim\textunderscore .
\section{Formiga-carregadeira}
\begin{itemize}
\item {Grp. gram.:f.}
\end{itemize}
\begin{itemize}
\item {Utilização:Bras. do Rio}
\end{itemize}
O mesmo que \textunderscore saúba\textunderscore .
\section{Formiga-de-roça}
\begin{itemize}
\item {Grp. gram.:f.}
\end{itemize}
\begin{itemize}
\item {Utilização:Bras. de Pernambuco}
\end{itemize}
O mesmo que \textunderscore saúba\textunderscore .
\section{Formigame}
\begin{itemize}
\item {Grp. gram.:m.}
\end{itemize}
O mesmo que \textunderscore betão\textunderscore .
\section{Formigamento}
\begin{itemize}
\item {Grp. gram.:m.}
\end{itemize}
\begin{itemize}
\item {Proveniência:(De \textunderscore formigar\textunderscore )}
\end{itemize}
O mesmo que \textunderscore formigueiro\textunderscore .
\section{Formigante}
\begin{itemize}
\item {Grp. gram.:adj.}
\end{itemize}
Que formiga.
\section{Formigão}
\begin{itemize}
\item {Grp. gram.:m.}
\end{itemize}
\begin{itemize}
\item {Utilização:T. escol. de Coimbra.}
\end{itemize}
\begin{itemize}
\item {Utilização:Coimbra}
\end{itemize}
\begin{itemize}
\item {Grp. gram.:Adj.}
\end{itemize}
\begin{itemize}
\item {Proveniência:(De \textunderscore formiga\textunderscore )}
\end{itemize}
Formiga grande.
Mistura de cal, saibro e cascalho, para construcções.
Rastilho, que communica o fogo a uma mina.
Estudante do seminário.
Diz-se do toiro, que tem as hastes pouco agudas.
\section{Formigar}
\begin{itemize}
\item {Grp. gram.:v. i.}
\end{itemize}
\begin{itemize}
\item {Grp. gram.:V. t.}
\end{itemize}
\begin{itemize}
\item {Proveniência:(Do lat. \textunderscore formicare\textunderscore )}
\end{itemize}
Sentir formigueiro, têr comichão.
Abundar, existir em grande número: \textunderscore formigavam feirantes\textunderscore .
Tratar das suas coisas com cuidado, com zêlo.
Governar bem a vida.
Passar, como formiga. Cf. Arn. Gama, \textunderscore Últ. Dona\textunderscore , 399.
Passar de mão em mão, (falando-se dos marnotos que, em linha, passam de mão em mão o torrão do sal, dos taburnos para os barcos).
\section{Formigo}
\begin{itemize}
\item {Grp. gram.:m.}
\end{itemize}
Depósito pulverulento de substância córnea do pé dos solípedes, entre o casco e o tecido vivo, numa doença chamada aguamento chrónico.
(Cp. \textunderscore formigueiro\textunderscore )
\section{Formigo}
\begin{itemize}
\item {Grp. gram.:m.}
\end{itemize}
\begin{itemize}
\item {Utilização:Prov.}
\end{itemize}
\begin{itemize}
\item {Utilização:trasm.}
\end{itemize}
Certa variedade de uva branca, de todas a mais temporan.
\section{Formigos}
\begin{itemize}
\item {Grp. gram.:m. pl.}
\end{itemize}
\begin{itemize}
\item {Utilização:Prov.}
\end{itemize}
\begin{itemize}
\item {Utilização:minh.}
\end{itemize}
\begin{itemize}
\item {Utilização:Prov.}
\end{itemize}
\begin{itemize}
\item {Utilização:trasm.}
\end{itemize}
\begin{itemize}
\item {Utilização:Prov.}
\end{itemize}
\begin{itemize}
\item {Utilização:trasm.}
\end{itemize}
Iguaria, feita de pão, mel, vinho, manteiga, ovos, açúcar, usada em festas do Natal e Anno Bom.
Chouriços de sangue e alhos.
O primeiro leite de vaca, fervido e misturado com mel.
\section{Formigueira}
\begin{itemize}
\item {Grp. gram.:f.}
\end{itemize}
Planta herbácea, (\textunderscore chenopodium ambrosioides\textunderscore ).
\section{Formigueirinho}
\begin{itemize}
\item {Grp. gram.:m.  e  adj.}
\end{itemize}
\begin{itemize}
\item {Utilização:pop.}
\end{itemize}
\begin{itemize}
\item {Utilização:Ant.}
\end{itemize}
\begin{itemize}
\item {Proveniência:(De \textunderscore formigueiro\textunderscore )}
\end{itemize}
Ladrão de coisas de pouca monta.
Ratoneiro, larápio.
\section{Formigueiro}
\begin{itemize}
\item {Grp. gram.:m.}
\end{itemize}
\begin{itemize}
\item {Utilização:Ext.}
\end{itemize}
\begin{itemize}
\item {Utilização:Fig.}
\end{itemize}
\begin{itemize}
\item {Grp. gram.:Adj.}
\end{itemize}
\begin{itemize}
\item {Proveniência:(De \textunderscore formiga\textunderscore )}
\end{itemize}
Grande porção de formigas.
Buraco ou toca, em que habitam formigas.
Muita gente reunida ou desfilando.
Muitos animaes.
Espécie de prurido ou sensação, semelhante á que produzem formigas, passando sôbre a pelle.
Doença de cavallos, o mesmo que \textunderscore formigo\textunderscore ^2.
Impaciencia.
Dizia-se do ladrão, que se esconde para furtar ou que furta coisas de pequeno valor.
\section{Formiguejar}
\begin{itemize}
\item {Grp. gram.:v. i.}
\end{itemize}
\begin{itemize}
\item {Proveniência:(De \textunderscore formiga\textunderscore )}
\end{itemize}
Andar ou mover-se, em grande quantidade, como formigueiro: \textunderscore a multidão formiguejava na praça\textunderscore .
\section{Formiguilho}
\begin{itemize}
\item {Grp. gram.:m.}
\end{itemize}
\begin{itemize}
\item {Proveniência:(Do rad. de \textunderscore formiga\textunderscore )}
\end{itemize}
Doença cavallar, resultante de um buraco entre o casco e o sauco.
O mesmo que \textunderscore formigo\textunderscore ^1?
\section{Formilha}
\begin{itemize}
\item {Grp. gram.:f.}
\end{itemize}
\begin{itemize}
\item {Proveniência:(De \textunderscore fórma\textunderscore )}
\end{itemize}
Peça de ferro, em que os rolheiros collocam a faca, com que fazem rolhas.
\section{Formilhão}
\begin{itemize}
\item {Grp. gram.:m.}
\end{itemize}
\begin{itemize}
\item {Proveniência:(De \textunderscore formilho\textunderscore )}
\end{itemize}
Instrumento de chapeleiro, para dar fórma ás abas dos chapéus.
\section{Formilhar}
\begin{itemize}
\item {Grp. gram.:v. i.}
\end{itemize}
Trabalhar com formilho.
\section{Formilho}
\begin{itemize}
\item {Grp. gram.:m.}
\end{itemize}
\begin{itemize}
\item {Proveniência:(De \textunderscore fôrma\textunderscore )}
\end{itemize}
Instrumento de chapeleiro, para dar fórma á bôca da copa dos chapeus.
\section{Formista}
\begin{itemize}
\item {Grp. gram.:m.}
\end{itemize}
Fabricante de fôrmas.
Formeiro.
\section{Formol}
\begin{itemize}
\item {Grp. gram.:m.}
\end{itemize}
Preparação antiséptica, applicável especialmente contra a mordedura venenosa de certos animaes.
(Cf. \textunderscore fórmico\textunderscore )
\section{Formolizador}
\begin{itemize}
\item {Grp. gram.:m.}
\end{itemize}
Apparelho, para desinfecções, por meio de formol.
\section{Formosa}
\begin{itemize}
\item {Grp. gram.:f.}
\end{itemize}
\begin{itemize}
\item {Utilização:Náut.}
\end{itemize}
\begin{itemize}
\item {Proveniência:(De \textunderscore formoso\textunderscore )}
\end{itemize}
Variedade de uva branca das vizinhanças de Lisbôa.
Uma das velas latinas que servem nos estais, e que também se chama \textunderscore cozinheira\textunderscore .
\section{Formosa-de-bèsteiros}
\begin{itemize}
\item {Grp. gram.:f.}
\end{itemize}
O mesmo que \textunderscore pêra-lemos\textunderscore .
\section{Formosa-de-darei}
\begin{itemize}
\item {Grp. gram.:f.}
\end{itemize}
Variedade de pêra portuguesa.
\section{Formosa-de-um-dia}
\begin{itemize}
\item {Grp. gram.:f.}
\end{itemize}
Planta liliácea.
\section{Formosano}
\begin{itemize}
\item {Grp. gram.:m.}
\end{itemize}
Língua malaiò-polynésica da Ilha Formosa.
\section{Formosear}
\begin{itemize}
\item {Grp. gram.:v. t.}
\end{itemize}
O mesmo que \textunderscore aformosear\textunderscore .
\section{Formosentar}
\begin{itemize}
\item {Grp. gram.:v. t.}
\end{itemize}
(V.aformosentar)
\section{Formosidade}
\begin{itemize}
\item {Grp. gram.:f.}
\end{itemize}
\begin{itemize}
\item {Utilização:Des.}
\end{itemize}
O mesmo que \textunderscore formosura\textunderscore .
\section{Formoso}
\begin{itemize}
\item {Grp. gram.:adj.}
\end{itemize}
\begin{itemize}
\item {Proveniência:(Lat. \textunderscore formosus\textunderscore )}
\end{itemize}
Que tem fórmas agradáveis ou aspecto agradável: \textunderscore mulher formosa\textunderscore .
Bem conformado.
Bello; deleitoso: \textunderscore dia formoso\textunderscore .
Brilhante.
Perfeito.
Que sôa bem, harmonioso.
\section{Formosura}
\begin{itemize}
\item {Grp. gram.:f.}
\end{itemize}
Qualidade daquelle ou daquillo que é formoso.
Pessôa formosa.
Coisa formosa.
Peixe dos Açores.
\section{Fórmula}
\begin{itemize}
\item {Grp. gram.:f.}
\end{itemize}
\begin{itemize}
\item {Proveniência:(Lat. \textunderscore formula\textunderscore )}
\end{itemize}
Expressão de um preceito ou de um principio.
Receita, escrita por um médico ou indicada em algum receituário ou pharmacopeia.
Expressão mathemática, para resolver problemas análogos.
Palavras, que servem invariavelmente na celebração de certos actos solennes.
\section{Formulação}
\begin{itemize}
\item {Grp. gram.:f.}
\end{itemize}
Acto ou effeito de formular.
\section{Formulador}
\begin{itemize}
\item {Grp. gram.:m.}
\end{itemize}
Aquelle que formúla.
\section{Formular}
\begin{itemize}
\item {Grp. gram.:v. t.}
\end{itemize}
Reduzir a uma fórmula.
Receitar.
Expor com precisão.
\section{Formulário}
\begin{itemize}
\item {Grp. gram.:m.}
\end{itemize}
\begin{itemize}
\item {Proveniência:(Lat. \textunderscore formularius\textunderscore )}
\end{itemize}
Collecção de fórmulas.
Livro de orações.
\section{Formulista}
\begin{itemize}
\item {Grp. gram.:m.}
\end{itemize}
Aquelle que prescreve fórmulas.
Aquelle que segue rigorosamente certas fórmulas; formalista.
\section{Fornaça}
\begin{itemize}
\item {Grp. gram.:f.}
\end{itemize}
\begin{itemize}
\item {Utilização:Ant.}
\end{itemize}
\begin{itemize}
\item {Proveniência:(Do lat. \textunderscore fornax\textunderscore )}
\end{itemize}
O mesmo que \textunderscore fornalha\textunderscore .
Casa, onde se cunhava moéda.
\section{Fornacaes}
\begin{itemize}
\item {Grp. gram.:m. pl.}
\end{itemize}
\begin{itemize}
\item {Proveniência:(Do lat. \textunderscore fornacalia\textunderscore )}
\end{itemize}
Sacrifícios, quo se faziam, em honra da deusa Fornaz, quando se secavam os trigos nos fornos.
\section{Fornacais}
\begin{itemize}
\item {Grp. gram.:m. pl.}
\end{itemize}
\begin{itemize}
\item {Proveniência:(Do lat. \textunderscore fornacalia\textunderscore )}
\end{itemize}
Sacrifícios, quo se faziam, em honra da deusa Fornaz, quando se secavam os trigos nos fornos.
\section{Fornaceiro}
\begin{itemize}
\item {Grp. gram.:m.}
\end{itemize}
\begin{itemize}
\item {Proveniência:(Do lat. \textunderscore fornacarius\textunderscore )}
\end{itemize}
Aquelle que trabalha nas fornalhas da casa da moéda.
\section{Fornaço}
\begin{itemize}
\item {Grp. gram.:m.}
\end{itemize}
\begin{itemize}
\item {Utilização:Prov.}
\end{itemize}
\begin{itemize}
\item {Utilização:trasm.}
\end{itemize}
Rosca de pão, que as mães fazem aos filhos, quando fazem a fornada.
(Cp. \textunderscore fornaça\textunderscore )
\section{Fornada}
\begin{itemize}
\item {Grp. gram.:f.}
\end{itemize}
\begin{itemize}
\item {Utilização:Fig.}
\end{itemize}
Os pães que se cozem ao mesmo tempo e no mesmo forno.
Aquillo que um forno coze de uma vez.
Quantidade de coisas que se fazem de uma vez, ou de pessôas que se nomeiam ao mesmo tempo para certos cargos.
\section{Fornalha}
\begin{itemize}
\item {Grp. gram.:f.}
\end{itemize}
\begin{itemize}
\item {Utilização:Fig.}
\end{itemize}
\begin{itemize}
\item {Proveniência:(Do lat. \textunderscore fornacula\textunderscore )}
\end{itemize}
Forno.
Parte de uma máquina, ou de um fogão, em que arde o combustível.
Lugar muito quente.
Calor intenso.
\section{Fornalheira}
\begin{itemize}
\item {Grp. gram.:f.}
\end{itemize}
\begin{itemize}
\item {Utilização:Prov.}
\end{itemize}
\begin{itemize}
\item {Utilização:trasm.}
\end{itemize}
\begin{itemize}
\item {Proveniência:(De \textunderscore fornalha\textunderscore )}
\end{itemize}
Tulha de cinza.
\section{Fornalheiro}
\begin{itemize}
\item {Grp. gram.:m.}
\end{itemize}
\begin{itemize}
\item {Proveniência:(De \textunderscore fornalha\textunderscore )}
\end{itemize}
O mesmo que \textunderscore fogueiro\textunderscore .
\section{Fornazinho}
\begin{itemize}
\item {Grp. gram.:adj.}
\end{itemize}
(V.fornezinho)
\section{Fornear}
\begin{itemize}
\item {Grp. gram.:v. i.}
\end{itemize}
O mesmo que \textunderscore fornejar\textunderscore .
\section{Fornecedor}
\begin{itemize}
\item {Grp. gram.:m.  e  adj.}
\end{itemize}
\begin{itemize}
\item {Proveniência:(De \textunderscore fornecer\textunderscore )}
\end{itemize}
O que fornece, ou se obriga a fornecer, certas provisões.
\section{Fornecer}
\begin{itemize}
\item {Grp. gram.:v. t.}
\end{itemize}
\begin{itemize}
\item {Grp. gram.:V. p.}
\end{itemize}
\begin{itemize}
\item {Proveniência:(De \textunderscore fornir\textunderscore )}
\end{itemize}
Prestar o necessário a: \textunderscore fornecer alimentação\textunderscore .
Ministrar.
Abastecer.
Guarnecer: \textunderscore fornecer uma praça\textunderscore .
Facilitar, proporcionar: \textunderscore fornecer livros a um estudante\textunderscore .
Fazer provisão ou acquisição.--Na acepção de \textunderscore dar\textunderscore , \textunderscore ministrar\textunderscore , é francesia ridícula, no conceito de Castilho.
\section{Fornecimento}
\begin{itemize}
\item {Grp. gram.:m.}
\end{itemize}
Acto ou effeito de fornecer.
\section{Forneco}
\begin{itemize}
\item {Grp. gram.:m.}
\end{itemize}
\begin{itemize}
\item {Utilização:Carp.}
\end{itemize}
Peça de madeira, que, na construcção dos telhados, liga a tacaniça ou rincão ao frechal.
\section{Forneira}
\begin{itemize}
\item {Grp. gram.:f.}
\end{itemize}
Mulher, que é dona de forno ou que trata de fornos.
O mesmo que \textunderscore bicho-de-conta\textunderscore .
\section{Forneiro}
\begin{itemize}
\item {Grp. gram.:m.}
\end{itemize}
\begin{itemize}
\item {Utilização:T. de Pare -de-Coira}
\end{itemize}
\begin{itemize}
\item {Utilização:des.}
\end{itemize}
Dono de forno.
Aquelle que trata do forno.
Queimada, fogueira nos campos, para se adubar a terra com a cinza.
\section{Fornejar}
\begin{itemize}
\item {Grp. gram.:v. i.}
\end{itemize}
\begin{itemize}
\item {Proveniência:(De \textunderscore forno\textunderscore )}
\end{itemize}
Exercer o mester de forneiro.
\section{Fornezinho}
\begin{itemize}
\item {Grp. gram.:adj.}
\end{itemize}
\begin{itemize}
\item {Utilização:Ant.}
\end{itemize}
Gerado por união illícita; bastardo.
(Por \textunderscore fornizinho\textunderscore , de \textunderscore fornízio\textunderscore )
\section{Fornicação}
\begin{itemize}
\item {Grp. gram.:f.}
\end{itemize}
\begin{itemize}
\item {Proveniência:(Lat. \textunderscore fornicatio\textunderscore )}
\end{itemize}
Acto de fornicar.
\section{Fornicador}
\begin{itemize}
\item {Grp. gram.:m.}
\end{itemize}
\begin{itemize}
\item {Proveniência:(Lat. \textunderscore fornicator\textunderscore )}
\end{itemize}
Aquelle que fornica.
\section{Fornicar}
\begin{itemize}
\item {Grp. gram.:v. t.}
\end{itemize}
\begin{itemize}
\item {Utilização:Chul.}
\end{itemize}
\begin{itemize}
\item {Grp. gram.:V. i}
\end{itemize}
\begin{itemize}
\item {Proveniência:(Lat. \textunderscore fornicari\textunderscore )}
\end{itemize}
Têr cóito com.
Importunar, apoquentar.
Têr cóito.
\section{Fornicária}
\begin{itemize}
\item {Grp. gram.:f.}
\end{itemize}
\begin{itemize}
\item {Utilização:Ant.}
\end{itemize}
\begin{itemize}
\item {Proveniência:(Lat. \textunderscore fornicaria\textunderscore )}
\end{itemize}
Meretriz.
\section{Fornicário}
\begin{itemize}
\item {Grp. gram.:m.  e  adj.}
\end{itemize}
\begin{itemize}
\item {Utilização:Ant.}
\end{itemize}
\begin{itemize}
\item {Proveniência:(Lat. \textunderscore fornicarius\textunderscore )}
\end{itemize}
Homem dissoluto.
\section{Fórnice}
\begin{itemize}
\item {Grp. gram.:f.}
\end{itemize}
\begin{itemize}
\item {Proveniência:(Do lat. \textunderscore fornix\textunderscore )}
\end{itemize}
Arco de porta em parede mestra.
Abóbada.
\section{Fornício}
\begin{itemize}
\item {Grp. gram.:m.}
\end{itemize}
\begin{itemize}
\item {Utilização:Ant.}
\end{itemize}
\begin{itemize}
\item {Proveniência:(Do b. lat. \textunderscore fornicium\textunderscore )}
\end{itemize}
O mesmo que fornicação.
\section{Fornicoques}
\begin{itemize}
\item {Grp. gram.:m. pl.}
\end{itemize}
\begin{itemize}
\item {Utilização:Pleb.}
\end{itemize}
\begin{itemize}
\item {Proveniência:(De \textunderscore fornicar\textunderscore ?)}
\end{itemize}
Cócegas.
Appetite; tentação.
\section{Fornigar}
\begin{itemize}
\item {Grp. gram.:v. t.}
\end{itemize}
\begin{itemize}
\item {Utilização:Ant.}
\end{itemize}
O mesmo que \textunderscore fornicar\textunderscore .
\section{Fornilha}
\begin{itemize}
\item {Grp. gram.:f.}
\end{itemize}
\begin{itemize}
\item {Utilização:Prov.}
\end{itemize}
\begin{itemize}
\item {Utilização:alg.}
\end{itemize}
O mesmo que \textunderscore fornilho\textunderscore .
\section{Fornilho}
\begin{itemize}
\item {Grp. gram.:m.}
\end{itemize}
Pequeno forno ou fogareiro.
Parte do cachimbo, onde arde o tabaco.
Caixão de pólvora, que se enterra, para o fazer explodir, em occasião de guerra.
\section{Fornimento}
\begin{itemize}
\item {Grp. gram.:m.}
\end{itemize}
Acto ou effeito de fornir.
Tabuado.
Robustez.
\section{Fornir}
\begin{itemize}
\item {Grp. gram.:v. t.}
\end{itemize}
Fornecer.
Tornar nutrido, robusto.
(B. lat. \textunderscore furnire\textunderscore )
\section{Fornitura}
\begin{itemize}
\item {Grp. gram.:f.}
\end{itemize}
\begin{itemize}
\item {Utilização:inútil}
\end{itemize}
\begin{itemize}
\item {Utilização:Gal}
\end{itemize}
\begin{itemize}
\item {Proveniência:(Fr. \textunderscore fourniture\textunderscore )}
\end{itemize}
O mesmo que \textunderscore fornecimento\textunderscore .
\section{Forniziada}
\begin{itemize}
\item {Grp. gram.:f.}
\end{itemize}
Acto de forniziar.
\section{Forniziar}
\begin{itemize}
\item {Grp. gram.:v. t.}
\end{itemize}
\begin{itemize}
\item {Utilização:Ant.}
\end{itemize}
\begin{itemize}
\item {Proveniência:(De \textunderscore fornízio\textunderscore )}
\end{itemize}
O mesmo que \textunderscore fornicar\textunderscore .
\section{Fornízio}
\begin{itemize}
\item {Grp. gram.:m.}
\end{itemize}
\begin{itemize}
\item {Utilização:Ant.}
\end{itemize}
Cóito peccaminoso.
(Cp. \textunderscore fornício\textunderscore )
\section{Forno}
\begin{itemize}
\item {fónica:fôr}
\end{itemize}
\begin{itemize}
\item {Grp. gram.:m.}
\end{itemize}
\begin{itemize}
\item {Utilização:Fig.}
\end{itemize}
\begin{itemize}
\item {Utilização:Prov.}
\end{itemize}
\begin{itemize}
\item {Utilização:minh.}
\end{itemize}
\begin{itemize}
\item {Proveniência:(Lat. \textunderscore furnus\textunderscore  = \textunderscore fornus\textunderscore )}
\end{itemize}
Construcção abobadada, com uma porta lateral, dentro do qual se coze pão, se assa carne, etc.
Construcção análoga, com abertura superior, para cozer loiça, cal, telha, etc.
Parte do fogão para fazer assados.
Lugar muito quente: \textunderscore esta casa é um forno\textunderscore .
Cavidade, gruta, que serve de abrigo de pastores.
\textunderscore Alto forno\textunderscore , o forno que é destinado á fundição de metaes.
\section{Foro}
\begin{itemize}
\item {fónica:fô}
\end{itemize}
\begin{itemize}
\item {Grp. gram.:m.}
\end{itemize}
\begin{itemize}
\item {Proveniência:(Lat. \textunderscore forum\textunderscore )}
\end{itemize}
Quantia ou pensão, que o emphyteuta de um prédio paga annualmente ao senhorio directo.
Domínio útil de um prédio.
Encargo habitual.
Uso ou privilégio, garantido pelo tempo ou pela lei.
Immunidade.
Tribunaes judiciaes; jurisdicção: \textunderscore o foro civil\textunderscore ; \textunderscore o foro commercial\textunderscore .
\section{Forol}
\begin{itemize}
\item {Grp. gram.:m.}
\end{itemize}
\begin{itemize}
\item {Utilização:Des.}
\end{itemize}
O mesmo que \textunderscore farol\textunderscore . Cf. \textunderscore Roteiro de Vasco da Gama\textunderscore .
\section{Forqueadura}
\begin{itemize}
\item {Grp. gram.:f.}
\end{itemize}
Acto ou effeito de forquear.
\section{Forquear}
\begin{itemize}
\item {Grp. gram.:v. t.}
\end{itemize}
\begin{itemize}
\item {Proveniência:(De \textunderscore fôrca\textunderscore )}
\end{itemize}
O mesmo que \textunderscore bifurcar\textunderscore .
\section{Forqueta}
\begin{itemize}
\item {fónica:quê}
\end{itemize}
\begin{itemize}
\item {Grp. gram.:f.}
\end{itemize}
\begin{itemize}
\item {Proveniência:(De \textunderscore fôrca\textunderscore )}
\end{itemize}
Tronco ou pau bifurcado; forquilha.
\section{Forquilha}
\begin{itemize}
\item {Grp. gram.:f.}
\end{itemize}
\begin{itemize}
\item {Proveniência:(De \textunderscore fôrca\textunderscore )}
\end{itemize}
Pequeno forcado de três pontas.
Vara bifurcada, em que descansa o braço do andor.
Garfo.
Espeque bifurcado, para esteio de árvores, etc.
Cabide; descanso.
Pequeno disco metállico com duas hastes, que se colloca por baixo de cada corda, nas harpas modernas.
Posição dos dedos indicador, médio e anular, quando o primeiro e o terceiro cobrem os respectivos orifícios nos instrumentos de sopro, em quanto o segundo está levantado.
\section{Forquilhar}
\begin{itemize}
\item {Grp. gram.:v. t.}
\end{itemize}
Bifurcar; converter em forquilha.
\section{Forquilhoso}
\begin{itemize}
\item {Grp. gram.:adj.}
\end{itemize}
Que termina em forquilha.
\section{Fôrra}
\begin{itemize}
\item {Grp. gram.:f.}
\end{itemize}
\begin{itemize}
\item {Utilização:Bras}
\end{itemize}
\begin{itemize}
\item {Proveniência:(De \textunderscore forrar\textunderscore ^1)}
\end{itemize}
Faixa, com que se fortalecem as velas do navio.
Chumaço, entretela.
Peça de mármore, para revestimento de uma construcção.
\section{Fôrra}
\begin{itemize}
\item {Grp. gram.:adj. f.}
\end{itemize}
\begin{itemize}
\item {Proveniência:(De \textunderscore forrar\textunderscore ^2)}
\end{itemize}
Diz-se da ovelha que não foi lançada ao carneiro ou que não está prenhe.
\section{Forração}
\begin{itemize}
\item {Grp. gram.:f.}
\end{itemize}
\begin{itemize}
\item {Utilização:Bras}
\end{itemize}
Acto ou effeito de \textunderscore forrar\textunderscore ^1.
\section{Forrador}
\begin{itemize}
\item {Grp. gram.:m.}
\end{itemize}
\begin{itemize}
\item {Proveniência:(De \textunderscore forrar\textunderscore ^1)}
\end{itemize}
Aquelle que forra.
\section{Fòrragaitas}
\begin{itemize}
\item {Grp. gram.:m.}
\end{itemize}
\begin{itemize}
\item {Utilização:Pop.}
\end{itemize}
\begin{itemize}
\item {Proveniência:(De \textunderscore forrar\textunderscore  + \textunderscore gaita\textunderscore )}
\end{itemize}
Avarento.
\section{Forrageador}
\begin{itemize}
\item {Grp. gram.:m.  e  adj.}
\end{itemize}
O que forrageia.
\section{Forrageal}
\begin{itemize}
\item {Grp. gram.:m.}
\end{itemize}
Campo de forragem.
\section{Forragear}
\begin{itemize}
\item {Grp. gram.:v. t.}
\end{itemize}
\begin{itemize}
\item {Utilização:Fig.}
\end{itemize}
\begin{itemize}
\item {Grp. gram.:V. i.}
\end{itemize}
Ceifar forragem em.
Remexer.
Respigar.
Compilar, plagiando.
Segar forragens.
\section{Forrageiro}
\begin{itemize}
\item {Grp. gram.:m.}
\end{itemize}
\begin{itemize}
\item {Grp. gram.:Adj.}
\end{itemize}
O mesmo que \textunderscore forrageador\textunderscore .
Relativo a forragem:«\textunderscore plantas forrageiras\textunderscore ». \textunderscore Jornal do Comm.\textunderscore , do Rio, de 16-XI-900.
\section{Forragem}
\begin{itemize}
\item {Grp. gram.:f.}
\end{itemize}
Erva, para alimentar gado.
Quantia, que se dá a funccionários, especialmente aos militares, para sustento do cavallo que lhes compete.
(Corr. de \textunderscore farragem\textunderscore ?)
\section{Forraginoso}
\begin{itemize}
\item {Grp. gram.:adj.}
\end{itemize}
Que serve para forragem; que produz forragem.
\section{Forrajoso}
\begin{itemize}
\item {Grp. gram.:adj.}
\end{itemize}
(V.forraginoso)
\section{Forramento}
\begin{itemize}
\item {Grp. gram.:m.}
\end{itemize}
Acto ou effeito de forrar.
\section{Forrar}
\begin{itemize}
\item {Grp. gram.:v. t.}
\end{itemize}
\begin{itemize}
\item {Grp. gram.:V. p.}
\end{itemize}
\begin{itemize}
\item {Proveniência:(De \textunderscore fôrro\textunderscore ^1)}
\end{itemize}
Pôr forro em.
Cobrir de papel, estôfo, lâminas de madeira ou metal.
Reforçar com entretela.
Poupar, economizar: \textunderscore forrar dinheiro\textunderscore .
Tirar a desforra.
\section{Forrar}
\begin{itemize}
\item {Grp. gram.:v. t.}
\end{itemize}
\begin{itemize}
\item {Proveniência:(De \textunderscore fôrro\textunderscore ^2)}
\end{itemize}
Tornar livre, dar alforria a.
\section{Forrear}
\begin{itemize}
\item {Grp. gram.:v. i.}
\end{itemize}
\begin{itemize}
\item {Utilização:Ant.}
\end{itemize}
O mesmo que \textunderscore forragear\textunderscore .
\section{Forreca}
\begin{itemize}
\item {Grp. gram.:m.}
\end{itemize}
\begin{itemize}
\item {Utilização:Bras. do N}
\end{itemize}
O mesmo que \textunderscore pobretão\textunderscore .
\section{Forrejar}
\begin{itemize}
\item {Grp. gram.:v. t.}
\end{itemize}
(V.forragear)
\section{Forreta}
\begin{itemize}
\item {fónica:rê}
\end{itemize}
\begin{itemize}
\item {Grp. gram.:m.  e  f.}
\end{itemize}
\begin{itemize}
\item {Proveniência:(De \textunderscore forrar\textunderscore ^1)}
\end{itemize}
Pessôa avarenta.
\section{Forrica}
\begin{itemize}
\item {Grp. gram.:f.}
\end{itemize}
\begin{itemize}
\item {Utilização:Prov.}
\end{itemize}
\begin{itemize}
\item {Utilização:minh.}
\end{itemize}
Dejecções quási líquidas. (Colhido em Barcellos)
\section{Forricar-se}
\begin{itemize}
\item {Grp. gram.:v. p.}
\end{itemize}
\begin{itemize}
\item {Utilização:Prov.}
\end{itemize}
\begin{itemize}
\item {Utilização:beir.}
\end{itemize}
\begin{itemize}
\item {Proveniência:(De \textunderscore fôrro\textunderscore ^2)}
\end{itemize}
Desligar-se de um compromisso.
\section{Forriques}
\begin{itemize}
\item {Grp. gram.:m. pl.}
\end{itemize}
\begin{itemize}
\item {Utilização:Prov.}
\end{itemize}
\begin{itemize}
\item {Utilização:trasm.}
\end{itemize}
\textunderscore Fazer os forriques\textunderscore , ganhar o jôgo das nécaras, aparando o chelim no último lance.
\section{Fôrro}
\begin{itemize}
\item {Grp. gram.:m.}
\end{itemize}
\begin{itemize}
\item {Grp. gram.:Loc. adv.}
\end{itemize}
Tudo que serve para encher ou reforçar interiormente algum artefacto.
Tecido, com que se cobre o assento de sofás, cadeiras, etc.
Tábuas, com que se reveste interiormente o tecto das casas.
Espaço, entre o telhado e o tecto das salas ou quartos.
Revestimento de paredes ou edifícios.
Revestimento exterior do fundo dos navios, das amuradas, etc.
\textunderscore A fôrro\textunderscore , sorrateiramente?«\textunderscore ...que já se conluiara com Haliadux e entrara em fôrro em sua perfidia...\textunderscore »Filinto, \textunderscore D. Man.\textunderscore , II, 7.
\textunderscore Fôrro negro\textunderscore , variedade de uva tinta da Arruda.
(Cp. fr. \textunderscore feurre\textunderscore , do germ.)
\section{Fôrro}
\begin{itemize}
\item {Grp. gram.:adj.}
\end{itemize}
\begin{itemize}
\item {Proveniência:(Do ár. \textunderscore horr\textunderscore ?)}
\end{itemize}
Que teve alforria; liberto; livre; desobrigado: \textunderscore escravo fôrro\textunderscore .
\section{Fôrro}
\begin{itemize}
\item {Grp. gram.:adj.}
\end{itemize}
\begin{itemize}
\item {Utilização:Prov.}
\end{itemize}
\begin{itemize}
\item {Utilização:trasm.}
\end{itemize}
\begin{itemize}
\item {Utilização:minh.}
\end{itemize}
\begin{itemize}
\item {Proveniência:(De \textunderscore forrar\textunderscore ^1)}
\end{itemize}
Que fez economias, que tem pé de meia; abonado.
\section{Forró}
\begin{itemize}
\item {Grp. gram.:m.}
\end{itemize}
\begin{itemize}
\item {Utilização:Bras. do N}
\end{itemize}
Baile de gente ordinária.
\section{Forrobodó}
\begin{itemize}
\item {Grp. gram.:m.}
\end{itemize}
\begin{itemize}
\item {Utilização:Bras. do Rio}
\end{itemize}
Baile reles, forró.
\section{Foróia}
\begin{itemize}
\item {Grp. gram.:f.}
\end{itemize}
\begin{itemize}
\item {Utilização:Bras. do N}
\end{itemize}
Égua velha.
\section{Forsítia}
\begin{itemize}
\item {Grp. gram.:f.}
\end{itemize}
Gênero de plantas oleáceas.
\section{Forsýthia}
\begin{itemize}
\item {Grp. gram.:f.}
\end{itemize}
Gênero de plantas oleáceas.
\section{Fortaçar}
\begin{itemize}
\item {Grp. gram.:v. t.}
\end{itemize}
Alisar com fortaço.
\section{Fortaço}
\begin{itemize}
\item {Grp. gram.:m.}
\end{itemize}
\begin{itemize}
\item {Utilização:Prov.}
\end{itemize}
\begin{itemize}
\item {Utilização:alg.}
\end{itemize}
\begin{itemize}
\item {Proveniência:(De \textunderscore forte\textunderscore )}
\end{itemize}
Peça quadrangular, de madeira, com uma péga numa das faces, para alisar o revestimento de cal e areia, nas paredes.
\section{Fortalecedor}
\begin{itemize}
\item {Grp. gram.:m.  e  adj.}
\end{itemize}
O que fortalece.
\section{Fortalecer}
\begin{itemize}
\item {Grp. gram.:v. t.}
\end{itemize}
\begin{itemize}
\item {Proveniência:(Do rad. de \textunderscore forte\textunderscore )}
\end{itemize}
Tornar forte; robustecer: \textunderscore o ar do campo fortaleceu-te\textunderscore .
Dar coragem a.
Fortificar.
Corroborar.
\section{Fortalecimento}
\begin{itemize}
\item {Grp. gram.:m.}
\end{itemize}
Acto ou effeito de fortalecer.
\section{Fortalegar}
\begin{itemize}
\item {Grp. gram.:v. t.}
\end{itemize}
\begin{itemize}
\item {Utilização:Ant.}
\end{itemize}
O mesmo que \textunderscore fortalecer\textunderscore .
(Cp. \textunderscore fortalezar\textunderscore )
\section{Fortaleza}
\begin{itemize}
\item {Grp. gram.:f.}
\end{itemize}
\begin{itemize}
\item {Proveniência:(Do b. lat. \textunderscore fortalitia\textunderscore )}
\end{itemize}
Qualidade daquillo que é forte.
Solidez.
Segurança.
Fortificação.
Praça fortificada.
Castello.
Forte.
Energia; constância.
\section{Fortalezar}
\begin{itemize}
\item {Grp. gram.:v. t.}
\end{itemize}
\begin{itemize}
\item {Utilização:Des.}
\end{itemize}
\begin{itemize}
\item {Proveniência:(De \textunderscore fortaleza\textunderscore )}
\end{itemize}
O mesmo que \textunderscore fortificar\textunderscore .
\section{Fortalheirão}
\begin{itemize}
\item {Grp. gram.:adj.}
\end{itemize}
\begin{itemize}
\item {Utilização:Prov.}
\end{itemize}
\begin{itemize}
\item {Utilização:trasm.}
\end{itemize}
Diz-se do tecido ou pano encorpado e forte.
(Cp. \textunderscore forte\textunderscore )
\section{Forte}
\begin{itemize}
\item {Grp. gram.:adj.}
\end{itemize}
\begin{itemize}
\item {Grp. gram.:M.}
\end{itemize}
\begin{itemize}
\item {Proveniência:(Lat. \textunderscore fortis\textunderscore )}
\end{itemize}
Que tem fôrça; valente; robusto: \textunderscore homem forte\textunderscore .
Sólido.
Corpulento.
Enérgico.
Animoso.
Intenso: \textunderscore calor forte\textunderscore .
Poderoso.
Alcoólico: \textunderscore bebidas fortes\textunderscore .
Substâncioso.
Castello.
Fortificação.
O lado ou a feição, por onde uma pessôa ou coisa offerece mais resistência: \textunderscore o seu forte é o latim\textunderscore .
Parceiro, que, no voltarete, compra cartas em seguida ao feito.
Antiga moéda de prata, do valor de 40 reis, no tempo de D. Dinís.
\section{Forte-piano}
\begin{itemize}
\item {Grp. gram.:m.}
\end{itemize}
Designação antiga do piano, especialmente do piano de cauda.
\section{Forteza}
\begin{itemize}
\item {Grp. gram.:f.}
\end{itemize}
\begin{itemize}
\item {Utilização:Pop.}
\end{itemize}
\begin{itemize}
\item {Proveniência:(De \textunderscore forte\textunderscore )}
\end{itemize}
Força.
Valentia.
Fortaleza.
\section{Fortidão}
\begin{itemize}
\item {Grp. gram.:f.}
\end{itemize}
\begin{itemize}
\item {Proveniência:(Do lat. \textunderscore fortitudo\textunderscore )}
\end{itemize}
Qualidade daquelle ou daquillo que é forte, consistente, sólido, excitante.
\section{Fortificação}
\begin{itemize}
\item {Grp. gram.:f.}
\end{itemize}
\begin{itemize}
\item {Proveniência:(Lat. \textunderscore fortificatio\textunderscore )}
\end{itemize}
Acto de fortificar.
Fortaleza, forte, baluarte.
Arte de fortificar e defender uma praça, um acampamento, etc.
\section{Fortificador}
\begin{itemize}
\item {Grp. gram.:m.  e  adj.}
\end{itemize}
O que fortifica.
\section{Fortificante}
\begin{itemize}
\item {Grp. gram.:m.  e  adj.}
\end{itemize}
\begin{itemize}
\item {Proveniência:(Lat. \textunderscore fortificans\textunderscore )}
\end{itemize}
O mesmo que \textunderscore fortificador\textunderscore .
\section{Fortificar}
\begin{itemize}
\item {Grp. gram.:v. t.}
\end{itemize}
\begin{itemize}
\item {Proveniência:(Lat. \textunderscore fortificare\textunderscore )}
\end{itemize}
Tornar forte.
Fortalecer.
Dar condições de defesa a.
\section{Fortim}
\begin{itemize}
\item {Grp. gram.:m.}
\end{itemize}
Pequeno forte.
\section{Fortuitamente}
\begin{itemize}
\item {Grp. gram.:adv.}
\end{itemize}
De modo fortuito.
\section{Fortuito}
\begin{itemize}
\item {Grp. gram.:adj.}
\end{itemize}
\begin{itemize}
\item {Proveniência:(Lat. \textunderscore fortuitus\textunderscore )}
\end{itemize}
Que succede por acaso, inesperado.
Casual.
\section{Fortum}
\begin{itemize}
\item {Grp. gram.:m.}
\end{itemize}
(V.fartum)
\section{Fortuna}
\begin{itemize}
\item {Grp. gram.:f.}
\end{itemize}
\begin{itemize}
\item {Utilização:dispensável}
\end{itemize}
\begin{itemize}
\item {Utilização:Gal}
\end{itemize}
\begin{itemize}
\item {Proveniência:(Lat. \textunderscore fortuna\textunderscore )}
\end{itemize}
Aquillo que succede por acaso.
Sucesso imprevisto.
Eventualidade.
Sorte: \textunderscore há bôa e má fortuna\textunderscore .
Bôa sorte; ventura; felicidade.
Estado ou condição de uma pessôa.
Revés da sorte; infortúnio.
Haveres, riqueza.
\section{Fortunar}
\begin{itemize}
\item {Grp. gram.:v. t.}
\end{itemize}
(V.afortunar)
\section{Fortunear}
\begin{itemize}
\item {Grp. gram.:v. i.}
\end{itemize}
\begin{itemize}
\item {Utilização:Des.}
\end{itemize}
Negociar com fortuna.
Sêr feliz em negócios.
\section{Fortúnio}
\begin{itemize}
\item {Grp. gram.:m.}
\end{itemize}
\begin{itemize}
\item {Utilização:Ant.}
\end{itemize}
\begin{itemize}
\item {Proveniência:(Do rad. de \textunderscore fortuna\textunderscore )}
\end{itemize}
Successo próspero.
\section{Fortunosamente}
\begin{itemize}
\item {Grp. gram.:adv.}
\end{itemize}
De modo fortunoso.
\section{Fortunoso}
\begin{itemize}
\item {Grp. gram.:adj.}
\end{itemize}
\begin{itemize}
\item {Utilização:Ant.}
\end{itemize}
\begin{itemize}
\item {Proveniência:(De \textunderscore fortuna\textunderscore )}
\end{itemize}
Afortunado.
Desafortunado, perseguido da fortuna.
\section{Fósca}
\begin{itemize}
\item {Grp. gram.:f.}
\end{itemize}
\begin{itemize}
\item {Grp. gram.:Pl.}
\end{itemize}
\begin{itemize}
\item {Utilização:Ant.}
\end{itemize}
\begin{itemize}
\item {Proveniência:(Do rad. de \textunderscore fôsco\textunderscore )}
\end{itemize}
O mesmo que \textunderscore fòsquinha\textunderscore .
O mesmo ou melhor que \textunderscore miragem\textunderscore .
\section{Foscagem}
\begin{itemize}
\item {Grp. gram.:f.}
\end{itemize}
Acto de foscar.
\section{Foscar}
\begin{itemize}
\item {Grp. gram.:v. t.}
\end{itemize}
Tornar fôsco.
\section{Fôsco}
\begin{itemize}
\item {Grp. gram.:adj.}
\end{itemize}
\begin{itemize}
\item {Proveniência:(Do lat. \textunderscore fuscus\textunderscore )}
\end{itemize}
Embaciado, sem brilho; escuro; que não é translúcido: \textunderscore vidro fôsco\textunderscore .
\section{Fosfoiodoglicina}
\begin{itemize}
\item {fónica:fo-i}
\end{itemize}
\begin{itemize}
\item {Grp. gram.:f.}
\end{itemize}
Producto farmacêutico.
\section{Fosfoiodoglycina}
\begin{itemize}
\item {fónica:fo-i}
\end{itemize}
\begin{itemize}
\item {Grp. gram.:f.}
\end{itemize}
Producto pharmacêutico.
\section{Fósforo}
\begin{itemize}
\item {Grp. gram.:adj.}
\end{itemize}
\begin{itemize}
\item {Utilização:Bras. do Rio}
\end{itemize}
Metediço; intruso.
\section{Fósga}
\begin{itemize}
\item {Grp. gram.:f.}
\end{itemize}
\begin{itemize}
\item {Utilização:Prov.}
\end{itemize}
\begin{itemize}
\item {Utilização:trasm.}
\end{itemize}
Buraco em a terra.
Cova.
Espaço estreito, entre o enxergão e a parede, nas camas de bancos.
(Cp. \textunderscore fósca\textunderscore )
\section{Fósmea}
\begin{itemize}
\item {Grp. gram.:f.}
\end{itemize}
Ideia confusa ou disparatada.
Coisa, de que se não póde dar a definição.
Concepção abstrusa. Cf. Filinto, I, 18 e 169; IX, 88; IV, 18 e 169.
\section{Fósmeo}
\begin{itemize}
\item {Grp. gram.:adj.}
\end{itemize}
Disparatado.
Imperceptível, incomprehensível; indefinível. Cf. Filinto, XIII, 165.
\section{Fòsquinha}
\begin{itemize}
\item {Grp. gram.:f.}
\end{itemize}
\begin{itemize}
\item {Proveniência:(De \textunderscore fósca\textunderscore )}
\end{itemize}
Gesto, momice.
Disfarce.
Festas.
\section{Fosquista}
\begin{itemize}
\item {Grp. gram.:m.  e  f.}
\end{itemize}
Pessôa que fósca, em certas indústrias. Cf. \textunderscore Inquér. Industr.\textunderscore , 2^a p., l. III, 230 e 231.
\section{Fosresínico}
\begin{itemize}
\item {Grp. gram.:adj.}
\end{itemize}
\begin{itemize}
\item {Proveniência:(De \textunderscore fóssil\textunderscore  + \textunderscore resina\textunderscore )}
\end{itemize}
Diz-se de um ácido amorpho e pulverulento, obtido pela acção do ácido nítrico sôbre uma resina fóssil.
\section{Fossa}
\begin{itemize}
\item {Grp. gram.:f.}
\end{itemize}
\begin{itemize}
\item {Grp. gram.:Pl.}
\end{itemize}
\begin{itemize}
\item {Proveniência:(Lat. \textunderscore fossa\textunderscore )}
\end{itemize}
Cova.
Cavidade subterrânea, em que se recolhem immundícies.
Pequena cavidade natural no queixo ou na face.
Cavidades, que, no organismo animal, apresentam abertura mais larga que o fundo: \textunderscore fossas nasaes\textunderscore .
\section{Fossada}
\begin{itemize}
\item {Grp. gram.:f.}
\end{itemize}
\begin{itemize}
\item {Utilização:Ant.}
\end{itemize}
Terreno fossado.
O mesmo que \textunderscore fossado\textunderscore .
\section{Fossadeira}
\begin{itemize}
\item {Grp. gram.:f.}
\end{itemize}
\begin{itemize}
\item {Proveniência:(Do b. lat. \textunderscore fossataria\textunderscore )}
\end{itemize}
Tributo dos que acompanhavam o Rei em correrias ou fossados.
\section{Fossado}
\begin{itemize}
\item {Grp. gram.:m.}
\end{itemize}
\begin{itemize}
\item {Proveniência:(De \textunderscore fossar\textunderscore )}
\end{itemize}
Fôsso.
Investida ou correria em território inimigo.
\section{Fossador}
\begin{itemize}
\item {Grp. gram.:adj.}
\end{itemize}
Que fossa.
\section{Fossadores}
\begin{itemize}
\item {Grp. gram.:m. pl.}
\end{itemize}
Nome, que alguns naturalistas deram a uma fam. de insectos hymenópteros, com antennas cerdosas.
(Pl. de \textunderscore fossador\textunderscore )
\section{Fossa-moira}
\begin{itemize}
\item {Grp. gram.:f.}
\end{itemize}
\begin{itemize}
\item {Utilização:Prov.}
\end{itemize}
\begin{itemize}
\item {Utilização:beir.}
\end{itemize}
Cavidade subterrânea, para receber immundicies. (Colhido na Guarda)
\section{Fossão}
\begin{itemize}
\item {Grp. gram.:m.  e  adj.}
\end{itemize}
\begin{itemize}
\item {Utilização:Fig.}
\end{itemize}
\begin{itemize}
\item {Utilização:Ant.}
\end{itemize}
O que fossa muito.
Glutão.
Cavador, trabalhador do campo.
\section{Fossar}
\begin{itemize}
\item {Grp. gram.:v. t.}
\end{itemize}
\begin{itemize}
\item {Grp. gram.:V. i.}
\end{itemize}
\begin{itemize}
\item {Utilização:Fig.}
\end{itemize}
\begin{itemize}
\item {Proveniência:(Lat. \textunderscore fossare\textunderscore )}
\end{itemize}
Revolver com o focinho (a terra).
Escavar, cavar.
Empregar-se em trabalhos grosseiros.
\section{Fossário}
\begin{itemize}
\item {Grp. gram.:m.}
\end{itemize}
\begin{itemize}
\item {Utilização:Ant.}
\end{itemize}
\begin{itemize}
\item {Proveniência:(Do lat. \textunderscore fossa\textunderscore )}
\end{itemize}
Lugar, em que há fossos.
Cemitério.
Serventuário ecclesiástico, que tinha a seu cargo o enterramento dos fiéis.
\section{Fosseta}
\begin{itemize}
\item {fónica:sê}
\end{itemize}
\begin{itemize}
\item {Grp. gram.:f.}
\end{itemize}
Fossazinha, pequena fossa.
\section{Fossete}
\begin{itemize}
\item {fónica:sê}
\end{itemize}
\begin{itemize}
\item {Grp. gram.:m.}
\end{itemize}
Pequeno fôsso.
\section{Fóssil}
\begin{itemize}
\item {Grp. gram.:adj.}
\end{itemize}
\begin{itemize}
\item {Grp. gram.:M.}
\end{itemize}
\begin{itemize}
\item {Proveniência:(Lat. \textunderscore fossilis\textunderscore )}
\end{itemize}
Que se extrái da terra.
Antiquado.
Que já se não usa.
Tudo que se extrái do seio da terra.
Concha, planta ou qualquer resto de corpos orgânicos, embebido em várias matérias solúveis, apresentando ainda os vestígios da fórma primitiva, apesar da petrificação.
Resto, que foi encontrado no seio da terra, (falando-se dos corpos organizados).
\section{Fossilífero}
\begin{itemize}
\item {Grp. gram.:adj.}
\end{itemize}
\begin{itemize}
\item {Proveniência:(Do lat. \textunderscore fossilis\textunderscore  + \textunderscore ferre\textunderscore )}
\end{itemize}
Em que há fósseis animaes ou vegetaes, (falando-se de terrenos).
\section{Fossilismo}
\begin{itemize}
\item {Grp. gram.:m.}
\end{itemize}
\begin{itemize}
\item {Proveniência:(De \textunderscore fóssil\textunderscore )}
\end{itemize}
Affeição a coisas antiquadas.
\section{Fossilista}
\begin{itemize}
\item {Grp. gram.:m.}
\end{itemize}
Aquelle que gosta de coisas antiquadas.
\section{Fossilização}
\begin{itemize}
\item {Grp. gram.:f.}
\end{itemize}
\begin{itemize}
\item {Proveniência:(De \textunderscore fossilizar\textunderscore )}
\end{itemize}
Qualidade ou estado daquillo que é fóssil.
\section{Fossilizar}
\begin{itemize}
\item {Grp. gram.:v. t.}
\end{itemize}
\begin{itemize}
\item {Grp. gram.:V. p.}
\end{itemize}
\begin{itemize}
\item {Proveniência:(De \textunderscore fóssil\textunderscore )}
\end{itemize}
Tornar fóssil.
Tornar-se retrógrado, inimigo do progresso.
\section{Fossípedes}
\begin{itemize}
\item {Grp. gram.:m.}
\end{itemize}
\begin{itemize}
\item {Proveniência:(Do lat. \textunderscore fossus\textunderscore  + \textunderscore pes\textunderscore )}
\end{itemize}
Grupo de mammíferos, cujos pés são próprios para remexer a terra.
\section{Fôsso}
\begin{itemize}
\item {Grp. gram.:m.}
\end{itemize}
\begin{itemize}
\item {Utilização:Prov.}
\end{itemize}
\begin{itemize}
\item {Utilização:alent.}
\end{itemize}
\begin{itemize}
\item {Proveniência:(Lat. \textunderscore fossus\textunderscore )}
\end{itemize}
Cova, barranco.
Cavidade na terra, rodeando fortificações, entrincheiramentos, etc., para difficultar aos inimigos o ataque.
Valla, valleta ou rêgo, para conducção de águas.
Jôgo de rapazes.
\section{Fóssula}
\begin{itemize}
\item {Grp. gram.:f.}
\end{itemize}
\begin{itemize}
\item {Proveniência:(Lat. \textunderscore fossula\textunderscore )}
\end{itemize}
Depressão, pequena fossa.
\section{Fóste}
\begin{itemize}
\item {Grp. gram.:m.}
\end{itemize}
\begin{itemize}
\item {Utilização:Ant.}
\end{itemize}
O mesmo que \textunderscore fuste\textunderscore .
\section{Fota}
\begin{itemize}
\item {Grp. gram.:f.}
\end{itemize}
\begin{itemize}
\item {Proveniência:(Do ár. \textunderscore futa\textunderscore )}
\end{itemize}
Turbante moirisco.
Faixa, enrolada em volta da cabeça, imitando aquelle turbante.
\section{Fota}
\begin{itemize}
\item {Grp. gram.:f.}
\end{itemize}
(?):«\textunderscore ...e meteose el Rei em huma grande galee que fora dos mouros, que passava quarenta cavallos so fota\textunderscore ». Fern. Lopes, \textunderscore Chrón. de D. Pedro\textunderscore , c. XXIV.
Talvez o mesmo que \textunderscore froto\textunderscore ; e assim \textunderscore so fota\textunderscore  seria abaixo do nível da água.
\section{Foteado}
\begin{itemize}
\item {Grp. gram.:adj.}
\end{itemize}
\begin{itemize}
\item {Proveniência:(De \textunderscore fotear\textunderscore )}
\end{itemize}
Semelhante á fota.
\section{Fotear}
\begin{itemize}
\item {Grp. gram.:v. t.}
\end{itemize}
\begin{itemize}
\item {Proveniência:(De \textunderscore fota\textunderscore ^1)}
\end{itemize}
Cingir com fota.
Pôr fota em:«\textunderscore ...e com esses estofos foteávão as cabeças...\textunderscore »Filinto, \textunderscore D. Man.\textunderscore , I. 379.
\section{Foto}
\begin{itemize}
\item {Grp. gram.:m.}
\end{itemize}
\begin{itemize}
\item {Utilização:Ant.}
\end{itemize}
\textunderscore Estar em foto\textunderscore , estar boiante (a embarcação), livre de baixios ou cachopos e poder navegar sem perigo. Cf. Azurara, \textunderscore Chrón. do Conde D. Pedro\textunderscore , c. LVII, 398.
(Cp. \textunderscore froto\textunderscore )
\section{Fouçada}
\begin{itemize}
\item {Grp. gram.:f.}
\end{itemize}
Golpe com fouce.
\section{Fouçar}
\begin{itemize}
\item {Grp. gram.:v. t.}
\end{itemize}
Cortar com fouce.
Ceifar; segar.
\section{Fouce}
\begin{itemize}
\item {Grp. gram.:f.}
\end{itemize}
\begin{itemize}
\item {Utilização:Fig.}
\end{itemize}
\begin{itemize}
\item {Proveniência:(Do lat. \textunderscore falx\textunderscore , \textunderscore falcis\textunderscore )}
\end{itemize}
Instrumento curvo para ceifar ou segar.
Membrana, que tem a configuração do peritonéu.
Instrumento symbólico, com que é representado o tempo.
\textunderscore Fouce roçadoura\textunderscore , o mesmo que \textunderscore roçadoura\textunderscore .
\section{Foucear}
\begin{itemize}
\item {Grp. gram.:v. i.}
\end{itemize}
\begin{itemize}
\item {Utilização:Neol.}
\end{itemize}
Meter a fouce; fazer golpes com a fouce.
\section{Fouciforme}
\begin{itemize}
\item {Grp. gram.:adj.}
\end{itemize}
\begin{itemize}
\item {Proveniência:(De \textunderscore foice\textunderscore  + \textunderscore forma\textunderscore )}
\end{itemize}
Que tem fórma de fouce.
\section{Foucinha}
\begin{itemize}
\item {Grp. gram.:f.}
\end{itemize}
O mesmo que \textunderscore foucinho\textunderscore .
\section{Foucinhão}
\begin{itemize}
\item {Grp. gram.:m.}
\end{itemize}
Fouce grande, com que se corta palha em miúdos, e que, para isso, está fixa em parede.
\section{Foucinho}
\begin{itemize}
\item {Grp. gram.:m.}
\end{itemize}
Pequena fouce.
\section{Foucisca}
\begin{itemize}
\item {Grp. gram.:m.}
\end{itemize}
\begin{itemize}
\item {Utilização:Prov.}
\end{itemize}
\begin{itemize}
\item {Utilização:dur.}
\end{itemize}
Fouce pequena.
\section{Fourierismo}
\begin{itemize}
\item {fónica:fu}
\end{itemize}
\begin{itemize}
\item {Grp. gram.:m.}
\end{itemize}
Theoria de organização social, preconizada por Fourier.
\section{Fourierista}
\begin{itemize}
\item {fónica:fu}
\end{itemize}
\begin{itemize}
\item {Grp. gram.:adj.}
\end{itemize}
\begin{itemize}
\item {Grp. gram.:M.}
\end{itemize}
Relativo ao fourierismo.
Partidário do fourierismo.
\section{Fouveiro}
\begin{itemize}
\item {Grp. gram.:adj.}
\end{itemize}
\begin{itemize}
\item {Proveniência:(Do lat. hyp. \textunderscore falbarius\textunderscore )}
\end{itemize}
O mesmo que \textunderscore ruivo\textunderscore .
\section{Fovente}
\begin{itemize}
\item {Grp. gram.:adj.}
\end{itemize}
\begin{itemize}
\item {Utilização:Poét.}
\end{itemize}
\begin{itemize}
\item {Proveniência:(Lat. \textunderscore fovens\textunderscore )}
\end{itemize}
Que favorece.
Propício.
\section{Fovila}
\begin{itemize}
\item {Grp. gram.:f.}
\end{itemize}
Líquido fecundante, contido na membrana interna do pólen.
O mesmo que \textunderscore favila\textunderscore .
(Corr. de \textunderscore favilla\textunderscore , que é t. preferível. Cf. Littré)
\section{Fovilla}
\begin{itemize}
\item {Grp. gram.:f.}
\end{itemize}
Líquido fecundante, contido na membrana interna do póllen.
O mesmo que \textunderscore favilla\textunderscore .
(Corr. de \textunderscore favilla\textunderscore , que é t. preferível. Cf. Littré)
\section{Fóz}
\begin{itemize}
\item {Grp. gram.:f.}
\end{itemize}
\begin{itemize}
\item {Utilização:Des.}
\end{itemize}
\begin{itemize}
\item {Proveniência:(Do lat. \textunderscore faux\textunderscore )}
\end{itemize}
Termo de um rio, ponto em que um rio desagúa noutro ou no mar.
Passagem estreita entre montanhas.
\section{Fr.}
(Abrev. de \textunderscore frei\textunderscore )
\section{Fracalhão}
\begin{itemize}
\item {Grp. gram.:m.  e  adj.}
\end{itemize}
Aquelle que é muito fraco.
Medroso; cobarde.
\section{Fracamente}
\begin{itemize}
\item {Grp. gram.:adv.}
\end{itemize}
De modo fraco.
Froixamente.
\section{Fraca-roupa}
\begin{itemize}
\item {Grp. gram.:m.}
\end{itemize}
\begin{itemize}
\item {Utilização:Fam.}
\end{itemize}
Pelintra; maltrapilho; farroupa, farroupilha.
\section{Fracassar}
\begin{itemize}
\item {Grp. gram.:v. t.}
\end{itemize}
\begin{itemize}
\item {Proveniência:(De \textunderscore fracasso\textunderscore )}
\end{itemize}
Despedaçar com estrépito.
Arruinar; quebrar:«\textunderscore ...rompendo de robustos cossoletes uma inteira cohorte, a fracassávão\textunderscore ». \textunderscore Viriato Trág.\textunderscore , IX, 74.
\section{Fracasso}
\begin{itemize}
\item {Grp. gram.:m.}
\end{itemize}
\begin{itemize}
\item {Proveniência:(It. \textunderscore fracasso\textunderscore )}
\end{itemize}
Estrondo de coisa que se parte ou cái.
Baque.
Ruína; desgraça.
\section{Fracatear}
\begin{itemize}
\item {Grp. gram.:v. i.}
\end{itemize}
\begin{itemize}
\item {Utilização:Bras. de Piauí}
\end{itemize}
\begin{itemize}
\item {Proveniência:(De \textunderscore fraco\textunderscore )}
\end{itemize}
Enfraquecer ou cansar na corrida.
\section{Fracção}
\begin{itemize}
\item {Grp. gram.:f.}
\end{itemize}
\begin{itemize}
\item {Utilização:Arith.}
\end{itemize}
\begin{itemize}
\item {Proveniência:(Lat. \textunderscore fractio\textunderscore )}
\end{itemize}
Acto de partir, de rasgar, de dividir.
Parte de um todo.
Fórmula indicativa de uma ou mais partes da unidade.
Quebrado.
\section{Fraccionamento}
\begin{itemize}
\item {Grp. gram.:m.}
\end{itemize}
Acto ou effeito de fraccionar.
\section{Fraccionar}
\begin{itemize}
\item {Grp. gram.:v. t.}
\end{itemize}
\begin{itemize}
\item {Proveniência:(Do lat. \textunderscore fractio\textunderscore )}
\end{itemize}
Dividir; partir em fracções.
\section{Fraccionário}
\begin{itemize}
\item {Grp. gram.:adj.}
\end{itemize}
\begin{itemize}
\item {Utilização:Arith.}
\end{itemize}
\begin{itemize}
\item {Proveniência:(Do lat. \textunderscore fractio\textunderscore )}
\end{itemize}
Em que há fracção ou número quebrado.
\section{Fracciúncula}
\begin{itemize}
\item {Grp. gram.:f.}
\end{itemize}
\begin{itemize}
\item {Proveniência:(Do lat. \textunderscore fractio\textunderscore )}
\end{itemize}
Pequena fracção; bocadinho; migalha.
\section{Fraco}
\begin{itemize}
\item {Grp. gram.:adj.}
\end{itemize}
\begin{itemize}
\item {Grp. gram.:M.}
\end{itemize}
\begin{itemize}
\item {Proveniência:(Do lat. \textunderscore flaccus\textunderscore )}
\end{itemize}
Que não tem fôrça: \textunderscore homem fraco\textunderscore .
Que não é sólido.
Que não tem importância.
Pouco espêsso.
Cobarde.
Mal fortificado, mal guarnecido.
Débil, debilitado: \textunderscore depois da doença, ficou muito fraco\textunderscore .
Froixo.
Medíocre.
Que vai afroixando, que vai esmorecendo.
Que se ouve mal, que sôa debilmente: \textunderscore voz fraca\textunderscore .
Indivíduo fraco.
Lado moral, por onde uma pessôa offerece menos resistência: \textunderscore as mulheres são o seu fraco\textunderscore .
Tendência; balda; disposição.
A parte mais fraca de qualquer coisa.
Parceiro que, no voltarete, compra cartas depois do forte, por têr menos trunfos que este.
\section{Fractura}
\begin{itemize}
\item {Grp. gram.:f.}
\end{itemize}
\begin{itemize}
\item {Proveniência:(Lat. \textunderscore fractura\textunderscore )}
\end{itemize}
Acto ou effeito de fracturar.
\section{Fracturar}
\begin{itemize}
\item {Grp. gram.:v. t.}
\end{itemize}
\begin{itemize}
\item {Proveniência:(De \textunderscore fractura\textunderscore )}
\end{itemize}
Partir (um osso).
Partir osso de (perna, braço, etc.).
Quebrar com fôrça.
\section{Fradaço}
\begin{itemize}
\item {Grp. gram.:m.}
\end{itemize}
O mesmo que \textunderscore fradalhão\textunderscore . Cf. Pant. de Aveiro, \textunderscore Itiner.\textunderscore , 142, (2.^a ed.).
\section{Fradalhada}
\begin{itemize}
\item {Grp. gram.:f.}
\end{itemize}
(V.fradaria)
\section{Fradalhão}
\begin{itemize}
\item {Grp. gram.:m.}
\end{itemize}
Frade corpulento ou pouco escrupuloso.
\section{Fradaria}
\begin{itemize}
\item {Grp. gram.:f.}
\end{itemize}
\begin{itemize}
\item {Utilização:Deprec.}
\end{itemize}
Classe dos frades.
Grande número de frades.
Espírito fradesco.
\section{Fradar-se}
\begin{itemize}
\item {Grp. gram.:v. p.}
\end{itemize}
Tornar-se frade ou freira.
\section{Frade}
\begin{itemize}
\item {Grp. gram.:m.}
\end{itemize}
\begin{itemize}
\item {Utilização:Náut.}
\end{itemize}
\begin{itemize}
\item {Utilização:T. de Leiria}
\end{itemize}
\begin{itemize}
\item {Utilização:Prov.}
\end{itemize}
\begin{itemize}
\item {Utilização:Prov.}
\end{itemize}
\begin{itemize}
\item {Utilização:Prov.}
\end{itemize}
\begin{itemize}
\item {Utilização:minh.}
\end{itemize}
\begin{itemize}
\item {Proveniência:(Do lat. \textunderscore frater\textunderscore )}
\end{itemize}
Membro de communidade religiosa, sujeita a determinada regra e insulada do trato commum.
Marco de pedra, na esquina de casas, á entrada de ruas ou becos, etc.
Chupeta.
Columna, á ré do mastro grande.
Ave palmípede, (\textunderscore recurvirostra avocetta\textunderscore ).
Variedade de feijão.
Nome de um peixe.
Grão de milho, que não estoira, quando se deita no braseiro para se comer assado.
O mesmo que \textunderscore rabo-branco\textunderscore .
O mesmo que \textunderscore alfaiate\textunderscore , ave.
Espécie de coleóptero, também chamado \textunderscore fedelho\textunderscore .
Variedade de cogumelo, com uma espécie de colleira.
O mesmo que \textunderscore cogumelo\textunderscore , em geral.
\section{Fradecida}
\begin{itemize}
\item {Grp. gram.:m.}
\end{itemize}
\begin{itemize}
\item {Utilização:Neol.}
\end{itemize}
Aquelle que mata frades; assassino de frades. Cf. Camillo, \textunderscore Cav. em Ruínas\textunderscore , 83.
(Preferível seria \textunderscore fradicida\textunderscore . Cp. \textunderscore fratricida\textunderscore )
\section{Fradeiro}
\begin{itemize}
\item {Grp. gram.:adj.}
\end{itemize}
\begin{itemize}
\item {Grp. gram.:M.}
\end{itemize}
\begin{itemize}
\item {Utilização:T. de Alcobaça}
\end{itemize}
\begin{itemize}
\item {Proveniência:(De \textunderscore frade\textunderscore )}
\end{itemize}
Afeiçoado aos frades.
Gavela de palha de milho, atada pelas pontas e escarranchada em varas ou cordas, para secar.
\section{Fradejar}
\begin{itemize}
\item {Grp. gram.:v. i.}
\end{itemize}
\begin{itemize}
\item {Utilização:Des.}
\end{itemize}
Armar enredos, murmurar dos collegas, como se diz que faziam frades.
\section{Fradel}
\begin{itemize}
\item {Grp. gram.:m.}
\end{itemize}
O mesmo que \textunderscore fardel\textunderscore . Cf. \textunderscore Aulegrafia\textunderscore , 161; \textunderscore Peregrinação\textunderscore , LXXVII.
\section{Fradépio}
\begin{itemize}
\item {Grp. gram.:m.}
\end{itemize}
\begin{itemize}
\item {Utilização:Deprec.}
\end{itemize}
Frade.
Marco das ruas, com o cimo arredondado; frade de pedra.
\section{Fradesco}
\begin{itemize}
\item {fónica:dês}
\end{itemize}
\begin{itemize}
\item {Grp. gram.:adj.}
\end{itemize}
Relativo a frades, a conventos; fradeiro.
\section{Fradete}
\begin{itemize}
\item {fónica:dê}
\end{itemize}
\begin{itemize}
\item {Grp. gram.:m.}
\end{itemize}
Parte dos fechos da espingarda, dentro da charneira.
\section{Fradice}
\begin{itemize}
\item {Grp. gram.:f.}
\end{itemize}
Acção ou expressões de frade.
\section{Fradinho}
\begin{itemize}
\item {Grp. gram.:m.}
\end{itemize}
\begin{itemize}
\item {Utilização:Prov.}
\end{itemize}
\begin{itemize}
\item {Proveniência:(De \textunderscore frade\textunderscore )}
\end{itemize}
Gênero de aves palmípedes, (\textunderscore paurus caudatus\textunderscore ).
O mesmo que \textunderscore megengra\textunderscore .
Espécie de feijão.
Espécie de crustáceo, (\textunderscore scyllarus arctus\textunderscore , Lin.).
\textunderscore Fradinho da mão furada\textunderscore , entidade mýthica, duende, trasgo.
\section{Fradisco}
\begin{itemize}
\item {Grp. gram.:m.}
\end{itemize}
\begin{itemize}
\item {Utilização:Prov.}
\end{itemize}
O mesmo que \textunderscore megengra\textunderscore .
\section{Fraga}
\begin{itemize}
\item {Grp. gram.:f.}
\end{itemize}
\begin{itemize}
\item {Proveniência:(Do rad. de \textunderscore fragoso\textunderscore )}
\end{itemize}
Terreno escabroso.
Penhasco; brenha.
\section{Fragal}
\begin{itemize}
\item {Grp. gram.:adj.}
\end{itemize}
\begin{itemize}
\item {Grp. gram.:M.}
\end{itemize}
O mesmo que \textunderscore fragoso\textunderscore .
O mesmo que \textunderscore fraguedo\textunderscore .
\section{Fragalho}
\textunderscore m.\textunderscore  (e der.)
(V. \textunderscore frangalho\textunderscore , etc.)
\section{Fragalhota}
\begin{itemize}
\item {Grp. gram.:f.}
\end{itemize}
\begin{itemize}
\item {Utilização:Ant.}
\end{itemize}
(V.fardel)
\section{Fragalhotear}
\textunderscore v. i.\textunderscore  (e der.)
O mesmo que \textunderscore frangalhotear\textunderscore , etc.
\section{Fragaredo}
\begin{itemize}
\item {fónica:garê}
\end{itemize}
\begin{itemize}
\item {Grp. gram.:m.}
\end{itemize}
\begin{itemize}
\item {Utilização:Prov.}
\end{itemize}
\begin{itemize}
\item {Utilização:trasm.}
\end{itemize}
O mesmo que \textunderscore fraguedo\textunderscore .
\section{Fragária}
\begin{itemize}
\item {Grp. gram.:f.}
\end{itemize}
\begin{itemize}
\item {Proveniência:(Lat. \textunderscore fragaria\textunderscore )}
\end{itemize}
O mesmo que \textunderscore morangueiro\textunderscore ^1.
Morango bravo.
\section{Fragata}
\begin{itemize}
\item {Grp. gram.:f.}
\end{itemize}
\begin{itemize}
\item {Utilização:Pop.}
\end{itemize}
\begin{itemize}
\item {Grp. gram.:M.}
\end{itemize}
\begin{itemize}
\item {Utilização:Pop.}
\end{itemize}
\begin{itemize}
\item {Proveniência:(It. \textunderscore fregata\textunderscore )}
\end{itemize}
Navio de guerra.
Barcaça forte, especialmente destinada a serviço de descargas no Tejo.
Ave de rapina, marítima, (\textunderscore tachypetes\textunderscore ).
Mulher corpulenta.
Homem bem apessoado e activo.
\section{Fragatear}
\begin{itemize}
\item {Grp. gram.:v. i.}
\end{itemize}
\begin{itemize}
\item {Utilização:Pop.}
\end{itemize}
\begin{itemize}
\item {Proveniência:(De \textunderscore fragata\textunderscore )}
\end{itemize}
Andar na pândega; vadiar.
\section{Fragateiro}
\begin{itemize}
\item {Grp. gram.:m.}
\end{itemize}
\begin{itemize}
\item {Grp. gram.:Adj.}
\end{itemize}
\begin{itemize}
\item {Utilização:Fam.}
\end{itemize}
Tripulante de fragata no Tejo.
Embarcação de carga.
Estróina, pândego, femeeiro.
\section{Fragatim}
\begin{itemize}
\item {Grp. gram.:m.}
\end{itemize}
\begin{itemize}
\item {Utilização:Ant.}
\end{itemize}
\begin{itemize}
\item {Proveniência:(De \textunderscore fragata\textunderscore )}
\end{itemize}
O mesmo que \textunderscore bergantim\textunderscore . Cf. Fern. Oliveira, \textunderscore Arte da Guerra do Mar\textunderscore , fl. 43, v.^o
\section{Fragífero}
\begin{itemize}
\item {Grp. gram.:adj.}
\end{itemize}
Que tem fragas. Cf. \textunderscore Viriato Trág.\textunderscore , II, 105.
\section{Frágil}
\begin{itemize}
\item {Grp. gram.:adj.}
\end{itemize}
\begin{itemize}
\item {Utilização:Fig.}
\end{itemize}
\begin{itemize}
\item {Proveniência:(Lat. \textunderscore fragilis\textunderscore )}
\end{itemize}
Que se parte facilmente; quebradiço.
Fraco.
Que tem pouca duração.
Que está sujeito a erros ou a culpas: \textunderscore a humanidade é fragil\textunderscore .
\section{Fragilidade}
\begin{itemize}
\item {Grp. gram.:f.}
\end{itemize}
\begin{itemize}
\item {Proveniência:(Lat. \textunderscore fragilitas\textunderscore )}
\end{itemize}
Qualidade daquelle ou de aquillo que é frágil: \textunderscore a fragilidade da mulher\textunderscore .
\section{Fragilmente}
\begin{itemize}
\item {Grp. gram.:adj.}
\end{itemize}
\begin{itemize}
\item {Proveniência:(De \textunderscore frágil\textunderscore )}
\end{itemize}
Com fragilidade.
\section{Fragmentação}
\begin{itemize}
\item {Grp. gram.:f.}
\end{itemize}
Acto ou effeito de fragmentar.
\section{Fragmentar}
\begin{itemize}
\item {Grp. gram.:v. t.}
\end{itemize}
Reduzir a fragmentos; partir em pedaços.
\section{Fragmentário}
\begin{itemize}
\item {Grp. gram.:adj.}
\end{itemize}
Relativo a fragmento.
Que se encontra em fragmentos.
\section{Fragmentista}
\begin{itemize}
\item {Grp. gram.:m.}
\end{itemize}
\begin{itemize}
\item {Proveniência:(De \textunderscore fragmento\textunderscore )}
\end{itemize}
Aquelle que fragmenta, ou que reúne fragmentos artisticos, literários, etc.
\section{Fragmento}
\begin{itemize}
\item {Grp. gram.:m.}
\end{itemize}
\begin{itemize}
\item {Proveniência:(Lat. \textunderscore fragmentum\textunderscore )}
\end{itemize}
Cada uma das partes, em que se divide um todo.
Fracção.
Pedaço; migalha.
\section{Frago}
\begin{itemize}
\item {Grp. gram.:m.}
\end{itemize}
\begin{itemize}
\item {Proveniência:(Do lat. \textunderscore fragare\textunderscore ?)}
\end{itemize}
Indícios de passagem de caça viva.
Excremento de animaes silvestres.
\section{Frágoa}
\begin{itemize}
\item {Grp. gram.:f.}
\end{itemize}
\begin{itemize}
\item {Utilização:Fig.}
\end{itemize}
\begin{itemize}
\item {Utilização:Ant.}
\end{itemize}
\begin{itemize}
\item {Proveniência:(Do lat. \textunderscore fabrica\textunderscore ?)}
\end{itemize}
Forja; fornalha.
Ardor, calor intenso.
Amargura.
O mesmo que \textunderscore fábrica\textunderscore . Cf. \textunderscore Port. Mon. Hist.\textunderscore , \textunderscore Script.\textunderscore , 246.
\section{Fragoar}
\begin{itemize}
\item {Grp. gram.:v. t.}
\end{itemize}
\begin{itemize}
\item {Utilização:Fig.}
\end{itemize}
\begin{itemize}
\item {Proveniência:(De \textunderscore frágua\textunderscore )}
\end{itemize}
O mesmo que \textunderscore forjar\textunderscore .
Amargurar.
\section{Fragoedo}
\begin{itemize}
\item {Grp. gram.:m.}
\end{itemize}
É palavra mal escrita, e dá lugar a pronúncia errada. Cf. Camillo, \textunderscore Perfil do Marquês\textunderscore , 63.(V.fraguedo)
\section{Fragoeiro}
\begin{itemize}
\item {Grp. gram.:m.}
\end{itemize}
\begin{itemize}
\item {Utilização:Prov.}
\end{itemize}
\begin{itemize}
\item {Utilização:beir.}
\end{itemize}
\begin{itemize}
\item {Grp. gram.:Adj.}
\end{itemize}
\begin{itemize}
\item {Utilização:Fig.}
\end{itemize}
\begin{itemize}
\item {Proveniência:(De \textunderscore frágua\textunderscore )}
\end{itemize}
Pau tôsco e comprido.
Estadulho.
Pau, em que se encaba o vassoiro, com que se varrem as cinzas e brasas do forno, para neste se deitar o pão que se vai cozer.
Ardente. Cf. Filinto, \textunderscore D. Manuel\textunderscore , I, 118.
Fogoso:«\textunderscore ...hum pirata... homem fragueiro e temido...\textunderscore »Barros, \textunderscore Dêc.\textunderscore 
\section{Fragor}
\begin{itemize}
\item {Grp. gram.:m.}
\end{itemize}
\begin{itemize}
\item {Proveniência:(Lat. \textunderscore fragor\textunderscore )}
\end{itemize}
Estampido; ruído forte.
\section{Fragorar}
\begin{itemize}
\item {Grp. gram.:v. i.}
\end{itemize}
\begin{itemize}
\item {Utilização:Neol.}
\end{itemize}
Produzir fragor, estrondear.
\section{Fragoroso}
\begin{itemize}
\item {Grp. gram.:adj.}
\end{itemize}
Que produz fragor.
\section{Fragosão}
\begin{itemize}
\item {Grp. gram.:m.}
\end{itemize}
\begin{itemize}
\item {Proveniência:(De \textunderscore fraga\textunderscore ? Ou corr. de \textunderscore folgosão\textunderscore ?)}
\end{itemize}
Casta de uva alentejana.
\section{Fragosidade}
\begin{itemize}
\item {Grp. gram.:f.}
\end{itemize}
Qualidade daquillo que é fragoso.
\section{Fragoso}
\begin{itemize}
\item {Grp. gram.:adj.}
\end{itemize}
\begin{itemize}
\item {Proveniência:(Lat. \textunderscore fragosus\textunderscore )}
\end{itemize}
Áspero; escabroso; penhascoso.
Difficilmente accessível.
\section{Fragrância}
\begin{itemize}
\item {Grp. gram.:f.}
\end{itemize}
\begin{itemize}
\item {Proveniência:(Lat. \textunderscore fragrantia\textunderscore )}
\end{itemize}
Qualidade daquillo que é fragrante.
Aroma, perfume.
\section{Fragrante}
\begin{itemize}
\item {Grp. gram.:adj.}
\end{itemize}
\begin{itemize}
\item {Proveniência:(Lat. \textunderscore fragrans\textunderscore )}
\end{itemize}
Odorífero.
Que exhala cheiro agradável; aromático.
\section{Frágua}
\begin{itemize}
\item {Grp. gram.:f.}
\end{itemize}
\begin{itemize}
\item {Utilização:Fig.}
\end{itemize}
\begin{itemize}
\item {Utilização:Ant.}
\end{itemize}
\begin{itemize}
\item {Proveniência:(Do lat. \textunderscore fabrica\textunderscore ?)}
\end{itemize}
Forja; fornalha.
Ardor, calor intenso.
Amargura.
O mesmo que \textunderscore fábrica\textunderscore . Cf. \textunderscore Port. Mon. Hist.\textunderscore , \textunderscore Script.\textunderscore , 246.
\section{Fraguar}
\begin{itemize}
\item {Grp. gram.:v. t.}
\end{itemize}
\begin{itemize}
\item {Utilização:Fig.}
\end{itemize}
\begin{itemize}
\item {Proveniência:(De \textunderscore frágua\textunderscore )}
\end{itemize}
O mesmo que \textunderscore forjar\textunderscore .
Amargurar.
\section{Fraguear}
\begin{itemize}
\item {Grp. gram.:v. i.}
\end{itemize}
\begin{itemize}
\item {Utilização:T. de Pare -de-Coira}
\end{itemize}
\begin{itemize}
\item {Utilização:des.}
\end{itemize}
O mesmo que \textunderscore defecar\textunderscore .
\section{Fraguedo}
\begin{itemize}
\item {fónica:guê}
\end{itemize}
\begin{itemize}
\item {Grp. gram.:m.}
\end{itemize}
Série de fragas; fraga extensa.
\section{Fragueirice}
\begin{itemize}
\item {Grp. gram.:f.}
\end{itemize}
\begin{itemize}
\item {Proveniência:(De \textunderscore fragueiro\textunderscore ^1)}
\end{itemize}
Acção de quem é \textunderscore fragueiro\textunderscore ^1; rudeza.
\section{Fragueiril}
\begin{itemize}
\item {Grp. gram.:adj.}
\end{itemize}
O mesmo que \textunderscore fragueiro\textunderscore ^1:«\textunderscore ...fragueiril monteador de veados\textunderscore ». Camillo, \textunderscore Regicida\textunderscore , 243.
\section{Fragueiro}
\begin{itemize}
\item {Grp. gram.:adj.}
\end{itemize}
\begin{itemize}
\item {Grp. gram.:M.}
\end{itemize}
\begin{itemize}
\item {Proveniência:(De \textunderscore fraga\textunderscore )}
\end{itemize}
Que tem vida trabalhosa.
Que anda por serras e fragas, moirejando.
Independente.
Rude, áspero, agreste.
Lenhador.
Aquelle que vive trabalhosamente por serras e fragas.
\section{Fragueiro}
\begin{itemize}
\item {Grp. gram.:m.}
\end{itemize}
\begin{itemize}
\item {Utilização:Ant.}
\end{itemize}
\begin{itemize}
\item {Proveniência:(Do rad. de \textunderscore fragata\textunderscore )}
\end{itemize}
Constructor de fragatas.
\section{Fragueiro}
\begin{itemize}
\item {fónica:gu-ei}
\end{itemize}
\begin{itemize}
\item {Grp. gram.:m.}
\end{itemize}
\begin{itemize}
\item {Utilização:Prov.}
\end{itemize}
\begin{itemize}
\item {Utilização:beir.}
\end{itemize}
\begin{itemize}
\item {Grp. gram.:Adj.}
\end{itemize}
\begin{itemize}
\item {Utilização:Fig.}
\end{itemize}
\begin{itemize}
\item {Proveniência:(De \textunderscore frágua\textunderscore )}
\end{itemize}
Pau tôsco e comprido.
Estadulho.
Pau, em que se encaba o vassoiro, com que se varrem as cinzas e brasas do forno, para neste se deitar o pão que se vai cozer.
Ardente. Cf. Filinto, \textunderscore D. Manuel\textunderscore , I, 118.
Fogoso:«\textunderscore ...hum pirata... homem fragueiro e temido...\textunderscore »Barros, \textunderscore Dêc.\textunderscore 
\section{Fraguice}
\begin{itemize}
\item {Grp. gram.:f.}
\end{itemize}
\begin{itemize}
\item {Utilização:Des.}
\end{itemize}
\begin{itemize}
\item {Proveniência:(De \textunderscore fraga\textunderscore )}
\end{itemize}
O mesmo que \textunderscore fraguedo\textunderscore . Cf. \textunderscore Viriato Trág.\textunderscore , VII, 94.
\section{Frágula}
\begin{itemize}
\item {Grp. gram.:f.}
\end{itemize}
Dardo, muito aguçado, usado antigamente na península hispânica.
(B. lat. \textunderscore fragula\textunderscore )
\section{Fragulho}
\begin{itemize}
\item {Grp. gram.:m.}
\end{itemize}
\begin{itemize}
\item {Utilização:Açor}
\end{itemize}
O mesmo que \textunderscore couve\textunderscore .
\section{Fragura}
\begin{itemize}
\item {Grp. gram.:f.}
\end{itemize}
O mesmo que \textunderscore fragosidade\textunderscore .
\section{Frainel}
\begin{itemize}
\item {Grp. gram.:m.}
\end{itemize}
\begin{itemize}
\item {Utilização:Náut.}
\end{itemize}
Botão, que, nos mastaréus, nas vêrgas de joanete, etc., se toma por pouco tempo, ordinariamente com fio de carrêta. Cf. Campos, \textunderscore Voc. Marujo\textunderscore .
\section{Fraire}
\begin{itemize}
\item {Grp. gram.:m.}
\end{itemize}
\begin{itemize}
\item {Utilização:Ant.}
\end{itemize}
O mesmo que \textunderscore freire\textunderscore .
\section{Fraita}
\begin{itemize}
\item {Grp. gram.:f.}
\end{itemize}
\begin{itemize}
\item {Utilização:Prov.}
\end{itemize}
\begin{itemize}
\item {Utilização:minh.}
\end{itemize}
O mesmo que \textunderscore frauta\textunderscore  ou \textunderscore flauta\textunderscore .
\section{Fralda}
\begin{itemize}
\item {Grp. gram.:f.}
\end{itemize}
\begin{itemize}
\item {Utilização:Ext.}
\end{itemize}
\begin{itemize}
\item {Proveniência:(Do b. lat. \textunderscore falda\textunderscore ?)}
\end{itemize}
Parte inferior da camisa.
Cueiro.
Abas.
Parte inferior, sopé, (de serra, monte, etc.).
\textunderscore Fralda do mar\textunderscore , praia.
\section{Fraldamento}
\begin{itemize}
\item {Grp. gram.:m.}
\end{itemize}
\begin{itemize}
\item {Utilização:Ant.}
\end{itemize}
\begin{itemize}
\item {Proveniência:(De \textunderscore fralda\textunderscore )}
\end{itemize}
Roupa branca de mulher? Cf. S. Viterbo, \textunderscore Artes e Artistas\textunderscore , 62.
\section{Fraldão}
\begin{itemize}
\item {Grp. gram.:m.}
\end{itemize}
\begin{itemize}
\item {Proveniência:(De \textunderscore fralda\textunderscore )}
\end{itemize}
Parte inferior da armadura.
\section{Fraldar}
\begin{itemize}
\item {Grp. gram.:v. t.}
\end{itemize}
\begin{itemize}
\item {Grp. gram.:V. p.}
\end{itemize}
Pôr fraldas a.
Vestir fraldão.
\section{Fraldear}
\begin{itemize}
\item {Grp. gram.:v. t.}
\end{itemize}
Caminhar pela fralda ou falda do (monte).
\section{Fraldeiro}
\begin{itemize}
\item {Grp. gram.:m.}
\end{itemize}
O mesmo que \textunderscore fraldiqueiro\textunderscore .
\section{Fraldejar}
\begin{itemize}
\item {Grp. gram.:v. i.}
\end{itemize}
Mostrar a fralda, andando.
Andar pelas fraldas da serra.
\section{Fraldelhim}
\begin{itemize}
\item {Grp. gram.:m.}
\end{itemize}
\begin{itemize}
\item {Utilização:Ant.}
\end{itemize}
O mesmo que \textunderscore fraldelim\textunderscore .
\section{Fraldelim}
\begin{itemize}
\item {Grp. gram.:m.}
\end{itemize}
\begin{itemize}
\item {Proveniência:(Do rad. de \textunderscore fralda\textunderscore )}
\end{itemize}
Brial; saia interior, aberta adiante.
Anágua; saiote.
\section{Fraldicurto}
\begin{itemize}
\item {Grp. gram.:adj.}
\end{itemize}
\begin{itemize}
\item {Proveniência:(De \textunderscore fralda\textunderscore  + \textunderscore curto\textunderscore )}
\end{itemize}
Que tem fraldas curtas.
\section{Fraldilha}
\begin{itemize}
\item {Grp. gram.:f.}
\end{itemize}
\begin{itemize}
\item {Utilização:Ant.}
\end{itemize}
\begin{itemize}
\item {Proveniência:(De \textunderscore fralda\textunderscore )}
\end{itemize}
Avental de coiro, de que usam os ferreiros.
Avental, que usavam os porta-machados.
Avental bordado, para senhoras.
Espécie de avental, usado por certo corpo de bèsteiros:«\textunderscore capitão dos besteiros de fraldilha\textunderscore ». Filinto, \textunderscore D. Man.\textunderscore , III, 21.
\section{Fraldiqueira}
\begin{itemize}
\item {Grp. gram.:f.}
\end{itemize}
\begin{itemize}
\item {Utilização:Deprec.}
\end{itemize}
O mesmo que \textunderscore algibeira\textunderscore .
(Por \textunderscore faldriqueira\textunderscore , do cast. \textunderscore faltriquera\textunderscore )
\section{Fraldiqueiro}
\begin{itemize}
\item {Grp. gram.:adj.}
\end{itemize}
Relativo a fraldas.
Effeminado, mulherengo.
E diz-se do cão, acostumado ao regaço das mulheres e ao aconchego e calor das saias.
(Por \textunderscore fraldeiro\textunderscore , de \textunderscore fralda\textunderscore )
\section{Fraldisqueira}
\begin{itemize}
\item {Grp. gram.:f.}
\end{itemize}
\begin{itemize}
\item {Utilização:T. do Fundão}
\end{itemize}
Rapariga metediça, que anda por todas as casas e gosta de ouvir e contar.
(Cp. \textunderscore fraldiqueiro\textunderscore )
\section{Fraldoso}
\begin{itemize}
\item {Grp. gram.:adj.}
\end{itemize}
\begin{itemize}
\item {Utilização:Fig.}
\end{itemize}
Que tem fraldas.
Palavroso, prolixo: \textunderscore estilo fraldoso\textunderscore .
\section{Frama}
\begin{itemize}
\item {Grp. gram.:f.}
\end{itemize}
\begin{itemize}
\item {Utilização:Ant.}
\end{itemize}
O mesmo que \textunderscore presunto\textunderscore .
\section{Framalha}
\begin{itemize}
\item {Grp. gram.:f.}
\end{itemize}
\begin{itemize}
\item {Utilização:Pop.}
\end{itemize}
O mesmo que \textunderscore faramalha\textunderscore .
\section{Framboêsa}
\begin{itemize}
\item {Grp. gram.:f.}
\end{itemize}
\begin{itemize}
\item {Proveniência:(Do fr. \textunderscore framboise\textunderscore )}
\end{itemize}
Fruto odorífero do framboeseiro.
\section{Framboeseiro}
\begin{itemize}
\item {fónica:bo-e}
\end{itemize}
\begin{itemize}
\item {Grp. gram.:m.}
\end{itemize}
\begin{itemize}
\item {Proveniência:(De \textunderscore framboêsa\textunderscore )}
\end{itemize}
Arbusto espinhoso, da fam. das rosáceas.
\section{Framboesia}
\begin{itemize}
\item {fónica:bo-e}
\end{itemize}
\begin{itemize}
\item {Grp. gram.:f.}
\end{itemize}
\begin{itemize}
\item {Proveniência:(De \textunderscore framboêsa\textunderscore )}
\end{itemize}
Moléstia, caracterizada por tumores, semelhantes, na fórma, a cogumelos ou a framboêsas.
\section{Frâmea}
\begin{itemize}
\item {Grp. gram.:f.}
\end{itemize}
\begin{itemize}
\item {Proveniência:(Lat. \textunderscore framea\textunderscore )}
\end{itemize}
Espécie de lança, entre os antigos Francos.
\section{Framengo}
\begin{itemize}
\item {Grp. gram.:m.}
\end{itemize}
\begin{itemize}
\item {Utilização:Ant.}
\end{itemize}
\begin{itemize}
\item {Grp. gram.:Adj.}
\end{itemize}
O mesmo que \textunderscore flamengo\textunderscore ^1.
Branco? Miudinho?:«\textunderscore dentinhos framengos...\textunderscore »Camões, \textunderscore Amphytriões\textunderscore .
\section{França}
\begin{itemize}
\item {Grp. gram.:m.}
\end{itemize}
\begin{itemize}
\item {Utilização:P. us.}
\end{itemize}
\begin{itemize}
\item {Grp. gram.:F.}
\end{itemize}
\begin{itemize}
\item {Grp. gram.:Adj.}
\end{itemize}
Rapaz casquilho.
Franchinote.
Mulher sécia, delambida e namorada.
Garrido. Cf. S. Monteiro, \textunderscore Auto dos Esquecidos\textunderscore , 50.
\section{Francalete}
\begin{itemize}
\item {fónica:lê}
\end{itemize}
\begin{itemize}
\item {Grp. gram.:m.}
\end{itemize}
Correia afivelada.
(Cast. \textunderscore francalete\textunderscore )
\section{Franca-marneca}
\begin{itemize}
\item {Grp. gram.:f.}
\end{itemize}
O mesmo que \textunderscore rabilha\textunderscore .
\section{Francamente}
\begin{itemize}
\item {Grp. gram.:adv.}
\end{itemize}
De modo franco.
Com franqueza.
\section{Franças}
\begin{itemize}
\item {Grp. gram.:f. pl.}
\end{itemize}
Os ramos mais altos das árvores; rama de arvoredo.
(Alter. de \textunderscore fronças\textunderscore , do lat. \textunderscore frondea\textunderscore ?)
\section{Franca-tripa}
\begin{itemize}
\item {Grp. gram.:f.}
\end{itemize}
\begin{itemize}
\item {Utilização:Bras}
\end{itemize}
Boneco, que se move por cordas de tripa, por arames, etc.; fantoche.
\section{Francear}
\begin{itemize}
\item {Grp. gram.:v. t.}
\end{itemize}
\begin{itemize}
\item {Grp. gram.:V. i.}
\end{itemize}
Cortar ou aparar as franças de.
Andar por cima das franças.
\section{Francela}
\begin{itemize}
\item {Grp. gram.:f.}
\end{itemize}
\begin{itemize}
\item {Utilização:Prov.}
\end{itemize}
\begin{itemize}
\item {Utilização:beir.}
\end{itemize}
\begin{itemize}
\item {Proveniência:(Do rad. de \textunderscore francelho\textunderscore )}
\end{itemize}
O mesmo que \textunderscore queijeira\textunderscore .
\section{Francelho}
\begin{itemize}
\item {fónica:cê}
\end{itemize}
\begin{itemize}
\item {Grp. gram.:m.}
\end{itemize}
\begin{itemize}
\item {Proveniência:(De \textunderscore francês\textunderscore )}
\end{itemize}
Barrileira ou mesa, que tem em volta um sulco, em que se junta, e donde cai para um balde, o soro da coalhada, nas queijarias.
Tagarela.
Indivíduo, que se exaggera no apêgo a francesismos e coisas francesas.
Espécie de falcão, (\textunderscore falco tinnunculus\textunderscore ).
O mesmo que \textunderscore gavião\textunderscore , (\textunderscore accipiter nisus\textunderscore ).
\section{Francella}
\begin{itemize}
\item {Grp. gram.:f.}
\end{itemize}
\begin{itemize}
\item {Utilização:Prov.}
\end{itemize}
\begin{itemize}
\item {Utilização:beir.}
\end{itemize}
\begin{itemize}
\item {Proveniência:(Do rad. de \textunderscore francelho\textunderscore )}
\end{itemize}
O mesmo que \textunderscore queijeira\textunderscore .
\section{Francês}
\begin{itemize}
\item {Grp. gram.:m.}
\end{itemize}
\begin{itemize}
\item {Grp. gram.:Adj.}
\end{itemize}
\begin{itemize}
\item {Utilização:Fig.}
\end{itemize}
\begin{itemize}
\item {Grp. gram.:Loc. adv.}
\end{itemize}
\begin{itemize}
\item {Utilização:fam.}
\end{itemize}
\begin{itemize}
\item {Proveniência:(Do b. lat. \textunderscore francensis\textunderscore )}
\end{itemize}
Indivíduo natural da França.
A língua dessa nação.
Relativo a França: \textunderscore o povo francês\textunderscore .
Que é fingidamente delicado.
Hypócrita; falso.
\textunderscore Á francesa\textunderscore , á grande, com esplendor.
\section{Francesear}
\begin{itemize}
\item {Grp. gram.:v. i.}
\end{itemize}
Falar francês, sabendo-o mal. Cf. Filinto, XII, 218; III, 212; \textunderscore Hyssope\textunderscore , 131.
\section{Francesia}
\begin{itemize}
\item {Grp. gram.:f.}
\end{itemize}
\begin{itemize}
\item {Proveniência:(De \textunderscore francês\textunderscore )}
\end{itemize}
Imitação dos costumes ou da linguagem dos Franceses; francesismo.
\section{Francesismo}
\begin{itemize}
\item {Grp. gram.:m.}
\end{itemize}
\begin{itemize}
\item {Utilização:Fig.}
\end{itemize}
\begin{itemize}
\item {Proveniência:(De \textunderscore francês\textunderscore )}
\end{itemize}
Palavra ou phrase, de sabor ou índole francesa.
Gallicismo.
Imitação affectada de costumes ou coisas francesas.
Delicadeza apparente, fingimento.
\section{Francesista}
\begin{itemize}
\item {Grp. gram.:m.  e  adj.}
\end{itemize}
O que usa francesismos ou gosta de francesear. Cf. Filinto, I, 60.
\section{Francêsmente}
\begin{itemize}
\item {Grp. gram.:adv.}
\end{itemize}
Á maneira dos Franceses.
\section{Franchado}
\begin{itemize}
\item {Grp. gram.:adj.}
\end{itemize}
\begin{itemize}
\item {Utilização:Heráld.}
\end{itemize}
\begin{itemize}
\item {Proveniência:(Do rad. do lat. \textunderscore fractus\textunderscore )}
\end{itemize}
Dividido diagonalmente em duas partes iguaes, (tratando-se de brasões).
\section{Franchão}
\begin{itemize}
\item {Grp. gram.:adj.}
\end{itemize}
\begin{itemize}
\item {Utilização:Des.}
\end{itemize}
Feio, mal encarado; repugnante.
\section{Franchinote}
\begin{itemize}
\item {Grp. gram.:m.}
\end{itemize}
\begin{itemize}
\item {Utilização:Deprec.}
\end{itemize}
Rapazelho.
Peralta.
Janota presumido.
O mesmo que \textunderscore francês\textunderscore . Cf. F. Manuel, \textunderscore Hospital das Letras\textunderscore .
(Corr. de \textunderscore franganote\textunderscore ?)
\section{Franchinótico}
\begin{itemize}
\item {Grp. gram.:adj.}
\end{itemize}
Relativo a franchinote:«\textunderscore ...pela parte franchinótica do seu carácter.\textunderscore »Garrett, \textunderscore Viagens\textunderscore , I, 123.
\section{Frância}
\begin{itemize}
\item {Grp. gram.:f.}
\end{itemize}
\begin{itemize}
\item {Proveniência:(De \textunderscore França\textunderscore , n. p.)}
\end{itemize}
Conto decamerónico, derivado dos antigos contos franceses, chamados \textunderscore fabliaux\textunderscore .
\section{Francica}
\begin{itemize}
\item {Grp. gram.:f.}
\end{itemize}
\begin{itemize}
\item {Proveniência:(Lat. \textunderscore francisca\textunderscore )}
\end{itemize}
Machado de dois gumes, usado pelos Francos. Cf. Filinto, XIV, 228.
\section{Frâncico}
\begin{itemize}
\item {Grp. gram.:adj.}
\end{itemize}
Relativo aos povos francos. Cf. Filinto, XIV, 217.
\section{Francisca}
\begin{itemize}
\item {Grp. gram.:f.}
\end{itemize}
O mesmo ou melhor que \textunderscore francica\textunderscore .
\section{Franciscana}
\begin{itemize}
\item {Grp. gram.:f.}
\end{itemize}
\begin{itemize}
\item {Proveniência:(De \textunderscore franciscano\textunderscore )}
\end{itemize}
A Ordem religiosa de San-Francisco. Cf. \textunderscore Hyssope\textunderscore , 131.
\section{Franciscanada}
\begin{itemize}
\item {Grp. gram.:f.}
\end{itemize}
\begin{itemize}
\item {Utilização:Pop.}
\end{itemize}
\begin{itemize}
\item {Proveniência:(De \textunderscore franciscano\textunderscore )}
\end{itemize}
Folia; patuscada.
\section{Franciscano}
\begin{itemize}
\item {Grp. gram.:adj.}
\end{itemize}
\begin{itemize}
\item {Utilização:Pop.}
\end{itemize}
\begin{itemize}
\item {Grp. gram.:M.}
\end{itemize}
\begin{itemize}
\item {Proveniência:(De \textunderscore Francisco\textunderscore , n. p.)}
\end{itemize}
Pertencente á Ordem de San-Francisco.
Diz-se da grande pobreza ou miséria.
Frade da Ordem de San-Francisco ou dos frades menores.
\section{Francisco}
\begin{itemize}
\item {Grp. gram.:adj.}
\end{itemize}
\begin{itemize}
\item {Utilização:Ant.}
\end{itemize}
\begin{itemize}
\item {Proveniência:(De \textunderscore França\textunderscore )}
\end{itemize}
Relativo á França ou aos Franceses.
\section{Francismo}
\begin{itemize}
\item {Grp. gram.:m.}
\end{itemize}
\begin{itemize}
\item {Utilização:Des.}
\end{itemize}
O mesmo que \textunderscore francesismo\textunderscore .
\section{Francisquinho}
\begin{itemize}
\item {Grp. gram.:m.}
\end{itemize}
\begin{itemize}
\item {Utilização:Gír.}
\end{itemize}
Copo de vinho.
\section{Franciú}
\begin{itemize}
\item {Grp. gram.:m.}
\end{itemize}
\begin{itemize}
\item {Utilização:Pop.}
\end{itemize}
O mesmo que \textunderscore francês\textunderscore , língua.
\section{Franc-mação}
\begin{itemize}
\item {Grp. gram.:m.}
\end{itemize}
(V.franco-mação)
\section{Franco}
\begin{itemize}
\item {Grp. gram.:adj.}
\end{itemize}
\begin{itemize}
\item {Grp. gram.:M.}
\end{itemize}
\begin{itemize}
\item {Grp. gram.:Pl.}
\end{itemize}
\begin{itemize}
\item {Proveniência:(Lat. \textunderscore francus\textunderscore )}
\end{itemize}
Desimpedido, desembaraçado: \textunderscore pôrto franco\textunderscore .
Sincero, que diz o que pensa.
Generoso; em que há franqueza, espontaneidade: \textunderscore procedimento franco\textunderscore .
Relativo aos Francos.
Moéda francesa de prata, do valor aproximado de 180 reis.
Unidade monetária em França.
Confederação de povos germânicos, que se espalharam pelas vizinhanças do Rheno.
\section{Franco...}
\begin{itemize}
\item {Grp. gram.:pref.}
\end{itemize}
\begin{itemize}
\item {Proveniência:(Do rad. de \textunderscore francês\textunderscore )}
\end{itemize}
(que entra na composição dos nomes, para designar associação ou mistura de Franceses com outro povo, ou alguma coisa commum á França e a outro povo)
\section{Frâncoa}
\begin{itemize}
\item {Grp. gram.:f.}
\end{itemize}
\begin{itemize}
\item {Proveniência:(De \textunderscore Franco\textunderscore , n. p.)}
\end{itemize}
Gênero de plantas herbáceas, que crescem no Chile.
\section{Francoáceas}
\begin{itemize}
\item {Grp. gram.:f. pl.}
\end{itemize}
Família de plantas, que têm por typo a frâncoa.
\section{Franco-alemão}
\begin{itemize}
\item {Grp. gram.:adj.}
\end{itemize}
Relativo á França e á Alemanha.
\section{Franco-árabe}
\begin{itemize}
\item {Grp. gram.:adj.}
\end{itemize}
Relativo a Franceses e Árabes.
\section{Franco-argentino}
\begin{itemize}
\item {Grp. gram.:adj.}
\end{itemize}
Relativo á França e á República Argentina.
\section{Franco-atirador}
\begin{itemize}
\item {Grp. gram.:m.}
\end{itemize}
Soldado, que fazia parte de uns corpos ligeiros, organizados durante as guerras da revolução francesa de 1789.
Cada um dos soldados franceses, que, no cêrco de Sebastopol, atiravam de emboscada sôbre o inimigo.
Membro de certas associações, que têm escola de tiro.
\section{Franco-austríaco}
\begin{itemize}
\item {Grp. gram.:adj.}
\end{itemize}
Relativo a Franceses e Austríacos.
\section{Franco-belga}
\begin{itemize}
\item {Grp. gram.:adj.}
\end{itemize}
Relativo a Franceses e Belgas.
\section{Franco-brasileiro}
\begin{itemize}
\item {Grp. gram.:adj.}
\end{itemize}
Relativo a Franceses e Brasileiros.
\section{Franco-búlgaro}
\begin{itemize}
\item {Grp. gram.:adj.}
\end{itemize}
Relativo a Franceses e Búlgaros.
\section{Franco-celta}
\begin{itemize}
\item {Grp. gram.:adj.}
\end{itemize}
Relativo a Franceses e Celtas.
\section{Franco-celtico}
\begin{itemize}
\item {Grp. gram.:adj.}
\end{itemize}
Relativo a Franceses e Celtas.
\section{Franco-chileno}
\begin{itemize}
\item {Grp. gram.:adj.}
\end{itemize}
Relativo a Franceses e Chilenos.
\section{Franco-china}
\begin{itemize}
\item {Grp. gram.:adj.}
\end{itemize}
Relativo a Franceses e Chineses.
\section{Franco-chinês}
\begin{itemize}
\item {Grp. gram.:adj.}
\end{itemize}
Relativo a Franceses e Chineses.
\section{Franco-de-oiro}
\begin{itemize}
\item {Grp. gram.:m.}
\end{itemize}
Antiga moéda portuguesa, que valia 120 reis.
\section{Franco-dinamarquês}
\begin{itemize}
\item {Grp. gram.:adj.}
\end{itemize}
Relativo a Franceses e Dinamarqueses.
\section{Franco-espanhol}
\begin{itemize}
\item {Grp. gram.:adj.}
\end{itemize}
Relativo a Franceses e a Espanhóis.
\section{Franco-gállico}
\begin{itemize}
\item {Grp. gram.:adj.}
\end{itemize}
Relativo aos Francos e gauleses ou antes Gallos \textunderscore ou Gállios\textunderscore .
\section{Franco-gaulês}
\begin{itemize}
\item {Grp. gram.:adj.}
\end{itemize}
Relativo aos Francos e gauleses ou antes Gallos \textunderscore ou Gállios\textunderscore .
\section{Franco-germânico}
\begin{itemize}
\item {Grp. gram.:adj.}
\end{itemize}
(V.franco-alemão)
\section{Franco-grego}
\begin{itemize}
\item {Grp. gram.:adj.}
\end{itemize}
Relativo a Franceses e Gregos.
\section{Franco-inglês}
\begin{itemize}
\item {Grp. gram.:adj.}
\end{itemize}
Relativo a Franceses e Ingleses.
\section{Franco-italiano}
\begin{itemize}
\item {Grp. gram.:adj.}
\end{itemize}
Relativo a Franceses e Italianos.
\section{Franco-japonês}
\begin{itemize}
\item {Grp. gram.:adj.}
\end{itemize}
Relativo a Franceses e Japoneses.
\section{Francolim}
\begin{itemize}
\item {Grp. gram.:m.}
\end{itemize}
Espécie de perdiz, (\textunderscore perdrix francolinus\textunderscore ).
\section{Franco-mação}
\begin{itemize}
\item {Grp. gram.:m.}
\end{itemize}
Membro da franco-maçonaria.
\section{Franco-maçonaria}
\begin{itemize}
\item {Grp. gram.:f.}
\end{itemize}
Sociedade secreta, que tem por fim principal o desenvolvimento do princípio da fraternidade e da philanthropia.
Sociedade de pedreiros-livres.
Maçonaria.
\section{Franco-mexicano}
\begin{itemize}
\item {Grp. gram.:adj.}
\end{itemize}
Relativo a Franceses e Mexicanos.
\section{Franco-persa}
\begin{itemize}
\item {Grp. gram.:adj.}
\end{itemize}
Relativo a Franceses e Persas.
\section{Franco-português}
\begin{itemize}
\item {Grp. gram.:adj.}
\end{itemize}
Relativo a Franceses e Portugueses.
\section{Franco-prussiano}
\begin{itemize}
\item {Grp. gram.:adj.}
\end{itemize}
Relativo a Franceses e Prussianos.
\section{Franco-russo}
\begin{itemize}
\item {Grp. gram.:adj.}
\end{itemize}
Relativo a Franceses e Russos.
\section{Franco-sérvio}
\begin{itemize}
\item {Grp. gram.:adj.}
\end{itemize}
Relativo a Franceses e Sérvios.
\section{Franco-turco}
\begin{itemize}
\item {Grp. gram.:adj.}
\end{itemize}
Relativo a Franceses e Turcos.
\section{Frandulagem}
\begin{itemize}
\item {Grp. gram.:f.}
\end{itemize}
Súcia de maltrapilhos.
Farraparia.
Bugigangas.
(Corr. de \textunderscore farandulagem\textunderscore ?)
\section{Franduleiro}
\begin{itemize}
\item {Grp. gram.:adj.}
\end{itemize}
\begin{itemize}
\item {Utilização:Des.}
\end{itemize}
\begin{itemize}
\item {Proveniência:(Do rad. de \textunderscore Frandes\textunderscore , = \textunderscore Flandres\textunderscore , n. p.)}
\end{itemize}
Estrangeiro.
\section{Franduno}
\begin{itemize}
\item {Grp. gram.:adj.}
\end{itemize}
\begin{itemize}
\item {Proveniência:(Do rad. de \textunderscore Frandes\textunderscore , = \textunderscore Flandres\textunderscore , n. p.)}
\end{itemize}
Que esteve em Flandres.
Que tem costumes estrangeirados.
Presumido, affectado.
\section{Franga}
\begin{itemize}
\item {Grp. gram.:f.}
\end{itemize}
\begin{itemize}
\item {Proveniência:(Do b. lat. \textunderscore frangana\textunderscore )}
\end{itemize}
Gallinha, que ainda não põe ovos.
\section{Frangaínha}
\begin{itemize}
\item {Grp. gram.:f.}
\end{itemize}
\begin{itemize}
\item {Proveniência:(Do lat. \textunderscore fringuilla\textunderscore )}
\end{itemize}
Pintaínha; franga.
\section{Frangaínho}
\begin{itemize}
\item {Grp. gram.:m.}
\end{itemize}
Pequeno frango.
(Cp. \textunderscore frangaínha\textunderscore )
\section{Frangaiola}
\begin{itemize}
\item {Grp. gram.:f.}
\end{itemize}
\begin{itemize}
\item {Utilização:T. de Turquel}
\end{itemize}
Mulher moça; rapariga.
\section{Frangalhada}
\begin{itemize}
\item {Grp. gram.:f.}
\end{itemize}
\begin{itemize}
\item {Utilização:Fam.}
\end{itemize}
Guisado de frangos.
\section{Frangalheiro}
\begin{itemize}
\item {Grp. gram.:m.  e  adj.}
\end{itemize}
Indivíduo, vestido de frangalhos; esfarrapado.
\section{Frangalho}
\begin{itemize}
\item {Grp. gram.:m.}
\end{itemize}
\begin{itemize}
\item {Proveniência:(Do rad. do lat. \textunderscore frangere\textunderscore )}
\end{itemize}
Farrapo.
\section{Frangalhona}
\begin{itemize}
\item {Grp. gram.:m.  e  adj.}
\end{itemize}
\begin{itemize}
\item {Proveniência:(De \textunderscore frangalho\textunderscore )}
\end{itemize}
Mulher desmazelada no traje.
Esfarrapada.
\section{Frangalhote}
\begin{itemize}
\item {Grp. gram.:m.}
\end{itemize}
\begin{itemize}
\item {Utilização:Pop.}
\end{itemize}
Frango já crescido.
Rapazola.
Rapaz estróina e femeeiro.
\section{Frangalhotear}
\begin{itemize}
\item {Grp. gram.:v. i.}
\end{itemize}
\begin{itemize}
\item {Proveniência:(De \textunderscore frangalhote\textunderscore )}
\end{itemize}
Sêr frascário, femeeiro.
Estroinar; folgar. Cf. Filinto, VIII, 230.
\section{Frangalhoteiro}
\begin{itemize}
\item {Grp. gram.:adj.}
\end{itemize}
Que frangalhoteia; que é doido por mulheres.
(Cp. \textunderscore frangalhotear\textunderscore )
\section{Frangam}
\begin{itemize}
\item {Grp. gram.:m.}
\end{itemize}
O mesmo que \textunderscore frango\textunderscore .
\section{Frângão}
\begin{itemize}
\item {Grp. gram.:m.}
\end{itemize}
O mesmo que \textunderscore frango\textunderscore .
\section{Franganada}
\begin{itemize}
\item {Grp. gram.:f.}
\end{itemize}
\begin{itemize}
\item {Proveniência:(De \textunderscore frangam\textunderscore )}
\end{itemize}
Bando de frangos.
\section{Franganito}
\begin{itemize}
\item {Grp. gram.:m.}
\end{itemize}
\begin{itemize}
\item {Utilização:Fig.}
\end{itemize}
\begin{itemize}
\item {Proveniência:(De \textunderscore frangam\textunderscore )}
\end{itemize}
Frangaínho.
Rapazinho vaidoso, empertigado.
\section{Franganota}
\begin{itemize}
\item {Grp. gram.:f.}
\end{itemize}
Rapariga casadoira.
\section{Franganote}
\begin{itemize}
\item {Grp. gram.:m.}
\end{itemize}
O mesmo que \textunderscore franganito\textunderscore .
\section{Frangelha}
\begin{itemize}
\item {fónica:gê}
\end{itemize}
\begin{itemize}
\item {Grp. gram.:f.}
\end{itemize}
\begin{itemize}
\item {Proveniência:(Do rad. do lat. \textunderscore frangere\textunderscore )}
\end{itemize}
Cinto ou arco, com que se aperta a massa do queijo.
Cincho.
\section{Franger}
\begin{itemize}
\item {Grp. gram.:v. t.}
\end{itemize}
\begin{itemize}
\item {Utilização:P. us.}
\end{itemize}
\begin{itemize}
\item {Proveniência:(Do lat. \textunderscore frangere\textunderscore )}
\end{itemize}
O mesmo que \textunderscore frangir\textunderscore .
Quebrar.
\section{Franges}
\begin{itemize}
\item {Grp. gram.:m. pl.}
\end{itemize}
\begin{itemize}
\item {Utilização:Ext.}
\end{itemize}
\begin{itemize}
\item {Proveniência:(T. ár. mas de or. europeia)}
\end{itemize}
Nome, com que no Oriente se designavam os franceses.
Nome, que, na Índia oriental, também se dava aos Portugueses. Cf. Pant. de Aveiro, \textunderscore Itiner.\textunderscore , 52, (2.^a ed.).
Estrangeiros.
\section{Frangibilidade}
\begin{itemize}
\item {Grp. gram.:f.}
\end{itemize}
Qualidade daquillo que é frangível.
\section{Frangícia}
\begin{itemize}
\item {Grp. gram.:f.}
\end{itemize}
\begin{itemize}
\item {Utilização:Med.}
\end{itemize}
\begin{itemize}
\item {Utilização:Ant.}
\end{itemize}
\begin{itemize}
\item {Proveniência:(Do lat. \textunderscore frangere\textunderscore )}
\end{itemize}
Quebradura, hérnia.
\section{Frangipana}
\begin{itemize}
\item {Grp. gram.:f.}
\end{itemize}
\begin{itemize}
\item {Utilização:Des.}
\end{itemize}
\begin{itemize}
\item {Proveniência:(De \textunderscore Frangipani\textunderscore , n. p.)}
\end{itemize}
Espécie de perfume.
Pastel ou bolo perfumado.
\section{Frangipano}
\begin{itemize}
\item {Grp. gram.:adj.}
\end{itemize}
Perfumado com frangipana.
\section{Frangir}
\begin{itemize}
\item {Grp. gram.:v. t.}
\end{itemize}
\begin{itemize}
\item {Utilização:Des.}
\end{itemize}
\begin{itemize}
\item {Proveniência:(Do lat. \textunderscore frangere\textunderscore )}
\end{itemize}
O mesmo que \textunderscore franzir\textunderscore .
\section{Frangível}
\begin{itemize}
\item {Grp. gram.:adj.}
\end{itemize}
\begin{itemize}
\item {Proveniência:(Do lat. \textunderscore frangere\textunderscore )}
\end{itemize}
Frágil.
\section{Frango}
\begin{itemize}
\item {Grp. gram.:m.}
\end{itemize}
\begin{itemize}
\item {Proveniência:(Do b. lat. \textunderscore franganus\textunderscore )}
\end{itemize}
O filho da gallinha, já crescido, mas antes de sêr gallo.
\section{Frango-de-água}
\begin{itemize}
\item {Grp. gram.:m.}
\end{itemize}
O mesmo que \textunderscore furamato\textunderscore .
\section{Frangolho}
\begin{itemize}
\item {fónica:gô}
\end{itemize}
\begin{itemize}
\item {Grp. gram.:m.}
\end{itemize}
Trigo mal pisado ou mal partido, com que se fazem papas.
(Cast. \textunderscore frangollo\textunderscore )
\section{Frangote}
\begin{itemize}
\item {Grp. gram.:m.}
\end{itemize}
O mesmo que \textunderscore franganote\textunderscore .
\section{Frangues}
\begin{itemize}
\item {Grp. gram.:m. pl.}
\end{itemize}
\begin{itemize}
\item {Utilização:Ant.}
\end{itemize}
O mesmo ou melhor que \textunderscore franges\textunderscore . Cf. Filinto, \textunderscore D. Man.\textunderscore , II, 41; Tenreiro, XX.
(Cp. \textunderscore franques\textunderscore )
\section{Frângula}
\begin{itemize}
\item {Grp. gram.:f.}
\end{itemize}
Espécie de abrunheiro, cuja madeira serve para a fabricação de pólvora.
Árvore rhamnácea, (\textunderscore rhamnus frangula\textunderscore , Lin.), de frutos comestíveis.
\section{Franguláceas}
\begin{itemize}
\item {Grp. gram.:f. pl.}
\end{itemize}
Classe de plantas dicotyledóneas, que têm por typo a frângula; o mesmo que \textunderscore rhamnáceas\textunderscore .
\section{Frangúlico}
\begin{itemize}
\item {Grp. gram.:adj.}
\end{itemize}
\begin{itemize}
\item {Proveniência:(De \textunderscore frângula\textunderscore )}
\end{itemize}
Diz-se de um ácido, que é o mesmo que a frangulina.
\section{Frangulina}
\begin{itemize}
\item {Grp. gram.:f.}
\end{itemize}
Substância amarga, extrahida da casca da frângula.
\section{Franja}
\begin{itemize}
\item {Grp. gram.:f.}
\end{itemize}
\begin{itemize}
\item {Utilização:Bot.}
\end{itemize}
\begin{itemize}
\item {Grp. gram.:Pl.}
\end{itemize}
\begin{itemize}
\item {Proveniência:(Fr. \textunderscore frange\textunderscore )}
\end{itemize}
Cadilhos de linho, algodão, seda, oiro, etc., para enfeitar ou guarnecer alguma peça de estôfo.
Faixa ou banda, com esses cadilhos pendentes.
Membrana dentada, por baixo do opérculo de alguns musgos.
Espécie de penteado, em que o cabello descái liso e curto sôbre a testa.
\section{Franjamento}
\begin{itemize}
\item {Grp. gram.:m.}
\end{itemize}
Acto ou effeito de franjar.
\section{Franjar}
\begin{itemize}
\item {Grp. gram.:v. t.}
\end{itemize}
\begin{itemize}
\item {Utilização:Fig.}
\end{itemize}
\begin{itemize}
\item {Proveniência:(De \textunderscore franja\textunderscore )}
\end{itemize}
Enfeitar ou guarnecer de franjas.
Rendilhar.
Desfiar.
Tornar garrido, pretensioso.
\section{Franjeado}
\begin{itemize}
\item {Grp. gram.:m.}
\end{itemize}
\begin{itemize}
\item {Utilização:Prov.}
\end{itemize}
\begin{itemize}
\item {Utilização:trasm.}
\end{itemize}
\begin{itemize}
\item {Proveniência:(De \textunderscore franja\textunderscore )}
\end{itemize}
Bazófia; presumpção infundada.
\section{Franjeira}
\begin{itemize}
\item {Grp. gram.:f.}
\end{itemize}
Mulher, que trabalha em franjas.
\section{Franjosca}
\begin{itemize}
\item {Utilização:Prov.}
\end{itemize}
\begin{itemize}
\item {Utilização:trasm.}
\end{itemize}
\begin{itemize}
\item {Proveniência:(De \textunderscore franja\textunderscore )}
\end{itemize}
Mulher impudica e provocante.
Concubina.
\section{Frankênia}
\begin{itemize}
\item {Grp. gram.:f.}
\end{itemize}
\begin{itemize}
\item {Proveniência:(De \textunderscore Frankenius\textunderscore , n. p.)}
\end{itemize}
Gênero de plantas, que crescem á beiramar, em todas as regiões temperadas.
\section{Frankeniáceas}
\begin{itemize}
\item {Grp. gram.:f. pl.}
\end{itemize}
Família de plantas dicotyledóneas, que têm por typo a frankênia.
\section{Frankliniano}
\begin{itemize}
\item {Grp. gram.:adj.}
\end{itemize}
Relativo a Benjamim Franklin, ás suas obras ou aos seus serviços humanitários.
\section{Franklinite}
\begin{itemize}
\item {Grp. gram.:f.}
\end{itemize}
\begin{itemize}
\item {Proveniência:(De \textunderscore Franklin\textunderscore , n. p.)}
\end{itemize}
Mineral, espécie de espinella.
\section{Franklinização}
\begin{itemize}
\item {Grp. gram.:f.}
\end{itemize}
\begin{itemize}
\item {Proveniência:(De \textunderscore Franklin\textunderscore , n. p.)}
\end{itemize}
Applicação médica da electricidade estática.
\section{Franquear}
\begin{itemize}
\item {Grp. gram.:v. t.}
\end{itemize}
Tornar franco.
Isentar de imposto.
Tornar fácil, desimpedido.
Pagar o transporte de: \textunderscore franquear cartas\textunderscore .
Conceder.
Tornar patente: \textunderscore franquear a sua casa\textunderscore .
\section{Franqueável}
\begin{itemize}
\item {Grp. gram.:adj.}
\end{itemize}
Que se póde franquear. Cf. Júlio Dinís, \textunderscore Morgadinha\textunderscore , 10.
\section{Franqueiro}
\begin{itemize}
\item {Grp. gram.:m.}
\end{itemize}
\begin{itemize}
\item {Utilização:Bras. do S}
\end{itemize}
Raça de bois corpulentos.
\section{Franquelete}
\begin{itemize}
\item {fónica:lê}
\end{itemize}
\begin{itemize}
\item {Grp. gram.:m.}
\end{itemize}
O mesmo que \textunderscore francalete\textunderscore . Cf. Chagas, \textunderscore Côrte de D. João V\textunderscore , 21.
\section{Franques}
\begin{itemize}
\item {Grp. gram.:m. pl.}
\end{itemize}
\begin{itemize}
\item {Utilização:Ant.}
\end{itemize}
Nome, que se dava aos christãos na Índia portuguesa.
(Cp. \textunderscore franges\textunderscore  e \textunderscore frangues\textunderscore )
\section{Franqueza}
\begin{itemize}
\item {Grp. gram.:f.}
\end{itemize}
Qualidade daquelle ou de aquillo que é franco.
Effeito de franquear.
Isenção, privilégio, regalia.
\section{Franquia}
\begin{itemize}
\item {Grp. gram.:f.}
\end{itemize}
\begin{itemize}
\item {Proveniência:(De \textunderscore franquiar\textunderscore , por \textunderscore franquear\textunderscore )}
\end{itemize}
Acto ou effeito de franquear.
Franqueza.
Refúgio.
\section{Franquir}
\begin{itemize}
\item {Grp. gram.:v. t.}
\end{itemize}
\begin{itemize}
\item {Utilização:Ant.}
\end{itemize}
\begin{itemize}
\item {Proveniência:(De \textunderscore franco\textunderscore ?)}
\end{itemize}
Arrotear, desbravar, (terrenos).
\section{Franzéria}
\begin{itemize}
\item {Grp. gram.:f.}
\end{itemize}
\begin{itemize}
\item {Proveniência:(De \textunderscore Franzer\textunderscore , n. p.)}
\end{itemize}
Gênero de plantas compostas.
\section{Franzido}
\begin{itemize}
\item {Grp. gram.:m.}
\end{itemize}
\begin{itemize}
\item {Proveniência:(De \textunderscore franzir\textunderscore )}
\end{itemize}
Coisa franzida.
\section{Franzimento}
\begin{itemize}
\item {Grp. gram.:m.}
\end{itemize}
Acto ou effeito de franzir.
\section{Franzino}
\begin{itemize}
\item {Grp. gram.:adj.}
\end{itemize}
\begin{itemize}
\item {Proveniência:(Do rad. de \textunderscore franzir\textunderscore )}
\end{itemize}
Delgado; débil: \textunderscore rapaz franzino\textunderscore .
Delicado de fórmas.
Tênue, pouco intenso.
Que tem pouca consistência: \textunderscore tecido franzino\textunderscore .
\section{Franzir}
\begin{itemize}
\item {Grp. gram.:v. t.}
\end{itemize}
Enrugar; preguear.
(Corr. de \textunderscore frangir\textunderscore ?)
\section{Fraque}
\begin{itemize}
\item {Grp. gram.:m.}
\end{itemize}
\begin{itemize}
\item {Proveniência:(Do al. \textunderscore frack\textunderscore )}
\end{itemize}
Casaco curto, cujas abas se afastam, do peito para baixo.
\section{Fraquear}
\begin{itemize}
\item {Grp. gram.:v. i.}
\end{itemize}
O mesmo que \textunderscore fraquejar\textunderscore .
Dobrar os joêlhos, caíndo:«\textunderscore fraqueou á terra, e só firme nos joelhos...\textunderscore »Filinto, \textunderscore D. Man.\textunderscore , II, 148.
\section{Fraqueira}
\begin{itemize}
\item {Grp. gram.:f.}
\end{itemize}
\begin{itemize}
\item {Utilização:Fam.}
\end{itemize}
\begin{itemize}
\item {Proveniência:(De \textunderscore fraco\textunderscore )}
\end{itemize}
Fraqueza, debilidade.
\section{Fraquejar}
\begin{itemize}
\item {Grp. gram.:v. i.}
\end{itemize}
Tornar-se fraco.
Perder o vigor, a coragem.
Afroixar.
\section{Fraquentar}
\begin{itemize}
\item {Grp. gram.:v. t.}
\end{itemize}
\begin{itemize}
\item {Utilização:Ant.}
\end{itemize}
O mesmo que \textunderscore enfraquecer\textunderscore .
\section{Fraqueza}
\begin{itemize}
\item {Grp. gram.:f.}
\end{itemize}
Qualidade daquelle ou daquillo que é fraco.
Compleição fraca: falta do robustez.
Debilidade.
Desânimo.
Timidez.
Defeito, imperfeição.
Falta de firmeza, de insistência.
Tendência para ceder a suggestões ou a imposições.
Fragilidade.
O lado fraco de um carácter ou de um objecto.
\section{Frasca}
\begin{itemize}
\item {Grp. gram.:f.}
\end{itemize}
\begin{itemize}
\item {Utilização:Des.}
\end{itemize}
\begin{itemize}
\item {Utilização:Prov.}
\end{itemize}
\begin{itemize}
\item {Utilização:alent.}
\end{itemize}
\begin{itemize}
\item {Proveniência:(Do rad. de \textunderscore frasco\textunderscore )}
\end{itemize}
Loiça de cozinha; baixela.
Provisões.
Faina de fazer bolos ou doces.
\section{Frascagem}
\begin{itemize}
\item {Grp. gram.:f.}
\end{itemize}
Frasca.
O mesmo que \textunderscore frascal\textunderscore .
\section{Frascal}
\begin{itemize}
\item {Grp. gram.:m.}
\end{itemize}
\begin{itemize}
\item {Utilização:Prov.}
\end{itemize}
\begin{itemize}
\item {Utilização:alent.}
\end{itemize}
Meda de palha, de fórma quadrangular.
Casa da eira.
Lugar, onde se guarda lenha sêca.
\section{Frascaria}
\begin{itemize}
\item {Grp. gram.:f.}
\end{itemize}
\begin{itemize}
\item {Utilização:Fig.}
\end{itemize}
Porção de frascos.
Qualidade de quem é frascário.
\section{Frascário}
\begin{itemize}
\item {Grp. gram.:adj.}
\end{itemize}
\begin{itemize}
\item {Utilização:Pop.}
\end{itemize}
\begin{itemize}
\item {Proveniência:(De \textunderscore frasca\textunderscore )}
\end{itemize}
Extravagante; libidinoso; dissoluto.
\section{Frasco}
\begin{itemize}
\item {Grp. gram.:m.}
\end{itemize}
\begin{itemize}
\item {Utilização:Bras. do Amaz}
\end{itemize}
\begin{itemize}
\item {Utilização:Des.}
\end{itemize}
\begin{itemize}
\item {Proveniência:(Do lat. \textunderscore vasculum\textunderscore ?)}
\end{itemize}
Vaso, ordinariamente de boca estreita, para líquidos principalmente.
Medida, correspondente a 2 litros.
O mesmo que \textunderscore penico\textunderscore .
\section{Frasqueira}
\begin{itemize}
\item {Grp. gram.:f.}
\end{itemize}
\begin{itemize}
\item {Utilização:Bras. do Amaz}
\end{itemize}
Caixa ou lugar, em que se juntam frascos.
Loja ou compartimento, em que se guardam bons vinhos engarrafados.
Vinhos engarrafados e guardados, para uso do lavrador.
Medida de capacidade ou garrafão de 24 litros.
\section{Frasqueiro}
\begin{itemize}
\item {Grp. gram.:adj.}
\end{itemize}
\begin{itemize}
\item {Utilização:Pop.}
\end{itemize}
O mesmo que \textunderscore frascário\textunderscore .
Pouco decente, muito decotado, (falando-se de vestuário).
\section{Frasquejar}
\begin{itemize}
\item {Grp. gram.:v. i.}
\end{itemize}
\begin{itemize}
\item {Utilização:Prov.}
\end{itemize}
\begin{itemize}
\item {Utilização:alent.}
\end{itemize}
\begin{itemize}
\item {Proveniência:(De \textunderscore frasca\textunderscore )}
\end{itemize}
Fazer bolos ou doces.
\section{Frasqueta}
\begin{itemize}
\item {fónica:quê}
\end{itemize}
\begin{itemize}
\item {Grp. gram.:f.}
\end{itemize}
Quadro de ferro com gonzos, com que se segura a fôlha do papel que se há de tirar do prelo.
\section{Frataxo}
\begin{itemize}
\item {Grp. gram.:m.}
\end{itemize}
\begin{itemize}
\item {Utilização:fam.}
\end{itemize}
\begin{itemize}
\item {Utilização:Ant.}
\end{itemize}
Frade de pouco mérito; fradépio.
\section{Fraterna}
\begin{itemize}
\item {Grp. gram.:f.}
\end{itemize}
\begin{itemize}
\item {Proveniência:(De \textunderscore fraterno\textunderscore )}
\end{itemize}
Reprehensão amigável.
\section{Fraternal}
\begin{itemize}
\item {Grp. gram.:adj.}
\end{itemize}
Fraterno, affectuoso.
\section{Fraternalmente}
\begin{itemize}
\item {Grp. gram.:adv.}
\end{itemize}
De modo fraternal.
\section{Fraternidade}
\begin{itemize}
\item {Grp. gram.:f.}
\end{itemize}
\begin{itemize}
\item {Proveniência:(Lat. \textunderscore fraternitas\textunderscore )}
\end{itemize}
Parentesco entre irmãos.
Amor ao próximo.
União ou convivência, como de irmãos.
Amizade; harmonia.
\section{Fraternização}
\begin{itemize}
\item {Grp. gram.:f.}
\end{itemize}
Acto ou effeito de fraternizar.
\section{Fraternizar}
\begin{itemize}
\item {Grp. gram.:v. t.}
\end{itemize}
\begin{itemize}
\item {Grp. gram.:V. i.}
\end{itemize}
\begin{itemize}
\item {Proveniência:(De \textunderscore fraterno\textunderscore )}
\end{itemize}
Unir com amizade estreita, intima.
Unir-se estreitamente, como entre irmãos.
Fazer alliança, travar amizade, sympathizar.
Associar-se.
Fazer causa commum: \textunderscore a tropa fraternizou com o povo\textunderscore .
Commungar nas mesmas ideias.
\section{Fraterno}
\begin{itemize}
\item {Grp. gram.:adj.}
\end{itemize}
\begin{itemize}
\item {Proveniência:(Lat. \textunderscore fraternus\textunderscore )}
\end{itemize}
Relativo a irmãos.
Próprio de irmãos; affectuoso.
\section{Fraticellos}
\begin{itemize}
\item {Grp. gram.:m. pl.}
\end{itemize}
Frades franciscanos, que se tornaram herejes, negando a utilidade dos sacramentos.
(B. lat. \textunderscore fraticelli\textunderscore , do rad. do lat. \textunderscore frater\textunderscore )
\section{Fraticelos}
\begin{itemize}
\item {Grp. gram.:m. pl.}
\end{itemize}
Frades franciscanos, que se tornaram herejes, negando a utilidade dos sacramentos.
(B. lat. \textunderscore fraticelli\textunderscore , do rad. do lat. \textunderscore frater\textunderscore )
\section{Fratricellos}
\begin{itemize}
\item {Grp. gram.:m. pl.}
\end{itemize}
O mesmo que \textunderscore fraticellos\textunderscore .
\section{Fratricelos}
\begin{itemize}
\item {Grp. gram.:m. pl.}
\end{itemize}
O mesmo que \textunderscore fraticelos\textunderscore .
\section{Fratricida}
\begin{itemize}
\item {Grp. gram.:m.}
\end{itemize}
\begin{itemize}
\item {Grp. gram.:Adj.}
\end{itemize}
\begin{itemize}
\item {Utilização:Ext.}
\end{itemize}
\begin{itemize}
\item {Proveniência:(Lat. \textunderscore fratricida\textunderscore )}
\end{itemize}
Assassino de irmão ou de irman.
Que concorre para a morte ou ruína de irmãos ou de pessôas que, como irmãos, se devem estimar.
Relativo a guerras civis.
\section{Fratricídio}
\begin{itemize}
\item {Grp. gram.:m.}
\end{itemize}
\begin{itemize}
\item {Utilização:Ext.}
\end{itemize}
\begin{itemize}
\item {Proveniência:(Lat. \textunderscore fratricidium\textunderscore )}
\end{itemize}
Crime de quem mata irmão ou irman.
Guerra civil.
\section{Fratrissa}
\begin{itemize}
\item {Grp. gram.:f.}
\end{itemize}
\begin{itemize}
\item {Proveniência:(Lat. \textunderscore fratrissa\textunderscore )}
\end{itemize}
Espécie de freira, que pertencia á Ordem de Malta, mas que não vivia em convento.
\section{Fraudação}
\begin{itemize}
\item {Grp. gram.:f.}
\end{itemize}
\begin{itemize}
\item {Proveniência:(Lat. \textunderscore fraudatio\textunderscore )}
\end{itemize}
Acto de fraudar; burla.
Má fé.
\section{Fraudador}
\begin{itemize}
\item {Grp. gram.:m.  e  adj.}
\end{itemize}
\begin{itemize}
\item {Proveniência:(Lat. \textunderscore fraudator\textunderscore )}
\end{itemize}
O que frauda.
\section{Fraudar}
\begin{itemize}
\item {Grp. gram.:v. t.}
\end{itemize}
\begin{itemize}
\item {Proveniência:(Lat. \textunderscore fraudare\textunderscore )}
\end{itemize}
Commeter fraude contra.
Enganar.
Privar.
Despojar fraudulentamente.
Frustrar.
\section{Fraudatório}
\begin{itemize}
\item {Grp. gram.:adj.}
\end{itemize}
\begin{itemize}
\item {Proveniência:(Lat. \textunderscore fraudatorius\textunderscore )}
\end{itemize}
Relativo a fraude; em que há fraude.
\section{Fraudável}
\begin{itemize}
\item {Grp. gram.:adj.}
\end{itemize}
Susceptível de fraude.
\section{Fraude}
\begin{itemize}
\item {Grp. gram.:f.}
\end{itemize}
\begin{itemize}
\item {Proveniência:(Lat. \textunderscore fraus\textunderscore , \textunderscore fraudis\textunderscore )}
\end{itemize}
Dolo; engano; burla.
Contrabando.
\section{Fraudento}
\begin{itemize}
\item {Grp. gram.:adj.}
\end{itemize}
\begin{itemize}
\item {Proveniência:(De \textunderscore fraude\textunderscore )}
\end{itemize}
O mesmo que \textunderscore fraudulento\textunderscore .
\section{Fraudulência}
\begin{itemize}
\item {Grp. gram.:f.}
\end{itemize}
\begin{itemize}
\item {Proveniência:(Lat. \textunderscore fraudulentia\textunderscore )}
\end{itemize}
O mesmo que \textunderscore fraude\textunderscore .
Astúcia.
\section{Fraudulento}
\begin{itemize}
\item {Grp. gram.:adj.}
\end{itemize}
\begin{itemize}
\item {Proveniência:(Lat. \textunderscore fraudulentus\textunderscore )}
\end{itemize}
Doloso; impostor.
Fallaz.
Em que há fraude.
Que é propenso á fraude.
\section{Fraudulosamente}
\begin{itemize}
\item {Grp. gram.:adv.}
\end{itemize}
De modo frauduloso.
\section{Frauduloso}
\begin{itemize}
\item {Grp. gram.:adj.}
\end{itemize}
\begin{itemize}
\item {Proveniência:(Lat. \textunderscore fraudulosus\textunderscore )}
\end{itemize}
O mesmo que \textunderscore fraudulento\textunderscore .
\section{Frauta}
\begin{itemize}
\item {Grp. gram.:f.}
\end{itemize}
\begin{itemize}
\item {Grp. gram.:Pl.}
\end{itemize}
\begin{itemize}
\item {Utilização:Fam.}
\end{itemize}
\begin{itemize}
\item {Grp. gram.:M.}
\end{itemize}
\begin{itemize}
\item {Utilização:Serralh.}
\end{itemize}
\begin{itemize}
\item {Proveniência:(Do lat. \textunderscore flata\textunderscore )}
\end{itemize}
O mesmo ou melhor que \textunderscore flauta\textunderscore .
Instrumento músico de sopro, cylíndrico e sem palheta.
Pífaro.
Utensílio de ferreiro, mais ou menos boleado, sôbre o qual se encurvam e se alisam certas peças.
Pernas delgadas.
Aquelle, que, num concêrto, toca flauta.
Peça para alisar o ferro.
\section{Frautado}
\begin{itemize}
\item {Grp. gram.:adj.}
\end{itemize}
\begin{itemize}
\item {Utilização:Ant.}
\end{itemize}
Alegre?:«\textunderscore estays muyto frautado e eu nada pera graças\textunderscore ». \textunderscore Aulegrafia\textunderscore , 27.
\section{Frauteira}
\textunderscore fem.\textunderscore  de \textunderscore frauteiro\textunderscore .
\section{Frauteiro}
\begin{itemize}
\item {Grp. gram.:m.}
\end{itemize}
\begin{itemize}
\item {Proveniência:(De \textunderscore frauta\textunderscore )}
\end{itemize}
O mesmo que \textunderscore flautista\textunderscore .
\section{Fraxineáceas}
\begin{itemize}
\item {fónica:csi}
\end{itemize}
\begin{itemize}
\item {Grp. gram.:f. pl.}
\end{itemize}
O mesmo ou melhor que \textunderscore fraxíneas\textunderscore .
\section{Fraxíneas}
\begin{itemize}
\item {fónica:csi}
\end{itemize}
\begin{itemize}
\item {Grp. gram.:f. pl.}
\end{itemize}
\begin{itemize}
\item {Proveniência:(De \textunderscore fraxíneo\textunderscore )}
\end{itemize}
Família de plantas, que têm por typo o freixo.
\section{Fraxinela}
\begin{itemize}
\item {fónica:csi}
\end{itemize}
\begin{itemize}
\item {Grp. gram.:f.}
\end{itemize}
\begin{itemize}
\item {Proveniência:(Do lat. \textunderscore fraxinus\textunderscore )}
\end{itemize}
Espécie de dictamno.
\section{Fraxíneo}
\begin{itemize}
\item {fónica:csi}
\end{itemize}
\begin{itemize}
\item {Grp. gram.:adj.}
\end{itemize}
\begin{itemize}
\item {Proveniência:(Lat. \textunderscore fraxineus\textunderscore )}
\end{itemize}
Que é da natureza do freixo.
Semelhante ao freixo.
\section{Fraxinina}
\begin{itemize}
\item {fónica:csi}
\end{itemize}
\begin{itemize}
\item {Grp. gram.:f.}
\end{itemize}
\begin{itemize}
\item {Proveniência:(Do lat. \textunderscore fraxinus\textunderscore )}
\end{itemize}
Alcali, que se extrái da casca do freixo.
\section{Frazangue}
\begin{itemize}
\item {Grp. gram.:m.}
\end{itemize}
Antiga medida itinerária da Pérsia, correspondente a uma légua portuguesa. Cf. Tenreiro, II.
\section{Frazão}
\begin{itemize}
\item {Grp. gram.:m.  e  adj.}
\end{itemize}
\begin{itemize}
\item {Utilização:Ant.}
\end{itemize}
Fragueiro, adestrado em marchas diffíceis?:«\textunderscore ...que na jornada era já frazão e pratico\textunderscore ». Fr. Gaspar de S. Bern., \textunderscore Itiner.\textunderscore , 132.
\section{Frebeliano}
\begin{itemize}
\item {Grp. gram.:adj.}
\end{itemize}
Relativo ao pedagogista Froebel ou ao seu systema.
\section{Frecha}
\begin{itemize}
\item {Grp. gram.:f.}
\end{itemize}
\begin{itemize}
\item {Utilização:Bras}
\end{itemize}
O mesmo ou melhor que \textunderscore flecha\textunderscore .
Arma offensiva, composta de uma haste, terminada em ferro triangular.
Seta.
Objecto em fórma de seta.
Parte do raio perpendicular á corda, entre esta e o arco, em Geometria.
Haste ou peça pyramidal, que termina superior e exteriormente alguns edifícios.
Cana dos foguetes.
(Cast. \textunderscore flecha\textunderscore )
\section{Frechada}
\begin{itemize}
\item {fónica:fré}
\end{itemize}
\begin{itemize}
\item {Grp. gram.:f.}
\end{itemize}
\begin{itemize}
\item {Grp. gram.:Pl.}
\end{itemize}
\begin{itemize}
\item {Utilização:Prov.}
\end{itemize}
\begin{itemize}
\item {Utilização:minh.}
\end{itemize}
Golpe de frecha.
Arremêsso.
Pauzinhos, a que estão presos os liços, no tear.
\section{Frechal}
\begin{itemize}
\item {fónica:fré}
\end{itemize}
\begin{itemize}
\item {Grp. gram.:m.}
\end{itemize}
\begin{itemize}
\item {Proveniência:(De \textunderscore frecha\textunderscore )}
\end{itemize}
Cada uma das vigas horizontaes, sôbre que se levantam os frontaes de cada pavimento.
Viga, em que se pregam os caibros, á beira do telhado.
\section{Frechar}
\begin{itemize}
\item {fónica:fré}
\end{itemize}
\begin{itemize}
\item {Grp. gram.:v. t.}
\end{itemize}
\begin{itemize}
\item {Utilização:Fig.}
\end{itemize}
\begin{itemize}
\item {Grp. gram.:V. i.}
\end{itemize}
\begin{itemize}
\item {Utilização:Bras. do N}
\end{itemize}
Ferir com frecha.
Maguar; satirizar.
Atravessar rapidamente, como uma frecha.
Ir ou vir em direitura: \textunderscore o boi frechou para o meu lado\textunderscore .
\section{Frecharia}
\begin{itemize}
\item {fónica:fré}
\end{itemize}
\begin{itemize}
\item {Grp. gram.:f.}
\end{itemize}
Porção de frechas.
\section{Frecheira}
\begin{itemize}
\item {fónica:fré}
\end{itemize}
\begin{itemize}
\item {Grp. gram.:f.}
\end{itemize}
\begin{itemize}
\item {Proveniência:(De \textunderscore frecha\textunderscore )}
\end{itemize}
O mesmo que \textunderscore seteira\textunderscore .
\section{Frecheiro}
\begin{itemize}
\item {fónica:fré}
\end{itemize}
\begin{itemize}
\item {Grp. gram.:m.}
\end{itemize}
\begin{itemize}
\item {Utilização:Ant.}
\end{itemize}
\begin{itemize}
\item {Utilização:Ant.}
\end{itemize}
\begin{itemize}
\item {Utilização:Pop.}
\end{itemize}
Soldado que atirava frechas.
Aquelle que usava de frecha na caça.
O mesmo que \textunderscore frecheira\textunderscore .
Namorador, galanteador.
\section{Freda}
\begin{itemize}
\item {Grp. gram.:f.}
\end{itemize}
\begin{itemize}
\item {Utilização:Ant.}
\end{itemize}
\begin{itemize}
\item {Proveniência:(Do germ. \textunderscore vride\textunderscore )}
\end{itemize}
Espécie de multa, que, em benefício do Estado, era cobrada daquelles que quebravam a paz estipulada. Cf. Herculano, \textunderscore Hist. de Port.\textunderscore , IV, 385.
\section{Frederico}
\begin{itemize}
\item {Grp. gram.:m.}
\end{itemize}
\begin{itemize}
\item {Proveniência:(De \textunderscore Frederico\textunderscore , n. p.)}
\end{itemize}
Moéda de oiro na Dinamarca.
\section{Freeiro}
\begin{itemize}
\item {Grp. gram.:m.}
\end{itemize}
Fabricante de freios.
\section{Fregão}
\begin{itemize}
\item {Grp. gram.:m.}
\end{itemize}
\begin{itemize}
\item {Utilização:Ant.}
\end{itemize}
(V.esfregão)
\section{Fregatola}
\begin{itemize}
\item {Grp. gram.:f.}
\end{itemize}
(V.mifongo)
\section{Frege}
\begin{itemize}
\item {Grp. gram.:m.}
\end{itemize}
\begin{itemize}
\item {Utilização:Bras. do Rio}
\end{itemize}
O mesmo que \textunderscore frege-môscas\textunderscore .
\section{Frege-môscas}
\begin{itemize}
\item {Grp. gram.:m.}
\end{itemize}
\begin{itemize}
\item {Utilização:Bras. do Rio}
\end{itemize}
\begin{itemize}
\item {Proveniência:(De \textunderscore frigir\textunderscore , e \textunderscore môsca\textunderscore )}
\end{itemize}
Tasca, em que se vende principalmente peixe frito.
\section{Fregista}
\begin{itemize}
\item {Grp. gram.:m.}
\end{itemize}
\begin{itemize}
\item {Utilização:Bras. do N}
\end{itemize}
Dono ou criado de frege.
\section{Fregona}
\begin{itemize}
\item {Grp. gram.:f.}
\end{itemize}
\begin{itemize}
\item {Utilização:Des.}
\end{itemize}
Criada ou serviçal de cozinha.
(Cast. \textunderscore fregona\textunderscore )
\section{Fregosão}
\begin{itemize}
\item {Grp. gram.:m.}
\end{itemize}
(V.fragosão)
\section{Freguês}
\begin{itemize}
\item {fónica:fré}
\end{itemize}
\begin{itemize}
\item {Grp. gram.:m.}
\end{itemize}
Habitante de uma freguesia.
Cliente.
Aquelle que compra ou vende habitualmente a determinada pessôa.
(Cast. \textunderscore feligrés\textunderscore , talvez do lat. \textunderscore filius gregis\textunderscore )
\section{Freguesa}
\begin{itemize}
\item {fónica:fré}
\end{itemize}
(\textunderscore fem.\textunderscore  de \textunderscore freguês\textunderscore )
\section{Freguesia}
\begin{itemize}
\item {fónica:fré}
\end{itemize}
\begin{itemize}
\item {Grp. gram.:f.}
\end{itemize}
\begin{itemize}
\item {Proveniência:(De \textunderscore freguês\textunderscore )}
\end{itemize}
Paróchia.
Igreja parochial.
Conjunto dos parochianos.
Hábito de comprar a certa pessôa ou em certo estabelecimento.
Concorrência de compradores a um estabelecimento ou a um vendedor.
Clientela.
\section{Frei}
\begin{itemize}
\item {Grp. gram.:m.}
\end{itemize}
(Abrev. de \textunderscore freire\textunderscore )
\section{Freicha}
\begin{itemize}
\item {Grp. gram.:f.}
\end{itemize}
\begin{itemize}
\item {Utilização:Prov.}
\end{itemize}
Cachão.
Cascata no rio.
\section{Freieiro}
\begin{itemize}
\item {Grp. gram.:m.}
\end{itemize}
\begin{itemize}
\item {Utilização:Des.}
\end{itemize}
O mesmo que \textunderscore freeiro\textunderscore .
\section{Frei-jorge}
\begin{itemize}
\item {Grp. gram.:m.}
\end{itemize}
\begin{itemize}
\item {Utilização:Bras}
\end{itemize}
O mesmo que \textunderscore quiri\textunderscore .
\section{Freima}
\begin{itemize}
\item {Grp. gram.:f.}
\end{itemize}
\begin{itemize}
\item {Proveniência:(Do lat. \textunderscore flegma\textunderscore ?)}
\end{itemize}
Impaciência.
Inquietação.
Actividade.
Pressa.
Cuidado.
\section{Freimão}
\begin{itemize}
\item {Grp. gram.:m.}
\end{itemize}
(V.fleimão)
\section{Freimático}
\begin{itemize}
\item {Grp. gram.:adj.}
\end{itemize}
Que tem freima. Cf. A. Pimentel, \textunderscore Chiado\textunderscore , 85.
\section{Freio}
\begin{itemize}
\item {Grp. gram.:m.}
\end{itemize}
\begin{itemize}
\item {Utilização:Fig.}
\end{itemize}
\begin{itemize}
\item {Proveniência:(Do lat. \textunderscore frenum\textunderscore )}
\end{itemize}
Peça de metal, que se mete na bôca das cavalgaduras, presa ás redeas, para govêrno dos mesmos animaes.
Apparelho, com que se regula o movimento das máquinas.
Carro, que tem êsse apparelho, nos caminhos de ferro.
Dobra membranosa, para reter um órgão do corpo.
Cada uma das queixadas do tôrno de serralheiro.
Tudo que reprime ou sujeita.
Obstáculo; impedimento.
\section{Freira}
\begin{itemize}
\item {Grp. gram.:f.}
\end{itemize}
\begin{itemize}
\item {Utilização:T. de Leiria}
\end{itemize}
\begin{itemize}
\item {Proveniência:(De \textunderscore freire\textunderscore )}
\end{itemize}
Mulher, que faz parte de communidade religiosa, sujeita ao celibato.
Grão de milho, que estoira quando se deita no braseiro, para se comer assado.
\textunderscore Barriga de freira\textunderscore , espécie de doce.
Variedade de peixe escômbrida.
Apreciada ave africana, (\textunderscore oestrellata mollis\textunderscore , Jould.).
\section{Freiral}
\begin{itemize}
\item {Grp. gram.:adj.}
\end{itemize}
O mesmo que \textunderscore freirático\textunderscore . Cf. Filinto, X, 123.
\section{Freirar}
\begin{itemize}
\item {Grp. gram.:v. t.}
\end{itemize}
\begin{itemize}
\item {Utilização:Des.}
\end{itemize}
\begin{itemize}
\item {Grp. gram.:V. i.}
\end{itemize}
\begin{itemize}
\item {Grp. gram.:V. p.}
\end{itemize}
\begin{itemize}
\item {Proveniência:(De \textunderscore freire\textunderscore  ou \textunderscore freira\textunderscore )}
\end{itemize}
Admittir ao lugar de freire de uma Ordem militar.
Seguir a vida do convento.
Tornar-se freire ou freira.
\section{Freiraria}
\begin{itemize}
\item {Grp. gram.:f.}
\end{itemize}
A classe das freiras.
As freiras.
Conjunto de freiras. Cf. Garrett, \textunderscore Romanceiro\textunderscore , II, 52.
\section{Freiras}
\begin{itemize}
\item {Grp. gram.:f. pl.}
\end{itemize}
\begin{itemize}
\item {Utilização:Pop.}
\end{itemize}
O mesmo que \textunderscore pipoca\textunderscore .
\section{Freirático}
\begin{itemize}
\item {Grp. gram.:adj.}
\end{itemize}
\begin{itemize}
\item {Grp. gram.:M.}
\end{itemize}
\begin{itemize}
\item {Proveniência:(Do rad. de \textunderscore freire\textunderscore  ou \textunderscore freira\textunderscore )}
\end{itemize}
Conventual; monástico.
Próprio de frades ou de freiras.
Affeiçoado aos costumes monacaes.
Aquelle que frequenta conventos de freiras.
Aquelle que sympathiza com a vida dos conventos.
\section{Freire}
\begin{itemize}
\item {Grp. gram.:m.}
\end{itemize}
\begin{itemize}
\item {Proveniência:(Fr. \textunderscore frère\textunderscore )}
\end{itemize}
Membro de Ordem religiosa e militar.
Frade.
\section{Freiria}
\begin{itemize}
\item {Grp. gram.:f.}
\end{itemize}
Convento de freires.
\section{Freirice}
\begin{itemize}
\item {Grp. gram.:f.}
\end{itemize}
Maneira ou acção próprias de freira.
\section{Freirinha}
\begin{itemize}
\item {Grp. gram.:f.}
\end{itemize}
\begin{itemize}
\item {Utilização:T. de Penafiel}
\end{itemize}
\begin{itemize}
\item {Grp. gram.:Pl.}
\end{itemize}
\begin{itemize}
\item {Utilização:Prov.}
\end{itemize}
\begin{itemize}
\item {Utilização:dur.}
\end{itemize}
Crustáceo decápode (\textunderscore calappa granulata\textunderscore ).
Nome, que, no Pôrto, se dá ao lugre ou ao pintasilgo verde.
Alfinete muito pequeno.
Nome de uma planta e da respectiva flôr.
\section{Freixal}
\begin{itemize}
\item {Grp. gram.:m.}
\end{itemize}
O mesmo que \textunderscore freixial\textunderscore .
\section{Freixial}
\begin{itemize}
\item {Grp. gram.:m.}
\end{itemize}
Lugar, onde crescem freixos.
\section{Freixieiro}
\begin{itemize}
\item {Grp. gram.:m.}
\end{itemize}
\begin{itemize}
\item {Utilização:Prov.}
\end{itemize}
\begin{itemize}
\item {Utilização:beir.}
\end{itemize}
O mesmo que \textunderscore freixo\textunderscore .
\section{Freixo}
\begin{itemize}
\item {Grp. gram.:m.}
\end{itemize}
\begin{itemize}
\item {Proveniência:(Do lat. \textunderscore fraxinus\textunderscore )}
\end{itemize}
Árvore oleagínea.
\section{Freixonita}
\begin{itemize}
\item {Grp. gram.:m.}
\end{itemize}
\begin{itemize}
\item {Utilização:Prov.}
\end{itemize}
\begin{itemize}
\item {Utilização:trasm.}
\end{itemize}
Habitante de Freixo-de-Espada-á-Cinta.
\section{Freme}
\begin{itemize}
\item {Grp. gram.:m.}
\end{itemize}
\begin{itemize}
\item {Utilização:Bras}
\end{itemize}
O mesmo que \textunderscore fleme\textunderscore .
\section{Fremebundo}
\begin{itemize}
\item {Grp. gram.:adj.}
\end{itemize}
\begin{itemize}
\item {Proveniência:(Lat. \textunderscore fremebundus\textunderscore )}
\end{itemize}
O mesmo que \textunderscore fremente\textunderscore .
\section{Fremente}
\begin{itemize}
\item {Grp. gram.:adj.}
\end{itemize}
\begin{itemize}
\item {Proveniência:(Lat. \textunderscore fremens\textunderscore )}
\end{itemize}
Que freme.
\section{Fremir}
\begin{itemize}
\item {Grp. gram.:v. i.}
\end{itemize}
\begin{itemize}
\item {Utilização:Fig.}
\end{itemize}
\begin{itemize}
\item {Proveniência:(Lat. \textunderscore fremere\textunderscore )}
\end{itemize}
Têr rumor surdo e áspero.
Bramir.
Rugir.
Tremer.
Vibrar; agitar-se.
Agitar-se com júbilo, estremecer de alegria.
\section{Frêmito}
\begin{itemize}
\item {Grp. gram.:m.}
\end{itemize}
\begin{itemize}
\item {Utilização:Fig.}
\end{itemize}
\begin{itemize}
\item {Proveniência:(Lat. \textunderscore fremitus\textunderscore )}
\end{itemize}
Rumor.
Som froixo, mas áspero.
Rugido.
Estrondo de coisa que freme.
Sussurro.
Estremecimento de alegria.
Sensação espasmódica.
\section{Frendente}
\begin{itemize}
\item {Grp. gram.:adj.}
\end{itemize}
\begin{itemize}
\item {Proveniência:(Lat. \textunderscore frendens\textunderscore )}
\end{itemize}
Que range os dentes.
\section{Frender}
\begin{itemize}
\item {Grp. gram.:v. i.}
\end{itemize}
\begin{itemize}
\item {Proveniência:(Lat. \textunderscore frendere\textunderscore )}
\end{itemize}
Ranger os dentes.
Bramir de raiva.
Irritar-se.
\section{Frendor}
\begin{itemize}
\item {Grp. gram.:m.}
\end{itemize}
\begin{itemize}
\item {Proveniência:(Lat. \textunderscore frendor\textunderscore )}
\end{itemize}
Acto de ranger os dentes.
\section{Frenela}
\begin{itemize}
\item {Grp. gram.:f.}
\end{itemize}
\begin{itemize}
\item {Proveniência:(De \textunderscore Frenel\textunderscore , n. p.)}
\end{itemize}
Gênero de plantas coníferas.
\section{Frenesi}
\begin{itemize}
\item {Grp. gram.:m.}
\end{itemize}
\begin{itemize}
\item {Utilização:Fig.}
\end{itemize}
\begin{itemize}
\item {Proveniência:(Gr. \textunderscore phrenesis\textunderscore )}
\end{itemize}
Inflammação cerebral.
Delírio, resultante dessa inflammação.
Loucura furiosa.
Inquietação moral.
Impaciência.
Impertinência.
Actividade, zêlo excessivo, amor ao trabalho: \textunderscore estudar com frenesi\textunderscore .
\section{Frenesiar}
\begin{itemize}
\item {Grp. gram.:v. t.}
\end{itemize}
(V.enfrenesiar)
\section{Frenesim}
\begin{itemize}
\item {Grp. gram.:m.}
\end{itemize}
(V.frenesi)
\section{Freneticamente}
\begin{itemize}
\item {Grp. gram.:adv.}
\end{itemize}
De modo frenético.
Com frenesi.
\section{Frenético}
\begin{itemize}
\item {Grp. gram.:adj.}
\end{itemize}
\begin{itemize}
\item {Proveniência:(Gr. \textunderscore phrenetikos\textunderscore )}
\end{itemize}
Que tem frenesi.
Impaciente, rabugento.
Convulso, agitado.
\section{Frenicoques}
\begin{itemize}
\item {Grp. gram.:m. pl.}
\end{itemize}
(V.fornicoques)
\section{Frente}
\begin{itemize}
\item {Grp. gram.:f.}
\end{itemize}
\begin{itemize}
\item {Utilização:Mathem.}
\end{itemize}
\begin{itemize}
\item {Utilização:Mathem.}
\end{itemize}
\begin{itemize}
\item {Grp. gram.:Loc. adv.}
\end{itemize}
Frontaria de edifício.
Parte anterior de qualquer coisa.
Parte deanteira.
Vanguarda: \textunderscore a frente do exército\textunderscore .
Face, rosto.
Presença: \textunderscore em frente do público\textunderscore .
\textunderscore Recta de frente\textunderscore , recta parallela ao quadro.
\textunderscore Plano de frente\textunderscore , plano parallelo ao quadro.
\textunderscore Em frente\textunderscore , perante, defronte, deante.
(Cast. \textunderscore fruente\textunderscore , do lat. \textunderscore frons\textunderscore , \textunderscore frontis\textunderscore )
\section{Frêo}
\begin{itemize}
\item {Grp. gram.:m.}
\end{itemize}
\begin{itemize}
\item {Utilização:Des.}
\end{itemize}
O mesmo que \textunderscore freio\textunderscore .
\section{Frequencia}
\begin{itemize}
\item {fónica:cu-en}
\end{itemize}
\begin{itemize}
\item {Grp. gram.:f.}
\end{itemize}
\begin{itemize}
\item {Proveniência:(Lat. \textunderscore frequentia\textunderscore )}
\end{itemize}
Repetição amiudada de actos ou successos.
Acceleração: \textunderscore frequência de pulsações\textunderscore .
Convivência: \textunderscore tem frequência da sociedade\textunderscore .
Acto de frequentar: \textunderscore têr bôa frequência na escola\textunderscore .
\section{Freilínia}
\begin{itemize}
\item {Grp. gram.:f.}
\end{itemize}
Gênero de plantas escrofularíneas.
\section{Frequentação}
\begin{itemize}
\item {fónica:cu-en}
\end{itemize}
\begin{itemize}
\item {Grp. gram.:f.}
\end{itemize}
\begin{itemize}
\item {Proveniência:(Lat. \textunderscore frequentatio\textunderscore )}
\end{itemize}
Acto ou effeito de frequentar.
\section{Frequentador}
\begin{itemize}
\item {fónica:cu-en}
\end{itemize}
\begin{itemize}
\item {Grp. gram.:m.  e  adj.}
\end{itemize}
\begin{itemize}
\item {Proveniência:(Lat. \textunderscore frequentator\textunderscore )}
\end{itemize}
O que frequenta.
\section{Frequentar}
\begin{itemize}
\item {fónica:cu-en}
\end{itemize}
\begin{itemize}
\item {Grp. gram.:v. t.}
\end{itemize}
\begin{itemize}
\item {Proveniência:(Lat. \textunderscore frequentare\textunderscore )}
\end{itemize}
Ir amiúde a: \textunderscore frequentar um clube\textunderscore .
Tratar familiarmente.
Conviver com: \textunderscore frequentar gente de bem\textunderscore .
Cursar, seguir (aula, disciplina, etc.).
\section{Frequentativo}
\begin{itemize}
\item {fónica:cu-en}
\end{itemize}
\begin{itemize}
\item {Grp. gram.:adj.}
\end{itemize}
\begin{itemize}
\item {Utilização:Gram.}
\end{itemize}
\begin{itemize}
\item {Proveniência:(Lat. \textunderscore frequentativus\textunderscore )}
\end{itemize}
Diz-se dos verbos, que exprimem acção repetida ou amiudada, como \textunderscore saltitar\textunderscore , e muitos outros.
\section{Frequente}
\begin{itemize}
\item {fónica:cu-en}
\end{itemize}
\begin{itemize}
\item {Grp. gram.:adj.}
\end{itemize}
\begin{itemize}
\item {Proveniência:(Lat. \textunderscore frequens\textunderscore )}
\end{itemize}
Continuado.
Amiudado.
Assíduo num lugar ou numa coisa.
Incansável; diligente.
\section{Frequentemente}
\begin{itemize}
\item {fónica:cu-en}
\end{itemize}
\begin{itemize}
\item {Grp. gram.:adv.}
\end{itemize}
De modo frequente.
Repetidas vezes.
\section{Fresca}
\begin{itemize}
\item {fónica:frês}
\end{itemize}
\begin{itemize}
\item {Grp. gram.:f.}
\end{itemize}
\begin{itemize}
\item {Grp. gram.:Loc. adv.}
\end{itemize}
\begin{itemize}
\item {Proveniência:(De \textunderscore fresco\textunderscore )}
\end{itemize}
Aragem agradável, que sopra ao cair da tarde, em alguns dias quentes: \textunderscore foi tomar a fresca\textunderscore .
Fresquidão.
\textunderscore Á fresca\textunderscore , em trajes leves, de maneira que a aragem refresque a epiderme.
\section{Frescaço}
\begin{itemize}
\item {Grp. gram.:adj.}
\end{itemize}
O mesmo que \textunderscore frescalhote\textunderscore :«\textunderscore ainda está frescaça\textunderscore ». Camillo, \textunderscore Myst. de Fafe.\textunderscore 
\section{Frescal}
\begin{itemize}
\item {Grp. gram.:adj.}
\end{itemize}
Quási fresco: \textunderscore bacalhau frescal\textunderscore .
Que tem pouco sal.
Que não está corrupto.
Fresco.
\section{Frescalhão}
\begin{itemize}
\item {Grp. gram.:adj.}
\end{itemize}
\begin{itemize}
\item {Utilização:Fam.}
\end{itemize}
Muito fresco.
Bem conservado, apesar da idade.
Abrejeirado.
\section{Frescalhota}
\begin{itemize}
\item {Grp. gram.:adj. f.}
\end{itemize}
\begin{itemize}
\item {Proveniência:(De \textunderscore frescalhote\textunderscore )}
\end{itemize}
Diz-se da mulher de certa idade, mas ainda bem conservada e com pretensões.
\section{Frescalhote}
\begin{itemize}
\item {Grp. gram.:adj.}
\end{itemize}
Um tanto fresco, sadio, sem apparência de velho, apesar da idade. Cf. Camillo, \textunderscore Corja\textunderscore , 12.
\section{Frescamente}
\begin{itemize}
\item {Grp. gram.:adv.}
\end{itemize}
De modo fresco.
Á fresca.
\section{Frescata}
\begin{itemize}
\item {Grp. gram.:f.}
\end{itemize}
\begin{itemize}
\item {Utilização:Pop.}
\end{itemize}
\begin{itemize}
\item {Proveniência:(De \textunderscore fresco\textunderscore )}
\end{itemize}
Digressão pelo campo.
Passeata; patuscada.
\section{Fresco}
\begin{itemize}
\item {fónica:frês}
\end{itemize}
\begin{itemize}
\item {Grp. gram.:adj.}
\end{itemize}
\begin{itemize}
\item {Utilização:Prov.}
\end{itemize}
\begin{itemize}
\item {Utilização:trasm.}
\end{itemize}
\begin{itemize}
\item {Utilização:Pop.}
\end{itemize}
\begin{itemize}
\item {Utilização:Irón.}
\end{itemize}
\begin{itemize}
\item {Grp. gram.:M.}
\end{itemize}
\begin{itemize}
\item {Proveniência:(Do b. lat. \textunderscore friscus\textunderscore )}
\end{itemize}
Não muito frio: \textunderscore água fresca\textunderscore .
Viçoso: \textunderscore rosas frescas\textunderscore .
Verdejante.
Sadio, vigoroso.
Que não está cansado.
Recente: \textunderscore notícias frescas\textunderscore .
Que não está estragado ou alterado: \textunderscore peixe fresco\textunderscore .
Cozido há pouco: \textunderscore pão fresco\textunderscore .
Molhado.
Bem arejado: \textunderscore casa fresca\textunderscore .
Limpo; lavado.
Asseado.
Licencioso; que desperta ideias obscenas: \textunderscore histórias frescas\textunderscore .
Mal succedido.
Ruim; que não presta: \textunderscore é fresco, o tal sujeito\textunderscore !
Aragem fresca.
Ar, um pouco frio.
Gênero de pintura, que consiste em revestir de argamassa uma parede, e sôbre a argamassa, ainda fresca, pintar a côres, embebendo-se as tintas na parede.
Quadro pintado por êsse processo: \textunderscore os frescos de Rafael\textunderscore .
\section{Frescor}
\begin{itemize}
\item {Grp. gram.:m.}
\end{itemize}
Qualidade daquillo que é fresco.
Lenitivo.
Viço.
Verdor.
Brilho.
Vento fresco.
\section{Frescum}
\begin{itemize}
\item {Grp. gram.:m.}
\end{itemize}
\begin{itemize}
\item {Utilização:Prov.}
\end{itemize}
\begin{itemize}
\item {Utilização:beir.}
\end{itemize}
\begin{itemize}
\item {Proveniência:(De \textunderscore fresco\textunderscore )}
\end{itemize}
Cheiro de carne fresca.
\section{Frescura}
\begin{itemize}
\item {Grp. gram.:f.}
\end{itemize}
\begin{itemize}
\item {Utilização:Prov.}
\end{itemize}
\begin{itemize}
\item {Utilização:minh.}
\end{itemize}
\begin{itemize}
\item {Utilização:Pop.}
\end{itemize}
\begin{itemize}
\item {Proveniência:(De \textunderscore fresco\textunderscore )}
\end{itemize}
O mesmo que \textunderscore frescor\textunderscore .
Asseio.
Limpeza.
Chulice.
Maneira pouca decorosa de falar ou de escrever.
\section{Frese}
\begin{itemize}
\item {Grp. gram.:m.}
\end{itemize}
\begin{itemize}
\item {Utilização:Gal}
\end{itemize}
\begin{itemize}
\item {Proveniência:(Fr. \textunderscore fraise\textunderscore )}
\end{itemize}
Lima redonda de relojoeiro.
Placa fina, que serve para serrear as rodas dos relógios.
\section{Fresquidão}
\begin{itemize}
\item {Grp. gram.:f.}
\end{itemize}
O mesmo que \textunderscore frescor\textunderscore .
\section{Fresquita}
\begin{itemize}
\item {Grp. gram.:f.}
\end{itemize}
\begin{itemize}
\item {Utilização:Prov.}
\end{itemize}
\begin{itemize}
\item {Utilização:trasm.}
\end{itemize}
\begin{itemize}
\item {Proveniência:(De \textunderscore fresco\textunderscore )}
\end{itemize}
Guisado de carne fresca ou de caça apanhada no dia, em que se faz o guisado.
\section{Fressura}
\begin{itemize}
\item {Grp. gram.:f.}
\end{itemize}
\begin{itemize}
\item {Proveniência:(Do lat. \textunderscore frixura\textunderscore )}
\end{itemize}
Conjunto das vísceras mais grossas de alguns animaes, como pulmões, coração, fígado.
\section{Fressureira}
\begin{itemize}
\item {Grp. gram.:f.}
\end{itemize}
\begin{itemize}
\item {Utilização:Chul.}
\end{itemize}
\begin{itemize}
\item {Proveniência:(De \textunderscore fressureiro\textunderscore )}
\end{itemize}
Mulher que vende fressura.
Mulher, que com outra satisfaz appetites sensuaes.
\section{Fressureiro}
\begin{itemize}
\item {Grp. gram.:m.}
\end{itemize}
Aquelle que vende fressuras.
\section{Fresta}
\begin{itemize}
\item {Grp. gram.:f.}
\end{itemize}
\begin{itemize}
\item {Proveniência:(Do b. lat. \textunderscore festra\textunderscore )}
\end{itemize}
\begin{itemize}
\item {Utilização:Ant.}
\end{itemize}
Pequena abertura em parede.
Janelinha.
Fisga; fenda.
O mesmo que \textunderscore janela\textunderscore . Cf. \textunderscore Rev. Lus.\textunderscore , XVI, 7.
(Cp. fr. \textunderscore fenêtre\textunderscore )
\section{Frestado}
\begin{itemize}
\item {Grp. gram.:adj.}
\end{itemize}
Que tem frestas.
\section{Frestão}
\begin{itemize}
\item {Grp. gram.:m.}
\end{itemize}
\begin{itemize}
\item {Proveniência:(De \textunderscore fresta\textunderscore )}
\end{itemize}
Janela alta e grande, bipartida, geralmente de estilo ogival.
\section{Fretador}
\begin{itemize}
\item {Grp. gram.:m.}
\end{itemize}
Aquelle que freta.
\section{Fretagem}
\begin{itemize}
\item {Grp. gram.:f.}
\end{itemize}
Retribuição de fretamento.
Acto ou trabalho de fretar.
\section{Fretamento}
\begin{itemize}
\item {Grp. gram.:m.}
\end{itemize}
Acto ou effeito de fretar.
\section{Fretar}
\begin{itemize}
\item {Grp. gram.:v. t.}
\end{itemize}
Alugar.
Tomar ou ceder a frete.
Equipar: \textunderscore fretar um navio\textunderscore .
\section{Frete}
\begin{itemize}
\item {Grp. gram.:m.}
\end{itemize}
\begin{itemize}
\item {Proveniência:(Do ant. al. \textunderscore freht\textunderscore )}
\end{itemize}
Aluguer de embarcação.
Transporte fluvial ou marítimo.
Carregamento de navio.
Aquillo que se paga pelo transporte de alguma coisa.
Coisa transportada.
Recado, incumbência: \textunderscore ó Romão, vai-me ali fazer um frete\textunderscore .
\section{Fretejador}
\begin{itemize}
\item {Grp. gram.:m.}
\end{itemize}
Aquelle que freteja.
\section{Fretejar}
\begin{itemize}
\item {Grp. gram.:v. i.}
\end{itemize}
Fazer fretes.
\section{Fretenir}
\begin{itemize}
\item {Grp. gram.:v. i.}
\end{itemize}
\begin{itemize}
\item {Proveniência:(T. onom.?)}
\end{itemize}
Diz-se da cigarra, quando canta. Cf. Castilho, \textunderscore Fastos\textunderscore , III, 324.
\section{Frêto}
\begin{itemize}
\item {Grp. gram.:m.}
\end{itemize}
\begin{itemize}
\item {Utilização:Poét.}
\end{itemize}
\begin{itemize}
\item {Proveniência:(Lat. \textunderscore fretum\textunderscore )}
\end{itemize}
Estreito; braço de mar.
\section{Freylínia}
\begin{itemize}
\item {Grp. gram.:f.}
\end{itemize}
Gênero de plantas escrofularíneas.
\section{Frezieira}
\begin{itemize}
\item {Grp. gram.:f.}
\end{itemize}
\begin{itemize}
\item {Proveniência:(De \textunderscore Freziers\textunderscore , n. p.)}
\end{itemize}
Gênero de árvores da América central.
\section{Friabilidade}
\begin{itemize}
\item {Grp. gram.:f.}
\end{itemize}
\begin{itemize}
\item {Proveniência:(Do lat. \textunderscore friabilis\textunderscore )}
\end{itemize}
Qualidade daquillo que é friável.
\section{Friacho}
\begin{itemize}
\item {Grp. gram.:adj.}
\end{itemize}
\begin{itemize}
\item {Utilização:Pop.}
\end{itemize}
\begin{itemize}
\item {Utilização:Fig.}
\end{itemize}
\begin{itemize}
\item {Grp. gram.:M.}
\end{itemize}
\begin{itemize}
\item {Proveniência:(Do rad. de \textunderscore frio\textunderscore )}
\end{itemize}
Um tanto frio.
Irresoluto, froixo.
Friagem.
\section{Friagem}
\begin{itemize}
\item {Grp. gram.:f.}
\end{itemize}
\begin{itemize}
\item {Proveniência:(De \textunderscore frio\textunderscore )}
\end{itemize}
O mesmo que \textunderscore frialdade\textunderscore .
Doença dos vegetaes, crestados pelo frio ou feridos pelo granizo.
\section{Frialdade}
\begin{itemize}
\item {Grp. gram.:f.}
\end{itemize}
Qualidade daquillo que é frio.
Tempo frio, frio atmosphérico.
Esterilidade.
Insensibilidade.
Desinteresse.
Negligência; froixidão.
\section{Friama}
\begin{itemize}
\item {Grp. gram.:f.}
\end{itemize}
\begin{itemize}
\item {Utilização:Prov.}
\end{itemize}
\begin{itemize}
\item {Utilização:alent.}
\end{itemize}
O mesmo que \textunderscore leitôa\textunderscore .
\section{Friame}
\begin{itemize}
\item {Grp. gram.:m.}
\end{itemize}
\begin{itemize}
\item {Proveniência:(Do lat. \textunderscore frigidamen\textunderscore )}
\end{itemize}
(Fórma port., em vez do castelhanismo \textunderscore fiambre\textunderscore )
\section{Friamente}
\begin{itemize}
\item {Grp. gram.:adv.}
\end{itemize}
De modo frio.
Com frieza.
\section{Friasco}
\begin{itemize}
\item {Grp. gram.:adj.}
\end{itemize}
\begin{itemize}
\item {Utilização:Ant.}
\end{itemize}
Frio; um tanto frio.
(Cp. \textunderscore friacho\textunderscore )
\section{Friável}
\begin{itemize}
\item {Grp. gram.:adj.}
\end{itemize}
\begin{itemize}
\item {Proveniência:(Lat. \textunderscore friabilis\textunderscore )}
\end{itemize}
Que póde reduzir-se a fragmentos.
Que se parte ou se esborôa facilmente.
\section{Fricandó}
\begin{itemize}
\item {Grp. gram.:m.}
\end{itemize}
\begin{itemize}
\item {Proveniência:(Fr. \textunderscore fricandeau\textunderscore )}
\end{itemize}
Variedade de assado culinário.
\section{Fricassé}
\begin{itemize}
\item {Grp. gram.:m.}
\end{itemize}
\begin{itemize}
\item {Utilização:Fig.}
\end{itemize}
\begin{itemize}
\item {Proveniência:(Fr. \textunderscore fricassée\textunderscore )}
\end{itemize}
Guisado de carne picada ou de aves partidas e còradas, ou meio fritas em manteiga com gemas de ovos.
Nome de outros preparados culinários.
Mistura de differentes coisas.
\section{Fricativa}
\begin{itemize}
\item {Grp. gram.:f.}
\end{itemize}
\begin{itemize}
\item {Utilização:Gram.}
\end{itemize}
\begin{itemize}
\item {Proveniência:(De \textunderscore fricativo\textunderscore )}
\end{itemize}
Cada uma das letras consoantes, que se produzem com estreitamento mas sem contacto das partes do tubo vocal, como o \textunderscore v\textunderscore  e o \textunderscore f\textunderscore . Cf. J. Ribeiro, \textunderscore Diccion. Gram.\textunderscore 
\section{Fricativo}
\begin{itemize}
\item {Grp. gram.:adj.}
\end{itemize}
\begin{itemize}
\item {Proveniência:(Do lat. \textunderscore fricatus\textunderscore )}
\end{itemize}
Que roça, que fricciona; que esfrega.
\section{Fricção}
\begin{itemize}
\item {Grp. gram.:f.}
\end{itemize}
\begin{itemize}
\item {Proveniência:(Lat. \textunderscore frictio\textunderscore )}
\end{itemize}
Acto de friccionar.
Medicamento para fomentações.
\section{Friccionador}
\begin{itemize}
\item {Grp. gram.:adj.}
\end{itemize}
Que fricciona.
\section{Friccionar}
\begin{itemize}
\item {Grp. gram.:v. t.}
\end{itemize}
\begin{itemize}
\item {Proveniência:(Do lat. \textunderscore frictio\textunderscore )}
\end{itemize}
Fazer fricção em; esfregar.
Dar fomentações em: \textunderscore friccionar um braço\textunderscore .
\section{Frictor}
\begin{itemize}
\item {Grp. gram.:m.}
\end{itemize}
\begin{itemize}
\item {Proveniência:(Do lat. \textunderscore frictus\textunderscore )}
\end{itemize}
Peça de cobre, com que se incendia a escorva, nas bôcas de fogo.
\section{Frieira}
\begin{itemize}
\item {Grp. gram.:f.}
\end{itemize}
\begin{itemize}
\item {Utilização:Fig.}
\end{itemize}
\begin{itemize}
\item {Proveniência:(De \textunderscore frio\textunderscore )}
\end{itemize}
Inflammação, produzida pelo frio e acompanhada de prurido e inchação.
Pessôa que come muito: \textunderscore em se sentando á mesa, é uma frieira\textunderscore .
\section{Friesta}
\begin{itemize}
\item {Grp. gram.:f.}
\end{itemize}
\begin{itemize}
\item {Utilização:Prov.}
\end{itemize}
\begin{itemize}
\item {Utilização:minh.}
\end{itemize}
O mesmo que \textunderscore fresta\textunderscore , fenda.
\section{Friez}
\begin{itemize}
\item {Grp. gram.:f.}
\end{itemize}
O mesmo que \textunderscore frieza\textunderscore :«\textunderscore desceu da campa á friez.\textunderscore »Fagundes Varela.
\section{Frieza}
\begin{itemize}
\item {Grp. gram.:f.}
\end{itemize}
\begin{itemize}
\item {Utilização:Fig.}
\end{itemize}
Qualidade daquelle ou daquillo que é frio.
Frialdade.
Indifferentismo.
Tibieza.
Falta de colorido.
\section{Frigideira}
\begin{itemize}
\item {Grp. gram.:f.}
\end{itemize}
\begin{itemize}
\item {Utilização:Prov.}
\end{itemize}
\begin{itemize}
\item {Utilização:Bras}
\end{itemize}
\begin{itemize}
\item {Utilização:minh.}
\end{itemize}
\begin{itemize}
\item {Grp. gram.:M.  e  f.}
\end{itemize}
\begin{itemize}
\item {Utilização:Fam.}
\end{itemize}
\begin{itemize}
\item {Proveniência:(De \textunderscore frigir\textunderscore )}
\end{itemize}
Utensílio de barro ou metal, para frigir.
Mulher, que frige.
Pastelão de carne, ovos, salsa e outras substâncias.
Pessôa, que se compraz em alardear importância, figurar em público, tornar-se temida.--A accepção familiar de \textunderscore pessôa que alardeia importância\textunderscore  ou \textunderscore gosta de figurar\textunderscore , talvez provenha de que se chamavam \textunderscore frigideiras\textunderscore  uns bonés de tampo largo, usados pelos soldados dos antigos batalhões nacionaes, e com que os militares julgavam deitar bôa figura; donde procederia serem êles designados pelos \textunderscore frigideiras\textunderscore , e tornar-se extensiva a significação. Cf. P. de Carvalho, \textunderscore Lisbôa de Outros Tempos\textunderscore , II, 195.
Pessôa impertinente, rabujenta.
\section{Frigideiro}
\begin{itemize}
\item {Grp. gram.:m.}
\end{itemize}
\begin{itemize}
\item {Proveniência:(De \textunderscore frigir\textunderscore )}
\end{itemize}
Aquelle que, em tabernas, frige iscas ou peixe.
\section{Frigidez}
\begin{itemize}
\item {Grp. gram.:f.}
\end{itemize}
Qualidade daquillo que é frígido.
Frieza; indifferença.
\section{Frígido}
\begin{itemize}
\item {Grp. gram.:adj.}
\end{itemize}
\begin{itemize}
\item {Proveniência:(Lat. \textunderscore frigidus\textunderscore )}
\end{itemize}
Que tem frio.
Gelado.
Álgido.
\section{Frigífugo}
\begin{itemize}
\item {Grp. gram.:adj.}
\end{itemize}
\begin{itemize}
\item {Proveniência:(Do lat. \textunderscore frigus\textunderscore  + \textunderscore fugere\textunderscore )}
\end{itemize}
Que evita o frio; que livra do frio:«\textunderscore ...frigífuga lenha...\textunderscore »Filinto, III, 261.
\section{Frigimenta}
\begin{itemize}
\item {Grp. gram.:f.}
\end{itemize}
\begin{itemize}
\item {Utilização:Prov.}
\end{itemize}
\begin{itemize}
\item {Utilização:alg.}
\end{itemize}
\begin{itemize}
\item {Proveniência:(De \textunderscore frigir\textunderscore )}
\end{itemize}
Cebola, azeite, etc., que se refoga para qualquer guisado.
\section{Frigir}
\begin{itemize}
\item {Grp. gram.:v. t.}
\end{itemize}
\begin{itemize}
\item {Grp. gram.:V. i.}
\end{itemize}
\begin{itemize}
\item {Utilização:Fam.}
\end{itemize}
\begin{itemize}
\item {Proveniência:(Do lat. \textunderscore frigere\textunderscore )}
\end{itemize}
Cozer com manteiga, ou com azeite, ou com outra substância oleosa, em frigideira.
Alardear importância, ostentar distincções.
Gostar de dar na vista.
\section{Frigoria}
\begin{itemize}
\item {Grp. gram.:f.}
\end{itemize}
\begin{itemize}
\item {Utilização:Phýs.}
\end{itemize}
\begin{itemize}
\item {Proveniência:(Do lat. \textunderscore frigus\textunderscore , \textunderscore frigoris\textunderscore )}
\end{itemize}
O contrário da caloria.
\section{Frigórico}
\begin{itemize}
\item {Grp. gram.:adj.}
\end{itemize}
\begin{itemize}
\item {Proveniência:(Do lat. \textunderscore frigus\textunderscore , \textunderscore frigoris\textunderscore )}
\end{itemize}
Dizia-se de um fluido imponderável, que se suppunha sêr a causa do frio, mas cuja existência ninguém hoje acceita.
\section{Frigorífero}
\begin{itemize}
\item {Grp. gram.:adj.}
\end{itemize}
\begin{itemize}
\item {Proveniência:(Do lat. \textunderscore frigus\textunderscore  + \textunderscore ferre\textunderscore )}
\end{itemize}
O mesmo que \textunderscore frigorífico\textunderscore .
\section{Frigorificação}
\begin{itemize}
\item {Grp. gram.:f.}
\end{itemize}
Acto ou effeito de frigorificar.
\section{Frigorificar}
\begin{itemize}
\item {Grp. gram.:v. t.}
\end{itemize}
\begin{itemize}
\item {Utilização:Neol.}
\end{itemize}
Manter frias ou em bom estado (certas substâncias alimentícias).
Sujeitar á acção do apparelho, chamado frigorífico.
\section{Frigorífico}
\begin{itemize}
\item {Grp. gram.:adj.}
\end{itemize}
\begin{itemize}
\item {Grp. gram.:M.}
\end{itemize}
\begin{itemize}
\item {Proveniência:(Lat. \textunderscore frigorificus\textunderscore )}
\end{itemize}
Que mantém o frio.
Que origina o frio: \textunderscore apparelho frigorífico\textunderscore .
Fluido, que faz fugir o calor.
Apparelho, com que se congelam certos corpos.
Apparelho, para manter frescas e em bom estado certas substâncias alimentícias, especialmente a bordo, nas longas viagens.
\section{Frija}
\begin{itemize}
\item {Grp. gram.:m.}
\end{itemize}
\begin{itemize}
\item {Utilização:Ant.}
\end{itemize}
\begin{itemize}
\item {Utilização:Pop.}
\end{itemize}
\begin{itemize}
\item {Proveniência:(Do rad. de \textunderscore frigir\textunderscore )}
\end{itemize}
Procurador de causas.
\section{Frima}
\begin{itemize}
\item {Grp. gram.:f.}
\end{itemize}
O mesmo que \textunderscore freima\textunderscore :«\textunderscore não te assustes, não tenhas tamanha frima.\textunderscore »Garrett, \textunderscore Romanceiro\textunderscore , II, 22.
\section{Frimário}
\begin{itemize}
\item {Grp. gram.:m.}
\end{itemize}
\begin{itemize}
\item {Proveniência:(Fr. \textunderscore frimaire\textunderscore )}
\end{itemize}
Terceiro mês, segundo o calendário da primeira república francesa, o qual vai de 24 de Novembro a 20 de Dezembro.
\section{Frincha}
\begin{itemize}
\item {Grp. gram.:f.}
\end{itemize}
O mesmo que \textunderscore fenda\textunderscore .
\section{Fringillo}
\begin{itemize}
\item {Grp. gram.:m.}
\end{itemize}
\begin{itemize}
\item {Proveniência:(Do lat. \textunderscore fringilla\textunderscore )}
\end{itemize}
Gênero de pássaros conirostros e granívoros.
\section{Fringilo}
\begin{itemize}
\item {Grp. gram.:m.}
\end{itemize}
\begin{itemize}
\item {Proveniência:(Do lat. \textunderscore fringilla\textunderscore )}
\end{itemize}
Gênero de pássaros conirostros e granívoros.
\section{Frio}
\begin{itemize}
\item {Grp. gram.:adj.}
\end{itemize}
\begin{itemize}
\item {Grp. gram.:M.}
\end{itemize}
\begin{itemize}
\item {Utilização:Fig.}
\end{itemize}
\begin{itemize}
\item {Proveniência:(Do b. lat. \textunderscore fridus\textunderscore , lat. \textunderscore frigidus\textunderscore )}
\end{itemize}
Que não tem calor.
Que perdeu o calor.
Desengraçado.
Inerte.
Inexpressivo.
Froixo.
Insensível.
Rude; cruel.
Ausência de calor.
Sensação produzida por essa ausência.
Frieza.
Abaixamento de temperatura.
Desânimo.
Indifferença; inércia.
\section{Frioleira}
\begin{itemize}
\item {Grp. gram.:f.}
\end{itemize}
Espécie de espiguilha, feita com lançadeira, para guarnições, enfeites, etc.
Insignificância; parvoíce.
(Cast. \textunderscore friolera\textunderscore , do lat. \textunderscore frivola\textunderscore )
\section{Friolento}
\begin{itemize}
\item {Grp. gram.:adj.}
\end{itemize}
\begin{itemize}
\item {Utilização:Prov.}
\end{itemize}
\begin{itemize}
\item {Utilização:alent.}
\end{itemize}
O mesmo ou melhor que \textunderscore friorento\textunderscore .
\section{Friorento}
\begin{itemize}
\item {Grp. gram.:adj.}
\end{itemize}
Muito sensível ao frio.
(Cp. cast. \textunderscore friolento\textunderscore )
\section{Frisa}
\begin{itemize}
\item {Grp. gram.:f.}
\end{itemize}
\begin{itemize}
\item {Proveniência:(De \textunderscore Frísia\textunderscore , n. p.)}
\end{itemize}
Tecido grosseiro de lan.
Pêlo de pano escrespado.
Porção de lan, com que se calafetam portinholas de navios, para impedir a entrada da água.
\section{Frisa}
\begin{itemize}
\item {Grp. gram.:f.}
\end{itemize}
\begin{itemize}
\item {Proveniência:(Do rad. de \textunderscore friso\textunderscore )}
\end{itemize}
Camarote, quási ao nível de uma plateia.
Friso.
\textunderscore Cavallo de frisa\textunderscore , trave, atravessada com puas de ferro, para defesa da fortificação ou de tropas de infantaria.
\section{Frisa}
\begin{itemize}
\item {Grp. gram.:m.}
\end{itemize}
\begin{itemize}
\item {Utilização:T. de Setúbal}
\end{itemize}
Capataz de argolistas, (baldeadores de sal).
\section{Frisada}
\begin{itemize}
\item {Grp. gram.:f.}
\end{itemize}
\begin{itemize}
\item {Utilização:Ant.}
\end{itemize}
\begin{itemize}
\item {Proveniência:(De \textunderscore frisado\textunderscore )}
\end{itemize}
Ave palmípede aquática.
Variedade de pomba.
Vestido felpudo.
\section{Frisado}
\begin{itemize}
\item {Grp. gram.:m.}
\end{itemize}
\begin{itemize}
\item {Grp. gram.:Adj.}
\end{itemize}
\begin{itemize}
\item {Proveniência:(De \textunderscore frisar\textunderscore )}
\end{itemize}
Cabello encrespado artificialmente.
Que tem frisas ou frisos.
\section{Frisador}
\begin{itemize}
\item {Grp. gram.:m.}
\end{itemize}
Instrumento para frisar.
Homem, que frisa.
\section{Frisagem}
\begin{itemize}
\item {Grp. gram.:f.}
\end{itemize}
Acto ou effeito de frisar.
\section{Frisante}
\begin{itemize}
\item {Grp. gram.:adj.}
\end{itemize}
\begin{itemize}
\item {Proveniência:(De \textunderscore frisar\textunderscore )}
\end{itemize}
Que frisa.
Que é próprio, apropriado.
Significativo.
Exacto.
Terminante; convincente: \textunderscore argumento frisante\textunderscore .
\section{Frisão}
\begin{itemize}
\item {Grp. gram.:m.}
\end{itemize}
\begin{itemize}
\item {Grp. gram.:Adj.}
\end{itemize}
Cavallo forte, robusto.
Aquelle que é natural da Frísia.
Língua dos antigos Frisões.
Relativo á Frísia.
\section{Frisar}
\begin{itemize}
\item {Grp. gram.:v. t.}
\end{itemize}
\begin{itemize}
\item {Grp. gram.:V. i.}
\end{itemize}
\begin{itemize}
\item {Utilização:Fig.}
\end{itemize}
\begin{itemize}
\item {Proveniência:(De \textunderscore friso\textunderscore  e \textunderscore frisa\textunderscore )}
\end{itemize}
Encrespar.
Tornar riço; anelar.
Pôr frisas em.
Citar ou referir opportunamente, apropriadamente.
Tornar saliente, sensível.
Encrespar-se.
Tocar quási.
Chegar perto.
Aproximar-se.
Orçar.
Assemelhar-se.
Sêr analogo, conforme. Cf. Bernárdez, \textunderscore Luz e Calor\textunderscore , 55.
\section{Frísico}
\begin{itemize}
\item {Grp. gram.:m.}
\end{itemize}
Dialecto da Frísia.
Frisão.
\section{Friso}
\begin{itemize}
\item {Grp. gram.:m.}
\end{itemize}
\begin{itemize}
\item {Proveniência:(Do ár. \textunderscore ifris\textunderscore )}
\end{itemize}
Espaço entre a cornija e a architrave.
Banda ou tira, pintada em parede.
Filete; ornato, disposto em friso.
Ornato de esculptura.
\section{Frisões}
\begin{itemize}
\item {Grp. gram.:m. pl.}
\end{itemize}
Antigo povo germânico.
\section{Frita}
\begin{itemize}
\item {Grp. gram.:f.}
\end{itemize}
\begin{itemize}
\item {Proveniência:(De \textunderscore frito\textunderscore )}
\end{itemize}
Cozimento dos ingredientes, com que se forma o vidro.
Tempo que dura êsse cozimento.
Acto de queimar as substâncias orgânicas, que se encontram nas misturas mineraes.
O mesmo que \textunderscore frito\textunderscore , m.
\section{Fritada}
\begin{itemize}
\item {Grp. gram.:f.}
\end{itemize}
\begin{itemize}
\item {Proveniência:(De \textunderscore fritar\textunderscore )}
\end{itemize}
Aquillo que se frige de uma vez: \textunderscore uma fritada de ovos\textunderscore .
\section{Fritadeira}
\begin{itemize}
\item {Grp. gram.:f.}
\end{itemize}
\begin{itemize}
\item {Utilização:Prov.}
\end{itemize}
\begin{itemize}
\item {Proveniência:(De \textunderscore fritar\textunderscore )}
\end{itemize}
O mesmo que \textunderscore frigideira\textunderscore .
\section{Fritalhada}
\begin{itemize}
\item {Grp. gram.:f.}
\end{itemize}
\begin{itemize}
\item {Utilização:Pop.}
\end{itemize}
O mesmo que \textunderscore fritangada\textunderscore .
\section{Fritangada}
\begin{itemize}
\item {Grp. gram.:f.}
\end{itemize}
\begin{itemize}
\item {Utilização:Pop.}
\end{itemize}
\begin{itemize}
\item {Proveniência:(De \textunderscore fritar\textunderscore )}
\end{itemize}
Fritada mal feita, mas abundante.
\section{Fritar}
\begin{itemize}
\item {Grp. gram.:v. t.}
\end{itemize}
\begin{itemize}
\item {Proveniência:(De \textunderscore frito\textunderscore )}
\end{itemize}
O mesmo que \textunderscore frigir\textunderscore .
\section{Fritido}
\begin{itemize}
\item {Grp. gram.:m.}
\end{itemize}
\begin{itemize}
\item {Utilização:Prov.}
\end{itemize}
\begin{itemize}
\item {Utilização:trasm.}
\end{itemize}
O mesmo que \textunderscore fritada\textunderscore .
(Colhido em Boticas)
\section{Fritilária}
\begin{itemize}
\item {Grp. gram.:f.}
\end{itemize}
Planta bulbosa e medicinal, da fam. das liliáceas, (\textunderscore fritillaria meleagris\textunderscore ).
\section{Fritillária}
\begin{itemize}
\item {Grp. gram.:f.}
\end{itemize}
Planta bulbosa e medicinal, da fam. das liliáceas, (\textunderscore fritillaria meleagris\textunderscore ).
\section{Fritillo}
\begin{itemize}
\item {Grp. gram.:m.}
\end{itemize}
\begin{itemize}
\item {Proveniência:(Lat. \textunderscore fritillus\textunderscore )}
\end{itemize}
Copo para jogar os dados.
\section{Fritilo}
\begin{itemize}
\item {Grp. gram.:m.}
\end{itemize}
\begin{itemize}
\item {Proveniência:(Lat. \textunderscore fritillus\textunderscore )}
\end{itemize}
Copo para jogar os dados.
\section{Fritir}
\begin{itemize}
\item {Grp. gram.:v. t.}
\end{itemize}
\begin{itemize}
\item {Utilização:Prov.}
\end{itemize}
\begin{itemize}
\item {Utilização:minh.}
\end{itemize}
\begin{itemize}
\item {Proveniência:(De \textunderscore frito\textunderscore )}
\end{itemize}
O mesmo que \textunderscore frigir\textunderscore .
\section{Frito}
\begin{itemize}
\item {Grp. gram.:adj.}
\end{itemize}
\begin{itemize}
\item {Grp. gram.:M.}
\end{itemize}
\begin{itemize}
\item {Proveniência:(Lat. \textunderscore frictus\textunderscore )}
\end{itemize}
Que se frigiu: \textunderscore peixe frito\textunderscore .
Filhó, coscorão, qualquer fritura.
\section{Fritura}
\begin{itemize}
\item {Grp. gram.:f.}
\end{itemize}
\begin{itemize}
\item {Proveniência:(De \textunderscore frito\textunderscore )}
\end{itemize}
Qualquer coisa frita; fritada.
\section{Friul}
\begin{itemize}
\item {Grp. gram.:m.}
\end{itemize}
\begin{itemize}
\item {Utilização:bras}
\end{itemize}
\begin{itemize}
\item {Utilização:Neol.}
\end{itemize}
O mesmo que \textunderscore friúra\textunderscore ?:«\textunderscore pensas talvez nesse friul tremendo... No dia, em que os gelos descerem...\textunderscore »Coelho Neto.
\section{Friulano}
\begin{itemize}
\item {Grp. gram.:m.}
\end{itemize}
Dialecto de Friul.
\section{Friúra}
\begin{itemize}
\item {Grp. gram.:f.}
\end{itemize}
\begin{itemize}
\item {Proveniência:(De \textunderscore frio\textunderscore )}
\end{itemize}
Estado daquillo que é frio ou daquillo que se acha frio; frialdade.
\section{Frivolamente}
\begin{itemize}
\item {Grp. gram.:adv.}
\end{itemize}
De modo frívolo.
\section{Frivolidade}
\begin{itemize}
\item {Grp. gram.:f.}
\end{itemize}
Qualidade daquelle ou daquillo que é frívolo.
\section{Frívolo}
\begin{itemize}
\item {Grp. gram.:adj.}
\end{itemize}
\begin{itemize}
\item {Proveniência:(Lat. \textunderscore frivolus\textunderscore )}
\end{itemize}
Que não tem importância.
Que é sem valor.
Vão.
Fútil.
Leviano; volúvel: \textunderscore mulher frívola\textunderscore .
\section{Frizante}
\begin{itemize}
\item {Grp. gram.:m.}
\end{itemize}
Moéda dos primeiros tempos de Portugal.
(Relaciona-se com \textunderscore besante\textunderscore ?)
\section{Frocado}
\begin{itemize}
\item {Grp. gram.:adj.}
\end{itemize}
\begin{itemize}
\item {Grp. gram.:M.}
\end{itemize}
Enfeitado com froco.
Enfeite de frocos.
\section{Frocadura}
\begin{itemize}
\item {Grp. gram.:f.}
\end{itemize}
\begin{itemize}
\item {Proveniência:(De \textunderscore frocado\textunderscore )}
\end{itemize}
Ornato de frocos.
\section{Froco}
\begin{itemize}
\item {Grp. gram.:m.}
\end{itemize}
\begin{itemize}
\item {Proveniência:(Lat. \textunderscore floccus\textunderscore )}
\end{itemize}
Flocco de neve.
Felpa de lan ou seda, cortada em bocadinhos, ou torcida em cordão, para ornatos de vestuário.
Conjunto de filamentos subtis, que esvoaçam ao simples impulso da aragem.
Felpa.
Tufo de pelos, na cauda de alguns animaes.
Vaporização.
Farfalha ou partícula de neve, que cái lentamente, esvoaçando como felpa branca.
(Cp. \textunderscore floco\textunderscore )
\section{Froina}
\begin{itemize}
\item {Grp. gram.:f.}
\end{itemize}
\begin{itemize}
\item {Utilização:Gír.}
\end{itemize}
Brôa.
(Cp. \textunderscore fronha\textunderscore )
\section{Froixamente}
\begin{itemize}
\item {Grp. gram.:adv.}
\end{itemize}
De modo froixo.
Com froixidão.
\section{Froixar}
\textunderscore v. t.\textunderscore  (e der.)
O mesmo que \textunderscore afroixar\textunderscore , etc. Cf. Filinto, III, 63.
\section{Froixel}
\begin{itemize}
\item {Grp. gram.:m.}
\end{itemize}
\begin{itemize}
\item {Proveniência:(Do cast. \textunderscore flogel\textunderscore )}
\end{itemize}
Penugem de aves.
\section{Froixelado}
\begin{itemize}
\item {Grp. gram.:adj.}
\end{itemize}
Que tem froixel; em que há froixel.
\section{Froixeleiro}
\begin{itemize}
\item {Grp. gram.:adj.}
\end{itemize}
\begin{itemize}
\item {Proveniência:(De \textunderscore froixel\textunderscore )}
\end{itemize}
O mesmo que \textunderscore froixelado\textunderscore . Cf. \textunderscore Panorama\textunderscore , III, 325.
\section{Froixeza}
\begin{itemize}
\item {Grp. gram.:f.}
\end{itemize}
\begin{itemize}
\item {Utilização:Fig.}
\end{itemize}
Qualidade daquelle ou daquillo que é froixo.
Molleza.
Falta de energia, de actividade.
\section{Froixidade}
\begin{itemize}
\item {Grp. gram.:f.}
\end{itemize}
\begin{itemize}
\item {Utilização:Fig.}
\end{itemize}
Qualidade daquelle ou daquillo que é froixo.
Molleza.
Falta de energia, de actividade.
\section{Froixidão}
\begin{itemize}
\item {Grp. gram.:f.}
\end{itemize}
\begin{itemize}
\item {Utilização:Fig.}
\end{itemize}
Qualidade daquelle ou daquillo que é froixo.
Molleza.
Falta de energia, de actividade.
\section{Froixo}
\begin{itemize}
\item {Grp. gram.:adj.}
\end{itemize}
\begin{itemize}
\item {Grp. gram.:M.}
\end{itemize}
\begin{itemize}
\item {Grp. gram.:Loc. adv.}
\end{itemize}
\begin{itemize}
\item {Proveniência:(Do cast. \textunderscore flojo\textunderscore )}
\end{itemize}
Molle.
Lânguido.
Indolente; brando.
Que não tem energia.
Fraco.
Fluxo.
\textunderscore Froixo de riso\textunderscore , risada.
\textunderscore A froixo\textunderscore , abundantemente:«\textunderscore beber a froixo\textunderscore ». Camillo, \textunderscore Filha do Regicida\textunderscore .
Unanimemente.
\section{Frôl}
\begin{itemize}
\item {Grp. gram.:f.}
\end{itemize}
\begin{itemize}
\item {Utilização:Ant.}
\end{itemize}
Flôr.
Escuma do mar.
(Cp. \textunderscore flôr\textunderscore )
\section{Frolir}
\begin{itemize}
\item {Grp. gram.:v. i.}
\end{itemize}
\begin{itemize}
\item {Utilização:Ant.}
\end{itemize}
\begin{itemize}
\item {Proveniência:(De \textunderscore frol\textunderscore )}
\end{itemize}
O mesmo que \textunderscore florir\textunderscore .
\section{Fronças}
\begin{itemize}
\item {Grp. gram.:f. pl.}
\end{itemize}
\begin{itemize}
\item {Proveniência:(Do lat. \textunderscore frondea\textunderscore )}
\end{itemize}
Outra fórma de \textunderscore franças\textunderscore .
\section{Froncil}
\begin{itemize}
\item {Grp. gram.:adj.}
\end{itemize}
\begin{itemize}
\item {Utilização:Ant.}
\end{itemize}
\begin{itemize}
\item {Grp. gram.:M.}
\end{itemize}
Que tem pregas.
Lenço de pregas.
(Cp. fr. \textunderscore froncé\textunderscore )
\section{Fronda}
\begin{itemize}
\item {Grp. gram.:f.}
\end{itemize}
\begin{itemize}
\item {Proveniência:(Fr. \textunderscore fronde\textunderscore , do lat. \textunderscore funda\textunderscore )}
\end{itemize}
Guerra civil em França, no tempo de Luís XIV.
\section{Frondar}
\begin{itemize}
\item {Grp. gram.:v. i.}
\end{itemize}
\begin{itemize}
\item {Utilização:bras}
\end{itemize}
\begin{itemize}
\item {Utilização:Neol.}
\end{itemize}
Tornar fronde, còpar:«\textunderscore frondava um vistoso ramalhete de palmeiras\textunderscore ». J. Alencar, \textunderscore Til\textunderscore .
\section{Fronde}
\begin{itemize}
\item {Grp. gram.:f.}
\end{itemize}
\begin{itemize}
\item {Utilização:Ext.}
\end{itemize}
\begin{itemize}
\item {Proveniência:(Do lat. \textunderscore frons\textunderscore , \textunderscore frondis\textunderscore )}
\end{itemize}
Folhagem de palmeiras e fêtos.
Ramo ou ramagem de árvore.
\section{Frondear}
\begin{itemize}
\item {Grp. gram.:v. t.  e  i.}
\end{itemize}
O mesmo que \textunderscore frondejar\textunderscore .
\section{Frondecer}
\begin{itemize}
\item {Grp. gram.:v. i.}
\end{itemize}
(V.frondescer)
\section{Frondejante}
\begin{itemize}
\item {Grp. gram.:adj.}
\end{itemize}
Que frondeja; que é frondoso.
\section{Frondejar}
\begin{itemize}
\item {Grp. gram.:v. t.}
\end{itemize}
\begin{itemize}
\item {Grp. gram.:V. i.}
\end{itemize}
\begin{itemize}
\item {Proveniência:(De \textunderscore fronde\textunderscore )}
\end{itemize}
Cobrir de fôlhas.
Encher de fôlhas.
Cobrir-se de fôlhas, criar fôlhas.
Sêr frondoso.
\section{Frondente}
\begin{itemize}
\item {Grp. gram.:adj.}
\end{itemize}
\begin{itemize}
\item {Proveniência:(Lat. \textunderscore frondens\textunderscore )}
\end{itemize}
Que tem frondes.
Que frondeja; frondoso; còpado.
\section{Frôndeo}
\begin{itemize}
\item {Grp. gram.:adj.}
\end{itemize}
\begin{itemize}
\item {Proveniência:(Lat. \textunderscore frondeus\textunderscore )}
\end{itemize}
O mesmo que \textunderscore frondente\textunderscore .
\section{Frondescência}
\begin{itemize}
\item {Grp. gram.:f.}
\end{itemize}
\begin{itemize}
\item {Proveniência:(De \textunderscore frondescer\textunderscore )}
\end{itemize}
O desenvolver das frondes.
Folheatura.
\section{Frondescente}
\begin{itemize}
\item {Grp. gram.:adj.}
\end{itemize}
\begin{itemize}
\item {Proveniência:(Lat. \textunderscore frondescens\textunderscore )}
\end{itemize}
Que frondesce; frondente.
\section{Frondescer}
\begin{itemize}
\item {Grp. gram.:v. i.}
\end{itemize}
\begin{itemize}
\item {Proveniência:(Lat. \textunderscore frondescere\textunderscore )}
\end{itemize}
Criar fôlhas.
Começar a enramar-se.
\section{Frondíbalo}
\begin{itemize}
\item {Grp. gram.:m.}
\end{itemize}
(Apparece êste t. no \textunderscore Diccion.\textunderscore  de Vieira, vb. \textunderscore balista\textunderscore . É êrro certamente. V. \textunderscore fundíbalo\textunderscore )
\section{Frondícola}
\begin{itemize}
\item {Grp. gram.:adj.}
\end{itemize}
\begin{itemize}
\item {Proveniência:(Do lat. \textunderscore frons\textunderscore , \textunderscore frondis\textunderscore  + \textunderscore colere\textunderscore )}
\end{itemize}
Que vive nos ramos das árvores.
\section{Frondífero}
\begin{itemize}
\item {Grp. gram.:adj.}
\end{itemize}
\begin{itemize}
\item {Proveniência:(Lat. \textunderscore frondifer\textunderscore )}
\end{itemize}
Que tem fôlhas.
Que cria fôlhas.
\section{Frondíparo}
\begin{itemize}
\item {Grp. gram.:adj.}
\end{itemize}
\begin{itemize}
\item {Utilização:Bot.}
\end{itemize}
\begin{itemize}
\item {Proveniência:(Do lat. \textunderscore frons\textunderscore  + \textunderscore parere\textunderscore )}
\end{itemize}
Diz-se das flôres que, por anomalia, produzem fôlhas.
\section{Frondista}
\begin{itemize}
\item {Grp. gram.:m.}
\end{itemize}
Partidário da fronda.
\section{Frondosidade}
\begin{itemize}
\item {Grp. gram.:f.}
\end{itemize}
Qualidade daquillo que é frondoso.
\section{Frondoso}
\begin{itemize}
\item {Grp. gram.:adj.}
\end{itemize}
\begin{itemize}
\item {Proveniência:(Lat. \textunderscore frondosus\textunderscore )}
\end{itemize}
Que tem muitas fôlhas.
Abundante de ramos.
Còpado.
Espêsso.
\section{Frôndula}
\begin{itemize}
\item {Grp. gram.:f.}
\end{itemize}
\begin{itemize}
\item {Utilização:Bot.}
\end{itemize}
\begin{itemize}
\item {Proveniência:(De \textunderscore fronde\textunderscore )}
\end{itemize}
Reunião de fôlhas nos musgos.
\section{Fronha}
\begin{itemize}
\item {Grp. gram.:f.}
\end{itemize}
\begin{itemize}
\item {Utilização:Gír.}
\end{itemize}
Espécie de saco, que, cheio de lan, de palha ou de outra substância flexível, fórma o travesseiro, a almofada, etc.
Peça de roupa, que envolve e resguarda o travesseiro ou a almofada da cama.
Pão.
(Cp. \textunderscore fronho\textunderscore )
\section{Fronho}
\begin{itemize}
\item {Grp. gram.:adj.}
\end{itemize}
\begin{itemize}
\item {Utilização:Prov.}
\end{itemize}
\begin{itemize}
\item {Utilização:minh.}
\end{itemize}
\begin{itemize}
\item {Proveniência:(Do lat. \textunderscore foraneus\textunderscore , que passou a \textunderscore foroneus\textunderscore )}
\end{itemize}
Diz-se do portal, por onde entram os bois na residência do lavrador.
E chama-se também \textunderscore portal fronho\textunderscore  a porta principal da casa.
\section{Fronta}
\begin{itemize}
\item {Grp. gram.:f.}
\end{itemize}
\begin{itemize}
\item {Utilização:Ant.}
\end{itemize}
\begin{itemize}
\item {Proveniência:(Do rad. de \textunderscore fronte\textunderscore )}
\end{itemize}
Apresentação; noticia.
\section{Fronta}
\begin{itemize}
\item {Grp. gram.:f.}
\end{itemize}
\begin{itemize}
\item {Utilização:Ant.}
\end{itemize}
Acto de frontar.
\section{Frontaberto}
\begin{itemize}
\item {Grp. gram.:adj.}
\end{itemize}
\begin{itemize}
\item {Proveniência:(De \textunderscore fronte\textunderscore  + \textunderscore aberto\textunderscore )}
\end{itemize}
Diz-se do cavallo, que tem malha branca na testa, de alto a baixo. Cf. \textunderscore Viriato Trág.\textunderscore  XI, 104.
\section{Frontada}
\begin{itemize}
\item {Grp. gram.:f.}
\end{itemize}
\begin{itemize}
\item {Utilização:Des.}
\end{itemize}
\begin{itemize}
\item {Proveniência:(De \textunderscore frontar\textunderscore )}
\end{itemize}
Pedra de cantaria, que concorre para as duas faces de uma parede.
\section{Frontal}
\begin{itemize}
\item {Grp. gram.:adj.}
\end{itemize}
\begin{itemize}
\item {Grp. gram.:M.}
\end{itemize}
\begin{itemize}
\item {Utilização:Anat.}
\end{itemize}
\begin{itemize}
\item {Proveniência:(Lat. \textunderscore frontalis\textunderscore )}
\end{itemize}
Relativo a fronte: \textunderscore osso frontal\textunderscore .
Faixa, que os Judeus usam em volta da cabeça.
Ornato architectónico, por cima de portas ou janelas.
Tabique, taipa.
Parapeito de baluarte.
Tela ou ornato, que reveste a frente do altar; frente de altar.
O mesmo que \textunderscore coronal\textunderscore , osso frontal.
\section{Frontaleira}
\begin{itemize}
\item {Grp. gram.:f.}
\end{itemize}
\begin{itemize}
\item {Proveniência:(De \textunderscore frontal\textunderscore )}
\end{itemize}
Tela com franjas, que guarnece a frente do altar.
\section{Frontão}
\begin{itemize}
\item {Grp. gram.:m.}
\end{itemize}
\begin{itemize}
\item {Utilização:Bras}
\end{itemize}
\begin{itemize}
\item {Utilização:Bras}
\end{itemize}
\begin{itemize}
\item {Proveniência:(De \textunderscore fronte\textunderscore )}
\end{itemize}
Peça architectónica, que adorna a parte superior de portas ou janelas, ou que corôa a entrada principal de um edifício.
Edifício, onde se joga a pelota.
Parede, contra a qual se joga a péla ou a pelota.
\section{Frontar}
\begin{itemize}
\item {Grp. gram.:v. t.}
\end{itemize}
\begin{itemize}
\item {Utilização:Ant.}
\end{itemize}
\begin{itemize}
\item {Proveniência:(De \textunderscore fronta\textunderscore )}
\end{itemize}
Requerer; pedir com instância.
\section{Frontaria}
\begin{itemize}
\item {Grp. gram.:f.}
\end{itemize}
\begin{itemize}
\item {Utilização:Ant.}
\end{itemize}
\begin{itemize}
\item {Proveniência:(De \textunderscore fronte\textunderscore )}
\end{itemize}
Fachada (de um edifício).
Frontispício.
Frente.
Guarnição militar de fronteiros.
\section{Fronte}
\begin{itemize}
\item {Grp. gram.:f.}
\end{itemize}
\begin{itemize}
\item {Grp. gram.:Loc. adv.}
\end{itemize}
\begin{itemize}
\item {Proveniência:(Do lat. \textunderscore frons\textunderscore , \textunderscore frontis\textunderscore )}
\end{itemize}
Testa.
Cabeça.
Frontaria.
Frente.
\textunderscore De frente\textunderscore , em frente, na frente; deante.
\section{Frontear}
\begin{itemize}
\item {Grp. gram.:v. i.}
\end{itemize}
\begin{itemize}
\item {Grp. gram.:V. t.}
\end{itemize}
\begin{itemize}
\item {Proveniência:(De \textunderscore fronte\textunderscore )}
\end{itemize}
Estar defronte; defrontar; estar em frente.
Estar defronte de.
Sêr situado em frente de. Cf. Latino, \textunderscore Memorial Biogr. do Gen. Claudino\textunderscore .
\section{Fronteira}
\begin{itemize}
\item {Grp. gram.:f.}
\end{itemize}
\begin{itemize}
\item {Utilização:Ant.}
\end{itemize}
\begin{itemize}
\item {Proveniência:(De \textunderscore fronteiro\textunderscore )}
\end{itemize}
Extremidade de um país ou região, do lado por onde confina com outro.
Linha divisória entre duas regiões ou países.
Confins.
Estrema; limite; fim.
Expedição militar, para defesa dos limites ou fronteiras de um país.
\section{Fronteirar}
\begin{itemize}
\item {Grp. gram.:v. t.}
\end{itemize}
\begin{itemize}
\item {Proveniência:(De \textunderscore fronteira\textunderscore )}
\end{itemize}
Tornar fronteiro.
Pôr de fronte.
\section{Fronteiriço}
\begin{itemize}
\item {Grp. gram.:adj.}
\end{itemize}
\begin{itemize}
\item {Proveniência:(De \textunderscore fronteira\textunderscore )}
\end{itemize}
Que vive ou está na fronteira ou na raia; raiano.
\section{Fronteiro}
\begin{itemize}
\item {Grp. gram.:adj.}
\end{itemize}
\begin{itemize}
\item {Utilização:Bras. do N}
\end{itemize}
\begin{itemize}
\item {Grp. gram.:M.}
\end{itemize}
\begin{itemize}
\item {Utilização:Ant.}
\end{itemize}
\begin{itemize}
\item {Proveniência:(Do b. lat. \textunderscore frontarius\textunderscore )}
\end{itemize}
Que esta defronte; situado na fronteira.
Diz-se do boi ou vaca, que tem a testa branca.
Capitão de uma praça de guerra, situada na fronteira.
\section{Frontino}
\begin{itemize}
\item {Grp. gram.:adj.}
\end{itemize}
\begin{itemize}
\item {Proveniência:(De \textunderscore fronte\textunderscore )}
\end{itemize}
Diz-se do cavallo, que tem malha branca na testa.
\section{Frontirostros}
\begin{itemize}
\item {fónica:rós}
\end{itemize}
\begin{itemize}
\item {Grp. gram.:m. pl.}
\end{itemize}
\begin{itemize}
\item {Proveniência:(De \textunderscore fronte\textunderscore  + \textunderscore rostro\textunderscore )}
\end{itemize}
Família de insectos hemípteros, cujo rostro parece nascer-lhes na fronte.
\section{Frontirrostros}
\begin{itemize}
\item {Grp. gram.:m. pl.}
\end{itemize}
\begin{itemize}
\item {Proveniência:(De \textunderscore fronte\textunderscore  + \textunderscore rostro\textunderscore )}
\end{itemize}
Família de insectos hemípteros, cujo rostro parece nascer-lhes na fronte.
\section{Frontispício}
\begin{itemize}
\item {Grp. gram.:m.}
\end{itemize}
\begin{itemize}
\item {Utilização:Fig.}
\end{itemize}
Frontaria; fachada.
A primeira página ou rosto de um livro ou folheto.
Rosto.
(B. lat. \textunderscore frontispicium\textunderscore )
\section{Fronto...}
\begin{itemize}
\item {Grp. gram.:pref.}
\end{itemize}
\begin{itemize}
\item {Proveniência:(Do rad. de \textunderscore fronte\textunderscore )}
\end{itemize}
(designativo de que alguma coisa é relativa á fronte e a outra parte do corpo)
\section{Fronto-nasal}
\begin{itemize}
\item {Grp. gram.:adj.}
\end{itemize}
\begin{itemize}
\item {Utilização:Anat.}
\end{itemize}
Diz-se do músculo pyramidal do nariz.
\section{Fronto-parietal}
\begin{itemize}
\item {Grp. gram.:adj.}
\end{itemize}
\begin{itemize}
\item {Utilização:Anat.}
\end{itemize}
Relativo á testa e região temporal.
\section{Froque}
\begin{itemize}
\item {Grp. gram.:m.}
\end{itemize}
\begin{itemize}
\item {Utilização:Pop.}
\end{itemize}
Cordãozinho, formado por fêlpa de lan ou seda.
O mesmo que \textunderscore froco\textunderscore . Cf. Camillo, \textunderscore Cav. em Ruínas\textunderscore , 116.
\section{Frôr}
\begin{itemize}
\item {Grp. gram.:f.}
\end{itemize}
\begin{itemize}
\item {Utilização:Ant.}
\end{itemize}
O mesmo que \textunderscore flôr\textunderscore .
\section{Frorão}
\begin{itemize}
\item {Grp. gram.:m.}
\end{itemize}
Florão?:«\textunderscore ...porque do envestir, que a Fusta fez em elle no quartel da popa com os frorões alagou-se...\textunderscore »Azurara, \textunderscore Chrón. do Conde D. Pedro\textunderscore , l. II, c. 22, 566.
\section{Frota}
\begin{itemize}
\item {Grp. gram.:f.}
\end{itemize}
\begin{itemize}
\item {Utilização:Ext.}
\end{itemize}
Porção de navios de guerra; armada.
Chusma, grande quantidade; multidão.
(Cp. cast. \textunderscore flota\textunderscore )
\section{Froto}
\begin{itemize}
\item {Grp. gram.:m.}
\end{itemize}
\begin{itemize}
\item {Utilização:Ant.}
\end{itemize}
\begin{itemize}
\item {Proveniência:(Do fr. \textunderscore flot\textunderscore )}
\end{itemize}
Usou-se \textunderscore a froto\textunderscore , em vez de \textunderscore a nado\textunderscore .
\section{Frouças}
\begin{itemize}
\item {Grp. gram.:f. pl.}
\end{itemize}
\begin{itemize}
\item {Utilização:Prov.}
\end{itemize}
\begin{itemize}
\item {Utilização:trasm.}
\end{itemize}
O mesmo que \textunderscore fronças\textunderscore .
\section{Frouva}
\begin{itemize}
\item {Grp. gram.:f.}
\end{itemize}
Espécie de corvo, (\textunderscore corvus frugilegus\textunderscore ).
\section{Frova}
\begin{itemize}
\item {fónica:frô}
\end{itemize}
\begin{itemize}
\item {Grp. gram.:f.}
\end{itemize}
O mesmo que \textunderscore frouva\textunderscore .
\section{Frouxamente}
\begin{itemize}
\item {Grp. gram.:adv.}
\end{itemize}
De modo frouxo.
Com frouxidão.
\section{Frouxar}
\textunderscore v. t.\textunderscore  (e der.)
O mesmo que \textunderscore afrouxar\textunderscore , etc. Cf. Filinto, III, 63.
\section{Frouxel}
\begin{itemize}
\item {Grp. gram.:m.}
\end{itemize}
\begin{itemize}
\item {Proveniência:(Do cast. \textunderscore flogel\textunderscore )}
\end{itemize}
Penugem de aves.
\section{Frouxelado}
\begin{itemize}
\item {Grp. gram.:adj.}
\end{itemize}
Que tem frouxel; em que há frouxel.
\section{Frouxeleiro}
\begin{itemize}
\item {Grp. gram.:adj.}
\end{itemize}
\begin{itemize}
\item {Proveniência:(De \textunderscore frouxel\textunderscore )}
\end{itemize}
O mesmo que \textunderscore frouxelado\textunderscore . Cf. \textunderscore Panorama\textunderscore , III, 325.
\section{Frouxeza}
\begin{itemize}
\item {Grp. gram.:f.}
\end{itemize}
\begin{itemize}
\item {Utilização:Fig.}
\end{itemize}
Qualidade daquelle ou daquillo que é frouxo.
Molleza.
Falta de energia, de actividade.
\section{Frouxidade}
\begin{itemize}
\item {Grp. gram.:f.}
\end{itemize}
\begin{itemize}
\item {Utilização:Fig.}
\end{itemize}
Qualidade daquelle ou daquillo que é frouxo.
Molleza.
Falta de energia, de actividade.
\section{Frouxidão}
\begin{itemize}
\item {Grp. gram.:f.}
\end{itemize}
\begin{itemize}
\item {Utilização:Fig.}
\end{itemize}
Qualidade daquelle ou daquillo que é frouxo.
Molleza.
Falta de energia, de actividade.
\section{Frouxo}
\begin{itemize}
\item {Grp. gram.:adj.}
\end{itemize}
\begin{itemize}
\item {Grp. gram.:M.}
\end{itemize}
\begin{itemize}
\item {Grp. gram.:Loc. adv.}
\end{itemize}
\begin{itemize}
\item {Proveniência:(Do cast. \textunderscore flojo\textunderscore )}
\end{itemize}
Molle.
Lânguido.
Indolente; brando.
Que não tem energia.
Fraco.
Fluxo.
\textunderscore Frouxo de riso\textunderscore , risada.
\textunderscore A frouxo\textunderscore , abundantemente:«\textunderscore beber a frouxo\textunderscore ». Camillo, \textunderscore Filha do Regicida\textunderscore .
Unanimemente.
\section{Fructa}
\textunderscore f.\textunderscore  (e der.)
O mesmo que \textunderscore fruta\textunderscore , etc.
\section{Fructicultor}
\begin{itemize}
\item {Proveniência:(Do lat. \textunderscore fructus\textunderscore  + \textunderscore cultor\textunderscore )}
\end{itemize}
Cultivador de árvores fructíferas.
Pomareiro.
\section{Fructicultura}
\begin{itemize}
\item {Grp. gram.:f.}
\end{itemize}
\begin{itemize}
\item {Proveniência:(Do lat. \textunderscore fructus\textunderscore  + \textunderscore cultura\textunderscore )}
\end{itemize}
Cultura de pomares ou de árvores fructíferas.
\section{Fructidor}
\begin{itemize}
\item {Grp. gram.:m.}
\end{itemize}
\begin{itemize}
\item {Proveniência:(Do lat. \textunderscore fructus\textunderscore  + gr. \textunderscore doron\textunderscore )}
\end{itemize}
Duodecimo mês do calendário da primeira república francesa, (18 de Agosto a 16 de Setembro).
\section{Fructífero}
\begin{itemize}
\item {Grp. gram.:adj.}
\end{itemize}
\begin{itemize}
\item {Utilização:Fig.}
\end{itemize}
\begin{itemize}
\item {Proveniência:(Lat. \textunderscore fructifer\textunderscore )}
\end{itemize}
Que dá frutos.
Proveitoso, útil.
\section{Fructificação}
\begin{itemize}
\item {Grp. gram.:f.}
\end{itemize}
\begin{itemize}
\item {Proveniência:(Lat. \textunderscore fructificatio\textunderscore )}
\end{itemize}
Acto ou effeito de fructificar.
\section{Fructificamento}
\begin{itemize}
\item {Grp. gram.:m.}
\end{itemize}
\begin{itemize}
\item {Utilização:Des.}
\end{itemize}
O mesmo que \textunderscore fructificação\textunderscore .
\section{Fructificar}
\begin{itemize}
\item {Grp. gram.:v. i.}
\end{itemize}
\begin{itemize}
\item {Utilização:Fig.}
\end{itemize}
\begin{itemize}
\item {Proveniência:(Lat. \textunderscore fructificare\textunderscore )}
\end{itemize}
Produzir frutos.
Dar resultado: \textunderscore o seu trabalho fructificou\textunderscore .
Sêr útil.
\section{Fructificativo}
\begin{itemize}
\item {Grp. gram.:adj.}
\end{itemize}
O mesmo que \textunderscore fructífero\textunderscore .
\section{Fructifloras}
\begin{itemize}
\item {Grp. gram.:f. pl.}
\end{itemize}
\begin{itemize}
\item {Utilização:Bot.}
\end{itemize}
\begin{itemize}
\item {Proveniência:(De \textunderscore fructifloro\textunderscore )}
\end{itemize}
Classe de plantas, que têm os estames sôbre os pistillos.
\section{Fructifloro}
\begin{itemize}
\item {Grp. gram.:adj.}
\end{itemize}
\begin{itemize}
\item {Utilização:Bot.}
\end{itemize}
\begin{itemize}
\item {Proveniência:(Do lat. \textunderscore fructus\textunderscore  + \textunderscore flos\textunderscore )}
\end{itemize}
Diz-se das plantas que têm o ovário livre.
\section{Fructiforme}
\begin{itemize}
\item {Grp. gram.:adj.}
\end{itemize}
\begin{itemize}
\item {Proveniência:(Do lat. \textunderscore fructus\textunderscore  + \textunderscore forma\textunderscore )}
\end{itemize}
Que tem fórma de fruto.
\section{Fructígero}
\begin{itemize}
\item {Grp. gram.:adj.}
\end{itemize}
\begin{itemize}
\item {Proveniência:(Do lat. \textunderscore fructus\textunderscore  + \textunderscore gerere\textunderscore )}
\end{itemize}
O mesmo que \textunderscore fructífero\textunderscore .
\section{Fructívoro}
\begin{itemize}
\item {Grp. gram.:adj.}
\end{itemize}
\begin{itemize}
\item {Proveniência:(Do lat. \textunderscore fructus\textunderscore  + \textunderscore vorare\textunderscore )}
\end{itemize}
Que se alimenta de frutos.
\section{Fructuária}
\begin{itemize}
\item {Grp. gram.:f.}
\end{itemize}
\begin{itemize}
\item {Proveniência:(De \textunderscore fructuário\textunderscore )}
\end{itemize}
Associação suíça de pequenos industriaes, para o fabríco do queijo, distribuíndo-se o lucro ou fruto proporcionalmente ao leite, com que cada associado contribuiu.
Instituição official, tentada há pouco entre nós, para exploração de lacticínios por conta do Estado.
Impropriamente, fábrica de lacticínios.
\section{Fructuário}
\begin{itemize}
\item {Grp. gram.:adj.}
\end{itemize}
\begin{itemize}
\item {Proveniência:(Lat. \textunderscore fructuarius\textunderscore )}
\end{itemize}
Relativo a frutos.
Fértil, fecundo.
Que dá bons resultados.
\section{Fructuosamente}
\begin{itemize}
\item {Grp. gram.:adv.}
\end{itemize}
De modo fructuoso.
\section{Fructuoso}
\begin{itemize}
\item {Grp. gram.:adj.}
\end{itemize}
\begin{itemize}
\item {Utilização:Fig.}
\end{itemize}
\begin{itemize}
\item {Proveniência:(Lat. \textunderscore fructuosus\textunderscore )}
\end{itemize}
Abundante em frutos.
Fecundante.
Proveitoso, útil.
\section{Frufru}
\begin{itemize}
\item {Grp. gram.:m.}
\end{itemize}
\begin{itemize}
\item {Proveniência:(Fr. \textunderscore frou-frou\textunderscore )}
\end{itemize}
Rumor de fôlhas.
Rumor de vestidos, mormente dos de seda.
\section{Frugal}
\begin{itemize}
\item {Grp. gram.:m.}
\end{itemize}
\begin{itemize}
\item {Proveniência:(Lat. \textunderscore frugalis\textunderscore )}
\end{itemize}
Relativo a frutos.
Moderado.
Prudente.
Sóbrio; moderado.
\section{Frugalidade}
\begin{itemize}
\item {Grp. gram.:f.}
\end{itemize}
\begin{itemize}
\item {Proveniência:(Lat. \textunderscore frugalitas\textunderscore )}
\end{itemize}
Qualidade daquelle ou daquillo que é frugal.
\section{Frugalmente}
\begin{itemize}
\item {Grp. gram.:adv.}
\end{itemize}
De modo frugal.
\section{Frugífero}
\begin{itemize}
\item {Grp. gram.:adj.}
\end{itemize}
\begin{itemize}
\item {Proveniência:(Lat. \textunderscore frugifer\textunderscore )}
\end{itemize}
O mesmo que \textunderscore fructífero\textunderscore .
\section{Frugívoro}
\begin{itemize}
\item {Grp. gram.:adj.}
\end{itemize}
\begin{itemize}
\item {Proveniência:(Do lat. \textunderscore frux\textunderscore  + \textunderscore vorare\textunderscore )}
\end{itemize}
Que se alimenta de frutos ou de outros vegetaes.
\section{Fruição}
\begin{itemize}
\item {fónica:frui-i}
\end{itemize}
\begin{itemize}
\item {Grp. gram.:f.}
\end{itemize}
Acto ou effeito de fruir.
\section{Fruir}
\begin{itemize}
\item {Grp. gram.:v. t.}
\end{itemize}
\begin{itemize}
\item {Grp. gram.:V. i.}
\end{itemize}
\begin{itemize}
\item {Proveniência:(Lat. \textunderscore frui\textunderscore )}
\end{itemize}
Desfrutar; estar na posse de.
Gozar.
\section{Fruita}
\begin{itemize}
\item {Grp. gram.:f.}
\end{itemize}
\begin{itemize}
\item {Utilização:Ant.}
\end{itemize}
\begin{itemize}
\item {Utilização:Bras. do N}
\end{itemize}
Fruta.
Espécie de bolo, feito de farinha de mandioca, açúcar e pimenta.
\section{Fruiteira}
\begin{itemize}
\item {Grp. gram.:f.}
\end{itemize}
\begin{itemize}
\item {Utilização:Bras}
\end{itemize}
\begin{itemize}
\item {Proveniência:(De \textunderscore fruita\textunderscore )}
\end{itemize}
Qualquer árvore fructífera.
Jaboticabeira.
\section{Fruitivo}
\begin{itemize}
\item {fónica:fru-i}
\end{itemize}
\begin{itemize}
\item {Grp. gram.:adj.}
\end{itemize}
\begin{itemize}
\item {Proveniência:(De \textunderscore fruir\textunderscore )}
\end{itemize}
Que frue, que possue, que goza.
Agradável, delicioso.
Que é digno de se fruir.
\section{Fruito}
\begin{itemize}
\item {Grp. gram.:m.}
\end{itemize}
\begin{itemize}
\item {Utilização:Des.}
\end{itemize}
O mesmo que \textunderscore fruto\textunderscore :«\textunderscore De teus annos colhendo o doce fruito.\textunderscore »\textunderscore Lusíadas\textunderscore , III, 120.
\section{Frumentação}
\begin{itemize}
\item {Grp. gram.:f.}
\end{itemize}
\begin{itemize}
\item {Proveniência:(Lat. \textunderscore frumentatio\textunderscore )}
\end{itemize}
Acto de forragear ou fazer provisões de cereaes, em tempo de guerra. Cf. Castilho, \textunderscore Fastos\textunderscore , III, 490.
\section{Frumentáceo}
\begin{itemize}
\item {Grp. gram.:adj.}
\end{itemize}
\begin{itemize}
\item {Proveniência:(Lat. \textunderscore frumentaceus\textunderscore )}
\end{itemize}
Semelhante a cereaes.
Que é da natureza de cereaes.
\section{Frumental}
\begin{itemize}
\item {Grp. gram.:adj.}
\end{itemize}
\begin{itemize}
\item {Grp. gram.:M.}
\end{itemize}
\begin{itemize}
\item {Proveniência:(De \textunderscore frumento\textunderscore )}
\end{itemize}
Relativo a cereaes.
Próprio para sementeira de cereaes.
Espécie de aveia.
\section{Frumentalite}
\begin{itemize}
\item {Grp. gram.:f.}
\end{itemize}
\begin{itemize}
\item {Proveniência:(De \textunderscore frumental\textunderscore )}
\end{itemize}
Variedade de pedras, que se julgou serem grãos de trigo fossilizados.
\section{Frumentário}
\begin{itemize}
\item {Grp. gram.:adj.}
\end{itemize}
O mesmo que \textunderscore frumentáceo\textunderscore . Cf. Latino, \textunderscore Oração da Corôa\textunderscore , 51; Herculano, \textunderscore Hist. de Port.\textunderscore , III, 367.
\section{Frumentício}
\begin{itemize}
\item {Grp. gram.:adj.}
\end{itemize}
(V.frumentáceo)
\section{Frumento}
\begin{itemize}
\item {Grp. gram.:m.}
\end{itemize}
\begin{itemize}
\item {Proveniência:(Lat. \textunderscore frumentum\textunderscore )}
\end{itemize}
O melhor trigo.
Cereaes.
\section{Frumentoso}
\begin{itemize}
\item {Grp. gram.:adj.}
\end{itemize}
\begin{itemize}
\item {Proveniência:(De \textunderscore frumento\textunderscore )}
\end{itemize}
Fértil em cereaes.
\section{Fruncho}
\begin{itemize}
\item {Grp. gram.:m.}
\end{itemize}
O mesmo que \textunderscore frunco\textunderscore .
\section{Frunco}
\begin{itemize}
\item {Grp. gram.:m.}
\end{itemize}
O mesmo que \textunderscore furúnculo\textunderscore .
\section{Frúnculo}
\begin{itemize}
\item {Grp. gram.:m.}
\end{itemize}
(V.furúnculo)
\section{Frusto}
\begin{itemize}
\item {Grp. gram.:adj.}
\end{itemize}
\begin{itemize}
\item {Utilização:Med.}
\end{itemize}
\begin{itemize}
\item {Grp. gram.:M.}
\end{itemize}
\begin{itemize}
\item {Proveniência:(It. \textunderscore frusto\textunderscore )}
\end{itemize}
Diz-se de medalha, esculptura ou pedra antiga, cujos caracteres ou lavores estão carcomidos pelo tempo.
Diz-se da fórma leve ou incompleta de uma doença.
Bloco de gêlo. Cf. João Ribeiro, \textunderscore Crepúsculos\textunderscore .
\section{Frustração}
\begin{itemize}
\item {Grp. gram.:f.}
\end{itemize}
\begin{itemize}
\item {Proveniência:(Lat. \textunderscore frustratio\textunderscore )}
\end{itemize}
Acto ou effeito de frustrar.
\section{Frustradamente}
\begin{itemize}
\item {Grp. gram.:adv.}
\end{itemize}
De modo frustrado.
Vanmente; debalde.
\section{Frustrado}
\begin{itemize}
\item {Grp. gram.:adj.}
\end{itemize}
Mallogrado.
Baldado.
Incompleto, imperfeito, que não chegou a desenvolver-se.
\section{Frustrador}
\begin{itemize}
\item {Grp. gram.:m.  e  adj.}
\end{itemize}
\begin{itemize}
\item {Proveniência:(Lat. \textunderscore frustrator\textunderscore )}
\end{itemize}
O que frustra.
\section{Frustraneamente}
\begin{itemize}
\item {Grp. gram.:adv.}
\end{itemize}
De modo frustrâneo.
\section{Frustrâneas}
\begin{itemize}
\item {Grp. gram.:f. pl.}
\end{itemize}
\begin{itemize}
\item {Proveniência:(De \textunderscore frustrâneo\textunderscore )}
\end{itemize}
Nome, dado por Linneu a uma ordem de uma das classes do seu systema, na qual se abrangem synanthéreas, cujas flôres discaes são hermaphrodritas e fecundas, sendo neutras ou femininas e estéreis as da circunferência.
\section{Frustrâneo}
\begin{itemize}
\item {Grp. gram.:adj.}
\end{itemize}
\begin{itemize}
\item {Utilização:Bot.}
\end{itemize}
\begin{itemize}
\item {Proveniência:(Do rad. de \textunderscore frustar\textunderscore )}
\end{itemize}
Frustrado, inútil.
Diz-se das plantas, cujos flósculos não produzem sementes.
\section{Frustrar}
\begin{itemize}
\item {Grp. gram.:v. t.}
\end{itemize}
\begin{itemize}
\item {Grp. gram.:V. p.}
\end{itemize}
\begin{itemize}
\item {Proveniência:(Lat. \textunderscore frustrari\textunderscore )}
\end{itemize}
Enganar a espectativa de.
Illudir.
Baldar; inutilizar: \textunderscore frustrar esforços\textunderscore .
Defraudar.
Mallograr-se.
Ficar sem effeito.
Não succeder (aquillo que se esperava).
\section{Frustratório}
\begin{itemize}
\item {Grp. gram.:adj.}
\end{itemize}
\begin{itemize}
\item {Proveniência:(Lat. \textunderscore frustratorius\textunderscore )}
\end{itemize}
Fallaz; illusório.
Dilatório.
\section{Frustulado}
\begin{itemize}
\item {Grp. gram.:adj.}
\end{itemize}
Dividido em frústulos.
\section{Frústulo}
\begin{itemize}
\item {Grp. gram.:m.}
\end{itemize}
\begin{itemize}
\item {Utilização:Bot.}
\end{itemize}
\begin{itemize}
\item {Proveniência:(Lat. \textunderscore frustulum\textunderscore )}
\end{itemize}
Pequenos corpos ou partículas movediças, que entram na formação de algumas algas.
\section{Fruta}
\begin{itemize}
\item {Grp. gram.:f.}
\end{itemize}
Fruto comestível.
(Cp. \textunderscore fruto\textunderscore )
\section{Fruta-cocta}
\begin{itemize}
\item {Grp. gram.:f.}
\end{itemize}
\begin{itemize}
\item {Utilização:Bot.}
\end{itemize}
Cucurbitácea da Índia portuguesa, (\textunderscore luffa egypciaca\textunderscore , Mill.).
\section{Fruta-de-conde}
\begin{itemize}
\item {Grp. gram.:f.}
\end{itemize}
\begin{itemize}
\item {Utilização:Bras}
\end{itemize}
Planta anonácea, (\textunderscore anona reticulata\textunderscore , Lin.).
Nome de várias espécies de anonas.
\section{Fruta-de-perdiz}
\begin{itemize}
\item {Grp. gram.:f.}
\end{itemize}
\begin{itemize}
\item {Utilização:Bras}
\end{itemize}
Palmeira do mato virgem.
\section{Fruta-de-pomba}
\begin{itemize}
\item {Grp. gram.:f.}
\end{itemize}
Árvore silvestre do Brasil.
\section{Fruta-dos-paulistas}
\begin{itemize}
\item {Grp. gram.:f.}
\end{itemize}
\begin{itemize}
\item {Utilização:Bras}
\end{itemize}
Planta cucurbitácea medicinal.
\section{Fruta-do-tucano}
\begin{itemize}
\item {Grp. gram.:f.}
\end{itemize}
\begin{itemize}
\item {Utilização:Bras}
\end{itemize}
Árvore silvestre, de madeira branca, de que se faz o assento dos tamancos.
\section{Fruta-pão}
\begin{itemize}
\item {Grp. gram.:f.}
\end{itemize}
\begin{itemize}
\item {Utilização:Bras}
\end{itemize}
Árvore artocarpácea, (\textunderscore artocarpus incisa\textunderscore , Lin.).
\section{Frutar}
\begin{itemize}
\item {Grp. gram.:v. t.}
\end{itemize}
\begin{itemize}
\item {Proveniência:(De \textunderscore fruto\textunderscore )}
\end{itemize}
Produzir.
Dar de si.
Dar origem a.
\section{Fruteador}
\begin{itemize}
\item {Grp. gram.:adj.}
\end{itemize}
Que fruteia ou fructifica; que faz fructificar.
\section{Frutear}
\begin{itemize}
\item {Grp. gram.:v. i.}
\end{itemize}
\begin{itemize}
\item {Grp. gram.:V. t.}
\end{itemize}
Dar frutos, fructificar.
Tornar fructífero. Cf. A. Candido, \textunderscore Philos. Polit.\textunderscore , 21.
\section{Frutegar}
\begin{itemize}
\item {Grp. gram.:v. t.}
\end{itemize}
\begin{itemize}
\item {Utilização:Ant.}
\end{itemize}
\begin{itemize}
\item {Proveniência:(Lat. \textunderscore fruticare\textunderscore )}
\end{itemize}
Cultivar; plantar de árvores.
\section{Fruteira}
\begin{itemize}
\item {Grp. gram.:f.}
\end{itemize}
\begin{itemize}
\item {Proveniência:(De \textunderscore fruteiro\textunderscore )}
\end{itemize}
Árvore fructífera.
Vaso ou cestinho, em que se põe a fruta, á mesa. Vendedora de fruta.
Nome de várias plantas brasileiras.
Local, disposto e preparado, para recolher os frutos, depois da colheita.
\section{Fruteiro}
\begin{itemize}
\item {Grp. gram.:m.}
\end{itemize}
\begin{itemize}
\item {Grp. gram.:Adj.}
\end{itemize}
\begin{itemize}
\item {Proveniência:(Do b. lat. \textunderscore fructarius\textunderscore )}
\end{itemize}
Vendedor de fruta.
Prato ou cestinho para fruta.
Lugar, onde se guarda fruta.
Que dá fruto, fructífero.
\section{Frutescência}
\begin{itemize}
\item {Grp. gram.:f.}
\end{itemize}
\begin{itemize}
\item {Proveniência:(Do lat. \textunderscore fructescere\textunderscore )}
\end{itemize}
Época do desenvolvimento dos frutos.
A sua maturação.
\section{Frutescente}
\begin{itemize}
\item {Grp. gram.:adj.}
\end{itemize}
\begin{itemize}
\item {Proveniência:(Lat. \textunderscore fructescens\textunderscore  e \textunderscore frutescens\textunderscore )}
\end{itemize}
Que cria frutos.
Arborescente.
\section{Frútice}
\begin{itemize}
\item {Grp. gram.:m.}
\end{itemize}
\begin{itemize}
\item {Proveniência:(Lat. \textunderscore frutex\textunderscore )}
\end{itemize}
Planta, que não attinge a grandeza de um arbusto.
Arbusto; arvoreta.
\section{Fruticeto}
\begin{itemize}
\item {Grp. gram.:m.}
\end{itemize}
\begin{itemize}
\item {Proveniência:(De \textunderscore frútice\textunderscore )}
\end{itemize}
Horto, plantado de fruteiras.
\section{Fruticoso}
\begin{itemize}
\item {Grp. gram.:adj.}
\end{itemize}
\begin{itemize}
\item {Proveniência:(Lat. \textunderscore fruticosus\textunderscore )}
\end{itemize}
O mesmo que \textunderscore frutescente\textunderscore .
Diz-se especialmente do tronco dos arbustos.
\section{Fruticuloso}
\begin{itemize}
\item {Grp. gram.:adj.}
\end{itemize}
Diz-se do tronco dos sub-arbustos.
(Cp. \textunderscore fruticoso\textunderscore )
\section{Fruticultor}
\begin{itemize}
\item {Grp. gram.:m.}
\end{itemize}
\begin{itemize}
\item {Proveniência:(Do lat. \textunderscore fructus\textunderscore  + \textunderscore cultor\textunderscore )}
\end{itemize}
Cultivador de árvores frutíferas.
Pomareiro.
\section{Fruticultura}
\begin{itemize}
\item {Grp. gram.:f.}
\end{itemize}
\begin{itemize}
\item {Proveniência:(Do lat. \textunderscore fructus\textunderscore  + \textunderscore cultura\textunderscore )}
\end{itemize}
Cultura de pomares ou de árvores frutíferas.
\section{Frutidor}
\begin{itemize}
\item {Grp. gram.:m.}
\end{itemize}
\begin{itemize}
\item {Proveniência:(Do lat. \textunderscore fructus\textunderscore  + gr. \textunderscore doron\textunderscore )}
\end{itemize}
Duodecimo mês do calendário da primeira república francesa, (18 de Agosto a 16 de Setembro).
\section{Frutif...}
O mesmo que \textunderscore fructif...\textunderscore 
\section{Frutífero}
\begin{itemize}
\item {Grp. gram.:adj.}
\end{itemize}
\begin{itemize}
\item {Utilização:Fig.}
\end{itemize}
\begin{itemize}
\item {Proveniência:(Lat. \textunderscore fructifer\textunderscore )}
\end{itemize}
Que dá frutos.
Proveitoso, útil.
\section{Frutificação}
\begin{itemize}
\item {Grp. gram.:f.}
\end{itemize}
\begin{itemize}
\item {Proveniência:(Lat. \textunderscore fructificatio\textunderscore )}
\end{itemize}
Acto ou efeito de frutificar.
\section{Frutificamento}
\begin{itemize}
\item {Grp. gram.:m.}
\end{itemize}
\begin{itemize}
\item {Utilização:Des.}
\end{itemize}
O mesmo que \textunderscore frutificação\textunderscore .
\section{Frutificar}
\begin{itemize}
\item {Grp. gram.:v. i.}
\end{itemize}
\begin{itemize}
\item {Utilização:Fig.}
\end{itemize}
\begin{itemize}
\item {Proveniência:(Lat. \textunderscore fructificare\textunderscore )}
\end{itemize}
Produzir frutos.
Dar resultado: \textunderscore o seu trabalho frutificou\textunderscore .
Sêr útil.
\section{Frutificativo}
\begin{itemize}
\item {Grp. gram.:adj.}
\end{itemize}
O mesmo que \textunderscore frutífero\textunderscore .
\section{Frutifloras}
\begin{itemize}
\item {Grp. gram.:f. pl.}
\end{itemize}
\begin{itemize}
\item {Utilização:Bot.}
\end{itemize}
\begin{itemize}
\item {Proveniência:(De \textunderscore frutifloro\textunderscore )}
\end{itemize}
Classe de plantas, que têm os estames sôbre os pistilos.
\section{Frutifloro}
\begin{itemize}
\item {Grp. gram.:adj.}
\end{itemize}
\begin{itemize}
\item {Utilização:Bot.}
\end{itemize}
\begin{itemize}
\item {Proveniência:(Do lat. \textunderscore fructus\textunderscore  + \textunderscore flos\textunderscore )}
\end{itemize}
Diz-se das plantas que têm o ovário livre.
\section{Frutiforme}
\begin{itemize}
\item {Grp. gram.:adj.}
\end{itemize}
\begin{itemize}
\item {Proveniência:(Do lat. \textunderscore fructus\textunderscore  + \textunderscore forma\textunderscore )}
\end{itemize}
Que tem fórma de fruto.
\section{Frutígero}
\begin{itemize}
\item {Grp. gram.:adj.}
\end{itemize}
\begin{itemize}
\item {Proveniência:(Do lat. \textunderscore fructus\textunderscore  + \textunderscore gerere\textunderscore )}
\end{itemize}
O mesmo que \textunderscore frutífero\textunderscore .
\section{Frutívoro}
\begin{itemize}
\item {Grp. gram.:adj.}
\end{itemize}
\begin{itemize}
\item {Proveniência:(Do lat. \textunderscore fructus\textunderscore  + \textunderscore vorare\textunderscore )}
\end{itemize}
Que se alimenta de frutos.
\section{Fruto}
\begin{itemize}
\item {Grp. gram.:m.}
\end{itemize}
\begin{itemize}
\item {Proveniência:(Lat. \textunderscore fructus\textunderscore )}
\end{itemize}
Tudo que a terra produz, para sustentação ou benefício do homem.
Parte productiva dos vegetaes, saída da flôr.
Prole.
Lucro; resultado, producto.
Rendimento.
Vantagem.
\section{Frutuária}
\begin{itemize}
\item {Grp. gram.:f.}
\end{itemize}
\begin{itemize}
\item {Proveniência:(De \textunderscore frutuário\textunderscore )}
\end{itemize}
Associação suíça de pequenos industriaes, para o fabríco do queijo, distribuíndo-se o lucro ou fruto proporcionalmente ao leite, com que cada associado contribuiu.
Instituição oficial, tentada há pouco entre nós, para exploração de lacticínios por conta do Estado.
Impropriamente, fábrica de lacticínios.
\section{Frutuário}
\begin{itemize}
\item {Grp. gram.:adj.}
\end{itemize}
\begin{itemize}
\item {Proveniência:(Lat. \textunderscore fructuarius\textunderscore )}
\end{itemize}
Relativo a frutos.
Fértil, fecundo.
Que dá bons resultados.
\section{Frutuosamente}
\begin{itemize}
\item {Grp. gram.:adv.}
\end{itemize}
De modo frutuoso.
\section{Frutuoso}
\begin{itemize}
\item {Grp. gram.:adj.}
\end{itemize}
\begin{itemize}
\item {Utilização:Fig.}
\end{itemize}
\begin{itemize}
\item {Proveniência:(Lat. \textunderscore fructuosus\textunderscore )}
\end{itemize}
Abundante em frutos.
Fecundante.
Proveitoso, útil.
\section{Fruxo}
\begin{itemize}
\item {Grp. gram.:m.}
\end{itemize}
\begin{itemize}
\item {Utilização:Ant.}
\end{itemize}
O mesmo que \textunderscore fluxo\textunderscore .
\section{Fu!}
\begin{itemize}
\item {Grp. gram.:interj.}
\end{itemize}
(designativa de nojo ou desprêzo)
\section{Fuá}
\begin{itemize}
\item {Grp. gram.:adj.}
\end{itemize}
\begin{itemize}
\item {Utilização:Bras}
\end{itemize}
\begin{itemize}
\item {Grp. gram.:M.}
\end{itemize}
\begin{itemize}
\item {Utilização:Bras. do N}
\end{itemize}
Diz-se do cavallo espantadiço e manhoso.
Caspa.
Pó tenuíssimo, que se desprende da pelle, quando esta é arranhada.
\section{Fuan}
\begin{itemize}
\item {Grp. gram.:f.}
\end{itemize}
(Flexão fem. de \textunderscore fuão\textunderscore ). Cf. Sousa, \textunderscore Vida do Arceb.\textunderscore , II, 70.
\section{Fuão}
\begin{itemize}
\item {Grp. gram.:m.}
\end{itemize}
(Contr. de \textunderscore fulano\textunderscore )
\section{Fuba}
\begin{itemize}
\item {Grp. gram.:f.}
\end{itemize}
Bebida vulgar entre os africanos, obtida com seiva vegetal. Cf. Serpa Pinto, I, 146.
Farinha da raiz de mandioca, com que se fazem as papas, chamadas \textunderscore infúndi\textunderscore . Cf. Capello e Ivens, I, 332.
\section{Fubá}
\begin{itemize}
\item {Grp. gram.:m.}
\end{itemize}
\begin{itemize}
\item {Utilização:Bras}
\end{itemize}
\begin{itemize}
\item {Grp. gram.:Adj.}
\end{itemize}
\begin{itemize}
\item {Utilização:Bras. do N}
\end{itemize}
Farinha para papas.
Diz-se do boi e da vaca alvacentos.
(Cp. \textunderscore fuba\textunderscore )
\section{Fubana}
\begin{itemize}
\item {Grp. gram.:f.}
\end{itemize}
\begin{itemize}
\item {Utilização:Bras. do N}
\end{itemize}
O mesmo que \textunderscore meretriz\textunderscore .
\section{Fubeca}
\begin{itemize}
\item {Grp. gram.:f.}
\end{itemize}
\begin{itemize}
\item {Utilização:Bras}
\end{itemize}
Sova, tunda.
Descompostura.
\section{Fuça}
\begin{itemize}
\item {Grp. gram.:f.}
\end{itemize}
\begin{itemize}
\item {Utilização:Chul.}
\end{itemize}
O mesmo que \textunderscore focinho\textunderscore , ventas.
\section{Fucáceas}
\begin{itemize}
\item {Grp. gram.:f. pl.}
\end{itemize}
Família de algas, que tem por typo o fuco.
\section{Fucamena}
\begin{itemize}
\item {Grp. gram.:f.}
\end{itemize}
Árvore brasileira, de fôlhas largas e crespas.
\section{Fucansengo}
\begin{itemize}
\item {Grp. gram.:m.}
\end{itemize}
Planta africana, trepadeira, de fôlhas muito distanciadas e flôres papilionáceas, côr de canário.
\section{Fúcaro}
\begin{itemize}
\item {Grp. gram.:m.}
\end{itemize}
\begin{itemize}
\item {Utilização:Ant.}
\end{itemize}
\begin{itemize}
\item {Proveniência:(Do cast. \textunderscore fúcar\textunderscore )}
\end{itemize}
Homem muito rico; argentário.
\section{Fúceas}
\begin{itemize}
\item {Grp. gram.:f. pl.}
\end{itemize}
\begin{itemize}
\item {Proveniência:(Do lat. \textunderscore fucus\textunderscore )}
\end{itemize}
O mesmo que \textunderscore hydróphytas\textunderscore .
\section{Fucense}
\begin{itemize}
\item {Grp. gram.:adj.}
\end{itemize}
\begin{itemize}
\item {Utilização:Agr.}
\end{itemize}
Diz-se de uma nova espécie de milho graúdo. Cf. \textunderscore Archivo Rur.\textunderscore , VI, 18.
\section{Fuchicar}
\begin{itemize}
\item {Grp. gram.:v. t.}
\end{itemize}
\begin{itemize}
\item {Utilização:Bras}
\end{itemize}
Emmaranhar ou revolver, produzindo rumor, como em papel sêco.
(Por \textunderscore fussicar\textunderscore , de \textunderscore fossar\textunderscore ?)
\section{Fúchsia}
\begin{itemize}
\item {Grp. gram.:f.}
\end{itemize}
Gênero de plantas onagrárias, vulgarmente conhecidas por \textunderscore brincos de princesa\textunderscore .
(Do \textunderscore Fuchs\textunderscore , n. p.)
\section{Fuchsina}
\begin{itemize}
\item {Grp. gram.:f.}
\end{itemize}
\begin{itemize}
\item {Proveniência:(De \textunderscore fúchsia\textunderscore )}
\end{itemize}
Substância encarnada, que se extrai da anilina.
\section{Fucícola}
\begin{itemize}
\item {Grp. gram.:adj.}
\end{itemize}
\begin{itemize}
\item {Proveniência:(Do lat. \textunderscore fucus\textunderscore  + \textunderscore colere\textunderscore )}
\end{itemize}
Que vive entre os fucos.
\section{Fuciforme}
\begin{itemize}
\item {Grp. gram.:adj.}
\end{itemize}
\begin{itemize}
\item {Proveniência:(Do lat. \textunderscore fucus\textunderscore  + \textunderscore forma\textunderscore )}
\end{itemize}
Que tem fórma de fuco.
\section{Fuco}
\begin{itemize}
\item {Grp. gram.:m.}
\end{itemize}
\begin{itemize}
\item {Utilização:Fig.}
\end{itemize}
\begin{itemize}
\item {Proveniência:(Lat. \textunderscore fucus\textunderscore )}
\end{itemize}
Espécie de alga marítima ou sargaço, de que se extrai uma substância, empregada em tinturaria.
Tintura para o rosto.
Arrebique feminino.
Disfarce, impostura.
\section{Fuco}
\begin{itemize}
\item {Grp. gram.:m.}
\end{itemize}
(V. \textunderscore fungo\textunderscore ^3)
\section{Fucoide}
\begin{itemize}
\item {Grp. gram.:adj.}
\end{itemize}
\begin{itemize}
\item {Proveniência:(Do gr. \textunderscore phukos\textunderscore  + \textunderscore eidos\textunderscore )}
\end{itemize}
O mesmo que \textunderscore fuciforme\textunderscore .
\section{Fucoídeas}
\begin{itemize}
\item {Grp. gram.:f. pl.}
\end{itemize}
O mesmo que \textunderscore fucáceas\textunderscore .
\section{Fúcsia}
\begin{itemize}
\item {Grp. gram.:f.}
\end{itemize}
Gênero de plantas onagrárias, vulgarmente conhecidas por \textunderscore brincos de princesa\textunderscore .
(Do \textunderscore Fuchs\textunderscore , n. p.)
\section{Fucsina}
\begin{itemize}
\item {Grp. gram.:f.}
\end{itemize}
\begin{itemize}
\item {Proveniência:(De \textunderscore fúcsia\textunderscore )}
\end{itemize}
Substância encarnada, que se extrai da anilina.
\section{Fueirada}
\begin{itemize}
\item {Grp. gram.:f.}
\end{itemize}
Pancada com fueiro^1.
\section{Fueireta}
\begin{itemize}
\item {fónica:eirê}
\end{itemize}
\begin{itemize}
\item {Grp. gram.:f.}
\end{itemize}
\begin{itemize}
\item {Utilização:Prov.}
\end{itemize}
\begin{itemize}
\item {Utilização:minh.}
\end{itemize}
Fueiro^1 pequeno ou fraco. (Colhido em Barcelos)
\section{Fueiro}
\begin{itemize}
\item {Grp. gram.:m.}
\end{itemize}
\begin{itemize}
\item {Proveniência:(Do lat. \textunderscore funarius\textunderscore )}
\end{itemize}
Cada uma das estacas, que, tendo a extremidade inferior segura no chedeiro do carro, servem para amparar a carga.
Estadulho.
\section{Fueiro}
\begin{itemize}
\item {Grp. gram.:m.}
\end{itemize}
\begin{itemize}
\item {Utilização:Bras. do N}
\end{itemize}
Parte da barriga do cavallo, entre o umbigo e os escrotos.
\section{Fueta}
\begin{itemize}
\item {fónica:fuê}
\end{itemize}
\begin{itemize}
\item {Grp. gram.:f.}
\end{itemize}
O mesmo que \textunderscore toirão\textunderscore .
\section{Fúfia}
\begin{itemize}
\item {Grp. gram.:f.}
\end{itemize}
\begin{itemize}
\item {Utilização:Chul.}
\end{itemize}
\begin{itemize}
\item {Grp. gram.:M.  e  f.}
\end{itemize}
Mulher pretensiosa e ridícula.
Pessôa sem mérito, mas engrandecida pelo acaso.
Empáfia.
\section{Fúfio}
\begin{itemize}
\item {Grp. gram.:adj.}
\end{itemize}
Reles, ordinário. Cf. Filinto, IV, 244.
(Cp. \textunderscore fúfia\textunderscore )
\section{Fuga}
\begin{itemize}
\item {Grp. gram.:f.}
\end{itemize}
\begin{itemize}
\item {Utilização:Prov.}
\end{itemize}
\begin{itemize}
\item {Utilização:trasm.}
\end{itemize}
\begin{itemize}
\item {Utilização:Mathem.}
\end{itemize}
\begin{itemize}
\item {Proveniência:(Lat. \textunderscore fuga\textunderscore )}
\end{itemize}
Acto ou effeito de fugir.
Saída; retirada.
Subterfúgio.
Orifício, por onde o folle toma vento.
Orifício dos apparelhos de destillação.
Composição musical, cujas diversas partes correspondem a diversos assumptos e que se ligam com grande difficuldade.
Parte da rabiça, entre o teiró e o ferro.
\textunderscore Pontos de fuga\textunderscore , pontos de duas rectas, divididas homographicamente, tendo cada um de uma dellas, por ponto homólogo na outra, o ponto ao infinito.
\section{Fugace}
\begin{itemize}
\item {Grp. gram.:adj.}
\end{itemize}
(V.fugaz)
\section{Fugacidade}
\begin{itemize}
\item {Grp. gram.:f.}
\end{itemize}
\begin{itemize}
\item {Proveniência:(Lat. \textunderscore fugacitas\textunderscore )}
\end{itemize}
Fuga, rapidez.
Qualidade daquillo que é fugaz.
\section{Fugafina}
\begin{itemize}
\item {Grp. gram.:f.}
\end{itemize}
Planta fantástica:«\textunderscore chá de flôr de fugafina, sustância de corritana.\textunderscore »Castilho, \textunderscore Méd. á Fôrça\textunderscore , 186.
\section{Fugalaça}
\begin{itemize}
\item {Grp. gram.:f.}
\end{itemize}
\begin{itemize}
\item {Utilização:Fig.}
\end{itemize}
\begin{itemize}
\item {Proveniência:(De \textunderscore fuga\textunderscore  + \textunderscore laçar\textunderscore )}
\end{itemize}
Corda comprida, que se atira aos animaes para os prender, dando-se-lhes folga para correr até perderem as fôrças.
Adiamento ou prazo, para se realizar alguma coisa.
\section{Fugar}
\begin{itemize}
\item {Grp. gram.:v. t.}
\end{itemize}
\begin{itemize}
\item {Utilização:Des.}
\end{itemize}
\begin{itemize}
\item {Proveniência:(Lat. \textunderscore fugare\textunderscore )}
\end{itemize}
Pôr em fuga.
Afugentar. Cf. \textunderscore Viriato Trág.\textunderscore , XIV, 84.
\section{Fugaz}
\begin{itemize}
\item {Grp. gram.:adj.}
\end{itemize}
\begin{itemize}
\item {Proveniência:(Lat. \textunderscore fugax\textunderscore )}
\end{itemize}
Que foge com rapidez.
Rápido, veloz.
Transitório.
\section{Fugazmente}
\begin{itemize}
\item {Grp. gram.:adv.}
\end{itemize}
De modo fugaz.
Rapidamente.
\section{Fuge}
\begin{itemize}
\item {Grp. gram.:m.}
\end{itemize}
O mesmo que \textunderscore quifuge\textunderscore .
\section{Fugeca}
\begin{itemize}
\item {Grp. gram.:f.}
\end{itemize}
\begin{itemize}
\item {Utilização:Gír.}
\end{itemize}
Mêdo; covardia.
(Relaciona-se com \textunderscore fugir\textunderscore )
\section{Fugente}
\begin{itemize}
\item {Grp. gram.:adj.}
\end{itemize}
\begin{itemize}
\item {Proveniência:(De \textunderscore fugir\textunderscore )}
\end{itemize}
Que parece fugir á vista, (em pintura)--Sería preferível \textunderscore fugiente\textunderscore .
\section{Fugião}
\begin{itemize}
\item {Grp. gram.:adj.}
\end{itemize}
\begin{itemize}
\item {Utilização:Ant.}
\end{itemize}
O mesmo que \textunderscore fujão\textunderscore .
\section{Fugida}
\begin{itemize}
\item {Grp. gram.:f.}
\end{itemize}
Acto ou effeito de fugir.
Fuga.
\section{Fugidiço}
\begin{itemize}
\item {Grp. gram.:adj.}
\end{itemize}
\begin{itemize}
\item {Proveniência:(De \textunderscore fugir\textunderscore )}
\end{itemize}
Acostumado a fugir; fugitivo.
Desertor.
Que se desvanece, que se some rapidamente: \textunderscore esperança fugidia\textunderscore .
Esquivo, arisco.
\section{Fugidio}
\begin{itemize}
\item {Grp. gram.:adj.}
\end{itemize}
\begin{itemize}
\item {Proveniência:(De \textunderscore fugir\textunderscore )}
\end{itemize}
Acostumado a fugir; fugitivo.
Desertor.
Que se desvanece, que se some rapidamente: \textunderscore esperança fugidia\textunderscore .
Esquivo, arisco.
\section{Fugiente}
\begin{itemize}
\item {Grp. gram.:adj.}
\end{itemize}
\begin{itemize}
\item {Proveniência:(Lat. \textunderscore fugiens\textunderscore )}
\end{itemize}
Que foge, que se afasta.
Que se vai perdendo de vista. Cf. Filinto, V, 275.
\section{Fugimento}
\begin{itemize}
\item {Grp. gram.:m.}
\end{itemize}
\begin{itemize}
\item {Utilização:Ant.}
\end{itemize}
\begin{itemize}
\item {Proveniência:(De \textunderscore fugir\textunderscore )}
\end{itemize}
O mesmo que \textunderscore fuga\textunderscore .
\section{Fuginte}
\begin{itemize}
\item {Grp. gram.:adj.}
\end{itemize}
O mesmo que \textunderscore fugiente\textunderscore . Cf. Filinto, VI, 98.
\section{Fugir}
\begin{itemize}
\item {Grp. gram.:v. i.}
\end{itemize}
\begin{itemize}
\item {Utilização:Prov.}
\end{itemize}
\begin{itemize}
\item {Grp. gram.:V. t.}
\end{itemize}
\begin{itemize}
\item {Proveniência:(Lat. \textunderscore fugere\textunderscore )}
\end{itemize}
Desviar-se rapidamente; livrar-se: \textunderscore fugir de um perigo\textunderscore .
Evitar alguma coisa: \textunderscore fugir das más companhias\textunderscore .
Escapar-se.
Desapparecer.
Correr rapidamente.
Escoar-se.
O mesmo que \textunderscore correr\textunderscore .
Evitar; escapar de.
Esquivar-se a: \textunderscore fugir de impertinências\textunderscore .
\section{Fugitivário}
\begin{itemize}
\item {Grp. gram.:m.}
\end{itemize}
\begin{itemize}
\item {Utilização:Ant.}
\end{itemize}
\begin{itemize}
\item {Proveniência:(Lat. \textunderscore fugitivarius\textunderscore )}
\end{itemize}
Aquelle que procurava os escravos fugitivos.
\section{Fugitivo}
\begin{itemize}
\item {Grp. gram.:adj.}
\end{itemize}
\begin{itemize}
\item {Grp. gram.:M.}
\end{itemize}
\begin{itemize}
\item {Proveniência:(Lat. \textunderscore fugitivus\textunderscore )}
\end{itemize}
Que fugiu.
Que desertou.
Fugaz; transitório.
Rápido.
Indeciso; que se entrevê apenas.
Indivíduo fugitivo; desertor.
\section{Fugueiro}
\begin{itemize}
\item {Grp. gram.:m.}
\end{itemize}
\begin{itemize}
\item {Utilização:Prov.}
\end{itemize}
O mesmo que \textunderscore fueiro\textunderscore ^1.
\section{Fuim}
\begin{itemize}
\item {Grp. gram.:m.}
\end{itemize}
\begin{itemize}
\item {Utilização:Prov.}
\end{itemize}
\begin{itemize}
\item {Utilização:alg.}
\end{itemize}
O mesmo que \textunderscore chincra\textunderscore .
\section{Fuínha}
\begin{itemize}
\item {Grp. gram.:f.}
\end{itemize}
\begin{itemize}
\item {Utilização:Prov.}
\end{itemize}
\begin{itemize}
\item {Grp. gram.:M.  e  f.}
\end{itemize}
\begin{itemize}
\item {Proveniência:(Do b. lat. \textunderscore fuina\textunderscore )}
\end{itemize}
Pequeno animal mammífero, damninho carnívoro, (\textunderscore mustela foina\textunderscore ).
O mesmo que \textunderscore folosa\textunderscore .
O mesmo que \textunderscore fuínho\textunderscore .
Pessôa avarenta.
Pessôa, magra como o fuínho.
Pessôa intriguista, mexeriqueira.
\section{Fuínhas}
\begin{itemize}
\item {Grp. gram.:m.  e  f.}
\end{itemize}
Pessôa magra e sovina.
(Cp. \textunderscore fuínha\textunderscore )
\section{Fuínho}
\begin{itemize}
\item {Grp. gram.:m.}
\end{itemize}
O mesmo que \textunderscore picancilho\textunderscore .
\section{Fujão}
\begin{itemize}
\item {Grp. gram.:m.  e  adj.}
\end{itemize}
\begin{itemize}
\item {Proveniência:(Do rad. de \textunderscore fugir\textunderscore )}
\end{itemize}
Indivíduo fugidiço.
\section{Fula}
\begin{itemize}
\item {Grp. gram.:f.}
\end{itemize}
\begin{itemize}
\item {Grp. gram.:Loc. adv.}
\end{itemize}
Pressa.
Empôla.
Cada uma das cavidades buccaes, onde se acumula a comida, quando se mastiga.
Grande quantidade.
\textunderscore Á fula\textunderscore , á pressa, com precipitação.
\section{Fula}
\begin{itemize}
\item {Grp. gram.:f.}
\end{itemize}
Preparação do feltro para chapéus.
Apparelho, para calandrar panos.
(Cp. lat. \textunderscore fullo\textunderscore )
\section{Fula}
\begin{itemize}
\item {Grp. gram.:m.}
\end{itemize}
Língua dos Fulas, na África.
\section{Fula}
\begin{itemize}
\item {Grp. gram.:f.}
\end{itemize}
\begin{itemize}
\item {Proveniência:(T. conc., que significa \textunderscore flôr\textunderscore )}
\end{itemize}
Nome, que, na Índia portuguesa, se dá á angélica branca e a outras plantas, \textunderscore fula-moirisca\textunderscore , \textunderscore fula-pipa\textunderscore , etc. Cf. D. G. Dalgado, \textunderscore Flora de Gôa\textunderscore .
\section{Fula-fula}
\begin{itemize}
\item {Grp. gram.:f.}
\end{itemize}
Muita pressa.
Confusão.
(Cp. \textunderscore fula\textunderscore ^1, e \textunderscore lufa-lufa\textunderscore )
\section{Fulagassa}
\begin{itemize}
\item {Grp. gram.:f.}
\end{itemize}
\begin{itemize}
\item {Utilização:Náut.}
\end{itemize}
O mesmo que \textunderscore falcassa\textunderscore .
(Provavelmente, metáth. de \textunderscore fugalassa\textunderscore , fórma incorrecta de \textunderscore fugalaça\textunderscore )
\section{Fulame}
\begin{itemize}
\item {Grp. gram.:m.}
\end{itemize}
\begin{itemize}
\item {Utilização:Ant.}
\end{itemize}
\begin{itemize}
\item {Proveniência:(De \textunderscore fula\textunderscore ^2)}
\end{itemize}
Porção de feltro para chapéus.
\section{Fulano}
\begin{itemize}
\item {Grp. gram.:m.}
\end{itemize}
\begin{itemize}
\item {Proveniência:(Do ár. \textunderscore fulan\textunderscore )}
\end{itemize}
Designação vaga de pessôa incerta ou de alguém que se não quer nomear.
\section{Fulão}
\begin{itemize}
\item {Grp. gram.:m.}
\end{itemize}
\begin{itemize}
\item {Utilização:Prov.}
\end{itemize}
\begin{itemize}
\item {Utilização:minh.}
\end{itemize}
\begin{itemize}
\item {Proveniência:(De \textunderscore fula\textunderscore ^2)}
\end{itemize}
Espécie de caldeira, para enfortir a fula dos chapeleiros.
Pisão, moínho de pisar panos.
\section{Fulão}
\begin{itemize}
\item {Grp. gram.:m.}
\end{itemize}
\begin{itemize}
\item {Utilização:Ant.}
\end{itemize}
O mesmo que \textunderscore fuão\textunderscore .
\section{Fulas}
\begin{itemize}
\item {Grp. gram.:m. pl.}
\end{itemize}
Povo da Senegâmbia, que parece proceder de uma mistura da raça branca com a raça negra.
\section{Fulcrado}
\begin{itemize}
\item {Grp. gram.:adj.}
\end{itemize}
\begin{itemize}
\item {Utilização:Bot.}
\end{itemize}
\begin{itemize}
\item {Proveniência:(De \textunderscore fulcro\textunderscore )}
\end{itemize}
Que produz novo caule, (falando-se do caule, cujas raízes, mergulhando na terra, dão origem a outro).
\section{Fulcro}
\begin{itemize}
\item {Grp. gram.:m.}
\end{itemize}
\begin{itemize}
\item {Utilização:Bot.}
\end{itemize}
\begin{itemize}
\item {Utilização:Náut.}
\end{itemize}
\begin{itemize}
\item {Proveniência:(Lat. \textunderscore fulcrum\textunderscore )}
\end{itemize}
Sustentáculo, apoio, amparo.
Designação genérica dos órgãos que protegem ou facilitam a vegetação.
O mesmo que \textunderscore tolete\textunderscore .
\section{Fulda}
\begin{itemize}
\item {Grp. gram.:f.}
\end{itemize}
\begin{itemize}
\item {Proveniência:(De \textunderscore Fulda\textunderscore , n. p.?)}
\end{itemize}
Túnica pontifícia, branca.
\section{Fulecar}
\begin{itemize}
\item {Grp. gram.:v. i.}
\end{itemize}
\begin{itemize}
\item {Utilização:Bras}
\end{itemize}
Perder, ao jôgo, todo o dinheiro que se levava.
\section{Fulecra}
\begin{itemize}
\item {Grp. gram.:f.}
\end{itemize}
\begin{itemize}
\item {Utilização:Prov.}
\end{itemize}
\begin{itemize}
\item {Utilização:trasm.}
\end{itemize}
\begin{itemize}
\item {Utilização:Fig.}
\end{itemize}
Espécie de pássaro pequenino e muito vivo.
Rapariga de pequeno corpo e leviana.
\section{Fulgência}
\begin{itemize}
\item {Grp. gram.:f.}
\end{itemize}
Qualidade daquillo que é fulgente.
\section{Fulgente}
\begin{itemize}
\item {Grp. gram.:adj.}
\end{itemize}
\begin{itemize}
\item {Proveniência:(Lat. \textunderscore fulgens\textunderscore )}
\end{itemize}
Que brilha, que tem fulgor, que fulge.
\section{Fulgentear}
\begin{itemize}
\item {Grp. gram.:v. t.}
\end{itemize}
Tornar fulgente.
Abrilhantar. Cf. Sousa Monteiro, \textunderscore Elog. de Latino\textunderscore .
\section{Fúlgido}
\begin{itemize}
\item {Grp. gram.:adj.}
\end{itemize}
\begin{itemize}
\item {Proveniência:(Lat. \textunderscore fulgidus\textunderscore )}
\end{itemize}
Que tem fulgor, brilho.
Fulgente.
\section{Fulgir}
\begin{itemize}
\item {Grp. gram.:v. t.}
\end{itemize}
\begin{itemize}
\item {Grp. gram.:V. i.}
\end{itemize}
\begin{itemize}
\item {Utilização:Fig.}
\end{itemize}
\begin{itemize}
\item {Proveniência:(Lat. \textunderscore fulgere\textunderscore )}
\end{itemize}
Fazer brilhar.
Brilhar.
Têr fulgor.
Tornar-se distinto, sobresaír.
\section{Fulgor}
\begin{itemize}
\item {Grp. gram.:m.}
\end{itemize}
\begin{itemize}
\item {Proveniência:(Lat. \textunderscore fulgor\textunderscore )}
\end{itemize}
Brilho.
Clarão.
Esplendor.
Luzeiro.
\section{Fulguração}
\begin{itemize}
\item {Grp. gram.:f.}
\end{itemize}
\begin{itemize}
\item {Utilização:Med.}
\end{itemize}
\begin{itemize}
\item {Proveniência:(Lat. \textunderscore fulguratio\textunderscore )}
\end{itemize}
Clarão, produzido na atmosphera pela electricidade, sem sêr acompanhado de estampido.
Scintillação; clarão rápido.
Perturbação, produzida no organismo vivo por descarga eléctrica, especialmente pelo raio.
\section{Fulgural}
\begin{itemize}
\item {Grp. gram.:adj.}
\end{itemize}
\begin{itemize}
\item {Proveniência:(Lat. \textunderscore fulguralis\textunderscore )}
\end{itemize}
Relativo ao raio ou a relâmpagos.
\section{Fulgurância}
\begin{itemize}
\item {Grp. gram.:f.}
\end{itemize}
Qualidade daquillo que é fulgurante. Cf. Alves Mendes, \textunderscore Herculano\textunderscore , 43.
\section{Fulgurante}
\begin{itemize}
\item {Grp. gram.:adj.}
\end{itemize}
\begin{itemize}
\item {Proveniência:(Lat. \textunderscore fulgurans\textunderscore )}
\end{itemize}
Que fulgura.
\section{Fulgurar}
\begin{itemize}
\item {Grp. gram.:v. i.}
\end{itemize}
\begin{itemize}
\item {Utilização:Fig.}
\end{itemize}
\begin{itemize}
\item {Proveniência:(Lat. \textunderscore fulgurare\textunderscore )}
\end{itemize}
Relampejar.
Fulgir.
Sobresaír.
\section{Fúlguras}
\begin{itemize}
\item {Grp. gram.:f. pl.}
\end{itemize}
O mesmo que \textunderscore fúlguros\textunderscore .
\section{Fulguri-crinante}
\begin{itemize}
\item {Grp. gram.:adj.}
\end{itemize}
\begin{itemize}
\item {Utilização:Poét.}
\end{itemize}
Relativo a cabellos que fulguram:«\textunderscore ...sobre as soltas bellas fulguri-crinantes tranças\textunderscore ». Garção.
\section{Fulgurite}
\begin{itemize}
\item {Grp. gram.:f.}
\end{itemize}
\begin{itemize}
\item {Proveniência:(Do lat. \textunderscore fulgur\textunderscore )}
\end{itemize}
Vitrificação, produzida na areia pela passagem do raio.
Explosivo, de invenção moderníssima, e cujos elementos são por ora desconhecidos do público.
\section{Fulgurómetro}
\begin{itemize}
\item {Grp. gram.:m.}
\end{itemize}
\begin{itemize}
\item {Proveniência:(T. hybr., do lat. \textunderscore fulgur\textunderscore  + gr. \textunderscore metron\textunderscore )}
\end{itemize}
Apparelho, para medir a intensidade da electricidade, em occasião de trovoada.
\section{Fúlguros}
\begin{itemize}
\item {Grp. gram.:m. pl.}
\end{itemize}
\begin{itemize}
\item {Proveniência:(Do lat. \textunderscore fulgur\textunderscore )}
\end{itemize}
Insectos hemípteros, que durante a noite expedem um brilho phosphorescente.
\section{Fulgurosamente}
\begin{itemize}
\item {Grp. gram.:adv.}
\end{itemize}
De modo fulguroso.
\section{Fulguroso}
\begin{itemize}
\item {Grp. gram.:adj.}
\end{itemize}
\begin{itemize}
\item {Proveniência:(Do lat. \textunderscore fulgur\textunderscore )}
\end{itemize}
O mesmo que \textunderscore fulgurante\textunderscore .
\section{Fulha-fulha}
\begin{itemize}
\item {Grp. gram.:m.}
\end{itemize}
\begin{itemize}
\item {Utilização:Açor}
\end{itemize}
Homem apressado.
(Cp. \textunderscore fula-fula\textunderscore )
\section{Fulharia}
\begin{itemize}
\item {Grp. gram.:f.}
\end{itemize}
(V.fulheira)
\section{Fulheira}
\begin{itemize}
\item {Grp. gram.:f.}
\end{itemize}
\begin{itemize}
\item {Proveniência:(Do rad. de \textunderscore fulheiro\textunderscore )}
\end{itemize}
Trapaça ao jôgo.
\section{Fulheiro}
\begin{itemize}
\item {Grp. gram.:m.  e  adj.}
\end{itemize}
O que faz trapaça ao jôgo.
(Cast. \textunderscore fullero\textunderscore )
\section{Fulicárias}
\begin{itemize}
\item {Grp. gram.:f. pl.}
\end{itemize}
\begin{itemize}
\item {Proveniência:(Do lat. \textunderscore fulica\textunderscore )}
\end{itemize}
Família de aves, que têm por typo a gaivota.
\section{Fuligem}
\begin{itemize}
\item {Grp. gram.:f.}
\end{itemize}
\begin{itemize}
\item {Proveniência:(Lat. \textunderscore fuligo\textunderscore )}
\end{itemize}
Substância escura, que, resultando da decomposição dos combustíveis, vai depositar-se nas paredes e tecto das cozinhas ou nos canos das chaminés.
\section{Fuliginosidade}
\begin{itemize}
\item {Grp. gram.:f.}
\end{itemize}
Qualidade daquillo que é fuliginoso.
\section{Fuliginoso}
\begin{itemize}
\item {Grp. gram.:adj.}
\end{itemize}
\begin{itemize}
\item {Proveniência:(Lat. \textunderscore fuliginosus\textunderscore )}
\end{itemize}
Que tem fuligem.
Denegrido pela fuligem.
Diz-se dos dentes, língua, etc., quando se cobrem de uma crosta escura, em certas enfermidades.
\section{Fulineiro}
\begin{itemize}
\item {Grp. gram.:m.}
\end{itemize}
\begin{itemize}
\item {Utilização:Pop.}
\end{itemize}
O mesmo que \textunderscore funileiro\textunderscore .
(Metáth.)
\section{Fulista}
\begin{itemize}
\item {Grp. gram.:m.}
\end{itemize}
\begin{itemize}
\item {Proveniência:(De \textunderscore fula\textunderscore ^2)}
\end{itemize}
Official de chapelaria, encarregado de preparar os feltros.
\section{Fulmi-algodão}
\begin{itemize}
\item {Grp. gram.:m.}
\end{itemize}
Explosivo, o mesmo que \textunderscore algodão-pólvora\textunderscore .
\section{Fulmi-farelo}
\begin{itemize}
\item {Grp. gram.:m.}
\end{itemize}
Nitrocellulose de farelo.
\section{Fulmi-lenho}
\begin{itemize}
\item {Grp. gram.:m.}
\end{itemize}
Nitrocellulose de madeira.
\section{Fulminação}
\begin{itemize}
\item {Grp. gram.:f.}
\end{itemize}
\begin{itemize}
\item {Proveniência:(Lat. \textunderscore fulminatio\textunderscore )}
\end{itemize}
Acto ou effeito de fulminar.
\section{Fulminador}
\begin{itemize}
\item {Grp. gram.:m.  e  adj.}
\end{itemize}
\begin{itemize}
\item {Proveniência:(Lat. \textunderscore fulminator\textunderscore )}
\end{itemize}
Aquillo que fulmina.
\section{Fulminante}
\begin{itemize}
\item {Grp. gram.:adj.}
\end{itemize}
\begin{itemize}
\item {Utilização:Fig.}
\end{itemize}
\begin{itemize}
\item {Grp. gram.:M.}
\end{itemize}
\begin{itemize}
\item {Proveniência:(Lat. \textunderscore fulminans\textunderscore )}
\end{itemize}
Que fulmina.
Terrível; cruel.
Indignado.
Cápsula metállica, que envolve a escorva da arma de fogo.
Rastilho de minas.
Pequeno explosivo, para brinquedo de crianças.
\section{Fulminar}
\begin{itemize}
\item {Grp. gram.:v. t.}
\end{itemize}
\begin{itemize}
\item {Grp. gram.:V. i.}
\end{itemize}
\begin{itemize}
\item {Proveniência:(Lat. \textunderscore fulminare\textunderscore )}
\end{itemize}
Lançar raios contra.
Ferir á maneira de raio.
Ferir, destruír, (falando-se do raio).
Desmoronar.
Aniquilar.
Despedir (excommunhão, censuras, etc.), comminar.
Apostrophar, invectivar.
Explodir.
Despedir raios.
Fulgurar.
\section{Fulminato}
\begin{itemize}
\item {Grp. gram.:m.}
\end{itemize}
\begin{itemize}
\item {Proveniência:(Do lat. \textunderscore fulmen\textunderscore )}
\end{itemize}
Sal, resultante da combinação do ácido fulmínico com uma base salificável.
\section{Fulminatório}
\begin{itemize}
\item {Grp. gram.:adj.}
\end{itemize}
Que fulmina.
\section{Fulmíneo}
\begin{itemize}
\item {Grp. gram.:adj.}
\end{itemize}
\begin{itemize}
\item {Utilização:Fig.}
\end{itemize}
\begin{itemize}
\item {Proveniência:(Lat. \textunderscore fulmineus\textunderscore )}
\end{itemize}
Relativo ao raio.
Brilhante ou destruidor, como o raio.
\section{Fulmínico}
\begin{itemize}
\item {Grp. gram.:adj.}
\end{itemize}
\begin{itemize}
\item {Proveniência:(Do lat. \textunderscore fulmen\textunderscore )}
\end{itemize}
Diz-se de um ácido, que é a combinação do cianogênio e do oxygênio.
\section{Fulminífero}
\begin{itemize}
\item {Grp. gram.:adj.}
\end{itemize}
\begin{itemize}
\item {Proveniência:(Do lat. \textunderscore fulmen\textunderscore  + \textunderscore ferre\textunderscore )}
\end{itemize}
Que fulmina.
\section{Fulminívomo}
\begin{itemize}
\item {Grp. gram.:adj.}
\end{itemize}
\begin{itemize}
\item {Proveniência:(Do lat. \textunderscore fulmen\textunderscore  + \textunderscore vomere\textunderscore )}
\end{itemize}
Que lança chammas, que dardeja fogo.
Que despede projécteis.
\section{Fulminoso}
\begin{itemize}
\item {Grp. gram.:adj.}
\end{itemize}
\begin{itemize}
\item {Proveniência:(Lat. \textunderscore fulminosus\textunderscore )}
\end{itemize}
O mesmo que \textunderscore fulmíneo\textunderscore .
\section{Fulmi-palha}
\begin{itemize}
\item {Grp. gram.:m.}
\end{itemize}
Nitrocellulose de palha.
\section{Fulo}
\begin{itemize}
\item {Grp. gram.:adj.}
\end{itemize}
\begin{itemize}
\item {Utilização:Fig.}
\end{itemize}
\begin{itemize}
\item {Utilização:Fam.}
\end{itemize}
\begin{itemize}
\item {Grp. gram.:M.}
\end{itemize}
\begin{itemize}
\item {Proveniência:(Do lat. \textunderscore fulvus\textunderscore )}
\end{itemize}
Diz-se dos pretos, cuja côr é tirante a amarelo.
Que empallidece ou muda de côr, por effeito de impressão violenta.
Muito zangado, irritado, furioso.
No jôgo do bóston, diz-se da côr opposta á favorita.
Pássaro syndáctylo da África occidental.
\section{Fulos}
\begin{itemize}
\item {Grp. gram.:m. pl.}
\end{itemize}
O mesmo ou melhor que \textunderscore fulas\textunderscore , povo da Guiné. Cf. Barros, \textunderscore Déc.\textunderscore  I, l. III, cap. 8.
\section{Fulosa}
\begin{itemize}
\item {Grp. gram.:f.}
\end{itemize}
\begin{itemize}
\item {Proveniência:(De \textunderscore fula\textunderscore ^2)}
\end{itemize}
Máquina de chapeleiro. Cf. \textunderscore Inquér. Industr.\textunderscore , P. II, l.^o 2.^o, 175.
\section{Fulverino}
\begin{itemize}
\item {Grp. gram.:m.}
\end{itemize}
\begin{itemize}
\item {Proveniência:(Do rad. de \textunderscore fulvo\textunderscore )}
\end{itemize}
Preparação, para se dar ao pano côr escura.
\section{Fulviana}
\begin{itemize}
\item {Grp. gram.:f.}
\end{itemize}
Planta diurética.
\section{Fulvicórneo}
\begin{itemize}
\item {Grp. gram.:adj.}
\end{itemize}
\begin{itemize}
\item {Utilização:Zool.}
\end{itemize}
\begin{itemize}
\item {Proveniência:(Do lat. \textunderscore fulvus\textunderscore  + \textunderscore cornu\textunderscore )}
\end{itemize}
Que tem as antennas fulvas.
\section{Fúlvido}
\begin{itemize}
\item {Grp. gram.:adj.}
\end{itemize}
\begin{itemize}
\item {Proveniência:(Lat. \textunderscore fulvidus\textunderscore )}
\end{itemize}
Fulvo e luzente.
Que tem côr de oiro. Cf. Alves Mendes, \textunderscore Herculano\textunderscore , 17.
\section{Fulvípede}
\begin{itemize}
\item {Grp. gram.:adj.}
\end{itemize}
\begin{itemize}
\item {Utilização:Zool.}
\end{itemize}
\begin{itemize}
\item {Proveniência:(Do lat. \textunderscore fulvus\textunderscore  + \textunderscore pes\textunderscore )}
\end{itemize}
Que tem os pés fulvos.
\section{Fulvipene}
\begin{itemize}
\item {Grp. gram.:adj.}
\end{itemize}
\begin{itemize}
\item {Utilização:Zool.}
\end{itemize}
\begin{itemize}
\item {Proveniência:(Do lat. \textunderscore fulvus\textunderscore  + \textunderscore penna\textunderscore )}
\end{itemize}
Que tem as penas fulvas.
\section{Fulvipenne}
\begin{itemize}
\item {Grp. gram.:adj.}
\end{itemize}
\begin{itemize}
\item {Utilização:Zool.}
\end{itemize}
\begin{itemize}
\item {Proveniência:(Do lat. \textunderscore fulvus\textunderscore  + \textunderscore penna\textunderscore )}
\end{itemize}
Que tem as penas fulvas.
\section{Fulvirostro}
\begin{itemize}
\item {fónica:rós}
\end{itemize}
\begin{itemize}
\item {Grp. gram.:adj.}
\end{itemize}
\begin{itemize}
\item {Utilização:Zool.}
\end{itemize}
\begin{itemize}
\item {Proveniência:(Do lat. \textunderscore fulvus\textunderscore  + \textunderscore rostrum\textunderscore )}
\end{itemize}
Que tem o bico fulvo.
\section{Fulvirrostro}
\begin{itemize}
\item {Grp. gram.:adj.}
\end{itemize}
\begin{itemize}
\item {Utilização:Zool.}
\end{itemize}
\begin{itemize}
\item {Proveniência:(Do lat. \textunderscore fulvus\textunderscore  + \textunderscore rostrum\textunderscore )}
\end{itemize}
Que tem o bico fulvo.
\section{Fulvo}
\begin{itemize}
\item {Grp. gram.:adj.}
\end{itemize}
\begin{itemize}
\item {Proveniência:(Lat. \textunderscore fulvus\textunderscore )}
\end{itemize}
Aloirado: \textunderscore cabellos fulvos\textunderscore .
Que tem côr amarelo-tostada.
\section{Fum}
Us. principalmente na loc. \textunderscore nem fum nem folle de ferreiro\textunderscore , como quem diz: nem uma palavra, nem pio, nem chus nem bus:«\textunderscore eu lá para o senhor não digo fum nem fum.\textunderscore »Castilho, \textunderscore Tartufo\textunderscore , 44.
\section{Fumaça}
\begin{itemize}
\item {Grp. gram.:f.}
\end{itemize}
\begin{itemize}
\item {Utilização:Fig.}
\end{itemize}
Grande porção de fumo.
Qualquer porção de fumo, que o fumista absorve de cada vez.
Vaidade.
\section{Fumaçada}
\begin{itemize}
\item {Grp. gram.:f.}
\end{itemize}
O mesmo que \textunderscore fumaça\textunderscore .
\section{Fumaceira}
\begin{itemize}
\item {Grp. gram.:f.}
\end{itemize}
\begin{itemize}
\item {Utilização:Pop.}
\end{itemize}
Grande fumaça.
Grande fumarada.
\section{Fumada}
\begin{itemize}
\item {Grp. gram.:f.}
\end{itemize}
\begin{itemize}
\item {Proveniência:(De \textunderscore fumo\textunderscore )}
\end{itemize}
Fumo, que se faz para sinal de rebate.
Fumaça.
Porção de fumo, que se tira por uma vez do cigarro, charuto ou cachimbo.
\section{Fumádego}
\begin{itemize}
\item {Grp. gram.:m.}
\end{itemize}
\begin{itemize}
\item {Utilização:Ant.}
\end{itemize}
(V.fumagem)
\section{Fumador}
\begin{itemize}
\item {Grp. gram.:m.  e  adj.}
\end{itemize}
\begin{itemize}
\item {Proveniência:(Lat. \textunderscore fumator\textunderscore )}
\end{itemize}
O que fuma.
\section{Fumagem}
\begin{itemize}
\item {Grp. gram.:f.}
\end{itemize}
Imposto, que incidia nas casas em que se accendesse lume.
Doiradura da prata.
Acto de fumar.
\section{Fumagina}
\begin{itemize}
\item {Grp. gram.:f.}
\end{itemize}
Doença das vinhas, o mesmo que \textunderscore mal-nero\textunderscore .
\section{Fumaguento}
\begin{itemize}
\item {Grp. gram.:adj.}
\end{itemize}
\begin{itemize}
\item {Utilização:Açor}
\end{itemize}
O mesmo que \textunderscore fumarento\textunderscore .
\section{Fumante}
\begin{itemize}
\item {Grp. gram.:m.}
\end{itemize}
\begin{itemize}
\item {Grp. gram.:Adj.}
\end{itemize}
\begin{itemize}
\item {Proveniência:(Lat. \textunderscore fumans\textunderscore )}
\end{itemize}
O mesmo que \textunderscore fumista\textunderscore .
Que lança fumo.
Que espuma.
\section{Fumão}
\begin{itemize}
\item {Grp. gram.:m.}
\end{itemize}
\begin{itemize}
\item {Utilização:Fam.}
\end{itemize}
Homem que fuma, fumista.
\section{Fumar}
\begin{itemize}
\item {Grp. gram.:v. t.}
\end{itemize}
\begin{itemize}
\item {Grp. gram.:V. i.}
\end{itemize}
\begin{itemize}
\item {Utilização:Ext.}
\end{itemize}
\begin{itemize}
\item {Utilização:T. de Turquel}
\end{itemize}
\begin{itemize}
\item {Proveniência:(Lat. \textunderscore fumare\textunderscore )}
\end{itemize}
Aspirar o fumo de: \textunderscore fumar um charuto\textunderscore .
Curar ao fumo: \textunderscore fumar chouriços\textunderscore .
Defumar.
Lançar o fumo.
Aspirar o fumo de cigarro, charuto ou cachimbo.
Lançar vapor.
Irritar-se.
Ir-se como o fumo, evaporar-se. Cf. Sim. Machado, 68, V.
O mesmo que \textunderscore fugir\textunderscore .
\section{Fumaraça}
\begin{itemize}
\item {Grp. gram.:f.}
\end{itemize}
O mesmo que \textunderscore fumarada\textunderscore .
\section{Fumarada}
\begin{itemize}
\item {Grp. gram.:f.}
\end{itemize}
\begin{itemize}
\item {Proveniência:(De \textunderscore fumar\textunderscore )}
\end{itemize}
O mesmo que \textunderscore fumaça\textunderscore .
\section{Fumarar}
\begin{itemize}
\item {Grp. gram.:v. i.}
\end{itemize}
\begin{itemize}
\item {Grp. gram.:V. t.}
\end{itemize}
Deitar fumo; fumegar.
Expellir ou diffundir como o fumo:«\textunderscore ...fica a fumarar náuseas a torcida da alampada.\textunderscore »Camillo, \textunderscore Mulher Fatal\textunderscore , 127.
(Cp. \textunderscore fumarada\textunderscore )
\section{Fumarato}
\begin{itemize}
\item {Grp. gram.:m.}
\end{itemize}
\begin{itemize}
\item {Proveniência:(De \textunderscore fumárico\textunderscore )}
\end{itemize}
Sal, resultante da combinação do ácido fumárico com uma base salificável.
\section{Fumareda}
\begin{itemize}
\item {Grp. gram.:f.}
\end{itemize}
Grossas nuvens de fumo, fumarada.
\section{Fumarela}
\begin{itemize}
\item {Grp. gram.:f.}
\end{itemize}
\begin{itemize}
\item {Utilização:Prov.}
\end{itemize}
\begin{itemize}
\item {Utilização:beir.}
\end{itemize}
O mesmo que \textunderscore fumaça\textunderscore .
\section{Fumarento}
\begin{itemize}
\item {Grp. gram.:adj.}
\end{itemize}
Que deita fumo ou fumarada. Cf. Ortigão, \textunderscore Hollanda\textunderscore , 42.
\section{Fumária}
\begin{itemize}
\item {Grp. gram.:f.}
\end{itemize}
Planta, também conhecida por \textunderscore fumo da terra\textunderscore , (\textunderscore fumaria officinalis\textunderscore ).
\section{Fumariáceas}
\begin{itemize}
\item {Grp. gram.:f. pl.}
\end{itemize}
Família de plantas dicotyledóneas polypétalas, que têm por typo a fumária.
\section{Fumárico}
\begin{itemize}
\item {Grp. gram.:adj.}
\end{itemize}
Diz-se de um ácido, extrahido da fumária.
\section{Fumarina}
\begin{itemize}
\item {Grp. gram.:f.}
\end{itemize}
Alcali, que se encontra na fumária.
\section{Fumarola}
\begin{itemize}
\item {Grp. gram.:f.}
\end{itemize}
\begin{itemize}
\item {Proveniência:(De \textunderscore fumar\textunderscore )}
\end{itemize}
Emanação vulcânica, com apparência de nuvem de fumo branco. Cf. G. Guimarães, \textunderscore Geologia\textunderscore , 181.
\section{Fumatório}
\begin{itemize}
\item {Grp. gram.:adj.}
\end{itemize}
\begin{itemize}
\item {Utilização:Neol.}
\end{itemize}
\begin{itemize}
\item {Proveniência:(De \textunderscore fumar\textunderscore )}
\end{itemize}
Diz-se do apparelho, com que se fuma.
\section{Fumável}
\begin{itemize}
\item {Grp. gram.:adj.}
\end{itemize}
Que se póde fumar; que é bom para se fumar.
\section{Fumbantsana}
\begin{itemize}
\item {Grp. gram.:f.}
\end{itemize}
Árvore medicinal, de que os indígenas de Moçambique fazem chiduras.
\section{Fumeante}
\begin{itemize}
\item {Grp. gram.:adj.}
\end{itemize}
\begin{itemize}
\item {Proveniência:(De \textunderscore fumear\textunderscore )}
\end{itemize}
Que deita fumo; que fumega. Cf. Filinto, VI, 107.
\section{Fumear}
\begin{itemize}
\item {Grp. gram.:v. i.}
\end{itemize}
O mesmo que \textunderscore fumegar\textunderscore .
\section{Fumegante}
\begin{itemize}
\item {Grp. gram.:adj.}
\end{itemize}
Que fumega.
\section{Fumegar}
\begin{itemize}
\item {Grp. gram.:v. i.}
\end{itemize}
\begin{itemize}
\item {Proveniência:(Do lat. \textunderscore fumigare\textunderscore )}
\end{itemize}
Lançar fumo.
Atear-se.
\section{Fumêgo}
\begin{itemize}
\item {Grp. gram.:m.}
\end{itemize}
\begin{itemize}
\item {Utilização:bras}
\end{itemize}
\begin{itemize}
\item {Utilização:Neol.}
\end{itemize}
Acto de fumegar (uma iguaria quente).--Us. por Coelho Neto.
\section{Fumeiro}
\begin{itemize}
\item {Grp. gram.:m.}
\end{itemize}
\begin{itemize}
\item {Utilização:Ext.}
\end{itemize}
\begin{itemize}
\item {Utilização:Prov.}
\end{itemize}
\begin{itemize}
\item {Utilização:alg.}
\end{itemize}
\begin{itemize}
\item {Proveniência:(Lat. \textunderscore fumarium\textunderscore )}
\end{itemize}
Chaminé.
Cano, por onde se eleva o fumo de um fogão, de uma cozinha, de uma máquina.
Fumarada.
Espaço, entre a lareira e o telhado, onde se pendura carne ensacada, para esta se curar ou defumar.
Carne ensacada.
O mesmo que \textunderscore almanxar\textunderscore .
\section{Fumélio}
\begin{itemize}
\item {Grp. gram.:m.}
\end{itemize}
\begin{itemize}
\item {Utilização:ant.}
\end{itemize}
\begin{itemize}
\item {Utilização:Gír.}
\end{itemize}
\begin{itemize}
\item {Proveniência:(De \textunderscore fumo\textunderscore )}
\end{itemize}
Tabaco para fumar.
\section{Fúmeo}
\begin{itemize}
\item {Grp. gram.:adj.}
\end{itemize}
\begin{itemize}
\item {Proveniência:(Lat. \textunderscore fumeus\textunderscore )}
\end{itemize}
O mesmo que \textunderscore fumífero\textunderscore .
\section{Fúmido}
\begin{itemize}
\item {Grp. gram.:adj.}
\end{itemize}
\begin{itemize}
\item {Proveniência:(Lat. \textunderscore fumidus\textunderscore )}
\end{itemize}
O mesmo que \textunderscore fumoso\textunderscore . Cf. \textunderscore Viriato Trág.\textunderscore , VIII, 123.
\section{Fumífero}
\begin{itemize}
\item {Grp. gram.:adj.}
\end{itemize}
O mesmo que \textunderscore fumoso\textunderscore .
\section{Fumífico}
\begin{itemize}
\item {Grp. gram.:adj.}
\end{itemize}
\begin{itemize}
\item {Proveniência:(Lat. \textunderscore fumificus\textunderscore )}
\end{itemize}
O mesmo que \textunderscore fumoso\textunderscore .
\section{Fumiflamante}
\begin{itemize}
\item {Grp. gram.:adj.}
\end{itemize}
\begin{itemize}
\item {Proveniência:(Do lat. \textunderscore fumus\textunderscore  + \textunderscore flammans\textunderscore )}
\end{itemize}
Que, ardendo, lança fumo.
\section{Fumiflammante}
\begin{itemize}
\item {Grp. gram.:adj.}
\end{itemize}
\begin{itemize}
\item {Proveniência:(Do lat. \textunderscore fumus\textunderscore  + \textunderscore flammans\textunderscore )}
\end{itemize}
Que, ardendo, lança fumo.
\section{Fumífugo}
\begin{itemize}
\item {Grp. gram.:adj.}
\end{itemize}
\begin{itemize}
\item {Grp. gram.:M.}
\end{itemize}
\begin{itemize}
\item {Proveniência:(Do lat. \textunderscore fumus\textunderscore  + \textunderscore fugere\textunderscore )}
\end{itemize}
Que afasta o fumo.
Apparelho, que, collocado na chaminé, impede a diffusão do fumo no interior das casas.
\section{Fumigação}
\begin{itemize}
\item {Grp. gram.:f.}
\end{itemize}
Acto de fumigar.
\section{Fumigar}
\begin{itemize}
\item {Grp. gram.:v. t.}
\end{itemize}
\begin{itemize}
\item {Proveniência:(Do lat. \textunderscore fumigare\textunderscore )}
\end{itemize}
Expor ao fumo.
Defumar.
Desinfectar, defumando.
\section{Fumigatório}
\begin{itemize}
\item {Grp. gram.:adj.}
\end{itemize}
\begin{itemize}
\item {Grp. gram.:M.}
\end{itemize}
Que serve para fumigar.
Fumigação.
\section{Fuminé}
\begin{itemize}
\item {Grp. gram.:f.}
\end{itemize}
\begin{itemize}
\item {Utilização:Prov.}
\end{itemize}
\begin{itemize}
\item {Utilização:dur.}
\end{itemize}
O mesmo que \textunderscore chaminé\textunderscore , (por infl. de \textunderscore fumo\textunderscore ).
\section{Fumista}
\begin{itemize}
\item {Grp. gram.:m.}
\end{itemize}
\begin{itemize}
\item {Proveniência:(De \textunderscore fumo\textunderscore )}
\end{itemize}
Aquelle que tem o hábito de fumar tabaco.
\section{Fumívomo}
\begin{itemize}
\item {Grp. gram.:adj.}
\end{itemize}
\begin{itemize}
\item {Proveniência:(Do lat. \textunderscore fumus\textunderscore  + \textunderscore vomere\textunderscore )}
\end{itemize}
O mesmo que \textunderscore fumante\textunderscore .
\section{Fumívoro}
\begin{itemize}
\item {Grp. gram.:adj.}
\end{itemize}
\begin{itemize}
\item {Grp. gram.:M.}
\end{itemize}
\begin{itemize}
\item {Proveniência:(Do lat. \textunderscore fumus\textunderscore  + \textunderscore vorare\textunderscore )}
\end{itemize}
Que aspira fumo.
Apparelho, que absorve o fumo dos bicos de gás.
\section{Fumo}
\begin{itemize}
\item {Grp. gram.:m.}
\end{itemize}
\begin{itemize}
\item {Utilização:Fig.}
\end{itemize}
\begin{itemize}
\item {Utilização:Pop.}
\end{itemize}
\begin{itemize}
\item {Proveniência:(Lat. \textunderscore fumus\textunderscore )}
\end{itemize}
Espécie de nuvem pardacenta ou escura, que se eleva dos corpos em combustão ou muito aquecidos, e ainda dos corpos húmidos e sujeitos a uma alta temperatura.
Vapor ou exhalação, de cheiro desagradável e que se eleva de corpos em decomposição.
Evaporação da água que se despenha, formando uma espécie de nuvem.
Faixa de crepe para luto.
Tabaco para fumar.
Vaidade, jactância.
Aquillo que se esvaece, que é transitório, rapidamente extinguível: \textunderscore a vida é fumo que vôa\textunderscore .
Fuligem, que entra na composição de certas tintas.
Esturro, bispo: \textunderscore a sopa tem fumo\textunderscore .
\section{Fumo-bravo}
\begin{itemize}
\item {Grp. gram.:m.}
\end{itemize}
Erva brasileira, medicinal, sudorífera.
O mesmo que \textunderscore erva-grossa\textunderscore .
\section{Fumosidade}
\begin{itemize}
\item {Grp. gram.:f.}
\end{itemize}
Qualidade daquillo que é fumoso.
\section{Fumoso}
\begin{itemize}
\item {Grp. gram.:adj.}
\end{itemize}
\begin{itemize}
\item {Utilização:Fig.}
\end{itemize}
\begin{itemize}
\item {Proveniência:(Lat. \textunderscore fumosus\textunderscore )}
\end{itemize}
Que lança fumo ou vapores.
Em que há fumo.
Cheio de fumo.
Vaidoso, jactancioso.
\section{Funambulesco}
\begin{itemize}
\item {fónica:lês}
\end{itemize}
\begin{itemize}
\item {Grp. gram.:adj.}
\end{itemize}
Relativo a funâmbulo; próprio de funâmbulo. Cf. Camillo, \textunderscore Cancion. Al.\textunderscore , 12.
\section{Funambulismo}
\begin{itemize}
\item {Grp. gram.:m.}
\end{itemize}
Offício de funâmbulo.
\section{Funâmbulo}
\begin{itemize}
\item {Grp. gram.:m.}
\end{itemize}
\begin{itemize}
\item {Utilização:Fig.}
\end{itemize}
\begin{itemize}
\item {Proveniência:(Lat. \textunderscore funambulus\textunderscore )}
\end{itemize}
Aquelle que anda ou dança em corda bamba.
Aquelle que muda facilmente de opinião ou de partido.
\section{Funante}
\begin{itemize}
\item {Grp. gram.:m.}
\end{itemize}
Negociante português, que, da costa da África, ia commerciar até o centro daquelle continente ou até alli mandava os seus pombeiros ou moçambazes. Cf. Ficalho, \textunderscore Plantas Úteis da Áfr. Port.\textunderscore 
\section{Funária}
\begin{itemize}
\item {Grp. gram.:f.}
\end{itemize}
\begin{itemize}
\item {Proveniência:(Lat. \textunderscore funaria\textunderscore )}
\end{itemize}
Gênero de vegetaes cryptogâmicos, da fam. dos musgos.
\section{Funarioides}
\begin{itemize}
\item {Grp. gram.:f. pl.}
\end{itemize}
\begin{itemize}
\item {Proveniência:(Do lat. \textunderscore funaria\textunderscore , gr. \textunderscore eidos\textunderscore )}
\end{itemize}
Vegetaes, que têm por typo a funária.
\section{Funca}
\begin{itemize}
\item {Grp. gram.:m. ,  f.  e  adj.}
\end{itemize}
\begin{itemize}
\item {Utilização:Bras}
\end{itemize}
Pessôa ou coisa de pouco préstimo.
\section{Funçanada}
\begin{itemize}
\item {Grp. gram.:f.}
\end{itemize}
O mesmo que \textunderscore funçanata\textunderscore .
\section{Funçanata}
\begin{itemize}
\item {Grp. gram.:f.}
\end{itemize}
\begin{itemize}
\item {Utilização:Fam.}
\end{itemize}
\begin{itemize}
\item {Proveniência:(Do rad. de \textunderscore função\textunderscore  = \textunderscore funcção\textunderscore )}
\end{itemize}
Divertimento.
Pândega; folia.
\section{Funçanista}
\begin{itemize}
\item {Grp. gram.:m. ,  f.  e  adj.}
\end{itemize}
\begin{itemize}
\item {Proveniência:(De \textunderscore função\textunderscore  = \textunderscore funcção\textunderscore )}
\end{itemize}
Pessôa, que é dada a funçanatas.
\section{Função}
\begin{itemize}
\item {Grp. gram.:f.}
\end{itemize}
\begin{itemize}
\item {Proveniência:(Lat. \textunderscore functio\textunderscore )}
\end{itemize}
Exercício.
Prática; uso.
Cargo.
Acto indispensável para o exercício dos fenómenos vitaes.
Solenidade, festa.
Funçanata.
Dependência de uma quantidade matemática, cujo valor se determina pelo que se dá a outra.
\section{Funcção}
\begin{itemize}
\item {Grp. gram.:f.}
\end{itemize}
\begin{itemize}
\item {Proveniência:(Lat. \textunderscore functio\textunderscore )}
\end{itemize}
Exercício.
Prática; uso.
Cargo.
Acto indispensável para o exercício dos phenómenos vitaes.
Solennidade, festa.
Funçanata.
Dependência de uma quantidade mathemática, cujo valor se determina pelo que se dá a outra.
\section{Funcciologia}
\begin{itemize}
\item {Grp. gram.:f.}
\end{itemize}
\begin{itemize}
\item {Utilização:Neol.}
\end{itemize}
\begin{itemize}
\item {Proveniência:(T. hybr., do lat. \textunderscore functio\textunderscore  + gr. \textunderscore logos\textunderscore )}
\end{itemize}
Tratado da funcção grammatical das palavras. Cf. Júl. Ribeiro, \textunderscore Estudos Philol.\textunderscore 
\section{Funcciológico}
\begin{itemize}
\item {Grp. gram.:adj.}
\end{itemize}
Relativo á funcciologia.
\section{Funccional}
\begin{itemize}
\item {Grp. gram.:adj.}
\end{itemize}
\begin{itemize}
\item {Proveniência:(Do lat. \textunderscore functio\textunderscore )}
\end{itemize}
Relativo a funcções vitaes.
\section{Funccionalismo}
\begin{itemize}
\item {Grp. gram.:m.}
\end{itemize}
\begin{itemize}
\item {Proveniência:(De \textunderscore funcional\textunderscore )}
\end{itemize}
A classe dos funccionários.
\section{Funccionalizar-se}
\begin{itemize}
\item {Grp. gram.:v. p.}
\end{itemize}
\begin{itemize}
\item {Utilização:Neol.}
\end{itemize}
Tornar-se funccionário:«\textunderscore funccionalizou-se num govêrno civil\textunderscore ». Camillo, \textunderscore Cancion. Al.\textunderscore , 304.
(Voc. mal formado, de \textunderscore funccional\textunderscore . Admissível seria \textunderscore funccionarizar-se\textunderscore , de \textunderscore funccionar\textunderscore )
\section{Funccionamento}
\begin{itemize}
\item {Grp. gram.:m.}
\end{itemize}
Acto ou effeito de funccionar.
\section{Funccionar}
\begin{itemize}
\item {Grp. gram.:v. i.}
\end{itemize}
\begin{itemize}
\item {Proveniência:(Do lat. \textunderscore functio\textunderscore )}
\end{itemize}
Exercer funcções.
Estar em exercício.
Realizar movimentos.
Trabalhar.
\section{Funccionário}
\begin{itemize}
\item {Grp. gram.:m.}
\end{itemize}
\begin{itemize}
\item {Proveniência:(Do lat. \textunderscore functio\textunderscore )}
\end{itemize}
Empregado público; empregado.
Aquelle que tem occupação permanente e retribuida.
\section{Funccionista}
\begin{itemize}
\item {Grp. gram.:m.}
\end{itemize}
\begin{itemize}
\item {Proveniência:(Do lat. \textunderscore functio\textunderscore )}
\end{itemize}
Aquelle que assiste a uma funcção ou toma parte nella.
\section{Funce}
\begin{itemize}
\item {Grp. gram.:m.}
\end{itemize}
Pequena embarcação asiática, de remos. Cp. \textunderscore Peregrinação\textunderscore , CXXXV.
(Cf. fr. \textunderscore fonce\textunderscore )
\section{Funchal}
\begin{itemize}
\item {Grp. gram.:m.}
\end{itemize}
Lugar, onde crescem funchos.
\section{Funcho}
\begin{itemize}
\item {Grp. gram.:m.}
\end{itemize}
\begin{itemize}
\item {Proveniência:(Do lat. \textunderscore fenuculum\textunderscore , por \textunderscore feniculum\textunderscore )}
\end{itemize}
Planta umbellífera, (\textunderscore anethum foeniculum vulgare\textunderscore ).
\section{Funcho-da-china}
\begin{itemize}
\item {Grp. gram.:m.}
\end{itemize}
\begin{itemize}
\item {Utilização:Bot.}
\end{itemize}
O mesmo que \textunderscore badiana\textunderscore .
\section{Funcho-de-água}
\begin{itemize}
\item {Grp. gram.:m.}
\end{itemize}
\begin{itemize}
\item {Utilização:Bras}
\end{itemize}
Planta medicinal, umbellífera, (\textunderscore phellandrium aquaticum\textunderscore ).
\section{Funciologia}
\begin{itemize}
\item {Grp. gram.:f.}
\end{itemize}
\begin{itemize}
\item {Utilização:Neol.}
\end{itemize}
\begin{itemize}
\item {Proveniência:(T. hybr., do lat. \textunderscore functio\textunderscore  + gr. \textunderscore logos\textunderscore )}
\end{itemize}
Tratado da função gramatical das palavras. Cf. Júl. Ribeiro, \textunderscore Estudos Philol.\textunderscore 
\section{Funciológico}
\begin{itemize}
\item {Grp. gram.:adj.}
\end{itemize}
Relativo á funciologia.
\section{Funcional}
\begin{itemize}
\item {Grp. gram.:adj.}
\end{itemize}
\begin{itemize}
\item {Proveniência:(Do lat. \textunderscore functio\textunderscore )}
\end{itemize}
Relativo a funções vitaes.
\section{Funcionalismo}
\begin{itemize}
\item {Grp. gram.:m.}
\end{itemize}
\begin{itemize}
\item {Proveniência:(De \textunderscore funcional\textunderscore )}
\end{itemize}
A classe dos funcionários.
\section{Funcionalizar-se}
\begin{itemize}
\item {Grp. gram.:v. p.}
\end{itemize}
\begin{itemize}
\item {Utilização:Neol.}
\end{itemize}
Tornar-se funcionário:«\textunderscore funcionalizou-se num govêrno civil\textunderscore ». Camillo, \textunderscore Cancion. Al.\textunderscore , 304.
(Voc. mal formado, de \textunderscore funcional\textunderscore . Admissível seria \textunderscore funcionarizar-se\textunderscore , de \textunderscore funcionar\textunderscore )
\section{Funcionamento}
\begin{itemize}
\item {Grp. gram.:m.}
\end{itemize}
Acto ou efeito de funcionar.
\section{Funcionar}
\begin{itemize}
\item {Grp. gram.:v. i.}
\end{itemize}
\begin{itemize}
\item {Proveniência:(Do lat. \textunderscore functio\textunderscore )}
\end{itemize}
Exercer funções.
Estar em exercício.
Realizar movimentos.
Trabalhar.
\section{Funcionário}
\begin{itemize}
\item {Grp. gram.:m.}
\end{itemize}
\begin{itemize}
\item {Proveniência:(Do lat. \textunderscore functio\textunderscore )}
\end{itemize}
Empregado público; empregado.
Aquele que tem occupação permanente e retribuida.
\section{Funcionista}
\begin{itemize}
\item {Grp. gram.:m.}
\end{itemize}
\begin{itemize}
\item {Proveniência:(Do lat. \textunderscore functio\textunderscore )}
\end{itemize}
Aquele que assiste a uma função ou toma parte nela.
\section{Funçonata}
\begin{itemize}
\item {Grp. gram.:f.}
\end{itemize}
\begin{itemize}
\item {Utilização:Bras. de Minas}
\end{itemize}
O mesmo que \textunderscore funçanata\textunderscore .
\section{Funda}
\begin{itemize}
\item {Grp. gram.:f.}
\end{itemize}
\begin{itemize}
\item {Proveniência:(Lat. \textunderscore funda\textunderscore )}
\end{itemize}
Apparelho, para arremêsso de pedras ou balas.
Utensílio cirúrgico, para ligar quebraduras.
\section{Funda}
\begin{itemize}
\item {Grp. gram.:f.}
\end{itemize}
\begin{itemize}
\item {Utilização:Prov.}
\end{itemize}
\begin{itemize}
\item {Proveniência:(De \textunderscore fundir\textunderscore ^1)}
\end{itemize}
Acto de produzir bem ou de produzir muito, (falando-se de uvas, azeitonas, searas, etc.): \textunderscore as uvas tiveram bôa funda\textunderscore .
\section{Fundação}
\begin{itemize}
\item {Grp. gram.:f.}
\end{itemize}
\begin{itemize}
\item {Proveniência:(Lat. \textunderscore fundatio\textunderscore )}
\end{itemize}
Acto ou effeito de fundar.
\section{Fundador}
\begin{itemize}
\item {Grp. gram.:m.  e  adj.}
\end{itemize}
\begin{itemize}
\item {Proveniência:(Lat. \textunderscore fundator\textunderscore )}
\end{itemize}
O que funda ou institue: \textunderscore o fundador da Ordem de Christo\textunderscore .
Iniciador.
\section{Fundagem}
\begin{itemize}
\item {Grp. gram.:f.}
\end{itemize}
\begin{itemize}
\item {Utilização:T. de Turquel}
\end{itemize}
\begin{itemize}
\item {Grp. gram.:Pl.}
\end{itemize}
\begin{itemize}
\item {Proveniência:(De \textunderscore fundo\textunderscore )}
\end{itemize}
Substância, que se deposita no fundo de um líquido; resíduo; fezes; bôrra.
Os tampos de tonéis, pipas ou vasilhas similares.
Pranchas, para o fabrico daquelles tampos.
\section{Fundal}
\begin{itemize}
\item {Grp. gram.:adj.}
\end{itemize}
\begin{itemize}
\item {Utilização:P. us.}
\end{itemize}
\begin{itemize}
\item {Grp. gram.:M.}
\end{itemize}
\begin{itemize}
\item {Utilização:T. de Pare -de-Coira}
\end{itemize}
\begin{itemize}
\item {Utilização:des.}
\end{itemize}
Situado ao fundo de um monte, no valle.
Fundeiro^1.
Campo baixo e regadio para semeadura.
\section{Fundalha}
\begin{itemize}
\item {Grp. gram.:f.}
\end{itemize}
\begin{itemize}
\item {Utilização:Pop.}
\end{itemize}
O mesmo que \textunderscore fundagem\textunderscore .
\section{Fundalho}
\begin{itemize}
\item {Grp. gram.:m.}
\end{itemize}
O mesmo que \textunderscore fundagem\textunderscore .
\section{Fundamentadamente}
\begin{itemize}
\item {Grp. gram.:adv.}
\end{itemize}
Com fundamento.
\section{Fundamental}
\begin{itemize}
\item {Grp. gram.:adj.}
\end{itemize}
Que serve de fundamento.
Essencial; necessário.
\section{Fundamentalmente}
\begin{itemize}
\item {Grp. gram.:adv.}
\end{itemize}
De modo fundamental.
\section{Fundamentar}
\begin{itemize}
\item {Grp. gram.:v. t.}
\end{itemize}
\begin{itemize}
\item {Proveniência:(Do b. lat. \textunderscore fundamentare\textunderscore )}
\end{itemize}
Dar fundamento a.
Alicerçar.
Firmar, estabelecer, basear: \textunderscore fundamentar argumentos\textunderscore .
Provar: \textunderscore fundamentar uma these\textunderscore .
\section{Fundamente}
\begin{itemize}
\item {Grp. gram.:adv.}
\end{itemize}
\begin{itemize}
\item {Proveniência:(De \textunderscore fundo\textunderscore )}
\end{itemize}
Com profundeza.
Fundamentadamente.
Em alto grau.
\section{Fundamento}
\begin{itemize}
\item {Grp. gram.:m.}
\end{itemize}
\begin{itemize}
\item {Proveniência:(Lat. \textunderscore fundamentum\textunderscore )}
\end{itemize}
Base, alicerce.
Sustentáculo.
Cimento.
Motivo, razão.
\section{Fundão}
\begin{itemize}
\item {Grp. gram.:m.}
\end{itemize}
\begin{itemize}
\item {Utilização:Pesc.}
\end{itemize}
\begin{itemize}
\item {Grp. gram.:Pl.}
\end{itemize}
\begin{itemize}
\item {Utilização:Bras}
\end{itemize}
\begin{itemize}
\item {Proveniência:(De \textunderscore fundo\textunderscore )}
\end{itemize}
O mesmo que \textunderscore pégo\textunderscore .
O mar alto.
Lugar, situado ao fundo de um monte ou de uma eminência.
Ermos, sitios distantes.
\section{Fundar}
\begin{itemize}
\item {Grp. gram.:v. t.}
\end{itemize}
\begin{itemize}
\item {Utilização:Fig.}
\end{itemize}
\begin{itemize}
\item {Utilização:Prov.}
\end{itemize}
\begin{itemize}
\item {Proveniência:(Lat. \textunderscore fundare\textunderscore )}
\end{itemize}
Construír; edificar, desde o fundo, desde os alicerces: \textunderscore fundar uma cidade\textunderscore .
Instituír.
Firmar; apoiar.
Pôr os fundos ou tampos em (tonel, pipa, etc.).
\section{Fundável}
\begin{itemize}
\item {Grp. gram.:adj.}
\end{itemize}
Diz-se do terreno, cuja camada arável tem muita espessura, ficando muito fundo o subsolo impermeável. Cf. \textunderscore Techn. Rur.\textunderscore , 48.
(Cp. \textunderscore fundar\textunderscore )
\section{Fund-dobre}
\begin{itemize}
\item {Grp. gram.:adj.}
\end{itemize}
\begin{itemize}
\item {Utilização:Des.}
\end{itemize}
Que tem dois fundos ou fundo duplo, (falando-se de caixas, vasos, etc.).--Extravagância morphológica de Filinto, XVI, 93.
\section{Fundeado}
\begin{itemize}
\item {Grp. gram.:adj.}
\end{itemize}
\begin{itemize}
\item {Proveniência:(De \textunderscore fundear\textunderscore )}
\end{itemize}
Que fundeou; que está ancorado.
\section{Fundeadoiro}
\begin{itemize}
\item {Grp. gram.:m.}
\end{itemize}
\begin{itemize}
\item {Proveniência:(De \textunderscore fundear\textunderscore )}
\end{itemize}
O mesmo que \textunderscore ancoradoiro\textunderscore .
\section{Fundeadouro}
\begin{itemize}
\item {Grp. gram.:m.}
\end{itemize}
\begin{itemize}
\item {Proveniência:(De \textunderscore fundear\textunderscore )}
\end{itemize}
O mesmo que \textunderscore ancoradouro\textunderscore .
\section{Fundear}
\begin{itemize}
\item {Grp. gram.:v. i.}
\end{itemize}
\begin{itemize}
\item {Proveniência:(De \textunderscore fundo\textunderscore )}
\end{itemize}
Deitar ferro ou âncora; ancorar.
Abicar; aportar.
\section{Fundego}
\begin{itemize}
\item {Grp. gram.:m.}
\end{itemize}
\begin{itemize}
\item {Utilização:Prov.}
\end{itemize}
\begin{itemize}
\item {Proveniência:(De \textunderscore fundo\textunderscore )}
\end{itemize}
Ribanceira.
Campo, ao fundo de uma ribanceira.
Terreno baixo ou fundo.--No Doiro, \textunderscore fundêgo\textunderscore ; na Beira, \textunderscore fundégo\textunderscore .
\section{Fundeiro}
\begin{itemize}
\item {Grp. gram.:adj.}
\end{itemize}
Que está ao fundo, na extremidade mais baixa: \textunderscore a janela fundeira\textunderscore .
Que está no fundo.
Que tem muito fundo ou altura.
\section{Fundeiro}
\begin{itemize}
\item {Grp. gram.:m.}
\end{itemize}
Aquelle que faz fundas.
Aquelle que usa de funda, como arma.
Fundibulário.
\section{Fundente}
\begin{itemize}
\item {Grp. gram.:adj.}
\end{itemize}
\begin{itemize}
\item {Grp. gram.:M.}
\end{itemize}
\begin{itemize}
\item {Proveniência:(Lat. \textunderscore fundens\textunderscore )}
\end{itemize}
Que está em fusão.
Que facilita a fusão.
Que liquefaz.
Substância, que auxilia a fusão dos metaes.
\section{Fundiário}
\begin{itemize}
\item {Grp. gram.:adj.}
\end{itemize}
\begin{itemize}
\item {Proveniência:(Do lat. \textunderscore fundus\textunderscore )}
\end{itemize}
Relativo a terrenos; terreal; agrário.
\section{Fundíbalo}
\begin{itemize}
\item {Grp. gram.:m.}
\end{itemize}
\begin{itemize}
\item {Proveniência:(Lat. \textunderscore fundibalum\textunderscore )}
\end{itemize}
Catapulta.
O mesmo que \textunderscore fundíbulo\textunderscore .
\section{Fundibulário}
\begin{itemize}
\item {Grp. gram.:m.}
\end{itemize}
\begin{itemize}
\item {Proveniência:(Lat. \textunderscore fundibularius\textunderscore )}
\end{itemize}
Aquelle que combate com a funda.
\section{Fundíbulo}
\begin{itemize}
\item {Grp. gram.:m.}
\end{itemize}
\begin{itemize}
\item {Proveniência:(Lat. \textunderscore fundibulum\textunderscore )}
\end{itemize}
Antigo apparelho guerreiro, para arremêsso de pedras e de outros projécteis.
Funda.
\section{Fundição}
\begin{itemize}
\item {Grp. gram.:f.}
\end{itemize}
\begin{itemize}
\item {Utilização:Fig.}
\end{itemize}
Acto, effeito, arte, ou fábrica de fundir^1.
Projecto.
Producção intellectual.
\section{Fundidor}
\begin{itemize}
\item {Grp. gram.:m.}
\end{itemize}
Aquelle que funde.
\section{Fundilhar}
\begin{itemize}
\item {Grp. gram.:v. t.}
\end{itemize}
Pôr fundilhos em.
\section{Fundilho}
\begin{itemize}
\item {Grp. gram.:m.}
\end{itemize}
\begin{itemize}
\item {Proveniência:(De \textunderscore fundo\textunderscore )}
\end{itemize}
Parte posterior das calças, no lugar correspondente ao assento.
Remendo nessa parte das calças.
\section{Fundinho}
\begin{itemize}
\item {Grp. gram.:m.}
\end{itemize}
\begin{itemize}
\item {Proveniência:(De \textunderscore fundo\textunderscore )}
\end{itemize}
Espécie de biombo, com duas fôlhas, atrás das portas dos salões, simulando corredor de communicação para outras divisões da casa.
\section{Fundir}
\begin{itemize}
\item {Grp. gram.:v. t.}
\end{itemize}
\begin{itemize}
\item {Utilização:Ext.}
\end{itemize}
\begin{itemize}
\item {Grp. gram.:V. i.}
\end{itemize}
\begin{itemize}
\item {Grp. gram.:V. p.}
\end{itemize}
\begin{itemize}
\item {Proveniência:(Lat. \textunderscore fundere\textunderscore )}
\end{itemize}
Derreter; liquefazer: \textunderscore fundir chumbo\textunderscore .
Unir, incorporar: \textunderscore fundir duas associações numa\textunderscore .
Moldar (metaes).
Dissipar.
Dar vantagem.
Sêr lucrativo; produzir muito: \textunderscore êste anno os olivaes fundiram muito\textunderscore .
Liquefazer-se.
Encorporar-se; confundir-se.
\section{Fundir}
\begin{itemize}
\item {Grp. gram.:v. t.}
\end{itemize}
\begin{itemize}
\item {Utilização:Des.}
\end{itemize}
\begin{itemize}
\item {Proveniência:(De \textunderscore fundo\textunderscore )}
\end{itemize}
O mesmo que \textunderscore afundar\textunderscore .
\section{Fundismo}
\begin{itemize}
\item {Grp. gram.:m.}
\end{itemize}
Bôrra ou felpa da lan, resultante da tosadura do pano. Cf. \textunderscore Diário do Govêrno\textunderscore  de 28-XII-1876.
\section{Fundível}
\begin{itemize}
\item {Grp. gram.:adj.}
\end{itemize}
\begin{itemize}
\item {Proveniência:(De \textunderscore fundir\textunderscore ^1)}
\end{itemize}
O mesmo que \textunderscore fusível\textunderscore .
\section{Fundo}
\begin{itemize}
\item {Grp. gram.:adj.}
\end{itemize}
\begin{itemize}
\item {Utilização:Fig.}
\end{itemize}
\begin{itemize}
\item {Grp. gram.:M.}
\end{itemize}
\begin{itemize}
\item {Utilização:Prov.}
\end{itemize}
\begin{itemize}
\item {Utilização:Fig.}
\end{itemize}
\begin{itemize}
\item {Utilização:Gír.}
\end{itemize}
\begin{itemize}
\item {Grp. gram.:Loc. adv.}
\end{itemize}
\begin{itemize}
\item {Grp. gram.:Adv.}
\end{itemize}
\begin{itemize}
\item {Grp. gram.:Pl.}
\end{itemize}
\begin{itemize}
\item {Proveniência:(Lat. \textunderscore fundus\textunderscore )}
\end{itemize}
Que está abaixo de uma superfície; que tem profundidade: \textunderscore poço fundo\textunderscore .
Reentrante, cavado: \textunderscore chaga funda\textunderscore .
Íntimo; arraigado: \textunderscore o mais fundo ódio\textunderscore .
Denso.
A parte que, num objecto ou numa cavidade, fica mais distante da superfície ou da abertura: \textunderscore o fundo de um copo\textunderscore .
O solo submarino.
A parte mais afastada, a parte mais baixa, mais interior: \textunderscore no fundo do coração\textunderscore .
Profundidade.
Decorações scênicas, as mais distantes da boca do palco.
O mesmo que \textunderscore tampo\textunderscore ^1.
Essência; fundamento: \textunderscore o fundo da questão\textunderscore .
O que há de mais íntimo no coração.
Cabedal.
Capital.
Soldado, sentinela.
Prisão.
\textunderscore A fundo\textunderscore , em cheio, com largueza, profundamente.
Profundamente.
\textunderscore Marchar a um de fundo\textunderscore , diz-se dos soldados que marcham um após outro, formando uma só fila.
Capital, haveres.
\textunderscore Fundos públicos\textunderscore , papéis de crédito, garantidos officialmente.
\section{Fundujo}
\begin{itemize}
\item {Grp. gram.:m.}
\end{itemize}
O mesmo que \textunderscore alçacu\textunderscore .
\section{Fundura}
\begin{itemize}
\item {Grp. gram.:f.}
\end{itemize}
\begin{itemize}
\item {Proveniência:(De \textunderscore fundo\textunderscore )}
\end{itemize}
Altura, desde a superfície até ao fundo, ou até á parte mais interior; profundidade: \textunderscore a fundura do lago\textunderscore .
\section{Funé}
\begin{itemize}
\item {Grp. gram.:f.}
\end{itemize}
\begin{itemize}
\item {Proveniência:(T. japon.)}
\end{itemize}
Pequena embarcação asiática. Cf. \textunderscore Peregrinação\textunderscore , CCXXIII.
\section{Fúnebre}
\begin{itemize}
\item {Grp. gram.:adj.}
\end{itemize}
\begin{itemize}
\item {Utilização:Fig.}
\end{itemize}
\begin{itemize}
\item {Proveniência:(Lat. \textunderscore funebris\textunderscore )}
\end{itemize}
Relativo á morte, ou a mortos, ou a coisas que se relacionam com os mortos: \textunderscore cortejo fúnebre\textunderscore .
Lúgubre; luctuoso: \textunderscore o pio fúnebre dos mochos\textunderscore .
\section{Funel}
\begin{itemize}
\item {Grp. gram.:m.}
\end{itemize}
\begin{itemize}
\item {Utilização:Prov.}
\end{itemize}
\begin{itemize}
\item {Utilização:minh.}
\end{itemize}
O mesmo que \textunderscore funéu\textunderscore .
\section{Fúnera}
\begin{itemize}
\item {Grp. gram.:f.}
\end{itemize}
\begin{itemize}
\item {Proveniência:(Lat. \textunderscore funera\textunderscore )}
\end{itemize}
Parenta mais próxima de um finado, a qual dirigia as carpideiras.
\section{Funeral}
\begin{itemize}
\item {Grp. gram.:adj.}
\end{itemize}
\begin{itemize}
\item {Grp. gram.:M.}
\end{itemize}
\begin{itemize}
\item {Utilização:Gír.}
\end{itemize}
Fúnebre.
Pompas fúnebres.
Ceremónias de enterramento.
Elogio.
\textunderscore Em funeral\textunderscore , em sinal de luto.
(B. lat. \textunderscore funeralis\textunderscore )
\section{Funerário}
\begin{itemize}
\item {Grp. gram.:adj.}
\end{itemize}
\begin{itemize}
\item {Proveniência:(Lat. \textunderscore funerarius\textunderscore )}
\end{itemize}
O mesmo que \textunderscore fúnebre\textunderscore .
\section{Funéreo}
\begin{itemize}
\item {Grp. gram.:adj.}
\end{itemize}
\begin{itemize}
\item {Proveniência:(Lat. \textunderscore funereus\textunderscore )}
\end{itemize}
O mesmo que \textunderscore fúnebre\textunderscore :«\textunderscore funérea campa com fragor rangeu\textunderscore ». S. Passos, \textunderscore Poesias\textunderscore .
\section{Funestação}
\begin{itemize}
\item {Grp. gram.:f.}
\end{itemize}
\begin{itemize}
\item {Proveniência:(Lat. \textunderscore funestatio\textunderscore )}
\end{itemize}
Acto ou effeito de funestar.
Luto.
\section{Funestador}
\begin{itemize}
\item {Grp. gram.:m.  e  adj.}
\end{itemize}
\begin{itemize}
\item {Proveniência:(Lat. \textunderscore funestator\textunderscore )}
\end{itemize}
O que funesta.
\section{Funestamente}
\begin{itemize}
\item {Grp. gram.:adv.}
\end{itemize}
De modo funesto.
\section{Funestar}
\begin{itemize}
\item {Grp. gram.:v. t.}
\end{itemize}
\begin{itemize}
\item {Proveniência:(Lat. \textunderscore funestare\textunderscore )}
\end{itemize}
Tornar funesto.
Infamar; estigmatizar.
\section{Funesto}
\begin{itemize}
\item {Grp. gram.:adj.}
\end{itemize}
\begin{itemize}
\item {Proveniência:(Lat. \textunderscore funestus\textunderscore )}
\end{itemize}
Que fere mortalmente; que produz morte: \textunderscore golpe funesto\textunderscore .
Que enluta.
Que destrói.
Que produz amargura: \textunderscore notícia funesta\textunderscore .
Que prognostica desgraça.
Fatal; cruel.
Desventurado.
\section{Funéu}
\begin{itemize}
\item {Grp. gram.:m.}
\end{itemize}
\begin{itemize}
\item {Utilização:T. do Pôrto}
\end{itemize}
Cordão ou corda, que passa por dentro de uma baínha, permittindo que esta se franza ou se desfranza.
(Cp. lat. \textunderscore funis\textunderscore )
\section{Fun-fun-gá-gá}
\begin{itemize}
\item {Grp. gram.:m.}
\end{itemize}
\begin{itemize}
\item {Utilização:Chul.}
\end{itemize}
\begin{itemize}
\item {Proveniência:(T. onom.)}
\end{itemize}
Philarmónica reles.
Musicata.
\section{Funga}
\begin{itemize}
\item {Grp. gram.:f.}
\end{itemize}
\begin{itemize}
\item {Proveniência:(De \textunderscore fungar\textunderscore )}
\end{itemize}
Doença de cães, caracterizada pela defluência de uma espécie de mormo, que lhes escorre das ventas.
\section{Fungação}
\begin{itemize}
\item {Grp. gram.:f.}
\end{itemize}
O mesmo que \textunderscore fungada\textunderscore .
\section{Fungada}
\begin{itemize}
\item {Grp. gram.:f.}
\end{itemize}
Acto de fungar.
\section{Fungadeira}
\begin{itemize}
\item {Grp. gram.:f.}
\end{itemize}
\begin{itemize}
\item {Utilização:Pop.}
\end{itemize}
\begin{itemize}
\item {Proveniência:(De \textunderscore fungar\textunderscore )}
\end{itemize}
Caixa de rapé; tabaqueira.
\section{Fungàgá}
\begin{itemize}
\item {Grp. gram.:m.}
\end{itemize}
O mesmo que \textunderscore fun-fun-gá-gá\textunderscore .
\section{Fungão}
\begin{itemize}
\item {Grp. gram.:m.  e  adj.}
\end{itemize}
\begin{itemize}
\item {Utilização:Pop.}
\end{itemize}
\begin{itemize}
\item {Utilização:Pop.}
\end{itemize}
\begin{itemize}
\item {Proveniência:(De \textunderscore fungar\textunderscore )}
\end{itemize}
O que toma rapé amiudadas vezes.
Nariz.
Criança, que chora.
\section{Fungão}
\begin{itemize}
\item {Grp. gram.:m.}
\end{itemize}
\begin{itemize}
\item {Proveniência:(Do lat. \textunderscore fungus\textunderscore )}
\end{itemize}
Gênero de cogumelos.
Cravagem.
Excrescência na pelle ou nas mucosas, em fórma de cogumelo.
\section{Fungar}
\begin{itemize}
\item {Grp. gram.:v. t.}
\end{itemize}
\begin{itemize}
\item {Grp. gram.:V. i.}
\end{itemize}
\begin{itemize}
\item {Utilização:Fam.}
\end{itemize}
\begin{itemize}
\item {Utilização:Fam.}
\end{itemize}
\begin{itemize}
\item {Proveniência:(T. onom.)}
\end{itemize}
Absorver pelo nariz.
Produzir som, absorvendo o ar pelo nariz ou tomando uma pitada de rapé.
Resmungar.
Sibilar.
Chorar, respirando só pelo nariz; choramingar.
\section{Fungicida}
\begin{itemize}
\item {Grp. gram.:adj.}
\end{itemize}
\begin{itemize}
\item {Utilização:Agr.}
\end{itemize}
\begin{itemize}
\item {Proveniência:(Do lat. \textunderscore fungus\textunderscore  + \textunderscore caedere\textunderscore )}
\end{itemize}
Que destrói os fungos ou fungões.
\section{Fúngico}
\begin{itemize}
\item {Grp. gram.:adj.}
\end{itemize}
\begin{itemize}
\item {Proveniência:(De \textunderscore fungo\textunderscore ^1)}
\end{itemize}
Diz-se de um ácido, extrahido de certos cogumelos.
\section{Fungícola}
\begin{itemize}
\item {Grp. gram.:adj.}
\end{itemize}
\begin{itemize}
\item {Proveniência:(Do lat. \textunderscore fungus\textunderscore  + \textunderscore colere\textunderscore )}
\end{itemize}
Que vive nos fungos.
\section{Fungiforme}
\begin{itemize}
\item {Grp. gram.:adj.}
\end{itemize}
\begin{itemize}
\item {Proveniência:(Do lat. \textunderscore fungus\textunderscore  + \textunderscore forma\textunderscore )}
\end{itemize}
Semelhante a cogumelo.
\section{Fungina}
\begin{itemize}
\item {Grp. gram.:f.}
\end{itemize}
\begin{itemize}
\item {Proveniência:(Do lat. \textunderscore fungus\textunderscore )}
\end{itemize}
Base orgânica do fungâo^2.
\section{Fungite}
\begin{itemize}
\item {Grp. gram.:f.}
\end{itemize}
\begin{itemize}
\item {Proveniência:(Do lat. \textunderscore fungus\textunderscore )}
\end{itemize}
Polypeiro fóssil.
\section{Fungível}
\begin{itemize}
\item {Grp. gram.:adj.}
\end{itemize}
\begin{itemize}
\item {Proveniência:(Do lat. \textunderscore fungi\textunderscore )}
\end{itemize}
Que se gasta; que se consome com o primeiro uso.
\section{Fungo}
\begin{itemize}
\item {Grp. gram.:m.}
\end{itemize}
\begin{itemize}
\item {Proveniência:(Lat. \textunderscore fungus\textunderscore . Cp. \textunderscore fungão\textunderscore ^2)}
\end{itemize}
Excrescência na pelle ou nas mucosas em fórma de cogumelo.
Fungão^2.
\section{Fungo}
\begin{itemize}
\item {Grp. gram.:m.}
\end{itemize}
\begin{itemize}
\item {Utilização:Neol.}
\end{itemize}
Acto de fungar, de farejar.
\section{Fungo}
\begin{itemize}
\item {Grp. gram.:m.}
\end{itemize}
\begin{itemize}
\item {Utilização:T. de Moçambique}
\end{itemize}
O mesmo que \textunderscore induna\textunderscore .
Fruto angolense, semelhante á ameixa. Cf. Capello e Ivens, I.
\section{Fungões}
\begin{itemize}
\item {Grp. gram.:m. pl.}
\end{itemize}
\begin{itemize}
\item {Utilização:Pop.}
\end{itemize}
\begin{itemize}
\item {Proveniência:(De \textunderscore fungão\textunderscore ^2)}
\end{itemize}
Ventas, cara: \textunderscore olha que te vou aos fungões\textunderscore .
\section{Fungosidade}
\begin{itemize}
\item {Grp. gram.:f.}
\end{itemize}
Qualidade daquillo que é fungoso.
Doença das vinhas, manifestada pela decomposição das raízes, separação da casca e formação de espêssa rede de filamentos brancos, em volta das raízes.
\section{Fungoso}
\begin{itemize}
\item {Grp. gram.:adj.}
\end{itemize}
\begin{itemize}
\item {Proveniência:(Lat. \textunderscore fungosus\textunderscore )}
\end{itemize}
Relativo ou semelhante a cogumelo.
Que é da natureza do fungo^1 ou do fungão^2.
Que tem poros, que é esponjoso.
\section{Funho}
\begin{itemize}
\item {Grp. gram.:m.}
\end{itemize}
\begin{itemize}
\item {Utilização:Prov.}
\end{itemize}
\begin{itemize}
\item {Utilização:alent.}
\end{itemize}
O mesmo que \textunderscore furúnculo\textunderscore .
Jôgo de rapazes, usado no inverno.
\section{Funicular}
\begin{itemize}
\item {Grp. gram.:adj.}
\end{itemize}
\begin{itemize}
\item {Proveniência:(De \textunderscore funículo\textunderscore )}
\end{itemize}
Composto de cordas.
Que funcciona por meio de cordas: \textunderscore um ascensor funicular\textunderscore .
\section{Funiculite}
\begin{itemize}
\item {Grp. gram.:f.}
\end{itemize}
\begin{itemize}
\item {Utilização:Med.}
\end{itemize}
\begin{itemize}
\item {Proveniência:(De \textunderscore funículo\textunderscore )}
\end{itemize}
Inflammação do cordão espermático.
\section{Funículo}
\begin{itemize}
\item {Grp. gram.:m.}
\end{itemize}
\begin{itemize}
\item {Utilização:Bot.}
\end{itemize}
\begin{itemize}
\item {Proveniência:(Lat. \textunderscore funiculus\textunderscore )}
\end{itemize}
Pequena corda.
Cordão umbilical.
Ligação, entre o grão e a placenta nos vegetaes.
\section{Funífero}
\begin{itemize}
\item {Grp. gram.:adj.}
\end{itemize}
\begin{itemize}
\item {Utilização:Bot.}
\end{itemize}
\begin{itemize}
\item {Proveniência:(Do lat. \textunderscore funis\textunderscore  + \textunderscore ferre\textunderscore )}
\end{itemize}
Diz-se das plantas, cujos compridos filamentos descaem perpendicularmente para o chão.
\section{Funiforme}
\begin{itemize}
\item {Grp. gram.:adj.}
\end{itemize}
\begin{itemize}
\item {Proveniência:(Do lat. \textunderscore funis\textunderscore  + \textunderscore forma\textunderscore )}
\end{itemize}
Semelhante a cordões.
\section{Funil}
\begin{itemize}
\item {Grp. gram.:m.}
\end{itemize}
\begin{itemize}
\item {Proveniência:(Do lat. \textunderscore infundile\textunderscore , por \textunderscore infundibulum\textunderscore )}
\end{itemize}
Utensílio, com a fórma de pyrâmide ou de cóne invertido, em cujo vértice há um tubo, e que serve para transvasar líquidos.
Objecto em fórma de funil.
\section{Funilaria}
\begin{itemize}
\item {Grp. gram.:f.}
\end{itemize}
\begin{itemize}
\item {Proveniência:(De \textunderscore funil\textunderscore )}
\end{itemize}
Loja de funileiro.
\section{Funileiro}
\begin{itemize}
\item {Grp. gram.:m.}
\end{itemize}
Fabricante de funis; latoeiro.
\section{Fúnkia}
\begin{itemize}
\item {Grp. gram.:f.}
\end{itemize}
\begin{itemize}
\item {Proveniência:(De \textunderscore Funk\textunderscore , n. p.)}
\end{itemize}
Gênero de plantas bulbosas, da fam. das liliáceas.
\section{Fura}
\begin{itemize}
\item {Utilização:Prov.}
\end{itemize}
\begin{itemize}
\item {Utilização:minh.}
\end{itemize}
\begin{itemize}
\item {Proveniência:(De \textunderscore furar\textunderscore )}
\end{itemize}
Furo, feito com verrumão grosso.
\section{Furabardo}
\begin{itemize}
\item {Grp. gram.:m.}
\end{itemize}
\begin{itemize}
\item {Utilização:T. da Madeira}
\end{itemize}
O mesmo que \textunderscore gavião\textunderscore .
O mesmo que \textunderscore tinge-burro\textunderscore .
\section{Furabolo}
\begin{itemize}
\item {Grp. gram.:m.  e  f.}
\end{itemize}
\begin{itemize}
\item {Utilização:Bras}
\end{itemize}
Pessôa curiosa, que se intromete em tudo.
\section{Furabolos}
\begin{itemize}
\item {Grp. gram.:m.}
\end{itemize}
\begin{itemize}
\item {Utilização:Pop.}
\end{itemize}
O dedo indicador.
\section{Furabosta}
\begin{itemize}
\item {Grp. gram.:m.}
\end{itemize}
\begin{itemize}
\item {Utilização:Açor}
\end{itemize}
O mesmo que \textunderscore melro\textunderscore .
\section{Furabugalhos}
\begin{itemize}
\item {Grp. gram.:m.}
\end{itemize}
\begin{itemize}
\item {Utilização:Prov.}
\end{itemize}
O mesmo que \textunderscore megengra\textunderscore .
O mesmo que \textunderscore abelheiro\textunderscore .
\section{Furabuxo}
\begin{itemize}
\item {Grp. gram.:m.}
\end{itemize}
Ave aquática, (\textunderscore puffinus anglorum\textunderscore , Tem.). Cf. Pero Vaz de Caminha, \textunderscore Carta a D. Man.\textunderscore 
\section{Furacamisas}
\begin{itemize}
\item {Grp. gram.:m.}
\end{itemize}
Crustáceo decápode, (\textunderscore grapsus varius\textunderscore ).
\section{Furacão}
\begin{itemize}
\item {Grp. gram.:m.}
\end{itemize}
\begin{itemize}
\item {Utilização:Fig.}
\end{itemize}
\begin{itemize}
\item {Utilização:Gír.}
\end{itemize}
Ventania violenta e súbita.
Tufão.
Tudo que destrói com violência e rapidez.
Morte de homem.
(Cast. \textunderscore huracan\textunderscore )
\section{Furacapa}
\begin{itemize}
\item {Grp. gram.:f.}
\end{itemize}
Planta gramínea, (\textunderscore stipa toriollis\textunderscore , Desf.).
\section{Furacar}
\begin{itemize}
\item {Grp. gram.:v. t.}
\end{itemize}
\begin{itemize}
\item {Utilização:Fam.}
\end{itemize}
\begin{itemize}
\item {Proveniência:(De \textunderscore furaco\textunderscore )}
\end{itemize}
O mesmo que \textunderscore esburacar\textunderscore .
\section{Furacidade}
\begin{itemize}
\item {Grp. gram.:f.}
\end{itemize}
\begin{itemize}
\item {Utilização:Des.}
\end{itemize}
\begin{itemize}
\item {Proveniência:(Lat. \textunderscore furacitas\textunderscore )}
\end{itemize}
Tendência para roubar.
Hábito de roubar.
\section{Furaco}
\begin{itemize}
\item {Grp. gram.:m.}
\end{itemize}
\begin{itemize}
\item {Utilização:Ant.}
\end{itemize}
O mesmo que \textunderscore buraco\textunderscore .
(Infl. de \textunderscore furo\textunderscore )
\section{Fura-crânio}
\begin{itemize}
\item {Grp. gram.:m.}
\end{itemize}
Instrumento cirúrgico, para abrir crânios. Cf. \textunderscore Tarifa das Alfândegas\textunderscore , no Brasil.
\section{Furado}
\begin{itemize}
\item {Grp. gram.:m.}
\end{itemize}
\begin{itemize}
\item {Utilização:Bras}
\end{itemize}
O mesmo que \textunderscore furo\textunderscore .
\section{Furadoiro}
\begin{itemize}
\item {Grp. gram.:m.}
\end{itemize}
\begin{itemize}
\item {Utilização:Ant.}
\end{itemize}
\begin{itemize}
\item {Proveniência:(De \textunderscore furar\textunderscore )}
\end{itemize}
Atalho, por onde alguém foge, sem sêr visto.
\section{Furador}
\begin{itemize}
\item {Grp. gram.:m.}
\end{itemize}
\begin{itemize}
\item {Grp. gram.:Adj.}
\end{itemize}
\begin{itemize}
\item {Utilização:Bras. de Minas}
\end{itemize}
\begin{itemize}
\item {Proveniência:(De \textunderscore furar\textunderscore )}
\end{itemize}
Utensílio de metal, osso ou marfim, para fazer furos ou ilhós.
Feliz; emprehendedor.
\section{Furadouro}
\begin{itemize}
\item {Grp. gram.:m.}
\end{itemize}
\begin{itemize}
\item {Utilização:Ant.}
\end{itemize}
\begin{itemize}
\item {Proveniência:(De \textunderscore furar\textunderscore )}
\end{itemize}
Atalho, por onde alguém foge, sem sêr visto.
\section{Furafigos}
\begin{itemize}
\item {Grp. gram.:m.}
\end{itemize}
Ave, o mesmo que \textunderscore papa-môscas\textunderscore .
\section{Furagem}
\begin{itemize}
\item {Grp. gram.:f.}
\end{itemize}
Acto de furar.
Processo para se obter das nascentes salgadas o líquido com o necessário grau da mineralização, furando o terreno e aspirando-se o líquido por meio de bombas. Cf. \textunderscore Museu Techn.\textunderscore , 121.
\section{Furageôlhos}
\begin{itemize}
\item {Grp. gram.:m.}
\end{itemize}
\begin{itemize}
\item {Utilização:Ant.}
\end{itemize}
\begin{itemize}
\item {Utilização:Pop.}
\end{itemize}
O mesmo que \textunderscore alfaiate\textunderscore .
\section{Fural}
\begin{itemize}
\item {Grp. gram.:m.}
\end{itemize}
\begin{itemize}
\item {Utilização:Açor}
\end{itemize}
\begin{itemize}
\item {Proveniência:(De \textunderscore furo\textunderscore )}
\end{itemize}
Rua estreita, travéssa.
\section{Furamar}
\begin{itemize}
\item {Grp. gram.:m.}
\end{itemize}
\begin{itemize}
\item {Utilização:Mad}
\end{itemize}
Ave, o mesmo que \textunderscore pata-garro\textunderscore .
\section{Furamato}
\begin{itemize}
\item {Grp. gram.:m.}
\end{itemize}
O mesmo que \textunderscore tiriba-pequeno\textunderscore .
Ave ribeirinha, (\textunderscore rallus aquaticua\textunderscore , Lin.).
\section{Fura-moiteiras}
\begin{itemize}
\item {Grp. gram.:m.}
\end{itemize}
\begin{itemize}
\item {Utilização:T. da Bairrada}
\end{itemize}
O mesmo que \textunderscore furaparedes\textunderscore .
\section{Furamoteiras}
\begin{itemize}
\item {Grp. gram.:m.}
\end{itemize}
\begin{itemize}
\item {Utilização:T. da Bairrada}
\end{itemize}
O mesmo que \textunderscore furaparedes\textunderscore .
\section{Furaneve}
\begin{itemize}
\item {Grp. gram.:f.}
\end{itemize}
\begin{itemize}
\item {Utilização:Bras}
\end{itemize}
Planta amaryllídea, emética.
\section{Furão}
\begin{itemize}
\item {Grp. gram.:m.}
\end{itemize}
\begin{itemize}
\item {Utilização:Fig.}
\end{itemize}
\begin{itemize}
\item {Utilização:Fam.}
\end{itemize}
\begin{itemize}
\item {Proveniência:(Do b. lat. \textunderscore furo\textunderscore )}
\end{itemize}
Pequeno mammífero vermiforme, que os caçadores empregam, para fazer saír os coêlhos das covas.
Pessôa bisbilhoteira, curiosa.
Pessôa, que come pouco: \textunderscore é um furão, a comer\textunderscore .
\section{Fura-panasco}
\begin{itemize}
\item {Grp. gram.:m.}
\end{itemize}
\begin{itemize}
\item {Utilização:Prov.}
\end{itemize}
O mesmo que \textunderscore licranço\textunderscore .
\section{Furaparedes}
\begin{itemize}
\item {Grp. gram.:m.  e  f.}
\end{itemize}
\begin{itemize}
\item {Utilização:Pop.}
\end{itemize}
Pessôa activa, esperta; fura-vidas.
\section{Furapastos}
\begin{itemize}
\item {Grp. gram.:m.}
\end{itemize}
\begin{itemize}
\item {Utilização:Prov.}
\end{itemize}
\begin{itemize}
\item {Utilização:alent.}
\end{itemize}
Pequeno reptil, (\textunderscore lacerta chalcides\textunderscore , Lin.).
\section{Furapaus}
\begin{itemize}
\item {Grp. gram.:m. pl.}
\end{itemize}
Insectos, da ordem dos coleópteros, de antennas filiformes e elytros duros.
\section{Furar}
\begin{itemize}
\item {Grp. gram.:v. t.}
\end{itemize}
\begin{itemize}
\item {Utilização:Fig.}
\end{itemize}
\begin{itemize}
\item {Grp. gram.:V. i.}
\end{itemize}
\begin{itemize}
\item {Proveniência:(Do lat. \textunderscore forare\textunderscore )}
\end{itemize}
Abrir furo em; esburacar.
Arrombar.
Romper.
Frustrar; transtornar: \textunderscore furar uma greve\textunderscore .
Irromper, saír.
\section{Furável}
\begin{itemize}
\item {Grp. gram.:adj.}
\end{itemize}
Que se póde furar.
\section{Furavidas}
\begin{itemize}
\item {Grp. gram.:m.  e  f.}
\end{itemize}
\begin{itemize}
\item {Utilização:Fam.}
\end{itemize}
Pessôa activa, que trata especialmente das suas commodidades e vantagens.
\section{Furbesco}
\begin{itemize}
\item {fónica:bês}
\end{itemize}
\begin{itemize}
\item {Grp. gram.:m.}
\end{itemize}
\begin{itemize}
\item {Proveniência:(It. \textunderscore furbesco\textunderscore )}
\end{itemize}
Gíria ou calão italiano.
\section{Furcate}
\begin{itemize}
\item {Grp. gram.:m.}
\end{itemize}
\begin{itemize}
\item {Utilização:Prov.}
\end{itemize}
\begin{itemize}
\item {Proveniência:(Do lat. \textunderscore furca\textunderscore )}
\end{itemize}
Espécie de collar de madeira, nas bêstas de tiro.
\section{Furcífero}
\begin{itemize}
\item {Grp. gram.:adj.}
\end{itemize}
\begin{itemize}
\item {Proveniência:(Do lat. \textunderscore furca\textunderscore  + \textunderscore ferre\textunderscore )}
\end{itemize}
Que tem uma parte do corpo bifurcado.
\section{Furco}
\begin{itemize}
\item {Grp. gram.:m.}
\end{itemize}
\begin{itemize}
\item {Utilização:Prov.}
\end{itemize}
\begin{itemize}
\item {Proveniência:(Do rad. do lat. \textunderscore furca\textunderscore )}
\end{itemize}
Medida ou distância, igual á que vai da extremidade do dedo pollegar á do indicador, e que corresponde a três quartos do palmo.
\section{Furcroia}
\begin{itemize}
\item {Grp. gram.:f.}
\end{itemize}
\begin{itemize}
\item {Proveniência:(De \textunderscore Fourcroy\textunderscore , n. p.)}
\end{itemize}
Gênero de plantas amarilídeas.
\section{Furcroya}
\begin{itemize}
\item {Grp. gram.:f.}
\end{itemize}
\begin{itemize}
\item {Proveniência:(De \textunderscore Fourcroy\textunderscore , n. p.)}
\end{itemize}
Gênero de plantas amarillýdeas.
\section{Fúrcula}
\begin{itemize}
\item {Grp. gram.:f.}
\end{itemize}
\begin{itemize}
\item {Utilização:Anat.}
\end{itemize}
\begin{itemize}
\item {Proveniência:(Lat. \textunderscore furcula\textunderscore )}
\end{itemize}
Designação antiga da parte superior do esterno.
\textunderscore Fúrcula do esterno\textunderscore , chanfradura na parte superior do esterno.
\section{Furda}
\begin{itemize}
\item {Grp. gram.:f.}
\end{itemize}
\begin{itemize}
\item {Utilização:Prov.}
\end{itemize}
Cabana; choça.
(Cp. \textunderscore alfurja\textunderscore )
\section{Furdunço}
\begin{itemize}
\item {Grp. gram.:m.}
\end{itemize}
\begin{itemize}
\item {Utilização:Bras. do N}
\end{itemize}
Barulho, desordem.
\section{Furegas}
\begin{itemize}
\item {Grp. gram.:m.}
\end{itemize}
\begin{itemize}
\item {Utilização:T. da Bairrada}
\end{itemize}
Indivíduo de feições miúdas.
Indivíduo que come pouco, biqueiro.
(Cp. \textunderscore furão\textunderscore )
\section{Furente}
\begin{itemize}
\item {Grp. gram.:adj.}
\end{itemize}
\begin{itemize}
\item {Utilização:Poét.}
\end{itemize}
\begin{itemize}
\item {Proveniência:(Lat. \textunderscore furens\textunderscore )}
\end{itemize}
Enfurecido, colérico. Cf. F. Barreto, \textunderscore Eneida\textunderscore , I, 13.
\section{Furfuração}
\begin{itemize}
\item {Grp. gram.:f.}
\end{itemize}
\begin{itemize}
\item {Utilização:Des.}
\end{itemize}
\begin{itemize}
\item {Proveniência:(Lat. \textunderscore furfuratio\textunderscore )}
\end{itemize}
Producção de caspa na cabeça.
\section{Furfuráceo}
\begin{itemize}
\item {Grp. gram.:adj.}
\end{itemize}
\begin{itemize}
\item {Proveniência:(Lat. \textunderscore furfuraceus\textunderscore )}
\end{itemize}
Relativo ou semelhante a farelo.
\section{Furfuramido}
\begin{itemize}
\item {Grp. gram.:m.}
\end{itemize}
Substância amarelada, produzida pela acção do ammoníaco sôbre o furfurol.
\section{Furfúreo}
\begin{itemize}
\item {Grp. gram.:adj.}
\end{itemize}
\begin{itemize}
\item {Proveniência:(Lat. \textunderscore furfureus\textunderscore )}
\end{itemize}
O mesmo que \textunderscore furfuráceo\textunderscore .
\section{Furfurina}
\begin{itemize}
\item {Grp. gram.:f.}
\end{itemize}
Alcaloide, produzido pela acção da potassa sôbre o furfuramido.
\section{Furfurol}
\begin{itemize}
\item {Grp. gram.:m.}
\end{itemize}
\begin{itemize}
\item {Proveniência:(Do lat. \textunderscore furfur\textunderscore  + \textunderscore oleum\textunderscore )}
\end{itemize}
Óleo quási incolor, obtido pela acção do ácido sulfúrico sôbre a farinha de aveia.
\section{Furgão}
\begin{itemize}
\item {Grp. gram.:m.}
\end{itemize}
\begin{itemize}
\item {Proveniência:(Fr. \textunderscore fourgon\textunderscore )}
\end{itemize}
Carro coberto, que faz parte de um combóio e serve para transporte de bagagens.
\section{Fúria}
\begin{itemize}
\item {Grp. gram.:f.}
\end{itemize}
\begin{itemize}
\item {Grp. gram.:Pl.}
\end{itemize}
\begin{itemize}
\item {Proveniência:(Lat. \textunderscore furia\textunderscore )}
\end{itemize}
Furor, ira, raiva.
Exaltação de ânimo.
Impeto.
Enthusiasmo.
Valentia.
Precipitação de procedimento.
Pessôa furiosa.
Divindades infernaes, segundo a Mythologia.
\section{Furial}
\begin{itemize}
\item {Grp. gram.:adj.}
\end{itemize}
\begin{itemize}
\item {Proveniência:(Lat. \textunderscore furialis\textunderscore )}
\end{itemize}
O mesmo que \textunderscore furioso\textunderscore .
\section{Furianos}
\begin{itemize}
\item {Grp. gram.:m. pl.}
\end{itemize}
Habitantes do Dar-Fur.
\section{Furibundo}
\begin{itemize}
\item {Grp. gram.:adj.}
\end{itemize}
\begin{itemize}
\item {Proveniência:(Lat. \textunderscore furibundus\textunderscore )}
\end{itemize}
O mesmo que \textunderscore furioso\textunderscore .
\section{Furifolha}
\begin{itemize}
\item {Grp. gram.:f.}
\end{itemize}
O mesmo que \textunderscore firafolha\textunderscore .
\section{Furifunar}
\begin{itemize}
\item {Grp. gram.:v. t.}
\end{itemize}
\begin{itemize}
\item {Utilização:Fam.}
\end{itemize}
Tocar mal ou desafinadamente:«\textunderscore os violinos furifunavam uma espécie de acompanhamento\textunderscore ». Ortigão, \textunderscore Hollanda\textunderscore .
\section{Furiosamente}
\begin{itemize}
\item {Grp. gram.:adv.}
\end{itemize}
De modo furioso.
\section{Furiosidade}
\begin{itemize}
\item {Grp. gram.:f.}
\end{itemize}
Qualidade daquelle ou daquillo que é furioso.
\section{Furioso}
\begin{itemize}
\item {Grp. gram.:adj.}
\end{itemize}
\begin{itemize}
\item {Proveniência:(Lat. \textunderscore furiosus\textunderscore )}
\end{itemize}
Que tem fúria.
Irritado, raivoso.
Impetuoso: \textunderscore vento furioso\textunderscore .
Enthusiasta.
Apaixonado.
Extraordinário.
Forte, resistente.
\section{Furjoco}
\begin{itemize}
\item {fónica:jô}
\end{itemize}
\begin{itemize}
\item {Grp. gram.:m.}
\end{itemize}
\begin{itemize}
\item {Utilização:Prov.}
\end{itemize}
\begin{itemize}
\item {Utilização:trasm.}
\end{itemize}
Buraco; caverna; gruta.
(Cp. \textunderscore alfurja\textunderscore )
\section{Furna}
\begin{itemize}
\item {Grp. gram.:f.}
\end{itemize}
\begin{itemize}
\item {Proveniência:(Do lat. \textunderscore furnus\textunderscore ?)}
\end{itemize}
Cova; caverna.
Antro; lapa; subterrâneo.
\section{Furo}
\begin{itemize}
\item {Grp. gram.:m.}
\end{itemize}
\begin{itemize}
\item {Utilização:Fam.}
\end{itemize}
\begin{itemize}
\item {Utilização:Bras}
\end{itemize}
\begin{itemize}
\item {Grp. gram.:Pl.}
\end{itemize}
\begin{itemize}
\item {Proveniência:(De \textunderscore furar\textunderscore )}
\end{itemize}
Abertura artificial.
Buraco; orifício.
Grau.
Modo de resolver um negócio.
Espaço navegável, entre arvoredos ou plantas aquáticas.
Desharmonia nas côres de um quadro.
Últimos retoques em alguns pontos de um quadro.
\section{Furoar}
\begin{itemize}
\item {Grp. gram.:v. t.}
\end{itemize}
\begin{itemize}
\item {Proveniência:(De \textunderscore furão\textunderscore )}
\end{itemize}
Procurar qualquer, á maneira de furão.
Pesquisar; investigar. Cf. Castilho, \textunderscore Fausto\textunderscore , 189.
\section{Furoeira}
\begin{itemize}
\item {Grp. gram.:f.}
\end{itemize}
\begin{itemize}
\item {Proveniência:(De \textunderscore furão\textunderscore )}
\end{itemize}
Caixa, onde se guarda ou se leva para a caça o furão.
\section{Furoeiro}
\begin{itemize}
\item {Grp. gram.:m.}
\end{itemize}
\begin{itemize}
\item {Proveniência:(De \textunderscore furão\textunderscore )}
\end{itemize}
Criador e negociador de furões.
\section{Furor}
\begin{itemize}
\item {Grp. gram.:m.}
\end{itemize}
\begin{itemize}
\item {Proveniência:(Lat. \textunderscore furor\textunderscore )}
\end{itemize}
Grande exaltação de ânimo; fúria.
Delírio.
Frenesi.
Impetuosidade.
Enthusiasmo.
\section{Furriel}
\begin{itemize}
\item {Grp. gram.:m.}
\end{itemize}
Pôsto militar, que não existe actualmente, e que era inferior ao de sargento.
(Cast. \textunderscore furriel\textunderscore )
\section{Furrinaes}
\begin{itemize}
\item {Grp. gram.:f. pl.}
\end{itemize}
\begin{itemize}
\item {Proveniência:(Lat. \textunderscore furrinalia\textunderscore )}
\end{itemize}
Festas romanas, que se celebravam a 25 de Julho, em honra da deusa Furrína ou Furina. Cf. Castilho, \textunderscore Fastos\textunderscore , I, 235.
\section{Furrinais}
\begin{itemize}
\item {Grp. gram.:f. pl.}
\end{itemize}
\begin{itemize}
\item {Proveniência:(Lat. \textunderscore furrinalia\textunderscore )}
\end{itemize}
Festas romanas, que se celebravam a 25 de Julho, em honra da deusa Furrína ou Furina. Cf. Castilho, \textunderscore Fastos\textunderscore , I, 235.
\section{Furrundu}
\begin{itemize}
\item {Grp. gram.:m.}
\end{itemize}
\begin{itemize}
\item {Utilização:Bras}
\end{itemize}
Espécie de doce, feito de cidra ralada, gengibre e açúcar mascavado.
Espécie de dança campestre.
\section{Furrundum}
\begin{itemize}
\item {Grp. gram.:m.}
\end{itemize}
\begin{itemize}
\item {Utilização:Bras}
\end{itemize}
O mesmo que \textunderscore furrundu\textunderscore .
\section{Furtacamisas}
\begin{itemize}
\item {Grp. gram.:m.}
\end{itemize}
(V.furacamisas)
\section{Furtacapa}
\begin{itemize}
\item {Grp. gram.:m.}
\end{itemize}
\begin{itemize}
\item {Utilização:Taur.}
\end{itemize}
O capinha.
\section{Furtacôr}
\begin{itemize}
\item {Grp. gram.:adj.}
\end{itemize}
\begin{itemize}
\item {Grp. gram.:M.}
\end{itemize}
Cambiante.
Que apresenta côr variada, segundo a projecção da luz.
A côr cambiante: \textunderscore um tecido de furtacôres\textunderscore .
\section{Furtadamente}
\begin{itemize}
\item {Grp. gram.:adv.}
\end{itemize}
\begin{itemize}
\item {Proveniência:(De \textunderscore furtar\textunderscore )}
\end{itemize}
Furtivamente; ás occultas.
\section{Furtadela}
\begin{itemize}
\item {Grp. gram.:f.}
\end{itemize}
\begin{itemize}
\item {Grp. gram.:Loc. adv.}
\end{itemize}
\begin{itemize}
\item {Proveniência:(De \textunderscore furtar\textunderscore )}
\end{itemize}
Acção de furtar ou de esconder.
\textunderscore Ás furtadelas\textunderscore , ás escondidas.
\section{Furtador}
\begin{itemize}
\item {Grp. gram.:m.}
\end{itemize}
\begin{itemize}
\item {Utilização:T. de Ceilão}
\end{itemize}
Aquelle que furta.
\section{Furtafogo}
\begin{itemize}
\item {Grp. gram.:m.}
\end{itemize}
Luzeiro escondido.
\textunderscore Lanterna de furta-fogo\textunderscore , apparelho, que esconde a luz, sem a apagar.
\section{Furtamontinho}
\begin{itemize}
\item {Grp. gram.:m.}
\end{itemize}
Espécie de jôgo popular. Cf. Rev. \textunderscore Tradição\textunderscore , II, 132.
\section{Furtapasso}
\begin{itemize}
\item {Grp. gram.:m.}
\end{itemize}
\begin{itemize}
\item {Utilização:Des.}
\end{itemize}
\begin{itemize}
\item {Grp. gram.:Loc. adv.}
\end{itemize}
Bôa andadura do cavallo, cômmoda ao cavalleiro.
\textunderscore A furta-passo\textunderscore , com cautela.
\section{Furtar}
\begin{itemize}
\item {Grp. gram.:v. t.}
\end{itemize}
\begin{itemize}
\item {Grp. gram.:Loc. adv.}
\end{itemize}
\begin{itemize}
\item {Utilização:Prov.}
\end{itemize}
Subtrahir fraudulentamente.
Apresentar como seu (aquillo que é de outrem).
Afastar, desviar.
Falsificar.
\textunderscore Á-furta-lhe-o-fato\textunderscore , furtivamente, dissimuladamente.
(B. lat. \textunderscore furtare\textunderscore )
\section{Furtivamente}
\begin{itemize}
\item {Grp. gram.:adv.}
\end{itemize}
De modo furtivo.
\section{Furtivo}
\begin{itemize}
\item {Grp. gram.:adj.}
\end{itemize}
\begin{itemize}
\item {Proveniência:(Lat. \textunderscore furtivus\textunderscore )}
\end{itemize}
Praticado a furto, occultamente.
Occulto: \textunderscore caçador furtivo\textunderscore .
\section{Furto}
\begin{itemize}
\item {Grp. gram.:m.}
\end{itemize}
\begin{itemize}
\item {Grp. gram.:Loc. Adv.}
\end{itemize}
\begin{itemize}
\item {Proveniência:(Lat. \textunderscore furtum\textunderscore )}
\end{itemize}
Acto ou effeito de furtar.
Aquillo que se furtou.
\textunderscore A furto\textunderscore , occultamente; dissimuladamente.
\section{Furuncular}
\begin{itemize}
\item {Grp. gram.:adj.}
\end{itemize}
Relativo a furúnculo.
Que é da natureza de furúnculo.
\section{Furúnculo}
\begin{itemize}
\item {Grp. gram.:m.}
\end{itemize}
\begin{itemize}
\item {Proveniência:(Lat. \textunderscore furunculus\textunderscore )}
\end{itemize}
Tumor pequeno e duro, formado na superfície da pelle, e acompanhado de inflammação e dôr.
Leicenço.
\section{Furunculose}
\begin{itemize}
\item {Grp. gram.:f.}
\end{itemize}
\begin{itemize}
\item {Utilização:Med.}
\end{itemize}
Erupção de uma série de furúnculos.
\section{Furunculoso}
\begin{itemize}
\item {Grp. gram.:adj.}
\end{itemize}
Relativo ou semelhante a furúnculo.
Attreito a furúnculos; que tem furúnculos.
\section{Fusa}
\begin{itemize}
\item {Grp. gram.:f.}
\end{itemize}
\begin{itemize}
\item {Proveniência:(It. \textunderscore fusa\textunderscore )}
\end{itemize}
Sinal musical, correspondente a metade da semi-colcheia.
\section{Fusa}
\begin{itemize}
\item {Grp. gram.:f.}
\end{itemize}
\begin{itemize}
\item {Utilização:Prov.}
\end{itemize}
\begin{itemize}
\item {Utilização:trasm.}
\end{itemize}
Espécie de fuso, com uma rodela no fundo.
\section{Fusada}
\begin{itemize}
\item {Grp. gram.:f.}
\end{itemize}
Porção de fio, enrolada no fuso.
Pancada ou golpe com o fuso.
\section{Fusano}
\begin{itemize}
\item {Grp. gram.:m.}
\end{itemize}
\begin{itemize}
\item {Proveniência:(Do rad. de \textunderscore fuso\textunderscore )}
\end{itemize}
Gênero de arbustos santaláceos da Austrália.
\section{Fusão}
\begin{itemize}
\item {Grp. gram.:f.}
\end{itemize}
\begin{itemize}
\item {Proveniência:(Lat. \textunderscore fusio\textunderscore )}
\end{itemize}
Acto ou effeito de fundir.
Mistura; alliança: \textunderscore fusão de dois partidos\textunderscore .
Associação.
\section{Fusário}
\begin{itemize}
\item {Grp. gram.:m.}
\end{itemize}
\begin{itemize}
\item {Proveniência:(De \textunderscore fuso\textunderscore )}
\end{itemize}
Gênero de cogumelos parasitos.
\section{Fúsaro}
\begin{itemize}
\item {Grp. gram.:m.}
\end{itemize}
O mesmo que \textunderscore sanguinheiro\textunderscore .
\section{Fusca}
\begin{itemize}
\item {Grp. gram.:f.}
\end{itemize}
\begin{itemize}
\item {Utilização:Gír.}
\end{itemize}
\begin{itemize}
\item {Proveniência:(De \textunderscore fusco\textunderscore )}
\end{itemize}
Espécie de pato selvagem, de peito, asas e lombo escuros.
A justiça.
\section{Fuscalvo}
\begin{itemize}
\item {Grp. gram.:adj.}
\end{itemize}
\begin{itemize}
\item {Proveniência:(De \textunderscore fusco\textunderscore  + \textunderscore alvo\textunderscore )}
\end{itemize}
Claro-escuro.
\section{Fuscar}
\begin{itemize}
\item {Grp. gram.:v. t.}
\end{itemize}
O mesmo que \textunderscore foscar\textunderscore .
\section{Fuscicollo}
\begin{itemize}
\item {Grp. gram.:adj.}
\end{itemize}
\begin{itemize}
\item {Utilização:Zool.}
\end{itemize}
\begin{itemize}
\item {Proveniência:(De \textunderscore fusco\textunderscore  + \textunderscore collo\textunderscore )}
\end{itemize}
Que tem o pescoço pardo.
\section{Fuscicolo}
\begin{itemize}
\item {Grp. gram.:adj.}
\end{itemize}
\begin{itemize}
\item {Utilização:Zool.}
\end{itemize}
\begin{itemize}
\item {Proveniência:(De \textunderscore fusco\textunderscore  + \textunderscore colo\textunderscore )}
\end{itemize}
Que tem o pescoço pardo.
\section{Fuscicórneo}
\begin{itemize}
\item {Grp. gram.:adj.}
\end{itemize}
\begin{itemize}
\item {Utilização:Zool.}
\end{itemize}
\begin{itemize}
\item {Proveniência:(De \textunderscore fusco\textunderscore  + \textunderscore córneo\textunderscore )}
\end{itemize}
Que tem as antennas pardas.
\section{Fuscímano}
\begin{itemize}
\item {Grp. gram.:adj.}
\end{itemize}
\begin{itemize}
\item {Utilização:Zool.}
\end{itemize}
\begin{itemize}
\item {Proveniência:(Do lat. \textunderscore fuscus\textunderscore  + \textunderscore manus\textunderscore )}
\end{itemize}
Que tem as patas anteriores escuras.
\section{Fuscina}
\begin{itemize}
\item {Grp. gram.:f.}
\end{itemize}
\begin{itemize}
\item {Proveniência:(De \textunderscore fusco\textunderscore )}
\end{itemize}
Substância escura, que se extrai do óleo animal de Dippel.
\section{Fuscina}
\begin{itemize}
\item {Grp. gram.:f.}
\end{itemize}
\begin{itemize}
\item {Proveniência:(Lat. \textunderscore fuscina\textunderscore , por \textunderscore furcina\textunderscore , de \textunderscore furca\textunderscore )}
\end{itemize}
O mesmo que \textunderscore fisga\textunderscore .
\section{Fuscipene}
\begin{itemize}
\item {Grp. gram.:adj.}
\end{itemize}
O mesmo que \textunderscore fuscipêneo\textunderscore .
\section{Fuscipêneo}
\begin{itemize}
\item {Grp. gram.:adj.}
\end{itemize}
\begin{itemize}
\item {Utilização:Zool.}
\end{itemize}
\begin{itemize}
\item {Proveniência:(De \textunderscore fusco\textunderscore  + \textunderscore pena\textunderscore )}
\end{itemize}
Que tem penas pardas.
\section{Fuscipenne}
\begin{itemize}
\item {Grp. gram.:adj.}
\end{itemize}
O mesmo que \textunderscore fuscipênneo\textunderscore .
\section{Fuscipênneo}
\begin{itemize}
\item {Grp. gram.:adj.}
\end{itemize}
\begin{itemize}
\item {Utilização:Zool.}
\end{itemize}
\begin{itemize}
\item {Proveniência:(De \textunderscore fusco\textunderscore  + \textunderscore penna\textunderscore )}
\end{itemize}
Que tem pennas pardas.
\section{Fuscirostro}
\begin{itemize}
\item {fónica:rós}
\end{itemize}
\begin{itemize}
\item {Grp. gram.:adj.}
\end{itemize}
\begin{itemize}
\item {Utilização:Zool.}
\end{itemize}
\begin{itemize}
\item {Proveniência:(De \textunderscore fusco\textunderscore  + \textunderscore rostro\textunderscore )}
\end{itemize}
Que tem o bico pardo.
\section{Fuscirrostro}
\begin{itemize}
\item {Grp. gram.:adj.}
\end{itemize}
\begin{itemize}
\item {Utilização:Zool.}
\end{itemize}
\begin{itemize}
\item {Proveniência:(De \textunderscore fusco\textunderscore  + \textunderscore rostro\textunderscore )}
\end{itemize}
Que tem o bico pardo.
\section{Fuscite}
\begin{itemize}
\item {Grp. gram.:f.}
\end{itemize}
\begin{itemize}
\item {Proveniência:(De \textunderscore fusco\textunderscore )}
\end{itemize}
Mineral norueguês, de côr parda.
\section{Fusco}
\begin{itemize}
\item {Grp. gram.:adj.}
\end{itemize}
\begin{itemize}
\item {Utilização:Ant.}
\end{itemize}
\begin{itemize}
\item {Grp. gram.:M.}
\end{itemize}
\begin{itemize}
\item {Proveniência:(Lat. \textunderscore fuscus\textunderscore )}
\end{itemize}
Escuro; pardo.
Triste, melancólico.
\textunderscore Lusco-fusco\textunderscore , o anoitecer.
\section{Fuscoito}
\begin{itemize}
\item {Grp. gram.:adj.}
\end{itemize}
\begin{itemize}
\item {Utilização:P. us.}
\end{itemize}
\begin{itemize}
\item {Utilização:Fig.}
\end{itemize}
\begin{itemize}
\item {Proveniência:(De \textunderscore fusco\textunderscore )}
\end{itemize}
Um tanto fusco.
Áspero; desabrido. Cf. Rebello, \textunderscore Contos e Lendas\textunderscore , 183.
\section{Fuseira}
\begin{itemize}
\item {Grp. gram.:f.}
\end{itemize}
Fuso grande.
\section{Fuseiro}
\begin{itemize}
\item {Grp. gram.:m.}
\end{itemize}
Fabricante de fusos.
Torneiro.
\section{Fusela}
\begin{itemize}
\item {Grp. gram.:f.}
\end{itemize}
\begin{itemize}
\item {Utilização:Heráld.}
\end{itemize}
\begin{itemize}
\item {Proveniência:(De \textunderscore fuso\textunderscore )}
\end{itemize}
Ornato do escudo, semelhante á lisonja, mas mais delgado ou afilado.
O mesmo que \textunderscore fuselo\textunderscore .
\section{Fuselado}
\begin{itemize}
\item {Grp. gram.:adj.}
\end{itemize}
Que tem fuselas.
Que tem fórma de fuso.
\section{Fuselo}
\begin{itemize}
\item {fónica:ê}
\end{itemize}
\begin{itemize}
\item {Grp. gram.:m.}
\end{itemize}
\begin{itemize}
\item {Proveniência:(De \textunderscore fuso\textunderscore )}
\end{itemize}
Cada um dos paus que sustêm as rodas parallelas do carrête.
Ave ribeirinha, (\textunderscore totanus calídris\textunderscore , Lin.).
\section{Fusibilidade}
\begin{itemize}
\item {Grp. gram.:f.}
\end{itemize}
Qualidade daquillo que é fusível.
\section{Fusiforme}
\begin{itemize}
\item {Grp. gram.:adj.}
\end{itemize}
\begin{itemize}
\item {Proveniência:(Do lat. \textunderscore fusus\textunderscore  + \textunderscore forma\textunderscore )}
\end{itemize}
Que tem fórma de fuso.
\section{Fusicórneos}
\begin{itemize}
\item {Grp. gram.:m. pl.}
\end{itemize}
\begin{itemize}
\item {Proveniência:(De \textunderscore fuso\textunderscore  + \textunderscore córneo\textunderscore )}
\end{itemize}
Família de insectos lepidópteros, com antennas grossas no meio.
\section{Fúsil}
\begin{itemize}
\item {Grp. gram.:adj.}
\end{itemize}
\begin{itemize}
\item {Proveniência:(Lat. \textunderscore fusilis\textunderscore )}
\end{itemize}
Que se póde fundir.
Fundido.
\section{Fusileira}
\begin{itemize}
\item {Grp. gram.:f.}
\end{itemize}
\begin{itemize}
\item {Utilização:Ant.}
\end{itemize}
\begin{itemize}
\item {Proveniência:(Do lat. \textunderscore fusilis\textunderscore )}
\end{itemize}
Fábrica de peças de artilheria? fundição? Cf. Castanheda, \textunderscore Descobr. da Índia\textunderscore , l. III, c. 75, l. V, c. 11.
\section{Fusilôa}
\begin{itemize}
\item {Grp. gram.:f.}
\end{itemize}
O mesmo que \textunderscore pernilongo\textunderscore , ave.
(Por \textunderscore fuselôa\textunderscore , de \textunderscore fuselo\textunderscore )
\section{Fusionar}
\begin{itemize}
\item {Grp. gram.:v. t.}
\end{itemize}
\begin{itemize}
\item {Utilização:Neol.}
\end{itemize}
\begin{itemize}
\item {Proveniência:(Do lat. \textunderscore fusio\textunderscore )}
\end{itemize}
Fazer a fusão de.
Fundir.
Confundir; amalgamar.
\section{Fusionista}
\begin{itemize}
\item {Grp. gram.:adj.}
\end{itemize}
\begin{itemize}
\item {Grp. gram.:M.}
\end{itemize}
\begin{itemize}
\item {Proveniência:(Do lat. \textunderscore fusio\textunderscore )}
\end{itemize}
Que entrou numa fusão partidária.
Relativo a fusão política.
Partidário de fusão política.
\section{Fusípede}
\begin{itemize}
\item {Grp. gram.:adj.}
\end{itemize}
\begin{itemize}
\item {Proveniência:(Do lat. \textunderscore fusus\textunderscore  + \textunderscore pes\textunderscore )}
\end{itemize}
Que tem os pés afusados.
\section{Fusível}
\begin{itemize}
\item {Grp. gram.:adj.}
\end{itemize}
\begin{itemize}
\item {Proveniência:(Do lat. \textunderscore fusus\textunderscore )}
\end{itemize}
Que se póde fundir.
\section{Fusmila}
\begin{itemize}
\item {Grp. gram.:f.}
\end{itemize}
Espécie de palmeira do Brasil.
\section{Fuso}
\begin{itemize}
\item {Grp. gram.:m.}
\end{itemize}
\begin{itemize}
\item {Proveniência:(Lat. \textunderscore fusus\textunderscore )}
\end{itemize}
Instrumento rolíço e ponteagudo, com que se fia até formar a maçaroca.
Parte de uma superfície esphérica, comprehendida entre dois grandes semi-círculos.
Peça de madeira, sulcada em espiral e que suspende a pedra que obriga a vara do lagar a espremer o bagaço.
Madeiro vertical e girante, a que se liga a mó nos lagares de azeite.
Instrumento para matar carneiros.
Nome de vários objectos que têm fórma de fuso.
Mollusco gasterópode.
O mesmo que \textunderscore fusela\textunderscore .
\section{Fusoide}
\begin{itemize}
\item {Grp. gram.:adj.}
\end{itemize}
\begin{itemize}
\item {Utilização:Hist. Nat.}
\end{itemize}
Que tem fórma de fuso.
\section{Fusório}
\begin{itemize}
\item {Grp. gram.:adj.}
\end{itemize}
\begin{itemize}
\item {Proveniência:(Lat. \textunderscore fusorius\textunderscore )}
\end{itemize}
Relativo á fundição.
\section{Fusta}
\begin{itemize}
\item {Grp. gram.:f.}
\end{itemize}
\begin{itemize}
\item {Utilização:Ant.}
\end{itemize}
\begin{itemize}
\item {Proveniência:(Do ar. \textunderscore futah\textunderscore , tecido com listras?)}
\end{itemize}
Espécie de chaile. Cf. \textunderscore Foral de Cintra\textunderscore .
\section{Fusta}
\begin{itemize}
\item {Grp. gram.:f.}
\end{itemize}
Embarcação longa e chata, de vela e remo.
(B. lat. \textunderscore fusta\textunderscore )
\section{Fusta}
\begin{itemize}
\item {Grp. gram.:f.}
\end{itemize}
\begin{itemize}
\item {Utilização:Ant.}
\end{itemize}
O mesmo que \textunderscore fustan\textunderscore .
\section{Fustã}
\begin{itemize}
\item {Grp. gram.:f.}
\end{itemize}
\begin{itemize}
\item {Utilização:Ant.}
\end{itemize}
\begin{itemize}
\item {Proveniência:(De \textunderscore fuste\textunderscore )}
\end{itemize}
Castigo com varas; fustigação.
\section{Fustal}
\begin{itemize}
\item {Grp. gram.:m.}
\end{itemize}
\begin{itemize}
\item {Utilização:Ant.}
\end{itemize}
O mesmo que \textunderscore fustão\textunderscore ^1.
\section{Fustalha}
\begin{itemize}
\item {Grp. gram.:f.}
\end{itemize}
\begin{itemize}
\item {Proveniência:(De \textunderscore fusta\textunderscore ^2)}
\end{itemize}
Grande porção de fustas.
\section{Fustan}
\begin{itemize}
\item {Grp. gram.:f.}
\end{itemize}
\begin{itemize}
\item {Utilização:Ant.}
\end{itemize}
\begin{itemize}
\item {Proveniência:(De \textunderscore fuste\textunderscore )}
\end{itemize}
Castigo com varas; fustigação.
\section{Fustão}
\begin{itemize}
\item {Grp. gram.:m.}
\end{itemize}
\begin{itemize}
\item {Proveniência:(Do b. lat. \textunderscore fustanum\textunderscore )}
\end{itemize}
Pano de algodão, linho, seda ou lan, tecido em cordão.
\section{Fustão}
\begin{itemize}
\item {Grp. gram.:m.}
\end{itemize}
\begin{itemize}
\item {Utilização:Ant.}
\end{itemize}
O mesmo que \textunderscore fustan\textunderscore .
\section{Fustarrão}
\begin{itemize}
\item {Grp. gram.:m.}
\end{itemize}
\begin{itemize}
\item {Proveniência:(De \textunderscore fusta\textunderscore ^2)}
\end{itemize}
Grande fusta.
\section{Fuste}
\begin{itemize}
\item {Grp. gram.:m.}
\end{itemize}
\begin{itemize}
\item {Utilização:Prov.}
\end{itemize}
\begin{itemize}
\item {Utilização:trasm.}
\end{itemize}
\begin{itemize}
\item {Proveniência:(Lat. \textunderscore fustis\textunderscore )}
\end{itemize}
Haste de madeira.
Peça, com que se esteiam os mastros de navio.
Parte da columna, entre o capitel e a base.
Pequeno pau, que tem numa das extremidades uma camada de betume, com que os ourives pegam nas peças miúdas que hão de lavrar.
O mesmo que \textunderscore graveto\textunderscore .
Mólho, feixe: \textunderscore um fuste de lenha\textunderscore .
O corpo principal do tambor e do bombo.
\section{Fusteína}
\begin{itemize}
\item {Grp. gram.:f.}
\end{itemize}
O mesmo ou melhor que \textunderscore fustina\textunderscore .
\section{Fustel}
\begin{itemize}
\item {Grp. gram.:m.}
\end{itemize}
O mesmo que \textunderscore fustete\textunderscore .
\section{Fustete}
\begin{itemize}
\item {fónica:tê}
\end{itemize}
\begin{itemize}
\item {Grp. gram.:m.}
\end{itemize}
O mesmo que \textunderscore tatajuba\textunderscore .
\section{Fustiga}
\begin{itemize}
\item {Grp. gram.:f.}
\end{itemize}
\begin{itemize}
\item {Utilização:Prov.}
\end{itemize}
\begin{itemize}
\item {Utilização:minh.}
\end{itemize}
O mesmo que \textunderscore fustigada\textunderscore .
\section{Fustigação}
\begin{itemize}
\item {Grp. gram.:f.}
\end{itemize}
Acto ou effeito de fustigar.
\section{Fustigada}
\begin{itemize}
\item {Grp. gram.:f.}
\end{itemize}
\begin{itemize}
\item {Utilização:Prov.}
\end{itemize}
\begin{itemize}
\item {Utilização:minh.}
\end{itemize}
Serviço gratuito, prestado no último dia da sacha pelos que se ajustaram para o trabalho geral da mesma sacha por uma quarta de milho em cada dia.
Acto de fustigar, fustigadela. Cf. Filinto, XX, 253.
\section{Fustigadela}
\begin{itemize}
\item {Grp. gram.:f.}
\end{itemize}
O mesmo que \textunderscore fustigação\textunderscore .
\section{Fustigar}
\begin{itemize}
\item {Grp. gram.:v. t.}
\end{itemize}
\begin{itemize}
\item {Proveniência:(Lat. \textunderscore fustigare\textunderscore )}
\end{itemize}
Bater com vara.
Açoitar.
Castigar; maltratar.
\section{Fustigo}
\begin{itemize}
\item {Grp. gram.:m.}
\end{itemize}
\begin{itemize}
\item {Proveniência:(De \textunderscore fustigar\textunderscore )}
\end{itemize}
Pancada de fuste ou de conto.
\section{Fustina}
\begin{itemize}
\item {Grp. gram.:f.}
\end{itemize}
Substância còrante da tatajuba.
(Cp. \textunderscore fustete\textunderscore )
\section{Fusto}
\begin{itemize}
\item {Grp. gram.:m.}
\end{itemize}
\begin{itemize}
\item {Utilização:Prov.}
\end{itemize}
\begin{itemize}
\item {Utilização:minh.}
\end{itemize}
\begin{itemize}
\item {Utilização:Prov.}
\end{itemize}
\begin{itemize}
\item {Utilização:trasm.}
\end{itemize}
O mesmo que \textunderscore fuste\textunderscore , vara.
O mesmo que \textunderscore mólho\textunderscore , \textunderscore feixe\textunderscore .
\section{Fusuê}
\begin{itemize}
\item {Grp. gram.:m.}
\end{itemize}
\begin{itemize}
\item {Utilização:Bras}
\end{itemize}
Barulho, motim.
\section{Futicar}
\begin{itemize}
\item {Grp. gram.:v. t.}
\end{itemize}
\begin{itemize}
\item {Utilização:Bras. do Rio}
\end{itemize}
Alinhavar ou coser a pontos largos.
\section{Fútil}
\begin{itemize}
\item {Grp. gram.:adj.}
\end{itemize}
\begin{itemize}
\item {Proveniência:(Lat. \textunderscore futilis\textunderscore )}
\end{itemize}
Vão; leviano; frívolo.
Insignificante: \textunderscore conversa fútil\textunderscore .
\section{Futila}
\begin{itemize}
\item {Grp. gram.:f.}
\end{itemize}
Pássaro tenuirostro da África occidental, (\textunderscore nectarinea fuliginosa\textunderscore ).
\section{Futileza}
\begin{itemize}
\item {Grp. gram.:f.}
\end{itemize}
O mesmo que \textunderscore futilidade\textunderscore . Cf. B. Pato, \textunderscore Rui Blas\textunderscore , VIII.
\section{Futilidade}
\begin{itemize}
\item {Grp. gram.:f.}
\end{itemize}
\begin{itemize}
\item {Proveniência:(Lat. \textunderscore futilitas\textunderscore )}
\end{itemize}
Qualidade daquillo que é fútil.
Coisa fútil.
\section{Futilizar}
\begin{itemize}
\item {Grp. gram.:v. i.}
\end{itemize}
\begin{itemize}
\item {Proveniência:(De \textunderscore fútil\textunderscore )}
\end{itemize}
Tratar de futilidades.
Dizer palavras ôcas.
\section{Futilmente}
\begin{itemize}
\item {Grp. gram.:adv.}
\end{itemize}
De modo fútil.
Inutilmente; baldadamente.
\section{Futre}
\begin{itemize}
\item {Grp. gram.:m.}
\end{itemize}
\begin{itemize}
\item {Utilização:Pop.}
\end{itemize}
\begin{itemize}
\item {Proveniência:(Fr. \textunderscore foutre\textunderscore )}
\end{itemize}
Bandalho.
Sovina.
Homem desprezível, mal vestido.
\section{Futrica}
\begin{itemize}
\item {Grp. gram.:f.}
\end{itemize}
\begin{itemize}
\item {Grp. gram.:M.}
\end{itemize}
\begin{itemize}
\item {Utilização:Prov.}
\end{itemize}
Baiuca.
Farraparia.
Porção de coisas velhas.
Designação escolar de quem não é estudante.
Homem ordinário, egoista, de sentimentos baixos.
(Relaciona-se provavelmente com \textunderscore futre\textunderscore )
\section{Futricada}
\begin{itemize}
\item {Grp. gram.:f.}
\end{itemize}
\begin{itemize}
\item {Utilização:Pop.}
\end{itemize}
\begin{itemize}
\item {Proveniência:(De \textunderscore futrica\textunderscore )}
\end{itemize}
Acção de futrica, acção reles.
Trastes usados; cacada.
\section{Futricagem}
\begin{itemize}
\item {Grp. gram.:f.}
\end{itemize}
O mesmo que \textunderscore futricada\textunderscore .
Acção de futrica.
\section{Futricar}
\begin{itemize}
\item {Grp. gram.:v. t.}
\end{itemize}
\begin{itemize}
\item {Grp. gram.:V. i.}
\end{itemize}
\begin{itemize}
\item {Utilização:Prov.}
\end{itemize}
\begin{itemize}
\item {Utilização:minh.}
\end{itemize}
\begin{itemize}
\item {Proveniência:(De \textunderscore futrica\textunderscore )}
\end{itemize}
Mercadejar.
Chatinar; negociar, trapaceando.
Realizar uma coisa aos poucos, aos bocadinhos.
\section{Futriqueiro}
\begin{itemize}
\item {Grp. gram.:m.}
\end{itemize}
\begin{itemize}
\item {Utilização:Prov.}
\end{itemize}
Aquelle que tem futrica ou loja de quinquilharias. Cf. \textunderscore Inquér. Industr.\textunderscore , 2.^a p., l. II, 250.
\section{Futriquice}
\begin{itemize}
\item {Grp. gram.:f.}
\end{itemize}
\begin{itemize}
\item {Utilização:Fam.}
\end{itemize}
Acção reles ou vil, própria de futriqueiro.
\section{Futura}
\begin{itemize}
\item {Grp. gram.:f.}
\end{itemize}
\begin{itemize}
\item {Utilização:Fam.}
\end{itemize}
\begin{itemize}
\item {Proveniência:(De \textunderscore futuro\textunderscore )}
\end{itemize}
O mesmo que \textunderscore noiva\textunderscore .
\section{Futuração}
\begin{itemize}
\item {Grp. gram.:f.}
\end{itemize}
Acto de futurar.
\section{Futurar}
\begin{itemize}
\item {Grp. gram.:v. t.}
\end{itemize}
\begin{itemize}
\item {Grp. gram.:V. i.}
\end{itemize}
\begin{itemize}
\item {Proveniência:(De \textunderscore futuro\textunderscore )}
\end{itemize}
Predizer.
Calcular.
Suppor.
Fazer vaticínio, mostrar bom agoiro. Cf. Camillo, \textunderscore Volcões\textunderscore , 90.
\section{Futurição}
\begin{itemize}
\item {Grp. gram.:f.}
\end{itemize}
\begin{itemize}
\item {Utilização:Des.}
\end{itemize}
\begin{itemize}
\item {Proveniência:(Do rad. de \textunderscore futuro\textunderscore )}
\end{itemize}
Coisa ou vida futura.
\section{Futuridade}
\begin{itemize}
\item {Grp. gram.:f.}
\end{itemize}
\begin{itemize}
\item {Proveniência:(De \textunderscore futuro\textunderscore )}
\end{itemize}
Qualidade de uma coisa que está por vir.
\section{Futuro}
\begin{itemize}
\item {Grp. gram.:m.}
\end{itemize}
\begin{itemize}
\item {Utilização:Gram.}
\end{itemize}
\begin{itemize}
\item {Grp. gram.:Adj.}
\end{itemize}
\begin{itemize}
\item {Proveniência:(Lat. \textunderscore futurus\textunderscore )}
\end{itemize}
Tempo que há de vir.
Destino: \textunderscore o futuro a Deus pertence\textunderscore .
Tempo dos verbos, relativo a acção que há de sêr.
Que há de sêr.
\section{Futuroso}
\begin{itemize}
\item {Grp. gram.:adj.}
\end{itemize}
\begin{itemize}
\item {Utilização:bras}
\end{itemize}
\begin{itemize}
\item {Utilização:Neol.}
\end{itemize}
Que tem bom futuro.
Prometedor, auspicioso: \textunderscore a nossa nascente e futurosa literatura...\textunderscore 
\section{Fuxicar}
\begin{itemize}
\item {Grp. gram.:v. t.}
\end{itemize}
\begin{itemize}
\item {Utilização:Prov.}
\end{itemize}
\begin{itemize}
\item {Utilização:minh.}
\end{itemize}
O mesmo que \textunderscore futicar\textunderscore .
O mesmo que \textunderscore mexer\textunderscore .
Amarrotar.--Nesta última accepção, registada por Boscoli, \textunderscore Gram.\textunderscore , há talvez a alter. de \textunderscore fossicar\textunderscore , de \textunderscore fossar\textunderscore .
\section{Fuxico}
\begin{itemize}
\item {Grp. gram.:m.}
\end{itemize}
\begin{itemize}
\item {Utilização:Bras}
\end{itemize}
Mexerico; intriga.
(Cp. \textunderscore fuxicar\textunderscore )
\section{Fuzil}
\begin{itemize}
\item {Grp. gram.:m.}
\end{itemize}
\begin{itemize}
\item {Utilização:Fig.}
\end{itemize}
\begin{itemize}
\item {Grp. gram.:Pl.}
\end{itemize}
\begin{itemize}
\item {Utilização:Zool.}
\end{itemize}
\begin{itemize}
\item {Proveniência:(It. \textunderscore focile\textunderscore )}
\end{itemize}
Peça de aço, com que se faz lume.
Relâmpago.
Elo de metal.
Anel de cadeia.
Cadeia, ligação.
Aro de ferro, que prende á testeira a serra grande dos serradores.
Pennas, que nascem no ângulo externo do coto das asas. Cf. Fern. Ferreira, \textunderscore Caça de Altan.\textunderscore 
\section{Fuzilação}
\begin{itemize}
\item {Grp. gram.:f.}
\end{itemize}
Acto de fuzilar.
Clarão, produzido pelo fuzil, ao ferir a pederneira.
\section{Fuzilada}
\begin{itemize}
\item {Grp. gram.:f.}
\end{itemize}
\begin{itemize}
\item {Proveniência:(De \textunderscore fuzil\textunderscore )}
\end{itemize}
Tiros de espingarda.
Pancada de fuzil em pederneira.
Relâmpagos longínquos.
\section{Fuzilado}
\begin{itemize}
\item {Grp. gram.:adj.}
\end{itemize}
\begin{itemize}
\item {Proveniência:(De \textunderscore fuzilar\textunderscore )}
\end{itemize}
Justiçado ou assassinado com arma de fogo.
\section{Fuzilador}
\begin{itemize}
\item {Grp. gram.:m.  e  adj.}
\end{itemize}
\begin{itemize}
\item {Proveniência:(De \textunderscore fuzilar\textunderscore )}
\end{itemize}
O que fuzila ou manda fuzilar ou espingardear.
\section{Fuzilamento}
\begin{itemize}
\item {Grp. gram.:m.}
\end{itemize}
Acto ou effeito de fuzilar.
\section{Fuzilante}
\begin{itemize}
\item {Grp. gram.:adj.}
\end{itemize}
Que fuzila, que expede clarões ou centelhas.
\section{Fuzilão}
\begin{itemize}
\item {Grp. gram.:m.}
\end{itemize}
O mesmo ou melhor que \textunderscore fuzilhão\textunderscore .
\section{Fuzilar}
\begin{itemize}
\item {Grp. gram.:v. t.}
\end{itemize}
\begin{itemize}
\item {Grp. gram.:V. i.}
\end{itemize}
\begin{itemize}
\item {Utilização:Fig.}
\end{itemize}
\begin{itemize}
\item {Proveniência:(Lat. \textunderscore focillare\textunderscore )}
\end{itemize}
Expedir de si, á maneira de raios ou scintillações.
Matar com arma de fogo.
Relampejar.
Brilhar muito.
Tornar-se ameaçador: \textunderscore os seus olhos fuzilavam\textunderscore .
\section{Fuzilaria}
\begin{itemize}
\item {Grp. gram.:f.}
\end{itemize}
\begin{itemize}
\item {Utilização:Fig.}
\end{itemize}
\begin{itemize}
\item {Proveniência:(De \textunderscore fuzil\textunderscore )}
\end{itemize}
Tiros simultâneos de espingardas.
Tiroteio entre inimigos.
Grande abundância.
\section{Fuzileiro}
\begin{itemize}
\item {Grp. gram.:m.}
\end{itemize}
\begin{itemize}
\item {Proveniência:(De \textunderscore fuzil\textunderscore )}
\end{itemize}
Soldado, armado de espingarda.
\section{Fuzilhão}
\begin{itemize}
\item {Grp. gram.:m.}
\end{itemize}
\begin{itemize}
\item {Proveniência:(De \textunderscore fuzil\textunderscore )}
\end{itemize}
\end{document}