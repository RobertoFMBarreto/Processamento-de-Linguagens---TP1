\documentclass{article}
\usepackage[portuguese]{babel}
\title{D}
\begin{document}
\textunderscore m.\textunderscore (e der.)
(V. \textunderscore tsar\textunderscore , etc.)
\section{Dacite}
\begin{itemize}
\item {Grp. gram.:f.}
\end{itemize}
Espécie de andesite quartzífera.
\section{Dafina}
\begin{itemize}
\item {Grp. gram.:f.}
\end{itemize}
O mesmo que \textunderscore defina\textunderscore .
\section{Damas-entre-verde}
\begin{itemize}
\item {Grp. gram.:f. pl.}
\end{itemize}
\begin{itemize}
\item {Utilização:Bot.}
\end{itemize}
O mesmo que \textunderscore barba-de-velho\textunderscore . Cf. P. Coutinho, \textunderscore Flora\textunderscore , 237.
\section{Dão-dão}
\begin{itemize}
\item {Grp. gram.:m.}
\end{itemize}
O mesmo que \textunderscore dandão\textunderscore , pesadelo nocturno. Cf. B. Pereira, \textunderscore Prosodia\textunderscore , vb. \textunderscore dusius\textunderscore .
\section{Dealmática}
\begin{itemize}
\item {Grp. gram.:f.}
\end{itemize}
O mesmo que \textunderscore dalmática\textunderscore . Cf. B. Pereira, \textunderscore Prosodia\textunderscore , vb. \textunderscore levitonarium\textunderscore .
\section{Decathlo}
\begin{itemize}
\item {Grp. gram.:m.}
\end{itemize}
\begin{itemize}
\item {Proveniência:(Do gr. \textunderscore deka\textunderscore  + \textunderscore athlos\textunderscore )}
\end{itemize}
Conjunto de déz exercícios, nos jogos olýmpicos.
\section{Decatlo}
\begin{itemize}
\item {Grp. gram.:m.}
\end{itemize}
\begin{itemize}
\item {Proveniência:(Do gr. \textunderscore deka\textunderscore  + \textunderscore athlos\textunderscore )}
\end{itemize}
Conjunto de déz exercícios, nos jogos olímpicos.
\section{Decídua}
\begin{itemize}
\item {Grp. gram.:f.}
\end{itemize}
Páreas ou secundinas. Cf. P. Moraes, \textunderscore Zool. Elem.\textunderscore , 159.
\section{Deciduados}
\begin{itemize}
\item {Grp. gram.:m. pl.}
\end{itemize}
Dizem-se os mammíferos, que têm decídua. Cf. P. Moraes, \textunderscore Zool. Elem.\textunderscore , 159.
\section{Decilitreiro}
\begin{itemize}
\item {Grp. gram.:m.}
\end{itemize}
\begin{itemize}
\item {Utilização:Burl.}
\end{itemize}
\begin{itemize}
\item {Proveniência:(De \textunderscore decilitro\textunderscore )}
\end{itemize}
Beberrão de tabernas. Cf. Fialho, \textunderscore Gatos\textunderscore , II, 9.
\section{Decontar}
\begin{itemize}
\item {Grp. gram.:v. t.}
\end{itemize}
\begin{itemize}
\item {Utilização:Ant.}
\end{itemize}
\begin{itemize}
\item {Proveniência:(De \textunderscore contar\textunderscore )}
\end{itemize}
Contar, narrar, referir.
\section{Decoral}
\begin{itemize}
\item {Grp. gram.:adj.}
\end{itemize}
O mesmo que \textunderscore decorativo\textunderscore .
\section{Degressivo}
\begin{itemize}
\item {Grp. gram.:adj.}
\end{itemize}
\begin{itemize}
\item {Proveniência:(Do lat. \textunderscore degressus\textunderscore )}
\end{itemize}
Que vai deminuindo sucessivamente: \textunderscore imposto degressivo\textunderscore .
\section{Deradelfo}
\begin{itemize}
\item {Grp. gram.:m.}
\end{itemize}
\begin{itemize}
\item {Utilização:Terat.}
\end{itemize}
\begin{itemize}
\item {Proveniência:(Do gr. \textunderscore dere\textunderscore  + \textunderscore adelphos\textunderscore )}
\end{itemize}
Nome, que se dá a monstros duplos monocéfalos, com os troncos separados do umbigo para baixo, três ou quatro membros torácicos e uma só cabeça.
\section{Deradelpho}
\begin{itemize}
\item {Grp. gram.:m.}
\end{itemize}
\begin{itemize}
\item {Utilização:Terat.}
\end{itemize}
\begin{itemize}
\item {Proveniência:(Do gr. \textunderscore dere\textunderscore  + \textunderscore adelphos\textunderscore )}
\end{itemize}
Nome, que se dá a monstros duplos monocéphalos, com os troncos separados do umbigo para baixo, três ou quatro membros thorácicos e uma só cabeça.
\section{Dermalgia}
\begin{itemize}
\item {Grp. gram.:f.}
\end{itemize}
\begin{itemize}
\item {Utilização:Med.}
\end{itemize}
O mesmo que \textunderscore dermatalgia\textunderscore .
\section{Dermatalgia}
\begin{itemize}
\item {Grp. gram.:f.}
\end{itemize}
\begin{itemize}
\item {Utilização:Med.}
\end{itemize}
\begin{itemize}
\item {Proveniência:(Do gr. \textunderscore derma\textunderscore , \textunderscore dermatos\textunderscore  + \textunderscore algos\textunderscore )}
\end{itemize}
Dôr espontânea, que se sente na pelle, independentemente de qualquer lesão apreciável do tegumento ou do systema nervoso.
\section{Dermatoma}
\begin{itemize}
\item {Grp. gram.:m.}
\end{itemize}
\begin{itemize}
\item {Utilização:Med.}
\end{itemize}
\begin{itemize}
\item {Proveniência:(Do gr. \textunderscore derma\textunderscore , \textunderscore dermatos\textunderscore )}
\end{itemize}
Neoplasma cutâneo.
\section{Dermatoterapia}
\begin{itemize}
\item {Grp. gram.:f.}
\end{itemize}
\begin{itemize}
\item {Utilização:Med.}
\end{itemize}
\begin{itemize}
\item {Proveniência:(Do gr. \textunderscore derma\textunderscore , \textunderscore dermatos\textunderscore  + \textunderscore therapeia\textunderscore )}
\end{itemize}
Tratamento das doenças da pele.
\section{Dermatotherapia}
\begin{itemize}
\item {Grp. gram.:f.}
\end{itemize}
\begin{itemize}
\item {Utilização:Med.}
\end{itemize}
\begin{itemize}
\item {Proveniência:(Do gr. \textunderscore derma\textunderscore , \textunderscore dermatos\textunderscore  + \textunderscore therapeia\textunderscore )}
\end{itemize}
Tratamento das doenças da pelle.
\section{Dermatorragia}
\begin{itemize}
\item {Grp. gram.:f.}
\end{itemize}
\begin{itemize}
\item {Utilização:Med.}
\end{itemize}
\begin{itemize}
\item {Proveniência:(Do gr. \textunderscore derma\textunderscore , \textunderscore dermatos\textunderscore  + \textunderscore rhagein\textunderscore )}
\end{itemize}
Hemorragia da pele.
\section{Dermatorrhagia}
\begin{itemize}
\item {Grp. gram.:f.}
\end{itemize}
\begin{itemize}
\item {Utilização:Med.}
\end{itemize}
\begin{itemize}
\item {Proveniência:(Do gr. \textunderscore derma\textunderscore , \textunderscore dermatos\textunderscore  + \textunderscore rhagein\textunderscore )}
\end{itemize}
Hemorrhagia da pelle.
\section{Dermóide}
\begin{itemize}
\item {Grp. gram.:adj.}
\end{itemize}
\begin{itemize}
\item {Utilização:Med.}
\end{itemize}
\begin{itemize}
\item {Proveniência:(Do gr. \textunderscore derma\textunderscore  + \textunderscore eidos\textunderscore )}
\end{itemize}
Que tem a estructura da pelle.
\section{Derodídimo}
\begin{itemize}
\item {Grp. gram.:m.}
\end{itemize}
\begin{itemize}
\item {Utilização:Terat.}
\end{itemize}
\begin{itemize}
\item {Proveniência:(Do gr. \textunderscore dere\textunderscore  + \textunderscore didumos\textunderscore )}
\end{itemize}
Monstro, com duas colunas vertebraes.
\section{Derodídymo}
\begin{itemize}
\item {Grp. gram.:m.}
\end{itemize}
\begin{itemize}
\item {Utilização:Terat.}
\end{itemize}
\begin{itemize}
\item {Proveniência:(Do gr. \textunderscore dere\textunderscore  + \textunderscore didumos\textunderscore )}
\end{itemize}
Monstro, com duas columnas vertebraes.
\section{Deródimo}
\begin{itemize}
\item {Grp. gram.:m.}
\end{itemize}
O mesmo que \textunderscore derodídimo\textunderscore , (por haplologia).
\section{Deródymo}
\begin{itemize}
\item {Grp. gram.:m.}
\end{itemize}
O mesmo que \textunderscore derodídymo\textunderscore , (por haplologia).
\section{Desajoujo}
\begin{itemize}
\item {Grp. gram.:m.}
\end{itemize}
Acto de desajoujar. Cf. B. Pereira, \textunderscore Prosodia\textunderscore , vb. \textunderscore dejugatio\textunderscore .
\section{Desassiso}
\begin{itemize}
\item {Grp. gram.:m.}
\end{itemize}
\begin{itemize}
\item {Proveniência:(De \textunderscore desassisar\textunderscore )}
\end{itemize}
Falta de siso; disparate. Cf. R. Jorge, \textunderscore El Greco\textunderscore , 26.
\section{Descanção}
\begin{itemize}
\item {Grp. gram.:m.}
\end{itemize}
\begin{itemize}
\item {Utilização:Ant.}
\end{itemize}
O mesmo que \textunderscore escanção\textunderscore . Cf. B. Pereira, \textunderscore Prosodia\textunderscore , vb. \textunderscore oenopta\textunderscore .
\section{Desbrochar}
\begin{itemize}
\item {Grp. gram.:v. i.}
\end{itemize}
O mesmo que \textunderscore desabrochar\textunderscore . Cf. B. Pereira, \textunderscore Prosodia\textunderscore , vb. \textunderscore expapillo\textunderscore .
\section{Descapsular}
\begin{itemize}
\item {Grp. gram.:v. t.}
\end{itemize}
Tirar a cápsula a.
\section{Descolamento}
\begin{itemize}
\item {Grp. gram.:m.}
\end{itemize}
Acto ou efeito de descolar: \textunderscore o descolamento da retina\textunderscore .
\section{Descollamento}
\begin{itemize}
\item {Grp. gram.:m.}
\end{itemize}
Acto ou effeito de descollar: \textunderscore o descollamento da retina\textunderscore .
\section{Descontentativo}
\begin{itemize}
\item {Grp. gram.:adj.}
\end{itemize}
\begin{itemize}
\item {Proveniência:(De \textunderscore descontentar\textunderscore )}
\end{itemize}
Próprio para produzir descontentamento:«\textunderscore degredo..., basta-lhe o nome para sêr descontentativo\textunderscore ». Arráiz, \textunderscore Diálogos\textunderscore , I, c. 3.
\section{Desenxabidez}
\begin{itemize}
\item {Grp. gram.:f.}
\end{itemize}
Estado ou qualidade de desenxabido. (Us. por Camillo)
\section{Desenxabimento}
\begin{itemize}
\item {Grp. gram.:adj.}
\end{itemize}
Desenxabidez; insipidez.
\section{Desértico}
\begin{itemize}
\item {Grp. gram.:adj.}
\end{itemize}
\begin{itemize}
\item {Utilização:Neol.}
\end{itemize}
\begin{itemize}
\item {Proveniência:(De \textunderscore deserto\textunderscore )}
\end{itemize}
Diz-se dos terrenos, que, embora não sejam perfeitos desertos, com êstes se parecem pelo aspecto do solo e pelo clima, como succede em grande parte da Sibéria.
\section{Desleitagem}
\begin{itemize}
\item {Grp. gram.:f.}
\end{itemize}
Acto de desleitar.
Operação, quási sempre manual, com que se tira ou se separa o leite da manteiga.
\section{Desmoídeo}
\begin{itemize}
\item {Grp. gram.:adj.}
\end{itemize}
\begin{itemize}
\item {Utilização:Anat.}
\end{itemize}
\begin{itemize}
\item {Proveniência:(Do gr. \textunderscore desmos\textunderscore  + \textunderscore eidos\textunderscore )}
\end{itemize}
Diz-se dos tecidos fibrosos do corpo animal, (ligamentos, tendões e membranas).
\section{Desnevado}
\begin{itemize}
\item {Grp. gram.:adj.}
\end{itemize}
\begin{itemize}
\item {Proveniência:(De \textunderscore des...\textunderscore , pref. de realce, e \textunderscore neve\textunderscore . Cp. \textunderscore desinquieto\textunderscore )}
\end{itemize}
Muito frio:«\textunderscore água fria e desnevada, na maior fôrça do sol do estio...\textunderscore »Sousa, \textunderscore Hist. de S. Dom.\textunderscore , II, 56.
\section{Desratização}
\begin{itemize}
\item {Grp. gram.:f.}
\end{itemize}
Acto ou effeito de desratizar.
\section{Desratizar}
\begin{itemize}
\item {Grp. gram.:v. t.}
\end{itemize}
\begin{itemize}
\item {Utilização:Neol.}
\end{itemize}
\begin{itemize}
\item {Proveniência:(De \textunderscore des...\textunderscore  + \textunderscore rato\textunderscore )}
\end{itemize}
Livrar dos ratos (uma região, uma casa, etc.).
\section{Desterroador}
\begin{itemize}
\item {Grp. gram.:m.}
\end{itemize}
Máquina agricola, para desterroar mecanicamente.
\section{Destilamento}
\begin{itemize}
\item {Grp. gram.:m.}
\end{itemize}
\begin{itemize}
\item {Utilização:Des.}
\end{itemize}
\begin{itemize}
\item {Proveniência:(De \textunderscore destilar\textunderscore )}
\end{itemize}
Gota; pingo.
\section{Destillamento}
\begin{itemize}
\item {Grp. gram.:m.}
\end{itemize}
\begin{itemize}
\item {Utilização:Des.}
\end{itemize}
\begin{itemize}
\item {Proveniência:(De \textunderscore destillar\textunderscore )}
\end{itemize}
Gota; pingo.
\section{Destroxo}
\begin{itemize}
\item {fónica:trô}
\end{itemize}
\begin{itemize}
\item {Grp. gram.:m.}
\end{itemize}
\begin{itemize}
\item {Utilização:T. de Odemira}
\end{itemize}
Disparate, asneira, parvoíce.
\section{Desvacação}
\begin{itemize}
\item {Grp. gram.:f.}
\end{itemize}
Acto ou effeito de desvacar.
\section{Desvacar}
\begin{itemize}
\item {Grp. gram.:v. t.}
\end{itemize}
\begin{itemize}
\item {Utilização:T. de Mossâme}
\end{itemize}
\begin{itemize}
\item {Utilização:des.}
\end{itemize}
\begin{itemize}
\item {Proveniência:(De \textunderscore des...\textunderscore  + \textunderscore vaca\textunderscore )}
\end{itemize}
Receber (do gentio) vacas, em pagamento de multas e de outras imposições.
\section{Detrítico}
\begin{itemize}
\item {Grp. gram.:adj.}
\end{itemize}
Relativo a detritos.
\section{Dextrocardia}
\begin{itemize}
\item {Grp. gram.:f.}
\end{itemize}
\begin{itemize}
\item {Utilização:Med.}
\end{itemize}
\begin{itemize}
\item {Proveniência:(Do lat. \textunderscore dexter\textunderscore  + gr. \textunderscore kardia\textunderscore )}
\end{itemize}
Deslocamento do coração para o lado direito, em consequência de um tumor intra-thorácico ou de outro incidente mórbido.
\section{Déz-reis}
\begin{itemize}
\item {Grp. gram.:m. pl.}
\end{itemize}
Moéda portuguesa, correspondente ao moderno centavo.
\section{Diabão}
\begin{itemize}
\item {Grp. gram.:m.}
\end{itemize}
\begin{itemize}
\item {Utilização:Pop.}
\end{itemize}
Grande diabo.
Pessôa ruim; um diabo alma.
\section{Diáboo}
\begin{itemize}
\item {Grp. gram.:m.}
\end{itemize}
\begin{itemize}
\item {Utilização:Ant.}
\end{itemize}
O mesmo que \textunderscore diabo\textunderscore .
\section{Diacatholicão}
\begin{itemize}
\item {Grp. gram.:m.}
\end{itemize}
Medicamento, o mesmo que \textunderscore catholicão\textunderscore .
\section{Diacatolicão}
\begin{itemize}
\item {Grp. gram.:m.}
\end{itemize}
Medicamento, o mesmo que \textunderscore catolicão\textunderscore .
\section{Diacúrcuma}
\begin{itemize}
\item {Grp. gram.:m.}
\end{itemize}
Unguento de cúrcuma. Cf. C. Guerreiro, \textunderscore Diccion. de Cous.\textunderscore 
\section{Dialaca}
\begin{itemize}
\item {Grp. gram.:f.}
\end{itemize}
Unguento, cuja base é a laca.
\section{Diálage}
\begin{itemize}
\item {Grp. gram.:f.}
\end{itemize}
\begin{itemize}
\item {Proveniência:(Do gr. \textunderscore diallage\textunderscore )}
\end{itemize}
Espécie de piroxênio, (silicato de magnésia, cálcio, ferro e alumínio).
\section{Dialágio}
\begin{itemize}
\item {Grp. gram.:m.}
\end{itemize}
\begin{itemize}
\item {Utilização:Miner.}
\end{itemize}
\begin{itemize}
\item {Proveniência:(Do gr. \textunderscore diallage\textunderscore )}
\end{itemize}
Espécie de piroxênio, (silicato de magnésia, cálcio, ferro e alumínio).
\section{Diállage}
\begin{itemize}
\item {Grp. gram.:f.}
\end{itemize}
\begin{itemize}
\item {Proveniência:(Do gr. \textunderscore diallage\textunderscore )}
\end{itemize}
Espécie de pyroxênio, (silicato de magnésia, cálcio, ferro e alumínio).
\section{Diallágio}
\begin{itemize}
\item {Grp. gram.:m.}
\end{itemize}
\begin{itemize}
\item {Utilização:Miner.}
\end{itemize}
\begin{itemize}
\item {Proveniência:(Do gr. \textunderscore diallage\textunderscore )}
\end{itemize}
Espécie de pyroxênio, (silicato de magnésia, cálcio, ferro e alumínio).
\section{Diamorão}
\begin{itemize}
\item {Grp. gram.:m.}
\end{itemize}
O mesmo que \textunderscore diamoro\textunderscore .
\section{Diamorusia}
\begin{itemize}
\item {Grp. gram.:f.}
\end{itemize}
O mesmo que \textunderscore diamoro\textunderscore . Cf. C. Guerreiro, \textunderscore Diccion. de Cous.\textunderscore 
\section{Diamusco}
\begin{itemize}
\item {Grp. gram.:m.}
\end{itemize}
Preparado pharmacêutico, cuja base é o almíscar.
\section{Diapruno}
\begin{itemize}
\item {Grp. gram.:m.}
\end{itemize}
Electuário purgativo de abrunhos. Cf. P. Rego, \textunderscore Instr. de Cavall. de Brida\textunderscore , 209.
\section{Diarremia}
\begin{itemize}
\item {Grp. gram.:f.}
\end{itemize}
Estado patológico, xaroposo, do sangue, em certas doenças como a hematuria. Cf. Macedo Pinto. \textunderscore Comp. de Veter.\textunderscore , I, 77.
\section{Diarrhemia}
\begin{itemize}
\item {Grp. gram.:f.}
\end{itemize}
Estado pathológico, xaroposo, do sangue, em certas doenças como a hematuria. Cf. Macedo Pinto. \textunderscore Comp. de Veter.\textunderscore , I, 77.
\section{Diarrhodão}
\begin{itemize}
\item {Grp. gram.:m.}
\end{itemize}
Electuário de rosas. Cf. P. Rego, \textunderscore Instr. de Cavall. de Brida\textunderscore , 210.
\section{Diarrodão}
\begin{itemize}
\item {Grp. gram.:m.}
\end{itemize}
Electuário de rosas. Cf. P. Rego, \textunderscore Instr. de Cavall. de Brida\textunderscore , 210.
\section{Diasene}
\begin{itemize}
\item {fónica:sê}
\end{itemize}
\begin{itemize}
\item {Grp. gram.:m.}
\end{itemize}
Pós purgativos de sene.
\section{Diassene}
\begin{itemize}
\item {Grp. gram.:m.}
\end{itemize}
Pós purgativos de sene.
\section{Diasaturno}
\begin{itemize}
\item {fónica:sa}
\end{itemize}
\begin{itemize}
\item {Grp. gram.:m.}
\end{itemize}
Electuário antigo, em que entrava o chumbo.
\section{Diassaturno}
\begin{itemize}
\item {Grp. gram.:m.}
\end{itemize}
Electuário antigo, em que entrava o chumbo.
\section{Diastasemia}
\begin{itemize}
\item {Grp. gram.:f.}
\end{itemize}
\begin{itemize}
\item {Proveniência:(Do gr. \textunderscore diastasis\textunderscore  + \textunderscore haima\textunderscore )}
\end{itemize}
Estado pathológico do sangue, cuja fibrina e cuja albumina se separam da hematosina.
\section{Difiodonte}
\begin{itemize}
\item {Grp. gram.:adj.}
\end{itemize}
\begin{itemize}
\item {Utilização:Zool.}
\end{itemize}
Diz-se dos animaes, que tem duas dentições.
\section{Dinanciano}
\begin{itemize}
\item {Grp. gram.:adj.}
\end{itemize}
\begin{itemize}
\item {Utilização:Geol.}
\end{itemize}
\begin{itemize}
\item {Proveniência:(De \textunderscore Dinant\textunderscore , n. p.)}
\end{itemize}
Diz-se dos terrenos de uma camada do carbonífero inferior.
\section{Diodoncéfalo}
\begin{itemize}
\item {Grp. gram.:m.}
\end{itemize}
\begin{itemize}
\item {Utilização:Terat.}
\end{itemize}
\begin{itemize}
\item {Proveniência:(Do gr. \textunderscore dis\textunderscore  + \textunderscore odous\textunderscore  + \textunderscore kephale\textunderscore )}
\end{itemize}
Monstro, que tem na cabeça uma fila dupla de dentes.
\section{Diodoncéphalo}
\begin{itemize}
\item {Grp. gram.:m.}
\end{itemize}
\begin{itemize}
\item {Utilização:Terat.}
\end{itemize}
\begin{itemize}
\item {Proveniência:(Do gr. \textunderscore dis\textunderscore  + \textunderscore odous\textunderscore  + \textunderscore kephale\textunderscore )}
\end{itemize}
Monstro, que tem na cabeça uma fila dupla de dentes.
\section{Diphyodonte}
\begin{itemize}
\item {Grp. gram.:adj.}
\end{itemize}
\begin{itemize}
\item {Utilização:Zool.}
\end{itemize}
Diz-se dos animaes, que tem duas dentições.
\section{Diospireiro}
\begin{itemize}
\item {Grp. gram.:m.}
\end{itemize}
Gênero de arbustos ebenáceos. Cf. B. Pereira, \textunderscore Prosodia\textunderscore , vb. \textunderscore diospyros\textunderscore .
\section{Diospyreiro}
\begin{itemize}
\item {Grp. gram.:m.}
\end{itemize}
Gênero de arbustos ebenáceos. Cf. B. Pereira, \textunderscore Prosodia\textunderscore , vb. \textunderscore diospyros\textunderscore .
\section{Diplocefalia}
\begin{itemize}
\item {Grp. gram.:f.}
\end{itemize}
Estado ou qualidade de diplocéfalo.
\section{Diplocéfalo}
\begin{itemize}
\item {Grp. gram.:m.}
\end{itemize}
\begin{itemize}
\item {Utilização:Terat.}
\end{itemize}
\begin{itemize}
\item {Proveniência:(Do gr. \textunderscore diploos\textunderscore  + \textunderscore kephale\textunderscore )}
\end{itemize}
Monstro, com duas cabeças num só corpo.
\section{Diplocephalia}
\begin{itemize}
\item {Grp. gram.:f.}
\end{itemize}
Estado ou qualidade de diplocéphalo.
\section{Diplocéphalo}
\begin{itemize}
\item {Grp. gram.:m.}
\end{itemize}
\begin{itemize}
\item {Utilização:Terat.}
\end{itemize}
\begin{itemize}
\item {Proveniência:(Do gr. \textunderscore diploos\textunderscore  + \textunderscore kephale\textunderscore )}
\end{itemize}
Monstro, com duas cabeças num só corpo.
\section{Diplococco}
\begin{itemize}
\item {Grp. gram.:m.}
\end{itemize}
\begin{itemize}
\item {Utilização:Med.}
\end{itemize}
\begin{itemize}
\item {Proveniência:(Do gr. \textunderscore diploos\textunderscore  + \textunderscore kokkos\textunderscore )}
\end{itemize}
Micrococcos, associados dois a dois.
\section{Diplococo}
\begin{itemize}
\item {Grp. gram.:m.}
\end{itemize}
\begin{itemize}
\item {Utilização:Med.}
\end{itemize}
\begin{itemize}
\item {Proveniência:(Do gr. \textunderscore diploos\textunderscore  + \textunderscore kokkos\textunderscore )}
\end{itemize}
Micrococos, associados dois a dois.
\section{Diplofonia}
\begin{itemize}
\item {Grp. gram.:f.}
\end{itemize}
\begin{itemize}
\item {Utilização:Med.}
\end{itemize}
\begin{itemize}
\item {Proveniência:(Do gr. \textunderscore diploos\textunderscore  + \textunderscore phone\textunderscore )}
\end{itemize}
Perturbação da voz, caracterizada pela formação simultânea de dois sons na laringe.
\section{Diplogênese}
\begin{itemize}
\item {Grp. gram.:f.}
\end{itemize}
\begin{itemize}
\item {Utilização:Med.}
\end{itemize}
\begin{itemize}
\item {Proveniência:(Do gr. \textunderscore diploos\textunderscore  + \textunderscore genesis\textunderscore )}
\end{itemize}
Fusão de dois fetos em estado diverso de desenvolvimento.
\section{Diplophonia}
\begin{itemize}
\item {Grp. gram.:f.}
\end{itemize}
\begin{itemize}
\item {Utilização:Med.}
\end{itemize}
\begin{itemize}
\item {Proveniência:(Do gr. \textunderscore diploos\textunderscore  + \textunderscore phone\textunderscore )}
\end{itemize}
Perturbação da voz, caracterizada pela formação simultânea de dois sons na larynge.
\section{Dispoer}
\begin{itemize}
\item {Grp. gram.:v. t.}
\end{itemize}
\begin{itemize}
\item {Utilização:Ant.}
\end{itemize}
O mesmo que \textunderscore dispor\textunderscore .
\section{Dispoimento}
\begin{itemize}
\item {Grp. gram.:m.}
\end{itemize}
\begin{itemize}
\item {Utilização:Ant.}
\end{itemize}
\begin{itemize}
\item {Proveniência:(De \textunderscore dispoer\textunderscore )}
\end{itemize}
O mesmo que \textunderscore disposição\textunderscore .
\section{Disputatório}
\begin{itemize}
\item {Grp. gram.:adj.}
\end{itemize}
Relativo a disputa:«\textunderscore campanhas disputatórias\textunderscore ». Cf. Macedo, \textunderscore Motim\textunderscore , III, 149.
\section{Dizedela}
\begin{itemize}
\item {Grp. gram.:f.}
\end{itemize}
\begin{itemize}
\item {Utilização:Fam.}
\end{itemize}
Acto de dizer; dito.
Aphorismo; prolóquio. Cf. R. Jorge, \textunderscore Epid. de Lisb.\textunderscore 
\section{Dobrum}
\begin{itemize}
\item {Grp. gram.:m.}
\end{itemize}
\begin{itemize}
\item {Utilização:Ant.}
\end{itemize}
O mesmo que \textunderscore debrum\textunderscore .
\section{Dínio}
\begin{itemize}
\item {Grp. gram.:m.}
\end{itemize}
\begin{itemize}
\item {Proveniência:(Do gr. \textunderscore dunamis\textunderscore )}
\end{itemize}
Unidade de medidas de fôrças eléctricas, igual a 1^{gr.}, 981.
\section{Discromia}
\begin{itemize}
\item {Grp. gram.:f.}
\end{itemize}
\begin{itemize}
\item {Proveniência:(Do gr. \textunderscore dus\textunderscore  + \textunderscore khroma\textunderscore )}
\end{itemize}
Designação genérica de todas as perturbações da pigmentação da pelle.
\section{Disdipsia}
\begin{itemize}
\item {Grp. gram.:f.}
\end{itemize}
\begin{itemize}
\item {Utilização:Med.}
\end{itemize}
\begin{itemize}
\item {Proveniência:(Do gr. \textunderscore dus\textunderscore  + \textunderscore dipsa\textunderscore )}
\end{itemize}
Dificuldade na deglutição dos líquidos.
\section{Disenteriforme}
\begin{itemize}
\item {Grp. gram.:adj.}
\end{itemize}
\begin{itemize}
\item {Utilização:Med.}
\end{itemize}
\begin{itemize}
\item {Proveniência:(De \textunderscore disenteria\textunderscore  + \textunderscore fórma\textunderscore )}
\end{itemize}
Diz-se das enterites, em que as evacuações se parecem com as da disenteria.
\section{Disgenético}
\begin{itemize}
\item {Grp. gram.:adj.}
\end{itemize}
O mesmo que \textunderscore disgenésico\textunderscore .
\section{Dislogia}
\begin{itemize}
\item {Grp. gram.:f.}
\end{itemize}
\begin{itemize}
\item {Utilização:Med.}
\end{itemize}
\begin{itemize}
\item {Proveniência:(Do gr. \textunderscore dus\textunderscore  + \textunderscore logos\textunderscore )}
\end{itemize}
Perturbação da linguagem, por defeito da inteligência.
Perturbação, que ocasiona a paragem súbita, em meio de uma frase.
\section{Dislógico}
\begin{itemize}
\item {Grp. gram.:adj.}
\end{itemize}
Relativo á dislogia.
\section{Disnervado}
\begin{itemize}
\item {Grp. gram.:adj.}
\end{itemize}
\begin{itemize}
\item {Utilização:Med.}
\end{itemize}
\begin{itemize}
\item {Proveniência:(Do gr. \textunderscore dus\textunderscore  + lat. \textunderscore nervus\textunderscore )}
\end{itemize}
Diz-se do órgão, perturbado na sua enervação.
\section{Distopia}
\begin{itemize}
\item {Grp. gram.:f.}
\end{itemize}
\begin{itemize}
\item {Utilização:Med.}
\end{itemize}
\begin{itemize}
\item {Proveniência:(Do gr. \textunderscore dus\textunderscore  + \textunderscore topos\textunderscore )}
\end{itemize}
Situação nómala de um órgão.
\section{Donzelice}
\begin{itemize}
\item {Grp. gram.:f.}
\end{itemize}
Estado de donzela. Cf. Camillo, \textunderscore Bohémia\textunderscore , 141.
\section{Donzellice}
\begin{itemize}
\item {Grp. gram.:f.}
\end{itemize}
Estado de donzella. Cf. Camillo, \textunderscore Bohémia\textunderscore , 141.
\section{Dothienteria}
\begin{itemize}
\item {Grp. gram.:f.}
\end{itemize}
\begin{itemize}
\item {Utilização:Med.}
\end{itemize}
\begin{itemize}
\item {Proveniência:(Do gr. \textunderscore dothien\textunderscore  + \textunderscore enteron\textunderscore )}
\end{itemize}
Febre typhóide.
\section{Dotienteria}
\begin{itemize}
\item {Grp. gram.:f.}
\end{itemize}
\begin{itemize}
\item {Utilização:Med.}
\end{itemize}
\begin{itemize}
\item {Proveniência:(Do gr. \textunderscore dothien\textunderscore  + \textunderscore enteron\textunderscore )}
\end{itemize}
Febre tifóide.
\section{Duodenostomia}
\begin{itemize}
\item {Grp. gram.:f.}
\end{itemize}
\begin{itemize}
\item {Utilização:Med.}
\end{itemize}
\begin{itemize}
\item {Proveniência:(De \textunderscore duodeno\textunderscore  + gr. \textunderscore stoma\textunderscore )}
\end{itemize}
Abertura artificial no duodeno, em caso de estenose pylórica.
\section{Duma}
\begin{itemize}
\item {Grp. gram.:f.}
\end{itemize}
Designação do Parlamento russo.
\section{Duodecaédro}
\begin{itemize}
\item {Grp. gram.:m.}
\end{itemize}
O mesmo que \textunderscore dodecaedro\textunderscore .
\section{Dura-sempre}
\begin{itemize}
\item {Grp. gram.:m.}
\end{itemize}
\begin{itemize}
\item {Utilização:Prov.}
\end{itemize}
Espécie de tecido, o mesmo que \textunderscore pelle-do-diabo\textunderscore .
\section{Dýnio}
\begin{itemize}
\item {Grp. gram.:m.}
\end{itemize}
\begin{itemize}
\item {Proveniência:(Do gr. \textunderscore dunamis\textunderscore )}
\end{itemize}
Unidade de medidas de fôrças eléctricas, igual a 1^{gr.}, 981.
\section{Dyschromia}
\begin{itemize}
\item {Grp. gram.:f.}
\end{itemize}
\begin{itemize}
\item {Proveniência:(Do gr. \textunderscore dus\textunderscore  + \textunderscore khroma\textunderscore )}
\end{itemize}
Designação genérica de todas as perturbações da pigmentação da pelle.
\section{Dysdipsia}
\begin{itemize}
\item {Grp. gram.:f.}
\end{itemize}
\begin{itemize}
\item {Utilização:Med.}
\end{itemize}
\begin{itemize}
\item {Proveniência:(Do gr. \textunderscore dus\textunderscore  + \textunderscore dipsa\textunderscore )}
\end{itemize}
Dificuldade na deglutição dos líquidos.
\section{Dysenteriforme}
\begin{itemize}
\item {Grp. gram.:adj.}
\end{itemize}
\begin{itemize}
\item {Utilização:Med.}
\end{itemize}
\begin{itemize}
\item {Proveniência:(De \textunderscore dysenteria\textunderscore  + \textunderscore fórma\textunderscore )}
\end{itemize}
Diz-se das enterites, em que as evacuações se parecem com as da dysenteria.
\section{Dysgenesia}
\begin{itemize}
\item {Grp. gram.:f.}
\end{itemize}
\begin{itemize}
\item {Utilização:Med.}
\end{itemize}
\begin{itemize}
\item {Proveniência:(Do gr. \textunderscore dus\textunderscore  + \textunderscore genesis\textunderscore )}
\end{itemize}
Perturbação da funcção reproductora.
Cruzamentos, cujos productos são estéreis entre si, mas fecundos com indivíduos de outra raça mãe.
\section{Dysgenésico}
\begin{itemize}
\item {Grp. gram.:adj.}
\end{itemize}
Relativo á dysgenesia.
Que torna diffícil a reproducção.
\section{Dysgenético}
\begin{itemize}
\item {Grp. gram.:adj.}
\end{itemize}
O mesmo que \textunderscore dysgenésico\textunderscore .
\section{Dyslogia}
\begin{itemize}
\item {Grp. gram.:f.}
\end{itemize}
\begin{itemize}
\item {Utilização:Med.}
\end{itemize}
\begin{itemize}
\item {Proveniência:(Do gr. \textunderscore dus\textunderscore  + \textunderscore logos\textunderscore )}
\end{itemize}
Perturbação da linguagem, por defeito da intelligência.
Perturbação, que occasiona a paragem súbita, em meio de uma phrase.
\section{Dyslógico}
\begin{itemize}
\item {Grp. gram.:adj.}
\end{itemize}
Relativo á dyslogia.
\section{Dysnervado}
\begin{itemize}
\item {Grp. gram.:adj.}
\end{itemize}
\begin{itemize}
\item {Utilização:Med.}
\end{itemize}
\begin{itemize}
\item {Proveniência:(Do gr. \textunderscore dus\textunderscore  + lat. \textunderscore nervus\textunderscore )}
\end{itemize}
Diz-se do órgão, perturbado na sua ennervação.
\section{Dystopia}
\begin{itemize}
\item {Grp. gram.:f.}
\end{itemize}
\begin{itemize}
\item {Utilização:Med.}
\end{itemize}
\begin{itemize}
\item {Proveniência:(Do gr. \textunderscore dus\textunderscore  + \textunderscore topos\textunderscore )}
\end{itemize}
Situação nómala de um órgão.
\section{D}
\begin{itemize}
\item {fónica:dê}
\end{itemize}
\begin{itemize}
\item {Grp. gram.:m.}
\end{itemize}
\begin{itemize}
\item {Utilização:Mús.}
\end{itemize}
\begin{itemize}
\item {Grp. gram.:Adj.}
\end{itemize}
\begin{itemize}
\item {Utilização:Mús.}
\end{itemize}
Quarta letra do alphabeto português.
Quarto grau da escala, na antiga notação alphabética.
Diz-se daquillo que, numa série, occupa o quarto lugar.
Diz-se do tom de ré.
Vale 500, em numeração romana.
Quando maiúsculo, seguido de ponto e antes de nome próprio, é abrev. de \textunderscore dom\textunderscore  ou \textunderscore dona\textunderscore .
\section{Da}
\begin{itemize}
\item {fónica:dâ}
\end{itemize}
Contr. da prep. \textunderscore de\textunderscore  e do art. f. \textunderscore a\textunderscore .
\section{Dabinício}
\begin{itemize}
\item {Grp. gram.:adv.}
\end{itemize}
\begin{itemize}
\item {Utilização:Ant.}
\end{itemize}
\begin{itemize}
\item {Proveniência:(De \textunderscore de\textunderscore  + loc. lat. \textunderscore ab-initio\textunderscore )}
\end{itemize}
Desde o princípio, desde sempre.
\section{Dabo}
\begin{itemize}
\item {Grp. gram.:m.}
\end{itemize}
\begin{itemize}
\item {Utilização:Gír.}
\end{itemize}
Pai.
(Or. ind.?)
\section{Dabom}
\begin{itemize}
\item {Grp. gram.:m.}
\end{itemize}
Arvore da Índia portuguesa.
\section{Dabua}
\begin{itemize}
\item {Grp. gram.:f.}
\end{itemize}
Espécie de víbora, adorada por negros da África.
\section{Dabula}
\begin{itemize}
\item {Grp. gram.:f.}
\end{itemize}
Palmeira da Índia.
\section{Dação}
\begin{itemize}
\item {Grp. gram.:f.}
\end{itemize}
\begin{itemize}
\item {Utilização:Des.}
\end{itemize}
\begin{itemize}
\item {Utilização:Jur.}
\end{itemize}
\begin{itemize}
\item {Proveniência:(Lat. \textunderscore datio\textunderscore )}
\end{itemize}
Acto de dar.
Entrega de uma coisa, em pagamento de outra que se devia.
\section{Dachém}
\begin{itemize}
\item {Grp. gram.:m.}
\end{itemize}
Pêso antigo de pau ou pedra, em Malaca e na China.
\section{Dacma}
\begin{itemize}
\item {Grp. gram.:m.}
\end{itemize}
Espécie de tôrre, aberta superiormente, na qual os cadáveres dos Parses são expostos á voracidade dos abutres.
\section{Daco}
\begin{itemize}
\item {Grp. gram.:m.}
\end{itemize}
\begin{itemize}
\item {Proveniência:(Do gr. \textunderscore dax\textunderscore )}
\end{itemize}
Verme, que rói a madeira.
\section{Dacos}
\begin{itemize}
\item {Grp. gram.:m. pl.}
\end{itemize}
\begin{itemize}
\item {Proveniência:(Lat. \textunderscore Daci\textunderscore )}
\end{itemize}
Antigos habitantes da Dácia.
\section{Dacota}
\begin{itemize}
\item {Grp. gram.:m.}
\end{itemize}
Língua dos Dacotas.
\section{Dacotas}
\begin{itemize}
\item {Grp. gram.:m. pl.}
\end{itemize}
Uma das tríbos do território indiano dos Estados-Unidos da América do Norte.
\section{Dacri...}
\begin{itemize}
\item {Grp. gram.:pref.}
\end{itemize}
\begin{itemize}
\item {Proveniência:(Gr. \textunderscore dakru\textunderscore )}
\end{itemize}
(design. de lágrima)
\section{Dacriadenite}
\begin{itemize}
\item {Grp. gram.:f.}
\end{itemize}
\begin{itemize}
\item {Utilização:Med.}
\end{itemize}
\begin{itemize}
\item {Proveniência:(De \textunderscore dacry...\textunderscore  + \textunderscore adenite\textunderscore )}
\end{itemize}
Inflammação da glândula lacrimal.
\section{Dacrídio}
\begin{itemize}
\item {Grp. gram.:m.}
\end{itemize}
\begin{itemize}
\item {Proveniência:(Do gr. \textunderscore dakru\textunderscore , lágrima)}
\end{itemize}
Planta resinosa, da fam. das taxíneas.
\section{Dacrina}
\begin{itemize}
\item {Grp. gram.:f.}
\end{itemize}
Gênero de cogumelos.
\section{Dacrioblenorreia}
\begin{itemize}
\item {Grp. gram.:f.}
\end{itemize}
\begin{itemize}
\item {Utilização:Med.}
\end{itemize}
Corrimento mucoso das lágrimas.
\section{Dacriocistalgia}
\begin{itemize}
\item {Grp. gram.:f.}
\end{itemize}
\begin{itemize}
\item {Utilização:Med.}
\end{itemize}
Dôr, no saco lacrimal.
\section{Dacriociste}
\begin{itemize}
\item {Grp. gram.:m.}
\end{itemize}
\begin{itemize}
\item {Proveniência:(Do gr. \textunderscore dakru\textunderscore  + \textunderscore kustis\textunderscore )}
\end{itemize}
Saco lacrimal.
\section{Dacriocistite}
\begin{itemize}
\item {Grp. gram.:f.}
\end{itemize}
\begin{itemize}
\item {Utilização:Med.}
\end{itemize}
\begin{itemize}
\item {Proveniência:(De \textunderscore dacryocyste\textunderscore )}
\end{itemize}
Inflammação do saco lacrimal.
\section{Dacrióide}
\begin{itemize}
\item {Grp. gram.:adj.}
\end{itemize}
\begin{itemize}
\item {Proveniência:(Do gr. \textunderscore dakru\textunderscore  + \textunderscore eidos\textunderscore )}
\end{itemize}
Que tem fórma de lágrima.
\section{Dacriolina}
\begin{itemize}
\item {Grp. gram.:f.}
\end{itemize}
\begin{itemize}
\item {Proveniência:(Do gr. \textunderscore dakru\textunderscore )}
\end{itemize}
Substância orgânica das lágrimas.
\section{Dacriolithíase}
\begin{itemize}
\item {Grp. gram.:f.}
\end{itemize}
\begin{itemize}
\item {Utilização:Med.}
\end{itemize}
\begin{itemize}
\item {Proveniência:(De \textunderscore dacryólitho\textunderscore )}
\end{itemize}
Formação de cálculos nas vias lacrimaes.
\section{Dacriólito}
\begin{itemize}
\item {Grp. gram.:m.}
\end{itemize}
\begin{itemize}
\item {Utilização:Med.}
\end{itemize}
\begin{itemize}
\item {Proveniência:(Do gr. \textunderscore dakru\textunderscore  + \textunderscore lithos\textunderscore )}
\end{itemize}
Concreção nos canalículos lacrimaes.
\section{Dácrion}
\begin{itemize}
\item {Grp. gram.:m.}
\end{itemize}
\begin{itemize}
\item {Utilização:Med.}
\end{itemize}
\begin{itemize}
\item {Proveniência:(Do gr. \textunderscore dakruos\textunderscore , lágrima)}
\end{itemize}
Ponto craniométrico, determinado pelo encontro dos ossos frontal e lacrimal com a apófise ascendente do maxilar superior.
\section{Dacriorreia}
\begin{itemize}
\item {Grp. gram.:f.}
\end{itemize}
\begin{itemize}
\item {Utilização:Med.}
\end{itemize}
Fluxo lacrimal.
\section{Dacry...}
\begin{itemize}
\item {Grp. gram.:pref.}
\end{itemize}
\begin{itemize}
\item {Proveniência:(Gr. \textunderscore dakru\textunderscore )}
\end{itemize}
(design. de lágrima)
\section{Dacryadenite}
\begin{itemize}
\item {Grp. gram.:f.}
\end{itemize}
\begin{itemize}
\item {Utilização:Med.}
\end{itemize}
\begin{itemize}
\item {Proveniência:(De \textunderscore dacry...\textunderscore  + \textunderscore adenite\textunderscore )}
\end{itemize}
Inflammação da glândula lacrimal.
\section{Dacrýdio}
\begin{itemize}
\item {Grp. gram.:m.}
\end{itemize}
\begin{itemize}
\item {Proveniência:(Do gr. \textunderscore dakru\textunderscore , lágrima)}
\end{itemize}
Planta resinosa, da fam. das taxíneas.
\section{Dacryna}
\begin{itemize}
\item {Grp. gram.:f.}
\end{itemize}
Gênero de cogumelos.
\section{Dacryoblennorrheia}
\begin{itemize}
\item {Grp. gram.:f.}
\end{itemize}
\begin{itemize}
\item {Utilização:Med.}
\end{itemize}
Corrimento mucoso das lágrimas.
\section{Dacryocystalgia}
\begin{itemize}
\item {Grp. gram.:f.}
\end{itemize}
\begin{itemize}
\item {Utilização:Med.}
\end{itemize}
Dôr, no saco lacrimal.
\section{Dacryocyste}
\begin{itemize}
\item {Grp. gram.:m.}
\end{itemize}
\begin{itemize}
\item {Proveniência:(Do gr. \textunderscore dakru\textunderscore  + \textunderscore kustis\textunderscore )}
\end{itemize}
Saco lacrimal.
\section{Dacryocystite}
\begin{itemize}
\item {Grp. gram.:f.}
\end{itemize}
\begin{itemize}
\item {Utilização:Med.}
\end{itemize}
\begin{itemize}
\item {Proveniência:(De \textunderscore dacryocyste\textunderscore )}
\end{itemize}
Inflammação do saco lacrimal.
\section{Dacryóide}
\begin{itemize}
\item {Grp. gram.:adj.}
\end{itemize}
\begin{itemize}
\item {Proveniência:(Do gr. \textunderscore dakru\textunderscore  + \textunderscore eidos\textunderscore )}
\end{itemize}
Que tem fórma de lágrima.
\section{Dacryolina}
\begin{itemize}
\item {Grp. gram.:f.}
\end{itemize}
\begin{itemize}
\item {Proveniência:(Do gr. \textunderscore dakru\textunderscore )}
\end{itemize}
Substância orgânica das lágrimas.
\section{Dacryolithíase}
\begin{itemize}
\item {Grp. gram.:f.}
\end{itemize}
\begin{itemize}
\item {Utilização:Med.}
\end{itemize}
\begin{itemize}
\item {Proveniência:(De \textunderscore dacryólitho\textunderscore )}
\end{itemize}
Formação de cálculos nas vias lacrimaes.
\section{Dacryólitho}
\begin{itemize}
\item {Grp. gram.:m.}
\end{itemize}
\begin{itemize}
\item {Utilização:Med.}
\end{itemize}
\begin{itemize}
\item {Proveniência:(Do gr. \textunderscore dakru\textunderscore  + \textunderscore lithos\textunderscore )}
\end{itemize}
Concreção nos canalículos lacrimaes.
\section{Dácryon}
\begin{itemize}
\item {Grp. gram.:m.}
\end{itemize}
\begin{itemize}
\item {Utilização:Med.}
\end{itemize}
\begin{itemize}
\item {Proveniência:(Do gr. \textunderscore dakruos\textunderscore , lágrima)}
\end{itemize}
Ponto craniométrico, determinado pelo encontro dos ossos frontal e lacrimal com a apóphyse ascendente do maxillar superior.
\section{Dacryorrheia}
\begin{itemize}
\item {Grp. gram.:f.}
\end{itemize}
\begin{itemize}
\item {Utilização:Med.}
\end{itemize}
Fluxo lacrimal.
\section{Dáctila}
\begin{itemize}
\item {Grp. gram.:f.}
\end{itemize}
Planta gramínea, vivaz, de fôlhas largas, (\textunderscore dactylis glomerata\textunderscore , Lin.).
\section{Dactílico}
\begin{itemize}
\item {Grp. gram.:adj.}
\end{itemize}
\begin{itemize}
\item {Proveniência:(Gr. \textunderscore daktulikos\textunderscore )}
\end{itemize}
Relativo a dáctilo.
\section{Dactilino}
\begin{itemize}
\item {Grp. gram.:adj.}
\end{itemize}
\begin{itemize}
\item {Proveniência:(Do gr. \textunderscore daktulos\textunderscore )}
\end{itemize}
Semelhante a um dedo.
\section{Dactílion}
\begin{itemize}
\item {Grp. gram.:m.}
\end{itemize}
\begin{itemize}
\item {Proveniência:(Do gr. \textunderscore daktulos\textunderscore )}
\end{itemize}
Maquinismo, inventado por H. Herz, para dar aos pianistas principiantes fôrça e agilidade aos dedos.
\section{Dactilite}
\begin{itemize}
\item {Grp. gram.:f.}
\end{itemize}
\begin{itemize}
\item {Utilização:Med.}
\end{itemize}
\begin{itemize}
\item {Proveniência:(Do gr. \textunderscore daktulos\textunderscore )}
\end{itemize}
Inflamação do dedo; panarício.
\section{Dáctilo}
\begin{itemize}
\item {Grp. gram.:m.}
\end{itemize}
\begin{itemize}
\item {Proveniência:(Gr. \textunderscore daktulos\textunderscore )}
\end{itemize}
Pé de verso, grego ou latino, de uma sílaba longa, seguida de duas breves.
Antiga medida linear, entre, os Gregos, correspondente a meia polegada.
\section{Dactilografia}
\begin{itemize}
\item {Grp. gram.:f.}
\end{itemize}
Arte de escrever com o dactilógrafo.
\section{Dactilográfico}
\begin{itemize}
\item {Grp. gram.:adj.}
\end{itemize}
Relativo á dactilografia.
\section{Dactilógrafo}
\begin{itemize}
\item {Grp. gram.:m.}
\end{itemize}
\begin{itemize}
\item {Proveniência:(Do gr. \textunderscore daktulos\textunderscore  + \textunderscore graphein\textunderscore )}
\end{itemize}
Máquina de escrever, posta em movimento por meio de um teclado.
Aquele que escreve com essa máquina.
\section{Dactilóide}
\begin{itemize}
\item {Grp. gram.:adj.}
\end{itemize}
\begin{itemize}
\item {Proveniência:(Do gr. \textunderscore daktulos\textunderscore  + \textunderscore eidos\textunderscore )}
\end{itemize}
Que tem a fórma de dedo.
\section{Dactilologia}
\begin{itemize}
\item {Grp. gram.:f.}
\end{itemize}
\begin{itemize}
\item {Proveniência:(Do gr. \textunderscore dactulos\textunderscore  + \textunderscore logos\textunderscore )}
\end{itemize}
Arte de conversar, por meio de sinaes feitos com os dedos.
\section{Dactilomancia}
\begin{itemize}
\item {Grp. gram.:f.}
\end{itemize}
\begin{itemize}
\item {Proveniência:(Do gr. \textunderscore dactulos\textunderscore  + \textunderscore manteia\textunderscore )}
\end{itemize}
Suposta adivinhação por meio dos dedos. Cf. Castilho, \textunderscore Fastos\textunderscore , III, p. 319.
\section{Dactilomântico}
\begin{itemize}
\item {Grp. gram.:adj.}
\end{itemize}
Relativo á dactilomância.
\section{Dactilonomia}
\begin{itemize}
\item {Grp. gram.:f.}
\end{itemize}
\begin{itemize}
\item {Proveniência:(Do gr. \textunderscore daktulos\textunderscore  + \textunderscore nomos\textunderscore )}
\end{itemize}
Arte de exprimir números pela posição dos dedos sôbre as mãos ou das mãos sôbre o corpo.
Suposta arte de adivinhar, por meio de anéis constelados.
\section{Dactilópteros}
\begin{itemize}
\item {Grp. gram.:m. pl.}
\end{itemize}
\begin{itemize}
\item {Proveniência:(Do gr. \textunderscore daktulos\textunderscore  + \textunderscore pteron\textunderscore )}
\end{itemize}
Gênero de peixes acantopterígios, chamados também peixes voadores.
\section{Dactiloscopia}
\begin{itemize}
\item {Grp. gram.:f.}
\end{itemize}
\begin{itemize}
\item {Proveniência:(Do gr. \textunderscore daktulos\textunderscore  + \textunderscore skopein\textunderscore )}
\end{itemize}
O mesmo que \textunderscore dactilomancia\textunderscore .
Moderno sistema de identificação dos criminosos, por meio das impressões digitaes em tinta.
\section{Dactiloscópico}
\begin{itemize}
\item {Grp. gram.:adj.}
\end{itemize}
Relativo á dactiloscopia.
\section{Dactiloteca}
\begin{itemize}
\item {Grp. gram.:f.}
\end{itemize}
\begin{itemize}
\item {Utilização:Zool.}
\end{itemize}
\begin{itemize}
\item {Proveniência:(Do gr. \textunderscore daktulos\textunderscore  + \textunderscore theke\textunderscore )}
\end{itemize}
Museu ou colecção de anéis, joias e pedras gravadas.
Pele, que envolve cada um dos dedos dos mamíferos.
\section{Dactilozoário}
\begin{itemize}
\item {Grp. gram.:m.}
\end{itemize}
\begin{itemize}
\item {Proveniência:(Do gr. \textunderscore daktulos\textunderscore  + \textunderscore zoon\textunderscore )}
\end{itemize}
Hidra que, no pólipo hidráceo, desempenha as funções da apreensão dos alimentos.
\section{Dáctyla}
\begin{itemize}
\item {Grp. gram.:f.}
\end{itemize}
Planta gramínea, vivaz, de fôlhas largas, (\textunderscore dactylis glomerata\textunderscore , Lin.).
\section{Dactýlico}
\begin{itemize}
\item {Grp. gram.:adj.}
\end{itemize}
\begin{itemize}
\item {Proveniência:(Gr. \textunderscore daktulikos\textunderscore )}
\end{itemize}
Relativo a dáctylo.
\section{Dactylino}
\begin{itemize}
\item {Grp. gram.:adj.}
\end{itemize}
\begin{itemize}
\item {Proveniência:(Do gr. \textunderscore daktulos\textunderscore )}
\end{itemize}
Semelhante a um dedo.
\section{Dactýlion}
\begin{itemize}
\item {Grp. gram.:m.}
\end{itemize}
\begin{itemize}
\item {Proveniência:(Do gr. \textunderscore daktulos\textunderscore )}
\end{itemize}
Maquinismo, inventado por H. Herz, para dar aos pianistas principiantes fôrça e agilidade aos dedos.
\section{Dactylite}
\begin{itemize}
\item {Grp. gram.:f.}
\end{itemize}
\begin{itemize}
\item {Utilização:Med.}
\end{itemize}
\begin{itemize}
\item {Proveniência:(Do gr. \textunderscore daktulos\textunderscore )}
\end{itemize}
Inflammação do dedo; panarício.
\section{Dáctylo}
\begin{itemize}
\item {Grp. gram.:m.}
\end{itemize}
\begin{itemize}
\item {Proveniência:(Gr. \textunderscore daktulos\textunderscore )}
\end{itemize}
Pé de verso, grego ou latino, de uma sýllaba longa, seguida de duas breves.
Antiga medida linear, entre, os Gregos, correspondente a meia pollegada.
\section{Dactylographia}
\begin{itemize}
\item {Grp. gram.:f.}
\end{itemize}
Arte de escrever com o dactylógrapho.
\section{Dactilográphico}
\begin{itemize}
\item {Grp. gram.:adj.}
\end{itemize}
Relativo á dactylographia.
\section{Dactylógrapho}
\begin{itemize}
\item {Grp. gram.:m.}
\end{itemize}
\begin{itemize}
\item {Proveniência:(Do gr. \textunderscore daktulos\textunderscore  + \textunderscore graphein\textunderscore )}
\end{itemize}
Máquina de escrever, posta em movimento por meio de um teclado.
Aquelle que escreve com essa máquina.
\section{Dactylóide}
\begin{itemize}
\item {Grp. gram.:adj.}
\end{itemize}
\begin{itemize}
\item {Proveniência:(Do gr. \textunderscore daktulos\textunderscore  + \textunderscore eidos\textunderscore )}
\end{itemize}
Que tem a fórma de dedo.
\section{Dactylologia}
\begin{itemize}
\item {Grp. gram.:f.}
\end{itemize}
\begin{itemize}
\item {Proveniência:(Do gr. \textunderscore dactulos\textunderscore  + \textunderscore logos\textunderscore )}
\end{itemize}
Arte de conversar, por meio de sinaes feitos com os dedos.
\section{Dactylomancia}
\begin{itemize}
\item {Grp. gram.:f.}
\end{itemize}
\begin{itemize}
\item {Proveniência:(Do gr. \textunderscore dactulos\textunderscore  + \textunderscore manteia\textunderscore )}
\end{itemize}
Supposta adivinhação por meio dos dedos. Cf. Castilho, \textunderscore Fastos\textunderscore , III, p. 319.
\section{Dactylomântico}
\begin{itemize}
\item {Grp. gram.:adj.}
\end{itemize}
Relativo á dactylomância.
\section{Dactylonomia}
\begin{itemize}
\item {Grp. gram.:f.}
\end{itemize}
\begin{itemize}
\item {Proveniência:(Do gr. \textunderscore daktulos\textunderscore  + \textunderscore nomos\textunderscore )}
\end{itemize}
Arte de exprimir números pela posição dos dedos sôbre as mãos ou das mãos sôbre o corpo.
Supposta arte de adivinhar, por meio de anéis constellados.
\section{Dactylópteros}
\begin{itemize}
\item {Grp. gram.:m. pl.}
\end{itemize}
\begin{itemize}
\item {Proveniência:(Do gr. \textunderscore daktulos\textunderscore  + \textunderscore pteron\textunderscore )}
\end{itemize}
Gênero de peixes acanthopterýgios, chamados também peixes voadores.
\section{Dactyloscopia}
\begin{itemize}
\item {Grp. gram.:f.}
\end{itemize}
\begin{itemize}
\item {Proveniência:(Do gr. \textunderscore daktulos\textunderscore  + \textunderscore skopein\textunderscore )}
\end{itemize}
O mesmo que \textunderscore dactylomancia\textunderscore .
Moderno systema de identificação dos criminosos, por meio das impressões digitaes em tinta.
\section{Dactyloscópico}
\begin{itemize}
\item {Grp. gram.:adj.}
\end{itemize}
Relativo á dactyloscopia.
\section{Dactylotheca}
\begin{itemize}
\item {Grp. gram.:f.}
\end{itemize}
\begin{itemize}
\item {Utilização:Zool.}
\end{itemize}
\begin{itemize}
\item {Proveniência:(Do gr. \textunderscore daktulos\textunderscore  + \textunderscore theke\textunderscore )}
\end{itemize}
Museu ou collecção de anéis, joias e pedras gravadas.
Pelle, que envolve cada um dos dedos dos mammíferos.
\section{Dactylozoário}
\begin{itemize}
\item {Grp. gram.:m.}
\end{itemize}
\begin{itemize}
\item {Proveniência:(Do gr. \textunderscore daktulos\textunderscore  + \textunderscore zoon\textunderscore )}
\end{itemize}
Hydra que, no pólypo hydráceo, desempenha as funcções da apprehensão dos alimentos.
\section{Dada}
\begin{itemize}
\item {Grp. gram.:f.}
\end{itemize}
\begin{itemize}
\item {Utilização:Des.}
\end{itemize}
\begin{itemize}
\item {Utilização:Prov.}
\end{itemize}
\begin{itemize}
\item {Utilização:Prov.}
\end{itemize}
\begin{itemize}
\item {Utilização:beir.}
\end{itemize}
\begin{itemize}
\item {Utilização:ant.}
\end{itemize}
\begin{itemize}
\item {Proveniência:(De \textunderscore dar\textunderscore )}
\end{itemize}
Acto de dar.
Abcesso no úbere da vaca, determinado pela febre do leite.
Quebranto.
\section{Dadan}
\begin{itemize}
\item {Grp. gram.:m.}
\end{itemize}
\begin{itemize}
\item {Utilização:Bras}
\end{itemize}
Ave de rapina.
\section{Dadane}
\begin{itemize}
\item {Grp. gram.:m.}
\end{itemize}
Nome, que, em varios pontos da África, se dá á doença do somno. Cf. Capello e Ivens, \textunderscore De Benguella ás Terras de Iaca\textunderscore , I, 125.
\section{Dadila}
\begin{itemize}
\item {Grp. gram.:f.}
\end{itemize}
\begin{itemize}
\item {Proveniência:(Do gr. \textunderscore das\textunderscore  + \textunderscore ule\textunderscore )}
\end{itemize}
Um dos dois óleos, que constituem o de terebentina.
\section{Dádiva}
\begin{itemize}
\item {Grp. gram.:f.}
\end{itemize}
Objecto, que se dá; presente.
Donativo.
(Talvez do lat. \textunderscore nativus\textunderscore , com deslocação de accento)
\section{Dadivar}
\begin{itemize}
\item {Grp. gram.:v. t.}
\end{itemize}
Fazer dádivas a.
Brindar; presentear. Cf. Castilho, \textunderscore Fastos\textunderscore , I, p. 303.
\section{Dadivosamente}
\begin{itemize}
\item {Grp. gram.:adj.}
\end{itemize}
Á maneira de dadivoso.
\section{Dadivoso}
\begin{itemize}
\item {Grp. gram.:adj.}
\end{itemize}
\begin{itemize}
\item {Proveniência:(De \textunderscore dádiva\textunderscore )}
\end{itemize}
Que gosta de dar.
Que tem liberalidade.
\section{Dado}
\begin{itemize}
\item {Grp. gram.:m.}
\end{itemize}
\begin{itemize}
\item {Utilização:Constr.}
\end{itemize}
\begin{itemize}
\item {Proveniência:(Lat. \textunderscore datum\textunderscore )}
\end{itemize}
Pequeno cubo, de osso ou marfim, que se usa em certos jogos.
Elemento, quantidade conhecida, que serve de base á resolução de um problema.
Cada um dos princípios, em que assenta uma discussão.
Elemento, base, para a formação de um juizo: \textunderscore não tens dados para tal affirmação\textunderscore .
Um dos membros, em que se subdividem os membros de cada ordem de architectura.--Os membros desta são: pedestal, columna e entablamento; e cada um dêstes membros se subdivide em outros três: base, dado e cornija.
\section{Dado}
\begin{itemize}
\item {Grp. gram.:adj.}
\end{itemize}
\begin{itemize}
\item {Proveniência:(De \textunderscore dar\textunderscore )}
\end{itemize}
Permittido; concedido: \textunderscore dado que assim seja...\textunderscore 
Gratuito: \textunderscore a cavallo dado não se lhe olha o dente\textunderscore .
Datado: \textunderscore êste alvará é dado de Salvaterra aos 4 de Maio\textunderscore .
\section{Dador}
\begin{itemize}
\item {Grp. gram.:m.}
\end{itemize}
\begin{itemize}
\item {Proveniência:(Lat. \textunderscore dator\textunderscore )}
\end{itemize}
Aquelle que dá, que concede: \textunderscore o dador da Carta Constitucional\textunderscore .
\section{Dadyla}
\begin{itemize}
\item {Grp. gram.:f.}
\end{itemize}
\begin{itemize}
\item {Proveniência:(Do gr. \textunderscore das\textunderscore  + \textunderscore ule\textunderscore )}
\end{itemize}
Um dos dois óleos, que constituem o de terebentina.
\section{Daganha}
\begin{itemize}
\item {Grp. gram.:f.}
\end{itemize}
\begin{itemize}
\item {Utilização:Ant.}
\end{itemize}
O mesmo que \textunderscore deganha\textunderscore .
\section{Daguerreotipar}
\begin{itemize}
\item {Grp. gram.:v. t.}
\end{itemize}
\begin{itemize}
\item {Proveniência:(De \textunderscore daguerreótipo\textunderscore )}
\end{itemize}
Reproduzir por daguerreótipo.
Reproduzir exactamente.
Apresentar ou representar fielmente.
\section{Daguerreotipia}
\begin{itemize}
\item {Grp. gram.:f.}
\end{itemize}
\begin{itemize}
\item {Proveniência:(De \textunderscore daguerreótipo\textunderscore )}
\end{itemize}
Arte de daguerreotipar.
\section{Daguerreótipo}
\begin{itemize}
\item {Grp. gram.:m.}
\end{itemize}
\begin{itemize}
\item {Proveniência:(De \textunderscore Daguerre\textunderscore , n. p.)}
\end{itemize}
Primitivo apparelho de fotografia, inventado por Daguerre.
Reprodução ou pintura exacta.
Imagem, reproduzida por aquele apparelho.
\section{Daguerreotypar}
\begin{itemize}
\item {Grp. gram.:v. t.}
\end{itemize}
\begin{itemize}
\item {Proveniência:(De \textunderscore daguerreótypo\textunderscore )}
\end{itemize}
Reproduzir por daguerreótypo.
Reproduzir exactamente.
Apresentar ou representar fielmente.
\section{Daguerreotypia}
\begin{itemize}
\item {Grp. gram.:f.}
\end{itemize}
\begin{itemize}
\item {Proveniência:(De \textunderscore daguerreótypo\textunderscore )}
\end{itemize}
Arte de daguerreotypar.
\section{Daguerreótypo}
\begin{itemize}
\item {Grp. gram.:m.}
\end{itemize}
\begin{itemize}
\item {Proveniência:(De \textunderscore Daguerre\textunderscore , n. p.)}
\end{itemize}
Primitivo apparelho de photographia, inventado por Daguerre.
Reproducção ou pintura exacta.
Imagem, reproduzida por aquelle apparelho.
\section{Dagussa}
\begin{itemize}
\item {Grp. gram.:f.}
\end{itemize}
Nome que, na Abyssínia, se dá ao nachenim.
\section{Dahi}
(V.daí)
\section{D'ahi}
(V.daí)
\section{Dáhlia}
\begin{itemize}
\item {Grp. gram.:f.}
\end{itemize}
\begin{itemize}
\item {Proveniência:(De \textunderscore Dahl\textunderscore , n. p.)}
\end{itemize}
Planta, de flôres variegadas mas inodoras.
A flôr dessa Planta.
\section{Dahlina}
Substância do bolbo da dáhlia.
\section{Dahomeano}
\begin{itemize}
\item {Grp. gram.:adj.}
\end{itemize}
\begin{itemize}
\item {Grp. gram.:M.}
\end{itemize}
Relativo ao Dahomé ou Daomé.
Homem natural do Dahomé.
\section{Daí}
(contr. da prep. \textunderscore de\textunderscore , e do adv. \textunderscore aí\textunderscore )
\section{Daia}
\begin{itemize}
\item {Grp. gram.:f.}
\end{itemize}
\begin{itemize}
\item {Utilização:T. de Gôa}
\end{itemize}
O mesmo que \textunderscore parteira\textunderscore .
\section{Daião}
\begin{itemize}
\item {Grp. gram.:f.}
\end{itemize}
\begin{itemize}
\item {Utilização:Ant.}
\end{itemize}
O mesmo que \textunderscore deão\textunderscore .
(Cp. fr. \textunderscore doyen\textunderscore )
\section{Daiaque}
\begin{itemize}
\item {Grp. gram.:m.}
\end{itemize}
\begin{itemize}
\item {Grp. gram.:Pl.}
\end{itemize}
Língua dos archipélagos meridionaes da Oceânia.
Povos de Bornéu.
\section{Daimiado}
\begin{itemize}
\item {Grp. gram.:m.}
\end{itemize}
Território, governado por um dáimio.
\section{Daimiato}
\begin{itemize}
\item {Grp. gram.:m.}
\end{itemize}
Território, governado por um dáimio.
\section{Dáimio}
\begin{itemize}
\item {Grp. gram.:m.}
\end{itemize}
Antigo chefe despótico, em tribos japonesas.
\section{Daimoso}
\begin{itemize}
\item {Grp. gram.:adj.}
\end{itemize}
\begin{itemize}
\item {Utilização:Prov.}
\end{itemize}
Dadivoso; caritativo.
Affável; carinhoso.
\section{Dainaca}
\begin{itemize}
\item {Grp. gram.:f.}
\end{itemize}
\begin{itemize}
\item {Utilização:Ant.}
\end{itemize}
Barcaça, para atravessar rios.
Ponte, feita de barcaças.
\section{Daineca}
\begin{itemize}
\item {Grp. gram.:f.}
\end{itemize}
O mesmo que \textunderscore dainaca\textunderscore .
\section{Dairena}
\begin{itemize}
\item {Grp. gram.:f.}
\end{itemize}
Planta medicinal da Guiana inglesa.
\section{Dairo}
\begin{itemize}
\item {Grp. gram.:m.}
\end{itemize}
Nome, que, no Japão, depois da revolução de 1585, se deu ao imperador espiritual, dando-se o de \textunderscore cubo\textunderscore  ao temporal.
\section{Dal}
\begin{itemize}
\item {Grp. gram.:m.}
\end{itemize}
Cesto de bambu, com que os Índios medem os cereaes.
(Do conc.)
\section{Dala}
\begin{itemize}
\item {Grp. gram.:f.}
\end{itemize}
\begin{itemize}
\item {Utilização:Prov.}
\end{itemize}
\begin{itemize}
\item {Utilização:minh.}
\end{itemize}
\begin{itemize}
\item {Proveniência:(Do ant. al. \textunderscore dal\textunderscore )}
\end{itemize}
Sulco ou calha, que dá vasão a águas ou outros líquidos.
Espécie de pia, onde se lava a loiça.
\section{Dala}
\begin{itemize}
\item {Grp. gram.:f.}
\end{itemize}
\begin{itemize}
\item {Utilização:Pesc.}
\end{itemize}
Cabo da rede das armações redondas de Peniche.
\section{Dala}
\begin{itemize}
\item {Grp. gram.:f.}
\end{itemize}
\begin{itemize}
\item {Proveniência:(Ingl. \textunderscore dale\textunderscore )}
\end{itemize}
Terreno ou caminho entre montanhas.
\section{Dala}
\begin{itemize}
\item {Grp. gram.:f.}
\end{itemize}
\begin{itemize}
\item {Utilização:T. do Porto}
\end{itemize}
\begin{itemize}
\item {Proveniência:(Fr. \textunderscore dalle\textunderscore )}
\end{itemize}
Mesa de cozinha, com tabuleiro de pedra ou loisa.
\section{Dalechâmpia}
\begin{itemize}
\item {Grp. gram.:f.}
\end{itemize}
\begin{itemize}
\item {Proveniência:(De \textunderscore Dalechamp\textunderscore , n. p.)}
\end{itemize}
Planta euphorbiácea, originária do México.
\section{Dali}
(contr. da prep. \textunderscore de\textunderscore  e do adv. \textunderscore ali\textunderscore )
\section{Dália}
\begin{itemize}
\item {Grp. gram.:f.}
\end{itemize}
\begin{itemize}
\item {Proveniência:(De \textunderscore Dahl\textunderscore , n. p.)}
\end{itemize}
Planta, de flôres variegadas mas inodoras.
A flôr dessa Planta.
\section{Dalina}
\begin{itemize}
\item {Grp. gram.:f.}
\end{itemize}
Substância do bolbo da dália.
\section{Dalli}
(contr. da prep. \textunderscore de\textunderscore  e do adv. \textunderscore alli\textunderscore )
\section{Dálmata}
\begin{itemize}
\item {Grp. gram.:adj.}
\end{itemize}
\begin{itemize}
\item {Grp. gram.:M.}
\end{itemize}
\begin{itemize}
\item {Proveniência:(Lat. \textunderscore dalmatae\textunderscore )}
\end{itemize}
Relativo á Dalmácia.
Homem natural da Dalmácia.
\section{Dalmática}
\begin{itemize}
\item {Grp. gram.:f.}
\end{itemize}
\begin{itemize}
\item {Proveniência:(Lat. \textunderscore dalmatica\textunderscore )}
\end{itemize}
Paramento, que os Diáconos e Subdiáconos vestem sobre a alva.
Antiga vestimenta de Bispos.
Túnica branca, bordada de púrpura, que se fabricava na Dalmácia.
\section{Dalmaticado}
\begin{itemize}
\item {Grp. gram.:adj.}
\end{itemize}
Vestido de dalmática.
\section{Daltónico}
\begin{itemize}
\item {Grp. gram.:adj.}
\end{itemize}
\begin{itemize}
\item {Grp. gram.:M.}
\end{itemize}
Relativo a daltonismo.
Aquelle que padece daltonismo.
\section{Daltonismo}
\begin{itemize}
\item {Grp. gram.:m.}
\end{itemize}
\begin{itemize}
\item {Proveniência:(De \textunderscore Dalton\textunderscore , n. p.)}
\end{itemize}
Incapacidade de distinguir cores.
Propriamente, falta de percepção do vermelho e do verde.
\section{Dama}
\begin{itemize}
\item {Grp. gram.:f.}
\end{itemize}
\begin{itemize}
\item {Grp. gram.:Pl.}
\end{itemize}
\begin{itemize}
\item {Proveniência:(Fr. \textunderscore dame\textunderscore , do lat. \textunderscore domina\textunderscore )}
\end{itemize}
Mulher nobre, bem educada.
Actriz.
Mulher, que toma parte num baile.
Uma das cartas de jogar.
Uma das peças do xadrez.
Uma das tábulas do jogo das damas, quando chegada á última linha do respectivo tabuleiro.
Nome de um jôgo.
Montículo, que se deixa em meio de uma escavação, para depois se conhecer e medir a profundidade desta.
\section{Damacuris}
\begin{itemize}
\item {Grp. gram.:f. pl.}
\end{itemize}
Indígenas brasileiros da região do Amazonas.
\section{Damado}
\begin{itemize}
\item {Grp. gram.:adj.}
\end{itemize}
\begin{itemize}
\item {Utilização:Ant.}
\end{itemize}
\begin{itemize}
\item {Proveniência:(De \textunderscore dama\textunderscore )}
\end{itemize}
Querido.
Amante. Cf. G. Vicente.
\section{Dama-dos-jardins}
\begin{itemize}
\item {Grp. gram.:f.}
\end{itemize}
\begin{itemize}
\item {Utilização:Bras}
\end{itemize}
Planta annual.
\section{Dama-entre-verdes}
\begin{itemize}
\item {Grp. gram.:f.}
\end{itemize}
\begin{itemize}
\item {Utilização:Bras}
\end{itemize}
Planta cucurbitácea, (\textunderscore nigella damascena\textunderscore , Lin.)
\section{Damaísmo}
\begin{itemize}
\item {Grp. gram.:m.}
\end{itemize}
Conjunto de damas.
As damas.
Modos de dama. Cf. Camillo, \textunderscore Narcóticos\textunderscore , II, p. 273.
\section{Daman}
\begin{itemize}
\item {Grp. gram.:m.}
\end{itemize}
Árvore indiana, (\textunderscore grewia tília\textunderscore ).
\section{Damanense}
\begin{itemize}
\item {Grp. gram.:m.  e  adj.}
\end{itemize}
O que é de Damão.
\section{Dama-nua}
\begin{itemize}
\item {Grp. gram.:f.}
\end{itemize}
O mesmo que \textunderscore cólchico\textunderscore .
\section{Damão}
\begin{itemize}
\item {Grp. gram.:m.}
\end{itemize}
Espécie de chibo africano.
\section{Dâmar}
\begin{itemize}
\item {Grp. gram.:m.}
\end{itemize}
\begin{itemize}
\item {Proveniência:(T. mal.)}
\end{itemize}
Espécie de resina, extrahida da dâmara.
\section{Dâmara}
\begin{itemize}
\item {Grp. gram.:f.}
\end{itemize}
\begin{itemize}
\item {Proveniência:(De \textunderscore dâmar\textunderscore )}
\end{itemize}
Gênero de plantas resinosas, da fam. das abietíneas.
\section{Damaras}
\begin{itemize}
\item {Grp. gram.:m. pl.}
\end{itemize}
Povo da África austral.
\section{Damasceno}
\begin{itemize}
\item {Grp. gram.:adj.}
\end{itemize}
O mesmo que \textunderscore damasquino\textunderscore .
\section{Damasco}
\begin{itemize}
\item {Grp. gram.:m.}
\end{itemize}
\begin{itemize}
\item {Utilização:Ext.}
\end{itemize}
\begin{itemize}
\item {Proveniência:(De \textunderscore Damasco\textunderscore , n. p.)}
\end{itemize}
Fruto do damasqueiro.
Tecido de seda com tafetá, fabricado primitivamente em Damasco.
Tecido, que imita o damasco.
\section{Damasónio}
\begin{itemize}
\item {Grp. gram.:m.}
\end{itemize}
\begin{itemize}
\item {Proveniência:(Gr. \textunderscore damasonion\textunderscore )}
\end{itemize}
Planta aquática, o mesmo que \textunderscore tanchagem\textunderscore  dos pântanos.
\section{Damasqueiro}
\begin{itemize}
\item {Grp. gram.:m.}
\end{itemize}
\begin{itemize}
\item {Proveniência:(De \textunderscore damasco\textunderscore )}
\end{itemize}
Árvore rosácea, da tríbo das amygdaláceas, cujo fruto se chama damasco.
\section{Damasquilho}
\begin{itemize}
\item {Grp. gram.:m.}
\end{itemize}
\begin{itemize}
\item {Proveniência:(De \textunderscore damasco\textunderscore )}
\end{itemize}
Tecido adamascado.
\section{Damasquim}
\begin{itemize}
\item {Grp. gram.:m.}
\end{itemize}
\begin{itemize}
\item {Utilização:Ant.}
\end{itemize}
\begin{itemize}
\item {Proveniência:(De \textunderscore damasco\textunderscore )}
\end{itemize}
Coberta ou outra peça de damasco.
\section{Dafnáceas}
\begin{itemize}
\item {Grp. gram.:f. pl.}
\end{itemize}
\begin{itemize}
\item {Proveniência:(De \textunderscore dafnáceo\textunderscore )}
\end{itemize}
Família de plantas, arbustivas e herbáceas, que tem por tipo a dafne.
\section{Dafnáceo}
\begin{itemize}
\item {Grp. gram.:adj.}
\end{itemize}
Relativo ou semelhante á dafne.
\section{Dafne}
\begin{itemize}
\item {Grp. gram.:f.}
\end{itemize}
\begin{itemize}
\item {Proveniência:(Gr. \textunderscore daphne\textunderscore , loireiro)}
\end{itemize}
Gênero de plantas, de fruto e casca medicinaes.
\section{Dafneforia}
\begin{itemize}
\item {Grp. gram.:f.}
\end{itemize}
\begin{itemize}
\item {Proveniência:(Do gr. \textunderscore daphne\textunderscore  + \textunderscore phoros\textunderscore )}
\end{itemize}
Festa, que os Beócios celebravam, de nove em nove anos, em honra de Apolo.
\section{Dafnina}
\begin{itemize}
\item {Grp. gram.:f.}
\end{itemize}
\begin{itemize}
\item {Proveniência:(Do gr. \textunderscore daphne\textunderscore )}
\end{itemize}
Substância volátil da casca da \textunderscore dafne alpina\textunderscore .
\section{Dafnite}
\begin{itemize}
\item {Grp. gram.:f.}
\end{itemize}
\begin{itemize}
\item {Proveniência:(De \textunderscore dafne\textunderscore )}
\end{itemize}
Espécie de loireiro, o mesmo que \textunderscore espirradeira\textunderscore .
\section{Dafnoídeas}
\begin{itemize}
\item {Grp. gram.:f. pl.}
\end{itemize}
(V.dafnáceas)
\section{Dafnomancia}
\begin{itemize}
\item {Grp. gram.:f.}
\end{itemize}
\begin{itemize}
\item {Proveniência:(Do gr. \textunderscore daphne\textunderscore  + \textunderscore manteia\textunderscore )}
\end{itemize}
Adivinhação por meio de fôlhas de loireiro queimadas.
\section{Dafnomântico}
\begin{itemize}
\item {Grp. gram.:adj.}
\end{itemize}
Relativo á dafnomancia.
\section{Damasquinaria}
\begin{itemize}
\item {Grp. gram.:f.}
\end{itemize}
\begin{itemize}
\item {Proveniência:(De \textunderscore damasquino\textunderscore )}
\end{itemize}
Arte de embutir desenhos de oiro ou prata num metal menos brilhante, que serve de fundo.
Tauxia.
\section{Damasquino}
\begin{itemize}
\item {Grp. gram.:adj.}
\end{itemize}
\begin{itemize}
\item {Proveniência:(De \textunderscore Damasco\textunderscore , n. p.)}
\end{itemize}
Relativo a Damasco.
Diz-se especialmente das armas brancas, que têm lavores, como as que se fabricavam em Damasco.
\section{Damba}
\begin{itemize}
\item {Grp. gram.:f.}
\end{itemize}
\begin{itemize}
\item {Utilização:T. de Angola}
\end{itemize}
Depressão ou desfiladeiro entre dois morros, por onde correm as águas pluviaes para algum rio. Cf. Capello e Ivens, I, 34.
\section{Dambi}
\begin{itemize}
\item {Grp. gram.:m.}
\end{itemize}
Indivíduo importante, entre os do séquito de alguns sobas. Cf. Capello e Ivens, II, p. 52.
\section{Damborá}
\begin{itemize}
\item {Grp. gram.:m.}
\end{itemize}
Pequena árvore, (\textunderscore conocarpus latifolia\textunderscore ), cujo tronco resistente é empregado em eixos de carrêtas, na Índia portuguesa.
\section{Damejar}
\begin{itemize}
\item {Grp. gram.:v. t.}
\end{itemize}
\begin{itemize}
\item {Proveniência:(De \textunderscore dama\textunderscore )}
\end{itemize}
Galantear; cortejar (damas).
\section{Dami}
\begin{itemize}
\item {Grp. gram.:m.}
\end{itemize}
\begin{itemize}
\item {Utilização:Ant.}
\end{itemize}
Certo pano de seda, ás vezes tecido de oiro.
\section{Damiana}
\begin{itemize}
\item {Grp. gram.:f.}
\end{itemize}
\begin{itemize}
\item {Utilização:Bras}
\end{itemize}
Planta portulácea e medicinal, (\textunderscore turnera aphrodisiaca\textunderscore ).
\section{Damice}
\begin{itemize}
\item {Grp. gram.:f.}
\end{itemize}
\begin{itemize}
\item {Utilização:Fam.}
\end{itemize}
\begin{itemize}
\item {Proveniência:(De \textunderscore dama\textunderscore )}
\end{itemize}
Melindre feminino.
Modos de dama affectada; affectação.
\section{Damnação}
\begin{itemize}
\item {Grp. gram.:f.}
\end{itemize}
\begin{itemize}
\item {Proveniência:(Lat. \textunderscore damnatio\textunderscore )}
\end{itemize}
Acto ou effeito de damnar.
\section{Damnador}
\begin{itemize}
\item {Grp. gram.:m.  e  adj.}
\end{itemize}
\begin{itemize}
\item {Proveniência:(Lat. \textunderscore damnator\textunderscore )}
\end{itemize}
O que damna.
\section{Damnamento}
\begin{itemize}
\item {Grp. gram.:m.}
\end{itemize}
(V.damnação)
\section{Damnar}
\begin{itemize}
\item {Grp. gram.:v. t.}
\end{itemize}
\begin{itemize}
\item {Utilização:Fig.}
\end{itemize}
\begin{itemize}
\item {Utilização:Des.}
\end{itemize}
\begin{itemize}
\item {Grp. gram.:V. i.}
\end{itemize}
\begin{itemize}
\item {Utilização:Bras. de Minas}
\end{itemize}
\begin{itemize}
\item {Utilização:Ant.}
\end{itemize}
\begin{itemize}
\item {Proveniência:(Lat. \textunderscore damnare\textunderscore )}
\end{itemize}
Tornar hydróphobo: \textunderscore o cão damnou-se\textunderscore .
Damnificar: \textunderscore o gado damnou a vinha\textunderscore .
Perverter.
Irritar: \textunderscore aquella offensa damnou-me\textunderscore .
Condemnar.
Reprovar, tornar réprobo, maldito.
\textunderscore Damnar do juízo\textunderscore , endoidecer.
Soffrer damno. Cf. \textunderscore Rev. Lus.\textunderscore , XVI, 2.
\section{Damnificação}
\begin{itemize}
\item {Grp. gram.:f.}
\end{itemize}
Acto ou effeito de damnificar.
\section{Damnificador}
\begin{itemize}
\item {Grp. gram.:m.  e  adj.}
\end{itemize}
O que damnifica.
\section{Damnificamento}
\begin{itemize}
\item {Grp. gram.:m.}
\end{itemize}
\begin{itemize}
\item {Utilização:Des.}
\end{itemize}
O mesmo que \textunderscore damnificação\textunderscore . Cf. \textunderscore Carta Régia\textunderscore , em que D. Manuel conferiu mercês a Vasco da Gama.
\section{Damnificar}
\begin{itemize}
\item {Grp. gram.:v. t.}
\end{itemize}
\begin{itemize}
\item {Proveniência:(Lat. \textunderscore damnificare\textunderscore )}
\end{itemize}
Causar damno a: \textunderscore a chuva damnificou a seara\textunderscore .
\section{Damnífico}
\begin{itemize}
\item {Grp. gram.:adj.}
\end{itemize}
\begin{itemize}
\item {Proveniência:(Lat. \textunderscore damnificus\textunderscore )}
\end{itemize}
Que causa damno.
\section{Damninho}
\begin{itemize}
\item {Grp. gram.:adj.}
\end{itemize}
Que causa damno: \textunderscore ervas damninhas\textunderscore .
\section{Damno}
\begin{itemize}
\item {Grp. gram.:m.}
\end{itemize}
\begin{itemize}
\item {Proveniência:(Lat. \textunderscore damnum\textunderscore )}
\end{itemize}
Mal, que se faz a alguém.
Prejuizo, deterioração feita em coisas alheias.
Perda.
\section{Damnosamente}
\begin{itemize}
\item {Grp. gram.:adv.}
\end{itemize}
De modo damnoso.
\section{Damnoso}
\begin{itemize}
\item {Grp. gram.:adj.}
\end{itemize}
\begin{itemize}
\item {Proveniência:(Lat. \textunderscore damnosus\textunderscore )}
\end{itemize}
Que causa damno.
\section{Damui}
\begin{itemize}
\item {Grp. gram.:m.}
\end{itemize}
Árvore de Damão.
\section{Danação}
\begin{itemize}
\item {Grp. gram.:f.}
\end{itemize}
\begin{itemize}
\item {Proveniência:(Lat. \textunderscore damnatio\textunderscore )}
\end{itemize}
Acto ou efeito de danar.
\section{Danador}
\begin{itemize}
\item {Grp. gram.:m.  e  adj.}
\end{itemize}
\begin{itemize}
\item {Proveniência:(Lat. \textunderscore damnator\textunderscore )}
\end{itemize}
O que dana.
\section{Danaide}
\begin{itemize}
\item {Grp. gram.:f.}
\end{itemize}
\begin{itemize}
\item {Proveniência:(Lat. \textunderscore danaides\textunderscore )}
\end{itemize}
Espécie de borboleta diurna.
Planta rubiácea.
Espécie de roda hydráulica.
Designação mythológica de cada uma das cincoenta irmans que, á excepção de uma, assassinaram seus maridos em a noite de núpcias e foram condemnadas no Inferno a encher um tonel sem fundo.
\section{Danais}
\begin{itemize}
\item {Grp. gram.:m.}
\end{itemize}
Gênero de plantas rubiáceas.
\section{Danamento}
\begin{itemize}
\item {Grp. gram.:m.}
\end{itemize}
(V.danação)
\section{Danar}
\begin{itemize}
\item {Grp. gram.:v. t.}
\end{itemize}
\begin{itemize}
\item {Utilização:Fig.}
\end{itemize}
\begin{itemize}
\item {Utilização:Des.}
\end{itemize}
\begin{itemize}
\item {Grp. gram.:V. i.}
\end{itemize}
\begin{itemize}
\item {Utilização:Bras. de Minas}
\end{itemize}
\begin{itemize}
\item {Proveniência:(Lat. \textunderscore damnare\textunderscore )}
\end{itemize}
Tornar hidrófobo: \textunderscore o cão danou-se\textunderscore .
Danificar: \textunderscore o gado danou a vinha\textunderscore .
Perverter.
Irritar: \textunderscore aquela ofensa danou-me\textunderscore .
Condenar.
Reprovar, tornar réprobo, maldito.
\textunderscore Danar do juízo\textunderscore , endoidecer.
\section{Dança}
\begin{itemize}
\item {Grp. gram.:f.}
\end{itemize}
\begin{itemize}
\item {Utilização:Fig.}
\end{itemize}
Saltos ou passos cadenciados, geralmente ao som e compasso de música.
Baile.
Labutação.
Questões.
Negócio intrincado: \textunderscore meteram-me numa bôa dança\textunderscore !
(Cp. cast. \textunderscore danza\textunderscore )
\section{Dançadeira}
\begin{itemize}
\item {Grp. gram.:f.}
\end{itemize}
Mulher que dança; dançarina, bailarina.
\section{Dançador}
\begin{itemize}
\item {Grp. gram.:m.}
\end{itemize}
Aquelle que dança.
Dançarino.
Aquelle que dança por offício.
\section{Dançante}
\begin{itemize}
\item {Grp. gram.:adj.}
\end{itemize}
Que dança.
Em que há dança: \textunderscore sarau dançante\textunderscore .
\section{Dançar}
\begin{itemize}
\item {Grp. gram.:v. i.}
\end{itemize}
\begin{itemize}
\item {Grp. gram.:V. t.}
\end{itemize}
\begin{itemize}
\item {Proveniência:(De \textunderscore dança\textunderscore )}
\end{itemize}
Dar saltos ou passos cadenciados, de ordinário ao som e compasso de música.
Saltar.
Girar.
Executar, dançando: \textunderscore dançar uma valsa\textunderscore .
\section{Dançarás}
\begin{itemize}
\item {Grp. gram.:m.}
\end{itemize}
\begin{itemize}
\item {Utilização:Pop.}
\end{itemize}
Bailarico. Cf. Castilho, \textunderscore Tartufo\textunderscore , 11.
\section{Dançarina}
\begin{itemize}
\item {Grp. gram.:f.}
\end{itemize}
Mulher, que dança por offício.
\section{Dançarino}
\begin{itemize}
\item {Grp. gram.:m.}
\end{itemize}
\begin{itemize}
\item {Grp. gram.:Adj.}
\end{itemize}
\begin{itemize}
\item {Proveniência:(De \textunderscore dançar\textunderscore )}
\end{itemize}
Homem, que dança por offício.
Relativo a dança:«\textunderscore o mesmo soldão cede aos impulsos dançarinos\textunderscore ». Filinto, VII, 110.
\section{Dançatriz}
\begin{itemize}
\item {Grp. gram.:adj. f.}
\end{itemize}
\begin{itemize}
\item {Utilização:Des.}
\end{itemize}
\begin{itemize}
\item {Grp. gram.:F.}
\end{itemize}
Que dança.
Que excita a dançar.
Mulher que dança; dançarina.
\section{Dance}
\begin{itemize}
\item {Grp. gram.:m.}
\end{itemize}
\begin{itemize}
\item {Utilização:T. da Bairrada}
\end{itemize}
O costume ou vício de dançar.
\section{Dandá}
\begin{itemize}
\item {Grp. gram.:m.}
\end{itemize}
\begin{itemize}
\item {Utilização:Bras}
\end{itemize}
Noz purgativa.
\section{Danda-multom}
\begin{itemize}
\item {Grp. gram.:m.}
\end{itemize}
Arbusto africano, de fôlhas simples e oppostas, flôres gamopétalas, de aroma suave.
\section{Dandão}
\begin{itemize}
\item {Grp. gram.:m.}
\end{itemize}
\begin{itemize}
\item {Utilização:Bras}
\end{itemize}
\begin{itemize}
\item {Utilização:fam.}
\end{itemize}
Pesadelo nocturno.
(Por \textunderscore duendão\textunderscore , de \textunderscore duende\textunderscore ?)
\section{Dandinar}
\begin{itemize}
\item {Grp. gram.:v. i.}
\end{itemize}
\begin{itemize}
\item {Utilização:Gal}
\end{itemize}
\begin{itemize}
\item {Proveniência:(Fr. \textunderscore dandiner\textunderscore )}
\end{itemize}
Caminhar, bamboleando-se. Cf. Garrett, \textunderscore Helena\textunderscore , 101.
\section{Danês}
\begin{itemize}
\item {Grp. gram.:m.  e  adj.}
\end{itemize}
\begin{itemize}
\item {Utilização:P. us.}
\end{itemize}
\begin{itemize}
\item {Proveniência:(Fr. \textunderscore danois\textunderscore )}
\end{itemize}
O mesmo que \textunderscore dinamarquês\textunderscore . Cf. J. Ribeiro, \textunderscore Esthética\textunderscore .
\section{Danificação}
\begin{itemize}
\item {Grp. gram.:f.}
\end{itemize}
Acto ou efeito de danificar.
\section{Danificador}
\begin{itemize}
\item {Grp. gram.:m.  e  adj.}
\end{itemize}
O que danifica.
\section{Danificamento}
\begin{itemize}
\item {Grp. gram.:m.}
\end{itemize}
\begin{itemize}
\item {Utilização:Des.}
\end{itemize}
O mesmo que \textunderscore danificação\textunderscore . Cf. \textunderscore Carta Régia\textunderscore , em que D. Manuel conferiu mercês a Vasco da Gama.
\section{Danificar}
\begin{itemize}
\item {Grp. gram.:v. t.}
\end{itemize}
\begin{itemize}
\item {Proveniência:(Lat. \textunderscore damnificare\textunderscore )}
\end{itemize}
Causar dano a: \textunderscore a chuva danificou a seara\textunderscore .
\section{Danífico}
\begin{itemize}
\item {Grp. gram.:adj.}
\end{itemize}
\begin{itemize}
\item {Proveniência:(Lat. \textunderscore damnificus\textunderscore )}
\end{itemize}
Que causa dano.
\section{Daninho}
\begin{itemize}
\item {Grp. gram.:adj.}
\end{itemize}
Que causa dano: \textunderscore ervas daninhas\textunderscore .
\section{Dânio}
\begin{itemize}
\item {Grp. gram.:adj.}
\end{itemize}
\begin{itemize}
\item {Proveniência:(De \textunderscore danos\textunderscore )}
\end{itemize}
Relativo á Dinamarca ou aos dinamarqueses.
\section{Danismo}
\begin{itemize}
\item {Grp. gram.:m.}
\end{itemize}
\begin{itemize}
\item {Utilização:Des.}
\end{itemize}
O mesmo que \textunderscore usura\textunderscore .
(Cp. \textunderscore danista\textunderscore )
\section{Danista}
\begin{itemize}
\item {Grp. gram.:m.}
\end{itemize}
\begin{itemize}
\item {Utilização:Des.}
\end{itemize}
\begin{itemize}
\item {Proveniência:(Lat. \textunderscore danista\textunderscore )}
\end{itemize}
O mesmo que \textunderscore usurário\textunderscore .
\section{Danístico}
\begin{itemize}
\item {Grp. gram.:adj.}
\end{itemize}
\begin{itemize}
\item {Proveniência:(De \textunderscore danista\textunderscore )}
\end{itemize}
Relativo á usura ou aos danistas.
\section{Dano}
\begin{itemize}
\item {Grp. gram.:m.}
\end{itemize}
\begin{itemize}
\item {Proveniência:(Lat. \textunderscore damnum\textunderscore )}
\end{itemize}
Mal, que se faz a alguém.
Prejuizo, deterioração feita em coisas alheias.
Perda.
\section{Danos}
\begin{itemize}
\item {Grp. gram.:m. pl.}
\end{itemize}
\begin{itemize}
\item {Proveniência:(Lat. \textunderscore dani\textunderscore )}
\end{itemize}
O mesmo que [[Dinamarqueses|dinamarquês]].
\section{Danosamente}
\begin{itemize}
\item {Grp. gram.:adv.}
\end{itemize}
De modo danoso.
\section{Danoso}
\begin{itemize}
\item {Grp. gram.:adj.}
\end{itemize}
\begin{itemize}
\item {Proveniência:(Lat. \textunderscore damnosus\textunderscore )}
\end{itemize}
Que causa dano.
\section{Danta}
\begin{itemize}
\item {Grp. gram.:f.}
\end{itemize}
Animal da Sibéria.
\section{Dante}
\begin{itemize}
\item {Grp. gram.:adj.}
\end{itemize}
\begin{itemize}
\item {Utilização:Ant.}
\end{itemize}
\begin{itemize}
\item {Proveniência:(Lat. \textunderscore dans\textunderscore )}
\end{itemize}
Que põe data.
Datado.
Que dá.
\section{Dantes}
\begin{itemize}
\item {Grp. gram.:adv.}
\end{itemize}
\begin{itemize}
\item {Proveniência:(De \textunderscore de\textunderscore  + \textunderscore antes\textunderscore )}
\end{itemize}
Noutro tempo.
Antigamente.
Antes de agora.
\section{Dantesco}
\begin{itemize}
\item {fónica:tês}
\end{itemize}
\begin{itemize}
\item {Grp. gram.:adj.}
\end{itemize}
Relatívo a Dante.
Que se parece a obras de Dante.
\section{Dântico}
\begin{itemize}
\item {Grp. gram.:adj.}
\end{itemize}
\begin{itemize}
\item {Utilização:P. us.}
\end{itemize}
O mesmo que \textunderscore dantesco\textunderscore . Cf. Garrett, \textunderscore Catão\textunderscore , 24.
\section{Danubiano}
\begin{itemize}
\item {Grp. gram.:adj.}
\end{itemize}
\begin{itemize}
\item {Proveniência:(De \textunderscore Danúbio\textunderscore , n. p.)}
\end{itemize}
Relativo ao Danúbio ou aos povos que o ladeiam.
\section{Daomeano}
\begin{itemize}
\item {Grp. gram.:adj.}
\end{itemize}
Relativo ao Dahomé ou Daomé.
\section{Daphnáceas}
\begin{itemize}
\item {Grp. gram.:f. pl.}
\end{itemize}
\begin{itemize}
\item {Proveniência:(De \textunderscore daphnáceo\textunderscore )}
\end{itemize}
Família de plantas, arbustivas e herbáceas, que tem por typo a daphne.
\section{Daphnáceo}
\begin{itemize}
\item {Grp. gram.:adj.}
\end{itemize}
Relativo ou semelhante á daphne.
\section{Daphne}
\begin{itemize}
\item {Grp. gram.:f.}
\end{itemize}
\begin{itemize}
\item {Proveniência:(Gr. \textunderscore daphne\textunderscore , loireiro)}
\end{itemize}
Gênero de plantas, de fruto e casca medicinaes.
\section{Daphnephoria}
\begin{itemize}
\item {Grp. gram.:f.}
\end{itemize}
\begin{itemize}
\item {Proveniência:(Do gr. \textunderscore daphne\textunderscore  + \textunderscore phoros\textunderscore )}
\end{itemize}
Festa, que os Beócios celebravam, de nove em nove annos, em honra de Apollo.
\section{Daphnina}
\begin{itemize}
\item {Grp. gram.:f.}
\end{itemize}
\begin{itemize}
\item {Proveniência:(Do gr. \textunderscore daphne\textunderscore )}
\end{itemize}
Substância volátil da casca da \textunderscore daphne alpina\textunderscore .
\section{Daphnite}
\begin{itemize}
\item {Grp. gram.:f.}
\end{itemize}
\begin{itemize}
\item {Proveniência:(De \textunderscore daphne\textunderscore )}
\end{itemize}
Espécie de loireiro, o mesmo que \textunderscore espirradeira\textunderscore .
\section{Daphnoídeas}
\begin{itemize}
\item {Grp. gram.:f. pl.}
\end{itemize}
(V.daphnáceas)
\section{Daphnomancia}
\begin{itemize}
\item {Grp. gram.:f.}
\end{itemize}
\begin{itemize}
\item {Proveniência:(Do gr. \textunderscore daphne\textunderscore  + \textunderscore manteia\textunderscore )}
\end{itemize}
Adivinhação por meio de fôlhas de loireiro queimadas.
\section{Daphnomântico}
\begin{itemize}
\item {Grp. gram.:adj.}
\end{itemize}
Relativo á daphnomancia.
\section{Dapífero}
\begin{itemize}
\item {Grp. gram.:m.}
\end{itemize}
\begin{itemize}
\item {Utilização:Ant.}
\end{itemize}
\begin{itemize}
\item {Proveniência:(Lat. \textunderscore dapifer\textunderscore )}
\end{itemize}
Aquelle que servia á mesa.
\section{Dapno}
\begin{itemize}
\item {Grp. gram.:m.}
\end{itemize}
\begin{itemize}
\item {Utilização:Ant.}
\end{itemize}
O mesmo que \textunderscore damno\textunderscore .
\section{Daquele}
\begin{itemize}
\item {fónica:quê}
\end{itemize}
(Contr. da prep. \textunderscore de\textunderscore , e do pron. \textunderscore aquele\textunderscore )
\section{Daquelle}
\begin{itemize}
\item {fónica:quê}
\end{itemize}
(Contr. da prep. \textunderscore de\textunderscore , e do pron. \textunderscore aquelle\textunderscore )
\section{Daqui}
(contr. da prep. \textunderscore de\textunderscore  e do adv. \textunderscore aqui\textunderscore )
\section{Daquillo}
(Contr. da prep. \textunderscore de\textunderscore , e do pron. \textunderscore aquillo\textunderscore )
\section{Daquilo}
(Contr. da prep. \textunderscore de\textunderscore , e do pron. \textunderscore aquilo\textunderscore )
\section{Dar}
\begin{itemize}
\item {Grp. gram.:v. t.}
\end{itemize}
\begin{itemize}
\item {Grp. gram.:V. i.}
\end{itemize}
\begin{itemize}
\item {Grp. gram.:Loc.}
\end{itemize}
\begin{itemize}
\item {Utilização:fam.}
\end{itemize}
\begin{itemize}
\item {Grp. gram.:Loc.}
\end{itemize}
\begin{itemize}
\item {Utilização:fam.}
\end{itemize}
\begin{itemize}
\item {Grp. gram.:Loc.}
\end{itemize}
\begin{itemize}
\item {Utilização:fam.}
\end{itemize}
\begin{itemize}
\item {Grp. gram.:Loc.}
\end{itemize}
\begin{itemize}
\item {Utilização:pop.}
\end{itemize}
\begin{itemize}
\item {Grp. gram.:V. p.}
\end{itemize}
\begin{itemize}
\item {Proveniência:(Lat. \textunderscore dare\textunderscore )}
\end{itemize}
Ceder gratuitamente: \textunderscore dar um cavallo\textunderscore .
Entregar.
Transferir.
Fazer doação de, doar.
Presentear com.
Fazer esmola de: \textunderscore dar um vintém a um pobre\textunderscore .
Produzir: \textunderscore árvore, que dá frutos\textunderscore .
Applicar: \textunderscore dar corda a um relógio\textunderscore .
Destinar.
Realizar: \textunderscore dar passos\textunderscore .
Incumbir.
Conferir.
Prescrever.
Ministrar: \textunderscore dar medicamentos\textunderscore .
Consentir, permittir: \textunderscore foi-me dado que o procurasse\textunderscore .
Admittir, suppor: \textunderscore mas demos que seja assim\textunderscore .
Julgar.
Apresentar: \textunderscore dou-lhe os meus parabens\textunderscore .
Expor, proferir: \textunderscore dar sentença\textunderscore ; \textunderscore dar opinião\textunderscore .
Exhalar: \textunderscore dar o último suspiro\textunderscore .
Constituir.
Bater: \textunderscore esta mulher dá muito nos filhos\textunderscore .
Ir de encontro: \textunderscore a carruagem foi dar na parede\textunderscore .
Encontrar-se.
Bastar.
Manifestar-se, sobrevir: \textunderscore deu-lhe um ataque\textunderscore .
Defrontar.
Têr vontade; tomar resolução: \textunderscore deu-lhe para embirrar comigo\textunderscore .
Fazer reparo.
Descobrir alguém ou alguma coisa: \textunderscore fui dar com êlle escondido\textunderscore .
Attender.
Communicar movimento.
\textunderscore Dar-lhe para ali\textunderscore , resolver á tôa: \textunderscore deu-lhe para ali ir banhar-se\textunderscore .
\textunderscore Dar para tabaco\textunderscore , bater, sovar; repreender severamente.
\textunderscore Dar-lhe\textunderscore , tomar ou fazer alguma coisa em grande quantidade ou depressa: \textunderscore com os dias tão pequenos, é preciso dar-lhe, para se vêr o serviço\textunderscore .
\textunderscore Dar de si\textunderscore , soffrer abalo ou deslocação: \textunderscore aquella parede deu de si\textunderscore .
\textunderscore Dar ao rabo\textunderscore , agitá-lo; saracotear-se, andando.
\textunderscore Dar ás trancas\textunderscore , fugir.
\textunderscore Dar ás de villa-diogo\textunderscore , fugir.
\textunderscore Dar de corpo\textunderscore , defecar.
Accommodar-se; estar de acôrdo: \textunderscore dá-se bem comigo\textunderscore .
Viver, passar: \textunderscore dá-se mal no campo\textunderscore .
Viver em harmonía: \textunderscore não me dou com êlle\textunderscore .
Conviver.
Realizar-se: \textunderscore deram-se successos imprevistos\textunderscore .
Importar: \textunderscore não se me dá que asneies\textunderscore .
\section{Darandela}
\begin{itemize}
\item {Grp. gram.:f.}
\end{itemize}
\begin{itemize}
\item {Utilização:Des.}
\end{itemize}
Antigo estôfo.
(Corr. de \textunderscore durandela\textunderscore , do cast. \textunderscore durando\textunderscore )
\section{Darandina}
\begin{itemize}
\item {Grp. gram.:f.}
\end{itemize}
Azáfama; lufa-lufa.
\section{Dardada}
\begin{itemize}
\item {Grp. gram.:f.}
\end{itemize}
Tiro de dardo.
\section{Dardanário}
\begin{itemize}
\item {Grp. gram.:m.}
\end{itemize}
\begin{itemize}
\item {Utilização:Des.}
\end{itemize}
\begin{itemize}
\item {Proveniência:(Do lat. \textunderscore dardanarius\textunderscore )}
\end{itemize}
Monopolizador de mercadorias, para as vender por alto preço.
\section{Dardânio}
\begin{itemize}
\item {Grp. gram.:adj.}
\end{itemize}
\begin{itemize}
\item {Proveniência:(Lat. \textunderscore dardanius\textunderscore )}
\end{itemize}
O mesmo que \textunderscore troiano\textunderscore .
\section{Dardar}
\begin{itemize}
\item {Grp. gram.:v. t.}
\end{itemize}
\begin{itemize}
\item {Utilização:Fig.}
\end{itemize}
\begin{itemize}
\item {Proveniência:(De \textunderscore dardo\textunderscore )}
\end{itemize}
Ferir com dardo.
Pungir, affligir muito. Cf. Filinto, I, 238.
\section{Dardejamento}
\begin{itemize}
\item {Grp. gram.:m.}
\end{itemize}
Acto de dardejar.
\section{Dardejante}
\begin{itemize}
\item {Grp. gram.:adj.}
\end{itemize}
Que dardeja.
\section{Dardejar}
\begin{itemize}
\item {Grp. gram.:v. t.}
\end{itemize}
\begin{itemize}
\item {Grp. gram.:V. i.}
\end{itemize}
\begin{itemize}
\item {Utilização:Fig.}
\end{itemize}
\begin{itemize}
\item {Proveniência:(De \textunderscore dardo\textunderscore )}
\end{itemize}
Atirar dardos contra.
Vibrar.
Arremessar.
Expellir, lançar.
Atirar dardos.
Scintillar: \textunderscore os teus olhos dardejam\textunderscore .
\section{Dardo}
\begin{itemize}
\item {Grp. gram.:m.}
\end{itemize}
\begin{itemize}
\item {Utilização:Fig.}
\end{itemize}
\begin{itemize}
\item {Utilização:Fig.}
\end{itemize}
Pequena lança.
Pau, terminado em ponta de ferro, que se arremessa com a mão.
Aquillo que fere ou magôa.
Ferrão de alguns inséctos.
Língua da cobra.
Censura, dito mordaz.
(Ant. alt. al. \textunderscore dart\textunderscore )
\section{Dares}
\begin{itemize}
\item {Grp. gram.:m. pl.}
\end{itemize}
\begin{itemize}
\item {Proveniência:(De \textunderscore dar\textunderscore )}
\end{itemize}
\textunderscore Dares e tomares\textunderscore , altercação, contenda.
\section{Darga}
\begin{itemize}
\item {Grp. gram.:f.}
\end{itemize}
\begin{itemize}
\item {Utilização:Ant.}
\end{itemize}
O mesmo que \textunderscore adarga\textunderscore .
\section{Dariás}
\begin{itemize}
\item {Grp. gram.:m. pl.}
\end{itemize}
Uma das castas indígenas de Damão.
\section{Dárico}
\begin{itemize}
\item {Grp. gram.:adj.}
\end{itemize}
\begin{itemize}
\item {Proveniência:(De \textunderscore Dario\textunderscore , n. p.)}
\end{itemize}
Antiga moéda persa, que também teve curso entre os Hebreus.
\section{Darlingtónia}
\begin{itemize}
\item {Grp. gram.:f.}
\end{itemize}
\begin{itemize}
\item {Proveniência:(De \textunderscore Darlington\textunderscore , n. p.)}
\end{itemize}
Gênero de plantas mimóseas.
\section{Darmadeira}
\begin{itemize}
\item {Grp. gram.:f.}
\end{itemize}
Escantilhão, para medir o calibre das balas.
(Talvez de \textunderscore adarme\textunderscore )
\section{Daroês}
\begin{itemize}
\item {Grp. gram.:m.}
\end{itemize}
\begin{itemize}
\item {Proveniência:(Do ár. \textunderscore daruix\textunderscore )}
\end{itemize}
Espécie de monge mahometano. Cf. Barros, \textunderscore Déc.\textunderscore , (\textunderscore passim\textunderscore ).--Corresponde ao estrangeirismo \textunderscore derviche\textunderscore .
\section{Darona}
\begin{itemize}
\item {Grp. gram.:f.}
\end{itemize}
\begin{itemize}
\item {Utilização:Gír.}
\end{itemize}
\begin{itemize}
\item {Proveniência:(De \textunderscore dar\textunderscore ?)}
\end{itemize}
Mãe.
\section{Darto}
\begin{itemize}
\item {Grp. gram.:m.}
\end{itemize}
O mesmo que \textunderscore dartro\textunderscore .
\section{Dartos}
\begin{itemize}
\item {Grp. gram.:m. pl.}
\end{itemize}
\begin{itemize}
\item {Proveniência:(Gr. \textunderscore dartos\textunderscore )}
\end{itemize}
A segunda membrana, que envolve os testículos.
\section{Dartoso}
\begin{itemize}
\item {Grp. gram.:adj.}
\end{itemize}
\begin{itemize}
\item {Proveniência:(De \textunderscore dartro\textunderscore )}
\end{itemize}
Que tem dartros; herpético.
\section{Dartro}
\begin{itemize}
\item {Grp. gram.:m.}
\end{itemize}
Impigem; herpes.
(Talvez do celt.)
\section{Dartroso}
\begin{itemize}
\item {Grp. gram.:adj.}
\end{itemize}
\begin{itemize}
\item {Proveniência:(De \textunderscore dartro\textunderscore )}
\end{itemize}
Que tem dartros; herpético.
\section{Daruês}
\begin{itemize}
\item {Grp. gram.:m.}
\end{itemize}
\begin{itemize}
\item {Proveniência:(Do ár. \textunderscore daruix\textunderscore )}
\end{itemize}
Espécie de monge mahometano. Cf. Barros, \textunderscore Déc.\textunderscore , (\textunderscore passim\textunderscore ).--Corresponde ao estrangeirismo \textunderscore derviche\textunderscore .
\section{Darwiniano}
\begin{itemize}
\item {fónica:dá-ru-í-ni-â-no}
\end{itemize}
\begin{itemize}
\item {Grp. gram.:adj.}
\end{itemize}
Relativo a Darwin ou á sua doutrina.
\section{Darwinismo}
\begin{itemize}
\item {fónica:dá-ru-i-nís-mo}
\end{itemize}
\begin{itemize}
\item {Grp. gram.:m.}
\end{itemize}
\begin{itemize}
\item {Proveniência:(De \textunderscore Darwin\textunderscore , n. p.)}
\end{itemize}
Systema de história natural, cuja conclusão extrema é o parentesco physiológico e a communhão de origem em todos os seres vivos.
\section{Darwinista}
\begin{itemize}
\item {fónica:da-ru-i}
\end{itemize}
\begin{itemize}
\item {Grp. gram.:m.}
\end{itemize}
Sectário do darwinismo.
\section{Dasimetria}
\begin{itemize}
\item {Grp. gram.:f.}
\end{itemize}
\begin{itemize}
\item {Proveniência:(De \textunderscore dasímetro\textunderscore )}
\end{itemize}
Medida da intensidade do ar atmosphérico.
\section{Dasímetro}
\begin{itemize}
\item {Grp. gram.:m.}
\end{itemize}
\begin{itemize}
\item {Proveniência:(Do gr. \textunderscore dasus\textunderscore  + \textunderscore metron\textunderscore )}
\end{itemize}
Instrumento, para medir a intensidade do ar atmosphérico.
\section{Dásipo}
\begin{itemize}
\item {Grp. gram.:m.}
\end{itemize}
\begin{itemize}
\item {Proveniência:(Gr. \textunderscore dasupous\textunderscore )}
\end{itemize}
Designação scientífica de algumas espécies de tatus.
\section{Dasiúro}
\begin{itemize}
\item {Grp. gram.:m.}
\end{itemize}
\begin{itemize}
\item {Proveniência:(Do gr. \textunderscore dasus\textunderscore  + \textunderscore oura\textunderscore )}
\end{itemize}
Quadrúpede marsupial.
\section{Dastur}
\begin{itemize}
\item {Grp. gram.:m.}
\end{itemize}
Chefe dos sacerdotes parses.
\section{Dasymetria}
\begin{itemize}
\item {Proveniência:(De \textunderscore dasímetro\textunderscore )}
\end{itemize}
Medida da intensidade do ar atmosphérico.
\section{Dasýmetro}
\begin{itemize}
\item {Grp. gram.:m.}
\end{itemize}
\begin{itemize}
\item {Proveniência:(Do gr. \textunderscore dasus\textunderscore  + \textunderscore metron\textunderscore )}
\end{itemize}
Instrumento, para medir a intensidade do ar atmosphérico.
\section{Dásypo}
\begin{itemize}
\item {Grp. gram.:m.}
\end{itemize}
\begin{itemize}
\item {Proveniência:(Gr. \textunderscore dasupous\textunderscore )}
\end{itemize}
Designação scientífica de algumas espécies de tatus.
\section{Dasyúro}
\begin{itemize}
\item {Grp. gram.:m.}
\end{itemize}
\begin{itemize}
\item {Proveniência:(Do gr. \textunderscore dasus\textunderscore  + \textunderscore oura\textunderscore )}
\end{itemize}
Quadrúpede marsupial.
\section{Data}
\begin{itemize}
\item {Grp. gram.:f.}
\end{itemize}
\begin{itemize}
\item {Utilização:Fam.}
\end{itemize}
\begin{itemize}
\item {Utilização:Bras. de Minas}
\end{itemize}
\begin{itemize}
\item {Proveniência:(Lat. \textunderscore data\textunderscore , fem. de \textunderscore datus\textunderscore , de \textunderscore dare\textunderscore )}
\end{itemize}
Indicação da época, anno, mês ou dia, em que se realizou um facto.
Acto de dar ou bater: \textunderscore apanhou uma data\textunderscore .
Dose.
Grande porção: \textunderscore uma data de pancadas\textunderscore .
Porção de terreno: \textunderscore comprei uma data para horta\textunderscore .
\section{Data-de-pão}
\begin{itemize}
\item {Grp. gram.:f.  Loc.}
\end{itemize}
\begin{itemize}
\item {Utilização:Loc. de Alcácer.}
\end{itemize}
Vinte e um pães.
\section{Datal}
\begin{itemize}
\item {Grp. gram.:adj.}
\end{itemize}
Relativo a data. Cf. Cortesão, \textunderscore Subs\textunderscore .
\section{Datar}
\begin{itemize}
\item {Grp. gram.:v. t.}
\end{itemize}
\begin{itemize}
\item {Grp. gram.:V. i.}
\end{itemize}
Indicar a data de.
Pôr data em: \textunderscore datar uma carta\textunderscore .
Têr principio.
Contar-se (desde certo tempo): \textunderscore o gongorismo data do século XVII\textunderscore .
\section{Dataria}
\begin{itemize}
\item {Grp. gram.:f.}
\end{itemize}
\begin{itemize}
\item {Proveniência:(Do lat. \textunderscore datarius\textunderscore )}
\end{itemize}
Tribunal pontifício, por onde correm os negócios, relativos a graças concedidas pela Cúria romana.
\section{Datário}
\begin{itemize}
\item {Grp. gram.:m.}
\end{itemize}
\begin{itemize}
\item {Proveniência:(Lat. \textunderscore datarius\textunderscore )}
\end{itemize}
Presidente da dataria.
\section{Datil}
\begin{itemize}
\item {Grp. gram.:m.}
\end{itemize}
Fruto de certa palmeira. Cf. \textunderscore Peregrinação\textunderscore , IV.
\section{Datisca}
\begin{itemize}
\item {Grp. gram.:f.}
\end{itemize}
Árvore medicinal, que serve de typo ás datísceas, e que, segundo outros, é capparídea. Cf. Caminhoá, \textunderscore Botán. Geral e Med.\textunderscore ; dr. J. Henriques, \textunderscore Plantas do Jardim Botânico\textunderscore .
\section{Datiscáceas}
\begin{itemize}
\item {Grp. gram.:f. pl.}
\end{itemize}
O mesmo ou melhor que \textunderscore datísceas\textunderscore .
\section{Datísceas}
\begin{itemize}
\item {Grp. gram.:f. pl.}
\end{itemize}
\begin{itemize}
\item {Proveniência:(De \textunderscore datisceo\textunderscore )}
\end{itemize}
Família de plantas annuaes ou arbóreas, de que há três gêneros e seis espécies.
\section{Datisceo}
\begin{itemize}
\item {Grp. gram.:adj.}
\end{itemize}
Relativo ou semelhante á datísca.
\section{Datiscina}
\begin{itemize}
\item {Grp. gram.:f.}
\end{itemize}
Matéria còrante da datisca.
\section{Dativo}
\begin{itemize}
\item {Grp. gram.:adj.}
\end{itemize}
\begin{itemize}
\item {Grp. gram.:M.}
\end{itemize}
\begin{itemize}
\item {Proveniência:(Lat. \textunderscore dativus\textunderscore )}
\end{itemize}
Nomeado por magistrado e não por lei: \textunderscore tutor dativo\textunderscore .
Em gram. gr. e lat., caso que exprime a relação de complemento terminativo.
\section{Datô}
\begin{itemize}
\item {Grp. gram.:m.}
\end{itemize}
\begin{itemize}
\item {Grp. gram.:Pl.}
\end{itemize}
Chefe de um suco ou de uma reunião de aldeias, em Timor.
Primeira classe social, entre os indígenas de Timor.
Espécie de sacerdote malaio? Cf. \textunderscore Peregrinação\textunderscore , XXXI.
\section{Datura}
\begin{itemize}
\item {Grp. gram.:f.}
\end{itemize}
\begin{itemize}
\item {Proveniência:(Lat. \textunderscore datura\textunderscore )}
\end{itemize}
Planta solânea.
\section{Daturáceas}
\begin{itemize}
\item {Grp. gram.:f. pl.}
\end{itemize}
\begin{itemize}
\item {Proveniência:(De \textunderscore datura\textunderscore )}
\end{itemize}
Família de plantas, de propriedades análogas ás das solâneas.
\section{Daturina}
\begin{itemize}
\item {Grp. gram.:f.}
\end{itemize}
Alcaloide das sementes da datura.
\section{Daucíneas}
\begin{itemize}
\item {Grp. gram.:f. pl.}
\end{itemize}
\begin{itemize}
\item {Proveniência:(Do lat. \textunderscore daucus\textunderscore )}
\end{itemize}
Família de plantas, que tem por typo a cenoira.
\section{Dauma}
\begin{itemize}
\item {Grp. gram.:m.}
\end{itemize}
Espécie de melro da Índia.
\section{Davália}
\begin{itemize}
\item {Grp. gram.:f.}
\end{itemize}
\begin{itemize}
\item {Proveniência:(De \textunderscore Davall\textunderscore , n. p.)}
\end{itemize}
Gênero de fêtos, cuja cepa é formada de rizomas escamosos, que rastejam pelo solo.
\section{Davállia}
\begin{itemize}
\item {Grp. gram.:f.}
\end{itemize}
\begin{itemize}
\item {Proveniência:(De \textunderscore Davall\textunderscore , n. p.)}
\end{itemize}
Gênero de fêtos, cuja cepa é formada de rhizomas escamosos, que rastejam pelo solo.
\section{Davandito}
\begin{itemize}
\item {Grp. gram.:adj.}
\end{itemize}
\begin{itemize}
\item {Utilização:Ant.}
\end{itemize}
O mesmo que \textunderscore sobredito\textunderscore .
(Condensação de \textunderscore de\textunderscore  + \textunderscore avante\textunderscore  + \textunderscore dito\textunderscore )
\section{Daves}
\begin{itemize}
\item {Grp. gram.:m.}
\end{itemize}
Bugio da Serra-Leôa.
\section{Dávida}
\begin{itemize}
\item {Grp. gram.:f.}
\end{itemize}
\begin{itemize}
\item {Utilização:Pop.}
\end{itemize}
O mesmo que \textunderscore dádiva\textunderscore .
(Metáth. de \textunderscore dádiva\textunderscore )
\section{Dazilírio}
\begin{itemize}
\item {Grp. gram.:m.}
\end{itemize}
Nome de uma planta, de que há exemplares no Bussaco.
\section{De}
\begin{itemize}
\item {Grp. gram.:prep.}
\end{itemize}
\begin{itemize}
\item {Proveniência:(Lat. \textunderscore de\textunderscore )}
\end{itemize}
Exprime restricção da palavra que a precede: \textunderscore filho de Pedro\textunderscore .
Ligação dos verbos auxiliares com o infinito de outros verbos: \textunderscore têr de fugir\textunderscore .
Relação de movimento, proveniência, propriedade, carácter, profissão.
Estado.
Naturalidade: \textunderscore sou de Portugal\textunderscore .
Situação, causa: \textunderscore morrer de febre amarela\textunderscore .
Mudança: \textunderscore saiu de Lisbôa\textunderscore .
Conformidade.
Formação: \textunderscore estátua de pedra\textunderscore .
Instrumento: \textunderscore a golpes de faca\textunderscore .
Dimensão: \textunderscore um metro de altura\textunderscore .
Modo; e ás vezes é elemento de loc. prep. e adv.: \textunderscore resolveu de per si\textunderscore .
\section{De...}
\begin{itemize}
\item {Grp. gram.:pref.}
\end{itemize}
\begin{itemize}
\item {Proveniência:(Lat. \textunderscore de\textunderscore )}
\end{itemize}
(indicativo de procedência, privação, conclusão, o contrário do que significa o rad., etc.)
\section{Dê}
\begin{itemize}
\item {Grp. gram.:prep.}
\end{itemize}
\begin{itemize}
\item {Utilização:ant.}
\end{itemize}
O mesmo que \textunderscore desde\textunderscore .
\section{De baixo}
\begin{itemize}
\item {Grp. gram.:adv.}
\end{itemize}
\begin{itemize}
\item {Grp. gram.:Loc. prep.}
\end{itemize}
\begin{itemize}
\item {Proveniência:(De \textunderscore de...\textunderscore  + \textunderscore baixo\textunderscore )}
\end{itemize}
Inferiormente.
Por baixo.
Subordinadamente.
Em decadência.
\textunderscore Debaixo de\textunderscore , em lugar inferior e encoberto por.
Em dependência de; em consequência de.
\section{De balde}
\begin{itemize}
\item {Grp. gram.:adv.}
\end{itemize}
Inutilmente.
Em vão; em-balde; baldadamente.
(Cp. \textunderscore balde\textunderscore  e \textunderscore baldo\textunderscore )
\section{Deado}
\begin{itemize}
\item {Grp. gram.:m.}
\end{itemize}
Cargo, funcções ou dignidade de deão.
\section{Dealbação}
\begin{itemize}
\item {Grp. gram.:f.}
\end{itemize}
Acto ou effeito de dealbar.
\section{Dealbar}
\begin{itemize}
\item {Grp. gram.:v. t.}
\end{itemize}
\begin{itemize}
\item {Proveniência:(Lat. \textunderscore dealbare\textunderscore )}
\end{itemize}
Tornar branco, branquear.
\section{Dealvar}
\textunderscore v. t.\textunderscore  (e der.)
(V. \textunderscore dealbar\textunderscore , etc.)
\section{Deambulação}
\begin{itemize}
\item {Grp. gram.:f.}
\end{itemize}
\begin{itemize}
\item {Utilização:Des.}
\end{itemize}
\begin{itemize}
\item {Proveniência:(Lat. \textunderscore deambulatio\textunderscore )}
\end{itemize}
Passeio.
\section{Deambular}
\begin{itemize}
\item {Grp. gram.:v. i.}
\end{itemize}
\begin{itemize}
\item {Proveniência:(Lat. \textunderscore deambulare\textunderscore )}
\end{itemize}
Passear, vaguear.
\section{Deambulatório}
\begin{itemize}
\item {Grp. gram.:m.}
\end{itemize}
\begin{itemize}
\item {Grp. gram.:Adj.}
\end{itemize}
\begin{itemize}
\item {Utilização:Fig.}
\end{itemize}
\begin{itemize}
\item {Proveniência:(Lat. \textunderscore deambulatorium\textunderscore )}
\end{itemize}
Galeria coberta, para nella se passear.
Relativo a passeio.
Erradio, desnorteado.
\section{Dean}
\begin{itemize}
\item {Grp. gram.:f.}
\end{itemize}
A mais velha das vestaes. Cf. Castilho, \textunderscore Fastos\textunderscore , II, p. 125.
(Cp. \textunderscore deão\textunderscore )
\section{Deante}
\begin{itemize}
\item {Grp. gram.:Prep.}
\end{itemize}
\begin{itemize}
\item {Proveniência:(De \textunderscore de\textunderscore  + \textunderscore ante\textunderscore )}
\end{itemize}
Em primeiro lugar: \textunderscore no rol dos doidos, o Alexandre é quem está adeante\textunderscore .
Em frente, defronte: \textunderscore ficava-nos adeante o casal\textunderscore .
Ante; perante:«\textunderscore deante as nossas pastoras\textunderscore ». R. Lobo, \textunderscore Côrte na Aldeia\textunderscore .
\textunderscore Ir por deante\textunderscore , continuar:«\textunderscore mais ia por deante o monstro horrendo\textunderscore ». \textunderscore Lusíadas\textunderscore .
\textunderscore Para deante\textunderscore , para a frente.
\section{Deanteira}
\begin{itemize}
\item {Grp. gram.:f.}
\end{itemize}
\begin{itemize}
\item {Proveniência:(De \textunderscore deanteiro\textunderscore )}
\end{itemize}
Frente; vanguarda: \textunderscore na deanteira das tropas\textunderscore .
\section{Deanteiro}
\begin{itemize}
\item {Grp. gram.:adj.}
\end{itemize}
\begin{itemize}
\item {Proveniência:(De \textunderscore deante\textunderscore )}
\end{itemize}
Que está ou vai adeante ou em primeiro lugar: \textunderscore o boi deanteiro\textunderscore .
\section{De-antemão}
\begin{itemize}
\item {Grp. gram.:adv.}
\end{itemize}
Previamente.
\section{Deão}
\begin{itemize}
\item {Grp. gram.:m.}
\end{itemize}
Dignitário ecclesiástico, que preside ao cabido.
(A fórma exacta seria \textunderscore dião\textunderscore , contr. de \textunderscore deião\textunderscore , do lat. \textunderscore decanus\textunderscore )
\section{De-arrancada}
\begin{itemize}
\item {Grp. gram.:adv.}
\end{itemize}
De-repente; com ímpeto.
\section{Dearticulação}
\begin{itemize}
\item {Grp. gram.:f.}
\end{itemize}
Acto de dearticular.
\section{Dearticular}
\begin{itemize}
\item {Grp. gram.:v. t.}
\end{itemize}
\begin{itemize}
\item {Proveniência:(De \textunderscore de...\textunderscore  + \textunderscore articular\textunderscore )}
\end{itemize}
Pronunciar com clareza.
\section{Deaurar}
\begin{itemize}
\item {Grp. gram.:v. t.}
\end{itemize}
\begin{itemize}
\item {Utilização:Des.}
\end{itemize}
Cobrir de oiro; doirar.
(B. lat. \textunderscore deaurare\textunderscore )
\section{Debacar-se}
\begin{itemize}
\item {Grp. gram.:v. p.}
\end{itemize}
\begin{itemize}
\item {Proveniência:(Lat. \textunderscore debacchari\textunderscore )}
\end{itemize}
Praticar actos de ébrio.
Irritar-se extremamente.
Enfurecer-se.
\section{Debacchar-se}
\begin{itemize}
\item {fónica:car}
\end{itemize}
\begin{itemize}
\item {Grp. gram.:v. p.}
\end{itemize}
\begin{itemize}
\item {Proveniência:(Lat. \textunderscore debacchari\textunderscore )}
\end{itemize}
Praticar actos de ébrio.
Irritar-se extremamente.
Enfurecer-se.
\section{Debagar}
\begin{itemize}
\item {Grp. gram.:v. t.}
\end{itemize}
\begin{itemize}
\item {Utilização:Prov.}
\end{itemize}
\begin{itemize}
\item {Utilização:trasm.}
\end{itemize}
\begin{itemize}
\item {Grp. gram.:V. p.}
\end{itemize}
\begin{itemize}
\item {Proveniência:(De \textunderscore bago\textunderscore )}
\end{itemize}
O mesmo que \textunderscore esbagoar\textunderscore  ou \textunderscore debulhar\textunderscore .
Cair em grande quantidade, (falando-se da chuva).
\section{Debaixar}
\begin{itemize}
\item {Grp. gram.:v. t.}
\end{itemize}
\begin{itemize}
\item {Utilização:Prov.}
\end{itemize}
\begin{itemize}
\item {Utilização:minh.}
\end{itemize}
Dar a primeira espadelada em (o linho).
(Colhido em Barcelos)
\section{Debaixo}
\begin{itemize}
\item {Grp. gram.:adv.}
\end{itemize}
\begin{itemize}
\item {Grp. gram.:Loc. prep.}
\end{itemize}
\begin{itemize}
\item {Proveniência:(De \textunderscore de...\textunderscore  + \textunderscore baixo\textunderscore )}
\end{itemize}
Inferiormente.
Por baixo.
Subordinadamente.
Em decadência.
\textunderscore Debaixo de\textunderscore , em lugar inferior e encoberto por.
Em dependência de; em consequência de.
\section{Debalde}
\begin{itemize}
\item {Grp. gram.:adv.}
\end{itemize}
Inutilmente.
Em vão; em-balde; baldadamente.
(Cp. \textunderscore balde\textunderscore  e \textunderscore baldo\textunderscore )
\section{Debanar}
\begin{itemize}
\item {Grp. gram.:v. t.}
\end{itemize}
\begin{itemize}
\item {Utilização:Prov.}
\end{itemize}
\begin{itemize}
\item {Utilização:trasm.}
\end{itemize}
O mesmo que \textunderscore dobar\textunderscore .
(Cf. cast. \textunderscore devanar\textunderscore )
\section{Debandada}
\begin{itemize}
\item {Grp. gram.:f.}
\end{itemize}
Acto e effeito de debandar; sinal para debandar.
\section{Debandar}
\begin{itemize}
\item {Grp. gram.:v. t.}
\end{itemize}
\begin{itemize}
\item {Grp. gram.:V. i.  e  p.}
\end{itemize}
\begin{itemize}
\item {Proveniência:(De \textunderscore de...\textunderscore  + \textunderscore bando\textunderscore )}
\end{itemize}
Pôr em fuga desordenadamente, confusamente.
Sair da ordem, desordenar-se.
Pôr-se em fuga desordenadamente.
Fugir; dispersar-se: \textunderscore as perdizes debandaram\textunderscore .
\section{Debangar}
\begin{itemize}
\item {Grp. gram.:v. t.}
\end{itemize}
\begin{itemize}
\item {Utilização:Prov.}
\end{itemize}
\begin{itemize}
\item {Utilização:minh.}
\end{itemize}
Dizer ou expôr em abundância: \textunderscore aquelle falou muito, mas não fez senão debangar mentiras\textunderscore .
(Cp. \textunderscore debagar\textunderscore )
\section{Debate}
\begin{itemize}
\item {Grp. gram.:m.}
\end{itemize}
\begin{itemize}
\item {Proveniência:(De \textunderscore debater\textunderscore )}
\end{itemize}
Discussão, altercação.
Contestação.
\section{Debater}
\begin{itemize}
\item {Grp. gram.:v. t.}
\end{itemize}
\begin{itemize}
\item {Grp. gram.:V. p.}
\end{itemize}
\begin{itemize}
\item {Proveniência:(De \textunderscore de...\textunderscore  + \textunderscore bater\textunderscore )}
\end{itemize}
Discutir.
Contestar.
Agitar-se, resistindo ou procurando soltar-se.
\section{Debatidiço}
\begin{itemize}
\item {Grp. gram.:adj.}
\end{itemize}
Que se debate muito.
\section{Debatidura}
\begin{itemize}
\item {Grp. gram.:f.}
\end{itemize}
Acto de se debater, para fugir, (falando-se de aves presas). Cf. Fern. Pereira, \textunderscore Caça de Altan.\textunderscore , p. II, c. 3.
\section{Debatimento}
\begin{itemize}
\item {Grp. gram.:m.}
\end{itemize}
\begin{itemize}
\item {Utilização:P. us.}
\end{itemize}
O mesmo que \textunderscore debate\textunderscore . Cf. Filinto, \textunderscore D. Manuel\textunderscore , I, p. 235.
\section{Debelação}
\begin{itemize}
\item {Grp. gram.:f.}
\end{itemize}
Acto ou efeito de debelar.
\section{Debelador}
\begin{itemize}
\item {Grp. gram.:m.}
\end{itemize}
\begin{itemize}
\item {Proveniência:(Lat. \textunderscore debellator\textunderscore )}
\end{itemize}
Aquele que debela.
\section{Debelar}
\begin{itemize}
\item {Grp. gram.:v. t.}
\end{itemize}
\begin{itemize}
\item {Proveniência:(Lat. \textunderscore debellare\textunderscore )}
\end{itemize}
Sujeitar, vencer.
Destruir, reprimir, combater, extinguir.
\section{Debelatório}
\begin{itemize}
\item {Grp. gram.:adj.}
\end{itemize}
Que debela.
\section{Debellação}
\begin{itemize}
\item {Grp. gram.:f.}
\end{itemize}
Acto ou effeito de debellar.
\section{Debellador}
\begin{itemize}
\item {Grp. gram.:m.}
\end{itemize}
\begin{itemize}
\item {Proveniência:(Lat. \textunderscore debellator\textunderscore )}
\end{itemize}
Aquelle que debella.
\section{Debellar}
\begin{itemize}
\item {Grp. gram.:v. t.}
\end{itemize}
\begin{itemize}
\item {Proveniência:(Lat. \textunderscore debellare\textunderscore )}
\end{itemize}
Sujeitar, vencer.
Destruir, reprimir, combater, extinguir.
\section{Debellatório}
\begin{itemize}
\item {Grp. gram.:adj.}
\end{itemize}
Que debella.
\section{Debentura}
\begin{itemize}
\item {Grp. gram.:f.}
\end{itemize}
\begin{itemize}
\item {Utilização:bras}
\end{itemize}
\begin{itemize}
\item {Utilização:Neol.}
\end{itemize}
\begin{itemize}
\item {Proveniência:(Ingl. \textunderscore debenture\textunderscore )}
\end{itemize}
Obrigação, (título).
\section{Debenturista}
\begin{itemize}
\item {Grp. gram.:m.}
\end{itemize}
\begin{itemize}
\item {Utilização:Bras}
\end{itemize}
Possuidor de debenturas.
\section{Debicador}
\begin{itemize}
\item {Grp. gram.:adj.}
\end{itemize}
Que debica.
\section{Debicar}
\begin{itemize}
\item {Grp. gram.:v. i.}
\end{itemize}
\begin{itemize}
\item {Utilização:Fig.}
\end{itemize}
\begin{itemize}
\item {Proveniência:(De \textunderscore de...\textunderscore  + \textunderscore bico\textunderscore )}
\end{itemize}
Picar alguma coisa com o bico: \textunderscore as aves debicam nas cerejas\textunderscore .
Comer pequena quantidade de uma coisa.
Comer pouco.
Motejar, provocar alguém com gracejos.
\section{Débil}
\begin{itemize}
\item {Grp. gram.:adj.}
\end{itemize}
\begin{itemize}
\item {Proveniência:(Lat. \textunderscore debilis\textunderscore , contr. de \textunderscore de\textunderscore  + \textunderscore habilis\textunderscore )}
\end{itemize}
Fraco; pusillânime.
Frouxo; pouco perceptível: \textunderscore sons débeis\textunderscore .
Insignificante: \textunderscore resultado débil\textunderscore .
\section{Debilicar}
\begin{itemize}
\item {Grp. gram.:v. t.}
\end{itemize}
\begin{itemize}
\item {Utilização:Ant.}
\end{itemize}
O mesmo que \textunderscore debilitar\textunderscore :«\textunderscore tanto amavão o seu rey, que se elle era debilicado dalgum membro, elles per si se aleijavão doutro\textunderscore ». J. Barros, \textunderscore Espelho de Casados\textunderscore .
\section{Debilidade}
\begin{itemize}
\item {Grp. gram.:f.}
\end{itemize}
\begin{itemize}
\item {Proveniência:(Lat. \textunderscore debilitas\textunderscore )}
\end{itemize}
Qualidade de quem ou daquillo que é débil.
Enfraquecimento.
Prostração de forças.
\section{Debilitação}
\begin{itemize}
\item {Grp. gram.:f.}
\end{itemize}
\begin{itemize}
\item {Proveniência:(Lat. \textunderscore debilitatio\textunderscore )}
\end{itemize}
Effeito de debilitar.
\section{Debilitador}
\begin{itemize}
\item {Grp. gram.:adj.}
\end{itemize}
O mesmo que \textunderscore debilitante\textunderscore .
\section{Debilitamento}
\begin{itemize}
\item {Grp. gram.:m.}
\end{itemize}
O mesmo que \textunderscore debilitação\textunderscore .
\section{Debilitante}
\begin{itemize}
\item {Grp. gram.:adj.}
\end{itemize}
\begin{itemize}
\item {Proveniência:(Lat. \textunderscore debilitans\textunderscore )}
\end{itemize}
Que debilita.
\section{Debilitar}
\begin{itemize}
\item {Grp. gram.:v. t.}
\end{itemize}
\begin{itemize}
\item {Proveniência:(Lat. \textunderscore debilitare\textunderscore )}
\end{itemize}
Tornar débil.
Tirar fôrças a.
Enfraquecer: \textunderscore o trabalho debilitou-o\textunderscore .
\section{Debilitável}
\begin{itemize}
\item {Grp. gram.:adj.}
\end{itemize}
Que se póde debilitar.
\section{Debilmente}
\begin{itemize}
\item {Grp. gram.:adv.}
\end{itemize}
De modo débil.
\section{Debique}
\begin{itemize}
\item {Grp. gram.:m.}
\end{itemize}
Acto de debicar.
Desfrute; troça.
\section{Debiqueiro}
\begin{itemize}
\item {Grp. gram.:adj.}
\end{itemize}
\begin{itemize}
\item {Utilização:Fam.}
\end{itemize}
\begin{itemize}
\item {Proveniência:(De \textunderscore debicar\textunderscore )}
\end{itemize}
Que debica; que come pouco.
Fastiento; biqueiro.
\section{Debitar}
\begin{itemize}
\item {Grp. gram.:v. t.}
\end{itemize}
\begin{itemize}
\item {Proveniência:(De \textunderscore débito\textunderscore )}
\end{itemize}
Constituir ou inscrever, como devedor.
\section{Débito}
\begin{itemize}
\item {Grp. gram.:m.}
\end{itemize}
\begin{itemize}
\item {Proveniência:(Lat. \textunderscore debitus\textunderscore )}
\end{itemize}
Aquillo que se deve.
Dívida.
Parte de uma conta commercial, onde se regista o que se fornece ou se paga.
Quantidade de água, fornecida por uma corrente ou fonte em uma unidade de tempo.
\section{Deblaterar}
\begin{itemize}
\item {Grp. gram.:v. i.}
\end{itemize}
\begin{itemize}
\item {Utilização:P. us.}
\end{itemize}
\begin{itemize}
\item {Proveniência:(Lat. \textunderscore deblaterare\textunderscore )}
\end{itemize}
Gritar ou declamar.
Barafustar contra alguém.
\section{Debloquear}
\begin{itemize}
\item {Grp. gram.:v. t.}
\end{itemize}
\begin{itemize}
\item {Proveniência:(De \textunderscore de...\textunderscore  + \textunderscore bloquear\textunderscore )}
\end{itemize}
Deixar de bloquear. Cf. Filinto, XXI, p. 115.
\section{Debo}
\begin{itemize}
\item {fónica:dê}
\end{itemize}
\begin{itemize}
\item {Grp. gram.:m.}
\end{itemize}
Casta de uva preta.
\section{Debochado}
\begin{itemize}
\item {Grp. gram.:adj.}
\end{itemize}
(Gal. inútil, por \textunderscore devasso\textunderscore , \textunderscore libertino\textunderscore , \textunderscore corrupto\textunderscore , \textunderscore extravagante\textunderscore )
\section{Debochar}
\begin{itemize}
\item {Grp. gram.:v. t.}
\end{itemize}
\begin{itemize}
\item {Proveniência:(Fr. \textunderscore débaucher\textunderscore )}
\end{itemize}
(Gal. inútil, por \textunderscore tornar devasso\textunderscore , \textunderscore prostituir\textunderscore , etc.)
\section{Deboche}
\begin{itemize}
\item {Grp. gram.:m.}
\end{itemize}
\begin{itemize}
\item {Proveniência:(Fr. \textunderscore débauche\textunderscore )}
\end{itemize}
(Gal. inútil, por \textunderscore libertinagem\textunderscore , \textunderscore devassidão\textunderscore , \textunderscore estroinice\textunderscore , etc.)
\section{Deboiçar}
\begin{itemize}
\item {Grp. gram.:v. t.}
\end{itemize}
\begin{itemize}
\item {Utilização:Prov.}
\end{itemize}
\begin{itemize}
\item {Utilização:minh.}
\end{itemize}
\begin{itemize}
\item {Proveniência:(De \textunderscore boiça\textunderscore )}
\end{itemize}
O mesmo que \textunderscore desbravar\textunderscore .
\section{De-boinas-a-boinas}
\begin{itemize}
\item {Grp. gram.:loc. adv.}
\end{itemize}
\begin{itemize}
\item {Utilização:Prov.}
\end{itemize}
\begin{itemize}
\item {Utilização:alg.}
\end{itemize}
De bom a melhor.
De vento em popa.
(Corr. da loc. lat. \textunderscore de bonis ad bona\textunderscore ?)
\section{Deborcar}
\begin{itemize}
\item {Grp. gram.:v. t.}
\end{itemize}
\begin{itemize}
\item {Proveniência:(De \textunderscore bôrco\textunderscore )}
\end{itemize}
Virar de bôrco. Cf. \textunderscore Technologia Rural\textunderscore , 102.
\section{Debotar}
\begin{itemize}
\item {Grp. gram.:v. i.}
\end{itemize}
O mesmo que \textunderscore desbotar\textunderscore .
\section{Debouçar}
\begin{itemize}
\item {Grp. gram.:v. t.}
\end{itemize}
\begin{itemize}
\item {Utilização:Prov.}
\end{itemize}
\begin{itemize}
\item {Utilização:trasm.}
\end{itemize}
Dar a primeira espadelada em (o linho).
(Gall. \textunderscore debouzar\textunderscore )
\section{Debouçar}
\begin{itemize}
\item {Grp. gram.:v. t.}
\end{itemize}
\begin{itemize}
\item {Utilização:Prov.}
\end{itemize}
\begin{itemize}
\item {Utilização:minh.}
\end{itemize}
\begin{itemize}
\item {Proveniência:(De \textunderscore boiça\textunderscore )}
\end{itemize}
O mesmo que \textunderscore desbravar\textunderscore .
\section{Debrear}
\begin{itemize}
\item {Grp. gram.:v. t.}
\end{itemize}
\begin{itemize}
\item {Utilização:Prov.}
\end{itemize}
\begin{itemize}
\item {Utilização:trasm.}
\end{itemize}
\begin{itemize}
\item {Utilização:beir.}
\end{itemize}
Moer com pancadas.
(Talvez de \textunderscore breu\textunderscore , por allusão ás nódoas que as pancadas podem produzir)
\section{Debruadeira}
\begin{itemize}
\item {Grp. gram.:f.}
\end{itemize}
Mulher, que trabalha em debruns.
\section{Debruar}
\begin{itemize}
\item {Grp. gram.:v. t.}
\end{itemize}
\begin{itemize}
\item {Utilização:Prov.}
\end{itemize}
\begin{itemize}
\item {Utilização:beir.}
\end{itemize}
\begin{itemize}
\item {Utilização:Fig.}
\end{itemize}
Guarnecer com debrum.
Orlar com friso (uma tábua).
Ornar.
\section{Debruçar}
\begin{itemize}
\item {Grp. gram.:v. t.}
\end{itemize}
\begin{itemize}
\item {Proveniência:(De \textunderscore de-bruços\textunderscore )}
\end{itemize}
Pôr de-bruços.
Tornar inclinado.
\section{De-bruços}
\begin{itemize}
\item {Grp. gram.:loc. adv.}
\end{itemize}
(Cp. \textunderscore bruços\textunderscore )
\section{Debrum}
\begin{itemize}
\item {Grp. gram.:m.}
\end{itemize}
Fita ou tira de pano, que se dobra e cose sôbre a orla de um tecido ou de uma peça de vestuário, para o enfeitar ou para lhe segurar a trama.
Orla.
Cordão, que se mostra á roda do casco do cavallo, doente de formigo.
(Talvez por \textunderscore dobrum\textunderscore , de \textunderscore dobrar\textunderscore )
\section{Debulha}
\begin{itemize}
\item {Grp. gram.:f.}
\end{itemize}
Acto de debulhar.
\section{Debulhador}
\begin{itemize}
\item {Grp. gram.:m.}
\end{itemize}
\begin{itemize}
\item {Proveniência:(De \textunderscore debulhar\textunderscore )}
\end{itemize}
Aquelle que debulha.
Máquina, para debulhar.
\section{Debulhadora}
\begin{itemize}
\item {Grp. gram.:f.}
\end{itemize}
Máquina de debulhar cereaes.
\section{Debulhar}
\begin{itemize}
\item {Grp. gram.:v. t.}
\end{itemize}
\begin{itemize}
\item {Proveniência:(Do lat. \textunderscore despoliare\textunderscore , se não de \textunderscore de-pileare\textunderscore )}
\end{itemize}
Esbagoar.
Separar do casulo (grãos de cereaes).
Descascar (frutos, tubérculos, etc.).
\section{Debulho}
\begin{itemize}
\item {Grp. gram.:m.}
\end{itemize}
\begin{itemize}
\item {Utilização:Bras. do N}
\end{itemize}
Alimentos, triturados e em princípio de digestão, no estômago dos ruminantes.
\section{Debutar}
\begin{itemize}
\item {Grp. gram.:v. i.}
\end{itemize}
\begin{itemize}
\item {Proveniência:(Fr. \textunderscore débuter\textunderscore )}
\end{itemize}
Estrear-se.--É gallicismo inadmissível.
\section{Debute}
\begin{itemize}
\item {Grp. gram.:m.}
\end{itemize}
\begin{itemize}
\item {Proveniência:(Fr. \textunderscore début\textunderscore )}
\end{itemize}
Estreia, comêço.--É gallicismo inadmissível.
\section{Debuxador}
\begin{itemize}
\item {Grp. gram.:m.}
\end{itemize}
Aquelle que debuxa.
\section{Debuxante}
\begin{itemize}
\item {Grp. gram.:adj.}
\end{itemize}
Que debuxa.
\section{Debuxar}
\begin{itemize}
\item {Grp. gram.:v. t.}
\end{itemize}
\begin{itemize}
\item {Utilização:Fig.}
\end{itemize}
Fazer o debuxo de.
Delinear.
Esboçar.
Planear.
\section{Debuxo}
\begin{itemize}
\item {Grp. gram.:m.}
\end{itemize}
\begin{itemize}
\item {Utilização:Fig.}
\end{itemize}
Esbôço, delineamento.
Desenho, que representa um objecto pelos contornos.
Estampa, colorida ou não, que serve de modêlo para bordados.
Plano, traça.
Instrumento de correeiro, com que se riscam as bordas das correias.
Peça das fábricas de estamparia, lavrada em relêvo, e sôbre a qual se applica tinta, para a estampagem das chitas.
(Cast. \textunderscore dibujo\textunderscore )
\section{Deca...}
\begin{itemize}
\item {Grp. gram.:pref.}
\end{itemize}
\begin{itemize}
\item {Proveniência:(Gr. \textunderscore deka\textunderscore )}
\end{itemize}
(Que significa \textunderscore déz\textunderscore )
\section{Deca}
\begin{itemize}
\item {Grp. gram.:f.}
\end{itemize}
\begin{itemize}
\item {Utilização:Pop.}
\end{itemize}
O mesmo que \textunderscore decalitro\textunderscore .
\section{Decacordo}
\begin{itemize}
\item {fónica:dé}
\end{itemize}
\begin{itemize}
\item {Grp. gram.:m.}
\end{itemize}
O mesmo que \textunderscore asor\textunderscore .
\section{Década}
\begin{itemize}
\item {Grp. gram.:f.}
\end{itemize}
\begin{itemize}
\item {Proveniência:(Do gr. \textunderscore dekas\textunderscore )}
\end{itemize}
Série de déz.
Dezena.
Espaço de déz dias, espaço de déz annos.
\section{Decadáctilo}
\begin{itemize}
\item {fónica:dé}
\end{itemize}
\begin{itemize}
\item {Grp. gram.:adj.}
\end{itemize}
Que tem déz dedos.
Diz-se especialmente do peixe, que tem déz espinhas em cada barbatana peitoral.
\section{Decadáctylo}
\begin{itemize}
\item {fónica:dé}
\end{itemize}
\begin{itemize}
\item {Grp. gram.:adj.}
\end{itemize}
Que tem déz dedos.
Diz-se especialmente do peixe, que tem déz espinhas em cada barbatana peitoral.
\section{Decadência}
\begin{itemize}
\item {Grp. gram.:f.}
\end{itemize}
Estado de quem ou daquillo que decai.
Acto de decair.
(B. lat. \textunderscore decadentia\textunderscore )
\section{Decadente}
\begin{itemize}
\item {Grp. gram.:adj.}
\end{itemize}
\begin{itemize}
\item {Grp. gram.:M.}
\end{itemize}
\begin{itemize}
\item {Proveniência:(Do lat. \textunderscore de\textunderscore  + \textunderscore cadens\textunderscore )}
\end{itemize}
Que decai.
Sectário do decadismo, decadista.
\section{Decadismo}
\begin{itemize}
\item {Grp. gram.:m.}
\end{itemize}
Supposta escola literária contemporânea, que procurou desviar-se dos processos conhecidos da versificação e dos preceitos communs da bôa elocução.
Nephelibatismo.
(Cp. \textunderscore decadista\textunderscore )
\section{Decadista}
\begin{itemize}
\item {Grp. gram.:adj.}
\end{itemize}
\begin{itemize}
\item {Utilização:Neol.}
\end{itemize}
\begin{itemize}
\item {Grp. gram.:M.}
\end{itemize}
\begin{itemize}
\item {Proveniência:(Do fr. \textunderscore décadi\textunderscore , último dia da década, no calendário republicano?)}
\end{itemize}
Relativo ao decadismo.
Sectário do decadismo.
\section{Decaédro}
\begin{itemize}
\item {Grp. gram.:m.}
\end{itemize}
\begin{itemize}
\item {Proveniência:(Do gr. \textunderscore deka\textunderscore  + \textunderscore hedra\textunderscore )}
\end{itemize}
Figura geométrica, com déz faces.
\section{Decafido}
\begin{itemize}
\item {Grp. gram.:adj.}
\end{itemize}
\begin{itemize}
\item {Utilização:Bot.}
\end{itemize}
\begin{itemize}
\item {Proveniência:(T. hybr., do gr. \textunderscore deka\textunderscore  + lat. \textunderscore findere\textunderscore )}
\end{itemize}
Diz-se do cálice e da corolla, quando o limbo apresenta déz incisões, que se prolongam pelo menos até á metade do seu comprimento total.
\section{Decágino}
\begin{itemize}
\item {Grp. gram.:adj.}
\end{itemize}
\begin{itemize}
\item {Utilização:Bot.}
\end{itemize}
\begin{itemize}
\item {Proveniência:(Do gr. \textunderscore deka\textunderscore  + \textunderscore gunè\textunderscore )}
\end{itemize}
Diz-se das plantas, que têm déz pistilos, ou déz estiletes, ou déz estigmas sésseis.
\section{Decagonal}
\begin{itemize}
\item {Grp. gram.:adj.}
\end{itemize}
Relativo a decágono.
\section{Decágono}
\begin{itemize}
\item {Grp. gram.:m.}
\end{itemize}
\begin{itemize}
\item {Proveniência:(Gr. \textunderscore dekagonos\textunderscore )}
\end{itemize}
Figura geométrica, que tem déz ângulos e déz lados.
\section{Decagrama}
\begin{itemize}
\item {Grp. gram.:m.}
\end{itemize}
\begin{itemize}
\item {Proveniência:(Do gr. \textunderscore deka\textunderscore  + \textunderscore gramma\textunderscore )}
\end{itemize}
Pêso de déz gramas.
\section{Decagramma}
\begin{itemize}
\item {Grp. gram.:m.}
\end{itemize}
\begin{itemize}
\item {Proveniência:(Do gr. \textunderscore deka\textunderscore  + \textunderscore gramma\textunderscore )}
\end{itemize}
Pêso de déz grammas.
\section{Decágyno}
\begin{itemize}
\item {Grp. gram.:adj.}
\end{itemize}
\begin{itemize}
\item {Utilização:Bot.}
\end{itemize}
\begin{itemize}
\item {Proveniência:(Do gr. \textunderscore deka\textunderscore  + \textunderscore gunè\textunderscore )}
\end{itemize}
Diz-se das plantas, que têm déz pistillos, ou déz estyletes, ou déz estigmas sésseis.
\section{Decaída}
\begin{itemize}
\item {Grp. gram.:f.}
\end{itemize}
Effeito de decair.
\section{Decaído}
\begin{itemize}
\item {Grp. gram.:adj.}
\end{itemize}
Que decaíu.
\section{Decaimento}
\begin{itemize}
\item {fónica:ca-i}
\end{itemize}
\begin{itemize}
\item {Grp. gram.:m.}
\end{itemize}
O mesmo que \textunderscore decadência\textunderscore .
\section{Decair}
\begin{itemize}
\item {Grp. gram.:v. i.}
\end{itemize}
\begin{itemize}
\item {Proveniência:(Do lat. \textunderscore de\textunderscore  + \textunderscore cadere\textunderscore )}
\end{itemize}
Aproximar-se da sua extincção.
Ir-se arruinando.
Pender.
Abater-se.
Empobrecer.
Enfraquecer.
Estragar-se.
\section{Decalcar}
\begin{itemize}
\item {Grp. gram.:v. t.}
\end{itemize}
Em desenho, o mesmo que \textunderscore calcar\textunderscore .
\section{Decalcomania}
\begin{itemize}
\item {Grp. gram.:f.}
\end{itemize}
\begin{itemize}
\item {Proveniência:(De \textunderscore decalcar\textunderscore  + lat. \textunderscore manus\textunderscore )}
\end{itemize}
Arte de produzir certos quadros, calcando com a mão, contra um papel, pequenos desenhos ou figuras, já estampadas noutro papel.
\section{Decalitro}
\begin{itemize}
\item {Grp. gram.:m.}
\end{itemize}
\begin{itemize}
\item {Proveniência:(Do gr. \textunderscore deka\textunderscore  + \textunderscore litra\textunderscore )}
\end{itemize}
Porção ou medida de déz litros.
\section{Decalobado}
\begin{itemize}
\item {fónica:dé}
\end{itemize}
\begin{itemize}
\item {Grp. gram.:adj.}
\end{itemize}
\begin{itemize}
\item {Utilização:Bot.}
\end{itemize}
\begin{itemize}
\item {Proveniência:(Do gr. \textunderscore deka\textunderscore  + \textunderscore lobos\textunderscore )}
\end{itemize}
Diz-se das partes vegetaes, cujo limbo offerece déz divisões arredondadas ou lóbulos.
\section{Decálogo}
\begin{itemize}
\item {Grp. gram.:m.}
\end{itemize}
\begin{itemize}
\item {Proveniência:(Gr. \textunderscore dekalogos\textunderscore )}
\end{itemize}
Os déz mandamentos da chamada lei de Deus.
\section{Decalque}
\begin{itemize}
\item {Grp. gram.:m.}
\end{itemize}
O mesmo que \textunderscore calco\textunderscore .
Acto de decalcar.
\section{Decalvação}
\begin{itemize}
\item {Grp. gram.:f.}
\end{itemize}
Acto ou effeito de decalvar.
\section{Decalvar}
\textunderscore v. t.\textunderscore  (e der.)
O mesmo que \textunderscore escalvar\textunderscore , etc.
\section{Decamerónico}
\begin{itemize}
\item {Grp. gram.:adj.}
\end{itemize}
Relativo ao Decameron, de Bocaccio ou á sua feição literária.
\section{Decâmeros}
\begin{itemize}
\item {Grp. gram.:m. pl.}
\end{itemize}
\begin{itemize}
\item {Proveniência:(Do gr. \textunderscore deka\textunderscore  + \textunderscore meros\textunderscore )}
\end{itemize}
Gênero de insectos coleópteros, que têm déz artículos nas antennas.
\section{Decâmetro}
\begin{itemize}
\item {Grp. gram.:m.}
\end{itemize}
\begin{itemize}
\item {Proveniência:(Do gr. \textunderscore deka\textunderscore  + \textunderscore metron\textunderscore )}
\end{itemize}
Medida de déz metros.
Extensão de déz metros.
\section{Decampamento}
\begin{itemize}
\item {Grp. gram.:m.}
\end{itemize}
Acto ou effeito de \textunderscore decampar\textunderscore .
\section{Decampar}
\begin{itemize}
\item {Grp. gram.:v. i.}
\end{itemize}
\begin{itemize}
\item {Proveniência:(De \textunderscore de...\textunderscore  + \textunderscore campo\textunderscore )}
\end{itemize}
Mudar de campo, de acampamento.
\section{Decanado}
\begin{itemize}
\item {Grp. gram.:m.}
\end{itemize}
\begin{itemize}
\item {Proveniência:(De \textunderscore decano\textunderscore )}
\end{itemize}
Dignidade de deão.
Qualidade de decano.
\section{Decanas}
\begin{itemize}
\item {Grp. gram.:m. pl.}
\end{itemize}
Antiga tríbo de indígenas do Pará.
\section{Decafilo}
\begin{itemize}
\item {Grp. gram.:adj.}
\end{itemize}
\begin{itemize}
\item {Utilização:Bot.}
\end{itemize}
\begin{itemize}
\item {Proveniência:(Do gr. \textunderscore deka\textunderscore  + \textunderscore phullon\textunderscore )}
\end{itemize}
Que tem déz fôlhas.
\section{Decandria}
\begin{itemize}
\item {Grp. gram.:f.}
\end{itemize}
\begin{itemize}
\item {Utilização:Bot.}
\end{itemize}
\begin{itemize}
\item {Proveniência:(De \textunderscore decandro\textunderscore )}
\end{itemize}
Qualidade de decandro.
Classe dos vegetaes que tem dez estames.
\section{Decandro}
\begin{itemize}
\item {Grp. gram.:adj.}
\end{itemize}
\begin{itemize}
\item {Proveniência:(Do gr. \textunderscore deka\textunderscore  + \textunderscore aner\textunderscore , \textunderscore andros\textunderscore )}
\end{itemize}
Que tem déz estames, livres entre si.
\section{Decangular}
\begin{itemize}
\item {Grp. gram.:adj.}
\end{itemize}
\begin{itemize}
\item {Proveniência:(De \textunderscore deca...\textunderscore  + \textunderscore angular\textunderscore )}
\end{itemize}
Que tem déz ângulos.
\section{Decania}
\begin{itemize}
\item {Grp. gram.:f.}
\end{itemize}
Qualidade de decano.
Corporação presidida por decano.
Grupo de déz pessôas.
Decanado.
Pelotão de déz homens, entre os Godos. Cf. C. Aires, \textunderscore Hist. do Exérc. Port.\textunderscore 
\section{Decânico}
\begin{itemize}
\item {Grp. gram.:adj.}
\end{itemize}
\begin{itemize}
\item {Proveniência:(De \textunderscore Decão\textunderscore , n. p.)}
\end{itemize}
Relativo ao Decão, na Índia.
Diz-se de um grupo de línguas asiáticas.
\section{Decanim}
\begin{itemize}
\item {Grp. gram.:m.}
\end{itemize}
Habitante do Decão, na Índia. Cf. G. Horta, \textunderscore Coll.\textunderscore , II.
\section{Decano}
\begin{itemize}
\item {Grp. gram.:m.}
\end{itemize}
\begin{itemize}
\item {Utilização:Chím.}
\end{itemize}
\begin{itemize}
\item {Proveniência:(Lat. \textunderscore decanus\textunderscore )}
\end{itemize}
O membro mais velho ou mais antigo de uma classe ou corporação.
Deão.

Variedade de carboneto do grupo formênico.
\section{Decantação}
\begin{itemize}
\item {Grp. gram.:f.}
\end{itemize}
Acto de decantar^2.
\section{Decantar}
\begin{itemize}
\item {Grp. gram.:v. t.}
\end{itemize}
\begin{itemize}
\item {Proveniência:(Lat. \textunderscore decantare\textunderscore )}
\end{itemize}
Celebrar em canto ou em verso.
Celebrar.
\section{Decantar}
\begin{itemize}
\item {Grp. gram.:v. t.}
\end{itemize}
\begin{itemize}
\item {Proveniência:(De \textunderscore de...\textunderscore  + gr. \textunderscore kanthos\textunderscore )}
\end{itemize}
Passar cautelosamente de um vaso para outro (um líquido), para separar as fezes dêste.
\section{Decapetaleado}
\begin{itemize}
\item {Grp. gram.:adj.}
\end{itemize}
\begin{itemize}
\item {Utilização:Bot.}
\end{itemize}
\begin{itemize}
\item {Proveniência:(Do gr. \textunderscore deka\textunderscore  + \textunderscore petalon\textunderscore )}
\end{itemize}
Diz-se das corollas, que são compostas de déz peças distintas.
\section{Decaphyllo}
\begin{itemize}
\item {Grp. gram.:adj.}
\end{itemize}
\begin{itemize}
\item {Utilização:Bot.}
\end{itemize}
\begin{itemize}
\item {Proveniência:(Do gr. \textunderscore deka\textunderscore  + \textunderscore phullon\textunderscore )}
\end{itemize}
Que tem déz fôlhas.
\section{Decapitação}
\begin{itemize}
\item {Grp. gram.:f.}
\end{itemize}
Acto de decapitar.
\section{Decapitar}
\begin{itemize}
\item {Grp. gram.:v. t.}
\end{itemize}
\begin{itemize}
\item {Utilização:Fig.}
\end{itemize}
\begin{itemize}
\item {Proveniência:(Lat. \textunderscore decapitare\textunderscore )}
\end{itemize}
Cortar a cabeça de.
Eliminar o chefe de: \textunderscore decapitar uma nação\textunderscore .
Tirar a parte superior de (alguma coisa): \textunderscore decapitar um mastro\textunderscore .
\section{Decápoda}
\begin{itemize}
\item {Grp. gram.:m.  e  adj.}
\end{itemize}
\begin{itemize}
\item {Utilização:Zool.}
\end{itemize}
\begin{itemize}
\item {Proveniência:(Do gr. \textunderscore deka\textunderscore  + \textunderscore pous\textunderscore , \textunderscore podos\textunderscore )}
\end{itemize}
Crustáceo, caracterizado por cinco pares de patas.
\section{Decápode}
\begin{itemize}
\item {Grp. gram.:m.  e  adj.}
\end{itemize}
\begin{itemize}
\item {Utilização:Zool.}
\end{itemize}
\begin{itemize}
\item {Proveniência:(Do gr. \textunderscore deka\textunderscore  + \textunderscore pous\textunderscore , \textunderscore podos\textunderscore )}
\end{itemize}
Crustáceo, caracterizado por cinco pares de patas.
\section{Decapódeo}
\begin{itemize}
\item {Grp. gram.:m.  e  adj.}
\end{itemize}
(V. decápode)
\section{Decaprotia}
\begin{itemize}
\item {Grp. gram.:f.}
\end{itemize}
\begin{itemize}
\item {Proveniência:(Gr. \textunderscore dekaproteia\textunderscore )}
\end{itemize}
Dignidade ou funcções dos decaprotos.
\section{Decaprotos}
\begin{itemize}
\item {Grp. gram.:m. pl.}
\end{itemize}
\begin{itemize}
\item {Proveniência:(Gr. \textunderscore dekaprotos\textunderscore )}
\end{itemize}
Os déz funccionários que, nas cidades e colónias gregas, dirigiam a cobrança dos impostos.
\section{Decassílabo}
\begin{itemize}
\item {Grp. gram.:adj.}
\end{itemize}
\begin{itemize}
\item {Grp. gram.:M.}
\end{itemize}
\begin{itemize}
\item {Proveniência:(Gr. \textunderscore dekasullabos\textunderscore )}
\end{itemize}
Diz-se do verso, que tem déz sílabas.
Esse verso.
\section{Decastere}
\begin{itemize}
\item {Grp. gram.:m.}
\end{itemize}
\begin{itemize}
\item {Proveniência:(Do gr. \textunderscore deka\textunderscore  + \textunderscore stereos\textunderscore )}
\end{itemize}
Medida de déz esteres.
\section{Decastilo}
\begin{itemize}
\item {Grp. gram.:m.}
\end{itemize}
\begin{itemize}
\item {Proveniência:(Do gr. \textunderscore deka\textunderscore  + \textunderscore stulos\textunderscore )}
\end{itemize}
Monumento ou edifício, com déz colunas na fachada.
\section{Decastylo}
\begin{itemize}
\item {Grp. gram.:m.}
\end{itemize}
\begin{itemize}
\item {Proveniência:(Do gr. \textunderscore deka\textunderscore  + \textunderscore stulos\textunderscore )}
\end{itemize}
Monumento ou edifício, com déz columnas na fachada.
\section{Decasýllabo}
\begin{itemize}
\item {fónica:si}
\end{itemize}
\begin{itemize}
\item {Grp. gram.:adj.}
\end{itemize}
\begin{itemize}
\item {Grp. gram.:M.}
\end{itemize}
\begin{itemize}
\item {Proveniência:(Gr. \textunderscore dekasullabos\textunderscore )}
\end{itemize}
Diz-se do verso, que tem déz sýllabas.
Esse verso.
\section{Decedura}
\begin{itemize}
\item {Grp. gram.:f.}
\end{itemize}
\begin{itemize}
\item {Utilização:Ant.}
\end{itemize}
Talvez o mesmo que \textunderscore occasião\textunderscore .
(Cp. lat. \textunderscore decedere\textunderscore )
\section{Deceinar}
\begin{itemize}
\item {Grp. gram.:v. t.}
\end{itemize}
\begin{itemize}
\item {Utilização:Ant.}
\end{itemize}
\begin{itemize}
\item {Proveniência:(Do lat. \textunderscore de\textunderscore  + \textunderscore cinis\textunderscore ?)}
\end{itemize}
Lavar (meadas), para lhes tirar a cinza da barrela.
Trazer de noite na mão (a ave), depois da muda, para a amansar de novo, (falando-se de volataria).
\section{Decélico}
\begin{itemize}
\item {Grp. gram.:adj.}
\end{itemize}
\begin{itemize}
\item {Proveniência:(Do gr. \textunderscore Dekeleia\textunderscore , n. p.)}
\end{itemize}
Relativo a Decelía, na Áttica.
Diz-se da guerra, que alli houve, entre Espartanos e Athenienses. Cf. Latino, \textunderscore Or. da Corôa\textunderscore , 20.
\section{Decemcellular}
\begin{itemize}
\item {Grp. gram.:adj.}
\end{itemize}
O mesmo que \textunderscore decemlocular\textunderscore .
\section{Decencelular}
\begin{itemize}
\item {Grp. gram.:adj.}
\end{itemize}
O mesmo que \textunderscore decemlocular\textunderscore .
\section{Decemestre}
\begin{itemize}
\item {Grp. gram.:m.}
\end{itemize}
\begin{itemize}
\item {Utilização:Des.}
\end{itemize}
O espaço de déz meses.
(Cp. \textunderscore trimestre\textunderscore  e \textunderscore semestre\textunderscore )
\section{Decemfido}
\begin{itemize}
\item {Grp. gram.:adj.}
\end{itemize}
O mesmo que \textunderscore decafido\textunderscore .
\section{Decemlocular}
\begin{itemize}
\item {Grp. gram.:adj.}
\end{itemize}
\begin{itemize}
\item {Utilização:Bot.}
\end{itemize}
\begin{itemize}
\item {Proveniência:(Do lat. \textunderscore decem\textunderscore  + \textunderscore loculus\textunderscore )}
\end{itemize}
Diz-se de um fruto ou de um ovário, quando dividido interiormente em déz lóculos.
\section{Decêmpeda}
\begin{itemize}
\item {Grp. gram.:f.}
\end{itemize}
\begin{itemize}
\item {Proveniência:(Lat. \textunderscore decempeda\textunderscore )}
\end{itemize}
Medida romana, linear, de déz pés de comprimento.
\section{Decemplicar}
\begin{itemize}
\item {Grp. gram.:v. t.}
\end{itemize}
O mesmo que \textunderscore decuplicar\textunderscore .
\section{Decemvirado}
\begin{itemize}
\item {Grp. gram.:m.}
\end{itemize}
\begin{itemize}
\item {Proveniência:(Lat. \textunderscore decemviratus\textunderscore )}
\end{itemize}
Dignidade ou govêrno dos decêmviros.
\section{Decemviral}
\begin{itemize}
\item {Grp. gram.:adj.}
\end{itemize}
\begin{itemize}
\item {Proveniência:(Lat. \textunderscore decemviralis\textunderscore )}
\end{itemize}
Relativo aos decêmviros.
\section{Decemvirato}
\begin{itemize}
\item {Grp. gram.:m.}
\end{itemize}
O mesmo que \textunderscore decemvirado\textunderscore .
\section{Decêmviro}
\begin{itemize}
\item {Grp. gram.:m.}
\end{itemize}
\begin{itemize}
\item {Proveniência:(Lat. \textunderscore decemvíri\textunderscore )}
\end{itemize}
Cada um dos déz magistrados, que na república romana foram encarregados de codificar as leis.
Cada um dos déz cidadãos, que, com o pretor, constituíam a magistratura judicial, entre os Romanos.
\section{Decenal}
\begin{itemize}
\item {Grp. gram.:adj.}
\end{itemize}
\begin{itemize}
\item {Proveniência:(Lat. \textunderscore decennalis\textunderscore )}
\end{itemize}
Que dura déz anos; que se realiza de déz em dez anos.
\section{Decenário}
\begin{itemize}
\item {Grp. gram.:m.}
\end{itemize}
\begin{itemize}
\item {Proveniência:(Do lat. \textunderscore deceni\textunderscore )}
\end{itemize}
Rosário, dividido em dezenas.
\section{Decência}
\begin{itemize}
\item {Grp. gram.:f.}
\end{itemize}
\begin{itemize}
\item {Proveniência:(Lat. \textunderscore decentia\textunderscore )}
\end{itemize}
Qualidade de quem ou daquillo que é decente.
\section{Decendial}
\begin{itemize}
\item {Grp. gram.:adj.}
\end{itemize}
\begin{itemize}
\item {Utilização:Jur.}
\end{itemize}
Relativo a decêndio.
\section{Decendimento}
\begin{itemize}
\item {Grp. gram.:m.}
\end{itemize}
\begin{itemize}
\item {Utilização:Prov.}
\end{itemize}
\begin{itemize}
\item {Utilização:trasm.}
\end{itemize}
\begin{itemize}
\item {Proveniência:(De \textunderscore decender\textunderscore )}
\end{itemize}
O mesmo que \textunderscore descida\textunderscore .
\section{Decêndio}
\begin{itemize}
\item {Grp. gram.:m.}
\end{itemize}
\begin{itemize}
\item {Proveniência:(Do lat. \textunderscore decem\textunderscore  + \textunderscore dies\textunderscore )}
\end{itemize}
Espaço de déz dias.
\section{Decenfido}
\begin{itemize}
\item {Grp. gram.:adj.}
\end{itemize}
O mesmo que \textunderscore decafido\textunderscore .
\section{Decênio}
\begin{itemize}
\item {Grp. gram.:m.}
\end{itemize}
\begin{itemize}
\item {Proveniência:(Lat. \textunderscore decennium\textunderscore )}
\end{itemize}
Espaço de déz anos.
\section{Decenlocular}
\begin{itemize}
\item {Grp. gram.:adj.}
\end{itemize}
\begin{itemize}
\item {Utilização:Bot.}
\end{itemize}
\begin{itemize}
\item {Proveniência:(Do lat. \textunderscore decem\textunderscore  + \textunderscore loculus\textunderscore )}
\end{itemize}
Diz-se de um fruto ou de um ovário, quando dividido interiormente em déz lóculos.
\section{Decennal}
\begin{itemize}
\item {Grp. gram.:adj.}
\end{itemize}
\begin{itemize}
\item {Proveniência:(Lat. \textunderscore decennalis\textunderscore )}
\end{itemize}
Que dura déz annos; que se realiza de déz em dez annos.
\section{Decênnio}
\begin{itemize}
\item {Grp. gram.:m.}
\end{itemize}
\begin{itemize}
\item {Proveniência:(Lat. \textunderscore decennium\textunderscore )}
\end{itemize}
Espaço de déz annos.
\section{Decennoval}
\begin{itemize}
\item {Grp. gram.:adj.}
\end{itemize}
\begin{itemize}
\item {Proveniência:(Lat. \textunderscore decennovenalis\textunderscore )}
\end{itemize}
Relativo ao espaço de dezanove.
\section{Decennovenal}
\begin{itemize}
\item {Grp. gram.:adj.}
\end{itemize}
\begin{itemize}
\item {Proveniência:(Lat. \textunderscore decennovenalis\textunderscore )}
\end{itemize}
Relativo ao espaço de dezanove.
\section{Decenoval}
\begin{itemize}
\item {Grp. gram.:adj.}
\end{itemize}
\begin{itemize}
\item {Proveniência:(Lat. \textunderscore decennovenalis\textunderscore )}
\end{itemize}
Relativo ao espaço de dezanove.
\section{Decenovenal}
\begin{itemize}
\item {Grp. gram.:adj.}
\end{itemize}
\begin{itemize}
\item {Proveniência:(Lat. \textunderscore decennovenalis\textunderscore )}
\end{itemize}
Relativo ao espaço de dezanove.
\section{Decente}
\begin{itemize}
\item {Grp. gram.:adj.}
\end{itemize}
\begin{itemize}
\item {Proveniência:(Lat. \textunderscore decens\textunderscore )}
\end{itemize}
Que fica bem, que é apropriado: \textunderscore casa decente\textunderscore .
Honesto.
Conveniente.
Decoroso.
Limpo, asseado: \textunderscore traje decente\textunderscore .
Bem comportado: \textunderscore pessôa decente\textunderscore .
\section{Decentemente}
\begin{itemize}
\item {Grp. gram.:adv.}
\end{itemize}
De modo decente, com decência.
\section{Decentralizar}
\textunderscore v. t.\textunderscore  (e der.)
(V. \textunderscore descentralizar\textunderscore , etc.)
\section{Decenvirado}
\begin{itemize}
\item {Grp. gram.:m.}
\end{itemize}
\begin{itemize}
\item {Proveniência:(Lat. \textunderscore decemviratus\textunderscore )}
\end{itemize}
Dignidade ou govêrno dos decênviros.
\section{Decenviral}
\begin{itemize}
\item {Grp. gram.:adj.}
\end{itemize}
\begin{itemize}
\item {Proveniência:(Lat. \textunderscore decemviralis\textunderscore )}
\end{itemize}
Relativo aos decênviros.
\section{Decenvirato}
\begin{itemize}
\item {Grp. gram.:m.}
\end{itemize}
O mesmo que \textunderscore decenvirado\textunderscore .
\section{Decênviro}
\begin{itemize}
\item {Grp. gram.:m.}
\end{itemize}
\begin{itemize}
\item {Proveniência:(Lat. \textunderscore decemvíri\textunderscore )}
\end{itemize}
Cada um dos déz magistrados, que na república romana foram encarregados de codificar as leis.
Cada um dos déz cidadãos, que, com o pretor, constituíam a magistratura judicial, entre os Romanos.
\section{Decepagem}
\begin{itemize}
\item {Grp. gram.:f.}
\end{itemize}
\begin{itemize}
\item {Proveniência:(De \textunderscore decepar\textunderscore )}
\end{itemize}
Córte de (árvores).
\section{Decepador}
\begin{itemize}
\item {Grp. gram.:adj.}
\end{itemize}
Que decepa.
\section{Decepamento}
\begin{itemize}
\item {Grp. gram.:m.}
\end{itemize}
Acto ou effeito de \textunderscore decepar\textunderscore .
\section{Decepar}
\begin{itemize}
\item {Grp. gram.:v. t.}
\end{itemize}
\begin{itemize}
\item {Proveniência:(De \textunderscore de...\textunderscore  + \textunderscore cepo\textunderscore ?)}
\end{itemize}
Amputar.
Mutilar.
Cortar (parte de um corpo ou de um todo): \textunderscore decepar um braço\textunderscore .
Interceptar: \textunderscore decepar uma conversa\textunderscore .
Desunir.
\section{Decepção}
\begin{itemize}
\item {Grp. gram.:f.}
\end{itemize}
\begin{itemize}
\item {Proveniência:(Lat. \textunderscore deceptio\textunderscore )}
\end{itemize}
Acto de enganar.
Lôgro.
Surpresa.
Desillusão.
\section{Decer}
\textunderscore v. i.\textunderscore  (e der.)
O mesmo que \textunderscore descer\textunderscore , etc.
\section{Decercar}
\begin{itemize}
\item {Grp. gram.:v. t.}
\end{itemize}
O mesmo que \textunderscore descercar\textunderscore . Cf. Usque, \textunderscore Tribulações\textunderscore , 43.
\section{Decertar}
\begin{itemize}
\item {Grp. gram.:v. i.}
\end{itemize}
\begin{itemize}
\item {Utilização:Des.}
\end{itemize}
\begin{itemize}
\item {Proveniência:(Lat. \textunderscore decertare\textunderscore )}
\end{itemize}
Pelejar.
\section{Decessor}
\begin{itemize}
\item {Grp. gram.:m.}
\end{itemize}
\begin{itemize}
\item {Utilização:Ant.}
\end{itemize}
\begin{itemize}
\item {Proveniência:(Lat. \textunderscore decessor\textunderscore )}
\end{itemize}
O mesmo que \textunderscore antecessor\textunderscore .
\section{Decho}
\begin{itemize}
\item {Grp. gram.:m.}
\end{itemize}
\begin{itemize}
\item {Utilização:Ant.}
\end{itemize}
O mesmo que \textunderscore diacho\textunderscore . Cf. G. Vicente, I, 174.
\section{Deci...}
\begin{itemize}
\item {Grp. gram.:pref.}
\end{itemize}
\begin{itemize}
\item {Proveniência:(Do lat. \textunderscore decem\textunderscore )}
\end{itemize}
(que se emprega nos nomes das medidas do systema métrico, e designa a décima parte da unidade): \textunderscore um decímetro\textunderscore .
\section{Decididamente}
\begin{itemize}
\item {Grp. gram.:adv.}
\end{itemize}
De modo decidido.
Resolutamente.
Certamente.
\section{Decidido}
\begin{itemize}
\item {Grp. gram.:adj.}
\end{itemize}
Corajoso.
Resoluto: \textunderscore homem decidido\textunderscore .
Inabalável.
Determinado: \textunderscore negócio decidido\textunderscore .
\section{Decidir}
\begin{itemize}
\item {Grp. gram.:v. t.}
\end{itemize}
\begin{itemize}
\item {Grp. gram.:V. i.}
\end{itemize}
\begin{itemize}
\item {Proveniência:(Lat. \textunderscore decidere\textunderscore )}
\end{itemize}
Resolver: \textunderscore decidir uma dúvida\textunderscore .
Determinar: \textunderscore decidir uma viagem\textunderscore .
Sêr causa de.
Esclarecer.
Sentenciar: \textunderscore decidir um pleito\textunderscore .
Opinar, emittir juízo: \textunderscore decidir de assumptos graves\textunderscore .
Tomar resolução: \textunderscore decidir da sorte de alguém\textunderscore .
\section{Decíduo}
\begin{itemize}
\item {Grp. gram.:adj.}
\end{itemize}
\begin{itemize}
\item {Utilização:Bot.}
\end{itemize}
\begin{itemize}
\item {Proveniência:(Lat. \textunderscore deciduus\textunderscore )}
\end{itemize}
Diz-se do cálice que cai depois de murcho.
\section{Decifração}
\begin{itemize}
\item {Grp. gram.:f.}
\end{itemize}
Acto ou effeito de decifrar.
\section{Decifrador}
\begin{itemize}
\item {Grp. gram.:m.}
\end{itemize}
Aquelle que decifra.
\section{Decifrar}
\begin{itemize}
\item {Grp. gram.:v. t.}
\end{itemize}
\begin{itemize}
\item {Grp. gram.:V. i.}
\end{itemize}
\begin{itemize}
\item {Proveniência:(De \textunderscore cifra\textunderscore )}
\end{itemize}
Lêr (o que está mal ou obscuramente escrito, ou escrito em cifra): \textunderscore decifrar uma inscripção\textunderscore .
Interpretar, comprehender (o que é intrincado ou obscuro): \textunderscore decifrar um enigma\textunderscore .
Adivinhar: \textunderscore decifrar o futuro\textunderscore .
Perceber: \textunderscore decifrar intenções\textunderscore .
Conhecer bem.
Executar promptamente (música).
Sêr intérprete, fazer interpretação: \textunderscore saber decifrar\textunderscore .
\section{Decifravel}
\begin{itemize}
\item {Grp. gram.:adj.}
\end{itemize}
Que se póde decifrar.
\section{Decigrama}
\begin{itemize}
\item {Grp. gram.:m.}
\end{itemize}
\begin{itemize}
\item {Proveniência:(De \textunderscore deci...\textunderscore  + \textunderscore gramma\textunderscore )}
\end{itemize}
Décima parte de um grama.
\section{Decigramma}
\begin{itemize}
\item {Grp. gram.:m.}
\end{itemize}
\begin{itemize}
\item {Proveniência:(De \textunderscore deci...\textunderscore  + \textunderscore gramma\textunderscore )}
\end{itemize}
Décima parte de um gramma.
\section{Decilitragem}
\begin{itemize}
\item {Grp. gram.:f.}
\end{itemize}
\begin{itemize}
\item {Utilização:Chul.}
\end{itemize}
Acto de \textunderscore decilitrar\textunderscore .
\section{Decilitrar}
\begin{itemize}
\item {Grp. gram.:v. i.}
\end{itemize}
\begin{itemize}
\item {Utilização:Chul.}
\end{itemize}
Beber vinhos aos decilitros.
Bebericar.
\section{Decilitro}
\begin{itemize}
\item {Grp. gram.:m.}
\end{itemize}
\begin{itemize}
\item {Proveniência:(De \textunderscore deci...\textunderscore  + \textunderscore litro\textunderscore )}
\end{itemize}
Décima parte do litro.
Medida de capacidade, igual á décima parte de um litro.
\section{Décima}
\begin{itemize}
\item {Grp. gram.:f.}
\end{itemize}
\begin{itemize}
\item {Utilização:Ext.}
\end{itemize}
\begin{itemize}
\item {Utilização:Vers.}
\end{itemize}
\begin{itemize}
\item {Proveniência:(Lat. \textunderscore decima\textunderscore )}
\end{itemize}
Dezena.
Imposto, que abrangia a décima parte de um rendimento.
Tributo, contribuição directa.
Estrophe de déz versos septisýllabos.
\section{De-cima}
\begin{itemize}
\item {Grp. gram.:loc. adv.}
\end{itemize}
Da parte superior.
Do alto: \textunderscore olhar de-cima\textunderscore .
\section{Decimal}
\begin{itemize}
\item {Grp. gram.:adj.}
\end{itemize}
\begin{itemize}
\item {Grp. gram.:F.}
\end{itemize}
\begin{itemize}
\item {Utilização:Arith.}
\end{itemize}
\begin{itemize}
\item {Proveniência:(De \textunderscore décimo\textunderscore )}
\end{itemize}
Que procede por dezenas, por grupos de déz: \textunderscore systema decimal de pesos e medidas\textunderscore .
Que abrange décimas, centésimas, milésimas, etc.
Algarismo de fracção decimal.
\section{Decímanos}
\begin{itemize}
\item {Grp. gram.:m. pl.}
\end{itemize}
\begin{itemize}
\item {Proveniência:(Lat. \textunderscore decimanus\textunderscore )}
\end{itemize}
Soldados da décima legião, entre os Romanos.
\section{Decimar}
\textunderscore v. t.\textunderscore  (e der.)
(V. \textunderscore dizimar\textunderscore , etc.)
\section{Decimável}
\begin{itemize}
\item {Grp. gram.:adj.}
\end{itemize}
\begin{itemize}
\item {Proveniência:(De \textunderscore décima\textunderscore )}
\end{itemize}
Sujeito a décima, a tributo; tributável.
\section{Decímetro}
\begin{itemize}
\item {Grp. gram.:m.}
\end{itemize}
\begin{itemize}
\item {Proveniência:(De \textunderscore deci...\textunderscore  + \textunderscore metro\textunderscore )}
\end{itemize}
Décima parte de um metro.
Extensão, correspondente a essa medida.
\section{Décimo}
\begin{itemize}
\item {Grp. gram.:adj.}
\end{itemize}
\begin{itemize}
\item {Grp. gram.:M.}
\end{itemize}
\begin{itemize}
\item {Proveniência:(Lat. \textunderscore decimus\textunderscore )}
\end{itemize}
Que occupa o lugar de déz numa série de coisas ou pessóas: \textunderscore Leão Décimo\textunderscore .
Décima parte.
\section{Decisão}
\begin{itemize}
\item {Grp. gram.:f.}
\end{itemize}
\begin{itemize}
\item {Proveniência:(Lat. \textunderscore decisio\textunderscore )}
\end{itemize}
Acto ou effeito de decidir.
Sentença.
Resolução.
Coragem, intrepidez: \textunderscore portar-se com decisão\textunderscore .
\section{Decisivamente}
\begin{itemize}
\item {Grp. gram.:adv.}
\end{itemize}
De modo decisivo.
\section{Decisivo}
\begin{itemize}
\item {Grp. gram.:adj.}
\end{itemize}
\begin{itemize}
\item {Proveniência:(Do lat. \textunderscore decisus\textunderscore )}
\end{itemize}
Que decide.
Em que não há dúvida.
Que resolve.
Que termina.
Claro, terminante: \textunderscore opinião decisiva\textunderscore .
\section{Decisor}
\begin{itemize}
\item {Grp. gram.:m.  e  adj.}
\end{itemize}
\begin{itemize}
\item {Proveniência:(Do lat. \textunderscore decisus\textunderscore )}
\end{itemize}
O que decide. Cf. Pato Moniz, \textunderscore Apparição\textunderscore , 56.
\section{Decisoriamente}
\begin{itemize}
\item {Grp. gram.:adv.}
\end{itemize}
De modo decisório.
\section{Decisório}
\begin{itemize}
\item {Grp. gram.:adj.}
\end{itemize}
\begin{itemize}
\item {Utilização:Jur.}
\end{itemize}
\begin{itemize}
\item {Proveniência:(Do lat. \textunderscore decisus\textunderscore )}
\end{itemize}
Que tem o poder de decidir, (falando-se do juramento, de que se torna dependente a decisão de um processo judicial).
\section{Decistere}
\begin{itemize}
\item {Grp. gram.:m.}
\end{itemize}
Décima parte de um estere.
\section{Declamação}
\begin{itemize}
\item {Grp. gram.:f.}
\end{itemize}
\begin{itemize}
\item {Proveniência:(Lat. \textunderscore declamatio\textunderscore )}
\end{itemize}
Acto, modo ou arte de declamar.
Modo pomposo de discursar.
Palavreado oco.
\section{Declamador}
\begin{itemize}
\item {Grp. gram.:m.}
\end{itemize}
\begin{itemize}
\item {Proveniência:(Lat. \textunderscore declamator\textunderscore )}
\end{itemize}
Aquelle que declama.
\section{Declamante}
\begin{itemize}
\item {Grp. gram.:m.  e  adj.}
\end{itemize}
O que declama. Cf. Filinto, XVI, p. 30.
\section{Declamar}
\begin{itemize}
\item {Grp. gram.:v. t.}
\end{itemize}
\begin{itemize}
\item {Grp. gram.:V. i.}
\end{itemize}
\begin{itemize}
\item {Proveniência:(Lat. \textunderscore declamare\textunderscore )}
\end{itemize}
Recitar, segundo a arte de declamação: \textunderscore declamar um poemeto\textunderscore .
Falar alto e com solennidade.
Discursar affectadamente, em estilo empolado e sem ideias.
Falar com violência, contra alguém ou contra alguma coisa: \textunderscore declamar contra o Govêrno\textunderscore .
\section{Declamatoriamente}
\begin{itemize}
\item {Grp. gram.:adv.}
\end{itemize}
De modo declamatório.
\section{Declamatório}
\begin{itemize}
\item {Grp. gram.:adj.}
\end{itemize}
\begin{itemize}
\item {Proveniência:(Lat. \textunderscore declamatorius\textunderscore )}
\end{itemize}
Relativo a declamação.
Em que há declamação: \textunderscore discurso declamatório\textunderscore .
\section{Declaração}
\begin{itemize}
\item {Grp. gram.:f.}
\end{itemize}
\begin{itemize}
\item {Utilização:Fam.}
\end{itemize}
\begin{itemize}
\item {Proveniência:(Lat. \textunderscore declaratio\textunderscore )}
\end{itemize}
Acto ou effeito de declarar.
Aquillo que se declara.
Documento.
Confissão de amor: \textunderscore encontrou-a e fez-lhe uma declaração\textunderscore .
\section{Declaradamente}
\begin{itemize}
\item {Grp. gram.:adv.}
\end{itemize}
De modo declarado.
\section{Declaradas}
\begin{itemize}
\item {Grp. gram.:f. pl. Loc. adv.}
\end{itemize}
\textunderscore Ás declaradas\textunderscore , ás claras, publicamente:«\textunderscore ás encobertas do principe, mas depois ás declaradas\textunderscore ». Filinto, \textunderscore D. Man.\textunderscore , II, 11.
\section{Declarador}
\begin{itemize}
\item {Grp. gram.:m.  e  adj.}
\end{itemize}
O mesmo que \textunderscore declarante\textunderscore .
\section{Declaramento}
\begin{itemize}
\item {Grp. gram.:m.}
\end{itemize}
\begin{itemize}
\item {Utilização:Des.}
\end{itemize}
O mesmo que \textunderscore declaração\textunderscore .
\section{Declarante}
\begin{itemize}
\item {Grp. gram.:m.  e  adj.}
\end{itemize}
\begin{itemize}
\item {Proveniência:(Lat. \textunderscore declarans\textunderscore )}
\end{itemize}
O que declara.
\section{Declarar}
\begin{itemize}
\item {Grp. gram.:v. t.}
\end{itemize}
\begin{itemize}
\item {Proveniência:(Lat. \textunderscore declarare\textunderscore )}
\end{itemize}
Expor claramente.
Manifestar.
Confessar: \textunderscore declarar o seu crime\textunderscore .
Explicar.
Publicar, annunciar: \textunderscore declarar o que se projecta\textunderscore .
Pronunciar, referir.
Nomear: \textunderscore declarar os autores de uma aggressão\textunderscore .
\section{Declarativo}
\begin{itemize}
\item {Grp. gram.:adj.}
\end{itemize}
\begin{itemize}
\item {Proveniência:(Lat. \textunderscore declarativus\textunderscore )}
\end{itemize}
Em que há declaração.
\section{Declaratório}
\begin{itemize}
\item {Grp. gram.:adj.}
\end{itemize}
(V.declarativo)
\section{Declareza}
\begin{itemize}
\item {Grp. gram.:f.}
\end{itemize}
\begin{itemize}
\item {Utilização:Pop.}
\end{itemize}
O mesmo que \textunderscore declaração\textunderscore .
\section{Declina}
\begin{itemize}
\item {Grp. gram.:f.}
\end{itemize}
\begin{itemize}
\item {Proveniência:(De \textunderscore declinar\textunderscore )}
\end{itemize}
Régua, que mostra os graus no astrolábio.
\section{Declinação}
\begin{itemize}
\item {Grp. gram.:f.}
\end{itemize}
\begin{itemize}
\item {Utilização:Astron.}
\end{itemize}
\begin{itemize}
\item {Utilização:Gram.}
\end{itemize}
\begin{itemize}
\item {Proveniência:(Lat. \textunderscore declinatio\textunderscore )}
\end{itemize}
Acto de declinar.
Inclinação.
Arco de um círculo máximo da esphera, entre o equador e um determinado astro.
Medida de um ângulo, formado entre a direcção do meridiano e a de uma agulha magnética.
Flexão de substantivos, adjectivos e pronomes.
Cada uma das classes de palavras, que se declinam da mesma fórma: \textunderscore palavras da 1.^a declinação, em latim\textunderscore .
\section{Declinador}
\begin{itemize}
\item {Grp. gram.:m.}
\end{itemize}
\begin{itemize}
\item {Proveniência:(De \textunderscore declinar\textunderscore )}
\end{itemize}
Instrumento, que determina a declinação do plano de um quadrante.
\section{Declinante}
\begin{itemize}
\item {Grp. gram.:adj.}
\end{itemize}
\begin{itemize}
\item {Proveniência:(Lat. \textunderscore declinans\textunderscore )}
\end{itemize}
Que declina.
\section{Declinar}
\begin{itemize}
\item {Grp. gram.:v. i.}
\end{itemize}
\begin{itemize}
\item {Grp. gram.:V.}
\end{itemize}
\begin{itemize}
\item {Utilização:t. Gram.}
\end{itemize}
\begin{itemize}
\item {Proveniência:(Lat. \textunderscore declinare\textunderscore )}
\end{itemize}
Desviar-se de um rumo ou de um ponto.
Inclinar-se.
Deminuir.
Avizinhar-se do seu termo: \textunderscore o sol declina\textunderscore .
Decaír: \textunderscore foi declinando a grandeza do império\textunderscore .
Enunciar as flexões de (substantivos, adjectivos ou pronomes).
Rejeitar, recusar: \textunderscore declinar a offerta\textunderscore .
Lançar de si, eximir-se a.
Desviar.
Rebaixar, abater.
\section{Declinar}
\begin{itemize}
\item {Grp. gram.:v. t.}
\end{itemize}
\begin{itemize}
\item {Utilização:Prov.}
\end{itemize}
Enxergar, divisar.
(Colhido em Turquel)
\section{Declinativo}
\begin{itemize}
\item {Grp. gram.:adj.}
\end{itemize}
\begin{itemize}
\item {Utilização:Philol.}
\end{itemize}
Diz-se das línguas, em que há declinações, segundo a classificação de Steinthal. Cf. C. Figueiredo, \textunderscore Man. da Sc. da Ling.\textunderscore , 203.
\section{Declinatória}
\begin{itemize}
\item {Grp. gram.:f.}
\end{itemize}
\begin{itemize}
\item {Proveniência:(De \textunderscore declinatório\textunderscore )}
\end{itemize}
Acto de declinar ou recusar a jurisdicção de um tribunal ou juiz.
Instrumento, semelhante á bússola, que dá precisamente a declinação da agulha magnética, e que se usa no levantamento dos planos, para orientar a plancheta.
\section{Declinatório}
\begin{itemize}
\item {Grp. gram.:adj.}
\end{itemize}
\begin{itemize}
\item {Proveniência:(Do rad. do lat. \textunderscore declinatus\textunderscore )}
\end{itemize}
Que declina.
Próprio para declinar jurisdicção.
\section{Declinável}
\begin{itemize}
\item {Grp. gram.:adj.}
\end{itemize}
\begin{itemize}
\item {Proveniência:(Lat. \textunderscore declinabilis\textunderscore )}
\end{itemize}
Que se póde declinar.
\section{Declínio}
\begin{itemize}
\item {Grp. gram.:m.}
\end{itemize}
\begin{itemize}
\item {Utilização:Des.}
\end{itemize}
Acto de declinar.
\section{De feito}
\begin{itemize}
\item {Grp. gram.:loc. adv.}
\end{itemize}
O mesmo que \textunderscore de-facto\textunderscore .
\section{Declinoso}
\begin{itemize}
\item {Grp. gram.:adj.}
\end{itemize}
Em que há declinação ou inclinação. Cf. Camillo, \textunderscore O Bem e O Mal\textunderscore , 92.
\section{Declivado}
\begin{itemize}
\item {Grp. gram.:adj.}
\end{itemize}
\begin{itemize}
\item {Proveniência:(De \textunderscore declivar\textunderscore )}
\end{itemize}
O mesmo que \textunderscore declivoso\textunderscore .
\section{Declivar}
\begin{itemize}
\item {Grp. gram.:v. i.}
\end{itemize}
\begin{itemize}
\item {Grp. gram.:V. t.}
\end{itemize}
Formar declive.
Tornar declive ou íngreme:«\textunderscore contribuira a declivar-lhe a ladeira\textunderscore ». Camillo, \textunderscore Filha do Reg.\textunderscore , 49.
\section{Declive}
\begin{itemize}
\item {Grp. gram.:adj.}
\end{itemize}
\begin{itemize}
\item {Utilização:Fig.}
\end{itemize}
\begin{itemize}
\item {Grp. gram.:M.}
\end{itemize}
\begin{itemize}
\item {Proveniência:(Lat. \textunderscore declivis\textunderscore )}
\end{itemize}
Inclinado; que fórma ladeira.
Que tende; que propende.
Pendor, inclinação de terreno.
\section{Declividade}
\begin{itemize}
\item {Grp. gram.:f.}
\end{itemize}
\begin{itemize}
\item {Proveniência:(Lat. \textunderscore declivitas\textunderscore )}
\end{itemize}
O mesmo que \textunderscore declive\textunderscore , \textunderscore m.\textunderscore 
\section{Declívio}
\begin{itemize}
\item {Grp. gram.:m.}
\end{itemize}
O mesmo que \textunderscore declive\textunderscore , \textunderscore m.\textunderscore 
\section{Declivoso}
\begin{itemize}
\item {Grp. gram.:adj.}
\end{itemize}
Em que há declive; ladeirento.
\section{Decoada}
\begin{itemize}
\item {Grp. gram.:f.}
\end{itemize}
\begin{itemize}
\item {Proveniência:(De \textunderscore de...\textunderscore  + \textunderscore coada\textunderscore )}
\end{itemize}
Lixívia.
Acto de coar a lixívia.
\section{Decoar}
\begin{itemize}
\item {Grp. gram.:v. t.}
\end{itemize}
Meter em decoada, pôr na barrela. Cf. \textunderscore Bibl. da Gente do Campo\textunderscore , 437.
(Cp. \textunderscore decoada\textunderscore )
\section{Decocção}
\begin{itemize}
\item {Grp. gram.:f.}
\end{itemize}
\begin{itemize}
\item {Proveniência:(Lat. \textunderscore decoctio\textunderscore )}
\end{itemize}
Acto de fazer ferver num líquido substâncias, a que se deseja extrahir os princípios solúveis.
\section{Decocto}
\begin{itemize}
\item {Grp. gram.:m.}
\end{itemize}
\begin{itemize}
\item {Proveniência:(Lat. \textunderscore decoctum\textunderscore )}
\end{itemize}
Cozimento.
Producto de decocção.
\section{Decolorar}
\textunderscore v. t.\textunderscore  (e der.)
(V. \textunderscore descolorar\textunderscore , etc.)
\section{Decombros}
\begin{itemize}
\item {Grp. gram.:m. pl.}
\end{itemize}
\begin{itemize}
\item {Utilização:Gal}
\end{itemize}
\begin{itemize}
\item {Proveniência:(Fr. \textunderscore décombres\textunderscore )}
\end{itemize}
O mesmo que \textunderscore escombros\textunderscore . Cf. Cortesão, \textunderscore Subs\textunderscore .
\section{Decomponente}
\begin{itemize}
\item {Grp. gram.:adj.}
\end{itemize}
Que decompõe.
(Cp. \textunderscore decompor\textunderscore )
\section{Decomponível}
\begin{itemize}
\item {Grp. gram.:adj.}
\end{itemize}
Que póde sêr decomposto.
\section{Decompor}
\begin{itemize}
\item {Grp. gram.:v. t.}
\end{itemize}
\begin{itemize}
\item {Proveniência:(De \textunderscore de\textunderscore  + \textunderscore compor\textunderscore )}
\end{itemize}
Separar as partes constitutivas de.
Modificar profundamente, alterar.
Estragar.
\section{Decomposição}
\begin{itemize}
\item {Grp. gram.:f.}
\end{itemize}
Acto ou effeito de decompor.
\section{De-cór}
\begin{itemize}
\item {Grp. gram.:loc. adv.}
\end{itemize}
\begin{itemize}
\item {Proveniência:(Do lat. \textunderscore cor\textunderscore , seg. alguns; e, mais provavelmente, da loc. cast. \textunderscore de coro\textunderscore . Cf. G. Viana, \textunderscore Apostilas\textunderscore , vb. \textunderscore decorar\textunderscore )}
\end{itemize}
De memória, de cabeça: \textunderscore repetir os«Lusiadas»de-cór\textunderscore .
\section{Decoração}
\begin{itemize}
\item {Grp. gram.:f.}
\end{itemize}
Acto ou effeito de decorar^2.
\section{Decorador}
\begin{itemize}
\item {Grp. gram.:m.}
\end{itemize}
Aquelle, que decora.
\section{Decoramento}
\begin{itemize}
\item {Grp. gram.:m.}
\end{itemize}
O mesmo que \textunderscore decoração\textunderscore .
\section{Decorar}
\begin{itemize}
\item {Grp. gram.:v. t.}
\end{itemize}
\begin{itemize}
\item {Proveniência:(De \textunderscore de\textunderscore  + \textunderscore cór\textunderscore )}
\end{itemize}
Aprender de cór; reter na memória: \textunderscore decorar versos\textunderscore .
\section{Decorar}
\begin{itemize}
\item {Grp. gram.:v. t.}
\end{itemize}
\begin{itemize}
\item {Proveniência:(Lat. \textunderscore decorare\textunderscore )}
\end{itemize}
Enfeitar, adornar.
Embellezar com pinturas, com estofos, etc.: \textunderscore decorar uma sala\textunderscore .
\section{Decorativo}
\begin{itemize}
\item {Grp. gram.:adj.}
\end{itemize}
\begin{itemize}
\item {Proveniência:(Do lat. \textunderscore decoratus\textunderscore )}
\end{itemize}
Que serve para decorar^2.
Que enfeita.
\section{Decoro}
\begin{itemize}
\item {Grp. gram.:m.}
\end{itemize}
\begin{itemize}
\item {Proveniência:(Lat. \textunderscore decorus\textunderscore )}
\end{itemize}
Belleza moral, que resulta da honestidade e decência.
Decência; honra.
Pundonor: \textunderscore homem sem decoro\textunderscore .
\section{Decorosamente}
\begin{itemize}
\item {Grp. gram.:adv.}
\end{itemize}
De modo decoroso.
\section{Decoroso}
\begin{itemize}
\item {Grp. gram.:adj.}
\end{itemize}
\begin{itemize}
\item {Proveniência:(Lat. \textunderscore decorosus\textunderscore )}
\end{itemize}
Que tem decoro; em que há decoro.
Digno, decente: \textunderscore actos decorosos\textunderscore .
Honroso.
\section{Decorrente}
\begin{itemize}
\item {Grp. gram.:adj.}
\end{itemize}
\begin{itemize}
\item {Proveniência:(Lat. \textunderscore decurrens\textunderscore )}
\end{itemize}
Que decorre.
\section{Decorrer}
\begin{itemize}
\item {Grp. gram.:v. i.}
\end{itemize}
\begin{itemize}
\item {Proveniência:(Lat. \textunderscore decurrere\textunderscore )}
\end{itemize}
Passar (o tempo).
Succeder-se.
\section{Decorrido}
\begin{itemize}
\item {Grp. gram.:adj.}
\end{itemize}
\begin{itemize}
\item {Proveniência:(De \textunderscore decorrer\textunderscore )}
\end{itemize}
Que decorreu; findo.
\section{Decorticação}
\begin{itemize}
\item {Grp. gram.:f.}
\end{itemize}
\begin{itemize}
\item {Proveniência:(Lat. \textunderscore decorticatio\textunderscore )}
\end{itemize}
Acto de decorticar.
\section{Decorticar}
\begin{itemize}
\item {Grp. gram.:v. t.}
\end{itemize}
\begin{itemize}
\item {Proveniência:(Lat. \textunderscore decorticare\textunderscore )}
\end{itemize}
Tirar a cortiça a.
Descascar.
\section{Decotador}
\begin{itemize}
\item {Grp. gram.:m.  e  adj.}
\end{itemize}
O que decota.
\section{Decotar}
\begin{itemize}
\item {Grp. gram.:v. t.}
\end{itemize}
\begin{itemize}
\item {Grp. gram.:V. p.}
\end{itemize}
\begin{itemize}
\item {Proveniência:(De \textunderscore decote\textunderscore )}
\end{itemize}
Cortar por cima ou em volta.
Aparar (ramos de árvores).
Cortar superiormente (vestuário de mulher, ficando mais ou menos descoberto o pescoço e as espáduas).
Trajar, descobrindo o pescoço e ombros.
\section{Decote}
\begin{itemize}
\item {Grp. gram.:m.}
\end{itemize}
\begin{itemize}
\item {Proveniência:(Do lat. \textunderscore decotes\textunderscore )}
\end{itemize}
Acto ou effeito de decotar.
Córte da ponta do sarmento, logo depois da floração, para concentrar a seiva nos olhos, que hão de constituir no anno seguinte os gomos fructíferos. Cp. \textunderscore belisca\textunderscore .
\section{Decrecer}
\textunderscore v. i.\textunderscore  (e der.)
O mesmo que \textunderscore decrescer\textunderscore , etc.
\section{Decremento}
\begin{itemize}
\item {Grp. gram.:m.}
\end{itemize}
\begin{itemize}
\item {Utilização:Des.}
\end{itemize}
\begin{itemize}
\item {Proveniência:(Lat. \textunderscore decrementum\textunderscore )}
\end{itemize}
O mesmo que \textunderscore decrescimento\textunderscore .
\section{Decrepidez}
\begin{itemize}
\item {Grp. gram.:f.}
\end{itemize}
Estado de quem ou daquillo que é decrépito; caducidade.
\section{Decrepitação}
\begin{itemize}
\item {Grp. gram.:f.}
\end{itemize}
Effeito de decrepitar.
\section{Decrepitar}
\begin{itemize}
\item {Grp. gram.:v. i.}
\end{itemize}
\begin{itemize}
\item {Utilização:Des.}
\end{itemize}
\begin{itemize}
\item {Grp. gram.:V. t.}
\end{itemize}
\begin{itemize}
\item {Proveniência:(De \textunderscore decrépito\textunderscore )}
\end{itemize}
Decaír, tornar-se decrépito.
Tornar decrépito.
Tornar velho. Cf. \textunderscore Viriato Trágico\textunderscore , III, 3.
\section{Decrépito}
\begin{itemize}
\item {Grp. gram.:adj.}
\end{itemize}
\begin{itemize}
\item {Proveniência:(Lat. \textunderscore decrepitus\textunderscore )}
\end{itemize}
Que é muito velho, muito idoso.
Arruinado, gasto, fraco.
\section{Decrepitude}
\begin{itemize}
\item {Grp. gram.:f.}
\end{itemize}
O mesmo que \textunderscore decrepidez\textunderscore .
\section{Decrescença}
\begin{itemize}
\item {Grp. gram.:f.}
\end{itemize}
O mesmo que \textunderscore decrescimento\textunderscore .
\section{Decrescendo}
\begin{itemize}
\item {Grp. gram.:adv.}
\end{itemize}
\begin{itemize}
\item {Utilização:Mús.}
\end{itemize}
\begin{itemize}
\item {Grp. gram.:M.}
\end{itemize}
\begin{itemize}
\item {Proveniência:(It. \textunderscore decrescendo\textunderscore )}
\end{itemize}
Deminuindo-se a intensidade do som.
Acto de deminuir a intensidade do som.
\section{Decrescente}
\begin{itemize}
\item {Grp. gram.:adj.}
\end{itemize}
\begin{itemize}
\item {Proveniência:(Lat. \textunderscore decrescens\textunderscore )}
\end{itemize}
Que decresce.
\section{Decrescer}
\begin{itemize}
\item {Grp. gram.:v. i.}
\end{itemize}
\begin{itemize}
\item {Proveniência:(Lat. \textunderscore decrescere\textunderscore )}
\end{itemize}
Tornar-se menor, deminuir: \textunderscore já vão decrescendo os dias\textunderscore .
\section{Decrescimento}
\begin{itemize}
\item {Grp. gram.:m.}
\end{itemize}
Acto de decrescer.
\section{Decretação}
\begin{itemize}
\item {Grp. gram.:f.}
\end{itemize}
Acto de decretar.
\section{Decretal}
\begin{itemize}
\item {Grp. gram.:f.}
\end{itemize}
\begin{itemize}
\item {Proveniência:(Lat. \textunderscore decretalis\textunderscore )}
\end{itemize}
Antiga Carta ou Constituição pontifícia, em resposta a consultas sôbre moral ou direito.
\section{Decretalista}
\begin{itemize}
\item {Grp. gram.:m.}
\end{itemize}
\begin{itemize}
\item {Proveniência:(De \textunderscore decretal\textunderscore )}
\end{itemize}
Jurisconsulto, versado em decretaes.
\section{Decretalmente}
\begin{itemize}
\item {Grp. gram.:adv.}
\end{itemize}
\begin{itemize}
\item {Proveniência:(De \textunderscore decretal\textunderscore )}
\end{itemize}
Por meio de decreto.
\section{Decretamento}
\begin{itemize}
\item {Grp. gram.:m.}
\end{itemize}
O mesmo que \textunderscore decretação\textunderscore .
\section{Decretar}
\begin{itemize}
\item {Grp. gram.:v. t.}
\end{itemize}
\begin{itemize}
\item {Utilização:Fig.}
\end{itemize}
\begin{itemize}
\item {Proveniência:(De \textunderscore decreto\textunderscore )}
\end{itemize}
Ordenar por decreto: \textunderscore decretar amnistia\textunderscore .
Determinar; mandar solennemente.
\section{Decreto}
\begin{itemize}
\item {Grp. gram.:m.}
\end{itemize}
\begin{itemize}
\item {Proveniência:(Lat. \textunderscore decretum\textunderscore )}
\end{itemize}
Determinação escrita, emanada do chefe do Estado, ou do poder executivo, ou de outra autoridade superior.
Qualquer lei, que não foi feita por Parlamento.
Ordenação.
Desígnio, vontade (de Deus, da Providência).
\section{Decretoriamente}
\begin{itemize}
\item {Grp. gram.:adv.}
\end{itemize}
De modo decretório.
\section{Decretório}
\begin{itemize}
\item {Grp. gram.:adj.}
\end{itemize}
\begin{itemize}
\item {Proveniência:(Lat. \textunderscore decretorius\textunderscore )}
\end{itemize}
Decisivo; que resolve.
\section{Decrinar}
\begin{itemize}
\item {Grp. gram.:v. t.}
\end{itemize}
\begin{itemize}
\item {Utilização:Ant.}
\end{itemize}
O mesmo que \textunderscore declinar\textunderscore ^1.
\section{Decroar}
\begin{itemize}
\item {Grp. gram.:v. t.}
\end{itemize}
\begin{itemize}
\item {Proveniência:(De \textunderscore crôa\textunderscore )}
\end{itemize}
Revolver ou desfazer a crôa a.
Dar a primeira lavra a (um terreno).
\section{Decrua}
\begin{itemize}
\item {Grp. gram.:f.}
\end{itemize}
\begin{itemize}
\item {Utilização:Prov.}
\end{itemize}
\begin{itemize}
\item {Utilização:minh.}
\end{itemize}
Acto de decruar.
Sova, tareia.
\section{Decruagem}
\begin{itemize}
\item {Grp. gram.:f.}
\end{itemize}
O mesmo que \textunderscore decrua\textunderscore .
\section{Decruar}
\begin{itemize}
\item {Grp. gram.:v. t.}
\end{itemize}
\begin{itemize}
\item {Proveniência:(De \textunderscore de...\textunderscore  + \textunderscore cru\textunderscore )}
\end{itemize}
Cozer ligeiramente.
Lavar (seda crua).
\section{Decúbito}
\begin{itemize}
\item {Grp. gram.:m.}
\end{itemize}
\begin{itemize}
\item {Proveniência:(Lat. \textunderscore decubitus\textunderscore )}
\end{itemize}
Posição de quem está deitado.
\section{Decúmano}
\begin{itemize}
\item {Grp. gram.:adj.}
\end{itemize}
\begin{itemize}
\item {Utilização:Des.}
\end{itemize}
\begin{itemize}
\item {Proveniência:(Lat. \textunderscore decumamus\textunderscore )}
\end{itemize}
O mesmo que \textunderscore décimo\textunderscore .
\section{Decumbente}
\begin{itemize}
\item {Grp. gram.:adj.}
\end{itemize}
\begin{itemize}
\item {Utilização:Des.}
\end{itemize}
\begin{itemize}
\item {Proveniência:(Lat. \textunderscore decumbens\textunderscore )}
\end{itemize}
Inclinado, deitado.
\section{Decuplar}
\begin{itemize}
\item {Grp. gram.:v. t.}
\end{itemize}
\begin{itemize}
\item {Proveniência:(Lat. \textunderscore decuplare\textunderscore )}
\end{itemize}
Multiplicar déz vezes.
Tornar déz vezes maior.
\section{Decuplicar}
\begin{itemize}
\item {Grp. gram.:v. t.}
\end{itemize}
\begin{itemize}
\item {Utilização:P. us.}
\end{itemize}
O mesmo que \textunderscore decuplar\textunderscore .
\section{Décuplo}
\begin{itemize}
\item {Grp. gram.:adj.}
\end{itemize}
\begin{itemize}
\item {Proveniência:(Lat. \textunderscore decuplus\textunderscore )}
\end{itemize}
Que contém déz vezes uma quantidade.
Déz vezes maior.
\section{Décuplo}
\begin{itemize}
\item {Grp. gram.:m.}
\end{itemize}
\begin{itemize}
\item {Proveniência:(Lat. \textunderscore decuplum\textunderscore )}
\end{itemize}
Quantidade décupla.
\section{Decúria}
\begin{itemize}
\item {Grp. gram.:f.}
\end{itemize}
\begin{itemize}
\item {Proveniência:(Lat. \textunderscore decuria\textunderscore )}
\end{itemize}
Quantidade de déz coisas.
Número de dez homens.
Corpo militar entre os Romanos, formado de cavallaria ou de infantaria.
Classe de alumnos numa escola, dirigidos por um dos mais adiantados, que se chama decurião.
\section{Decuriado}
\begin{itemize}
\item {Grp. gram.:m.}
\end{itemize}
\begin{itemize}
\item {Proveniência:(Lat. \textunderscore decuriatus\textunderscore )}
\end{itemize}
Cargo de decurião.
\section{Decurião}
\begin{itemize}
\item {Grp. gram.:m.}
\end{itemize}
\begin{itemize}
\item {Proveniência:(Lat. \textunderscore decurio\textunderscore )}
\end{itemize}
Chefe ou director de decúria.
Estudante, que, numa escola, dirige uma classe de alumnos ou lhes toma lição.
\section{Decuriato}
\begin{itemize}
\item {Grp. gram.:m.}
\end{itemize}
O mesmo que \textunderscore decuriado\textunderscore .
\section{Decurionato}
\begin{itemize}
\item {Grp. gram.:m.}
\end{itemize}
O mesmo que \textunderscore decuriado\textunderscore . Cf. Herculano, \textunderscore Hist. de Port.\textunderscore , IV, p. 14.
\section{Decurr...}
(V.decorr...)
\section{Decursivamente}
\begin{itemize}
\item {Grp. gram.:adv.}
\end{itemize}
De modo decursivo.
\section{Decursivo}
\begin{itemize}
\item {Grp. gram.:adv.}
\end{itemize}
(V.decorrente)
\section{Decurso}
\begin{itemize}
\item {Grp. gram.:m.}
\end{itemize}
\begin{itemize}
\item {Proveniência:(Lat. \textunderscore decursus\textunderscore )}
\end{itemize}
Successão.
Acto de decorrer; percurso.
\section{Decússis}
\begin{itemize}
\item {Grp. gram.:m.}
\end{itemize}
\begin{itemize}
\item {Proveniência:(Lat. \textunderscore decussis\textunderscore )}
\end{itemize}
Antiga moéda romana de déz asses.
\section{Dedada}
\begin{itemize}
\item {Grp. gram.:f.}
\end{itemize}
\begin{itemize}
\item {Proveniência:(De \textunderscore dedo\textunderscore )}
\end{itemize}
Porção de uma substância, que adheriu a um dedo.
Nódoa, que o dedo deixa num objecto.
\section{Dedal}
\begin{itemize}
\item {Grp. gram.:m.}
\end{itemize}
\begin{itemize}
\item {Grp. gram.:Pl.}
\end{itemize}
\begin{itemize}
\item {Proveniência:(Do lat. \textunderscore digitalis\textunderscore )}
\end{itemize}
Utensílio, que se encaixa no terceiro dedo da mão direita de quem cose, para empurrar a agulha sem que esta fira o dedo.
Jôgo popular.
\section{Dedalário}
\begin{itemize}
\item {Grp. gram.:m.}
\end{itemize}
\begin{itemize}
\item {Utilização:Bot.}
\end{itemize}
O mesmo que \textunderscore dedaleira\textunderscore .
\section{Dedaleira}
\begin{itemize}
\item {Grp. gram.:f.}
\end{itemize}
\begin{itemize}
\item {Proveniência:(De \textunderscore dedal\textunderscore )}
\end{itemize}
Planta escrofularínea, medicinal, (\textunderscore digitalis purpurea\textunderscore ).
O mesmo que \textunderscore digital\textunderscore .
Estojo de dedal.
\section{Dedáleo}
\begin{itemize}
\item {Grp. gram.:adj.}
\end{itemize}
\begin{itemize}
\item {Proveniência:(Lat. \textunderscore daedaleus\textunderscore )}
\end{itemize}
Relativo a dédalo.
Intrincado.
Engenhoso.
\section{Dédalo}
\begin{itemize}
\item {Grp. gram.:m.}
\end{itemize}
\begin{itemize}
\item {Proveniência:(De \textunderscore Dédalo\textunderscore , n. p.)}
\end{itemize}
Labyrintho.
Encruzamento confuso.
Confusão.
\section{Dédalo}
\begin{itemize}
\item {Grp. gram.:adj.}
\end{itemize}
\begin{itemize}
\item {Utilização:Ant.}
\end{itemize}
\begin{itemize}
\item {Proveniência:(Lat. \textunderscore daedalus\textunderscore )}
\end{itemize}
Ornado ricamente.
Esmaltado de flôres.
Florígero. Cf. \textunderscore Laura de Anfriso\textunderscore , 96, v.^o.
\section{Dedecorar}
\begin{itemize}
\item {Grp. gram.:v. t.}
\end{itemize}
\begin{itemize}
\item {Proveniência:(Lat. \textunderscore dedecorare\textunderscore )}
\end{itemize}
Tornar indecoroso, deshonesto.
\section{Dedeira}
\begin{itemize}
\item {Grp. gram.:f.}
\end{itemize}
\begin{itemize}
\item {Proveniência:(De \textunderscore dedo\textunderscore )}
\end{itemize}
Pedaço de pano ou coiro, com que se reveste o dedo.
\section{Dédica}
\begin{itemize}
\item {Grp. gram.:f.}
\end{itemize}
O mesmo que \textunderscore dedicação\textunderscore . Cf. Macedo, \textunderscore Couto\textunderscore , 12.
(Der. extravagante de \textunderscore dedicar\textunderscore )
\section{Dedicação}
\begin{itemize}
\item {Grp. gram.:f.}
\end{itemize}
\begin{itemize}
\item {Proveniência:(Lat. \textunderscore dedicatio\textunderscore )}
\end{itemize}
Qualidade de quem se dedica.
Acto de dedicar.
\section{Dedicador}
\begin{itemize}
\item {Grp. gram.:m.}
\end{itemize}
\begin{itemize}
\item {Proveniência:(Lat. \textunderscore dedicator\textunderscore )}
\end{itemize}
Aquelle que dedica.
\section{Dedicar}
\begin{itemize}
\item {Grp. gram.:v. t.}
\end{itemize}
\begin{itemize}
\item {Grp. gram.:V. p.}
\end{itemize}
\begin{itemize}
\item {Proveniência:(Lat. \textunderscore dedicare\textunderscore )}
\end{itemize}
Consagrar, votar, tributar: \textunderscore dedicar affeição\textunderscore .
Offerecer affectuosamente (um livro, etc.).
Applicar.
Estar sinceramente disposto a servir alguém.
Sacrificar-se por alguém.
Destinar-se: \textunderscore dedicar-se á vida diplomática\textunderscore .
\section{Dedicatória}
\begin{itemize}
\item {Grp. gram.:f.}
\end{itemize}
\begin{itemize}
\item {Proveniência:(Do rad. do lat. \textunderscore dedicatus\textunderscore )}
\end{itemize}
Palavras escritas, com que se dedica a alguém uma publicação ou trabalho escrito.
\section{Dedignação}
\begin{itemize}
\item {Grp. gram.:f.}
\end{itemize}
\begin{itemize}
\item {Proveniência:(Lat. \textunderscore dedignatio\textunderscore )}
\end{itemize}
Acto de dedignar-se.
\section{Dedignar-se}
\begin{itemize}
\item {Grp. gram.:v. p.}
\end{itemize}
\begin{itemize}
\item {Proveniência:(Lat. \textunderscore dedignari\textunderscore )}
\end{itemize}
Julgar indigno de si: \textunderscore não se dedigne de me responder\textunderscore .
Rebaixar-se.
\section{Dedilhação}
\begin{itemize}
\item {Grp. gram.:f.}
\end{itemize}
O mesmo que \textunderscore dedilhamento\textunderscore .
\section{Dedilhamento}
\begin{itemize}
\item {Grp. gram.:m.}
\end{itemize}
Acto de dedilhar. Cf. Ortigão, \textunderscore Hollanda\textunderscore , 35.
\section{Dedilhar}
\begin{itemize}
\item {Grp. gram.:v. t.}
\end{itemize}
\begin{itemize}
\item {Proveniência:(De \textunderscore dedo\textunderscore )}
\end{itemize}
Fazer vibrar com os dedos: \textunderscore dedilhar uma harpa\textunderscore .
Executar com os dedos num instrumento de cordas: \textunderscore dedilhar músicas populares\textunderscore .
\section{Dediróseo}
\begin{itemize}
\item {fónica:ró}
\end{itemize}
\begin{itemize}
\item {Grp. gram.:adj.}
\end{itemize}
Que tem dedos rosados:«\textunderscore ... a dedirósea aurora\textunderscore ». Filinto, X, 5.
\section{Dedirróseo}
\begin{itemize}
\item {Grp. gram.:adj.}
\end{itemize}
Que tem dedos rosados:«\textunderscore ... a dedirósea aurora\textunderscore ». Filinto, X, 5.
\section{Dedo}
\begin{itemize}
\item {Grp. gram.:m.}
\end{itemize}
\begin{itemize}
\item {Utilização:Fig.}
\end{itemize}
\begin{itemize}
\item {Grp. gram.:Loc.}
\end{itemize}
\begin{itemize}
\item {Utilização:fam.}
\end{itemize}
\begin{itemize}
\item {Proveniência:(Lat. \textunderscore digitus\textunderscore )}
\end{itemize}
Cada uma das partes distintas e articuladas, que terminam as mãos e os pés do homem.
Cada um dos prolongamentos, que terminam os pés de alguns animaes.
Cada uma das partes da luva, correspondentes aos dedos.
Extensão, equivalente á largura de um dedo.
Aptidão.
Vestígio de aptidões.
Poder dirigente.
\textunderscore Dois dedos de conversa\textunderscore , um pouco de cavaco, de parola.
\section{Dedo-de-alicante}
\begin{itemize}
\item {Grp. gram.:m.}
\end{itemize}
Casta de uva branca e temporan do Cartaxo.
\section{Dedo-de-dama}
\begin{itemize}
\item {Grp. gram.:m.}
\end{itemize}
Casta de uva alambreada clara, de bagos grandes e compridos.
\section{Dedução}
\begin{itemize}
\item {Grp. gram.:f.}
\end{itemize}
\begin{itemize}
\item {Utilização:Mús.}
\end{itemize}
\begin{itemize}
\item {Utilização:ant.}
\end{itemize}
\begin{itemize}
\item {Proveniência:(Lat. \textunderscore deductio\textunderscore )}
\end{itemize}
Acto ou efeito de deduzir.
Série de notas, que sobem diatonicamente.
\section{Deducção}
\begin{itemize}
\item {Grp. gram.:f.}
\end{itemize}
\begin{itemize}
\item {Utilização:Mús.}
\end{itemize}
\begin{itemize}
\item {Utilização:ant.}
\end{itemize}
\begin{itemize}
\item {Proveniência:(Lat. \textunderscore deductio\textunderscore )}
\end{itemize}
Acto ou effeito de deduzir.
Série de notas, que sobem diatonicamente.
\section{Deductivo}
\begin{itemize}
\item {Grp. gram.:adj.}
\end{itemize}
\begin{itemize}
\item {Proveniência:(Lat. \textunderscore deductivus\textunderscore )}
\end{itemize}
Que procede por deducção.
\section{Dedutivo}
\begin{itemize}
\item {Grp. gram.:adj.}
\end{itemize}
\begin{itemize}
\item {Proveniência:(Lat. \textunderscore deductivus\textunderscore )}
\end{itemize}
Que procede por dedução.
\section{Deduzir}
\begin{itemize}
\item {Grp. gram.:v. t.}
\end{itemize}
\begin{itemize}
\item {Proveniência:(Lat. \textunderscore deducere\textunderscore )}
\end{itemize}
Deminuir.
Descontar: \textunderscore deduzir duas prestações de uma dívida\textunderscore .
Enumerar minuciosamente.
Inferir; tirar como consequência: \textunderscore dêsse arrozado nada se deduz\textunderscore .
Concluir.
\section{De-facto}
\begin{itemize}
\item {Grp. gram.:loc. adv.}
\end{itemize}
Effectivamente; realmente.
Com effeito.
\section{Defecação}
\begin{itemize}
\item {Grp. gram.:f.}
\end{itemize}
\begin{itemize}
\item {Proveniência:(Lat. \textunderscore defecatio\textunderscore )}
\end{itemize}
Acto de defecar.
\section{Defecador}
\begin{itemize}
\item {Grp. gram.:m.}
\end{itemize}
\begin{itemize}
\item {Proveniência:(De \textunderscore defecar\textunderscore )}
\end{itemize}
Vaso ou outro utensílio, em que se desonera o ventre; bacio. Cf. \textunderscore Inquérito Industrial\textunderscore , 2.^a P., liv. III, 290.
\section{Defecar}
\begin{itemize}
\item {Grp. gram.:v. t.}
\end{itemize}
\begin{itemize}
\item {Grp. gram.:V. i.}
\end{itemize}
\begin{itemize}
\item {Grp. gram.:V. p.}
\end{itemize}
\begin{itemize}
\item {Proveniência:(Lat. \textunderscore defaecare\textunderscore )}
\end{itemize}
Purificar, separar as fezes de.
Limpar.
Expellir naturalmente os excrementos.
Emmagrecer.
\section{Defecatório}
\begin{itemize}
\item {Grp. gram.:adj.}
\end{itemize}
\begin{itemize}
\item {Proveniência:(De \textunderscore defecar\textunderscore )}
\end{itemize}
Que defeca ou faz defecar.
\section{Defecção}
\begin{itemize}
\item {Grp. gram.:f.}
\end{itemize}
\begin{itemize}
\item {Proveniência:(Lat. \textunderscore defectio\textunderscore )}
\end{itemize}
Desapparecimento.
Deserção.
Rebellião.
Apostasía.
\section{Defectibilidade}
\begin{itemize}
\item {Grp. gram.:f.}
\end{itemize}
Qualidade daquillo que é defectível.
\section{Defectível}
\begin{itemize}
\item {Grp. gram.:adj.}
\end{itemize}
\begin{itemize}
\item {Proveniência:(Do rad. do lat. \textunderscore defectus\textunderscore )}
\end{itemize}
Que tem defeito.
Que póde enganar-se.
\section{Defectivo}
\begin{itemize}
\item {Grp. gram.:adj.}
\end{itemize}
\begin{itemize}
\item {Utilização:Gram.}
\end{itemize}
\begin{itemize}
\item {Proveniência:(Lat. \textunderscore defectivus\textunderscore )}
\end{itemize}
Em que falta alguma coisa.
Imperfeito.
Que não tem algum número, caso, tempo, modo ou pessoa.
\section{Defectuoso}
\begin{itemize}
\item {Grp. gram.:adj.}
\end{itemize}
(V.defeituoso)
\section{Defedação}
\begin{itemize}
\item {Grp. gram.:f.}
\end{itemize}
\begin{itemize}
\item {Proveniência:(Do lat. \textunderscore de\textunderscore  + \textunderscore foedatio\textunderscore )}
\end{itemize}
Mancha da pelle. Cf. \textunderscore Âncora Med.\textunderscore , 58.
\section{Defeito}
\begin{itemize}
\item {Grp. gram.:m.}
\end{itemize}
\begin{itemize}
\item {Utilização:Prov.}
\end{itemize}
\begin{itemize}
\item {Proveniência:(Lat. \textunderscore defectus\textunderscore )}
\end{itemize}
Falta de alguma coisa.
Imperfeição: \textunderscore não há ninguém sem defeitos\textunderscore .
Deformidade: \textunderscore um defeito numa perna\textunderscore .
Mancha.
Vício.
Obstáculo.
\section{De-feito}
\begin{itemize}
\item {Grp. gram.:loc. adv.}
\end{itemize}
O mesmo que \textunderscore de-facto\textunderscore .
\section{Defeituosamente}
\begin{itemize}
\item {Grp. gram.:adv.}
\end{itemize}
De modo defeituoso.
\section{Defeituoso}
\begin{itemize}
\item {Grp. gram.:adj.}
\end{itemize}
Que tem defeito; em que há defeito: \textunderscore obra defeituosa\textunderscore .
\section{Defendente}
\begin{itemize}
\item {Grp. gram.:m.  e  adj.}
\end{itemize}
\begin{itemize}
\item {Proveniência:(Lat. \textunderscore defendens\textunderscore )}
\end{itemize}
Aquelle que defende.
\section{Defender}
\begin{itemize}
\item {Grp. gram.:v. t.}
\end{itemize}
\begin{itemize}
\item {Proveniência:(Lat. \textunderscore defendere\textunderscore )}
\end{itemize}
Desviar qualquer mal de: \textunderscore Deus nos defenda das más linguas\textunderscore .
Dar auxílio a; soccorrer.
Desculpar.
Falar a favor de: \textunderscore o advogado defende o réu\textunderscore .
Proteger; abrigar.
Livrar.
Impedir, prohibir: \textunderscore os regulamentos defendem a caça em certa época do anno\textunderscore .
\section{Defendimento}
\begin{itemize}
\item {Grp. gram.:m.}
\end{itemize}
(V.defesa)
\section{Defendível}
\begin{itemize}
\item {Grp. gram.:adj.}
\end{itemize}
Que se póde defender.
\section{Defengular}
\begin{itemize}
\item {Grp. gram.:v. t.}
\end{itemize}
\begin{itemize}
\item {Utilização:Ant.}
\end{itemize}
Defender? Cf. G. Vicente, I, 144.
\section{Defensa}
\begin{itemize}
\item {Grp. gram.:f.}
\end{itemize}
\begin{itemize}
\item {Grp. gram.:Pl.}
\end{itemize}
\begin{itemize}
\item {Proveniência:(Lat. \textunderscore defensa\textunderscore )}
\end{itemize}
O mesmo que \textunderscore defesa\textunderscore .
Peças, que se collocam fóra do costado do navio, para o defender de sêr roçado nas atracações.
\section{Defensão}
\begin{itemize}
\item {Grp. gram.:f.}
\end{itemize}
O mesmo que \textunderscore defesa\textunderscore .
\section{Defensar}
\begin{itemize}
\item {Proveniência:(De \textunderscore defensa\textunderscore )}
\end{itemize}
\textunderscore v. t.\textunderscore  (e der.)
O mesmo que \textunderscore defender\textunderscore , etc.
\section{Defensável}
\begin{itemize}
\item {Grp. gram.:adj.}
\end{itemize}
\begin{itemize}
\item {Utilização:Ant.}
\end{itemize}
\begin{itemize}
\item {Proveniência:(De \textunderscore defensar\textunderscore )}
\end{itemize}
Que póde têr defensa.
Que serve para defensa; defensivo:«\textunderscore as armas defensáveis de todos eram bacinetes\textunderscore ». Fern. Lopes, \textunderscore Chrón. de D. João I.\textunderscore 
\section{Defensavelmente}
\begin{itemize}
\item {Grp. gram.:adv.}
\end{itemize}
De modo defensável.
\section{Defensiva}
\begin{itemize}
\item {Grp. gram.:f.}
\end{itemize}
\begin{itemize}
\item {Proveniência:(De \textunderscore defensivo\textunderscore )}
\end{itemize}
Posição de quem se defende: \textunderscore estou na defensiva\textunderscore .
\section{Defensível}
\begin{itemize}
\item {Grp. gram.:adj.}
\end{itemize}
\begin{itemize}
\item {Proveniência:(Lat. \textunderscore defensibilis\textunderscore )}
\end{itemize}
O mesmo que \textunderscore defensável\textunderscore .
\section{Defensivo}
\begin{itemize}
\item {Grp. gram.:adj.}
\end{itemize}
\begin{itemize}
\item {Grp. gram.:M.}
\end{itemize}
\begin{itemize}
\item {Proveniência:(De \textunderscore defensa\textunderscore )}
\end{itemize}
Que serve para defensa.
Preservativo.
\section{Defensor}
\begin{itemize}
\item {Grp. gram.:m.}
\end{itemize}
\begin{itemize}
\item {Proveniência:(Lat. \textunderscore defensor\textunderscore )}
\end{itemize}
Aquelle que defende.
\section{Defensório}
\begin{itemize}
\item {Grp. gram.:adj.}
\end{itemize}
\begin{itemize}
\item {Proveniência:(Lat. \textunderscore defensorius\textunderscore )}
\end{itemize}
Relativo a defensa.
\section{Deferência}
\begin{itemize}
\item {Grp. gram.:f.}
\end{itemize}
Acatamento.
Condescendência respeitosa.
(Cp. \textunderscore deferente\textunderscore )
\section{Deferente}
\begin{itemize}
\item {Grp. gram.:adj.}
\end{itemize}
\begin{itemize}
\item {Utilização:Anat.}
\end{itemize}
\begin{itemize}
\item {Proveniência:(Lat. \textunderscore deferens\textunderscore )}
\end{itemize}
Que defere.
Que condescende.
Diz-se de cada um dos vasos excretores dos testículos.
\section{Deferimento}
\begin{itemize}
\item {Grp. gram.:m.}
\end{itemize}
Acto de deferir.
\section{Deferir}
\begin{itemize}
\item {Grp. gram.:v. t.}
\end{itemize}
\begin{itemize}
\item {Grp. gram.:V. i.}
\end{itemize}
\begin{itemize}
\item {Proveniência:(Lat. \textunderscore deferre\textunderscore )}
\end{itemize}
Ceder.
Conceder.
Annuir ao que se pede ou requer: \textunderscore deferir um requerimento\textunderscore .
Condescender.
Têr attenção, acatamento.
\section{Deferível}
\begin{itemize}
\item {Grp. gram.:adj.}
\end{itemize}
Que se póde deferir.
\section{Defervescência}
\begin{itemize}
\item {Grp. gram.:f.}
\end{itemize}
\begin{itemize}
\item {Utilização:Med.}
\end{itemize}
\begin{itemize}
\item {Proveniência:(De \textunderscore de...\textunderscore  + \textunderscore ferver\textunderscore )}
\end{itemize}
Declínio ou cessação da febre.
\section{Defesa}
\begin{itemize}
\item {fónica:fê}
\end{itemize}
\begin{itemize}
\item {Grp. gram.:f.}
\end{itemize}
\begin{itemize}
\item {Proveniência:(Lat. \textunderscore defensa\textunderscore )}
\end{itemize}
Acto de defender.
Aquillo que serve para defender.
Contestação de uma accusação.
Acto de repellir um ataque.
Pessôa, que em juízo, patrocina outra.
Preservativo, resguardo.
Dentes caninos de alguns animaes.
Cornos.
Impedimento, prohibição.
\section{Defeso}
\begin{itemize}
\item {fónica:fê}
\end{itemize}
\begin{itemize}
\item {Grp. gram.:m.}
\end{itemize}
\begin{itemize}
\item {Proveniência:(Do lat. \textunderscore defensus\textunderscore )}
\end{itemize}
Época do anno, em que é prohibida a caça.
\section{Defesso}
\begin{itemize}
\item {fónica:fê}
\end{itemize}
\begin{itemize}
\item {Grp. gram.:adj.}
\end{itemize}
\begin{itemize}
\item {Proveniência:(Lat. \textunderscore defessus\textunderscore )}
\end{itemize}
Cansado.
\section{Deficiência}
\begin{itemize}
\item {Grp. gram.:f.}
\end{itemize}
\begin{itemize}
\item {Proveniência:(Lat. \textunderscore deficientia\textunderscore )}
\end{itemize}
Imperfeição, falta.
\section{Deficiente}
\begin{itemize}
\item {Grp. gram.:adj.}
\end{itemize}
\begin{itemize}
\item {Utilização:Arith.}
\end{itemize}
\begin{itemize}
\item {Proveniência:(Lat. \textunderscore deficiens\textunderscore )}
\end{itemize}
Em que há deficiência.
Imperfeito.
Diz-se de um número, cujas partes alíquotas dão, depois de sommadas, um total menor que êsse número.
\section{Deficit}
\begin{itemize}
\item {fónica:déficid'}
\end{itemize}
\begin{itemize}
\item {Grp. gram.:m.}
\end{itemize}
\begin{itemize}
\item {Proveniência:(T. lat.)}
\end{itemize}
Aquillo que falta.
Saldo negativo entre a receita e a despesa.
Excesso desta sôbre aquella.
\section{Defina}
\begin{itemize}
\item {Grp. gram.:f.}
\end{itemize}
\begin{itemize}
\item {Utilização:T. de Beja}
\end{itemize}
O mesmo que \textunderscore sarrabulho\textunderscore .
\section{Definhamento}
\begin{itemize}
\item {Grp. gram.:m.}
\end{itemize}
Acto ou effeito de definhar.
\section{Definhar}
\begin{itemize}
\item {Grp. gram.:v. t.}
\end{itemize}
\begin{itemize}
\item {Grp. gram.:V. i.}
\end{itemize}
\begin{itemize}
\item {Proveniência:(De \textunderscore de\textunderscore  + \textunderscore fim\textunderscore )}
\end{itemize}
Tornar magro, extenuado.
Enfraquecer-se gradualmente.
Extenuar-se; emmagrecer.
Decair.
\section{Definibilidade}
\begin{itemize}
\item {Grp. gram.:f.}
\end{itemize}
Qualidade do que é definível.
\section{Definição}
\begin{itemize}
\item {Grp. gram.:f.}
\end{itemize}
\begin{itemize}
\item {Proveniência:(Lat. \textunderscore definitio\textunderscore )}
\end{itemize}
Acto de definir.
Palavras, com que se define.
\section{Definido}
\begin{itemize}
\item {Grp. gram.:m.}
\end{itemize}
\begin{itemize}
\item {Grp. gram.:Adj.}
\end{itemize}
Aquillo que se definiu.
Determinado.
Fixo.
\section{Definidor}
\begin{itemize}
\item {Grp. gram.:m.}
\end{itemize}
\begin{itemize}
\item {Proveniência:(Lat. \textunderscore definitor\textunderscore )}
\end{itemize}
Aquelle que define.
Funccionário superior de certas Ordens religiosas.
\section{Definir}
\begin{itemize}
\item {Grp. gram.:v. t.}
\end{itemize}
\begin{itemize}
\item {Proveniência:(Lat. \textunderscore definire\textunderscore )}
\end{itemize}
Determinar a extensão ou os limites de: \textunderscore definir uma área\textunderscore .
Enunciar os attributos e qualidades de (uma coisa), por fórma que esta se não confunda com outra.
Explicar a significação de: \textunderscore definir uma palavra\textunderscore .
Expôr as diversas faces ou lados de.
Decidir; expôr com precisão; fixar: \textunderscore definir uma questão\textunderscore .
Tornar conhecido.
\section{Definitivamente}
\begin{itemize}
\item {Grp. gram.:adv.}
\end{itemize}
De modo definitivo.
\section{Definitivo}
\begin{itemize}
\item {Grp. gram.:adj.}
\end{itemize}
\begin{itemize}
\item {Proveniência:(Lat. \textunderscore definitivus\textunderscore )}
\end{itemize}
Que define.
Final.
Que termina; decisivo: \textunderscore resolução definitiva\textunderscore .
\section{Definito}
\begin{itemize}
\item {Grp. gram.:adj.}
\end{itemize}
\begin{itemize}
\item {Grp. gram.:M.}
\end{itemize}
\begin{itemize}
\item {Utilização:Gram.}
\end{itemize}
O mesmo que \textunderscore definido\textunderscore .
Nome determinativo, que exprime exactidão de referência, como \textunderscore dois\textunderscore , \textunderscore nenhum\textunderscore , \textunderscore todos\textunderscore . Cf. J. Ribeiro, \textunderscore Dicc. Gram.\textunderscore 
\section{Definitório}
\begin{itemize}
\item {Grp. gram.:m.}
\end{itemize}
Assembleia dos definidores, num convento.
Lugar dessa assembleia.
(Cp. \textunderscore definidor\textunderscore )
\section{Definível}
\begin{itemize}
\item {Grp. gram.:adj.}
\end{itemize}
Que se póde definir.
\section{Deflagração}
\begin{itemize}
\item {Grp. gram.:f.}
\end{itemize}
\begin{itemize}
\item {Utilização:Fig.}
\end{itemize}
\begin{itemize}
\item {Proveniência:(Lat. \textunderscore deflagratio\textunderscore )}
\end{itemize}
Combustão activa, com chamma intensa e sem explosão.
Acto de difundir-se ou communicar-se como incêndio:«\textunderscore Se a deflagração do riso não fosse geral...\textunderscore ». Cf. Camillo, \textunderscore Brasileira\textunderscore , 209.
\section{Deflagrador}
\begin{itemize}
\item {Grp. gram.:m.}
\end{itemize}
\begin{itemize}
\item {Proveniência:(De \textunderscore deflagrar\textunderscore )}
\end{itemize}
Apparelho de Phýsica, para incendiar substâncias explosivas.
\section{Deflagrar}
\begin{itemize}
\item {Grp. gram.:v. i.}
\end{itemize}
\begin{itemize}
\item {Proveniência:(Lat. \textunderscore deflagrare\textunderscore )}
\end{itemize}
Arder, formando grande chamma.
\section{Deflexão}
\begin{itemize}
\item {Grp. gram.:f.}
\end{itemize}
Movimento, com que se abandona uma linha que se descrevia, para seguir outra.
(Cp. lat. \textunderscore deflectere\textunderscore )
\section{Defloração}
\begin{itemize}
\item {Grp. gram.:f.}
\end{itemize}
Acto de deflorar.
\section{Deflorador}
\begin{itemize}
\item {Grp. gram.:m.}
\end{itemize}
Aquelle que deflora.
\section{Deflorar}
\begin{itemize}
\item {Grp. gram.:v. t.}
\end{itemize}
\begin{itemize}
\item {Proveniência:(Lat. \textunderscore deflorare\textunderscore )}
\end{itemize}
O mesmo que \textunderscore desflorar\textunderscore .
Prelibar, encetar, provar. Cf. Rebello, \textunderscore Mocidade\textunderscore , II, p. 252.
\section{Defluência}
\begin{itemize}
\item {Grp. gram.:f.}
\end{itemize}
O mesmo que \textunderscore influência\textunderscore .
\section{Defluente}
\begin{itemize}
\item {Grp. gram.:adj.}
\end{itemize}
Que deflue.
\section{Defluir}
\begin{itemize}
\item {Grp. gram.:v. i.}
\end{itemize}
\begin{itemize}
\item {Proveniência:(Lat. \textunderscore defluere\textunderscore )}
\end{itemize}
Manar.
Ir correndo.
Derivar.
\section{Deflúvio}
\begin{itemize}
\item {Grp. gram.:m.}
\end{itemize}
\begin{itemize}
\item {Proveniência:(Lat. \textunderscore defluvium\textunderscore )}
\end{itemize}
Escoamento de líquidos.
Acto de defluir.
\section{Defluxão}
\begin{itemize}
\item {Grp. gram.:f.}
\end{itemize}
\begin{itemize}
\item {Proveniência:(Lat. \textunderscore defluxio\textunderscore )}
\end{itemize}
Deflúvio.
Escoamento de humores.
Defluxo.
\section{Defluxeira}
\begin{itemize}
\item {Grp. gram.:f.}
\end{itemize}
\begin{itemize}
\item {Utilização:Fam.}
\end{itemize}
O mesmo que \textunderscore defluxo\textunderscore .
\section{Defluxo}
\begin{itemize}
\item {Grp. gram.:m.}
\end{itemize}
\begin{itemize}
\item {Proveniência:(Lat. \textunderscore defluxus\textunderscore )}
\end{itemize}
Catarro.
Escoamento de humores, proveniente de coriza ou de inflammação das mucosas nasaes.
\section{Defoliação}
\begin{itemize}
\item {Grp. gram.:f.}
\end{itemize}
(V.desfolhação)
\section{Deformação}
\begin{itemize}
\item {Grp. gram.:f.}
\end{itemize}
\begin{itemize}
\item {Proveniência:(Lat. \textunderscore deformatio\textunderscore )}
\end{itemize}
Acto ou effeito de deformar.
\section{Deformador}
\begin{itemize}
\item {Grp. gram.:m.  e  adj.}
\end{itemize}
Aquelle ou aquillo que deforma.
\section{Deformar}
\begin{itemize}
\item {Grp. gram.:v. t.}
\end{itemize}
\begin{itemize}
\item {Proveniência:(Lat. \textunderscore deformare\textunderscore )}
\end{itemize}
Alterar a fórma de.
Tornar deforme.
\section{Deformatório}
\begin{itemize}
\item {Grp. gram.:adj.}
\end{itemize}
\begin{itemize}
\item {Utilização:Des.}
\end{itemize}
\begin{itemize}
\item {Proveniência:(De \textunderscore deformar\textunderscore )}
\end{itemize}
Que produz deformidade.
\section{Deforme}
\begin{itemize}
\item {Grp. gram.:adj.}
\end{itemize}
\begin{itemize}
\item {Proveniência:(Lat. \textunderscore deformis\textunderscore )}
\end{itemize}
Que perdeu a fórma habitual.
Cuja fórma é irregular e desagradável.
Disforme.
Repellente.
\section{Deformemente}
\begin{itemize}
\item {Grp. gram.:adv.}
\end{itemize}
\begin{itemize}
\item {Proveniência:(De \textunderscore deforme\textunderscore )}
\end{itemize}
Com deformidade.
\section{Deformidade}
\begin{itemize}
\item {Grp. gram.:f.}
\end{itemize}
\begin{itemize}
\item {Proveniência:(Lat. \textunderscore deformitas\textunderscore )}
\end{itemize}
Estado ou qualidade daquillo que é deforme ou de quem é deforme.
\section{Defraudação}
\begin{itemize}
\item {Grp. gram.:f.}
\end{itemize}
\begin{itemize}
\item {Proveniência:(Lat. \textunderscore defraudatio\textunderscore )}
\end{itemize}
Acto de defraudar.
\section{Defraudador}
\begin{itemize}
\item {Grp. gram.:m.}
\end{itemize}
\begin{itemize}
\item {Proveniência:(Lat. \textunderscore defraudator\textunderscore )}
\end{itemize}
Aquelle que defrauda.
\section{Defraudamento}
\begin{itemize}
\item {Grp. gram.:m.}
\end{itemize}
(V.defraudação)
\section{Defraudar}
\begin{itemize}
\item {Grp. gram.:v. t.}
\end{itemize}
\begin{itemize}
\item {Proveniência:(Do lat. \textunderscore defraudare\textunderscore )}
\end{itemize}
Espoliar com fraude: \textunderscore defraudar um herdeiro\textunderscore .
Privar dolosamente de: \textunderscore defraudar uma herança\textunderscore .
Contrariar, illudindo.
\section{Defraudo}
\begin{itemize}
\item {Grp. gram.:m.}
\end{itemize}
(V.defraudação)
\section{Defrontação}
\begin{itemize}
\item {Grp. gram.:f.}
\end{itemize}
Estado daquillo que defronta.
Acto de defrontar.
\section{Defrontadamente}
\begin{itemize}
\item {Grp. gram.:adv.}
\end{itemize}
\begin{itemize}
\item {Proveniência:(De \textunderscore defrontar\textunderscore )}
\end{itemize}
Com defrontação.
Por meio de confronto.
A par.
\section{Defrontante}
\begin{itemize}
\item {Grp. gram.:adj.}
\end{itemize}
Que defronta.
\section{Defrontar}
\begin{itemize}
\item {Grp. gram.:v. i.}
\end{itemize}
\begin{itemize}
\item {Grp. gram.:V. t.}
\end{itemize}
\begin{itemize}
\item {Proveniência:(De \textunderscore Defronte\textunderscore )}
\end{itemize}
Estar defronte.
Pôr-se defronte.
Pôr-se defronte de. Cf. Camillo, \textunderscore Críticos de Canc. Al.\textunderscore , 18.
\section{Defronte}
\begin{itemize}
\item {Grp. gram.:adv.}
\end{itemize}
\begin{itemize}
\item {Grp. gram.:Loc. prep.}
\end{itemize}
\begin{itemize}
\item {Proveniência:(De \textunderscore de\textunderscore  + \textunderscore fronte\textunderscore )}
\end{itemize}
Em face.
Frente a frente.
\textunderscore Defronte de\textunderscore , em opposição a; em frente de.
\section{De-fronte}
\begin{itemize}
\item {Grp. gram.:adv.}
\end{itemize}
\begin{itemize}
\item {Grp. gram.:Loc. prep.}
\end{itemize}
\begin{itemize}
\item {Proveniência:(De \textunderscore de\textunderscore  + \textunderscore fronte\textunderscore )}
\end{itemize}
Em face.
Frente a frente.
\textunderscore De-fronte de\textunderscore , em opposição a; em frente de.
\section{Defumação}
\begin{itemize}
\item {Grp. gram.:f.}
\end{itemize}
\begin{itemize}
\item {Utilização:Bras. do N}
\end{itemize}
Processo para a preparação da borracha.
\section{Defumadoiro}
\begin{itemize}
\item {Grp. gram.:m.}
\end{itemize}
Substância, que defuma.
Vaso, em que se queimam substâncias aromáticas.
Defumador.
Lugar, em que se defuma alguma coisa; fumeiro.
Acto de defumar.
\section{Defumador}
\begin{itemize}
\item {Grp. gram.:m.}
\end{itemize}
Aquelle que defuma.
Vaso, em que se queimam substâncias para defumar ou perfumar.
\section{Defumadouro}
\begin{itemize}
\item {Grp. gram.:m.}
\end{itemize}
Substância, que defuma.
Vaso, em que se queimam substâncias aromáticas.
Defumador.
Lugar, em que se defuma alguma coisa; fumeiro.
Acto de defumar.
\section{Defumadura}
\begin{itemize}
\item {Grp. gram.:f.}
\end{itemize}
\begin{itemize}
\item {Utilização:Des.}
\end{itemize}
Acto de defumar.
\section{Defumar}
\begin{itemize}
\item {Grp. gram.:v. t.}
\end{itemize}
Expôr ao fumo.
Tornar negro com fumo.
Curar ou secar, com o fumo: \textunderscore defumar chouriços\textunderscore .
Perfumar: \textunderscore defumar a sala\textunderscore .
\section{Defunção}
\begin{itemize}
\item {Grp. gram.:f.}
\end{itemize}
\begin{itemize}
\item {Proveniência:(Lat. \textunderscore defunctio\textunderscore )}
\end{itemize}
Óbito, falecimento. Cf. Camillo, \textunderscore Maria da Fonte\textunderscore , p. 101 e 102.
\section{Defuncção}
\begin{itemize}
\item {Grp. gram.:f.}
\end{itemize}
\begin{itemize}
\item {Proveniência:(Lat. \textunderscore defunctio\textunderscore )}
\end{itemize}
Óbito, fallecimento. Cf. Camillo, \textunderscore Maria da Fonte\textunderscore , p. 101 e 102.
\section{Defunta}
\begin{itemize}
\item {Grp. gram.:adj. f.}
\end{itemize}
\begin{itemize}
\item {Utilização:T. de Trancoso}
\end{itemize}
Diz-se de uma variedade de pêra.
\section{Defunteiro}
\begin{itemize}
\item {Grp. gram.:m.}
\end{itemize}
\begin{itemize}
\item {Utilização:Bras. de Pelotas}
\end{itemize}
\begin{itemize}
\item {Proveniência:(De \textunderscore defunto\textunderscore )}
\end{itemize}
Aquelle que trata de enterros; gato-pingado.
\section{Defunto}
\begin{itemize}
\item {Grp. gram.:adj.}
\end{itemize}
\begin{itemize}
\item {Grp. gram.:M.}
\end{itemize}
\begin{itemize}
\item {Proveniência:(Lat. \textunderscore defunctus\textunderscore )}
\end{itemize}
Que morreu.
Extinto.
Esquecido.
Cadáver.
Pessôa que morreu.
\section{Deganha}
\begin{itemize}
\item {Grp. gram.:f.}
\end{itemize}
\begin{itemize}
\item {Utilização:Ant.}
\end{itemize}
Terra adquirida e cultivada, sendo antes desaproveitada e inculta.
(Cast. \textunderscore decana\textunderscore , herdade, granja)
\section{Deganho}
\begin{itemize}
\item {Grp. gram.:m.}
\end{itemize}
\begin{itemize}
\item {Utilização:Ant.}
\end{itemize}
Priorado ou igreja rural.
(Cp. \textunderscore deganha\textunderscore )
\section{Degastador}
\begin{itemize}
\item {Grp. gram.:adj.}
\end{itemize}
\begin{itemize}
\item {Proveniência:(De \textunderscore degastar\textunderscore )}
\end{itemize}
Gastador, perdulário.
\section{Degastar}
\begin{itemize}
\item {Grp. gram.:v. t.}
\end{itemize}
\begin{itemize}
\item {Utilização:Ant.}
\end{itemize}
\begin{itemize}
\item {Proveniência:(Do lat. \textunderscore devastare\textunderscore )}
\end{itemize}
Assolar, devastar.
Desbaratar.
\section{Degelador}
\begin{itemize}
\item {Grp. gram.:adj.}
\end{itemize}
Que degela.
\section{Degelar}
\begin{itemize}
\item {Grp. gram.:v. t.}
\end{itemize}
\begin{itemize}
\item {Utilização:Fig.}
\end{itemize}
\begin{itemize}
\item {Proveniência:(De \textunderscore de...\textunderscore  + \textunderscore gêlo\textunderscore )}
\end{itemize}
Derreter (o que estava congelado).
Aquecer.
\section{Degêlo}
\begin{itemize}
\item {Grp. gram.:m.}
\end{itemize}
Acto de degelar.
\section{Degeneração}
\begin{itemize}
\item {Grp. gram.:f.}
\end{itemize}
Acto ou effeito de degenerar.
\section{Degenerado}
\begin{itemize}
\item {Grp. gram.:adj.}
\end{itemize}
Que degenerou.
\section{Degenerante}
\begin{itemize}
\item {Grp. gram.:adj.}
\end{itemize}
\begin{itemize}
\item {Utilização:Constr.}
\end{itemize}
\begin{itemize}
\item {Proveniência:(De \textunderscore degenerar\textunderscore )}
\end{itemize}
Diz-se do arco, em que a linha do intradorso e a do extradorso se tornaram rectas e parallelas, ficando obliquas as juntas das pedras.
\section{Degenerar}
\begin{itemize}
\item {Grp. gram.:v. i.}
\end{itemize}
\begin{itemize}
\item {Utilização:Fig.}
\end{itemize}
\begin{itemize}
\item {Proveniência:(Lat. \textunderscore degenerare\textunderscore )}
\end{itemize}
Perder as qualidades que teve na sua origem.
Desviar-se das qualidades de sua raça.
Estragar-se, adulterar-se.
Passar a mau ou pior estado.
\section{Degenerativo}
\begin{itemize}
\item {Grp. gram.:adj.}
\end{itemize}
\begin{itemize}
\item {Utilização:Neol.}
\end{itemize}
\begin{itemize}
\item {Proveniência:(De \textunderscore degenerar\textunderscore )}
\end{itemize}
Que revela degeneração.
\section{Degenerescência}
\begin{itemize}
\item {Grp. gram.:f.}
\end{itemize}
Acto ou effeito de degenerar.
Degeneração.
Disposição para degenerar.
Alteração dos caracteres de um corpo organizado.
(Cp. \textunderscore degenerescente\textunderscore )
\section{Degenerescente}
\begin{itemize}
\item {Grp. gram.:adj.}
\end{itemize}
\begin{itemize}
\item {Proveniência:(De um hypoth. \textunderscore degenerescer\textunderscore , de \textunderscore degenerar\textunderscore )}
\end{itemize}
Em que há degenerescência.
\section{Deglutição}
\begin{itemize}
\item {Grp. gram.:f.}
\end{itemize}
\begin{itemize}
\item {Proveniência:(Lat. \textunderscore deglutitio\textunderscore )}
\end{itemize}
Acto de deglutir.
\section{Deglutir}
\begin{itemize}
\item {Grp. gram.:v. t.}
\end{itemize}
\begin{itemize}
\item {Proveniência:(Lat. \textunderscore deglutire\textunderscore )}
\end{itemize}
O mesmo que \textunderscore engulir\textunderscore .
\section{Degola}
\begin{itemize}
\item {Grp. gram.:f.}
\end{itemize}
O mesmo que \textunderscore degolação\textunderscore .
\section{Degolação}
\begin{itemize}
\item {Grp. gram.:f.}
\end{itemize}
\begin{itemize}
\item {Proveniência:(Lat. \textunderscore decollatio\textunderscore )}
\end{itemize}
Acto ou efeito de degolar.
\section{Degolador}
\begin{itemize}
\item {Grp. gram.:m.}
\end{itemize}
\begin{itemize}
\item {Utilização:Serralh.}
\end{itemize}
\begin{itemize}
\item {Proveniência:(De \textunderscore degolar\textunderscore )}
\end{itemize}
Aquele que degola.
Peça, para fazer no ferro uma espécie de garganta ou meia cana.
\section{Degoladouro}
\begin{itemize}
\item {Grp. gram.:m.}
\end{itemize}
\begin{itemize}
\item {Utilização:Prov.}
\end{itemize}
\begin{itemize}
\item {Utilização:minh.}
\end{itemize}
Lugar, em que se degola, em que se mata.
\textunderscore Posta de degoladouro\textunderscore , primeira posta, que se tira do peixe, depois de degolado.
\section{Degoladura}
\begin{itemize}
\item {Grp. gram.:f.}
\end{itemize}
(V.degolação)
\section{Degolar}
\begin{itemize}
\item {Grp. gram.:v. t.}
\end{itemize}
\begin{itemize}
\item {Proveniência:(Do lat. \textunderscore decollare\textunderscore )}
\end{itemize}
Cortar o pescoço de.
Cortar a cabeça a.
Decapitar.
\section{Degolla}
\begin{itemize}
\item {Grp. gram.:f.}
\end{itemize}
O mesmo que \textunderscore degollação\textunderscore .
\section{Degollação}
\begin{itemize}
\item {Proveniência:(Lat. \textunderscore decollatio\textunderscore )}
\end{itemize}
Acto ou effeito de degollar.
\section{Degolladoiro}
\begin{itemize}
\item {Grp. gram.:m.}
\end{itemize}
\begin{itemize}
\item {Utilização:Prov.}
\end{itemize}
\begin{itemize}
\item {Utilização:minh.}
\end{itemize}
Lugar, em que se degolla, em que se mata.
\textunderscore Posta de degolladoiro\textunderscore , primeira posta, que se tira do peixe, depois de degollado.
\section{Degollador}
\begin{itemize}
\item {Utilização:Serralh.}
\end{itemize}
\begin{itemize}
\item {Proveniência:(De \textunderscore degollar\textunderscore )}
\end{itemize}
\textunderscore m.\textunderscore 
Aquelle que degolla.
Peça, para fazer no ferro uma espécie de garganta ou meia cana.
\section{Degolladura}
\begin{itemize}
\item {Grp. gram.:f.}
\end{itemize}
(V.degollação)
\section{Degollar}
\begin{itemize}
\item {Grp. gram.:v. t.}
\end{itemize}
\begin{itemize}
\item {Proveniência:(Do lat. \textunderscore decollare\textunderscore )}
\end{itemize}
Cortar o pescoço de.
Cortar a cabeça a.
Decapitar.
\section{Degote}
\textunderscore m.\textunderscore  (e der.)
O mesmo que \textunderscore decote\textunderscore , etc.--Na linguagem popular, \textunderscore degote\textunderscore  e \textunderscore degotar\textunderscore , são fórmulas usadíssimas.
\section{Degradação}
\begin{itemize}
\item {Grp. gram.:f.}
\end{itemize}
\begin{itemize}
\item {Utilização:Geol.}
\end{itemize}
Acto ou effeito de degradar.
Desgaste das rochas da superfície do globo, determinado principalmente pelos agentes atmosphéricos, variação da temperatura, chuvas, etc.
\section{Degradado}
\begin{itemize}
\item {Grp. gram.:m.  e  adj.}
\end{itemize}
(V.degredado)
\section{Degradamento}
\begin{itemize}
\item {Grp. gram.:m.}
\end{itemize}
Acto de degradar^1.
Infâmia.
Aviltamento.
\section{Degradamento}
\begin{itemize}
\item {Grp. gram.:m.}
\end{itemize}
O mesmo que \textunderscore degrêdo\textunderscore .
\section{Degradante}
\begin{itemize}
\item {Grp. gram.:adj.}
\end{itemize}
\begin{itemize}
\item {Proveniência:(De \textunderscore degradar\textunderscore ^1)}
\end{itemize}
Que degrada, que rebaixa, que avilta: \textunderscore procedimento degradante\textunderscore .
\section{Degradar}
\begin{itemize}
\item {Grp. gram.:v. t.}
\end{itemize}
\begin{itemize}
\item {Proveniência:(Lat. \textunderscore degradare\textunderscore )}
\end{itemize}
Privar de grau ou dignidade, por modo infamante.
Aviltar, tornar desprezível.
\section{Degradar}
\begin{itemize}
\item {Grp. gram.:v. t.}
\end{itemize}
O mesmo que \textunderscore degredar\textunderscore .
\section{Degranadeira}
\begin{itemize}
\item {Grp. gram.:f.}
\end{itemize}
\begin{itemize}
\item {Proveniência:(De \textunderscore degranar\textunderscore )}
\end{itemize}
Grande ciranda, para desengaçar uvas.
\section{Degranar}
\begin{itemize}
\item {Grp. gram.:v. t.}
\end{itemize}
\begin{itemize}
\item {Proveniência:(Do lat. \textunderscore de\textunderscore  + \textunderscore granum\textunderscore )}
\end{itemize}
Tirar o grão a.
\section{Degranhar}
\begin{itemize}
\item {Grp. gram.:v. t.}
\end{itemize}
\begin{itemize}
\item {Utilização:Prov.}
\end{itemize}
\begin{itemize}
\item {Utilização:trasm.}
\end{itemize}
O mesmo que \textunderscore degranar\textunderscore .
\section{Degráo}
\begin{itemize}
\item {Grp. gram.:m.}
\end{itemize}
\begin{itemize}
\item {Utilização:Fig.}
\end{itemize}
\begin{itemize}
\item {Proveniência:(Do lat. \textunderscore de\textunderscore  + \textunderscore gradus\textunderscore )}
\end{itemize}
Cada uma das partes da escada, em que se põe o pé para subir ou descer.
Grau.
Meio, para se elevar ou conseguir certo fim.
\section{Degrau}
\begin{itemize}
\item {Grp. gram.:m.}
\end{itemize}
\begin{itemize}
\item {Utilização:Fig.}
\end{itemize}
\begin{itemize}
\item {Proveniência:(Do lat. \textunderscore de\textunderscore  + \textunderscore gradus\textunderscore )}
\end{itemize}
Cada uma das partes da escada, em que se põe o pé para subir ou descer.
Grau.
Meio, para se elevar ou conseguir certo fim.
\section{Degredado}
\begin{itemize}
\item {Grp. gram.:m.}
\end{itemize}
Aquelle que soffre a pena de degrêdo.
\section{Degredar}
\begin{itemize}
\item {Grp. gram.:v. t.}
\end{itemize}
\begin{itemize}
\item {Proveniência:(Do lat. \textunderscore degradare\textunderscore ?)}
\end{itemize}
Impor a pena de degrêdo a.
Desterrar.
\section{Degrêdo}
\begin{itemize}
\item {Grp. gram.:m.}
\end{itemize}
\begin{itemize}
\item {Utilização:Ant.}
\end{itemize}
O mesmo que \textunderscore decreto\textunderscore .
\section{Degredo}
\begin{itemize}
\item {Grp. gram.:m.}
\end{itemize}
\begin{itemize}
\item {Proveniência:(De \textunderscore degredar\textunderscore )}
\end{itemize}
Pena de destêrro, imposta judicialmente em castigo de um crime.
Lugar, onde se expia essa pena.
Exílio.
\section{Deguélia}
\begin{itemize}
\item {Grp. gram.:f.}
\end{itemize}
\begin{itemize}
\item {Proveniência:(De \textunderscore Deguel\textunderscore , n. p.)}
\end{itemize}
Arbusto sarmentoso da Guiana.
\section{Degustação}
\begin{itemize}
\item {Grp. gram.:f.}
\end{itemize}
\begin{itemize}
\item {Proveniência:(Lat. \textunderscore degustatio\textunderscore )}
\end{itemize}
Acto de degustar.
\section{Degustar}
\begin{itemize}
\item {Grp. gram.:v. t.}
\end{itemize}
\begin{itemize}
\item {Proveniência:(Lat. \textunderscore degustare\textunderscore )}
\end{itemize}
Avaliar, por meio do paladar, o sabor de.
Provar.
\section{Dehiscência}
\begin{itemize}
\item {Grp. gram.:f.}
\end{itemize}
\begin{itemize}
\item {Utilização:Bot.}
\end{itemize}
Acto de se abrirem espontaneamente as válvulas de um órgão vegetal.
(Cp. \textunderscore dehiscente\textunderscore )
\section{Dehiscente}
\begin{itemize}
\item {Grp. gram.:adj.}
\end{itemize}
\begin{itemize}
\item {Proveniência:(Lat. \textunderscore dehiscens\textunderscore )}
\end{itemize}
Que se abre espontaneamente por suturas preexistentes, (falando-se de órgãos vegetaes)
\section{Dei}
\begin{itemize}
\item {Grp. gram.:m.}
\end{itemize}
\begin{itemize}
\item {Proveniência:(T. turco)}
\end{itemize}
Presidente de corporação administrativa, entre os Moiros.
Título do chefe bárbaro, que governava o districto de Argel.
\section{Deia}
\begin{itemize}
\item {Grp. gram.:f.}
\end{itemize}
Deusa. Cf. \textunderscore Lusíadas\textunderscore , I, 34.
\section{Deicida}
\begin{itemize}
\item {Grp. gram.:m.  e  adj.}
\end{itemize}
\begin{itemize}
\item {Proveniência:(Lat. \textunderscore deicida\textunderscore )}
\end{itemize}
Cada um dos que cooperaram na morte de Christo.
\section{Deicídio}
\begin{itemize}
\item {Grp. gram.:m.}
\end{itemize}
Morte, que os Judeus deram a Christo.
(Cp. \textunderscore deicida\textunderscore )
\section{Deícola}
\begin{itemize}
\item {Grp. gram.:m.}
\end{itemize}
\begin{itemize}
\item {Proveniência:(Do lat. \textunderscore deus\textunderscore  + \textunderscore colere\textunderscore )}
\end{itemize}
O mesmo que \textunderscore deísta\textunderscore .
\section{Deidade}
\begin{itemize}
\item {Grp. gram.:f.}
\end{itemize}
\begin{itemize}
\item {Utilização:Fig.}
\end{itemize}
\begin{itemize}
\item {Proveniência:(Lat. deitas)}
\end{itemize}
Divindade.
Mulher muito formosa.
\section{Deidâmia}
\begin{itemize}
\item {Grp. gram.:f.}
\end{itemize}
Gênero de plantas passiflóreas.
\section{Deificação}
\begin{itemize}
\item {Grp. gram.:f.}
\end{itemize}
Acto de deificar.
\section{Deificador}
\begin{itemize}
\item {Grp. gram.:m.}
\end{itemize}
Aquelle que deifica.
\section{Deificar}
\begin{itemize}
\item {Grp. gram.:v. t.}
\end{itemize}
\begin{itemize}
\item {Proveniência:(Lat. \textunderscore deificare\textunderscore )}
\end{itemize}
Incluir em um número dos deuses.
Divinizar.
Fazer a apothéose de.
\section{Deífico}
\begin{itemize}
\item {Grp. gram.:adj.}
\end{itemize}
\begin{itemize}
\item {Proveniência:(Lat. \textunderscore deificus\textunderscore )}
\end{itemize}
Que deifica.
\section{Deilo}
\begin{itemize}
\item {Grp. gram.:m.}
\end{itemize}
\begin{itemize}
\item {Proveniência:(Gr. \textunderscore deilos\textunderscore )}
\end{itemize}
Insecto longicórneo do sul da Europa.
\section{Deípara}
\begin{itemize}
\item {Grp. gram.:f.}
\end{itemize}
\begin{itemize}
\item {Proveniência:(Lat. \textunderscore deipara\textunderscore )}
\end{itemize}
Mãe de Deus.
\section{Deiscência}
\begin{itemize}
\item {fónica:de-is}
\end{itemize}
\begin{itemize}
\item {Grp. gram.:f.}
\end{itemize}
\begin{itemize}
\item {Utilização:Bot.}
\end{itemize}
Acto de se abrirem espontaneamente as válvulas de um órgão vegetal.
(Cp. \textunderscore deiscente\textunderscore )
\section{Deiscente}
\begin{itemize}
\item {fónica:de-is}
\end{itemize}
\begin{itemize}
\item {Grp. gram.:adj.}
\end{itemize}
\begin{itemize}
\item {Proveniência:(Lat. \textunderscore dehiscens\textunderscore )}
\end{itemize}
Que se abre espontaneamente por suturas preexistentes, (falando-se de órgãos vegetaes).
\section{Deísmo}
\begin{itemize}
\item {Grp. gram.:m.}
\end{itemize}
\begin{itemize}
\item {Proveniência:(De \textunderscore Deus\textunderscore )}
\end{itemize}
Systema dos que crêem em Deus, rejeitando a revelação.
\section{Deísta}
\begin{itemize}
\item {Grp. gram.:m.}
\end{itemize}
\begin{itemize}
\item {Proveniência:(De \textunderscore Deus\textunderscore )}
\end{itemize}
Sectário do deísmo.
\section{Deita}
\begin{itemize}
\item {Grp. gram.:f.}
\end{itemize}
\begin{itemize}
\item {Utilização:Fam.}
\end{itemize}
\begin{itemize}
\item {Proveniência:(De \textunderscore deitar\textunderscore )}
\end{itemize}
Acto de deitar-se alguém para dormir: \textunderscore vamos á deita, que está o somno á espreita\textunderscore . Cf. Camillo, \textunderscore Brasileira\textunderscore , 323.
\section{Deitada}
\begin{itemize}
\item {Grp. gram.:f.}
\end{itemize}
\begin{itemize}
\item {Utilização:Pop.}
\end{itemize}
Acto de deitar-se.
\section{Deitadura}
\begin{itemize}
\item {Grp. gram.:f.}
\end{itemize}
Acto de deitar.
\section{Deitar}
\begin{itemize}
\item {Grp. gram.:v. t.}
\end{itemize}
\begin{itemize}
\item {Grp. gram.:V. i.}
\end{itemize}
\begin{itemize}
\item {Utilização:Fam.}
\end{itemize}
\begin{itemize}
\item {Grp. gram.:V. p.}
\end{itemize}
\begin{itemize}
\item {Proveniência:(Lat. \textunderscore dejectare\textunderscore )}
\end{itemize}
Atirar, arremessar.
Expellir: \textunderscore deitar espuma pela boca\textunderscore .
Expulsar.
Pôr no chão.
Abater.
Inclinar: \textunderscore o vento deita as searas\textunderscore .
Fazer cair.
Estender horizontalmente: \textunderscore deitar um pau no chão\textunderscore .
Entornar.
Exhalar: \textunderscore deitar aroma\textunderscore .
Attribuir: \textunderscore deitar culpas a alguém\textunderscore .
Resumar.
Produzir: \textunderscore deitar rebentos\textunderscore .
Pôr.
Ostentar: \textunderscore deitar vestidos de luxo\textunderscore .
Voltar-se andando.
Andar.
Estender-se no chão: \textunderscore o cavallo deitou-se\textunderscore 
Estender-se ou meter-se na cama: \textunderscore fui deitar-me\textunderscore .
\section{Deixa}
\begin{itemize}
\item {Grp. gram.:f.}
\end{itemize}
\begin{itemize}
\item {Proveniência:(De \textunderscore deixar\textunderscore )}
\end{itemize}
Acto ou effeito de deixar.
Legado.
Palavra que, nos papéis dos actores, indica que um acabou de falar e que outro deve começar.
\section{Deixação}
\begin{itemize}
\item {Grp. gram.:f.}
\end{itemize}
\begin{itemize}
\item {Proveniência:(De \textunderscore deixar\textunderscore )}
\end{itemize}
O mesmo que \textunderscore deixa\textunderscore .
\textunderscore Deixação de si mesmo\textunderscore , abnegação, desprendimento de sua pessôa. Cf. \textunderscore Luz e Calor\textunderscore , 184; B. de Figueiredo, \textunderscore Odivellas\textunderscore , 266.
\section{Deixamento}
\begin{itemize}
\item {Grp. gram.:m.}
\end{itemize}
Acto ou effeito de deixar. Cf. Filinto, XIX, 221; XX, 88.
\section{Deixar}
\begin{itemize}
\item {Grp. gram.:v. t.}
\end{itemize}
Separar-se de.
Lançar de si.
Largar: \textunderscore deixar o mêdo\textunderscore .
Pôr de lado:«\textunderscore deixo, deuses, atrás a fama antiga\textunderscore ». \textunderscore Lusíadas\textunderscore .
Abandonar: \textunderscore deixar os filhos\textunderscore .
Permittir: \textunderscore deixe-me falar\textunderscore .
Cessar.
Resistir.
Adiar: \textunderscore deixar para mais tarde\textunderscore .
Ceder.
Omittir.
(B. lat. \textunderscore delaxare\textunderscore )
\section{Dejarretar}
\begin{itemize}
\item {Grp. gram.:v. t.}
\end{itemize}
Cortar pelo jarrête.
\section{Dejatata}
\begin{itemize}
\item {Grp. gram.:f.}
\end{itemize}
\begin{itemize}
\item {Utilização:Prov.}
\end{itemize}
\begin{itemize}
\item {Utilização:trasm.}
\end{itemize}
O mesmo que \textunderscore descampatória\textunderscore .
\section{Dejecção}
\begin{itemize}
\item {Grp. gram.:f.}
\end{itemize}
\begin{itemize}
\item {Proveniência:(Lat. \textunderscore dejectio\textunderscore )}
\end{itemize}
Evacuação de excremento.
Substância expellida por vulcões.
\section{Dejectar}
\begin{itemize}
\item {Grp. gram.:v. t.}
\end{itemize}
\begin{itemize}
\item {Proveniência:(De \textunderscore dejecto\textunderscore )}
\end{itemize}
Fazer dejecção; defecar.
\section{Dejecto}
\begin{itemize}
\item {Grp. gram.:m.}
\end{itemize}
\begin{itemize}
\item {Proveniência:(Lat. \textunderscore dejectus\textunderscore )}
\end{itemize}
Acto de evacuar excrementos.
Matérias fecaes, expellidas por uma vez.
\section{Dejectório}
\begin{itemize}
\item {Grp. gram.:m.}
\end{itemize}
\begin{itemize}
\item {Utilização:bras}
\end{itemize}
\begin{itemize}
\item {Utilização:Neol.}
\end{itemize}
\begin{itemize}
\item {Proveniência:(Do lat. \textunderscore dejectus\textunderscore )}
\end{itemize}
O mesmo que \textunderscore latrina\textunderscore .
\section{Dejejuadoiro}
\begin{itemize}
\item {Grp. gram.:m.}
\end{itemize}
Acto de dejejuar.
Parva.
\section{Dejejuadouro}
\begin{itemize}
\item {Grp. gram.:m.}
\end{itemize}
Acto de dejejuar.
Parva.
\section{Dejejuar}
\begin{itemize}
\item {Grp. gram.:v. i.}
\end{itemize}
\begin{itemize}
\item {Proveniência:(De \textunderscore de...\textunderscore  + \textunderscore jejuar\textunderscore )}
\end{itemize}
Quebrar o jejum, comendo alguma coisa antes de almôço.
\section{Dejungir}
\textunderscore v. t.\textunderscore  (e der.)
O mesmo que \textunderscore desjungir\textunderscore , etc.
\section{Del}
Contracção \textunderscore ant.\textunderscore  da prep. \textunderscore de\textunderscore  e do art. \textunderscore el\textunderscore :«\textunderscore cêrca das ribas del mar\textunderscore ». Canção algarvia.
\section{Dela}
\begin{itemize}
\item {Grp. gram.:Pl.}
\end{itemize}
Fórma contraida de \textunderscore de\textunderscore  + \textunderscore ela\textunderscore .
Alguns, parte:«\textunderscore e as náos que achou no porto, delas afundou; e taes rombos fez ás mais...\textunderscore »Filinto. \textunderscore D. Man.\textunderscore , I, 121.
\section{Delação}
\begin{itemize}
\item {Grp. gram.:f.}
\end{itemize}
\begin{itemize}
\item {Proveniência:(Lat. \textunderscore delatio\textunderscore )}
\end{itemize}
Acto de delatar.
\section{Deladeiro}
\begin{itemize}
\item {Grp. gram.:m.}
\end{itemize}
\begin{itemize}
\item {Utilização:Prov.}
\end{itemize}
\begin{itemize}
\item {Utilização:trasm.}
\end{itemize}
O mesmo que \textunderscore desladeiro\textunderscore .
\section{Delaidar}
\begin{itemize}
\item {Grp. gram.:v. t.}
\end{itemize}
\begin{itemize}
\item {Utilização:Ant.}
\end{itemize}
\begin{itemize}
\item {Proveniência:(Do fr. ant. \textunderscore laider\textunderscore )}
\end{itemize}
Tornar feio; desfigurar.
\section{Delaidinha}
\begin{itemize}
\item {Grp. gram.:f.}
\end{itemize}
\begin{itemize}
\item {Utilização:Prov.}
\end{itemize}
\begin{itemize}
\item {Utilização:alent.}
\end{itemize}
Dança de roda.
(Por \textunderscore Adelaidinha\textunderscore , de \textunderscore Adelaide\textunderscore , n. p.)
\section{Delamber-se}
\begin{itemize}
\item {Grp. gram.:v. p.}
\end{itemize}
O mesmo que [[lamber-se|lamber]].
\section{Delambido}
\begin{itemize}
\item {Grp. gram.:m.}
\end{itemize}
\begin{itemize}
\item {Utilização:Fig.}
\end{itemize}
Indivíduo affectado, presumido.
\section{Delanteira}
\begin{itemize}
\item {Grp. gram.:f.}
\end{itemize}
\begin{itemize}
\item {Utilização:Ant.}
\end{itemize}
O mesmo que \textunderscore deanteira\textunderscore  ou \textunderscore vanguarda\textunderscore . Cf. Herculano, \textunderscore Hist. de Port.\textunderscore , IV, 415.
\section{Delapidar}
\textunderscore v. t.\textunderscore  (e der.)
O mesmo que \textunderscore dilapidar\textunderscore .
\section{Delatar}
\begin{itemize}
\item {Grp. gram.:v. t.}
\end{itemize}
\begin{itemize}
\item {Proveniência:(Do rad. do lat. \textunderscore delatus\textunderscore )}
\end{itemize}
Denunciar (crime).
\section{Delatável}
\begin{itemize}
\item {Grp. gram.:adj.}
\end{itemize}
\begin{itemize}
\item {Proveniência:(De \textunderscore delatar\textunderscore )}
\end{itemize}
Que deve sêr delatado.
\section{Delator}
\begin{itemize}
\item {Grp. gram.:m.}
\end{itemize}
\begin{itemize}
\item {Proveniência:(Lat. \textunderscore delator\textunderscore )}
\end{itemize}
Aquelle que delata.
\section{Delatório}
\begin{itemize}
\item {Grp. gram.:adj.}
\end{itemize}
\begin{itemize}
\item {Proveniência:(Lat. \textunderscore delatorius\textunderscore )}
\end{itemize}
Relativo a delação.
\section{Dêle}
Fórma contraida de \textunderscore de\textunderscore  + \textunderscore êle\textunderscore .
\section{Delegação}
\begin{itemize}
\item {Grp. gram.:f.}
\end{itemize}
\begin{itemize}
\item {Proveniência:(Lat. \textunderscore delegatio\textunderscore )}
\end{itemize}
Acto de delegar.
Delegacia.
\section{Delegacia}
\begin{itemize}
\item {Grp. gram.:f.}
\end{itemize}
\begin{itemize}
\item {Proveniência:(Do lat. \textunderscore delegatus\textunderscore )}
\end{itemize}
Cargo do delegado.
Repartição, em que trabalha o delegado.
\section{Delegado}
\begin{itemize}
\item {Grp. gram.:m.}
\end{itemize}
\begin{itemize}
\item {Proveniência:(De \textunderscore delegar\textunderscore )}
\end{itemize}
Aquelle que é autorizado por outrem, para o representar.
Commissário.
Aquelle que tem a seu cargo serviço público, dependente de autoridade superior.
\section{Delegante}
\begin{itemize}
\item {Grp. gram.:m.  e  adj.}
\end{itemize}
\begin{itemize}
\item {Proveniência:(Lat. \textunderscore delegans\textunderscore )}
\end{itemize}
Aquelle que delega.
\section{Delegar}
\begin{itemize}
\item {Grp. gram.:v. t.}
\end{itemize}
\begin{itemize}
\item {Proveniência:(Lat. \textunderscore delegare\textunderscore )}
\end{itemize}
Investir na faculdade de proceder em nome de outrem.
Commeter, incumbir.
\section{Delegatório}
\begin{itemize}
\item {Grp. gram.:adj.}
\end{itemize}
\begin{itemize}
\item {Proveniência:(Lat. \textunderscore delegatorius\textunderscore )}
\end{itemize}
Em que há delegação.
\section{Deleitação}
\begin{itemize}
\item {Grp. gram.:f.}
\end{itemize}
\begin{itemize}
\item {Proveniência:(Lat. \textunderscore delectatio\textunderscore )}
\end{itemize}
O mesmo que \textunderscore deleite\textunderscore .
\section{Deleitamento}
\begin{itemize}
\item {Grp. gram.:m.}
\end{itemize}
(V.deleite)
\section{Deleitante}
\begin{itemize}
\item {Grp. gram.:adj.}
\end{itemize}
\begin{itemize}
\item {Proveniência:(Lat. \textunderscore delectans\textunderscore )}
\end{itemize}
Que deleita.
\section{Deleitar}
\begin{itemize}
\item {Grp. gram.:v. t.}
\end{itemize}
\begin{itemize}
\item {Proveniência:(Do lat. \textunderscore delectare\textunderscore )}
\end{itemize}
Causar deleite a.
Dar prazer a.
Deliciar.
\section{Deleitável}
\begin{itemize}
\item {Grp. gram.:adj.}
\end{itemize}
\begin{itemize}
\item {Proveniência:(Lat. \textunderscore delectabilis\textunderscore )}
\end{itemize}
O mesmo que \textunderscore deleitoso\textunderscore .
\section{Deleitavelmente}
\begin{itemize}
\item {Grp. gram.:adv.}
\end{itemize}
De modo deleitável.
\section{Deleite}
\begin{itemize}
\item {Grp. gram.:m.}
\end{itemize}
\begin{itemize}
\item {Proveniência:(De \textunderscore deleitar\textunderscore )}
\end{itemize}
Prazer íntimo e suave; delícia.
\section{Deleitosamente}
\begin{itemize}
\item {Grp. gram.:adv.}
\end{itemize}
De modo deleitoso.
\section{Deleitoso}
\begin{itemize}
\item {Grp. gram.:adj.}
\end{itemize}
Que causa deleite.
\section{Deleixar}
\textunderscore v. t.\textunderscore  (e der.)
(V. [[desleixar|desleixar-se]], etc.)
\section{Deletério}
\begin{itemize}
\item {Grp. gram.:adj.}
\end{itemize}
\begin{itemize}
\item {Proveniência:(Gr. \textunderscore deleterios\textunderscore )}
\end{itemize}
Que destrói, que corrompe.
Nocivo á saúde; prejudicial: \textunderscore substância deletéria\textunderscore .
Damnoso; desmoralizador: \textunderscore doutrinas deletérias\textunderscore .
\section{Deletrear}
\begin{itemize}
\item {Grp. gram.:v. t.}
\end{itemize}
\begin{itemize}
\item {Proveniência:(De \textunderscore letra\textunderscore )}
\end{itemize}
Soletrar; lêr mal.
\section{Delével}
\begin{itemize}
\item {Grp. gram.:adj.}
\end{itemize}
\begin{itemize}
\item {Proveniência:(Lat. \textunderscore delebitis\textunderscore )}
\end{itemize}
Que se póde destruir, expungir, apagar.
\section{Délfico}
\begin{itemize}
\item {Grp. gram.:m.}
\end{itemize}
\begin{itemize}
\item {Proveniência:(Gr. \textunderscore delphikos\textunderscore )}
\end{itemize}
Pequena mesa de três pés, ou pequeno aparador, para exposição de baixelas ricas.
\section{Delfim}
\begin{itemize}
\item {Grp. gram.:m.}
\end{itemize}
\begin{itemize}
\item {Grp. gram.:Pl.}
\end{itemize}
\begin{itemize}
\item {Utilização:Ant.}
\end{itemize}
\begin{itemize}
\item {Proveniência:(Lat. \textunderscore delphin\textunderscore )}
\end{itemize}
Cetáceo, o mesmo que \textunderscore golfinho\textunderscore .
Constellação do Norte.
Espécie de asas, no segundo refôrço dos antigos canhões de bronze.
\section{Delfim}
\begin{itemize}
\item {Grp. gram.:m.}
\end{itemize}
Nome, que se deu aos senadores do Delfinado, em França, por allusão ao delphim ou golfinho que figurava em suas armas.
Depois de 1349, deu-se êsse título ao herdeiro presumptivo da Corôa de França.
\section{Delfina}
\begin{itemize}
\item {Grp. gram.:f.}
\end{itemize}
(V.delfinina)
\section{Delfinado}
\begin{itemize}
\item {Grp. gram.:m.}
\end{itemize}
Terras, que pertenciam ao delfim de França.
\section{Delfínico}
\begin{itemize}
\item {Grp. gram.:adj.}
\end{itemize}
(V.delfinino)
\section{Delfinina}
\begin{itemize}
\item {Grp. gram.:f.}
\end{itemize}
\begin{itemize}
\item {Proveniência:(De \textunderscore delphinio\textunderscore )}
\end{itemize}
Alcaloide, que se encontra no paparrás.
\section{Delfinino}
\begin{itemize}
\item {Grp. gram.:adj.}
\end{itemize}
\begin{itemize}
\item {Grp. gram.:M. pl.}
\end{itemize}
Relativo ou semelhante ao delfim^1.
Família de mammíferos cetáceos, que têm por typo o delfim.
\section{Delfínio}
\begin{itemize}
\item {Grp. gram.:m.}
\end{itemize}
\begin{itemize}
\item {Proveniência:(Do lat. \textunderscore delphin\textunderscore , pela pequena semelhança dos seus nectários a um delfim)}
\end{itemize}
Planta ranunculácea.
\section{Delfino}
\begin{itemize}
\item {Grp. gram.:m.}
\end{itemize}
\begin{itemize}
\item {Utilização:Bras. do N}
\end{itemize}
Defunto; finado.
\section{Delfínula}
\begin{itemize}
\item {Grp. gram.:f.}
\end{itemize}
Gênero de conchas marítimas, nacaradas e erriçadas de espinhos.
\section{Delgadamente}
\begin{itemize}
\item {Grp. gram.:adv.}
\end{itemize}
De fórma delgada.
\section{Delgadeza}
\begin{itemize}
\item {Grp. gram.:f.}
\end{itemize}
Qualidade de quem ou daquillo que é delgado.
\section{Delgadicho}
\begin{itemize}
\item {Grp. gram.:adj.}
\end{itemize}
\begin{itemize}
\item {Utilização:T. de Viana}
\end{itemize}
Muito delgado.
\section{Delgado}
\begin{itemize}
\item {Grp. gram.:adj.}
\end{itemize}
\begin{itemize}
\item {Grp. gram.:M.}
\end{itemize}
\begin{itemize}
\item {Proveniência:(Do lat. \textunderscore delicatus\textunderscore )}
\end{itemize}
Pouco espêsso, tênue.
Magro.
Que tem pouca grossura: \textunderscore homem delgado\textunderscore .
Fino: \textunderscore corda delgada\textunderscore .
Pouco volumoso.
Facilmente digerível; fino: \textunderscore águas delgadas\textunderscore . Cf. L. Cardoso, \textunderscore Diccion. Geogr.\textunderscore , I, 377 e 452.
A parte delgada de alguns objectos.
\section{Délias}
\begin{itemize}
\item {Grp. gram.:f. pl.}
\end{itemize}
Festas, em honra de Apollo, que se celebravam em Delos.
\section{Delibação}
\begin{itemize}
\item {Grp. gram.:f.}
\end{itemize}
\begin{itemize}
\item {Proveniência:(Lat. \textunderscore delibatio\textunderscore )}
\end{itemize}
Acto de delibar.
\section{Delibar}
\begin{itemize}
\item {Grp. gram.:v. t.}
\end{itemize}
\begin{itemize}
\item {Proveniência:(Lat. \textunderscore delibare\textunderscore )}
\end{itemize}
Libar.
Tocar com os lábios.
Provar, bebendo.
Saborear.
\section{Deliberação}
\begin{itemize}
\item {Grp. gram.:f.}
\end{itemize}
\begin{itemize}
\item {Proveniência:(Lat. \textunderscore deliberatio\textunderscore )}
\end{itemize}
Acto de deliberar.
Resolução.
\section{Deliberadamente}
\begin{itemize}
\item {Grp. gram.:adv.}
\end{itemize}
Com deliberação.
Propositadamente.
\section{Deliberante}
\begin{itemize}
\item {Grp. gram.:m.  e  adj.}
\end{itemize}
\begin{itemize}
\item {Proveniência:(Lat. \textunderscore deliberans\textunderscore )}
\end{itemize}
Quem delibera.
\section{Deliberar}
\begin{itemize}
\item {Grp. gram.:v. t.}
\end{itemize}
\begin{itemize}
\item {Grp. gram.:V. i.}
\end{itemize}
\begin{itemize}
\item {Proveniência:(Lat. \textunderscore deliberare\textunderscore )}
\end{itemize}
Resolver, com precedência de discussão ou exame.
Decidir.
Reflectir, ponderar consigo próprio ou juntamente com outrem.
\section{Deliberativo}
\begin{itemize}
\item {Grp. gram.:adj.}
\end{itemize}
\begin{itemize}
\item {Proveniência:(Lat. \textunderscore deliberativus\textunderscore )}
\end{itemize}
Relativo a deliberação.
\section{Delicadamente}
\begin{itemize}
\item {Grp. gram.:adv.}
\end{itemize}
De modo delicado.
Com delicadeza.
\section{Delicadeza}
\begin{itemize}
\item {Grp. gram.:f.}
\end{itemize}
Qualidade de quem ou daquillo que é delicado.
Cortesia, urbanidade.
\section{Delicado}
\begin{itemize}
\item {Grp. gram.:adj.}
\end{itemize}
\begin{itemize}
\item {Proveniência:(Lat. \textunderscore delicatus\textunderscore )}
\end{itemize}
Fraco.
Frágil: \textunderscore planta delicada\textunderscore .
Molle.
Brando.
Delgado, fino.
Elegante: \textunderscore cintura delicada\textunderscore .
Suave.
Affectuoso, meigo.
Leve.
Que se altera facilmente: \textunderscore côr delicada\textunderscore .
Mimoso.
Apurado.
Sensível.
Attencioso, cortês: \textunderscore cavalheiro delicado\textunderscore .
Complicado, diffícil: \textunderscore isso é negócio delicado\textunderscore .
\section{Delícia}
\begin{itemize}
\item {Grp. gram.:f.}
\end{itemize}
\begin{itemize}
\item {Proveniência:(Lat. \textunderscore delicia\textunderscore )}
\end{itemize}
O mesmo que \textunderscore deleite\textunderscore .
Encanto.
Volúpia.
Aquillo que causa delícia: \textunderscore êste bolo é uma delícia\textunderscore .
\section{Deliciar}
\begin{itemize}
\item {Grp. gram.:v. t.}
\end{itemize}
\begin{itemize}
\item {Grp. gram.:V. p.}
\end{itemize}
\begin{itemize}
\item {Proveniência:(Lat. \textunderscore deliciari\textunderscore )}
\end{itemize}
Causar delícia a.
Sentir delícia.
\section{Deliciosa-da-beira}
\begin{itemize}
\item {Grp. gram.:f.}
\end{itemize}
O mesmo que \textunderscore bella-feia\textunderscore .
\section{Deliciosamente}
\begin{itemize}
\item {Grp. gram.:adv.}
\end{itemize}
De modo delicioso.
Com delícia.
\section{Delicioso}
\begin{itemize}
\item {Grp. gram.:adj.}
\end{itemize}
\begin{itemize}
\item {Proveniência:(Lat. \textunderscore deliciosus\textunderscore )}
\end{itemize}
Que causa delícia: \textunderscore manjar delicioso\textunderscore .
Perfeito, bem acabado.
\section{Delicodoce}
\begin{itemize}
\item {fónica:dé}
\end{itemize}
\begin{itemize}
\item {Grp. gram.:adj.}
\end{itemize}
\begin{itemize}
\item {Utilização:Deprec.}
\end{itemize}
\begin{itemize}
\item {Proveniência:(De \textunderscore delicado\textunderscore  + \textunderscore doce\textunderscore )}
\end{itemize}
Muito presumido.
Piegas.
\section{Delicto}
\begin{itemize}
\item {Grp. gram.:m.}
\end{itemize}
\begin{itemize}
\item {Proveniência:(Lat. \textunderscore delictum\textunderscore )}
\end{itemize}
Facto, que a lei declara punível; crime.
Culpa.
\section{Delictuoso}
\begin{itemize}
\item {Grp. gram.:adj.}
\end{itemize}
Em que há delicto: \textunderscore procedimento delictuoso\textunderscore .
\section{Deligação}
\begin{itemize}
\item {Grp. gram.:f.}
\end{itemize}
\begin{itemize}
\item {Proveniência:(Lat. \textunderscore deligatio\textunderscore )}
\end{itemize}
Ligadura; applicação de ligaduras.
\section{Delimitação}
\begin{itemize}
\item {Grp. gram.:f.}
\end{itemize}
\begin{itemize}
\item {Proveniência:(Lat. \textunderscore delimitatio\textunderscore )}
\end{itemize}
Acto de delimitar.
\section{Delimitador}
\begin{itemize}
\item {Grp. gram.:m.  e  adj.}
\end{itemize}
O que delimita.
\section{Delimitar}
\begin{itemize}
\item {Grp. gram.:v. t.}
\end{itemize}
\begin{itemize}
\item {Proveniência:(Lat. des. \textunderscore delimitare\textunderscore )}
\end{itemize}
Fixar os limites de: \textunderscore delimitar uma discussão\textunderscore .
Estremar; demarcar: \textunderscore Delimitar uma herdade\textunderscore .
\section{Delineação}
\begin{itemize}
\item {Grp. gram.:f.}
\end{itemize}
\begin{itemize}
\item {Proveniência:(Lat. \textunderscore delineatio\textunderscore )}
\end{itemize}
Acto ou effeito de delinear.
\section{Delineador}
\begin{itemize}
\item {Grp. gram.:m.}
\end{itemize}
Aquelle que delineia.
\section{Delineamento}
\begin{itemize}
\item {Grp. gram.:m.}
\end{itemize}
O mesmo que \textunderscore delineação\textunderscore .
\section{Delinear}
\begin{itemize}
\item {Grp. gram.:v. t.}
\end{itemize}
\begin{itemize}
\item {Proveniência:(Lat. \textunderscore delineare\textunderscore )}
\end{itemize}
Esboçar.
Fazer os traços geraes de.
Desenhar.
Descrever succintamente, de um modo geral.
Projectar, planear.
Demarcar.
\section{Delineativo}
\begin{itemize}
\item {Grp. gram.:adj.}
\end{itemize}
\begin{itemize}
\item {Proveniência:(De \textunderscore delinear\textunderscore )}
\end{itemize}
Relativo a delineação.
\section{Delingar}
\begin{itemize}
\item {Grp. gram.:v. t.}
\end{itemize}
\begin{itemize}
\item {Utilização:Prov.}
\end{itemize}
\begin{itemize}
\item {Utilização:trasm.}
\end{itemize}
Tornar pendente.
Soltar, deixando pender, (o cabello, etc.).
(Cp. \textunderscore linga\textunderscore ^1)
\section{Delinquência}
\begin{itemize}
\item {fónica:cu-en}
\end{itemize}
\begin{itemize}
\item {Grp. gram.:f.}
\end{itemize}
Estado ou qualidade de delinquente.
\section{Delinquente}
\begin{itemize}
\item {fónica:cu-en}
\end{itemize}
\begin{itemize}
\item {Grp. gram.:m. ,  f.  e  adj.}
\end{itemize}
\begin{itemize}
\item {Proveniência:(Lat. \textunderscore delinquens\textunderscore )}
\end{itemize}
Pessôa que delinquiu.
\section{Delinquir}
\begin{itemize}
\item {fónica:cu-ir}
\end{itemize}
\begin{itemize}
\item {Grp. gram.:v. t.}
\end{itemize}
\begin{itemize}
\item {Proveniência:(Lat. \textunderscore delinquere\textunderscore )}
\end{itemize}
Commeter delicto.
\section{Deliquar}
\begin{itemize}
\item {Grp. gram.:v.}
\end{itemize}
\begin{itemize}
\item {Utilização:t. Chím.}
\end{itemize}
\begin{itemize}
\item {Utilização:des.}
\end{itemize}
\begin{itemize}
\item {Proveniência:(Lat. \textunderscore deliquare\textunderscore )}
\end{itemize}
Pôr a derreter; decantar^2.
\section{Deliquescência}
\begin{itemize}
\item {fónica:cu-es}
\end{itemize}
\begin{itemize}
\item {Grp. gram.:f.}
\end{itemize}
Qualidade, que têm alguns corpos sólidos e mineraes, de absorver a humidade do ar e dissolver-se.
(Cp. \textunderscore deliquescente\textunderscore )
\section{Deliquescente}
\begin{itemize}
\item {fónica:cu-es}
\end{itemize}
\begin{itemize}
\item {Grp. gram.:adj.}
\end{itemize}
\begin{itemize}
\item {Proveniência:(Lat. \textunderscore deliquescens\textunderscore )}
\end{itemize}
Em que há deliquescência.
\section{Delíquio}
\begin{itemize}
\item {Grp. gram.:m.}
\end{itemize}
\begin{itemize}
\item {Proveniência:(Lat. \textunderscore deliquium\textunderscore )}
\end{itemize}
Acto de liquefazer-se, sôb a acção da humidade do ar.
Desmaio, sýncope.
\section{Delir}
\begin{itemize}
\item {Grp. gram.:v. t.}
\end{itemize}
\begin{itemize}
\item {Proveniência:(Lat. \textunderscore delere\textunderscore )}
\end{itemize}
Desfazer.
Apagar; destruir: \textunderscore o tempo deliu a pintura\textunderscore .
\section{Delirante}
\begin{itemize}
\item {Grp. gram.:adj.}
\end{itemize}
\begin{itemize}
\item {Proveniência:(Lat. \textunderscore delirans\textunderscore )}
\end{itemize}
Que delira.
Próprio de quem delira: \textunderscore palavras delirantes\textunderscore .
Semelhante ao delirio.
Excessivo, extraordinário: \textunderscore alegria delirante\textunderscore .
\section{Delirar}
\begin{itemize}
\item {Grp. gram.:v. t.}
\end{itemize}
\begin{itemize}
\item {Proveniência:(Lat. \textunderscore delirare\textunderscore )}
\end{itemize}
Têr delírio.
Estar muito apaixonado.
Exaltar-se.
Disparatar.
\section{Delírio}
\begin{itemize}
\item {Grp. gram.:m.}
\end{itemize}
\begin{itemize}
\item {Proveniência:(Lat. \textunderscore delirium\textunderscore )}
\end{itemize}
Perturbação intellectual, produzida por doença.
Exaltação.
Excesso de sentimento.
Enthusiasmo: \textunderscore aquella festa foi um delírio\textunderscore .
\section{Delirioso}
\begin{itemize}
\item {Grp. gram.:adj.}
\end{itemize}
\begin{itemize}
\item {Proveniência:(De \textunderscore delírio\textunderscore )}
\end{itemize}
Que tem delírio.
Que delira.
Em que há delírio.
Resultante de delírio. Cf. Filinto, XVI, p. 267.
\section{Delitescência}
\begin{itemize}
\item {Grp. gram.:f.}
\end{itemize}
\begin{itemize}
\item {Utilização:Med.}
\end{itemize}
\begin{itemize}
\item {Proveniência:(Do lat. \textunderscore delitescere\textunderscore )}
\end{itemize}
Desapparecimento súbito de uma affecção local e exterior.
\section{Delito}
\begin{itemize}
\item {Grp. gram.:m.}
\end{itemize}
\begin{itemize}
\item {Proveniência:(Lat. \textunderscore delictum\textunderscore )}
\end{itemize}
Facto, que a lei declara punível; crime.
Culpa.
\section{Delitoso}
\begin{itemize}
\item {Grp. gram.:adj.}
\end{itemize}
O mesmo que \textunderscore delictuoso\textunderscore . Cf. Usque, \textunderscore Tribulações\textunderscore , 20 v.^o
\section{Delituoso}
\begin{itemize}
\item {Grp. gram.:adj.}
\end{itemize}
Em que há delito: \textunderscore procedimento delituoso\textunderscore .
\section{Delivramento}
\begin{itemize}
\item {Grp. gram.:m.}
\end{itemize}
\begin{itemize}
\item {Utilização:Ant.}
\end{itemize}
Expulsão das secundinas ou páreas, em seguida ao parto.
(Cf. fr. \textunderscore délivrance\textunderscore )
\section{Delivrar}
\begin{itemize}
\item {Grp. gram.:v. t.}
\end{itemize}
\begin{itemize}
\item {Utilização:Ant.}
\end{itemize}
Dar liberdade a.
(Cf. fr. \textunderscore délivrer\textunderscore )
\section{Delivrar}
\begin{itemize}
\item {Grp. gram.:v. t.}
\end{itemize}
\begin{itemize}
\item {Utilização:Ant.}
\end{itemize}
\begin{itemize}
\item {Grp. gram.:V. p.}
\end{itemize}
\begin{itemize}
\item {Utilização:Prov.}
\end{itemize}
\begin{itemize}
\item {Utilização:Ant.}
\end{itemize}
\begin{itemize}
\item {Proveniência:(Lat. \textunderscore deliberare\textunderscore )}
\end{itemize}
O mesmo que \textunderscore deliberar\textunderscore . Cf. D. Nunes do Lião, etc.
Expellir as páreas ou secundinas (a parturiente).
\section{Della}
\begin{itemize}
\item {Grp. gram.:Pl.}
\end{itemize}
Fórma contrahida de \textunderscore de\textunderscore  + \textunderscore ella\textunderscore .
Alguns, parte:«\textunderscore e as náos que achou no porto, dellas afundou; e taes rombos fez ás mais...\textunderscore »Filinto. \textunderscore D. Man.\textunderscore , I, 121.
\section{Dêlle}
Fórma contrahida de \textunderscore de\textunderscore  + \textunderscore êlle\textunderscore .
\section{Delombar}
\begin{itemize}
\item {Grp. gram.:v. t.}
\end{itemize}
(V.deslombar)
\section{Delonga}
\begin{itemize}
\item {Grp. gram.:f.}
\end{itemize}
\begin{itemize}
\item {Proveniência:(De \textunderscore delongar\textunderscore )}
\end{itemize}
Demora; dilação.
Adiamento.
\section{Delongador}
\begin{itemize}
\item {Grp. gram.:m.}
\end{itemize}
\begin{itemize}
\item {Proveniência:(De \textunderscore delongar\textunderscore )}
\end{itemize}
Aquelle que delonga.
\section{Delongamento}
\begin{itemize}
\item {Grp. gram.:m.}
\end{itemize}
\begin{itemize}
\item {Utilização:Des.}
\end{itemize}
O mesmo que \textunderscore delonga\textunderscore .
\section{Delongar}
\begin{itemize}
\item {Grp. gram.:v. t.}
\end{itemize}
\begin{itemize}
\item {Proveniência:(De \textunderscore longo\textunderscore )}
\end{itemize}
Demorar.
Adiar.
\section{Delóstomo}
\begin{itemize}
\item {Grp. gram.:m.}
\end{itemize}
Gênero de plantas bignoniáceas.
\section{Délphico}
\begin{itemize}
\item {Grp. gram.:m.}
\end{itemize}
\begin{itemize}
\item {Proveniência:(Gr. \textunderscore delphikos\textunderscore )}
\end{itemize}
Pequena mesa de três pés, ou pequeno aparador, para exposição de baixelas ricas.
\section{Délphico}
\begin{itemize}
\item {Grp. gram.:adj.}
\end{itemize}
Relativo a Delphos ou ao seu oráculo.
\section{Delta}
\begin{itemize}
\item {Grp. gram.:m.}
\end{itemize}
\begin{itemize}
\item {Proveniência:(Gr. \textunderscore delta\textunderscore )}
\end{itemize}
Quarta letra do alphabeto grego, a qual corresponde ao nosso D e tem a fórma de Δ.
Terreno triangular, da fórma de um delta, e em que dois lados são formados por dois braços de um rio e outro pela costa do mar, como o \textunderscore delta do Nilo\textunderscore .
\section{Deltocarpo}
\begin{itemize}
\item {Grp. gram.:adj.}
\end{itemize}
\begin{itemize}
\item {Proveniência:(Do gr. \textunderscore delta\textunderscore  + \textunderscore karpos\textunderscore )}
\end{itemize}
Que tem frutos triangulares.
\section{Deitoidal}
\begin{itemize}
\item {Grp. gram.:adj.}
\end{itemize}
\begin{itemize}
\item {Grp. gram.:M.}
\end{itemize}
\begin{itemize}
\item {Utilização:Anat.}
\end{itemize}
\begin{itemize}
\item {Utilização:Mathem.}
\end{itemize}
\begin{itemize}
\item {Grp. gram.:M. pl.}
\end{itemize}
\begin{itemize}
\item {Proveniência:(Do gr. \textunderscore delta\textunderscore  + \textunderscore eidos\textunderscore )}
\end{itemize}
Que tem fórma de delta.
Um dos músculos da espádua.
Trapezoide, com duas diagonaes rectangulares, uma das quaes divide a figura em dois triângulos escalenos symétricos, e a outra em dois triângulos isósceles iguaes, justapostos pela base.
Família de insectos nocturnos.
\section{Deltoide}
\begin{itemize}
\item {Grp. gram.:adj.}
\end{itemize}
\begin{itemize}
\item {Grp. gram.:M.}
\end{itemize}
\begin{itemize}
\item {Utilização:Anat.}
\end{itemize}
\begin{itemize}
\item {Utilização:Mathem.}
\end{itemize}
\begin{itemize}
\item {Grp. gram.:M. pl.}
\end{itemize}
\begin{itemize}
\item {Proveniência:(Do gr. \textunderscore delta\textunderscore  + \textunderscore eidos\textunderscore )}
\end{itemize}
Que tem fórma de delta.
Um dos músculos da espádua.
Trapezoide, com duas diagonaes rectangulares, uma das quaes divide a figura em dois triângulos escalenos symétricos, e a outra em dois triângulos isósceles iguaes, justapostos pela base.
Família de insectos nocturnos.
\section{Deltoideano}
\begin{itemize}
\item {Grp. gram.:adj.}
\end{itemize}
\begin{itemize}
\item {Utilização:Anat.}
\end{itemize}
Relativo ao músculo deltoide.
\section{Deltoídeo}
\begin{itemize}
\item {Grp. gram.:adj.}
\end{itemize}
Que tem fórma de delta.
O mesmo que \textunderscore deltoide\textunderscore  e \textunderscore deltoideano\textunderscore .
\section{Deltota}
\begin{itemize}
\item {Grp. gram.:m.}
\end{itemize}
\begin{itemize}
\item {Proveniência:(Gr. \textunderscore deltoton\textunderscore )}
\end{itemize}
Triângulo, formado por muitas estrellas, junto da constellação de Andrómeda.
\section{Delubro}
\begin{itemize}
\item {Grp. gram.:m.}
\end{itemize}
\begin{itemize}
\item {Utilização:Poét.}
\end{itemize}
\begin{itemize}
\item {Proveniência:(Lat. \textunderscore delubrum\textunderscore )}
\end{itemize}
Templo pagão.
\section{Delúcia}
\begin{itemize}
\item {Grp. gram.:f.}
\end{itemize}
\begin{itemize}
\item {Proveniência:(De \textunderscore Deluc\textunderscore , n. p.)}
\end{itemize}
Planta do México.
\section{Deludir}
\textunderscore v. t.\textunderscore  (e der.)
(V. \textunderscore illudir\textunderscore , etc.)
\section{Delusão}
\begin{itemize}
\item {Grp. gram.:m.}
\end{itemize}
\begin{itemize}
\item {Utilização:Des.}
\end{itemize}
\begin{itemize}
\item {Proveniência:(Lat. \textunderscore delusio\textunderscore )}
\end{itemize}
Engano, burla.
\section{Deluso}
\begin{itemize}
\item {Grp. gram.:adj.}
\end{itemize}
\begin{itemize}
\item {Proveniência:(Lat. \textunderscore delusus\textunderscore )}
\end{itemize}
Que engana.
\section{Delusório}
\begin{itemize}
\item {Grp. gram.:adj.}
\end{itemize}
O mesmo que \textunderscore illusório\textunderscore . Cf. Latino, \textunderscore Elogios Acad.\textunderscore , I, 143.
\section{Deluzir-se}
\begin{itemize}
\item {Grp. gram.:v. p.}
\end{itemize}
\begin{itemize}
\item {Proveniência:(De \textunderscore de...\textunderscore  + \textunderscore luzir\textunderscore )}
\end{itemize}
Perder a luz ou o brilho, desvanecer-se, apagar-se:«\textunderscore deluz-se-nos na alma o que mais amamos\textunderscore ». Camillo, \textunderscore Freira no Subterr.\textunderscore , 220.
\section{Demacuris}
\begin{itemize}
\item {Grp. gram.:m. pl.}
\end{itemize}
Tríbo de Índios do Brasil, nas margens do Rio Negro.
\section{Demagogia}
\begin{itemize}
\item {Grp. gram.:f.}
\end{itemize}
\begin{itemize}
\item {Proveniência:(Gr. \textunderscore demagogia\textunderscore )}
\end{itemize}
Preponderância ou govêrno de facções populares.
Anarchia.
\section{Demagógico}
\begin{itemize}
\item {Grp. gram.:adj.}
\end{itemize}
Relativo a demagogia.
\section{Demagogo}
\begin{itemize}
\item {Grp. gram.:m.}
\end{itemize}
\begin{itemize}
\item {Proveniência:(Gr. \textunderscore demagogos\textunderscore , de \textunderscore demos\textunderscore , povo, e \textunderscore agein\textunderscore , conduzir)}
\end{itemize}
Chefe de facções populares, na Grécia antiga.
Sectário da demagogia.
Homem, que excita as paixões populares; revolucionário.
\section{Demais}
\begin{itemize}
\item {Grp. gram.:adv.}
\end{itemize}
\begin{itemize}
\item {Grp. gram.:Adj.}
\end{itemize}
\begin{itemize}
\item {Grp. gram.:Loc. adv.}
\end{itemize}
\begin{itemize}
\item {Proveniência:(Lat. \textunderscore demagis\textunderscore , us. na decadência)}
\end{itemize}
Excessivamente.
Além d'isso.
Demasiado; excessivo.
\textunderscore Por demais\textunderscore , indifferentemente, inutilmente.
\section{Demanda}
\begin{itemize}
\item {Grp. gram.:f.}
\end{itemize}
Acto de demandar.
Acção judicial, por processo civil.
Litigio.
Combate; discussão.
\section{Demandador}
\begin{itemize}
\item {Grp. gram.:m.}
\end{itemize}
Aquelle que demanda.
\section{Demandança}
\begin{itemize}
\item {Grp. gram.:f.}
\end{itemize}
\begin{itemize}
\item {Utilização:Ant.}
\end{itemize}
Acto de demandar.
Questão, contenda.
\section{Demandante}
\begin{itemize}
\item {Grp. gram.:m.  e  adj.}
\end{itemize}
\begin{itemize}
\item {Proveniência:(Lat. \textunderscore demandans\textunderscore )}
\end{itemize}
O que demanda.
\section{Demandão}
\begin{itemize}
\item {Grp. gram.:m.}
\end{itemize}
\begin{itemize}
\item {Utilização:Pop.}
\end{itemize}
O mesmo que \textunderscore demandista\textunderscore .
\section{Demandar}
\begin{itemize}
\item {Grp. gram.:v. t.}
\end{itemize}
\begin{itemize}
\item {Grp. gram.:V. i.}
\end{itemize}
\begin{itemize}
\item {Proveniência:(Lat. \textunderscore demandare\textunderscore )}
\end{itemize}
Ir em procura de.
Buscar: \textunderscore demandar sítios ermos\textunderscore .
Reclamar.
Pedir.
Precisar: \textunderscore êste barco demanda trinta pés de água\textunderscore .
Intentar litígio judicial contra.
O mesmo que \textunderscore perguntar\textunderscore .
\section{Demandista}
\begin{itemize}
\item {Grp. gram.:m.  e  f.}
\end{itemize}
\begin{itemize}
\item {Proveniência:(De \textunderscore demandar\textunderscore )}
\end{itemize}
Pessôa, que intenta demandas.
Pessôa muito dada a pleitos judiciaes.
\section{Demão}
\begin{itemize}
\item {Grp. gram.:f.}
\end{itemize}
Camada de tinta ou cal, que se estende numa superfície.
Cada uma das vezes em que se recomeça um trabalho ou em que se retoma um assumpto: \textunderscore está dando a última demão num poema\textunderscore .
Ajuda, auxílio: \textunderscore dá aqui uma demão neste mólho\textunderscore .
\section{Demarcação}
\begin{itemize}
\item {Grp. gram.:f.}
\end{itemize}
Acto de demarcar.
\section{Demarcadamente}
\begin{itemize}
\item {Grp. gram.:adv.}
\end{itemize}
Com demarcação.
\section{Demarcador}
\begin{itemize}
\item {Grp. gram.:m.}
\end{itemize}
Aquelle que demarca.
\section{Demarcar}
\begin{itemize}
\item {Grp. gram.:v. t.}
\end{itemize}
\begin{itemize}
\item {Proveniência:(De \textunderscore marcar\textunderscore )}
\end{itemize}
Traçar os limites de.
Estremar.
Definir.
Determinar.
Separar.
\section{Demarcável}
\begin{itemize}
\item {Grp. gram.:adj.}
\end{itemize}
Que se póde demarcar.
\section{Demasia}
\begin{itemize}
\item {Grp. gram.:f.}
\end{itemize}
Aquillo que é demais; excesso.
Sobejo; resto; sobras.
Intemperança.
Desregramento.
Abuso; temeridade.
(Por \textunderscore demaisia\textunderscore , de \textunderscore demais\textunderscore )
\section{Demasiadamente}
\begin{itemize}
\item {Grp. gram.:adv.}
\end{itemize}
De modo demasiado.
\section{Demasiado}
\begin{itemize}
\item {Grp. gram.:adj.}
\end{itemize}
\begin{itemize}
\item {Proveniência:(De \textunderscore demasia\textunderscore )}
\end{itemize}
Que passa dos justos limites.
Excessivo; supérfluo.
Desregrado; abusivo.
\section{Demasiar-se}
\begin{itemize}
\item {Grp. gram.:v. p.}
\end{itemize}
\begin{itemize}
\item {Proveniência:(De \textunderscore demasia\textunderscore )}
\end{itemize}
Exceder-se.
Passar além do limite razoável.
\section{Dembe}
\begin{itemize}
\item {Grp. gram.:m.}
\end{itemize}
O mesmo que \textunderscore dembo\textunderscore . Cf. Capello e Ivens, \textunderscore De Benguela\textunderscore , II, 200 e 354.
\section{Dembo}
\begin{itemize}
\item {Grp. gram.:m.}
\end{itemize}
Chefe de tríbo, ao norte de Angola.
Espécie de tambor dos negros de Loango.
\section{Dembos}
\begin{itemize}
\item {Grp. gram.:m. pl.}
\end{itemize}
Povos independentes, entre o Dande superior e o Lombije.
\section{Demear}
\begin{itemize}
\item {Grp. gram.:v. t.}
\end{itemize}
\begin{itemize}
\item {Utilização:Des.}
\end{itemize}
\begin{itemize}
\item {Proveniência:(De \textunderscore meio\textunderscore )}
\end{itemize}
Encher até metade.
Occupar metade de.
\section{Demedian}
\begin{itemize}
\item {Grp. gram.:m.}
\end{itemize}
\begin{itemize}
\item {Utilização:Ant.}
\end{itemize}
Moéda de liga de cobre e prata, em Cambaia.
\section{De-meiacara}
\begin{itemize}
\item {Grp. gram.:loc. adv.}
\end{itemize}
\begin{itemize}
\item {Utilização:Bras}
\end{itemize}
De graça.
\section{Demência}
\begin{itemize}
\item {Grp. gram.:f.}
\end{itemize}
\begin{itemize}
\item {Proveniência:(Lat. dementia)}
\end{itemize}
Estado de demente.
Loucura; insensatez.
Acção insensata.
\section{Demenso}
\begin{itemize}
\item {Grp. gram.:m.}
\end{itemize}
\begin{itemize}
\item {Proveniência:(Lat. \textunderscore demensus\textunderscore )}
\end{itemize}
Medida do pão, que se dava aos escravos, entre os Romanos.
\section{Dementação}
\begin{itemize}
\item {Grp. gram.:f.}
\end{itemize}
\begin{itemize}
\item {Proveniência:(De \textunderscore dementar\textunderscore )}
\end{itemize}
Demência.
\section{Dementar}
\begin{itemize}
\item {Grp. gram.:v. t.}
\end{itemize}
\begin{itemize}
\item {Proveniência:(Do lat. \textunderscore dementare\textunderscore )}
\end{itemize}
Tornar demente, louco, doido.
\section{Demente}
\begin{itemize}
\item {Grp. gram.:m.  e  adj.}
\end{itemize}
\begin{itemize}
\item {Proveniência:(Lat. \textunderscore demens\textunderscore )}
\end{itemize}
Louco; insensato.
\section{Dementre}
\begin{itemize}
\item {Grp. gram.:adv.}
\end{itemize}
\begin{itemize}
\item {Utilização:Ant.}
\end{itemize}
Entretanto.
Em quanto não.
(Cp. \textunderscore entrementes\textunderscore )
\section{Demerger}
\begin{itemize}
\item {Grp. gram.:v. t.}
\end{itemize}
\begin{itemize}
\item {Utilização:Des.}
\end{itemize}
Applicar; inclinar:«\textunderscore demerge ta orelha, ó prove\textunderscore ». Herculano, \textunderscore Cister\textunderscore , I, 7.
\section{Demérito}
\begin{itemize}
\item {Grp. gram.:m.}
\end{itemize}
\begin{itemize}
\item {Grp. gram.:Adj.}
\end{itemize}
O mesmo que \textunderscore desmerecimento\textunderscore .
Falta de mérito.
Que não tem merecimento, que perdeu o merecimento. Cf. Filinto, IX, 160.
\section{Demeritório}
\begin{itemize}
\item {Grp. gram.:adj.}
\end{itemize}
Relativo a demérito.
\section{Demigola}
\begin{itemize}
\item {Grp. gram.:f.}
\end{itemize}
(V.semigola)
\section{Demigolla}
\begin{itemize}
\item {Grp. gram.:f.}
\end{itemize}
(V.semigolla)
\section{Demi-lunar}
\begin{itemize}
\item {Grp. gram.:adj.}
\end{itemize}
\begin{itemize}
\item {Proveniência:(T. hýbr., do fr. \textunderscore demi\textunderscore  + lat. \textunderscore luna\textunderscore )}
\end{itemize}
Que tem fórma de meia lua. Cf. Garrett, \textunderscore Dona Branca\textunderscore , 51.
\section{Deminuendo}
\begin{itemize}
\item {Grp. gram.:m.}
\end{itemize}
\begin{itemize}
\item {Utilização:Arith.}
\end{itemize}
\begin{itemize}
\item {Proveniência:(Lat. \textunderscore deminuendus\textunderscore )}
\end{itemize}
Número, de que se subtrai outro.
\section{Deminuente}
\begin{itemize}
\item {Grp. gram.:adj.}
\end{itemize}
\begin{itemize}
\item {Proveniência:(Lat. \textunderscore deminuens\textunderscore )}
\end{itemize}
Que deminue.
\section{Deminuição}
\begin{itemize}
\item {fónica:nu-i}
\end{itemize}
\begin{itemize}
\item {Grp. gram.:f.}
\end{itemize}
Acto ou effeito de deminuir.
\section{Deminuidor}
\begin{itemize}
\item {fónica:nu-i}
\end{itemize}
\begin{itemize}
\item {Grp. gram.:m.}
\end{itemize}
\begin{itemize}
\item {Grp. gram.:M.}
\end{itemize}
\begin{itemize}
\item {Utilização:Arith.}
\end{itemize}
Que deminue.
Número, que se subtrai do deminuendo.
\section{Deminuir}
\begin{itemize}
\item {Grp. gram.:v. t.}
\end{itemize}
\begin{itemize}
\item {Grp. gram.:V. i.}
\end{itemize}
\begin{itemize}
\item {Proveniência:(Lat. \textunderscore deminuere\textunderscore )}
\end{itemize}
Tornar menor em dimensões ou quantidade.
Subtrahir (um número) de outro.
Tornar raro: \textunderscore a hygiene diminue as doenças\textunderscore .
Abreviar: \textunderscore os excessos deminuem os dias da vida\textunderscore .
Abater: \textunderscore deminuir as fôrças\textunderscore .
Attenuar: \textunderscore o arrependimento deminue as responsabilidades do crime\textunderscore .
Decrescer.
Tornar-se menor: \textunderscore os dias agora já deminuem\textunderscore .
Apoucar-se.
Emmagrecer.
Moderar-se.
\section{Deminutamente}
\begin{itemize}
\item {Grp. gram.:adv.}
\end{itemize}
De modo deminuto.
Parcamente.
\section{Deminutivamente}
\begin{itemize}
\item {Grp. gram.:adv.}
\end{itemize}
De modo deminutivo.
\section{Deminutivo}
\begin{itemize}
\item {Grp. gram.:adv.}
\end{itemize}
\begin{itemize}
\item {Utilização:Gram.}
\end{itemize}
\begin{itemize}
\item {Grp. gram.:M.}
\end{itemize}
\begin{itemize}
\item {Utilização:Gram.}
\end{itemize}
\begin{itemize}
\item {Proveniência:(Lat. \textunderscore deminutivus\textunderscore )}
\end{itemize}
Que deminue.
Que attenua ou adoça a significação de uma palavra: \textunderscore avòzinha é vocábulo deminutivo\textunderscore .
Palavra deminutiva: \textunderscore filhinho, casinha, pequenito, asseadinha, burrico, portella, são deminutivos\textunderscore .
\section{Deminuto}
\begin{itemize}
\item {Grp. gram.:adj.}
\end{itemize}
\begin{itemize}
\item {Proveniência:(Lat. \textunderscore deminutus\textunderscore )}
\end{itemize}
Muito pouco.
Reduzido a pequenas dimensões.
Pequenino; escasso: \textunderscore um rendimento deminuto\textunderscore .
\section{Demissão}
\begin{itemize}
\item {Grp. gram.:f.}
\end{itemize}
\begin{itemize}
\item {Proveniência:(Lat. \textunderscore demissio\textunderscore )}
\end{itemize}
Acto ou effeito de demittir.
\section{Demissionário}
\begin{itemize}
\item {Grp. gram.:adj.}
\end{itemize}
\begin{itemize}
\item {Proveniência:(Do lat. \textunderscore demissio\textunderscore )}
\end{itemize}
Que se demittiu.
Que pediu a demissão: \textunderscore o Ministério está demissionário\textunderscore .
\section{Demisso}
\begin{itemize}
\item {Grp. gram.:adj.}
\end{itemize}
\begin{itemize}
\item {Proveniência:(Lat. \textunderscore demissus\textunderscore )}
\end{itemize}
O mesmo que \textunderscore demittido\textunderscore .
\section{Demissor}
\begin{itemize}
\item {Grp. gram.:adj.}
\end{itemize}
O mesmo que \textunderscore demissório\textunderscore .
\section{Demissório}
\begin{itemize}
\item {Grp. gram.:adj.}
\end{itemize}
(V.dimissório)
\section{Demittente}
\begin{itemize}
\item {Grp. gram.:adj.}
\end{itemize}
\begin{itemize}
\item {Proveniência:(Lat. \textunderscore demittens\textunderscore )}
\end{itemize}
O mesmo que \textunderscore demissionário\textunderscore .
\section{Demittido}
\begin{itemize}
\item {Grp. gram.:adj.}
\end{itemize}
\begin{itemize}
\item {Proveniência:(De \textunderscore demittir\textunderscore )}
\end{itemize}
Que teve demissão.
\section{Demittir}
\begin{itemize}
\item {Grp. gram.:v. t.}
\end{itemize}
\begin{itemize}
\item {Proveniência:(Lat. \textunderscore demittere\textunderscore )}
\end{itemize}
Tirar um cargo a.
Destituir de emprêgo.
Renunciar.
Despedir; exonerar.
\section{Demiúrgico}
\begin{itemize}
\item {Grp. gram.:adj.}
\end{itemize}
Relativo ao demiurgo.
\section{Demiurgo}
\begin{itemize}
\item {Grp. gram.:m.}
\end{itemize}
\begin{itemize}
\item {Proveniência:(Gr. \textunderscore demiourgos\textunderscore )}
\end{itemize}
Nome, que os philósophos platónicos davam ao criador dos homens.
\section{Demo}
\begin{itemize}
\item {Grp. gram.:m.}
\end{itemize}
\begin{itemize}
\item {Utilização:Fam.}
\end{itemize}
O mesmo que \textunderscore demónio\textunderscore .
Pessôa turbulenta.
Pessôa astuciosa.
\section{Demo}
\begin{itemize}
\item {Grp. gram.:m.}
\end{itemize}
\begin{itemize}
\item {Grp. gram.:Pl.}
\end{itemize}
\begin{itemize}
\item {Proveniência:(Gr. \textunderscore demos\textunderscore )}
\end{itemize}
Cada um dos burgos que constituíam certas tríbos da África.
Conjunto de organismos vivos, no primeiro momento da sua expansão social.
\section{Demóboro}
\begin{itemize}
\item {Grp. gram.:adj.}
\end{itemize}
Anti-democrático:«\textunderscore ...e se desligou da funesta, demóbora alliança dos reis\textunderscore ». Garrett, \textunderscore Port. na Balança\textunderscore , 128.
\section{Democracia}
\begin{itemize}
\item {Grp. gram.:f.}
\end{itemize}
\begin{itemize}
\item {Proveniência:(Gr. \textunderscore demokratia\textunderscore )}
\end{itemize}
Soberania popular.
Govêrno do povo.
Influência do povo na governação pública.
Classe social, que comprehende o operariado e a população ínfima.
\section{Democrata}
\begin{itemize}
\item {Grp. gram.:m.}
\end{itemize}
\begin{itemize}
\item {Proveniência:(Gr. \textunderscore demokratos\textunderscore )}
\end{itemize}
Sectário do govêrno democrático.
Aquelle que pertence á classe popular ou que não gosta da aristocracia.--A pronúncia exacta é \textunderscore demócrata\textunderscore , mas não se usa.
\section{Democraticamente}
\begin{itemize}
\item {Grp. gram.:adv.}
\end{itemize}
De modo democrático.
Conformemente á democracia: \textunderscore proceder democraticamente\textunderscore .
\section{Democrático}
\begin{itemize}
\item {Grp. gram.:adj.}
\end{itemize}
\begin{itemize}
\item {Proveniência:(Gr. \textunderscore demokratikos\textunderscore )}
\end{itemize}
Relativo á democracia.
\section{Democratismo}
\begin{itemize}
\item {Grp. gram.:m.}
\end{itemize}
O mesmo que \textunderscore democracia\textunderscore .
\section{Democratização}
\begin{itemize}
\item {Grp. gram.:f.}
\end{itemize}
Acto ou effeito de \textunderscore democratizar\textunderscore .
\section{Democratizar}
\begin{itemize}
\item {Grp. gram.:v. t.}
\end{itemize}
\begin{itemize}
\item {Proveniência:(De \textunderscore democrata\textunderscore  ou \textunderscore democrático\textunderscore )}
\end{itemize}
Tornar democrata ou democrático.
\section{Democrítico}
\begin{itemize}
\item {Grp. gram.:adj.}
\end{itemize}
Relativo a Demócrito ou ao seu systema.
Que se ri de tudo.
\section{Demografia}
\begin{itemize}
\item {Grp. gram.:f.}
\end{itemize}
\begin{itemize}
\item {Proveniência:(Do gr. \textunderscore demos\textunderscore  + \textunderscore graphein\textunderscore )}
\end{itemize}
Estatistica da população.
\section{Demográfico}
\begin{itemize}
\item {Grp. gram.:adj.}
\end{itemize}
Relativo á demografia.
\section{Demografista}
\begin{itemize}
\item {Grp. gram.:m.}
\end{itemize}
Aquele que se dedica a estudos demográficos.
\section{Demógrafo}
\begin{itemize}
\item {Grp. gram.:m.}
\end{itemize}
Aquele que se ocupa de demografia.
\section{Demographia}
\begin{itemize}
\item {Grp. gram.:f.}
\end{itemize}
\begin{itemize}
\item {Proveniência:(Do gr. \textunderscore demos\textunderscore  + \textunderscore graphein\textunderscore )}
\end{itemize}
Estatistica da população.
\section{Demográphico}
\begin{itemize}
\item {Grp. gram.:adj.}
\end{itemize}
Relativo á demographia.
\section{Demographista}
\begin{itemize}
\item {Grp. gram.:m.}
\end{itemize}
Aquelle que se dedica a estudos demográphicos.
\section{Demógrapho}
\begin{itemize}
\item {Grp. gram.:m.}
\end{itemize}
Aquelle que se occupa de demographia.
\section{Demoisella}
\begin{itemize}
\item {Grp. gram.:f.}
\end{itemize}
\begin{itemize}
\item {Proveniência:(Fr. \textunderscore demoiselle\textunderscore )}
\end{itemize}
Mulher francesa, ainda solteira ou virgem. Cf. Filinto, XIX, p. 87.
\section{Demolhar}
\begin{itemize}
\item {Grp. gram.:v. t.}
\end{itemize}
\begin{itemize}
\item {Proveniência:(De \textunderscore molhar\textunderscore )}
\end{itemize}
Pôr de môlho em água: \textunderscore demolhar o bacalhau\textunderscore .
\section{Demolição}
\begin{itemize}
\item {Grp. gram.:f.}
\end{itemize}
\begin{itemize}
\item {Proveniência:(Lat. \textunderscore demolitio\textunderscore )}
\end{itemize}
Acto ou effeito de demolir.
\section{Demolidor}
\begin{itemize}
\item {Grp. gram.:m.}
\end{itemize}
Aquelle que demole.
\section{Demolir}
\begin{itemize}
\item {Grp. gram.:v. t}
\end{itemize}
\begin{itemize}
\item {Proveniência:(Lat. \textunderscore demoliri\textunderscore )}
\end{itemize}
Desfazer (qualquer construcção): \textunderscore demolir um prédio\textunderscore .
Destruir.
Arrasar.
Aniquilar: \textunderscore demolir projectos\textunderscore .
\section{Demolitório}
\begin{itemize}
\item {Grp. gram.:adj.}
\end{itemize}
\begin{itemize}
\item {Proveniência:(De \textunderscore demolir\textunderscore )}
\end{itemize}
Próprio para demolir.
\section{Demonarca}
\begin{itemize}
\item {Grp. gram.:m.}
\end{itemize}
\begin{itemize}
\item {Proveniência:(Do gr. \textunderscore daimon\textunderscore  + \textunderscore arkhein\textunderscore )}
\end{itemize}
O demónio principal.
\section{Demonarcha}
\begin{itemize}
\item {Grp. gram.:m.}
\end{itemize}
\begin{itemize}
\item {Proveniência:(Do gr. \textunderscore daimon\textunderscore  + \textunderscore arkhein\textunderscore )}
\end{itemize}
O demónio principal.
\section{Demonete}
\begin{itemize}
\item {fónica:nê}
\end{itemize}
\begin{itemize}
\item {Grp. gram.:m.}
\end{itemize}
\begin{itemize}
\item {Proveniência:(De \textunderscore demónio\textunderscore )}
\end{itemize}
Criança endiabrada, travêssa.
\section{Demonetização}
\begin{itemize}
\item {Grp. gram.:f.}
\end{itemize}
Acto de demonetizar.
\section{Demonetizar}
\begin{itemize}
\item {Grp. gram.:v. t.}
\end{itemize}
\begin{itemize}
\item {Proveniência:(Do lat. \textunderscore de\textunderscore  + \textunderscore moneta\textunderscore )}
\end{itemize}
Tirar o valor de moéda a; desamoedar.
\section{Demonho}
\begin{itemize}
\item {Grp. gram.:m.}
\end{itemize}
\begin{itemize}
\item {Utilização:Prov.}
\end{itemize}
O mesmo que \textunderscore demónio\textunderscore .
\section{Demónia}
\begin{itemize}
\item {Grp. gram.:f.}
\end{itemize}
\begin{itemize}
\item {Utilização:P. us.}
\end{itemize}
\begin{itemize}
\item {Utilização:Fam.}
\end{itemize}
\begin{itemize}
\item {Proveniência:(De \textunderscore demónio\textunderscore )}
\end{itemize}
Mulher má.
\section{Demoníaco}
\begin{itemize}
\item {Grp. gram.:adj.}
\end{itemize}
Relativo ao demónio.
\section{Demonico}
\begin{itemize}
\item {Grp. gram.:m.}
\end{itemize}
O mesmo que \textunderscore demonete\textunderscore .
\section{Demonifúgio}
\begin{itemize}
\item {Grp. gram.:m.}
\end{itemize}
Exorcismo, água benta ou qualquer acto, a que se attribue a virtude de afugentar o demónio.
(Cp. \textunderscore demonífugo\textunderscore )
\section{Demonífugo}
\begin{itemize}
\item {Grp. gram.:adj.}
\end{itemize}
\begin{itemize}
\item {Proveniência:(Do lat. \textunderscore demonium\textunderscore  + \textunderscore fugere\textunderscore )}
\end{itemize}
Que afugenta o demónio ou as más tentações. Cf. Camillo, \textunderscore Olho de Vidro\textunderscore , 162.
\section{Demoninhado}
\begin{itemize}
\item {Grp. gram.:adj.}
\end{itemize}
\begin{itemize}
\item {Utilização:Des.}
\end{itemize}
O mesmo que \textunderscore endemoninhado\textunderscore . Cf. G. Vicente, I, 169.
\section{Demónio}
\begin{itemize}
\item {Grp. gram.:m.}
\end{itemize}
\begin{itemize}
\item {Grp. gram.:Loc. adv.}
\end{itemize}
\begin{itemize}
\item {Grp. gram.:Loc. interj.}
\end{itemize}
\begin{itemize}
\item {Proveniência:(Gr. \textunderscore daimonion\textunderscore )}
\end{itemize}
Espírito sobrenatural, que, segundo o Polytheísmo, presidia ao destino de cada homem.
Gênio do mal ou espírito sobrenatural, que, segundo o Christianismo, procura a perdição dos homens.
Diabo, satanás, belzebu.
Pessôa ruim.
Pessôa turbulenta.
Pessôa feia.
\textunderscore Como um demónio\textunderscore , muito; ardentemente.
\textunderscore Com mil demónios!\textunderscore  ou \textunderscore com seiscentos demónios!\textunderscore  ou \textunderscore com os demónios!\textunderscore  (indicativa de admiração, espanto ou terror)
\section{Demoniomania}
\begin{itemize}
\item {Grp. gram.:f.}
\end{itemize}
\begin{itemize}
\item {Proveniência:(De \textunderscore demónio\textunderscore  + \textunderscore mania\textunderscore )}
\end{itemize}
Crença supersticiosa na influência e poder dos demónios.
\section{Demoniomaníaco}
\begin{itemize}
\item {Grp. gram.:adj.}
\end{itemize}
\begin{itemize}
\item {Grp. gram.:M.}
\end{itemize}
Relativo á demoniomania.
Aquelle que tem demoniomania.
\section{Demonismo}
\begin{itemize}
\item {Grp. gram.:m.}
\end{itemize}
\begin{itemize}
\item {Proveniência:(De \textunderscore demónio\textunderscore )}
\end{itemize}
Crença em demónios.
\section{Demonista}
\begin{itemize}
\item {Grp. gram.:m.}
\end{itemize}
Sectário do demonismo.
\section{Demonografia}
\begin{itemize}
\item {Grp. gram.:f.}
\end{itemize}
\begin{itemize}
\item {Proveniência:(De \textunderscore demonógrafo\textunderscore )}
\end{itemize}
Tratado da natureza ou influência dos demónios.
\section{Demonógrafo}
\begin{itemize}
\item {Grp. gram.:m.}
\end{itemize}
\begin{itemize}
\item {Proveniência:(Do gr. \textunderscore daimon\textunderscore  + \textunderscore graphein\textunderscore )}
\end{itemize}
Aquele que trata de demonografia.
\section{Demonographia}
\begin{itemize}
\item {Grp. gram.:f.}
\end{itemize}
\begin{itemize}
\item {Proveniência:(De \textunderscore demonógrapho\textunderscore )}
\end{itemize}
Tratado da natureza ou influência dos demónios.
\section{Demonógrapho}
\begin{itemize}
\item {Grp. gram.:m.}
\end{itemize}
\begin{itemize}
\item {Proveniência:(Do gr. \textunderscore daimon\textunderscore  + \textunderscore graphein\textunderscore )}
\end{itemize}
Aquelle que trata de demonografia.
\section{Demonólatra}
\begin{itemize}
\item {Grp. gram.:m.}
\end{itemize}
\begin{itemize}
\item {Proveniência:(Do gr. \textunderscore daimon\textunderscore  + \textunderscore latreuein\textunderscore )}
\end{itemize}
Adorador de demónios.
\section{Demonolátrico}
\begin{itemize}
\item {Grp. gram.:adj.}
\end{itemize}
Relativo á demonolatria.
\section{Demonolatria}
\begin{itemize}
\item {Grp. gram.:f.}
\end{itemize}
Culto dos demónios.
(Cp. \textunderscore demonólatra\textunderscore )
\section{Demonologia}
\begin{itemize}
\item {Grp. gram.:f.}
\end{itemize}
O mesmo que \textunderscore demonographia\textunderscore .
\section{Demonológico}
\begin{itemize}
\item {Grp. gram.:adj.}
\end{itemize}
Relativo á demonologia.
\section{Demonólogo}
\begin{itemize}
\item {Grp. gram.:adj.}
\end{itemize}
O mesmo que \textunderscore demonógrafo\textunderscore .
\section{Demonomancia}
\begin{itemize}
\item {Grp. gram.:f.}
\end{itemize}
\begin{itemize}
\item {Proveniência:(Do gr. \textunderscore daimon\textunderscore  + \textunderscore manteia\textunderscore )}
\end{itemize}
Adivinhação por influência dos demónios.
\section{Demonomania}
\begin{itemize}
\item {Grp. gram.:f.}
\end{itemize}
\begin{itemize}
\item {Proveniência:(De \textunderscore demónio\textunderscore  + \textunderscore mania\textunderscore )}
\end{itemize}
Estado de um louco, que se crê possuido do demónio.
\section{Demonomaníaco}
\begin{itemize}
\item {Grp. gram.:m.  e  adj.}
\end{itemize}
Aquelle que tem demonomania.
\section{Demonómano}
\begin{itemize}
\item {Grp. gram.:m.}
\end{itemize}
O mesmo que \textunderscore demonomaníaco\textunderscore  ou \textunderscore demoniomaniaco\textunderscore . Cf. Camillo, \textunderscore Brasileira\textunderscore , 368.
\section{Demonoscopia}
\begin{itemize}
\item {Grp. gram.:f.}
\end{itemize}
\begin{itemize}
\item {Proveniência:(Do gr. \textunderscore daimon\textunderscore  + \textunderscore skopein\textunderscore )}
\end{itemize}
O mesmo que \textunderscore demonomancia\textunderscore .
\section{Demonstrabilidade}
\begin{itemize}
\item {Grp. gram.:f.}
\end{itemize}
\begin{itemize}
\item {Proveniência:(Do lat. \textunderscore demonstrabilis\textunderscore )}
\end{itemize}
Qualidade daquillo que é demonstrável.
\section{Demonstração}
\begin{itemize}
\item {Grp. gram.:f.}
\end{itemize}
\begin{itemize}
\item {Proveniência:(Lat. \textunderscore demonstratio\textunderscore )}
\end{itemize}
Acto de demonstrar.
Prova.
Raciocínio, de que se infere a verdade de uma proposição.
Sinal, manifestação: \textunderscore demonstração de pesar\textunderscore .
Estratégia militar, com que se encobre um plano.
\section{Demonstrador}
\begin{itemize}
\item {Grp. gram.:m.}
\end{itemize}
\begin{itemize}
\item {Proveniência:(Lat. \textunderscore demonstrator\textunderscore )}
\end{itemize}
Aquelle que demonstra.
\section{Demonstrante}
\begin{itemize}
\item {Grp. gram.:adj.}
\end{itemize}
\begin{itemize}
\item {Proveniência:(Lat. \textunderscore demonstrans\textunderscore )}
\end{itemize}
Que demonstra.
\section{Demonstrar}
\begin{itemize}
\item {Grp. gram.:v. t.}
\end{itemize}
\begin{itemize}
\item {Proveniência:(Lat. \textunderscore demonstrare\textunderscore )}
\end{itemize}
Provar por meio de raciocínio concludente.
Provar.
Mostrar; revelar: \textunderscore demonstrar habilidade\textunderscore .
Ensinar praticamente.
\section{Demonstrativamente}
\begin{itemize}
\item {Grp. gram.:adv.}
\end{itemize}
De modo demonstrativo.
\section{Demonstrativo}
\begin{itemize}
\item {Grp. gram.:adj.}
\end{itemize}
\begin{itemize}
\item {Utilização:Gram.}
\end{itemize}
\begin{itemize}
\item {Proveniência:(Lat. \textunderscore demonstrativus\textunderscore )}
\end{itemize}
Que demonstra.
Próprio para demonstrar.
Que louva ou vitupera, tratando-se de eloquência.
Que mostra ou indica ordem ou situação: \textunderscore pronome demonstrativo\textunderscore .
\section{Demonstrável}
\begin{itemize}
\item {Grp. gram.:adj.}
\end{itemize}
\begin{itemize}
\item {Proveniência:(Lat. \textunderscore demonstrabilis\textunderscore )}
\end{itemize}
Que póde sêr demonstrado.
\section{Demonstres}
\begin{itemize}
\item {Grp. gram.:m.}
\end{itemize}
\begin{itemize}
\item {Utilização:Prov.}
\end{itemize}
Demónio; inimigo.
(Colhido em Vianna e V. Real)
\section{Demopsicologia}
\begin{itemize}
\item {Grp. gram.:f.}
\end{itemize}
\begin{itemize}
\item {Proveniência:(Do gr. \textunderscore demos\textunderscore  + \textunderscore psukhe\textunderscore  + \textunderscore logos\textunderscore )}
\end{itemize}
Estudo psíquico de um povo.
\section{Demopsicológico}
\begin{itemize}
\item {Grp. gram.:adj.}
\end{itemize}
Relativo á demopsicologia.
\section{Demopsychologia}
\begin{itemize}
\item {fónica:co}
\end{itemize}
\begin{itemize}
\item {Grp. gram.:f.}
\end{itemize}
\begin{itemize}
\item {Proveniência:(Do gr. \textunderscore demos\textunderscore  + \textunderscore psukhe\textunderscore  + \textunderscore logos\textunderscore )}
\end{itemize}
Estudo psýchico de um povo.
\section{Demopsychológico}
\begin{itemize}
\item {fónica:co}
\end{itemize}
\begin{itemize}
\item {Grp. gram.:adj.}
\end{itemize}
Relativo á demopsychologia.
\section{Demora}
\begin{itemize}
\item {Grp. gram.:f.}
\end{itemize}
Acto de demorar.
Detença.
Dilação; paragem.
\section{Demoradamente}
\begin{itemize}
\item {Grp. gram.:adv.}
\end{itemize}
De modo demorado.
Com demora.
\section{Demorança}
\begin{itemize}
\item {Grp. gram.:f.}
\end{itemize}
\begin{itemize}
\item {Utilização:Ant.}
\end{itemize}
O mesmo que \textunderscore demora\textunderscore .
\section{Demorar}
\begin{itemize}
\item {Grp. gram.:v. t.}
\end{itemize}
\begin{itemize}
\item {Grp. gram.:V. i.}
\end{itemize}
\begin{itemize}
\item {Proveniência:(Lat. \textunderscore demorari\textunderscore )}
\end{itemize}
Deter.
Sustar.
Retardar: \textunderscore demorar uma resposta\textunderscore .
Fazer que espere: \textunderscore demorar alguém\textunderscore .
Fazer parar.
Estar situado.
Habitar: \textunderscore demorar em Pedroiços\textunderscore .
Permanecer.
\section{Demoroso}
\begin{itemize}
\item {Grp. gram.:adj.}
\end{itemize}
O mesmo que \textunderscore moroso\textunderscore .
\section{Demostênico}
\begin{itemize}
\item {Grp. gram.:adj.}
\end{itemize}
\begin{itemize}
\item {Proveniência:(De \textunderscore Demóstenes\textunderscore , n. p.)}
\end{itemize}
Relativo a Demóstenes, ao seu estilo ou á sua eloquência.
\section{Demosthênico}
\begin{itemize}
\item {Grp. gram.:adj.}
\end{itemize}
\begin{itemize}
\item {Proveniência:(De \textunderscore Demósthenes\textunderscore , n. p.)}
\end{itemize}
Relativo a Demósthenes, ao seu estilo ou á sua eloquência.
\section{Demostração}
\begin{itemize}
\item {Grp. gram.:f.}
\end{itemize}
Acto de demostrar.
\section{Demostrador}
\begin{itemize}
\item {Grp. gram.:m.}
\end{itemize}
Aquelle que demostra.
\section{Demostrar}
\begin{itemize}
\item {Grp. gram.:v. t.}
\end{itemize}
\begin{itemize}
\item {Proveniência:(De \textunderscore mostrar\textunderscore )}
\end{itemize}
Mostrar, patentear: \textunderscore demostrar alegria\textunderscore .
\section{Demostrativo}
\begin{itemize}
\item {Grp. gram.:adj.}
\end{itemize}
\begin{itemize}
\item {Utilização:Ant.}
\end{itemize}
O mesmo que \textunderscore demonstrativo\textunderscore .
\section{Demótico}
\begin{itemize}
\item {Grp. gram.:adj.}
\end{itemize}
\begin{itemize}
\item {Proveniência:(Gr. \textunderscore demotikos\textunderscore )}
\end{itemize}
Dizia-se da escritura ou dos caracteres correntes ou vulgares entre os Egýpcios.
\section{Demover}
\begin{itemize}
\item {Grp. gram.:v. t.}
\end{itemize}
\begin{itemize}
\item {Proveniência:(Lat. \textunderscore demovere\textunderscore )}
\end{itemize}
Deslocar: \textunderscore demover um pedregulho\textunderscore .
Desviar, apartar.
Dissuadir: \textunderscore ninguém me demove desta ideia\textunderscore .
\section{Demudadamente}
\begin{itemize}
\item {Grp. gram.:adv.}
\end{itemize}
Com demudamento.
\section{Demudamento}
\begin{itemize}
\item {Grp. gram.:m.}
\end{itemize}
Acto ou effeito de demudar.
\section{Demudar}
\begin{itemize}
\item {Grp. gram.:v. t.}
\end{itemize}
\begin{itemize}
\item {Proveniência:(Do lat. \textunderscore demutare\textunderscore )}
\end{itemize}
Mudar.
Alterar.
Transformar: \textunderscore a doença demudou-o\textunderscore .
\section{Demulcente}
\begin{itemize}
\item {Grp. gram.:m.  e  adj.}
\end{itemize}
\begin{itemize}
\item {Proveniência:(Lat. \textunderscore demulcens\textunderscore )}
\end{itemize}
Diz-se do medicamento que abranda ou adoça.
\section{Demulcir}
\begin{itemize}
\item {Grp. gram.:v. t.}
\end{itemize}
\begin{itemize}
\item {Utilização:Des.}
\end{itemize}
\begin{itemize}
\item {Proveniência:(Lat. \textunderscore demulcere\textunderscore )}
\end{itemize}
Amollecer, abrandar.
\section{Denário}
\begin{itemize}
\item {Grp. gram.:m.}
\end{itemize}
\begin{itemize}
\item {Grp. gram.:Adj.}
\end{itemize}
\begin{itemize}
\item {Proveniência:(Lat. \textunderscore denarius\textunderscore )}
\end{itemize}
Moéda romana.
Antigo pêso de pharmácia.
Que contém déz.
\section{Dende}
\begin{itemize}
\item {Grp. gram.:adj.}
\end{itemize}
\begin{itemize}
\item {Utilização:Ant.}
\end{itemize}
Daí.
\section{Dendê}
\begin{itemize}
\item {Grp. gram.:m.}
\end{itemize}
\begin{itemize}
\item {Utilização:Bras}
\end{itemize}
Espécie de palmeira.
Fruto dessa árvore.
\section{Dendezeiro}
\begin{itemize}
\item {Grp. gram.:m.}
\end{itemize}
\begin{itemize}
\item {Utilização:Bras}
\end{itemize}
O mesmo que \textunderscore dendê\textunderscore , árvore.
\section{Dendo}
\begin{itemize}
\item {Grp. gram.:m.}
\end{itemize}
Árvore angolense, (\textunderscore diospyros dendo\textunderscore ).
\section{Dendrite}
\begin{itemize}
\item {Grp. gram.:f.}
\end{itemize}
\begin{itemize}
\item {Proveniência:(Gr. \textunderscore dendrites\textunderscore )}
\end{itemize}
Pedra, que representa uma árvore.
\section{Dendrítica}
\begin{itemize}
\item {Grp. gram.:adj.  e  f.}
\end{itemize}
\begin{itemize}
\item {Proveniência:(Do gr. \textunderscore dendron\textunderscore )}
\end{itemize}
Diz-se da fórma irregular do mineral, semelhante á disposição dos ramos de uma árvore.
\section{Dendróbata}
\begin{itemize}
\item {Grp. gram.:adj.}
\end{itemize}
\begin{itemize}
\item {Proveniência:(Do gr. \textunderscore dendron\textunderscore  + \textunderscore bates\textunderscore )}
\end{itemize}
Que vive habitualmente nas árvores.
\section{Dendróbio}
\begin{itemize}
\item {Grp. gram.:m.}
\end{itemize}
\begin{itemize}
\item {Proveniência:(Do gr. \textunderscore dendron\textunderscore  + \textunderscore bios\textunderscore )}
\end{itemize}
Gênero de orchídeas parasitas.
\section{Dendrocarpo}
\begin{itemize}
\item {Grp. gram.:m.}
\end{itemize}
\begin{itemize}
\item {Proveniência:(Do gr. \textunderscore dendron\textunderscore  + \textunderscore karpos\textunderscore )}
\end{itemize}
Fruto de árvore.
\section{Dendrocelos}
\begin{itemize}
\item {Grp. gram.:m.}
\end{itemize}
\begin{itemize}
\item {Proveniência:(Do gr. \textunderscore dendron\textunderscore  + \textunderscore koilia\textunderscore )}
\end{itemize}
Vermes intestinaes.
\section{Dendroclasta}
\begin{itemize}
\item {Grp. gram.:m.  e  adj.}
\end{itemize}
\begin{itemize}
\item {Utilização:Neol.}
\end{itemize}
\begin{itemize}
\item {Proveniência:(Do gr. \textunderscore dendron\textunderscore  + \textunderscore klaein\textunderscore )}
\end{itemize}
Aquelle que não respeita as árvores. Cf. B. Pato, \textunderscore Livro do Monte\textunderscore , 121.
\section{Dendroclastia}
\begin{itemize}
\item {Grp. gram.:f.}
\end{itemize}
Qualidade de \textunderscore dendroclasta\textunderscore .
\section{Dendróforo}
\begin{itemize}
\item {Grp. gram.:m.}
\end{itemize}
\begin{itemize}
\item {Proveniência:(Gr. \textunderscore dendrophoros\textunderscore )}
\end{itemize}
Aquele que, em certas festas gregas, levava nas mãos arbustos ou ramos.
\section{Dendrografia}
\begin{itemize}
\item {Grp. gram.:f.}
\end{itemize}
\begin{itemize}
\item {Proveniência:(Do gr. \textunderscore dendron\textunderscore  + \textunderscore graphein\textunderscore )}
\end{itemize}
Tratado das árvores.
\section{Dendrographia}
\begin{itemize}
\item {Grp. gram.:f.}
\end{itemize}
\begin{itemize}
\item {Proveniência:(Do gr. \textunderscore dendron\textunderscore  + \textunderscore graphein\textunderscore )}
\end{itemize}
Tratado das árvores.
\section{Dendroide}
\begin{itemize}
\item {Grp. gram.:adj.}
\end{itemize}
\begin{itemize}
\item {Proveniência:(Do gr. \textunderscore dendron\textunderscore  + \textunderscore eidos\textunderscore )}
\end{itemize}
Diz-se das plantas cryptogâmicas, que na sua organização apresentam ramificações semelhantes ás das árvores.
\section{Dendroídeo}
\begin{itemize}
\item {Grp. gram.:adj.}
\end{itemize}
\begin{itemize}
\item {Utilização:Bot.}
\end{itemize}
\begin{itemize}
\item {Proveniência:(Do gr. \textunderscore dendron\textunderscore  + \textunderscore eidos\textunderscore )}
\end{itemize}
Diz-se das plantas cryptogâmicas, que na sua organização apresentam ramificações semelhantes ás das árvores.
\section{Dendrolatria}
\begin{itemize}
\item {Grp. gram.:f.}
\end{itemize}
\begin{itemize}
\item {Proveniência:(Do gr. \textunderscore dendron\textunderscore  + \textunderscore latreia\textunderscore )}
\end{itemize}
Culto das árvores.
\section{Dendrólitho}
\begin{itemize}
\item {Grp. gram.:m.}
\end{itemize}
\begin{itemize}
\item {Proveniência:(Do gr. \textunderscore dendron\textunderscore  + \textunderscore lithos\textunderscore )}
\end{itemize}
Árvore petrificada, fóssil.
\section{Dendrólito}
\begin{itemize}
\item {Grp. gram.:m.}
\end{itemize}
\begin{itemize}
\item {Proveniência:(Do gr. \textunderscore dendron\textunderscore  + \textunderscore lithos\textunderscore )}
\end{itemize}
Árvore petrificada, fóssil.
\section{Dendrologia}
\begin{itemize}
\item {Grp. gram.:f.}
\end{itemize}
O mesmo que \textunderscore dendrographia\textunderscore .
\section{Dendrómetro}
\begin{itemize}
\item {Grp. gram.:m.}
\end{itemize}
\begin{itemize}
\item {Proveniência:(Do gr. \textunderscore dendron\textunderscore  + \textunderscore metron\textunderscore )}
\end{itemize}
Instrumento, para determinar a altura das árvores, conhecida a distância ellas e o ponto em que está o observador.
\section{Dendróphoro}
\begin{itemize}
\item {Grp. gram.:m.}
\end{itemize}
\begin{itemize}
\item {Proveniência:(Gr. \textunderscore dendrophoros\textunderscore )}
\end{itemize}
Aquelle que, em certas festas gregas, levava nas mãos arbustos ou ramos.
\section{Dendrotomia}
\begin{itemize}
\item {Grp. gram.:f.}
\end{itemize}
\begin{itemize}
\item {Proveniência:(Do gr. \textunderscore dendron\textunderscore  + \textunderscore tome\textunderscore )}
\end{itemize}
Córte de árvores.
\section{Denegação}
\begin{itemize}
\item {Grp. gram.:f.}
\end{itemize}
\begin{itemize}
\item {Proveniência:(Lat. \textunderscore denegatio\textunderscore )}
\end{itemize}
Acto de denegar.
\section{Denegar}
\begin{itemize}
\item {Grp. gram.:v. t.}
\end{itemize}
\begin{itemize}
\item {Proveniência:(Lat. \textunderscore denegare\textunderscore )}
\end{itemize}
Negar.
Recusar.
Indeferir.
\section{Denegrecer}
\begin{itemize}
\item {Grp. gram.:v. t.}
\end{itemize}
(V.denegrir)
\section{Denegrir}
\begin{itemize}
\item {Grp. gram.:v. t.}
\end{itemize}
Tornar negro, escuro.
Manchar, macular: \textunderscore denegrir a reputação de alguém\textunderscore .
(Cf. lat. \textunderscore denigrare\textunderscore )
\section{Deneguil}
\begin{itemize}
\item {Grp. gram.:adj.}
\end{itemize}
\begin{itemize}
\item {Utilização:Ant.}
\end{itemize}
O mesmo que \textunderscore dengue\textunderscore ^1? Cf. G. Vicente, I, 252.
\section{Denesde}
\begin{itemize}
\item {Grp. gram.:prep.}
\end{itemize}
\begin{itemize}
\item {Utilização:Ant.}
\end{itemize}
O mesmo que \textunderscore desde\textunderscore .
\section{Dengosa}
\begin{itemize}
\item {Grp. gram.:f.}
\end{itemize}
\begin{itemize}
\item {Utilização:Bras. do N}
\end{itemize}
Aguardente de cana; cachaça.
\section{Dengoso}
\begin{itemize}
\item {Grp. gram.:adj.}
\end{itemize}
\begin{itemize}
\item {Utilização:Bras. do N}
\end{itemize}
\begin{itemize}
\item {Proveniência:(De \textunderscore dengue\textunderscore ^1)}
\end{itemize}
Que tem hábitos de pessôa dengue.
Pretensioso, affectado.
Choramingas; birrento, (falando-se de crianças).
\section{Dengue}
\begin{itemize}
\item {Grp. gram.:adj.}
\end{itemize}
\begin{itemize}
\item {Grp. gram.:M.}
\end{itemize}
\begin{itemize}
\item {Utilização:Bras. do N}
\end{itemize}
Presumido.
Affectado.
Vaidoso.
Mulherengo.
Choradeira, birra de crianças.
(Cast. \textunderscore dengue\textunderscore )
\section{Dengue}
\begin{itemize}
\item {Grp. gram.:adj.}
\end{itemize}
\begin{itemize}
\item {Proveniência:(T. as.)}
\end{itemize}
Diz-se de uma febre epidêmica, acompanhada de erupção e dôres articulares, análoga á grippe.
\section{Dengue-de-mané}
\begin{itemize}
\item {Grp. gram.:m.}
\end{itemize}
\begin{itemize}
\item {Utilização:Bras. do N}
\end{itemize}
Nome de uma flôr silvestre.
\section{Denguedengue}
\begin{itemize}
\item {Grp. gram.:m.}
\end{itemize}
\begin{itemize}
\item {Utilização:Prov.}
\end{itemize}
\begin{itemize}
\item {Utilização:alg.}
\end{itemize}
O mesmo que \textunderscore machado\textunderscore .
\section{Dengueiro}
\begin{itemize}
\item {Grp. gram.:adj.}
\end{itemize}
Que tem denguice. Cf. Eça de Queiroz, \textunderscore P. Basilio\textunderscore , 21.
\section{Denguice}
\begin{itemize}
\item {Grp. gram.:f.}
\end{itemize}
Qualidade de quem é dengue^1.
\section{Denguim}
\begin{itemize}
\item {Grp. gram.:m.}
\end{itemize}
Embarcação asiática.
\section{Denigração}
\begin{itemize}
\item {Grp. gram.:f.}
\end{itemize}
\begin{itemize}
\item {Proveniência:(Lat. \textunderscore denigratio\textunderscore )}
\end{itemize}
Acto de denegrir.
\section{Denigrativo}
\begin{itemize}
\item {Grp. gram.:adj.}
\end{itemize}
\begin{itemize}
\item {Proveniência:(Do lat. \textunderscore denigrare\textunderscore )}
\end{itemize}
Que denigre.
\section{Denigrir}
\begin{itemize}
\item {Grp. gram.:v. t.}
\end{itemize}
(V.denegrir)
\section{Denodadamente}
\begin{itemize}
\item {Grp. gram.:adv.}
\end{itemize}
De modo denodado.
Com denôdo.
\section{Denodado}
\begin{itemize}
\item {Grp. gram.:adj.}
\end{itemize}
Que tem denôdo.
Ousado; impetuoso.
\section{Denodar}
\begin{itemize}
\item {Grp. gram.:v. t.}
\end{itemize}
Cortar o nó de.
Desatar, desembaraçar.
(Cp. lat. \textunderscore enodare\textunderscore )
\section{Denôdo}
\begin{itemize}
\item {Grp. gram.:m.}
\end{itemize}
\begin{itemize}
\item {Proveniência:(De \textunderscore denodar\textunderscore )}
\end{itemize}
Intrepidez; ousadia.
Desembaraço.
\section{Denominação}
\begin{itemize}
\item {Grp. gram.:f.}
\end{itemize}
\begin{itemize}
\item {Proveniência:(Lat. \textunderscore denominatio\textunderscore )}
\end{itemize}
Acto de denominar.
\section{Denominador}
\begin{itemize}
\item {Grp. gram.:m.  e  adj.}
\end{itemize}
\begin{itemize}
\item {Utilização:Arith.}
\end{itemize}
\begin{itemize}
\item {Proveniência:(Lat. \textunderscore denominator\textunderscore )}
\end{itemize}
Aquelle ou aquillo que denomina.
Número, que indica em quantas partes está dividida uma unidade.
\section{Denominar}
\begin{itemize}
\item {Grp. gram.:v. t.}
\end{itemize}
\begin{itemize}
\item {Proveniência:(Lat. \textunderscore denominare\textunderscore )}
\end{itemize}
Indicar o nome de.
Nomear.
\section{Denominativo}
\begin{itemize}
\item {Grp. gram.:adj.}
\end{itemize}
\begin{itemize}
\item {Proveniência:(Lat. \textunderscore denominativus\textunderscore )}
\end{itemize}
Próprio para denominar.
\section{Denotação}
\begin{itemize}
\item {Grp. gram.:f.}
\end{itemize}
\begin{itemize}
\item {Proveniência:(Lat. \textunderscore denotatio\textunderscore )}
\end{itemize}
Acto de denotar.
\section{Denotador}
\begin{itemize}
\item {Grp. gram.:adj.}
\end{itemize}
\begin{itemize}
\item {Grp. gram.:M.}
\end{itemize}
\begin{itemize}
\item {Proveniência:(Lat. \textunderscore denotator\textunderscore )}
\end{itemize}
Que denota.
Aquelle que denota.
\section{Denotar}
\begin{itemize}
\item {Grp. gram.:v. t.}
\end{itemize}
\begin{itemize}
\item {Proveniência:(Lat. \textunderscore denotare\textunderscore )}
\end{itemize}
Mostrar, significar, por meio de certos sinaes: \textunderscore aquelle sujeito denota que tem mêdo\textunderscore .
\section{Densamente}
\begin{itemize}
\item {Grp. gram.:adv.}
\end{itemize}
De modo denso.
\section{Densar}
\begin{itemize}
\item {Grp. gram.:v. t.}
\end{itemize}
\begin{itemize}
\item {Utilização:Des.}
\end{itemize}
\begin{itemize}
\item {Proveniência:(Lat. \textunderscore densare\textunderscore )}
\end{itemize}
Tornar denso.
\section{Densidade}
\begin{itemize}
\item {Grp. gram.:f.}
\end{itemize}
\begin{itemize}
\item {Utilização:Phýs.}
\end{itemize}
\begin{itemize}
\item {Proveniência:(Lat. \textunderscore densitas\textunderscore )}
\end{itemize}
Qualidade daquillo que é denso: \textunderscore a densidade do bosque\textunderscore .
Relação entre a massa de um corpo e o seu volume: \textunderscore a densidade da água\textunderscore .
\section{Densidão}
\begin{itemize}
\item {Grp. gram.:f.}
\end{itemize}
(V.densidade)
\section{Densifoliado}
\begin{itemize}
\item {Grp. gram.:adj.}
\end{itemize}
\begin{itemize}
\item {Utilização:Bot.}
\end{itemize}
\begin{itemize}
\item {Proveniência:(Do lat. \textunderscore densus\textunderscore  + \textunderscore folium\textunderscore )}
\end{itemize}
Que tem muitas fôlhas juntas.
\section{Densímetro}
\begin{itemize}
\item {Grp. gram.:m.}
\end{itemize}
\begin{itemize}
\item {Proveniência:(T. hybr., do lat. \textunderscore densus\textunderscore  + gr. \textunderscore metron\textunderscore )}
\end{itemize}
Instrumento, para avaliar a densidade dos líquidos. Cf. \textunderscore Techn. Rur.\textunderscore , 38 e 58.
\section{Denso}
\begin{itemize}
\item {Grp. gram.:adj.}
\end{itemize}
\begin{itemize}
\item {Utilização:Fig.}
\end{itemize}
\begin{itemize}
\item {Proveniência:(Lat. \textunderscore densus\textunderscore )}
\end{itemize}
Diz-se de um corpo que tem mais pêso e massa, que outros do mesmo volume.
Cujas moléculas estão muito apertadas umas contra outras.
Espêsso; compacto, cerrado.
Escuro: \textunderscore noite densa\textunderscore .
\section{Dentada}
\begin{itemize}
\item {Grp. gram.:f.}
\end{itemize}
\begin{itemize}
\item {Utilização:Fig.}
\end{itemize}
Ferimento com os dentes; mordedura.
Vestigio de mordedura.
Dito picante, mordaz.
\section{Dentado}
\begin{itemize}
\item {Grp. gram.:adj.}
\end{itemize}
Ferido ou cortado com os dentes: \textunderscore uma pêra dentada\textunderscore .
Guarnecido de dentes: \textunderscore uma roda dentada\textunderscore .
\section{Dentadura}
\begin{itemize}
\item {Grp. gram.:f.}
\end{itemize}
\begin{itemize}
\item {Proveniência:(De \textunderscore dentar\textunderscore )}
\end{itemize}
Conjunto dos dentes, nas pessôas e nos animaes.
Série de dentes artificiaes.
Totalidade dos dentes que guarnecem certas rodas.
\section{Dentaes}
\begin{itemize}
\item {Grp. gram.:m. pl.}
\end{itemize}
\begin{itemize}
\item {Utilização:T. da Bairrada}
\end{itemize}
\begin{itemize}
\item {Proveniência:(Lat. \textunderscore dentalia\textunderscore )}
\end{itemize}
As aivecas do arado.
\section{Dental}
\begin{itemize}
\item {Grp. gram.:adj.}
\end{itemize}
\begin{itemize}
\item {Utilização:Gram.}
\end{itemize}
\begin{itemize}
\item {Grp. gram.:F.}
\end{itemize}
\begin{itemize}
\item {Proveniência:(Lat. \textunderscore dentalis\textunderscore )}
\end{itemize}
Relativo aos dentes: \textunderscore hygiene dental\textunderscore .
Diz-se das letras, que se não podem pronunciar, sem que a língua toque nos dentes, \textunderscore ou\textunderscore , pelo menos, nas gengivas.
Letra dental.
Dente do arado.
\section{Dentálio}
\begin{itemize}
\item {Grp. gram.:m.}
\end{itemize}
\begin{itemize}
\item {Proveniência:(Do rad. de \textunderscore dente\textunderscore )}
\end{itemize}
Mollusco marinho.
\section{Dentálitho}
\begin{itemize}
\item {Grp. gram.:m.}
\end{itemize}
\begin{itemize}
\item {Proveniência:(De \textunderscore dentálio\textunderscore  + gr. \textunderscore lithos\textunderscore )}
\end{itemize}
Dentálio fóssil.
\section{Dentálito}
\begin{itemize}
\item {Grp. gram.:m.}
\end{itemize}
\begin{itemize}
\item {Proveniência:(De \textunderscore dentálio\textunderscore  + gr. \textunderscore lithos\textunderscore )}
\end{itemize}
Dentálio fóssil.
\section{Dentão}
\begin{itemize}
\item {Grp. gram.:m.}
\end{itemize}
\begin{itemize}
\item {Proveniência:(De \textunderscore dente\textunderscore )}
\end{itemize}
Peixe de grandes dentes.
\section{Dentar}
\begin{itemize}
\item {Grp. gram.:v. t.}
\end{itemize}
\begin{itemize}
\item {Grp. gram.:V. i.}
\end{itemize}
\begin{itemize}
\item {Proveniência:(De \textunderscore dente\textunderscore )}
\end{itemize}
Dar dentada em; morder.
Dentear.
Começar a têr dentes.
\section{Dentária}
\begin{itemize}
\item {Grp. gram.:f.}
\end{itemize}
\begin{itemize}
\item {Proveniência:(De \textunderscore dentário\textunderscore )}
\end{itemize}
Planta crucífera, de raízes denteadas.
\section{Dentário}
\begin{itemize}
\item {Grp. gram.:adj.}
\end{itemize}
\begin{itemize}
\item {Proveniência:(Lat. \textunderscore dentarius\textunderscore )}
\end{itemize}
Relativo aos dentes: \textunderscore clínica dentária\textunderscore .
\section{Dente}
\begin{itemize}
\item {Grp. gram.:m.}
\end{itemize}
\begin{itemize}
\item {Utilização:Agr.}
\end{itemize}
\begin{itemize}
\item {Grp. gram.:Loc.}
\end{itemize}
\begin{itemize}
\item {Utilização:fam.}
\end{itemize}
\begin{itemize}
\item {Grp. gram.:Loc. adv.}
\end{itemize}
\begin{itemize}
\item {Utilização:Fig.}
\end{itemize}
\begin{itemize}
\item {Grp. gram.:Loc.}
\end{itemize}
\begin{itemize}
\item {Utilização:fam.}
\end{itemize}
\begin{itemize}
\item {Grp. gram.:Pl.}
\end{itemize}
\begin{itemize}
\item {Grp. gram.:Loc.}
\end{itemize}
\begin{itemize}
\item {Utilização:fam.}
\end{itemize}
\begin{itemize}
\item {Utilização:Fig.}
\end{itemize}
\begin{itemize}
\item {Utilização:Fig.}
\end{itemize}
\begin{itemize}
\item {Proveniência:(Lat. \textunderscore dens\textunderscore , \textunderscore dentis\textunderscore )}
\end{itemize}
Cada um dos pequenos ossos, que guarnecem as maxillas do homem e de outros animaes, e servem especialmente para a trituração dos alimentos.
Peça do arado, que assenta no leito do rêgo; o mesmo que \textunderscore vessadoiro\textunderscore  ou \textunderscore cepo\textunderscore .
Cada uma das longas pontas ou defesas, que guarnecem a maxilla superior do elephante e de alguns outros animaes.
Cada uma das pontas ou saliências, que guarnecem ou formam a engrenagem de certos objectos ou instrumentos: \textunderscore os dentes do forcado\textunderscore .
Cada uma das pequenas saliências na borda de certos órgãos vegetaes.
Objecto, cuja configuração dá ideia de um dente: \textunderscore um dente de alho\textunderscore .
\textunderscore Dente de leão\textunderscore , planta, da fam. das compostas.
\textunderscore Dente de coêlho\textunderscore , difficuldade, negócio intrincado.
\textunderscore Ôlho por ôlho, dente por dente\textunderscore , com desforra igual á offensa; applicando-se a pena de Talião.
\textunderscore Apanhar a dente\textunderscore , aprender de cór, adquirir imperfeito conhecimento de.
\textunderscore Não meter dente\textunderscore , não comer, não provar.
Não comprehender nada de uma coisa.
\textunderscore Dar ao dente\textunderscore , mastigar, comer.
\textunderscore Dente de lobo\textunderscore , brunidor ou instrumento, de que usam os doiradores e outros artistas, para brunir suas obras.
\textunderscore Dente de gato\textunderscore , planta leguminosa, o mesmo que \textunderscore resta-boi\textunderscore .
\textunderscore Aguçar os dentes\textunderscore , dispor-se para comer; dispor-se para gozar o que se desejava.
\textunderscore Com unhas e dentes\textunderscore , com toda a fôrça, com todos os recursos de que se dispõe.
\textunderscore Custar os dentes da boca\textunderscore , custar muito, sêr caro; adquirir-se á custa de muitos sacrificios.
\textunderscore Falar entre dentes\textunderscore , ou \textunderscore por entre dentes\textunderscore , rosnar, resmungar.
\textunderscore Mostrar os dentes\textunderscore , rir-se; ameaçar.
\textunderscore Trazer entre os dentes\textunderscore , dispor-se para fazer mal, planear desforra ou offensa, contra alguém.
\section{Denteação}
\begin{itemize}
\item {Grp. gram.:f.}
\end{itemize}
Acto de \textunderscore dentear\textunderscore .
\section{Dentear}
\begin{itemize}
\item {Grp. gram.:v. t.}
\end{itemize}
Formar dentes em.
Recortar; chanfrar.
\section{Dentebrum}
\begin{itemize}
\item {Grp. gram.:m.}
\end{itemize}
Espécie de féto, (\textunderscore polypodium filix mas\textunderscore , Lin.).
\section{Dente-de-cão}
\begin{itemize}
\item {Grp. gram.:m.}
\end{itemize}
Designação vulgar do \textunderscore gramão\textunderscore .
Nome que, na Madeira, se dá á geada, que apparece nas serras, durante o Inverno. Cf. \textunderscore Bol. da Sociedade de Geogr.\textunderscore , XXX, 572.
\section{Dente-de-velha}
\begin{itemize}
\item {Grp. gram.:m.}
\end{itemize}
\begin{itemize}
\item {Utilização:Bras}
\end{itemize}
O mesmo que \textunderscore gangão\textunderscore ^1.
\section{Denteira}
\begin{itemize}
\item {Grp. gram.:f.}
\end{itemize}
\begin{itemize}
\item {Utilização:Pop.}
\end{itemize}
\begin{itemize}
\item {Utilização:Des.}
\end{itemize}
Embotamento dos dentes.
O mesmo que \textunderscore dentuça\textunderscore . Cf. G. Vicente, I, 347.
(Cast. \textunderscore dentera\textunderscore )
\section{Dentel}
\begin{itemize}
\item {Grp. gram.:m.}
\end{itemize}
\begin{itemize}
\item {Utilização:Carp.}
\end{itemize}
Entalhe, para regular a altura das prateleiras.
\section{Dentelar}
\begin{itemize}
\item {Grp. gram.:v. t.}
\end{itemize}
(V.dentear)
\section{Dentelária}
\begin{itemize}
\item {Grp. gram.:f.}
\end{itemize}
(V.dentilária)
\section{Dentelha}
\begin{itemize}
\item {fónica:tê}
\end{itemize}
\begin{itemize}
\item {Grp. gram.:f.}
\end{itemize}
\begin{itemize}
\item {Proveniência:(De \textunderscore dente\textunderscore )}
\end{itemize}
Peixe da costa do Algarve, provavelmente o mesmo que \textunderscore dentilha\textunderscore .
\section{Dentelo}
\begin{itemize}
\item {fónica:tê}
\end{itemize}
\begin{itemize}
\item {Grp. gram.:m.}
\end{itemize}
\begin{itemize}
\item {Proveniência:(De \textunderscore dente\textunderscore )}
\end{itemize}
O mesmo que \textunderscore dentículo\textunderscore , em architectura.
\section{Dentição}
\begin{itemize}
\item {Grp. gram.:f.}
\end{itemize}
\begin{itemize}
\item {Proveniência:(Lat. \textunderscore dentitio\textunderscore )}
\end{itemize}
Erupção natural dos dentes.
\section{Deflegmação}
\begin{itemize}
\item {Grp. gram.:f.}
\end{itemize}
Acto de deflegmar.
\section{Deflegmar}
\begin{itemize}
\item {Grp. gram.:v. t.}
\end{itemize}
\begin{itemize}
\item {Proveniência:(Do gr. \textunderscore phlegma\textunderscore )}
\end{itemize}
Destilar, para separar de uma substância a parte aquosa.
\section{Denticida}
\begin{itemize}
\item {Grp. gram.:adj.}
\end{itemize}
\begin{itemize}
\item {Utilização:Bot.}
\end{itemize}
\begin{itemize}
\item {Proveniência:(Do lat. \textunderscore dens\textunderscore  + \textunderscore caedere\textunderscore )}
\end{itemize}
Diz-se da dehiscência, que se produz pelo afastamento dos dentes no cimo das carpellas.
\section{Denticórneo}
\begin{itemize}
\item {Grp. gram.:adj.}
\end{itemize}
\begin{itemize}
\item {Utilização:Zool.}
\end{itemize}
\begin{itemize}
\item {Proveniência:(De \textunderscore dente\textunderscore  + \textunderscore córneo\textunderscore )}
\end{itemize}
Que tem antennas denteadas.
\section{Denticulado}
\begin{itemize}
\item {Grp. gram.:adj.}
\end{itemize}
\begin{itemize}
\item {Proveniência:(Lat. \textunderscore denticulatus\textunderscore )}
\end{itemize}
Guarnecido de dentículos.
Recortado.
\textunderscore Part.\textunderscore  de \textunderscore denticular\textunderscore ?
\section{Denticular}
\begin{itemize}
\item {Grp. gram.:adj.}
\end{itemize}
Que tem dentículos ou entalhos em fórma de dentes.
\section{Denticular}
\begin{itemize}
\item {Grp. gram.:v. t.}
\end{itemize}
\begin{itemize}
\item {Proveniência:(De \textunderscore dentículo\textunderscore )}
\end{itemize}
Recortar, formando dentes.
\section{Dentículo}
\begin{itemize}
\item {Grp. gram.:m.}
\end{itemize}
\begin{itemize}
\item {Proveniência:(Lat. \textunderscore denticulus\textunderscore )}
\end{itemize}
Pequeno dente.
Entalho em fórma de dente, nas obras de architectura.
Pequeno recorte nas bordas das fôlhas de certos vegetaes.
\section{Dentificação}
\begin{itemize}
\item {Grp. gram.:f.}
\end{itemize}
\begin{itemize}
\item {Proveniência:(Do lat. \textunderscore dens\textunderscore  + \textunderscore facere\textunderscore )}
\end{itemize}
Formação dos dentes ou da sua substância.
\section{Dentiforme}
\begin{itemize}
\item {Grp. gram.:adj.}
\end{itemize}
\begin{itemize}
\item {Proveniência:(Do lat. \textunderscore dens\textunderscore  + \textunderscore forma\textunderscore )}
\end{itemize}
Que tem fórma de dentes.
\section{Dentifrício}
\begin{itemize}
\item {Grp. gram.:m.}
\end{itemize}
\begin{itemize}
\item {Proveniência:(Lat. \textunderscore dentifricium\textunderscore )}
\end{itemize}
Aquillo que serve para limpar os dentes.
\section{Dentífrico}
\begin{itemize}
\item {Grp. gram.:adj.}
\end{itemize}
Que serve para limpar os dentes.
(Cp. \textunderscore dentifrício\textunderscore )
\section{Dentígero}
\begin{itemize}
\item {Grp. gram.:adj.}
\end{itemize}
\begin{itemize}
\item {Proveniência:(Do lat. \textunderscore dens\textunderscore  + \textunderscore gerere\textunderscore )}
\end{itemize}
Que tem dentes.
\section{Dentilabial}
\begin{itemize}
\item {Grp. gram.:adj.}
\end{itemize}
\begin{itemize}
\item {Grp. gram.:adj.}
\end{itemize}
\begin{itemize}
\item {Utilização:Gram.}
\end{itemize}
\begin{itemize}
\item {Grp. gram.:F.}
\end{itemize}
\begin{itemize}
\item {Proveniência:(De \textunderscore dente\textunderscore  + \textunderscore labial\textunderscore )}
\end{itemize}
O mesmo ou melhor que \textunderscore dentolabial\textunderscore .
Diz-se das letras consoantes, que se formam, applicando-se o lábio inferior aos incisivos superiores.
Consoante dentilabial.
\section{Dentilária}
\begin{itemize}
\item {Grp. gram.:f.}
\end{itemize}
Planta plumbagínea, (\textunderscore plumbago europaea\textunderscore , Lin.).
\section{Dentilha}
\begin{itemize}
\item {Grp. gram.:f.}
\end{itemize}
Peixe da Póvoa de Varzim, (\textunderscore labrus mixtus\textunderscore , Lin.).
\section{Dentilhão}
\begin{itemize}
\item {Grp. gram.:m.}
\end{itemize}
\begin{itemize}
\item {Proveniência:(Do rad. de \textunderscore dente\textunderscore )}
\end{itemize}
Dente muito grande.
Pedra, que resai lateralmente de uma parede, mostrando que a construcção prosegue.
\section{Dentina}
\begin{itemize}
\item {Grp. gram.:f.}
\end{itemize}
\begin{itemize}
\item {Proveniência:(De \textunderscore dente\textunderscore )}
\end{itemize}
Substância própria dos dentes.
\section{Dentirostro}
\begin{itemize}
\item {fónica:rós}
\end{itemize}
\begin{itemize}
\item {Grp. gram.:adj.}
\end{itemize}
\begin{itemize}
\item {Utilização:Zool.}
\end{itemize}
\begin{itemize}
\item {Proveniência:(Do lat. \textunderscore dens\textunderscore  + \textunderscore rostrum\textunderscore )}
\end{itemize}
Que tem o bico denteado.
\section{Dentirrostro}
\begin{itemize}
\item {Grp. gram.:adj.}
\end{itemize}
\begin{itemize}
\item {Utilização:Zool.}
\end{itemize}
\begin{itemize}
\item {Proveniência:(Do lat. \textunderscore dens\textunderscore  + \textunderscore rostrum\textunderscore )}
\end{itemize}
Que tem o bico denteado.
\section{Dentista}
\begin{itemize}
\item {Grp. gram.:m.}
\end{itemize}
\begin{itemize}
\item {Utilização:Fam.}
\end{itemize}
\begin{itemize}
\item {Proveniência:(De \textunderscore dente\textunderscore )}
\end{itemize}
Aquelle que trata de moléstias dentárias.
Aquelle que tem por officio tirar dentes.
Charlatão.
\section{Dentola}
\begin{itemize}
\item {Grp. gram.:f.}
\end{itemize}
\begin{itemize}
\item {Utilização:Fam.}
\end{itemize}
\begin{itemize}
\item {Proveniência:(De \textunderscore dente\textunderscore )}
\end{itemize}
Dente grande.
Dentuça.
\section{Dentolabial}
\begin{itemize}
\item {Grp. gram.:adj.}
\end{itemize}
\begin{itemize}
\item {Utilização:Gram.}
\end{itemize}
\begin{itemize}
\item {Grp. gram.:F.}
\end{itemize}
\begin{itemize}
\item {Proveniência:(De \textunderscore dente\textunderscore  + \textunderscore labial\textunderscore )}
\end{itemize}
Diz-se das letras consoantes, que se formam, applicando-se o lábio inferior aos incisivos superiores.
Consoante dentolabial.
\section{Dentolas}
\begin{itemize}
\item {Grp. gram.:m.}
\end{itemize}
\begin{itemize}
\item {Utilização:Pop.}
\end{itemize}
\begin{itemize}
\item {Proveniência:(De \textunderscore dente\textunderscore )}
\end{itemize}
Pessôa, que tem dentes grandes e feios.
\section{Dentona}
\begin{itemize}
\item {Grp. gram.:f.}
\end{itemize}
\begin{itemize}
\item {Utilização:Bras}
\end{itemize}
\begin{itemize}
\item {Utilização:fam.}
\end{itemize}
Mulher, que tem os dentes muito saídos.
\section{Dentoneira}
\begin{itemize}
\item {Grp. gram.:f.}
\end{itemize}
\begin{itemize}
\item {Proveniência:(De \textunderscore dentão\textunderscore )}
\end{itemize}
Embarcação de pesca, tripulada geralmente por déz homens e usada antigamente nas costas do Algarve.
\section{Dentremes}
\begin{itemize}
\item {Grp. gram.:m.}
\end{itemize}
\begin{itemize}
\item {Utilização:Gír.}
\end{itemize}
Bolso interior de casaco ou collete.
\section{Dentro}
\begin{itemize}
\item {Grp. gram.:adv.}
\end{itemize}
\begin{itemize}
\item {Grp. gram.:Loc. prep.}
\end{itemize}
\begin{itemize}
\item {Proveniência:(Do lat. \textunderscore de\textunderscore  + \textunderscore intro\textunderscore )}
\end{itemize}
Do lado interior.
Interiormente.
\textunderscore Dentro de\textunderscore  ou \textunderscore dentro em\textunderscore , no interior de; no espaço de; no íntimo de.
\section{Dentuça}
\begin{itemize}
\item {Grp. gram.:f.}
\end{itemize}
\begin{itemize}
\item {Utilização:Fam.}
\end{itemize}
\begin{itemize}
\item {Grp. gram.:M.  e  f.}
\end{itemize}
\begin{itemize}
\item {Proveniência:(De \textunderscore dente\textunderscore )}
\end{itemize}
Dentes grandes e resaídos.
Dentadura.
Pessôa, que tem dentes grandes e mal feitos.
\section{Dentuço}
\begin{itemize}
\item {Grp. gram.:m.}
\end{itemize}
\begin{itemize}
\item {Utilização:Bras}
\end{itemize}
Aquelle que tem os dentes grandes.
\section{Dentudo}
\begin{itemize}
\item {Grp. gram.:adj.}
\end{itemize}
\begin{itemize}
\item {Grp. gram.:M.}
\end{itemize}
\begin{itemize}
\item {Proveniência:(De \textunderscore dente\textunderscore )}
\end{itemize}
Que tem dentuça.
Grande peixe, do gênero dos esqualos.
\section{Denuar}
\begin{itemize}
\item {Grp. gram.:v. t.}
\end{itemize}
\begin{itemize}
\item {Utilização:Ant.}
\end{itemize}
O mesmo que \textunderscore denudar\textunderscore .
\section{Denudação}
\begin{itemize}
\item {Grp. gram.:f.}
\end{itemize}
\begin{itemize}
\item {Proveniência:(Lat. \textunderscore denudatio\textunderscore )}
\end{itemize}
Acto ou effeito de denudar.
\section{Denudar}
\begin{itemize}
\item {Grp. gram.:v. t.}
\end{itemize}
\begin{itemize}
\item {Proveniência:(Lat. \textunderscore denudare\textunderscore )}
\end{itemize}
Tornar nu.
Despir.
Privar de vestidos, ornatos ou envoltórios naturaes.
Descobrir.
\section{Denúncia}
\begin{itemize}
\item {Grp. gram.:f.}
\end{itemize}
Acto de denunciar.
\section{Denunciação}
\begin{itemize}
\item {Grp. gram.:f.}
\end{itemize}
\begin{itemize}
\item {Proveniência:(Lat. \textunderscore denunciatio\textunderscore )}
\end{itemize}
(V.denúncia)
\section{Denunciador}
\begin{itemize}
\item {Grp. gram.:adj.}
\end{itemize}
\begin{itemize}
\item {Grp. gram.:M.}
\end{itemize}
Que denuncía.
Que mostra, que revela.
Aquelle que denuncía.
\section{Denunciante}
\begin{itemize}
\item {Grp. gram.:m. ,  f.  e  adj.}
\end{itemize}
\begin{itemize}
\item {Proveniência:(Lat. \textunderscore denuncians\textunderscore )}
\end{itemize}
Pessôa que denuncía.
\section{Denunciar}
\begin{itemize}
\item {Grp. gram.:v. t.}
\end{itemize}
\begin{itemize}
\item {Proveniência:(Lat. \textunderscore denunciare\textunderscore )}
\end{itemize}
Accusar secretamente; delatar: \textunderscore denunciar conspiradores\textunderscore .
Tornar público; annunciar.
Dar a conhecer.
Mostrar: \textunderscore denunciar interesse pelas letras\textunderscore .
Participar o termo de, (falando-se de tratados internacionaes).
\section{Denunciativo}
\begin{itemize}
\item {Grp. gram.:adj.}
\end{itemize}
\begin{itemize}
\item {Proveniência:(Lat. \textunderscore denunciativus\textunderscore )}
\end{itemize}
Que denuncía.
\section{Denunciatório}
\begin{itemize}
\item {Grp. gram.:adj.}
\end{itemize}
\begin{itemize}
\item {Proveniência:(De \textunderscore denunciar\textunderscore )}
\end{itemize}
Em que há denúncia: \textunderscore declarações denunciatórias\textunderscore .
\section{Denunciável}
\begin{itemize}
\item {Grp. gram.:adj.}
\end{itemize}
Que se póde denunciar.
\section{Deodorar}
\begin{itemize}
\item {Grp. gram.:v. t.}
\end{itemize}
\begin{itemize}
\item {Utilização:Ant.}
\end{itemize}
\begin{itemize}
\item {Proveniência:(Do lat. \textunderscore de\textunderscore  + \textunderscore odorare\textunderscore )}
\end{itemize}
Dar bom cheiro a; aromatizar. Cf. Cortesão, \textunderscore Subs.\textunderscore 
\section{Deontologia}
\begin{itemize}
\item {Grp. gram.:f.}
\end{itemize}
\begin{itemize}
\item {Proveniência:(Do gr. \textunderscore deon\textunderscore  + \textunderscore logos\textunderscore )}
\end{itemize}
Sciência dos deveres.
\section{Deontologico}
\begin{itemize}
\item {Grp. gram.:adj.}
\end{itemize}
Relativo á deontologia.
\section{Deoperculado}
\begin{itemize}
\item {Grp. gram.:adj.}
\end{itemize}
\begin{itemize}
\item {Proveniência:(De \textunderscore de...\textunderscore  + \textunderscore opérculo\textunderscore )}
\end{itemize}
Que não tem opérculos.
\section{Deparador}
\begin{itemize}
\item {Grp. gram.:adj.}
\end{itemize}
\begin{itemize}
\item {Grp. gram.:M.}
\end{itemize}
Que depara.
Aquelle que depara.
\section{Deparar}
\begin{itemize}
\item {Grp. gram.:v. t.}
\end{itemize}
\begin{itemize}
\item {Grp. gram.:V. i.}
\end{itemize}
\begin{itemize}
\item {Proveniência:(Do lat. \textunderscore de\textunderscore  + \textunderscore parare\textunderscore )}
\end{itemize}
Fazer apparecer.
Apresentar inesperadamente: \textunderscore deparou-me o acaso um amigo\textunderscore .
--Tem-se usado com a significação de \textunderscore encontrar alguém\textunderscore  ou \textunderscore alguma coisa\textunderscore , \textunderscore topar\textunderscore , mas não é correcto.
\section{Departamental}
\begin{itemize}
\item {Grp. gram.:adj.}
\end{itemize}
Relativo a departamento.
\section{Departamento}
\begin{itemize}
\item {Grp. gram.:m.}
\end{itemize}
\begin{itemize}
\item {Proveniência:(Fr. \textunderscore département\textunderscore )}
\end{itemize}
Circunscripção marítima, que comprehende várias capitanias de portos.
Divisão administrativa da França e de algumas outras nações.
\section{Departidamente}
\begin{itemize}
\item {Grp. gram.:adv.}
\end{itemize}
\begin{itemize}
\item {Utilização:Ant.}
\end{itemize}
\begin{itemize}
\item {Proveniência:(De \textunderscore departir\textunderscore )}
\end{itemize}
Por miúdo, por partes.
Minuciosamente.
\section{Departimento}
\begin{itemize}
\item {Grp. gram.:m.}
\end{itemize}
Acto ou effeito de departir.
\section{Departir}
\begin{itemize}
\item {Grp. gram.:v. t.}
\end{itemize}
\begin{itemize}
\item {Utilização:Ant.}
\end{itemize}
\begin{itemize}
\item {Proveniência:(De \textunderscore de...\textunderscore  + \textunderscore partir\textunderscore )}
\end{itemize}
Repartir, dividir.
Narrar minuciosamente.
Contar com familiaridade. Cf. Castilho, \textunderscore D. Quixote\textunderscore , p. 173.
\section{Depassar}
\textunderscore v. t.\textunderscore  (e der.)
(Gallicismo inútil, em vez de exceder, etc.). Cf. Capello e Ivens, \textunderscore De Benguela\textunderscore , I, 246.
\section{Depauperação}
\begin{itemize}
\item {Grp. gram.:f.}
\end{itemize}
Acto ou effeito de depauperar.
\section{Depauperador}
\begin{itemize}
\item {Grp. gram.:adj.}
\end{itemize}
Que depaupera.
\section{Depauperamento}
\begin{itemize}
\item {Grp. gram.:adj.}
\end{itemize}
Acto de depauperar.
\section{Depauperar}
\begin{itemize}
\item {Grp. gram.:v. t.}
\end{itemize}
\begin{itemize}
\item {Proveniência:(Lat. \textunderscore depauperare\textunderscore )}
\end{itemize}
Tornar pobre.
Esgotar os recursos de: \textunderscore depauperar as indústrias\textunderscore .
Extenuar.
\section{Depenado}
\begin{itemize}
\item {Grp. gram.:adj.}
\end{itemize}
\begin{itemize}
\item {Utilização:Fam.}
\end{itemize}
\begin{itemize}
\item {Proveniência:(De \textunderscore depenar\textunderscore )}
\end{itemize}
A que tiraram as penas, ou que perdeu penas: \textunderscore galinha depenada\textunderscore .
A quem fizeram gastar o dinheiro que tinha; que ficou sem cinco reis: \textunderscore fui jogar e fiquei depenado\textunderscore .
\section{Depenador}
\begin{itemize}
\item {Grp. gram.:m.}
\end{itemize}
Aquele que depena.
\section{Depenar}
\begin{itemize}
\item {Grp. gram.:v. t.}
\end{itemize}
\begin{itemize}
\item {Utilização:Fam.}
\end{itemize}
\begin{itemize}
\item {Proveniência:(De \textunderscore de...\textunderscore  + \textunderscore pena\textunderscore )}
\end{itemize}
Tirar as penas a.
Espoliar astuciosamente.
\section{Depenar-se}
\begin{itemize}
\item {Grp. gram.:v. p.}
\end{itemize}
\begin{itemize}
\item {Utilização:Ant.}
\end{itemize}
\begin{itemize}
\item {Proveniência:(De \textunderscore pena\textunderscore )}
\end{itemize}
Lamentar-se; prantear.
\section{Dependência}
\begin{itemize}
\item {Grp. gram.:f.}
\end{itemize}
\begin{itemize}
\item {Proveniência:(De \textunderscore dependente\textunderscore )}
\end{itemize}
Estado de quem ou daquillo que é dependente.
Subordinação; sujeição: \textunderscore estar na dependência de alguém\textunderscore .
Annexo acessório: \textunderscore as dependências de um prédio\textunderscore .
\section{Dependente}
\begin{itemize}
\item {Grp. gram.:adj.}
\end{itemize}
\begin{itemize}
\item {Proveniência:(Lat. \textunderscore dependens\textunderscore )}
\end{itemize}
Que depende.
\section{Dependentemente}
\begin{itemize}
\item {Grp. gram.:adv.}
\end{itemize}
De modo dependente.
\section{Depender}
\begin{itemize}
\item {Grp. gram.:v. i.}
\end{itemize}
\begin{itemize}
\item {Proveniência:(Lat. \textunderscore dependere\textunderscore )}
\end{itemize}
Estar em relação immediata.
Estar subordinado, sujeito.
Proceder, resultar: \textunderscore do que depende uma resolução\textunderscore .
Estar ligado, fazer parte.
\section{Dependura}
\begin{itemize}
\item {Grp. gram.:f.}
\end{itemize}
\begin{itemize}
\item {Utilização:Fam.}
\end{itemize}
\begin{itemize}
\item {Utilização:Restrict.}
\end{itemize}
\begin{itemize}
\item {Proveniência:(De \textunderscore dependurar\textunderscore )}
\end{itemize}
Acto ou effeito de dependurar.
Ruína extrema.
Perigo de vida.
Objecto ou objectos pendurados.
Cacho de uvas ou grupo de cachos que, depois de cortados, se guardam pendurados ordinariamente dentro de casa.
\section{Dependurão}
\begin{itemize}
\item {Grp. gram.:m.}
\end{itemize}
\begin{itemize}
\item {Utilização:Prov.}
\end{itemize}
\begin{itemize}
\item {Utilização:beir.}
\end{itemize}
O mesmo que dependura, us. na loc. adv. \textunderscore ao dependurão\textunderscore .
\section{Dependurar}
\begin{itemize}
\item {Grp. gram.:v. t.}
\end{itemize}
O mesmo que \textunderscore pendurar\textunderscore .
\section{Dependuro}
\begin{itemize}
\item {Grp. gram.:m.}
\end{itemize}
O mesmo que \textunderscore dependura\textunderscore .
\section{Depenicar}
\begin{itemize}
\item {Grp. gram.:v. t.}
\end{itemize}
\begin{itemize}
\item {Utilização:Pop.}
\end{itemize}
\begin{itemize}
\item {Grp. gram.:V. i.}
\end{itemize}
Tirar penas a, uma a uma, pouco a pouco.
Comer ou tirar para comer (pequenina porção de qualquer iguaria).
Comer muito pouco ou tirar pequenas porções para comer.
(Freq. de \textunderscore depenar\textunderscore )
\section{Depennado}
\begin{itemize}
\item {Grp. gram.:adj.}
\end{itemize}
\begin{itemize}
\item {Utilização:Fam.}
\end{itemize}
\begin{itemize}
\item {Proveniência:(De \textunderscore depennar\textunderscore )}
\end{itemize}
A que tiraram as pennas, ou que perdeu pennas: \textunderscore gallinha depennada\textunderscore .
A quem fizeram gastar o dinheiro que tinha; que ficou sem cinco reis: \textunderscore fui jogar e fiquei depennado\textunderscore .
\section{Depennador}
\begin{itemize}
\item {Grp. gram.:m.}
\end{itemize}
Aquelle que depenna.
\section{Depennar}
\begin{itemize}
\item {Grp. gram.:v. t.}
\end{itemize}
\begin{itemize}
\item {Utilização:Fam.}
\end{itemize}
\begin{itemize}
\item {Proveniência:(De \textunderscore de...\textunderscore  + \textunderscore penna\textunderscore )}
\end{itemize}
Tirar as pennas a.
Espoliar astuciosamente.
\section{Depennicar}
\begin{itemize}
\item {Grp. gram.:v. t.}
\end{itemize}
\begin{itemize}
\item {Utilização:Pop.}
\end{itemize}
\begin{itemize}
\item {Grp. gram.:V. i.}
\end{itemize}
Tirar pennas a, uma a uma, pouco a pouco.
Comer ou tirar para comer (pequenina porção de qualquer iguaria)
Comer muito pouco ou tirar pequenas porções para comer.
(Freq. de \textunderscore depennar\textunderscore )
\section{Deperder}
\begin{itemize}
\item {Grp. gram.:v. t.}
\end{itemize}
\begin{itemize}
\item {Utilização:Ant.}
\end{itemize}
\begin{itemize}
\item {Proveniência:(Lat. \textunderscore deperdere\textunderscore )}
\end{itemize}
O mesmo que perder.
Dissipar.
Arruinar. Cf. Usque, \textunderscore Tribulações\textunderscore , 21.
\section{Depérdito}
\begin{itemize}
\item {Grp. gram.:adj.}
\end{itemize}
\begin{itemize}
\item {Utilização:Des.}
\end{itemize}
\begin{itemize}
\item {Proveniência:(Lat. \textunderscore deperditus\textunderscore )}
\end{itemize}
O mesmo que \textunderscore perdido\textunderscore .
\section{Deperecer}
\begin{itemize}
\item {Grp. gram.:v. i.}
\end{itemize}
\begin{itemize}
\item {Proveniência:(De \textunderscore perecer\textunderscore )}
\end{itemize}
Perecer a pouco e pouco. Cf. Camillo, \textunderscore Doze Casamentos\textunderscore , 205.
\section{Deperecimento}
\begin{itemize}
\item {Grp. gram.:m.}
\end{itemize}
Acto de deperecer.
\section{Dephlegmação}
\begin{itemize}
\item {Grp. gram.:f.}
\end{itemize}
Acto de dephlegmar.
\section{Dephlegmar}
\begin{itemize}
\item {Grp. gram.:v. t.}
\end{itemize}
\begin{itemize}
\item {Proveniência:(Do gr. \textunderscore phlegma\textunderscore )}
\end{itemize}
Destillar, para separar de uma substância a parte aquosa.
\section{Depilação}
\begin{itemize}
\item {Grp. gram.:f.}
\end{itemize}
Acto ou effeito de depilar.
\section{Depilar}
\begin{itemize}
\item {Grp. gram.:v. t.}
\end{itemize}
\begin{itemize}
\item {Proveniência:(Lat. \textunderscore depilare\textunderscore )}
\end{itemize}
Pelar; arrancar ou fazer caír o pêlo ou o cabello de.
\section{Depilatório}
\begin{itemize}
\item {Grp. gram.:adj.}
\end{itemize}
\begin{itemize}
\item {Grp. gram.:M.}
\end{itemize}
\begin{itemize}
\item {Proveniência:(De \textunderscore depilar\textunderscore )}
\end{itemize}
Que depila.
Substância ou medicamento que depila.
\section{Depleção}
\begin{itemize}
\item {Grp. gram.:f.}
\end{itemize}
\begin{itemize}
\item {Proveniência:(Lat. \textunderscore depletio\textunderscore )}
\end{itemize}
Acto de deminuir os humores de um corpo vivo.
\section{Depletivo}
\begin{itemize}
\item {Grp. gram.:adj.}
\end{itemize}
\begin{itemize}
\item {Proveniência:(Do lat. \textunderscore depletus\textunderscore )}
\end{itemize}
Que produz depleção.
\section{Deploração}
\begin{itemize}
\item {Grp. gram.:f.}
\end{itemize}
\begin{itemize}
\item {Proveniência:(Lat. \textunderscore deploratio\textunderscore )}
\end{itemize}
Acto de deplorar.
Palavras com que se deplora.
\section{Deplorador}
\begin{itemize}
\item {Grp. gram.:m.}
\end{itemize}
Aquelle que deplora.
\section{Deplorar}
\begin{itemize}
\item {Grp. gram.:v. t.}
\end{itemize}
\begin{itemize}
\item {Proveniência:(Lat. \textunderscore deplorare\textunderscore )}
\end{itemize}
Lastimar, lamentar.
\section{Deplorativo}
\begin{itemize}
\item {Grp. gram.:adj.}
\end{itemize}
\begin{itemize}
\item {Proveniência:(De \textunderscore deplorar\textunderscore )}
\end{itemize}
Que deplora; lastimoso.
Deploratório. Cf. Camillo, \textunderscore Cavar em Ruinas\textunderscore , 127.
\section{Deploratório}
\begin{itemize}
\item {Grp. gram.:adj.}
\end{itemize}
Relativo á deploração.
\section{Deplorável}
\begin{itemize}
\item {Grp. gram.:adj.}
\end{itemize}
\begin{itemize}
\item {Proveniência:(De \textunderscore deplorar\textunderscore )}
\end{itemize}
Digno de deploração.
Lastimável; detestável: \textunderscore procedimento deplorável\textunderscore .
\section{Deploravelmente}
\begin{itemize}
\item {Grp. gram.:adv.}
\end{itemize}
De modo deplorável.
\section{Deplumar}
\begin{itemize}
\item {Grp. gram.:v. t.}
\end{itemize}
(V.desplumar)
\section{Depodado}
\begin{itemize}
\item {Grp. gram.:adj.}
\end{itemize}
\begin{itemize}
\item {Utilização:Ant.}
\end{itemize}
O mesmo que \textunderscore deputado\textunderscore . Cf. Fr. Fortun., \textunderscore Inéd.\textunderscore , I, 304.
\section{Depoência}
\begin{itemize}
\item {Grp. gram.:f.}
\end{itemize}
\begin{itemize}
\item {Utilização:Gram.}
\end{itemize}
\begin{itemize}
\item {Proveniência:(De \textunderscore depoente\textunderscore )}
\end{itemize}
Carácter das fórmas verbaes do latim, que na passiva têm o significado da voz activa. Cf. J. Ribeiro, \textunderscore Dicc. Gram.\textunderscore 
\section{Depoente}
\begin{itemize}
\item {Grp. gram.:m. f.}
\end{itemize}
\begin{itemize}
\item {Grp. gram.:Adj.}
\end{itemize}
\begin{itemize}
\item {Utilização:Gram.}
\end{itemize}
\begin{itemize}
\item {Proveniência:(Lat. \textunderscore deponens\textunderscore )}
\end{itemize}
Pessôa, que depõe em juízo como testemunha.
Diz-se dos verbos latinos que, tendo a fórma da voz passiva, têm significação e regência da activa.
\section{Depoimento}
\begin{itemize}
\item {fónica:po-i}
\end{itemize}
\begin{itemize}
\item {Grp. gram.:m.}
\end{itemize}
Acto ou effeito de depôr.
\section{Depois}
\begin{itemize}
\item {Grp. gram.:adv.}
\end{itemize}
\begin{itemize}
\item {Grp. gram.:Loc. prep.}
\end{itemize}
\begin{itemize}
\item {Proveniência:(Do lat. \textunderscore de\textunderscore  + \textunderscore post\textunderscore )}
\end{itemize}
Posteriormente.
Do lado detrás.
Em seguida.
Além disso.
\textunderscore Depois de\textunderscore , seguidamente a; atrás de.
\section{Depolarização}
\begin{itemize}
\item {Grp. gram.:f.}
\end{itemize}
Acto de depolarizar.
\section{Depolarizar}
\begin{itemize}
\item {Grp. gram.:v. t.}
\end{itemize}
\begin{itemize}
\item {Proveniência:(De \textunderscore de...\textunderscore  + \textunderscore polarizar\textunderscore )}
\end{itemize}
Fazer cessar em (alguma coisa) o estado de polaridade ou polarização.
\section{Depolir}
\textunderscore v. t.\textunderscore  (e der.)
O mesmo que \textunderscore despolir\textunderscore , etc.
\section{Deponente}
\begin{itemize}
\item {Grp. gram.:adj.}
\end{itemize}
\begin{itemize}
\item {Proveniência:(Lat. \textunderscore deponens\textunderscore )}
\end{itemize}
O mesmo que \textunderscore depoente\textunderscore .
\section{Depopulação}
\begin{itemize}
\item {Grp. gram.:f.}
\end{itemize}
\begin{itemize}
\item {Utilização:Des.}
\end{itemize}
\begin{itemize}
\item {Proveniência:(Lat. \textunderscore depopulatio\textunderscore )}
\end{itemize}
Assolação; devastação.
Ruína.
\section{Depopular}
\textunderscore v. t.\textunderscore  (e der.)
(V. \textunderscore despovoar\textunderscore , etc.)
\section{Depopularizar}
\begin{itemize}
\item {Grp. gram.:v. t.}
\end{itemize}
(V.despopularizar)
\section{Depor}
\begin{itemize}
\item {Grp. gram.:v. t.}
\end{itemize}
\begin{itemize}
\item {Proveniência:(Lat. \textunderscore deponere\textunderscore )}
\end{itemize}
Pôr de lado.
Largar: \textunderscore depor um fardo\textunderscore .
Renunciar: \textunderscore depor um emprêgo\textunderscore .
Demittir, despojar de cargo, dignidade, etc.: \textunderscore depor um empregado público\textunderscore .
Fazer depoimento; prestar declarações como testemunha.
Depositar.
\section{Deportação}
\begin{itemize}
\item {Grp. gram.:f.}
\end{itemize}
\begin{itemize}
\item {Proveniência:(Lat. \textunderscore deportatio\textunderscore )}
\end{itemize}
Acto ou effeito de deportar.
\section{Deportado}
\begin{itemize}
\item {Grp. gram.:adj.}
\end{itemize}
\begin{itemize}
\item {Grp. gram.:M.}
\end{itemize}
\begin{itemize}
\item {Proveniência:(De \textunderscore deportar\textunderscore )}
\end{itemize}
Condemnado a deportação.
Indivíduo deportado: \textunderscore relação dos deportados\textunderscore .
\section{Deportar}
\begin{itemize}
\item {Grp. gram.:v. t.}
\end{itemize}
\begin{itemize}
\item {Proveniência:(Lat. \textunderscore deportare\textunderscore )}
\end{itemize}
Levar para fóra, para longe.
Desterrar.
Impor a pena de degrêdo a.
\section{Deporte}
\begin{itemize}
\item {Grp. gram.:m.}
\end{itemize}
Acto de deportar; transporte:«\textunderscore a presa foi tão ampla, que tres dias se dispendêrão no deporte della do castello para o arraial\textunderscore ». Filinto, \textunderscore D. Man.\textunderscore , III, 108.
\section{Deporte}
\begin{itemize}
\item {Grp. gram.:m.}
\end{itemize}
\begin{itemize}
\item {Utilização:Ant.}
\end{itemize}
Distracção, diversão, recreio, o mesmo que \textunderscore desporte\textunderscore :«\textunderscore estou mais morta que a morte sem deporte\textunderscore ». G. Vicente, I, 20.
\section{Depôrto}
\begin{itemize}
\item {Grp. gram.:m.}
\end{itemize}
O mesmo que \textunderscore deporte\textunderscore ^2.
\section{Depós}
\begin{itemize}
\item {Grp. gram.:prep.}
\end{itemize}
O mesmo que \textunderscore após\textunderscore .
\section{Deposição}
\begin{itemize}
\item {Grp. gram.:f.}
\end{itemize}
\begin{itemize}
\item {Proveniência:(Lat. \textunderscore depositio\textunderscore )}
\end{itemize}
Acto de depor.
\section{Depositador}
\begin{itemize}
\item {Grp. gram.:m.}
\end{itemize}
\begin{itemize}
\item {Utilização:Des.}
\end{itemize}
Aquelle que deposita.
\section{Depositante}
\begin{itemize}
\item {Grp. gram.:m. ,  f.  e  adj.}
\end{itemize}
Pessôa, que deposita.
\section{Depositar}
\begin{itemize}
\item {Grp. gram.:v. t.}
\end{itemize}
\begin{itemize}
\item {Grp. gram.:V. p.}
\end{itemize}
\begin{itemize}
\item {Proveniência:(De \textunderscore depósito\textunderscore )}
\end{itemize}
Pôr em depósito.
Confiar, dar a guardar, temporariamente.
Entregar solennemente, confiar com solennidades jurídicas.
Depor, largar.
Têr (confiança): \textunderscore não depositei confiança nelle\textunderscore .
Fazer inscrever nos depósitos officiaes (um modêlo ou desenho industrial).
Assentar no fundo, (falando de uma substância que estava em suspensão num líquido).
\section{Depositário}
\begin{itemize}
\item {Grp. gram.:m.}
\end{itemize}
\begin{itemize}
\item {Utilização:Fig.}
\end{itemize}
\begin{itemize}
\item {Proveniência:(Lat. \textunderscore depositarius\textunderscore )}
\end{itemize}
Indivíduo, que recebe em depósito.
Pessôa, a quem se communicou particularmente algum segrêdo.
Objecto, lugar, etc., que foi, por assim dizer, testemunha de successos particulares.
\section{Depósito}
\begin{itemize}
\item {Grp. gram.:m.}
\end{itemize}
\begin{itemize}
\item {Proveniência:(Lat. \textunderscore depositus\textunderscore )}
\end{itemize}
Acto de confiar ou de dar a guardar.
Aquillo que se depositou: \textunderscore um depósito de 100$000 reis\textunderscore .
Estado ou lugar daquillo que se depositou.
Substâncias ou impurezas, que, misturadas com um líquido, se depositam no fundo do respectivo vaso.
Sedimento.
Reservatório (de água).
\section{Depraça}
\begin{itemize}
\item {Grp. gram.:adv.}
\end{itemize}
\begin{itemize}
\item {Utilização:Ant.}
\end{itemize}
\begin{itemize}
\item {Proveniência:(De \textunderscore praça\textunderscore )}
\end{itemize}
Publicamente; sem vergonha nem rodeios.
\section{Depravação}
\begin{itemize}
\item {Grp. gram.:f.}
\end{itemize}
\begin{itemize}
\item {Proveniência:(Lat. \textunderscore depravatio\textunderscore )}
\end{itemize}
Acto ou effeito de depravar.
Perversão.
\section{Depravadamente}
\begin{itemize}
\item {Grp. gram.:adv.}
\end{itemize}
De modo depravado.
\section{Depravado}
\begin{itemize}
\item {Grp. gram.:adj.}
\end{itemize}
\begin{itemize}
\item {Proveniência:(De \textunderscore depravar\textunderscore )}
\end{itemize}
Malvado; perverso.
Corrompido.
\section{Depravador}
\begin{itemize}
\item {Grp. gram.:m.  e  adj.}
\end{itemize}
O que deprava.
\section{Depravar}
\begin{itemize}
\item {Grp. gram.:v. t.}
\end{itemize}
\begin{itemize}
\item {Proveniência:(Lat. \textunderscore depravare\textunderscore )}
\end{itemize}
Perverter.
Tornar mau.
Alterar prejudicialmente.
Estragar.
\section{Deprecação}
\begin{itemize}
\item {Grp. gram.:f.}
\end{itemize}
\begin{itemize}
\item {Proveniência:(Lat. \textunderscore deprecatio\textunderscore )}
\end{itemize}
Acto de deprecar.
\section{Deprecada}
\begin{itemize}
\item {Grp. gram.:f.}
\end{itemize}
\begin{itemize}
\item {Proveniência:(De \textunderscore deprecar\textunderscore )}
\end{itemize}
Documento, em que um juiz ou um tribunal pede a outro que realize algum acto ou diligência judicial.
\section{Deprecante}
\begin{itemize}
\item {Grp. gram.:m.  e  adj.}
\end{itemize}
\begin{itemize}
\item {Proveniência:(Lat. \textunderscore deprecans\textunderscore )}
\end{itemize}
O que depreca.
\section{Deprecar}
\begin{itemize}
\item {Grp. gram.:v. t.}
\end{itemize}
\begin{itemize}
\item {Grp. gram.:V. i.}
\end{itemize}
\begin{itemize}
\item {Proveniência:(Lat. \textunderscore deprecari\textunderscore )}
\end{itemize}
Supplicar, pedir submissamente.
Pedir (um juiz a outro o cumprimento de um mandado ou uma diligência judicial).
Expedir deprecada.
\section{Deprecativamente}
\begin{itemize}
\item {Grp. gram.:adv.}
\end{itemize}
De modo deprecativo.
\section{Deprecativo}
\begin{itemize}
\item {Grp. gram.:adj.}
\end{itemize}
\begin{itemize}
\item {Proveniência:(Lat. \textunderscore deprecativus\textunderscore )}
\end{itemize}
Em que há deprecação.
\section{Deprecatório}
\begin{itemize}
\item {Grp. gram.:adj.}
\end{itemize}
\begin{itemize}
\item {Proveniência:(Lat. \textunderscore deprecatorius\textunderscore )}
\end{itemize}
Relativo a deprecação.
\section{Depreciação}
\begin{itemize}
\item {Grp. gram.:f.}
\end{itemize}
Acto ou effeito de depreciar.
\section{Depreciador}
\begin{itemize}
\item {Grp. gram.:adj.}
\end{itemize}
\begin{itemize}
\item {Grp. gram.:M.}
\end{itemize}
\begin{itemize}
\item {Proveniência:(Lat. \textunderscore depreciator\textunderscore )}
\end{itemize}
Que deprecia.
Aquelle que deprecia.
\section{Depreciar}
\begin{itemize}
\item {Grp. gram.:v. t.}
\end{itemize}
\begin{itemize}
\item {Proveniência:(Lat. \textunderscore depretiare\textunderscore )}
\end{itemize}
Deminuir ou tirar o valor de.
Rebaixar: \textunderscore depreciar os merecimentos de alguém\textunderscore .
Aviltar, desprezar.
\section{Depreciativo}
\begin{itemize}
\item {Grp. gram.:adj.}
\end{itemize}
\begin{itemize}
\item {Proveniência:(De \textunderscore depreciar\textunderscore )}
\end{itemize}
Em que há depreciação.
\section{Depreciável}
\begin{itemize}
\item {Grp. gram.:adj.}
\end{itemize}
Que se póde depreciar.
\section{Depredação}
\begin{itemize}
\item {Grp. gram.:f.}
\end{itemize}
\begin{itemize}
\item {Proveniência:(Lat. \textunderscore depraedatio\textunderscore )}
\end{itemize}
Acto de depredar.
\section{Depredador}
\begin{itemize}
\item {Grp. gram.:m.}
\end{itemize}
\begin{itemize}
\item {Proveniência:(Lat. \textunderscore depraedator\textunderscore )}
\end{itemize}
Aquelle que depreda.
\section{Depredar}
\begin{itemize}
\item {Grp. gram.:v. t.}
\end{itemize}
\begin{itemize}
\item {Proveniência:(Lat. \textunderscore depraedari\textunderscore )}
\end{itemize}
Fazer presa em.
Saquear.
Espoliar.
Assolar, talar.
\section{Depredatório}
\begin{itemize}
\item {Grp. gram.:adj.}
\end{itemize}
\begin{itemize}
\item {Proveniência:(De \textunderscore depredar\textunderscore )}
\end{itemize}
Em que há depredação, ou que tem por fim depredação.
\section{Depreender}
\begin{itemize}
\item {Grp. gram.:v. t.}
\end{itemize}
\begin{itemize}
\item {Proveniência:(Lat. \textunderscore deprehendere\textunderscore )}
\end{itemize}
Compreender, chegar ao conhecimento de.
Inferir, deduzir: \textunderscore é o que se depreende do que dizes\textunderscore .
\section{Deprehender}
\begin{itemize}
\item {Grp. gram.:v. t.}
\end{itemize}
\begin{itemize}
\item {Proveniência:(Lat. \textunderscore deprehendere\textunderscore )}
\end{itemize}
Comprehender, chegar ao conhecimento de.
Inferir, deduzir: \textunderscore é o que se deprehende do que dizes\textunderscore .
\section{Depressa}
\begin{itemize}
\item {Grp. gram.:adv.}
\end{itemize}
\begin{itemize}
\item {Proveniência:(De \textunderscore pressa\textunderscore )}
\end{itemize}
Com pressa, apressadamente.
Com rapidez; immediatamente; em breve tempo.
\section{De-pressa}
\begin{itemize}
\item {Grp. gram.:adv.}
\end{itemize}
\begin{itemize}
\item {Proveniência:(De \textunderscore pressa\textunderscore )}
\end{itemize}
Com pressa, apressadamente.
Com rapidez; immediatamente; em breve tempo.
\section{Depressão}
\begin{itemize}
\item {Grp. gram.:f.}
\end{itemize}
\begin{itemize}
\item {Utilização:Fig.}
\end{itemize}
\begin{itemize}
\item {Proveniência:(Lat. \textunderscore depressio\textunderscore )}
\end{itemize}
Acto de deprimir.
Abaixamento por effeito de pressão.
Deminuição.
Terreno mais baixo que o que lhe fica aos lados.
Pequena cavidade.
Abatimento phýsico ou moral.
\section{Depressivo}
\begin{itemize}
\item {Grp. gram.:adj.}
\end{itemize}
\begin{itemize}
\item {Proveniência:(De \textunderscore depresso\textunderscore )}
\end{itemize}
Em que há depressão.
Que causa depressão.
Deprimente. Cf. Garrett, \textunderscore Port. na Balança\textunderscore , 36 e 127.
\section{Depresso}
\begin{itemize}
\item {Grp. gram.:adj.}
\end{itemize}
\begin{itemize}
\item {Proveniência:(Lat. \textunderscore depressus\textunderscore )}
\end{itemize}
Em que há depressão.
\section{Depressor}
\begin{itemize}
\item {Grp. gram.:adj.}
\end{itemize}
\begin{itemize}
\item {Grp. gram.:M.}
\end{itemize}
\begin{itemize}
\item {Proveniência:(Lat. \textunderscore depressor\textunderscore )}
\end{itemize}
Que deprime.
Aquelle que deprime.
\section{Deprimente}
\begin{itemize}
\item {Grp. gram.:adj.}
\end{itemize}
\begin{itemize}
\item {Proveniência:(Lat. \textunderscore deprimens\textunderscore )}
\end{itemize}
Que deprime.
Aviltante: \textunderscore boato deprimente\textunderscore . Cf. Th. Ribeiro, \textunderscore Jornadas\textunderscore , II, 176.
\section{Deprimir}
\begin{itemize}
\item {Grp. gram.:v. t.}
\end{itemize}
\begin{itemize}
\item {Proveniência:(Lat. \textunderscore deprimere\textunderscore )}
\end{itemize}
Abaixar, abater.
Aviltar.
Humilhar.
\section{Depuração}
\begin{itemize}
\item {Grp. gram.:f.}
\end{itemize}
Acto de depurar.
\section{Depurador}
\begin{itemize}
\item {Grp. gram.:adj.}
\end{itemize}
\begin{itemize}
\item {Grp. gram.:m.}
\end{itemize}
Que depura.
Aquelle que depura.
\section{Depurante}
\begin{itemize}
\item {Grp. gram.:adj.}
\end{itemize}
Que depura; depurador.
\section{Depurar}
\begin{itemize}
\item {Grp. gram.:v. t.}
\end{itemize}
\begin{itemize}
\item {Proveniência:(De \textunderscore puro\textunderscore )}
\end{itemize}
Tornar puro.
Limpar.
\section{Depurativo}
\begin{itemize}
\item {Grp. gram.:adj.}
\end{itemize}
\begin{itemize}
\item {Grp. gram.:M.}
\end{itemize}
\begin{itemize}
\item {Proveniência:(De \textunderscore depurar\textunderscore )}
\end{itemize}
Que depura.
Qualquer coisa que depura.
\section{Depuratório}
\begin{itemize}
\item {Grp. gram.:adj.}
\end{itemize}
(V.depurativo)
\section{Deputação}
\begin{itemize}
\item {Grp. gram.:f.}
\end{itemize}
\begin{itemize}
\item {Proveniência:(Lat. \textunderscore deputatio\textunderscore )}
\end{itemize}
Acto de deputar.
Pessôas deputadas.
\section{Deputado}
\begin{itemize}
\item {Grp. gram.:m.}
\end{itemize}
\begin{itemize}
\item {Proveniência:(De \textunderscore deputar\textunderscore )}
\end{itemize}
Aquelle que é commissionado para curar de negócios de outrem.
Membro eleito de assembleia legislativa.
Membro eleito de outras corporações.
Vogal nomeado de certas corporações.
\section{Deputar}
\begin{itemize}
\item {Grp. gram.:v. t.}
\end{itemize}
\begin{itemize}
\item {Proveniência:(Do lat. \textunderscore deputare\textunderscore )}
\end{itemize}
Delegar.
Mandar em comissão.
Incumbir.
\section{Dequitação}
\begin{itemize}
\item {Grp. gram.:f.}
\end{itemize}
O mesmo que \textunderscore dequitadura\textunderscore .
\section{Dequitadura}
\begin{itemize}
\item {Grp. gram.:f.}
\end{itemize}
\begin{itemize}
\item {Utilização:Med.}
\end{itemize}
\begin{itemize}
\item {Proveniência:(De \textunderscore dequitar-se\textunderscore )}
\end{itemize}
Quéda da placenta, na occasião do parto.
\section{Dequitar-se}
\begin{itemize}
\item {Grp. gram.:v. p.}
\end{itemize}
\begin{itemize}
\item {Utilização:Med.}
\end{itemize}
Diz-se da parturiente, a quem a placenta caiu.
Dar á luz.
(Cp. \textunderscore quitar\textunderscore )
\section{Deramar}
\begin{itemize}
\item {fónica:ra}
\end{itemize}
\begin{itemize}
\item {Grp. gram.:v. t.}
\end{itemize}
O mesmo que \textunderscore derramar\textunderscore . Cf. Castilho, \textunderscore Felic. pela Agr.\textunderscore ,19.
\section{Derelição}
\begin{itemize}
\item {Grp. gram.:f.}
\end{itemize}
\begin{itemize}
\item {Proveniência:(Lat. \textunderscore derelictio\textunderscore )}
\end{itemize}
Abandono, desamparo.
\section{Derelicção}
\begin{itemize}
\item {Grp. gram.:f.}
\end{itemize}
\begin{itemize}
\item {Proveniência:(Lat. \textunderscore derelictio\textunderscore )}
\end{itemize}
Abandono, desamparo.
\section{Derelicto}
\begin{itemize}
\item {Grp. gram.:adj.}
\end{itemize}
\begin{itemize}
\item {Proveniência:(Lat. \textunderscore derelictus\textunderscore )}
\end{itemize}
Abandonado; desprezado.
\section{Derelito}
\begin{itemize}
\item {Grp. gram.:adj.}
\end{itemize}
\begin{itemize}
\item {Proveniência:(Lat. \textunderscore derelictus\textunderscore )}
\end{itemize}
Abandonado; desprezado.
\section{Derengue}
\begin{itemize}
\item {Grp. gram.:m.}
\end{itemize}
O mesmo que \textunderscore derrengue\textunderscore :«\textunderscore agradeceu com... derengues de cintura.\textunderscore »Camillo, \textunderscore Corja\textunderscore , 252.«\textunderscore Com um garboso derengue de cinta.\textunderscore »Id., 318.
\section{Derepastar}
\begin{itemize}
\item {fónica:re}
\end{itemize}
\begin{itemize}
\item {Grp. gram.:v. t.}
\end{itemize}
Tirar do pasto.
\section{Derrepastar}
\begin{itemize}
\item {Grp. gram.:v. t.}
\end{itemize}
Tirar do pasto.
\section{De-repente}
\begin{itemize}
\item {Grp. gram.:adv.}
\end{itemize}
De súbito.
Repentinamente.
De improviso.
(Ant. lat. \textunderscore derepente\textunderscore , de \textunderscore repente\textunderscore )
\section{Derepente}
\begin{itemize}
\item {fónica:re}
\end{itemize}
\begin{itemize}
\item {Grp. gram.:adv.}
\end{itemize}
De súbito.
Repentinamente.
De improviso.
(Ant. lat. \textunderscore derepente\textunderscore , de \textunderscore repente\textunderscore )
\section{Deres}
\begin{itemize}
\item {Grp. gram.:m. pl.}
\end{itemize}
Uma das castas inferiores da Índia, em Dio.
\section{Deríbia}
\begin{itemize}
\item {Grp. gram.:f.}
\end{itemize}
\begin{itemize}
\item {Proveniência:(Do gr. \textunderscore deris\textunderscore  + \textunderscore bios\textunderscore )}
\end{itemize}
Espécie de pyrilampo.
\section{Derisão}
\begin{itemize}
\item {fónica:ri}
\end{itemize}
\begin{itemize}
\item {Grp. gram.:f.}
\end{itemize}
\begin{itemize}
\item {Proveniência:(Lat. \textunderscore derisio\textunderscore )}
\end{itemize}
Riso de desprêzo.
Escárneo.
\section{Derisca}
\begin{itemize}
\item {fónica:ris}
\end{itemize}
\begin{itemize}
\item {Grp. gram.:f.}
\end{itemize}
Desobriga.
Acto de \textunderscore deriscar\textunderscore .
\section{Deriscar}
\begin{itemize}
\item {fónica:ris}
\end{itemize}
\begin{itemize}
\item {Grp. gram.:v. t.}
\end{itemize}
\begin{itemize}
\item {Proveniência:(De \textunderscore riscar\textunderscore )}
\end{itemize}
Riscar (o nome no rol dos que se hão de confessar).
\section{Derisor}
\begin{itemize}
\item {fónica:ri}
\end{itemize}
\begin{itemize}
\item {Grp. gram.:m.}
\end{itemize}
\begin{itemize}
\item {Proveniência:(Lat. \textunderscore derisor\textunderscore )}
\end{itemize}
Aquelle que moteja ou escarnece.
\section{Derisoriamente}
\begin{itemize}
\item {fónica:ri}
\end{itemize}
\begin{itemize}
\item {Grp. gram.:adv.}
\end{itemize}
De modo derisório.
\section{Derisório}
\begin{itemize}
\item {fónica:ri}
\end{itemize}
\begin{itemize}
\item {Grp. gram.:adj.}
\end{itemize}
\begin{itemize}
\item {Proveniência:(Lat. \textunderscore derisorius\textunderscore )}
\end{itemize}
Em que há derisão.
\section{Derivação}
\begin{itemize}
\item {Grp. gram.:f.}
\end{itemize}
\begin{itemize}
\item {Utilização:Gram.}
\end{itemize}
\begin{itemize}
\item {Proveniência:(Lat. \textunderscore derivatio\textunderscore )}
\end{itemize}
Acto ou effeito de derivar.
Formação de palavras, que, tendo determinado thema ou raíz, variam na terminação ou nos suffixos.
\section{Derivado}
\begin{itemize}
\item {Grp. gram.:m.}
\end{itemize}
\begin{itemize}
\item {Grp. gram.:M. pl.}
\end{itemize}
\begin{itemize}
\item {Utilização:Mús.}
\end{itemize}
\begin{itemize}
\item {Utilização:Mús.}
\end{itemize}
\begin{itemize}
\item {Proveniência:(De \textunderscore derivar\textunderscore )}
\end{itemize}
Palavra que deriva de outra.
Dizem-se os compassos compostos.
Dizem-se os acordes invertidos, alterados, e outros, que alguns harmonistas não consideram naturaes.
\section{Derivante}
\begin{itemize}
\item {Grp. gram.:adj.}
\end{itemize}
\begin{itemize}
\item {Proveniência:(Lat. \textunderscore derivans\textunderscore )}
\end{itemize}
Que deriva.
\section{Derivar}
\begin{itemize}
\item {Grp. gram.:v. t.}
\end{itemize}
\begin{itemize}
\item {Grp. gram.:V. i.}
\end{itemize}
\begin{itemize}
\item {Utilização:Náut.}
\end{itemize}
\begin{itemize}
\item {Proveniência:(Lat. \textunderscore derivare\textunderscore )}
\end{itemize}
Desviar (curso de águas).
Formar (palavras) com a raíz de outras e suffixos.
Fazer provir.
Correr (falando-se de rios ou regatos).
Originar-se, provir: \textunderscore daqui derivou aquella desordem\textunderscore .
Sêr arrastado por um curso de água: \textunderscore o madeiro derivou na corrente\textunderscore .
Descender: \textunderscore os Gamas derivam de bôa estirpe\textunderscore .
Descaír, apartar-se do rumo: \textunderscore a barca derivou para Léste\textunderscore .
\section{Derivativo}
\begin{itemize}
\item {Grp. gram.:adj.}
\end{itemize}
\begin{itemize}
\item {Grp. gram.:M.}
\end{itemize}
\begin{itemize}
\item {Proveniência:(Lat. \textunderscore derivativus\textunderscore )}
\end{itemize}
Relativo a derivação.
Revulsivo: \textunderscore applicar um derivativo\textunderscore .
\section{Derivatório}
\begin{itemize}
\item {Grp. gram.:adj.}
\end{itemize}
(V.derivativo)
\section{Derivável}
\begin{itemize}
\item {Grp. gram.:adj.}
\end{itemize}
\begin{itemize}
\item {Proveniência:(De \textunderscore derivar\textunderscore )}
\end{itemize}
Que se póde derivar.
Que tem derivação provável. Cf. Latino, \textunderscore Elogios Acad\textunderscore ., I, p. 66.
\section{Derma}
\begin{itemize}
\item {Grp. gram.:f.}
\end{itemize}
\begin{itemize}
\item {Utilização:Physiol.}
\end{itemize}
\begin{itemize}
\item {Proveniência:(Gr. \textunderscore derma\textunderscore )}
\end{itemize}
Tecido, que fórma a espessura da pelle.
\section{Dermatina}
\begin{itemize}
\item {Grp. gram.:f.}
\end{itemize}
\begin{itemize}
\item {Proveniência:(Do rad. de \textunderscore derma\textunderscore )}
\end{itemize}
Variedade de magnesita.
\section{Dermatite}
\begin{itemize}
\item {Grp. gram.:f.}
\end{itemize}
\begin{itemize}
\item {Proveniência:(De \textunderscore derma\textunderscore )}
\end{itemize}
Inflammação da pelle.
\section{Dermatocarpo}
\begin{itemize}
\item {Grp. gram.:m.}
\end{itemize}
\begin{itemize}
\item {Utilização:Bot.}
\end{itemize}
\begin{itemize}
\item {Proveniência:(Do gr. \textunderscore derma\textunderscore  + \textunderscore karpos\textunderscore )}
\end{itemize}
Espécie de cogumelo, cujas sementes se acham espalhadas sôbre uma membrana fructífera.
\section{Dermatodonte}
\begin{itemize}
\item {Grp. gram.:adj.}
\end{itemize}
\begin{itemize}
\item {Proveniência:(Do gr. \textunderscore derma\textunderscore  + \textunderscore odous\textunderscore )}
\end{itemize}
Guarnecido de dentículos membranosos.
\section{Dermatogastro}
\begin{itemize}
\item {Grp. gram.:m.}
\end{itemize}
\begin{itemize}
\item {Utilização:Bot.}
\end{itemize}
\begin{itemize}
\item {Proveniência:(Do gr. \textunderscore derma\textunderscore  + \textunderscore gaster\textunderscore )}
\end{itemize}
Tríbo de cogumelos.
\section{Dermatogênio}
\begin{itemize}
\item {Grp. gram.:m.}
\end{itemize}
\begin{itemize}
\item {Utilização:Bot.}
\end{itemize}
\begin{itemize}
\item {Proveniência:(Do gr. \textunderscore derma\textunderscore  + \textunderscore genos\textunderscore )}
\end{itemize}
Parte externa do meristema differenciado, donde se fórma a parte externa do systema tegumentar.
\section{Dermatografia}
\begin{itemize}
\item {Grp. gram.:f.}
\end{itemize}
\begin{itemize}
\item {Proveniência:(Do gr. \textunderscore derma\textunderscore  + \textunderscore graphein\textunderscore )}
\end{itemize}
Descripção da pele.
\section{Dermatographia}
\begin{itemize}
\item {Grp. gram.:f.}
\end{itemize}
\begin{itemize}
\item {Proveniência:(Do gr. \textunderscore derma\textunderscore  + \textunderscore graphein\textunderscore )}
\end{itemize}
Descripção da pelle.
\section{Dermatoide}
\begin{itemize}
\item {Grp. gram.:adj.}
\end{itemize}
\begin{itemize}
\item {Proveniência:(Do gr. \textunderscore derma\textunderscore  + \textunderscore eidos\textunderscore )}
\end{itemize}
Semelhante ao coiro ou á pelle.
\section{Dermatol}
\begin{itemize}
\item {Grp. gram.:m.}
\end{itemize}
\begin{itemize}
\item {Proveniência:(Do gr. \textunderscore derma\textunderscore )}
\end{itemize}
Pó medicamentoso, para enfermidades de pelle.
\section{Dermatologia}
\begin{itemize}
\item {Grp. gram.:f.}
\end{itemize}
\begin{itemize}
\item {Proveniência:(Do gr. \textunderscore derma\textunderscore  + \textunderscore logos\textunderscore )}
\end{itemize}
Tratado da pelle.
\section{Dermatológico}
\begin{itemize}
\item {Grp. gram.:adj.}
\end{itemize}
Relativo a dermatologia.
\section{Dermatoneurose}
\begin{itemize}
\item {Grp. gram.:f.}
\end{itemize}
\begin{itemize}
\item {Utilização:Med.}
\end{itemize}
\begin{itemize}
\item {Proveniência:(Do gr. \textunderscore derma\textunderscore  + \textunderscore neuron\textunderscore )}
\end{itemize}
Moléstia de pelle, de origem nervosa.
\section{Dermatopathia}
\begin{itemize}
\item {Grp. gram.:f.}
\end{itemize}
\begin{itemize}
\item {Proveniência:(Do gr. \textunderscore derma\textunderscore  + \textunderscore pathos\textunderscore )}
\end{itemize}
Designação genérica das moléstias da pelle.
\section{Dermatopatia}
\begin{itemize}
\item {Grp. gram.:f.}
\end{itemize}
\begin{itemize}
\item {Proveniência:(Do gr. \textunderscore derma\textunderscore  + \textunderscore pathos\textunderscore )}
\end{itemize}
Designação genérica das moléstias da pele.
\section{Dermatópodes}
\begin{itemize}
\item {Grp. gram.:f. pl.}
\end{itemize}
\begin{itemize}
\item {Proveniência:(Do gr. \textunderscore derma\textunderscore  + \textunderscore pous\textunderscore , \textunderscore podos\textunderscore )}
\end{itemize}
Classe de aves, que comprehende as que têm os pés cobertos de pelle coriácea e rugosa.
\section{Dermatose}
\begin{itemize}
\item {Grp. gram.:f.}
\end{itemize}
O mesmo que \textunderscore dermatopathia\textunderscore .
\section{Dermatotomia}
\begin{itemize}
\item {Grp. gram.:f.}
\end{itemize}
\begin{itemize}
\item {Proveniência:(Do gr. \textunderscore derma\textunderscore  + \textunderscore tome\textunderscore )}
\end{itemize}
Dissecção da pelle.
\section{Derme}
\begin{itemize}
\item {Grp. gram.:f.}
\end{itemize}
O mesmo que \textunderscore derma\textunderscore .
\section{Dermesto}
\begin{itemize}
\item {Grp. gram.:m.}
\end{itemize}
\begin{itemize}
\item {Proveniência:(Gr. \textunderscore dermestos\textunderscore )}
\end{itemize}
Insecto clavicórneo, nocivo ás carnes e ás pelles.
\section{Dérmico}
\begin{itemize}
\item {Grp. gram.:adj.}
\end{itemize}
Relativo á derma.
\section{Dermite}
\begin{itemize}
\item {Grp. gram.:f.}
\end{itemize}
O mesmo que \textunderscore dermatite\textunderscore .
\section{Dermoblasto}
\begin{itemize}
\item {Grp. gram.:m.}
\end{itemize}
\begin{itemize}
\item {Utilização:Bot.}
\end{itemize}
\begin{itemize}
\item {Proveniência:(Do gr. \textunderscore derma\textunderscore  + \textunderscore blassein\textunderscore )}
\end{itemize}
Embryão, cujo cotylédone é formado de uma membrana, que se rompe irregularmente, como Wildenow julga que se dá nos cogumelos.
\section{Dermodontes}
\begin{itemize}
\item {Grp. gram.:m. pl.}
\end{itemize}
\begin{itemize}
\item {Proveniência:(Do gr. \textunderscore derma\textunderscore  + \textunderscore odous\textunderscore )}
\end{itemize}
Peixes, que têm os dentes insertos fóra das maxillas.
\section{Dermogenia}
\begin{itemize}
\item {Grp. gram.:f.}
\end{itemize}
\begin{itemize}
\item {Proveniência:(Do gr. \textunderscore derma\textunderscore  + \textunderscore genos\textunderscore )}
\end{itemize}
Theoria da formação da pelle.
\section{Dermol}
\begin{itemize}
\item {Grp. gram.:m.}
\end{itemize}
Certo medicamento, para doenças de pelle.
\section{Dermolisia}
\begin{itemize}
\item {Grp. gram.:f.}
\end{itemize}
\begin{itemize}
\item {Proveniência:(Do gr. \textunderscore derma\textunderscore  + \textunderscore lusis\textunderscore )}
\end{itemize}
Paralisia da pele.
\section{Dermologia}
\begin{itemize}
\item {Grp. gram.:f.}
\end{itemize}
O mesmo que \textunderscore dermatologia\textunderscore .
\section{Dermolysia}
\begin{itemize}
\item {Grp. gram.:f.}
\end{itemize}
\begin{itemize}
\item {Proveniência:(Do gr. \textunderscore derma\textunderscore  + \textunderscore lusis\textunderscore )}
\end{itemize}
Paralysia da pelle.
\section{Dermóptero}
\begin{itemize}
\item {Grp. gram.:adj.}
\end{itemize}
\begin{itemize}
\item {Utilização:Zool.}
\end{itemize}
\begin{itemize}
\item {Proveniência:(Do gr. \textunderscore derma\textunderscore  + \textunderscore pteron\textunderscore )}
\end{itemize}
Que tem asas membranosas.
\section{Dermoterapismo}
\begin{itemize}
\item {Grp. gram.:m.}
\end{itemize}
\begin{itemize}
\item {Utilização:Med.}
\end{itemize}
\begin{itemize}
\item {Proveniência:(Do gr. \textunderscore derma\textunderscore  + \textunderscore therapeia\textunderscore )}
\end{itemize}
Tratamento de doenças internas, pela aplicação externa de productos vegetaes ou mineraes; iatralíptica.
\section{Dermotherapismo}
\begin{itemize}
\item {Grp. gram.:m.}
\end{itemize}
\begin{itemize}
\item {Utilização:Med.}
\end{itemize}
\begin{itemize}
\item {Proveniência:(Do gr. \textunderscore derma\textunderscore  + \textunderscore therapeia\textunderscore )}
\end{itemize}
Tratamento de doenças internas, pela applicação externa de productos vegetaes ou mineraes; iatralíptica.
\section{Deroga}
\begin{itemize}
\item {fónica:ró}
\end{itemize}
\begin{itemize}
\item {Grp. gram.:f.}
\end{itemize}
O mesmo que \textunderscore derogação\textunderscore . Cf. Filinto, II, 30.
\section{Derogação}
\begin{itemize}
\item {fónica:ro}
\end{itemize}
\begin{itemize}
\item {Grp. gram.:f.}
\end{itemize}
\begin{itemize}
\item {Proveniência:(Lat. \textunderscore derogatio\textunderscore )}
\end{itemize}
Acto de derogar.
\section{Derogador}
\begin{itemize}
\item {fónica:ro}
\end{itemize}
\begin{itemize}
\item {Grp. gram.:m.}
\end{itemize}
\begin{itemize}
\item {Proveniência:(Lat. \textunderscore derogator\textunderscore )}
\end{itemize}
Aquele que deroga.
\section{Derogamento}
\begin{itemize}
\item {fónica:ro}
\end{itemize}
\begin{itemize}
\item {Grp. gram.:m.}
\end{itemize}
Acto de derogar.
\section{Derogante}
\begin{itemize}
\item {fónica:ro}
\end{itemize}
\begin{itemize}
\item {Grp. gram.:adj.}
\end{itemize}
\begin{itemize}
\item {Proveniência:(Lat. \textunderscore derogans\textunderscore )}
\end{itemize}
Que deroga.
\section{Derogar}
\begin{itemize}
\item {fónica:ro}
\end{itemize}
\begin{itemize}
\item {Grp. gram.:v. t.}
\end{itemize}
\begin{itemize}
\item {Grp. gram.:V. i.}
\end{itemize}
\begin{itemize}
\item {Utilização:Des.}
\end{itemize}
\begin{itemize}
\item {Proveniência:(Lat. \textunderscore derogare\textunderscore )}
\end{itemize}
Annullar, abolir.
Substituir (preceitos legaes) por outros.
Causar derogação.
Produzir alteração essencial.
\section{Derogatório}
\begin{itemize}
\item {fónica:ro}
\end{itemize}
\begin{itemize}
\item {Grp. gram.:adj.}
\end{itemize}
\begin{itemize}
\item {Proveniência:(Lat. \textunderscore derogatorius\textunderscore )}
\end{itemize}
Que envolve derogação.
\section{De-romania}
\begin{itemize}
\item {Grp. gram.:loc. adv.}
\end{itemize}
\begin{itemize}
\item {Utilização:Ant.}
\end{itemize}
De repente, de chofre.
\section{Derrabar}
\begin{itemize}
\item {Grp. gram.:v. t.}
\end{itemize}
\begin{itemize}
\item {Utilização:Ant.}
\end{itemize}
Cortar o rabo ou cauda de.
Cortar as abas de (um vestuário).
Cortar a parte posterior de (um objecto).
Aprisionar ou inutilizar a retaguarda de (um exército). Cf. Barros, \textunderscore Dec\textunderscore . II, l. III, cap. II.
\section{Derradeiramente}
\begin{itemize}
\item {Grp. gram.:adv.}
\end{itemize}
\begin{itemize}
\item {Proveniência:(De \textunderscore derradeiro\textunderscore )}
\end{itemize}
Por fim.
\section{Derradeiro}
\begin{itemize}
\item {Grp. gram.:adj.}
\end{itemize}
\begin{itemize}
\item {Proveniência:(Do lat. \textunderscore de\textunderscore  + \textunderscore retro\textunderscore )}
\end{itemize}
Último; final.
Que vem atrás; que está depois.
\section{Derraigar}
\begin{itemize}
\item {Grp. gram.:v. t.}
\end{itemize}
\begin{itemize}
\item {Utilização:Prov.}
\end{itemize}
\begin{itemize}
\item {Utilização:trasm.}
\end{itemize}
Surribar, decroar.
(Cp. \textunderscore desarraigar\textunderscore )
\section{Derrama}
\begin{itemize}
\item {Grp. gram.:f.}
\end{itemize}
\begin{itemize}
\item {Proveniência:(Do cast. \textunderscore garrama\textunderscore ? ou de \textunderscore derramar\textunderscore ?)}
\end{itemize}
Tributo local, proporcionado aos rendimentos de cada contribuinte.
\section{Derramação}
\begin{itemize}
\item {Grp. gram.:f.}
\end{itemize}
Acto de derramar.
O mesmo que \textunderscore derramamento\textunderscore .
\section{Derramadamente}
\begin{itemize}
\item {Grp. gram.:adv.}
\end{itemize}
De modo derramado.
Com derramamento.
\section{Derramado}
\begin{itemize}
\item {Grp. gram.:adj.}
\end{itemize}
\begin{itemize}
\item {Utilização:Pop.}
\end{itemize}
\begin{itemize}
\item {Proveniência:(De \textunderscore derramar\textunderscore )}
\end{itemize}
Enfurecido, raivoso.
Hydróphobo: \textunderscore um cão derramado\textunderscore .
\section{Derramador}
\begin{itemize}
\item {Grp. gram.:m.}
\end{itemize}
Aquelle que derrama.
\section{Derramamento}
\begin{itemize}
\item {Grp. gram.:m.}
\end{itemize}
Acto ou effeito de derramar.
Hydrophobia.
\section{Derramar}
\begin{itemize}
\item {Grp. gram.:v. t.}
\end{itemize}
\begin{itemize}
\item {Grp. gram.:V. p.}
\end{itemize}
\begin{itemize}
\item {Proveniência:(De \textunderscore ramo\textunderscore )}
\end{itemize}
Aparar, cortar, os ramos de; desarmar.
Diffundir, espalhar: \textunderscore derramar notícias falsas\textunderscore .
Entornar: \textunderscore derramar uma bilha de azeite\textunderscore .
Distribuir (um imposto).
Prodigalizar: \textunderscore derramar auxílios\textunderscore .
Produzir.
Vulgarizar: \textunderscore derramar ideias úteis\textunderscore .
Sêr derramado.
Tornar-se hydróphobo.--Com a accepção de \textunderscore distribuir\textunderscore  (um imposto), seria voc. distinto, der. de \textunderscore derrama\textunderscore , se êste subst. se relacionasse mais com o cast. \textunderscore garrama\textunderscore , do que com o nosso clássico \textunderscore derramar\textunderscore , como me inclino a crêr.
\section{Derrame}
\begin{itemize}
\item {Grp. gram.:m.}
\end{itemize}
O mesmo que \textunderscore derramamento\textunderscore .
\section{Derrancado}
\begin{itemize}
\item {Grp. gram.:adj.}
\end{itemize}
\begin{itemize}
\item {Utilização:Prov.}
\end{itemize}
\begin{itemize}
\item {Utilização:beir.}
\end{itemize}
\begin{itemize}
\item {Proveniência:(De \textunderscore derrancar\textunderscore )}
\end{itemize}
Estragado: corrompido: \textunderscore vinho derrancado\textunderscore .
Hydróphobo: \textunderscore um cão derrancado\textunderscore .
\section{Derrancamento}
\begin{itemize}
\item {Grp. gram.:m.}
\end{itemize}
Acto de derrancar.
\section{Derrancar}
\begin{itemize}
\item {Grp. gram.:v. t.}
\end{itemize}
\begin{itemize}
\item {Utilização:Des.}
\end{itemize}
\begin{itemize}
\item {Grp. gram.:V. p.}
\end{itemize}
\begin{itemize}
\item {Utilização:Pop.}
\end{itemize}
\begin{itemize}
\item {Proveniência:(Do lat. \textunderscore rancus?\textunderscore )}
\end{itemize}
Estragar.
Corromper.
Tornar rançoso.
Irritar; encolerizar.
Arrancar.
Tornar-se hydróphobo.
\section{Derrancar}
\begin{itemize}
\item {Grp. gram.:v. t.}
\end{itemize}
\begin{itemize}
\item {Utilização:Prov.}
\end{itemize}
Partir a perna de; tornar manco. (Colhido na Bairrada)
(Alter. de \textunderscore derrengar\textunderscore ?)
\section{Derranco}
\begin{itemize}
\item {Grp. gram.:m.}
\end{itemize}
O mesmo que \textunderscore derrancamento\textunderscore .
\section{Derrangadeira}
\begin{itemize}
\item {Grp. gram.:f.}
\end{itemize}
\begin{itemize}
\item {Utilização:Prov.}
\end{itemize}
\begin{itemize}
\item {Utilização:alg.}
\end{itemize}
O mesmo que \textunderscore mó\textunderscore ^1.
\section{Derreaço}
\begin{itemize}
\item {Grp. gram.:m.}
\end{itemize}
\begin{itemize}
\item {Utilização:Prov.}
\end{itemize}
\begin{itemize}
\item {Utilização:trasm.}
\end{itemize}
\begin{itemize}
\item {Proveniência:(De \textunderscore derrear\textunderscore )}
\end{itemize}
Estado de pessôa derreada; cansaço.
\section{Derreadela}
\begin{itemize}
\item {Grp. gram.:f.}
\end{itemize}
\begin{itemize}
\item {Utilização:Prov.}
\end{itemize}
\begin{itemize}
\item {Utilização:trasm.}
\end{itemize}
O mesmo que \textunderscore derreaço\textunderscore .
\section{Derreador}
\begin{itemize}
\item {Grp. gram.:m.}
\end{itemize}
Aquelle que derreia.
\section{Derreamento}
\begin{itemize}
\item {Grp. gram.:m.}
\end{itemize}
Effeito de derrear.
\section{Derrear}
\begin{itemize}
\item {Grp. gram.:v. t.}
\end{itemize}
\begin{itemize}
\item {Utilização:Fig.}
\end{itemize}
Fazer curvar as costas de, com pancadas ou com pêso.
Alquebrar.
Curvar por effeito da velhice.
Extenuar.
Desacreditar.
\section{Derredor}
\begin{itemize}
\item {Grp. gram.:adv.}
\end{itemize}
\begin{itemize}
\item {Grp. gram.:Loc. prep.}
\end{itemize}
\begin{itemize}
\item {Grp. gram.:M.}
\end{itemize}
\begin{itemize}
\item {Utilização:Des.}
\end{itemize}
\begin{itemize}
\item {Proveniência:(De \textunderscore redor\textunderscore )}
\end{itemize}
Em roda, á volta.
Em \textunderscore derredor de\textunderscore , em volta de.
Circuito, róda.
\section{De-redor}
\begin{itemize}
\item {Grp. gram.:adv.}
\end{itemize}
\begin{itemize}
\item {Grp. gram.:Loc. prep.}
\end{itemize}
\begin{itemize}
\item {Grp. gram.:M.}
\end{itemize}
\begin{itemize}
\item {Utilização:Des.}
\end{itemize}
\begin{itemize}
\item {Proveniência:(De \textunderscore redor\textunderscore )}
\end{itemize}
Em roda, á volta.
Em \textunderscore derredor de\textunderscore , em volta de.
Circuito, róda.
\section{Derreeira}
\begin{itemize}
\item {Grp. gram.:f.}
\end{itemize}
\begin{itemize}
\item {Utilização:Prov.}
\end{itemize}
O mesmo que \textunderscore derreamento\textunderscore .
Prostração de fôrças.
(Cp. \textunderscore derrear\textunderscore )
\section{Derregador}
\begin{itemize}
\item {Grp. gram.:m.}
\end{itemize}
\begin{itemize}
\item {Utilização:Agr.}
\end{itemize}
\begin{itemize}
\item {Proveniência:(De \textunderscore derregar\textunderscore )}
\end{itemize}
Utensílio, para abrir os rêgos, por onde se rega.
\section{Derregar}
\begin{itemize}
\item {Grp. gram.:v. t.}
\end{itemize}
Abrir novos regos em (uma terra), para receberem a água pluvial e desviarem-na da mesma terra.
\section{Derreigar}
\textunderscore v. t.\textunderscore  (e der.)
O mesmo que \textunderscore derraigar\textunderscore .
\section{Derrengar}
\begin{itemize}
\item {Grp. gram.:v. t.}
\end{itemize}
\begin{itemize}
\item {Grp. gram.:V. p.}
\end{itemize}
\begin{itemize}
\item {Utilização:Bras. do N}
\end{itemize}
\begin{itemize}
\item {Proveniência:(T. cast.)}
\end{itemize}
(V.derrear)
Requebrar-se, fazer trejeitos affectados.
Curvar excessivamente a columna vertebral do (gado cavallar), por demasía de carga.
\section{Derrengo}
\begin{itemize}
\item {Grp. gram.:m.}
\end{itemize}
O mesmo que \textunderscore derrengue\textunderscore . Cf. Filinto, V, 147.
\section{Derrengue}
\begin{itemize}
\item {Grp. gram.:m.}
\end{itemize}
Effeito de derrengar.
Requebro de corpo.
Trejeito dengue:«\textunderscore ...saltar com tregeitos desenvoltos, num derrengue arregaçado\textunderscore ». Camillo, \textunderscore Vinho do Porto\textunderscore , 80.
\section{Derrête}
\begin{itemize}
\item {Grp. gram.:m.}
\end{itemize}
\begin{itemize}
\item {Utilização:T. de Sintra}
\end{itemize}
\begin{itemize}
\item {Proveniência:(De \textunderscore derreter\textunderscore )}
\end{itemize}
Namôro, galanteio.
\section{Derretedura}
\begin{itemize}
\item {Grp. gram.:f.}
\end{itemize}
Acto ou effeito de derreter.
\section{Derreter}
\begin{itemize}
\item {Grp. gram.:v. t.}
\end{itemize}
\begin{itemize}
\item {Utilização:Fig.}
\end{itemize}
\begin{itemize}
\item {Proveniência:(De \textunderscore de\textunderscore  + \textunderscore reter\textunderscore ?)}
\end{itemize}
Tornar líquido: \textunderscore derreter gêlo\textunderscore .
Fundir: \textunderscore derreter metaes\textunderscore .
Commover.
Consumir; malbaratar: \textunderscore derreteu quanto tinha\textunderscore .
\section{Derretimento}
\begin{itemize}
\item {Grp. gram.:m.}
\end{itemize}
O mesmo que \textunderscore derretedura\textunderscore .
\section{Derribador}
\begin{itemize}
\item {Grp. gram.:adj.}
\end{itemize}
\begin{itemize}
\item {Grp. gram.:M.}
\end{itemize}
Que derriba.
Aquelle que derriba.
\section{Derribamento}
\begin{itemize}
\item {Grp. gram.:m.}
\end{itemize}
Acto de derribar.
\section{Derribar}
\begin{itemize}
\item {Grp. gram.:v. t.}
\end{itemize}
\begin{itemize}
\item {Proveniência:(De \textunderscore riba\textunderscore )}
\end{itemize}
Tirar de riba.
Abater: \textunderscore derribar árvores\textunderscore .
Inclinar.
Deitar abaixo.
Despenhar.
Demolir: \textunderscore derribar um muro\textunderscore .
Prostrar: \textunderscore derribar o contendor\textunderscore .
Aniquilar.
Exonerar, destituir.
\section{Derriça}
\begin{itemize}
\item {Grp. gram.:f.}
\end{itemize}
\begin{itemize}
\item {Utilização:Pop.}
\end{itemize}
Acto de derriçar.
Contenda.
\section{Derriçador}
\begin{itemize}
\item {Grp. gram.:m.}
\end{itemize}
\begin{itemize}
\item {Utilização:Pop.}
\end{itemize}
Namorador.
\section{Derriçar}
\begin{itemize}
\item {Grp. gram.:v. t.}
\end{itemize}
\begin{itemize}
\item {Grp. gram.:V. i.}
\end{itemize}
\begin{itemize}
\item {Utilização:Fig.}
\end{itemize}
\begin{itemize}
\item {Utilização:Pop.}
\end{itemize}
\begin{itemize}
\item {Proveniência:(De \textunderscore riço\textunderscore )}
\end{itemize}
Desenriçar.
Destramar.
Dirigir motejos.
Troçar.
Contender.
Namorar.
\section{Derriço}
\begin{itemize}
\item {Grp. gram.:m.}
\end{itemize}
\begin{itemize}
\item {Utilização:Pop.}
\end{itemize}
\begin{itemize}
\item {Proveniência:(De \textunderscore derriçar\textunderscore )}
\end{itemize}
Namôro.
\section{Derrisão}
\begin{itemize}
\item {Grp. gram.:f.}
\end{itemize}
\begin{itemize}
\item {Proveniência:(Lat. \textunderscore derisio\textunderscore )}
\end{itemize}
Riso de desprêzo.
Escárneo.
\section{Derrisca}
\begin{itemize}
\item {Grp. gram.:f.}
\end{itemize}
Desobriga.
Acto de \textunderscore derriscar\textunderscore .
\section{Derriscar}
\begin{itemize}
\item {Grp. gram.:v. t.}
\end{itemize}
\begin{itemize}
\item {Proveniência:(De \textunderscore riscar\textunderscore )}
\end{itemize}
\textunderscore v. t.\textunderscore  (e der.)
O mesmo que \textunderscore deriscar\textunderscore , etc.
Riscar (o nome no rol dos que se hão de confessar).
\section{Derrisor}
\begin{itemize}
\item {Grp. gram.:m.}
\end{itemize}
\begin{itemize}
\item {Proveniência:(Lat. \textunderscore derisor\textunderscore )}
\end{itemize}
Aquelle que moteja ou escarnece.
\section{Derrisoriamente}
\begin{itemize}
\item {Grp. gram.:adv.}
\end{itemize}
De modo derrisório.
\section{Derrisório}
\begin{itemize}
\item {Grp. gram.:adj.}
\end{itemize}
\begin{itemize}
\item {Proveniência:(Lat. \textunderscore derisorius\textunderscore )}
\end{itemize}
Em que há derrisão.
\section{Derrocada}
\begin{itemize}
\item {Grp. gram.:f.}
\end{itemize}
O mesmo que \textunderscore derrocamento\textunderscore .
\section{Derrocador}
\begin{itemize}
\item {Grp. gram.:adj.}
\end{itemize}
\begin{itemize}
\item {Grp. gram.:M.}
\end{itemize}
Que derroca.
Aquelle que derroca.
\section{Derrocamento}
\begin{itemize}
\item {Grp. gram.:m.}
\end{itemize}
Acto de derrocar.
\section{Derrocar}
\begin{itemize}
\item {Grp. gram.:v. t.}
\end{itemize}
\begin{itemize}
\item {Proveniência:(De \textunderscore roca\textunderscore ^2?)}
\end{itemize}
Derribar.
Destruir.
Arrasar.
Humilhar.
\section{Derroga}
\begin{itemize}
\item {Grp. gram.:f.}
\end{itemize}
O mesmo que \textunderscore derrogação\textunderscore . Cf. Filinto, II, 30.
\section{Derrogação}
\begin{itemize}
\item {Grp. gram.:f.}
\end{itemize}
\begin{itemize}
\item {Proveniência:(Lat. \textunderscore derogatio\textunderscore )}
\end{itemize}
Acto de derrogar.
\section{Derrogador}
\begin{itemize}
\item {Grp. gram.:m.}
\end{itemize}
\begin{itemize}
\item {Proveniência:(Lat. \textunderscore derogator\textunderscore )}
\end{itemize}
Aquelle que derroga.
\section{Derrogamento}
\begin{itemize}
\item {Grp. gram.:m.}
\end{itemize}
Acto de derrogar.
\section{Derrogante}
\begin{itemize}
\item {Grp. gram.:adj.}
\end{itemize}
\begin{itemize}
\item {Proveniência:(Lat. \textunderscore derogans\textunderscore )}
\end{itemize}
Que derroga.
\section{Derrogar}
\begin{itemize}
\item {Grp. gram.:v. t.}
\end{itemize}
\begin{itemize}
\item {Grp. gram.:V. i.}
\end{itemize}
\begin{itemize}
\item {Utilização:Des.}
\end{itemize}
\begin{itemize}
\item {Proveniência:(Lat. \textunderscore derogare\textunderscore )}
\end{itemize}
Annullar, abolir.
Substituir (preceitos legaes) por outros.
Causar derrogação.
Produzir alteração essencial.
\section{Derrogatório}
\begin{itemize}
\item {Grp. gram.:adj.}
\end{itemize}
\begin{itemize}
\item {Proveniência:(Lat. \textunderscore derogatorius\textunderscore )}
\end{itemize}
Que envolve derrogação.
\section{Derronchar}
\begin{itemize}
\item {Grp. gram.:v. t.}
\end{itemize}
\begin{itemize}
\item {Utilização:Prov.}
\end{itemize}
\begin{itemize}
\item {Utilização:trasm.}
\end{itemize}
Derrubar, deitar por terra.
\section{Derrota}
\begin{itemize}
\item {Grp. gram.:f.}
\end{itemize}
\begin{itemize}
\item {Utilização:Ext.}
\end{itemize}
\begin{itemize}
\item {Proveniência:(De \textunderscore rota\textunderscore )}
\end{itemize}
Frota.
Rumo das embarcações.
Caminho.
Roteiro: \textunderscore seguir a derrota da India\textunderscore .
Viagem; percurso.
Modo de proceder.
\section{Derrota}
\begin{itemize}
\item {Grp. gram.:f.}
\end{itemize}
Acto ou effeito de derrotar^2.
\section{Derrotado}
\begin{itemize}
\item {Grp. gram.:adj.}
\end{itemize}
\begin{itemize}
\item {Utilização:Des.}
\end{itemize}
\begin{itemize}
\item {Proveniência:(De \textunderscore derrotar\textunderscore ^1)}
\end{itemize}
Que se afastou da rota (falando-se do navio).
\section{Derrotador}
\begin{itemize}
\item {Grp. gram.:adj.}
\end{itemize}
\begin{itemize}
\item {Grp. gram.:M.}
\end{itemize}
Que derrota.
Aquelle que derrota.
\section{Derrotar}
\begin{itemize}
\item {Grp. gram.:v. t.}
\end{itemize}
\begin{itemize}
\item {Grp. gram.:V. i.}
\end{itemize}
\begin{itemize}
\item {Proveniência:(De \textunderscore derrota\textunderscore ^1)}
\end{itemize}
Afastar da rota.
Desviar-se do rumo que se seguia.
Desencaminhar-se.
\section{Derrotar}
\begin{itemize}
\item {Grp. gram.:v. t.}
\end{itemize}
\begin{itemize}
\item {Proveniência:(Do lat. \textunderscore diruptus\textunderscore )}
\end{itemize}
Vencer.
Desbaratar: \textunderscore derrotar um exército\textunderscore .
Destruir.
Cansar.
\section{Derrote}
\begin{itemize}
\item {Grp. gram.:m.}
\end{itemize}
\begin{itemize}
\item {Utilização:Taur.}
\end{itemize}
\begin{itemize}
\item {Utilização:Pop.}
\end{itemize}
\begin{itemize}
\item {Proveniência:(De \textunderscore derrotar\textunderscore ^2)}
\end{itemize}
Acto, em que o toiro levanta a cabeça, depois de a têr baixado para marrar.
Acto ou effeito de derrubar árvores.
Destruição de grande parte de uma plantação.
\section{Derroteiro}
\begin{itemize}
\item {Grp. gram.:m.}
\end{itemize}
\begin{itemize}
\item {Proveniência:(De \textunderscore derrota\textunderscore ^1)}
\end{itemize}
O mesmo que \textunderscore roteiro\textunderscore ^1.
\section{Derrôto}
\begin{itemize}
\item {Grp. gram.:adj.}
\end{itemize}
\begin{itemize}
\item {Utilização:Ant.}
\end{itemize}
\begin{itemize}
\item {Proveniência:(De \textunderscore derrotar\textunderscore )}
\end{itemize}
Destruído; desbaratado.
\section{Derruba}
\begin{itemize}
\item {Grp. gram.:f.}
\end{itemize}
\begin{itemize}
\item {Utilização:Bras}
\end{itemize}
O mesmo que \textunderscore carpa\textunderscore ^2.
\section{Derrubada}
\begin{itemize}
\item {Grp. gram.:f.}
\end{itemize}
\begin{itemize}
\item {Utilização:Bras}
\end{itemize}
\begin{itemize}
\item {Proveniência:(De \textunderscore derrubar\textunderscore )}
\end{itemize}
Acto de abater grandes árvores, com o fim de preparar o terreno para plantações. Cf. \textunderscore Vocabulário de Estradas\textunderscore .
\section{Derrubador}
\begin{itemize}
\item {Grp. gram.:m.}
\end{itemize}
\begin{itemize}
\item {Utilização:Gír.}
\end{itemize}
Fadista.
\section{Derrubamento}
\begin{itemize}
\item {Grp. gram.:m.}
\end{itemize}
Acto de derrubar.
\section{Derrubar}
\begin{itemize}
\item {Grp. gram.:v. t.}
\end{itemize}
O mesmo que \textunderscore derribar\textunderscore .
\section{Derruimento}
\begin{itemize}
\item {fónica:ru-i}
\end{itemize}
\begin{itemize}
\item {Grp. gram.:m.}
\end{itemize}
Acto ou effeito de derruir.
\section{Derruir}
\begin{itemize}
\item {Grp. gram.:v. t.}
\end{itemize}
\begin{itemize}
\item {Proveniência:(Lat. \textunderscore deruere\textunderscore )}
\end{itemize}
O mesmo que \textunderscore derribar\textunderscore .
\section{Derviche}
\begin{itemize}
\item {Grp. gram.:m.}
\end{itemize}
(Expressão francesa, correspondente ao português \textunderscore daroês\textunderscore . V. \textunderscore daroês\textunderscore )
\section{Dês}
\begin{itemize}
\item {Grp. gram.:prep.}
\end{itemize}
O mesmo que \textunderscore desde\textunderscore .
\section{Des...}
\begin{itemize}
\item {Grp. gram.:pref.}
\end{itemize}
O mesmo que \textunderscore de...\textunderscore  e que \textunderscore es...\textunderscore  ou \textunderscore ex...\textunderscore 
\section{Desabafar}
\begin{itemize}
\item {Grp. gram.:v. t.}
\end{itemize}
\begin{itemize}
\item {Grp. gram.:V. i.}
\end{itemize}
\begin{itemize}
\item {Proveniência:(De \textunderscore des...\textunderscore  + \textunderscore abafar\textunderscore )}
\end{itemize}
Descobrir.
Desagasalhar.
Desafrontar.
Tornar livre (a respiração).
Dizer com franqueza: \textunderscore desabafar resentimentos\textunderscore .
Expandir.
Reanimar.
Respirar livremente.
Expandir-se.
\section{Desabafo}
\begin{itemize}
\item {Grp. gram.:m.}
\end{itemize}
Acto ou effeito de desabafar.
\section{Desabaladamente}
\begin{itemize}
\item {Grp. gram.:adv.}
\end{itemize}
De modo desabalado.
\section{Desabalado}
\begin{itemize}
\item {Grp. gram.:adj.}
\end{itemize}
\begin{itemize}
\item {Utilização:Pop.}
\end{itemize}
\begin{itemize}
\item {Proveniência:(De \textunderscore des...\textunderscore  + \textunderscore abalado\textunderscore )}
\end{itemize}
Excessivo: \textunderscore rancor desabalado\textunderscore .
Immenso, desmedido.
Precipitado: \textunderscore corrida desabalada\textunderscore .
\section{Desabaldar}
\begin{itemize}
\item {Grp. gram.:v. t.}
\end{itemize}
(?):«\textunderscore ...esteve por desabaldar o mulato\textunderscore  (mu)». Soropita, Prosas, p. 102.
\section{Desabalroar}
\begin{itemize}
\item {Grp. gram.:v. t.}
\end{itemize}
\begin{itemize}
\item {Proveniência:(De \textunderscore des...\textunderscore  + \textunderscore abalroar\textunderscore )}
\end{itemize}
Desatracar.
\section{Desabamento}
\begin{itemize}
\item {Grp. gram.:m.}
\end{itemize}
Acto ou effeito de desabar.
\section{Desabar}
\begin{itemize}
\item {Grp. gram.:v. t.}
\end{itemize}
\begin{itemize}
\item {Grp. gram.:V. i.}
\end{itemize}
\begin{itemize}
\item {Proveniência:(De \textunderscore des...\textunderscore  + \textunderscore aba\textunderscore )}
\end{itemize}
Abaixar a aba de: \textunderscore desabar o chapéu\textunderscore .
Deitar abaixo, arruinar:«\textunderscore receio muito que as superstições venham a desabar o Catholicismo\textunderscore ». Camillo, \textunderscore Brasileira\textunderscore , 357.
Desmoronar-se.
Abater-se, caír: \textunderscore o telhado desabou\textunderscore .
\section{Desabastado}
\begin{itemize}
\item {Grp. gram.:adj.}
\end{itemize}
\begin{itemize}
\item {Proveniência:(De \textunderscore des...\textunderscore  + \textunderscore abastado\textunderscore )}
\end{itemize}
Que não é abastado; que perdeu a abastança. Cf. Filinto, XVIII, 195.
\section{Desabe}
\begin{itemize}
\item {Grp. gram.:m.}
\end{itemize}
\begin{itemize}
\item {Proveniência:(De \textunderscore desabar\textunderscore )}
\end{itemize}
Desabamento.
Parte desabada de uma construcção.
\section{Desabelhar}
\begin{itemize}
\item {Grp. gram.:v. i.}
\end{itemize}
\begin{itemize}
\item {Proveniência:(De \textunderscore des...\textunderscore  + \textunderscore abelha\textunderscore )}
\end{itemize}
Debandar, partir em bandos, como um enxame de abelhas:«\textunderscore ...a gente foge, desabelha com calor.\textunderscore »J. de Deus, \textunderscore Flores do Campo\textunderscore , 2.^a ed., 154.
\section{Desabitar}
\begin{itemize}
\item {Grp. gram.:v. i.}
\end{itemize}
\begin{itemize}
\item {Utilização:Náut.}
\end{itemize}
\begin{itemize}
\item {Proveniência:(De \textunderscore abita\textunderscore )}
\end{itemize}
Tirar as voltas que a amarra tem na abita.
\section{Desaboçar}
\begin{itemize}
\item {Grp. gram.:v. t.}
\end{itemize}
Tirar as boças a.
\section{Desabonadamente}
\begin{itemize}
\item {Grp. gram.:adv.}
\end{itemize}
Com desabono.
Desfavoravelmente.
\section{Desabonador}
\begin{itemize}
\item {Grp. gram.:adj.}
\end{itemize}
\begin{itemize}
\item {Grp. gram.:M.}
\end{itemize}
Que desabona.
Aquelle que desabona.
\section{Desabonar}
\begin{itemize}
\item {Grp. gram.:v. t.}
\end{itemize}
\begin{itemize}
\item {Proveniência:(De \textunderscore des...\textunderscore  + \textunderscore abonar\textunderscore )}
\end{itemize}
Desacreditar.
Depreciar.
\section{Desabono}
\begin{itemize}
\item {Grp. gram.:m.}
\end{itemize}
Acto ou effeito de desabonar.
Desfavor.
Depreciação: \textunderscore falar em desabono de alguém\textunderscore .
\section{Desaborcar}
\textunderscore v. t.\textunderscore  (e der.)
O mesmo que \textunderscore desemborcar\textunderscore , etc. Cf. Filinto, XVIII, 206.
\section{Desabordamento}
\begin{itemize}
\item {Grp. gram.:m.}
\end{itemize}
Acto de desabordar.
\section{Desabordar}
\begin{itemize}
\item {Grp. gram.:v. t.}
\end{itemize}
\begin{itemize}
\item {Grp. gram.:V. i.}
\end{itemize}
\begin{itemize}
\item {Utilização:Náut.}
\end{itemize}
\begin{itemize}
\item {Proveniência:(De \textunderscore des...\textunderscore  + \textunderscore abordar\textunderscore )}
\end{itemize}
Separar (um navio) de outro, a que estava abordado.
Separar-se de bordo, largar a nau, que se tinha abordado:«\textunderscore não acabárão de desabordar das nossas naos, quando...\textunderscore »Filinto, \textunderscore D. Man.\textunderscore , II, 101.
\section{Desabotinado}
\begin{itemize}
\item {Grp. gram.:adj.}
\end{itemize}
\begin{itemize}
\item {Utilização:Bras. de Minas}
\end{itemize}
\begin{itemize}
\item {Proveniência:(De \textunderscore botina\textunderscore )}
\end{itemize}
Ligeiro.
Valente.
Amalucado.
\section{Desabotoadura}
\begin{itemize}
\item {Grp. gram.:f.}
\end{itemize}
O mesmo que \textunderscore desabotoamento\textunderscore .
\section{Desabotoamento}
\begin{itemize}
\item {Grp. gram.:m.}
\end{itemize}
Acto ou effeito de desabotoar.
\section{Desabotoar}
\begin{itemize}
\item {Grp. gram.:v. t.}
\end{itemize}
\begin{itemize}
\item {Grp. gram.:V. i.}
\end{itemize}
\begin{itemize}
\item {Proveniência:(De \textunderscore des...\textunderscore  + \textunderscore abotoar\textunderscore )}
\end{itemize}
Tirar da casa os botões de.
Abrir ou desapertar, desabotoando: \textunderscore desabotoar o casaco\textunderscore .
Abrir os botões, desabrochar, (falando-se de flôres ou rebentos).
\section{Desabraçar}
\begin{itemize}
\item {Grp. gram.:v. t.}
\end{itemize}
\begin{itemize}
\item {Proveniência:(De \textunderscore des...\textunderscore  + \textunderscore abraçar\textunderscore )}
\end{itemize}
Desprender dos braços (o que estava abraçado).
\section{Desabragalar}
\begin{itemize}
\item {Grp. gram.:v. t.}
\end{itemize}
\begin{itemize}
\item {Utilização:Prov.}
\end{itemize}
\begin{itemize}
\item {Utilização:trasm.}
\end{itemize}
Abrir a braguilha de.
Desabotoar.
(Por \textunderscore desbraguilhar\textunderscore , de \textunderscore des...\textunderscore  + \textunderscore braguilha\textunderscore )
\section{Desabridamente}
\begin{itemize}
\item {Grp. gram.:adv.}
\end{itemize}
De modo desabrido.
\section{Desabrido}
\begin{itemize}
\item {Grp. gram.:adj.}
\end{itemize}
\begin{itemize}
\item {Proveniência:(De \textunderscore desabrir\textunderscore )}
\end{itemize}
Rude; insolente: \textunderscore modo desabrido\textunderscore .
Áspero, violento: \textunderscore vento desabrido\textunderscore .
\section{Desabrigadamente}
\begin{itemize}
\item {Grp. gram.:adv.}
\end{itemize}
Sem abrigo.
\section{Desabrigar}
\begin{itemize}
\item {Grp. gram.:v. t.}
\end{itemize}
\begin{itemize}
\item {Proveniência:(De \textunderscore des...\textunderscore  + \textunderscore abrigar\textunderscore )}
\end{itemize}
Tirar o abrigo a.
Abandonar.
\section{Desabrigo}
\begin{itemize}
\item {Grp. gram.:m.}
\end{itemize}
Falta de abrigo.
Effeito de desabrigar.
\section{Desabrigoso}
\begin{itemize}
\item {Grp. gram.:adj.}
\end{itemize}
\begin{itemize}
\item {Utilização:Prov.}
\end{itemize}
\begin{itemize}
\item {Utilização:alg.}
\end{itemize}
\begin{itemize}
\item {Proveniência:(De \textunderscore des...\textunderscore  + \textunderscore abrigoso\textunderscore )}
\end{itemize}
Que não abriga.
\section{Desabrimento}
\begin{itemize}
\item {Grp. gram.:m.}
\end{itemize}
\begin{itemize}
\item {Proveniência:(De \textunderscore desabrir\textunderscore )}
\end{itemize}
Estado ou qualidade de quem ou daquillo que é desabrido.
\section{Desabrir}
\begin{itemize}
\item {Grp. gram.:v. t.}
\end{itemize}
\begin{itemize}
\item {Utilização:Des.}
\end{itemize}
\begin{itemize}
\item {Grp. gram.:V. p.}
\end{itemize}
\begin{itemize}
\item {Proveniência:(De \textunderscore des...\textunderscore  + \textunderscore abrir\textunderscore )}
\end{itemize}
Abrir mão de.
Abandonar.
Irritar-se.
Malquistar-se com alguém.
\section{Desabrochamento}
\begin{itemize}
\item {Grp. gram.:m.}
\end{itemize}
Acto ou effeito de desabrochar.
\section{Desabrochar}
\begin{itemize}
\item {Grp. gram.:v. t.}
\end{itemize}
\begin{itemize}
\item {Grp. gram.:V. i.}
\end{itemize}
\begin{itemize}
\item {Proveniência:(De \textunderscore des...\textunderscore  + \textunderscore abrochar\textunderscore )}
\end{itemize}
Desapertar (o que estava preso com broche, colchete, atacador, botões, etc.).
Desabotoar.
Abrir, desprender.
Descerrar.
Abrir-se, (falando-se de flôres).
\section{Desabrolhar}
\begin{itemize}
\item {Grp. gram.:v. i.}
\end{itemize}
\begin{itemize}
\item {Proveniência:(De \textunderscore abrolhar\textunderscore )}
\end{itemize}
Abrolhar, brotar.
Germinar; desabrochar.
\section{Desabusadamente}
\begin{itemize}
\item {Grp. gram.:adv.}
\end{itemize}
De modo desabusado.
\section{Desabusado}
\begin{itemize}
\item {Grp. gram.:adj.}
\end{itemize}
Petulante; inconveniente.
Atrevido.
Isento de abusões ou preconceitos; sensato:«\textunderscore ...o público desabusado e imparcial\textunderscore ». S. R. de Viterbo, \textunderscore Diccion. Portát.\textunderscore , (no prólogo).
\section{Desabusar}
\begin{itemize}
\item {Grp. gram.:v. t.}
\end{itemize}
\begin{itemize}
\item {Proveniência:(De \textunderscore des...\textunderscore  + \textunderscore abusar\textunderscore )}
\end{itemize}
Livrar de abusões.
Desenganar, desilludir.
\section{Desabuso}
\begin{itemize}
\item {Grp. gram.:m.}
\end{itemize}
Acto ou effeito de desabusar. Cf. \textunderscore Parnaso Lusitano\textunderscore , V, 188.
\section{Desaçaimar}
\textunderscore v. t.\textunderscore  (e der.)
O mesmo que \textunderscore desaçamar\textunderscore , etc. Cf. Castilho, \textunderscore Palavras de um Crente\textunderscore , 103.
\section{Desacamar}
\begin{itemize}
\item {Grp. gram.:v. t.}
\end{itemize}
\begin{itemize}
\item {Proveniência:(De \textunderscore des...\textunderscore  + \textunderscore acamar\textunderscore )}
\end{itemize}
Fazer que não esteja acamado.
Desacommodar. Cf. A. Candido, \textunderscore Or. Fún. de A. Herc.\textunderscore , 14.
\section{Desaçamar}
\begin{itemize}
\item {Grp. gram.:v. t.}
\end{itemize}
\begin{itemize}
\item {Proveniência:(De \textunderscore des...\textunderscore  + \textunderscore açamar\textunderscore )}
\end{itemize}
Tirar o açamo a.
\section{Desacampar}
\begin{itemize}
\item {Grp. gram.:v. i.}
\end{itemize}
\begin{itemize}
\item {Proveniência:(De \textunderscore des...\textunderscore  + \textunderscore acampar\textunderscore )}
\end{itemize}
Levantar arraial.
Deixar de estar acampado. Cf. Camillo, \textunderscore Maria da Fonte\textunderscore , 147.
\section{Desacanhar}
\begin{itemize}
\item {Grp. gram.:v. t.}
\end{itemize}
\begin{itemize}
\item {Proveniência:(De \textunderscore des...\textunderscore  + \textunderscore acanhar\textunderscore )}
\end{itemize}
Livrar de acanhamento.
\section{Desacasalar}
\begin{itemize}
\item {Grp. gram.:v. t.}
\end{itemize}
\begin{itemize}
\item {Proveniência:(De \textunderscore des...\textunderscore  + \textunderscore acasalar\textunderscore )}
\end{itemize}
Separar (os que estavam acasalados).
\section{Desacatadamente}
\begin{itemize}
\item {Grp. gram.:adv.}
\end{itemize}
Com desacato.
\section{Desacatamento}
\begin{itemize}
\item {Grp. gram.:m.}
\end{itemize}
Acto de desacatar.
\section{Desacatar}
\begin{itemize}
\item {Grp. gram.:v. t.}
\end{itemize}
\begin{itemize}
\item {Proveniência:(De \textunderscore des...\textunderscore  + \textunderscore acatar\textunderscore )}
\end{itemize}
Faltar ao respeito que se deve a.
Desrespeitar.
Afrontar.
\section{Desacato}
\begin{itemize}
\item {Grp. gram.:m.}
\end{itemize}
Falta de acatamento; acto de desacatar.
\section{Desacaudilhado}
\begin{itemize}
\item {Grp. gram.:adj.}
\end{itemize}
\begin{itemize}
\item {Proveniência:(De \textunderscore des...\textunderscore  + \textunderscore acaudilhado\textunderscore )}
\end{itemize}
Que não tem caudilho.
Abandonado pelo caudilho.
\section{Desacauteladamente}
\begin{itemize}
\item {Grp. gram.:adv.}
\end{itemize}
\begin{itemize}
\item {Proveniência:(De \textunderscore desacautelar\textunderscore )}
\end{itemize}
Sem cautela.
\section{Desacautelar}
\begin{itemize}
\item {Grp. gram.:v. t.}
\end{itemize}
\begin{itemize}
\item {Proveniência:(De \textunderscore des...\textunderscore  + \textunderscore acautelar\textunderscore )}
\end{itemize}
Não têr cautela com; descuidar-se de.
\section{Desacavalar}
\begin{itemize}
\item {Grp. gram.:v. t.}
\end{itemize}
\begin{itemize}
\item {Proveniência:(De \textunderscore des...\textunderscore  + \textunderscore acavalar\textunderscore )}
\end{itemize}
Separar (o que estava sobreposto, acavalado)
\section{Desacavallar}
\begin{itemize}
\item {Grp. gram.:v. t.}
\end{itemize}
\begin{itemize}
\item {Proveniência:(De \textunderscore des...\textunderscore  + \textunderscore acavallar\textunderscore )}
\end{itemize}
Separar (o que estava sobreposto, acavallado)
\section{Desacceitar}
\begin{itemize}
\item {Grp. gram.:v. t.}
\end{itemize}
\begin{itemize}
\item {Proveniência:(De \textunderscore des...\textunderscore  + \textunderscore acceitar\textunderscore )}
\end{itemize}
Não acceitar; rejeitar. Cf. Filinto, VI, 161.
\section{Desacceito}
\begin{itemize}
\item {Grp. gram.:adj.}
\end{itemize}
\begin{itemize}
\item {Proveniência:(De \textunderscore desacceitar\textunderscore )}
\end{itemize}
Não acceito.
Recusado.
Mal apreciado.
\section{Desaccentuar}
\begin{itemize}
\item {Grp. gram.:v. t.}
\end{itemize}
\begin{itemize}
\item {Proveniência:(De \textunderscore des...\textunderscore  + \textunderscore accentuar\textunderscore )}
\end{itemize}
Tirar a accentuação a.
Simplificar.
\section{Desaccorrer}
\begin{itemize}
\item {Grp. gram.:v. t.  e  i.}
\end{itemize}
\begin{itemize}
\item {Utilização:Ant.}
\end{itemize}
\begin{itemize}
\item {Proveniência:(De \textunderscore des...\textunderscore  + \textunderscore acorrer\textunderscore )}
\end{itemize}
Não dar soccorro.
Deixar ao abandono.
\section{Desaccorrimento}
\begin{itemize}
\item {Grp. gram.:m.}
\end{itemize}
\begin{itemize}
\item {Utilização:Ant.}
\end{itemize}
Acto de desaccorrer.
\section{Desaccorro}
\begin{itemize}
\item {fónica:cô}
\end{itemize}
\begin{itemize}
\item {Grp. gram.:m.}
\end{itemize}
\begin{itemize}
\item {Utilização:Ant.}
\end{itemize}
O mesmo que \textunderscore desaccorrimento\textunderscore .
\section{Desaceitar}
\begin{itemize}
\item {Grp. gram.:v. t.}
\end{itemize}
\begin{itemize}
\item {Proveniência:(De \textunderscore des...\textunderscore  + \textunderscore aceitar\textunderscore )}
\end{itemize}
Não aceitar; rejeitar. Cf. Filinto, VI, 161.
\section{Desaceito}
\begin{itemize}
\item {Grp. gram.:adj.}
\end{itemize}
\begin{itemize}
\item {Proveniência:(De \textunderscore desaceitar\textunderscore )}
\end{itemize}
Não aceito.
Recusado.
Mal apreciado.
\section{Desacelebrada}
\begin{itemize}
\item {Grp. gram.:adj.  e  fem.}
\end{itemize}
\begin{itemize}
\item {Utilização:Prov.}
\end{itemize}
\begin{itemize}
\item {Utilização:trasm.}
\end{itemize}
Diz-se de mulher adoidada, desorientada, de cabeça leve.
(Por \textunderscore descerebrada\textunderscore , de \textunderscore des...\textunderscore  + \textunderscore cérebro\textunderscore )
\section{Desacentuar}
\begin{itemize}
\item {Grp. gram.:v. t.}
\end{itemize}
\begin{itemize}
\item {Proveniência:(De \textunderscore des...\textunderscore  + \textunderscore acentuar\textunderscore )}
\end{itemize}
Tirar a acentuação a.
Simplificar.
\section{Desacerbar}
\begin{itemize}
\item {Grp. gram.:v. t.}
\end{itemize}
\begin{itemize}
\item {Proveniência:(De \textunderscore des...\textunderscore  + \textunderscore acerbo\textunderscore )}
\end{itemize}
Tirar o amargor de.
Suavizar, adoçar.
\section{Desacertadamente}
\begin{itemize}
\item {Grp. gram.:adv.}
\end{itemize}
Com desacêrto.
\section{Desacertar}
\begin{itemize}
\item {Grp. gram.:v. t.}
\end{itemize}
\begin{itemize}
\item {Grp. gram.:V. i.}
\end{itemize}
\begin{itemize}
\item {Proveniência:(De \textunderscore des...\textunderscore  + \textunderscore acertar\textunderscore )}
\end{itemize}
Fazer com desacêrto.
Tornar desacertado.
Não alcançar.
Proceder com desacêrto.
\section{Desacêrto}
\begin{itemize}
\item {Grp. gram.:m.}
\end{itemize}
Falta de acêrto.
\section{Desachegar}
\begin{itemize}
\item {Grp. gram.:v. t.}
\end{itemize}
\begin{itemize}
\item {Proveniência:(De \textunderscore des...\textunderscore  + \textunderscore achegar\textunderscore )}
\end{itemize}
Separar (o que estava chegado).
\section{Desacidificante}
\begin{itemize}
\item {Grp. gram.:adj.}
\end{itemize}
Que desacidifica.
\section{Desacidificar}
\begin{itemize}
\item {Grp. gram.:v. t.}
\end{itemize}
\begin{itemize}
\item {Proveniência:(De \textunderscore des...\textunderscore  + \textunderscore acidificar\textunderscore )}
\end{itemize}
Tirar a acidez a.
\section{Desaclimar}
\begin{itemize}
\item {Grp. gram.:v. t.}
\end{itemize}
\begin{itemize}
\item {Proveniência:(De \textunderscore des...\textunderscore  + \textunderscore aclimar\textunderscore )}
\end{itemize}
Deshabituar de um clima.
\section{Desacobardar}
\begin{itemize}
\item {Grp. gram.:v. t.}
\end{itemize}
\begin{itemize}
\item {Proveniência:(De \textunderscore des...\textunderscore  + \textunderscore acobardar\textunderscore )}
\end{itemize}
Dar coragem a.
Animar.
\section{Desacoimadamente}
\begin{itemize}
\item {Grp. gram.:adv.}
\end{itemize}
\begin{itemize}
\item {Proveniência:(De \textunderscore desacoimar\textunderscore )}
\end{itemize}
Sem cóima.
\section{Desacoimar}
\begin{itemize}
\item {Grp. gram.:v. t.}
\end{itemize}
\begin{itemize}
\item {Proveniência:(De \textunderscore des...\textunderscore  + \textunderscore acoimar\textunderscore )}
\end{itemize}
Livrar de cóima.
Rehabilitar, restabelecer o crédito de.
\section{Desacoitar}
\begin{itemize}
\item {Grp. gram.:v. t.}
\end{itemize}
\begin{itemize}
\item {Proveniência:(De \textunderscore des...\textunderscore  + \textunderscore acoitar\textunderscore )}
\end{itemize}
Tirar o abrigo a.
Fazer saír do coito ou guarida.
\section{Desacolchetar}
\begin{itemize}
\item {Grp. gram.:v. t.}
\end{itemize}
\begin{itemize}
\item {Proveniência:(De \textunderscore des...\textunderscore  + \textunderscore acolchetar\textunderscore )}
\end{itemize}
Desprender dos colchetes.
Desapertar, desprendendo os colchetes.
\section{Desacolchoar}
\begin{itemize}
\item {Grp. gram.:v. t.}
\end{itemize}
\begin{itemize}
\item {Proveniência:(De \textunderscore des...\textunderscore  + \textunderscore acolchoar\textunderscore )}
\end{itemize}
Desmanchar (o que estava acolchoado).
\section{Desacolher}
\begin{itemize}
\item {Grp. gram.:v. t.}
\end{itemize}
\begin{itemize}
\item {Proveniência:(De \textunderscore des...\textunderscore  + \textunderscore acolher\textunderscore )}
\end{itemize}
Negar abrigo a.
Receber mal.
\section{Desacolhimento}
\begin{itemize}
\item {Grp. gram.:m.}
\end{itemize}
Acto de desacolher.
\section{Desacommodadamente}
\begin{itemize}
\item {Grp. gram.:adv.}
\end{itemize}
Sem cômmodo.
Incommodadamente.
\section{Desacommodar}
\begin{itemize}
\item {Grp. gram.:v. t.}
\end{itemize}
\begin{itemize}
\item {Proveniência:(De \textunderscore des...\textunderscore  + \textunderscore acommodar\textunderscore )}
\end{itemize}
Tirar acommodação a.
Desorganizar, desordenar.
Destituir, desempregar: \textunderscore desacommodar um serviçal\textunderscore .
Incommodar.
Deslocar: \textunderscore desacommodar um móvel\textunderscore .
\section{Desacomodadamente}
\begin{itemize}
\item {Grp. gram.:adv.}
\end{itemize}
Sem cômodo.
Incomodadamente.
\section{Desacomodar}
\begin{itemize}
\item {Grp. gram.:v. t.}
\end{itemize}
\begin{itemize}
\item {Proveniência:(De \textunderscore des...\textunderscore  + \textunderscore acomodar\textunderscore )}
\end{itemize}
Tirar acomodação a.
Desorganizar, desordenar.
Destituir, desempregar: \textunderscore desacomodar um serviçal\textunderscore .
Incomodar.
Deslocar: \textunderscore desacomodar um móvel\textunderscore .
\section{Desacompanhadamente}
\begin{itemize}
\item {Grp. gram.:adv.}
\end{itemize}
\begin{itemize}
\item {Proveniência:(De \textunderscore desacompanhar\textunderscore )}
\end{itemize}
Sem companhia.
\section{Desacompanhar}
\begin{itemize}
\item {Grp. gram.:v. t.}
\end{itemize}
\begin{itemize}
\item {Proveniência:(De \textunderscore des...\textunderscore  + \textunderscore acompanhar\textunderscore )}
\end{itemize}
Deixar de acompanhar.
Deixar de proteger.
Discordar de.
Separar.
\section{Desaconchegar}
\textunderscore v. t.\textunderscore  (e der.)
O mesmo que \textunderscore desconchegar\textunderscore , etc. Cf. Serpa Pinto, \textunderscore África\textunderscore , II, 166.
\section{Desaconselhadamente}
\begin{itemize}
\item {Grp. gram.:adv.}
\end{itemize}
\begin{itemize}
\item {Proveniência:(De \textunderscore desaconselhar\textunderscore )}
\end{itemize}
Sem bom conselho; imprudentemente.
\section{Desaconselhar}
\begin{itemize}
\item {Grp. gram.:v. t.}
\end{itemize}
\begin{itemize}
\item {Proveniência:(De \textunderscore des...\textunderscore  + \textunderscore aconselhar\textunderscore )}
\end{itemize}
Despersuadir.
Desviar de uma resolução.
\section{Desacorçoar}
\begin{itemize}
\item {Grp. gram.:v. i.}
\end{itemize}
\begin{itemize}
\item {Utilização:pop.}
\end{itemize}
\begin{itemize}
\item {Utilização:Ant.}
\end{itemize}
O mesmo que \textunderscore descoroçoar\textunderscore . Cf. \textunderscore Peregrinação\textunderscore , X.
\section{Desacordadamente}
\begin{itemize}
\item {Grp. gram.:adv.}
\end{itemize}
\begin{itemize}
\item {Proveniência:(De \textunderscore desacordar\textunderscore )}
\end{itemize}
Sem acôrdo, com desacôrdo.
\section{Desacordante}
\begin{itemize}
\item {Grp. gram.:adj.}
\end{itemize}
\begin{itemize}
\item {Proveniência:(De \textunderscore desacordar\textunderscore )}
\end{itemize}
Que desacorda.
\section{Desacordar}
\begin{itemize}
\item {Grp. gram.:v. t.}
\end{itemize}
\begin{itemize}
\item {Grp. gram.:V. i.}
\end{itemize}
\begin{itemize}
\item {Proveniência:(De \textunderscore des...\textunderscore  + \textunderscore acordar\textunderscore )}
\end{itemize}
Pôr em desacôrdo.
Tirar o acôrdo a.
Estontear.
Fazer perder os sentidos a.
Discordar.
Desafinar: \textunderscore a orchestra desacordava\textunderscore .
Falar despropositadamente.
Perder os sentidos.
\section{Desacorde}
\begin{itemize}
\item {Grp. gram.:adj.}
\end{itemize}
\begin{itemize}
\item {Proveniência:(De \textunderscore des...\textunderscore  + \textunderscore acorde\textunderscore )}
\end{itemize}
Dissonante.
Desharmónico.
Discordante.
\section{Desacôrdo}
\begin{itemize}
\item {Grp. gram.:m.}
\end{itemize}
\begin{itemize}
\item {Proveniência:(De \textunderscore des...\textunderscore  + \textunderscore acôrdo\textunderscore )}
\end{itemize}
Falta de acôrdo.
Discordância.
Desharmonia.
Delíquio, desmaio.
\section{Desacoroçoar}
\begin{itemize}
\item {Grp. gram.:v. t.}
\end{itemize}
O mesmo que \textunderscore descoroçoar\textunderscore , etc.
\section{Desacorrentar}
\begin{itemize}
\item {Grp. gram.:v. t.}
\end{itemize}
\begin{itemize}
\item {Proveniência:(De \textunderscore des...\textunderscore  + \textunderscore acorrentar\textunderscore )}
\end{itemize}
Desligar da corrente.
Desprender.
Soltar.
\section{Desacorrer}
\begin{itemize}
\item {Grp. gram.:v. t.  e  i.}
\end{itemize}
\begin{itemize}
\item {Utilização:Ant.}
\end{itemize}
\begin{itemize}
\item {Proveniência:(De \textunderscore des...\textunderscore  + \textunderscore acorrer\textunderscore )}
\end{itemize}
Não dar socorro.
Deixar ao abandono.
\section{Desacorrimento}
\begin{itemize}
\item {Grp. gram.:m.}
\end{itemize}
\begin{itemize}
\item {Utilização:Ant.}
\end{itemize}
Acto de desacorrer.
\section{Desacorro}
\begin{itemize}
\item {fónica:cô}
\end{itemize}
\begin{itemize}
\item {Grp. gram.:m.}
\end{itemize}
\begin{itemize}
\item {Utilização:Ant.}
\end{itemize}
O mesmo que \textunderscore desacorrimento\textunderscore .
\section{Desacostumadamente}
\begin{itemize}
\item {Grp. gram.:adv.}
\end{itemize}
\begin{itemize}
\item {Proveniência:(De \textunderscore des...\textunderscore  + \textunderscore acostumar\textunderscore )}
\end{itemize}
Contra o costume.
\section{Desacostumar}
\begin{itemize}
\item {Grp. gram.:v. t.}
\end{itemize}
\begin{itemize}
\item {Proveniência:(De \textunderscore des...\textunderscore  + \textunderscore acostumar\textunderscore )}
\end{itemize}
Fazer perder um costume a.
Fazer saír do que é habitual.
\section{Desacreditado}
\begin{itemize}
\item {Grp. gram.:adj.}
\end{itemize}
\begin{itemize}
\item {Proveniência:(De \textunderscore desacreditar\textunderscore )}
\end{itemize}
Que não tem crédito, que perdeu o crédito.
Depreciado.
\section{Desacreditador}
\begin{itemize}
\item {Grp. gram.:adj.}
\end{itemize}
\begin{itemize}
\item {Grp. gram.:M.}
\end{itemize}
\begin{itemize}
\item {Proveniência:(De \textunderscore desacreditar\textunderscore )}
\end{itemize}
Que desacredita.
Aquelle que desacredita.
\section{Desacreditar}
\begin{itemize}
\item {Grp. gram.:v. t.}
\end{itemize}
\begin{itemize}
\item {Proveniência:(De \textunderscore des...\textunderscore  + \textunderscore acreditar\textunderscore )}
\end{itemize}
Fazer perder o crédito a.
Diffamar; depreciar.
\section{Desacuar}
\begin{itemize}
\item {Grp. gram.:v. i.}
\end{itemize}
\begin{itemize}
\item {Utilização:Bras. do N}
\end{itemize}
Deixar de estar acuado ou empacado (um animal).
\section{Desacumular}
\begin{itemize}
\item {Grp. gram.:v. t.}
\end{itemize}
\begin{itemize}
\item {Proveniência:(De \textunderscore des...\textunderscore  + \textunderscore acumular\textunderscore )}
\end{itemize}
Separar (o que estava acumulado).
\section{Desacunhar}
\begin{itemize}
\item {Grp. gram.:v. t.}
\end{itemize}
\begin{itemize}
\item {Proveniência:(De \textunderscore des...\textunderscore  + \textunderscore acunhar\textunderscore )}
\end{itemize}
Tirar as cunhas a. Cf. M. de Aguiar, \textunderscore Dicc. de Marinha\textunderscore .
\section{Desadmoestar}
\textunderscore v. t.\textunderscore  (e der.)
O mesmo que \textunderscore desaconselhar\textunderscore , etc.
\section{Desadoração}
\begin{itemize}
\item {Grp. gram.:f.}
\end{itemize}
Acto ou effeito de desadorar.
\section{Desadorado}
\begin{itemize}
\item {Grp. gram.:adj.}
\end{itemize}
\begin{itemize}
\item {Utilização:Bras. do N}
\end{itemize}
\begin{itemize}
\item {Proveniência:(De \textunderscore desadorar\textunderscore ^2)}
\end{itemize}
Atacado de dôr violenta.
Impertinente.
\section{Desadorador}
\begin{itemize}
\item {Grp. gram.:adj.}
\end{itemize}
\begin{itemize}
\item {Grp. gram.:M.}
\end{itemize}
\begin{itemize}
\item {Proveniência:(De \textunderscore desadorar\textunderscore ^1)}
\end{itemize}
Que desadora.
Aquelle que desadora.
\section{Desadorar}
\begin{itemize}
\item {Grp. gram.:v. t.}
\end{itemize}
\begin{itemize}
\item {Grp. gram.:V. i.}
\end{itemize}
\begin{itemize}
\item {Utilização:Des.}
\end{itemize}
\begin{itemize}
\item {Proveniência:(De \textunderscore des...\textunderscore  + \textunderscore adorar\textunderscore )}
\end{itemize}
Não adorar.
Menosprezar.
Não gostar de.
Detestar: \textunderscore desadorar lisonjas\textunderscore .
Irar-se, vociferar.
\section{Desadorar}
\begin{itemize}
\item {Grp. gram.:v. i.}
\end{itemize}
\begin{itemize}
\item {Utilização:Bras. do N}
\end{itemize}
\begin{itemize}
\item {Grp. gram.:V. t.}
\end{itemize}
\begin{itemize}
\item {Proveniência:(De \textunderscore dôr\textunderscore , com o mesmo prefixo de \textunderscore desinquieto, desinfeliz, deslavrar\textunderscore , etc.)}
\end{itemize}
Soffrer dôr violenta.
Importunar, maçar.
\section{Desadormecer}
\begin{itemize}
\item {Grp. gram.:v. t.}
\end{itemize}
\begin{itemize}
\item {Proveniência:(De \textunderscore des...\textunderscore  + \textunderscore adormecer\textunderscore )}
\end{itemize}
Despertar.
\section{Desadormentar}
\textunderscore v. t.\textunderscore  (e der.)
O mesmo que \textunderscore desadormecer\textunderscore , etc.
\section{Desadornadamente}
\begin{itemize}
\item {Grp. gram.:adv.}
\end{itemize}
Com desadôrno.
\section{Desadornar}
\begin{itemize}
\item {Grp. gram.:v. t.}
\end{itemize}
Tirar o adôrno a; desenfeitar.
\section{Desadôrno}
\begin{itemize}
\item {Grp. gram.:m.}
\end{itemize}
Falta de adôrno.
Desalinho.
\section{Desadoro}
\begin{itemize}
\item {Grp. gram.:m.}
\end{itemize}
\begin{itemize}
\item {Utilização:Bras. do N}
\end{itemize}
Acto de desadorar^2.
\section{Desadunado}
\begin{itemize}
\item {Grp. gram.:adj.}
\end{itemize}
\begin{itemize}
\item {Proveniência:(De \textunderscore des...\textunderscore  + \textunderscore adunado\textunderscore )}
\end{itemize}
Desunido; distinto.
\section{Desadvertido}
\begin{itemize}
\item {Grp. gram.:adj.}
\end{itemize}
\begin{itemize}
\item {Proveniência:(De \textunderscore des...\textunderscore  + \textunderscore advertido\textunderscore )}
\end{itemize}
Indiscreto; que não reflecte; inadvertido.
\section{Desafabilidade}
\begin{itemize}
\item {Grp. gram.:f.}
\end{itemize}
Qualidade de quem ou daquilo que é desafável.
\section{Desafaimar}
\begin{itemize}
\item {Grp. gram.:v. t.}
\end{itemize}
\begin{itemize}
\item {Proveniência:(De \textunderscore des...\textunderscore  + \textunderscore afaimar\textunderscore )}
\end{itemize}
Tirar a fome a; saciar.
\section{Desafamar}
\begin{itemize}
\item {Grp. gram.:v. t.}
\end{itemize}
\begin{itemize}
\item {Proveniência:(De \textunderscore des...\textunderscore  + \textunderscore afamar\textunderscore )}
\end{itemize}
Desacreditar.
Diffamar; infamar.
\section{Desafável}
\begin{itemize}
\item {Grp. gram.:adj.}
\end{itemize}
Que não é afável.
\section{Desafazer}
\begin{itemize}
\item {Grp. gram.:v. t.}
\end{itemize}
\begin{itemize}
\item {Proveniência:(De \textunderscore des...\textunderscore  + \textunderscore afazer\textunderscore )}
\end{itemize}
Desacostumar.
\section{Desafear}
\begin{itemize}
\item {Grp. gram.:v. t.}
\end{itemize}
\begin{itemize}
\item {Proveniência:(De \textunderscore des...\textunderscore  + \textunderscore afear\textunderscore )}
\end{itemize}
Tirar a fealdade a. Cf. Filinto, XVII, p. 202.
\section{Desafectação}
\begin{itemize}
\item {Grp. gram.:f.}
\end{itemize}
\begin{itemize}
\item {Proveniência:(De \textunderscore des...\textunderscore  + \textunderscore afectação\textunderscore )}
\end{itemize}
Qualidade de quem ou daquilo que é desafectado.
\section{Desafectadamente}
\begin{itemize}
\item {Grp. gram.:adv.}
\end{itemize}
De modo desafectado.
\section{Desafectado}
\begin{itemize}
\item {Grp. gram.:adj.}
\end{itemize}
\begin{itemize}
\item {Proveniência:(De \textunderscore des...\textunderscore  + \textunderscore afectado\textunderscore )}
\end{itemize}
Que não tem afectação: em que não há afectação: \textunderscore modos desafectados\textunderscore .
\section{Desafecto}
\begin{itemize}
\item {Grp. gram.:adj.}
\end{itemize}
\begin{itemize}
\item {Grp. gram.:M.}
\end{itemize}
Que não tem afecto.
Falta de afecto.
\section{Desafeição}
\begin{itemize}
\item {Grp. gram.:f.}
\end{itemize}
Falta de afeição; desafecto.
\section{Desafeiçoar}
\begin{itemize}
\item {Grp. gram.:v. t.}
\end{itemize}
\begin{itemize}
\item {Proveniência:(De \textunderscore des...\textunderscore  + \textunderscore afeiçoar\textunderscore )}
\end{itemize}
Tirar a feição de.
Desfigurar.
Alterar.
\section{Desafeitar}
\textunderscore v. t.\textunderscore  (e der.)
O mesmo que \textunderscore desenfeitar\textunderscore , etc. Cf. Garrett, \textunderscore Filippa\textunderscore , VI.
\section{Desafeito}
\begin{itemize}
\item {Grp. gram.:adj.}
\end{itemize}
\begin{itemize}
\item {Proveniência:(De \textunderscore des...\textunderscore  + \textunderscore afeito\textunderscore )}
\end{itemize}
Desacostumado, deshabituado.
\section{Desaferrar}
\begin{itemize}
\item {Grp. gram.:v. t.}
\end{itemize}
\begin{itemize}
\item {Utilização:Fig.}
\end{itemize}
\begin{itemize}
\item {Grp. gram.:V. i.}
\end{itemize}
\begin{itemize}
\item {Utilização:Náut.}
\end{itemize}
\begin{itemize}
\item {Proveniência:(De \textunderscore des...\textunderscore  + \textunderscore aferrar\textunderscore )}
\end{itemize}
Desprender (o que estava preso com o ferro).
Soltar.
Despersuadir.
Levantar ferro (um navio); desancorar.
\section{Desaferrolhar}
\begin{itemize}
\item {Grp. gram.:v. t.}
\end{itemize}
\begin{itemize}
\item {Proveniência:(De \textunderscore des...\textunderscore  + \textunderscore aferrolhar\textunderscore )}
\end{itemize}
Abrir (o que estava fechado com o ferrolho).
Correr o ferrolho a.
Patentear.
Tirar (o que estava aferrolhado).
Soltar.
\section{Desafervorar}
\begin{itemize}
\item {Grp. gram.:v. t.}
\end{itemize}
\begin{itemize}
\item {Proveniência:(De \textunderscore des...\textunderscore  + \textunderscore afervorar\textunderscore )}
\end{itemize}
Afroixar o fervor de.
\section{Desaffabilidade}
\begin{itemize}
\item {Grp. gram.:f.}
\end{itemize}
Qualidade de quem ou daquillo que é desaffável.
\section{Desaffável}
\begin{itemize}
\item {Grp. gram.:adj.}
\end{itemize}
Que não é affável.
\section{Desaffectação}
\begin{itemize}
\item {Grp. gram.:f.}
\end{itemize}
\begin{itemize}
\item {Proveniência:(De \textunderscore des...\textunderscore  + \textunderscore affectação\textunderscore )}
\end{itemize}
Qualidade de quem ou daquillo que é desaffectado.
\section{Desaffectadamente}
\begin{itemize}
\item {Grp. gram.:adv.}
\end{itemize}
De modo desaffectado.
\section{Desaffectado}
\begin{itemize}
\item {Grp. gram.:adj.}
\end{itemize}
\begin{itemize}
\item {Proveniência:(De \textunderscore des...\textunderscore  + \textunderscore affectado\textunderscore )}
\end{itemize}
Que não tem affectação: em que não há affectação: \textunderscore modos desaffectados\textunderscore .
\section{Desaffecto}
\begin{itemize}
\item {Grp. gram.:adj.}
\end{itemize}
\begin{itemize}
\item {Grp. gram.:M.}
\end{itemize}
Que não tem affecto.
Falta de affecto.
\section{Desaffeição}
\begin{itemize}
\item {Grp. gram.:f.}
\end{itemize}
Falta de affeição; desaffecto.
\section{Desaffeiçoar}
\begin{itemize}
\item {Grp. gram.:v. t.}
\end{itemize}
Tirar a affeição a.
\section{Desaffixar}
\begin{itemize}
\item {Grp. gram.:v. t.}
\end{itemize}
\begin{itemize}
\item {Proveniência:(De \textunderscore des...\textunderscore  + \textunderscore affixar\textunderscore )}
\end{itemize}
Despegar, soltar, (aquillo que estava affixado).
\section{Desafiação}
\begin{itemize}
\item {Grp. gram.:f.}
\end{itemize}
O mesmo que \textunderscore desafio\textunderscore :«\textunderscore eu lhe agradeço muito suas desafiações\textunderscore ». Fern. Lopes, \textunderscore Chrón, de D. João I.\textunderscore 
\section{Desafiador}
\begin{itemize}
\item {Grp. gram.:m.}
\end{itemize}
Aquelle que desafia.
\section{Desafiante}
\begin{itemize}
\item {Grp. gram.:adj.}
\end{itemize}
\begin{itemize}
\item {Grp. gram.:M.}
\end{itemize}
\begin{itemize}
\item {Proveniência:(De \textunderscore desafiar\textunderscore )}
\end{itemize}
Que desafia.
Desafiador.
\section{Desafiar}
\begin{itemize}
\item {Grp. gram.:v. t.}
\end{itemize}
\begin{itemize}
\item {Proveniência:(Do it. \textunderscore disfidare\textunderscore )}
\end{itemize}
Provocar para luta ou duello.
Provocar.
Excitar.
Estimular.
Convidar.
\section{Desafiar}
\begin{itemize}
\item {Grp. gram.:v. t.}
\end{itemize}
\begin{itemize}
\item {Proveniência:(De \textunderscore des...\textunderscore  + \textunderscore afiar\textunderscore )}
\end{itemize}
Tirar o fio a.
Embotar.
\section{Desafinação}
\begin{itemize}
\item {Grp. gram.:f.}
\end{itemize}
Estado daquillo que é desafinado.
Acto de desafinar.
\section{Desafinadamente}
\begin{itemize}
\item {Grp. gram.:adv.}
\end{itemize}
De modo desafinado.
\section{Desafinado}
\begin{itemize}
\item {Grp. gram.:adj.}
\end{itemize}
\begin{itemize}
\item {Proveniência:(De \textunderscore desafinar\textunderscore )}
\end{itemize}
Dissonante, desacorde.
\section{Desafinamento}
\begin{itemize}
\item {Grp. gram.:m.}
\end{itemize}
(V.desafinação)
\section{Desafinar}
\begin{itemize}
\item {Grp. gram.:v. t.}
\end{itemize}
\begin{itemize}
\item {Grp. gram.:V. i.}
\end{itemize}
\begin{itemize}
\item {Utilização:Fig.}
\end{itemize}
\begin{itemize}
\item {Proveniência:(De \textunderscore des...\textunderscore  + \textunderscore afinar\textunderscore )}
\end{itemize}
Fazer perder a afinação de: \textunderscore desafinar a rabeca\textunderscore .
Perder a afinação: \textunderscore a rabeca desafinou\textunderscore .
Irritar-se.
\section{Desafio}
\begin{itemize}
\item {Grp. gram.:m.}
\end{itemize}
Acto de desafiar^1.
Provocação.
\textunderscore Cantar ao desafio\textunderscore , cantar improvisada e alternadamente com outra ou outras pessôas, provocando-as graciosamente, em esfolhadas e noutros folguedos populares.
\section{Desafivelar}
\begin{itemize}
\item {Grp. gram.:v. t.}
\end{itemize}
\begin{itemize}
\item {Proveniência:(De \textunderscore des...\textunderscore  + \textunderscore afivelar\textunderscore )}
\end{itemize}
Desapertar a fivela de.
\section{Desafixar}
\begin{itemize}
\item {Grp. gram.:v. t.}
\end{itemize}
\begin{itemize}
\item {Proveniência:(De \textunderscore des...\textunderscore  + \textunderscore afixar\textunderscore )}
\end{itemize}
Despegar, soltar, (aquilo que estava afixado).
\section{Desafogadamente}
\begin{itemize}
\item {Grp. gram.:adv.}
\end{itemize}
Com desafôgo.
\section{Desafogar}
\begin{itemize}
\item {Grp. gram.:v. t.}
\end{itemize}
\begin{itemize}
\item {Proveniência:(De \textunderscore des...\textunderscore  + \textunderscore afogar\textunderscore )}
\end{itemize}
Desafrontar.
Desopprimir.
Alliviar.
Desapertar: \textunderscore desafogar o casaco\textunderscore .
Tornar livre.
Expandir: \textunderscore desafogar tristezas\textunderscore .
\section{Desafôgo}
\begin{itemize}
\item {Grp. gram.:m.}
\end{itemize}
Acto ou effeito de desafogar.
\section{Desafoguear}
\begin{itemize}
\item {Grp. gram.:v. t.}
\end{itemize}
\begin{itemize}
\item {Proveniência:(De \textunderscore des...\textunderscore  + \textunderscore afoguear\textunderscore )}
\end{itemize}
Tirar o calor a.
Refrescar.
Refrigerar.
\section{Desaforadamente}
\begin{itemize}
\item {Grp. gram.:adv.}
\end{itemize}
De modo desaforado.
\section{Desaforado}
\begin{itemize}
\item {Grp. gram.:adj.}
\end{itemize}
\begin{itemize}
\item {Utilização:Jur.}
\end{itemize}
\begin{itemize}
\item {Proveniência:(De \textunderscore desaforar\textunderscore )}
\end{itemize}
Livre ou isento de um foro.
Que desacata a honestidade ou a cortesia: \textunderscore palavras desaforadas\textunderscore .
Inconveniente.
Insolente: \textunderscore homem desaforado\textunderscore .
Libertino.
\section{Desaforamento}
\begin{itemize}
\item {Grp. gram.:m.}
\end{itemize}
Desafôro.
Acto de desaforar.
\section{Desaforar}
\begin{itemize}
\item {Grp. gram.:v. t.}
\end{itemize}
\begin{itemize}
\item {Utilização:Ant.}
\end{itemize}
\begin{itemize}
\item {Proveniência:(De \textunderscore des...\textunderscore  + \textunderscore aforar\textunderscore )}
\end{itemize}
Tornar insolente, impudico, atrevido.
Desobrigar de um foro.
\section{Desaforido}
\begin{itemize}
\item {Grp. gram.:adj.}
\end{itemize}
\begin{itemize}
\item {Utilização:Prov.}
\end{itemize}
\begin{itemize}
\item {Utilização:trasm.}
\end{itemize}
\begin{itemize}
\item {Utilização:alent.}
\end{itemize}
Desenfreado.
Libidinoso.
(Relaciona-se com \textunderscore desafôro\textunderscore ?)
\section{Desafôro}
\begin{itemize}
\item {Grp. gram.:m.}
\end{itemize}
\begin{itemize}
\item {Proveniência:(De \textunderscore desaforar\textunderscore )}
\end{itemize}
Impudência.
Atrevimento.
Petulância.
\section{Desafortunadamente}
\begin{itemize}
\item {Grp. gram.:adv.}
\end{itemize}
De modo desafortunado.
\section{Desafortunado}
\begin{itemize}
\item {Grp. gram.:adj.}
\end{itemize}
\begin{itemize}
\item {Proveniência:(De \textunderscore des...\textunderscore  + \textunderscore afortunado\textunderscore )}
\end{itemize}
Que não tem fortuna; infeliz.
\section{Desafreguesado}
\begin{itemize}
\item {Grp. gram.:adj.}
\end{itemize}
\begin{itemize}
\item {Proveniência:(De \textunderscore desafreguesar\textunderscore )}
\end{itemize}
Que não tem fregueses ou que os perdeu.
\section{Desafreguesar}
\begin{itemize}
\item {Grp. gram.:v. t.}
\end{itemize}
Desviar os fregueses de.
\section{Desafronta}
\begin{itemize}
\item {Grp. gram.:f.}
\end{itemize}
Acto ou effeito de desafrontar.
\section{Desafrontadamente}
\begin{itemize}
\item {Grp. gram.:adv.}
\end{itemize}
De modo desafrontado.
\section{Desafrontado}
\begin{itemize}
\item {Grp. gram.:adj.}
\end{itemize}
\begin{itemize}
\item {Utilização:Fig.}
\end{itemize}
\begin{itemize}
\item {Proveniência:(De \textunderscore desafrontar\textunderscore )}
\end{itemize}
Livre da calma ou do calor.
Alliviado, desopprimido.
Desafogado; aberto.
Vingado; desaggravado.
\section{Desafrontador}
\begin{itemize}
\item {Grp. gram.:adj.}
\end{itemize}
\begin{itemize}
\item {Grp. gram.:M.}
\end{itemize}
Que desafronta.
Aquelle que desafronta.
\section{Desafrontamento}
\begin{itemize}
\item {Grp. gram.:m.}
\end{itemize}
Acto de desafrontar.
\section{Desafrontar}
\begin{itemize}
\item {Grp. gram.:v. t.}
\end{itemize}
Livrar de afronta.
Vingar.
Desaggravar.
Desafogar.
Tirar a calma a.
Alliviar.
\section{Desafuar}
\begin{itemize}
\item {Grp. gram.:v. i.}
\end{itemize}
\begin{itemize}
\item {Utilização:Prov.}
\end{itemize}
\begin{itemize}
\item {Utilização:minh.}
\end{itemize}
Desatar as cordas dos fueiros.
(Por \textunderscore desafueirar\textunderscore )
\section{Desafundar}
\begin{itemize}
\item {Grp. gram.:v. t.}
\end{itemize}
\begin{itemize}
\item {Proveniência:(De \textunderscore des...\textunderscore  + \textunderscore afundar\textunderscore )}
\end{itemize}
Tirar do fundo de águas.
\section{Desagaloar}
\begin{itemize}
\item {Grp. gram.:v. t.}
\end{itemize}
Tirar os galões de.
Desguarnecer.
\section{Desagarrar}
\begin{itemize}
\item {Grp. gram.:v. t.}
\end{itemize}
\begin{itemize}
\item {Proveniência:(De \textunderscore des...\textunderscore  + \textunderscore agarrar\textunderscore )}
\end{itemize}
Despegar, desprender.
\section{Desagasalhar}
\textunderscore v. t.\textunderscore  (e der.)
O mesmo que desabrigar, etc.
\section{Desagastamento}
\begin{itemize}
\item {Grp. gram.:m.}
\end{itemize}
Acto de desagastar.
\section{Desagastar}
\begin{itemize}
\item {Grp. gram.:v. t.}
\end{itemize}
\begin{itemize}
\item {Proveniência:(De \textunderscore des...\textunderscore  + \textunderscore agastar\textunderscore )}
\end{itemize}
Fazer cessar a ira de.
Tranquillizar; reconciliar.
\section{Desageitar}
\textunderscore v. t.\textunderscore  (e der.)
(V. \textunderscore desajeitar\textunderscore , etc.)
\section{Desagglomerar}
\begin{itemize}
\item {Grp. gram.:v. t.}
\end{itemize}
\begin{itemize}
\item {Proveniência:(De \textunderscore des...\textunderscore  + \textunderscore agglomerar\textunderscore )}
\end{itemize}
Desacumular.
\section{Desaggravador}
\begin{itemize}
\item {Grp. gram.:adj.}
\end{itemize}
\begin{itemize}
\item {Grp. gram.:M.}
\end{itemize}
Que desaggrava.
Aquelle que desaggrava.
\section{Desaggravar}
\begin{itemize}
\item {Grp. gram.:v. t.}
\end{itemize}
Livrar de aggravo.
Alliviar, suavizar.
Desafrontar.
Dar reparação de (um aggravo).
Emendar judicialmente (aggravo feito a litigante).
\section{Desaggravo}
\begin{itemize}
\item {Grp. gram.:m.}
\end{itemize}
Acto ou effeito de desaggravar.
\section{Desaggregação}
\begin{itemize}
\item {Grp. gram.:f.}
\end{itemize}
Acto ou effeito de desaggregar.
\section{Desaggregante}
\begin{itemize}
\item {Grp. gram.:adj.}
\end{itemize}
Que desaggrega.
\section{Desaggregar}
\begin{itemize}
\item {Grp. gram.:v. t.}
\end{itemize}
\begin{itemize}
\item {Proveniência:(De \textunderscore des...\textunderscore  + \textunderscore aggregar\textunderscore )}
\end{itemize}
Desligar (o que estava aggregado).
\section{Desaggregável}
\begin{itemize}
\item {Grp. gram.:adj.}
\end{itemize}
Que se póde desaggregar.
\section{Desaglomerar}
\begin{itemize}
\item {Grp. gram.:v. t.}
\end{itemize}
\begin{itemize}
\item {Proveniência:(De \textunderscore des...\textunderscore  + \textunderscore aglomerar\textunderscore )}
\end{itemize}
Desacumular.
\section{Desagoar}
\textunderscore v. i.\textunderscore  (e der.)
O mesmo que \textunderscore desaguar\textunderscore , etc.
\section{Desagora}
\begin{itemize}
\item {Grp. gram.:adv.}
\end{itemize}
\begin{itemize}
\item {Proveniência:(De \textunderscore dês\textunderscore  + \textunderscore agora\textunderscore )}
\end{itemize}
Desde já. Cf. D. Bernárdez, \textunderscore Lima\textunderscore , 163 e 210.
\section{Desagradar}
\begin{itemize}
\item {Grp. gram.:v. i.}
\end{itemize}
Não agradar.
Descontentar.
\section{Desagradável}
\begin{itemize}
\item {Grp. gram.:adj.}
\end{itemize}
Que desagrada.
\section{Desagradavelmente}
\begin{itemize}
\item {Grp. gram.:adv.}
\end{itemize}
De modo desagradável.
\section{Desagradecer}
\begin{itemize}
\item {Grp. gram.:v. t.}
\end{itemize}
Não agradecer.
Retribuir com ingratidão.
\section{Desagradecidamente}
\begin{itemize}
\item {Grp. gram.:adv.}
\end{itemize}
De modo desagradecido.
\section{Desagradecido}
\begin{itemize}
\item {Grp. gram.:adj.}
\end{itemize}
\begin{itemize}
\item {Proveniência:(De \textunderscore desagradecer\textunderscore )}
\end{itemize}
Ingrato.
Que não agradeceu.
\section{Desagradecimento}
\begin{itemize}
\item {Grp. gram.:m.}
\end{itemize}
Ingratidão.
Acto de desagradecer.
\section{Desagrado}
\begin{itemize}
\item {Grp. gram.:m.}
\end{itemize}
Falta de agrado.
Desprazer.
Rudeza.
Falta de estima.
Acto de desagradar.
\section{Desagravador}
\begin{itemize}
\item {Grp. gram.:adj.}
\end{itemize}
\begin{itemize}
\item {Grp. gram.:M.}
\end{itemize}
Que desagrava.
Aquele que desagrava.
\section{Desagravar}
\begin{itemize}
\item {Grp. gram.:v. t.}
\end{itemize}
Livrar de agravo.
Aliviar, suavizar.
Desafrontar.
Dar reparação de (um agravo).
Emendar judicialmente (agravo feito a litigante).
\section{Desagravo}
\begin{itemize}
\item {Grp. gram.:m.}
\end{itemize}
Acto ou efeito de desagravar.
\section{Desagregação}
\begin{itemize}
\item {Grp. gram.:f.}
\end{itemize}
Acto ou efeito de desagregar.
\section{Desagregante}
\begin{itemize}
\item {Grp. gram.:adj.}
\end{itemize}
Que desagrega.
\section{Desagregar}
\begin{itemize}
\item {Grp. gram.:v. t.}
\end{itemize}
\begin{itemize}
\item {Proveniência:(De \textunderscore des...\textunderscore  + \textunderscore agregar\textunderscore )}
\end{itemize}
Desligar (o que estava agregado)
\section{Desagregável}
\begin{itemize}
\item {Grp. gram.:adj.}
\end{itemize}
Que se póde desagregar.
\section{Desagreste}
\begin{itemize}
\item {Grp. gram.:adj.}
\end{itemize}
\begin{itemize}
\item {Utilização:Prov.}
\end{itemize}
\begin{itemize}
\item {Utilização:trasm.}
\end{itemize}
O mesmo que \textunderscore agreste\textunderscore .
(Cp. \textunderscore desinquieto\textunderscore , por analogia de formação)
\section{Desagrilhoamento}
\begin{itemize}
\item {Grp. gram.:m.}
\end{itemize}
Acto de desagrilhoar.
\section{Desagrilhoar}
\begin{itemize}
\item {Grp. gram.:v. t.}
\end{itemize}
Livrar de grilhões; libertar. Cf. Castilho, \textunderscore Metam.\textunderscore , 288.
\section{Desaguadeiro}
\begin{itemize}
\item {Grp. gram.:m.}
\end{itemize}
(V.desaguadoiro)
\section{Desaguadoiro}
\begin{itemize}
\item {Grp. gram.:m.}
\end{itemize}
\begin{itemize}
\item {Proveniência:(De \textunderscore desaguar\textunderscore )}
\end{itemize}
Valla, rêgo, sargeta, para escoamento de águas.
\section{Desaguador}
\begin{itemize}
\item {Grp. gram.:adj.}
\end{itemize}
Que desagua.
\section{Desaguadouro}
\begin{itemize}
\item {Grp. gram.:m.}
\end{itemize}
\begin{itemize}
\item {Proveniência:(De \textunderscore desaguar\textunderscore )}
\end{itemize}
Valla, rêgo, sargeta, para escoamento de águas.
\section{Desaguamento}
\begin{itemize}
\item {Grp. gram.:m.}
\end{itemize}
Acto ou effeito de desaguar.
\section{Desaguar}
\begin{itemize}
\item {Grp. gram.:v. t.}
\end{itemize}
\begin{itemize}
\item {Utilização:Pop.}
\end{itemize}
\begin{itemize}
\item {Grp. gram.:V. i.}
\end{itemize}
\begin{itemize}
\item {Proveniência:(De \textunderscore des...\textunderscore  + \textunderscore aguar\textunderscore )}
\end{itemize}
Tirar a água de.
Enxugar.
Dar alguma coisa a comer a (animaes), para não aguarem.
Lançar as suas águas, (falando-se de rios ou regatos): \textunderscore o Tejo desagua no mar\textunderscore .
Despejar-se.
Desembocar.
\section{Desaguisado}
\begin{itemize}
\item {Grp. gram.:m.}
\end{itemize}
\begin{itemize}
\item {Proveniência:(De \textunderscore des...\textunderscore  + \textunderscore aguisado\textunderscore )}
\end{itemize}
Rixa; contenda.
\section{Desainado}
\begin{itemize}
\item {Grp. gram.:adj.}
\end{itemize}
\begin{itemize}
\item {Proveniência:(De \textunderscore desainar\textunderscore )}
\end{itemize}
Emmagrecido.
\section{Desainadura}
\begin{itemize}
\item {Grp. gram.:f.}
\end{itemize}
\begin{itemize}
\item {Proveniência:(De \textunderscore desainar\textunderscore )}
\end{itemize}
Doença nos cascos dos cavallos folgados.
\section{Desainar}
\begin{itemize}
\item {Grp. gram.:v. t.}
\end{itemize}
\begin{itemize}
\item {Utilização:Ant.}
\end{itemize}
\begin{itemize}
\item {Grp. gram.:V. i.}
\end{itemize}
Amansar (o falcão, privando-o de carne).
Gritar enraivecido, como o falcão privado de carne.
(Relaciona-se com \textunderscore sanha\textunderscore , do lat. \textunderscore insania\textunderscore ?)
\section{Desairadamente}
\begin{itemize}
\item {Grp. gram.:adv.}
\end{itemize}
Com desaire.
\section{Desairar}
\begin{itemize}
\item {Grp. gram.:v. t.}
\end{itemize}
\begin{itemize}
\item {Utilização:Des.}
\end{itemize}
\begin{itemize}
\item {Proveniência:(De \textunderscore desaire\textunderscore )}
\end{itemize}
Tornar desairoso.
Causar desaire a.
\section{Desaire}
\begin{itemize}
\item {Grp. gram.:m.}
\end{itemize}
\begin{itemize}
\item {Proveniência:(De \textunderscore des...\textunderscore  + cast. \textunderscore aire\textunderscore )}
\end{itemize}
Falta de elegância.
Qualidade de quem ou daquillo que é desajeitado.
Inconveniência.
Falta de decoro.
Mancha: \textunderscore aquelle procedimento é um desaire\textunderscore .
\section{Desairosamente}
\begin{itemize}
\item {Grp. gram.:adv.}
\end{itemize}
De modo desairoso.
\section{Desairoso}
\begin{itemize}
\item {Grp. gram.:adj.}
\end{itemize}
\begin{itemize}
\item {Proveniência:(De \textunderscore des...\textunderscore  + \textunderscore airoso\textunderscore )}
\end{itemize}
Que não é airoso.
Em que há desaire.
Que tem desaire.
\section{Desajeitado}
\begin{itemize}
\item {Grp. gram.:adj.}
\end{itemize}
\begin{itemize}
\item {Proveniência:(De \textunderscore desajeitar\textunderscore )}
\end{itemize}
Que não tem jeito.
Bronco, lorpa.
\section{Desajeitamento}
\begin{itemize}
\item {Grp. gram.:m.}
\end{itemize}
Acto ou modos de desajeitado.
\section{Desajeitar}
\begin{itemize}
\item {Grp. gram.:v. t.}
\end{itemize}
\begin{itemize}
\item {Proveniência:(De \textunderscore des...\textunderscore  + \textunderscore ajeitar\textunderscore )}
\end{itemize}
Tirar o jeito a; deformar.
\section{Desajoujar}
\begin{itemize}
\item {Grp. gram.:v. t.}
\end{itemize}
\begin{itemize}
\item {Proveniência:(De \textunderscore des...\textunderscore  + \textunderscore ajoujar\textunderscore )}
\end{itemize}
Desligar do ajoujo.
Soltar.
\section{Desajudar}
\begin{itemize}
\item {Grp. gram.:v. t.}
\end{itemize}
\begin{itemize}
\item {Proveniência:(De \textunderscore des...\textunderscore  + \textunderscore ajudar\textunderscore )}
\end{itemize}
Não ajudar.
\section{Desajuizado}
\begin{itemize}
\item {fónica:ju-i}
\end{itemize}
\begin{itemize}
\item {Grp. gram.:adj.}
\end{itemize}
\begin{itemize}
\item {Proveniência:(De \textunderscore desajuizar\textunderscore )}
\end{itemize}
Que perdeu o juízo; insensato.
\section{Desajuizar}
\begin{itemize}
\item {fónica:ju-i}
\end{itemize}
\begin{itemize}
\item {Grp. gram.:v. t.}
\end{itemize}
\begin{itemize}
\item {Proveniência:(De \textunderscore des...\textunderscore  + \textunderscore ajuizar\textunderscore )}
\end{itemize}
Tirar o juízo a.
Entontecer.
\section{Desajuntar}
\begin{itemize}
\item {Grp. gram.:v. t.}
\end{itemize}
\begin{itemize}
\item {Proveniência:(De \textunderscore des...\textunderscore  + \textunderscore ajuntar\textunderscore )}
\end{itemize}
Desunir.
\section{Desajustar}
\begin{itemize}
\item {Grp. gram.:v. t.}
\end{itemize}
\begin{itemize}
\item {Proveniência:(De \textunderscore des...\textunderscore  + \textunderscore ajustar\textunderscore )}
\end{itemize}
Desfazer o ajuste de.
Transtornar, desordenar.
Desajuntar.
\section{Desajuste}
\begin{itemize}
\item {Grp. gram.:m.}
\end{itemize}
Acto de desajustar.
\section{Desalagar}
\begin{itemize}
\item {Grp. gram.:v. t.}
\end{itemize}
\begin{itemize}
\item {Utilização:Fig.}
\end{itemize}
\begin{itemize}
\item {Proveniência:(De \textunderscore des...\textunderscore  + \textunderscore alagar\textunderscore )}
\end{itemize}
Livrar da água (aquillo que estava alagado por ella).
Evacuar.
Desembaraçar.
\section{Desalastrar}
\begin{itemize}
\item {Grp. gram.:v. t.}
\end{itemize}
\begin{itemize}
\item {Proveniência:(De \textunderscore des...\textunderscore  + \textunderscore alastrar\textunderscore )}
\end{itemize}
Alliviar do lastro, tirar o lastro a.
\section{Desalbardar}
\begin{itemize}
\item {Grp. gram.:v. t.}
\end{itemize}
\begin{itemize}
\item {Proveniência:(De \textunderscore des...\textunderscore  + \textunderscore albardar\textunderscore )}
\end{itemize}
Tirar a albarda a.
\section{Desalcançar}
\begin{itemize}
\item {Grp. gram.:v. t.}
\end{itemize}
\begin{itemize}
\item {Utilização:bras}
\end{itemize}
\begin{itemize}
\item {Utilização:Neol.}
\end{itemize}
Não alcançar, não conseguir.
\section{Desalegrar}
\begin{itemize}
\item {Grp. gram.:v. t.}
\end{itemize}
\begin{itemize}
\item {Proveniência:(De \textunderscore desalegre\textunderscore )}
\end{itemize}
Tirar a alegria a; entristecer.
\section{Desalegre}
\begin{itemize}
\item {Grp. gram.:adj.}
\end{itemize}
\begin{itemize}
\item {Utilização:Des.}
\end{itemize}
Não alegre.
Triste.
\section{Desaleitar}
\begin{itemize}
\item {Grp. gram.:v. t.}
\end{itemize}
\begin{itemize}
\item {Proveniência:(De \textunderscore des...\textunderscore  + \textunderscore aleitar\textunderscore )}
\end{itemize}
O mesmo que \textunderscore desmamar\textunderscore .
\section{Desalentador}
\begin{itemize}
\item {Grp. gram.:adj.}
\end{itemize}
Que desalenta.
\section{Desalentar}
\begin{itemize}
\item {Grp. gram.:v. t.}
\end{itemize}
\begin{itemize}
\item {Grp. gram.:V. i.}
\end{itemize}
\begin{itemize}
\item {Proveniência:(De \textunderscore des...\textunderscore  + \textunderscore alentar\textunderscore )}
\end{itemize}
Tirar o alento a.
Desanimar.
Desanimar-se.
\section{Desalento}
\begin{itemize}
\item {Grp. gram.:m.}
\end{itemize}
Falta de alento, de ânimo.
\section{Desalfaiar}
\begin{itemize}
\item {Grp. gram.:v. t.}
\end{itemize}
Tirar as alfaias a.
\section{Desalforjar}
\begin{itemize}
\item {Grp. gram.:v. t.}
\end{itemize}
\begin{itemize}
\item {Utilização:Fig.}
\end{itemize}
Tirar do alforge.
Tirar da algibeira.
Despejar.
\section{Desalgemar}
\begin{itemize}
\item {Grp. gram.:v. t.}
\end{itemize}
Soltar das algemas; libertar.
\section{Desalhar}
\begin{itemize}
\item {Grp. gram.:v. t.}
\end{itemize}
\begin{itemize}
\item {Utilização:Ant.}
\end{itemize}
Alienar (bens).
(Por \textunderscore desalhear\textunderscore , de \textunderscore alhear\textunderscore )
\section{Desaliança}
\begin{itemize}
\item {Grp. gram.:f.}
\end{itemize}
Falta de aliança.
Quebra de aliança.
Ruptura de relações.
\section{Desaliar}
\begin{itemize}
\item {Grp. gram.:v. t.}
\end{itemize}
\begin{itemize}
\item {Proveniência:(De \textunderscore des...\textunderscore  + \textunderscore aliar\textunderscore )}
\end{itemize}
Separar (os que estavam aliados)
\section{Desalijar}
\begin{itemize}
\item {Grp. gram.:v. t.}
\end{itemize}
\begin{itemize}
\item {Proveniência:(De \textunderscore des...\textunderscore  + \textunderscore alijar\textunderscore )}
\end{itemize}
Alliviar da carga.
Alliviar.
\section{Desalinhadamente}
\begin{itemize}
\item {Grp. gram.:adv.}
\end{itemize}
De modo desalinhado.
\section{Desalinhado}
\begin{itemize}
\item {Grp. gram.:adj.}
\end{itemize}
\begin{itemize}
\item {Proveniência:(De \textunderscore desalinhar\textunderscore )}
\end{itemize}
Que não tem alinho.
Descuidado.
Desordenado: \textunderscore vestuário desalinhado\textunderscore .
Despretensioso.
\section{Desalinhar}
\begin{itemize}
\item {Grp. gram.:v. t.}
\end{itemize}
\begin{itemize}
\item {Proveniência:(De \textunderscore des...\textunderscore  + \textunderscore alinhar\textunderscore )}
\end{itemize}
Tirar do alinhamento.
Desordenar; desenfeitar: \textunderscore desalinhar os cabellos\textunderscore .
\section{Desalinhavado}
\begin{itemize}
\item {Grp. gram.:m.}
\end{itemize}
\begin{itemize}
\item {Proveniência:(De desalinhavar)}
\end{itemize}
O mesmo que \textunderscore desalinho\textunderscore .
\section{Desalinhavar}
\begin{itemize}
\item {Grp. gram.:v. t.}
\end{itemize}
\begin{itemize}
\item {Proveniência:(De \textunderscore des...\textunderscore  + \textunderscore alinhavar\textunderscore )}
\end{itemize}
Tirar os alinhavos de.
\section{Desalinhavo}
\begin{itemize}
\item {Grp. gram.:m.}
\end{itemize}
\begin{itemize}
\item {Utilização:bras}
\end{itemize}
\begin{itemize}
\item {Utilização:Neol.}
\end{itemize}
O mesmo que \textunderscore desalinho\textunderscore .
\section{Desalinho}
\begin{itemize}
\item {Grp. gram.:m.}
\end{itemize}
Falta de alinho.
Desordem, descuido, no trajar.
Desconcêrto, desarranjo.
Perturbação de ânimo.
\section{Desalistar}
\begin{itemize}
\item {Grp. gram.:v. t.}
\end{itemize}
\begin{itemize}
\item {Proveniência:(De \textunderscore des...\textunderscore  + \textunderscore alistar\textunderscore )}
\end{itemize}
Tirar da lista.
\section{Desaliviar}
\begin{itemize}
\item {Grp. gram.:v. t.}
\end{itemize}
\begin{itemize}
\item {Utilização:Des.}
\end{itemize}
\begin{itemize}
\item {Proveniência:(De \textunderscore aliviar\textunderscore , com o mesmo pref. de \textunderscore desinquieto\textunderscore )}
\end{itemize}
Aliviar.
\section{Desalliança}
\begin{itemize}
\item {Grp. gram.:f.}
\end{itemize}
Falta de alliança.
Quebra de alliança.
Ruptura de relações.
\section{Desalliar}
\begin{itemize}
\item {Grp. gram.:v. t.}
\end{itemize}
\begin{itemize}
\item {Proveniência:(De \textunderscore des...\textunderscore  + \textunderscore alliar\textunderscore )}
\end{itemize}
Separar (os que estavam alliados).
\section{Desalliviar}
\begin{itemize}
\item {Grp. gram.:v. t.}
\end{itemize}
\begin{itemize}
\item {Utilização:Des.}
\end{itemize}
\begin{itemize}
\item {Proveniência:(De \textunderscore alliviar\textunderscore , com o mesmo pref. de \textunderscore desinquieto\textunderscore )}
\end{itemize}
Alliviar.
\section{Desalmadamente}
\begin{itemize}
\item {Grp. gram.:adv.}
\end{itemize}
De modo desalmado.
\section{Desalmado}
\begin{itemize}
\item {Grp. gram.:adj.}
\end{itemize}
\begin{itemize}
\item {Proveniência:(De \textunderscore des...\textunderscore  + \textunderscore alma\textunderscore )}
\end{itemize}
Cruel; deshumano.
Que não tem coração.
Que não tem consciência.
Que mostra maus sentimentos.
\section{Desalojamento}
\begin{itemize}
\item {Grp. gram.:m.}
\end{itemize}
Acto ou effeito de desalojar.
\section{Desalojar}
\begin{itemize}
\item {Grp. gram.:v. t.}
\end{itemize}
\begin{itemize}
\item {Grp. gram.:V. i.}
\end{itemize}
\begin{itemize}
\item {Utilização:Des.}
\end{itemize}
\begin{itemize}
\item {Proveniência:(De \textunderscore des...\textunderscore  + \textunderscore alojar\textunderscore )}
\end{itemize}
Lançar fóra de um alojamento.
Expellir de um pôsto.
Tirar de um lugar (quem ou aquillo que nelle estava guardado).
Mudar de acampamento.
\section{Desalterar}
\begin{itemize}
\item {Grp. gram.:v. t.}
\end{itemize}
Talvez \textunderscore gal.\textunderscore 
Acalmar, abrandar.
(Cp. fr. \textunderscore désaltérer\textunderscore )
\section{Desalumiado}
\begin{itemize}
\item {Grp. gram.:adj.}
\end{itemize}
\begin{itemize}
\item {Utilização:Fig.}
\end{itemize}
\begin{itemize}
\item {Proveniência:(De \textunderscore des...\textunderscore  + \textunderscore alumiado\textunderscore )}
\end{itemize}
Que não tem luz.
Ignorante: \textunderscore espírito desalumiado\textunderscore .
\section{Desalvorar}
\begin{itemize}
\item {Grp. gram.:v. t.}
\end{itemize}
\begin{itemize}
\item {Utilização:Pop.}
\end{itemize}
O mesmo que \textunderscore desarvorar\textunderscore :«\textunderscore ...e huma das naos desalvorada recuou para Lisboa\textunderscore ». Filinto, \textunderscore D. Man.\textunderscore , I, 143.
\section{Desamabilidade}
\begin{itemize}
\item {Grp. gram.:f.}
\end{itemize}
Qualidade de quem ou daquillo que é desamável.
\section{Desamador}
\begin{itemize}
\item {Grp. gram.:m.}
\end{itemize}
Aquelle que desama.
\section{Desamalgamar}
\begin{itemize}
\item {Grp. gram.:v. t.}
\end{itemize}
\begin{itemize}
\item {Proveniência:(De \textunderscore des...\textunderscore  + \textunderscore amalgamar\textunderscore )}
\end{itemize}
Separar (aquillo que estava amalgamado).
\section{Desamamentar}
\begin{itemize}
\item {Grp. gram.:v. t.}
\end{itemize}
\begin{itemize}
\item {Proveniência:(De \textunderscore des...\textunderscore  + \textunderscore amamentar\textunderscore )}
\end{itemize}
O mesmo que \textunderscore desmamar\textunderscore .
\section{Desamantilhar}
\begin{itemize}
\item {Grp. gram.:v.}
\end{itemize}
\begin{itemize}
\item {Utilização:t. Náut.}
\end{itemize}
\begin{itemize}
\item {Proveniência:(De \textunderscore des...\textunderscore  + \textunderscore amantilho\textunderscore )}
\end{itemize}
Alar os amantilhos de (um navio), ficando os de umas vêrgas oppostos aos das outras.
\section{Desàmão}
\begin{itemize}
\item {Grp. gram.:f.}
\end{itemize}
\begin{itemize}
\item {Utilização:Prov.}
\end{itemize}
\begin{itemize}
\item {Utilização:minh.}
\end{itemize}
\begin{itemize}
\item {Proveniência:(De \textunderscore des...\textunderscore  + \textunderscore á\textunderscore  + \textunderscore mão\textunderscore )}
\end{itemize}
Us. na loc. \textunderscore á desàmão\textunderscore , fóra de caminho; fóra de jeito.
\section{Desamar}
\begin{itemize}
\item {Grp. gram.:v. t.}
\end{itemize}
\begin{itemize}
\item {Proveniência:(De \textunderscore des...\textunderscore  + \textunderscore amar\textunderscore )}
\end{itemize}
Não amar.
Odiar; deixar de amar.
\section{Desamarrar}
\begin{itemize}
\item {Grp. gram.:v. t.}
\end{itemize}
\begin{itemize}
\item {Grp. gram.:V. i.}
\end{itemize}
\begin{itemize}
\item {Proveniência:(De \textunderscore des...\textunderscore  + \textunderscore amarrar\textunderscore )}
\end{itemize}
Desprender (aquillo que estava amarrado).
Soltar da amarra.
Deslocar.
Demover.
Levantar amarra.
\section{Desamarrotar}
\begin{itemize}
\item {Grp. gram.:v. t.}
\end{itemize}
\begin{itemize}
\item {Proveniência:(De \textunderscore des...\textunderscore  + \textunderscore amarrotar\textunderscore )}
\end{itemize}
Alisar (aquillo que estava amarrotado).
\section{Desamartelar}
\begin{itemize}
\item {Grp. gram.:v. t.}
\end{itemize}
\begin{itemize}
\item {Utilização:Prov.}
\end{itemize}
\begin{itemize}
\item {Utilização:minh.}
\end{itemize}
Endireitar (o que está amartelado ou amolgado); desamolgar.
\section{Desamassar}
\begin{itemize}
\item {Grp. gram.:v. t.}
\end{itemize}
\begin{itemize}
\item {Proveniência:(De \textunderscore des...\textunderscore  + \textunderscore amassar\textunderscore )}
\end{itemize}
Desfazer (a amassadura), para que esta não levede depressa.
\section{Desamável}
\begin{itemize}
\item {Grp. gram.:adj.}
\end{itemize}
Que não é amável; descortês.
\section{Desambição}
\begin{itemize}
\item {Grp. gram.:f.}
\end{itemize}
Falta de ambição; desinteresse.
Isenção.
Abnegação.
\section{Desambicioso}
\begin{itemize}
\item {Grp. gram.:adj.}
\end{itemize}
Que não tem ambição.
\section{Desamigar}
\begin{itemize}
\item {Grp. gram.:v. t.}
\end{itemize}
\begin{itemize}
\item {Proveniência:(De \textunderscore des...\textunderscore  + \textunderscore amigar\textunderscore )}
\end{itemize}
Quebrar a amizade de.
\section{Desamizade}
\begin{itemize}
\item {Grp. gram.:f.}
\end{itemize}
Falta de amizade.
\section{Desamoedação}
\begin{itemize}
\item {fónica:mo-e}
\end{itemize}
\begin{itemize}
\item {Grp. gram.:f.}
\end{itemize}
Acto de desamoedar.
\section{Desamoedar}
\begin{itemize}
\item {fónica:mo-é}
\end{itemize}
\begin{itemize}
\item {Grp. gram.:v. t.}
\end{itemize}
(V.desmonetizar)
\section{Desamolgar}
\begin{itemize}
\item {Grp. gram.:v. t.}
\end{itemize}
Endireitar ou aplanar (o que estava amolgado).
\section{Desamontar}
\begin{itemize}
\item {Grp. gram.:v. t.}
\end{itemize}
\begin{itemize}
\item {Utilização:Prov.}
\end{itemize}
\begin{itemize}
\item {Utilização:alg.}
\end{itemize}
O mesmo que \textunderscore desmontar\textunderscore .
\section{Desamontoar}
\begin{itemize}
\item {Grp. gram.:v. t.}
\end{itemize}
O mesmo que \textunderscore desacumular\textunderscore .
\section{Desamor}
\begin{itemize}
\item {Grp. gram.:m.}
\end{itemize}
Falta de amor.
Desprêzo.
\section{Desamorado}
\begin{itemize}
\item {Grp. gram.:adj.}
\end{itemize}
Que tem desamor.
\section{Desamorável}
\begin{itemize}
\item {Grp. gram.:adj.}
\end{itemize}
Que não é amorável.
\section{Desamoravelmente}
\begin{itemize}
\item {Grp. gram.:adv.}
\end{itemize}
De modo desamorável.
Com desamor.
\section{Desamorosamente}
\begin{itemize}
\item {Grp. gram.:adv.}
\end{itemize}
De modo desamoroso.
\section{Desamoroso}
\begin{itemize}
\item {Grp. gram.:adj.}
\end{itemize}
O mesmo que \textunderscore desamorável\textunderscore .
\section{Desamortalhar}
\begin{itemize}
\item {Grp. gram.:v. t.}
\end{itemize}
\begin{itemize}
\item {Proveniência:(De \textunderscore des...\textunderscore  + \textunderscore amortalhar\textunderscore )}
\end{itemize}
Tirar a mortalha a.
\section{Desamortização}
\begin{itemize}
\item {Grp. gram.:f.}
\end{itemize}
Acto de desamortizar.
\section{Desamortizar}
\begin{itemize}
\item {Grp. gram.:v. t.}
\end{itemize}
\begin{itemize}
\item {Proveniência:(De \textunderscore des...\textunderscore  + \textunderscore amortizar\textunderscore )}
\end{itemize}
Sujeitar ao direito commum (bens de mão morta).
\section{Desamortizável}
\begin{itemize}
\item {Grp. gram.:adj.}
\end{itemize}
Que se póde desamortizar.
\section{Desamotinar}
\begin{itemize}
\item {Grp. gram.:v. t.}
\end{itemize}
\begin{itemize}
\item {Proveniência:(De \textunderscore des...\textunderscore  + \textunderscore amotinar\textunderscore )}
\end{itemize}
Fazer cessar o motim de.
\section{Desamparadamente}
\begin{itemize}
\item {Grp. gram.:adv.}
\end{itemize}
\begin{itemize}
\item {Proveniência:(De \textunderscore desamparado\textunderscore )}
\end{itemize}
Com desamparo.
\section{Desamparado}
\begin{itemize}
\item {Grp. gram.:adj.}
\end{itemize}
\begin{itemize}
\item {Proveniência:(De \textunderscore desamparar\textunderscore )}
\end{itemize}
Que não tem amparo; abandonado, desprotegido.
\section{Desamparador}
\begin{itemize}
\item {Grp. gram.:m.}
\end{itemize}
Aquelle que desampara. Cf. Usque, \textunderscore Tribulações\textunderscore , 52.
\section{Desamparar}
\begin{itemize}
\item {Grp. gram.:v. t.}
\end{itemize}
Deixar de amparar.
Não auxiliar.
Abandonar.
Deixar de sustentar: \textunderscore desamparar os filhos\textunderscore .
Desviar-se de: \textunderscore desamparar uma povoação\textunderscore .
Não tratar de.
\section{Desamparo}
\begin{itemize}
\item {Grp. gram.:m.}
\end{itemize}
Acto ou effeito de desamparar.
Falta de amparo.
\section{Desamuador}
\begin{itemize}
\item {Grp. gram.:m.}
\end{itemize}
\begin{itemize}
\item {Utilização:Bras}
\end{itemize}
\begin{itemize}
\item {Proveniência:(De \textunderscore desamuar\textunderscore )}
\end{itemize}
Instrumento de calafates e carpinteiros, para sacar cavilhas, abrir orifícios, etc. Cf. M. de Aguiar, \textunderscore Diccion. de Marinha\textunderscore .
\section{Desamuar}
\begin{itemize}
\item {Grp. gram.:v. t.}
\end{itemize}
\begin{itemize}
\item {Proveniência:(De \textunderscore des...\textunderscore  + \textunderscore amuar\textunderscore )}
\end{itemize}
Tirar o amúo a.
\section{Desamurizar}
\begin{itemize}
\item {Grp. gram.:v. t.}
\end{itemize}
\begin{itemize}
\item {Utilização:Prov.}
\end{itemize}
\begin{itemize}
\item {Utilização:trasm.}
\end{itemize}
\begin{itemize}
\item {Proveniência:(De \textunderscore des...\textunderscore  + \textunderscore muro\textunderscore )}
\end{itemize}
Destruir o muro de.
\section{Desancador}
\begin{itemize}
\item {Grp. gram.:m.}
\end{itemize}
Aquelle que desanca.
\section{Desancamento}
\begin{itemize}
\item {Grp. gram.:m.}
\end{itemize}
Acto de desancar.
\section{Desancar}
\begin{itemize}
\item {Grp. gram.:v. t.}
\end{itemize}
\begin{itemize}
\item {Utilização:Fig.}
\end{itemize}
\begin{itemize}
\item {Proveniência:(De \textunderscore des...\textunderscore  + \textunderscore anca\textunderscore )}
\end{itemize}
Derrear com pancadas.
Bater muito.
Maltratar, vencer, em discussão ou com crítica.
\section{Desancorar}
\begin{itemize}
\item {Grp. gram.:v. t.}
\end{itemize}
\begin{itemize}
\item {Grp. gram.:V. i.}
\end{itemize}
\begin{itemize}
\item {Proveniência:(De \textunderscore des...\textunderscore  + \textunderscore ancorar\textunderscore )}
\end{itemize}
Levantar a âncora de: \textunderscore desancorar um navio\textunderscore .
Levantar âncora.
\section{Desanda}
\begin{itemize}
\item {Grp. gram.:f.}
\end{itemize}
\begin{itemize}
\item {Utilização:Pop.}
\end{itemize}
\begin{itemize}
\item {Proveniência:(De \textunderscore desandar\textunderscore )}
\end{itemize}
Reprehensão; descompostura.
\section{Desandador}
\begin{itemize}
\item {Grp. gram.:m.}
\end{itemize}
\begin{itemize}
\item {Proveniência:(De \textunderscore desandar\textunderscore )}
\end{itemize}
Utensílio de ferro, com que se fazem desandar parafusos.
Instrumento náutico, para fazer desandar o corpo da sonda.
\section{Desandança}
\begin{itemize}
\item {Grp. gram.:f.}
\end{itemize}
\begin{itemize}
\item {Utilização:Prov.}
\end{itemize}
\begin{itemize}
\item {Proveniência:(De \textunderscore desandar\textunderscore )}
\end{itemize}
Contratempo, revés. (Colhido em Turquel)
\section{Desandar}
\begin{itemize}
\item {Grp. gram.:v. t.}
\end{itemize}
\begin{itemize}
\item {Grp. gram.:V. i.}
\end{itemize}
\begin{itemize}
\item {Proveniência:(De \textunderscore des...\textunderscore  + \textunderscore andar\textunderscore )}
\end{itemize}
Fazer andar para trás.: \textunderscore desandar um parafuso\textunderscore .
Desatarrachar.
Supprimir (uma maracha), para de duas salinas fazer uma só.
Andar para trás.
Reverter, redundar.
Piorar: \textunderscore o doente desandou\textunderscore .
\section{Desanelar}
\begin{itemize}
\item {Grp. gram.:v. t.}
\end{itemize}
\begin{itemize}
\item {Proveniência:(De \textunderscore des...\textunderscore  + \textunderscore anelar\textunderscore )}
\end{itemize}
Desmanchar os anéis de.
\section{Desanexação}
\begin{itemize}
\item {fónica:csa}
\end{itemize}
\begin{itemize}
\item {Grp. gram.:f.}
\end{itemize}
Acto ou efeito de desanexar.
\section{Desanexadamente}
\begin{itemize}
\item {fónica:csa}
\end{itemize}
\begin{itemize}
\item {Grp. gram.:adv.}
\end{itemize}
\begin{itemize}
\item {Proveniência:(De \textunderscore desanexar\textunderscore )}
\end{itemize}
Sem nexo, sem ligação.
\section{Desanexar}
\begin{itemize}
\item {fónica:csar}
\end{itemize}
\begin{itemize}
\item {Grp. gram.:v. t.}
\end{itemize}
\begin{itemize}
\item {Proveniência:(De \textunderscore des...\textunderscore  + \textunderscore anexar\textunderscore )}
\end{itemize}
Separar (aquilo que era anexo)
Desligar.
\section{Desanexo}
\begin{itemize}
\item {fónica:cso}
\end{itemize}
\begin{itemize}
\item {Grp. gram.:m.}
\end{itemize}
Aquilo que não está anexo. Cf. Camillo, \textunderscore Livro Negro\textunderscore , 223.
\section{Desangradeiro}
\begin{itemize}
\item {fónica:san}
\end{itemize}
\begin{itemize}
\item {Grp. gram.:m.}
\end{itemize}
\begin{itemize}
\item {Proveniência:(De \textunderscore desangrar\textunderscore )}
\end{itemize}
O mesmo que \textunderscore bueiro\textunderscore .
\section{Desangrar}
\begin{itemize}
\item {fónica:san}
\end{itemize}
\begin{itemize}
\item {Grp. gram.:v. t.}
\end{itemize}
\begin{itemize}
\item {Proveniência:(De \textunderscore de...\textunderscore  + \textunderscore sangrar\textunderscore )}
\end{itemize}
Tirar o sangre a.
Extenuar.
Extrahir algum líquido de (poça, pote, pipa, etc.), por meio de orifício, abaixo do nivel do mesmo líquido.
\section{Desanichar}
\begin{itemize}
\item {Grp. gram.:v. t.}
\end{itemize}
Tirar de um nicho.
Deslocar (quem se anichou num emprêgo).
\section{Desanimação}
\begin{itemize}
\item {Grp. gram.:f.}
\end{itemize}
Falta de animação: desânimo.
\section{Desanimadamente}
\begin{itemize}
\item {Grp. gram.:adv.}
\end{itemize}
De modo desanimado.
\section{Desanimado}
\begin{itemize}
\item {Grp. gram.:adj.}
\end{itemize}
\begin{itemize}
\item {Proveniência:(De \textunderscore desanimar\textunderscore )}
\end{itemize}
Que perde o ânimo ou a coragem; desalentado.
\section{Desanimar}
\begin{itemize}
\item {Grp. gram.:v. t.}
\end{itemize}
\begin{itemize}
\item {Grp. gram.:V. i.  e  p.}
\end{itemize}
\begin{itemize}
\item {Proveniência:(De \textunderscore des...\textunderscore  + \textunderscore animar\textunderscore )}
\end{itemize}
Tirar o ânimo, a coragem a.
Desalentar.
Desalentar-se.
\section{Desânimo}
\begin{itemize}
\item {Grp. gram.:m.}
\end{itemize}
Falta de ânimo.
\section{Desaninhar}
\begin{itemize}
\item {Grp. gram.:v. t.}
\end{itemize}
\begin{itemize}
\item {Proveniência:(De \textunderscore des...\textunderscore  + \textunderscore aninhar\textunderscore )}
\end{itemize}
Tirar do ninho.
Deslocar.
Desanichar.
\section{Desannexação}
\begin{itemize}
\item {fónica:csa}
\end{itemize}
\begin{itemize}
\item {Grp. gram.:f.}
\end{itemize}
Acto ou effeito de desannexar.
\section{Desannexadamente}
\begin{itemize}
\item {fónica:csa}
\end{itemize}
\begin{itemize}
\item {Grp. gram.:adv.}
\end{itemize}
\begin{itemize}
\item {Proveniência:(De \textunderscore desannexar\textunderscore )}
\end{itemize}
Sem nexo, sem ligação.
\section{Desannexar}
\begin{itemize}
\item {fónica:csar}
\end{itemize}
\begin{itemize}
\item {Grp. gram.:v. t.}
\end{itemize}
\begin{itemize}
\item {Proveniência:(De \textunderscore des...\textunderscore  + \textunderscore annexar\textunderscore )}
\end{itemize}
Separar (aquillo que era annexo).
Desligar.
\section{Desannexo}
\begin{itemize}
\item {fónica:cso}
\end{itemize}
\begin{itemize}
\item {Grp. gram.:m.}
\end{itemize}
Aquillo que não está annexo. Cf. Camillo, \textunderscore Livro Negro\textunderscore , 223.
\section{Desanojar}
\begin{itemize}
\item {Grp. gram.:v. t.}
\end{itemize}
\begin{itemize}
\item {Proveniência:(De \textunderscore des...\textunderscore  + \textunderscore anojar\textunderscore )}
\end{itemize}
Tirar o nojo a.
Dar pêsames a.
Tirar o enfado a.
\section{Desanojo}
\begin{itemize}
\item {Grp. gram.:m.}
\end{itemize}
Acto de desanojar.
\section{Desanuviar}
\begin{itemize}
\item {Grp. gram.:v. t.}
\end{itemize}
\begin{itemize}
\item {Proveniência:(De \textunderscore des...\textunderscore  + \textunderscore anuviar\textunderscore )}
\end{itemize}
Tirar as nuvens de.
Desassombrar.
\section{Desapacientar}
\begin{itemize}
\item {Grp. gram.:v. t.}
\end{itemize}
\begin{itemize}
\item {Utilização:Prov.}
\end{itemize}
\begin{itemize}
\item {Utilização:trasm.}
\end{itemize}
Tornar impaciente, zangar.
\section{Desapadrinhar}
\begin{itemize}
\item {Grp. gram.:v. t.}
\end{itemize}
\begin{itemize}
\item {Proveniência:(De \textunderscore des...\textunderscore  + \textunderscore apadrinhar\textunderscore )}
\end{itemize}
Tirar a protecção a, tornar abandonado.
\section{Desapagar}
\begin{itemize}
\item {Grp. gram.:v. t.}
\end{itemize}
\begin{itemize}
\item {Proveniência:(De \textunderscore apagar\textunderscore , com o mesmo pref. de \textunderscore desinquieto\textunderscore )}
\end{itemize}
Apagar bem.
Expungir.
\section{Desapaixonadamente}
\begin{itemize}
\item {Grp. gram.:adv.}
\end{itemize}
\begin{itemize}
\item {Proveniência:(De \textunderscore desapaixonado\textunderscore )}
\end{itemize}
Sem paixão.
\section{Desapaixonado}
\begin{itemize}
\item {Grp. gram.:adj.}
\end{itemize}
\begin{itemize}
\item {Proveniência:(De \textunderscore des...\textunderscore  + \textunderscore apaixonar\textunderscore )}
\end{itemize}
Quem não tem paixão.
Impassivel.
Indifferente.
\section{Desapaixonar}
\begin{itemize}
\item {Grp. gram.:v. t.}
\end{itemize}
\begin{itemize}
\item {Proveniência:(De \textunderscore des...\textunderscore  + \textunderscore apaixonar\textunderscore )}
\end{itemize}
Fazer perder uma paixão a.
Distrahir.
\section{Desaparafusar}
\begin{itemize}
\item {Grp. gram.:v. t.}
\end{itemize}
\begin{itemize}
\item {Proveniência:(De \textunderscore des...\textunderscore  + \textunderscore aparafusar\textunderscore )}
\end{itemize}
Desandar os parafusos de.
\section{Desaparecer}
\begin{itemize}
\item {Grp. gram.:v. i.}
\end{itemize}
\begin{itemize}
\item {Proveniência:(De \textunderscore des...\textunderscore  + \textunderscore aparecer\textunderscore )}
\end{itemize}
Deixar de aparecer.
Esconder-se.
Morrer.
Perder-se.
\section{Desaparentado}
\begin{itemize}
\item {Grp. gram.:adj.}
\end{itemize}
Que não é aparentado.
Que não tem parentesco.
\section{Desapartar}
\begin{itemize}
\item {Grp. gram.:v. t.}
\end{itemize}
\begin{itemize}
\item {Utilização:Pop.}
\end{itemize}
\begin{itemize}
\item {Proveniência:(De \textunderscore apartar\textunderscore , com o mesmo pref. de \textunderscore desinquieto\textunderscore )}
\end{itemize}
Apartar.
\section{Desapavorar}
\begin{itemize}
\item {Grp. gram.:v. t.}
\end{itemize}
\begin{itemize}
\item {Proveniência:(De \textunderscore des...\textunderscore  + \textunderscore apavorar\textunderscore )}
\end{itemize}
Tirar o pavor a.
\section{Desapear}
\begin{itemize}
\item {Grp. gram.:v. t.}
\end{itemize}
O mesmo que \textunderscore apear\textunderscore , (com o mesmo pref. de \textunderscore desinquieto\textunderscore ).
\section{Desapeçonhentar}
\begin{itemize}
\item {Grp. gram.:v. t.}
\end{itemize}
\begin{itemize}
\item {Proveniência:(De \textunderscore peçonhento\textunderscore )}
\end{itemize}
Tirar a peçonha a.
\section{Desapegadamente}
\begin{itemize}
\item {Grp. gram.:adv.}
\end{itemize}
Com desapêgo.
\section{Desapegamento}
\begin{itemize}
\item {Grp. gram.:m.}
\end{itemize}
O mesmo que \textunderscore desapêgo\textunderscore .
\section{Desapegar}
\begin{itemize}
\item {Grp. gram.:v. t.}
\end{itemize}
O mesmo que \textunderscore despegar\textunderscore .
\section{Desapêgo}
\begin{itemize}
\item {Grp. gram.:m.}
\end{itemize}
\begin{itemize}
\item {Proveniência:(De \textunderscore des...\textunderscore  + \textunderscore apêgo\textunderscore )}
\end{itemize}
Falta de affeição.
Desamor.
Desinteresse.
\section{Desaperceber}
\begin{itemize}
\item {Grp. gram.:v. t.}
\end{itemize}
Deixar de aperceber.
\section{Desapercebidamente}
\begin{itemize}
\item {Grp. gram.:adv.}
\end{itemize}
De modo desapercebido.
\section{Desapercebido}
\begin{itemize}
\item {Grp. gram.:adj.}
\end{itemize}
Desprovido; desguarnecido.--Como synónymo de \textunderscore despercebido\textunderscore , não é expressão autorizada.
\section{Desapercebimento}
\begin{itemize}
\item {Grp. gram.:m.}
\end{itemize}
Falta de apercebimento.
\section{Desaperrar}
\begin{itemize}
\item {Grp. gram.:v. t.}
\end{itemize}
\begin{itemize}
\item {Proveniência:(De \textunderscore des...\textunderscore  + \textunderscore aperrar\textunderscore )}
\end{itemize}
Desengatilhar.
Pôr no descanso (o cão da espingarda).
\section{Desapertadamente}
\begin{itemize}
\item {Grp. gram.:adv.}
\end{itemize}
Com desapêrto; sem apêrto.
\section{Desapertar}
\begin{itemize}
\item {Grp. gram.:v. t.}
\end{itemize}
\begin{itemize}
\item {Proveniência:(De \textunderscore des...\textunderscore  + \textunderscore apertar\textunderscore )}
\end{itemize}
Tirar de apêrto.
Alargar.
Desabotoar.
Alliviar.
Soltar.
\section{Desapêrto}
\begin{itemize}
\item {Grp. gram.:m.}
\end{itemize}
Acto ou effeito de desapertar.
\section{Desapiedadamente}
\begin{itemize}
\item {Grp. gram.:adv.}
\end{itemize}
Sem piedade.
Cruelmente.
\section{Desapiedar}
\begin{itemize}
\item {Grp. gram.:v. t.}
\end{itemize}
\begin{itemize}
\item {Proveniência:(De \textunderscore des...\textunderscore  + \textunderscore apiedar\textunderscore )}
\end{itemize}
Tirar a piedade a.
Tornar duro, cruel.
\section{Desapoderado}
\begin{itemize}
\item {Grp. gram.:adj.}
\end{itemize}
\begin{itemize}
\item {Utilização:Fig.}
\end{itemize}
\begin{itemize}
\item {Proveniência:(De \textunderscore desapoderar\textunderscore )}
\end{itemize}
Desenfreado; furioso.
Desabalado:«\textunderscore galgou paredes e searas em desapoderada fuga\textunderscore ». Camillo, \textunderscore Brasileira\textunderscore , 291.
\section{Desapoderar}
\begin{itemize}
\item {Grp. gram.:v. t.}
\end{itemize}
\begin{itemize}
\item {Proveniência:(De \textunderscore des...\textunderscore  + \textunderscore apoderar\textunderscore )}
\end{itemize}
Privar do poder.
Tirar a posse de alguma coisa a.
\section{Desapoiar}
\begin{itemize}
\item {Grp. gram.:v. t.}
\end{itemize}
\begin{itemize}
\item {Proveniência:(De \textunderscore des...\textunderscore  + \textunderscore apoiar\textunderscore )}
\end{itemize}
Privar de apoio.
Discordar de.
Abandonar.
\section{Desapoio}
\begin{itemize}
\item {Grp. gram.:m.}
\end{itemize}
Acto de desapoiar.
\section{Desapolvilhar}
\begin{itemize}
\item {Grp. gram.:v. t.}
\end{itemize}
\begin{itemize}
\item {Proveniência:(De \textunderscore des...\textunderscore  + \textunderscore apolvilhar\textunderscore )}
\end{itemize}
Limpar dos pós.
\section{Desapontadamente}
\begin{itemize}
\item {Grp. gram.:adv.}
\end{itemize}
Com desapontamento.
\section{Desapontado}
\begin{itemize}
\item {Grp. gram.:adj.}
\end{itemize}
\begin{itemize}
\item {Proveniência:(De \textunderscore desapontar\textunderscore ^2)}
\end{itemize}
Que soffreu desapontamento.
\section{Desapontamento}
\begin{itemize}
\item {Grp. gram.:m.}
\end{itemize}
\begin{itemize}
\item {Utilização:Angl}
\end{itemize}
\begin{itemize}
\item {Proveniência:(Do ingl. \textunderscore disappointment\textunderscore )}
\end{itemize}
Surpresa.
Successo desagradável, que surprehende.--É voc., introduzido por Garrett. Cf. \textunderscore Dona Branca\textunderscore .
\section{Desapontar}
\begin{itemize}
\item {Grp. gram.:v. t.}
\end{itemize}
\begin{itemize}
\item {Proveniência:(De \textunderscore des...\textunderscore  + \textunderscore apontar\textunderscore )}
\end{itemize}
Apontar mal.
Tirar da pontaria.
\section{Desapontar}
\begin{itemize}
\item {Grp. gram.:v. t.}
\end{itemize}
\begin{itemize}
\item {Utilização:Angl}
\end{itemize}
Causar desapontamento a.
\section{Desapoquentar}
\begin{itemize}
\item {Grp. gram.:v. t.}
\end{itemize}
\begin{itemize}
\item {Proveniência:(De \textunderscore des...\textunderscore  + \textunderscore apoquentar\textunderscore )}
\end{itemize}
Tranquïllizar.
Alliviar.
\section{Desaportuguesar}
\begin{itemize}
\item {Grp. gram.:v. t.}
\end{itemize}
\begin{itemize}
\item {Proveniência:(De \textunderscore des...\textunderscore  + \textunderscore aportuguesar\textunderscore )}
\end{itemize}
Tirar a feição portuguesa a. Cf. Herculano, \textunderscore Lendas\textunderscore , I, 276.
\section{Desaposentar}
\begin{itemize}
\item {Grp. gram.:v. t.}
\end{itemize}
\begin{itemize}
\item {Proveniência:(De \textunderscore des...\textunderscore  + \textunderscore aposentar\textunderscore )}
\end{itemize}
Privar de aposento.
\section{Desapossar}
\begin{itemize}
\item {Grp. gram.:v. t.}
\end{itemize}
\begin{itemize}
\item {Proveniência:(De \textunderscore des...\textunderscore  + \textunderscore apossar\textunderscore )}
\end{itemize}
Privar da posse; despojar.
\section{Desapparecer}
\begin{itemize}
\item {Grp. gram.:v. i.}
\end{itemize}
\begin{itemize}
\item {Proveniência:(De \textunderscore des...\textunderscore  + \textunderscore apparecer\textunderscore )}
\end{itemize}
Deixar de apparecer.
Esconder-se.
Morrer.
Perder-se.
\section{Desaparecido}
\begin{itemize}
\item {Grp. gram.:adj.}
\end{itemize}
\begin{itemize}
\item {Proveniência:(De \textunderscore desaparecer\textunderscore )}
\end{itemize}
Que desapareceu.
\section{Desaparecimento}
\begin{itemize}
\item {Grp. gram.:m.}
\end{itemize}
O mesmo que \textunderscore desaparição\textunderscore .
\section{Desaparelhadamente}
\begin{itemize}
\item {Grp. gram.:adv.}
\end{itemize}
Sem aparelho.
Sem ornatos.
\section{Desaparelhamento}
\begin{itemize}
\item {Grp. gram.:m.}
\end{itemize}
Acto de desaparelhar.
\section{Desaparelhar}
\begin{itemize}
\item {Grp. gram.:v. t.}
\end{itemize}
\begin{itemize}
\item {Proveniência:(De \textunderscore des...\textunderscore  + \textunderscore aparelhar\textunderscore )}
\end{itemize}
Tirar o aparelho a.
Desguarnecer.
\section{Desaparelho}
\begin{itemize}
\item {Grp. gram.:m.}
\end{itemize}
Acto ou efeito de desaparelhar.
\section{Desaparição}
\begin{itemize}
\item {Grp. gram.:f.}
\end{itemize}
Acto de desaparecer.
\section{Desaplaudir}
\begin{itemize}
\item {Grp. gram.:v. t.}
\end{itemize}
\begin{itemize}
\item {Proveniência:(De \textunderscore des...\textunderscore  + \textunderscore aplaudir\textunderscore )}
\end{itemize}
Não aplaudir.
Desaprovar.
\section{Desaplauso}
\begin{itemize}
\item {Grp. gram.:m.}
\end{itemize}
Falta de aplauso.
Reprovação.
\section{Desaplicação}
\begin{itemize}
\item {Grp. gram.:f.}
\end{itemize}
Falta de aplicação.
Acto de tirar aquilo que estava aplicado.
\section{Desaplicadamente}
\begin{itemize}
\item {Grp. gram.:adv.}
\end{itemize}
Com desaplicação.
\section{Desaplicar}
\begin{itemize}
\item {Grp. gram.:v. t.}
\end{itemize}
\begin{itemize}
\item {Proveniência:(De \textunderscore des...\textunderscore  + \textunderscore aplicar\textunderscore )}
\end{itemize}
Desviar a aplicação de.
Tirar (aquilo que estava aplicado).
\section{Desapor}
\begin{itemize}
\item {Grp. gram.:v. t.}
\end{itemize}
\begin{itemize}
\item {Utilização:Prov.}
\end{itemize}
\begin{itemize}
\item {Utilização:minh.}
\end{itemize}
\begin{itemize}
\item {Proveniência:(De \textunderscore des...\textunderscore  + \textunderscore apor\textunderscore )}
\end{itemize}
Tirar o jugo a (bois).
Tirar do cabeçalho (a chavelha) e pô-la no chão, para que os bois descansem.
\section{Desapparecido}
\begin{itemize}
\item {Grp. gram.:adj.}
\end{itemize}
\begin{itemize}
\item {Proveniência:(De \textunderscore desapparecer\textunderscore )}
\end{itemize}
Que desappareceu.
\section{Desapparecimento}
\begin{itemize}
\item {Grp. gram.:m.}
\end{itemize}
O mesmo que \textunderscore desapparição\textunderscore .
\section{Desapparelhadamente}
\begin{itemize}
\item {Grp. gram.:adv.}
\end{itemize}
Sem apparelho.
Sem ornatos.
\section{Desapparelhamento}
\begin{itemize}
\item {Grp. gram.:m.}
\end{itemize}
Acto de desapparelhar.
\section{Desapparelhar}
\begin{itemize}
\item {Grp. gram.:v. t.}
\end{itemize}
\begin{itemize}
\item {Proveniência:(De \textunderscore des...\textunderscore  + \textunderscore apparelhar\textunderscore )}
\end{itemize}
Tirar o apparelho a.
Desguarnecer.
\section{Desapparelho}
\begin{itemize}
\item {Grp. gram.:m.}
\end{itemize}
Acto ou effeito de desapparelhar.
\section{Desapparição}
\begin{itemize}
\item {Grp. gram.:f.}
\end{itemize}
Acto de desapparecer.
\section{Desapplaudir}
\begin{itemize}
\item {Grp. gram.:v. t.}
\end{itemize}
\begin{itemize}
\item {Proveniência:(De \textunderscore des...\textunderscore  + \textunderscore applaudir\textunderscore )}
\end{itemize}
Não applaudir.
Desapprovar.
\section{Desapplauso}
\begin{itemize}
\item {Grp. gram.:m.}
\end{itemize}
Falta de applauso.
Reprovação.
\section{Desapplicação}
\begin{itemize}
\item {Grp. gram.:f.}
\end{itemize}
Falta de applicação.
Acto de tirar aquillo que estava applicado.
\section{Desapplicadamente}
\begin{itemize}
\item {Grp. gram.:adv.}
\end{itemize}
Com desapplicação.
\section{Desapplicar}
\begin{itemize}
\item {Grp. gram.:v. t.}
\end{itemize}
\begin{itemize}
\item {Proveniência:(De \textunderscore des...\textunderscore  + \textunderscore applicar\textunderscore )}
\end{itemize}
Desviar a applicação de.
Tirar (aquillo que estava applicado).
\section{Desappor}
\begin{itemize}
\item {Grp. gram.:v. t.}
\end{itemize}
\begin{itemize}
\item {Utilização:Prov.}
\end{itemize}
\begin{itemize}
\item {Utilização:minh.}
\end{itemize}
\begin{itemize}
\item {Proveniência:(De \textunderscore des...\textunderscore  + \textunderscore appor\textunderscore )}
\end{itemize}
Tirar o jugo a (bois).
Tirar do cabeçalho (a chavelha) e pô-la no chão, para que os bois descansem.
\section{Desapprender}
\begin{itemize}
\item {Grp. gram.:v. t.}
\end{itemize}
\begin{itemize}
\item {Proveniência:(De \textunderscore des...\textunderscore  + \textunderscore apprender\textunderscore )}
\end{itemize}
Esquecer-se de (aquillo que se tinha apprendido): \textunderscore desapprender gymnástica\textunderscore .
\section{Desapprovação}
\begin{itemize}
\item {Grp. gram.:f.}
\end{itemize}
Acto de desapprovar.
\section{Desapprovadamente}
\begin{itemize}
\item {Grp. gram.:adv.}
\end{itemize}
Com desapprovação.
\section{Desapprovador}
\begin{itemize}
\item {Grp. gram.:adj.}
\end{itemize}
\begin{itemize}
\item {Grp. gram.:M.}
\end{itemize}
Que desapprova.
Aquelle que desapprova.
\section{Desapprovar}
\begin{itemize}
\item {Grp. gram.:v. t.}
\end{itemize}
Não approvar.
Reprovar.
\section{Desapprovativo}
\begin{itemize}
\item {Grp. gram.:adj.}
\end{itemize}
\begin{itemize}
\item {Proveniência:(De \textunderscore des...\textunderscore  + \textunderscore approvativo\textunderscore )}
\end{itemize}
Em que há desapprovação.
\section{Desaprazer}
\begin{itemize}
\item {Grp. gram.:v. i.}
\end{itemize}
Não aprazer.
Desagradar; causar contrariedade.
\section{Desaprazível}
\begin{itemize}
\item {Grp. gram.:adj.}
\end{itemize}
\begin{itemize}
\item {Proveniência:(De \textunderscore des...\textunderscore  + \textunderscore aprazível\textunderscore )}
\end{itemize}
Que desapraz.
\section{Desapreciar}
\begin{itemize}
\item {Grp. gram.:v. t.}
\end{itemize}
\begin{itemize}
\item {Proveniência:(De \textunderscore des...\textunderscore  + \textunderscore apreciar\textunderscore )}
\end{itemize}
Não dar apreço a.
Amesquinhar.
\section{Desapreço}
\begin{itemize}
\item {Grp. gram.:m.}
\end{itemize}
Falta de apreço.
Menosprêzo.
\section{Desaprender}
\begin{itemize}
\item {Grp. gram.:v. t.}
\end{itemize}
\begin{itemize}
\item {Proveniência:(De \textunderscore des...\textunderscore  + \textunderscore aprender\textunderscore )}
\end{itemize}
Esquecer-se de (aquilo que se tinha aprendido): \textunderscore desaprender ginástica\textunderscore .
\section{Desapressar-se}
\begin{itemize}
\item {Grp. gram.:v. p.}
\end{itemize}
\begin{itemize}
\item {Proveniência:(De \textunderscore des...\textunderscore  + \textunderscore apressar\textunderscore )}
\end{itemize}
Deixar de têr pressa.
Fazer-se vagaroso.
\section{Desapresto}
\begin{itemize}
\item {Grp. gram.:m.}
\end{itemize}
Falta de apresto.
\section{Desaprimorado}
\begin{itemize}
\item {Grp. gram.:adj.}
\end{itemize}
\begin{itemize}
\item {Proveniência:(De \textunderscore des...\textunderscore  + \textunderscore aprimorado\textunderscore )}
\end{itemize}
Que não tem primor.
Indelicado.
\section{Desapropósito}
\textunderscore m.\textunderscore  (e der.)
(V. \textunderscore despropósito\textunderscore , etc.)
\section{Desapropriação}
\begin{itemize}
\item {Grp. gram.:f.}
\end{itemize}
Acto ou effeito de desapropriar.
\section{Desapropriamento}
\begin{itemize}
\item {Grp. gram.:m.}
\end{itemize}
O mesmo que \textunderscore desapropriação\textunderscore .
\section{Desapropriar}
\begin{itemize}
\item {Grp. gram.:v. t.}
\end{itemize}
\begin{itemize}
\item {Utilização:Fig.}
\end{itemize}
\begin{itemize}
\item {Proveniência:(De \textunderscore des...\textunderscore  + \textunderscore apropriar\textunderscore )}
\end{itemize}
Privar da propriedade.
Desapossar.
Usar impropriamente: \textunderscore desapropriar um vocábulo\textunderscore .
\section{Desaprovação}
\begin{itemize}
\item {Grp. gram.:f.}
\end{itemize}
Acto de desaprovar.
\section{Desaprovadamente}
\begin{itemize}
\item {Grp. gram.:adv.}
\end{itemize}
Com desaprovação.
\section{Desaprovador}
\begin{itemize}
\item {Grp. gram.:adj.}
\end{itemize}
\begin{itemize}
\item {Grp. gram.:M.}
\end{itemize}
Que desaprova.
Aquele que desaprova.
\section{Desaprovar}
\begin{itemize}
\item {Grp. gram.:v. t.}
\end{itemize}
Não aprovar.
Reprovar.
\section{Desaprovativo}
\begin{itemize}
\item {Grp. gram.:adj.}
\end{itemize}
\begin{itemize}
\item {Proveniência:(De \textunderscore des...\textunderscore  + \textunderscore aprovativo\textunderscore )}
\end{itemize}
Em que há desaprovação.
\section{Desaproveitadamente}
\begin{itemize}
\item {Grp. gram.:adv.}
\end{itemize}
Sem proveito.
\section{Desaproveitamento}
\begin{itemize}
\item {Grp. gram.:m.}
\end{itemize}
Acto ou effeito de desaproveitar.
\section{Desaproveitar}
\begin{itemize}
\item {Grp. gram.:v. t.}
\end{itemize}
Não aproveitar.
\section{Desaproximar}
\begin{itemize}
\item {fónica:ci}
\end{itemize}
\begin{itemize}
\item {Grp. gram.:v. t.}
\end{itemize}
\begin{itemize}
\item {Proveniência:(De \textunderscore des...\textunderscore  + \textunderscore aproximar\textunderscore )}
\end{itemize}
Afastar.
Separar.
Distanciar.
\section{Desaprumar}
\begin{itemize}
\item {Grp. gram.:v. t.}
\end{itemize}
\begin{itemize}
\item {Utilização:Fig.}
\end{itemize}
\begin{itemize}
\item {Grp. gram.:V. i.}
\end{itemize}
\begin{itemize}
\item {Proveniência:(De \textunderscore des...\textunderscore  + \textunderscore aprumar\textunderscore )}
\end{itemize}
Desviar do prumo.
Aviltar.
Desviar-se do prumo.
\section{Desaprumo}
\begin{itemize}
\item {Grp. gram.:m.}
\end{itemize}
Desvio do prumo.
Effeito de desaprumar.
\section{Desapurado}
\begin{itemize}
\item {Grp. gram.:adj.}
\end{itemize}
Feito com desapuro.
\section{Desapuro}
\begin{itemize}
\item {Grp. gram.:m.}
\end{itemize}
Falta de apuro; pouco cuidado.
\section{Desaquartelar}
\begin{itemize}
\item {Grp. gram.:v. t.}
\end{itemize}
\begin{itemize}
\item {Proveniência:(De \textunderscore des...\textunderscore  + \textunderscore aquartelar\textunderscore )}
\end{itemize}
Tirar de quartel; desalojar.
\section{Desaquecer}
\begin{itemize}
\item {fónica:qué}
\end{itemize}
\begin{itemize}
\item {Grp. gram.:v. t.}
\end{itemize}
\begin{itemize}
\item {Proveniência:(De \textunderscore des...\textunderscore  + \textunderscore aquecer\textunderscore )}
\end{itemize}
Fazer esfriar.
\section{Desaquinhoar}
\begin{itemize}
\item {Grp. gram.:v. t.}
\end{itemize}
\begin{itemize}
\item {Proveniência:(De \textunderscore des...\textunderscore  + \textunderscore aquinhoar\textunderscore )}
\end{itemize}
Privar de quinhão.
\section{Desar}
\begin{itemize}
\item {Grp. gram.:m.}
\end{itemize}
\begin{itemize}
\item {Proveniência:(De \textunderscore des...\textunderscore  + \textunderscore ar\textunderscore . Cp. \textunderscore desaire\textunderscore )}
\end{itemize}
Acto indecoroso.
Falta de elegância; desaire.
Mancha.
Desventura.
\section{Desaranhar}
\begin{itemize}
\item {Grp. gram.:v. t.}
\end{itemize}
\begin{itemize}
\item {Utilização:Pop.}
\end{itemize}
\begin{itemize}
\item {Proveniência:(De \textunderscore des...\textunderscore  + \textunderscore aranha\textunderscore )}
\end{itemize}
Tirar teias de aranhas a.
\section{Desarar}
\begin{itemize}
\item {Grp. gram.:v. i.}
\end{itemize}
\begin{itemize}
\item {Grp. gram.:V. t.}
\end{itemize}
\begin{itemize}
\item {Utilização:Prov.}
\end{itemize}
\begin{itemize}
\item {Utilização:trasm.}
\end{itemize}
\begin{itemize}
\item {Utilização:Fig.}
\end{itemize}
\begin{itemize}
\item {Proveniência:(De \textunderscore des...\textunderscore  + \textunderscore aro\textunderscore )}
\end{itemize}
Despegar-se, (falando-se do casco das bêstas).
Fazer caír (os cascos):«\textunderscore Succede caír o humor em tal quantidade, que lhes desara os cascos.\textunderscore »Galvão, \textunderscore Alveitaria\textunderscore , 550.
Tirar os aros ou arcos de (pipas ou tonéis).
Desarranjar, pôr em desordem.
\section{Desarborizar}
\begin{itemize}
\item {Grp. gram.:v. i.}
\end{itemize}
\begin{itemize}
\item {Proveniência:(De \textunderscore des...\textunderscore  + \textunderscore arborizar\textunderscore )}
\end{itemize}
Privar de árvores.
Cortar as árvores de.
\section{Desarcar}
\begin{itemize}
\item {Grp. gram.:v. t.}
\end{itemize}
\begin{itemize}
\item {Proveniência:(De \textunderscore des...\textunderscore  + \textunderscore arcar\textunderscore )}
\end{itemize}
Tirar os arcos de.
Desconjuntar.
\section{Desarear}
\begin{itemize}
\item {Grp. gram.:v. t.}
\end{itemize}
\begin{itemize}
\item {Proveniência:(De \textunderscore des...\textunderscore  + \textunderscore arear\textunderscore )}
\end{itemize}
Limpar de areia; tirar a areia a.
\section{Desaristado}
\begin{itemize}
\item {Grp. gram.:adj.}
\end{itemize}
\begin{itemize}
\item {Utilização:Pop.}
\end{itemize}
\begin{itemize}
\item {Proveniência:(De \textunderscore des...\textunderscore  + lat. \textunderscore arista\textunderscore )}
\end{itemize}
Diz-se do casulo, cujas válvulas não têm praganas como succede no escalracho, no milho, etc.
\section{Desarmação}
\begin{itemize}
\item {Grp. gram.:f.}
\end{itemize}
Acto de desarmar.
\section{Desarmador}
\begin{itemize}
\item {Grp. gram.:m.}
\end{itemize}
\begin{itemize}
\item {Utilização:P. us.}
\end{itemize}
\begin{itemize}
\item {Proveniência:(De \textunderscore desarmar\textunderscore )}
\end{itemize}
Aquelle que desarma.
O mesmo que \textunderscore gatilho\textunderscore .
\section{Desarmamento}
\begin{itemize}
\item {Grp. gram.:m.}
\end{itemize}
Acto ou effeito de desarmar.
Reducção ou licenceamento de tropas.
\section{Desarmar}
\begin{itemize}
\item {Grp. gram.:v. t.}
\end{itemize}
\begin{itemize}
\item {Grp. gram.:V. i.}
\end{itemize}
\begin{itemize}
\item {Proveniência:(De \textunderscore des...\textunderscore  + \textunderscore armar\textunderscore )}
\end{itemize}
Tirar as armas a: \textunderscore desarmar o assassino\textunderscore .
Desguarnecer de fôrça militar: \textunderscore desarmar uma praça\textunderscore .
Tirar os meios de defesa ou de ataque a.
Tirar a armadura a.
Applacar: \textunderscore aquellas palavras desarmaram-me\textunderscore .
Frustrar.
Pôr no descanso o cão de (uma arma de cano).
Tirar os adornos, a armação, de: \textunderscore desarmar uma sala\textunderscore .
Desadornar.
Desapparelhar; desmanchar: \textunderscore desarmar um bailéu\textunderscore .
Licencear ou reduzir (tropas).
Largar as armas; pôr de lado o armamento.
\section{Desaromar}
\begin{itemize}
\item {Grp. gram.:v. t.}
\end{itemize}
\begin{itemize}
\item {Proveniência:(De \textunderscore des...\textunderscore  + \textunderscore aroma\textunderscore )}
\end{itemize}
O mesmo que \textunderscore desaromatizar\textunderscore .
\section{Desaromatizar}
\begin{itemize}
\item {Grp. gram.:v. t.}
\end{itemize}
Fazer perder o aroma a.
Tirar o aroma de.
\section{Desarquear}
\begin{itemize}
\item {Grp. gram.:v. i.}
\end{itemize}
\begin{itemize}
\item {Proveniência:(De \textunderscore des...\textunderscore  + \textunderscore arquear\textunderscore )}
\end{itemize}
Tirar a fórma de arco a.
Tirar os arcos de: \textunderscore desarquear um tonel\textunderscore .
\section{Desarraigamento}
\begin{itemize}
\item {Grp. gram.:m.}
\end{itemize}
Acto ou effeito de desarraigar.
\section{Desarraigar}
\begin{itemize}
\item {Grp. gram.:v. t.}
\end{itemize}
\begin{itemize}
\item {Utilização:Fig.}
\end{itemize}
\begin{itemize}
\item {Proveniência:(De \textunderscore des...\textunderscore  + \textunderscore arraigar\textunderscore )}
\end{itemize}
Arrancar pela raiz.
Tirar inteiramente.
Extirpar: \textunderscore desarraigar preconceitos\textunderscore .
Destruir.
\section{Desarrancar}
\begin{itemize}
\item {Grp. gram.:v. i.}
\end{itemize}
\begin{itemize}
\item {Utilização:Ant.}
\end{itemize}
Investir; irromper. Cf. \textunderscore Port. Mon. Hist., Script.\textunderscore , 279.
(Cp. \textunderscore arrancar\textunderscore )
\section{Desarranchar}
\begin{itemize}
\item {Grp. gram.:v. t.}
\end{itemize}
\begin{itemize}
\item {Grp. gram.:V. i.}
\end{itemize}
\begin{itemize}
\item {Proveniência:(De \textunderscore des...\textunderscore  + \textunderscore arranchar\textunderscore )}
\end{itemize}
Tirar do rancho.
Separar-se do rancho.
\section{Desarranjador}
\begin{itemize}
\item {Grp. gram.:m.  e  adj.}
\end{itemize}
Aquillo ou aquelle que desarranja.
\section{Desarranjar}
\begin{itemize}
\item {Grp. gram.:v. t.}
\end{itemize}
\begin{itemize}
\item {Proveniência:(De \textunderscore des...\textunderscore  + \textunderscore arranjar\textunderscore )}
\end{itemize}
Pôr em desordem.
Desconcertar.
Embaraçar.
\section{Desarranjo}
\begin{itemize}
\item {Grp. gram.:m.}
\end{itemize}
\begin{itemize}
\item {Utilização:Pop.}
\end{itemize}
\begin{itemize}
\item {Proveniência:(De \textunderscore des...\textunderscore  + \textunderscore arranjo\textunderscore )}
\end{itemize}
Falta de arranjo.
Desordem.
Contratempo.
O mesmo que \textunderscore abôrto\textunderscore .
\section{Desarrazoadamente}
\begin{itemize}
\item {Grp. gram.:adv.}
\end{itemize}
De modo desarrazoado.
\section{Desarrazoado}
\begin{itemize}
\item {Grp. gram.:adj.}
\end{itemize}
\begin{itemize}
\item {Proveniência:(De \textunderscore des...\textunderscore  + \textunderscore arrazoado\textunderscore )}
\end{itemize}
Que não tem razão.
Injusto; despropositado.
\section{Desarrazoamento}
\begin{itemize}
\item {Grp. gram.:m.}
\end{itemize}
Acto de desarrazoar.
\section{Desarrazoar}
\begin{itemize}
\item {Grp. gram.:v. i.}
\end{itemize}
\begin{itemize}
\item {Proveniência:(De \textunderscore des...\textunderscore  + \textunderscore arrazoar\textunderscore )}
\end{itemize}
Proceder ou falar sem razão, ou sem bom senso.
Disparatar.
\section{Desarrear}
\begin{itemize}
\item {Grp. gram.:v. t.}
\end{itemize}
\begin{itemize}
\item {Proveniência:(De \textunderscore des...\textunderscore  + \textunderscore arrear\textunderscore )}
\end{itemize}
Tirar os arreios a.
\section{Desarredondar}
\begin{itemize}
\item {Grp. gram.:v. t.}
\end{itemize}
\begin{itemize}
\item {Proveniência:(De \textunderscore des...\textunderscore  + \textunderscore arredondar\textunderscore )}
\end{itemize}
Tirar a fôrma redonda a.
\section{Desarregaçar}
\begin{itemize}
\item {Grp. gram.:v. t.}
\end{itemize}
\begin{itemize}
\item {Proveniência:(De \textunderscore des...\textunderscore  + \textunderscore arregaçar\textunderscore )}
\end{itemize}
Soltar, fazer descer ou cair (aquillo que estava arregaçado): \textunderscore desarregaçar as calças\textunderscore .
\section{Desarrenegar-se}
\begin{itemize}
\item {Grp. gram.:v. p.}
\end{itemize}
\begin{itemize}
\item {Utilização:Fam.}
\end{itemize}
O mesmo que [[desagastar-se|desgastar]].
\section{Desarrimar}
\begin{itemize}
\item {Grp. gram.:v. t.}
\end{itemize}
\begin{itemize}
\item {Proveniência:(De \textunderscore des...\textunderscore  + \textunderscore arrimar\textunderscore )}
\end{itemize}
Desencostar.
Abandonar.
\section{Desarrimo}
\begin{itemize}
\item {Grp. gram.:m.}
\end{itemize}
Falta de arrimo.
\section{Desarrisca}
\begin{itemize}
\item {Grp. gram.:f.}
\end{itemize}
Acto de desarriscar.
\section{Desarriscar}
\begin{itemize}
\item {Grp. gram.:v. t.}
\end{itemize}
\begin{itemize}
\item {Utilização:Fam.}
\end{itemize}
O mesmo que \textunderscore deriscar\textunderscore .
Apagar o risco ou a nota de (uma dívida na tenda ou na taberna).
(Por \textunderscore desriscar\textunderscore , de \textunderscore des...\textunderscore  + \textunderscore risco\textunderscore )
\section{Desarrochar}
\begin{itemize}
\item {Grp. gram.:v. t.}
\end{itemize}
\begin{itemize}
\item {Proveniência:(De \textunderscore des...\textunderscore  + \textunderscore arrochar\textunderscore )}
\end{itemize}
Desapertar (aquillo que estava arrochado): \textunderscore desarrochar um jumento\textunderscore .
\section{Desarrolhar}
\begin{itemize}
\item {Grp. gram.:v. t.}
\end{itemize}
\begin{itemize}
\item {Proveniência:(De \textunderscore des...\textunderscore  + \textunderscore arrolhar\textunderscore )}
\end{itemize}
Tirar a rolha a; abrir (aquillo que tinha rolha): \textunderscore desarrolhar uma garrafa\textunderscore .
\section{Desarrotar}
\begin{itemize}
\item {Grp. gram.:v.}
\end{itemize}
\begin{itemize}
\item {Utilização:t. Náut.}
\end{itemize}
Tirar a rotadura a.
\section{Desarrufar}
\begin{itemize}
\item {Grp. gram.:v. t.}
\end{itemize}
\begin{itemize}
\item {Utilização:Fam.}
\end{itemize}
\begin{itemize}
\item {Proveniência:(De \textunderscore des...\textunderscore  + \textunderscore arrufar\textunderscore )}
\end{itemize}
Reconciliar.
\section{Desarrufo}
\begin{itemize}
\item {Grp. gram.:m.}
\end{itemize}
Acto de desarrufar.
\section{Desarrugamento}
\begin{itemize}
\item {Grp. gram.:m.}
\end{itemize}
Acto ou effeito de desarrugar.
\section{Desarrugar}
\begin{itemize}
\item {Grp. gram.:v. t.}
\end{itemize}
O mesmo que \textunderscore desenrugar\textunderscore .
\section{Desarrumação}
\begin{itemize}
\item {Grp. gram.:f.}
\end{itemize}
Acto ou effeito de desarrumar.
\section{Desarrumadamente}
\begin{itemize}
\item {Grp. gram.:adv.}
\end{itemize}
\begin{itemize}
\item {Proveniência:(De \textunderscore desarrumar\textunderscore )}
\end{itemize}
Em desordem.
\section{Desarrumar}
\begin{itemize}
\item {Grp. gram.:v. t.}
\end{itemize}
\begin{itemize}
\item {Proveniência:(De \textunderscore des...\textunderscore  + \textunderscore arrumar\textunderscore )}
\end{itemize}
Tirar do arrumo.
Desordenar; desarranjar: \textunderscore desarrumar uma livraria\textunderscore .
\section{Desarticulação}
\begin{itemize}
\item {Grp. gram.:f.}
\end{itemize}
Acto ou effeito de desarticular.
\section{Desarticular}
\begin{itemize}
\item {Grp. gram.:v. t.}
\end{itemize}
\begin{itemize}
\item {Proveniência:(De \textunderscore des...\textunderscore  + \textunderscore articular\textunderscore )}
\end{itemize}
Cortar pela articulação.
Desconjuntar: \textunderscore desarticular uma perna\textunderscore .
\section{Desartificioso}
\begin{itemize}
\item {Grp. gram.:adj.}
\end{itemize}
\begin{itemize}
\item {Proveniência:(De \textunderscore des...\textunderscore  + \textunderscore artificioso\textunderscore )}
\end{itemize}
Em que não há artifício.
Simples, modesto, singelo.
\section{Desarvorado}
\begin{itemize}
\item {Grp. gram.:adj.}
\end{itemize}
\begin{itemize}
\item {Utilização:Fam.}
\end{itemize}
\begin{itemize}
\item {Proveniência:(De \textunderscore desarvorar\textunderscore )}
\end{itemize}
Que fugiu desordenadamente.
\section{Desarvoramento}
\begin{itemize}
\item {Grp. gram.:m.}
\end{itemize}
Acto ou effeito de desarvorar.
\section{Desarvorar}
\begin{itemize}
\item {Grp. gram.:v. t.}
\end{itemize}
\begin{itemize}
\item {Utilização:Náut.}
\end{itemize}
\begin{itemize}
\item {Grp. gram.:V. i.}
\end{itemize}
\begin{itemize}
\item {Utilização:Fam.}
\end{itemize}
\begin{itemize}
\item {Proveniência:(De \textunderscore des...\textunderscore  + \textunderscore arvorar\textunderscore )}
\end{itemize}
Abater (aquillo que estava arvorado).
Tirar mastros e enxárcias a.
Desapparelhar.
Abalar ou fugir desordenadamente, á tôa.
\section{Desasadamente}
\begin{itemize}
\item {Grp. gram.:adv.}
\end{itemize}
De modo desasado; sem jeito.
\section{Desàsado}
\begin{itemize}
\item {Grp. gram.:adj.}
\end{itemize}
\begin{itemize}
\item {Utilização:Ant.}
\end{itemize}
\begin{itemize}
\item {Proveniência:(De \textunderscore desaso\textunderscore )}
\end{itemize}
Que tem desaso; que não é jeitoso.
Imprópio.
\section{Desasar}
\begin{itemize}
\item {Grp. gram.:v. t.}
\end{itemize}
\begin{itemize}
\item {Proveniência:(De \textunderscore des...\textunderscore  + \textunderscore asa\textunderscore )}
\end{itemize}
Partir ou abater as asas de.
Bater.
Derrear.
\section{Desasir}
\begin{itemize}
\item {Grp. gram.:v. t.}
\end{itemize}
\begin{itemize}
\item {Utilização:Des.}
\end{itemize}
\begin{itemize}
\item {Proveniência:(De \textunderscore des...\textunderscore  + \textunderscore asir\textunderscore )}
\end{itemize}
Soltar da mão, largar. Cf. Latino, \textunderscore Camões\textunderscore , p. 115.
\section{Desasnar}
\begin{itemize}
\item {Grp. gram.:v. t.}
\end{itemize}
\begin{itemize}
\item {Proveniência:(De \textunderscore des...\textunderscore  + \textunderscore asno\textunderscore )}
\end{itemize}
Dar tino a.
Dissipar a ignorância de.
Desilludir.
\section{Desaso}
\begin{itemize}
\item {Grp. gram.:m.}
\end{itemize}
\begin{itemize}
\item {Proveniência:(De \textunderscore des...\textunderscore  + \textunderscore aso\textunderscore )}
\end{itemize}
Falta de aso.
Inaptidão.
Descuido.
\section{Desassanhar}
\begin{itemize}
\item {Grp. gram.:v. t.}
\end{itemize}
\begin{itemize}
\item {Proveniência:(De \textunderscore des...\textunderscore  + \textunderscore assanhar\textunderscore )}
\end{itemize}
Applacar a sanha de.
Tranquillizar.
\section{Desassazonado}
\begin{itemize}
\item {Grp. gram.:adj.}
\end{itemize}
\begin{itemize}
\item {Proveniência:(De \textunderscore des...\textunderscore  + \textunderscore assazonado\textunderscore )}
\end{itemize}
Inopportuno.
Que vem fóra de tempo.
Verde.
\section{Desassear}
\begin{itemize}
\item {Grp. gram.:v. t.}
\end{itemize}
\begin{itemize}
\item {Proveniência:(De \textunderscore des...\textunderscore  + \textunderscore assear\textunderscore )}
\end{itemize}
Sujar.
Tirar a limpeza de.
\section{Desassegar}
\textunderscore v. t.\textunderscore  (e der.)
(Contr. ant. de \textunderscore desassossegar\textunderscore , etc.)
\section{Desasseio}
\begin{itemize}
\item {Grp. gram.:m.}
\end{itemize}
Falta de asseio.
\section{Desasselvajar}
\begin{itemize}
\item {Grp. gram.:v. t.}
\end{itemize}
\begin{itemize}
\item {Proveniência:(De \textunderscore des...\textunderscore  + \textunderscore asselvajar\textunderscore )}
\end{itemize}
Tirar do estado selvagem.
\section{Desassemelhar}
\begin{itemize}
\item {Grp. gram.:v. t.}
\end{itemize}
Tornar dessemelhante.
\section{Desassenhorear}
\begin{itemize}
\item {Grp. gram.:v. t.}
\end{itemize}
\begin{itemize}
\item {Proveniência:(De \textunderscore des...\textunderscore  + \textunderscore assenhorear\textunderscore )}
\end{itemize}
Desapossar.
Tirar a qualidade de senhor a.
\section{Desassestar}
\begin{itemize}
\item {Grp. gram.:v. t.}
\end{itemize}
\begin{itemize}
\item {Proveniência:(De \textunderscore des...\textunderscore  + \textunderscore assestar\textunderscore )}
\end{itemize}
Deslocar ou alterar a direcção de (aquillo que estava assestado).
\section{Desassimilação}
\begin{itemize}
\item {Grp. gram.:f.}
\end{itemize}
Acto de desassimilar.
\section{Desassimilador}
\begin{itemize}
\item {Grp. gram.:adj.}
\end{itemize}
Que desassimila.
\section{Desassimilar}
\begin{itemize}
\item {Grp. gram.:v. t.}
\end{itemize}
\begin{itemize}
\item {Proveniência:(De \textunderscore des...\textunderscore  + \textunderscore assimilar\textunderscore )}
\end{itemize}
Tirar a assimilação a. Cf. Camillo, \textunderscore Vinho do Porto\textunderscore , 59.
\section{Desassimilhar}
\begin{itemize}
\item {Grp. gram.:v. t.}
\end{itemize}
(V.desassemelhar)
\section{Desassisadamente}
\begin{itemize}
\item {Grp. gram.:adv.}
\end{itemize}
De modo desassisado.
\section{Desassisar}
\begin{itemize}
\item {Grp. gram.:v. t.}
\end{itemize}
\begin{itemize}
\item {Proveniência:(De \textunderscore des...\textunderscore  + \textunderscore assisado\textunderscore )}
\end{itemize}
Tirar o siso a.
Tornar louco ou maníaco.
\section{Desassistir}
\begin{itemize}
\item {Grp. gram.:v. t.}
\end{itemize}
\begin{itemize}
\item {Grp. gram.:V. i.}
\end{itemize}
Não assistir a.
Desauxiliar.
Não assistir.
Deixar de assistir.
Retirar auxílio ou protecção. Cf. Filinto, XXI, 37.
\section{Desassociar}
\begin{itemize}
\item {Grp. gram.:v. t.}
\end{itemize}
\begin{itemize}
\item {Proveniência:(De \textunderscore des...\textunderscore  + \textunderscore associar\textunderscore )}
\end{itemize}
Desligar (aquelle ou aquillo que estava associado).
\section{Desassolvar}
\begin{itemize}
\item {Grp. gram.:v. t.}
\end{itemize}
\begin{itemize}
\item {Utilização:Des.}
\end{itemize}
Tirar (de canhões) a pólvora húmida. Cf. Isla, \textunderscore Artilharia\textunderscore , 66.
(Cp. \textunderscore ensolvar\textunderscore )
\section{Desassombradamente}
\begin{itemize}
\item {Grp. gram.:adv.}
\end{itemize}
Com desassombro.
\section{Desassombramento}
\begin{itemize}
\item {Grp. gram.:m.}
\end{itemize}
\begin{itemize}
\item {Utilização:Des.}
\end{itemize}
(V.desassombro)
\section{Desassombrar}
\begin{itemize}
\item {Grp. gram.:v. t.}
\end{itemize}
\begin{itemize}
\item {Utilização:Fig.}
\end{itemize}
\begin{itemize}
\item {Proveniência:(De \textunderscore des...\textunderscore  + \textunderscore assombrar\textunderscore )}
\end{itemize}
Tirar da sombra.
Illuminar.
Desanuvear.
Tirar a tristeza a.
Livrar do susto; tranquillizar.
\section{Desassombro}
\begin{itemize}
\item {Grp. gram.:m.}
\end{itemize}
\begin{itemize}
\item {Proveniência:(De \textunderscore des...\textunderscore  + \textunderscore assombro\textunderscore )}
\end{itemize}
Franqueza.
Intrepidez.
Falta de assombro.
\section{Desassoreamento}
\begin{itemize}
\item {Grp. gram.:m.}
\end{itemize}
Acto de desassorear.
\section{Desassorear}
\begin{itemize}
\item {Grp. gram.:v. t.}
\end{itemize}
\begin{itemize}
\item {Proveniência:(De \textunderscore des...\textunderscore  + \textunderscore assorear\textunderscore )}
\end{itemize}
Tirar o assoreamento a.
\section{Desassossegadamente}
\begin{itemize}
\item {Grp. gram.:adv.}
\end{itemize}
Com desassossêgo.
\section{Desassossegador}
\begin{itemize}
\item {Grp. gram.:adj.}
\end{itemize}
\begin{itemize}
\item {Grp. gram.:M.}
\end{itemize}
Que desassossega.
Aquelle que desassossega.
\section{Desassossegar}
\begin{itemize}
\item {Grp. gram.:v. t.}
\end{itemize}
\begin{itemize}
\item {Proveniência:(De \textunderscore des...\textunderscore  + \textunderscore assossegar\textunderscore )}
\end{itemize}
Tirar o sossêgo a.
Inquietar.
\section{Desassossêgo}
\begin{itemize}
\item {Grp. gram.:m.}
\end{itemize}
\begin{itemize}
\item {Proveniência:(De \textunderscore desassossegar\textunderscore )}
\end{itemize}
Falta de sossêgo.
Inquietação.
Receio.
\section{Desassustadamente}
\begin{itemize}
\item {Grp. gram.:adv.}
\end{itemize}
Sem susto.
\section{Desassustar}
\begin{itemize}
\item {Grp. gram.:v. t.}
\end{itemize}
\begin{itemize}
\item {Proveniência:(De \textunderscore des...\textunderscore  + \textunderscore assustar\textunderscore )}
\end{itemize}
Livrar de susto.
Desassombrar.
\section{Desastradamente}
\begin{itemize}
\item {Grp. gram.:adv.}
\end{itemize}
De modo desastrado.
\section{Desastrado}
\begin{itemize}
\item {Grp. gram.:adj.}
\end{itemize}
\begin{itemize}
\item {Proveniência:(De \textunderscore desastre\textunderscore )}
\end{itemize}
Que redundou em desastre.
Funesto.
Resultante de um desastre.
Desgraçado.
Que não tem jeito para nada; desajeitado: \textunderscore é desastrada a minha criada\textunderscore .
Que não tem graça.
\section{Desastramento}
\begin{itemize}
\item {Grp. gram.:m.}
\end{itemize}
(V.desastre)
\section{Desastre}
\begin{itemize}
\item {Grp. gram.:m.}
\end{itemize}
\begin{itemize}
\item {Proveniência:(De \textunderscore des...\textunderscore  + \textunderscore astre\textunderscore )}
\end{itemize}
Desgraça.
Sinistro.
Fatalidade.
\section{Desastrosamente}
\begin{itemize}
\item {Grp. gram.:adv.}
\end{itemize}
De modo desastroso.
\section{Desastroso}
\begin{itemize}
\item {Grp. gram.:adj.}
\end{itemize}
\begin{itemize}
\item {Proveniência:(De \textunderscore desastre\textunderscore )}
\end{itemize}
Em que há desastre; que produz desastre: \textunderscore tempestade desastrosa\textunderscore .
\section{Desatabafadamente}
\begin{itemize}
\item {Grp. gram.:adv.}
\end{itemize}
\begin{itemize}
\item {Proveniência:(De \textunderscore desatabafar\textunderscore )}
\end{itemize}
Com desafôgo; com desassombro.
\section{Desatabafar}
\begin{itemize}
\item {Grp. gram.:v. t.}
\end{itemize}
\begin{itemize}
\item {Grp. gram.:V. i.}
\end{itemize}
\begin{itemize}
\item {Proveniência:(De \textunderscore des...\textunderscore  + \textunderscore atabafar\textunderscore )}
\end{itemize}
Desafogar, desassombrar.
Respirar ou falar livremente.
Desabafar.
\section{Desatacar}
\begin{itemize}
\item {Grp. gram.:v. t.}
\end{itemize}
\begin{itemize}
\item {Proveniência:(De \textunderscore des...\textunderscore  + \textunderscore atacar\textunderscore )}
\end{itemize}
Desprender a ataca de.
Desunir (aquillo que estava ligado por atacador).
Desabotoar.
Descarregar.
Despejar.
\section{Desatadamente}
\begin{itemize}
\item {Grp. gram.:adv.}
\end{itemize}
\begin{itemize}
\item {Proveniência:(De \textunderscore desatar\textunderscore )}
\end{itemize}
Com desembaraço.
Desordenadamente, sem ligação.
\section{Desatador}
\begin{itemize}
\item {Grp. gram.:m.}
\end{itemize}
Aquelle que desata.
\section{Desatadura}
\begin{itemize}
\item {Grp. gram.:f.}
\end{itemize}
Acto ou effeito de desatar.
\section{Desatamento}
\begin{itemize}
\item {Grp. gram.:m.}
\end{itemize}
O mesmo que \textunderscore desatadura\textunderscore .
\section{Desatar}
\begin{itemize}
\item {Grp. gram.:v. t.}
\end{itemize}
\begin{itemize}
\item {Utilização:Fig.}
\end{itemize}
\begin{itemize}
\item {Proveniência:(De \textunderscore des...\textunderscore  + \textunderscore atar\textunderscore )}
\end{itemize}
Desprender.
Desligar.
Desfazer (um nó).
Explicar.
Libertar.
Rescindir: \textunderscore desatar uma combinação\textunderscore .
Começar com intensidade ou ímpeto: \textunderscore e desatou a berrar\textunderscore .
\section{Desatarrachar}
\begin{itemize}
\item {Grp. gram.:v. t.}
\end{itemize}
\begin{itemize}
\item {Proveniência:(De \textunderscore des...\textunderscore  + \textunderscore atarrachar\textunderscore )}
\end{itemize}
Tirar a tarracha a; desaparafusar.
\section{Desatascar}
\begin{itemize}
\item {Grp. gram.:v. t.}
\end{itemize}
\begin{itemize}
\item {Proveniência:(De \textunderscore des...\textunderscore  + \textunderscore atascar\textunderscore )}
\end{itemize}
Livrar de atascadeiro.
Tirar da lama.
\section{Desataviadamente}
\begin{itemize}
\item {Grp. gram.:adv.}
\end{itemize}
Com desatavio.
\section{Desataviar}
\begin{itemize}
\item {Grp. gram.:v. t.}
\end{itemize}
\begin{itemize}
\item {Proveniência:(De \textunderscore des...\textunderscore  + \textunderscore ataviar\textunderscore )}
\end{itemize}
Tirar o atavio a; desadornar.
Despir.
\section{Desatavio}
\begin{itemize}
\item {Grp. gram.:m.}
\end{itemize}
\begin{itemize}
\item {Proveniência:(De \textunderscore des...\textunderscore  + \textunderscore atavio\textunderscore )}
\end{itemize}
Falta de atavio.
Desalinho.
\section{Desate}
\begin{itemize}
\item {Grp. gram.:m.}
\end{itemize}
Acto ou effeito de desatar.
Desfecho, desenlace. Cf. F. Alexandre Lobo, III, 406.
\section{Desatediar}
\begin{itemize}
\item {Grp. gram.:v. t.}
\end{itemize}
(V.desentediar)
\section{Desatemorizador}
\begin{itemize}
\item {Grp. gram.:m.}
\end{itemize}
Aquelle que desatemoriza.
\section{Desatemorizar}
\begin{itemize}
\item {Grp. gram.:v. t.}
\end{itemize}
\begin{itemize}
\item {Proveniência:(De \textunderscore des...\textunderscore  + \textunderscore atemorizar\textunderscore )}
\end{itemize}
Tirar o temor a; animar.
Dar coragem a.
\section{Desatenção}
\begin{itemize}
\item {Grp. gram.:f.}
\end{itemize}
Falta de atenção.
Descortesia.
\section{Desatencioso}
\begin{itemize}
\item {Grp. gram.:adj.}
\end{itemize}
\begin{itemize}
\item {Proveniência:(De \textunderscore des...\textunderscore  + \textunderscore atencioso\textunderscore )}
\end{itemize}
Que não dá atenção.
Descortês.
\section{Desatender}
\begin{itemize}
\item {Grp. gram.:v. t.}
\end{itemize}
\begin{itemize}
\item {Proveniência:(De \textunderscore des...\textunderscore  + \textunderscore atender\textunderscore )}
\end{itemize}
Não atender: \textunderscore desatender conselhos\textunderscore .
Desconsiderar.
\section{Desatendível}
\begin{itemize}
\item {Grp. gram.:adj.}
\end{itemize}
\begin{itemize}
\item {Proveniência:(De \textunderscore des...\textunderscore  + \textunderscore atendível\textunderscore )}
\end{itemize}
Que não é digno de atenção.
\section{Desatentadamente}
\begin{itemize}
\item {Grp. gram.:adv.}
\end{itemize}
(V.desatentamente)
\section{Desatentamente}
\begin{itemize}
\item {Grp. gram.:adv.}
\end{itemize}
\begin{itemize}
\item {Proveniência:(De \textunderscore desatento\textunderscore )}
\end{itemize}
Sem tento; com desatenção.
\section{Desatentar}
\begin{itemize}
\item {Grp. gram.:v. t.}
\end{itemize}
\begin{itemize}
\item {Proveniência:(De \textunderscore des...\textunderscore  + \textunderscore atentar\textunderscore )}
\end{itemize}
Não dar atenção a.
\section{Desatento}
\begin{itemize}
\item {Grp. gram.:adj.}
\end{itemize}
Distraido; abstracto.
Leviano.
\section{Desaterrar}
\begin{itemize}
\item {Grp. gram.:v. t.}
\end{itemize}
\begin{itemize}
\item {Proveniência:(De \textunderscore des...\textunderscore  + \textunderscore aterrar\textunderscore )}
\end{itemize}
Escavar.
Desfazer (atêrro).
Desobstruir ou aplanar (terreno).
\section{Desatêrro}
\begin{itemize}
\item {Grp. gram.:m.}
\end{itemize}
Acto ou effeito de desaterrar.
\section{Desatilado}
\begin{itemize}
\item {Grp. gram.:adj.}
\end{itemize}
Que não é atilado.
\section{Desatilhar}
\begin{itemize}
\item {Grp. gram.:v. t.}
\end{itemize}
Soltar dos atilhos.
Desatar.
\section{Desatinação}
\begin{itemize}
\item {Grp. gram.:f.}
\end{itemize}
Acto ou effeito de desatinar.
\section{Desatinadamente}
\begin{itemize}
\item {Grp. gram.:adv.}
\end{itemize}
Com desatino.
\section{Desatinar}
\begin{itemize}
\item {Grp. gram.:v. t.}
\end{itemize}
\begin{itemize}
\item {Grp. gram.:V. i.}
\end{itemize}
\begin{itemize}
\item {Proveniência:(De \textunderscore des...\textunderscore  + \textunderscore atinar\textunderscore )}
\end{itemize}
Tirar o tino a.
Tornar doido; insensato.
Perder o tino. Fazer ou dizer desatinos.
\section{Desatino}
\begin{itemize}
\item {Grp. gram.:m.}
\end{itemize}
\begin{itemize}
\item {Proveniência:(De \textunderscore desatinar\textunderscore )}
\end{itemize}
Falta de tino.
Acto ou palavras de desatinado.
\section{Desatolar}
\begin{itemize}
\item {Grp. gram.:v. t.}
\end{itemize}
O mesmo que \textunderscore desatascar\textunderscore .
\section{Desatordoar}
\begin{itemize}
\item {Grp. gram.:v. t.}
\end{itemize}
\begin{itemize}
\item {Proveniência:(De \textunderscore des...\textunderscore  + \textunderscore atordoar\textunderscore )}
\end{itemize}
Tirar de atordoamento.
\section{Desatracação}
\begin{itemize}
\item {Grp. gram.:f.}
\end{itemize}
Acto de \textunderscore desatracar\textunderscore .
\section{Desatracar}
\begin{itemize}
\item {Grp. gram.:v. t.}
\end{itemize}
\begin{itemize}
\item {Grp. gram.:V. i.}
\end{itemize}
\begin{itemize}
\item {Proveniência:(De \textunderscore des...\textunderscore  + \textunderscore atracar\textunderscore )}
\end{itemize}
Desamarrar.
Desprender.
Desprender-se, desamarrar-se.
\section{Desatrancar}
\begin{itemize}
\item {Grp. gram.:v. t.}
\end{itemize}
(V.destrancar)
\section{Desatravancar}
\begin{itemize}
\item {Grp. gram.:v. t.}
\end{itemize}
\begin{itemize}
\item {Proveniência:(De \textunderscore des...\textunderscore  + \textunderscore atravancar\textunderscore )}
\end{itemize}
Tirar as travancas a.
Desimpedir.
Desobstruir.
\section{Desatravessar}
\begin{itemize}
\item {Grp. gram.:v. t.}
\end{itemize}
\begin{itemize}
\item {Proveniência:(De \textunderscore des...\textunderscore  + \textunderscore atravessar\textunderscore )}
\end{itemize}
Tirar as travessas a.
Desembaraçar.
Desatravancar.
\section{Desatrelar}
\begin{itemize}
\item {Grp. gram.:v. t.}
\end{itemize}
\begin{itemize}
\item {Proveniência:(De \textunderscore des...\textunderscore  + \textunderscore atrelar\textunderscore )}
\end{itemize}
Desligar da trela.
Separar.
\section{Desatremado}
\begin{itemize}
\item {Grp. gram.:adj.}
\end{itemize}
\begin{itemize}
\item {Proveniência:(De \textunderscore desatremar\textunderscore )}
\end{itemize}
Que desatremou.
Desarvorado.
Tonto.
\section{Desatremar}
\begin{itemize}
\item {Grp. gram.:v. i.}
\end{itemize}
\begin{itemize}
\item {Proveniência:(De \textunderscore des...\textunderscore  + \textunderscore atremar\textunderscore )}
\end{itemize}
Perder o tino.
Desarvorar.
Desviar-se do bom caminho.
\section{Desattenção}
\begin{itemize}
\item {Grp. gram.:f.}
\end{itemize}
Falta de attenção.
Descortesia.
\section{Desattencioso}
\begin{itemize}
\item {Grp. gram.:adj.}
\end{itemize}
\begin{itemize}
\item {Proveniência:(De \textunderscore des...\textunderscore  + \textunderscore attencioso\textunderscore )}
\end{itemize}
Que não dá attenção.
Descortês.
\section{Desattender}
\begin{itemize}
\item {Grp. gram.:v. t.}
\end{itemize}
\begin{itemize}
\item {Proveniência:(De \textunderscore des...\textunderscore  + \textunderscore attender\textunderscore )}
\end{itemize}
Não attender: \textunderscore desattender conselhos\textunderscore .
Desconsiderar.
\section{Desattendível}
\begin{itemize}
\item {Grp. gram.:adj.}
\end{itemize}
\begin{itemize}
\item {Proveniência:(De \textunderscore des...\textunderscore  + \textunderscore attendível\textunderscore )}
\end{itemize}
Que não é digno de attenção.
\section{Desattentadamente}
\begin{itemize}
\item {Grp. gram.:adv.}
\end{itemize}
(V.desattentamente)
\section{Desattentamente}
\begin{itemize}
\item {Grp. gram.:adv.}
\end{itemize}
\begin{itemize}
\item {Proveniência:(De \textunderscore desattento\textunderscore )}
\end{itemize}
Sem tento; com desattenção.
\section{Desattentar}
\begin{itemize}
\item {Grp. gram.:v. t.}
\end{itemize}
\begin{itemize}
\item {Proveniência:(De \textunderscore des...\textunderscore  + \textunderscore attentar\textunderscore )}
\end{itemize}
Não dar attenção a.
\section{Desattento}
\begin{itemize}
\item {Grp. gram.:adj.}
\end{itemize}
Distrahido; abstracto.
Leviano.
\section{Desaugar}
\begin{itemize}
\item {Grp. gram.:v. t.}
\end{itemize}
\begin{itemize}
\item {Utilização:Prov.}
\end{itemize}
Dar a (uma criança ou a um animal) qualquer coisa de comida que vejam e appeteçam.
Tirar o aguamento a.
(Por \textunderscore desaguar\textunderscore , de \textunderscore des...\textunderscore  + \textunderscore aguar\textunderscore )
\section{Desaurido}
\begin{itemize}
\item {Grp. gram.:adj.}
\end{itemize}
\begin{itemize}
\item {Utilização:Prov.}
\end{itemize}
\begin{itemize}
\item {Proveniência:(De \textunderscore desaurir\textunderscore )}
\end{itemize}
Que desauriu.
Alucinado, estonteado.
\section{Desaurir}
\begin{itemize}
\item {Grp. gram.:v. i.}
\end{itemize}
\begin{itemize}
\item {Utilização:Prov.}
\end{itemize}
O mesmo que \textunderscore aurir\textunderscore , fugir.
\section{Desaustinado}
\begin{itemize}
\item {Grp. gram.:m.  e  adj.}
\end{itemize}
\begin{itemize}
\item {Utilização:Prov.}
\end{itemize}
Turbulento.
Inquieto.
Vadio.
Sem tino.
Desarvorado. Cf. Camillo, \textunderscore Corja\textunderscore , 197.
(Cp. \textunderscore austinado\textunderscore )
\section{Desautoração}
\begin{itemize}
\item {Grp. gram.:f.}
\end{itemize}
Acto ou effeito de desautorar.
\section{Desautorar}
\begin{itemize}
\item {Grp. gram.:v. t.}
\end{itemize}
\begin{itemize}
\item {Proveniência:(De \textunderscore des...\textunderscore  + \textunderscore autor\textunderscore )}
\end{itemize}
Privar de cargo ou dignidade, por castigo: \textunderscore desautorar um majór\textunderscore .
Desautorizar.
\section{Desautoridade}
\begin{itemize}
\item {Grp. gram.:f.}
\end{itemize}
\begin{itemize}
\item {Utilização:Ant.}
\end{itemize}
O mesmo que \textunderscore desautorização\textunderscore .
\section{Desautorização}
\begin{itemize}
\item {Grp. gram.:f.}
\end{itemize}
Acto ou effeito de desautorizar.
\section{Desautorizamento}
\begin{itemize}
\item {Grp. gram.:m.}
\end{itemize}
(V.desautorização)
\section{Desautorizar}
\begin{itemize}
\item {Grp. gram.:v. t.}
\end{itemize}
\begin{itemize}
\item {Proveniência:(De \textunderscore des...\textunderscore  + \textunderscore autorizar\textunderscore )}
\end{itemize}
Tirar o prestigio a; privar de autoridade, de crédito.
\section{Desauxiliado}
\begin{itemize}
\item {Grp. gram.:adj.}
\end{itemize}
Que não tem auxílio.
\section{Desauxiliar}
\begin{itemize}
\item {Grp. gram.:v. t.}
\end{itemize}
Privar de auxílio.
\section{Desavagar}
\begin{itemize}
\item {Grp. gram.:v. t.}
\end{itemize}
Arrancar (a ferradura), depois de se lhe cortarem os rebites.
\section{Desavantajoso}
\begin{itemize}
\item {Grp. gram.:adj.}
\end{itemize}
\begin{itemize}
\item {Proveniência:(De \textunderscore des...\textunderscore  + \textunderscore avantagem\textunderscore )}
\end{itemize}
Que não dá vantagem.
Desvantajoso. Cf. Filinto, XX, 106 e 177.
\section{Desavença}
\begin{itemize}
\item {Grp. gram.:f.}
\end{itemize}
\begin{itemize}
\item {Proveniência:(De \textunderscore des...\textunderscore  + \textunderscore avença\textunderscore )}
\end{itemize}
Quebra de bôas relações.
Contenda.
Inimizade.
\section{Desaventura}
\begin{itemize}
\item {Grp. gram.:f.}
\end{itemize}
\begin{itemize}
\item {Utilização:Ant.}
\end{itemize}
O mesmo que \textunderscore desventura\textunderscore . Cf. \textunderscore Eufrosina\textunderscore , 297.
\section{Desaventurado}
\begin{itemize}
\item {Grp. gram.:adj.}
\end{itemize}
O mesmo que \textunderscore desventurado\textunderscore .
\section{Desaventurável}
\begin{itemize}
\item {Grp. gram.:adj.}
\end{itemize}
\begin{itemize}
\item {Utilização:Ant.}
\end{itemize}
O mesmo que \textunderscore desventurado\textunderscore . Cf. \textunderscore Aulegrafia\textunderscore , 18.
\section{Desaverbar}
\begin{itemize}
\item {Grp. gram.:v. t.}
\end{itemize}
\begin{itemize}
\item {Proveniência:(De \textunderscore des...\textunderscore  + \textunderscore averbar\textunderscore )}
\end{itemize}
Cancellar.
Riscar.
\section{Desavergonhadamente}
\begin{itemize}
\item {Grp. gram.:adv.}
\end{itemize}
Sem vergonha.
\section{Desavergonhado}
\begin{itemize}
\item {Grp. gram.:adj.}
\end{itemize}
\begin{itemize}
\item {Proveniência:(De \textunderscore desavergonhar\textunderscore )}
\end{itemize}
Que não tem vergonha.
Descarado, insolente.
\section{Desavergonhamento}
\begin{itemize}
\item {Grp. gram.:m.}
\end{itemize}
(V.desvergonha)
\section{Desavergonhar}
\begin{itemize}
\item {Grp. gram.:v. t.}
\end{itemize}
\begin{itemize}
\item {Utilização:Des.}
\end{itemize}
Fazer perder o pundonor ou a vergonha a.
Tornar descarado, impudente.
(Cp. \textunderscore desvergonha\textunderscore )
\section{Desaveriguado}
\begin{itemize}
\item {Grp. gram.:adj.}
\end{itemize}
Não averiguado.
Duvidoso; incerto. Cf. Filinto, \textunderscore D. Man.\textunderscore , III, 244.
\section{Desavesso}
\begin{itemize}
\item {fónica:vê}
\end{itemize}
\begin{itemize}
\item {Grp. gram.:adj.}
\end{itemize}
\begin{itemize}
\item {Utilização:Prov.}
\end{itemize}
\begin{itemize}
\item {Utilização:trasm.}
\end{itemize}
\textunderscore Não sêr desavesso\textunderscore , não sêr mau de todo.
Poder servir.
(Cp. \textunderscore avesso\textunderscore )
\section{Desavèxar}
\begin{itemize}
\item {Grp. gram.:v. t.}
\end{itemize}
\begin{itemize}
\item {Proveniência:(De \textunderscore des...\textunderscore  + \textunderscore avèxar\textunderscore )}
\end{itemize}
Livrar de vèxame.
Tirar vèxame a. Cf. Filinto, \textunderscore D. Man.\textunderscore , II, 238.
\section{Desavezar}
\begin{itemize}
\item {Grp. gram.:v. t.}
\end{itemize}
\begin{itemize}
\item {Proveniência:(De \textunderscore des...\textunderscore  + \textunderscore acostumar\textunderscore )}
\end{itemize}
Desacostumar.
Tirar o vêzo a.
\section{Desavêzo}
\begin{itemize}
\item {Grp. gram.:m.}
\end{itemize}
Acto ou effeito de desavezar.
\section{Desaviamento}
\begin{itemize}
\item {Grp. gram.:m.}
\end{itemize}
\begin{itemize}
\item {Utilização:Des.}
\end{itemize}
Falta de aviamento.
\section{Desaviar}
\begin{itemize}
\item {Grp. gram.:v. t.}
\end{itemize}
\begin{itemize}
\item {Utilização:Des.}
\end{itemize}
\begin{itemize}
\item {Proveniência:(De \textunderscore des...\textunderscore  + \textunderscore aviar\textunderscore )}
\end{itemize}
Não aviar.
Desviar; impedir.
\section{Desavincar}
\textunderscore v. t.\textunderscore  (e der.)
O mesmo que \textunderscore desvincar\textunderscore . Cf. Camillo, \textunderscore Brasileira\textunderscore , 135; \textunderscore Maria da Fonte\textunderscore , 373; \textunderscore Corja\textunderscore , 195.
\section{Desavindo}
\begin{itemize}
\item {Grp. gram.:adj.}
\end{itemize}
\begin{itemize}
\item {Proveniência:(De \textunderscore desavir\textunderscore )}
\end{itemize}
Mal avindo; que anda em desavença.
Desacorde.
\section{Desavinhar}
\begin{itemize}
\item {Grp. gram.:v. i.}
\end{itemize}
\begin{itemize}
\item {Proveniência:(De \textunderscore des...\textunderscore  + \textunderscore avinhar\textunderscore )}
\end{itemize}
Diz-se das videiras ou dos cachos que soffreram desavinho.
\section{Desavinho}
\begin{itemize}
\item {Grp. gram.:m.}
\end{itemize}
\begin{itemize}
\item {Proveniência:(De \textunderscore desavinhar\textunderscore )}
\end{itemize}
Abôrto das flôres da videira; abôrto parcial do cacho, desenvolvendo-se desigualmente os bagos e não amadurecendo todos.
\section{Desavir}
\begin{itemize}
\item {Grp. gram.:v. t.}
\end{itemize}
\begin{itemize}
\item {Proveniência:(De \textunderscore des...\textunderscore  + \textunderscore avir\textunderscore )}
\end{itemize}
Indispor; criar desavenças entre.
\section{Desavisadamente}
\begin{itemize}
\item {Grp. gram.:adv.}
\end{itemize}
Com desaviso.
\section{Desavisamento}
\begin{itemize}
\item {Grp. gram.:m.}
\end{itemize}
Qualidade daquelle ou daquillo que é desavisado.
\section{Desavisar}
\begin{itemize}
\item {Grp. gram.:v. t.}
\end{itemize}
\begin{itemize}
\item {Proveniência:(De \textunderscore des...\textunderscore  + \textunderscore avisar\textunderscore )}
\end{itemize}
Dar contra-aviso a.
Tornar imprudente, leviano.
\section{Desaviso}
\begin{itemize}
\item {Grp. gram.:m.}
\end{itemize}
\begin{itemize}
\item {Proveniência:(De \textunderscore des...\textunderscore  + \textunderscore aviso\textunderscore )}
\end{itemize}
Contra-aviso.
Imprudência; leviandade.
\section{Desavistar}
\begin{itemize}
\item {Grp. gram.:v. t.}
\end{itemize}
\begin{itemize}
\item {Proveniência:(De \textunderscore des...\textunderscore  + \textunderscore avistar\textunderscore )}
\end{itemize}
Deixar de vêr; perder de vista.
\section{Desbabar}
\begin{itemize}
\item {Grp. gram.:v. t.}
\end{itemize}
\begin{itemize}
\item {Utilização:Des.}
\end{itemize}
\begin{itemize}
\item {Grp. gram.:V. i.}
\end{itemize}
\begin{itemize}
\item {Utilização:Prov.}
\end{itemize}
\begin{itemize}
\item {Utilização:trasm.}
\end{itemize}
\begin{itemize}
\item {Proveniência:(De \textunderscore des...\textunderscore  + \textunderscore babar\textunderscore )}
\end{itemize}
Limpar a baba de.
Perder a fé que tinha em alguém.
Desenganar-se.
\section{Desbadalar}
\begin{itemize}
\item {Grp. gram.:v. i.}
\end{itemize}
Deixar de badalar: emmudecer. Cf. Filinto, IX, 115.
\section{Desbagamento}
\begin{itemize}
\item {Grp. gram.:m.}
\end{itemize}
\begin{itemize}
\item {Utilização:Marn.}
\end{itemize}
Passagem da água, dos compartimentos de nível inferior para os de nível immediatamente superior, nas salinas do Algarve. Cf. \textunderscore Museu Techn.\textunderscore , 102.
\section{Desbagoar}
\begin{itemize}
\item {Grp. gram.:v. t.}
\end{itemize}
(V.esbagoar)
\section{Desbagulhar}
\begin{itemize}
\item {Grp. gram.:v. t.}
\end{itemize}
\begin{itemize}
\item {Proveniência:(De \textunderscore des...\textunderscore  + \textunderscore bagulho\textunderscore )}
\end{itemize}
Tirar o bagulho de.
\section{Desbalçar}
\begin{itemize}
\item {Grp. gram.:v. t.}
\end{itemize}
\begin{itemize}
\item {Proveniência:(De \textunderscore des...\textunderscore  + \textunderscore balçar\textunderscore )}
\end{itemize}
Cortar as balças a.
\section{Desbalisar}
\begin{itemize}
\item {Grp. gram.:v. t.}
\end{itemize}
O mesmo que \textunderscore desvalijar\textunderscore . Cf. Camillo, \textunderscore Caveira\textunderscore , p. XXVII.
\section{Desbalizar}
\begin{itemize}
\item {Grp. gram.:v. t.}
\end{itemize}
Tirar as balizas de.
\section{Desbambar}
\begin{itemize}
\item {Grp. gram.:v. t.}
\end{itemize}
\begin{itemize}
\item {Proveniência:(De \textunderscore des...\textunderscore  + \textunderscore bambo\textunderscore )}
\end{itemize}
Tirar o estado de bambo a; retesar.
\section{Desbancar}
\begin{itemize}
\item {Grp. gram.:v. t.}
\end{itemize}
\begin{itemize}
\item {Utilização:Fig.}
\end{itemize}
\begin{itemize}
\item {Utilização:Prov.}
\end{itemize}
\begin{itemize}
\item {Utilização:T. da Bairrada}
\end{itemize}
\begin{itemize}
\item {Proveniência:(De \textunderscore des...\textunderscore  + \textunderscore banca\textunderscore )}
\end{itemize}
Ganhar o dinheiro da banca a.
Exceder.
Vencer.
Levar as lampas a.
O mesmo que \textunderscore mantear\textunderscore  (a terra).
\section{Desbandalhar}
\textunderscore v. t.\textunderscore  (e der.)
O mesmo que \textunderscore esbandalhar\textunderscore , etc. Cf. Filinto, IX, 154.
\section{Desbandar}
\begin{itemize}
\item {Grp. gram.:v. t.}
\end{itemize}
\begin{itemize}
\item {Utilização:Prov.}
\end{itemize}
O mesmo que \textunderscore debandar\textunderscore .
\section{Desbandeira}
\begin{itemize}
\item {Grp. gram.:f.}
\end{itemize}
Acto de desbandeirar ou de tirar o panículo ao milho.
\section{Desbandeirar}
\begin{itemize}
\item {Grp. gram.:v. t.}
\end{itemize}
\begin{itemize}
\item {Proveniência:(De \textunderscore des...\textunderscore  + \textunderscore bandeira\textunderscore )}
\end{itemize}
Tirar a bandeira a.
Tirar o panículo a (o milho).
\section{Desbaptizar}
\begin{itemize}
\item {Grp. gram.:v. t.}
\end{itemize}
\begin{itemize}
\item {Proveniência:(De \textunderscore des...\textunderscore  + \textunderscore baptizar\textunderscore )}
\end{itemize}
Excommungar.
Privar da graça baptismal.
Tirar ou mudar o nome de.
\section{Desbarar}
\begin{itemize}
\item {Grp. gram.:v. i.}
\end{itemize}
\begin{itemize}
\item {Utilização:Prov.}
\end{itemize}
\begin{itemize}
\item {Utilização:trasm.}
\end{itemize}
Resvalar; escorregar.
(Relaciona-se com \textunderscore disparar\textunderscore ? Ou por \textunderscore desbarrar\textunderscore  = \textunderscore esbarrar\textunderscore ?)
\section{Desbaratadamente}
\begin{itemize}
\item {Grp. gram.:adv.}
\end{itemize}
Com desbarato.
\section{Desbaratador}
\begin{itemize}
\item {Grp. gram.:adj.}
\end{itemize}
\begin{itemize}
\item {Grp. gram.:M.}
\end{itemize}
\begin{itemize}
\item {Proveniência:(De \textunderscore desbaratar\textunderscore )}
\end{itemize}
Que desbarata.
Aquelle que desbarata.
\section{Desbaratamento}
\begin{itemize}
\item {Grp. gram.:m.}
\end{itemize}
Acto ou effeito de \textunderscore desbaratar\textunderscore .
\section{Desbaratar}
\begin{itemize}
\item {Grp. gram.:v. t.}
\end{itemize}
\begin{itemize}
\item {Utilização:Ant.}
\end{itemize}
\begin{itemize}
\item {Proveniência:(De \textunderscore des...\textunderscore  + \textunderscore barato\textunderscore )}
\end{itemize}
Desperdiçar: \textunderscore desbaratar riquezas\textunderscore .
Perder; arruinar, estragar.
Derrotar: \textunderscore desbaratar os inimigos\textunderscore .
Pôr em desordem.
Tratar sem cuidado.
Gastar com economia e acêrto.
\section{Desbarate}
\begin{itemize}
\item {Grp. gram.:m.}
\end{itemize}
\begin{itemize}
\item {Grp. gram.:Loc. adv.}
\end{itemize}
Acto ou effeito de desbaratar.
\textunderscore Ao desbarato\textunderscore , por baixo preço, com grande prejuizo: \textunderscore vender ao desbarato\textunderscore .
\section{Desbarato}
\begin{itemize}
\item {Grp. gram.:m.}
\end{itemize}
\begin{itemize}
\item {Grp. gram.:Loc. adv.}
\end{itemize}
Acto ou effeito de desbaratar.
\textunderscore Ao desbarato\textunderscore , por baixo preço, com grande prejuizo: \textunderscore vender ao desbarato\textunderscore .
\section{Desbarbador}
\begin{itemize}
\item {Grp. gram.:m.}
\end{itemize}
\begin{itemize}
\item {Proveniência:(De \textunderscore desbarbar\textunderscore )}
\end{itemize}
Mecanismo agrícola, em que se lima ou corta a ponta dos grãos de trigo.
\section{Desbarbar}
\begin{itemize}
\item {Grp. gram.:v. t.}
\end{itemize}
\begin{itemize}
\item {Proveniência:(De \textunderscore des...\textunderscore  + \textunderscore barbar\textunderscore )}
\end{itemize}
Tirar a barba a. Tirar os pêlos de. Cortar com o desbarbador.
\section{Desbarbedo}
\begin{itemize}
\item {fónica:bé}
\end{itemize}
\begin{itemize}
\item {Grp. gram.:m.}
\end{itemize}
\begin{itemize}
\item {Proveniência:(De \textunderscore desbarbar\textunderscore )}
\end{itemize}
Acto de cortar (pêlos de um pano, de um bordado, etc.).
\section{Desbaria}
\begin{itemize}
\item {Grp. gram.:f.}
\end{itemize}
\begin{itemize}
\item {Utilização:Prov.}
\end{itemize}
\begin{itemize}
\item {Utilização:trasm.}
\end{itemize}
\begin{itemize}
\item {Proveniência:(De \textunderscore desbarar\textunderscore )}
\end{itemize}
Fraga.
Plano inclinado, em que se póde resvalar.
\section{Desbarrancamento}
\begin{itemize}
\item {Grp. gram.:m.}
\end{itemize}
Acto ou effeito de desbarrancar.
\section{Desbarrancar}
\begin{itemize}
\item {Grp. gram.:v. t.}
\end{itemize}
\begin{itemize}
\item {Utilização:Neol.}
\end{itemize}
Escavar profundamente, desaterrar.
Fazer barrancos em.--Us. pelo gram. bras. Júl. Ribeiro, por \textunderscore esbarrancar\textunderscore , de \textunderscore barranco\textunderscore .
\section{Desbarranco}
\begin{itemize}
\item {Grp. gram.:m.}
\end{itemize}
\begin{itemize}
\item {Proveniência:(De \textunderscore desbarrancar\textunderscore )}
\end{itemize}
O mesmo que \textunderscore desentulho\textunderscore .
\section{Desbarrar}
\begin{itemize}
\item {Grp. gram.:v. t.}
\end{itemize}
\begin{itemize}
\item {Proveniência:(De \textunderscore des...\textunderscore  + \textunderscore barro\textunderscore  e \textunderscore barra\textunderscore )}
\end{itemize}
Tirar a barra a.
Tirar o barro de.
\section{Desbarretar}
\begin{itemize}
\item {Grp. gram.:v. t.}
\end{itemize}
\begin{itemize}
\item {Grp. gram.:V. p.}
\end{itemize}
\begin{itemize}
\item {Proveniência:(De \textunderscore des...\textunderscore  + \textunderscore barrete\textunderscore )}
\end{itemize}
Tirar o barrete da cabeça de.
Descobrir-se.
Cumprimentar, descobrindo a cabeça.
\section{Desbarrigado}
\begin{itemize}
\item {Grp. gram.:adj.}
\end{itemize}
\begin{itemize}
\item {Utilização:Fam.}
\end{itemize}
\begin{itemize}
\item {Utilização:Bras. de Minas}
\end{itemize}
\begin{itemize}
\item {Proveniência:(De \textunderscore des...\textunderscore  + \textunderscore barriga\textunderscore )}
\end{itemize}
Que tem a barriga deprimida.
Que traz desapertado o collete na cintura, deixando vêr camisa ou ceroilas.
Diz-se do animal faminto ou mal alimentado.
\section{Desbastação}
\begin{itemize}
\item {Grp. gram.:f.}
\end{itemize}
O mesmo que \textunderscore desbaste\textunderscore .
\section{Desbastador}
\begin{itemize}
\item {Grp. gram.:adj.}
\end{itemize}
\begin{itemize}
\item {Grp. gram.:m.}
\end{itemize}
Que desbasta.
Aquelle que desbasta.
\section{Desbastamento}
\begin{itemize}
\item {Grp. gram.:m.}
\end{itemize}
O mesmo que \textunderscore desbaste\textunderscore .
\section{Desbastar}
\begin{itemize}
\item {Grp. gram.:v. t.}
\end{itemize}
\begin{itemize}
\item {Proveniência:(De \textunderscore des...\textunderscore  + \textunderscore basto\textunderscore )}
\end{itemize}
Tornar menos basto, mais raro, menos denso: \textunderscore desbastar um pinhal\textunderscore .
Desengrossar (uma peça), cortando: \textunderscore desbastar um madeiro\textunderscore .
Aperfeiçoar, cortando.
Aperfeiçoar.
\section{Desbastardar}
\begin{itemize}
\item {Grp. gram.:v. t.}
\end{itemize}
\begin{itemize}
\item {Proveniência:(De \textunderscore des...\textunderscore  + \textunderscore bastardo\textunderscore )}
\end{itemize}
Tornar legítimo (aquelle ou aquillo que era bastardo).
Legitimar.
\section{Desbaste}
\begin{itemize}
\item {Grp. gram.:m.}
\end{itemize}
Acto ou effeito de desbastar.
\section{Desbastecer}
\begin{itemize}
\item {Grp. gram.:v. t.}
\end{itemize}
(V.desbastar)
\section{Desbatocar}
\begin{itemize}
\item {Grp. gram.:v. t.}
\end{itemize}
Tirar o batoque a.
\section{Desbeiçar}
\begin{itemize}
\item {Grp. gram.:v. t.}
\end{itemize}
\begin{itemize}
\item {Utilização:Fig.}
\end{itemize}
\begin{itemize}
\item {Proveniência:(De \textunderscore des...\textunderscore  + \textunderscore beiço\textunderscore )}
\end{itemize}
Cortar o beiço ou beiços de.
Cortar ou quebrar as bordas de: \textunderscore desbeiçar um tacho\textunderscore .
\section{Desbloquear}
\begin{itemize}
\item {Grp. gram.:v. t.}
\end{itemize}
\begin{itemize}
\item {Proveniência:(De \textunderscore des...\textunderscore  + \textunderscore bloquear\textunderscore )}
\end{itemize}
Cortar ou desfazer o bloqueio de.
\section{Desbloqueio}
\begin{itemize}
\item {Grp. gram.:m.}
\end{itemize}
Acto de desbloquear.
\section{Desbocadamente}
\begin{itemize}
\item {Grp. gram.:adv.}
\end{itemize}
De modo desbocado.
\section{Desbocado}
\begin{itemize}
\item {Grp. gram.:adj.}
\end{itemize}
\begin{itemize}
\item {Utilização:Equit.}
\end{itemize}
\begin{itemize}
\item {Utilização:Fig.}
\end{itemize}
\begin{itemize}
\item {Proveniência:(De \textunderscore desbocar\textunderscore )}
\end{itemize}
Desenfreado.
Que não obedece ao freio.
Impudico.
Inconveniente.
Desaforado.
\section{Desbocar}
\begin{itemize}
\item {Grp. gram.:v. t.}
\end{itemize}
\begin{itemize}
\item {Utilização:Fig.}
\end{itemize}
\begin{itemize}
\item {Grp. gram.:V. p.}
\end{itemize}
\begin{itemize}
\item {Utilização:Fig.}
\end{itemize}
\begin{itemize}
\item {Proveniência:(De \textunderscore des...\textunderscore  + \textunderscore bôca\textunderscore )}
\end{itemize}
Callejar a boca de (um cavallo).
Entornar, despejar.
Tornar impudente, descarado.
Não obedecer (o cavallo) ao freio.
Usar linguagem despejada, inconveniente.
\section{Desbòiar}
\begin{itemize}
\item {Grp. gram.:v. t.}
\end{itemize}
\begin{itemize}
\item {Utilização:Prov.}
\end{itemize}
\begin{itemize}
\item {Utilização:alent.}
\end{itemize}
\begin{itemize}
\item {Proveniência:(De \textunderscore des...\textunderscore  + \textunderscore bóia\textunderscore )}
\end{itemize}
Tirar ao (sobreiro) a primeira cortiça, a qual, por sêr de má qualidade, só serve para bóias.
\section{Desbolinar}
\begin{itemize}
\item {Grp. gram.:v.}
\end{itemize}
\begin{itemize}
\item {Utilização:t. Náut.}
\end{itemize}
\begin{itemize}
\item {Proveniência:(De \textunderscore des...\textunderscore  + \textunderscore bolina\textunderscore )}
\end{itemize}
Tirar a (um cabo), a tendência para criar nó, em razão da sua cocha. Cf. J. P. Bandeira, \textunderscore Appar. do Navio\textunderscore , p. XVII.
\section{Desboqueirar}
\begin{itemize}
\item {Grp. gram.:v. t.}
\end{itemize}
\begin{itemize}
\item {Utilização:Prov.}
\end{itemize}
\begin{itemize}
\item {Utilização:trasm.}
\end{itemize}
Abrir, despedaçando:«\textunderscore os lobos haviam desboqueirado a cancela do curral...\textunderscore »Deusdado, \textunderscore Escorços Trasm.\textunderscore , 252.
(Cp. \textunderscore boqueirão\textunderscore )
\section{Desborcar}
\begin{itemize}
\item {Grp. gram.:v. i.}
\end{itemize}
\begin{itemize}
\item {Proveniência:(De \textunderscore bôrco\textunderscore )}
\end{itemize}
Entornar-se, despejar-se.
Esvaziar-se, voltando-se de bôrco.
\section{Desborcelar}
\begin{itemize}
\item {Grp. gram.:v. t.}
\end{itemize}
O mesmo que \textunderscore esborcelar\textunderscore .
\section{Desborcinar}
\begin{itemize}
\item {Grp. gram.:v. t.}
\end{itemize}
(V.esborcinar)
\section{Desbordante}
\begin{itemize}
\item {Grp. gram.:adj.}
\end{itemize}
Que desborda.
\section{Desbordar}
\begin{itemize}
\item {Grp. gram.:v. i.}
\end{itemize}
O mesmo que \textunderscore trasbordar\textunderscore .
\section{Desboroar}
\begin{itemize}
\item {Grp. gram.:v. t.}
\end{itemize}
(V.esboroar)
\section{Desborrar}
\begin{itemize}
\item {Grp. gram.:v. t.}
\end{itemize}
\begin{itemize}
\item {Proveniência:(De \textunderscore des...\textunderscore  + \textunderscore bôrra\textunderscore )}
\end{itemize}
Tirar as bôrras a.
\section{Desbotadura}
\begin{itemize}
\item {Grp. gram.:f.}
\end{itemize}
Acto ou effeito de desbotar.
\section{Desbotamento}
\begin{itemize}
\item {Grp. gram.:m.}
\end{itemize}
Acto ou effeito de desbotar.
\section{Desbotar}
\begin{itemize}
\item {Grp. gram.:v. t.}
\end{itemize}
\begin{itemize}
\item {Grp. gram.:V. i.}
\end{itemize}
Fazer desvanecer a côr ou brilho de.
Tornar menos viva a côr de.
Alterar a côr de.
Perder a viveza da côr: \textunderscore esta chita desbota\textunderscore .
Soffrer mudança da côr.
\section{Desbotoar}
\textunderscore v. t.\textunderscore  (e der.)
O mesmo que \textunderscore desabotoar\textunderscore , etc. Cf. Camillo, \textunderscore Canc. Alegre\textunderscore , 356; \textunderscore Noites\textunderscore , IX, 31.
\section{Desbragado}
\begin{itemize}
\item {Grp. gram.:adj.}
\end{itemize}
Impudico; indecoroso: \textunderscore phrases desbragadas\textunderscore .
\section{Desbragar}
\begin{itemize}
\item {Grp. gram.:v. t.}
\end{itemize}
\begin{itemize}
\item {Utilização:Fig.}
\end{itemize}
\begin{itemize}
\item {Proveniência:(De \textunderscore des...\textunderscore  + \textunderscore braga\textunderscore )}
\end{itemize}
Desprender da braga.
Tornar libertino, dissoluto.
\section{Desbravador}
\begin{itemize}
\item {Grp. gram.:adj.}
\end{itemize}
Que desbrava.
\section{Desbravamento}
\begin{itemize}
\item {Grp. gram.:m.}
\end{itemize}
Acto de desbravar.
\section{Desbravar}
\begin{itemize}
\item {Grp. gram.:v. t.}
\end{itemize}
\begin{itemize}
\item {Proveniência:(De \textunderscore des...\textunderscore  + \textunderscore bravo\textunderscore )}
\end{itemize}
Tornar manso.
Arrotear.
Preparar para cultura: \textunderscore desbravar uma charneca\textunderscore .
\section{Desbravecer}
\begin{itemize}
\item {Grp. gram.:v. i.}
\end{itemize}
O mesmo que \textunderscore desembravecer\textunderscore . Cf. Filinto, XVI, 167.
\section{Desbriar}
\begin{itemize}
\item {Grp. gram.:v. t.}
\end{itemize}
Tirar o brio a.
Causar desbrio a.
\section{Desbridar}
\begin{itemize}
\item {Grp. gram.:v. t.}
\end{itemize}
Tirar a brida a.
\section{Desbrilho}
\begin{itemize}
\item {Grp. gram.:m.}
\end{itemize}
\begin{itemize}
\item {Proveniência:(De \textunderscore des...\textunderscore  + \textunderscore brilho\textunderscore )}
\end{itemize}
O mesmo que \textunderscore deslustre\textunderscore . Cf. Camillo, \textunderscore Sebenta\textunderscore , II, 10.
\section{Desbrio}
\begin{itemize}
\item {Grp. gram.:m.}
\end{itemize}
Falta de brio, de pundonor.
\section{Desbrioso}
\begin{itemize}
\item {Grp. gram.:adj.}
\end{itemize}
Que não é brioso, que não tem brio.
\section{Desbuchar}
\begin{itemize}
\item {Grp. gram.:v. t.}
\end{itemize}
(V.desembuchar)
\section{Desbulhar}
\begin{itemize}
\item {Grp. gram.:v. t.}
\end{itemize}
(V.debulhar)
\section{Desbulhar}
\begin{itemize}
\item {Grp. gram.:v. t.}
\end{itemize}
O mesmo que \textunderscore esbulhar\textunderscore . Cf. \textunderscore Parnaso Lusitano\textunderscore , V, 172.
\section{Desbulho}
\begin{itemize}
\item {Grp. gram.:m.}
\end{itemize}
(V.debulho)
\section{Descabaçar}
\begin{itemize}
\item {Grp. gram.:v. t.}
\end{itemize}
\begin{itemize}
\item {Utilização:Bras. do N}
\end{itemize}
Tirar a virgindade a.
(Cp. \textunderscore cabaço\textunderscore ^2)
\section{Descabeçador}
\begin{itemize}
\item {Grp. gram.:m.}
\end{itemize}
Aquelle que tira a cabeça a outrem.
Aquelle que degolla.
\section{Descabeçamento}
\begin{itemize}
\item {Grp. gram.:m.}
\end{itemize}
Acto de descabeçar.
\section{Descabeçar}
\begin{itemize}
\item {Grp. gram.:v. t.}
\end{itemize}
\begin{itemize}
\item {Utilização:Des.}
\end{itemize}
\begin{itemize}
\item {Proveniência:(De \textunderscore des...\textunderscore  + \textunderscore cabeça\textunderscore )}
\end{itemize}
Cortar a cabeça de.
Cortar a ponta de.
\section{Descabelado}
\begin{itemize}
\item {Grp. gram.:adj.}
\end{itemize}
\begin{itemize}
\item {Utilização:Fig.}
\end{itemize}
Diz-se daquele, a quém caiu o cabelo, ou a quem o arrancaram.

Que leva coiro e cabelo.
Ofensivo, violento.
\section{Descabelar}
\begin{itemize}
\item {Grp. gram.:v. t.}
\end{itemize}
\begin{itemize}
\item {Grp. gram.:V. p.}
\end{itemize}
Tirar os cabelos a.
Arrancar os cabelos.
Arrepelar-se.
Irritar-se.
\section{Descabellado}
\begin{itemize}
\item {Grp. gram.:adj.}
\end{itemize}
\begin{itemize}
\item {Utilização:Fig.}
\end{itemize}
Diz-se daquelle, a quém caiu o cabello, ou a quem o arrancaram.

Que leva coiro e cabello.
Offensivo, violento.
\section{Descabellar}
\begin{itemize}
\item {Grp. gram.:v. t.}
\end{itemize}
\begin{itemize}
\item {Grp. gram.:V. p.}
\end{itemize}
Tirar os cabellos a.
Arrancar os cabellos.
Arrepelar-se.
Irritar-se.
\section{Descabello}
\begin{itemize}
\item {Grp. gram.:m.}
\end{itemize}
\begin{itemize}
\item {Utilização:Taur.}
\end{itemize}
\begin{itemize}
\item {Proveniência:(De descabellar)}
\end{itemize}
Recurso, em que o toireiro pica com o estoque a medulla espinhal, junto ao testo, para mais rapidamente matar o toiro já ferido.
\section{Descabelo}
\begin{itemize}
\item {fónica:bê}
\end{itemize}
\begin{itemize}
\item {Grp. gram.:m.}
\end{itemize}
\begin{itemize}
\item {Utilização:Taur.}
\end{itemize}
\begin{itemize}
\item {Proveniência:(De descabelar)}
\end{itemize}
Recurso, em que o toireiro pica com o estoque a medula espinhal, junto ao testo, para mais rapidamente matar o toiro já ferido.
\section{Descaber}
\begin{itemize}
\item {Grp. gram.:v. i.}
\end{itemize}
\begin{itemize}
\item {Proveniência:(De \textunderscore des...\textunderscore  + \textunderscore caber\textunderscore )}
\end{itemize}
Não ter cabida.
Não pertencer.
Não vir a propósito.
\section{Descabido}
\begin{itemize}
\item {Grp. gram.:adj.}
\end{itemize}
\begin{itemize}
\item {Proveniência:(De \textunderscore descaber\textunderscore )}
\end{itemize}
Que não vem a ponto.
Inopportuno.
Inconveniente.
\section{Descabreado}
\begin{itemize}
\item {Grp. gram.:adj.}
\end{itemize}
\begin{itemize}
\item {Proveniência:(De \textunderscore descabrear\textunderscore )}
\end{itemize}
O mesmo que [[escabreado|escabrear]]. Cf. Camillo, \textunderscore Cav. em Ruínas\textunderscore , 45.
\section{Descabrear}
\begin{itemize}
\item {Grp. gram.:v. i.}
\end{itemize}
O mesmo que \textunderscore escabrear\textunderscore .
\section{Descaçar-se}
\begin{itemize}
\item {Grp. gram.:v. p.}
\end{itemize}
\begin{itemize}
\item {Utilização:Prov.}
\end{itemize}
\begin{itemize}
\item {Utilização:fam.}
\end{itemize}
\begin{itemize}
\item {Proveniência:(De \textunderscore des...\textunderscore  + \textunderscore caçar\textunderscore )}
\end{itemize}
Desacostumar-se, deshabituar-se.
\section{Descachaçar}
\begin{itemize}
\item {Grp. gram.:v. t.}
\end{itemize}
\begin{itemize}
\item {Proveniência:(De \textunderscore des...\textunderscore  + \textunderscore cachaça\textunderscore )}
\end{itemize}
Limpar da cachaça ou das escumas grossas (o caldo da cana de açúcar).
\section{Descachar}
\begin{itemize}
\item {Grp. gram.:v. t.}
\end{itemize}
(V.descachaçar)
\section{Descadeirado}
\begin{itemize}
\item {Grp. gram.:adj.}
\end{itemize}
\begin{itemize}
\item {Proveniência:(De \textunderscore descadeirar\textunderscore )}
\end{itemize}
Desnalgado.
\section{Descadeirar}
\begin{itemize}
\item {Grp. gram.:v. t.}
\end{itemize}
\begin{itemize}
\item {Utilização:Pop.}
\end{itemize}
\begin{itemize}
\item {Proveniência:(De \textunderscore des...\textunderscore  + \textunderscore cadeira\textunderscore )}
\end{itemize}
Bater nas ancas de.
Derrear com pancadas.
\section{Descaída}
\begin{itemize}
\item {Grp. gram.:f.}
\end{itemize}
\begin{itemize}
\item {Utilização:Fam.}
\end{itemize}
Acto de descaír.
Lapso, descuido.
\section{Descaidela}
\begin{itemize}
\item {fónica:ca-i}
\end{itemize}
\begin{itemize}
\item {Grp. gram.:f.}
\end{itemize}
\begin{itemize}
\item {Utilização:Pop.}
\end{itemize}
O mesmo que \textunderscore descaída\textunderscore .
\section{Descaimento}
\begin{itemize}
\item {fónica:ca-i}
\end{itemize}
\begin{itemize}
\item {Grp. gram.:m.}
\end{itemize}
Acto ou effeito de descair.
\section{Descair}
\begin{itemize}
\item {Grp. gram.:v. t.}
\end{itemize}
\begin{itemize}
\item {Grp. gram.:V. i.}
\end{itemize}
\begin{itemize}
\item {Proveniência:(De \textunderscore des...\textunderscore  + \textunderscore cair\textunderscore )}
\end{itemize}
Deixar caír.
Inclinar.
Decair: \textunderscore aquella riqueza vai descaindo\textunderscore .
Curvar-se.
Enfraquecer.
Incidir.
Desviar-se de um rumo.
Abrandar: \textunderscore descair o calor\textunderscore .
\section{Descalabrar}
\begin{itemize}
\item {Grp. gram.:v. t.}
\end{itemize}
\begin{itemize}
\item {Proveniência:(De \textunderscore descalabro\textunderscore )}
\end{itemize}
O mesmo que \textunderscore escalavrar\textunderscore . Cf. \textunderscore Viriato Trág.\textunderscore , XIII, 87.
\section{Descalabro}
\begin{itemize}
\item {Grp. gram.:m.}
\end{itemize}
\begin{itemize}
\item {Proveniência:(T. cast.)}
\end{itemize}
Grande damno.
Perda.
Ruína.
Desgraça; derrota.
\section{Descalçadeira}
\begin{itemize}
\item {Grp. gram.:f.}
\end{itemize}
\begin{itemize}
\item {Utilização:Fig.}
\end{itemize}
\begin{itemize}
\item {Proveniência:(De \textunderscore descalçar\textunderscore )}
\end{itemize}
Instrumento, para ajudar a tirar o calçado dos pés.
Descompostura.
\section{Descalçadela}
\begin{itemize}
\item {Grp. gram.:f.}
\end{itemize}
\begin{itemize}
\item {Utilização:Pop.}
\end{itemize}
\begin{itemize}
\item {Proveniência:(De \textunderscore descalçar\textunderscore )}
\end{itemize}
Descompostura.
\section{Descalçador}
\begin{itemize}
\item {Grp. gram.:m.}
\end{itemize}
Utensílio, para ajudar a tirar dos pés o calçado.
Descalçadeira.
\section{Descalçadura}
\begin{itemize}
\item {Grp. gram.:f.}
\end{itemize}
Acto de descalçar.
\section{Descalçar}
\begin{itemize}
\item {Grp. gram.:v. t.}
\end{itemize}
\begin{itemize}
\item {Utilização:Fig.}
\end{itemize}
\begin{itemize}
\item {Utilização:Fam.}
\end{itemize}
\begin{itemize}
\item {Proveniência:(De \textunderscore des...\textunderscore  + \textunderscore calçar\textunderscore )}
\end{itemize}
Tirar (aquillo que vestia a perna, o pé ou a mão): \textunderscore descalçar as luvas\textunderscore .
Tirar (a mão, a perna ou pé) daquillo que os vestia.
Tirar o calço de.
Tirar o empedramento de (rua, estrada, etc.).
Desamparar.
\textunderscore Descalçar a bota\textunderscore , vencer uma difficuldade.
\section{Descalcez}
\begin{itemize}
\item {Grp. gram.:f.}
\end{itemize}
\begin{itemize}
\item {Proveniência:(De \textunderscore descalço\textunderscore )}
\end{itemize}
Qualidade de descalço. Cf. Macedo, \textunderscore Burros\textunderscore , 292.
Designação genérica da ordem religiosa dos Carmelitas Descalços. Cf. \textunderscore Luz e Calor\textunderscore , 158.
\section{Descalço}
\begin{itemize}
\item {Grp. gram.:adj.}
\end{itemize}
\begin{itemize}
\item {Utilização:pop.}
\end{itemize}
\begin{itemize}
\item {Utilização:Fig.}
\end{itemize}
\begin{itemize}
\item {Proveniência:(De \textunderscore descalçar\textunderscore )}
\end{itemize}
Que não está calçado.
Que tem os pés nus ou só calçados com meias.
Desprevenido: \textunderscore não me apanhou descalço\textunderscore .
Que não está empedrado, (falando-se de um caminho ou rua).
\section{Descalhoar}
\begin{itemize}
\item {Grp. gram.:v. t.}
\end{itemize}
Limpar dos calhaus.
Tirar os calhaus ou pedras de (terreno cultivável). Cf. \textunderscore Biblioth. da Gente do Campo\textunderscore , 304.
\section{Descaliçar}
\begin{itemize}
\item {Grp. gram.:v. t.}
\end{itemize}
Tirar a caliça a.
\section{Descalícino}
\begin{itemize}
\item {Grp. gram.:adj.}
\end{itemize}
\begin{itemize}
\item {Utilização:Bot.}
\end{itemize}
\begin{itemize}
\item {Proveniência:(De \textunderscore des...\textunderscore  + \textunderscore calícino\textunderscore )}
\end{itemize}
Que não tem cálice.
\section{Descalvar}
\begin{itemize}
\item {Grp. gram.:v. t.}
\end{itemize}
(V.escalvar)
\section{Descamação}
\begin{itemize}
\item {Grp. gram.:f.}
\end{itemize}
Acto de descamar.
\section{Descamar}
\begin{itemize}
\item {Grp. gram.:v. t.}
\end{itemize}
\begin{itemize}
\item {Proveniência:(Do lat. \textunderscore dísquamare\textunderscore )}
\end{itemize}
O mesmo que \textunderscore escamar\textunderscore .
\section{Descambação}
\begin{itemize}
\item {Grp. gram.:f.}
\end{itemize}
Acto de descambar.
\section{Descambada}
\begin{itemize}
\item {Grp. gram.:f.}
\end{itemize}
\begin{itemize}
\item {Utilização:Bras}
\end{itemize}
\begin{itemize}
\item {Proveniência:(De \textunderscore descambado\textunderscore )}
\end{itemize}
Encosta.
Vertente.
\section{Descambadela}
\begin{itemize}
\item {Grp. gram.:f.}
\end{itemize}
O mesmo que \textunderscore descambação\textunderscore .
\section{Descambado}
\begin{itemize}
\item {Grp. gram.:adj.}
\end{itemize}
Que descambou.
\section{Descambar}
\begin{itemize}
\item {Grp. gram.:v. i.}
\end{itemize}
\begin{itemize}
\item {Proveniência:(De \textunderscore des...\textunderscore  + \textunderscore cambar\textunderscore )}
\end{itemize}
Cair.
Derivar.
Incidir.
Redundar.
\section{Descaminhadamente}
\begin{itemize}
\item {Grp. gram.:adv.}
\end{itemize}
Com extravio.
\section{Descaminhar}
\begin{itemize}
\item {Grp. gram.:v. t.}
\end{itemize}
O mesmo que \textunderscore desencaminhar\textunderscore .
\section{Descaminho}
\begin{itemize}
\item {Grp. gram.:m.}
\end{itemize}
\begin{itemize}
\item {Proveniência:(De \textunderscore des...\textunderscore  + \textunderscore caminho\textunderscore )}
\end{itemize}
Acto de descaminhar.
Extravio.
Acto de pretender introduzir numa localidade, principalmente por meios fraudulentos, objectos sujeitos a direitos alfandegários, para não pagar êsses direitos.
Objecto, descaminhado nesse intuito.
\section{Descamisa}
\begin{itemize}
\item {Grp. gram.:f.}
\end{itemize}
O mesmo que \textunderscore descamisada\textunderscore .
\section{Descamisada}
\begin{itemize}
\item {Grp. gram.:f.}
\end{itemize}
Acto de descamisar (o milho).
\section{Descamisado}
\begin{itemize}
\item {Grp. gram.:adj.}
\end{itemize}
\begin{itemize}
\item {Grp. gram.:M.}
\end{itemize}
\begin{itemize}
\item {Proveniência:(De \textunderscore descamisar\textunderscore )}
\end{itemize}
Que não tem camisa.
Aquelle que não tem camisa.
Maltrapilho:«\textunderscore ...ainda que se formasse um ministério de descamisados...\textunderscore »Garrett, \textunderscore Port. na Balança\textunderscore , 2.^a ed., 222.
\section{Descamisar}
\begin{itemize}
\item {Grp. gram.:v. t.}
\end{itemize}
\begin{itemize}
\item {Proveniência:(De \textunderscore des...\textunderscore  + \textunderscore camisa\textunderscore )}
\end{itemize}
Tirar a camisa de.
Tirar as fôlhas que envolvem a maçaroca de (o milho).
\section{Desçamoucar}
\begin{itemize}
\item {Grp. gram.:v. t.}
\end{itemize}
Tirar o çamouco a.
\section{Descampado}
\begin{itemize}
\item {Grp. gram.:m.}
\end{itemize}
Campo inculto e deshabitado.
(Por \textunderscore escampado\textunderscore , de \textunderscore campo\textunderscore )
\section{Descampar}
\begin{itemize}
\item {Grp. gram.:v. i.}
\end{itemize}
\begin{itemize}
\item {Utilização:Náut.}
\end{itemize}
\begin{itemize}
\item {Proveniência:(De \textunderscore des\textunderscore  + \textunderscore campo\textunderscore )}
\end{itemize}
Correr pelo campo.
Desapparecer:«\textunderscore pedir-lhe a filha e descampar com ella\textunderscore ». Filinto, VII, 92.
Pôr ao largo, guiar para o largo (uma embarcação):«\textunderscore mandou pôr pulso aos remos e descampar as galés\textunderscore ». Filinto, \textunderscore D. Man.\textunderscore , II, 60.
\section{Descampatória}
\begin{itemize}
\item {Grp. gram.:f.}
\end{itemize}
\begin{itemize}
\item {Utilização:Prov.}
\end{itemize}
\begin{itemize}
\item {Utilização:trasm.}
\end{itemize}
O mesmo que \textunderscore destampatório\textunderscore .
\section{Descanar}
\begin{itemize}
\item {Grp. gram.:v. t.}
\end{itemize}
\begin{itemize}
\item {Utilização:Prov.}
\end{itemize}
\begin{itemize}
\item {Utilização:beir.}
\end{itemize}
\begin{itemize}
\item {Proveniência:(De \textunderscore des...\textunderscore  + \textunderscore cana\textunderscore )}
\end{itemize}
Cortar a cana do (milho) por cima da maçaroca; desbandeirar.
\section{Descancarado}
\begin{itemize}
\item {Grp. gram.:adj.}
\end{itemize}
\begin{itemize}
\item {Utilização:Ant.}
\end{itemize}
\begin{itemize}
\item {Proveniência:(De \textunderscore escancarar\textunderscore )}
\end{itemize}
Descarado, desavergonhado. Cf. G. Vicente, \textunderscore Inês Pereira\textunderscore .
\section{Descancelar}
\begin{itemize}
\item {Grp. gram.:v. t.}
\end{itemize}
\begin{itemize}
\item {Proveniência:(De \textunderscore des...\textunderscore  + \textunderscore cancela\textunderscore )}
\end{itemize}
Abrir a cancela de.
\section{Descancellar}
\begin{itemize}
\item {Grp. gram.:v. t.}
\end{itemize}
\begin{itemize}
\item {Proveniência:(De \textunderscore des...\textunderscore  + \textunderscore cancella\textunderscore )}
\end{itemize}
Abrir a cancella de.
\section{Descangar}
\begin{itemize}
\item {Grp. gram.:v. t.}
\end{itemize}
\begin{itemize}
\item {Utilização:Ant.}
\end{itemize}
\begin{itemize}
\item {Proveniência:(De \textunderscore des...\textunderscore  + \textunderscore cangar\textunderscore )}
\end{itemize}
Tirar a canga a.
Tirar os paus, que seguram o colmo ou as giestas, com que se cobrem as casas.
\section{Descangotado}
\begin{itemize}
\item {Grp. gram.:adj.}
\end{itemize}
\begin{itemize}
\item {Utilização:Bras}
\end{itemize}
Combalido, abatido.
\section{Descansadamente}
\begin{itemize}
\item {Grp. gram.:adv.}
\end{itemize}
\begin{itemize}
\item {Proveniência:(De \textunderscore descansado\textunderscore )}
\end{itemize}
Com descanso.
\section{Descansadeiro}
\begin{itemize}
\item {Grp. gram.:m.}
\end{itemize}
\begin{itemize}
\item {Proveniência:(De \textunderscore descansar\textunderscore )}
\end{itemize}
Lugar, em que se descansa.
\section{Descansado}
\begin{itemize}
\item {Grp. gram.:adj.}
\end{itemize}
\begin{itemize}
\item {Proveniência:(De \textunderscore descansar\textunderscore )}
\end{itemize}
Que está em descanso.
Tranquillo.
Que não tem cuidados.
Vagaroso.
\section{Descansar}
\begin{itemize}
\item {Grp. gram.:v. t.}
\end{itemize}
\begin{itemize}
\item {Utilização:Fig.}
\end{itemize}
\begin{itemize}
\item {Grp. gram.:V. i.}
\end{itemize}
\begin{itemize}
\item {Proveniência:(De \textunderscore des...\textunderscore  + \textunderscore cansar\textunderscore )}
\end{itemize}
Alliviar da fadiga.
Firmar sôbre alguma coisa: \textunderscore descansar o rosto nas mãos\textunderscore .
Tranquillizar.
Tomar repoiso.
Estar de poisio.
Interromper-se.
Dormir ou estar deitado na cama: \textunderscore a menina está descansando\textunderscore .
Tranquillizar-se.
Sentar-se.
Apoiar-se sôbre alguma coisa.
\section{Descanso}
\begin{itemize}
\item {Grp. gram.:m.}
\end{itemize}
\begin{itemize}
\item {Proveniência:(De \textunderscore descansar\textunderscore )}
\end{itemize}
Acto ou effeito de descansar.
Lugar, em que se descansa.
Objecto, em que assenta outro.
Sossêgo.
Demora, pachorra.
Commodidade.
Ocio.
Allivio: \textunderscore soffrer sem descanso\textunderscore .
Pessôa, a quem outrem confia tranquillamente a gestão dos seus negócios.
Somno.
\section{Descantar}
\begin{itemize}
\item {Grp. gram.:v. t.  e  i.}
\end{itemize}
\begin{itemize}
\item {Proveniência:(De \textunderscore cantar\textunderscore )}
\end{itemize}
Cantar ao som de instrumento.
\section{Descante}
\begin{itemize}
\item {Grp. gram.:m.}
\end{itemize}
Acto de descantar.
* (\textunderscore Prov. minh.\textunderscore )
Viola pequena.
\section{Descanto}
\begin{itemize}
\item {Grp. gram.:m.}
\end{itemize}
\begin{itemize}
\item {Utilização:Ant.}
\end{itemize}
\begin{itemize}
\item {Proveniência:(De \textunderscore descantar\textunderscore )}
\end{itemize}
Canto a vozes, música de estante. Cf. Cortesão, \textunderscore Subs.\textunderscore 
\section{Descapacitar-se}
\begin{itemize}
\item {Grp. gram.:v. p.}
\end{itemize}
\begin{itemize}
\item {Proveniência:(De \textunderscore des...\textunderscore  + \textunderscore capacitar\textunderscore )}
\end{itemize}
Despersuadir-se.
\section{Descapelada}
\begin{itemize}
\item {Grp. gram.:f.}
\end{itemize}
Acto de descapelar.
\section{Descapelar}
\begin{itemize}
\item {Grp. gram.:v. t.}
\end{itemize}
\begin{itemize}
\item {Utilização:Prov.}
\end{itemize}
\begin{itemize}
\item {Proveniência:(De \textunderscore des...\textunderscore  + \textunderscore capela\textunderscore )}
\end{itemize}
Descamisar (o milho). (Colhido na Bairrada)
\section{Descapellada}
\begin{itemize}
\item {Grp. gram.:f.}
\end{itemize}
Acto de descapellar.
\section{Descapellar}
\begin{itemize}
\item {Grp. gram.:v. t.}
\end{itemize}
\begin{itemize}
\item {Utilização:Prov.}
\end{itemize}
\begin{itemize}
\item {Proveniência:(De \textunderscore des...\textunderscore  + \textunderscore capella\textunderscore )}
\end{itemize}
Descamisar (o milho). (Colhido na Bairrada)
\section{Descapitalizar}
\begin{itemize}
\item {Grp. gram.:v. t.}
\end{itemize}
\begin{itemize}
\item {Proveniência:(De \textunderscore des...\textunderscore  + \textunderscore capitalizar\textunderscore )}
\end{itemize}
Desviar, despender ou pôr em circulação (quantias ou valores capitalizados).
\section{Descaracterizar}
\begin{itemize}
\item {Grp. gram.:v. t.}
\end{itemize}
\begin{itemize}
\item {Proveniência:(De \textunderscore des...\textunderscore  + \textunderscore caracterizar\textunderscore )}
\end{itemize}
Tirar o carácter a.
Desfazer a caracterização de.
\section{Descaradamente}
\begin{itemize}
\item {Grp. gram.:adv.}
\end{itemize}
De modo descarado.
\section{Descarado}
\begin{itemize}
\item {Grp. gram.:adj.}
\end{itemize}
\begin{itemize}
\item {Proveniência:(De \textunderscore descarar-se\textunderscore )}
\end{itemize}
Impudente; insolente.
Desavergonhado.
\section{Descaramento}
\begin{itemize}
\item {Grp. gram.:m.}
\end{itemize}
Estado daquelle ou daquillo que é descarado.
\section{Descarapuçar}
\begin{itemize}
\item {Grp. gram.:v. t.}
\end{itemize}
\begin{itemize}
\item {Proveniência:(De \textunderscore des...\textunderscore  + \textunderscore carapuça\textunderscore )}
\end{itemize}
Tirar a carapuça a.
\section{Descarar-se}
\begin{itemize}
\item {Grp. gram.:v. p.}
\end{itemize}
\begin{itemize}
\item {Proveniência:(De \textunderscore des...\textunderscore  + \textunderscore cara\textunderscore )}
\end{itemize}
Perder a vergonha.
Tornar-se impudente.
\section{Descarbonizar}
\begin{itemize}
\item {Grp. gram.:v. t.}
\end{itemize}
\begin{itemize}
\item {Proveniência:(De \textunderscore des...\textunderscore  + \textunderscore carbone\textunderscore )}
\end{itemize}
Tirar o carbone a.
\section{Descarda}
\begin{itemize}
\item {Grp. gram.:f.}
\end{itemize}
\begin{itemize}
\item {Utilização:Prov.}
\end{itemize}
\begin{itemize}
\item {Utilização:alent.}
\end{itemize}
\begin{itemize}
\item {Proveniência:(De \textunderscore des\textunderscore  + \textunderscore cardo\textunderscore )}
\end{itemize}
Córte dos cardos, que se faz em terras ainda não semeadas, para impedir que os cardos dêem sementes e sujem muito a terra.
\section{Descarecer}
\begin{itemize}
\item {Grp. gram.:v. i.}
\end{itemize}
\begin{itemize}
\item {Utilização:P. us.}
\end{itemize}
\begin{itemize}
\item {Proveniência:(De \textunderscore des...\textunderscore  + \textunderscore carecer\textunderscore )}
\end{itemize}
Não carecer.
\section{Descarecido}
\begin{itemize}
\item {Grp. gram.:adj.}
\end{itemize}
\begin{itemize}
\item {Proveniência:(De \textunderscore descarecer\textunderscore )}
\end{itemize}
Que descarece.
Privado. Cf. Camillo, \textunderscore Narcóticos\textunderscore , I, 121.
\section{Descarga}
\begin{itemize}
\item {Grp. gram.:f.}
\end{itemize}
Acto de descargar.
Tiro de espingarda ou de peça de artilharia.
Tiros, disparados conjuntamente.
\section{Descargar}
\begin{itemize}
\item {Grp. gram.:v. t.}
\end{itemize}
\begin{itemize}
\item {Utilização:Des.}
\end{itemize}
O mesmo que descarregar. Cf. Filinto, V, 286.
\section{Descargo}
\begin{itemize}
\item {Grp. gram.:m.}
\end{itemize}
\begin{itemize}
\item {Proveniência:(De \textunderscore des...\textunderscore  + \textunderscore cargo\textunderscore )}
\end{itemize}
Acto de desobrigar-se.
Cumprimento, de uma obrigação.
Defesa de uma imputação.
Allívio: \textunderscore descargo de consciência\textunderscore .
\section{Descaridade}
\begin{itemize}
\item {Grp. gram.:f.}
\end{itemize}
Falta de caridade.
\section{Descaridosamente}
\begin{itemize}
\item {Grp. gram.:adv.}
\end{itemize}
De modo descaridoso.
\section{Descaridoso}
\begin{itemize}
\item {Grp. gram.:adj.}
\end{itemize}
\begin{itemize}
\item {Proveniência:(De \textunderscore des...\textunderscore  + \textunderscore caridoso\textunderscore )}
\end{itemize}
Que não tem caridade.
Em que não há caridade.
\section{Descarinhosamente}
\begin{itemize}
\item {Grp. gram.:adv.}
\end{itemize}
De modo descarinhoso.
\section{Descarinhoso}
\begin{itemize}
\item {Grp. gram.:adj.}
\end{itemize}
\begin{itemize}
\item {Proveniência:(De \textunderscore des...\textunderscore  + \textunderscore carinhoso\textunderscore )}
\end{itemize}
Que não tem carinho.
Em que não há carinho.
\section{Descarnador}
\begin{itemize}
\item {Grp. gram.:adj.}
\end{itemize}
\begin{itemize}
\item {Grp. gram.:M.}
\end{itemize}
\begin{itemize}
\item {Proveniência:(De \textunderscore descarnar\textunderscore )}
\end{itemize}
Que descarna.
Instrumento, com que se descarnam os dentes.
\section{Descarnadura}
\begin{itemize}
\item {Grp. gram.:f.}
\end{itemize}
Acto de descarnar.
\section{Descarnar}
\begin{itemize}
\item {Grp. gram.:v. t.}
\end{itemize}
\begin{itemize}
\item {Utilização:Fig.}
\end{itemize}
Separar da carne: \textunderscore descarnar os dentes\textunderscore .
Escavar.
Tornar magro.
\section{Descaro}
\begin{itemize}
\item {Grp. gram.:m.}
\end{itemize}
O mesmo que \textunderscore descaramento\textunderscore .
\section{Descaroado}
\begin{itemize}
\item {Grp. gram.:adj.}
\end{itemize}
O mesmo que \textunderscore descaroável\textunderscore .
\section{Descaroável}
\begin{itemize}
\item {Grp. gram.:adj.}
\end{itemize}
\begin{itemize}
\item {Proveniência:(De \textunderscore des...\textunderscore  + \textunderscore caroável\textunderscore )}
\end{itemize}
Descaridoso.
Que não tem carinho; inclemente.
\section{Descaroçador}
\begin{itemize}
\item {Grp. gram.:adj.}
\end{itemize}
\begin{itemize}
\item {Grp. gram.:M.}
\end{itemize}
\begin{itemize}
\item {Proveniência:(De \textunderscore des...\textunderscore  + \textunderscore caroço\textunderscore )}
\end{itemize}
Que descaroça.
Instrumento, para descaroçar.
\section{Descaroçamento}
\begin{itemize}
\item {Grp. gram.:m.}
\end{itemize}
Acto de descaroçar.
\section{Descaroçar}
\begin{itemize}
\item {Grp. gram.:v. t.}
\end{itemize}
\begin{itemize}
\item {Utilização:Prov.}
\end{itemize}
\begin{itemize}
\item {Utilização:minh.}
\end{itemize}
\begin{itemize}
\item {Utilização:Fam.}
\end{itemize}
Tirar o caroço de.
Explicar, deslindar.
Contar minuciosamente.
\section{Descaroçar}
\begin{itemize}
\item {Grp. gram.:v. t.}
\end{itemize}
\begin{itemize}
\item {Utilização:Prov.}
\end{itemize}
Tirar a caroça a (o linho).
\section{Descarolar}
\begin{itemize}
\item {Grp. gram.:v. t.}
\end{itemize}
Tirar a crosta a. Cf. Camillo, \textunderscore Narcóticos\textunderscore , II, 244.
(Cp. \textunderscore escarolar\textunderscore )
\section{Descarrar}
\begin{itemize}
\item {Grp. gram.:v. t.}
\end{itemize}
Tirar do carro: \textunderscore foram precisos quatro homens para descarrar aquella madeira\textunderscore .
\section{Descarregadeira}
\begin{itemize}
\item {Grp. gram.:f.}
\end{itemize}
\begin{itemize}
\item {Utilização:Marn.}
\end{itemize}
\begin{itemize}
\item {Proveniência:(De \textunderscore descarregar\textunderscore )}
\end{itemize}
Cano de manilhas, entre uma peça da salina e o caldeirão, para alliviar a peça, quando tem água de mais.
\section{Descarregadoiro}
\begin{itemize}
\item {Grp. gram.:m.}
\end{itemize}
\begin{itemize}
\item {Proveniência:(De \textunderscore des...\textunderscore  + \textunderscore carregadoiro\textunderscore )}
\end{itemize}
Lugar, em que se descarrega alguma coisa.
\section{Descarregador}
\begin{itemize}
\item {Grp. gram.:m.}
\end{itemize}
Aquelle que descarrega.
\section{Descarregadouro}
\begin{itemize}
\item {Grp. gram.:m.}
\end{itemize}
\begin{itemize}
\item {Proveniência:(De \textunderscore des...\textunderscore  + \textunderscore carregadouro\textunderscore )}
\end{itemize}
Lugar, em que se descarrega alguma coisa.
\section{Descarregamento}
\begin{itemize}
\item {Grp. gram.:m.}
\end{itemize}
Acto de descarregar.
\section{Descarregar}
\begin{itemize}
\item {Grp. gram.:v. t.}
\end{itemize}
\begin{itemize}
\item {Utilização:Fig.}
\end{itemize}
\begin{itemize}
\item {Proveniência:(De \textunderscore des...\textunderscore  + \textunderscore carregar\textunderscore )}
\end{itemize}
Tirar a carga de.
Tirar de (carro, navio, etc.) carregação que traz.
Alliviar.
Isentar.
Desabafar.
Desobrigar.
Evacuar.
Disparar: \textunderscore descarregar uma arma de fogo\textunderscore .
Dar com fôrça: \textunderscore descarregar pancadas\textunderscore .
Arremessar.
\section{Descarrêgo}
\begin{itemize}
\item {Grp. gram.:m.}
\end{itemize}
\begin{itemize}
\item {Utilização:Des.}
\end{itemize}
O mesmo que \textunderscore descargo\textunderscore .
Acto de descarregar.
\section{Descarreirar}
\begin{itemize}
\item {Grp. gram.:v. t.}
\end{itemize}
(V.descaminhar)
\section{Descarrilamento}
\begin{itemize}
\item {Grp. gram.:m.}
\end{itemize}
Acto de descarrilar.
\section{Descarrilar}
\begin{itemize}
\item {Grp. gram.:v. t.}
\end{itemize}
\begin{itemize}
\item {Grp. gram.:V. i.}
\end{itemize}
\begin{itemize}
\item {Utilização:Fig.}
\end{itemize}
\begin{itemize}
\item {Proveniência:(De \textunderscore des...\textunderscore  + \textunderscore carril\textunderscore )}
\end{itemize}
Desviar do carril.
Fazer saír dos carrís.
Saír (uma carruagem) para fóra dos carrís, sôbre que ia rodando.
Desorientar-se.
Perder o tino.
Disparatar.
Portar-se mal: \textunderscore a pequena descarrilou\textunderscore .
\section{Descartar-se}
\begin{itemize}
\item {Grp. gram.:v. p.}
\end{itemize}
\begin{itemize}
\item {Proveniência:(De \textunderscore des...\textunderscore  + \textunderscore carta\textunderscore )}
\end{itemize}
Baldar-se.
Pôr de lado ou rejeitar certas cartas, ao jôgo.
Libertar-se (de pessoas ou coisas impertinentes ou incômmodas).
\section{Descarte}
\begin{itemize}
\item {Grp. gram.:m.}
\end{itemize}
Acto de descartar-se.
\section{Descasalar}
\begin{itemize}
\item {Grp. gram.:v. t.}
\end{itemize}
O mesmo que \textunderscore desacasalar\textunderscore .
\section{Descasadura}
\begin{itemize}
\item {Grp. gram.:f.}
\end{itemize}
O mesmo que \textunderscore descasamento\textunderscore .
\section{Descasamento}
\begin{itemize}
\item {Grp. gram.:m.}
\end{itemize}
Acto de descasar.
\section{Descasar}
\begin{itemize}
\item {Grp. gram.:v. t.}
\end{itemize}
\begin{itemize}
\item {Utilização:Fam.}
\end{itemize}
\begin{itemize}
\item {Utilização:Fig.}
\end{itemize}
\begin{itemize}
\item {Proveniência:(De \textunderscore des...\textunderscore  + \textunderscore casar\textunderscore )}
\end{itemize}
Desfazer o casamento de.
Desacasalar.
Desirmanar.
\section{Descascação}
\begin{itemize}
\item {Grp. gram.:f.}
\end{itemize}
O mesmo que \textunderscore descascamento\textunderscore . Cf. Castilho, \textunderscore Fastos\textunderscore , III, 473.
\section{Descascador}
\begin{itemize}
\item {Grp. gram.:m.}
\end{itemize}
Aquelle que descasca.
\section{Descascadura}
\begin{itemize}
\item {Grp. gram.:f.}
\end{itemize}
O mesmo que \textunderscore descascamento\textunderscore .
\section{Descascamento}
\begin{itemize}
\item {Grp. gram.:m.}
\end{itemize}
Acto de descascar.
\section{Descascar}
\begin{itemize}
\item {Grp. gram.:v. t.}
\end{itemize}
\begin{itemize}
\item {Grp. gram.:V. i.}
\end{itemize}
Tirar a casca de.
Largar a casca.
Deixar caír o casco (a bêsta).
\section{Descaspar}
\begin{itemize}
\item {Grp. gram.:v. t.}
\end{itemize}
\begin{itemize}
\item {Utilização:Des.}
\end{itemize}
\begin{itemize}
\item {Proveniência:(De \textunderscore des...\textunderscore  + \textunderscore caspa\textunderscore )}
\end{itemize}
Tirar a caspa da cabeça de.
\section{Descasque}
\begin{itemize}
\item {Grp. gram.:m.}
\end{itemize}
O mesmo que \textunderscore descascamento\textunderscore .
Acto de descasquejar, de tirar o casco ou surro.
\section{Descasquejado}
\begin{itemize}
\item {Grp. gram.:adj.}
\end{itemize}
\begin{itemize}
\item {Utilização:Prov.}
\end{itemize}
\begin{itemize}
\item {Utilização:fam.}
\end{itemize}
\begin{itemize}
\item {Proveniência:(De \textunderscore descasquejar\textunderscore )}
\end{itemize}
Lavado e barbeado.
Muito limpo.
\section{Descasquejar}
\begin{itemize}
\item {Grp. gram.:v. t.}
\end{itemize}
\begin{itemize}
\item {Proveniência:(De \textunderscore des...\textunderscore  + \textunderscore casco\textunderscore )}
\end{itemize}
Limpar a immundície de.
Tirar a crosta ou casco formado pelo surro.
Cp. \textunderscore escasquear\textunderscore .
\section{Descatholizar}
\begin{itemize}
\item {Grp. gram.:v. t.}
\end{itemize}
\begin{itemize}
\item {Proveniência:(De \textunderscore des...\textunderscore  + \textunderscore catholizar\textunderscore )}
\end{itemize}
Tirar a fé cathólica a.
Desviar do Catholicismo.
\section{Descatolizar}
\begin{itemize}
\item {Grp. gram.:v. t.}
\end{itemize}
\begin{itemize}
\item {Proveniência:(De \textunderscore des...\textunderscore  + \textunderscore catolizar\textunderscore )}
\end{itemize}
Tirar a fé católica a.
Desviar do Catolicismo.
\section{Descaudado}
\begin{itemize}
\item {Grp. gram.:adj.}
\end{itemize}
Que não tem cauda.
\section{Descaudar}
\begin{itemize}
\item {Grp. gram.:v. t.}
\end{itemize}
Tirar a cauda a.
\section{Descaudato}
\begin{itemize}
\item {Grp. gram.:adj.}
\end{itemize}
(V.descaudado)
\section{Descaulino}
\begin{itemize}
\item {Grp. gram.:adj.}
\end{itemize}
\begin{itemize}
\item {Utilização:Bot.}
\end{itemize}
\begin{itemize}
\item {Proveniência:(De \textunderscore des...\textunderscore  + \textunderscore caule\textunderscore )}
\end{itemize}
Que não tem caule.
\section{Descautela}
\begin{itemize}
\item {Grp. gram.:f.}
\end{itemize}
Falta de cautela.
\section{Descavalgamento}
\begin{itemize}
\item {Grp. gram.:m.}
\end{itemize}
Acto de descavalgar.
\section{Descavalgar}
\begin{itemize}
\item {Grp. gram.:v. t.}
\end{itemize}
\begin{itemize}
\item {Grp. gram.:V. i.}
\end{itemize}
\begin{itemize}
\item {Proveniência:(De \textunderscore des...\textunderscore  + \textunderscore cavalgar\textunderscore )}
\end{itemize}
Desmontar.
Apear.
Apear-se.
\section{Descavar}
\begin{itemize}
\item {Grp. gram.:v. t.}
\end{itemize}
(V.escavar)
\section{Descaveirado}
\begin{itemize}
\item {Grp. gram.:adj.}
\end{itemize}
(V.escaveirado)
\section{Descedura}
\begin{itemize}
\item {Grp. gram.:f.}
\end{itemize}
O mesmo que \textunderscore descimento\textunderscore .
\section{Descegar}
\begin{itemize}
\item {Grp. gram.:v. t.}
\end{itemize}
\begin{itemize}
\item {Proveniência:(De \textunderscore des...\textunderscore  + \textunderscore cegar\textunderscore )}
\end{itemize}
Restituir a vista a.
\section{Descellular}
\begin{itemize}
\item {Grp. gram.:v. t.}
\end{itemize}
\begin{itemize}
\item {Proveniência:(De \textunderscore des...\textunderscore  + \textunderscore céllula\textunderscore )}
\end{itemize}
Desfazer ou tirar as céllulas a. Cf. Alves Mendes, \textunderscore Discursos\textunderscore , 15.
\section{Descelular}
\begin{itemize}
\item {Grp. gram.:v. t.}
\end{itemize}
\begin{itemize}
\item {Proveniência:(De \textunderscore des...\textunderscore  + \textunderscore célula\textunderscore )}
\end{itemize}
Desfazer ou tirar as células a. Cf. Alves Mendes, \textunderscore Discursos\textunderscore , 15.
\section{Descendência}
\begin{itemize}
\item {Grp. gram.:f.}
\end{itemize}
\begin{itemize}
\item {Proveniência:(De \textunderscore descendente\textunderscore )}
\end{itemize}
Série de pessôas que procedem de um tronco commum.
\section{Descendente}
\begin{itemize}
\item {Grp. gram.:adj.}
\end{itemize}
\begin{itemize}
\item {Grp. gram.:M.  e  f.}
\end{itemize}
\begin{itemize}
\item {Grp. gram.:M. pl.}
\end{itemize}
\begin{itemize}
\item {Grp. gram.:F.}
\end{itemize}
\begin{itemize}
\item {Grp. gram.:M.}
\end{itemize}
\begin{itemize}
\item {Proveniência:(Lat. \textunderscore descendens\textunderscore )}
\end{itemize}
Que descende.
Que desce.
Que vai de cima para baixo.
Que decresce.
Em caminhos de ferro, diz-se de tudo que está do lado opposto ao da origem da linha férrea.
Pessôa, que descende de outra ou de uma raça.
Indivíduos, que constituem uma descendência.
Descida.
Curso de água.
Vasante.
Tecto inclinado, que acompanha uma escada.

Na Índia port., indivíduo, que descende, ou pretende descender, de portugueses.
\section{Descender}
\begin{itemize}
\item {Grp. gram.:v. i.}
\end{itemize}
\begin{itemize}
\item {Proveniência:(Lat. \textunderscore descendere\textunderscore )}
\end{itemize}
Descer.
Proceder por geração.
Derivar; originar-se.
\section{Descendimento}
\begin{itemize}
\item {Grp. gram.:m.}
\end{itemize}
Acto de descer ou sêr descido.
\section{Descensão}
\begin{itemize}
\item {Grp. gram.:f.}
\end{itemize}
\begin{itemize}
\item {Proveniência:(Lat. \textunderscore descensio\textunderscore )}
\end{itemize}
O mesmo que \textunderscore descenso\textunderscore .
\section{Descensional}
\begin{itemize}
\item {Grp. gram.:adj.}
\end{itemize}
Relativo a descensão.
\section{Descenso}
\begin{itemize}
\item {Grp. gram.:m.}
\end{itemize}
\begin{itemize}
\item {Proveniência:(Lat. \textunderscore descensus\textunderscore )}
\end{itemize}
Descida.
Baixamento.
\section{Descentração}
\begin{itemize}
\item {Grp. gram.:f.}
\end{itemize}
Acto ou effeito de descentrar.
\section{Descentralização}
\begin{itemize}
\item {Grp. gram.:f.}
\end{itemize}
Acto ou effeito de descentralizar.
\section{Descentralizador}
\begin{itemize}
\item {Grp. gram.:adj.}
\end{itemize}
\begin{itemize}
\item {Grp. gram.:M.}
\end{itemize}
Que descentraliza.
Aquelle que descentraliza.
\section{Descentralizar}
\begin{itemize}
\item {Grp. gram.:v. t.}
\end{itemize}
\begin{itemize}
\item {Proveniência:(De \textunderscore des...\textunderscore  + \textunderscore centralizar\textunderscore )}
\end{itemize}
Afastar do centro.
Distribuir pelas localidades ou corporações locaes (atribuições de administração pública).
\section{Descentralizável}
\begin{itemize}
\item {Grp. gram.:adj.}
\end{itemize}
Que se póde descentralizar.
\section{Descentrar}
\begin{itemize}
\item {Grp. gram.:v.}
\end{itemize}
\begin{itemize}
\item {Utilização:t. Mecan.}
\end{itemize}
Tirar ou desviar do centro geométrico: \textunderscore a fôrça centrífuga póde fazer descentrar um veio, se houver folga nas chumaceiras\textunderscore .
\section{Descepar}
\begin{itemize}
\item {Grp. gram.:v. t.}
\end{itemize}
\begin{itemize}
\item {Utilização:Prov.}
\end{itemize}
\begin{itemize}
\item {Utilização:alent.}
\end{itemize}
Cortar ou arrancar as cepas de: \textunderscore descepar a vinha\textunderscore .
\section{Descer}
\begin{itemize}
\item {Grp. gram.:v. i.}
\end{itemize}
\begin{itemize}
\item {Utilização:Fig.}
\end{itemize}
\begin{itemize}
\item {Grp. gram.:V. t.}
\end{itemize}
\begin{itemize}
\item {Proveniência:(Lat. \textunderscore descendere\textunderscore )}
\end{itemize}
Mover-se de cima para baixo.
Declinar.
Proceder.
Rebaixar-se.
Apoucar-se.
Abaixar.
Pôr em baixo.
Percorrer, descendo: \textunderscore descer uma ladeira\textunderscore .
Fazer pender.
Apear.
\section{Descercador}
\begin{itemize}
\item {Grp. gram.:m.}
\end{itemize}
Aquelle que descerca.
\section{Descercar}
\begin{itemize}
\item {Grp. gram.:v. t.}
\end{itemize}
\begin{itemize}
\item {Proveniência:(De \textunderscore des...\textunderscore  + \textunderscore cercar\textunderscore )}
\end{itemize}
Tirar o cêrco a.
Tirar o que rodeia a.
\section{Descêrco}
\begin{itemize}
\item {Grp. gram.:m.}
\end{itemize}
Acto de descercar.
\section{Descerebração}
\begin{itemize}
\item {Grp. gram.:f.}
\end{itemize}
Acto ou effeito de descerebrar.
\section{Descerebrado}
\begin{itemize}
\item {Grp. gram.:m.  e  adj.}
\end{itemize}
\begin{itemize}
\item {Utilização:Fig.}
\end{itemize}
\begin{itemize}
\item {Proveniência:(De \textunderscore descerebrar\textunderscore )}
\end{itemize}
Ignorante.
Idiota; cretino.
\section{Descerebrar}
\begin{itemize}
\item {Grp. gram.:v. t.}
\end{itemize}
\begin{itemize}
\item {Utilização:Neol.}
\end{itemize}
\begin{itemize}
\item {Proveniência:(De \textunderscore des...\textunderscore  + \textunderscore cérebro\textunderscore )}
\end{itemize}
Tirar o juízo a.
Tornar idiota ou cretino.
\section{Desceremoniosamente}
\begin{itemize}
\item {Grp. gram.:adv.}
\end{itemize}
De modo desceremonioso.
\section{Desceremonioso}
\begin{itemize}
\item {Grp. gram.:adj.}
\end{itemize}
Que não é ceremonioso; em que não há ceremónia.
\section{Descerrar}
\begin{itemize}
\item {Grp. gram.:v. t.}
\end{itemize}
\begin{itemize}
\item {Proveniência:(De \textunderscore des...\textunderscore  + \textunderscore cerrar\textunderscore )}
\end{itemize}
Abrir (o que estava cerrado ou junto).
Patentear (aquillo que estava escondido).
\section{Deschancelar}
\begin{itemize}
\item {Grp. gram.:v. t.}
\end{itemize}
Tirar a chancela a.
\section{Deschancellar}
\begin{itemize}
\item {Grp. gram.:v. t.}
\end{itemize}
Tirar a chancella a.
\section{Deschloretação}
\begin{itemize}
\item {Grp. gram.:f.}
\end{itemize}
Acto de deschloretar.
\section{Deschloretar}
\begin{itemize}
\item {Grp. gram.:v. t.}
\end{itemize}
Separar o chloreto de.
\section{Deschristianização}
\begin{itemize}
\item {Grp. gram.:f.}
\end{itemize}
Acto ou effeito de deschristianizar.
\section{Deschristianizar}
\begin{itemize}
\item {Grp. gram.:v. t.}
\end{itemize}
\begin{itemize}
\item {Utilização:Neol.}
\end{itemize}
\begin{itemize}
\item {Proveniência:(De \textunderscore des\textunderscore  + \textunderscore christianizar\textunderscore )}
\end{itemize}
Fazer perder a qualidade de christão a.
Tirar as crenças christans a.
\section{Descida}
\begin{itemize}
\item {Grp. gram.:f.}
\end{itemize}
\begin{itemize}
\item {Proveniência:(De \textunderscore descido\textunderscore )}
\end{itemize}
Acto de descer.
Terreno inclinado, ladeira.
Deminuição, abaixamento: \textunderscore descida de cotação de fundos\textunderscore .
\section{Descifrar}
\textunderscore v. t.\textunderscore  (e der.)
O mesmo ou melhor que decifrar, etc. Cf. Filinto, XV, 224; XVII, 159; XVIII, 34; XIX, 121; XX, 235.
\section{Descimbração}
\begin{itemize}
\item {Grp. gram.:f.}
\end{itemize}
Acto de descimbrar.
\section{Descimbramento}
\begin{itemize}
\item {Grp. gram.:m.}
\end{itemize}
Acto de descimbrar.
\section{Descimbrar}
\begin{itemize}
\item {Grp. gram.:v. t.}
\end{itemize}
Tirar os cimbres a: \textunderscore descimbrar uma abóbada\textunderscore .
\section{Descimentar}
\begin{itemize}
\item {Grp. gram.:v. t.}
\end{itemize}
\begin{itemize}
\item {Proveniência:(De \textunderscore des...\textunderscore  + \textunderscore cimentar\textunderscore )}
\end{itemize}
Desfazer o cimento de.
Abalar.
Arruinar.
\section{Descimento}
\begin{itemize}
\item {Grp. gram.:m.}
\end{itemize}
Acto de descer.
\section{Descingir}
\begin{itemize}
\item {Grp. gram.:v. t.}
\end{itemize}
\begin{itemize}
\item {Proveniência:(De \textunderscore des...\textunderscore  + \textunderscore cingir\textunderscore )}
\end{itemize}
Desapertar.
Alargar.
Tirar (aquillo que cingia).
\section{Desclaridade}
\begin{itemize}
\item {Grp. gram.:f.}
\end{itemize}
Falta de claridade. Cf. Filinto, \textunderscore D. Man.\textunderscore , III, 245.
\section{Desclassificação}
\begin{itemize}
\item {Grp. gram.:f.}
\end{itemize}
Acto de desclassificar.
\section{Desclassificado}
\begin{itemize}
\item {Grp. gram.:m.  e  adj.}
\end{itemize}
\begin{itemize}
\item {Utilização:Neol.}
\end{itemize}
\begin{itemize}
\item {Proveniência:(De \textunderscore desclassificar\textunderscore )}
\end{itemize}
Indivíduo, que pelo seu procedimento, é indigno de consideração; desacreditado: \textunderscore não me bato com um desclassificado\textunderscore .
\section{Desclassificar}
\begin{itemize}
\item {Grp. gram.:v. t.}
\end{itemize}
\begin{itemize}
\item {Utilização:Neol.}
\end{itemize}
\begin{itemize}
\item {Proveniência:(De \textunderscore des...\textunderscore  + \textunderscore classificar\textunderscore )}
\end{itemize}
Tirar ou deslocar de uma classe.
Deshonrar moralmente; desacreditar; aviltar.
\section{Descloretação}
\begin{itemize}
\item {Grp. gram.:f.}
\end{itemize}
Acto de descloretar.
\section{Descloretar}
\begin{itemize}
\item {Grp. gram.:v. t.}
\end{itemize}
Separar o cloreto de.
\section{Descoagulação}
\begin{itemize}
\item {Grp. gram.:f.}
\end{itemize}
Acto ou effeito de descoagular.
\section{Descoagulamento}
\begin{itemize}
\item {Grp. gram.:m.}
\end{itemize}
(V.descoagulação)
\section{Descoagulante}
\begin{itemize}
\item {Grp. gram.:adj.}
\end{itemize}
Que descoagula.
\section{Descoagular}
\begin{itemize}
\item {Grp. gram.:v. t.}
\end{itemize}
\begin{itemize}
\item {Proveniência:(De \textunderscore des...\textunderscore  + \textunderscore coagular\textunderscore )}
\end{itemize}
Tornar líquido (aquillo que estava coalhado).
Fundir.
\section{Descoalhar}
\begin{itemize}
\item {Grp. gram.:v. t.}
\end{itemize}
O mesmo que \textunderscore descoagular\textunderscore .
\section{Descoalho}
\begin{itemize}
\item {Grp. gram.:m.}
\end{itemize}
O mesmo que \textunderscore degêlo\textunderscore .
Acto de descoagular.
\section{Descoberta}
\begin{itemize}
\item {Grp. gram.:f.}
\end{itemize}
\begin{itemize}
\item {Proveniência:(De \textunderscore descoberto\textunderscore )}
\end{itemize}
Coisa, que se descobriu.
Invento.
Terra, que se descobriu de novo ou pela primeira vez.--Por \textunderscore acto de descobrir\textunderscore , é termo afrancesado; os asseados no escrever, disse Castilho numas notas inéditas ao diccionário de Moraes, só escrevem \textunderscore descobrimento\textunderscore .
\section{Descobertamente}
\begin{itemize}
\item {Grp. gram.:adv.}
\end{itemize}
\begin{itemize}
\item {Proveniência:(De \textunderscore descoberto\textunderscore )}
\end{itemize}
Publicamente; ás claras.
\section{Descoberto}
\begin{itemize}
\item {Grp. gram.:Adj.}
\end{itemize}
\begin{itemize}
\item {Grp. gram.:M.}
\end{itemize}
\begin{itemize}
\item {Utilização:Bras. de Minas}
\end{itemize}
\begin{itemize}
\item {Proveniência:(De \textunderscore des...\textunderscore  + \textunderscore coberto\textunderscore )}
\end{itemize}
Que não está coberto.
Patente.
Divulgado: segrêdo descoberto.
Descobrimento (de minas ou de veios mineraes): \textunderscore fizeram-se novos descobertos de lavras auríferas\textunderscore .
\section{Descobertura}
\begin{itemize}
\item {Grp. gram.:f.}
\end{itemize}
(V.descoberta)
\section{Descobridor}
\begin{itemize}
\item {Grp. gram.:adj.}
\end{itemize}
\begin{itemize}
\item {Grp. gram.:M.}
\end{itemize}
Que descobre.
Aquelle que descobre ou descobriu: \textunderscore o descobridor da América\textunderscore .
Em legislação fiscal, diz-se \textunderscore descobridor\textunderscore  o indivíduo que primeiro dá notícia de uma fraude e que, por isso, tem a melhor parte na multa que fôr imposta.
\section{Descobrimento}
\begin{itemize}
\item {Grp. gram.:m.}
\end{itemize}
Acto ou effeito de descobrir.
\section{Descobrir}
\begin{itemize}
\item {Grp. gram.:v. t.}
\end{itemize}
\begin{itemize}
\item {Grp. gram.:V. i.}
\end{itemize}
\begin{itemize}
\item {Grp. gram.:V. p.}
\end{itemize}
\begin{itemize}
\item {Proveniência:(De \textunderscore des...\textunderscore  + \textunderscore cobrir\textunderscore )}
\end{itemize}
Levantar ou tirar aquillo que cobria (qualquer coisa).
Patentear, pôr á vista: \textunderscore descobrir o peito\textunderscore .
Inventar: \textunderscore não descobriu a pólvora\textunderscore .
Avistar: \textunderscore descobrir um vulto\textunderscore .
Manifestar.
Divulgar.
Denunciar: \textunderscore descobrir contrabando\textunderscore .
Encontrar.
Reconhecer.
Clarear a atmosphera.
Tirar da cabeça o chapéu.
\section{Descocadamente}
\begin{itemize}
\item {Grp. gram.:adv.}
\end{itemize}
Com descôco.
\section{Descocar-se}
\begin{itemize}
\item {Grp. gram.:v. p.}
\end{itemize}
Proceder com descôco.
\section{Descochado}
\begin{itemize}
\item {Grp. gram.:adj.}
\end{itemize}
\begin{itemize}
\item {Utilização:Bras}
\end{itemize}
Que não tem brio; desvergonhado.
(Relaciona-se com \textunderscore descochar\textunderscore ?)
\section{Descochar}
\begin{itemize}
\item {Grp. gram.:v.}
\end{itemize}
\begin{itemize}
\item {Utilização:t. Náut.}
\end{itemize}
\begin{itemize}
\item {Proveniência:(De \textunderscore des...\textunderscore  + \textunderscore cochar\textunderscore )}
\end{itemize}
Destorcer (cabos de navio), para se utilizarem separadamente os cordões.
\section{Descôco}
\begin{itemize}
\item {Grp. gram.:m.}
\end{itemize}
\begin{itemize}
\item {Utilização:Fam.}
\end{itemize}
\begin{itemize}
\item {Proveniência:(De \textunderscore des...\textunderscore  + \textunderscore côco\textunderscore ^2)}
\end{itemize}
Descaramento; atrevimento.
Insensatez; disparate.
\section{Descodear}
\begin{itemize}
\item {Grp. gram.:v. t.}
\end{itemize}
Tirar a côdea a.
\section{Descoimar}
\begin{itemize}
\item {Grp. gram.:v. t.}
\end{itemize}
Alliviar da cóima.
Desobrigar do pagamento de multa. Cf. Camillo, \textunderscore Cav. em Ruínas\textunderscore , 8.
\section{Descoitado}
\begin{itemize}
\item {Grp. gram.:adj.}
\end{itemize}
\begin{itemize}
\item {Utilização:Prov.}
\end{itemize}
\begin{itemize}
\item {Utilização:beir.}
\end{itemize}
Diz-se das árvores, ou das propriedades rústicas, a que já se colheram os frutos: \textunderscore uma pereira descoitada\textunderscore .
\section{Descoitar}
\begin{itemize}
\item {Grp. gram.:v. t.}
\end{itemize}
\begin{itemize}
\item {Proveniência:(De \textunderscore des...\textunderscore  + \textunderscore coitar\textunderscore )}
\end{itemize}
Tirar os privilégios de coito a (uma propriedade). Cf. Herculano, \textunderscore M. de Cister\textunderscore , II, 64.
\section{Descollar}
\begin{itemize}
\item {Grp. gram.:v. t.}
\end{itemize}
Desligar, despegar (aquillo que estava collado).
\section{Descolmar}
\begin{itemize}
\item {Grp. gram.:v. t.}
\end{itemize}
\begin{itemize}
\item {Proveniência:(De \textunderscore des...\textunderscore  + \textunderscore colmar\textunderscore )}
\end{itemize}
Tirar ou arrancar o colmo a.
Desguarnecer de colmo: \textunderscore descolmar um casebre\textunderscore .
\section{Descoloração}
\begin{itemize}
\item {Grp. gram.:f.}
\end{itemize}
Acto de descolorar.
\section{Descolorante}
\begin{itemize}
\item {Grp. gram.:adj.}
\end{itemize}
O mesmo que \textunderscore descòrante\textunderscore .
\section{Descolorar}
\begin{itemize}
\item {Grp. gram.:v. t.}
\end{itemize}
\begin{itemize}
\item {Proveniência:(De \textunderscore des...\textunderscore  + \textunderscore colorar\textunderscore )}
\end{itemize}
Privar de côr.
Descòrar.
\section{Descolorir}
\begin{itemize}
\item {Grp. gram.:v. t.}
\end{itemize}
\begin{itemize}
\item {Grp. gram.:V. i.}
\end{itemize}
O mesmo que \textunderscore descolorar\textunderscore .
Perder a côr, desbotar: \textunderscore êste tecido descoloriu\textunderscore .
\section{Descomedidamente}
\begin{itemize}
\item {Grp. gram.:adv.}
\end{itemize}
Com descomedimento.
\section{Descomedimento}
\begin{itemize}
\item {Grp. gram.:m.}
\end{itemize}
Acto de descomedir-se.
\section{Descomer}
\begin{itemize}
\item {Grp. gram.:v. i.}
\end{itemize}
\begin{itemize}
\item {Utilização:Chul.}
\end{itemize}
\begin{itemize}
\item {Proveniência:(De \textunderscore des...\textunderscore  + \textunderscore comer\textunderscore )}
\end{itemize}
Evacuar os intestinos; defecar.
\section{Descometer}
\begin{itemize}
\item {Grp. gram.:v. t.}
\end{itemize}
\begin{itemize}
\item {Proveniência:(De \textunderscore des...\textunderscore  + \textunderscore cometer\textunderscore )}
\end{itemize}
Livrar do encargo de.
Desobrigar.
\section{Descommedidamente}
\begin{itemize}
\item {Grp. gram.:adv.}
\end{itemize}
Com descommedimento.
\section{Descommedimento}
\begin{itemize}
\item {Grp. gram.:m.}
\end{itemize}
Acto de descommedir-se.
\section{Descommedir-se}
\begin{itemize}
\item {Grp. gram.:v. p.}
\end{itemize}
\begin{itemize}
\item {Proveniência:(De \textunderscore des...\textunderscore  + \textunderscore commedir\textunderscore )}
\end{itemize}
Praticar excessos.
Sêr inconveniente.
Disparatar.
\section{Descommeter}
\begin{itemize}
\item {Grp. gram.:v. t.}
\end{itemize}
\begin{itemize}
\item {Proveniência:(De \textunderscore des...\textunderscore  + \textunderscore commeter\textunderscore )}
\end{itemize}
Livrar do encargo de.
Desobrigar.
\section{Descommodidade}
\begin{itemize}
\item {Grp. gram.:f.}
\end{itemize}
Falta de commodidade.
\section{Descómmodo}
\begin{itemize}
\item {Grp. gram.:m.}
\end{itemize}
\begin{itemize}
\item {Utilização:Des.}
\end{itemize}
O mesmo que \textunderscore incômmodo\textunderscore ^1.
\section{Descommover}
\begin{itemize}
\item {Grp. gram.:v. t.}
\end{itemize}
\begin{itemize}
\item {Proveniência:(De \textunderscore des...\textunderscore  + \textunderscore commover\textunderscore )}
\end{itemize}
Tirar a commoção a.
Serenar. Cf. Camillo, \textunderscore Estrêl. Prop.\textunderscore , 108.
\section{Descommunal}
\begin{itemize}
\item {Grp. gram.:adj.}
\end{itemize}
\begin{itemize}
\item {Proveniência:(De \textunderscore des...\textunderscore  + \textunderscore communal\textunderscore )}
\end{itemize}
Que é fóra do commum.
Extraordinário; colossal.
\section{Descommunaleza}
\begin{itemize}
\item {Grp. gram.:f.}
\end{itemize}
\begin{itemize}
\item {Utilização:Des.}
\end{itemize}
Qualidade de descommunal.
\section{Descommunalmente}
\begin{itemize}
\item {Grp. gram.:adv.}
\end{itemize}
De modo descommunal.
\section{Descommungar}
\begin{itemize}
\item {Grp. gram.:v. t.}
\end{itemize}
Levantar a excommunhão a.
(Por \textunderscore desexcommungar\textunderscore , de \textunderscore des...\textunderscore  + \textunderscore excommungar\textunderscore )
\section{Descommunhão}
\begin{itemize}
\item {Grp. gram.:f.}
\end{itemize}
Acto de descommungar.
\section{Descomodidade}
\begin{itemize}
\item {Grp. gram.:f.}
\end{itemize}
Falta de comodidade.
\section{Descómodo}
\begin{itemize}
\item {Grp. gram.:m.}
\end{itemize}
\begin{itemize}
\item {Utilização:Des.}
\end{itemize}
O mesmo que \textunderscore incômodo\textunderscore ^1.
\section{Descomover}
\begin{itemize}
\item {Grp. gram.:v. t.}
\end{itemize}
\begin{itemize}
\item {Proveniência:(De \textunderscore des...\textunderscore  + \textunderscore comover\textunderscore )}
\end{itemize}
Tirar a comoção a.
Serenar. Cf. Camillo, \textunderscore Estrêl. Prop.\textunderscore , 108.
\section{Descompadrar}
\begin{itemize}
\item {Grp. gram.:v. t.}
\end{itemize}
\begin{itemize}
\item {Utilização:Pop.}
\end{itemize}
\begin{itemize}
\item {Proveniência:(De \textunderscore des...\textunderscore  + \textunderscore compadre\textunderscore )}
\end{itemize}
Malquistar; tornar inimigo: \textunderscore a rivalidade descompadrou-os\textunderscore .
\section{Descompaixão}
\begin{itemize}
\item {Grp. gram.:f.}
\end{itemize}
Falta de compaixão.
\section{Descompanhar}
\textunderscore v. t.\textunderscore  (e der.)
O mesmo que \textunderscore desacompanhar\textunderscore . Cf. Filinto, XIV, 20.
\section{Descompassadamente}
\begin{itemize}
\item {Grp. gram.:adv.}
\end{itemize}
De modo descompassado.
Enormemente; extraordinariamente.
\section{Descompassado}
\begin{itemize}
\item {Grp. gram.:adj.}
\end{itemize}
Enorme; desmedido:«\textunderscore ...tão descompassada bêsta\textunderscore ». Camillo.
\section{Descompassar}
\begin{itemize}
\item {Grp. gram.:v. t.}
\end{itemize}
\begin{itemize}
\item {Proveniência:(De \textunderscore des...\textunderscore  + \textunderscore compassar\textunderscore )}
\end{itemize}
Estender muito; realizar sem medida, sem limites.
Desviar das conveniências.
\section{Descompasso}
\begin{itemize}
\item {Grp. gram.:m.}
\end{itemize}
\begin{itemize}
\item {Proveniência:(De \textunderscore des...\textunderscore  + \textunderscore compasso\textunderscore )}
\end{itemize}
Falta de compasso ou de medida; irregularidade.
\section{Descompensar}
\begin{itemize}
\item {Grp. gram.:v. t.}
\end{itemize}
\begin{itemize}
\item {Utilização:Ant.}
\end{itemize}
\begin{itemize}
\item {Proveniência:(De \textunderscore des...\textunderscore  + \textunderscore compensar\textunderscore )}
\end{itemize}
Dispensar.
Exceptuar da retribuição.
Abater; descontar.
\section{Descomponenda}
\begin{itemize}
\item {Grp. gram.:f.}
\end{itemize}
\begin{itemize}
\item {Utilização:Fam.}
\end{itemize}
Descompostura, reprehensão.
(Cp. \textunderscore descompor\textunderscore )
\section{Descompor}
\begin{itemize}
\item {Grp. gram.:v. t.}
\end{itemize}
\begin{itemize}
\item {Utilização:Fig.}
\end{itemize}
\begin{itemize}
\item {Proveniência:(De \textunderscore des...\textunderscore  + \textunderscore compor\textunderscore )}
\end{itemize}
Tirar do lugar próprio.
Tirar a feição regular de; alterar: \textunderscore o mêdo descompôs-lhe as feições\textunderscore .
Desordenar.
Desadornar; despir.
Injuriar.
Censurar acremente.
\section{Descomposição}
\begin{itemize}
\item {Grp. gram.:f.}
\end{itemize}
Acto de descompor.
Decomposição.
\section{Descompostamente}
\begin{itemize}
\item {Grp. gram.:adv.}
\end{itemize}
\begin{itemize}
\item {Proveniência:(De \textunderscore descompor\textunderscore )}
\end{itemize}
Em desordem.
Com acrimónia.
\section{Descompostura}
\begin{itemize}
\item {Grp. gram.:f.}
\end{itemize}
Acto ou effeito de descompor.
Censura acrimoniosa: \textunderscore dar uma descompostura em alguém\textunderscore .
\section{Descomprazente}
\begin{itemize}
\item {Grp. gram.:adj.}
\end{itemize}
Que descompraz.
\section{Descomprazer}
\begin{itemize}
\item {Grp. gram.:v. i.}
\end{itemize}
Não comprazer.
Não condescender.
\section{Descomprensada}
\begin{itemize}
\item {Grp. gram.:f.  e  adj.}
\end{itemize}
\begin{itemize}
\item {Utilização:Prov.}
\end{itemize}
\begin{itemize}
\item {Utilização:trasm.}
\end{itemize}
\begin{itemize}
\item {Proveniência:(De \textunderscore des...\textunderscore  + \textunderscore com\textunderscore  + \textunderscore pressa\textunderscore ?)}
\end{itemize}
Diz-se da mulher indolente, desleixada.
\section{Descomunal}
\begin{itemize}
\item {Grp. gram.:adj.}
\end{itemize}
\begin{itemize}
\item {Proveniência:(De \textunderscore des...\textunderscore  + \textunderscore comunal\textunderscore )}
\end{itemize}
Que é fóra do comum.
Extraordinário; colossal.
\section{Descomunaleza}
\begin{itemize}
\item {Grp. gram.:f.}
\end{itemize}
\begin{itemize}
\item {Utilização:Des.}
\end{itemize}
Qualidade de descomunal.
\section{Descomunalmente}
\begin{itemize}
\item {Grp. gram.:adv.}
\end{itemize}
De modo descomunal.
\section{Descomungar}
\begin{itemize}
\item {Grp. gram.:v. t.}
\end{itemize}
Levantar a excomunhão a.
(Por \textunderscore desexcomungar\textunderscore , de \textunderscore des...\textunderscore  + \textunderscore excomungar\textunderscore )
\section{Descomunhão}
\begin{itemize}
\item {Grp. gram.:f.}
\end{itemize}
Acto de descomungar.
\section{Desconceito}
\begin{itemize}
\item {Grp. gram.:m.}
\end{itemize}
\begin{itemize}
\item {Proveniência:(De \textunderscore des...\textunderscore  + \textunderscore conceito\textunderscore )}
\end{itemize}
Mau conceito.
Descrédito; má fama.
\section{Desconceituar}
\begin{itemize}
\item {Grp. gram.:v. t.}
\end{itemize}
\begin{itemize}
\item {Proveniência:(De \textunderscore des...\textunderscore  + \textunderscore conceituar\textunderscore )}
\end{itemize}
Tirar o bom conceito de; desacreditar.
\section{Desconcentrar}
\begin{itemize}
\item {Grp. gram.:v. t.}
\end{itemize}
\begin{itemize}
\item {Proveniência:(De \textunderscore des...\textunderscore  + \textunderscore concentrar\textunderscore )}
\end{itemize}
Tirar do centro; descentralizar. Cf. Garrett, \textunderscore Helena\textunderscore , 15.
\section{Desconcertadamente}
\begin{itemize}
\item {Grp. gram.:adv.}
\end{itemize}
Com desconcêrto.
\section{Desconcertador}
\begin{itemize}
\item {Grp. gram.:adj.}
\end{itemize}
\begin{itemize}
\item {Grp. gram.:M.}
\end{itemize}
Que desconcerta.
Aquelle que desconcerta.
\section{Desconcertante}
\begin{itemize}
\item {Grp. gram.:adj.}
\end{itemize}
Que desconcerta.
Que embaraça, que atordôa.
\section{Desconcertar}
\begin{itemize}
\item {Grp. gram.:v. t.}
\end{itemize}
\begin{itemize}
\item {Grp. gram.:V. i.}
\end{itemize}
\begin{itemize}
\item {Proveniência:(De \textunderscore des...\textunderscore  + \textunderscore concertar\textunderscore )}
\end{itemize}
Fazer perder o concêrto, a bôa disposição, de.
Descompor, desarranjar: \textunderscore desconcertar um relógio\textunderscore .
Pôr em divergência: \textunderscore a política desconcertou-os\textunderscore .
Discordar.
Disparatar.
\section{Desconcêrto}
\begin{itemize}
\item {Grp. gram.:m.}
\end{itemize}
Acto ou effeito de desconcertar.
Desarranjo; desordem:«\textunderscore vêde dêste mundo o desconcerto!\textunderscore »Bocage.
\section{Desconchavar}
\begin{itemize}
\item {Grp. gram.:v. t.}
\end{itemize}
\begin{itemize}
\item {Utilização:Fig.}
\end{itemize}
\begin{itemize}
\item {Grp. gram.:V. i.}
\end{itemize}
\begin{itemize}
\item {Proveniência:(De \textunderscore des...\textunderscore  + \textunderscore conchavar\textunderscore )}
\end{itemize}
Desencaixar.
Desligar.
Malquistar.
Tornar desavindo.
Disparatar.
\section{Desconchavo}
\begin{itemize}
\item {Grp. gram.:m.}
\end{itemize}
Despautério, tolíce.
Acto de desconchavar.
\section{Desconchegar}
\begin{itemize}
\item {Grp. gram.:v. t.}
\end{itemize}
\begin{itemize}
\item {Proveniência:(De \textunderscore des...\textunderscore  + \textunderscore conchegar\textunderscore )}
\end{itemize}
Separar (aquillo que estava conchegado).
Desaproximar.
Desacommodar.
\section{Desconchego}
\begin{itemize}
\item {fónica:chê}
\end{itemize}
\begin{itemize}
\item {Grp. gram.:m.}
\end{itemize}
Falta de conchego.
Acto de desconchegar.
\section{Desconciliação}
\begin{itemize}
\item {Grp. gram.:f.}
\end{itemize}
Acto de desconciliar.
\section{Desconciliar}
\begin{itemize}
\item {Grp. gram.:v. t.}
\end{itemize}
\begin{itemize}
\item {Proveniência:(De \textunderscore des...\textunderscore  + \textunderscore conciliar\textunderscore )}
\end{itemize}
Tornar desavindo.
Quebrar a conciliação de.
\section{Desconcordação}
\begin{itemize}
\item {Grp. gram.:f.}
\end{itemize}
O mesmo que \textunderscore desconcordância\textunderscore .
\section{Desconcordância}
\begin{itemize}
\item {Grp. gram.:f.}
\end{itemize}
Falta de concordância.
\section{Desconcordante}
\begin{itemize}
\item {Grp. gram.:adj.}
\end{itemize}
Que desconcorda.
\section{Desconcordar}
\begin{itemize}
\item {Grp. gram.:v. t.}
\end{itemize}
\begin{itemize}
\item {Grp. gram.:V. i.}
\end{itemize}
\begin{itemize}
\item {Proveniência:(De \textunderscore des...\textunderscore  + \textunderscore concordar\textunderscore )}
\end{itemize}
Desavir; separar por desacôrdo.
Não concordar.
\section{Desconcorde}
\begin{itemize}
\item {Grp. gram.:adj.}
\end{itemize}
(V.desconcordante)
\section{Desconcórdia}
\begin{itemize}
\item {Grp. gram.:f.}
\end{itemize}
Falta de concórdia.
\section{Descondensar}
\begin{itemize}
\item {Grp. gram.:v. t.}
\end{itemize}
\begin{itemize}
\item {Proveniência:(De \textunderscore des...\textunderscore  + \textunderscore condensar\textunderscore )}
\end{itemize}
Tirar a qualidade de denso a.
Dissolver; tornar tênue:«\textunderscore descondensaria o mais espêsso da sua escuridão\textunderscore ». Camillo, \textunderscore Caveira\textunderscore , 10.
\section{Desconexão}
\begin{itemize}
\item {fónica:csão}
\end{itemize}
\begin{itemize}
\item {Grp. gram.:f.}
\end{itemize}
Falta de conexão.
\section{Desconexo}
\begin{itemize}
\item {fónica:cso}
\end{itemize}
\begin{itemize}
\item {Grp. gram.:adj.}
\end{itemize}
\begin{itemize}
\item {Proveniência:(De \textunderscore des...\textunderscore  + \textunderscore conexo\textunderscore )}
\end{itemize}
Que não tem conexão.
\section{Desconfeito}
\begin{itemize}
\item {Grp. gram.:adj.}
\end{itemize}
\begin{itemize}
\item {Utilização:P. us.}
\end{itemize}
Desconjuntado.
Prestes a desmanchar-se:«\textunderscore baloiça o baixel torpe e desconfeito\textunderscore ». Garrett, \textunderscore Fl. sem Fruto\textunderscore , 197.
\section{Desconfessar}
\begin{itemize}
\item {Grp. gram.:v. t.}
\end{itemize}
\begin{itemize}
\item {Proveniência:(De \textunderscore des...\textunderscore  + \textunderscore confessar\textunderscore )}
\end{itemize}
Desdizer-se de.
Dar por não dito ou não confessado. Cf. Benalcanfor, \textunderscore Cartas de Viagem\textunderscore , XLIV.
\section{Desconfiadamente}
\begin{itemize}
\item {Grp. gram.:adv.}
\end{itemize}
Com desconfiança.
\section{Desconfiado}
\begin{itemize}
\item {Grp. gram.:adj.}
\end{itemize}
\begin{itemize}
\item {Utilização:Fam.}
\end{itemize}
\begin{itemize}
\item {Proveniência:(De \textunderscore desconfiar\textunderscore )}
\end{itemize}
Que desconfia.
Que não confia.
Receoso.
Que se agasta facilmente.
Que toma em mau sentido palavras ou actos inoffensivos.
\section{Desconfiança}
\begin{itemize}
\item {Grp. gram.:f.}
\end{itemize}
Falta de confiança.
\section{Desconfiante}
\begin{itemize}
\item {Grp. gram.:adj.}
\end{itemize}
\begin{itemize}
\item {Proveniência:(De \textunderscore desconfiar\textunderscore )}
\end{itemize}
Que tem desconfiança.
\section{Desconfiar}
\begin{itemize}
\item {Grp. gram.:v. t.}
\end{itemize}
\begin{itemize}
\item {Grp. gram.:V. i.}
\end{itemize}
\begin{itemize}
\item {Utilização:Gír.}
\end{itemize}
\begin{itemize}
\item {Proveniência:(De \textunderscore des...\textunderscore  + \textunderscore confiar\textunderscore )}
\end{itemize}
Conjecturar.
Têr supposição de: \textunderscore desconfio que me enganas\textunderscore .
Deixar de têr confiança.
Duvidar: \textunderscore desconfío da tua promessa\textunderscore .
Ir-se embora.
Zangar-se.
\section{Desconfioso}
\begin{itemize}
\item {Grp. gram.:adj.}
\end{itemize}
\begin{itemize}
\item {Utilização:P. us.}
\end{itemize}
O mesmo que \textunderscore desconfiado\textunderscore . Cf. Filinto, \textunderscore D. Man.\textunderscore , III, 162.
\section{Desconforme}
\begin{itemize}
\item {Grp. gram.:adj.}
\end{itemize}
\begin{itemize}
\item {Proveniência:(De \textunderscore des...\textunderscore  + \textunderscore conforme\textunderscore )}
\end{itemize}
Que não é conforme.
Que se não conforma.
Grandioso; descommunal.
\section{Desconformemente}
\begin{itemize}
\item {Grp. gram.:adv.}
\end{itemize}
De modo desconforme.
\section{Desconformidade}
\begin{itemize}
\item {Grp. gram.:f.}
\end{itemize}
Falta de conformidade.
Divergência; desharmonia.
\section{Desconfortadamente}
\begin{itemize}
\item {Grp. gram.:adv.}
\end{itemize}
Sem confôrto.
\section{Desconfortar}
\begin{itemize}
\item {Grp. gram.:v. t.}
\end{itemize}
\begin{itemize}
\item {Proveniência:(De \textunderscore des...\textunderscore  + \textunderscore confortar\textunderscore )}
\end{itemize}
Tirar o confôrto a.
Desalentar; desconsolar.
\section{Desconfortável}
\begin{itemize}
\item {Grp. gram.:adj.}
\end{itemize}
Que se póde desconfortar.
\section{Desconfôrto}
\begin{itemize}
\item {Grp. gram.:m.}
\end{itemize}
Falta de confôrto.
\section{Desconfranger}
\begin{itemize}
\item {Grp. gram.:v. t.}
\end{itemize}
\begin{itemize}
\item {Proveniência:(De \textunderscore des...\textunderscore  + \textunderscore confranger\textunderscore )}
\end{itemize}
Tirar o confrangimento a. Cf. Arn. Gama, \textunderscore Um Motim\textunderscore , 558.
\section{Descongelação}
\begin{itemize}
\item {Grp. gram.:f.}
\end{itemize}
Acto de descongelar.
\section{Descongelar}
\begin{itemize}
\item {Grp. gram.:v. t.}
\end{itemize}
\begin{itemize}
\item {Proveniência:(De \textunderscore des...\textunderscore  + \textunderscore congelar\textunderscore )}
\end{itemize}
Derreter (aquillo que estava congelado).
\section{Descongestionar}
\begin{itemize}
\item {Grp. gram.:v. t.}
\end{itemize}
\begin{itemize}
\item {Proveniência:(De \textunderscore des...\textunderscore  + \textunderscore congestionar\textunderscore )}
\end{itemize}
Livrar de congestão a.
\section{Desconhecedor}
\begin{itemize}
\item {Grp. gram.:adj.}
\end{itemize}
\begin{itemize}
\item {Grp. gram.:M.}
\end{itemize}
Que desconhece.
Aquelle que desconhece.
\section{Desconhecer}
\begin{itemize}
\item {Grp. gram.:v. t.}
\end{itemize}
Não conhecer.
Não se lembrar de.
Desagradecer: \textunderscore desconhecer obséquios\textunderscore .
\section{Desconhecidamente}
\begin{itemize}
\item {Grp. gram.:adv.}
\end{itemize}
Com desconhecimento.
\section{Desconhecido}
\begin{itemize}
\item {Grp. gram.:m.}
\end{itemize}
Indivíduo, que não é conhecido.
\section{Desconhecimento}
\begin{itemize}
\item {Grp. gram.:m.}
\end{itemize}
Acto de desconhecer.
\section{Desconhecível}
\begin{itemize}
\item {Grp. gram.:adj.}
\end{itemize}
Que se não póde conhecer:«\textunderscore mais desfigurado, mais desconhecível, mais horrendo\textunderscore ». Castilho, \textunderscore Escav. Poét.\textunderscore , 47.
\section{Desconjunção}
\begin{itemize}
\item {Grp. gram.:f.}
\end{itemize}
O mesmo que desconjuntamento.
\section{Desconjuncção}
\begin{itemize}
\item {Grp. gram.:f.}
\end{itemize}
O mesmo que desconjuntamento.
\section{Desconjuntamento}
\begin{itemize}
\item {Grp. gram.:m.}
\end{itemize}
Acto ou effeito de desconjuntar.
\section{Desconjuntar}
\begin{itemize}
\item {Grp. gram.:v. t.}
\end{itemize}
\begin{itemize}
\item {Proveniência:(De \textunderscore des...\textunderscore  + \textunderscore conjunto\textunderscore )}
\end{itemize}
Tirar das articulações.
Desligar.
Deslocar.
\section{Desconjunto}
\begin{itemize}
\item {Grp. gram.:adj.}
\end{itemize}
\begin{itemize}
\item {Proveniência:(De \textunderscore des...\textunderscore  + \textunderscore conjunto\textunderscore )}
\end{itemize}
Separado.
Discordante.
\section{Desconjuntura}
\begin{itemize}
\item {Grp. gram.:f.}
\end{itemize}
O mesmo que \textunderscore desconjuntamento\textunderscore .
\section{Desconjurar}
\begin{itemize}
\item {Grp. gram.:v. t.}
\end{itemize}
O mesmo que \textunderscore esconjurar\textunderscore .
Offender; desacatar:«\textunderscore João Machado se apaixonava de vêr aquelle homem desconjurar a sua santa fé\textunderscore ». Filinto, \textunderscore D. Man.\textunderscore , II, 266.
\section{Desconnexão}
\begin{itemize}
\item {fónica:csão}
\end{itemize}
\begin{itemize}
\item {Grp. gram.:f.}
\end{itemize}
Falta de connexão.
\section{Desconnexo}
\begin{itemize}
\item {fónica:cso}
\end{itemize}
\begin{itemize}
\item {Grp. gram.:adj.}
\end{itemize}
\begin{itemize}
\item {Proveniência:(De \textunderscore des...\textunderscore  + \textunderscore connexo\textunderscore )}
\end{itemize}
Que não tem connexão.
\section{Desconsagração}
\begin{itemize}
\item {Grp. gram.:f.}
\end{itemize}
Acto de desconsagrar.
\section{Desconsagrar}
\begin{itemize}
\item {Grp. gram.:v. t.}
\end{itemize}
\begin{itemize}
\item {Proveniência:(De \textunderscore des...\textunderscore  + \textunderscore consagrar\textunderscore )}
\end{itemize}
Profanar.
\section{Desconsciência}
\begin{itemize}
\item {Grp. gram.:f.}
\end{itemize}
\begin{itemize}
\item {Utilização:Ant.}
\end{itemize}
Falta de consciência.
Inconsciência.
\section{Desconselhar}
\textunderscore v. t.\textunderscore  (e der.)
O mesmo que \textunderscore desaconselhar\textunderscore , etc. Cf. Filinto, XVIII, 202.
\section{Desconsentimento}
\begin{itemize}
\item {Grp. gram.:m.}
\end{itemize}
Acto ou effeito de desconsentir.
\section{Desconsentir}
\begin{itemize}
\item {Grp. gram.:v. t.  e  i.}
\end{itemize}
Não consentir.
\section{Desconsideração}
\begin{itemize}
\item {Grp. gram.:f.}
\end{itemize}
Falta de consideração; desrespeito.
\section{Desconsiderar}
\begin{itemize}
\item {Grp. gram.:v. t.}
\end{itemize}
Não considerar.
Tratar sem respeito: \textunderscore desconsiderar os pais\textunderscore .
\section{Desconsoante}
\begin{itemize}
\item {Grp. gram.:adj.}
\end{itemize}
\begin{itemize}
\item {Proveniência:(De \textunderscore des...\textunderscore  + \textunderscore consoante\textunderscore )}
\end{itemize}
Que não está de acôrdo.
Divergente. Cf. Filinto, VI, 227.
\section{Desconsolação}
\begin{itemize}
\item {Grp. gram.:f.}
\end{itemize}
Falta de consolação.
\section{Desconsoladamente}
\begin{itemize}
\item {Grp. gram.:adv.}
\end{itemize}
De modo desconsolado.
\section{Desconsoladeza}
\begin{itemize}
\item {Grp. gram.:f.}
\end{itemize}
\begin{itemize}
\item {Utilização:Burl.}
\end{itemize}
\begin{itemize}
\item {Proveniência:(De \textunderscore desconsolar\textunderscore )}
\end{itemize}
Desconsolação.
\section{Desconsolado}
\begin{itemize}
\item {Grp. gram.:adj.}
\end{itemize}
\begin{itemize}
\item {Utilização:Fam.}
\end{itemize}
\begin{itemize}
\item {Proveniência:(De \textunderscore des...\textunderscore  + \textunderscore consolado\textunderscore )}
\end{itemize}
Que não tem consolação; consternado.
Que não tem graça nem animação: \textunderscore cara desconsolada\textunderscore .
Insípido: \textunderscore uma comida desconsolada\textunderscore .
\section{Desconsolador}
\begin{itemize}
\item {Grp. gram.:adj.}
\end{itemize}
\begin{itemize}
\item {Grp. gram.:M.}
\end{itemize}
Que desconsola.
Aquelle, que desconsola.
\section{Desconsolar}
\begin{itemize}
\item {Grp. gram.:v. t.}
\end{itemize}
\begin{itemize}
\item {Proveniência:(De \textunderscore des...\textunderscore  + \textunderscore consolar\textunderscore )}
\end{itemize}
Causar desconsolação a.
\section{Desconsolativo}
\begin{itemize}
\item {Grp. gram.:adj.}
\end{itemize}
\begin{itemize}
\item {Proveniência:(De \textunderscore desconsolar\textunderscore )}
\end{itemize}
Que desconsola.
\section{Desconsolatriz}
\begin{itemize}
\item {Grp. gram.:adj. f.}
\end{itemize}
Que desconsola. Cf. Filinto, XI, 173.
(Cp. \textunderscore desconsolador\textunderscore )
\section{Desconsolável}
\begin{itemize}
\item {Grp. gram.:adj.}
\end{itemize}
O mesmo que inconsolável. Cf. Camillo, \textunderscore Livro Negro\textunderscore , 299.
\section{Desconsôlo}
\begin{itemize}
\item {Grp. gram.:m.}
\end{itemize}
O mesmo que \textunderscore desconsolação\textunderscore .
\section{Desconsoloso}
\begin{itemize}
\item {Grp. gram.:adj.}
\end{itemize}
Que não tem consôlo.
Que manifesta desconsôlo.
\section{Desconstrangido}
\begin{itemize}
\item {Grp. gram.:adj.}
\end{itemize}
Não constrangido.
Que não tem constrangimento. Cf. P. Chagas, \textunderscore Côrte de D. João V\textunderscore , 33.
\section{Desconstruir}
\begin{itemize}
\item {Grp. gram.:v. t.}
\end{itemize}
\begin{itemize}
\item {Proveniência:(De \textunderscore des...\textunderscore  + \textunderscore construir\textunderscore )}
\end{itemize}
Desfazer a construcção de.
Destruir. Cf. Filinto, \textunderscore D. Man.\textunderscore , III, 134.
\section{Descontar}
\begin{itemize}
\item {Grp. gram.:v. t.}
\end{itemize}
\begin{itemize}
\item {Utilização:Fig.}
\end{itemize}
\begin{itemize}
\item {Proveniência:(De \textunderscore des...\textunderscore  + \textunderscore contar\textunderscore )}
\end{itemize}
Fazer desconto de.
Abater de uma conta.
Deduzir.
Prescindir de.
Não meter em conta: \textunderscore descontar rapaziadas\textunderscore .
\section{Descontentadiço}
\begin{itemize}
\item {Grp. gram.:adj.}
\end{itemize}
\begin{itemize}
\item {Proveniência:(De \textunderscore descontentar\textunderscore )}
\end{itemize}
Diffícil de contentar.
\section{Descontentamento}
\begin{itemize}
\item {Grp. gram.:m.}
\end{itemize}
Falta de contentamento; desprazer.
\section{Descontentar}
\begin{itemize}
\item {Grp. gram.:v. t.}
\end{itemize}
\begin{itemize}
\item {Proveniência:(De \textunderscore des...\textunderscore  + \textunderscore contentar\textunderscore )}
\end{itemize}
Tornar descontente.
Desgostar; contrariar.
\section{Descontente}
\begin{itemize}
\item {Grp. gram.:adj.}
\end{itemize}
Que não está contente.
Triste; desgostoso.
\section{Descontento}
\begin{itemize}
\item {Grp. gram.:m.}
\end{itemize}
O mesmo que \textunderscore descontentamento\textunderscore . Cf. Garret, \textunderscore Catão\textunderscore .
\section{Descontinência}
\begin{itemize}
\item {Grp. gram.:f.}
\end{itemize}
O mesmo que \textunderscore incontinência\textunderscore . Cf. F. Manuel, \textunderscore Carta de Guia\textunderscore , 76.
\section{Descontinuação}
\begin{itemize}
\item {Grp. gram.:f.}
\end{itemize}
Acto ou effeito de descontinuar.
\section{Descontinuadamente}
\begin{itemize}
\item {Grp. gram.:adv.}
\end{itemize}
Com descontinuação.
\section{Descontinuador}
\begin{itemize}
\item {Grp. gram.:adj.}
\end{itemize}
\begin{itemize}
\item {Grp. gram.:M.}
\end{itemize}
Que descontinua.
Aquelle, que descontinua.
\section{Descontinuar}
\begin{itemize}
\item {Grp. gram.:v. t.}
\end{itemize}
Não continuar.
Interromper.
Suspender.
\section{Descontinuidade}
\begin{itemize}
\item {Grp. gram.:f.}
\end{itemize}
Qualidade daquillo que é descontinuo. Cf. Júl. Dinís, \textunderscore Morgadinha\textunderscore , 256.
\section{Descontínuo}
\begin{itemize}
\item {Grp. gram.:adj.}
\end{itemize}
Não continuo, interrompido.
\section{Desconto}
\begin{itemize}
\item {Grp. gram.:m.}
\end{itemize}
\begin{itemize}
\item {Proveniência:(De \textunderscore des...\textunderscore  + \textunderscore conto\textunderscore )}
\end{itemize}
Acto de descontar.
Deminuição.
Aquillo que se deminue ou se abate de uma conta: \textunderscore os militares têm desconto no preço da passagem\textunderscore .
\section{Descontratar}
\begin{itemize}
\item {Grp. gram.:v. t.}
\end{itemize}
\begin{itemize}
\item {Proveniência:(De \textunderscore des...\textunderscore  + \textunderscore contratar\textunderscore )}
\end{itemize}
Desfazer um contrato sôbre: \textunderscore descontratar um empréstimo\textunderscore .
\section{Desconvencer}
\begin{itemize}
\item {Grp. gram.:v. t.}
\end{itemize}
\begin{itemize}
\item {Proveniência:(De \textunderscore des...\textunderscore  + \textunderscore convencer\textunderscore )}
\end{itemize}
Despersuadir, fazer mudar de convicção.
\section{Desconveniência}
\begin{itemize}
\item {Grp. gram.:f.}
\end{itemize}
Falta de conveniência.
Desconformidade.
\section{Desconveniente}
\begin{itemize}
\item {Grp. gram.:adj.}
\end{itemize}
Que não é conveniente.
\section{Desconversação}
\begin{itemize}
\item {Grp. gram.:f.}
\end{itemize}
Falta de conversação.
\section{Desconversar}
\begin{itemize}
\item {Grp. gram.:v. t.  e  i.}
\end{itemize}
\begin{itemize}
\item {Grp. gram.:V. i.}
\end{itemize}
\begin{itemize}
\item {Utilização:Prov.}
\end{itemize}
Deixar de conversar.
Descambar no gracejo ou na chocarrice. (Colhido em Turquel)
\section{Desconversável}
\begin{itemize}
\item {Grp. gram.:adj.}
\end{itemize}
Que não é conversável.
Intratável. Cf. G. Frutuoso, \textunderscore Saud. da Terra\textunderscore .
\section{Desconversavelmente}
\begin{itemize}
\item {Grp. gram.:adv.}
\end{itemize}
De modo desconversável.
\section{Desconverter}
\begin{itemize}
\item {Grp. gram.:v. t.}
\end{itemize}
\begin{itemize}
\item {Utilização:Des.}
\end{itemize}
\begin{itemize}
\item {Proveniência:(De \textunderscore des...\textunderscore  + \textunderscore converter\textunderscore )}
\end{itemize}
Desfazer a conversão de.
\section{Desconvidar}
\begin{itemize}
\item {Grp. gram.:v. t.}
\end{itemize}
\begin{itemize}
\item {Proveniência:(De \textunderscore des...\textunderscore  + \textunderscore convidar\textunderscore )}
\end{itemize}
Fazer contra-aviso de um convite a.
\section{Desconvir}
\begin{itemize}
\item {Grp. gram.:v. i.}
\end{itemize}
\begin{itemize}
\item {Grp. gram.:V. t.}
\end{itemize}
Não convir.
O mesmo que \textunderscore desavir\textunderscore . Cf. Filinto, \textunderscore D. Man.\textunderscore , III, 205.
\section{Desconvizinho}
\begin{itemize}
\item {Grp. gram.:adj.}
\end{itemize}
Não vizinho; afastado, distante. Cf. Camillo, \textunderscore Coisas Leves\textunderscore , 26.
\section{Descoordenação}
\begin{itemize}
\item {Grp. gram.:f.}
\end{itemize}
Acto de descoordenar.
\section{Descoordenar}
\begin{itemize}
\item {Grp. gram.:v. t.}
\end{itemize}
\begin{itemize}
\item {Proveniência:(De \textunderscore des...\textunderscore  + \textunderscore coordenar\textunderscore )}
\end{itemize}
Tirar a coordenação a.
\section{Descôr}
\begin{itemize}
\item {Grp. gram.:f.}
\end{itemize}
Falta de côr. Cf. Castilho, \textunderscore N. do Castello\textunderscore .
\section{Descoraçoar}
\begin{itemize}
\item {Grp. gram.:v. t.}
\end{itemize}
O mesmo ou melhor que \textunderscore descoroçoar\textunderscore .
\section{Descòrado}
\begin{itemize}
\item {Grp. gram.:adj.}
\end{itemize}
\begin{itemize}
\item {Proveniência:(De \textunderscore des...\textunderscore  + \textunderscore còrado\textunderscore )}
\end{itemize}
Que não tem côr.
Cuja côr está alterada.
Pállido.
\section{Descòramento}
\begin{itemize}
\item {Grp. gram.:m.}
\end{itemize}
Acto ou effeito de descòrar.
\section{Descòrante}
\begin{itemize}
\item {Grp. gram.:adj.}
\end{itemize}
\begin{itemize}
\item {Proveniência:(De \textunderscore descòrar\textunderscore )}
\end{itemize}
Que tira a côr a.
\section{Descòrar}
\begin{itemize}
\item {Grp. gram.:v. t.}
\end{itemize}
\begin{itemize}
\item {Grp. gram.:V. i.}
\end{itemize}
\begin{itemize}
\item {Proveniência:(De \textunderscore des...\textunderscore  + \textunderscore còrar\textunderscore )}
\end{itemize}
Modificar a côr de.
Tirar a côr a.
Empallidecer.
\section{Descorar}
\begin{itemize}
\item {Grp. gram.:v. t.}
\end{itemize}
\begin{itemize}
\item {Proveniência:(De \textunderscore des...\textunderscore  + \textunderscore cór\textunderscore )}
\end{itemize}
Deixar de têr de cór.
Esquecer-se de (aquillo que estava decorado). Cf. Camillo, \textunderscore Mystérios de Lisbôa\textunderscore , II, 168.
\section{Descorchar}
\begin{itemize}
\item {Grp. gram.:v. t.}
\end{itemize}
(V.escorchar)
\section{Descorçoamento}
\begin{itemize}
\item {Grp. gram.:m.}
\end{itemize}
Acto de descorçoar.
\section{Descorçoar}
\begin{itemize}
\item {Grp. gram.:v. t.}
\end{itemize}
(V.descoroçoar)
\section{Descordar}
\begin{itemize}
\item {Grp. gram.:v. i.}
\end{itemize}
\begin{itemize}
\item {Proveniência:(De \textunderscore des...\textunderscore  + \textunderscore corda\textunderscore )}
\end{itemize}
Diz-se do toireiro, que corta com o estoque a medulla espinhal do toiro, fazendo-o caír, sem que se possa levantar mais.
\section{Descorentar}
\begin{itemize}
\item {Grp. gram.:v. i.}
\end{itemize}
\begin{itemize}
\item {Utilização:Neol.}
\end{itemize}
\begin{itemize}
\item {Proveniência:(De \textunderscore des...\textunderscore  + \textunderscore côr\textunderscore )}
\end{itemize}
Tirar a côr a; tornar descorado.
\section{Descornamento}
\begin{itemize}
\item {Grp. gram.:m.}
\end{itemize}
Acto de descornar.
\section{Descornar}
\begin{itemize}
\item {Grp. gram.:v. t.}
\end{itemize}
\begin{itemize}
\item {Proveniência:(De \textunderscore des...\textunderscore  + \textunderscore côrno\textunderscore )}
\end{itemize}
Tirar os cornos a.
\section{Descoroar}
\begin{itemize}
\item {Grp. gram.:v. t.}
\end{itemize}
\begin{itemize}
\item {Proveniência:(De \textunderscore des...\textunderscore  + \textunderscore coroar\textunderscore )}
\end{itemize}
Tirar a corôa ou o remate a.
\section{Descoroçoar}
\begin{itemize}
\item {Grp. gram.:v. t. e i.}
\end{itemize}
\begin{itemize}
\item {Grp. gram.:V. i.}
\end{itemize}
Tirar o ânimo ou a coragem a.
Perder a coragem.
Desanimar.
(Por \textunderscore descoraçoar\textunderscore , de \textunderscore des...\textunderscore  + \textunderscore coração\textunderscore )
\section{Descorolado}
\begin{itemize}
\item {Grp. gram.:adj.}
\end{itemize}
Que não tem corola.
\section{Descorollado}
\begin{itemize}
\item {Grp. gram.:adj.}
\end{itemize}
Que não tem corolla.
\section{Descorregedor}
\begin{itemize}
\item {Grp. gram.:m.}
\end{itemize}
\begin{itemize}
\item {Utilização:Ant.}
\end{itemize}
\begin{itemize}
\item {Proveniência:(De \textunderscore descorreger\textunderscore )}
\end{itemize}
Aquelle, que não corrige.
\section{Descorreger}
\begin{itemize}
\item {Grp. gram.:v. t.}
\end{itemize}
\begin{itemize}
\item {Utilização:Ant.}
\end{itemize}
\begin{itemize}
\item {Proveniência:(De \textunderscore des...\textunderscore  + \textunderscore correger\textunderscore )}
\end{itemize}
Não corrigir.
\section{Descorrelacionar}
\begin{itemize}
\item {Grp. gram.:v. t.}
\end{itemize}
Tirar a correlação a.
\section{Descorrentar}
\textunderscore v. t.\textunderscore  (e der.)
O mesmo que \textunderscore desacorrentar\textunderscore . Cf. Castilho, \textunderscore D. Quixote\textunderscore , I, 153.
\section{Descortejar}
\begin{itemize}
\item {Grp. gram.:v. t.}
\end{itemize}
\begin{itemize}
\item {Proveniência:(De \textunderscore des...\textunderscore  + \textunderscore cortejar\textunderscore )}
\end{itemize}
Desconsiderar.
Não cumprimentar, não cortejar.
\section{Descortês}
\begin{itemize}
\item {Grp. gram.:adj.}
\end{itemize}
Que não é cortês.
Indelicado; grosseiro.
\section{Descortesia}
\begin{itemize}
\item {Grp. gram.:f.}
\end{itemize}
Falta de cortesia.
Acção descortês.
\section{Descortesmente}
\begin{itemize}
\item {fónica:tês}
\end{itemize}
\begin{itemize}
\item {Grp. gram.:adv.}
\end{itemize}
De modo descortês.
Grosseiramente.
\section{Descorticação}
\begin{itemize}
\item {Grp. gram.:f.}
\end{itemize}
\begin{itemize}
\item {Proveniência:(Do lat. \textunderscore decorticatio\textunderscore )}
\end{itemize}
Operação cirúrgica de uma membrana da túnica vaginal.
\section{Descortiçamento}
\begin{itemize}
\item {Grp. gram.:m.}
\end{itemize}
Acto de descortiçar.
\section{Descorticar}
\begin{itemize}
\item {Grp. gram.:v. t.}
\end{itemize}
\begin{itemize}
\item {Proveniência:(Do lat. \textunderscore decorticare\textunderscore )}
\end{itemize}
Triturar a casca de, para a tirar.
\section{Descortiçar}
\begin{itemize}
\item {Grp. gram.:v. t.}
\end{itemize}
Tirar a cortiça a: \textunderscore descortiçar sobreiros\textunderscore .
\section{Descortinar}
\begin{itemize}
\item {Grp. gram.:v. t.}
\end{itemize}
\begin{itemize}
\item {Utilização:Fig.}
\end{itemize}
\begin{itemize}
\item {Proveniência:(De \textunderscore des...\textunderscore  + \textunderscore cortina\textunderscore )}
\end{itemize}
Correr a cortina para deixar vêr.
Tirar a cortina a.
Avistar; descobrir ao longe: \textunderscore os mareantes descortinaram terra\textunderscore .
\section{Descortinável}
\begin{itemize}
\item {Grp. gram.:adj.}
\end{itemize}
Que se póde descortinar.
\section{Descortino}
\begin{itemize}
\item {Grp. gram.:m.}
\end{itemize}
\begin{itemize}
\item {Utilização:bras}
\end{itemize}
\begin{itemize}
\item {Utilização:Neol.}
\end{itemize}
Acto de descortinar ou avistar.
\section{Descosedura}
\begin{itemize}
\item {Grp. gram.:f.}
\end{itemize}
Acto ou effeito de descoser.
\section{Descoser}
\begin{itemize}
\item {Grp. gram.:v. t.}
\end{itemize}
\begin{itemize}
\item {Utilização:Fig.}
\end{itemize}
\begin{itemize}
\item {Utilização:Fam.}
\end{itemize}
\begin{itemize}
\item {Grp. gram.:V. p.}
\end{itemize}
\begin{itemize}
\item {Utilização:Fam.}
\end{itemize}
\begin{itemize}
\item {Proveniência:(De \textunderscore des...\textunderscore  + \textunderscore coser\textunderscore )}
\end{itemize}
Desmanchar a costura de: \textunderscore descoser um casaco\textunderscore .
Desunir.
Rasgar.
Divulgar.
Dizer tudo que se sente.
Desabafar.
\section{Descostar}
\textunderscore v. t.\textunderscore  (e der.)
O mesmo que \textunderscore desencostar\textunderscore , etc.
\section{Descostumar}
\begin{itemize}
\item {Grp. gram.:v. t.}
\end{itemize}
(V.desacostumar)
\section{Descostume}
\begin{itemize}
\item {Grp. gram.:m.}
\end{itemize}
Falta de costume.
\section{Descotoar}
\begin{itemize}
\item {Grp. gram.:v. t.}
\end{itemize}
Tirar o cotão a.
\section{Descoutar}
\begin{itemize}
\item {Grp. gram.:v. t.}
\end{itemize}
\begin{itemize}
\item {Proveniência:(De \textunderscore des...\textunderscore  + \textunderscore coutar\textunderscore )}
\end{itemize}
Tirar os privilégios de couto a (uma propriedade). Cf. Herculano, \textunderscore M. de Cister\textunderscore , II, 64.
\section{Descraseado}
\begin{itemize}
\item {Grp. gram.:adj.}
\end{itemize}
\begin{itemize}
\item {Proveniência:(De \textunderscore des...\textunderscore  + \textunderscore crase\textunderscore )}
\end{itemize}
Diz-se do sangue, que não tem crase, isto é, que tem desequilibrados os elementos que o constituem.
Dessorado.--Camillo escreveu mal a palavra:«\textunderscore ...derivaram o seu sangue descraziado de Guadalete\textunderscore ». \textunderscore Maria da Fonte\textunderscore , 232.
\section{Descravar}
\begin{itemize}
\item {Grp. gram.:v. t.}
\end{itemize}
O mesmo que \textunderscore desencravar\textunderscore .
O mesmo que \textunderscore desfitar\textunderscore . Cf. Filinto, \textunderscore D. Man.\textunderscore , III, 146.
Arrancar (ferro). Cf. Castilho, \textunderscore Metam.\textunderscore , 35.
\section{Descravejar}
\begin{itemize}
\item {Grp. gram.:v. t.}
\end{itemize}
\begin{itemize}
\item {Proveniência:(De \textunderscore des...\textunderscore  + \textunderscore cravejar\textunderscore )}
\end{itemize}
Tirar os cravos a: \textunderscore descravejar a mula\textunderscore .
Desmanchar (aquillo que estava cravejado): \textunderscore descravejar uma jóia\textunderscore .
\section{Descravizar}
\begin{itemize}
\item {Grp. gram.:v. t.}
\end{itemize}
\begin{itemize}
\item {Proveniência:(De \textunderscore de...\textunderscore  + \textunderscore escravizar\textunderscore )}
\end{itemize}
Livrar da escravidão. Cf. Camillo, \textunderscore Caveira\textunderscore , 230.
\section{Descreditar}
\begin{itemize}
\item {Grp. gram.:v. t.}
\end{itemize}
(V.desacreditar)
\section{Descrédito}
\begin{itemize}
\item {Grp. gram.:m.}
\end{itemize}
Falta ou perda de crédito; deshonra.
\section{Descrença}
\begin{itemize}
\item {Grp. gram.:f.}
\end{itemize}
Falta ou perda de crença; incredulidade.
\section{Descrente}
\begin{itemize}
\item {Grp. gram.:adj.}
\end{itemize}
\begin{itemize}
\item {Grp. gram.:M.}
\end{itemize}
\begin{itemize}
\item {Proveniência:(De \textunderscore des...\textunderscore  + \textunderscore crente\textunderscore )}
\end{itemize}
Que não crê; que perdeu a crença.
Incrédulo.
Pessôa descrente.
\section{Descrer}
\begin{itemize}
\item {Grp. gram.:v. t.}
\end{itemize}
\begin{itemize}
\item {Grp. gram.:V. i.}
\end{itemize}
Não crer.
Não têr fé.
Não dar crédito: \textunderscore descrer de uma promessa\textunderscore .
\section{Descreúdo}
\begin{itemize}
\item {Grp. gram.:adj.}
\end{itemize}
\begin{itemize}
\item {Utilização:Ant.}
\end{itemize}
Descrido.
\section{Descrever}
\begin{itemize}
\item {Grp. gram.:v. t.}
\end{itemize}
\begin{itemize}
\item {Proveniência:(Lat. \textunderscore describere\textunderscore )}
\end{itemize}
Narrar circunstanciadamente: \textunderscore descrever uma batalha\textunderscore .
Representar, falando ou escrevendo; traçar: \textunderscore descrever uma personagem\textunderscore .
Percorrer.
\section{Descriado}
\begin{itemize}
\item {Grp. gram.:adj.}
\end{itemize}
\begin{itemize}
\item {Utilização:Fam.}
\end{itemize}
\begin{itemize}
\item {Proveniência:(De \textunderscore des...\textunderscore  + \textunderscore criado\textunderscore )}
\end{itemize}
Que já não é criança.
\section{Descrição}
\begin{itemize}
\item {Grp. gram.:f.}
\end{itemize}
\begin{itemize}
\item {Proveniência:(Lat. \textunderscore descriptio\textunderscore )}
\end{itemize}
Acto ou efeito de descrever.
\section{Descrido}
\begin{itemize}
\item {Grp. gram.:m.  e  adj.}
\end{itemize}
O mesmo que \textunderscore descrente\textunderscore .
\section{Descriminar}
\begin{itemize}
\item {Grp. gram.:v. t.}
\end{itemize}
\begin{itemize}
\item {Proveniência:(De \textunderscore des...\textunderscore  + \textunderscore criminar\textunderscore )}
\end{itemize}
Tirar a culpa a.
Absolver de crime.
\section{Descripção}
\begin{itemize}
\item {Grp. gram.:f.}
\end{itemize}
\begin{itemize}
\item {Proveniência:(Lat. \textunderscore descriptio\textunderscore )}
\end{itemize}
Acto ou effeito de descrever.
\section{Descriptível}
\begin{itemize}
\item {Grp. gram.:adj.}
\end{itemize}
Que se póde descrever.
\section{Descriptivo}
\begin{itemize}
\item {Grp. gram.:adj.}
\end{itemize}
\begin{itemize}
\item {Proveniência:(Lat. \textunderscore descriptivus\textunderscore )}
\end{itemize}
Em que há descripção.
Próprio para descrever; relativo a descripções: \textunderscore gênero descriptivo\textunderscore .
\section{Descriptor}
\begin{itemize}
\item {Grp. gram.:adj.}
\end{itemize}
\begin{itemize}
\item {Utilização:Des.}
\end{itemize}
\begin{itemize}
\item {Grp. gram.:M.}
\end{itemize}
\begin{itemize}
\item {Proveniência:(Lat. \textunderscore descriptor\textunderscore )}
\end{itemize}
Que descreve.
Aquelle que descreve.
\section{Descristianização}
\begin{itemize}
\item {Grp. gram.:f.}
\end{itemize}
Acto ou efeito de descristianizar.
\section{Descristianizar}
\begin{itemize}
\item {Grp. gram.:v. t.}
\end{itemize}
\begin{itemize}
\item {Utilização:Neol.}
\end{itemize}
\begin{itemize}
\item {Proveniência:(De \textunderscore des\textunderscore  + \textunderscore cristianizar\textunderscore )}
\end{itemize}
Fazer perder a qualidade de cristão a.
Tirar as crenças cristans a.
\section{Descritível}
\begin{itemize}
\item {Grp. gram.:adj.}
\end{itemize}
Que se póde descrever.
\section{Descritivo}
\begin{itemize}
\item {Grp. gram.:adj.}
\end{itemize}
\begin{itemize}
\item {Proveniência:(Lat. \textunderscore descriptivus\textunderscore )}
\end{itemize}
Em que há descrição.
Próprio para descrever; relativo a descrições: \textunderscore gênero descritivo\textunderscore .
\section{Descritor}
\begin{itemize}
\item {Grp. gram.:adj.}
\end{itemize}
\begin{itemize}
\item {Utilização:Des.}
\end{itemize}
\begin{itemize}
\item {Grp. gram.:M.}
\end{itemize}
\begin{itemize}
\item {Proveniência:(Lat. \textunderscore descriptor\textunderscore )}
\end{itemize}
Que descreve.
Aquele que descreve.
\section{Descruzar}
\begin{itemize}
\item {Grp. gram.:v. t.}
\end{itemize}
\begin{itemize}
\item {Proveniência:(De \textunderscore des...\textunderscore  + \textunderscore cruzar\textunderscore )}
\end{itemize}
Separar ou desarranjar (aquillo que estava cruzado): \textunderscore descruzar os braços\textunderscore .
\section{Descubiçoso}
\begin{itemize}
\item {Grp. gram.:adj.}
\end{itemize}
\begin{itemize}
\item {Proveniência:(De \textunderscore des...\textunderscore  + \textunderscore cubiçoso\textunderscore )}
\end{itemize}
Que não tem cubiça.
\section{Descudar}
\begin{itemize}
\item {Grp. gram.:v. t.}
\end{itemize}
\begin{itemize}
\item {Utilização:Prov.}
\end{itemize}
\begin{itemize}
\item {Utilização:beir.}
\end{itemize}
\begin{itemize}
\item {Utilização:Ant.}
\end{itemize}
O mesmo que \textunderscore descuidar\textunderscore . Cf. \textunderscore Viriato Trág.\textunderscore , II, 106.
\section{Descudo}
\begin{itemize}
\item {Grp. gram.:m.}
\end{itemize}
\begin{itemize}
\item {Utilização:Prov.}
\end{itemize}
\begin{itemize}
\item {Utilização:beir.}
\end{itemize}
\begin{itemize}
\item {Utilização:Ant.}
\end{itemize}
O mesmo que \textunderscore descuido\textunderscore .
(Cp. \textunderscore descudar\textunderscore )
\section{Descuidadamente}
\begin{itemize}
\item {Grp. gram.:adv.}
\end{itemize}
Com descuido.
\section{Descuidado}
\begin{itemize}
\item {Grp. gram.:adj.}
\end{itemize}
\begin{itemize}
\item {Proveniência:(De \textunderscore descuidar\textunderscore )}
\end{itemize}
Que não tem cuidado.
Desleixado, indolente.
Que manifesta descuido ou desleixo.
Precipitado, irreflectido: \textunderscore acto descuidado\textunderscore .
Tranquillo, sereno.
\section{Descuidadoso}
\begin{itemize}
\item {Grp. gram.:adj.}
\end{itemize}
\begin{itemize}
\item {Utilização:Des.}
\end{itemize}
\begin{itemize}
\item {Proveniência:(De \textunderscore des...\textunderscore  + \textunderscore cuidadoso\textunderscore )}
\end{itemize}
Descuidado.
\section{Descuidar}
\begin{itemize}
\item {Grp. gram.:v. t.}
\end{itemize}
\begin{itemize}
\item {Grp. gram.:V. p.}
\end{itemize}
\begin{itemize}
\item {Proveniência:(De \textunderscore des...\textunderscore  + \textunderscore cuidar\textunderscore )}
\end{itemize}
Não têr cuidado em.
Não fazer caso de.
Desprezar: \textunderscore descuidar o govêrno da casa\textunderscore .
Esquecer-se.
Não têr cuidado: \textunderscore descuidar-se dos deveres\textunderscore .
\section{Descuido}
\begin{itemize}
\item {Grp. gram.:m.}
\end{itemize}
\begin{itemize}
\item {Utilização:Fam.}
\end{itemize}
\begin{itemize}
\item {Proveniência:(De \textunderscore descuidar\textunderscore )}
\end{itemize}
Falta de cuidado.
Falta.
Êrro; inadvertência: \textunderscore enganar-se por descuido.\textunderscore 
Ventosidade.
\section{Descuidosamente}
\begin{itemize}
\item {Grp. gram.:adv.}
\end{itemize}
De modo descuidoso.
\section{Descuidoso}
\begin{itemize}
\item {Grp. gram.:adj.}
\end{itemize}
(V.descuidado)
\section{Desculpa}
\begin{itemize}
\item {Grp. gram.:f.}
\end{itemize}
Acto de desculpar.
Ausência de culpa.
Indulgência; perdão: \textunderscore pedir desculpa\textunderscore .
\section{Desculpação}
\begin{itemize}
\item {Grp. gram.:f.}
\end{itemize}
\begin{itemize}
\item {Utilização:Ant.}
\end{itemize}
\begin{itemize}
\item {Proveniência:(De \textunderscore desculpar\textunderscore )}
\end{itemize}
O mesmo que \textunderscore desculpa\textunderscore .
\section{Desculpador}
\begin{itemize}
\item {Grp. gram.:adj.}
\end{itemize}
\begin{itemize}
\item {Grp. gram.:M.}
\end{itemize}
Que desculpa.
Aquelle que desculpa.
\section{Desculpar}
\begin{itemize}
\item {Grp. gram.:v. t.}
\end{itemize}
\begin{itemize}
\item {Grp. gram.:V. p.}
\end{itemize}
\begin{itemize}
\item {Proveniência:(De \textunderscore des...\textunderscore  + \textunderscore culpar\textunderscore )}
\end{itemize}
Attenuar ou destruir a culpa de.
Justificar.
Perdoar a culpa de.
Dispensar.
Expor a razão que allivia ou destrói a própria culpa.
Pretextar.
Pedir escusa: \textunderscore desculpar-se de não comparecer\textunderscore .
\section{Desculpável}
\begin{itemize}
\item {Grp. gram.:adj.}
\end{itemize}
\begin{itemize}
\item {Proveniência:(De \textunderscore des...\textunderscore  + \textunderscore culpável\textunderscore )}
\end{itemize}
Susceptível de desculpa; que se póde ou se deve desculpar.
\section{Desculpavelmente}
\begin{itemize}
\item {Grp. gram.:adv.}
\end{itemize}
De modo desculpável.
\section{Descultivar}
\begin{itemize}
\item {Grp. gram.:v. t.}
\end{itemize}
Não cultivar.
Deixar de cultivar.
Conservar inculto. Cf. Camillo, \textunderscore Pombal\textunderscore , 102.
\section{Desculto}
\begin{itemize}
\item {Grp. gram.:m.}
\end{itemize}
Ausência de culto; falta de amor.
Irreverência. Cf. Camillo, \textunderscore Amor de Salv.\textunderscore , 73.
\section{Descumprir}
\begin{itemize}
\item {Grp. gram.:v. t.}
\end{itemize}
Não cumprir. Cf. Castilho, \textunderscore Misanthropo\textunderscore , 115.
\section{Descuradamente}
\begin{itemize}
\item {Grp. gram.:adv.}
\end{itemize}
\begin{itemize}
\item {Proveniência:(De \textunderscore descurar\textunderscore )}
\end{itemize}
Desleixadamente.
Com descuramento.
\section{Descuramento}
\begin{itemize}
\item {Grp. gram.:m.}
\end{itemize}
Acto de descurar.
\section{Descurar}
\begin{itemize}
\item {Grp. gram.:v. t.}
\end{itemize}
\begin{itemize}
\item {Grp. gram.:V. i.}
\end{itemize}
Não curar de.
Desprezar; abandonar.
Não tratar, não cuidar.
\section{Descuriosamente}
\begin{itemize}
\item {Grp. gram.:adv.}
\end{itemize}
\begin{itemize}
\item {Utilização:Des.}
\end{itemize}
\begin{itemize}
\item {Proveniência:(De \textunderscore des...\textunderscore  + \textunderscore curiosamente\textunderscore )}
\end{itemize}
Sem curiosidade.
\section{Descuriosidade}
\begin{itemize}
\item {Grp. gram.:f.}
\end{itemize}
Falta de curiosidade.
\section{Descurioso}
\begin{itemize}
\item {Grp. gram.:adj.}
\end{itemize}
Que não é curioso.
\section{Descurvar}
\begin{itemize}
\item {Grp. gram.:v. t.}
\end{itemize}
\begin{itemize}
\item {Utilização:Des.}
\end{itemize}
\begin{itemize}
\item {Proveniência:(De \textunderscore des\textunderscore .. + \textunderscore curvar\textunderscore )}
\end{itemize}
Endireitar.
Desencurvar.
\section{Desdar}
\begin{itemize}
\item {Grp. gram.:v. t.}
\end{itemize}
\begin{itemize}
\item {Proveniência:(De \textunderscore des...\textunderscore  + \textunderscore dar\textunderscore )}
\end{itemize}
Desatar (um nó).
Retomar (aquillo que se deu).
\section{Desde}
\begin{itemize}
\item {Grp. gram.:prep.}
\end{itemize}
\begin{itemize}
\item {Grp. gram.:Loc. conj.}
\end{itemize}
\begin{itemize}
\item {Proveniência:(De \textunderscore dês\textunderscore  + \textunderscore de\textunderscore )}
\end{itemize}
A começar de: \textunderscore desde a implantação da República\textunderscore .
\textunderscore Desde que\textunderscore , visto que; depois que.
\section{Desdeixado}
\begin{itemize}
\item {Grp. gram.:adj.}
\end{itemize}
O mesmo que \textunderscore desleixado\textunderscore .
(Cp. \textunderscore deixar\textunderscore , e sua etym.)
\section{Desdém}
\begin{itemize}
\item {Grp. gram.:m.}
\end{itemize}
Desprêzo.
Altivez.
Desaffectação.
Acto do desdenhar.
(Cast. \textunderscore desdén\textunderscore )
\section{Desdenhador}
\begin{itemize}
\item {Grp. gram.:adj.}
\end{itemize}
\begin{itemize}
\item {Grp. gram.:M.}
\end{itemize}
Que desdenha.
Aquelle que desdenha.
\section{Desdenhar}
\begin{itemize}
\item {Grp. gram.:v. t.}
\end{itemize}
\begin{itemize}
\item {Proveniência:(Do lat. \textunderscore dedignari\textunderscore ?)}
\end{itemize}
Mostrar ou têr desdém por.
Desprezar com sobrançaria: \textunderscore desdenhar ameaças\textunderscore .
Motejar.
\section{Desdenhativo}
\begin{itemize}
\item {Grp. gram.:adj.}
\end{itemize}
\begin{itemize}
\item {Proveniência:(De \textunderscore desdenhar\textunderscore )}
\end{itemize}
Que envolve desdém; depreciativo: \textunderscore expressões desdenhativas\textunderscore .
\section{Desdenhável}
\begin{itemize}
\item {Grp. gram.:adj.}
\end{itemize}
\begin{itemize}
\item {Proveniência:(De \textunderscore desdenhar\textunderscore )}
\end{itemize}
Que merece desdém.
\section{Desdenhosamente}
\begin{itemize}
\item {Grp. gram.:adv.}
\end{itemize}
De modo desdenhoso.
\section{Desdenhoso}
\begin{itemize}
\item {Grp. gram.:adj.}
\end{itemize}
\begin{itemize}
\item {Proveniência:(De \textunderscore desdenhar\textunderscore )}
\end{itemize}
Em que há desdém; que tem desdém.
\section{Desdentados}
\begin{itemize}
\item {Grp. gram.:m. pl.}
\end{itemize}
\begin{itemize}
\item {Utilização:Zool.}
\end{itemize}
Ordem de animaes, a que faltam os dentes da frente.
\section{Desdentar}
\begin{itemize}
\item {Grp. gram.:v. t.}
\end{itemize}
\begin{itemize}
\item {Grp. gram.:V. p.}
\end{itemize}
Tirar os dentes a.
Perder os dentes.
\section{Desdita}
\begin{itemize}
\item {Grp. gram.:f.}
\end{itemize}
Falta de dita.
Desgraça, infelicidade.
\section{Desditado}
\begin{itemize}
\item {Grp. gram.:adj.}
\end{itemize}
O mesmo que \textunderscore desditoso\textunderscore . Cf. Castilho, \textunderscore Fastos\textunderscore , II, 477.
\section{Desdito}
\begin{itemize}
\item {Grp. gram.:adj.}
\end{itemize}
\begin{itemize}
\item {Utilização:Ant.}
\end{itemize}
\begin{itemize}
\item {Proveniência:(De \textunderscore desdita\textunderscore )}
\end{itemize}
O mesmo que \textunderscore desditoso\textunderscore .
\section{Desditosamente}
\begin{itemize}
\item {Grp. gram.:adv.}
\end{itemize}
De modo desditoso.
\section{Desditoso}
\begin{itemize}
\item {Grp. gram.:adj.}
\end{itemize}
\begin{itemize}
\item {Proveniência:(De \textunderscore des...\textunderscore  + \textunderscore ditoso\textunderscore )}
\end{itemize}
Em que há desdita.
Desventurado.
\section{Desdizer}
\begin{itemize}
\item {Grp. gram.:v. t.}
\end{itemize}
\begin{itemize}
\item {Grp. gram.:V. i.}
\end{itemize}
\begin{itemize}
\item {Proveniência:(De \textunderscore des...\textunderscore  + \textunderscore dizer\textunderscore )}
\end{itemize}
Contradizer.
Desmentir.
Impugnar.
Negar.
Contradizer-se: \textunderscore diz e desdiz\textunderscore .
Discordar.
Desconvir.
\section{Desdobramento}
\begin{itemize}
\item {Grp. gram.:m.}
\end{itemize}
Acto de desdobrar.
\section{Desdobrar}
\begin{itemize}
\item {Grp. gram.:v. t.}
\end{itemize}
\begin{itemize}
\item {Proveniência:(De \textunderscore des...\textunderscore  + \textunderscore dobrar\textunderscore )}
\end{itemize}
Abrir ou estender (aquillo que estava dobrado): \textunderscore desdobrar uma toalha\textunderscore .
Desenvolver: \textunderscore desdobrar um assumpto\textunderscore .
\section{Desdoiramento}
\begin{itemize}
\item {Grp. gram.:m.}
\end{itemize}
Acto de desdoirar.
\section{Desdoirar}
\begin{itemize}
\item {Grp. gram.:v. t.}
\end{itemize}
\begin{itemize}
\item {Utilização:Fig.}
\end{itemize}
\begin{itemize}
\item {Proveniência:(De \textunderscore des...\textunderscore  + \textunderscore doirar\textunderscore )}
\end{itemize}
Tirar a doiradura a.
Causar desdoiro a.
Deslustrar.
Desacreditar.
\section{Desdoiro}
\begin{itemize}
\item {Grp. gram.:m.}
\end{itemize}
\begin{itemize}
\item {Utilização:Fig.}
\end{itemize}
Acto ou effeito de desdoirar.
Deslustre, mácula.
Vergonha.
Descrédito.
\section{Desdouramento}
\begin{itemize}
\item {Grp. gram.:m.}
\end{itemize}
Acto de desdourar.
\section{Desdourar}
\begin{itemize}
\item {Grp. gram.:v. t.}
\end{itemize}
\begin{itemize}
\item {Utilização:Fig.}
\end{itemize}
\begin{itemize}
\item {Proveniência:(De \textunderscore des...\textunderscore  + \textunderscore dourar\textunderscore )}
\end{itemize}
Tirar a douradura a.
Causar desdouro a.
Deslustrar.
Desacreditar.
\section{Desdouro}
\begin{itemize}
\item {Grp. gram.:m.}
\end{itemize}
\begin{itemize}
\item {Utilização:Fig.}
\end{itemize}
Acto ou effeito de desdourar.
Deslustre, mácula.
Vergonha.
Descrédito.
\section{Desdoutrinação}
\begin{itemize}
\item {Grp. gram.:f.}
\end{itemize}
Acto de desdoutrinar.
\section{Desdoutrinar}
\begin{itemize}
\item {Grp. gram.:v. t.}
\end{itemize}
\begin{itemize}
\item {Proveniência:(De \textunderscore des...\textunderscore  + \textunderscore doutrinar\textunderscore )}
\end{itemize}
Tornar ignorante.
Fazer esquecer da doutrina. Cf. Filinto, IX, 220.
\section{Desecação}
\begin{itemize}
\item {fónica:se}
\end{itemize}
\begin{itemize}
\item {Grp. gram.:f.}
\end{itemize}
Acto ou effeito de desecar.
\section{Desecamento}
\begin{itemize}
\item {fónica:se}
\end{itemize}
\begin{itemize}
\item {Grp. gram.:m.}
\end{itemize}
O mesmo que \textunderscore desecação\textunderscore .
\section{Desecante}
\begin{itemize}
\item {fónica:se}
\end{itemize}
\begin{itemize}
\item {Grp. gram.:adj.}
\end{itemize}
\begin{itemize}
\item {Proveniência:(Do lat. \textunderscore desiccans\textunderscore )}
\end{itemize}
Que deseca.
\section{Desecar}
\begin{itemize}
\item {fónica:se}
\end{itemize}
\begin{itemize}
\item {Grp. gram.:v. t.}
\end{itemize}
\begin{itemize}
\item {Utilização:Fig.}
\end{itemize}
\begin{itemize}
\item {Proveniência:(Lat. \textunderscore desiccare\textunderscore , se não é gallicismo. Cp. fr. \textunderscore dessécher\textunderscore )}
\end{itemize}
Enxugar; tornar sêco, árido.
Fazer cicatrizar.
Mirrar.
Tornar insensivel.
\section{Desecativo}
\begin{itemize}
\item {fónica:se}
\end{itemize}
\begin{itemize}
\item {Grp. gram.:adj.}
\end{itemize}
\begin{itemize}
\item {Proveniência:(Lat. desiccativus)}
\end{itemize}
Que faz desecar.
\section{Deseclipsar}
\begin{itemize}
\item {Grp. gram.:v. t.}
\end{itemize}
\begin{itemize}
\item {Grp. gram.:V. i.}
\end{itemize}
\begin{itemize}
\item {Proveniência:(De \textunderscore des...\textunderscore  + \textunderscore eclipsar\textunderscore )}
\end{itemize}
Desvendar, descobrir.
Tornar a apparecer, tendo-se eclipsado.
\section{Desedificação}
\begin{itemize}
\item {Grp. gram.:f.}
\end{itemize}
Acto ou effeito de desedificar.
\section{Desedificador}
\begin{itemize}
\item {Grp. gram.:adj.}
\end{itemize}
\begin{itemize}
\item {Grp. gram.:M.}
\end{itemize}
Que desedifica.
Aquelle que desedifica.
\section{Desedificar}
\begin{itemize}
\item {Grp. gram.:v. t.}
\end{itemize}
\begin{itemize}
\item {Proveniência:(De \textunderscore des...\textunderscore  + \textunderscore edificar\textunderscore )}
\end{itemize}
Dar maus exemplos a.
Desviar da crença religiosa ou da moral.
\section{Desedificativo}
\begin{itemize}
\item {Grp. gram.:adj.}
\end{itemize}
\begin{itemize}
\item {Proveniência:(De \textunderscore desedificar\textunderscore )}
\end{itemize}
Que desedifica.
\section{Deseixar}
\begin{itemize}
\item {Grp. gram.:v. t.}
\end{itemize}
\begin{itemize}
\item {Proveniência:(De \textunderscore des...\textunderscore  + \textunderscore eixo\textunderscore )}
\end{itemize}
Tirar do eixo. Cf. Filinto, VII, 166.
\section{Desejador}
\begin{itemize}
\item {Grp. gram.:adj.}
\end{itemize}
\begin{itemize}
\item {Grp. gram.:M.}
\end{itemize}
Que deseja.
Aquelle que deseja.
\section{Desejar}
\begin{itemize}
\item {Grp. gram.:v. t.}
\end{itemize}
\begin{itemize}
\item {Proveniência:(Do b. lat. \textunderscore disediare\textunderscore )}
\end{itemize}
Têr appetite de.
Querer.
Ambicionar: \textunderscore desejar riquezas\textunderscore .
Fazer empenho em.
Cubiçar.
Querer a posse de: \textunderscore desejar uma casa\textunderscore .
\textunderscore Deixar a desejar\textunderscore , não satisfazer a espectativa ou o que é legítimo esperar ou exigir.
\section{Desejável}
\begin{itemize}
\item {Grp. gram.:adj.}
\end{itemize}
Que é digno de se desejar.
\section{Desejo}
\begin{itemize}
\item {Grp. gram.:m.}
\end{itemize}
\begin{itemize}
\item {Proveniência:(Do b. lat. \textunderscore disedium\textunderscore )}
\end{itemize}
Acto de desejar.
Cubiça.
Appetite.
\section{Desejosamente}
\begin{itemize}
\item {Grp. gram.:adv.}
\end{itemize}
\begin{itemize}
\item {Proveniência:(De \textunderscore desejoso\textunderscore )}
\end{itemize}
Com desejo.
\section{Desejoso}
\begin{itemize}
\item {Grp. gram.:adj.}
\end{itemize}
Que tem desejo: \textunderscore desejoso de gulodices\textunderscore .
\section{Deselegância}
\begin{itemize}
\item {Grp. gram.:f.}
\end{itemize}
Falta de elegância.
\section{Deselegante}
\begin{itemize}
\item {Grp. gram.:adj.}
\end{itemize}
Que não é elegante.
\section{Deseliminar}
\begin{itemize}
\item {Grp. gram.:v. t.}
\end{itemize}
\begin{itemize}
\item {Proveniência:(De \textunderscore des...\textunderscore  + \textunderscore eliminar\textunderscore )}
\end{itemize}
Restabelecer, restituir ao antigo estado.
Rehabilitar:«\textunderscore instava quisesse deseliminar e fazer mercê a parentes.\textunderscore »Filinto, \textunderscore D. Man.\textunderscore , I, 30.
\section{Desembaçar}
\begin{itemize}
\item {Grp. gram.:v. t.}
\end{itemize}
\begin{itemize}
\item {Utilização:pop.}
\end{itemize}
\begin{itemize}
\item {Utilização:Fig.}
\end{itemize}
\begin{itemize}
\item {Proveniência:(De \textunderscore des...\textunderscore  + \textunderscore embaçar\textunderscore )}
\end{itemize}
Tirar a côr baça, a pallidez, a.
Reanimar, fazer voltar a si (quem estava embaçado).
\section{Desembaciar}
\begin{itemize}
\item {Grp. gram.:v. t.}
\end{itemize}
\begin{itemize}
\item {Proveniência:(De \textunderscore des...\textunderscore  + \textunderscore embaciar\textunderscore )}
\end{itemize}
Desempanar.
Limpar (aquillo que estava embaciado).
\section{Desembahular}
\begin{itemize}
\item {fónica:ba-u}
\end{itemize}
\begin{itemize}
\item {Grp. gram.:v. t.}
\end{itemize}
\begin{itemize}
\item {Proveniência:(De \textunderscore des...\textunderscore  + \textunderscore embahular\textunderscore )}
\end{itemize}
Tirar do bahu.
Despejar.
\section{Desembainhar}
\begin{itemize}
\item {fónica:ba-i}
\end{itemize}
\begin{itemize}
\item {Grp. gram.:v. t.}
\end{itemize}
\begin{itemize}
\item {Proveniência:(De \textunderscore des...\textunderscore  + \textunderscore embainhar\textunderscore )}
\end{itemize}
Fazer saír da bainha: \textunderscore desembainhar a espada\textunderscore .
Desmanchar a bainha de (uma costura).
\section{Desembalçar}
\begin{itemize}
\item {Grp. gram.:v. t.}
\end{itemize}
\begin{itemize}
\item {Proveniência:(De \textunderscore des...\textunderscore  + \textunderscore embalçar\textunderscore )}
\end{itemize}
Tirar da balça (o vinho).
\section{Desembandeirar}
\begin{itemize}
\item {Grp. gram.:v. t.}
\end{itemize}
\begin{itemize}
\item {Proveniência:(De \textunderscore des...\textunderscore  + \textunderscore embandeirar\textunderscore )}
\end{itemize}
Tirar a bandeira ou bandeiras a.
\section{Desembaraçadamente}
\begin{itemize}
\item {Grp. gram.:adv.}
\end{itemize}
Com desembaraço.
\section{Desembaraçador}
\begin{itemize}
\item {Grp. gram.:adj.}
\end{itemize}
\begin{itemize}
\item {Grp. gram.:M.}
\end{itemize}
Que desembaraça.
Aquelle que desembaraça.
\section{Desembaraçar}
\begin{itemize}
\item {Grp. gram.:v. t.}
\end{itemize}
\begin{itemize}
\item {Proveniência:(De \textunderscore des...\textunderscore  + \textunderscore embaraçar\textunderscore )}
\end{itemize}
Desimpedir.
Remover o embaraço de: \textunderscore desembaraçar uma pretensão\textunderscore .
Desenredar: \textunderscore desembaraçar meadas\textunderscore .
\section{Desembaraço}
\begin{itemize}
\item {Grp. gram.:m.}
\end{itemize}
\begin{itemize}
\item {Proveniência:(De \textunderscore des...\textunderscore  + \textunderscore embaraço\textunderscore )}
\end{itemize}
Falta de embaraço.
Agilidade; coragem.
Acto de desembaraçar.
\section{Desembaralhar}
\begin{itemize}
\item {Grp. gram.:v. t.}
\end{itemize}
\begin{itemize}
\item {Proveniência:(De \textunderscore des...\textunderscore  + \textunderscore embaralhar\textunderscore )}
\end{itemize}
Pôr em ordem (aquillo que estava embaralhado).
Desembaraçar.
\section{Desembarcação}
\begin{itemize}
\item {Grp. gram.:f.}
\end{itemize}
(V.desembarque)
Lugar de desembarque; desembarcadoiro:«\textunderscore carece de desembarcação\textunderscore ». \textunderscore Rot. do Mar-Verm.\textunderscore , 36.
\section{Desembarcadeiro}
\begin{itemize}
\item {Grp. gram.:m.}
\end{itemize}
(V.desembarcadoiro)
\section{Desembarcadoiro}
\begin{itemize}
\item {Grp. gram.:m.}
\end{itemize}
\begin{itemize}
\item {Proveniência:(De \textunderscore des...\textunderscore  + \textunderscore embarcadoiro\textunderscore )}
\end{itemize}
Lugar de desembarque.
\section{Desembarcadouro}
\begin{itemize}
\item {Grp. gram.:m.}
\end{itemize}
\begin{itemize}
\item {Proveniência:(De \textunderscore des...\textunderscore  + \textunderscore embarcadouro\textunderscore )}
\end{itemize}
Lugar de desembarque.
\section{Desembarcar}
\begin{itemize}
\item {Grp. gram.:v. t.}
\end{itemize}
\begin{itemize}
\item {Grp. gram.:V. i.}
\end{itemize}
\begin{itemize}
\item {Proveniência:(De \textunderscore des...\textunderscore  + \textunderscore embarcar\textunderscore )}
\end{itemize}
Tirar de uma embarcação: \textunderscore desembarcar fardos\textunderscore .
Saír de uma embarcação.
Apear-se de um combóio.
\section{Desembarco}
\begin{itemize}
\item {Grp. gram.:m.}
\end{itemize}
O mesmo que \textunderscore desembarque\textunderscore . Cf. \textunderscore Hist. Insulana\textunderscore , II, 33 e 41.
\section{Desembargadamente}
\begin{itemize}
\item {Grp. gram.:adv.}
\end{itemize}
\begin{itemize}
\item {Proveniência:(De \textunderscore desembargar\textunderscore )}
\end{itemize}
Sem embargo.
Resolutamente.
\section{Desembargador}
\begin{itemize}
\item {Grp. gram.:m.}
\end{itemize}
\begin{itemize}
\item {Proveniência:(De \textunderscore desembargar\textunderscore )}
\end{itemize}
Designação antiga de cada juiz do tribunal da Relação.
Antigo magistrado do Desembargo do Paço.
Membro do tribunal ecclesiástico do patriarchado.
\section{Desembargar}
\begin{itemize}
\item {Grp. gram.:v. t.}
\end{itemize}
\begin{itemize}
\item {Proveniência:(De \textunderscore des...\textunderscore  + \textunderscore embargar\textunderscore )}
\end{itemize}
Tirar o embargo a.
Pôr desembargo a.
Despachar.
Resolver.
\section{Desembargo}
\begin{itemize}
\item {Grp. gram.:m.}
\end{itemize}
Acto de desembargar.
Antiga magistratura de Desembargadores.
\section{Desembarque}
\begin{itemize}
\item {Grp. gram.:m.}
\end{itemize}
Acto do desembarcar.
\section{Desembarrancar}
\begin{itemize}
\item {Grp. gram.:v. t.}
\end{itemize}
\begin{itemize}
\item {Grp. gram.:V. i.}
\end{itemize}
\begin{itemize}
\item {Utilização:Prov.}
\end{itemize}
\begin{itemize}
\item {Utilização:trasm.}
\end{itemize}
\begin{itemize}
\item {Proveniência:(De \textunderscore des...\textunderscore  + \textunderscore barranco\textunderscore )}
\end{itemize}
Desatascar.
Desatolar.
Tirar de um barranco.
Tomar uma resolução, dar resposta decisiva.
\section{Desembarrilar}
\begin{itemize}
\item {Grp. gram.:v. t.}
\end{itemize}
\begin{itemize}
\item {Utilização:Fig.}
\end{itemize}
\begin{itemize}
\item {Proveniência:(De \textunderscore des...\textunderscore  + \textunderscore embarrilar\textunderscore )}
\end{itemize}
Tirar do barril.
Desenganar.
\section{Desembatiar}
\begin{itemize}
\item {Grp. gram.:v.}
\end{itemize}
\begin{itemize}
\item {Utilização:t. Marn.}
\end{itemize}
\begin{itemize}
\item {Proveniência:(De \textunderscore des...\textunderscore  + \textunderscore embate\textunderscore ?)}
\end{itemize}
Nivelar pelo centro (as peças das salinas), desfazendo as saliências até a pervinca.
\section{Desembaular}
\begin{itemize}
\item {Grp. gram.:v. t.}
\end{itemize}
\begin{itemize}
\item {Proveniência:(De \textunderscore des...\textunderscore  + \textunderscore embaular\textunderscore )}
\end{itemize}
Tirar do bau.
Despejar.
\section{Desembebedar}
\begin{itemize}
\item {Grp. gram.:v. t.}
\end{itemize}
\begin{itemize}
\item {Proveniência:(De \textunderscore des...\textunderscore  + \textunderscore embebedar\textunderscore )}
\end{itemize}
Fazer passar a embriaguez a.
\section{Desembestadamente}
\begin{itemize}
\item {Grp. gram.:adv.}
\end{itemize}
Desenfreadamente.
De modo desembestado.
\section{Desembestado}
\begin{itemize}
\item {Grp. gram.:adj.}
\end{itemize}
Desenfreado.
Devasso.
\section{Desembestar}
\begin{itemize}
\item {Grp. gram.:v. t.}
\end{itemize}
\begin{itemize}
\item {Grp. gram.:V. i.}
\end{itemize}
\begin{itemize}
\item {Proveniência:(De \textunderscore bésta\textunderscore )}
\end{itemize}
Arremessar, atirar, (como bésta).
Soltar, despedir, (afrontas, injúrias, etc.).
Correr impetuosamente.
\section{Desembezerrar}
\begin{itemize}
\item {Grp. gram.:v. t.}
\end{itemize}
\begin{itemize}
\item {Proveniência:(De \textunderscore des...\textunderscore  + \textunderscore embezerrar\textunderscore )}
\end{itemize}
Desamuar.
\section{Desembirrar}
\begin{itemize}
\item {Grp. gram.:v. t.}
\end{itemize}
\begin{itemize}
\item {Grp. gram.:V. i.}
\end{itemize}
\begin{itemize}
\item {Proveniência:(De \textunderscore des...\textunderscore  + \textunderscore embirrar\textunderscore )}
\end{itemize}
Tirar birra a.
Deixar a birra.
\section{Desemblinhar-se}
\begin{itemize}
\item {Grp. gram.:v. p.}
\end{itemize}
\begin{itemize}
\item {Utilização:Prov.}
\end{itemize}
\begin{itemize}
\item {Utilização:trasm.}
\end{itemize}
Correr a toda a brida.
Desembaraçar-se.
Aviar-se.
(Relaciona-se com \textunderscore desbolinar\textunderscore ?)
\section{Desembocadura}
\begin{itemize}
\item {Grp. gram.:f.}
\end{itemize}
Acto de desembocar.
Lugar, onde um rio desemboca.
\section{Desembocar}
\begin{itemize}
\item {Grp. gram.:v. t.}
\end{itemize}
\begin{itemize}
\item {Grp. gram.:V. i.}
\end{itemize}
\begin{itemize}
\item {Proveniência:(De \textunderscore des...\textunderscore  + \textunderscore embocar\textunderscore )}
\end{itemize}
Fazer sair.
Ir dar; desaguar: \textunderscore o Zézere desemboca no Tejo\textunderscore .
Terminar: \textunderscore esta rua desemboca na Sé\textunderscore .
\section{Desembolar}
\begin{itemize}
\item {Grp. gram.:v. t.}
\end{itemize}
\begin{itemize}
\item {Grp. gram.:V. p.}
\end{itemize}
\begin{itemize}
\item {Proveniência:(De \textunderscore des...\textunderscore  + \textunderscore embolar\textunderscore )}
\end{itemize}
Tirar as bolas a (o toiro).
Livrar-se das bolas que lhe guarnecem as pontas (o toiro).
\section{Desembolsar}
\begin{itemize}
\item {Grp. gram.:v. t.}
\end{itemize}
\begin{itemize}
\item {Proveniência:(De \textunderscore des...\textunderscore  + \textunderscore embolsar\textunderscore )}
\end{itemize}
Tirar da bolsa.
Gastar: \textunderscore desembolsei muito dinheiro\textunderscore .
\section{Desembôlso}
\begin{itemize}
\item {Grp. gram.:m.}
\end{itemize}
Acto de desembolsar.
Aquillo que se gastou ou que se pagou.
Aquillo que se pagou adeantadamente, para se rehaver o mesmo ou o equivalente: \textunderscore estou no desembôlso de 4 escudos\textunderscore .
Despesa.
\section{Desemborcar}
\begin{itemize}
\item {Grp. gram.:v. t.}
\end{itemize}
\begin{itemize}
\item {Proveniência:(De \textunderscore des...\textunderscore  + \textunderscore emborcar\textunderscore )}
\end{itemize}
Voltar para cima (aquillo que estava emborcado).
\section{Desemborrachar}
\begin{itemize}
\item {Grp. gram.:v. t.}
\end{itemize}
O mesmo que \textunderscore desembebedar\textunderscore .
\section{Desemborrascar}
\begin{itemize}
\item {Grp. gram.:v. t.}
\end{itemize}
\begin{itemize}
\item {Proveniência:(De \textunderscore des...\textunderscore  + \textunderscore borrasca\textunderscore )}
\end{itemize}
Desassombrar.
Tornar sereno. Cf. Cortesão, \textunderscore Subs.\textunderscore 
\section{Desemboscar}
\begin{itemize}
\item {Grp. gram.:v. t.}
\end{itemize}
\begin{itemize}
\item {Proveniência:(De \textunderscore des...\textunderscore  + \textunderscore emboscar\textunderscore )}
\end{itemize}
Fazer sair do bosque ou da emboscada.
\section{Desembotar}
\begin{itemize}
\item {Grp. gram.:v. i.}
\end{itemize}
\begin{itemize}
\item {Proveniência:(De \textunderscore des...\textunderscore  + \textunderscore embotar\textunderscore )}
\end{itemize}
Tornar cortante, afiar.
Tornar ágil.
Fazer que (os dentes) deixem de estar botos.
\section{Desembraçar}
\begin{itemize}
\item {Grp. gram.:v. t.}
\end{itemize}
\begin{itemize}
\item {Proveniência:(De \textunderscore des...\textunderscore  + \textunderscore embraçar\textunderscore )}
\end{itemize}
Largar (aquillo que estava embraçado).
\section{Desembravecer}
\begin{itemize}
\item {Grp. gram.:v. t.}
\end{itemize}
\begin{itemize}
\item {Grp. gram.:V. i.}
\end{itemize}
\begin{itemize}
\item {Proveniência:(De \textunderscore des...\textunderscore  + \textunderscore embravecer\textunderscore )}
\end{itemize}
Amansar.
Acalmar.
Tirar a braveza a.
Tornar-se manso.
Perder a braveza.
\section{Desembrear}
\begin{itemize}
\item {Grp. gram.:v. t.}
\end{itemize}
\begin{itemize}
\item {Proveniência:(De \textunderscore des...\textunderscore  + \textunderscore embrear\textunderscore )}
\end{itemize}
Limpar do breu, do alcatrão.
\section{Desembrechar}
\begin{itemize}
\item {Grp. gram.:v. t.}
\end{itemize}
\begin{itemize}
\item {Proveniência:(De \textunderscore des...\textunderscore  + \textunderscore embrechar\textunderscore )}
\end{itemize}
Tirar o embrechado a.
\section{Desembrenhar}
\begin{itemize}
\item {Grp. gram.:v. t.}
\end{itemize}
\begin{itemize}
\item {Grp. gram.:V. p.}
\end{itemize}
\begin{itemize}
\item {Proveniência:(De \textunderscore des...\textunderscore  + \textunderscore embrenhar\textunderscore )}
\end{itemize}
Fazer sair das brenhas.
Tirar para fóra.
Sair das brenhas.
Sair para fóra.
\section{Desembriagar}
\begin{itemize}
\item {Grp. gram.:v. t.}
\end{itemize}
O mesmo que \textunderscore desembebedar\textunderscore .
\section{Desembridar}
\begin{itemize}
\item {Grp. gram.:v. t.}
\end{itemize}
(V.desbridar)
\section{Desembrulhadamente}
\begin{itemize}
\item {Grp. gram.:adv.}
\end{itemize}
Com clareza.
\section{Desembrulhar}
\begin{itemize}
\item {Grp. gram.:v. t.}
\end{itemize}
\begin{itemize}
\item {Proveniência:(De \textunderscore des...\textunderscore  + \textunderscore embrulhar\textunderscore )}
\end{itemize}
Tirar de embrulho.
Desdobrar.
Desenredar.
Esclarecer: \textunderscore desembrulhar uma intriga\textunderscore .
\section{Desembrulho}
\begin{itemize}
\item {Grp. gram.:m.}
\end{itemize}
Acto de desembrulhar.
\section{Desembruscar}
\begin{itemize}
\item {Grp. gram.:v. t.}
\end{itemize}
\begin{itemize}
\item {Proveniência:(De \textunderscore des...\textunderscore  + \textunderscore embruscar\textunderscore )}
\end{itemize}
Desanuvear.
Tornar limpo ou claro.
\section{Desembrutecer}
\begin{itemize}
\item {Grp. gram.:v. t.}
\end{itemize}
\begin{itemize}
\item {Proveniência:(De \textunderscore des...\textunderscore  + \textunderscore embrutecer\textunderscore )}
\end{itemize}
Tirar a bruteza a.
\section{Desembruxar}
\begin{itemize}
\item {Grp. gram.:v. t.}
\end{itemize}
\begin{itemize}
\item {Proveniência:(De \textunderscore des...\textunderscore  + \textunderscore embruxar\textunderscore )}
\end{itemize}
O mesmo que \textunderscore desenfeitiçar\textunderscore .
Livrar de bruxarias.
\section{Desembuçadamente}
\begin{itemize}
\item {Grp. gram.:adv.}
\end{itemize}
Com franqueza.
\section{Desembuçar}
\begin{itemize}
\item {Grp. gram.:v. t.}
\end{itemize}
\begin{itemize}
\item {Proveniência:(De \textunderscore des...\textunderscore  + \textunderscore embuçar\textunderscore )}
\end{itemize}
Tirar o embuço a.
Patentear.
Esclarecer: \textunderscore desembuçar a verdade\textunderscore .
\section{Desembuchar}
\begin{itemize}
\item {Grp. gram.:v. t.}
\end{itemize}
\begin{itemize}
\item {Grp. gram.:V. i.}
\end{itemize}
\begin{itemize}
\item {Proveniência:(De \textunderscore des...\textunderscore  + \textunderscore embuchar\textunderscore )}
\end{itemize}
Desimpedir (aquelle ou aquillo que estava embuchado).
Expandir, expor francamente (aquillo que se pensa).
Desabafar, falando.
\section{Desembuço}
\begin{itemize}
\item {Grp. gram.:m.}
\end{itemize}
Acto de desembuçar.
\section{Desemburilhar}
\begin{itemize}
\item {Grp. gram.:v. t.}
\end{itemize}
\begin{itemize}
\item {Utilização:Ant.}
\end{itemize}
O mesmo que \textunderscore desembrulhar\textunderscore . Cf. Fern. Lopes, \textunderscore Chrón. de D. João I\textunderscore , p. 2.^a, c. XLV.
\section{Desemburrar}
\begin{itemize}
\item {Grp. gram.:v. t.}
\end{itemize}
\begin{itemize}
\item {Utilização:Fam.}
\end{itemize}
\begin{itemize}
\item {Proveniência:(De \textunderscore des...\textunderscore  + \textunderscore em...\textunderscore  + \textunderscore burro\textunderscore )}
\end{itemize}
Livrar da ignorância.
Ensinar rudimentos a: \textunderscore desemburrar uma criança\textunderscore .
\section{Desemburricar}
\begin{itemize}
\item {Grp. gram.:v. t.}
\end{itemize}
\begin{itemize}
\item {Utilização:Fam.}
\end{itemize}
\begin{itemize}
\item {Proveniência:(De \textunderscore des...\textunderscore  + \textunderscore em\textunderscore  + \textunderscore burrico\textunderscore )}
\end{itemize}
Dar o primeiro ensino a.
Desemburrar.
\section{Desemendar-se}
\begin{itemize}
\item {Grp. gram.:v. p.}
\end{itemize}
\begin{itemize}
\item {Utilização:Prov.}
\end{itemize}
\begin{itemize}
\item {Utilização:trasm.}
\end{itemize}
Emendar-se.
\section{Desemmaçar}
\begin{itemize}
\item {Grp. gram.:v. t.}
\end{itemize}
\begin{itemize}
\item {Proveniência:(De \textunderscore des...\textunderscore  + \textunderscore emmaçar\textunderscore )}
\end{itemize}
Separar (aquillo que estava reunido em maço): \textunderscore desemmaçar notas do Banco\textunderscore .
\section{Desemmadeirar}
\begin{itemize}
\item {Grp. gram.:v. t.}
\end{itemize}
\begin{itemize}
\item {Proveniência:(De \textunderscore des...\textunderscore  + \textunderscore emmadeirar\textunderscore )}
\end{itemize}
Tirar o madeiramento a.
\section{Desemmalar}
\begin{itemize}
\item {Grp. gram.:v. t.}
\end{itemize}
\begin{itemize}
\item {Proveniência:(De \textunderscore des...\textunderscore  + \textunderscore emmalar\textunderscore )}
\end{itemize}
Tirar da mala.
\section{Desemmalhar}
\begin{itemize}
\item {Grp. gram.:v. t.}
\end{itemize}
\begin{itemize}
\item {Proveniência:(De \textunderscore des...\textunderscore  + \textunderscore emmalhar\textunderscore )}
\end{itemize}
Tirar das malhas da rêde.
\section{Desemmalhetar}
\begin{itemize}
\item {Grp. gram.:v. t.}
\end{itemize}
\begin{itemize}
\item {Proveniência:(De \textunderscore des...\textunderscore  + \textunderscore emmalhetar\textunderscore )}
\end{itemize}
Destravar (aquillo que estava emmalhetado).
\section{Desemmaranhar}
\begin{itemize}
\item {Grp. gram.:v. t.}
\end{itemize}
\begin{itemize}
\item {Proveniência:(De \textunderscore des...\textunderscore  + \textunderscore emmaranhar\textunderscore )}
\end{itemize}
Desenredar.
Esclarecer.
\section{Desemmastear}
\begin{itemize}
\item {Proveniência:(De \textunderscore des...\textunderscore  + \textunderscore emmastrear\textunderscore )}
\end{itemize}
\textunderscore v. t.\textunderscore  (e der)
O mesmo que \textunderscore desmastrear\textunderscore , etc.
\section{Desemmedar}
\begin{itemize}
\item {Grp. gram.:v. t.}
\end{itemize}
\begin{itemize}
\item {Proveniência:(De \textunderscore des...\textunderscore  + \textunderscore emmedar\textunderscore )}
\end{itemize}
Desmanchar as medas de.
\section{Desemmoinhar}
\begin{itemize}
\item {Grp. gram.:v. t.}
\end{itemize}
\begin{itemize}
\item {Proveniência:(De \textunderscore moínha\textunderscore )}
\end{itemize}
Tirar a moínha a.
\section{Desemmoldurar}
\begin{itemize}
\item {Grp. gram.:v. t.}
\end{itemize}
\begin{itemize}
\item {Proveniência:(De \textunderscore des...\textunderscore  + \textunderscore emmoldurar\textunderscore )}
\end{itemize}
Tirar da moldura; desenquadrar; desencaixilhar: \textunderscore desemmoldurar um quadro\textunderscore .
\section{Desemmudecer}
\begin{itemize}
\item {Grp. gram.:v. t.}
\end{itemize}
\begin{itemize}
\item {Grp. gram.:V. i.}
\end{itemize}
\begin{itemize}
\item {Proveniência:(De \textunderscore des...\textunderscore  + \textunderscore emmudecer\textunderscore )}
\end{itemize}
Fazer sair do silêncio a.
Deixar de estar silencioso.
Recuperar a fala.
\section{Desempacar}
\begin{itemize}
\item {Grp. gram.:v. t.}
\end{itemize}
\begin{itemize}
\item {Utilização:Bras}
\end{itemize}
\begin{itemize}
\item {Proveniência:(De \textunderscore des\textunderscore  + \textunderscore empacar\textunderscore )}
\end{itemize}
Desemperrar (a cavalgadura).
\section{Desempachadamente}
\begin{itemize}
\item {Grp. gram.:adv.}
\end{itemize}
\begin{itemize}
\item {Proveniência:(De \textunderscore desempachar\textunderscore )}
\end{itemize}
Com desembaraço; com allívio.
\section{Desempachar}
\begin{itemize}
\item {Grp. gram.:v. t.}
\end{itemize}
\begin{itemize}
\item {Proveniência:(De \textunderscore des...\textunderscore  + \textunderscore empachar\textunderscore )}
\end{itemize}
Desobstruir; alliviar.
\section{Desempacho}
\begin{itemize}
\item {Grp. gram.:m.}
\end{itemize}
Acto de desempachar.
\section{Desempacotamento}
\begin{itemize}
\item {Grp. gram.:m.}
\end{itemize}
Acto de desempacotar.
\section{Desempacotar}
\begin{itemize}
\item {Grp. gram.:v. t.}
\end{itemize}
\begin{itemize}
\item {Proveniência:(De \textunderscore des...\textunderscore  + \textunderscore empacotar\textunderscore )}
\end{itemize}
Tirar do pacote.
\section{Desempadralhar-se}
\begin{itemize}
\item {Grp. gram.:v. p.}
\end{itemize}
\begin{itemize}
\item {Utilização:Burl.}
\end{itemize}
Deixar de pertencer á padralhada.
Despadrar-se.
(Cp. \textunderscore padralhada\textunderscore )
\section{Desempalhar}
\begin{itemize}
\item {Grp. gram.:v. t.}
\end{itemize}
\begin{itemize}
\item {Proveniência:(De \textunderscore des...\textunderscore  + \textunderscore empalhar\textunderscore )}
\end{itemize}
Tirar da palha ou do palheiro.
Tirar a palha que envolve ou que enche.
\section{Desempalmar}
\begin{itemize}
\item {Grp. gram.:v. t.}
\end{itemize}
\begin{itemize}
\item {Proveniência:(De \textunderscore des...\textunderscore  + \textunderscore empalmar\textunderscore )}
\end{itemize}
Largar ou mostrar (aquillo que estava empalmado).
\section{Desempambado}
\begin{itemize}
\item {Grp. gram.:adj.}
\end{itemize}
\begin{itemize}
\item {Utilização:Bras. da Baia}
\end{itemize}
Desembaraçado; franco.
Positivo.
\section{Desempanar}
\begin{itemize}
\item {Grp. gram.:v. t.}
\end{itemize}
\begin{itemize}
\item {Utilização:Fig.}
\end{itemize}
\begin{itemize}
\item {Proveniência:(De \textunderscore des...\textunderscore  + \textunderscore empanar\textunderscore )}
\end{itemize}
Tirar os panos a.
Esclarecer.
Dar brilho a.
\section{Desempapelar}
\begin{itemize}
\item {Grp. gram.:v. t.}
\end{itemize}
\begin{itemize}
\item {Proveniência:(De \textunderscore des...\textunderscore  + \textunderscore empapelar\textunderscore )}
\end{itemize}
Tirar do papel ou de papéis.
Desembrulhar.
\section{Desempar}
\begin{itemize}
\item {Grp. gram.:v. t.}
\end{itemize}
\begin{itemize}
\item {Proveniência:(De \textunderscore des...\textunderscore  + \textunderscore empar\textunderscore )}
\end{itemize}
Tirar as estacas que sustentam (as videiras).
\section{Desemparceirar}
\begin{itemize}
\item {Grp. gram.:v. t.}
\end{itemize}
\begin{itemize}
\item {Proveniência:(De \textunderscore des...\textunderscore  + \textunderscore emparceirar\textunderscore )}
\end{itemize}
Separar (quem estava emparceirado).
\section{Desemparedar}
\begin{itemize}
\item {Grp. gram.:v. t.}
\end{itemize}
\begin{itemize}
\item {Proveniência:(De \textunderscore des...\textunderscore  + \textunderscore emparedar\textunderscore )}
\end{itemize}
Soltar (aquelle ou aquillo que estava emparedado).
\section{Desemparelhar}
\begin{itemize}
\item {Grp. gram.:v. t.}
\end{itemize}
\begin{itemize}
\item {Proveniência:(De \textunderscore des...\textunderscore  + \textunderscore emparelhar\textunderscore )}
\end{itemize}
Separar (aquillo que estava emparelhado).
\section{Desempastar}
\begin{itemize}
\item {Grp. gram.:v. t.}
\end{itemize}
\begin{itemize}
\item {Proveniência:(De \textunderscore des...\textunderscore  + \textunderscore empastar\textunderscore )}
\end{itemize}
Desmanchar (aquillo que estava empastado).
\section{Desempatar}
\begin{itemize}
\item {Grp. gram.:v. t.}
\end{itemize}
\begin{itemize}
\item {Proveniência:(De \textunderscore des...\textunderscore  + \textunderscore empatar\textunderscore )}
\end{itemize}
Tirar o empate a.
Resolver: \textunderscore desempatar uma questão\textunderscore .
\section{Desempate}
\begin{itemize}
\item {Grp. gram.:m.}
\end{itemize}
Acto de desempatar.
\section{Desempavesar}
\begin{itemize}
\item {Grp. gram.:v. t.}
\end{itemize}
\begin{itemize}
\item {Proveniência:(De \textunderscore des...\textunderscore  + \textunderscore empavesar\textunderscore )}
\end{itemize}
Tirar os paveses a.
\section{Desempeçadamente}
\begin{itemize}
\item {Grp. gram.:adv.}
\end{itemize}
\begin{itemize}
\item {Proveniência:(De \textunderscore desempeçar\textunderscore )}
\end{itemize}
Desembaraçadamente.
\section{Desempeçar}
\begin{itemize}
\item {Grp. gram.:v. t.}
\end{itemize}
\begin{itemize}
\item {Proveniência:(De \textunderscore des...\textunderscore  + \textunderscore empeçar\textunderscore )}
\end{itemize}
Desenredar.
Tirar o empecilho de.
\section{Desempecer}
\begin{itemize}
\item {Grp. gram.:v. t.}
\end{itemize}
O mesmo que \textunderscore desempeçar\textunderscore .
\section{Desempecilhar}
\begin{itemize}
\item {Grp. gram.:v. t.}
\end{itemize}
\begin{itemize}
\item {Proveniência:(De \textunderscore des...\textunderscore  + \textunderscore empecilho\textunderscore )}
\end{itemize}
Tirar o empecilho a.
Desembaraçar. Cf. Filinto, XIII, 94.
\section{Desempêço}
\begin{itemize}
\item {Grp. gram.:m.}
\end{itemize}
\begin{itemize}
\item {Proveniência:(De \textunderscore desempeçar\textunderscore )}
\end{itemize}
Allívio.
Desobstrucção.
Desembaraço.
\section{Desempedernecer}
\begin{itemize}
\item {Grp. gram.:v. t.}
\end{itemize}
(V.desempedernir)
\section{Desempedernir}
\begin{itemize}
\item {Grp. gram.:v. t.}
\end{itemize}
\begin{itemize}
\item {Proveniência:(De \textunderscore des...\textunderscore  + \textunderscore empedernir\textunderscore )}
\end{itemize}
Abrandar, tornar molle, (aquelle ou aquillo que estava empedernido).
\section{Desempedrar}
\begin{itemize}
\item {Grp. gram.:v. t.}
\end{itemize}
\begin{itemize}
\item {Proveniência:(De \textunderscore des...\textunderscore  + \textunderscore empedrar\textunderscore )}
\end{itemize}
Tirar o empedramento de.
Tirar as pedras a.
\section{Desempègar}
\begin{itemize}
\item {Grp. gram.:v. t.}
\end{itemize}
\begin{itemize}
\item {Proveniência:(De \textunderscore des...\textunderscore  + \textunderscore empègar\textunderscore )}
\end{itemize}
Tirar do pego.
Tirar a água de (um pego, um reservatório, etc.).
\section{Desempego}
\begin{itemize}
\item {Grp. gram.:m.}
\end{itemize}
Acto de desempègar.
\section{Desempenadamente}
\begin{itemize}
\item {Grp. gram.:adv.}
\end{itemize}
Com desempeno.
\section{Desempenadeira}
\begin{itemize}
\item {Grp. gram.:f.}
\end{itemize}
\begin{itemize}
\item {Proveniência:(De \textunderscore desempenar\textunderscore )}
\end{itemize}
Instrumento de madeira, que o caiador segura por uma asa, para alastrar e alisar a cal na parede.
\section{Desempenar}
\begin{itemize}
\item {Grp. gram.:v. t.}
\end{itemize}
\begin{itemize}
\item {Proveniência:(De \textunderscore des...\textunderscore  + \textunderscore empenar\textunderscore )}
\end{itemize}
Tirar o empeno a.
Endireitar: \textunderscore desempenar uma tábua\textunderscore .
Vêr se há empeno em.
\section{Desempenhar}
\begin{itemize}
\item {Grp. gram.:v. t.}
\end{itemize}
\begin{itemize}
\item {Proveniência:(De \textunderscore des...\textunderscore  + \textunderscore empenhar\textunderscore )}
\end{itemize}
Resgatar (aquillo que estava penhorado): \textunderscore desempenhar um relógio\textunderscore .
Cumprir (aquillo a que se estava obrigado): \textunderscore desempenhar uma commissão\textunderscore .
Desobrigar: \textunderscore desempenhar alguém de uma obrigação\textunderscore .
Livrar de dividas.
Representar em scena: \textunderscore desempenhar um auto de Gil Vicente\textunderscore .
\section{Desempenho}
\begin{itemize}
\item {Grp. gram.:m.}
\end{itemize}
Acto ou effeito de desempenhar: \textunderscore o desempenho de um drama\textunderscore .
\section{Desempeno}
\begin{itemize}
\item {Grp. gram.:m.}
\end{itemize}
Acto ou effeito de desempenar.
Cada uma das réguas, com que o carpinteiro verifica se uma peça está plana ou desempenada.
Aprumo.
Agilidade.
Galhardia, elegância.
\section{Desemperramento}
\begin{itemize}
\item {Grp. gram.:m.}
\end{itemize}
Acto de desemperrar.
\section{Desemperrar}
\begin{itemize}
\item {Grp. gram.:v. t.}
\end{itemize}
\begin{itemize}
\item {Proveniência:(De \textunderscore des...\textunderscore  + \textunderscore emperrar\textunderscore )}
\end{itemize}
Tirar a perrice a.
Alargar (aquillo que estava perro).
\section{Desemperro}
\begin{itemize}
\item {fónica:pê}
\end{itemize}
\begin{itemize}
\item {Grp. gram.:m.}
\end{itemize}
O mesmo que \textunderscore desemperramento\textunderscore .
\section{Desempestar}
\begin{itemize}
\item {Grp. gram.:v. t.}
\end{itemize}
\begin{itemize}
\item {Proveniência:(De \textunderscore des...\textunderscore  + \textunderscore empestar\textunderscore )}
\end{itemize}
Desinfeccionar.
Livrar da peste.
\section{Desempilhar}
\begin{itemize}
\item {Grp. gram.:v. t.}
\end{itemize}
\begin{itemize}
\item {Proveniência:(De \textunderscore des...\textunderscore  + \textunderscore empilhar\textunderscore )}
\end{itemize}
Desarrumar (aquillo que estava empilhado).
\section{Desemplasto}
\textunderscore m.\textunderscore  (e der.)
(V. \textunderscore desemplastro\textunderscore , etc.)
\section{Desemplastrar}
\begin{itemize}
\item {Grp. gram.:v. t.}
\end{itemize}
\begin{itemize}
\item {Proveniência:(De \textunderscore des...\textunderscore  + \textunderscore emplastar\textunderscore )}
\end{itemize}
Tirar o emplastro a.
\section{Desemplastro}
\begin{itemize}
\item {Grp. gram.:m.}
\end{itemize}
Acto de desemplastrar.
\section{Desemplumar}
\begin{itemize}
\item {Grp. gram.:v. t.}
\end{itemize}
\begin{itemize}
\item {Proveniência:(De \textunderscore des...\textunderscore  + \textunderscore emplumar\textunderscore )}
\end{itemize}
Tirar as pennas ou as plumas a.
\section{Desempoado}
\begin{itemize}
\item {Grp. gram.:adj.}
\end{itemize}
\begin{itemize}
\item {Proveniência:(De \textunderscore desempoar\textunderscore )}
\end{itemize}
Lhano, tratável.
Modesto.
\section{Desempoar}
\begin{itemize}
\item {Grp. gram.:v. t.}
\end{itemize}
\begin{itemize}
\item {Utilização:Fig.}
\end{itemize}
\begin{itemize}
\item {Proveniência:(De \textunderscore des...\textunderscore  + \textunderscore empoar\textunderscore )}
\end{itemize}
Tirar o pó a.
Sacudir.
Tirar preconceitos a.
Tornar modesto, affável.
\section{Desempobrecer}
\begin{itemize}
\item {Grp. gram.:v. t.}
\end{itemize}
\begin{itemize}
\item {Grp. gram.:V. i.}
\end{itemize}
\begin{itemize}
\item {Proveniência:(De \textunderscore des...\textunderscore  + \textunderscore empobrecer\textunderscore )}
\end{itemize}
Tirar da pobreza.
Saír da pobreza.
\section{Desempoçar}
\begin{itemize}
\item {Grp. gram.:v. t.}
\end{itemize}
\begin{itemize}
\item {Proveniência:(De \textunderscore des...\textunderscore  + \textunderscore empoçar\textunderscore )}
\end{itemize}
Tirar de poço.
Tirar água de (um poço).
\section{Desempoeirado}
\begin{itemize}
\item {Grp. gram.:adj.}
\end{itemize}
\begin{itemize}
\item {Utilização:Fam.}
\end{itemize}
Que não tem vaidade nem soberba.
Modesto.
\section{Desempoeirar}
\begin{itemize}
\item {Grp. gram.:v. t.}
\end{itemize}
\begin{itemize}
\item {Proveniência:(De \textunderscore poeira\textunderscore )}
\end{itemize}
O mesmo que \textunderscore desempoar\textunderscore .
\section{Desempolar}
\begin{itemize}
\item {Grp. gram.:v. t.}
\end{itemize}
\begin{itemize}
\item {Proveniência:(De \textunderscore des...\textunderscore  + \textunderscore empolar\textunderscore )}
\end{itemize}
Tirar empôla ou empôlas a.
Aplanar, alisar.
\section{Desempoleirar}
\begin{itemize}
\item {Grp. gram.:v. i.}
\end{itemize}
\begin{itemize}
\item {Utilização:pop.}
\end{itemize}
\begin{itemize}
\item {Utilização:Fig.}
\end{itemize}
\begin{itemize}
\item {Proveniência:(De \textunderscore des...\textunderscore  + \textunderscore empoleirar\textunderscore )}
\end{itemize}
Tirar do poleiro.
Abater, fazer descer de posição elevada.
\section{Desempolgadura}
\begin{itemize}
\item {Grp. gram.:f.}
\end{itemize}
Acto de desempolgar.
\section{Desempolgar}
\begin{itemize}
\item {Grp. gram.:v. t.}
\end{itemize}
\begin{itemize}
\item {Proveniência:(De \textunderscore des...\textunderscore  + \textunderscore empolgar\textunderscore )}
\end{itemize}
Largar da mão ou da garra.
\section{Desempossar}
\begin{itemize}
\item {Grp. gram.:v. t.}
\end{itemize}
(V.desapossar)
\section{Desempregado}
\begin{itemize}
\item {Grp. gram.:adj.}
\end{itemize}
Que está fóra de emprêgo.
Que deixou de exercer offício ou profissão.
\section{Desempregar}
\begin{itemize}
\item {Grp. gram.:v. t.}
\end{itemize}
\begin{itemize}
\item {Proveniência:(De \textunderscore des...\textunderscore  + \textunderscore empregar\textunderscore )}
\end{itemize}
Tirar do emprêgo.
Exonerar, destituir.
\section{Desemprenhar}
\begin{itemize}
\item {Grp. gram.:v. i.}
\end{itemize}
\begin{itemize}
\item {Utilização:Fig.}
\end{itemize}
O mesmo que \textunderscore parir\textunderscore .
Falar sem rebuço.
Expor quanto se sente:«\textunderscore Ora pois acabay de desemprenhar, saibamos o que temos\textunderscore ». \textunderscore Eufrosina\textunderscore , 62.
\section{Desemproar}
\begin{itemize}
\item {Grp. gram.:v. t.}
\end{itemize}
\begin{itemize}
\item {Utilização:Fig.}
\end{itemize}
\begin{itemize}
\item {Proveniência:(De \textunderscore des...\textunderscore  + \textunderscore emproar\textunderscore )}
\end{itemize}
Abater o orgulho ou a vaidade de.
\section{Desempulhar-se}
\begin{itemize}
\item {Grp. gram.:v. p.}
\end{itemize}
\begin{itemize}
\item {Proveniência:(De \textunderscore des...\textunderscore  + \textunderscore empulhar\textunderscore )}
\end{itemize}
Livrar-se de pulhas ou de troça.
Desforrar-se de troças ou motejos. Cf. Filinto, V, 13.
\section{Desempunhar}
\begin{itemize}
\item {Grp. gram.:v. t.}
\end{itemize}
\begin{itemize}
\item {Proveniência:(De \textunderscore des...\textunderscore  + \textunderscore empunhar\textunderscore )}
\end{itemize}
Largar do punho ou da mão.
\section{Desenamorar-se}
\begin{itemize}
\item {Grp. gram.:v. p.}
\end{itemize}
\begin{itemize}
\item {Proveniência:(De \textunderscore des...\textunderscore  + \textunderscore enamorar\textunderscore )}
\end{itemize}
Deixar de estar enamorado. Cf. Castilho, \textunderscore D. Quixote\textunderscore , II, 374.
\section{Desenarcado}
\begin{itemize}
\item {fónica:desen-ar}
\end{itemize}
\begin{itemize}
\item {Grp. gram.:adj.}
\end{itemize}
\begin{itemize}
\item {Utilização:Prov.}
\end{itemize}
\begin{itemize}
\item {Utilização:trasm.}
\end{itemize}
\begin{itemize}
\item {Proveniência:(De \textunderscore des...\textunderscore  + \textunderscore en...\textunderscore  + \textunderscore arco\textunderscore )}
\end{itemize}
Diz-se da pipa ou do tonel, a que se tiraram arcos.
\section{Desencabar}
\begin{itemize}
\item {Grp. gram.:v. t.}
\end{itemize}
\begin{itemize}
\item {Proveniência:(De \textunderscore des...\textunderscore  + \textunderscore encabar\textunderscore )}
\end{itemize}
Tirar do cabo (utensílio ou instrumento).
\section{Desencabeçar}
\begin{itemize}
\item {Grp. gram.:v. t.}
\end{itemize}
\begin{itemize}
\item {Proveniência:(De \textunderscore des...\textunderscore  + \textunderscore encabeçar\textunderscore )}
\end{itemize}
Tirar da cabeça, da ideia.
\section{Desencabrestadamente}
\begin{itemize}
\item {Grp. gram.:adv.}
\end{itemize}
\begin{itemize}
\item {Utilização:Fig.}
\end{itemize}
\begin{itemize}
\item {Proveniência:(De \textunderscore desencabrestar\textunderscore )}
\end{itemize}
Desenfreadamente.
Com impeto.
\section{Desencabrestar}
\begin{itemize}
\item {Grp. gram.:v. t.}
\end{itemize}
\begin{itemize}
\item {Utilização:Fig.}
\end{itemize}
\begin{itemize}
\item {Proveniência:(De \textunderscore des...\textunderscore  + \textunderscore encabrestar\textunderscore )}
\end{itemize}
Tirar o cabresto a.
Tornar desenfreado.
\section{Desencabritar}
\begin{itemize}
\item {Grp. gram.:v. i.}
\end{itemize}
\begin{itemize}
\item {Utilização:Bras}
\end{itemize}
\begin{itemize}
\item {Utilização:fam.}
\end{itemize}
Fugir apressadamente.
Embarafustar.
\section{Desencachar}
\begin{itemize}
\item {Grp. gram.:v. t.}
\end{itemize}
\begin{itemize}
\item {Proveniência:(De \textunderscore des...\textunderscore  + \textunderscore encachar\textunderscore )}
\end{itemize}
Tirar a tanga a.
\section{Desencadeamento}
\begin{itemize}
\item {Grp. gram.:m.}
\end{itemize}
Acto de desencadear.
\section{Desencadear}
\begin{itemize}
\item {Grp. gram.:v. t.}
\end{itemize}
\begin{itemize}
\item {Grp. gram.:V. i.}
\end{itemize}
\begin{itemize}
\item {Grp. gram.:V. p.}
\end{itemize}
\begin{itemize}
\item {Proveniência:(De \textunderscore des...\textunderscore  + \textunderscore encadear\textunderscore )}
\end{itemize}
Soltar (aquillo que estava preso ou encadeado).
Desligar.
Irritar.
Sublevar: \textunderscore desencadear iras\textunderscore .
Caír com força e aturadamente, (falando-se da chuva).
Apparecer com estrondo, (falando-se especialmente da tempestade).
\section{Desencadernação}
\begin{itemize}
\item {Grp. gram.:f.}
\end{itemize}
Acto ou effeito de desencadernar.
\section{Desencadernadamente}
\begin{itemize}
\item {Grp. gram.:adv.}
\end{itemize}
\begin{itemize}
\item {Utilização:Fig.}
\end{itemize}
\begin{itemize}
\item {Proveniência:(De \textunderscore desencadernado\textunderscore )}
\end{itemize}
Desordenadamente.
Immoderadamente:«\textunderscore e comeu, sendo livreiro, desencadernadamente\textunderscore ». N. Tolentino.
\section{Desencadernado}
\begin{itemize}
\item {Grp. gram.:adj.}
\end{itemize}
\begin{itemize}
\item {Utilização:Fig.}
\end{itemize}
Não encadernado.
Destrambelhado, desconnexo:«\textunderscore consultas desencadernadas\textunderscore ». F. Manuel, \textunderscore Apólogos\textunderscore .
\section{Desencadernar}
\begin{itemize}
\item {Grp. gram.:v. t.}
\end{itemize}
\begin{itemize}
\item {Proveniência:(De \textunderscore des...\textunderscore  + \textunderscore encadernar\textunderscore )}
\end{itemize}
Tirar a encadernação a.
Desconjuntar.
\section{Desencaiporar}
\begin{itemize}
\item {Grp. gram.:v. t.}
\end{itemize}
\begin{itemize}
\item {Utilização:Bras}
\end{itemize}
\begin{itemize}
\item {Grp. gram.:V. i.}
\end{itemize}
\begin{itemize}
\item {Proveniência:(De \textunderscore des...\textunderscore  + \textunderscore encaiporar\textunderscore )}
\end{itemize}
Fazer cessar a infelicidade de.
Deixar de sêr infeliz.
\section{Desencaixadura}
\begin{itemize}
\item {Grp. gram.:f.}
\end{itemize}
O mesmo que \textunderscore desencaixamento\textunderscore .
\section{Desencaixamento}
\begin{itemize}
\item {Grp. gram.:m.}
\end{itemize}
Acto ou effeito de desencaixar.
\section{Desencaixar}
\begin{itemize}
\item {Grp. gram.:v. t.}
\end{itemize}
\begin{itemize}
\item {Utilização:Fig.}
\end{itemize}
\begin{itemize}
\item {Proveniência:(De \textunderscore des...\textunderscore  + \textunderscore encaixar\textunderscore .)}
\end{itemize}
Tirar do encaixe.
Dizer desapropositadamente: \textunderscore desencaixar offensas\textunderscore .
\section{Desencaixe}
\begin{itemize}
\item {Grp. gram.:m.}
\end{itemize}
O mesmo que \textunderscore desencaixamento\textunderscore .
\section{Desencaixilhar}
\begin{itemize}
\item {Grp. gram.:v. t.}
\end{itemize}
\begin{itemize}
\item {Proveniência:(De \textunderscore des...\textunderscore  + \textunderscore encaixilhar\textunderscore )}
\end{itemize}
Tirar o caixilho a.
Tirar do caixilho.
Desenquadrar.
\section{Desencaixotamento}
\begin{itemize}
\item {Grp. gram.:m.}
\end{itemize}
Acto de desencaixotar.
\section{Desencaixotar}
\begin{itemize}
\item {Grp. gram.:v. t.}
\end{itemize}
\begin{itemize}
\item {Proveniência:(De \textunderscore des...\textunderscore  + \textunderscore encaixotar\textunderscore )}
\end{itemize}
Tirar do caixote, de caixa ou de mala.
\section{Desencalacração}
\begin{itemize}
\item {Grp. gram.:f.}
\end{itemize}
Acto de desencalacrar.
\section{Desencalacrar}
\begin{itemize}
\item {Grp. gram.:v. t.}
\end{itemize}
\begin{itemize}
\item {Utilização:Pop.}
\end{itemize}
\begin{itemize}
\item {Proveniência:(De \textunderscore des...\textunderscore  + \textunderscore encalacrar\textunderscore )}
\end{itemize}
Tirar de difficuldades, de dívidas.
\section{Desencalhar}
\begin{itemize}
\item {Grp. gram.:v. t.}
\end{itemize}
\begin{itemize}
\item {Proveniência:(De \textunderscore des...\textunderscore  + \textunderscore encalhar\textunderscore )}
\end{itemize}
Tirar do encalhe, desimpedir: \textunderscore desencalhar um navio\textunderscore .
\section{Desencalhe}
\begin{itemize}
\item {Grp. gram.:m.}
\end{itemize}
Acto ou effeito de desencalhar.
\section{Desencalho}
\begin{itemize}
\item {Grp. gram.:m.}
\end{itemize}
(V.desencalhe)
\section{Desencalmadamente}
\begin{itemize}
\item {Grp. gram.:adv.}
\end{itemize}
\begin{itemize}
\item {Utilização:Fig.}
\end{itemize}
\begin{itemize}
\item {Proveniência:(De \textunderscore desencalmar\textunderscore )}
\end{itemize}
Sem calor.
Francamente. Cf. Aulegrafia, 31.
\section{Desencalmar}
\begin{itemize}
\item {Grp. gram.:v. t.}
\end{itemize}
\begin{itemize}
\item {Utilização:Fig.}
\end{itemize}
\begin{itemize}
\item {Proveniência:(De \textunderscore des...\textunderscore  + \textunderscore encalmar\textunderscore )}
\end{itemize}
Tirar a calma a.
Refrescar.
Tranquillizar.
\section{Desencambar}
\begin{itemize}
\item {Grp. gram.:v. t.}
\end{itemize}
Tirar (alguma coisa) da cambada.
Tirar o cambão de.
\section{Desencaminhadamente}
\begin{itemize}
\item {Grp. gram.:adv.}
\end{itemize}
\begin{itemize}
\item {Proveniência:(De \textunderscore desencaminhar\textunderscore )}
\end{itemize}
Com descaminho.
\section{Desencaminhador}
\begin{itemize}
\item {Grp. gram.:adj.}
\end{itemize}
\begin{itemize}
\item {Grp. gram.:M.}
\end{itemize}
\begin{itemize}
\item {Proveniência:(De \textunderscore des...\textunderscore  + \textunderscore encaminhador\textunderscore )}
\end{itemize}
Que desencaminha.
Aquelle que desencaminha.
\section{Desencaminhamento}
\begin{itemize}
\item {Grp. gram.:m.}
\end{itemize}
(V.descaminho)
\section{Desencaminhar}
\begin{itemize}
\item {Grp. gram.:v. t.}
\end{itemize}
\begin{itemize}
\item {Utilização:Fig.}
\end{itemize}
\begin{itemize}
\item {Proveniência:(De \textunderscore des...\textunderscore  + \textunderscore encaminhar\textunderscore )}
\end{itemize}
Tirar do caminho.
Perverter: \textunderscore desencaminhar uma criança\textunderscore .
Alliciar para o mal.
Subtrahir ao pagamento de direitos aduaneiros: \textunderscore desencaminhar mercadorias\textunderscore .
\section{Desencamisar}
\begin{itemize}
\item {Grp. gram.:v. t.}
\end{itemize}
\begin{itemize}
\item {Utilização:Pop.}
\end{itemize}
O mesmo que \textunderscore descamisar\textunderscore .
\section{Desencampar}
\begin{itemize}
\item {Grp. gram.:v. t.}
\end{itemize}
\begin{itemize}
\item {Proveniência:(De \textunderscore des...\textunderscore  + \textunderscore encampar\textunderscore )}
\end{itemize}
Receber (aquillo que estava encampado).
\section{Desencanar}
\begin{itemize}
\item {Grp. gram.:v. t.}
\end{itemize}
\begin{itemize}
\item {Proveniência:(De \textunderscore des...\textunderscore  + \textunderscore encanar\textunderscore )}
\end{itemize}
Tirar do cano.
\section{Desencanastrar}
\begin{itemize}
\item {Grp. gram.:v. t.}
\end{itemize}
\begin{itemize}
\item {Proveniência:(De \textunderscore des...\textunderscore  + \textunderscore encanastrar\textunderscore )}
\end{itemize}
Tirar da canastra.
Desentrançar.
\section{Desencantação}
\begin{itemize}
\item {Grp. gram.:f.}
\end{itemize}
Acto de desencantar.
\section{Desencantador}
\begin{itemize}
\item {Grp. gram.:adj.}
\end{itemize}
\begin{itemize}
\item {Grp. gram.:M.}
\end{itemize}
Que desencanta.
Aquelle que desencanta.
\section{Desencantamento}
\begin{itemize}
\item {Grp. gram.:m.}
\end{itemize}
Acto ou effeito de desencantar.
\section{Desencantar}
\begin{itemize}
\item {Grp. gram.:v. t.}
\end{itemize}
\begin{itemize}
\item {Utilização:Fam.}
\end{itemize}
\begin{itemize}
\item {Proveniência:(De \textunderscore des...\textunderscore  + \textunderscore encantar\textunderscore )}
\end{itemize}
Tirar o encanto a.
Desenganar.
Achar ou encontrar (aquillo que estava perdido ou era diffícil de encontrar).
\section{Desencanto}
\begin{itemize}
\item {Grp. gram.:m.}
\end{itemize}
O mesmo que \textunderscore desencantamento\textunderscore .
\section{Desencantoar}
\begin{itemize}
\item {Grp. gram.:v. t.}
\end{itemize}
\begin{itemize}
\item {Utilização:Des.}
\end{itemize}
\begin{itemize}
\item {Proveniência:(De \textunderscore des...\textunderscore  + \textunderscore encantoar\textunderscore )}
\end{itemize}
Tirar do canto.
Patentear.
\section{Desencanudar}
\begin{itemize}
\item {Grp. gram.:v. t.}
\end{itemize}
\begin{itemize}
\item {Proveniência:(De \textunderscore des...\textunderscore  + \textunderscore encanudar\textunderscore )}
\end{itemize}
Alisar (aquillo que estava encanudado).
\section{Desencaparar}
\begin{itemize}
\item {Grp. gram.:v. t.}
\end{itemize}
Soltar?:«\textunderscore depois de desencaparado o grande e ligeiro nebri...\textunderscore »Usque, \textunderscore Tribulações\textunderscore , VI, v.^o
\section{Desencapelar}
\begin{itemize}
\item {Grp. gram.:v. t.}
\end{itemize}
\begin{itemize}
\item {Grp. gram.:V. i.}
\end{itemize}
\begin{itemize}
\item {Proveniência:(De \textunderscore des...\textunderscore  + \textunderscore encapelar\textunderscore )}
\end{itemize}
Tirar o capelo a.
Tirar do calcês do mastro (os cabos que estavam encapelados).
Serenar.
Sair das ondas encapeladas.
\section{Desencapellar}
\begin{itemize}
\item {Grp. gram.:v. t.}
\end{itemize}
\begin{itemize}
\item {Grp. gram.:V. i.}
\end{itemize}
\begin{itemize}
\item {Proveniência:(De \textunderscore des...\textunderscore  + \textunderscore encapellar\textunderscore )}
\end{itemize}
Tirar o capello a.
Tirar do calcês do mastro (os cabos que estavam encapellados).
Serenar.
Sair das ondas encapelladas.
\section{Desencapoeirar}
\begin{itemize}
\item {Grp. gram.:v. t.}
\end{itemize}
\begin{itemize}
\item {Utilização:Ext.}
\end{itemize}
\begin{itemize}
\item {Proveniência:(De \textunderscore des...\textunderscore  + \textunderscore encapoeirar\textunderscore )}
\end{itemize}
Tirar da capoeira.
Trazer para fóra.
\section{Desencapotadamente}
\begin{itemize}
\item {Grp. gram.:adv.}
\end{itemize}
\begin{itemize}
\item {Proveniência:(De \textunderscore des...\textunderscore  + \textunderscore encapotadamente\textunderscore )}
\end{itemize}
Sem rebuço; claramente.
\section{Desencapotar}
\begin{itemize}
\item {Grp. gram.:v. t.}
\end{itemize}
\begin{itemize}
\item {Proveniência:(De \textunderscore des...\textunderscore  + \textunderscore encapotar\textunderscore )}
\end{itemize}
Tirar o capote a.
Descobrir.
Patentear.
\section{Desencaracolar}
\begin{itemize}
\item {Grp. gram.:v. t.}
\end{itemize}
\begin{itemize}
\item {Proveniência:(De \textunderscore des...\textunderscore  + \textunderscore encaracolar\textunderscore )}
\end{itemize}
Desenrolar, desmanchar, (caracóes ou anéis).
\section{Desencarapelar}
\begin{itemize}
\item {Grp. gram.:v. t.}
\end{itemize}
\begin{itemize}
\item {Proveniência:(De \textunderscore des...\textunderscore  + \textunderscore encarapelar\textunderscore )}
\end{itemize}
Desencrespar.
Desenredar.
\section{Desencarapinhar}
\begin{itemize}
\item {Grp. gram.:v. t.}
\end{itemize}
\begin{itemize}
\item {Proveniência:(De \textunderscore des...\textunderscore  + \textunderscore encarapinhar\textunderscore )}
\end{itemize}
Desencrespar.
\section{Desencarcerar}
\begin{itemize}
\item {Grp. gram.:v. t.}
\end{itemize}
\begin{itemize}
\item {Proveniência:(De \textunderscore des...\textunderscore  + \textunderscore encarcerar\textunderscore )}
\end{itemize}
Tirar do cárcere; libertar.
\section{Desencardimento}
\begin{itemize}
\item {Grp. gram.:m.}
\end{itemize}
Acto de desencardir.
\section{Desencardir}
\begin{itemize}
\item {Grp. gram.:v. t.}
\end{itemize}
\begin{itemize}
\item {Utilização:Fam.}
\end{itemize}
\begin{itemize}
\item {Proveniência:(De \textunderscore des...\textunderscore  + \textunderscore encardir\textunderscore )}
\end{itemize}
Limpar.
Lavar.
Tornar puro.
\section{Desencarecer}
\begin{itemize}
\item {Grp. gram.:v. t.  e  i.}
\end{itemize}
\begin{itemize}
\item {Proveniência:(De \textunderscore des...\textunderscore  + \textunderscore encarecer\textunderscore )}
\end{itemize}
Depreciar; aviltar.
\section{Desencarquilhar}
\begin{itemize}
\item {Grp. gram.:v. t.}
\end{itemize}
\begin{itemize}
\item {Proveniência:(De \textunderscore des...\textunderscore  + \textunderscore encarquilhar\textunderscore )}
\end{itemize}
Tirar as rugas a.
Alisar.
\section{Desencarrancar}
\begin{itemize}
\item {Grp. gram.:v. t.}
\end{itemize}
Tirar a carranca a.
Fazer que deixe de estar carrancudo. Cf. Arn. Gama, \textunderscore Última Dona\textunderscore , 284.
\section{Desencarrar}
\textunderscore v. t.\textunderscore  (e der.)
O mesmo que \textunderscore descarrar\textunderscore , etc.
\section{Desencarregamento}
\begin{itemize}
\item {Grp. gram.:m.}
\end{itemize}
Acto ou effeito de desencarregar.
\section{Desencarregar}
\begin{itemize}
\item {Grp. gram.:v. t.}
\end{itemize}
\begin{itemize}
\item {Proveniência:(De \textunderscore des...\textunderscore  + \textunderscore encarregar\textunderscore )}
\end{itemize}
Desobrigar de encargo, culpa, etc.
Alliviar.
Destituir de emprêgo.
\section{Desencarreirar}
\begin{itemize}
\item {Grp. gram.:v. t.}
\end{itemize}
\begin{itemize}
\item {Proveniência:(De \textunderscore des...\textunderscore  + \textunderscore encarreirar\textunderscore )}
\end{itemize}
O mesmo que \textunderscore desencaminhar\textunderscore .
\section{Desencarretar}
\begin{itemize}
\item {Grp. gram.:v. t.}
\end{itemize}
\begin{itemize}
\item {Proveniência:(De \textunderscore des...\textunderscore  + \textunderscore encarretar\textunderscore )}
\end{itemize}
Tirar da carrêta (a peça de artilharia).
\section{Desencarrilar}
\begin{itemize}
\item {Grp. gram.:v. t.}
\end{itemize}
(V. descarrilar)
\section{Desencarrilhar}
\begin{itemize}
\item {Grp. gram.:v. t.}
\end{itemize}
(V. descarrilar)
\section{Desencartar}
\begin{itemize}
\item {Grp. gram.:v. t.}
\end{itemize}
\begin{itemize}
\item {Proveniência:(De \textunderscore des...\textunderscore  + \textunderscore encartar\textunderscore )}
\end{itemize}
Tirar o encarte a.
Destituir de emprêgo, em que havia encarte.
\section{Desencasacar-se}
\begin{itemize}
\item {Grp. gram.:v. p.}
\end{itemize}
\begin{itemize}
\item {Proveniência:(De \textunderscore des...\textunderscore  + \textunderscore encasacar-se\textunderscore )}
\end{itemize}
Tirar a casaca.
Pôr-se á vontade.
\section{Desencasar}
\begin{itemize}
\item {Grp. gram.:v. t.}
\end{itemize}
\begin{itemize}
\item {Proveniência:(De \textunderscore des...\textunderscore  + \textunderscore encasar\textunderscore )}
\end{itemize}
Tirar da casa, da encarna, do encaixe.
\section{Desencascar}
\begin{itemize}
\item {Grp. gram.:v. t.}
\end{itemize}
O mesmo que \textunderscore desencardir\textunderscore .
\section{Desencascar}
\begin{itemize}
\item {Grp. gram.:v. t.}
\end{itemize}
\begin{itemize}
\item {Proveniência:(De \textunderscore des...\textunderscore  + \textunderscore casco\textunderscore )}
\end{itemize}
Tirar do casco, da pipa ou do tonel, (o líquido que encerram).
\section{Desencasquetar}
\begin{itemize}
\item {Grp. gram.:v. t.}
\end{itemize}
\begin{itemize}
\item {Utilização:Fam.}
\end{itemize}
\begin{itemize}
\item {Proveniência:(De \textunderscore des...\textunderscore  + \textunderscore encasquetar\textunderscore )}
\end{itemize}
Dissuadir.
Tirar da cabeça.
\section{Desencastelador}
\begin{itemize}
\item {Grp. gram.:m.}
\end{itemize}
\begin{itemize}
\item {Proveniência:(De \textunderscore desencastelar\textunderscore )}
\end{itemize}
Instrumento de ferrador, o mesmo que \textunderscore platinópode\textunderscore .
\section{Desencastelar}
\begin{itemize}
\item {Grp. gram.:v. t.}
\end{itemize}
\begin{itemize}
\item {Proveniência:(De \textunderscore des...\textunderscore  + \textunderscore encastelar\textunderscore )}
\end{itemize}
Expulsar do castelo.
Desmanchar castelos a.
\section{Desencastellador}
\begin{itemize}
\item {Grp. gram.:m.}
\end{itemize}
\begin{itemize}
\item {Proveniência:(De \textunderscore desencastellar\textunderscore )}
\end{itemize}
Instrumento de ferrador, o mesmo que \textunderscore platynópode\textunderscore .
\section{Desencastellar}
\begin{itemize}
\item {Grp. gram.:v. t.}
\end{itemize}
\begin{itemize}
\item {Proveniência:(De \textunderscore des...\textunderscore  + \textunderscore encastellar\textunderscore )}
\end{itemize}
Expulsar do castello.
Desmanchar castellos a.
\section{Desencastoar}
\begin{itemize}
\item {Grp. gram.:v. t.}
\end{itemize}
\begin{itemize}
\item {Proveniência:(De \textunderscore des...\textunderscore  + \textunderscore encastoar\textunderscore )}
\end{itemize}
Tirar o castão a.
Tirar do engaste.
\section{Desencatarroar}
\begin{itemize}
\item {Grp. gram.:v. t.}
\end{itemize}
\begin{itemize}
\item {Proveniência:(De \textunderscore des...\textunderscore  + \textunderscore encatarroar\textunderscore )}
\end{itemize}
Curar o catarro a.
\section{Desencavalgar}
\begin{itemize}
\item {Grp. gram.:v. t.}
\end{itemize}
(V. descavalgar)
\section{Desencavernar}
\begin{itemize}
\item {Grp. gram.:v. t.}
\end{itemize}
\begin{itemize}
\item {Proveniência:(De \textunderscore des...\textunderscore  + \textunderscore encavernar\textunderscore )}
\end{itemize}
Tirar da caverna; desencovilar. Cf. Alves Mendes, \textunderscore Discursos\textunderscore , 44.
\section{Desencavilhar}
\begin{itemize}
\item {Grp. gram.:v. t.}
\end{itemize}
\begin{itemize}
\item {Proveniência:(De \textunderscore des...\textunderscore  + \textunderscore encavilhar\textunderscore )}
\end{itemize}
Desunir (aquillo que estava encavilhado).
\section{Desencepar}
\begin{itemize}
\item {Grp. gram.:v. t.}
\end{itemize}
\begin{itemize}
\item {Proveniência:(De \textunderscore des...\textunderscore  + \textunderscore encepar\textunderscore )}
\end{itemize}
Tirar do cepo.
\section{Desencerar}
\begin{itemize}
\item {Grp. gram.:v. t.}
\end{itemize}
\begin{itemize}
\item {Proveniência:(De \textunderscore des...\textunderscore  + \textunderscore encerar\textunderscore )}
\end{itemize}
Tirar a cera que reveste.
\section{Desencerramento}
\begin{itemize}
\item {Grp. gram.:m.}
\end{itemize}
Acto ou effeito de desencerrar.
\section{Desencerrar}
\begin{itemize}
\item {Grp. gram.:v. t.}
\end{itemize}
\begin{itemize}
\item {Utilização:Fig.}
\end{itemize}
\begin{itemize}
\item {Proveniência:(De \textunderscore des...\textunderscore  + \textunderscore encerrar\textunderscore )}
\end{itemize}
Soltar do encêrro.
Libertar.
Descobrir, patentear.
\section{Desencharcar}
\begin{itemize}
\item {Grp. gram.:v. t.}
\end{itemize}
\begin{itemize}
\item {Proveniência:(De \textunderscore des\textunderscore .. + \textunderscore encharcar\textunderscore )}
\end{itemize}
Tirar do charco.
Enxugar.
\section{Desenchavetadeira}
\begin{itemize}
\item {Grp. gram.:f.}
\end{itemize}
Utensílio de ferreiro, para desenchavetar.
\section{Desenchavetar}
\begin{itemize}
\item {Grp. gram.:v. t.}
\end{itemize}
\begin{itemize}
\item {Proveniência:(De \textunderscore enchavetar\textunderscore )}
\end{itemize}
Tirar chaveta ou chavetas a.
\section{Desencher}
\begin{itemize}
\item {Grp. gram.:v. t.}
\end{itemize}
\begin{itemize}
\item {Proveniência:(De \textunderscore des...\textunderscore  + \textunderscore encher\textunderscore )}
\end{itemize}
Despejar, esvasiar.
\section{Desencilhar}
\begin{itemize}
\item {Grp. gram.:v. t.}
\end{itemize}
\begin{itemize}
\item {Proveniência:(De \textunderscore des...\textunderscore  + \textunderscore encilhar\textunderscore )}
\end{itemize}
Tirar a cilha ou os arreios a.
\section{Desenclaustrar}
\begin{itemize}
\item {Grp. gram.:v. t.}
\end{itemize}
\begin{itemize}
\item {Proveniência:(De \textunderscore des...\textunderscore  + \textunderscore claustrar\textunderscore )}
\end{itemize}
Tirar do claustro.
\section{Desenclavinhar}
\begin{itemize}
\item {Grp. gram.:v. t.}
\end{itemize}
\begin{itemize}
\item {Proveniência:(De \textunderscore des...\textunderscore  + \textunderscore enclavinhar\textunderscore )}
\end{itemize}
Desimpedir ou destravar (aquillo que estava enclavinhado). Cf. Castilho, \textunderscore D. Quixote\textunderscore , I, 360.
\section{Desencobrir}
\begin{itemize}
\item {Grp. gram.:v. t.}
\end{itemize}
\begin{itemize}
\item {Proveniência:(De \textunderscore des...\textunderscore  + \textunderscore encobrir\textunderscore )}
\end{itemize}
O mesmo que \textunderscore descobrir\textunderscore .
\section{Desencofrar}
\begin{itemize}
\item {Grp. gram.:v. t.}
\end{itemize}
\begin{itemize}
\item {Utilização:Neol.}
\end{itemize}
\begin{itemize}
\item {Proveniência:(De \textunderscore des...\textunderscore  + \textunderscore encofrar\textunderscore )}
\end{itemize}
Tirar de um cofre.
\section{Desencoifar}
\begin{itemize}
\item {Grp. gram.:v. t}
\end{itemize}
\begin{itemize}
\item {Proveniência:(De \textunderscore des...\textunderscore  + \textunderscore encoifar\textunderscore )}
\end{itemize}
Tirar a coifa a.
Tirar o capello de pergaminho, que resguarda a espoleta e preserva a escorva de (peça de artilharia).
\section{Desencolar}
\begin{itemize}
\item {Grp. gram.:v.}
\end{itemize}
\begin{itemize}
\item {Utilização:t. Carp.}
\end{itemize}
Desbastar a borda de (uma tábua).
(Cp. \textunderscore colar\textunderscore ^1)
\section{Desencolerizar}
\begin{itemize}
\item {Grp. gram.:v. t.}
\end{itemize}
\begin{itemize}
\item {Proveniência:(De \textunderscore des...\textunderscore  + \textunderscore encolerizar\textunderscore )}
\end{itemize}
Amansar.
Fazer passar a cólera a.
\section{Desencolher}
\begin{itemize}
\item {Grp. gram.:v. t.}
\end{itemize}
\begin{itemize}
\item {Utilização:Fig.}
\end{itemize}
\begin{itemize}
\item {Proveniência:(De \textunderscore des...\textunderscore  + \textunderscore encolher\textunderscore )}
\end{itemize}
Estender.
Tirar o acanhamento a.
\section{Desencolhimento}
\begin{itemize}
\item {Grp. gram.:m.}
\end{itemize}
Acto ou effeito de desencolher.
\section{Desencollar}
\begin{itemize}
\item {Grp. gram.:v.}
\end{itemize}
\begin{itemize}
\item {Utilização:t. Carp.}
\end{itemize}
Desbastar a borda de (uma tábua).
(Cp. \textunderscore collar\textunderscore ^1)
\section{Desencommendar}
\begin{itemize}
\item {Grp. gram.:v. t.}
\end{itemize}
\begin{itemize}
\item {Proveniência:(De \textunderscore des...\textunderscore  + \textunderscore encommendar\textunderscore )}
\end{itemize}
Avisar ou ordenar que se não faça (aquillo que estava encommendado).
\section{Desenconchar}
\begin{itemize}
\item {Grp. gram.:v. t.}
\end{itemize}
\begin{itemize}
\item {Utilização:Fig.}
\end{itemize}
\begin{itemize}
\item {Proveniência:(De \textunderscore des...\textunderscore  + \textunderscore enconchar\textunderscore )}
\end{itemize}
Tirar da concha.
Soltar, libertar.
Fazer saír (aquelle ou aquillo que estava occulto).
\section{Desencontrado}
\begin{itemize}
\item {Grp. gram.:adj.}
\end{itemize}
\begin{itemize}
\item {Proveniência:(De \textunderscore desencontrar\textunderscore )}
\end{itemize}
Que vai em direcção opposta á de outrem.
\section{Desencontrar}
\begin{itemize}
\item {Grp. gram.:v. t.}
\end{itemize}
\begin{itemize}
\item {Grp. gram.:V. p.}
\end{itemize}
\begin{itemize}
\item {Proveniência:(De \textunderscore des...\textunderscore  + \textunderscore encontrar\textunderscore )}
\end{itemize}
Fazer que (dois ou mais individuos) sigam caminhos diversos, não se encontrando.
Não se encontrar com outrem.
Discordar.
\section{Desencontro}
\begin{itemize}
\item {Grp. gram.:m.}
\end{itemize}
Acto ou effeito de desencontrar.
\section{Desencordoar}
\begin{itemize}
\item {Grp. gram.:v. t.}
\end{itemize}
\begin{itemize}
\item {Utilização:Pop.}
\end{itemize}
\begin{itemize}
\item {Proveniência:(De \textunderscore des...\textunderscore  + \textunderscore encordoar\textunderscore )}
\end{itemize}
Tirar as cordas a.
Desamuar-se.
\section{Desencorpar}
\begin{itemize}
\item {Grp. gram.:v. t.}
\end{itemize}
\begin{itemize}
\item {Proveniência:(De \textunderscore des...\textunderscore  + \textunderscore encorpar\textunderscore )}
\end{itemize}
Fazer deminuir em corpo ou volume.
\section{Desencorporação}
\begin{itemize}
\item {Grp. gram.:f.}
\end{itemize}
Acto ou effeito de desencorporar.
\section{Desencorporar}
\begin{itemize}
\item {Grp. gram.:v. t.}
\end{itemize}
\begin{itemize}
\item {Proveniência:(De \textunderscore des...\textunderscore  + \textunderscore encorporar\textunderscore )}
\end{itemize}
Tirar de uma corporação.
Separar (aquelle ou aquillo que estava encorporado).
\section{Desencorrear}
\begin{itemize}
\item {Grp. gram.:v. t.}
\end{itemize}
\begin{itemize}
\item {Grp. gram.:V. i.}
\end{itemize}
\begin{itemize}
\item {Proveniência:(De \textunderscore des...\textunderscore  + \textunderscore encorrear\textunderscore )}
\end{itemize}
Soltar (aquillo que estava atado com correia).
Perder a rijeza própria da correia ou do coiro.
\section{Desencortiçar}
\begin{itemize}
\item {Grp. gram.:v. t.}
\end{itemize}
\begin{itemize}
\item {Utilização:Fig.}
\end{itemize}
\begin{itemize}
\item {Proveniência:(De \textunderscore des...\textunderscore  + \textunderscore encortiçar\textunderscore )}
\end{itemize}
Desenrugar.
\section{Desencoscorar}
\begin{itemize}
\item {Grp. gram.:v. t.}
\end{itemize}
\begin{itemize}
\item {Proveniência:(De \textunderscore des...\textunderscore  + \textunderscore encoscorar\textunderscore )}
\end{itemize}
Desencrespar.
Tirar a crósta de.
Desencortiçar.
\section{Desencostar}
\begin{itemize}
\item {Grp. gram.:v. t.}
\end{itemize}
\begin{itemize}
\item {Proveniência:(De \textunderscore des...\textunderscore  + \textunderscore encostar\textunderscore )}
\end{itemize}
Desviar do encôsto; endireitar.
\section{Desencovador}
\begin{itemize}
\item {Grp. gram.:adj.}
\end{itemize}
Que desencova; que faz saír da cova. Cf. \textunderscore Hist. Insulana\textunderscore , II, 82.
\section{Desencovar}
\begin{itemize}
\item {Grp. gram.:v. t.}
\end{itemize}
\begin{itemize}
\item {Proveniência:(De \textunderscore des...\textunderscore  + \textunderscore encovar\textunderscore )}
\end{itemize}
Tirar da cova.
Patentear.
\section{Desencovilar}
\begin{itemize}
\item {Grp. gram.:v. t.}
\end{itemize}
\begin{itemize}
\item {Proveniência:(De \textunderscore des...\textunderscore  + \textunderscore encovilar\textunderscore )}
\end{itemize}
Tirar de covil. Cf. Castilho, \textunderscore D. Quixote\textunderscore , I, 195.
\section{Desencravar}
\begin{itemize}
\item {Grp. gram.:v. t.}
\end{itemize}
\begin{itemize}
\item {Proveniência:(De \textunderscore des...\textunderscore  + \textunderscore encravar\textunderscore )}
\end{itemize}
Despregar.
Tirar da ferradura cravo, pedra, etc., que embaraça os movimentos a (um cavallo).
Separar da carne (as unhas nellas cravadas).
\section{Desencravilhar}
\begin{itemize}
\item {Grp. gram.:v. t.}
\end{itemize}
\begin{itemize}
\item {Utilização:Pop.}
\end{itemize}
\begin{itemize}
\item {Proveniência:(De \textunderscore des...\textunderscore  + \textunderscore encravilhar\textunderscore )}
\end{itemize}
Desencravar.
Desentalar.
Desencalacrar.
\section{Desencrespar}
\begin{itemize}
\item {Grp. gram.:v. t.}
\end{itemize}
\begin{itemize}
\item {Utilização:Fig.}
\end{itemize}
\begin{itemize}
\item {Proveniência:(De \textunderscore des...\textunderscore  + \textunderscore encrespar\textunderscore )}
\end{itemize}
Tirar o encrespamento a.
Alisar; amaciar.
Desencapellar.
\section{Desencruar}
\begin{itemize}
\item {Grp. gram.:v. t.}
\end{itemize}
\begin{itemize}
\item {Proveniência:(De \textunderscore des...\textunderscore  + \textunderscore encruar\textunderscore )}
\end{itemize}
Fazer perder a qualidade de encruado a. Cf. Castilho, \textunderscore Fastos\textunderscore , III, 41.
\section{Desencruzar}
\begin{itemize}
\item {Grp. gram.:v. t.}
\end{itemize}
O mesmo que \textunderscore descruzar\textunderscore .
\section{Desencurralar}
\begin{itemize}
\item {Grp. gram.:v. t.}
\end{itemize}
\begin{itemize}
\item {Proveniência:(De \textunderscore des...\textunderscore  + \textunderscore encurralar\textunderscore )}
\end{itemize}
Soltar do curral.
Desalojar.
Desencantoar.
\section{Desencurvar}
\begin{itemize}
\item {Grp. gram.:v. t.}
\end{itemize}
\begin{itemize}
\item {Proveniência:(De \textunderscore des...\textunderscore  + \textunderscore encurvar\textunderscore )}
\end{itemize}
Tornar direito (aquillo que era curvo).
\section{Desende}
\begin{itemize}
\item {Grp. gram.:adv.}
\end{itemize}
\begin{itemize}
\item {Utilização:Ant.}
\end{itemize}
Dalli; daquelle lugar.
\section{Desendemoninhar}
\begin{itemize}
\item {Grp. gram.:v. t.}
\end{itemize}
\begin{itemize}
\item {Utilização:Fig.}
\end{itemize}
\begin{itemize}
\item {Proveniência:(De \textunderscore des...\textunderscore  + \textunderscore endemoninhar\textunderscore )}
\end{itemize}
Tirar o demónio do corpo de.
Desencolerizar.
\section{Desendeusar}
\begin{itemize}
\item {Grp. gram.:v. t.}
\end{itemize}
\begin{itemize}
\item {Proveniência:(De \textunderscore des...\textunderscore  + \textunderscore endeusar\textunderscore )}
\end{itemize}
Rejeitar a apothéose de.
Não reconhecer o carácter divino de.
Privar dêsse carácter.
Recusar adoração a.
\section{Desendividar}
\begin{itemize}
\item {Grp. gram.:v. t.}
\end{itemize}
\begin{itemize}
\item {Proveniência:(De \textunderscore des...\textunderscore  + \textunderscore endividar\textunderscore )}
\end{itemize}
Solver dívida ou dívidas de.
Desobrigar.
Dar quitação a.
\section{Desenervação}
\begin{itemize}
\item {Grp. gram.:f.}
\end{itemize}
Acto de desenervar.
\section{Desenervar}
\begin{itemize}
\item {Grp. gram.:v. t.}
\end{itemize}
\begin{itemize}
\item {Proveniência:(De \textunderscore des...\textunderscore  + \textunderscore enervar\textunderscore )}
\end{itemize}
Tirar a enervação a.
Tonificar.
\section{Desenfadadamente}
\begin{itemize}
\item {Grp. gram.:adv.}
\end{itemize}
Com desenfado.
\section{Desenfadadiço}
\begin{itemize}
\item {Grp. gram.:adj.}
\end{itemize}
\begin{itemize}
\item {Proveniência:(De \textunderscore des...\textunderscore  + \textunderscore enfadadiço\textunderscore )}
\end{itemize}
Que desenfada:«\textunderscore a preguiça é mais desenfadadiça que bom pomar\textunderscore ». G. Vicente, \textunderscore Juiz da Beira\textunderscore .
\section{Desenfadamento}
\begin{itemize}
\item {Grp. gram.:m.}
\end{itemize}
O mesmo que \textunderscore desenfado\textunderscore .
\section{Desenfadar}
\begin{itemize}
\item {Grp. gram.:v. t.}
\end{itemize}
\begin{itemize}
\item {Proveniência:(De \textunderscore des...\textunderscore  + \textunderscore enfadar\textunderscore )}
\end{itemize}
Tirar o enfado a.
Distrahir.
Divertir.
\section{Desenfado}
\begin{itemize}
\item {Grp. gram.:m.}
\end{itemize}
Acto ou effeito de desenfadar.
\section{Desenfaixar}
\begin{itemize}
\item {Grp. gram.:v. t.}
\end{itemize}
\begin{itemize}
\item {Proveniência:(De \textunderscore des...\textunderscore  + \textunderscore enfaixar\textunderscore )}
\end{itemize}
Tirar as faixas a.
\section{Desenfardar}
\begin{itemize}
\item {Grp. gram.:v. t.}
\end{itemize}
\begin{itemize}
\item {Proveniência:(De \textunderscore des...\textunderscore  + \textunderscore enfardar\textunderscore )}
\end{itemize}
Soldar ou tirar para fóra (aquillo que estava enfardado).
Tirar do fardo.
\section{Desenfardelar}
\begin{itemize}
\item {Grp. gram.:v. t.}
\end{itemize}
\begin{itemize}
\item {Utilização:Fig.}
\end{itemize}
\begin{itemize}
\item {Proveniência:(De \textunderscore des...\textunderscore  + \textunderscore enfardelar\textunderscore )}
\end{itemize}
Tirar do fardel.
Tirar do saco.
Patentear.
\section{Desenfardo}
\begin{itemize}
\item {Grp. gram.:m.}
\end{itemize}
Acto de desenfardar.
\section{Desenfarpelar}
\begin{itemize}
\item {Grp. gram.:v. t.}
\end{itemize}
\begin{itemize}
\item {Utilização:Pop.}
\end{itemize}
\begin{itemize}
\item {Proveniência:(De \textunderscore des...\textunderscore  + \textunderscore enfarpelar\textunderscore )}
\end{itemize}
Despir ou tirar a farpela a.
\section{Desenfarruscar}
\begin{itemize}
\item {Grp. gram.:v. t.}
\end{itemize}
\begin{itemize}
\item {Proveniência:(De \textunderscore des...\textunderscore  + \textunderscore enfarruscar\textunderscore )}
\end{itemize}
Tirar farruscas a.
Limpar.
\section{Desenfartar}
\begin{itemize}
\item {Grp. gram.:v. t.}
\end{itemize}
\begin{itemize}
\item {Proveniência:(De \textunderscore des...\textunderscore  + \textunderscore enfartar\textunderscore )}
\end{itemize}
Tirar o enfarte a.
\section{Desenfastiadamente}
\begin{itemize}
\item {Grp. gram.:adv.}
\end{itemize}
Com desfastio.
\section{Desenfastiadiço}
\begin{itemize}
\item {Grp. gram.:adj.}
\end{itemize}
\begin{itemize}
\item {Proveniência:(De \textunderscore desenfastiar\textunderscore )}
\end{itemize}
Próprio para desenfastiar.
\section{Desenfastiar}
\begin{itemize}
\item {Grp. gram.:v. t.}
\end{itemize}
\begin{itemize}
\item {Utilização:Fig.}
\end{itemize}
\begin{itemize}
\item {Proveniência:(De \textunderscore des...\textunderscore  + \textunderscore enfastiar\textunderscore )}
\end{itemize}
Tirar o fastio a.
Despertar o appetite em.
Tornar appetitoso.
Recrear.
Tornar ameno, suave: \textunderscore desenfastiar uma narração\textunderscore .
\section{Desenfeitar}
\begin{itemize}
\item {Grp. gram.:v. t.}
\end{itemize}
\begin{itemize}
\item {Proveniência:(De \textunderscore des...\textunderscore  + \textunderscore enfeitar\textunderscore )}
\end{itemize}
Tirar os enfeites a; desataviar.
\section{Desenfeitiçar}
\begin{itemize}
\item {Grp. gram.:v. t.}
\end{itemize}
\begin{itemize}
\item {Utilização:Fig.}
\end{itemize}
\begin{itemize}
\item {Proveniência:(De \textunderscore des...\textunderscore  + \textunderscore enfeitiçar\textunderscore )}
\end{itemize}
Livrar do feitiço; desencantar.
Livrar do amor.
\section{Desenfeixar}
\begin{itemize}
\item {Grp. gram.:v. t.}
\end{itemize}
\begin{itemize}
\item {Proveniência:(De \textunderscore des...\textunderscore  + \textunderscore enfeixar\textunderscore )}
\end{itemize}
Tirar do feixe.
Desmanchar (aquillo que estava enfeixado).
Desunir.
\section{Desenferençar}
\begin{itemize}
\item {Grp. gram.:v. t.}
\end{itemize}
\begin{itemize}
\item {Utilização:Ant.}
\end{itemize}
O mesmo que \textunderscore differençar\textunderscore . Cf. Cortesão, \textunderscore Subs.\textunderscore 
\section{Desenfermar}
\begin{itemize}
\item {Grp. gram.:v. i.}
\end{itemize}
\begin{itemize}
\item {Proveniência:(De \textunderscore des...\textunderscore  + \textunderscore enfermar\textunderscore )}
\end{itemize}
Deixar de estar enfermo.
\section{Desenferrujar}
\begin{itemize}
\item {Grp. gram.:v. t.}
\end{itemize}
\begin{itemize}
\item {Utilização:Fig.}
\end{itemize}
\begin{itemize}
\item {Proveniência:(De \textunderscore des...\textunderscore  + \textunderscore enferrujar\textunderscore )}
\end{itemize}
Tirar a ferrugem a.
Mover a (língua), falando.
\section{Desenfestado}
\begin{itemize}
\item {Grp. gram.:adj.}
\end{itemize}
Que não é enfestado.
\section{Desenfezar}
\begin{itemize}
\item {fónica:fé}
\end{itemize}
\begin{itemize}
\item {Grp. gram.:v. t.}
\end{itemize}
\begin{itemize}
\item {Utilização:Fig.}
\end{itemize}
\begin{itemize}
\item {Proveniência:(De \textunderscore des...\textunderscore  + \textunderscore enfezar\textunderscore )}
\end{itemize}
Tirar o enfezamento a.
Desenfadar; desencolerizar.
\section{Desenfiamento}
\begin{itemize}
\item {Grp. gram.:m.}
\end{itemize}
\begin{itemize}
\item {Proveniência:(De \textunderscore desenfiar\textunderscore )}
\end{itemize}
Organização de uma bataria, para tiros de enfiada.
\section{Desenfiar}
\begin{itemize}
\item {Grp. gram.:v. t.}
\end{itemize}
\begin{itemize}
\item {Proveniência:(De \textunderscore des...\textunderscore  + \textunderscore enfiar\textunderscore )}
\end{itemize}
Tirar do fio.
Desmanchar (aquillo que estava enfiado): \textunderscore desenfiar um collar de pérolas\textunderscore .
Tirar o fio de.
Guarnecer (praça ou reducto), com peças de artilharia, para tiros de enfiada.
\section{Desenfileirar}
\begin{itemize}
\item {Grp. gram.:v. t.}
\end{itemize}
\begin{itemize}
\item {Proveniência:(De \textunderscore des...\textunderscore  + \textunderscore enfileirar\textunderscore )}
\end{itemize}
Tirar da fileira.
\section{Desenflorar}
\begin{itemize}
\item {Grp. gram.:v. t.}
\end{itemize}
\begin{itemize}
\item {Grp. gram.:V. i.}
\end{itemize}
\begin{itemize}
\item {Proveniência:(De \textunderscore des...\textunderscore  + \textunderscore enflorar\textunderscore )}
\end{itemize}
Fazer caír as flôres a.
Tirar as flôres de.
Despojar-se das flôres, perder as flôres. Cf. Júl. Dinis, \textunderscore Serões\textunderscore , 236.
\section{Desenforcar}
\begin{itemize}
\item {Grp. gram.:v. t.}
\end{itemize}
\begin{itemize}
\item {Proveniência:(De \textunderscore des...\textunderscore  + \textunderscore enforcar\textunderscore )}
\end{itemize}
Desprender da forca.
\section{Desenforido}
\begin{itemize}
\item {Grp. gram.:adj.}
\end{itemize}
\begin{itemize}
\item {Utilização:Prov.}
\end{itemize}
\begin{itemize}
\item {Utilização:trasm.}
\end{itemize}
O mesmo que \textunderscore desaforido\textunderscore .
\section{Desenforjar}
\begin{itemize}
\item {Grp. gram.:v. t.}
\end{itemize}
\begin{itemize}
\item {Proveniência:(De \textunderscore des...\textunderscore  + \textunderscore enforjar\textunderscore )}
\end{itemize}
Tirar da forja.
\section{Desenfornagem}
\begin{itemize}
\item {Grp. gram.:f.}
\end{itemize}
Acto de desenfornar.
\section{Desenfornar}
\begin{itemize}
\item {Grp. gram.:v. t.}
\end{itemize}
\begin{itemize}
\item {Proveniência:(De \textunderscore des...\textunderscore  + \textunderscore enfornar\textunderscore )}
\end{itemize}
Tirar do forno.
\section{Desenfrascar}
\begin{itemize}
\item {Grp. gram.:v. t.}
\end{itemize}
\begin{itemize}
\item {Grp. gram.:V. p.}
\end{itemize}
\begin{itemize}
\item {Utilização:Pop.}
\end{itemize}
\begin{itemize}
\item {Proveniência:(De \textunderscore des...\textunderscore  + \textunderscore enfrascar\textunderscore )}
\end{itemize}
Tirar do frasco ou de frascos.
Desembebedar-se.
\section{Desenfreadamente}
\begin{itemize}
\item {Grp. gram.:adv.}
\end{itemize}
\begin{itemize}
\item {Utilização:Fig.}
\end{itemize}
\begin{itemize}
\item {Proveniência:(De \textunderscore desenfrear\textunderscore )}
\end{itemize}
Sem freio, á solta.
Com arrebatamento.
Excessivamente.
\section{Desenfreamento}
\begin{itemize}
\item {Grp. gram.:m.}
\end{itemize}
Acto ou effeito de desenfrear.
\section{Desenfrear}
\begin{itemize}
\item {Grp. gram.:v. t.}
\end{itemize}
\begin{itemize}
\item {Grp. gram.:V. p.}
\end{itemize}
\begin{itemize}
\item {Utilização:Fig.}
\end{itemize}
\begin{itemize}
\item {Proveniência:(De \textunderscore des...\textunderscore  + \textunderscore enfrear\textunderscore )}
\end{itemize}
Tirar o freio a: \textunderscore desenfrear um cavallo\textunderscore .
Soltar.
Tomar o freio nos dentes.
Arremessar-se impetuosamente.
Irritar-se.
Tornar-se dissoluto, libertino.
\section{Desenfrechar}
\begin{itemize}
\item {fónica:fré}
\end{itemize}
\begin{itemize}
\item {Grp. gram.:v. t.}
\end{itemize}
Despedir como frecha. Cf. R. Ortigão, \textunderscore Hollanda\textunderscore , 123.
\section{Desenfreio}
\begin{itemize}
\item {Grp. gram.:m.}
\end{itemize}
O mesmo que \textunderscore desenfreamento\textunderscore .
\section{Desenfrenar}
\begin{itemize}
\item {Grp. gram.:v. t.}
\end{itemize}
\begin{itemize}
\item {Utilização:Bras. do S}
\end{itemize}
O mesmo que \textunderscore desenfrear\textunderscore .
\section{Desenfronhar}
\begin{itemize}
\item {Grp. gram.:v. t.}
\end{itemize}
\begin{itemize}
\item {Utilização:Ext.}
\end{itemize}
\begin{itemize}
\item {Utilização:Fig.}
\end{itemize}
\begin{itemize}
\item {Proveniência:(De \textunderscore des...\textunderscore  + \textunderscore enfronhar\textunderscore )}
\end{itemize}
Tirar da fronha.
Despir.
Patentear.
\section{Desenfueirar}
\begin{itemize}
\item {Grp. gram.:v. t.}
\end{itemize}
\begin{itemize}
\item {Proveniência:(De \textunderscore des...\textunderscore  + \textunderscore enfueirar\textunderscore )}
\end{itemize}
Tirar os fueiros a.
\section{Desenfunar-se}
\begin{itemize}
\item {Grp. gram.:v. p.}
\end{itemize}
\begin{itemize}
\item {Utilização:Fig.}
\end{itemize}
\begin{itemize}
\item {Proveniência:(De \textunderscore des...\textunderscore  + \textunderscore enfunar\textunderscore )}
\end{itemize}
Deixar do estar enfunado.
Deixar de sêr vaidoso.
\section{Desenfurecer}
\begin{itemize}
\item {Grp. gram.:v. t.}
\end{itemize}
\begin{itemize}
\item {Proveniência:(De \textunderscore des...\textunderscore  + \textunderscore enfurecer\textunderscore )}
\end{itemize}
O mesmo que \textunderscore desencolerizar\textunderscore .
\section{Desenfurnar}
\begin{itemize}
\item {Grp. gram.:v.}
\end{itemize}
\begin{itemize}
\item {Utilização:t. Náut.}
\end{itemize}
\begin{itemize}
\item {Proveniência:(De \textunderscore des...\textunderscore  + \textunderscore enfurnar\textunderscore )}
\end{itemize}
Tirar do seu lugar (os mastros de navio).
\section{Desengaçadamente}
\begin{itemize}
\item {Grp. gram.:adv.}
\end{itemize}
\begin{itemize}
\item {Utilização:Pop.}
\end{itemize}
\begin{itemize}
\item {Proveniência:(De \textunderscore desengaçar\textunderscore )}
\end{itemize}
Descommedidamente.
\section{Desengaçadeira}
\begin{itemize}
\item {Grp. gram.:f.}
\end{itemize}
O mesmo que \textunderscore desengaçador\textunderscore .
\section{Desengaçador}
\begin{itemize}
\item {Grp. gram.:m.}
\end{itemize}
\begin{itemize}
\item {Proveniência:(De \textunderscore desengaçar\textunderscore )}
\end{itemize}
Instrumento, com que se separam do engaço os bagos da uva.
\section{Desengaçar}
\begin{itemize}
\item {Grp. gram.:v. t.}
\end{itemize}
\begin{itemize}
\item {Utilização:Pop.}
\end{itemize}
\begin{itemize}
\item {Proveniência:(De \textunderscore des...\textunderscore  + \textunderscore engaçar\textunderscore )}
\end{itemize}
Separar do engaço (bagos de uvas).
Comer desmedidamente.
\section{Desengace}
\begin{itemize}
\item {Grp. gram.:m.}
\end{itemize}
O mesmo que \textunderscore desengaço\textunderscore .
\section{Desengaço}
\begin{itemize}
\item {Grp. gram.:m.}
\end{itemize}
Acto de desengaçar.
\section{Desengaiolar}
\begin{itemize}
\item {Grp. gram.:v. t.}
\end{itemize}
\begin{itemize}
\item {Utilização:Fig.}
\end{itemize}
\begin{itemize}
\item {Proveniência:(De \textunderscore des...\textunderscore  + \textunderscore engaiolar\textunderscore )}
\end{itemize}
Tirar da gaiola.
Soltar da prisão.
Libertar.
\section{Desengalapar}
\begin{itemize}
\item {Grp. gram.:v. i.}
\end{itemize}
\begin{itemize}
\item {Utilização:Prov.}
\end{itemize}
\begin{itemize}
\item {Utilização:trasm.}
\end{itemize}
\begin{itemize}
\item {Proveniência:(De \textunderscore des...\textunderscore  + \textunderscore engalapar\textunderscore )}
\end{itemize}
Tornar-se desempenado, nivelar-se, (falando-se de madeira).
\section{Desengalfinhar}
\begin{itemize}
\item {Grp. gram.:v. t.}
\end{itemize}
\begin{itemize}
\item {Utilização:Pop.}
\end{itemize}
\begin{itemize}
\item {Proveniência:(De \textunderscore des...\textunderscore  + \textunderscore engalfinhar\textunderscore )}
\end{itemize}
Separar (quem estava engalfinhado).
\section{Desengaliar-se}
\begin{itemize}
\item {Grp. gram.:v. p.}
\end{itemize}
\begin{itemize}
\item {Utilização:Prov.}
\end{itemize}
\begin{itemize}
\item {Utilização:trasm.}
\end{itemize}
\begin{itemize}
\item {Proveniência:(De \textunderscore des...\textunderscore  + \textunderscore engaliar-se\textunderscore )}
\end{itemize}
Desagarrar-se na luta.
\section{Desenganadamente}
\begin{itemize}
\item {Grp. gram.:adv.}
\end{itemize}
\begin{itemize}
\item {Proveniência:(De \textunderscore desenganar\textunderscore )}
\end{itemize}
Francamente.
\section{Desenganador}
\begin{itemize}
\item {Grp. gram.:adj.}
\end{itemize}
\begin{itemize}
\item {Grp. gram.:M.}
\end{itemize}
Que desengana.
Aquelle que desengana.
\section{Desenganar}
\begin{itemize}
\item {Grp. gram.:v. t.}
\end{itemize}
\begin{itemize}
\item {Proveniência:(De \textunderscore des...\textunderscore  + \textunderscore enganar\textunderscore )}
\end{itemize}
Tirar do engano.
Desilludir.
Tirar esperanças a.
\section{Desenganchar}
\begin{itemize}
\item {Grp. gram.:v. t.}
\end{itemize}
\begin{itemize}
\item {Proveniência:(De \textunderscore des...\textunderscore  + \textunderscore enganchar\textunderscore )}
\end{itemize}
Separar, soltar (aquillo que estava preso por gancho).
\section{Desengano}
\begin{itemize}
\item {Grp. gram.:m.}
\end{itemize}
\begin{itemize}
\item {Proveniência:(De \textunderscore des...\textunderscore  + \textunderscore engano\textunderscore )}
\end{itemize}
Acto ou effeito de desenganar.
Desillusão.
Franqueza.
Experiência.
\section{Desengarrafar}
\begin{itemize}
\item {Grp. gram.:v. t.}
\end{itemize}
\begin{itemize}
\item {Proveniência:(De \textunderscore des...\textunderscore  + \textunderscore engarrafar\textunderscore )}
\end{itemize}
Tirar da garrafa.
\section{Desengasgar}
\begin{itemize}
\item {Grp. gram.:v. t.}
\end{itemize}
\begin{itemize}
\item {Proveniência:(De \textunderscore des...\textunderscore  + \textunderscore engasgar\textunderscore )}
\end{itemize}
Tirar o engasgo a.
\section{Desengasgue}
\begin{itemize}
\item {Grp. gram.:m.}
\end{itemize}
Acto de desengasgar.
\section{Desengastalhar}
\begin{itemize}
\item {Grp. gram.:v. t.}
\end{itemize}
\begin{itemize}
\item {Proveniência:(De \textunderscore des...\textunderscore  + \textunderscore engastalhar\textunderscore )}
\end{itemize}
Tirar o gastalho de.
\section{Desengastar}
\begin{itemize}
\item {Grp. gram.:v. t.}
\end{itemize}
\begin{itemize}
\item {Proveniência:(De \textunderscore des...\textunderscore  + \textunderscore engastar\textunderscore )}
\end{itemize}
Tirar do engaste.
\section{Desengatar}
\begin{itemize}
\item {Grp. gram.:v. t.}
\end{itemize}
\begin{itemize}
\item {Proveniência:(De \textunderscore des...\textunderscore  + \textunderscore engatar\textunderscore )}
\end{itemize}
Soltar do engate.
\section{Desengatilhar}
\begin{itemize}
\item {Grp. gram.:v. t.}
\end{itemize}
\begin{itemize}
\item {Proveniência:(De \textunderscore des...\textunderscore  + \textunderscore engatilhar\textunderscore )}
\end{itemize}
Desfechar, disparar: \textunderscore desengatilhar a pistola\textunderscore .
\section{Desengenhosamente}
\begin{itemize}
\item {Grp. gram.:adv.}
\end{itemize}
De modo desengenhoso.
Sem engenho.
\section{Desengenhoso}
\begin{itemize}
\item {Grp. gram.:adj.}
\end{itemize}
\begin{itemize}
\item {Proveniência:(De \textunderscore des...\textunderscore  + \textunderscore engenhoso\textunderscore )}
\end{itemize}
Que não tem engenho.
Em que não há engenho.
\section{Desenglobar}
\begin{itemize}
\item {Grp. gram.:v. t.}
\end{itemize}
\begin{itemize}
\item {Proveniência:(De \textunderscore des...\textunderscore  + \textunderscore englobar\textunderscore )}
\end{itemize}
Separar (aquillo que estava englobado).
\section{Desengodar}
\begin{itemize}
\item {Grp. gram.:v. t.}
\end{itemize}
\begin{itemize}
\item {Utilização:Fig.}
\end{itemize}
\begin{itemize}
\item {Proveniência:(De \textunderscore des...\textunderscore  + \textunderscore engodar\textunderscore )}
\end{itemize}
Tirar o engôdo a.
Desilludir.
\section{Desengolfar}
\begin{itemize}
\item {Grp. gram.:v. t.}
\end{itemize}
\begin{itemize}
\item {Proveniência:(De \textunderscore des...\textunderscore  + \textunderscore engolfar\textunderscore )}
\end{itemize}
Tirar do golfo.
Livrar do abysmo.
\section{Desengomar}
\begin{itemize}
\item {Grp. gram.:v. t.}
\end{itemize}
\begin{itemize}
\item {Utilização:Gír. de gatunos.}
\end{itemize}
\begin{itemize}
\item {Utilização:Gír. lisb.}
\end{itemize}
\begin{itemize}
\item {Grp. gram.:V. i.}
\end{itemize}
\begin{itemize}
\item {Utilização:Gír. de gatunos.}
\end{itemize}
\begin{itemize}
\item {Proveniência:(De \textunderscore des...\textunderscore  + \textunderscore engomar\textunderscore )}
\end{itemize}
Tirar a goma a.
Furtar (cadeia de relógio).
Abrir, descerrar, (porta, gaveta, etc.).
Tirar o botão da respectiva casa, para subtrahir uma carteira da algibeira interior.
\section{Desengonçadamente}
\begin{itemize}
\item {Grp. gram.:adv.}
\end{itemize}
\begin{itemize}
\item {Proveniência:(De desengonçado)}
\end{itemize}
Fóra dos gonzos.
Desajeitadamente.
\section{Desengonçado}
\begin{itemize}
\item {Grp. gram.:adj.}
\end{itemize}
\begin{itemize}
\item {Proveniência:(De \textunderscore desengonçar\textunderscore )}
\end{itemize}
Que saiu dos gonzos.
Cujos gonzos ou articulações estão lassas: \textunderscore navalha desengonçada\textunderscore .
Que póde dobrar os braços e as pernas, como os sáltimbancos.
\section{Desengonçar}
\begin{itemize}
\item {Grp. gram.:v. t.}
\end{itemize}
\begin{itemize}
\item {Grp. gram.:V. p.}
\end{itemize}
\begin{itemize}
\item {Utilização:Fig.}
\end{itemize}
\begin{itemize}
\item {Proveniência:(De \textunderscore des...\textunderscore  + \textunderscore engonçar\textunderscore )}
\end{itemize}
Tirar dos engonços.
Desconjuntar.
Mover-se descompostamente.
\section{Desengonço}
\begin{itemize}
\item {Grp. gram.:m.}
\end{itemize}
Acto ou effeito de desengonçar.
\section{Desengordar}
\begin{itemize}
\item {Grp. gram.:v. t.}
\end{itemize}
\begin{itemize}
\item {Proveniência:(De \textunderscore des...\textunderscore  + \textunderscore engordar\textunderscore )}
\end{itemize}
Tirar ou deminuir a gordura a.
Tornar magro.
\section{Desengordurar}
\begin{itemize}
\item {Grp. gram.:v. t.}
\end{itemize}
\begin{itemize}
\item {Proveniência:(De \textunderscore des...\textunderscore  + \textunderscore engordurar\textunderscore )}
\end{itemize}
Tirar manchas de gordura a.
Tirar a gordura a: \textunderscore desengordurar a caldo\textunderscore .
\section{Desengraçadamente}
\begin{itemize}
\item {Grp. gram.:adv.}
\end{itemize}
De modo desengraçado.
\section{Desengraçadão}
\begin{itemize}
\item {Grp. gram.:m.}
\end{itemize}
\begin{itemize}
\item {Utilização:Fam.}
\end{itemize}
\begin{itemize}
\item {Proveniência:(De \textunderscore desengraçado\textunderscore )}
\end{itemize}
Indivíduo que tem modos muito desengraçados.
\section{Desengraçado}
\begin{itemize}
\item {Grp. gram.:adj.}
\end{itemize}
\begin{itemize}
\item {Utilização:Fig.}
\end{itemize}
Que não é engraçado.
Deselegante.
A quem falta a animação.
Insípido.
\section{Desengraçar}
\begin{itemize}
\item {Grp. gram.:v. t.}
\end{itemize}
\begin{itemize}
\item {Grp. gram.:V. i.}
\end{itemize}
\begin{itemize}
\item {Utilização:Pop.}
\end{itemize}
\begin{itemize}
\item {Proveniência:(De \textunderscore des...\textunderscore  + \textunderscore engraçar\textunderscore )}
\end{itemize}
Tirar a graça a.
Antipathizar.
\section{Desengrainhar}
\begin{itemize}
\item {fónica:gra-i}
\end{itemize}
\begin{itemize}
\item {Grp. gram.:v. t.}
\end{itemize}
\begin{itemize}
\item {Proveniência:(De \textunderscore des...\textunderscore  + \textunderscore en...\textunderscore  + \textunderscore graínha\textunderscore )}
\end{itemize}
Tirar a graínha de.
\section{Desengraixar}
\begin{itemize}
\item {Grp. gram.:v. t.}
\end{itemize}
\begin{itemize}
\item {Utilização:Pop.}
\end{itemize}
\begin{itemize}
\item {Proveniência:(De \textunderscore des...\textunderscore  + \textunderscore engraixar\textunderscore )}
\end{itemize}
Tirar a graixa ou o lustro de.
Destingir (aquillo que estava pintado de preto).
\section{Desengrandecer}
\begin{itemize}
\item {Grp. gram.:v. t.}
\end{itemize}
\begin{itemize}
\item {Proveniência:(De \textunderscore des...\textunderscore  + \textunderscore engrandecer\textunderscore )}
\end{itemize}
Apoucar; aviltar.
\section{Desengranzar}
\begin{itemize}
\item {Grp. gram.:v. t.}
\end{itemize}
\begin{itemize}
\item {Proveniência:(De \textunderscore des...\textunderscore  + \textunderscore engranzar\textunderscore )}
\end{itemize}
Desprender, soltar (aquillo que estava engranzado).
\section{Desengravecer}
\begin{itemize}
\item {Grp. gram.:v. t.}
\end{itemize}
\begin{itemize}
\item {Proveniência:(De \textunderscore des...\textunderscore  + \textunderscore engravecer\textunderscore )}
\end{itemize}
Tirar a gravidade de.
Tornar menos grave.
\section{Desengrazar}
\begin{itemize}
\item {Grp. gram.:v. t.}
\end{itemize}
(V.desengranzar)
\section{Desengrenhar}
\begin{itemize}
\item {Grp. gram.:v. t.}
\end{itemize}
O mesmo que \textunderscore desgrenhar\textunderscore .
\section{Desengrilar-se}
\begin{itemize}
\item {Grp. gram.:v. p.}
\end{itemize}
\begin{itemize}
\item {Utilização:Pop.}
\end{itemize}
\begin{itemize}
\item {Proveniência:(De \textunderscore des...\textunderscore  + \textunderscore engrilar\textunderscore )}
\end{itemize}
Desenfurecer-se.
\section{Desengrillar-se}
\begin{itemize}
\item {Grp. gram.:v. p.}
\end{itemize}
\begin{itemize}
\item {Utilização:Pop.}
\end{itemize}
\begin{itemize}
\item {Proveniência:(De \textunderscore des...\textunderscore  + \textunderscore engrillar\textunderscore )}
\end{itemize}
Desenfurecer-se.
\section{Desengrimpar-se}
\begin{itemize}
\item {Grp. gram.:v. p.}
\end{itemize}
\begin{itemize}
\item {Utilização:Pop.}
\end{itemize}
\begin{itemize}
\item {Utilização:Fig.}
\end{itemize}
\begin{itemize}
\item {Proveniência:(De \textunderscore des...\textunderscore  + \textunderscore engrimpar-se\textunderscore )}
\end{itemize}
Descer das grimpas.
Abater-se.
\section{Desengrimponar-se}
\begin{itemize}
\item {Grp. gram.:v. p.}
\end{itemize}
(V.desengrimpar-se)
\section{Desengrinaldar}
\begin{itemize}
\item {Grp. gram.:v. t.}
\end{itemize}
\begin{itemize}
\item {Proveniência:(De \textunderscore des...\textunderscore  + \textunderscore engrinaldar\textunderscore )}
\end{itemize}
Tirar a grinalda a.
\section{Desengrossar}
\begin{itemize}
\item {Grp. gram.:v. t.}
\end{itemize}
\begin{itemize}
\item {Proveniência:(De \textunderscore des...\textunderscore  + \textunderscore engrossar\textunderscore )}
\end{itemize}
Tornar menos grosso.
Desbastar.
\section{Desengrôsso}
\begin{itemize}
\item {Grp. gram.:m.}
\end{itemize}
Acto de desengrossar.
\section{Desengrumar}
\begin{itemize}
\item {Grp. gram.:v. t.}
\end{itemize}
(V.desgrumar)
\section{Desengrunhir}
\begin{itemize}
\item {Grp. gram.:v. t.}
\end{itemize}
\begin{itemize}
\item {Utilização:Prov.}
\end{itemize}
\begin{itemize}
\item {Proveniência:(De \textunderscore des...\textunderscore  + \textunderscore engrunhir\textunderscore )}
\end{itemize}
Desentorpecer.
Tirar o torpor ou a preguiça a.
\section{Desenguiçar}
\begin{itemize}
\item {Grp. gram.:v. t.}
\end{itemize}
\begin{itemize}
\item {Proveniência:(De \textunderscore des...\textunderscore  + \textunderscore enguiçar\textunderscore )}
\end{itemize}
Tirar o enguiço a.
\section{Desenguiçar}
\begin{itemize}
\item {Grp. gram.:v. t.}
\end{itemize}
\begin{itemize}
\item {Utilização:Prov.}
\end{itemize}
\begin{itemize}
\item {Utilização:trasm.}
\end{itemize}
Desenredar, alisar, (o cabello).
(Por \textunderscore desenriçar\textunderscore )
\section{Desenguiço}
\begin{itemize}
\item {Grp. gram.:m.}
\end{itemize}
\begin{itemize}
\item {Utilização:Prov.}
\end{itemize}
\begin{itemize}
\item {Utilização:trasm.}
\end{itemize}
\begin{itemize}
\item {Proveniência:(De \textunderscore desenguiçar\textunderscore ^2)}
\end{itemize}
Pente grande, para alisar o cabello.
\section{Desengulhar}
\begin{itemize}
\item {Grp. gram.:v. t.}
\end{itemize}
\begin{itemize}
\item {Proveniência:(De \textunderscore des...\textunderscore  + \textunderscore engulhar\textunderscore )}
\end{itemize}
Livrar de engulho.
\section{Desengulir}
\begin{itemize}
\item {Grp. gram.:v. t.}
\end{itemize}
\begin{itemize}
\item {Proveniência:(De \textunderscore des...\textunderscore  + \textunderscore engulir\textunderscore )}
\end{itemize}
Vomitar. Cf. G. Vicente, \textunderscore Farça dos Almocreves\textunderscore .
\section{Desengurgitar}
\begin{itemize}
\item {Grp. gram.:v.}
\end{itemize}
\begin{itemize}
\item {Utilização:t. Med.}
\end{itemize}
\begin{itemize}
\item {Proveniência:(De \textunderscore des...\textunderscore  + \textunderscore engurgitar\textunderscore )}
\end{itemize}
Desobstruir (um vaso ou ducto excretor).
\section{Desenhador}
\begin{itemize}
\item {Grp. gram.:m.}
\end{itemize}
Aquelle que desenha.
\section{Desenhar}
\begin{itemize}
\item {Grp. gram.:v. t.}
\end{itemize}
\begin{itemize}
\item {Utilização:Fig.}
\end{itemize}
\begin{itemize}
\item {Grp. gram.:V. i.}
\end{itemize}
\begin{itemize}
\item {Grp. gram.:V. p.}
\end{itemize}
\begin{itemize}
\item {Proveniência:(Do lat. \textunderscore designare\textunderscore )}
\end{itemize}
Fazer o desenho de.
Representar por meio de linhas e sombras: \textunderscore desenhar uma paisagem\textunderscore .
Delinear os contornos de: \textunderscore desenhar uma figura\textunderscore .
Fazer resaír, dar relêvo a: \textunderscore aquelle vestido desenha-lhe as fórmas\textunderscore .
Representar por palavras.
Descrever.
Planear. Cf. Usque, \textunderscore Tribulações\textunderscore , 30.
Traçar desenhos.
Resaír; relevar-se.
\section{Desenhista}
\begin{itemize}
\item {Grp. gram.:m.  e  f.}
\end{itemize}
\begin{itemize}
\item {Proveniência:(De \textunderscore desenho\textunderscore )}
\end{itemize}
Pessôa que desenha. Cf. R. Ortigão, \textunderscore Hollanda\textunderscore , 257; Júl. de Castilho, \textunderscore Manuelinas\textunderscore , 179.
\section{Desenho}
\begin{itemize}
\item {Grp. gram.:m.}
\end{itemize}
Emprêgo de linhas e sombras, para representar objectos.
Representação dos objectos, por meio de traços ou linhas e sombras.
Delineamento ou traçado geral de um quadro.
Delineamento, plano: \textunderscore o desenho de uma construcção\textunderscore .
Figuras de estampagem.
Delineamento de contornos.
Arte de desenhar: \textunderscore estudar desenho\textunderscore .
Acto de desenhar.
Intento, desígnio, plano. Cf. R. Lobo, \textunderscore Côrte na Aldeia\textunderscore , II, 87; \textunderscore Peregrinação\textunderscore , XXXIII.«\textunderscore Como fosse meu desenho offerecer ao Conde o fruito...\textunderscore »\textunderscore Jorn. de Áfr.\textunderscore , dedicatória.
\section{Desenjoar}
\begin{itemize}
\item {Grp. gram.:v. t.}
\end{itemize}
\begin{itemize}
\item {Proveniência:(De \textunderscore des...\textunderscore  + \textunderscore enjoar\textunderscore )}
\end{itemize}
Tirar o enjôo a.
Distrahir.
\section{Desenjoativo}
\begin{itemize}
\item {Grp. gram.:adj.}
\end{itemize}
\begin{itemize}
\item {Grp. gram.:M.}
\end{itemize}
\begin{itemize}
\item {Proveniência:(De \textunderscore des...\textunderscore  + \textunderscore enjoativo\textunderscore )}
\end{itemize}
Que desenjôa.
Iguaria, que desperta o appetite ou que tira o enjôo produzido por outra.
\section{Desenlaçamento}
\begin{itemize}
\item {Grp. gram.:m.}
\end{itemize}
Acto ou effeito de desenlaçar.
\section{Desenlaçar}
\begin{itemize}
\item {Grp. gram.:v. t.}
\end{itemize}
\begin{itemize}
\item {Utilização:Fig.}
\end{itemize}
\begin{itemize}
\item {Proveniência:(De \textunderscore des...\textunderscore  + \textunderscore enlaçar\textunderscore )}
\end{itemize}
Desprender do laço.
Desenredar.
Dar solução a.
\section{Desenlace}
\begin{itemize}
\item {Grp. gram.:m.}
\end{itemize}
O mesmo que \textunderscore desenlaçamento\textunderscore ; desfecho; solução: \textunderscore o desenlace da tragedia\textunderscore .
\section{Desenlambuzar}
\begin{itemize}
\item {Grp. gram.:v. t.}
\end{itemize}
\begin{itemize}
\item {Utilização:Pop.}
\end{itemize}
\begin{itemize}
\item {Proveniência:(De \textunderscore des...\textunderscore  + \textunderscore enlambuzar\textunderscore )}
\end{itemize}
Limpar (aquillo que estava enlambuzado).
Tirar nódoas ou porcaria a.
\section{Desenlamear}
\begin{itemize}
\item {Grp. gram.:v. t.}
\end{itemize}
\begin{itemize}
\item {Utilização:Fig.}
\end{itemize}
\begin{itemize}
\item {Proveniência:(De \textunderscore des...\textunderscore  + \textunderscore enlamear\textunderscore )}
\end{itemize}
Tirar a lama a.
Restabelecer o crédito de.
\section{Desenlapar}
\begin{itemize}
\item {Grp. gram.:v. t.}
\end{itemize}
\begin{itemize}
\item {Proveniência:(De \textunderscore des...\textunderscore  + \textunderscore enlapar\textunderscore )}
\end{itemize}
Fazer saír da lapa.
\section{Desenleado}
\begin{itemize}
\item {Grp. gram.:adj.}
\end{itemize}
\begin{itemize}
\item {Proveniência:(De \textunderscore desenlear\textunderscore )}
\end{itemize}
Expedito; franco.
\section{Desenlear}
\begin{itemize}
\item {Grp. gram.:v. t.}
\end{itemize}
\begin{itemize}
\item {Utilização:Fig.}
\end{itemize}
\begin{itemize}
\item {Proveniência:(De \textunderscore des...\textunderscore  + \textunderscore enlear\textunderscore )}
\end{itemize}
Desfazer o enleio a.
Desenredar.
Livrar de difficuldades.
\section{Desenleio}
\begin{itemize}
\item {Grp. gram.:m.}
\end{itemize}
Acto ou effeito de desenlear.
\section{Desenlevar}
\begin{itemize}
\item {Grp. gram.:v. t.}
\end{itemize}
\begin{itemize}
\item {Proveniência:(De \textunderscore des...\textunderscore  + \textunderscore enlevar\textunderscore )}
\end{itemize}
Tirar o enlêvo a.
Desilludir.
\section{Desenliçar}
\begin{itemize}
\item {Grp. gram.:v. t.}
\end{itemize}
\begin{itemize}
\item {Proveniência:(De \textunderscore des...\textunderscore  + \textunderscore enliçar\textunderscore )}
\end{itemize}
Desenredar; destrinçar.
\section{Desenlodar}
\begin{itemize}
\item {Grp. gram.:v. t.}
\end{itemize}
\begin{itemize}
\item {Proveniência:(De \textunderscore des...\textunderscore  + \textunderscore enlodar\textunderscore )}
\end{itemize}
Limpar do lodo.
Tirar o lodo a.
Desenlamear.
\section{Desenlouquecer}
\begin{itemize}
\item {Grp. gram.:v. t.}
\end{itemize}
\begin{itemize}
\item {Grp. gram.:V. i.}
\end{itemize}
\begin{itemize}
\item {Proveniência:(De \textunderscore des...\textunderscore  + \textunderscore enlouquecer\textunderscore )}
\end{itemize}
Curar a loucura a.
Recuperar o juízo.
\section{Desenlutar}
\begin{itemize}
\item {Grp. gram.:v. t.}
\end{itemize}
\begin{itemize}
\item {Proveniência:(De \textunderscore des...\textunderscore  + \textunderscore enlutar\textunderscore )}
\end{itemize}
Tirar o luto a.
Dar alegria a.
Consolar (quem estava enlutado).
\section{Desennastrar}
\begin{itemize}
\item {Grp. gram.:v. t.}
\end{itemize}
\begin{itemize}
\item {Proveniência:(De \textunderscore des...\textunderscore  + \textunderscore ennastrar\textunderscore )}
\end{itemize}
Soltar do nastro ou nastros.
\section{Desennatar}
\begin{itemize}
\item {Grp. gram.:v. t.}
\end{itemize}
O mesmo que \textunderscore desnatar\textunderscore .
\section{Desennegrecer}
\begin{itemize}
\item {Grp. gram.:v. t.}
\end{itemize}
\begin{itemize}
\item {Proveniência:(De \textunderscore des...\textunderscore  + \textunderscore ennegrecer\textunderscore )}
\end{itemize}
Aclarar.
\section{Desennevoar}
\begin{itemize}
\item {Grp. gram.:v. t.}
\end{itemize}
\begin{itemize}
\item {Utilização:Fig.}
\end{itemize}
\begin{itemize}
\item {Proveniência:(De \textunderscore des...\textunderscore  + \textunderscore ennevoar\textunderscore )}
\end{itemize}
Limpar de nuvens ou de névoa.
Alegrar, aclarar: \textunderscore desennevoar o espírito\textunderscore .
\section{Desennobrecer}
\begin{itemize}
\item {Grp. gram.:v. t.}
\end{itemize}
\begin{itemize}
\item {Proveniência:(De \textunderscore des...\textunderscore  + \textunderscore ennobrecer\textunderscore )}
\end{itemize}
Tirar a nobreza a.
Desdoirar.
\section{Desennodoamento}
\begin{itemize}
\item {Grp. gram.:m.}
\end{itemize}
Acto de desennodoar. Cf. Castilho, \textunderscore Fastos\textunderscore , II, 323.
\section{Desennodoar}
\begin{itemize}
\item {Grp. gram.:v. t.}
\end{itemize}
\begin{itemize}
\item {Proveniência:(De \textunderscore des...\textunderscore  + \textunderscore ennodoar\textunderscore )}
\end{itemize}
Tirar as nódoas a.
Limpar.
Desenlamear.
\section{Desennojar}
\begin{itemize}
\item {Grp. gram.:v. t.}
\end{itemize}
O mesmo que \textunderscore desanojar\textunderscore .
\section{Desennovelar}
\begin{itemize}
\item {Grp. gram.:v. t.}
\end{itemize}
\begin{itemize}
\item {Proveniência:(De \textunderscore des...\textunderscore  + \textunderscore ennovelar\textunderscore )}
\end{itemize}
Desenrolar (aquillo que estava ennovelado).
\section{Desennublar}
\begin{itemize}
\item {Grp. gram.:v. t.}
\end{itemize}
O mesmo que \textunderscore desennevoar\textunderscore .
\section{Desenquadrar}
\begin{itemize}
\item {Grp. gram.:v. t.}
\end{itemize}
\begin{itemize}
\item {Proveniência:(De \textunderscore des...\textunderscore  + \textunderscore enquadrar\textunderscore )}
\end{itemize}
Tirar do quadro ou de moldura.
\section{Desenraiar}
\begin{itemize}
\item {Grp. gram.:v. t.}
\end{itemize}
\begin{itemize}
\item {Proveniência:(De \textunderscore des...\textunderscore  + \textunderscore enraiar\textunderscore )}
\end{itemize}
Destravar (uma roda de carro).
\section{Desenraivar}
\begin{itemize}
\item {Grp. gram.:v. t.}
\end{itemize}
O mesmo que \textunderscore desenraivecer\textunderscore .
\section{Desenraivecer}
\begin{itemize}
\item {Grp. gram.:v. t.}
\end{itemize}
\begin{itemize}
\item {Proveniência:(De \textunderscore des...\textunderscore  + \textunderscore enraivecer\textunderscore )}
\end{itemize}
Tornar sereno, abrandar, tirar a raiva a.
\section{Desenraizar}
\begin{itemize}
\item {fónica:ra-i}
\end{itemize}
\begin{itemize}
\item {Grp. gram.:v. t.}
\end{itemize}
O mesmo que \textunderscore desarraigar\textunderscore .
\section{Desenramar}
\begin{itemize}
\item {Grp. gram.:v. t.}
\end{itemize}
\begin{itemize}
\item {Proveniência:(De \textunderscore des...\textunderscore  + \textunderscore enramar\textunderscore )}
\end{itemize}
Tirar os ramos a.
\section{Desenrascar}
\begin{itemize}
\item {Grp. gram.:v. t.}
\end{itemize}
\begin{itemize}
\item {Proveniência:(De \textunderscore des...\textunderscore  + \textunderscore enrascar\textunderscore )}
\end{itemize}
Desembaraçar (aquillo que estava enrascado).
\section{Desenredador}
\begin{itemize}
\item {Grp. gram.:m.}
\end{itemize}
Aquelle que desenreda.
\section{Desenredar}
\begin{itemize}
\item {Grp. gram.:v. t.}
\end{itemize}
\begin{itemize}
\item {Proveniência:(De \textunderscore des...\textunderscore  + \textunderscore enredar\textunderscore )}
\end{itemize}
Desfazer o enrêdo de.
Desembaraçar.
Explicar.
Desenlaçar.
Dar solução a.
Perscrutar.
\section{Desenrêdo}
\begin{itemize}
\item {Grp. gram.:m.}
\end{itemize}
Acto ou effeito de desenredar.
\section{Desenregelamento}
\begin{itemize}
\item {Grp. gram.:m.}
\end{itemize}
Acto de desenregelar.
\section{Desenregelar}
\begin{itemize}
\item {Grp. gram.:v. t.}
\end{itemize}
\begin{itemize}
\item {Utilização:Fig.}
\end{itemize}
\begin{itemize}
\item {Proveniência:(De \textunderscore des...\textunderscore  + \textunderscore enregelar\textunderscore )}
\end{itemize}
Desgelar.
Aquecer.
\section{Desenriçar}
\begin{itemize}
\item {Grp. gram.:v. t.}
\end{itemize}
\begin{itemize}
\item {Proveniência:(De \textunderscore des...\textunderscore  + \textunderscore enriçar\textunderscore )}
\end{itemize}
Desencrespar.
Desemmaranhar: \textunderscore desenriçar o cabello\textunderscore .
\section{Desenrijar}
\begin{itemize}
\item {Grp. gram.:v. t.}
\end{itemize}
\begin{itemize}
\item {Proveniência:(De \textunderscore des...\textunderscore  + \textunderscore enrijar\textunderscore )}
\end{itemize}
Tornar brando, molle.
\section{Desenriquecer}
\begin{itemize}
\item {Grp. gram.:v. t.}
\end{itemize}
\begin{itemize}
\item {Grp. gram.:V. i.}
\end{itemize}
\begin{itemize}
\item {Proveniência:(De \textunderscore des...\textunderscore  + \textunderscore enriquecer\textunderscore )}
\end{itemize}
Tirar a riqueza a.
Tornar pobre.
Tornar-se pobre.
\section{Desenristar}
\begin{itemize}
\item {Grp. gram.:v. t.}
\end{itemize}
\begin{itemize}
\item {Proveniência:(De \textunderscore des...\textunderscore  + \textunderscore enristar\textunderscore )}
\end{itemize}
Tirar do riste.
\section{Desenrizar}
\begin{itemize}
\item {Grp. gram.:v.}
\end{itemize}
\begin{itemize}
\item {Utilização:t. Náut.}
\end{itemize}
\begin{itemize}
\item {Proveniência:(De \textunderscore des...\textunderscore  + \textunderscore enrizar\textunderscore )}
\end{itemize}
Tirar dos rizes.
\section{Desenrodilhar}
\begin{itemize}
\item {Grp. gram.:v. t.}
\end{itemize}
\begin{itemize}
\item {Proveniência:(De \textunderscore des...\textunderscore  + \textunderscore enrodilhar\textunderscore )}
\end{itemize}
Desenrolar, estender.
\section{Desenrolamento}
\begin{itemize}
\item {Grp. gram.:m.}
\end{itemize}
Acto de desenrolar.
\section{Desenrolar}
\begin{itemize}
\item {Grp. gram.:v. t.}
\end{itemize}
\begin{itemize}
\item {Utilização:Fig.}
\end{itemize}
\begin{itemize}
\item {Proveniência:(De \textunderscore des...\textunderscore  + \textunderscore enrolar\textunderscore )}
\end{itemize}
Desfazer o rôlo de.
Estender (aquillo que estava enrolado): \textunderscore desenrolar uma bandeira\textunderscore .
Desenvolver, expôr minuciosamente: \textunderscore desenrolar episódios\textunderscore .
\section{Desenrolhar}
\begin{itemize}
\item {Grp. gram.:v. t.}
\end{itemize}
O mesmo que \textunderscore desarrolhar\textunderscore . Cf. Macedo, \textunderscore Burros\textunderscore , 5.
\section{Desenroscar}
\begin{itemize}
\item {Grp. gram.:v. t.}
\end{itemize}
\begin{itemize}
\item {Proveniência:(De \textunderscore des...\textunderscore  + \textunderscore enroscar\textunderscore )}
\end{itemize}
Estender (aquillo que estava enroscado).
Estirar; desaparafusar.
\section{Desenroupar}
\begin{itemize}
\item {Grp. gram.:v. t.}
\end{itemize}
\begin{itemize}
\item {Proveniência:(De \textunderscore des...\textunderscore  + \textunderscore enroupar\textunderscore )}
\end{itemize}
Tirar a roupa a.
Despir.
\section{Desenrouquecer}
\begin{itemize}
\item {Grp. gram.:v. t.}
\end{itemize}
\begin{itemize}
\item {Proveniência:(De \textunderscore des...\textunderscore  + \textunderscore enrouquecer\textunderscore )}
\end{itemize}
Tirar a rouquidão a.
\section{Desenrubescer}
\begin{itemize}
\item {Grp. gram.:v. t.}
\end{itemize}
\begin{itemize}
\item {Grp. gram.:V. i.}
\end{itemize}
\begin{itemize}
\item {Proveniência:(De \textunderscore des...\textunderscore  + \textunderscore enrubescer\textunderscore )}
\end{itemize}
Fazer perder a côr vermelha a.
Deixar de sêr corado.
Empallidecer.
\section{Desenrugar}
\begin{itemize}
\item {Grp. gram.:v. t.}
\end{itemize}
\begin{itemize}
\item {Proveniência:(De \textunderscore des...\textunderscore  + \textunderscore enrugar\textunderscore )}
\end{itemize}
Tirar as rugas a.
Desencarquilhar.
\section{Desensaboar}
\begin{itemize}
\item {Grp. gram.:v. t.}
\end{itemize}
\begin{itemize}
\item {Proveniência:(De \textunderscore des...\textunderscore  + \textunderscore ensaboar\textunderscore )}
\end{itemize}
Limpar, tirando o sabão. Cf. Filinto, V, 185.
\section{Desensacar}
\begin{itemize}
\item {Grp. gram.:v. t.}
\end{itemize}
\begin{itemize}
\item {Proveniência:(De \textunderscore des...\textunderscore  + \textunderscore ensacar\textunderscore )}
\end{itemize}
Tirar do saco.
\section{Desensandecer}
\begin{itemize}
\item {Grp. gram.:v. t.  e  i.}
\end{itemize}
O mesmo que \textunderscore desenlouquecer\textunderscore .
\section{Desensanguentar}
\begin{itemize}
\item {fónica:gu-en}
\end{itemize}
\begin{itemize}
\item {Grp. gram.:v. t.}
\end{itemize}
\begin{itemize}
\item {Proveniência:(De \textunderscore des...\textunderscore  + \textunderscore ensanguentar\textunderscore )}
\end{itemize}
Limpar o sangue a.
\section{Desensarilhar}
\begin{itemize}
\item {Grp. gram.:v. t.}
\end{itemize}
\begin{itemize}
\item {Proveniência:(De \textunderscore des...\textunderscore  + \textunderscore ensarilhar\textunderscore )}
\end{itemize}
Separar (aquillo que estava ensarilhado): \textunderscore desensarilhar armas\textunderscore .
\section{Desensebar}
\begin{itemize}
\item {Grp. gram.:v. t.}
\end{itemize}
\begin{itemize}
\item {Proveniência:(De \textunderscore des...\textunderscore  + \textunderscore ensebar\textunderscore )}
\end{itemize}
Limpar do sebo; tirar as manchas do sebo a.
\section{Desensinador}
\begin{itemize}
\item {Grp. gram.:m.}
\end{itemize}
Aquelle que desensina.
\section{Desensinar}
\begin{itemize}
\item {Grp. gram.:v. t.}
\end{itemize}
\begin{itemize}
\item {Proveniência:(De \textunderscore des...\textunderscore  + \textunderscore ensinar\textunderscore )}
\end{itemize}
Fazer desapprender.
Fazer esquecer (aquillo que se tinha apprendido).
\section{Desensino}
\begin{itemize}
\item {Grp. gram.:m.}
\end{itemize}
Acto ou effeito de desensinar.
\section{Desensoberbecer}
\begin{itemize}
\item {Grp. gram.:v. t.}
\end{itemize}
\begin{itemize}
\item {Proveniência:(De \textunderscore des...\textunderscore  + \textunderscore ensoberbecer\textunderscore )}
\end{itemize}
Tirar a soberba a.
Humilhar.
\section{Desensolvar}
\begin{itemize}
\item {Grp. gram.:v. t.}
\end{itemize}
O mesmo que \textunderscore desassolvar\textunderscore .
\section{Desensombrar}
\begin{itemize}
\item {Grp. gram.:v. t.}
\end{itemize}
\begin{itemize}
\item {Proveniência:(De \textunderscore des...\textunderscore  + \textunderscore ensombrar\textunderscore )}
\end{itemize}
Tirar aquillo que fazia sombra a.
Desennevoar.
Tornar claro ou alegre.
\section{Desensopar}
\begin{itemize}
\item {Grp. gram.:v. t.}
\end{itemize}
\begin{itemize}
\item {Proveniência:(De \textunderscore des...\textunderscore  + \textunderscore ensopar\textunderscore )}
\end{itemize}
Enxugar.
\section{Desensurdecer}
\begin{itemize}
\item {Grp. gram.:v. t.}
\end{itemize}
\begin{itemize}
\item {Grp. gram.:V. i.}
\end{itemize}
\begin{itemize}
\item {Proveniência:(De \textunderscore des...\textunderscore  + \textunderscore ensurdecer\textunderscore )}
\end{itemize}
Tirar a surdez a.
Curar-se da surdez.
\section{Desentabuar}
\begin{itemize}
\item {Grp. gram.:v. t.}
\end{itemize}
\begin{itemize}
\item {Proveniência:(De \textunderscore des...\textunderscore  + \textunderscore entabuar\textunderscore )}
\end{itemize}
Tirar o sobrado ou o fôrro a (uma casa).
\section{Desentabular}
\begin{itemize}
\item {Grp. gram.:v. t.}
\end{itemize}
\begin{itemize}
\item {Proveniência:(De \textunderscore des...\textunderscore  + \textunderscore entabular\textunderscore )}
\end{itemize}
Desmanchar (aquillo que estava entabulado).
\section{Desentaipar}
\begin{itemize}
\item {Grp. gram.:v. t.}
\end{itemize}
\begin{itemize}
\item {Utilização:Fig.}
\end{itemize}
\begin{itemize}
\item {Proveniência:(De \textunderscore des...\textunderscore  + \textunderscore entaipar\textunderscore )}
\end{itemize}
Tirar de entre taipas ou taipaes.
Desembaraçar, libertar.
Desafrontar.
\section{Desentalar}
\begin{itemize}
\item {Grp. gram.:v. t.}
\end{itemize}
\begin{itemize}
\item {Utilização:Fig.}
\end{itemize}
\begin{itemize}
\item {Proveniência:(De \textunderscore des...\textunderscore  + \textunderscore entalar\textunderscore )}
\end{itemize}
Tirar das talas.
Livrar de difficuldades.
\section{Desentaramelar}
\begin{itemize}
\item {Grp. gram.:v. t.}
\end{itemize}
\begin{itemize}
\item {Proveniência:(De \textunderscore des...\textunderscore  + \textunderscore entaramelar\textunderscore )}
\end{itemize}
Desembaraçar (a lingua), falando-se muito ou á tôa.
\section{Desentarraxar}
\begin{itemize}
\item {Grp. gram.:v. t.}
\end{itemize}
\begin{itemize}
\item {Proveniência:(De \textunderscore des...\textunderscore  + \textunderscore entarraxar\textunderscore )}
\end{itemize}
Tirar a tarraxa a.
Desapertar, laxar.
Desatarraxar. Cf. Camillo, \textunderscore Estrêl. Prop.\textunderscore  99.
\section{Desentediar}
\begin{itemize}
\item {Grp. gram.:v. t.}
\end{itemize}
\begin{itemize}
\item {Proveniência:(De \textunderscore des...\textunderscore  + \textunderscore entediar\textunderscore )}
\end{itemize}
Tirar o tédio a.
Desenjoar.
Distrahir.
\section{Desentender}
\begin{itemize}
\item {Grp. gram.:v. t.}
\end{itemize}
\begin{itemize}
\item {Proveniência:(De \textunderscore des...\textunderscore  + \textunderscore entender\textunderscore )}
\end{itemize}
Fingir que não entende.
\section{Desentendidamente}
\begin{itemize}
\item {Grp. gram.:adv.}
\end{itemize}
De modo desentendido.
Com dissimulação.
\section{Desentendido}
\begin{itemize}
\item {Grp. gram.:adj.}
\end{itemize}
\begin{itemize}
\item {Proveniência:(De \textunderscore desentender\textunderscore )}
\end{itemize}
Que não entende.
Fazer-se desentendido, mostrar-se indifferente; não fazer caso do que se diz ou se faz; fingir que não entende.
\section{Desentendimento}
\begin{itemize}
\item {Grp. gram.:adj.}
\end{itemize}
\begin{itemize}
\item {Proveniência:(De \textunderscore des...\textunderscore  + \textunderscore entendimento\textunderscore )}
\end{itemize}
Falta de entendimento.
Inépcia.
\section{Desentenebrecer}
\begin{itemize}
\item {Grp. gram.:v. t.}
\end{itemize}
\begin{itemize}
\item {Proveniência:(De \textunderscore des...\textunderscore  + \textunderscore entenebrecer\textunderscore )}
\end{itemize}
Aclarar; dissipar as trevas de.
\section{Desenternecer}
\begin{itemize}
\item {Grp. gram.:v. t.}
\end{itemize}
\begin{itemize}
\item {Proveniência:(De \textunderscore des...\textunderscore  + \textunderscore enternecer\textunderscore )}
\end{itemize}
Fazer perder a ternura a.
\section{Desenterrador}
\begin{itemize}
\item {Grp. gram.:adj.}
\end{itemize}
\begin{itemize}
\item {Grp. gram.:M.}
\end{itemize}
Que desenterra.
Aquelle que desenterra.
\section{Desenterramento}
\begin{itemize}
\item {Grp. gram.:m.}
\end{itemize}
Acto de desenterrar.
\section{Desenterrar}
\begin{itemize}
\item {Grp. gram.:v. t.}
\end{itemize}
\begin{itemize}
\item {Utilização:Fig.}
\end{itemize}
\begin{itemize}
\item {Proveniência:(De \textunderscore des...\textunderscore  + \textunderscore enterrar\textunderscore )}
\end{itemize}
Tirar da terra.
Exhumar: \textunderscore desenterrar um cadáver\textunderscore .
Patentear.
\section{Desenterroar}
\begin{itemize}
\item {Grp. gram.:v. t.}
\end{itemize}
(V.esterroar)
\section{Desentesar}
\begin{itemize}
\item {Grp. gram.:v. t.}
\end{itemize}
\begin{itemize}
\item {Utilização:Fig.}
\end{itemize}
\begin{itemize}
\item {Proveniência:(De \textunderscore des...\textunderscore  + \textunderscore entesar\textunderscore )}
\end{itemize}
Tornar lasso, froixo.
Humilhar.
\section{Desentesoirador}
\begin{itemize}
\item {Grp. gram.:m.}
\end{itemize}
Aquele que desentesoira.
\section{Desentesoirar}
\begin{itemize}
\item {Grp. gram.:v. t.}
\end{itemize}
\begin{itemize}
\item {Utilização:Fig.}
\end{itemize}
\begin{itemize}
\item {Proveniência:(De \textunderscore des...\textunderscore  + \textunderscore entesoirar\textunderscore )}
\end{itemize}
Tirar do tesoiro.
Desencantar.
\section{Desentesourar}
\begin{itemize}
\item {Grp. gram.:v. t.}
\end{itemize}
\begin{itemize}
\item {Utilização:Fig.}
\end{itemize}
\begin{itemize}
\item {Proveniência:(De \textunderscore des...\textunderscore  + \textunderscore entesourar\textunderscore )}
\end{itemize}
Tirar do tesouro.
Desencantar.
\section{Desenthesoirador}
\begin{itemize}
\item {Grp. gram.:m.}
\end{itemize}
Aquelle que desenthesoira.
\section{Desenthesoirar}
\begin{itemize}
\item {Grp. gram.:v. t.}
\end{itemize}
\begin{itemize}
\item {Utilização:Fig.}
\end{itemize}
\begin{itemize}
\item {Proveniência:(De \textunderscore des...\textunderscore  + \textunderscore enthesoirar\textunderscore )}
\end{itemize}
Tirar do thesoiro.
Desencantar.
\section{Desenthronizar}
\begin{itemize}
\item {Grp. gram.:v. t.}
\end{itemize}
O mesmo que \textunderscore desthronar\textunderscore .
\section{Desentibiar}
\begin{itemize}
\item {Grp. gram.:v. t.}
\end{itemize}
\begin{itemize}
\item {Proveniência:(De \textunderscore des...\textunderscore  + \textunderscore entibiar\textunderscore )}
\end{itemize}
Tirar a tibieza a.
\section{Desentoação}
\begin{itemize}
\item {Grp. gram.:f.}
\end{itemize}
Acto ou effeito de desentoar.
\section{Desentoadamente}
\begin{itemize}
\item {Grp. gram.:adv.}
\end{itemize}
De modo desentoado.
\section{Desentoado}
\begin{itemize}
\item {Grp. gram.:adj.}
\end{itemize}
\begin{itemize}
\item {Proveniência:(De \textunderscore desentoar\textunderscore )}
\end{itemize}
Desafinado, dissonante.
\section{Desentoamento}
\begin{itemize}
\item {Grp. gram.:m.}
\end{itemize}
O mesmo que \textunderscore desentoação\textunderscore .
\section{Desentoar}
\begin{itemize}
\item {Grp. gram.:v. t.}
\end{itemize}
\begin{itemize}
\item {Grp. gram.:V. i.}
\end{itemize}
\begin{itemize}
\item {Utilização:Fig.}
\end{itemize}
\begin{itemize}
\item {Proveniência:(De \textunderscore des...\textunderscore  + \textunderscore entoar\textunderscore )}
\end{itemize}
Cantar, desafinando: \textunderscore desentoar o hymno nacional\textunderscore .
Destoar.
Fazer ou dizer inconveniências.
\section{Desentocar}
\begin{itemize}
\item {Grp. gram.:v. t.}
\end{itemize}
\begin{itemize}
\item {Proveniência:(De \textunderscore des...\textunderscore  + \textunderscore entocar\textunderscore )}
\end{itemize}
Tirar de toca ou de cova.
\section{Desentolher}
\begin{itemize}
\item {Grp. gram.:v. t.}
\end{itemize}
\begin{itemize}
\item {Proveniência:(De \textunderscore des...\textunderscore  + \textunderscore en...\textunderscore  + \textunderscore tolher\textunderscore )}
\end{itemize}
Tirar o entorpecimento a.
\section{Desentonar}
\begin{itemize}
\item {Grp. gram.:v. t.}
\end{itemize}
\begin{itemize}
\item {Proveniência:(De \textunderscore des...\textunderscore  + \textunderscore entonar\textunderscore )}
\end{itemize}
Abater, humilhar.
\section{Desentorpecer}
\begin{itemize}
\item {Grp. gram.:v. t.}
\end{itemize}
\begin{itemize}
\item {Utilização:Fig.}
\end{itemize}
\begin{itemize}
\item {Proveniência:(De \textunderscore des...\textunderscore  + \textunderscore entorpecer\textunderscore )}
\end{itemize}
Tirar do torpor.
Reanimar.
\section{Desentorpecimento}
\begin{itemize}
\item {Grp. gram.:m.}
\end{itemize}
Acto ou effeito de desentorpecer.
\section{Desentorroar}
\begin{itemize}
\item {Grp. gram.:v. t.}
\end{itemize}
(V.esterroar)
\section{Desentortar}
\begin{itemize}
\item {Grp. gram.:v. t.}
\end{itemize}
\begin{itemize}
\item {Proveniência:(De \textunderscore des...\textunderscore  + \textunderscore entortar\textunderscore )}
\end{itemize}
Tirar a qualidade de torto a.
Endireitar. Cf. Camillo, \textunderscore Maria da Fonte\textunderscore , 412.
\section{Desentralhar}
\begin{itemize}
\item {Grp. gram.:v. t.}
\end{itemize}
\begin{itemize}
\item {Proveniência:(De \textunderscore des...\textunderscore  + \textunderscore entralhar\textunderscore )}
\end{itemize}
Tirar das tralhas.
Desenredar.
Desprender (aquillo que estava entralhado); desentalar.
\section{Desentrançar}
\begin{itemize}
\item {Grp. gram.:v. t.}
\end{itemize}
\begin{itemize}
\item {Proveniência:(De \textunderscore des...\textunderscore  + \textunderscore entrançar\textunderscore )}
\end{itemize}
Desmanchar (aquillo que estava entrançado).
\section{Desentranhar}
\begin{itemize}
\item {Grp. gram.:v. t.}
\end{itemize}
\begin{itemize}
\item {Grp. gram.:V. p.}
\end{itemize}
\begin{itemize}
\item {Proveniência:(De \textunderscore des...\textunderscore  + \textunderscore entranhar\textunderscore )}
\end{itemize}
Tirar das entranhas.
Estripar.
Tirar de lugar occulto.
Tirar do seio, do coração: \textunderscore desentranhar suspiros\textunderscore .
Patentear o que tem no íntimo ou no coração.
Desafogar-se.
Desabafar.
\section{Desentravar}
\begin{itemize}
\item {Grp. gram.:v. t.}
\end{itemize}
O mesmo que \textunderscore destravar\textunderscore .
\section{Desentrecho}
\begin{itemize}
\item {fónica:trê}
\end{itemize}
\begin{itemize}
\item {Grp. gram.:m.}
\end{itemize}
\begin{itemize}
\item {Utilização:P. us.}
\end{itemize}
\begin{itemize}
\item {Proveniência:(De \textunderscore des...\textunderscore  + \textunderscore entrecho\textunderscore )}
\end{itemize}
Desfecho, desenlace.
\section{Desentrincheirar}
\begin{itemize}
\item {Grp. gram.:v. t.}
\end{itemize}
\begin{itemize}
\item {Proveniência:(De \textunderscore des...\textunderscore  + \textunderscore entrincheirar\textunderscore )}
\end{itemize}
Romper as trincheiras de.
Desalojar de atrás das trincheiras.
\section{Desentristecer}
\begin{itemize}
\item {Grp. gram.:v. t.}
\end{itemize}
\begin{itemize}
\item {Grp. gram.:V. i.}
\end{itemize}
\begin{itemize}
\item {Proveniência:(De \textunderscore des...\textunderscore  + \textunderscore entristecer\textunderscore )}
\end{itemize}
Alegrar; tirar a tristeza a.
Perder a tristeza.
\section{Desentroixar}
\begin{itemize}
\item {Grp. gram.:v. t.}
\end{itemize}
\begin{itemize}
\item {Proveniência:(De \textunderscore des...\textunderscore  + \textunderscore entroixar\textunderscore )}
\end{itemize}
Tirar da troixa.
Desmanchar (aquillo que estava entroixado).
\section{Desentronizar}
\begin{itemize}
\item {Grp. gram.:v. t.}
\end{itemize}
O mesmo que \textunderscore destronar\textunderscore .
\section{Desentrouxar}
\begin{itemize}
\item {Grp. gram.:v. t.}
\end{itemize}
\begin{itemize}
\item {Proveniência:(De \textunderscore des...\textunderscore  + \textunderscore entrouxar\textunderscore )}
\end{itemize}
Tirar da trouxa.
Desmanchar (aquillo que estava entrouxado).
\section{Desentulhador}
\begin{itemize}
\item {Grp. gram.:m.}
\end{itemize}
Aquelle que desentulha.
\section{Desentulhar}
\begin{itemize}
\item {Grp. gram.:v. t.}
\end{itemize}
\begin{itemize}
\item {Proveniência:(De \textunderscore des...\textunderscore  + \textunderscore entulhar\textunderscore )}
\end{itemize}
Tirar da tulha.
Tirar (aquillo que estava entulhado): \textunderscore desentulhar caliça\textunderscore .
\section{Desentulho}
\begin{itemize}
\item {Grp. gram.:m.}
\end{itemize}
Acto ou effeito de desentulhar.
\section{Desentupimento}
\begin{itemize}
\item {Grp. gram.:m.}
\end{itemize}
Acto ou effeito de desentupir.
\section{Desentupir}
\begin{itemize}
\item {Grp. gram.:v. t.}
\end{itemize}
\begin{itemize}
\item {Proveniência:(De \textunderscore des...\textunderscore  + \textunderscore entupir\textunderscore )}
\end{itemize}
Desobstruir.
Desembaraçar, desimpedir, (aquillo estava entupido).
\section{Desenturvar}
\begin{itemize}
\item {Grp. gram.:v. t.}
\end{itemize}
\begin{itemize}
\item {Proveniência:(De \textunderscore des...\textunderscore  + \textunderscore enturvar\textunderscore )}
\end{itemize}
Tirar a enturvação a.
Tornar claro ou transparente (aquillo que era turvo). Cf. Alves Mendes, \textunderscore Discursos\textunderscore , 250.
\section{Desenvasar}
\begin{itemize}
\item {Grp. gram.:v. t.}
\end{itemize}
\begin{itemize}
\item {Proveniência:(De \textunderscore des...\textunderscore  + \textunderscore envasar\textunderscore )}
\end{itemize}
Tirar da vasa.
Pôr a nado (o navio)
\section{Desenvasilhar}
\begin{itemize}
\item {Grp. gram.:v. t.}
\end{itemize}
\begin{itemize}
\item {Proveniência:(De \textunderscore des...\textunderscore  + \textunderscore envasilhar\textunderscore )}
\end{itemize}
Tirar da vasilha: \textunderscore desenvasilhar azeite\textunderscore .
\section{Desenvenenar}
\begin{itemize}
\item {Grp. gram.:v. t.}
\end{itemize}
\begin{itemize}
\item {Proveniência:(De \textunderscore des...\textunderscore  + \textunderscore envenenar\textunderscore )}
\end{itemize}
Curar os effeitos do veneno em.
\section{Desenvergar}
\begin{itemize}
\item {Grp. gram.:v.}
\end{itemize}
\begin{itemize}
\item {Utilização:t. Náut.}
\end{itemize}
\begin{itemize}
\item {Utilização:Fam.}
\end{itemize}
\begin{itemize}
\item {Proveniência:(De \textunderscore des...\textunderscore  + \textunderscore envergar\textunderscore )}
\end{itemize}
Tirar das vêrgas.
Despir.
\section{Desenvergonhado}
\begin{itemize}
\item {Grp. gram.:adj.}
\end{itemize}
O mesmo que \textunderscore desavergonhado\textunderscore . Cf. Pant. do Aveiro, \textunderscore Itiner.\textunderscore , 36 v.^o, (2.^a ed.)
\section{Desenvernizar}
\begin{itemize}
\item {Grp. gram.:v. t.}
\end{itemize}
\begin{itemize}
\item {Proveniência:(De \textunderscore des...\textunderscore  + \textunderscore envernizar\textunderscore )}
\end{itemize}
Tirar o verniz de.
Deslustrar.
\section{Desenvièzar}
\begin{itemize}
\item {Grp. gram.:v. t.}
\end{itemize}
\begin{itemize}
\item {Proveniência:(De \textunderscore des...\textunderscore  + \textunderscore envièsar\textunderscore )}
\end{itemize}
Tirar o viés a.
\section{Desenvincilhar}
\begin{itemize}
\item {Grp. gram.:v. t.}
\end{itemize}
\begin{itemize}
\item {Utilização:Fig.}
\end{itemize}
\begin{itemize}
\item {Utilização:Fig.}
\end{itemize}
\begin{itemize}
\item {Proveniência:(De \textunderscore des...\textunderscore  + \textunderscore envincilhar\textunderscore )}
\end{itemize}
Soltar do vincilho.
Desatar; desemmaranhar: \textunderscore desenvincilhar meadas\textunderscore .
Aclarar, resolver: \textunderscore desenvincilhar questões\textunderscore .
\section{Desenviolar}
\begin{itemize}
\item {Grp. gram.:v. t.}
\end{itemize}
\begin{itemize}
\item {Utilização:Ant.}
\end{itemize}
Tomar apto para usos honestos ou sagrados (aquillo que estava polluto e contaminado ou profanado).
(Por \textunderscore desviolar\textunderscore , de \textunderscore des\textunderscore  + \textunderscore violar\textunderscore )
\section{Desenviscar}
\begin{itemize}
\item {Grp. gram.:v. t.}
\end{itemize}
\begin{itemize}
\item {Proveniência:(De \textunderscore des...\textunderscore  + \textunderscore enviscar\textunderscore )}
\end{itemize}
Tirar o visco a.
\section{Desenvolto}
\begin{itemize}
\item {fónica:vôl}
\end{itemize}
\begin{itemize}
\item {Grp. gram.:adj.}
\end{itemize}
Desembaraçado.
Travêsso.
Libertino, deshonesto: \textunderscore linguagem desenvolta\textunderscore .
(\textunderscore Part. irr.\textunderscore  de \textunderscore desenvolver\textunderscore )
\section{Desenvoltura}
\begin{itemize}
\item {Grp. gram.:f.}
\end{itemize}
Qualidade daquelle ou daquillo que é desenvolto.
\section{Desenvolução}
\begin{itemize}
\item {Grp. gram.:f.}
\end{itemize}
O mesmo que \textunderscore desenvolvimento\textunderscore .
\section{Desenvolvente}
\begin{itemize}
\item {Grp. gram.:adj.}
\end{itemize}
Que desenvolve.
\section{Desenvolver}
\begin{itemize}
\item {Grp. gram.:v. t.}
\end{itemize}
\begin{itemize}
\item {Proveniência:(De \textunderscore des...\textunderscore  + \textunderscore envolver\textunderscore )}
\end{itemize}
Tirar de invólucro; desenrolar.
Fazer crescer: \textunderscore o sol desenvolve as plantas\textunderscore .
Expor minuciosamente: \textunderscore desenvolver um assumpto\textunderscore .
Executar as operações, implícitas em (um cálculo mathemático).
Descobrir todos os termos incluídos em (uma série ou funcção mathemática).
Aumentar as faculdades intellectuaes de: \textunderscore a Mathemática desenvolve o estudante\textunderscore .
Tirar a timidez a.
Representar num plano todos os lados de (uma construcção).
\section{Desenvolvida}
\begin{itemize}
\item {Grp. gram.:f.}
\end{itemize}
\begin{itemize}
\item {Utilização:Geom.}
\end{itemize}
Linha curva, da qual se suppõe que, sendo desenvolvida, formaria outra.
\section{Desenvolvidamente}
\begin{itemize}
\item {Grp. gram.:adv.}
\end{itemize}
De modo desenvolvido.
Minuciosamente.
\section{Desenvolvido}
\begin{itemize}
\item {Grp. gram.:adj.}
\end{itemize}
\begin{itemize}
\item {Proveniência:(De \textunderscore desenvolver\textunderscore )}
\end{itemize}
Que revela intelligência ou desenvolvimento: \textunderscore rapaz desenvolvido\textunderscore .
Crescido: \textunderscore searas desenvolvidas\textunderscore .
Aumentado.
Amplificado.
Minucioso: \textunderscore um relatório desenvolvido\textunderscore .
\section{Desenvolvimento}
\begin{itemize}
\item {Grp. gram.:m.}
\end{itemize}
Acto ou effeito de desenvolver.
Crescimento.
Ampliação.
Minuciosidade.
\section{Desenvolvível}
\begin{itemize}
\item {Grp. gram.:adj.}
\end{itemize}
Que se póde desenvolver.
\section{Desenxabidamente}
\begin{itemize}
\item {Grp. gram.:adv.}
\end{itemize}
De modo desenxabido.
\section{Desenxabido}
\begin{itemize}
\item {Grp. gram.:adj.}
\end{itemize}
\begin{itemize}
\item {Utilização:Fig.}
\end{itemize}
\begin{itemize}
\item {Proveniência:(De \textunderscore des...\textunderscore  + \textunderscore en...\textunderscore  + \textunderscore sápido\textunderscore )}
\end{itemize}
Insípido.
Desgracioso.
Que não tem animação.
\section{Desenxabir}
\begin{itemize}
\item {Grp. gram.:v. t.}
\end{itemize}
Tornar desenxabido. Cf. Filinto. V, 304.
\section{Desenxamear}
\begin{itemize}
\item {Grp. gram.:v. t.}
\end{itemize}
\begin{itemize}
\item {Proveniência:(De \textunderscore des...\textunderscore  + \textunderscore enxamear\textunderscore )}
\end{itemize}
Dispersar (aquillo que enxameava).
\section{Desenxarciar}
\begin{itemize}
\item {Grp. gram.:v. t.}
\end{itemize}
\begin{itemize}
\item {Utilização:Ant.}
\end{itemize}
\begin{itemize}
\item {Proveniência:(De \textunderscore des...\textunderscore  + \textunderscore enxárcia\textunderscore )}
\end{itemize}
Tirar as enxárcias a.
\section{Desenxoframento}
\begin{itemize}
\item {Grp. gram.:m.}
\end{itemize}
Acto de desenxofrar.
\section{Desenxofrar}
\begin{itemize}
\item {Grp. gram.:v. t.}
\end{itemize}
\begin{itemize}
\item {Utilização:Fig.}
\end{itemize}
\begin{itemize}
\item {Proveniência:(De \textunderscore des...\textunderscore  + \textunderscore enxofrar\textunderscore )}
\end{itemize}
Limpar do enxôfre.
Desencolerizar.
\section{Desenxovalhado}
\begin{itemize}
\item {Grp. gram.:adj.}
\end{itemize}
\begin{itemize}
\item {Proveniência:(De \textunderscore desenxovalhar\textunderscore )}
\end{itemize}
Limpo; asseado bem pôsto: \textunderscore é uma rapariga desenxovalhada\textunderscore .
\section{Desenxovalhar}
\begin{itemize}
\item {Grp. gram.:v. t.}
\end{itemize}
\begin{itemize}
\item {Utilização:Fig.}
\end{itemize}
\begin{itemize}
\item {Proveniência:(De \textunderscore des...\textunderscore  + \textunderscore enxovalhar\textunderscore )}
\end{itemize}
Lavar, limpar.
Desamarrotar.
Desafrontar.
\section{Desenxovalho}
\begin{itemize}
\item {Grp. gram.:m.}
\end{itemize}
Acto ou effeito de desenxovalhar.
\section{Desenxovar}
\begin{itemize}
\item {Grp. gram.:v. t.}
\end{itemize}
\begin{itemize}
\item {Proveniência:(De \textunderscore des...\textunderscore  + \textunderscore enxovar\textunderscore )}
\end{itemize}
Tirar de enxovia. Cf. Camillo, \textunderscore Noites de Insómn.\textunderscore , VIII, 72.
\section{Desequilibrado}
\begin{itemize}
\item {Grp. gram.:adj.}
\end{itemize}
\begin{itemize}
\item {Utilização:Fam.}
\end{itemize}
Que não tem equilíbrio; que não está equilibrado.
Que tem desarranjo mental.
\section{Desequilibrar}
\begin{itemize}
\item {Grp. gram.:v. t.}
\end{itemize}
\begin{itemize}
\item {Proveniência:(De \textunderscore des...\textunderscore  + \textunderscore equilibrar\textunderscore )}
\end{itemize}
Tirar o equilíbrio a.
\section{Desequilíbrio}
\begin{itemize}
\item {Grp. gram.:m.}
\end{itemize}
\begin{itemize}
\item {Utilização:Fam.}
\end{itemize}
\begin{itemize}
\item {Proveniência:(De \textunderscore des...\textunderscore  + \textunderscore equilíbrio\textunderscore )}
\end{itemize}
Falta de equilíbrio.
Desarranjo mental.
\section{Deserção}
\begin{itemize}
\item {Grp. gram.:f.}
\end{itemize}
Acto de desertar.
\section{Desertar}
\begin{itemize}
\item {Grp. gram.:v. t.}
\end{itemize}
\begin{itemize}
\item {Grp. gram.:V. i.}
\end{itemize}
\begin{itemize}
\item {Proveniência:(De \textunderscore deserto\textunderscore )}
\end{itemize}
Despovoar.
Desamparar: \textunderscore desertar o seu lugar\textunderscore .
Renunciar.
Ausentar-se; afastar-se.
Fugir do serviço militar.
\section{Deserto}
\begin{itemize}
\item {Grp. gram.:m.}
\end{itemize}
\begin{itemize}
\item {Grp. gram.:Adj.}
\end{itemize}
\begin{itemize}
\item {Utilização:Pop.}
\end{itemize}
\begin{itemize}
\item {Proveniência:(Lat. \textunderscore desertus\textunderscore )}
\end{itemize}
Região despovoada e árida.
Lugar solitário.
Solidão.
Ermo.
Em que não vive gente: \textunderscore planície deserta\textunderscore .
Cujos habitantes pereceram ou se ausentaram: \textunderscore a aldeia ficou deserta\textunderscore .
Impaciente, ansioso.
\section{Desertor}
\begin{itemize}
\item {Grp. gram.:m.}
\end{itemize}
\begin{itemize}
\item {Proveniência:(Lat. \textunderscore desertor\textunderscore )}
\end{itemize}
Aquelle que desertou do serviço militar.
\section{Deservir}
\begin{itemize}
\item {fónica:ser}
\end{itemize}
\begin{itemize}
\item {Grp. gram.:v. t.}
\end{itemize}
O mesmo que \textunderscore desservir\textunderscore ; negar-se a servir. Cf. Pant. de Aveiro, \textunderscore Itiner.\textunderscore , 281, (2.^a ed.)
\section{Desesperação}
\begin{itemize}
\item {Grp. gram.:f.}
\end{itemize}
Acto ou effeito de desesperar.
\section{Desesperadamente}
\begin{itemize}
\item {Grp. gram.:adv.}
\end{itemize}
De modo desesperado.
Com desespêro.
\section{Desesperado}
\begin{itemize}
\item {Grp. gram.:adj.}
\end{itemize}
\begin{itemize}
\item {Grp. gram.:M.}
\end{itemize}
\begin{itemize}
\item {Proveniência:(De \textunderscore desesperar\textunderscore )}
\end{itemize}
Que não espera, que perdeu a esperança.
Que não dá esperança: \textunderscore doença desesperada\textunderscore .
Encarniçado: \textunderscore luta desesperada\textunderscore .
Arrebatado.
Indivíduo alucinado.
\section{Desesperador}
\begin{itemize}
\item {Grp. gram.:adj.}
\end{itemize}
Que faz desesperar.
\section{Desesperança}
\begin{itemize}
\item {Grp. gram.:f.}
\end{itemize}
Falta de esperança.
Desesperação.
\section{Desesperançar}
\begin{itemize}
\item {Grp. gram.:v. t.}
\end{itemize}
\begin{itemize}
\item {Proveniência:(De \textunderscore des...\textunderscore  + \textunderscore esperançar\textunderscore )}
\end{itemize}
Tirar a esperança a.
\section{Desesperante}
\begin{itemize}
\item {Grp. gram.:adj.}
\end{itemize}
\begin{itemize}
\item {Proveniência:(De \textunderscore desesperar\textunderscore )}
\end{itemize}
Que causa desesperação.
\section{Desesperar}
\begin{itemize}
\item {Grp. gram.:v. t.}
\end{itemize}
\begin{itemize}
\item {Grp. gram.:V. i.}
\end{itemize}
\begin{itemize}
\item {Proveniência:(De \textunderscore des...\textunderscore  + \textunderscore esperar\textunderscore )}
\end{itemize}
Tirar a esperança a.
Causar desespêro a.
Irritar; affligir intensamente.
Não esperar; não têr esperança de:«\textunderscore mas desespéro esse &amp; todo outro remedio\textunderscore ». \textunderscore Eufrosina\textunderscore , 25.
Perder a esperança.
Irritar-se.
\section{Desesperativo}
\begin{itemize}
\item {Grp. gram.:adj.}
\end{itemize}
Que faz desesperar. Cf. Filinto, XX, 14.
\section{Desespêro}
\begin{itemize}
\item {Grp. gram.:m.}
\end{itemize}
O mesmo que \textunderscore desesperação\textunderscore .
\section{Desestagnação}
\begin{itemize}
\item {Grp. gram.:f.}
\end{itemize}
Acto de desestagnar.
\section{Desestagnar}
\begin{itemize}
\item {Grp. gram.:v. t.}
\end{itemize}
\begin{itemize}
\item {Proveniência:(De \textunderscore des...\textunderscore  + \textunderscore estagnar\textunderscore )}
\end{itemize}
Fazer correr (água que estava estagnada); tirar a estagnação de.
\section{Desesteirar}
\begin{itemize}
\item {Grp. gram.:v. t.}
\end{itemize}
\begin{itemize}
\item {Proveniência:(De \textunderscore des...\textunderscore  + \textunderscore esteirar\textunderscore )}
\end{itemize}
Tirar as esteiras de.
Descobrir, levantando as esteiras: \textunderscore desesteirar uma sala\textunderscore .
\section{Desestima}
\begin{itemize}
\item {Grp. gram.:f.}
\end{itemize}
Falta de estima.
\section{Desestimação}
\begin{itemize}
\item {Grp. gram.:f.}
\end{itemize}
O mesmo que \textunderscore desestima\textunderscore .
\section{Desestimadamente}
\begin{itemize}
\item {Grp. gram.:adv.}
\end{itemize}
\begin{itemize}
\item {Proveniência:(De \textunderscore desestimar\textunderscore )}
\end{itemize}
Desprezivelmente; com desestima.
\section{Desestimador}
\begin{itemize}
\item {Grp. gram.:adj.}
\end{itemize}
\begin{itemize}
\item {Grp. gram.:M.}
\end{itemize}
Que desestima.
Aquelle que desestima.
\section{Desestimar}
\begin{itemize}
\item {Grp. gram.:v. t.}
\end{itemize}
Não estimar; depreciar.
\section{Desestorvar}
\begin{itemize}
\item {Grp. gram.:v. t.}
\end{itemize}
\begin{itemize}
\item {Proveniência:(De \textunderscore des...\textunderscore  + \textunderscore estorvar\textunderscore )}
\end{itemize}
Tirar o estôrvo a.
\section{Desestôrvo}
\begin{itemize}
\item {Grp. gram.:m.}
\end{itemize}
Acto de desestorvar.
\section{Desestradamente}
\begin{itemize}
\item {Grp. gram.:adv.}
\end{itemize}
De modo desestrado. Cf. \textunderscore Peregrinação\textunderscore , CXVI.
\section{Desestrado}
\begin{itemize}
\item {Grp. gram.:adj.}
\end{itemize}
Que não tem jeito.
Que faz tudo mal.--Us. por Filinto, Castilho e Garrett, sob a infl. de \textunderscore estro\textunderscore . Cf. G. Viana, \textunderscore Apostilas\textunderscore , vb. \textunderscore desastrado\textunderscore .
\section{Desestramento}
\begin{itemize}
\item {Grp. gram.:m.}
\end{itemize}
Falta de jeito.
Acto ou modos do desestrado. Cf. Filinto, XVII, 161.
(Cp. \textunderscore desestrado\textunderscore )
\section{Desevangelizar}
\begin{itemize}
\item {Grp. gram.:v. t.}
\end{itemize}
\begin{itemize}
\item {Proveniência:(De \textunderscore des...\textunderscore  + \textunderscore evangelizar\textunderscore )}
\end{itemize}
Tirar a doutrina evangélica a.
Inutilizar a propaganda de (certa doutrina ou systema).
\section{Desexcommungar}
\begin{itemize}
\item {Grp. gram.:v. t.}
\end{itemize}
\begin{itemize}
\item {Proveniência:(De \textunderscore des...\textunderscore  + \textunderscore excommungar\textunderscore )}
\end{itemize}
Levantar a excommunhão a.
\section{Desexcommunhão}
\begin{itemize}
\item {Grp. gram.:f.}
\end{itemize}
Acto de desexcommungar.
\section{Desexcomungar}
\begin{itemize}
\item {Grp. gram.:v. t.}
\end{itemize}
\begin{itemize}
\item {Proveniência:(De \textunderscore des...\textunderscore  + \textunderscore excomungar\textunderscore )}
\end{itemize}
Levantar a excomunhão a.
\section{Desexcomunhão}
\begin{itemize}
\item {Grp. gram.:f.}
\end{itemize}
Acto de desexcomungar.
\section{Desfabricar}
\begin{itemize}
\item {Grp. gram.:v. t.}
\end{itemize}
\begin{itemize}
\item {Proveniência:(De \textunderscore des...\textunderscore  + \textunderscore fabricar\textunderscore )}
\end{itemize}
Desmanchar (aquillo que se tinha fabricado).
\section{Desfabular}
\begin{itemize}
\item {Grp. gram.:v. t.}
\end{itemize}
Desfazer a fábula de; mostrar a verdade de. Cf. Castilho, \textunderscore Mont'Alverne\textunderscore .
\section{Desfaçadamente}
\begin{itemize}
\item {Grp. gram.:adv.}
\end{itemize}
\begin{itemize}
\item {Proveniência:(De \textunderscore desfaçado\textunderscore )}
\end{itemize}
Com desfaçatez.
\section{Desfaçado}
\begin{itemize}
\item {Grp. gram.:adj.}
\end{itemize}
\begin{itemize}
\item {Proveniência:(De \textunderscore desfaçar-se\textunderscore )}
\end{itemize}
Que tem desfaçatez.
Desavergonhado.
\section{Desfaçamento}
\begin{itemize}
\item {Grp. gram.:m.}
\end{itemize}
O mesmo que \textunderscore desfaçatez\textunderscore . Cf. Herculano, \textunderscore Bobo\textunderscore , 81.
\section{Desfaçar-se}
\begin{itemize}
\item {Grp. gram.:v. p.}
\end{itemize}
\begin{itemize}
\item {Proveniência:(De \textunderscore des...\textunderscore  + \textunderscore face\textunderscore )}
\end{itemize}
Tornar-se descarado, insolente, atrevido, impudente.
\section{Desfaçatez}
\begin{itemize}
\item {Grp. gram.:f.}
\end{itemize}
Qualidade de quem é desfaçado.
\section{Desfadiga}
\begin{itemize}
\item {Grp. gram.:f.}
\end{itemize}
Acto de desfadigar.
Descanso; allívio. Cf. Filinto, IX, 44; XVII, 61; XVIII, 177 e 250.
\section{Desfadigar}
\begin{itemize}
\item {Grp. gram.:v. t.}
\end{itemize}
\begin{itemize}
\item {Proveniência:(De \textunderscore des...\textunderscore  + \textunderscore fadiga\textunderscore )}
\end{itemize}
Tirar a fadiga a. Alliviar o cansaço de. Cf. Castilho, \textunderscore Palavras de um Crente\textunderscore , 62.
\section{Desfaiar-se}
\begin{itemize}
\item {Grp. gram.:v. p.}
\end{itemize}
\begin{itemize}
\item {Utilização:Prov.}
\end{itemize}
\begin{itemize}
\item {Utilização:trasm.}
\end{itemize}
Despenhar-se de um fraguedo.
\section{Desfalcamento}
\begin{itemize}
\item {Grp. gram.:m.}
\end{itemize}
Acto ou effeito de desfalcar.
\section{Desfalcar}
\begin{itemize}
\item {Grp. gram.:v. t.}
\end{itemize}
Tirar parte de: \textunderscore desfalcar bens alheios\textunderscore .
Defraudar.
(B. lat. \textunderscore defalcare\textunderscore )
\section{Desfalco}
\begin{itemize}
\item {Grp. gram.:m.}
\end{itemize}
O mesmo que \textunderscore desfalque\textunderscore .
\section{Desfalcoar}
\begin{itemize}
\item {Grp. gram.:v. i.}
\end{itemize}
\begin{itemize}
\item {Utilização:Prov.}
\end{itemize}
\begin{itemize}
\item {Utilização:alg.}
\end{itemize}
\begin{itemize}
\item {Proveniência:(De \textunderscore desfalco\textunderscore )}
\end{itemize}
O mesmo que \textunderscore desfalcar\textunderscore .
\section{Desfalência}
\begin{itemize}
\item {Grp. gram.:f.}
\end{itemize}
O mesmo que \textunderscore desfalecimento\textunderscore . Cf. Castilho, \textunderscore Palavras de um Crente\textunderscore , 61.
\section{Desfalecente}
\begin{itemize}
\item {Grp. gram.:adj.}
\end{itemize}
Que desfalece.
\section{Desfalecer}
\begin{itemize}
\item {Grp. gram.:v. t.}
\end{itemize}
\begin{itemize}
\item {Grp. gram.:V. i.}
\end{itemize}
\begin{itemize}
\item {Utilização:Fig.}
\end{itemize}
\begin{itemize}
\item {Proveniência:(De \textunderscore des...\textunderscore  + \textunderscore falecer\textunderscore )}
\end{itemize}
Tirar as fôrças a.
Desalentar.
Desamparar.
Perder as fôrças.
Esmorecer.
Desmaiar.
Afroixar; decaír: \textunderscore o calor vai desfalecendo\textunderscore .
\section{Desfalecimento}
\begin{itemize}
\item {Grp. gram.:m.}
\end{itemize}
Acto ou efeito de desfalecer.
Desânimo.
Desmaio.
\section{Desfallência}
\begin{itemize}
\item {Grp. gram.:f.}
\end{itemize}
O mesmo que \textunderscore desfallecimento\textunderscore . Cf. Castilho, \textunderscore Palavras de um Crente\textunderscore , 61.
\section{Desfallecente}
\begin{itemize}
\item {Grp. gram.:adj.}
\end{itemize}
Que desfallece.
\section{Desfallecer}
\begin{itemize}
\item {Grp. gram.:v. t.}
\end{itemize}
\begin{itemize}
\item {Utilização:Fig.}
\end{itemize}
\begin{itemize}
\item {Proveniência:(De \textunderscore des...\textunderscore  + \textunderscore fallecer\textunderscore )}
\end{itemize}
Tirar as fôrças a.
Desalentar.
Desamparar.
\textunderscore V. i.\textunderscore l
Perder as fôrças.
Esmorecer.
Desmaiar.
Afroixar; decaír: \textunderscore o calor vai desfallecendo\textunderscore .
\section{Desfallecimento}
\begin{itemize}
\item {Grp. gram.:m.}
\end{itemize}
Acto ou effeito de desfallecer.
Desânimo.
Desmaio.
\section{Desfalleçudo}
\begin{itemize}
\item {Grp. gram.:adj.}
\end{itemize}
\begin{itemize}
\item {Utilização:Ant.}
\end{itemize}
Minguado; que não tem o que precisa.
(Part. ant. de \textunderscore desfallecer\textunderscore )
\section{Desfalque}
\begin{itemize}
\item {Grp. gram.:m.}
\end{itemize}
Acto ou effeito de desfalcar.
Somma ou quantia, igual á differença para menos na quantia desfalcada: \textunderscore um desfalque de mil escudos\textunderscore .
\section{Desfanatizar}
\begin{itemize}
\item {Grp. gram.:v. t.}
\end{itemize}
\begin{itemize}
\item {Proveniência:(De \textunderscore des...\textunderscore  + \textunderscore fanatizar\textunderscore )}
\end{itemize}
Fazer que (alguém) deixe de sêr fanático.
\section{Desfarelar}
\begin{itemize}
\item {Grp. gram.:v. t.}
\end{itemize}
(V.esfarelar)
\section{Desfastio}
\begin{itemize}
\item {Grp. gram.:m.}
\end{itemize}
Falta de fastio.
Graça, bom humor.
\section{Desfavor}
\begin{itemize}
\item {Grp. gram.:m.}
\end{itemize}
Falta de favor.
Malquerença; desprêzo.
\section{Desfavorável}
\begin{itemize}
\item {Grp. gram.:adj.}
\end{itemize}
Que não é favorável.
\section{Desfavoravelmente}
\begin{itemize}
\item {Grp. gram.:adv.}
\end{itemize}
De modo desfavorável.
\section{Desfavorecedor}
\begin{itemize}
\item {Grp. gram.:adj.}
\end{itemize}
\begin{itemize}
\item {Grp. gram.:M.}
\end{itemize}
Que desfavorece.
Aquelle que desfavorece.
\section{Desfavorecer}
\begin{itemize}
\item {Grp. gram.:v. t.}
\end{itemize}
Não favorecer.
Sêr desfavorável a.
Desestimar; contrariar.
\section{Desfazedor}
\begin{itemize}
\item {Grp. gram.:adj.}
\end{itemize}
\begin{itemize}
\item {Grp. gram.:M.}
\end{itemize}
Que desfaz.
Aquelle que desfaz.
\section{Desfazer}
\begin{itemize}
\item {Grp. gram.:v. t.}
\end{itemize}
\begin{itemize}
\item {Grp. gram.:V. i.}
\end{itemize}
\begin{itemize}
\item {Grp. gram.:V. p.}
\end{itemize}
\begin{itemize}
\item {Proveniência:(De \textunderscore des...\textunderscore  + \textunderscore fazer\textunderscore )}
\end{itemize}
Alterar a fórma de, substituindo-a por outra.
Desorganizar.
Desmanchar: \textunderscore desfazer um andaime\textunderscore .
Partir em pedaços: \textunderscore desfazer um casaco\textunderscore .
Destruir.
Dissolver: \textunderscore desfazer uma associação\textunderscore .
Annullar: \textunderscore desfazer um contrato\textunderscore .
Resolver.
Derrotar.
Malbaratar.
Desvanecer.
Abater.
Livrar.
Desgastar.
Refutar: \textunderscore desfazer argumentos\textunderscore .
Amesquinhar alguém; dirigir motejos.
Desapossar-se, largar do seu poder: \textunderscore desfazer-se de um cavallo\textunderscore .
Desencadear-se; tornar-se copioso, violento.
\section{Desfazimento}
\begin{itemize}
\item {Grp. gram.:m.}
\end{itemize}
Acto de desfazer.
\section{Desfear}
\begin{itemize}
\item {Grp. gram.:v. t.}
\end{itemize}
O mesmo que \textunderscore afear\textunderscore .
\section{Desfechar}
\begin{itemize}
\item {Grp. gram.:v. t.}
\end{itemize}
\begin{itemize}
\item {Grp. gram.:V. i.}
\end{itemize}
\begin{itemize}
\item {Proveniência:(De \textunderscore des...\textunderscore  + \textunderscore fechar\textunderscore )}
\end{itemize}
Tirar o fecho ou o sêllo a.
Descarregar (arma de fogo).
Arremessar.
Soltar.
Concluir.
Desencadear-se; desafogar.
Concluir-se: \textunderscore a conversa desfechou em injúrias\textunderscore .
\section{Desfecho}
\begin{itemize}
\item {fónica:fê}
\end{itemize}
\begin{itemize}
\item {Grp. gram.:m.}
\end{itemize}
\begin{itemize}
\item {Proveniência:(De \textunderscore desfechar\textunderscore )}
\end{itemize}
Conclusão ou remate de um drama, poema, etc.
Resultado; solução.
\section{Desfeita}
\begin{itemize}
\item {Grp. gram.:f.}
\end{itemize}
\begin{itemize}
\item {Utilização:Pop.}
\end{itemize}
\begin{itemize}
\item {Utilização:Ant.}
\end{itemize}
\begin{itemize}
\item {Proveniência:(De \textunderscore desfeito\textunderscore )}
\end{itemize}
Injúria; offensa: \textunderscore recebi delle muitas desfeitas\textunderscore .
Espécie de pureia.
Iguaria, feita de bacalhau desfiado, grão de bico, cebola, etc.
Imitação, paródia. Cf. \textunderscore Peregrinação\textunderscore , LXVIII.
\section{Desfeiteador}
\begin{itemize}
\item {Grp. gram.:m.}
\end{itemize}
Aquelle que desfeiteia.
\section{Desfeitear}
\begin{itemize}
\item {Grp. gram.:v. t.}
\end{itemize}
\begin{itemize}
\item {Proveniência:(De \textunderscore desfeita\textunderscore )}
\end{itemize}
Injuriar; offender.
Fazer pirraça a.
\section{Desfeito}
\begin{itemize}
\item {Grp. gram.:adj.}
\end{itemize}
\begin{itemize}
\item {Grp. gram.:M.}
\end{itemize}
\begin{itemize}
\item {Utilização:pop.}
\end{itemize}
Impetuoso, violento: \textunderscore um temporal desfeito\textunderscore .
Iguaria, o mesmo que \textunderscore desfeita\textunderscore .
(\textunderscore Part. irr.\textunderscore  de \textunderscore desfazer\textunderscore )
\section{Desferimento}
\begin{itemize}
\item {Grp. gram.:m.}
\end{itemize}
Acto de desferir.
\section{Desferir}
\begin{itemize}
\item {Grp. gram.:v. t.}
\end{itemize}
\begin{itemize}
\item {Proveniência:(De \textunderscore des...\textunderscore  + \textunderscore ferir\textunderscore )}
\end{itemize}
Fazer vibrar: \textunderscore desferir as cordas do alaúde\textunderscore .
Atirar, despedir: \textunderscore desferir frechas\textunderscore .
Desfraldar; patentear.
\section{Desferrar}
\begin{itemize}
\item {Grp. gram.:v. t.}
\end{itemize}
\begin{itemize}
\item {Utilização:Náut.}
\end{itemize}
\begin{itemize}
\item {Proveniência:(De \textunderscore des...\textunderscore  + \textunderscore ferrar\textunderscore )}
\end{itemize}
Fazer cair a ferradura de.
Soltar, desfraldar (velas).
\section{Desferrolhar}
\begin{itemize}
\item {Grp. gram.:v. t.}
\end{itemize}
(V.desaferrolhar)
\section{Desferrujar}
\begin{itemize}
\item {Proveniência:(De \textunderscore des...\textunderscore  + \textunderscore ferrugem\textunderscore )}
\end{itemize}
\textunderscore v. t.\textunderscore  (e der.)
O mesmo que \textunderscore desenferrujar\textunderscore , etc. Cf. Filinto, III, 206.
\section{Desfervoroso}
\begin{itemize}
\item {Grp. gram.:adj.}
\end{itemize}
\begin{itemize}
\item {Proveniência:(De \textunderscore des...\textunderscore  + \textunderscore fervoroso\textunderscore )}
\end{itemize}
Que não tem fervor, que não é fervoroso.
\section{Desfiada}
\begin{itemize}
\item {Grp. gram.:f.}
\end{itemize}
\begin{itemize}
\item {Utilização:T. de Tôrres-Novas}
\end{itemize}
\begin{itemize}
\item {Proveniência:(De \textunderscore desfiar\textunderscore )}
\end{itemize}
O mesmo que \textunderscore esfiada\textunderscore .
\section{Desfiado}
\begin{itemize}
\item {Grp. gram.:adj.}
\end{itemize}
\begin{itemize}
\item {Grp. gram.:M.}
\end{itemize}
\begin{itemize}
\item {Proveniência:(De \textunderscore desfiar\textunderscore )}
\end{itemize}
Desfeito em fios.
Esmiuçado.
Aquillo que se desfia. Cf. \textunderscore Eufrosina\textunderscore , 129.
\section{Desfiadura}
\begin{itemize}
\item {Grp. gram.:f.}
\end{itemize}
Acto ou effeito de desfiar.
\section{Desfiar}
\begin{itemize}
\item {Grp. gram.:v. t.}
\end{itemize}
\begin{itemize}
\item {Grp. gram.:V. i.}
\end{itemize}
\begin{itemize}
\item {Proveniência:(De \textunderscore des...\textunderscore  + \textunderscore fiar\textunderscore )}
\end{itemize}
Reduzir a fios.
Expor por miúdo.
Espalhar: \textunderscore desfiar pérolas\textunderscore .
Desenfiar.
Correr em fio.
\section{Desfia-trapos}
\begin{itemize}
\item {Grp. gram.:m.}
\end{itemize}
Nome de vários maquinismos destinados a desfiar trapos de lan velha, que se mistura com lan nova, para servir em tecidos ordinários.
\section{Desfibramento}
\begin{itemize}
\item {Grp. gram.:m.}
\end{itemize}
Acto de desfibrar.
\section{Desfibrante}
\begin{itemize}
\item {Grp. gram.:adj.}
\end{itemize}
Que desfibra.
\section{Desfibrar}
\begin{itemize}
\item {Grp. gram.:v. t.}
\end{itemize}
Tirar as fibras a.
\section{Desfribinar}
\begin{itemize}
\item {Grp. gram.:v. t.}
\end{itemize}
Tirar a fibrina a.
\section{Desfiguração}
\begin{itemize}
\item {Grp. gram.:f.}
\end{itemize}
Acto ou effeito de desfigurar.
\section{Desfigurado}
\begin{itemize}
\item {Grp. gram.:adj.}
\end{itemize}
\begin{itemize}
\item {Proveniência:(De \textunderscore desfigurar\textunderscore )}
\end{itemize}
Que mudou de figura.
Alterado.
Demudado nas feições.
\section{Desfigurador}
\begin{itemize}
\item {Grp. gram.:adj.}
\end{itemize}
\begin{itemize}
\item {Grp. gram.:M.}
\end{itemize}
Que desfigura.
Aquelle que desfigura.
\section{Desfigurar}
\begin{itemize}
\item {Grp. gram.:v. t.}
\end{itemize}
\begin{itemize}
\item {Proveniência:(De \textunderscore des...\textunderscore  + \textunderscore figurar\textunderscore )}
\end{itemize}
Alterar a figura ou o aspecto de: \textunderscore a doença desfigurou-o\textunderscore .
Afear.
Adulterar, deturpar.
\section{Desfigurável}
\begin{itemize}
\item {Grp. gram.:adj.}
\end{itemize}
Que se póde desfigurar.
\section{Desfilada}
\begin{itemize}
\item {Grp. gram.:f.}
\end{itemize}
Acto de desfilar; rapidez: \textunderscore correr á desfilada\textunderscore .
\section{Desfiladeiro}
\begin{itemize}
\item {Grp. gram.:m.}
\end{itemize}
\begin{itemize}
\item {Proveniência:(De \textunderscore desfilar\textunderscore )}
\end{itemize}
Garganta ou passagem estreita, entre montanhas.
\section{Desfiladora}
\begin{itemize}
\item {Grp. gram.:f.}
\end{itemize}
\begin{itemize}
\item {Proveniência:(De \textunderscore des...\textunderscore  + lat. \textunderscore filum\textunderscore )}
\end{itemize}
Máquina, para destramar tecidos, usada nas fábricas de papel e de nitrocellulose.
\section{Desfilar}
\begin{itemize}
\item {Grp. gram.:v. t.}
\end{itemize}
\begin{itemize}
\item {Proveniência:(De \textunderscore des...\textunderscore  + \textunderscore fila\textunderscore )}
\end{itemize}
Passar em filas.
Marchar successivamente; succeder-se: \textunderscore desfilavam tropas\textunderscore .
\section{Desfile}
\begin{itemize}
\item {Grp. gram.:m.}
\end{itemize}
Acto de desfilar.
\section{Desfilhar}
\begin{itemize}
\item {Grp. gram.:v. t.}
\end{itemize}
\begin{itemize}
\item {Utilização:Ant.}
\end{itemize}
\begin{itemize}
\item {Proveniência:(De \textunderscore des...\textunderscore  + \textunderscore filhar\textunderscore )}
\end{itemize}
Tirar os rebentos demasiados a.
Separar parte das abelhas de (uma colmeia).
Tirar os filhos a:«\textunderscore virlheey ao encõtro como urso desfilhado, e espedaçarei\textunderscore ». Usque, \textunderscore Tribulações\textunderscore , XVII.
\section{Desfitar}
\begin{itemize}
\item {Grp. gram.:v. t.}
\end{itemize}
Não fitar.
Desviar (os olhos).
\section{Desfivelar}
\begin{itemize}
\item {Grp. gram.:v. t.}
\end{itemize}
(V.desafivelar)
\section{Desfloração}
\begin{itemize}
\item {Grp. gram.:f.}
\end{itemize}
Acto ou effeito de desflorar.
\section{Desflorador}
\begin{itemize}
\item {Grp. gram.:adj.}
\end{itemize}
\begin{itemize}
\item {Grp. gram.:M.}
\end{itemize}
Que desflora.
Aquelle que desflora.
\section{Desfloramento}
\begin{itemize}
\item {Grp. gram.:m.}
\end{itemize}
O mesmo que \textunderscore desfloração\textunderscore .
\section{Desflorar}
\begin{itemize}
\item {Grp. gram.:v. t.}
\end{itemize}
\begin{itemize}
\item {Utilização:Gal}
\end{itemize}
\begin{itemize}
\item {Utilização:Bras. do S}
\end{itemize}
\begin{itemize}
\item {Utilização:Bras. do S}
\end{itemize}
\begin{itemize}
\item {Proveniência:(Lat. \textunderscore deflorare\textunderscore )}
\end{itemize}
Tirar as flôres a.
Provar, encetar: \textunderscore desflorar uma iguaria\textunderscore .
Offender ou tirar a pureza, a virgindade, de: \textunderscore desflorar uma donzella\textunderscore .
Tocar, passar, á superficie de.
Encrespar ligeiramente a superficie de (mar, lago, etc.).
Arruinar (um cavallo).
Tirar as melhores reses a (uma tropa de éguas).
\section{Desflorecer}
\begin{itemize}
\item {Grp. gram.:v. i.}
\end{itemize}
\begin{itemize}
\item {Proveniência:(De \textunderscore des...\textunderscore  + \textunderscore florecer\textunderscore )}
\end{itemize}
Perder as flôres.
Emmurchecer.
Perder o frescor, o brilho.
\section{Desflorecimento}
\begin{itemize}
\item {Grp. gram.:m.}
\end{itemize}
Acto ou effeito de desflorecer.
\section{Desflorir}
\begin{itemize}
\item {Grp. gram.:v. t.}
\end{itemize}
O mesmo que \textunderscore desflorecer\textunderscore .
\section{Desfogonar-se}
\begin{itemize}
\item {Grp. gram.:v. p.}
\end{itemize}
\begin{itemize}
\item {Proveniência:(De \textunderscore des...\textunderscore  + \textunderscore fogão\textunderscore )}
\end{itemize}
Diz-se da peça de artilharia, que tem o ouvido ou fogão já gasto ou arruinado pelo uso.
\section{Desfolegar}
\begin{itemize}
\item {Grp. gram.:v. i.}
\end{itemize}
(V.resfolegar)
\section{Desfôlha}
\begin{itemize}
\item {Grp. gram.:f.}
\end{itemize}
O mesmo que \textunderscore desfolhação\textunderscore .
Época em que as fôlhas cáem das árvores.
\section{Desfólha}
\begin{itemize}
\item {Grp. gram.:f.}
\end{itemize}
O mesmo que \textunderscore desfolhação\textunderscore .
Época em que as fôlhas cáem das árvores.
\section{Desfolhação}
\begin{itemize}
\item {Grp. gram.:f.}
\end{itemize}
Acto de desfolhar.
Phylloptosia.
\section{Desfolhada}
\begin{itemize}
\item {Grp. gram.:f.}
\end{itemize}
\begin{itemize}
\item {Proveniência:(De \textunderscore desfolhar\textunderscore )}
\end{itemize}
O mesmo que \textunderscore descamisada\textunderscore .
\section{Desfolhador}
\begin{itemize}
\item {Grp. gram.:adj.}
\end{itemize}
\begin{itemize}
\item {Grp. gram.:M.}
\end{itemize}
Que desfolha.
Aquelle que desfolha.
\section{Desfolhadura}
\begin{itemize}
\item {Grp. gram.:f.}
\end{itemize}
(V.desfolhamento)
\section{Desfolhamento}
\begin{itemize}
\item {Grp. gram.:m.}
\end{itemize}
Acto de desfolhar.
\section{Desfolhar}
\begin{itemize}
\item {Grp. gram.:v. t.}
\end{itemize}
Tirar as folhas ou as pétalas a.
Descamisar (milho).
\section{Desfôlho}
\begin{itemize}
\item {Grp. gram.:m.}
\end{itemize}
(V.desfolhada)
\section{Desforçador}
\begin{itemize}
\item {Grp. gram.:m.}
\end{itemize}
Aquelle que desforça.
\section{Desforçamento}
\begin{itemize}
\item {Grp. gram.:m.}
\end{itemize}
Acto ou effeito de desforçar. Cf. Assis Teixeira, \textunderscore Águas\textunderscore .
\section{Desforçar}
\begin{itemize}
\item {Grp. gram.:v. t.}
\end{itemize}
\begin{itemize}
\item {Proveniência:(De \textunderscore des...\textunderscore  + \textunderscore forçar\textunderscore )}
\end{itemize}
Dar ou tomar satisfação de (uma afronta); vingar; indemnizar-se de.
\section{Desfôrço}
\begin{itemize}
\item {Grp. gram.:m.}
\end{itemize}
Acto de desforçar.
Desforra.
Vingança.
\section{Desformar}
\begin{itemize}
\item {Utilização:Mil.}
\end{itemize}
\textunderscore v. t.\textunderscore  (e der.)
O mesmo que \textunderscore deformar\textunderscore , etc. Cf. Camillo, \textunderscore Noites\textunderscore , VII, 87.
Sair ou tirar da linha de formatura: \textunderscore mandou desformar as tropas\textunderscore .
\section{Desformoso}
\begin{itemize}
\item {Grp. gram.:adj.}
\end{itemize}
Que não é formoso. Cf. Filinto, X, 278.
\section{Desforra}
\begin{itemize}
\item {Grp. gram.:f.}
\end{itemize}
Acto de desforrar.
Desfôrço.
\section{Desforrar}
\begin{itemize}
\item {Grp. gram.:v. t.}
\end{itemize}
\begin{itemize}
\item {Proveniência:(De \textunderscore des...\textunderscore  + \textunderscore forrar\textunderscore )}
\end{itemize}
Tirar o fôrro a.
Desforçar, vingar.
\section{Desforro}
\begin{itemize}
\item {fónica:fô}
\end{itemize}
\begin{itemize}
\item {Grp. gram.:m.}
\end{itemize}
\begin{itemize}
\item {Utilização:P. us.}
\end{itemize}
O mesmo que \textunderscore desforra\textunderscore . Cf. Filinto, IX, 85.
\section{Desfortalecer}
\begin{itemize}
\item {Grp. gram.:v. t.}
\end{itemize}
\begin{itemize}
\item {Proveniência:(De \textunderscore des...\textunderscore  + \textunderscore fortalecer\textunderscore )}
\end{itemize}
Tirar a fortaleza ou a fôrça a.
Desguarnecer. Cf. Filinto, \textunderscore D. Man.\textunderscore , I, 176.
\section{Desfortuna}
\begin{itemize}
\item {Grp. gram.:f.}
\end{itemize}
Falta de fortuna.
Infortúnio; desgraça.
\section{Desfortúnio}
\begin{itemize}
\item {Grp. gram.:m.}
\end{itemize}
O mesmo que \textunderscore infortúnio\textunderscore . Cf. Filinto, V, 276.
\section{Desfortunoso}
\begin{itemize}
\item {Grp. gram.:adj.}
\end{itemize}
\begin{itemize}
\item {Proveniência:(De \textunderscore des...\textunderscore  + \textunderscore fortunoso\textunderscore )}
\end{itemize}
Que não tem fortuna.
Desgraçado.
\section{Desfradar}
\begin{itemize}
\item {Grp. gram.:v. t.}
\end{itemize}
\begin{itemize}
\item {Proveniência:(De \textunderscore des...\textunderscore  + \textunderscore frade\textunderscore )}
\end{itemize}
Tirar a qualidade de frade a.
\section{Desfraldar}
\begin{itemize}
\item {Grp. gram.:v. t.}
\end{itemize}
\begin{itemize}
\item {Proveniência:(De \textunderscore des...\textunderscore  + \textunderscore fralda\textunderscore )}
\end{itemize}
Soltar ao vento: \textunderscore desfraldar bandeiras; desfraldar velas\textunderscore .
\section{Desfrangir}
\begin{itemize}
\item {Grp. gram.:v. t.}
\end{itemize}
\begin{itemize}
\item {Utilização:Des.}
\end{itemize}
Alterar a superfície de; desenrugar.
(Cp. \textunderscore desfranzir\textunderscore )
\section{Desfranjar}
\begin{itemize}
\item {Grp. gram.:v. t.}
\end{itemize}
\begin{itemize}
\item {Proveniência:(De \textunderscore des...\textunderscore  + \textunderscore franjar\textunderscore )}
\end{itemize}
Tirar a franja a.
\section{Desfranquear}
\begin{itemize}
\item {Grp. gram.:v. t.}
\end{itemize}
\begin{itemize}
\item {Proveniência:(De \textunderscore des...\textunderscore  + \textunderscore franquear\textunderscore )}
\end{itemize}
Tornar preso.
Acanhar, circunscrever. Cf. Filinto, XVIII, 218.
\section{Desfranzir}
\begin{itemize}
\item {Grp. gram.:v. t.}
\end{itemize}
\begin{itemize}
\item {Proveniência:(De \textunderscore des...\textunderscore  + \textunderscore franzir\textunderscore )}
\end{itemize}
Tirar o franzido a.
Tirar as pregas ou as rugas a: \textunderscore desfranzir a testa\textunderscore .
Abrir (o riso).
\section{Desfrear}
\begin{itemize}
\item {Grp. gram.:v. t.}
\end{itemize}
\begin{itemize}
\item {Proveniência:(De \textunderscore des...\textunderscore  + \textunderscore freio\textunderscore )}
\end{itemize}
O mesmo que \textunderscore desenfrear\textunderscore . Cf. Filinto, VI, 182.
\section{Desfrechar}
\begin{itemize}
\item {Grp. gram.:v. t.}
\end{itemize}
\begin{itemize}
\item {Proveniência:(De \textunderscore des...\textunderscore  + \textunderscore frechar\textunderscore )}
\end{itemize}
Atirar (setas, frechas).
Arremessar.
\section{Desfrequentado}
\begin{itemize}
\item {fónica:cu-en}
\end{itemize}
\begin{itemize}
\item {Grp. gram.:adj.}
\end{itemize}
Não frequentado.
Deserto.
\section{Desfrisante}
\begin{itemize}
\item {Grp. gram.:adj.}
\end{itemize}
\begin{itemize}
\item {Proveniência:(De \textunderscore desfrisar\textunderscore )}
\end{itemize}
Que não é frisante, que não quadra bem. Cf. Camillo, \textunderscore Cavar em Ruínas\textunderscore , 115.
\section{Desfrisar}
\begin{itemize}
\item {Grp. gram.:v. t.}
\end{itemize}
\begin{itemize}
\item {Proveniência:(De \textunderscore des...\textunderscore  + \textunderscore frisar\textunderscore )}
\end{itemize}
Desfazer o frisado de.
Esguedelhar.
Pôr em desalinho (os cabellos).
\section{Desfruidor}
\begin{itemize}
\item {fónica:fru-i}
\end{itemize}
\begin{itemize}
\item {Grp. gram.:adj.}
\end{itemize}
Que desfrue.
\section{Desfruir}
\begin{itemize}
\item {Grp. gram.:v. t.}
\end{itemize}
O mesmo que \textunderscore desfrutar\textunderscore  ou \textunderscore gozar\textunderscore . Cf. Camillo, \textunderscore Sereia\textunderscore , 8.
\section{Desfrutação}
\begin{itemize}
\item {Grp. gram.:f.}
\end{itemize}
O mesmo que \textunderscore desfrute\textunderscore . Cf. Castilho, \textunderscore Fausto\textunderscore , 143.
\section{Desfrutador}
\begin{itemize}
\item {Grp. gram.:adj.}
\end{itemize}
\begin{itemize}
\item {Grp. gram.:M.}
\end{itemize}
Que desfruta.
Aquelle que desfruta.
\section{Desfrutar}
\begin{itemize}
\item {Grp. gram.:v. t.}
\end{itemize}
\begin{itemize}
\item {Proveniência:(De \textunderscore des...\textunderscore  + \textunderscore fruto\textunderscore )}
\end{itemize}
Lograr os frutos de.
Viver á custa de.
Zombar de: \textunderscore desfrutar os parvos\textunderscore .
\section{Desabilidade}
\begin{itemize}
\item {Grp. gram.:f.}
\end{itemize}
Falta de habilidade.
\section{Desabilitar}
\begin{itemize}
\item {Grp. gram.:v. t.}
\end{itemize}
\begin{itemize}
\item {Proveniência:(De \textunderscore des...\textunderscore  + \textunderscore habilitar\textunderscore )}
\end{itemize}
Tornar inhábil ou inapto.
\section{Desabitar}
\begin{itemize}
\item {Grp. gram.:v. t}
\end{itemize}
\begin{itemize}
\item {Proveniência:(De \textunderscore des...\textunderscore  + \textunderscore habitar\textunderscore )}
\end{itemize}
Desamparar (casa ou lugar que se habitava).
\section{Desábito}
\begin{itemize}
\item {Grp. gram.:m.}
\end{itemize}
Falta de hábito; descostume.
\section{Desabituação}
\begin{itemize}
\item {Grp. gram.:f.}
\end{itemize}
Acto de desabituar.
\section{Desabituar}
\begin{itemize}
\item {Grp. gram.:v. t.}
\end{itemize}
\begin{itemize}
\item {Proveniência:(De \textunderscore des...\textunderscore  + \textunderscore habituar\textunderscore )}
\end{itemize}
Fazer perder o hábito ou o costume a.
\section{Desarmonia}
\begin{itemize}
\item {Grp. gram.:f.}
\end{itemize}
\begin{itemize}
\item {Utilização:Fig.}
\end{itemize}
Falta de harmonia.
Discordância.
Desproporção ou má disposição das partes que formam um todo.
\section{Desarmónico}
\begin{itemize}
\item {Grp. gram.:adj.}
\end{itemize}
Em que há desarmonia.
\section{Desarmonizador}
\begin{itemize}
\item {Grp. gram.:m.}
\end{itemize}
Aquelle que desarmoniza.
\section{Desarmonizar}
\begin{itemize}
\item {Grp. gram.:v. t.}
\end{itemize}
\begin{itemize}
\item {Proveniência:(De \textunderscore des...\textunderscore  + \textunderscore harmonizar\textunderscore )}
\end{itemize}
Produzir a desarmonia de ou entre.
\section{Deserdação}
\begin{itemize}
\item {Grp. gram.:f.}
\end{itemize}
Acto ou effeito de deserdar.
\section{Deserdar}
\begin{itemize}
\item {Grp. gram.:v. t.}
\end{itemize}
\begin{itemize}
\item {Proveniência:(De \textunderscore des...\textunderscore  + \textunderscore herdar\textunderscore )}
\end{itemize}
Privar do direito a uma herança.
Privar de benefícios que outros compartilham.
\section{Deshumanamente}
\begin{itemize}
\item {Grp. gram.:adv.}
\end{itemize}
De modo deshumano.
\section{Deshumanar}
\begin{itemize}
\item {Grp. gram.:v. t.}
\end{itemize}
\begin{itemize}
\item {Utilização:Des.}
\end{itemize}
\begin{itemize}
\item {Proveniência:(De \textunderscore des...\textunderscore  + \textunderscore humanar\textunderscore )}
\end{itemize}
Tornar deshumano.
\section{Deshumanidade}
\begin{itemize}
\item {Grp. gram.:f.}
\end{itemize}
Acto deshumano.
Falta de humanidade.
\section{Deshumanizar}
\begin{itemize}
\item {Grp. gram.:v. t.}
\end{itemize}
O mesmo que \textunderscore deshumanar\textunderscore .
\section{Deshumano}
\begin{itemize}
\item {Grp. gram.:adj.}
\end{itemize}
Que não é humano.
Que é ferino, bestial, cruel.
\section{Deshydratação}
\begin{itemize}
\item {Grp. gram.:f.}
\end{itemize}
Acto de deshydratar.
\section{Deshydratar}
\begin{itemize}
\item {Grp. gram.:v. t.}
\end{itemize}
\begin{itemize}
\item {Proveniência:(De \textunderscore des...\textunderscore  + \textunderscore hydratar\textunderscore )}
\end{itemize}
Separar a água de.
\section{Deshydrogenação}
\begin{itemize}
\item {Grp. gram.:f.}
\end{itemize}
Acto de deshydrogenar.
\section{Deshydrogenar}
\begin{itemize}
\item {Grp. gram.:v. t.}
\end{itemize}
\begin{itemize}
\item {Proveniência:(De \textunderscore des...\textunderscore  + \textunderscore hydrogenar\textunderscore )}
\end{itemize}
Separar o hydrogênio de.
\section{Desí}
\begin{itemize}
\item {Grp. gram.:adv.}
\end{itemize}
\begin{itemize}
\item {Utilização:Ant.}
\end{itemize}
Desde aí, desde então.
\section{Desiderandum}
\begin{itemize}
\item {fónica:derân}
\end{itemize}
\begin{itemize}
\item {Grp. gram.:m.}
\end{itemize}
\begin{itemize}
\item {Proveniência:(T. lat.)}
\end{itemize}
Aquillo que se deve desejar.
Aspiração justa. Cf. Castilho, \textunderscore Fastos\textunderscore , 310.
\section{Desiderativo}
\begin{itemize}
\item {Grp. gram.:adj.}
\end{itemize}
\begin{itemize}
\item {Proveniência:(Lat. \textunderscore desiderativus\textunderscore )}
\end{itemize}
Que exprime desejo.
\section{Desiderato}
\begin{itemize}
\item {Grp. gram.:m.}
\end{itemize}
O mesmo que \textunderscore desideratum\textunderscore .
\section{Desideratum}
\begin{itemize}
\item {fónica:derá}
\end{itemize}
\begin{itemize}
\item {Grp. gram.:m.}
\end{itemize}
\begin{itemize}
\item {Proveniência:(T. lat.)}
\end{itemize}
Aquillo que se deseja.
Aquillo a que se aspira; aspiração.
\section{Desídia}
\begin{itemize}
\item {Grp. gram.:f.}
\end{itemize}
\begin{itemize}
\item {Proveniência:(Lat. \textunderscore desidia\textunderscore )}
\end{itemize}
Preguiça; indolência.
\section{Desídio}
\begin{itemize}
\item {Grp. gram.:m.}
\end{itemize}
O mesmo que \textunderscore desídia\textunderscore . Cf. Latino, \textunderscore Elogios Acad.\textunderscore , I, 212.
\section{Desidiosamente}
\begin{itemize}
\item {Grp. gram.:adv.}
\end{itemize}
\begin{itemize}
\item {Proveniência:(De \textunderscore desidioso\textunderscore )}
\end{itemize}
Com desídia.
\section{Desidioso}
\begin{itemize}
\item {Grp. gram.:adj.}
\end{itemize}
\begin{itemize}
\item {Proveniência:(Lat. \textunderscore de idiosus\textunderscore )}
\end{itemize}
Que tem desídia.
\section{Desidratação}
\begin{itemize}
\item {Grp. gram.:f.}
\end{itemize}
Acto de desidratar.
\section{Desidratar}
\begin{itemize}
\item {Grp. gram.:v. t.}
\end{itemize}
\begin{itemize}
\item {Proveniência:(De \textunderscore des...\textunderscore  + \textunderscore hidratar\textunderscore )}
\end{itemize}
Separar a água de.
\section{Desidrogenação}
\begin{itemize}
\item {Grp. gram.:f.}
\end{itemize}
Acto de desidrogenar.
\section{Desidrogenar}
\begin{itemize}
\item {Grp. gram.:v. t.}
\end{itemize}
\begin{itemize}
\item {Proveniência:(De \textunderscore des...\textunderscore  + \textunderscore hidrogenar\textunderscore )}
\end{itemize}
Separar o hidrogênio de.
\section{Designação}
\begin{itemize}
\item {Grp. gram.:f.}
\end{itemize}
Acto de designar.
Significação.
Indicação.
\section{Designadamente}
\begin{itemize}
\item {Grp. gram.:adv.}
\end{itemize}
\begin{itemize}
\item {Proveniência:(De \textunderscore designar\textunderscore )}
\end{itemize}
Particularmente; de modo especificado.
\section{Designador}
\begin{itemize}
\item {Grp. gram.:adj.}
\end{itemize}
\begin{itemize}
\item {Grp. gram.:M.}
\end{itemize}
Que designa.
Aquelle que designa.
\section{Designar}
\begin{itemize}
\item {Grp. gram.:v. t.}
\end{itemize}
\begin{itemize}
\item {Proveniência:(Lat. \textunderscore designare\textunderscore )}
\end{itemize}
Apontar; indicar.
Assignalar.
Mostrar, significar.
Nomear.
\section{Designativo}
\begin{itemize}
\item {Grp. gram.:adj.}
\end{itemize}
\begin{itemize}
\item {Proveniência:(Lat. \textunderscore designativus\textunderscore )}
\end{itemize}
Próprio para designar.
Que designa.
\section{Desígnio}
\begin{itemize}
\item {Grp. gram.:m.}
\end{itemize}
\begin{itemize}
\item {Proveniência:(Do rad. do lat. \textunderscore designare\textunderscore )}
\end{itemize}
Intento.
Plano; projecto.
Tenção.
\section{Desigual}
\begin{itemize}
\item {Grp. gram.:adj.}
\end{itemize}
\begin{itemize}
\item {Proveniência:(De \textunderscore des...\textunderscore  + \textunderscore igual\textunderscore )}
\end{itemize}
Que não é igual.
Que faz differença: \textunderscore temperamentos desiguaes\textunderscore .
Variável: \textunderscore o tempo vai muito desigual\textunderscore .
Irregular.
Desproporcionado.
Injusto.
\section{Desigualança}
\begin{itemize}
\item {Grp. gram.:f.}
\end{itemize}
\begin{itemize}
\item {Utilização:Ant.}
\end{itemize}
Acto ou effeito de desigualar; desigualdade.
\section{Desigualar}
\begin{itemize}
\item {Grp. gram.:v. t.}
\end{itemize}
\begin{itemize}
\item {Grp. gram.:V. i.}
\end{itemize}
\begin{itemize}
\item {Proveniência:(De \textunderscore des...\textunderscore  + \textunderscore igualar\textunderscore )}
\end{itemize}
Tornar desigual.
Divergir.
\section{Desigualdade}
\begin{itemize}
\item {Grp. gram.:f.}
\end{itemize}
Estado daquelle ou daquillo que é desigual.
\section{Desigualmente}
\begin{itemize}
\item {Grp. gram.:adv.}
\end{itemize}
De modo desigual.
\section{Desilludido}
\begin{itemize}
\item {Grp. gram.:adj.}
\end{itemize}
\begin{itemize}
\item {Proveniência:(De \textunderscore desilludir\textunderscore )}
\end{itemize}
Que perdeu illusão.
A quem desilludiram.
\section{Desilludir}
\begin{itemize}
\item {Grp. gram.:v. t.}
\end{itemize}
\begin{itemize}
\item {Proveniência:(De \textunderscore des...\textunderscore  + \textunderscore illudir\textunderscore )}
\end{itemize}
Tirar illusão a.
Desenganar.
Causar decepção a.
\section{Desilluminado}
\begin{itemize}
\item {Grp. gram.:adj.}
\end{itemize}
O mesmo que \textunderscore desalumiado\textunderscore . Cf. Corvo, \textunderscore Anno na Côrte\textunderscore , c. II.
\section{Desillusão}
\begin{itemize}
\item {Grp. gram.:f.}
\end{itemize}
Perda de illusão.
Effeito de desilludir.
\section{Desilluso}
\begin{itemize}
\item {Grp. gram.:adj.}
\end{itemize}
O mesmo que \textunderscore desilludido\textunderscore .
\section{Desiludido}
\begin{itemize}
\item {Grp. gram.:adj.}
\end{itemize}
\begin{itemize}
\item {Proveniência:(De \textunderscore desiludir\textunderscore )}
\end{itemize}
Que perdeu ilusão.
A quem desiludiram.
\section{Desiludir}
\begin{itemize}
\item {Grp. gram.:v. t.}
\end{itemize}
\begin{itemize}
\item {Proveniência:(De \textunderscore des...\textunderscore  + \textunderscore iludir\textunderscore )}
\end{itemize}
Tirar ilusão a.
Desenganar.
Causar decepção a.
\section{Desiluminado}
\begin{itemize}
\item {Grp. gram.:adj.}
\end{itemize}
O mesmo que \textunderscore desalumiado\textunderscore . Cf. Corvo, \textunderscore Anno na Côrte\textunderscore , c. II.
\section{Desilusão}
\begin{itemize}
\item {Grp. gram.:f.}
\end{itemize}
Perda de ilusão.
Efeito de desiludir.
\section{Desiluso}
\begin{itemize}
\item {Grp. gram.:adj.}
\end{itemize}
O mesmo que \textunderscore desiludido\textunderscore .
\section{Desimaginar}
\begin{itemize}
\item {Grp. gram.:v. t.}
\end{itemize}
\begin{itemize}
\item {Proveniência:(De \textunderscore des...\textunderscore  + \textunderscore imaginar\textunderscore )}
\end{itemize}
Tirar da imaginação a.
Dissuadir.
\section{Desfrutável}
\begin{itemize}
\item {Grp. gram.:adv.}
\end{itemize}
Que se póde desfrutar; que se presta a zombarias.
\section{Desfrute}
\begin{itemize}
\item {Grp. gram.:m.}
\end{itemize}
Acto de desfrutar.
Zombaria.
\section{Desfruto}
\begin{itemize}
\item {Grp. gram.:m.}
\end{itemize}
O mesmo que \textunderscore desfrute\textunderscore .
\section{Desfunchar}
\begin{itemize}
\item {Grp. gram.:v.}
\end{itemize}
\begin{itemize}
\item {Utilização:t. Veter.}
\end{itemize}
\begin{itemize}
\item {Proveniência:(De \textunderscore funcho\textunderscore ?)}
\end{itemize}
Friccionar a espinha a (os cães), a fim de se não damnarem.
\section{Desfundar}
\begin{itemize}
\item {Grp. gram.:v. t.}
\end{itemize}
\begin{itemize}
\item {Proveniência:(De \textunderscore des...\textunderscore  + \textunderscore fundar\textunderscore )}
\end{itemize}
Tirar o fundo a.
\section{Desgabador}
\begin{itemize}
\item {Grp. gram.:m.}
\end{itemize}
Aquelle que desgaba.
\section{Desgabar}
\begin{itemize}
\item {Grp. gram.:v. t.}
\end{itemize}
\begin{itemize}
\item {Proveniência:(De \textunderscore des...\textunderscore  + \textunderscore gabar\textunderscore )}
\end{itemize}
Falar mal de; depreciar.
\section{Desgabo}
\begin{itemize}
\item {Grp. gram.:m.}
\end{itemize}
Acto de desgabar.
\section{Desgadelhar}
\textunderscore v. t.\textunderscore  (e der.)
O mesmo que \textunderscore desguedelhar\textunderscore , etc.
\section{Desgaira}
\begin{itemize}
\item {Grp. gram.:f. Loc. adv.}
\end{itemize}
\begin{itemize}
\item {Utilização:Prov.}
\end{itemize}
\begin{itemize}
\item {Utilização:trasm.}
\end{itemize}
\textunderscore Á desgaira\textunderscore , indifferentemente; sem dar importância a qualquer coisa que se faz.
\section{Desgalante}
\begin{itemize}
\item {Grp. gram.:adj.}
\end{itemize}
Que não é galante.
Que é descortês.
\section{Desgalgar}
\begin{itemize}
\item {Grp. gram.:v. t.}
\end{itemize}
\begin{itemize}
\item {Proveniência:(De \textunderscore des...\textunderscore  + \textunderscore galgar\textunderscore )}
\end{itemize}
Lançar por um declive.
Despenhar.
\section{Desgalhar}
\begin{itemize}
\item {Grp. gram.:v. t.}
\end{itemize}
\begin{itemize}
\item {Proveniência:(De \textunderscore des...\textunderscore  + \textunderscore galho\textunderscore )}
\end{itemize}
Cortar os galhos de.
\section{Desgarrada}
\begin{itemize}
\item {Grp. gram.:f.}
\end{itemize}
\begin{itemize}
\item {Proveniência:(De \textunderscore desgarrado\textunderscore )}
\end{itemize}
Cantiga popular, por desafio.
\section{Desgarradamente}
\begin{itemize}
\item {Grp. gram.:adv.}
\end{itemize}
\begin{itemize}
\item {Proveniência:(De \textunderscore desgarrado\textunderscore )}
\end{itemize}
Com desgarre.
\section{Desgarrado}
\begin{itemize}
\item {Grp. gram.:adj.}
\end{itemize}
\begin{itemize}
\item {Proveniência:(De \textunderscore desgarrar\textunderscore )}
\end{itemize}
Que se desgarrou.
Extraviado; erradio: \textunderscore andava desgarrado pelos montes\textunderscore .
Pervertido; devasso: \textunderscore uma criatura desgarrada\textunderscore .
\section{Desgarrão}
\begin{itemize}
\item {Grp. gram.:adj.}
\end{itemize}
\begin{itemize}
\item {Grp. gram.:M.}
\end{itemize}
\begin{itemize}
\item {Proveniência:(De \textunderscore desgarrar\textunderscore )}
\end{itemize}
Que desgarra violentamente.
Impulso violento.
\section{Desgarrar}
\begin{itemize}
\item {Grp. gram.:v. t.}
\end{itemize}
\begin{itemize}
\item {Grp. gram.:V. i.  e  p.}
\end{itemize}
\begin{itemize}
\item {Proveniência:(De \textunderscore des...\textunderscore  + \textunderscore garrar\textunderscore )}
\end{itemize}
Desviar de rumo (um navio).
Extraviar.
Perverter.
Tornar extravagante, excêntrico.
Desencaminhar-se.
Garrar.
Tresmalhar-se: \textunderscore o gado desgarrou\textunderscore .
\section{Desgarre}
\begin{itemize}
\item {Grp. gram.:m.}
\end{itemize}
Acto ou effeito de desgarrar.
Audácia; elegância; bizarria.
Canto á desgarrada; cantarola:«\textunderscore métrico desgarre\textunderscore ». Abbade de Jazente, II, 33.
\section{Desgarro}
\begin{itemize}
\item {Grp. gram.:m.}
\end{itemize}
(V.desgarre)
\section{Desgarronar}
\begin{itemize}
\item {Grp. gram.:v. t.}
\end{itemize}
\begin{itemize}
\item {Utilização:Bras. do S}
\end{itemize}
\begin{itemize}
\item {Proveniência:(De \textunderscore des...\textunderscore  + \textunderscore garrão\textunderscore )}
\end{itemize}
Cortar o garrão ou jarrete a (o cavallo).
\section{Desgastar}
\begin{itemize}
\item {Grp. gram.:v. t.}
\end{itemize}
\begin{itemize}
\item {Utilização:Pop.}
\end{itemize}
\begin{itemize}
\item {Proveniência:(De \textunderscore gastar\textunderscore )}
\end{itemize}
Gastar a pouco e pouco, consumir pelo attrito.
Digerir.
\section{Desgaste}
\begin{itemize}
\item {Grp. gram.:m.}
\end{itemize}
\begin{itemize}
\item {Grp. gram.:Adj.}
\end{itemize}
Acto de desgastar.
Desbastado ou gasto pelo uso.
\section{Desgasto}
\begin{itemize}
\item {Grp. gram.:m.}
\end{itemize}
\begin{itemize}
\item {Grp. gram.:Adj.}
\end{itemize}
Acto de desgastar.
Desbastado ou gasto pelo uso.
\section{Desgelador}
\begin{itemize}
\item {Grp. gram.:adj.}
\end{itemize}
Que desgela.
\section{Desgelar}
\begin{itemize}
\item {Grp. gram.:v. t.}
\end{itemize}
O mesmo que \textunderscore degelar\textunderscore .
\section{Desgorgomilado}
\begin{itemize}
\item {Grp. gram.:adj.}
\end{itemize}
\begin{itemize}
\item {Proveniência:(De \textunderscore des...\textunderscore  + \textunderscore gorgomilos\textunderscore )}
\end{itemize}
Que é glutão.
Perdulário.
\section{Desgorjado}
\begin{itemize}
\item {Grp. gram.:adj.}
\end{itemize}
O mesmo que [[esgorjado|esgorjar]].
\section{Desgornir}
\begin{itemize}
\item {Grp. gram.:v. t.}
\end{itemize}
\begin{itemize}
\item {Proveniência:(De \textunderscore des...\textunderscore  + \textunderscore gornir\textunderscore )}
\end{itemize}
Fazer saír do gorne.
\section{Desgostante}
\begin{itemize}
\item {Grp. gram.:adj.}
\end{itemize}
\begin{itemize}
\item {Proveniência:(De \textunderscore desgostar\textunderscore )}
\end{itemize}
Que causa desgôsto.
\section{Desgostar}
\begin{itemize}
\item {Grp. gram.:v. t.}
\end{itemize}
\begin{itemize}
\item {Proveniência:(De \textunderscore des...\textunderscore  + \textunderscore gostar\textunderscore )}
\end{itemize}
Causar desgôsto a.
Mortificar.
Fazer perder o gôsto.
\section{Desgôsto}
\begin{itemize}
\item {Grp. gram.:m.}
\end{itemize}
Ausência de gôsto ou prazer.
Pesar; descontentamento.
Repugnância.
\section{Desgostosamente}
\begin{itemize}
\item {Grp. gram.:adv.}
\end{itemize}
\begin{itemize}
\item {Proveniência:(De \textunderscore desgostoso\textunderscore )}
\end{itemize}
Com desgôsto.
\section{Desgostoso}
\begin{itemize}
\item {Grp. gram.:adj.}
\end{itemize}
\begin{itemize}
\item {Proveniência:(De \textunderscore des...\textunderscore  + \textunderscore gostoso\textunderscore )}
\end{itemize}
Que tem desgôsto: \textunderscore estou muito desgostoso\textunderscore .
Que desgosta.
Que tem mau sabor: \textunderscore pastel desgostoso\textunderscore .
\section{Desgovernação}
\begin{itemize}
\item {Grp. gram.:f.}
\end{itemize}
\begin{itemize}
\item {Utilização:Deprec.}
\end{itemize}
\begin{itemize}
\item {Proveniência:(De \textunderscore desgovernar\textunderscore )}
\end{itemize}
Falta de govêrno.
Mau govêrno, má administração.
\section{Desgovernadamente}
\begin{itemize}
\item {Grp. gram.:adv.}
\end{itemize}
\begin{itemize}
\item {Proveniência:(De \textunderscore desgovernado\textunderscore )}
\end{itemize}
Com desgovêrno.
Governando mal.
Com desperdício.
\section{Desgovernado}
\begin{itemize}
\item {Grp. gram.:adj.}
\end{itemize}
\begin{itemize}
\item {Proveniência:(De \textunderscore desgovernar\textunderscore )}
\end{itemize}
Que não tem govêrno.
Gastador; perdulário.
\section{Desgovernar}
\begin{itemize}
\item {Grp. gram.:v. t.}
\end{itemize}
\begin{itemize}
\item {Grp. gram.:V. i.}
\end{itemize}
\begin{itemize}
\item {Grp. gram.:V. p.}
\end{itemize}
\begin{itemize}
\item {Proveniência:(De \textunderscore des...\textunderscore  + \textunderscore governar\textunderscore )}
\end{itemize}
Governar mal.
Navegar sem govêrno (uma embarcação).
Desregrar-se; governar-se mal.
Sêr perdulário.
\section{Desgovêrno}
\begin{itemize}
\item {Grp. gram.:m.}
\end{itemize}
Falta de govêrno.
Mau govêrno.
Desregramento.
Esbanjamento.
\section{Desgraça}
\begin{itemize}
\item {Grp. gram.:f.}
\end{itemize}
\begin{itemize}
\item {Proveniência:(De \textunderscore des...\textunderscore  + \textunderscore graça\textunderscore )}
\end{itemize}
Falta de ventura.
Má fortuna; infortúnio.
Successo funesto.
Miséria.
Angústia.
Desaire.
Desfavor.
\section{Desgraçadamente}
\begin{itemize}
\item {Grp. gram.:adv.}
\end{itemize}
De modo desgraçado.
Infelizmente.
\section{Desgraçado}
\begin{itemize}
\item {Grp. gram.:adj.}
\end{itemize}
\begin{itemize}
\item {Grp. gram.:M.}
\end{itemize}
\begin{itemize}
\item {Proveniência:(De \textunderscore desgraçar\textunderscore )}
\end{itemize}
Infeliz.
Agoirento.
Lastimoso.
Miserável.
Inhábil: \textunderscore proceder de modo desgraçado\textunderscore .
Funesto.
Desprezível.
Despropositado.
Indivíduo muito pobre: \textunderscore tenham dó dos desgraçados\textunderscore .
Indivíduo desprezivel, pelo seu proceder.
\section{Desgraçar}
\begin{itemize}
\item {Grp. gram.:v. t.}
\end{itemize}
Causar desgraça a.
Tornar infeliz.
\section{Desgrácia}
\begin{itemize}
\item {Grp. gram.:f.}
\end{itemize}
(Fórma pop. de \textunderscore desgraça\textunderscore )
\section{Desgraciado}
\begin{itemize}
\item {Grp. gram.:adj.}
\end{itemize}
\begin{itemize}
\item {Utilização:P. us.}
\end{itemize}
O mesmo que \textunderscore desgraçado\textunderscore .
O mesmo que \textunderscore desgracioso\textunderscore :«\textunderscore dizia-se que a freira estava completamente acalcanhada e desgraciada.\textunderscore »Camillo, \textunderscore Caveira\textunderscore , 276.
\section{Desgraciamento}
\begin{itemize}
\item {Grp. gram.:m.}
\end{itemize}
O mesmo que \textunderscore desgraça\textunderscore . Cf. Castilho, \textunderscore Fastos\textunderscore , I, 183.
\section{Desgraciar}
\begin{itemize}
\item {Grp. gram.:v. t.}
\end{itemize}
\begin{itemize}
\item {Proveniência:(De \textunderscore desgrácia\textunderscore )}
\end{itemize}
Desgraçar.
Lamentar a desgraça de.
Lamentar. Cf. Rui Barb., 157.
\section{Desgracioso}
\begin{itemize}
\item {Grp. gram.:adj.}
\end{itemize}
\begin{itemize}
\item {Proveniência:(De \textunderscore des...\textunderscore  + \textunderscore gracioso\textunderscore )}
\end{itemize}
Que não tem graça.
Falta de elegância; desajeitado.
\section{Desgraduar}
\begin{itemize}
\item {Grp. gram.:v. t.}
\end{itemize}
O mesmo que \textunderscore degradar\textunderscore ^1.
\section{Desgranar}
\begin{itemize}
\item {Grp. gram.:v. t.}
\end{itemize}
\begin{itemize}
\item {Proveniência:(De \textunderscore des...\textunderscore  + lat. \textunderscore granum\textunderscore )}
\end{itemize}
Tirar levemente as rugosidades a (um objecto que se vai doirar).
\section{Desgravidação}
\begin{itemize}
\item {Grp. gram.:f.}
\end{itemize}
\begin{itemize}
\item {Proveniência:(De \textunderscore desgravidar\textunderscore )}
\end{itemize}
Acto de deixar de estar grávida (a mulher).
Parto. Cf. Filinto, VIII, 275 e 258.
\section{Desgravidar}
\begin{itemize}
\item {Grp. gram.:v. t.}
\end{itemize}
\begin{itemize}
\item {Grp. gram.:V. i.}
\end{itemize}
\begin{itemize}
\item {Proveniência:(De \textunderscore des...\textunderscore  + \textunderscore grávido\textunderscore )}
\end{itemize}
Tirar a gravidez a.
Dar á luz, parir. Cf. Filinto, IX, 115.
\section{Desgraxamento}
\begin{itemize}
\item {Grp. gram.:m.}
\end{itemize}
Acto de desgraxar.
\section{Desgraxar}
\begin{itemize}
\item {Grp. gram.:v. t.}
\end{itemize}
Tirar a gordura a. Cf. \textunderscore Diário do Governo\textunderscore  de 24-IX-903.
\section{Desgregar}
\textunderscore v. t.\textunderscore  (e der.)
O mesmo que \textunderscore desaggregar\textunderscore , etc.
\section{Desgrenhado}
\begin{itemize}
\item {Grp. gram.:adj.}
\end{itemize}
\begin{itemize}
\item {Utilização:Fig.}
\end{itemize}
\begin{itemize}
\item {Proveniência:(De \textunderscore desgrenhar\textunderscore )}
\end{itemize}
Diz-se do cabello despenteado ou revolto.
Que traz o cabello revolto ou emmaranhado.
Desordenado, irregular, no falar ou escrever.
\section{Desgrenhamento}
\begin{itemize}
\item {Grp. gram.:m.}
\end{itemize}
Acto ou effeito de desgrenhar.
\section{Desgrenhar}
\begin{itemize}
\item {Grp. gram.:v. t.}
\end{itemize}
\begin{itemize}
\item {Proveniência:(De \textunderscore des...\textunderscore  + \textunderscore grenha\textunderscore )}
\end{itemize}
Despentear.
Desconcertar (a grenha, o cabello).
\section{Desgrilhoar}
\begin{itemize}
\item {Grp. gram.:v. t.}
\end{itemize}
(V.desagrilhoar)
\section{Desgrinaldar}
\begin{itemize}
\item {Grp. gram.:v. t.}
\end{itemize}
Tirar grinalda a.
\section{Desgrojado}
\begin{itemize}
\item {Grp. gram.:adj.}
\end{itemize}
(V.desgorjado)
\section{Desgrudar}
\begin{itemize}
\item {Grp. gram.:v. t.}
\end{itemize}
\begin{itemize}
\item {Proveniência:(De \textunderscore des...\textunderscore  + \textunderscore grudar\textunderscore )}
\end{itemize}
Desligar (aquillo que estava grudado).
\section{Desgrumar}
\begin{itemize}
\item {Grp. gram.:v. t.}
\end{itemize}
\begin{itemize}
\item {Proveniência:(De \textunderscore des...\textunderscore  + \textunderscore grumar\textunderscore )}
\end{itemize}
Desfazer os grumos de.
\section{Desguardar}
\begin{itemize}
\item {Grp. gram.:v. t.}
\end{itemize}
\begin{itemize}
\item {Proveniência:(De \textunderscore des...\textunderscore  + \textunderscore guardar\textunderscore )}
\end{itemize}
Não guardar.
\section{Desguaritar}
\begin{itemize}
\item {Grp. gram.:v. t.}
\end{itemize}
\begin{itemize}
\item {Utilização:Bras}
\end{itemize}
\begin{itemize}
\item {Proveniência:(De \textunderscore des...\textunderscore  + \textunderscore guarita\textunderscore )}
\end{itemize}
Desviar do bando, tresmalhar.
\section{Desguarnecer}
\begin{itemize}
\item {Grp. gram.:v. t.}
\end{itemize}
\begin{itemize}
\item {Proveniência:(De \textunderscore des...\textunderscore  + \textunderscore guarnecer\textunderscore )}
\end{itemize}
Privar de guarnição.
Privar de fórças militares ou de munições de guerra.
Tirar os enfeites a.
Desmobilar: \textunderscore desguarnecer uma casa\textunderscore .
\section{Desguedelhar}
\begin{itemize}
\item {Grp. gram.:v. t.}
\end{itemize}
\begin{itemize}
\item {Proveniência:(De \textunderscore des...\textunderscore  + \textunderscore guedelha\textunderscore )}
\end{itemize}
O mesmo que \textunderscore desgrenhar\textunderscore .
\section{Deshabilidade}
\begin{itemize}
\item {Grp. gram.:f.}
\end{itemize}
Falta de habilidade.
\section{Deshabilitar}
\begin{itemize}
\item {Grp. gram.:v. t.}
\end{itemize}
\begin{itemize}
\item {Proveniência:(De \textunderscore des...\textunderscore  + \textunderscore habilitar\textunderscore )}
\end{itemize}
Tornar inhábil ou inapto.
\section{Deshabitar}
\begin{itemize}
\item {Grp. gram.:v. t}
\end{itemize}
\begin{itemize}
\item {Proveniência:(De \textunderscore des...\textunderscore  + \textunderscore habitar\textunderscore )}
\end{itemize}
Desamparar (casa ou lugar que se habitava).
\section{Deshábito}
\begin{itemize}
\item {Grp. gram.:m.}
\end{itemize}
Falta de hábito; descostume.
\section{Deshabituação}
\begin{itemize}
\item {Grp. gram.:f.}
\end{itemize}
Acto de deshabituar.
\section{Deshabituar}
\begin{itemize}
\item {Grp. gram.:v. t.}
\end{itemize}
\begin{itemize}
\item {Proveniência:(De \textunderscore des...\textunderscore  + \textunderscore habituar\textunderscore )}
\end{itemize}
Fazer perder o hábito ou o costume a.
\section{Desharmonia}
\begin{itemize}
\item {Grp. gram.:f.}
\end{itemize}
\begin{itemize}
\item {Utilização:Fig.}
\end{itemize}
Falta de harmonia.
Discordância.
Desproporção ou má disposição das partes que formam um todo.
\section{Desharmónico}
\begin{itemize}
\item {Grp. gram.:adj.}
\end{itemize}
Em que há desharmonia.
\section{Desharmonizador}
\begin{itemize}
\item {Grp. gram.:m.}
\end{itemize}
Aquelle que desharmoniza.
\section{Desharmonizar}
\begin{itemize}
\item {Grp. gram.:v. t.}
\end{itemize}
\begin{itemize}
\item {Proveniência:(De \textunderscore des...\textunderscore  + \textunderscore harmonizar\textunderscore )}
\end{itemize}
Produzir a desharmonia de ou entre.
\section{Desherdação}
\begin{itemize}
\item {Grp. gram.:f.}
\end{itemize}
Acto ou effeito de desherdar.
\section{Desherdar}
\begin{itemize}
\item {Grp. gram.:v. t.}
\end{itemize}
\begin{itemize}
\item {Proveniência:(De \textunderscore des...\textunderscore  + \textunderscore herdar\textunderscore )}
\end{itemize}
Privar do direito a uma herança.
Privar de benefícios que outros compartilham.
\section{Deshonestamente}
\begin{itemize}
\item {Grp. gram.:adv.}
\end{itemize}
De modo deshonesto.
\section{Deshonestar}
\begin{itemize}
\item {Grp. gram.:v. t.}
\end{itemize}
\begin{itemize}
\item {Utilização:Ant.}
\end{itemize}
\begin{itemize}
\item {Proveniência:(Do lat. \textunderscore dehonestare\textunderscore )}
\end{itemize}
Deshonrar.
Injuriar.
\section{Deshonestidade}
\begin{itemize}
\item {Grp. gram.:f.}
\end{itemize}
\begin{itemize}
\item {Proveniência:(De \textunderscore des...\textunderscore  + \textunderscore honestidade\textunderscore )}
\end{itemize}
Falta de honestidade.
Impudência.
\section{Deshonesto}
\begin{itemize}
\item {Grp. gram.:m.}
\end{itemize}
\begin{itemize}
\item {Proveniência:(De \textunderscore des...\textunderscore  + \textunderscore honesto\textunderscore )}
\end{itemize}
Que não tem honestidade; impudico; devasso.
\section{Deshonra}
\begin{itemize}
\item {Grp. gram.:f.}
\end{itemize}
\begin{itemize}
\item {Proveniência:(De \textunderscore des...\textunderscore  + \textunderscore honra\textunderscore )}
\end{itemize}
Falta de honra.
Perda de honra.
Descrédito.
\section{Deshonradamente}
\begin{itemize}
\item {Proveniência:(De \textunderscore des...\textunderscore  + \textunderscore honradamente\textunderscore )}
\end{itemize}
Com deshonra.
\section{Deshonradez}
\begin{itemize}
\item {Grp. gram.:f.}
\end{itemize}
Estado de deshonrado.
Deshonra.
\section{Deshonradiço}
\begin{itemize}
\item {Grp. gram.:adj.}
\end{itemize}
\begin{itemize}
\item {Utilização:Ant.}
\end{itemize}
\begin{itemize}
\item {Proveniência:(De \textunderscore deshonrar\textunderscore )}
\end{itemize}
Que deshonra ou offende.
\section{Deshonrador}
\begin{itemize}
\item {Grp. gram.:adj.}
\end{itemize}
\begin{itemize}
\item {Grp. gram.:M.}
\end{itemize}
Que deshonra.
Aquelle que deshonra.
\section{Deshonrante}
\begin{itemize}
\item {Grp. gram.:adj.}
\end{itemize}
Que deshonra, que avilta.
\section{Deshonrar}
\begin{itemize}
\item {Grp. gram.:v. t.}
\end{itemize}
\begin{itemize}
\item {Proveniência:(De \textunderscore des...\textunderscore  + \textunderscore honrar\textunderscore )}
\end{itemize}
Offender a honra, o pudor ou o crédito, de.
Desflorar.
Infamar; deslustrar.
\section{Deshonrosamente}
\begin{itemize}
\item {Grp. gram.:adv.}
\end{itemize}
De modo deshonroso.
\section{Deshonroso}
\begin{itemize}
\item {Grp. gram.:adj.}
\end{itemize}
\begin{itemize}
\item {Proveniência:(De \textunderscore des...\textunderscore  + \textunderscore honroso\textunderscore )}
\end{itemize}
Que deshonra.
Em que há deshonra.
\section{Deshorado}
\begin{itemize}
\item {Grp. gram.:adj.}
\end{itemize}
\begin{itemize}
\item {Proveniência:(De \textunderscore deshoras\textunderscore )}
\end{itemize}
Que vem fóra de horas.
Intempestivo.
Inopportuno:«\textunderscore tratão-no de cruel e deshorado inimigo...\textunderscore »Filinto, \textunderscore D. Man.\textunderscore , I, 342.
\section{Deshoras}
\begin{itemize}
\item {Grp. gram.:f. pl.}
\end{itemize}
\begin{itemize}
\item {Proveniência:(De \textunderscore des...\textunderscore  + \textunderscore hora\textunderscore )}
\end{itemize}
(Só us. na \textunderscore loc. adv.\textunderscore  \textunderscore a deshoras\textunderscore , por fôra de horas, inopportunamente, tarde)
\section{Desencarnado}
\begin{itemize}
\item {Grp. gram.:adj.}
\end{itemize}
\begin{itemize}
\item {Utilização:Espir.}
\end{itemize}
Diz-se do espírito, separado da carne e que êlle animou. Cf. M. Velho, \textunderscore Man. do Espirita\textunderscore .
\section{Desimpedidamente}
\begin{itemize}
\item {Grp. gram.:adv.}
\end{itemize}
\begin{itemize}
\item {Proveniência:(De \textunderscore desimpedir\textunderscore )}
\end{itemize}
Sem impedimento.
\section{Desimpedimento}
\begin{itemize}
\item {Grp. gram.:m.}
\end{itemize}
Acto de desimpedir.
\section{Desimpedir}
\begin{itemize}
\item {Grp. gram.:v. t.}
\end{itemize}
\begin{itemize}
\item {Proveniência:(De \textunderscore des...\textunderscore  + \textunderscore impedir\textunderscore )}
\end{itemize}
Tirar o impedimento a.
Desembaraçar.
Desatravancar; desobstruir: \textunderscore desimpedir o caminho\textunderscore .
\section{Desimplicar}
\begin{itemize}
\item {Grp. gram.:v. t.}
\end{itemize}
\begin{itemize}
\item {Proveniência:(De \textunderscore des...\textunderscore  + \textunderscore implicar\textunderscore )}
\end{itemize}
Separar (aquelle ou aquillo que estava implicado).
\section{Desimprensar}
\begin{itemize}
\item {Grp. gram.:v. t.}
\end{itemize}
\begin{itemize}
\item {Proveniência:(De \textunderscore des...\textunderscore  + \textunderscore imprensar\textunderscore )}
\end{itemize}
Tirar a estampagem a (os panos).
\section{Desimpressionar}
\begin{itemize}
\item {Grp. gram.:v. t.}
\end{itemize}
\begin{itemize}
\item {Proveniência:(De \textunderscore des...\textunderscore  + \textunderscore impressionar\textunderscore )}
\end{itemize}
Fazer desvanecer uma impressão moral a.
\section{Desimpureza}
\begin{itemize}
\item {Grp. gram.:f.}
\end{itemize}
Falta de pureza. Cf. Castilho, \textunderscore Metam.\textunderscore , XIV.
\section{Desinçar}
\begin{itemize}
\item {Grp. gram.:v. t.}
\end{itemize}
\begin{itemize}
\item {Proveniência:(De \textunderscore des...\textunderscore  + \textunderscore inçar\textunderscore )}
\end{itemize}
Livrar de coisas ou pessôas nocivas.
\section{Desincarnado}
\begin{itemize}
\item {Grp. gram.:adj.}
\end{itemize}
\begin{itemize}
\item {Utilização:Espir.}
\end{itemize}
Diz-se do espírito, separado da carne e que êlle animou. Cf. M. Velho, \textunderscore Man. do Espirita\textunderscore .
\section{Desinchação}
\begin{itemize}
\item {Grp. gram.:f.}
\end{itemize}
Acto ou effeito de desinchar.
\section{Desinchar}
\begin{itemize}
\item {Grp. gram.:v. t.}
\end{itemize}
\begin{itemize}
\item {Utilização:Fig.}
\end{itemize}
\begin{itemize}
\item {Grp. gram.:V. i.  e  p.}
\end{itemize}
\begin{itemize}
\item {Proveniência:(De \textunderscore des...\textunderscore  + \textunderscore inchar\textunderscore )}
\end{itemize}
Tirar ou deminuir a inchação a.
Humilhar.
Deixar de estar inchado: \textunderscore a perna desinchou\textunderscore .
\section{Desinclinação}
\begin{itemize}
\item {Grp. gram.:f.}
\end{itemize}
Acto ou effeito de desinclinar.
\section{Desinclinar}
\begin{itemize}
\item {Grp. gram.:v. t.}
\end{itemize}
\begin{itemize}
\item {Proveniência:(De \textunderscore des...\textunderscore  + \textunderscore inclinar\textunderscore )}
\end{itemize}
Tirar a inclinação a.
Aprumar.
\section{Desinço}
\begin{itemize}
\item {Grp. gram.:m.}
\end{itemize}
\begin{itemize}
\item {Utilização:Prov.}
\end{itemize}
\begin{itemize}
\item {Utilização:Prov.}
\end{itemize}
\begin{itemize}
\item {Utilização:trasm.}
\end{itemize}
Acto de desinçar.
Coisa ou pessôa, que estraga ou destrói.
Pequeno pente, para desinçar os piolhos da cabeça.
\section{Desincompatibilizar}
\begin{itemize}
\item {Grp. gram.:v. t.}
\end{itemize}
Tirar incompatibilidade a.
\section{Desindiciar}
\begin{itemize}
\item {Grp. gram.:v.}
\end{itemize}
\begin{itemize}
\item {Utilização:t. Jur.}
\end{itemize}
\begin{itemize}
\item {Proveniência:(De \textunderscore des...\textunderscore  + \textunderscore indiciar\textunderscore )}
\end{itemize}
Declarar que (alguém) não deve sêr processado criminalmente.
\section{Desinência}
\begin{itemize}
\item {Grp. gram.:f.}
\end{itemize}
\begin{itemize}
\item {Utilização:Gram.}
\end{itemize}
\begin{itemize}
\item {Utilização:Gram.}
\end{itemize}
\begin{itemize}
\item {Proveniência:(Do lat. \textunderscore desinere\textunderscore )}
\end{itemize}
Letra ou sýllaba que, posposta ao radical das palavras, as termina.
Flexão dos verbos e dos nomes.
Fim; extremidade.
\section{Desinfamar}
\begin{itemize}
\item {Grp. gram.:v. t.}
\end{itemize}
Limpar da infâmia.
Rehabilitar moralmente. Cf. Camillo, \textunderscore Cav. em Ruínas\textunderscore , 94.
\section{Desinfecção}
\begin{itemize}
\item {Grp. gram.:f.}
\end{itemize}
Acto ou effeito de desinfectar.
\section{Desinfeccionar}
\begin{itemize}
\item {Proveniência:(De \textunderscore desinfecção\textunderscore )}
\end{itemize}
\textunderscore v. t.\textunderscore  (e der.)
O mesmo que \textunderscore desinficionar\textunderscore , etc.
\section{Desinfectador}
\begin{itemize}
\item {Grp. gram.:m.}
\end{itemize}
\begin{itemize}
\item {Grp. gram.:Adj.}
\end{itemize}
Apparelho, que desinfecta.
O mesmo que \textunderscore desinfectante\textunderscore .
\section{Desinfectante}
\begin{itemize}
\item {Grp. gram.:adj.}
\end{itemize}
\begin{itemize}
\item {Grp. gram.:M.}
\end{itemize}
Que desinfecta.
Substância que desinfecta: \textunderscore a latrina precisa desinfectantes\textunderscore .
\section{Desinfectar}
\begin{itemize}
\item {Grp. gram.:v. t.}
\end{itemize}
\begin{itemize}
\item {Proveniência:(De \textunderscore des...\textunderscore  + \textunderscore infectar\textunderscore )}
\end{itemize}
Livrar daquillo que infecta; sanear.
\section{Desinfectório}
\begin{itemize}
\item {Grp. gram.:m.}
\end{itemize}
\begin{itemize}
\item {Utilização:Bras}
\end{itemize}
\begin{itemize}
\item {Proveniência:(De \textunderscore desinfectar\textunderscore )}
\end{itemize}
Lugar, onde se fazem desinfecções; pôsto de desinfecção.
\section{Desinfelicidade}
\begin{itemize}
\item {Grp. gram.:f.}
\end{itemize}
\begin{itemize}
\item {Utilização:Pop.}
\end{itemize}
Qualidade de desinfeliz.
\section{Desinfeliz}
\begin{itemize}
\item {Grp. gram.:adj.}
\end{itemize}
\begin{itemize}
\item {Utilização:Pop.}
\end{itemize}
\begin{itemize}
\item {Proveniência:(De formação análoga á de \textunderscore desinquieto\textunderscore )}
\end{itemize}
O mesmo que \textunderscore infeliz\textunderscore .
\section{Desinfestar}
\begin{itemize}
\item {Grp. gram.:v. t.}
\end{itemize}
\begin{itemize}
\item {Proveniência:(De \textunderscore des...\textunderscore  + \textunderscore infestar\textunderscore )}
\end{itemize}
Livrar daquillo que infesta.
\section{Desinficionar}
\begin{itemize}
\item {Grp. gram.:v. t.}
\end{itemize}
\begin{itemize}
\item {Proveniência:(De \textunderscore des...\textunderscore  + \textunderscore inficionar\textunderscore )}
\end{itemize}
O mesmo que \textunderscore desinfectar\textunderscore .
\section{Desinflamação}
\begin{itemize}
\item {Grp. gram.:f.}
\end{itemize}
Acto ou efeito de desinflamar.
\section{Desinflamar}
\begin{itemize}
\item {Grp. gram.:v. t.}
\end{itemize}
\begin{itemize}
\item {Proveniência:(De \textunderscore des...\textunderscore  + \textunderscore inflamar\textunderscore )}
\end{itemize}
Tirar a inflamação a.
Fazer deminuir a inflamação de.
\section{Desinflammação}
\begin{itemize}
\item {Grp. gram.:f.}
\end{itemize}
Acto ou effeito de desinflammar.
\section{Desinflammar}
\begin{itemize}
\item {Grp. gram.:v. t.}
\end{itemize}
\begin{itemize}
\item {Proveniência:(De \textunderscore des...\textunderscore  + \textunderscore inflammar\textunderscore )}
\end{itemize}
Tirar a inflammação a.
Fazer deminuir a inflammação de.
\section{Desinquietação}
\begin{itemize}
\item {Grp. gram.:f.}
\end{itemize}
Acto de desinquietar.
\section{Desinquietador}
\begin{itemize}
\item {Grp. gram.:adj.}
\end{itemize}
\begin{itemize}
\item {Grp. gram.:M.}
\end{itemize}
Que desinquieta.
Aquelle que desinquieta.
\section{Desinquietar}
\begin{itemize}
\item {Grp. gram.:v. t.}
\end{itemize}
\begin{itemize}
\item {Utilização:Pop.}
\end{itemize}
\begin{itemize}
\item {Proveniência:(De \textunderscore desinquieto\textunderscore )}
\end{itemize}
O mesmo que \textunderscore inquietar\textunderscore .
\section{Desinquieto}
\begin{itemize}
\item {Grp. gram.:adj.}
\end{itemize}
\begin{itemize}
\item {Utilização:Fam.}
\end{itemize}
\begin{itemize}
\item {Proveniência:(De \textunderscore des...\textunderscore , pref. de realce ou alter. do pref. \textunderscore tres...\textunderscore  = \textunderscore tri\textunderscore , e \textunderscore inquieto\textunderscore )}
\end{itemize}
Inquieto; traquinas.
\section{Desinsculpir}
\begin{itemize}
\item {Grp. gram.:v. t.}
\end{itemize}
\begin{itemize}
\item {Proveniência:(De \textunderscore des...\textunderscore  + \textunderscore insculpir\textunderscore )}
\end{itemize}
Desfazer a esculptura de.
Expungir. Cf. Filinto, XXI, 52.
\section{Desinsoffrido}
\begin{itemize}
\item {Grp. gram.:adj.}
\end{itemize}
\begin{itemize}
\item {Utilização:Prov.}
\end{itemize}
\begin{itemize}
\item {Utilização:alent.}
\end{itemize}
O mesmo que \textunderscore insoffrido\textunderscore .
(Cp. \textunderscore desinquieto\textunderscore )
\section{Desinsofrido}
\begin{itemize}
\item {Grp. gram.:adj.}
\end{itemize}
\begin{itemize}
\item {Utilização:Prov.}
\end{itemize}
\begin{itemize}
\item {Utilização:alent.}
\end{itemize}
O mesmo que \textunderscore insofrido\textunderscore .
(Cp. \textunderscore desinquieto\textunderscore )
\section{Desintegração}
\begin{itemize}
\item {Grp. gram.:f.}
\end{itemize}
Acto de desintegrar.
\section{Desintegrar}
\begin{itemize}
\item {Grp. gram.:v. t.}
\end{itemize}
\begin{itemize}
\item {Proveniência:(De \textunderscore des...\textunderscore  + \textunderscore integrar\textunderscore )}
\end{itemize}
Tirar a integração de.
Separar de um todo; insular.
Tirar a qualidade de integral a.
\section{Desintelligência}
\begin{itemize}
\item {Grp. gram.:f.}
\end{itemize}
\begin{itemize}
\item {Proveniência:(De \textunderscore des...\textunderscore  + \textunderscore intelligência\textunderscore )}
\end{itemize}
Divergência; inimizade: \textunderscore há desintelligência entre os irmãos\textunderscore .
\section{Desintencionado}
\begin{itemize}
\item {Grp. gram.:adj.}
\end{itemize}
O mesmo que \textunderscore desintencional\textunderscore .
\section{Desintencional}
\begin{itemize}
\item {Grp. gram.:adj.}
\end{itemize}
\begin{itemize}
\item {Proveniência:(De \textunderscore des...\textunderscore  + \textunderscore intencional\textunderscore )}
\end{itemize}
Não intencional; involuntário.
\section{Desinteressadamente}
\begin{itemize}
\item {Grp. gram.:adj.}
\end{itemize}
De modo desinteressado.
Sem interesse.
\section{Desinteressado}
\begin{itemize}
\item {Grp. gram.:adj.}
\end{itemize}
Que não tem interesse.
Que tem abnegação.
Incorruptível; imparcial.
\section{Desinteressante}
\begin{itemize}
\item {Grp. gram.:adj.}
\end{itemize}
Que não é interessante.
\section{Desinteressar}
\begin{itemize}
\item {Grp. gram.:v. t.}
\end{itemize}
\begin{itemize}
\item {Grp. gram.:V. p.}
\end{itemize}
\begin{itemize}
\item {Proveniência:(De \textunderscore des...\textunderscore  + \textunderscore interessar\textunderscore )}
\end{itemize}
Tirar interesse a.
Tirar lucros a.
Não fazer empenho, não têr interesse: \textunderscore desinteresso-me da questão\textunderscore .
\section{Desinterésse}
\begin{itemize}
\item {Grp. gram.:m.}
\end{itemize}
Falta de interesse.
Abnegação; generosidade.
\section{Desinterêsse}
\begin{itemize}
\item {Grp. gram.:m.}
\end{itemize}
Falta de interesse.
Abnegação; generosidade.
\section{Desinteresseiro}
\begin{itemize}
\item {Grp. gram.:adj.}
\end{itemize}
Que tem desinteresse.
Que não é interesseiro.
\section{Desinternar}
\begin{itemize}
\item {Grp. gram.:v. t.}
\end{itemize}
\begin{itemize}
\item {Proveniência:(De \textunderscore des...\textunderscore  + \textunderscore internar\textunderscore )}
\end{itemize}
Fazer saír do interior.
Tirar a qualidade de interno a: \textunderscore desinternar um colleqial\textunderscore .
\section{Desintestinar}
\begin{itemize}
\item {Grp. gram.:v. t.}
\end{itemize}
\begin{itemize}
\item {Proveniência:(De \textunderscore des...\textunderscore  + \textunderscore intestino\textunderscore )}
\end{itemize}
Tirar dos intestinos. Cf. Camillo, \textunderscore Amor de Salvação\textunderscore , 220.
\section{Desintrincar}
\begin{itemize}
\item {Grp. gram.:v. t.}
\end{itemize}
\begin{itemize}
\item {Proveniência:(De \textunderscore des...\textunderscore  + \textunderscore intrincado\textunderscore )}
\end{itemize}
Desemmaranhar.
Tornar simples, claro. Cf. Filinto, III, 97.
\section{Desinvejoso}
\begin{itemize}
\item {Grp. gram.:adj.}
\end{itemize}
\begin{itemize}
\item {Proveniência:(De \textunderscore des...\textunderscore  + \textunderscore invejoso\textunderscore )}
\end{itemize}
Que não tem inveja; que não revela inveja.
\section{Desinvernar}
\begin{itemize}
\item {Grp. gram.:v. i.}
\end{itemize}
\begin{itemize}
\item {Proveniência:(De \textunderscore des...\textunderscore  + \textunderscore invernar\textunderscore )}
\end{itemize}
Perder a aspereza do inverno.
Deixar quartéis de inverno.
\section{Desinvestir}
\begin{itemize}
\item {Grp. gram.:v. t.}
\end{itemize}
\begin{itemize}
\item {Proveniência:(De \textunderscore des...\textunderscore  + \textunderscore investir\textunderscore )}
\end{itemize}
Tirar a investidura a.
Exonerar, demittir.
\section{Desinviolar}
\begin{itemize}
\item {Grp. gram.:v. t.}
\end{itemize}
\begin{itemize}
\item {Utilização:Ant.}
\end{itemize}
Tornar honesto e respeitável (o que dantes o não era).
Restabelecer o crédito de.
(Palavra, de formação irregular, como \textunderscore desinfeliz, desinquieto\textunderscore , etc.)
\section{Desirmanadamente}
\begin{itemize}
\item {Grp. gram.:adv.}
\end{itemize}
Á maneira de desirmanado.
\section{Desirmanado}
\begin{itemize}
\item {Grp. gram.:adj.}
\end{itemize}
\begin{itemize}
\item {Proveniência:(De \textunderscore desirmanar\textunderscore )}
\end{itemize}
Que se separou ou desuniu de coisa ou pessôa, com que estava emparelhado.
\section{Desirmanar}
\begin{itemize}
\item {Grp. gram.:v. t.}
\end{itemize}
\begin{itemize}
\item {Proveniência:(De \textunderscore des...\textunderscore  + \textunderscore irmanar\textunderscore )}
\end{itemize}
Separar (aquillo que estava emparelhado).
Quebrar relações de irmão ou confrade entre.
\section{Desirmão}
\begin{itemize}
\item {Grp. gram.:adj.}
\end{itemize}
\begin{itemize}
\item {Proveniência:(De \textunderscore des...\textunderscore  + \textunderscore irmão\textunderscore )}
\end{itemize}
Desirmanado; desigual. Cf. Camillo, \textunderscore Noites\textunderscore , VII, 30.
\section{Desiscar}
\begin{itemize}
\item {Grp. gram.:v. t.}
\end{itemize}
\begin{itemize}
\item {Proveniência:(De \textunderscore des...\textunderscore  + \textunderscore iscar\textunderscore )}
\end{itemize}
Tirar a isca de.
\section{Desistência}
\begin{itemize}
\item {Grp. gram.:f.}
\end{itemize}
Acto ou effeito de desistir.
\section{Desistente}
\begin{itemize}
\item {Grp. gram.:adj.}
\end{itemize}
Que desiste.
(lat. \textunderscore desistens\textunderscore )
\section{Desistição}
\begin{itemize}
\item {Grp. gram.:f.}
\end{itemize}
O mesmo que \textunderscore desistência\textunderscore .
\section{Desistir}
\begin{itemize}
\item {Grp. gram.:v. i.}
\end{itemize}
\begin{itemize}
\item {Proveniência:(Do lat. \textunderscore desistere\textunderscore )}
\end{itemize}
Abster-se: \textunderscore desisto de ir passear\textunderscore .
Fazer renúncia: \textunderscore desistir de um emprêgo\textunderscore .
\section{Desistória}
\begin{itemize}
\item {Grp. gram.:f.}
\end{itemize}
\begin{itemize}
\item {Utilização:Ant.}
\end{itemize}
\begin{itemize}
\item {Proveniência:(De \textunderscore des...\textunderscore  + \textunderscore história\textunderscore ?)}
\end{itemize}
Patranha, fábula, invenção. Cf. G. Vianna, \textunderscore Apostilas\textunderscore .
\section{Desitivo}
\begin{itemize}
\item {Grp. gram.:adj.}
\end{itemize}
\begin{itemize}
\item {Proveniência:(Do lat. \textunderscore desitus\textunderscore )}
\end{itemize}
Que denota desinência, deminuição ou termo de acção.
\section{Desjarretar}
\begin{itemize}
\item {Grp. gram.:v. t.}
\end{itemize}
O mesmo que \textunderscore dejarretar\textunderscore .
\section{Desjeito}
\begin{itemize}
\item {Grp. gram.:m.}
\end{itemize}
Falta de jeito. Cf. Filinto, XVII, 161.
\section{Desjeitoso}
\begin{itemize}
\item {Grp. gram.:adj.}
\end{itemize}
\begin{itemize}
\item {Proveniência:(De \textunderscore des\textunderscore  + \textunderscore jeitoso\textunderscore )}
\end{itemize}
Que não tem jeito; desajeitado.
\section{Desjejuar}
\begin{itemize}
\item {Grp. gram.:v. i.}
\end{itemize}
(V.dejejuar)
\section{Desjuizar}
\begin{itemize}
\item {fónica:ju-i}
\end{itemize}
\begin{itemize}
\item {Grp. gram.:v. t.}
\end{itemize}
(V.desajuizar)
\section{Desjungir}
\begin{itemize}
\item {Grp. gram.:v. t.}
\end{itemize}
\begin{itemize}
\item {Proveniência:(De \textunderscore des...\textunderscore  + \textunderscore jungir\textunderscore )}
\end{itemize}
Separar (aquillo que estava jungido).
\section{Deslaçar}
\textunderscore v. t.\textunderscore  (e der.)
O mesmo que \textunderscore desenlaçar\textunderscore .
\section{Deslacrar}
\begin{itemize}
\item {Grp. gram.:v. t.}
\end{itemize}
\begin{itemize}
\item {Proveniência:(De \textunderscore des...\textunderscore  + \textunderscore lacrar\textunderscore )}
\end{itemize}
Partir ou tirar o lacre que fecha ou sella.
\section{Desladeiro}
\begin{itemize}
\item {Grp. gram.:m.}
\end{itemize}
\begin{itemize}
\item {Utilização:Prov.}
\end{itemize}
\begin{itemize}
\item {Utilização:trasm.}
\end{itemize}
\textunderscore Subir ao desladeiro\textunderscore , subir ladeando (uma encosta).
(Cp. \textunderscore ladeira\textunderscore )
\section{Desladrilhar}
\begin{itemize}
\item {Grp. gram.:v. t.}
\end{itemize}
\begin{itemize}
\item {Proveniência:(De \textunderscore des...\textunderscore  + \textunderscore ladrilhar\textunderscore )}
\end{itemize}
Tirar os ladrilhos a.
\section{Desladrilho}
\begin{itemize}
\item {Grp. gram.:m.}
\end{itemize}
Acto de desladrilhar.
\section{Deslageamento}
\begin{itemize}
\item {Grp. gram.:m.}
\end{itemize}
Acto de deslagear.
\section{Deslagear}
\begin{itemize}
\item {Grp. gram.:v. t.}
\end{itemize}
\begin{itemize}
\item {Proveniência:(De \textunderscore des...\textunderscore  + \textunderscore lagear\textunderscore )}
\end{itemize}
Arrancar as lages a.
\section{Deslaiado}
\begin{itemize}
\item {Grp. gram.:adj.}
\end{itemize}
\begin{itemize}
\item {Utilização:Prov.}
\end{itemize}
\begin{itemize}
\item {Utilização:alg.}
\end{itemize}
\begin{itemize}
\item {Proveniência:(De \textunderscore deslaio\textunderscore )}
\end{itemize}
Desanimado, desalentado.
\section{Deslaio}
\begin{itemize}
\item {Grp. gram.:m.}
\end{itemize}
\begin{itemize}
\item {Utilização:Prov.}
\end{itemize}
\begin{itemize}
\item {Utilização:alg.}
\end{itemize}
Desânimo, desalento.
\section{Deslanar}
\begin{itemize}
\item {Grp. gram.:v. t.}
\end{itemize}
\begin{itemize}
\item {Proveniência:(De \textunderscore des...\textunderscore  + lat. \textunderscore lana\textunderscore )}
\end{itemize}
Cortar ou tosquiar a lan a. Cf. A. Baganha, \textunderscore Hygiene Pecuária\textunderscore , 225 e 226.
\section{Deslandesiano}
\begin{itemize}
\item {Grp. gram.:adj.}
\end{itemize}
Relativo aos impressores Deslandes: \textunderscore uma edição deslandesiana\textunderscore .
\section{Deslapar}
\begin{itemize}
\item {Grp. gram.:v. t.}
\end{itemize}
\begin{itemize}
\item {Proveniência:(De \textunderscore des...\textunderscore  + \textunderscore lapa\textunderscore )}
\end{itemize}
Fazer saír da lapa.
\section{Deslapidado}
\begin{itemize}
\item {Grp. gram.:adj.}
\end{itemize}
\begin{itemize}
\item {Proveniência:(De \textunderscore des...\textunderscore  + \textunderscore lapidar\textunderscore )}
\end{itemize}
Que perdeu o brilho. Cf. \textunderscore Eufrosina\textunderscore , 222.
\section{Deslassar}
\begin{itemize}
\item {Grp. gram.:v. t.}
\end{itemize}
\begin{itemize}
\item {Proveniência:(De \textunderscore des...\textunderscore  + \textunderscore lasso\textunderscore )}
\end{itemize}
\begin{itemize}
\item {Proveniência:(Cp. \textunderscore desenviolar\textunderscore )
}
\end{itemize}
Tornar lasso, torno froixo; alargar.
\section{Deslasso}
\begin{itemize}
\item {Grp. gram.:adj.}
\end{itemize}
\begin{itemize}
\item {Utilização:Prov.}
\end{itemize}
\begin{itemize}
\item {Utilização:dur.}
\end{itemize}
Diz-se do algodão lasso, não torcido. (Colhido no Pôrto)
\section{Deslastrador}
\begin{itemize}
\item {Grp. gram.:m.}
\end{itemize}
O que deslastra.
\section{Deslastrar}
\begin{itemize}
\item {Grp. gram.:v. t.}
\end{itemize}
\begin{itemize}
\item {Proveniência:(De \textunderscore des...\textunderscore  + \textunderscore lastrar\textunderscore )}
\end{itemize}
Tirar o lastro a.
\section{Deslastre}
\begin{itemize}
\item {Grp. gram.:m.}
\end{itemize}
Acto ou effeito de deslastrar.
\section{Deslavado}
\begin{itemize}
\item {Grp. gram.:adj.}
\end{itemize}
\begin{itemize}
\item {Proveniência:(De \textunderscore deslavar\textunderscore )}
\end{itemize}
Descarado; petulante; atrevido.
\section{Deslavamento}
\begin{itemize}
\item {Grp. gram.:m.}
\end{itemize}
Acto ou effeito de deslavar.
\section{Deslavar}
\begin{itemize}
\item {Grp. gram.:v. t.}
\end{itemize}
\begin{itemize}
\item {Utilização:Fig.}
\end{itemize}
\begin{itemize}
\item {Proveniência:(De \textunderscore des...\textunderscore  + \textunderscore lavar\textunderscore )}
\end{itemize}
Fazer perder a côr a.
Fazer desbotar.
Tornar descarado, desavergonhado.
\section{Deslavra}
\begin{itemize}
\item {Grp. gram.:f.}
\end{itemize}
Acto de deslavrar.
\section{Deslavrar}
\begin{itemize}
\item {Grp. gram.:v. t.}
\end{itemize}
\begin{itemize}
\item {Proveniência:(De \textunderscore des...\textunderscore  + \textunderscore lavrar\textunderscore . Cp. \textunderscore desinviolar\textunderscore )}
\end{itemize}
Lavrar (um terreno) através de outra lavra que já teve, ou através de um alqueive.
\section{Desleal}
\begin{itemize}
\item {Grp. gram.:adj.}
\end{itemize}
Que não é leal; infiel.
Tredo.
\section{Deslealdade}
\begin{itemize}
\item {Grp. gram.:f.}
\end{itemize}
Falta de lealdade.
Qualidade de quem é desleal.
\section{Deslealdar}
\begin{itemize}
\item {Grp. gram.:v. t.}
\end{itemize}
\begin{itemize}
\item {Proveniência:(De \textunderscore deslealdade\textunderscore )}
\end{itemize}
Sêr desleal a.
Tratar com deslealdade.
Traír. Cf. Filinto, D. Man., III, 46.
\section{Deslealdoso}
\begin{itemize}
\item {Grp. gram.:adj.}
\end{itemize}
Que tem deslealdade; desleal. Cf. Filinto, \textunderscore D. Man.\textunderscore , II, 129.
(Por \textunderscore deslealdadoso\textunderscore , de \textunderscore deslealdade\textunderscore )
\section{Deslealmente}
\begin{itemize}
\item {Grp. gram.:adv.}
\end{itemize}
De modo desleal.
\section{Desléctico}
\begin{itemize}
\item {Grp. gram.:adj.}
\end{itemize}
Relativo á deslexia.
Que soffre deslexia.
\section{Desleitar}
\begin{itemize}
\item {Grp. gram.:v. t.}
\end{itemize}
\begin{itemize}
\item {Proveniência:(De \textunderscore des...\textunderscore  + \textunderscore leite\textunderscore )}
\end{itemize}
Tirar o leite a; desmamar.
\section{Desleixação}
\begin{itemize}
\item {Grp. gram.:f.}
\end{itemize}
O mesmo que \textunderscore desleixo\textunderscore .
\section{Desleixadamente}
\begin{itemize}
\item {Grp. gram.:adv.}
\end{itemize}
\begin{itemize}
\item {Proveniência:(De \textunderscore desleixado\textunderscore )}
\end{itemize}
Com desleixo.
\section{Desleixado}
\begin{itemize}
\item {Grp. gram.:adj.}
\end{itemize}
\begin{itemize}
\item {Proveniência:(De \textunderscore desleixar-se\textunderscore )}
\end{itemize}
Descuidado, negligente, mollangueirão.
\section{Desleixamento}
\begin{itemize}
\item {Grp. gram.:m.}
\end{itemize}
(V.desleixo)
\section{Desleixar-se}
\begin{itemize}
\item {Grp. gram.:v. p.}
\end{itemize}
\begin{itemize}
\item {Proveniência:(De \textunderscore des...\textunderscore  + \textunderscore leixar\textunderscore )}
\end{itemize}
Descuidar-se.
Tornar-se negligente.
\section{Desleixo}
\begin{itemize}
\item {Grp. gram.:m.}
\end{itemize}
Acto de desleixar-se.
Negligência; descuido.
Inércia.
\section{Deslembrança}
\begin{itemize}
\item {Grp. gram.:f.}
\end{itemize}
Falta de lembrança.
Esquécimento.
\section{Deslembrar}
\begin{itemize}
\item {Grp. gram.:v. t.}
\end{itemize}
Não lembrar.
Esquecer-se de.
Não mencionar por esquecimento.
\section{Deslembrativo}
\begin{itemize}
\item {Grp. gram.:adj.}
\end{itemize}
\begin{itemize}
\item {Proveniência:(De \textunderscore deslembrar\textunderscore )}
\end{itemize}
Que faz deslembrar.
Que revela esquecimento. Cf. Filinto, VIII, 233.
\section{Deslendear}
\begin{itemize}
\item {Grp. gram.:v. t.}
\end{itemize}
\begin{itemize}
\item {Proveniência:(De \textunderscore des...\textunderscore  + \textunderscore lêndea\textunderscore )}
\end{itemize}
Tirar as lêndeas a.
\section{Deslexia}
\begin{itemize}
\item {fónica:csi}
\end{itemize}
\begin{itemize}
\item {Grp. gram.:f.}
\end{itemize}
\begin{itemize}
\item {Proveniência:(De \textunderscore des...\textunderscore  + gr. \textunderscore legein\textunderscore )}
\end{itemize}
Repugnância ou difficuldade de lêr, mental e pathológica.
\section{Deslialdade}
\begin{itemize}
\item {Grp. gram.:f.}
\end{itemize}
Falta de lialdade.
Qualidade de quem é deslial.
\section{Deslialdar}
\begin{itemize}
\item {Grp. gram.:v. t.}
\end{itemize}
\begin{itemize}
\item {Proveniência:(De \textunderscore deslialdade\textunderscore )}
\end{itemize}
Sêr deslial a.
Tratar com deslialdade.
Traír. Cf. Filinto, D. Man., III, 46.
\section{Deslialdoso}
\begin{itemize}
\item {Grp. gram.:adj.}
\end{itemize}
Que tem deslialdade; deslial. Cf. Filinto, \textunderscore D. Man.\textunderscore , II, 129.
(Por \textunderscore deslialdadoso\textunderscore , de \textunderscore deslialdade\textunderscore )
\section{Deslialmente}
\begin{itemize}
\item {Grp. gram.:adv.}
\end{itemize}
De modo deslial.
\section{Desliar}
\begin{itemize}
\item {Grp. gram.:v. t.}
\end{itemize}
\begin{itemize}
\item {Proveniência:(De \textunderscore des...\textunderscore  + \textunderscore liar\textunderscore )}
\end{itemize}
Desatar, desligar, separar.
\section{Desligadura}
\begin{itemize}
\item {Grp. gram.:f.}
\end{itemize}
Acto ou effeito de desligar.
\section{Desligamento}
\begin{itemize}
\item {Grp. gram.:m.}
\end{itemize}
Falta de ligamento ou ligação.
\section{Desligar}
\begin{itemize}
\item {Grp. gram.:v. t.}
\end{itemize}
\begin{itemize}
\item {Utilização:Fig.}
\end{itemize}
\begin{itemize}
\item {Proveniência:(De \textunderscore des...\textunderscore  + \textunderscore ligar\textunderscore )}
\end{itemize}
Desunir (aquillo que estava ligado).
Desobrigar: \textunderscore desligar de uma promessa\textunderscore .
\section{Deslimar}
\begin{itemize}
\item {Grp. gram.:v. i.}
\end{itemize}
\begin{itemize}
\item {Utilização:Prov.}
\end{itemize}
\begin{itemize}
\item {Utilização:Veter.}
\end{itemize}
\begin{itemize}
\item {Proveniência:(De \textunderscore des...\textunderscore  + \textunderscore limo\textunderscore )}
\end{itemize}
Deitar pela vagina humores ou muco, que os lavradores chamam limo, (falando-se de vacas aluadas). Cf. Baganha, \textunderscore Hig. Pec.\textunderscore 
\section{Deslindação}
\begin{itemize}
\item {Grp. gram.:f.}
\end{itemize}
O mesmo que \textunderscore deslindamento\textunderscore .
\section{Deslindador}
\begin{itemize}
\item {Grp. gram.:m.}
\end{itemize}
Aquelle que deslinda.
\section{Deslindamento}
\begin{itemize}
\item {Grp. gram.:m.}
\end{itemize}
Acto ou effeito de deslindar.
\section{Deslindar}
\begin{itemize}
\item {Grp. gram.:v. t.}
\end{itemize}
\begin{itemize}
\item {Proveniência:(De \textunderscore des...\textunderscore  + \textunderscore lindar\textunderscore )}
\end{itemize}
O mesmo que \textunderscore lindar\textunderscore .
Demarcar.
Investigar.
Desenredar.
Aclarar: \textunderscore deslindar uma questão\textunderscore .
\section{Deslinde}
\begin{itemize}
\item {Grp. gram.:m.}
\end{itemize}
O mesmo que \textunderscore deslindamento\textunderscore .
\section{Deslinguadamente}
\begin{itemize}
\item {Grp. gram.:adv.}
\end{itemize}
\begin{itemize}
\item {Proveniência:(De \textunderscore deslinguado\textunderscore )}
\end{itemize}
Desbocadamente,despejadamente. Cf. Camillo, \textunderscore Ratos da Inquis.\textunderscore , 47.
\section{Deslinguado}
\begin{itemize}
\item {Grp. gram.:m. adj.}
\end{itemize}
\begin{itemize}
\item {Proveniência:(De \textunderscore deslinguar\textunderscore )}
\end{itemize}
Maldizente:«\textunderscore o fel da enveja, que nos deslinguados domina...\textunderscore »Amador Arráiz.
\section{Deslinguar}
\begin{itemize}
\item {Grp. gram.:v. t.}
\end{itemize}
\begin{itemize}
\item {Grp. gram.:V. p.}
\end{itemize}
\begin{itemize}
\item {Utilização:Fig.}
\end{itemize}
\begin{itemize}
\item {Proveniência:(De \textunderscore des...\textunderscore  + \textunderscore língua\textunderscore )}
\end{itemize}
Tirar a língua a.
Falar muito, sem vergonha.
\section{Deslisura}
\begin{itemize}
\item {Grp. gram.:f.}
\end{itemize}
Falta de lisura.
Doblez. Cf. Filinto, VII, 134.
\section{Deslizadeiro}
\begin{itemize}
\item {Grp. gram.:m.}
\end{itemize}
\begin{itemize}
\item {Proveniência:(De \textunderscore deslizar\textunderscore )}
\end{itemize}
O mesmo que \textunderscore resvaladoiro\textunderscore .
\section{Deslizamento}
\begin{itemize}
\item {Grp. gram.:m.}
\end{itemize}
Acto de deslizar.
\section{Deslizar}
\begin{itemize}
\item {Grp. gram.:v. i.}
\end{itemize}
\begin{itemize}
\item {Grp. gram.:V. i.}
\end{itemize}
\begin{itemize}
\item {Grp. gram.:V. p.}
\end{itemize}
Escorregar brandamente.
Resvalar.
Derivar suavemente.
Desviar-se do bom caminho.
Divergir, discordar. Cf. Camillo, \textunderscore Noites de Insómn.\textunderscore 
(a mesma significação do \textunderscore v. i.\textunderscore )
(Cp. cast. \textunderscore deslizar\textunderscore )
\section{Deslize}
\begin{itemize}
\item {Grp. gram.:m.}
\end{itemize}
\begin{itemize}
\item {Utilização:P. us.}
\end{itemize}
Deslizamento.
Desvio do dever:«\textunderscore ...se houvessem descido as escaleiras da necessidade sem deslize da honra.\textunderscore »Camillo, \textunderscore Vinho do Porto\textunderscore , 80.
Engano, equivoco; incorrecção involuntária.
(Cp. cast. \textunderscore desliz\textunderscore )
\section{Deslizo}
\begin{itemize}
\item {Grp. gram.:m.}
\end{itemize}
O mesmo que \textunderscore deslizamento\textunderscore . Cf. Filinto. IV, 265.
(Cp. cast. ant. \textunderscore deslizo\textunderscore )
\section{Deslocação}
\begin{itemize}
\item {Grp. gram.:f.}
\end{itemize}
Acto ou effeito de deslocar.
\section{Deslocador}
\begin{itemize}
\item {Grp. gram.:adj.}
\end{itemize}
Que desloca.
\section{Deslocamento}
\begin{itemize}
\item {Grp. gram.:m.}
\end{itemize}
O mesmo que \textunderscore deslocação\textunderscore . Cf. Castilho, \textunderscore Fastos\textunderscore , III, 413.
\section{Deslocar}
\begin{itemize}
\item {Grp. gram.:v. t.}
\end{itemize}
\begin{itemize}
\item {Proveniência:(Do lat. \textunderscore des...\textunderscore  + \textunderscore locare\textunderscore )}
\end{itemize}
Tirar de um lugar para outro.
Afastar.
Desconjuntar: \textunderscore deslocar um braço\textunderscore .
\section{Deslodamento}
\begin{itemize}
\item {Grp. gram.:m.}
\end{itemize}
Acto ou effeito de deslodar.
\section{Deslodar}
\begin{itemize}
\item {Grp. gram.:v. t.}
\end{itemize}
\begin{itemize}
\item {Proveniência:(De \textunderscore des...\textunderscore  + \textunderscore lodo\textunderscore )}
\end{itemize}
Tirar o lodo a.
Tirar do lodo; desenlamear.
\section{Deslograr}
\begin{itemize}
\item {Grp. gram.:v. t.}
\end{itemize}
\begin{itemize}
\item {Proveniência:(De \textunderscore des...\textunderscore  + \textunderscore lograr\textunderscore )}
\end{itemize}
Não lograr.
Deixar de lograr: \textunderscore deslograr a estima pública\textunderscore . Cf. Filinto, XX, 123.
\section{Deslombar}
\begin{itemize}
\item {Grp. gram.:v. t.}
\end{itemize}
\begin{itemize}
\item {Utilização:Pop.}
\end{itemize}
\begin{itemize}
\item {Utilização:Fig.}
\end{itemize}
\begin{itemize}
\item {Proveniência:(De \textunderscore des...\textunderscore  + \textunderscore lombo\textunderscore )}
\end{itemize}
Bater muito em.
Derrear com pancadas.
Abater, vencer.
\section{Desloucar}
\begin{itemize}
\item {Grp. gram.:v. t.}
\end{itemize}
Gradar ligeiramente (a terra). Cf. \textunderscore Bibl. da Gente do Campo\textunderscore , 304.
(Por \textunderscore deslòcar\textunderscore , de \textunderscore loca\textunderscore ?)
\section{Deslouvar}
\begin{itemize}
\item {Grp. gram.:v. t.}
\end{itemize}
\begin{itemize}
\item {Proveniência:(De \textunderscore des...\textunderscore  + \textunderscore louvar\textunderscore )}
\end{itemize}
O mesmo que \textunderscore desgabar\textunderscore .
\section{Deslouvor}
\begin{itemize}
\item {Grp. gram.:m.}
\end{itemize}
Ausência de louvor.
O mesmo que \textunderscore desgabo\textunderscore .
\section{Deslumbradamente}
\begin{itemize}
\item {Grp. gram.:adv.}
\end{itemize}
\begin{itemize}
\item {Proveniência:(De \textunderscore deslumbrar\textunderscore )}
\end{itemize}
Com deslumbramento.
\section{Deslumbrador}
\begin{itemize}
\item {Grp. gram.:adj.}
\end{itemize}
\begin{itemize}
\item {Grp. gram.:M.}
\end{itemize}
Que deslumbra.
Aquelle que deslumbra.
\section{Deslumbramento}
\begin{itemize}
\item {Grp. gram.:m.}
\end{itemize}
Acto ou effeito de deslumbrar.
\section{Deslumbrante}
\begin{itemize}
\item {Grp. gram.:adj.}
\end{itemize}
Que deslumbra.
\section{Deslumbrar}
\begin{itemize}
\item {Grp. gram.:v. t.}
\end{itemize}
Turvar a vista de, pela acção de demasiada luz.
Causar assombro em.
Maravilhar: \textunderscore a festa deslumbrava-nos\textunderscore .
Seduzir; perturbar o entendimento de.
(Cast. \textunderscore deslumbrar\textunderscore )
\section{Deslumbrativo}
\begin{itemize}
\item {Grp. gram.:adj.}
\end{itemize}
\begin{itemize}
\item {Proveniência:(De \textunderscore deslumbrar\textunderscore )}
\end{itemize}
Capaz de deslumbrar; deslumbrante. Cf. Filinto, VIII, 233.
\section{Deslumbroso}
\begin{itemize}
\item {Grp. gram.:adj.}
\end{itemize}
(V.deslumbrante)
\section{Deslustral}
\begin{itemize}
\item {Grp. gram.:adj.}
\end{itemize}
Que não é lustral.
Que deslustra. Cf. Filinto, X, 120.
\section{Deslustrar}
\begin{itemize}
\item {Grp. gram.:v. t.}
\end{itemize}
\begin{itemize}
\item {Utilização:Fig.}
\end{itemize}
\begin{itemize}
\item {Proveniência:(De \textunderscore des...\textunderscore  + \textunderscore lustrar\textunderscore )}
\end{itemize}
Tirar ou deminuir o lustre de.
Empanar.
Desacreditar, manchar.
\section{Deslustre}
\begin{itemize}
\item {Grp. gram.:m.}
\end{itemize}
Acto ou effeito de deslustrar.
\section{Deslustro}
\begin{itemize}
\item {Grp. gram.:m.}
\end{itemize}
O mesmo que \textunderscore deslustre\textunderscore . Cf. J. Dinis, \textunderscore Fidalgos\textunderscore , I, 201.
\section{Deslustroso}
\begin{itemize}
\item {Grp. gram.:adj.}
\end{itemize}
\begin{itemize}
\item {Proveniência:(De \textunderscore deslustrar\textunderscore )}
\end{itemize}
Que não tem lustre.
Que deslustra.
\section{Desluzidamente}
\begin{itemize}
\item {Grp. gram.:adv.}
\end{itemize}
\begin{itemize}
\item {Proveniência:(De \textunderscore desluzir\textunderscore )}
\end{itemize}
Com desluzimento.
\section{Desluzido}
\begin{itemize}
\item {Grp. gram.:adj.}
\end{itemize}
\begin{itemize}
\item {Proveniência:(De \textunderscore desluzir\textunderscore )}
\end{itemize}
Deslustrado.
Depreciado.
Escasso ou minguado em pêso ou medida: \textunderscore a manteiga, que veio da tenda, é muito desluzida\textunderscore .
\section{Desluzidor}
\begin{itemize}
\item {Grp. gram.:adj.}
\end{itemize}
\begin{itemize}
\item {Grp. gram.:M.}
\end{itemize}
Que desluz.
Aquelle que desluz.
\section{Desluzimento}
\begin{itemize}
\item {Grp. gram.:m.}
\end{itemize}
Acto ou effeito de desluzir.
\section{Desluzir}
\begin{itemize}
\item {Grp. gram.:v. t.}
\end{itemize}
\begin{itemize}
\item {Utilização:Fig.}
\end{itemize}
\begin{itemize}
\item {Proveniência:(De \textunderscore des...\textunderscore  + \textunderscore luzir\textunderscore )}
\end{itemize}
Deslustrar.
Depreciar; deminuir o mérito de.
\section{Desmaginar}
\textunderscore v. t.\textunderscore  (e der.)
O mesmo que \textunderscore desimaginar\textunderscore , etc. Cf. Filinto, XIX, 175; XXI, 101; Castilho, \textunderscore D. Quixote\textunderscore , I, 171.
\section{Desmagnetização}
\begin{itemize}
\item {Grp. gram.:f.}
\end{itemize}
Acto de desmagnetizar.
\section{Desmagnetizar}
\begin{itemize}
\item {Grp. gram.:v. t.}
\end{itemize}
\begin{itemize}
\item {Proveniência:(De \textunderscore des...\textunderscore  + \textunderscore magnetizar\textunderscore )}
\end{itemize}
Tirar o fluido magnético a.
Subtrahir á acção magnética. Cf. F. Lapa, \textunderscore Phýsica e Chímica\textunderscore , 118.
\section{Desmaiadamente}
\begin{itemize}
\item {Grp. gram.:adv.}
\end{itemize}
\begin{itemize}
\item {Proveniência:(De \textunderscore desmaiado\textunderscore )}
\end{itemize}
Descòradamente.
De modo froixo.
\section{Desmaiado}
\begin{itemize}
\item {Grp. gram.:adj.}
\end{itemize}
\begin{itemize}
\item {Utilização:Ant.}
\end{itemize}
\begin{itemize}
\item {Proveniência:(De \textunderscore desmaiar\textunderscore )}
\end{itemize}
Que desmaiou, que perdeu os sentidos.
Desbotado, que tem pouco brilho; pállido: \textunderscore flôr desmaiada\textunderscore .
Quási imperceptível: \textunderscore som desmaiado\textunderscore .
Desanimado, desalentado.
\section{Desmaiar}
\begin{itemize}
\item {Grp. gram.:v. t.}
\end{itemize}
\begin{itemize}
\item {Utilização:Fig.}
\end{itemize}
\begin{itemize}
\item {Grp. gram.:V. i.}
\end{itemize}
\begin{itemize}
\item {Proveniência:(De um rad. germ. \textunderscore magan\textunderscore ?)}
\end{itemize}
Fazer descòrar.
Fazer perder os sentidos.
Menoscabar.
Perder a côr; perder os sentidos.
Desalentar-se.
Enfraquecer.
\section{Desmaio}
\begin{itemize}
\item {Grp. gram.:m.}
\end{itemize}
Acto ou effeito de desmaiar.
\section{Desmalhar}
\begin{itemize}
\item {Grp. gram.:v. t.}
\end{itemize}
\begin{itemize}
\item {Proveniência:(De \textunderscore des...\textunderscore  + \textunderscore malha\textunderscore )}
\end{itemize}
Tirar as malhas a.
\section{Desmaliciado}
\begin{itemize}
\item {Grp. gram.:adj.}
\end{itemize}
\begin{itemize}
\item {Proveniência:(De \textunderscore des...\textunderscore  + \textunderscore malícia\textunderscore )}
\end{itemize}
Que não tem malícia; desmalicioso. Cf. Latino, \textunderscore Humboldt\textunderscore , 426.
\section{Desmalicioso}
\begin{itemize}
\item {Grp. gram.:adj.}
\end{itemize}
\begin{itemize}
\item {Proveniência:(De \textunderscore des...\textunderscore  + \textunderscore malicioso\textunderscore )}
\end{itemize}
Que não tem malícia.
Em que não há malícia.
\section{Desmaltas}
\begin{itemize}
\item {Grp. gram.:f. pl.}
\end{itemize}
\begin{itemize}
\item {Utilização:Prov.}
\end{itemize}
\begin{itemize}
\item {Utilização:trasm.}
\end{itemize}
Ralhos, questões, vivas.
\section{Desmama}
\begin{itemize}
\item {Grp. gram.:f.}
\end{itemize}
Acto de desmamar.
\section{Desmamação}
\begin{itemize}
\item {Grp. gram.:f.}
\end{itemize}
O mesmo que \textunderscore desmama\textunderscore .
\section{Desmamadeira}
\begin{itemize}
\item {Grp. gram.:f.}
\end{itemize}
\begin{itemize}
\item {Proveniência:(De \textunderscore desmamar\textunderscore )}
\end{itemize}
Boneca de trapinhos, embebidos em substância amarga, que se dão a chupar ás crianças, na época da desmamação, quando pedem mama, para que esta lhes repugne.
\section{Desmamamento}
\begin{itemize}
\item {Grp. gram.:m.}
\end{itemize}
O mesmo que \textunderscore desmama\textunderscore .
\section{Desmamar}
\begin{itemize}
\item {Grp. gram.:v. t.}
\end{itemize}
\begin{itemize}
\item {Utilização:pop.}
\end{itemize}
\begin{itemize}
\item {Utilização:Fig.}
\end{itemize}
\begin{itemize}
\item {Utilização:Marn.}
\end{itemize}
\begin{itemize}
\item {Proveniência:(De \textunderscore des...\textunderscore  + \textunderscore mamar\textunderscore )}
\end{itemize}
Suspender a amamentação de.
Concluir a criação de.
Emancipar.
Tirar dos meios das salinas (a água já concentrada).
\section{Desmame}
\begin{itemize}
\item {Grp. gram.:m.}
\end{itemize}
O mesmo que \textunderscore desmama\textunderscore . Cf. A. Baganha, \textunderscore Hygiene Pecuária\textunderscore , 92 e 181.
\section{Desmammar}
\textunderscore v. t.\textunderscore  (e der.)
O mesmo que \textunderscore desmamar\textunderscore , etc.
\section{Desmamo}
\begin{itemize}
\item {Grp. gram.:m.}
\end{itemize}
O mesmo que \textunderscore desmame\textunderscore . Cf. Ed. Magalhães, \textunderscore Hyg. Alim.\textunderscore , I, 27.
\section{Desmanar}
\begin{itemize}
\item {Grp. gram.:v. t.}
\end{itemize}
Separar da manada.
Tresmalhar.
(Por \textunderscore desmanadar\textunderscore , de \textunderscore des...\textunderscore  + \textunderscore manada\textunderscore )
\section{Desmancha}
\begin{itemize}
\item {Grp. gram.:f.}
\end{itemize}
\begin{itemize}
\item {Utilização:Bras. do N}
\end{itemize}
Acto de desmanchar (a mandioca).
\section{Desmanchadamente}
\begin{itemize}
\item {Grp. gram.:adv.}
\end{itemize}
\begin{itemize}
\item {Proveniência:(De \textunderscore desmanchar\textunderscore )}
\end{itemize}
Descompostamente; em desordem.
\section{Desmanchadão}
\begin{itemize}
\item {Grp. gram.:m.  e  adj.}
\end{itemize}
\begin{itemize}
\item {Utilização:Fam.}
\end{itemize}
\begin{itemize}
\item {Proveniência:(De \textunderscore desmanchar\textunderscore )}
\end{itemize}
Homem desmazelado; desajeitado.
\section{Desmanchadela}
\begin{itemize}
\item {Grp. gram.:f.}
\end{itemize}
Acto de desmanchar.
\section{Desmanchadiço}
\begin{itemize}
\item {Grp. gram.:adj.}
\end{itemize}
Que se desmancha facilmente.
\section{Desmanchadoiro}
\begin{itemize}
\item {Grp. gram.:adj.}
\end{itemize}
\begin{itemize}
\item {Utilização:Ant.}
\end{itemize}
\begin{itemize}
\item {Proveniência:(De \textunderscore desmanchar\textunderscore )}
\end{itemize}
Dizia-se do indivíduo desajeitado, vestido com desalinho.
\section{Desmanchadouro}
\begin{itemize}
\item {Grp. gram.:adj.}
\end{itemize}
\begin{itemize}
\item {Utilização:Ant.}
\end{itemize}
\begin{itemize}
\item {Proveniência:(De \textunderscore desmanchar\textunderscore )}
\end{itemize}
Dizia-se do indivíduo desajeitado, vestido com desalinho.
\section{Desmancha-prazeres}
\begin{itemize}
\item {Grp. gram.:m.  e  f.}
\end{itemize}
Pessôa que estorva a outrem prazer ou divertimemto.
Empecilho.
\section{Desmanchar}
\begin{itemize}
\item {Grp. gram.:v. t.}
\end{itemize}
\begin{itemize}
\item {Utilização:Bras. do N}
\end{itemize}
\begin{itemize}
\item {Grp. gram.:V. i.}
\end{itemize}
\begin{itemize}
\item {Utilização:Mús.}
\end{itemize}
Desfazer.
Dessarranjar: \textunderscore desmanchar a cama\textunderscore .
Descompor; deslocar.
Inutilizar: \textunderscore desmanchar uma teia\textunderscore .
Demolir: \textunderscore desmanchar uma parede\textunderscore .
Tornar descommedido, dissoluto.
Reduzir (a mandioca) a farinha.
Tirar a mão esquerda da sua posição natural, nos instrumentos de corda, para collocar os dedos na extremidade do ponto, próximo do cavallete, a fim de produzir as notas mais agudas.
(Por \textunderscore desmachar\textunderscore , do lat. \textunderscore emasculare\textunderscore ?)
\section{Desmancho}
\begin{itemize}
\item {Grp. gram.:m.}
\end{itemize}
\begin{itemize}
\item {Utilização:Fam.}
\end{itemize}
Acto ou effeito de desmanchar.
Abôrto.
\section{Desmandadamente}
\begin{itemize}
\item {Grp. gram.:adv.}
\end{itemize}
\begin{itemize}
\item {Proveniência:(De \textunderscore desmandar\textunderscore )}
\end{itemize}
Com desmando.
\section{Desmandamento}
\begin{itemize}
\item {Grp. gram.:m.}
\end{itemize}
O mesmo que \textunderscore desmando\textunderscore . Cf. Filinto, \textunderscore D. Man.\textunderscore , I, 354.
\section{Desmandar}
\begin{itemize}
\item {Grp. gram.:v. t.}
\end{itemize}
\begin{itemize}
\item {Grp. gram.:V. p.}
\end{itemize}
\begin{itemize}
\item {Proveniência:(De \textunderscore des...\textunderscore  + \textunderscore mandar\textunderscore )}
\end{itemize}
Contramandar.
Exceder ordens recebidas.
Exorbitar.
Desregrar-se.
Tornar-se dissoluto.
\section{Desmandibulação}
\begin{itemize}
\item {Grp. gram.:f.}
\end{itemize}
Supplício antigo de arrancar as mandíbulas ou queixos. Cf. Cortesão, \textunderscore Subs.\textunderscore 
\section{Desmando}
\begin{itemize}
\item {Grp. gram.:m.}
\end{itemize}
Acto ou effeito de desmandar e de desmandar-se.
\section{Desmanear}
\begin{itemize}
\item {Grp. gram.:v. t.}
\end{itemize}
\begin{itemize}
\item {Utilização:Bras. do S}
\end{itemize}
Tirar a maneia a.
\section{Desmangolado}
\begin{itemize}
\item {Grp. gram.:adj.}
\end{itemize}
\begin{itemize}
\item {Utilização:Bras. do N}
\end{itemize}
Desajeitado.
Mal feito do corpo.
\section{Desmanho}
\begin{itemize}
\item {Grp. gram.:m.}
\end{itemize}
\begin{itemize}
\item {Utilização:Ant.}
\end{itemize}
Debandada.
Desordem; confusão.
(Por \textunderscore desamanho\textunderscore , de \textunderscore des...\textunderscore  + \textunderscore amanho\textunderscore )
\section{Desmanhoso}
\begin{itemize}
\item {Grp. gram.:adj.}
\end{itemize}
Que não é manhoso, que não tem manha.
\section{Desmaninhar}
\begin{itemize}
\item {Grp. gram.:v. t.}
\end{itemize}
\begin{itemize}
\item {Proveniência:(De \textunderscore des...\textunderscore  + \textunderscore maninho\textunderscore )}
\end{itemize}
Tornar cultivados (terrenos que eram maninhos).
\section{Desmanivar}
\begin{itemize}
\item {Grp. gram.:v. t.}
\end{itemize}
\begin{itemize}
\item {Utilização:Bras}
\end{itemize}
\begin{itemize}
\item {Utilização:Fig.}
\end{itemize}
Desramar (mandíoca).
Desembaraçar.
Facilitar.
Dissipar; desbaratar.
\section{Desmantadela}
\begin{itemize}
\item {Grp. gram.:f.}
\end{itemize}
\begin{itemize}
\item {Utilização:Prov.}
\end{itemize}
\begin{itemize}
\item {Utilização:dur.}
\end{itemize}
\begin{itemize}
\item {Proveniência:(De \textunderscore desmantar\textunderscore )}
\end{itemize}
Acto de esfolhar ou descamisar (o milho); descamisada.
\section{Desmantar}
\begin{itemize}
\item {Grp. gram.:v. t.}
\end{itemize}
\begin{itemize}
\item {Utilização:Prov.}
\end{itemize}
\begin{itemize}
\item {Utilização:dur.}
\end{itemize}
\begin{itemize}
\item {Proveniência:(De \textunderscore des...\textunderscore  + \textunderscore manta\textunderscore )}
\end{itemize}
Tirar a casca ou folhelho a (o milho); descamisar.
\section{Desmanteladamente}
\begin{itemize}
\item {Grp. gram.:adv.}
\end{itemize}
Com desmantelamento.
\section{Desmantelador}
\begin{itemize}
\item {Grp. gram.:adj.}
\end{itemize}
Que desmantela.
\section{Desmantelamento}
\begin{itemize}
\item {Grp. gram.:m.}
\end{itemize}
Acto ou effeito de desmantelar.
\section{Desmantelar}
\begin{itemize}
\item {Grp. gram.:v. t.}
\end{itemize}
\begin{itemize}
\item {Proveniência:(De \textunderscore des...\textunderscore  + \textunderscore mantel\textunderscore )}
\end{itemize}
Derribar (muralhas, fortificações).
Desmanchar.
Arruinar.
\section{Desmantelo}
\begin{itemize}
\item {fónica:tê}
\end{itemize}
\begin{itemize}
\item {Grp. gram.:m.}
\end{itemize}
Acto ou effeito de desmantelar. Cf. F. Alexandre Lobo, III, 350, 386 e 387.
\section{Desmantho}
\begin{itemize}
\item {Grp. gram.:m.}
\end{itemize}
\begin{itemize}
\item {Proveniência:(Do gr. \textunderscore desmos\textunderscore  + \textunderscore anthos\textunderscore )}
\end{itemize}
Gênero de acácias.
\section{Desmanto}
\begin{itemize}
\item {Grp. gram.:m.}
\end{itemize}
\begin{itemize}
\item {Proveniência:(Do gr. \textunderscore desmos\textunderscore  + \textunderscore anthos\textunderscore )}
\end{itemize}
Gênero de acácias.
\section{Desmanzelado}
\begin{itemize}
\item {Grp. gram.:adj.}
\end{itemize}
\begin{itemize}
\item {Utilização:Prov.}
\end{itemize}
O mesmo que \textunderscore desmazelado\textunderscore .
\section{Desmaranhado}
\begin{itemize}
\item {Grp. gram.:adj.}
\end{itemize}
\begin{itemize}
\item {Utilização:Prov.}
\end{itemize}
\begin{itemize}
\item {Utilização:alg.}
\end{itemize}
O mesmo que \textunderscore despassarinhado\textunderscore .
\section{Desmaranhar}
\begin{itemize}
\item {Grp. gram.:v. t.}
\end{itemize}
O mesmo que \textunderscore desemmaranhar\textunderscore . Cf. Filinto, II, 22.
\section{Desmaranho}
\begin{itemize}
\item {Grp. gram.:m.}
\end{itemize}
\begin{itemize}
\item {Utilização:Prov.}
\end{itemize}
\begin{itemize}
\item {Utilização:alg.}
\end{itemize}
\begin{itemize}
\item {Proveniência:(De \textunderscore desmaranhar\textunderscore )}
\end{itemize}
Desalinho; desordem.
\section{Desmarcadamente}
\begin{itemize}
\item {Grp. gram.:adv.}
\end{itemize}
De modo desmarcado.
Enormemente.
\section{Desmarcado}
\begin{itemize}
\item {Grp. gram.:adj.}
\end{itemize}
\begin{itemize}
\item {Proveniência:(De \textunderscore desmarcar\textunderscore )}
\end{itemize}
Enorme.
Descompassado; immenso: \textunderscore ambição desmarcada\textunderscore .
\section{Desmarcar}
\begin{itemize}
\item {Grp. gram.:v. t.}
\end{itemize}
\begin{itemize}
\item {Proveniência:(De \textunderscore des...\textunderscore  + \textunderscore marcar\textunderscore )}
\end{itemize}
Tirar as marcas a.
Tirar os marcos a.
Tornar excessivo, desmedido.
\section{Desmarear}
\begin{itemize}
\item {Grp. gram.:v. t.}
\end{itemize}
\begin{itemize}
\item {Grp. gram.:V. p.}
\end{itemize}
\begin{itemize}
\item {Proveniência:(De \textunderscore des...\textunderscore  + \textunderscore marear\textunderscore ^1)}
\end{itemize}
Tirar as manchas a.
Diz-se das embarcações que perdem o govêrno, á falta de mareação.
\section{Desmarelecer}
\begin{itemize}
\item {Grp. gram.:v. i.}
\end{itemize}
Deixar de sêr amarelo.
Perder a pallidez, tomando côr de saúde. Cf. Filinto, VII, 159.
(Por \textunderscore desamarelecer\textunderscore , de \textunderscore des...\textunderscore  + \textunderscore amarelo\textunderscore )
\section{Desmascaramento}
\begin{itemize}
\item {Grp. gram.:m.}
\end{itemize}
Acto de desmarcarar.
\section{Desmascarar}
\begin{itemize}
\item {Grp. gram.:v. t.}
\end{itemize}
\begin{itemize}
\item {Utilização:Fig.}
\end{itemize}
\begin{itemize}
\item {Proveniência:(De \textunderscore des...\textunderscore  + \textunderscore mascarar\textunderscore )}
\end{itemize}
Tirar a máscara a.
Dar a conhecer, descobrir (aquillo que se intentava ou que era segrêdo): \textunderscore desmascarar trapaças\textunderscore .
\section{Desmasiadamente}
\begin{itemize}
\item {Grp. gram.:adv.}
\end{itemize}
\begin{itemize}
\item {Utilização:Des.}
\end{itemize}
O mesmo que \textunderscore demasiadamente\textunderscore . Cf. Pant. de Aveiro, \textunderscore Itiner.\textunderscore , 25, (2.^a ed.).
\section{Desmastrar}
\begin{itemize}
\item {Grp. gram.:v. t.}
\end{itemize}
O mesmo que \textunderscore desmastrear\textunderscore .
\section{Desmastreamento}
\begin{itemize}
\item {Grp. gram.:m.}
\end{itemize}
Acto de desmastrear.
\section{Desmastrear}
\begin{itemize}
\item {Grp. gram.:v.}
\end{itemize}
\begin{itemize}
\item {Utilização:t. Náut.}
\end{itemize}
\begin{itemize}
\item {Proveniência:(De \textunderscore des...\textunderscore  + \textunderscore mastrear\textunderscore )}
\end{itemize}
Desarvorar os mastros de.
Desapparelhar (embarcação).
\section{Desmaterialização}
\begin{itemize}
\item {Grp. gram.:f.}
\end{itemize}
Acto de desmaterializar-se.
\section{Desmaterializar-se}
\begin{itemize}
\item {Grp. gram.:v. p.}
\end{itemize}
\begin{itemize}
\item {Utilização:Espir.}
\end{itemize}
Perder a supposta fórma material, (falando-se do espírito que se materializara).
\section{Desmazelado}
\begin{itemize}
\item {Grp. gram.:adj.}
\end{itemize}
Que se desmazela.
Desleixado, negligente.
\section{Desmazelar-se}
\begin{itemize}
\item {Grp. gram.:v. p.}
\end{itemize}
\begin{itemize}
\item {Proveniência:(De \textunderscore mazela\textunderscore )}
\end{itemize}
Desleixar-se; tornar-se negligente, desalinhado.
\section{Desmazêlo}
\begin{itemize}
\item {Grp. gram.:m.}
\end{itemize}
Acto ou effeito de desmazelar-se.
\section{Desmazio}
\begin{itemize}
\item {Grp. gram.:m.}
\end{itemize}
\begin{itemize}
\item {Utilização:Açor}
\end{itemize}
O mesmo que \textunderscore abôrto\textunderscore  ou \textunderscore desmancho\textunderscore .
\section{Desmear}
\begin{itemize}
\item {Grp. gram.:v. t.}
\end{itemize}
\begin{itemize}
\item {Utilização:Prov.}
\end{itemize}
\begin{itemize}
\item {Utilização:minh.}
\end{itemize}
\begin{itemize}
\item {Utilização:Ext.}
\end{itemize}
\begin{itemize}
\item {Proveniência:(De \textunderscore des\textunderscore  + \textunderscore meio\textunderscore )}
\end{itemize}
Serrar ao meio (tábua), no sentido do comprimento.
Serrar metade de (duas ou mais tábuas).
\section{Desmedidamente}
\begin{itemize}
\item {Grp. gram.:adv.}
\end{itemize}
De modo desmedido.
\section{Desmedido}
\begin{itemize}
\item {Grp. gram.:adj.}
\end{itemize}
\begin{itemize}
\item {Proveniência:(De \textunderscore desmedir-se\textunderscore )}
\end{itemize}
Muito extenso.
Incommensurável.
Enorme.
\section{Desmedir-se}
\begin{itemize}
\item {Grp. gram.:v. p.}
\end{itemize}
O mesmo que \textunderscore descommedir-se\textunderscore .
\section{Desmedrado}
\begin{itemize}
\item {Grp. gram.:adj.}
\end{itemize}
\begin{itemize}
\item {Utilização:Prov.}
\end{itemize}
Desleixado, indolente. (Colhido em Turquel)
\section{Desmedramento}
\begin{itemize}
\item {Grp. gram.:m.}
\end{itemize}
Acto de desmedrar.
\section{Desmedrança}
\begin{itemize}
\item {Grp. gram.:f.}
\end{itemize}
Falta de medrança.
\section{Desmedrar}
\begin{itemize}
\item {Grp. gram.:v. t.}
\end{itemize}
\begin{itemize}
\item {Grp. gram.:V. i.}
\end{itemize}
\begin{itemize}
\item {Proveniência:(De \textunderscore des...\textunderscore  + \textunderscore medrar\textunderscore )}
\end{itemize}
Impedir a medrança de.
Não medrar.
Crescer pouco.
\section{Desmedro}
\begin{itemize}
\item {Grp. gram.:m.}
\end{itemize}
(V.desmedrança)
\section{Desmedroso}
\begin{itemize}
\item {Grp. gram.:adj.}
\end{itemize}
Que não é medroso, que não tem mêdo.
Intrépido.
\section{Desmedular}
\begin{itemize}
\item {Grp. gram.:v. t.}
\end{itemize}
Tirar a medula ou o miolo a.
\section{Desmedullar}
\begin{itemize}
\item {Grp. gram.:v. t.}
\end{itemize}
Tirar a medulla ou o miolo a.
\section{Desmelancolizar}
\begin{itemize}
\item {Grp. gram.:v. t.}
\end{itemize}
Tirar a melancolia a.
Alegrar.
\section{Desmelindrar}
\begin{itemize}
\item {Grp. gram.:v. t.}
\end{itemize}
Livrar de melindres.
Desaggravar.
\section{Desmembração}
\begin{itemize}
\item {Grp. gram.:f.}
\end{itemize}
Acto ou effeito de desmembrar.
\section{Desmembramento}
\begin{itemize}
\item {Grp. gram.:m.}
\end{itemize}
O mesmo que \textunderscore desmembração\textunderscore .
\section{Desmembrar}
\begin{itemize}
\item {Grp. gram.:v. t.}
\end{itemize}
Cortar os membros ou algum membro de.
Separar os membros de.
\section{Desmemória}
\begin{itemize}
\item {Grp. gram.:f.}
\end{itemize}
Falta de memória.
Esquecimento. Cf. Filinto, XVIII, 225.
\section{Desmemoriar}
\begin{itemize}
\item {Grp. gram.:v. t.}
\end{itemize}
\begin{itemize}
\item {Proveniência:(De \textunderscore des...\textunderscore  + \textunderscore memória\textunderscore )}
\end{itemize}
Fazer perder a memória a.
Tornar esquecido.
\section{Desmensurar}
\begin{itemize}
\item {Grp. gram.:v. t.}
\end{itemize}
(V.desmesurar)
\section{Desmentar}
\textunderscore v. t.\textunderscore  (e der.)
O mesmo que \textunderscore dementar\textunderscore , etc. Cf. Arn. Gama, \textunderscore Um Motim\textunderscore , 342, 423 e 439.
\section{Desmentido}
\begin{itemize}
\item {Grp. gram.:m.}
\end{itemize}
\begin{itemize}
\item {Proveniência:(De \textunderscore desmentir\textunderscore )}
\end{itemize}
Palavras ou declaração, com que se desmente.
\section{Desmentidor}
\begin{itemize}
\item {Grp. gram.:m.}
\end{itemize}
Aquelle que desmente.
\section{Desmentidura}
\begin{itemize}
\item {Grp. gram.:f.}
\end{itemize}
\begin{itemize}
\item {Utilização:Bras. do N}
\end{itemize}
\begin{itemize}
\item {Proveniência:(De \textunderscore desmentir\textunderscore )}
\end{itemize}
Luxação ou deslocação de um osso.
Torcedura.
\section{Desmentir}
\begin{itemize}
\item {Grp. gram.:v. t.}
\end{itemize}
\begin{itemize}
\item {Utilização:Bras}
\end{itemize}
\begin{itemize}
\item {Utilização:Fig.}
\end{itemize}
\begin{itemize}
\item {Grp. gram.:V. p.}
\end{itemize}
\begin{itemize}
\item {Utilização:Prov.}
\end{itemize}
\begin{itemize}
\item {Proveniência:(Do b. lat. \textunderscore desmentire\textunderscore )}
\end{itemize}
Contradizer, contraditar: \textunderscore desmentir calúmnias\textunderscore .
Deslocar, entorsar (uma articulação): \textunderscore desmentir um pé\textunderscore .
Destoar.
Discrepar de.
Não ajustar com exactidão (uma peça noutra).
\section{Desmerecedor}
\begin{itemize}
\item {Grp. gram.:adj.}
\end{itemize}
Que desmerece.
\section{Desmerecer}
\begin{itemize}
\item {Grp. gram.:v. t.}
\end{itemize}
\begin{itemize}
\item {Grp. gram.:V. i.}
\end{itemize}
Não merecer.
Ser indigno de: \textunderscore desmerecer encómios\textunderscore .
Não sêr digno.
Perder o merecimento.
Desbotar: \textunderscore esta fazenda tem desmerecido\textunderscore .
\section{Desmerecimento}
\begin{itemize}
\item {Grp. gram.:m.}
\end{itemize}
Acto ou effeito de desmerecer.
\section{Desmergulhar}
\begin{itemize}
\item {Grp. gram.:v. t.}
\end{itemize}
\begin{itemize}
\item {Proveniência:(De \textunderscore des...\textunderscore  + \textunderscore mergulhar\textunderscore )}
\end{itemize}
Tirar de debaixo da água.
Fazer emergir. Cf. Filinto, \textunderscore D. Man.\textunderscore , I, 382.
\section{Desmérito}
\begin{itemize}
\item {Grp. gram.:m.}
\end{itemize}
Falta de mérito.
Demérito.
\section{Desmesura}
\begin{itemize}
\item {Grp. gram.:f.}
\end{itemize}
\begin{itemize}
\item {Proveniência:(De \textunderscore des...\textunderscore  + \textunderscore mesura\textunderscore )}
\end{itemize}
Falta de cortesia.
Indelicadeza.
\section{Desmesurabilidade}
\begin{itemize}
\item {Grp. gram.:f.}
\end{itemize}
Qualidade de desmesurável.
\section{Desmesurável}
\begin{itemize}
\item {Grp. gram.:adj.}
\end{itemize}
\begin{itemize}
\item {Proveniência:(De \textunderscore desmesurar\textunderscore )}
\end{itemize}
Que se não póde medir; desmedido; immenso.
\section{Desmesuradamente}
\begin{itemize}
\item {Grp. gram.:adv.}
\end{itemize}
De modo desmesurado.
\section{Desmesurado}
\begin{itemize}
\item {Grp. gram.:adj.}
\end{itemize}
\begin{itemize}
\item {Proveniência:(De \textunderscore desmesurar\textunderscore )}
\end{itemize}
Desmedido; enorme.
\section{Desmesurar}
\begin{itemize}
\item {Grp. gram.:v. t.}
\end{itemize}
\begin{itemize}
\item {Grp. gram.:V. p.}
\end{itemize}
\begin{itemize}
\item {Proveniência:(De \textunderscore des...\textunderscore  + \textunderscore mesura\textunderscore )}
\end{itemize}
Estender muito.
Descommedir-se.
\section{Desmethódico}
\begin{itemize}
\item {Grp. gram.:adj.}
\end{itemize}
Que não é methódico.
Em que não há méthodo. Cf. Filinto, I, 177.
\section{Desmetódico}
\begin{itemize}
\item {Grp. gram.:adj.}
\end{itemize}
Que não é metódico.
Em que não há método. Cf. Filinto, I, 177.
\section{Desmigado}
\begin{itemize}
\item {Grp. gram.:adj.}
\end{itemize}
\begin{itemize}
\item {Utilização:Ant.}
\end{itemize}
Que deixou de sêr amigo; indisposto com alguém.
(Por \textunderscore desamigado\textunderscore )
\section{Desmineralizar}
\begin{itemize}
\item {Grp. gram.:v. t.}
\end{itemize}
\begin{itemize}
\item {Utilização:bras}
\end{itemize}
\begin{itemize}
\item {Utilização:Neol.}
\end{itemize}
\begin{itemize}
\item {Proveniência:(De \textunderscore des...\textunderscore  + \textunderscore mineral\textunderscore )}
\end{itemize}
Tirar o mineral de.
\section{Desmiolar}
\begin{itemize}
\item {Grp. gram.:v. t.}
\end{itemize}
\begin{itemize}
\item {Utilização:Fig.}
\end{itemize}
Tirar o miolo ou os miolos a.
Tirar o juízo a.
Tornar louco.
\section{Desmiudar}
\begin{itemize}
\item {fónica:mi-u}
\end{itemize}
\begin{itemize}
\item {Grp. gram.:v. t.}
\end{itemize}
\begin{itemize}
\item {Proveniência:(De \textunderscore des...\textunderscore  + \textunderscore miúdo\textunderscore )}
\end{itemize}
Converter em miúdos.
Esmiuçar.
Pormenorizar.
\section{Desmobilado}
\begin{itemize}
\item {Grp. gram.:adj.}
\end{itemize}
\begin{itemize}
\item {Proveniência:(De \textunderscore desmobilar\textunderscore )}
\end{itemize}
Em que não há mobilia; desguarnecido de mobília: \textunderscore casa desmobilada\textunderscore .
\section{Desmobilar}
\begin{itemize}
\item {Grp. gram.:v. t.}
\end{itemize}
\begin{itemize}
\item {Proveniência:(De \textunderscore des...\textunderscore  + \textunderscore mobilar\textunderscore )}
\end{itemize}
Tirar a mobilia de (uma casa, um compartimento, etc.).
\section{Desmobilização}
\begin{itemize}
\item {Grp. gram.:f.}
\end{itemize}
Acto ou effeito de desmobilizar.
\section{Desmobilizar}
\begin{itemize}
\item {Grp. gram.:v. t.}
\end{itemize}
Deixar de mobilizar (um exército).
\section{Desmobilizável}
\begin{itemize}
\item {Grp. gram.:adj.}
\end{itemize}
Que se póde desmobilizar.
\section{Desmochar}
\begin{itemize}
\item {Grp. gram.:v. t.}
\end{itemize}
\begin{itemize}
\item {Utilização:Fig.}
\end{itemize}
Tornar mocho.
Estragar.
Derramar.
\section{Desmoderado}
\begin{itemize}
\item {Grp. gram.:adj.}
\end{itemize}
(V.immoderado)
\section{Desmódio}
\begin{itemize}
\item {Grp. gram.:m.}
\end{itemize}
\begin{itemize}
\item {Proveniência:(Do gr. \textunderscore desmos\textunderscore , ligação)}
\end{itemize}
Planta papilionácea.
\section{Desmoflogia}
\begin{itemize}
\item {Grp. gram.:f.}
\end{itemize}
\begin{itemize}
\item {Utilização:Med.}
\end{itemize}
Inchação inflamatoria dos ligamentos.
\section{Desmografia}
\begin{itemize}
\item {Grp. gram.:f.}
\end{itemize}
\begin{itemize}
\item {Utilização:Anat.}
\end{itemize}
\begin{itemize}
\item {Proveniência:(Do gr. \textunderscore desmos\textunderscore  + \textunderscore graphein\textunderscore )}
\end{itemize}
Descripção dos ligamentos.
\section{Desmógrafo}
\begin{itemize}
\item {Grp. gram.:m.}
\end{itemize}
Tratadista de desmografia.
\section{Desmographia}
\begin{itemize}
\item {Grp. gram.:f.}
\end{itemize}
\begin{itemize}
\item {Utilização:Anat.}
\end{itemize}
\begin{itemize}
\item {Proveniência:(Do gr. \textunderscore desmos\textunderscore  + \textunderscore graphein\textunderscore )}
\end{itemize}
Descripção dos ligamentos.
\section{Desmógrapho}
\begin{itemize}
\item {Grp. gram.:m.}
\end{itemize}
Tratadista de desmographia.
\section{Desmoirar}
\begin{itemize}
\item {Grp. gram.:v. t.}
\end{itemize}
\begin{itemize}
\item {Grp. gram.:V. p.}
\end{itemize}
Tirar a qualidade de moiro a.
Deixar de sêr moiro. Cf. Serpa, \textunderscore Solaus\textunderscore , 155.
\section{Desmoita}
\begin{itemize}
\item {Grp. gram.:f.}
\end{itemize}
Acto de desmoitar.
\section{Desmoitador}
\begin{itemize}
\item {Grp. gram.:adj.}
\end{itemize}
\begin{itemize}
\item {Grp. gram.:M.}
\end{itemize}
Que desmoita.
Aquelle que desmoita.
\section{Desmoitar}
\begin{itemize}
\item {Grp. gram.:v. t.}
\end{itemize}
\begin{itemize}
\item {Utilização:Fig.}
\end{itemize}
\begin{itemize}
\item {Proveniência:(De \textunderscore des...\textunderscore  + \textunderscore moita\textunderscore )}
\end{itemize}
Desbravar, arrotear, cortar o mato de, para cultivar.
Desbastar.
Tornar culto, civilizado.
\section{Desmologia}
\begin{itemize}
\item {Grp. gram.:f.}
\end{itemize}
\begin{itemize}
\item {Utilização:Anat.}
\end{itemize}
\begin{itemize}
\item {Proveniência:(Do gr. \textunderscore desmos\textunderscore  + \textunderscore logos\textunderscore )}
\end{itemize}
Tratado dos ligamentos.
\section{Desmonetização}
\begin{itemize}
\item {Grp. gram.:f.}
\end{itemize}
Acto de desmonetizar.
\section{Desmonetizar}
\begin{itemize}
\item {Grp. gram.:v. t.}
\end{itemize}
\begin{itemize}
\item {Proveniência:(De \textunderscore des...\textunderscore  + lat. \textunderscore moneta\textunderscore )}
\end{itemize}
Tirar a qualidade de moéda a.
\section{Desmontada}
\begin{itemize}
\item {Grp. gram.:f.}
\end{itemize}
Acto de desmontar.
\section{Desmontar}
\begin{itemize}
\item {Grp. gram.:v. t.}
\end{itemize}
\begin{itemize}
\item {Utilização:Fig.}
\end{itemize}
\begin{itemize}
\item {Proveniência:(De \textunderscore des...\textunderscore  + \textunderscore montar\textunderscore )}
\end{itemize}
Descavalgar.
Fazer descer; tirar de cima de.
Abater.
Demittir.
\section{Desmontável}
\begin{itemize}
\item {Grp. gram.:adj.}
\end{itemize}
\begin{itemize}
\item {Utilização:Constr.}
\end{itemize}
Que se póde desmontar ou desarmar: \textunderscore solhos desmontáveis\textunderscore .
\section{Desmonte}
\begin{itemize}
\item {Grp. gram.:m.}
\end{itemize}
Acto de desmontar.
Acção de tirar dos jazigos o minério.
Conjunto de seixos e areia.
\section{Desmopathia}
\begin{itemize}
\item {Grp. gram.:f.}
\end{itemize}
\begin{itemize}
\item {Utilização:Med.}
\end{itemize}
\begin{itemize}
\item {Proveniência:(Do gr. \textunderscore desmos\textunderscore  + \textunderscore pathos\textunderscore )}
\end{itemize}
Doença dos ligamentos.
\section{Desmopatia}
\begin{itemize}
\item {Grp. gram.:f.}
\end{itemize}
\begin{itemize}
\item {Utilização:Med.}
\end{itemize}
\begin{itemize}
\item {Proveniência:(Do gr. \textunderscore desmos\textunderscore  + \textunderscore pathos\textunderscore )}
\end{itemize}
Doença dos ligamentos.
\section{Desmophlogia}
\begin{itemize}
\item {Grp. gram.:f.}
\end{itemize}
\begin{itemize}
\item {Utilização:Med.}
\end{itemize}
Inchação inflammatoria dos ligamentos.
\section{Desmoralização}
\begin{itemize}
\item {Grp. gram.:f.}
\end{itemize}
Acto ou effeito de desmoralizar.
\section{Desmoralizado}
\begin{itemize}
\item {Grp. gram.:adj.}
\end{itemize}
\begin{itemize}
\item {Proveniência:(De \textunderscore desmoralizar\textunderscore )}
\end{itemize}
Pervertido; corrupto.
\section{Desmoralizador}
\begin{itemize}
\item {Grp. gram.:adj.}
\end{itemize}
\begin{itemize}
\item {Grp. gram.:M.}
\end{itemize}
Que desmoraliza.
Aquelle que desmoraliza.
\section{Desmoralizar}
\begin{itemize}
\item {Grp. gram.:v. t.}
\end{itemize}
\begin{itemize}
\item {Proveniência:(De \textunderscore des...\textunderscore  + \textunderscore moralizar\textunderscore )}
\end{itemize}
Tornar immoral.
Perverter.
Corromper.
\section{Desmorder}
\begin{itemize}
\item {Grp. gram.:v. i.}
\end{itemize}
\begin{itemize}
\item {Utilização:P. us.}
\end{itemize}
Deixar de morder.
\section{Desmoronadiço}
\begin{itemize}
\item {Grp. gram.:adj.}
\end{itemize}
\begin{itemize}
\item {Proveniência:(De \textunderscore desmoronar\textunderscore )}
\end{itemize}
Que se desmorona facilmente.
\section{Desmoronamento}
\begin{itemize}
\item {Grp. gram.:m.}
\end{itemize}
Acto ou effeito de desmoronar.
\section{Desmoronar}
\begin{itemize}
\item {Grp. gram.:v. t.}
\end{itemize}
Demolir.
Abater; derribar.
(Cast. \textunderscore desmoronar\textunderscore )
\section{Desmorrer}
\begin{itemize}
\item {Grp. gram.:v. i.}
\end{itemize}
\begin{itemize}
\item {Utilização:P. us.}
\end{itemize}
\begin{itemize}
\item {Proveniência:(De \textunderscore des...\textunderscore  + \textunderscore morrer\textunderscore )}
\end{itemize}
Estar quási a morrer e restabelecer-se.
\section{Desmortalhar}
\textunderscore v. t.\textunderscore  (e der.)
O mesmo que \textunderscore desamortalhar\textunderscore , etc. Cf. Castilho, \textunderscore Fausto\textunderscore , 67.
\section{Desmortes}
\begin{itemize}
\item {Grp. gram.:f. pl.}
\end{itemize}
\begin{itemize}
\item {Utilização:Prov.}
\end{itemize}
\begin{itemize}
\item {Utilização:trasm.}
\end{itemize}
\textunderscore Bater ás desmortes\textunderscore , bater ás cegas, á tôa, a matar.
\section{Desmortificar}
\begin{itemize}
\item {Grp. gram.:v. t.}
\end{itemize}
Tirar mortificação a:«\textunderscore mortifica os pés, desgraçado, desmortifica-os depois\textunderscore ». M. Assis, \textunderscore Brás Cubas\textunderscore .
\section{Desmotivado}
\begin{itemize}
\item {Grp. gram.:adj.}
\end{itemize}
\begin{itemize}
\item {Proveniência:(De \textunderscore des...\textunderscore  + \textunderscore motivado\textunderscore )}
\end{itemize}
Que não tem motivo ou fundamento; infundado.
Que não tem justificação. Cf. Castilho, \textunderscore Fastos\textunderscore , I, 328.
\section{Desmotomia}
\begin{itemize}
\item {Grp. gram.:f.}
\end{itemize}
\begin{itemize}
\item {Utilização:Anat.}
\end{itemize}
\begin{itemize}
\item {Proveniência:(Do gr. \textunderscore desmos\textunderscore  + \textunderscore tome\textunderscore )}
\end{itemize}
Dissecação dos ligamentos.
\section{Desmouchar}
\begin{itemize}
\item {Grp. gram.:v. t.}
\end{itemize}
\begin{itemize}
\item {Utilização:Des.}
\end{itemize}
\begin{itemize}
\item {Proveniência:(De \textunderscore des...\textunderscore  + \textunderscore mouchão\textunderscore )}
\end{itemize}
Podar.
Desmoitar.
\section{Desmoutador}
\begin{itemize}
\item {Grp. gram.:adj.}
\end{itemize}
\begin{itemize}
\item {Grp. gram.:M.}
\end{itemize}
Que desmouta.
Aquelle que desmouta.
\section{Desmoutar}
\begin{itemize}
\item {Grp. gram.:v. t.}
\end{itemize}
\begin{itemize}
\item {Utilização:Fig.}
\end{itemize}
\begin{itemize}
\item {Proveniência:(De \textunderscore des...\textunderscore  + \textunderscore mouta\textunderscore )}
\end{itemize}
Desbravar, arrotear, cortar o mato de, para cultivar.
Desbastar.
Tornar culto, civilizado.
\section{Desmudança}
\begin{itemize}
\item {Grp. gram.:f.}
\end{itemize}
O mesmo que \textunderscore mudança\textunderscore . Cf. Filinto, X, 259.
\section{Desmudo}
\begin{itemize}
\item {Grp. gram.:adj.}
\end{itemize}
Que deixou de sêr mudo. Cf. Filinto, V, 201.
\section{Desmurar}
\begin{itemize}
\item {Grp. gram.:v. t.}
\end{itemize}
\begin{itemize}
\item {Proveniência:(De \textunderscore des...\textunderscore  + \textunderscore murar\textunderscore )}
\end{itemize}
Derribar os muros de.
\section{Desmurchar}
\begin{itemize}
\item {Grp. gram.:v. i.}
\end{itemize}
\begin{itemize}
\item {Utilização:P. us.}
\end{itemize}
\begin{itemize}
\item {Proveniência:(De \textunderscore des...\textunderscore  + \textunderscore murchar\textunderscore )}
\end{itemize}
Deixar de estar murcho.
\section{Desmúsica}
\begin{itemize}
\item {Grp. gram.:f.}
\end{itemize}
Ausência de música ou de harmonia. Cf. \textunderscore Eufrosina\textunderscore , 170.
\section{Desnacional}
\begin{itemize}
\item {Grp. gram.:adj.}
\end{itemize}
Que não tem carácter nacional; antipatriótico.
\section{Desnacionalização}
\begin{itemize}
\item {Grp. gram.:f.}
\end{itemize}
Acto ou effeito de desnacionalizar.
\section{Desnacionalizador}
\begin{itemize}
\item {Grp. gram.:adj.}
\end{itemize}
Que desnacionaliza.
\section{Desnacionalizar}
\begin{itemize}
\item {Grp. gram.:v. t.}
\end{itemize}
\begin{itemize}
\item {Proveniência:(De \textunderscore des...\textunderscore  + \textunderscore nacionalizar\textunderscore )}
\end{itemize}
Tirar a feição nacional a.
\section{Desnalgado}
\begin{itemize}
\item {Grp. gram.:adj.}
\end{itemize}
\begin{itemize}
\item {Proveniência:(De \textunderscore desnalgar-se\textunderscore )}
\end{itemize}
Que tem ancas pequenas e magras.
Que se desnalgou.
\section{Desnalgar-se}
\begin{itemize}
\item {Grp. gram.:v. p.}
\end{itemize}
\begin{itemize}
\item {Utilização:P. us.}
\end{itemize}
\begin{itemize}
\item {Proveniência:(De \textunderscore des...\textunderscore  + \textunderscore nalga\textunderscore )}
\end{itemize}
Mostrar as nalgas levantando os vestidos.
\section{Desnarigar}
\begin{itemize}
\item {Grp. gram.:v. t.}
\end{itemize}
\begin{itemize}
\item {Proveniência:(Do lat. \textunderscore de\textunderscore  + \textunderscore naricare\textunderscore )}
\end{itemize}
Tirar o nariz a.
\section{Desnasalação}
\begin{itemize}
\item {Grp. gram.:f.}
\end{itemize}
\begin{itemize}
\item {Utilização:Gram.}
\end{itemize}
\begin{itemize}
\item {Proveniência:(De \textunderscore desnasalar\textunderscore )}
\end{itemize}
Transformação de uma vogal nasal em vogal oral, em seguida á sýncope do \textunderscore n\textunderscore : \textunderscore luna\textunderscore  &lt; \textunderscore lũa\textunderscore  &lt; \textunderscore lua\textunderscore .
\section{Desnasalar}
\textunderscore v. t.\textunderscore  (e der.)
O mesmo que \textunderscore desnasalizar\textunderscore .
\section{Desnasalizar}
\begin{itemize}
\item {Grp. gram.:v.}
\end{itemize}
\begin{itemize}
\item {Utilização:t. Gram.}
\end{itemize}
\begin{itemize}
\item {Proveniência:(De \textunderscore des...\textunderscore  + \textunderscore nasalizar\textunderscore )}
\end{itemize}
Tirar o som nasal a.
\section{Desnascer}
\begin{itemize}
\item {Grp. gram.:v. i.}
\end{itemize}
\begin{itemize}
\item {Utilização:P. us.}
\end{itemize}
Deixar de nascer, tendo começado a vir á luz.
\section{Desnastrar}
\textunderscore v. t.\textunderscore  (e der.)
O mesmo que \textunderscore desennastrar\textunderscore , etc. Cf. Rebello, \textunderscore Contos e Lendas\textunderscore , 115.
\section{Desnatação}
\begin{itemize}
\item {Grp. gram.:f.}
\end{itemize}
Acto ou effeito de desnatar.
\section{Desnatadeira}
\begin{itemize}
\item {Grp. gram.:f.}
\end{itemize}
\begin{itemize}
\item {Proveniência:(De \textunderscore desnatar\textunderscore )}
\end{itemize}
Apparelho que separa do leite a nata ou o creme, para o fabríco da manteiga.
\section{Desnatar}
\begin{itemize}
\item {Grp. gram.:v. t.}
\end{itemize}
Tirar a nata ou o nateiro a.
\section{Desnaturação}
\begin{itemize}
\item {Grp. gram.:f.}
\end{itemize}
Acto ou effeito de desnaturar.
\section{Desnaturadamente}
\begin{itemize}
\item {Grp. gram.:adv.}
\end{itemize}
De modo desnaturado.
\section{Desnaturado}
\begin{itemize}
\item {Grp. gram.:adj.}
\end{itemize}
\begin{itemize}
\item {Grp. gram.:M.}
\end{itemize}
\begin{itemize}
\item {Proveniência:(De \textunderscore desnaturar\textunderscore )}
\end{itemize}
Não conforme á natureza ou aos sentimentos naturaes.
Descaroável, cruel.
Indivíduo desnaturado.
\section{Desnatural}
\begin{itemize}
\item {Grp. gram.:adj.}
\end{itemize}
Que não é natural.
Excêntrico.
Extraordinário.
\section{Desnaturalismo}
\begin{itemize}
\item {Grp. gram.:m.}
\end{itemize}
Falta de naturalismo. Cf. Camillo, \textunderscore Narcóticos\textunderscore , I, 281.
\section{Desnaturalização}
\begin{itemize}
\item {Grp. gram.:f.}
\end{itemize}
Acto ou effeito de desnaturalizar.
\section{Desnaturalizar}
\begin{itemize}
\item {Grp. gram.:v. t.}
\end{itemize}
\begin{itemize}
\item {Proveniência:(De \textunderscore des...\textunderscore  + \textunderscore naturalizar\textunderscore )}
\end{itemize}
Tirar os direitos de cidadão de um país a.
\section{Desnaturante}
\begin{itemize}
\item {Grp. gram.:adj.}
\end{itemize}
Que desnatura.
Que altera ou adultera uma substância. Cf. \textunderscore Diário do Govêrno\textunderscore , de 13-XI-901.
\section{Desnaturar}
\begin{itemize}
\item {Grp. gram.:v. t.}
\end{itemize}
\begin{itemize}
\item {Proveniência:(De \textunderscore des...\textunderscore  + \textunderscore natura\textunderscore )}
\end{itemize}
Tornar opposto aos sentimentos que são naturaes ao homem.
Alterar a natureza de.
Tornar deshumano, cruel.
O mesmo que \textunderscore desnaturalizar\textunderscore .
\section{Desnavegável}
\begin{itemize}
\item {Grp. gram.:adj.}
\end{itemize}
\begin{itemize}
\item {Proveniência:(De \textunderscore des...\textunderscore  + \textunderscore navegável\textunderscore )}
\end{itemize}
Em que se não póde navegar.
Que não admitte navegação ou em que se não póde ancorar. Cf. F. Borges, \textunderscore Diccion. Jur.\textunderscore 
\section{Desnecessariamente}
\begin{itemize}
\item {Grp. gram.:adv.}
\end{itemize}
De modo desnecessário.
\section{Desnecessário}
\begin{itemize}
\item {Grp. gram.:adj.}
\end{itemize}
Que não é necessário.
Dispensável; escusado.
\section{Desnecessidade}
\begin{itemize}
\item {Grp. gram.:f.}
\end{itemize}
Falta de necessidade.
Inutilidade.
\section{Desnegar}
\begin{itemize}
\item {Grp. gram.:v. t.}
\end{itemize}
\begin{itemize}
\item {Utilização:Pop.}
\end{itemize}
\begin{itemize}
\item {Proveniência:(De \textunderscore negar\textunderscore )}
\end{itemize}
Negar.
\section{Desnegociar}
\begin{itemize}
\item {Grp. gram.:v. t.}
\end{itemize}
\begin{itemize}
\item {Proveniência:(De \textunderscore des...\textunderscore  + \textunderscore negociar\textunderscore )}
\end{itemize}
Desfazer o negócio ou a combinação á cêrca de: \textunderscore desnegociar um contrato\textunderscore . Cf. Filinto, IV, 254.
\section{Desneixar}
\begin{itemize}
\item {Grp. gram.:v. t.}
\end{itemize}
\begin{itemize}
\item {Utilização:Prov.}
\end{itemize}
\begin{itemize}
\item {Utilização:trasm.}
\end{itemize}
Desconjuntar (os ossos, etc.)
(Por desnexar, de \textunderscore des...\textunderscore  + \textunderscore nexo\textunderscore )
\section{Desneixo}
\begin{itemize}
\item {Grp. gram.:adj.}
\end{itemize}
\begin{itemize}
\item {Utilização:Prov.}
\end{itemize}
\begin{itemize}
\item {Utilização:trasm.}
\end{itemize}
\begin{itemize}
\item {Proveniência:(De \textunderscore desneixar\textunderscore )}
\end{itemize}
Roto, maltrapilho, desprezível.
\section{Desnervamento}
\begin{itemize}
\item {Grp. gram.:m.}
\end{itemize}
Acto de desnervar.
\section{Desnervar}
\begin{itemize}
\item {Grp. gram.:v. t.}
\end{itemize}
O mesmo que \textunderscore enervar\textunderscore ^1.
\section{Desnevada}
\begin{itemize}
\item {Grp. gram.:f.}
\end{itemize}
\begin{itemize}
\item {Utilização:Prov.}
\end{itemize}
\begin{itemize}
\item {Utilização:trasm.}
\end{itemize}
Acto de desnevar.
O mesmo que \textunderscore descampatória\textunderscore .
\section{Desnevar}
\begin{itemize}
\item {Grp. gram.:v. i.}
\end{itemize}
O mesmo que \textunderscore degelar\textunderscore .
\section{Desninhar}
\begin{itemize}
\item {Grp. gram.:v. t.}
\end{itemize}
(V.desaninhar)
\section{Desniquelagem}
\begin{itemize}
\item {Grp. gram.:f.}
\end{itemize}
Acto de desniquelar.
\section{Desniquelar}
\begin{itemize}
\item {Grp. gram.:v. t.}
\end{itemize}
\begin{itemize}
\item {Proveniência:(De \textunderscore des...\textunderscore  + \textunderscore niquelar\textunderscore )}
\end{itemize}
Separar ou tirar o níquel a.
\section{Desnível}
\begin{itemize}
\item {Grp. gram.:m.}
\end{itemize}
\begin{itemize}
\item {Proveniência:(De \textunderscore des...\textunderscore  + \textunderscore nível\textunderscore )}
\end{itemize}
Differença do nível.
\section{Desnivél}
\begin{itemize}
\item {Grp. gram.:m.}
\end{itemize}
\begin{itemize}
\item {Proveniência:(De \textunderscore des...\textunderscore  + \textunderscore nível\textunderscore )}
\end{itemize}
Differença do nível.
\section{Desnivelamento}
\begin{itemize}
\item {Grp. gram.:m.}
\end{itemize}
Acto ou effeito de desnivelar.
\section{Desnivelar}
\begin{itemize}
\item {Grp. gram.:v. t.}
\end{itemize}
\begin{itemize}
\item {Proveniência:(De \textunderscore des...\textunderscore  + \textunderscore nivelar\textunderscore )}
\end{itemize}
Tirar do nivelamento.
\section{Desnobre}
\begin{itemize}
\item {Grp. gram.:adj.}
\end{itemize}
Que não é nobre.
Vil.
Reles. Cf. Filinto, XIV, 149.
\section{Desnobrecer}
\begin{itemize}
\item {Grp. gram.:v. t.}
\end{itemize}
(V.desennobrecer)
\section{Desnocamento}
\begin{itemize}
\item {Grp. gram.:m.}
\end{itemize}
Acto de desnocar.
\section{Desnocar}
\begin{itemize}
\item {Grp. gram.:v. t.}
\end{itemize}
\begin{itemize}
\item {Utilização:Pop.}
\end{itemize}
\begin{itemize}
\item {Proveniência:(De \textunderscore deslocar\textunderscore , talvez sôb a infl. de \textunderscore nó\textunderscore )}
\end{itemize}
Desarticular (um dedo, um braço, uma perna).
Tirar da articulação ou junta, que o povo chama nó. Cf. Camillo, \textunderscore Maria da Fonte\textunderscore , 344.
\section{Desnodar}
\begin{itemize}
\item {Grp. gram.:v. t.}
\end{itemize}
(V.denodar)
\section{Desnodoante}
\begin{itemize}
\item {Grp. gram.:adj.}
\end{itemize}
Que desnodôa.
\section{Desnodoar}
\begin{itemize}
\item {Grp. gram.:v. t.}
\end{itemize}
Tirar nódoas a.
\section{Desnodoso}
\begin{itemize}
\item {Grp. gram.:adj.}
\end{itemize}
\begin{itemize}
\item {Proveniência:(De \textunderscore des...\textunderscore  + \textunderscore nodoso\textunderscore )}
\end{itemize}
Que não tem nós.
\section{Desnoivar}
\begin{itemize}
\item {Grp. gram.:v. t.}
\end{itemize}
\begin{itemize}
\item {Utilização:Fig.}
\end{itemize}
\begin{itemize}
\item {Proveniência:(De \textunderscore des...\textunderscore  + \textunderscore noivo\textunderscore )}
\end{itemize}
Apartar (noivos).
Dissolver os esponsaes de.
Privar.
\section{Desnorteado}
\begin{itemize}
\item {Grp. gram.:adj.}
\end{itemize}
\begin{itemize}
\item {Proveniência:(De \textunderscore desnortear\textunderscore )}
\end{itemize}
Que não tem juizo; maníaco.
\section{Desnorteamento}
\begin{itemize}
\item {Grp. gram.:m.}
\end{itemize}
Acto ou effeito de desnortear.
\section{Desnortear}
\begin{itemize}
\item {Grp. gram.:v. t.}
\end{itemize}
\begin{itemize}
\item {Utilização:Fig.}
\end{itemize}
\begin{itemize}
\item {Proveniência:(De \textunderscore des...\textunderscore  + \textunderscore nortear\textunderscore )}
\end{itemize}
Desviar do rumo.
Perturbar, desorientar.
\section{Desnotar}
\begin{itemize}
\item {Grp. gram.:v. t.}
\end{itemize}
\begin{itemize}
\item {Proveniência:(De \textunderscore des...\textunderscore  + \textunderscore notar\textunderscore )}
\end{itemize}
Tirar a nota de.
\section{Desnovelar}
\begin{itemize}
\item {Grp. gram.:v. t.}
\end{itemize}
(V.desennovelar)
\section{Desnuar}
\begin{itemize}
\item {Grp. gram.:v. t.}
\end{itemize}
(V.denudar)
\section{Desnublar}
\begin{itemize}
\item {Grp. gram.:v. t.}
\end{itemize}
\begin{itemize}
\item {Proveniência:(De \textunderscore des...\textunderscore  + \textunderscore nublar\textunderscore )}
\end{itemize}
Tirar as nuvens de.
Aclarar, esclarecer.
\section{Desnudação}
\begin{itemize}
\item {Grp. gram.:f.}
\end{itemize}
O mesmo que \textunderscore desnudamento\textunderscore .
\section{Desnudamento}
\begin{itemize}
\item {Grp. gram.:m.}
\end{itemize}
Acto ou effeito de desnudar.
\section{Desnudar}
\begin{itemize}
\item {Grp. gram.:v. t.}
\end{itemize}
(V.denudar)
\section{Desnudez}
\begin{itemize}
\item {Grp. gram.:f.}
\end{itemize}
(V.nudez)
\section{Desnudo}
\begin{itemize}
\item {Grp. gram.:adj.}
\end{itemize}
\begin{itemize}
\item {Proveniência:(De \textunderscore des...\textunderscore  + lat. \textunderscore nudus\textunderscore . Cp. \textunderscore desinquieto\textunderscore )}
\end{itemize}
Despido, nu.
\section{Desnutrição}
\begin{itemize}
\item {Grp. gram.:f.}
\end{itemize}
Falta de nutrição; emmagrecimento.
\section{Desnuviar}
\textunderscore v. t.\textunderscore  (e der.)
O mesmo que \textunderscore desanuviar\textunderscore , etc. Cf. Castilho, \textunderscore Fastos\textunderscore , III, 163.
\section{Desobedecer}
\begin{itemize}
\item {Grp. gram.:v. i.}
\end{itemize}
\begin{itemize}
\item {Grp. gram.:V. t.}
\end{itemize}
Não obedecer.
Recalcitrar.
Não obedecer a. Cf. Usque, \textunderscore Tribulações\textunderscore , 33.
\section{Desobedecido}
\begin{itemize}
\item {Grp. gram.:adj.}
\end{itemize}
Que não é obedecido.
\section{Desobediência}
\begin{itemize}
\item {Grp. gram.:f.}
\end{itemize}
Falta de obediência.
\section{Desobediente}
\begin{itemize}
\item {Grp. gram.:adj.}
\end{itemize}
\begin{itemize}
\item {Proveniência:(De \textunderscore des...\textunderscore  + \textunderscore obediente\textunderscore )}
\end{itemize}
Que desobedece.
\section{Desobedientemente}
\begin{itemize}
\item {Grp. gram.:adv.}
\end{itemize}
\begin{itemize}
\item {Proveniência:(De \textunderscore desobediente\textunderscore )}
\end{itemize}
Com desobediência.
\section{Desobra}
\begin{itemize}
\item {Grp. gram.:f.}
\end{itemize}
\begin{itemize}
\item {Utilização:P. us.}
\end{itemize}
\begin{itemize}
\item {Proveniência:(De \textunderscore des...\textunderscore  + \textunderscore obra\textunderscore )}
\end{itemize}
Falta de actividade.
Ausência de trabalho.
Inércia. Cf. Filinto, D. \textunderscore Man.\textunderscore , II, 39.
\section{Desobriga}
\begin{itemize}
\item {Grp. gram.:f.}
\end{itemize}
\begin{itemize}
\item {Proveniência:(De \textunderscore desobrigar\textunderscore )}
\end{itemize}
O mesmo que \textunderscore desarrisca\textunderscore .
Desobrigação.
\section{Desobrigação}
\begin{itemize}
\item {Grp. gram.:f.}
\end{itemize}
Acto ou effeito de desobrigar.
\section{Desobrigadas}
\begin{itemize}
\item {Grp. gram.:f. pl.}
\end{itemize}
\begin{itemize}
\item {Utilização:Prov.}
\end{itemize}
\begin{itemize}
\item {Utilização:minh.}
\end{itemize}
\begin{itemize}
\item {Proveniência:(De \textunderscore desobrigar\textunderscore )}
\end{itemize}
Laranjas, que se comem antes do nono dia da quaresma e que fazem mal.
\section{Desobrigar}
\begin{itemize}
\item {Grp. gram.:v. t.}
\end{itemize}
\begin{itemize}
\item {Grp. gram.:V. p.}
\end{itemize}
\begin{itemize}
\item {Proveniência:(De \textunderscore des...\textunderscore  + \textunderscore obrigar\textunderscore )}
\end{itemize}
Livrar de uma obrigação.
Cumprir o preceito da confissão e communhão annuaes.
\section{Desobrigatório}
\begin{itemize}
\item {Grp. gram.:adj.}
\end{itemize}
\begin{itemize}
\item {Proveniência:(De \textunderscore des...\textunderscore  + \textunderscore obrigatório\textunderscore )}
\end{itemize}
Que desobriga.
\section{Desobscurecer}
\begin{itemize}
\item {Grp. gram.:v. t.}
\end{itemize}
\begin{itemize}
\item {Proveniência:(De \textunderscore des...\textunderscore  + \textunderscore obscurecer\textunderscore )}
\end{itemize}
Dissipar as sombras de.
Aclarar, esclarecer.
\section{Desobstrução}
\begin{itemize}
\item {Grp. gram.:f.}
\end{itemize}
Acto ou efeito de desobstruir.
Desimpedimento.
\section{Desobstrucção}
\begin{itemize}
\item {Grp. gram.:f.}
\end{itemize}
Acto ou effeito de desobstruir.
Desimpedimento.
\section{Desobstruimento}
\begin{itemize}
\item {fónica:tru-i}
\end{itemize}
\begin{itemize}
\item {Grp. gram.:m.}
\end{itemize}
Acto ou effeito de desobstruir; desobstrucção. Cf. Herculano, \textunderscore Hist. de Port.\textunderscore , IV, 137.
\section{Desobstruir}
\begin{itemize}
\item {Grp. gram.:v. t.}
\end{itemize}
\begin{itemize}
\item {Proveniência:(De \textunderscore des...\textunderscore  + \textunderscore obstruir\textunderscore )}
\end{itemize}
Desimpedir.
Desatravancar; desembaraçar: \textunderscore desobstruir um camínho\textunderscore .
\section{Desobstrutivo}
\begin{itemize}
\item {Grp. gram.:adj.}
\end{itemize}
Que desobstrue.
\section{Desoccupação}
\begin{itemize}
\item {Grp. gram.:f.}
\end{itemize}
Estado de quem se acha desoccupado.
Acto ou effeito de desoccupar.
Ociosidade.
\section{Desoccupadamente}
\begin{itemize}
\item {Grp. gram.:adv.}
\end{itemize}
\begin{itemize}
\item {Proveniência:(De \textunderscore desoccupado\textunderscore )}
\end{itemize}
Sem occupação.
\section{Desoccupado}
\begin{itemize}
\item {Grp. gram.:adj.}
\end{itemize}
Que não tem occupação.
Ocioso; \textunderscore que não tem que fazer\textunderscore .
Abandonado; devoluto: \textunderscore casa desoccupada\textunderscore .
\section{Desoccupar}
\begin{itemize}
\item {Grp. gram.:v. t.}
\end{itemize}
\begin{itemize}
\item {Proveniência:(De \textunderscore des...\textunderscore  + \textunderscore occupar\textunderscore )}
\end{itemize}
Deixar de occupar.
Deixar livre; desimpedir: \textunderscore desoccupar uma loja\textunderscore .
\section{Desocupação}
\begin{itemize}
\item {Grp. gram.:f.}
\end{itemize}
Estado de quem se acha desocupado.
Acto ou efeito de desocupar.
Ociosidade.
\section{Desocupadamente}
\begin{itemize}
\item {Grp. gram.:adv.}
\end{itemize}
\begin{itemize}
\item {Proveniência:(De \textunderscore desocupado\textunderscore )}
\end{itemize}
Sem ocupação.
\section{Desocupado}
\begin{itemize}
\item {Grp. gram.:adj.}
\end{itemize}
Que não tem ocupação.
Ocioso; que não tem que fazer.
Abandonado; devoluto: \textunderscore casa desocupada\textunderscore .
\section{Desocupar}
\begin{itemize}
\item {Grp. gram.:v. t.}
\end{itemize}
\begin{itemize}
\item {Proveniência:(De \textunderscore des...\textunderscore  + \textunderscore ocupar\textunderscore )}
\end{itemize}
Deixar de ocupar.
Deixar livre; desimpedir: \textunderscore desocupar uma loja\textunderscore .
\section{Desodorizado}
\begin{itemize}
\item {Grp. gram.:adj.}
\end{itemize}
\begin{itemize}
\item {Proveniência:(De \textunderscore desodorizar\textunderscore )}
\end{itemize}
Que perdeu o cheiro ou odor.
\section{Desodorizante}
\begin{itemize}
\item {Grp. gram.:adj.}
\end{itemize}
Que desodoriza.
\section{Desodorizar}
\begin{itemize}
\item {Grp. gram.:v. t.}
\end{itemize}
\begin{itemize}
\item {Proveniência:(De \textunderscore des...\textunderscore  + \textunderscore odor\textunderscore )}
\end{itemize}
Supprimir o cheiro de.
\section{Desoffuscar}
\begin{itemize}
\item {Grp. gram.:v. t.}
\end{itemize}
\begin{itemize}
\item {Proveniência:(De \textunderscore des...\textunderscore  + \textunderscore offuscar\textunderscore )}
\end{itemize}
Desanuvear.
Tornar claro, esclarecer.
\section{Desofuscar}
\begin{itemize}
\item {Grp. gram.:v. t.}
\end{itemize}
\begin{itemize}
\item {Proveniência:(De \textunderscore des...\textunderscore  + \textunderscore ofuscar\textunderscore )}
\end{itemize}
Desanuvear.
Tornar claro, esclarecer.
\section{Desolação}
\begin{itemize}
\item {Grp. gram.:f.}
\end{itemize}
Acto ou effeito de desolar.
\section{Desolado}
\begin{itemize}
\item {Grp. gram.:adj.}
\end{itemize}
Solitário, triste:«\textunderscore a desolada viuva precisava de alguém\textunderscore ». Camillo, \textunderscore Onde está a Fel.\textunderscore , 22.
\section{Desolador}
\begin{itemize}
\item {Grp. gram.:adj.}
\end{itemize}
\begin{itemize}
\item {Grp. gram.:M.}
\end{itemize}
Que desola.
Aquelle que desola.
\section{Desolar}
\begin{itemize}
\item {Grp. gram.:v. t.}
\end{itemize}
\begin{itemize}
\item {Proveniência:(Lat. \textunderscore desolare\textunderscore )}
\end{itemize}
Despovoar; devastar.
Tornar solitário, triste.
\section{Desolhado}
\begin{itemize}
\item {Grp. gram.:adj.}
\end{itemize}
Que tem olhos mortiços, quási cerrados:«\textunderscore dias depois Corina saíra do seu quarto, pállida, desolhada e triste\textunderscore ». Camillo, \textunderscore Estrêl. Prop.\textunderscore , 178.
\section{Desolhar}
\begin{itemize}
\item {Grp. gram.:v. t.}
\end{itemize}
\begin{itemize}
\item {Proveniência:(De \textunderscore des...\textunderscore  + \textunderscore ôlho\textunderscore )}
\end{itemize}
Tirar os olhos de (algumas plantas).
\section{Desoneração}
\begin{itemize}
\item {Grp. gram.:f.}
\end{itemize}
Acto de desonerar.
\section{Desonerar}
\begin{itemize}
\item {Grp. gram.:v. t.}
\end{itemize}
O mesmo que \textunderscore exonerar\textunderscore .
\section{Desonestamente}
\begin{itemize}
\item {Grp. gram.:adv.}
\end{itemize}
De modo desonesto.
\section{Desonestar}
\begin{itemize}
\item {Grp. gram.:v. t.}
\end{itemize}
\begin{itemize}
\item {Utilização:Ant.}
\end{itemize}
\begin{itemize}
\item {Proveniência:(Do lat. \textunderscore dehonestare\textunderscore )}
\end{itemize}
Desonrar.
Injuriar.
\section{Desonestidade}
\begin{itemize}
\item {Grp. gram.:f.}
\end{itemize}
\begin{itemize}
\item {Proveniência:(De \textunderscore des...\textunderscore  + \textunderscore honestidade\textunderscore )}
\end{itemize}
Falta de honestidade.
Impudência.
\section{Desonesto}
\begin{itemize}
\item {Grp. gram.:m.}
\end{itemize}
\begin{itemize}
\item {Proveniência:(De \textunderscore des...\textunderscore  + \textunderscore honesto\textunderscore )}
\end{itemize}
Que não tem honestidade; impudico; devasso.
\section{Desonra}
\begin{itemize}
\item {Grp. gram.:f.}
\end{itemize}
\begin{itemize}
\item {Proveniência:(De \textunderscore des...\textunderscore  + \textunderscore honra\textunderscore )}
\end{itemize}
Falta de honra.
Perda de honra.
Descrédito.
\section{Desonradamente}
\begin{itemize}
\item {Grp. gram.:adv.}
\end{itemize}
\begin{itemize}
\item {Proveniência:(De \textunderscore des...\textunderscore  + \textunderscore honradamente\textunderscore )}
\end{itemize}
Com desonra.
\section{Desonradez}
\begin{itemize}
\item {Grp. gram.:f.}
\end{itemize}
Estado de desonrado.
Desonra.
\section{Desonradiço}
\begin{itemize}
\item {Grp. gram.:adj.}
\end{itemize}
\begin{itemize}
\item {Utilização:Ant.}
\end{itemize}
\begin{itemize}
\item {Proveniência:(De \textunderscore desonrar\textunderscore )}
\end{itemize}
Que desonra ou ofende.
\section{Desonrador}
\begin{itemize}
\item {Grp. gram.:adj.}
\end{itemize}
\begin{itemize}
\item {Grp. gram.:M.}
\end{itemize}
Que desonra.
Aquele que desonra.
\section{Desonrante}
\begin{itemize}
\item {Grp. gram.:adj.}
\end{itemize}
Que desonra, que avilta.
\section{Desonrar}
\begin{itemize}
\item {Grp. gram.:v. t.}
\end{itemize}
\begin{itemize}
\item {Proveniência:(De \textunderscore des...\textunderscore  + \textunderscore honrar\textunderscore )}
\end{itemize}
Ofender a honra, o pudor ou o crédito, de.
Desflorar.
Infamar; deslustrar.
\section{Desonrosamente}
\begin{itemize}
\item {Grp. gram.:adv.}
\end{itemize}
De modo desonroso.
\section{Desonroso}
\begin{itemize}
\item {Grp. gram.:adj.}
\end{itemize}
\begin{itemize}
\item {Proveniência:(De \textunderscore des...\textunderscore  + \textunderscore honroso\textunderscore )}
\end{itemize}
Que desonra.
Em que há desonra.
\section{Desopilação}
\begin{itemize}
\item {Grp. gram.:f.}
\end{itemize}
Acto ou effeito de desopilar.
\section{Desopilante}
\begin{itemize}
\item {Grp. gram.:adj.}
\end{itemize}
Que desopila.
\section{Desopilar}
\begin{itemize}
\item {Grp. gram.:v. t.}
\end{itemize}
\begin{itemize}
\item {Utilização:Fig.}
\end{itemize}
\begin{itemize}
\item {Proveniência:(De \textunderscore des...\textunderscore  + \textunderscore opilar\textunderscore )}
\end{itemize}
Alliviar; desobstruir.
\textunderscore Desopilar o figado\textunderscore , communicar alegria ou bem-estar.
\section{Desopilativo}
\begin{itemize}
\item {Grp. gram.:adj.}
\end{itemize}
O mesmo que \textunderscore desopilante\textunderscore .
\section{Desoportuno}
\begin{itemize}
\item {Grp. gram.:adj.}
\end{itemize}
O mesmo que \textunderscore inoportuno\textunderscore . Cf. Filinto, \textunderscore D. Man.\textunderscore , II, 55.
\section{Desopportuno}
\begin{itemize}
\item {Grp. gram.:adj.}
\end{itemize}
O mesmo que \textunderscore inopportuno\textunderscore . Cf. Filinto, \textunderscore D. Man.\textunderscore , II, 55.
\section{Desoppressão}
\begin{itemize}
\item {Grp. gram.:f.}
\end{itemize}
Acto ou effeito de desopprimir.
Allívio, desafôgo.
\section{Desoppressar}
\begin{itemize}
\item {Grp. gram.:v. t.}
\end{itemize}
\begin{itemize}
\item {Utilização:Des.}
\end{itemize}
\begin{itemize}
\item {Proveniência:(De \textunderscore desoppresso\textunderscore )}
\end{itemize}
O mesmo que \textunderscore desopprimir\textunderscore .
\section{Desoppressor}
\begin{itemize}
\item {Grp. gram.:adj.}
\end{itemize}
\begin{itemize}
\item {Grp. gram.:M.}
\end{itemize}
\begin{itemize}
\item {Proveniência:(De \textunderscore des...\textunderscore  + \textunderscore oppressor\textunderscore )}
\end{itemize}
Que desopprime.
Aquelle que desopprime.
\section{Desopprimir}
\begin{itemize}
\item {Grp. gram.:v. t.}
\end{itemize}
\begin{itemize}
\item {Proveniência:(De \textunderscore des...\textunderscore  + \textunderscore opprimir\textunderscore )}
\end{itemize}
Livrar de oppressão; alliviar.
\section{Desopressar}
\begin{itemize}
\item {Grp. gram.:v. t.}
\end{itemize}
\begin{itemize}
\item {Utilização:Des.}
\end{itemize}
\begin{itemize}
\item {Proveniência:(De \textunderscore desopresso\textunderscore )}
\end{itemize}
O mesmo que \textunderscore desoprimir\textunderscore .
\section{Desopressor}
\begin{itemize}
\item {Grp. gram.:adj.}
\end{itemize}
\begin{itemize}
\item {Grp. gram.:M.}
\end{itemize}
\begin{itemize}
\item {Proveniência:(De \textunderscore des...\textunderscore  + \textunderscore opressor\textunderscore )}
\end{itemize}
Que desoprime.
Aquele que desoprime.
\section{Desoprimir}
\begin{itemize}
\item {Grp. gram.:v. t.}
\end{itemize}
\begin{itemize}
\item {Proveniência:(De \textunderscore des...\textunderscore  + \textunderscore oprimir\textunderscore )}
\end{itemize}
Livrar de opressão; aliviar.
\section{Desorado}
\begin{itemize}
\item {Grp. gram.:adj.}
\end{itemize}
\begin{itemize}
\item {Proveniência:(De \textunderscore desoras\textunderscore )}
\end{itemize}
Que vem fóra de horas.
Intempestivo.
Inoportuno:«\textunderscore tratão-no de cruel e desorado inimigo...\textunderscore »Filinto, \textunderscore D. Man.\textunderscore , I, 342.
\section{Desorbitar}
\begin{itemize}
\item {Grp. gram.:v. t.}
\end{itemize}
\begin{itemize}
\item {Proveniência:(De \textunderscore des...\textunderscore  + \textunderscore órbita\textunderscore )}
\end{itemize}
Tirar da órbita.
Fazer saír da órbita. Cf. R. Ortigão, \textunderscore Hollanda\textunderscore , 158.
\section{Desordeiro}
\begin{itemize}
\item {Grp. gram.:m.  e  adj.}
\end{itemize}
\begin{itemize}
\item {Proveniência:(De \textunderscore des...\textunderscore  + \textunderscore ordeiro\textunderscore )}
\end{itemize}
O que costuma promover desordens.
Aquelle que gosta de arruaças.
\section{Desordem}
\begin{itemize}
\item {Grp. gram.:f.}
\end{itemize}
Falta de ordem.
Desarranjo.
Desalinho: \textunderscore trazia o fato em desordem\textunderscore .
Desvairamento.
Motim; barulho; rixa: \textunderscore promover desordens\textunderscore .
\section{Desordenadamente}
\begin{itemize}
\item {Grp. gram.:adv.}
\end{itemize}
\begin{itemize}
\item {Proveniência:(De \textunderscore desordenar\textunderscore )}
\end{itemize}
Em desordem.
Sem nexo.
Sem systema.
Á tôa.
\section{Desordenador}
\begin{itemize}
\item {Grp. gram.:adj.}
\end{itemize}
\begin{itemize}
\item {Grp. gram.:M.}
\end{itemize}
Que desordena.
Aquelle que desordena.
\section{Desordenar}
\begin{itemize}
\item {Grp. gram.:v. t.}
\end{itemize}
\begin{itemize}
\item {Proveniência:(De \textunderscore des...\textunderscore  + \textunderscore ordenar\textunderscore )}
\end{itemize}
Tirar da ordem.
Amotinar.
Confundir.
\section{Desordinário}
\begin{itemize}
\item {Grp. gram.:adj.}
\end{itemize}
Que não é ordinário. Cf. Filinto, XIII, 213.
\section{Desorelhamento}
\begin{itemize}
\item {Grp. gram.:m.}
\end{itemize}
Acto de desorelhar. Cf. Arn. Gama, \textunderscore Ultima Dona\textunderscore , 23 e 434.
\section{Desorelhar}
\begin{itemize}
\item {Grp. gram.:v. t.}
\end{itemize}
\begin{itemize}
\item {Utilização:Fig.}
\end{itemize}
Tirar as orelhas a.
Tirar as arrecadas das orelhas de.
\section{Desorganização}
\begin{itemize}
\item {Grp. gram.:f.}
\end{itemize}
Acto ou effeito de desorganizar.
\section{Desorganizador}
\begin{itemize}
\item {Grp. gram.:adj.}
\end{itemize}
\begin{itemize}
\item {Grp. gram.:M.}
\end{itemize}
Que desorganiza.
Aquelle que desorganiza.
\section{Desorganizar}
\begin{itemize}
\item {Grp. gram.:v. t.}
\end{itemize}
\begin{itemize}
\item {Proveniência:(De \textunderscore des...\textunderscore  + \textunderscore organizar\textunderscore )}
\end{itemize}
Desordenar; destruir a organização de: \textunderscore desorganizar uma associação\textunderscore .
Dissolver.
\section{Desorientação}
\begin{itemize}
\item {Grp. gram.:f.}
\end{itemize}
Acto de desorientar.
Insensatez.
\section{Desorientadamente}
\begin{itemize}
\item {Grp. gram.:adv.}
\end{itemize}
Com desorientação.
Á tôa.
Loucamente.
\section{Desorientado}
\begin{itemize}
\item {Grp. gram.:adj.}
\end{itemize}
\begin{itemize}
\item {Proveniência:(De \textunderscore desorientar\textunderscore )}
\end{itemize}
Desequilibrado; maníaco.
\section{Desorientamento}
\begin{itemize}
\item {Grp. gram.:m.}
\end{itemize}
O mesmo que \textunderscore desorientação\textunderscore .
\section{Desorientar}
\begin{itemize}
\item {Grp. gram.:v. t.}
\end{itemize}
\begin{itemize}
\item {Proveniência:(De \textunderscore des...\textunderscore  + \textunderscore orientar\textunderscore )}
\end{itemize}
O mesmo que \textunderscore desnortear\textunderscore .
Desvairar.
Ensandecer.
\section{Desornar}
\begin{itemize}
\item {Grp. gram.:v. t.}
\end{itemize}
\begin{itemize}
\item {Proveniência:(De \textunderscore des...\textunderscore  + \textunderscore ornar\textunderscore )}
\end{itemize}
O mesmo que \textunderscore desenfeitar\textunderscore .
\section{De-sorte}
\begin{itemize}
\item {Grp. gram.:loc. adv.}
\end{itemize}
\begin{itemize}
\item {Utilização:Prov.}
\end{itemize}
\begin{itemize}
\item {Proveniência:(Do lat. \textunderscore sortito\textunderscore ?)}
\end{itemize}
Sem probabilidade; só por acaso.
\section{Desossamento}
\begin{itemize}
\item {Grp. gram.:m.}
\end{itemize}
Acto ou effeito de desossar.
\section{Desossar}
\begin{itemize}
\item {Grp. gram.:v. t.}
\end{itemize}
Tirar os ossos a.
\section{Desougar}
\textunderscore v. t.\textunderscore  (e der.)
O mesmo que \textunderscore desaugar\textunderscore , etc.
\section{Desova}
\begin{itemize}
\item {Grp. gram.:f.}
\end{itemize}
Acto de desovar.
\section{Desovação}
\begin{itemize}
\item {Grp. gram.:f.}
\end{itemize}
O mesmo que \textunderscore desova\textunderscore . Cf. Ortigão, \textunderscore Praias\textunderscore , 14.
\section{Desovamento}
\begin{itemize}
\item {Grp. gram.:m.}
\end{itemize}
O mesmo que \textunderscore desova\textunderscore .
\section{Desovar}
\begin{itemize}
\item {Grp. gram.:v. i.}
\end{itemize}
Pôr ovos, (falando-se especialmente dos peixes).
\section{Desoxidação}
\begin{itemize}
\item {fónica:csi}
\end{itemize}
\begin{itemize}
\item {Grp. gram.:f.}
\end{itemize}
Acto ou efeito de desoxidar.
\section{Desoxidante}
\begin{itemize}
\item {fónica:csi}
\end{itemize}
\begin{itemize}
\item {Grp. gram.:adj.}
\end{itemize}
Que desoxida.
\section{Desoxidar}
\begin{itemize}
\item {fónica:csi}
\end{itemize}
\begin{itemize}
\item {Grp. gram.:v. t.}
\end{itemize}
\begin{itemize}
\item {Proveniência:(De \textunderscore des...\textunderscore  + \textunderscore oxidar\textunderscore )}
\end{itemize}
Tirar o óxido a.
Tirar a ferrugem a.
\section{Desoxigenação}
\begin{itemize}
\item {fónica:csi}
\end{itemize}
\begin{itemize}
\item {Grp. gram.:f.}
\end{itemize}
Acto ou efeito de desoxigenar.
\section{Desoxigenante}
\begin{itemize}
\item {fónica:csi}
\end{itemize}
\begin{itemize}
\item {Grp. gram.:adj.}
\end{itemize}
Que desoxigena.
\section{Desoxigenar}
\begin{itemize}
\item {fónica:csi}
\end{itemize}
\begin{itemize}
\item {Grp. gram.:v. t.}
\end{itemize}
\begin{itemize}
\item {Proveniência:(De \textunderscore des...\textunderscore  + \textunderscore oxigenar\textunderscore )}
\end{itemize}
O mesmo que \textunderscore desoxidar\textunderscore .
\section{Desoxydação}
\begin{itemize}
\item {Grp. gram.:f.}
\end{itemize}
Acto ou effeito de desoxydar.
\section{Desoxydante}
\begin{itemize}
\item {Grp. gram.:adj.}
\end{itemize}
Que desoxyda.
\section{Desoxydar}
\begin{itemize}
\item {Grp. gram.:v. t.}
\end{itemize}
\begin{itemize}
\item {Proveniência:(De \textunderscore des...\textunderscore  + \textunderscore oxydar\textunderscore )}
\end{itemize}
Tirar o óxydo a.
Tirar a ferrugem a.
\section{Desoxygenação}
\begin{itemize}
\item {Grp. gram.:f.}
\end{itemize}
Acto ou effeito de desoxygenar.
\section{Desoxygenante}
\begin{itemize}
\item {Grp. gram.:adj.}
\end{itemize}
Que desoxygena.
\section{Desoxygenar}
\begin{itemize}
\item {Grp. gram.:v. t.}
\end{itemize}
\begin{itemize}
\item {Proveniência:(De \textunderscore des...\textunderscore  + \textunderscore oxygenar\textunderscore )}
\end{itemize}
O mesmo que \textunderscore desoxydar\textunderscore .
\section{Despachadamente}
\begin{itemize}
\item {Grp. gram.:adv.}
\end{itemize}
\begin{itemize}
\item {Proveniência:(De \textunderscore despachado\textunderscore )}
\end{itemize}
Com desembaraço.
\section{Despachado}
\begin{itemize}
\item {Grp. gram.:adj.}
\end{itemize}
\begin{itemize}
\item {Utilização:Fam.}
\end{itemize}
\begin{itemize}
\item {Proveniência:(De \textunderscore despachar\textunderscore )}
\end{itemize}
Resolvido, concluido: \textunderscore negócio despachado\textunderscore .
Que teve despacho: \textunderscore requerimento despachado\textunderscore .
Expedito, desembaraçado.
Morto, assassinado.
\section{Despachador}
\begin{itemize}
\item {Grp. gram.:adj.}
\end{itemize}
\begin{itemize}
\item {Grp. gram.:M.}
\end{itemize}
Que despacha.
Aquelle que despacha.
\section{Despachante}
\begin{itemize}
\item {Grp. gram.:m.  e  adj.}
\end{itemize}
O mesmo que \textunderscore despachador\textunderscore .
\section{Despachar}
\begin{itemize}
\item {Grp. gram.:v. t.}
\end{itemize}
\begin{itemize}
\item {Proveniência:(Lat. hyp. \textunderscore dispactiare\textunderscore . Cp. it. \textunderscore dispacciare\textunderscore )}
\end{itemize}
Dar despacho a: \textunderscore despachar um pedido\textunderscore .
Resolver: \textunderscore despachar uma questão\textunderscore .
Desembaraçar.
Expedir: \textunderscore despachar mercadorias\textunderscore .
Aviar: \textunderscore não te demores, despacha-te\textunderscore .
\section{Despacho}
\begin{itemize}
\item {Grp. gram.:m.}
\end{itemize}
\begin{itemize}
\item {Proveniência:(De \textunderscore despachar\textunderscore )}
\end{itemize}
Acto de despachar: \textunderscore hoje o Ministro não dá despacho\textunderscore .
Nota de deferimento ou indeferimento, posta por uma autoridade em petição ou requerimento.
Provimento de emprêgo público.
Desfecho.
Telegramma.
\section{Despaciência}
\begin{itemize}
\item {Grp. gram.:f.}
\end{itemize}
\begin{itemize}
\item {Utilização:Prov.}
\end{itemize}
\begin{itemize}
\item {Utilização:trasm.}
\end{itemize}
O mesmo que \textunderscore impaciência\textunderscore .
\section{Despadrar}
\begin{itemize}
\item {Grp. gram.:v. t.}
\end{itemize}
\begin{itemize}
\item {Grp. gram.:V. p.}
\end{itemize}
Tirar a qualidade de padre a.
Deixar de sêr padre.
\section{Despalhar}
\begin{itemize}
\item {Grp. gram.:v. t.}
\end{itemize}
Tirar a palha a.
\section{Despalmar}
\begin{itemize}
\item {Grp. gram.:v. t.}
\end{itemize}
\begin{itemize}
\item {Proveniência:(De \textunderscore des...\textunderscore  + \textunderscore palma\textunderscore )}
\end{itemize}
Cortar ao cavallo (a palma ou a parte do casco, que assenta sôbre a ferradura).
\section{Despalmilhado}
\begin{itemize}
\item {Grp. gram.:adj.}
\end{itemize}
\begin{itemize}
\item {Utilização:Bras. do S}
\end{itemize}
\begin{itemize}
\item {Proveniência:(De \textunderscore despalmilhar-se\textunderscore )}
\end{itemize}
Diz-se do cavallo, molestado na parte molle do casco.
\section{Despalmilhar-se}
\begin{itemize}
\item {Grp. gram.:v. p.}
\end{itemize}
\begin{itemize}
\item {Utilização:Bras. do S}
\end{itemize}
\begin{itemize}
\item {Proveniência:(De \textunderscore des...\textunderscore  + \textunderscore palmilha\textunderscore )}
\end{itemize}
Tornar-se despalmilhado.
\section{Despampanar}
\begin{itemize}
\item {Grp. gram.:v. t.}
\end{itemize}
Tirar os pâmpanos a.
\section{Despampar}
\begin{itemize}
\item {Grp. gram.:v. t.}
\end{itemize}
O mesmo que \textunderscore despampanar\textunderscore .
\section{Despanda}
\begin{itemize}
\item {Grp. gram.:m.}
\end{itemize}
Chefe de povoação, na Índia portuguesa.
\section{Despapar}
\begin{itemize}
\item {Grp. gram.:v. i.  e  p.}
\end{itemize}
\begin{itemize}
\item {Proveniência:(De \textunderscore des...\textunderscore  + \textunderscore papo\textunderscore )}
\end{itemize}
Erguer muito o focinho, andando, (relativamente ao cavallo).
\section{Desparafusar}
\begin{itemize}
\item {Grp. gram.:v. t.}
\end{itemize}
O mesmo que \textunderscore desaparafusar\textunderscore .
\section{Desparamentar}
\begin{itemize}
\item {Grp. gram.:v. t.}
\end{itemize}
Tirar os paramentos a.
\section{Desparecer}
\begin{itemize}
\item {Grp. gram.:v. i.}
\end{itemize}
(Contr. de \textunderscore desapparecer\textunderscore ):«\textunderscore ri, desparece e nunca mais voltou\textunderscore ». Th. Ribeiro, \textunderscore Judia\textunderscore .
\section{Desparelhar}
\textunderscore v. t.\textunderscore  (e der.)
O mesmo que \textunderscore desaparelhar\textunderscore , etc. Cf. Filinto, XX, 244.
\section{Despargir}
\begin{itemize}
\item {Grp. gram.:v. t.}
\end{itemize}
O mesmo que \textunderscore espargir\textunderscore . Cf. \textunderscore Viriato Trág.\textunderscore , III, 108.
\section{Desparra}
\begin{itemize}
\item {Grp. gram.:f.}
\end{itemize}
Acto de desparrar.
\section{Desparrar}
\begin{itemize}
\item {Grp. gram.:v. t.}
\end{itemize}
Tirar as parras a.
\section{Despartir}
\begin{itemize}
\item {Grp. gram.:v. t.}
\end{itemize}
\begin{itemize}
\item {Grp. gram.:V. p.}
\end{itemize}
\begin{itemize}
\item {Utilização:Ant.}
\end{itemize}
(V.dispartir)
Ausentar-se, partir. Cf. \textunderscore Eufrosina\textunderscore , act. I, sc. 3.
\section{Desparzir}
\begin{itemize}
\item {Grp. gram.:v. t.}
\end{itemize}
O mesmo que \textunderscore espargir\textunderscore .
\section{Despassar}
\begin{itemize}
\item {Grp. gram.:v. t.}
\end{itemize}
Passar àlém de; transpor:«\textunderscore apenas D. João Coutinho despassara a ponte...\textunderscore »Filinto, \textunderscore D. Man.\textunderscore , III, 150.
\section{Despassarinhado}
\begin{itemize}
\item {Grp. gram.:adj.}
\end{itemize}
\begin{itemize}
\item {Utilização:Prov.}
\end{itemize}
\begin{itemize}
\item {Utilização:alg.}
\end{itemize}
Desajeitado; inhábil.
Que não tem engenho.
\section{Despastar}
\begin{itemize}
\item {Grp. gram.:v. t.}
\end{itemize}
O mesmo que \textunderscore pastar\textunderscore .
\section{Despatriota}
\begin{itemize}
\item {Grp. gram.:adj.}
\end{itemize}
O mesmo que \textunderscore antipatriótico\textunderscore . Cf. Camillo, \textunderscore Canc. Alegre\textunderscore , 327.
\section{Despatriótico}
\begin{itemize}
\item {Grp. gram.:adj.}
\end{itemize}
O mesmo ou melhor que \textunderscore despatriota\textunderscore .
\section{Despautério}
\begin{itemize}
\item {Grp. gram.:m.}
\end{itemize}
\begin{itemize}
\item {Utilização:Fam.}
\end{itemize}
\begin{itemize}
\item {Proveniência:(De \textunderscore Despautere\textunderscore , n. p. de um grammático flamengo)}
\end{itemize}
Grande disparate.
Tolice graúda; desconchavo.
\section{Despavorido}
\begin{itemize}
\item {Grp. gram.:adj.}
\end{itemize}
\begin{itemize}
\item {Proveniência:(De \textunderscore des...\textunderscore  + \textunderscore pavor\textunderscore )}
\end{itemize}
Que perdeu o pavor, que deixou de têr medo. Cf. Filinto, III, 377.
\section{Despavorir}
\begin{itemize}
\item {Grp. gram.:v. t.}
\end{itemize}
\begin{itemize}
\item {Proveniência:(T. cast.)}
\end{itemize}
Espavorir, causar susto a.
\section{Despeadamente}
\begin{itemize}
\item {Grp. gram.:adv.}
\end{itemize}
\begin{itemize}
\item {Proveniência:(De \textunderscore despear\textunderscore ^2)}
\end{itemize}
Sem peias.
\section{Despear}
\begin{itemize}
\item {Grp. gram.:v. t.}
\end{itemize}
\begin{itemize}
\item {Proveniência:(De \textunderscore des...\textunderscore  + \textunderscore pear\textunderscore )}
\end{itemize}
Tirar as peias a.
\section{Despear}
\begin{itemize}
\item {Grp. gram.:v. t.}
\end{itemize}
\begin{itemize}
\item {Proveniência:(De \textunderscore des...\textunderscore  + \textunderscore pé\textunderscore )}
\end{itemize}
Molestar muito os pés de.
Gastar os cascos de (uma cavalgadura).
\section{Despeçar}
\textunderscore v. t.\textunderscore  (e der.)
O mesmo que \textunderscore desempeçar\textunderscore , etc. Cf. Filinto, XIV, 149.
\section{Despecuniado}
\begin{itemize}
\item {Grp. gram.:adj.}
\end{itemize}
\begin{itemize}
\item {Utilização:Fam.}
\end{itemize}
\begin{itemize}
\item {Proveniência:(De \textunderscore despecuniar\textunderscore )}
\end{itemize}
Que não tem pecúnia.
Que não é endinheirado. Cf. Filinto, \textunderscore D. Man.\textunderscore , III, 154.
\section{Despecuniar}
\begin{itemize}
\item {Grp. gram.:v. t.}
\end{itemize}
\begin{itemize}
\item {Proveniência:(De \textunderscore des...\textunderscore  + \textunderscore pecúnia\textunderscore )}
\end{itemize}
Obrigar a despesas excessivas; privar de recursos pecuniários. Cf. Filinto, \textunderscore D. Man.\textunderscore , III, 154.
\section{Despedaçador}
\begin{itemize}
\item {Grp. gram.:adj.}
\end{itemize}
Que despedaça.
\section{Despedaçamento}
\begin{itemize}
\item {Grp. gram.:m.}
\end{itemize}
Acto ou effeito de despedaçar.
\section{Despedaçar}
\begin{itemize}
\item {Grp. gram.:v. t.}
\end{itemize}
\begin{itemize}
\item {Utilização:Fig.}
\end{itemize}
\begin{itemize}
\item {Proveniência:(De \textunderscore des...\textunderscore  + \textunderscore pedaço\textunderscore )}
\end{itemize}
Partir em pedaços.
Partir; rasgar: \textunderscore despedaçar um casaso\textunderscore .
Lancinar, pungir: \textunderscore infortúnios, que despedaçam o coração\textunderscore .
\section{Despedição}
\begin{itemize}
\item {Grp. gram.:f.}
\end{itemize}
\begin{itemize}
\item {Utilização:P. us.}
\end{itemize}
\begin{itemize}
\item {Proveniência:(De \textunderscore despedir\textunderscore )}
\end{itemize}
Despedida.
Expedição. Cf. Filinto, \textunderscore D. Man.\textunderscore , I, 337.
\section{Despedida}
\begin{itemize}
\item {Grp. gram.:f.}
\end{itemize}
\begin{itemize}
\item {Utilização:Fig.}
\end{itemize}
Acto de despedir ou despedir-se.
Termo, fim: \textunderscore os chrysântemos, a que o povo chama despedidas do verão...\textunderscore 
\section{Despedidas-do-verão}
\begin{itemize}
\item {Grp. gram.:f. pl.}
\end{itemize}
Designação vulgar dos chrysântemos.
\section{Despediente}
\begin{itemize}
\item {Grp. gram.:adj.}
\end{itemize}
\begin{itemize}
\item {Proveniência:(De \textunderscore despedir\textunderscore )}
\end{itemize}
Que despede. Cf. Filinto, \textunderscore D. Man.\textunderscore , II, 66.
\section{Despedimento}
\begin{itemize}
\item {Grp. gram.:m.}
\end{itemize}
Acto de despedir.
\section{Despedir}
\begin{itemize}
\item {Grp. gram.:v. t.}
\end{itemize}
\begin{itemize}
\item {Grp. gram.:V. i.}
\end{itemize}
\begin{itemize}
\item {Grp. gram.:V. p.}
\end{itemize}
\begin{itemize}
\item {Proveniência:(Do lat. \textunderscore de\textunderscore  + \textunderscore expedire\textunderscore )}
\end{itemize}
Fazer saír.
Dispensar os serviços de: \textunderscore despedir um criado\textunderscore .
Separar-se de.
Lançar, arremessar: \textunderscore despedir setas\textunderscore .
Expedir.
Aviar.
Fazer despedida.
Renunciar.
Apartar-se, cumprimentando: \textunderscore foi despedir-se do tio\textunderscore .
\section{Despedradamente}
\begin{itemize}
\item {Grp. gram.:adv.}
\end{itemize}
De modo despedrado: \textunderscore de modo desabrido\textunderscore .
\section{Despedrado}
\begin{itemize}
\item {Grp. gram.:adj.}
\end{itemize}
\begin{itemize}
\item {Utilização:Prov.}
\end{itemize}
\begin{itemize}
\item {Utilização:trasm.}
\end{itemize}
\begin{itemize}
\item {Proveniência:(De \textunderscore pedra\textunderscore )}
\end{itemize}
Ríspido; desabrido.
Estridente.
Aspero.
\section{Despegar}
\begin{itemize}
\item {Grp. gram.:V. i.}
\end{itemize}
\begin{itemize}
\item {Utilização:Pop.}
\end{itemize}
\textunderscore v. t.\textunderscore  (e der.)
O mesmo que \textunderscore desapegar\textunderscore , etc.
Cessar; abrir mão; deixar de se applicar: \textunderscore ás duas horas, despegou do trabalho\textunderscore .
\section{Despêgo}
\begin{itemize}
\item {Grp. gram.:m.}
\end{itemize}
O mesmo que \textunderscore desapêgo\textunderscore .
\section{Despeitador}
\begin{itemize}
\item {Grp. gram.:m.}
\end{itemize}
Aquelle que despeita.
\section{Despeitamente}
\begin{itemize}
\item {Grp. gram.:m.}
\end{itemize}
\begin{itemize}
\item {Utilização:Ant.}
\end{itemize}
O mesmo que \textunderscore despeito\textunderscore .
\section{Despeitar}
\begin{itemize}
\item {Grp. gram.:v. t.}
\end{itemize}
Causar despeito a.
Tornar amuado.
\section{Despeito}
\begin{itemize}
\item {Grp. gram.:m.}
\end{itemize}
\begin{itemize}
\item {Grp. gram.:Loc. prep.}
\end{itemize}
\begin{itemize}
\item {Proveniência:(Lat. \textunderscore despectus\textunderscore )}
\end{itemize}
Desgôsto, produzido por desconsideração ou ligeira offensa.
Pesar.
\textunderscore A despeito de\textunderscore , apesar de.
\section{Despeitoramento}
\begin{itemize}
\item {Grp. gram.:m.}
\end{itemize}
Acto de despeitorar.
\section{Despeitorar}
\begin{itemize}
\item {Grp. gram.:v. t.}
\end{itemize}
\begin{itemize}
\item {Utilização:Des.}
\end{itemize}
\begin{itemize}
\item {Grp. gram.:V. p.}
\end{itemize}
\begin{itemize}
\item {Proveniência:(Do lat. \textunderscore de\textunderscore  + \textunderscore expectorare\textunderscore )}
\end{itemize}
Desabafar.
Descobrir muito o peito, decotar-se. Cf. Camillo, \textunderscore Noites\textunderscore , V, 5.
\section{Despeitoso}
\begin{itemize}
\item {Grp. gram.:adj.}
\end{itemize}
Que causa despeito.
\section{Despejadamente}
\begin{itemize}
\item {Grp. gram.:adv.}
\end{itemize}
\begin{itemize}
\item {Proveniência:(De \textunderscore despejado\textunderscore )}
\end{itemize}
Sem pejo.
\section{Despejado}
\begin{itemize}
\item {Grp. gram.:adj.}
\end{itemize}
\begin{itemize}
\item {Proveniência:(De \textunderscore despejar\textunderscore )}
\end{itemize}
Desoccupado, vazio: \textunderscore um tonel despejado\textunderscore .
Em que não há pejo ou decência: \textunderscore linguagem despejada\textunderscore .
\section{Despejamento}
\begin{itemize}
\item {Grp. gram.:m.}
\end{itemize}
Acto de despejar.
\section{Despejar}
\begin{itemize}
\item {Grp. gram.:v. t.}
\end{itemize}
\begin{itemize}
\item {Utilização:Fig.}
\end{itemize}
\begin{itemize}
\item {Grp. gram.:V. i.}
\end{itemize}
\begin{itemize}
\item {Proveniência:(De \textunderscore des...\textunderscore  + \textunderscore pejar\textunderscore )}
\end{itemize}
Tirar ou desviar (aquillo que estorva ou peia).
Desembaraçar, desoccupar: \textunderscore despejar uma casa\textunderscore .
Vasar o que está contido em: \textunderscore despejar o vinho de uma garrafa\textunderscore .
Tornar vazio: \textunderscore despejar uma garrafa\textunderscore .
Tirar o pejo a.
Desmobilar a própria habitação.
Deixar uma habitação.
Despedir-se de um arrendamento.
\section{Despejo}
\begin{itemize}
\item {Grp. gram.:m.}
\end{itemize}
\begin{itemize}
\item {Utilização:Fig.}
\end{itemize}
Acto ou effeito de despejar.
Aquillo que se despeja.
Lixo; dejectos.
Falta de pejo.
Impudor, impudência.
Agilidade.
Intrepidez.
\section{Despela}
\begin{itemize}
\item {Grp. gram.:f.}
\end{itemize}
Acto de despelar.
\section{Despelar}
\begin{itemize}
\item {Grp. gram.:v. t.}
\end{itemize}
\begin{itemize}
\item {Proveniência:(De \textunderscore des...\textunderscore  + \textunderscore pele\textunderscore )}
\end{itemize}
Tirar a pele a.
Tirar a casca de.
\section{Despella}
\begin{itemize}
\item {Grp. gram.:f.}
\end{itemize}
Acto de despellar.
\section{Despellar}
\begin{itemize}
\item {Grp. gram.:v. t.}
\end{itemize}
\begin{itemize}
\item {Proveniência:(De \textunderscore des...\textunderscore  + \textunderscore pelle\textunderscore )}
\end{itemize}
Tirar a pelle a.
Tirar a casca de.
\section{Despenadora}
\begin{itemize}
\item {Grp. gram.:f.}
\end{itemize}
\begin{itemize}
\item {Utilização:Des.}
\end{itemize}
\begin{itemize}
\item {Proveniência:(De \textunderscore despenar\textunderscore )}
\end{itemize}
Mulher que, á semelhança da seita dos abafadores, suffocava os moribundos, para lhes abreviar a agonia.
\section{Despenar}
\begin{itemize}
\item {Grp. gram.:v. t.}
\end{itemize}
Livrar de penas.
Consolar:«\textunderscore daqui me despeno da coima de linguareiro\textunderscore ». Camillo, \textunderscore Estrêl. Fun.\textunderscore , prólogo.(V.depenar)
\section{Despencar}
\begin{itemize}
\item {Grp. gram.:v. t.}
\end{itemize}
\begin{itemize}
\item {Utilização:Bras}
\end{itemize}
\begin{itemize}
\item {Grp. gram.:V. i.}
\end{itemize}
\begin{itemize}
\item {Proveniência:(De \textunderscore des...\textunderscore  + \textunderscore penca\textunderscore )}
\end{itemize}
Separar do cacho (pencas de banana).
Caír desastradamente de muito alto.
\section{Despendedor}
\begin{itemize}
\item {Grp. gram.:m.}
\end{itemize}
Aquelle que despende.
\section{Despender}
\begin{itemize}
\item {Grp. gram.:v. t.}
\end{itemize}
\begin{itemize}
\item {Utilização:Fig.}
\end{itemize}
\begin{itemize}
\item {Proveniência:(Do lat. \textunderscore dispendere\textunderscore )}
\end{itemize}
Fazer dispêndio de; gastar: \textunderscore despendeu muito dinheiro\textunderscore .
Empregar: \textunderscore despender cuidados\textunderscore .
Espargir.
\section{Despendurar}
\begin{itemize}
\item {Grp. gram.:v. t.}
\end{itemize}
\begin{itemize}
\item {Proveniência:(De \textunderscore des...\textunderscore  + \textunderscore pendurar\textunderscore )}
\end{itemize}
Tirar do seu lugar (aquillo que estava pendurado).
\section{Despenhadamente}
\begin{itemize}
\item {Grp. gram.:adv.}
\end{itemize}
\begin{itemize}
\item {Proveniência:(De \textunderscore despenhar\textunderscore )}
\end{itemize}
Com despenho.
\section{Despenhadeiro}
\begin{itemize}
\item {Grp. gram.:m.}
\end{itemize}
\begin{itemize}
\item {Proveniência:(De \textunderscore despenhar\textunderscore )}
\end{itemize}
Alcantil.
Precipício.
\section{Despenhamento}
\begin{itemize}
\item {Grp. gram.:m.}
\end{itemize}
Acto de despenhar.
\section{Despenhar}
\begin{itemize}
\item {Grp. gram.:v. t.}
\end{itemize}
\begin{itemize}
\item {Utilização:Fig.}
\end{itemize}
\begin{itemize}
\item {Proveniência:(De \textunderscore des...\textunderscore  + \textunderscore penha\textunderscore )}
\end{itemize}
Lançar ou arremessar de penha ou lugar alcantilado.
Precipitar.
Fazer caír de alta posição na desgraça.
Arruinar.
Fazer correr com impeto.
\section{Despenho}
\begin{itemize}
\item {Grp. gram.:m.}
\end{itemize}
O mesmo que \textunderscore despenhamento\textunderscore .
\section{Despenhoso}
\begin{itemize}
\item {Grp. gram.:adj.}
\end{itemize}
\begin{itemize}
\item {Proveniência:(De \textunderscore despenho\textunderscore )}
\end{itemize}
Alcantilado.
Fragoso.
\section{Despennar}
\begin{itemize}
\item {Grp. gram.:v. t.}
\end{itemize}
(V.depennar)
\section{Despensa}
\begin{itemize}
\item {Grp. gram.:f.}
\end{itemize}
\begin{itemize}
\item {Utilização:Ant.}
\end{itemize}
\begin{itemize}
\item {Proveniência:(Do lat. \textunderscore dispensa\textunderscore )}
\end{itemize}
Casa ou armário, em que se guardam provisões culinárias ou gêneros alimentícios, para uso doméstico.
O mesmo que \textunderscore despesa\textunderscore .
\section{Despenseiro}
\begin{itemize}
\item {Grp. gram.:m.}
\end{itemize}
\begin{itemize}
\item {Proveniência:(De \textunderscore despensa\textunderscore )}
\end{itemize}
Aquelle, que tem a seu cargo a despensa.
Aquelle que distribue os dons da munificência alheia.
\section{Despentear}
\begin{itemize}
\item {Grp. gram.:v. t.}
\end{itemize}
\begin{itemize}
\item {Proveniência:(De \textunderscore des...\textunderscore  + \textunderscore pentear\textunderscore )}
\end{itemize}
Desmanchar o penteado de.
\section{Desperação}
\begin{itemize}
\item {Grp. gram.:f.}
\end{itemize}
\begin{itemize}
\item {Utilização:Ant.}
\end{itemize}
O mesmo que \textunderscore desesperação\textunderscore .
\section{Desperança}
\begin{itemize}
\item {Grp. gram.:f.}
\end{itemize}
O mesmo que \textunderscore desesperança\textunderscore . Cf. Filinto, IX, 270.
\section{Desperceber}
\begin{itemize}
\item {Grp. gram.:v. t.}
\end{itemize}
Não perceber.
Não notar.
Não dar attenção a.
\section{Despercebido}
\begin{itemize}
\item {Grp. gram.:adj.}
\end{itemize}
Que se não viu ou que se não ouviu.
A que se não deu attenção: \textunderscore passou-lhe despercebido o aviso\textunderscore .
\section{Despercebimento}
\begin{itemize}
\item {Grp. gram.:m.}
\end{itemize}
Acto ou effeito de desperceber.
\section{Desperdiçadamente}
\begin{itemize}
\item {Grp. gram.:adv.}
\end{itemize}
\begin{itemize}
\item {Proveniência:(De \textunderscore desperdiçar\textunderscore )}
\end{itemize}
Com desperdício.
\section{Desperdiçado}
\begin{itemize}
\item {Grp. gram.:adj.}
\end{itemize}
\begin{itemize}
\item {Grp. gram.:M.}
\end{itemize}
\begin{itemize}
\item {Proveniência:(De \textunderscore desperdiçar\textunderscore )}
\end{itemize}
Gasto sem proveito.
Desbaratado: \textunderscore dinheiro desperdiçado\textunderscore .
Que gasta muito, sem necessidade nem proveito: \textunderscore aquelle homem é muito desperdiçado\textunderscore .
O mesmo que \textunderscore desperdiçador\textunderscore .
\section{Desperdiçador}
\begin{itemize}
\item {Grp. gram.:m.}
\end{itemize}
Aquelle que desperdiça.
\section{Desperdiçar}
\begin{itemize}
\item {Grp. gram.:v. t.}
\end{itemize}
\begin{itemize}
\item {Proveniência:(De \textunderscore des...\textunderscore  + \textunderscore perder\textunderscore )}
\end{itemize}
Gastar sem proveito.
Malbaratar.
Desaproveitar: \textunderscore desperdiçar auxílios\textunderscore .
\section{Desperdício}
\begin{itemize}
\item {Grp. gram.:m.}
\end{itemize}
Acto ou effeito de desperdiçar.
\section{Desperecer}
\begin{itemize}
\item {Grp. gram.:v. i.}
\end{itemize}
\begin{itemize}
\item {Utilização:Ant.}
\end{itemize}
\begin{itemize}
\item {Proveniência:(Do lat. \textunderscore deperire\textunderscore )}
\end{itemize}
O mesmo que \textunderscore deperecer\textunderscore , morrer. Cf. \textunderscore Port. Mon. Hist., Script.\textunderscore , 279.
\section{Desperecimento}
\begin{itemize}
\item {Grp. gram.:m.}
\end{itemize}
\begin{itemize}
\item {Utilização:Ant.}
\end{itemize}
\begin{itemize}
\item {Proveniência:(De \textunderscore desperecer\textunderscore )}
\end{itemize}
O mesmo que \textunderscore perecimento\textunderscore .
\section{Desperfilamento}
\begin{itemize}
\item {Grp. gram.:m.}
\end{itemize}
Acto ou effeito de desperfilar.
\section{Desperfilar}
\begin{itemize}
\item {Grp. gram.:v. t.}
\end{itemize}
\begin{itemize}
\item {Proveniência:(De \textunderscore des...\textunderscore  + \textunderscore perfilar\textunderscore )}
\end{itemize}
Tirar do alinhamento.
Desarranjar (aquelle ou aquillo que estava perfilado).
\section{Despersonalização}
\begin{itemize}
\item {Grp. gram.:f.}
\end{itemize}
Acto de despersonalizar-se.
\section{Despersonalizar-se}
\begin{itemize}
\item {Grp. gram.:v. p.}
\end{itemize}
\begin{itemize}
\item {Utilização:bras}
\end{itemize}
\begin{itemize}
\item {Utilização:Neol.}
\end{itemize}
\begin{itemize}
\item {Proveniência:(De \textunderscore des...\textunderscore  + \textunderscore personalizar\textunderscore )}
\end{itemize}
Enjeitar a própria personalidade; proceder contrariamente ao seu carácter.
\section{Despersuadir}
\begin{itemize}
\item {Grp. gram.:v. t.}
\end{itemize}
\begin{itemize}
\item {Proveniência:(De \textunderscore des...\textunderscore  + \textunderscore persuadir\textunderscore )}
\end{itemize}
Fazer mudar de opinião.
\section{Despersuasão}
\begin{itemize}
\item {Grp. gram.:f.}
\end{itemize}
Acto ou effeito de despersuadir.
\section{Despertador}
\begin{itemize}
\item {Grp. gram.:m.}
\end{itemize}
\begin{itemize}
\item {Grp. gram.:Adj.}
\end{itemize}
Relógio, que tem annexo um apparelho, para soar em hora determinada, acordando quem dorme perto.
Que desperta.
\section{Despertar}
\begin{itemize}
\item {Grp. gram.:v. i.}
\end{itemize}
\begin{itemize}
\item {Proveniência:(Do lat. \textunderscore de\textunderscore  + \textunderscore experrectus\textunderscore )}
\end{itemize}
Acordar, tirar do somno a.
Estimular, provocar: \textunderscore despertar appetites\textunderscore .
Dar origem a (no espirito de alguém).
Manifestar.
\section{Despertez}
\begin{itemize}
\item {Grp. gram.:f.}
\end{itemize}
Estado de quem é desperto. Cf. Filinto, \textunderscore D. Man.\textunderscore , II, 265.
\section{Desperto}
\begin{itemize}
\item {Grp. gram.:adj.}
\end{itemize}
\begin{itemize}
\item {Proveniência:(De \textunderscore despertar\textunderscore )}
\end{itemize}
Que despertou; que está acordado.
\section{Despesa}
\begin{itemize}
\item {fónica:pê}
\end{itemize}
\begin{itemize}
\item {Grp. gram.:f.}
\end{itemize}
\begin{itemize}
\item {Proveniência:(Do lat. \textunderscore dispensa\textunderscore )}
\end{itemize}
Acto ou effeito de despender.
\section{Despeso}
\begin{itemize}
\item {Grp. gram.:adj.}
\end{itemize}
\begin{itemize}
\item {Utilização:P. us.}
\end{itemize}
\begin{itemize}
\item {Proveniência:(Do lat. \textunderscore dispensus\textunderscore )}
\end{itemize}
Que se despendeu; gasto: \textunderscore dinheiro despeso\textunderscore .
\section{Despetaleado}
\begin{itemize}
\item {Grp. gram.:adj.}
\end{itemize}
O mesmo que \textunderscore apétalo\textunderscore .
\section{Despicador}
\begin{itemize}
\item {Grp. gram.:m.}
\end{itemize}
Aquelle que despica.
\section{Despicar}
\begin{itemize}
\item {Grp. gram.:v. t.}
\end{itemize}
\begin{itemize}
\item {Grp. gram.:V. p.}
\end{itemize}
\begin{itemize}
\item {Proveniência:(De \textunderscore picar\textunderscore )}
\end{itemize}
Vingar, desforrar.
Portar-se bizarramente, na devida altura.
\section{Despicativo}
\begin{itemize}
\item {Grp. gram.:adj.}
\end{itemize}
\begin{itemize}
\item {Proveniência:(Do lat. \textunderscore despicatus\textunderscore )}
\end{itemize}
O mesmo que \textunderscore desprezativo\textunderscore . Cf. Filinto, II, 226.
\section{Despiciendo}
\begin{itemize}
\item {Grp. gram.:adj.}
\end{itemize}
\begin{itemize}
\item {Utilização:Fam.}
\end{itemize}
\begin{itemize}
\item {Proveniência:(Lat. \textunderscore despiciendus\textunderscore )}
\end{itemize}
Digno de desdém, de desprêzo. Cf. Filinto, I, 102.
\section{Despiciente}
\begin{itemize}
\item {Grp. gram.:adj.}
\end{itemize}
\begin{itemize}
\item {Proveniência:(Lat. \textunderscore despiciens\textunderscore )}
\end{itemize}
Que desdenha, que despreza, que olha de alto.
\section{Despiedadamente}
\begin{itemize}
\item {Grp. gram.:adv.}
\end{itemize}
De modo despiedado.
Sem piedade.
\section{Despiedade}
\begin{itemize}
\item {Grp. gram.:f.}
\end{itemize}
Falta de piedade.
Deshumanidade.
\section{Despiedado}
\begin{itemize}
\item {Grp. gram.:adj.}
\end{itemize}
\begin{itemize}
\item {Proveniência:(De \textunderscore despiedar\textunderscore )}
\end{itemize}
Que não tem piedade; desapiedado.
\section{Despiedar}
\begin{itemize}
\item {Grp. gram.:v. t.}
\end{itemize}
O mesmo que \textunderscore desapiedar\textunderscore .
\section{Despiedosamente}
\begin{itemize}
\item {Grp. gram.:adv.}
\end{itemize}
De modo despiedoso.
\section{Despiedoso}
\begin{itemize}
\item {Grp. gram.:adj.}
\end{itemize}
\begin{itemize}
\item {Proveniência:(De \textunderscore des...\textunderscore  + \textunderscore piedoso\textunderscore )}
\end{itemize}
Que não tem piedade.
Em que não há piedade: \textunderscore tratamento despiedoso\textunderscore .
\section{Despinça}
\begin{itemize}
\item {Grp. gram.:f.}
\end{itemize}
\begin{itemize}
\item {Proveniência:(De \textunderscore despinçar\textunderscore )}
\end{itemize}
Salina, como algumas do norte, em que se não fórma o cozimento ou cásco. Cf. \textunderscore Museu Techn.\textunderscore , 105.
\section{Despinçadeira}
\begin{itemize}
\item {Grp. gram.:f.}
\end{itemize}
O mesmo que \textunderscore espinçadeira\textunderscore .
\section{Despinçar}
\begin{itemize}
\item {Grp. gram.:v. t.}
\end{itemize}
Tirar com pinça.
\section{Despintar}
\begin{itemize}
\item {Grp. gram.:v. t.}
\end{itemize}
Apagar a pintura de.
Destingir:«\textunderscore os que pintam ou despintam os objectos\textunderscore ». Vieira.
\section{Despiolhar}
\begin{itemize}
\item {Grp. gram.:v. t.}
\end{itemize}
(V.espiolhar)
\section{Despique}
\begin{itemize}
\item {Grp. gram.:m.}
\end{itemize}
Acto de despicar.
\section{Despir}
\begin{itemize}
\item {Grp. gram.:v. t.}
\end{itemize}
Tirar o vestuário a: \textunderscore despir uma criança\textunderscore .
Tirar do corpo (o vestuário): \textunderscore despir o casaco\textunderscore .
Tirar a cobertura ou o invólucro de.
Deixar, abandonar: \textunderscore despir vaidades\textunderscore .
Despojar.
(Contr. de \textunderscore despedir\textunderscore )
\section{Despistar}
\begin{itemize}
\item {Grp. gram.:v. t.}
\end{itemize}
\begin{itemize}
\item {Proveniência:(De \textunderscore des...\textunderscore  + \textunderscore pista\textunderscore )}
\end{itemize}
Fazer perder a pista: \textunderscore Cervera partiu para Martínica, despistando os navios americanos\textunderscore .
\section{Despitorrado}
\begin{itemize}
\item {Grp. gram.:adj.}
\end{itemize}
\begin{itemize}
\item {Proveniência:(De \textunderscore des...\textunderscore  + \textunderscore pitorra\textunderscore )}
\end{itemize}
Diz-se do toiro, que tem partida uma pequena parte da ponta das hastes.
\section{Desplantar}
\begin{itemize}
\item {Grp. gram.:v. t.}
\end{itemize}
\begin{itemize}
\item {Proveniência:(De \textunderscore des...\textunderscore  + \textunderscore plantar\textunderscore )}
\end{itemize}
Arrancar as plantas de: \textunderscore desplantar um canteiro\textunderscore .
\section{Desplante}
\begin{itemize}
\item {Grp. gram.:m.}
\end{itemize}
\begin{itemize}
\item {Utilização:Fig.}
\end{itemize}
\begin{itemize}
\item {Proveniência:(De \textunderscore desplantar\textunderscore )}
\end{itemize}
Uma das posições no jôgo de esgrima.
Ousadia, audácia.
\section{Desplumar}
\begin{itemize}
\item {Grp. gram.:v. t.}
\end{itemize}
Tirar as plumas ou pennas a.
\section{Despoer}
\begin{itemize}
\item {Grp. gram.:v. t.}
\end{itemize}
\begin{itemize}
\item {Utilização:Ant.}
\end{itemize}
O mesmo que \textunderscore dispor\textunderscore .
\section{Despoético}
\begin{itemize}
\item {Grp. gram.:adj.}
\end{itemize}
Não poético.
Opposto á poesia. Cf. Latino, \textunderscore Humboldt\textunderscore , 122.
\section{Despoetizador}
\begin{itemize}
\item {fónica:po-e}
\end{itemize}
\begin{itemize}
\item {Grp. gram.:m.}
\end{itemize}
Aquelle que despoetiza.
\section{Despoetizar}
\begin{itemize}
\item {fónica:po-e}
\end{itemize}
\begin{itemize}
\item {Grp. gram.:v. t.}
\end{itemize}
\begin{itemize}
\item {Proveniência:(De \textunderscore des...\textunderscore  + \textunderscore poetizar\textunderscore )}
\end{itemize}
Tirar a feição poética a.
\section{Despoimento}
\begin{itemize}
\item {fónica:po-i}
\end{itemize}
\begin{itemize}
\item {Grp. gram.:m.}
\end{itemize}
\begin{itemize}
\item {Utilização:Ant.}
\end{itemize}
\begin{itemize}
\item {Proveniência:(De \textunderscore despoer\textunderscore )}
\end{itemize}
Disposição, determinação.
\section{Despois}
\begin{itemize}
\item {Grp. gram.:adv.}
\end{itemize}
O mesmo que \textunderscore depois\textunderscore .
\section{Despojador}
\begin{itemize}
\item {Grp. gram.:adj.}
\end{itemize}
\begin{itemize}
\item {Grp. gram.:M.}
\end{itemize}
Que despoja.
Aquelle que despoja.
\section{Despojamento}
\begin{itemize}
\item {Grp. gram.:m.}
\end{itemize}
Acto de despojar.
\section{Despojar}
\begin{itemize}
\item {Grp. gram.:v. t.}
\end{itemize}
\begin{itemize}
\item {Proveniência:(Lat. \textunderscore despoliare\textunderscore )}
\end{itemize}
Privar; desapossar.
Despir.
\section{Despôjo}
\begin{itemize}
\item {Grp. gram.:m.}
\end{itemize}
Acto ou effeito de despojar.
Espólio.
Presa.
Aquillo que caiu ou foi arrancado, tendo servido de revestimento ou adôrno.
\section{Despoliar}
\begin{itemize}
\item {Grp. gram.:v. t.}
\end{itemize}
\begin{itemize}
\item {Utilização:Ant.}
\end{itemize}
O mesmo que \textunderscore espoliar\textunderscore .
\section{Despolidez}
\begin{itemize}
\item {Grp. gram.:f.}
\end{itemize}
Falta de polidez; indelicadeza. Cf. Filinto. XXI. 74, 76 e 85.
\section{Despolimento}
\begin{itemize}
\item {Grp. gram.:m.}
\end{itemize}
Acto ou effeito de despolir.
\section{Despolir}
\begin{itemize}
\item {Grp. gram.:v. t.}
\end{itemize}
\begin{itemize}
\item {Proveniência:(De \textunderscore des...\textunderscore  + \textunderscore polir\textunderscore )}
\end{itemize}
Tirar o polimento a.
\section{Despolpador}
\begin{itemize}
\item {Grp. gram.:m.}
\end{itemize}
Aquelle que despolpa.
Instrumento para despolpar.
\section{Despolpar}
\begin{itemize}
\item {Grp. gram.:v. t.}
\end{itemize}
\begin{itemize}
\item {Utilização:Bras}
\end{itemize}
Tirar a pôlpa a.
Descascar (o grão de café).
\section{Desponderar}
\begin{itemize}
\item {Grp. gram.:v. t.}
\end{itemize}
Não ponderar.
Fazer inadvertidamente.
\section{Despongar}
\begin{itemize}
\item {Grp. gram.:v. i.}
\end{itemize}
\begin{itemize}
\item {Utilização:Bras. do N}
\end{itemize}
\begin{itemize}
\item {Proveniência:(De \textunderscore des...\textunderscore  + \textunderscore pongar\textunderscore )}
\end{itemize}
Saltar do carro ou bonde, sem que êste pare.
\section{Desponsório}
\begin{itemize}
\item {Grp. gram.:m.}
\end{itemize}
(V.desposório)
\section{Despontado}
\begin{itemize}
\item {Grp. gram.:adj.}
\end{itemize}
\begin{itemize}
\item {Proveniência:(De \textunderscore des...\textunderscore  + \textunderscore pontar\textunderscore )}
\end{itemize}
A que se tirou ou de gastou a ponta: \textunderscore navalha despontada\textunderscore .
Embotado.
\section{Despontante}
\begin{itemize}
\item {Grp. gram.:adj.}
\end{itemize}
Que desponta.
\section{Despontar}
\begin{itemize}
\item {Grp. gram.:v. t.}
\end{itemize}
\begin{itemize}
\item {Grp. gram.:V. i.}
\end{itemize}
\begin{itemize}
\item {Proveniência:(De \textunderscore des...\textunderscore  + \textunderscore ponta\textunderscore )}
\end{itemize}
Gastar a ponta de.
Embotar.
Começar a surgir; nascer: \textunderscore despontou a aurora\textunderscore .
Vir á lembrança.
\section{Despopularizar}
\begin{itemize}
\item {Grp. gram.:v. t.}
\end{itemize}
\begin{itemize}
\item {Proveniência:(De \textunderscore des...\textunderscore  + \textunderscore popularizar\textunderscore )}
\end{itemize}
Tornar impopular.
\section{Despor}
\begin{itemize}
\item {Utilização:Prov.}
\end{itemize}
\textunderscore v. t.\textunderscore  (e der.)
O mesmo que \textunderscore depor\textunderscore , \textunderscore demittir\textunderscore . Cf. Castanheda, \textunderscore passim\textunderscore ; Pant. de Aveiro, \textunderscore Itiner.\textunderscore , 235, (2.^a ed.).
Plantar (arbustos, hortaliças, etc.).
\section{Desporobo}
\begin{itemize}
\item {Grp. gram.:m.}
\end{itemize}
Autoridade local, na Índia portuguesa.
\section{Desporte}
\begin{itemize}
\item {Grp. gram.:m.}
\end{itemize}
O mesmo que \textunderscore desporto\textunderscore .
\section{Desportilhado}
\begin{itemize}
\item {Grp. gram.:adj.}
\end{itemize}
\begin{itemize}
\item {Proveniência:(De \textunderscore desportilhar\textunderscore )}
\end{itemize}
Diz-se do casco dos solípedes, que tem falhas ou desigualdades no bôrdo inferior da tapa, por separação accidental de porções de substância córnea.
\section{Desportilhar}
\begin{itemize}
\item {Grp. gram.:v. t.}
\end{itemize}
\begin{itemize}
\item {Proveniência:(De \textunderscore des...\textunderscore  + \textunderscore portilho\textunderscore )}
\end{itemize}
Derribar as portas de.
Deteriorar o bôrdo inferior das tapas do casco de (um cavallo).
\section{Desportivo}
\begin{itemize}
\item {Grp. gram.:adj.}
\end{itemize}
Relativo ao desporto: \textunderscore vida desportiva\textunderscore .
\section{Desporto}
\begin{itemize}
\item {Grp. gram.:m.}
\end{itemize}
\begin{itemize}
\item {Utilização:Des.}
\end{itemize}
\begin{itemize}
\item {Utilização:Ant.}
\end{itemize}
\begin{itemize}
\item {Proveniência:(It. \textunderscore diporto\textunderscore )}
\end{itemize}
Recreio; diversão.
Lugar insulado; solidão; retiro.
\section{Despós}
\begin{itemize}
\item {Grp. gram.:prep.}
\end{itemize}
\begin{itemize}
\item {Utilização:ant.}
\end{itemize}
Após, atrás de. Cf. \textunderscore Filodemo\textunderscore , II, VIII.
\section{Desposar}
\begin{itemize}
\item {Grp. gram.:v. t.}
\end{itemize}
\begin{itemize}
\item {Proveniência:(De \textunderscore esposar\textunderscore )}
\end{itemize}
Celebrar esponsaes com.
Firmar promessa de casamento com; ajustar casamento com.
Casar com.
\section{Desposoiro}
\begin{itemize}
\item {Grp. gram.:m.}
\end{itemize}
\begin{itemize}
\item {Utilização:Ant.}
\end{itemize}
O mesmo que \textunderscore desposório\textunderscore . Cf. \textunderscore Eufrosina\textunderscore , 149.
\section{Desposório}
\begin{itemize}
\item {Grp. gram.:m.}
\end{itemize}
\begin{itemize}
\item {Proveniência:(De \textunderscore desposar\textunderscore )}
\end{itemize}
Esponsaes.
Casamento.
\section{Despossessão}
\begin{itemize}
\item {Grp. gram.:f.}
\end{itemize}
\begin{itemize}
\item {Proveniência:(De \textunderscore des...\textunderscore  + \textunderscore possessão\textunderscore )}
\end{itemize}
Acto de sêr desapossado.
\section{Despossesso}
\begin{itemize}
\item {Grp. gram.:adj.}
\end{itemize}
Que não está possesso. Cf. Filinto, XIII, 181; Garrett, \textunderscore D. Branca\textunderscore , 191.
\section{Despossuído}
\begin{itemize}
\item {Grp. gram.:adj.}
\end{itemize}
\begin{itemize}
\item {Proveniência:(De \textunderscore despossuir\textunderscore )}
\end{itemize}
Não possuído.
Despossesso. Cf. Alves Mendes, \textunderscore Discursos\textunderscore , 249.
\section{Despossuir}
\begin{itemize}
\item {Grp. gram.:v. t.}
\end{itemize}
\begin{itemize}
\item {Utilização:Des.}
\end{itemize}
O mesmo que \textunderscore desapossar\textunderscore :«\textunderscore El-rei de Cochim despossuído do reino pelos poderes do Çamarim...\textunderscore »Filinto, D. \textunderscore Man.\textunderscore , I, 242.
\section{Despostiçar}
\begin{itemize}
\item {Grp. gram.:v. t.}
\end{itemize}
\begin{itemize}
\item {Utilização:Prov.}
\end{itemize}
\begin{itemize}
\item {Utilização:trasm.}
\end{itemize}
\begin{itemize}
\item {Proveniência:(De \textunderscore des...\textunderscore  + \textunderscore pôsto\textunderscore )}
\end{itemize}
Despedir, pôr fóra de casa, á má cara.
\section{Despostigar}
\begin{itemize}
\item {Grp. gram.:v. t.}
\end{itemize}
Tirar o postigo a.
\section{Déspota}
\begin{itemize}
\item {Grp. gram.:m.}
\end{itemize}
\begin{itemize}
\item {Proveniência:(Gr. \textunderscore despotes\textunderscore )}
\end{itemize}
Aquelle que exerce poder absoluto e arbitrário.
Aquelle que domina tyrannicamente.
Tyranno.
\section{Despoticamente}
\begin{itemize}
\item {Grp. gram.:adv.}
\end{itemize}
De modo despótico.
\section{Despótico}
\begin{itemize}
\item {Grp. gram.:adj.}
\end{itemize}
Relativo a déspota.
Em que há despotismo.
Tyrânnico.
\section{Despotismo}
\begin{itemize}
\item {Grp. gram.:m.}
\end{itemize}
\begin{itemize}
\item {Utilização:Bras. de Minas}
\end{itemize}
Autoridade de um déspota.
Poder absoluto e arbitrário.
Acto próprio de déspota.
Grande porção: \textunderscore vi lá um despotismo de gente\textunderscore .
\section{Despovoação}
\begin{itemize}
\item {Grp. gram.:f.}
\end{itemize}
Acto ou effeito de despovoar.
\section{Despovoado}
\begin{itemize}
\item {Grp. gram.:m.}
\end{itemize}
\begin{itemize}
\item {Proveniência:(De \textunderscore despovoar\textunderscore )}
\end{itemize}
Lugar, que não é habitado.
\section{Despovoador}
\begin{itemize}
\item {Grp. gram.:adj.}
\end{itemize}
\begin{itemize}
\item {Grp. gram.:M.}
\end{itemize}
Que despovôa.
Aquelle que despovôa.
\section{Despovoamento}
\begin{itemize}
\item {Grp. gram.:m.}
\end{itemize}
O mesmo que \textunderscore despovoação\textunderscore .
\section{Despovoar}
\begin{itemize}
\item {Grp. gram.:v. t.}
\end{itemize}
\begin{itemize}
\item {Utilização:Fig.}
\end{itemize}
\begin{itemize}
\item {Proveniência:(De \textunderscore des...\textunderscore  + \textunderscore povoar\textunderscore )}
\end{itemize}
Privar de povoadores; tornar deshabitado.
Deminuir ou reduzir a população de.
Tirar os objectos que guarnecem ou adornam.
\section{Despovorar}
\begin{itemize}
\item {Grp. gram.:v. t.}
\end{itemize}
\begin{itemize}
\item {Utilização:Ant.}
\end{itemize}
O mesmo que \textunderscore despovoar\textunderscore .
Cf. \textunderscore Inéd. da Hist. Port.\textunderscore , I, 131.
\section{Despratear}
\begin{itemize}
\item {Grp. gram.:v. t.}
\end{itemize}
\begin{itemize}
\item {Proveniência:(De \textunderscore des...\textunderscore  + \textunderscore pratear\textunderscore )}
\end{itemize}
Tirar camada de prata a.
Tirar a côr de prata a.
\section{Desprazer}
\begin{itemize}
\item {Grp. gram.:m.}
\end{itemize}
\begin{itemize}
\item {Grp. gram.:V. i.}
\end{itemize}
Falta de prazer.
Desgôsto.
Desagradar: \textunderscore são coisas que desprazem\textunderscore .
\section{Desprazimento}
\begin{itemize}
\item {Grp. gram.:m.}
\end{itemize}
O mesmo que \textunderscore desprazer\textunderscore .
Desagrado; desgôsto.
\section{Desprazível}
\begin{itemize}
\item {Grp. gram.:adj.}
\end{itemize}
\begin{itemize}
\item {Proveniência:(De \textunderscore desprazer\textunderscore )}
\end{itemize}
Que despraz.
\section{Despreçamento}
\begin{itemize}
\item {Grp. gram.:m.}
\end{itemize}
\begin{itemize}
\item {Utilização:Ant.}
\end{itemize}
\begin{itemize}
\item {Proveniência:(De \textunderscore despreçar\textunderscore )}
\end{itemize}
O mesmo que \textunderscore desprêzo\textunderscore .
\section{Despreçar}
\begin{itemize}
\item {Grp. gram.:v. t.}
\end{itemize}
\begin{itemize}
\item {Utilização:Ant.}
\end{itemize}
O mesmo que \textunderscore desprezar\textunderscore .
\section{Desprecatadamente}
\begin{itemize}
\item {Grp. gram.:adv.}
\end{itemize}
\begin{itemize}
\item {Proveniência:(De \textunderscore desprecatar-se\textunderscore )}
\end{itemize}
Sem cautela, sem cuidado.
\section{Desprecatar-se}
\begin{itemize}
\item {Grp. gram.:v. p.}
\end{itemize}
\begin{itemize}
\item {Proveniência:(De \textunderscore des...\textunderscore  + \textunderscore precatar\textunderscore )}
\end{itemize}
Não tomar cautela.
Desacautelar-se.
Descuidar-se.
\section{Desprecaver}
\begin{itemize}
\item {Grp. gram.:v. t.}
\end{itemize}
\begin{itemize}
\item {Proveniência:(De \textunderscore des...\textunderscore  + \textunderscore precaver\textunderscore )}
\end{itemize}
Desacautelar.
\section{Despreciar}
\begin{itemize}
\item {Grp. gram.:v. t.}
\end{itemize}
O mesmo que \textunderscore desprezar\textunderscore .
\section{Despregadura}
\begin{itemize}
\item {Grp. gram.:f.}
\end{itemize}
Acto ou effeito de despregar^1.
\section{Despregar}
\begin{itemize}
\item {Grp. gram.:v. t.}
\end{itemize}
\begin{itemize}
\item {Proveniência:(De \textunderscore des...\textunderscore  + \textunderscore pregar\textunderscore )}
\end{itemize}
Descravar.
Arrancar ou separar (aquillo que estava pregado).
Desviar.
Emittir.
\section{Despregar}
\begin{itemize}
\item {Grp. gram.:v. t.}
\end{itemize}
\begin{itemize}
\item {Proveniência:(De \textunderscore des...\textunderscore  + \textunderscore prega\textunderscore )}
\end{itemize}
Desfazer as pregas de.
Desenrugar.
Desenrolar.
Estender.
Desfraldar: \textunderscore despregar a vela do barco\textunderscore .
\section{Despreguiçar}
\begin{itemize}
\item {Grp. gram.:v. t.}
\end{itemize}
(V.espreguiçar)
\section{Despremiar}
\begin{itemize}
\item {Grp. gram.:v. t.}
\end{itemize}
Não premiar.
\section{Desprendado}
\begin{itemize}
\item {Grp. gram.:adj.}
\end{itemize}
\begin{itemize}
\item {Proveniência:(De \textunderscore des...\textunderscore  + \textunderscore prendado\textunderscore )}
\end{itemize}
Que não tem prendas, habilidade ou talento.
\section{Desprender}
\begin{itemize}
\item {Grp. gram.:v. t.}
\end{itemize}
\begin{itemize}
\item {Proveniência:(De \textunderscore des...\textunderscore  + \textunderscore prender\textunderscore )}
\end{itemize}
Desligar.
Soltar; desatar.
Dar liberdade a: \textunderscore desprender um cão\textunderscore .
\section{Desprendido}
\begin{itemize}
\item {Grp. gram.:adj.}
\end{itemize}
\begin{itemize}
\item {Proveniência:(De \textunderscore desprender\textunderscore )}
\end{itemize}
Que tem desprendimento, que tem abnegação, que tem independência.
\section{Desprendimento}
\begin{itemize}
\item {Grp. gram.:m.}
\end{itemize}
Acto ou effeito de desprender.
Abnegação.
Altruismo.
Independência.
\section{Desprenhar}
\begin{itemize}
\item {Grp. gram.:v. t.}
\end{itemize}
\begin{itemize}
\item {Utilização:Des.}
\end{itemize}
\begin{itemize}
\item {Proveniência:(De \textunderscore des...\textunderscore  + \textunderscore prenhe\textunderscore )}
\end{itemize}
Tirar a prenhez a.
Despejar. Cf. Filinto, \textunderscore D. Man.\textunderscore , I, 410.
\section{Despreoccupação}
\begin{itemize}
\item {Grp. gram.:f.}
\end{itemize}
Estado de quem se acha despreoccupado.
\section{Despreoccupadamente}
\begin{itemize}
\item {Grp. gram.:adv.}
\end{itemize}
\begin{itemize}
\item {Proveniência:(De \textunderscore despreoccupar\textunderscore )}
\end{itemize}
Sem preoccupação.
Com desafôgo.
\section{Despreoccupar}
\begin{itemize}
\item {Grp. gram.:v. t.}
\end{itemize}
\begin{itemize}
\item {Proveniência:(De \textunderscore des...\textunderscore  + \textunderscore preoccupar\textunderscore )}
\end{itemize}
Livrar de preoccupação.
\section{Despreocupação}
\begin{itemize}
\item {Grp. gram.:f.}
\end{itemize}
Estado de quem se acha despreocupado.
\section{Despreocupadamente}
\begin{itemize}
\item {Grp. gram.:adv.}
\end{itemize}
\begin{itemize}
\item {Proveniência:(De \textunderscore despreocupar\textunderscore )}
\end{itemize}
Sem preocupação.
Com desafôgo.
\section{Despreocupar}
\begin{itemize}
\item {Grp. gram.:v. t.}
\end{itemize}
\begin{itemize}
\item {Proveniência:(De \textunderscore des...\textunderscore  + \textunderscore preocupar\textunderscore )}
\end{itemize}
Livrar de preocupação.
\section{Despreparo}
\begin{itemize}
\item {Grp. gram.:m.}
\end{itemize}
\begin{itemize}
\item {Proveniência:(De \textunderscore des...\textunderscore  + \textunderscore preparo\textunderscore )}
\end{itemize}
Desarranjo, desorganização.
\section{Desprestigiar}
\begin{itemize}
\item {Grp. gram.:v. t.}
\end{itemize}
\begin{itemize}
\item {Proveniência:(De \textunderscore desprestígio\textunderscore )}
\end{itemize}
Tirar o prestígio a.
Depreciar; desacreditar.
\section{Desprestígio}
\begin{itemize}
\item {Grp. gram.:m.}
\end{itemize}
\begin{itemize}
\item {Proveniência:(De \textunderscore des...\textunderscore  + \textunderscore prestígio\textunderscore )}
\end{itemize}
Acto ou effeito de desprestigiar.
\section{Despretensão}
\begin{itemize}
\item {Grp. gram.:f.}
\end{itemize}
Falta de pretensão.
Falta de ambição.
Modéstia.
Desaffectação.
\section{Despretensiosamente}
\begin{itemize}
\item {Grp. gram.:adv.}
\end{itemize}
De modo despretensioso.
\section{Despretensioso}
\begin{itemize}
\item {Grp. gram.:adj.}
\end{itemize}
\begin{itemize}
\item {Proveniência:(De \textunderscore des...\textunderscore  + \textunderscore pretensioso\textunderscore )}
\end{itemize}
Que não tem pretensões; em que não há pretensões.
Modesto, franco.
\section{Desprevenção}
\begin{itemize}
\item {Grp. gram.:f.}
\end{itemize}
Falta de prevenção.
Imprevidência.
\section{Desprevenidamente}
\begin{itemize}
\item {Grp. gram.:adv.}
\end{itemize}
\begin{itemize}
\item {Proveniência:(De \textunderscore desprevenir\textunderscore )}
\end{itemize}
Sem prevenção.
\section{Desprevenir}
\begin{itemize}
\item {Grp. gram.:v. t.}
\end{itemize}
Não prevenir.
Desacautelar.
\section{Desprezadoiro}
\begin{itemize}
\item {Grp. gram.:adj.}
\end{itemize}
\begin{itemize}
\item {Utilização:Des.}
\end{itemize}
\begin{itemize}
\item {Proveniência:(De \textunderscore desprezar\textunderscore )}
\end{itemize}
Digno de desprêzo, desprezível.
\section{Desprezador}
\begin{itemize}
\item {Grp. gram.:adj.}
\end{itemize}
\begin{itemize}
\item {Grp. gram.:M.}
\end{itemize}
Que despreza.
Aquelle que despreza.
\section{Desprezadouro}
\begin{itemize}
\item {Grp. gram.:adj.}
\end{itemize}
\begin{itemize}
\item {Utilização:Des.}
\end{itemize}
\begin{itemize}
\item {Proveniência:(De \textunderscore desprezar\textunderscore )}
\end{itemize}
Digno de desprêzo, desprezível.
\section{Desprezamento}
\begin{itemize}
\item {Grp. gram.:m.}
\end{itemize}
\begin{itemize}
\item {Utilização:Ant.}
\end{itemize}
Acto de desprezar.
\section{Desprezar}
\begin{itemize}
\item {Grp. gram.:v. t.}
\end{itemize}
\begin{itemize}
\item {Grp. gram.:V. p.}
\end{itemize}
Não prezar.
Desconsiderar, menosprezar.
Desattender.
Rebaixar-se, envergonhar-se.
\section{Desprezativo}
\begin{itemize}
\item {Grp. gram.:adj.}
\end{itemize}
\begin{itemize}
\item {Proveniência:(De \textunderscore desprezar\textunderscore )}
\end{itemize}
Em que há desprêzo; que revela desprêzo.
O mesmo que \textunderscore depreciativo\textunderscore .
\section{Desprezável}
\begin{itemize}
\item {Grp. gram.:adj.}
\end{itemize}
\begin{itemize}
\item {Utilização:Des.}
\end{itemize}
O mesmo que \textunderscore desprezível\textunderscore .
\section{Desprezilho}
\begin{itemize}
\item {Grp. gram.:m.}
\end{itemize}
\begin{itemize}
\item {Utilização:Des.}
\end{itemize}
Indício de desprêzo; desdém. Cf. \textunderscore Anat. Joc.\textunderscore , 3.
\section{Desprezível}
\begin{itemize}
\item {Grp. gram.:adj.}
\end{itemize}
Digno de desprêzo; vergonhoso, abjecto.
\section{Desprezivelmente}
\begin{itemize}
\item {Grp. gram.:adv.}
\end{itemize}
De modo desprezível.
\section{Desprezivo}
\begin{itemize}
\item {Grp. gram.:adj.}
\end{itemize}
O mesmo que \textunderscore desprezativo\textunderscore . Cf. Filinto, III, 179.
\section{Desprêzo}
\begin{itemize}
\item {Grp. gram.:m.}
\end{itemize}
Acto de desprezar.
Desdém; falta de apreço.
\section{Desprimor}
\begin{itemize}
\item {Grp. gram.:m.}
\end{itemize}
Falta de primor.
Indelicadeza; descortesia.
\section{Desprimorar}
\begin{itemize}
\item {Grp. gram.:v. t.}
\end{itemize}
\begin{itemize}
\item {Proveniência:(De \textunderscore desprimor\textunderscore )}
\end{itemize}
Tirar o primor a; depreciar.
\section{Desprimorosamente}
\begin{itemize}
\item {Grp. gram.:adv.}
\end{itemize}
De modo desprimoroso.
\section{Desprimoroso}
\begin{itemize}
\item {Grp. gram.:adj.}
\end{itemize}
Que não tem primor; que não é perfeito.
Incivil; descortês.
\section{Desprivar}
\begin{itemize}
\item {Grp. gram.:v. t.}
\end{itemize}
\begin{itemize}
\item {Utilização:Des.}
\end{itemize}
\begin{itemize}
\item {Proveniência:(De \textunderscore des...\textunderscore  + \textunderscore privar\textunderscore )}
\end{itemize}
Tirar a privança ou o valimento a.
\section{Desprivilegiar}
\begin{itemize}
\item {Grp. gram.:v. t.}
\end{itemize}
\begin{itemize}
\item {Proveniência:(De \textunderscore des...\textunderscore  + \textunderscore privilegiar\textunderscore )}
\end{itemize}
Tirar privilégio a.
Tornar commum.
Generalizar.
\section{Despronúncia}
\begin{itemize}
\item {Grp. gram.:f.}
\end{itemize}
Acto de despronunciar.
\section{Despronunciar}
\begin{itemize}
\item {Grp. gram.:v.}
\end{itemize}
\begin{itemize}
\item {Utilização:t. Jur.}
\end{itemize}
\begin{itemize}
\item {Proveniência:(De \textunderscore des...\textunderscore  + \textunderscore pronunciar\textunderscore )}
\end{itemize}
Tornar nulla a pronúncia de (um réu).
\section{Desproporção}
\begin{itemize}
\item {Grp. gram.:f.}
\end{itemize}
Falta de proporção.
Monstruosidade.
\section{Desproporcionadamente}
\begin{itemize}
\item {Grp. gram.:adv.}
\end{itemize}
\begin{itemize}
\item {Proveniência:(De \textunderscore desproporcionado\textunderscore )}
\end{itemize}
Com desproporção.
Desigualmente.
\section{Desproporcionado}
\begin{itemize}
\item {Grp. gram.:adj.}
\end{itemize}
\begin{itemize}
\item {Proveniência:(De \textunderscore des...\textunderscore  + \textunderscore proporcionar\textunderscore )}
\end{itemize}
Que não tem proporções.
Desigual.
Enorme: \textunderscore um sujeito desproporcionado\textunderscore .
\section{Desproporcionar}
\begin{itemize}
\item {Grp. gram.:v. t.}
\end{itemize}
\begin{itemize}
\item {Proveniência:(De \textunderscore des...\textunderscore  + \textunderscore proporcionar\textunderscore )}
\end{itemize}
Tirar a proporção a.
Tornar desconforme.
Afear.
\section{Despropositadamente}
\begin{itemize}
\item {Grp. gram.:adv.}
\end{itemize}
De modo despropositado.
Á tôa.
Sem razão.
\section{Despropositado}
\begin{itemize}
\item {Grp. gram.:adj.}
\end{itemize}
\begin{itemize}
\item {Proveniência:(De \textunderscore despropósito\textunderscore )}
\end{itemize}
Que não vem a propósito.
Inopportuno.
Que fala sem propósito, sem tino.
\section{Despropositar}
\begin{itemize}
\item {Grp. gram.:v. i.}
\end{itemize}
\begin{itemize}
\item {Proveniência:(De \textunderscore despropósito\textunderscore )}
\end{itemize}
Proceder sem propósito.
Disparatar.
Falar com arrebatamento.
\section{Despropósito}
\begin{itemize}
\item {Grp. gram.:m.}
\end{itemize}
Falta de propósito.
Descommedimento.
Imprudência.
Aquillo que se faz ou se diz, fóra de propósito.
\section{Desprotecção}
\begin{itemize}
\item {Grp. gram.:f.}
\end{itemize}
Falta de protecção.
Desamparo, abandono.
\section{Desproteger}
\begin{itemize}
\item {Grp. gram.:v. t.}
\end{itemize}
Não proteger.
Retirar a protecção a.
Desauxiliar; desamparar.
\section{Desproveito}
\begin{itemize}
\item {Grp. gram.:m.}
\end{itemize}
O mesmo que \textunderscore desaproveitamento\textunderscore .
\section{Desprover}
\begin{itemize}
\item {Grp. gram.:v. t.}
\end{itemize}
\begin{itemize}
\item {Proveniência:(De \textunderscore des...\textunderscore  + \textunderscore prover\textunderscore )}
\end{itemize}
Tirar as provisões a.
Recusar as provisões necessárias a.
\section{Desprovidamente}
\begin{itemize}
\item {Grp. gram.:adv.}
\end{itemize}
\begin{itemize}
\item {Proveniência:(De \textunderscore desprover\textunderscore )}
\end{itemize}
Sem provisões.
\section{Desprovimento}
\begin{itemize}
\item {Grp. gram.:m.}
\end{itemize}
Acto ou effeito de desprover.
\section{Despumação}
\begin{itemize}
\item {Grp. gram.:f.}
\end{itemize}
Acto de despumar.
\section{Despumar}
\textunderscore v. t.\textunderscore  (e der.)(V.espumar)
\section{Despundonor}
\begin{itemize}
\item {Grp. gram.:m.}
\end{itemize}
Falta de pundonor. Cf. Camillo, \textunderscore Estrêl. Fun.\textunderscore , 157.
\section{Despundonoroso}
\begin{itemize}
\item {Grp. gram.:adj.}
\end{itemize}
Que tem despundonor.
Que revela despundonor.
\section{Desquadrilhar}
\begin{itemize}
\item {Grp. gram.:v. t.}
\end{itemize}
\begin{itemize}
\item {Proveniência:(De \textunderscore des...\textunderscore  + \textunderscore quadril\textunderscore )}
\end{itemize}
Torcer os quadris a.
Derrengar:«\textunderscore fariam desquadrilhar de riso os tristes...\textunderscore ». Camillo, \textunderscore Maria da Fonte\textunderscore , 1.^a ed., 347.
\section{Desqualificação}
\begin{itemize}
\item {Grp. gram.:f.}
\end{itemize}
Acto ou effeito de desqualificar.
\section{Desqualificadamente}
\begin{itemize}
\item {Grp. gram.:adv.}
\end{itemize}
\begin{itemize}
\item {Proveniência:(De \textunderscore desqualificado\textunderscore )}
\end{itemize}
Sem qualificação.
\section{Desqualificado}
\begin{itemize}
\item {Grp. gram.:adj.}
\end{itemize}
\begin{itemize}
\item {Proveniência:(De \textunderscore desqualificar\textunderscore )}
\end{itemize}
Que é inhábil ou que perdeu as qualidades, com que se poderia recommendar.
O mesmo que \textunderscore desclassificado\textunderscore .
\section{Desqualificador}
\begin{itemize}
\item {Grp. gram.:adj.}
\end{itemize}
\begin{itemize}
\item {Grp. gram.:M.}
\end{itemize}
Que desqualifica.
Aquelle que desqualifica.
\section{Desqualificar}
\begin{itemize}
\item {Grp. gram.:v. t.}
\end{itemize}
\begin{itemize}
\item {Proveniência:(De \textunderscore des...\textunderscore  + \textunderscore qualificar\textunderscore )}
\end{itemize}
Tirar qualificação a.
Inhabilitar.
\section{Desqualificativo}
\begin{itemize}
\item {Grp. gram.:adj.}
\end{itemize}
Que desqualifica.
\section{Desquamar}
\textunderscore v. t.\textunderscore  (e der.)
O mesmo que \textunderscore descamar\textunderscore , etc.
\section{Desquartinar}
\begin{itemize}
\item {Grp. gram.:v. t.}
\end{itemize}
\begin{itemize}
\item {Utilização:P. us.}
\end{itemize}
O mesmo que \textunderscore descortinar\textunderscore .
\section{Desque}
\begin{itemize}
\item {Grp. gram.:loc. prep.}
\end{itemize}
Desde que.
\section{Desquebrar}
\begin{itemize}
\item {Grp. gram.:v. i.}
\end{itemize}
\begin{itemize}
\item {Utilização:Prov.}
\end{itemize}
\begin{itemize}
\item {Utilização:trasm.}
\end{itemize}
\begin{itemize}
\item {Utilização:dur.}
\end{itemize}
\begin{itemize}
\item {Utilização:Prov.}
\end{itemize}
\begin{itemize}
\item {Utilização:beir.}
\end{itemize}
Enfraquecer, deteriorar-se: \textunderscore o vinho nas dornas, exposto ao ar, desquebra\textunderscore .
Arrefecer um pouco, tornar-se morno (aquillo que era quente): \textunderscore não te laves em água tão quente, deixa-a desquebrar\textunderscore .
\section{Desqueixador}
\begin{itemize}
\item {Grp. gram.:m.}
\end{itemize}
Aquelle que desqueixa.
\section{Desqueixar}
\begin{itemize}
\item {Grp. gram.:v. t.}
\end{itemize}
Partir os queixos a.
\section{Desqueixelado}
\begin{itemize}
\item {Grp. gram.:adj.}
\end{itemize}
\begin{itemize}
\item {Utilização:Bras}
\end{itemize}
Admirado, espantado, de queixo caído.
\section{Desquerer}
\begin{itemize}
\item {Grp. gram.:v. t.}
\end{itemize}
Deixar de querer a.
Não querer bem a.
Abandonar.
\section{Desquiar}
\begin{itemize}
\item {Grp. gram.:v. t.}
\end{itemize}
\begin{itemize}
\item {Utilização:Prov.}
\end{itemize}
\begin{itemize}
\item {Utilização:trasm.}
\end{itemize}
O mesmo que \textunderscore tosquiar\textunderscore .
\section{Desquiciar}
\begin{itemize}
\item {Grp. gram.:v. t.}
\end{itemize}
\begin{itemize}
\item {Grp. gram.:V. i.}
\end{itemize}
Tirar os quicios.
Sair dos quicios.
\section{Desquietar}
\begin{itemize}
\item {Proveniência:(De \textunderscore des...\textunderscore  + \textunderscore quieto\textunderscore )}
\end{itemize}
\textunderscore v. t.\textunderscore  (e der.)
O mesmo que \textunderscore inquietar\textunderscore , etc.
\section{Desquilar}
\begin{itemize}
\item {Grp. gram.:v. t.}
\end{itemize}
\begin{itemize}
\item {Utilização:Prov.}
\end{itemize}
\begin{itemize}
\item {Utilização:trasm.}
\end{itemize}
O mesmo que \textunderscore tosquiar\textunderscore .
\section{Desquitação}
\begin{itemize}
\item {Grp. gram.:f.}
\end{itemize}
O mesmo que \textunderscore desquite\textunderscore .
\section{Desquitar}
\begin{itemize}
\item {Grp. gram.:v. t.}
\end{itemize}
\begin{itemize}
\item {Utilização:Fig.}
\end{itemize}
\begin{itemize}
\item {Proveniência:(De \textunderscore des...\textunderscore  + \textunderscore quitar\textunderscore )}
\end{itemize}
Divorciar; separar legalmente (cônjuges).
Compensar.
Deixar.
\section{Desquite}
\begin{itemize}
\item {Grp. gram.:m.}
\end{itemize}
Acto de desquitar.
Separação judicial de cônjuges.
\section{Desrabar}
\begin{itemize}
\item {Grp. gram.:v. t.}
\end{itemize}
(V.derrabar)
\section{Desraigar}
\textunderscore v. t.\textunderscore  (e der.)
O mesmo que \textunderscore desarraigar\textunderscore , etc.
\section{Desramar}
\begin{itemize}
\item {Grp. gram.:v. t.}
\end{itemize}
\begin{itemize}
\item {Proveniência:(De \textunderscore des...\textunderscore  + \textunderscore ramo\textunderscore )}
\end{itemize}
Cortar os ramos a: \textunderscore desramar árvores\textunderscore .
Derramar.
\section{Desrazão}
\begin{itemize}
\item {Grp. gram.:f.}
\end{itemize}
Falta de razão.
Sem-razão.
\section{Desrebuçado}
\begin{itemize}
\item {Grp. gram.:adj.}
\end{itemize}
\begin{itemize}
\item {Proveniência:(De \textunderscore des...\textunderscore  + \textunderscore rebuçado\textunderscore )}
\end{itemize}
Que não tem rebuço.
Em que não há rebuço.
\section{Desrefolho}
\begin{itemize}
\item {fónica:fô}
\end{itemize}
\begin{itemize}
\item {Grp. gram.:m.}
\end{itemize}
\begin{itemize}
\item {Proveniência:(De \textunderscore des...\textunderscore  + \textunderscore refolho\textunderscore )}
\end{itemize}
Acto de devassar os refolhos ou a parte mais íntima de alguma coisa. Cf. Filinto, XI. 292.
\section{Desregradamente}
\begin{itemize}
\item {Grp. gram.:adv.}
\end{itemize}
\begin{itemize}
\item {Proveniência:(De \textunderscore desregrado\textunderscore )}
\end{itemize}
Com desregramento.
Descommedidamente.
\section{Desregrado}
\begin{itemize}
\item {Grp. gram.:adj.}
\end{itemize}
\begin{itemize}
\item {Proveniência:(De \textunderscore desregrar\textunderscore )}
\end{itemize}
Alheio ás bôas regras.
Irregular.
Descommedido.
Perdulário.
Desmoralizado; devasso: \textunderscore homem desregrado\textunderscore .
\section{Desregramento}
\begin{itemize}
\item {Grp. gram.:m.}
\end{itemize}
\begin{itemize}
\item {Proveniência:(De \textunderscore desregrar\textunderscore )}
\end{itemize}
Falta de regularidade.
Descommedimento.
Desordem; abuso.
Libertinagem.
\section{Desregrar}
\begin{itemize}
\item {Grp. gram.:v. t.}
\end{itemize}
\begin{itemize}
\item {Proveniência:(De \textunderscore des...\textunderscore  + \textunderscore regrar\textunderscore )}
\end{itemize}
Afastar da regra.
Tornar irregular, descommedido.
\section{Desrelvar}
\begin{itemize}
\item {Grp. gram.:v. t.}
\end{itemize}
\begin{itemize}
\item {Proveniência:(De \textunderscore des...\textunderscore  + \textunderscore relva\textunderscore )}
\end{itemize}
Cortar a relva de: \textunderscore desrelvar um campo\textunderscore .
\section{Desrespeitador}
\begin{itemize}
\item {Grp. gram.:adj.}
\end{itemize}
\begin{itemize}
\item {Grp. gram.:M.}
\end{itemize}
Que desrespeita.
Aquelle que desrespeita.
\section{Desrespeitar}
\begin{itemize}
\item {Grp. gram.:v. t.}
\end{itemize}
\begin{itemize}
\item {Proveniência:(De \textunderscore des...\textunderscore  + \textunderscore respeitar\textunderscore )}
\end{itemize}
Desacatar. Faltar ao respeito que se deve a.
\section{Desrespeito}
\begin{itemize}
\item {Grp. gram.:m.}
\end{itemize}
Falta de respeito.
\section{Desresponsabilizar}
\begin{itemize}
\item {Grp. gram.:v. t.}
\end{itemize}
\begin{itemize}
\item {Proveniência:(De \textunderscore des...\textunderscore  + \textunderscore responsabilizar\textunderscore )}
\end{itemize}
Livrar de responsabilidade.
Tornar irresponsável.
\section{Desreverência}
\begin{itemize}
\item {Grp. gram.:f.}
\end{itemize}
O mesmo que \textunderscore irreverência\textunderscore .
\section{Desrevestir-se}
\begin{itemize}
\item {Grp. gram.:v. p.}
\end{itemize}
\begin{itemize}
\item {Proveniência:(De \textunderscore des...\textunderscore  + \textunderscore revestir\textunderscore )}
\end{itemize}
Despir-se; tirar as vestes, que o sacerdote usa quando celebra Missa.
\section{Desriçar}
\begin{itemize}
\item {Grp. gram.:v. t.}
\end{itemize}
(V.desenriçar)
\section{Desrisonhar}
\begin{itemize}
\item {Grp. gram.:v. t.}
\end{itemize}
\begin{itemize}
\item {Utilização:Des.}
\end{itemize}
\begin{itemize}
\item {Proveniência:(De \textunderscore des...\textunderscore  + \textunderscore risonho\textunderscore )}
\end{itemize}
Tirar o riso ou a alegria a. Filinto, VII, 44.
\section{Desrolhar}
\textunderscore v. t.\textunderscore  (e der.)
O mesmo que \textunderscore desarrolhar\textunderscore , etc.
\section{Desrugar}
\begin{itemize}
\item {Grp. gram.:v. t.}
\end{itemize}
(V.desenrugar)
\section{Desruidoso}
\begin{itemize}
\item {fónica:ru-i}
\end{itemize}
\begin{itemize}
\item {Grp. gram.:adj.}
\end{itemize}
Não ruidoso.
Que não faz ruido. Cf. Filinto, XVI, 189.
\section{Dessabença}
\begin{itemize}
\item {Grp. gram.:f.}
\end{itemize}
\begin{itemize}
\item {Utilização:Ant.}
\end{itemize}
\begin{itemize}
\item {Proveniência:(De \textunderscore dessaber\textunderscore )}
\end{itemize}
O mesmo que \textunderscore ignorância\textunderscore . Cf. Frei Fortun., I, 305.
\section{Dessaber}
\begin{itemize}
\item {Grp. gram.:v. i.}
\end{itemize}
\begin{itemize}
\item {Utilização:Des.}
\end{itemize}
\begin{itemize}
\item {Proveniência:(De \textunderscore des...\textunderscore  + \textunderscore saber\textunderscore )}
\end{itemize}
Proceder como ignorante.
Esquecer-se do que sabia. Cf. Castilho, \textunderscore Geórgicas\textunderscore , 153.
\section{Dessabor}
\begin{itemize}
\item {Grp. gram.:m.}
\end{itemize}
Falta de sabor.
\section{Dessaborar}
\begin{itemize}
\item {Grp. gram.:v. t.}
\end{itemize}
\begin{itemize}
\item {Proveniência:(De \textunderscore dessabor\textunderscore )}
\end{itemize}
Tirar o sabor a.
\section{Dessaborear}
\begin{itemize}
\item {Grp. gram.:v. t.}
\end{itemize}
O mesmo que \textunderscore dessaborar\textunderscore . Cf. Filinto, XII, 62.
\section{Dessaborido}
\begin{itemize}
\item {Grp. gram.:adj.}
\end{itemize}
O mesmo que \textunderscore sem-sabor\textunderscore .
\section{Dessaboroso}
\begin{itemize}
\item {Grp. gram.:adj.}
\end{itemize}
(V.dessaborido)
\section{Dessagrar}
\begin{itemize}
\item {Grp. gram.:v. t.}
\end{itemize}
\begin{itemize}
\item {Proveniência:(De \textunderscore des...\textunderscore  + \textunderscore sagrar\textunderscore )}
\end{itemize}
Tirar as ordens sacras a.
Tirar a qualidade de sagrado a.
Profanar.
\section{Dessainar}
\begin{itemize}
\item {Grp. gram.:v. t.}
\end{itemize}
\begin{itemize}
\item {Utilização:Prov.}
\end{itemize}
Zangar; irritar.
\section{Dessais}
\begin{itemize}
\item {Grp. gram.:m. pl.}
\end{itemize}
Uma das tríbos da principal classe dos indígenas de Satari, na Índia port.
\section{Dessalgação}
\begin{itemize}
\item {Grp. gram.:f.}
\end{itemize}
Acto de dessalgar.
\section{Dessalgar}
\begin{itemize}
\item {Grp. gram.:v. t.}
\end{itemize}
\begin{itemize}
\item {Proveniência:(De \textunderscore des...\textunderscore  + \textunderscore salgar\textunderscore )}
\end{itemize}
Tirar o sal a.
Tornar insípido ou sem-sabor.
Livrar do feitiço, que o povo chama salgação.
\section{Dessangradeiro}
\begin{itemize}
\item {Grp. gram.:m.}
\end{itemize}
\begin{itemize}
\item {Proveniência:(De \textunderscore dessangrar\textunderscore )}
\end{itemize}
O mesmo que \textunderscore bueiro\textunderscore .
\section{Dessangrar}
\begin{itemize}
\item {Grp. gram.:v. t.}
\end{itemize}
\begin{itemize}
\item {Utilização:Fig.}
\end{itemize}
\begin{itemize}
\item {Proveniência:(De \textunderscore des...\textunderscore  + \textunderscore sangrar\textunderscore )}
\end{itemize}
Tirar todo o sangue a.
Extenuar.
Debilitar.
Privar de recursos.
Esgotar: \textunderscore dessangrou os cofres do pai\textunderscore .
Extrahir algum líquido de (poça, pote, pipa, etc.), por meio de orifício, abaixo do nivel do mesmo líquido.
\section{Dessar}
\begin{itemize}
\item {Grp. gram.:v. t.}
\end{itemize}
\begin{itemize}
\item {Utilização:Prov.}
\end{itemize}
O mesmo que \textunderscore dessalgar\textunderscore .
(Contr. de \textunderscore dessalar\textunderscore , de \textunderscore des...\textunderscore  + \textunderscore sal\textunderscore )
\section{Dessatisfação}
\begin{itemize}
\item {Grp. gram.:f.}
\end{itemize}
\begin{itemize}
\item {Utilização:Des.}
\end{itemize}
Falta de satisfação.
\section{Dessatisfeito}
\begin{itemize}
\item {Grp. gram.:adj.}
\end{itemize}
\begin{itemize}
\item {Utilização:Des.}
\end{itemize}
Não satisfeito; descontente.
\section{Dessaudar}
\begin{itemize}
\item {fónica:sa-u}
\end{itemize}
\begin{itemize}
\item {Grp. gram.:v. t.}
\end{itemize}
Não saudar; desacatar. Cf. Filinto, XVI, 25.
\section{Dessaudoso}
\begin{itemize}
\item {fónica:sa-u}
\end{itemize}
\begin{itemize}
\item {Grp. gram.:adj.}
\end{itemize}
Que não é saudoso. Cf. Filinto, XI, 112.
\section{Dessazonar}
\begin{itemize}
\item {Grp. gram.:v. t.}
\end{itemize}
\begin{itemize}
\item {Proveniência:(De \textunderscore des...\textunderscore  + \textunderscore sazonar\textunderscore )}
\end{itemize}
Tirar o sabor a.
Destemperar.
\section{Dessecação}
\begin{itemize}
\item {Grp. gram.:f.}
\end{itemize}
Acto ou efeito de dessecar.
\section{Dessecamento}
\begin{itemize}
\item {Grp. gram.:m.}
\end{itemize}
O mesmo que \textunderscore dessecação\textunderscore .
\section{Dessecante}
\begin{itemize}
\item {Grp. gram.:adj.}
\end{itemize}
\begin{itemize}
\item {Proveniência:(Do lat. \textunderscore desiccans\textunderscore )}
\end{itemize}
Que desseca.
\section{Dessecar}
\begin{itemize}
\item {Grp. gram.:v. t.}
\end{itemize}
\begin{itemize}
\item {Utilização:Fig.}
\end{itemize}
\begin{itemize}
\item {Proveniência:(Lat. \textunderscore desiccare\textunderscore , se não é gallicismo. Cp. fr. \textunderscore dessécher\textunderscore )}
\end{itemize}
Enxugar; tornar sêco, árido.
Fazer cicatrizar.
Mirrar.
Tornar insensivel.
\section{Dessecativo}
\begin{itemize}
\item {Grp. gram.:adj.}
\end{itemize}
\begin{itemize}
\item {Proveniência:(Lat. \textunderscore desiccativus\textunderscore )}
\end{itemize}
Que faz dessecar.
\section{Dessedentar}
\begin{itemize}
\item {Grp. gram.:v. t.}
\end{itemize}
\begin{itemize}
\item {Proveniência:(De \textunderscore des...\textunderscore  + \textunderscore sedento\textunderscore )}
\end{itemize}
Matar a sêde a.
\section{Dessegredo}
\begin{itemize}
\item {fónica:grê}
\end{itemize}
\begin{itemize}
\item {Grp. gram.:m.}
\end{itemize}
Falta de segrêdo.
Publicidade. Cf. Filinto, XXI, 152.
\section{Desseguir}
\begin{itemize}
\item {Grp. gram.:v. t.}
\end{itemize}
Deixar de seguir.
Desacompanhar.
\section{Dessegurança}
\begin{itemize}
\item {Grp. gram.:f.}
\end{itemize}
\begin{itemize}
\item {Utilização:Ant.}
\end{itemize}
Falta de segurança.
\section{Dessegurar}
\begin{itemize}
\item {Grp. gram.:v. t.}
\end{itemize}
\begin{itemize}
\item {Proveniência:(De \textunderscore des...\textunderscore  + \textunderscore segurar\textunderscore )}
\end{itemize}
Tirar ou deminuir a segurança de.
\section{Dessellar}
\begin{itemize}
\item {Grp. gram.:v. t.}
\end{itemize}
\begin{itemize}
\item {Proveniência:(De \textunderscore des...\textunderscore  + \textunderscore sellar\textunderscore ^1)}
\end{itemize}
Tirar a sella a (cavalgaduras).
\section{Dessellar}
\begin{itemize}
\item {Grp. gram.:v. t.}
\end{itemize}
\begin{itemize}
\item {Proveniência:(De \textunderscore des...\textunderscore  + \textunderscore sellar\textunderscore ^2)}
\end{itemize}
Tirar o sêllo a.
\section{Dessemelhar}
\textunderscore v. t.\textunderscore  (e der.)
(V. \textunderscore dissemelhar\textunderscore , etc.)
\section{Dessentir}
\begin{itemize}
\item {Grp. gram.:v. t.}
\end{itemize}
Não sentir.
\section{Dessepultar}
\begin{itemize}
\item {Grp. gram.:v. t.}
\end{itemize}
\begin{itemize}
\item {Proveniência:(De \textunderscore des...\textunderscore  + \textunderscore sepultar\textunderscore )}
\end{itemize}
Tirar da sepultura.
Exhumar. Cf. Castilho, \textunderscore Geórgicas\textunderscore , 55.
\section{Dessepulto}
\begin{itemize}
\item {Grp. gram.:adj.}
\end{itemize}
O mesmo que \textunderscore insepulto\textunderscore .
\section{Dessér}
\begin{itemize}
\item {Grp. gram.:m.}
\end{itemize}
\begin{itemize}
\item {Proveniência:(Fr. \textunderscore dessert\textunderscore )}
\end{itemize}
(Gal., us. por Filinto, VIII, 244)
Sobremesa, pospasto.
\section{Desserviçal}
\begin{itemize}
\item {Grp. gram.:adj.}
\end{itemize}
Que não é serviçal.
Que desserve. Cf. Filinto, \textunderscore D. Man.\textunderscore , III, 19.
\section{Desserviço}
\begin{itemize}
\item {Grp. gram.:m.}
\end{itemize}
\begin{itemize}
\item {Proveniência:(De \textunderscore des...\textunderscore  + \textunderscore serviço\textunderscore )}
\end{itemize}
Mau serviço.
\section{Desservidor}
\begin{itemize}
\item {Grp. gram.:m.}
\end{itemize}
Aquelle que desserve.
\section{Desservir}
\begin{itemize}
\item {Grp. gram.:v. t.}
\end{itemize}
\begin{itemize}
\item {Proveniência:(De \textunderscore des...\textunderscore  + \textunderscore servir\textunderscore )}
\end{itemize}
Fazer desserviço a.
\section{Dessesmar}
\begin{itemize}
\item {Grp. gram.:v. t.}
\end{itemize}
Desfazer a divisão de, feita em sesmas ou sesmarias. Cf. Costa Lobo, \textunderscore Sátiras\textunderscore , 216.
\section{Dessexuado}
\begin{itemize}
\item {fónica:csu}
\end{itemize}
\begin{itemize}
\item {Grp. gram.:adj.}
\end{itemize}
\begin{itemize}
\item {Utilização:Fig.}
\end{itemize}
Que não tem sexo.
Que não conheceu nunca appetites sexuaes. Cf. Camillo, \textunderscore Brasileira\textunderscore , 259. (De \textunderscore des...\textunderscore  + \textunderscore sexo\textunderscore )
\section{Dessinalado}
\begin{itemize}
\item {Grp. gram.:adj.}
\end{itemize}
\begin{itemize}
\item {Proveniência:(De \textunderscore des...\textunderscore  + \textunderscore sinal\textunderscore )}
\end{itemize}
Que não tem sinal. Cf. Filinto, XV, 134.
\section{Dessiso}
\begin{itemize}
\item {Grp. gram.:m.}
\end{itemize}
Falta de siso. Cf. Filinto, XVI, 19.
\section{Dessisudo}
\begin{itemize}
\item {Grp. gram.:adj.}
\end{itemize}
Que não é sisudo. Cf. Filinto, VII, 100, e XIII, 192.
\section{Dessitiar}
\begin{itemize}
\item {Grp. gram.:v. t.}
\end{itemize}
\begin{itemize}
\item {Proveniência:(De \textunderscore des...\textunderscore  + \textunderscore sitiar\textunderscore )}
\end{itemize}
Livrar de sítio ou de cêrco; descercar. Cf. Filinto, \textunderscore D. Man.\textunderscore , III, 76.
\section{Dessoante}
\begin{itemize}
\item {Grp. gram.:adj.}
\end{itemize}
O mesmo que \textunderscore dissonante\textunderscore .
\section{Dessoar}
\begin{itemize}
\item {Grp. gram.:v. i.}
\end{itemize}
O mesmo que \textunderscore dissonar\textunderscore ; destoar. Cf. Castilho, \textunderscore Primavera\textunderscore , 32.
\section{Dessocar}
\begin{itemize}
\item {Grp. gram.:v. t.}
\end{itemize}
\begin{itemize}
\item {Utilização:Bras. do S}
\end{itemize}
\begin{itemize}
\item {Proveniência:(De \textunderscore des...\textunderscore  + \textunderscore sóco\textunderscore ?)}
\end{itemize}
Fazer certa operação nas mãos ou incisão nos tendões de (cavallo matreiro), para lhe difficultar a carreira.
\section{Dessociável}
\begin{itemize}
\item {Grp. gram.:adj.}
\end{itemize}
Não sociável.
Intratável. Cf. Camillo, \textunderscore Estrêl. Prop.\textunderscore , 40.
\section{Dessoçobrar}
\begin{itemize}
\item {Grp. gram.:v. t.}
\end{itemize}
\begin{itemize}
\item {Grp. gram.:V. i.}
\end{itemize}
\begin{itemize}
\item {Proveniência:(De \textunderscore des...\textunderscore  + \textunderscore soçobrar\textunderscore )}
\end{itemize}
Tirar de dentro.
Desafundar.
Saír de dentro ou do fundo da água. Cf. Filinto, XV, 251.
\section{Dessoçôbro}
\begin{itemize}
\item {Grp. gram.:m.}
\end{itemize}
Acto de dessoçobrar.
\section{Dessocorrer}
\begin{itemize}
\item {Grp. gram.:v. t.}
\end{itemize}
Não socorrer; desprezar.
\section{Dessoldar}
\begin{itemize}
\item {Grp. gram.:v. t.}
\end{itemize}
\begin{itemize}
\item {Proveniência:(De \textunderscore des...\textunderscore  + \textunderscore soldar\textunderscore )}
\end{itemize}
Desligar (aquillo que estava ligado com solda).
\section{Dessolhar}
\begin{itemize}
\item {Grp. gram.:v. t.}
\end{itemize}
\begin{itemize}
\item {Proveniência:(De \textunderscore des...\textunderscore  + \textunderscore solhar\textunderscore )}
\end{itemize}
Tirar o solho a.
\section{Dessorar}
\begin{itemize}
\item {Grp. gram.:v. t.}
\end{itemize}
\begin{itemize}
\item {Proveniência:(De \textunderscore des...\textunderscore  + \textunderscore soro\textunderscore )}
\end{itemize}
Converter em soro.
Tirar a substância a.
Enfraquecer.
\section{Dessossegar}
\begin{itemize}
\item {Grp. gram.:v. t.}
\end{itemize}
O mesmo que \textunderscore desassossegar\textunderscore .
\section{Dessossobrar}
\begin{itemize}
\item {Grp. gram.:v. t.}
\end{itemize}
\begin{itemize}
\item {Grp. gram.:V. i.}
\end{itemize}
\begin{itemize}
\item {Proveniência:(De \textunderscore des...\textunderscore  + \textunderscore sossobrar\textunderscore )}
\end{itemize}
Tirar de dentro.
Desafundar.
Saír de dentro ou do fundo da água. Cf. Filinto, XV, 251.
\section{Dessossobro}
\begin{itemize}
\item {Grp. gram.:m.}
\end{itemize}
Acto de dessossobrar.
\section{Dessoterrar}
\begin{itemize}
\item {Grp. gram.:v. t.}
\end{itemize}
O mesmo que \textunderscore desenterrar\textunderscore .
\section{Dessovado}
\begin{itemize}
\item {Grp. gram.:adj.}
\end{itemize}
\begin{itemize}
\item {Proveniência:(De \textunderscore des...\textunderscore  + \textunderscore sovar\textunderscore )}
\end{itemize}
Que não tem apanhado sova, que está folgado, (falando-se de animaes). Cf. Arn. Gama, \textunderscore Ultima Dona\textunderscore , 182.
\section{Dessuar}
\begin{itemize}
\item {Grp. gram.:v. i.}
\end{itemize}
\begin{itemize}
\item {Proveniência:(De \textunderscore des...\textunderscore  + \textunderscore suar\textunderscore )}
\end{itemize}
Deixar de suar.
Enxugar o suor:«\textunderscore e lhes dessua a fronte.\textunderscore »Castilho, \textunderscore Escav. Poét.\textunderscore , 43.
\section{Dessubjugar}
\begin{itemize}
\item {Grp. gram.:v. t.}
\end{itemize}
\begin{itemize}
\item {Proveniência:(De \textunderscore des...\textunderscore  + \textunderscore subjugar\textunderscore )}
\end{itemize}
Livrar do jugo ou da sujeição. Cf. Sousa, \textunderscore Vida do Arceb.\textunderscore , III, 75.
\section{Dessubstanciar}
\begin{itemize}
\item {Grp. gram.:v. t.}
\end{itemize}
Tirar a substância de. Cf. Rui Barb., \textunderscore Réplica\textunderscore , 157.
\section{Dessuetude}
\begin{itemize}
\item {Grp. gram.:f.}
\end{itemize}
\begin{itemize}
\item {Proveniência:(De \textunderscore des...\textunderscore  + lat. \textunderscore suetudo\textunderscore . Cp. lat. \textunderscore dissuescere\textunderscore )}
\end{itemize}
Descostume.
Deshábito. Cf. Garrett, \textunderscore Port. na Balança\textunderscore , 301.
\section{Dessujar}
\begin{itemize}
\item {Grp. gram.:v. t.}
\end{itemize}
\begin{itemize}
\item {Proveniência:(De \textunderscore des...\textunderscore  + \textunderscore sujar\textunderscore )}
\end{itemize}
Tirar a sujidade a.
Limpar. Cf. Filinto, XVIII, 69.
\section{Dessujeito}
\begin{itemize}
\item {Grp. gram.:adj.}
\end{itemize}
Não sujeito; livre. Cf. \textunderscore Viriato Trág.\textunderscore , X, 1.
\section{Dessujo}
\begin{itemize}
\item {Grp. gram.:adj.}
\end{itemize}
Que não está sujo.
Limpo.
\section{Dessulfuração}
\begin{itemize}
\item {Grp. gram.:f.}
\end{itemize}
Acto de dessulfurar.
\section{Dessulfurar}
\begin{itemize}
\item {Grp. gram.:v. t.}
\end{itemize}
\begin{itemize}
\item {Proveniência:(De \textunderscore des...\textunderscore  + \textunderscore sulfurar\textunderscore )}
\end{itemize}
Tirar o enxôfre a.
Separar o enxôfre de.
\section{Dessulfurizar}
\begin{itemize}
\item {Grp. gram.:v. t.}
\end{itemize}
(V.dessulfurar)
\section{Dessurdo}
\begin{itemize}
\item {Grp. gram.:adj.}
\end{itemize}
Que não é surdo, que ouve bem. Cf. Filinto, V, 112.
\section{Dessimetria}
\begin{itemize}
\item {Grp. gram.:f.}
\end{itemize}
Falta de simetria. Cf. Júl. Castilho, \textunderscore Lisb. Antiga\textunderscore .--Preferível \textunderscore asimetria\textunderscore .
\section{Dessimétrico}
\begin{itemize}
\item {Grp. gram.:adj.}
\end{itemize}
Que não é simétrico; em que não há simetria. Cf. Camillo, \textunderscore Quéda\textunderscore , 50.
\section{Dessuspeitoso}
\begin{itemize}
\item {Grp. gram.:adj.}
\end{itemize}
Não suspeitoso; em que não há suspeita ou que não causa suspeita. Cf. Filinto, VII, 15.
\section{Dessymetria}
\begin{itemize}
\item {Grp. gram.:f.}
\end{itemize}
Falta de symetria. Cf. Júl. Castilho, \textunderscore Lisb. Antiga\textunderscore .--Preferível \textunderscore asymetria\textunderscore .
\section{Dessymétrico}
\begin{itemize}
\item {Grp. gram.:adj.}
\end{itemize}
Que não é symétrico; em que não há symetria. Cf. Camillo, \textunderscore Quéda\textunderscore , 50.
\section{Destabocado}
\begin{itemize}
\item {Grp. gram.:adj.}
\end{itemize}
\begin{itemize}
\item {Utilização:Bras}
\end{itemize}
Adoidado; que não respeita as conveniências.
\section{Destacado}
\begin{itemize}
\item {Grp. gram.:m.}
\end{itemize}
\begin{itemize}
\item {Utilização:Mús.}
\end{itemize}
Maneira de fazer succeder as notas, separando umas das outras.
(Cp. it. \textunderscore staccato\textunderscore )
\section{Destacamento}
\begin{itemize}
\item {Grp. gram.:m.}
\end{itemize}
Porção de tropa, que se separa de um regimento, para fazer serviço fóra da séde do mesmo regimento.
Acto de destacar.
\section{Destacar}
\begin{itemize}
\item {Grp. gram.:v. t.}
\end{itemize}
\begin{itemize}
\item {Utilização:Gal}
\end{itemize}
\begin{itemize}
\item {Utilização:Mús.}
\end{itemize}
\begin{itemize}
\item {Grp. gram.:V. i.}
\end{itemize}
\begin{itemize}
\item {Utilização:Gal}
\end{itemize}
Expedir ou enviar (um troço de tropas, que formam destacamento).
Expedir.
Despedir.
Tornar saliente; dar relêvo a.
Separar.
Articular (notas successivas), separando-as, como se fôssem cortadas por pequenas pausas.
Fazer parte de um destacamento.
Sobrelevar, tomar relêvo; distinguir-se; separar-se.
(Cp. it. \textunderscore staccare\textunderscore )
\section{Destalhar}
\begin{itemize}
\item {Grp. gram.:v. i.}
\end{itemize}
\begin{itemize}
\item {Utilização:Prov.}
\end{itemize}
Talhar, decompor-se, (o leite).
(Colhido no \textunderscore Pôrto\textunderscore )
\section{Destalingar}
\begin{itemize}
\item {Grp. gram.:v.}
\end{itemize}
\begin{itemize}
\item {Utilização:t. Náut.}
\end{itemize}
\begin{itemize}
\item {Proveniência:(De \textunderscore des...\textunderscore  + \textunderscore talingar\textunderscore )}
\end{itemize}
Desatar (cabos que estão talingados).
\section{Destampadamente}
\begin{itemize}
\item {Grp. gram.:adv.}
\end{itemize}
\begin{itemize}
\item {Proveniência:(De \textunderscore destampar\textunderscore )}
\end{itemize}
Disparatadamente.
\section{Destampado}
\begin{itemize}
\item {Grp. gram.:adj.}
\end{itemize}
\begin{itemize}
\item {Utilização:Fig.}
\end{itemize}
\begin{itemize}
\item {Proveniência:(De \textunderscore destampar\textunderscore )}
\end{itemize}
Abundante, copioso:«\textunderscore os destampados chuveiros\textunderscore ». Filinto, \textunderscore D. Man.\textunderscore , I, 348.
\section{Destampar}
\begin{itemize}
\item {Grp. gram.:v. t.}
\end{itemize}
\begin{itemize}
\item {Grp. gram.:V. i.}
\end{itemize}
\begin{itemize}
\item {Utilização:Fam.}
\end{itemize}
\begin{itemize}
\item {Proveniência:(De \textunderscore des...\textunderscore  + \textunderscore tampo\textunderscore  ou \textunderscore tampa\textunderscore )}
\end{itemize}
Tirar o tampo ou tampa a.
Disparatar.
\section{Destampatória}
\begin{itemize}
\item {Grp. gram.:f.}
\end{itemize}
O mesmo que \textunderscore destampatório\textunderscore .
\section{Destampatório}
\begin{itemize}
\item {Grp. gram.:m.}
\end{itemize}
\begin{itemize}
\item {Utilização:Fam.}
\end{itemize}
\begin{itemize}
\item {Proveniência:(De \textunderscore destampar\textunderscore )}
\end{itemize}
Gritaria.
Despropósito, desconchavo.
\section{Destampice}
\begin{itemize}
\item {Grp. gram.:f.}
\end{itemize}
\begin{itemize}
\item {Utilização:ant.}
\end{itemize}
\begin{itemize}
\item {Utilização:Fam.}
\end{itemize}
\begin{itemize}
\item {Proveniência:(De \textunderscore destampar\textunderscore )}
\end{itemize}
Disparate.
Destampatório.
\section{Destampo}
\begin{itemize}
\item {Grp. gram.:m.}
\end{itemize}
Acto de destampar; destampice. Cf. \textunderscore Agostinheida\textunderscore , 72.
\section{Destanizar}
\begin{itemize}
\item {Grp. gram.:v. t.}
\end{itemize}
Tirar o tanino a. Cf. \textunderscore Techn. Rur.\textunderscore , 240.
(Por \textunderscore destaninizar\textunderscore , de \textunderscore des...\textunderscore  + \textunderscore tanino\textunderscore )
\section{Destannizar}
\begin{itemize}
\item {Grp. gram.:v. t.}
\end{itemize}
Tirar o tannino a. Cf. \textunderscore Techn. Rur.\textunderscore , 240.
(Por \textunderscore destanninizar\textunderscore , de \textunderscore des...\textunderscore  + \textunderscore tannino\textunderscore )
\section{Destapamento}
\begin{itemize}
\item {Grp. gram.:m.}
\end{itemize}
Acto de destapar.
\section{Destapar}
\begin{itemize}
\item {Grp. gram.:v. t.}
\end{itemize}
\begin{itemize}
\item {Proveniência:(De \textunderscore des...\textunderscore  + \textunderscore tapar\textunderscore )}
\end{itemize}
Descobrir (aquillo que estava tapado).
Destampar.
\section{Destaque}
\begin{itemize}
\item {Grp. gram.:m.}
\end{itemize}
\begin{itemize}
\item {Utilização:Gal}
\end{itemize}
\begin{itemize}
\item {Proveniência:(De \textunderscore destacar\textunderscore )}
\end{itemize}
Acção de desligar os floretes, cambiando-lhes a posição, no jôgo da esgrima.
Qualidade daquillo que sobresái, que se destaca.
\section{Destecedura}
\begin{itemize}
\item {Grp. gram.:f.}
\end{itemize}
Acto de destecer.
\section{Destecer}
\begin{itemize}
\item {Grp. gram.:v. t.}
\end{itemize}
\begin{itemize}
\item {Utilização:Fig.}
\end{itemize}
\begin{itemize}
\item {Proveniência:(De \textunderscore des...\textunderscore  + \textunderscore tecer\textunderscore )}
\end{itemize}
Desmanchar (aquillo que estava tecido).
Desenredar.
Dissipar:«\textunderscore as trevas da razão se desteciam.\textunderscore »Camillo, \textunderscore Corja\textunderscore , 136.
\section{Destelado}
\begin{itemize}
\item {Grp. gram.:adj.}
\end{itemize}
\begin{itemize}
\item {Utilização:Prov.}
\end{itemize}
Que destelou.
\section{Destelar}
\begin{itemize}
\item {Grp. gram.:v. i.}
\end{itemize}
\begin{itemize}
\item {Utilização:Prov.}
\end{itemize}
\begin{itemize}
\item {Proveniência:(De \textunderscore destêlo\textunderscore )}
\end{itemize}
Caír das árvores, em consequência do vento, ou por têr attingido o maior grau de maturação, (falando-se do fruto do castanheiro e da oliveira).
\section{Destelhamento}
\begin{itemize}
\item {Grp. gram.:m.}
\end{itemize}
Acto de destelhar.
\section{Destelhar}
\begin{itemize}
\item {Grp. gram.:v. t.}
\end{itemize}
Tirar as telhas de (um prédio).
\section{Destêlo}
\begin{itemize}
\item {Grp. gram.:m.}
\end{itemize}
\begin{itemize}
\item {Utilização:Prov.}
\end{itemize}
Acto de destelar.
Fruto que destela.
(Talvez do rad. de \textunderscore destillar\textunderscore , como quem diz \textunderscore gotejar\textunderscore , \textunderscore deixar cair a pouco e pouco\textunderscore )
\section{Destemer}
\begin{itemize}
\item {Grp. gram.:v. t.}
\end{itemize}
Não temer.
\section{Destemidamente}
\begin{itemize}
\item {Grp. gram.:adv.}
\end{itemize}
Corajosamente; de modo destemido.
\section{Destemidez}
\begin{itemize}
\item {Grp. gram.:f.}
\end{itemize}
Qualidade de quem é destemido.
\section{Destemido}
\begin{itemize}
\item {Grp. gram.:adj.}
\end{itemize}
\begin{itemize}
\item {Proveniência:(De \textunderscore destemer\textunderscore )}
\end{itemize}
Que não tem temor.
Corajoso; que revela coragem.
\section{Destemor}
\begin{itemize}
\item {Grp. gram.:m.}
\end{itemize}
Falta de temor.
Intrepidez; audácia.
\section{Destêmpera}
\begin{itemize}
\item {Grp. gram.:f.}
\end{itemize}
\begin{itemize}
\item {Proveniência:(De \textunderscore des...\textunderscore  + \textunderscore têmpera\textunderscore )}
\end{itemize}
Acto de destemperar (o aço).
\section{Destemperadamente}
\begin{itemize}
\item {Grp. gram.:adv.}
\end{itemize}
De modo destemperado.
Disparatadamente.
\section{Destemperado}
\begin{itemize}
\item {Grp. gram.:adj.}
\end{itemize}
\begin{itemize}
\item {Proveniência:(De \textunderscore destemperar\textunderscore )}
\end{itemize}
Disparatado; desordenado.
\section{Destemperamento}
\begin{itemize}
\item {Grp. gram.:m.}
\end{itemize}
Acto ou effeito de destemperar.
\section{Destemperança}
\begin{itemize}
\item {Grp. gram.:f.}
\end{itemize}
O mesmo que \textunderscore intemperança\textunderscore .
\section{Destemperar}
\begin{itemize}
\item {Grp. gram.:v. t.}
\end{itemize}
\begin{itemize}
\item {Grp. gram.:V. i.}
\end{itemize}
\begin{itemize}
\item {Grp. gram.:V. p.}
\end{itemize}
\begin{itemize}
\item {Proveniência:(De \textunderscore des...\textunderscore  + \textunderscore temperar\textunderscore )}
\end{itemize}
Deminuir a temperatura ou a fôrça de: \textunderscore destemperar água quente\textunderscore .
Alterar o sabor de, tornando-o menos pronunciado.
Fazer perder a têmpera de.
Desafinar: \textunderscore destemperar uma rabeca\textunderscore .
Desorganizar.
Desconcertar (as funcções intestinaes), produzindo diarreia.
Descommedir-se.
Disparatar.
Perder a têmpera, (falando-se do aço).
Perder a têmpera.
Descommedir-se.
\section{Destempêro}
\begin{itemize}
\item {Grp. gram.:m.}
\end{itemize}
Acto ou effeito de destemperar.
Desconchavo, dislate.
\section{Destempo}
\begin{itemize}
\item {Grp. gram.:m. Loc. adv.}
\end{itemize}
\begin{itemize}
\item {Proveniência:(De \textunderscore des...\textunderscore  + \textunderscore tempo\textunderscore )}
\end{itemize}
\textunderscore A destempo\textunderscore , fóra de horas; inopportunamente.
\section{Desteridade}
\begin{itemize}
\item {Grp. gram.:f.}
\end{itemize}
\begin{itemize}
\item {Proveniência:(Lat. \textunderscore desteritas\textunderscore )}
\end{itemize}
O mesmo que \textunderscore destreza\textunderscore .
\section{Desterrador}
\begin{itemize}
\item {Grp. gram.:m.  e  adj.}
\end{itemize}
O que desterra ou impelle a destêrro. Cf. Castilho, \textunderscore Montalverne\textunderscore .
\section{Desterramento}
\begin{itemize}
\item {Grp. gram.:m.}
\end{itemize}
\begin{itemize}
\item {Utilização:Des.}
\end{itemize}
O mesmo que \textunderscore destêrro\textunderscore .
\section{Desterrar}
\begin{itemize}
\item {Grp. gram.:v. t.}
\end{itemize}
\begin{itemize}
\item {Proveniência:(De \textunderscore des...\textunderscore  + \textunderscore terra\textunderscore )}
\end{itemize}
Expulsar da residência ou da pátria, por castigo.
Expatriar.
Insular.
Afastar; afugentar: \textunderscore desterrar pesares\textunderscore .
Fazer cessar.
\section{Destêrro}
\begin{itemize}
\item {Grp. gram.:m.}
\end{itemize}
Acto ou effeito de desterrar.
Degrêdo.
Pena de degrêdo.
Lugar, onde vive o desterrado.
Solidão.
\section{Desterroar}
\textunderscore v. t.\textunderscore  (e der.)
O mesmo que \textunderscore esterroar\textunderscore .
Tirar terra de:«\textunderscore o coveiro que desterroava uma sepultura\textunderscore ». Camillo, \textunderscore Caveira\textunderscore , 410.
\section{Destetar}
\begin{itemize}
\item {Grp. gram.:v. t.}
\end{itemize}
\begin{itemize}
\item {Proveniência:(De \textunderscore des...\textunderscore  + \textunderscore têta\textunderscore )}
\end{itemize}
O mesmo que \textunderscore desmamar\textunderscore ; desleitar.
\section{Desthronar}
\begin{itemize}
\item {Grp. gram.:v. t.}
\end{itemize}
\begin{itemize}
\item {Utilização:Fig.}
\end{itemize}
\begin{itemize}
\item {Proveniência:(De \textunderscore des...\textunderscore  + \textunderscore throno\textunderscore )}
\end{itemize}
Tirar do throno.
Destituir da soberania.
Abater, humilhar.
\section{Desthronização}
\begin{itemize}
\item {Grp. gram.:f.}
\end{itemize}
Acto ou effeito de desthronizar.
\section{Desthronizar}
\begin{itemize}
\item {Grp. gram.:v. t.}
\end{itemize}
O mesmo que \textunderscore desthronar\textunderscore .
\section{Destilação}
\begin{itemize}
\item {Grp. gram.:f.}
\end{itemize}
Acto de destilar.
\section{Destiladeira}
\begin{itemize}
\item {Grp. gram.:f.}
\end{itemize}
\begin{itemize}
\item {Utilização:Bras. do N}
\end{itemize}
\begin{itemize}
\item {Proveniência:(De \textunderscore destilar\textunderscore )}
\end{itemize}
Espécie de cesto de palha, afunilado, em que se deitam as cinzas das coivaras e água, para destilar o líquido escuro e cáustico, que se aproveita no fabrico do sabão.
\section{Destilar}
\begin{itemize}
\item {Grp. gram.:v. t.}
\end{itemize}
\begin{itemize}
\item {Utilização:Fig.}
\end{itemize}
\begin{itemize}
\item {Grp. gram.:V. i.}
\end{itemize}
\begin{itemize}
\item {Proveniência:(Lat. \textunderscore destillare\textunderscore )}
\end{itemize}
Gotejar.
Deixar cair, gota a gota: \textunderscore destilar suor\textunderscore .
Separar, por meio do fogo e em vasos fechados, das partes fixas de uma substância (a parte volátil).
Condensar (os vapores de um líquido que se fez evaporar pelo calor): \textunderscore destilar aguardente\textunderscore .
Instilar, infundir a pouco e pouco.
Insinuar.
Cair, gota a gota: \textunderscore o sangue destilava-lhe da ferida\textunderscore .
\section{Destilaria}
\begin{itemize}
\item {Grp. gram.:f.}
\end{itemize}
\begin{itemize}
\item {Utilização:Neol.}
\end{itemize}
\begin{itemize}
\item {Proveniência:(De \textunderscore destilar\textunderscore )}
\end{itemize}
Fábrica de destilação. Cf. R. Ortigão, \textunderscore Hollanda\textunderscore , p. 65.
\section{Destilatório}
\begin{itemize}
\item {Grp. gram.:adj.}
\end{itemize}
\begin{itemize}
\item {Proveniência:(De \textunderscore destilar\textunderscore )}
\end{itemize}
Que serve para destilar: \textunderscore aparelho destilatório\textunderscore .
\section{Destillação}
\begin{itemize}
\item {Grp. gram.:f.}
\end{itemize}
Acto de destillar.
\section{Destilladeira}
\begin{itemize}
\item {Grp. gram.:f.}
\end{itemize}
\begin{itemize}
\item {Utilização:Bras. do N}
\end{itemize}
\begin{itemize}
\item {Proveniência:(De \textunderscore destillar\textunderscore )}
\end{itemize}
Espécie de cesto de palha, afunilado, em que se deitam as cinzas das coivaras e água, para destillar o líquido escuro e cáustico, que se aproveita no fabrico do sabão.
\section{Destillar}
\begin{itemize}
\item {Grp. gram.:v. t.}
\end{itemize}
\begin{itemize}
\item {Utilização:Fig.}
\end{itemize}
\begin{itemize}
\item {Grp. gram.:V. i.}
\end{itemize}
\begin{itemize}
\item {Proveniência:(Lat. \textunderscore destillare\textunderscore )}
\end{itemize}
Gotejar.
Deixar cair, gota a gota: \textunderscore destillar suor\textunderscore .
Separar, por meio do fogo e em vasos fechados, das partes fixas de uma substância (a parte volátil).
Condensar (os vapores de um líquido que se fez evaporar pelo calor): \textunderscore destillar aguardente\textunderscore .
Instillar, infundir a pouco e pouco.
Insinuar.
Cair, gota a gota: \textunderscore o sangue destillava-lhe da ferida\textunderscore .
\section{Destillaria}
\begin{itemize}
\item {Grp. gram.:f.}
\end{itemize}
\begin{itemize}
\item {Utilização:Neol.}
\end{itemize}
\begin{itemize}
\item {Proveniência:(De \textunderscore destillar\textunderscore )}
\end{itemize}
Fábrica de destillação. Cf. R. Ortigão, \textunderscore Hollanda\textunderscore , p. 65.
\section{Destillatório}
\begin{itemize}
\item {Grp. gram.:adj.}
\end{itemize}
\begin{itemize}
\item {Proveniência:(De \textunderscore destillar\textunderscore )}
\end{itemize}
Que serve para destillar: \textunderscore apparelho destillatório\textunderscore .
\section{Destimidez}
\begin{itemize}
\item {Grp. gram.:f.}
\end{itemize}
Qualidade ou estado de destímido.
\section{Destímido}
\begin{itemize}
\item {Grp. gram.:adj.}
\end{itemize}
Que não é timido; que perdeu o temor.
\section{Destinação}
\begin{itemize}
\item {Grp. gram.:f.}
\end{itemize}
\begin{itemize}
\item {Proveniência:(Lat. \textunderscore destinatio\textunderscore )}
\end{itemize}
Acto de destinar, destino.
\section{Destinador}
\begin{itemize}
\item {Grp. gram.:adj.}
\end{itemize}
\begin{itemize}
\item {Grp. gram.:M.}
\end{itemize}
Que destina.
Aquelle que destina.
\section{Destinar}
\begin{itemize}
\item {Grp. gram.:v. t.}
\end{itemize}
\begin{itemize}
\item {Proveniência:(Lat. \textunderscore destinare\textunderscore )}
\end{itemize}
Determinar com antecípação.
Fixar préviamente.
Designar o objecto ou o fim de.
Reservar para determinado emprêgo ou fim.
Reservar.
Resolver, decidir.
\section{Destinatário}
\begin{itemize}
\item {Grp. gram.:m.}
\end{itemize}
\begin{itemize}
\item {Proveniência:(De destinar)}
\end{itemize}
Aquelle a quem se envia ou destina alguma coisa.
\section{Destincto}
\begin{itemize}
\item {Grp. gram.:adj.}
\end{itemize}
\begin{itemize}
\item {Proveniência:(De \textunderscore des...\textunderscore  + \textunderscore tinto\textunderscore )}
\end{itemize}
Que se destingiu.
\section{Destingir}
\begin{itemize}
\item {Grp. gram.:v. t.}
\end{itemize}
\begin{itemize}
\item {Grp. gram.:V. i.}
\end{itemize}
\begin{itemize}
\item {Proveniência:(De \textunderscore des...\textunderscore  + \textunderscore tingir\textunderscore )}
\end{itemize}
Tirar a côr a.
Fazer perder a tinta de.
Fazer desbotar: \textunderscore o tempo destingiu a chita\textunderscore .
Perder a côr ou a tinta.
Desbotar: \textunderscore esta fazenda destinge facilmente\textunderscore .
\section{Destino}
\begin{itemize}
\item {Grp. gram.:m.}
\end{itemize}
\begin{itemize}
\item {Proveniência:(De \textunderscore destinar\textunderscore )}
\end{itemize}
Sucessão dos factos, suppostamente necessários; fatalidade: \textunderscore ninguém foge ao seu destino\textunderscore .
Sorte.
Objecto, fim, para que se reserva ou se designa alguma coisa.
Applicação.
Existência.
Lugar, a que se dirige alguma coisa ou pessôa; direcção: \textunderscore indicar o destino de uma carta\textunderscore .
\section{Destino}
\begin{itemize}
\item {Grp. gram.:m.}
\end{itemize}
\begin{itemize}
\item {Utilização:Bras}
\end{itemize}
\begin{itemize}
\item {Utilização:pop.}
\end{itemize}
O mesmo que \textunderscore desatino\textunderscore .
\section{Destinto}
\begin{itemize}
\item {Grp. gram.:adj.}
\end{itemize}
\begin{itemize}
\item {Proveniência:(De \textunderscore des...\textunderscore  + \textunderscore tinto\textunderscore )}
\end{itemize}
Que se destingiu.
\section{Destinto}
\begin{itemize}
\item {Grp. gram.:m.}
\end{itemize}
\begin{itemize}
\item {Utilização:T. do Fundão}
\end{itemize}
Coisa ou pessôa que gasta muito, que consome ou destrói: \textunderscore êste rapaz é o destinto do meu dinheiro\textunderscore .
(Relaciona-se com \textunderscore destinto\textunderscore , de \textunderscore destingir\textunderscore ?)
\section{Destituição}
\begin{itemize}
\item {fónica:tu-i}
\end{itemize}
\begin{itemize}
\item {Grp. gram.:f.}
\end{itemize}
\begin{itemize}
\item {Proveniência:(Lat. \textunderscore destitutio\textunderscore )}
\end{itemize}
Acto ou effeito de destituir.
Demissão.
\section{Destituir}
\begin{itemize}
\item {Grp. gram.:v. t.}
\end{itemize}
\begin{itemize}
\item {Proveniência:(Lat. \textunderscore destituere\textunderscore )}
\end{itemize}
Depor.
Privar de emprêgo ou dignidade.
Privar: \textunderscore destituir alguém de recursos\textunderscore .
\section{Destoante}
\begin{itemize}
\item {Grp. gram.:adj.}
\end{itemize}
Que destoa; discordante; divergente.
\section{Destoar}
\begin{itemize}
\item {Grp. gram.:v. i.}
\end{itemize}
\begin{itemize}
\item {Utilização:Fig.}
\end{itemize}
\begin{itemize}
\item {Grp. gram.:V.}
\end{itemize}
\begin{itemize}
\item {Utilização:t. Mús.}
\end{itemize}
\begin{itemize}
\item {Proveniência:(De \textunderscore des...\textunderscore  + \textunderscore toar\textunderscore )}
\end{itemize}
Saír do tom, desafinar.
Soar mal.
Desagradar.
Divergir: \textunderscore actos, que destôam do dever\textunderscore .
Não se conformar.
Desentoar, desafinar.
\section{Destocamento}
\begin{itemize}
\item {Grp. gram.:m.}
\end{itemize}
Acto de destocar^1. Cf. \textunderscore Vocabulário de Estradas\textunderscore .
\section{Destocar}
\begin{itemize}
\item {Grp. gram.:v. t.}
\end{itemize}
\begin{itemize}
\item {Utilização:Bras. de Minas}
\end{itemize}
\begin{itemize}
\item {Proveniência:(De \textunderscore des...\textunderscore  + \textunderscore toco\textunderscore )}
\end{itemize}
Limpar de tocos (um campo).
Escanhoar (a barba).
\section{Destocar}
\begin{itemize}
\item {Grp. gram.:v. t.}
\end{itemize}
\begin{itemize}
\item {Proveniência:(De \textunderscore des...\textunderscore  + \textunderscore tocar\textunderscore )}
\end{itemize}
Desligar, separar, abrir:«\textunderscore amiudados terremotos, muitos edificios foram destocados...\textunderscore »Filinto, \textunderscore D. Man.\textunderscore , I, 259.
\section{Destoituçada}
\begin{itemize}
\item {Grp. gram.:adj. f.}
\end{itemize}
\begin{itemize}
\item {Utilização:Prov.}
\end{itemize}
\begin{itemize}
\item {Utilização:trasm.}
\end{itemize}
Diz-se da rapariga leviana, estouvada.
(Por \textunderscore destoitiçada\textunderscore , de \textunderscore des...\textunderscore  + \textunderscore toitiço\textunderscore )
\section{Destoldar}
\begin{itemize}
\item {Grp. gram.:v. t.}
\end{itemize}
\begin{itemize}
\item {Utilização:Fig.}
\end{itemize}
\begin{itemize}
\item {Proveniência:(De \textunderscore des...\textunderscore  + \textunderscore toldar\textunderscore )}
\end{itemize}
Tirar o tôldo a.
Descobrir.
Tornar claro, límpido.
\section{Destom}
\begin{itemize}
\item {Grp. gram.:m.}
\end{itemize}
\begin{itemize}
\item {Proveniência:(De \textunderscore des...\textunderscore  + \textunderscore tom\textunderscore )}
\end{itemize}
Desharmonia.
Divergência.
Degeneração.
\section{Destopetear}
\begin{itemize}
\item {Grp. gram.:v. t.}
\end{itemize}
Tirar o topête a.
\section{Destorar}
\begin{itemize}
\item {Grp. gram.:v. t.}
\end{itemize}
Cortar os toros a.
\section{Destorcer}
\begin{itemize}
\item {Grp. gram.:v. t.}
\end{itemize}
\begin{itemize}
\item {Proveniência:(De \textunderscore des...\textunderscore  + \textunderscore torcer\textunderscore )}
\end{itemize}
Torcer em sentido opposto àquelle em que se torceu (uma corda, cordel, etc.).
Desmanchar a torcedura de.
Tornar direito (aquillo que era torcido).
\section{Destorpecer}
\begin{itemize}
\item {Grp. gram.:v. t.}
\end{itemize}
O mesmo que \textunderscore desentorpecer\textunderscore .
\section{Destorpecido}
\begin{itemize}
\item {Grp. gram.:adj.}
\end{itemize}
O mesmo que [[desentorpecido|desentorpecer]]. Cf. Filinto, XIII, 193.
\section{Destorroar}
\textunderscore v. t.\textunderscore  (e der.)
O mesmo que \textunderscore estorroar\textunderscore , etc.
\section{Destoucar}
\begin{itemize}
\item {Grp. gram.:v. t.}
\end{itemize}
\begin{itemize}
\item {Utilização:Fig.}
\end{itemize}
\begin{itemize}
\item {Proveniência:(De \textunderscore des...\textunderscore  + \textunderscore toucar\textunderscore )}
\end{itemize}
Tirar a touca a.
Desarranjar o toucado de.
Desenfeitar.
\section{Destra}
\begin{itemize}
\item {Grp. gram.:f.}
\end{itemize}
\begin{itemize}
\item {Proveniência:(Do lat. \textunderscore dexter\textunderscore )}
\end{itemize}
A mão direita.
\section{Destragar}
\begin{itemize}
\item {Grp. gram.:v. t.}
\end{itemize}
\begin{itemize}
\item {Proveniência:(De \textunderscore de...\textunderscore  + \textunderscore estragar\textunderscore )}
\end{itemize}
O mesmo que \textunderscore estragar\textunderscore . Cf. Camillo, \textunderscore Judeu\textunderscore , 138.
\section{Destramar}
\begin{itemize}
\item {Grp. gram.:v. t.}
\end{itemize}
\begin{itemize}
\item {Proveniência:(De \textunderscore des...\textunderscore  + \textunderscore tramar\textunderscore )}
\end{itemize}
Desmanchar a trama de.
Desenredar.
\section{Destrambelhado}
\begin{itemize}
\item {Grp. gram.:adj.}
\end{itemize}
\begin{itemize}
\item {Utilização:Pop.}
\end{itemize}
\begin{itemize}
\item {Proveniência:(De \textunderscore destrambelho\textunderscore )}
\end{itemize}
Disparatado.
Desorganizado.
Desnorteado; desordenado.
\section{Destrambelho}
\begin{itemize}
\item {fónica:bê}
\end{itemize}
\begin{itemize}
\item {Grp. gram.:m.}
\end{itemize}
\begin{itemize}
\item {Utilização:Pop.}
\end{itemize}
\begin{itemize}
\item {Proveniência:(De \textunderscore des...\textunderscore  + \textunderscore trambelho\textunderscore )}
\end{itemize}
Desordem, desarranjo.
Disparate.
\section{Destramente}
\begin{itemize}
\item {Grp. gram.:adv.}
\end{itemize}
\begin{itemize}
\item {Proveniência:(De \textunderscore destro\textunderscore ^1)}
\end{itemize}
Com destreza.
Habilmente.
Promptamente.
\section{Destrancar}
\begin{itemize}
\item {Grp. gram.:v. t.}
\end{itemize}
\begin{itemize}
\item {Proveniência:(De \textunderscore des...\textunderscore  + \textunderscore trancar\textunderscore )}
\end{itemize}
Tirar a tranca a.
\section{Destrançar}
\begin{itemize}
\item {Grp. gram.:v. t.}
\end{itemize}
O mesmo que \textunderscore desentrançar\textunderscore .
\section{Destratar}
\begin{itemize}
\item {Grp. gram.:v. t.}
\end{itemize}
\begin{itemize}
\item {Utilização:Bras}
\end{itemize}
\begin{itemize}
\item {Proveniência:(De \textunderscore des...\textunderscore  + \textunderscore tratar\textunderscore )}
\end{itemize}
Insultar; maltratar com palavras.
\section{Destravado}
\begin{itemize}
\item {Grp. gram.:adj.}
\end{itemize}
\begin{itemize}
\item {Utilização:Prov.}
\end{itemize}
\begin{itemize}
\item {Utilização:minh.}
\end{itemize}
\begin{itemize}
\item {Proveniência:(De \textunderscore destravar\textunderscore )}
\end{itemize}
Desbocado; linguareiro.
Insolente.
\section{Destravancado}
\begin{itemize}
\item {Grp. gram.:adj.}
\end{itemize}
\begin{itemize}
\item {Utilização:Bras. de Minas}
\end{itemize}
\begin{itemize}
\item {Proveniência:(De \textunderscore destravancar\textunderscore )}
\end{itemize}
Incorrigível.
\section{Destravancar}
\textunderscore v. t.\textunderscore  (e der.)
O mesmo que \textunderscore desatravancar\textunderscore .
\section{Destravar}
\begin{itemize}
\item {Grp. gram.:v. t.}
\end{itemize}
\begin{itemize}
\item {Grp. gram.:V. i.}
\end{itemize}
\begin{itemize}
\item {Utilização:Pop.}
\end{itemize}
\begin{itemize}
\item {Proveniência:(De \textunderscore des...\textunderscore  + \textunderscore travar\textunderscore )}
\end{itemize}
Desligar do travão.
Desapertar (aquillo que estava travado).
Evacuar excrementos.
\section{Destrelar}
\begin{itemize}
\item {Grp. gram.:v. t.}
\end{itemize}
O mesmo que \textunderscore desatrelar\textunderscore .
\section{Destrenga-Deus}
\begin{itemize}
\item {Grp. gram.:loc. interj.}
\end{itemize}
\begin{itemize}
\item {Utilização:Ant.}
\end{itemize}
Permitta Deus.
Oxala.
\section{Destrepar}
\begin{itemize}
\item {Grp. gram.:v. i.}
\end{itemize}
\begin{itemize}
\item {Proveniência:(De \textunderscore des...\textunderscore  + \textunderscore trepar\textunderscore )}
\end{itemize}
Descer do lugar, aonde tinha trepado.
\section{Destreza}
\begin{itemize}
\item {Grp. gram.:f.}
\end{itemize}
Qualidade de quem é destro^1.
\section{Destribar-se}
\begin{itemize}
\item {Grp. gram.:v. p.}
\end{itemize}
\begin{itemize}
\item {Proveniência:(De \textunderscore de...\textunderscore  + \textunderscore estribar\textunderscore )}
\end{itemize}
Perder os estribos, o apoio.
\section{Destridade}
\begin{itemize}
\item {Grp. gram.:f.}
\end{itemize}
O mesmo que \textunderscore destreza\textunderscore .
\section{Destrigar}
\begin{itemize}
\item {Grp. gram.:v. t.}
\end{itemize}
\begin{itemize}
\item {Proveniência:(De \textunderscore de...\textunderscore  + \textunderscore estriga\textunderscore )}
\end{itemize}
Desfiar da estriga. Cf. Filinto, XII, 179.
\section{Destrinça}
\begin{itemize}
\item {Grp. gram.:f.}
\end{itemize}
\begin{itemize}
\item {Utilização:Prov.}
\end{itemize}
\begin{itemize}
\item {Utilização:dur.}
\end{itemize}
Acto de destrinçar.
Distribuição (das águas de rega) por vários prédios.
\section{Destrinçadamente}
\begin{itemize}
\item {Grp. gram.:adv.}
\end{itemize}
\begin{itemize}
\item {Proveniência:(De \textunderscore destrinçar\textunderscore )}
\end{itemize}
Minuciosamente.
\section{Destrinçador}
\begin{itemize}
\item {Grp. gram.:adj.}
\end{itemize}
\begin{itemize}
\item {Grp. gram.:M.}
\end{itemize}
Que destrinça.
Aquelle que destrinça.
\section{Destrinçar}
\begin{itemize}
\item {Grp. gram.:v. t.}
\end{itemize}
\begin{itemize}
\item {Proveniência:(Do lat. hyp. \textunderscore strinctiare\textunderscore , de \textunderscore strinctus\textunderscore , por \textunderscore strictus\textunderscore )}
\end{itemize}
Individualizar.
Expor minuciosamente: \textunderscore destrinçar uma questão\textunderscore .
Desenredar.
Dividir proporcionalmente um fôro por.
\section{Destrincar}
\textunderscore v. t.\textunderscore  (e der.)
O mesmo que \textunderscore estrincar\textunderscore ^1, etc.
\section{Destrincar}
\begin{itemize}
\item {Grp. gram.:v.}
\end{itemize}
\begin{itemize}
\item {Utilização:t. Náut.}
\end{itemize}
\begin{itemize}
\item {Proveniência:(De \textunderscore des...\textunderscore  + \textunderscore trincar\textunderscore ^1)}
\end{itemize}
Tirar as trincas a.
Desconcertar, desapparelhar:«\textunderscore hum temporal lhes destrincou os baixéis\textunderscore ». Filinto, \textunderscore D. Man.\textunderscore , III, 145.
\section{Destrinçável}
\begin{itemize}
\item {Grp. gram.:adj.}
\end{itemize}
Que se póde destrinçar.
\section{Destripar}
\begin{itemize}
\item {Grp. gram.:v. t.}
\end{itemize}
O mesmo que \textunderscore estripar\textunderscore .
\section{Destripular}
\begin{itemize}
\item {Grp. gram.:v. t.}
\end{itemize}
\begin{itemize}
\item {Proveniência:(De \textunderscore des...\textunderscore  + \textunderscore tripular\textunderscore )}
\end{itemize}
Tirar a tripulação a.
\section{Destristecer}
\begin{itemize}
\item {Grp. gram.:v. t.}
\end{itemize}
\begin{itemize}
\item {Grp. gram.:V. i.}
\end{itemize}
\begin{itemize}
\item {Proveniência:(De \textunderscore des...\textunderscore  + \textunderscore triste\textunderscore )}
\end{itemize}
Tirar a tristeza a.
Deixar de estar triste. Cf. Filinto, V, 305.
\section{Destro}
\begin{itemize}
\item {Grp. gram.:adj.}
\end{itemize}
\begin{itemize}
\item {Proveniência:(Do lat. \textunderscore dester\textunderscore )}
\end{itemize}
Direito.
Que fica ao lado direito.
Perito.
Agil.
Sagaz.
\section{Destro}
\begin{itemize}
\item {Grp. gram.:m.}
\end{itemize}
Animálculo, o mesmo que berro^2.
\section{Destro}
\begin{itemize}
\item {Grp. gram.:m.}
\end{itemize}
\begin{itemize}
\item {Utilização:Ant.}
\end{itemize}
O mesmo que \textunderscore passal\textunderscore .
\section{Dessudação}
\begin{itemize}
\item {Grp. gram.:f.}
\end{itemize}
\begin{itemize}
\item {Proveniência:(Lat. \textunderscore desudatio\textunderscore )}
\end{itemize}
Acto de suar muito.
\section{Dessultor}
\begin{itemize}
\item {Grp. gram.:m.}
\end{itemize}
\begin{itemize}
\item {Proveniência:(Lat. \textunderscore desultor\textunderscore )}
\end{itemize}
Cavalleiro romano, que, nos jogos públicos, saltava de um cavallo para outro.
\section{Dessultório}
\begin{itemize}
\item {Grp. gram.:adj.}
\end{itemize}
\begin{itemize}
\item {Proveniência:(Lat. \textunderscore desultorius\textunderscore )}
\end{itemize}
Que salta de um lado para outro; que volteia:«\textunderscore nestas desultórias conversações...\textunderscore »Garrett, \textunderscore Arco\textunderscore , I, 122.
\section{Dessumir}
\begin{itemize}
\item {Grp. gram.:v. t.}
\end{itemize}
\begin{itemize}
\item {Utilização:Des.}
\end{itemize}
\begin{itemize}
\item {Proveniência:(Lat. \textunderscore desumere\textunderscore )}
\end{itemize}
Inferir, deduzir.
\section{Destiranizar}
\begin{itemize}
\item {Grp. gram.:v. t.}
\end{itemize}
\begin{itemize}
\item {Proveniência:(De \textunderscore des...\textunderscore  + \textunderscore tiranizar\textunderscore )}
\end{itemize}
Livrar da tirania. Cf. Filinto, \textunderscore D. Man.\textunderscore , II, 31.
\section{Destroca}
\begin{itemize}
\item {Grp. gram.:f.}
\end{itemize}
Acto ou effeito de destrocar.
\section{Destroçador}
\begin{itemize}
\item {Grp. gram.:adj.}
\end{itemize}
\begin{itemize}
\item {Grp. gram.:M.}
\end{itemize}
\begin{itemize}
\item {Proveniência:(De \textunderscore destroçar\textunderscore )}
\end{itemize}
Que destroça.
Aquelle que destroça.
Aquelle que cresta as colmeias.
\section{Destrocar}
\begin{itemize}
\item {Grp. gram.:v. t.}
\end{itemize}
\begin{itemize}
\item {Proveniência:(De \textunderscore des...\textunderscore  + \textunderscore trocar\textunderscore )}
\end{itemize}
Desmanchar a troca de.
\section{Destroçar}
\begin{itemize}
\item {Grp. gram.:v. t.}
\end{itemize}
\begin{itemize}
\item {Proveniência:(Do lat. hyp. \textunderscore destructiare\textunderscore )}
\end{itemize}
Pôr em debandada.
Dividir em troços.
Dispersar.
Despedaçar.
Desbaratar: \textunderscore destroçar um exército\textunderscore .
\section{Destrôço}
\begin{itemize}
\item {Grp. gram.:m.}
\end{itemize}
Acto ou effeito de destroçar.
Coisas partidas.
Ruínas.
\section{Destronar}
\begin{itemize}
\item {Grp. gram.:v. t.}
\end{itemize}
\begin{itemize}
\item {Utilização:Fig.}
\end{itemize}
\begin{itemize}
\item {Proveniência:(De \textunderscore des...\textunderscore  + \textunderscore trono\textunderscore )}
\end{itemize}
Tirar do trono.
Destituir da soberania.
Abater, humilhar.
\section{Destroncar}
\begin{itemize}
\item {Grp. gram.:v. t.}
\end{itemize}
Separar do tronco.
Decepar; desmembrar.
Separar os membros de.
\section{Destronização}
\begin{itemize}
\item {Grp. gram.:f.}
\end{itemize}
Acto ou efeito de destronizar.
\section{Destronizar}
\begin{itemize}
\item {Grp. gram.:v. t.}
\end{itemize}
O mesmo que \textunderscore destronar\textunderscore .
\section{Destronque}
\begin{itemize}
\item {Grp. gram.:m.}
\end{itemize}
\begin{itemize}
\item {Proveniência:(De \textunderscore destroncar\textunderscore )}
\end{itemize}
Perturbação, soffrida pelo toiro, depois de rabejado ou passado de muleta.
\section{Destronquecido}
\begin{itemize}
\item {Grp. gram.:adj.}
\end{itemize}
\begin{itemize}
\item {Proveniência:(De \textunderscore des...\textunderscore  + \textunderscore tronco\textunderscore )}
\end{itemize}
Diz-se dos vegetaes que não têm tronco algum. Cf. A. A. F. Benevides, \textunderscore Glossologia\textunderscore .
\section{Destructibilidade}
\begin{itemize}
\item {Grp. gram.:f.}
\end{itemize}
\begin{itemize}
\item {Proveniência:(Do lat. \textunderscore destructibilis\textunderscore )}
\end{itemize}
Qualidade daquillo que é destructível.
\section{Destructivamente}
\begin{itemize}
\item {Grp. gram.:adv.}
\end{itemize}
De modo destructivo.
Fazendo destruição.
\section{Destructível}
\begin{itemize}
\item {Grp. gram.:adj.}
\end{itemize}
\begin{itemize}
\item {Proveniência:(Do lat. \textunderscore destructibilis\textunderscore )}
\end{itemize}
Que póde sêr destruído.
\section{Destructivo}
\begin{itemize}
\item {Grp. gram.:adj.}
\end{itemize}
\begin{itemize}
\item {Proveniência:(Lat. \textunderscore destructivus\textunderscore )}
\end{itemize}
Que destrói.
\section{Destructor}
\begin{itemize}
\item {Grp. gram.:m.  e  adj.}
\end{itemize}
\begin{itemize}
\item {Proveniência:(Lat. \textunderscore destructor\textunderscore )}
\end{itemize}
O mesmo que \textunderscore destruidor\textunderscore .
\section{Destruição}
\begin{itemize}
\item {fónica:tru-i}
\end{itemize}
\begin{itemize}
\item {Grp. gram.:f.}
\end{itemize}
\begin{itemize}
\item {Proveniência:(Do lat. \textunderscore destructio\textunderscore )}
\end{itemize}
Acto ou effeito de destruir.
\section{Destruidor}
\begin{itemize}
\item {fónica:tru-i}
\end{itemize}
\begin{itemize}
\item {Grp. gram.:adj.}
\end{itemize}
\begin{itemize}
\item {Grp. gram.:M.}
\end{itemize}
Que destrói.
Aquelle que destrói.
Moderno navio de guerra, próprio para destruir torpedos.
\section{Destruir}
\begin{itemize}
\item {Grp. gram.:v. t.}
\end{itemize}
\begin{itemize}
\item {Proveniência:(Lat. \textunderscore destruere\textunderscore )}
\end{itemize}
Desfazer, desmanchar.
Demolir: \textunderscore destruir um prédio\textunderscore .
Assolar.
Fazer desapparecer, exterminar: \textunderscore destruir uma raça\textunderscore .
Desbaratar: \textunderscore destruir um exército\textunderscore .
\section{Destrunfar}
\begin{itemize}
\item {Grp. gram.:v. t.}
\end{itemize}
\begin{itemize}
\item {Proveniência:(De \textunderscore des...\textunderscore  + \textunderscore trunfar\textunderscore )}
\end{itemize}
Obrigar a jogar trunfo.
\section{Destrutibilidade}
\begin{itemize}
\item {Grp. gram.:f.}
\end{itemize}
\begin{itemize}
\item {Proveniência:(Do lat. \textunderscore destructibilis\textunderscore )}
\end{itemize}
Qualidade daquilo que é destrutível.
\section{Destrutivamente}
\begin{itemize}
\item {Grp. gram.:adv.}
\end{itemize}
De modo destrutivo.
Fazendo destruição.
\section{Destrutível}
\begin{itemize}
\item {Grp. gram.:adj.}
\end{itemize}
\begin{itemize}
\item {Proveniência:(Do lat. \textunderscore destructibilis\textunderscore )}
\end{itemize}
Que póde sêr destruído.
\section{Destrutor}
\begin{itemize}
\item {Grp. gram.:m.  e  adj.}
\end{itemize}
\begin{itemize}
\item {Proveniência:(Lat. \textunderscore destructor\textunderscore )}
\end{itemize}
O mesmo que \textunderscore destruidor\textunderscore .
\section{Desturvar}
\begin{itemize}
\item {Proveniência:(De \textunderscore des...\textunderscore  + \textunderscore turvo\textunderscore )}
\end{itemize}
\textunderscore v. t.\textunderscore  (e der.)
O mesmo que \textunderscore desenturvar\textunderscore , etc.
\section{Destyrannizar}
\begin{itemize}
\item {Grp. gram.:v. t.}
\end{itemize}
\begin{itemize}
\item {Proveniência:(De \textunderscore des...\textunderscore  + \textunderscore tyrannizar\textunderscore )}
\end{itemize}
Livrar da tyrannia. Cf. Filinto, \textunderscore D. Man.\textunderscore , II, 31.
\section{Desudação}
\begin{itemize}
\item {fónica:su}
\end{itemize}
\begin{itemize}
\item {Grp. gram.:f.}
\end{itemize}
\begin{itemize}
\item {Proveniência:(Lat. \textunderscore desudatio\textunderscore )}
\end{itemize}
Acto de suar muito.
\section{Desultor}
\begin{itemize}
\item {fónica:sul}
\end{itemize}
\begin{itemize}
\item {Grp. gram.:m.}
\end{itemize}
\begin{itemize}
\item {Proveniência:(Lat. \textunderscore desultor\textunderscore )}
\end{itemize}
Cavalleiro romano, que, nos jogos públicos, saltava de um cavallo para outro.
\section{Desultório}
\begin{itemize}
\item {fónica:sul}
\end{itemize}
\begin{itemize}
\item {Grp. gram.:adj.}
\end{itemize}
\begin{itemize}
\item {Proveniência:(Lat. \textunderscore desultorius\textunderscore )}
\end{itemize}
Que salta de um lado para outro; que volteia:«\textunderscore nestas desultórias conversações...\textunderscore »Garrett, \textunderscore Arco\textunderscore , I, 122.
\section{Desultrajar-se}
\begin{itemize}
\item {Grp. gram.:v. p.}
\end{itemize}
\begin{itemize}
\item {Proveniência:(De \textunderscore des...\textunderscore  + \textunderscore ultraje\textunderscore )}
\end{itemize}
Desaggravar-se, desforçar-se.
\section{Desumanamente}
\begin{itemize}
\item {Grp. gram.:adv.}
\end{itemize}
De modo desumano.
\section{Desumanar}
\begin{itemize}
\item {Grp. gram.:v. t.}
\end{itemize}
\begin{itemize}
\item {Utilização:Des.}
\end{itemize}
\begin{itemize}
\item {Proveniência:(De \textunderscore des...\textunderscore  + \textunderscore humanar\textunderscore )}
\end{itemize}
Tornar desumano.
\section{Desumanidade}
\begin{itemize}
\item {Grp. gram.:f.}
\end{itemize}
Acto desumano.
Falta de humanidade.
\section{Desumanizar}
\begin{itemize}
\item {Grp. gram.:v. t.}
\end{itemize}
O mesmo que \textunderscore desumanar\textunderscore .
\section{Desumano}
\begin{itemize}
\item {Grp. gram.:adj.}
\end{itemize}
Que não é humano.
Que é ferino, bestial, cruel.
\section{Desumir}
\begin{itemize}
\item {fónica:su}
\end{itemize}
\begin{itemize}
\item {Grp. gram.:v. t.}
\end{itemize}
\begin{itemize}
\item {Utilização:Des.}
\end{itemize}
\begin{itemize}
\item {Proveniência:(Lat. \textunderscore desumere\textunderscore )}
\end{itemize}
Inferir, deduzir.
\section{Desunhar}
\begin{itemize}
\item {Grp. gram.:v. t.}
\end{itemize}
\begin{itemize}
\item {Grp. gram.:V. p.}
\end{itemize}
\begin{itemize}
\item {Utilização:Prov.}
\end{itemize}
\begin{itemize}
\item {Proveniência:(De \textunderscore des...\textunderscore  + \textunderscore unha\textunderscore )}
\end{itemize}
Arrancar as unhas a.
Fatigar.
Fazer andar muito.
Fazer qualquer coisa desembaraçadamente: \textunderscore e come, que se desunha\textunderscore !
\section{Desunião}
\begin{itemize}
\item {Grp. gram.:f.}
\end{itemize}
Acto ou effeito de desunir.
Desavença.
Rixa.
\section{Desunidamente}
\begin{itemize}
\item {Grp. gram.:adv.}
\end{itemize}
\begin{itemize}
\item {Proveniência:(De \textunderscore desunir\textunderscore )}
\end{itemize}
Sem união.
\section{Desunificar}
\begin{itemize}
\item {Grp. gram.:v. t.}
\end{itemize}
\begin{itemize}
\item {Proveniência:(De \textunderscore des...\textunderscore  + \textunderscore unificar\textunderscore )}
\end{itemize}
Tirar a unificação a.
\section{Desunir}
\begin{itemize}
\item {Grp. gram.:v. t.}
\end{itemize}
\begin{itemize}
\item {Proveniência:(De \textunderscore des...\textunderscore  + \textunderscore unir\textunderscore )}
\end{itemize}
Desfazer a união de.
Desligar.
Desmembrar.
Produzir discórdia entre: \textunderscore desunir irmãos\textunderscore .
\section{Desurdir}
\begin{itemize}
\item {Grp. gram.:v. t.}
\end{itemize}
\begin{itemize}
\item {Proveniência:(De \textunderscore des...\textunderscore  + \textunderscore urdir\textunderscore )}
\end{itemize}
Desmanchar a urdidura de; desmanchar; desunir:«\textunderscore matérias que por froixas embaraçassem as balas sem desurdir-se\textunderscore ». Filinto, \textunderscore D. Man.\textunderscore , II, 248.
\section{Desusadamente}
\begin{itemize}
\item {Grp. gram.:adv.}
\end{itemize}
\begin{itemize}
\item {Proveniência:(De \textunderscore desusar\textunderscore )}
\end{itemize}
Fóra do uso.
\section{Desusança}
\begin{itemize}
\item {Grp. gram.:f.}
\end{itemize}
\begin{itemize}
\item {Utilização:Ant.}
\end{itemize}
O mesmo que \textunderscore desuso\textunderscore .
\section{Desusar}
\begin{itemize}
\item {Grp. gram.:v. t.}
\end{itemize}
\begin{itemize}
\item {Grp. gram.:V. p.}
\end{itemize}
\begin{itemize}
\item {Proveniência:(De \textunderscore des...\textunderscore  + \textunderscore usar\textunderscore )}
\end{itemize}
Não usar.
Estar fóra do uso.
\section{Desuso}
\begin{itemize}
\item {Grp. gram.:m.}
\end{itemize}
Acto ou effeito de desusar.
\section{Desútil}
\begin{itemize}
\item {Grp. gram.:adj.}
\end{itemize}
O mesmo que \textunderscore inútil\textunderscore . Cf. \textunderscore Panorama\textunderscore , VII, 1.
\section{Desvaecer}
\begin{itemize}
\item {fónica:va-e}
\end{itemize}
\begin{itemize}
\item {Grp. gram.:v. i.}
\end{itemize}
Esvaecer-se; desvanecer-se.
Desapparecer. Cf. Filinto, \textunderscore D. Man.\textunderscore , III, 230.
(Cp. \textunderscore desvanecer\textunderscore )
\section{Desvaidade}
\begin{itemize}
\item {Grp. gram.:f.}
\end{itemize}
Ausência de vaidade; modéstia.
\section{Desvaír-se}
\begin{itemize}
\item {Grp. gram.:v. p.}
\end{itemize}
(V. [[esvaír-se|esvair]])
\section{Desvairadamente}
\begin{itemize}
\item {Grp. gram.:adv.}
\end{itemize}
De modo desvairado.
\section{Desvairado}
\begin{itemize}
\item {Grp. gram.:adj.}
\end{itemize}
\begin{itemize}
\item {Utilização:Ant.}
\end{itemize}
\begin{itemize}
\item {Grp. gram.:M.}
\end{itemize}
\begin{itemize}
\item {Proveniência:(De \textunderscore desvairar\textunderscore )}
\end{itemize}
Que perdeu o juízo.
Alucinado, desnorteado.
Vário, diverso: \textunderscore andou por desvairados caminhos\textunderscore .
Estroina; valdevinos.
\section{Desvairador}
\begin{itemize}
\item {Grp. gram.:adj.}
\end{itemize}
Que produz desvario.
Que entontece.
Que faz perder o tino.
\section{Desvairamento}
\begin{itemize}
\item {Grp. gram.:m.}
\end{itemize}
Acto ou effeito de desvairar.
\section{Desvairança}
\begin{itemize}
\item {Grp. gram.:f.}
\end{itemize}
\begin{itemize}
\item {Utilização:Ant.}
\end{itemize}
\begin{itemize}
\item {Proveniência:(De \textunderscore desvairar\textunderscore )}
\end{itemize}
O mesmo que \textunderscore desvario\textunderscore .
Differença, diversidade.
\section{Desvairar}
\begin{itemize}
\item {Grp. gram.:v. t.}
\end{itemize}
\begin{itemize}
\item {Grp. gram.:V. i.}
\end{itemize}
\begin{itemize}
\item {Utilização:Ant.}
\end{itemize}
Tornar alucinado.
Endoidecer.
Dar maus conselhos a.
Discordar.
Desencaminhar-se.
Desorientar-se.
Alucinar-se.
Desvariar.
Variar, diversificar.
(Methát. de \textunderscore desvariar\textunderscore )
\section{Desvaire}
\begin{itemize}
\item {Grp. gram.:m.}
\end{itemize}
Acto de desvairar.
Desvairo. Cf. Filinto, IX, 282.
\section{Desvairo}
\begin{itemize}
\item {Grp. gram.:m.}
\end{itemize}
\begin{itemize}
\item {Utilização:Ant.}
\end{itemize}
\begin{itemize}
\item {Proveniência:(De \textunderscore desvairar\textunderscore )}
\end{itemize}
Discordância, desunião.
\section{Desvalado}
\begin{itemize}
\item {Grp. gram.:adj.}
\end{itemize}
\begin{itemize}
\item {Proveniência:(De \textunderscore des...\textunderscore  + \textunderscore valado\textunderscore )}
\end{itemize}
Que não é fechado por fossos ou valados.
Raso, franqueado, (falando-se de terrenos).
\section{Desvalente}
\begin{itemize}
\item {Grp. gram.:adj.}
\end{itemize}
Que desvale, que não ampara. Cf. Filinto, X, 122.
\section{Desvaler}
\begin{itemize}
\item {Grp. gram.:v. t.}
\end{itemize}
\begin{itemize}
\item {Grp. gram.:V. i.}
\end{itemize}
Não valer a; não acudir a.
Não dar amparo a.
Não valer.
\section{Desvalia}
\begin{itemize}
\item {Grp. gram.:f.}
\end{itemize}
O mesmo que \textunderscore desvalimento\textunderscore .
\section{Desvaliar}
\begin{itemize}
\item {Grp. gram.:v. t.}
\end{itemize}
\begin{itemize}
\item {Proveniência:(De \textunderscore des...\textunderscore  + \textunderscore valia\textunderscore )}
\end{itemize}
Tirar ou deminuir a valia a.
Depreciar.
\section{Desvalidar}
\textunderscore v. t.\textunderscore  (e der.)
O mesmo que \textunderscore invalidar\textunderscore , etc. Cf. Filinto, XXI, 235.
\section{Desvalido}
\begin{itemize}
\item {Grp. gram.:m.}
\end{itemize}
\begin{itemize}
\item {Proveniência:(De \textunderscore desvaler\textunderscore )}
\end{itemize}
Aquelle, que não tem valimento.
Homem desgraçado, miserável.
\section{Desvalijar}
\begin{itemize}
\item {Grp. gram.:v. t.}
\end{itemize}
Roubar a mala ou os alforges a.
Despojar; roubar. Cf. Bernárdez, \textunderscore Confiança em Deus\textunderscore , c. 57.
(Cast. \textunderscore desbalijar\textunderscore )
\section{Desvalimento}
\begin{itemize}
\item {Grp. gram.:m.}
\end{itemize}
Falta de valimento.
\section{Desvalioso}
\begin{itemize}
\item {Grp. gram.:adj.}
\end{itemize}
Que não tem valia.
Não valioso.
\section{Desvalisar}
\begin{itemize}
\item {Grp. gram.:v. t.}
\end{itemize}
\begin{itemize}
\item {Utilização:Gal}
\end{itemize}
(V.desvalijar)
\section{Desvallado}
\begin{itemize}
\item {Grp. gram.:adj.}
\end{itemize}
\begin{itemize}
\item {Proveniência:(De \textunderscore des...\textunderscore  + \textunderscore vallado\textunderscore )}
\end{itemize}
Que não é fechado por fossos ou vallados.
Raso, franqueado, (falando-se de terrenos).
\section{Desvalor}
\begin{itemize}
\item {Grp. gram.:m.}
\end{itemize}
Falta de valor.
\section{Desvalorização}
\begin{itemize}
\item {Grp. gram.:f.}
\end{itemize}
Acto de desvalorizar.
\section{Desvalorizar}
\begin{itemize}
\item {Grp. gram.:v. t.}
\end{itemize}
\begin{itemize}
\item {Proveniência:(De \textunderscore des...\textunderscore  + \textunderscore valorizar\textunderscore )}
\end{itemize}
Tirar o valor a.
Depreciar.
\section{Desvalvulado}
\begin{itemize}
\item {Grp. gram.:adj.}
\end{itemize}
Que não tem válvulas.
\section{Desvanecedor}
\begin{itemize}
\item {Grp. gram.:m.  e  adj.}
\end{itemize}
Aquelle ou aquillo que desvanece: \textunderscore elogios desvanecedores\textunderscore .
\section{Desvanecer}
\begin{itemize}
\item {Grp. gram.:v. t.}
\end{itemize}
\begin{itemize}
\item {Grp. gram.:V. p.}
\end{itemize}
\begin{itemize}
\item {Proveniência:(Do lat. \textunderscore de\textunderscore  + \textunderscore vanescere\textunderscore )}
\end{itemize}
Fazer desapparecer.
Expungir; apagar: \textunderscore o tempo desvanece mágoas\textunderscore .
Causar vaidade a.
Tornar orgulhoso.
Frustar.
Esmorecer; desapparecer: \textunderscore desvaneceu-se a última illusão\textunderscore .
\section{Desvanecidamente}
\begin{itemize}
\item {Grp. gram.:adv.}
\end{itemize}
De modo desvanecido.
Com desvanecimento.
\section{Desvanecido}
\begin{itemize}
\item {Grp. gram.:adj.}
\end{itemize}
Dissipado, desfeito.
Apagado.
Desmaiado, desbotado.
Presumido, vaidoso.
\section{Desvanecimento}
\begin{itemize}
\item {Grp. gram.:m.}
\end{itemize}
Acto ou effeito de desvanecer.
Satisfação, que alguém tem, de si próprio.
Vaidade.
\section{Desvanecível}
\begin{itemize}
\item {Grp. gram.:adj.}
\end{itemize}
Que se póde desvanecer.
\section{Desvaneio}
\begin{itemize}
\item {Grp. gram.:m.}
\end{itemize}
O mesmo que \textunderscore devaneio\textunderscore . Cf. Herculano, \textunderscore Quest. Pub.\textunderscore , I, 142.
\section{Desvantagem}
\begin{itemize}
\item {Grp. gram.:f.}
\end{itemize}
Falta de vantagem.
\section{Desvantajosamente}
\begin{itemize}
\item {Grp. gram.:adv.}
\end{itemize}
\begin{itemize}
\item {Proveniência:(De \textunderscore desvantajoso\textunderscore )}
\end{itemize}
Com desvantagem.
\section{Desvantajoso}
\begin{itemize}
\item {Grp. gram.:adj.}
\end{itemize}
\begin{itemize}
\item {Proveniência:(De \textunderscore des...\textunderscore  + \textunderscore vantajoso\textunderscore )}
\end{itemize}
Em que não há vantagem; prejudicial.
\section{Desvão}
\begin{itemize}
\item {Grp. gram.:m.}
\end{itemize}
\begin{itemize}
\item {Proveniência:(De \textunderscore des...\textunderscore  + \textunderscore vão\textunderscore )}
\end{itemize}
Espaço entre o telhado e o fôrro de uma casa.
Pavimento superior de uma casa.
Recanto; esconderijo; esconso.
\section{Desvarar}
\begin{itemize}
\item {Grp. gram.:v.}
\end{itemize}
\begin{itemize}
\item {Utilização:t. Náut.}
\end{itemize}
\begin{itemize}
\item {Proveniência:(De \textunderscore des...\textunderscore  + \textunderscore varar\textunderscore )}
\end{itemize}
O mesmo que \textunderscore desencalhar\textunderscore  (embarcação):«\textunderscore os três outros vasos portugueses, como não foi possível desvarallos, deitou-lhes fogo\textunderscore ». Filinto, \textunderscore D. Man.\textunderscore , I, 262.
\section{Desvariar}
\begin{itemize}
\item {Grp. gram.:v. t.}
\end{itemize}
\begin{itemize}
\item {Grp. gram.:V. i.}
\end{itemize}
\begin{itemize}
\item {Proveniência:(De \textunderscore des...\textunderscore  + \textunderscore variar\textunderscore )}
\end{itemize}
Fazer caír em desvario.
Delirar; desvairar.
\section{Desvaricado}
\begin{itemize}
\item {Grp. gram.:adj.}
\end{itemize}
\begin{itemize}
\item {Utilização:Bot.}
\end{itemize}
\begin{itemize}
\item {Proveniência:(Do lat. \textunderscore divaricatus\textunderscore )}
\end{itemize}
Diz-se do tronco, quando, pouco acima da raiz, se divide em muitos ramos, que formam entre si ângulos obtusos.
\section{Desvario}
\begin{itemize}
\item {Grp. gram.:m.}
\end{itemize}
\begin{itemize}
\item {Proveniência:(De \textunderscore desvariar\textunderscore )}
\end{itemize}
Delírio.
Acto de loucura; extravagância.
\section{Desvassallar}
\begin{itemize}
\item {Grp. gram.:v. t.}
\end{itemize}
\begin{itemize}
\item {Proveniência:(De \textunderscore des...\textunderscore  + \textunderscore vassallo\textunderscore )}
\end{itemize}
Tirar a qualidade ou o estado de vassallo a. Cf. Filinto, \textunderscore D. Man.\textunderscore , III, 315.
\section{Desveladamente}
\begin{itemize}
\item {Grp. gram.:adv.}
\end{itemize}
\begin{itemize}
\item {Proveniência:(De \textunderscore desvelado\textunderscore ^1)}
\end{itemize}
Com desvelo.
Com o maior cuidado; dedicadamente.
\section{Desvelado}
\begin{itemize}
\item {Grp. gram.:adj.}
\end{itemize}
\begin{itemize}
\item {Proveniência:(De \textunderscore desvelar\textunderscore ^1)}
\end{itemize}
Cuidadoso; vigilante; estremoso: \textunderscore enfermeira desvelada\textunderscore .
\section{Desvelado}
\begin{itemize}
\item {Grp. gram.:adj.}
\end{itemize}
\begin{itemize}
\item {Proveniência:(De \textunderscore desvelar\textunderscore ^2)}
\end{itemize}
Não velado, patente, descoberto.
Manifestado, revelado.
Sereno, claro: \textunderscore atmosphera desvelada\textunderscore .
\section{Desvelar}
\begin{itemize}
\item {Grp. gram.:v. t.}
\end{itemize}
\begin{itemize}
\item {Grp. gram.:V. p.}
\end{itemize}
\begin{itemize}
\item {Proveniência:(De \textunderscore des...\textunderscore  + \textunderscore velar\textunderscore ^2)}
\end{itemize}
Causar vigília a.
Não deixar dormir: \textunderscore os cuidados desvelam-me\textunderscore .
Têr muito cuidado; diligenciar: \textunderscore desvelar-se por cumprir os seus deveres\textunderscore .
\section{Desvelar}
\begin{itemize}
\item {Grp. gram.:v. t.}
\end{itemize}
\begin{itemize}
\item {Proveniência:(De \textunderscore des...\textunderscore  + \textunderscore velar\textunderscore ^1)}
\end{itemize}
Tirar o véu a.
Descobrir.
Revelar; aclarar: \textunderscore desvelar um segrêdo\textunderscore .
\section{Desvelejar}
\begin{itemize}
\item {Grp. gram.:v. i.}
\end{itemize}
\begin{itemize}
\item {Proveniência:(De \textunderscore des...\textunderscore  + \textunderscore velejar\textunderscore )}
\end{itemize}
Navegar em direcção opposta á que se seguia.
Amainar as velas.
\section{Desvelo}
\begin{itemize}
\item {Grp. gram.:m.}
\end{itemize}
\begin{itemize}
\item {Proveniência:(De \textunderscore desvelar\textunderscore ^1)}
\end{itemize}
Acto de diligenciar.
Cuidado carinhoso.
Vigilância.
Dedicação: \textunderscore servir com desvelo\textunderscore .
O objecto dêsses sentimentos.
\section{Desvencilhar}
\begin{itemize}
\item {Grp. gram.:v. t.}
\end{itemize}
(V.desenvincilhar)
\section{Desvendar}
\begin{itemize}
\item {Grp. gram.:v. t.}
\end{itemize}
\begin{itemize}
\item {Proveniência:(De \textunderscore des...\textunderscore  + \textunderscore vendar\textunderscore )}
\end{itemize}
Tirar a venda dos olhos de.
Descobrir.
Revelar: \textunderscore desvendar uma intriga\textunderscore .
\section{Desveneração}
\begin{itemize}
\item {Grp. gram.:f.}
\end{itemize}
Falta de veneração.
Desacatamento.
\section{Desvenerar}
\begin{itemize}
\item {Grp. gram.:v. t.}
\end{itemize}
\begin{itemize}
\item {Proveniência:(De \textunderscore des...\textunderscore  + \textunderscore venerar\textunderscore )}
\end{itemize}
Desrespeitar, desacatar.
\section{Desvenosa}
\begin{itemize}
\item {Grp. gram.:adj. f.}
\end{itemize}
\begin{itemize}
\item {Utilização:Bot.}
\end{itemize}
\begin{itemize}
\item {Proveniência:(De \textunderscore des...\textunderscore  + \textunderscore venoso\textunderscore )}
\end{itemize}
Diz-se das fôlhas, que não têm veias ou nervuras.
\section{Desventoso}
\begin{itemize}
\item {Grp. gram.:adj.}
\end{itemize}
Que não é ventoso; sereno. Cf. Castilho, \textunderscore Geórgicas\textunderscore , 107.
\section{Desventrar}
\begin{itemize}
\item {Grp. gram.:v. t.}
\end{itemize}
\begin{itemize}
\item {Proveniência:(De \textunderscore des...\textunderscore  + \textunderscore ventre\textunderscore )}
\end{itemize}
Rasgar o ventre a.
Estripar.
\section{Desventura}
\begin{itemize}
\item {Grp. gram.:f.}
\end{itemize}
Falta de ventura.
Sorte má; desgraça.
\section{Desventuradamente}
\begin{itemize}
\item {Grp. gram.:adv.}
\end{itemize}
De modo desventurado.
Com desventura.
Desgraçadamente.
\section{Desventurado}
\begin{itemize}
\item {Grp. gram.:adj.}
\end{itemize}
\begin{itemize}
\item {Proveniência:(De \textunderscore desventurar\textunderscore )}
\end{itemize}
Infeliz, desgraçado.
\section{Desventurar}
\begin{itemize}
\item {Grp. gram.:v. t.}
\end{itemize}
\begin{itemize}
\item {Proveniência:(De \textunderscore desventura\textunderscore )}
\end{itemize}
Tornar infeliz; tirar a ventura a.
\section{Desventuroso}
\begin{itemize}
\item {Grp. gram.:adj.}
\end{itemize}
Que não é venturoso.
Infeliz, desgraçado.
\section{Desverde}
\begin{itemize}
\item {fónica:vêr}
\end{itemize}
\begin{itemize}
\item {Grp. gram.:adj.}
\end{itemize}
\begin{itemize}
\item {Proveniência:(De \textunderscore des...\textunderscore  + \textunderscore verde\textunderscore )}
\end{itemize}
Que perdeu o verdor, que deixou de sêr verde.
\section{Desverdecer}
\begin{itemize}
\item {Grp. gram.:v. i.}
\end{itemize}
\begin{itemize}
\item {Proveniência:(De \textunderscore des...\textunderscore  + \textunderscore verdecer\textunderscore )}
\end{itemize}
Perder a côr verde.
Murchar.
\section{Desvergonha}
\begin{itemize}
\item {Grp. gram.:f.}
\end{itemize}
Falta de vergonha, desaforo.
Descaramento.
Desfaçatez.
\section{Desvergonhamento}
\begin{itemize}
\item {Grp. gram.:m.}
\end{itemize}
O mesmo que \textunderscore desvergonha\textunderscore .
\section{Desvergonhar}
\textunderscore v. t.\textunderscore  (e der.)
O mesmo que \textunderscore desavergonhar\textunderscore , etc.
\section{Desvestir}
\begin{itemize}
\item {Grp. gram.:v. t.}
\end{itemize}
\begin{itemize}
\item {Proveniência:(De \textunderscore des...\textunderscore  + \textunderscore vestir\textunderscore )}
\end{itemize}
O mesmo que \textunderscore despir\textunderscore . Cf. Filinto,\textunderscore D. Man.\textunderscore , III, 241.
\section{Desvezado}
\begin{itemize}
\item {Grp. gram.:adj.}
\end{itemize}
\begin{itemize}
\item {Proveniência:(De \textunderscore des...\textunderscore  + \textunderscore vez\textunderscore )}
\end{itemize}
Desacostumado. Cf. Castilho, \textunderscore Sonho de Uma Noite de Verão\textunderscore .
\section{Desviar}
\begin{itemize}
\item {Grp. gram.:v. t.}
\end{itemize}
\begin{itemize}
\item {Proveniência:(Do b. lat. \textunderscore deviare\textunderscore )}
\end{itemize}
Tirar do caminho.
Afastar; deslocar.
Mudar a posição de.
Repellir: \textunderscore desviar accusações\textunderscore .
Desencaminhar: \textunderscore desviar fundos públicos\textunderscore .
Impedir.
Dissuadir.
Tornar divergente.
Alterar a direcção ou o destino de.
\section{Desvidrar-se}
\begin{itemize}
\item {Grp. gram.:v. p.}
\end{itemize}
\begin{itemize}
\item {Proveniência:(De \textunderscore des...\textunderscore  + \textunderscore vidrar\textunderscore )}
\end{itemize}
Deixar de ser vidrado.
Perder o brilho ou a transparência.
Deslustrar-se.
\section{Desvigar}
\begin{itemize}
\item {Grp. gram.:v. t.}
\end{itemize}
\begin{itemize}
\item {Proveniência:(De \textunderscore des...\textunderscore  + \textunderscore vigar\textunderscore )}
\end{itemize}
Tirar o vigamento a.
\section{Desvigiar}
\begin{itemize}
\item {Grp. gram.:v. t.}
\end{itemize}
Deixar de vigiar.
Desattender.
Não têr cuidado em.
\section{Desvigorar}
\begin{itemize}
\item {Grp. gram.:v. t.}
\end{itemize}
\begin{itemize}
\item {Grp. gram.:V. p.}
\end{itemize}
\begin{itemize}
\item {Proveniência:(De \textunderscore des...\textunderscore  + \textunderscore vigor\textunderscore )}
\end{itemize}
Tirar o vigor a.
Perder o vigor, enfraquecer. Cf. Camillo, \textunderscore Scenas da Hora Final\textunderscore , 20.
\section{Desvigorizar}
\textunderscore v. t.\textunderscore  (e der.)
O mesmo que \textunderscore desvigorar\textunderscore , etc. Cf. Camillo, \textunderscore Brasileira\textunderscore , 245.
\section{Desvincar}
\begin{itemize}
\item {Grp. gram.:v. t.}
\end{itemize}
\begin{itemize}
\item {Proveniência:(De \textunderscore des...\textunderscore  + \textunderscore vincar\textunderscore )}
\end{itemize}
Tirar os vincos a.
Desenrugar.
\section{Desvincilhar}
\begin{itemize}
\item {Grp. gram.:v. t.}
\end{itemize}
O mesmo que \textunderscore desenvincilhar\textunderscore .
\section{Desvinculação}
\begin{itemize}
\item {Grp. gram.:f.}
\end{itemize}
Acto de desvincular.
\section{Desvincular}
\begin{itemize}
\item {Grp. gram.:v. t.}
\end{itemize}
\begin{itemize}
\item {Proveniência:(De \textunderscore des...\textunderscore  + \textunderscore vincular\textunderscore )}
\end{itemize}
Desatar ou desligar (aquillo que estava vinculado).
Desligar.
Tornar alienável (bens que constituíam vínculo).
\section{Desvio}
\begin{itemize}
\item {Grp. gram.:m.}
\end{itemize}
Acto ou effeito de desviar.
Desvão.
Volta; rodeio: \textunderscore escondeu-se num desvio\textunderscore .
Falta, culpa.
Nos caminhos de ferro, linha secundária, ligada á linha geral, para abrigar ou depositar vehículos, etc.
\section{Desvirar}
\begin{itemize}
\item {Grp. gram.:v. t.}
\end{itemize}
\begin{itemize}
\item {Proveniência:(De \textunderscore des...\textunderscore  + \textunderscore virar\textunderscore )}
\end{itemize}
Voltar do avesso.
Voltar de dentro para fóra.
\section{Desvirgar}
\begin{itemize}
\item {Grp. gram.:v. t.}
\end{itemize}
\begin{itemize}
\item {Utilização:Pop.}
\end{itemize}
\begin{itemize}
\item {Proveniência:(De \textunderscore des...\textunderscore  + lat. \textunderscore virgo\textunderscore )}
\end{itemize}
Tirar a virgindade a; desflorar.
Deshonestar.
\section{Desvirginamento}
\begin{itemize}
\item {Grp. gram.:m.}
\end{itemize}
Acto de desvirginar.
\section{Desvirginar}
\begin{itemize}
\item {Grp. gram.:v. t.}
\end{itemize}
\begin{itemize}
\item {Proveniência:(De \textunderscore des...\textunderscore  + lat. \textunderscore virgo\textunderscore , \textunderscore virginis\textunderscore )}
\end{itemize}
Tirar a virgindade a. Cf. R. Ortigão, \textunderscore Hollanda\textunderscore , 73.
\section{Desvirginizar}
\begin{itemize}
\item {Grp. gram.:v. t.}
\end{itemize}
O mesmo que \textunderscore desvirginar\textunderscore .
Cf. Pacheco da Silva, \textunderscore Promptuário\textunderscore , 20.
\section{Desvirilizar}
\begin{itemize}
\item {Grp. gram.:v. t.}
\end{itemize}
\begin{itemize}
\item {Proveniência:(De \textunderscore des...\textunderscore  + \textunderscore viril\textunderscore )}
\end{itemize}
Tirar a virilidade a.
\section{Desvirtuação}
\begin{itemize}
\item {Grp. gram.:f.}
\end{itemize}
Acto de desvirtuar.
Depreciação.
\section{Desvirtuar}
\begin{itemize}
\item {Grp. gram.:v. t.}
\end{itemize}
\begin{itemize}
\item {Proveniência:(Do lat. \textunderscore de\textunderscore  + \textunderscore virtus\textunderscore )}
\end{itemize}
Depreciar a virtude de.
Tirar o mérito ou o prestígio a.
Julgar desfavoravelmente.
Tomar em mau sentido: \textunderscore desvirtuar intenções\textunderscore .
\section{Desvirtude}
\begin{itemize}
\item {Grp. gram.:f.}
\end{itemize}
Ausência de virtude.
Defeito.
Peccado.
\section{Desvirtuoso}
\begin{itemize}
\item {Grp. gram.:adj.}
\end{itemize}
Que não tem virtude; que não é virtuoso.
\section{Desvisgar}
\begin{itemize}
\item {Grp. gram.:v. t.}
\end{itemize}
Tirar o visgo a. Cf. G. Viana, \textunderscore Apostilas\textunderscore .
\section{Desvitrificação}
\begin{itemize}
\item {Grp. gram.:f.}
\end{itemize}
\begin{itemize}
\item {Utilização:Geol.}
\end{itemize}
\begin{itemize}
\item {Proveniência:(De \textunderscore des...\textunderscore  + \textunderscore vitrificação\textunderscore )}
\end{itemize}
Desenvolvimento espontâneo e lento da textura crystallina das rochas compostas que, quando resultam de um magma fundido, apresentam geralmente aspecto vítreo.
\section{Desviver}
\begin{itemize}
\item {Grp. gram.:v. t.}
\end{itemize}
\begin{itemize}
\item {Grp. gram.:V. i.}
\end{itemize}
\begin{itemize}
\item {Proveniência:(De \textunderscore des...\textunderscore  + \textunderscore viver\textunderscore )}
\end{itemize}
Tirar a vida a.
Deixar de viver; morrer.
\section{Desvizinhar}
\begin{itemize}
\item {Grp. gram.:v. i.}
\end{itemize}
Deixar de sêr vizinho; não sêr vizinho. Cf. Herculano, \textunderscore Hist. de Port.\textunderscore , IV, 86.
\section{Desvolvado}
\begin{itemize}
\item {Grp. gram.:adj.}
\end{itemize}
\begin{itemize}
\item {Proveniência:(De \textunderscore des...\textunderscore  + \textunderscore volva\textunderscore )}
\end{itemize}
Que não tem volva.
\section{Desxadrezar}
\begin{itemize}
\item {Grp. gram.:v. t.}
\end{itemize}
\begin{itemize}
\item {Proveniência:(De \textunderscore des...\textunderscore  + \textunderscore xadrez\textunderscore )}
\end{itemize}
Desmanchar o xadrez a.
Tirar a fôrma de xadrez a.
\section{Deszelar}
\begin{itemize}
\item {Grp. gram.:v. t.}
\end{itemize}
\begin{itemize}
\item {Proveniência:(De \textunderscore des...\textunderscore  + \textunderscore zelar\textunderscore )}
\end{itemize}
Não têr zelo, a respeito de.
Descurar, desvigiar.
\section{Detalhar}
\begin{itemize}
\item {Grp. gram.:v. t.}
\end{itemize}
\begin{itemize}
\item {Utilização:Gal}
\end{itemize}
\begin{itemize}
\item {Proveniência:(Fr. \textunderscore détailler\textunderscore )}
\end{itemize}
Narrar minuciosamente.
Particularizar.
Distribuir (serviços militares).
\section{Detalhe}
\begin{itemize}
\item {Grp. gram.:m.}
\end{itemize}
\begin{itemize}
\item {Utilização:inútil}
\end{itemize}
\begin{itemize}
\item {Utilização:Gal}
\end{itemize}
Acto ou effeito de detalhar.
Particularidade; minuciosidade.
\section{Detardar}
\begin{itemize}
\item {Grp. gram.:v. t.}
\end{itemize}
\begin{itemize}
\item {Utilização:Ant.}
\end{itemize}
O mesmo que \textunderscore retardar\textunderscore .
\section{Detectiva}
\begin{itemize}
\item {Grp. gram.:f.}
\end{itemize}
\begin{itemize}
\item {Utilização:Phot.}
\end{itemize}
\begin{itemize}
\item {Proveniência:(Do lat. \textunderscore detectus\textunderscore )}
\end{itemize}
Câmara escura especial.
\section{Detẽedor}
\begin{itemize}
\item {Grp. gram.:m.  e  adj.}
\end{itemize}
\begin{itemize}
\item {Utilização:Ant.}
\end{itemize}
O mesmo que \textunderscore detentor\textunderscore .
\section{Deteira}
\begin{itemize}
\item {Grp. gram.:f.}
\end{itemize}
Expressão?:«\textunderscore tornarei todalas vossas cantigas em deteiras lamentosas.\textunderscore »Usque, \textunderscore Tribulações\textunderscore , 16.
(Por \textunderscore diteira\textunderscore , de \textunderscore dito\textunderscore ?)
\section{Deteirar}
\begin{itemize}
\item {Grp. gram.:v. t.}
\end{itemize}
\begin{itemize}
\item {Proveniência:(De \textunderscore deteira\textunderscore )}
\end{itemize}
Exprimir por palavras? Cf. Usque, \textunderscore Tribulações\textunderscore , 48.
\section{Detença}
\begin{itemize}
\item {Grp. gram.:f.}
\end{itemize}
\begin{itemize}
\item {Proveniência:(De \textunderscore deter\textunderscore )}
\end{itemize}
Demora, delonga.
\section{Detenção}
\begin{itemize}
\item {Grp. gram.:f.}
\end{itemize}
Acto ou effeito de deter.
Prisão preventiva.
\section{Detençoso}
\begin{itemize}
\item {Grp. gram.:adj.}
\end{itemize}
\begin{itemize}
\item {Proveniência:(De \textunderscore detença\textunderscore )}
\end{itemize}
Que tem detença.
Que se detém:«\textunderscore ...com esperas mui detençosas.\textunderscore »\textunderscore Luz e Calor\textunderscore , 238. Cf. Camillo, \textunderscore Narcóticos\textunderscore , II, 291.
\section{Detentor}
\begin{itemize}
\item {Grp. gram.:m.}
\end{itemize}
\begin{itemize}
\item {Proveniência:(Lat. \textunderscore detentor\textunderscore )}
\end{itemize}
Aquelle que detém.
Depositário.
\section{Deter}
\begin{itemize}
\item {Grp. gram.:v. t.}
\end{itemize}
\begin{itemize}
\item {Grp. gram.:V. i.}
\end{itemize}
\begin{itemize}
\item {Proveniência:(Lat. \textunderscore detinere\textunderscore )}
\end{itemize}
Sustar.
Não deixar ir por deante: \textunderscore deter um cavallo\textunderscore .
Fazer cessar.
Reter em seu poder.
Demorar, dilatar, adiar.
Parar. Cf. Rui Barb., \textunderscore Réplica\textunderscore , 159.
\section{Detergente}
\begin{itemize}
\item {Grp. gram.:adj.}
\end{itemize}
\begin{itemize}
\item {Grp. gram.:M.}
\end{itemize}
\begin{itemize}
\item {Proveniência:(Lat. \textunderscore detergens\textunderscore )}
\end{itemize}
Que deterge.
Medicamento que deterge.
\section{Detergir}
\begin{itemize}
\item {Grp. gram.:v. t.}
\end{itemize}
\begin{itemize}
\item {Proveniência:(Lat. \textunderscore detergere\textunderscore )}
\end{itemize}
Limpar, enxugar.
\section{Deterioração}
\begin{itemize}
\item {Grp. gram.:f.}
\end{itemize}
Acto ou effeito de deteriorar.
Estrago.
Ruína.
\section{Deterioramento}
\begin{itemize}
\item {Grp. gram.:m.}
\end{itemize}
O mesmo que \textunderscore deterioração\textunderscore .
\section{Deteriorante}
\begin{itemize}
\item {Grp. gram.:adj.}
\end{itemize}
\begin{itemize}
\item {Proveniência:(Lat. \textunderscore deteriorans\textunderscore )}
\end{itemize}
Que deteriora.
\section{Deteriorar}
\begin{itemize}
\item {Grp. gram.:v. t.}
\end{itemize}
Damnificar.
Dissipar.
Alterar, adulterar: \textunderscore deteriorar uma bebida\textunderscore .
Estragar: \textunderscore deteriorar um jardim\textunderscore .
Arruinar.
Tornar degenerado.
(B. lat. \textunderscore deteriorare\textunderscore )
\section{Deteriorável}
\begin{itemize}
\item {Grp. gram.:adj.}
\end{itemize}
Que se póde deteriorar.
\section{Determinação}
\begin{itemize}
\item {Grp. gram.:f.}
\end{itemize}
Acto ou effeito de determinar.
Resolução.
Ordem superior.
\section{Determinadamente}
\begin{itemize}
\item {Grp. gram.:adv.}
\end{itemize}
Com determinação.
\section{Determinador}
\begin{itemize}
\item {Grp. gram.:adj.}
\end{itemize}
\begin{itemize}
\item {Grp. gram.:M.}
\end{itemize}
Que determina.
Aquelle que determina.
\section{Determinança}
\begin{itemize}
\item {Grp. gram.:f.}
\end{itemize}
\begin{itemize}
\item {Utilização:Ant.}
\end{itemize}
O mesmo que \textunderscore determinação\textunderscore .
\section{Determinante}
\begin{itemize}
\item {Grp. gram.:adj.}
\end{itemize}
\begin{itemize}
\item {Grp. gram.:F.}
\end{itemize}
\begin{itemize}
\item {Utilização:Mathem.}
\end{itemize}
Que determina.
Somma algébrica de todos os productos de vários factores, dispostos em linhas e columnas, que formam quadrado, tomando-se um elemento de cada columna e um de cada linha.
\section{Determinar}
\begin{itemize}
\item {Grp. gram.:v. t.}
\end{itemize}
\begin{itemize}
\item {Proveniência:(Lat. \textunderscore determinare\textunderscore )}
\end{itemize}
Marcar termo a.
Delimitar.
Fixar.
Indicar com precisão.
Resolver; ordenar: \textunderscore determinar serviços\textunderscore .
Motivar: \textunderscore determinar queixas\textunderscore .
Induzir.
Distinguir.
\section{Determinativo}
\begin{itemize}
\item {Grp. gram.:adj.}
\end{itemize}
\begin{itemize}
\item {Utilização:Gram.}
\end{itemize}
Que determina.
Que restringe.
Que restringe o nome: \textunderscore adjectivo determinativo\textunderscore .
\section{Determinável}
\begin{itemize}
\item {Grp. gram.:adj.}
\end{itemize}
Que se póde determinar.
\section{Determinhar}
\begin{itemize}
\item {Grp. gram.:v. t.}
\end{itemize}
\begin{itemize}
\item {Utilização:Ant.}
\end{itemize}
O mesmo que \textunderscore determinar\textunderscore .
\section{Determinismo}
\begin{itemize}
\item {Grp. gram.:m.}
\end{itemize}
\begin{itemize}
\item {Proveniência:(De \textunderscore determinar\textunderscore )}
\end{itemize}
Systema philosóphico, que subordinava as determinações humanas á acção providencial.
\section{Determinista}
\begin{itemize}
\item {Grp. gram.:m.}
\end{itemize}
\begin{itemize}
\item {Grp. gram.:Adj.}
\end{itemize}
\begin{itemize}
\item {Proveniência:(De \textunderscore determinar\textunderscore )}
\end{itemize}
Sectário do determinismo.
Relativo ao determinismo.
\section{Detersão}
\begin{itemize}
\item {Grp. gram.:f.}
\end{itemize}
\begin{itemize}
\item {Proveniência:(Do lat. \textunderscore detersus\textunderscore )}
\end{itemize}
Acto ou effeito de detergir.
\section{Detersivo}
\begin{itemize}
\item {Grp. gram.:adj.}
\end{itemize}
\begin{itemize}
\item {Proveniência:(Do lat. \textunderscore detersus\textunderscore )}
\end{itemize}
O mesmo que \textunderscore detergente\textunderscore .
\section{Detersório}
\begin{itemize}
\item {Grp. gram.:adj.}
\end{itemize}
O mesmo que \textunderscore detersivo\textunderscore .
\section{Detestação}
\begin{itemize}
\item {Grp. gram.:f.}
\end{itemize}
\begin{itemize}
\item {Proveniência:(Lat. \textunderscore detestatio\textunderscore )}
\end{itemize}
Acto de detestar.
Aversão, ódio.
\section{Detestando}
\begin{itemize}
\item {Grp. gram.:adj.}
\end{itemize}
\begin{itemize}
\item {Proveniência:(Lat. \textunderscore detestandus\textunderscore )}
\end{itemize}
O mesmo que \textunderscore detestável\textunderscore .
\section{Detestar}
\begin{itemize}
\item {Grp. gram.:v. t.}
\end{itemize}
\begin{itemize}
\item {Proveniência:(Lat. \textunderscore detestari\textunderscore )}
\end{itemize}
Repellir.
Têr aversão a.
Odiar; antipathizar com: \textunderscore sempre detestei lisonjas\textunderscore .
\section{Detestável}
\begin{itemize}
\item {Grp. gram.:adj.}
\end{itemize}
\begin{itemize}
\item {Proveniência:(Lat. \textunderscore detestabilis\textunderscore )}
\end{itemize}
Que se deve detestar.
Que merece detestação.
Péssimo.
Abominável.
\section{Detestavelmente}
\begin{itemize}
\item {Grp. gram.:adv.}
\end{itemize}
De modo detestável.
\section{Deteúdo}
\begin{itemize}
\item {Grp. gram.:adj.}
\end{itemize}
\begin{itemize}
\item {Proveniência:(De \textunderscore deter\textunderscore )}
\end{itemize}
Detido:«\textunderscore ...foram roubados e deteúdus\textunderscore ». Rui de Pina. \textunderscore Chrón. de Aff. V\textunderscore , c. CXXXI.
\section{Detidamente}
\begin{itemize}
\item {Grp. gram.:adv.}
\end{itemize}
\begin{itemize}
\item {Proveniência:(De \textunderscore detido\textunderscore )}
\end{itemize}
Com detença, demoradamente.
Minuciosamente.
\section{Detido}
\begin{itemize}
\item {Grp. gram.:adj.}
\end{itemize}
\begin{itemize}
\item {Proveniência:(De \textunderscore deter\textunderscore )}
\end{itemize}
Retardado, demorado.
Prêso provisoriamente.
\section{Detonação}
\begin{itemize}
\item {Grp. gram.:f.}
\end{itemize}
\begin{itemize}
\item {Proveniência:(Lat. \textunderscore detonatio\textunderscore )}
\end{itemize}
Ruído subitâneo, produzido por explosão.
Em balística, explosão de primeira ordem ou explosão da maior violência.
\section{Detonador}
\begin{itemize}
\item {Grp. gram.:m.}
\end{itemize}
\begin{itemize}
\item {Proveniência:(De \textunderscore detonar\textunderscore )}
\end{itemize}
Artifício, que provoca a detonação das cargas, nas peças de artilharia.
\section{Detonante}
\begin{itemize}
\item {Grp. gram.:adj.}
\end{itemize}
\begin{itemize}
\item {Proveniência:(Lat. \textunderscore detonans\textunderscore )}
\end{itemize}
Que detona.
\section{Detonar}
\begin{itemize}
\item {Grp. gram.:v. i.}
\end{itemize}
\begin{itemize}
\item {Proveniência:(Lat. \textunderscore detonare\textunderscore )}
\end{itemize}
Estrondear, explodindo.
\section{Detorar}
\begin{itemize}
\item {Grp. gram.:v. t.}
\end{itemize}
(V.destorar)
\section{Detracção}
\begin{itemize}
\item {Grp. gram.:f.}
\end{itemize}
\begin{itemize}
\item {Proveniência:(Lat. \textunderscore detractio\textunderscore )}
\end{itemize}
Acto de detrahir.
Depreciação; menosprêzo.
\section{Detractivo}
\begin{itemize}
\item {Grp. gram.:adj.}
\end{itemize}
\begin{itemize}
\item {Proveniência:(Do lat. \textunderscore detractus\textunderscore )}
\end{itemize}
Que detrái.
Desprezativo.
\section{Detractor}
\begin{itemize}
\item {Grp. gram.:m.}
\end{itemize}
\begin{itemize}
\item {Proveniência:(Lat. \textunderscore detractor\textunderscore )}
\end{itemize}
Aquelle que detrái, aquelle que diffama.
\section{Detrahidor}
\begin{itemize}
\item {Grp. gram.:m.}
\end{itemize}
O mesmo que \textunderscore detractor\textunderscore . Cf. Camillo, \textunderscore Quéda de Um Anjo\textunderscore , 80.
\section{Detrahir}
\begin{itemize}
\item {Grp. gram.:v. t.}
\end{itemize}
\begin{itemize}
\item {Proveniência:(Lat. \textunderscore detrahere\textunderscore )}
\end{itemize}
Abater o crédito de.
Infamar; depreciar.
\section{Detraidor}
\begin{itemize}
\item {fónica:tra-i}
\end{itemize}
\begin{itemize}
\item {Grp. gram.:m.}
\end{itemize}
O mesmo que \textunderscore detractor\textunderscore . Cf. Camillo, \textunderscore Quéda de Um Anjo\textunderscore , 80.
\section{Detrair}
\begin{itemize}
\item {Grp. gram.:v. t.}
\end{itemize}
\begin{itemize}
\item {Proveniência:(Lat. \textunderscore detrahere\textunderscore )}
\end{itemize}
Abater o crédito de.
Infamar; depreciar.
\section{Detrás}
\begin{itemize}
\item {Grp. gram.:adv.}
\end{itemize}
\begin{itemize}
\item {Proveniência:(De \textunderscore de\textunderscore  + \textunderscore trás\textunderscore )}
\end{itemize}
Na parte posterior.
Posteriormente, depois.
\section{Detrectar}
\begin{itemize}
\item {Grp. gram.:v. t.}
\end{itemize}
\begin{itemize}
\item {Utilização:Des.}
\end{itemize}
\begin{itemize}
\item {Proveniência:(Lat. \textunderscore detrectare\textunderscore )}
\end{itemize}
Recusar; renunciar. Cf. Filinto, \textunderscore D. Man.\textunderscore , I, 153.
\section{Detrição}
\begin{itemize}
\item {Grp. gram.:f.}
\end{itemize}
\begin{itemize}
\item {Proveniência:(Lat. \textunderscore detritio\textunderscore )}
\end{itemize}
Acto de desfazer ou gastar, por attrito.
\section{Detrimento}
\begin{itemize}
\item {Grp. gram.:m.}
\end{itemize}
\begin{itemize}
\item {Proveniência:(Lat. \textunderscore detrimentum\textunderscore )}
\end{itemize}
Perda; damno: \textunderscore proceder, com detrimento alheio\textunderscore .
\section{Detrimentoso}
\begin{itemize}
\item {Grp. gram.:adj.}
\end{itemize}
\begin{itemize}
\item {Utilização:Des.}
\end{itemize}
\begin{itemize}
\item {Proveniência:(Lat. \textunderscore detrimentosus\textunderscore )}
\end{itemize}
Que causa detrimento; prejudicial.
\section{Detrinçar}
\begin{itemize}
\item {Grp. gram.:v. t.}
\end{itemize}
(V.destrinçar)
\section{Detrito}
\begin{itemize}
\item {Grp. gram.:m.}
\end{itemize}
\begin{itemize}
\item {Proveniência:(Lat. \textunderscore detritus\textunderscore )}
\end{itemize}
Restos, resíduo de uma substância desorganizada.
\section{Detruncar}
\begin{itemize}
\item {Grp. gram.:v. t.}
\end{itemize}
\begin{itemize}
\item {Proveniência:(Lat. \textunderscore detruncare\textunderscore )}
\end{itemize}
O mesmo que \textunderscore truncar\textunderscore .
\section{Detrusão}
\begin{itemize}
\item {Grp. gram.:f.}
\end{itemize}
\begin{itemize}
\item {Utilização:Des.}
\end{itemize}
Acto de enclausurar por castigo, (no regime monástico).
(B. lat. \textunderscore detrusio\textunderscore )
\section{Detruso}
\begin{itemize}
\item {Grp. gram.:m.}
\end{itemize}
\begin{itemize}
\item {Utilização:Des.}
\end{itemize}
\begin{itemize}
\item {Proveniência:(Lat. \textunderscore detrusus\textunderscore )}
\end{itemize}
Aquelle que soffria a pena de detrusão.
\section{Detuanás}
\begin{itemize}
\item {Grp. gram.:m. pl.}
\end{itemize}
Índios selvagens das margens do Apaporis, no Brasil.
\section{Detumescência}
\begin{itemize}
\item {Grp. gram.:f.}
\end{itemize}
\begin{itemize}
\item {Proveniência:(Do lat. \textunderscore detumescere\textunderscore )}
\end{itemize}
Desinchação.
\section{Deturbação}
\begin{itemize}
\item {Grp. gram.:f.}
\end{itemize}
\begin{itemize}
\item {Proveniência:(Lat. \textunderscore deturbatio\textunderscore )}
\end{itemize}
Acto de deturbar.
Perturbação.
\section{Deturbar}
\begin{itemize}
\item {Grp. gram.:v. t.}
\end{itemize}
\begin{itemize}
\item {Utilização:Des.}
\end{itemize}
\begin{itemize}
\item {Proveniência:(Lat. \textunderscore deturbare\textunderscore )}
\end{itemize}
Agitar, perturbar.
Expellir, eliminar. Cf. Cortesão, \textunderscore Subs.\textunderscore 
\section{Deturpação}
\begin{itemize}
\item {Grp. gram.:f.}
\end{itemize}
Acto ou effeito de deturpar.
Desfiguração.
\section{Deturpador}
\begin{itemize}
\item {Grp. gram.:adj.}
\end{itemize}
\begin{itemize}
\item {Grp. gram.:M.}
\end{itemize}
Que deturpa.
Aquelle que deturpa.
\section{Deturpar}
\begin{itemize}
\item {Grp. gram.:v. t.}
\end{itemize}
\begin{itemize}
\item {Proveniência:(Lat. \textunderscore deturpare\textunderscore )}
\end{itemize}
Tornar torpe, feio.
Desfigurar: \textunderscore deturpar o sentido de uma phrase\textunderscore .
Corromper, estragar.
\section{Deu}
\begin{itemize}
\item {Grp. gram.:m.}
\end{itemize}
\begin{itemize}
\item {Utilização:Ant.}
\end{itemize}
O mesmo que \textunderscore Deus\textunderscore . Cf. Usque, \textunderscore Tribulações\textunderscore , 37.
\section{Déu}
\begin{itemize}
\item {Grp. gram.:m.}
\end{itemize}
\begin{itemize}
\item {Utilização:Prov.}
\end{itemize}
\begin{itemize}
\item {Utilização:trasm.}
\end{itemize}
\textunderscore Andar de déu em déu\textunderscore , andar de casa em casa, de porta em porta, á procura de qualquer coisa.
Andar ás cambalhotas. (Colhido em Turquel)
\section{Deunx}
\begin{itemize}
\item {fónica:dé-uncs}
\end{itemize}
\begin{itemize}
\item {Grp. gram.:m.}
\end{itemize}
Onze duodécimas partes. Cf. Castilho, \textunderscore Fastos\textunderscore , I, 355.
(Fórma lat., que poderíamos substituir por \textunderscore deunce\textunderscore )
\section{Deus}
\begin{itemize}
\item {Grp. gram.:m.}
\end{itemize}
\begin{itemize}
\item {Utilização:Fig.}
\end{itemize}
\begin{itemize}
\item {Proveniência:(Lat. \textunderscore Deus\textunderscore )}
\end{itemize}
Principio supremo, que as religiões consideram superior á natureza.
Sêr infinito, perfeito, criador e conservador do universo.
Divindade.
Cada uma das pessôas da Trindade christan.
Ente sobrenatural, que, segundo o Polytheísmo, presidia a certa ordem de phenómenos.
Indivíduo ou personagem, que, por qualidades extraordinárias, se impõe á adoração ou ao amor dos homens.
Objecto de um culto ou de um desejo ardente, que se antepõe a todos os outros desejos ou affectos: \textunderscore o deus do agiota é o oiro\textunderscore .
\section{Deusa}
\begin{itemize}
\item {Grp. gram.:f.}
\end{itemize}
\begin{itemize}
\item {Utilização:Fig.}
\end{itemize}
Cada uma das divindades femininas do polytheísmo.
Mulher adorável, extremamente formosa.
(Fem. de \textunderscore deus\textunderscore )
\section{Deus-dará}
\begin{itemize}
\item {Grp. gram.:m.}
\end{itemize}
Us. na loc. \textunderscore ao Deus-dará\textunderscore , á tôa, ao acaso, á ventura, á gandaia.
\section{Deutergia}
\begin{itemize}
\item {Grp. gram.:f.}
\end{itemize}
\begin{itemize}
\item {Utilização:Med.}
\end{itemize}
\begin{itemize}
\item {Proveniência:(Do gr. \textunderscore deutos\textunderscore  + \textunderscore ergon\textunderscore )}
\end{itemize}
Conjunto dos effeitos secundários de um medicamento.
\section{Deuteria}
\begin{itemize}
\item {Grp. gram.:f.}
\end{itemize}
\begin{itemize}
\item {Utilização:Med.}
\end{itemize}
\begin{itemize}
\item {Proveniência:(Gr. \textunderscore deutérion\textunderscore )}
\end{itemize}
Accidentes, resultantes da retenção das secundinas.
\section{Deuterocanónico}
\begin{itemize}
\item {Grp. gram.:adj.}
\end{itemize}
\begin{itemize}
\item {Proveniência:(Do gr. \textunderscore deuteros\textunderscore  + \textunderscore kanonikos\textunderscore )}
\end{itemize}
Diz-se dos livros santos, que não entraram, desde os primeiros concílios, nos cânones da Escritura.
\section{Deuterogamia}
\begin{itemize}
\item {Grp. gram.:f.}
\end{itemize}
Estado do deuterógamo.
\section{Deuterógamo}
\begin{itemize}
\item {Grp. gram.:m.}
\end{itemize}
\begin{itemize}
\item {Proveniência:(Do gr. \textunderscore deuteros\textunderscore  + \textunderscore gamos\textunderscore )}
\end{itemize}
Aquelle que se casa segunda vez.
\section{Deuterologia}
\begin{itemize}
\item {Grp. gram.:f.}
\end{itemize}
\begin{itemize}
\item {Proveniência:(Do gr. \textunderscore deutera\textunderscore  + \textunderscore logos\textunderscore )}
\end{itemize}
Tratado das secundinas.
\section{Deuterologia}
\begin{itemize}
\item {Grp. gram.:f.}
\end{itemize}
\begin{itemize}
\item {Proveniência:(Do gr. \textunderscore deuteros\textunderscore  + \textunderscore logos\textunderscore )}
\end{itemize}
Discurso, que o defensor officioso, nos tribunaes de Athenas, fazia em seguida ao discurso do accusado, que tinha sempre de falar primeiro.
\section{Deuteronómio}
\begin{itemize}
\item {Grp. gram.:m.}
\end{itemize}
\begin{itemize}
\item {Proveniência:(Gr. \textunderscore deuteronomion\textunderscore , de \textunderscore deuteros\textunderscore  + \textunderscore nomos\textunderscore )}
\end{itemize}
O quinto livro do \textunderscore Pentateuco\textunderscore .
\section{Deuteropathia}
\begin{itemize}
\item {Grp. gram.:f.}
\end{itemize}
\begin{itemize}
\item {Proveniência:(Do gr. \textunderscore deuteros\textunderscore  + \textunderscore pathos\textunderscore )}
\end{itemize}
Doença secundária, desenvolvida pela influência de outra.
\section{Deuteropáthico}
\begin{itemize}
\item {Grp. gram.:adj.}
\end{itemize}
Relativo á deuteropathia.
\section{Deuteropatia}
\begin{itemize}
\item {Grp. gram.:f.}
\end{itemize}
\begin{itemize}
\item {Proveniência:(Do gr. \textunderscore deuteros\textunderscore  + \textunderscore pathos\textunderscore )}
\end{itemize}
Doença secundária, desenvolvida pela influência de outra.
\section{Deuteropático}
\begin{itemize}
\item {Grp. gram.:adj.}
\end{itemize}
Relativo á deuteropatia.
\section{Deuteropirâmide}
\begin{itemize}
\item {Grp. gram.:f.}
\end{itemize}
\begin{itemize}
\item {Utilização:Miner.}
\end{itemize}
\begin{itemize}
\item {Proveniência:(Do gr. \textunderscore deuteros\textunderscore  + \textunderscore puramis\textunderscore )}
\end{itemize}
Forma holoédrica dos mineraes dos sistemas tetragonal, e hexagonal.
\section{Deuteroprisma}
\begin{itemize}
\item {Grp. gram.:m.}
\end{itemize}
\begin{itemize}
\item {Utilização:Miner.}
\end{itemize}
\begin{itemize}
\item {Proveniência:(Do gr. \textunderscore deuteros\textunderscore  + \textunderscore prisma\textunderscore )}
\end{itemize}
Uma das fórmas holoédricas dos mineraes do sistema hexagonal.
\section{Deuteropyrâmide}
\begin{itemize}
\item {Grp. gram.:f.}
\end{itemize}
\begin{itemize}
\item {Utilização:Miner.}
\end{itemize}
\begin{itemize}
\item {Proveniência:(Do gr. \textunderscore deuteros\textunderscore  + \textunderscore puramis\textunderscore )}
\end{itemize}
Forma holoédrica dos mineraes dos systemas tetragonal, e hexagonal.
\section{Deuteroscopia}
\begin{itemize}
\item {Grp. gram.:f.}
\end{itemize}
\begin{itemize}
\item {Proveniência:(Do gr. \textunderscore deuteros\textunderscore  + \textunderscore skopein\textunderscore )}
\end{itemize}
Estado mórbido, em que o paciente julga vêr coisas futuras \textunderscore ou\textunderscore muito afastadas.
\section{Deuteroses}
\begin{itemize}
\item {Grp. gram.:f. pl.}
\end{itemize}
\begin{itemize}
\item {Utilização:Des.}
\end{itemize}
\begin{itemize}
\item {Proveniência:(Gr. \textunderscore deuterosis\textunderscore )}
\end{itemize}
Tradições.
\section{Deutiodeto}
\begin{itemize}
\item {fónica:dê}
\end{itemize}
\begin{itemize}
\item {Grp. gram.:m.}
\end{itemize}
\begin{itemize}
\item {Proveniência:(De \textunderscore deuto...\textunderscore  + \textunderscore iodeto\textunderscore )}
\end{itemize}
Preparado chímico, em que entram duas proporções de iodo.
\section{Deuto...}
\begin{itemize}
\item {Grp. gram.:pref.}
\end{itemize}
\begin{itemize}
\item {Proveniência:(Do gr. \textunderscore deutos\textunderscore )}
\end{itemize}
(indicativo do segundo grau de uma combinação chímica)
\section{Deutochloreto}
\begin{itemize}
\item {Grp. gram.:m.}
\end{itemize}
\begin{itemize}
\item {Utilização:Chím.}
\end{itemize}
\begin{itemize}
\item {Proveniência:(De \textunderscore deuto...\textunderscore  + \textunderscore chloreto\textunderscore )}
\end{itemize}
A segunda das combinações, que o chloreto fórma com um corpo simples, quando póde produzir muitas.
\section{Deutocloreto}
\begin{itemize}
\item {Grp. gram.:m.}
\end{itemize}
\begin{itemize}
\item {Utilização:Chím.}
\end{itemize}
\begin{itemize}
\item {Proveniência:(De \textunderscore deuto...\textunderscore  + \textunderscore cloreto\textunderscore )}
\end{itemize}
A segunda das combinações, que o cloreto fórma com um corpo simples, quando póde produzir muitas.
\section{Deutóxido}
\begin{itemize}
\item {Grp. gram.:m.}
\end{itemize}
\begin{itemize}
\item {Utilização:Chím.}
\end{itemize}
\begin{itemize}
\item {Proveniência:(De \textunderscore deuto...\textunderscore  + \textunderscore óxido\textunderscore )}
\end{itemize}
Segundo grau de oxidação de um corpo, que póde combinar-se em muitas proporções com o oxigênio.
\section{Deutóxydo}
\begin{itemize}
\item {Grp. gram.:m.}
\end{itemize}
\begin{itemize}
\item {Utilização:Chím.}
\end{itemize}
\begin{itemize}
\item {Proveniência:(De \textunderscore deuto...\textunderscore  + \textunderscore óxydo\textunderscore )}
\end{itemize}
Segundo grau de oxydação de um corpo, que póde combinar-se em muitas proporções com o oxygênio.
\section{Dêutzia}
\begin{itemize}
\item {Grp. gram.:f.}
\end{itemize}
\begin{itemize}
\item {Proveniência:(De \textunderscore Deutz\textunderscore , n. p.)}
\end{itemize}
Planta philadélphea.
\section{Devação}
\begin{itemize}
\item {Grp. gram.:f.}
\end{itemize}
(Fórma antiga, por \textunderscore devoção\textunderscore . Cf. Pant. de Aveiro, \textunderscore Itiner.\textunderscore , 29, 38, 41, 47, 52, 2.^a ed.)
\section{Devagar}
\begin{itemize}
\item {Grp. gram.:adv.}
\end{itemize}
O mesmo que \textunderscore vagarosamente\textunderscore .
\section{De-vagar}
\begin{itemize}
\item {Grp. gram.:adv.}
\end{itemize}
O mesmo que \textunderscore vagarosamente\textunderscore .
\section{Devanágari}
\begin{itemize}
\item {Grp. gram.:adj.}
\end{itemize}
\begin{itemize}
\item {Grp. gram.:M.}
\end{itemize}
\begin{itemize}
\item {Proveniência:(Do sãnscr. \textunderscore deva\textunderscore  + \textunderscore nagari\textunderscore )}
\end{itemize}
Diz-se dos caracteres do alphabeto sãnscrito.
Fórma de escrita ou caracteres sãnscríticos.
\section{Devanágrico}
\begin{itemize}
\item {Grp. gram.:adj.}
\end{itemize}
O mesmo que \textunderscore devanágari\textunderscore .
Escrito em devanágari: \textunderscore poema devanágrico\textunderscore .
\section{Devandito}
\begin{itemize}
\item {Grp. gram.:adj.}
\end{itemize}
\begin{itemize}
\item {Utilização:Ant.}
\end{itemize}
O mesmo que \textunderscore sobredito\textunderscore .
\section{Devaneador}
\begin{itemize}
\item {Grp. gram.:adj.}
\end{itemize}
\begin{itemize}
\item {Grp. gram.:M.}
\end{itemize}
Que devaneia.
Aquelle que devaneia.
\section{Devanear}
\begin{itemize}
\item {Grp. gram.:v. t.}
\end{itemize}
\begin{itemize}
\item {Grp. gram.:V. i.}
\end{itemize}
\begin{itemize}
\item {Proveniência:(Do lat. \textunderscore devanus\textunderscore )}
\end{itemize}
Fantasiar, idear.
Meditar; pensar vagamente em.
Imaginar, dizer, coisas sem nexo.
Absorver-se em vagas meditações.
Desvairar.
\section{Devaneio}
\begin{itemize}
\item {Grp. gram.:m.}
\end{itemize}
\begin{itemize}
\item {Utilização:Ant.}
\end{itemize}
Acto de devanear.
Arrogância; desvanecimento.
\section{Devas}
\begin{itemize}
\item {Grp. gram.:m. pl.}
\end{itemize}
Os gênios bons, na Theogonia brahmânica.
\section{Devassa}
\begin{itemize}
\item {Grp. gram.:f.}
\end{itemize}
\begin{itemize}
\item {Utilização:Ant.}
\end{itemize}
\begin{itemize}
\item {Proveniência:(De \textunderscore devassar\textunderscore )}
\end{itemize}
Syndicância a um acto criminoso.
Acto de reunir depoimentos e outras provas, concernentes a um facto criminoso.
Processo, que contém essas provas.
\section{Devassador}
\begin{itemize}
\item {Grp. gram.:adj.}
\end{itemize}
\begin{itemize}
\item {Grp. gram.:M.}
\end{itemize}
Que devassa.
Aquelle que devassa.
\section{Devassamente}
\begin{itemize}
\item {Grp. gram.:adv.}
\end{itemize}
\begin{itemize}
\item {Utilização:Ant.}
\end{itemize}
\begin{itemize}
\item {Proveniência:(De \textunderscore devasso\textunderscore )}
\end{itemize}
Licenciosamente.
Á maneira de devassa.
\section{Devassamento}
\begin{itemize}
\item {Grp. gram.:m.}
\end{itemize}
Acto ou effeito de devassar.
\section{Devassante}
\begin{itemize}
\item {Grp. gram.:adj.}
\end{itemize}
\begin{itemize}
\item {Utilização:Ant.}
\end{itemize}
Que tira ou faz devassa.
\section{Devassar}
\begin{itemize}
\item {Grp. gram.:v. i.}
\end{itemize}
\begin{itemize}
\item {Grp. gram.:V. i.}
\end{itemize}
\begin{itemize}
\item {Utilização:Ant.}
\end{itemize}
\begin{itemize}
\item {Proveniência:(Do lat. \textunderscore de\textunderscore  + \textunderscore fassus\textunderscore , part. de \textunderscore fateor\textunderscore ?)}
\end{itemize}
Invadir (aquillo que é defeso ou vedado): \textunderscore os pastores devassaram a seara\textunderscore .
Divulgar.
Tornar lasso.
Tornar relaxado, licencioso.
Pesquizar; penetrar: \textunderscore devassar segredos\textunderscore .
Fazer inquirição.
Tirar devassa.
\section{Devassidade}
\begin{itemize}
\item {Grp. gram.:f.}
\end{itemize}
O mesmo que \textunderscore devassidão\textunderscore . Cf. Alves Mendes, \textunderscore Discursos\textunderscore , 253.
\section{Devassidão}
\begin{itemize}
\item {Grp. gram.:f.}
\end{itemize}
Qualidade daquelle ou daquillo que é devasso.
Libertinagem.
\section{Devasso}
\begin{itemize}
\item {Grp. gram.:adj.}
\end{itemize}
\begin{itemize}
\item {Grp. gram.:M.}
\end{itemize}
\begin{itemize}
\item {Utilização:Ant.}
\end{itemize}
\begin{itemize}
\item {Proveniência:(De \textunderscore devassar\textunderscore )}
\end{itemize}
Libertino, licencioso: \textunderscore homem devasso\textunderscore .
Devassado: \textunderscore terrenos devassos\textunderscore .
Aquelle que é devasso.
Abandono, (falando-se de terrenos).
\section{Devastação}
\begin{itemize}
\item {Grp. gram.:f.}
\end{itemize}
\begin{itemize}
\item {Proveniência:(Lat. \textunderscore devastatio\textunderscore )}
\end{itemize}
Acto ou effeito de devastar.
\section{Devastador}
\begin{itemize}
\item {Grp. gram.:adj.}
\end{itemize}
\begin{itemize}
\item {Grp. gram.:M.}
\end{itemize}
\begin{itemize}
\item {Proveniência:(Lat. \textunderscore devastator\textunderscore )}
\end{itemize}
Que devasta.
Aquelle que devasta.
\section{Devastar}
\begin{itemize}
\item {Grp. gram.:v. t.}
\end{itemize}
\begin{itemize}
\item {Proveniência:(Lat. \textunderscore devastare\textunderscore )}
\end{itemize}
Talar, assolar: \textunderscore a invasão francesa devastou a Beira\textunderscore .
Damnificar.
Tornar deserto.
\section{Deve}
\begin{itemize}
\item {Grp. gram.:m.}
\end{itemize}
\begin{itemize}
\item {Utilização:Prov.}
\end{itemize}
\begin{itemize}
\item {Utilização:trasm.}
\end{itemize}
\begin{itemize}
\item {Proveniência:(De \textunderscore dever\textunderscore )}
\end{itemize}
Débito ou despesa de um estabelecimento commercial, indicada no livro chamado \textunderscore Razão\textunderscore .
Certo jogo de pião.
\section{Devedado}
\begin{itemize}
\item {Grp. gram.:adj.}
\end{itemize}
\begin{itemize}
\item {Utilização:Ant.}
\end{itemize}
O mesmo que \textunderscore vedado\textunderscore , prohibido.
\section{Devedor}
\begin{itemize}
\item {Grp. gram.:adj.}
\end{itemize}
\begin{itemize}
\item {Grp. gram.:M.}
\end{itemize}
\begin{itemize}
\item {Proveniência:(Do lat. \textunderscore debitor\textunderscore )}
\end{itemize}
Que deve.
Aquelle que deve.
\section{De-vedro}
\begin{itemize}
\item {Grp. gram.:loc. adv.}
\end{itemize}
\begin{itemize}
\item {Utilização:Ant.}
\end{itemize}
\begin{itemize}
\item {Proveniência:(Do lat. \textunderscore de\textunderscore  + \textunderscore vetero\textunderscore )}
\end{itemize}
Nos tempos antigos.
\section{Devenir}
\begin{itemize}
\item {Grp. gram.:v. i.}
\end{itemize}
(?):«\textunderscore começado que é o segundo momento da sua existencia, do seu devenir incessante (para nos servirmos de um verbo francez, correspondente ao «werden» allemão, que não tem exacto equivalente em português...\textunderscore »Latino, \textunderscore Elogios Acad.\textunderscore , I, 86).
\section{Deventre}
\begin{itemize}
\item {Grp. gram.:m.}
\end{itemize}
\begin{itemize}
\item {Proveniência:(De \textunderscore de...\textunderscore  + \textunderscore ventre\textunderscore )}
\end{itemize}
Intestinos dos animaes. Cf. \textunderscore Port. Mon. Hist.\textunderscore , \textunderscore Script.\textunderscore , 259.
\section{Dever}
\begin{itemize}
\item {Grp. gram.:v. t.}
\end{itemize}
\begin{itemize}
\item {Grp. gram.:V. i.}
\end{itemize}
\begin{itemize}
\item {Grp. gram.:M.}
\end{itemize}
\begin{itemize}
\item {Proveniência:(Lat. \textunderscore debere\textunderscore )}
\end{itemize}
Têr por obrigação: \textunderscore o filho deve obediência aos pais\textunderscore .
Estar obrigado a.
Têr que dar ou prestar: \textunderscore dever muito dinheiro\textunderscore .
Estar em agradecimento de.
Têr que restituir.
Têr de.
Sêr provavel: \textunderscore hoje, deve chover\textunderscore .
Têr dívidas; têr deveres: \textunderscore quem não deve não teme\textunderscore .
Obrigação de fazer ou deixar de fazer alguma coisa.
Obrigação: \textunderscore não conhece os seus deveres\textunderscore .
\section{Devéras}
\begin{itemize}
\item {Grp. gram.:adv.}
\end{itemize}
\begin{itemize}
\item {Proveniência:(De \textunderscore de...\textunderscore  + \textunderscore véras\textunderscore )}
\end{itemize}
Verdadeiramente; realmente: \textunderscore mas encontraste-o, devéras\textunderscore ?
Muito, em alto grau: \textunderscore tem soffrido devéras\textunderscore .
\section{Devesa}
\begin{itemize}
\item {fónica:vé}
\end{itemize}
\begin{itemize}
\item {Grp. gram.:f.}
\end{itemize}
\begin{itemize}
\item {Proveniência:(Do lat. \textunderscore defensa\textunderscore )}
\end{itemize}
Alameda, que delimita um terreno.
Mata cercada.
Quinta murada.
Soito.
Variedade de pêra do Minho e do Doiro.
\section{Devesal}
\begin{itemize}
\item {Grp. gram.:m.}
\end{itemize}
\begin{itemize}
\item {Utilização:Des.}
\end{itemize}
\begin{itemize}
\item {Proveniência:(De \textunderscore devesa\textunderscore )}
\end{itemize}
Lugar abundante de árvores ou pastos.
\section{Deviação}
\begin{itemize}
\item {Grp. gram.:f.}
\end{itemize}
\begin{itemize}
\item {Proveniência:(Do lat. \textunderscore deviare\textunderscore )}
\end{itemize}
Desvio ou mudança de viagem.
\section{Devida}
\begin{itemize}
\item {Grp. gram.:f.}
\end{itemize}
\begin{itemize}
\item {Utilização:Ant.}
\end{itemize}
O mesmo que \textunderscore dívida\textunderscore . Cf. D. de Góes, \textunderscore in\textunderscore  Tôrre do Tombo, gav. 22, maço 4.^o, n.^o 2.
\section{Devidamente}
\begin{itemize}
\item {Grp. gram.:adv.}
\end{itemize}
De modo devido.
Conforme ao dever.
\section{Devido}
\begin{itemize}
\item {Grp. gram.:m.}
\end{itemize}
\begin{itemize}
\item {Utilização:Ant.}
\end{itemize}
Aquillo que se deve.
Parentesco, obrigação.
Obrigação social.
Relações de nobreza.
\section{Dévio}
\begin{itemize}
\item {Grp. gram.:adj.}
\end{itemize}
\begin{itemize}
\item {Proveniência:(Lat. \textunderscore devius\textunderscore )}
\end{itemize}
Desencaminhado, extraviado: \textunderscore rebanhos dévios\textunderscore .
Intransitável: \textunderscore caminhos dévios\textunderscore .
\section{Devisado}
\begin{itemize}
\item {Grp. gram.:adj.}
\end{itemize}
\begin{itemize}
\item {Utilização:Ant.}
\end{itemize}
Vistoso.
Marcado.
(Por \textunderscore divisado\textunderscore )
\section{Devisar}
\begin{itemize}
\item {Grp. gram.:v. t.}
\end{itemize}
\begin{itemize}
\item {Utilização:Des.}
\end{itemize}
O mesmo que \textunderscore planejar\textunderscore . Cf. Rui Barb., \textunderscore Réplica\textunderscore , 157.
\section{Devitrificação}
\begin{itemize}
\item {Grp. gram.:f.}
\end{itemize}
Acto de devitrificar.
Cp. \textunderscore desvitrificação\textunderscore .
\section{Devitrificar}
\begin{itemize}
\item {Grp. gram.:v. t.}
\end{itemize}
\begin{itemize}
\item {Proveniência:(De \textunderscore de...\textunderscore  + \textunderscore vitrificar\textunderscore )}
\end{itemize}
Tirar o estado de vitrificação a.
Tirar a apparência de vidro a.
\section{Devoção}
\begin{itemize}
\item {Grp. gram.:f.}
\end{itemize}
\begin{itemize}
\item {Proveniência:(Lat. \textunderscore devotio\textunderscore )}
\end{itemize}
Sentimento religioso: \textunderscore orar com devoção\textunderscore .
Práticas religiosas: \textunderscore gastava muito tempo em devoções\textunderscore .
Dedicação ás coisas religiosas.
Dedicação íntima.
Objecto de especial veneração: \textunderscore tu és a minha devoção\textunderscore .
Veneração; affecto.
\section{Devocionário}
\begin{itemize}
\item {Grp. gram.:m.}
\end{itemize}
\begin{itemize}
\item {Proveniência:(Do lat. \textunderscore devotio\textunderscore )}
\end{itemize}
Livro de orações.
\section{Devocionista}
\begin{itemize}
\item {Grp. gram.:m. ,  f.  e  adj.}
\end{itemize}
\begin{itemize}
\item {Proveniência:(Do lat. \textunderscore devotio\textunderscore )}
\end{itemize}
Pessôa que se entrega a devoções, que é igrejeira.
\section{Devolução}
\begin{itemize}
\item {Grp. gram.:f.}
\end{itemize}
Acto ou effeito de devolver.
\section{Devolutivo}
\begin{itemize}
\item {Grp. gram.:adj.}
\end{itemize}
\begin{itemize}
\item {Proveniência:(De \textunderscore devoluto\textunderscore )}
\end{itemize}
Que devolve, que determina devolução.
\section{Devoluto}
\begin{itemize}
\item {Grp. gram.:adj.}
\end{itemize}
\begin{itemize}
\item {Proveniência:(Lat. \textunderscore devolutus\textunderscore )}
\end{itemize}
Adquirido por devolução.
Vazio.
Deshabitado; desoccupado: \textunderscore uma casa devoluta\textunderscore .
\section{Devolutório}
\begin{itemize}
\item {Grp. gram.:adj.}
\end{itemize}
(V.devolutivo)
\section{Devolver}
\begin{itemize}
\item {Grp. gram.:v. t.}
\end{itemize}
\begin{itemize}
\item {Proveniência:(Lat. \textunderscore devolvere\textunderscore )}
\end{itemize}
Enviar ou entregar a alguém (aquillo que êste havia entregado ou enviado).
Restituir.
Transferir.
Recusar.
Dar.
Desenvolver.
\section{Devoniano}
\begin{itemize}
\item {Grp. gram.:adj.}
\end{itemize}
O mesmo que \textunderscore devónico\textunderscore . Cf. \textunderscore Museu Technol.\textunderscore , 36.
\section{Devónico}
\begin{itemize}
\item {Grp. gram.:adj.}
\end{itemize}
\begin{itemize}
\item {Proveniência:(De \textunderscore Devonshire\textunderscore , n. p.)}
\end{itemize}
Diz-se do systema geológico, que comprehende as camadas intermediárias ao systema silúrico e ao permo-carbónico, na Inglaterra meridional.
\section{Devoração}
\begin{itemize}
\item {Grp. gram.:f.}
\end{itemize}
\begin{itemize}
\item {Utilização:Des.}
\end{itemize}
\begin{itemize}
\item {Proveniência:(Lat. \textunderscore devoratio\textunderscore )}
\end{itemize}
Acto de devorar.
\section{Devorador}
\begin{itemize}
\item {Grp. gram.:adj.}
\end{itemize}
\begin{itemize}
\item {Grp. gram.:M.}
\end{itemize}
\begin{itemize}
\item {Proveniência:(Lat. devorator)}
\end{itemize}
Que devora.
Insaciável: \textunderscore fome devoradora\textunderscore .
Aquelle que devora.
\section{Devorante}
\begin{itemize}
\item {Grp. gram.:adj.}
\end{itemize}
\begin{itemize}
\item {Grp. gram.:F.}
\end{itemize}
\begin{itemize}
\item {Utilização:Pop.}
\end{itemize}
\begin{itemize}
\item {Proveniência:(Lat. \textunderscore devorans\textunderscore )}
\end{itemize}
Devorador.
Grande appetite de comer: \textunderscore vens com uma devorante\textunderscore !
\section{Devorar}
\begin{itemize}
\item {Grp. gram.:v. t.}
\end{itemize}
\begin{itemize}
\item {Proveniência:(Lat. \textunderscore devorare\textunderscore )}
\end{itemize}
Comer com rapidez.
Tragar soffregamente.
Destruir, corroer: \textunderscore o tempo tudo devora\textunderscore .
Usurpar, roubar.
Assolar.
Esgotar, absorver.
Agitar profundamente; agitar.
Sumir dentro de si.
Lêr avidamente: \textunderscore devorar muitas páginas\textunderscore .
Cubiçar: \textunderscore devorava-a com os olhos\textunderscore .
Supportar com resignação.
\section{Devorismo}
\begin{itemize}
\item {Grp. gram.:m.}
\end{itemize}
\begin{itemize}
\item {Utilização:Fig.}
\end{itemize}
\begin{itemize}
\item {Proveniência:(De \textunderscore devorar\textunderscore )}
\end{itemize}
Gasto exaggerado e injustificado.
Dissipação da fazenda pública, em proveito próprio.
\section{Devorista}
\begin{itemize}
\item {Grp. gram.:m.  e  adj.}
\end{itemize}
Aquelle que pratíca o devorismo.
\section{Devotação}
\begin{itemize}
\item {Grp. gram.:f.}
\end{itemize}
Acto ou effeito de devotar.
\section{Devotado}
\begin{itemize}
\item {Grp. gram.:adj.}
\end{itemize}
Offerecido em voto.
Destinado.
Affeiçoado, dedicado: \textunderscore seu devotado amigo\textunderscore .
\section{Devotamente}
\begin{itemize}
\item {Grp. gram.:adv.}
\end{itemize}
\begin{itemize}
\item {Proveniência:(De \textunderscore devoto\textunderscore )}
\end{itemize}
Com devoção; dedicadamente.
\section{Devotamento}
\begin{itemize}
\item {Grp. gram.:m.}
\end{itemize}
Acto de devotar.
Dedicação.
\section{Devotar}
\begin{itemize}
\item {Grp. gram.:v. t.}
\end{itemize}
\begin{itemize}
\item {Proveniência:(Lat. \textunderscore devotare\textunderscore )}
\end{itemize}
Offerecer em voto.
Dedicar, consagrar.
Tributar.
\section{Devoto}
\begin{itemize}
\item {Grp. gram.:adj.}
\end{itemize}
\begin{itemize}
\item {Grp. gram.:M.}
\end{itemize}
\begin{itemize}
\item {Proveniência:(Lat. \textunderscore devotus\textunderscore )}
\end{itemize}
Que tem devoção: \textunderscore pessôa devota\textunderscore .
Que revela devoção.
Produzido por devoção.
Extremamente religioso, fanático.
Indivíduo devoto.
Admirador.
Amigo dedicado.
Aquelle que aprecia muito: \textunderscore um devoto de bons pitéus\textunderscore .
\section{Dexar}
\begin{itemize}
\item {Grp. gram.:v. t.}
\end{itemize}
\begin{itemize}
\item {Utilização:Ant.}
\end{itemize}
O mesmo que \textunderscore deixar\textunderscore .
\section{Déxemo}
\begin{itemize}
\item {Grp. gram.:m.}
\end{itemize}
\begin{itemize}
\item {Utilização:Ant.}
\end{itemize}
O mesmo que \textunderscore demo\textunderscore ^1. Cf. G. Vicente, \textunderscore Auto Pastoril\textunderscore .
\section{Dexiocardia}
\begin{itemize}
\item {fónica:csi}
\end{itemize}
\begin{itemize}
\item {Grp. gram.:f.}
\end{itemize}
\begin{itemize}
\item {Utilização:Med.}
\end{itemize}
\begin{itemize}
\item {Proveniência:(Do gr. \textunderscore dexios\textunderscore , direito + \textunderscore kardia\textunderscore )}
\end{itemize}
Desvio do coração para o lado direito do thórax.
\section{Dextans}
\begin{itemize}
\item {Grp. gram.:m.}
\end{itemize}
\begin{itemize}
\item {Proveniência:(Lat. \textunderscore dextans\textunderscore )}
\end{itemize}
Pêso de dez onças, entre os antigos Romanos. Cf. Castilho, \textunderscore Fastos\textunderscore , I, 353 e 355.
\section{Dextante}
\begin{itemize}
\item {Grp. gram.:m.}
\end{itemize}
\begin{itemize}
\item {Proveniência:(Lat. \textunderscore dextans\textunderscore )}
\end{itemize}
Pêso de dez onças, entre os antigos Romanos. Cf. Castilho, \textunderscore Fastos\textunderscore , I, 353 e 355.
\section{Dexteridade}
\begin{itemize}
\item {Grp. gram.:f.}
\end{itemize}
\begin{itemize}
\item {Proveniência:(Lat. \textunderscore dexteritas\textunderscore )}
\end{itemize}
O mesmo que \textunderscore destreza\textunderscore .
\section{Dextra}
\begin{itemize}
\item {Grp. gram.:f.}
\end{itemize}
O mesmo que \textunderscore destra\textunderscore .
\section{Dextrina}
\begin{itemize}
\item {Grp. gram.:f.}
\end{itemize}
\begin{itemize}
\item {Proveniência:(Do lat. \textunderscore dexter\textunderscore )}
\end{itemize}
Substância gomosa, em que se transformam os glóbulos do amido, sôb a influência dos ácidos, óxydos, etc.
\section{Dextro}
\begin{itemize}
\item {Grp. gram.:adj.}
\end{itemize}
(V. \textunderscore destro\textunderscore ^1)
\section{Dextrogiro}
\begin{itemize}
\item {Grp. gram.:adj.}
\end{itemize}
\begin{itemize}
\item {Utilização:Phýs.}
\end{itemize}
\begin{itemize}
\item {Grp. gram.:adj.}
\end{itemize}
\begin{itemize}
\item {Proveniência:(Do gr. \textunderscore dexter\textunderscore  + \textunderscore gyrare\textunderscore )}
\end{itemize}
Diz-se dos corpos, que têm a propriedade de desviar para a direita o plano de polarização da luz.
Cp. \textunderscore levogiro\textunderscore .
Que faz voltar para a direita, (falando-se, por exemplo, de certas febres, que determinam a rotação dos olhos para a direita).
\section{Dextrogyro}
\begin{itemize}
\item {Grp. gram.:adj.}
\end{itemize}
\begin{itemize}
\item {Utilização:Phýs.}
\end{itemize}
\begin{itemize}
\item {Grp. gram.:adj.}
\end{itemize}
\begin{itemize}
\item {Proveniência:(Do gr. \textunderscore dexter\textunderscore  + \textunderscore gyrare\textunderscore )}
\end{itemize}
Diz-se dos corpos, que têm a propriedade de desviar para a direita o plano de polarização da luz.
Cp. \textunderscore levogyro\textunderscore .
Que faz voltar para a direita, (falando-se, por exemplo, de certas febres, que determinam a rotação dos olhos para a direita).
\section{Dextrovolúvel}
\begin{itemize}
\item {Grp. gram.:adj.}
\end{itemize}
\begin{itemize}
\item {Utilização:Bot.}
\end{itemize}
\begin{itemize}
\item {Proveniência:(Do lat. \textunderscore dexter\textunderscore  + \textunderscore volubilis\textunderscore )}
\end{itemize}
Diz-se do tronco e das gavinhas, quando se encaracolam para o lado direito.
\section{Déz}
\begin{itemize}
\item {Grp. gram.:adj.}
\end{itemize}
\begin{itemize}
\item {Grp. gram.:M.}
\end{itemize}
\begin{itemize}
\item {Grp. gram.:Loc. adv.}
\end{itemize}
\begin{itemize}
\item {Utilização:pop.}
\end{itemize}
\begin{itemize}
\item {Proveniência:(Do lat. \textunderscore decem\textunderscore )}
\end{itemize}
Diz-se do número cardinal, formado de duas vezes cinco ou de nove e mais um.
Décimo.
Aquelle ou aquillo que tem o décimo lugar numa série.
\textunderscore Como um déz\textunderscore , com toda a certeza, infallivelmente. Cf. Camillo, \textunderscore Brasileira\textunderscore , 128.
\section{Dezanove}
\begin{itemize}
\item {Grp. gram.:adj.}
\end{itemize}
\begin{itemize}
\item {Grp. gram.:M.}
\end{itemize}
\begin{itemize}
\item {Proveniência:(De \textunderscore dez\textunderscore  + \textunderscore a\textunderscore  + \textunderscore nove\textunderscore )}
\end{itemize}
Diz-se do número cardinal, formado de déz e mais nove.
Aquelle ou aquillo que numa série occupa o lugar décimo nono.
\section{Dezanovena}
\begin{itemize}
\item {Grp. gram.:f.}
\end{itemize}
\begin{itemize}
\item {Utilização:Mús.}
\end{itemize}
\begin{itemize}
\item {Proveniência:(De \textunderscore dezanove\textunderscore )}
\end{itemize}
Registo de órgão, que só se emprega, combinado com outros.
\section{Dezaseis}
\begin{itemize}
\item {fónica:seis}
\end{itemize}
\begin{itemize}
\item {Grp. gram.:adj.}
\end{itemize}
\begin{itemize}
\item {Grp. gram.:M.}
\end{itemize}
\begin{itemize}
\item {Proveniência:(De \textunderscore déz\textunderscore  + \textunderscore a\textunderscore  + \textunderscore seis\textunderscore )}
\end{itemize}
Diz-se do número cardinal, formado de déz e mais seis.
Aquelle ou aquillo que numa série occupa o lugar décimo sexto.
\section{Dezasete}
\begin{itemize}
\item {fónica:sé}
\end{itemize}
\begin{itemize}
\item {Grp. gram.:adj.}
\end{itemize}
\begin{itemize}
\item {Grp. gram.:M.}
\end{itemize}
\begin{itemize}
\item {Proveniência:(De \textunderscore déz\textunderscore  + \textunderscore a\textunderscore  + \textunderscore sete\textunderscore )}
\end{itemize}
Diz-se do número cardinal, formado de déz e mais sete.
Aquelle ou aquillo que numa série occupa o lugar décimo sétimo.
\section{Dezasseis}
\begin{itemize}
\item {Grp. gram.:adj.}
\end{itemize}
\begin{itemize}
\item {Grp. gram.:M.}
\end{itemize}
\begin{itemize}
\item {Proveniência:(De \textunderscore déz\textunderscore  + \textunderscore a\textunderscore  + \textunderscore seis\textunderscore )}
\end{itemize}
Diz-se do número cardinal, formado de déz e mais seis.
Aquelle ou aquillo que numa série occupa o lugar décimo sexto.
\section{Dezassete}
\begin{itemize}
\item {Grp. gram.:adj.}
\end{itemize}
\begin{itemize}
\item {Grp. gram.:M.}
\end{itemize}
\begin{itemize}
\item {Proveniência:(De \textunderscore déz\textunderscore  + \textunderscore a\textunderscore  + \textunderscore sete\textunderscore )}
\end{itemize}
Diz-se do número cardinal, formado de déz e mais sete.
Aquelle ou aquillo que numa série occupa o lugar décimo sétimo.
\section{Déz-bofas}
\begin{itemize}
\item {Grp. gram.:m. pl.}
\end{itemize}
\begin{itemize}
\item {Utilização:ant.}
\end{itemize}
\begin{itemize}
\item {Utilização:Gír.}
\end{itemize}
Moéda de déz reis.
\section{Dezembro}
\begin{itemize}
\item {Grp. gram.:m.}
\end{itemize}
\begin{itemize}
\item {Proveniência:(Do lat. \textunderscore december\textunderscore , de \textunderscore decem\textunderscore , déz, porque o antigo anno latino, que começava em Março, tinha déz meses, sendo Dezembro o último)}
\end{itemize}
Duodécimo e último mês do nosso anno.
\section{Dezena}
\begin{itemize}
\item {Grp. gram.:f.}
\end{itemize}
Grupo de déz.
Espaço de déz dias. Cf. \textunderscore Auto do Dia de Juízo\textunderscore , cit. por Castilho.
(B. lat. \textunderscore decena\textunderscore )
\section{Dezenário}
\begin{itemize}
\item {Grp. gram.:m.}
\end{itemize}
\begin{itemize}
\item {Utilização:Ant.}
\end{itemize}
\begin{itemize}
\item {Proveniência:(De \textunderscore dezena\textunderscore )}
\end{itemize}
Período ou espaço de déz dias. (Colhido num testamento do séc. XVII)
\section{Dezeno}
\begin{itemize}
\item {Grp. gram.:adj.}
\end{itemize}
\begin{itemize}
\item {Utilização:P. us.}
\end{itemize}
Que numa série de déz occupa o último lugar. Cf. Castilho, \textunderscore Fastos\textunderscore , I, 125.
(Cp. \textunderscore dezena\textunderscore )
\section{Dezenove}
\begin{itemize}
\item {Grp. gram.:adj.}
\end{itemize}
(V.dezanove)
\section{Dezeseis}
\begin{itemize}
\item {Grp. gram.:adj.}
\end{itemize}
(V.dezaseis)
\section{Dezesete}
\begin{itemize}
\item {Grp. gram.:adj.}
\end{itemize}
(V.dezasete)
\section{Déz-e-um}
\begin{itemize}
\item {Grp. gram.:m.}
\end{itemize}
\begin{itemize}
\item {Utilização:Bras. de Minas}
\end{itemize}
Alcoviteiro.
(Variante da expressão \textunderscore onze-letras\textunderscore )
\section{Dezóito}
\begin{itemize}
\item {Grp. gram.:adj.}
\end{itemize}
\begin{itemize}
\item {Grp. gram.:M.}
\end{itemize}
\begin{itemize}
\item {Proveniência:(De \textunderscore déz\textunderscore  + \textunderscore a\textunderscore  + \textunderscore oito\textunderscore )}
\end{itemize}
Diz-se do número cardinal, formado de déz e mais oito.
Aquelle ou aquillo que numa série occupa o lugar décimo oitavo.
\section{Dhália}
\begin{itemize}
\item {Grp. gram.:f.}
\end{itemize}
\begin{itemize}
\item {Proveniência:(De \textunderscore Dhal\textunderscore , n. p.)}
\end{itemize}
Planta tuberculosa dos jardins.
\section{Dhalina}
\begin{itemize}
\item {Grp. gram.:f.}
\end{itemize}
Substância dos bolbos da dhália.
\section{Di...}
\begin{itemize}
\item {Grp. gram.:pref.}
\end{itemize}
O mesmo que \textunderscore bis...\textunderscore 
O mesmo que \textunderscore dis...\textunderscore 
\section{Dia}
\begin{itemize}
\item {Grp. gram.:m.}
\end{itemize}
\begin{itemize}
\item {Utilização:Prov.}
\end{itemize}
\begin{itemize}
\item {Utilização:alent.}
\end{itemize}
\begin{itemize}
\item {Utilização:Fig.}
\end{itemize}
\begin{itemize}
\item {Grp. gram.:Loc. adv.}
\end{itemize}
\begin{itemize}
\item {Grp. gram.:Loc. adv.}
\end{itemize}
\begin{itemize}
\item {Grp. gram.:Pl.}
\end{itemize}
\begin{itemize}
\item {Grp. gram.:Loc. interj.}
\end{itemize}
\begin{itemize}
\item {Proveniência:(Do lat. \textunderscore dies\textunderscore )}
\end{itemize}
Claridade, que o sol dá á terra.
Espaço de tempo, que vai do nascer ao pôr do sol.
Espaço de vinte e quatro horas.
Qualquer parte ou momento dessas vinte e quatro horas: \textunderscore a revolução foi no dia 1 de Dezembro\textunderscore .
Jôgo de rapazes.
Estado atmosphérico: \textunderscore hoje está mau dia\textunderscore .
\textunderscore Passou um mau dia\textunderscore , esteve incommodado ou soffreu muito durante o dia.
\textunderscore Algum dia\textunderscore , noutro tempo.
\textunderscore Um dia\textunderscore , em certa occasião, em certa época.
\textunderscore Perder o dia\textunderscore , não trabalhar nas horas do dia, em que precisava ou costumava trabalhar.
\textunderscore Ganhar o dia\textunderscore , receber salário, relativo ao dia em que se trabalhou, ou têr direito a esse salário.
\textunderscore Andar em dia\textunderscore , têr as contas saldadas ou bem reguladas.
\textunderscore Dia de semana\textunderscore , qualquer dia, excepto os Domingos e dias santos.
\textunderscore Dia santo\textunderscore  ou \textunderscore santificado\textunderscore , aquelle que, embora não seja Domingo, é consagrado ao culto religioso e em que a Igreja prohibe trabalhar; também se chama \textunderscore dia santo de guarda\textunderscore .
\textunderscore Dia santo dispensado\textunderscore , aquelle em que, embora consagrado pela Igreja a certas solennidades, não é por ella prohibido trabalhar.
\textunderscore Dia de juízo\textunderscore , aquelle em que, segundo a fé christan, as almas se hão de juntar aos corpos, para serem julgadas por Deus.
\textunderscore Dia nefasto\textunderscore , dia em que houve prantos e clamores.
\textunderscore Dia feriado\textunderscore , aquelle, em que não funccionam as escolas e as Repartições públicas.
\textunderscore Dia cheio\textunderscore , aquelle que se passou regaladamente.
\textunderscore Dia útil\textunderscore , o mesmo que \textunderscore dia de semana\textunderscore .
\textunderscore Dia lectivo\textunderscore , aquelle em que funccionam escolas.
\textunderscore Dia de annos\textunderscore , anniversário natalício.
\textunderscore Dia a dia\textunderscore , em todos os dias; successivamente.
\textunderscore Hoje em dia\textunderscore , nos tempos de agora.
\textunderscore Dia de anno bom\textunderscore , o primeiro dia do anno.
\textunderscore Dia de gala\textunderscore , dia que, por lei, é considerado como de festa nacional.
\textunderscore Dia de finados\textunderscore , o dia 2 de Novembro, em que a Igreja commemora e suffraga as almas dos defuntos.
Existência; vida: \textunderscore corriam-lhe os dias tranquillos\textunderscore .
Época.
\textunderscore Bons dias\textunderscore ! ou \textunderscore bom dia\textunderscore ! (para saudar ou cumprimentar, em quanto não é meio-dia).
\textunderscore De dias\textunderscore , que nasceu poucos dias antes: \textunderscore uma criança de dias\textunderscore .
Que succedeu poucos dias antes.
\textunderscore Homem de dias\textunderscore , homem velho.
\textunderscore Têr dias\textunderscore , estar ora bem, ora mal.
Apresentar aspectos differentes, segundo o tempo.
\textunderscore Mulher a dias\textunderscore , mulher que trabalha em casa alheia, recebendo salário diariamente.
\section{Dia...}
\begin{itemize}
\item {Grp. gram.:pref.}
\end{itemize}
\begin{itemize}
\item {Proveniência:(Do gr. \textunderscore dia\textunderscore )}
\end{itemize}
Através.
\section{Diaba}
\begin{itemize}
\item {Grp. gram.:f.}
\end{itemize}
Caixa cylíndrica de fibras de bordão, na Lunda.
\section{Diaba}
\begin{itemize}
\item {Grp. gram.:f.}
\end{itemize}
\begin{itemize}
\item {Utilização:Pop.}
\end{itemize}
A mulher do Diabo.
\section{Diábase}
\begin{itemize}
\item {Grp. gram.:f.}
\end{itemize}
\begin{itemize}
\item {Proveniência:(Lat. \textunderscore diabasis\textunderscore )}
\end{itemize}
Rocha massiça, granulosa, de côr verde-escura.
\section{Diabelha}
\begin{itemize}
\item {fónica:bê}
\end{itemize}
\begin{itemize}
\item {Grp. gram.:f.}
\end{itemize}
Planta plantagínea, (\textunderscore plantago coronopus\textunderscore , Lin.).
\section{Diabete}
\begin{itemize}
\item {Grp. gram.:m.  ou  f.}
\end{itemize}
\begin{itemize}
\item {Proveniência:(Gr. \textunderscore diabetes\textunderscore )}
\end{itemize}
Doença, caracterizada pela emissão de urinas abundantes e saccharinas.
\section{Diabetes}
\begin{itemize}
\item {Grp. gram.:f.}
\end{itemize}
(V.diabete)
\section{Diabético}
\begin{itemize}
\item {Grp. gram.:adj.}
\end{itemize}
\begin{itemize}
\item {Grp. gram.:M.}
\end{itemize}
Relativo á diabete.
Aquelle que padece diabete.
\section{Diabinho}
\begin{itemize}
\item {Grp. gram.:m.}
\end{itemize}
O mesmo que \textunderscore diabrete\textunderscore .
\section{Diabo}
\begin{itemize}
\item {Grp. gram.:m.}
\end{itemize}
\begin{itemize}
\item {Utilização:Fig.}
\end{itemize}
\begin{itemize}
\item {Utilização:T. da Bairrada}
\end{itemize}
\begin{itemize}
\item {Proveniência:(Do lat. \textunderscore diabolus\textunderscore )}
\end{itemize}
Gênio do mal, em geral.
Demónio.
Satanás.
Cada um dos anjos maus ou rebeldes.
Pessôa de mau gênio, de más qualidades, feia.
Variedade de pêra.
Um dos apparelhos das fábricas de tecidos.
Jôgo popular, o mesmo que \textunderscore diabrete\textunderscore .--Este t. faz parte de muitas loc. interjectivas, e é t. de comparação, de carácter vago, applicado amiúde, mormente em loc. fam.
\section{Diáboa}
\begin{itemize}
\item {Grp. gram.:f.}
\end{itemize}
\begin{itemize}
\item {Utilização:Des.}
\end{itemize}
Mulher endiabrada, muito viva e astuta. Cf. \textunderscore Eufrosina\textunderscore , 223.
(Cp. \textunderscore diaba\textunderscore ^2)
\section{Diabo-alma}
\begin{itemize}
\item {Grp. gram.:m.}
\end{itemize}
\begin{itemize}
\item {Utilização:Pop.}
\end{itemize}
Má pessôa.
Alma damnada. Cf. Camillo, \textunderscore Mystérios de Lisbôa\textunderscore , I, 135.
\section{Diabolicamente}
\begin{itemize}
\item {Grp. gram.:adv.}
\end{itemize}
De modo diabólico.
\section{Diabólico}
\begin{itemize}
\item {Grp. gram.:adj.}
\end{itemize}
\begin{itemize}
\item {Utilização:Fig.}
\end{itemize}
\begin{itemize}
\item {Proveniência:(Gr. \textunderscore diabolikos\textunderscore )}
\end{itemize}
Relativo ao diabo.
Que se attribue ao diabo.
Maligno.
Detestável.
Infernal.
Intrincado: \textunderscore problema diabólico\textunderscore .
\section{Diabótano}
\begin{itemize}
\item {Grp. gram.:m.}
\end{itemize}
\begin{itemize}
\item {Proveniência:(Gr. \textunderscore diabotanon\textunderscore )}
\end{itemize}
Emplastro vegetal, desusado.
\section{Diabra}
\begin{itemize}
\item {Grp. gram.:f.}
\end{itemize}
\begin{itemize}
\item {Utilização:Prov.}
\end{itemize}
\begin{itemize}
\item {Utilização:trasm.}
\end{itemize}
Sartan, em cujo fundo se deita sal e aguardente, e que os rapazes vão accender aos fiadoiros, depois de um têr ido previamente apagar a candeia, a fim de meter mêdo, pelo tom sinistro que a chamma da aguardente dá ao rosto das pessôas, em frente de quem ella se apresenta.
(Cp. cast. \textunderscore diabla\textunderscore )
\section{Diabre}
\begin{itemize}
\item {Grp. gram.:m.}
\end{itemize}
\begin{itemize}
\item {Utilização:Prov.}
\end{itemize}
\begin{itemize}
\item {Utilização:trasm.}
\end{itemize}
O mesmo que \textunderscore diabo\textunderscore .
\section{Diabrete}
\begin{itemize}
\item {fónica:brê}
\end{itemize}
\begin{itemize}
\item {Grp. gram.:m.}
\end{itemize}
\begin{itemize}
\item {Utilização:Fam.}
\end{itemize}
Pequeno diabo.
Criança travêssa.
Espécie de jôgo de cartas.
Apparelho das officinas de cardação, nas fábricas de tecidos. Cf. \textunderscore Diabo\textunderscore .
(Por \textunderscore diab'lete\textunderscore , contr. de \textunderscore diabolete\textunderscore , do lat. \textunderscore diabolus\textunderscore )
\section{Diabril}
\begin{itemize}
\item {Grp. gram.:adj.}
\end{itemize}
Relativo a diabrete.
Próprio de diabrete.
Travesso. Cf. Alves Mendes, \textunderscore Discursos\textunderscore , 59.
(Cp. \textunderscore diabrete\textunderscore )
\section{Diabrilmente}
\begin{itemize}
\item {Grp. gram.:adv.}
\end{itemize}
De modo diabril; diabolicamente.
\section{Diabrose}
\begin{itemize}
\item {Grp. gram.:f.}
\end{itemize}
\begin{itemize}
\item {Utilização:Med.}
\end{itemize}
\begin{itemize}
\item {Proveniência:(Gr. \textunderscore diabrosis\textunderscore )}
\end{itemize}
Erosão numa parte do corpo, por acção de substância corrosiva.
\section{Diabrótico}
\begin{itemize}
\item {Grp. gram.:adj.}
\end{itemize}
Relativo á diabrose; que produz diabrose.
\section{Diabrura}
\begin{itemize}
\item {Grp. gram.:f.}
\end{itemize}
Coisa diabólica.
Obra do diabo.
Transtôrno.
Travessura, traquinice.
Maldade.
(Cp. \textunderscore diabrete\textunderscore )
\section{Diacantho}
\begin{itemize}
\item {Grp. gram.:m.}
\end{itemize}
Gênero de plantas synanthéreas.
\section{Diacanto}
\begin{itemize}
\item {Grp. gram.:m.}
\end{itemize}
Gênero de plantas synanthéreas.
\section{Diacético}
\begin{itemize}
\item {Grp. gram.:adj.}
\end{itemize}
\begin{itemize}
\item {Utilização:Chím.}
\end{itemize}
Diz-se de um ácido, que se encontra nas urinas de alguns diabéticos.
\section{Diaceturia}
\begin{itemize}
\item {Grp. gram.:f.}
\end{itemize}
Existência do ácido diacético na urina.
\section{Diacho}
\begin{itemize}
\item {Grp. gram.:m.}
\end{itemize}
\begin{itemize}
\item {Utilização:Pop.}
\end{itemize}
O mesmo que \textunderscore diabo\textunderscore .
\section{Diachylão}
\begin{itemize}
\item {fónica:qui}
\end{itemize}
\begin{itemize}
\item {Grp. gram.:m.}
\end{itemize}
\begin{itemize}
\item {Proveniência:(Do gr. \textunderscore dia\textunderscore  + \textunderscore khulos\textunderscore )}
\end{itemize}
Espécie de emplastro agglutinativo.
\section{Diáclase}
\begin{itemize}
\item {Grp. gram.:f.}
\end{itemize}
\begin{itemize}
\item {Proveniência:(Do gr. \textunderscore dia\textunderscore  + \textunderscore klao\textunderscore )}
\end{itemize}
Grande fractura natural de rocha, uma das fórmas da lithóclase.
\section{Diacódio}
\begin{itemize}
\item {Grp. gram.:m.}
\end{itemize}
\begin{itemize}
\item {Proveniência:(Do gr. \textunderscore dia\textunderscore  + \textunderscore khodia\textunderscore )}
\end{itemize}
Xarope, preparado com cabeças de papoilas brancas.
\section{Diaconado}
\begin{itemize}
\item {Grp. gram.:m.}
\end{itemize}
O mesmo que \textunderscore diaconato\textunderscore .
\section{Diaconal}
\begin{itemize}
\item {Grp. gram.:adj.}
\end{itemize}
Relativo a diácono.
\section{Diaconato}
\begin{itemize}
\item {Grp. gram.:m.}
\end{itemize}
Dignidade de diácono.
(B. lat. \textunderscore diaconatus\textunderscore )
\section{Diaconia}
\begin{itemize}
\item {Grp. gram.:f.}
\end{itemize}
\begin{itemize}
\item {Utilização:Ant.}
\end{itemize}
Lugar, onde se estabeleciam diáconos, para distribuir esmolas.
(B. lat. \textunderscore diaconia\textunderscore )
\section{Diacónico}
\begin{itemize}
\item {Grp. gram.:m.}
\end{itemize}
\begin{itemize}
\item {Proveniência:(Lat. \textunderscore diaconicum\textunderscore )}
\end{itemize}
Cada um dos recintos, á direita e á esquerda da tribuna em as antigas basílicas.
Espécie de sacristia.
\section{Diaconisa}
\begin{itemize}
\item {Grp. gram.:f.}
\end{itemize}
Mulher, que a Igreja investia em funcções ecclesiásticas, análogas ás dos diáconos, especialmente para o baptismo das mulheres e para outros actos, em que o pudor aconselhava a ausência de homens.
Mulher de diácono, (na Igreja primitiva).
(B. lat. \textunderscore diaconissa\textunderscore )
\section{Diácono}
\begin{itemize}
\item {Grp. gram.:m.}
\end{itemize}
Clérigo, a que só falta a terceira ordem sacra e cuja principal attribuição é a leitura do Evangelho nas Missas cantadas.
(B. lat. \textunderscore diaconus\textunderscore )
\section{Diácope}
\begin{itemize}
\item {Grp. gram.:f.}
\end{itemize}
\begin{itemize}
\item {Proveniência:(Gr. \textunderscore diakope\textunderscore )}
\end{itemize}
Figura grammatical, o mesmo que \textunderscore tmese\textunderscore .
Incisão longitudinal num osso.
\section{Diacoprégia}
\begin{itemize}
\item {Grp. gram.:f.}
\end{itemize}
\begin{itemize}
\item {Proveniência:(Do gr. \textunderscore dia\textunderscore  + \textunderscore copron\textunderscore  + \textunderscore algos\textunderscore )}
\end{itemize}
Emplastro de excrementos de cabra, desusado.
\section{Diacosmético}
\begin{itemize}
\item {Grp. gram.:adj.}
\end{itemize}
\begin{itemize}
\item {Proveniência:(Do gr. \textunderscore dia\textunderscore  + \textunderscore kosmos\textunderscore )}
\end{itemize}
Dizia-se, na philosophia grega, do princípio regulador da ordem e da harmonia.
\section{Diacraniano}
\begin{itemize}
\item {Grp. gram.:adj.}
\end{itemize}
\begin{itemize}
\item {Utilização:Anat.}
\end{itemize}
\begin{itemize}
\item {Proveniência:(Do gr. \textunderscore dia\textunderscore  + \textunderscore kranion\textunderscore )}
\end{itemize}
Que se articula com o crânio.
\section{Diacrítico}
\begin{itemize}
\item {Grp. gram.:adj.}
\end{itemize}
\begin{itemize}
\item {Utilização:Gram.}
\end{itemize}
\begin{itemize}
\item {Proveniência:(Lat. \textunderscore diacriticus\textunderscore )}
\end{itemize}
Diz-se dos sinaes gráphicos, destinados especialmente a distinguir a modulação das vogaes ou a pronúncia de certas palavras, que, sem elles, offerecem confusão com outras.
\section{Diacústica}
\begin{itemize}
\item {Grp. gram.:f.}
\end{itemize}
\begin{itemize}
\item {Utilização:Phýs.}
\end{itemize}
\begin{itemize}
\item {Proveniência:(Do gr. \textunderscore dia\textunderscore  e \textunderscore acústica\textunderscore )}
\end{itemize}
Parte da acústica, que estuda a refracção e as propriedades do som, que passa de um fluido a outro.
\section{Diadelfia}
\begin{itemize}
\item {Grp. gram.:f.}
\end{itemize}
\begin{itemize}
\item {Utilização:Bot.}
\end{itemize}
União dos estames da flôr, que formam dois feixes.
(Cp. \textunderscore diadelfo\textunderscore )
\section{Diadelfo}
\begin{itemize}
\item {Grp. gram.:adj.}
\end{itemize}
\begin{itemize}
\item {Utilização:Bot.}
\end{itemize}
\begin{itemize}
\item {Proveniência:(De \textunderscore di...\textunderscore  + gr. \textunderscore adelphos\textunderscore , irmão)}
\end{itemize}
Diz-se dos estames, que estão reunidos pelos seus filetes em dois corpos.
\section{Diadelphia}
\begin{itemize}
\item {Grp. gram.:f.}
\end{itemize}
\begin{itemize}
\item {Utilização:Bot.}
\end{itemize}
União dos estames da flôr, que formam dois feixes.
(Cp. \textunderscore diadelpho\textunderscore )
\section{Diadelpho}
\begin{itemize}
\item {Grp. gram.:adj.}
\end{itemize}
\begin{itemize}
\item {Utilização:Bot.}
\end{itemize}
\begin{itemize}
\item {Proveniência:(De \textunderscore di...\textunderscore  + gr. \textunderscore adelphos\textunderscore , irmão)}
\end{itemize}
Diz-se dos estames, que estão reunidos pelos seus filetes em dois corpos.
\section{Diadema}
\begin{itemize}
\item {Grp. gram.:m.}
\end{itemize}
\begin{itemize}
\item {Proveniência:(Gr. \textunderscore diadema\textunderscore )}
\end{itemize}
Faixa ornamental, de metal ou de estôfo, com que os soberanos cingem a cabeça.
Ornato circular, com que as senhoras cingem o toucado.
Corôa; autoridade soberana.
\section{Diademado}
\begin{itemize}
\item {Grp. gram.:adj.}
\end{itemize}
Que tem diadema ou ornato semelhante.
\section{Diadérmico}
\begin{itemize}
\item {Grp. gram.:adj.}
\end{itemize}
\begin{itemize}
\item {Utilização:Med.}
\end{itemize}
O mesmo que \textunderscore endérmico\textunderscore .
\section{Diadexia}
\begin{itemize}
\item {fónica:csi}
\end{itemize}
\begin{itemize}
\item {Grp. gram.:f.}
\end{itemize}
\begin{itemize}
\item {Utilização:Med.}
\end{itemize}
\begin{itemize}
\item {Proveniência:(Do gr. \textunderscore diadexis\textunderscore )}
\end{itemize}
Transformação de uma doença em outra, de natureza differente e em differente órgão.
\section{Diadose}
\begin{itemize}
\item {Grp. gram.:f.}
\end{itemize}
\begin{itemize}
\item {Utilização:Med.}
\end{itemize}
Deminuição ou desaparecimento de uma doença.
\section{Diafa}
\begin{itemize}
\item {Grp. gram.:f.}
\end{itemize}
\begin{itemize}
\item {Utilização:T. caboverd}
\end{itemize}
\begin{itemize}
\item {Proveniência:(Do ár. \textunderscore diafa\textunderscore )}
\end{itemize}
Gratificação ou beberete, que se dá aos trabalhadores, depois de concluírem a sua tarefa.
Descante popular de homens e mulheres; côro.
\section{Diafenicão}
\begin{itemize}
\item {Grp. gram.:m.}
\end{itemize}
\begin{itemize}
\item {Utilização:Pharm.}
\end{itemize}
Electuário, que é um drástico enérgico e se emprega numa espécie de cólica, chamada \textunderscore cólica-dos-pintores\textunderscore .
\section{Diagal}
\begin{itemize}
\item {Grp. gram.:m.  e  adj.}
\end{itemize}
\begin{itemize}
\item {Utilização:Pop.}
\end{itemize}
O mesmo que \textunderscore diagalves\textunderscore .
\section{Diagalves}
\begin{itemize}
\item {Grp. gram.:m.  e  adj.}
\end{itemize}
Variedade de uva branca.
(Provavelmente de \textunderscore Diogo Alves\textunderscore , n. p.)
\section{Diagnose}
\begin{itemize}
\item {Grp. gram.:f.}
\end{itemize}
\begin{itemize}
\item {Proveniência:(Gr. \textunderscore diagnosis\textunderscore )}
\end{itemize}
Conhecimento das doenças, resultante da observação dos symptomas.
\section{Diagnosticar}
\begin{itemize}
\item {Grp. gram.:v. t.}
\end{itemize}
Fazer o diagnóstico de (uma doença).
\section{Diagnosticável}
\begin{itemize}
\item {Grp. gram.:adj.}
\end{itemize}
Que se póde diagnosticar.
\section{Diagnóstico}
\begin{itemize}
\item {Grp. gram.:m.}
\end{itemize}
\begin{itemize}
\item {Grp. gram.:Adj.}
\end{itemize}
\begin{itemize}
\item {Proveniência:(Gr. \textunderscore diagnostikos\textunderscore )}
\end{itemize}
Conhecimento ou determinação de uma doença, pela observação dos seus symptomas.
Conjunto dos symptomas, em que se funda essa determinação.
Relativo á diagnose.
\section{Diagometria}
\begin{itemize}
\item {Grp. gram.:f.}
\end{itemize}
\begin{itemize}
\item {Utilização:Phýs.}
\end{itemize}
Comparação das conductibilidades eléctricas.
(Cp. \textunderscore diagómetro\textunderscore )
\section{Diagométrico}
\begin{itemize}
\item {Grp. gram.:adj.}
\end{itemize}
Relativo á diagometria.
\section{Diagómetro}
\begin{itemize}
\item {Grp. gram.:m.}
\end{itemize}
\begin{itemize}
\item {Proveniência:(Do gr. \textunderscore diagein\textunderscore  + \textunderscore metron\textunderscore )}
\end{itemize}
Apparelho, para medir pequenas quantidades de electricidade e demomstrar a grande ductibilidade de certas substâncias.
\section{Diagonal}
\begin{itemize}
\item {Grp. gram.:adj.}
\end{itemize}
\begin{itemize}
\item {Utilização:Geom.}
\end{itemize}
\begin{itemize}
\item {Grp. gram.:F.}
\end{itemize}
\begin{itemize}
\item {Proveniência:(Lat. \textunderscore diagonalis\textunderscore )}
\end{itemize}
Diz-se da linha recta que vai de um ângulo a outro ângulo opposto, ou não adjacente, numa figura rectilínea.
Oblíquo.
Linha diagonal.
Direcção oblíqua.
\section{Diagonalmente}
\begin{itemize}
\item {Grp. gram.:adv.}
\end{itemize}
Á maneira de diagonal.
\section{Diágrafo}
\begin{itemize}
\item {Grp. gram.:m.}
\end{itemize}
\begin{itemize}
\item {Proveniência:(Do gr. \textunderscore dia\textunderscore  + \textunderscore graphein\textunderscore )}
\end{itemize}
Instrumento para traçar, por movimento contínuo, quaesquer figuras rectilíneas e curvilíneas.
\section{Diagrama}
\begin{itemize}
\item {Grp. gram.:f.}
\end{itemize}
\begin{itemize}
\item {Utilização:Mús.}
\end{itemize}
\begin{itemize}
\item {Utilização:ant.}
\end{itemize}
\begin{itemize}
\item {Proveniência:(Gr. \textunderscore diagramma\textunderscore )}
\end{itemize}
Representação por meio de linhas.
Delineação; bosquejo.
Tabella ou modêlo, em que se apresentava a extensão geral de um sistema, e que correspondia proximamente ao que hoje chamamos \textunderscore escala\textunderscore .
\section{Diagramma}
\begin{itemize}
\item {Grp. gram.:f.}
\end{itemize}
\begin{itemize}
\item {Utilização:Mús.}
\end{itemize}
\begin{itemize}
\item {Utilização:ant.}
\end{itemize}
\begin{itemize}
\item {Proveniência:(Gr. \textunderscore diagramma\textunderscore )}
\end{itemize}
Representação por meio de linhas.
Delineação; bosquejo.
Tabella ou modêlo, em que se apresentava a extensão geral de um systema, e que correspondia proximamente ao que hoje chamamos \textunderscore escala\textunderscore .
\section{Diágrapho}
\begin{itemize}
\item {Grp. gram.:m.}
\end{itemize}
\begin{itemize}
\item {Proveniência:(Do gr. \textunderscore dia\textunderscore  + \textunderscore graphein\textunderscore )}
\end{itemize}
Instrumento para traçar, por movimento contínuo, quaesquer figuras rectilíneas e curvilíneas.
\section{Dialectação}
\begin{itemize}
\item {Grp. gram.:f.}
\end{itemize}
\begin{itemize}
\item {Utilização:Neol.}
\end{itemize}
Conversão em dialecto.
Formação de dialecto. Cf. J. Ribeiro, \textunderscore Diccion. Gram.\textunderscore , 64.
\section{Dialectal}
\begin{itemize}
\item {Grp. gram.:adj.}
\end{itemize}
Relativo a dialecto.
\section{Dialéctica}
\begin{itemize}
\item {Grp. gram.:f.}
\end{itemize}
\begin{itemize}
\item {Proveniência:(Lat. \textunderscore dialectica\textunderscore )}
\end{itemize}
Arte de argumentar ou discutir.
Argumentação dialogada, segundo a philosophia antiga.
Segundo Platão, era o diálogo, como méthodo de investigação scientífica, e era a sciência das ideias e do sêr em si.
\section{Dialecticamente}
\begin{itemize}
\item {Grp. gram.:adv.}
\end{itemize}
\begin{itemize}
\item {Proveniência:(De \textunderscore dialéctico\textunderscore )}
\end{itemize}
Segundo as regras da dialéctica.
\section{Dialéctico}
\begin{itemize}
\item {Grp. gram.:adj.}
\end{itemize}
\begin{itemize}
\item {Grp. gram.:M.}
\end{itemize}
Relativo á dialéctica.
Aquelle que argumenta bem.
\section{Dialéctico}
\begin{itemize}
\item {Grp. gram.:adj.}
\end{itemize}
O mesmo que \textunderscore dialectal\textunderscore .
\section{Dialectismo}
\begin{itemize}
\item {Grp. gram.:m.}
\end{itemize}
\begin{itemize}
\item {Proveniência:(De \textunderscore dialecto\textunderscore )}
\end{itemize}
Emprêgo de palavras, de origem popular, usadas numa região e desconhecidas noutras, embora dentro do território em que se fala uma língua.
\section{Dialecto}
\begin{itemize}
\item {Grp. gram.:m.}
\end{itemize}
\begin{itemize}
\item {Proveniência:(Gr. \textunderscore dialektos\textunderscore )}
\end{itemize}
Linguagem peculiar a uma região, não differindo essencialmente da linguagem das regiões vizinhas.
Cada uma das línguas, que se consideram variedades do mesmo typo fundamental.
\section{Dialectologia}
\begin{itemize}
\item {Grp. gram.:f.}
\end{itemize}
Complexo de estudos ou conhecimentos à cêrca de dialectos.
(Cp. \textunderscore dialectólogo\textunderscore )
\section{Dialectólogo}
\begin{itemize}
\item {Grp. gram.:m.}
\end{itemize}
\begin{itemize}
\item {Proveniência:(Do gr. \textunderscore dialektos\textunderscore  + \textunderscore logos\textunderscore )}
\end{itemize}
Aquelle que se dedica ao estudo de dialectos.
Aquelle que é versado em dialectologia.
\section{Dialelo}
\begin{itemize}
\item {Grp. gram.:m.}
\end{itemize}
\begin{itemize}
\item {Utilização:Rhet.}
\end{itemize}
\begin{itemize}
\item {Proveniência:(Gr. \textunderscore diallelos\textunderscore )}
\end{itemize}
Espécie de antimetábole, como na expressão: \textunderscore é o mais rico dos sábios e o mais sábio dos ricos\textunderscore .
\section{Dialho}
\begin{itemize}
\item {Grp. gram.:m.}
\end{itemize}
\begin{itemize}
\item {Utilização:Pop.}
\end{itemize}
O mesmo que \textunderscore Diabo\textunderscore .
\section{Dialisador}
\begin{itemize}
\item {Grp. gram.:adj.}
\end{itemize}
\begin{itemize}
\item {Grp. gram.:M.}
\end{itemize}
Que dialisa.
Instrumento para dialisar.
\section{Dialisar}
\begin{itemize}
\item {Grp. gram.:v. t.}
\end{itemize}
\begin{itemize}
\item {Proveniência:(Do gr. \textunderscore dialuo\textunderscore )}
\end{itemize}
Separar (substâncias), que estão misturadas com outras, por meio de um instrumento, compôsto de pergaminho.
\section{Diálise}
\begin{itemize}
\item {Grp. gram.:f.}
\end{itemize}
\begin{itemize}
\item {Proveniência:(Gr. \textunderscore dialusis\textunderscore )}
\end{itemize}
Acto de dialisar.
Faculdade que possuem as membranas, de deixar passar através dos seus poros certas substâncias, com exclusão de outras. Cf. \textunderscore Museu Techn.\textunderscore , 90.
\section{Diallelo}
\begin{itemize}
\item {Grp. gram.:m.}
\end{itemize}
\begin{itemize}
\item {Utilização:Rhet.}
\end{itemize}
\begin{itemize}
\item {Proveniência:(Gr. \textunderscore diallelos\textunderscore )}
\end{itemize}
Espécie de antimetábole, como na expressão: \textunderscore é o mais rico dos sábios e o mais sábio dos ricos\textunderscore .
\section{Dialoés}
\begin{itemize}
\item {Grp. gram.:m.}
\end{itemize}
\begin{itemize}
\item {Utilização:Pharm.}
\end{itemize}
Medicamento, em que entra o aloés.
\section{Dialogador}
\begin{itemize}
\item {Grp. gram.:m.}
\end{itemize}
Aquelle que dialoga.
\section{Dialogal}
\begin{itemize}
\item {Grp. gram.:adj.}
\end{itemize}
Relativo a diálogo.
\section{Dialogalmente}
\begin{itemize}
\item {Grp. gram.:adv.}
\end{itemize}
\begin{itemize}
\item {Proveniência:(De \textunderscore dialogal\textunderscore )}
\end{itemize}
Em fórma de diálogo.
\section{Dialogar}
\begin{itemize}
\item {Grp. gram.:v. t.}
\end{itemize}
\begin{itemize}
\item {Grp. gram.:V. i.}
\end{itemize}
\begin{itemize}
\item {Proveniência:(De \textunderscore diálogo\textunderscore )}
\end{itemize}
Dizer ou escrever em fórma de diálogo.
Conversar.
\section{Dialogia}
\begin{itemize}
\item {Grp. gram.:f.}
\end{itemize}
\begin{itemize}
\item {Utilização:Rhet.}
\end{itemize}
Emprêgo repetido da mesma palavra em sentidos differentes.
(Cp. \textunderscore diálogo\textunderscore )
\section{Dialógico}
\begin{itemize}
\item {Grp. gram.:adj.}
\end{itemize}
O mesmo que \textunderscore dialogal\textunderscore .
\section{Dialogismo}
\begin{itemize}
\item {Grp. gram.:m.}
\end{itemize}
\begin{itemize}
\item {Proveniência:(Gr. \textunderscore dialogismos\textunderscore )}
\end{itemize}
Arte de dialogar.
\section{Dialogista}
\begin{itemize}
\item {Grp. gram.:f. m.}
\end{itemize}
\begin{itemize}
\item {Proveniência:(Gr. \textunderscore dialogistes\textunderscore )}
\end{itemize}
Aquelle que escreve diálogos.
Aquelle que discute bem; o mesmo que \textunderscore dialéctico\textunderscore ^1.
\section{Dialogístico}
\begin{itemize}
\item {Grp. gram.:adj.}
\end{itemize}
(V.dialogal)
\section{Dialogita}
\begin{itemize}
\item {Grp. gram.:f.}
\end{itemize}
O mesmo que \textunderscore dialogito\textunderscore .
\section{Dialogito}
\begin{itemize}
\item {Grp. gram.:m.}
\end{itemize}
\begin{itemize}
\item {Utilização:Miner.}
\end{itemize}
\begin{itemize}
\item {Proveniência:(Do gr. \textunderscore dialoge\textunderscore )}
\end{itemize}
Carbonato de manganés.
\section{Diálogo}
\begin{itemize}
\item {Grp. gram.:m.}
\end{itemize}
\begin{itemize}
\item {Utilização:Ext.}
\end{itemize}
\begin{itemize}
\item {Proveniência:(Gr. \textunderscore dialogos\textunderscore )}
\end{itemize}
Conversação entre duas pessôas.
Conversação entre muitas pessôas.
Obra literária ou scientífica em fórma dialogada.
\section{Dialysador}
\begin{itemize}
\item {Grp. gram.:adj.}
\end{itemize}
\begin{itemize}
\item {Grp. gram.:M.}
\end{itemize}
Que dialysa.
Instrumento para dialysar.
\section{Dialysar}
\begin{itemize}
\item {Grp. gram.:v. t.}
\end{itemize}
\begin{itemize}
\item {Proveniência:(Do gr. \textunderscore dialuo\textunderscore )}
\end{itemize}
Separar (substâncias), que estão misturadas com outras, por meio de um instrumento, compôsto de pergaminho.
\section{Diályse}
\begin{itemize}
\item {Grp. gram.:f.}
\end{itemize}
\begin{itemize}
\item {Proveniência:(Gr. \textunderscore dialusis\textunderscore )}
\end{itemize}
Acto de dialysar.
Faculdade que possuem as membranas, de deixar passar através dos seus poros certas substâncias, com exclusão de outras. Cf. \textunderscore Museu Techn.\textunderscore , 90.
\section{Diamagnético}
\begin{itemize}
\item {Grp. gram.:adj.}
\end{itemize}
\begin{itemize}
\item {Proveniência:(De \textunderscore dia...\textunderscore  + \textunderscore magnético\textunderscore )}
\end{itemize}
Que é repellido pelos magnetes.
\section{Diamagnetismo}
\begin{itemize}
\item {Grp. gram.:m.}
\end{itemize}
\begin{itemize}
\item {Proveniência:(De \textunderscore dia...\textunderscore  + \textunderscore magnetismo\textunderscore )}
\end{itemize}
Tratado de phenómenos dos corpos diamagnéticos.
\section{Diamantado}
\begin{itemize}
\item {Grp. gram.:adj.}
\end{itemize}
(V.adiamantado)
\section{Diamante}
\begin{itemize}
\item {Grp. gram.:m.}
\end{itemize}
\begin{itemize}
\item {Grp. gram.:Adj.}
\end{itemize}
\begin{itemize}
\item {Proveniência:(Do gr. \textunderscore adamas\textunderscore )}
\end{itemize}
A mais dura e brilhante pedra preciosa, crystal de carbóne puro.
Diz-se das edições ou impressões, de pequeno formato, e feitas com caracteres muito finos, mas muito nitidos e de apparência elegante.
\section{Diamante-rosa}
\begin{itemize}
\item {Grp. gram.:m.}
\end{itemize}
Diamante que tem muitas facetas. Cf. \textunderscore Luz e Calor\textunderscore , 309.
\section{Diamântico}
\begin{itemize}
\item {Grp. gram.:adj.}
\end{itemize}
O mesmo que \textunderscore diamantino\textunderscore . Cf. Camillo, \textunderscore Cav. em Ruínas\textunderscore , 8.
\section{Diamantífero}
\begin{itemize}
\item {Grp. gram.:adj.}
\end{itemize}
\begin{itemize}
\item {Proveniência:(De \textunderscore diamante\textunderscore  + lat. \textunderscore ferre\textunderscore )}
\end{itemize}
Em que há diamantes (falando-se de terrenos).
\section{Diamantinamente}
\begin{itemize}
\item {Grp. gram.:adv.}
\end{itemize}
\begin{itemize}
\item {Proveniência:(De \textunderscore diamantino\textunderscore )}
\end{itemize}
Á maneira do diamante.
\section{Diamantino}
\begin{itemize}
\item {Grp. gram.:adj.}
\end{itemize}
\begin{itemize}
\item {Utilização:Fig.}
\end{itemize}
\begin{itemize}
\item {Proveniência:(De \textunderscore diamante\textunderscore )}
\end{itemize}
Que na dureza e brilho se parece ao diamante.
Precioso: \textunderscore carácter diamantino\textunderscore .
\section{Diamantista}
\begin{itemize}
\item {Grp. gram.:m.}
\end{itemize}
Aquelle que trabalha em diamantes.
Aquelle que compra e vende diamantes.
\section{Diamantizar}
\begin{itemize}
\item {Grp. gram.:v. t.}
\end{itemize}
\begin{itemize}
\item {Utilização:Fig.}
\end{itemize}
\begin{itemize}
\item {Proveniência:(De \textunderscore diamante\textunderscore )}
\end{itemize}
Tornar precioso.
Dar grande valor a. Cf. Alves Mendes, \textunderscore Discursos\textunderscore , 14.
\section{Diamantoide}
\begin{itemize}
\item {Grp. gram.:m.}
\end{itemize}
\begin{itemize}
\item {Proveniência:(De \textunderscore diamante\textunderscore  + gr. \textunderscore eidos\textunderscore )}
\end{itemize}
Pedra, que tem todas as propriedades do diamante, á excepção do brilho, e cujo pó serve para polir pedras preciosas.
\section{Diamão}
\begin{itemize}
\item {Grp. gram.:m.}
\end{itemize}
\begin{itemize}
\item {Utilização:Ant.}
\end{itemize}
O mesmo que \textunderscore diamante\textunderscore . Cf. \textunderscore Chancell. de D. João III\textunderscore , na Tôrre do Tombo.
\section{Diamba}
\begin{itemize}
\item {Grp. gram.:f.}
\end{itemize}
\begin{itemize}
\item {Utilização:Bras. do N}
\end{itemize}
Espécie de cânhamo.
\section{Diambe}
\begin{itemize}
\item {Grp. gram.:m.}
\end{itemize}
Espécie de pomba africana.
\section{Diametral}
\begin{itemize}
\item {Grp. gram.:adj.}
\end{itemize}
Concernente a diâmetro.
\section{Diametralmente}
\begin{itemize}
\item {Grp. gram.:adv.}
\end{itemize}
\begin{itemize}
\item {Proveniência:(De \textunderscore diametral\textunderscore )}
\end{itemize}
De modo análogo á posição dos pontos extremos do diâmetro.
\section{Diâmetro}
\begin{itemize}
\item {Grp. gram.:m.}
\end{itemize}
\begin{itemize}
\item {Proveniência:(Gr. \textunderscore diametros\textunderscore )}
\end{itemize}
Linha recta que passa pelo centro de um círculo, terminando de ambos os lados na círcunferência ou peripheria.
\section{Diamido}
\begin{itemize}
\item {Grp. gram.:m.}
\end{itemize}
Corpo resultante da perda de duas ou quatro moléculas de água do oxalato de ammoníaco neutro.
\section{Diaminas}
\begin{itemize}
\item {Grp. gram.:f. pl.}
\end{itemize}
\begin{itemize}
\item {Utilização:Chím.}
\end{itemize}
\begin{itemize}
\item {Proveniência:(De \textunderscore di...\textunderscore  + \textunderscore aminas\textunderscore )}
\end{itemize}
Aminas, derivadas de duas moléculas de ammoníaco condensadas.
\section{Diamoro}
\begin{itemize}
\item {Grp. gram.:m.}
\end{itemize}
\begin{itemize}
\item {Proveniência:(De \textunderscore di...\textunderscore  + \textunderscore amora\textunderscore )}
\end{itemize}
Xarope de amoras.
\section{Diandria}
\begin{itemize}
\item {Grp. gram.:f.}
\end{itemize}
\begin{itemize}
\item {Proveniência:(De \textunderscore diandro\textunderscore )}
\end{itemize}
Qualidade dos vegetaes diandros.
Conjunto dêsses vegetaes.
\section{Diandro}
\begin{itemize}
\item {Grp. gram.:adj.}
\end{itemize}
\begin{itemize}
\item {Utilização:Bot.}
\end{itemize}
\begin{itemize}
\item {Proveniência:(Do gr. \textunderscore dis\textunderscore  + \textunderscore aner\textunderscore )}
\end{itemize}
Que tem dois estames.
\section{Dianela}
\begin{itemize}
\item {Grp. gram.:f.}
\end{itemize}
Gênero de plantas asparagíneas.
\section{Dianella}
\begin{itemize}
\item {Grp. gram.:f.}
\end{itemize}
Gênero de plantas asparagíneas.
\section{Dianho}
\begin{itemize}
\item {Grp. gram.:m.}
\end{itemize}
\begin{itemize}
\item {Utilização:Prov.}
\end{itemize}
O mesmo que \textunderscore Diabo\textunderscore .
\section{Diante}
\textunderscore adv.\textunderscore  (e der.)
(V. \textunderscore deante\textunderscore , etc.)
\section{Diantho}
\begin{itemize}
\item {Utilização:Bot.}
\end{itemize}
\begin{itemize}
\item {Proveniência:(Do gr. \textunderscore dis\textunderscore  + \textunderscore anthos\textunderscore )}
\end{itemize}
Que tem duas flôres.
Formado de duas flôres.
\section{Diantho}
\begin{itemize}
\item {Grp. gram.:m.}
\end{itemize}
\begin{itemize}
\item {Proveniência:(Do gr. \textunderscore dios\textunderscore  + \textunderscore anthos\textunderscore )}
\end{itemize}
Tribo de plantas caryophylláceas, a que pertencem os cravos, as mauritânias, etc.
\section{Dianto}
\begin{itemize}
\item {Utilização:Bot.}
\end{itemize}
\begin{itemize}
\item {Proveniência:(Do gr. \textunderscore dis\textunderscore  + \textunderscore anthos\textunderscore )}
\end{itemize}
Que tem duas flôres.
Formado de duas flôres.
\section{Dianto}
\begin{itemize}
\item {Grp. gram.:m.}
\end{itemize}
\begin{itemize}
\item {Proveniência:(Do gr. \textunderscore dios\textunderscore  + \textunderscore anthos\textunderscore )}
\end{itemize}
Tribo de plantas cariofyláceas, a que pertencem os cravos, as mauritânias, etc.
\section{Dianuco}
\begin{itemize}
\item {Grp. gram.:m.}
\end{itemize}
\begin{itemize}
\item {Proveniência:(De \textunderscore dia...\textunderscore  + lat. \textunderscore nux\textunderscore )}
\end{itemize}
Arrobe, feito de mel e nozes verdes.
\section{Diapalmo}
\begin{itemize}
\item {Grp. gram.:m.}
\end{itemize}
\begin{itemize}
\item {Proveniência:(De \textunderscore dia...\textunderscore  + \textunderscore palma\textunderscore )}
\end{itemize}
Unguento deseccante.
\section{Diapasão}
\begin{itemize}
\item {Grp. gram.:m.}
\end{itemize}
\begin{itemize}
\item {Utilização:Mús.}
\end{itemize}
\begin{itemize}
\item {Proveniência:(Do gr. \textunderscore dia\textunderscore  + \textunderscore pason\textunderscore )}
\end{itemize}
Extensão das notas de uma voz ou de um instrumento, entre o som mais grave e o mais agudo.
Lamiré.
Intervallo de oitava.
Totalidade dos sons praticáveis em cada voz ou em cada instrumento.
A nota, estabelecida fixamente pelo lamiré.
\section{Diafaneidade}
\begin{itemize}
\item {Grp. gram.:f.}
\end{itemize}
Propriedade daquilo que é diáfano; transparência.
\section{Diáfano}
\begin{itemize}
\item {Grp. gram.:adj.}
\end{itemize}
\begin{itemize}
\item {Utilização:Fig.}
\end{itemize}
\begin{itemize}
\item {Proveniência:(Gr. \textunderscore diaphanos\textunderscore )}
\end{itemize}
Que, sendo compacto, dá passagem á luz.
Transparente.
Magro.
\section{Diafanógeno}
\begin{itemize}
\item {Grp. gram.:adj.}
\end{itemize}
\begin{itemize}
\item {Utilização:Phýs.}
\end{itemize}
\begin{itemize}
\item {Proveniência:(Do gr. \textunderscore diaphanos\textunderscore  + \textunderscore genes\textunderscore )}
\end{itemize}
Que produz diafaneidade.
\section{Diafanometria}
\begin{itemize}
\item {Grp. gram.:f.}
\end{itemize}
Medida da diafaneidade.
\section{Diafanométrico}
\begin{itemize}
\item {Grp. gram.:adj.}
\end{itemize}
Relativo á diafanometria.
\section{Diafanómetro}
\begin{itemize}
\item {Grp. gram.:m.}
\end{itemize}
\begin{itemize}
\item {Proveniência:(Do gr. \textunderscore diaphanes\textunderscore  + \textunderscore metron\textunderscore )}
\end{itemize}
Aparelho, para avaliar as variações da transparência do ar.
\section{Diafanorama}
\begin{itemize}
\item {Grp. gram.:m.}
\end{itemize}
\begin{itemize}
\item {Proveniência:(Do gr. \textunderscore diaphanes\textunderscore  + \textunderscore orama\textunderscore )}
\end{itemize}
Quadro de uma cidade ou região, representado em perspectiva e convenientemente iluminado.
\section{Diafisário}
\begin{itemize}
\item {Grp. gram.:adj.}
\end{itemize}
Relativo á diáfise.
\section{Diafise}
\begin{itemize}
\item {Grp. gram.:f.}
\end{itemize}
\begin{itemize}
\item {Proveniência:(Gr. \textunderscore diaphusis\textunderscore )}
\end{itemize}
Parte média de um osso comprido.
Separação.
\section{Diafonia}
\begin{itemize}
\item {Grp. gram.:f.}
\end{itemize}
\begin{itemize}
\item {Proveniência:(Do gr. \textunderscore dia\textunderscore  + \textunderscore phone\textunderscore )}
\end{itemize}
Intervalo dissonante, na música grega.
\section{Diáfora}
\begin{itemize}
\item {Grp. gram.:f.}
\end{itemize}
\begin{itemize}
\item {Utilização:Rhet.}
\end{itemize}
\begin{itemize}
\item {Proveniência:(Gr. \textunderscore diaphora\textunderscore )}
\end{itemize}
Repetição de uma palavra em sentidos diferentes.
\section{Diaforese}
\begin{itemize}
\item {Grp. gram.:f.}
\end{itemize}
\begin{itemize}
\item {Proveniência:(Gr. \textunderscore diaphoresis\textunderscore )}
\end{itemize}
Transpiração.
Suor abundante.
\section{Diaforético}
\begin{itemize}
\item {Grp. gram.:m.}
\end{itemize}
\begin{itemize}
\item {Grp. gram.:Adj.}
\end{itemize}
Medicamento sudorífico.
Relativo a diaforese.
\section{Diafragma}
\begin{itemize}
\item {Grp. gram.:m.}
\end{itemize}
\begin{itemize}
\item {Proveniência:(Gr. \textunderscore diaphragma\textunderscore )}
\end{itemize}
Músculo largo, entre o peito e o abdome.
Divisão transversal, que separa um fruto capsular.
Placa ou outro objecto, que divide duas cavidades, que intercepta a luz ou a transmissão do calor, etc.
Chapa perfurada, usada em aparelhos ópticos.
\section{Diafragmático}
\begin{itemize}
\item {Grp. gram.:adj.}
\end{itemize}
Relativo ao diafragma.
\section{Diafragmite}
\begin{itemize}
\item {Grp. gram.:f.}
\end{itemize}
Inflamação do diafragma.
\section{Diapédese}
\begin{itemize}
\item {Grp. gram.:f.}
\end{itemize}
\begin{itemize}
\item {Utilização:Med.}
\end{itemize}
\begin{itemize}
\item {Proveniência:(Gr. \textunderscore diapedesis\textunderscore )}
\end{itemize}
Saída, para fóra dos vasos, dos glóbulos brancos do sangue.
\section{Diapente}
\begin{itemize}
\item {Grp. gram.:m.}
\end{itemize}
\begin{itemize}
\item {Proveniência:(Do gr. \textunderscore dia\textunderscore  + \textunderscore pente\textunderscore )}
\end{itemize}
Espaço de uma quinta, na música.
\section{Diápero}
\begin{itemize}
\item {Grp. gram.:m.}
\end{itemize}
\begin{itemize}
\item {Proveniência:(Do gr. \textunderscore diaperein\textunderscore )}
\end{itemize}
Gênero de insectos coleópteros heterómeros.
\section{Diaphaneidade}
\begin{itemize}
\item {Grp. gram.:f.}
\end{itemize}
Propriedade daquillo que é diáphano; transparência.
\section{Diáphano}
\begin{itemize}
\item {Grp. gram.:adj.}
\end{itemize}
\begin{itemize}
\item {Utilização:Fig.}
\end{itemize}
\begin{itemize}
\item {Proveniência:(Gr. \textunderscore diaphanos\textunderscore )}
\end{itemize}
Que, sendo compacto, dá passagem á luz.
Transparente.
Magro.
\section{Diaphanógeno}
\begin{itemize}
\item {Grp. gram.:adj.}
\end{itemize}
\begin{itemize}
\item {Utilização:Phýs.}
\end{itemize}
\begin{itemize}
\item {Proveniência:(Do gr. \textunderscore diaphanos\textunderscore  + \textunderscore genes\textunderscore )}
\end{itemize}
Que produz diaphaneidade.
\section{Diaphanometria}
\begin{itemize}
\item {Grp. gram.:f.}
\end{itemize}
Medida da diaphaneidade.
\section{Diaphanométrico}
\begin{itemize}
\item {Grp. gram.:adj.}
\end{itemize}
Relativo á diaphanometria.
\section{Diaphanómetro}
\begin{itemize}
\item {Grp. gram.:m.}
\end{itemize}
\begin{itemize}
\item {Proveniência:(Do gr. \textunderscore diaphanes\textunderscore  + \textunderscore metron\textunderscore )}
\end{itemize}
Apparelho, para avaliar as variações da transparência do ar.
\section{Diaphanorama}
\begin{itemize}
\item {Grp. gram.:m.}
\end{itemize}
\begin{itemize}
\item {Proveniência:(Do gr. \textunderscore diaphanes\textunderscore  + \textunderscore orama\textunderscore )}
\end{itemize}
Quadro de uma cidade ou região, representado em perspectiva e convenientemente illuminado.
\section{Diaphonia}
\begin{itemize}
\item {Grp. gram.:f.}
\end{itemize}
\begin{itemize}
\item {Proveniência:(Do gr. \textunderscore dia\textunderscore  + \textunderscore phone\textunderscore )}
\end{itemize}
Intervallo dissonante, na música grega.
\section{Diáphora}
\begin{itemize}
\item {Grp. gram.:f.}
\end{itemize}
\begin{itemize}
\item {Utilização:Rhet.}
\end{itemize}
\begin{itemize}
\item {Proveniência:(Gr. \textunderscore diaphora\textunderscore )}
\end{itemize}
Repetição de uma palavra em sentidos differentes.
\section{Diaphorese}
\begin{itemize}
\item {Grp. gram.:f.}
\end{itemize}
\begin{itemize}
\item {Proveniência:(Gr. \textunderscore diaphoresis\textunderscore )}
\end{itemize}
Transpiração.
Suor abundante.
\section{Diaphorético}
\begin{itemize}
\item {Grp. gram.:m.}
\end{itemize}
\begin{itemize}
\item {Grp. gram.:Adj.}
\end{itemize}
Medicamento sudorífico.
Relativo a diaphorese.
\section{Diaphragma}
\begin{itemize}
\item {Grp. gram.:m.}
\end{itemize}
\begin{itemize}
\item {Proveniência:(Gr. \textunderscore diaphragma\textunderscore )}
\end{itemize}
Músculo largo, entre o peito e o abdome.
Divisão transversal, que separa um fruto capsular.
Placa ou outro objecto, que divide duas cavidades, que intercepta a luz ou a transmissão do calor, etc.
Chapa perfurada, usada em apparelhos ópticos.
\section{Diaphragmático}
\begin{itemize}
\item {Grp. gram.:adj.}
\end{itemize}
Relativo ao diaphragma.
\section{Diaphragmite}
\begin{itemize}
\item {Grp. gram.:f.}
\end{itemize}
Inflammação do diaphragma.
\section{Diaphysário}
\begin{itemize}
\item {Grp. gram.:adj.}
\end{itemize}
Relativo á diáphyse.
\section{Diaphyse}
\begin{itemize}
\item {Grp. gram.:f.}
\end{itemize}
\begin{itemize}
\item {Proveniência:(Gr. \textunderscore diaphusis\textunderscore )}
\end{itemize}
Parte média de um osso comprido.
Separação.
\section{Diapiético}
\begin{itemize}
\item {Grp. gram.:adj.}
\end{itemize}
\begin{itemize}
\item {Proveniência:(Gr. \textunderscore diapuetikos\textunderscore )}
\end{itemize}
Que promove a supuração.
\section{Diapnoico}
\begin{itemize}
\item {Grp. gram.:adj.}
\end{itemize}
\begin{itemize}
\item {Grp. gram.:M.}
\end{itemize}
\begin{itemize}
\item {Proveniência:(Do gr. \textunderscore diapnoia\textunderscore )}
\end{itemize}
Que excita uma ligeira transpiração.
Medicamento diapnoico.
\section{Diaptose}
\begin{itemize}
\item {Grp. gram.:f.}
\end{itemize}
\begin{itemize}
\item {Proveniência:(Gr. \textunderscore diaptosis\textunderscore )}
\end{itemize}
Intercadência, na música.
\section{Diapyético}
\begin{itemize}
\item {Grp. gram.:adj.}
\end{itemize}
\begin{itemize}
\item {Proveniência:(Gr. \textunderscore diapuetikos\textunderscore )}
\end{itemize}
Que promove a supuração.
\section{Diaquilão}
\begin{itemize}
\item {Grp. gram.:m.}
\end{itemize}
\begin{itemize}
\item {Proveniência:(Do gr. \textunderscore dia\textunderscore  + \textunderscore khulos\textunderscore )}
\end{itemize}
Espécie de emplastro aglutinativo.
\section{Diaras}
\begin{itemize}
\item {Grp. gram.:m. pl.}
\end{itemize}
Uma das tríbos indígenas do Sudão.
\section{Diária}
\begin{itemize}
\item {Grp. gram.:f.}
\end{itemize}
\begin{itemize}
\item {Proveniência:(De \textunderscore diário\textunderscore )}
\end{itemize}
Ganho, correspondente ao trabalho de um dia.
Ração diária.
\section{Diariamente}
\begin{itemize}
\item {Grp. gram.:adv.}
\end{itemize}
\begin{itemize}
\item {Proveniência:(De \textunderscore diário\textunderscore )}
\end{itemize}
Em cada dia.
Todos os dias.
\section{Diário}
\begin{itemize}
\item {Grp. gram.:adj.}
\end{itemize}
\begin{itemize}
\item {Grp. gram.:M.}
\end{itemize}
\begin{itemize}
\item {Utilização:Fam.}
\end{itemize}
\begin{itemize}
\item {Proveniência:(Lat. \textunderscore diarium\textunderscore )}
\end{itemize}
Que se faz todos os dias.
Que succede todos os dias.
Quotidiano.
Relação do que se faz ou daquillo que succede em cada dia.
Jornal ou periódico, que se publica todos os dias.
Despesa diária.
\section{Diarista}
\begin{itemize}
\item {Grp. gram.:m.}
\end{itemize}
\begin{itemize}
\item {Utilização:P. us.}
\end{itemize}
\begin{itemize}
\item {Proveniência:(De \textunderscore diário\textunderscore )}
\end{itemize}
Redactor de uma fôlha diária.
\section{Diarista}
\begin{itemize}
\item {Grp. gram.:m.}
\end{itemize}
\begin{itemize}
\item {Utilização:Bras}
\end{itemize}
\begin{itemize}
\item {Proveniência:(De \textunderscore diária\textunderscore )}
\end{itemize}
Trabalhador, que não tem vencimento fixo, ganhando só nos dias em que trabalha.
\section{Diarizar}
\begin{itemize}
\item {Grp. gram.:v. t.}
\end{itemize}
\begin{itemize}
\item {Proveniência:(De \textunderscore diário\textunderscore )}
\end{itemize}
Tornar diário.
Fazer diariamente. Cf. Filinto, XVIII, 261.
\section{Diarreia}
\begin{itemize}
\item {Grp. gram.:f.}
\end{itemize}
\begin{itemize}
\item {Proveniência:(Gr. \textunderscore diarrhoia\textunderscore )}
\end{itemize}
Evacuação de ventre, líquida e frequente; fluxo de ventre.
\section{Diarreico}
\begin{itemize}
\item {Grp. gram.:adj.}
\end{itemize}
\begin{itemize}
\item {Grp. gram.:M.}
\end{itemize}
Relativo á diarreia.
Aquele que padece diarreia.
\section{Diarrheia}
\begin{itemize}
\item {Grp. gram.:f.}
\end{itemize}
\begin{itemize}
\item {Proveniência:(Gr. \textunderscore diarrhoia\textunderscore )}
\end{itemize}
Evacuação de ventre, líquida e frequente; fluxo de ventre.
\section{Diarthrose}
\begin{itemize}
\item {Grp. gram.:f.}
\end{itemize}
\begin{itemize}
\item {Proveniência:(Gr. \textunderscore diarthrosis\textunderscore )}
\end{itemize}
Articulação, que permitte o movimento dos ossos em todos os sentidos.
\section{Diartrose}
\begin{itemize}
\item {Grp. gram.:f.}
\end{itemize}
\begin{itemize}
\item {Proveniência:(Gr. \textunderscore diarthrosis\textunderscore )}
\end{itemize}
Articulação, que permitte o movimento dos ossos em todos os sentidos.
\section{Diascevasta}
\begin{itemize}
\item {Grp. gram.:m.}
\end{itemize}
\begin{itemize}
\item {Proveniência:(Gr. \textunderscore diaskeuastes\textunderscore )}
\end{itemize}
Crítico, que revê e corrige obras alheias.--Deu-se especialmente êste nome aos críticos alexandrinos, que discutiram textos homéricos, sua correcção, authenticidade, etc.
\section{Diáscia}
\begin{itemize}
\item {Grp. gram.:f.}
\end{itemize}
Planta irídea.
\section{Diasco}
\begin{itemize}
\item {Grp. gram.:m.}
\end{itemize}
\begin{itemize}
\item {Utilização:Prov.}
\end{itemize}
\begin{itemize}
\item {Utilização:trasm.}
\end{itemize}
Diabo, dianho, demo.
\section{Diasóstica}
\begin{itemize}
\item {Grp. gram.:f.}
\end{itemize}
\begin{itemize}
\item {Utilização:Des.}
\end{itemize}
\begin{itemize}
\item {Proveniência:(De \textunderscore diasóstico\textunderscore )}
\end{itemize}
Arte de conservar a saúde.
Hygiene.
\section{Diasóstico}
\begin{itemize}
\item {Grp. gram.:adj.}
\end{itemize}
\begin{itemize}
\item {Utilização:Des.}
\end{itemize}
\begin{itemize}
\item {Proveniência:(Gr. \textunderscore diasostikos\textunderscore )}
\end{itemize}
Relativo á conservação da saúde.
Hygiênico.
\section{Diáspide}
\begin{itemize}
\item {Grp. gram.:f.}
\end{itemize}
\begin{itemize}
\item {Proveniência:(Do gr. \textunderscore dia\textunderscore  + \textunderscore aspis\textunderscore )}
\end{itemize}
Gênero de insectos hemípteros.
\section{Diásporo}
\begin{itemize}
\item {Grp. gram.:m.}
\end{itemize}
\begin{itemize}
\item {Proveniência:(Gr. \textunderscore diaspora\textunderscore )}
\end{itemize}
Mineral raro, espécie de jaspe.
\section{Diasporometria}
\begin{itemize}
\item {Grp. gram.:f.}
\end{itemize}
Emprêgo do diasporómetro.
\section{Diasporómetro}
\begin{itemize}
\item {Grp. gram.:m.}
\end{itemize}
\begin{itemize}
\item {Proveniência:(Do gr. \textunderscore diaspora\textunderscore  + \textunderscore metron\textunderscore )}
\end{itemize}
Instrumento de Phýsica, que mede o ângulo necessário paraestabelecer o achromatismo de dois prismas de crystaes differentes.
\section{Diaspro}
\begin{itemize}
\item {Grp. gram.:m.}
\end{itemize}
(V.diásporo)
\section{Diastáltico}
\begin{itemize}
\item {Grp. gram.:adj.}
\end{itemize}
\begin{itemize}
\item {Utilização:Anat.}
\end{itemize}
\begin{itemize}
\item {Proveniência:(Gr. \textunderscore diastaltikos\textunderscore )}
\end{itemize}
Dá-se o nome de arcos diastálticos a um conjunto de nervos, que se suppõe saírem da medulla espinhal, (motores), entrarem nella, (sensitivos), e como que unirem-se através da medulla, para fazer contrahir os músculos.
Dizia-se do gênero de melopeia, que, entre os Gregos, era destinado a incutir sentimentos generosos e coragem.
\section{Diástase}
\begin{itemize}
\item {Grp. gram.:f.}
\end{itemize}
\begin{itemize}
\item {Utilização:Anat.}
\end{itemize}
\begin{itemize}
\item {Proveniência:(Gr. \textunderscore diastasis\textunderscore )}
\end{itemize}
Desvio ou deslocação de dois ossos, que têm articulação contígua.
Substância vegetal, que faz fermentar o amido, separando nelle a parte gomosa da tegumentar.
\section{Diastasímetro}
\begin{itemize}
\item {Grp. gram.:m.}
\end{itemize}
\begin{itemize}
\item {Proveniência:(Do gr. \textunderscore diastasis\textunderscore  + \textunderscore metron\textunderscore )}
\end{itemize}
Instrumento, que foi proposto para medir as distâncias geodésicas.
\section{Diastatoma}
\begin{itemize}
\item {Grp. gram.:m.}
\end{itemize}
\begin{itemize}
\item {Proveniência:(Do gr. \textunderscore diastatos\textunderscore  + \textunderscore omma\textunderscore )}
\end{itemize}
Gênero de insectos neurópteros da China.
\section{Diastatomma}
\begin{itemize}
\item {Grp. gram.:m.}
\end{itemize}
\begin{itemize}
\item {Proveniência:(Do gr. \textunderscore diastatos\textunderscore  + \textunderscore omma\textunderscore )}
\end{itemize}
Gênero de insectos neurópteros da China.
\section{Diastema}
\begin{itemize}
\item {Grp. gram.:m.}
\end{itemize}
\begin{itemize}
\item {Proveniência:(Gr. \textunderscore diastema\textunderscore )}
\end{itemize}
Espaço entre os dentes caninos e os molares de muitos animaes mammíferos.
Poros, que não podem sêr observados directamente, mas que se revelam pela penetração dos líquidos.
Intervallo simples, em música.
\section{Diastematia}
\begin{itemize}
\item {Grp. gram.:f.}
\end{itemize}
Monstruosidade, caracterizada da por uma fenda na linha média do côrpo.
(Cp. \textunderscore diastema\textunderscore )
\section{Diastemático}
\begin{itemize}
\item {Grp. gram.:adj.}
\end{itemize}
\begin{itemize}
\item {Utilização:Mús.}
\end{itemize}
Dizia-se antigamente da voz cantante, por opposição á voz falante ou contínua.
(Cp. \textunderscore diastema\textunderscore )
\section{Diástole}
\begin{itemize}
\item {Grp. gram.:f.}
\end{itemize}
\begin{itemize}
\item {Utilização:Mús.}
\end{itemize}
\begin{itemize}
\item {Utilização:ant.}
\end{itemize}
\begin{itemize}
\item {Proveniência:(Gr. \textunderscore diastole\textunderscore )}
\end{itemize}
Movimento de dilatação do coração e das artérias.
Figura poética, com que se torna longa uma sýllaba breve.
Acto de levantar a mão, no bater no campasso.
\section{Diastólico}
\begin{itemize}
\item {Grp. gram.:adj.}
\end{itemize}
Relativo á diástole.
\section{Diastrofia}
\begin{itemize}
\item {Grp. gram.:f.}
\end{itemize}
\begin{itemize}
\item {Proveniência:(Do gr. \textunderscore diastrophe\textunderscore )}
\end{itemize}
Luxação de ossos.
Deslocamento de músculos, etc.
\section{Diastrophia}
\begin{itemize}
\item {Grp. gram.:f.}
\end{itemize}
\begin{itemize}
\item {Proveniência:(Do gr. \textunderscore diastrophe\textunderscore )}
\end{itemize}
Luxação de ossos.
Deslocamento de músculos, etc.
\section{Diasirmo}
\begin{itemize}
\item {Grp. gram.:m.}
\end{itemize}
\begin{itemize}
\item {Utilização:Rhet.}
\end{itemize}
\begin{itemize}
\item {Proveniência:(Gr. \textunderscore diasurmos\textunderscore )}
\end{itemize}
Espécie de hipérbole, com que se exalta ou se encarece coisa ordinária ou ridícula.
\section{Diástilo}
\begin{itemize}
\item {Grp. gram.:m.}
\end{itemize}
\begin{itemize}
\item {Proveniência:(Gr. \textunderscore diastulos\textunderscore )}
\end{itemize}
Espaço entre columnas, na extensão de três vezes o diâmetro de cada uma.
Edifício, construído em diástilo.
\section{Diástylo}
\begin{itemize}
\item {Grp. gram.:m.}
\end{itemize}
\begin{itemize}
\item {Proveniência:(Gr. \textunderscore diastulos\textunderscore )}
\end{itemize}
Espaço entre columnas, na extensão de três vezes o diâmetro de cada uma.
Edifício, construído em diástylo.
\section{Diasyrmo}
\begin{itemize}
\item {Grp. gram.:m.}
\end{itemize}
\begin{itemize}
\item {Utilização:Rhet.}
\end{itemize}
\begin{itemize}
\item {Proveniência:(Gr. \textunderscore diasurmos\textunderscore )}
\end{itemize}
Espécie de hypérbole, com que se exalta ou se encarece coisa ordinária ou ridícula.
\section{Diate}
\begin{itemize}
\item {Grp. gram.:m.}
\end{itemize}
\begin{itemize}
\item {Utilização:Ant.}
\end{itemize}
Certo pano, talvez fabricado em Dio. Cf. \textunderscore Lembranças das Cousas da India\textunderscore , nos \textunderscore Subsidios\textunderscore  de Felner.
\section{Diatermanismo}
\begin{itemize}
\item {Grp. gram.:m.}
\end{itemize}
\begin{itemize}
\item {Utilização:Phýs.}
\end{itemize}
\begin{itemize}
\item {Proveniência:(De \textunderscore diatérmano\textunderscore )}
\end{itemize}
Faculdade, que certos raios de calor têm, de atravessar facilmente um dado meio.
\section{Diatérmano}
\begin{itemize}
\item {Grp. gram.:adj.}
\end{itemize}
\begin{itemize}
\item {Proveniência:(Do gr. \textunderscore dia\textunderscore  + \textunderscore thermos\textunderscore )}
\end{itemize}
Que deixa passar facilmente o calor.
\section{Diatérmico}
\begin{itemize}
\item {Grp. gram.:adj.}
\end{itemize}
O mesmo que \textunderscore diatérmano\textunderscore .
\section{Diátese}
\begin{itemize}
\item {Grp. gram.:f.}
\end{itemize}
\begin{itemize}
\item {Utilização:Med.}
\end{itemize}
\begin{itemize}
\item {Proveniência:(Gr. \textunderscore diathesis\textunderscore )}
\end{itemize}
Disposição, que o indivíduo tem, para sêr atacado de muitas doenças locaes da mesma natureza.
\section{Diatésico}
\begin{itemize}
\item {Grp. gram.:adj.}
\end{itemize}
Relativo a diátese.
\section{Diatessarão}
\begin{itemize}
\item {Grp. gram.:m.}
\end{itemize}
\begin{itemize}
\item {Utilização:Mús.}
\end{itemize}
\begin{itemize}
\item {Proveniência:(Do gr. \textunderscore dia\textunderscore  + \textunderscore tessares\textunderscore )}
\end{itemize}
Medicamento, em que entram quatro ingredientes.
Intervallo de quarta perfeita.
\section{Diathermanismo}
\begin{itemize}
\item {Grp. gram.:m.}
\end{itemize}
\begin{itemize}
\item {Utilização:Phýs.}
\end{itemize}
\begin{itemize}
\item {Proveniência:(De \textunderscore diathérmano\textunderscore )}
\end{itemize}
Faculdade, que certos raios de calor têm, de atravessar facilmente um dado meio.
\section{Diathérmano}
\begin{itemize}
\item {Grp. gram.:adj.}
\end{itemize}
\begin{itemize}
\item {Proveniência:(Do gr. \textunderscore dia\textunderscore  + \textunderscore thermos\textunderscore )}
\end{itemize}
Que deixa passar facilmente o calor.
\section{Diathérmico}
\begin{itemize}
\item {Grp. gram.:adj.}
\end{itemize}
O mesmo que \textunderscore diathérmano\textunderscore .
\section{Diáthese}
\begin{itemize}
\item {Grp. gram.:f.}
\end{itemize}
\begin{itemize}
\item {Utilização:Med.}
\end{itemize}
\begin{itemize}
\item {Proveniência:(Gr. \textunderscore diathesis\textunderscore )}
\end{itemize}
Disposição, que o indivíduo tem, para sêr atacado de muitas doenças locaes da mesma natureza.
\section{Diathésico}
\begin{itemize}
\item {Grp. gram.:adj.}
\end{itemize}
Relativo a diáthese.
\section{Diatima}
\begin{itemize}
\item {Grp. gram.:f.}
\end{itemize}
Arvore angolense, no Duque-de-Bragança.
\section{Diatomáceas}
\begin{itemize}
\item {Grp. gram.:f. pl.}
\end{itemize}
\begin{itemize}
\item {Proveniência:(Do gr. \textunderscore diatomos\textunderscore )}
\end{itemize}
Algas microscópicas, que vivem nas águas doces ou salgadas, e também na terra húmida, formando sôbre o solo uma espécie de vasa gelatinosa, pardacenta.
\section{Diátomo}
\begin{itemize}
\item {Grp. gram.:m.}
\end{itemize}
Gênero de plantas phýceas.
\section{Diatonicamente}
\begin{itemize}
\item {Grp. gram.:adv.}
\end{itemize}
\begin{itemize}
\item {Proveniência:(De \textunderscore diatónico\textunderscore )}
\end{itemize}
Por graus diatónicos.
\section{Diatónico}
\begin{itemize}
\item {Grp. gram.:adj.}
\end{itemize}
\begin{itemize}
\item {Utilização:Mús.}
\end{itemize}
\begin{itemize}
\item {Proveniência:(Do gr. \textunderscore dia\textunderscore  + \textunderscore tonos\textunderscore )}
\end{itemize}
Que consta de tons e meios tons.
\section{Diatribe}
\begin{itemize}
\item {Grp. gram.:f.}
\end{itemize}
\begin{itemize}
\item {Proveniência:(Gr. \textunderscore diatribe\textunderscore )}
\end{itemize}
Crítica severa.
Escrito violento e injurioso.
\section{Diaulo}
\begin{itemize}
\item {Grp. gram.:m.}
\end{itemize}
Flauta dupla, entre os Gregos.
\section{Diazoma}
\begin{itemize}
\item {Grp. gram.:m.}
\end{itemize}
\begin{itemize}
\item {Proveniência:(Lat. \textunderscore diazoma\textunderscore )}
\end{itemize}
Espaço estreito, que separava os assentos, nos theatros gregos e romanos.
\section{Dibixi}
\begin{itemize}
\item {Grp. gram.:m.}
\end{itemize}
Arvoreta bixácea de Angola, (\textunderscore oncoba dentata\textunderscore , Oliver).
\section{Dibranchiado}
\begin{itemize}
\item {fónica:qui}
\end{itemize}
\begin{itemize}
\item {Grp. gram.:adj.}
\end{itemize}
\begin{itemize}
\item {Proveniência:(De \textunderscore di...\textunderscore  + \textunderscore brânchia\textunderscore )}
\end{itemize}
Diz-se dos molluscos, que têm duas brânchias.
\section{Dibranchiaes}
\begin{itemize}
\item {fónica:qui}
\end{itemize}
\begin{itemize}
\item {Grp. gram.:m. pl.}
\end{itemize}
Família de molluscos da classe dos cephalópodes.
\section{Dibrânchio}
\begin{itemize}
\item {fónica:qui}
\end{itemize}
\begin{itemize}
\item {Grp. gram.:adj.}
\end{itemize}
O mesmo que \textunderscore dibranchiado\textunderscore .
\section{Dibranquiado}
\begin{itemize}
\item {Grp. gram.:adj.}
\end{itemize}
\begin{itemize}
\item {Proveniência:(De \textunderscore di...\textunderscore  + \textunderscore brânchia\textunderscore )}
\end{itemize}
Diz-se dos moluscos, que têm duas brânquias.
\section{Dibranquiaes}
\begin{itemize}
\item {Grp. gram.:m. pl.}
\end{itemize}
Família de moluscos da classe dos cefalópodes.
\section{Dibranquiais}
\begin{itemize}
\item {Grp. gram.:m. pl.}
\end{itemize}
Família de moluscos da classe dos cefalópodes.
\section{Dibrânquio}
\begin{itemize}
\item {Grp. gram.:adj.}
\end{itemize}
O mesmo que \textunderscore dibranquiado\textunderscore .
\section{Dibulo}
\begin{itemize}
\item {Grp. gram.:m.}
\end{itemize}
Copado arbusto africano, de fôlhas pubescentes, muito longas, e flôres de corolla carmesim.
\section{Dica}
\begin{itemize}
\item {Grp. gram.:f.}
\end{itemize}
\begin{itemize}
\item {Utilização:Gír.}
\end{itemize}
\textunderscore Dar a dica\textunderscore , denunciar.
\section{Dicaba}
\begin{itemize}
\item {Grp. gram.:f.}
\end{itemize}
Espécie de palmeira, conhecida por \textunderscore palmeira de leque\textunderscore .
\section{Dicação}
\begin{itemize}
\item {Grp. gram.:f.}
\end{itemize}
Acto de dicar.
\section{Dicacidade}
\begin{itemize}
\item {Grp. gram.:f.}
\end{itemize}
\begin{itemize}
\item {Proveniência:(Lat. \textunderscore dicacitas\textunderscore )}
\end{itemize}
Qualidade de quem é dicaz.
\section{Dição}
\begin{itemize}
\item {Grp. gram.:f.}
\end{itemize}
\begin{itemize}
\item {Proveniência:(Lat. \textunderscore dictio\textunderscore )}
\end{itemize}
Maneira de dizer.
Expressão; vocábulo.
\section{Dição}
\begin{itemize}
\item {Grp. gram.:f.}
\end{itemize}
\begin{itemize}
\item {Utilização:Ant.}
\end{itemize}
O mesmo que \textunderscore domínio\textunderscore . Cf. \textunderscore Vida da Rainha Santa Isabel\textunderscore , 66.
\section{Dicar}
\begin{itemize}
\item {Grp. gram.:v. t.}
\end{itemize}
\begin{itemize}
\item {Proveniência:(Lat. \textunderscore dicare\textunderscore )}
\end{itemize}
Dedicar.
Tributar.
Sacrificar. Cf. Castilho, \textunderscore Fastos\textunderscore , III, 79.
\section{Dicarpelar}
\begin{itemize}
\item {Grp. gram.:adj.}
\end{itemize}
\begin{itemize}
\item {Proveniência:(De \textunderscore di...\textunderscore  + \textunderscore carpelar\textunderscore )}
\end{itemize}
Que tem duas carpelas.
\section{Dicastéria}
\begin{itemize}
\item {Grp. gram.:f.}
\end{itemize}
\begin{itemize}
\item {Proveniência:(Do gr. \textunderscore dikasterion\textunderscore )}
\end{itemize}
Antigo tribunal de justiça, em Athenas.
\section{Dicastério}
\begin{itemize}
\item {Grp. gram.:m.}
\end{itemize}
O mesmo ou melhor que \textunderscore dicastéria\textunderscore . Cf. Latino, \textunderscore Oração da Corôa\textunderscore , CXLIII.
\section{Dicaz}
\begin{itemize}
\item {Grp. gram.:adj.}
\end{itemize}
\begin{itemize}
\item {Proveniência:(Lat. \textunderscore dicax\textunderscore )}
\end{itemize}
Mordaz, severo em critica.
Satírico.
\section{Dicção}
\begin{itemize}
\item {Grp. gram.:f.}
\end{itemize}
\begin{itemize}
\item {Proveniência:(Lat. \textunderscore dictio\textunderscore )}
\end{itemize}
Maneira de dizer.
Expressão; vocábulo.
\section{Diccionário}
\begin{itemize}
\item {Grp. gram.:m.}
\end{itemize}
\begin{itemize}
\item {Proveniência:(Do lat. \textunderscore dictio\textunderscore )}
\end{itemize}
Collecção alphabetada dos vocábulos de uma língua, ou dos termos próprios de uma sciência ou arte, com a significação delles, ou com a sua traducção em outra língua.
\section{Diccionarista}
\begin{itemize}
\item {Grp. gram.:m.}
\end{itemize}
\begin{itemize}
\item {Proveniência:(De \textunderscore diccionário\textunderscore )}
\end{itemize}
Autor de diccionário ou de diccionários.
Lexicógrapho.
\section{Diccionarizar}
\begin{itemize}
\item {Grp. gram.:v. t.}
\end{itemize}
Organizar sob a fórma de diccionário. Cf. Ruy Barb., \textunderscore Réplica\textunderscore , 157.
\section{Dicéfalo}
\begin{itemize}
\item {Grp. gram.:adj.}
\end{itemize}
\begin{itemize}
\item {Proveniência:(Do gr. \textunderscore dis\textunderscore  + \textunderscore kephale\textunderscore )}
\end{itemize}
Que tem duas cabeças.
\section{Dicélias}
\begin{itemize}
\item {Grp. gram.:f. pl.}
\end{itemize}
\begin{itemize}
\item {Proveniência:(Do gr. \textunderscore deikelon\textunderscore )}
\end{itemize}
Farças, comédias licenciosas, na antiga Grécia.
\section{Dicelista}
\begin{itemize}
\item {Grp. gram.:m.}
\end{itemize}
Actor de dicélias.
\section{Dicentra}
\begin{itemize}
\item {Grp. gram.:f.}
\end{itemize}
O mesmo que \textunderscore dicentro\textunderscore .
\section{Dicentro}
\begin{itemize}
\item {Grp. gram.:m.}
\end{itemize}
\begin{itemize}
\item {Proveniência:(Do gr. \textunderscore dis...\textunderscore  + \textunderscore kentron\textunderscore )}
\end{itemize}
Planta papaverácea.
\section{Diceosine}
\begin{itemize}
\item {Grp. gram.:f.}
\end{itemize}
\begin{itemize}
\item {Proveniência:(Gr. \textunderscore dikeosine\textunderscore )}
\end{itemize}
Parte da Philosophia, que se occupa da justiça.
\section{Dicéphalo}
\begin{itemize}
\item {Grp. gram.:adj.}
\end{itemize}
\begin{itemize}
\item {Proveniência:(Do gr. \textunderscore dis\textunderscore  + \textunderscore kephale\textunderscore )}
\end{itemize}
Que tem duas cabeças.
\section{Diche}
\begin{itemize}
\item {Grp. gram.:m.}
\end{itemize}
\begin{itemize}
\item {Utilização:Ant.}
\end{itemize}
O mesmo que \textunderscore dicho\textunderscore .
\section{Di-chlorado}
\begin{itemize}
\item {Grp. gram.:m.}
\end{itemize}
\begin{itemize}
\item {Utilização:Chím.}
\end{itemize}
Chloreto de methylo.
\section{Dicho}
\begin{itemize}
\item {Grp. gram.:m.  e  adj.}
\end{itemize}
\begin{itemize}
\item {Utilização:Ant.}
\end{itemize}
\begin{itemize}
\item {Proveniência:(T. cast.)}
\end{itemize}
O mesmo que \textunderscore dito\textunderscore . Cf. \textunderscore Aulegraphia\textunderscore , 86.
\section{Dichogamia}
\begin{itemize}
\item {fónica:co}
\end{itemize}
\begin{itemize}
\item {Grp. gram.:f.}
\end{itemize}
\begin{itemize}
\item {Proveniência:(Do gr. \textunderscore dikha\textunderscore  + \textunderscore gamos\textunderscore )}
\end{itemize}
Modo de fecundação nos vegetaes, cujas flôres apresentam ambos os órgãos sexuaes, que se não desenvolvem completamente ao mesmo tempo.
\section{Dichoreu}
\begin{itemize}
\item {fónica:co}
\end{itemize}
\begin{itemize}
\item {Grp. gram.:adj.}
\end{itemize}
\begin{itemize}
\item {Proveniência:(Lat. \textunderscore dichoreus\textunderscore )}
\end{itemize}
Dizia-se de um pé de verso grego; choreu duplo.
\section{Dichote}
\begin{itemize}
\item {Grp. gram.:m.}
\end{itemize}
\begin{itemize}
\item {Proveniência:(De \textunderscore dicho\textunderscore )}
\end{itemize}
Motejo.
Chufa; expressão jocosa.
\section{Dichotomal}
\begin{itemize}
\item {fónica:co}
\end{itemize}
\begin{itemize}
\item {Grp. gram.:adj.}
\end{itemize}
O mesmo que \textunderscore dichotómico\textunderscore .
Diz-se, em Botânica, do pedúnculo, quando nasce do ângulo formado por dois ramúsculos de um tronco dichótomo.
\section{Dichotomia}
\begin{itemize}
\item {fónica:co}
\end{itemize}
\begin{itemize}
\item {Grp. gram.:f.}
\end{itemize}
\begin{itemize}
\item {Utilização:Bot.}
\end{itemize}
Divisão em dois ramos ou pedúnculos.
Classificação, em que se divide cada coisa ou cada proposição em duas, subdividindo-se cada uma destas em outras duas, e assim successivamente.
Phase da lua, em que esta apresenta metade do seu disco.
(Cp. \textunderscore dichótomo\textunderscore )
\section{Dichotómico}
\begin{itemize}
\item {fónica:co}
\end{itemize}
\begin{itemize}
\item {Grp. gram.:adj.}
\end{itemize}
O mesmo que \textunderscore dichótomo\textunderscore .
\section{Dichótomo}
\begin{itemize}
\item {fónica:có}
\end{itemize}
\begin{itemize}
\item {Grp. gram.:adj.}
\end{itemize}
\begin{itemize}
\item {Proveniência:(Gr. \textunderscore dikhotomos\textunderscore )}
\end{itemize}
Que se divide e subdivide em dois.
Bifurcado.
Diz-se da Lua, quando só deixa vêr metade do seu disco.
\section{Dichrocéphala}
\begin{itemize}
\item {Grp. gram.:f.}
\end{itemize}
\begin{itemize}
\item {Proveniência:(Do gr. \textunderscore dis\textunderscore  + \textunderscore khroa\textunderscore  + \textunderscore kephale\textunderscore )}
\end{itemize}
Gênero de plantas compostas.
\section{Dichroico}
\begin{itemize}
\item {Grp. gram.:adj.}
\end{itemize}
Diz-se dos mineraes em que há dichroismo.
(Cp. \textunderscore dichroismo\textunderscore )
\section{Dichroismo}
\begin{itemize}
\item {Grp. gram.:m.}
\end{itemize}
\begin{itemize}
\item {Proveniência:(Do gr. \textunderscore dis\textunderscore  + \textunderscore khroizo\textunderscore )}
\end{itemize}
Qualidade, que os crystaes birefringentes têm, de apresentar duas côres differentes.
\section{Dichromático}
\begin{itemize}
\item {Grp. gram.:adj.}
\end{itemize}
\begin{itemize}
\item {Proveniência:(Do gr. \textunderscore dis\textunderscore  + \textunderscore khroma\textunderscore )}
\end{itemize}
Que póde apresentar duas côres.
\section{Dichroscópio}
\begin{itemize}
\item {Grp. gram.:m.}
\end{itemize}
\begin{itemize}
\item {Proveniência:(Do gr. \textunderscore dikhroia\textunderscore  + \textunderscore skopein\textunderscore )}
\end{itemize}
Instrumento, com que se observam duas imagens de côres differentes.
\section{Dicionário}
\begin{itemize}
\item {Grp. gram.:m.}
\end{itemize}
\begin{itemize}
\item {Proveniência:(Do lat. \textunderscore dictio\textunderscore )}
\end{itemize}
Colecção alfabetada dos vocábulos de uma língua, ou dos termos próprios de uma ciência ou arte, com a significação deles, ou com a sua tradução em outra língua.
\section{Dicionarista}
\begin{itemize}
\item {Grp. gram.:m.}
\end{itemize}
\begin{itemize}
\item {Proveniência:(De \textunderscore dicionário\textunderscore )}
\end{itemize}
Autor de dicionário ou de dicionários.
Lexicógrafo.
\section{Dicionarizar}
\begin{itemize}
\item {Grp. gram.:v. t.}
\end{itemize}
Organizar sob a fórma de dicionário. Cf. Ruy Barb., \textunderscore Réplica\textunderscore , 157.
\section{Diclinismo}
\begin{itemize}
\item {Grp. gram.:m.}
\end{itemize}
\begin{itemize}
\item {Utilização:Bot.}
\end{itemize}
Separação dos dois sexos, na mesma planta.
(Cp. \textunderscore diclino\textunderscore )
\section{Diclino}
\begin{itemize}
\item {Grp. gram.:adj.}
\end{itemize}
\begin{itemize}
\item {Proveniência:(Do gr. \textunderscore dis\textunderscore  + \textunderscore kline\textunderscore )}
\end{itemize}
Diz-se das plantas unisexuaes.
\section{Declíptero}
\begin{itemize}
\item {Grp. gram.:m.}
\end{itemize}
Gênero de plantas acantháceas.
\section{Dicliso}
\begin{itemize}
\item {Grp. gram.:m.}
\end{itemize}
Gênero de plantas escrofularíneas.
\section{Diclorado}
\begin{itemize}
\item {Grp. gram.:m.}
\end{itemize}
\begin{itemize}
\item {Utilização:Chím.}
\end{itemize}
Cloretode metilo.
\section{Diclória}
\begin{itemize}
\item {Grp. gram.:f.}
\end{itemize}
Gênero de plantas marinhas do Atlântico.
\section{Dicogamia}
\begin{itemize}
\item {Grp. gram.:f.}
\end{itemize}
\begin{itemize}
\item {Proveniência:(Do gr. \textunderscore dikha\textunderscore  + \textunderscore gamos\textunderscore )}
\end{itemize}
Modo de fecundação nos vegetaes, cujas flôres apresentam ambos os órgãos sexuaes, que se não desenvolvem completamente ao mesmo tempo.
\section{Dicole}
\begin{itemize}
\item {Grp. gram.:m.}
\end{itemize}
Designação de algumas espécies de aves africanas.
\section{Dicongo}
\begin{itemize}
\item {Grp. gram.:m.}
\end{itemize}
Banquete selvagem, entre os Negros da África occidental, e composto de carne de gente, de boi e de cabra. Cf. Capello e Ivens, I, 300.
\section{Diconroque}
\begin{itemize}
\item {Grp. gram.:m.}
\end{itemize}
Árvore brasileira.
\section{Dicórdio}
\begin{itemize}
\item {Grp. gram.:m.}
\end{itemize}
Instrumento de arco, que se usou na Idade-Média, e que consistia apenas numa comprida caixa harmónica, sem braço e com duas cordas.
\section{Dicoreu}
\begin{itemize}
\item {Grp. gram.:adj.}
\end{itemize}
\begin{itemize}
\item {Proveniência:(Lat. \textunderscore dichoreus\textunderscore )}
\end{itemize}
Dizia-se de um pé de verso grego; coreu duplo.
\section{Dicoria}
\begin{itemize}
\item {Grp. gram.:f.}
\end{itemize}
\begin{itemize}
\item {Utilização:Anat.}
\end{itemize}
\begin{itemize}
\item {Proveniência:(Do gr. \textunderscore dis\textunderscore  + \textunderscore core\textunderscore )}
\end{itemize}
Duplicidade da pupilla ocular.
\section{Dicosema}
\begin{itemize}
\item {Grp. gram.:m.}
\end{itemize}
Gênero de plantas papilionáceas.
\section{Dicotíleas}
\begin{itemize}
\item {Grp. gram.:f. pl.}
\end{itemize}
O mesmo que \textunderscore dicotiledóneas\textunderscore .
\section{Dicotiledóneas}
\begin{itemize}
\item {Grp. gram.:f. pl.}
\end{itemize}
\begin{itemize}
\item {Proveniência:(De \textunderscore dicotiledóneo\textunderscore )}
\end{itemize}
Familia de plantas, cujo embrião tem dois cotilédones.
\section{Dicotiledóneo}
\begin{itemize}
\item {Grp. gram.:adj.}
\end{itemize}
\begin{itemize}
\item {Proveniência:(De \textunderscore di...\textunderscore  + \textunderscore cotiledóneo\textunderscore )}
\end{itemize}
Que tem dois cotilédones.
\section{Dicótilo}
\begin{itemize}
\item {Grp. gram.:adj.}
\end{itemize}
\begin{itemize}
\item {Proveniência:(Do gr. \textunderscore dis\textunderscore  + \textunderscore kotule\textunderscore )}
\end{itemize}
O mesmo que \textunderscore dicotiledóneo\textunderscore .
\section{Dicotomal}
\begin{itemize}
\item {Grp. gram.:adj.}
\end{itemize}
O mesmo que \textunderscore dicotómico\textunderscore .
Diz-se, em Botânica, do pedúnculo, quando nasce do ângulo formado por dois ramúsculos de um tronco dicótomo.
\section{Dicotomia}
\begin{itemize}
\item {Grp. gram.:f.}
\end{itemize}
\begin{itemize}
\item {Utilização:Bot.}
\end{itemize}
Divisão em dois ramos ou pedúnculos.
Classificação, em que se divide cada coisa ou cada proposição em duas, subdividindo-se cada uma destas em outras duas, e assim sucessivamente.
Fase da lua, em que esta apresenta metade do seu disco.
(Cp. \textunderscore dicótomo\textunderscore )
\section{Dicotómico}
\begin{itemize}
\item {Grp. gram.:adj.}
\end{itemize}
O mesmo que \textunderscore dicótomo\textunderscore .
\section{Dicótomo}
\begin{itemize}
\item {Grp. gram.:adj.}
\end{itemize}
\begin{itemize}
\item {Proveniência:(Gr. \textunderscore dikhotomos\textunderscore )}
\end{itemize}
Que se divide e subdivide em dois.
Bifurcado.
Diz-se da Lua, quando só deixa vêr metade do seu disco.
\section{Dicotýleas}
\begin{itemize}
\item {Grp. gram.:f. pl.}
\end{itemize}
O mesmo que \textunderscore dicotyledóneas\textunderscore .
\section{Dicotyledóneas}
\begin{itemize}
\item {Grp. gram.:f. pl.}
\end{itemize}
\begin{itemize}
\item {Proveniência:(De \textunderscore dicotyledóneo\textunderscore )}
\end{itemize}
Família de plantas, cujo embryão tem dois cotylédones.
\section{Dicotyledóneo}
\begin{itemize}
\item {Grp. gram.:adj.}
\end{itemize}
\begin{itemize}
\item {Proveniência:(De \textunderscore di...\textunderscore  + \textunderscore cotiledóneo\textunderscore )}
\end{itemize}
Que tem dois cotylédones.
\section{Dicótylo}
\begin{itemize}
\item {Grp. gram.:adj.}
\end{itemize}
\begin{itemize}
\item {Proveniência:(Do gr. \textunderscore dis\textunderscore  + \textunderscore kotule\textunderscore )}
\end{itemize}
O mesmo que \textunderscore dicotyledóneo\textunderscore .
\section{Dicrocéfala}
\begin{itemize}
\item {Grp. gram.:f.}
\end{itemize}
\begin{itemize}
\item {Proveniência:(Do gr. \textunderscore dis\textunderscore  + \textunderscore khroa\textunderscore  + \textunderscore kephale\textunderscore )}
\end{itemize}
Gênero de plantas compostas.
\section{Dicroico}
\begin{itemize}
\item {Grp. gram.:adj.}
\end{itemize}
Diz-se dos mineraes em que há dicroismo.
(Cp. \textunderscore dicroismo\textunderscore )
\section{Dicroismo}
\begin{itemize}
\item {Grp. gram.:m.}
\end{itemize}
\begin{itemize}
\item {Proveniência:(Do gr. \textunderscore dis\textunderscore  + \textunderscore khroizo\textunderscore )}
\end{itemize}
Qualidade, que os cristaes birefringentes têm, de apresentar duas côres diferentes.
\section{Dicromático}
\begin{itemize}
\item {Grp. gram.:adj.}
\end{itemize}
\begin{itemize}
\item {Proveniência:(Do gr. \textunderscore dis\textunderscore  + \textunderscore khroma\textunderscore )}
\end{itemize}
Que póde apresentar duas côres.
\section{Dicroscópio}
\begin{itemize}
\item {Grp. gram.:m.}
\end{itemize}
\begin{itemize}
\item {Proveniência:(Do gr. \textunderscore dikhroia\textunderscore  + \textunderscore skopein\textunderscore )}
\end{itemize}
Instrumento, com que se observam duas imagens de côres diferentes.
\section{Dicrotismo}
\begin{itemize}
\item {Grp. gram.:m.}
\end{itemize}
\begin{itemize}
\item {Utilização:Med.}
\end{itemize}
\begin{itemize}
\item {Proveniência:(De \textunderscore dicroto\textunderscore )}
\end{itemize}
Pulsação dupla para cada sýstole cardiaca.
\section{Dicroto}
\begin{itemize}
\item {Grp. gram.:m.}
\end{itemize}
\begin{itemize}
\item {Utilização:Med.}
\end{itemize}
Diz-se do pulso, que dá pancada dupla, por cada sýstole cardíaca.
\section{Dicteríade}
\begin{itemize}
\item {Grp. gram.:f.}
\end{itemize}
\begin{itemize}
\item {Proveniência:(Do gr. \textunderscore dikterias\textunderscore )}
\end{itemize}
Mulher mundana, que, entre os antigos Gregos, representava sátiras ou pantomimas.
\section{Didacta}
\begin{itemize}
\item {Grp. gram.:m.}
\end{itemize}
\begin{itemize}
\item {Utilização:Neol.}
\end{itemize}
Aquelle que instrue.
(Cp. \textunderscore didáctico\textunderscore )
\section{Didáctica}
\begin{itemize}
\item {Grp. gram.:f.}
\end{itemize}
\begin{itemize}
\item {Proveniência:(De \textunderscore didáctico\textunderscore )}
\end{itemize}
Arte de ensinar.
\section{Didacticamente}
\begin{itemize}
\item {Grp. gram.:adv.}
\end{itemize}
De modo didáctico.
\section{Didáctico}
\begin{itemize}
\item {Grp. gram.:adj.}
\end{itemize}
\begin{itemize}
\item {Proveniência:(Gr. \textunderscore didaktikos\textunderscore )}
\end{itemize}
Relativo ao ensino.
Próprio para instruir.
Relativo a uma sciência.
Escolar.
\section{Didáctilo}
\begin{itemize}
\item {Grp. gram.:adj.}
\end{itemize}
\begin{itemize}
\item {Utilização:Zool.}
\end{itemize}
\begin{itemize}
\item {Proveniência:(Do gr. \textunderscore dis\textunderscore  + \textunderscore daktulos\textunderscore )}
\end{itemize}
Diz-se do animal, que tem dois dedos em cada pé.
\section{Didactologia}
\begin{itemize}
\item {Grp. gram.:f.}
\end{itemize}
\begin{itemize}
\item {Proveniência:(De \textunderscore didacta\textunderscore  + gr. \textunderscore logos\textunderscore )}
\end{itemize}
Theoria do ensino.
Pedagogia.
O gênero didactico, em composições literárias.
\section{Didactológico}
\begin{itemize}
\item {Grp. gram.:adj.}
\end{itemize}
Relativo á didactologia.
\section{Didáctylo}
\begin{itemize}
\item {Grp. gram.:adj.}
\end{itemize}
\begin{itemize}
\item {Utilização:Zool.}
\end{itemize}
\begin{itemize}
\item {Proveniência:(Do gr. \textunderscore dis\textunderscore  + \textunderscore daktulos\textunderscore )}
\end{itemize}
Diz-se do animal, que tem dois dedos em cada pé.
\section{Didal}
\begin{itemize}
\item {Grp. gram.:m.}
\end{itemize}
(Fórma pop. de \textunderscore dedal\textunderscore . Cf. Bandarra, \textunderscore Profecias\textunderscore , 20 v.^o)
\section{Didascália}
\begin{itemize}
\item {Grp. gram.:f.}
\end{itemize}
\begin{itemize}
\item {Proveniência:(Gr. \textunderscore didaskalia\textunderscore )}
\end{itemize}
Instrucção, que os actores gregos recebiam dos poétas.
Annotação ou crítica de peça theatral, entre os Latinos.
\section{Didascálica}
\begin{itemize}
\item {Grp. gram.:f.}
\end{itemize}
\begin{itemize}
\item {Proveniência:(Gr. \textunderscore didascalike\textunderscore )}
\end{itemize}
O gênero didascálico.
\section{Didascálico}
\begin{itemize}
\item {Grp. gram.:adj.}
\end{itemize}
\begin{itemize}
\item {Proveniência:(Gr. \textunderscore didaskalikos\textunderscore )}
\end{itemize}
O mesmo que \textunderscore didáctico\textunderscore .
Que annotava, comentava ou criticava peças theatraes.
\section{Didelfoide}
\begin{itemize}
\item {Grp. gram.:adj.}
\end{itemize}
Semelhante aos didelfos.
\section{Didelfos}
\begin{itemize}
\item {Grp. gram.:m. pl.}
\end{itemize}
\begin{itemize}
\item {Proveniência:(Do gr. \textunderscore dia\textunderscore  + \textunderscore delphos\textunderscore )}
\end{itemize}
Grupo de animaes, o mesmo que \textunderscore marsupiaes\textunderscore .
\section{Didelphoide}
\begin{itemize}
\item {Grp. gram.:adj.}
\end{itemize}
Semelhante aos didelphos.
\section{Didelphos}
\begin{itemize}
\item {Grp. gram.:m. pl.}
\end{itemize}
\begin{itemize}
\item {Proveniência:(Do gr. \textunderscore dia\textunderscore  + \textunderscore delphos\textunderscore )}
\end{itemize}
Grupo de animaes, o mesmo que \textunderscore marsupiaes\textunderscore .
\section{Didi}
\begin{itemize}
\item {Grp. gram.:f.}
\end{itemize}
Planta brasileira.
\section{Dídia}
\begin{itemize}
\item {Grp. gram.:adj. f.}
\end{itemize}
\begin{itemize}
\item {Proveniência:(Lat. \textunderscore didia\textunderscore )}
\end{itemize}
Dizia-se de uma lei que regulava o luxo, entre os Romanos.
\section{Didimalgia}
\begin{itemize}
\item {Grp. gram.:f.}
\end{itemize}
\begin{itemize}
\item {Proveniência:(Do gr. \textunderscore didumoi\textunderscore  + \textunderscore algos\textunderscore )}
\end{itemize}
Dôr nos testículos.
\section{Didímio}
\begin{itemize}
\item {Grp. gram.:m.}
\end{itemize}
Metal novo, descoberto na cerita, o mesmo que \textunderscore dídimo\textunderscore .
\section{Didimite}
\begin{itemize}
\item {Grp. gram.:f.}
\end{itemize}
\begin{itemize}
\item {Proveniência:(Do gr. \textunderscore didumos\textunderscore )}
\end{itemize}
Inflamação nos testículos.
Orchite.
\section{Dídimo}
\begin{itemize}
\item {Grp. gram.:adj.}
\end{itemize}
\begin{itemize}
\item {Utilização:Bot.}
\end{itemize}
\begin{itemize}
\item {Grp. gram.:M.}
\end{itemize}
\begin{itemize}
\item {Proveniência:(Do gr. \textunderscore didumos\textunderscore )}
\end{itemize}
Diz-se dos órgãos vegetaes, compostos de duas partes arredondadas ligadas por um ponto comum das suas periferias.
Metal novo, descoberto na cerita.
\section{Didinamia}
\begin{itemize}
\item {Grp. gram.:f.}
\end{itemize}
\begin{itemize}
\item {Utilização:Bot.}
\end{itemize}
\begin{itemize}
\item {Proveniência:(De \textunderscore dis\textunderscore  + \textunderscore dunamis\textunderscore )}
\end{itemize}
Disposição de quatro estames na flôr, tendo dois os filetes maiores que os outros dois.
\section{Didínamo}
\begin{itemize}
\item {Grp. gram.:adj.}
\end{itemize}
\begin{itemize}
\item {Utilização:Bot.}
\end{itemize}
\begin{itemize}
\item {Proveniência:(Do gr. \textunderscore dis...\textunderscore  + \textunderscore dunamis\textunderscore )}
\end{itemize}
Diz-se dos estames que são em número de quatro, dois dos quaes são mais compridos que os outros; e diz-se dos vegetaes que têm estames didínamos.
\section{Didisco}
\begin{itemize}
\item {Grp. gram.:m.}
\end{itemize}
\begin{itemize}
\item {Proveniência:(Do gr. \textunderscore dis\textunderscore  + \textunderscore diskos\textunderscore )}
\end{itemize}
Gênero de plantas umbellíferas.
\section{Didrachma}
\begin{itemize}
\item {Grp. gram.:m.}
\end{itemize}
\begin{itemize}
\item {Proveniência:(Lat. \textunderscore didrachma\textunderscore )}
\end{itemize}
Drachma duplo. Cf. Vieira, XI, 55.
\section{Didracma}
\begin{itemize}
\item {Grp. gram.:m.}
\end{itemize}
\begin{itemize}
\item {Proveniência:(Lat. \textunderscore didrachma\textunderscore )}
\end{itemize}
Dracma duplo. Cf. Vieira, XI, 55.
\section{Didução}
\begin{itemize}
\item {Grp. gram.:f.}
\end{itemize}
\begin{itemize}
\item {Proveniência:(Lat. \textunderscore diductio\textunderscore )}
\end{itemize}
Movimento lateral da maxila inferior dos herbívoros quando mastigam, e dos ruminantes quando ruminam.
\section{Diducção}
\begin{itemize}
\item {Grp. gram.:f.}
\end{itemize}
\begin{itemize}
\item {Proveniência:(Lat. \textunderscore diductio\textunderscore )}
\end{itemize}
Movimento lateral da maxilla inferior dos herbívoros quando mastigam, e dos ruminantes quando ruminam.
\section{Didymalgia}
\begin{itemize}
\item {Grp. gram.:f.}
\end{itemize}
\begin{itemize}
\item {Proveniência:(Do gr. \textunderscore didumoi\textunderscore  + \textunderscore algos\textunderscore )}
\end{itemize}
Dôr nos testículos.
\section{Didýmio}
\begin{itemize}
\item {Grp. gram.:m.}
\end{itemize}
Metal novo, descoberto na cerita, o mesmo que \textunderscore dídymo\textunderscore .
\section{Didymite}
\begin{itemize}
\item {Grp. gram.:f.}
\end{itemize}
\begin{itemize}
\item {Proveniência:(Do gr. \textunderscore didumos\textunderscore )}
\end{itemize}
Inflammação nos testículos.
Orchite.
\section{Dídymo}
\begin{itemize}
\item {Grp. gram.:adj.}
\end{itemize}
\begin{itemize}
\item {Utilização:Bot.}
\end{itemize}
\begin{itemize}
\item {Grp. gram.:M.}
\end{itemize}
\begin{itemize}
\item {Proveniência:(Do gr. \textunderscore didumos\textunderscore )}
\end{itemize}
Diz-se dos órgãos vegetaes, compostos de duas partes arredondadas ligadas por um ponto commum das suas peripherias.
Metal novo, descoberto na cerita.
\section{Didynamia}
\begin{itemize}
\item {Grp. gram.:f.}
\end{itemize}
\begin{itemize}
\item {Utilização:Bot.}
\end{itemize}
\begin{itemize}
\item {Proveniência:(De \textunderscore dis\textunderscore  + \textunderscore dunamis\textunderscore )}
\end{itemize}
Disposição de quatro estames na flôr, tendo dois os filetes maiores que os outros dois.
\section{Didýnamo}
\begin{itemize}
\item {Grp. gram.:adj.}
\end{itemize}
\begin{itemize}
\item {Utilização:Bot.}
\end{itemize}
\begin{itemize}
\item {Proveniência:(Do gr. \textunderscore dis...\textunderscore  + \textunderscore dunamis\textunderscore )}
\end{itemize}
Diz-se dos estames que são em número de quatro, dois dos quaes são mais compridos que os outros; e diz-se dos vegetaes que têm estames didýnamos.
\section{Diecia}
\begin{itemize}
\item {Grp. gram.:f.}
\end{itemize}
\begin{itemize}
\item {Utilização:Bot.}
\end{itemize}
\begin{itemize}
\item {Proveniência:(Do gr. \textunderscore dis\textunderscore  + \textunderscore oikia\textunderscore )}
\end{itemize}
Classe das plantas, que têm as flôres masculinas e as femininas em pés differentes.
\section{Diécico}
\begin{itemize}
\item {Grp. gram.:adj.}
\end{itemize}
Relativo á diecia.
\section{Diédrico}
\begin{itemize}
\item {Grp. gram.:adj.}
\end{itemize}
Relativo aos ângulos diedros.
\section{Diedro}
\begin{itemize}
\item {Grp. gram.:m.  e  adj.}
\end{itemize}
\begin{itemize}
\item {Utilização:Geom.}
\end{itemize}
\begin{itemize}
\item {Proveniência:(Do gr. \textunderscore dis\textunderscore  + \textunderscore edra\textunderscore )}
\end{itemize}
Diz-se do ângulo, formado pelo encontro de dois planos.
\section{Diedrogonómetro}
\begin{itemize}
\item {Grp. gram.:m.}
\end{itemize}
\begin{itemize}
\item {Proveniência:(Do gr. \textunderscore dis\textunderscore  + \textunderscore edra\textunderscore  + \textunderscore gomos\textunderscore  + \textunderscore metron\textunderscore )}
\end{itemize}
Instrumento, para medir ângulos diedros.
\section{Dieiro}
\begin{itemize}
\item {Grp. gram.:m.}
\end{itemize}
\begin{itemize}
\item {Utilização:Ant.}
\end{itemize}
O mesmo que \textunderscore dinheiro\textunderscore .
\section{Dieléctrico}
\begin{itemize}
\item {Grp. gram.:m.}
\end{itemize}
Substância ou objecto, que é insulador da electricidade.
\section{Dielitra}
\begin{itemize}
\item {Grp. gram.:f.}
\end{itemize}
Gênero de plantas fumariáceas.
\section{Dielytra}
\begin{itemize}
\item {Grp. gram.:f.}
\end{itemize}
Gênero de plantas fumariáceas.
\section{Diênia}
\begin{itemize}
\item {Grp. gram.:f.}
\end{itemize}
\begin{itemize}
\item {Proveniência:(Do gr. \textunderscore dienos\textunderscore )}
\end{itemize}
Planta bisannual, espécie de orchídea das regiões tropicaes.
\section{Diérese}
\begin{itemize}
\item {Grp. gram.:f.}
\end{itemize}
\begin{itemize}
\item {Utilização:Gram.}
\end{itemize}
Divisão de um ditongo em duas sýllabas.
Sinal orthográphico dessa divisão; trema.
Separação dos tecidos orgânicos, cuja contiguidade poderia sêr nociva.
\section{Dierético}
\begin{itemize}
\item {Grp. gram.:adj.}
\end{itemize}
\begin{itemize}
\item {Proveniência:(Gr. \textunderscore diairetikos\textunderscore )}
\end{itemize}
Relativo a diérese; em que há diérese.
\section{Diese}
\begin{itemize}
\item {Grp. gram.:f.}
\end{itemize}
\begin{itemize}
\item {Utilização:Mús.}
\end{itemize}
\begin{itemize}
\item {Utilização:Ant.}
\end{itemize}
\begin{itemize}
\item {Proveniência:(Gr. \textunderscore diesis\textunderscore )}
\end{itemize}
Sustenido; elevação de um meio tom, independentemente do sinal respectivo.
Meio tom.
Quarto de tom.
\section{Dieta}
\begin{itemize}
\item {Grp. gram.:f.}
\end{itemize}
\begin{itemize}
\item {Proveniência:(Gr. \textunderscore diaita\textunderscore )}
\end{itemize}
Modo de empregar o que é necessário á conservação da vida, tanto em saúde como em doença.
Hvgiene.
Abstenção de alguns ou de todos os alimentos, em caso de doença.
Privação de alimento.
Casa de recreio, pavilhão ou caramanchel, dentro de jardim.
\section{Dieta}
\begin{itemize}
\item {Grp. gram.:f.}
\end{itemize}
\begin{itemize}
\item {Utilização:Ant.}
\end{itemize}
Assembleia politica ou legislativa, em alguns países.
Reunião dos capítulos de certas Ordens religiosas.
Viagem, que se fazia num dia.
Trabalho de um dia.
Geira ou porção de terra, que se podía lavrar num dia com um só arado.
(B. lat. \textunderscore dieta\textunderscore )
\section{Dietário}
\begin{itemize}
\item {Grp. gram.:m.}
\end{itemize}
\begin{itemize}
\item {Utilização:Des.}
\end{itemize}
\begin{itemize}
\item {Proveniência:(De \textunderscore dieta\textunderscore ^1)}
\end{itemize}
Tratadista de dietas.
\section{Dietética}
\begin{itemize}
\item {Grp. gram.:f.}
\end{itemize}
\begin{itemize}
\item {Proveniência:(De \textunderscore dietético\textunderscore )}
\end{itemize}
Parte da Medicina, que trata da dieta^1.
\section{Dieteticamente}
\begin{itemize}
\item {Grp. gram.:adv.}
\end{itemize}
Segundo os preceitos da dietética.
\section{Dietético}
\begin{itemize}
\item {Grp. gram.:adj.}
\end{itemize}
\begin{itemize}
\item {Proveniência:(Gr. \textunderscore diaitetikos\textunderscore )}
\end{itemize}
Relativo a dieta^1.
\section{Difamação}
\begin{itemize}
\item {Grp. gram.:f.}
\end{itemize}
Acto ou efeito de difamar.
\section{Difamadamente}
\begin{itemize}
\item {Grp. gram.:adv.}
\end{itemize}
\begin{itemize}
\item {Proveniência:(De \textunderscore difamar\textunderscore )}
\end{itemize}
Com difamação.
\section{Difamador}
\begin{itemize}
\item {Grp. gram.:adj.}
\end{itemize}
\begin{itemize}
\item {Grp. gram.:M.}
\end{itemize}
Que difama.
Aquele que difama.
\section{Difamante}
\begin{itemize}
\item {Grp. gram.:adj.}
\end{itemize}
\begin{itemize}
\item {Proveniência:(Lat. \textunderscore diffamans\textunderscore )}
\end{itemize}
Que difama.
\section{Difamar}
\begin{itemize}
\item {Grp. gram.:v. t.}
\end{itemize}
\begin{itemize}
\item {Proveniência:(Lat. \textunderscore diffamare\textunderscore )}
\end{itemize}
Tirar a bôa fama ou o crédito a.
Desacreditar publicamente.
Caluniar.
\section{Difamatório}
\begin{itemize}
\item {Grp. gram.:adj.}
\end{itemize}
\begin{itemize}
\item {Proveniência:(Lat. \textunderscore diffamatorius\textunderscore )}
\end{itemize}
Que difama.
Em que há difamação: \textunderscore escrever um artigo difamatório\textunderscore .
\section{Difarreação}
\begin{itemize}
\item {Grp. gram.:f.}
\end{itemize}
\begin{itemize}
\item {Proveniência:(Lat. \textunderscore difarreatio\textunderscore )}
\end{itemize}
Dissolução solenne de casamento entre os Romanos, em que se fazia a offerta de um bolo de farinha.
Cp. \textunderscore confarreação\textunderscore .
\section{Diffamação}
\begin{itemize}
\item {Grp. gram.:f.}
\end{itemize}
Acto ou effeito de diffamar.
\section{Diffamadamente}
\begin{itemize}
\item {Grp. gram.:adv.}
\end{itemize}
\begin{itemize}
\item {Proveniência:(De \textunderscore diffamar\textunderscore )}
\end{itemize}
Com diffamação.
\section{Diffamador}
\begin{itemize}
\item {Grp. gram.:adj.}
\end{itemize}
\begin{itemize}
\item {Grp. gram.:M.}
\end{itemize}
Que diffama.
Aquelle que diffama.
\section{Diffamante}
\begin{itemize}
\item {Grp. gram.:adj.}
\end{itemize}
\begin{itemize}
\item {Proveniência:(Lat. \textunderscore diffamans\textunderscore )}
\end{itemize}
Que diffama.
\section{Diffamar}
\begin{itemize}
\item {Grp. gram.:v. t.}
\end{itemize}
\begin{itemize}
\item {Proveniência:(Lat. \textunderscore diffamare\textunderscore )}
\end{itemize}
Tirar a bôa fama ou o crédito a.
Desacreditar publicamente.
Calumniar.
\section{Diffamatório}
\begin{itemize}
\item {Grp. gram.:adj.}
\end{itemize}
\begin{itemize}
\item {Proveniência:(Lat. \textunderscore diffamatorius\textunderscore )}
\end{itemize}
Que diffama.
Em que há diffamação: \textunderscore escrever um artigo diffamatório\textunderscore .
\section{Diferença}
\begin{itemize}
\item {Grp. gram.:f.}
\end{itemize}
\begin{itemize}
\item {Proveniência:(Lat. \textunderscore differentia\textunderscore )}
\end{itemize}
Qualidade de quem ou daquilo que é diferente.
Falta de semelhança.
Alteração.
Desconformidade.
Diversidade.
Divergência.
Desavença.
Inexactidão.
Prejuizo, transtôrno.
\section{Diferençar}
\begin{itemize}
\item {Grp. gram.:v. t.}
\end{itemize}
\begin{itemize}
\item {Proveniência:(De \textunderscore diferença\textunderscore )}
\end{itemize}
Fazer diferença ou distinção entre.
Distinguir.
Tornár diverso.
Discriminar, notar.
\section{Diferençável}
\begin{itemize}
\item {Grp. gram.:adj.}
\end{itemize}
Que se póde diferençar.
\section{Diferenciação}
\begin{itemize}
\item {Grp. gram.:f.}
\end{itemize}
Acto ou efeito de \textunderscore diferenciar\textunderscore .
\section{Diferencial}
\begin{itemize}
\item {Grp. gram.:adj.}
\end{itemize}
\begin{itemize}
\item {Utilização:Mathem.}
\end{itemize}
\begin{itemize}
\item {Grp. gram.:F.}
\end{itemize}
\begin{itemize}
\item {Proveniência:(Do lat. \textunderscore differentia\textunderscore )}
\end{itemize}
Diz-se da quantidade ou do cálculo que procede por diferenças infinitamente pequenas.
Diz-se da taxa aduaneira, que varia segundo as procedências das marcadorias.
Aumento infinitamente pequeno de uma quantidade variável.
\section{Diferenciar}
\begin{itemize}
\item {Grp. gram.:v. t.}
\end{itemize}
\begin{itemize}
\item {Proveniência:(Do lat. \textunderscore differentia\textunderscore )}
\end{itemize}
Achar a diferencial de.
Diferençar.
\section{Diferente}
\begin{itemize}
\item {Grp. gram.:adj.}
\end{itemize}
\begin{itemize}
\item {Proveniência:(Lat. \textunderscore dífferens\textunderscore )}
\end{itemize}
Que difere.
Que é diverso.
Não semelhante.
Alterado.
Variado.
Desavindo.
\section{Diferentemente}
\begin{itemize}
\item {Grp. gram.:adv.}
\end{itemize}
De modo diferente.
\section{Diferir}
\begin{itemize}
\item {Grp. gram.:v. t.}
\end{itemize}
\begin{itemize}
\item {Grp. gram.:V. i.}
\end{itemize}
\begin{itemize}
\item {Proveniência:(Do lat. \textunderscore differre\textunderscore )}
\end{itemize}
Adiar.
Demorar; delongar.
Procrastinar.
Sêr diferente.
Distinguir-se
Discordar.
\section{Differença}
\begin{itemize}
\item {Grp. gram.:f.}
\end{itemize}
\begin{itemize}
\item {Proveniência:(Lat. \textunderscore differentia\textunderscore )}
\end{itemize}
Qualidade de quem ou daquillo que é differente.
Falta de semelhança.
Alteração.
Desconformidade.
Diversidade.
Divergência.
Desavença.
Inexactidão.
Prejuizo, transtôrno.
\section{Differençar}
\begin{itemize}
\item {Grp. gram.:v. t.}
\end{itemize}
\begin{itemize}
\item {Proveniência:(De \textunderscore differença\textunderscore )}
\end{itemize}
Fazer differença ou distincção entre.
Distinguir.
Tornár diverso.
Discriminar, notar.
\section{Differençável}
\begin{itemize}
\item {Grp. gram.:adj.}
\end{itemize}
Que se póde differençar.
\section{Differenciação}
\begin{itemize}
\item {Grp. gram.:f.}
\end{itemize}
Acto ou effeito de \textunderscore differenciar\textunderscore .
\section{Differencial}
\begin{itemize}
\item {Grp. gram.:adj.}
\end{itemize}
\begin{itemize}
\item {Utilização:Mathem.}
\end{itemize}
\begin{itemize}
\item {Grp. gram.:F.}
\end{itemize}
\begin{itemize}
\item {Proveniência:(Do lat. \textunderscore differentia\textunderscore )}
\end{itemize}
Diz-se da quantidade ou do cálculo que procede por differenças infinitamente pequenas.
Diz-se da taxa aduaneira, que varia segundo as procedências das marcadorias.
Aumento infinitamente pequeno de uma quantidade variável.
\section{Differenciar}
\begin{itemize}
\item {Grp. gram.:v. t.}
\end{itemize}
\begin{itemize}
\item {Proveniência:(Do lat. \textunderscore differentia\textunderscore )}
\end{itemize}
Achar a differencial de.
Differençar.
\section{Differente}
\begin{itemize}
\item {Grp. gram.:adj.}
\end{itemize}
\begin{itemize}
\item {Proveniência:(Lat. \textunderscore dífferens\textunderscore )}
\end{itemize}
Que differe.
Que é diverso.
Não semelhante.
Alterado.
Variado.
Desavindo.
\section{Differentemente}
\begin{itemize}
\item {Grp. gram.:adv.}
\end{itemize}
De modo differente.
\section{Differir}
\begin{itemize}
\item {Grp. gram.:v. t.}
\end{itemize}
\begin{itemize}
\item {Grp. gram.:V. i.}
\end{itemize}
\begin{itemize}
\item {Proveniência:(Do lat. \textunderscore differre\textunderscore )}
\end{itemize}
Adiar.
Demorar; delongar.
Procrastinar.
Sêr differente.
Distinguir-se
Discordar.
\section{Diffícil}
\begin{itemize}
\item {Grp. gram.:adj.}
\end{itemize}
\begin{itemize}
\item {Proveniência:(Lat. \textunderscore difficilis\textunderscore )}
\end{itemize}
Que não é fácil.
Árduo.
Trabalhoso.
Intrincado: \textunderscore problema diffícil\textunderscore .
Obscuro.
Penoso: \textunderscore momentos diffíceis\textunderscore .
Contrário.
Exigente.
Improvável: \textunderscore é diffícil que isso aconteça\textunderscore .
Intransitável: \textunderscore caminhos diffíceis\textunderscore .
\section{Difficíllimo}
\begin{itemize}
\item {Grp. gram.:adj.}
\end{itemize}
\begin{itemize}
\item {Proveniência:(Lat. \textunderscore difficillimus\textunderscore )}
\end{itemize}
Muito diffícil.
\section{Difficilíssimo}
\begin{itemize}
\item {Grp. gram.:adj.}
\end{itemize}
O mesmo que \textunderscore difficíllimo\textunderscore .
\section{Difficilmente}
\begin{itemize}
\item {Grp. gram.:adv.}
\end{itemize}
\begin{itemize}
\item {Proveniência:(De diffícil)}
\end{itemize}
Com difficuldade.
\section{Difficuldade}
\begin{itemize}
\item {Grp. gram.:f.}
\end{itemize}
\begin{itemize}
\item {Proveniência:(Lat. \textunderscore difficultas\textunderscore )}
\end{itemize}
Qualidade daquillo que é diffícil.
Aquillo que é diffícil.
Obstáculo, impedimento.
Objecção; repugnância.
Situação crítica.
\section{Difficultação}
\begin{itemize}
\item {Grp. gram.:f.}
\end{itemize}
Acto de difficultar. Cp. Herculano, \textunderscore Quest. Púb.\textunderscore , II, 105.
\section{Difficultar}
\begin{itemize}
\item {Grp. gram.:v. t.}
\end{itemize}
\begin{itemize}
\item {Grp. gram.:V. p.}
\end{itemize}
\begin{itemize}
\item {Proveniência:(Lat. \textunderscore difficultare\textunderscore )}
\end{itemize}
Tornar diffícil; apresentar como diffícil.
Recusar-se.
Não condescender.
\section{Difficultosamente}
\begin{itemize}
\item {Grp. gram.:adv.}
\end{itemize}
De modo difficultoso.
\section{Difficultoso}
\begin{itemize}
\item {Grp. gram.:adj.}
\end{itemize}
\begin{itemize}
\item {Proveniência:(Do lat. \textunderscore disfficultas\textunderscore )}
\end{itemize}
Em que há difficuldade; diffícil.
\section{Diffidência}
\begin{itemize}
\item {Grp. gram.:f.}
\end{itemize}
\begin{itemize}
\item {Utilização:Des.}
\end{itemize}
\begin{itemize}
\item {Proveniência:(Lat. \textunderscore diffidentia\textunderscore )}
\end{itemize}
Desconfiança.
\section{Diffidente}
\begin{itemize}
\item {Grp. gram.:adj.}
\end{itemize}
\begin{itemize}
\item {Utilização:Des.}
\end{itemize}
\begin{itemize}
\item {Proveniência:(Lat. \textunderscore diffidens\textunderscore )}
\end{itemize}
Desconfiado.
\section{Diffluência}
\begin{itemize}
\item {Grp. gram.:f.}
\end{itemize}
\begin{itemize}
\item {Proveniência:(Lat. \textunderscore diffluentia\textunderscore )}
\end{itemize}
Qualidade daquillo que é diffluente.
\section{Diffluente}
\begin{itemize}
\item {Grp. gram.:adj.}
\end{itemize}
\begin{itemize}
\item {Proveniência:(Lat. \textunderscore diffluens\textunderscore )}
\end{itemize}
Que difflue.
\section{Diffluir}
\begin{itemize}
\item {Grp. gram.:v. i.}
\end{itemize}
\begin{itemize}
\item {Proveniência:(Do lat. \textunderscore disffluere\textunderscore )}
\end{itemize}
Espalhar-se; diffundir-se.
\section{Diffracção}
\begin{itemize}
\item {Grp. gram.:f.}
\end{itemize}
\begin{itemize}
\item {Utilização:Phýs.}
\end{itemize}
\begin{itemize}
\item {Proveniência:(Lat. \textunderscore disffractio\textunderscore )}
\end{itemize}
Desvio dos raios luminosos, ao tocarem num corpo opaco.
\section{Diffractar}
\begin{itemize}
\item {Grp. gram.:v. t.}
\end{itemize}
\begin{itemize}
\item {Proveniência:(Do lat. \textunderscore diffractus\textunderscore )}
\end{itemize}
Fazer a diffracção de.
\section{Diffractivo}
\begin{itemize}
\item {Grp. gram.:adj.}
\end{itemize}
\begin{itemize}
\item {Proveniência:(Do lat. \textunderscore diffractus\textunderscore )}
\end{itemize}
Que póde causar diffracção.
\section{Diffringente}
\begin{itemize}
\item {Grp. gram.:adj.}
\end{itemize}
\begin{itemize}
\item {Proveniência:(Lat. \textunderscore diffringens\textunderscore )}
\end{itemize}
Que difracta.
\section{Diffundir}
\begin{itemize}
\item {Grp. gram.:v. t.}
\end{itemize}
\begin{itemize}
\item {Proveniência:(Lat. \textunderscore diffundere\textunderscore )}
\end{itemize}
Derramar, espalhar.
Dilatar, estender.
Divulgar.
\section{Diffusamente}
\begin{itemize}
\item {Grp. gram.:adv.}
\end{itemize}
De modo diffuso.
\section{Diffusão}
\begin{itemize}
\item {Grp. gram.:f.}
\end{itemize}
\begin{itemize}
\item {Proveniência:(Lat. \textunderscore diffusio\textunderscore )}
\end{itemize}
Acto de diffundir.
Prolixidade.
\section{Diffusível}
\begin{itemize}
\item {Grp. gram.:adj.}
\end{itemize}
\begin{itemize}
\item {Proveniência:(De \textunderscore diffuso\textunderscore )}
\end{itemize}
Que se póde diffundir.
\section{Diffusivo}
\begin{itemize}
\item {Grp. gram.:adj.}
\end{itemize}
\begin{itemize}
\item {Proveniência:(De \textunderscore diffuso\textunderscore )}
\end{itemize}
Diffusível.
Que tem acção rápida e enérgica sôbre o organismo.
\section{Diffuso}
\begin{itemize}
\item {Grp. gram.:adj.}
\end{itemize}
\begin{itemize}
\item {Proveniência:(Lat. \textunderscore diffusus\textunderscore )}
\end{itemize}
Diffundido.
Em que há diffusão.
\section{Difícil}
\begin{itemize}
\item {Grp. gram.:adj.}
\end{itemize}
\begin{itemize}
\item {Proveniência:(Lat. \textunderscore difficilis\textunderscore )}
\end{itemize}
Que não é fácil.
Árduo.
Trabalhoso.
Intrincado: \textunderscore problema difícil\textunderscore .
Obscuro.
Penoso: \textunderscore momentos difíceis\textunderscore .
Contrário.
Exigente.
Improvável: \textunderscore é difícil que isso aconteça\textunderscore .
Intransitável: \textunderscore caminhos difíceis\textunderscore .
\section{Dificílimo}
\begin{itemize}
\item {Grp. gram.:adj.}
\end{itemize}
\begin{itemize}
\item {Proveniência:(Lat. \textunderscore difficillimus\textunderscore )}
\end{itemize}
Muito difícil.
\section{Dificilíssimo}
\begin{itemize}
\item {Grp. gram.:adj.}
\end{itemize}
O mesmo que \textunderscore dificílimo\textunderscore .
\section{Dificilmente}
\begin{itemize}
\item {Grp. gram.:adv.}
\end{itemize}
\begin{itemize}
\item {Proveniência:(De difícil)}
\end{itemize}
Com dificuldade.
\section{Dificuldade}
\begin{itemize}
\item {Grp. gram.:f.}
\end{itemize}
\begin{itemize}
\item {Proveniência:(Lat. \textunderscore difficultas\textunderscore )}
\end{itemize}
Qualidade daquilo que é difícil.
Aquilo que é difícil.
Obstáculo, impedimento.
Objecção; repugnância.
Situação crítica.
\section{Dificultação}
\begin{itemize}
\item {Grp. gram.:f.}
\end{itemize}
Acto de dificultar. Cp. Herculano, \textunderscore Quest. Púb.\textunderscore , II, 105.
\section{Dificultar}
\begin{itemize}
\item {Grp. gram.:v. t.}
\end{itemize}
\begin{itemize}
\item {Grp. gram.:V. p.}
\end{itemize}
\begin{itemize}
\item {Proveniência:(Lat. \textunderscore difficultare\textunderscore )}
\end{itemize}
Tornar difícil; apresentar como difícil.
Recusar-se.
Não condescender.
\section{Dificultosamente}
\begin{itemize}
\item {Grp. gram.:adv.}
\end{itemize}
De modo dificultoso.
\section{Dificultoso}
\begin{itemize}
\item {Grp. gram.:adj.}
\end{itemize}
\begin{itemize}
\item {Proveniência:(Do lat. \textunderscore disfficultas\textunderscore )}
\end{itemize}
Em que há dificuldade; difícil.
\section{Difidência}
\begin{itemize}
\item {Grp. gram.:f.}
\end{itemize}
\begin{itemize}
\item {Utilização:Des.}
\end{itemize}
\begin{itemize}
\item {Proveniência:(Lat. \textunderscore diffidentia\textunderscore )}
\end{itemize}
Desconfiança.
\section{Difidente}
\begin{itemize}
\item {Grp. gram.:adj.}
\end{itemize}
\begin{itemize}
\item {Utilização:Des.}
\end{itemize}
\begin{itemize}
\item {Proveniência:(Lat. \textunderscore diffidens\textunderscore )}
\end{itemize}
Desconfiado.
\section{Diflorígero}
\begin{itemize}
\item {Grp. gram.:adj.}
\end{itemize}
\begin{itemize}
\item {Utilização:Bot.}
\end{itemize}
\begin{itemize}
\item {Proveniência:(De \textunderscore di...\textunderscore  + \textunderscore florigero\textunderscore )}
\end{itemize}
Que tem duas flôres.
\section{Difluência}
\begin{itemize}
\item {Grp. gram.:f.}
\end{itemize}
\begin{itemize}
\item {Proveniência:(Lat. \textunderscore diffluentia\textunderscore )}
\end{itemize}
Qualidade daquilo que é difluente.
\section{Difluente}
\begin{itemize}
\item {Grp. gram.:adj.}
\end{itemize}
\begin{itemize}
\item {Proveniência:(Lat. \textunderscore diffluens\textunderscore )}
\end{itemize}
Que diflue.
\section{Difracção}
\begin{itemize}
\item {Grp. gram.:f.}
\end{itemize}
\begin{itemize}
\item {Utilização:Phýs.}
\end{itemize}
\begin{itemize}
\item {Proveniência:(Lat. \textunderscore disffractio\textunderscore )}
\end{itemize}
Desvio dos raios luminosos, ao tocarem num corpo opaco.
\section{Difractar}
\begin{itemize}
\item {Grp. gram.:v. t.}
\end{itemize}
\begin{itemize}
\item {Proveniência:(Do lat. \textunderscore diffractus\textunderscore )}
\end{itemize}
Fazer a difracção de.
\section{Difractivo}
\begin{itemize}
\item {Grp. gram.:adj.}
\end{itemize}
\begin{itemize}
\item {Proveniência:(Do lat. \textunderscore diffractus\textunderscore )}
\end{itemize}
Que póde causar difracção.
\section{Difringente}
\begin{itemize}
\item {Grp. gram.:adj.}
\end{itemize}
\begin{itemize}
\item {Proveniência:(Lat. \textunderscore diffringens\textunderscore )}
\end{itemize}
Que difracta.
\section{Difundir}
\begin{itemize}
\item {Grp. gram.:v. t.}
\end{itemize}
\begin{itemize}
\item {Proveniência:(Lat. \textunderscore diffundere\textunderscore )}
\end{itemize}
Derramar, espalhar.
Dilatar, estender.
Divulgar.
\section{Difusamente}
\begin{itemize}
\item {Grp. gram.:adv.}
\end{itemize}
De modo difuso.
\section{Difusão}
\begin{itemize}
\item {Grp. gram.:f.}
\end{itemize}
\begin{itemize}
\item {Proveniência:(Lat. \textunderscore diffusio\textunderscore )}
\end{itemize}
Acto de difundir.
Prolixidade.
\section{Difusível}
\begin{itemize}
\item {Grp. gram.:adj.}
\end{itemize}
\begin{itemize}
\item {Proveniência:(De \textunderscore difuso\textunderscore )}
\end{itemize}
Que se póde difundir.
\section{Difusivo}
\begin{itemize}
\item {Grp. gram.:adj.}
\end{itemize}
\begin{itemize}
\item {Proveniência:(De \textunderscore difuso\textunderscore )}
\end{itemize}
Difusível.
Que tem acção rápida e enérgica sôbre o organismo.
\section{Difuso}
\begin{itemize}
\item {Grp. gram.:adj.}
\end{itemize}
\begin{itemize}
\item {Proveniência:(Lat. \textunderscore diffusus\textunderscore )}
\end{itemize}
Difundido.
Em que há difusão.
\section{Dígamo}
\begin{itemize}
\item {Grp. gram.:adj.}
\end{itemize}
\begin{itemize}
\item {Proveniência:(Do gr. \textunderscore dis\textunderscore  + \textunderscore gamos\textunderscore )}
\end{itemize}
Que participa dos dois sexos.
\section{Digar}
\begin{itemize}
\item {Grp. gram.:m.}
\end{itemize}
\begin{itemize}
\item {Utilização:Ant.}
\end{itemize}
Funccionário judicial na Índia portuguesa.
\section{Digástrico}
\begin{itemize}
\item {Grp. gram.:adj.}
\end{itemize}
\begin{itemize}
\item {Utilização:Anat.}
\end{itemize}
\begin{itemize}
\item {Proveniência:(De \textunderscore di...\textunderscore  + \textunderscore gástrico\textunderscore )}
\end{itemize}
Diz-se dos músculos, formados de duas partes carnosas, ligadas por um tendão.
\section{Diger}
\begin{itemize}
\item {Grp. gram.:v. t.}
\end{itemize}
\begin{itemize}
\item {Utilização:pop.}
\end{itemize}
\begin{itemize}
\item {Utilização:Ant.}
\end{itemize}
O mesmo que \textunderscore dizer\textunderscore . Cp. Lobo, \textunderscore Auto do Nascimento\textunderscore .
\section{Digerido}
\begin{itemize}
\item {Grp. gram.:adj.}
\end{itemize}
Que se digeriu.
Transformado pela digestão.
\section{Digerir}
\begin{itemize}
\item {Grp. gram.:v. t.}
\end{itemize}
\begin{itemize}
\item {Utilização:Fig.}
\end{itemize}
\begin{itemize}
\item {Proveniência:(Lat. \textunderscore digerere\textunderscore )}
\end{itemize}
Transformar pela digestão: \textunderscore digerir carne\textunderscore .
Supportar: \textunderscore digerir afrontas\textunderscore .
Estudar com proveito: \textunderscore não digere o que lê\textunderscore .
Macerar num liquido.
\section{Digerível}
\begin{itemize}
\item {Grp. gram.:adj.}
\end{itemize}
Que se digere facilmente.
Que se póde digerir.
\section{Digestão}
\begin{itemize}
\item {Grp. gram.:f.}
\end{itemize}
\begin{itemize}
\item {Proveniência:(Lat. \textunderscore digestio\textunderscore )}
\end{itemize}
Elaboração dos alimentos nas vias digestivas, ou funcção orgânica, caracterizada pela dissolução, liquefacção, e absorpção dos alimentos, com dejecção dos residuos.
Acto de digerir, (em sentido fig).
\section{Digestibilidade}
\begin{itemize}
\item {Grp. gram.:f.}
\end{itemize}
Qualidade daquillo que é digestível.
\section{Digestir}
\begin{itemize}
\item {Grp. gram.:v. t.}
\end{itemize}
\begin{itemize}
\item {Utilização:Ant.}
\end{itemize}
Digerir.
Supportar.
(Cp. \textunderscore digestão\textunderscore )
\section{Digestível}
\begin{itemize}
\item {Grp. gram.:adj.}
\end{itemize}
(V.digerível)
\section{Digestivo}
\begin{itemize}
\item {Grp. gram.:adj.}
\end{itemize}
\begin{itemize}
\item {Grp. gram.:M.}
\end{itemize}
\begin{itemize}
\item {Proveniência:(Lat. \textunderscore digestivus\textunderscore )}
\end{itemize}
Relativo á digestão.
Que facilita a digestão.
Que promove a supuração dos tumores.
Medicamento digestivo.
\section{Digesto}
\begin{itemize}
\item {Grp. gram.:m.}
\end{itemize}
\begin{itemize}
\item {Grp. gram.:Adj.}
\end{itemize}
\begin{itemize}
\item {Proveniência:(Lat. \textunderscore digestus\textunderscore )}
\end{itemize}
Compilação de leis romanas, organizada por ordem do imperador Justiniano.
Compilação de regras ou decisões jurídicas.
O mesmo que \textunderscore digerido\textunderscore .
\section{Diélia}
\begin{itemize}
\item {Grp. gram.:f.}
\end{itemize}
\begin{itemize}
\item {Utilização:Astron.}
\end{itemize}
\begin{itemize}
\item {Proveniência:(Do gr. \textunderscore dia\textunderscore  + \textunderscore helios\textunderscore )}
\end{itemize}
Ordenada da elipse terrestre, que passa pelo foco em que está o Sol.
\section{Digestor}
\begin{itemize}
\item {Grp. gram.:adj.}
\end{itemize}
\begin{itemize}
\item {Grp. gram.:M.}
\end{itemize}
\begin{itemize}
\item {Proveniência:(Do lat. \textunderscore digestus\textunderscore )}
\end{itemize}
Digestivo.
Apparelho, para a cocção de certas substâncias.
\section{Digestório}
\begin{itemize}
\item {Grp. gram.:adj.}
\end{itemize}
\begin{itemize}
\item {Proveniência:(Lat. \textunderscore digestorius\textunderscore )}
\end{itemize}
Que tem a fôrça ou a propriedade de digerir.
\section{Digínia}
\begin{itemize}
\item {Grp. gram.:f.}
\end{itemize}
\begin{itemize}
\item {Proveniência:(De \textunderscore digino\textunderscore )}
\end{itemize}
Nome de várias classes de plantas, no sistema de Lineu.
\section{Dígino}
\begin{itemize}
\item {Grp. gram.:adj.}
\end{itemize}
\begin{itemize}
\item {Utilização:Bot.}
\end{itemize}
\begin{itemize}
\item {Proveniência:(Do gr. \textunderscore dis\textunderscore  + \textunderscore gune\textunderscore )}
\end{itemize}
Diz-se das plantas, que têm dois pistilos distintos.
\section{Digitação}
\begin{itemize}
\item {Grp. gram.:f.}
\end{itemize}
\begin{itemize}
\item {Proveniência:(Do lat. \textunderscore digitatus\textunderscore )}
\end{itemize}
Qualidade do que é digitado.
\section{Digitado}
\begin{itemize}
\item {Grp. gram.:adj.}
\end{itemize}
\begin{itemize}
\item {Proveniência:(Lat. \textunderscore digitatus\textunderscore )}
\end{itemize}
Que tem fórma de dedos.
\section{Digital}
\begin{itemize}
\item {Grp. gram.:adj.}
\end{itemize}
\begin{itemize}
\item {Grp. gram.:F.}
\end{itemize}
\begin{itemize}
\item {Proveniência:(Lat. \textunderscore digitalis\textunderscore )}
\end{itemize}
Relativo aos dedos.
Planta, o mesmo que \textunderscore dedaleira\textunderscore .
\section{Digitalina}
\begin{itemize}
\item {Grp. gram.:f.}
\end{itemize}
\begin{itemize}
\item {Proveniência:(De \textunderscore digital\textunderscore )}
\end{itemize}
Substância medicinal, extrahida da dedaleira.
\section{Dígite}
\begin{itemize}
\item {Grp. gram.:m.}
\end{itemize}
\begin{itemize}
\item {Proveniência:(Do lat. \textunderscore digitus\textunderscore )}
\end{itemize}
Amuleto, em fórma de mão fechada.
Figa.
O mesmo que \textunderscore dixe\textunderscore .
\section{Digitifoliado}
\begin{itemize}
\item {Grp. gram.:adj.}
\end{itemize}
\begin{itemize}
\item {Proveniência:(Do lat. \textunderscore digitus\textunderscore  + \textunderscore folium\textunderscore )}
\end{itemize}
Que tem fôlhas digitadas.
\section{Digitiforme}
\begin{itemize}
\item {Grp. gram.:adj.}
\end{itemize}
\begin{itemize}
\item {Proveniência:(Do lat. \textunderscore digitus\textunderscore  + \textunderscore forma\textunderscore )}
\end{itemize}
O mesmo que \textunderscore digitado\textunderscore .
\section{Digitígrado}
\begin{itemize}
\item {Grp. gram.:adj.}
\end{itemize}
\begin{itemize}
\item {Grp. gram.:M. pl.}
\end{itemize}
\begin{itemize}
\item {Proveniência:(Do lat. \textunderscore digitus\textunderscore  + \textunderscore gradi\textunderscore )}
\end{itemize}
Que anda sobre as pontas dos dedos.
Animaes carnivoros, que andam sôbre as pontas dos dedos.
\section{Digito}
\begin{itemize}
\item {Grp. gram.:adj.}
\end{itemize}
\begin{itemize}
\item {Utilização:Arith.}
\end{itemize}
\begin{itemize}
\item {Grp. gram.:M.}
\end{itemize}
\begin{itemize}
\item {Utilização:Poét.}
\end{itemize}
\begin{itemize}
\item {Proveniência:(Lat. \textunderscore digitus\textunderscore )}
\end{itemize}
Diz-se de cada um dos números, dêsde um até déz.
Cada uma das doze partes, em que se divide o diâmetro da Lua ou do Sol, para se calcularem os eclipses.
O mesmo que \textunderscore dedo\textunderscore .
\section{Digladiador}
\begin{itemize}
\item {Grp. gram.:m.}
\end{itemize}
Aquelle que digladia.
\section{Digladiar}
\begin{itemize}
\item {Grp. gram.:v. i.}
\end{itemize}
\begin{itemize}
\item {Utilização:Fig.}
\end{itemize}
\begin{itemize}
\item {Proveniência:(Lat. \textunderscore digladiari\textunderscore )}
\end{itemize}
Combater com a espada.
Discutir com ardor.
\section{Digleno}
\begin{itemize}
\item {Grp. gram.:m.}
\end{itemize}
\begin{itemize}
\item {Proveniência:(Do gr. \textunderscore dis\textunderscore  + \textunderscore glene\textunderscore )}
\end{itemize}
Gênero de infusórios, que têm dois olhos e o pé fendido.
\section{Diglifo}
\begin{itemize}
\item {Grp. gram.:m.}
\end{itemize}
\begin{itemize}
\item {Utilização:Archit.}
\end{itemize}
\begin{itemize}
\item {Proveniência:(Gr. \textunderscore digluphos\textunderscore )}
\end{itemize}
Modilhão com duas estrias.
\section{Diglypho}
\begin{itemize}
\item {Grp. gram.:m.}
\end{itemize}
\begin{itemize}
\item {Utilização:Archit.}
\end{itemize}
\begin{itemize}
\item {Proveniência:(Gr. \textunderscore digluphos\textunderscore )}
\end{itemize}
Modilhão com duas estrias.
\section{Dignação}
\begin{itemize}
\item {Grp. gram.:f.}
\end{itemize}
\begin{itemize}
\item {Utilização:Des.}
\end{itemize}
\begin{itemize}
\item {Proveniência:(Lat. \textunderscore dignatio\textunderscore )}
\end{itemize}
Acto de dignar-se:«\textunderscore ó dignação incomparavel!\textunderscore »\textunderscore Luz e Calor\textunderscore , 213.
\section{Dignamente}
\begin{itemize}
\item {Grp. gram.:adv.}
\end{itemize}
De modo digno.
Honradamente: decorosamente.
\section{Dignar-se}
\begin{itemize}
\item {Grp. gram.:v. p.}
\end{itemize}
\begin{itemize}
\item {Proveniência:(Lat. \textunderscore dignari\textunderscore )}
\end{itemize}
Condescender.
Fazer mercê, favor; têr a generosidade, têr a bondade.
Sêr servido.
\section{Dignidade}
\begin{itemize}
\item {Grp. gram.:f.}
\end{itemize}
\begin{itemize}
\item {Utilização:Ant.}
\end{itemize}
\begin{itemize}
\item {Proveniência:(Lat. \textunderscore dignitas\textunderscore )}
\end{itemize}
Titulo ou cargo de graduação elevada.
Honraria.
Qualidade daquelle ou daquillo que é nobre e grande.
Modo de proceder, que se impõe ao respeito público.
Respeitabilidade.
Pundonor.
Seriedade; autoridade.
Nobreza.
Dignitário.
\section{Dignificação}
\begin{itemize}
\item {Grp. gram.:f.}
\end{itemize}
Acto de dignificar.
\section{Dignificador}
\begin{itemize}
\item {Grp. gram.:adj.}
\end{itemize}
Que dignifica.
\section{Dignificar}
\begin{itemize}
\item {Grp. gram.:v. t.}
\end{itemize}
\begin{itemize}
\item {Proveniência:(Do lat. \textunderscore dignus\textunderscore  + \textunderscore facere\textunderscore )}
\end{itemize}
Tornar digno.
Engrandecer; elevar a uma dignidade.
\section{Dignitário}
\begin{itemize}
\item {Grp. gram.:m.}
\end{itemize}
\begin{itemize}
\item {Proveniência:(Do lat. \textunderscore dignitas\textunderscore )}
\end{itemize}
Aquelle que exerce cargo elevado ou possue alta graduação honorífica.
\section{Digno}
\begin{itemize}
\item {Grp. gram.:adj.}
\end{itemize}
\begin{itemize}
\item {Proveniência:(Lat. \textunderscore dignus\textunderscore )}
\end{itemize}
Merecedor: \textunderscore digno de aprêço\textunderscore .
Apropriado.
Que tem dignidade.
Honrado, exemplar: \textunderscore é um homem digno\textunderscore .
Illustre.
Que vale a pena.
Hábil, habilitado.
Capaz.
\section{Dígono}
\begin{itemize}
\item {Grp. gram.:adj.}
\end{itemize}
\begin{itemize}
\item {Utilização:Geom.}
\end{itemize}
\begin{itemize}
\item {Proveniência:(Do gr. \textunderscore dis\textunderscore  + \textunderscore gonos\textunderscore )}
\end{itemize}
Que tem dois ângulos.
\section{Digrama}
\begin{itemize}
\item {Grp. gram.:m.}
\end{itemize}
\begin{itemize}
\item {Proveniência:(Do gr. \textunderscore dis\textunderscore  + \textunderscore gramma\textunderscore )}
\end{itemize}
Grupo de duas letras, que representa um só som ou uma só articulação.
\section{Digramma}
\begin{itemize}
\item {Grp. gram.:m.}
\end{itemize}
\begin{itemize}
\item {Proveniência:(Do gr. \textunderscore dis\textunderscore  + \textunderscore gramma\textunderscore )}
\end{itemize}
Grupo de duas letras, que representa um só som ou uma só articulação.
\section{Dígrafo}
\begin{itemize}
\item {Grp. gram.:m.}
\end{itemize}
O mesmo que \textunderscore digrama\textunderscore . Cf. Rui Barb., \textunderscore Réplica\textunderscore , 157.
\section{Dígrapho}
\begin{itemize}
\item {Grp. gram.:m.}
\end{itemize}
O mesmo que \textunderscore digramma\textunderscore . Cf. Rui Barb., \textunderscore Réplica\textunderscore , 157.
\section{Digressão}
\begin{itemize}
\item {Grp. gram.:f.}
\end{itemize}
\begin{itemize}
\item {Proveniência:(Lat. \textunderscore digressio\textunderscore )}
\end{itemize}
Acção de se separar.
Desvio, divagação.
Excursão, passeio.
Subterfúgio.
Desvio de um planeta, relativamente ao Sol.
\section{Digressivamente}
\begin{itemize}
\item {Grp. gram.:adv.}
\end{itemize}
De modo digressivo.
\section{Digressivo}
\begin{itemize}
\item {Grp. gram.:adj.}
\end{itemize}
\begin{itemize}
\item {Proveniência:(Do lat. \textunderscore digressus\textunderscore )}
\end{itemize}
Que se afasta.
Em que há digressão.
\section{Digresso}
\begin{itemize}
\item {Grp. gram.:m.}
\end{itemize}
\begin{itemize}
\item {Proveniência:(Lat. \textunderscore digressus\textunderscore )}
\end{itemize}
O mesmo que \textunderscore digressão\textunderscore .
\section{Digressoar}
\begin{itemize}
\item {Grp. gram.:v. i.}
\end{itemize}
\begin{itemize}
\item {Utilização:Neol.}
\end{itemize}
\begin{itemize}
\item {Proveniência:(De \textunderscore digressão\textunderscore )}
\end{itemize}
Fazer digressão ou digressões. Cf. Camillo, \textunderscore O Bem e o Mal\textunderscore , 157.
\section{Diguice}
\begin{itemize}
\item {Grp. gram.:f.}
\end{itemize}
\begin{itemize}
\item {Utilização:Bras}
\end{itemize}
Tolice; disparate.
\section{Digýnia}
\begin{itemize}
\item {Grp. gram.:f.}
\end{itemize}
\begin{itemize}
\item {Proveniência:(De \textunderscore digyno\textunderscore )}
\end{itemize}
Nome de várias classes de plantas, no systema de Linneu.
\section{Dígyno}
\begin{itemize}
\item {Grp. gram.:adj.}
\end{itemize}
\begin{itemize}
\item {Utilização:Bot.}
\end{itemize}
\begin{itemize}
\item {Proveniência:(Do gr. \textunderscore dis\textunderscore  + \textunderscore gune\textunderscore )}
\end{itemize}
Diz-se das plantas, que têm dois pistillos distintos.
\section{Dihélia}
\begin{itemize}
\item {Grp. gram.:f.}
\end{itemize}
\begin{itemize}
\item {Utilização:Astron.}
\end{itemize}
\begin{itemize}
\item {Proveniência:(Do gr. \textunderscore dia\textunderscore  + \textunderscore helios\textunderscore )}
\end{itemize}
Ordenada da ellipse terrestre, que passa pelo foco em que está o Sol.
\section{Dihýdrico}
\begin{itemize}
\item {Grp. gram.:adj.}
\end{itemize}
\begin{itemize}
\item {Proveniência:(De \textunderscore dí...\textunderscore  + rad. de \textunderscore hydrogênio\textunderscore )}
\end{itemize}
Diz-se das combinações chímicas, em que entram duas proporções de hydrogênio, por uma de outra substância componente.
\section{Diídrico}
\begin{itemize}
\item {Grp. gram.:adj.}
\end{itemize}
\begin{itemize}
\item {Proveniência:(De \textunderscore dí...\textunderscore  + rad. de \textunderscore hidrogênio\textunderscore )}
\end{itemize}
Diz-se das combinações químicas, em que entram duas proporções de hidrogênio, por uma de outra substância componente.
\section{Dijâmbico}
\begin{itemize}
\item {Grp. gram.:adj.}
\end{itemize}
Relativo ao dijambo.
\section{Dijambo}
\begin{itemize}
\item {Grp. gram.:m.}
\end{itemize}
\begin{itemize}
\item {Proveniência:(Gr. \textunderscore diiambos\textunderscore )}
\end{itemize}
Pé métrico, na versificação grega e latina, o qual consta de dois jambos.
\section{Dijole}
\begin{itemize}
\item {Grp. gram.:m.}
\end{itemize}
\begin{itemize}
\item {Proveniência:(T. lund.)}
\end{itemize}
Robusta árvore africana, de fôlhas alternas, em cruz, e flôres hermaphroditas, em amentilhos esphéricos.
\section{Dilação}
\begin{itemize}
\item {Grp. gram.:f.}
\end{itemize}
\begin{itemize}
\item {Proveniência:(Lat. \textunderscore dilatio\textunderscore )}
\end{itemize}
Acto ou effeito de dilatar.
Adiamento; demora.
Prazo.
\section{Dilaceração}
\begin{itemize}
\item {Grp. gram.:f.}
\end{itemize}
\begin{itemize}
\item {Proveniência:(Lat. \textunderscore dilaceratio\textunderscore )}
\end{itemize}
Acto ou effeito de dilacerar.
\section{Dilacerador}
\begin{itemize}
\item {Grp. gram.:m.}
\end{itemize}
Aquelle que dilacera. O mesmo que \textunderscore dilacerante\textunderscore .
\section{Dilaceramento}
\begin{itemize}
\item {Grp. gram.:m.}
\end{itemize}
Acto ou effeito de dilacerar.
\section{Dilacerante}
\begin{itemize}
\item {Grp. gram.:adj.}
\end{itemize}
\begin{itemize}
\item {Proveniência:(Lat. \textunderscore dilacerans\textunderscore )}
\end{itemize}
Que dilacera.
\section{Dilacerar}
\begin{itemize}
\item {Grp. gram.:v. t.}
\end{itemize}
\begin{itemize}
\item {Utilização:Fig.}
\end{itemize}
\begin{itemize}
\item {Proveniência:(Lat. \textunderscore dilacerare\textunderscore )}
\end{itemize}
Rasgar em pedaços; despedaçar com violência.
Affligir muito: \textunderscore desgraças, que dilaceram a alma\textunderscore .
\section{Dilaniar}
\begin{itemize}
\item {Grp. gram.:v. t.}
\end{itemize}
\begin{itemize}
\item {Utilização:Fig.}
\end{itemize}
\begin{itemize}
\item {Proveniência:(Lat. \textunderscore dilaniare\textunderscore )}
\end{itemize}
Dilacerar; espedaçar.
Criticar severamente.
\section{Dilapidação}
\begin{itemize}
\item {Grp. gram.:f.}
\end{itemize}
\begin{itemize}
\item {Proveniência:(Lat. \textunderscore dilapidatio\textunderscore )}
\end{itemize}
Acto de dilapidar.
\section{Dilapidador}
\begin{itemize}
\item {Grp. gram.:m.}
\end{itemize}
Aquelle que dilapida.
\section{Dilapidar}
\begin{itemize}
\item {Grp. gram.:v. t.}
\end{itemize}
\begin{itemize}
\item {Proveniência:(Lat. \textunderscore dilapidare\textunderscore )}
\end{itemize}
Dissipar, gastar extraordinariamente.
Estragar.
\section{Dilargado}
\begin{itemize}
\item {Grp. gram.:adj.}
\end{itemize}
\begin{itemize}
\item {Utilização:Ant.}
\end{itemize}
Dilatado, extenso.
\section{Dilatabilidade}
\begin{itemize}
\item {Grp. gram.:f.}
\end{itemize}
Propriedade daquillo que é dilatável.
\section{Dilatação}
\begin{itemize}
\item {Grp. gram.:f.}
\end{itemize}
\begin{itemize}
\item {Proveniência:(Lat. \textunderscore dilatatio\textunderscore )}
\end{itemize}
Acto ou effeito de dilatar; acto de avolumar.
Acto de alargar-se.
Operação cirúrgica, com que se alarga um canal.
Incremento.
Prolongamento.
Prorogação.
Expansão: \textunderscore a dilatação do Protestantismo\textunderscore .
\section{Dilatadamente}
\begin{itemize}
\item {Grp. gram.:adv.}
\end{itemize}
\begin{itemize}
\item {Proveniência:(De \textunderscore dilatado\textunderscore )}
\end{itemize}
Com dilatação.
\section{Dilatado}
\begin{itemize}
\item {Grp. gram.:adj.}
\end{itemize}
\begin{itemize}
\item {Proveniência:(De \textunderscore dilatar\textunderscore )}
\end{itemize}
Extenso; amplo: \textunderscore campos dilatados\textunderscore .
Aumentado.
Engrandecido; desenvolvido: \textunderscore o dilatado poderio da Inglaterra\textunderscore .
\section{Dilatador}
\begin{itemize}
\item {Grp. gram.:adj.}
\end{itemize}
\begin{itemize}
\item {Grp. gram.:M.}
\end{itemize}
\begin{itemize}
\item {Proveniência:(Lat. \textunderscore dilatator\textunderscore )}
\end{itemize}
Que dilata.
Instrumento cirúrgico, para alargar uma abertura ou separar tecidos.
\section{Dilatante}
\begin{itemize}
\item {Grp. gram.:adj.}
\end{itemize}
Que dilata.
\section{Dilatar}
\begin{itemize}
\item {Grp. gram.:v. t.}
\end{itemize}
\begin{itemize}
\item {Proveniência:(Lat. \textunderscore dilatare\textunderscore )}
\end{itemize}
Tornar largo, mais extenso, amplo.
Estender.
Espalhar, divulgar.
Demorar; retardar.
Fazer durar.
Prolongar.
Desenvolver.
Aumentar.
\section{Dilatativo}
\begin{itemize}
\item {Grp. gram.:adj.}
\end{itemize}
\begin{itemize}
\item {Utilização:Anat.}
\end{itemize}
\begin{itemize}
\item {Utilização:ant.}
\end{itemize}
Que serve para dilatar.
\section{Dilatável}
\begin{itemize}
\item {Grp. gram.:adj.}
\end{itemize}
Que se póde dilatar.
\section{Dilatório}
\begin{itemize}
\item {Grp. gram.:adj.}
\end{itemize}
\begin{itemize}
\item {Proveniência:(Lat. \textunderscore dilatorius\textunderscore )}
\end{itemize}
Que faz adiar.
Que retarda: \textunderscore desculpas dilatórias\textunderscore .
\section{Dilecção}
\begin{itemize}
\item {Grp. gram.:f.}
\end{itemize}
\begin{itemize}
\item {Proveniência:(Lat. \textunderscore dilectio\textunderscore )}
\end{itemize}
Estima.
Affecto racional, esclarecido.
\section{Dilecto}
\begin{itemize}
\item {Grp. gram.:adj.}
\end{itemize}
\begin{itemize}
\item {Proveniência:(Lat. \textunderscore dilectus\textunderscore )}
\end{itemize}
Amado, com inteiro conhecimento das suas bôas qualidades.
\section{Dilema}
\begin{itemize}
\item {Grp. gram.:m.}
\end{itemize}
\begin{itemize}
\item {Utilização:Fig.}
\end{itemize}
\begin{itemize}
\item {Proveniência:(Gr. \textunderscore dilemma\textunderscore )}
\end{itemize}
Argumento, que coloca o adversário entre duas proposições opostas, dificultando-lhe a saída ou a réplica.
Situação embaraçosa, de que não há saída senão por um de dois modos, ambos difíceis ou penosos.
\section{Dilemático}
\begin{itemize}
\item {Grp. gram.:adj.}
\end{itemize}
Relativo a dilema.
\section{Dilemma}
\begin{itemize}
\item {Grp. gram.:m.}
\end{itemize}
\begin{itemize}
\item {Utilização:Fig.}
\end{itemize}
\begin{itemize}
\item {Proveniência:(Gr. \textunderscore dilemma\textunderscore )}
\end{itemize}
Argumento, que colloca o adversário entre duas proposições oppostas, difficultando-lhe a saída ou a réplica.
Situação embaraçosa, de que não há saída senão por um de dois modos, ambos diffíceis ou penosos.
\section{Dilemmático}
\begin{itemize}
\item {Grp. gram.:adj.}
\end{itemize}
Relativo a dilemma.
\section{Dilênia}
\begin{itemize}
\item {Grp. gram.:f.}
\end{itemize}
Gênero de árvores copadas.
\section{Dileniáceas}
\begin{itemize}
\item {Grp. gram.:f. pl.}
\end{itemize}
\begin{itemize}
\item {Grp. gram.:f.}
\end{itemize}
\begin{itemize}
\item {Utilização:Bot.}
\end{itemize}
\begin{itemize}
\item {Proveniência:(De \textunderscore Dillen\textunderscore , n. p.)}
\end{itemize}
Família de plantas, estabelecida por De-Candole, e formada de árvores e arbustos, de que há 180 espécies conhecidas.
Família de plantas dicotiledóneas, polipétalas e hipogíneas, estabelecida por De-Candolle.
\section{Dilépido}
\begin{itemize}
\item {Grp. gram.:adj.}
\end{itemize}
\begin{itemize}
\item {Utilização:Bot.}
\end{itemize}
\begin{itemize}
\item {Proveniência:(Do gr. \textunderscore dis\textunderscore  + \textunderscore lepis\textunderscore )}
\end{itemize}
Que tem duas escamas unicamente.
\section{Dilettante}
\begin{itemize}
\item {Grp. gram.:m.  e  adj.}
\end{itemize}
\begin{itemize}
\item {Proveniência:(T. it.)}
\end{itemize}
Apreciador de música.
Amador.
Aquelle que exerce uma arte ou se dedica a um assumpto, exclusivamente por gôsto, e não por offício ou obrigação.
\section{Dilettantismo}
\begin{itemize}
\item {Grp. gram.:m.}
\end{itemize}
Qualidade de quem é dilettante.
(Italianismo, de \textunderscore dilettante\textunderscore )
\section{Diligência}
\begin{itemize}
\item {Grp. gram.:f.}
\end{itemize}
\begin{itemize}
\item {Proveniência:(Lat. \textunderscore diligentia\textunderscore )}
\end{itemize}
Zêlo, cuidado activo; actividade: \textunderscore trabalhar com diligência\textunderscore .
Investigação official; pesquisa.
Execução de certos serviços judiciaes.
Carruagem para serviço público ordinário, entre dois pontos ou entre duas localidades.
\section{Diligenciador}
\begin{itemize}
\item {Grp. gram.:m.}
\end{itemize}
Aquelle que diligencía.
\section{Diligenciar}
\begin{itemize}
\item {Grp. gram.:v. t.}
\end{itemize}
Procurar com diligência; esforçar-se por.
Empregar os meios para.
\section{Diligente}
\begin{itemize}
\item {Grp. gram.:adj.}
\end{itemize}
\begin{itemize}
\item {Proveniência:(Lat. \textunderscore diligens\textunderscore )}
\end{itemize}
Zeloso, que tem diligência.
Ligeiro, rápido.
\section{Diligentemente}
\begin{itemize}
\item {Grp. gram.:adv.}
\end{itemize}
De modo diligente.
\section{Dilleniáceas}
\begin{itemize}
\item {Grp. gram.:f. pl.}
\end{itemize}
\begin{itemize}
\item {Grp. gram.:f.}
\end{itemize}
\begin{itemize}
\item {Utilização:Bot.}
\end{itemize}
\begin{itemize}
\item {Proveniência:(De \textunderscore Dillen\textunderscore , n. p.)}
\end{itemize}
Família de plantas, estabelecida por De-Candolle, e formada de árvores e arbustos, de que há 180 espécies conhecidas.
Família de plantas dicotiledóneas, polipétalas e hipogíneas, estabelecida por De-Candolle.
\section{Dilobulado}
\begin{itemize}
\item {Grp. gram.:adj.}
\end{itemize}
\begin{itemize}
\item {Proveniência:(De \textunderscore di...\textunderscore  + \textunderscore lobulado\textunderscore )}
\end{itemize}
Que tem dois lóbulos.
\section{Dilochia}
\begin{itemize}
\item {fónica:qui}
\end{itemize}
\begin{itemize}
\item {Grp. gram.:f.}
\end{itemize}
\begin{itemize}
\item {Proveniência:(Do gr. \textunderscore dis\textunderscore  + \textunderscore lokhos\textunderscore )}
\end{itemize}
Reunião de dois lochos em formatura, na phalange macedónica.
\section{Dilogia}
\begin{itemize}
\item {Grp. gram.:f.}
\end{itemize}
(V.dialogia)
\section{Dilolo}
\begin{itemize}
\item {fónica:lô}
\end{itemize}
\begin{itemize}
\item {Grp. gram.:m.}
\end{itemize}
O mesmo que \textunderscore malolo\textunderscore .
\section{Diloquia}
\begin{itemize}
\item {Grp. gram.:f.}
\end{itemize}
\begin{itemize}
\item {Proveniência:(Do gr. \textunderscore dis\textunderscore  + \textunderscore lokhos\textunderscore )}
\end{itemize}
Reunião de dois lochos em formatura, na phalange macedónica.
\section{Dilucidamento}
\begin{itemize}
\item {Grp. gram.:m.}
\end{itemize}
Acto de dilucidar. Cf. Ortigão, \textunderscore Hollanda\textunderscore , 320.
\section{Dilucidar}
\textunderscore v. t.\textunderscore  (e der.)
O mesmo que \textunderscore elucidar\textunderscore , etc.
\section{Dilúcido}
\begin{itemize}
\item {Grp. gram.:adj.}
\end{itemize}
\begin{itemize}
\item {Proveniência:(Lat. \textunderscore dilucidus\textunderscore )}
\end{itemize}
O mesmo que \textunderscore lúcido\textunderscore .
\section{Dilucular}
\begin{itemize}
\item {Grp. gram.:adj.}
\end{itemize}
Relativo ao dilúculo.
\section{Dilúculo}
\begin{itemize}
\item {Grp. gram.:m.}
\end{itemize}
\begin{itemize}
\item {Proveniência:(Lat. \textunderscore diluculum\textunderscore )}
\end{itemize}
Crepúsculo da manhan.
Alvorada.
\section{Diluente}
\begin{itemize}
\item {Grp. gram.:adj.}
\end{itemize}
\begin{itemize}
\item {Proveniência:(Lat. \textunderscore diluens\textunderscore )}
\end{itemize}
Que dilue.
\section{Diluia}
\begin{itemize}
\item {Grp. gram.:f.}
\end{itemize}
Arbusto rubiáceo de Angola.
\section{Diluição}
\begin{itemize}
\item {fónica:lu-i}
\end{itemize}
\begin{itemize}
\item {Grp. gram.:f.}
\end{itemize}
Acto ou effeito de diluir.
\section{Diluimento}
\begin{itemize}
\item {fónica:lu-i}
\end{itemize}
\begin{itemize}
\item {Grp. gram.:m.}
\end{itemize}
O mesmo que \textunderscore diluição\textunderscore .
\section{Diluir}
\begin{itemize}
\item {Grp. gram.:v. t.}
\end{itemize}
\begin{itemize}
\item {Proveniência:(Lat. \textunderscore diluere\textunderscore )}
\end{itemize}
Misturar com água.
Deminuir a intensidade de, misturando-lhe água.
Dissolver.
\section{Diluto}
\begin{itemize}
\item {Grp. gram.:adj.}
\end{itemize}
\begin{itemize}
\item {Proveniência:(Lat. \textunderscore dilutus\textunderscore )}
\end{itemize}
Que se diluiu.
Dissolvido.
\section{Diluvial}
\begin{itemize}
\item {Grp. gram.:adj.}
\end{itemize}
\begin{itemize}
\item {Utilização:Fig.}
\end{itemize}
\begin{itemize}
\item {Proveniência:(De \textunderscore dilúvio\textunderscore )}
\end{itemize}
Relativo ao dilúvio universal.
Relativo a qualquer dilúvio.
Relativo a alluviões prehistóricas.
Torrencial; abundante.
\section{Diluviano}
\begin{itemize}
\item {Grp. gram.:adj.}
\end{itemize}
\begin{itemize}
\item {Utilização:Fig.}
\end{itemize}
\begin{itemize}
\item {Proveniência:(De \textunderscore dilúvio\textunderscore )}
\end{itemize}
Relativo ao dilúvio universal.
Relativo a qualquer dilúvio.
Relativo a alluviões prehistóricas.
Torrencial; abundante.
\section{Diluvião}
\begin{itemize}
\item {Grp. gram.:f.}
\end{itemize}
\begin{itemize}
\item {Proveniência:(Lat. \textunderscore diluvio\textunderscore )}
\end{itemize}
Terreno, em que há vestígios de alluviões anteriores aos tempos históricos.
\section{Diluviar}
\begin{itemize}
\item {Grp. gram.:v. i.}
\end{itemize}
\begin{itemize}
\item {Utilização:Neol.}
\end{itemize}
\begin{itemize}
\item {Proveniência:(De \textunderscore dilúvio\textunderscore )}
\end{itemize}
Chover copiosamente.
\section{Dilúvio}
\begin{itemize}
\item {Grp. gram.:m.}
\end{itemize}
\begin{itemize}
\item {Utilização:Gír.}
\end{itemize}
\begin{itemize}
\item {Proveniência:(Lat. \textunderscore diluvium\textunderscore )}
\end{itemize}
Inundação extraordinária.
Inundação universal, de que fala a \textunderscore Biblia\textunderscore .
Enorme porção de líquidos.
Innumerável affluência de homens ou de coisas, que se pricipitam como em dilúvio.
Grande abundância.
Caldo.
\section{Diluvioso}
\begin{itemize}
\item {Grp. gram.:adj.}
\end{itemize}
\begin{itemize}
\item {Proveniência:(De \textunderscore dilúvio\textunderscore )}
\end{itemize}
Torrencial.
Que produz inundação: \textunderscore temporal diluvioso\textunderscore .
\section{Diluvo}
\begin{itemize}
\item {Grp. gram.:m.}
\end{itemize}
Planta africana, herbácea, annual, de caule ramoso, quadrangular, fôlhas simples e glabras, e flôres irregulares.
\section{Dima}
\begin{itemize}
\item {Grp. gram.:f.}
\end{itemize}
Animal herbívoro africano, do tamanho de uma pequena cabra. Cf. Serpa Pinto, I, 299.
\section{Dimanação}
\begin{itemize}
\item {Grp. gram.:f.}
\end{itemize}
Acto de dimanar.
\section{Dimanante}
\begin{itemize}
\item {Grp. gram.:adj.}
\end{itemize}
\begin{itemize}
\item {Proveniência:(Lat. \textunderscore dimanans\textunderscore )}
\end{itemize}
Que dimana.
\section{Dimanar}
\begin{itemize}
\item {Grp. gram.:v. i.}
\end{itemize}
\begin{itemize}
\item {Proveniência:(Lat. \textunderscore dimanare\textunderscore )}
\end{itemize}
Brotar, fluir.
Derivar-se.
Provir; resultar: \textunderscore a sua pobreza dimanou da sua falta de juízo\textunderscore .
\section{Dimbedimbe}
\begin{itemize}
\item {Grp. gram.:m.}
\end{itemize}
Pássaro dentirostro da África Occidental.
\section{Dimensão}
\begin{itemize}
\item {Grp. gram.:f.}
\end{itemize}
\begin{itemize}
\item {Proveniência:(Lat. \textunderscore dimensio\textunderscore )}
\end{itemize}
Extensão, em qualquer sentido.
Grau de uma potência ou de uma equação, em álgebra.
\section{Dimensível}
\begin{itemize}
\item {Grp. gram.:adj.}
\end{itemize}
\begin{itemize}
\item {Proveniência:(Do lat. \textunderscore dimensus\textunderscore )}
\end{itemize}
Que se póde medir.
\section{Dimensório}
\begin{itemize}
\item {Grp. gram.:adj.}
\end{itemize}
\begin{itemize}
\item {Proveniência:(Do lat. \textunderscore dimensus\textunderscore )}
\end{itemize}
Relativo a dimensões.
\section{Dímero}
\begin{itemize}
\item {Grp. gram.:adj.}
\end{itemize}
\begin{itemize}
\item {Proveniência:(Do gr. \textunderscore dis\textunderscore  + \textunderscore meros\textunderscore )}
\end{itemize}
Que é composto de dois segmentos.
\section{Dimétrico}
\begin{itemize}
\item {Grp. gram.:adj.}
\end{itemize}
\begin{itemize}
\item {Proveniência:(Do gr. \textunderscore dis\textunderscore  + \textunderscore metron\textunderscore )}
\end{itemize}
Diz-se dos systemas crystallográphicos, que se definem por duas dimensões axiaes, uma para o eixo principal e outra para o secundário.
\section{Dímetro}
\begin{itemize}
\item {Grp. gram.:m.  e  adj.}
\end{itemize}
\begin{itemize}
\item {Proveniência:(Do gr. \textunderscore dis\textunderscore  + \textunderscore metron\textunderscore )}
\end{itemize}
Verso grego ou latino de dois pés.
\section{Dimidiação}
\begin{itemize}
\item {Grp. gram.:f.}
\end{itemize}
\begin{itemize}
\item {Proveniência:(Lat. \textunderscore dimidiatio\textunderscore )}
\end{itemize}
Acto ou effeito de dimidiar.
\section{Dimidiar}
\begin{itemize}
\item {Grp. gram.:v. t.}
\end{itemize}
\begin{itemize}
\item {Utilização:Des.}
\end{itemize}
\begin{itemize}
\item {Proveniência:(Lat. \textunderscore dimidiare\textunderscore )}
\end{itemize}
Dividir pelo meio; mear.
\section{Diminuir}
\textunderscore v. t.\textunderscore  (e der)
(V. \textunderscore deminuir\textunderscore , etc.)
\section{Dimissórias}
\begin{itemize}
\item {Grp. gram.:f.  e  adj. pl.}
\end{itemize}
\begin{itemize}
\item {Proveniência:(Lat. \textunderscore dimissoriae\textunderscore )}
\end{itemize}
Diz-se das letras ou cartas, pelas quaes um Prelado autoriza outro a conferir Ordens sacras a um diocesano daquelle.
\section{Dimittir}
\textunderscore v. t.\textunderscore  (e der.)
(V. \textunderscore demittir\textunderscore , etc.)
\section{Dimorfia}
\begin{itemize}
\item {Grp. gram.:f.}
\end{itemize}
O mesmo que \textunderscore dimorfismo\textunderscore .
\section{Dimorfismo}
\begin{itemize}
\item {Grp. gram.:m.}
\end{itemize}
Qualidade daquilo que é dimorfo.
\section{Dimorfo}
\begin{itemize}
\item {Grp. gram.:adj.}
\end{itemize}
\begin{itemize}
\item {Proveniência:(Do gr. \textunderscore dis\textunderscore  + \textunderscore morphe\textunderscore )}
\end{itemize}
Que póde tomar duas fórmas diferentes.
\section{Dimorphia}
\begin{itemize}
\item {Grp. gram.:f.}
\end{itemize}
O mesmo que \textunderscore dimorphismo\textunderscore .
\section{Dimorphismo}
\begin{itemize}
\item {Grp. gram.:m.}
\end{itemize}
Qualidade daquillo que é dimorpho.
\section{Dimorpho}
\begin{itemize}
\item {Grp. gram.:adj.}
\end{itemize}
\begin{itemize}
\item {Proveniência:(Do gr. \textunderscore dis\textunderscore  + \textunderscore morphe\textunderscore )}
\end{itemize}
Que póde tomar duas fórmas differentes.
\section{Dinamarquês}
\begin{itemize}
\item {Grp. gram.:adj.}
\end{itemize}
\begin{itemize}
\item {Grp. gram.:M.}
\end{itemize}
Relativo á Dinamarca.
Aquelle que é natural da Dinamarca.
Idioma da Dinamarca.
\section{Dinar}
\begin{itemize}
\item {Grp. gram.:m.}
\end{itemize}
Moéda indiana.
(Ár. \textunderscore dinar\textunderscore )
\section{Dincas}
\begin{itemize}
\item {Grp. gram.:m. pl.}
\end{itemize}
Povos da África oriental.
\section{Dindié}
\begin{itemize}
\item {Grp. gram.:m.}
\end{itemize}
Espécie de rôla africana.
\section{Dindinha}
\begin{itemize}
\item {Grp. gram.:f.}
\end{itemize}
\begin{itemize}
\item {Utilização:Bras}
\end{itemize}
\begin{itemize}
\item {Utilização:infant.}
\end{itemize}
Madrinha.
\section{Dinemo}
\begin{itemize}
\item {Grp. gram.:adj.}
\end{itemize}
\begin{itemize}
\item {Proveniência:(Do gr. \textunderscore dis\textunderscore  + \textunderscore nema\textunderscore )}
\end{itemize}
Que tem dois tentáculos.
\section{Dinga}
\begin{itemize}
\item {Grp. gram.:f.}
\end{itemize}
Barco da costa do Malabar.
\section{Dinguiânguia}
\begin{itemize}
\item {Grp. gram.:f.}
\end{itemize}
Espécie de codorniz africana.
\section{Dinhângoa}
\begin{itemize}
\item {Grp. gram.:f.}
\end{itemize}
Planta cucurbitácea de Angola, de grandes frutos que têm polpa côr de laranja.
\section{Dinheirada}
\begin{itemize}
\item {Grp. gram.:f.}
\end{itemize}
\begin{itemize}
\item {Utilização:Pop.}
\end{itemize}
\begin{itemize}
\item {Utilização:Ant.}
\end{itemize}
\begin{itemize}
\item {Proveniência:(De \textunderscore dinheiro\textunderscore )}
\end{itemize}
O mesmo que \textunderscore dinheirama\textunderscore .
Fazenda ou qualquer objecto, que valesse precisamente um dinheiro.
\section{Dinheiral}
\begin{itemize}
\item {Grp. gram.:m.}
\end{itemize}
O mesmo que \textunderscore dinheirama\textunderscore .
\section{Difalangarchia}
\begin{itemize}
\item {Grp. gram.:f.}
\end{itemize}
\begin{itemize}
\item {Proveniência:(De \textunderscore di...\textunderscore  + \textunderscore falangarchia\textunderscore )}
\end{itemize}
Formatura de 8192 homens ou duas falangarchias.
\section{Difia}
\begin{itemize}
\item {Grp. gram.:f.}
\end{itemize}
Peixe marítimo, tão diáfano, que mal se distingue na água. Cf. A. F. Simões, \textunderscore Cartas da Beira mar\textunderscore , 15.
\section{Difilo}
\begin{itemize}
\item {Grp. gram.:adj.}
\end{itemize}
\begin{itemize}
\item {Utilização:Bot.}
\end{itemize}
\begin{itemize}
\item {Proveniência:(Do gr. \textunderscore dis\textunderscore  + \textunderscore phulon\textunderscore )}
\end{itemize}
Que tem duas fôlhas.
\section{Difococo}
\begin{itemize}
\item {Grp. gram.:m.}
\end{itemize}
Micrococo, associado, dois a dois.
\section{Difonia}
\begin{itemize}
\item {Grp. gram.:f.}
\end{itemize}
\begin{itemize}
\item {Utilização:Mús.}
\end{itemize}
\begin{itemize}
\item {Proveniência:(Do gr. \textunderscore dis\textunderscore  + \textunderscore phone\textunderscore )}
\end{itemize}
Harmonia a duas partes.
\section{Difteria}
\begin{itemize}
\item {Grp. gram.:f.}
\end{itemize}
\begin{itemize}
\item {Proveniência:(Do gr. \textunderscore diphtera\textunderscore )}
\end{itemize}
Doença, caracterizada pela formação de falsas membranas nas mucosas da bôca e garganta e ainda sobre a pele.
Crupe; garrotilho.
\section{Diftérico}
\begin{itemize}
\item {Grp. gram.:adj.}
\end{itemize}
Relativo á difteria.
\section{Difterite}
\begin{itemize}
\item {Grp. gram.:f.}
\end{itemize}
O mesmo que \textunderscore difteria\textunderscore .
\section{Dinheirama}
\begin{itemize}
\item {Grp. gram.:f.}
\end{itemize}
\begin{itemize}
\item {Utilização:Pop.}
\end{itemize}
\begin{itemize}
\item {Proveniência:(De \textunderscore dinheiro\textunderscore )}
\end{itemize}
Muito dinheiro.
\section{Dinheirame}
\begin{itemize}
\item {Grp. gram.:m.}
\end{itemize}
O mesmo que \textunderscore dinheirama\textunderscore . Cf. Camillo, \textunderscore Narcóticos\textunderscore , 168.
\section{Dinheirão}
\begin{itemize}
\item {Grp. gram.:m.}
\end{itemize}
Grande e indeterminada quantia de dinheiro.
\section{Dinheiro}
\begin{itemize}
\item {Grp. gram.:m.}
\end{itemize}
\begin{itemize}
\item {Utilização:Fam.}
\end{itemize}
\begin{itemize}
\item {Proveniência:(Lat. \textunderscore denarius\textunderscore )}
\end{itemize}
Antiga moéda romana, que valeu 10 asses.
Antiga moéda portuguesa de cobre, cujo primitivo valor é desconhecido, mas que em tempo de Affonso V valia a terça parte de um ceitil ou duas mealhas.
Nome commum a todas as moédas: \textunderscore o dinheiro em prata é mais cómmodo que o dinheiro em cobre\textunderscore .
Quantia; numerário: \textunderscore isto custou muito dinheiro\textunderscore .
O mesmo que \textunderscore riqueza\textunderscore : \textunderscore o Neves é homem de dinheiro\textunderscore .
\section{Dinheiro-de-raposa}
\begin{itemize}
\item {Grp. gram.:m.}
\end{itemize}
\begin{itemize}
\item {Utilização:Prov.}
\end{itemize}
\begin{itemize}
\item {Utilização:trasm.}
\end{itemize}
\begin{itemize}
\item {Utilização:beir.}
\end{itemize}
Designação vulgar da mica, ou das lâminas brilhantes e finíssimas, que sobresaem na maior parte dos areaes, quando nellas incide a luz solar.
\section{Dinheiros-molhados}
\begin{itemize}
\item {Grp. gram.:m. pl.}
\end{itemize}
\begin{itemize}
\item {Utilização:Ant.}
\end{itemize}
Cereaes.
Gêneros alimentícios.
\section{Dinheiroso}
\begin{itemize}
\item {Grp. gram.:adj.}
\end{itemize}
\begin{itemize}
\item {Proveniência:(De \textunderscore dinheiro\textunderscore )}
\end{itemize}
Que tem muito dinheiro.
Rico.
\section{Dinheiros-secos}
\begin{itemize}
\item {Grp. gram.:m. pl.}
\end{itemize}
\begin{itemize}
\item {Utilização:Ant.}
\end{itemize}
Moéda.
\section{Dinheirudo}
\begin{itemize}
\item {Grp. gram.:adj.}
\end{itemize}
\begin{itemize}
\item {Utilização:Bras. do N}
\end{itemize}
O mesmo que \textunderscore dinheiroso\textunderscore .
\section{Dino}
\textunderscore adj.\textunderscore  (e der.)
(Fórma ant. de \textunderscore digno\textunderscore , etc., ainda usada na ilha de S. Jorge)
\section{Dinosáurios}
\begin{itemize}
\item {fónica:sau}
\end{itemize}
\begin{itemize}
\item {Grp. gram.:m. pl.}
\end{itemize}
Espécie fóssil de reptís marinhos.
\section{Dinossanios}
\begin{itemize}
\item {Grp. gram.:m. pl.}
\end{itemize}
Espécie fóssil de reptís marinhos.
\section{Dinotério}
\begin{itemize}
\item {Grp. gram.:m.}
\end{itemize}
\begin{itemize}
\item {Proveniência:(Do gr. \textunderscore deinos\textunderscore  + \textunderscore therion\textunderscore )}
\end{itemize}
Pachiderme fóssil dos terrenos terciários.
\section{Dinothério}
\begin{itemize}
\item {Grp. gram.:m.}
\end{itemize}
\begin{itemize}
\item {Proveniência:(Do gr. \textunderscore deinos\textunderscore  + \textunderscore therion\textunderscore )}
\end{itemize}
Pachiderme fóssil dos terrenos terciários.
\section{Dintel}
\begin{itemize}
\item {Grp. gram.:m.}
\end{itemize}
Vêrga de pedra ou ferro, que fórma a parte superior das portas ou janelas.
Cada um dos degraus lateraes, em que assentam as prateleiras da estante.
Lintel.(V.lintel)
\section{Dintorno}
\begin{itemize}
\item {fónica:tôr}
\end{itemize}
\begin{itemize}
\item {Grp. gram.:m.}
\end{itemize}
\begin{itemize}
\item {Proveniência:(De \textunderscore de\textunderscore  + \textunderscore em\textunderscore  + \textunderscore tôrno\textunderscore )}
\end{itemize}
Delineamentos de uma figura, contidos no contôrno.
\section{Dinumeração}
\begin{itemize}
\item {Grp. gram.:f.}
\end{itemize}
\begin{itemize}
\item {Proveniência:(Lat. \textunderscore dinumeratio\textunderscore )}
\end{itemize}
Acto de dinumerar.
\section{Dinumerar}
\begin{itemize}
\item {Grp. gram.:v. t.}
\end{itemize}
\begin{itemize}
\item {Proveniência:(Lat. \textunderscore dinumerare\textunderscore )}
\end{itemize}
Contar uma série ou collectividade de, por cada um dos seus elementos ou individualidades. Cf. Castilho, \textunderscore Fastos\textunderscore , I, 287.
\section{Dinúvio}
\begin{itemize}
\item {Grp. gram.:m.}
\end{itemize}
\begin{itemize}
\item {Utilização:Prov.}
\end{itemize}
\begin{itemize}
\item {Utilização:trasm.}
\end{itemize}
O mesmo que \textunderscore dilúvio\textunderscore . (Colhido em Sabrosa)
\section{Dio}
\begin{itemize}
\item {Grp. gram.:m.}
\end{itemize}
\begin{itemize}
\item {Proveniência:(T. it.)}
\end{itemize}
O mesmo que \textunderscore Deus\textunderscore . Cf. Usque, \textunderscore Passim\textunderscore .
\section{Diocesano}
\begin{itemize}
\item {Grp. gram.:adj.}
\end{itemize}
\begin{itemize}
\item {Grp. gram.:M.}
\end{itemize}
Relativo a diocese.
Aquelle que é de uma diocese.
\section{Diocese}
\begin{itemize}
\item {Grp. gram.:f.}
\end{itemize}
\begin{itemize}
\item {Proveniência:(Do gr. \textunderscore dioikesis\textunderscore )}
\end{itemize}
Antiga circunscripção administrativa em algumas províncias romanas.
Circunscripção territorial, sujeita á administração ecclesiástica de um Bispo, Arcebispo ou Patriarcha.
\section{Diódia}
\begin{itemize}
\item {Grp. gram.:f.}
\end{itemize}
\begin{itemize}
\item {Proveniência:(Do gr. \textunderscore diodos\textunderscore )}
\end{itemize}
Gênero de plantas rubiáceas.
\section{Dióico}
\begin{itemize}
\item {Grp. gram.:adj.}
\end{itemize}
\begin{itemize}
\item {Utilização:Bot.}
\end{itemize}
\begin{itemize}
\item {Proveniência:(Do gr. \textunderscore dis\textunderscore  + \textunderscore oikos\textunderscore )}
\end{itemize}
Relativo á disposição de flôres masculinas e femininas em dois indivíduos vegetaes differentes.
\section{Dioneia}
\begin{itemize}
\item {Grp. gram.:f.}
\end{itemize}
\begin{itemize}
\item {Proveniência:(Do gr. \textunderscore Dione\textunderscore , n. p.)}
\end{itemize}
Sensitiva americana, cujas fôlhas, contrahindo-se, apanham os insectos que nellas poisam.
\section{Dionina}
\begin{itemize}
\item {Grp. gram.:f.}
\end{itemize}
Medicamento sedativo e expectorante.
\section{Dionisíacas}
\begin{itemize}
\item {Grp. gram.:f. pl.}
\end{itemize}
\begin{itemize}
\item {Proveniência:(De \textunderscore dionisíaco\textunderscore )}
\end{itemize}
Festas antigas, em honra de Baco.
\section{Dionisíaco}
\begin{itemize}
\item {Grp. gram.:adj.}
\end{itemize}
\begin{itemize}
\item {Proveniência:(Lat. \textunderscore dionysiacus\textunderscore )}
\end{itemize}
Relativo a Baco, que também se chamava Dionísio.
Relativo ao rei português D. Dinis ou ao seu tempo.
\section{Dionysíacas}
\begin{itemize}
\item {Grp. gram.:f. pl.}
\end{itemize}
\begin{itemize}
\item {Proveniência:(De \textunderscore dionysíaco\textunderscore )}
\end{itemize}
Festas antigas, em honra de Baccho.
\section{Dionysíaco}
\begin{itemize}
\item {Grp. gram.:adj.}
\end{itemize}
\begin{itemize}
\item {Proveniência:(Lat. \textunderscore dionysiacus\textunderscore )}
\end{itemize}
Relativo a Baccho, que também se chamava Dionýsio.
Relativo ao rei português D. Dinis ou ao seu tempo.
\section{Diope}
\begin{itemize}
\item {Grp. gram.:f.}
\end{itemize}
Frauta simples, só com dois orifícios na extremidade do tubo, usada entre os Gregos.
\section{Dioptria}
\begin{itemize}
\item {Grp. gram.:f.}
\end{itemize}
\begin{itemize}
\item {Proveniência:(Do gr. \textunderscore dia\textunderscore  + \textunderscore optomai\textunderscore )}
\end{itemize}
Fôrça refringente de uma lente, que tenha um metro de distância focal.
\section{Dióptrica}
\begin{itemize}
\item {Grp. gram.:f.}
\end{itemize}
\begin{itemize}
\item {Proveniência:(Gr. \textunderscore dioptrikos\textunderscore )}
\end{itemize}
Parte da Phýsica, que trata dos phenómenos produzidos pela luz refractada, quando atravessa meios de densidades differentes.
\section{Dioptro}
\begin{itemize}
\item {Grp. gram.:m.}
\end{itemize}
\begin{itemize}
\item {Proveniência:(Gr. \textunderscore dioptron\textunderscore , de \textunderscore dia\textunderscore  + \textunderscore optomai\textunderscore )}
\end{itemize}
O mesmo que \textunderscore espéculo\textunderscore .
\section{Diorama}
\begin{itemize}
\item {Grp. gram.:m.}
\end{itemize}
\begin{itemize}
\item {Proveniência:(Do gr. \textunderscore dia\textunderscore  + \textunderscore orama\textunderscore )}
\end{itemize}
Quadro em tela, illuminado superiormente, e observado de lugar escuro, produzindo-se illusão óptica.
\section{Diorâmico}
\begin{itemize}
\item {Grp. gram.:adj.}
\end{itemize}
Relativo a diorama.
\section{Diorite}
\begin{itemize}
\item {Grp. gram.:f.}
\end{itemize}
(V. diorito)
\section{Diorítico}
\begin{itemize}
\item {Grp. gram.:adj.}
\end{itemize}
Que diz respeito ao diorito.
\section{Diorito}
\begin{itemize}
\item {Grp. gram.:m.}
\end{itemize}
\begin{itemize}
\item {Proveniência:(Do gr. \textunderscore diorizo\textunderscore )}
\end{itemize}
Rocha massiça, granulosa, cuja côr é geralmente semelhante á da diábase.
\section{Dioscórea}
\begin{itemize}
\item {Grp. gram.:f.}
\end{itemize}
\begin{itemize}
\item {Grp. gram.:F. pl.}
\end{itemize}
O mesmo que \textunderscore inhame\textunderscore .
Família de plantas, estabelecida por Brown á custa das asparagíneas de Jussieu.
\section{Diosma}
\begin{itemize}
\item {Grp. gram.:f.}
\end{itemize}
\begin{itemize}
\item {Proveniência:(Do gr. \textunderscore dios\textunderscore  + \textunderscore osme\textunderscore )}
\end{itemize}
Planta do Cabo da Bôa-Esperança, muito aromática.
\section{Diósmeas}
\begin{itemize}
\item {Grp. gram.:f. pl.}
\end{itemize}
\begin{itemize}
\item {Proveniência:(De \textunderscore diósmeo\textunderscore )}
\end{itemize}
Família de plantas, estabelecida por Brown, á custa das rutáceas de Jussieu, e que tem por typo a diosma.
Secção das rutáceas segundo De-Candolle.
\section{Diósmeo}
\begin{itemize}
\item {Grp. gram.:adj.}
\end{itemize}
Relativo ou semelhante á diosma.
\section{Diôso}
\begin{itemize}
\item {Grp. gram.:adj.}
\end{itemize}
\begin{itemize}
\item {Utilização:Prov.}
\end{itemize}
\begin{itemize}
\item {Utilização:minh.}
\end{itemize}
\begin{itemize}
\item {Utilização:Ant.}
\end{itemize}
\begin{itemize}
\item {Proveniência:(De \textunderscore dia\textunderscore , se não é metáth. de \textunderscore idoso\textunderscore )}
\end{itemize}
Que tem muitos dias.
Velho.
Ancião.
\section{Dióspiro}
\begin{itemize}
\item {Grp. gram.:m.}
\end{itemize}
\begin{itemize}
\item {Proveniência:(Lat. \textunderscore diospyron\textunderscore )}
\end{itemize}
Planta, conhecida também por \textunderscore litospermo\textunderscore  e \textunderscore erva-das-sete-sangrias\textunderscore .
Fruto do diospireiro.
\section{Dióspyro}
\begin{itemize}
\item {Grp. gram.:m.}
\end{itemize}
\begin{itemize}
\item {Proveniência:(Lat. \textunderscore diospyron\textunderscore )}
\end{itemize}
Planta, conhecida também por \textunderscore lithospermo\textunderscore  e \textunderscore erva-das-sete-sangrias\textunderscore .
Fruto do diospyreiro.
\section{Dioxia}
\begin{itemize}
\item {fónica:csi}
\end{itemize}
\begin{itemize}
\item {Grp. gram.:f.}
\end{itemize}
\begin{itemize}
\item {Utilização:Mús.}
\end{itemize}
Consonância de quinta perfeita, entre os Gregos.
\section{Dipa}
\begin{itemize}
\item {Grp. gram.:f.}
\end{itemize}
Pequeno peixe escamoso da região dos Ambuelas, na África.
\section{Dipétalo}
\begin{itemize}
\item {Grp. gram.:adj.}
\end{itemize}
\begin{itemize}
\item {Proveniência:(De \textunderscore di...\textunderscore  + \textunderscore pétala\textunderscore )}
\end{itemize}
Que tem duas pétalas.
\section{Diphalangarchia}
\begin{itemize}
\item {fónica:qui}
\end{itemize}
\begin{itemize}
\item {Grp. gram.:f.}
\end{itemize}
\begin{itemize}
\item {Proveniência:(De \textunderscore di...\textunderscore  + \textunderscore phalangarchia\textunderscore )}
\end{itemize}
Formatura de 8192 homens ou duas phalangarchias.
\section{Diphia}
\begin{itemize}
\item {Grp. gram.:f.}
\end{itemize}
Peixe marítimo, tão diáphano, que mal se distingue na água. Cf. A. F. Simões, \textunderscore Cartas da Beira mar\textunderscore , 15.
\section{Diphococco}
\begin{itemize}
\item {Grp. gram.:m.}
\end{itemize}
Micrococco, associado, dois a dois.
\section{Diphonia}
\begin{itemize}
\item {Grp. gram.:f.}
\end{itemize}
\begin{itemize}
\item {Utilização:Mús.}
\end{itemize}
\begin{itemize}
\item {Proveniência:(Do gr. \textunderscore dis\textunderscore  + \textunderscore phone\textunderscore )}
\end{itemize}
Harmonia a duas partes.
\section{Diphteria}
\begin{itemize}
\item {Grp. gram.:f.}
\end{itemize}
\begin{itemize}
\item {Proveniência:(Do gr. \textunderscore diphtera\textunderscore )}
\end{itemize}
Doença, caracterizada pela formação de falsas membranas nas mucosas da bôca e garganta e ainda sobre a pelle.
Crupe; garrotilho.
\section{Diphtérico}
\begin{itemize}
\item {Grp. gram.:adj.}
\end{itemize}
Relativo á diphteria.
\section{Diphterite}
\begin{itemize}
\item {Grp. gram.:f.}
\end{itemize}
O mesmo que \textunderscore diphteria\textunderscore .
\section{Diphthongo}
\begin{itemize}
\item {Grp. gram.:m.}
\end{itemize}
\begin{itemize}
\item {Proveniência:(Gr. \textunderscore diphthongos\textunderscore )}
\end{itemize}
\textunderscore m.\textunderscore  (e der.)
O mesmo que \textunderscore ditongo\textunderscore , etc.
Reunião de vogaes, que se pronunciam com uma só emissão de voz e formam uma só sýllaba.
\section{Diphyllo}
\begin{itemize}
\item {Grp. gram.:adj.}
\end{itemize}
\begin{itemize}
\item {Utilização:Bot.}
\end{itemize}
\begin{itemize}
\item {Proveniência:(Do gr. \textunderscore dis\textunderscore  + \textunderscore phulon\textunderscore )}
\end{itemize}
Que tem duas fôlhas.
\section{Dipirenado}
\begin{itemize}
\item {Grp. gram.:adj.}
\end{itemize}
\begin{itemize}
\item {Utilização:Bot.}
\end{itemize}
\begin{itemize}
\item {Proveniência:(Do gr. \textunderscore dis\textunderscore  + \textunderscore puren\textunderscore )}
\end{itemize}
Cujo fruto tem dois caroços.
\section{Dipírrico}
\begin{itemize}
\item {Grp. gram.:m.}
\end{itemize}
\begin{itemize}
\item {Proveniência:(De \textunderscore di...\textunderscore  + \textunderscore pírrico\textunderscore )}
\end{itemize}
Pé de verso antigo, com dois pírricos.
\section{Diplasiasmo}
\begin{itemize}
\item {Grp. gram.:m.}
\end{itemize}
\begin{itemize}
\item {Proveniência:(Gr. \textunderscore diplasiasmos\textunderscore )}
\end{itemize}
Injustificada duplicação de letras, na escrita de uma palavra.
\section{Diplegia}
\begin{itemize}
\item {Grp. gram.:f.}
\end{itemize}
\begin{itemize}
\item {Utilização:Med.}
\end{itemize}
Paralysia bilateral.
\section{Dipleiscópio}
\begin{itemize}
\item {Grp. gram.:m.}
\end{itemize}
Instrumento, para determinar precisamente o meio-dia.
\section{Díploe}
\begin{itemize}
\item {Grp. gram.:m.}
\end{itemize}
\begin{itemize}
\item {Utilização:Anat.}
\end{itemize}
\begin{itemize}
\item {Utilização:Ant.}
\end{itemize}
\begin{itemize}
\item {Proveniência:(Gr. \textunderscore diploe\textunderscore )}
\end{itemize}
Tecido esponjoso, entre as duas lâminas de tecido compacto, que formam os ossos do crânio.
Cada uma dessas duas lâminas.
\section{Diploico}
\begin{itemize}
\item {Grp. gram.:adj.}
\end{itemize}
Relativo ao díploe.
\section{Diploide}
\begin{itemize}
\item {Grp. gram.:m.}
\end{itemize}
\begin{itemize}
\item {Proveniência:(Gr. \textunderscore diplois\textunderscore )}
\end{itemize}
Vestido ou manto, que dava duas voltas ao corpo, e que era usado entre os antigos orientaes.
\section{Diploma}
\begin{itemize}
\item {Grp. gram.:m.}
\end{itemize}
\begin{itemize}
\item {Proveniência:(Gr. \textunderscore diploma\textunderscore )}
\end{itemize}
Título ou documento official, com que se confere um cargo, dignidade, mercê ou privilégio.
Título de contrato.
\section{Diplomacia}
\begin{itemize}
\item {Grp. gram.:f.}
\end{itemize}
\begin{itemize}
\item {Utilização:Fig.}
\end{itemize}
Conhecimento do direito, relações e interesses internacionaes.
Relações internacionaes, por meio de embaixadas ou legações.
Pessoal diplomático.
Procedimento astucioso.
Circunspecção ou discrição, observada na vida particular, á semelhança da que se usa entre diplomatas.
(Cp. \textunderscore diplomata\textunderscore )
\section{Diplomado}
\begin{itemize}
\item {Grp. gram.:adj.}
\end{itemize}
\begin{itemize}
\item {Proveniência:(De \textunderscore diploma\textunderscore )}
\end{itemize}
Que tem diploma ou título justificativo de certas habilitações literárias ou scientíficas.
\section{Diplomata}
\begin{itemize}
\item {Grp. gram.:m.}
\end{itemize}
\begin{itemize}
\item {Utilização:Fig.}
\end{itemize}
\begin{itemize}
\item {Proveniência:(Do rad. de \textunderscore diploma\textunderscore )}
\end{itemize}
Aquelle que faz parte do pessoal diplomático.
Aquelle que trata de diplomacia.
Homem astucioso.
Homem zeloso das conveniências sociaes.
\section{Diplomática}
\begin{itemize}
\item {Grp. gram.:f.}
\end{itemize}
\begin{itemize}
\item {Proveniência:(De \textunderscore diplomático\textunderscore ^2)}
\end{itemize}
Arte de lêr e conhecer os diplomas antigos.
Paleographia.
\section{Diplomaticamente}
\begin{itemize}
\item {Grp. gram.:adv.}
\end{itemize}
De modo diplomático^1.
\section{Diplomático}
\begin{itemize}
\item {Grp. gram.:adj.}
\end{itemize}
\begin{itemize}
\item {Utilização:Fig.}
\end{itemize}
\begin{itemize}
\item {Proveniência:(De \textunderscore diplomata\textunderscore )}
\end{itemize}
Relativo á diplomacia.
Discreto; grave.
Cortês; elegante.
\section{Diplomático}
\begin{itemize}
\item {Grp. gram.:adj.}
\end{itemize}
\begin{itemize}
\item {Grp. gram.:M.}
\end{itemize}
Relativo a diploma.
Aquelle que é versado em diplomática.
(Cp. \textunderscore diplomática\textunderscore )
\section{Diplomatista}
\begin{itemize}
\item {Grp. gram.:m.}
\end{itemize}
Aquelle que é versado em diplomática.
\section{Diplopapo}
\begin{itemize}
\item {Grp. gram.:m.}
\end{itemize}
Gênero de plantas compostas.
\section{Diplopappo}
\begin{itemize}
\item {Grp. gram.:m.}
\end{itemize}
Gênero de plantas compostas.
\section{Diplopia}
\begin{itemize}
\item {Grp. gram.:f.}
\end{itemize}
\begin{itemize}
\item {Utilização:Med.}
\end{itemize}
\begin{itemize}
\item {Proveniência:(Do gr. \textunderscore diploos\textunderscore  + \textunderscore ops\textunderscore )}
\end{itemize}
Doença dos olhos, que duplica a imagem dos objetos da visão.
\section{Diplóstemo}
\begin{itemize}
\item {Grp. gram.:adj.}
\end{itemize}
O mesmo que \textunderscore diplostêmone\textunderscore .
\section{Diplostêmone}
\begin{itemize}
\item {Grp. gram.:adj.}
\end{itemize}
\begin{itemize}
\item {Utilização:Bot.}
\end{itemize}
\begin{itemize}
\item {Proveniência:(Do gr. \textunderscore diploos\textunderscore  + \textunderscore stemon\textunderscore )}
\end{itemize}
Diz-se da flôr, em que o número dos estames é duplo do das pétalas, como succede no feijoeiro.
\section{Diplotáxis}
\begin{itemize}
\item {Grp. gram.:m.}
\end{itemize}
\begin{itemize}
\item {Proveniência:(Do gr. \textunderscore diploos\textunderscore  + \textunderscore taxis\textunderscore )}
\end{itemize}
Gênero de plantas crucíferas.
\section{Dipneumóneo}
\begin{itemize}
\item {Grp. gram.:adj.}
\end{itemize}
\begin{itemize}
\item {Proveniência:(Do gr. \textunderscore dis\textunderscore  + \textunderscore pneumon\textunderscore )}
\end{itemize}
Que tem dois pulmões ou dois sacos pneumonares.
\section{Dipneus}
\begin{itemize}
\item {Grp. gram.:m. pl.}
\end{itemize}
\begin{itemize}
\item {Proveniência:(Do gr. \textunderscore dis\textunderscore  + \textunderscore pneumon\textunderscore )}
\end{itemize}
Animaes, que têm dois pulmões.
\section{Dipneustas}
\begin{itemize}
\item {Grp. gram.:m. pl.}
\end{itemize}
\begin{itemize}
\item {Proveniência:(Do gr. \textunderscore dis\textunderscore  + \textunderscore pneumon\textunderscore )}
\end{itemize}
Animaes, que têm dois pulmões.
\section{Dipnodos}
\begin{itemize}
\item {Grp. gram.:m. pl.}
\end{itemize}
Subclasse de peixes, segundo Muller.
\section{Dipnoico}
\begin{itemize}
\item {Grp. gram.:adj.}
\end{itemize}
O mesmo que \textunderscore pneumobrânchio\textunderscore .
\section{Dípode}
\begin{itemize}
\item {Grp. gram.:adj.}
\end{itemize}
\begin{itemize}
\item {Utilização:Zool.}
\end{itemize}
\begin{itemize}
\item {Proveniência:(Do gr. \textunderscore dis\textunderscore  + \textunderscore pous\textunderscore , podos)}
\end{itemize}
Que tem dois pés ou dois membros análogos aos pés.
\section{Dipodia}
\begin{itemize}
\item {Grp. gram.:f.}
\end{itemize}
\begin{itemize}
\item {Proveniência:(Gr. \textunderscore dipodia\textunderscore )}
\end{itemize}
Reunião de dois pés de verso grego ou latino.
\section{Dipôndio}
\begin{itemize}
\item {Grp. gram.:m.}
\end{itemize}
O mesmo que \textunderscore dupôndio\textunderscore .
\section{Dipsáceas}
\begin{itemize}
\item {Grp. gram.:f. pl.}
\end{itemize}
Família de plantas, que têm por typo o dípsaco.
\section{Dípsaco}
\begin{itemize}
\item {Grp. gram.:m.}
\end{itemize}
\begin{itemize}
\item {Proveniência:(Gr. \textunderscore dipsakos\textunderscore )}
\end{itemize}
Planta, também conhecida por \textunderscore cardo penteador\textunderscore .
\section{Dípsada}
\begin{itemize}
\item {Grp. gram.:f.}
\end{itemize}
Serpente venenosa da África.
\section{Dipsético}
\begin{itemize}
\item {Grp. gram.:adj.}
\end{itemize}
\begin{itemize}
\item {Proveniência:(Do gr. \textunderscore dipsios\textunderscore )}
\end{itemize}
Que produz sêde.
\section{Dipsomania}
\begin{itemize}
\item {Grp. gram.:f.}
\end{itemize}
\begin{itemize}
\item {Utilização:Med.}
\end{itemize}
\begin{itemize}
\item {Proveniência:(Do gr. \textunderscore dipsa\textunderscore  + \textunderscore mania\textunderscore )}
\end{itemize}
Impulsão mórbida, que faz que o indivíduo tenha muita sêde, ou que o degenerado beba em excesso.
\section{Dipsomaníaco}
\begin{itemize}
\item {Grp. gram.:adj.}
\end{itemize}
\begin{itemize}
\item {Grp. gram.:M.}
\end{itemize}
Relativo a dipsomania.
Aquelle que soffre dipsomania.
\section{Dipteracantho}
\begin{itemize}
\item {Grp. gram.:m.}
\end{itemize}
\begin{itemize}
\item {Proveniência:(Do gr. \textunderscore dipteron\textunderscore  + \textunderscore akantha\textunderscore )}
\end{itemize}
Gênero de plantas acantháceas.
\section{Dipteracanto}
\begin{itemize}
\item {Grp. gram.:m.}
\end{itemize}
\begin{itemize}
\item {Proveniência:(Do gr. \textunderscore dipteron\textunderscore  + \textunderscore akantha\textunderscore )}
\end{itemize}
Gênero de plantas acantáceas.
\section{Dipterácea}
\begin{itemize}
\item {Grp. gram.:adj. f.}
\end{itemize}
Diz-se de uma ordem de plantas, cujas sementes são guarnecidas de duas asas.
(Cp. \textunderscore díptero\textunderscore )
\section{Diptérico}
\begin{itemize}
\item {Grp. gram.:adj.}
\end{itemize}
Relativo a díptero.
\section{Dipterígios}
\begin{itemize}
\item {Grp. gram.:m. pl.}
\end{itemize}
\begin{itemize}
\item {Proveniência:(Do gr. \textunderscore dis\textunderscore  + \textunderscore pterux\textunderscore )}
\end{itemize}
Peixes, que têm duas barbatanas.
\section{Díptero}
\begin{itemize}
\item {Grp. gram.:adj.}
\end{itemize}
\begin{itemize}
\item {Grp. gram.:M.}
\end{itemize}
\begin{itemize}
\item {Grp. gram.:Pl.}
\end{itemize}
\begin{itemize}
\item {Proveniência:(Do gr. \textunderscore dis\textunderscore  + \textunderscore pteron\textunderscore )}
\end{itemize}
Que têm duas asas.
Templo antígo, com duas ordens de columnas, sendo oito anteriores e oito posteriores.
Ordem de insectos, que têm duas asas.
\section{Dipterologia}
\begin{itemize}
\item {Grp. gram.:f.}
\end{itemize}
Tratado á cêrca dos insectos dípteros.
\section{Dipterológico}
\begin{itemize}
\item {Grp. gram.:adj.}
\end{itemize}
Relativo a dipterologia.
\section{Dipterólogo}
\begin{itemize}
\item {Grp. gram.:m.}
\end{itemize}
Naturalista, que se dedica á dipterologia.
\section{Dipterýgia}
\begin{itemize}
\item {Grp. gram.:f.}
\end{itemize}
Gênero de plantas crucíferas.
\section{Dipterýgios}
\begin{itemize}
\item {Grp. gram.:m. pl.}
\end{itemize}
\begin{itemize}
\item {Proveniência:(Do gr. \textunderscore dis\textunderscore  + \textunderscore pterux\textunderscore )}
\end{itemize}
Peixes, que têm duas barbatanas.
\section{Dípticos}
\begin{itemize}
\item {Grp. gram.:m. pl.}
\end{itemize}
\begin{itemize}
\item {Utilização:Ant.}
\end{itemize}
\begin{itemize}
\item {Proveniência:(Gr. \textunderscore diptukos\textunderscore )}
\end{itemize}
Tábuas, em que se inscrevia o nome dos principaes magistrados.
Registos monásticos, que tinham os nomes dos Bispos e bem-feitores, por quem se devia rezar.
\section{Díptycos}
\begin{itemize}
\item {Grp. gram.:m. pl.}
\end{itemize}
\begin{itemize}
\item {Utilização:Ant.}
\end{itemize}
\begin{itemize}
\item {Proveniência:(Gr. \textunderscore diptukos\textunderscore )}
\end{itemize}
Tábuas, em que se inscrevia o nome dos principaes magistrados.
Registos monásticos, que tinham os nomes dos Bispos e bem-feitores, por quem se devia rezar.
\section{Dipyrenado}
\begin{itemize}
\item {Grp. gram.:adj.}
\end{itemize}
\begin{itemize}
\item {Utilização:Bot.}
\end{itemize}
\begin{itemize}
\item {Proveniência:(Do gr. \textunderscore dis\textunderscore  + \textunderscore puren\textunderscore )}
\end{itemize}
Cujo fruto tem dois caroços.
\section{Dipýrrhico}
\begin{itemize}
\item {Grp. gram.:m.}
\end{itemize}
\begin{itemize}
\item {Proveniência:(De \textunderscore di...\textunderscore  + \textunderscore pýrrhico\textunderscore )}
\end{itemize}
Pé de verso antigo, com dois pýrrhicos.
\section{Dique}
\begin{itemize}
\item {Grp. gram.:m.}
\end{itemize}
\begin{itemize}
\item {Proveniência:(Do hol. \textunderscore dyk\textunderscore ?)}
\end{itemize}
Construcção para represar águas correntes.
Represa.
Reservatório, com comporta.
Doca.
Açude.
\section{Diquara}
\begin{itemize}
\item {Grp. gram.:f.}
\end{itemize}
\begin{itemize}
\item {Utilização:Bras. dos sertões do N}
\end{itemize}
O mesmo que \textunderscore tiquara\textunderscore .
\section{Dir}
Fórma do infinito do verbo \textunderscore dizer\textunderscore , a qual serve para formar os futuros e os condicionaes periphrásticos: \textunderscore dir-te-ei\textunderscore , \textunderscore dir-se-á\textunderscore .
(Evolução do lat. \textunderscore dic(e)re\textunderscore  &lt; \textunderscore digre\textunderscore  &lt; \textunderscore dire\textunderscore )
\section{Diradicar}
\begin{itemize}
\item {fónica:ra}
\end{itemize}
\begin{itemize}
\item {Proveniência:(Do lat. \textunderscore de\textunderscore  + \textunderscore radix\textunderscore )}
\end{itemize}
\textunderscore v. t.\textunderscore  (e der.)
O mesmo que \textunderscore desarraigar\textunderscore , etc.--Us. por Latino, \textunderscore Hist. Pol. e Mil.\textunderscore , 115; mas melhor seria \textunderscore deradicar\textunderscore .
\section{Dirandela}
\begin{itemize}
\item {Grp. gram.:f.}
\end{itemize}
O mesmo que \textunderscore arandela\textunderscore .
\section{Dirandina}
\begin{itemize}
\item {Grp. gram.:f.}
\end{itemize}
Movimento circular?:«\textunderscore ... uma roda que andava em dirandina, á luz de uma vela...\textunderscore »Filinto, I, 142.
\section{Direcção}
\begin{itemize}
\item {Grp. gram.:f.}
\end{itemize}
\begin{itemize}
\item {Proveniência:(Lat. \textunderscore directio\textunderscore )}
\end{itemize}
Acto de dirigir.
Administração: \textunderscore a direcção de um Banco\textunderscore .
Cargo de director.
Secretaria, que está a cargo de um director.
Circunscripção sujeita a um director.
Banda ou lado, para que está voltada uma pessôa ou coisa, ou para onde alguém caminha: \textunderscore tomou a direcção do mar\textunderscore .
Escopo, fim.
Linha recta, direitura.
\section{Directamente}
\begin{itemize}
\item {Grp. gram.:adv.}
\end{itemize}
De modo directo: \textunderscore tratar directamente com alguém\textunderscore .
\section{Directivo}
\begin{itemize}
\item {Grp. gram.:adj.}
\end{itemize}
\begin{itemize}
\item {Proveniência:(De \textunderscore directo\textunderscore )}
\end{itemize}
Que dirige.
\section{Directo}
\begin{itemize}
\item {Grp. gram.:adj.}
\end{itemize}
\begin{itemize}
\item {Utilização:Gram.}
\end{itemize}
\begin{itemize}
\item {Utilização:Gram.}
\end{itemize}
\begin{itemize}
\item {Utilização:Astron.}
\end{itemize}
\begin{itemize}
\item {Utilização:Mús.}
\end{itemize}
\begin{itemize}
\item {Proveniência:(Lat. \textunderscore directus\textunderscore )}
\end{itemize}
Que está ou vai em linha recta.
Que não tem intermediário; immediato.
Recto.
Direito.
Que não tem rodeios ou circunlóquios: \textunderscore accusação directa\textunderscore .
Relativo ao parentesco ascendente ou descendente.
Diz-se das contribuições, que incidem directamente nos bens ou nas pessôas.
Diz-se da linguagem, em que se observa a ordem lógica das palavras.
Diz-se do objecto, em que recái a acção de um verbo, geralmente sem proposição intermediária.
Diz-se do movimento dos astros, do Occidente para o Oriente, e diz-se dos planetas que tem êsse movimento.
Diz-se do intervallo, que se harmoniza com o som fundamental que o produz.
\section{Director}
\begin{itemize}
\item {Grp. gram.:m.  e  adj.}
\end{itemize}
\begin{itemize}
\item {Proveniência:(Lat. \textunderscore director\textunderscore )}
\end{itemize}
O que dirige ou administra.
Mentor; guia: \textunderscore um director espiritual\textunderscore .
\section{Directorado}
\begin{itemize}
\item {Grp. gram.:m.}
\end{itemize}
Cargo de director.
\section{Directoria}
\begin{itemize}
\item {Grp. gram.:f.}
\end{itemize}
\begin{itemize}
\item {Proveniência:(De \textunderscore director\textunderscore )}
\end{itemize}
Acto de dirigir.
Cargo de director; corporação que elle dirige.
\section{Directorial}
\begin{itemize}
\item {Grp. gram.:adj.}
\end{itemize}
Relativo a directório.
\section{Directório}
\begin{itemize}
\item {Grp. gram.:adj.}
\end{itemize}
\begin{itemize}
\item {Grp. gram.:M.}
\end{itemize}
\begin{itemize}
\item {Proveniência:(Lat. \textunderscore directorius\textunderscore )}
\end{itemize}
Que dirige.
Papel, folheto ou livro, que contém direcções e indicações necessárias para o desempenho de certo mister ou para a execução de determinados negócios.
Conselho, encarregado da gerência de negócios públicos.
Commissão directora: \textunderscore o Directório do partido republicano\textunderscore .
\section{Directriz}
\begin{itemize}
\item {Grp. gram.:adj. f.}
\end{itemize}
\begin{itemize}
\item {Grp. gram.:F.}
\end{itemize}
\begin{itemize}
\item {Utilização:Geom.}
\end{itemize}
Que dirige.
Linha, ao longo ou em volta da qual se faz correr outra linha ou uma superfície, para produzir uma figura plana ou um sólido.
Linha, que regula o traçado de um caminho, de uma estrada.
(Fem. de \textunderscore director\textunderscore )
\section{Direita}
\begin{itemize}
\item {Grp. gram.:f.}
\end{itemize}
\begin{itemize}
\item {Proveniência:(De \textunderscore direito\textunderscore )}
\end{itemize}
O mesmo que \textunderscore destra\textunderscore .
Lado direito: \textunderscore fiquei á direita do presidente\textunderscore .
Grupo parlamentar, que se senta ao lado direito do presidente da respectiva assembleia.
\section{Direitamente}
\begin{itemize}
\item {Grp. gram.:adv.}
\end{itemize}
O mesmo que \textunderscore directamente\textunderscore .
De modo direito, com justiça, com razão. Cf. \textunderscore Luz e Calor\textunderscore , 320.
\section{Direiteza}
\begin{itemize}
\item {Grp. gram.:f.}
\end{itemize}
(V.direitura)
\section{Direito}
\begin{itemize}
\item {Grp. gram.:adj.}
\end{itemize}
\begin{itemize}
\item {Grp. gram.:Adv.}
\end{itemize}
\begin{itemize}
\item {Proveniência:(Do lat. \textunderscore directus\textunderscore )}
\end{itemize}
Directo.
Que segue ou se estende em linha recta: \textunderscore camínho direito\textunderscore .
Plano.
Aprumado.
Recto, íntegro.
Directamente.
Acertadamente.
\section{Direito}
\begin{itemize}
\item {Grp. gram.:m.}
\end{itemize}
\begin{itemize}
\item {Proveniência:(Do lat. \textunderscore directum\textunderscore )}
\end{itemize}
Aquillo que é recto, que é justo.
Aquillo que se conforma com a lei ou com a justiça: \textunderscore julgar contra direito\textunderscore .
Faculdade legal de praticar ou não praticar um acto: \textunderscore não há o direito de calumniar\textunderscore .
Conjunto de leis ou de regras à cêrca das relações sociaes: \textunderscore o direito romano\textunderscore .
Regalía.
Tributo, imposto: \textunderscore os direitos de alfândega\textunderscore .
\section{Direitura}
\begin{itemize}
\item {Grp. gram.:f.}
\end{itemize}
\begin{itemize}
\item {Utilização:Des.}
\end{itemize}
Qualidade daquillo que é direito, do que é recto.
Direcção em linha recta.
Rectidão.
\section{Diriano}
\begin{itemize}
\item {Grp. gram.:m.}
\end{itemize}
Língua uralo-altaica do grupo ugro-filandês.
\section{Diribitor}
\begin{itemize}
\item {Grp. gram.:m.}
\end{itemize}
\begin{itemize}
\item {Proveniência:(Lat. \textunderscore diribitor\textunderscore )}
\end{itemize}
Aquelle que, entre os Romanos, distribuía pelo povo as listas eleitoraes.
\section{Diribitório}
\begin{itemize}
\item {Grp. gram.:m.}
\end{itemize}
\begin{itemize}
\item {Proveniência:(Lat. \textunderscore diribitorium\textunderscore )}
\end{itemize}
Lugar ou edifício, onde os diribitores distribuíam as listas eleitoraes.
Lugar, onde se distribuíam donativos pelo povo. Cf. Castilho, \textunderscore Fastos\textunderscore , III, 231.
\section{Dirigente}
\begin{itemize}
\item {Grp. gram.:adj.}
\end{itemize}
\begin{itemize}
\item {Grp. gram.:M.}
\end{itemize}
\begin{itemize}
\item {Proveniência:(Lat. \textunderscore dirigens\textunderscore )}
\end{itemize}
Que dirige.
Aquelle que dirige.
Director.
\section{Diriginte}
\begin{itemize}
\item {Grp. gram.:adj.}
\end{itemize}
\begin{itemize}
\item {Utilização:Des.}
\end{itemize}
\begin{itemize}
\item {Proveniência:(De \textunderscore dirigir\textunderscore )}
\end{itemize}
Que dirige.
Dirigente. Cf. Garrett, \textunderscore Catão\textunderscore , 237.
\section{Dirigir}
\begin{itemize}
\item {Grp. gram.:v. t.}
\end{itemize}
\begin{itemize}
\item {Proveniência:(Lat. \textunderscore dirigere\textunderscore )}
\end{itemize}
Pôr em linha recta.
Dar direcção a.
Administrar: \textunderscore dirigir um Banco\textunderscore .
Guiar: \textunderscore dirigir um carro\textunderscore .
Governar.
Superintender em.
Enviar: \textunderscore dirigir uma carta\textunderscore .
Consagrar, dedicar.
Volver.
\section{Dirigível}
\begin{itemize}
\item {Grp. gram.:adj.}
\end{itemize}
\begin{itemize}
\item {Grp. gram.:M.}
\end{itemize}
Que se póde dirigir.
Balão, que se póde dirigir.
\section{Dirimente}
\begin{itemize}
\item {Grp. gram.:adj.}
\end{itemize}
\begin{itemize}
\item {Proveniência:(Lat. \textunderscore dirimens\textunderscore )}
\end{itemize}
Que dirime.
Que annulla sem remédio.
Que obsta fundamentalmente: \textunderscore impedimentos dirimentes do matrimónio\textunderscore .
\section{Dirimir}
\begin{itemize}
\item {Grp. gram.:v. t.}
\end{itemize}
\begin{itemize}
\item {Proveniência:(Lat. \textunderscore dirimere\textunderscore )}
\end{itemize}
Tornar nullo.
Impedir absolutamente.
Extinguir, decidir: \textunderscore dirimir questões\textunderscore .
\section{Diro}
\begin{itemize}
\item {Grp. gram.:adj.}
\end{itemize}
\begin{itemize}
\item {Utilização:Des.}
\end{itemize}
\begin{itemize}
\item {Proveniência:(Lat. \textunderscore dirus\textunderscore )}
\end{itemize}
Duro; cruel.
\section{Dirradicar}
\begin{itemize}
\item {Proveniência:(Do lat. \textunderscore de\textunderscore  + \textunderscore radix\textunderscore )}
\end{itemize}
\textunderscore v. t.\textunderscore  (e der.)
O mesmo que \textunderscore desarraigar\textunderscore , etc.--Us. por Latino, \textunderscore Hist. Pol. e Mil.\textunderscore , 115; mas melhor seria \textunderscore deradicar\textunderscore .
\section{Diruir}
\begin{itemize}
\item {Grp. gram.:v. t.}
\end{itemize}
(V.derruir)
\section{Dirupção}
\begin{itemize}
\item {Grp. gram.:f.}
\end{itemize}
\begin{itemize}
\item {Utilização:Des.}
\end{itemize}
\begin{itemize}
\item {Proveniência:(Lat. \textunderscore diruptio\textunderscore )}
\end{itemize}
Ruina.
\section{Diruptivo}
\begin{itemize}
\item {Grp. gram.:adj.}
\end{itemize}
\begin{itemize}
\item {Utilização:Des.}
\end{itemize}
\begin{itemize}
\item {Proveniência:(Do lat. \textunderscore diruptus\textunderscore )}
\end{itemize}
Que arruína, que rasga.
\section{Dis...}
\begin{itemize}
\item {Proveniência:(Do lat. \textunderscore dis\textunderscore )}
\end{itemize}
\textunderscore pref.\textunderscore  (indicativo de separação, negação, deminuição, aumento)
Antes de consoante que não seja \textunderscore s\textunderscore , diz-se geralmente \textunderscore di\textunderscore : \textunderscore dimanar\textunderscore , \textunderscore difundir\textunderscore , etc.
\section{Disaco}
\begin{itemize}
\item {Grp. gram.:m.}
\end{itemize}
Árvore sapotácea de Angola, de frutos comestíveis, semelhantes ás cerejas.
\section{Disbarate}
\begin{itemize}
\item {Grp. gram.:f.}
\end{itemize}
\begin{itemize}
\item {Utilização:Des.}
\end{itemize}
O mesmo que \textunderscore disparate\textunderscore .
\section{Discante}
\begin{itemize}
\item {Grp. gram.:m.}
\end{itemize}
O mesmo que \textunderscore descante\textunderscore .
\section{Discente}
\begin{itemize}
\item {Grp. gram.:adj.}
\end{itemize}
\begin{itemize}
\item {Proveniência:(Lat. \textunderscore discens\textunderscore )}
\end{itemize}
Que apprende.
Relativo a alumnos: \textunderscore a população discente do lyceu\textunderscore .
\section{Disceptação}
\begin{itemize}
\item {Grp. gram.:f.}
\end{itemize}
\begin{itemize}
\item {Utilização:Des.}
\end{itemize}
\begin{itemize}
\item {Proveniência:(Lat. \textunderscore disceptatio\textunderscore )}
\end{itemize}
Discussão.
\section{Discernente}
\begin{itemize}
\item {Grp. gram.:adj.}
\end{itemize}
\begin{itemize}
\item {Proveniência:(Lat. \textunderscore discernens\textunderscore )}
\end{itemize}
Que discerne.
\section{Discernículo}
\begin{itemize}
\item {Grp. gram.:m.}
\end{itemize}
\begin{itemize}
\item {Proveniência:(Lat. \textunderscore discerniculum\textunderscore )}
\end{itemize}
Agulha, com que as romanas apartavam o cabello na cabeça.
\section{Discernimento}
\begin{itemize}
\item {Grp. gram.:m.}
\end{itemize}
Acto de discernir.
Prudência.
Juízo.
Escolha; distincção.
Apreciação.
\section{Discernir}
\begin{itemize}
\item {Grp. gram.:v. t.}
\end{itemize}
\begin{itemize}
\item {Proveniência:(Lat. \textunderscore discernere\textunderscore )}
\end{itemize}
Separar.
Distinguir.
Vêr claramente, discriminar: \textunderscore discernir o falso do verdadeiro\textunderscore .
Apreciar.
Medir; avaliar bem.
\section{Discernível}
\begin{itemize}
\item {Grp. gram.:adj.}
\end{itemize}
Que se póde discernir.
\section{Disciforme}
\begin{itemize}
\item {Grp. gram.:adj.}
\end{itemize}
\begin{itemize}
\item {Utilização:Bot.}
\end{itemize}
\begin{itemize}
\item {Proveniência:(De \textunderscore disco\textunderscore  + \textunderscore fórma\textunderscore )}
\end{itemize}
Que tem fórma de disco.
Diz-se das fôlhas vegetaes de fórma quási circular.
\section{Discina}
\begin{itemize}
\item {Grp. gram.:f.}
\end{itemize}
Gênero de conchas bivalves.
\section{Disciplina}
\begin{itemize}
\item {Grp. gram.:f.}
\end{itemize}
\begin{itemize}
\item {Grp. gram.:Pl.}
\end{itemize}
\begin{itemize}
\item {Proveniência:(Lat. \textunderscore disciplina\textunderscore )}
\end{itemize}
Instrucção e educação.
Ensino.
Relações de submissão, de quem é ensinado para com aquelle que ensina.
Observância de preceitos ou ordens: \textunderscore nesta escola há disciplina\textunderscore .
Procedimento correcto.
Doutrina.
Conjunto de conhecimentos, que se professam em cada cadeira de um instituto escolar: \textunderscore frequentar duas disciplinas\textunderscore .
Autoridade.
Obediência á autoridade.
Correias, para açoitar por penitência ou por castigo.
\section{Disciplinadamente}
\begin{itemize}
\item {Grp. gram.:adv.}
\end{itemize}
\begin{itemize}
\item {Proveniência:(De \textunderscore disciplinar\textunderscore )}
\end{itemize}
Com disciplina.
\section{Disciplina-de-freira}
\begin{itemize}
\item {Grp. gram.:f.}
\end{itemize}
\begin{itemize}
\item {Utilização:Bras}
\end{itemize}
O mesmo que \textunderscore amaranto\textunderscore .
\section{Disciplinador}
\begin{itemize}
\item {Grp. gram.:adj.}
\end{itemize}
\begin{itemize}
\item {Proveniência:(De \textunderscore disciplinar\textunderscore ^1)}
\end{itemize}
Que disciplina.
Que faz observar a disciplina: \textunderscore um general disciplinador\textunderscore .
Amigo da disciplina.
\section{Disciplinamento}
\begin{itemize}
\item {Grp. gram.:m.}
\end{itemize}
Acto de disciplinar.
\section{Disciplinante}
\begin{itemize}
\item {Grp. gram.:adj.}
\end{itemize}
\begin{itemize}
\item {Grp. gram.:M.}
\end{itemize}
\begin{itemize}
\item {Proveniência:(De \textunderscore disciplinar\textunderscore )}
\end{itemize}
Que disciplina.
Penitente, que se disciplina.
\section{Disciplinar}
\begin{itemize}
\item {Grp. gram.:v. t.}
\end{itemize}
\begin{itemize}
\item {Proveniência:(Do lat. \textunderscore disciplinari\textunderscore )}
\end{itemize}
Sujeitar á disciplina.
Corrigir.
Instruir methodicamente.
Açoitar com disciplinas.
\section{Disciplinar}
\begin{itemize}
\item {Grp. gram.:adj.}
\end{itemize}
\begin{itemize}
\item {Proveniência:(Lat. \textunderscore disciplinaris\textunderscore )}
\end{itemize}
Relativo a disciplina: \textunderscore havia um Conselho disciplinar\textunderscore .
\section{Disciplinarmente}
\begin{itemize}
\item {Grp. gram.:adv.}
\end{itemize}
De modo disciplinar.
\section{Disciplinável}
\begin{itemize}
\item {Grp. gram.:adj.}
\end{itemize}
\begin{itemize}
\item {Proveniência:(Lat. \textunderscore disciplinabilis\textunderscore )}
\end{itemize}
Que póde sêr disciplinado.
\section{Discipulado}
\begin{itemize}
\item {Grp. gram.:m.}
\end{itemize}
\begin{itemize}
\item {Proveniência:(Lat. \textunderscore discipulatus\textunderscore )}
\end{itemize}
Conjunto dos alumnos de uma escola.
Aprendizado.
Estado de quem é discípulo.
\section{Discípulo}
\begin{itemize}
\item {Grp. gram.:m.}
\end{itemize}
\begin{itemize}
\item {Grp. gram.:Adj.}
\end{itemize}
\begin{itemize}
\item {Utilização:Mús.}
\end{itemize}
\begin{itemize}
\item {Proveniência:(Lat. \textunderscore discipulus\textunderscore )}
\end{itemize}
Aquelle que recebe ensino de alguém.
Aquelle que apprende.
Alumno.
Aquelle que segue a doutrina de outrem.
Diz-se dos quatro tons plagaes, no cantochão.
\section{Discissão}
\begin{itemize}
\item {Grp. gram.:f.}
\end{itemize}
\begin{itemize}
\item {Proveniência:(Lat. \textunderscore discissio\textunderscore )}
\end{itemize}
Operação, com que se fende a cápsula do crystallino, e que é um dos processos para a extirpação da cataracta.
\section{Disco}
\begin{itemize}
\item {Grp. gram.:m.}
\end{itemize}
\begin{itemize}
\item {Proveniência:(Gr. \textunderscore diskos\textunderscore )}
\end{itemize}
Chapa redonda e pesada, para arremêsso, na gymnástica antiga.
Qualquer peça circular e chata.
\section{Discóbolo}
\begin{itemize}
\item {Grp. gram.:m.}
\end{itemize}
\begin{itemize}
\item {Proveniência:(Gr. \textunderscore discobolos\textunderscore )}
\end{itemize}
Gymnasta, que arremessava o disco, nos jogos gregos.
\section{Discoide}
\begin{itemize}
\item {Grp. gram.:adj.}
\end{itemize}
\begin{itemize}
\item {Grp. gram.:F. pl.}
\end{itemize}
\begin{itemize}
\item {Utilização:Bot.}
\end{itemize}
\begin{itemize}
\item {Proveniência:(Do gr. \textunderscore diskos\textunderscore  + \textunderscore eidos\textunderscore )}
\end{itemize}
Que tem fórma de disco.
Disciforme.
Secção de plantas synanthéreas, segundo Goertner.
\section{Discômmodo}
\begin{itemize}
\item {Grp. gram.:adj.}
\end{itemize}
\begin{itemize}
\item {Utilização:Ant.}
\end{itemize}
O mesmo que \textunderscore incômmodo\textunderscore ^1.
\section{Discontínuo}
\begin{itemize}
\item {Grp. gram.:adj.}
\end{itemize}
(V.descontínuo)
\section{Discordância}
\begin{itemize}
\item {Grp. gram.:f.}
\end{itemize}
\begin{itemize}
\item {Proveniência:(De \textunderscore discordar\textunderscore )}
\end{itemize}
Estado daquelle ou daquíllo que discorda.
Incompatibilidade.
Disparidade.
Desharmonia.
Desafinação.
\section{Discordante}
\begin{itemize}
\item {Grp. gram.:adj.}
\end{itemize}
\begin{itemize}
\item {Proveniência:(Lat. \textunderscore discordans\textunderscore )}
\end{itemize}
Que discorda.
\section{Discordantemente}
\begin{itemize}
\item {Grp. gram.:adv.}
\end{itemize}
\begin{itemize}
\item {Proveniência:(De \textunderscore discordante\textunderscore )}
\end{itemize}
De modo discorde, sem harmonia.
\section{Discordar}
\begin{itemize}
\item {Grp. gram.:v. i.}
\end{itemize}
\begin{itemize}
\item {Proveniência:(Lat. \textunderscore discordare\textunderscore )}
\end{itemize}
Não concordar.
Estar em desharmonia.
Divergir.
Desafinar.
Sêr incompatível.
\section{Discorde}
\begin{itemize}
\item {Grp. gram.:adj.}
\end{itemize}
\begin{itemize}
\item {Proveniência:(Lat. \textunderscore discors\textunderscore )}
\end{itemize}
Que discorda.
Destoante.
Desproporcionado.
Incompatível.
Discordante.
\section{Discordemente}
\begin{itemize}
\item {Grp. gram.:adv.}
\end{itemize}
O mesmo que \textunderscore discordantemente\textunderscore .
\section{Discórdia}
\begin{itemize}
\item {Grp. gram.:f.}
\end{itemize}
\begin{itemize}
\item {Proveniência:(Lat. \textunderscore discordia\textunderscore )}
\end{itemize}
Discordância.
Luta.
Desintelligência.
Desordem.
\section{Discorrer}
\begin{itemize}
\item {Grp. gram.:v. i.}
\end{itemize}
\begin{itemize}
\item {Utilização:Fig.}
\end{itemize}
\begin{itemize}
\item {Grp. gram.:V. t.}
\end{itemize}
\begin{itemize}
\item {Proveniência:(Lat. \textunderscore discurrere\textunderscore )}
\end{itemize}
Correr para differentes lados.
Diffundir-se.
Correr em determinada direcção.
Fazer pequena viagem, passear: \textunderscore discorrer pela província\textunderscore .
Falar; discursar.
Pensar.
Decorrer.
Percorrer.
Meditar em.
Observar; examinar.
\section{Discorrimento}
\begin{itemize}
\item {Grp. gram.:m.}
\end{itemize}
\begin{itemize}
\item {Utilização:Des.}
\end{itemize}
Acto de discorrer.
\section{Discreção}
\textunderscore f.\textunderscore  (e der.)
(V. \textunderscore discrição\textunderscore , etc.)
\section{Discrepancia}
\begin{itemize}
\item {Grp. gram.:f.}
\end{itemize}
\begin{itemize}
\item {Proveniência:(Lat. \textunderscore discrepantia\textunderscore )}
\end{itemize}
Estado ou qualidade daquelle ou daquillo que discrepa.
Disparidade.
Divergência.
\section{Discrepante}
\begin{itemize}
\item {Grp. gram.:adj.}
\end{itemize}
\begin{itemize}
\item {Proveniência:(Lat. \textunderscore discrepans\textunderscore )}
\end{itemize}
Que discrepa.
Divergente; diverso.
\section{Discrepar}
\begin{itemize}
\item {Grp. gram.:v. i.}
\end{itemize}
\begin{itemize}
\item {Proveniência:(Lat. \textunderscore discrepare\textunderscore )}
\end{itemize}
Discordar.
Sêr diverso.
\section{Discretamente}
\begin{itemize}
\item {Grp. gram.:adv.}
\end{itemize}
De modo discreto.
Com discrição.
\section{Discreteador}
\begin{itemize}
\item {Grp. gram.:m.}
\end{itemize}
Aquelle que discreteia.
\section{Discretear}
\begin{itemize}
\item {Grp. gram.:v. i.}
\end{itemize}
\begin{itemize}
\item {Proveniência:(De \textunderscore discreto\textunderscore )}
\end{itemize}
Discorrer ou falar com discrição, placidamente.
\section{Discretissimamente}
\begin{itemize}
\item {Grp. gram.:adv.}
\end{itemize}
De modo muito discreto, com grande discrição.
\section{Discretivo}
\begin{itemize}
\item {Grp. gram.:adj.}
\end{itemize}
\begin{itemize}
\item {Proveniência:(Lat. \textunderscore discretivus\textunderscore )}
\end{itemize}
Próprio para discernir.
\section{Discreto}
\begin{itemize}
\item {Grp. gram.:adj.}
\end{itemize}
\begin{itemize}
\item {Proveniência:(Lat. \textunderscore discretus\textunderscore )}
\end{itemize}
Que tem discrição: \textunderscore pessôa discreta\textunderscore .
Em que ha discrição: \textunderscore acções discretas\textunderscore .
Que exprime objectos distintos.
Que se manifesta por sinaes separados.
\section{Discrição}
\begin{itemize}
\item {Grp. gram.:f.}
\end{itemize}
\begin{itemize}
\item {Grp. gram.:Loc. adv.}
\end{itemize}
\begin{itemize}
\item {Proveniência:(Do lat. \textunderscore discretio\textunderscore .--Como \textunderscore processio\textunderscore  produziu \textunderscore procissão\textunderscore , \textunderscore professio\textunderscore  profissão, \textunderscore confessio\textunderscore  confissão, etc., \textunderscore discretio\textunderscore  produziu \textunderscore discrição\textunderscore , que é como se diz, e não \textunderscore discreção\textunderscore , como muitos escrevem pretensiosamente)}
\end{itemize}
Qualidade daquelle ou daquillo que é discreto.
Acto de discernir.
Reserva; qualidade de quem sabe guardar segrêdo.
Modéstia.
\textunderscore Á discrição\textunderscore , á vontade, sem restricções.
\section{Discricionário}
\begin{itemize}
\item {Grp. gram.:adj.}
\end{itemize}
\begin{itemize}
\item {Proveniência:(De \textunderscore discrição\textunderscore )}
\end{itemize}
Que não tem condições; arbitrário, caprichoso: \textunderscore procedimento discricionário\textunderscore .
\section{Discrime}
\begin{itemize}
\item {Grp. gram.:m.}
\end{itemize}
\begin{itemize}
\item {Utilização:Des.}
\end{itemize}
\begin{itemize}
\item {Proveniência:(Lat. \textunderscore discrimen\textunderscore )}
\end{itemize}
Acto de discernir.
Differença.
Combate; luta.
\section{Discrimen}
\begin{itemize}
\item {Grp. gram.:m.}
\end{itemize}
(V.discrime)
\section{Discriminação}
\begin{itemize}
\item {Grp. gram.:f.}
\end{itemize}
\begin{itemize}
\item {Proveniência:(Lat. \textunderscore discriminatio\textunderscore )}
\end{itemize}
Acto de discriminar.
\section{Discriminador}
\begin{itemize}
\item {Grp. gram.:m.}
\end{itemize}
\begin{itemize}
\item {Proveniência:(Lat. \textunderscore discriminator\textunderscore )}
\end{itemize}
Aquelle que discrimina.
\section{Discriminal}
\begin{itemize}
\item {Grp. gram.:adj.}
\end{itemize}
\begin{itemize}
\item {Utilização:Des.}
\end{itemize}
Que serve para discriminar.
(B. lat. \textunderscore discriminalis\textunderscore )
\section{Discriminar}
\begin{itemize}
\item {Grp. gram.:v. t.}
\end{itemize}
\begin{itemize}
\item {Proveniência:(Lat. \textunderscore discriminare\textunderscore )}
\end{itemize}
Discernir: \textunderscore discriminar as razões de uma theoria\textunderscore .
Separar; differençar: \textunderscore discriminar o bem e o mal\textunderscore .
\section{Discriminável}
\begin{itemize}
\item {Grp. gram.:adj.}
\end{itemize}
Que se póde discriminar.
\section{Discursador}
\begin{itemize}
\item {Grp. gram.:m.}
\end{itemize}
Aquelle que discursa.
\section{Discursar}
\begin{itemize}
\item {Grp. gram.:v. t.}
\end{itemize}
\begin{itemize}
\item {Grp. gram.:V. i.}
\end{itemize}
\begin{itemize}
\item {Proveniência:(Lat. \textunderscore discursare\textunderscore )}
\end{itemize}
Pronunciar, expor methodicamente.
Fazer discurso.
Discorrer.
Perorar.
Dar largas explicações.
\section{Discursista}
\begin{itemize}
\item {Grp. gram.:m.  e  f.}
\end{itemize}
\begin{itemize}
\item {Proveniência:(De \textunderscore discursar\textunderscore )}
\end{itemize}
Pessôa que discursa. Cf. Latino, \textunderscore Elogios Acad.\textunderscore , 361.
\section{Discursivo}
\begin{itemize}
\item {Grp. gram.:adj.}
\end{itemize}
\begin{itemize}
\item {Proveniência:(De \textunderscore discurso\textunderscore )}
\end{itemize}
Que procede por meio de raciocínio.
Deductivo.
Que costuma discursar ou gosta de discursar; palrador:«\textunderscore o administrador era discursivo e não perdia lanço de eloquência\textunderscore ». Camillo, \textunderscore Brasileira\textunderscore , 57.
\section{Discurso}
\begin{itemize}
\item {Grp. gram.:m.}
\end{itemize}
\begin{itemize}
\item {Utilização:Ant.}
\end{itemize}
\begin{itemize}
\item {Utilização:Fam.}
\end{itemize}
\begin{itemize}
\item {Proveniência:(Lat. \textunderscore discursus\textunderscore )}
\end{itemize}
Conjunto ordenado ou methódico de phrases, pronunciadas em público, ou escritas como se tivessem de sêr proferidas ou lidas em público.
Arrazoado; oração.
Raciocínio.
Decurso: \textunderscore tudo que vai succedendo no discurso do tempo\textunderscore .
Palavreado oco.
\section{Discussão}
\begin{itemize}
\item {Grp. gram.:f.}
\end{itemize}
\begin{itemize}
\item {Proveniência:(Lat. \textunderscore discussio\textunderscore )}
\end{itemize}
Acto de discutir.
Contenda; disputa.
\section{Discutidor}
\begin{itemize}
\item {Grp. gram.:m.}
\end{itemize}
Aquelle que discute.
\section{Discutinhar}
\begin{itemize}
\item {Grp. gram.:v. i.}
\end{itemize}
\begin{itemize}
\item {Utilização:Fam.}
\end{itemize}
\begin{itemize}
\item {Proveniência:(De \textunderscore discutir\textunderscore )}
\end{itemize}
Discutir mal e enfadonhamente.
\section{Discutir}
\begin{itemize}
\item {Grp. gram.:v. t.}
\end{itemize}
\begin{itemize}
\item {Grp. gram.:V. i.}
\end{itemize}
\begin{itemize}
\item {Proveniência:(Lat. \textunderscore discutere\textunderscore )}
\end{itemize}
Examinar, questionando.
Questionar.
Defender ou atacar (assumpto controvertido).
Excutir.
Fazer questão.
Questionar.
\section{Discutível}
\begin{itemize}
\item {Grp. gram.:adj.}
\end{itemize}
Que se póde discutir.
\section{Disema}
\begin{itemize}
\item {Grp. gram.:m.}
\end{itemize}
Gênero de plantas passiflóreas.
\section{Disemia}
\begin{itemize}
\item {Grp. gram.:f.}
\end{itemize}
\begin{itemize}
\item {Utilização:Med.}
\end{itemize}
\begin{itemize}
\item {Proveniência:(Do gr. \textunderscore dis\textunderscore  + \textunderscore haima\textunderscore )}
\end{itemize}
Alteração do sangue.
\section{Disépalo}
\begin{itemize}
\item {fónica:sé}
\end{itemize}
\begin{itemize}
\item {Grp. gram.:adj.}
\end{itemize}
\begin{itemize}
\item {Utilização:Bot.}
\end{itemize}
\begin{itemize}
\item {Proveniência:(De \textunderscore di...\textunderscore  + \textunderscore sépala\textunderscore )}
\end{itemize}
Que tem duas sépalas distintas.
\section{Disertamente}
\begin{itemize}
\item {Grp. gram.:adv.}
\end{itemize}
De modo diserto.
\section{Diserto}
\begin{itemize}
\item {Grp. gram.:adj.}
\end{itemize}
\begin{itemize}
\item {Proveniência:(Lat. \textunderscore disertus\textunderscore )}
\end{itemize}
Facundo.
Que se exprime com simplicidade e elegância.
\section{Disfarçadamente}
\begin{itemize}
\item {Grp. gram.:adv.}
\end{itemize}
De modo disfarçado.
\section{Disfarçado}
\begin{itemize}
\item {Grp. gram.:adj.}
\end{itemize}
Que se disfarçou.
Fingido.
Falso.
Mascarado.
\section{Disfarçar}
\begin{itemize}
\item {Grp. gram.:v. t.}
\end{itemize}
\begin{itemize}
\item {Proveniência:(De \textunderscore dis...\textunderscore  + \textunderscore farça\textunderscore )}
\end{itemize}
Mascarar.
Encobrir, vestir, de modo que se não veja ou se não conheça.
Simular.
Dissimular.
\section{Disfarce}
\begin{itemize}
\item {Grp. gram.:m.}
\end{itemize}
Acto de disfarçar.
Aquillo que serve para disfarçar.
\section{Disfarçuda}
\begin{itemize}
\item {Grp. gram.:f.}
\end{itemize}
\begin{itemize}
\item {Utilização:Prov.}
\end{itemize}
\begin{itemize}
\item {Utilização:alent.}
\end{itemize}
\begin{itemize}
\item {Proveniência:(De \textunderscore disfarçar\textunderscore )}
\end{itemize}
O mesmo que \textunderscore mascarada\textunderscore .
\section{Disferir}
\begin{itemize}
\item {Grp. gram.:v. t.}
\end{itemize}
\begin{itemize}
\item {Proveniência:(Do lat. \textunderscore disferre\textunderscore )}
\end{itemize}
Dilatar; engrandecer. Cf. Filinto. XVII, 230; XIX, 27.
\section{Disformar}
\begin{itemize}
\item {Grp. gram.:v. t.}
\end{itemize}
(V.deformar)
\section{Disforme}
\begin{itemize}
\item {Grp. gram.:adj.}
\end{itemize}
\begin{itemize}
\item {Proveniência:(De \textunderscore dis...\textunderscore  + \textunderscore fórma\textunderscore )}
\end{itemize}
Extraordinário.
Muito grande.
Monstruoso.
\section{Disfrutar}
\textunderscore v. t.\textunderscore  (e der.)
O mesmo ou melhor que \textunderscore desfrutar\textunderscore , etc.
\section{Disgenesia}
\begin{itemize}
\item {Grp. gram.:f.}
\end{itemize}
\begin{itemize}
\item {Grp. gram.:f.}
\end{itemize}
\begin{itemize}
\item {Utilização:Med.}
\end{itemize}
\begin{itemize}
\item {Proveniência:(Do gr. \textunderscore dus\textunderscore  + \textunderscore genesis\textunderscore )}
\end{itemize}
Qualidade do que é disgenesico.
Perturbação da funcção reproductora.
Cruzamentos, cujos productos são estéreis entre si, mas fecundos com indivíduos de outra raça mãe.
\section{Disgenésico}
\begin{itemize}
\item {Grp. gram.:adj.}
\end{itemize}
\begin{itemize}
\item {Proveniência:(Do gr. \textunderscore dis\textunderscore  + \textunderscore genos\textunderscore )}
\end{itemize}
Diz-se dos indivíduos, que são infecundos entre si, mas fecundos com indivíduos de outras raças.
Relativo á disgenesia.
Que torna diffícil a reproducção.
\section{Disgra}
\begin{itemize}
\item {Grp. gram.:f.}
\end{itemize}
\begin{itemize}
\item {Utilização:Bras}
\end{itemize}
Desgraça.
(Do port.)
\section{Disgraça}
\begin{itemize}
\item {Grp. gram.:f.}
\end{itemize}
\begin{itemize}
\item {Utilização:Des.}
\end{itemize}
O mesmo que \textunderscore desgraça\textunderscore . Cf. \textunderscore Luz e Calor\textunderscore , (\textunderscore passim\textunderscore ).
\section{Disgregar}
\begin{itemize}
\item {Proveniência:(Lat. \textunderscore disgregare\textunderscore )}
\end{itemize}
\textunderscore v. t.\textunderscore  (e der.)
O mesmo que desaggregar, etc.
\section{Disgregativo}
\begin{itemize}
\item {Grp. gram.:adj.}
\end{itemize}
\begin{itemize}
\item {Proveniência:(De \textunderscore disgregar\textunderscore )}
\end{itemize}
Que disgrega. Cf. Vieira, X, 165.
\section{Disjunção}
\begin{itemize}
\item {Grp. gram.:f.}
\end{itemize}
\begin{itemize}
\item {Utilização:Gram.}
\end{itemize}
\begin{itemize}
\item {Proveniência:(Lat. \textunderscore disjunctio\textunderscore )}
\end{itemize}
Separação.
Supressão de conjunção copulativa, entre várias frases.
\section{Disjuncção}
\begin{itemize}
\item {Grp. gram.:f.}
\end{itemize}
\begin{itemize}
\item {Utilização:Gram.}
\end{itemize}
\begin{itemize}
\item {Proveniência:(Lat. \textunderscore disjunctio\textunderscore )}
\end{itemize}
Separação.
Suppressão de conjuncção copulativa, entre várias phrases.
\section{Disjungir}
\begin{itemize}
\item {Grp. gram.:v. t.}
\end{itemize}
\begin{itemize}
\item {Proveniência:(Lat. \textunderscore disjungere\textunderscore )}
\end{itemize}
Tirar do jugo.
Separar.
\section{Disjunta}
\begin{itemize}
\item {Grp. gram.:f.}
\end{itemize}
\begin{itemize}
\item {Utilização:Des.}
\end{itemize}
\begin{itemize}
\item {Proveniência:(De \textunderscore disjunto\textunderscore )}
\end{itemize}
Passagem de um tom para outro, em música.
\section{Disjuntivamente}
\begin{itemize}
\item {Grp. gram.:adv.}
\end{itemize}
De modo disjuntivo.
\section{Disjuntivo}
\begin{itemize}
\item {Grp. gram.:adj.}
\end{itemize}
\begin{itemize}
\item {Utilização:Gram.}
\end{itemize}
\begin{itemize}
\item {Proveniência:(Lat. \textunderscore disjunctivus\textunderscore )}
\end{itemize}
Próprio para desunir.
Diz-se do movimento musical, que se chamava \textunderscore disjunta\textunderscore .
Diz-se da proposição, em que há dois predicativos, um dos quaes tem de convir ao sujeito, com exclusão do outro.
E diz-se da palavra que estabelece distincção ou alternativa, entre os membros de uma phrase.
\section{Disjunto}
\begin{itemize}
\item {Grp. gram.:adj.}
\end{itemize}
\begin{itemize}
\item {Proveniência:(Lat. \textunderscore disjunctus\textunderscore )}
\end{itemize}
Diz-se dos graus musicaes que não são conjuntos.
\section{Dislate}
\begin{itemize}
\item {Grp. gram.:m.}
\end{itemize}
Disparate, despautério.
(Cast. \textunderscore dislate\textunderscore )
\section{Disna}
\begin{itemize}
\item {Grp. gram.:f.}
\end{itemize}
Casa africana, circular e de tecto cónico. Cf. Capello e Ivens, I, 184.
\section{Dispamparar}
\begin{itemize}
\item {Grp. gram.:v. t.  e  i.}
\end{itemize}
\begin{itemize}
\item {Utilização:Bras}
\end{itemize}
Disparar.
\section{Díspar}
\begin{itemize}
\item {Grp. gram.:adj.}
\end{itemize}
\begin{itemize}
\item {Proveniência:(Lat. \textunderscore dispar\textunderscore )}
\end{itemize}
Desigual.
Differente.
\section{Disparada}
\begin{itemize}
\item {Grp. gram.:f.}
\end{itemize}
\begin{itemize}
\item {Utilização:Bras}
\end{itemize}
\begin{itemize}
\item {Proveniência:(De \textunderscore disparar\textunderscore )}
\end{itemize}
Acto de tresmalhar-se o gado.
\section{Disparador}
\begin{itemize}
\item {Grp. gram.:m.}
\end{itemize}
\begin{itemize}
\item {Utilização:Des.}
\end{itemize}
\begin{itemize}
\item {Grp. gram.:M.  e  adj.}
\end{itemize}
\begin{itemize}
\item {Utilização:Bras}
\end{itemize}
\begin{itemize}
\item {Proveniência:(De \textunderscore disparar\textunderscore )}
\end{itemize}
Gatilho.
Diz-se do animal que foge, quando o querem prender.
\section{Disparar}
\begin{itemize}
\item {Grp. gram.:v. t.}
\end{itemize}
\begin{itemize}
\item {Utilização:Fig.}
\end{itemize}
\begin{itemize}
\item {Grp. gram.:V. i.}
\end{itemize}
\begin{itemize}
\item {Utilização:Bras}
\end{itemize}
\begin{itemize}
\item {Proveniência:(Lat. \textunderscore disparare\textunderscore )}
\end{itemize}
Arrojar.
Arremessar.
Soltar.
Fazer fogo com: \textunderscore disparar a pistola\textunderscore .
Soltar (invectivas, injúrias, etc.).
Dispersar-se (o gado).
Tresmalhar-se.
Partir apressadamente:«\textunderscore entrão os inimigos em tal pavor, que dispárão em declarada fugida...\textunderscore »Filinto, \textunderscore D. Man.\textunderscore , II, 33.
\section{Disparatadamente}
\begin{itemize}
\item {Grp. gram.:adv.}
\end{itemize}
De modo disparatado.
\section{Disparatado}
\begin{itemize}
\item {Grp. gram.:adj.}
\end{itemize}
\begin{itemize}
\item {Proveniência:(De \textunderscore disparatar\textunderscore )}
\end{itemize}
Que diz ou faz disparates.
Que revela disparate.
Em que há disparate.
\section{Disparatar}
\begin{itemize}
\item {Grp. gram.:v. i.}
\end{itemize}
\begin{itemize}
\item {Utilização:Des.}
\end{itemize}
\begin{itemize}
\item {Proveniência:(De \textunderscore disparate\textunderscore )}
\end{itemize}
Desvairar.
Despropositar.
Fazer ou dizer disparates.
Discordar.
\section{Disparate}
\begin{itemize}
\item {Grp. gram.:m.}
\end{itemize}
\begin{itemize}
\item {Utilização:Bras. de Minas}
\end{itemize}
\begin{itemize}
\item {Proveniência:(T. cast.)}
\end{itemize}
Falta de propósito.
Sem-razão; tolice.
Desvario.
Absurdo.
Grande quantidade: \textunderscore um disparate de dinheiro\textunderscore .
\section{Disparateiro}
\begin{itemize}
\item {Grp. gram.:adj.}
\end{itemize}
\begin{itemize}
\item {Utilização:Prov.}
\end{itemize}
\begin{itemize}
\item {Utilização:minh.}
\end{itemize}
Que diz disparates para agradar; engraçado; divertido.
\section{Disparecer}
\begin{itemize}
\item {Grp. gram.:v. i.}
\end{itemize}
(V.desparecer)
\section{Disparidade}
\begin{itemize}
\item {Grp. gram.:f.}
\end{itemize}
Qualidade do que é dispar.
Desigualdade.
\section{Disparisonante}
\begin{itemize}
\item {fónica:so}
\end{itemize}
\begin{itemize}
\item {Grp. gram.:adj.}
\end{itemize}
\begin{itemize}
\item {Utilização:Gram.}
\end{itemize}
\begin{itemize}
\item {Proveniência:(Do lat. \textunderscore díspar\textunderscore  + \textunderscore sonans\textunderscore )}
\end{itemize}
Diz-se da conjugação irregular, em algumas línguas, como o alemão.
\section{Disparissonante}
\begin{itemize}
\item {Grp. gram.:adj.}
\end{itemize}
\begin{itemize}
\item {Utilização:Gram.}
\end{itemize}
\begin{itemize}
\item {Proveniência:(Do lat. \textunderscore díspar\textunderscore  + \textunderscore sonans\textunderscore )}
\end{itemize}
Diz-se da conjugação irregular, em algumas línguas, como o alemão.
\section{Disparo}
\begin{itemize}
\item {Grp. gram.:m.}
\end{itemize}
Acto de disparar.
\section{Dispartir}
\begin{itemize}
\item {Grp. gram.:v. t.}
\end{itemize}
\begin{itemize}
\item {Proveniência:(Lat. \textunderscore dispartire\textunderscore )}
\end{itemize}
Distribuir.
Separar em diversos sentidos.
\section{Dispautério}
\begin{itemize}
\item {Grp. gram.:m.}
\end{itemize}
(V.despautério)
\section{Dispêndio}
\begin{itemize}
\item {Grp. gram.:m.}
\end{itemize}
\begin{itemize}
\item {Utilização:Fig.}
\end{itemize}
\begin{itemize}
\item {Proveniência:(Lat. \textunderscore dispendium\textunderscore )}
\end{itemize}
Despesa; consumo.
Prejuízo, damno.
\section{Dispendiosamente}
\begin{itemize}
\item {Grp. gram.:adv.}
\end{itemize}
De modo dispendioso.
\section{Dispendioso}
\begin{itemize}
\item {Grp. gram.:adj.}
\end{itemize}
\begin{itemize}
\item {Proveniência:(Lat. \textunderscore dispendiosus\textunderscore )}
\end{itemize}
Que importa grande despesa.
Custoso.
\section{Dispensa}
\begin{itemize}
\item {Grp. gram.:f.}
\end{itemize}
\begin{itemize}
\item {Utilização:Prov.}
\end{itemize}
\begin{itemize}
\item {Utilização:minh.}
\end{itemize}
\begin{itemize}
\item {Proveniência:(De \textunderscore dispensar\textunderscore )}
\end{itemize}
Acto de sêr desobrigado.
Escusa: \textunderscore pedir dispensa de não comparecer\textunderscore .
Documento ou acto, em que se pede dispensa.
Documento em que ella se concede.
Autorização, com que se falta a um preceito legal.
O mesmo ou melhor que \textunderscore despensa\textunderscore , copa ou casa de arrecadação de mantimentos.
O mesmo que \textunderscore água-pé\textunderscore ^1.
\section{Dispensa}
\begin{itemize}
\item {Grp. gram.:f.}
\end{itemize}
\begin{itemize}
\item {Utilização:Ant.}
\end{itemize}
\begin{itemize}
\item {Proveniência:(Do lat. \textunderscore dispensa\textunderscore )}
\end{itemize}
Casa ou armário, em que se guardam provisões culinárias ou gêneros alimentícios, para uso doméstico.
O mesmo que \textunderscore despesa\textunderscore .
\section{Dispensabilidade}
\begin{itemize}
\item {Grp. gram.:f.}
\end{itemize}
Qualidade daquelle ou daquillo que é dispensavel.
\section{Dispensação}
\begin{itemize}
\item {Grp. gram.:f.}
\end{itemize}
\begin{itemize}
\item {Proveniência:(Lat. \textunderscore dispensatio\textunderscore )}
\end{itemize}
Acto de dispensar.
Dispensa.
\section{Dispensador}
\begin{itemize}
\item {Grp. gram.:m.}
\end{itemize}
\begin{itemize}
\item {Proveniência:(Lat. \textunderscore dispensator\textunderscore )}
\end{itemize}
Aquelle que dispensa.
\section{Dispensar}
\begin{itemize}
\item {Grp. gram.:v. t.}
\end{itemize}
\begin{itemize}
\item {Proveniência:(Lat. \textunderscore dispensare\textunderscore )}
\end{itemize}
Dar dispensa a.
Desobrigar.
Dar; conceder: \textunderscore dispensar attenção\textunderscore .
Ceder provisoriamente.
Distribuir.
\section{Dispensário}
\begin{itemize}
\item {Grp. gram.:m.}
\end{itemize}
\begin{itemize}
\item {Proveniência:(De \textunderscore dispensar\textunderscore )}
\end{itemize}
Estabelecimento de beneficência, para dar gratuitamente cuidados e medicamentos aos doentes pobres, que podem ser tratados no seu domicílio. Cf. Brás Luis de Abreu, \textunderscore Portugal Médico\textunderscore .
\section{Dispensatário}
\begin{itemize}
\item {Grp. gram.:m.}
\end{itemize}
\begin{itemize}
\item {Proveniência:(De \textunderscore dispensar\textunderscore )}
\end{itemize}
Aquelle que dá dispensa.
\section{Dispensativo}
\begin{itemize}
\item {Grp. gram.:adj.}
\end{itemize}
\begin{itemize}
\item {Proveniência:(Lat. \textunderscore dispensativus\textunderscore )}
\end{itemize}
Que dispensa; que é motivo de dispensa.
\section{Dispensatório}
\begin{itemize}
\item {Grp. gram.:m.}
\end{itemize}
\begin{itemize}
\item {Proveniência:(Lat. \textunderscore dispensatorius\textunderscore )}
\end{itemize}
Estabelecimento, annexo ás aulas de pharmácia, para demonstrações práticas do ensino.
Laboratório de drogas pharmacêuticas.
\section{Dispensável}
\begin{itemize}
\item {Grp. gram.:adj.}
\end{itemize}
Que se póde dispensar.
\section{Dispenseiro}
\begin{itemize}
\item {Grp. gram.:m.}
\end{itemize}
\begin{itemize}
\item {Proveniência:(De \textunderscore dispensa\textunderscore ^2)}
\end{itemize}
Aquelle, que tem a seu cargo a dispensa.
Aquelle que distribue os dons da munificência alheia.
\section{Disperder}
\begin{itemize}
\item {Grp. gram.:v. t.}
\end{itemize}
\begin{itemize}
\item {Proveniência:(Lat. \textunderscore disperdere\textunderscore )}
\end{itemize}
Destruir, arruinar; aniquilar. Cf. Garrett, \textunderscore Camões\textunderscore .
\section{Dispermático}
\begin{itemize}
\item {Grp. gram.:adj.}
\end{itemize}
\begin{itemize}
\item {Proveniência:(Do gr. \textunderscore dis\textunderscore  + \textunderscore sperma\textunderscore )}
\end{itemize}
Que encerra duas sementes.
\section{Dispermo}
\begin{itemize}
\item {Grp. gram.:adj.}
\end{itemize}
\begin{itemize}
\item {Utilização:Bot.}
\end{itemize}
\begin{itemize}
\item {Proveniência:(Do gr. \textunderscore dis\textunderscore  + \textunderscore sperma\textunderscore )}
\end{itemize}
Que encerra duas sementes.
\section{Dispersador}
\begin{itemize}
\item {Grp. gram.:adj}
\end{itemize}
Que dispersa.
\section{Dispersamente}
\begin{itemize}
\item {Grp. gram.:adv.}
\end{itemize}
\begin{itemize}
\item {Proveniência:(De \textunderscore disperso\textunderscore )}
\end{itemize}
Com dispersão; diffusamente.
\section{Dispersão}
\begin{itemize}
\item {Grp. gram.:f.}
\end{itemize}
\begin{itemize}
\item {Proveniência:(Do lat. \textunderscore dispersus\textunderscore )}
\end{itemize}
Acto ou effeito de dispersar.
\section{Dispersar}
\begin{itemize}
\item {Grp. gram.:v. t.}
\end{itemize}
\begin{itemize}
\item {Proveniência:(De \textunderscore disperso\textunderscore )}
\end{itemize}
Impellir para differentes partes.
Disseminar.
Pôr em debandada, em desordem.
Desbaratar.
Afugentar.
\section{Dispersivo}
\begin{itemize}
\item {Grp. gram.:adj.}
\end{itemize}
\begin{itemize}
\item {Proveniência:(De \textunderscore disperso\textunderscore )}
\end{itemize}
Que produz dispersão.
\section{Disperso}
\begin{itemize}
\item {Grp. gram.:adj.}
\end{itemize}
\begin{itemize}
\item {Proveniência:(Lat. \textunderscore dispersus\textunderscore )}
\end{itemize}
Espalhado.
Dividido, separado, em debandada.
Desordenado.
\section{Displicência}
\begin{itemize}
\item {Grp. gram.:f.}
\end{itemize}
\begin{itemize}
\item {Proveniência:(Lat. \textunderscore displicentia\textunderscore )}
\end{itemize}
Estado de quem se acha descontente.
Desagrado; aborrecimento.
\section{Displicente}
\begin{itemize}
\item {Grp. gram.:adj.}
\end{itemize}
\begin{itemize}
\item {Proveniência:(Lat. \textunderscore displicens\textunderscore )}
\end{itemize}
Que produz displicência.
\section{Dispoéta}
\begin{itemize}
\item {Grp. gram.:adj.}
\end{itemize}
\begin{itemize}
\item {Utilização:Des.}
\end{itemize}
\begin{itemize}
\item {Proveniência:(De \textunderscore dis...\textunderscore  + \textunderscore poêta\textunderscore )}
\end{itemize}
Avesso ou contrário á poesia.
Que não é poéta. Cf. Castilho, \textunderscore Fastos\textunderscore , I, 340.
\section{Dispondeu}
\begin{itemize}
\item {Grp. gram.:m.}
\end{itemize}
\begin{itemize}
\item {Proveniência:(Lat. \textunderscore dispondeus\textunderscore )}
\end{itemize}
Espondeu duplo.
\section{Disponente}
\begin{itemize}
\item {Grp. gram.:adj.}
\end{itemize}
\begin{itemize}
\item {Grp. gram.:M.}
\end{itemize}
\begin{itemize}
\item {Proveniência:(Lat. \textunderscore disponens\textunderscore )}
\end{itemize}
Que dispõe.
Aquelle que faz disposição de bens, em favor de alguém.
\section{Disponibilidade}
\begin{itemize}
\item {Grp. gram.:f.}
\end{itemize}
Qualidade daquelle ou daquillo que é ou está disponível.
Coisa disponivel ou conjunto de coisas disponíveis: \textunderscore o que me pedes não cabe nas minhas disponibilidades\textunderscore .
\section{Disponível}
\begin{itemize}
\item {Grp. gram.:adj.}
\end{itemize}
\begin{itemize}
\item {Proveniência:(Do lat. \textunderscore disponere\textunderscore )}
\end{itemize}
De que se póde dispor.
Que está desembaraçado.
\section{Dispor}
\begin{itemize}
\item {Grp. gram.:v. t.}
\end{itemize}
\begin{itemize}
\item {Grp. gram.:V. i.}
\end{itemize}
\begin{itemize}
\item {Grp. gram.:V. p.}
\end{itemize}
\begin{itemize}
\item {Grp. gram.:M.}
\end{itemize}
\begin{itemize}
\item {Proveniência:(Do lat. \textunderscore disponere\textunderscore )}
\end{itemize}
Pôr methodicamente, collocar por ordem.
Arrumar em lugar apropriado: \textunderscore dispor a mobília de uma casa\textunderscore .
Coordenar.
Graduar.
Subdividir.
Marcar.
Apropriar.
Determinar: \textunderscore dispor a execução de um serviço\textunderscore .
Planear.
Preparar: \textunderscore dispor as malas para a viagem\textunderscore .
Enfeitar com.
Incitar.
Persuadir.
Acommodar.
Dirigir, encaminhar.
Habituar.
Educar.
Transplantar: \textunderscore dispor cebolinho\textunderscore .
Usar livremente: \textunderscore eu disponho do meu cavallo\textunderscore .
Alienar bens: \textunderscore dispôs, em vida, de quanto tinha\textunderscore .
Têr a posse.
Sêr senhor.
Têr influência: \textunderscore dispor de muitos eleitores\textunderscore .
Dar applicação.
Decidir.
Dar ordens.
Fazer uso, utilizar-se.
Prescindir.
Fazer cedência.
Têr para dar ou emprestar.
Estar prompto ou resolvído: \textunderscore dispor-se a trabalhar\textunderscore .
Preparar-se.
Formar plano.
Resolver-se.
Applicar-se.
Têr desejo ou vontade.
Disposição: \textunderscore estou sempre ao seu dispor\textunderscore .
Talante.
Arbítrio.
\section{Disposar}
\begin{itemize}
\item {Grp. gram.:v. t.}
\end{itemize}
\begin{itemize}
\item {Utilização:Des.}
\end{itemize}
O mesmo que \textunderscore dispor\textunderscore .
\section{Disposição}
\begin{itemize}
\item {Grp. gram.:f.}
\end{itemize}
\begin{itemize}
\item {Proveniência:(Lat. \textunderscore dispositio\textunderscore )}
\end{itemize}
Acto ou effeito de dispor.
Tendência; aptidão: \textunderscore tem disposição para a música\textunderscore .
Situação.
Preceito, prescripção legal: \textunderscore as disposições do Codigo Civil\textunderscore .
Distribuição ordenada das partes de um discurso.
Constituição phýsica.
Temperamento.
Compleição; estado de saúde.
Subordinação, dependência: \textunderscore aquelle magistrado ficou á disposição do Ministério da Justiça\textunderscore .
\section{Dispositivamente}
\begin{itemize}
\item {Grp. gram.:adv.}
\end{itemize}
De modo dispositivo.
\section{Dispositivo}
\begin{itemize}
\item {Grp. gram.:adj.}
\end{itemize}
\begin{itemize}
\item {Proveniência:(Do lat. \textunderscore dispositus\textunderscore )}
\end{itemize}
Que contém disposição, ordem, prescripção.
\section{Dispôsto}
\begin{itemize}
\item {Grp. gram.:m.}
\end{itemize}
\begin{itemize}
\item {Proveniência:(Lat. \textunderscore dispositus\textunderscore )}
\end{itemize}
Aquillo que se preceituou.
Regra, preceito: \textunderscore segundo o disposto no artigo 20 da Constituição\textunderscore .
\section{Disputa}
\begin{itemize}
\item {Grp. gram.:f.}
\end{itemize}
Acto de disputar.
Contenda.
Discussão.
Rixa.
\section{Disputação}
\begin{itemize}
\item {Grp. gram.:f.}
\end{itemize}
\begin{itemize}
\item {Proveniência:(Lat. \textunderscore disputatio\textunderscore )}
\end{itemize}
O mesmo que \textunderscore disputa\textunderscore :«\textunderscore ...sem intervir na sua disputação\textunderscore ». Camillo, \textunderscore Voltareis, ó Christo?\textunderscore , 5.
\section{Disputador}
\begin{itemize}
\item {Grp. gram.:m.}
\end{itemize}
\begin{itemize}
\item {Proveniência:(Lat. \textunderscore disputator\textunderscore )}
\end{itemize}
Aquelle que disputa.
\section{Disputal}
\begin{itemize}
\item {Grp. gram.:adj.}
\end{itemize}
Relativo a disputa:«\textunderscore tal era a ré no disputal discrime\textunderscore ». Filinto, XVIII, 158.
\section{Disputante}
\begin{itemize}
\item {Grp. gram.:adj.}
\end{itemize}
\begin{itemize}
\item {Proveniência:(Lat. \textunderscore disputans\textunderscore )}
\end{itemize}
Que disputa.
\section{Disputar}
\begin{itemize}
\item {Grp. gram.:v. t.}
\end{itemize}
\begin{itemize}
\item {Grp. gram.:V. i.}
\end{itemize}
\begin{itemize}
\item {Proveniência:(Lat. \textunderscore disputare\textunderscore )}
\end{itemize}
Oppor-se a.
Contestar.
Pleitear.
Tornar objecto de contenda: \textunderscore disputar uma herança\textunderscore .
Lutar por: \textunderscore disputar um prêmio\textunderscore .
Sustentar em discussão.
Têr discussão, discutir.
Contender.
Rivalizar.
\section{Disputativo}
\begin{itemize}
\item {Grp. gram.:adj.}
\end{itemize}
\begin{itemize}
\item {Proveniência:(Lat. \textunderscore disputativus\textunderscore )}
\end{itemize}
Relativo a discussão scientífica.
Que gosta de disputar.
Que é objecto de discussão.
\section{Disputável}
\begin{itemize}
\item {Grp. gram.:adj.}
\end{itemize}
\begin{itemize}
\item {Proveniência:(Lat. \textunderscore disputabilis\textunderscore )}
\end{itemize}
Que póde sêr objecto de disputa.
\section{Disquisição}
\begin{itemize}
\item {Grp. gram.:f.}
\end{itemize}
\begin{itemize}
\item {Proveniência:(Lat. \textunderscore disquisitio\textunderscore )}
\end{itemize}
Investigação, pesquisa. Cf. Latino, \textunderscore Humboldt\textunderscore , 414 e 524.
\section{Disquisitório}
\begin{itemize}
\item {Grp. gram.:m.}
\end{itemize}
\begin{itemize}
\item {Utilização:bras}
\end{itemize}
\begin{itemize}
\item {Utilização:Neol.}
\end{itemize}
Repetição de exames ou observações, em alienados. Cf. \textunderscore Jornal do Comm.\textunderscore , do Rio, de 8-VIII-901.
\section{Dissabôr}
\begin{itemize}
\item {Grp. gram.:m.}
\end{itemize}
\begin{itemize}
\item {Proveniência:(De \textunderscore dis...\textunderscore  + \textunderscore sabor\textunderscore )}
\end{itemize}
Desprazer.
Mágoa.
Desgôsto; contratempo.
\section{Dissaborear}
\begin{itemize}
\item {Grp. gram.:v. t.}
\end{itemize}
Causar dissabor a.
\section{Dissaborido}
\begin{itemize}
\item {Grp. gram.:adj.}
\end{itemize}
O mesmo que \textunderscore dissaboroso\textunderscore . Cf. Camillo, \textunderscore Esqueleto\textunderscore , 54.
\section{Dissaboroso}
\begin{itemize}
\item {Grp. gram.:adj.}
\end{itemize}
\begin{itemize}
\item {Proveniência:(De \textunderscore dis...\textunderscore  + \textunderscore saboroso\textunderscore )}
\end{itemize}
Que não tem sabor.
Desgostoso.
Triste.
Que dissaboreia. Cf. Camillo. \textunderscore Estrêl. Fun.\textunderscore , 234.
\section{Dissaco}
\begin{itemize}
\item {Grp. gram.:m.}
\end{itemize}
Árvore intertropical, da fam. das sapotáceas, notável por seus agradáveis frutos.
\section{Dissecação}
\begin{itemize}
\item {Grp. gram.:f.}
\end{itemize}
Acto de dissecar.
\section{Dissecar}
\begin{itemize}
\item {Grp. gram.:v. t.}
\end{itemize}
\begin{itemize}
\item {Utilização:Fig.}
\end{itemize}
\begin{itemize}
\item {Proveniência:(Lat. \textunderscore dissecare\textunderscore )}
\end{itemize}
Cortar, dividir.
Separar com instrumento cirúrgico (qualquer órgão do corpo).
Dividir em partes (um corpo morto), para estudo.
Separar cortando (órgãos ou tecidos vegetaes), para estudo.
Analysar miudamente, rigorosamente: \textunderscore dissecar uma obra literária\textunderscore .
\section{Dissecção}
\begin{itemize}
\item {Grp. gram.:f.}
\end{itemize}
\begin{itemize}
\item {Proveniência:(Lat. \textunderscore dissectio\textunderscore )}
\end{itemize}
O mesmo que \textunderscore dissecação\textunderscore .
\section{Dissector}
\begin{itemize}
\item {Grp. gram.:m.}
\end{itemize}
\begin{itemize}
\item {Proveniência:(Do lat. \textunderscore dissectus\textunderscore )}
\end{itemize}
Aquelle que disseca.
Instrumento, com que se disseca.
\section{Dissegar}
\begin{itemize}
\item {Grp. gram.:v. t.}
\end{itemize}
\begin{itemize}
\item {Utilização:Ant.}
\end{itemize}
\begin{itemize}
\item {Proveniência:(De \textunderscore dis...\textunderscore  + \textunderscore segar\textunderscore , ou alter. de \textunderscore dissecar\textunderscore )}
\end{itemize}
Separar, dividir.
\section{Dissemelhança}
\begin{itemize}
\item {Grp. gram.:f.}
\end{itemize}
\begin{itemize}
\item {Proveniência:(De \textunderscore dis...\textunderscore  + \textunderscore semelhança\textunderscore )}
\end{itemize}
Falta de semelhança.
Desigualdade; differença.
\section{Dissemelhante}
\begin{itemize}
\item {Grp. gram.:adj.}
\end{itemize}
\begin{itemize}
\item {Proveniência:(De \textunderscore dis...\textunderscore  + \textunderscore semelhante\textunderscore )}
\end{itemize}
Que não é semelhante.
Differente.
\section{Dissemelhantemente}
\begin{itemize}
\item {Grp. gram.:adv.}
\end{itemize}
De modo dissemelhante.
\section{Dissemelhar}
\begin{itemize}
\item {Grp. gram.:v. t.}
\end{itemize}
Tornar dissemelhante.
\section{Disseminação}
\begin{itemize}
\item {Grp. gram.:f.}
\end{itemize}
\begin{itemize}
\item {Proveniência:(Lat. \textunderscore disseminatio\textunderscore )}
\end{itemize}
Acto ou effeito de disseminar.
Diffusão.
Vulgarização.
\section{Disseminador}
\begin{itemize}
\item {Grp. gram.:m.}
\end{itemize}
Aquelle que dissemina.
\section{Disseminar}
\begin{itemize}
\item {Grp. gram.:v. t.}
\end{itemize}
\begin{itemize}
\item {Proveniência:(Lat. \textunderscore disseminare\textunderscore )}
\end{itemize}
Semear.
Espalhar; diffundir; vulgarizar: \textunderscore disseminar ideias justas\textunderscore .
\section{Dissena}
\begin{itemize}
\item {Grp. gram.:f.}
\end{itemize}
Aparelho piscatório, em fórma de barco, na Lunda.
\section{Dissensão}
\begin{itemize}
\item {Grp. gram.:f.}
\end{itemize}
\begin{itemize}
\item {Proveniência:(Lat. \textunderscore dissensio\textunderscore )}
\end{itemize}
Acto de dissentir.
Divergência.
Desavença.
Contraste.
\section{Dissentâneo}
\begin{itemize}
\item {Grp. gram.:adj.}
\end{itemize}
\begin{itemize}
\item {Proveniência:(Lat. \textunderscore dissentaneus\textunderscore )}
\end{itemize}
Que dissente.
\section{Dissentimento}
\begin{itemize}
\item {Grp. gram.:m.}
\end{itemize}
\begin{itemize}
\item {Proveniência:(De \textunderscore dissentir\textunderscore )}
\end{itemize}
O mesmo que \textunderscore dissensão\textunderscore .
\section{Dissentir}
\begin{itemize}
\item {Grp. gram.:v. i.}
\end{itemize}
\begin{itemize}
\item {Proveniência:(Lat. \textunderscore dissentire\textunderscore )}
\end{itemize}
Sentir diversamente.
Não concordar.
Divergir.
Desavir-se.
\section{Dissépalo}
\begin{itemize}
\item {Grp. gram.:adj.}
\end{itemize}
\begin{itemize}
\item {Utilização:Bot.}
\end{itemize}
\begin{itemize}
\item {Proveniência:(De \textunderscore di...\textunderscore  + \textunderscore sépala\textunderscore )}
\end{itemize}
Que tem duas sépalas distintas.
\section{Dissertação}
\begin{itemize}
\item {Grp. gram.:f.}
\end{itemize}
\begin{itemize}
\item {Proveniência:(Lat. \textunderscore dissertatio\textunderscore )}
\end{itemize}
Exame, desenvolvimento, exposição escrita ou oral, de um ponto doutrinário.
\section{Dissertador}
\begin{itemize}
\item {Grp. gram.:m.}
\end{itemize}
\begin{itemize}
\item {Proveniência:(Lat. \textunderscore dissertator\textunderscore )}
\end{itemize}
Aquelle que disserta.
\section{Dissertar}
\begin{itemize}
\item {Grp. gram.:v. i.}
\end{itemize}
\begin{itemize}
\item {Proveniência:(Lat. \textunderscore dissertare\textunderscore )}
\end{itemize}
Fazer dissertação.
Discursar, discorrer methodicamcnte.
Tratar desenvolvidamente de um ponto doutrinário.
Discretear.
\section{Dissidência}
\begin{itemize}
\item {Grp. gram.:f.}
\end{itemize}
\begin{itemize}
\item {Proveniência:(Lat. \textunderscore dissidentia\textunderscore )}
\end{itemize}
O mesmo que \textunderscore dissensão\textunderscore .
\section{Dissidente}
\begin{itemize}
\item {Grp. gram.:m.  e  adj.}
\end{itemize}
\begin{itemize}
\item {Proveniência:(Lat. \textunderscore dissidens\textunderscore )}
\end{itemize}
O que se não conforma com as opiniões de outrem.
Aquelle que diverge da opinião geral.
\section{Dissidiar}
\begin{itemize}
\item {Grp. gram.:v. i.}
\end{itemize}
\begin{itemize}
\item {Proveniência:(De \textunderscore dissídio\textunderscore )}
\end{itemize}
Tornar dissidente, desunir.
\section{Dissídio}
\begin{itemize}
\item {Grp. gram.:m.}
\end{itemize}
\begin{itemize}
\item {Proveniência:(Lat. \textunderscore dissidium\textunderscore )}
\end{itemize}
O mesmo que \textunderscore dissensão\textunderscore .
\section{Dissímil}
\begin{itemize}
\item {Grp. gram.:adj.}
\end{itemize}
\begin{itemize}
\item {Proveniência:(Lat. \textunderscore dissimilis\textunderscore )}
\end{itemize}
O mesmo que \textunderscore dissemelhante\textunderscore .
\section{Dissimilação}
\begin{itemize}
\item {Grp. gram.:f.}
\end{itemize}
\begin{itemize}
\item {Utilização:Gram.}
\end{itemize}
Phenómeno, que é opposto ao da assimilação, e que consiste na modificação de vocábulos, letras ou sýllabas, evitando-se a repetição de sons similares, como \textunderscore particular\textunderscore , comparado com \textunderscore geral\textunderscore , visto que o \textunderscore al\textunderscore , em \textunderscore particular\textunderscore , se converteu em \textunderscore ar\textunderscore , por \textunderscore dissimilação\textunderscore  do \textunderscore l\textunderscore  de \textunderscore particular\textunderscore .
(Cp. \textunderscore assimilação\textunderscore )
\section{Dissimilar}
\begin{itemize}
\item {Grp. gram.:adj.}
\end{itemize}
\begin{itemize}
\item {Proveniência:(De \textunderscore dissimil\textunderscore )}
\end{itemize}
Que é de diverso gênero ou espécie.
\section{Dissimilitude}
\begin{itemize}
\item {Grp. gram.:f.}
\end{itemize}
\begin{itemize}
\item {Proveniência:(Lat. \textunderscore dissimilitudo\textunderscore )}
\end{itemize}
O mesmo que \textunderscore dissemelhança\textunderscore .
\section{Dissimulação}
\begin{itemize}
\item {Grp. gram.:f.}
\end{itemize}
\begin{itemize}
\item {Proveniência:(Lat. \textunderscore dissimulatio\textunderscore )}
\end{itemize}
Acto de dissimular.
\section{Dissimuladamente}
\begin{itemize}
\item {Grp. gram.:adv.}
\end{itemize}
De modo dissimulado.
\section{Dissimulado}
\begin{itemize}
\item {Grp. gram.:adj.}
\end{itemize}
\begin{itemize}
\item {Proveniência:(De \textunderscore dissimular\textunderscore )}
\end{itemize}
Astuto.
Disfarçado; encoberto: \textunderscore ódio dissimulado\textunderscore .
\section{Dissimulador}
\begin{itemize}
\item {Grp. gram.:m.  e  adj.}
\end{itemize}
\begin{itemize}
\item {Proveniência:(Lat. \textunderscore dissimulator\textunderscore )}
\end{itemize}
O que dissimula.
\section{Dissimular}
\begin{itemize}
\item {Grp. gram.:v. t.}
\end{itemize}
\begin{itemize}
\item {Grp. gram.:V. i.}
\end{itemize}
\begin{itemize}
\item {Proveniência:(Lat. \textunderscore dissimulare\textunderscore )}
\end{itemize}
Occultar astutamente.
Disfarçar.
Apresentar como differente.
Attenuar.
Calar: \textunderscore dissimular resentimentos\textunderscore .
Sêr dissimulado, têr reserva.
Proceder com dissimulação.
\section{Dissimulável}
\begin{itemize}
\item {Grp. gram.:adj.}
\end{itemize}
Que se póde dissimular.
\section{Dictado}
\begin{itemize}
\item {Grp. gram.:m.}
\end{itemize}
\begin{itemize}
\item {Proveniência:(De \textunderscore ditar\textunderscore )}
\end{itemize}
Aquillo que se dita, ou que se ditou.
Anexim; provérbio.
Brocardo.
\section{Dictador}
\begin{itemize}
\item {Grp. gram.:m.}
\end{itemize}
\begin{itemize}
\item {Utilização:Ext.}
\end{itemize}
\begin{itemize}
\item {Proveniência:(Lat. \textunderscore dictator\textunderscore )}
\end{itemize}
Antigo magistrado romano, que exercia poder absoluto.
Aquelle que reúne em si temporariamente todos os poderes públicos.
Pessôa autoritária, despótica.
\section{Dictadura}
\begin{itemize}
\item {Grp. gram.:f.}
\end{itemize}
\begin{itemize}
\item {Proveniência:(Lat. \textunderscore dictatura\textunderscore )}
\end{itemize}
Dignidade ou cargo de ditador.
Govêrno, em que o poder executivo absorve o legislativo ou o dispensa.
Autoridade absoluta.
\section{Dictame}
\begin{itemize}
\item {Grp. gram.:m.}
\end{itemize}
\begin{itemize}
\item {Proveniência:(Do lat. \textunderscore dictamen\textunderscore )}
\end{itemize}
Aquillo que se dita.
Aviso.
Regra.
Ordem.
Doutrina.
\section{Dictamno}
\begin{itemize}
\item {Grp. gram.:m.}
\end{itemize}
\begin{itemize}
\item {Proveniência:(Lat. \textunderscore dictamnum\textunderscore )}
\end{itemize}
Planta rutácea, muito aromática.
\section{Dictar}
\begin{itemize}
\item {Grp. gram.:v. t.}
\end{itemize}
\begin{itemize}
\item {Proveniência:(Lat. \textunderscore dictare\textunderscore )}
\end{itemize}
Dizer em voz alta, para que outrem escreva o que se vai dizendo.
Inspirar.
Prescrever.
Impor.
\section{Dictatorial}
\begin{itemize}
\item {Grp. gram.:adj.}
\end{itemize}
\begin{itemize}
\item {Proveniência:(De \textunderscore dictatório\textunderscore )}
\end{itemize}
Relativo a dictador ou a dictadura.
\section{Dictatório}
\begin{itemize}
\item {Grp. gram.:adj.}
\end{itemize}
\begin{itemize}
\item {Proveniência:(Lat. \textunderscore dictatorius\textunderscore )}
\end{itemize}
O mesmo que \textunderscore dictatorial\textunderscore .
\section{Dictério}
\begin{itemize}
\item {Grp. gram.:m.}
\end{itemize}
\begin{itemize}
\item {Proveniência:(Do gr. \textunderscore deikterion\textunderscore )}
\end{itemize}
Motejo.
Chufa; dichote.
\section{Dictinho}
\begin{itemize}
\item {Grp. gram.:m.}
\end{itemize}
Mexerico.
(Dem. de \textunderscore dicto\textunderscore )
\section{Dicto}
\begin{itemize}
\item {Grp. gram.:m.}
\end{itemize}
\begin{itemize}
\item {Grp. gram.:Adj.}
\end{itemize}
\begin{itemize}
\item {Proveniência:(Lat. \textunderscore dictus\textunderscore )}
\end{itemize}
Palavra, expressão.
Máxima.
Phrase.
Que se disse: \textunderscore aquillo que fica dicto\textunderscore .
\section{Dissilábico}
\begin{itemize}
\item {Grp. gram.:adj.}
\end{itemize}
Diz-se das línguas oceânicas, em que as palavras são compostas de duas sílabas.
O mesmo que \textunderscore dissílabo\textunderscore .
\section{Dissílabo}
\begin{itemize}
\item {Grp. gram.:adj.}
\end{itemize}
\begin{itemize}
\item {Grp. gram.:M.}
\end{itemize}
\begin{itemize}
\item {Proveniência:(Gr. \textunderscore disullabos\textunderscore )}
\end{itemize}
Que tem duas sílabas.
Palavra de duas sílabas.
\section{Dissímulo}
\begin{itemize}
\item {Grp. gram.:m.}
\end{itemize}
O mesmo que \textunderscore dissimulação\textunderscore . Cf. Filinto, XV, 270.
\section{Dissipação}
\begin{itemize}
\item {Grp. gram.:f.}
\end{itemize}
\begin{itemize}
\item {Proveniência:(Lat. \textunderscore dissipatio\textunderscore )}
\end{itemize}
Acto ou effeito de dissipar.
Devassidão, libertinagem.
\section{Dissipadamente}
\begin{itemize}
\item {Grp. gram.:adv.}
\end{itemize}
\begin{itemize}
\item {Proveniência:(De \textunderscore dissipar\textunderscore )}
\end{itemize}
Com dissipação.
\section{Dissipador}
\begin{itemize}
\item {Grp. gram.:m.  e  adj.}
\end{itemize}
\begin{itemize}
\item {Proveniência:(Lat. \textunderscore dissipator\textunderscore )}
\end{itemize}
O que dissipa.
Esbanjador.
\section{Dissipar}
\begin{itemize}
\item {Grp. gram.:v. t.}
\end{itemize}
\begin{itemize}
\item {Proveniência:(Lat. \textunderscore dissipare\textunderscore )}
\end{itemize}
Dispersar.
Fazer desvanecer; fazer cessar: \textunderscore a luz dissipa as trevas\textunderscore .
Afastar.
Gastar prodigamente; esbanjar; desperdiçar; consumir inutilmente: \textunderscore dissipar riquezas\textunderscore .
Perder (saúde ou fôrças) por excessos.
\section{Dissipável}
\begin{itemize}
\item {Grp. gram.:adj.}
\end{itemize}
\begin{itemize}
\item {Proveniência:(Lat. \textunderscore dissipabilis\textunderscore )}
\end{itemize}
Que se dissipa facilmente.
\section{Disso}
(contr. de \textunderscore de\textunderscore  + \textunderscore isso\textunderscore )
\section{Dissociabilidade}
\begin{itemize}
\item {Grp. gram.:f.}
\end{itemize}
Qualidade daquelle ou daquillo que é dissociável.
\section{Dissociação}
\begin{itemize}
\item {Grp. gram.:f.}
\end{itemize}
\begin{itemize}
\item {Proveniência:(Lat. \textunderscore dissociatio\textunderscore )}
\end{itemize}
Acto ou effeito de dissociar.
\section{Dissocial}
\begin{itemize}
\item {Grp. gram.:adj.}
\end{itemize}
\begin{itemize}
\item {Proveniência:(Lat. \textunderscore dissocialis\textunderscore )}
\end{itemize}
Que não se póde associar.
Insociável.
\section{Dissociar}
\begin{itemize}
\item {Grp. gram.:v. t.}
\end{itemize}
\begin{itemize}
\item {Proveniência:(Lat. \textunderscore dissociare\textunderscore )}
\end{itemize}
Desaggregar.
Dissolver (aquillo que estava associado).
Decompor chimicamente.
\section{Dissociável}
\begin{itemize}
\item {Grp. gram.:adj.}
\end{itemize}
\begin{itemize}
\item {Proveniência:(Lat. \textunderscore dissociabilis\textunderscore )}
\end{itemize}
Que se não póde associar.
\section{Dissoltamente}
\begin{itemize}
\item {Grp. gram.:adv.}
\end{itemize}
\begin{itemize}
\item {Utilização:Ant.}
\end{itemize}
O mesmo que \textunderscore dissolutamente\textunderscore .
\section{Dissolubilidade}
\begin{itemize}
\item {Grp. gram.:f.}
\end{itemize}
Qualidade daquillo que é dissolúvel.
\section{Dissolução}
\begin{itemize}
\item {Grp. gram.:f.}
\end{itemize}
\begin{itemize}
\item {Proveniência:(Lat. \textunderscore dissolutio\textunderscore )}
\end{itemize}
Acto ou effeito de dissolver.
Decomposição.
Desaggregação de moléculas.
Líquido, em que se dissolveu substância sólida.
Extincção de contrato ou sociedade.
Perversão de costumes; devassidão; corrupção.
\section{Dissolutamente}
\begin{itemize}
\item {Grp. gram.:adv.}
\end{itemize}
De modo dissoluto.
\section{Dissolutivo}
\begin{itemize}
\item {Grp. gram.:adj.}
\end{itemize}
\begin{itemize}
\item {Proveniência:(Lat. \textunderscore dissolutivus\textunderscore )}
\end{itemize}
Que dissolve.
\section{Dissoluto}
\begin{itemize}
\item {Grp. gram.:adj.}
\end{itemize}
\begin{itemize}
\item {Proveniência:(Lat. \textunderscore dissolutus\textunderscore )}
\end{itemize}
Libertino, devasso.
\section{Dissolúvel}
\begin{itemize}
\item {Grp. gram.:adj.}
\end{itemize}
\begin{itemize}
\item {Proveniência:(Lat. \textunderscore dissolubitis\textunderscore )}
\end{itemize}
Que póde sêr dissolvido.
\section{Dissolvência}
\begin{itemize}
\item {Grp. gram.:f.}
\end{itemize}
(V.dissolução)
\section{Dissolvente}
\begin{itemize}
\item {Grp. gram.:adj.}
\end{itemize}
\begin{itemize}
\item {Grp. gram.:M.}
\end{itemize}
\begin{itemize}
\item {Proveniência:(Lat. \textunderscore dissolvens\textunderscore )}
\end{itemize}
Que dissolve.
Aquillo que dissolve.
\section{Dissolver}
\begin{itemize}
\item {Grp. gram.:v. t.}
\end{itemize}
\begin{itemize}
\item {Proveniência:(Lat. \textunderscore dissolvere\textunderscore )}
\end{itemize}
Desligar.
Desaggregar.
Desfazer.
Fazer evaporar, fazer desapparecer.
Quebrar a ligação de.
Tornar nullo, invalidar: \textunderscore dissolver o casamento\textunderscore .
Desmembrar, separar os membros de (uma corporação).
Desmembrar.
Corromper.
Tornar dissoluto, devasso; produzir o desregramento de.
\section{Dissonância}
\begin{itemize}
\item {Grp. gram.:f.}
\end{itemize}
Qualidade de dissonante.
Falta de consonância ou de harmonia.
Desproporção entre as partes de um todo, nas côres de um quadro, no estilo, nas fórmas, etc.
\section{Dissonante}
\begin{itemize}
\item {Grp. gram.:adj.}
\end{itemize}
\begin{itemize}
\item {Proveniência:(Lat. \textunderscore dissonans\textunderscore )}
\end{itemize}
Que dissona; em que há dissonância.
\section{Dissonar}
\begin{itemize}
\item {Grp. gram.:v. i.}
\end{itemize}
\begin{itemize}
\item {Proveniência:(Lat. \textunderscore dissonare\textunderscore )}
\end{itemize}
Fazer dissonância.
\section{Díssono}
\begin{itemize}
\item {Grp. gram.:adj.}
\end{itemize}
\begin{itemize}
\item {Proveniência:(Lat. \textunderscore dissonus\textunderscore )}
\end{itemize}
O mesmo que \textunderscore dissonante\textunderscore .
\section{Dissonoro}
\begin{itemize}
\item {Grp. gram.:adj.}
\end{itemize}
\begin{itemize}
\item {Proveniência:(Lat. \textunderscore dissonorus\textunderscore )}
\end{itemize}
O mesmo que \textunderscore dissonante\textunderscore .
\section{Dissuadimento}
\begin{itemize}
\item {Grp. gram.:m.}
\end{itemize}
Acto ou effeito de dissuadir. Cf. Filinto, \textunderscore D. Man.\textunderscore , II, 69.
\section{Dissuadir}
\begin{itemize}
\item {Grp. gram.:v. i.}
\end{itemize}
\begin{itemize}
\item {Proveniência:(Lat. \textunderscore dissuadere\textunderscore )}
\end{itemize}
Despersuadir.
Desaconselhar.
Fazer mudar de opinião.
\section{Dissuasão}
\begin{itemize}
\item {Grp. gram.:f.}
\end{itemize}
\begin{itemize}
\item {Proveniência:(Lat. \textunderscore dissuasio\textunderscore )}
\end{itemize}
Acto ou effeito de dissuadir.
\section{Dissuasivo}
\begin{itemize}
\item {Grp. gram.:adj.}
\end{itemize}
\begin{itemize}
\item {Proveniência:(Do lat. \textunderscore dissuasus\textunderscore )}
\end{itemize}
Próprio para dissuadir.
\section{Dissuasor}
\begin{itemize}
\item {Grp. gram.:m.  e  adj.}
\end{itemize}
\begin{itemize}
\item {Utilização:Des.}
\end{itemize}
\begin{itemize}
\item {Proveniência:(Lat. \textunderscore dissuasor\textunderscore )}
\end{itemize}
O que dissuade.
\section{Dissuasório}
\begin{itemize}
\item {Grp. gram.:adj.}
\end{itemize}
O mesmo que \textunderscore dissuasivo\textunderscore .
\section{Distal}
\begin{itemize}
\item {Grp. gram.:adj.}
\end{itemize}
\begin{itemize}
\item {Utilização:Anat.}
\end{itemize}
Que fica para o lado dos dedos.
Voltado para o extremo dos membros.
(Relaciona-se com \textunderscore digital\textunderscore  e com o gr. \textunderscore deikein\textunderscore )
\section{Distância}
\begin{itemize}
\item {Grp. gram.:f.}
\end{itemize}
\begin{itemize}
\item {Proveniência:(Lat. \textunderscore distantia\textunderscore )}
\end{itemize}
Espaço entre duas coisas ou pessôas.
Intervallo.
Afastamento, separação.
Espaço entre duas épocas.
Grande differença.
\section{Distanciadamente}
\begin{itemize}
\item {Grp. gram.:adv.}
\end{itemize}
\begin{itemize}
\item {Proveniência:(De \textunderscore distanciar\textunderscore )}
\end{itemize}
Com distância; a certa distância.
\section{Distanciador}
\begin{itemize}
\item {Grp. gram.:adj.}
\end{itemize}
Que põe distante.
Que desaproxima, que distancia. Cf. J. Lourenço Pinto, \textunderscore Senhor Deputado\textunderscore , 41.
\section{Distanciar}
\begin{itemize}
\item {Grp. gram.:v. t.}
\end{itemize}
\begin{itemize}
\item {Proveniência:(De \textunderscore distância\textunderscore )}
\end{itemize}
Afastar, pôr distante.
\section{Distanciómetro}
\begin{itemize}
\item {Grp. gram.:m.}
\end{itemize}
\begin{itemize}
\item {Proveniência:(T. hybr., do lat. \textunderscore distantia\textunderscore  + gr. \textunderscore metron\textunderscore )}
\end{itemize}
Instrumento, inventado há poucos annos por D. Martinho da França Pereira Coutinho, e próprio para medir as distâncias a pontos inacessíveis.
\section{Distante}
\begin{itemize}
\item {Grp. gram.:adj.}
\end{itemize}
\begin{itemize}
\item {Proveniência:(Lat. \textunderscore distans\textunderscore )}
\end{itemize}
Que dista.
Que está longe.
Remoto: \textunderscore tempos distantes\textunderscore .
Que se avista ao longe.
Que se faz ouvir de longe: \textunderscore sons distantes\textunderscore .
\section{Distantemente}
\begin{itemize}
\item {Grp. gram.:adv.}
\end{itemize}
\begin{itemize}
\item {Proveniência:(De \textunderscore distante\textunderscore )}
\end{itemize}
Ao longe.
\section{Distar}
\begin{itemize}
\item {Grp. gram.:v. i.}
\end{itemize}
\begin{itemize}
\item {Proveniência:(Lat. \textunderscore distare\textunderscore )}
\end{itemize}
Estar ou sêr distante.
Divergir.
\section{Distender}
\begin{itemize}
\item {Grp. gram.:v. t.}
\end{itemize}
\begin{itemize}
\item {Proveniência:(Lat. \textunderscore distendere\textunderscore )}
\end{itemize}
Estender para vários lados: \textunderscore distender a vista\textunderscore .
Estender muito.
Desenvolver.
Retesar: \textunderscore distender uma corda\textunderscore .
Dilatar.
\section{Disteno}
\begin{itemize}
\item {Grp. gram.:m.}
\end{itemize}
\begin{itemize}
\item {Proveniência:(Do gr. \textunderscore dis\textunderscore  + \textunderscore stenos\textunderscore )}
\end{itemize}
Mineral, que apresenta geralmente a fórma de dois cristaes alongados.
\section{Distensão}
\begin{itemize}
\item {Grp. gram.:f.}
\end{itemize}
\begin{itemize}
\item {Proveniência:(Lat. \textunderscore distensio\textunderscore )}
\end{itemize}
Acto ou effeito de distender.
Torção violenta dos ligamentos de uma articulação.
\section{Distenso}
\begin{itemize}
\item {Grp. gram.:adj.}
\end{itemize}
\begin{itemize}
\item {Proveniência:(Lat. \textunderscore distensus\textunderscore )}
\end{itemize}
Que soffreu distensão.
Que se estendeu.
\section{Distensor}
\begin{itemize}
\item {Grp. gram.:adj.}
\end{itemize}
\begin{itemize}
\item {Proveniência:(Do lat. \textunderscore distensus\textunderscore )}
\end{itemize}
Que distende.
\textunderscore M.\textunderscore  Aquillo que distende.
\section{Distheno}
\begin{itemize}
\item {Grp. gram.:m.}
\end{itemize}
\begin{itemize}
\item {Proveniência:(Do gr. \textunderscore dis\textunderscore  + \textunderscore stenos\textunderscore )}
\end{itemize}
Mineral, que apresenta geralmente a fórma de dois crystaes alongados.
\section{Disticado}
\begin{itemize}
\item {Grp. gram.:adj.}
\end{itemize}
\begin{itemize}
\item {Utilização:Bot.}
\end{itemize}
\begin{itemize}
\item {Proveniência:(Do gr. \textunderscore dis\textunderscore  + \textunderscore stikhos\textunderscore )}
\end{itemize}
Diz-se dos órgãos vegetaes, dispostos em duas séries opostas, ao longo de um eixo comum, como os ramos do olmo.
\section{Distichado}
\begin{itemize}
\item {fónica:ca}
\end{itemize}
\begin{itemize}
\item {Grp. gram.:adj.}
\end{itemize}
\begin{itemize}
\item {Utilização:Bot.}
\end{itemize}
\begin{itemize}
\item {Proveniência:(Do gr. \textunderscore dis\textunderscore  + \textunderscore stikhos\textunderscore )}
\end{itemize}
Diz-se dos órgãos vegetaes, dispostos em duas séries oppostas, ao longo de um eixo commum, como os ramos do olmo.
\section{Distichíase}
\begin{itemize}
\item {fónica:qui}
\end{itemize}
\begin{itemize}
\item {Grp. gram.:f.}
\end{itemize}
\begin{itemize}
\item {Proveniência:(Gr. \textunderscore distikhiasis\textunderscore )}
\end{itemize}
Anomalia, caracterizada por duas ordens de pestanas sobrepostas, cuja extremidade se volta para o globo do ôlho.
\section{Dístico}
\begin{itemize}
\item {Grp. gram.:adj.}
\end{itemize}
\begin{itemize}
\item {Grp. gram.:M.}
\end{itemize}
\begin{itemize}
\item {Proveniência:(Lat. \textunderscore disticus\textunderscore )}
\end{itemize}
Que tem duas séries ao longo de um eixo commum.
Grupo de dois versos.
Máxima em dois versos.
Divisa.
Rótudo; letreiro.
\section{Distillar}
\textunderscore v. t.\textunderscore  (e der.)
(V. \textunderscore destillar\textunderscore , etc.)
\section{Distilo}
\begin{itemize}
\item {Grp. gram.:adj.}
\end{itemize}
\begin{itemize}
\item {Utilização:Bot.}
\end{itemize}
\begin{itemize}
\item {Proveniência:(Do gr. \textunderscore dis\textunderscore  + \textunderscore stulos\textunderscore )}
\end{itemize}
Diz-se dos vegetaes que tem dois estiletes.
\section{Distinção}
\begin{itemize}
\item {Grp. gram.:f.}
\end{itemize}
\begin{itemize}
\item {Proveniência:(Lat. \textunderscore distinctio\textunderscore )}
\end{itemize}
Acto ou efeito de distinguir.
Diferença.
Prerogativa.
Sinal ou qualidade, por que uma coisa se diferença de outra.
Urbanidade; educação apurada.
Nobreza de porte: \textunderscore pessôa de distinção\textunderscore .
Elegância.
Correcção de procedimento.
Clareza.
\section{Distincção}
\begin{itemize}
\item {Grp. gram.:f.}
\end{itemize}
\begin{itemize}
\item {Proveniência:(Lat. \textunderscore distinctio\textunderscore )}
\end{itemize}
Acto ou effeito de distinguir.
Differença.
Prerogativa.
Sinal ou qualidade, por que uma coisa se differença de outra.
Urbanidade; educação apurada.
Nobreza de porte: \textunderscore pessôa de distincção\textunderscore .
Elegância.
Correcção de procedimento.
Clareza.
\section{Distinctamente}
\begin{itemize}
\item {Grp. gram.:adv.}
\end{itemize}
De modo distincto.
Com distincção.
\section{Distinctivamente}
\begin{itemize}
\item {Grp. gram.:adv.}
\end{itemize}
De modo distinctivo.
\section{Distinctivo}
\begin{itemize}
\item {Grp. gram.:adj.}
\end{itemize}
\begin{itemize}
\item {Grp. gram.:M.}
\end{itemize}
\begin{itemize}
\item {Proveniência:(De \textunderscore distincto\textunderscore )}
\end{itemize}
Próprio para distinguir.
Coisa que distingue.
Sinal; emblema: \textunderscore o distinctivo de marinheiro\textunderscore .
\section{Distincto}
\begin{itemize}
\item {Grp. gram.:adj.}
\end{itemize}
\begin{itemize}
\item {Proveniência:(Lat. \textunderscore distinctus\textunderscore )}
\end{itemize}
Que differe outrem ou de outra coisa.
Que se não confunde: \textunderscore duas coisas distinctas\textunderscore .
Que tem distincção.
Illustre.
Eminente; que sobresái: \textunderscore literato distincto\textunderscore .
\section{Distinguidor}
\begin{itemize}
\item {Grp. gram.:m.}
\end{itemize}
Aquelle que distingue.
\section{Distinguir}
\begin{itemize}
\item {Grp. gram.:v. t.}
\end{itemize}
\begin{itemize}
\item {Proveniência:(Lat. \textunderscore distinguere\textunderscore )}
\end{itemize}
Separar, differençar.
Discriminar.
Caracterizar.
Especificar o sentido de.
Sentir, perceber.
Tornar notável, superior.
Marcar.
Avistar: \textunderscore distinguir um vulto\textunderscore .
Dar preferência a.
\section{Distinguível}
\begin{itemize}
\item {Grp. gram.:adj.}
\end{itemize}
Que se póde distinguir.
Que se distingue.
Digno de distincção.
\section{Distintamente}
\begin{itemize}
\item {Grp. gram.:adv.}
\end{itemize}
De modo distinto.
Com distinção.
\section{Distintivamente}
\begin{itemize}
\item {Grp. gram.:adv.}
\end{itemize}
De modo distintivo.
\section{Distinto}
\begin{itemize}
\item {Grp. gram.:adj.}
\end{itemize}
\begin{itemize}
\item {Proveniência:(Lat. \textunderscore distinctus\textunderscore )}
\end{itemize}
Que difere outrem ou de outra coisa.
Que se não confunde: \textunderscore duas coisas distintas\textunderscore .
Que tem distinção.
Ilustre.
Eminente; que sobresái: \textunderscore literato distinto\textunderscore .
\section{Distiquíase}
\begin{itemize}
\item {Grp. gram.:f.}
\end{itemize}
\begin{itemize}
\item {Proveniência:(Gr. \textunderscore distikhiasis\textunderscore )}
\end{itemize}
Anomalia, caracterizada por duas ordens de pestanas sobrepostas, cuja extremidade se volta para o globo do ôlho.
\section{Dístoma}
\begin{itemize}
\item {Grp. gram.:m.}
\end{itemize}
\begin{itemize}
\item {Proveniência:(Do gr. \textunderscore dis\textunderscore  + \textunderscore stoma\textunderscore )}
\end{itemize}
Entozoário, que existe na veia-porta e suas ramificações, e que é vulgar no homem de algumas regiões da África.
\section{Distomatose}
\begin{itemize}
\item {Grp. gram.:f.}
\end{itemize}
Cachexia aquosa.
Papeira.
(Cp. \textunderscore dístoma\textunderscore )
\section{Distomíase}
\begin{itemize}
\item {Grp. gram.:f.}
\end{itemize}
\begin{itemize}
\item {Utilização:Med.}
\end{itemize}
\begin{itemize}
\item {Proveniência:(Do gr. \textunderscore dis\textunderscore  + \textunderscore stoma\textunderscore )}
\end{itemize}
Cachexia verminosa.
\section{Dístomo}
\begin{itemize}
\item {Grp. gram.:adj.}
\end{itemize}
\begin{itemize}
\item {Proveniência:(Do gr. \textunderscore dis\textunderscore  + \textunderscore stoma\textunderscore )}
\end{itemize}
Que tem duas bôcas.
\section{Distorção}
\begin{itemize}
\item {Grp. gram.:f.}
\end{itemize}
\begin{itemize}
\item {Utilização:Miner.}
\end{itemize}
\begin{itemize}
\item {Proveniência:(Do lat. \textunderscore distortus\textunderscore )}
\end{itemize}
Acto de distorcer.
Desenvolvimento desigual dos elementos crystallographicamente homólogos dos crystaes.
\section{Distorcer}
\begin{itemize}
\item {Grp. gram.:v. t.}
\end{itemize}
(V.destorcer)
\section{Distracção}
\begin{itemize}
\item {Grp. gram.:f.}
\end{itemize}
\begin{itemize}
\item {Proveniência:(Lat. \textunderscore distractio\textunderscore )}
\end{itemize}
Acto de distrahir.
Inadvertência, estado de quem se acha distrahido.
Irreflexão.
Palavra ou acto irreflectido.
Aquillo que distrai: \textunderscore procurar distracções\textunderscore .
\section{Distractar}
\begin{itemize}
\item {Grp. gram.:v. t.}
\end{itemize}
\begin{itemize}
\item {Proveniência:(Do lat. \textunderscore distractus\textunderscore )}
\end{itemize}
Tornar nullo, desfazer (contrato ou pacto).
\section{Distracte}
\begin{itemize}
\item {Grp. gram.:m.}
\end{itemize}
Acto de distractar.
\section{Distráctil}
\begin{itemize}
\item {Grp. gram.:adj.}
\end{itemize}
\begin{itemize}
\item {Utilização:Bot.}
\end{itemize}
\begin{itemize}
\item {Proveniência:(Do lat. \textunderscore distractus\textunderscore )}
\end{itemize}
Diz-se do órgão vegetal, que, ligado ás duas céllulas da anthera, as tem sensivelmente afastadas uma da outra.
\section{Distractivo}
\begin{itemize}
\item {Grp. gram.:adj.}
\end{itemize}
\begin{itemize}
\item {Proveniência:(Do lat. \textunderscore distractus\textunderscore )}
\end{itemize}
Que distrai.
\section{Distracto}
\begin{itemize}
\item {Grp. gram.:m.}
\end{itemize}
O mesmo que \textunderscore distracte\textunderscore .
\section{Distrahidamente}
\begin{itemize}
\item {Grp. gram.:adv.}
\end{itemize}
De modo distrahido.
Irreflectidamente.
\section{Distrahido}
\begin{itemize}
\item {Grp. gram.:adj.}
\end{itemize}
\begin{itemize}
\item {Proveniência:(De \textunderscore distrahir\textunderscore )}
\end{itemize}
Sujeito a distracções.
Concentrado.
Entretido.
Descuidado: \textunderscore apanhou-me distrahido...\textunderscore 
\section{Distrahimento}
\begin{itemize}
\item {Grp. gram.:m.}
\end{itemize}
O mesmo que \textunderscore distracção\textunderscore .
\section{Distrahir}
\begin{itemize}
\item {Grp. gram.:v. t.}
\end{itemize}
\begin{itemize}
\item {Proveniência:(Lat. \textunderscore distrahere\textunderscore )}
\end{itemize}
Separar em diversas direcções.
Tirar; desviar: \textunderscore distrahir dinheiro de um cofre\textunderscore .
Tornar desattento, esquecido.
Divertir, entreter: \textunderscore a leitura distrai-me\textunderscore .
Desencaminhar.
Attrahir, chamar, de um ponto para outro.
\section{Distraidamente}
\begin{itemize}
\item {fónica:tra-i}
\end{itemize}
\begin{itemize}
\item {Grp. gram.:adv.}
\end{itemize}
De modo distraido.
Irreflectidamente.
\section{Distraído}
\begin{itemize}
\item {Grp. gram.:adj.}
\end{itemize}
\begin{itemize}
\item {Proveniência:(De \textunderscore distrair\textunderscore )}
\end{itemize}
Sujeito a distrações.
Concentrado.
Entretido.
Descuidado: \textunderscore apanhou-me distraído...\textunderscore 
\section{Distraimento}
\begin{itemize}
\item {fónica:tra-i}
\end{itemize}
\begin{itemize}
\item {Grp. gram.:m.}
\end{itemize}
O mesmo que \textunderscore distracção\textunderscore .
\section{Distrair}
\begin{itemize}
\item {Grp. gram.:v. t.}
\end{itemize}
\begin{itemize}
\item {Proveniência:(Lat. \textunderscore distrahere\textunderscore )}
\end{itemize}
Separar em diversas direcções.
Tirar; desviar: \textunderscore distrair dinheiro de um cofre\textunderscore .
Tornar desatento, esquecido.
Divertir, entreter: \textunderscore a leitura distrai-me\textunderscore .
Desencaminhar.
Atrair, chamar, de um ponto para outro.
\section{Distribuição}
\begin{itemize}
\item {fónica:bu-i}
\end{itemize}
\begin{itemize}
\item {Grp. gram.:f.}
\end{itemize}
\begin{itemize}
\item {Proveniência:(Lat. \textunderscore distributio\textunderscore )}
\end{itemize}
Acto de distribuir.
\section{Distribuidor}
\begin{itemize}
\item {fónica:bu-i}
\end{itemize}
\begin{itemize}
\item {Grp. gram.:adj.}
\end{itemize}
\begin{itemize}
\item {Grp. gram.:M.}
\end{itemize}
\begin{itemize}
\item {Proveniência:(Lat. \textunderscore distributor\textunderscore )}
\end{itemize}
Que distribue.
Aquelle que distribue.
\section{Distribuir}
\begin{itemize}
\item {Grp. gram.:v. t.}
\end{itemize}
\begin{itemize}
\item {Proveniência:(Lat. \textunderscore distribuere\textunderscore )}
\end{itemize}
Dar a diversas pessôas, ou por diversos modos; repartir: \textunderscore distribuir esmolas\textunderscore .
Dispensar.
Espalhar em differentes sentidos; lançar para pontos diversos: \textunderscore distribuir impressos\textunderscore .
Classificar, pôr por ordem: \textunderscore distribuir os alumnos de uma classe\textunderscore .
Entregar ou commeter (uma causa judicial) ao juiz, para que a examine ou julgue, ou a um escrivão para que a processe.
\section{Distributivamente}
\begin{itemize}
\item {Grp. gram.:adv.}
\end{itemize}
De modo distributivo.
\section{Distributivo}
\begin{itemize}
\item {Grp. gram.:adj.}
\end{itemize}
\begin{itemize}
\item {Proveniência:(Lat. \textunderscore distributivus\textunderscore )}
\end{itemize}
Que distribue.
Equitativo.
\section{Distrição}
\begin{itemize}
\item {Grp. gram.:f.}
\end{itemize}
\begin{itemize}
\item {Utilização:Des.}
\end{itemize}
\begin{itemize}
\item {Proveniência:(Lat. \textunderscore districtio\textunderscore )}
\end{itemize}
Embaraço; aflição.
\section{Districção}
\begin{itemize}
\item {Grp. gram.:f.}
\end{itemize}
\begin{itemize}
\item {Utilização:Des.}
\end{itemize}
\begin{itemize}
\item {Proveniência:(Lat. \textunderscore districtio\textunderscore )}
\end{itemize}
Embaraço; afflicção.
\section{Districtal}
\begin{itemize}
\item {Grp. gram.:adj.}
\end{itemize}
Relativo a districto.
\section{Districto}
\begin{itemize}
\item {Grp. gram.:m.}
\end{itemize}
\begin{itemize}
\item {Proveniência:(Lat. \textunderscore districtus\textunderscore )}
\end{itemize}
Área de uma jurisdicção.
Competência.
Divisão administrativa de alguns países.
Secção de talho, nas salinas.
\section{Distrital}
\begin{itemize}
\item {Grp. gram.:adj.}
\end{itemize}
Relativo a distrito.
\section{Distrito}
\begin{itemize}
\item {Grp. gram.:m.}
\end{itemize}
\begin{itemize}
\item {Proveniência:(Lat. \textunderscore districtus\textunderscore )}
\end{itemize}
Área de uma jurisdicção.
Competência.
Divisão administrativa de alguns países.
Secção de talho, nas salinas.
\section{Disturbar}
\begin{itemize}
\item {Grp. gram.:v. t.}
\end{itemize}
\begin{itemize}
\item {Proveniência:(Lat. \textunderscore disturbare\textunderscore )}
\end{itemize}
O mesmo que \textunderscore perturbar\textunderscore .
\section{Distúrbio}
\begin{itemize}
\item {Grp. gram.:m.}
\end{itemize}
Acto de disturbar.
Motim.
Traquinice.
(B. lat. \textunderscore disturbium\textunderscore )
\section{Distylo}
\begin{itemize}
\item {Grp. gram.:adj.}
\end{itemize}
\begin{itemize}
\item {Utilização:Bot.}
\end{itemize}
\begin{itemize}
\item {Proveniência:(Do gr. \textunderscore dis\textunderscore  + \textunderscore stulos\textunderscore )}
\end{itemize}
Diz-se dos vegetaes que tem dois estyletes.
\section{Disyllábico}
\begin{itemize}
\item {fónica:si}
\end{itemize}
\begin{itemize}
\item {Grp. gram.:adj.}
\end{itemize}
Diz-se das línguas oceânicas, em que as palavras são compostas de duas sýllabas.
O mesmo que \textunderscore disýllabo\textunderscore .
\section{Disýllabo}
\begin{itemize}
\item {fónica:si}
\end{itemize}
\begin{itemize}
\item {Grp. gram.:adj.}
\end{itemize}
\begin{itemize}
\item {Grp. gram.:M.}
\end{itemize}
\begin{itemize}
\item {Proveniência:(Gr. \textunderscore disullabos\textunderscore )}
\end{itemize}
Que tem duas sýllabas.
Palavra de duas sýllabas.
\section{Dita}
\begin{itemize}
\item {Grp. gram.:f.}
\end{itemize}
\begin{itemize}
\item {Proveniência:(Do rad. do lat. \textunderscore ditare\textunderscore ?)}
\end{itemize}
Ventura, fortuna.
Sorte feliz.
\section{Ditado}
\begin{itemize}
\item {Grp. gram.:m.}
\end{itemize}
\begin{itemize}
\item {Proveniência:(De \textunderscore ditar\textunderscore )}
\end{itemize}
Aquillo que se dita, ou que se ditou.
Anexim; provérbio.
Brocardo.
\section{Ditador}
\begin{itemize}
\item {Grp. gram.:m.}
\end{itemize}
\begin{itemize}
\item {Utilização:Ext.}
\end{itemize}
\begin{itemize}
\item {Proveniência:(Lat. \textunderscore dictator\textunderscore )}
\end{itemize}
Antigo magistrado romano, que exercia poder absoluto.
Aquelle que reúne em si temporariamente todos os poderes públicos.
Pessôa autoritária, despótica.
\section{Ditadura}
\begin{itemize}
\item {Grp. gram.:f.}
\end{itemize}
\begin{itemize}
\item {Proveniência:(Lat. \textunderscore dictatura\textunderscore )}
\end{itemize}
Dignidade ou cargo de ditador.
Govêrno, em que o poder executivo absorve o legislativo ou o dispensa.
Autoridade absoluta.
\section{Ditame}
\begin{itemize}
\item {Grp. gram.:m.}
\end{itemize}
\begin{itemize}
\item {Proveniência:(Do lat. \textunderscore dictamen\textunderscore )}
\end{itemize}
Aquillo que se dita.
Aviso.
Regra.
Ordem.
Doutrina.
\section{Ditamno}
\begin{itemize}
\item {Grp. gram.:m.}
\end{itemize}
\begin{itemize}
\item {Proveniência:(Lat. \textunderscore dictamnum\textunderscore )}
\end{itemize}
Planta rutácea, muito aromática.
\section{Ditar}
\begin{itemize}
\item {Grp. gram.:v. t.}
\end{itemize}
\begin{itemize}
\item {Proveniência:(Lat. \textunderscore dictare\textunderscore )}
\end{itemize}
Dizer em voz alta, para que outrem escreva o que se vai dizendo.
Inspirar.
Prescrever.
Impor.
\section{Ditassôa}
\begin{itemize}
\item {Grp. gram.:f.}
\end{itemize}
Peixe africano da região do Quanza. Cf. Serpa Pinto, I, 211.
\section{Ditatorial}
\begin{itemize}
\item {Grp. gram.:adj.}
\end{itemize}
\begin{itemize}
\item {Proveniência:(De \textunderscore ditatório\textunderscore )}
\end{itemize}
Relativo a ditador ou a ditadura.
\section{Ditatório}
\begin{itemize}
\item {Grp. gram.:adj.}
\end{itemize}
\begin{itemize}
\item {Proveniência:(Lat. \textunderscore dictatorius\textunderscore )}
\end{itemize}
O mesmo que \textunderscore ditatorial\textunderscore .
\section{Diteque}
\begin{itemize}
\item {Grp. gram.:m.}
\end{itemize}
Arvoreta bixacea, procedente da América e muito conhecida em Angola, (\textunderscore bixa orellana\textunderscore , Lin.).
\section{Ditério}
\begin{itemize}
\item {Grp. gram.:m.}
\end{itemize}
\begin{itemize}
\item {Proveniência:(Do gr. \textunderscore deikterion\textunderscore )}
\end{itemize}
Motejo.
Chufa; dichote.
\section{Dithyrâmbico}
\begin{itemize}
\item {Grp. gram.:adj.}
\end{itemize}
Relativo a dithyrambo.
\section{Dithyrambo}
\begin{itemize}
\item {Grp. gram.:m.}
\end{itemize}
\begin{itemize}
\item {Utilização:Ant.}
\end{itemize}
\begin{itemize}
\item {Proveniência:(Gr. \textunderscore Dithurambos\textunderscore , sobrenome de Baccho)}
\end{itemize}
Composição poética de versos e estâncias irregulares, que exprime enthusiasmo ou delírio.
Hymno em honra de Baccho.
\section{Ditinho}
\begin{itemize}
\item {Grp. gram.:m.}
\end{itemize}
Mexerico.
(Dem. de \textunderscore dito\textunderscore )
\section{Ditirâmbico}
\begin{itemize}
\item {Grp. gram.:adj.}
\end{itemize}
Relativo a ditirambo.
\section{Ditirambo}
\begin{itemize}
\item {Grp. gram.:m.}
\end{itemize}
\begin{itemize}
\item {Utilização:Ant.}
\end{itemize}
\begin{itemize}
\item {Proveniência:(Gr. \textunderscore Dithurambos\textunderscore , sobrenome de Baco)}
\end{itemize}
Composição poética de versos e estâncias irregulares, que exprime entusiasmo ou delírio.
Hino em honra de Baco.
\section{Dito}
\begin{itemize}
\item {Grp. gram.:m.}
\end{itemize}
\begin{itemize}
\item {Grp. gram.:Adj.}
\end{itemize}
\begin{itemize}
\item {Proveniência:(Lat. \textunderscore dictus\textunderscore )}
\end{itemize}
Palavra, expressão.
Máxima.
Frase.
Que se disse: \textunderscore aquilo que fica dito\textunderscore .
\section{Dítome}
\begin{itemize}
\item {Grp. gram.:adj.}
\end{itemize}
\begin{itemize}
\item {Proveniência:(De \textunderscore di...\textunderscore  + gr. \textunderscore tomè\textunderscore )}
\end{itemize}
(V.bivalve)
\section{Ditongação}
\begin{itemize}
\item {Grp. gram.:f.}
\end{itemize}
Acto de ditongar.
\section{Ditongal}
\begin{itemize}
\item {Grp. gram.:adj.}
\end{itemize}
Relativo a ditongo.
Que fórma ditongo.
\section{Ditongar}
\begin{itemize}
\item {Grp. gram.:v. t.}
\end{itemize}
\begin{itemize}
\item {Proveniência:(De \textunderscore ditongo\textunderscore )}
\end{itemize}
Formar ditongo de.
Converter em ditongo.
\section{Ditongo}
\begin{itemize}
\item {Grp. gram.:m.}
\end{itemize}
\begin{itemize}
\item {Proveniência:(Gr. \textunderscore diphthongos\textunderscore )}
\end{itemize}
Reunião de vogaes, que se pronunciam com uma só emissão de voz e formam uma só sílaba.
\section{Dítono}
\begin{itemize}
\item {Grp. gram.:m.}
\end{itemize}
\begin{itemize}
\item {Proveniência:(Do gr. \textunderscore dis\textunderscore  + \textunderscore tonos\textunderscore )}
\end{itemize}
Intervallo de dois tons, em música.
\section{Ditosamente}
\begin{itemize}
\item {Grp. gram.:adv.}
\end{itemize}
Felizmente, de modo ditoso.
\section{Ditoso}
\begin{itemize}
\item {Grp. gram.:adj.}
\end{itemize}
Que tem dita; feliz; venturoso.
\section{Ditote}
\begin{itemize}
\item {Grp. gram.:m.}
\end{itemize}
\begin{itemize}
\item {Utilização:Fam.}
\end{itemize}
O mesmo que \textunderscore dichote\textunderscore .
\section{Ditriglifo}
\begin{itemize}
\item {Grp. gram.:m.}
\end{itemize}
\begin{itemize}
\item {Proveniência:(De \textunderscore di...\textunderscore  + \textunderscore triglifo\textunderscore )}
\end{itemize}
Espaço entre dois triglifos.
\section{Ditriglypho}
\begin{itemize}
\item {Grp. gram.:m.}
\end{itemize}
\begin{itemize}
\item {Proveniência:(De \textunderscore di...\textunderscore  + \textunderscore triglypho\textunderscore )}
\end{itemize}
Espaço entre dois triglyphos.
\section{Ditrocheu}
\begin{itemize}
\item {fónica:queu}
\end{itemize}
\begin{itemize}
\item {Grp. gram.:m.}
\end{itemize}
\begin{itemize}
\item {Proveniência:(De \textunderscore di...\textunderscore  + \textunderscore trocheu\textunderscore )}
\end{itemize}
Pé de verso grego ou latino, composto de dois trocheus.
\section{Ditroqueu}
\begin{itemize}
\item {Grp. gram.:m.}
\end{itemize}
\begin{itemize}
\item {Proveniência:(De \textunderscore di...\textunderscore  + \textunderscore troqueu\textunderscore )}
\end{itemize}
Pé de verso grego ou latino, composto de dois troqueus.
\section{Ditua}
\begin{itemize}
\item {Grp. gram.:f.}
\end{itemize}
Ave pernalta da África occidental.
\section{Diurese}
\begin{itemize}
\item {fónica:di-u}
\end{itemize}
\begin{itemize}
\item {Grp. gram.:f.}
\end{itemize}
\begin{itemize}
\item {Proveniência:(Gr. \textunderscore diouresis\textunderscore )}
\end{itemize}
Secreção copiosa de urina.
\section{Diurético}
\begin{itemize}
\item {Grp. gram.:adj.}
\end{itemize}
\begin{itemize}
\item {Grp. gram.:M.}
\end{itemize}
\begin{itemize}
\item {Proveniência:(Gr. \textunderscore diouretikos\textunderscore )}
\end{itemize}
Que facilita a secreção urinária.
Medicamento diurético.
\section{Diuretina}
\begin{itemize}
\item {Grp. gram.:f.}
\end{itemize}
Medicamento diurético, contra a hydropisia.
\section{Diurnal}
\begin{itemize}
\item {Grp. gram.:adj.}
\end{itemize}
\begin{itemize}
\item {Grp. gram.:M.}
\end{itemize}
\begin{itemize}
\item {Proveniência:(Lat. \textunderscore diurnalis\textunderscore )}
\end{itemize}
Diário.
Livro de orações para todos os dias.
\section{Diurno}
\begin{itemize}
\item {Grp. gram.:adj.}
\end{itemize}
\begin{itemize}
\item {Grp. gram.:M.}
\end{itemize}
\begin{itemize}
\item {Proveniência:(Lat. \textunderscore diurnus\textunderscore )}
\end{itemize}
Que se faz ou succede num dia.
Que se faz ou succede entre o sol nado e o sol posto: \textunderscore trabalho diurno\textunderscore .
Que só apparece de dia.
Espécie de breviario.
\section{Diuturnidade}
\begin{itemize}
\item {Grp. gram.:f.}
\end{itemize}
\begin{itemize}
\item {Proveniência:(Lat. \textunderscore diuturnitas\textunderscore )}
\end{itemize}
Largo espaço de tempo.
Duração longa.
\section{Diuturno}
\begin{itemize}
\item {Grp. gram.:adj.}
\end{itemize}
\begin{itemize}
\item {Proveniência:(Lat. \textunderscore diuturnus\textunderscore )}
\end{itemize}
Que vive muito tempo.
Que dura muito.
\section{Diva}
\begin{itemize}
\item {Grp. gram.:f.}
\end{itemize}
\begin{itemize}
\item {Utilização:Fig.}
\end{itemize}
\begin{itemize}
\item {Proveniência:(Lat. \textunderscore diva\textunderscore )}
\end{itemize}
Deusa.
Mulher formosa.
Epítheto de cantora notavel.
\section{Divagação}
\begin{itemize}
\item {Grp. gram.:f.}
\end{itemize}
Acto de divagar.
\section{Divagador}
\begin{itemize}
\item {Grp. gram.:m.}
\end{itemize}
Aquelle que divaga.
\section{Divagante}
\begin{itemize}
\item {Grp. gram.:adj.}
\end{itemize}
Que divaga.
\section{Divagar}
\begin{itemize}
\item {Grp. gram.:v. i.}
\end{itemize}
\begin{itemize}
\item {Proveniência:(Lat. \textunderscore divagari\textunderscore )}
\end{itemize}
Andar em differentes sentidos.
Andar ao acaso; vagabundear.
Desviar-se arbitrariamente do assumpto que se estava tratando.
Devanear.
\section{Divan}
\begin{itemize}
\item {Grp. gram.:m.}
\end{itemize}
\begin{itemize}
\item {Proveniência:(T. persa)}
\end{itemize}
Conselho de Estado, na Turquia.
Sala, onde funcciona êsse Conselho.
Sala luxuosa entre os Turcos.
Banco almofadado, espécie de sofá, sem encôsto.
Cancioneiro árabe.
\section{Divedo}
\begin{itemize}
\item {fónica:vê}
\end{itemize}
\begin{itemize}
\item {Grp. gram.:m.}
\end{itemize}
Preto, que faz feitiços.
Feiticeiro africano. Cf. Capello e Ivens, I, 283.
\section{Divelente}
\begin{itemize}
\item {Grp. gram.:adj.}
\end{itemize}
\begin{itemize}
\item {Utilização:Des.}
\end{itemize}
\begin{itemize}
\item {Proveniência:(Lat. \textunderscore divellens\textunderscore )}
\end{itemize}
Que separa, que desliga.
\section{Divellente}
\begin{itemize}
\item {Grp. gram.:adj.}
\end{itemize}
\begin{itemize}
\item {Utilização:Des.}
\end{itemize}
\begin{itemize}
\item {Proveniência:(Lat. \textunderscore divellens\textunderscore )}
\end{itemize}
Que separa, que desliga.
\section{Divergência}
\begin{itemize}
\item {Grp. gram.:f.}
\end{itemize}
\begin{itemize}
\item {Proveniência:(Lat. \textunderscore divergentia\textunderscore )}
\end{itemize}
Posição de duas linhas que se separam progressivamente.
Direcção dos raios, que, partindo do mesmo ponto, se separam progressivamente.
Discordância.
Acto de divergir.
\section{Divergente}
\begin{itemize}
\item {Grp. gram.:adj.}
\end{itemize}
\begin{itemize}
\item {Proveniência:(Lat. \textunderscore devergens\textunderscore )}
\end{itemize}
Que diverge.
Em que há divergência.
\section{Diverginérveo}
\begin{itemize}
\item {Grp. gram.:adj.}
\end{itemize}
\begin{itemize}
\item {Utilização:Bot.}
\end{itemize}
\begin{itemize}
\item {Proveniência:(De \textunderscore divergir\textunderscore  + \textunderscore nervo\textunderscore )}
\end{itemize}
Diz-se das fôlhas, cujas nervuras se dirigem, divergindo, da base para o ápice.
\section{Divergir}
\begin{itemize}
\item {Grp. gram.:v. i.}
\end{itemize}
\begin{itemize}
\item {Utilização:Fig.}
\end{itemize}
\begin{itemize}
\item {Proveniência:(Lat. \textunderscore devergere\textunderscore )}
\end{itemize}
Desviar-se.
Afastar-se progressivamente.
Não concordar.
\section{Divergivenoso}
\begin{itemize}
\item {Grp. gram.:adj.}
\end{itemize}
\begin{itemize}
\item {Proveniência:(Do lat. \textunderscore devergens\textunderscore  + \textunderscore vena\textunderscore )}
\end{itemize}
Diz-se, segundo alguns botânicos, das fôlhas divergentes, sem veias apparentes na superficie.
\section{Diversamente}
\begin{itemize}
\item {Grp. gram.:adv.}
\end{itemize}
De modo diverso.
\section{Diversão}
\begin{itemize}
\item {Grp. gram.:f.}
\end{itemize}
\begin{itemize}
\item {Proveniência:(Do lat. \textunderscore diversus\textunderscore )}
\end{itemize}
Acto de voltar para uma e outra parte.
Acto ou effeito de divertir.
Distracção.
Desvio.
\section{Diversicolor}
\begin{itemize}
\item {Grp. gram.:adj.}
\end{itemize}
Que tem diversas côres.
Variegado.
\section{Diversidade}
\begin{itemize}
\item {Grp. gram.:f.}
\end{itemize}
\begin{itemize}
\item {Proveniência:(Lat. \textunderscore diversitas\textunderscore )}
\end{itemize}
Qualidade daquelle ou daquillo que é diverso.
\section{Diversificação}
\begin{itemize}
\item {Grp. gram.:f.}
\end{itemize}
Acto ou effeito de diversificar.
\section{Diversificante}
\begin{itemize}
\item {Grp. gram.:adj.}
\end{itemize}
\begin{itemize}
\item {Proveniência:(Lat. \textunderscore diversificans\textunderscore )}
\end{itemize}
Que diversifica.
\section{Diversificar}
\begin{itemize}
\item {Grp. gram.:v. t.}
\end{itemize}
\begin{itemize}
\item {Grp. gram.:V. i.}
\end{itemize}
\begin{itemize}
\item {Proveniência:(Do lat. \textunderscore diversus\textunderscore  + \textunderscore facere\textunderscore )}
\end{itemize}
Tornar diverso.
Sêr diverso, variar.
\section{Diversificável}
\begin{itemize}
\item {Grp. gram.:adj.}
\end{itemize}
Que se póde diversificar.
\section{Diversifloro}
\begin{itemize}
\item {Grp. gram.:adj.}
\end{itemize}
\begin{itemize}
\item {Utilização:Bot.}
\end{itemize}
\begin{itemize}
\item {Proveniência:(De \textunderscore diverso\textunderscore  + \textunderscore flôr\textunderscore )}
\end{itemize}
Diz-se da inflorescência, em que as flôres do centro são regulares e as da circunferência irregulares.
\section{Diversivo}
\begin{itemize}
\item {Grp. gram.:adj.}
\end{itemize}
\begin{itemize}
\item {Proveniência:(De \textunderscore diverso\textunderscore )}
\end{itemize}
Em que há diversão.
Revulsivo.
\section{Diverso}
\begin{itemize}
\item {Grp. gram.:adj.}
\end{itemize}
\begin{itemize}
\item {Grp. gram.:Pl.}
\end{itemize}
\begin{itemize}
\item {Proveniência:(Lat. \textunderscore diversus\textunderscore )}
\end{itemize}
Que offerece vários aspectos.
Differente.
Alterado.
Discordante.
Vários; alguns: \textunderscore lê-se em diversos escritores...\textunderscore 
\section{Diversório}
\begin{itemize}
\item {Grp. gram.:adj.}
\end{itemize}
\begin{itemize}
\item {Grp. gram.:M.}
\end{itemize}
O mesmo que \textunderscore diversivo\textunderscore .
Aquillo que diverte.
Diversão. Cf. Vieira, VI, 175.
\section{Divertículo}
\begin{itemize}
\item {Grp. gram.:m.}
\end{itemize}
\begin{itemize}
\item {Proveniência:(Lat. \textunderscore diverticulum\textunderscore )}
\end{itemize}
Appêndice ôco, sem saída, como o que ás vezes apresenta o intestino delgado.
\section{Divertidamente}
\begin{itemize}
\item {Grp. gram.:adv.}
\end{itemize}
De modo divertido.
Alegremente.
\section{Divertido}
\begin{itemize}
\item {Grp. gram.:adj.}
\end{itemize}
\begin{itemize}
\item {Proveniência:(De \textunderscore divertir\textunderscore )}
\end{itemize}
Alegre; folgazão.
\section{Divertimento}
\begin{itemize}
\item {Grp. gram.:m.}
\end{itemize}
Acto de divertir.
Entretenimento.
Distracção; recreio.
\section{Divertir}
\begin{itemize}
\item {Grp. gram.:v. t.}
\end{itemize}
\begin{itemize}
\item {Grp. gram.:V. p.}
\end{itemize}
\begin{itemize}
\item {Utilização:Ant.}
\end{itemize}
\begin{itemize}
\item {Proveniência:(Lat. \textunderscore divertere\textunderscore )}
\end{itemize}
Desviar a attenção de.
Deshabituar.
Distrahir.
Despersuadir.
Recrear.

Afastar-se, desviar-se. Cf. Pant. de Aveiro, \textunderscore Itiner.\textunderscore , 95, (2.^a ed.).
\section{Divícia}
\begin{itemize}
\item {Grp. gram.:f.}
\end{itemize}
\begin{itemize}
\item {Utilização:Poét.}
\end{itemize}
\begin{itemize}
\item {Proveniência:(Lat. \textunderscore divitia\textunderscore )}
\end{itemize}
Riqueza. Cf. \textunderscore Lusíadas\textunderscore , VII, 8.
\section{Dívida}
\begin{itemize}
\item {Grp. gram.:f.}
\end{itemize}
\begin{itemize}
\item {Proveniência:(Do lat. \textunderscore debita\textunderscore )}
\end{itemize}
Aquillo que se deve.
Obrigação; dever moral.
\section{Dividendo}
\begin{itemize}
\item {Grp. gram.:adj.}
\end{itemize}
\begin{itemize}
\item {Grp. gram.:M.}
\end{itemize}
\begin{itemize}
\item {Proveniência:(Lat. \textunderscore dividendus\textunderscore )}
\end{itemize}
Que se há de dividir ou que se deve dividir.
Número, que se há de dividir.
Lucros de uma empresa, que se hão de dividir pelos societários ou accionistas.
Valores ou quantias que, numa liquidação commercial, competem a cada um dos interessados.
\section{Divididor}
\begin{itemize}
\item {Grp. gram.:m.  e  adj.}
\end{itemize}
\begin{itemize}
\item {Utilização:Des.}
\end{itemize}
(V.divisor)
\section{Dividimento}
\begin{itemize}
\item {Grp. gram.:m.}
\end{itemize}
\begin{itemize}
\item {Utilização:Des.}
\end{itemize}
\begin{itemize}
\item {Proveniência:(De \textunderscore dividir\textunderscore )}
\end{itemize}
Divisão.
Repartimento.
\section{Dividir}
\begin{itemize}
\item {Grp. gram.:v. t.}
\end{itemize}
\begin{itemize}
\item {Proveniência:(Lat. \textunderscore dividere\textunderscore )}
\end{itemize}
Separar de.
Separar em partes: \textunderscore dividir uma casa\textunderscore .
Repartir: \textunderscore dividir lucros\textunderscore .
Afastar.
Desligar.
Demarcar.
Desavir, pôr em discórdia.
Sulcar: \textunderscore a nau divide os mares\textunderscore .
\section{Dividivi}
\begin{itemize}
\item {Grp. gram.:m.}
\end{itemize}
Planta medicinal da Colômbia, (\textunderscore caesalpinia coriaria\textunderscore ).
\section{Dívido}
\begin{itemize}
\item {Grp. gram.:m.}
\end{itemize}
\begin{itemize}
\item {Utilização:Ant.}
\end{itemize}
Relações de grande amizade.
Parentesco.
(Cp. \textunderscore dívida\textunderscore )
\section{Dividual}
\begin{itemize}
\item {Grp. gram.:adj.}
\end{itemize}
\begin{itemize}
\item {Utilização:Des.}
\end{itemize}
O mesmo que \textunderscore dividundo\textunderscore .
\section{Dividundo}
\begin{itemize}
\item {Grp. gram.:adj.}
\end{itemize}
O mesmo que \textunderscore divíduo\textunderscore . Cf. Assis Teixeira, \textunderscore Águas\textunderscore , 185.
\section{Divíduo}
\begin{itemize}
\item {Grp. gram.:adj.}
\end{itemize}
\begin{itemize}
\item {Utilização:Gram.}
\end{itemize}
\begin{itemize}
\item {Proveniência:(Lat. \textunderscore dividuus\textunderscore )}
\end{itemize}
Divisível.
Diz-se do phonema, produzido pela expulsão do ar, após a separação súbita de dois órgãos factores, como nas consoantes \textunderscore p\textunderscore , \textunderscore b\textunderscore , \textunderscore t\textunderscore , \textunderscore d\textunderscore , \textunderscore k\textunderscore , \textunderscore g\textunderscore .
\section{Divina}
\textunderscore f. Loc. adv.\textunderscore  e \textunderscore pop\textunderscore . \textunderscore Á divina\textunderscore , sem um real: \textunderscore estou á divina\textunderscore .
\section{Divinação}
\begin{itemize}
\item {Grp. gram.:f.}
\end{itemize}
\begin{itemize}
\item {Proveniência:(Lat. \textunderscore divinatio\textunderscore )}
\end{itemize}
(V.adivinhação)
\section{Divinador}
\begin{itemize}
\item {Grp. gram.:m.}
\end{itemize}
\begin{itemize}
\item {Proveniência:(Lat. \textunderscore divinator\textunderscore )}
\end{itemize}
(V.adivinhador)
\section{Divinal}
\begin{itemize}
\item {Grp. gram.:adj.}
\end{itemize}
\begin{itemize}
\item {Proveniência:(Lat. \textunderscore divinalis\textunderscore )}
\end{itemize}
O mesmo que \textunderscore divino\textunderscore .
\section{Divinalmente}
\begin{itemize}
\item {Grp. gram.:adj.}
\end{itemize}
O mesmo que \textunderscore divinamente\textunderscore .
\section{Divinamente}
\begin{itemize}
\item {Grp. gram.:adv.}
\end{itemize}
De modo divino; magnificamente.
Esplendidamente.
Deliciosamente.
\section{Divinatório}
\begin{itemize}
\item {Grp. gram.:adj.}
\end{itemize}
\begin{itemize}
\item {Proveniência:(Do lat. \textunderscore divinatus\textunderscore )}
\end{itemize}
Relativo á adivinhação.
\section{Divinatriz}
\begin{itemize}
\item {Grp. gram.:adj. f.}
\end{itemize}
Que adivinha:«\textunderscore faculdade divinatriz\textunderscore ». Camillo, \textunderscore J. Christo\textunderscore , 133.
(Cp. lat. \textunderscore divinator\textunderscore )
\section{Divindade}
\begin{itemize}
\item {Grp. gram.:f.}
\end{itemize}
\begin{itemize}
\item {Utilização:Fig.}
\end{itemize}
\begin{itemize}
\item {Proveniência:(Lat. \textunderscore divinitas\textunderscore )}
\end{itemize}
Qualidade de quem ou daquillo que é divino.
Natureza divina.
Deus.
Coisa ou pessôa que se adora.
Mulher formosa.
\section{Divinhar}
\textunderscore v. t.\textunderscore  (e der.)
O mesmo que \textunderscore adivinhar\textunderscore , etc. Cf. Filinto, III, 164.
\section{Divinização}
\begin{itemize}
\item {Grp. gram.:f.}
\end{itemize}
Acto de divinizar.
\section{Divinizador}
\begin{itemize}
\item {Grp. gram.:adj.}
\end{itemize}
\begin{itemize}
\item {Grp. gram.:M.}
\end{itemize}
Que diviniza.
Aquelle que diviniza.
\section{Divinizante}
\begin{itemize}
\item {Grp. gram.:adj.}
\end{itemize}
Que diviniza.
\section{Divinizar}
\begin{itemize}
\item {Grp. gram.:v. t.}
\end{itemize}
\begin{itemize}
\item {Utilização:Fig.}
\end{itemize}
\begin{itemize}
\item {Proveniência:(De \textunderscore divino\textunderscore )}
\end{itemize}
Considerar divino.
Attribuir divindade a.
Elevar, exaltar.
Tornar sublime, adorável.
\section{Divino}
\begin{itemize}
\item {Grp. gram.:adj.}
\end{itemize}
\begin{itemize}
\item {Utilização:Fig.}
\end{itemize}
\begin{itemize}
\item {Grp. gram.:M.}
\end{itemize}
\begin{itemize}
\item {Proveniência:(Lat. \textunderscore divinus\textunderscore )}
\end{itemize}
Relativo a Deus.
Sobrenatural.
Sublime.
Perfeito.
Encantador.
Divindade.
Coisas sagradas: \textunderscore cantar lôas ao divino\textunderscore .
\section{Divisa}
\begin{itemize}
\item {Grp. gram.:f.}
\end{itemize}
\begin{itemize}
\item {Utilização:Ant.}
\end{itemize}
\begin{itemize}
\item {Proveniência:(Lat. \textunderscore divisa\textunderscore )}
\end{itemize}
Marca.
Raia.
Marco, sinal que divide.
Distintivo de cargo, de bandeira, de brazões, etc.
Emblema.
Sentença ou phrase, que symboliza a ideia ou sentimento de alguém, a norma de um partido, etc.: \textunderscore «pátria e liberdade»era a sua divisa\textunderscore .
Cada um dos galões que os officiaes do exército usam no braço.
Bens, herdados de pais e divididos pelos filhos.
\section{Divisadamente}
\begin{itemize}
\item {Grp. gram.:adv.}
\end{itemize}
\begin{itemize}
\item {Proveniência:(De \textunderscore divisar\textunderscore )}
\end{itemize}
Distintamente.
\section{Divisamente}
\begin{itemize}
\item {Grp. gram.:adv.}
\end{itemize}
\begin{itemize}
\item {Proveniência:(De \textunderscore diviso\textunderscore )}
\end{itemize}
Com divisão.
\section{Divisão}
\begin{itemize}
\item {Grp. gram.:f.}
\end{itemize}
\begin{itemize}
\item {Proveniência:(Lat. \textunderscore divisio\textunderscore )}
\end{itemize}
Acto ou effeito de dividir: \textunderscore a divisão de uma herança\textunderscore .
Fragmentação.
Cada uma das partes em que se divide um todo: \textunderscore uma casa com vinte divisões\textunderscore .
Linha de separação; estrema.
Porção.
Operação, com que se procura achar quantas vezes um número se contém noutro.
Parte de um exército, formado de brigadas.
Parte de uma esquadra, composta de vários navios de guerra.
Área de certas jurisdicções.
Classificação: \textunderscore a divisão do reino vegetal\textunderscore .
Partilha.
\section{Divisar}
\begin{itemize}
\item {Grp. gram.:v. t.}
\end{itemize}
\begin{itemize}
\item {Proveniência:(Do lat. \textunderscore divisus\textunderscore )}
\end{itemize}
Delimitar.
Marcar.
Avistar; distinguir.
Conhecer claramente.
\section{Diviseiro}
\begin{itemize}
\item {Grp. gram.:m.}
\end{itemize}
\begin{itemize}
\item {Utilização:Ant.}
\end{itemize}
\begin{itemize}
\item {Proveniência:(De \textunderscore divisar\textunderscore )}
\end{itemize}
Aquelle que fazia demarcações.
Avindor de pleitos numa behetria.
Aquelle que tinha parte nos bens da divisa.
Cp. \textunderscore divisa\textunderscore . Cf. \textunderscore Port. Mon. Hist., Script.\textunderscore , 262.
\section{Divisibilidade}
\begin{itemize}
\item {Grp. gram.:f.}
\end{itemize}
Qualidade daquillo que é divisível.
\section{Divisional}
\begin{itemize}
\item {Grp. gram.:adj.}
\end{itemize}
\begin{itemize}
\item {Proveniência:(Do lat. \textunderscore divisio\textunderscore )}
\end{itemize}
Relativo a divisão.
\section{Divisionário}
\begin{itemize}
\item {Grp. gram.:adj.}
\end{itemize}
\begin{itemize}
\item {Utilização:Bras}
\end{itemize}
\begin{itemize}
\item {Proveniência:(Do lat. \textunderscore divisio\textunderscore )}
\end{itemize}
Relativo a divisão militar.
Diz-se da moéda secundária, para trocos.
\section{Divisível}
\begin{itemize}
\item {Grp. gram.:adj.}
\end{itemize}
\begin{itemize}
\item {Proveniência:(Lat. \textunderscore divisibilis\textunderscore )}
\end{itemize}
Que se póde dividir.
\section{Divisivo}
\begin{itemize}
\item {Grp. gram.:adj.}
\end{itemize}
\begin{itemize}
\item {Utilização:P. us.}
\end{itemize}
\begin{itemize}
\item {Proveniência:(Lat. \textunderscore divisivus\textunderscore )}
\end{itemize}
O mesmo que \textunderscore divisível\textunderscore .
\section{Diviso}
\begin{itemize}
\item {Grp. gram.:adj.}
\end{itemize}
\begin{itemize}
\item {Proveniência:(Lat. \textunderscore divisus\textunderscore )}
\end{itemize}
Que se dividiu: \textunderscore aquelle casal ficou diviso\textunderscore .
\section{Divisor}
\begin{itemize}
\item {Grp. gram.:adj.}
\end{itemize}
\begin{itemize}
\item {Grp. gram.:M.}
\end{itemize}
\begin{itemize}
\item {Proveniência:(Lat. \textunderscore divisor\textunderscore )}
\end{itemize}
Que divide.
Número, pelo qual se divide outro.
\section{Divisória}
\begin{itemize}
\item {Grp. gram.:f.}
\end{itemize}
\begin{itemize}
\item {Proveniência:(De \textunderscore divisório\textunderscore )}
\end{itemize}
Linha, que divide ou separa objectos ou compartimentos próximos.
Tapume, parede ou biombo, que divide uma casa ou um compartimento.
\section{Divisório}
\begin{itemize}
\item {Grp. gram.:adj.}
\end{itemize}
\begin{itemize}
\item {Grp. gram.:M.}
\end{itemize}
\begin{itemize}
\item {Proveniência:(Do lat. \textunderscore divisus\textunderscore )}
\end{itemize}
Que divide.
Que serve para delimitar.
Relativo a divisão.
Peça de madeira, em que o compositor typográphico fixa o original com um ou dois mordentes.
\section{Divo}
\begin{itemize}
\item {Grp. gram.:adj.}
\end{itemize}
\begin{itemize}
\item {Grp. gram.:M.}
\end{itemize}
\begin{itemize}
\item {Proveniência:(Lat. \textunderscore divus\textunderscore )}
\end{itemize}
Divino.
Homem divinizado.
Deus.
\section{Divodignos}
\begin{itemize}
\item {Grp. gram.:m. pl.}
\end{itemize}
Sociedade secreta, que havia em Coímbra, e a que pertencíam os estudantes que, em 1828, assassinaram os Lentes que iam em commissão apresentar-se a D. Miguel.
\section{Divorciar}
\begin{itemize}
\item {Grp. gram.:v. t.}
\end{itemize}
\begin{itemize}
\item {Utilização:Fig.}
\end{itemize}
Determinar o divórcio de.
Separar, desunir.
(B. lat. \textunderscore divortiare\textunderscore )
\section{Divórcio}
\begin{itemize}
\item {Grp. gram.:m.}
\end{itemize}
\begin{itemize}
\item {Utilização:Fig.}
\end{itemize}
\begin{itemize}
\item {Proveniência:(Lat. \textunderscore divorcium\textunderscore )}
\end{itemize}
Dissolução judicial do casamento.
Desunião, separação.
\section{Divulgação}
\begin{itemize}
\item {Grp. gram.:f.}
\end{itemize}
\begin{itemize}
\item {Proveniência:(Lat. \textunderscore divulgatio\textunderscore )}
\end{itemize}
Acto ou effeito de divulgar.
\section{Divulgador}
\begin{itemize}
\item {Grp. gram.:adj.}
\end{itemize}
\begin{itemize}
\item {Grp. gram.:M.}
\end{itemize}
\begin{itemize}
\item {Proveniência:(Lat. \textunderscore divulgator\textunderscore )}
\end{itemize}
Que divulga.
Aquelle que divulga.
\section{Divulgar}
\begin{itemize}
\item {Grp. gram.:v. t.}
\end{itemize}
\begin{itemize}
\item {Proveniência:(Lat. \textunderscore divulgare\textunderscore )}
\end{itemize}
Tornar conhecido do público: \textunderscore divulgar um boato\textunderscore .
Publicar; propagar: \textunderscore divulgar um opúsculo\textunderscore .
\section{Divulsão}
\begin{itemize}
\item {Grp. gram.:f.}
\end{itemize}
\begin{itemize}
\item {Proveniência:(Lat. \textunderscore divulsio\textunderscore )}
\end{itemize}
Acto de separar com violência.
Acto de rasgar.
\section{Dixe}
\begin{itemize}
\item {Grp. gram.:m.}
\end{itemize}
Ornamento de oiro ou pedraria.
Jóia.
Enfeite.
Pequeno objecto para brinquedo.
(Cast. \textunderscore dije\textunderscore )
\section{Díxeme-díxeme}
\begin{itemize}
\item {Grp. gram.:m.}
\end{itemize}
\begin{itemize}
\item {Utilização:Pop.}
\end{itemize}
Mexerico:«\textunderscore tudo é díxeme-díxeme, andar espreitando\textunderscore ». \textunderscore Eufrosina\textunderscore , 97.
(Fórma ant. de \textunderscore disse-me\textunderscore  + \textunderscore disse-me\textunderscore )
\section{Díxemes}
\begin{itemize}
\item {Grp. gram.:m. pl.}
\end{itemize}
\begin{itemize}
\item {Utilização:Pop.}
\end{itemize}
O mesmo que \textunderscore díxeme-díxeme\textunderscore .
\section{Dizedor}
\begin{itemize}
\item {Grp. gram.:m.  e  adj.}
\end{itemize}
\begin{itemize}
\item {Proveniência:(De \textunderscore dizer\textunderscore )}
\end{itemize}
Falador.
Indivíduo que conta anecdotas.
Gracejador.
O que fala muito.
\section{Dizente}
(fórma ant. do gerúndio de \textunderscore dizer\textunderscore , hoje substituída por \textunderscore dizendo\textunderscore )
\section{Dizer}
\begin{itemize}
\item {Grp. gram.:v. t.}
\end{itemize}
\begin{itemize}
\item {Grp. gram.:V. i.}
\end{itemize}
\begin{itemize}
\item {Grp. gram.:V. p.}
\end{itemize}
\begin{itemize}
\item {Grp. gram.:M.}
\end{itemize}
\begin{itemize}
\item {Proveniência:(Lat. \textunderscore dicere\textunderscore )}
\end{itemize}
Expor, exprimir, por meio de palavras.
Referir: \textunderscore dizer histórias\textunderscore .
Designar.
Enunciar por escrito.
Recitar.
Rezar: \textunderscore dizer uma oração\textunderscore .
Estar tentado a crer.
Significar: \textunderscore isso não diz ao caso\textunderscore .
Lêr.
Exprimir por música, tocando ou cantando.
Afirmar.
Explicar.
Confessar.
Bradar.
Pensar.
Indicar.
Aconselhar: \textunderscore digo-te que não cases\textunderscore .
Denominar.
Notar.
Falar.
Fazer allegações.
Quadrar; sêr apropriado; conformar-se:«\textunderscore vêde como diz o estilo de pregar no céu\textunderscore ». Vieira.
\textunderscore Dizer em crédito de\textunderscore , abonar:«\textunderscore o successo diz em crédito do moço\textunderscore ». Camillo.
Têr como nome; chamar-se: \textunderscore um patife, que se diz Leal\textunderscore .
Referir-se: \textunderscore diz-se que o Govêrno cai\textunderscore .
Dar como pretexto.
Dito, expressão: \textunderscore no dizer dos sábios\textunderscore .
Linguagem falada: \textunderscore correcto no dizer\textunderscore .
Maneira de exprimir, estilo.
\section{Dize-tu, direi-eu}
\begin{itemize}
\item {Grp. gram.:m.}
\end{itemize}
Disputa acalorada.
Altercação, em que os dois contendores falam quási ao mesmo tempo.
\section{Dízima}
\begin{itemize}
\item {Grp. gram.:f.}
\end{itemize}
\begin{itemize}
\item {Utilização:Ext.}
\end{itemize}
\begin{itemize}
\item {Proveniência:(Lat. \textunderscore decima\textunderscore )}
\end{itemize}
Contribuição, equivalente á décima parte de um rendimento.
Décima.
Imposto.
Fracção decimal, que resulta de uma fracção ordinária.
\section{Dizimação}
\begin{itemize}
\item {Grp. gram.:f.}
\end{itemize}
Acto de dizimar.
\section{Dizimador}
\begin{itemize}
\item {Grp. gram.:adj.}
\end{itemize}
\begin{itemize}
\item {Grp. gram.:M.}
\end{itemize}
Que dizima.
Aquelle que dizima.
\section{Dizimar}
\begin{itemize}
\item {Grp. gram.:v. t.}
\end{itemize}
\begin{itemize}
\item {Utilização:Fig.}
\end{itemize}
\begin{itemize}
\item {Proveniência:(Do lat. \textunderscore decimare\textunderscore )}
\end{itemize}
Matar (um soldado), em cada grupo ou número de dêz.
Lançar imposto da dízima sôbre.
Destruir parte de: \textunderscore Dizimar um exército\textunderscore .
Destruir.
Desfalcar.
Deminuir.
Arruinar.
Tornar raro.
\section{Dizimaria}
\begin{itemize}
\item {Grp. gram.:f.}
\end{itemize}
\begin{itemize}
\item {Utilização:Des.}
\end{itemize}
\begin{itemize}
\item {Proveniência:(De \textunderscore dizimar\textunderscore )}
\end{itemize}
Lugar, onde se depositava o imposto da dízima.
\section{Dizimeiro}
\begin{itemize}
\item {Grp. gram.:m.}
\end{itemize}
\begin{itemize}
\item {Proveniência:(De \textunderscore dízimo\textunderscore )}
\end{itemize}
Cobrador da dízima ou dos dízimos.
\section{Dizimento}
\begin{itemize}
\item {Grp. gram.:m.}
\end{itemize}
\begin{itemize}
\item {Utilização:Ant.}
\end{itemize}
Acto de dizer.
\section{Dízimo}
\begin{itemize}
\item {Grp. gram.:adj.}
\end{itemize}
\begin{itemize}
\item {Grp. gram.:M.}
\end{itemize}
\begin{itemize}
\item {Utilização:Prov.}
\end{itemize}
\begin{itemize}
\item {Utilização:minh.}
\end{itemize}
\begin{itemize}
\item {Proveniência:(Do lat. \textunderscore decimus\textunderscore )}
\end{itemize}
Décimo.
A décima parte.
Contribuição, que se pagava á Igreja e que consistia na décima parte dos frutos recolhidos.
Imposto do pescado, cobrado pela Guarda-Fiscal.
\section{Dizível}
\begin{itemize}
\item {Grp. gram.:adj.}
\end{itemize}
\begin{itemize}
\item {Proveniência:(Lat. \textunderscore dicibilis\textunderscore )}
\end{itemize}
Que se póde dizer.
\section{Dizombole}
\begin{itemize}
\item {Grp. gram.:m.}
\end{itemize}
Arbusto africano, da fam. das leguminosas, de fôlhas amplas, pecioladas, e flôres miúdas, côr de palha.
\section{Dizonho}
\begin{itemize}
\item {Grp. gram.:adj.}
\end{itemize}
\begin{itemize}
\item {Utilização:Pop.}
\end{itemize}
\begin{itemize}
\item {Proveniência:(De \textunderscore dizer\textunderscore )}
\end{itemize}
O mesmo que \textunderscore respondão\textunderscore .
\section{Do}
(contr. de \textunderscore de\textunderscore  + \textunderscore o\textunderscore )
\section{Do}
\begin{itemize}
\item {Grp. gram.:adv.}
\end{itemize}
\begin{itemize}
\item {Utilização:Ant.}
\end{itemize}
O mesmo que \textunderscore onde\textunderscore :«\textunderscore por do vás, como vires, assim faz.\textunderscore »\textunderscore Eufrosina\textunderscore , 305.
\section{Dó}
\begin{itemize}
\item {Grp. gram.:m.}
\end{itemize}
Commiseração.
Compaixão, lástima: \textunderscore tive dó do pequeno\textunderscore .
Tristeza.
Luto: \textunderscore vestido de dó\textunderscore .
(Cast. \textunderscore duelo\textunderscore , provavelmente do lat. \textunderscore dolus\textunderscore )
\section{Dó}
\begin{itemize}
\item {Grp. gram.:m.}
\end{itemize}
Primeira nota da moderna escala musical.
Quarta corda do violoncello; quarta corda da violeta.
(Sýllaba adoptada pelos italianos e depois por todos os mestres de canto, em substituição da \textunderscore sýllaba ut\textunderscore , que era a primeira nota da antiga escala musical)
\section{Doação}
\begin{itemize}
\item {Grp. gram.:f.}
\end{itemize}
\begin{itemize}
\item {Proveniência:(Do lat. \textunderscore donatio\textunderscore )}
\end{itemize}
Acto ou effeito de doar.
Aquillo que se doou.
Documento, que assegura e legaliza a doação.
\section{Doador}
\begin{itemize}
\item {Grp. gram.:m.}
\end{itemize}
\begin{itemize}
\item {Proveniência:(Do lat. \textunderscore donator\textunderscore )}
\end{itemize}
Aquelle que doa ou faz doação.
\section{Doairo}
\begin{itemize}
\item {Grp. gram.:m.}
\end{itemize}
\begin{itemize}
\item {Utilização:Prov.}
\end{itemize}
\begin{itemize}
\item {Utilização:alent.}
\end{itemize}
\begin{itemize}
\item {Utilização:beir.}
\end{itemize}
\begin{itemize}
\item {Utilização:Ant.}
\end{itemize}
Modo, jeito, ademanes.
Semblante.
Parecença.
Donaire.
(Cp. \textunderscore donaire\textunderscore )
\section{Doar}
\begin{itemize}
\item {Grp. gram.:v. t.}
\end{itemize}
\begin{itemize}
\item {Proveniência:(Do lat. \textunderscore donare\textunderscore )}
\end{itemize}
Transmittir gratuitamente a outrem (bens, presentes).
\section{Dôas}
\begin{itemize}
\item {Grp. gram.:f. pl.}
\end{itemize}
\begin{itemize}
\item {Utilização:Ant.}
\end{itemize}
\begin{itemize}
\item {Proveniência:(De \textunderscore doar\textunderscore )}
\end{itemize}
O mesmo que [[dons|dom:1]], \textunderscore donativos\textunderscore , \textunderscore presentes\textunderscore .
\section{Dobadeira}
\begin{itemize}
\item {Grp. gram.:f.}
\end{itemize}
Mulher que doba.
\section{Dobadoira}
\begin{itemize}
\item {Grp. gram.:f.}
\end{itemize}
\begin{itemize}
\item {Utilização:Fam.}
\end{itemize}
\begin{itemize}
\item {Utilização:T. da Ericeira}
\end{itemize}
\begin{itemize}
\item {Utilização:Prov.}
\end{itemize}
\begin{itemize}
\item {Utilização:beir.}
\end{itemize}
\begin{itemize}
\item {Utilização:Prov.}
\end{itemize}
\begin{itemize}
\item {Utilização:trasm.}
\end{itemize}
\begin{itemize}
\item {Utilização:fam.}
\end{itemize}
\begin{itemize}
\item {Proveniência:(De \textunderscore dobar\textunderscore )}
\end{itemize}
Apparelho, com que se doba.
Azáfama, roda viva: \textunderscore andar numa dobadoira\textunderscore .
Astéria ou estrêlla-do-mar.
O mesmo que \textunderscore urdideira\textunderscore . (Colhido na Guarda)
\textunderscore Dobadoira sem pés\textunderscore , pessôa leviana, sem juizo.
\section{Dobadoura}
\begin{itemize}
\item {Grp. gram.:f.}
\end{itemize}
\begin{itemize}
\item {Utilização:Fam.}
\end{itemize}
\begin{itemize}
\item {Utilização:T. da Ericeira}
\end{itemize}
\begin{itemize}
\item {Utilização:Prov.}
\end{itemize}
\begin{itemize}
\item {Utilização:beir.}
\end{itemize}
\begin{itemize}
\item {Utilização:Prov.}
\end{itemize}
\begin{itemize}
\item {Utilização:trasm.}
\end{itemize}
\begin{itemize}
\item {Utilização:fam.}
\end{itemize}
\begin{itemize}
\item {Proveniência:(De \textunderscore dobar\textunderscore )}
\end{itemize}
Apparelho, com que se doba.
Azáfama, roda viva: \textunderscore andar numa dobadoura\textunderscore .
Astéria ou estrêlla-do-mar.
O mesmo que \textunderscore urdideira\textunderscore . (Colhido na Guarda)
\textunderscore Dobadoira sem pés\textunderscore , pessôa leviana, sem juizo.
\section{Dobagem}
\begin{itemize}
\item {Grp. gram.:f.}
\end{itemize}
Acto de dobar.
Officina, onde se doba, nas fábricas de fiação.
\section{Dobar}
\begin{itemize}
\item {Grp. gram.:v. t.}
\end{itemize}
\begin{itemize}
\item {Utilização:Fig.}
\end{itemize}
\begin{itemize}
\item {Grp. gram.:V. i.}
\end{itemize}
\begin{itemize}
\item {Proveniência:(Do lat. \textunderscore depanare\textunderscore )}
\end{itemize}
Ennovelar, enrolar em novelos, (o fio da meada), com ou sem auxílio de dobadoira.
Voltear.
Fazer novelos.
\section{Doble}
\begin{itemize}
\item {Grp. gram.:adj.}
\end{itemize}
O mesmo que \textunderscore dobre\textunderscore .
Fingido; velhaco.
\section{Doblete}
\begin{itemize}
\item {fónica:blê}
\end{itemize}
\begin{itemize}
\item {Grp. gram.:m.}
\end{itemize}
\begin{itemize}
\item {Proveniência:(De \textunderscore doble\textunderscore )}
\end{itemize}
Pedaço de vidro, que imita pedra preciosa.
\section{Doblez}
\begin{itemize}
\item {Grp. gram.:f.}
\end{itemize}
O mesmo que \textunderscore dobrez\textunderscore .
\section{Dobra}
\begin{itemize}
\item {Grp. gram.:f.}
\end{itemize}
\begin{itemize}
\item {Utilização:Fig.}
\end{itemize}
\begin{itemize}
\item {Proveniência:(De \textunderscore dobrar\textunderscore )}
\end{itemize}
Parte de um objecto, que voltada, fica sobreposta a outra.
Vinco; prega.
Coisa que envolve, enrolando-se.
Nome de várias moédas antigas.
\section{Dobrada}
\begin{itemize}
\item {Grp. gram.:f.}
\end{itemize}
\begin{itemize}
\item {Utilização:Bras}
\end{itemize}
\begin{itemize}
\item {Proveniência:(De \textunderscore dobrar\textunderscore )}
\end{itemize}
Parte das visceras do boi ou vaca, para guisar.
Guisado feito com essas vísceras.
Ondulação do terreno; quebrada.
\section{Dobradamente}
\begin{itemize}
\item {Grp. gram.:adv.}
\end{itemize}
\begin{itemize}
\item {Proveniência:(De \textunderscore dobrar\textunderscore )}
\end{itemize}
Em dôbro; duplamente.
\section{Dobradeira}
\begin{itemize}
\item {Grp. gram.:f.}
\end{itemize}
\begin{itemize}
\item {Proveniência:(De \textunderscore dobrar\textunderscore )}
\end{itemize}
Instrumento de encadernador, para dobrar fôlhas e capas de livros, antes de batidas e cosidas.
Mulher, que dobra fôlhas impressas, que se hão de brochar ou encadernar.
Apparelho para dobrar, nas fábricas de fiação e tecidos.
\section{Dobradiça}
\begin{itemize}
\item {Grp. gram.:f.}
\end{itemize}
\begin{itemize}
\item {Proveniência:(De \textunderscore dobradiço\textunderscore )}
\end{itemize}
Peça de metal, formada de duas chapas, unidas por um eixo commum, e sôbre a qual gira a janela, porta, etc.
Bisagra, gonzo.
Charneira.
Tábua móvel, que nas plateias se estende de um a outro banco, de uma a outra cadeira, ou de uma cadeira á parede, formando assento supplementar.
\section{Dobradiço}
\begin{itemize}
\item {Grp. gram.:adj.}
\end{itemize}
\begin{itemize}
\item {Proveniência:(De \textunderscore dobrar\textunderscore )}
\end{itemize}
Flexível; que se dobra facilmente: \textunderscore vara, dobradiça\textunderscore .
\section{Dobrado}
\begin{itemize}
\item {Grp. gram.:adj.}
\end{itemize}
\begin{itemize}
\item {Grp. gram.:M.}
\end{itemize}
\begin{itemize}
\item {Proveniência:(De \textunderscore dobrar\textunderscore )}
\end{itemize}
O mesmo que \textunderscore duplicado\textunderscore .
Enrolado.
Marcha militar.
Alteação do cantar dos pássaros.
\section{Dobradura}
\begin{itemize}
\item {Grp. gram.:f.}
\end{itemize}
\begin{itemize}
\item {Utilização:Ant.}
\end{itemize}
\begin{itemize}
\item {Proveniência:(De \textunderscore dobrar\textunderscore )}
\end{itemize}
Acto ou effeito de dobrar.
Dobrez, deslealdade, fingimento.
\section{Dobral}
\begin{itemize}
\item {Grp. gram.:m.}
\end{itemize}
\begin{itemize}
\item {Utilização:Ant.}
\end{itemize}
\begin{itemize}
\item {Proveniência:(De \textunderscore dobra\textunderscore )}
\end{itemize}
Rôlo.
Carteira.
Bôlsa para dinheiro.
\section{Dobramento}
\begin{itemize}
\item {Grp. gram.:m.}
\end{itemize}
O mesmo que \textunderscore dobradura\textunderscore .
\section{Dobrão}
\begin{itemize}
\item {Grp. gram.:m.}
\end{itemize}
\begin{itemize}
\item {Utilização:Bras. do N}
\end{itemize}
\begin{itemize}
\item {Proveniência:(De \textunderscore dobra\textunderscore )}
\end{itemize}
Antiga moéda de oiro, que valia 24$000 reis.
Moéda de cobre, do valor de 40 reis.
\section{Dobrar}
\begin{itemize}
\item {Grp. gram.:v. t.}
\end{itemize}
\begin{itemize}
\item {Grp. gram.:V. i.}
\end{itemize}
\begin{itemize}
\item {Utilização:Fig.}
\end{itemize}
\begin{itemize}
\item {Utilização:Prov.}
\end{itemize}
\begin{itemize}
\item {Utilização:trasm.}
\end{itemize}
\begin{itemize}
\item {Proveniência:(Do b. lat. \textunderscore duplare\textunderscore )}
\end{itemize}
Multiplicar por dois.
Tornar duas vezes maior.
Duplicar.
Fazer dobras em.
Augmentar, multiplicar.
Enrolar; entroixar.
Curvar: \textunderscore dobrar a cerviz\textunderscore .
Abatar; domar.
Modificar; abrandar: \textunderscore dobrar o mau gênio de alguem\textunderscore .
Passar além de, torneando: \textunderscore dobrar uma esquina\textunderscore .
Fazer dar volta a (o sino).
Curvar-se.
Aumentar.
Dar voltas a (um sino).
Ceder.
Galopar.
\section{Dobrável}
\begin{itemize}
\item {Grp. gram.:adj.}
\end{itemize}
Que se póde dobrar.
\section{Dobre}
\begin{itemize}
\item {Grp. gram.:adj.}
\end{itemize}
\begin{itemize}
\item {Utilização:Fig.}
\end{itemize}
\begin{itemize}
\item {Grp. gram.:M.}
\end{itemize}
Dobrado, duplicado.
Fingido.
Que revela differentes sentimentos, segundo os fins que mais lhe convêm.
Acto de dobrar os sinos.
Repetição da mesma palavra ou fórmula em certos lugares de uma estrophe. Ch. C. Michaëlis, \textunderscore Geschichte der Port. Litter.\textunderscore , 196.
(Cp. \textunderscore dôbro\textunderscore )
\section{Dobrez}
\begin{itemize}
\item {Grp. gram.:f.}
\end{itemize}
\begin{itemize}
\item {Proveniência:(De \textunderscore dobre\textunderscore )}
\end{itemize}
Falta de sinceridade.
Fingimento.
Qualidade de quem é dobre.
\section{Dôbro}
\begin{itemize}
\item {Grp. gram.:m.}
\end{itemize}
\begin{itemize}
\item {Utilização:Bras. do S}
\end{itemize}
\begin{itemize}
\item {Proveniência:(Do lat. \textunderscore duplus\textunderscore )}
\end{itemize}
Duplicação de uma coisa; duplo.
Repetição de uma quantidade.
Producto de uma quantidade ou número duplicado por dois.
Pequeno volume, que se põe em cima da carga dos tropeiros.
\section{Doca}
\begin{itemize}
\item {Grp. gram.:f.}
\end{itemize}
\begin{itemize}
\item {Proveniência:(Do ingl. \textunderscore dock\textunderscore )}
\end{itemize}
Parte de um pôrto, ladeado de muros ou caes, na qual se abrigam os navios, e onde tomam ou deixam carga.
Dique; estaleiro.
\section{Doca}
\begin{itemize}
\item {Grp. gram.:adj.}
\end{itemize}
\begin{itemize}
\item {Utilização:Bras}
\end{itemize}
Cego de um ôlho.
\section{Doçaína}
\begin{itemize}
\item {Grp. gram.:f.}
\end{itemize}
Espécie de charamela grande, que se usou desde o século XII ao XVII:«\textunderscore ...alguns instrumentos músicos, em que entravam duas doçaínas\textunderscore ». \textunderscore Peregrinação\textunderscore , 251.
\section{Doçaínha}
\begin{itemize}
\item {Grp. gram.:f.}
\end{itemize}
O mesmo que \textunderscore doçaína\textunderscore . Cf. \textunderscore Peregrinação\textunderscore , LXIX.
\section{Doçal}
\begin{itemize}
\item {Grp. gram.:m.  e  adj.}
\end{itemize}
\begin{itemize}
\item {Proveniência:(De \textunderscore doce\textunderscore )}
\end{itemize}
Casta de uva preta.
\section{Doçal-branco}
\begin{itemize}
\item {Grp. gram.:m.}
\end{itemize}
Casta de uva de Monção.
\section{Docar}
\begin{itemize}
\item {Grp. gram.:m.}
\end{itemize}
\begin{itemize}
\item {Proveniência:(Do ingl. \textunderscore dog-cart\textunderscore , carro de cães)}
\end{itemize}
Espécie de carruagem moderna, de duas rodas.
\section{Doçar}
\begin{itemize}
\item {Grp. gram.:adj.}
\end{itemize}
\begin{itemize}
\item {Proveniência:(De \textunderscore doce\textunderscore )}
\end{itemize}
Diz-se de uma variedade de pêra.
E diz-se de uma casta de uva preta, o mesmo que \textunderscore doçal\textunderscore .
\section{Doçaria}
\begin{itemize}
\item {Grp. gram.:f.}
\end{itemize}
Lugar, onde se vende ou fabríca doce.
Grande porção de doce.
\section{Doce}
\begin{itemize}
\item {Grp. gram.:adj.}
\end{itemize}
\begin{itemize}
\item {Utilização:Náut.}
\end{itemize}
\begin{itemize}
\item {Grp. gram.:Adv.}
\end{itemize}
\begin{itemize}
\item {Utilização:Náut.}
\end{itemize}
\begin{itemize}
\item {Grp. gram.:M.}
\end{itemize}
\begin{itemize}
\item {Utilização:Prov.}
\end{itemize}
\begin{itemize}
\item {Utilização:Bras}
\end{itemize}
\begin{itemize}
\item {Proveniência:(Do lat. \textunderscore dulcis\textunderscore )}
\end{itemize}
Que tem sabor agradável; que agrada ao paladar.
Que tem sabor mais ou menos semelhante ao do açúcar ou do mel.
Que não é salgado.
Temperado ou misturado com açúcar ou com outra substância saccharina.
Que exerce nos sentidos impressão agradável.
Meigo, affectuoso: \textunderscore palavras doces\textunderscore .
Suave: \textunderscore uma doce canção\textunderscore .
Que não é escabroso, que não é fatigante.
Benigno.
Ditoso.
Que agrada ao espírito.
Que encanta.
\textunderscore Doce de borda\textunderscore , diz-se o navio que facilmente se inclina para sotavento.
Suavemente.
\textunderscore Içar doce\textunderscore , içar sem custo.
Aquillo que é doce.
Confeição culinária, em que entra açúcar, ou em que é elemento principal o açúcar ou outra substância saccharina.
Açúcar.
\section{Doce-amarga}
\begin{itemize}
\item {Grp. gram.:f.}
\end{itemize}
O mesmo que \textunderscore dulcamara\textunderscore .
\section{Doce-crasto}
\begin{itemize}
\item {Grp. gram.:m.}
\end{itemize}
Variedade de maçan.
\section{Doce-de-pimenta}
\begin{itemize}
\item {Grp. gram.:m.}
\end{itemize}
\begin{itemize}
\item {Utilização:Bras. do N}
\end{itemize}
O mesmo que \textunderscore fruita\textunderscore .
\section{Doceira}
\begin{itemize}
\item {Grp. gram.:f.}
\end{itemize}
\begin{itemize}
\item {Proveniência:(De \textunderscore doceiro\textunderscore )}
\end{itemize}
Mulher, que faz ou vende doces.
Confeiteira.
\section{Doceiro}
\begin{itemize}
\item {Grp. gram.:m.}
\end{itemize}
\begin{itemize}
\item {Proveniência:(Do b. lat. \textunderscore dulciarius\textunderscore )}
\end{itemize}
Aquelle que faz ou vende doces.
Aquelle que tem confeitaria.
\section{Docel}
\begin{itemize}
\item {Grp. gram.:m.}
\end{itemize}
Fórma usual, mas incorrecta, por \textunderscore dossel\textunderscore .(V.dossel)
\section{Docemente}
\begin{itemize}
\item {fónica:dô}
\end{itemize}
\begin{itemize}
\item {Grp. gram.:adv.}
\end{itemize}
\begin{itemize}
\item {Proveniência:(De \textunderscore doce\textunderscore )}
\end{itemize}
Com doçura.
Suavemente.
Brandamente.
Agradavelmente.
\section{Docente}
\begin{itemize}
\item {Grp. gram.:adj.}
\end{itemize}
\begin{itemize}
\item {Proveniência:(Lat. \textunderscore docens\textunderscore )}
\end{itemize}
Que ensina.
Relativo a professores: \textunderscore o pessoal docente do lyceu\textunderscore .
\section{Docidão}
\begin{itemize}
\item {Grp. gram.:f.}
\end{itemize}
\begin{itemize}
\item {Utilização:Ant.}
\end{itemize}
\begin{itemize}
\item {Proveniência:(Do lat. \textunderscore dulcedo\textunderscore )}
\end{itemize}
O mesmo que \textunderscore doçura\textunderscore .
\section{Dócil}
\begin{itemize}
\item {Grp. gram.:adj.}
\end{itemize}
\begin{itemize}
\item {Proveniência:(Lat. \textunderscore docilis\textunderscore )}
\end{itemize}
Que se submete ao ensino: \textunderscore cavallo dócil\textunderscore .
Submisso.
Obediente: \textunderscore criança dócil\textunderscore .
Que se move ou se curva facilmente.
Que se póde lavrar ou obrar com facilidade.
Flexível.
\section{Docilidade}
\begin{itemize}
\item {Grp. gram.:f.}
\end{itemize}
\begin{itemize}
\item {Proveniência:(Lat. \textunderscore docilitas\textunderscore )}
\end{itemize}
Qualidade daquelle ou daquillo que é dócil.
\section{Docilizar}
\begin{itemize}
\item {Grp. gram.:v. t.}
\end{itemize}
Tornar dócil.
\section{Docíllimo}
\begin{itemize}
\item {Grp. gram.:adj.}
\end{itemize}
Muito dócil.
\section{Docilmente}
\begin{itemize}
\item {fónica:dó}
\end{itemize}
\begin{itemize}
\item {Grp. gram.:adv.}
\end{itemize}
\begin{itemize}
\item {Proveniência:(De \textunderscore dócil\textunderscore )}
\end{itemize}
Com docilidade.
\section{Docimásia}
\begin{itemize}
\item {Grp. gram.:f.}
\end{itemize}
\begin{itemize}
\item {Proveniência:(Gr. \textunderscore dokimasia\textunderscore )}
\end{itemize}
Determinação da proporção, em que os metaes se contêm nos minérios ou em productos artificiaes.
Experiência médica, com que se procura determinar se uma criança nasceu morta ou viva.
\section{Docimástico}
\begin{itemize}
\item {Grp. gram.:adj.}
\end{itemize}
Relativo a docimásia.
\section{Docíssimo}
\begin{itemize}
\item {Grp. gram.:adj.}
\end{itemize}
\begin{itemize}
\item {Utilização:Pop.}
\end{itemize}
\begin{itemize}
\item {Proveniência:(De \textunderscore doce\textunderscore )}
\end{itemize}
O mesmo que \textunderscore dulcíssimo\textunderscore .
\section{Docleia}
\begin{itemize}
\item {Grp. gram.:f.}
\end{itemize}
Crustáceo, vulgarmente conhecido por \textunderscore aranha do mar\textunderscore .
\section{Doco}
\begin{itemize}
\item {Grp. gram.:m.}
\end{itemize}
Peixe africano, semelhante á enguia, (\textunderscore protopterus annectens\textunderscore ).
\section{Doctilóquio}
\begin{itemize}
\item {Grp. gram.:m.}
\end{itemize}
Qualidade de doctíloquo.
Discurso erudito.
Eloquência.
\section{Doctíloquo}
\begin{itemize}
\item {Grp. gram.:adj.}
\end{itemize}
\begin{itemize}
\item {Utilização:Poét.}
\end{itemize}
\begin{itemize}
\item {Proveniência:(Lat. \textunderscore doctiloquus\textunderscore )}
\end{itemize}
Que fala doutamente; eloquente.
\section{Docto}
\begin{itemize}
\item {Grp. gram.:adj.}
\end{itemize}
\begin{itemize}
\item {Utilização:Des.}
\end{itemize}
O mesmo que \textunderscore douto\textunderscore . Cf. Sousa, \textunderscore Vida do Arceb.\textunderscore , I, 15.
\section{Doctor}
\textunderscore m.\textunderscore  (e der.)
O mesmo que \textunderscore doutor\textunderscore , etc.
\section{Doctrina}
\textunderscore f.\textunderscore  (e der.)
O mesmo que \textunderscore doutrina\textunderscore , etc.
\section{Documentação}
\begin{itemize}
\item {Grp. gram.:f.}
\end{itemize}
Acto ou effeito de documentar.
\section{Documentadamente}
\begin{itemize}
\item {Grp. gram.:adv.}
\end{itemize}
\begin{itemize}
\item {Proveniência:(De \textunderscore documentar\textunderscore )}
\end{itemize}
Por meio de documentos.
\section{Documentado}
\begin{itemize}
\item {Grp. gram.:adj.}
\end{itemize}
\begin{itemize}
\item {Proveniência:(De \textunderscore documentar\textunderscore )}
\end{itemize}
Fundado em documento ou documentos; acompanhado de documento ou documentos: \textunderscore um protesto documentado\textunderscore .
\section{Documental}
\begin{itemize}
\item {Grp. gram.:adj.}
\end{itemize}
Relativo a documento.
Fundado em documentos.
\section{Documentar}
\begin{itemize}
\item {Grp. gram.:v. t.}
\end{itemize}
Juntar documentos a.
Provar com documentos.
\section{Documentário}
\begin{itemize}
\item {Grp. gram.:adj.}
\end{itemize}
Relativo a documentos; que tem o valor de documento.
\section{Documentativo}
\begin{itemize}
\item {Grp. gram.:adj.}
\end{itemize}
Que serve para documentar.
\section{Documentável}
\begin{itemize}
\item {Grp. gram.:adj.}
\end{itemize}
Que se póde documentar.
\section{Documento}
\begin{itemize}
\item {Grp. gram.:m.}
\end{itemize}
\begin{itemize}
\item {Utilização:Des.}
\end{itemize}
\begin{itemize}
\item {Proveniência:(Lat. \textunderscore documentum\textunderscore )}
\end{itemize}
Aquillo que ensina, que serve de exemplo.
Aquillo que serve de prova.
Título.
Declaração escrita, para servir de prova.
Demonstração.
Recommendação; preceito.
\section{Doçura}
\begin{itemize}
\item {Grp. gram.:f.}
\end{itemize}
Qualidade daquillo que é doce.
Suavidade; brandura.
Simplicidade.
\section{Dodecaédrico}
Relativo ao dodecaedro.
\section{Dodecaedro}
\begin{itemize}
\item {Grp. gram.:m.}
\end{itemize}
\begin{itemize}
\item {Utilização:Mathem.}
\end{itemize}
\begin{itemize}
\item {Proveniência:(Gr. \textunderscore dodekaedros\textunderscore )}
\end{itemize}
Sólido com doze faces.
\section{Dodecagínia}
\begin{itemize}
\item {Grp. gram.:f.}
\end{itemize}
\begin{itemize}
\item {Utilização:Bot.}
\end{itemize}
\begin{itemize}
\item {Proveniência:(Do gr. \textunderscore dodeka\textunderscore  + \textunderscore gune\textunderscore )}
\end{itemize}
Uma das ordens de uma das classes, em que se divide o sistema de Linneu, a qual abrange as flôres, cujos estames e pistilos são em número de doze pelo menos.
\section{Dodecágino}
\begin{itemize}
\item {Grp. gram.:adj.}
\end{itemize}
\begin{itemize}
\item {Utilização:Bot.}
\end{itemize}
\begin{itemize}
\item {Proveniência:(Do gr. \textunderscore dodeka\textunderscore  + \textunderscore gune\textunderscore )}
\end{itemize}
Diz-se da flôr que tem doze pistilos, estiletes ou estigma sésseis.
\section{Dodecagonal}
\begin{itemize}
\item {Grp. gram.:adj.}
\end{itemize}
Relativo ao dodecágono.
\section{Dodecágono}
\begin{itemize}
\item {Grp. gram.:m.}
\end{itemize}
\begin{itemize}
\item {Utilização:Mathem.}
\end{itemize}
\begin{itemize}
\item {Proveniência:(Gr. \textunderscore dodekagonos\textunderscore )}
\end{itemize}
Polýgono de doze lados.
\section{Dodecagýnia}
\begin{itemize}
\item {Grp. gram.:f.}
\end{itemize}
\begin{itemize}
\item {Utilização:Bot.}
\end{itemize}
\begin{itemize}
\item {Proveniência:(Do gr. \textunderscore dodeka\textunderscore  + \textunderscore gune\textunderscore )}
\end{itemize}
Uma das ordens de uma das classes, em que se divide o systema de Linneu, a qual abrange as flôres, cujos estames e pistillos são em número de doze pelo menos.
\section{Dodecágyno}
\begin{itemize}
\item {Grp. gram.:adj.}
\end{itemize}
\begin{itemize}
\item {Utilização:Bot.}
\end{itemize}
\begin{itemize}
\item {Proveniência:(Do gr. \textunderscore dodeka\textunderscore  + \textunderscore gune\textunderscore )}
\end{itemize}
Diz-se da flôr que tem doze pistillos, estyletes ou estigma sésseis.
\section{Dodecandria}
\begin{itemize}
\item {Grp. gram.:f.}
\end{itemize}
Qualidade de dodecandro.
Classe, dos vegetaes que têm doze estames.
\section{Dodecandro}
\begin{itemize}
\item {Grp. gram.:adj.}
\end{itemize}
\begin{itemize}
\item {Utilização:Bot.}
\end{itemize}
\begin{itemize}
\item {Proveniência:(Do gr. \textunderscore dodeka\textunderscore  + \textunderscore aner\textunderscore , \textunderscore andros\textunderscore )}
\end{itemize}
Que tem doze estames, livres entre si.
\section{Dodecapétalo}
\begin{itemize}
\item {Grp. gram.:adj.}
\end{itemize}
\begin{itemize}
\item {Utilização:Bot.}
\end{itemize}
\begin{itemize}
\item {Proveniência:(Do gr. \textunderscore dodeka\textunderscore  + \textunderscore petalon\textunderscore )}
\end{itemize}
Que tem doze pétalas.
\section{Dòdó}
\begin{itemize}
\item {Grp. gram.:m.}
\end{itemize}
Espécie de cysne, que se encontrava na vizinhança de Madagáscar, e cuja raça parece extinta.
\section{Dodoneáceas}
\begin{itemize}
\item {Grp. gram.:f. pl.}
\end{itemize}
\begin{itemize}
\item {Utilização:Bot.}
\end{itemize}
Tríbo de sapindáceas, estabelecida por Kunt, e que tem por typo a dodoneia.
\section{Dodoneia}
\begin{itemize}
\item {Grp. gram.:f.}
\end{itemize}
\begin{itemize}
\item {Proveniência:(De \textunderscore Dodoens\textunderscore , n. p. de um bot. hol.)}
\end{itemize}
Gênero de plantas sapindáceas.
\section{Dodrans}
\begin{itemize}
\item {Grp. gram.:m.}
\end{itemize}
(V.dodrante)
\section{Dodrantal}
\begin{itemize}
\item {Grp. gram.:adj.}
\end{itemize}
\begin{itemize}
\item {Proveniência:(De \textunderscore dodrante\textunderscore )}
\end{itemize}
Dizia-se do castello, cuja defensa era a três quartos do tiro de mosquete.
\section{Dodrante}
\begin{itemize}
\item {Grp. gram.:m.}
\end{itemize}
\begin{itemize}
\item {Proveniência:(Lat. \textunderscore dodrans\textunderscore )}
\end{itemize}
Três quartas partes da unidade das moédas, dos pesos, das medidas, etc., entre os antigos Romanos. Cf. Castilho, \textunderscore Fastos\textunderscore , I, 353.
\section{Dóe}
\begin{itemize}
\item {Grp. gram.:m.}
\end{itemize}
\begin{itemize}
\item {Utilização:Infant.}
\end{itemize}
\begin{itemize}
\item {Proveniência:(De \textunderscore doer\textunderscore )}
\end{itemize}
O mesmo que \textunderscore axe\textunderscore ^1.
\section{Doença}
\begin{itemize}
\item {Grp. gram.:f.}
\end{itemize}
\begin{itemize}
\item {Utilização:Bras. de Minas}
\end{itemize}
\begin{itemize}
\item {Utilização:Fig.}
\end{itemize}
\begin{itemize}
\item {Proveniência:(Do lat. \textunderscore dolentia\textunderscore )}
\end{itemize}
Alteração na saúde.
Falta de saúde.
Moléstia.
Mal.
O mesmo que \textunderscore parto\textunderscore .
Tarefa laboriosa, diffícil.
Defeito, mania.
\section{Doente}
\begin{itemize}
\item {Grp. gram.:adj.}
\end{itemize}
\begin{itemize}
\item {Grp. gram.:M.  e  f.}
\end{itemize}
\begin{itemize}
\item {Proveniência:(Lat. \textunderscore dolens\textunderscore )}
\end{itemize}
Que tem doença.
Que soffre incômmodo phýsico ou moral.
Achacadiço.
Fraco.
Pessôa doente.
\section{Doentio}
\begin{itemize}
\item {Grp. gram.:adj.}
\end{itemize}
\begin{itemize}
\item {Proveniência:(De \textunderscore doente\textunderscore )}
\end{itemize}
Que adoéce facilmente.
Débil.
Prejudícial a saúde.
Que causa doença.
Em que há mania ou achaque moral.
\section{Doer}
\begin{itemize}
\item {Grp. gram.:v. t.}
\end{itemize}
\begin{itemize}
\item {Proveniência:(Do lat. \textunderscore dolere\textunderscore )}
\end{itemize}
Causar pena, dôr, dó.
Soffrer.
Estar dorido, sentir dôr.
\section{Dões}
\begin{itemize}
\item {Grp. gram.:m. pl.}
\end{itemize}
(Pl. de \textunderscore dom\textunderscore , us. por Vieira e, antes, por Tarrago. Cf. \textunderscore Laura de Anfriso\textunderscore , 20)
\section{Doestador}
\begin{itemize}
\item {fónica:do-es}
\end{itemize}
\begin{itemize}
\item {Grp. gram.:adj.}
\end{itemize}
\begin{itemize}
\item {Grp. gram.:M.}
\end{itemize}
Que doésta.
Aquelle que doésta.
\section{Doestar}
\begin{itemize}
\item {fónica:do-es}
\end{itemize}
\begin{itemize}
\item {Grp. gram.:v. t.}
\end{itemize}
\begin{itemize}
\item {Proveniência:(Do lat. hyp. \textunderscore dehonestare\textunderscore )}
\end{itemize}
Injuriar; dirigir doéstos a.
\section{Doésto}
\begin{itemize}
\item {Grp. gram.:m.}
\end{itemize}
\begin{itemize}
\item {Proveniência:(De \textunderscore doestar\textunderscore )}
\end{itemize}
Injúria.
Acção deshonrosa, de que se accusa alguém.
\section{Dogal}
\begin{itemize}
\item {Grp. gram.:adj.}
\end{itemize}
Relativo a doge; próprio de doge. Cf. Sousa Monteiro, \textunderscore Elog. de Lat.\textunderscore 
\section{Dogalina}
\begin{itemize}
\item {Grp. gram.:f.}
\end{itemize}
\begin{itemize}
\item {Proveniência:(De \textunderscore dogal\textunderscore )}
\end{itemize}
Espécie de vestimenta privativa, que os antigos senadores de Veneza usavam sôbre o vestuário commum.
\section{Dogaressa}
\begin{itemize}
\item {fónica:garê}
\end{itemize}
\begin{itemize}
\item {Grp. gram.:f.}
\end{itemize}
\begin{itemize}
\item {Proveniência:(It. \textunderscore dogaressa\textunderscore )}
\end{itemize}
Mulher de um doge. Cf. S. Monteiro, \textunderscore Camafeus\textunderscore .
\section{Doge}
\begin{itemize}
\item {Grp. gram.:m.}
\end{itemize}
\begin{itemize}
\item {Proveniência:(It. \textunderscore doge\textunderscore , do lat. \textunderscore dux\textunderscore )}
\end{itemize}
Magistrado superior nas repúblicas de Veneza e Gênova.
\section{Dogesa}
\begin{itemize}
\item {Grp. gram.:f.}
\end{itemize}
Mulher de doge.
Dogaressa.
\section{Dógicos}
\begin{itemize}
\item {Grp. gram.:m. pl.}
\end{itemize}
Espécie de noviços nas confrarias búdicas japonesas.
(Talvez do japon. \textunderscore dogaku\textunderscore )
\section{Dogma}
\begin{itemize}
\item {Grp. gram.:m.}
\end{itemize}
\begin{itemize}
\item {Utilização:Fig.}
\end{itemize}
\begin{itemize}
\item {Proveniência:(Gr. \textunderscore dogma\textunderscore )}
\end{itemize}
Cada um dos pontos fundamentaes e indiscutíveis de uma crença religiosa.
Conjunto das doutrínas fundamentaes do Christianismo.
Proposição, apresentada como incontestável e indiscutível.
\section{Dogmaticamente}
\begin{itemize}
\item {Grp. gram.:adv.}
\end{itemize}
De modo dogmático.
\section{Dogmático}
\begin{itemize}
\item {Grp. gram.:adj.}
\end{itemize}
\begin{itemize}
\item {Utilização:Fig.}
\end{itemize}
Relativo a dogma.
Autoritário; sentencioso.
\section{Dogmatismo}
\begin{itemize}
\item {Grp. gram.:m.}
\end{itemize}
\begin{itemize}
\item {Proveniência:(De \textunderscore dogma\textunderscore )}
\end{itemize}
Systema dos que acceitam dogmas.
Systema dos que não acceitam discussão do que allegam ou affirmam.
\section{Dogmatista}
\begin{itemize}
\item {Grp. gram.:m.}
\end{itemize}
\begin{itemize}
\item {Grp. gram.:M.  e  f.}
\end{itemize}
\begin{itemize}
\item {Proveniência:(De \textunderscore dogma\textunderscore )}
\end{itemize}
Sectário do dogmatismo.
Indivíduo autoritário nas suas ideias.
\section{Dogmatizador}
\begin{itemize}
\item {Grp. gram.:m.}
\end{itemize}
Aquelle que dogmatiza.
\section{Dogmatizante}
\begin{itemize}
\item {Grp. gram.:m.  e  adj.}
\end{itemize}
Aquelle que dogmatiza.
\section{Dogmatizar}
\begin{itemize}
\item {Grp. gram.:v. t.}
\end{itemize}
\begin{itemize}
\item {Grp. gram.:V. i.}
\end{itemize}
\begin{itemize}
\item {Proveniência:(Do lat. \textunderscore dogmatizare\textunderscore )}
\end{itemize}
Proclamar como dogma.
Ensinar autoritariamente.
Estabelecer dogmas.
Dar ás suas affirmações o valor de indiscutíveis.
\section{Dogmatologia}
\begin{itemize}
\item {Grp. gram.:f.}
\end{itemize}
Tratado dos dogmas.
\section{Dogre}
\begin{itemize}
\item {Grp. gram.:m.}
\end{itemize}
\begin{itemize}
\item {Proveniência:(Do hol. \textunderscore doggerook\textunderscore )}
\end{itemize}
Barco de pesca, na Hollanda.
\section{Dogue}
\begin{itemize}
\item {Grp. gram.:m.}
\end{itemize}
\begin{itemize}
\item {Proveniência:(Do ingl. \textunderscore dog\textunderscore )}
\end{itemize}
Variedade, de cães, de pequeno corpo, focinho chato e indole bravia.
\section{Doi}
\begin{itemize}
\item {Grp. gram.:m.}
\end{itemize}
\begin{itemize}
\item {Utilização:Infant.}
\end{itemize}
\begin{itemize}
\item {Proveniência:(De \textunderscore doer\textunderscore )}
\end{itemize}
O mesmo que \textunderscore axe\textunderscore ^1.
\section{Doida}
\begin{itemize}
\item {Grp. gram.:f.}
\end{itemize}
\begin{itemize}
\item {Proveniência:(De \textunderscore doido\textunderscore )}
\end{itemize}
Moléstia, que dá nos miolos do gado lanigero.
\section{Doidamente}
\begin{itemize}
\item {Grp. gram.:adv.}
\end{itemize}
De modo doido.
Com doidice.
Tolamente.
\section{Doidaria}
\begin{itemize}
\item {Grp. gram.:f.}
\end{itemize}
\begin{itemize}
\item {Proveniência:(De \textunderscore doido\textunderscore )}
\end{itemize}
Doidice; os doidos. Cf. Filinto, XIII, 136.
\section{Doidarrão}
\begin{itemize}
\item {Grp. gram.:adj.}
\end{itemize}
\begin{itemize}
\item {Utilização:Pop.}
\end{itemize}
\begin{itemize}
\item {Proveniência:(De \textunderscore doido\textunderscore )}
\end{itemize}
Muito doido.
Idiota.
Pateta.
\section{Doidarraz}
\begin{itemize}
\item {Grp. gram.:m.}
\end{itemize}
O mesmo que \textunderscore doidarrão\textunderscore . Cf. Filinto, XIX, 247.
\section{Doidejante}
\begin{itemize}
\item {Grp. gram.:adj.}
\end{itemize}
Que doideja.
\section{Doidejar}
\begin{itemize}
\item {Grp. gram.:v. i.}
\end{itemize}
\begin{itemize}
\item {Proveniência:(De \textunderscore doido\textunderscore )}
\end{itemize}
Fazer doidices, loucuras.
Dizer ou fazer disparates.
Brincar.
Vagabundear.
\section{Doidejo}
\begin{itemize}
\item {Grp. gram.:m.}
\end{itemize}
Acto de doidejar.
\section{Doidelas}
\begin{itemize}
\item {Grp. gram.:m.}
\end{itemize}
\begin{itemize}
\item {Utilização:Prov.}
\end{itemize}
\begin{itemize}
\item {Utilização:trasm.}
\end{itemize}
\begin{itemize}
\item {Proveniência:(De \textunderscore doido\textunderscore )}
\end{itemize}
Doidivanas, homem estavanado.
\section{Doidete}
\begin{itemize}
\item {Grp. gram.:m.}
\end{itemize}
Aquelle que tem pouco juízo:«\textunderscore o lobo apanha pelo pescoço o doidete\textunderscore ». Sá de Miranda.
\section{Doidice}
\begin{itemize}
\item {Grp. gram.:f.}
\end{itemize}
\begin{itemize}
\item {Proveniência:(De \textunderscore doido\textunderscore )}
\end{itemize}
Estado de quem é doido.
Palavras ou actos próprios de doido.
Extravagância; excesso.
\section{Doidinha}
\begin{itemize}
\item {Grp. gram.:f.}
\end{itemize}
O mesmo que \textunderscore papa-formigas\textunderscore .
\section{Doidivana}
\begin{itemize}
\item {Grp. gram.:m.  e  f.}
\end{itemize}
O mesmo que \textunderscore doidivanas\textunderscore . Cf. J. Dinís, \textunderscore Fidalgos\textunderscore , I, 128.
\section{Doidivanas}
\begin{itemize}
\item {Grp. gram.:m.  e  f.}
\end{itemize}
\begin{itemize}
\item {Utilização:Fam.}
\end{itemize}
\begin{itemize}
\item {Proveniência:(Do rad. de \textunderscore doido\textunderscore )}
\end{itemize}
Indivíduo leviano, imprudente.
Pateta.
\section{Doido}
\begin{itemize}
\item {Grp. gram.:adj.}
\end{itemize}
\begin{itemize}
\item {Grp. gram.:M.}
\end{itemize}
\begin{itemize}
\item {Proveniência:(Do ingl. \textunderscore dold\textunderscore ?)}
\end{itemize}
Que não tem juízo.
Que perdeu o uso da razão.
Louco.
Extravagante.
Temerário.
Insensato.
Que é indício de falta de juízo ou de falta de prudência: \textunderscore doidas correrias\textunderscore .
Arrebatado, enthusiasta: \textunderscore sou doido por flôres\textunderscore .
Muito contente.
Vaidoso.
Extraordináriamente affectuoso: \textunderscore é doido por ella\textunderscore .
Indivíduo doido.
\section{Doído}
\begin{itemize}
\item {Grp. gram.:adj.}
\end{itemize}
\begin{itemize}
\item {Proveniência:(De \textunderscore doer\textunderscore )}
\end{itemize}
Que sente dor.
Maguado.
Que revela soffrimento.
\section{Doilo}
\begin{itemize}
\item {Grp. gram.:m.}
\end{itemize}
\begin{itemize}
\item {Utilização:Ant.}
\end{itemize}
\begin{itemize}
\item {Proveniência:(Do cast. \textunderscore duelo\textunderscore )}
\end{itemize}
Dó; luto; mágoa. Cf. \textunderscore Eufrosina\textunderscore , 45 e 107.
\section{Doirada}
\begin{itemize}
\item {Grp. gram.:f.}
\end{itemize}
\begin{itemize}
\item {Proveniência:(De \textunderscore doirado\textunderscore )}
\end{itemize}
Nome de algumas espécies de peixes.
Variedade de uva beirôa e minhota.
\section{Doiradilho}
\begin{itemize}
\item {Grp. gram.:adj.}
\end{itemize}
\begin{itemize}
\item {Utilização:Bras}
\end{itemize}
\begin{itemize}
\item {Proveniência:(De \textunderscore doirado\textunderscore )}
\end{itemize}
Diz-se dos cavallos de côr avermelhada.
Diz-se do cavallo castanho.
\section{Doiradinha}
\begin{itemize}
\item {Grp. gram.:f.}
\end{itemize}
\begin{itemize}
\item {Proveniência:(De \textunderscore doirado\textunderscore )}
\end{itemize}
Espécie de fêto.
Ave pernalta, o mesmo que \textunderscore tarambola\textunderscore .
Planta, da fam. das compostas, (\textunderscore senecio incrassatus\textunderscore , Lin.).
Variedade de pêra.
Espécie de jôgo de cartas.
A dama de oiros, nesse jôgo.
\section{Doirado}
\begin{itemize}
\item {Grp. gram.:adj.}
\end{itemize}
\begin{itemize}
\item {Grp. gram.:M.}
\end{itemize}
\begin{itemize}
\item {Utilização:Prov.}
\end{itemize}
\begin{itemize}
\item {Proveniência:(De \textunderscore doirar\textunderscore )}
\end{itemize}
Revestido de camada de oiro: \textunderscore tecto doirado\textunderscore .
Que tem côr de oiro: \textunderscore cabello doirado\textunderscore .
Doiradura.
Casta de uva preta de Collares.
Peixe de Portugal.
O mesmo que \textunderscore tarambola\textunderscore .
\section{Doirador}
\begin{itemize}
\item {Grp. gram.:m.}
\end{itemize}
\begin{itemize}
\item {Proveniência:(Do lat. \textunderscore deaurator\textunderscore )}
\end{itemize}
Aquelle que doira.
\section{Doiradura}
\begin{itemize}
\item {Grp. gram.:f.}
\end{itemize}
Camada ou fôlha de oiro sôbre um objecto.
Coisa doirada.
Arte ou acto de doirar.
\section{Doiramento}
\begin{itemize}
\item {Grp. gram.:m.}
\end{itemize}
Acto de doirar.
\section{Doirar}
\begin{itemize}
\item {Grp. gram.:v. t.}
\end{itemize}
\begin{itemize}
\item {Utilização:Fig.}
\end{itemize}
\begin{itemize}
\item {Proveniência:(Do lat. \textunderscore deaurare\textunderscore )}
\end{itemize}
Revestir com camada de oiro em fôlha ou em dissolução.
Dar a côr de oiro a.
Disfarçar: \textunderscore doirar a pílula\textunderscore .
Desculpar.
Tornar brilhante, formoso, feliz.
Adornar.
\section{Dois}
\begin{itemize}
\item {Grp. gram.:adj.}
\end{itemize}
\begin{itemize}
\item {Grp. gram.:M.}
\end{itemize}
\begin{itemize}
\item {Proveniência:(Do lat. \textunderscore duo\textunderscore )}
\end{itemize}
Diz-se do número cardinal, formado de um e mais um.
Segundo.
Algarismo, representativo dêsse número.
Carta de jogar ou peça do dominó, que tem dois pontos.
Aquelle ou aquillo que numa série de dois occupa o último lugar.
\section{Doiseiro}
\begin{itemize}
\item {Grp. gram.:adj.}
\end{itemize}
\begin{itemize}
\item {Utilização:T. de Ceilão}
\end{itemize}
\begin{itemize}
\item {Proveniência:(De \textunderscore dois\textunderscore )}
\end{itemize}
O mesmo que \textunderscore segundo\textunderscore ^1.
\section{Dolabela}
\begin{itemize}
\item {Grp. gram.:f.}
\end{itemize}
Espécie de molusco.
\section{Dolabella}
\begin{itemize}
\item {Grp. gram.:f.}
\end{itemize}
Espécie de molusco.
\section{Dolabriforme}
\begin{itemize}
\item {Grp. gram.:adj.}
\end{itemize}
\begin{itemize}
\item {Utilização:Bot.}
\end{itemize}
\begin{itemize}
\item {Proveniência:(Do lat. \textunderscore dolabra\textunderscore  + \textunderscore forma\textunderscore )}
\end{itemize}
Diz-se das fôlhas carnudas, quási cylíndricas na base, achatadas no ápice, com dois bordos, sendo um espêsso e rectilíneo, e o outro largo e circular.
\section{Dolente}
\begin{itemize}
\item {Grp. gram.:adj.}
\end{itemize}
\begin{itemize}
\item {Proveniência:(Lat. \textunderscore dolens\textunderscore )}
\end{itemize}
Que revela dôr.
Magoado: \textunderscore versos dolentes\textunderscore .
Lastimoso.
\section{Doler}
\begin{itemize}
\item {Grp. gram.:v. i.}
\end{itemize}
\begin{itemize}
\item {Utilização:Ant.}
\end{itemize}
\begin{itemize}
\item {Proveniência:(Lat. \textunderscore dolere\textunderscore )}
\end{itemize}
O mesmo que \textunderscore doer\textunderscore .
\section{Dolerina}
\begin{itemize}
\item {Grp. gram.:f.}
\end{itemize}
O mesmo que \textunderscore dolerite\textunderscore .
\section{Dolerite}
\begin{itemize}
\item {Grp. gram.:f.}
\end{itemize}
\begin{itemize}
\item {Proveniência:(Do gr. \textunderscore doleros\textunderscore )}
\end{itemize}
Matéria ignívoma, que, segundo Estrabão, constitue a montanha Strômboli. Cf. Castilho, \textunderscore Geórgicas\textunderscore , 420.
\section{Dólicho}
\begin{itemize}
\item {fónica:co}
\end{itemize}
\begin{itemize}
\item {Grp. gram.:m.}
\end{itemize}
\begin{itemize}
\item {Proveniência:(Gr. \textunderscore dolikhos\textunderscore )}
\end{itemize}
Planta leguminosa das regiões intertropicaes.
\section{Dólico}
\begin{itemize}
\item {Grp. gram.:m.}
\end{itemize}
\begin{itemize}
\item {Proveniência:(Gr. \textunderscore dolikhos\textunderscore )}
\end{itemize}
Planta leguminosa das regiões intertropicaes.
\section{Dolicocefalia}
\begin{itemize}
\item {Grp. gram.:f.}
\end{itemize}
Estado de dolicocéfalo.
\section{Dolicocéfalo}
\begin{itemize}
\item {Grp. gram.:adj.}
\end{itemize}
\begin{itemize}
\item {Proveniência:(Do gr. \textunderscore dolikhos\textunderscore  + \textunderscore kephale\textunderscore )}
\end{itemize}
Que tem oval o crânio, sendo o diâmetro transversal mais pequeno uma quarta parte que o longitudinal.
\section{Dolichocephalia}
\begin{itemize}
\item {fónica:co}
\end{itemize}
\begin{itemize}
\item {Grp. gram.:f.}
\end{itemize}
Estado de dolichocéphalo.
\section{Dolichocéphalo}
\begin{itemize}
\item {fónica:co}
\end{itemize}
\begin{itemize}
\item {Grp. gram.:adj.}
\end{itemize}
\begin{itemize}
\item {Proveniência:(Do gr. \textunderscore dolikhos\textunderscore  + \textunderscore kephale\textunderscore )}
\end{itemize}
Que tem oval o crânio, sendo o diâmetro transversal mais pequeno uma quarta parte que o longitudinal.
\section{Dolichópode}
\begin{itemize}
\item {fónica:co}
\end{itemize}
\begin{itemize}
\item {Grp. gram.:adj.}
\end{itemize}
\begin{itemize}
\item {Utilização:Zool.}
\end{itemize}
\begin{itemize}
\item {Proveniência:(Do gr. \textunderscore dolikhos\textunderscore  + \textunderscore pous\textunderscore , \textunderscore podos\textunderscore )}
\end{itemize}
Que tem patas grandes.
\section{Dolicópode}
\begin{itemize}
\item {Grp. gram.:adj.}
\end{itemize}
\begin{itemize}
\item {Utilização:Zool.}
\end{itemize}
\begin{itemize}
\item {Proveniência:(Do gr. \textunderscore dolikhos\textunderscore  + \textunderscore pous\textunderscore , \textunderscore podos\textunderscore )}
\end{itemize}
Que tem patas grandes.
\section{Dolicopse}
\begin{itemize}
\item {Grp. gram.:adj.}
\end{itemize}
\begin{itemize}
\item {Utilização:Anthrop.}
\end{itemize}
Diz-se da face, em que a altura predomina sôbre o diâmetro bi-zygomático, segundo Quatrefages.
\section{Dóliman}
\begin{itemize}
\item {Grp. gram.:m.}
\end{itemize}
Capa turca, de mangas estreitas.
(Do turco \textunderscore thoulamet\textunderscore )
\section{Dóllar}
\begin{itemize}
\item {Grp. gram.:m.}
\end{itemize}
\begin{itemize}
\item {Proveniência:(T. ingl.)}
\end{itemize}
Moéda de prata nos Estados-Unidos da America do Norte, onde é unidade monetária, correspondente a 932 reis do dinheiro português.
\section{Dólman}
\begin{itemize}
\item {Grp. gram.:m.}
\end{itemize}
Capa curta, que faz parte do uniforme dos hússares.
Casaco curto e justo de militares.
(Contr. de \textunderscore dóliman\textunderscore )
\section{Dólmen}
\begin{itemize}
\item {Grp. gram.:m.}
\end{itemize}
\begin{itemize}
\item {Proveniência:(Fr. \textunderscore dolmen\textunderscore )}
\end{itemize}
Pedra de anta; o mesmo que \textunderscore anta\textunderscore .
\section{Dolmênico}
\begin{itemize}
\item {Grp. gram.:adj.}
\end{itemize}
O mesmo que \textunderscore dolmético\textunderscore .
\section{Dolmético}
\begin{itemize}
\item {Grp. gram.:adj.}
\end{itemize}
Relativo a dólmen.
\section{Dólmin}
\begin{itemize}
\item {Grp. gram.:m.}
\end{itemize}
O mesmo que \textunderscore dólmen\textunderscore , anta. Cf. Filinto, XV, 52.
\section{Dolo}
\begin{itemize}
\item {Grp. gram.:m.}
\end{itemize}
\begin{itemize}
\item {Proveniência:(Lat. \textunderscore dolus\textunderscore )}
\end{itemize}
Fraude.
Astúcia.
Engano.
Traição.
\section{Dolo}
\begin{itemize}
\item {Grp. gram.:m.}
\end{itemize}
Espécie de punhal, metido em baínha de madeira, á semelhança de estoque, e usado antigamente na península hispânica. Cf. C. Aires, \textunderscore Hist. do Exérc. Port.\textunderscore 
\section{Dolomia}
\begin{itemize}
\item {Grp. gram.:f.}
\end{itemize}
\begin{itemize}
\item {Proveniência:(De \textunderscore Dolomieu\textunderscore , n. p.)}
\end{itemize}
Variedade de carbonato de cal e de magnésia.
\section{Dolomite}
\begin{itemize}
\item {Grp. gram.:f.}
\end{itemize}
\begin{itemize}
\item {Proveniência:(De \textunderscore dolomia\textunderscore )}
\end{itemize}
Carbonato, de fórma hexagonal.
\section{Dolomítico}
\begin{itemize}
\item {Grp. gram.:adj.}
\end{itemize}
Que contém dolomia.
\section{Dolomização}
\begin{itemize}
\item {Grp. gram.:f.}
\end{itemize}
Formação de rochas dolomíticas.
\section{Dolópio}
\begin{itemize}
\item {Grp. gram.:m.}
\end{itemize}
\begin{itemize}
\item {Proveniência:(Gr. \textunderscore dolopoios\textunderscore )}
\end{itemize}
Gênero de insectos coleópteros pentâmeros.
\section{Dolor}
\begin{itemize}
\item {Grp. gram.:m.}
\end{itemize}
\begin{itemize}
\item {Utilização:Des.}
\end{itemize}
\begin{itemize}
\item {Proveniência:(Lat. \textunderscore dolor\textunderscore )}
\end{itemize}
O mesmo que \textunderscore dôr\textunderscore .
\section{Dolorido}
\begin{itemize}
\item {Grp. gram.:adj.}
\end{itemize}
\begin{itemize}
\item {Proveniência:(De \textunderscore dolor\textunderscore )}
\end{itemize}
Que tem dôr.
Dorido.
Magoado.
Lastimoso.
\section{Dolorífico}
\begin{itemize}
\item {Grp. gram.:adj.}
\end{itemize}
\begin{itemize}
\item {Proveniência:(Lat. \textunderscore dolorificus\textunderscore )}
\end{itemize}
Doloroso.
Que produz dôr.
\section{Dolório}
\begin{itemize}
\item {Grp. gram.:m.}
\end{itemize}
\begin{itemize}
\item {Utilização:Açor}
\end{itemize}
O mesmo que \textunderscore desgôsto\textunderscore .
(Cp. lat. \textunderscore dolor\textunderscore )
\section{Dolorosamente}
\begin{itemize}
\item {Grp. gram.:adv.}
\end{itemize}
De modo doloroso.
Lastimosamente.
\section{Doloroso}
\begin{itemize}
\item {Grp. gram.:adj.}
\end{itemize}
\begin{itemize}
\item {Proveniência:(Lat. \textunderscore dolorosus\textunderscore )}
\end{itemize}
Que produz dôr.
Dorido.
Lastimoso.
Amargurado: \textunderscore vida dolorosa\textunderscore .
Que revela dôr.
\section{Dolosamente}
\begin{itemize}
\item {Grp. gram.:adv.}
\end{itemize}
De modo doloso. Com dolo^1.
\section{Doloso}
\begin{itemize}
\item {Grp. gram.:adj.}
\end{itemize}
\begin{itemize}
\item {Proveniência:(Lat. \textunderscore dolosus\textunderscore )}
\end{itemize}
Que procede com dolo.
Em que há dolo.
Causado por dolo.
\section{Dom}
\begin{itemize}
\item {Grp. gram.:m.}
\end{itemize}
\begin{itemize}
\item {Utilização:Fig.}
\end{itemize}
\begin{itemize}
\item {Proveniência:(Do lat. \textunderscore donum\textunderscore )}
\end{itemize}
Donativo.
Acto de dar.
Dádiva.
Benefício da natureza.
Merecimento.
Vantagem.
Poder.
Privilégio.
\section{Dom}
\begin{itemize}
\item {Grp. gram.:m.}
\end{itemize}
\begin{itemize}
\item {Proveniência:(Do lat. \textunderscore dominus\textunderscore )}
\end{itemize}
Título honorífico, que precede os nomes próprios masculinos, em certas categorias sociáes.
\section{Dóma}
\begin{itemize}
\item {Grp. gram.:f.}
\end{itemize}
\begin{itemize}
\item {Utilização:Ant.}
\end{itemize}
\begin{itemize}
\item {Proveniência:(Do lat. \textunderscore hebdomada\textunderscore )}
\end{itemize}
O mesmo que \textunderscore semana\textunderscore . Cf. G. Vicente, I, 127.
\section{Domá}
\begin{itemize}
\item {Grp. gram.:f.}
\end{itemize}
Moéda antiga de Castella.
\section{Dómaa}
\begin{itemize}
\item {Grp. gram.:f.}
\end{itemize}
\begin{itemize}
\item {Utilização:Ant.}
\end{itemize}
O mesmo que \textunderscore domá\textunderscore . Cf. Moraes, \textunderscore Diccion\textunderscore .
\section{Domador}
\begin{itemize}
\item {Grp. gram.:adj.}
\end{itemize}
\begin{itemize}
\item {Grp. gram.:M.}
\end{itemize}
\begin{itemize}
\item {Proveniência:(Lat. \textunderscore domator\textunderscore )}
\end{itemize}
Que doma.
Aquelle que doma ou domestica.
\section{Domar}
\begin{itemize}
\item {Grp. gram.:v. t.}
\end{itemize}
\begin{itemize}
\item {Proveniência:(Lat. \textunderscore domare\textunderscore )}
\end{itemize}
Vencer a resistência de.
Tornar manso.
Domesticar: \textunderscore domar um urso\textunderscore .
Sujeitar.
Abater: \textunderscore domar orgulhosos\textunderscore .
Fazer ceder.
Refrear: \textunderscore domar paixões\textunderscore .
\section{Domário}
\begin{itemize}
\item {Grp. gram.:m.}
\end{itemize}
\begin{itemize}
\item {Utilização:Ant.}
\end{itemize}
\begin{itemize}
\item {Proveniência:(Do lat. \textunderscore domus\textunderscore ?)}
\end{itemize}
Sacerdote, que celebrava a Missa no Offício de Defuntos, e presidia ao mesmo Offício.
\section{Domável}
\begin{itemize}
\item {Grp. gram.:adj.}
\end{itemize}
\begin{itemize}
\item {Proveniência:(Lat. \textunderscore domabilis\textunderscore )}
\end{itemize}
Que póde sêr domado.
\section{Dom-barqueiro}
\begin{itemize}
\item {Grp. gram.:m.}
\end{itemize}
Espécie de jôgo popular.
\section{Dom-bernardo}
\begin{itemize}
\item {Grp. gram.:m.}
\end{itemize}
Arbusto rubiáceo do Brasil.
\section{Dombuela}
\begin{itemize}
\item {Grp. gram.:f.}
\end{itemize}
Ave pernalta da África.
\section{Domena}
\begin{itemize}
\item {Grp. gram.:f.}
\end{itemize}
\begin{itemize}
\item {Utilização:Prov.}
\end{itemize}
\begin{itemize}
\item {Utilização:alg.}
\end{itemize}
O mesmo que \textunderscore andaina\textunderscore  (de fato).
\section{Domesticação}
\begin{itemize}
\item {Grp. gram.:f.}
\end{itemize}
Acto de domesticar.
\section{Domesticador}
\begin{itemize}
\item {Grp. gram.:adj.}
\end{itemize}
\begin{itemize}
\item {Grp. gram.:M.}
\end{itemize}
Que domestica.
Aquelle que domestica.
\section{Domesticamente}
\begin{itemize}
\item {Grp. gram.:adv.}
\end{itemize}
De modo doméstico.
\section{Domesticar}
\begin{itemize}
\item {Grp. gram.:v. t.}
\end{itemize}
\begin{itemize}
\item {Utilização:Fig.}
\end{itemize}
\begin{itemize}
\item {Proveniência:(De \textunderscore doméstico\textunderscore )}
\end{itemize}
Tornar doméstico.
Domar.
Tornar culto, civilizar.
\section{Domesticável}
\begin{itemize}
\item {Grp. gram.:adj.}
\end{itemize}
Que se póde domesticar.
\section{Domesticidade}
\begin{itemize}
\item {Grp. gram.:f.}
\end{itemize}
Estado ou qualidade daquelle ou daquillo que é doméstico.
\section{Doméstico}
\begin{itemize}
\item {Grp. gram.:adj.}
\end{itemize}
\begin{itemize}
\item {Grp. gram.:M.}
\end{itemize}
\begin{itemize}
\item {Utilização:Bras. do N}
\end{itemize}
\begin{itemize}
\item {Proveniência:(Lat. \textunderscore domesticus\textunderscore )}
\end{itemize}
Relativo a casa, á vida íntima da família: \textunderscore desavenças domésticas\textunderscore .
Relativo ao govêrno da casa: \textunderscore despesas domésticas\textunderscore .
Familiar.
Que serve por soldada.
Diz-se do animal que vive ou é criado dentro de casa ou em dependências desta.
Relativo ao interior de um país, em opposição a estranho.
Aquelle que serve por soldada.
Criado.
Tecido grosseiro de algodão.
\section{Domestiqueza}
\begin{itemize}
\item {Grp. gram.:f.}
\end{itemize}
(V.domesticidade)
\section{Dom-fafe}
\begin{itemize}
\item {Grp. gram.:m.}
\end{itemize}
\begin{itemize}
\item {Utilização:Zool.}
\end{itemize}
\begin{itemize}
\item {Proveniência:(Do al. \textunderscore dompfaff\textunderscore , clérigo de cathedral, de \textunderscore dom\textunderscore , cathedral, e \textunderscore pfaffe\textunderscore , clérigo)}
\end{itemize}
Variedade de pisco, também conhecido por \textunderscore pisco-chilreiro\textunderscore , (\textunderscore pyrrhula vulgaris\textunderscore ).
\section{Domiciliar}
\begin{itemize}
\item {Grp. gram.:v. t.}
\end{itemize}
\begin{itemize}
\item {Grp. gram.:V. p.}
\end{itemize}
\begin{itemize}
\item {Proveniência:(De \textunderscore domicílio\textunderscore )}
\end{itemize}
Dar domicílio a.
Fazer fixar domicílio.
Fixar residência.
\section{Domiciliariamente}
\begin{itemize}
\item {Grp. gram.:adv.}
\end{itemize}
\begin{itemize}
\item {Proveniência:(De \textunderscore domiciliário\textunderscore )}
\end{itemize}
Como em domicílio; nos domicílios.
\section{Domiciliário}
\begin{itemize}
\item {Grp. gram.:adj.}
\end{itemize}
Relativo a domicílio; feito no domicílio.
\section{Domicílio}
\begin{itemize}
\item {Grp. gram.:m.}
\end{itemize}
\begin{itemize}
\item {Proveniência:(Lat. \textunderscore domicilium\textunderscore )}
\end{itemize}
Casa de residência.
Habitação fixa.
Povoação ou lugar, em que se reside com permanência.
Lugar, que, embora não seja residência fixa, é considerada legalmente como domicílio, por o cidadão exercer alli certos direitos ou funcções.
\section{Dominação}
\begin{itemize}
\item {Grp. gram.:f.}
\end{itemize}
\begin{itemize}
\item {Grp. gram.:Pl.}
\end{itemize}
\begin{itemize}
\item {Utilização:Theol.}
\end{itemize}
\begin{itemize}
\item {Proveniência:(Lat. \textunderscore dominatio\textunderscore )}
\end{itemize}
Acto ou effeito de dominar.
Um dos nove coros dos anjos.
Cp. \textunderscore jerarchia\textunderscore .
\section{Dominador}
\begin{itemize}
\item {Grp. gram.:adj.}
\end{itemize}
\begin{itemize}
\item {Grp. gram.:M.}
\end{itemize}
\begin{itemize}
\item {Proveniência:(Lat. \textunderscore dominator\textunderscore )}
\end{itemize}
Que domina; que infunde respeito.
Que sobresái.
Aquelle que subjugou ou conquistou; conquistador.
\section{Dominância}
\begin{itemize}
\item {Grp. gram.:f.}
\end{itemize}
\begin{itemize}
\item {Utilização:bras}
\end{itemize}
\begin{itemize}
\item {Utilização:Neol.}
\end{itemize}
Qualidade de dominante.
\section{Dominante}
\begin{itemize}
\item {Grp. gram.:m.  e  adj.}
\end{itemize}
\begin{itemize}
\item {Grp. gram.:F.}
\end{itemize}
\begin{itemize}
\item {Utilização:Mús.}
\end{itemize}
\begin{itemize}
\item {Proveniência:(Lat. \textunderscore dominans\textunderscore )}
\end{itemize}
Dominador.
Nota, que domina o tom, ou quinta nota acima da tónica.
\section{Dominar}
\begin{itemize}
\item {Grp. gram.:v. t.}
\end{itemize}
\begin{itemize}
\item {Grp. gram.:V. i.}
\end{itemize}
\begin{itemize}
\item {Proveniência:(Lat. \textunderscore dominari\textunderscore )}
\end{itemize}
Exercer domínio sôbre.
Sêr senhor de.
Conter.
Reprimir: \textunderscore dominar impulsos ruins\textunderscore .
Vencer.
Elevar-se acima de; sêr sobranceiro a: \textunderscore aquelle monte domina o rio\textunderscore .
Têr influência sôbre.
Preponderar sôbre.
Abranger, occupar.
Exercer domínio.
Preponderar.
\section{Dominativo}
\begin{itemize}
\item {Grp. gram.:adj.}
\end{itemize}
\begin{itemize}
\item {Utilização:Des.}
\end{itemize}
O mesmo que \textunderscore dominante\textunderscore .
\section{Dominável}
\begin{itemize}
\item {Grp. gram.:adj.}
\end{itemize}
Que se póde dominar.
\section{Dominga}
\begin{itemize}
\item {Grp. gram.:f.}
\end{itemize}
O mesmo que \textunderscore Domingo\textunderscore .
\section{Domingar}
\begin{itemize}
\item {Grp. gram.:v. i.}
\end{itemize}
Usar ou trazer ao Domingo: \textunderscore um casaco de domingar\textunderscore .
(Colhido na Guarda)
\section{Domingas}
\begin{itemize}
\item {Grp. gram.:f.}
\end{itemize}
Variedade de pêra, parecida com a amerim.
\section{Domingo}
\begin{itemize}
\item {Grp. gram.:m.}
\end{itemize}
\begin{itemize}
\item {Proveniência:(Do lat. \textunderscore dominicum\textunderscore )}
\end{itemize}
Dia do Senhor, dia consagrado ao descanso e á oração.
Primeiro dia da semana.
\section{Domingueiramente}
\begin{itemize}
\item {Grp. gram.:adv.}
\end{itemize}
De modo domingueiro.
\section{Domingueiro}
\begin{itemize}
\item {Grp. gram.:adj.}
\end{itemize}
\begin{itemize}
\item {Utilização:Fam.}
\end{itemize}
\begin{itemize}
\item {Proveniência:(De \textunderscore Domingo\textunderscore )}
\end{itemize}
Relativo ao Domingo.
Que se usa aos Domingos: \textunderscore fato domingueiro\textunderscore .
Festivo; garrido.
\section{Dominial}
\begin{itemize}
\item {Grp. gram.:adj.}
\end{itemize}
Relativo a domínio.
(B. lat. \textunderscore dominialis\textunderscore )
\section{Dominião}
\begin{itemize}
\item {Grp. gram.:f.}
\end{itemize}
\begin{itemize}
\item {Utilização:T. de Ceilão}
\end{itemize}
O mesmo que \textunderscore domínio\textunderscore .
\section{Dominical}
\begin{itemize}
\item {Grp. gram.:adj.}
\end{itemize}
\begin{itemize}
\item {Proveniência:(Lat. \textunderscore dominicalis\textunderscore )}
\end{itemize}
Relativo a quem tem o domínio, ao senhor.
Relativo a Deus.
Relativo ao Domingo e á letra que no calendário designa o Domingo.
\section{Dominicanas}
\begin{itemize}
\item {Grp. gram.:f. pl.}
\end{itemize}
Congregação de religiosas, fundada por San-Domingos, em 1206 e reformada por Santa Catherina de Sena, no século XVI.
\section{Dominicano}
\begin{itemize}
\item {Grp. gram.:m.}
\end{itemize}
\begin{itemize}
\item {Grp. gram.:Adj.}
\end{itemize}
\begin{itemize}
\item {Proveniência:(Do lat. \textunderscore dominicus\textunderscore )}
\end{itemize}
Frade da Ordem de San-Domingos, fundada em 1215.
Indivíduo, natural da república dominicana.
Relativo á Ordem de San-Domingos.
Relativo á república americana de San-Domingos ou de Santo-Domingo.
\section{Dominíco}
\begin{itemize}
\item {Grp. gram.:m.}
\end{itemize}
O mesmo que frade \textunderscore dominicano\textunderscore .
\section{Dominim}
\begin{itemize}
\item {Grp. gram.:m.}
\end{itemize}
Árvore da Índia portuguesa.
\section{Domínio}
\begin{itemize}
\item {Grp. gram.:m.}
\end{itemize}
\begin{itemize}
\item {Proveniência:(Lat. \textunderscore dominium\textunderscore )}
\end{itemize}
O mesmo que \textunderscore dominação\textunderscore .
Faculdade de dispor de alguma coisa, como senhor della.
Qualidade de proprietário.
Propriedade: \textunderscore têr o domínio de um prédio\textunderscore .
Território extenso, que pertence a um indivíduo ou ao Estado.
Espaço occupado.
Autoridade.
Possessão: \textunderscore os nossos domínios ultramarinos\textunderscore .
Aquillo que uma arte ou sciência comprehende: \textunderscore no domínio da Phýsica\textunderscore .
Pertença.
Esphera de acção.
\textunderscore Domínio directo\textunderscore , o do senhorio que recebe o foro de um prazo.
\textunderscore Domínio útil\textunderscore , o do emphyteuta ou subemphyteuta, isto é, do que paga foro de um prazo.
\section{Dominó}
\begin{itemize}
\item {Grp. gram.:m.}
\end{itemize}
\begin{itemize}
\item {Proveniência:(T. fr.)}
\end{itemize}
Traje de pessôa mascarada, formado de uma longa túnica, com capuz e mangas.
Pessôa, que veste êsse traje.
Jôgo, composto de 28 peças, geralmente de madeira, com pontos marcados, podendo com ellas formar-se variadas combinações de números, desde um até seis.
\section{Dom-jorge}
\begin{itemize}
\item {Grp. gram.:m.}
\end{itemize}
Planta malgipiácea da Índia portuguesa, (gaertnera racemosa, Roxb.).
\section{Domnío}
\begin{itemize}
\item {Grp. gram.:m.}
\end{itemize}
\begin{itemize}
\item {Utilização:Prov.}
\end{itemize}
O mesmo que \textunderscore domínio\textunderscore .
\section{Dómo}
\begin{itemize}
\item {Grp. gram.:m.}
\end{itemize}
\begin{itemize}
\item {Proveniência:(Do lat. \textunderscore domus\textunderscore )}
\end{itemize}
Parte superior de um edifício, a qual tem fórma esphérica ou convexa, e cuja parte inferior, côncava, é a cúpula.
Zimbório.
\section{Dom-pedro}
\begin{itemize}
\item {Grp. gram.:m.}
\end{itemize}
Variedade de videira brasileira.
\section{Dom-quixotismo}
\begin{itemize}
\item {Grp. gram.:m.}
\end{itemize}
Modos ou hábitos, parecidos aos de Dom Quixote. Cf. Castilho, \textunderscore D. Quixote\textunderscore , XVII e XVIII.
\section{Dom-solidom}
\begin{itemize}
\item {Grp. gram.:m.}
\end{itemize}
\begin{itemize}
\item {Utilização:Prov.}
\end{itemize}
Dança de roda.
(Das notas musicaes \textunderscore dó\textunderscore , \textunderscore sol\textunderscore  e \textunderscore dó\textunderscore ?)
\section{Dona}
\begin{itemize}
\item {Grp. gram.:f.}
\end{itemize}
\begin{itemize}
\item {Utilização:Des.}
\end{itemize}
\begin{itemize}
\item {Utilização:Bras. de Minas}
\end{itemize}
\begin{itemize}
\item {Proveniência:(Do lat. \textunderscore domina\textunderscore )}
\end{itemize}
Senhora de alguma coisa; proprietária.
Mulher, que governa ou administra (uma casa).
Governanta.
Título e tratamento honorífico, que precede os nomes próprios das senhoras.
Dama, senhora nobre.
O mesmo que \textunderscore espôsa\textunderscore : \textunderscore como passa a sua dona\textunderscore ?
\section{Dona-branca}
\begin{itemize}
\item {Grp. gram.:f.}
\end{itemize}
Variedade de uva branca.
\section{Dona-brites}
\begin{itemize}
\item {Grp. gram.:f.}
\end{itemize}
Casta de uva alentejana.
\section{Donácia}
\begin{itemize}
\item {Grp. gram.:f.}
\end{itemize}
\begin{itemize}
\item {Proveniência:(Do gr. \textunderscore donax\textunderscore )}
\end{itemize}
Mollusco acéphalo.
Gênero de insectos coleópteros tetrâmeros.
\section{Donadio}
\begin{itemize}
\item {Grp. gram.:m.}
\end{itemize}
\begin{itemize}
\item {Utilização:Ant.}
\end{itemize}
\begin{itemize}
\item {Proveniência:(Do lat. \textunderscore donativum\textunderscore )}
\end{itemize}
Donativo, dádiva.
\section{Dona-inês}
\begin{itemize}
\item {Grp. gram.:f.}
\end{itemize}
Variedade de pêra sucosa e aromática.
\section{Donaire}
\begin{itemize}
\item {Grp. gram.:m.}
\end{itemize}
\begin{itemize}
\item {Utilização:Des.}
\end{itemize}
Graça; garbo.
Gentileza.
Enfeite.
Crinolina.
(Cast. \textunderscore donaire\textunderscore )
\section{Donairear}
\begin{itemize}
\item {Grp. gram.:v. i.}
\end{itemize}
\begin{itemize}
\item {Proveniência:(De \textunderscore donaire\textunderscore )}
\end{itemize}
Apresentar-se com garbo.
Mostrar gentilêza.
\section{Donairíssimo}
\begin{itemize}
\item {Grp. gram.:adj.}
\end{itemize}
Muito donairoso. Cf. Filinto, XI, 107.
(Sup. mal formado, de \textunderscore donaire\textunderscore )
\section{Donairosamente}
\begin{itemize}
\item {Grp. gram.:adv.}
\end{itemize}
De modo donairoso.
Com donaire.
\section{Donairoso}
\begin{itemize}
\item {Grp. gram.:adj.}
\end{itemize}
Em que há donaire; que tem donaire.
\section{Dona-joana}
\begin{itemize}
\item {Grp. gram.:f.}
\end{itemize}
O mesmo que \textunderscore joana\textunderscore .
\section{Dona-joaquina}
\begin{itemize}
\item {Grp. gram.:f.}
\end{itemize}
Variedade de pêra sumarenta, também conhecida por pêra-de-água.
\section{Dona-maria-alonsa}
\begin{itemize}
\item {Grp. gram.:f.}
\end{itemize}
Espécie de jôgo popular.
\section{Donário}
\begin{itemize}
\item {Grp. gram.:m.}
\end{itemize}
\begin{itemize}
\item {Utilização:P. us.}
\end{itemize}
\begin{itemize}
\item {Proveniência:(Do lat. \textunderscore donarium\textunderscore )}
\end{itemize}
Lugar, onde se guardam presentes ou jóias.
Guarda-jóias. Cf. Castilho, \textunderscore Geórgicas\textunderscore , 215.
\section{Donatal}
\begin{itemize}
\item {Grp. gram.:adj.}
\end{itemize}
Relativo aos donatos.
\section{Donataria}
\begin{itemize}
\item {Grp. gram.:f.}
\end{itemize}
Jurisdição de um donatário.
\section{Donatário}
\begin{itemize}
\item {Grp. gram.:m.}
\end{itemize}
\begin{itemize}
\item {Proveniência:(Lat. \textunderscore donatarius\textunderscore )}
\end{itemize}
Aquelle que recebeu alguma doação.
\section{Donatismo}
\begin{itemize}
\item {Grp. gram.:m.}
\end{itemize}
Heresia dos donatistas.
\section{Donatista}
\begin{itemize}
\item {Grp. gram.:m.}
\end{itemize}
Membro de uma seita religiosa, que teve por fundador a Donato, bispo de Carthago.
\section{Donativo}
\begin{itemize}
\item {Grp. gram.:m.}
\end{itemize}
\begin{itemize}
\item {Proveniência:(Lat. \textunderscore donativum\textunderscore )}
\end{itemize}
Dom, dádiva.
Presente.
\section{Donato}
\begin{itemize}
\item {Grp. gram.:m.}
\end{itemize}
\begin{itemize}
\item {Proveniência:(Lat. \textunderscore donatus\textunderscore )}
\end{itemize}
Leigo, que servia num convento, e que de frade só tinha o hábito.
\section{Donde}
(contr. de \textunderscore de\textunderscore  + \textunderscore onde\textunderscore )
\section{Dondico}
\begin{itemize}
\item {Grp. gram.:m.}
\end{itemize}
Árvore da Índia portuguesa.
\section{Dondo}
\begin{itemize}
\item {Grp. gram.:adj.}
\end{itemize}
\begin{itemize}
\item {Utilização:Prov.}
\end{itemize}
\begin{itemize}
\item {Utilização:trasm.}
\end{itemize}
\begin{itemize}
\item {Proveniência:(Do lat. \textunderscore domitus\textunderscore )}
\end{itemize}
Nédio, macio, lustroso.
Brando, flexível.
Que não tem consistência, que está mal cozido, (falando-se do pão).
\section{Donear}
\begin{itemize}
\item {Grp. gram.:v. t.}
\end{itemize}
\begin{itemize}
\item {Proveniência:(De \textunderscore dona\textunderscore )}
\end{itemize}
Cortejar ou galantear (mulheres). Cf. Fr. Fortun., \textunderscore Inéd.\textunderscore , I, 305.
\section{Dongaluta}
\begin{itemize}
\item {Grp. gram.:f.}
\end{itemize}
Planta leguminosa de Angola, (\textunderscore dolichos dongaluta\textunderscore , Welw.).
\section{Dongo}
\begin{itemize}
\item {Grp. gram.:m.}
\end{itemize}
Barco africano, formado de um tronco de árvore.
\section{Doninha}
\begin{itemize}
\item {Grp. gram.:f.}
\end{itemize}
Pequeno mammífero, (\textunderscore mustela\textunderscore ).
Grande peixe marítimo do Brasil.
(Dem. de \textunderscore dona\textunderscore . Cp. gall. \textunderscore donacinha\textunderscore , furão)
\section{Donjuanesco}
\begin{itemize}
\item {fónica:nês}
\end{itemize}
\begin{itemize}
\item {Grp. gram.:adj.}
\end{itemize}
Que tem modos de \textunderscore Don Juan\textunderscore , typo espanhol do galanteador, por offício ou por hábito.
\section{Dono}
\begin{itemize}
\item {Grp. gram.:m.}
\end{itemize}
\begin{itemize}
\item {Proveniência:(Do lat. \textunderscore dominus\textunderscore )}
\end{itemize}
Proprietário.
Senhor.
Chefe (de uma casa).
\section{Donosamente}
\begin{itemize}
\item {Grp. gram.:adv.}
\end{itemize}
De modo donoso.
\section{Donoso}
\begin{itemize}
\item {Grp. gram.:adj.}
\end{itemize}
O mesmo que \textunderscore donairoso\textunderscore .
(Cast. \textunderscore donoso\textunderscore )
\section{Donzel}
\begin{itemize}
\item {Grp. gram.:adj.}
\end{itemize}
\begin{itemize}
\item {Grp. gram.:M.}
\end{itemize}
\begin{itemize}
\item {Utilização:Ant.}
\end{itemize}
\begin{itemize}
\item {Proveniência:(Do b. lat. \textunderscore domicellus\textunderscore )}
\end{itemize}
Dócil.
Ingênuo.
Puro.
Diz-se de uma variedade de pinheiro.
Espécie de pagem.
Criado de honra.
Moço nobre e galante.
\section{Donzela}
\begin{itemize}
\item {Grp. gram.:f.}
\end{itemize}
\begin{itemize}
\item {Utilização:Ant.}
\end{itemize}
\begin{itemize}
\item {Utilização:Des.}
\end{itemize}
\begin{itemize}
\item {Grp. gram.:Adj.}
\end{itemize}
\begin{itemize}
\item {Proveniência:(Do b. lat. \textunderscore domicella\textunderscore )}
\end{itemize}
Mulher solteira.
Virgem.
Aia.
Criada de honra.
Banca de quarto, junto á cabeceira da cama.
Nome de um peixe, (\textunderscore molva vulgaris\textunderscore ).
Solteira.
Virginal.
\section{Donzelaria}
\begin{itemize}
\item {Grp. gram.:f.}
\end{itemize}
Acompanhamento ou comitiva de donzelas.
\section{Donzelha}
\begin{itemize}
\item {fónica:zê}
\end{itemize}
\begin{itemize}
\item {Grp. gram.:f.}
\end{itemize}
\begin{itemize}
\item {Utilização:T. de Turquel}
\end{itemize}
Planta bulbosa.
\section{Donzelinha}
\begin{itemize}
\item {Grp. gram.:f.}
\end{itemize}
Insecto ortóptero, (\textunderscore libellulavirgo\textunderscore ), também conhecido por \textunderscore libelinha\textunderscore , \textunderscore libélula\textunderscore  e \textunderscore lavadeira\textunderscore .
\section{Donzelinho}
\begin{itemize}
\item {Grp. gram.:m.}
\end{itemize}
Designação de algumas espécies de uva, na região do Doiro.
\section{Donzelinho-branco}
\begin{itemize}
\item {Grp. gram.:m.}
\end{itemize}
Variedade de donzelinho.
\section{Donzelinho-do-castello}
\begin{itemize}
\item {Grp. gram.:m.}
\end{itemize}
Variedade de uva preta da região do Doiro.
\section{Donzelinho-gallego}
\begin{itemize}
\item {Grp. gram.:m.}
\end{itemize}
Casta de uva preta, na região do Doiro.
\section{Donzelinho-malhado}
\begin{itemize}
\item {Grp. gram.:m.}
\end{itemize}
Casta de uva, da região do Doiro.
\section{Donzella}
\begin{itemize}
\item {Grp. gram.:f.}
\end{itemize}
\begin{itemize}
\item {Utilização:Ant.}
\end{itemize}
\begin{itemize}
\item {Utilização:Des.}
\end{itemize}
\begin{itemize}
\item {Grp. gram.:Adj.}
\end{itemize}
\begin{itemize}
\item {Proveniência:(Do b. lat. \textunderscore domicella\textunderscore )}
\end{itemize}
Mulher solteira.
Virgem.
Aia.
Criada de honra.
Banca de quarto, junto á cabeceira da cama.
Nome de um peixe, (\textunderscore molva vulgaris\textunderscore ).
Solteira.
Virginal.
\section{Donzellaria}
\begin{itemize}
\item {Grp. gram.:f.}
\end{itemize}
Acompanhamento ou comitiva de donzellas.
\section{Donzella-verde}
\begin{itemize}
\item {Grp. gram.:f.}
\end{itemize}
\begin{itemize}
\item {Utilização:Prov.}
\end{itemize}
O mesmo que \textunderscore abibe\textunderscore .
\section{Donzellinha}
\begin{itemize}
\item {Grp. gram.:f.}
\end{itemize}
Insecto orthóptero, (\textunderscore libellulavirgo\textunderscore ), também conhecido por \textunderscore libellinha\textunderscore , \textunderscore libéllula\textunderscore  e \textunderscore lavadeira\textunderscore .
\section{Donzello}
\begin{itemize}
\item {Grp. gram.:m.}
\end{itemize}
O mesmo que \textunderscore donzellinha\textunderscore .
\section{Donzellona}
\begin{itemize}
\item {Grp. gram.:f.}
\end{itemize}
\begin{itemize}
\item {Utilização:Fam.}
\end{itemize}
\begin{itemize}
\item {Proveniência:(De \textunderscore donzella\textunderscore )}
\end{itemize}
Mulher solteira, que já não é moça; solteirona.
\section{Donzelo}
\begin{itemize}
\item {Grp. gram.:m.}
\end{itemize}
O mesmo que \textunderscore donzelinha\textunderscore .
\section{Donzelona}
\begin{itemize}
\item {Grp. gram.:f.}
\end{itemize}
\begin{itemize}
\item {Utilização:Fam.}
\end{itemize}
\begin{itemize}
\item {Proveniência:(De \textunderscore donzela\textunderscore )}
\end{itemize}
Mulher solteira, que já não é moça; solteirona.
\section{Donzilha}
\begin{itemize}
\item {Grp. gram.:f.}
\end{itemize}
\begin{itemize}
\item {Utilização:Açor}
\end{itemize}
O mesmo que \textunderscore donzella\textunderscore .
\section{Door}
\begin{itemize}
\item {Grp. gram.:f.}
\end{itemize}
\begin{itemize}
\item {Utilização:Ant.}
\end{itemize}
\begin{itemize}
\item {Proveniência:(Do lat. \textunderscore dolor\textunderscore )}
\end{itemize}
O mesmo que \textunderscore dôr\textunderscore .
\section{Dopo}
\begin{itemize}
\item {Grp. gram.:m.}
\end{itemize}
\begin{itemize}
\item {Utilização:Ant.}
\end{itemize}
Tenda do commandante em acampamento militar, na China. Cf. \textunderscore Peregrinação\textunderscore , CXVIII.
\section{Doque}
\begin{itemize}
\item {Grp. gram.:m.}
\end{itemize}
Espécie de macaco asiático, (\textunderscore simia nemoeus\textunderscore ).
\section{Dôr}
\begin{itemize}
\item {Grp. gram.:f.}
\end{itemize}
\begin{itemize}
\item {Utilização:Fig.}
\end{itemize}
\begin{itemize}
\item {Utilização:Gír.}
\end{itemize}
\begin{itemize}
\item {Utilização:Pop.}
\end{itemize}
\begin{itemize}
\item {Proveniência:(Lat. \textunderscore dolor\textunderscore )}
\end{itemize}
Impressão desagradável ou penosa, produzida por lesão, por contusão ou por um estado anómalo do organismo ou de uma parte do organismo.
Soffrimento phýsico.
Soffrimento moral.
Condolência.
Dó.
Remorso.
Manifestação de sentimento.
Ciúme.
\textunderscore Dôr de cotovelo\textunderscore , ciúme.
\section{Dór}
\begin{itemize}
\item {Grp. gram.:m.}
\end{itemize}
Oração, que os Parses fazem a Deus, ao meio-dia. Cf. Barros, \textunderscore Déc.\textunderscore  II, l. X, c. VI.
\section{Dora}
\begin{itemize}
\item {Grp. gram.:f.}
\end{itemize}
\begin{itemize}
\item {Proveniência:(Do ár. \textunderscore doura\textunderscore )}
\end{itemize}
Espécie de sorgo ou de milho da Índia.
\section{Dòravante}
\begin{itemize}
\item {Grp. gram.:adv.}
\end{itemize}
\begin{itemize}
\item {Proveniência:(De \textunderscore de\textunderscore  + \textunderscore ora\textunderscore  + \textunderscore avante\textunderscore )}
\end{itemize}
Daqui em deante.
Para o futuro.
\section{Dórcada}
\begin{itemize}
\item {Grp. gram.:f.}
\end{itemize}
\begin{itemize}
\item {Proveniência:(Do gr. \textunderscore dorkas\textunderscore )}
\end{itemize}
Espécie de cabra silvestre.
\section{Dórea}
\begin{itemize}
\item {Grp. gram.:f.}
\end{itemize}
\begin{itemize}
\item {Utilização:Des.}
\end{itemize}
Pano de algodão, que vinha da Índia.
\section{Dória}
\begin{itemize}
\item {Grp. gram.:f.}
\end{itemize}
\begin{itemize}
\item {Utilização:Des.}
\end{itemize}
Planta vulnerária.
\section{Dórico}
\begin{itemize}
\item {Grp. gram.:adj.}
\end{itemize}
\begin{itemize}
\item {Grp. gram.:M.}
\end{itemize}
\begin{itemize}
\item {Proveniência:(Gr. \textunderscore dorikos\textunderscore )}
\end{itemize}
Relativo aos Dórios.
Diz-se especialmente de uma das ordens clássicas de architectura.
Dialecto dos Dórios.
\section{Doridamente}
\begin{itemize}
\item {Grp. gram.:adv.}
\end{itemize}
De modo dorido.
\section{Dóride}
\begin{itemize}
\item {Grp. gram.:f.}
\end{itemize}
\begin{itemize}
\item {Proveniência:(Do lat. \textunderscore doris\textunderscore )}
\end{itemize}
Gênero de molluscos gasterópodes.
\section{Dorido}
\begin{itemize}
\item {Grp. gram.:adj.}
\end{itemize}
\begin{itemize}
\item {Utilização:Fig.}
\end{itemize}
\begin{itemize}
\item {Grp. gram.:M.}
\end{itemize}
\begin{itemize}
\item {Proveniência:(De \textunderscore dôr\textunderscore )}
\end{itemize}
Que tem dôr.
Em que há dôr: \textunderscore um braço muito dorido\textunderscore .
Magoado.
Consternado.
Triste.
Compadecido.
Indivíduo, a quem, recentemente morreu pessôa de família. Cf. Camillo, \textunderscore Maria da Fonte\textunderscore , 39.
\section{Dorífora}
\begin{itemize}
\item {Grp. gram.:f.}
\end{itemize}
Insecto microscópico, que ataca os batataes. Cf. \textunderscore Gaz. das Aldeias\textunderscore , de Julho de 1902.
\section{Dório}
\begin{itemize}
\item {Grp. gram.:m.  e  adj.}
\end{itemize}
(V.dórico)
\section{Dórios}
\begin{itemize}
\item {Grp. gram.:m. pl.}
\end{itemize}
\begin{itemize}
\item {Proveniência:(Gr. \textunderscore Dorieis\textunderscore )}
\end{itemize}
Povos da Dórida, na Grécia.
\section{Dóris}
\begin{itemize}
\item {Grp. gram.:m.}
\end{itemize}
Mollusco sem concha, que se arrasta sôbre as algas marinhas. Cf. R. Ortigão, \textunderscore Praias\textunderscore , 8.
\section{Dormentar}
\textunderscore v. t.\textunderscore  (e der.)
O mesmo que \textunderscore adormentar\textunderscore , etc. Cf. Filinto, I, 188.
\section{Dormente}
\begin{itemize}
\item {Grp. gram.:adj.}
\end{itemize}
\begin{itemize}
\item {Utilização:Fig.}
\end{itemize}
\begin{itemize}
\item {Grp. gram.:M.}
\end{itemize}
\begin{itemize}
\item {Proveniência:(De \textunderscore dormir\textunderscore )}
\end{itemize}
Que dorme.
Que está em torpor, em quietação.
Calmo.
Estagnado: \textunderscore águas dormentes\textunderscore .
Diz-se das plantas, cujas fôlhas se enrolam durante a noite.
Cada um dos paus da coberta de um navio.
Uma das peças da atafona.
Cada uma das travessas, em que assentam os carris das linhas férreas.
Cada uma das traves, em que se prega o soalho.
Trave.
\section{Dormida}
\begin{itemize}
\item {Grp. gram.:f.}
\end{itemize}
Estado de quem dorme.
Poisada, em que se pernoita.
O tempo em que se dorme.
\section{Dormideira}
\begin{itemize}
\item {Grp. gram.:f.}
\end{itemize}
\begin{itemize}
\item {Proveniência:(De \textunderscore dormir\textunderscore )}
\end{itemize}
Espécie de papoila, (\textunderscore papaver somniferum\textunderscore , Lin.).
Cápsula dessa planta, que tem qualidades sedativas e narcóticas.
\section{Dormidoiro}
\begin{itemize}
\item {Grp. gram.:m.}
\end{itemize}
\begin{itemize}
\item {Utilização:Des.}
\end{itemize}
(V.dormitório)
\section{Dormidor}
\begin{itemize}
\item {Grp. gram.:m.  e  adj.}
\end{itemize}
O mesmo que \textunderscore dorminhoco\textunderscore .
\section{Dormindinho}
\begin{itemize}
\item {Utilização:Bras}
\end{itemize}
É deminutivo do gerúndio \textunderscore dormindo\textunderscore , peculiar ao Brasil:«\textunderscore ...me folgo em vel-o dormindiño cal um anxel.\textunderscore »Saco Arce, \textunderscore Gram. Gallega.\textunderscore  Cf. Pacheco da Silva, \textunderscore Promptuário\textunderscore , 28.--O deminutivo \textunderscore dormindinho\textunderscore  denota meiguice ou carinho, e emprega-se falando de crianças: \textunderscore O Lulu está dormindinho\textunderscore .
\section{Dorminhoca}
\begin{itemize}
\item {Grp. gram.:f.}
\end{itemize}
Peixe dos Açores.
\section{Dorminhocamente}
\begin{itemize}
\item {Grp. gram.:adv.}
\end{itemize}
\begin{itemize}
\item {Proveniência:(De \textunderscore dorminhoco\textunderscore )}
\end{itemize}
Á maneira de pessôa dorminhoca.
\section{Dorminhoco}
\begin{itemize}
\item {fónica:nhô}
\end{itemize}
\begin{itemize}
\item {Grp. gram.:adj.}
\end{itemize}
\begin{itemize}
\item {Utilização:Fam.}
\end{itemize}
\begin{itemize}
\item {Grp. gram.:M.}
\end{itemize}
\begin{itemize}
\item {Utilização:Gír.}
\end{itemize}
\begin{itemize}
\item {Utilização:Bras}
\end{itemize}
\begin{itemize}
\item {Proveniência:(De \textunderscore dormir\textunderscore )}
\end{itemize}
Que dorme muito.
Ópio.
Passarinho, também chamado \textunderscore mano-tolo\textunderscore .
\section{Dorminte}
\begin{itemize}
\item {Grp. gram.:adj.}
\end{itemize}
Que está dormindo. Cf. Herculano, \textunderscore Lendas e Narrativas\textunderscore , II, 36.
\section{Dormir}
\begin{itemize}
\item {Grp. gram.:v. i.}
\end{itemize}
\begin{itemize}
\item {Grp. gram.:V. t.}
\end{itemize}
\begin{itemize}
\item {Grp. gram.:V. p.}
\end{itemize}
\begin{itemize}
\item {Utilização:Prov.}
\end{itemize}
\begin{itemize}
\item {Utilização:trasm.}
\end{itemize}
\begin{itemize}
\item {Grp. gram.:M.}
\end{itemize}
\begin{itemize}
\item {Proveniência:(Lat. \textunderscore dormire\textunderscore )}
\end{itemize}
Descansar no somno.
Passar as noites: \textunderscore dormir fóra de casa\textunderscore .
Estar immóvel.
Perder o movimento orgânico.
Descansar no somno da morte.
Tranquillizar-se.
Sêr descuidado.
Estar entorpecido.
Estar latente.
Passar (um espaço de tempo), dormindo: \textunderscore dormir seis horas\textunderscore .
O mesmo que \textunderscore adormecer\textunderscore .
Estado de quem dorme: \textunderscore tem um dormir agitado\textunderscore .
\section{Dormitar}
\begin{itemize}
\item {Grp. gram.:v. i.}
\end{itemize}
\begin{itemize}
\item {Utilização:Fig.}
\end{itemize}
\begin{itemize}
\item {Proveniência:(Lat. \textunderscore dormitare\textunderscore )}
\end{itemize}
Dormir levemente.
Têr um somno rápido.
Descansar.
\section{Dormitivo}
\begin{itemize}
\item {Grp. gram.:adj.}
\end{itemize}
\begin{itemize}
\item {Proveniência:(Do lat. \textunderscore dormitum\textunderscore )}
\end{itemize}
Que provoca o somno, que é narcótico.
Soporífero.
\section{Dormitólogo}
\begin{itemize}
\item {Grp. gram.:m.}
\end{itemize}
\begin{itemize}
\item {Proveniência:(Do lat. \textunderscore dormitum\textunderscore  + gr. \textunderscore logos\textunderscore )}
\end{itemize}
Aquelle que dá regra ou preceitos ácêrca do somno.--É termo hýbr., que julgo inventado por Camillo. Cf. \textunderscore Maria da Fonte\textunderscore , 1.^a ed., 413.
\section{Dormitório}
\begin{itemize}
\item {Grp. gram.:m.}
\end{itemize}
\begin{itemize}
\item {Proveniência:(Lat. \textunderscore dormitorium\textunderscore )}
\end{itemize}
Sala de uma communidade, em que há muitos leitos.
Corredor, ladeado de cellas.
\section{Dorna}
\begin{itemize}
\item {Grp. gram.:f.}
\end{itemize}
\begin{itemize}
\item {Utilização:Fam.}
\end{itemize}
\begin{itemize}
\item {Utilização:Prov.}
\end{itemize}
\begin{itemize}
\item {Utilização:dur.}
\end{itemize}
\begin{itemize}
\item {Proveniência:(Do b. lat. \textunderscore durna\textunderscore )}
\end{itemize}
Grande vasilha de aduelas sem tampa, e destinada á pisa das uvas ou ao transporte dellas para o lagar.
Mulher gorda e baixa.
Grande sorvedoiro, que fórma redemoínho na corrente do rio.
\section{Dornacho}
\begin{itemize}
\item {Grp. gram.:m.}
\end{itemize}
Pequena dorna.
\section{Dornada}
\begin{itemize}
\item {Grp. gram.:f.}
\end{itemize}
Aquillo que uma dorna póde conter.
\section{Dorneira}
\begin{itemize}
\item {Grp. gram.:f.}
\end{itemize}
\begin{itemize}
\item {Proveniência:(De \textunderscore dorna\textunderscore )}
\end{itemize}
O mesmo que \textunderscore canoira\textunderscore .
\section{Dorónico}
\begin{itemize}
\item {Grp. gram.:m.}
\end{itemize}
Gênero de plantas synanthéreas medicinaes, (\textunderscore doronicum plantagineum\textunderscore , Lin.).
\section{Dorosamente}
\begin{itemize}
\item {Grp. gram.:adv.}
\end{itemize}
\begin{itemize}
\item {Utilização:Ant.}
\end{itemize}
De modo doroso.
\section{Doroso}
\begin{itemize}
\item {Grp. gram.:adj.}
\end{itemize}
\begin{itemize}
\item {Utilização:Ant.}
\end{itemize}
\begin{itemize}
\item {Proveniência:(De \textunderscore dôr\textunderscore )}
\end{itemize}
O mesmo que \textunderscore doloroso\textunderscore .
\section{Dorsal}
\begin{itemize}
\item {Grp. gram.:adj.}
\end{itemize}
Relativo ao dorso.
\section{Dorsel}
\begin{itemize}
\item {Grp. gram.:m.}
\end{itemize}
\begin{itemize}
\item {Utilização:Ant.}
\end{itemize}
\begin{itemize}
\item {Utilização:Ant.}
\end{itemize}
O mesmo que \textunderscore dossel\textunderscore .
Espaldar ou costas da cadeira. Cf. Madureira Feijó, \textunderscore Ortogr.\textunderscore 
(Corr. de \textunderscore dorsal\textunderscore )
\section{Dorsibrânchias}
\begin{itemize}
\item {fónica:qui}
\end{itemize}
\begin{itemize}
\item {Grp. gram.:f. pl.}
\end{itemize}
Secção de crustáceos anelídeos.
\section{Dorsibrânquias}
\begin{itemize}
\item {Grp. gram.:f. pl.}
\end{itemize}
Secção de crustáceos anelídeos.
\section{Dorsífero}
\begin{itemize}
\item {Grp. gram.:adj.}
\end{itemize}
\begin{itemize}
\item {Utilização:Bot.}
\end{itemize}
\begin{itemize}
\item {Proveniência:(Lat. \textunderscore dorsifer\textunderscore )}
\end{itemize}
Diz-se das fôlhas, que têm sôbre o dorso os órgãos da frutificação.
\section{Dorsifixo}
\begin{itemize}
\item {Grp. gram.:adj.}
\end{itemize}
Fixo no dorso.
\section{Dorso}
\begin{itemize}
\item {Grp. gram.:m.}
\end{itemize}
\begin{itemize}
\item {Utilização:Fig.}
\end{itemize}
\begin{itemize}
\item {Proveniência:(Lat. \textunderscore dorsum\textunderscore )}
\end{itemize}
Parte posterior do corpo humano, entre os ombros e os rins.
Parte superior dos animaes, lombo.
Reverso; parte posterior.
Parte superior convexa.
\section{Dorýphora}
\begin{itemize}
\item {Grp. gram.:f.}
\end{itemize}
Insecto microscópico, que ataca os batataes. Cf. \textunderscore Gaz. das Aldeias\textunderscore , de Julho de 1902.
\section{Dos}
(contr. de \textunderscore de\textunderscore  + \textunderscore os\textunderscore )
\section{Dosa}
\begin{itemize}
\item {Grp. gram.:f.}
\end{itemize}
\begin{itemize}
\item {Utilização:Prov.}
\end{itemize}
\begin{itemize}
\item {Utilização:trasm.}
\end{itemize}
O mesmo que \textunderscore tosa\textunderscore ^2, tareia.
O mesmo que \textunderscore bebedeira\textunderscore .
\section{Dosagem}
\begin{itemize}
\item {Grp. gram.:f.}
\end{itemize}
Acto de dosar.
\section{Dosar}
\begin{itemize}
\item {Grp. gram.:v. t.}
\end{itemize}
\begin{itemize}
\item {Proveniência:(De \textunderscore dose\textunderscore )}
\end{itemize}
Combinar a mistura de.
Regular por dose.
Misturar em proporções devidas.
\section{Dose}
\begin{itemize}
\item {Grp. gram.:f.}
\end{itemize}
\begin{itemize}
\item {Proveniência:(Gr. \textunderscore dosis\textunderscore )}
\end{itemize}
Quantidade fixa de uma substância, que entra na composição de um medicamento ou numa combinação chímica.
Porção medicamentosa que se toma de uma vez.
Quantidade, porção.
\section{Doseamento}
\begin{itemize}
\item {Grp. gram.:m.}
\end{itemize}
O mesmo que \textunderscore dosagem\textunderscore . Cf. \textunderscore Techn. Rur.\textunderscore , 45.
\section{Dosear}
\textunderscore v. t.\textunderscore  (e der.)
O mesmo que \textunderscore dosar\textunderscore , etc.
\section{Dosificar}
\begin{itemize}
\item {Grp. gram.:v. t.}
\end{itemize}
\begin{itemize}
\item {Proveniência:(Do lat. \textunderscore dosis\textunderscore  + \textunderscore facere\textunderscore )}
\end{itemize}
Dividir em doses.
Reduzir a doses.
\section{Dosimetria}
\begin{itemize}
\item {Grp. gram.:f.}
\end{itemize}
\begin{itemize}
\item {Proveniência:(Do gr. \textunderscore dosis\textunderscore  + \textunderscore metron\textunderscore )}
\end{itemize}
Systema médico e pharmacêutico, que consiste em confeccionar e applicar medicamentos em fórma de grânulos, em que se contêm apenas os princípios activos das substâncias medicamentosas.
\section{Dosimétrico}
\begin{itemize}
\item {Grp. gram.:adj.}
\end{itemize}
Relativo á dosimetria.
\section{Dossel}
\begin{itemize}
\item {Grp. gram.:m.}
\end{itemize}
\begin{itemize}
\item {Utilização:Fig.}
\end{itemize}
\begin{itemize}
\item {Proveniência:(Lat. \textunderscore dossellum\textunderscore )}
\end{itemize}
Armação saliente, forrada de damasco ou de outro estôfo, e franjada, que se colloca como ornato sôbre altares, thronos, camas, etc.; sobrecéu.
Copa de verdura.
Cobertura ornamental, mais ou menos alta.
\section{Dotação}
\begin{itemize}
\item {Grp. gram.:f.}
\end{itemize}
\begin{itemize}
\item {Proveniência:(Lat. \textunderscore dotatio\textunderscore )}
\end{itemize}
Acto de dotar.
\section{Dotador}
\begin{itemize}
\item {Grp. gram.:m.}
\end{itemize}
Aquelle que dota.
\section{Dotal}
\begin{itemize}
\item {Grp. gram.:adj.}
\end{itemize}
Relativo a dote: \textunderscore escritura dotal\textunderscore .
\section{Dotalício}
\begin{itemize}
\item {Grp. gram.:adj.}
\end{itemize}
O mesmo que \textunderscore dotal\textunderscore .
\section{Dotar}
\begin{itemize}
\item {Grp. gram.:v. t.}
\end{itemize}
\begin{itemize}
\item {Utilização:Fig.}
\end{itemize}
\begin{itemize}
\item {Proveniência:(Lat. \textunderscore dotare\textunderscore )}
\end{itemize}
Dar dote a.
Dar em doação.
Beneficiar com algum dom natural: \textunderscore dotou-o a natureza de grande coragem\textunderscore .
\section{Dote}
\begin{itemize}
\item {Grp. gram.:m.}
\end{itemize}
\begin{itemize}
\item {Utilização:Fig.}
\end{itemize}
\begin{itemize}
\item {Utilização:Bras. do N}
\end{itemize}
\begin{itemize}
\item {Proveniência:(Do lat. \textunderscore dos\textunderscore , \textunderscore dotis\textunderscore )}
\end{itemize}
Bens próprios e exclusivos da mulher casada.
Bens, que a freira levava para o convento e que são administrados pelas communidades.
Quantia, que a Misericórdia de Lisbôa dá a algumas enjeitadas, quando se casam, se foram criadas por ella.
Dinheiro ou propriedades, que alguém dá, espontânea e gratuitamente a uma noiva ou a uns noivos, para accréscimo dos bens dêstes.
Merecimento; bôas qualidades; prenda: \textunderscore a Dora tem dotes apreciáveis\textunderscore .
Preço exorbitante.
\section{Dóti}
\begin{itemize}
\item {Grp. gram.:m.}
\end{itemize}
Medida africana de quatro jardas.
\section{Douda}
\begin{itemize}
\item {Grp. gram.:f.}
\end{itemize}
\begin{itemize}
\item {Proveniência:(De \textunderscore doudo\textunderscore )}
\end{itemize}
Moléstia, que dá nos miolos do gado lanigero.
\section{Doudamente}
\begin{itemize}
\item {Grp. gram.:adv.}
\end{itemize}
De modo doudo.
Com doudice.
Tolamente.
\section{Doudaria}
\begin{itemize}
\item {Grp. gram.:f.}
\end{itemize}
\begin{itemize}
\item {Proveniência:(De \textunderscore doudo\textunderscore )}
\end{itemize}
Doudice; os doudos. Cf. Filinto, XIII, 136.
\section{Doudarrão}
\begin{itemize}
\item {Grp. gram.:adj.}
\end{itemize}
\begin{itemize}
\item {Utilização:Pop.}
\end{itemize}
\begin{itemize}
\item {Proveniência:(De \textunderscore doudo\textunderscore )}
\end{itemize}
Muito doudo.
Idiota.
Pateta.
\section{Doudarraz}
\begin{itemize}
\item {Grp. gram.:m.}
\end{itemize}
O mesmo que \textunderscore doudarrão\textunderscore . Cf. Filinto, XIX, 247.
\section{Doudejante}
\begin{itemize}
\item {Grp. gram.:adj.}
\end{itemize}
Que doudeja.
\section{Doudejar}
\begin{itemize}
\item {Grp. gram.:v. i.}
\end{itemize}
\begin{itemize}
\item {Proveniência:(De \textunderscore doudo\textunderscore )}
\end{itemize}
Fazer doudices, loucuras.
Dizer ou fazer disparates.
Brincar.
Vagabundear.
\section{Doudejo}
\begin{itemize}
\item {Grp. gram.:m.}
\end{itemize}
Acto de doudejar.
\section{Doudelas}
\begin{itemize}
\item {Grp. gram.:m.}
\end{itemize}
\begin{itemize}
\item {Utilização:Prov.}
\end{itemize}
\begin{itemize}
\item {Utilização:trasm.}
\end{itemize}
\begin{itemize}
\item {Proveniência:(De \textunderscore doudo\textunderscore )}
\end{itemize}
Doudivanas, homem estavanado.
\section{Doudete}
\begin{itemize}
\item {fónica:dê}
\end{itemize}
\begin{itemize}
\item {Grp. gram.:m.}
\end{itemize}
Aquelle que tem pouco juízo:«\textunderscore o lobo apanha pelo pescoço o doudete\textunderscore ». Sá de Miranda.
\section{Doudice}
\begin{itemize}
\item {Grp. gram.:f.}
\end{itemize}
\begin{itemize}
\item {Proveniência:(De \textunderscore doudo\textunderscore )}
\end{itemize}
Estado de quem é doudo.
Palavras ou actos próprios de doudo.
Extravagância; excesso.
\section{Doudinha}
\begin{itemize}
\item {Grp. gram.:f.}
\end{itemize}
O mesmo que \textunderscore papa-formigas\textunderscore .
\section{Doudivana}
\begin{itemize}
\item {Grp. gram.:m.  e  f.}
\end{itemize}
O mesmo que \textunderscore doudivanas\textunderscore . Cf. J. Dinís, \textunderscore Fidalgos\textunderscore , I, 128.
\section{Doudivanas}
\begin{itemize}
\item {Grp. gram.:m.  e  f.}
\end{itemize}
\begin{itemize}
\item {Utilização:Fam.}
\end{itemize}
\begin{itemize}
\item {Proveniência:(Do rad. de \textunderscore doudo\textunderscore )}
\end{itemize}
Indivíduo leviano, imprudente.
Pateta.
\section{Doudo}
\begin{itemize}
\item {Grp. gram.:adj.}
\end{itemize}
\begin{itemize}
\item {Grp. gram.:M.}
\end{itemize}
\begin{itemize}
\item {Proveniência:(Do ingl. \textunderscore dold\textunderscore ?)}
\end{itemize}
Que não tem juízo.
Que perdeu o uso da razão.
Louco.
Extravagante.
Temerário.
Insensato.
Que é indício de falta de juízo ou de falta de prudência: \textunderscore doudas correrias\textunderscore .
Arrebatado, enthusiasta: \textunderscore sou doudo por flôres\textunderscore .
Muito contente.
Vaidoso.
Extraordináriamente affectuoso: \textunderscore é doudo por ella\textunderscore .
Indivíduo doudo.
\section{Doundo}
\begin{itemize}
\item {Grp. gram.:m.}
\end{itemize}
Árvore da Índia portuguesa.
\section{Dourada}
\begin{itemize}
\item {Grp. gram.:f.}
\end{itemize}
\begin{itemize}
\item {Proveniência:(De \textunderscore dourado\textunderscore )}
\end{itemize}
Nome de algumas espécies de peixes.
Variedade de uva beirôa e minhota.
\section{Douradilho}
\begin{itemize}
\item {Grp. gram.:adj.}
\end{itemize}
\begin{itemize}
\item {Utilização:Bras}
\end{itemize}
\begin{itemize}
\item {Proveniência:(De \textunderscore dourado\textunderscore )}
\end{itemize}
Diz-se dos cavallos de côr avermelhada.
Diz-se do cavallo castanho.
\section{Douradinha}
\begin{itemize}
\item {Grp. gram.:f.}
\end{itemize}
\begin{itemize}
\item {Proveniência:(De \textunderscore dourado\textunderscore )}
\end{itemize}
Espécie de fêto.
Ave pernalta, o mesmo que \textunderscore tarambola\textunderscore .
Planta, da fam. das compostas, (\textunderscore senecio incrassatus\textunderscore , Lin.).
Variedade de pêra.
Espécie de jôgo de cartas.
A dama de ouros, nesse jôgo.
\section{Dourado}
\begin{itemize}
\item {Grp. gram.:adj.}
\end{itemize}
\begin{itemize}
\item {Grp. gram.:M.}
\end{itemize}
\begin{itemize}
\item {Utilização:Prov.}
\end{itemize}
\begin{itemize}
\item {Proveniência:(De \textunderscore dourar\textunderscore )}
\end{itemize}
Revestido de camada de ouro: \textunderscore tecto dourado\textunderscore .
Que tem côr de ouro: \textunderscore cabello dourado\textunderscore .
Douradura.
Casta de uva preta de Collares.
Peixe de Portugal.
O mesmo que \textunderscore tarambola\textunderscore .
\section{Dourador}
\begin{itemize}
\item {Grp. gram.:m.}
\end{itemize}
\begin{itemize}
\item {Proveniência:(Do lat. \textunderscore deaurator\textunderscore )}
\end{itemize}
Aquelle que doura.
\section{Douradura}
\begin{itemize}
\item {Grp. gram.:f.}
\end{itemize}
Camada ou fôlha de ouro sôbre um objecto.
Coisa dourada.
Arte ou acto de dourar.
\section{Douramento}
\begin{itemize}
\item {Grp. gram.:m.}
\end{itemize}
Acto de dourar.
\section{Dourar}
\begin{itemize}
\item {Grp. gram.:v. t.}
\end{itemize}
\begin{itemize}
\item {Utilização:Fig.}
\end{itemize}
\begin{itemize}
\item {Proveniência:(Do lat. \textunderscore deaurare\textunderscore )}
\end{itemize}
Revestir com camada de ouro em fôlha ou em dissolução.
Dar a côr de ouro a.
Disfarçar: \textunderscore dourar a pílula\textunderscore .
Desculpar.
Tornar brilhante, formoso, feliz.
Adornar.
\section{Dous}
\begin{itemize}
\item {Grp. gram.:adj.}
\end{itemize}
\begin{itemize}
\item {Grp. gram.:M.}
\end{itemize}
\begin{itemize}
\item {Proveniência:(Do lat. \textunderscore duo\textunderscore )}
\end{itemize}
Diz-se do número cardinal, formado de um e mais um.
Segundo.
Algarismo, representativo dêsse número.
Carta de jogar ou peça do dominó, que tem dous pontos.
Aquelle ou aquillo que numa série de dous occupa o último lugar.
\section{Douseiro}
\begin{itemize}
\item {Grp. gram.:adj.}
\end{itemize}
\begin{itemize}
\item {Utilização:T. de Ceilão}
\end{itemize}
\begin{itemize}
\item {Proveniência:(De \textunderscore dous\textunderscore )}
\end{itemize}
O mesmo que \textunderscore segundo\textunderscore ^1.
\section{Doutamente}
\begin{itemize}
\item {Grp. gram.:adv.}
\end{itemize}
\begin{itemize}
\item {Proveniência:(De \textunderscore douto\textunderscore )}
\end{itemize}
Com erudição.
\section{Dou-te-lo-vivo}
\begin{itemize}
\item {Grp. gram.:m.}
\end{itemize}
\begin{itemize}
\item {Utilização:Prov.}
\end{itemize}
Jôgo popular, em que um palito acceso passa de mão em mão, perdendo no jôgo aquelle que deixou apagar o palito, antes de responder a certa pergunta.
(Colhido na Bairrada)
\section{Doutíloquo}
\begin{itemize}
\item {Grp. gram.:adj.}
\end{itemize}
\begin{itemize}
\item {Proveniência:(Do lat. \textunderscore doctus\textunderscore  + \textunderscore loqui\textunderscore . \textunderscore Doctiloquo\textunderscore  é fórma preferível)}
\end{itemize}
Que fala doutamente.
\section{Douto}
\begin{itemize}
\item {Grp. gram.:adj.}
\end{itemize}
\begin{itemize}
\item {Proveniência:(Do lat. \textunderscore doctus\textunderscore )}
\end{itemize}
Que aprendeu muito; erudito; muito instruído.
Em que se revela erudição: \textunderscore doutas annotações\textunderscore .
\section{Doutor}
\begin{itemize}
\item {Grp. gram.:m.}
\end{itemize}
\begin{itemize}
\item {Utilização:Ext.}
\end{itemize}
\begin{itemize}
\item {Utilização:Pop.}
\end{itemize}
\begin{itemize}
\item {Utilização:Fam.}
\end{itemize}
\begin{itemize}
\item {Utilização:Prov.}
\end{itemize}
\begin{itemize}
\item {Utilização:alg.}
\end{itemize}
\begin{itemize}
\item {Proveniência:(Do lat. \textunderscore doctor\textunderscore )}
\end{itemize}
Aquelle que ensina.
Homem erudito.
Aquelle que recebeu de uma Faculdade universitária a mais alta graduação desta.
Bacharel formado.
Advogado.
Médico.
Homem que tem presumpção de sábio.
O mesmo que \textunderscore bispote\textunderscore .
\section{Doutora}
\begin{itemize}
\item {Grp. gram.:f.}
\end{itemize}
\begin{itemize}
\item {Utilização:Fam.}
\end{itemize}
\begin{itemize}
\item {Utilização:Gír.}
\end{itemize}
Mulher, que recebeu o grau de doutor: \textunderscore a doutora Carolina Michaëlis\textunderscore .
Mulher sabichona.
Cabeça.
(Fem. de \textunderscore doutor\textunderscore )
\section{Doutoraço}
\begin{itemize}
\item {Grp. gram.:m.}
\end{itemize}
\begin{itemize}
\item {Utilização:Pop.}
\end{itemize}
\begin{itemize}
\item {Proveniência:(De \textunderscore doutor\textunderscore )}
\end{itemize}
Homem, que ridiculamente presume de sábio.
\section{Doutorado}
\begin{itemize}
\item {Grp. gram.:m.}
\end{itemize}
\begin{itemize}
\item {Proveniência:(De \textunderscore doutorar\textunderscore )}
\end{itemize}
Graduação de doutor.
\section{Doutoral}
\begin{itemize}
\item {Grp. gram.:adj.}
\end{itemize}
\begin{itemize}
\item {Grp. gram.:M. pl.}
\end{itemize}
\begin{itemize}
\item {Proveniência:(De \textunderscore doutor\textunderscore )}
\end{itemize}
Relativo a doutor: \textunderscore o capello doutoral\textunderscore .
Próprio de doutor: \textunderscore gravidade doutoral\textunderscore .
Bancada, em que se assentam os doutores, na sala dos capellos da Universidade.
\section{Doutoramento}
\begin{itemize}
\item {Grp. gram.:m.}
\end{itemize}
Acto ou effeito de doutorar.
\section{Doutorando}
\begin{itemize}
\item {Grp. gram.:m.}
\end{itemize}
\begin{itemize}
\item {Proveniência:(De \textunderscore doutorar\textunderscore )}
\end{itemize}
Aquelle que se prepara para receber o grau de doutor.
\section{Doutorão}
\begin{itemize}
\item {Grp. gram.:m.}
\end{itemize}
\begin{itemize}
\item {Utilização:Irón.}
\end{itemize}
Grande doutor. Cf. Castilho, \textunderscore Sabichonas\textunderscore , 46.
\section{Doutorar}
\begin{itemize}
\item {Grp. gram.:v. t.}
\end{itemize}
\begin{itemize}
\item {Grp. gram.:V. p.}
\end{itemize}
Dar o grau de doutor.
Receber o grau de doutor.
\section{Doutorato}
\begin{itemize}
\item {Grp. gram.:m.}
\end{itemize}
(V.doutorado)
\section{Doutorice}
\begin{itemize}
\item {Grp. gram.:f.}
\end{itemize}
\begin{itemize}
\item {Utilização:Deprec.}
\end{itemize}
\begin{itemize}
\item {Proveniência:(De \textunderscore doutor\textunderscore )}
\end{itemize}
Modos de doutor. Parlapatice; ditos de sabichão.
\section{Doutrina}
\begin{itemize}
\item {Grp. gram.:f.}
\end{itemize}
\begin{itemize}
\item {Proveniência:(Do lat. \textunderscore doctrina\textunderscore )}
\end{itemize}
Conjunto de princípios, em que se baseia um systema religioso, político ou philosóphico.
Opinião, em assumptos scientíficos.
Disciplina.
Instrucção.
Modo de pensar, de proceder.
\section{Doutrinação}
\begin{itemize}
\item {Grp. gram.:f.}
\end{itemize}
Acto, ou effeito de doutrinar.
\section{Doutrinador}
\begin{itemize}
\item {Grp. gram.:m}
\end{itemize}
Aquelle que doutrina.
\section{Doutrinal}
\begin{itemize}
\item {Grp. gram.:adj.}
\end{itemize}
\begin{itemize}
\item {Proveniência:(Do lat. \textunderscore doctrinalis\textunderscore )}
\end{itemize}
Relativo a doutrina.
\section{Doutrinalmente}
\begin{itemize}
\item {Grp. gram.:adv.}
\end{itemize}
De modo doutrinal.
\section{Doutrinamento}
\begin{itemize}
\item {Grp. gram.:m.}
\end{itemize}
O mesmo que \textunderscore doutrinação\textunderscore .
\section{Doutrinando}
\begin{itemize}
\item {Grp. gram.:adj.}
\end{itemize}
\begin{itemize}
\item {Proveniência:(De \textunderscore doutrinar\textunderscore )}
\end{itemize}
Aquelle que recebe doutrina; aquelle, que anda apprendendo.
\section{Doutrinante}
\begin{itemize}
\item {Grp. gram.:m.}
\end{itemize}
O mesmo que \textunderscore doutrinador\textunderscore .
\section{Doutrinar}
\begin{itemize}
\item {Grp. gram.:v. t.}
\end{itemize}
Transmittir doutrina a.
Ensinar.
\section{Doutrinariamente}
\begin{itemize}
\item {Grp. gram.:adv.}
\end{itemize}
Segundo o systema dos doutrinários.
\section{Doutrinário}
\begin{itemize}
\item {Grp. gram.:m.}
\end{itemize}
\begin{itemize}
\item {Grp. gram.:Adj.}
\end{itemize}
\begin{itemize}
\item {Proveniência:(De \textunderscore doutrina\textunderscore )}
\end{itemize}
Sectário de um systema político, que em França occupava o meio termo entre a democracia e o tradicionalismo.
O mesmo que \textunderscore doutrinal\textunderscore .
\section{Doutrinarismo}
\begin{itemize}
\item {Grp. gram.:m.}
\end{itemize}
\begin{itemize}
\item {Proveniência:(De \textunderscore doutrina\textunderscore )}
\end{itemize}
Systema político, que em França occupava o meio termo entre a democracia e o tradicionalismo.
\section{Doutrinável}
\begin{itemize}
\item {Grp. gram.:adj.}
\end{itemize}
Que se póde doutrinar; dócil.
\section{Doutrineiro}
\begin{itemize}
\item {Grp. gram.:m.}
\end{itemize}
\begin{itemize}
\item {Utilização:Deprec.}
\end{itemize}
\begin{itemize}
\item {Grp. gram.:Adj.}
\end{itemize}
\begin{itemize}
\item {Proveniência:(De \textunderscore doutrinar\textunderscore )}
\end{itemize}
Aquelle que doutrina, que ensina.
Que espalha doutrinas.
\section{Doxógrafo}
\begin{itemize}
\item {fónica:csó}
\end{itemize}
\begin{itemize}
\item {Grp. gram.:m.}
\end{itemize}
Nome, que se deu aos compiladores gregos, que reuniam extractos dos filósofos antigos.
\section{Doxógrapho}
\begin{itemize}
\item {fónica:csó}
\end{itemize}
\begin{itemize}
\item {Grp. gram.:m.}
\end{itemize}
Nome, que se deu aos compiladores gregos, que reuniam extractos dos philósophos antigos.
\section{Doxomania}
\begin{itemize}
\item {Grp. gram.:f.}
\end{itemize}
Paixão de glória.
\section{Dozão}
\begin{itemize}
\item {Grp. gram.:m.}
\end{itemize}
\begin{itemize}
\item {Proveniência:(De \textunderscore doze\textunderscore )}
\end{itemize}
Antiga medida para líquidos, correspondente á duodécima parte do almude ou a uma canada.
Antiga medida para cereaes, correspondente á duodécima parte de um moio ou a cinco alqueires.
\section{Doze}
\begin{itemize}
\item {fónica:dô}
\end{itemize}
\begin{itemize}
\item {Grp. gram.:adj.}
\end{itemize}
\begin{itemize}
\item {Grp. gram.:M.}
\end{itemize}
\begin{itemize}
\item {Proveniência:(Do lat. \textunderscore duodecim\textunderscore )}
\end{itemize}
Diz-se do número cardinal, formado de déz e mais dois ou de duas vezes seis.
Duodécimo:«\textunderscore no doze capítulo de Tobias...\textunderscore »Azurara.
Aquelle ou aquillo que tem o duodécimo lugar numa série.
\section{Dozena}
\begin{itemize}
\item {Grp. gram.:f.}
\end{itemize}
\begin{itemize}
\item {Utilização:Mús.}
\end{itemize}
Registo de órgão, também chamado \textunderscore quinta real\textunderscore .
(\textunderscore De dozeno\textunderscore )
\section{Dozeno}
\begin{itemize}
\item {Grp. gram.:adj.}
\end{itemize}
\begin{itemize}
\item {Utilização:Des.}
\end{itemize}
\begin{itemize}
\item {Proveniência:(De \textunderscore doze\textunderscore )}
\end{itemize}
O mesmo que duodécimo.
\section{Dr.}
(Abrev. de \textunderscore doutor\textunderscore )
\section{Dracena}
\begin{itemize}
\item {Grp. gram.:f.}
\end{itemize}
O mesmo ou melhor que \textunderscore dracina\textunderscore .
\section{Drachma}
\begin{itemize}
\item {Grp. gram.:f.}
\end{itemize}
\begin{itemize}
\item {Utilização:Pharm.}
\end{itemize}
\begin{itemize}
\item {Proveniência:(Gr. \textunderscore drakhme\textunderscore )}
\end{itemize}
Moéda antiga, entre os Gregos e os Romanos, ainda hoje usada na Grécia e correspondente a 180 reis da nossa moéda.
Pêso antigo, em vários países.
Antigo pêso, igual á oitava.
\section{Dracina}
\begin{itemize}
\item {Grp. gram.:f.}
\end{itemize}
\begin{itemize}
\item {Proveniência:(Do gr. \textunderscore drakaina\textunderscore )}
\end{itemize}
Substância, que se extrái do sangue de drago.
\section{Dracma}
\begin{itemize}
\item {Grp. gram.:f.}
\end{itemize}
\begin{itemize}
\item {Utilização:Pharm.}
\end{itemize}
\begin{itemize}
\item {Proveniência:(Gr. \textunderscore drakhme\textunderscore )}
\end{itemize}
Moéda antiga, entre os Gregos e os Romanos, ainda hoje usada na Grécia e correspondente a 180 reis da nossa moéda.
Pêso antigo, em vários países.
Antigo pêso, igual á oitava.
\section{Dracocéfalo}
\begin{itemize}
\item {Grp. gram.:m.}
\end{itemize}
\begin{itemize}
\item {Proveniência:(Do gr. \textunderscore drakon\textunderscore  + \textunderscore kephale\textunderscore )}
\end{itemize}
Planta ornamental.
\section{Dracocéphalo}
\begin{itemize}
\item {Grp. gram.:m.}
\end{itemize}
\begin{itemize}
\item {Proveniência:(Do gr. \textunderscore drakon\textunderscore  + \textunderscore kephale\textunderscore )}
\end{itemize}
Planta ornamental.
\section{Dracogrifo}
\begin{itemize}
\item {Grp. gram.:m.}
\end{itemize}
Animal fabuloso, meio águia e meio dragão, representado na armaria.
(Fórma lat., do gr. \textunderscore drakon\textunderscore  + \textunderscore grups\textunderscore )
\section{Dracogrypho}
\begin{itemize}
\item {Grp. gram.:m.}
\end{itemize}
Animal fabuloso, meio águia e meio dragão, representado na armaria.
(Fórma lat., do gr. \textunderscore drakon\textunderscore  + \textunderscore grups\textunderscore )
\section{Draconário}
\begin{itemize}
\item {Grp. gram.:m.}
\end{itemize}
\begin{itemize}
\item {Utilização:Ant.}
\end{itemize}
\begin{itemize}
\item {Proveniência:(Lat. \textunderscore draconarius\textunderscore )}
\end{itemize}
O mesmo que \textunderscore porta-bandeira\textunderscore . Cf. \textunderscore Viriato Trág.\textunderscore , II, 107.
\section{Draconiano}
\begin{itemize}
\item {Grp. gram.:adj.}
\end{itemize}
\begin{itemize}
\item {Utilização:Ext.}
\end{itemize}
\begin{itemize}
\item {Proveniência:(De \textunderscore Dracon\textunderscore , n. p.)}
\end{itemize}
Relativo ás leis severas, promulgadas pelo legislador atheniense Dracon.
Excessivamente rigoroso.
Muito severo.
\section{Draconígena}
\begin{itemize}
\item {Grp. gram.:m.  e  f.}
\end{itemize}
\begin{itemize}
\item {Proveniência:(Lat. \textunderscore draconigena\textunderscore )}
\end{itemize}
Filho ou filha de dragão.
\section{Draconina}
\begin{itemize}
\item {Grp. gram.:f.}
\end{itemize}
O mesmo que \textunderscore dracina\textunderscore .
\section{Draconite}
\begin{itemize}
\item {Grp. gram.:f.}
\end{itemize}
\begin{itemize}
\item {Proveniência:(Do gr. \textunderscore drakon\textunderscore , dragão)}
\end{itemize}
Polypeiro fóssil.
\section{Draconteia}
\begin{itemize}
\item {Grp. gram.:f.}
\end{itemize}
O mesmo que \textunderscore dragonteia\textunderscore .
\section{Dracontíase}
\begin{itemize}
\item {Grp. gram.:f.}
\end{itemize}
O mesmo que \textunderscore dracunculose\textunderscore .
\section{Dracontite}
\begin{itemize}
\item {Grp. gram.:f.}
\end{itemize}
\begin{itemize}
\item {Proveniência:(Gr. \textunderscore drakontites\textunderscore )}
\end{itemize}
Jóia fabulosa, que se suppunha encontrar-se na cabeça do dragão, se a êste, em quanto vivo, fôsse extrahido o cérebro.
\section{Dracontocéfalo}
\begin{itemize}
\item {Grp. gram.:adj.}
\end{itemize}
\begin{itemize}
\item {Proveniência:(Do gr. \textunderscore drakon\textunderscore , \textunderscore drakontos\textunderscore  + \textunderscore kephalè\textunderscore )}
\end{itemize}
Que tem cabeça de dragão.
\section{Dracontocéphalo}
\begin{itemize}
\item {Grp. gram.:adj.}
\end{itemize}
\begin{itemize}
\item {Proveniência:(Do gr. \textunderscore drakon\textunderscore , \textunderscore drakontos\textunderscore  + \textunderscore kephalè\textunderscore )}
\end{itemize}
Que tem cabeça de dragão.
\section{Dracunculose}
\begin{itemize}
\item {Grp. gram.:f.}
\end{itemize}
\begin{itemize}
\item {Utilização:Med.}
\end{itemize}
\begin{itemize}
\item {Proveniência:(Do lat. \textunderscore dracunculus\textunderscore )}
\end{itemize}
Moléstia, causada pela chamada filária de Medina, e que se manifesta por um tumor subcutâneo, donde se póde extrahir o parasito.
\section{Draga}
\begin{itemize}
\item {Grp. gram.:f.}
\end{itemize}
\begin{itemize}
\item {Utilização:Prov.}
\end{itemize}
\begin{itemize}
\item {Utilização:dur.}
\end{itemize}
\begin{itemize}
\item {Grp. gram.:Pl.}
\end{itemize}
\begin{itemize}
\item {Proveniência:(Do ingl. \textunderscore drag\textunderscore )}
\end{itemize}
Apparelho, com que se tira areia, lodo, entulho, etc., do fundo dos rios, dos portos de mar, etc.
Cada um dos dois madeiros de castanho, que revestem, em todo comprimento a extremidade superior do costado do barco.
Peças madeira, com que se escora uma embarcação quando está em sêco.
\section{Dragador}
\begin{itemize}
\item {Grp. gram.:m.}
\end{itemize}
\begin{itemize}
\item {Proveniência:(De \textunderscore dragar\textunderscore )}
\end{itemize}
Aquelle que trabalha com draga.
\section{Dragagem}
\begin{itemize}
\item {Grp. gram.:f.}
\end{itemize}
Acto ou trabalho de dragar.
\section{Draganes}
\begin{itemize}
\item {Grp. gram.:m. pl.}
\end{itemize}
Antigos habitadores da Galliza.
\section{Dragão}
\begin{itemize}
\item {Grp. gram.:m.}
\end{itemize}
\begin{itemize}
\item {Utilização:Fig.}
\end{itemize}
\begin{itemize}
\item {Utilização:Prov.}
\end{itemize}
\begin{itemize}
\item {Utilização:beir.}
\end{itemize}
\begin{itemize}
\item {Proveniência:(Do lat. \textunderscore draco\textunderscore )}
\end{itemize}
Monstro fabuloso, que se representa com cauda de serpente, garras e asas.
Pessôa de má índole.
Diabo.
Antigo soldado de cavallaria, que também combatia de pé.
Constellação boreal.
Antiga peça de artilharia.
Cataracta, no cavallo.
Valentão.
\section{Dragão-marinho}
\begin{itemize}
\item {Grp. gram.:m.}
\end{itemize}
Peixe acanthopterýgio, da fam. dos pércidas.
\section{Dragar}
\begin{itemize}
\item {Grp. gram.:v. t.}
\end{itemize}
\begin{itemize}
\item {Proveniência:(De \textunderscore draga\textunderscore )}
\end{itemize}
Limpar com draga.
Rocegar.
\section{Drago}
\begin{itemize}
\item {Grp. gram.:m.}
\end{itemize}
O mesmo que \textunderscore dragão\textunderscore .
\textunderscore Sangue de drago\textunderscore , espécie de resina, extrahida de uma planta liliácea.
\section{Dragoeira}
\begin{itemize}
\item {Grp. gram.:f.}
\end{itemize}
Peixe do mar das Índias.
\section{Dragoeiro}
\begin{itemize}
\item {Grp. gram.:m.}
\end{itemize}
\begin{itemize}
\item {Proveniência:(De \textunderscore dragão\textunderscore )}
\end{itemize}
Planta liliácea, de que se extrai uma resina, conhecida por \textunderscore sangue de drago\textunderscore .
\section{Dragona}
\begin{itemize}
\item {Grp. gram.:f.}
\end{itemize}
\begin{itemize}
\item {Proveniência:(De \textunderscore dragão\textunderscore )}
\end{itemize}
Galão com franjas ou sem ellas, ou peça de metal amarelo, que os militares usam no ombro, como distinctivo.
\section{Dragonada}
\begin{itemize}
\item {Grp. gram.:f.}
\end{itemize}
\begin{itemize}
\item {Proveniência:(Fr. \textunderscore dragonade\textunderscore )}
\end{itemize}
Perseguição religiosa, movida contra os protestantes por Luis XIV, e em que se empregou a cavallaria de dragões.
\section{Dragonete}
\begin{itemize}
\item {fónica:nê}
\end{itemize}
\begin{itemize}
\item {Grp. gram.:m.}
\end{itemize}
\begin{itemize}
\item {Proveniência:(De \textunderscore dragão\textunderscore )}
\end{itemize}
Sýmbolo heráldico, que figura uma cabeça de dragão com a bôca aberta.
Peixe rei.
\section{Dragonteia}
\begin{itemize}
\item {Grp. gram.:f.}
\end{itemize}
\begin{itemize}
\item {Proveniência:(Lat. \textunderscore dracontea\textunderscore )}
\end{itemize}
Planta, cuja raíz tem semelhança com o dragão, e que é também conhecida por \textunderscore serpentária\textunderscore .
\section{Dragontino}
\begin{itemize}
\item {Grp. gram.:m.}
\end{itemize}
\begin{itemize}
\item {Proveniência:(Do lat. \textunderscore dracon\textunderscore , \textunderscore dracontis\textunderscore )}
\end{itemize}
Relativo a dragão.
\section{Drainador}
\begin{itemize}
\item {Grp. gram.:m.}
\end{itemize}
Aquelle que trabalha em drainagem.
\section{Drainagem}
\begin{itemize}
\item {Grp. gram.:f.}
\end{itemize}
\begin{itemize}
\item {Proveniência:(De \textunderscore drainar\textunderscore )}
\end{itemize}
Escoamento das águas de terrenos alagadiços, por meio de tubos, vallas ou fossos.
\section{Drainar}
\begin{itemize}
\item {Grp. gram.:v. t.}
\end{itemize}
\begin{itemize}
\item {Proveniência:(De \textunderscore draino\textunderscore )}
\end{itemize}
Praticar a drainagem em.--Melhor se diria \textunderscore gaivar\textunderscore .
\section{Draino}
\begin{itemize}
\item {Grp. gram.:m.}
\end{itemize}
\begin{itemize}
\item {Utilização:Neol.}
\end{itemize}
\begin{itemize}
\item {Proveniência:(Do ingl. \textunderscore drain\textunderscore )}
\end{itemize}
Valla para drainagem.
Tubo de barro, fabricado especialmente para drainagem.
\section{Draiva}
\begin{itemize}
\item {Grp. gram.:f.}
\end{itemize}
\begin{itemize}
\item {Utilização:Náut.}
\end{itemize}
Uma das velas de ré.
\section{Drama}
\begin{itemize}
\item {Grp. gram.:m.}
\end{itemize}
\begin{itemize}
\item {Proveniência:(Lat. \textunderscore drama\textunderscore )}
\end{itemize}
Peça theatral.
Composição theatral, que occupa, quanto á sua índole e fórma, o meio termo entre a tragédia e a comédia, quando não participa de ambas.
Acontecimento commovente.
Narrativa, que representa com vivacidade acontecimentos commoventes.
\section{Drama}
\begin{itemize}
\item {Grp. gram.:m.}
\end{itemize}
Pêso grego, correspondente a 0,0032^k.
\section{Dramadeira}
\begin{itemize}
\item {Grp. gram.:f.}
\end{itemize}
Escantilhão, com buracos proporcionados aos adarmes ou calibres das balas.
(Corr. de \textunderscore adarmadeira\textunderscore , de \textunderscore adarme\textunderscore )
\section{Dramalhão}
\begin{itemize}
\item {Grp. gram.:m.}
\end{itemize}
\begin{itemize}
\item {Utilização:Fam.}
\end{itemize}
\begin{itemize}
\item {Proveniência:(De \textunderscore drama\textunderscore )}
\end{itemize}
Drama de pouco merecimento, mas cheio de lances trágicos, ou que expõe actos de requintada perversidade. Cf. Castilho, \textunderscore Misanthropo\textunderscore , 32.
\section{Dramaticamente}
\begin{itemize}
\item {Grp. gram.:adv.}
\end{itemize}
De modo dramático.
\section{Dramaticidade}
\begin{itemize}
\item {Grp. gram.:f.}
\end{itemize}
\begin{itemize}
\item {Utilização:bras}
\end{itemize}
\begin{itemize}
\item {Utilização:Neol.}
\end{itemize}
Qualidade daquillo que é dramático.
\section{Dramático}
\begin{itemize}
\item {Grp. gram.:adj.}
\end{itemize}
\begin{itemize}
\item {Proveniência:(Lat. \textunderscore dramaticus\textunderscore )}
\end{itemize}
Relativo a drama.
Que é da natureza ou do gênero do drama.
Commovente.
\section{Dramatização}
\begin{itemize}
\item {Grp. gram.:f.}
\end{itemize}
Acto ou effeito de dramatizar. Cf. Camillo, \textunderscore Corja\textunderscore , 150.
\section{Dramatizar}
\begin{itemize}
\item {Grp. gram.:v. t.}
\end{itemize}
\begin{itemize}
\item {Proveniência:(Do gr. \textunderscore drama\textunderscore )}
\end{itemize}
Dar fórma de drama a.
Tornar dramático, interessante ou commovente, como um drama.
\section{Dramatologia}
\begin{itemize}
\item {Grp. gram.:f.}
\end{itemize}
\begin{itemize}
\item {Proveniência:(Do gr. \textunderscore drama\textunderscore  + \textunderscore logos\textunderscore )}
\end{itemize}
Arte dramática.
\section{Dramatológico}
\begin{itemize}
\item {Grp. gram.:adj.}
\end{itemize}
Relativo, á dramatologia. Cf. Camillo, \textunderscore Brasileira\textunderscore , 204.
\section{Dramaturgia}
\begin{itemize}
\item {Grp. gram.:f.}
\end{itemize}
\begin{itemize}
\item {Utilização:Des.}
\end{itemize}
\begin{itemize}
\item {Proveniência:(De \textunderscore dramaturgo\textunderscore )}
\end{itemize}
O mesmo que arte \textunderscore dramatologia\textunderscore .
\section{Dramaturgo}
\begin{itemize}
\item {Grp. gram.:m.}
\end{itemize}
\begin{itemize}
\item {Proveniência:(Gr. \textunderscore dramatourgos\textunderscore )}
\end{itemize}
Aquelle que escreve dramas.
\section{Dramma}
\begin{itemize}
\item {Grp. gram.:m.}
\end{itemize}
Pêso grego, correspondente a 0,0032^k.
\section{Dramo}
\begin{itemize}
\item {Grp. gram.:m.}
\end{itemize}
Moéda de Gôa, anterior á dominação portuguesa.
\section{Dranja}
\begin{itemize}
\item {Grp. gram.:f.}
\end{itemize}
(V.hydranja)
\section{Drastério}
\begin{itemize}
\item {Grp. gram.:m.}
\end{itemize}
\begin{itemize}
\item {Proveniência:(Gr. \textunderscore drasterios\textunderscore )}
\end{itemize}
Gênero de insectos coleópteros pentâmeros.
\section{Drástico}
\begin{itemize}
\item {Grp. gram.:adj.}
\end{itemize}
\begin{itemize}
\item {Grp. gram.:M.}
\end{itemize}
\begin{itemize}
\item {Proveniência:(Gr. \textunderscore drastikos\textunderscore )}
\end{itemize}
Diz-se de um purgante enérgico.
Purgante enérgico.
\section{Drávidas}
\begin{itemize}
\item {Grp. gram.:f. pl.}
\end{itemize}
Indígenas do sul da Índia e do norte de Ceilão.
\section{Dravídico}
\begin{itemize}
\item {Grp. gram.:m.}
\end{itemize}
\begin{itemize}
\item {Grp. gram.:Adj.}
\end{itemize}
\begin{itemize}
\item {Proveniência:(De \textunderscore Drávida\textunderscore , n. p.)}
\end{itemize}
Grupo das línguas indianas, falada no sul da Índia.
Relativo a essas línguas.
\section{Dreno}
\textunderscore m.\textunderscore  (e der.)
(V. \textunderscore draino\textunderscore , etc.)
\section{Dribo}
\begin{itemize}
\item {Grp. gram.:m.}
\end{itemize}
Macaco africano de cabeça preta.
\section{Driça}
\begin{itemize}
\item {Grp. gram.:f.}
\end{itemize}
\begin{itemize}
\item {Proveniência:(It. \textunderscore drizza\textunderscore )}
\end{itemize}
Corda, com que se içam pavilhões, vêrgas do navio, etc.
O mesmo que \textunderscore adriça\textunderscore .
\section{Dril}
\begin{itemize}
\item {Grp. gram.:m.}
\end{itemize}
Pano branco de linho, inglês, muito usado no Brasil em vestuário de homem.
\section{Drilo}
\begin{itemize}
\item {Grp. gram.:m.}
\end{itemize}
\begin{itemize}
\item {Proveniência:(Gr. \textunderscore drilos\textunderscore )}
\end{itemize}
Gênero de insectos coleópteros pentâmeros.
\section{Drímia}
\begin{itemize}
\item {Grp. gram.:f.}
\end{itemize}
\begin{itemize}
\item {Proveniência:(Do gr. \textunderscore drimus\textunderscore )}
\end{itemize}
Planta bulbosa, lilíácea, do Cabo da Boa-Esperança.
\section{Drímya}
\begin{itemize}
\item {Grp. gram.:f.}
\end{itemize}
\begin{itemize}
\item {Proveniência:(Do gr. \textunderscore drimus\textunderscore )}
\end{itemize}
Planta bulbosa, lilíácea, do Cabo da Boa-Esperança.
\section{Drinça}
\begin{itemize}
\item {Grp. gram.:f.}
\end{itemize}
\begin{itemize}
\item {Utilização:Prov.}
\end{itemize}
\begin{itemize}
\item {Utilização:dur.}
\end{itemize}
O mesmo que \textunderscore driça\textunderscore .
\section{Droca}
\begin{itemize}
\item {Grp. gram.:f.}
\end{itemize}
\begin{itemize}
\item {Proveniência:(T. ind.)}
\end{itemize}
Antiga moêda de Cambaia.
\section{Drofa}
\begin{itemize}
\item {Grp. gram.:f.}
\end{itemize}
\begin{itemize}
\item {Utilização:Gír. lisb.}
\end{itemize}
Vitrina.
Porta.
\section{Droga}
\begin{itemize}
\item {Grp. gram.:f.}
\end{itemize}
\begin{itemize}
\item {Utilização:Fam.}
\end{itemize}
\begin{itemize}
\item {Proveniência:(T. cast.)}
\end{itemize}
Qualquer substância ou ingrediente, applicado em tinturaria, pharmácia, etc.
Tecido ligeiro de seda ou lan.
Bagatela; nada: \textunderscore dar em droga\textunderscore , arruinar-se, dar em pantana; não têr resultado nenhum.
\section{Drogaria}
\begin{itemize}
\item {Grp. gram.:f.}
\end{itemize}
Porção de drogas.
Estabelecimento, em que se vendem drogas.
\section{Drogueta}
\begin{itemize}
\item {fónica:guê}
\end{itemize}
\begin{itemize}
\item {Grp. gram.:f.}
\end{itemize}
O mesmo que \textunderscore droguete\textunderscore .
\section{Droguete}
\begin{itemize}
\item {fónica:guê}
\end{itemize}
\begin{itemize}
\item {Grp. gram.:m.}
\end{itemize}
\begin{itemize}
\item {Proveniência:(De \textunderscore droga\textunderscore )}
\end{itemize}
Estôfo ordinário, geralmente de lan.
\section{Dromedário}
\begin{itemize}
\item {Grp. gram.:m.}
\end{itemize}
Espécie de camelo, de pescoço curto e uma só corcova.
(B. lat. \textunderscore dromadarius\textunderscore )
\section{Dromomania}
\begin{itemize}
\item {Grp. gram.:f.}
\end{itemize}
\begin{itemize}
\item {Utilização:Med.}
\end{itemize}
\begin{itemize}
\item {Proveniência:(Do gr. \textunderscore dromos\textunderscore  + \textunderscore mania\textunderscore )}
\end{itemize}
Impulsão á fuga.
Automatismo ambulatório.
Impulsão mórbida para andar.
\section{Dromómetro}
\begin{itemize}
\item {Grp. gram.:m.}
\end{itemize}
\begin{itemize}
\item {Proveniência:(Do gr. \textunderscore dromos\textunderscore  + \textunderscore metron\textunderscore )}
\end{itemize}
Apparelho, com que se mede uma distância percorrida.
\section{Dria}
\begin{itemize}
\item {Grp. gram.:f.}
\end{itemize}
O mesmo que \textunderscore dríada\textunderscore .
\section{Dríada}
\begin{itemize}
\item {Grp. gram.:f.}
\end{itemize}
\begin{itemize}
\item {Proveniência:(Do gr. \textunderscore druas\textunderscore )}
\end{itemize}
Antiga divindade silvestre.
Planta rosácea.
\section{Dríade}
\begin{itemize}
\item {Grp. gram.:f.}
\end{itemize}
O mesmo que \textunderscore dríada\textunderscore .
\section{Driádeas}
\begin{itemize}
\item {Grp. gram.:f. pl.}
\end{itemize}
Tríbo de plantas rosáceas, que tem por tipo a dríada.
\section{Drímis}
\begin{itemize}
\item {Grp. gram.:m.}
\end{itemize}
Planta ranunculácea.
\section{Drimónia}
\begin{itemize}
\item {Grp. gram.:f.}
\end{itemize}
\begin{itemize}
\item {Proveniência:(Do gr. \textunderscore drumos\textunderscore , floresta)}
\end{itemize}
Gênero de plantas gesneráceas.
\section{Driopiteco}
\begin{itemize}
\item {Grp. gram.:m.}
\end{itemize}
\begin{itemize}
\item {Proveniência:(Do gr. \textunderscore drus\textunderscore , \textunderscore druos\textunderscore , árvore, e \textunderscore píthekos\textunderscore , macaco)}
\end{itemize}
O mesmo que \textunderscore antropopiteco\textunderscore .
\section{Dromo}
\begin{itemize}
\item {Grp. gram.:m.}
\end{itemize}
\begin{itemize}
\item {Utilização:Ant.}
\end{itemize}
\begin{itemize}
\item {Proveniência:(Gr. \textunderscore dromos\textunderscore )}
\end{itemize}
Liça, estádio; corrida.
Avenida, ladeada de esphynges, que, nos templos egýcios, levava da entrada do recinto á fachada do edifício.
\section{Dromornito}
\begin{itemize}
\item {Grp. gram.:m.}
\end{itemize}
\begin{itemize}
\item {Proveniência:(Do gr. \textunderscore dromos\textunderscore  + \textunderscore ornis\textunderscore )}
\end{itemize}
Nome commum ás aves que não vôam, mas correm.
\section{Dromos}
\begin{itemize}
\item {Grp. gram.:m.}
\end{itemize}
\begin{itemize}
\item {Utilização:Ant.}
\end{itemize}
\begin{itemize}
\item {Proveniência:(Gr. \textunderscore dromos\textunderscore )}
\end{itemize}
Liça, estádio; corrida.
Avenida, ladeada de esphynges, que, nos templos egýcios, levava da entrada do recinto á fachada do edifício.
\section{Dromunda}
\begin{itemize}
\item {Grp. gram.:f.}
\end{itemize}
Antiga embarcação árabe, mais ligeira que as outras do seu tempo. Cf. Herculano, Hist. de Port., 4.^a ed., I, 438.
\section{Drongo}
\begin{itemize}
\item {Grp. gram.:m.}
\end{itemize}
Pássaro dentírostro, (\textunderscore edolius cristatus\textunderscore ).
\section{Dronte}
\begin{itemize}
\item {Grp. gram.:m.}
\end{itemize}
Ave giganteia, que existia em Madagáscar.
\section{Dropacismo}
\begin{itemize}
\item {Grp. gram.:m.}
\end{itemize}
\begin{itemize}
\item {Utilização:Ant.}
\end{itemize}
\begin{itemize}
\item {Proveniência:(Do lat. \textunderscore dopracismus\textunderscore )}
\end{itemize}
Certo medicamento depilatório.
\section{Dropaz}
\begin{itemize}
\item {Grp. gram.:m.}
\end{itemize}
\begin{itemize}
\item {Utilização:Ant.}
\end{itemize}
\begin{itemize}
\item {Proveniência:(Lat. \textunderscore dropax\textunderscore )}
\end{itemize}
Certo unguento depilatório.
\section{Drósera}
\begin{itemize}
\item {Grp. gram.:f.}
\end{itemize}
\begin{itemize}
\item {Proveniência:(Do gr. \textunderscore droseros\textunderscore , orvalho, por allusão aos pêlos das folhas, que segregam um líquido transparente)}
\end{itemize}
Planta herbácea, de fôlhas alternas.
\section{Droseráceas}
\begin{itemize}
\item {Grp. gram.:f. pl.}
\end{itemize}
Família de plantas, que tem por typo a drósera.
\section{Drosofilo}
\begin{itemize}
\item {Grp. gram.:m.}
\end{itemize}
\begin{itemize}
\item {Proveniência:(Do gr. \textunderscore drosos\textunderscore  + \textunderscore phullon\textunderscore )}
\end{itemize}
Gênero de plantas droseráceas.
\section{Drosometria}
\begin{itemize}
\item {Grp. gram.:f.}
\end{itemize}
Applicação do drosómetro.
\section{Drosométrico}
\begin{itemize}
\item {Grp. gram.:adj.}
\end{itemize}
Relativo á drosometria.
\section{Drosómetro}
\begin{itemize}
\item {Grp. gram.:m.}
\end{itemize}
\begin{itemize}
\item {Proveniência:(Do gr. \textunderscore drosos\textunderscore  + \textunderscore metron\textunderscore )}
\end{itemize}
Instrumento, com que se avalia o orvalho que se fórma diariamente.
\section{Drosophyllo}
\begin{itemize}
\item {Grp. gram.:m.}
\end{itemize}
\begin{itemize}
\item {Proveniência:(Do gr. \textunderscore drosos\textunderscore  + \textunderscore phullon\textunderscore )}
\end{itemize}
Gênero de plantas droseráceas.
\section{Drudária}
\begin{itemize}
\item {Grp. gram.:f.}
\end{itemize}
\begin{itemize}
\item {Utilização:Ant.}
\end{itemize}
Espécie de prestação ou presente obrigatório, que se dava ás mulheres dos senhores e dos juizes.
Acto de peitar, por intervenção de mulheres.
(B. lat. \textunderscore drudaria\textunderscore )
\section{Druida}
\begin{itemize}
\item {Grp. gram.:m.}
\end{itemize}
\begin{itemize}
\item {Proveniência:(Lat. \textunderscore druida\textunderscore )}
\end{itemize}
Antigo sacerdote da Gállia e da Britânnia.
\section{Druídico}
\begin{itemize}
\item {Grp. gram.:adj.}
\end{itemize}
Relativo aos druidas ou ao druidismo.
\section{Druidismo}
\begin{itemize}
\item {Grp. gram.:m.}
\end{itemize}
Religião dos druidas.
\section{Drupa}
\begin{itemize}
\item {Grp. gram.:f.}
\end{itemize}
\begin{itemize}
\item {Proveniência:(Lat. \textunderscore drupa\textunderscore )}
\end{itemize}
Fruto carnudo, que tem caroço duro.
\section{Drupáceas}
\begin{itemize}
\item {Grp. gram.:f. pl.}
\end{itemize}
Tríbo de plantas rosáceas, cujos frutos são drupas.
\section{Drupéola}
\begin{itemize}
\item {Grp. gram.:f.}
\end{itemize}
Pequena drupa.
\section{Drupeolado}
\begin{itemize}
\item {Grp. gram.:adj.}
\end{itemize}
Semelhante á drupéola.
\section{Drusa}
\begin{itemize}
\item {Grp. gram.:f.}
\end{itemize}
\begin{itemize}
\item {Proveniência:(Do al. \textunderscore druse\textunderscore )}
\end{itemize}
Aggregação de crystaes num mineral differente.
\section{Drusiforme}
\begin{itemize}
\item {Grp. gram.:adj.}
\end{itemize}
Que tem fórma de drusa.
\section{Drusos}
\begin{itemize}
\item {Grp. gram.:m. pl.}
\end{itemize}
Seita religiosa, originária do Líbano, e em cuja moral o ignorante é destinado aos mesmos castigos que o criminoso.
\section{Drya}
\begin{itemize}
\item {Grp. gram.:f.}
\end{itemize}
O mesmo que \textunderscore drýada\textunderscore .
\section{Drýada}
\begin{itemize}
\item {Grp. gram.:f.}
\end{itemize}
\begin{itemize}
\item {Proveniência:(Do gr. \textunderscore druas\textunderscore )}
\end{itemize}
Antiga divindade silvestre.
Planta rosácea.
\section{Drýade}
\begin{itemize}
\item {Grp. gram.:f.}
\end{itemize}
O mesmo que \textunderscore drýada\textunderscore .
\section{Dryádeas}
\begin{itemize}
\item {Grp. gram.:f. pl.}
\end{itemize}
Tríbo de plantas rosáceas, que tem por typo a drýada.
\section{Drýmis}
\begin{itemize}
\item {Grp. gram.:m.}
\end{itemize}
Planta ranunculácea.
\section{Drymónia}
\begin{itemize}
\item {Grp. gram.:f.}
\end{itemize}
\begin{itemize}
\item {Proveniência:(Do gr. \textunderscore drumos\textunderscore , floresta)}
\end{itemize}
Gênero de plantas gesneráceas.
\section{Dryopitheco}
\begin{itemize}
\item {Grp. gram.:m.}
\end{itemize}
\begin{itemize}
\item {Proveniência:(Do gr. \textunderscore drus\textunderscore , \textunderscore druos\textunderscore , árvore, e \textunderscore píthekos\textunderscore , macaco)}
\end{itemize}
O mesmo que \textunderscore anthropopitheco\textunderscore .
\section{Du}
\begin{itemize}
\item {Grp. gram.:m.}
\end{itemize}
\begin{itemize}
\item {Utilização:Ant.}
\end{itemize}
\begin{itemize}
\item {Proveniência:(Do lat. \textunderscore dux\textunderscore )}
\end{itemize}
Chefe.
\section{Du}
\begin{itemize}
\item {Grp. gram.:adv.}
\end{itemize}
\begin{itemize}
\item {Utilização:Ant.}
\end{itemize}
O mesmo que \textunderscore donde\textunderscore .
\section{Duaire}
\begin{itemize}
\item {Grp. gram.:m.}
\end{itemize}
\begin{itemize}
\item {Utilização:Pop.}
\end{itemize}
O mesmo que \textunderscore donaire\textunderscore .
Aspecto; retrato. Cf. Filinto, XIII, 304.
\section{Duairo}
\begin{itemize}
\item {Grp. gram.:m.}
\end{itemize}
\begin{itemize}
\item {Utilização:Pop.}
\end{itemize}
O mesmo que \textunderscore donaire\textunderscore .
Aspecto; retrato. Cf. Filinto, XIII, 304.
\section{Dual}
\begin{itemize}
\item {Grp. gram.:adj.}
\end{itemize}
\begin{itemize}
\item {Proveniência:(Lat. \textunderscore dualis\textunderscore )}
\end{itemize}
Relativo a dois.
Diz-se do número, que na declinação e conjugação de algumas linguas, como a grega e a sânscrítica, designa duas coisas ou duas pessôas.
\section{Dualidade}
\begin{itemize}
\item {Grp. gram.:f.}
\end{itemize}
\begin{itemize}
\item {Proveniência:(Lat. \textunderscore dualitas\textunderscore )}
\end{itemize}
Carácter daquillo que é duplo.
\section{Dualismo}
\begin{itemize}
\item {Grp. gram.:m.}
\end{itemize}
\begin{itemize}
\item {Proveniência:(De \textunderscore dual\textunderscore )}
\end{itemize}
Systema religioso ou philosóphico, que admitte a coexistencia de dois princípios eternos, necessários e oppostos.
Antiga theoria chímica, que suppunha todos os corpos formados de um elemento electro-negativo.
\section{Dualista}
\begin{itemize}
\item {Grp. gram.:adj.}
\end{itemize}
\begin{itemize}
\item {Grp. gram.:M.}
\end{itemize}
\begin{itemize}
\item {Proveniência:(De \textunderscore dual\textunderscore )}
\end{itemize}
Relativo a dualismo.
Sectário do dualismo.
\section{Dualístico}
\begin{itemize}
\item {Grp. gram.:adj.}
\end{itemize}
O mesmo que \textunderscore dualista\textunderscore , adj.
\section{Dualizador}
\begin{itemize}
\item {Grp. gram.:adj.}
\end{itemize}
Que dualiza.
\section{Dualizar}
\begin{itemize}
\item {Grp. gram.:v. t.}
\end{itemize}
\begin{itemize}
\item {Utilização:Neol.}
\end{itemize}
Tornar dual.
Referir a duas coisas conjuntamente.
\section{Duas}
\begin{itemize}
\item {Grp. gram.:adj.}
\end{itemize}
(fem. de \textunderscore dois\textunderscore )
\section{Dubador}
\begin{itemize}
\item {Grp. gram.:m.}
\end{itemize}
\begin{itemize}
\item {Utilização:Ant.}
\end{itemize}
Aquelle que concertava roupas ou calçado velho.
\section{Dubiamente}
\begin{itemize}
\item {Grp. gram.:adv.}
\end{itemize}
De modo dúbio.
\section{Dubiedade}
\begin{itemize}
\item {Grp. gram.:f.}
\end{itemize}
\begin{itemize}
\item {Proveniência:(Lat. \textunderscore dubietas\textunderscore )}
\end{itemize}
Qualidade de dúbio.
\section{Dúbio}
\begin{itemize}
\item {Grp. gram.:adj.}
\end{itemize}
\begin{itemize}
\item {Proveniência:(Lat. \textunderscore dubius\textunderscore )}
\end{itemize}
Duvidoso.
Hesitante.
Que é diffícil de se definir; indefinível; vago.
\section{Dubitação}
\begin{itemize}
\item {Grp. gram.:f.}
\end{itemize}
\begin{itemize}
\item {Utilização:Des.}
\end{itemize}
\begin{itemize}
\item {Proveniência:(Lat. \textunderscore dubitatio\textunderscore )}
\end{itemize}
O mesmo que \textunderscore dúvida\textunderscore .
\section{Dubitativamente}
\begin{itemize}
\item {Grp. gram.:adv.}
\end{itemize}
De modo dubitativo.
\section{Dubitativo}
\begin{itemize}
\item {Grp. gram.:adj.}
\end{itemize}
\begin{itemize}
\item {Proveniência:(Lat. \textunderscore dubitativus\textunderscore )}
\end{itemize}
Em que há dúvida.
\section{Dubitável}
\begin{itemize}
\item {Grp. gram.:adj.}
\end{itemize}
\begin{itemize}
\item {Proveniência:(Lat. \textunderscore dubitabilis\textunderscore )}
\end{itemize}
De que se póde duvidar.
\section{Dublás}
\begin{itemize}
\item {Grp. gram.:m. pl.}
\end{itemize}
Uma das castas indígenas de Damão.
\section{Duboisina}
\begin{itemize}
\item {fónica:bo-i}
\end{itemize}
\begin{itemize}
\item {Grp. gram.:f.}
\end{itemize}
\begin{itemize}
\item {Proveniência:(De \textunderscore Dubois\textunderscore , n. p.)}
\end{itemize}
Alcaloide de uma planta escrofularinea, (\textunderscore duboisia\textunderscore ), e que, ainda mais que a atropina, tem a propriedade de dilatar as pupillas.
\section{Duc}
\begin{itemize}
\item {Grp. gram.:m.}
\end{itemize}
\begin{itemize}
\item {Utilização:Ant.}
\end{itemize}
O mesmo que \textunderscore Duque\textunderscore . Cf. \textunderscore Port. Mon. Hist., Script.\textunderscore , 253.
\section{Ducado}
\begin{itemize}
\item {Grp. gram.:m.}
\end{itemize}
Território, que fórma o domínio de um Duque.
Estado independente, que tem um Duque por soberano.
Dignidade de Duque.
Moéda de oiro, de valor differente, segundo os tempos e os países.
(B. lat. \textunderscore ducatus\textunderscore )
\section{Ducado-de-águia}
\begin{itemize}
\item {Grp. gram.:m.}
\end{itemize}
Moéda de oiro, na Rússia, do valor aproximado de 2$124 reis.
\section{Ducado-imperial}
\begin{itemize}
\item {Grp. gram.:m.}
\end{itemize}
Moéda austro-húngara, de oiro, do valor de 2$126 reis.
\section{Ducal}
\begin{itemize}
\item {Grp. gram.:adj.}
\end{itemize}
\begin{itemize}
\item {Proveniência:(Lat. \textunderscore ducalis\textunderscore )}
\end{itemize}
Relativo a duque.
\section{Ducandar}
\begin{itemize}
\item {Grp. gram.:m.}
\end{itemize}
\begin{itemize}
\item {Utilização:Ant.}
\end{itemize}
Vendedor de gêneros por miúdo, nos ducões.
(Cp. \textunderscore ducão\textunderscore )
\section{Ducão}
\begin{itemize}
\item {Grp. gram.:m.}
\end{itemize}
\begin{itemize}
\item {Utilização:Ant.}
\end{itemize}
\begin{itemize}
\item {Proveniência:(T. ind.)}
\end{itemize}
Loja, onde se vendiam gêneros por miúdo, na Índia portuguesa.
\section{Ducatão}
\begin{itemize}
\item {Grp. gram.:m.}
\end{itemize}
\begin{itemize}
\item {Proveniência:(Do rad. de \textunderscore ducado\textunderscore )}
\end{itemize}
Moéda antiga de oiro.
\section{Ducentésimo}
\begin{itemize}
\item {Grp. gram.:adj.}
\end{itemize}
\begin{itemize}
\item {Proveniência:(Do lat. \textunderscore ducenti\textunderscore )}
\end{itemize}
Que numa série de 200 occupa o último lugar.
\section{Ducha}
\begin{itemize}
\item {Grp. gram.:f.}
\end{itemize}
\begin{itemize}
\item {Proveniência:(Fr. \textunderscore douche\textunderscore . Cp. cast. \textunderscore ducha\textunderscore )}
\end{itemize}
Termo que deveria substituir \textunderscore duche\textunderscore .
\section{Duchamento}
\begin{itemize}
\item {Grp. gram.:m.}
\end{itemize}
Acto de duchar.
\section{Duchar}
\begin{itemize}
\item {Grp. gram.:v. t.}
\end{itemize}
\begin{itemize}
\item {Utilização:Neol.}
\end{itemize}
Applicar duches a.
\section{Duche}
\begin{itemize}
\item {Grp. gram.:m.}
\end{itemize}
\begin{itemize}
\item {Utilização:Neol.}
\end{itemize}
\begin{itemize}
\item {Proveniência:(Fr. \textunderscore douche\textunderscore )}
\end{itemize}
Jôrro de água, que se arremessa sôbre o corpo de alguém, com fins therapêuticos.
Emborcação.
\section{Ducina}
\begin{itemize}
\item {Grp. gram.:f.}
\end{itemize}
\begin{itemize}
\item {Proveniência:(Fr. \textunderscore doucine\textunderscore )}
\end{itemize}
Moldura de cornija, meio convexa e meio côncava.
\section{Dúctil}
\begin{itemize}
\item {Grp. gram.:adj.}
\end{itemize}
\begin{itemize}
\item {Utilização:Fig.}
\end{itemize}
\begin{itemize}
\item {Proveniência:(Lat. \textunderscore ductilis\textunderscore )}
\end{itemize}
Que póde sêr comprimido ou estirado sem se partir; malleável; elástico.
Dócil.
Contemporizador.
\section{Ductilidade}
\begin{itemize}
\item {Grp. gram.:f.}
\end{itemize}
Qualidade daquillo que é dúctil.
\section{Ductilímetro}
\begin{itemize}
\item {Grp. gram.:m.}
\end{itemize}
\begin{itemize}
\item {Proveniência:(De \textunderscore dúctil\textunderscore  + gr. \textunderscore metron\textunderscore )}
\end{itemize}
Martelo, com que se avalia a ductilidade dos metaes.
\section{Ducto}
\begin{itemize}
\item {Grp. gram.:m.}
\end{itemize}
\begin{itemize}
\item {Proveniência:(Lat. \textunderscore ductus\textunderscore )}
\end{itemize}
Meato; canal no organismo animal.
Oscillação do thuríbulo.
\section{Dudongo}
\begin{itemize}
\item {Grp. gram.:m.  e  adj.}
\end{itemize}
Diz-se de uma variedade de trichecos.
\section{Duelar}
\begin{itemize}
\item {Grp. gram.:adj.}
\end{itemize}
Relativo a duelo.
\section{Duelista}
\begin{itemize}
\item {Grp. gram.:m.}
\end{itemize}
Aquele que tem o hábito de se bater em duelo.
Aquele que se bate em duelo.
\section{Duellar}
\begin{itemize}
\item {Grp. gram.:adj.}
\end{itemize}
Relativo a duello.
\section{Duellista}
\begin{itemize}
\item {Grp. gram.:m.}
\end{itemize}
Aquelle que tem o hábito de se bater em duello.
Aquelle que se bate em duello.
\section{Duello}
\begin{itemize}
\item {Grp. gram.:m.}
\end{itemize}
\begin{itemize}
\item {Proveniência:(Lat. \textunderscore duellum\textunderscore )}
\end{itemize}
Combate entre duas pessôas.
Contenda entre duas pessôas, entre dois Estados.
\section{Duelo}
\begin{itemize}
\item {Grp. gram.:m.}
\end{itemize}
\begin{itemize}
\item {Proveniência:(Lat. \textunderscore duellum\textunderscore )}
\end{itemize}
Combate entre duas pessôas.
Contenda entre duas pessôas, entre dois Estados.
\section{Duende}
\begin{itemize}
\item {Grp. gram.:m.}
\end{itemize}
Entidade mythológica ou espírito sobrenatural, que se suppunha fazer travessuras, de noite, dentro das casas.
(Cast. \textunderscore duende\textunderscore )
\section{Duerno}
\begin{itemize}
\item {Grp. gram.:m.}
\end{itemize}
\begin{itemize}
\item {Proveniência:(Do rad. do lat. \textunderscore duo\textunderscore . Cp. \textunderscore caderno\textunderscore )}
\end{itemize}
Duas fôlhas de papel de impressão, contida uma na outra.
\section{Dueto}
\begin{itemize}
\item {fónica:ê}
\end{itemize}
\begin{itemize}
\item {Grp. gram.:m.}
\end{itemize}
\begin{itemize}
\item {Utilização:Fam.}
\end{itemize}
\begin{itemize}
\item {Proveniência:(It. \textunderscore duetto\textunderscore )}
\end{itemize}
Composição musical, cantada por duas vozes ou tocada por dois instrumentos.
Conversa de duas pessôas.
\section{Dugongo}
\begin{itemize}
\item {Grp. gram.:m.}
\end{itemize}
\begin{itemize}
\item {Proveniência:(De \textunderscore Dugong\textunderscore , n. p.)}
\end{itemize}
Mammífero amphíbio do Mar das Índias, dotado de fórma extravagante e a que o vulgo chama \textunderscore homem-peixe\textunderscore .
\section{Duidade}
\begin{itemize}
\item {fónica:du-i}
\end{itemize}
\begin{itemize}
\item {Grp. gram.:f.}
\end{itemize}
\begin{itemize}
\item {Utilização:P. us.}
\end{itemize}
\begin{itemize}
\item {Proveniência:(Lat. \textunderscore duitas\textunderscore )}
\end{itemize}
O número dois.
\section{Dulçaína}
\begin{itemize}
\item {Grp. gram.:f.}
\end{itemize}
O mesmo que \textunderscore doçaína\textunderscore .
\section{Dulcamara}
\begin{itemize}
\item {Grp. gram.:f.}
\end{itemize}
\begin{itemize}
\item {Proveniência:(Do lat. \textunderscore dulcis\textunderscore  + \textunderscore amara\textunderscore )}
\end{itemize}
Planta solánea, cujas fólhas e hastes, quando mascadas, offerecem primeiro um sabor doce e depois amargo.
Videira, cuja uva é conhecida por \textunderscore uva de cão\textunderscore .
\section{Dulcamarina}
\begin{itemize}
\item {Grp. gram.:f.}
\end{itemize}
Princípio alcalino, descoberto na dulcamara.
\section{Dulciana}
\begin{itemize}
\item {Grp. gram.:f.}
\end{itemize}
\begin{itemize}
\item {Utilização:Mús.}
\end{itemize}
Registo de órgão, composto de tubos de chumbo.
\section{Dulcidão}
\begin{itemize}
\item {Grp. gram.:f.}
\end{itemize}
\begin{itemize}
\item {Utilização:Des.}
\end{itemize}
\begin{itemize}
\item {Proveniência:(Lat. \textunderscore dulcedo\textunderscore )}
\end{itemize}
O mesmo que \textunderscore doçura\textunderscore .
\section{Dulcificação}
\begin{itemize}
\item {Grp. gram.:f.}
\end{itemize}
Acto ou effeito de dulcificar.
\section{Dulcificador}
\begin{itemize}
\item {Grp. gram.:adj.}
\end{itemize}
O mesmo que \textunderscore dulcificante\textunderscore .
\section{Dulcificante}
\begin{itemize}
\item {Grp. gram.:adj.}
\end{itemize}
\begin{itemize}
\item {Proveniência:(Lat. \textunderscore dulcificans\textunderscore )}
\end{itemize}
Que dulcifica.
\section{Dulcificar}
\begin{itemize}
\item {Grp. gram.:v. t.}
\end{itemize}
\begin{itemize}
\item {Proveniência:(Lat. \textunderscore dulcificare\textunderscore )}
\end{itemize}
Tornar doce.
Suavizar.
Mitigar.
\section{Dulcífico}
\begin{itemize}
\item {Grp. gram.:adj.}
\end{itemize}
\begin{itemize}
\item {Proveniência:(De \textunderscore dulcificar\textunderscore )}
\end{itemize}
Que dulcifica.
Que é doce.
\section{Dulcífluo}
\begin{itemize}
\item {Grp. gram.:adj.}
\end{itemize}
\begin{itemize}
\item {Proveniência:(Lat. \textunderscore dulcifluus\textunderscore )}
\end{itemize}
Que corre suavemente; mellífluo.
\section{Dulcíloquo}
\begin{itemize}
\item {Grp. gram.:adj.}
\end{itemize}
\begin{itemize}
\item {Proveniência:(Lat. \textunderscore dulciloquus\textunderscore )}
\end{itemize}
Que fala docemente.
\section{Dulcineia}
\begin{itemize}
\item {Grp. gram.:f.}
\end{itemize}
\begin{itemize}
\item {Utilização:Fam.}
\end{itemize}
\begin{itemize}
\item {Proveniência:(De \textunderscore Dulcineia\textunderscore , n. p. da amante de D. Quixote)}
\end{itemize}
Namorada.
\section{Dulcísono}
\begin{itemize}
\item {Grp. gram.:adj.}
\end{itemize}
\begin{itemize}
\item {Proveniência:(Lat. \textunderscore dulcisonus\textunderscore )}
\end{itemize}
Que sôa docemente.
\section{Dulcíssimo}
\begin{itemize}
\item {Grp. gram.:adj.}
\end{itemize}
\begin{itemize}
\item {Proveniência:(Lat. \textunderscore dulcissimus\textunderscore )}
\end{itemize}
Muito doce.
\section{Dulcíssono}
\begin{itemize}
\item {Grp. gram.:adj.}
\end{itemize}
\begin{itemize}
\item {Proveniência:(Lat. \textunderscore dulcisonus\textunderscore )}
\end{itemize}
Que sôa docemente.
\section{Dulçor}
\begin{itemize}
\item {Grp. gram.:m.}
\end{itemize}
\begin{itemize}
\item {Utilização:Des.}
\end{itemize}
O mesmo que \textunderscore doçura\textunderscore .
(Cast. \textunderscore dulzor\textunderscore )
\section{Dulçoroso}
\begin{itemize}
\item {Grp. gram.:adj.}
\end{itemize}
\begin{itemize}
\item {Utilização:Des.}
\end{itemize}
\begin{itemize}
\item {Proveniência:(De \textunderscore dulçor\textunderscore )}
\end{itemize}
Que tem doçura.
\section{Dulçura}
\begin{itemize}
\item {Grp. gram.:f.}
\end{itemize}
\begin{itemize}
\item {Utilização:Ant.}
\end{itemize}
O mesmo que \textunderscore doçura\textunderscore .
(Cp. \textunderscore dulçor\textunderscore )
\section{Dulia}
\begin{itemize}
\item {Grp. gram.:f.}
\end{itemize}
\begin{itemize}
\item {Proveniência:(Gr. \textunderscore douleia\textunderscore )}
\end{itemize}
Culto que se presta aos santos e aos anjos.
\section{Dúliman}
\begin{itemize}
\item {Grp. gram.:m.}
\end{itemize}
\begin{itemize}
\item {Utilização:Ant.}
\end{itemize}
O mesmo que \textunderscore dóliman\textunderscore .
\section{Dulocracia}
\begin{itemize}
\item {Grp. gram.:f.}
\end{itemize}
\begin{itemize}
\item {Proveniência:(Do gr. \textunderscore doulos\textunderscore  + \textunderscore krateia\textunderscore )}
\end{itemize}
Preponderância de escravos.
\section{Dum-Dum}
\begin{itemize}
\item {Grp. gram.:m.}
\end{itemize}
\begin{itemize}
\item {Proveniência:(De \textunderscore Dum Dum\textunderscore , n. p. geogr.)}
\end{itemize}
Bala explosiva, de invenção moderna.
\section{Dumo}
\begin{itemize}
\item {Grp. gram.:m.}
\end{itemize}
Árvore santhomense, de madeira avermelhada.
\section{Duna}
\begin{itemize}
\item {Grp. gram.:f.}
\end{itemize}
\begin{itemize}
\item {Utilização:Gal}
\end{itemize}
Elevação de areias, acumuladas pelo vento, á beira-mar.(V.mêdo)
\section{Duna}
\begin{itemize}
\item {Grp. gram.:f.}
\end{itemize}
\begin{itemize}
\item {Utilização:Ant.}
\end{itemize}
O mesmo que \textunderscore dona\textunderscore . Cf. \textunderscore Eufrosina\textunderscore , 217.
\section{Dunália}
\begin{itemize}
\item {Grp. gram.:f.}
\end{itemize}
\begin{itemize}
\item {Proveniência:(De \textunderscore Dunal\textunderscore , n. p.)}
\end{itemize}
Arbusto americano, da fam. das solâneas.
\section{Dundum}
\begin{itemize}
\item {Grp. gram.:m.}
\end{itemize}
\begin{itemize}
\item {Proveniência:(De \textunderscore Dum Dum\textunderscore , n. p. geogr.)}
\end{itemize}
Bala explosiva, de invenção moderna.
\section{Duneta}
\begin{itemize}
\item {fónica:nê}
\end{itemize}
\begin{itemize}
\item {Grp. gram.:f.}
\end{itemize}
\begin{itemize}
\item {Proveniência:(De \textunderscore duna\textunderscore )}
\end{itemize}
O ponto mais elevado da popa do navio.
\section{Dunfa}
\begin{itemize}
\item {Grp. gram.:m.}
\end{itemize}
Espécie de lundum em San-Thomé.
\section{Dunga}
\begin{itemize}
\item {Grp. gram.:m.}
\end{itemize}
\begin{itemize}
\item {Utilização:Bras. do N}
\end{itemize}
Homem poderoso, influente.
Valentão.
O dois de paus, no jôgo chamado rodinha. Cf. Beaurepaire-Rohan, \textunderscore Diccion. de Voc. Bras.\textunderscore 
\section{Dungo-angila}
\begin{itemize}
\item {Grp. gram.:m.}
\end{itemize}
Pássaro dentirostro da África Occidental.
\section{Dunguinha}
\begin{itemize}
\item {Grp. gram.:f.}
\end{itemize}
\begin{itemize}
\item {Utilização:Bras}
\end{itemize}
\begin{itemize}
\item {Utilização:fam.}
\end{itemize}
\begin{itemize}
\item {Proveniência:(De \textunderscore dunga\textunderscore )}
\end{itemize}
Amigo útil, carinhoso.
O dois de paus, num baralho.
Criançola.
Homem insignificante.
\section{Dúnia}
\begin{itemize}
\item {Grp. gram.:f.}
\end{itemize}
\begin{itemize}
\item {Utilização:Prov.}
\end{itemize}
\begin{itemize}
\item {Utilização:trasm.}
\end{itemize}
Parte, quinhão.
Facção, partido.
(Colhido em Caçarelhos)
\section{Dunzongo}
\begin{itemize}
\item {Grp. gram.:m.}
\end{itemize}
Planta africana, annual, da fam. das malváceas, de fôlhas simples, e flôres completas sôbre longos pecíolos.
\section{Duóbolo}
\begin{itemize}
\item {Grp. gram.:m.}
\end{itemize}
Antiga moéda grega, que valia dois óbolos.
\section{Duodécima}
\begin{itemize}
\item {Grp. gram.:f.}
\end{itemize}
\begin{itemize}
\item {Utilização:Mús.}
\end{itemize}
\begin{itemize}
\item {Proveniência:(De \textunderscore duodécimo\textunderscore )}
\end{itemize}
Intervallo de décima segunda, composto da quinta sôbre a oitava.
\section{Duodecimal}
\begin{itemize}
\item {Grp. gram.:adj.}
\end{itemize}
\begin{itemize}
\item {Proveniência:(De \textunderscore duodécimo\textunderscore )}
\end{itemize}
Que se divide ou se conta por séries de doze.
Que tem por base o número doze.
\section{Duodécimo}
\begin{itemize}
\item {Grp. gram.:adj.}
\end{itemize}
\begin{itemize}
\item {Proveniência:(Lat. \textunderscore duodecimus\textunderscore )}
\end{itemize}
Que numa série occupa o lugar do número doze.
\section{Duodécuplo}
\begin{itemize}
\item {Grp. gram.:adj.}
\end{itemize}
\begin{itemize}
\item {Proveniência:(Do lat. \textunderscore duo\textunderscore  + \textunderscore decuplum\textunderscore )}
\end{itemize}
Que contém alguma coisa doze vezes; repetido doze vezes.
\section{Duodenal}
\begin{itemize}
\item {Grp. gram.:adj.}
\end{itemize}
Relativo a duodeno.
\section{Duodenário}
\begin{itemize}
\item {Grp. gram.:adj.}
\end{itemize}
\begin{itemize}
\item {Utilização:Des.}
\end{itemize}
\begin{itemize}
\item {Proveniência:(Lat. \textunderscore duodenarius\textunderscore )}
\end{itemize}
Disposto em séries de doze.
\section{Duodenite}
\begin{itemize}
\item {Grp. gram.:f.}
\end{itemize}
Inflammação no duodeno.
\section{Duodeno}
\begin{itemize}
\item {Grp. gram.:m.}
\end{itemize}
\begin{itemize}
\item {Grp. gram.:Adj.}
\end{itemize}
\begin{itemize}
\item {Utilização:Des.}
\end{itemize}
\begin{itemize}
\item {Proveniência:(Do lat. \textunderscore duodeni\textunderscore )}
\end{itemize}
Primeira parte do intestino delgado, entre o estômago e o jejuno.
Duodécimo.
\section{Duotal}
\begin{itemize}
\item {Grp. gram.:m.}
\end{itemize}
\begin{itemize}
\item {Utilização:Pharm.}
\end{itemize}
Carboneto de guaiacol, antiséptico intestinal.
\section{Duplamente}
\begin{itemize}
\item {Grp. gram.:adv.}
\end{itemize}
\begin{itemize}
\item {Proveniência:(De \textunderscore duplo\textunderscore )}
\end{itemize}
Em dôbro; duplicadamente.
\section{Duplicação}
\begin{itemize}
\item {Grp. gram.:f.}
\end{itemize}
\begin{itemize}
\item {Proveniência:(Lat. \textunderscore duplicatio\textunderscore )}
\end{itemize}
Acto ou effeito de duplicar.
\section{Duplicadamente}
\begin{itemize}
\item {Grp. gram.:adv.}
\end{itemize}
Em duplicado.
Com duplicação.
\section{Duplicado}
\begin{itemize}
\item {Grp. gram.:m.}
\end{itemize}
\begin{itemize}
\item {Proveniência:(De \textunderscore duplicar\textunderscore )}
\end{itemize}
Reproducção: \textunderscore passei recibo em duplicado\textunderscore .
Traslado; cópia.
\section{Duplicador}
\begin{itemize}
\item {Grp. gram.:m.  e  adj.}
\end{itemize}
\begin{itemize}
\item {Proveniência:(Lat. \textunderscore duplicator\textunderscore )}
\end{itemize}
O que duplica.
\section{Duplicante}
\begin{itemize}
\item {Grp. gram.:adj.}
\end{itemize}
\begin{itemize}
\item {Proveniência:(Lat. \textunderscore duplicans\textunderscore )}
\end{itemize}
Que duplica. Cf. Camillo, \textunderscore Estrêl. Fun.\textunderscore , 146.
\section{Duplicar}
\begin{itemize}
\item {Grp. gram.:v. t.}
\end{itemize}
\begin{itemize}
\item {Utilização:Ext.}
\end{itemize}
\begin{itemize}
\item {Proveniência:(Lat. \textunderscore duplicare\textunderscore )}
\end{itemize}
Repetir duas vezes; multiplicar por dois; dobrar.
Aumentar.
Fortificar; engrandecer.
\section{Duplicário}
\begin{itemize}
\item {Grp. gram.:m.}
\end{itemize}
\begin{itemize}
\item {Proveniência:(Lat. \textunderscore duplicarius\textunderscore )}
\end{itemize}
Soldado romano, que, a título de recompensa, recebia sôldo duplicado.
\section{Duplicata}
\begin{itemize}
\item {Grp. gram.:f.}
\end{itemize}
O mesmo que \textunderscore duplicado\textunderscore , m.
\section{Duplicativo}
\begin{itemize}
\item {Grp. gram.:adj.}
\end{itemize}
Que duplica.
\section{Duplicatura}
\begin{itemize}
\item {Grp. gram.:f.}
\end{itemize}
\begin{itemize}
\item {Proveniência:(De \textunderscore duplicar\textunderscore )}
\end{itemize}
Estado de uma coisa que se dobra sôbre si mesmo.
\section{Duplicável}
\begin{itemize}
\item {Grp. gram.:adj.}
\end{itemize}
Que se póde duplicar.
\section{Dúplice}
\begin{itemize}
\item {Grp. gram.:adj.}
\end{itemize}
\begin{itemize}
\item {Utilização:Fig.}
\end{itemize}
\begin{itemize}
\item {Proveniência:(Lat. \textunderscore duplex\textunderscore )}
\end{itemize}
Duplo, duplicado.
Que tem fingimento ou dobrez.
\section{Díada}
\begin{itemize}
\item {Grp. gram.:f.}
\end{itemize}
\begin{itemize}
\item {Proveniência:(Do gr. \textunderscore duas\textunderscore )}
\end{itemize}
Grupo de dois; um par.
\section{Díade}
\begin{itemize}
\item {Grp. gram.:f.}
\end{itemize}
\begin{itemize}
\item {Proveniência:(Do gr. \textunderscore duas\textunderscore )}
\end{itemize}
Grupo de dois; um par.
\section{Dinamia}
\begin{itemize}
\item {Grp. gram.:f.}
\end{itemize}
\begin{itemize}
\item {Proveniência:(Do gr. \textunderscore dunamis\textunderscore )}
\end{itemize}
Unidade, com que se mede o trabalho mecânico.
Desenvolvimento das propriedades vitaes dos tecidos orgânicos.
\section{Dinâmica}
\begin{itemize}
\item {Grp. gram.:f.}
\end{itemize}
\begin{itemize}
\item {Proveniência:(De \textunderscore dinâmico\textunderscore )}
\end{itemize}
Parte da Matemática, que trata do movimento ou do estudo das fôrças.
\section{Dinâmico}
\begin{itemize}
\item {Grp. gram.:adj.}
\end{itemize}
\begin{itemize}
\item {Proveniência:(Gr. \textunderscore dunamikos\textunderscore )}
\end{itemize}
Relativo ao movimento, ás forças.
Relativo ao organismo em actividade.
\section{Dinamiologia}
\begin{itemize}
\item {Grp. gram.:f.}
\end{itemize}
\begin{itemize}
\item {Proveniência:(Do gr. \textunderscore dunamis\textunderscore  + \textunderscore logos\textunderscore )}
\end{itemize}
Tratado das fôrças, consideradas em abstracto.
\section{Dinamismo}
\begin{itemize}
\item {Grp. gram.:m.}
\end{itemize}
\begin{itemize}
\item {Proveniência:(De \textunderscore dinamia\textunderscore )}
\end{itemize}
Sistema filosófico, que considera a matéria animada por fôrças próprias e imanentes, rejeitando as fôrças extrínsecas e mecânicas.
\section{Dinamista}
\begin{itemize}
\item {Grp. gram.:m.}
\end{itemize}
Sectário do dinamismo.
\section{Dinamitaria}
\begin{itemize}
\item {Grp. gram.:f.}
\end{itemize}
\begin{itemize}
\item {Utilização:Neol.}
\end{itemize}
Fábrica de dinamite.
\section{Dinamite}
\begin{itemize}
\item {Grp. gram.:f.}
\end{itemize}
\begin{itemize}
\item {Proveniência:(De \textunderscore dinamia\textunderscore )}
\end{itemize}
Matéria explosiva, formada de nitro-glicerina, misturada com areia quartzoza.
\section{Dinamiteiro}
\begin{itemize}
\item {Grp. gram.:adj.}
\end{itemize}
\begin{itemize}
\item {Grp. gram.:M.}
\end{itemize}
Relativo a dinamite.
O mesmo que \textunderscore dinamitista\textunderscore .
\section{Dinamitista}
\begin{itemize}
\item {Grp. gram.:m.}
\end{itemize}
Aquele que faz uso de dinamite.
Aquele que a fabríca.
\section{Dinamitizar}
\begin{itemize}
\item {Grp. gram.:v. t.}
\end{itemize}
Compor com dinamite, juntar dinamite a.
\section{Dinamização}
\begin{itemize}
\item {Grp. gram.:f.}
\end{itemize}
Acto ou efeito de dinamizar.
\section{Dinamizar}
\begin{itemize}
\item {Grp. gram.:v. t.}
\end{itemize}
\begin{itemize}
\item {Utilização:Neol.}
\end{itemize}
\begin{itemize}
\item {Proveniência:(De \textunderscore dinamia\textunderscore )}
\end{itemize}
Concentrar, dar carácter dinâmico a.
Elevar, concentrando, a energia terapêutica de, segundo a Medicina homeopática.
\section{Dínamo}
\begin{itemize}
\item {Grp. gram.:m.}
\end{itemize}
\begin{itemize}
\item {Utilização:Phýs.}
\end{itemize}
\begin{itemize}
\item {Proveniência:(Do gr. \textunderscore dunamis\textunderscore )}
\end{itemize}
Nome, que se dá, por abreviatura, á máquina electro-dinâmica, que transforma a energia mecânica em energia eléctrica.
O mesmo que dinamia.
\section{Dinamóforo}
\begin{itemize}
\item {Grp. gram.:m.}
\end{itemize}
\begin{itemize}
\item {Utilização:Bot.}
\end{itemize}
\begin{itemize}
\item {Proveniência:(Do gr. \textunderscore dunamis\textunderscore  + \textunderscore phero\textunderscore )}
\end{itemize}
Órgão de certos óvulos vegetaes, que fecha o micrópilo, depois da fecundação do óvulo. Cf. Caminhoá, \textunderscore Bot. Ger. e Med.\textunderscore , II, 1002.
\section{Dinamogenia}
\begin{itemize}
\item {Grp. gram.:f.}
\end{itemize}
\begin{itemize}
\item {Proveniência:(Do gr. \textunderscore dunamis\textunderscore  + \textunderscore genes\textunderscore )}
\end{itemize}
Exaltação funccional de um órgão, sob a influência de qualquer excitação.
\section{Dinamomagnético}
\begin{itemize}
\item {Grp. gram.:adj.}
\end{itemize}
\begin{itemize}
\item {Proveniência:(De \textunderscore dinamia\textunderscore  + \textunderscore magnético\textunderscore )}
\end{itemize}
Relativo á dinâmica do magnetismo.
\section{Dinamometria}
\begin{itemize}
\item {Grp. gram.:f.}
\end{itemize}
\begin{itemize}
\item {Proveniência:(Do gr. \textunderscore dunamis\textunderscore  + \textunderscore metron\textunderscore )}
\end{itemize}
Aplicação do dinamómetro.
\section{Dinamométrico}
\begin{itemize}
\item {Grp. gram.:adj.}
\end{itemize}
Relativo á dinamometria.
\section{Dinamómetro}
\begin{itemize}
\item {Grp. gram.:m.}
\end{itemize}
\begin{itemize}
\item {Proveniência:(Do gr. \textunderscore dunamis\textunderscore  + \textunderscore metron\textunderscore )}
\end{itemize}
Instrumento, para avaliar em pêso a fôrça e os efeitos de uma máquina.
Aparelho, para medir a fôrça muscular.
\section{Dinamoscopia}
Observação patológica, que consiste em colocar um dedo do doente no ouvido do observador, para avaliar a fôrça e saúde do observado, segundo a fôrça e a continuidade do zumbido ou rumor que se ouve.
(Cp. \textunderscore dinamoscópio\textunderscore )
\section{Dinamoscópio}
\begin{itemize}
\item {Grp. gram.:m.}
\end{itemize}
\begin{itemize}
\item {Proveniência:(Do gr. \textunderscore dunamis\textunderscore  + \textunderscore skopein\textunderscore )}
\end{itemize}
Instrumento, que se aplica na dinamoscopia.
\section{Dinasta}
\begin{itemize}
\item {Grp. gram.:m.}
\end{itemize}
\begin{itemize}
\item {Proveniência:(Gr. \textunderscore dunastes\textunderscore )}
\end{itemize}
Título antigo de príncipes soberanos.
Partidário de uma dinastia.
\section{Dinastia}
\begin{itemize}
\item {Grp. gram.:f.}
\end{itemize}
\begin{itemize}
\item {Utilização:Ext.}
\end{itemize}
\begin{itemize}
\item {Proveniência:(Gr. \textunderscore dunasteia\textunderscore )}
\end{itemize}
Série de soberanos, pertencentes á mesma família.
Série de reis.
Série de homens ilustres na mesma família.
\section{Dinástico}
\begin{itemize}
\item {Grp. gram.:adj.}
\end{itemize}
Relativo a dinastia.
\section{Diostilo}
\begin{itemize}
\item {Grp. gram.:m.}
\end{itemize}
\begin{itemize}
\item {Proveniência:(Do gr. \textunderscore duo\textunderscore  + \textunderscore stulos\textunderscore )}
\end{itemize}
Fachada de duas colunas.
\section{Diótipa}
\begin{itemize}
\item {Grp. gram.:f.}
\end{itemize}
Máquina de composição tipográfica, inventada em 1905.
\section{Disartria}
\begin{itemize}
\item {Grp. gram.:f.}
\end{itemize}
Dificuldade na pronúncia e articulação das palavras.
\section{Disbasia}
\begin{itemize}
\item {Grp. gram.:f.}
\end{itemize}
\begin{itemize}
\item {Proveniência:(Do gr. \textunderscore dus\textunderscore  + \textunderscore basis\textunderscore )}
\end{itemize}
Aberração no andar, procedente de perturbação do sIstema nervoso.
\section{Discinesia}
\begin{itemize}
\item {Grp. gram.:f.}
\end{itemize}
\begin{itemize}
\item {Utilização:Med.}
\end{itemize}
\begin{itemize}
\item {Proveniência:(Do gr. \textunderscore dus\textunderscore  + \textunderscore kinesís\textunderscore )}
\end{itemize}
Deminuição ou extinção dos movimentos voluntários.
\section{Díscolo}
\begin{itemize}
\item {Grp. gram.:adj.}
\end{itemize}
\begin{itemize}
\item {Grp. gram.:M.}
\end{itemize}
\begin{itemize}
\item {Proveniência:(Lat. \textunderscore dyscolus\textunderscore )}
\end{itemize}
Que é áspero no trato.
Mal humorado.
Desordeiro.
Dissidente.
Aquele, que tem mau gênio.
Brigão.
\section{Discrasia}
\begin{itemize}
\item {Grp. gram.:f.}
\end{itemize}
\begin{itemize}
\item {Proveniência:(Gr. \textunderscore duscrasia\textunderscore )}
\end{itemize}
Maus humores, má constituição física.
Alteração de humores.
\section{Discrásico}
\begin{itemize}
\item {Grp. gram.:adj.}
\end{itemize}
\begin{itemize}
\item {Grp. gram.:M.}
\end{itemize}
Relativo á discrasia.
Aquele que padece discrasia.
\section{Discroia}
\begin{itemize}
\item {Grp. gram.:f.}
\end{itemize}
\begin{itemize}
\item {Utilização:Med.}
\end{itemize}
\begin{itemize}
\item {Proveniência:(Do gr. \textunderscore dus\textunderscore  + \textunderscore khroia\textunderscore )}
\end{itemize}
Má côr da pele.
\section{Discromático}
\begin{itemize}
\item {Grp. gram.:adj.}
\end{itemize}
\begin{itemize}
\item {Proveniência:(Do gr. \textunderscore dus\textunderscore  + \textunderscore khroma\textunderscore )}
\end{itemize}
Que altera as cores.
Que não tem bôa côr.
\section{Discromatopsia}
\begin{itemize}
\item {Grp. gram.:f.}
\end{itemize}
\begin{itemize}
\item {Proveniência:(Do gr. \textunderscore dus\textunderscore  + \textunderscore khroma\textunderscore  + \textunderscore opsis\textunderscore )}
\end{itemize}
Espécie de daltonismo, em que a vista confunde certas côres com outras que distingue.
\section{Disenteria}
\begin{itemize}
\item {Grp. gram.:f.}
\end{itemize}
\begin{itemize}
\item {Proveniência:(Gr. \textunderscore dusenteria\textunderscore )}
\end{itemize}
Inflamação dos intestinos, de que resultam evacuações mucosas ou purulentas, ás vezes misturadas com sangue.
\section{Disentérico}
\begin{itemize}
\item {Grp. gram.:adj.}
\end{itemize}
\begin{itemize}
\item {Grp. gram.:M.}
\end{itemize}
Relativo á disenteria.
Aquele que sofre disenteria.
\section{Disestesia}
\begin{itemize}
\item {Grp. gram.:f.}
\end{itemize}
\begin{itemize}
\item {Utilização:Med.}
\end{itemize}
\begin{itemize}
\item {Proveniência:(Do gr. \textunderscore dus\textunderscore  + \textunderscore aisthesis\textunderscore )}
\end{itemize}
Enfraquecimento ou extinção da acção dos sentidos.
\section{Disfagia}
\begin{itemize}
\item {Grp. gram.:f.}
\end{itemize}
\begin{itemize}
\item {Proveniência:(Do gr. \textunderscore dus\textunderscore  + \textunderscore phagein\textunderscore )}
\end{itemize}
Dificuldade em engulir.
\section{Disfasia}
\begin{itemize}
\item {Grp. gram.:f.}
\end{itemize}
\begin{itemize}
\item {Utilização:Med.}
\end{itemize}
Qualquer dificuldade no falar.
\section{Disfonia}
\begin{itemize}
\item {Grp. gram.:f.}
\end{itemize}
\begin{itemize}
\item {Utilização:Med.}
\end{itemize}
\begin{itemize}
\item {Proveniência:(Do gr. \textunderscore dus\textunderscore  + \textunderscore phone\textunderscore )}
\end{itemize}
Alteração da voz e da palavra.
\section{Disforia}
\begin{itemize}
\item {Grp. gram.:f.}
\end{itemize}
\begin{itemize}
\item {Proveniência:(Do gr. \textunderscore dus\textunderscore  + \textunderscore pherein\textunderscore )}
\end{itemize}
Indisposição mórbida; mal-estar.
\section{Dislalia}
\begin{itemize}
\item {Grp. gram.:f.}
\end{itemize}
\begin{itemize}
\item {Proveniência:(Do gr. \textunderscore dus\textunderscore  + \textunderscore lalein\textunderscore )}
\end{itemize}
Dificuldade em articular palavras.
\section{Dislexia}
\begin{itemize}
\item {fónica:csi}
\end{itemize}
\begin{itemize}
\item {Grp. gram.:f.}
\end{itemize}
Dificuldade de lêr e compreender a escrita.
\section{Dismenia}
\begin{itemize}
\item {Grp. gram.:f.}
\end{itemize}
O mesmo que \textunderscore dismenorreia\textunderscore .
\section{Dismenorreia}
\begin{itemize}
\item {Grp. gram.:f.}
\end{itemize}
\begin{itemize}
\item {Proveniência:(Do gr. \textunderscore dus\textunderscore  + \textunderscore men\textunderscore  + \textunderscore rhein\textunderscore )}
\end{itemize}
Corrimento menstrual, difícil e doloroso.
Dificuldade da menstruação.
\section{Dismenorreico}
\begin{itemize}
\item {Grp. gram.:adj.}
\end{itemize}
Relativo a dismenorreia.
\section{Dismnesia}
\begin{itemize}
\item {Grp. gram.:f.}
\end{itemize}
\begin{itemize}
\item {Proveniência:(Do gr. \textunderscore dus\textunderscore  + \textunderscore mnesis\textunderscore )}
\end{itemize}
Enfraquecimento da memória.
\section{Disodia}
\begin{itemize}
\item {Grp. gram.:f.}
\end{itemize}
\begin{itemize}
\item {Proveniência:(Do gr. \textunderscore dus\textunderscore  + \textunderscore ozein\textunderscore )}
\end{itemize}
Mau cheiro das secreções.
\section{Disopia}
\begin{itemize}
\item {Grp. gram.:f.}
\end{itemize}
\begin{itemize}
\item {Proveniência:(Do gr. \textunderscore dus\textunderscore  + \textunderscore ops\textunderscore )}
\end{itemize}
Enfraquecimento da vista.
\section{Disorexia}
\begin{itemize}
\item {fónica:csi}
\end{itemize}
\begin{itemize}
\item {Grp. gram.:f.}
\end{itemize}
Falta de apetite.
\section{Disosmia}
\begin{itemize}
\item {Grp. gram.:f.}
\end{itemize}
\begin{itemize}
\item {Proveniência:(Do gr. \textunderscore dus\textunderscore  + \textunderscore osme\textunderscore )}
\end{itemize}
Enfraquecimento do olfato.
\section{Dispepsia}
\begin{itemize}
\item {Grp. gram.:f.}
\end{itemize}
\begin{itemize}
\item {Proveniência:(Gr. \textunderscore duspepsia\textunderscore )}
\end{itemize}
Dificuldade na digestão.
Má digestão.
\section{Dispéptico}
\begin{itemize}
\item {Grp. gram.:adj.}
\end{itemize}
\begin{itemize}
\item {Grp. gram.:M.}
\end{itemize}
Relativo a dispepsia.
Aquele que sofre dispepsia.
\section{Dispneia}
\begin{itemize}
\item {Grp. gram.:f.}
\end{itemize}
\begin{itemize}
\item {Proveniência:(Gr. \textunderscore duspnoia\textunderscore )}
\end{itemize}
Dificuldade de respirar.
\section{Dispneico}
\begin{itemize}
\item {Grp. gram.:adj.}
\end{itemize}
Relativo á dispneia.
\section{Dissimetria}
\begin{itemize}
\item {Grp. gram.:f.}
\end{itemize}
\begin{itemize}
\item {Proveniência:(Do gr. \textunderscore dus\textunderscore  + \textunderscore summetria\textunderscore )}
\end{itemize}
Falta de simetria.
\section{Dissimétrico}
\begin{itemize}
\item {Grp. gram.:adj.}
\end{itemize}
Em que ha dissimetria.
\section{Distanásia}
\begin{itemize}
\item {Grp. gram.:f.}
\end{itemize}
\begin{itemize}
\item {Proveniência:(Do gr. \textunderscore dus\textunderscore  + \textunderscore thanatos\textunderscore )}
\end{itemize}
Morte dolorosa.
\section{Distócia}
\begin{itemize}
\item {Grp. gram.:f.}
\end{itemize}
\begin{itemize}
\item {Proveniência:(Do gr. \textunderscore dus\textunderscore  + \textunderscore tokos\textunderscore )}
\end{itemize}
Parto difícil.
\section{Distocíaco}
\begin{itemize}
\item {Grp. gram.:adj}
\end{itemize}
Relativo á distocia.
\section{Distrofia}
\begin{itemize}
\item {Grp. gram.:f.}
\end{itemize}
\begin{itemize}
\item {Utilização:Med.}
\end{itemize}
Perturbação da nutrição.
Doença, produzida por má nutrição.
\section{Distrófico}
\begin{itemize}
\item {Grp. gram.:adj.}
\end{itemize}
\begin{itemize}
\item {Proveniência:(Do gr. \textunderscore dus\textunderscore  + \textunderscore trophe\textunderscore )}
\end{itemize}
Relativo á distrofia.
Que prejudica a nutrição.
Que se alimenta mal.
\section{Disuria}
\begin{itemize}
\item {Grp. gram.:f.}
\end{itemize}
\begin{itemize}
\item {Proveniência:(Gr. \textunderscore dusouria\textunderscore )}
\end{itemize}
Dificuldade em urinar.
\section{Disúrico}
\begin{itemize}
\item {Grp. gram.:adj.}
\end{itemize}
\begin{itemize}
\item {Grp. gram.:M.}
\end{itemize}
Relativo á disuria.
Aquelle que padece disuria.
\section{Dítico}
\begin{itemize}
\item {Grp. gram.:adj.}
\end{itemize}
\begin{itemize}
\item {Grp. gram.:m.}
\end{itemize}
\begin{itemize}
\item {Grp. gram.:M. pl.}
\end{itemize}
\begin{itemize}
\item {Proveniência:(Do gr. \textunderscore duein\textunderscore )}
\end{itemize}
Mergulhador.
Coleóptero amphíbio, espécie de carocha com asas, (\textunderscore dyticus marginalis\textunderscore ). Cf. P. Moraes, \textunderscore Zool. Elem.\textunderscore , 586.
Família de aves, que têm o hábito de mergulhar.
\section{Drudaria}
\begin{itemize}
\item {Grp. gram.:f.}
\end{itemize}
\begin{itemize}
\item {Utilização:Ant.}
\end{itemize}
O mesmo que \textunderscore adultério\textunderscore . Cf. \textunderscore Port. Mon. Hist.\textunderscore , \textunderscore Script.\textunderscore , 296.
\section{Dupliciário}
\begin{itemize}
\item {Grp. gram.:m.}
\end{itemize}
\begin{itemize}
\item {Proveniência:(Lat. \textunderscore dupliciarius\textunderscore )}
\end{itemize}
O mesmo que \textunderscore duplicário\textunderscore .
\section{Duplicidade}
\begin{itemize}
\item {Grp. gram.:f.}
\end{itemize}
\begin{itemize}
\item {Utilização:Fig.}
\end{itemize}
\begin{itemize}
\item {Proveniência:(Lat. \textunderscore duplicitas\textunderscore )}
\end{itemize}
Estado daquelle ou daquillo que é dúplice.
Dobrez.
\section{Duplipenes}
\begin{itemize}
\item {Grp. gram.:m. pl.}
\end{itemize}
\begin{itemize}
\item {Proveniência:(De \textunderscore duplo\textunderscore  + \textunderscore penna\textunderscore )}
\end{itemize}
Nome, que os naturalistas deram a uma família de insectos, de quatro asas, dobradas longitudinalmente, e com antennas claviformes.
\section{Duplipennes}
\begin{itemize}
\item {Grp. gram.:m. pl.}
\end{itemize}
\begin{itemize}
\item {Proveniência:(De \textunderscore duplo\textunderscore  + \textunderscore penna\textunderscore )}
\end{itemize}
Nome, que os naturalistas deram a uma família de insectos, de quatro asas, dobradas longitudinalmente, e com antennas claviformes.
\section{Duplo}
\begin{itemize}
\item {Grp. gram.:m.}
\end{itemize}
\begin{itemize}
\item {Grp. gram.:Adj.}
\end{itemize}
\begin{itemize}
\item {Proveniência:(Lat. \textunderscore duplus\textunderscore )}
\end{itemize}
Dôbro.
Dobrado.
Formado de duas coisas análogas.
\section{Dupôndio}
\begin{itemize}
\item {Grp. gram.:m.}
\end{itemize}
\begin{itemize}
\item {Proveniência:(Lat. \textunderscore dupondium\textunderscore )}
\end{itemize}
Moéda romana, do valor do dois asses.
\section{Duque}
\begin{itemize}
\item {Grp. gram.:m.}
\end{itemize}
\begin{itemize}
\item {Utilização:Gír.}
\end{itemize}
\begin{itemize}
\item {Proveniência:(Do lat. \textunderscore dux\textunderscore )}
\end{itemize}
Chefe de um ducado.
Título nobre em Portugal, immediatamente superior ao de Marquês.
Carta de jogar, que tem dois pontos.
Cão.
Variedade de videira.
\section{Duquesa}
\begin{itemize}
\item {Grp. gram.:f.}
\end{itemize}
\begin{itemize}
\item {Proveniência:(De \textunderscore duque\textunderscore )}
\end{itemize}
Senhora, que tem o título honorífico, correspondente ao de Duque.
Soberana de um ducado.
Espécie de tecido antigo:«\textunderscore uma saia de lanilha e outra de duquesa\textunderscore ». (De um testamento de 1694)
Espécie de sofá.
\section{Dura}
\begin{itemize}
\item {Grp. gram.:f.}
\end{itemize}
O mesmo que \textunderscore duração\textunderscore : \textunderscore sol de inverno é de pouca dura\textunderscore .
\section{Durabilidade}
\begin{itemize}
\item {Grp. gram.:f.}
\end{itemize}
\begin{itemize}
\item {Proveniência:(Lat. \textunderscore durabilitas\textunderscore )}
\end{itemize}
Qualidade daquillo que é durável.
\section{Duração}
\begin{itemize}
\item {Grp. gram.:f.}
\end{itemize}
\begin{itemize}
\item {Proveniência:(De \textunderscore durar\textunderscore )}
\end{itemize}
O tempo que uma coisa dura: \textunderscore a duração de uma guerra\textunderscore .
Qualidade daquillo que dura: \textunderscore êste tecido não é de duração\textunderscore .
\section{Duradoiro}
\begin{itemize}
\item {Grp. gram.:adj.}
\end{itemize}
Que dura muito.
Que póde durar muito.
\section{Durador}
\begin{itemize}
\item {Grp. gram.:adj.}
\end{itemize}
O mesmo que \textunderscore duradoiro\textunderscore .
\section{Duradouro}
\begin{itemize}
\item {Grp. gram.:adj.}
\end{itemize}
Que dura muito.
Que póde durar muito.
\section{Duramáter}
\begin{itemize}
\item {Grp. gram.:f.}
\end{itemize}
\begin{itemize}
\item {Proveniência:(Do lat. \textunderscore dura\textunderscore  + \textunderscore mater\textunderscore )}
\end{itemize}
Membrana exterior, que envolve o cérebro e a medulla espinal.
\section{Dura-máter}
\begin{itemize}
\item {Grp. gram.:f.}
\end{itemize}
\begin{itemize}
\item {Utilização:Anat.}
\end{itemize}
\begin{itemize}
\item {Proveniência:(Do lat. \textunderscore dura\textunderscore  + \textunderscore mater\textunderscore )}
\end{itemize}
Membrana exterior, que envolve o cérebro e a medulla espinal.
\section{Durame}
\begin{itemize}
\item {Grp. gram.:m.}
\end{itemize}
\begin{itemize}
\item {Proveniência:(Lat. \textunderscore duramen\textunderscore )}
\end{itemize}
O mesmo que \textunderscore cerne\textunderscore .
\section{Durâmen}
\begin{itemize}
\item {Grp. gram.:m.}
\end{itemize}
\begin{itemize}
\item {Proveniência:(Lat. \textunderscore duramen\textunderscore )}
\end{itemize}
O mesmo que \textunderscore cerne\textunderscore .
\section{Duramente}
\begin{itemize}
\item {Grp. gram.:adv.}
\end{itemize}
Com dureza, de modo duro.
Asperamente, severamente; cruelmente.
\section{Durança}
\begin{itemize}
\item {Grp. gram.:f.}
\end{itemize}
\begin{itemize}
\item {Utilização:Ant.}
\end{itemize}
O mesmo que \textunderscore duração\textunderscore .
\section{Duranta}
\begin{itemize}
\item {Grp. gram.:f.}
\end{itemize}
\begin{itemize}
\item {Proveniência:(De \textunderscore Durantes\textunderscore , n. p.)}
\end{itemize}
Planta verbenácea, espécie de lantana.
\section{Durante}
\begin{itemize}
\item {Grp. gram.:prep.}
\end{itemize}
\begin{itemize}
\item {Grp. gram.:M.}
\end{itemize}
\begin{itemize}
\item {Grp. gram.:Adj.}
\end{itemize}
\begin{itemize}
\item {Proveniência:(Lat. \textunderscore durans\textunderscore )}
\end{itemize}
No tempo de: \textunderscore durante a epidemia\textunderscore .
No espaço de: \textunderscore durante quarenta annos\textunderscore .
Tecido de lan, lustroso como setim.
Diz-se de uma variedade de maçan.
\section{Duraque}
\begin{itemize}
\item {Grp. gram.:m.}
\end{itemize}
Tecido forte e consistente, que se applicou especialmente no calçado de senhoras.
\section{Durar}
\begin{itemize}
\item {Grp. gram.:v. i.}
\end{itemize}
\begin{itemize}
\item {Proveniência:(Lat. \textunderscore durare\textunderscore )}
\end{itemize}
Sêr duro, resistente.
Permanecer.
Não se gastar.
Persistir.
Prolongar-se.
Existir; viver: \textunderscore homem, que durou cem annos\textunderscore .
Conservar-se, mantendo as mesmas qualidades.
\section{Duras}
\begin{itemize}
\item {Grp. gram.:f. pl. Loc. adv.}
\end{itemize}
\begin{itemize}
\item {Grp. gram.:Loc. adv.}
\end{itemize}
\textunderscore Ás duras\textunderscore , difficilmente, trabalhosamente.
\textunderscore A-duras\textunderscore , difficilmente. Cf. \textunderscore Eufrosina\textunderscore , 86.
\section{Durasnal}
\begin{itemize}
\item {Grp. gram.:m.}
\end{itemize}
\begin{itemize}
\item {Utilização:Bras. do S}
\end{itemize}
\begin{itemize}
\item {Proveniência:(Do cast. \textunderscore durazno\textunderscore , durázio)}
\end{itemize}
Pomar de pessegueiros, de fruto durázio.
\section{Durável}
\begin{itemize}
\item {Grp. gram.:adj.}
\end{itemize}
\begin{itemize}
\item {Proveniência:(Lat. \textunderscore durabilis\textunderscore )}
\end{itemize}
O mesmo que \textunderscore duradoiro\textunderscore .
\section{Duraz}
\begin{itemize}
\item {Grp. gram.:adj.}
\end{itemize}
O mesmo que \textunderscore durázio\textunderscore .
\section{Durázia}
\begin{itemize}
\item {Grp. gram.:f.}
\end{itemize}
Espécie de azeitona, também conhecida por \textunderscore lentisca\textunderscore  e \textunderscore salgueira\textunderscore .
\section{Durázio}
\begin{itemize}
\item {Grp. gram.:adj.}
\end{itemize}
\begin{itemize}
\item {Utilização:Fam.}
\end{itemize}
\begin{itemize}
\item {Proveniência:(Do lat. \textunderscore duracinus\textunderscore )}
\end{itemize}
Diz-se de alguns frutos, que têm a casca dura.
Diz-se da amêndoa, que só quebra a martelo, (por opposição á amêndoa côca).
Que está na idade madura: \textunderscore a Dona Annica já é durázia\textunderscore .
\section{Durdaria}
\begin{itemize}
\item {Grp. gram.:f.}
\end{itemize}
\begin{itemize}
\item {Utilização:Ant.}
\end{itemize}
O mesmo que \textunderscore adultério\textunderscore . Cf. \textunderscore Port. Mon. Hist.\textunderscore , \textunderscore Script.\textunderscore , 296.
\section{Dureiro}
\begin{itemize}
\item {Grp. gram.:adj.}
\end{itemize}
\begin{itemize}
\item {Utilização:Ant.}
\end{itemize}
\begin{itemize}
\item {Proveniência:(De \textunderscore duro\textunderscore )}
\end{itemize}
Que offerece difficuldade ou dureza.
\section{Durez}
\begin{itemize}
\item {Grp. gram.:f.}
\end{itemize}
O mesmo que \textunderscore dureza\textunderscore . Cf. Camillo, \textunderscore Estrêl. Funestas\textunderscore , 82.
\section{Dureza}
\begin{itemize}
\item {Grp. gram.:f.}
\end{itemize}
\begin{itemize}
\item {Utilização:Bras. do N}
\end{itemize}
\begin{itemize}
\item {Proveniência:(Do lat. \textunderscore duritia\textunderscore )}
\end{itemize}
Qualidade daquillo que é duro.
Acção dura, cruel.
Rigor.
Inflammação do baço, em resultado de sezões.
\section{Durguete}
\begin{itemize}
\item {fónica:guê}
\end{itemize}
\begin{itemize}
\item {Grp. gram.:f.}
\end{itemize}
\begin{itemize}
\item {Utilização:Ant.}
\end{itemize}
Espécie de tecido.
\section{Duríade}
\begin{itemize}
\item {Grp. gram.:f.}
\end{itemize}
\begin{itemize}
\item {Utilização:Poét.}
\end{itemize}
\begin{itemize}
\item {Proveniência:(Do lat. \textunderscore Durius\textunderscore , n. p.)}
\end{itemize}
Nympha do Doiro. Cf. Garrett, \textunderscore Retrato de Venus\textunderscore , 79.
\section{Durião}
\begin{itemize}
\item {Grp. gram.:m.}
\end{itemize}
\begin{itemize}
\item {Proveniência:(Do mal. \textunderscore durian\textunderscore )}
\end{itemize}
Fruto asiático, semelhante ás alcachofras, (\textunderscore durio Zibelhinus\textunderscore ).
\section{Duriense}
\begin{itemize}
\item {Grp. gram.:adj.}
\end{itemize}
\begin{itemize}
\item {Grp. gram.:M.}
\end{itemize}
\begin{itemize}
\item {Proveniência:(Lat. \textunderscore duriensis\textunderscore )}
\end{itemize}
Relativo á provincia do Doiro ou ao rio Doiro.
Aquelle que é natural da provincia do Doiro.
\section{Durindana}
\begin{itemize}
\item {Grp. gram.:f.}
\end{itemize}
\begin{itemize}
\item {Utilização:Pop.}
\end{itemize}
\begin{itemize}
\item {Proveniência:(De \textunderscore durandal\textunderscore , espada de Roldão)}
\end{itemize}
Espada.
\section{Duriúsculo}
\begin{itemize}
\item {Grp. gram.:adj.}
\end{itemize}
\begin{itemize}
\item {Utilização:Burl.}
\end{itemize}
Um tanto duro. Cf. Filinto, XIII, 70.
\section{Duro}
\begin{itemize}
\item {Grp. gram.:adj.}
\end{itemize}
\begin{itemize}
\item {Utilização:Fam.}
\end{itemize}
\begin{itemize}
\item {Grp. gram.:M.}
\end{itemize}
\begin{itemize}
\item {Proveniência:(Lat. \textunderscore durus\textunderscore )}
\end{itemize}
Que não é tenro, que não é molle: \textunderscore pão duro\textunderscore .
Que é diffícil de penetrar.
Sólido; rijo; consistente: \textunderscore tem carnes duras\textunderscore .
Coagulado.
Resistente.
Arduo, áspero.
Desagradável ao ouvido.
Rigoroso, cruel, implacável: \textunderscore um duro destino\textunderscore .
Enérgico; forte.
Penoso, molesto.
Que está na idade madura.
\textunderscore Duro de cabeça\textunderscore , pertinaz, casmurro.
\textunderscore Duro de ouvido\textunderscore , que não ouve bem ou que não distingue bem as modalidades do som.
\textunderscore Duro de roer\textunderscore , que custa muito soffrer.
Moéda espanhola, de prata, que vale aproximadamente 920 reis da nossa moéda.
\textunderscore Duro com duro não faz bom muro\textunderscore , indoles obstinadas ou teimosas não se harmonizam.
\section{Duro-a-fogo}
\begin{itemize}
\item {Grp. gram.:m.}
\end{itemize}
\begin{itemize}
\item {Utilização:Bras. do N}
\end{itemize}
\begin{itemize}
\item {Utilização:Fig.}
\end{itemize}
Tabaco ruím, que arde com difficuldade.
Indivíduo, insensível ás reprehensões.
\section{Durura}
\begin{itemize}
\item {Grp. gram.:f.}
\end{itemize}
\begin{itemize}
\item {Utilização:T. de Moçambique}
\end{itemize}
Planta, também conhecida por \textunderscore erva-do-leite\textunderscore .
\section{Dussia}
\begin{itemize}
\item {Grp. gram.:f.}
\end{itemize}
O mesmo que \textunderscore adussia\textunderscore . Cf. \textunderscore Testamento de El-rei D. Dinís\textunderscore ; e Castanheda, \textunderscore Descobr. da Índia\textunderscore .
\section{Duumvirado}
\begin{itemize}
\item {Grp. gram.:m.}
\end{itemize}
\begin{itemize}
\item {Proveniência:(Lat. \textunderscore duumviratus\textunderscore )}
\end{itemize}
Cargo de duúmviro.
Duração desse cargo.
\section{Duumviral}
\begin{itemize}
\item {Grp. gram.:adj.}
\end{itemize}
Relativo a duúmviro.
\section{Duumviralício}
\begin{itemize}
\item {Grp. gram.:adj.}
\end{itemize}
O mesmo que duumviral. Cf. Herculano, \textunderscore Hist. de Port.\textunderscore , VI, 47 e 181.
\section{Duumvirato}
\begin{itemize}
\item {Grp. gram.:m.}
\end{itemize}
(V.duumvirado)
\section{Duúmviro}
\begin{itemize}
\item {Grp. gram.:m.}
\end{itemize}
\begin{itemize}
\item {Proveniência:(Lat. \textunderscore duumvir\textunderscore )}
\end{itemize}
Cada um de dois magistrados romanos, que funccionavam juntamente.
\section{Duvália}
\begin{itemize}
\item {Grp. gram.:f.}
\end{itemize}
\begin{itemize}
\item {Proveniência:(De \textunderscore Duval\textunderscore , n. p.)}
\end{itemize}
Gênero de plantas cryptògâmicas da Alemanha.
\section{Dúvida}
\begin{itemize}
\item {Grp. gram.:f.}
\end{itemize}
\begin{itemize}
\item {Grp. gram.:Loc. adv.}
\end{itemize}
\begin{itemize}
\item {Proveniência:(De \textunderscore duvidar\textunderscore )}
\end{itemize}
Incerteza á cêrca da realidade de um facto ou da verdade de uma asserção; hesitação.
Suspeita.
Sceptícismo.
Difficuldade em acreditar.
Objecção: \textunderscore oppor dúvidas\textunderscore .
Escrúpulo.
\textunderscore Por sem dúvida\textunderscore , com toda a certeza. Cf. \textunderscore Eufrosina\textunderscore , 151.
\section{Duvidador}
\begin{itemize}
\item {Grp. gram.:m.}
\end{itemize}
\begin{itemize}
\item {Proveniência:(Do lat. \textunderscore dubitator\textunderscore )}
\end{itemize}
Aquelle que duvida.
\section{Duvidança}
\begin{itemize}
\item {Grp. gram.:f.}
\end{itemize}
\begin{itemize}
\item {Utilização:Ant.}
\end{itemize}
\begin{itemize}
\item {Proveniência:(De \textunderscore duvidar\textunderscore )}
\end{itemize}
Dúvida, incerteza.
\section{Duvidar}
\begin{itemize}
\item {Grp. gram.:v. t.}
\end{itemize}
\begin{itemize}
\item {Grp. gram.:V. i.}
\end{itemize}
\begin{itemize}
\item {Proveniência:(Lat. \textunderscore dubitare\textunderscore )}
\end{itemize}
Têr dúvida de; não acreditar.
Estar na dúvida, na incerteza: \textunderscore a êsse respeito, duvido\textunderscore .
Têr desconfiança; não têr confiança: \textunderscore todos duvidam delle\textunderscore .
Não acreditar; sêr scéptico.
Hesitar.
Nutrir suspeitas.
\section{Duvidosamente}
\begin{itemize}
\item {Grp. gram.:adv.}
\end{itemize}
De modo duvidoso.
\section{Duvidoso}
\begin{itemize}
\item {Grp. gram.:adj.}
\end{itemize}
\begin{itemize}
\item {Proveniência:(De \textunderscore dúvida\textunderscore )}
\end{itemize}
Incerto, sujeito a dúvidas.
Hesitante.
Receoso.
Suspeito.
Que duvida; indeciso.
\section{Duzentos}
\begin{itemize}
\item {Grp. gram.:adj.}
\end{itemize}
\begin{itemize}
\item {Proveniência:(Lat. \textunderscore ducentum\textunderscore )}
\end{itemize}
Duas vezes cem.
\section{Dúzia}
\begin{itemize}
\item {Grp. gram.:f.}
\end{itemize}
\begin{itemize}
\item {Grp. gram.:Pl.}
\end{itemize}
\begin{itemize}
\item {Utilização:Fam.}
\end{itemize}
Reunião de doze objectos da mesma natureza: \textunderscore uma dúzia de laranjas\textunderscore .
Porção, quantidade: \textunderscore tenho dúzias de prédios\textunderscore .
(Da mesma or. que \textunderscore doze\textunderscore )
\section{Duzir}
\begin{itemize}
\item {Grp. gram.:v. t.}
\end{itemize}
\begin{itemize}
\item {Utilização:Ant.}
\end{itemize}
\begin{itemize}
\item {Proveniência:(Lat. \textunderscore ducere\textunderscore )}
\end{itemize}
Conduzir.
\section{Dýada}
\begin{itemize}
\item {Grp. gram.:f.}
\end{itemize}
\begin{itemize}
\item {Proveniência:(Do gr. \textunderscore duas\textunderscore )}
\end{itemize}
Grupo de dois; um par.
\section{Dynamia}
\begin{itemize}
\item {Grp. gram.:f.}
\end{itemize}
\begin{itemize}
\item {Proveniência:(Do gr. \textunderscore dunamis\textunderscore )}
\end{itemize}
Unidade, com que se mede o trabalho mecânico.
Desenvolvimento das propriedades vitaes dos tecidos orgânicos.
\section{Dynâmica}
\begin{itemize}
\item {Grp. gram.:f.}
\end{itemize}
\begin{itemize}
\item {Proveniência:(De \textunderscore dynâmico\textunderscore )}
\end{itemize}
Parte da Mathemática, que trata do movimento ou do estudo das fôrças.
\section{Dynâmico}
\begin{itemize}
\item {Grp. gram.:adj.}
\end{itemize}
\begin{itemize}
\item {Proveniência:(Gr. \textunderscore dunamikos\textunderscore )}
\end{itemize}
Relativo ao movimento, ás forças.
Relativo ao organismo em actividade.
\section{Dynamiologia}
\begin{itemize}
\item {Grp. gram.:f.}
\end{itemize}
\begin{itemize}
\item {Proveniência:(Do gr. \textunderscore dunamis\textunderscore  + \textunderscore logos\textunderscore )}
\end{itemize}
Tratado das fôrças, consideradas em abstracto.
\section{Dynamismo}
\begin{itemize}
\item {Grp. gram.:m.}
\end{itemize}
\begin{itemize}
\item {Proveniência:(De \textunderscore dynamia\textunderscore )}
\end{itemize}
Systema philosóphico, que considera a matéria animada por fôrças próprias e immanentes, rejeitando as fôrças extrínsecas e mecânicas.
\section{Dynamista}
\begin{itemize}
\item {Grp. gram.:m.}
\end{itemize}
Sectário do dynamismo.
\section{Dynamitaria}
\begin{itemize}
\item {Grp. gram.:f.}
\end{itemize}
\begin{itemize}
\item {Utilização:Neol.}
\end{itemize}
Fábrica de dynamite.
\section{Dynamite}
\begin{itemize}
\item {Grp. gram.:f.}
\end{itemize}
\begin{itemize}
\item {Proveniência:(De \textunderscore dynamia\textunderscore )}
\end{itemize}
Matéria explosiva, formada de nitro-glycerina, misturada com areia quartzoza.
\section{Dynamiteiro}
\begin{itemize}
\item {Grp. gram.:adj.}
\end{itemize}
\begin{itemize}
\item {Grp. gram.:M.}
\end{itemize}
Relativo a dynamite.
O mesmo que \textunderscore dynamitista\textunderscore .
\section{Dynamitista}
\begin{itemize}
\item {Grp. gram.:m.}
\end{itemize}
Aquelle que faz uso de dynamite.
Aquelle que a fabríca.
\section{Dynamitizar}
\begin{itemize}
\item {Grp. gram.:v. t.}
\end{itemize}
Compor com dynamite, juntar dynamite a.
\section{Dynamização}
\begin{itemize}
\item {Grp. gram.:f.}
\end{itemize}
Acto ou effeito de dynamizar.
\section{Dynamizar}
\begin{itemize}
\item {Grp. gram.:v. t.}
\end{itemize}
\begin{itemize}
\item {Utilização:Neol.}
\end{itemize}
\begin{itemize}
\item {Proveniência:(De \textunderscore dynamia\textunderscore )}
\end{itemize}
Concentrar, dar carácter dynâmico a.
Elevar, concentrando, a energia therapêutica de, segundo a Medicina homeopáthica.
\section{Dýnamo}
\begin{itemize}
\item {Grp. gram.:m.}
\end{itemize}
\begin{itemize}
\item {Utilização:Phýs.}
\end{itemize}
\begin{itemize}
\item {Proveniência:(Do gr. \textunderscore dunamis\textunderscore )}
\end{itemize}
Nome, que se dá, por abreviatura, á máquina electro-dynâmica, que transforma a energia mecânica em energia eléctrica.
O mesmo que dynamia.
\section{Dynamogenia}
\begin{itemize}
\item {Grp. gram.:f.}
\end{itemize}
\begin{itemize}
\item {Proveniência:(Do gr. \textunderscore dunamis\textunderscore  + \textunderscore genes\textunderscore )}
\end{itemize}
Exaltação funccional de um órgão, sob a influência de qualquer excitação.
\section{Dynamomagnético}
\begin{itemize}
\item {Grp. gram.:adj.}
\end{itemize}
\begin{itemize}
\item {Proveniência:(De \textunderscore dynamia\textunderscore  + \textunderscore magnético\textunderscore )}
\end{itemize}
Relativo á dynâmica do magnetismo.
\section{Dynamometria}
\begin{itemize}
\item {Grp. gram.:f.}
\end{itemize}
\begin{itemize}
\item {Proveniência:(Do gr. \textunderscore dunamis\textunderscore  + \textunderscore metron\textunderscore )}
\end{itemize}
Applicação do dynamómetro.
\section{Dynamométrico}
\begin{itemize}
\item {Grp. gram.:adj.}
\end{itemize}
Relativo á dynamometria.
\section{Dynamómetro}
\begin{itemize}
\item {Grp. gram.:m.}
\end{itemize}
\begin{itemize}
\item {Proveniência:(Do gr. \textunderscore dunamis\textunderscore  + \textunderscore metron\textunderscore )}
\end{itemize}
Instrumento, para avaliar em pêso a fôrça e os effeitos de uma máquina.
Apparelho, para medir a fôrça muscular.
\section{Dynamóphoro}
\begin{itemize}
\item {Grp. gram.:m.}
\end{itemize}
\begin{itemize}
\item {Utilização:Bot.}
\end{itemize}
\begin{itemize}
\item {Proveniência:(Do gr. \textunderscore dunamis\textunderscore  + \textunderscore phero\textunderscore )}
\end{itemize}
Órgão de certos óvulos vegetaes, que fecha o micrópylo, depois da fecundação do óvulo. Cf. Caminhoá, \textunderscore Bot. Ger. e Med.\textunderscore , II, 1002.
\section{Dynamoscopia}
\begin{itemize}
\item {Grp. gram.:f.}
\end{itemize}
Observação pathológica, que consiste em collocar um dedo do doente no ouvido do observador, para avaliar a fôrça e saúde do observado, segundo a fôrça e a continuidade do zumbido ou rumor que se ouve.
(Cp. \textunderscore dinamoscópio\textunderscore )
\section{Dynamoscópio}
\begin{itemize}
\item {Grp. gram.:m.}
\end{itemize}
\begin{itemize}
\item {Proveniência:(Do gr. \textunderscore dunamis\textunderscore  + \textunderscore skopein\textunderscore )}
\end{itemize}
Instrumento, que se applica na dynamoscopia.
\section{Dynasta}
\begin{itemize}
\item {Grp. gram.:m.}
\end{itemize}
\begin{itemize}
\item {Proveniência:(Gr. \textunderscore dunastes\textunderscore )}
\end{itemize}
Título antigo de príncipes soberanos.
Partidário de uma dynastia.
\section{Dynastia}
\begin{itemize}
\item {Grp. gram.:f.}
\end{itemize}
\begin{itemize}
\item {Utilização:Ext.}
\end{itemize}
\begin{itemize}
\item {Proveniência:(Gr. \textunderscore dunasteia\textunderscore )}
\end{itemize}
Série de soberanos, pertencentes á mesma família.
Série de reis.
Série de homens illustres na mesma família.
\section{Dynástico}
\begin{itemize}
\item {Grp. gram.:adj.}
\end{itemize}
Relativo a dynastia.
\section{Dyostylo}
\begin{itemize}
\item {Grp. gram.:m.}
\end{itemize}
\begin{itemize}
\item {Proveniência:(Do gr. \textunderscore duo\textunderscore  + \textunderscore stulos\textunderscore )}
\end{itemize}
Fachada de duas columnas.
\section{Dyótypa}
\begin{itemize}
\item {Grp. gram.:f.}
\end{itemize}
Máquina de composição typográphica, inventada em 1905.
\section{Dysarthria}
\begin{itemize}
\item {Grp. gram.:f.}
\end{itemize}
Difficuldade na pronúncia e articulação das palavras.
\section{Dysbasia}
\begin{itemize}
\item {Grp. gram.:f.}
\end{itemize}
\begin{itemize}
\item {Proveniência:(Do gr. \textunderscore dus\textunderscore  + \textunderscore basis\textunderscore )}
\end{itemize}
Aberração no andar, procedente de perturbação do systema nervoso.
\section{Dyschroia}
\begin{itemize}
\item {Grp. gram.:f.}
\end{itemize}
\begin{itemize}
\item {Utilização:Med.}
\end{itemize}
\begin{itemize}
\item {Proveniência:(Do gr. \textunderscore dus\textunderscore  + \textunderscore khroia\textunderscore )}
\end{itemize}
Má côr da pelle.
\section{Dyschromático}
\begin{itemize}
\item {Grp. gram.:adj.}
\end{itemize}
\begin{itemize}
\item {Proveniência:(Do gr. \textunderscore dus\textunderscore  + \textunderscore khroma\textunderscore )}
\end{itemize}
Que altera as cores.
Que não tem bôa côr.
\section{Dyschromatopsia}
\begin{itemize}
\item {Grp. gram.:f.}
\end{itemize}
\begin{itemize}
\item {Proveniência:(Do gr. \textunderscore dus\textunderscore  + \textunderscore khroma\textunderscore  + \textunderscore opsis\textunderscore )}
\end{itemize}
Espécie de daltonismo, em que a vista confunde certas côres com outras que distingue.
\section{Dyscinesia}
\begin{itemize}
\item {Grp. gram.:f.}
\end{itemize}
\begin{itemize}
\item {Utilização:Med.}
\end{itemize}
\begin{itemize}
\item {Proveniência:(Do gr. \textunderscore dus\textunderscore  + \textunderscore kinesís\textunderscore )}
\end{itemize}
Deminuição ou extincção dos movimentos voluntários.
\section{Dýscolo}
\begin{itemize}
\item {Grp. gram.:adj.}
\end{itemize}
\begin{itemize}
\item {Grp. gram.:M.}
\end{itemize}
\begin{itemize}
\item {Proveniência:(Lat. \textunderscore dyscolus\textunderscore )}
\end{itemize}
Que é áspero no trato.
Mal humorado.
Desordeiro.
Dissidente.
Aquelle, que tem mau gênio.
Brigão.
\section{Dyscrasia}
\begin{itemize}
\item {Grp. gram.:f.}
\end{itemize}
\begin{itemize}
\item {Proveniência:(Gr. \textunderscore duscrasia\textunderscore )}
\end{itemize}
Maus humores, má constituição phýsica.
Alteração de humores.
\section{Dyscrásico}
\begin{itemize}
\item {Grp. gram.:adj.}
\end{itemize}
\begin{itemize}
\item {Grp. gram.:M.}
\end{itemize}
Relativo á dyscrasia.
Aquelle que padece dyscrasia.
\section{Dysenteria}
\begin{itemize}
\item {Grp. gram.:f.}
\end{itemize}
\begin{itemize}
\item {Proveniência:(Gr. \textunderscore dusenteria\textunderscore )}
\end{itemize}
Inflammação dos intestinos, de que resultam evacuações mucosas ou purulentas, ás vezes misturadas com sangue.
\section{Dysentérico}
\begin{itemize}
\item {Grp. gram.:adj.}
\end{itemize}
\begin{itemize}
\item {Grp. gram.:M.}
\end{itemize}
Relativo á dysenteria.
Aquelle que soffre dysenteria.
\section{Dysesthesia}
\begin{itemize}
\item {Grp. gram.:f.}
\end{itemize}
\begin{itemize}
\item {Utilização:Med.}
\end{itemize}
\begin{itemize}
\item {Proveniência:(Do gr. \textunderscore dus\textunderscore  + \textunderscore aisthesis\textunderscore )}
\end{itemize}
Enfraquecimento ou extincção da acção dos sentidos.
\section{Dyslalia}
\begin{itemize}
\item {Grp. gram.:f.}
\end{itemize}
\begin{itemize}
\item {Proveniência:(Do gr. \textunderscore dus\textunderscore  + \textunderscore lalein\textunderscore )}
\end{itemize}
Difficuldade em articular palavras.
\section{Dyslexia}
\begin{itemize}
\item {fónica:csi}
\end{itemize}
\begin{itemize}
\item {Grp. gram.:f.}
\end{itemize}
Difficuldade de lêr e comprehender a escrita.
\section{Dysmenia}
\begin{itemize}
\item {Grp. gram.:f.}
\end{itemize}
O mesmo que \textunderscore dysmenorrheia\textunderscore .
\section{Dysmenorrheia}
\begin{itemize}
\item {Grp. gram.:f.}
\end{itemize}
\begin{itemize}
\item {Proveniência:(Do gr. \textunderscore dus\textunderscore  + \textunderscore men\textunderscore  + \textunderscore rhein\textunderscore )}
\end{itemize}
Corrimento menstrual, diffícil e doloroso.
Difficuldade da menstruação.
\section{Dysmenorrheico}
\begin{itemize}
\item {Grp. gram.:adj.}
\end{itemize}
Relativo a dysmenorrheia.
\section{Dysmnesia}
\begin{itemize}
\item {Grp. gram.:f.}
\end{itemize}
\begin{itemize}
\item {Proveniência:(Do gr. \textunderscore dus\textunderscore  + \textunderscore mnesis\textunderscore )}
\end{itemize}
Enfraquecimento da memória.
\section{Dysodia}
\begin{itemize}
\item {Grp. gram.:f.}
\end{itemize}
\begin{itemize}
\item {Proveniência:(Do gr. \textunderscore dus\textunderscore  + \textunderscore ozein\textunderscore )}
\end{itemize}
Mau cheiro das secreções.
\section{Dysopia}
\begin{itemize}
\item {Grp. gram.:f.}
\end{itemize}
\begin{itemize}
\item {Proveniência:(Do gr. \textunderscore dus\textunderscore  + \textunderscore ops\textunderscore )}
\end{itemize}
Enfraquecimento da vista.
\section{Dysorexia}
\begin{itemize}
\item {fónica:csi}
\end{itemize}
\begin{itemize}
\item {Grp. gram.:f.}
\end{itemize}
Falta de appetite.
\section{Dysosmia}
\begin{itemize}
\item {Grp. gram.:f.}
\end{itemize}
\begin{itemize}
\item {Proveniência:(Do gr. \textunderscore dus\textunderscore  + \textunderscore osme\textunderscore )}
\end{itemize}
Enfraquecimento do olfato.
\section{Dyspepsia}
\begin{itemize}
\item {Grp. gram.:f.}
\end{itemize}
\begin{itemize}
\item {Proveniência:(Gr. \textunderscore duspepsia\textunderscore )}
\end{itemize}
Difficuldade na digestão.
Má digestão.
\section{Dyspéptico}
\begin{itemize}
\item {Grp. gram.:adj.}
\end{itemize}
\begin{itemize}
\item {Grp. gram.:M.}
\end{itemize}
Relativo a dyspepsia.
Aquelle que soffre dyspepsia.
\section{Dysphagia}
\begin{itemize}
\item {Grp. gram.:f.}
\end{itemize}
\begin{itemize}
\item {Proveniência:(Do gr. \textunderscore dus\textunderscore  + \textunderscore phagein\textunderscore )}
\end{itemize}
Difficuldade em engulir.
\section{Dysphasia}
\begin{itemize}
\item {Grp. gram.:f.}
\end{itemize}
\begin{itemize}
\item {Utilização:Med.}
\end{itemize}
Qualquer difficuldade no falar.
\section{Dysphonia}
\begin{itemize}
\item {Grp. gram.:f.}
\end{itemize}
\begin{itemize}
\item {Utilização:Med.}
\end{itemize}
\begin{itemize}
\item {Proveniência:(Do gr. \textunderscore dus\textunderscore  + \textunderscore phone\textunderscore )}
\end{itemize}
Alteração da voz e da palavra.
\section{Dysphoria}
\begin{itemize}
\item {Grp. gram.:f.}
\end{itemize}
\begin{itemize}
\item {Proveniência:(Do gr. \textunderscore dus\textunderscore  + \textunderscore pherein\textunderscore )}
\end{itemize}
Indisposição mórbida; mal-estar.
\section{Dyspneia}
\begin{itemize}
\item {Grp. gram.:f.}
\end{itemize}
\begin{itemize}
\item {Proveniência:(Gr. \textunderscore duspnoia\textunderscore )}
\end{itemize}
Difficuldade de respirar.
\section{Dyspneico}
\begin{itemize}
\item {Grp. gram.:adj.}
\end{itemize}
Relativo á dyspneia.
\section{Dyssymetria}
\begin{itemize}
\item {Grp. gram.:f.}
\end{itemize}
\begin{itemize}
\item {Proveniência:(Do gr. \textunderscore dus\textunderscore  + \textunderscore summetria\textunderscore )}
\end{itemize}
Falta de symetria.
\section{Dyssymétrico}
\begin{itemize}
\item {Grp. gram.:adj.}
\end{itemize}
Em que ha dyssymetria.
\section{Dysthanásia}
\begin{itemize}
\item {Grp. gram.:f.}
\end{itemize}
\begin{itemize}
\item {Proveniência:(Do gr. \textunderscore dus\textunderscore  + \textunderscore thanatos\textunderscore )}
\end{itemize}
Morte dolorosa.
\section{Dystócia}
\begin{itemize}
\item {Grp. gram.:f.}
\end{itemize}
\begin{itemize}
\item {Proveniência:(Do gr. \textunderscore dus\textunderscore  + \textunderscore tokos\textunderscore )}
\end{itemize}
Parto diffícil.
\section{Dystocíaco}
\begin{itemize}
\item {Grp. gram.:adj}
\end{itemize}
Relativo á dystocia.
\section{Dystrophia}
\begin{itemize}
\item {Grp. gram.:f.}
\end{itemize}
\begin{itemize}
\item {Utilização:Med.}
\end{itemize}
Perturbação da nutrição.
Doença, produzida por má nutrição.
\section{Dystróphico}
\begin{itemize}
\item {Grp. gram.:adj.}
\end{itemize}
\begin{itemize}
\item {Proveniência:(Do gr. \textunderscore dus\textunderscore  + \textunderscore trophe\textunderscore )}
\end{itemize}
Relativo á dystrophia.
Que prejudica a nutrição.
Que se alimenta mal.
\section{Dysuria}
\begin{itemize}
\item {Grp. gram.:f.}
\end{itemize}
\begin{itemize}
\item {Proveniência:(Gr. \textunderscore dusouria\textunderscore )}
\end{itemize}
Difficuldade em urinar.
\section{Dysúrico}
\begin{itemize}
\item {Grp. gram.:adj.}
\end{itemize}
\begin{itemize}
\item {Grp. gram.:M.}
\end{itemize}
Relativo á dysuria.
Aquelle que padece dysuria.
\section{Dýtico}
\begin{itemize}
\item {Grp. gram.:adj.}
\end{itemize}
\begin{itemize}
\item {Grp. gram.:m.}
\end{itemize}
\begin{itemize}
\item {Grp. gram.:M. pl.}
\end{itemize}
\begin{itemize}
\item {Proveniência:(Do gr. \textunderscore duein\textunderscore )}
\end{itemize}
\end{document}