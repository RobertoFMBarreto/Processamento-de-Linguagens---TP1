\documentclass{article}
\usepackage[portuguese]{babel}
\title{T}
\begin{document}
O mesmo que \textunderscore syzýgia\textunderscore .
Gênero de plantas myrtáceas.
(Cp. \textunderscore syzýgia\textunderscore )
\section{Tremido}
\begin{itemize}
\item {Grp. gram.:adj.}
\end{itemize}
\begin{itemize}
\item {Utilização:Fam.}
\end{itemize}
\begin{itemize}
\item {Grp. gram.:M.}
\end{itemize}
Duvidoso, arriscado: \textunderscore isso é negócio muito tremido\textunderscore .
Tremor.
Tortuoidade; linha sinuosa.
\section{Tremifusa}
\begin{itemize}
\item {Grp. gram.:f.}
\end{itemize}
\begin{itemize}
\item {Proveniência:(De \textunderscore tremer\textunderscore  e \textunderscore fusa\textunderscore )}
\end{itemize}
O mesmo que \textunderscore trifusa\textunderscore .
\section{Tremilongo}
\begin{itemize}
\item {Grp. gram.:m.}
\end{itemize}
O mesmo que \textunderscore pernilongo\textunderscore , ave.
\section{Tremisse}
\begin{itemize}
\item {Grp. gram.:m.}
\end{itemize}
\begin{itemize}
\item {Utilização:Ant.}
\end{itemize}
\begin{itemize}
\item {Proveniência:(Do lat. \textunderscore tremis\textunderscore , \textunderscore tremissis\textunderscore )}
\end{itemize}
A têrça parte de um sôldo. Cf. S. C. Viterbo, \textunderscore Elucidário\textunderscore .
\section{Trêmito}
\begin{itemize}
\item {Grp. gram.:m.}
\end{itemize}
\begin{itemize}
\item {Utilização:bras}
\end{itemize}
\begin{itemize}
\item {Utilização:Neol.}
\end{itemize}
\begin{itemize}
\item {Proveniência:(Do lat. \textunderscore tremitus\textunderscore )}
\end{itemize}
O mesmo que \textunderscore frêmito\textunderscore .
\section{Tremó}
\begin{itemize}
\item {Grp. gram.:m.}
\end{itemize}
\begin{itemize}
\item {Proveniência:(Do gr. \textunderscore trumeau\textunderscore )}
\end{itemize}
Apparador antigo, com espêlho alto, que cobre a parte da parede comprehendida entre duas janelas.
Espaço de parede entre duas janelas.
\section{Tremoçada}
\begin{itemize}
\item {Grp. gram.:f.}
\end{itemize}
\begin{itemize}
\item {Utilização:Prov.}
\end{itemize}
Grande porção de tremoços.
Arremêsso de tremoços, em folganças de Carnaval.
\section{Tremoçal}
\begin{itemize}
\item {Grp. gram.:m.}
\end{itemize}
Terreno, onde crescem tremoços.
\section{Tremoção}
\begin{itemize}
\item {Grp. gram.:m.}
\end{itemize}
Planta leguminosa, (\textunderscore astragalus lusitanicus\textunderscore , Lam.).
\section{Tremoceira}
\begin{itemize}
\item {Grp. gram.:f.}
\end{itemize}
Vendedeira ambulante de tremoços.
\section{Tremoceiro}
\begin{itemize}
\item {Grp. gram.:m.}
\end{itemize}
\begin{itemize}
\item {Proveniência:(De \textunderscore tremoço\textunderscore )}
\end{itemize}
Planta leguminosa, cujas vagens têm grãos comestíveis.
\section{Tremocilho}
\begin{itemize}
\item {Grp. gram.:m.}
\end{itemize}
\begin{itemize}
\item {Utilização:Prov.}
\end{itemize}
\begin{itemize}
\item {Utilização:alent.}
\end{itemize}
Tremoço bravo.
\section{Tremoço}
\begin{itemize}
\item {fónica:mô}
\end{itemize}
\begin{itemize}
\item {Grp. gram.:m.}
\end{itemize}
\begin{itemize}
\item {Proveniência:(Do ár. \textunderscore at-tormos\textunderscore )}
\end{itemize}
Grão de tremoceiro, em fórma de disco.
O tremoceiro.
\section{Tremoços-de-chelro}
\begin{itemize}
\item {Grp. gram.:m. Pl.}
\end{itemize}
\begin{itemize}
\item {Utilização:Bras}
\end{itemize}
Planta, o mesmo que \textunderscore lupino\textunderscore .
\section{Tremóia}
\begin{itemize}
\item {Grp. gram.:f.}
\end{itemize}
\begin{itemize}
\item {Utilização:Prov.}
\end{itemize}
O mesmo que \textunderscore tremonha\textunderscore .
\section{Tremoicela}
\begin{itemize}
\item {Grp. gram.:f.}
\end{itemize}
\begin{itemize}
\item {Utilização:Prov.}
\end{itemize}
\begin{itemize}
\item {Utilização:trasm.}
\end{itemize}
Peça de madeira, que serve como de temão no canamão da trilha.
\section{Tremolito}
\begin{itemize}
\item {Grp. gram.:m.}
\end{itemize}
\begin{itemize}
\item {Utilização:Miner.}
\end{itemize}
\begin{itemize}
\item {Proveniência:(De \textunderscore Tremola\textunderscore , n. p.)}
\end{itemize}
Espécie de silicato, descoberto nos Alpes pelo padre Pini.
\section{Tremonha}
\begin{itemize}
\item {Grp. gram.:f.}
\end{itemize}
\begin{itemize}
\item {Proveniência:(Do lat. \textunderscore trimodia\textunderscore ?)}
\end{itemize}
Peça do moínho, em fórma de pyrâmide quadrada e invertida, por cuja extremidade inferior passa o grão que vai ser moído; canoira; dorneira.
Pyrâmide ôca, composta de differentes zonas de crystaes precipitados, cujos fragmentos são as pedras de sal.
\section{Tremonhado}
\begin{itemize}
\item {Grp. gram.:m.}
\end{itemize}
\begin{itemize}
\item {Proveniência:(De \textunderscore tremonha\textunderscore )}
\end{itemize}
Lugar, utensílio ou vaso, onde cái a farinha que se vai moendo.
\section{Tremonhal}
\begin{itemize}
\item {Grp. gram.:m.}
\end{itemize}
\begin{itemize}
\item {Utilização:Prov.}
\end{itemize}
\begin{itemize}
\item {Utilização:trasm.}
\end{itemize}
O mesmo que \textunderscore tremonhado\textunderscore .
\section{Tremontelo}
\begin{itemize}
\item {fónica:tê}
\end{itemize}
\begin{itemize}
\item {Grp. gram.:m.}
\end{itemize}
Variedade de tomilho bravo.
\section{Tremor}
\begin{itemize}
\item {Grp. gram.:m.}
\end{itemize}
\begin{itemize}
\item {Proveniência:(Lat. \textunderscore tremor\textunderscore )}
\end{itemize}
Acto ou effeito de tremer.
Movimento convulsivo.
Temor.
\section{Trempe}
\begin{itemize}
\item {Grp. gram.:f.}
\end{itemize}
\begin{itemize}
\item {Utilização:Bras. do N}
\end{itemize}
\begin{itemize}
\item {Utilização:Fam.}
\end{itemize}
\begin{itemize}
\item {Utilização:Náut.}
\end{itemize}
\begin{itemize}
\item {Utilização:Fig.}
\end{itemize}
\begin{itemize}
\item {Proveniência:(Do gr. \textunderscore tripous\textunderscore )}
\end{itemize}
Arco de ferro, sustentado por três peças perpendiculares, que constitue o utensílio culinário, sôbre que se assenta a caçarola ou a panela.
Jôgo de cartas, espécie de manilha, com três parceiros.
Conjunto de três pedras, em que se assenta a panela, ao lume.
Conjunto de três pessôas, reunidas para o mesmo fim ou por interesse commum.
Jangada de três paus.
Armadilha, laço:«\textunderscore ...e trempe me não armas.\textunderscore »Filinto, XII, 157.
\section{Trempem}
\begin{itemize}
\item {Grp. gram.:f.}
\end{itemize}
O mesmo que \textunderscore trempe\textunderscore . Cf. \textunderscore Aulegrafia\textunderscore , 89.
\section{Tremudar}
\textunderscore v. t.\textunderscore  (e der.)
Fórma ant. de \textunderscore transmudar\textunderscore , etc.
\section{Tremulação}
\begin{itemize}
\item {Grp. gram.:f.}
\end{itemize}
\begin{itemize}
\item {Utilização:Neol.}
\end{itemize}
Acto de tremular.
\section{Tremulamente}
\begin{itemize}
\item {Grp. gram.:adv.}
\end{itemize}
De modo trêmulo.
\section{Tremulante}
\begin{itemize}
\item {Grp. gram.:adj.}
\end{itemize}
\begin{itemize}
\item {Proveniência:(Lat. \textunderscore tremulans\textunderscore )}
\end{itemize}
Que tremúla.
\section{Tçar}
\begin{itemize}
\item {Grp. gram.:m.}
\end{itemize}
\begin{itemize}
\item {Proveniência:(De russo \textunderscore tçari\textunderscore )}
\end{itemize}
Título de Soberano da Rússia.
\section{Tçarina}
\begin{itemize}
\item {Grp. gram.:f.}
\end{itemize}
\begin{itemize}
\item {Proveniência:(De \textunderscore tçar\textunderscore )}
\end{itemize}
Título da Imperatriz da Rússia.
\section{Triglo}
\begin{itemize}
\item {Grp. gram.:m.}
\end{itemize}
Peixe das profundidades do Oceano Índico, e cujo fígado é venenoso.
\section{Tripanossomíase}
\begin{itemize}
\item {Grp. gram.:f.}
\end{itemize}
O mesmo que \textunderscore tripanossomose\textunderscore .
\section{Tripanossomo}
\begin{itemize}
\item {Grp. gram.:m.}
\end{itemize}
\begin{itemize}
\item {Utilização:Med.}
\end{itemize}
\begin{itemize}
\item {Proveniência:(Do gr. \textunderscore trupanon\textunderscore  + \textunderscore soma\textunderscore )}
\end{itemize}
Protozoário, parasito do sangue e causador de várias doenças.
\section{Tripanossomose}
\begin{itemize}
\item {Grp. gram.:f.}
\end{itemize}
Doença, produzida pelo tripanosomo.
Doença do sono.
\section{Troteiro}
\begin{itemize}
\item {Grp. gram.:m.  e  adj.}
\end{itemize}
\begin{itemize}
\item {Utilização:Ant.}
\end{itemize}
\begin{itemize}
\item {Proveniência:(De \textunderscore trotar\textunderscore )}
\end{itemize}
O que anda a trote.
Postilhão.
\section{Trouçar}
\begin{itemize}
\item {Grp. gram.:v. i.}
\end{itemize}
\begin{itemize}
\item {Utilização:Ant.}
\end{itemize}
Passar adiante; têr a primazia.
Ser mais attendido.
\section{Trouciar}
\begin{itemize}
\item {Grp. gram.:v. i.}
\end{itemize}
\begin{itemize}
\item {Utilização:Ant.}
\end{itemize}
Passar adiante; têr a primazia.
Ser mais attendido.
\section{Trova}
\begin{itemize}
\item {Grp. gram.:f.}
\end{itemize}
\begin{itemize}
\item {Proveniência:(De \textunderscore trovar\textunderscore )}
\end{itemize}
Composição lýrica, ligeira e de carácter mais ou menos popular.
Cantiga, canção.
\section{Trovão}
\begin{itemize}
\item {Grp. gram.:m.}
\end{itemize}
Ruído, produzido por descarga de electricidade atmosphérica.
Grande estrondo.
Coisa ruidosa ou espantosa.
(Talvez por \textunderscore troão\textunderscore , de \textunderscore trom\textunderscore )
\section{Trovador}
\begin{itemize}
\item {Grp. gram.:m.}
\end{itemize}
\begin{itemize}
\item {Utilização:Ext.}
\end{itemize}
\begin{itemize}
\item {Proveniência:(De \textunderscore trovar\textunderscore ^1)}
\end{itemize}
Diz-se especialmente dos poétas da língua de oc., que floresceram desde o século XI ao XIV e cultivaram a poésia lýrica.
Diz-se dos poétas lýricos portugueses que, naquelles tempos, imitaram a poesia provençal.
Poéta, lýrico.
\section{Trovante}
\begin{itemize}
\item {Grp. gram.:m.}
\end{itemize}
\begin{itemize}
\item {Utilização:Des.}
\end{itemize}
\begin{itemize}
\item {Proveniência:(De \textunderscore trovar\textunderscore ^1)}
\end{itemize}
Aquelle que faz trovas. Cf. \textunderscore Fenix Renasc.\textunderscore , IV, 43.
\section{Trovar}
\begin{itemize}
\item {Grp. gram.:v. i.}
\end{itemize}
\begin{itemize}
\item {Grp. gram.:V. t.}
\end{itemize}
Fazer ou cantar trovas.
Exprimir em cantigas.
(Provn. \textunderscore trobar\textunderscore )
\section{Trovar}
\begin{itemize}
\item {Grp. gram.:v. t.}
\end{itemize}
\begin{itemize}
\item {Utilização:Ant.}
\end{itemize}
O mesmo que \textunderscore turvar\textunderscore .
\section{Troveiro}
\begin{itemize}
\item {Grp. gram.:m.}
\end{itemize}
\begin{itemize}
\item {Proveniência:(Fr. \textunderscore trouvère\textunderscore )}
\end{itemize}
Diz-se dos trovadores ou poétas da língua de oíl, que cultivaram a poésia épica medieval, e que floresceram em França desde o século XI ao XIV. Cf. Garrett, \textunderscore Romanceiro\textunderscore , II, 30.
\section{Trovejante}
\begin{itemize}
\item {Grp. gram.:adj.}
\end{itemize}
Que troveja.
\section{Trovejar}
\begin{itemize}
\item {Grp. gram.:v. i.}
\end{itemize}
\begin{itemize}
\item {Utilização:Fig.}
\end{itemize}
\begin{itemize}
\item {Grp. gram.:V. t.}
\end{itemize}
\begin{itemize}
\item {Grp. gram.:M.}
\end{itemize}
\begin{itemize}
\item {Proveniência:(De \textunderscore trovão\textunderscore )}
\end{itemize}
Soar o trovão.
Haver trovões.
Soar com grande estrondo; estrondear.
Clamar.
Pronunciar ou emittir com grande ruído: \textunderscore trovejar pragas\textunderscore .
Trovão; estampido.
\section{Troviscada}
\begin{itemize}
\item {Grp. gram.:f.}
\end{itemize}
\begin{itemize}
\item {Proveniência:(De \textunderscore trovisco\textunderscore )}
\end{itemize}
Porção de trovisco, com que se envenenam os peixes para os pescar.
Entroviscada.
\section{Troviscal}
\begin{itemize}
\item {Grp. gram.:m.}
\end{itemize}
Terreno, onde crescem troviscos.
\section{Troviscar}
\begin{itemize}
\item {Grp. gram.:v. i.}
\end{itemize}
\begin{itemize}
\item {Utilização:Pop.}
\end{itemize}
\begin{itemize}
\item {Proveniência:(De \textunderscore trovão\textunderscore )}
\end{itemize}
Trovejar um pouco.
\section{Trovisco}
\begin{itemize}
\item {Grp. gram.:m.}
\end{itemize}
\begin{itemize}
\item {Proveniência:(Do lat. \textunderscore turviscus\textunderscore )}
\end{itemize}
Arbusto thymeliáceo.
\section{Trovisco-macho}
\begin{itemize}
\item {Grp. gram.:m.}
\end{itemize}
Gênero de plantas daphnáceas, (\textunderscore daphne gnidium\textunderscore , Lin.).
\section{Trovisqueira}
\begin{itemize}
\item {Grp. gram.:f.}
\end{itemize}
O mesmo que \textunderscore trovisco\textunderscore .
\section{Trovista}
\begin{itemize}
\item {Grp. gram.:m.}
\end{itemize}
\begin{itemize}
\item {Utilização:Deprec.}
\end{itemize}
\begin{itemize}
\item {Proveniência:(De \textunderscore trova\textunderscore )}
\end{itemize}
Aquelle que faz trovas.
O mesmo que \textunderscore poetastro\textunderscore .
\section{Trovoada}
\begin{itemize}
\item {Grp. gram.:f.}
\end{itemize}
\begin{itemize}
\item {Utilização:Fig.}
\end{itemize}
\begin{itemize}
\item {Utilização:Prov.}
\end{itemize}
\begin{itemize}
\item {Utilização:trasm.}
\end{itemize}
\begin{itemize}
\item {Utilização:T. de Alijó}
\end{itemize}
\begin{itemize}
\item {Proveniência:(De \textunderscore trovoar\textunderscore )}
\end{itemize}
Qualidade de trovões successivos.
Grande estrondo.
Altercação.
Balbúrdia.
Ralhos.
Aspecto carregado.
Bebedeira.
\section{Trovoar}
\begin{itemize}
\item {Grp. gram.:v. t.  e  i.}
\end{itemize}
\begin{itemize}
\item {Proveniência:(De \textunderscore trovão\textunderscore )}
\end{itemize}
O mesmo que \textunderscore trovejar\textunderscore .
\section{Trovoso}
\begin{itemize}
\item {Grp. gram.:adj.}
\end{itemize}
\begin{itemize}
\item {Proveniência:(De \textunderscore trovão\textunderscore )}
\end{itemize}
Trovejante; que faz grande estrondo.
\section{Troxe-moxe}
\begin{itemize}
\item {Grp. gram.:m.}
\end{itemize}
O mesmo que [[troixe-moixe, a]].
\section{Truanaz}
\begin{itemize}
\item {Grp. gram.:m.}
\end{itemize}
O mesmo que \textunderscore truão\textunderscore .
\section{Truanear}
\begin{itemize}
\item {Grp. gram.:v. i.}
\end{itemize}
\begin{itemize}
\item {Proveniência:(De \textunderscore truão\textunderscore )}
\end{itemize}
Fazer truanices.
\section{Truanesco}
\begin{itemize}
\item {fónica:nês}
\end{itemize}
\begin{itemize}
\item {Grp. gram.:adj.}
\end{itemize}
Relativo a truão.
Semelhante a truanices.
(Cast. \textunderscore truhanesco\textunderscore )
\section{Truania}
\begin{itemize}
\item {Grp. gram.:f.}
\end{itemize}
Acto ou dito de truão.
\section{Truanice}
\begin{itemize}
\item {Grp. gram.:f.}
\end{itemize}
Acto ou dito de truão.
\section{Truão}
\begin{itemize}
\item {Grp. gram.:m.}
\end{itemize}
Bobo.
Saltimbanco.
Palhaço; pelotiqueiro.
(Cast. \textunderscore truhán\textunderscore )
\section{Trucar}
\begin{itemize}
\item {Grp. gram.:v. i.}
\end{itemize}
Propor a primeira parada, no truque.
\section{Truchemão}
\begin{itemize}
\item {Grp. gram.:m.}
\end{itemize}
(V.turgimão). Cf. Filinto, XIII, 371.
\section{Trucidação}
\begin{itemize}
\item {Grp. gram.:f.}
\end{itemize}
Acto ou effeito de \textunderscore trucidar\textunderscore .
\section{Trucidar}
\begin{itemize}
\item {Grp. gram.:v. t.}
\end{itemize}
\begin{itemize}
\item {Proveniência:(Lat. \textunderscore trucidare\textunderscore )}
\end{itemize}
Matar barbaramente ou com crueldade.
Degollar.
\section{Trucilar}
\begin{itemize}
\item {Grp. gram.:m.}
\end{itemize}
\begin{itemize}
\item {Grp. gram.:V. i.}
\end{itemize}
\begin{itemize}
\item {Proveniência:(Lat. \textunderscore trucilare\textunderscore )}
\end{itemize}
O cantar do tordo.
Cantar, (falando-se dos tordos).
\section{Trusquiar}
\begin{itemize}
\item {Grp. gram.:v. t.}
\end{itemize}
\begin{itemize}
\item {Utilização:Ant.}
\end{itemize}
\begin{itemize}
\item {Grp. gram.:V. p.}
\end{itemize}
O mesmo que \textunderscore tosquiar\textunderscore .
Morrer, (por allusão ao antigo costume de se tosquiarem os defuntos). Cf. S. R. Viterbo, \textunderscore Elucidário\textunderscore .
(Cp. \textunderscore tosquiar\textunderscore )
\section{Truta}
\begin{itemize}
\item {Grp. gram.:f.}
\end{itemize}
\begin{itemize}
\item {Proveniência:(Lat. \textunderscore tructa\textunderscore )}
\end{itemize}
Peixe salmonídeo.
\section{Trutífero}
\begin{itemize}
\item {Grp. gram.:adj.}
\end{itemize}
\begin{itemize}
\item {Proveniência:(Do lat. \textunderscore tructa\textunderscore  + \textunderscore ferre\textunderscore )}
\end{itemize}
Que produz trutas:«\textunderscore ...o trutífero Moura.\textunderscore »\textunderscore Vir. Trág.\textunderscore , IV, 90.
\section{Truxálidos}
\begin{itemize}
\item {fónica:csá}
\end{itemize}
\begin{itemize}
\item {Grp. gram.:m. pl.}
\end{itemize}
\begin{itemize}
\item {Proveniência:(Do gr. \textunderscore truzein\textunderscore  + \textunderscore eidos\textunderscore )}
\end{itemize}
Tríbo de insectos, que tem por typo o trúxalo.
\section{Trúxalo}
\begin{itemize}
\item {fónica:csa}
\end{itemize}
\begin{itemize}
\item {Grp. gram.:m.}
\end{itemize}
\begin{itemize}
\item {Proveniência:(Do gr. \textunderscore truzein\textunderscore )}
\end{itemize}
Gênero de insectos saltadores.
\section{Truz!}
\begin{itemize}
\item {Grp. gram.:interj.}
\end{itemize}
\begin{itemize}
\item {Grp. gram.:M.}
\end{itemize}
\begin{itemize}
\item {Grp. gram.:Loc. adj.}
\end{itemize}
\begin{itemize}
\item {Proveniência:(T. onom.)}
\end{itemize}
Voz imitativa do som de uma quéda ou de uma explosão.
Pancada.
Acto de bater.
\textunderscore De truz\textunderscore , magnífico; de primeira ordem.
\section{Truz-truz}
\begin{itemize}
\item {Grp. gram.:m.  e  interj.}
\end{itemize}
\begin{itemize}
\item {Proveniência:(T. onom.)}
\end{itemize}
Voz imitativa do som produzido por quem bate a uma porta.
\section{Tryglo}
\begin{itemize}
\item {Grp. gram.:m.}
\end{itemize}
Peixe das profundidades do Oceano Índico, e cujo fígado é venenoso.
\section{Trypanosomíase}
\begin{itemize}
\item {fónica:só}
\end{itemize}
\begin{itemize}
\item {Grp. gram.:f.}
\end{itemize}
O mesmo que \textunderscore trypanosomose\textunderscore .
\section{Trypanosomo}
\begin{itemize}
\item {fónica:só}
\end{itemize}
\begin{itemize}
\item {Grp. gram.:m.}
\end{itemize}
\begin{itemize}
\item {Utilização:Med.}
\end{itemize}
\begin{itemize}
\item {Proveniência:(Do gr. \textunderscore trupanon\textunderscore  + \textunderscore soma\textunderscore )}
\end{itemize}
Protozoário, parasito do sangue e causador de várias doenças.
\section{Trypanosomose}
\begin{itemize}
\item {fónica:so}
\end{itemize}
\begin{itemize}
\item {Grp. gram.:f.}
\end{itemize}
Doença, produzida pelo trypanosomo.
Doença do somno.
\section{Tsar}
\begin{itemize}
\item {Grp. gram.:m.}
\end{itemize}
\begin{itemize}
\item {Proveniência:(De russo \textunderscore tçari\textunderscore )}
\end{itemize}
Título de Soberano da Rússia.
\section{Tsarina}
\begin{itemize}
\item {Grp. gram.:f.}
\end{itemize}
\begin{itemize}
\item {Proveniência:(De \textunderscore tsar\textunderscore )}
\end{itemize}
Título da Imperatriz da Rússia.
\section{Tsetsé}
\begin{itemize}
\item {Grp. gram.:m.}
\end{itemize}
Espécie de môsca venenosa do interior da África meridional.
\section{Tu}
\begin{itemize}
\item {Grp. gram.:pron. pess.}
\end{itemize}
\begin{itemize}
\item {Grp. gram.:M.}
\end{itemize}
\begin{itemize}
\item {Proveniência:(Lat. \textunderscore tu\textunderscore )}
\end{itemize}
Pron. pess. da segunda pessôa do singular, com que se designa a pessôa com quem se fala.
Tratamento de tu.
\section{Tu}
\begin{itemize}
\item {Grp. gram.:adj.}
\end{itemize}
O mesmo que \textunderscore banto\textunderscore .
\section{Tua}
\begin{itemize}
\item {Grp. gram.:adj.}
\end{itemize}
\begin{itemize}
\item {Proveniência:(Lat. \textunderscore tua\textunderscore )}
\end{itemize}
(Fem. de \textunderscore teu\textunderscore )
\section{Tua}
\begin{itemize}
\item {Grp. gram.:f.}
\end{itemize}
Ave pernalta da África.
\section{Tuaca}
\begin{itemize}
\item {Grp. gram.:f.}
\end{itemize}
O mesmo que \textunderscore sagu\textunderscore .
\section{Tuaiuçu}
\begin{itemize}
\item {Grp. gram.:m.}
\end{itemize}
O mesmo que \textunderscore utuapoca\textunderscore .
\section{Tuaqueira}
\begin{itemize}
\item {Grp. gram.:f.}
\end{itemize}
Espécie de palmeira, de que se extrai a tuaca.
\section{Tuaupoca}
\begin{itemize}
\item {Grp. gram.:f.}
\end{itemize}
\begin{itemize}
\item {Utilização:Bras}
\end{itemize}
O memso que \textunderscore tuapoca\textunderscore .
\section{Tuba}
\begin{itemize}
\item {Grp. gram.:f.}
\end{itemize}
\begin{itemize}
\item {Utilização:Poét.}
\end{itemize}
\begin{itemize}
\item {Proveniência:(Lat. \textunderscore tuba\textunderscore )}
\end{itemize}
O mesmo que \textunderscore trombeta\textunderscore ^1.
\section{Tuba}
\begin{itemize}
\item {Grp. gram.:f.}
\end{itemize}
\begin{itemize}
\item {Utilização:Bras}
\end{itemize}
O mesmo que \textunderscore tiba\textunderscore ^2.
\section{Tubáceo}
\begin{itemize}
\item {Grp. gram.:adj.}
\end{itemize}
Que tem fórma de tuba.
\section{Tubagem}
\begin{itemize}
\item {Grp. gram.:f.}
\end{itemize}
Conjunto de tubos.
Systema, por que estão dispostos ou por que funccionam certos tubos.
\section{Túbara}
\begin{itemize}
\item {Grp. gram.:f.}
\end{itemize}
(V.túbera)
\section{Tubarana}
\begin{itemize}
\item {Grp. gram.:f.}
\end{itemize}
\begin{itemize}
\item {Utilização:Bras}
\end{itemize}
Peixe fluvial, muito apreciado.
\section{Tubarão}
\begin{itemize}
\item {Grp. gram.:m.}
\end{itemize}
Cetáceo, notável pela sua voracidade e pelo seu comportamento, que attinge 10 metros.
(Cp. cast. \textunderscore tiburón\textunderscore )
\section{Tubário}
\begin{itemize}
\item {Grp. gram.:adj.}
\end{itemize}
\begin{itemize}
\item {Utilização:Anat.}
\end{itemize}
\begin{itemize}
\item {Proveniência:(De \textunderscore tubo\textunderscore )}
\end{itemize}
Relativo aos tubos dos brônchos.
Relativo aos canaes, que partem dos ângulos superiores da madre.
\section{Túbaro}
\begin{itemize}
\item {Grp. gram.:m.}
\end{itemize}
\begin{itemize}
\item {Utilização:Prov.}
\end{itemize}
\begin{itemize}
\item {Utilização:alent.}
\end{itemize}
O mesmo que \textunderscore túbera\textunderscore , no sentido de testículo.
\section{Tubeira}
\begin{itemize}
\item {Grp. gram.:f.}
\end{itemize}
Bôca ou extremidade de um tubo.
\section{Tubel}
\begin{itemize}
\item {Grp. gram.:m.}
\end{itemize}
\begin{itemize}
\item {Proveniência:(T. ár.)}
\end{itemize}
Escama, que resalta do metal candente, quando êste se bate.
\section{Túbera}
\begin{itemize}
\item {Grp. gram.:f.}
\end{itemize}
\begin{itemize}
\item {Utilização:Veter.}
\end{itemize}
\begin{itemize}
\item {Utilização:Prov.}
\end{itemize}
\begin{itemize}
\item {Proveniência:(Lat. \textunderscore tuber\textunderscore )}
\end{itemize}
Gênero de cogumelos.
Cogumelo subterrâneo, aromático e comestivel, o mesmo que \textunderscore trufa\textunderscore .
Endurecimento cutâneo; callo. Cf. Mac. Pinto, \textunderscore Comp de Veter.\textunderscore , I, 132.
Testículo de animal.
\section{Tubífero}
\begin{itemize}
\item {Grp. gram.:adj.}
\end{itemize}
\begin{itemize}
\item {Proveniência:(Do lat. \textunderscore tubus\textunderscore  + \textunderscore ferre\textunderscore )}
\end{itemize}
Que tem tubos.
\section{Tubifloras}
\begin{itemize}
\item {Grp. gram.:f. pl.}
\end{itemize}
\begin{itemize}
\item {Proveniência:(De \textunderscore tubifloro\textunderscore )}
\end{itemize}
Ordem de plantas, que comprehende as borragíneas e outras.
\section{Tubifloro}
\begin{itemize}
\item {Grp. gram.:adj.}
\end{itemize}
\begin{itemize}
\item {Utilização:Bot.}
\end{itemize}
\begin{itemize}
\item {Proveniência:(De \textunderscore tubo\textunderscore  + \textunderscore flôr\textunderscore )}
\end{itemize}
Diz-se da flôr, cua corolla tem o tubo muito alongado.
\section{Tubiforme}
\begin{itemize}
\item {Grp. gram.:adj.}
\end{itemize}
\begin{itemize}
\item {Proveniência:(De \textunderscore tubo\textunderscore  + \textunderscore fórma\textunderscore )}
\end{itemize}
Que tem fórma de tubo.
\section{Tubilústria}
\begin{itemize}
\item {Grp. gram.:f.}
\end{itemize}
\begin{itemize}
\item {Proveniência:(Lat. \textunderscore tubilustria\textunderscore , pl. de \textunderscore tubilustrium\textunderscore )}
\end{itemize}
Festa da benção das trombetas, empregadas nos sacrifícios, entre os Romanos. Cf. Castilho, \textunderscore Fastos\textunderscore , III, 87.
\section{Tubilústrio}
\begin{itemize}
\item {Grp. gram.:adj.}
\end{itemize}
Relativo á tubilústria. Cf. Castilho, \textunderscore Fastos\textunderscore , II, 97.
\section{Tubim}
\begin{itemize}
\item {Grp. gram.:m.}
\end{itemize}
\begin{itemize}
\item {Utilização:Bras}
\end{itemize}
O mesmo que \textunderscore tubi\textunderscore .
\section{Tubípora}
\begin{itemize}
\item {Grp. gram.:f.}
\end{itemize}
\begin{itemize}
\item {Proveniência:(T. hybr., do lat. \textunderscore tubus\textunderscore  + gr. \textunderscore poros\textunderscore )}
\end{itemize}
Gênero de pólypos, que apresenta tubos semelhantes aos de um órgão.
\section{Tubiporíneos}
\begin{itemize}
\item {Grp. gram.:m. pl.}
\end{itemize}
Família de pólypos, que tem por typo a \textunderscore tubipora\textunderscore .
\section{Tubitelos}
\begin{itemize}
\item {Grp. gram.:m. pl.}
\end{itemize}
\begin{itemize}
\item {Utilização:Zool.}
\end{itemize}
Família de aranhas, que comprehende as que fiam as suas teias tubulares em fórma de cesto.
\section{Tubo}
\begin{itemize}
\item {Grp. gram.:m.}
\end{itemize}
\begin{itemize}
\item {Proveniência:(Do lat. \textunderscore tubus\textunderscore )}
\end{itemize}
Canal, mais ou menos cylíndrico, por onde passam ou saem fluidos, liquídos, etc.
Vaso cylíndrico de vidro.
Qualquer canal de organismo humano.
Objecto, que tem a apparência de um tubo.
\section{Tubulação}
\begin{itemize}
\item {Grp. gram.:f.}
\end{itemize}
\begin{itemize}
\item {Proveniência:(Do lat. \textunderscore tubulatio\textunderscore )}
\end{itemize}
Collocação de um ou mais tubos; tubagem.
\section{Tubulado}
\begin{itemize}
\item {Grp. gram.:adj.}
\end{itemize}
\begin{itemize}
\item {Proveniência:(Do lat. \textunderscore tubulatus\textunderscore )}
\end{itemize}
O mesmo que \textunderscore tubiforme\textunderscore .
Cavado como um tubo.
\section{Tubuladura}
\begin{itemize}
\item {Grp. gram.:f.}
\end{itemize}
\begin{itemize}
\item {Proveniência:(De \textunderscore tubulado\textunderscore )}
\end{itemize}
Abertura num vaso, para adaptação de um tubo.
\section{Tubular}
\begin{itemize}
\item {Grp. gram.:adj.}
\end{itemize}
\begin{itemize}
\item {Proveniência:(De \textunderscore túbulo\textunderscore )}
\end{itemize}
O mesmo que \textunderscore tubiforme\textunderscore .
Que tem tabuladura ou tabuladuras.
\section{Tubulária}
\begin{itemize}
\item {Grp. gram.:f.}
\end{itemize}
Gênero de pólypos anthozoários.
(Cp. \textunderscore tubular\textunderscore )
\section{Tubulibrânchios}
\begin{itemize}
\item {fónica:qui}
\end{itemize}
\begin{itemize}
\item {Grp. gram.:m. pl.}
\end{itemize}
\begin{itemize}
\item {Proveniência:(De \textunderscore túbulo\textunderscore  + \textunderscore brânchias\textunderscore )}
\end{itemize}
Ordem de molluscos gasterópodes, de concha tubulosa.
\section{Tubulibrânquios}
\begin{itemize}
\item {Grp. gram.:m. pl.}
\end{itemize}
\begin{itemize}
\item {Proveniência:(De \textunderscore túbulo\textunderscore  + \textunderscore brânquias\textunderscore )}
\end{itemize}
Ordem de moluscos gasterópodes, de concha tubulosa.
\section{Tubulífero}
\begin{itemize}
\item {Grp. gram.:adj.}
\end{itemize}
\begin{itemize}
\item {Utilização:Zool.}
\end{itemize}
\begin{itemize}
\item {Grp. gram.:M. Pl.}
\end{itemize}
\begin{itemize}
\item {Proveniência:(Do lat. \textunderscore tubulus\textunderscore  + \textunderscore ferre\textunderscore )}
\end{itemize}
Que apresenta na sua superfície uma multidão de pequenos tubos, como certas esponjas.
Família de insectos hymenópteros.
\section{Tubulifloro}
\begin{itemize}
\item {Grp. gram.:adj.}
\end{itemize}
\begin{itemize}
\item {Utilização:Bot.}
\end{itemize}
\begin{itemize}
\item {Proveniência:(De \textunderscore túbulo\textunderscore  + \textunderscore flôr\textunderscore )}
\end{itemize}
Que tem flôres de corollas tubulosas.
\section{Tubuliforme}
\begin{itemize}
\item {Grp. gram.:adj.}
\end{itemize}
\begin{itemize}
\item {Proveniência:(De \textunderscore túbulo\textunderscore  + \textunderscore fórma\textunderscore )}
\end{itemize}
Que tem a fórma de pequeno tubo.
\section{Tubulita}
\begin{itemize}
\item {Grp. gram.:f.}
\end{itemize}
\begin{itemize}
\item {Proveniência:(De \textunderscore túbulo\textunderscore )}
\end{itemize}
Artéria fóssil.
Zoóphyto fóssil, de fórma tubular.
\section{Tubulite}
\begin{itemize}
\item {Grp. gram.:f.}
\end{itemize}
\begin{itemize}
\item {Proveniência:(De \textunderscore túbulo\textunderscore )}
\end{itemize}
Artéria fóssil.
Zoóphyto fóssil, de fórma tubular.
\section{Tubulito}
\begin{itemize}
\item {Grp. gram.:m.}
\end{itemize}
O mesmo ou melhór que \textunderscore tubulita\textunderscore .
\section{Túbulo}
\begin{itemize}
\item {Grp. gram.:m.}
\end{itemize}
\begin{itemize}
\item {Proveniência:(Lat. \textunderscore tubulus\textunderscore )}
\end{itemize}
Pequeno tubo.
\section{Tubuloso}
\begin{itemize}
\item {Grp. gram.:adj.}
\end{itemize}
\begin{itemize}
\item {Proveniência:(De \textunderscore túbulo\textunderscore )}
\end{itemize}
O mesmo que \textunderscore tubiforme\textunderscore .
Formado por um tubo.
\section{Tubulura}
\begin{itemize}
\item {Grp. gram.:f.}
\end{itemize}
\begin{itemize}
\item {Utilização:Chím.}
\end{itemize}
\begin{itemize}
\item {Proveniência:(De \textunderscore túbulo\textunderscore )}
\end{itemize}
Abertura de certos vasos, empregados pelos chímicos, á qual se adapta uma rolha ou tampa, passando por esta um pequeno tubo.
\section{Tuca}
\begin{itemize}
\item {Grp. gram.:f.}
\end{itemize}
O mesmo que \textunderscore júvia\textunderscore .
\section{Tucaíra}
\begin{itemize}
\item {Grp. gram.:f.}
\end{itemize}
Planta amaryllídea do Brasil.
\section{Tucanabóia}
\begin{itemize}
\item {Grp. gram.:f.}
\end{itemize}
\begin{itemize}
\item {Utilização:Bras}
\end{itemize}
Espécie de cobra das regiões do Amazonas.
\section{Tucanaré}
\begin{itemize}
\item {Grp. gram.:m.}
\end{itemize}
\begin{itemize}
\item {Utilização:Bras}
\end{itemize}
(V.tucunaré)
\section{Tucano}
\begin{itemize}
\item {Grp. gram.:m.}
\end{itemize}
\begin{itemize}
\item {Utilização:Bras}
\end{itemize}
Ave trepadora da América do Sul, de bico muito longo.
Constellação austral.
Árvore silvestre.
\section{Tucari}
\begin{itemize}
\item {Grp. gram.:m.}
\end{itemize}
O mesmo que \textunderscore júvia\textunderscore .
\section{Tuchado}
\begin{itemize}
\item {Grp. gram.:adj.}
\end{itemize}
Atuchado; endurecido; reforçado:«\textunderscore elle é de carnes tuchadas\textunderscore ». Lapa, \textunderscore Proc. de Vín.\textunderscore , 13. Cf. Castilho, \textunderscore Fausto\textunderscore , 6.
\section{Tuge}
\begin{itemize}
\item {Grp. gram.:m.}
\end{itemize}
\begin{itemize}
\item {Utilização:Ant.}
\end{itemize}
\begin{itemize}
\item {Utilização:Fam.}
\end{itemize}
\begin{itemize}
\item {Proveniência:(De \textunderscore tugir\textunderscore )}
\end{itemize}
Excremento humano.
\section{Tugido}
\begin{itemize}
\item {Grp. gram.:m.}
\end{itemize}
Acto de \textunderscore tugir\textunderscore .
\section{Tugir}
\begin{itemize}
\item {Grp. gram.:v. i.}
\end{itemize}
Falar baixo; dar sinal de si.
\section{Tugúrio}
\begin{itemize}
\item {Grp. gram.:m.}
\end{itemize}
\begin{itemize}
\item {Utilização:Ext.}
\end{itemize}
\begin{itemize}
\item {Proveniência:(Lat. \textunderscore tugurium\textunderscore )}
\end{itemize}
Habitação rústica.
Choça; cabana.
Refúgio, abrigo.
\section{Tui-aíca}
\begin{itemize}
\item {Grp. gram.:m.}
\end{itemize}
Espécie de papagaio do Brasil.
\section{Tuias}
\begin{itemize}
\item {Grp. gram.:m. pl.}
\end{itemize}
Homens de lavoira, entre os habitantes da antiga Índia Portuguesa.
\section{Tuição}
\begin{itemize}
\item {fónica:tu-i}
\end{itemize}
\begin{itemize}
\item {Grp. gram.:f.}
\end{itemize}
\begin{itemize}
\item {Utilização:Jur.}
\end{itemize}
\begin{itemize}
\item {Utilização:Des.}
\end{itemize}
\begin{itemize}
\item {Proveniência:(Do lat. \textunderscore tuitio\textunderscore )}
\end{itemize}
Acto de defender ou patrocinar; defesa judicial.
\section{Tuidara}
\begin{itemize}
\item {fónica:tu-i}
\end{itemize}
\begin{itemize}
\item {Grp. gram.:f.}
\end{itemize}
\begin{itemize}
\item {Utilização:Bras}
\end{itemize}
O mesmo que \textunderscore coruja\textunderscore .
\section{Tuiengia}
\begin{itemize}
\item {Grp. gram.:f.}
\end{itemize}
Embarcação chinesa.
\section{Tuijuva}
\begin{itemize}
\item {fónica:tu-i}
\end{itemize}
\begin{itemize}
\item {Grp. gram.:f.}
\end{itemize}
O mesmo que \textunderscore tatajuba\textunderscore .
\section{Tuim}
\begin{itemize}
\item {Grp. gram.:m.}
\end{itemize}
\begin{itemize}
\item {Utilização:Bras}
\end{itemize}
Pequeno papagaio.
\section{Tuínha}
\begin{itemize}
\item {Grp. gram.:f.}
\end{itemize}
\begin{itemize}
\item {Utilização:Prov.}
\end{itemize}
O mesmo que \textunderscore chincra\textunderscore .
\section{Tuíra}
\begin{itemize}
\item {Grp. gram.:adj.}
\end{itemize}
\begin{itemize}
\item {Utilização:Bras}
\end{itemize}
Pardo; cinzento.
(Do tupi)
\section{Tuita}
\begin{itemize}
\item {fónica:tu-i}
\end{itemize}
\begin{itemize}
\item {Grp. gram.:f.}
\end{itemize}
Acto ou effeito de tuitar.
\section{Tuitar}
\begin{itemize}
\item {fónica:tu-i}
\end{itemize}
\begin{itemize}
\item {Grp. gram.:v. t.}
\end{itemize}
\begin{itemize}
\item {Utilização:Ant.}
\end{itemize}
\begin{itemize}
\item {Proveniência:(Do lat. \textunderscore tuitus\textunderscore )}
\end{itemize}
Defender; proteger.
\section{Tuitivo}
\begin{itemize}
\item {fónica:tu-i}
\end{itemize}
\begin{itemize}
\item {Grp. gram.:adj.}
\end{itemize}
\begin{itemize}
\item {Proveniência:(Do lat. \textunderscore tuitus\textunderscore )}
\end{itemize}
Que defende; próprio para defesa.
\section{Tuiúca}
\begin{itemize}
\item {Grp. gram.:f.}
\end{itemize}
\begin{itemize}
\item {Utilização:T. do Amazonas}
\end{itemize}
O mesmo que \textunderscore tujuco\textunderscore .
\section{Tuiuiú}
\begin{itemize}
\item {fónica:tu-i}
\end{itemize}
\begin{itemize}
\item {Grp. gram.:m.}
\end{itemize}
Grande ave do Brasil.
(Cp. \textunderscore tujuju\textunderscore )
\section{Tuixinau}
\begin{itemize}
\item {Grp. gram.:m.}
\end{itemize}
Espécie de reitor de certas escolas da China. Cf. \textunderscore Peregrinação\textunderscore , CXXVI.
\section{Tujuca}
\begin{itemize}
\item {Grp. gram.:f.}
\end{itemize}
O mesmo que \textunderscore tujuco\textunderscore .
\section{Tujucal}
\begin{itemize}
\item {Grp. gram.:m.}
\end{itemize}
O mesmo que \textunderscore tijucal\textunderscore .
\section{Tujuco}
\begin{itemize}
\item {Grp. gram.:m.}
\end{itemize}
O mesmo que \textunderscore tijuco\textunderscore .
\section{Tujuju}
\begin{itemize}
\item {Grp. gram.:m.}
\end{itemize}
\begin{itemize}
\item {Utilização:Bras}
\end{itemize}
Ave ribeirinha, que se alimenta de peixes.
(Do tupi)
\section{Tujupar}
\begin{itemize}
\item {Grp. gram.:m.}
\end{itemize}
\begin{itemize}
\item {Utilização:Bras}
\end{itemize}
O mesmo que \textunderscore tijupar\textunderscore :«\textunderscore ...não he necessario offerecer cidades... Basta acenar o diabo (no Maranhão) cõ um tujupar de pindoba, &amp; dous Tapuyas &amp; logo está adorado com ambos os joelhos.\textunderscore »Vieira, XII, 324.
\section{Tujupi}
\begin{itemize}
\item {Grp. gram.:m.}
\end{itemize}
\begin{itemize}
\item {Utilização:Bras}
\end{itemize}
Ave ribeirinha do Amazonas.
\section{Tula}
\begin{itemize}
\item {Grp. gram.:f.}
\end{itemize}
Gênero das plantas rubiáceas.
\section{Tulangue}
\begin{itemize}
\item {Grp. gram.:m.}
\end{itemize}
Insecto africano, preto, de muitas pernas.
\section{Tulha}
\begin{itemize}
\item {Grp. gram.:f.}
\end{itemize}
\begin{itemize}
\item {Utilização:Ext.}
\end{itemize}
\begin{itemize}
\item {Utilização:Prov.}
\end{itemize}
\begin{itemize}
\item {Utilização:alent.}
\end{itemize}
\begin{itemize}
\item {Utilização:Pop.}
\end{itemize}
\begin{itemize}
\item {Utilização:Prov.}
\end{itemize}
\begin{itemize}
\item {Proveniência:(Do lat. \textunderscore tudicula\textunderscore )}
\end{itemize}
Lugar, onde se ajunta e deposita a azeitona, antes de sêr levada ao moinho.
Casa ou compartimento, onde se depositam ou guardam cereaes em grão.
Porção de azeitona, contida no respectivo depósito.
Montão de cereaes ou frutos sêcos.
O mesmo que \textunderscore cama\textunderscore .
Cheiro e saibo desagradável, que a azeitona adquire, quando entulhada por muito tempo.
\section{Tulheira}
\begin{itemize}
\item {Grp. gram.:f.}
\end{itemize}
\begin{itemize}
\item {Utilização:Des.}
\end{itemize}
Monja, encarregada da tulha, em certos conventos, como no de Semide.
\section{Tulípa}
\begin{itemize}
\item {Grp. gram.:f.}
\end{itemize}
Gênero de plantas liliáceas.
A flôr de uma espécie daquelle gênero.
Nome de várias conchas.--Alguns diccion., e os que só conhecem o t. por o terem lido, dizem \textunderscore túlipa\textunderscore . Ao povo da Beira ouvi sempre \textunderscore tulípa\textunderscore  e assim pronunciavam os antigos. Cf. \textunderscore tolipa\textunderscore  nos \textunderscore Vestígios da Ling. Arab.\textunderscore  de fr. João de Sousa.
(Do turco \textunderscore tolipend\textunderscore ?)
\section{Tulipáceas}
\begin{itemize}
\item {Grp. gram.:f. pl.}
\end{itemize}
\begin{itemize}
\item {Proveniência:(De \textunderscore tulipáceo\textunderscore )}
\end{itemize}
Tribo de plantas liliáceas, cujo typo é a tulípa.
\section{Tulipáceo}
\begin{itemize}
\item {Grp. gram.:adj.}
\end{itemize}
Relativo ou semelhante á tulipa.
\section{Tulipeiro}
\begin{itemize}
\item {Grp. gram.:m.}
\end{itemize}
Árvore magnoliácea, de flôres semelhantes á da tulípa e chamada em Coímbra \textunderscore árvore do ponto\textunderscore , (\textunderscore liriodendrum tulipiferum\textunderscore ).
\section{Tulipomania}
\begin{itemize}
\item {Grp. gram.:f.}
\end{itemize}
Nome, que se deu á mania dos colleccionadores de variedades de tulípas, no séc. XVII.
\section{Tumulto}
\begin{itemize}
\item {Grp. gram.:m.}
\end{itemize}
\begin{itemize}
\item {Utilização:Fig.}
\end{itemize}
\begin{itemize}
\item {Proveniência:(Lat. \textunderscore tumultus\textunderscore )}
\end{itemize}
Movimento desordenado.
Motim.
Agitação; discórdia.
Agitação moral; perturbação.
\section{Tumultuador}
\begin{itemize}
\item {Grp. gram.:m.  e  adj.}
\end{itemize}
O que tumultua; agitador. Cf. Rui Barb., \textunderscore Réplica\textunderscore , 158.
\section{Tumultuante}
\begin{itemize}
\item {Grp. gram.:adj.}
\end{itemize}
Que tumultua.
\section{Tumultuar}
\begin{itemize}
\item {Grp. gram.:v. t.}
\end{itemize}
\begin{itemize}
\item {Grp. gram.:V. i.}
\end{itemize}
\begin{itemize}
\item {Proveniência:(Lat. \textunderscore tumultuari\textunderscore )}
\end{itemize}
Excitar á desordem.
Amotinar; agitar.
Mover-se desordenadamente.
Amotinar-se.
Revolver-se.
Atropelar-se.
Fazer grande barulho ou estrondo.
\section{Tumultuariamente}
\begin{itemize}
\item {Grp. gram.:adv.}
\end{itemize}
De modo tumultuário.
Desordenadamente; em confusão.
\section{Tumultuário}
\begin{itemize}
\item {Grp. gram.:adj.}
\end{itemize}
\begin{itemize}
\item {Proveniência:(Lat. \textunderscore tumultuarius\textunderscore )}
\end{itemize}
Feito desordenadamente; desordenado.
Ruidoso; amotinado; confuso.
\section{Tumultuosamente}
\begin{itemize}
\item {Grp. gram.:adv.}
\end{itemize}
De modo tumultuoso; tumultuariamente.
\section{Tumultuoso}
\begin{itemize}
\item {Grp. gram.:adj.}
\end{itemize}
\begin{itemize}
\item {Proveniência:(Lat. \textunderscore tumultuosus\textunderscore )}
\end{itemize}
Tumultuário; em que há tumulto.
\section{Tumungão}
\begin{itemize}
\item {Grp. gram.:m.}
\end{itemize}
O mesmo que \textunderscore datô\textunderscore .
\section{Tuna}
\begin{itemize}
\item {Grp. gram.:f.}
\end{itemize}
Ociosidade.
Vadiagem.
Grupo de estudantes, que vagueiam por várias terras, organizando concertos musicaes.
(Cast. \textunderscore tuna\textunderscore )
\section{Tuna}
\begin{itemize}
\item {Grp. gram.:f.}
\end{itemize}
O mesmo que \textunderscore tunal\textunderscore .
\section{Tunador}
\begin{itemize}
\item {Grp. gram.:m.  e  adj.}
\end{itemize}
O mesmo que \textunderscore tunante\textunderscore .
\section{Tunal}
\begin{itemize}
\item {Grp. gram.:m.}
\end{itemize}
O mesmo que \textunderscore nopal\textunderscore .
\section{Tunantaria}
\begin{itemize}
\item {Grp. gram.:f.}
\end{itemize}
Qualidade ou acto de tunante; os tunantes. Cf. Camillo, \textunderscore Ôlho de Vidro\textunderscore , 61.
\section{Tunante}
\begin{itemize}
\item {Grp. gram.:m.  e  adj.}
\end{itemize}
\begin{itemize}
\item {Utilização:Taur.}
\end{itemize}
\begin{itemize}
\item {Proveniência:(De \textunderscore tuna\textunderscore ^1)}
\end{itemize}
O que anda á tuna; vadio.
Trampolineiro.
Estudante, que faz parte de uma tuna.
Diz-se do toiro, que já conhece a arena, por têr sido picado, ou que revela má intenção.
\section{Tunar}
\begin{itemize}
\item {Grp. gram.:v. i.}
\end{itemize}
Andar á tuna; vadiar. Cf. Rui Barb., \textunderscore Réplica\textunderscore , 158.
\section{Tunco}
\begin{itemize}
\item {Grp. gram.:m.}
\end{itemize}
\begin{itemize}
\item {Utilização:Bras}
\end{itemize}
O mesmo que \textunderscore muxoxo\textunderscore .
\section{Tunda}
\begin{itemize}
\item {Grp. gram.:f.}
\end{itemize}
\begin{itemize}
\item {Utilização:Fig.}
\end{itemize}
\begin{itemize}
\item {Proveniência:(Do lat. \textunderscore tundere\textunderscore )}
\end{itemize}
Pancadaria, sova.
Crítica severa.
\section{Tundá}
\begin{itemize}
\item {Grp. gram.:m.}
\end{itemize}
\begin{itemize}
\item {Utilização:Bras}
\end{itemize}
Anquinhas, com que as mulheres alargam posteriormente o vestido, e a que os Franceses e francesistas chamam \textunderscore tournure\textunderscore .
\section{Tundar}
\begin{itemize}
\item {Grp. gram.:v. t.}
\end{itemize}
Dar tunda em; sovar.
\section{Tundo}
\begin{itemize}
\item {Grp. gram.:m.}
\end{itemize}
Chefe de sacerdotes gentios, na África.
\section{Tundo}
\begin{itemize}
\item {Grp. gram.:m.}
\end{itemize}
Espécie de doutor, em escolas japonesas. Cf. \textunderscore Peregrinação\textunderscore , XXIV.
\section{Tunesino}
\begin{itemize}
\item {Grp. gram.:adj.}
\end{itemize}
\begin{itemize}
\item {Grp. gram.:M.}
\end{itemize}
Relativo a Tunes.
Habitante de Tunes.
\section{Tunetano}
\begin{itemize}
\item {Grp. gram.:m.}
\end{itemize}
\begin{itemize}
\item {Utilização:Zool.}
\end{itemize}
\begin{itemize}
\item {Utilização:Ant.}
\end{itemize}
Dizia-se de uma ave de rapina, originária de Tunes.
\section{Tunga}
\begin{itemize}
\item {Grp. gram.:f.}
\end{itemize}
Espécie de pulga do Brasil, que ataca a pelle dos pés; o mesmo que \textunderscore nígua\textunderscore .
\section{Tunga}
\begin{itemize}
\item {Grp. gram.:f.}
\end{itemize}
\begin{itemize}
\item {Proveniência:(Do quimb. \textunderscore túngu\textunderscore )}
\end{itemize}
Árvore de Angola.
\section{Tungada}
\begin{itemize}
\item {Grp. gram.:f.}
\end{itemize}
\begin{itemize}
\item {Utilização:Bras}
\end{itemize}
Pancada; choque.
\section{Tungstatado}
\begin{itemize}
\item {Grp. gram.:adj.}
\end{itemize}
\begin{itemize}
\item {Utilização:Miner.}
\end{itemize}
\begin{itemize}
\item {Proveniência:(De \textunderscore tungstato\textunderscore )}
\end{itemize}
Que contém tungstênio.
\section{Tungstato}
\begin{itemize}
\item {Grp. gram.:f.}
\end{itemize}
\begin{itemize}
\item {Utilização:Chím.}
\end{itemize}
Sal, produzido pela combinação do ácido túngstico com uma base.
\section{Tungstênico}
\begin{itemize}
\item {Grp. gram.:adj.}
\end{itemize}
Relativo ao tungstênio.
\section{Tungstênio}
\begin{itemize}
\item {Grp. gram.:m.}
\end{itemize}
\begin{itemize}
\item {Utilização:Miner.}
\end{itemize}
\begin{itemize}
\item {Proveniência:(Do al. \textunderscore tungstein\textunderscore )}
\end{itemize}
Metal pardacento, cujo aspecto díffere pouco do ferro.
\section{Tungsteno}
\begin{itemize}
\item {Grp. gram.:m.}
\end{itemize}
O mesmo que \textunderscore tungstênio\textunderscore .
\section{Túngstico}
\begin{itemize}
\item {Grp. gram.:adj.}
\end{itemize}
Relativo ao tungstênio.
Que encerra tungstênio.
\section{Tungstídeos}
\begin{itemize}
\item {Grp. gram.:m. pl.}
\end{itemize}
\begin{itemize}
\item {Utilização:Miner.}
\end{itemize}
\begin{itemize}
\item {Proveniência:(De \textunderscore tungstênio\textunderscore  + gr. \textunderscore eidos\textunderscore )}
\end{itemize}
Família de mineraes, que comprehende o tungstênio e as suas combinações.
\section{Tungula}
\begin{itemize}
\item {Grp. gram.:f.}
\end{itemize}
Ave trepadora africana.
\section{Tungúsio}
\begin{itemize}
\item {Grp. gram.:m.}
\end{itemize}
Língua uralô-altaica da Ásia setentrional.
\section{Tupixás}
\begin{itemize}
\item {Grp. gram.:m. pl.}
\end{itemize}
Indígenas do norte do Brasil.--Terá havido êrro ou equívoco nos ethnógraphos brasileiros, que dariam dois nomes a um só povo, \textunderscore tupixás\textunderscore  e \textunderscore tupinás\textunderscore ?
\section{Tupixava}
\begin{itemize}
\item {Grp. gram.:f.}
\end{itemize}
O mesmo que \textunderscore vassourinha-de-varrer\textunderscore .
\section{Tupurapo}
\begin{itemize}
\item {Grp. gram.:m.}
\end{itemize}
O mesmo que \textunderscore jacuruaru\textunderscore .
\section{Tuputá}
\begin{itemize}
\item {Grp. gram.:m.}
\end{itemize}
Espécie de ave indiana.
\section{Tuputu}
\begin{itemize}
\item {Grp. gram.:m.}
\end{itemize}
O mesmo que \textunderscore tuputá\textunderscore .
\section{Tuquianês}
\begin{itemize}
\item {Grp. gram.:m.}
\end{itemize}
\begin{itemize}
\item {Proveniência:(De \textunderscore Tuquian\textunderscore , n. p.)}
\end{itemize}
Um dos principaes dialectos chineses.
\section{Tuquira}
\begin{itemize}
\item {Grp. gram.:m.}
\end{itemize}
O mesmo que \textunderscore tucaíra\textunderscore .--Caminhoá, \textunderscore Bot. Ger. e Med.\textunderscore , escreveu \textunderscore tuquirá\textunderscore .
\section{Turaco}
\begin{itemize}
\item {Grp. gram.:m.}
\end{itemize}
Espécie de cuco africano.
\section{Turanianos}
\begin{itemize}
\item {Grp. gram.:m. pl.}
\end{itemize}
Descendentes do Patriarcha hebreu Tur.
\section{Turânico}
\begin{itemize}
\item {Grp. gram.:adj.}
\end{itemize}
Relativo aos Turanianos. Cf. Latino, \textunderscore Elog.\textunderscore , 66.
\section{Turari}
\begin{itemize}
\item {Grp. gram.:m.}
\end{itemize}
Planta sapindácea e trepadeira do Brasil.
\section{Turba}
\begin{itemize}
\item {Grp. gram.:f.}
\end{itemize}
\begin{itemize}
\item {Proveniência:(Lat. \textunderscore turba\textunderscore )}
\end{itemize}
Multidão desordenada.
Muita gente reunida.
O povo, o vulgo, as multidões.
Côro de vozes.
\section{Turbação}
\begin{itemize}
\item {Grp. gram.:f.}
\end{itemize}
\begin{itemize}
\item {Proveniência:(Do lat. \textunderscore turbatio\textunderscore )}
\end{itemize}
Acto ou effeito de turbar.
\section{Turbadamente}
\begin{itemize}
\item {Grp. gram.:adv.}
\end{itemize}
De modo turbado.
Com perturbação; embaraçadamente.
\section{Turbador}
\begin{itemize}
\item {Grp. gram.:m.  e  adj.}
\end{itemize}
\begin{itemize}
\item {Proveniência:(Do lat. \textunderscore turbator\textunderscore )}
\end{itemize}
O que turba; perturbador; agitador.
\section{Turbamento}
\begin{itemize}
\item {Grp. gram.:m.}
\end{itemize}
\begin{itemize}
\item {Proveniência:(Lat. \textunderscore turbamentum\textunderscore )}
\end{itemize}
O mesmo que \textunderscore turbação\textunderscore .
\section{Turbamulta}
\begin{itemize}
\item {Grp. gram.:f.}
\end{itemize}
\begin{itemize}
\item {Proveniência:(Do lat. \textunderscore turba\textunderscore  + \textunderscore multus\textunderscore )}
\end{itemize}
Grande multidão em desordem; agrupamento de gente; tropel.
\section{Turbante}
\begin{itemize}
\item {Grp. gram.:m.}
\end{itemize}
\begin{itemize}
\item {Proveniência:(Do ár. \textunderscore dulband\textunderscore )}
\end{itemize}
Cobertura ou ornato para a cabeça, usado por povos do Oriente.
Toucado feminino, semelhante àquelle ornato.
Enfeite ou cobertura semelhante a um turbante.
\section{Turbão}
\begin{itemize}
\item {Grp. gram.:m.}
\end{itemize}
\begin{itemize}
\item {Utilização:Ant.}
\end{itemize}
O mesmo que \textunderscore turbante\textunderscore . Cf. Pant. de Aveiro, \textunderscore Itiner.\textunderscore , 50 v.^o e 292, (2.^a ed.).
\section{Turbar}
\begin{itemize}
\item {Grp. gram.:v. t.}
\end{itemize}
\begin{itemize}
\item {Proveniência:(Lat. \textunderscore turbare\textunderscore )}
\end{itemize}
O mesmo que \textunderscore turvar\textunderscore .
Perturbar.
Revolver, agitar.
Transtornar; inquietar.
\section{Turbativo}
\begin{itemize}
\item {Grp. gram.:adj.}
\end{itemize}
\begin{itemize}
\item {Proveniência:(De \textunderscore turbar\textunderscore )}
\end{itemize}
Que causa perturbação.
\section{Turbelariados}
\begin{itemize}
\item {Grp. gram.:m. pl.}
\end{itemize}
\begin{itemize}
\item {Utilização:Zool.}
\end{itemize}
Ordem de vermes chatos, que vivem na terra húmida.
\section{Turbelinho}
\begin{itemize}
\item {Grp. gram.:m.}
\end{itemize}
O mesmo que \textunderscore torvelinho\textunderscore .
Movimento desordenado:«\textunderscore ...e neste empeço, e turbellino da cidade.\textunderscore »Filinto, \textunderscore D. Man.\textunderscore , II, 5.
\section{Túrbido}
\begin{itemize}
\item {Grp. gram.:adj.}
\end{itemize}
\begin{itemize}
\item {Proveniência:(Lat. \textunderscore turbidus\textunderscore )}
\end{itemize}
Que perturba.
Perturbado.
Escuro; turvo.
\section{Turbilhão}
\begin{itemize}
\item {Grp. gram.:m.}
\end{itemize}
\begin{itemize}
\item {Utilização:Fig.}
\end{itemize}
Redemoínho de vento.
Movimento forte e giratório de águas.
Revolução de um planeta.
Aquillo que é comparável a um redemoínho de vento.
Aquillo que impelle ou excita com violência.
Turbamulta.
(Por \textunderscore turbinão\textunderscore , do lat. \textunderscore turbo\textunderscore , \textunderscore turbinis\textunderscore )
\section{Turbilho}
\begin{itemize}
\item {Grp. gram.:m.}
\end{itemize}
\begin{itemize}
\item {Proveniência:(Do lat. \textunderscore turbo\textunderscore )}
\end{itemize}
Mollusco gasterópode.
\section{Turbilhonar}
\begin{itemize}
\item {Grp. gram.:v. i.}
\end{itemize}
\begin{itemize}
\item {Utilização:Neol.}
\end{itemize}
Formar turbilhão.
\section{Turbina}
\begin{itemize}
\item {Grp. gram.:f.}
\end{itemize}
\begin{itemize}
\item {Proveniência:(Do lat. \textunderscore turbo\textunderscore , \textunderscore turbinis\textunderscore )}
\end{itemize}
Roda hydráulica, cujo eixo vertical gira debaixo de água.
\section{Turbináceos}
\begin{itemize}
\item {Grp. gram.:m. pl.}
\end{itemize}
\begin{itemize}
\item {Proveniência:(De \textunderscore turbina\textunderscore )}
\end{itemize}
Família de molluscos phytóphagos.
\section{Turbinado}
\begin{itemize}
\item {Grp. gram.:adj.}
\end{itemize}
\begin{itemize}
\item {Grp. gram.:M.}
\end{itemize}
\begin{itemize}
\item {Proveniência:(Do lat. \textunderscore turbinatus\textunderscore )}
\end{itemize}
Semelhante a um cóne invertido.
Diz-se da concha univalve, cuja espiral fórma um cóne pouco alongado e bastante largo na base.
Diz-se de dois pequenos ossos na raíz do nariz.
Concha univalve, de espiral pouco alongada.
\section{Turbinar}
\begin{itemize}
\item {Grp. gram.:v. i.}
\end{itemize}
\begin{itemize}
\item {Utilização:Ant.}
\end{itemize}
\begin{itemize}
\item {Proveniência:(Lat. \textunderscore turbinare\textunderscore )}
\end{itemize}
O mesmo que \textunderscore redemoinhar\textunderscore , (falando-se de águas).
\section{Tifa}
\begin{itemize}
\item {Grp. gram.:f.}
\end{itemize}
\begin{itemize}
\item {Proveniência:(Lat. \textunderscore typhe\textunderscore )}
\end{itemize}
Nome científico da espadana.
\section{Tifáceas}
\begin{itemize}
\item {Grp. gram.:f. pl.}
\end{itemize}
Família de plantas, que tem por tipo a espadana.
\section{Tifão}
\begin{itemize}
\item {Grp. gram.:m.}
\end{itemize}
\begin{itemize}
\item {Utilização:Geol.}
\end{itemize}
\begin{itemize}
\item {Proveniência:(Do gr. \textunderscore Tuphon\textunderscore , n. p. de uma divindade egípcia)}
\end{itemize}
Massa de terreno, não estratificada, na crosta da terra.
\section{Tife}
\begin{itemize}
\item {Grp. gram.:f.}
\end{itemize}
O mesmo ou melhór que \textunderscore tifa\textunderscore .
\section{Tífico}
\begin{itemize}
\item {Grp. gram.:adj.}
\end{itemize}
Relativo ao tifo ou que tem a natureza dêle.
\section{Tifíneas}
\begin{itemize}
\item {Grp. gram.:f. pl.}
\end{itemize}
(V.tifáceas)
\section{Tifismo}
\begin{itemize}
\item {Grp. gram.:m.}
\end{itemize}
\begin{itemize}
\item {Proveniência:(De \textunderscore tifo\textunderscore )}
\end{itemize}
Caracter tífico de certas febres.
\section{Tiflina}
\begin{itemize}
\item {Grp. gram.:f.}
\end{itemize}
Gênero de reptís ofídios.
\section{Tiflite}
\begin{itemize}
\item {Grp. gram.:f.}
\end{itemize}
\begin{itemize}
\item {Proveniência:(Do gr. \textunderscore tuphlos\textunderscore )}
\end{itemize}
Inflamação do ceco.
\section{Tiflografia}
\begin{itemize}
\item {Grp. gram.:f.}
\end{itemize}
Arte de escrever em relêvo, para uso dos cegos.
(Cp. \textunderscore tiflógrafo\textunderscore )
\section{Tiflógrafo}
\begin{itemize}
\item {Grp. gram.:m.}
\end{itemize}
\begin{itemize}
\item {Proveniência:(Do gr. \textunderscore tuphlos\textunderscore , cego, e \textunderscore graphein\textunderscore , escrever)}
\end{itemize}
Instrumento, com que os cegos podem escrever.
\section{Tiflologia}
\begin{itemize}
\item {Grp. gram.:f.}
\end{itemize}
\begin{itemize}
\item {Proveniência:(Do gr. \textunderscore tuphlos\textunderscore  + \textunderscore logos\textunderscore )}
\end{itemize}
Tratado sôbre a instrução dos cegos.
\section{Tiflológico}
\begin{itemize}
\item {Grp. gram.:adj.}
\end{itemize}
Relativo á tiflologia.
\section{Tiflólogo}
\begin{itemize}
\item {Grp. gram.:m.}
\end{itemize}
\begin{itemize}
\item {Proveniência:(Do gr. \textunderscore tuphlos\textunderscore  + \textunderscore logos\textunderscore )}
\end{itemize}
Aquele que se ocupa da instrução dos cegos.
\section{Tifo}
\begin{itemize}
\item {Grp. gram.:m.}
\end{itemize}
\begin{itemize}
\item {Proveniência:(Do gr. \textunderscore tuphos\textunderscore )}
\end{itemize}
Febre contínua, geralmente contagiosa, e produzida por influências miasmáticas.
Variedade de epizootia.
\section{Tifoemia}
\begin{itemize}
\item {Grp. gram.:f.}
\end{itemize}
\begin{itemize}
\item {Proveniência:(Do gr. \textunderscore tuphos\textunderscore  + \textunderscore haima\textunderscore )}
\end{itemize}
Alteração de sangue por influências miasmáticas.
\section{Tifóide}
\begin{itemize}
\item {Grp. gram.:adj.}
\end{itemize}
\begin{itemize}
\item {Proveniência:(Do gr. \textunderscore tuphos\textunderscore  + \textunderscore eidos\textunderscore )}
\end{itemize}
Que tem caracteres de tifo; semelhante ao tifo.
\section{Tifoídeo}
\begin{itemize}
\item {Grp. gram.:adj.}
\end{itemize}
O mesmo que \textunderscore tifóide\textunderscore .
\section{Tifomania}
\begin{itemize}
\item {Grp. gram.:f.}
\end{itemize}
\begin{itemize}
\item {Proveniência:(De \textunderscore tifo\textunderscore  + \textunderscore mania\textunderscore )}
\end{itemize}
Delírio, que se manifesta na doença do tifo.
\section{Tifónico}
\begin{itemize}
\item {Grp. gram.:adj.}
\end{itemize}
\begin{itemize}
\item {Utilização:Geol.}
\end{itemize}
Relativo a tifão.
Diz-se especialmente dos terrenos ou vales, limitados por séries de falhas e com o fundo levantado através dos terrenos mais recentes, com os quaes se acha actualmente em contacto em todo o seu perímetro. Cf. Gonç. Guimarães, \textunderscore Geologia\textunderscore , 144.
\section{Tifoso}
\begin{itemize}
\item {Grp. gram.:adj.}
\end{itemize}
\begin{itemize}
\item {Grp. gram.:M.}
\end{itemize}
\begin{itemize}
\item {Proveniência:(De \textunderscore tifo\textunderscore )}
\end{itemize}
O mesmo que \textunderscore tifóide\textunderscore .
Diz-se dos fenómenos atáxicos e adinâmicos, que complicam a marcha de uma doença.
Indivíduo, atacado de tifo.
\section{Tímpano}
\begin{itemize}
\item {Grp. gram.:m.}
\end{itemize}
\begin{itemize}
\item {Utilização:Constr.}
\end{itemize}
\begin{itemize}
\item {Grp. gram.:Pl.}
\end{itemize}
\begin{itemize}
\item {Proveniência:(Lat. \textunderscore tympanum\textunderscore )}
\end{itemize}
Cavidade irregular, na base do rochedo auricular.
Peça de escultura, limitada por arcos ou linhas.
Espécie de tambor oco, com repartimentos em espiral, por intermédio dos quaes se eleva a água de um depósito ou de uma corrente.
Timbale.
Caixilho de ferro, recoberto de estôfo de algodão, ligado ao quadro do prelo por dois gonzos, e no qual se colocam as puncturas, se faz o alceamento, se regula a margem e se coloca sucessivamente cada uma das fôlhas a imprimir.
Planto rectangular, formado por uma linha que une as extremidades inferiores de um arco ou abóbada, por outra, vertical áquela, e por uma terceira, que liga as duas mencionadas.
Ouvidos.
\section{Tipa}
\begin{itemize}
\item {Grp. gram.:f.}
\end{itemize}
\begin{itemize}
\item {Utilização:Chul.}
\end{itemize}
Qualquer mulhér.
Mulhér de costumes fáceis.
(Cp. \textunderscore tipo\textunderscore )
\section{Tipicamente}
\begin{itemize}
\item {Grp. gram.:adv.}
\end{itemize}
De modo típico.
Á maneira de tipo.
Com carácter de tipo.
Simbolicamente.
\section{Típico}
\begin{itemize}
\item {Grp. gram.:adj.}
\end{itemize}
\begin{itemize}
\item {Proveniência:(Lat. \textunderscore typicus\textunderscore )}
\end{itemize}
Que serve de tipo; que caracteriza; simbólico.
\section{Tipo}
\begin{itemize}
\item {Grp. gram.:m.}
\end{itemize}
\begin{itemize}
\item {Utilização:Fam.}
\end{itemize}
\begin{itemize}
\item {Utilização:Burl.}
\end{itemize}
\begin{itemize}
\item {Proveniência:(Lat. \textunderscore typus\textunderscore )}
\end{itemize}
Cunho, ou cada um dos caracteres tipográficos.
Aquilo que produz fé, como modêlo.
Coisa, que reúne em si os caracteres que distinguem uma classe.
Reunião dos caracteres que distinguem uma raça ou classe.
Símbolo; exemplar; modêlo: \textunderscore aquele é tipo da honradez\textunderscore .
Ordem, por que e manifestam ou succedem os sintomas de uma doença.
Pessôa excêntrica.
Qualquer indivíduo, pessôa pouco respeitável: \textunderscore não conheces aquele tipo?\textunderscore 
\section{Turgente}
\begin{itemize}
\item {Grp. gram.:adj.}
\end{itemize}
\begin{itemize}
\item {Proveniência:(Lat. \textunderscore turgens\textunderscore )}
\end{itemize}
O mesmo que \textunderscore túrgido\textunderscore .
\section{Turgescência}
\begin{itemize}
\item {Grp. gram.:f.}
\end{itemize}
Qualidade do que é turgescente.
\section{Turgescente}
\begin{itemize}
\item {Grp. gram.:adj.}
\end{itemize}
\begin{itemize}
\item {Proveniência:(Lat. \textunderscore turgescens\textunderscore )}
\end{itemize}
Que turgesce.
\section{Turgescer}
\begin{itemize}
\item {Grp. gram.:v. t.}
\end{itemize}
\begin{itemize}
\item {Grp. gram.:V. i.  e  p.}
\end{itemize}
\begin{itemize}
\item {Proveniência:(Lat. \textunderscore turgescere\textunderscore )}
\end{itemize}
Tornar túrgido.
Tornar-se túrgido.
\section{Turgidez}
\begin{itemize}
\item {Grp. gram.:f.}
\end{itemize}
Estado do que é túrgido.
Inchação; entumecimento.
\section{Túrgido}
\begin{itemize}
\item {Grp. gram.:adj.}
\end{itemize}
\begin{itemize}
\item {Proveniência:(Lat. \textunderscore turgidus\textunderscore )}
\end{itemize}
Dilatado, por conter grande porção de humores.
Inchado; túmido.
\section{Turgimão}
\begin{itemize}
\item {Grp. gram.:m.}
\end{itemize}
\begin{itemize}
\item {Proveniência:(Do ár. \textunderscore turgeman\textunderscore )}
\end{itemize}
Intérprete levantino, geralmente ao serviço das legações e consulados europeus.
O mesmo que \textunderscore alcoviteiro\textunderscore .
\section{Turião}
\begin{itemize}
\item {Grp. gram.:m.}
\end{itemize}
\begin{itemize}
\item {Proveniência:(Do lat. \textunderscore turio\textunderscore )}
\end{itemize}
Rebento de ervas vivazes, que saem da parte subterrânea do caule.
\section{Turimua}
\begin{itemize}
\item {Grp. gram.:f.}
\end{itemize}
Planta rosácea, medicinal.
\section{Turina}
\begin{itemize}
\item {Grp. gram.:adj. f.}
\end{itemize}
Diz-se de uma casta de vacas leiteiras, variedade portuguesa, de raça hollandesa. Cf. Filito, III, 223; v. 16.
(Talvez de \textunderscore Tubingen\textunderscore , n. p.)
\section{Turino}
\begin{itemize}
\item {Grp. gram.:adj.}
\end{itemize}
\begin{itemize}
\item {Proveniência:(Lat. \textunderscore turinus\textunderscore )}
\end{itemize}
Relativo a incenso.
\section{Turismo}
\begin{itemize}
\item {Grp. gram.:m.}
\end{itemize}
\begin{itemize}
\item {Utilização:Neol.}
\end{itemize}
\begin{itemize}
\item {Proveniência:(Fr. \textunderscore tourisme\textunderscore )}
\end{itemize}
Gôsto por viagens.
Viagens para recreio.
\section{Turista}
\begin{itemize}
\item {Grp. gram.:m.}
\end{itemize}
\begin{itemize}
\item {Utilização:Neol.}
\end{itemize}
\begin{itemize}
\item {Proveniência:(Fr. \textunderscore touriste\textunderscore )}
\end{itemize}
Pessôa, que viaja para se recrear.
\section{Turiúa}
\begin{itemize}
\item {Grp. gram.:f.}
\end{itemize}
\begin{itemize}
\item {Utilização:Bras. do N}
\end{itemize}
Dança, o mesmo que \textunderscore sairé\textunderscore .
\section{Turma}
\begin{itemize}
\item {Grp. gram.:f.}
\end{itemize}
\begin{itemize}
\item {Proveniência:(Lat. \textunderscore turma\textunderscore )}
\end{itemize}
Reunião de trinta cavalleiros com três decuriões, na Roma antiga.
Esquadrão.
Porção de bandos.
Cada um dos grupos de pessôas que se revezam em certos actos.
Cada um dos grupos, em que se divide uma numerosa classe de estudantes: \textunderscore o Lopes é o n.^o 25 da 2.^a turma da 5.^a classe\textunderscore .
\section{Turma}
\begin{itemize}
\item {Grp. gram.:f.}
\end{itemize}
Antigo pêso asiático. Cf. \textunderscore Peregrinação\textunderscore , CLXXX.
\section{Turmalina}
\begin{itemize}
\item {Grp. gram.:f.}
\end{itemize}
\begin{itemize}
\item {Utilização:Miner.}
\end{itemize}
\begin{itemize}
\item {Proveniência:(T. \textunderscore cingalês\textunderscore )}
\end{itemize}
Pedra dura, formada de silicato com base de cal ou magnésia, e contendo ácido bórico ou fluor, pedra que se electriza ao aquecer-se.
\section{Turmalinoso}
\begin{itemize}
\item {Grp. gram.:adj.}
\end{itemize}
Que tem a natureza da turmalina.
\section{Turna}
\begin{itemize}
\item {Grp. gram.:f.}
\end{itemize}
\begin{itemize}
\item {Utilização:Prov.}
\end{itemize}
\begin{itemize}
\item {Utilização:trasm.}
\end{itemize}
Marrada de animal.
\section{Túrnepo}
\begin{itemize}
\item {Grp. gram.:m.}
\end{itemize}
\begin{itemize}
\item {Proveniência:(Ingl. \textunderscore turnep\textunderscore )}
\end{itemize}
Variedade de nabo.
\section{Turnera}
\begin{itemize}
\item {Grp. gram.:f.}
\end{itemize}
Gênero de plantas, que serviu de typo ás turneráceas.
\section{Turneráceas}
\begin{itemize}
\item {Grp. gram.:f. pl.}
\end{itemize}
Família de plantas, estabelecida por Humboldt, e que pertence ás calicifloras de De-Candolle.
\section{Turnerito}
\begin{itemize}
\item {Grp. gram.:m.}
\end{itemize}
Substância mineral, que se suppõe sêr um silicato ou aluminato de cal.
\section{Turnês}
\begin{itemize}
\item {Grp. gram.:m.}
\end{itemize}
O mesmo que \textunderscore tornês\textunderscore .
\section{Turno}
\begin{itemize}
\item {Grp. gram.:m.}
\end{itemize}
\begin{itemize}
\item {Proveniência:(Do gr. \textunderscore tornos\textunderscore )}
\end{itemize}
Cada um dos grupos de pessôas, que se revezam em certos actos: \textunderscore para pegar nas borlas do caixão mortuário, formaram-se três turnos\textunderscore .
Turma.
Vez: \textunderscore depois de ferido, feriu por seu turno o aggressor\textunderscore .
\section{Turonense}
\begin{itemize}
\item {Grp. gram.:adj.}
\end{itemize}
\begin{itemize}
\item {Proveniência:(Lat. \textunderscore turonensis\textunderscore )}
\end{itemize}
Relativo aos antigos habitantes da cidade de Tours. Cf. Herculano, \textunderscore Hist. de Port.\textunderscore , II, 48.
\section{Turoniano}
\begin{itemize}
\item {Grp. gram.:adj.}
\end{itemize}
\begin{itemize}
\item {Utilização:Geol.}
\end{itemize}
Diz-se de uma das espécies do terreno cretáceo.
(Cp. \textunderscore turonense\textunderscore )
\section{Turpelina}
\begin{itemize}
\item {Grp. gram.:f.}
\end{itemize}
(V.turmalina)
\section{Turpilóquio}
\begin{itemize}
\item {Grp. gram.:m.}
\end{itemize}
\begin{itemize}
\item {Proveniência:(Lat. \textunderscore turpíloquium\textunderscore )}
\end{itemize}
Dito torpe.
Palavrada; expressão obscena.
\section{Turpínia}
\begin{itemize}
\item {Grp. gram.:f.}
\end{itemize}
\begin{itemize}
\item {Proveniência:(De \textunderscore Turpin\textunderscore , n. p.)}
\end{itemize}
Gênero de plantas leguminosas.
\section{Tus}
\begin{itemize}
\item {Grp. gram.:adj.}
\end{itemize}
O mesmo que \textunderscore banto\textunderscore .
\section{Tusco}
\begin{itemize}
\item {Grp. gram.:adj.}
\end{itemize}
\begin{itemize}
\item {Proveniência:(Lat. \textunderscore tuscus\textunderscore )}
\end{itemize}
O mesmo que \textunderscore toscano\textunderscore ^1. Cf. Castilho, \textunderscore Fastos\textunderscore , I, 99.
\section{Tusculano}
\begin{itemize}
\item {Grp. gram.:adj.}
\end{itemize}
\begin{itemize}
\item {Proveniência:(Lat. \textunderscore tusculanus\textunderscore )}
\end{itemize}
Relativo á antiga cidade de Túsculo. Cf. Herculano, \textunderscore Hist. de Post.\textunderscore , III, 124 e 125.
\section{Tusébio}
\begin{itemize}
\item {Grp. gram.:m.}
\end{itemize}
Espécie de mármore preto.
\section{Tussícula}
\begin{itemize}
\item {Grp. gram.:f.}
\end{itemize}
\begin{itemize}
\item {Proveniência:(Lat. \textunderscore tussicula\textunderscore )}
\end{itemize}
Tosse benigna, pequena tosse.
\section{Tussilagem}
\begin{itemize}
\item {Grp. gram.:f.}
\end{itemize}
\begin{itemize}
\item {Proveniência:(Do lat. \textunderscore tussis\textunderscore , tosse, por allusão ás propriedades medicinaes da planta)}
\end{itemize}
Gênero medicinal de plantas synanthéreas, (\textunderscore tussilago\textunderscore , Tournf.).
\section{Tussilagíneas}
\begin{itemize}
\item {Grp. gram.:f. pl.}
\end{itemize}
\begin{itemize}
\item {Proveniência:(Do lat. \textunderscore tussilago\textunderscore )}
\end{itemize}
Tríbo de synanthéreas, no systema de Cassini.
\section{Tussol}
\begin{itemize}
\item {Grp. gram.:m.}
\end{itemize}
\begin{itemize}
\item {Utilização:Pharm.}
\end{itemize}
Medicamento narcótico, que é um amygdalato da antipyrina, e que se applica especialmente contra a coqueluche.
\section{Tústio}
\begin{itemize}
\item {Grp. gram.:m.}
\end{itemize}
\begin{itemize}
\item {Utilização:Gír.}
\end{itemize}
O mesmo que \textunderscore tostão\textunderscore ^1.
\section{Tuta!}
\begin{itemize}
\item {Grp. gram.:interj.}
\end{itemize}
\begin{itemize}
\item {Utilização:T. da Bairrada}
\end{itemize}
\begin{itemize}
\item {Utilização:chul.}
\end{itemize}
O mesmo que \textunderscore chiça!\textunderscore .
\section{Tuta-e-meia}
\begin{itemize}
\item {Grp. gram.:f.}
\end{itemize}
\begin{itemize}
\item {Utilização:Fam.}
\end{itemize}
Pouca coisa; bagatela; quási nada.
Pouco dinheiro: \textunderscore comprou aquillo por tuta-e-meia\textunderscore .
(Cp. \textunderscore matuta-e-meia\textunderscore )
\section{Tutanaga}
\begin{itemize}
\item {Grp. gram.:f.}
\end{itemize}
Substância metállica, cobre da China.
\section{Tutano}
\begin{itemize}
\item {Grp. gram.:m.}
\end{itemize}
\begin{itemize}
\item {Utilização:Fig.}
\end{itemize}
Substância gordurosa do interior dos ossos.
A parte mais íntima, a essência, o âmago.
(Cp. cast. \textunderscore tuétano\textunderscore )
\section{Tutão}
\begin{itemize}
\item {Grp. gram.:m.}
\end{itemize}
Antigo e principal dignitário na côrte da China. Cf. \textunderscore Peregrinação\textunderscore , 111.
\section{Tutar}
\begin{itemize}
\item {Grp. gram.:v. i.}
\end{itemize}
\begin{itemize}
\item {Utilização:Prov.}
\end{itemize}
\begin{itemize}
\item {Utilização:beir.}
\end{itemize}
O mesmo que \textunderscore soprar\textunderscore , (na loc. \textunderscore tutar por um corno\textunderscore , lamentar-se, carpir-se).
(Talvez por \textunderscore totar\textunderscore , alter. arbitrária de \textunderscore tocar\textunderscore )
\section{Tuta-riambula}
\begin{itemize}
\item {Grp. gram.:f.}
\end{itemize}
Planta crassulácea de Angola.
\section{Tute, a}
\begin{itemize}
\item {Grp. gram.:loc. adv.}
\end{itemize}
\begin{itemize}
\item {Utilização:Ant.}
\end{itemize}
Abundantemente; a rodos.
\section{Tutear}
\begin{itemize}
\item {Grp. gram.:v. t.}
\end{itemize}
Tratar por tu, atuar.
(Cast. \textunderscore tutear\textunderscore )
\section{Tutela}
\begin{itemize}
\item {Grp. gram.:f.}
\end{itemize}
\begin{itemize}
\item {Proveniência:(Lat. \textunderscore tutela\textunderscore )}
\end{itemize}
Autoridade, dada por lei, para velar a pessôa e bens de um menos ou de um interdito.
Defesa, protecção.
Dependência, sujeição vexatória.
\section{Tutelado}
\begin{itemize}
\item {Grp. gram.:adj.}
\end{itemize}
\begin{itemize}
\item {Grp. gram.:M.}
\end{itemize}
\begin{itemize}
\item {Proveniência:(De \textunderscore tutelar\textunderscore )}
\end{itemize}
Protegido.
Sujeito a tutela.
Indivíduo tutelado.
\section{Tutelagem}
\begin{itemize}
\item {Grp. gram.:f.}
\end{itemize}
Acto ou cargo de tutelar^2.
\section{Tutelar}
\begin{itemize}
\item {Grp. gram.:adj.}
\end{itemize}
\begin{itemize}
\item {Proveniência:(Lat. \textunderscore tutelaris\textunderscore )}
\end{itemize}
Relativo a tutela; protector.
\section{Tutelar}
\begin{itemize}
\item {Grp. gram.:v. t.}
\end{itemize}
\begin{itemize}
\item {Proveniência:(De \textunderscore tutela\textunderscore )}
\end{itemize}
Cuidar de, como tutor; exercer tutela sôbre.
Proteger, defender.
\section{Tutia}
\begin{itemize}
\item {Grp. gram.:f.}
\end{itemize}
\begin{itemize}
\item {Proveniência:(Do ár. \textunderscore tutia\textunderscore )}
\end{itemize}
Oxydo de zinco impuro, que adhere, sob a fórma de camada dura e pardacenta, ás chaminés dor fornos em que se calcinam certos minérios.
\section{Tutia}
\begin{itemize}
\item {Grp. gram.:f.}
\end{itemize}
Peixe dos Açores.
\section{Tutinegra}
\begin{itemize}
\item {Grp. gram.:f.}
\end{itemize}
O mesmo que \textunderscore toutinegra\textunderscore .
\section{Tutinegro}
\begin{itemize}
\item {fónica:nê}
\end{itemize}
\begin{itemize}
\item {Grp. gram.:m.}
\end{itemize}
\begin{itemize}
\item {Utilização:Med.}
\end{itemize}
Pequena ave madeirense, (\textunderscore silvia atricapilla\textunderscore ).
\section{Tuto}
\begin{itemize}
\item {Grp. gram.:m.}
\end{itemize}
(V.ungui)
\section{Tutor}
\begin{itemize}
\item {Grp. gram.:m.}
\end{itemize}
\begin{itemize}
\item {Proveniência:(Lat. \textunderscore tutor\textunderscore )}
\end{itemize}
Indivíduo, encarregado legalmente de tutelar alguém.
Defensor; protector.
Estaca ou vara, com que se ampara um arbusto ou árvore flexível.
\section{Tutorar}
\begin{itemize}
\item {Grp. gram.:v. t.}
\end{itemize}
\begin{itemize}
\item {Utilização:P. us.}
\end{itemize}
\begin{itemize}
\item {Proveniência:(De \textunderscore tutor\textunderscore )}
\end{itemize}
O mesmo que \textunderscore tutelar\textunderscore ^2.
\section{Tutorear}
\begin{itemize}
\item {Grp. gram.:v. t.}
\end{itemize}
O mesmo que \textunderscore tutorar\textunderscore .
\section{Tutoria}
\begin{itemize}
\item {Grp. gram.:f.}
\end{itemize}
Cargo ou autoridade do tutor.
Exercício de tutela; tutela.
Protecção, defesa: \textunderscore há uma instituição official para tutoria da infância\textunderscore .
\section{Tutriz}
\begin{itemize}
\item {Grp. gram.:f.}
\end{itemize}
\begin{itemize}
\item {Proveniência:(Lat. \textunderscore tutrix\textunderscore )}
\end{itemize}
Tutora; defensora.
\section{Týmpano}
\begin{itemize}
\item {Grp. gram.:m.}
\end{itemize}
\begin{itemize}
\item {Utilização:Constr.}
\end{itemize}
\begin{itemize}
\item {Grp. gram.:Pl.}
\end{itemize}
\begin{itemize}
\item {Proveniência:(Lat. \textunderscore tympanum\textunderscore )}
\end{itemize}
Cavidade irregular, na base do rochedo auricular.
Peça de escultura, limitada por arcos ou linhas.
Espécie de tambor oco, com repartimentos em espiral, por intermédio dos quaes se eleva a água de um depósito ou de uma corrente.
Timbale.
Caixilho de ferro, recoberto de estôfo de algodão, ligado ao quadro do prelo por dois gonzos, e no qual se collocam as puncturas, se faz o alceamento, se regula a margem e se colloca successivamente cada uma das fôlhas a imprimir.
Planto rectangular, formado por uma linha que une as extremidades inferiores de um arco ou abóbada, por outra, vertical áquella, e por uma terceira, que liga as duas mencionadas.
Ouvidos.
\section{Typa}
\begin{itemize}
\item {Grp. gram.:f.}
\end{itemize}
\begin{itemize}
\item {Utilização:Chul.}
\end{itemize}
Qualquer mulhér.
Mulhér de costumes fáceis.
(Cp. \textunderscore typo\textunderscore )
\section{Typha}
\begin{itemize}
\item {Grp. gram.:f.}
\end{itemize}
\begin{itemize}
\item {Proveniência:(Lat. \textunderscore typhe\textunderscore )}
\end{itemize}
Nome scientífico da espadana.
\section{Typháceas}
\begin{itemize}
\item {Grp. gram.:f. pl.}
\end{itemize}
Família de plantas, que tem por typo a espadana.
\section{Typhão}
\begin{itemize}
\item {Grp. gram.:m.}
\end{itemize}
\begin{itemize}
\item {Utilização:Geol.}
\end{itemize}
\begin{itemize}
\item {Proveniência:(Do gr. \textunderscore Tuphon\textunderscore , n. p. de uma divindade egýpcia)}
\end{itemize}
Massa de terreno, não estratificada, na crosta da terra.
\section{Typhe}
\begin{itemize}
\item {Grp. gram.:f.}
\end{itemize}
O mesmo ou melhór que \textunderscore typha\textunderscore .
\section{Týphia}
\begin{itemize}
\item {Grp. gram.:f.}
\end{itemize}
Gênero de insectos hymenópteros.
\section{Týphico}
\begin{itemize}
\item {Grp. gram.:adj.}
\end{itemize}
Relativo ao typho ou que tem a natureza dêlle.
\section{Typhíneas}
\begin{itemize}
\item {Grp. gram.:f. pl.}
\end{itemize}
(V.typháceas)
\section{Typhismo}
\begin{itemize}
\item {Grp. gram.:m.}
\end{itemize}
\begin{itemize}
\item {Proveniência:(De \textunderscore typho\textunderscore )}
\end{itemize}
Caracter týphico de certas febres.
\section{Typhlina}
\begin{itemize}
\item {Grp. gram.:f.}
\end{itemize}
Gênero de reptís ophídios.
\section{Typhlite}
\begin{itemize}
\item {Grp. gram.:f.}
\end{itemize}
\begin{itemize}
\item {Proveniência:(Do gr. \textunderscore tuphlos\textunderscore )}
\end{itemize}
Inflammação do ceco.
\section{Typhlographia}
\begin{itemize}
\item {Grp. gram.:f.}
\end{itemize}
Arte de escrever em relêvo, para uso dos cegos.
(Cp. \textunderscore typhlógrapho\textunderscore )
\section{Typhlógrapho}
\begin{itemize}
\item {Grp. gram.:m.}
\end{itemize}
\begin{itemize}
\item {Proveniência:(Do gr. \textunderscore tuphlos\textunderscore , cego, e \textunderscore graphein\textunderscore , escrever)}
\end{itemize}
Instrumento, com que os cegos podem escrever.
\section{Typhlologia}
\begin{itemize}
\item {Grp. gram.:f.}
\end{itemize}
\begin{itemize}
\item {Proveniência:(Do gr. \textunderscore tuphlos\textunderscore  + \textunderscore logos\textunderscore )}
\end{itemize}
Tratado sôbre a instrucção dos cegos.
\section{Typhlológico}
\begin{itemize}
\item {Grp. gram.:adj.}
\end{itemize}
Relativo á typhlologia.
\section{Typhlólogo}
\begin{itemize}
\item {Grp. gram.:m.}
\end{itemize}
\begin{itemize}
\item {Proveniência:(Do gr. \textunderscore tuphlos\textunderscore  + \textunderscore logos\textunderscore )}
\end{itemize}
Aquelle que se occupa da instrucção dos cegos.
\section{Typho}
\begin{itemize}
\item {Grp. gram.:m.}
\end{itemize}
\begin{itemize}
\item {Proveniência:(Do gr. \textunderscore tuphos\textunderscore )}
\end{itemize}
Febre contínua, geralmente contagiosa, e produzida por influências miasmáticas.
Variedade de epizootia.
\section{Typhoemia}
\begin{itemize}
\item {Grp. gram.:f.}
\end{itemize}
\begin{itemize}
\item {Proveniência:(Do gr. \textunderscore tuphos\textunderscore  + \textunderscore haima\textunderscore )}
\end{itemize}
Alteração de sangue por influências miasmáticas.
\section{Typhóide}
\begin{itemize}
\item {Grp. gram.:adj.}
\end{itemize}
\begin{itemize}
\item {Proveniência:(Do gr. \textunderscore tuphos\textunderscore  + \textunderscore eidos\textunderscore )}
\end{itemize}
Que tem caracteres de typho; semelhante ao typho.
\section{Typhoídeo}
\begin{itemize}
\item {Grp. gram.:adj.}
\end{itemize}
O mesmo que \textunderscore typhóide\textunderscore .
\section{Typhomania}
\begin{itemize}
\item {Grp. gram.:f.}
\end{itemize}
\begin{itemize}
\item {Proveniência:(De \textunderscore typho\textunderscore  + \textunderscore mania\textunderscore )}
\end{itemize}
Delírio, que se manifesta na doença do typho.
\section{Typhónico}
\begin{itemize}
\item {Grp. gram.:adj.}
\end{itemize}
\begin{itemize}
\item {Utilização:Geol.}
\end{itemize}
Relativo a typhão.
Diz-se especialmente dos terrenos ou valles, limitados por séries de falhas e com o fundo levantado através dos terrenos mais recentes, com os quaes se acha actualmente em contacto em todo o seu perímetro. Cf. Gonç. Guimarães, \textunderscore Geologia\textunderscore , 144.
\section{Typhoso}
\begin{itemize}
\item {Grp. gram.:adj.}
\end{itemize}
\begin{itemize}
\item {Grp. gram.:M.}
\end{itemize}
\begin{itemize}
\item {Proveniência:(De \textunderscore typho\textunderscore )}
\end{itemize}
O mesmo que \textunderscore typhóide\textunderscore .
Diz-se dos phenómenos atáxicos e adynâmicos, que complicam a marcha de uma doença.
Indivíduo, atacado de typho.
\section{Typicamente}
\begin{itemize}
\item {Grp. gram.:adv.}
\end{itemize}
De modo týpico.
Á maneira de typo.
Com carácter de typo.
Symbolicamente.
\section{Týpico}
\begin{itemize}
\item {Grp. gram.:adj.}
\end{itemize}
\begin{itemize}
\item {Proveniência:(Lat. \textunderscore typicus\textunderscore )}
\end{itemize}
Que serve de typo; que caracteriza; symbólico.
\section{Typo}
\begin{itemize}
\item {Grp. gram.:m.}
\end{itemize}
\begin{itemize}
\item {Utilização:Fam.}
\end{itemize}
\begin{itemize}
\item {Utilização:Burl.}
\end{itemize}
\begin{itemize}
\item {Proveniência:(Lat. \textunderscore typus\textunderscore )}
\end{itemize}
Cunho, ou cada um dos caracteres typográphicos.
Aquilo que produz fé, como modêlo.
Coisa, que reúne em si os caracteres que distinguem uma classe.
Reunião dos caracteres que distinguem uma raça ou classe.
Sýmbolo; exemplar; modêlo: \textunderscore aquelle é typo da honradez\textunderscore .
Ordem, por que e manifestam ou succedem os symptomas de uma doença.
Pessôa excêntrica.
Qualquer indivíduo, pessôa pouco respeitável: \textunderscore não conheces aquelle typo?\textunderscore 
\section{Trechedipna}
\begin{itemize}
\item {fónica:que}
\end{itemize}
\begin{itemize}
\item {Grp. gram.:f.}
\end{itemize}
\begin{itemize}
\item {Proveniência:(Lat. \textunderscore trechedipna\textunderscore , pl. de \textunderscore trechedipnum\textunderscore )}
\end{itemize}
Sapato á moda grega, de fórma hoje desconhecida:«\textunderscore calçam gregas trequedipnas\textunderscore ». C. Lobo, \textunderscore Sát. de Juv.\textunderscore , I, 147.
\section{Trechedipno}
\begin{itemize}
\item {fónica:que}
\end{itemize}
\begin{itemize}
\item {Grp. gram.:m.}
\end{itemize}
\begin{itemize}
\item {Proveniência:(Lat. \textunderscore trechedipnum\textunderscore )}
\end{itemize}
O mesmo ou melhor que \textunderscore trequedipna\textunderscore .
\section{Tremular}
\begin{itemize}
\item {Grp. gram.:v. t.}
\end{itemize}
\begin{itemize}
\item {Grp. gram.:V. i.}
\end{itemize}
\begin{itemize}
\item {Proveniência:(Lat. \textunderscore tremulare\textunderscore )}
\end{itemize}
Mover com tremor.
Agitar.
Desfraldar.
Mover-se com tremor.
Tremeluzir.
Vacillar.
\section{Trepar}
\begin{itemize}
\item {Grp. gram.:v. t.}
\end{itemize}
\begin{itemize}
\item {Utilização:Prov.}
\end{itemize}
\begin{itemize}
\item {Utilização:minh.}
\end{itemize}
\begin{itemize}
\item {Proveniência:(De \textunderscore trepa\textunderscore ^1)}
\end{itemize}
Calcar aos pés.
\section{Trepeço}
\begin{itemize}
\item {Grp. gram.:m.}
\end{itemize}
\begin{itemize}
\item {Utilização:Prov.}
\end{itemize}
O mesmo que \textunderscore tripeça\textunderscore .
\section{Trepelada}
\begin{itemize}
\item {Grp. gram.:f.}
\end{itemize}
\begin{itemize}
\item {Utilização:Prov.}
\end{itemize}
\begin{itemize}
\item {Utilização:trasm.}
\end{itemize}
Pancadaria, trepa^1.
\section{Trepicar}
\begin{itemize}
\item {Grp. gram.:v. i.}
\end{itemize}
\begin{itemize}
\item {Utilização:Pop.}
\end{itemize}
Implicar, contender.
\section{Trepidação}
\begin{itemize}
\item {Grp. gram.:f.}
\end{itemize}
\begin{itemize}
\item {Proveniência:(Do lat. \textunderscore trepidatio\textunderscore )}
\end{itemize}
Acto ou efeito de trepidar.
Tremura dos nervos.
Pequeno tremor de terra.
\section{Trepidamente}
\begin{itemize}
\item {Grp. gram.:adv.}
\end{itemize}
De modo trépido; com susto.
\section{Trepidante}
\begin{itemize}
\item {Grp. gram.:adj.}
\end{itemize}
\begin{itemize}
\item {Proveniência:(Lat. \textunderscore trepidans\textunderscore )}
\end{itemize}
Que trepida; trêmulo.
Assustado.
\section{Trepidar}
\begin{itemize}
\item {Grp. gram.:v. i.}
\end{itemize}
\begin{itemize}
\item {Proveniência:(Lat. \textunderscore trepidari\textunderscore )}
\end{itemize}
Tremer com susto.
Andar ou apoiar-se, tremendo.
Vacillar.
\section{Trepidez}
\begin{itemize}
\item {Grp. gram.:f.}
\end{itemize}
Estado do que é trépido.
Tremura.
\section{Trépido}
\begin{itemize}
\item {Grp. gram.:adj.}
\end{itemize}
\begin{itemize}
\item {Proveniência:(Lat. \textunderscore trepidus\textunderscore )}
\end{itemize}
Trêmulo de susto; assustado.
Que corre ou flue, tremendo: \textunderscore o trépido regato\textunderscore .
\section{Tréplica}
\begin{itemize}
\item {Grp. gram.:f.}
\end{itemize}
Resposta a uma réplica; acto de treplicar.
\section{Treplicar}
\begin{itemize}
\item {Grp. gram.:v. t.}
\end{itemize}
\begin{itemize}
\item {Proveniência:(Do lat. \textunderscore triplicare\textunderscore )}
\end{itemize}
Responder a (uma réplica).
Refutar com tréplica.
\section{Treplos}
\begin{itemize}
\item {fónica:trê}
\end{itemize}
\begin{itemize}
\item {Grp. gram.:m. Pl.}
\end{itemize}
O mesmo que \textunderscore trêpulos\textunderscore .
\section{Trêpo}
\begin{itemize}
\item {Grp. gram.:m.}
\end{itemize}
\begin{itemize}
\item {Utilização:Prov.}
\end{itemize}
\begin{itemize}
\item {Utilização:minh.}
\end{itemize}
A parte da árvore, que fica na terra com as raízes, depois de cortada pelo fundo do tronco.
\section{Trepola}
\begin{itemize}
\item {fónica:pô}
\end{itemize}
\begin{itemize}
\item {Grp. gram.:f.}
\end{itemize}
\begin{itemize}
\item {Utilização:Prov.}
\end{itemize}
\begin{itemize}
\item {Utilização:trasm.}
\end{itemize}
Pôla grossa, grosso braço de árvore.
\section{Treposta}
\begin{itemize}
\item {Grp. gram.:f.}
\end{itemize}
\begin{itemize}
\item {Utilização:T. de Resende}
\end{itemize}
Batente de uma porta.
\section{Trêpulos}
\begin{itemize}
\item {Grp. gram.:m. Pl.}
\end{itemize}
\begin{itemize}
\item {Utilização:Prov.}
\end{itemize}
\begin{itemize}
\item {Utilização:beir.}
\end{itemize}
Grelos de hortaliça, cozidos e temperados com azeite, vinagre e sal.
Propriamente, espigos de nabo.
(Cp. [[túrnepos|túrnepo]])
\section{Trequedipna}
\begin{itemize}
\item {fónica:que}
\end{itemize}
\begin{itemize}
\item {Grp. gram.:f.}
\end{itemize}
\begin{itemize}
\item {Proveniência:(Lat. \textunderscore trechedipna\textunderscore , pl. de \textunderscore trechedipnum\textunderscore )}
\end{itemize}
Sapato á moda grega, de fórma hoje desconhecida:«\textunderscore calçam gregas trequedipnas\textunderscore ». C. Lobo, \textunderscore Sát. de Juv.\textunderscore , I, 147.
\section{Trequedipno}
\begin{itemize}
\item {fónica:que}
\end{itemize}
\begin{itemize}
\item {Grp. gram.:m.}
\end{itemize}
\begin{itemize}
\item {Proveniência:(Lat. \textunderscore trechedipnum\textunderscore )}
\end{itemize}
O mesmo ou melhor que \textunderscore trequedipna\textunderscore .
\section{Três}
\begin{itemize}
\item {Grp. gram.:adj.}
\end{itemize}
\begin{itemize}
\item {Grp. gram.:M.}
\end{itemize}
\begin{itemize}
\item {Proveniência:(Lat. \textunderscore tres\textunderscore )}
\end{itemize}
Diz-se do número cardinal, formado de dois e mais um.
Terceiro.
Algarismo, que representa três.
Dado ou carta de jogar, com três pintas.
\section{Tres...}
\begin{itemize}
\item {Grp. gram.:pref.}
\end{itemize}
O mesmo que \textunderscore trans...\textunderscore 
\section{Tres...}
\begin{itemize}
\item {Grp. gram.:pref.}
\end{itemize}
\begin{itemize}
\item {Proveniência:(Do lat. \textunderscore tres\textunderscore )}
\end{itemize}
(designativo de aumento, multiplicação ou intensidade)
O mesmo que \textunderscore tris...\textunderscore 
\section{Tresandar}
\begin{itemize}
\item {Grp. gram.:v. t.}
\end{itemize}
\begin{itemize}
\item {Grp. gram.:V. t.}
\end{itemize}
\begin{itemize}
\item {Proveniência:(De \textunderscore tres...\textunderscore ^1 + \textunderscore andar\textunderscore )}
\end{itemize}
Fazer voltar para trás.
Transformar.
Transtornar.
Exhalar (mau cheiro).
Cheirar muito mal.
\section{Tresavó}
\begin{itemize}
\item {Grp. gram.:f.}
\end{itemize}
O mesmo \textunderscore trisavó\textunderscore .
\section{Tresavô}
\begin{itemize}
\item {Grp. gram.:m.}
\end{itemize}
O mesmo que \textunderscore trisavô\textunderscore .
\section{Trescalante}
\begin{itemize}
\item {Grp. gram.:adj.}
\end{itemize}
Que trescala.
\section{Trescalar}
\begin{itemize}
\item {Grp. gram.:v. t.  e  i.}
\end{itemize}
\begin{itemize}
\item {Proveniência:(De \textunderscore tres...\textunderscore ^1 + \textunderscore calar\textunderscore )}
\end{itemize}
Emittir (cheiro).
Lançar de si, exhalar.
\section{Tresdobradura}
\begin{itemize}
\item {Grp. gram.:f.}
\end{itemize}
Acto ou effeito de \textunderscore tresdobrar\textunderscore .
\section{Tresdobrar}
\begin{itemize}
\item {Grp. gram.:v. t.}
\end{itemize}
\begin{itemize}
\item {Grp. gram.:V. i.}
\end{itemize}
\begin{itemize}
\item {Proveniência:(De \textunderscore tresdôbro\textunderscore )}
\end{itemize}
Dobrar três vezes.
Triplicar.
Aumentar-se três vezes.
\section{Tresdobre}
\begin{itemize}
\item {Grp. gram.:adj.}
\end{itemize}
\begin{itemize}
\item {Grp. gram.:M.}
\end{itemize}
\begin{itemize}
\item {Utilização:Pop.}
\end{itemize}
\begin{itemize}
\item {Proveniência:(De \textunderscore tres...\textunderscore ^2 + \textunderscore dobrar\textunderscore )}
\end{itemize}
Dizia-se de certa evolução militar.
Triplicado.
Tresdôbro.
\section{Tresdôbro}
\begin{itemize}
\item {Grp. gram.:m.}
\end{itemize}
\begin{itemize}
\item {Proveniência:(De \textunderscore tres...\textunderscore ^2 + \textunderscore dôbro\textunderscore )}
\end{itemize}
O mesmo que \textunderscore triplo\textunderscore .
\section{Três-em-prato}
\begin{itemize}
\item {Grp. gram.:f.}
\end{itemize}
Variedade de pêra, o mesmo que \textunderscore pêra-de-arratel\textunderscore .
\section{Tresfegar}
\begin{itemize}
\item {Grp. gram.:v. t.}
\end{itemize}
O mesmo que \textunderscore trasfegar\textunderscore .
\section{Tresfiar}
\begin{itemize}
\item {Grp. gram.:v. i.}
\end{itemize}
Entreabrirem-se (as aduelas dos cascos), por effeito do calor.
\section{Trespor}
\begin{itemize}
\item {Grp. gram.:v. t.}
\end{itemize}
\begin{itemize}
\item {Grp. gram.:V. p.}
\end{itemize}
O mesmo que \textunderscore transplantar\textunderscore . Cf. G. Vicente, I, 187.
O mesmo que [[pôr-se|pôr]], (falando-se do Sol):«\textunderscore por se lhes haver tresposto e desapparecido o Sol...\textunderscore »J. F. Castilho, \textunderscore Grinalda\textunderscore .
\section{Tresportalecer}
\begin{itemize}
\item {Grp. gram.:v. t.}
\end{itemize}
\begin{itemize}
\item {Utilização:Ant.}
\end{itemize}
O mesmo que \textunderscore trasportalecer\textunderscore .
\section{Tresposta}
\begin{itemize}
\item {Grp. gram.:f.}
\end{itemize}
\begin{itemize}
\item {Utilização:Ant.}
\end{itemize}
Terreno accidentado?«\textunderscore ...convem que seja a terra limpa de árvores, sem haver nella cabeços nem trespostas...\textunderscore »Fern. Pereira, \textunderscore Caça de Altan.\textunderscore , p. II, c. 4.
\section{Tresquiáltera}
\begin{itemize}
\item {Grp. gram.:f.}
\end{itemize}
\begin{itemize}
\item {Proveniência:(De \textunderscore tres...\textunderscore ^2 + \textunderscore quiáltera\textunderscore )}
\end{itemize}
Quiáltera, de três figuras, que tomam o lugar de duas.
\section{Tresse}
\begin{itemize}
\item {Grp. gram.:m.}
\end{itemize}
\begin{itemize}
\item {Proveniência:(Lat. \textunderscore tresis\textunderscore )}
\end{itemize}
O valor de três ases, entre os Romanos. Cf. Castilho, \textunderscore Fastos\textunderscore , I, 354.
\section{Tresselim}
\begin{itemize}
\item {Grp. gram.:m.}
\end{itemize}
Variedade de dança antiga:«\textunderscore cruzam pares, avançam-se, recuam com rodas, tresselins, chatés, cadeiras...\textunderscore »Filinto, X, 26.
(Por \textunderscore trancelim\textunderscore ?)
\section{Três-setes}
\begin{itemize}
\item {Grp. gram.:m. pl.}
\end{itemize}
Variedade de jôgo de cartas, em que não há trunfo, e em que a carta de maior valor é o três em cada naipe.
\section{Tréssis}
\begin{itemize}
\item {Grp. gram.:m.}
\end{itemize}
\begin{itemize}
\item {Proveniência:(Lat. \textunderscore tresis\textunderscore )}
\end{itemize}
O valor de três ases, entre os Romanos. Cf. Castilho, \textunderscore Fastos\textunderscore , I, 354.
\section{Tressuante}
\begin{itemize}
\item {Grp. gram.:adj.}
\end{itemize}
Que tressua.
\section{Tressuar}
\begin{itemize}
\item {Grp. gram.:v. i.}
\end{itemize}
\begin{itemize}
\item {Proveniência:(De \textunderscore tres...\textunderscore ^2 + \textunderscore suar\textunderscore )}
\end{itemize}
Suar muito.
\section{Trestampar}
\begin{itemize}
\item {Grp. gram.:v. i.}
\end{itemize}
\begin{itemize}
\item {Utilização:Des.}
\end{itemize}
Dizer destempêros.
(Por \textunderscore destemperar\textunderscore , de \textunderscore destempêro\textunderscore )
\section{Tresvariado}
\begin{itemize}
\item {Grp. gram.:adj.}
\end{itemize}
\begin{itemize}
\item {Proveniência:(De \textunderscore trasvariar\textunderscore )}
\end{itemize}
Que tresvariou; que delira.
\section{Tresvariar}
\begin{itemize}
\item {Grp. gram.:v. i.}
\end{itemize}
\begin{itemize}
\item {Proveniência:(De \textunderscore tres...\textunderscore ^1 + \textunderscore variar\textunderscore )}
\end{itemize}
Dizer ou fazer desvarios; delirar.
\section{Tresvario}
\begin{itemize}
\item {Grp. gram.:m.}
\end{itemize}
Acto ou effeito de \textunderscore tresvariar\textunderscore .
\section{Tresver}
\begin{itemize}
\item {Grp. gram.:v. t.}
\end{itemize}
\begin{itemize}
\item {Proveniência:(De \textunderscore tres...\textunderscore ^2 + \textunderscore vêr\textunderscore )}
\end{itemize}
Vêr com maus olhos? aborrecer?:«\textunderscore ela, que o tres-viu sempre como a morte!\textunderscore »Filinto, III, 90.
\section{Tresvoltear}
\begin{itemize}
\item {Grp. gram.:v. t.}
\end{itemize}
\begin{itemize}
\item {Proveniência:(De \textunderscore tres...\textunderscore ^2 + \textunderscore voltear\textunderscore )}
\end{itemize}
Fazer dar volta por três vezes.
Fazer dar muitas voltas.
\section{Treta}
\begin{itemize}
\item {fónica:trê}
\end{itemize}
\begin{itemize}
\item {Grp. gram.:f.}
\end{itemize}
\begin{itemize}
\item {Grp. gram.:Pl.}
\end{itemize}
Destreza na luta ou na esgrima.
Ardil; manha; estratagema:«\textunderscore ...tretas, de que os lutadores usão, para derribar seu competidor.\textunderscore »\textunderscore Luz e Calor\textunderscore , 261.
Palavreado, para enganar: \textunderscore deixa-te de tretas, e fala claro\textunderscore .
(Cp. \textunderscore treita\textunderscore )
\section{Treu}
\begin{itemize}
\item {Grp. gram.:m.}
\end{itemize}
\begin{itemize}
\item {Utilização:Náut.}
\end{itemize}
Vela latina, que se usava em occasião de temporal.
Pano para velas de navio, que se fabricava no termo do Pôrto.
\section{Treva}
\begin{itemize}
\item {Grp. gram.:f.}
\end{itemize}
\begin{itemize}
\item {Utilização:P. us.}
\end{itemize}
O mesmo que \textunderscore trevas\textunderscore . Cf. Camillo, \textunderscore Caveira\textunderscore , 157; Garrett, \textunderscore Folhas Caídas\textunderscore .
\section{Trevagem}
\begin{itemize}
\item {Grp. gram.:f.}
\end{itemize}
\begin{itemize}
\item {Proveniência:(De \textunderscore trevo\textunderscore )}
\end{itemize}
Erva rasteira e damninha, de folhas semelhantes ás do trevo, e da fam. das leguminosas, (\textunderscore medicago ciliaris\textunderscore , Lin.), também conhecida por \textunderscore luzerna brava\textunderscore .
\section{Trevas}
\begin{itemize}
\item {Grp. gram.:f. pl.}
\end{itemize}
\begin{itemize}
\item {Utilização:Fig.}
\end{itemize}
\begin{itemize}
\item {Proveniência:(Do lat. \textunderscore tenebrae\textunderscore )}
\end{itemize}
Privação ou ausência da luz.
Escuridão.
Noite.
Ignorância.
Dias, em que, na Semana Santa, se não deixa entrar a luz do dia nas igrejas.
As ceremónias eclesiásticas dêsses dias.
\section{Trevina}
\begin{itemize}
\item {Grp. gram.:f.}
\end{itemize}
\begin{itemize}
\item {Proveniência:(De \textunderscore trevo\textunderscore )}
\end{itemize}
Planta leguminosa, (\textunderscore pedrosia glauca\textunderscore , Ait).
\section{Treviscar}
\begin{itemize}
\item {Grp. gram.:v. i.}
\end{itemize}
\begin{itemize}
\item {Utilização:T. do Ribatejo}
\end{itemize}
\begin{itemize}
\item {Proveniência:(De \textunderscore tre...\textunderscore  + \textunderscore visco\textunderscore )}
\end{itemize}
Tornar-se viscoso, peganhento (o terreno), com a chuva miúda: \textunderscore a terra já trevisca\textunderscore .
\section{Trevite}
\begin{itemize}
\item {Grp. gram.:m.}
\end{itemize}
\begin{itemize}
\item {Proveniência:(De \textunderscore trevo\textunderscore ?)}
\end{itemize}
Droga medicinal da Índia.
\section{Trevo}
\begin{itemize}
\item {fónica:trê}
\end{itemize}
\begin{itemize}
\item {Grp. gram.:m.}
\end{itemize}
\begin{itemize}
\item {Proveniência:(Do lat. \textunderscore trifolium\textunderscore )}
\end{itemize}
Gênero de plantas leguminosas, entre cujas espécies as mais importantes são o trevo commum ou vermelho, o trevo branco ou pequeno trevo, e o trevo encarnado ou feno vermelho.
\section{Trevoada}
\begin{itemize}
\item {Grp. gram.:f.}
\end{itemize}
\begin{itemize}
\item {Utilização:ant.}
\end{itemize}
\begin{itemize}
\item {Utilização:Pop.}
\end{itemize}
O mesmo que \textunderscore trovoada\textunderscore .
\section{Triangulação}
\begin{itemize}
\item {Grp. gram.:f.}
\end{itemize}
Acto ou effeito de triangular.
\section{Triangulado}
\begin{itemize}
\item {Grp. gram.:adj.}
\end{itemize}
\begin{itemize}
\item {Proveniência:(De \textunderscore triangular\textunderscore ^1)}
\end{itemize}
Dividido em triângulos.
\section{Triangular}
\begin{itemize}
\item {Grp. gram.:v. t.}
\end{itemize}
Dividir em triângulos.
\section{Triangular}
\begin{itemize}
\item {Grp. gram.:adj.}
\end{itemize}
\begin{itemize}
\item {Proveniência:(Lat. \textunderscore triangularis\textunderscore )}
\end{itemize}
O mesmo que \textunderscore triangulado\textunderscore .
Que tem por base um triângulo.
Que tem a fórma de um triângulo.
Que tem três ângulos.
\section{Triangularmente}
\begin{itemize}
\item {Grp. gram.:adv.}
\end{itemize}
De modo triangular.
\section{Triângulo}
\begin{itemize}
\item {Grp. gram.:m.}
\end{itemize}
\begin{itemize}
\item {Proveniência:(Lat. \textunderscore triângulos\textunderscore )}
\end{itemize}
Figura geométrica de três lados e três ângulos.
Qualquer objecto de fórma triangular.
Constellação boreal.
Instrumento musical, o mesmo que \textunderscore ferrinhos\textunderscore .
O mesmo que \textunderscore fôrca\textunderscore :«\textunderscore vêr pernear um justiçado no triângulo...\textunderscore »Camillo, \textunderscore Homem de Brios\textunderscore , VIII.
\section{Trianno}
\begin{itemize}
\item {Grp. gram.:m.}
\end{itemize}
\begin{itemize}
\item {Utilização:Des.}
\end{itemize}
O mesmo que \textunderscore triênnio\textunderscore :«\textunderscore tenho a capitania... de Chaul por dous triannos\textunderscore ». (De um testamento do séc. XVII)
\section{Triano}
\begin{itemize}
\item {Grp. gram.:m.}
\end{itemize}
\begin{itemize}
\item {Utilização:Des.}
\end{itemize}
O mesmo que \textunderscore triênio\textunderscore :«\textunderscore tenho a capitania... de Chaul por dous trianos\textunderscore ». (De um testamento do séc. XVII)
\section{Triantema}
\begin{itemize}
\item {Grp. gram.:f.}
\end{itemize}
\begin{itemize}
\item {Proveniência:(Do gr. \textunderscore tri\textunderscore  + \textunderscore anthemon\textunderscore )}
\end{itemize}
Gênero de plantas portuláceas.
\section{Trianthema}
\begin{itemize}
\item {Grp. gram.:f.}
\end{itemize}
\begin{itemize}
\item {Proveniência:(Do gr. \textunderscore tri\textunderscore  + \textunderscore anthemon\textunderscore )}
\end{itemize}
Gênero de plantas portuláceas.
\section{Triantho}
\begin{itemize}
\item {Grp. gram.:m.}
\end{itemize}
\begin{itemize}
\item {Proveniência:(Do gr. \textunderscore tri\textunderscore  + \textunderscore anthos\textunderscore )}
\end{itemize}
Gênero de plantas, da fam. das compostas.
\section{Trianto}
\begin{itemize}
\item {Grp. gram.:m.}
\end{itemize}
\begin{itemize}
\item {Proveniência:(Do gr. \textunderscore tri\textunderscore  + \textunderscore anthos\textunderscore )}
\end{itemize}
Gênero de plantas, da fam. das compostas.
\section{Triarchia}
\begin{itemize}
\item {fónica:qui}
\end{itemize}
\begin{itemize}
\item {Grp. gram.:f.}
\end{itemize}
\begin{itemize}
\item {Proveniência:(Gr. \textunderscore triarkhia\textunderscore )}
\end{itemize}
Govêrno exercido por três indivíduos.
Conjunto de três Estados.
Triunvirado.
\section{Triarestado}
\begin{itemize}
\item {Grp. gram.:adj.}
\end{itemize}
\begin{itemize}
\item {Utilização:Bot.}
\end{itemize}
\begin{itemize}
\item {Proveniência:(De \textunderscore tri...\textunderscore  + \textunderscore aresta\textunderscore )}
\end{itemize}
Que tem três arestas.
\section{Triarquia}
\begin{itemize}
\item {Grp. gram.:f.}
\end{itemize}
\begin{itemize}
\item {Proveniência:(Gr. \textunderscore triarkhia\textunderscore )}
\end{itemize}
Govêrno exercido por três indivíduos.
Conjunto de três Estados.
Triunvirado.
\section{Triários}
\begin{itemize}
\item {Grp. gram.:m. pl.}
\end{itemize}
\begin{itemize}
\item {Proveniência:(Lat. \textunderscore triarii\textunderscore )}
\end{itemize}
Soldados romanos, que combatiam de joêlhos, na terceira fila do exército.
Veteranos ou reformados, que iam de refôrço na retaguarda das tropas. Cf. \textunderscore Viriato Trág.\textunderscore , XVI, 62 e 63.
\section{Triarticulado}
\begin{itemize}
\item {Grp. gram.:adj.}
\end{itemize}
\begin{itemize}
\item {Proveniência:(De \textunderscore tri...\textunderscore  + \textunderscore articulado\textunderscore )}
\end{itemize}
Que tem três artículos.
\section{Trias}
\begin{itemize}
\item {Grp. gram.:m.}
\end{itemize}
\begin{itemize}
\item {Proveniência:(Gr. \textunderscore trias\textunderscore )}
\end{itemize}
Formação geológica, que succede immediatamente ao lias, na ordem descendente.
\section{Triásico}
\begin{itemize}
\item {Grp. gram.:adj.}
\end{itemize}
\begin{itemize}
\item {Proveniência:(De \textunderscore trias\textunderscore )}
\end{itemize}
O mesmo que \textunderscore triádico\textunderscore .
\section{Triatera}
\begin{itemize}
\item {Grp. gram.:f.}
\end{itemize}
\begin{itemize}
\item {Proveniência:(Do gr. \textunderscore tri\textunderscore  + \textunderscore ather\textunderscore )}
\end{itemize}
Gênero de plantas gramíneas.
\section{Triathera}
\begin{itemize}
\item {Grp. gram.:f.}
\end{itemize}
\begin{itemize}
\item {Proveniência:(Do gr. \textunderscore tri\textunderscore  + \textunderscore ather\textunderscore )}
\end{itemize}
Gênero de plantas gramíneas.
\section{Triatomicidade}
\begin{itemize}
\item {Grp. gram.:f.}
\end{itemize}
Qualidade de um átomo triatómico.
\section{Triatómico}
\begin{itemize}
\item {Grp. gram.:adj.}
\end{itemize}
\begin{itemize}
\item {Proveniência:(De \textunderscore tri...\textunderscore  + \textunderscore atómico\textunderscore )}
\end{itemize}
Diz-se do átomo, que tem três pontos de attracção.
\section{Triaxífero}
\begin{itemize}
\item {fónica:csi}
\end{itemize}
\begin{itemize}
\item {Grp. gram.:adj.}
\end{itemize}
\begin{itemize}
\item {Utilização:Bot.}
\end{itemize}
\begin{itemize}
\item {Proveniência:(De \textunderscore tri...\textunderscore  + \textunderscore axífero\textunderscore )}
\end{itemize}
Que tem três eixos.
\section{Tríbade}
\begin{itemize}
\item {Grp. gram.:f.}
\end{itemize}
\begin{itemize}
\item {Proveniência:(Do gr. \textunderscore tribein\textunderscore )}
\end{itemize}
Mulhér, dada a práticas homosexuaes.
\section{Tribadia}
\begin{itemize}
\item {Grp. gram.:f.}
\end{itemize}
O mesmo que \textunderscore tribadismo\textunderscore .
\section{Tribadismo}
\begin{itemize}
\item {Grp. gram.:m.}
\end{itemize}
Vício ou práticas de tríbade.
\section{Tribasicidade}
\begin{itemize}
\item {Grp. gram.:f.}
\end{itemize}
\begin{itemize}
\item {Utilização:Chím.}
\end{itemize}
Qualidade de tribásico.
\section{Tribásico}
\begin{itemize}
\item {Grp. gram.:adj.}
\end{itemize}
\begin{itemize}
\item {Proveniência:(De \textunderscore tri...\textunderscore  + \textunderscore básico\textunderscore )}
\end{itemize}
Diz-se dos saes, que contêm três equivalentes de base por um ácido.
Diz-se do ácido que, em combinação, não póde sêr neutralizado, senão por três equivalentes de uma base.
\section{Tribasilar}
\begin{itemize}
\item {Grp. gram.:adj.}
\end{itemize}
\begin{itemize}
\item {Utilização:Anat.}
\end{itemize}
\begin{itemize}
\item {Proveniência:(De \textunderscore tri...\textunderscore  + \textunderscore basilar\textunderscore )}
\end{itemize}
Diz-se do osso da base do crânio, formado pela soldadura do occipital com o esphenóide.
\section{Tribo}
\begin{itemize}
\item {Grp. gram.:f.}
\end{itemize}
\begin{itemize}
\item {Utilização:Hist. Nat.}
\end{itemize}
\begin{itemize}
\item {Proveniência:(Lat. \textunderscore tribus\textunderscore )}
\end{itemize}
Cada uma das divisões de um povo, em algumas nações antigas.
Conjunto dos descendentes de cada um de doze patriarchas, entre os Judeus.
Pequeno povo.
Sociedade rudimentar.
Divisão de famílias--É voc. masculino, em escritos antigos. Cf. Pant. de Aveiro, \textunderscore Itiner.\textunderscore , 86, (2.^a ed.).
\section{Tribofe}
\begin{itemize}
\item {Grp. gram.:m.}
\end{itemize}
\begin{itemize}
\item {Utilização:Bras}
\end{itemize}
Trapaça, em corridas \textunderscore hippodrómicas\textunderscore .
\section{Tribometria}
\begin{itemize}
\item {Grp. gram.:f.}
\end{itemize}
Applicação do tribómetro.
\section{Tribométrico}
\begin{itemize}
\item {Grp. gram.:adj.}
\end{itemize}
Relativo á tribometria ou ao \textunderscore tribómetro\textunderscore .
\section{Tributo}
\begin{itemize}
\item {Grp. gram.:m.}
\end{itemize}
\begin{itemize}
\item {Proveniência:(Lat. \textunderscore tributum\textunderscore )}
\end{itemize}
Aquillo que um Estado para a outro, em sinal de dependência.
Qualquer imposto.
Aquillo que se concede por hábito ou necessidade.
O que se é obrigado a soffrer; homenagem.
\section{Trica}
\begin{itemize}
\item {Grp. gram.:f.}
\end{itemize}
\begin{itemize}
\item {Proveniência:(Lat. \textunderscore trica\textunderscore )}
\end{itemize}
Intriga.
Chicana.
Trapaça.
Futilidade, nica.
\section{Tricalho}
\begin{itemize}
\item {Grp. gram.:m.}
\end{itemize}
\begin{itemize}
\item {Utilização:Prov.}
\end{itemize}
\begin{itemize}
\item {Utilização:beir.}
\end{itemize}
\begin{itemize}
\item {Proveniência:(De \textunderscore trica\textunderscore )}
\end{itemize}
Espécie de jôgo antigo, que se prestava á burla ou á trica.
\section{Tricalho}
\begin{itemize}
\item {Grp. gram.:m.}
\end{itemize}
\begin{itemize}
\item {Utilização:Prov.}
\end{itemize}
\begin{itemize}
\item {Utilização:beir.}
\end{itemize}
O mesmo que \textunderscore trapicalho\textunderscore .
Indivíduo torpe ou immundo.
\section{Tricalísia}
\begin{itemize}
\item {Grp. gram.:f.}
\end{itemize}
Gênero de plantas rubiáceas.
\section{Tricalýsia}
\begin{itemize}
\item {Grp. gram.:f.}
\end{itemize}
Gênero de plantas rubiáceas.
\section{Tricâmeron}
\begin{itemize}
\item {Grp. gram.:m.}
\end{itemize}
Edifício de três andares ou de três abóbadas sobrepostas.
(Cp. lat. \textunderscore tricameratus\textunderscore )
\section{Tricâmero}
\begin{itemize}
\item {Grp. gram.:adj.}
\end{itemize}
\begin{itemize}
\item {Utilização:Bot.}
\end{itemize}
\begin{itemize}
\item {Proveniência:(De \textunderscore tri\textunderscore  + \textunderscore câmara\textunderscore )}
\end{itemize}
Que tem três câmaras.
\section{Tricana}
\begin{itemize}
\item {Grp. gram.:f.}
\end{itemize}
Espécie de burel antigo.
Saia dêsse tecido.
Rapariga do povo ou do campo; camponesa.
\section{Tricanto}
\begin{itemize}
\item {Grp. gram.:m.}
\end{itemize}
\begin{itemize}
\item {Proveniência:(Do gr. \textunderscore trix\textunderscore , \textunderscore trikhos\textunderscore  + \textunderscore anthos\textunderscore )}
\end{itemize}
Gênero de plantas gesneriáceas.
\section{Tricapsular}
\begin{itemize}
\item {Grp. gram.:adj.}
\end{itemize}
\begin{itemize}
\item {Utilização:Bot.}
\end{itemize}
\begin{itemize}
\item {Proveniência:(De \textunderscore tri...\textunderscore  + \textunderscore capsular\textunderscore )}
\end{itemize}
Que tem três capsulas.
\section{Tricéfalo}
\begin{itemize}
\item {Grp. gram.:m.  e  adj.}
\end{itemize}
\begin{itemize}
\item {Proveniência:(Do gr. \textunderscore treis\textunderscore  + \textunderscore kephale\textunderscore )}
\end{itemize}
O que tem três cabeças.
\section{Tricellular}
\begin{itemize}
\item {Grp. gram.:adj.}
\end{itemize}
\begin{itemize}
\item {Utilização:Bot.}
\end{itemize}
\begin{itemize}
\item {Proveniência:(De \textunderscore tri...\textunderscore  + \textunderscore cellular\textunderscore )}
\end{itemize}
Que tem três céllulas.
\section{Tricelular}
\begin{itemize}
\item {Grp. gram.:adj.}
\end{itemize}
\begin{itemize}
\item {Utilização:Bot.}
\end{itemize}
\begin{itemize}
\item {Proveniência:(De \textunderscore tri...\textunderscore  + \textunderscore celular\textunderscore )}
\end{itemize}
Que tem três células.
\section{Tricenal}
\begin{itemize}
\item {Grp. gram.:adj.}
\end{itemize}
Que dura trinta annos.
(Cp. lat. \textunderscore tricenarius\textunderscore )
\section{Tricentenário}
\begin{itemize}
\item {Grp. gram.:adj.}
\end{itemize}
\begin{itemize}
\item {Grp. gram.:M.}
\end{itemize}
\begin{itemize}
\item {Proveniência:(De \textunderscore tri...\textunderscore  + \textunderscore centenário\textunderscore )}
\end{itemize}
Que tem trezentos annos.
Commemoração de facto notável, succedido há trezentos annos.
\section{Tricentésimo}
\begin{itemize}
\item {Grp. gram.:adj.}
\end{itemize}
O mesmo que \textunderscore trecentésimo\textunderscore .
\section{Tricêntrico}
\begin{itemize}
\item {Grp. gram.:adj.}
\end{itemize}
\begin{itemize}
\item {Utilização:Constr.}
\end{itemize}
Que tem três centros ou três arcos successivos:«\textunderscore a portada da igreja, de arco tricêntrico, firmado em pilares...\textunderscore »Herculano, \textunderscore Párocho\textunderscore , C. I.
\section{Tricéphalo}
\begin{itemize}
\item {Grp. gram.:m.  e  adj.}
\end{itemize}
\begin{itemize}
\item {Proveniência:(Do gr. \textunderscore treis\textunderscore  + \textunderscore kephale\textunderscore )}
\end{itemize}
O que tem três cabeças.
\section{Trícera}
\begin{itemize}
\item {Grp. gram.:f.}
\end{itemize}
\begin{itemize}
\item {Proveniência:(Do gr. \textunderscore treis\textunderscore  + \textunderscore keras\textunderscore )}
\end{itemize}
Gênero de plantas euphorbiáceas.
\section{Tricerasta}
\begin{itemize}
\item {Grp. gram.:f.}
\end{itemize}
\begin{itemize}
\item {Proveniência:(Do gr. \textunderscore treis\textunderscore  + \textunderscore kerastes\textunderscore )}
\end{itemize}
Gênero de plantas datiscáceas.
\section{Trícero}
\begin{itemize}
\item {Grp. gram.:m.}
\end{itemize}
Gênero de plantas burseráceas.
(Cp. \textunderscore trícera\textunderscore )
\section{Tricésimo}
\begin{itemize}
\item {Grp. gram.:adj.}
\end{itemize}
\begin{itemize}
\item {Proveniência:(Lat. \textunderscore tricesineus\textunderscore )}
\end{itemize}
O mesmo que \textunderscore trigésimo\textunderscore .
\section{Trichantho}
\begin{itemize}
\item {fónica:can}
\end{itemize}
\begin{itemize}
\item {Grp. gram.:m.}
\end{itemize}
\begin{itemize}
\item {Proveniência:(Do gr. \textunderscore trix\textunderscore , \textunderscore trikhos\textunderscore  + \textunderscore anthos\textunderscore )}
\end{itemize}
Gênero de plantas gesneriáceas.
\section{Trichecho}
\begin{itemize}
\item {fónica:quéco}
\end{itemize}
\begin{itemize}
\item {Grp. gram.:m.}
\end{itemize}
\begin{itemize}
\item {Proveniência:(Do gr. \textunderscore thrix\textunderscore  + \textunderscore ekhein\textunderscore )}
\end{itemize}
Grande animal mammífero, que attinge 7 a 10 metros de comprimento e que também é conhecido por \textunderscore cavallo-marinho\textunderscore .
\section{Trichíase}
\begin{itemize}
\item {fónica:chi}
\end{itemize}
\begin{itemize}
\item {Grp. gram.:f.}
\end{itemize}
\begin{itemize}
\item {Utilização:Med.}
\end{itemize}
\begin{itemize}
\item {Proveniência:(Lat. \textunderscore trichiasis\textunderscore )}
\end{itemize}
Affecção mórbida, em que as pestanas, desviadas da direcção natural, se põem em contacto com o globo do ôlho.
\section{Trichíasis}
\begin{itemize}
\item {fónica:qui}
\end{itemize}
\begin{itemize}
\item {Grp. gram.:f.}
\end{itemize}
O mesmo que \textunderscore trichíase\textunderscore .
\section{Trichília}
\begin{itemize}
\item {fónica:qui}
\end{itemize}
\begin{itemize}
\item {Grp. gram.:f.}
\end{itemize}
\begin{itemize}
\item {Proveniência:(Do gr. \textunderscore tria\textunderscore  + \textunderscore kheilos\textunderscore )}
\end{itemize}
Árvore meliácea.
\section{Trichina}
\begin{itemize}
\item {fónica:qui}
\end{itemize}
\begin{itemize}
\item {Grp. gram.:f.}
\end{itemize}
\begin{itemize}
\item {Proveniência:(Gr. \textunderscore trikhinos\textunderscore )}
\end{itemize}
Gênero de vermes intestinaes.
\section{Trichinado}
\begin{itemize}
\item {fónica:qui}
\end{itemize}
\begin{itemize}
\item {Grp. gram.:adj.}
\end{itemize}
Que tem trichinas.
\section{Trichinoscópio}
\begin{itemize}
\item {fónica:qui}
\end{itemize}
\begin{itemize}
\item {Grp. gram.:m.}
\end{itemize}
\begin{itemize}
\item {Proveniência:(Do gr. \textunderscore trikhinos\textunderscore  + \textunderscore skopein\textunderscore )}
\end{itemize}
Apparelho para investigação e exame da trichina, nas carnes.
\section{Trichinose}
\begin{itemize}
\item {fónica:qui}
\end{itemize}
\begin{itemize}
\item {Grp. gram.:f.}
\end{itemize}
Doença causada pelas trichinas.
\section{Trichinoso}
\begin{itemize}
\item {fónica:qui}
\end{itemize}
\begin{itemize}
\item {Grp. gram.:adj.}
\end{itemize}
O mesmo que \textunderscore trichinado\textunderscore .
\section{Tríchio}
\begin{itemize}
\item {fónica:qui}
\end{itemize}
\begin{itemize}
\item {Grp. gram.:m.}
\end{itemize}
\begin{itemize}
\item {Proveniência:(Do gr. \textunderscore thrix\textunderscore , \textunderscore trikhos\textunderscore )}
\end{itemize}
Gênero de insectos coleópteros, da fam. dos lamellicórneos.
Gênero de molluscos gasterópodes.
Gênero de crustáceos decápodes.
\section{Triqueco}
\begin{itemize}
\item {Grp. gram.:m.}
\end{itemize}
\begin{itemize}
\item {Proveniência:(Do gr. \textunderscore thrix\textunderscore  + \textunderscore ekhein\textunderscore )}
\end{itemize}
Grande animal mamífero, que atinge 7 a 10 metros de comprimento e que também é conhecido por \textunderscore cavalo-marinho\textunderscore .
\section{Triquíase}
\begin{itemize}
\item {Grp. gram.:f.}
\end{itemize}
\begin{itemize}
\item {Utilização:Med.}
\end{itemize}
\begin{itemize}
\item {Proveniência:(Lat. \textunderscore trichiasis\textunderscore )}
\end{itemize}
Afecção mórbida, em que as pestanas, desviadas da direcção natural, se põem em contacto com o globo do ôlho.
\section{Triquíasis}
\begin{itemize}
\item {Grp. gram.:f.}
\end{itemize}
O mesmo que \textunderscore triquíase\textunderscore .
\section{Triquília}
\begin{itemize}
\item {Grp. gram.:f.}
\end{itemize}
\begin{itemize}
\item {Proveniência:(Do gr. \textunderscore tria\textunderscore  + \textunderscore kheilos\textunderscore )}
\end{itemize}
Árvore meliácea.
\section{Triquina}
\begin{itemize}
\item {Grp. gram.:f.}
\end{itemize}
\begin{itemize}
\item {Proveniência:(Gr. \textunderscore trikhinos\textunderscore )}
\end{itemize}
Gênero de vermes intestinaes.
\section{Triquinado}
\begin{itemize}
\item {Grp. gram.:adj.}
\end{itemize}
Que tem triquinas.
\section{Triquinoscópio}
\begin{itemize}
\item {Grp. gram.:m.}
\end{itemize}
\begin{itemize}
\item {Proveniência:(Do gr. \textunderscore trikhinos\textunderscore  + \textunderscore skopein\textunderscore )}
\end{itemize}
Aparelho para investigação e exame da triquina, nas carnes.
\section{Triquinose}
\begin{itemize}
\item {Grp. gram.:f.}
\end{itemize}
Doença causada pelas triquinas.
\section{Triquinoso}
\begin{itemize}
\item {Grp. gram.:adj.}
\end{itemize}
O mesmo que \textunderscore triquinado\textunderscore .
\section{Tríquio}
\begin{itemize}
\item {Grp. gram.:m.}
\end{itemize}
\begin{itemize}
\item {Proveniência:(Do gr. \textunderscore thrix\textunderscore , \textunderscore trikhos\textunderscore )}
\end{itemize}
Gênero de insectos coleópteros, da fam. dos lamelicórneos.
Gênero de moluscos gasterópodes.
Gênero de crustáceos decápodes.
\section{Trichosoma}
\begin{itemize}
\item {fónica:cosso}
\end{itemize}
\begin{itemize}
\item {Grp. gram.:f.}
\end{itemize}
\begin{itemize}
\item {Proveniência:(Do gr. \textunderscore trix\textunderscore , \textunderscore trikhos\textunderscore  + \textunderscore soma\textunderscore )}
\end{itemize}
Gênero de insectos lepidópteros nocturnos.
\section{Trichóspira}
\begin{itemize}
\item {fónica:cós}
\end{itemize}
\begin{itemize}
\item {Grp. gram.:f.}
\end{itemize}
Gênero de plantas, da fam. das compostas.
\section{Trichóstomo}
\begin{itemize}
\item {fónica:cos}
\end{itemize}
\begin{itemize}
\item {Grp. gram.:m.}
\end{itemize}
\begin{itemize}
\item {Proveniência:(Do gr. \textunderscore trix\textunderscore , \textunderscore trikhos\textunderscore  + \textunderscore stoma\textunderscore )}
\end{itemize}
Gênero de musgos.
\section{Trichomia}
\begin{itemize}
\item {fónica:co}
\end{itemize}
\begin{itemize}
\item {Grp. gram.:f.}
\end{itemize}
\begin{itemize}
\item {Utilização:Bot.}
\end{itemize}
Divisão de um caule de três galhos ou braços, e dos galhos em três ramos, e assim por diante.
(Cp. \textunderscore trichótomo\textunderscore )
\section{Trichotómico}
\begin{itemize}
\item {fónica:co}
\end{itemize}
\begin{itemize}
\item {Grp. gram.:adj.}
\end{itemize}
Relativo á trichotomia.
\section{Trichótomo}
\begin{itemize}
\item {fónica:có}
\end{itemize}
\begin{itemize}
\item {Grp. gram.:adj.}
\end{itemize}
\begin{itemize}
\item {Proveniência:(Do gr. \textunderscore trikha\textunderscore  + \textunderscore tome\textunderscore )}
\end{itemize}
Dividido em três.
Que se faz por divisões sucessivas de três.
\section{Trichotósia}
\begin{itemize}
\item {fónica:co}
\end{itemize}
\begin{itemize}
\item {Grp. gram.:f.}
\end{itemize}
Gênero de orchídeas.
\section{Trichróico}
\begin{itemize}
\item {Grp. gram.:adj.}
\end{itemize}
\begin{itemize}
\item {Utilização:Miner.}
\end{itemize}
Diz-se de metaes que têm a qualdiade de trichroísmo.
\section{Trichroísmo}
\begin{itemize}
\item {Grp. gram.:m.}
\end{itemize}
\begin{itemize}
\item {Utilização:Miner.}
\end{itemize}
\begin{itemize}
\item {Proveniência:(Do gr. \textunderscore tri\textunderscore  + \textunderscore khroa\textunderscore )}
\end{itemize}
Propriedade, que os mineraes de dois eixos de dupla refracção têm, de oferecer três côres diferentes, quando olhados em diversos sentidos.
\section{Trichróito}
\begin{itemize}
\item {Grp. gram.:adj.}
\end{itemize}
O mesmo que \textunderscore trichróico\textunderscore .
\section{Tricipital}
\begin{itemize}
\item {Grp. gram.:adj.}
\end{itemize}
\begin{itemize}
\item {Utilização:Anat.}
\end{itemize}
Relativo ao osso que se chama tricípite.
\section{Tricípite}
\begin{itemize}
\item {Grp. gram.:m.  e  adj.}
\end{itemize}
\begin{itemize}
\item {Utilização:Anat.}
\end{itemize}
\begin{itemize}
\item {Proveniência:(Do lat. hypoth. \textunderscore triceps\textunderscore , \textunderscore tricipitis\textunderscore )}
\end{itemize}
Diz-se de um osso com três pontas, situado no braço, e de outro situado na coxa.
\section{Tricládia}
\begin{itemize}
\item {Grp. gram.:f.}
\end{itemize}
\begin{itemize}
\item {Proveniência:(Do gr. \textunderscore treis\textunderscore  + \textunderscore klados\textunderscore )}
\end{itemize}
Gênero de plantas phýceas.
\section{Tricliniarca}
\begin{itemize}
\item {Grp. gram.:m.}
\end{itemize}
\begin{itemize}
\item {Proveniência:(Lat. \textunderscore tricliniarca\textunderscore )}
\end{itemize}
O encarregado do banquete e dos adornos do triclínio, entre os antigos.
\section{Tricliniarcha}
\begin{itemize}
\item {fónica:ca}
\end{itemize}
\begin{itemize}
\item {Grp. gram.:m.}
\end{itemize}
\begin{itemize}
\item {Proveniência:(Lat. \textunderscore tricliniarca\textunderscore )}
\end{itemize}
O encarregado do banquete e dos adornos do triclínio, entre os antigos.
\section{Tricliniário}
\begin{itemize}
\item {Grp. gram.:m.}
\end{itemize}
\begin{itemize}
\item {Proveniência:(Lat. \textunderscore tricliniarius\textunderscore )}
\end{itemize}
Escravo, que servia á mesa, nos triclínios.
\section{Triclínico}
\begin{itemize}
\item {Grp. gram.:adj.}
\end{itemize}
\begin{itemize}
\item {Utilização:Miner.}
\end{itemize}
\begin{itemize}
\item {Proveniência:(Do gr. \textunderscore treis\textunderscore  + \textunderscore klinein\textunderscore )}
\end{itemize}
Diz-se do systema crystallográphico, em que há três eixos desiguaes e oblíquos.
\section{Triclínio}
\begin{itemize}
\item {Grp. gram.:m.}
\end{itemize}
\begin{itemize}
\item {Proveniência:(Lat. \textunderscore triclinium\textunderscore )}
\end{itemize}
Entre os antigos Romanos, sala para refeições, com três leitos, em cada um dos quaes se sentavam três convivas.
\section{Tricobáltico}
\begin{itemize}
\item {Grp. gram.:adj.}
\end{itemize}
\begin{itemize}
\item {Utilização:Chím.}
\end{itemize}
\begin{itemize}
\item {Proveniência:(De \textunderscore tri...\textunderscore  + \textunderscore cobáltico\textunderscore )}
\end{itemize}
Diz-se do sal cobáltico, que contém três vezes tanta base como o sal neutro correspondente.
\section{Tricocas}
\begin{itemize}
\item {Grp. gram.:f. pl.}
\end{itemize}
\begin{itemize}
\item {Utilização:Bot.}
\end{itemize}
\begin{itemize}
\item {Proveniência:(De \textunderscore tricoco\textunderscore )}
\end{itemize}
Ordem de plantas, que abrange as euforbiáceas e outras.
\section{Tricocco}
\begin{itemize}
\item {Grp. gram.:adj.}
\end{itemize}
\begin{itemize}
\item {Proveniência:(Do gr. \textunderscore trikokkon\textunderscore )}
\end{itemize}
Que tem três céllulas ôcas.
\section{Tricoccas}
\begin{itemize}
\item {Grp. gram.:f. pl.}
\end{itemize}
\begin{itemize}
\item {Utilização:Bot.}
\end{itemize}
\begin{itemize}
\item {Proveniência:(De \textunderscore tricocco\textunderscore )}
\end{itemize}
Ordem de plantas, que abrange as euphorbiáceas e outras.
\section{Tricoco}
\begin{itemize}
\item {Grp. gram.:adj.}
\end{itemize}
\begin{itemize}
\item {Proveniência:(Do gr. \textunderscore trikokkon\textunderscore )}
\end{itemize}
Que tem três células ôcas.
\section{Tricolor}
\begin{itemize}
\item {Grp. gram.:adj.}
\end{itemize}
\begin{itemize}
\item {Proveniência:(Lat. \textunderscore tricolor\textunderscore )}
\end{itemize}
Que tem três côres.
\section{Tricolóreo}
\begin{itemize}
\item {Grp. gram.:adj.}
\end{itemize}
O mesmo que \textunderscore tricolor\textunderscore . Cf. \textunderscore Elegíada\textunderscore , 85.
\section{Tricomia}
\begin{itemize}
\item {Grp. gram.:f.}
\end{itemize}
\begin{itemize}
\item {Utilização:Bot.}
\end{itemize}
Divisão de um caule de três galhos ou braços, e dos galhos em três ramos, e assim por diante.
(Cp. \textunderscore tricótomo\textunderscore )
\section{Tricorde}
\begin{itemize}
\item {Grp. gram.:adj.}
\end{itemize}
\begin{itemize}
\item {Proveniência:(De \textunderscore tri...\textunderscore  + \textunderscore corda\textunderscore )}
\end{itemize}
Que tem três cordas. Cf. Filinto, I, 223.
\section{Tricórdio}
\begin{itemize}
\item {Grp. gram.:m.}
\end{itemize}
\begin{itemize}
\item {Proveniência:(De \textunderscore tri...\textunderscore  + \textunderscore corda\textunderscore )}
\end{itemize}
Instrumento de três cordas.
\section{Tricorne}
\begin{itemize}
\item {Grp. gram.:adj.}
\end{itemize}
\begin{itemize}
\item {Proveniência:(Lat. \textunderscore tricornis\textunderscore )}
\end{itemize}
Que tem três cornos, pontas ou bicos.
\section{Tricórnio}
\begin{itemize}
\item {Grp. gram.:m.}
\end{itemize}
\begin{itemize}
\item {Proveniência:(Lat. \textunderscore tricornium\textunderscore )}
\end{itemize}
Chapéu de três bicos.
\section{Tricorpóreo}
\begin{itemize}
\item {Grp. gram.:adj.}
\end{itemize}
\begin{itemize}
\item {Proveniência:(De \textunderscore tri...\textunderscore  + \textunderscore corpóreo\textunderscore )}
\end{itemize}
Que tem três corpos.
\section{Tricorina}
\begin{itemize}
\item {Grp. gram.:f.}
\end{itemize}
Gênero de plantas liliáceas.
\section{Tricoryna}
\begin{itemize}
\item {Grp. gram.:f.}
\end{itemize}
Gênero de plantas liliáceas.
\section{Tricóspira}
\begin{itemize}
\item {Grp. gram.:f.}
\end{itemize}
Gênero de plantas, da fam. das compostas.
\section{Tricossoma}
\begin{itemize}
\item {Grp. gram.:f.}
\end{itemize}
\begin{itemize}
\item {Proveniência:(Do gr. \textunderscore trix\textunderscore , \textunderscore trikhos\textunderscore  + \textunderscore soma\textunderscore )}
\end{itemize}
Gênero de insectos lepidópteros nocturnos.
\section{Tricóstomo}
\begin{itemize}
\item {Grp. gram.:m.}
\end{itemize}
\begin{itemize}
\item {Proveniência:(Do gr. \textunderscore trix\textunderscore , \textunderscore trikhos\textunderscore  + \textunderscore stoma\textunderscore )}
\end{itemize}
Gênero de musgos.
\section{Tricotiledóneo}
\begin{itemize}
\item {Grp. gram.:adj.}
\end{itemize}
\begin{itemize}
\item {Utilização:Bot.}
\end{itemize}
\begin{itemize}
\item {Proveniência:(De \textunderscore tri...\textunderscore  + \textunderscore cotilédono\textunderscore )}
\end{itemize}
Diz-se da semente, provida de três cotilédones, como a do pinheiro negro.
\section{Tricotómico}
\begin{itemize}
\item {Grp. gram.:adj.}
\end{itemize}
Relativo á tricotomia.
\section{Tricótomo}
\begin{itemize}
\item {Grp. gram.:adj.}
\end{itemize}
\begin{itemize}
\item {Proveniência:(Do gr. \textunderscore trikha\textunderscore  + \textunderscore tome\textunderscore )}
\end{itemize}
Dividido em três.
Que se faz por divisões sucessivas de três.
\section{Tricotósia}
\begin{itemize}
\item {Grp. gram.:f.}
\end{itemize}
Gênero de orquídeas.
\section{Tricotyledóneo}
\begin{itemize}
\item {Grp. gram.:adj.}
\end{itemize}
\begin{itemize}
\item {Utilização:Bot.}
\end{itemize}
\begin{itemize}
\item {Proveniência:(De \textunderscore tri...\textunderscore  + \textunderscore cotylédono\textunderscore )}
\end{itemize}
Diz-se da semente, provida de três cotylédones, como a do pinheiro negro.
\section{Tricróico}
\begin{itemize}
\item {Grp. gram.:adj.}
\end{itemize}
\begin{itemize}
\item {Utilização:Miner.}
\end{itemize}
Diz-se de metaes que têm a qualdiade de tricroismo.
\section{Tricroísmo}
\begin{itemize}
\item {Grp. gram.:m.}
\end{itemize}
\begin{itemize}
\item {Utilização:Miner.}
\end{itemize}
\begin{itemize}
\item {Proveniência:(Do gr. \textunderscore tri\textunderscore  + \textunderscore khroa\textunderscore )}
\end{itemize}
Propriedade, que os mineraes de dois eixos de dupla refracção têm, de oferecer três côres diferentes, quando olhados em diversos sentidos.
\section{Tricróito}
\begin{itemize}
\item {Grp. gram.:adj.}
\end{itemize}
O mesmo que \textunderscore tricróico\textunderscore .
\section{Trierarcho}
\begin{itemize}
\item {fónica:co}
\end{itemize}
\begin{itemize}
\item {Grp. gram.:m.}
\end{itemize}
\begin{itemize}
\item {Proveniência:(Lat. \textunderscore trierarchus\textunderscore )}
\end{itemize}
Commandante de uma embarcação trireme, entre os antigos.
\section{Trierarco}
\begin{itemize}
\item {Grp. gram.:m.}
\end{itemize}
\begin{itemize}
\item {Proveniência:(Lat. \textunderscore trierarchus\textunderscore )}
\end{itemize}
Comandante de uma embarcação trireme, entre os antigos.
\section{Trietérico}
\begin{itemize}
\item {Grp. gram.:adj.}
\end{itemize}
\begin{itemize}
\item {Proveniência:(Lat. \textunderscore trietericus\textunderscore )}
\end{itemize}
Que comprehende três annos.
\section{Trietéride}
\begin{itemize}
\item {Grp. gram.:f.}
\end{itemize}
\begin{itemize}
\item {Proveniência:(Lat. \textunderscore trieteris\textunderscore , \textunderscore trieteridis\textunderscore )}
\end{itemize}
Triênnio, no calendário atheniense.
\section{Trietérides}
\begin{itemize}
\item {Grp. gram.:f. pl.}
\end{itemize}
\begin{itemize}
\item {Proveniência:(Gr. \textunderscore trieterides\textunderscore )}
\end{itemize}
Festas ou orgias nocturnas que, de três em três annos, se celebravam em honra de Baccho.
\section{Trifacial}
\begin{itemize}
\item {Grp. gram.:adj.}
\end{itemize}
\begin{itemize}
\item {Utilização:Anat.}
\end{itemize}
\begin{itemize}
\item {Grp. gram.:M.}
\end{itemize}
\begin{itemize}
\item {Proveniência:(De \textunderscore tri...\textunderscore  + \textunderscore facial\textunderscore )}
\end{itemize}
Diz-se de um nervo, cujos três ramos principaes se distribuem pela face.
Nervo trifacial.
\section{Trifauce}
\begin{itemize}
\item {Grp. gram.:adj.}
\end{itemize}
\begin{itemize}
\item {Utilização:Poét.}
\end{itemize}
\begin{itemize}
\item {Proveniência:(Lat. \textunderscore trifaux\textunderscore )}
\end{itemize}
Que tem três fauces.
\section{Triférrico}
\begin{itemize}
\item {Grp. gram.:adj.}
\end{itemize}
\begin{itemize}
\item {Utilização:Chím.}
\end{itemize}
\begin{itemize}
\item {Proveniência:(De \textunderscore tri...\textunderscore  + \textunderscore férrico\textunderscore )}
\end{itemize}
Diz-se de um sal férrico, que tem três vezes tanta base como o sal neutro correspondente.
\section{Triferrina}
\begin{itemize}
\item {Grp. gram.:f.}
\end{itemize}
\begin{itemize}
\item {Utilização:Med.}
\end{itemize}
Medicamente, usado como ferruginoso.
(Cp. \textunderscore triférrico\textunderscore )
\section{Triferroso}
\begin{itemize}
\item {Grp. gram.:adj.}
\end{itemize}
\begin{itemize}
\item {Utilização:Chím.}
\end{itemize}
\begin{itemize}
\item {Proveniência:(De \textunderscore tri...\textunderscore  + \textunderscore ferroso\textunderscore )}
\end{itemize}
Diz-se de um sal ferroso, que tem tríplice quantidade de base da do sal neutro correspondente.
\section{Trífido}
\begin{itemize}
\item {Grp. gram.:adj.}
\end{itemize}
\begin{itemize}
\item {Proveniência:(Lat. \textunderscore trifidus\textunderscore )}
\end{itemize}
Dividido em três; tríplice.
\section{Trifloro}
\begin{itemize}
\item {Grp. gram.:adj.}
\end{itemize}
\begin{itemize}
\item {Utilização:Poét.}
\end{itemize}
\begin{itemize}
\item {Proveniência:(Do lat. \textunderscore tres\textunderscore  + \textunderscore flos\textunderscore , \textunderscore floris\textunderscore )}
\end{itemize}
Que tem três flôres.
\section{Trifólia}
\begin{itemize}
\item {Grp. gram.:adj. f.}
\end{itemize}
Diz-se da charrua de três quinas ou arestas.
(Cp. \textunderscore trifólio\textunderscore )
\section{Trifoliáceas}
\begin{itemize}
\item {Grp. gram.:f. pl.}
\end{itemize}
\begin{itemize}
\item {Proveniência:(De \textunderscore trifólio\textunderscore )}
\end{itemize}
Tribo de plantas, que tem por typo o trevo.
\section{Trifoliado}
\begin{itemize}
\item {Grp. gram.:adj.}
\end{itemize}
\begin{itemize}
\item {Proveniência:(Do lat. \textunderscore tres\textunderscore  + \textunderscore folium\textunderscore )}
\end{itemize}
Que tem três fôlhas.
\section{Trifólio}
\begin{itemize}
\item {Grp. gram.:m.}
\end{itemize}
\begin{itemize}
\item {Grp. gram.:Adj.}
\end{itemize}
\begin{itemize}
\item {Utilização:Prov.}
\end{itemize}
\begin{itemize}
\item {Proveniência:(Lat. \textunderscore trifolium\textunderscore )}
\end{itemize}
Trevo.
Ornato em fórma de trevo.
Diz-se do arado, cujo ferro tem três quinas ou arestas.
\section{Trifoliose}
\begin{itemize}
\item {Grp. gram.:f.}
\end{itemize}
\begin{itemize}
\item {Proveniência:(Do lat. \textunderscore trifolium\textunderscore )}
\end{itemize}
Envenenamento do cavallo, causado pelo trevo hýbrido.
\section{Triforfai}
\begin{itemize}
\item {Grp. gram.:m.}
\end{itemize}
\begin{itemize}
\item {Utilização:T. do Fundão}
\end{itemize}
\begin{itemize}
\item {Proveniência:(Do ingl. \textunderscore tree\textunderscore  + \textunderscore four\textunderscore  + \textunderscore five\textunderscore . Quando as tropas inglezas estiveram no Fundão, em princípios do séc. XIX, os soldados eram castigados com varadas, que se contavam em voz alta: \textunderscore ...three, four, five...\textunderscore , três, quatro, cinco. Daqui a formação pop. \textunderscore triforfai\textunderscore )}
\end{itemize}
Sova, pancadaria.
\section{Trifório}
\begin{itemize}
\item {Grp. gram.:m.}
\end{itemize}
Galeria estreita, sôbre os arcos da nave central, nas igrejas ogivaes.
Galeria, sôbre as naves lateraes das igrejas.
(Cp. lat. \textunderscore triforis\textunderscore )
\section{Triforme}
\begin{itemize}
\item {Grp. gram.:adj.}
\end{itemize}
\begin{itemize}
\item {Proveniência:(Lat. \textunderscore triformis\textunderscore )}
\end{itemize}
Que tem três fórmas.
\section{Trifurcação}
\begin{itemize}
\item {Grp. gram.:f.}
\end{itemize}
Acto ou effeito de \textunderscore trifurcar\textunderscore .
\section{Trifurcar}
\begin{itemize}
\item {Grp. gram.:v. t.}
\end{itemize}
\begin{itemize}
\item {Proveniência:(Do lat. \textunderscore trifurcus\textunderscore )}
\end{itemize}
Dividir em três partes ou ramos.
\section{Trifusa}
\begin{itemize}
\item {Grp. gram.:f.}
\end{itemize}
\begin{itemize}
\item {Utilização:Mús.}
\end{itemize}
\begin{itemize}
\item {Proveniência:(De \textunderscore tri...\textunderscore  + \textunderscore fusa\textunderscore )}
\end{itemize}
Figura de nota, que raramente se emprega e que vale metade de uma semifusa. Cf. E. Vieira, \textunderscore Diccion. Mus.\textunderscore 
\section{Triga}
\begin{itemize}
\item {Grp. gram.:f.}
\end{itemize}
Acto ou effeito de trigar.
Pressa.
\section{Triga}
\begin{itemize}
\item {Grp. gram.:f.}
\end{itemize}
\begin{itemize}
\item {Utilização:Ant.}
\end{itemize}
\begin{itemize}
\item {Proveniência:(Lat. \textunderscore triga\textunderscore )}
\end{itemize}
Carro, puxado por três cavallos.
\section{Triga}
\begin{itemize}
\item {Grp. gram.:adj. f.}
\end{itemize}
\begin{itemize}
\item {Utilização:Prov.}
\end{itemize}
\begin{itemize}
\item {Utilização:beir.}
\end{itemize}
Diz-se de farinha de trigo.
\section{Trigada}
\begin{itemize}
\item {Grp. gram.:f.}
\end{itemize}
\begin{itemize}
\item {Utilização:Prov.}
\end{itemize}
\begin{itemize}
\item {Utilização:beir.}
\end{itemize}
Seara de trigo.
\section{Trigado}
\begin{itemize}
\item {Grp. gram.:adj.}
\end{itemize}
O mesmo que \textunderscore atrigado\textunderscore .
\section{Trigães}
\begin{itemize}
\item {Grp. gram.:f.}
\end{itemize}
Variedade antigo de pêra portuguesa. Cf. Rui Fernandes, \textunderscore Inéd. da Hist. Port.\textunderscore 
\section{Trigal}
\begin{itemize}
\item {Grp. gram.:m.}
\end{itemize}
\begin{itemize}
\item {Grp. gram.:Adj.}
\end{itemize}
\begin{itemize}
\item {Proveniência:(Do b. lat. \textunderscore triticalis\textunderscore )}
\end{itemize}
Campo de trigo; seara.
Diz-se de uma espécie de cereja vermelha e um pouco amargosa.
\section{Trigamia}
\begin{itemize}
\item {Grp. gram.:f.}
\end{itemize}
\begin{itemize}
\item {Proveniência:(Lat. \textunderscore trigamia\textunderscore )}
\end{itemize}
Estado ou crime daquelle que é trigamo.
\section{Trigamilha}
\begin{itemize}
\item {Grp. gram.:f.}
\end{itemize}
\begin{itemize}
\item {Proveniência:(De \textunderscore trigo\textunderscore  + \textunderscore milho\textunderscore )}
\end{itemize}
Pão, feito de farinha de trigo e de farinha de milho.
\section{Triexaédro}
\begin{itemize}
\item {Grp. gram.:adj.}
\end{itemize}
\begin{itemize}
\item {Utilização:Miner.}
\end{itemize}
\begin{itemize}
\item {Proveniência:(De \textunderscore tri...\textunderscore  + \textunderscore hexaédro\textunderscore )}
\end{itemize}
Diz-se do sólido, resultante da reunião de dois romboédros iguaes, colocados de modo simétrico, como se observa em alguns carbonatos de cal.
\section{Triginia}
\begin{itemize}
\item {Grp. gram.:f.}
\end{itemize}
\begin{itemize}
\item {Utilização:Bot.}
\end{itemize}
Qualidade de trigino, ou conjunto das plantas, cujas flôres têm três pistilos.
\section{Trigínico}
\begin{itemize}
\item {Grp. gram.:adj.}
\end{itemize}
Relativo á triginia.
O mesmo que \textunderscore trigínio\textunderscore .
\section{Trigínio}
\begin{itemize}
\item {Grp. gram.:adj.}
\end{itemize}
\begin{itemize}
\item {Utilização:Bot.}
\end{itemize}
\begin{itemize}
\item {Proveniência:(Do gr. \textunderscore tri\textunderscore  + \textunderscore gune\textunderscore )}
\end{itemize}
Que tem três pistilos.
\section{Trígino}
\begin{itemize}
\item {Grp. gram.:adj.}
\end{itemize}
O mesmo ou melhór que \textunderscore trigínio\textunderscore .
\section{Trigonocarpo}
\begin{itemize}
\item {Grp. gram.:adj.}
\end{itemize}
\begin{itemize}
\item {Utilização:Bot.}
\end{itemize}
\begin{itemize}
\item {Proveniência:(Do gr. \textunderscore trigonos\textunderscore  + \textunderscore karpos\textunderscore )}
\end{itemize}
Diz-se da planta, cujos frutos são triangulares.
\section{Trigonocefalia}
\begin{itemize}
\item {Grp. gram.:f.}
\end{itemize}
Estado ou qualidade de trigonocéfalo.
\section{Trigonocéfalo}
\begin{itemize}
\item {Grp. gram.:m.  e  adj.}
\end{itemize}
\begin{itemize}
\item {Proveniência:(Do gr. \textunderscore trigonos\textunderscore  + \textunderscore kephale\textunderscore )}
\end{itemize}
O que tem cabeça triangular.
\section{Trigonocephalia}
\begin{itemize}
\item {Grp. gram.:f.}
\end{itemize}
Estado ou qualidade de trigonocéphalo.
\section{Trigonocéphalo}
\begin{itemize}
\item {Grp. gram.:m.  e  adj.}
\end{itemize}
\begin{itemize}
\item {Proveniência:(Do gr. \textunderscore trigonos\textunderscore  + \textunderscore kephale\textunderscore )}
\end{itemize}
O que tem cabeça triangular.
\section{Trigonódoro}
\begin{itemize}
\item {Grp. gram.:m.}
\end{itemize}
\begin{itemize}
\item {Proveniência:(Do gr. \textunderscore trigonos\textunderscore  + \textunderscore dere\textunderscore )}
\end{itemize}
Gênero de insectos hymenópteros.
\section{Trigonóforo}
\begin{itemize}
\item {Grp. gram.:m.}
\end{itemize}
\begin{itemize}
\item {Proveniência:(Do gr. \textunderscore trigonos\textunderscore  + \textunderscore phoros\textunderscore )}
\end{itemize}
Gênero de insectos coleópteros pentâmeros.
\section{Trigonometria}
\begin{itemize}
\item {Grp. gram.:f.}
\end{itemize}
\begin{itemize}
\item {Proveniência:(Gr. \textunderscore trigonometria\textunderscore )}
\end{itemize}
Sciência, que tem por objecto determinar pelo cálculo os ângulos e os lados dos triângulos, partindo de certos números dados.
\section{Trigonometricamente}
\begin{itemize}
\item {Grp. gram.:adv.}
\end{itemize}
De modo trigonométrico.
Segundo as regras da Trigonometria.
\section{Trigonométrico}
\begin{itemize}
\item {Grp. gram.:adj.}
\end{itemize}
Relativo á Trigonometria.
Conforme as regras da Trigonometria.
\section{Trigonóphoro}
\begin{itemize}
\item {Grp. gram.:m.}
\end{itemize}
\begin{itemize}
\item {Proveniência:(Do gr. \textunderscore trigonos\textunderscore  + \textunderscore phoros\textunderscore )}
\end{itemize}
Gênero de insectos coleópteros pentâmeros.
\section{Trigonosomo}
\begin{itemize}
\item {fónica:sô}
\end{itemize}
\begin{itemize}
\item {Grp. gram.:m.}
\end{itemize}
\begin{itemize}
\item {Proveniência:(Do gr. \textunderscore trigonos\textunderscore  + \textunderscore soma\textunderscore )}
\end{itemize}
Gênero de insectos hemípteros.
\section{Trigonospermo}
\begin{itemize}
\item {Grp. gram.:m.}
\end{itemize}
\begin{itemize}
\item {Proveniência:(Do gr. \textunderscore trigonos\textunderscore  + \textunderscore sperma\textunderscore )}
\end{itemize}
Gênero de plantas, da fam. das compostas.
\section{Trigonossomo}
\begin{itemize}
\item {Grp. gram.:m.}
\end{itemize}
\begin{itemize}
\item {Proveniência:(Do gr. \textunderscore trigonos\textunderscore  + \textunderscore soma\textunderscore )}
\end{itemize}
Gênero de insectos hemípteros.
\section{Trigo-preto}
\begin{itemize}
\item {Grp. gram.:m.}
\end{itemize}
Planta polygónea do Brasil, (\textunderscore polygonum fagopyrum\textunderscore , Lin.).
\section{Trigosamente}
\begin{itemize}
\item {Grp. gram.:adv.}
\end{itemize}
\begin{itemize}
\item {Utilização:Ant.}
\end{itemize}
De modo trigoso; apressadamente.
\section{Trigo-sarraceno}
\begin{itemize}
\item {Grp. gram.:m.}
\end{itemize}
O mesmo que \textunderscore trigo-preto\textunderscore .
\section{Trigoso}
\begin{itemize}
\item {Grp. gram.:adj.}
\end{itemize}
\begin{itemize}
\item {Utilização:Ant.}
\end{itemize}
\begin{itemize}
\item {Proveniência:(De \textunderscore triga\textunderscore ^1)}
\end{itemize}
O mesmo que \textunderscore apressado\textunderscore .
\section{Trigrama}
\begin{itemize}
\item {Grp. gram.:m.}
\end{itemize}
\begin{itemize}
\item {Proveniência:(Do gr. \textunderscore tri\textunderscore  + \textunderscore gramma\textunderscore )}
\end{itemize}
Palavra de três letras.
Sinal, composto de três caracteres reunidos.
\section{Trigramma}
\begin{itemize}
\item {Grp. gram.:m.}
\end{itemize}
\begin{itemize}
\item {Proveniência:(Do gr. \textunderscore tri\textunderscore  + \textunderscore gramma\textunderscore )}
\end{itemize}
Palavra de três letras.
Sinal, composto de três caracteres reunidos.
\section{Trigueira}
\begin{itemize}
\item {Grp. gram.:f.}
\end{itemize}
\begin{itemize}
\item {Utilização:Prov.}
\end{itemize}
\begin{itemize}
\item {Utilização:trasm.}
\end{itemize}
Vendedeira de pão de trigo.
\section{Trigueirão}
\begin{itemize}
\item {Grp. gram.:m.}
\end{itemize}
\begin{itemize}
\item {Proveniência:(De \textunderscore trigueiro\textunderscore )}
\end{itemize}
Pássaro conirostro, (\textunderscore miliaria europaea\textunderscore , Swains, ou \textunderscore emberiza millicria\textunderscore , Lin.).
\section{Trigueiro}
\begin{itemize}
\item {Grp. gram.:adj.}
\end{itemize}
\begin{itemize}
\item {Grp. gram.:M.}
\end{itemize}
\begin{itemize}
\item {Proveniência:(De \textunderscore trigo\textunderscore )}
\end{itemize}
Que tem a côr do trigo maduro.
Moreno.
Pássaro conirostro, espécie de verdelhão.
Indivíduo trigueiro.
\section{Triguenho}
\begin{itemize}
\item {Grp. gram.:adj.}
\end{itemize}
Relativo ou semelhante ao trigo.
Trigueiro.
(Cast. \textunderscore trigueño\textunderscore )
\section{Triguera}
\begin{itemize}
\item {Grp. gram.:f.}
\end{itemize}
Gênero de plantas solanáceas.
(Cast. \textunderscore triguera\textunderscore , alpiste)
\section{Trigynia}
\begin{itemize}
\item {Grp. gram.:f.}
\end{itemize}
\begin{itemize}
\item {Utilização:Bot.}
\end{itemize}
Qualidade de trigyno, ou conjunto das plantas, cujas flôres têm três pistillos.
\section{Trigýnico}
\begin{itemize}
\item {Grp. gram.:adj.}
\end{itemize}
Relativo á trigynia.
O mesmo que \textunderscore trigýnio\textunderscore .
\section{Trigýnio}
\begin{itemize}
\item {Grp. gram.:adj.}
\end{itemize}
\begin{itemize}
\item {Utilização:Bot.}
\end{itemize}
\begin{itemize}
\item {Proveniência:(Do gr. \textunderscore tri\textunderscore  + \textunderscore gune\textunderscore )}
\end{itemize}
Que tem três pistillos.
\section{Trígyno}
\begin{itemize}
\item {Grp. gram.:adj.}
\end{itemize}
O mesmo ou melhór que \textunderscore trigýnio\textunderscore .
\section{Trihexaédro}
\begin{itemize}
\item {Grp. gram.:adj.}
\end{itemize}
\begin{itemize}
\item {Utilização:Miner.}
\end{itemize}
\begin{itemize}
\item {Proveniência:(De \textunderscore tri...\textunderscore  + \textunderscore hexaédro\textunderscore )}
\end{itemize}
Diz-se do sólido, resultante da reunião de dois rhomboédros iguaes, collocados de modo symétrico, como se observa em alguns carbonatos de cal.
\section{Trihýdrico}
\begin{itemize}
\item {Grp. gram.:adj.}
\end{itemize}
\begin{itemize}
\item {Utilização:Chím.}
\end{itemize}
\begin{itemize}
\item {Proveniência:(De \textunderscore tri...\textunderscore  + \textunderscore hýdrico\textunderscore )}
\end{itemize}
Diz-se de um composto, que contém três proporções de hydrogênio.
\section{Triídrico}
\begin{itemize}
\item {Grp. gram.:adj.}
\end{itemize}
\begin{itemize}
\item {Utilização:Chím.}
\end{itemize}
\begin{itemize}
\item {Proveniência:(De \textunderscore tri...\textunderscore  + \textunderscore hýdrico\textunderscore )}
\end{itemize}
Diz-se de um composto, que contém três proporções de hydrogênio.
\section{Trijugado}
\begin{itemize}
\item {Grp. gram.:adj.}
\end{itemize}
\begin{itemize}
\item {Utilização:Bot.}
\end{itemize}
\begin{itemize}
\item {Proveniência:(Do lat. \textunderscore trijugus\textunderscore )}
\end{itemize}
Composto de três pares de foliólos.
\section{Trilado}
\begin{itemize}
\item {Grp. gram.:m.}
\end{itemize}
\begin{itemize}
\item {Proveniência:(De \textunderscore trilar\textunderscore )}
\end{itemize}
O mesmo que \textunderscore trilo\textunderscore .
\section{Trilar}
\begin{itemize}
\item {Grp. gram.:v. t.  e  i.}
\end{itemize}
Cantar, formando trilos.
Trinar; gorgear.
\section{Trilogia}
\begin{itemize}
\item {Grp. gram.:f.}
\end{itemize}
\begin{itemize}
\item {Utilização:Ext.}
\end{itemize}
\begin{itemize}
\item {Proveniência:(Gr. \textunderscore trilogia\textunderscore )}
\end{itemize}
Poema dramático, composto de três tragédias e destinado aos concursos, nos jogos solennes da Grécia.
Peça theatral ou literária dividida em três partes; tríade.
\section{Trilógico}
\begin{itemize}
\item {Grp. gram.:adj.}
\end{itemize}
Relativo á trilogia.
\section{Trílogo}
\begin{itemize}
\item {Grp. gram.:m.}
\end{itemize}
\begin{itemize}
\item {Utilização:Des.}
\end{itemize}
\begin{itemize}
\item {Proveniência:(Do gr. \textunderscore tri\textunderscore  + \textunderscore logos\textunderscore )}
\end{itemize}
Conversação entre três pessôas.
\section{Trilongo}
\begin{itemize}
\item {Grp. gram.:m.  e  adj.}
\end{itemize}
\begin{itemize}
\item {Proveniência:(Lat. \textunderscore trilongus\textunderscore )}
\end{itemize}
Diz-se do verso, que tem três sýllabas longas.
\section{Trimaculado}
\begin{itemize}
\item {Grp. gram.:adj.}
\end{itemize}
\begin{itemize}
\item {Proveniência:(De \textunderscore tri...\textunderscore  + \textunderscore maculado\textunderscore )}
\end{itemize}
Que tem três malhas ou manchas.
\section{Trimbolar}
\begin{itemize}
\item {Grp. gram.:v. i.}
\end{itemize}
\begin{itemize}
\item {Utilização:Gír.}
\end{itemize}
Guiar uma carruagem.
\section{Trimbolim}
\begin{itemize}
\item {Grp. gram.:m.}
\end{itemize}
\begin{itemize}
\item {Utilização:Prov.}
\end{itemize}
Carruagem velha e desconjuntada.
(Talvez se relacione com o ingl. \textunderscore tilbury\textunderscore )
\section{Trimembre}
\begin{itemize}
\item {Grp. gram.:adj.}
\end{itemize}
\begin{itemize}
\item {Proveniência:(Lat. \textunderscore trimembris\textunderscore )}
\end{itemize}
Que tem três membros.
\section{Trimensal}
\begin{itemize}
\item {Grp. gram.:adj.}
\end{itemize}
\begin{itemize}
\item {Proveniência:(Do lat. \textunderscore trimensis\textunderscore )}
\end{itemize}
Que dura três meses.
Que se realiza de três em três meses.
\section{Triméria}
\begin{itemize}
\item {Grp. gram.:f.}
\end{itemize}
\begin{itemize}
\item {Proveniência:(Do gr. \textunderscore tremeres\textunderscore )}
\end{itemize}
Gênero de plantas homalíneas.
\section{Trímero}
\begin{itemize}
\item {Grp. gram.:adj.}
\end{itemize}
\begin{itemize}
\item {Grp. gram.:M. pl.}
\end{itemize}
\begin{itemize}
\item {Utilização:Zool.}
\end{itemize}
\begin{itemize}
\item {Proveniência:(Gr. \textunderscore trimeres\textunderscore )}
\end{itemize}
Dividido em três partes.
Quarta divisão da ordem dos coleópteros, que comprehende os gêneros que têm os tarsos divididos em três partes.
\section{Trimestral}
\begin{itemize}
\item {Grp. gram.:adj.}
\end{itemize}
\begin{itemize}
\item {Proveniência:(De \textunderscore trimestre\textunderscore )}
\end{itemize}
O mesmo que \textunderscore trimensal\textunderscore .
\section{Trimestralmente}
\begin{itemize}
\item {Grp. gram.:adv.}
\end{itemize}
\begin{itemize}
\item {Proveniência:(De \textunderscore trimestral\textunderscore )}
\end{itemize}
De três em três meses.
\section{Trimestre}
\begin{itemize}
\item {Grp. gram.:m.}
\end{itemize}
\begin{itemize}
\item {Grp. gram.:Adj.}
\end{itemize}
\begin{itemize}
\item {Proveniência:(Lat. \textunderscore trimestris\textunderscore )}
\end{itemize}
Espaço de três meses.
Aquillo que se tem de pagar, no fim de cada período de três meses.
Trimensal.
\section{Trímetra}
\begin{itemize}
\item {Grp. gram.:f.}
\end{itemize}
\begin{itemize}
\item {Proveniência:(Do gr. \textunderscore treis\textunderscore  + \textunderscore metra\textunderscore )}
\end{itemize}
Gênero de plantas, da fam. das compostas.
\section{Trimétrico}
\begin{itemize}
\item {Grp. gram.:adj.}
\end{itemize}
\begin{itemize}
\item {Proveniência:(De \textunderscore trímetro\textunderscore )}
\end{itemize}
Que se refere a três medidas differentes.
\section{Trímetro}
\begin{itemize}
\item {Grp. gram.:m.  e  adj.}
\end{itemize}
\begin{itemize}
\item {Proveniência:(Lat. \textunderscore trimetrus\textunderscore )}
\end{itemize}
Diz-se do verso de três pés.
\section{Trimódio}
\begin{itemize}
\item {Grp. gram.:m.}
\end{itemize}
\begin{itemize}
\item {Proveniência:(Lat. \textunderscore trimodium\textunderscore )}
\end{itemize}
Vaso romano, com a capacidade de três módios.
\section{Trimorfia}
\begin{itemize}
\item {Grp. gram.:f.}
\end{itemize}
O mesmo que \textunderscore trimorfismo\textunderscore .
\section{Trimorfismo}
\begin{itemize}
\item {Grp. gram.:m.}
\end{itemize}
Qualidade ou estado do que é trimorfo.
\section{Trimorfo}
\begin{itemize}
\item {Grp. gram.:adj.}
\end{itemize}
\begin{itemize}
\item {Proveniência:(Gr. \textunderscore trimorphos\textunderscore )}
\end{itemize}
Diz-se de uma substância, que se póde cristalizar de três fórmas diversas.
Diz-se das flôres, que tem estames de três tamanhos diferentes.
\section{Trimorphia}
\begin{itemize}
\item {Grp. gram.:f.}
\end{itemize}
O mesmo que \textunderscore trimorphismo\textunderscore .
\section{Trimorphismo}
\begin{itemize}
\item {Grp. gram.:m.}
\end{itemize}
Qualidade ou estado do que é trimorpho.
\section{Trimorpho}
\begin{itemize}
\item {Grp. gram.:adj.}
\end{itemize}
\begin{itemize}
\item {Proveniência:(Gr. \textunderscore trimorphos\textunderscore )}
\end{itemize}
Diz-se de uma substância, que se póde crystallizar de três fórmas diversas.
Diz-se das flôres, que tem estames de três tamanhos differentes.
\section{Trimurti}
\begin{itemize}
\item {Grp. gram.:f.}
\end{itemize}
Trindade dos Índios.
(Sanscr. \textunderscore trimurti\textunderscore )
\section{Trinácrio}
\begin{itemize}
\item {Grp. gram.:m.  e  adj.}
\end{itemize}
\begin{itemize}
\item {Proveniência:(Lat. \textunderscore trinacrius\textunderscore )}
\end{itemize}
O mesmo que \textunderscore siciliano\textunderscore .
\section{Trinado}
\begin{itemize}
\item {Grp. gram.:m.}
\end{itemize}
\begin{itemize}
\item {Proveniência:(De \textunderscore trinar\textunderscore )}
\end{itemize}
O mesmo que \textunderscore trino\textunderscore ^2.
\section{Trinador}
\begin{itemize}
\item {Grp. gram.:adj.}
\end{itemize}
Que trina. Cf. Th. Ribeiro, \textunderscore Jornadas\textunderscore , I, 368.
\section{Trinalidade}
\begin{itemize}
\item {Grp. gram.:f.}
\end{itemize}
Estado do que é trino^1.
\section{Trinar}
\begin{itemize}
\item {Grp. gram.:v. t.}
\end{itemize}
\begin{itemize}
\item {Grp. gram.:V. i.}
\end{itemize}
Exprimir ou cantar com trinos.
Soltar trinos.
Ferir tremulamente as cordas de um instrumento.
\section{Trinca}
\begin{itemize}
\item {Grp. gram.:f.}
\end{itemize}
\begin{itemize}
\item {Utilização:Náut.}
\end{itemize}
Reunião de três coisas análogas.
Conjunto de três cartas de jôgo, do mesmo valor.
Espécie de cabo, que dá três voltas.
Nome de outros cabos.
(Cp. cast. \textunderscore trinca\textunderscore )
\section{Trinca-cevaca}
\begin{itemize}
\item {Grp. gram.:f.}
\end{itemize}
Espécie de jôgo popular.
\section{Trincadeira}
\begin{itemize}
\item {Grp. gram.:f.}
\end{itemize}
\begin{itemize}
\item {Utilização:Pop.}
\end{itemize}
Casta de uva preta, de pellícula resistente.
Espécie de uva branca.
Acto de trincar ou comer.
Aquillo que se come.
Pechincha.
\section{Trincadeira-branca}
\begin{itemize}
\item {Grp. gram.:f.}
\end{itemize}
Casta de uva do Cartaxo. Cf. \textunderscore Rev. Agron.\textunderscore , I, 18.
\section{Trincadente}
\begin{itemize}
\item {Grp. gram.:f.  e  adj.}
\end{itemize}
\begin{itemize}
\item {Proveniência:(De \textunderscore trincar\textunderscore  + \textunderscore dente\textunderscore )}
\end{itemize}
Espécie de uva branca, o mesmo que \textunderscore agudenho\textunderscore .
\section{Trincadentes}
\begin{itemize}
\item {Grp. gram.:f.  e  adj.}
\end{itemize}
O mesmo que \textunderscore trincadente\textunderscore .
\section{Trincadura}
\begin{itemize}
\item {Grp. gram.:f.}
\end{itemize}
Espécie de lancha com dois mastros, usada pelos pescadores da costa da Biscaia.
\section{Trifânio}
\begin{itemize}
\item {Grp. gram.:m.}
\end{itemize}
O mesmo ou melhór que \textunderscore trífano\textunderscore .
\section{Trifanito}
\begin{itemize}
\item {Grp. gram.:m.}
\end{itemize}
\begin{itemize}
\item {Utilização:Miner.}
\end{itemize}
\begin{itemize}
\item {Proveniência:(De \textunderscore trífano\textunderscore )}
\end{itemize}
Mineral côr de rosa e de fractura xistosa.
\section{Trífano}
\begin{itemize}
\item {Grp. gram.:m.}
\end{itemize}
\begin{itemize}
\item {Utilização:Miner.}
\end{itemize}
\begin{itemize}
\item {Proveniência:(Do gr. \textunderscore treis\textunderscore  + \textunderscore phainesthai\textunderscore )}
\end{itemize}
Variedade de feldspato de litina.
\section{Trifármaco}
\begin{itemize}
\item {Grp. gram.:m.}
\end{itemize}
\begin{itemize}
\item {Proveniência:(Do gr. \textunderscore treis\textunderscore  + \textunderscore pharmakon\textunderscore )}
\end{itemize}
Antigo medicamento, composto de três drogas.
\section{Trifélia}
\begin{itemize}
\item {Grp. gram.:f.}
\end{itemize}
Gênero de plantas mirtáceas.
\section{Trifenina}
\begin{itemize}
\item {Grp. gram.:f.}
\end{itemize}
Medicamento antineurálgico e antipirético.
\section{Trifonia}
\begin{itemize}
\item {Grp. gram.:f.}
\end{itemize}
\begin{itemize}
\item {Utilização:Mús.}
\end{itemize}
\begin{itemize}
\item {Proveniência:(Do gr. \textunderscore treis\textunderscore  + \textunderscore phone\textunderscore )}
\end{itemize}
Contraponto a três vozes, na Idade-Média.
\section{Trincheirar}
\begin{itemize}
\item {Grp. gram.:v. t.  e  p.}
\end{itemize}
O mesmo que \textunderscore entrincheirar\textunderscore :«\textunderscore ...devia mandar trincheirar o campo...\textunderscore »\textunderscore Jornada de África\textunderscore , c. v.
\section{Trincheiro}
\begin{itemize}
\item {Grp. gram.:m.}
\end{itemize}
Socalco ou degrau, feito numa trincheira ou barreira, para facilitar a subida.
\section{Trinchete}
\begin{itemize}
\item {fónica:chê}
\end{itemize}
\begin{itemize}
\item {Grp. gram.:m.}
\end{itemize}
\begin{itemize}
\item {Proveniência:(Fr. \textunderscore trinchet\textunderscore )}
\end{itemize}
Faca de sapateiro, terminada em faceta e mais ou menos curva.
\section{Trincho}
\begin{itemize}
\item {Grp. gram.:m.}
\end{itemize}
\begin{itemize}
\item {Utilização:Fig.}
\end{itemize}
\begin{itemize}
\item {Proveniência:(De \textunderscore trinchar\textunderscore )}
\end{itemize}
Travéssa ou prato grande, em que se trincham viandas.
Acto ou modo de trinchar; lado da peça de vianda, por onde se trincha mais facilmente.
Tábua, em que assenta a massa do queijo, apertada pelo cincho.
Peça nas prensas de fuso fixo.
Faca, o mesmo que \textunderscore trincha\textunderscore .
O meio mais fácil de resolver uma difficuldade.
\section{Trinco}
\begin{itemize}
\item {Grp. gram.:m.}
\end{itemize}
\begin{itemize}
\item {Proveniência:(De \textunderscore trincar\textunderscore )}
\end{itemize}
Espécie de pequena tranca, com que se fecham porta, e que se levanta com auxílio de chave, cordão ou aldrava.
Espécie de fechadura, por onde entra a chave, que levanta essa tranqueta.
Estalido com os dedos; som análogo a êsse estalido.
\section{Trincolejar}
\begin{itemize}
\item {Grp. gram.:v. i.}
\end{itemize}
O mesmo que \textunderscore telintar\textunderscore :«\textunderscore ouviste-me não sei quê trincolejar na algibeira.\textunderscore »J. de Deus.
\section{Trincolejo}
\begin{itemize}
\item {Grp. gram.:m.}
\end{itemize}
Acto de trincolejar.
\section{Trincolhos-brincolhos}
\begin{itemize}
\item {fónica:cô}
\end{itemize}
\begin{itemize}
\item {Grp. gram.:m. Pl.}
\end{itemize}
Brinquedos de criança.
\section{Trindade}
\begin{itemize}
\item {Grp. gram.:f.}
\end{itemize}
\begin{itemize}
\item {Utilização:Ext.}
\end{itemize}
\begin{itemize}
\item {Utilização:Fig.}
\end{itemize}
\begin{itemize}
\item {Grp. gram.:Pl.}
\end{itemize}
\begin{itemize}
\item {Proveniência:(Lat. \textunderscore trinitas\textunderscore )}
\end{itemize}
Um só Deus em três pessôas distintas.
Domingo, immediato ao de Pentecostes.
Divindade tríplice, nas religiões pagans.
Designação de uma Ordem religiosa: \textunderscore os frades da Trindade\textunderscore , ou frades trinos.
Grupo de três pessôas ou três coisas análogas.
O toque das ave-marias.
Hora do toque das ave-marias; tardinha.
\section{Trinervado}
\begin{itemize}
\item {Grp. gram.:adj.}
\end{itemize}
O mesmo que \textunderscore trinérveo\textunderscore .
\section{Trinérveo}
\begin{itemize}
\item {Grp. gram.:adj.}
\end{itemize}
\begin{itemize}
\item {Utilização:Bot.}
\end{itemize}
\begin{itemize}
\item {Proveniência:(De \textunderscore tri...\textunderscore  + \textunderscore nérveo\textunderscore )}
\end{itemize}
Que tem três nervos ou nervuras.
\section{Trineta}
\begin{itemize}
\item {Grp. gram.:f.}
\end{itemize}
Filha do bisneto ou da bisneta.
(de \textunderscore tri...\textunderscore  + \textunderscore neta\textunderscore )
\section{Trineto}
\begin{itemize}
\item {Grp. gram.:m.}
\end{itemize}
\begin{itemize}
\item {Proveniência:(De \textunderscore tri...\textunderscore  + \textunderscore neto\textunderscore )}
\end{itemize}
Filho do bisneto ou da bisneta.
\section{Trinfar}
\begin{itemize}
\item {Grp. gram.:v. i.}
\end{itemize}
\begin{itemize}
\item {Grp. gram.:M.}
\end{itemize}
Soltar a voz, (falando-se da andorinha).
A voz da andorinha. Cf. Castilho, \textunderscore Fastos\textunderscore , III, 324.
\section{Tringalha}
\begin{itemize}
\item {Grp. gram.:f.}
\end{itemize}
\begin{itemize}
\item {Utilização:Prov.}
\end{itemize}
O mesmo que \textunderscore pinguelete\textunderscore .
Pênis de criança.
\section{Tringalheira}
\begin{itemize}
\item {Grp. gram.:f.}
\end{itemize}
\begin{itemize}
\item {Utilização:Prov.}
\end{itemize}
\begin{itemize}
\item {Utilização:minh.}
\end{itemize}
\begin{itemize}
\item {Proveniência:(De \textunderscore tringalho\textunderscore )}
\end{itemize}
Pessôa esfarrapada.
\section{Tringalho}
\begin{itemize}
\item {Grp. gram.:m.}
\end{itemize}
\begin{itemize}
\item {Utilização:Prov.}
\end{itemize}
\begin{itemize}
\item {Utilização:minh.}
\end{itemize}
O mesmo que \textunderscore farrapo\textunderscore .
\section{Trínia}
\begin{itemize}
\item {Grp. gram.:f.}
\end{itemize}
\begin{itemize}
\item {Proveniência:(De \textunderscore Trinius\textunderscore , n. p.)}
\end{itemize}
Gênero de plantas umbellíferas.
\section{Trinitário}
\begin{itemize}
\item {Grp. gram.:adj.}
\end{itemize}
\begin{itemize}
\item {Grp. gram.:M.}
\end{itemize}
\begin{itemize}
\item {Proveniência:(Do lat. \textunderscore trinitas\textunderscore )}
\end{itemize}
Pertencente á Ordem religiosa da Trindade.

Frade trinitário.
\section{Trinitrado}
\begin{itemize}
\item {Grp. gram.:adj.}
\end{itemize}
\begin{itemize}
\item {Proveniência:(De \textunderscore tri...\textunderscore  + \textunderscore nitrado\textunderscore )}
\end{itemize}
Diz-se das combinações chímicas, em que o nitro entra na proporção de três por um.
\section{Trinitrina}
\begin{itemize}
\item {Grp. gram.:f.}
\end{itemize}
Substância explosiva, o mesmo que \textunderscore nitro-glycerina\textunderscore .
\section{Trinitrocellulose}
\begin{itemize}
\item {Grp. gram.:f.}
\end{itemize}
Explosivo, o mesmo que \textunderscore algodão-pólvora\textunderscore .
\section{Trinitrofênico}
\begin{itemize}
\item {Grp. gram.:adj.}
\end{itemize}
Diz-se de um ácido desinfectante, mais conhecido por ácido pícrico.(V.pícrico)
\section{Trinitrophênico}
\begin{itemize}
\item {Grp. gram.:adj.}
\end{itemize}
Diz-se de um ácido desinfectante, mais conhecido por ácido pícrico.(V.pícrico)
\section{Trino}
\begin{itemize}
\item {Grp. gram.:adj.}
\end{itemize}
\begin{itemize}
\item {Grp. gram.:M.  e  adj.}
\end{itemize}
\begin{itemize}
\item {Proveniência:(Lat. \textunderscore trinus\textunderscore )}
\end{itemize}
Composto de três.
O mesmo que \textunderscore trinitário\textunderscore .
\section{Trino}
\begin{itemize}
\item {Grp. gram.:m.}
\end{itemize}
Acto de trinar; gorgeio.
\section{Trinogeto}
\begin{itemize}
\item {Grp. gram.:m.}
\end{itemize}
Gênero de plantas solanáceas.
\section{Trinômine}
\begin{itemize}
\item {Grp. gram.:adj.}
\end{itemize}
\begin{itemize}
\item {Utilização:Poét.}
\end{itemize}
\begin{itemize}
\item {Proveniência:(Lat. \textunderscore trinominis\textunderscore )}
\end{itemize}
Que tem três nomes.
\section{Trinómio}
\begin{itemize}
\item {Grp. gram.:m.}
\end{itemize}
\begin{itemize}
\item {Utilização:Ext.}
\end{itemize}
\begin{itemize}
\item {Grp. gram.:Adj.}
\end{itemize}
\begin{itemize}
\item {Proveniência:(Do gr. \textunderscore treis\textunderscore  + \textunderscore nomos\textunderscore )}
\end{itemize}
Polynómio de três termos.
Aquillo que tem três partes.
Que tem três termos ou partes.
\section{Trióico}
\begin{itemize}
\item {Grp. gram.:adj.}
\end{itemize}
\begin{itemize}
\item {Utilização:Bot.}
\end{itemize}
Que tem flôres masculinas, femininas e hermaphroditas em três indivíduos separados.
(Deviração incorrecta, do gr. \textunderscore treis\textunderscore  + \textunderscore oikos\textunderscore . A fórma exacta seria \textunderscore trieco\textunderscore . V. \textunderscore triécico\textunderscore  e \textunderscore triecia\textunderscore )
\section{Trional}
\begin{itemize}
\item {Grp. gram.:m.}
\end{itemize}
\begin{itemize}
\item {Utilização:Pharm.}
\end{itemize}
Pó medicinal, applicado contra insómnias nervosas.
\section{Triónicho}
\begin{itemize}
\item {fónica:co}
\end{itemize}
\begin{itemize}
\item {Grp. gram.:m.}
\end{itemize}
\begin{itemize}
\item {Proveniência:(Do gr. \textunderscore treis\textunderscore  + \textunderscore onux\textunderscore , \textunderscore onukos\textunderscore )}
\end{itemize}
Gênero de insectos coleópteros pentâmeros.
\section{Triónico}
\begin{itemize}
\item {Grp. gram.:m.}
\end{itemize}
\begin{itemize}
\item {Proveniência:(Do gr. \textunderscore treis\textunderscore  + \textunderscore onux\textunderscore , \textunderscore onukos\textunderscore )}
\end{itemize}
Gênero de insectos coleópteros pentâmeros.
\section{Trioptérido}
\begin{itemize}
\item {Grp. gram.:m.}
\end{itemize}
\begin{itemize}
\item {Proveniência:(Do gr. \textunderscore treis\textunderscore  + \textunderscore pterus\textunderscore  + \textunderscore eidos\textunderscore )}
\end{itemize}
Gênero de plantas malpigiáceas.
\section{Trioptérydo}
\begin{itemize}
\item {Grp. gram.:m.}
\end{itemize}
\begin{itemize}
\item {Proveniência:(Do gr. \textunderscore treis\textunderscore  + \textunderscore pterus\textunderscore  + \textunderscore eidos\textunderscore )}
\end{itemize}
Gênero de plantas malpigiáceas.
\section{Triovulado}
\begin{itemize}
\item {Grp. gram.:adj.}
\end{itemize}
\begin{itemize}
\item {Proveniência:(De \textunderscore tri...\textunderscore  + \textunderscore ovulado\textunderscore )}
\end{itemize}
Que tem três óvulos.
\section{Tripa}
\begin{itemize}
\item {Grp. gram.:f.}
\end{itemize}
\begin{itemize}
\item {Utilização:Náut.}
\end{itemize}
\begin{itemize}
\item {Grp. gram.:Loc.}
\end{itemize}
\begin{itemize}
\item {Utilização:fam.}
\end{itemize}
\begin{itemize}
\item {Grp. gram.:Loc. adv.}
\end{itemize}
\begin{itemize}
\item {Grp. gram.:Pl.}
\end{itemize}
\begin{itemize}
\item {Utilização:Fam.}
\end{itemize}
\begin{itemize}
\item {Proveniência:(Do al. \textunderscore strippe\textunderscore )}
\end{itemize}
O mesmo que \textunderscore intestino\textunderscore ^1.
Estralheira, que ajuda os amantilhos, para içar e arrear as vêrgas dos papafigos.
\textunderscore Tripa de lobo\textunderscore , qualidade do indivíduo que come muito e anda sempre magro.
\textunderscore Á tripa fôrra\textunderscore , gratuitamente; de borla; á larga. Cf. Filinto, III; 223.
Fôrças, coragem.
\section{Tripa-de-orelha}
\begin{itemize}
\item {Grp. gram.:f.}
\end{itemize}
Planta, da famíllia das compostas, (\textunderscore andryala integrifolia\textunderscore , Lin.), também conhecida por \textunderscore camareira\textunderscore  e \textunderscore alface dos montes\textunderscore .
\section{Tripagem}
\begin{itemize}
\item {Grp. gram.:f.}
\end{itemize}
Porção de tripas.
\section{Tripalhada}
\begin{itemize}
\item {Grp. gram.:f.}
\end{itemize}
O mesmo que \textunderscore tripagem\textunderscore .
\section{Tripango}
\begin{itemize}
\item {Grp. gram.:m.}
\end{itemize}
\begin{itemize}
\item {Utilização:Bras}
\end{itemize}
O mesmo que \textunderscore holothúria\textunderscore .
\section{Triparia}
\begin{itemize}
\item {Grp. gram.:f.}
\end{itemize}
\begin{itemize}
\item {Utilização:Des.}
\end{itemize}
\begin{itemize}
\item {Proveniência:(De \textunderscore tripa\textunderscore )}
\end{itemize}
Lugar, onde se vendem tripas.
Arruamento de tripeiros.--Em Coimbra, havia a \textunderscore rua da Triparia\textunderscore . Cf. o periódico \textunderscore Conimbricense\textunderscore  de 1 de Abril de 1899.
\section{Tríparo}
\begin{itemize}
\item {Grp. gram.:adj.}
\end{itemize}
\begin{itemize}
\item {Utilização:Bot.}
\end{itemize}
\begin{itemize}
\item {Proveniência:(Do lat. \textunderscore tres\textunderscore  + \textunderscore parere\textunderscore )}
\end{itemize}
Que se produz e reproduz em grupos de três.
\section{Tripartir}
\begin{itemize}
\item {Grp. gram.:v. t.}
\end{itemize}
\begin{itemize}
\item {Proveniência:(Do lat. \textunderscore tripartire\textunderscore )}
\end{itemize}
Partir em três partes.
\section{Tripartível}
\begin{itemize}
\item {Grp. gram.:adj.}
\end{itemize}
Que se póde tripartir.
\section{Tripe}
\begin{itemize}
\item {Grp. gram.:m.}
\end{itemize}
\begin{itemize}
\item {Utilização:Ant.}
\end{itemize}
\begin{itemize}
\item {Proveniência:(Fr. \textunderscore tripe\textunderscore )}
\end{itemize}
Espécie de estôfo avelludado:«\textunderscore calções de tripe azul.\textunderscore »(De um testamento de 1693)
\section{Tripé}
\begin{itemize}
\item {Grp. gram.:m.}
\end{itemize}
O mesmo que \textunderscore tripeça\textunderscore .
Apparelho portátil, que se arma com três escoras, e sôbre o qual se assenta a máquina photográphica, para focar objectos ou pessôas, ou se assenta o telescópio, etc.
\section{Tripeça}
\begin{itemize}
\item {Grp. gram.:f.}
\end{itemize}
\begin{itemize}
\item {Utilização:Fig.}
\end{itemize}
\begin{itemize}
\item {Utilização:Burl.}
\end{itemize}
\begin{itemize}
\item {Proveniência:(Do bt. lat. \textunderscore tripetia\textunderscore )}
\end{itemize}
Banco de três pés.
Officio de sapateiro.
Reunião de três pessôas.
\section{Tripecinha}
\begin{itemize}
\item {Grp. gram.:f.}
\end{itemize}
Pequena tripeça.
\section{Tripeço}
\begin{itemize}
\item {Grp. gram.:m.}
\end{itemize}
\begin{itemize}
\item {Utilização:Prov.}
\end{itemize}
Espécie de tripeça.
\section{Tripeira}
\begin{itemize}
\item {Grp. gram.:f.}
\end{itemize}
\begin{itemize}
\item {Utilização:Prov.}
\end{itemize}
\begin{itemize}
\item {Utilização:trasm.}
\end{itemize}
\begin{itemize}
\item {Proveniência:(De \textunderscore tripeiro\textunderscore )}
\end{itemize}
Vendedeira de tripas.
Mulhér rota e suja.
\section{Tripeiro}
\begin{itemize}
\item {Grp. gram.:m.}
\end{itemize}
\begin{itemize}
\item {Utilização:Deprec.}
\end{itemize}
\begin{itemize}
\item {Proveniência:(De \textunderscore tripa\textunderscore )}
\end{itemize}
Vendedor de tripas.
Aquelle que se sustenta de tripas.
Habitante do Pôrto.
\section{Tripétalo}
\begin{itemize}
\item {Grp. gram.:adj.}
\end{itemize}
\begin{itemize}
\item {Proveniência:(De \textunderscore tri...\textunderscore  + \textunderscore pétala\textunderscore )}
\end{itemize}
Que tem três pétalas.
\section{Tripetrepe}
\begin{itemize}
\item {Grp. gram.:adv.}
\end{itemize}
\begin{itemize}
\item {Proveniência:(T. onom.)}
\end{itemize}
De manso, pé ante pé; dissimuladamente.
\section{Triphânio}
\begin{itemize}
\item {Grp. gram.:m.}
\end{itemize}
O mesmo ou melhór que \textunderscore tríphano\textunderscore .
\section{Triphanito}
\begin{itemize}
\item {Grp. gram.:m.}
\end{itemize}
\begin{itemize}
\item {Utilização:Miner.}
\end{itemize}
\begin{itemize}
\item {Proveniência:(De \textunderscore tríphano\textunderscore )}
\end{itemize}
Mineral côr de rosa e de fractura xistosa.
\section{Tríphano}
\begin{itemize}
\item {Grp. gram.:m.}
\end{itemize}
\begin{itemize}
\item {Utilização:Miner.}
\end{itemize}
\begin{itemize}
\item {Proveniência:(Do gr. \textunderscore treis\textunderscore  + \textunderscore phainesthai\textunderscore )}
\end{itemize}
Variedade de feldspatho de lithina.
\section{Triphármaco}
\begin{itemize}
\item {Grp. gram.:m.}
\end{itemize}
\begin{itemize}
\item {Proveniência:(Do gr. \textunderscore treis\textunderscore  + \textunderscore pharmakon\textunderscore )}
\end{itemize}
Antigo medicamento, composto de três drogas.
\section{Triphélia}
\begin{itemize}
\item {Grp. gram.:f.}
\end{itemize}
Gênero de plantas myrtáceas.
\section{Triphenina}
\begin{itemize}
\item {Grp. gram.:f.}
\end{itemize}
Medicamento antineurálgico e antipyrético.
\section{Triphonia}
\begin{itemize}
\item {Grp. gram.:f.}
\end{itemize}
\begin{itemize}
\item {Utilização:Mús.}
\end{itemize}
\begin{itemize}
\item {Proveniência:(Do gr. \textunderscore treis\textunderscore  + \textunderscore phone\textunderscore )}
\end{itemize}
Contraponto a três vozes, na Idade-Média.
\section{Tripterígio}
\begin{itemize}
\item {Grp. gram.:m.}
\end{itemize}
\begin{itemize}
\item {Utilização:Zool.}
\end{itemize}
\begin{itemize}
\item {Proveniência:(Do gr. \textunderscore tri\textunderscore  + \textunderscore pterux\textunderscore )}
\end{itemize}
Gênero de peixes gobióides.
\section{Tripterýgio}
\begin{itemize}
\item {Grp. gram.:m.}
\end{itemize}
\begin{itemize}
\item {Utilização:Zool.}
\end{itemize}
\begin{itemize}
\item {Proveniência:(Do gr. \textunderscore tri\textunderscore  + \textunderscore pterux\textunderscore )}
\end{itemize}
Gênero de peixes gobióides.
\section{Tripterospermo}
\begin{itemize}
\item {Grp. gram.:m.}
\end{itemize}
\begin{itemize}
\item {Proveniência:(Do gr. \textunderscore treis\textunderscore  + \textunderscore pteron\textunderscore  + \textunderscore sperma\textunderscore )}
\end{itemize}
Gênero de plantas gencianáceas.
\section{Tríptico}
\begin{itemize}
\item {Grp. gram.:m.}
\end{itemize}
\begin{itemize}
\item {Proveniência:(Gr. \textunderscore triptukhos\textunderscore )}
\end{itemize}
Quadro, sôbre três panos que se dobram.
Painel, coberto por duas meias portas, cujas faces internas, e ás vezes as externas, são trabalhadas como o painel propriamente dito.
Livrinho de três fôlhas.
\section{Triptólemo}
\begin{itemize}
\item {Grp. gram.:m.}
\end{itemize}
\begin{itemize}
\item {Proveniência:(Lat. \textunderscore Triptolemus\textunderscore , n. p.)}
\end{itemize}
A constellação dos Gêmeos.
\section{Tríptycho}
\begin{itemize}
\item {fónica:co}
\end{itemize}
\begin{itemize}
\item {Grp. gram.:m.}
\end{itemize}
\begin{itemize}
\item {Proveniência:(Gr. \textunderscore triptukhos\textunderscore )}
\end{itemize}
Quadro, sôbre três panos que se dobram.
Painel, coberto por duas meias portas, cujas faces internas, e ás vezes as externas, são trabalhadas como o painel propriamente dito.
Livrinho de três fôlhas.
\section{Tripudiante}
\begin{itemize}
\item {Grp. gram.:adj.}
\end{itemize}
\begin{itemize}
\item {Grp. gram.:m.}
\end{itemize}
\begin{itemize}
\item {Proveniência:(Lat. \textunderscore tripudians\textunderscore )}
\end{itemize}
Que tripudia.
Aquelle que tripudia. Cf. Castilho, \textunderscore Fastos\textunderscore , II, 584.
\section{Tripudiar}
\begin{itemize}
\item {Grp. gram.:v. t.}
\end{itemize}
\begin{itemize}
\item {Utilização:Fig.}
\end{itemize}
\begin{itemize}
\item {Proveniência:(Lat. \textunderscore tripudiare\textunderscore )}
\end{itemize}
Saltar ou dançar, batendo com os pés.
Folgar desenvoltamente.
Viver no crime ou no vício.
\section{Tripúdio}
\begin{itemize}
\item {Grp. gram.:m.}
\end{itemize}
\begin{itemize}
\item {Utilização:Fig.}
\end{itemize}
\begin{itemize}
\item {Proveniência:(Lat. \textunderscore tripudium\textunderscore )}
\end{itemize}
Acto ou effeito de tripudiar.
Libertinagem.
\section{Tripulação}
\begin{itemize}
\item {Grp. gram.:f.}
\end{itemize}
\begin{itemize}
\item {Proveniência:(De \textunderscore tripular\textunderscore )}
\end{itemize}
Conjunto dos marinheiros, que trabalham num navio.
\section{Tripulante}
\begin{itemize}
\item {Grp. gram.:m.  e  adj.}
\end{itemize}
O que tripula; marinheiro.
\section{Tripular}
\begin{itemize}
\item {Grp. gram.:v. t.}
\end{itemize}
Prover de pessoal necessário para as manobras e mais serviço de (um navio).
Dirigir ou governar (uma embarcação).
\section{Triques}
\begin{itemize}
\item {Grp. gram.:adj.}
\end{itemize}
\begin{itemize}
\item {Utilização:Gír.}
\end{itemize}
Aperaltado, liró.
\section{Triquestroques}
\begin{itemize}
\item {Grp. gram.:m.}
\end{itemize}
\begin{itemize}
\item {Utilização:Pleb.}
\end{itemize}
Jôgo de palavras; trocadilho; equívoco.
(Infl. de \textunderscore trocar\textunderscore )
\section{Triquete}
\begin{itemize}
\item {fónica:quê}
\end{itemize}
\begin{itemize}
\item {Grp. gram.:m.}
\end{itemize}
O mesmo que \textunderscore passo\textunderscore ^1, mas só us. na phrase \textunderscore a cada triquete\textunderscore , a cada passo.
(Cast. \textunderscore triquete\textunderscore )
\section{Triquetraque}
\begin{itemize}
\item {Grp. gram.:m.}
\end{itemize}
\begin{itemize}
\item {Proveniência:(T. onom.)}
\end{itemize}
Peça de fôgo de artifício, que dá estalos repetidos.
\section{Triquetraque}
\begin{itemize}
\item {Grp. gram.:m.}
\end{itemize}
\begin{itemize}
\item {Utilização:Des.}
\end{itemize}
\begin{itemize}
\item {Proveniência:(Fr. \textunderscore tric-trac\textunderscore )}
\end{itemize}
Jôgo do gamão.
Tabuleiro do gamão.
\section{Triquetraz}
\begin{itemize}
\item {Grp. gram.:m.}
\end{itemize}
O mesmo que \textunderscore traquinas\textunderscore .
\section{Tríquetro}
\begin{itemize}
\item {Grp. gram.:adj.}
\end{itemize}
\begin{itemize}
\item {Grp. gram.:M.}
\end{itemize}
\begin{itemize}
\item {Proveniência:(Lat. \textunderscore triquetrus\textunderscore )}
\end{itemize}
Que tem três ângulos.
Reunião de três coxas, com as respectivas pernas e pés, que se observam em algumas medalhas antigas.
\section{Triradiado}
\begin{itemize}
\item {fónica:ra}
\end{itemize}
\begin{itemize}
\item {Grp. gram.:adj.}
\end{itemize}
\begin{itemize}
\item {Utilização:Hist. nat.}
\end{itemize}
\begin{itemize}
\item {Proveniência:(De \textunderscore tri...\textunderscore  + \textunderscore radiado\textunderscore )}
\end{itemize}
Que tem três raios.
\section{Triramoso}
\begin{itemize}
\item {fónica:ra}
\end{itemize}
\begin{itemize}
\item {Grp. gram.:adj.}
\end{itemize}
\begin{itemize}
\item {Proveniência:(De \textunderscore tri...\textunderscore  + \textunderscore ramoso\textunderscore )}
\end{itemize}
Que tem três ramos.
\section{Trirectângulo}
\begin{itemize}
\item {fónica:re}
\end{itemize}
\begin{itemize}
\item {Grp. gram.:adj.}
\end{itemize}
\begin{itemize}
\item {Proveniência:(De \textunderscore tri...\textunderscore  + \textunderscore rectângulo\textunderscore )}
\end{itemize}
Que tem três ângulos rectos.
\section{Trirháphide}
\begin{itemize}
\item {fónica:rá}
\end{itemize}
\begin{itemize}
\item {Grp. gram.:m.}
\end{itemize}
\begin{itemize}
\item {Proveniência:(Do gr. \textunderscore treis\textunderscore  + \textunderscore rhaphis\textunderscore )}
\end{itemize}
Gênero de plantas gramíneas.
\section{Triregno}
\begin{itemize}
\item {fónica:re}
\end{itemize}
\begin{itemize}
\item {Grp. gram.:m.}
\end{itemize}
\begin{itemize}
\item {Proveniência:(De \textunderscore tri...\textunderscore  + \textunderscore regno\textunderscore )}
\end{itemize}
Domínio de três reinos; triarchia.
\section{Trireme}
\begin{itemize}
\item {fónica:rê}
\end{itemize}
\begin{itemize}
\item {Grp. gram.:f.}
\end{itemize}
\begin{itemize}
\item {Proveniência:(Lat. \textunderscore triremis\textunderscore )}
\end{itemize}
Antiga galera, de três ordens de remos.
\section{Trirhomboidal}
\begin{itemize}
\item {fónica:rom}
\end{itemize}
\begin{itemize}
\item {Grp. gram.:adj.}
\end{itemize}
\begin{itemize}
\item {Utilização:Miner.}
\end{itemize}
\begin{itemize}
\item {Proveniência:(De \textunderscore tri...\textunderscore  + \textunderscore rhombóide\textunderscore )}
\end{itemize}
Diz-se do mineral, cuja fórma participa de três rhombóides differentes.
\section{Trirogma}
\begin{itemize}
\item {fónica:ró}
\end{itemize}
\begin{itemize}
\item {Grp. gram.:f.}
\end{itemize}
Gênero de insectos hymenópteros.
\section{Trirradiado}
\begin{itemize}
\item {Grp. gram.:adj.}
\end{itemize}
\begin{itemize}
\item {Utilização:Hist. nat.}
\end{itemize}
\begin{itemize}
\item {Proveniência:(De \textunderscore tri...\textunderscore  + \textunderscore radiado\textunderscore )}
\end{itemize}
Que tem três raios.
\section{Trirráfide}
\begin{itemize}
\item {Grp. gram.:m.}
\end{itemize}
\begin{itemize}
\item {Proveniência:(Do gr. \textunderscore treis\textunderscore  + \textunderscore rhaphis\textunderscore )}
\end{itemize}
Gênero de plantas gramíneas.
\section{Trirramoso}
\begin{itemize}
\item {Grp. gram.:adj.}
\end{itemize}
\begin{itemize}
\item {Proveniência:(De \textunderscore tri...\textunderscore  + \textunderscore ramoso\textunderscore )}
\end{itemize}
Que tem três ramos.
\section{Trirrectângulo}
\begin{itemize}
\item {Grp. gram.:adj.}
\end{itemize}
\begin{itemize}
\item {Proveniência:(De \textunderscore tri...\textunderscore  + \textunderscore rectângulo\textunderscore )}
\end{itemize}
Que tem três ângulos rectos.
\section{Trirregno}
\begin{itemize}
\item {Grp. gram.:m.}
\end{itemize}
\begin{itemize}
\item {Proveniência:(De \textunderscore tri...\textunderscore  + \textunderscore regno\textunderscore )}
\end{itemize}
Domínio de três reinos; triarquia.
\section{Trirreme}
\begin{itemize}
\item {Grp. gram.:f.}
\end{itemize}
\begin{itemize}
\item {Proveniência:(Lat. \textunderscore triremis\textunderscore )}
\end{itemize}
Antiga galera, de três ordens de remos.
\section{Trirrogma}
\begin{itemize}
\item {Grp. gram.:f.}
\end{itemize}
Gênero de insectos himenópteros.
\section{Trirromboidal}
\begin{itemize}
\item {Grp. gram.:adj.}
\end{itemize}
\begin{itemize}
\item {Utilização:Miner.}
\end{itemize}
\begin{itemize}
\item {Proveniência:(De \textunderscore tri...\textunderscore  + \textunderscore rombóide\textunderscore )}
\end{itemize}
Diz-se do mineral, cuja fórma participa de três rombóides diferentes.
\section{Tris}
\begin{itemize}
\item {Grp. gram.:m.}
\end{itemize}
Um quási nada; \textunderscore escapou por um tris\textunderscore .
\section{Tris}
Voz imitativa de ruído, feito por qualquer coisa que se parte, especialmente vidros.
\section{Tris...}
\begin{itemize}
\item {Grp. gram.:pref.}
\end{itemize}
O mesmo que \textunderscore tres...\textunderscore ^2
\section{Trisacramental}
\begin{itemize}
\item {fónica:sa}
\end{itemize}
\begin{itemize}
\item {Grp. gram.:adj.}
\end{itemize}
\begin{itemize}
\item {Proveniência:(De \textunderscore tri...\textunderscore  + \textunderscore sacramento\textunderscore )}
\end{itemize}
Diz-se dos que, como os Anglicanos, só admittem três sacramentos.
\section{Triságio}
\begin{itemize}
\item {Grp. gram.:m.}
\end{itemize}
\begin{itemize}
\item {Proveniência:(Gr. \textunderscore trisagios\textunderscore )}
\end{itemize}
Hymno religioso, que começa pelas palavras latinas \textunderscore Sanctus, Sanctus, Sanctus\textunderscore .
\section{Trissacramental}
\begin{itemize}
\item {Grp. gram.:adj.}
\end{itemize}
\begin{itemize}
\item {Proveniência:(De \textunderscore tri...\textunderscore  + \textunderscore sacramento\textunderscore )}
\end{itemize}
Diz-se dos que, como os Anglicanos, só admitem três sacramentos.
\section{Trissilábico}
\begin{itemize}
\item {Grp. gram.:adj.}
\end{itemize}
\begin{itemize}
\item {Proveniência:(De \textunderscore tri\textunderscore  + \textunderscore silábico\textunderscore )}
\end{itemize}
Que tem três sílabas.
\section{Trissílabo}
\begin{itemize}
\item {Grp. gram.:m.  e  adj.}
\end{itemize}
\begin{itemize}
\item {Proveniência:(Lat. \textunderscore trisyllabus\textunderscore )}
\end{itemize}
Diz-se da palavra, que tem três sílabas.
\section{Trissulco}
\begin{itemize}
\item {Grp. gram.:adj.}
\end{itemize}
\begin{itemize}
\item {Proveniência:(Lat. \textunderscore trisulcus\textunderscore )}
\end{itemize}
O mesmo que \textunderscore trífido\textunderscore .
\section{Trissulfureto}
\begin{itemize}
\item {Grp. gram.:m.}
\end{itemize}
\begin{itemize}
\item {Proveniência:(De \textunderscore tri...\textunderscore  + \textunderscore sulfureto\textunderscore )}
\end{itemize}
Sulfureto, que tem três proporções de enxôfre.
\section{Triste}
\begin{itemize}
\item {Grp. gram.:adj.}
\end{itemize}
\begin{itemize}
\item {Grp. gram.:M.  e  f.}
\end{itemize}
\begin{itemize}
\item {Grp. gram.:M. Pl.}
\end{itemize}
\begin{itemize}
\item {Utilização:Ant.}
\end{itemize}
\begin{itemize}
\item {Proveniência:(Lat. \textunderscore tristis\textunderscore )}
\end{itemize}
Que tem mágoa ou afflicção.
Que não tem alegria.
Que, pela contracção ou vincos da fronte, exprime cuidados ou melancolia.
Severo.
Infeliz.
Lastimoso.
Sombrio; lúgubre: \textunderscore noite triste\textunderscore .
Que causa melancolia.
Deprimido.
Enfezado.
Insignificante: \textunderscore por uns tristes vintens\textunderscore .
Caricato.
Pessôa infeliz.
Anéis, com que as mulhéres ornavam a cabeça.
\section{Trístega}
\begin{itemize}
\item {Grp. gram.:f.}
\end{itemize}
\begin{itemize}
\item {Utilização:Ant.}
\end{itemize}
\begin{itemize}
\item {Proveniência:(Lat. \textunderscore tristega\textunderscore )}
\end{itemize}
Terceiro andar.
Águas-furtadas.
Eirado.
Mirante.
Edifício de três andares.
Coberta de navio, formada de três sobrados.
\section{Tristemente}
\begin{itemize}
\item {Grp. gram.:adv.}
\end{itemize}
De modo triste; com tristeza.
Melancolicamente.
\section{Tristernal}
\begin{itemize}
\item {Grp. gram.:adj.}
\end{itemize}
\begin{itemize}
\item {Utilização:Anat.}
\end{itemize}
\begin{itemize}
\item {Proveniência:(De \textunderscore tri...\textunderscore  + \textunderscore esterno\textunderscore )}
\end{itemize}
Relativo á terceira peça do esterno.
\section{Tristésio}
\begin{itemize}
\item {Grp. gram.:adj.}
\end{itemize}
\begin{itemize}
\item {Utilização:T. de Turquel}
\end{itemize}
Alegre; bem disposto.
\section{Tristeza}
\begin{itemize}
\item {Grp. gram.:f.}
\end{itemize}
\begin{itemize}
\item {Proveniência:(Do lat. \textunderscore tristitia\textunderscore )}
\end{itemize}
Qualidade ou estado do que é triste.
Melancolia.
Consternação.
Aspecto de quem revela mágoa ou afflicção.
\section{Trísticha}
\begin{itemize}
\item {fónica:ca}
\end{itemize}
\begin{itemize}
\item {Grp. gram.:f.}
\end{itemize}
Gênero de plantas podostemmáceas.
(Cp. \textunderscore trísticho\textunderscore )
\section{Trística}
\begin{itemize}
\item {Grp. gram.:f.}
\end{itemize}
Gênero de plantas podostemmáceas.
(Cp. \textunderscore trísticho\textunderscore )
\section{Trísticho}
\begin{itemize}
\item {fónica:co}
\end{itemize}
\begin{itemize}
\item {Grp. gram.:adj.}
\end{itemize}
\begin{itemize}
\item {Utilização:Hist. Nat.}
\end{itemize}
\begin{itemize}
\item {Proveniência:(Do gr. \textunderscore treis\textunderscore  + \textunderscore stikhos\textunderscore )}
\end{itemize}
Diz-se de certos órgãos, dispostos em três ordens.
\section{Trístico}
\begin{itemize}
\item {Grp. gram.:adj.}
\end{itemize}
\begin{itemize}
\item {Utilização:Hist. Nat.}
\end{itemize}
\begin{itemize}
\item {Proveniência:(Do gr. \textunderscore treis\textunderscore  + \textunderscore stikhos\textunderscore )}
\end{itemize}
Diz-se de certos órgãos, dispostos em três ordens.
\section{Tristimania}
\begin{itemize}
\item {Grp. gram.:f.}
\end{itemize}
\begin{itemize}
\item {Utilização:Med.}
\end{itemize}
\begin{itemize}
\item {Proveniência:(De \textunderscore triste\textunderscore  + \textunderscore mania\textunderscore )}
\end{itemize}
Monomania com tristeza.
Tristeza habitual, sem fundamento razoável.
\section{Trístoma}
\begin{itemize}
\item {Grp. gram.:m.}
\end{itemize}
\begin{itemize}
\item {Proveniência:(Do gr. \textunderscore treis\textunderscore  + \textunderscore stoma\textunderscore )}
\end{itemize}
Gênero de helminthos intestinaes.
\section{Trístomo}
\begin{itemize}
\item {Grp. gram.:m.}
\end{itemize}
O mesmo ou melhór que \textunderscore tristoma\textunderscore .
\section{Tristonho}
\begin{itemize}
\item {Grp. gram.:adj.}
\end{itemize}
\begin{itemize}
\item {Proveniência:(De \textunderscore triste\textunderscore )}
\end{itemize}
Que mostra tristeza.
Melancólico; sorumbático.
Que infunde tristeza.
Medonho.
\section{Tristura}
\begin{itemize}
\item {Grp. gram.:f.}
\end{itemize}
O mesmo que \textunderscore tristeza\textunderscore .
\section{Trisulco}
\begin{itemize}
\item {fónica:sul}
\end{itemize}
\begin{itemize}
\item {Grp. gram.:adj.}
\end{itemize}
\begin{itemize}
\item {Proveniência:(Lat. \textunderscore trisulcus\textunderscore )}
\end{itemize}
O mesmo que \textunderscore trífido\textunderscore .
\section{Trisulfureto}
\begin{itemize}
\item {fónica:sul}
\end{itemize}
\begin{itemize}
\item {Grp. gram.:m.}
\end{itemize}
\begin{itemize}
\item {Proveniência:(De \textunderscore tri...\textunderscore  + \textunderscore sulfureto\textunderscore )}
\end{itemize}
Sulfureto, que tem três proporções de enxôfre.
\section{Trisyllábico}
\begin{itemize}
\item {fónica:si}
\end{itemize}
\begin{itemize}
\item {Grp. gram.:adj.}
\end{itemize}
\begin{itemize}
\item {Proveniência:(De \textunderscore tri\textunderscore  + \textunderscore syllábico\textunderscore )}
\end{itemize}
Que tem três sýllabas.
\section{Trisýllabo}
\begin{itemize}
\item {fónica:si}
\end{itemize}
\begin{itemize}
\item {Grp. gram.:m.  e  adj.}
\end{itemize}
\begin{itemize}
\item {Proveniência:(Lat. \textunderscore trisyllabus\textunderscore )}
\end{itemize}
Diz-se da palavra, que tem três sýllabas.
\section{Tritão}
\begin{itemize}
\item {Grp. gram.:m.}
\end{itemize}
\begin{itemize}
\item {Proveniência:(Lat. \textunderscore triton\textunderscore )}
\end{itemize}
Qualquer divindade marítima, de ordem inferior, (em Mythologia).
Espécie de salamandra.
Gênero de conchas univalves.
\section{Triteísmo}
\begin{itemize}
\item {Grp. gram.:m.}
\end{itemize}
\begin{itemize}
\item {Proveniência:(Do gr. \textunderscore treis\textunderscore  + \textunderscore theos\textunderscore )}
\end{itemize}
Doutrina dos que sustentam que em Deus não há só três pessôas, mas também três essências, três substâncias e três deuses.
\section{Triteísta}
\begin{itemize}
\item {Grp. gram.:m.  e  adj.}
\end{itemize}
Sectário do triteísmo.
\section{Triteofia}
\begin{itemize}
\item {Grp. gram.:f.}
\end{itemize}
\begin{itemize}
\item {Utilização:Med.}
\end{itemize}
\begin{itemize}
\item {Proveniência:(Do gr. \textunderscore tritaios\textunderscore  + \textunderscore phuein\textunderscore )}
\end{itemize}
Febre intermitente terçan.
\section{Triteophya}
\begin{itemize}
\item {Grp. gram.:f.}
\end{itemize}
\begin{itemize}
\item {Utilização:Med.}
\end{itemize}
\begin{itemize}
\item {Proveniência:(Do gr. \textunderscore tritaios\textunderscore  + \textunderscore phuein\textunderscore )}
\end{itemize}
Febre intermittente terçan.
\section{Triternado}
\begin{itemize}
\item {Grp. gram.:adj.}
\end{itemize}
\begin{itemize}
\item {Utilização:Bot.}
\end{itemize}
\begin{itemize}
\item {Proveniência:(De \textunderscore tri...\textunderscore  + \textunderscore terno\textunderscore ^1)}
\end{itemize}
Diz-se da fôlha composta de um peciolo que se subdivide em três ramos, cada um dos quaes tem três folíolos inseridos no mesmo ponto.
\section{Tritheísmo}
\begin{itemize}
\item {Grp. gram.:m.}
\end{itemize}
\begin{itemize}
\item {Proveniência:(Do gr. \textunderscore treis\textunderscore  + \textunderscore theos\textunderscore )}
\end{itemize}
Doutrina dos que sustentam que em Deus não há só três pessôas, mas também três essências, três substâncias e três deuses.
\section{Tritheísta}
\begin{itemize}
\item {Grp. gram.:m.  e  adj.}
\end{itemize}
Sectário do tritheísmo.
\section{Tritíceo}
\begin{itemize}
\item {Grp. gram.:adj.}
\end{itemize}
\begin{itemize}
\item {Proveniência:(Lat. \textunderscore triticeus\textunderscore )}
\end{itemize}
Relativo ao trigo.
Que tem algumas qualidades do trigo.
\section{Triticina}
\begin{itemize}
\item {Grp. gram.:f.}
\end{itemize}
\begin{itemize}
\item {Proveniência:(Do lat. \textunderscore triticum\textunderscore )}
\end{itemize}
Glúten da farinha do trigo.
\section{Triticita}
\begin{itemize}
\item {Grp. gram.:f.}
\end{itemize}
\begin{itemize}
\item {Proveniência:(Do lat. \textunderscore triticum\textunderscore )}
\end{itemize}
Espiga do trigo fóssil.
Sulfureto de cobre, que se descobriu em Frankenberg sob a fórma de espiga de trigo.
\section{Tríticito}
\begin{itemize}
\item {Grp. gram.:m.}
\end{itemize}
O mesmo ou melhór que \textunderscore triticita\textunderscore .
\section{Trivalente}
\begin{itemize}
\item {Grp. gram.:adj.}
\end{itemize}
\begin{itemize}
\item {Utilização:Chím.}
\end{itemize}
\begin{itemize}
\item {Proveniência:(De \textunderscore tri...\textunderscore  + \textunderscore valer\textunderscore )}
\end{itemize}
Que se pode ser saturado por um corpo triatómico.
\section{Trívia}
\begin{itemize}
\item {Grp. gram.:f.}
\end{itemize}
\begin{itemize}
\item {Proveniência:(Lat. \textunderscore Trivia\textunderscore )}
\end{itemize}
O mesmo que \textunderscore Lua\textunderscore . Cf. Garrett, \textunderscore Retr. de Vénus\textunderscore , 11.
\section{Triviá}
\begin{itemize}
\item {Grp. gram.:m.}
\end{itemize}
\begin{itemize}
\item {Utilização:Bras}
\end{itemize}
Trem de cozinha.
(Por \textunderscore trivial\textunderscore ?)
\section{Trivial}
\begin{itemize}
\item {Grp. gram.:adj.}
\end{itemize}
\begin{itemize}
\item {Proveniência:(Lat. \textunderscore trivialis\textunderscore )}
\end{itemize}
Que é sabido de todos; notório.
Usado.
Commum; vulgar, ordinário.
Baixo.
\section{Trivialidade}
\begin{itemize}
\item {Grp. gram.:f.}
\end{itemize}
Qualidade do que é trivial.
Coisa trivial.
\section{Trivializar}
\begin{itemize}
\item {Grp. gram.:v. t.}
\end{itemize}
Tornar trivial.
\section{Trivialmente}
\begin{itemize}
\item {Grp. gram.:adv.}
\end{itemize}
De modo trivial.
\section{Trívio}
\begin{itemize}
\item {Grp. gram.:m.}
\end{itemize}
\begin{itemize}
\item {Grp. gram.:Adj.}
\end{itemize}
Lugar, onde se encontram três ruas ou caminhos.
Divisão inferior das artes liberaes, comprehendendo, segundo a organização das universidades da Idade-Média, a Grammática, a Rhetórica e a Dialéctica.
Que se reparte em três caminhos. Cf. Castilho, \textunderscore Fastos\textunderscore , I, 17.
\section{Trivogal}
\begin{itemize}
\item {Grp. gram.:f.}
\end{itemize}
\begin{itemize}
\item {Utilização:Gram.}
\end{itemize}
\begin{itemize}
\item {Proveniência:(De \textunderscore tri...\textunderscore  + \textunderscore vogal\textunderscore )}
\end{itemize}
O mesmo que \textunderscore tritongo\textunderscore .
\section{Triz}
\begin{itemize}
\item {Grp. gram.:f.}
\end{itemize}
\begin{itemize}
\item {Utilização:Prov.}
\end{itemize}
\begin{itemize}
\item {Utilização:beir.}
\end{itemize}
(Fórma pop. de \textunderscore icterícia\textunderscore )
\section{Trízia}
\begin{itemize}
\item {Grp. gram.:f.}
\end{itemize}
\begin{itemize}
\item {Utilização:Prov.}
\end{itemize}
\begin{itemize}
\item {Utilização:beir.}
\end{itemize}
(Fórma pop. de \textunderscore icterícia\textunderscore )
\section{Trizíncico}
\begin{itemize}
\item {Grp. gram.:adj.}
\end{itemize}
\begin{itemize}
\item {Utilização:Chím.}
\end{itemize}
\begin{itemize}
\item {Proveniência:(De \textunderscore tri...\textunderscore  + \textunderscore zíncico\textunderscore )}
\end{itemize}
Diz-se do sal zíncico, que contém três vezes tanta base como o sal neutro correspondente.
\section{Trizircónico}
\begin{itemize}
\item {Grp. gram.:adj.}
\end{itemize}
\begin{itemize}
\item {Utilização:Chím.}
\end{itemize}
\begin{itemize}
\item {Proveniência:(De \textunderscore tri...\textunderscore  + \textunderscore zircónico\textunderscore )}
\end{itemize}
Diz-se do sal zircónico, que contém três vezes tanta base como o sal neutro correspondente.
\section{Troada}
\begin{itemize}
\item {Grp. gram.:f.}
\end{itemize}
Acto ou effeito de troar; tiroteio.
\section{Troante}
\begin{itemize}
\item {Grp. gram.:adj.}
\end{itemize}
Que trôa.
\section{Troar}
\begin{itemize}
\item {Grp. gram.:v. i.}
\end{itemize}
\begin{itemize}
\item {Grp. gram.:M.}
\end{itemize}
\begin{itemize}
\item {Proveniência:(De \textunderscore trom\textunderscore )}
\end{itemize}
Trovejar.
Estrondear.
Resoar fortemente.
Fazer estrondo, tocando instrumentos músicos.
Troada.
\section{Troca}
\begin{itemize}
\item {Grp. gram.:f.}
\end{itemize}
Acto ou effeito de trocar; permutação; escambo.
\section{Troça}
\begin{itemize}
\item {Grp. gram.:f.}
\end{itemize}
\begin{itemize}
\item {Utilização:Náut.}
\end{itemize}
\begin{itemize}
\item {Utilização:Pop.}
\end{itemize}
Acto ou effeito de troçar.
Cabo, que atraca as antennas ao mastro.
Grande porção.
(Cp. cast. \textunderscore troza\textunderscore )
\section{Troça}
\begin{itemize}
\item {Grp. gram.:adj. f.}
\end{itemize}
\begin{itemize}
\item {Utilização:Prov.}
\end{itemize}
\begin{itemize}
\item {Utilização:minh.}
\end{itemize}
Diz-se da estôpa fraca ou de inferior qualidade. (Colhido em Barcelos)
\section{Trocadamente}
\begin{itemize}
\item {Grp. gram.:adv.}
\end{itemize}
De modo trocado; com troca, com permutação.
\section{Trocadilhar}
\begin{itemize}
\item {Grp. gram.:v. i.}
\end{itemize}
Fazer trocadilhos.
\section{Trocadilho}
\begin{itemize}
\item {Grp. gram.:m.}
\end{itemize}
\begin{itemize}
\item {Proveniência:(De \textunderscore trocado\textunderscore )}
\end{itemize}
Uso de expressões ambiguas.
Jôgo de palavras, por ornato ou por gracejo.
\section{Trocado}
\begin{itemize}
\item {Grp. gram.:adj.}
\end{itemize}
\begin{itemize}
\item {Proveniência:(De \textunderscore trocar\textunderscore )}
\end{itemize}
Permutado.
Substituido.
Confundido com outro.
\section{Trocador}
\begin{itemize}
\item {Grp. gram.:m.  e  adj.}
\end{itemize}
O que troca.
\section{Troçador}
\begin{itemize}
\item {Grp. gram.:m.  e  adj.}
\end{itemize}
O que troça.
\section{Trocados}
\begin{itemize}
\item {Grp. gram.:m. pl.}
\end{itemize}
\begin{itemize}
\item {Proveniência:(De \textunderscore trocar\textunderscore )}
\end{itemize}
Trocadilhos.
Lavores antigos, em panos de armas ou em vestidos.
\section{Trocamento}
\begin{itemize}
\item {Grp. gram.:m.}
\end{itemize}
\begin{itemize}
\item {Utilização:T. de Ceilão}
\end{itemize}
O mesmo que \textunderscore troca\textunderscore .
\section{Trocano}
\begin{itemize}
\item {Grp. gram.:m.}
\end{itemize}
\begin{itemize}
\item {Utilização:Bras}
\end{itemize}
Espécie de tambor guerreiro, feito de um tôro de madeira, concavado, e fechado nas extremidades por tábuas furadas no centro.
\section{Troca-queixos}
\begin{itemize}
\item {Grp. gram.:m.}
\end{itemize}
\begin{itemize}
\item {Utilização:Prov.}
\end{itemize}
\begin{itemize}
\item {Utilização:alg.}
\end{itemize}
O mesmo que \textunderscore sôco\textunderscore .
\section{Trocar}
\begin{itemize}
\item {Grp. gram.:v. t.}
\end{itemize}
Dar (uma coisa por outra).
Permutar.
Substituir.
Transformar.
Confundir.
Alterar.
Pôr de través; cruzar.
Alternar.
\section{Troçar}
\begin{itemize}
\item {Grp. gram.:v. t.}
\end{itemize}
\begin{itemize}
\item {Utilização:Pop.}
\end{itemize}
\begin{itemize}
\item {Grp. gram.:V. i.}
\end{itemize}
\begin{itemize}
\item {Proveniência:(De \textunderscore troça\textunderscore )}
\end{itemize}
Fazer escárneo ou zombaria de.
Escarnecer.
Motejar, zombar.
\section{Trocarte}
\begin{itemize}
\item {Grp. gram.:m.}
\end{itemize}
\begin{itemize}
\item {Proveniência:(Fr. \textunderscore trocart\textunderscore )}
\end{itemize}
Instrumento, usado em Veterinária para operações, como a punção do rumen.
\section{Trocas-baldrocas}
\begin{itemize}
\item {Grp. gram.:f. pl.}
\end{itemize}
\begin{itemize}
\item {Utilização:Pop.}
\end{itemize}
Negócios fraudulentos; tricas.
\section{Trochocéphalo}
\begin{itemize}
\item {fónica:co}
\end{itemize}
\begin{itemize}
\item {Grp. gram.:adj.}
\end{itemize}
\begin{itemize}
\item {Utilização:Zool.}
\end{itemize}
\begin{itemize}
\item {Proveniência:(Do gr. \textunderscore trokhos\textunderscore  + \textunderscore kephale\textunderscore )}
\end{itemize}
Que tem cabeça redonda.
\section{Trochodendro}
\begin{itemize}
\item {fónica:co}
\end{itemize}
\begin{itemize}
\item {Grp. gram.:m.}
\end{itemize}
\begin{itemize}
\item {Proveniência:(Do gr. \textunderscore trokhos\textunderscore  + \textunderscore dendron\textunderscore )}
\end{itemize}
Gênero de plantas magnoliáceas.
\section{Trochoéla}
\begin{itemize}
\item {Grp. gram.:f.}
\end{itemize}
\begin{itemize}
\item {Utilização:Prov.}
\end{itemize}
Bacalhau.
\section{Trochóide}
\begin{itemize}
\item {fónica:coi}
\end{itemize}
\begin{itemize}
\item {Grp. gram.:adj.}
\end{itemize}
\begin{itemize}
\item {Utilização:Anat.}
\end{itemize}
\begin{itemize}
\item {Proveniência:(Do gr. \textunderscore trokos\textunderscore  + \textunderscore eidos\textunderscore )}
\end{itemize}
Semelhante a uma roda.
Diz-se da articulação em que um ôsso gira sôbre outro.
\section{Trochoídeo}
\begin{itemize}
\item {fónica:co}
\end{itemize}
\begin{itemize}
\item {Grp. gram.:adj.}
\end{itemize}
\begin{itemize}
\item {Utilização:Anat.}
\end{itemize}
\begin{itemize}
\item {Proveniência:(Do gr. \textunderscore trokos\textunderscore  + \textunderscore eidos\textunderscore )}
\end{itemize}
Semelhante a uma roda.
Diz-se da articulação em que um ôsso gira sôbre outro.
\section{Trochólica}
\begin{itemize}
\item {fónica:có}
\end{itemize}
\begin{itemize}
\item {Grp. gram.:f.}
\end{itemize}
\begin{itemize}
\item {Proveniência:(Do gr. \textunderscore trokhos\textunderscore )}
\end{itemize}
Nome, que alguns déram á parte da Mecânica, que trata dos movimentos circulares.
\section{Trochoséride}
\begin{itemize}
\item {fónica:co}
\end{itemize}
\begin{itemize}
\item {Grp. gram.:f.}
\end{itemize}
Gênero de plantas, da fam. das synanthéreas.
\section{Trociscação}
\begin{itemize}
\item {Grp. gram.:f.}
\end{itemize}
Acto ou effeito de trociscar.
\section{Trociscar}
\begin{itemize}
\item {Grp. gram.:v. t.}
\end{itemize}
\begin{itemize}
\item {Proveniência:(De \textunderscore trocisco\textunderscore ^2)}
\end{itemize}
Reduzir a trociscos; reduzir a fragmentos.
\section{Trocisco}
\begin{itemize}
\item {Grp. gram.:m.}
\end{itemize}
\begin{itemize}
\item {Proveniência:(Do lat. \textunderscore trochiscus\textunderscore )}
\end{itemize}
Medicamento sólido, composto de substâncias pulverizadas e tendo fórma cónica, pyramidal, cúbica, etc.
Cp. \textunderscore trochisco\textunderscore .
\section{Trocisco}
\begin{itemize}
\item {Grp. gram.:m.}
\end{itemize}
\begin{itemize}
\item {Utilização:Ant.}
\end{itemize}
\begin{itemize}
\item {Proveniência:(De \textunderscore trôço\textunderscore )}
\end{itemize}
Pequeno trôço, fragmento.
\section{Trocista}
\begin{itemize}
\item {Grp. gram.:m. ,  f.  e  adj.}
\end{itemize}
\begin{itemize}
\item {Proveniência:(De \textunderscore troça\textunderscore )}
\end{itemize}
Pessôa, que gosta de troçar ou escarnecer.
\section{Trôco}
\begin{itemize}
\item {Grp. gram.:m.}
\end{itemize}
\begin{itemize}
\item {Utilização:Fam.}
\end{itemize}
\begin{itemize}
\item {Proveniência:(De \textunderscore trocar\textunderscore )}
\end{itemize}
O mesmo que \textunderscore troca\textunderscore .
Pequenas moédas, que constituem valor igual ao de uma só moéda, ou nota do Banco.
Demasia; o que se recebe do vendedor a quem se pagou um objecto com moéda superior ao preço ajustado.
Réplica.
Resposta prompta, resposta a tempo: \textunderscore offendeu-me, mas dei-lhe o trôco\textunderscore .
\section{Trôço}
\begin{itemize}
\item {Grp. gram.:m.}
\end{itemize}
\begin{itemize}
\item {Grp. gram.:Pl.}
\end{itemize}
\begin{itemize}
\item {Utilização:Prov.}
\end{itemize}
\begin{itemize}
\item {Utilização:minh.}
\end{itemize}
\begin{itemize}
\item {Utilização:Prov.}
\end{itemize}
\begin{itemize}
\item {Utilização:alg.}
\end{itemize}
O mesmo que \textunderscore trôcho\textunderscore .
Pedaço ou fragmento de madeira.
Cada uma das aduelas de um molde de canhão.
Obra de marinheiro, feita de fios.
Corpo de tropas; porção de gente; rancho, grupo.
Couve de pé alto.
Erva ou palha traçada ou cortada:«\textunderscore tu sustentas-te a favas e eu a troços.\textunderscore »J. de Deus.
(Cp. cast. \textunderscore trozo\textunderscore )
\section{Trococéfalo}
\begin{itemize}
\item {Grp. gram.:adj.}
\end{itemize}
\begin{itemize}
\item {Utilização:Zool.}
\end{itemize}
\begin{itemize}
\item {Proveniência:(Do gr. \textunderscore trokhos\textunderscore  + \textunderscore kephale\textunderscore )}
\end{itemize}
Que tem cabeça redonda.
\section{Trocodendro}
\begin{itemize}
\item {Grp. gram.:m.}
\end{itemize}
\begin{itemize}
\item {Proveniência:(Do gr. \textunderscore trokhos\textunderscore  + \textunderscore dendron\textunderscore )}
\end{itemize}
Gênero de plantas magnoliáceas.
\section{Trocóide}
\begin{itemize}
\item {Grp. gram.:adj.}
\end{itemize}
\begin{itemize}
\item {Utilização:Anat.}
\end{itemize}
\begin{itemize}
\item {Proveniência:(Do gr. \textunderscore trokos\textunderscore  + \textunderscore eidos\textunderscore )}
\end{itemize}
Semelhante a uma roda.
Diz-se da articulação em que um ôsso gira sôbre outro.
\section{Trocoídeo}
\begin{itemize}
\item {Grp. gram.:adj.}
\end{itemize}
\begin{itemize}
\item {Utilização:Anat.}
\end{itemize}
\begin{itemize}
\item {Proveniência:(Do gr. \textunderscore trokos\textunderscore  + \textunderscore eidos\textunderscore )}
\end{itemize}
Semelhante a uma roda.
Diz-se da articulação em que um ôsso gira sôbre outro.
\section{Trocólica}
\begin{itemize}
\item {Grp. gram.:f.}
\end{itemize}
\begin{itemize}
\item {Proveniência:(Do gr. \textunderscore trokhos\textunderscore )}
\end{itemize}
Nome, que alguns déram á parte da Mecânica, que trata dos movimentos circulares.
\section{Trocoséride}
\begin{itemize}
\item {Grp. gram.:f.}
\end{itemize}
Gênero de plantas, da fam. das sinantéreas.
\section{Troços-grossos}
\begin{itemize}
\item {Grp. gram.:m. Pl.}
\end{itemize}
Maquinismo, para a primeira torcedura, nas fábricas de fiação.
\section{Troçulho}
\begin{itemize}
\item {Grp. gram.:m.}
\end{itemize}
\begin{itemize}
\item {Utilização:Bras. do N}
\end{itemize}
\begin{itemize}
\item {Utilização:Pleb.}
\end{itemize}
\begin{itemize}
\item {Proveniência:(De \textunderscore troço\textunderscore )}
\end{itemize}
O mesmo que \textunderscore cagalhão\textunderscore .
\section{Trofa}
\begin{itemize}
\item {Grp. gram.:f.}
\end{itemize}
\begin{itemize}
\item {Utilização:Prov.}
\end{itemize}
Capa de junco; palhota.
\section{Troféu}
\begin{itemize}
\item {Grp. gram.:m.}
\end{itemize}
\begin{itemize}
\item {Utilização:Fig.}
\end{itemize}
\begin{itemize}
\item {Proveniência:(Do lat. \textunderscore tropaeum\textunderscore )}
\end{itemize}
Despojos de inimigo vencido.
Qualquer objecto, que se expõe em público, em commemoração de uma victória.
Reunião de armas, ou de outros objectos guerreiros, que servem de ornamento ou commemoram uma victória.
Representação dos attributos peculiares a uma sciência ou arte, em Pintura e Escultura.
O mesmo que \textunderscore victória\textunderscore .
\section{Trogalheira}
\begin{itemize}
\item {Grp. gram.:f.}
\end{itemize}
\begin{itemize}
\item {Utilização:Fam.}
\end{itemize}
\begin{itemize}
\item {Proveniência:(De \textunderscore trogalho\textunderscore )}
\end{itemize}
Mulhér desajeitada ou mal vestida.
\section{Trogalho}
\begin{itemize}
\item {Grp. gram.:m.}
\end{itemize}
\begin{itemize}
\item {Utilização:Pop.}
\end{itemize}
\begin{itemize}
\item {Utilização:Prov.}
\end{itemize}
\begin{itemize}
\item {Utilização:trasm.}
\end{itemize}
Pequena corda para atar.
Pessôa desajeitada.
(Por \textunderscore torgalho\textunderscore , do lat. \textunderscore torquere\textunderscore ?)
\section{Troge}
\begin{itemize}
\item {Grp. gram.:m.}
\end{itemize}
\begin{itemize}
\item {Proveniência:(Do gr. \textunderscore troa\textunderscore )}
\end{itemize}
Gênero de insectos coleópteros pentâmeros.
\section{Trógio}
\begin{itemize}
\item {Grp. gram.:m.}
\end{itemize}
\begin{itemize}
\item {Proveniência:(Do gr. \textunderscore troa\textunderscore )}
\end{itemize}
Gênero de insectos coleópteros pentâmeros.
\section{Troglodita}
\begin{itemize}
\item {Grp. gram.:m.  e  f.}
\end{itemize}
\begin{itemize}
\item {Proveniência:(Lat. \textunderscore troglodytae\textunderscore )}
\end{itemize}
Pessôa, que vive debaixo da terra, como os mineiros.
Membro de um antigo povo africano que vivia em cavernas.
Membro de qualquer tríbo prehistórica, que vivia em cavernas ou construía moradas subterrâneas.
Gênero de quadrúmanos, que compreende o chimpanzé.
Gênero de pássaros dentirostros, a que pertence a carriça.
\section{Troglodite}
\begin{itemize}
\item {Grp. gram.:adj.}
\end{itemize}
O mesmo ou melhór que \textunderscore troglodita\textunderscore .
(Cp. lat. \textunderscore troglodytis\textunderscore )
\section{Troglodyta}
\begin{itemize}
\item {Grp. gram.:m.  e  f.}
\end{itemize}
\begin{itemize}
\item {Proveniência:(Lat. \textunderscore troglodytae\textunderscore )}
\end{itemize}
Pessôa, que vive debaixo da terra, como os mineiros.
Membro de um antigo povo africano que vivia em cavernas.
Membro de qualquer tríbo prehistórica, que vivia em cavernas ou construía moradas subterrâneas.
Gênero de quadrúmanos, que comprehende o chimpanzé.
Gênero de pássaros dentirostros, a que pertence a carriça.
\section{Troglodyte}
\begin{itemize}
\item {Grp. gram.:adj.}
\end{itemize}
O mesmo ou melhór que \textunderscore troglodyta\textunderscore .
(Cp. lat. \textunderscore troglodytis\textunderscore )
\section{Troficidade}
\begin{itemize}
\item {Grp. gram.:f.}
\end{itemize}
\begin{itemize}
\item {Proveniência:(De \textunderscore trófico\textunderscore )}
\end{itemize}
Uma das quatro funções do sistema nervoso, (troficidade psiquismo, sensibilidade e motilidade) Cf. Al. de Castro, \textunderscore Desordens da Marcha\textunderscore .
\section{Trófico}
\begin{itemize}
\item {Grp. gram.:adj.}
\end{itemize}
\begin{itemize}
\item {Proveniência:(Do gr. \textunderscore trophe\textunderscore )}
\end{itemize}
Relativo á alimentação.
\section{Trófide}
\begin{itemize}
\item {Grp. gram.:f.}
\end{itemize}
Gênero de plantas artocárpeas.
\section{Trofologia}
\begin{itemize}
\item {Grp. gram.:f.}
\end{itemize}
\begin{itemize}
\item {Proveniência:(Do gr. \textunderscore trophe\textunderscore  + \textunderscore logos\textunderscore )}
\end{itemize}
Tratado, á cêrca do regime alimentar.
\section{Trofológico}
\begin{itemize}
\item {Grp. gram.:adj.}
\end{itemize}
Relativo á trofologia.
\section{Trofoneurose}
\begin{itemize}
\item {Grp. gram.:f.}
\end{itemize}
\begin{itemize}
\item {Proveniência:(Do gr. \textunderscore trophe\textunderscore  + \textunderscore neuron\textunderscore )}
\end{itemize}
Neurose, que ataca os tecidos, alterando-lhes a fórma ou o volume.
\section{Trofosperma}
\begin{itemize}
\item {Grp. gram.:m.}
\end{itemize}
\begin{itemize}
\item {Proveniência:(Do gr. \textunderscore trophe\textunderscore  + \textunderscore sperma\textunderscore )}
\end{itemize}
Saliência da cavidade interior do pericarpo, a que estão ligadas as sementes.
\section{Trombeiro}
\begin{itemize}
\item {Grp. gram.:m.}
\end{itemize}
\begin{itemize}
\item {Utilização:Ant.}
\end{itemize}
\begin{itemize}
\item {Proveniência:(De \textunderscore tromba\textunderscore )}
\end{itemize}
Gênero de peixes acanthopterýgios.
Tocador de trombeta ou de outros instrumentos de sôpro. Cf. Fern. Lopes, \textunderscore Chrón. de D. Pedro.\textunderscore 
\section{Trombejar}
\begin{itemize}
\item {Grp. gram.:v. i.}
\end{itemize}
\begin{itemize}
\item {Utilização:Fig.}
\end{itemize}
Agitar a tromba.
Bater com a tromba.
Tornar-se trombudo.
\section{Trombelão}
\begin{itemize}
\item {Grp. gram.:m.}
\end{itemize}
Espécie de estramónio.
\section{Trombeta}
\begin{itemize}
\item {fónica:bê}
\end{itemize}
\begin{itemize}
\item {Grp. gram.:f.}
\end{itemize}
\begin{itemize}
\item {Utilização:Fig.}
\end{itemize}
\begin{itemize}
\item {Grp. gram.:M.}
\end{itemize}
\begin{itemize}
\item {Proveniência:(Do gr. \textunderscore trompette\textunderscore )}
\end{itemize}
Instrumento de sopro, feito de cobre ou de outro metal.
Pessôa chocalheira.
Aquelle que toca trombeta.
\section{Trombeta}
\begin{itemize}
\item {fónica:bê}
\end{itemize}
\begin{itemize}
\item {Grp. gram.:f.}
\end{itemize}
\begin{itemize}
\item {Utilização:Bras}
\end{itemize}
\begin{itemize}
\item {Proveniência:(De \textunderscore tromba\textunderscore )}
\end{itemize}
Peixe de Portugal.
Máscara de coiro, apposta ao focinho dos cavallos, para que êstes não comam nem bebam, fóra da ração.
\section{Trombeta-branca}
\begin{itemize}
\item {Grp. gram.:f.}
\end{itemize}
Espécie de datura, (\textunderscore datura suaveolens\textunderscore , Humb.).
\section{Trombetada}
\begin{itemize}
\item {Grp. gram.:f.}
\end{itemize}
Toque rápido de trombeta.
\section{Trombetão}
\begin{itemize}
\item {Grp. gram.:m.}
\end{itemize}
\begin{itemize}
\item {Proveniência:(De \textunderscore tronbeta\textunderscore ^2)}
\end{itemize}
Planta solânea, de grandes flôres brancas muito aromáticas, em fórma de cálice.
Planta análoga, de flôres roxas.
\section{Trombetear}
\begin{itemize}
\item {Grp. gram.:v. i.}
\end{itemize}
\begin{itemize}
\item {Utilização:Bras}
\end{itemize}
Tocar trombeta. Cf. Camillo, \textunderscore Corja\textunderscore , 237.
Imitar o som da trombeta.
\section{Trombeteira}
\begin{itemize}
\item {Grp. gram.:f.}
\end{itemize}
O mesmo que \textunderscore trombetão\textunderscore .
\section{Trombeteiro}
\begin{itemize}
\item {Grp. gram.:m.}
\end{itemize}
\begin{itemize}
\item {Proveniência:(De \textunderscore trombeta\textunderscore ^1)}
\end{itemize}
Tocador de trombeta.
Fabricante de trombetas.
Espécie de mosquito.
Ave pernalta da América do Sul.
\section{Trombicar}
\begin{itemize}
\item {Grp. gram.:v. i.}
\end{itemize}
\begin{itemize}
\item {Utilização:beir.}
\end{itemize}
\begin{itemize}
\item {Utilização:Pleb.}
\end{itemize}
\begin{itemize}
\item {Utilização:Fig.}
\end{itemize}
\begin{itemize}
\item {Proveniência:(De \textunderscore tromba\textunderscore ? ou relaciona-se com \textunderscore trompicar\textunderscore ?)}
\end{itemize}
O mesmo que \textunderscore fornicar\textunderscore .
O mesmo que \textunderscore burlar\textunderscore .
\section{Trombífero}
\begin{itemize}
\item {Grp. gram.:m.}
\end{itemize}
\begin{itemize}
\item {Utilização:Burl.}
\end{itemize}
\begin{itemize}
\item {Proveniência:(Fr. \textunderscore tromblon\textunderscore )}
\end{itemize}
Chapéu alto, dos que íam alargando até o tampo.
\section{Trombombó}
\begin{itemize}
\item {Grp. gram.:m.}
\end{itemize}
\begin{itemize}
\item {Utilização:Bras. do Rio}
\end{itemize}
Modo especial de pescar taínhas, fixando esteiras nos bordos da canôa.
\section{Trombóne}
\begin{itemize}
\item {Grp. gram.:m.}
\end{itemize}
\begin{itemize}
\item {Proveniência:(It. \textunderscore trombone\textunderscore )}
\end{itemize}
Instrumento músico de metal, composto de dois tubos encaixados um no outro, e cujo timbre é mais ou menos semelhante ao da trombeta.
Tocador de trombóne.
\section{Trombonista}
\begin{itemize}
\item {Grp. gram.:m.}
\end{itemize}
Tocador de trombóne.
\section{Trombóno}
\begin{itemize}
\item {Grp. gram.:adj.}
\end{itemize}
Que sôa como trombóne: \textunderscore voz trombóna\textunderscore . Cf. Filinto, XII, 200.
\section{Trombudo}
\begin{itemize}
\item {Grp. gram.:adj.}
\end{itemize}
\begin{itemize}
\item {Utilização:Fig.}
\end{itemize}
Que tem tromba.
Carrancudo; que tem aspecto sombrio ou torvo.
\section{Trompa}
\begin{itemize}
\item {Grp. gram.:f.}
\end{itemize}
\begin{itemize}
\item {Utilização:Phýs.}
\end{itemize}
Instrumento de sopro, semelhante á trombeta, mas maior e mais sonoro.
Nome de alguns órgãos animaes, de fórma tubular.
Instrumento de vidro, usado em laboratórios chímicos e destinado a fazer a aspiração do ar, para auxiliar, por exemplo, as filtrações.
(Cp. \textunderscore tromba\textunderscore )
\section{Trompa}
\begin{itemize}
\item {Grp. gram.:f.}
\end{itemize}
\begin{itemize}
\item {Proveniência:(Fr. \textunderscore trompe\textunderscore )}
\end{itemize}
O mesmo que \textunderscore perxina\textunderscore .
\section{Trompão}
\begin{itemize}
\item {Grp. gram.:m.}
\end{itemize}
Grande trompa; trombão, trombóne.
\section{Trompázio}
\begin{itemize}
\item {Grp. gram.:m.}
\end{itemize}
\begin{itemize}
\item {Utilização:Bras}
\end{itemize}
Grande bofetada na bôca ou no nariz.
\section{Trompicão}
\begin{itemize}
\item {Grp. gram.:m.}
\end{itemize}
\begin{itemize}
\item {Utilização:Prov.}
\end{itemize}
\begin{itemize}
\item {Utilização:trasm.}
\end{itemize}
\begin{itemize}
\item {Utilização:alg.}
\end{itemize}
\begin{itemize}
\item {Proveniência:(De \textunderscore trompicar\textunderscore )}
\end{itemize}
Tropeção de bêstas.
Qualquer tropeção.
\section{Trompicar}
\begin{itemize}
\item {Grp. gram.:v. i.}
\end{itemize}
\begin{itemize}
\item {Utilização:Prov.}
\end{itemize}
\begin{itemize}
\item {Utilização:trasm.}
\end{itemize}
\begin{itemize}
\item {Utilização:alg.}
\end{itemize}
O mesmo que \textunderscore tropeçar\textunderscore .
(Cast. \textunderscore trompicar\textunderscore )
\section{Trompim}
\begin{itemize}
\item {Grp. gram.:m.}
\end{itemize}
\begin{itemize}
\item {Utilização:Des.}
\end{itemize}
Trompa pequena.
\section{Trompista}
\begin{itemize}
\item {Grp. gram.:m.}
\end{itemize}
Fabricante de trompas. Cf. E. Vieira, \textunderscore Diccion. Mus.\textunderscore , vb. \textunderscore sax-horn.\textunderscore 
\section{Tromplear}
\begin{itemize}
\item {Grp. gram.:v. t.}
\end{itemize}
Diz-se do toiro, quando toca o toireiro com o focinho, sem contudo o fazer cair.
(Cp. cast. \textunderscore trompillar\textunderscore )
\section{Tronante}
\begin{itemize}
\item {Grp. gram.:adj.}
\end{itemize}
Que trona.
\section{Tronar}
\begin{itemize}
\item {Grp. gram.:v. i.}
\end{itemize}
\begin{itemize}
\item {Proveniência:(De \textunderscore trom\textunderscore )}
\end{itemize}
Tronar; trovejar.
\section{Troncácia}
\begin{itemize}
\item {Grp. gram.:f.}
\end{itemize}
\begin{itemize}
\item {Proveniência:(De \textunderscore tronco\textunderscore ^1)}
\end{itemize}
Espécie de imposto, que se pagava ao tronqueiro-mór, pelo peixe que se pescava em dias defesos.
\section{Tropeçamento}
\begin{itemize}
\item {Grp. gram.:m.}
\end{itemize}
Acto ou effeito de tropeçar.
\section{Tropeção}
\begin{itemize}
\item {Grp. gram.:m.}
\end{itemize}
O mesmo que \textunderscore tropeçamento\textunderscore .
\section{Tropeçar}
\begin{itemize}
\item {Grp. gram.:v. i.}
\end{itemize}
\begin{itemize}
\item {Utilização:Fig.}
\end{itemize}
Dar com o pé involuntariamente.
Esbarrar.
Incorrer.
Errar.
Cambalear.
Hesitar.
(Cast. \textunderscore tropezar\textunderscore )
\section{Tropecina}
\begin{itemize}
\item {Grp. gram.:f.}
\end{itemize}
\begin{itemize}
\item {Utilização:Prov.}
\end{itemize}
\begin{itemize}
\item {Utilização:trasm.}
\end{itemize}
\begin{itemize}
\item {Proveniência:(De \textunderscore tropeçar\textunderscore )}
\end{itemize}
O mesmo que \textunderscore bebedeira\textunderscore .
\section{Tropêço}
\begin{itemize}
\item {Grp. gram.:m.}
\end{itemize}
\begin{itemize}
\item {Utilização:Fig.}
\end{itemize}
\begin{itemize}
\item {Proveniência:(De \textunderscore tropeçar\textunderscore )}
\end{itemize}
Objecto, em que se tropeça.
Obstáculo, travanca.
\section{Tropêço}
\begin{itemize}
\item {Grp. gram.:m.}
\end{itemize}
\begin{itemize}
\item {Utilização:T. do Fundão}
\end{itemize}
O mesmo que \textunderscore tripeça\textunderscore .
\section{Tropeçudo}
\begin{itemize}
\item {Grp. gram.:adj.}
\end{itemize}
Que tropeça muitas vezes.
\section{Trôpego}
\begin{itemize}
\item {Grp. gram.:adj.}
\end{itemize}
Que anda com difficuldade.
Que não póde mover os membros ou que os move difficilmente.
(Cp. \textunderscore tropicar\textunderscore )
\section{Tropeiro}
\begin{itemize}
\item {Grp. gram.:m.}
\end{itemize}
\begin{itemize}
\item {Utilização:Bras}
\end{itemize}
\begin{itemize}
\item {Proveniência:(De \textunderscore tropa\textunderscore )}
\end{itemize}
Recoveiro.
Aquelle que conduz bêstas de carga ou manadas de gado grosso, como cavallos, bois, etc.
Empresário de transportes.
\section{Tropel}
\begin{itemize}
\item {Grp. gram.:m.}
\end{itemize}
\begin{itemize}
\item {Proveniência:(De \textunderscore tropa\textunderscore )}
\end{itemize}
Ruído ou tumulto de grande porção de gente, que anda ou se agita.
Balbúrdia, grande confusão.
Estrondo, feito com os pés.
Conjunto de muitas coisas em desordem, movendo-se.
Grande ruído, produzido pelo andar de cavallos.
\section{Tropelão}
\begin{itemize}
\item {Grp. gram.:adj.}
\end{itemize}
\begin{itemize}
\item {Utilização:Bras. do N}
\end{itemize}
Diz-se do cavallo, que tropeça muito ou dá muitas vezes com o pé.
(Cp. \textunderscore tropel\textunderscore )
\section{Tropelas}
\begin{itemize}
\item {fónica:trô-pê}
\end{itemize}
\begin{itemize}
\item {Grp. gram.:m.}
\end{itemize}
\begin{itemize}
\item {Utilização:Prov.}
\end{itemize}
\begin{itemize}
\item {Utilização:minh.}
\end{itemize}
Homem pesado, que se move com difficuldade.
(Cp. \textunderscore trôpego\textunderscore )
\section{Tropelha}
\begin{itemize}
\item {fónica:pê}
\end{itemize}
\begin{itemize}
\item {Grp. gram.:f.}
\end{itemize}
\begin{itemize}
\item {Utilização:Bras}
\end{itemize}
Magote de cavallos, com uma égua branca.
O mesmo que \textunderscore tropilha\textunderscore ?
\section{Tropelia}
\begin{itemize}
\item {Grp. gram.:f.}
\end{itemize}
\begin{itemize}
\item {Utilização:Fig.}
\end{itemize}
Effeito de tropel.
Barulho ou damno, produzido por gente em tropel.
Astúcia.
Travessura.
Prejuízo; maus tratos.
\section{Tropeliar}
\begin{itemize}
\item {Grp. gram.:v. i.}
\end{itemize}
Fazer tropel ou tropelia. Cf. Castilho, \textunderscore Metam.\textunderscore , XLII.
\section{Tropeóleas}
\begin{itemize}
\item {Grp. gram.:f. pl.}
\end{itemize}
Família de plantas, que tem por typo as chagas.
(Fem. pl. de \textunderscore tropeóleo\textunderscore )
\section{Tropeóleo}
\begin{itemize}
\item {Grp. gram.:adj.}
\end{itemize}
\begin{itemize}
\item {Proveniência:(Do gr. \textunderscore tropaiolos\textunderscore )}
\end{itemize}
Relativo ou semelhante ás chaga, (planta).
\section{Trophéu}
\begin{itemize}
\item {Grp. gram.:m.}
\end{itemize}
(V. \textunderscore troféu\textunderscore . \textunderscore Trophéu\textunderscore  não tem razão etym.; é naturalmente imitação francesa)
\section{Trophicidade}
\begin{itemize}
\item {Grp. gram.:f.}
\end{itemize}
\begin{itemize}
\item {Proveniência:(De \textunderscore tróphico\textunderscore )}
\end{itemize}
Uma das quatro funcções do systema nervoso, (trophicidade psychismo, sensibilidade e motilidade) Cf. Al. de Castro, \textunderscore Desordens da Marcha\textunderscore .
\section{Tróphico}
\begin{itemize}
\item {Grp. gram.:adj.}
\end{itemize}
\begin{itemize}
\item {Proveniência:(Do gr. \textunderscore trophe\textunderscore )}
\end{itemize}
Relativo á alimentação.
\section{Tróphide}
\begin{itemize}
\item {Grp. gram.:f.}
\end{itemize}
Gênero de plantas artocárpeas.
\section{Trophologia}
\begin{itemize}
\item {Grp. gram.:f.}
\end{itemize}
\begin{itemize}
\item {Proveniência:(Do gr. \textunderscore trophe\textunderscore  + \textunderscore logos\textunderscore )}
\end{itemize}
Tratado, á cêrca do regime alimentar.
\section{Trophológico}
\begin{itemize}
\item {Grp. gram.:adj.}
\end{itemize}
Relativo á trophologia.
\section{Trophoneurose}
\begin{itemize}
\item {Grp. gram.:f.}
\end{itemize}
\begin{itemize}
\item {Proveniência:(Do gr. \textunderscore trophe\textunderscore  + \textunderscore neuron\textunderscore )}
\end{itemize}
Neurose, que ataca os tecidos, alterando-lhes a fórma ou o volume.
\section{Trophosperma}
\begin{itemize}
\item {Grp. gram.:m.}
\end{itemize}
\begin{itemize}
\item {Proveniência:(Do gr. \textunderscore trophe\textunderscore  + \textunderscore sperma\textunderscore )}
\end{itemize}
Saliência da cavidade interior do pericarpo, a que estão ligadas as sementes.
\section{Tropical}
\begin{itemize}
\item {Grp. gram.:adj.}
\end{itemize}
\begin{itemize}
\item {Proveniência:(De \textunderscore trópico\textunderscore )}
\end{itemize}
Relativo aos trópicos.
Relativo ás regiões que ficam entre os trópicos: \textunderscore plantas tropicaes\textunderscore .
Relativo ao clima dessas regiões: \textunderscore calor tropical\textunderscore .
O mesmo que \textunderscore intertropical\textunderscore .
\section{Tropicão}
\begin{itemize}
\item {Grp. gram.:m.}
\end{itemize}
Acto ou effeito de \textunderscore tropicar\textunderscore .
\section{Tropicar}
\begin{itemize}
\item {Grp. gram.:v. i.}
\end{itemize}
Tropeçar muitas vezes.
(Cp. \textunderscore trompicar\textunderscore )
\section{Trópico}
\begin{itemize}
\item {Grp. gram.:m.}
\end{itemize}
\begin{itemize}
\item {Utilização:Geogr.}
\end{itemize}
\begin{itemize}
\item {Utilização:Ext.}
\end{itemize}
\begin{itemize}
\item {Grp. gram.:Adj.}
\end{itemize}
\begin{itemize}
\item {Proveniência:(Gr. \textunderscore tropikos\textunderscore )}
\end{itemize}
Cada um dos dois parallelos terrestres, que separam a zona tórrida das zonas temperadas.
Regiões, comprehendidas entre os trópicos.
Ave palmípede, que vive entre os trópicos.
Diz-se do anno, que comprehende o tempo decorrido entre duas passagens successivas do centro do Sol pelo equinócio da Primavera.
\section{T}
\begin{itemize}
\item {fónica:tê}
\end{itemize}
\begin{itemize}
\item {Grp. gram.:m.}
\end{itemize}
\begin{itemize}
\item {Utilização:Ant.}
\end{itemize}
\begin{itemize}
\item {Grp. gram.:Adj.}
\end{itemize}
Vigésima letra do alphabeto português.
Como letra numeral, valia 160 e, com um til, 160:000, e valia também 1:000.
Que, numa série de vinte, occupa o último lugar.
\section{Tá}
\begin{itemize}
\item {Grp. gram.:prep.}
\end{itemize}
\begin{itemize}
\item {Utilização:Ant.}
\end{itemize}
O mesmo que \textunderscore atá\textunderscore  ou \textunderscore até\textunderscore . Cf. G. Vicente.
\section{Tá!}
\begin{itemize}
\item {Grp. gram.:interj.}
\end{itemize}
(designativa de \textunderscore suspensão\textunderscore , \textunderscore interrupção\textunderscore )
O mesmo que \textunderscore tá-tá!\textunderscore .
\section{Ta}
\begin{itemize}
\item {fónica:tâ}
\end{itemize}
\begin{itemize}
\item {Grp. gram.:pron.}
\end{itemize}
\begin{itemize}
\item {Utilização:Ant.}
\end{itemize}
O mesmo que \textunderscore tua\textunderscore ^1:«\textunderscore e ta mãy bôa mulher\textunderscore ». G. Vicente, \textunderscore Juíz da Beira\textunderscore .
\section{Taá}
\begin{itemize}
\item {Grp. gram.:f.}
\end{itemize}
\begin{itemize}
\item {Utilização:Ant.}
\end{itemize}
Cabeça de concelho ou de julgado.
\section{Taalique}
\begin{itemize}
\item {Grp. gram.:m.}
\end{itemize}
Gênero de caracteres de escrita, de que se servem os Persas.
\section{Taba}
\begin{itemize}
\item {Grp. gram.:f.}
\end{itemize}
Habitação de Índio, na América do Sul.
Pequena povoação de indígenas do Brasil. Cf. \textunderscore Caramuru\textunderscore , X, 24.
\section{Tabacal}
\begin{itemize}
\item {Grp. gram.:m.}
\end{itemize}
\begin{itemize}
\item {Grp. gram.:Adj.}
\end{itemize}
\begin{itemize}
\item {Proveniência:(De \textunderscore tabaco\textunderscore )}
\end{itemize}
Plantação de tabaco.
Nome de uma erva, cujo pé se usa como rapé.
O mesmo que \textunderscore tabaqueiro\textunderscore .
\section{Tabacão}
\begin{itemize}
\item {Grp. gram.:m.}
\end{itemize}
\begin{itemize}
\item {Utilização:Des.}
\end{itemize}
\begin{itemize}
\item {Utilização:Pop.}
\end{itemize}
\begin{itemize}
\item {Proveniência:(De \textunderscore tabaco\textunderscore )}
\end{itemize}
Aquelle que tabaqueia demasiadamente.
\section{Tabacaria}
\begin{itemize}
\item {Grp. gram.:f.}
\end{itemize}
Loja ou lugar, onde se vende tabaco.
\section{Tabacino}
\begin{itemize}
\item {Grp. gram.:adj.}
\end{itemize}
Relativo a tabaco.
Diz-se especialmente da ophthalmia, procedida do abuso do tabaco.
\section{Tabaco}
\begin{itemize}
\item {Grp. gram.:m.}
\end{itemize}
\begin{itemize}
\item {Proveniência:(T. caraíba?)}
\end{itemize}
Gênero de plantas solanáceas; nicociana.
Nome de várias preparações de fôlhas desta planta, para se fumarem, cheirarem ou mascarem.
\section{Tabacose}
\begin{itemize}
\item {Grp. gram.:f.}
\end{itemize}
\begin{itemize}
\item {Utilização:Med.}
\end{itemize}
Pneumoconiose, peculiar aos operários, que se empregam na fabricação do tabaco.
\section{Tabacoso}
\begin{itemize}
\item {Grp. gram.:adj.}
\end{itemize}
\begin{itemize}
\item {Utilização:Prov.}
\end{itemize}
\begin{itemize}
\item {Proveniência:(De \textunderscore tabaco\textunderscore )}
\end{itemize}
Diz-se do lenço, com que se limpa o pingo do rapé, que cai do nariz.
\section{Tabafeia}
\begin{itemize}
\item {Grp. gram.:f.}
\end{itemize}
\begin{itemize}
\item {Utilização:Prov.}
\end{itemize}
\begin{itemize}
\item {Utilização:trasm.}
\end{itemize}
Chouriço, recheado de carne e intestinos de várias qualidades, e próprio para se comer logo depois de feito.
\section{Tabafeira}
\begin{itemize}
\item {Grp. gram.:f.}
\end{itemize}
\begin{itemize}
\item {Utilização:Prov.}
\end{itemize}
\begin{itemize}
\item {Utilização:trasm.}
\end{itemize}
O mesmo que \textunderscore tabafeia\textunderscore .
\section{Tabagismo}
\begin{itemize}
\item {Grp. gram.:m.}
\end{itemize}
\begin{itemize}
\item {Utilização:Neol.}
\end{itemize}
\begin{itemize}
\item {Proveniência:(Do fr. \textunderscore tabagie\textunderscore )}
\end{itemize}
Abuso do fumo do tabaco e dos que o mascam.
\section{Tabaibo}
\begin{itemize}
\item {Grp. gram.:m.}
\end{itemize}
\begin{itemize}
\item {Utilização:T. de Cabo Verde}
\end{itemize}
Planta, o mesmo que \textunderscore figueira-do-inferno\textunderscore .
\section{Tabajaras}
\begin{itemize}
\item {Grp. gram.:m. pl.}
\end{itemize}
Nação de Índios do Ceará, hoje civilizada.
\section{Tabalo}
\begin{itemize}
\item {Grp. gram.:m.}
\end{itemize}
Gênero de plantas capparídeas da Índia Portuguesa, (\textunderscore capparis heyneana\textunderscore , Wall.).
\section{Tabanca}
\begin{itemize}
\item {Grp. gram.:f.}
\end{itemize}
Povoação ou localidade, em alguns pontos da África, geralmente fortificada.
\section{Tabanga}
\begin{itemize}
\item {Grp. gram.:f.}
\end{itemize}
O mesmo que \textunderscore tabanca\textunderscore .
\section{Tabão}
\begin{itemize}
\item {Grp. gram.:m.}
\end{itemize}
O mesmo que \textunderscore tavão\textunderscore . Cf. B. Pereira, \textunderscore Prosódia\textunderscore , vb. \textunderscore oestrum\textunderscore .
\section{Tabaque}
\begin{itemize}
\item {Grp. gram.:m.}
\end{itemize}
\begin{itemize}
\item {Proveniência:(T. afr.)}
\end{itemize}
Pequena e elegante árvore da ilha de San-Thomé.
\section{Tabaque}
\begin{itemize}
\item {Grp. gram.:m.}
\end{itemize}
\begin{itemize}
\item {Utilização:Bras}
\end{itemize}
Espécie de tambor, feito de um tronco oco, guarnecido de coiro em uma das extremidades, e em que os Índios batem com as mãos em vez de baquetas.
(Cp. \textunderscore tabaque\textunderscore ^1)
\section{Tabaqueação}
\begin{itemize}
\item {Grp. gram.:f.}
\end{itemize}
Acto de \textunderscore tabaquear\textunderscore .
\section{Tabaquear}
\begin{itemize}
\item {Grp. gram.:v. t.  e  i.}
\end{itemize}
\begin{itemize}
\item {Proveniência:(De \textunderscore tabaco\textunderscore )}
\end{itemize}
Tomar pitadas de rapé ou tabaco.
Fumar.
\section{Tabaqueira}
\begin{itemize}
\item {Grp. gram.:f.}
\end{itemize}
\begin{itemize}
\item {Utilização:Des.}
\end{itemize}
\begin{itemize}
\item {Grp. gram.:Pl.}
\end{itemize}
\begin{itemize}
\item {Utilização:Pop.}
\end{itemize}
Caixa ou bolsa para tabaco.
Dava-se êste nome á palma da mão esquerda, encurvada, aproximando-se o pollegar do index, e na qual alguns tabaquistas deitavam o tabaco, para o aspirar pelo nariz.
Ventas.
\section{Tabaqueiro}
\begin{itemize}
\item {Grp. gram.:adj.}
\end{itemize}
\begin{itemize}
\item {Grp. gram.:M.  e  adj.}
\end{itemize}
Relativo a tabaco.
O que usa tabaco.
\section{Tabaquista}
\begin{itemize}
\item {Grp. gram.:m.  e  f.}
\end{itemize}
\begin{itemize}
\item {Proveniência:(De \textunderscore tabaco\textunderscore )}
\end{itemize}
Pessôa, que toma ou fuma tabaco.
\section{Tabardão}
\begin{itemize}
\item {Grp. gram.:m.}
\end{itemize}
\begin{itemize}
\item {Proveniência:(De \textunderscore tabardo\textunderscore )}
\end{itemize}
Homem rude, mal vestido? Camillo, \textunderscore Noites de Insómn.\textunderscore , VIII, 65.
\section{Tabardilha}
\begin{itemize}
\item {Grp. gram.:f.}
\end{itemize}
Pequeno tabardo.
\section{Tabardilho}
\begin{itemize}
\item {Grp. gram.:m.}
\end{itemize}
Febre de mau carácter, acompanhada de exanthemas.
(Cast. \textunderscore tabardillo\textunderscore )
\section{Tabardo}
\begin{itemize}
\item {Grp. gram.:m.}
\end{itemize}
\begin{itemize}
\item {Utilização:Prov.}
\end{itemize}
\begin{itemize}
\item {Utilização:minh.}
\end{itemize}
Capote antigo de capuz e mangas.
Casaco antigo de mulhér.
(Cast. \textunderscore tabardo\textunderscore )
\section{Tabaréo}
\begin{itemize}
\item {Grp. gram.:m.}
\end{itemize}
\begin{itemize}
\item {Utilização:Fig.}
\end{itemize}
\begin{itemize}
\item {Utilização:Bras}
\end{itemize}
\begin{itemize}
\item {Utilização:Des.}
\end{itemize}
Soldado bisonho.
Homem inhenho, acanhado.
Matuto; caipira.
Official ordinário ou mandrião.
\section{Tabaréu}
\begin{itemize}
\item {Grp. gram.:m.}
\end{itemize}
\begin{itemize}
\item {Utilização:Fig.}
\end{itemize}
\begin{itemize}
\item {Utilização:Bras}
\end{itemize}
\begin{itemize}
\item {Utilização:Des.}
\end{itemize}
Soldado bisonho.
Homem inhenho, acanhado.
Matuto; caipira.
Official ordinário ou mandrião.
\section{Tabarito}
\begin{itemize}
\item {Grp. gram.:m.}
\end{itemize}
\begin{itemize}
\item {Utilização:T. de Lamego}
\end{itemize}
Jôgo popular, espécie de chinquilho.
\section{Tabarôa}
\begin{itemize}
\item {Grp. gram.:f.}
\end{itemize}
\begin{itemize}
\item {Proveniência:(De \textunderscore tabaréu\textunderscore )}
\end{itemize}
Mulhér acanhada.
\section{Tabarro}
\begin{itemize}
\item {Grp. gram.:m.}
\end{itemize}
\begin{itemize}
\item {Proveniência:(It. \textunderscore tabarro\textunderscore )}
\end{itemize}
O mesmo que \textunderscore tabardo\textunderscore .
\section{Tabatinga}
\begin{itemize}
\item {Grp. gram.:f.}
\end{itemize}
\begin{itemize}
\item {Utilização:Bras}
\end{itemize}
Argilla, geralmente amarela, e, ás vezes, de outras côres, com que se caiam as paredes em algumas localidades.
(Corr. do tupi, \textunderscore tobatinga\textunderscore )
\section{Tabaxir}
\begin{itemize}
\item {Grp. gram.:m.}
\end{itemize}
Substância saccharina, extrahida do bambu.
Giz de alfaiate.
(Do ár.)
\section{Tabebuia}
\begin{itemize}
\item {Grp. gram.:f.}
\end{itemize}
O mesmo ou melhor que \textunderscore tabibuia\textunderscore .
\section{Tabeca}
\begin{itemize}
\item {Grp. gram.:f.}
\end{itemize}
\begin{itemize}
\item {Utilização:Ant.}
\end{itemize}
\begin{itemize}
\item {Proveniência:(De \textunderscore tábua\textunderscore )}
\end{itemize}
Fôrro de madeira, em que se pregam as ripas do telhado.
\section{Tabedáe}
\begin{itemize}
\item {Grp. gram.:m.}
\end{itemize}
Dança, peculiar aos povos de Timor.
\section{Tabefe}
\begin{itemize}
\item {Grp. gram.:m.}
\end{itemize}
\begin{itemize}
\item {Utilização:Pleb.}
\end{itemize}
\begin{itemize}
\item {Proveniência:(Do ár. \textunderscore tabikh\textunderscore )}
\end{itemize}
Iguaria, espécie de caldo grosso, feito de leite, açúcar e ovos.
Sôro de leite coalhado.
Bofetada, sopapo.
\section{Tabela}
\begin{itemize}
\item {Grp. gram.:f.}
\end{itemize}
\begin{itemize}
\item {Grp. gram.:Loc. adv.}
\end{itemize}
\begin{itemize}
\item {Proveniência:(Lat. \textunderscore tabella\textunderscore )}
\end{itemize}
Tábua pequena, quadro ou papel, em que se regista qualquer coisa.
Relação; catálogo; lista.
Índice.
Parte interna da borda do bilhar.
Caixilho, que contém as bolas móvies, com que se marca o número das carambolas, no bilhar.
Electuário em pastilhas.
\textunderscore Por tabela\textunderscore , o mesmo que \textunderscore indirectamente\textunderscore .
\section{Tabelar}
\begin{itemize}
\item {Grp. gram.:adj.}
\end{itemize}
Relativo a tabela; que tem fórma de tabela.
\section{Tabelário}
\begin{itemize}
\item {Grp. gram.:m.}
\end{itemize}
\begin{itemize}
\item {Utilização:Ant.}
\end{itemize}
\begin{itemize}
\item {Proveniência:(Lat. \textunderscore tabellarius\textunderscore )}
\end{itemize}
Correio; postilhão.
\section{Tabelhão}
\begin{itemize}
\item {Grp. gram.:m.}
\end{itemize}
\begin{itemize}
\item {Utilização:T. da Bairrada}
\end{itemize}
Tôrno, tarugo.
Tôrno de madeira, na ponta do apo.
(Provavelmente por \textunderscore trabelhão\textunderscore , de \textunderscore trabelho\textunderscore )
\section{Tabeliado}
\begin{itemize}
\item {Grp. gram.:m.}
\end{itemize}
\begin{itemize}
\item {Proveniência:(De \textunderscore tabelião\textunderscore )}
\end{itemize}
O mesmo que \textunderscore tabelionado\textunderscore .
Antigo imposto, que os tabeliães pagavam.
\section{Tabelião}
\begin{itemize}
\item {Grp. gram.:m.}
\end{itemize}
\begin{itemize}
\item {Proveniência:(Lat. \textunderscore tabellio\textunderscore )}
\end{itemize}
Notário, funcionário público que faz escrituras e outros instrumentos jurídicos, que reconhece assinaturas, etc.
\section{Tabeliar}
\begin{itemize}
\item {Grp. gram.:v. i.}
\end{itemize}
Exercer as funcções de tabelião.
\section{Tabeliôa}
\begin{itemize}
\item {Grp. gram.:f.  e  adj.}
\end{itemize}
\begin{itemize}
\item {Grp. gram.:F.}
\end{itemize}
\begin{itemize}
\item {Utilização:Pop.}
\end{itemize}
\begin{itemize}
\item {Proveniência:(De \textunderscore tabelião\textunderscore )}
\end{itemize}
Diz-se da letra larga e mal feita e de certas palavras que constituem uma fórma usual.
Mulhér de tabelião.
\section{Tabelionado}
\begin{itemize}
\item {Grp. gram.:m.}
\end{itemize}
\begin{itemize}
\item {Proveniência:(Do lat. \textunderscore tabellio\textunderscore , \textunderscore tabellionis\textunderscore )}
\end{itemize}
Cargo de tabelião.
\section{Tabelionar}
\begin{itemize}
\item {Grp. gram.:adj.}
\end{itemize}
Relativo a tabelião.
\section{Tabelionato}
\begin{itemize}
\item {Grp. gram.:m.}
\end{itemize}
O mesmo que \textunderscore tabelionado\textunderscore .
\section{Tabella}
\begin{itemize}
\item {Grp. gram.:f.}
\end{itemize}
\begin{itemize}
\item {Grp. gram.:Loc. adv.}
\end{itemize}
\begin{itemize}
\item {Proveniência:(Lat. \textunderscore tabella\textunderscore )}
\end{itemize}
Tábua pequena, quadro ou papel, em que se regista qualquer coisa.
Relação; catálogo; lista.
Índice.
Parte interna da borda do bilhar.
Caixilho, que contém as bolas móvies, com que se marca o número das carambolas, no bilhar.
Electuário em pastilhas.
\textunderscore Por tabella\textunderscore , o mesmo que \textunderscore indirectamente\textunderscore .
\section{Tabellar}
\begin{itemize}
\item {Grp. gram.:adj.}
\end{itemize}
Relativo a tabella; que tem fórma de tabella.
\section{Tabellário}
\begin{itemize}
\item {Grp. gram.:m.}
\end{itemize}
\begin{itemize}
\item {Utilização:Ant.}
\end{itemize}
\begin{itemize}
\item {Proveniência:(Lat. \textunderscore tabellarius\textunderscore )}
\end{itemize}
Correio; postilhão.
\section{Tabelliado}
\begin{itemize}
\item {Grp. gram.:m.}
\end{itemize}
\begin{itemize}
\item {Proveniência:(De \textunderscore tabellião\textunderscore )}
\end{itemize}
O mesmo que \textunderscore tabellionado\textunderscore .
Antigo imposto, que os tabelliães pagavam.
\section{Tabellião}
\begin{itemize}
\item {Grp. gram.:m.}
\end{itemize}
\begin{itemize}
\item {Proveniência:(Lat. \textunderscore tabellio\textunderscore )}
\end{itemize}
Notário, funccionário público que faz escrituras e outros instrumentos jurídicos, que reconhece assinaturas, etc.
\section{Tabelliar}
\begin{itemize}
\item {Grp. gram.:v. i.}
\end{itemize}
Exercer as funcções de tabellião.
\section{Tabelliôa}
\begin{itemize}
\item {Grp. gram.:f.  e  adj.}
\end{itemize}
\begin{itemize}
\item {Grp. gram.:F.}
\end{itemize}
\begin{itemize}
\item {Utilização:Pop.}
\end{itemize}
\begin{itemize}
\item {Proveniência:(De \textunderscore tabellião\textunderscore )}
\end{itemize}
Diz-se da letra larga e mal feita e de certas palavras que constituem uma fórma usual.
Mulhér de tabellião.
\section{Tabellionado}
\begin{itemize}
\item {Grp. gram.:m.}
\end{itemize}
\begin{itemize}
\item {Proveniência:(Do lat. \textunderscore tabellio\textunderscore , \textunderscore tabellionis\textunderscore )}
\end{itemize}
Cargo de tabellião.
\section{Tabellionar}
\begin{itemize}
\item {Grp. gram.:adj.}
\end{itemize}
Relativo a tabellião.
\section{Tabellionato}
\begin{itemize}
\item {Grp. gram.:m.}
\end{itemize}
O mesmo que \textunderscore tabellionado\textunderscore .
\section{Taberna}
\begin{itemize}
\item {Grp. gram.:f.}
\end{itemize}
\begin{itemize}
\item {Utilização:Fig.}
\end{itemize}
\begin{itemize}
\item {Proveniência:(Lat. \textunderscore taberna\textunderscore )}
\end{itemize}
Loja ou lugar, onde se vende vinho por miúdo.
Casa de pasto ordinária, tasca.
Casa immunda.
\section{Tabernáculo}
\begin{itemize}
\item {Grp. gram.:m.}
\end{itemize}
\begin{itemize}
\item {Utilização:Fam.}
\end{itemize}
\begin{itemize}
\item {Proveniência:(Lat. \textunderscore tabernaculum\textunderscore )}
\end{itemize}
Tenda portátil, em que os Hebreus faziam os seus sacrifícios.
Parte do templo israelita, em que se conservava a Arca da Alliança.
Mesa, em que trabalham os ourives.
Lugar nas galeras, donde o capitão fazia o commando.
Morada, lares.
\section{Tabernal}
\begin{itemize}
\item {Grp. gram.:adj.}
\end{itemize}
Relativo a taberna.
Immundo.
\section{Tabernal}
\begin{itemize}
\item {Grp. gram.:m.}
\end{itemize}
\begin{itemize}
\item {Utilização:Mad}
\end{itemize}
\begin{itemize}
\item {Utilização:Pop.}
\end{itemize}
O mesmo que \textunderscore tribunal\textunderscore .
\section{Tabernário}
\begin{itemize}
\item {Grp. gram.:adj.}
\end{itemize}
\begin{itemize}
\item {Proveniência:(Lat. \textunderscore tabernarius\textunderscore )}
\end{itemize}
Tabernal.
Próprio de taberneiro.
\section{Taberneira}
\begin{itemize}
\item {Grp. gram.:f.}
\end{itemize}
\begin{itemize}
\item {Utilização:Fig.}
\end{itemize}
\begin{itemize}
\item {Proveniência:(De \textunderscore taberneiro\textunderscore )}
\end{itemize}
Mulhér de taberneiro.
Mulhér, que vende vinho a retalho.
Mulhér suja ou grosseira.
\section{Taberneiro}
\begin{itemize}
\item {Grp. gram.:m.}
\end{itemize}
\begin{itemize}
\item {Utilização:Fig.}
\end{itemize}
Dono de taberna.
Aquelle que vende em taberna.
Homem pouco limpo; homem grosseiro.
\section{Tabernola}
\begin{itemize}
\item {Grp. gram.:f.}
\end{itemize}
Taberna ordinária, tasca.
\section{Tabernória}
\begin{itemize}
\item {Grp. gram.:f.}
\end{itemize}
O mesmo que \textunderscore tabernola\textunderscore .
\section{Tabes}
\begin{itemize}
\item {Grp. gram.:f.}
\end{itemize}
\begin{itemize}
\item {Utilização:Med.}
\end{itemize}
\begin{itemize}
\item {Proveniência:(Lat. \textunderscore tabes\textunderscore )}
\end{itemize}
Ataxia progressiva dos membros locomotores.
Degeneração dos cordões posteriores da medulla espinhal.
\section{Tabescente}
\begin{itemize}
\item {Grp. gram.:adj.}
\end{itemize}
\begin{itemize}
\item {Utilização:P. us.}
\end{itemize}
\begin{itemize}
\item {Proveniência:(Lat. \textunderscore tabescens\textunderscore )}
\end{itemize}
O mesmo que \textunderscore tábido\textunderscore :«\textunderscore dos tabescentes lívidos cadáveres...\textunderscore »Garrett, \textunderscore Catão\textunderscore , 79.
\section{Tabi}
\begin{itemize}
\item {Grp. gram.:m.}
\end{itemize}
\begin{itemize}
\item {Proveniência:(Do ár. \textunderscore attabi\textunderscore )}
\end{itemize}
Espécie de tafetá grosso.
\section{Tabibuia}
\begin{itemize}
\item {Grp. gram.:f.}
\end{itemize}
\begin{itemize}
\item {Utilização:Bras}
\end{itemize}
Árvore apocýnea, que cresce em lugares húmidos, e cuja madeira é utilizada por tamanqueiros.
\section{Tabica}
\begin{itemize}
\item {Grp. gram.:f.}
\end{itemize}
\begin{itemize}
\item {Utilização:Náut.}
\end{itemize}
\begin{itemize}
\item {Utilização:Bras}
\end{itemize}
\begin{itemize}
\item {Utilização:Bras}
\end{itemize}
\begin{itemize}
\item {Proveniência:(Do ár. \textunderscore tatbica\textunderscore )}
\end{itemize}
Peça da borda do navio, sôbre o tôpo das aposturas.
Cunha, encravada no tôpo de um madeiro que se está serrando, afim de facilitar a serragem.
Cipó, de que se fazem chibatas.
O mesmo que \textunderscore chibata\textunderscore .
\section{Tabicar}
\begin{itemize}
\item {Grp. gram.:v. t.}
\end{itemize}
Pôr tabicas em.
\section{Tabidamente}
\begin{itemize}
\item {Grp. gram.:adj.}
\end{itemize}
Em estado tábido; em estado de corrupção material.
\section{Tabidez}
\begin{itemize}
\item {Grp. gram.:f.}
\end{itemize}
Estado ou qualidade de tábido:«\textunderscore testemunho dera eu de tabidez intellectual...\textunderscore »Camillo, \textunderscore Mulher Fatal\textunderscore , 196.
\section{Tábido}
\begin{itemize}
\item {Grp. gram.:adj.}
\end{itemize}
\begin{itemize}
\item {Proveniência:(Lat. \textunderscore tabidus\textunderscore )}
\end{itemize}
Sanioso, pôdre.
Em que há podridão.
\section{Tabífico}
\begin{itemize}
\item {Grp. gram.:adj.}
\end{itemize}
\begin{itemize}
\item {Proveniência:(Lat. \textunderscore tabificus\textunderscore )}
\end{itemize}
Que faz apodrecer; que corrompe.
\section{Tabique}
\begin{itemize}
\item {Grp. gram.:m.}
\end{itemize}
\begin{itemize}
\item {Proveniência:(Do ár. \textunderscore taxbik\textunderscore )}
\end{itemize}
O mesmo que \textunderscore taipa\textunderscore .
Construcção delgada, geralmente de madeira, com que se divide verticalmente o interior de uma casa.
Divisória.
Membrana, que separa dois órgãos ou duas cavidades.
Parede estreita, de tijolo.
\section{Tabira}
\begin{itemize}
\item {Grp. gram.:f.}
\end{itemize}
Festa, entre os Tupis.
\section{Tabizar}
\begin{itemize}
\item {Grp. gram.:v. t.}
\end{itemize}
Tornar ondeado como o tabi.
\section{Tabla}
\begin{itemize}
\item {Grp. gram.:f.}
\end{itemize}
\begin{itemize}
\item {Grp. gram.:Adj.}
\end{itemize}
Chapa, lâmina.
Diz-se do diamante chato e lapidado.
(Contr. de \textunderscore tábula\textunderscore )
\section{Tablado}
\begin{itemize}
\item {Grp. gram.:m.}
\end{itemize}
\begin{itemize}
\item {Proveniência:(Do cast. \textunderscore tabla\textunderscore )}
\end{itemize}
Estrado; palanque; palco.
\section{Tablatura}
\begin{itemize}
\item {Grp. gram.:f.}
\end{itemize}
O mesmo que \textunderscore tavolatura\textunderscore .
\section{Tablilha}
\begin{itemize}
\item {Grp. gram.:f.}
\end{itemize}
\begin{itemize}
\item {Utilização:Fig.}
\end{itemize}
\begin{itemize}
\item {Proveniência:(De \textunderscore tabla\textunderscore ; ou por \textunderscore tabellilha\textunderscore , de \textunderscore tabella\textunderscore )}
\end{itemize}
Tabella de bilhar.
Meio indirecto:«\textunderscore ...veio por tablilha.\textunderscore »Filinto, XI, 195.
\section{Tablino}
\begin{itemize}
\item {Grp. gram.:m.}
\end{itemize}
\begin{itemize}
\item {Utilização:Des.}
\end{itemize}
\begin{itemize}
\item {Proveniência:(Lat. \textunderscore tablinum\textunderscore )}
\end{itemize}
Gabinete, guarnecido de quadros.
Gabinete de pintura.
Cartório.
\section{Tabo}
\begin{itemize}
\item {Grp. gram.:m.}
\end{itemize}
Embarcação asiática.
\section{Táboa}
\begin{itemize}
\item {Grp. gram.:f.}
\end{itemize}
\begin{itemize}
\item {Utilização:Fig.}
\end{itemize}
\begin{itemize}
\item {Utilização:Bras}
\end{itemize}
\begin{itemize}
\item {Grp. gram.:Loc.}
\end{itemize}
\begin{itemize}
\item {Utilização:fam.}
\end{itemize}
\begin{itemize}
\item {Grp. gram.:Pl.}
\end{itemize}
\begin{itemize}
\item {Proveniência:(Do lat. \textunderscore tabula\textunderscore )}
\end{itemize}
Peça plana de madeira e mais ou menos delgada.
Mappa.
Tela para pintura.
Indice.
Tabella.
Mesa de jôgo.
Mesa, onde se come.
Pedaço de mármore plano.
Lâmina interior e exterior dos ossos cranianos.
Cada uma das faces lateraes do pescoço do cavallo.
Recusa a pedido de casamento.
\textunderscore Mandar á táboa\textunderscore , mandar bugiar, mandar á fava.
Quaesquer escrituras, exaradas em pedra, em madeira ou noutra substância: \textunderscore lei das Doze Táboas\textunderscore .
\section{Taboca}
\begin{itemize}
\item {Grp. gram.:f.}
\end{itemize}
\begin{itemize}
\item {Utilização:T. da Baía}
\end{itemize}
Espécie de bambu brasileiro.
Doce sêco, com a apparência de papel pardo.
\section{Taboca}
\begin{itemize}
\item {Grp. gram.:f.}
\end{itemize}
\begin{itemize}
\item {Utilização:Bras}
\end{itemize}
Lôgro; decepção.
\textunderscore Passar a taboca\textunderscore , deixar noivo ou noiva, para casar com outrem.
Não dançar com quem se escolheu para par.
\section{Tabocal}
\begin{itemize}
\item {Grp. gram.:m.}
\end{itemize}
\begin{itemize}
\item {Proveniência:(De \textunderscore taboca\textunderscore ^1)}
\end{itemize}
Terreno, onde crescem tabocas.
\section{Tabocas}
\begin{itemize}
\item {Grp. gram.:m. pl.}
\end{itemize}
Tríbo de aborígenes do Pará.
\section{Taboeira}
\begin{itemize}
\item {Grp. gram.:f.}
\end{itemize}
\begin{itemize}
\item {Utilização:Bras}
\end{itemize}
Qualquer planta, que cresce mal ou que se desenvolve pouco.
(Cp. \textunderscore tamboeira\textunderscore )
\section{Taboquear}
\begin{itemize}
\item {Grp. gram.:v. t.}
\end{itemize}
\begin{itemize}
\item {Utilização:Bras}
\end{itemize}
\begin{itemize}
\item {Grp. gram.:V. i.}
\end{itemize}
\begin{itemize}
\item {Proveniência:(De \textunderscore taboca\textunderscore ^2, se não do ant. port. \textunderscore atabucar\textunderscore )}
\end{itemize}
Lograr.
Desilludir.
Passar a taboca.
\section{Taboqueiro}
\begin{itemize}
\item {Grp. gram.:adj.}
\end{itemize}
\begin{itemize}
\item {Utilização:Bras. do N}
\end{itemize}
Que vende caro.
\section{Taboquinha}
\begin{itemize}
\item {Grp. gram.:f.}
\end{itemize}
\begin{itemize}
\item {Proveniência:(De \textunderscore taboca\textunderscore ^1)}
\end{itemize}
Planta herbácea do Brasil.
\section{Tabordas}
\begin{itemize}
\item {Grp. gram.:m.}
\end{itemize}
\begin{itemize}
\item {Utilização:Prov.}
\end{itemize}
\begin{itemize}
\item {Utilização:minh.}
\end{itemize}
Indivíduo despropositado; escalda-favaes. (Colhido em Barcelos)
\section{Tabu}
\begin{itemize}
\item {Grp. gram.:m.}
\end{itemize}
\begin{itemize}
\item {Utilização:Bras}
\end{itemize}
Açúcar, mal coalhado.
Açúcar mascavado.
Ceremónia supersticiosa na Guiné.
\section{Tabu}
\begin{itemize}
\item {Grp. gram.:m.}
\end{itemize}
\begin{itemize}
\item {Utilização:Bras. da Baía}
\end{itemize}
O mesmo que \textunderscore tabúa\textunderscore , de que se fazem esteiras.
\section{Tabúa}
\begin{itemize}
\item {Grp. gram.:f.}
\end{itemize}
Planta leguminosa, de flôres em corymbos.
Planta typhácea, (\textunderscore typha minor\textunderscore ).
\section{Tábua}
\begin{itemize}
\item {Grp. gram.:f.}
\end{itemize}
\begin{itemize}
\item {Utilização:Fig.}
\end{itemize}
\begin{itemize}
\item {Utilização:Bras}
\end{itemize}
\begin{itemize}
\item {Grp. gram.:Loc.}
\end{itemize}
\begin{itemize}
\item {Utilização:fam.}
\end{itemize}
\begin{itemize}
\item {Grp. gram.:Pl.}
\end{itemize}
\begin{itemize}
\item {Proveniência:(Do lat. \textunderscore tabula\textunderscore )}
\end{itemize}
Peça plana de madeira e mais ou menos delgada.
Mappa.
Tela para pintura.
Indice.
Tabella.
Mesa de jôgo.
Mesa, onde se come.
Pedaço de mármore plano.
Lâmina interior e exterior dos ossos cranianos.
Cada uma das faces lateraes do pescoço do cavallo.
Recusa a pedido de casamento.
\textunderscore Mandar á tábua\textunderscore , mandar bugiar, mandar á fava.
Quaesquer escrituras, exaradas em pedra, em madeira ou noutra substância: \textunderscore lei das Doze Tábuas\textunderscore .
\section{Tabuada}
\begin{itemize}
\item {Grp. gram.:f.}
\end{itemize}
\begin{itemize}
\item {Utilização:Fam.}
\end{itemize}
\begin{itemize}
\item {Proveniência:(De \textunderscore tábua\textunderscore )}
\end{itemize}
Tabella; índice.
Quadro, em que se expõem e ensinam operações Arithméticas.
Livrinho, em que se ensina a numeração e os primeiros rudimentos de Arithmética.
Repertório, série.
\section{Tabuada}
\begin{itemize}
\item {Grp. gram.:f.}
\end{itemize}
\begin{itemize}
\item {Utilização:Prov.}
\end{itemize}
\begin{itemize}
\item {Utilização:alent.}
\end{itemize}
Canteiro nas hortas.
(Relaciona-se com \textunderscore tábua\textunderscore  ou com \textunderscore tabúa\textunderscore ?)
\section{Tabuado}
\begin{itemize}
\item {Grp. gram.:m.}
\end{itemize}
Porção de tábuas; sobrado.
\section{Tabual}
\begin{itemize}
\item {Grp. gram.:m.}
\end{itemize}
Terreno, onde crescem tabúas.
\section{Tabuão}
\begin{itemize}
\item {Grp. gram.:m.}
\end{itemize}
Tábua grande, prancha.
\section{Tabueira}
\begin{itemize}
\item {Grp. gram.:f.}
\end{itemize}
\begin{itemize}
\item {Utilização:Bras}
\end{itemize}
Qualquer planta, que cresce mal ou que se desenvolve pouco.
(Cp. \textunderscore tamboeira\textunderscore )
\section{Tabuiaiá}
\begin{itemize}
\item {Grp. gram.:m.}
\end{itemize}
\begin{itemize}
\item {Utilização:Bras}
\end{itemize}
Nome de uma ave.
\section{Tabuínha}
\begin{itemize}
\item {Grp. gram.:f.}
\end{itemize}
\begin{itemize}
\item {Grp. gram.:Pl.}
\end{itemize}
\begin{itemize}
\item {Proveniência:(De \textunderscore tábua\textunderscore )}
\end{itemize}
Tábua muito delgada.
Conjunto de ripas ou fasquias, sobrepostas horizontalmente e enfiadas em cordas ou fitas, das quaes se suspendem nos vãos das janelas ou sacadas, para resguardarem do sol ou das vistas estranhas o interior da habitação.
\section{Tábula}
\begin{itemize}
\item {Grp. gram.:f.}
\end{itemize}
\begin{itemize}
\item {Utilização:Ant.}
\end{itemize}
\begin{itemize}
\item {Utilização:Des.}
\end{itemize}
\begin{itemize}
\item {Proveniência:(Lat. \textunderscore tabula\textunderscore )}
\end{itemize}
Pequena peça redonda, empregada em vários jogos de tabuleiro.
Mesa, especialmente a mesa do jôgo.
\textunderscore Jôgo das tábulas\textunderscore , jôgo do gamão.
\section{Tabulado}
\begin{itemize}
\item {Grp. gram.:m.}
\end{itemize}
\begin{itemize}
\item {Proveniência:(Lat. \textunderscore tabulatum\textunderscore )}
\end{itemize}
Tapume de tábuas.
Soalho.
Resguardo de madeira.
\section{Tabulageiro}
\begin{itemize}
\item {Grp. gram.:f.}
\end{itemize}
Dono de tabulagem.
Aquelle que entra em jogos de asar.
\section{Tabulagem}
\begin{itemize}
\item {Grp. gram.:f.}
\end{itemize}
\begin{itemize}
\item {Utilização:Ant.}
\end{itemize}
\begin{itemize}
\item {Proveniência:(De \textunderscore tábula\textunderscore )}
\end{itemize}
Casa em que há jôgo de tábulas.
Casa de jôgo.
Jôgo, vício do jôgo.
\section{Tabulão}
\begin{itemize}
\item {Grp. gram.:m.}
\end{itemize}
\begin{itemize}
\item {Proveniência:(De \textunderscore tábula\textunderscore )}
\end{itemize}
Mesa ou tabernáculo de ourives.
\section{Tabulão}
\begin{itemize}
\item {Grp. gram.:adj.}
\end{itemize}
\begin{itemize}
\item {Utilização:T. de Pare -de-Coira}
\end{itemize}
\begin{itemize}
\item {Utilização:des.}
\end{itemize}
Teimoso.
\section{Tabular}
\begin{itemize}
\item {Grp. gram.:adj.}
\end{itemize}
\begin{itemize}
\item {Utilização:Miner.}
\end{itemize}
\begin{itemize}
\item {Proveniência:(Lat. \textunderscore tabularis\textunderscore )}
\end{itemize}
Relativo a tábua.
Que tem fórma de tábua.
Que tem fórma de tabella.
Diz-se de um typo das fórmas irregulares dos crystaes.
\section{Tabulário}
\begin{itemize}
\item {Grp. gram.:adj.}
\end{itemize}
\begin{itemize}
\item {Proveniência:(Lat. \textunderscore tabularium\textunderscore )}
\end{itemize}
Que tem gravuras em madeira, (falando-se de livros).
\section{Tabulato}
\begin{itemize}
\item {Grp. gram.:m.}
\end{itemize}
\begin{itemize}
\item {Utilização:Ant.}
\end{itemize}
\begin{itemize}
\item {Proveniência:(Lat. \textunderscore tabulatum\textunderscore )}
\end{itemize}
O mesmo que \textunderscore tablado\textunderscore .
Theatro.
Palanque. Cf. B. Pereira, \textunderscore Prosódia\textunderscore , vb. \textunderscore scena\textunderscore .
\section{Tabuleiro}
\begin{itemize}
\item {Grp. gram.:m.}
\end{itemize}
\begin{itemize}
\item {Utilização:Mús.}
\end{itemize}
\begin{itemize}
\item {Utilização:Bras. do N}
\end{itemize}
\begin{itemize}
\item {Proveniência:(De \textunderscore tábula\textunderscore )}
\end{itemize}
Peça plana de madeira ou de outra substância, e com rebordo saliente.
Bandeja.
Quadro de madeira ou de outra substância, sôbre que se joga o xadrez, as damas, etc.
Espaço plano, dentro de uma igreja ou de outro edifício.
Patamar.
Varanda.
Pavimento de uma ponte, com seus resguardos.
Pedaço de jardim, limitado por bordadura.
Canteiro.
Horta.
Talho das salinas.
Soalho do carro.
A parte do piano, onde assenta o teclado.
Terreno pedregoso, de pequena vegetação.
\section{Tabuleta}
\begin{itemize}
\item {fónica:lê}
\end{itemize}
\begin{itemize}
\item {Grp. gram.:f.}
\end{itemize}
\begin{itemize}
\item {Utilização:Fig.}
\end{itemize}
\begin{itemize}
\item {Proveniência:(De \textunderscore tábula\textunderscore )}
\end{itemize}
Peça plana de madeira ou de outra substância, collocada na frente de um estabelecimento, de uma repartição ou de qualquer construcção, indicando o que nesta se vende ou os fins a que ella se destina.
Mostrador de loja de ourives.
Indicação, sinal.
\section{Tabulhão}
\begin{itemize}
\item {Grp. gram.:m.}
\end{itemize}
«\textunderscore ...fez grande pendor, e o tabulhão debaixo do costado, e aparelhos dados no masto grande, que fizerão vir a nao á banda tanto que lhe descobrirão a quilha...\textunderscore »Gaspar Correia, \textunderscore Lendas\textunderscore , I, 29.
\section{Tabulista}
\begin{itemize}
\item {Grp. gram.:m.  e  f.}
\end{itemize}
Pessôa, que faz tábulas ou tabellas astronómicas, geométricas, etc.
\section{Taburnar}
\begin{itemize}
\item {Grp. gram.:v. i.}
\end{itemize}
\begin{itemize}
\item {Utilização:T. de Pare -de-Coira}
\end{itemize}
\begin{itemize}
\item {Utilização:des.}
\end{itemize}
O mesmo que \textunderscore urinar\textunderscore .
\section{Taburno}
\begin{itemize}
\item {Grp. gram.:m.}
\end{itemize}
\begin{itemize}
\item {Proveniência:(It. \textunderscore tamburo\textunderscore )}
\end{itemize}
Estrado; suppedâneo.
Peça de madeira, em fórma de telha, em que se transportam torrões para os muros das marinhas.
\section{Taca}
\begin{itemize}
\item {Grp. gram.:f.}
\end{itemize}
\begin{itemize}
\item {Utilização:Bras}
\end{itemize}
O mesmo que \textunderscore pancada\textunderscore .
(Cp. \textunderscore tacada\textunderscore )
\section{Taca}
\begin{itemize}
\item {Grp. gram.:f.}
\end{itemize}
\begin{itemize}
\item {Utilização:Bras}
\end{itemize}
Correia, o mesmo que \textunderscore manguá\textunderscore .
\section{Taca}
\begin{itemize}
\item {Grp. gram.:f.}
\end{itemize}
Gênero de plantas, (\textunderscore tacca pinnatifida\textunderscore , Forster), que serve de tipo ás tacáceas.
\section{Taça}
\begin{itemize}
\item {Grp. gram.:f.}
\end{itemize}
\begin{itemize}
\item {Proveniência:(Do ár. \textunderscore taça\textunderscore )}
\end{itemize}
Vaso para beber, de boca um pouco larga.
Pequena malga, tigelinha.
Copo.
Nome de uma constellação. Cf. B. Pereira, \textunderscore Prosódia\textunderscore , vb. \textunderscore urna\textunderscore .
\section{Tacacá}
\begin{itemize}
\item {Grp. gram.:m.}
\end{itemize}
Iguaria picante do norte do Brasil, espécie de caldo grosso de mandioca.
\section{Tacáceas}
\begin{itemize}
\item {Grp. gram.:f. pl}
\end{itemize}
Família de plantas monocotiledóneas, que compreende vegetaes herbáceos, vivazes, de raízes tuberosas e feculentas.
\section{Tacada}
\begin{itemize}
\item {Grp. gram.:f.}
\end{itemize}
Pancada de taco.
\section{Taçada}
\begin{itemize}
\item {Grp. gram.:f.}
\end{itemize}
\begin{itemize}
\item {Utilização:Pleb.}
\end{itemize}
Aquillo que uma taça pode conter.
Bebedeira.
\section{Taçado}
\begin{itemize}
\item {Grp. gram.:adj.}
\end{itemize}
\begin{itemize}
\item {Utilização:Pleb.}
\end{itemize}
\begin{itemize}
\item {Proveniência:(De \textunderscore taça\textunderscore )}
\end{itemize}
O mesmo que \textunderscore bêbedo\textunderscore .
\section{Tacamaca}
\begin{itemize}
\item {Grp. gram.:f.}
\end{itemize}
Árvore terebinthácea.
Árvore gutífera.
Resina destas árvores.
\section{Tacamagueiro}
\begin{itemize}
\item {Grp. gram.:m.}
\end{itemize}
(V.tacamaca)
\section{Tacana}
\begin{itemize}
\item {Grp. gram.:f.}
\end{itemize}
\begin{itemize}
\item {Utilização:Bras. do Amazonas}
\end{itemize}
O mesmo que \textunderscore frecheira\textunderscore .
\section{Tacanas}
\begin{itemize}
\item {Grp. gram.:m. pl.}
\end{itemize}
Tríbo indígena do Peru.
\section{Tacanhamente}
\begin{itemize}
\item {Grp. gram.:adv.}
\end{itemize}
De modo tacanho.
\section{Tacanharia}
\begin{itemize}
\item {Grp. gram.:f.}
\end{itemize}
Qualidade ou acto do que é tacanho.
\section{Tacanhear}
\begin{itemize}
\item {Grp. gram.:v. i.}
\end{itemize}
Proceder como tacanho.
\section{Tacanhez}
\begin{itemize}
\item {Grp. gram.:f.}
\end{itemize}
O mesmo que \textunderscore tacanharia\textunderscore .
\section{Tacanheza}
\begin{itemize}
\item {Grp. gram.:f.}
\end{itemize}
O mesmo que \textunderscore tacanharia\textunderscore .
\section{Tacanhice}
\begin{itemize}
\item {Grp. gram.:f.}
\end{itemize}
O mesmo que \textunderscore tacanharia\textunderscore .
\section{Tacanho}
\begin{itemize}
\item {Grp. gram.:adj.}
\end{itemize}
Que tem pequena estatura.
Apoucado, inhenho.
Sovina, avarento.
Velhaco.
(Cast. \textunderscore tacaño\textunderscore )
\section{Tacanhunas}
\begin{itemize}
\item {Grp. gram.:m. pl.}
\end{itemize}
Tríbo de Índios tupinambás, no Pará.
\section{Tacaniça}
\begin{itemize}
\item {Grp. gram.:f.}
\end{itemize}
\begin{itemize}
\item {Utilização:Carp.}
\end{itemize}
Parte do telhado, que cobre ou abriga os lados do edifício.
Peça de madeira que, nos telhados ou madeiramentos de três ou quatro correntes, vai do extremo da fileira a qualquer ângulo de duas frentes, e que também se chama \textunderscore rincão\textunderscore .
\section{Tacão}
\begin{itemize}
\item {Grp. gram.:m.}
\end{itemize}
\begin{itemize}
\item {Utilização:Fig.}
\end{itemize}
\begin{itemize}
\item {Proveniência:(De \textunderscore taco\textunderscore ^1?)}
\end{itemize}
Pedaço ou pedaços de sola, semi-circulares, em que assenta a parte posterior do calçado.
Parte do calçado, correspondente á base do calcanhar.
Salto do calçado.
Pateada: \textunderscore o actor foi recebido a tacão\textunderscore .
\section{Tacão}
\begin{itemize}
\item {Grp. gram.:m.}
\end{itemize}
(V.tacanho)
\section{Tacape}
\begin{itemize}
\item {Grp. gram.:m.}
\end{itemize}
Arma gentílica, espécie de clava, entre os Índios da América.
\section{Tacar}
\begin{itemize}
\item {Grp. gram.:v. i.}
\end{itemize}
\begin{itemize}
\item {Utilização:Prov.}
\end{itemize}
\begin{itemize}
\item {Utilização:minh.}
\end{itemize}
\begin{itemize}
\item {Proveniência:(De \textunderscore taco\textunderscore ^2)}
\end{itemize}
Tomar qualquer coisa de comer, entre o almôço e o jantar; lanchar.
\section{Tacaré}
\begin{itemize}
\item {Grp. gram.:m.}
\end{itemize}
\begin{itemize}
\item {Utilização:Bras}
\end{itemize}
Espécie de mandioca.
\section{Tacca}
\begin{itemize}
\item {Grp. gram.:f.}
\end{itemize}
Gênero de plantas, (\textunderscore tacca pinnatifida\textunderscore , Forster), que serve de typo ás taccáceas.
\section{Taccáceas}
\begin{itemize}
\item {Grp. gram.:f. pl}
\end{itemize}
Família de plantas monocotyledóneas, que comprehende vegetaes herbáceos, vivazes, de raízes tuberosas e feculentas.
\section{Taccóide}
\begin{itemize}
\item {Grp. gram.:m.}
\end{itemize}
Gênero de aves trepadoras.
\section{Taceira}
\begin{itemize}
\item {Grp. gram.:f.}
\end{itemize}
\begin{itemize}
\item {Proveniência:(De \textunderscore taça\textunderscore )}
\end{itemize}
Tabuleta ou mostrador, em que se expõem taças e outros artefactos de ourives.
Espécie de doce, que se fabricava no convento de Santa-Clara, em Beja.
\section{Taceiro}
\begin{itemize}
\item {Grp. gram.:m.}
\end{itemize}
Aquelle que, na secretaria do antigo Ministério do Reino, recebia a chamada propina da taça.
\section{Tacelo}
\begin{itemize}
\item {fónica:cê}
\end{itemize}
\begin{itemize}
\item {Grp. gram.:m.}
\end{itemize}
Cada uma das peças, de que se compõe uma estátua ou um modelo, em esculptura.
(Cp. it. \textunderscore tasselo\textunderscore )
\section{Tacha}
\begin{itemize}
\item {Grp. gram.:f.}
\end{itemize}
\begin{itemize}
\item {Utilização:Fig.}
\end{itemize}
\begin{itemize}
\item {Utilização:Pop.}
\end{itemize}
\begin{itemize}
\item {Utilização:Prov.}
\end{itemize}
\begin{itemize}
\item {Utilização:beir.}
\end{itemize}
Pequeno prego, de cabeça chata.
Brocha.
Nódoa.
Defeito; mancha.
Dente.
Dentadura, cartucheira.
(Cp. cast. \textunderscore tacha\textunderscore )
\section{Tacha}
\begin{itemize}
\item {Grp. gram.:f.}
\end{itemize}
\begin{itemize}
\item {Utilização:Bras}
\end{itemize}
Tacho grande, usado nos engenhos de açúcar.
(Cp. \textunderscore tacho\textunderscore )
\section{Tachada}
\begin{itemize}
\item {Grp. gram.:f.}
\end{itemize}
\begin{itemize}
\item {Utilização:Pleb.}
\end{itemize}
\begin{itemize}
\item {Proveniência:(De \textunderscore tacho\textunderscore )}
\end{itemize}
O que um tacho póde conter; tacho cheio.
O mesmo que \textunderscore bebedeira\textunderscore .
\section{Tachado}
\begin{itemize}
\item {Grp. gram.:adj.}
\end{itemize}
\begin{itemize}
\item {Utilização:Pleb.}
\end{itemize}
O mesmo que \textunderscore bêbedo\textunderscore .
(Cp. \textunderscore tachada\textunderscore )
\section{Tachador}
\begin{itemize}
\item {Grp. gram.:m.  e  adj.}
\end{itemize}
O que tacha.
\section{Tachão}
\begin{itemize}
\item {Grp. gram.:m.}
\end{itemize}
Grande tacha ou mancha:«\textunderscore ...círculos ou tachões de pedra negra...\textunderscore »Sousa, \textunderscore Vida do Arceb.\textunderscore , III, 318.
(Cp. cast. \textunderscore tachón\textunderscore )
\section{Tachão}
\begin{itemize}
\item {Grp. gram.:m.}
\end{itemize}
Tacho grande.
\section{Tachar}
\begin{itemize}
\item {Grp. gram.:v. t.}
\end{itemize}
\begin{itemize}
\item {Proveniência:(De \textunderscore tacha\textunderscore ^1)}
\end{itemize}
Notar defeito em.
Notar.
Qualificar.
\section{Tacheometria}
\begin{itemize}
\item {fónica:que}
\end{itemize}
\begin{itemize}
\item {Grp. gram.:f.}
\end{itemize}
Conjunto de princípios e operações, que constituem o processo mais económico, mais rápido, e menos fatigante, para se obter, sem damno das propriedades, o relêvo de um terreno.
(Cp. \textunderscore tacheómetro\textunderscore )
\section{Tacheómetro}
\begin{itemize}
\item {fónica:que}
\end{itemize}
\begin{itemize}
\item {Grp. gram.:m.}
\end{itemize}
Instrumento, com que se pratíca a tacheometria.
(Palavra mal formada, do gr. \textunderscore takhus\textunderscore  + \textunderscore metron\textunderscore . Cp. \textunderscore tachýmetro\textunderscore )
\section{Tachim}
\begin{itemize}
\item {Grp. gram.:m.}
\end{itemize}
\begin{itemize}
\item {Proveniência:(Do al. \textunderscore tasche\textunderscore ?)}
\end{itemize}
Invólucro de coiro, para resguardar uma encadernação rica.
\section{Tachinha}
\begin{itemize}
\item {Grp. gram.:f.}
\end{itemize}
Pequena tacha^1.
\section{Tacho}
\begin{itemize}
\item {Grp. gram.:m.}
\end{itemize}
\begin{itemize}
\item {Utilização:Fam.}
\end{itemize}
\begin{itemize}
\item {Utilização:Ant.}
\end{itemize}
\begin{itemize}
\item {Utilização:Gír.}
\end{itemize}
\begin{itemize}
\item {Proveniência:(Do ár. \textunderscore taxt\textunderscore )}
\end{itemize}
Vaso largo e pouco fundo, geralmente com asas, e destinado especialmente a usos culinários.
O mesmo que \textunderscore cozinheira\textunderscore . Cf. Eça, \textunderscore P. Amaro\textunderscore , 84.
Medida de capacidade, correspondente a 25 litros.
Espécie de jôgo de bilhar, que também se chama \textunderscore jôgo do prato\textunderscore , e que consiste em que dois ou mais jogadores reúnem num prato, ao centro do bilhar, as \textunderscore entradas\textunderscore  e os \textunderscore castigos\textunderscore  convencionaes, formando um \textunderscore bolo\textunderscore , que é levantado pelo jogador que primeiro fizer determinado número de carambolas, sem de permeio haver tocado no prato com alguma das bolas.
Cara.
\section{Tachometria}
\begin{itemize}
\item {fónica:co}
\end{itemize}
\begin{itemize}
\item {Grp. gram.:f.}
\end{itemize}
Applicação do tachómetro.
\section{Tachómetro}
\begin{itemize}
\item {fónica:có}
\end{itemize}
\begin{itemize}
\item {Grp. gram.:m.}
\end{itemize}
\begin{itemize}
\item {Proveniência:(Do gr. \textunderscore takhos\textunderscore  + \textunderscore metron\textunderscore )}
\end{itemize}
Instrumento, com que se determina a velocidade dos movimentos de uma máquina.
\section{Tachonar}
\begin{itemize}
\item {Grp. gram.:v. t.}
\end{itemize}
\begin{itemize}
\item {Proveniência:(De \textunderscore tachão\textunderscore ^1)}
\end{itemize}
Pregar ou segurar com tachas grandes.
Esmaltar com tachas; mosquear. Cf. Gabr. Pereira, \textunderscore Ulisseia\textunderscore ; Filinto, XV, 30.
\section{Táchya}
\begin{itemize}
\item {fónica:qui}
\end{itemize}
\begin{itemize}
\item {Grp. gram.:f.}
\end{itemize}
Gênero de plantas gencianáceas.
\section{Tachyádeno}
\begin{itemize}
\item {fónica:qui}
\end{itemize}
\begin{itemize}
\item {Grp. gram.:f.}
\end{itemize}
\begin{itemize}
\item {Proveniência:(Do gr. \textunderscore takhus\textunderscore  + \textunderscore aden\textunderscore )}
\end{itemize}
Gênero de plantas gencianáceas.
\section{Tachycardia}
\begin{itemize}
\item {fónica:qui}
\end{itemize}
\begin{itemize}
\item {Grp. gram.:f.}
\end{itemize}
\begin{itemize}
\item {Proveniência:(Do gr. \textunderscore takhus\textunderscore  + \textunderscore kardia\textunderscore )}
\end{itemize}
Rapidez de pulsações.
\section{Táchyde}
\begin{itemize}
\item {fónica:qui}
\end{itemize}
\begin{itemize}
\item {Grp. gram.:f.}
\end{itemize}
Gênero de insectos coleópteros pentâmeros.
\section{Tachydrito}
\begin{itemize}
\item {fónica:qui}
\end{itemize}
\begin{itemize}
\item {Grp. gram.:m.}
\end{itemize}
\begin{itemize}
\item {Utilização:Miner.}
\end{itemize}
\begin{itemize}
\item {Proveniência:(Do gr. \textunderscore takhus\textunderscore  + \textunderscore hudor\textunderscore )}
\end{itemize}
Chloreto hydratado de cálcio e magnésio.
\section{Tachydromia}
\begin{itemize}
\item {fónica:qui}
\end{itemize}
\begin{itemize}
\item {Grp. gram.:f.}
\end{itemize}
\begin{itemize}
\item {Proveniência:(Do gr. \textunderscore takhus\textunderscore  + \textunderscore dromos\textunderscore )}
\end{itemize}
Gênero de insectos dípteros.
\section{Tachygalia}
\begin{itemize}
\item {fónica:qui}
\end{itemize}
\begin{itemize}
\item {Grp. gram.:f.}
\end{itemize}
\begin{itemize}
\item {Proveniência:(Do gr. \textunderscore takhus\textunderscore  + \textunderscore gala\textunderscore )}
\end{itemize}
Gênero de plantas leguminosas.
\section{Tachýgono}
\begin{itemize}
\item {fónica:qui}
\end{itemize}
\begin{itemize}
\item {Grp. gram.:m.}
\end{itemize}
\begin{itemize}
\item {Proveniência:(Do gr. \textunderscore takhus\textunderscore  + \textunderscore gonos\textunderscore )}
\end{itemize}
Gênero de insectos coleópteros tetrâmeros.
\section{Tachygraphar}
\begin{itemize}
\item {fónica:qui}
\end{itemize}
\begin{itemize}
\item {Grp. gram.:v. t.}
\end{itemize}
\begin{itemize}
\item {Proveniência:(De \textunderscore tachýgrapho\textunderscore )}
\end{itemize}
Escrever tachygraphicamente.
\section{Tachygraphia}
\begin{itemize}
\item {fónica:qui}
\end{itemize}
\begin{itemize}
\item {Grp. gram.:f.}
\end{itemize}
\begin{itemize}
\item {Proveniência:(De \textunderscore tachýgrapho\textunderscore )}
\end{itemize}
Systema de escrita, por meio do qual se escreve quási tão depressa como se fala.
\section{Tachygraphicamente}
\begin{itemize}
\item {fónica:qui}
\end{itemize}
\begin{itemize}
\item {Grp. gram.:adv.}
\end{itemize}
De modo tachygráphico; segundo os processos da tachygraphia.
\section{Tachygráphico}
\begin{itemize}
\item {fónica:qui}
\end{itemize}
\begin{itemize}
\item {Grp. gram.:adj.}
\end{itemize}
Relativo á tachygraphia.
\section{Tachýgrapho}
\begin{itemize}
\item {fónica:qui}
\end{itemize}
\begin{itemize}
\item {Grp. gram.:m.}
\end{itemize}
\begin{itemize}
\item {Proveniência:(Do gr. \textunderscore takhus\textunderscore  + \textunderscore graphein\textunderscore )}
\end{itemize}
Aquelle que escreve tachygraphicamente.
Tratadista de tachygraphia.
\section{Tachýlitha}
\begin{itemize}
\item {fónica:qui}
\end{itemize}
\begin{itemize}
\item {Grp. gram.:f.}
\end{itemize}
\begin{itemize}
\item {Proveniência:(Do gr. \textunderscore takhus\textunderscore  + \textunderscore lithos\textunderscore )}
\end{itemize}
Silicato de alumina e bases protoxydadas, que se encontra no basalto.
\section{Tachýlitho}
\begin{itemize}
\item {fónica:quí}
\end{itemize}
\begin{itemize}
\item {Grp. gram.:m.}
\end{itemize}
O mesmo ou melhór que \textunderscore tachýlitha\textunderscore .
\section{Tachýmetro}
\begin{itemize}
\item {fónica:quí}
\end{itemize}
\begin{itemize}
\item {Grp. gram.:m.}
\end{itemize}
\begin{itemize}
\item {Proveniência:(Do gr. \textunderscore takhus\textunderscore  + \textunderscore metron\textunderscore )}
\end{itemize}
Instrumento, com que se procura determinar ou avaliar a velocidade de uma máquina.--É preferível \textunderscore tachómetro\textunderscore .
V. \textunderscore tachómetro\textunderscore .
\section{Tachypneia}
\begin{itemize}
\item {fónica:qui}
\end{itemize}
\begin{itemize}
\item {Grp. gram.:f.}
\end{itemize}
\begin{itemize}
\item {Utilização:Med.}
\end{itemize}
\begin{itemize}
\item {Proveniência:(Do gr. \textunderscore takhus\textunderscore  + \textunderscore pnein\textunderscore )}
\end{itemize}
Grande acceleração do rythmo respiratório.
\section{Tachyplóteros}
\begin{itemize}
\item {fónica:qui}
\end{itemize}
\begin{itemize}
\item {Grp. gram.:m. pl.}
\end{itemize}
\begin{itemize}
\item {Utilização:Zool.}
\end{itemize}
\begin{itemize}
\item {Proveniência:(Do gr. \textunderscore takhus\textunderscore  + \textunderscore ploter\textunderscore )}
\end{itemize}
Uma das divisões da família das anátides.
\section{Táchypo}
\begin{itemize}
\item {fónica:qui}
\end{itemize}
\begin{itemize}
\item {Grp. gram.:m.}
\end{itemize}
\begin{itemize}
\item {Proveniência:(Do gr. \textunderscore takhus\textunderscore  + \textunderscore pous\textunderscore )}
\end{itemize}
Gênero de insectos coleópteros pentâmeros.
\section{Tachýporo}
\begin{itemize}
\item {fónica:qui}
\end{itemize}
\begin{itemize}
\item {Grp. gram.:m.}
\end{itemize}
\begin{itemize}
\item {Proveniência:(Do gr. \textunderscore takhus\textunderscore  + \textunderscore poros\textunderscore )}
\end{itemize}
Gênero de insectos coleópteros pentâmeros.
\section{Tácia}
\begin{itemize}
\item {Grp. gram.:f.}
\end{itemize}
Planta da serra de Sintra.
\section{Tacitamente}
\begin{itemize}
\item {Grp. gram.:adv.}
\end{itemize}
De modo tácito; implicitamente; sem declaração expressa, mas subentendida.
\section{Tacitífluo}
\begin{itemize}
\item {Grp. gram.:adj.}
\end{itemize}
\begin{itemize}
\item {Proveniência:(Do lat. \textunderscore tacitus\textunderscore  + \textunderscore fluere\textunderscore )}
\end{itemize}
Que corre ou mana silenciosamente. Cf. Castilho, \textunderscore Fastos\textunderscore , I, 9.
\section{Tácito}
\begin{itemize}
\item {Grp. gram.:adj.}
\end{itemize}
\begin{itemize}
\item {Proveniência:(Lat. \textunderscore tacitus\textunderscore )}
\end{itemize}
Silencioso.
Que se não exprime por palavras.
Subentendido; implícito: \textunderscore consentimento tácito\textunderscore .
\section{Taciturnamente}
\begin{itemize}
\item {Grp. gram.:adv.}
\end{itemize}
De modo taciturno; de modo sombrio ou carrancudo.
\section{Taciturnidade}
\begin{itemize}
\item {Grp. gram.:f.}
\end{itemize}
\begin{itemize}
\item {Proveniência:(Lat. \textunderscore taciturnitas\textunderscore )}
\end{itemize}
Qualidade do que é taciturno.
Misanthropia; solidão.
\section{Taciturno}
\begin{itemize}
\item {Grp. gram.:adj.}
\end{itemize}
\begin{itemize}
\item {Proveniência:(Lat. \textunderscore taciturnus\textunderscore )}
\end{itemize}
Que por natureza é de poucas palavras; triste.
\section{Taco}
\begin{itemize}
\item {Grp. gram.:m.}
\end{itemize}
Pau redondo e comprido, com que, no jôgo do bilhar, se impellem as bolas.
O jôgo do bilhar. Cf. M. Feijó, \textunderscore Orthogr.\textunderscore 
Peça de madeira, com que se fecha o rombo, feito por artilharia no costado de um navio.
Peça em que assenta o carrete da atafona.
Bucha de peça de artilharia.
Tarugo.
Peça de madeira, com que se fecha qualquer orifício ou abertura; tapulho.
(Cp. cast. \textunderscore taco\textunderscore )
\section{Taco}
\begin{itemize}
\item {Grp. gram.:m.}
\end{itemize}
\begin{itemize}
\item {Utilização:Bras. do N}
\end{itemize}
\begin{itemize}
\item {Utilização:Prov.}
\end{itemize}
\begin{itemize}
\item {Utilização:trasm.}
\end{itemize}
Pedaço, bocado.
Pequena refeição, usada por trabalhadores entre o almôço e o jantar; piqueta.
\section{Tacóide}
\begin{itemize}
\item {Grp. gram.:m.}
\end{itemize}
Gênero de aves trepadoras.
\section{Tacoila}
\begin{itemize}
\item {Grp. gram.:f.}
\end{itemize}
\begin{itemize}
\item {Utilização:Prov.}
\end{itemize}
Joelheira ou utensílio de madeira, sôbre que ajoelha a pessôa que anda lavando pavimentos ou sobrados.
\section{Tacometria}
\begin{itemize}
\item {Grp. gram.:f.}
\end{itemize}
Aplicação do tacómetro.
\section{Tacómetro}
\begin{itemize}
\item {Grp. gram.:m.}
\end{itemize}
\begin{itemize}
\item {Proveniência:(Do gr. \textunderscore takhos\textunderscore  + \textunderscore metron\textunderscore )}
\end{itemize}
Instrumento, com que se determina a velocidade dos movimentos de uma máquina.
\section{Tacôs}
\begin{itemize}
\item {Grp. gram.:m.}
\end{itemize}
Espécie de feijão chinês. Cf. \textunderscore Ásia Sínica\textunderscore , 59.
\section{Taco-taraco}
\begin{itemize}
\item {Grp. gram.:m.}
\end{itemize}
\begin{itemize}
\item {Utilização:Ant.}
\end{itemize}
Saltos desordenados:«\textunderscore se V. Senhoria acha consonancia nisto, hé capaz de chamar dança ao taco taraco\textunderscore ». \textunderscore Anat. Joc.\textunderscore , II, 424.
\section{Tacteabilidade}
\begin{itemize}
\item {Grp. gram.:f.}
\end{itemize}
\begin{itemize}
\item {Utilização:Phýs.}
\end{itemize}
Qualidade de tacteável ou de táctil.
O facto positivo da resistência táctil.
\section{Tacteadamente}
\begin{itemize}
\item {Grp. gram.:adv.}
\end{itemize}
\begin{itemize}
\item {Proveniência:(De \textunderscore tactear\textunderscore )}
\end{itemize}
Por meio do tacto; apalpando.
\section{Tacteamento}
\begin{itemize}
\item {Grp. gram.:m.}
\end{itemize}
Acto de tactear.
\section{Tacteante}
\begin{itemize}
\item {Grp. gram.:adj.}
\end{itemize}
Que tacteia.
\section{Tactear}
\begin{itemize}
\item {Grp. gram.:v. t.}
\end{itemize}
\begin{itemize}
\item {Utilização:Fig.}
\end{itemize}
\begin{itemize}
\item {Proveniência:(De \textunderscore tacto\textunderscore )}
\end{itemize}
Applicar o tacto a.
Apalpar.
Investigar, pesquisar; examinar.
\section{Tacteável}
\begin{itemize}
\item {Grp. gram.:adj.}
\end{itemize}
Que se póde tactear.
\section{Táctica}
\begin{itemize}
\item {Grp. gram.:f.}
\end{itemize}
\begin{itemize}
\item {Utilização:Fig.}
\end{itemize}
\begin{itemize}
\item {Proveniência:(Do gr. \textunderscore taktihe\textunderscore )}
\end{itemize}
Arte de combater.
Habilidade em regular ou dirigir um negócio.
\section{Táctico}
\begin{itemize}
\item {Grp. gram.:adj.}
\end{itemize}
\begin{itemize}
\item {Grp. gram.:M.}
\end{itemize}
\begin{itemize}
\item {Proveniência:(Lat. \textunderscore tacticus\textunderscore )}
\end{itemize}
Relativo á táctica.
Indivíduo, que é perito em táctica.
\section{Tacticografia}
\begin{itemize}
\item {Grp. gram.:f.}
\end{itemize}
\begin{itemize}
\item {Proveniência:(Do gr. \textunderscore taktike\textunderscore  + \textunderscore graphein\textunderscore )}
\end{itemize}
Delineamento de manobras militares.
Representação gráfica de evoluções guerreiras.
\section{Tacticographia}
\begin{itemize}
\item {Grp. gram.:f.}
\end{itemize}
\begin{itemize}
\item {Proveniência:(Do gr. \textunderscore taktike\textunderscore  + \textunderscore graphein\textunderscore )}
\end{itemize}
Delineamento de manobras militares.
Representação gráphica de evoluções guerreiras.
\section{Táctil}
\begin{itemize}
\item {Grp. gram.:adj.}
\end{itemize}
\begin{itemize}
\item {Proveniência:(Lat. \textunderscore tactilis\textunderscore )}
\end{itemize}
Relativo ao tacto; que se póde tactear.
\section{Tactilidade}
\begin{itemize}
\item {Grp. gram.:f.}
\end{itemize}
Qualidade de táctil.
Carácter das substâncias, que exercem acção especial no sentido do tacto, como as unctuosas, as ásperas, as macias. Cf. Gonç. Guimarães, \textunderscore Geologia\textunderscore , 67.
\section{Tactilmente}
\begin{itemize}
\item {Grp. gram.:adv.}
\end{itemize}
De modo táctil; por meio do tacto.
\section{Tacto}
\begin{itemize}
\item {Grp. gram.:m.}
\end{itemize}
\begin{itemize}
\item {Utilização:Fig.}
\end{itemize}
\begin{itemize}
\item {Proveniência:(Lat. \textunderscore tactus\textunderscore )}
\end{itemize}
Um dos cinco sentidos, que nos permitte avaliar a solidez, a temperatura, a fórma, a extensão e outras qualidades dos corpos.
Tactura.
Sensação, que nos causam os objectos que apalpamos.
Prudência; habilidade.
Vocação.
\section{Tacto}
\begin{itemize}
\item {Grp. gram.:adj.}
\end{itemize}
\begin{itemize}
\item {Utilização:Bras}
\end{itemize}
Tremulo.
Bambo; incerto.
\section{Tactura}
\begin{itemize}
\item {Grp. gram.:f.}
\end{itemize}
\begin{itemize}
\item {Proveniência:(De \textunderscore tacto\textunderscore ^1)}
\end{itemize}
Acto ou effeito de tactear.
\section{Tacuará}
\begin{itemize}
\item {Grp. gram.:m.}
\end{itemize}
O mesmo que \textunderscore taboca\textunderscore ^1.
\section{Tacuará-açu}
\begin{itemize}
\item {Grp. gram.:m.}
\end{itemize}
Grande bambu.
\section{Tacuaral}
\begin{itemize}
\item {Grp. gram.:m.}
\end{itemize}
O mesmo que \textunderscore tabocal\textunderscore .
\section{Tacuaré}
\begin{itemize}
\item {Grp. gram.:m.}
\end{itemize}
Um dos nomes do castanheiro do Maranhão.
\section{Tacuari}
\begin{itemize}
\item {Grp. gram.:m.}
\end{itemize}
Planta gramínea, (\textunderscore panicum horizontale\textunderscore ).
Nome de outras plantas americanas.
\section{Tacula}
\begin{itemize}
\item {Grp. gram.:f.}
\end{itemize}
Árvore africana, cuja madeira é muito apreciada e usada em tinturaria.--Alguns diccion. dizem \textunderscore tácula\textunderscore . Opto pela prosódia de Ficalho, \textunderscore Plantas úteis\textunderscore .
\section{Tacumba-iva}
\begin{itemize}
\item {Grp. gram.:f.}
\end{itemize}
Variedade de coqueiro.
\section{Tacunás}
\begin{itemize}
\item {Grp. gram.:m. pl.}
\end{itemize}
Numerosa tríbo de Índios do Brasil, espalhada hoje pelas villas Fonte-Boa, Olivença e San-José.
\section{Tacuru}
\begin{itemize}
\item {Grp. gram.:m.}
\end{itemize}
\begin{itemize}
\item {Utilização:Bras}
\end{itemize}
O mesmo que \textunderscore tacuruba\textunderscore .
\section{Tacuru}
\begin{itemize}
\item {Grp. gram.:m.}
\end{itemize}
\begin{itemize}
\item {Utilização:Bras. do S}
\end{itemize}
Montículo de terra, em meio de um charco coberto de ervagem.
\section{Taçuru}
\begin{itemize}
\item {Grp. gram.:m.}
\end{itemize}
\begin{itemize}
\item {Utilização:Bras. do S}
\end{itemize}
Pulga do pé, pulga penetrante.
\section{Tacuruba}
\begin{itemize}
\item {Grp. gram.:f.}
\end{itemize}
\begin{itemize}
\item {Utilização:Bras}
\end{itemize}
Trempe, formada por três pedras sôltas, em que se assenta a panela.
(Do tupi \textunderscore itacuruba\textunderscore )
\section{Tacus}
\begin{itemize}
\item {Grp. gram.:m. pl.}
\end{itemize}
Indígenas do norte do Brasil.
\section{Tádega}
\begin{itemize}
\item {Grp. gram.:f.}
\end{itemize}
Planta, da fam. das synanthéreas, (\textunderscore conyza squarrosa\textunderscore ).
\section{Tadorna}
\begin{itemize}
\item {Grp. gram.:f.}
\end{itemize}
Espécie de pato bravo, (\textunderscore anas tadorna\textunderscore , Lin.).
\section{Tadorno}
\begin{itemize}
\item {fónica:dôr}
\end{itemize}
\begin{itemize}
\item {Grp. gram.:adj.}
\end{itemize}
Diz-se de uma espécie de pato bravo, (\textunderscore anas tadorna\textunderscore , Lin.).
\section{Taeiga}
\begin{itemize}
\item {Grp. gram.:f.}
\end{itemize}
\begin{itemize}
\item {Utilização:Ant.}
\end{itemize}
O mesmo que \textunderscore taleiga\textunderscore . Cf. Meyer-Lübke, \textunderscore Gram. des Lang. Rom.\textunderscore , I, 410.
\section{Taél}
\begin{itemize}
\item {Grp. gram.:m.}
\end{itemize}
\begin{itemize}
\item {Proveniência:(Do mal. \textunderscore tail\textunderscore )}
\end{itemize}
Unidade de pêso, que na China tem valor monetário.
\section{Táes}
\begin{itemize}
\item {Grp. gram.:m.}
\end{itemize}
Espécie de bigorna, para uso dos cutileiros.
\section{Táes}
\begin{itemize}
\item {Grp. gram.:m.}
\end{itemize}
Pano de algodão, com que os guerreiros indígenas de Timor cobrem o corpo, desde a cintura ao joelho.
\section{Tafecira}
\begin{itemize}
\item {Grp. gram.:f.}
\end{itemize}
O mesmo que \textunderscore taficira\textunderscore . Cf. Felner, \textunderscore Subsídios\textunderscore .
\section{Tafetá}
\begin{itemize}
\item {Grp. gram.:m.}
\end{itemize}
\begin{itemize}
\item {Proveniência:(Do pers. \textunderscore taftah\textunderscore )}
\end{itemize}
Tecido lustroso de sêda.
\section{Tafe-tafe}
\begin{itemize}
\item {Grp. gram.:m.}
\end{itemize}
\begin{itemize}
\item {Utilização:Pop.}
\end{itemize}
O mesmo que \textunderscore tefe-tefe\textunderscore .
\section{Tafiá}
\begin{itemize}
\item {Grp. gram.:m.}
\end{itemize}
Aguardente de cana; cachaça.
\section{Taficira}
\begin{itemize}
\item {Grp. gram.:f.}
\end{itemize}
Espécie de chita da Índia.
\section{Tafife}
\begin{itemize}
\item {Grp. gram.:m.}
\end{itemize}
\begin{itemize}
\item {Utilização:Prov.}
\end{itemize}
Lasca de madeira, com que se tapam as junturas de porta ou janela, produzidas pelo secar da madeira destas.
\section{Tafilete}
\begin{itemize}
\item {Grp. gram.:m.}
\end{itemize}
Marroquim fino, fabricado em Tafilete, cidade marroquina.
\section{Tafona}
\begin{itemize}
\item {Grp. gram.:f.}
\end{itemize}
\begin{itemize}
\item {Utilização:Bras}
\end{itemize}
O mesmo que \textunderscore atafona\textunderscore .
\section{Tafoné}
\begin{itemize}
\item {Grp. gram.:m.}
\end{itemize}
Carolo ou cascudo dado com as cabeças dos dedos. Cf. Filinto, VIII, 22.
\section{Taforéa}
\begin{itemize}
\item {Grp. gram.:f.}
\end{itemize}
Antiga embarcação portuguesa, empregada em transporte de cavallos, e ainda como navio de guerra. Cf. Filinto, \textunderscore D. Man.\textunderscore , II, 151.
(Cast. ant. \textunderscore taforea\textunderscore )
\section{Taforeia}
\begin{itemize}
\item {Grp. gram.:f.}
\end{itemize}
Antiga embarcação portuguesa, empregada em transporte de cavallos, e ainda como navio de guerra. Cf. Filinto, \textunderscore D. Man.\textunderscore , II, 151.
(Cast. ant. \textunderscore taforea\textunderscore )
\section{Tafu}
\begin{itemize}
\item {Grp. gram.:m.}
\end{itemize}
Bebida chinesa, preparada com uma espécie de feijões.
\section{Taful}
\begin{itemize}
\item {Grp. gram.:m.  e  adj.}
\end{itemize}
\begin{itemize}
\item {Utilização:Fig.}
\end{itemize}
Janota, peralta.
Garrido.
Casquilho.
Jogador de profissão ou por hábito.
O que sabe do seu offício.
(Cp. \textunderscore tafur\textunderscore )
\section{Tafular}
\begin{itemize}
\item {Grp. gram.:v. i.}
\end{itemize}
Têr vida de taful; enfeitar-se, janotar.
\section{Tafularia}
\begin{itemize}
\item {Grp. gram.:f.}
\end{itemize}
\begin{itemize}
\item {Proveniência:(De \textunderscore taful\textunderscore )}
\end{itemize}
Acto ou effeito de tafular.
Reunião de tafues.
\section{Tafulhar}
\begin{itemize}
\item {Grp. gram.:v. t.}
\end{itemize}
O mesmo que \textunderscore atafulhar\textunderscore .
\section{Tafulho}
\begin{itemize}
\item {Grp. gram.:m.}
\end{itemize}
Acto ou effeito de tafulhar.
Tacha ou bucha, com que se tapa um orifício.
(Infl. de \textunderscore tapulho\textunderscore ?)
\section{Tafulice}
\begin{itemize}
\item {Grp. gram.:f.}
\end{itemize}
O mesmo que \textunderscore tafularia\textunderscore .
\section{Tafulo}
\begin{itemize}
\item {Grp. gram.:adj.}
\end{itemize}
O mesmo que \textunderscore taful\textunderscore .
\section{Tafur}
\begin{itemize}
\item {Grp. gram.:m.}
\end{itemize}
\begin{itemize}
\item {Utilização:Ant.}
\end{itemize}
O mesmo que \textunderscore taful\textunderscore . Cp. G. Resende, \textunderscore Miscellânea\textunderscore .
(Cast. \textunderscore tahur\textunderscore )
\section{Tagala}
\begin{itemize}
\item {Grp. gram.:m.}
\end{itemize}
O mesmo que \textunderscore tagalo\textunderscore .
\section{Tagalar}
\begin{itemize}
\item {Grp. gram.:m.}
\end{itemize}
O mesmo que \textunderscore tagalo\textunderscore .
\section{Tagalo}
\begin{itemize}
\item {Grp. gram.:m.}
\end{itemize}
Homem natural das Filippinas.
Dialecto malaio, um dos das Filippinas.
\section{Tagana}
\begin{itemize}
\item {Grp. gram.:f.}
\end{itemize}
\begin{itemize}
\item {Utilização:Des.}
\end{itemize}
Nome, que no Ribatejo se dava á taínha.
\section{Tagantada}
\begin{itemize}
\item {Grp. gram.:f.}
\end{itemize}
Acto ou effeito de tagantar.
\section{Tagantar}
\begin{itemize}
\item {Grp. gram.:v. t.}
\end{itemize}
Bater com tagante.
\section{Tagante}
\begin{itemize}
\item {Grp. gram.:m.}
\end{itemize}
Azorrague antigo.
(Cp. cast. \textunderscore tajar\textunderscore )
\section{Tagantear}
\begin{itemize}
\item {Grp. gram.:v. t.}
\end{itemize}
O mesmo que \textunderscore tagantar\textunderscore .
\section{Tagarela}
\begin{itemize}
\item {Grp. gram.:adj.}
\end{itemize}
\begin{itemize}
\item {Grp. gram.:M.  e  f.}
\end{itemize}
\begin{itemize}
\item {Proveniência:(De \textunderscore tagarelar\textunderscore )}
\end{itemize}
Diz-se de quem fala muito ou de quem é indiscreto.
Pessôa muito faladora ou indiscreta.
Barulho, gritaria.
\section{Tagarelar}
\begin{itemize}
\item {Grp. gram.:v. i.}
\end{itemize}
Falar muito.
Sêr indiscreto.
Parolar.
\section{Tagarelice}
\begin{itemize}
\item {Grp. gram.:f.}
\end{itemize}
\begin{itemize}
\item {Proveniência:(De \textunderscore tagarela\textunderscore )}
\end{itemize}
Hábito de tagarelar; modos de tagarela; indiscrição.
\section{Tagarinos}
\begin{itemize}
\item {Grp. gram.:m. pl.}
\end{itemize}
Nome, que se deu aos Moiros, nascidos entre os Christãos da Espanha e falando correntemente a língua castelhana.
(Cp. cast. \textunderscore tagarino\textunderscore )
\section{Tagaris}
\begin{itemize}
\item {Grp. gram.:m. pl.}
\end{itemize}
Tríbo do Alto Amazonas.
\section{Tagarote}
\begin{itemize}
\item {Grp. gram.:m.}
\end{itemize}
\begin{itemize}
\item {Utilização:Fig.}
\end{itemize}
\begin{itemize}
\item {Proveniência:(Do ár. \textunderscore taorti\textunderscore ?)}
\end{itemize}
Espécie de falcão africano.
Indivíduo pobre, que come á custa alheia.
\section{Tagarra}
\begin{itemize}
\item {Grp. gram.:f.}
\end{itemize}
Peixe marítimo da costa de Portugal.
(Cp. \textunderscore tagana\textunderscore )
\section{Tagarrilha}
\begin{itemize}
\item {Grp. gram.:f.}
\end{itemize}
\begin{itemize}
\item {Utilização:Prov.}
\end{itemize}
\begin{itemize}
\item {Utilização:alent.}
\end{itemize}
O mesmo que \textunderscore tagarrina\textunderscore .
\section{Tagarrina}
\begin{itemize}
\item {Grp. gram.:f.}
\end{itemize}
\begin{itemize}
\item {Utilização:Prov.}
\end{itemize}
\begin{itemize}
\item {Utilização:alent.}
\end{itemize}
Cardo comestível, o mesmo que \textunderscore carrasquinha\textunderscore .
(Provavelmente do cast. \textunderscore tagarnina\textunderscore )
\section{Tagaté}
\begin{itemize}
\item {Grp. gram.:m.}
\end{itemize}
\begin{itemize}
\item {Utilização:Fam.}
\end{itemize}
Festa com a mão, carícia; afago; lisonja.
\section{Tagaz}
\begin{itemize}
\item {Grp. gram.:m.}
\end{itemize}
O mesmo que \textunderscore chagaz\textunderscore .
\section{Tage}
\begin{itemize}
\item {Proveniência:(Do ár. \textunderscore tage\textunderscore )}
\end{itemize}
\textunderscore m.\textunderscore  Grande carapuça, usada no Oriente. Cf. Tenreiro, \textunderscore Itiner.\textunderscore , c. VIII.
\section{Tagênia}
\begin{itemize}
\item {Grp. gram.:f.}
\end{itemize}
\begin{itemize}
\item {Proveniência:(Lat. \textunderscore tagenia\textunderscore )}
\end{itemize}
Gênero de insectos coleópteros heterómeros.
\section{Tagenitos}
\begin{itemize}
\item {Grp. gram.:m. pl.}
\end{itemize}
Tríbo de insectos coleópteros heterómeros.
\section{Tageto}
\begin{itemize}
\item {Grp. gram.:m.}
\end{itemize}
Gênero de plantas synanthéreas.
\section{Tágico}
\begin{itemize}
\item {Grp. gram.:adj.}
\end{itemize}
\begin{itemize}
\item {Utilização:Poét.}
\end{itemize}
\begin{itemize}
\item {Proveniência:(Do lat. \textunderscore Tagus\textunderscore , n. p.)}
\end{itemize}
Relativo ao Tejo.
\section{Tágide}
\begin{itemize}
\item {Grp. gram.:f.}
\end{itemize}
\begin{itemize}
\item {Utilização:Poét.}
\end{itemize}
\begin{itemize}
\item {Proveniência:(Do lat. \textunderscore Tagus\textunderscore , n. p.)}
\end{itemize}
Nimpha do Tejo.
\section{Tagílitho}
\begin{itemize}
\item {Grp. gram.:m.}
\end{itemize}
Variedade de phosphato ou cobre hydratado, de côr verde.
\section{Tagílito}
\begin{itemize}
\item {Grp. gram.:m.}
\end{itemize}
Variedade de fosfato ou cobre hidratado, de côr verde.
\section{Tágoa-uva}
\begin{itemize}
\item {Grp. gram.:f.}
\end{itemize}
O mesmo que \textunderscore tatajuba\textunderscore .
\section{Tagona}
\begin{itemize}
\item {Grp. gram.:f.}
\end{itemize}
Gênero de insectos coleópteros heterómeros.
\section{Tagra}
\begin{itemize}
\item {Grp. gram.:f.}
\end{itemize}
Medida antiga, equivalente á canada.
\section{Taguá}
\begin{itemize}
\item {Grp. gram.:m.}
\end{itemize}
O mesmo que \textunderscore cabo-negro\textunderscore .
\section{Taguari}
\begin{itemize}
\item {Grp. gram.:m.}
\end{itemize}
Espécie de cana de Mazagão.
\section{Tágueda}
\begin{itemize}
\item {Grp. gram.:f.}
\end{itemize}
(V.tádega)
\section{Tahanhé}
\begin{itemize}
\item {Grp. gram.:m.}
\end{itemize}
Planta, o mesmo que \textunderscore orelha-de-rato\textunderscore .
\section{Taia}
\begin{itemize}
\item {Grp. gram.:f.}
\end{itemize}
Cobra venenosa, temível e vulgar em Nova-Granada.
\section{Taiá}
\begin{itemize}
\item {Grp. gram.:f.}
\end{itemize}
(V.taioba)
\section{Taiaboeira}
\begin{itemize}
\item {Grp. gram.:f.}
\end{itemize}
O mesmo que \textunderscore tamboeira\textunderscore .
\section{Taiataia}
\begin{itemize}
\item {Grp. gram.:m.}
\end{itemize}
Ave palmípede dos mares da América.
\section{Taia-uva}
\begin{itemize}
\item {Grp. gram.:f.}
\end{itemize}
O mesmo que \textunderscore taioba\textunderscore .
\section{Taibo}
\begin{itemize}
\item {Grp. gram.:m.}
\end{itemize}
\begin{itemize}
\item {Utilização:Ant.}
\end{itemize}
\begin{itemize}
\item {Grp. gram.:Adv.}
\end{itemize}
Lugar ou posição indecorosa?
Sem sabôr? Cf. Camões, \textunderscore Rei Seleuco\textunderscore , cit. por Moraes, \textunderscore Diccion.\textunderscore --Júl. Moreira, \textunderscore Estudos da Língua Port.\textunderscore , I, 211, dá-lhe a significação de \textunderscore bem\textunderscore  e de \textunderscore bom\textunderscore , e suppõe-no procedente do árabe.
\section{Taifa}
\begin{itemize}
\item {Grp. gram.:f.}
\end{itemize}
\begin{itemize}
\item {Utilização:Náut.}
\end{itemize}
Conjunto de soldados e marinheiros que combatem na tolda e no castello da prôa.
(Cp. cast. \textunderscore teifa\textunderscore )
\section{Taifeiro}
\begin{itemize}
\item {Grp. gram.:m.}
\end{itemize}
Cada uma das unidades da taifa.
Serviçal de navios de guerra, aos quaes, em postos de combate, pertencia parte do serviço da taifa.
\section{Taimado}
\begin{itemize}
\item {Grp. gram.:adj.}
\end{itemize}
Malicioso; finório; velhaco.
(Cast. \textunderscore taimado\textunderscore )
\section{Taimbé}
\begin{itemize}
\item {Grp. gram.:m.}
\end{itemize}
\begin{itemize}
\item {Utilização:Bras. do S}
\end{itemize}
O mesmo que \textunderscore itaimbé\textunderscore .
\section{Taina}
\begin{itemize}
\item {Grp. gram.:m.}
\end{itemize}
\begin{itemize}
\item {Utilização:Prov.}
\end{itemize}
\begin{itemize}
\item {Utilização:trasm.}
\end{itemize}
O mesmo que \textunderscore pancadaria\textunderscore .
\section{Taínha}
\begin{itemize}
\item {Grp. gram.:f.}
\end{itemize}
Nome de vários peixes.
Mugem; tinca; fataça.
\section{Taio}
\begin{itemize}
\item {Grp. gram.:m.}
\end{itemize}
Quadrúpede dos bosques da Califórnia, de carne saborosa e do tamanho de um bezerro.
\section{Taioba}
\begin{itemize}
\item {fónica:ta-i}
\end{itemize}
\begin{itemize}
\item {Grp. gram.:f.}
\end{itemize}
Planta arácea, o mesmo que \textunderscore jarro\textunderscore .
\section{Taioca}
\begin{itemize}
\item {fónica:ta-i}
\end{itemize}
\begin{itemize}
\item {Grp. gram.:f.}
\end{itemize}
Formiga negra do Brasil.
\section{Taipa}
\begin{itemize}
\item {Grp. gram.:f.}
\end{itemize}
\begin{itemize}
\item {Proveniência:(Do cast. \textunderscore tapia\textunderscore )}
\end{itemize}
Parede de barro, calcado entre enxaiméis atravessados por fasquias.
Substância córnea, que envolve as partes vivas do pé do cavallo, o mesmo que \textunderscore tapa\textunderscore . Cf. Leon., \textunderscore Arte de Ferrar\textunderscore , 19.
\section{Taipal}
\begin{itemize}
\item {Grp. gram.:m.}
\end{itemize}
\begin{itemize}
\item {Grp. gram.:Adj.}
\end{itemize}
\begin{itemize}
\item {Grp. gram.:M. pl.}
\end{itemize}
\begin{itemize}
\item {Proveniência:(De \textunderscore taipa\textunderscore )}
\end{itemize}
Série de enxaiméis, entre os quaes se calca o barro da taipa.
Diz-se do carro, guarnecido de taipaes.
Espécie de portas ou anteparos, com que se resguardam vidraças.
Sebe, com que se alteiam as bordas de um carro, e que serve de anteparo á carga; anteparo.
\section{Taipão}
\begin{itemize}
\item {Grp. gram.:m.}
\end{itemize}
O mesmo que \textunderscore taipal\textunderscore .
\section{Taipar}
\begin{itemize}
\item {Grp. gram.:v. t.}
\end{itemize}
Calcar o barro em (a taipa).
Construir com taipa.
Limitar ou dividir com taipa: \textunderscore taipar uma sala\textunderscore .
\section{Taipeiro}
\begin{itemize}
\item {Grp. gram.:m.  e  adj.}
\end{itemize}
O que trabalha em taipas.
\section{Taipoca}
\begin{itemize}
\item {Grp. gram.:f.}
\end{itemize}
\begin{itemize}
\item {Utilização:Bras}
\end{itemize}
Gênero de árvores silvestres.
\section{Taíra}
\begin{itemize}
\item {Grp. gram.:f.}
\end{itemize}
Animal carnívoro da América.
\section{Tairoca}
\begin{itemize}
\item {Grp. gram.:f.}
\end{itemize}
O mesmo que \textunderscore taroca\textunderscore .
\section{Tais}
\begin{itemize}
\item {Grp. gram.:m.}
\end{itemize}
Espécie de bigorna, para uso dos cutileiros.
\section{Tais}
\begin{itemize}
\item {Grp. gram.:m.}
\end{itemize}
Pano de algodão, com que os guerreiros indígenas de Timor cobrem o corpo, desde a cintura ao joelho.
\section{Taititu}
\begin{itemize}
\item {Grp. gram.:m.}
\end{itemize}
\begin{itemize}
\item {Utilização:Bras. do N}
\end{itemize}
Mammífero, o mesmo que \textunderscore caititu\textunderscore .
\section{Taiuiá}
\begin{itemize}
\item {Grp. gram.:m.}
\end{itemize}
\begin{itemize}
\item {Utilização:Bras}
\end{itemize}
Gênero de plantas cucurbitáceas.
\section{Taiuva}
\begin{itemize}
\item {Grp. gram.:f.}
\end{itemize}
Árvore brasileira.
\section{Taixi}
\begin{itemize}
\item {Grp. gram.:m.}
\end{itemize}
O mesmo que \textunderscore pau-formiga\textunderscore .
\section{Tajá}
\begin{itemize}
\item {Grp. gram.:m.}
\end{itemize}
\begin{itemize}
\item {Utilização:Bras. do N}
\end{itemize}
O mesmo que \textunderscore taioba\textunderscore .
\section{Tajabemba}
\begin{itemize}
\item {Grp. gram.:f.}
\end{itemize}
\begin{itemize}
\item {Utilização:Bras}
\end{itemize}
Erva medicinal das regiões do Amazonas.
\section{Tajabuçu}
\begin{itemize}
\item {Grp. gram.:m.}
\end{itemize}
O mesmo que \textunderscore taioba\textunderscore .
\section{Tajabussu}
\begin{itemize}
\item {Grp. gram.:m.}
\end{itemize}
O mesmo que \textunderscore taioba\textunderscore .
\section{Tajaçu}
\begin{itemize}
\item {Grp. gram.:m.}
\end{itemize}
Espécie de porco bravo da América.
\section{Tajal}
\begin{itemize}
\item {Grp. gram.:m.}
\end{itemize}
O mesmo que \textunderscore taioba\textunderscore .
\section{Tajuba}
\begin{itemize}
\item {Grp. gram.:f.}
\end{itemize}
O mesmo que \textunderscore tatajuba\textunderscore .
\section{Tajujá}
\begin{itemize}
\item {Grp. gram.:m.}
\end{itemize}
Nome de várias plantas cucurbitáceas do Brasil.
\section{Tajurá}
\begin{itemize}
\item {Grp. gram.:m.}
\end{itemize}
O mesmo que \textunderscore tinhorão\textunderscore .
\section{Tal}
\begin{itemize}
\item {Grp. gram.:adj.}
\end{itemize}
\begin{itemize}
\item {Utilização:Ant.}
\end{itemize}
\begin{itemize}
\item {Grp. gram.:M.}
\end{itemize}
\begin{itemize}
\item {Grp. gram.:Loc. adv.}
\end{itemize}
\begin{itemize}
\item {Proveniência:(Lat. \textunderscore talis\textunderscore )}
\end{itemize}
Que tem a mesma qualidade ou natureza: \textunderscore tal pai, tal filho\textunderscore .
Que tem certa qualidade.
Semelhante.
Algum.
Tão bom, tão grande, tão elevado, etc.: \textunderscore é tal a sua audácia, que espanta\textunderscore .
Simples.
\textunderscore Mura de pedra tal\textunderscore , muro de pedra insossa, sem barro.
\textunderscore Lâmpada tal\textunderscore , lâmpada simples de vidro. Cf. Pant. de Aveiro, \textunderscore Itiner.\textunderscore , 8 e 89, (2.^a ed.).
Aquelle, um certo: \textunderscore não vi tal homem\textunderscore .
Isso, aquillo: \textunderscore não acredito tal\textunderscore .
\textunderscore Tal qual\textunderscore , exactamente o mesmo que. Cf. \textunderscore Filodemo\textunderscore , V, 4; Castilho, \textunderscore Sabichonas\textunderscore , 36.
Assim mesmo. Cf. Castilho, \textunderscore Sabichonas\textunderscore , 165; Camillo, \textunderscore Caveira\textunderscore , 241.
\section{Tala}
\begin{itemize}
\item {Grp. gram.:f.}
\end{itemize}
\begin{itemize}
\item {Utilização:Chapel.}
\end{itemize}
\begin{itemize}
\item {Grp. gram.:Pl.}
\end{itemize}
\begin{itemize}
\item {Utilização:Fig.}
\end{itemize}
\begin{itemize}
\item {Proveniência:(Do lat. \textunderscore tabula\textunderscore ?)}
\end{itemize}
Lâmina de madeira ou de outra substância, contra a qual se comprime uma parte do corpo, para esta se manter immóvel, como meio de cura.
Peça, com que se alonga interiormente a circunferência do chapéu.
Embaraços; difficuldades; apertos: \textunderscore viu-se em talas\textunderscore .
\section{Tala}
\begin{itemize}
\item {Grp. gram.:f.}
\end{itemize}
Acto ou effeito de talar^1.
\section{Talabardão}
\begin{itemize}
\item {Grp. gram.:m.}
\end{itemize}
\begin{itemize}
\item {Utilização:Náut.}
\end{itemize}
\begin{itemize}
\item {Proveniência:(De \textunderscore talabarte\textunderscore ?)}
\end{itemize}
Série de pranchões, que liga os dormentes da tolda aos do castello de prôa.
\section{Talabarte}
\begin{itemize}
\item {Grp. gram.:m.}
\end{itemize}
O mesmo que \textunderscore boldrié\textunderscore .
\section{Talaca}
\begin{itemize}
\item {Grp. gram.:f.}
\end{itemize}
\begin{itemize}
\item {Utilização:Ant.}
\end{itemize}
\begin{itemize}
\item {Proveniência:(Do ár. \textunderscore talaque\textunderscore )}
\end{itemize}
O mesmo que \textunderscore divórcio\textunderscore .
\section{Talado}
\begin{itemize}
\item {Grp. gram.:m.}
\end{itemize}
Arco da broca dos ourives.
\section{Talador}
\begin{itemize}
\item {Grp. gram.:m.  e  adj.}
\end{itemize}
O que tala.
\section{Talaga}
\begin{itemize}
\item {Grp. gram.:f.}
\end{itemize}
Grande palmeira asiática.
\section{Talagaça}
\begin{itemize}
\item {Grp. gram.:f.}
\end{itemize}
O mesmo que \textunderscore talagarça\textunderscore . Cf. G. Guerreiro, \textunderscore Diccion. de Cons.\textunderscore , 48.
\section{Talagarça}
\begin{itemize}
\item {Grp. gram.:f.}
\end{itemize}
Pano, mais ou menos grosso, de fios ralos, sôbre o qual se borda.
\section{Talagaxa}
\begin{itemize}
\item {Grp. gram.:f.}
\end{itemize}
Espécie de tecido fino de linho.
\section{Talagóia}
\begin{itemize}
\item {Grp. gram.:f.}
\end{itemize}
Peixe de Dio.
\section{Talagrepo}
\begin{itemize}
\item {Grp. gram.:m.}
\end{itemize}
\begin{itemize}
\item {Utilização:Ant.}
\end{itemize}
Sacerdote em Sião. Cf. \textunderscore Peregrinação\textunderscore .
\section{Talambor}
\begin{itemize}
\item {Grp. gram.:m.}
\end{itemize}
Fechadura de segrêdo, cuja lingueta é movida por peça interior, deixando vêr por fóra só o orifício, por onde entra uma chave especial.
\section{Talamento}
\begin{itemize}
\item {Grp. gram.:m.}
\end{itemize}
Acto ou effeito de talar.
\section{Talan}
\begin{itemize}
\item {Grp. gram.:m.}
\end{itemize}
\begin{itemize}
\item {Utilização:Ant.}
\end{itemize}
O mesmo que \textunderscore talante\textunderscore .
\section{Talanqueira}
\begin{itemize}
\item {Grp. gram.:f.}
\end{itemize}
\begin{itemize}
\item {Utilização:T. de Miranda}
\end{itemize}
Tablado, mesa ou qualquer construcção improvisada, em que se espera o acompanhamento dos noivos, e onde o padrinho tem de dar dinheiro a quem se apresenta. Cf. \textunderscore Portugalia\textunderscore , II, 102.
(Cast. \textunderscore talanquera\textunderscore )
\section{Talante}
\begin{itemize}
\item {Grp. gram.:m.}
\end{itemize}
Vontade, arbítrio: \textunderscore proceder a seu talante\textunderscore .
(Provn. e fr. \textunderscore talant\textunderscore , tenção, propósito)
\section{Talão}
\begin{itemize}
\item {Grp. gram.:m.}
\end{itemize}
\begin{itemize}
\item {Utilização:Mús.}
\end{itemize}
\begin{itemize}
\item {Proveniência:(Do lat. \textunderscore talo\textunderscore , \textunderscore talonis\textunderscore , seg. Körting)}
\end{itemize}
Parte posterior do pé, formada pelo maior osso do tarso.
Parte do calçado, correspondente ao calcanhar.
Moldura, côncava de um lado e convexa do outro.
Instrumento, com que se faz essa moldura.
Entalhe numa viga, para assentar o chincharel.
Cada uma das duas partes da face exterior de uma muralha.
Vara de videira, que se deixa junto á terra, na occasião da poda.
Parte de uma fôlha, de que se corta um recibo, e onde fica a indicação summária do mesmo recibo.
Extremidade dos ramos das ferraduras, também chamada \textunderscore collo\textunderscore .
A parte inferior do arco da rabeca.
\section{Talão}
\begin{itemize}
\item {Grp. gram.:m.}
\end{itemize}
(V.telão)
\section{Talão-balão}
\begin{itemize}
\item {Grp. gram.:m.}
\end{itemize}
(V.tão-balalão). Cf. Filinto, III, 152.
\section{Talapate}
\begin{itemize}
\item {Grp. gram.:m.}
\end{itemize}
\begin{itemize}
\item {Utilização:Ant.}
\end{itemize}
Imposto que, na Índia Portuguesa, pagavam os boticários, ourives, etc.
\section{Talar}
\begin{itemize}
\item {Grp. gram.:v. t.}
\end{itemize}
\begin{itemize}
\item {Utilização:Fig.}
\end{itemize}
Abrir sulcos em.
Fazer escoadoiros em (campos).
Subverter.
Assolar, devastar.
Derribar.
(Cp. cast. \textunderscore talar\textunderscore )
\section{Talar}
\begin{itemize}
\item {Grp. gram.:adj.}
\end{itemize}
\begin{itemize}
\item {Grp. gram.:M. Pl.}
\end{itemize}
\begin{itemize}
\item {Proveniência:(Lat. \textunderscore talaris\textunderscore )}
\end{itemize}
Relativo ao talão.
Que desce até aos talões ou ao calcanhar, (falando-se de vestuários): \textunderscore um padre, de hábito talar\textunderscore .
Asas que, segundo a Mythologia, Mercúrio tinha nos calcanhares.
\section{Talarejo}
\begin{itemize}
\item {Grp. gram.:m.}
\end{itemize}
\begin{itemize}
\item {Utilização:P. us.}
\end{itemize}
Peça do freio do cavallo.
\section{Talária}
\begin{itemize}
\item {Grp. gram.:f.}
\end{itemize}
\begin{itemize}
\item {Utilização:Ant.}
\end{itemize}
Talha para vinho. Cf. S. R. Viterbo, \textunderscore Elucidário\textunderscore .
\section{Talaveira}
\begin{itemize}
\item {Grp. gram.:m.}
\end{itemize}
\begin{itemize}
\item {Utilização:Bras}
\end{itemize}
\begin{itemize}
\item {Utilização:Bras. do S}
\end{itemize}
\begin{itemize}
\item {Proveniência:(Do cast. \textunderscore Talavera\textunderscore , n. p.?)}
\end{itemize}
Antiga designação burlesca de qualquer criado do paço.
O mesmo que \textunderscore português\textunderscore .
Maturrango.
\section{Talaveirada}
\begin{itemize}
\item {Grp. gram.:f.}
\end{itemize}
\begin{itemize}
\item {Utilização:Bras. do S}
\end{itemize}
Porção de talaveiras ou portugueses.
\section{Talavez}
\begin{itemize}
\item {Grp. gram.:adv.}
\end{itemize}
\begin{itemize}
\item {Utilização:Ant.}
\end{itemize}
\begin{itemize}
\item {Proveniência:(De \textunderscore tão\textunderscore  + \textunderscore a\textunderscore  + \textunderscore la\textunderscore  + \textunderscore vez\textunderscore )}
\end{itemize}
Um tanto; um pouco; raramente.
\section{Tálcico}
\begin{itemize}
\item {Grp. gram.:adj.}
\end{itemize}
Composto de talco.
\section{Talco}
\begin{itemize}
\item {Grp. gram.:m.}
\end{itemize}
\begin{itemize}
\item {Proveniência:(Do ár. \textunderscore talg\textunderscore , sebo)}
\end{itemize}
Silicato de magnésia, esverdeado, esbranquiçado ou pardo, susceptível de se dividir em lâminas, mais ou menos transparentes.
Pedra falsa:«\textunderscore ...chatins de talcos e avelórios.\textunderscore »Castilho, \textunderscore Sabichonas\textunderscore , 183.
Brilho falso.
\section{Talco-micáceo}
\begin{itemize}
\item {Grp. gram.:adj.}
\end{itemize}
\begin{itemize}
\item {Utilização:Miner.}
\end{itemize}
Que contém talco e mica.
\section{Talco-quartzoso}
\begin{itemize}
\item {Grp. gram.:adj.}
\end{itemize}
\begin{itemize}
\item {Utilização:Miner.}
\end{itemize}
Que contém talco e quartzo.
\section{Talcoso}
\begin{itemize}
\item {Grp. gram.:adj.}
\end{itemize}
Diz-se do terreno, em que há talco.
\section{Talcoxisto}
\begin{itemize}
\item {Grp. gram.:m.}
\end{itemize}
\begin{itemize}
\item {Utilização:Miner.}
\end{itemize}
\begin{itemize}
\item {Proveniência:(De \textunderscore talco\textunderscore  + \textunderscore xisto\textunderscore )}
\end{itemize}
Rocha primitiva, esverdeada e untuosa ao tacto.
\section{Talefe}
\begin{itemize}
\item {Grp. gram.:m.}
\end{itemize}
\begin{itemize}
\item {Utilização:Prov.}
\end{itemize}
\begin{itemize}
\item {Utilização:trasm.}
\end{itemize}
Marco geodésico, no alto dos montes.
(Por \textunderscore telégrafo\textunderscore ?)
\section{Taleiga}
\begin{itemize}
\item {Grp. gram.:f.}
\end{itemize}
\begin{itemize}
\item {Proveniência:(Do b. lat. \textunderscore talica\textunderscore )}
\end{itemize}
Saco, de maiores ou menores dimensões, e destinado especialmente á conducção de cereaes para os moínhos e da farinha que nestes se fabrica.
Antiga medida para líquidos e cereaes.
\section{Taleigada}
\begin{itemize}
\item {Grp. gram.:f.}
\end{itemize}
\begin{itemize}
\item {Proveniência:(De \textunderscore taleiga\textunderscore )}
\end{itemize}
O que uma taleiga póde conter.
Taleiga cheia.
\section{Taleigo}
\begin{itemize}
\item {Grp. gram.:m.}
\end{itemize}
\begin{itemize}
\item {Utilização:Prov.}
\end{itemize}
Taleiga pequena.
\textunderscore Dar aos taleigos\textunderscore , parolar, dar á língua. Cf. Camillo, \textunderscore Cav. em Ruínas\textunderscore , 168.
\section{Taleira}
\begin{itemize}
\item {Grp. gram.:f.}
\end{itemize}
\begin{itemize}
\item {Utilização:Carp.}
\end{itemize}
\begin{itemize}
\item {Utilização:Prov.}
\end{itemize}
\begin{itemize}
\item {Utilização:alent.}
\end{itemize}
\begin{itemize}
\item {Proveniência:(Do lat. \textunderscore tabularia\textunderscore )}
\end{itemize}
Cada uma das peças de madeira, que unem as falcas das carrêtas, na artilharia náutica.
Peça de madeira, que se mete na espessura de uma porta ou de um taipal, para lhe evitar o empeno.
Cada uma das travessas que ligam as chedas á pírtiga.
\section{Talen}
\begin{itemize}
\item {Grp. gram.:m.}
\end{itemize}
\begin{itemize}
\item {Utilização:Ant.}
\end{itemize}
O mesmo que \textunderscore talante\textunderscore . Cf. \textunderscore Port. Mon. Hist.\textunderscore , \textunderscore Scrip.\textunderscore , 244.
\section{Talentaço}
\begin{itemize}
\item {Grp. gram.:m.}
\end{itemize}
\begin{itemize}
\item {Utilização:Fam.}
\end{itemize}
Talento elevado.
Homem de elevado talento.
\section{Talentão}
\begin{itemize}
\item {Grp. gram.:m.}
\end{itemize}
Grande talento.
\section{Talentário}
\begin{itemize}
\item {Grp. gram.:m.}
\end{itemize}
\begin{itemize}
\item {Proveniência:(Lat. \textunderscore talentarius\textunderscore )}
\end{itemize}
Antiga máquina de guerra, com a qual se arremessavam pedras, que tinham o pêso de um talento.
\section{Talente}
\begin{itemize}
\item {Grp. gram.:m.}
\end{itemize}
(V.talante)
\section{Talento}
\begin{itemize}
\item {Grp. gram.:m.}
\end{itemize}
\begin{itemize}
\item {Utilização:Fig.}
\end{itemize}
\begin{itemize}
\item {Utilização:Prov.}
\end{itemize}
\begin{itemize}
\item {Utilização:bras}
\end{itemize}
\begin{itemize}
\item {Proveniência:(Lat. \textunderscore talentum\textunderscore )}
\end{itemize}
Nome de um pêso e de uma moéda, na antiguidade grega e romana.
Capacidade, distinta aptidão, habilidade natural.
Intelligência; engenho.
Homem talentoso.
Força phýsica, muscular; alento, esfôrço.
\section{Talento}
\begin{itemize}
\item {Grp. gram.:m.}
\end{itemize}
\begin{itemize}
\item {Utilização:Ant.}
\end{itemize}
O mesmo que \textunderscore talante\textunderscore .
\section{Talentoso}
\begin{itemize}
\item {Grp. gram.:adj.}
\end{itemize}
\begin{itemize}
\item {Proveniência:(De \textunderscore talento\textunderscore )}
\end{itemize}
Que tem talento; intelligente.
Hábil.
\section{Talentoso}
\begin{itemize}
\item {Grp. gram.:adj.}
\end{itemize}
\begin{itemize}
\item {Proveniência:(De \textunderscore talento\textunderscore ^2)}
\end{itemize}
Relativo a talante.
\section{Talépora}
\begin{itemize}
\item {Grp. gram.:f.}
\end{itemize}
Gênero de insectos coleópteros.
\section{Táler}
\begin{itemize}
\item {Grp. gram.:m.}
\end{itemize}
\begin{itemize}
\item {Proveniência:(Do al. \textunderscore thaler\textunderscore )}
\end{itemize}
Moéda aleman, que vale proximamente 675 reis.
\section{Taleto}
\begin{itemize}
\item {Grp. gram.:m.}
\end{itemize}
Nome de um quadrúpede brasileiro, pouco conhecido.
\section{Taleva}
\begin{itemize}
\item {Grp. gram.:f.}
\end{itemize}
Gênero de aves pernaltas.
\section{Talguênea}
\begin{itemize}
\item {Grp. gram.:f.}
\end{itemize}
Gênero de plantas rhamnáceas.
\section{Talha}
\begin{itemize}
\item {Grp. gram.:f.}
\end{itemize}
\begin{itemize}
\item {Utilização:Ant.}
\end{itemize}
\begin{itemize}
\item {Utilização:Ant.}
\end{itemize}
\begin{itemize}
\item {Utilização:Med.}
\end{itemize}
\begin{itemize}
\item {Proveniência:(Do lat. \textunderscore talea\textunderscore )}
\end{itemize}
Acto ou effeito de talhar.
Córte.
Entalhe.
Porção de metal, que o buril tira quando lavra.
Corda, que se prende á cana do leme, para melhor govêrno.
Apparelho náutico, composto de moitão, cadernal e cabo gornido.
Cabo, na testa de qualquer gávea, para facilitar a manobra de meter nos rizes.
Certo número de feixes de lenha de pinho ou de tojo: \textunderscore cincoenta mólhos de pinho fazem uma talha\textunderscore .
Representação do valor de uma entrada no voltarete.
Cartada.
Salário.
Espécie de tributo ou derrama.
Pedaço de madeira ou de um ramo, dividido diagonalmente em duas partes, em cada uma das quaes se escreviam algumas letras ou se punham alguns sinaes, indicando uma divida ou quitação della, e ficando uma daquellas partes nas mãos do credor.
Operação, que consiste em fazer uma abertura na bexiga, para se extrahirem calculos.
O mesmo que \textunderscore cystotomia\textunderscore .--A \textunderscore operação da talha\textunderscore  também designa, por analogia, a extracção de um corpo extranho de uma cavidade natural; o que se dá na talha \textunderscore intestinal\textunderscore , na \textunderscore estomacal\textunderscore , etc.
\section{Talha}
\begin{itemize}
\item {Grp. gram.:f.}
\end{itemize}
Vaso de barro, de grande bôjo.
Pote.
Grande vaso de lata.
Certo número de alqueires de sal, nas marinhas.
\section{Talha}
\begin{itemize}
\item {Grp. gram.:adj.}
\end{itemize}
\begin{itemize}
\item {Utilização:Ant.}
\end{itemize}
Dizia-se de uma longura, que constituía uma das medidas usadas pelos tanoeiros. Cf. Fern. Oliveira, \textunderscore Liv. da Fábr. das Naus\textunderscore , ms., 14.
\section{Talhada}
\begin{itemize}
\item {Grp. gram.:f.}
\end{itemize}
\begin{itemize}
\item {Proveniência:(De \textunderscore talhar\textunderscore )}
\end{itemize}
Porção cortada de certos corpos: \textunderscore uma talhada de melancia\textunderscore .
Fatia; naco.
\section{Talhadeira}
\begin{itemize}
\item {Grp. gram.:f.}
\end{itemize}
\begin{itemize}
\item {Proveniência:(De \textunderscore talhar\textunderscore )}
\end{itemize}
Instrumento, com que se talha.
\section{Tàlhadente}
\begin{itemize}
\item {Grp. gram.:m.}
\end{itemize}
\begin{itemize}
\item {Proveniência:(De \textunderscore talhar\textunderscore  + \textunderscore dente\textunderscore )}
\end{itemize}
Planta gramínea.
\section{Talhadia}
\begin{itemize}
\item {Grp. gram.:f.}
\end{itemize}
\begin{itemize}
\item {Proveniência:(De \textunderscore talhar\textunderscore )}
\end{itemize}
Operação de arboricultura, que consiste em desbastar os braços das árvores ou corta-los na extremidade.
\section{Talhadiço}
\begin{itemize}
\item {Grp. gram.:adj.}
\end{itemize}
\begin{itemize}
\item {Utilização:Bras}
\end{itemize}
\begin{itemize}
\item {Proveniência:(De \textunderscore talhar\textunderscore )}
\end{itemize}
Que se póde cortar ou roçar, (falando-se do mato).
\section{Talhado}
\begin{itemize}
\item {Grp. gram.:m.}
\end{itemize}
\begin{itemize}
\item {Utilização:Bras. do N}
\end{itemize}
\begin{itemize}
\item {Proveniência:(De \textunderscore talhar\textunderscore )}
\end{itemize}
Precipício, despenhadeiro.
\section{Talhadoiro}
\begin{itemize}
\item {Grp. gram.:m.}
\end{itemize}
\begin{itemize}
\item {Utilização:Prov.}
\end{itemize}
\begin{itemize}
\item {Proveniência:(De \textunderscore talhar\textunderscore )}
\end{itemize}
Lugar, onde se talha ou corta a água de rega.
\section{Talhador}
\begin{itemize}
\item {Grp. gram.:m.  e  adj.}
\end{itemize}
\begin{itemize}
\item {Grp. gram.:M.}
\end{itemize}
\begin{itemize}
\item {Proveniência:(De \textunderscore talhar\textunderscore )}
\end{itemize}
O que talha.
Aquelle que corta a carne nos açougues; cortador.
Cutello para cortar carne.
Prato, em que se trincha carne.
\section{Talhadouro}
\begin{itemize}
\item {Grp. gram.:m.}
\end{itemize}
\begin{itemize}
\item {Utilização:Prov.}
\end{itemize}
\begin{itemize}
\item {Proveniência:(De \textunderscore talhar\textunderscore )}
\end{itemize}
Lugar, onde se talha ou corta a água de rega.
\section{Talhadura}
\begin{itemize}
\item {Grp. gram.:f.}
\end{itemize}
\begin{itemize}
\item {Utilização:Ant.}
\end{itemize}
Acto ou effeito de talhar.
Porção de água que, de uma fonte ou poço, pertence ao dono de um terreno, para a rega dêste.
\section{Tàlhafrio}
\begin{itemize}
\item {Grp. gram.:m.}
\end{itemize}
\begin{itemize}
\item {Proveniência:(De \textunderscore talhar\textunderscore )}
\end{itemize}
Instrumento de marceneiro, para lavrar madeira.
\section{Tàlhamar}
\begin{itemize}
\item {Grp. gram.:m.}
\end{itemize}
\begin{itemize}
\item {Proveniência:(De \textunderscore talhar\textunderscore  + \textunderscore mar\textunderscore )}
\end{itemize}
Beque do navio.
Construcção de pedra, em fórma angular, num caes ou numa ponte, para quebrar a fôrça da corrente.
Ave, o mesmo que \textunderscore taiataia\textunderscore .
\section{Talhame}
\begin{itemize}
\item {Grp. gram.:m.}
\end{itemize}
\begin{itemize}
\item {Utilização:Náut.}
\end{itemize}
\begin{itemize}
\item {Proveniência:(De \textunderscore talha\textunderscore ^1)}
\end{itemize}
Conjunto de talhas, candeliças, etc., que se guardam no trem de manobra dos arsenaes marítimos ou no paiol do mestre, a bordo.
\section{Talhamento}
\begin{itemize}
\item {Grp. gram.:m.}
\end{itemize}
\begin{itemize}
\item {Utilização:Ant.}
\end{itemize}
Acto ou effeito de talhar.
O mesmo que \textunderscore talha\textunderscore ^1, imposto.
\section{Talhante}
\begin{itemize}
\item {Grp. gram.:adj.}
\end{itemize}
\begin{itemize}
\item {Grp. gram.:M.}
\end{itemize}
\begin{itemize}
\item {Utilização:Constr.}
\end{itemize}
Que talha.
Talhamar.
Peça de ferro para armar inferiormente a corôa de madeira, que é base das columnas ocas de alvenaria, que se enterram no solo, para depois de cheias de areia e betão, servirem, nos terrenos molles ou aquíferos, de pilares dos arcos ou abóbadas, que sustentam as alvenarias superiores de certas construcções.
\section{Talhão}
\begin{itemize}
\item {Grp. gram.:m.}
\end{itemize}
\begin{itemize}
\item {Proveniência:(De \textunderscore talhar\textunderscore )}
\end{itemize}
Terreno cultivado ou próprio para cultura, entre dois regos.
Tabuleiro.
\section{Talhão}
\begin{itemize}
\item {Grp. gram.:m.}
\end{itemize}
\begin{itemize}
\item {Utilização:Açor}
\end{itemize}
\begin{itemize}
\item {Proveniência:(De \textunderscore talha\textunderscore ^1)}
\end{itemize}
Pote para água.
\section{Talhar}
\begin{itemize}
\item {Grp. gram.:v. t.}
\end{itemize}
\begin{itemize}
\item {Utilização:Pop.}
\end{itemize}
\begin{itemize}
\item {Utilização:Fig.}
\end{itemize}
\begin{itemize}
\item {Grp. gram.:V. i.}
\end{itemize}
\begin{itemize}
\item {Utilização:Ant.}
\end{itemize}
\begin{itemize}
\item {Grp. gram.:V. p.}
\end{itemize}
\begin{itemize}
\item {Proveniência:(Do lat. \textunderscore taleare\textunderscore )}
\end{itemize}
Cortar; golpear.
Gravar, esculpir.
Talar.
Cortar por medida (pano para fato, coiro para calçado, etc.).
Ajustar, adaptar.
Sulcar.
Dividir; repartir.
Atalhar, obstar a.
Preparar; predispor.
Cortar o pano para fato, (falando-se do alfaiate ou costureira).
Decompor-se, (falando-se do leite).
Fazer banca ou sêr banqueiro (no jôgo do monte ou de asar).
Tomar caminho direito, ir em certa direcção.
Decompor-se (o leite).
Abrir-se, rachar-se.
Embotar-se, (falando-se dos dentes).
\section{Talharia}
\begin{itemize}
\item {Grp. gram.:f.}
\end{itemize}
\begin{itemize}
\item {Proveniência:(De \textunderscore talho\textunderscore  ou \textunderscore talha\textunderscore )}
\end{itemize}
Grande porção de talhos ou talhas.
\section{Talharim}
\begin{itemize}
\item {Grp. gram.:m.}
\end{itemize}
\begin{itemize}
\item {Proveniência:(Do it. \textunderscore taglierini\textunderscore )}
\end{itemize}
Massa, em fórma de tiras, para sopa.
\section{Talharola}
\begin{itemize}
\item {Grp. gram.:f.}
\end{itemize}
Instrumento, para cortar os fios ou orelhas, que ficam fóra da trama, no fabríco do velludo. Cf. G. Viana, \textunderscore Apostilas\textunderscore .
\section{Talhe}
\begin{itemize}
\item {Grp. gram.:m.}
\end{itemize}
\begin{itemize}
\item {Proveniência:(De \textunderscore talhar\textunderscore )}
\end{itemize}
Conformação do corpo.
Estatura.
Feição.
O mesmo que \textunderscore talho\textunderscore .
\section{Talhér}
\begin{itemize}
\item {Grp. gram.:m.}
\end{itemize}
\begin{itemize}
\item {Utilização:Fig.}
\end{itemize}
\begin{itemize}
\item {Proveniência:(Do it. \textunderscore tagliere\textunderscore )}
\end{itemize}
Reunião de garfo, colhér e faca.
Galheteiro.
Lugar para cada pessôa, á mesa.
\section{Talhinha}
\begin{itemize}
\item {Grp. gram.:f.}
\end{itemize}
Apparelho náutico, para levantar pequenos pesos.
\section{Talho}
\begin{itemize}
\item {Grp. gram.:m.}
\end{itemize}
\begin{itemize}
\item {Utilização:Fig.}
\end{itemize}
\begin{itemize}
\item {Utilização:Prov.}
\end{itemize}
\begin{itemize}
\item {Utilização:trasm.}
\end{itemize}
\begin{itemize}
\item {Proveniência:(Do b. lat. \textunderscore talium\textunderscore )}
\end{itemize}
O mesmo que \textunderscore talhamento\textunderscore .
Compartimento nas salinas.
Córte da carne, no açougue.
Desbaste dos ramos das árvores.
Talhão, leira.
Acto de atalhar ou reprimir.
Cepo, sôbre que se corta a carne, no açougue.
O mesmo que \textunderscore açougue\textunderscore .
Feição.
Banco pequeno e tôsco.
\textunderscore Vir a talho\textunderscore , ou \textunderscore vir a talho de foice\textunderscore , vir a propósito. Cf. Camillo, \textunderscore Onde está a Felic.\textunderscore , 379.
\section{Tali}
\begin{itemize}
\item {Grp. gram.:m.}
\end{itemize}
O mesmo que \textunderscore talim\textunderscore .
\section{Tália}
\begin{itemize}
\item {Grp. gram.:f.}
\end{itemize}
Espécie de uva branca dos Açores e da Extremadura.
\section{Talião}
\begin{itemize}
\item {Grp. gram.:m.}
\end{itemize}
\begin{itemize}
\item {Proveniência:(Lat. \textunderscore talio\textunderscore )}
\end{itemize}
Desforra, igual á offensa.
Pena, igual á culpa.
Castigo, análogo ao acto punido.
\section{Talicão}
\begin{itemize}
\item {Grp. gram.:m.}
\end{itemize}
\begin{itemize}
\item {Utilização:Prov.}
\end{itemize}
Fragmento curto de vide, que se deixa ás videiras, na occasião da poda, para futura rebentação.
\section{Taliga}
\begin{itemize}
\item {Grp. gram.:f.}
\end{itemize}
(V.taleiga)
\section{Talim}
\begin{itemize}
\item {Grp. gram.:m.}
\end{itemize}
\begin{itemize}
\item {Proveniência:(Do ár. \textunderscore tahlil\textunderscore )}
\end{itemize}
Correia, que se aperta á cinta e donde pende a espada; boldrié.
\section{Talina}
\begin{itemize}
\item {Grp. gram.:f.}
\end{itemize}
Gênero de plantas portuláceas.
\section{Talinga}
\begin{itemize}
\item {Grp. gram.:f.}
\end{itemize}
\begin{itemize}
\item {Utilização:Náut.}
\end{itemize}
Cabo, amarra.
\section{Talingadura}
\begin{itemize}
\item {Grp. gram.:f.}
\end{itemize}
Acto ou effeito de talingar.
\section{Talingar}
\begin{itemize}
\item {Grp. gram.:v.}
\end{itemize}
\begin{itemize}
\item {Utilização:t. Náut.}
\end{itemize}
Ligar ou atar com talinga; ligar.
\section{Talinheira}
\begin{itemize}
\item {Grp. gram.:f.}
\end{itemize}
\begin{itemize}
\item {Utilização:Prov.}
\end{itemize}
\begin{itemize}
\item {Utilização:trasm.}
\end{itemize}
Azinhaga; quelha.
\section{Talinhos}
\begin{itemize}
\item {Grp. gram.:m. pl.}
\end{itemize}
Espécie de jôgo popular.
\section{Talintona}
\begin{itemize}
\item {Grp. gram.:f.  e  adj.}
\end{itemize}
\begin{itemize}
\item {Utilização:Ant.}
\end{itemize}
Mulhér diligente no govêrno de sua casa.
(Provavelmente por \textunderscore talentona\textunderscore , de \textunderscore talento\textunderscore ^1)
\section{Talionar}
\begin{itemize}
\item {Grp. gram.:v. t.}
\end{itemize}
\begin{itemize}
\item {Proveniência:(Do lat. \textunderscore talio\textunderscore , \textunderscore talionis\textunderscore )}
\end{itemize}
Applicar o talião a.
\section{Talionato}
\begin{itemize}
\item {Grp. gram.:m.}
\end{itemize}
\begin{itemize}
\item {Proveniência:(Do lat. \textunderscore talio\textunderscore , \textunderscore talionis\textunderscore )}
\end{itemize}
Pena de talião.
\section{Talisca}
\begin{itemize}
\item {Grp. gram.:f.}
\end{itemize}
Fenda.
Estilha; pequena lasca.
\section{Talísia}
\begin{itemize}
\item {Grp. gram.:f.}
\end{itemize}
Gênero de plantas sapindáceas.
\section{Talismã}
\begin{itemize}
\item {Grp. gram.:m.}
\end{itemize}
\begin{itemize}
\item {Utilização:Fig.}
\end{itemize}
\begin{itemize}
\item {Proveniência:(Do ár. \textunderscore telsaman\textunderscore )}
\end{itemize}
Figura ou caracteres, gravados em pedra ou metal, e a que se attribuem qualidades ou virtudes sobrenaturaes.
Objecto, a que se attribuem propriedades sobrenaturaes, como a de livrar de perigos ou de certos males quem o traz consigo.
Amuleto.
Encanto.
\section{Talisman}
\begin{itemize}
\item {Grp. gram.:m.}
\end{itemize}
\begin{itemize}
\item {Utilização:Fig.}
\end{itemize}
\begin{itemize}
\item {Proveniência:(Do ár. \textunderscore telsaman\textunderscore )}
\end{itemize}
Figura ou caracteres, gravados em pedra ou metal, e a que se attribuem qualidades ou virtudes sobrenaturaes.
Objecto, a que se attribuem propriedades sobrenaturaes, como a de livrar de perigos ou de certos males quem o traz consigo.
Amuleto.
Encanto.
\section{Talismânico}
\begin{itemize}
\item {Grp. gram.:adj.}
\end{itemize}
Relativo ao talisman; que tem a supposta virtude dos talismans.
\section{Talitre}
\begin{itemize}
\item {Grp. gram.:m.}
\end{itemize}
\begin{itemize}
\item {Proveniência:(Lat. \textunderscore talitrum\textunderscore )}
\end{itemize}
Piparote.
Nó, na articulação dos dedos.
Gênero de crustáceos amphípodes.
\section{Talitro}
\begin{itemize}
\item {Grp. gram.:m.}
\end{itemize}
\begin{itemize}
\item {Proveniência:(Lat. \textunderscore talitrum\textunderscore )}
\end{itemize}
Piparote.
Nó, na articulação dos dedos.
Gênero de crustáceos amphípodes.
\section{Talmud}
\begin{itemize}
\item {Grp. gram.:m.}
\end{itemize}
\begin{itemize}
\item {Proveniência:(Hebr. \textunderscore talmud\textunderscore )}
\end{itemize}
Antiga collecção de leis, tradições e costumes, compilada pelos doutores judeus.
\section{Talmúdico}
\begin{itemize}
\item {Grp. gram.:adj.}
\end{itemize}
Relativo ao \textunderscore Talmud\textunderscore .
\section{Talmudista}
\begin{itemize}
\item {Grp. gram.:m. ,  f.  e  adj.}
\end{itemize}
Diz se da pessôa que explica ou segue as opiniões do \textunderscore Talmud\textunderscore .
\section{Talo}
\begin{itemize}
\item {Grp. gram.:m.}
\end{itemize}
\begin{itemize}
\item {Utilização:Anat.}
\end{itemize}
\begin{itemize}
\item {Proveniência:(Gr. \textunderscore thallos\textunderscore )}
\end{itemize}
O mesmo que \textunderscore caule\textunderscore .
Pecíolo.
Fibra grossa, que corre ao meio das fôlhas da planta, confundindo-se com o pecíolo.
Parte de um pêlo ou cabello, para fóra da pelle.
Taioba.
Fuste ou tronco de columna, sem base nem capitel.
\section{Taloca}
\begin{itemize}
\item {Grp. gram.:f.}
\end{itemize}
\begin{itemize}
\item {Utilização:Prov.}
\end{itemize}
O mesmo que \textunderscore toca\textunderscore ^1.
\section{Talocha}
\begin{itemize}
\item {Grp. gram.:f.}
\end{itemize}
Pequena tábua ou espécie de esparavel, com que os pedreiros ou estucadores põem cal e areia nos cantos dos tectos das casas.
\section{Taloeira}
\begin{itemize}
\item {Grp. gram.:f.}
\end{itemize}
\begin{itemize}
\item {Utilização:Pesc.}
\end{itemize}
\begin{itemize}
\item {Proveniência:(De \textunderscore tala\textunderscore ^1)}
\end{itemize}
Apparelho, para fisgar lulas e chocos.
\section{Talona}
\begin{itemize}
\item {Grp. gram.:f.}
\end{itemize}
Gênero de plantas apocýneas da Índia Portuguesa, o mesmo que \textunderscore pau-de-cobra\textunderscore , (\textunderscore ranwolfia serpentina\textunderscore , Benth.)
\section{Talondos}
\begin{itemize}
\item {Grp. gram.:m.}
\end{itemize}
Pêso grego, de 150 quilogrammas.
\section{Taloso}
\begin{itemize}
\item {Grp. gram.:adj.}
\end{itemize}
Que tem talos.
Relativo a talos.
\section{Taloucada}
\begin{itemize}
\item {Grp. gram.:f.}
\end{itemize}
\begin{itemize}
\item {Utilização:T. da Bairrada}
\end{itemize}
Bordoada, paulada.
\section{Talpária}
\begin{itemize}
\item {Grp. gram.:f.}
\end{itemize}
\begin{itemize}
\item {Utilização:Med.}
\end{itemize}
\begin{itemize}
\item {Proveniência:(Do lat. \textunderscore talpa\textunderscore )}
\end{itemize}
Abscessos no pericrânio.
\section{Tal-qualmente}
\begin{itemize}
\item {Grp. gram.:loc. adv.  e  conj.}
\end{itemize}
\begin{itemize}
\item {Utilização:Fam.}
\end{itemize}
Igualmente; da mesma maneira que:«\textunderscore ...tal qualmente a intervenção do Estado se accentua...\textunderscore »Sousa Martins, \textunderscore Nosographia\textunderscore . Cf. Garrett. \textunderscore Port. na Balança\textunderscore , 221.
\section{Taluda}
\begin{itemize}
\item {Grp. gram.:f.}
\end{itemize}
\begin{itemize}
\item {Utilização:Pop.}
\end{itemize}
\begin{itemize}
\item {Proveniência:(De \textunderscore taludo\textunderscore )}
\end{itemize}
O prêmio maior, nas lotarias.
\section{Taludão}
\begin{itemize}
\item {Grp. gram.:m.}
\end{itemize}
Rapaz muito taludo ou muito desenvolvido physicamente.
\section{Taludar}
\begin{itemize}
\item {Grp. gram.:v. t.}
\end{itemize}
Dar talude ou inclinação a.
Dispôr em talude.
\section{Talude}
\begin{itemize}
\item {Grp. gram.:m.}
\end{itemize}
Inclinação na superfície lateral de um terreno, de um muro ou de qualquer obra; rampa, escarpa.
(T. mal formado do lat. \textunderscore talus\textunderscore , como se o genitivo fosse \textunderscore taludis\textunderscore , á semelhança de \textunderscore palus\textunderscore , \textunderscore paludis\textunderscore )?
\section{Taludo}
\begin{itemize}
\item {Grp. gram.:adj.}
\end{itemize}
\begin{itemize}
\item {Proveniência:(De \textunderscore talo\textunderscore )}
\end{itemize}
Que tem talo duro.
Corpulento.
Muito desenvolvido.
Grande.
\section{Talvez}
\begin{itemize}
\item {Grp. gram.:adv.}
\end{itemize}
\begin{itemize}
\item {Proveniência:(De \textunderscore tal\textunderscore  + \textunderscore vez\textunderscore )}
\end{itemize}
É possível; porventura; quiçá.
\section{Tam}
\begin{itemize}
\item {Grp. gram.:adv.}
\end{itemize}
(V.tão)
\section{Tamacarica}
\begin{itemize}
\item {Grp. gram.:f.}
\end{itemize}
\begin{itemize}
\item {Utilização:Bras}
\end{itemize}
Tolde de embarcação.
\section{Tamalanes}
\begin{itemize}
\item {Grp. gram.:adj.}
\end{itemize}
\begin{itemize}
\item {Utilização:Ant.}
\end{itemize}
Desassisado.
Imprudente; atoleimado.
\section{Tamalavez}
\begin{itemize}
\item {Grp. gram.:adv.}
\end{itemize}
\begin{itemize}
\item {Utilização:Ant.}
\end{itemize}
Um tanto; de algum modo:«\textunderscore ...com arroz cozido..., tamalaves, sobe-lo seco...\textunderscore »Fr. Gaspar da Cruz, \textunderscore Tratado\textunderscore , cap. X. Cf. G. Vicente, I, 250; F. Manuel, \textunderscore Carta de Guia\textunderscore , 148; Bernardim Ribeiro, etc.
\section{Tamanca}
\begin{itemize}
\item {Grp. gram.:f.}
\end{itemize}
\begin{itemize}
\item {Utilização:Prov.}
\end{itemize}
Tamanco baixo e de entrada muito aberta, usado por mulheres.
E o mesmo que \textunderscore tamancada\textunderscore .
\section{Tamancada}
\begin{itemize}
\item {Grp. gram.:f.}
\end{itemize}
Pancada com tamanco.
\section{Tamancaria}
\begin{itemize}
\item {Grp. gram.:f.}
\end{itemize}
Officina, onde se fabricam tamancos, ou lugar onde se vendem. Cf. \textunderscore Inquér. Industr.\textunderscore , III p., 283.
\section{Tamanco}
\begin{itemize}
\item {Grp. gram.:m.}
\end{itemize}
Calçado grosseiro, sem talão ou com talão muito baixo, e com a base de madeira em vez de sola.
Utensílio de marnoto, também conhecido por \textunderscore pé de pau\textunderscore .
\section{Tamanduá}
\begin{itemize}
\item {Grp. gram.:m.}
\end{itemize}
\begin{itemize}
\item {Utilização:Bras}
\end{itemize}
Quadrúpede desdentado.
\section{Tamanduá}
\begin{itemize}
\item {Grp. gram.:m.}
\end{itemize}
\begin{itemize}
\item {Utilização:Bras}
\end{itemize}
Questão moral, de diffícil solução.
Grande mentira, carapetão.
\section{Tamanhão}
\begin{itemize}
\item {Grp. gram.:adj.}
\end{itemize}
\begin{itemize}
\item {Grp. gram.:M.  e  f.}
\end{itemize}
\begin{itemize}
\item {Utilização:Fam.}
\end{itemize}
\begin{itemize}
\item {Proveniência:(De \textunderscore tamanho\textunderscore )}
\end{itemize}
Muito grande.
Pessôa encorpada.
Pessôa robusta e alta.
\section{Tamanhinho}
\begin{itemize}
\item {Grp. gram.:adj.}
\end{itemize}
\begin{itemize}
\item {Proveniência:(De \textunderscore tamanho\textunderscore )}
\end{itemize}
Muito pequeno; pequenino.
\section{Tamanho}
\begin{itemize}
\item {Grp. gram.:adj.}
\end{itemize}
\begin{itemize}
\item {Grp. gram.:M.}
\end{itemize}
\begin{itemize}
\item {Proveniência:(Do lat. \textunderscore tam\textunderscore  + \textunderscore magnus\textunderscore )}
\end{itemize}
Tão grande; tão distinto.
Volume, grandeza.
\section{Tamanhouço}
\begin{itemize}
\item {Grp. gram.:adj.}
\end{itemize}
\begin{itemize}
\item {Utilização:Des.}
\end{itemize}
O mesmo que \textunderscore tamanhão\textunderscore :«\textunderscore era um zote tamanhouço.\textunderscore »G. Vicente, \textunderscore Inês Pereira\textunderscore .
\section{Tamanino}
\begin{itemize}
\item {Grp. gram.:adj.}
\end{itemize}
\begin{itemize}
\item {Grp. gram.:M.}
\end{itemize}
\begin{itemize}
\item {Utilização:Ant.}
\end{itemize}
O mesmo que \textunderscore tamanhinho\textunderscore :«\textunderscore ...um bichinho tamanino, que quási escapa da vista.\textunderscore »\textunderscore Luz e Calor\textunderscore , 554.
Bocadinho, pequena porção. Cf. G. Vicente.
\section{Tamanquear}
\begin{itemize}
\item {Grp. gram.:v. i.}
\end{itemize}
Andar de tamancos.
Fazer bulha, andando de tamancos.
\section{Tamanqueira}
\begin{itemize}
\item {Grp. gram.:f.}
\end{itemize}
\begin{itemize}
\item {Utilização:Bras}
\end{itemize}
Árvore silvestre.
O mesmo que \textunderscore tamaquaré\textunderscore ?
\section{Tamanqueiro}
\begin{itemize}
\item {Grp. gram.:m.}
\end{itemize}
Aquelle que faz ou vende tamancos.
\section{Tamão}
\begin{itemize}
\item {Grp. gram.:m.}
\end{itemize}
\begin{itemize}
\item {Utilização:Prov.}
\end{itemize}
\begin{itemize}
\item {Utilização:trasm.}
\end{itemize}
O mesmo que \textunderscore temão\textunderscore .
\section{Tamaotarana}
\begin{itemize}
\item {Grp. gram.:f.}
\end{itemize}
O mesmo que \textunderscore mendobi\textunderscore .
\section{Tamaquaré}
\begin{itemize}
\item {Grp. gram.:m.}
\end{itemize}
Árvore do Norte do Brasil.
Óleo medicinal, fabricado com a seiva da mesma árvore.
\section{Tâmara}
\begin{itemize}
\item {Grp. gram.:f.}
\end{itemize}
\begin{itemize}
\item {Proveniência:(Do ár. \textunderscore tamr\textunderscore )}
\end{itemize}
Fruto da tamareira e de outras palmeiras.
Casta de uva branca de Ourém.
\section{Tamaral}
\begin{itemize}
\item {Grp. gram.:m.}
\end{itemize}
Arvoredo de tamareiras:«\textunderscore ...na queimada África prenhe de formosos tamaraes.\textunderscore »Usque.
\section{Tamarana}
\begin{itemize}
\item {Grp. gram.:m.}
\end{itemize}
\begin{itemize}
\item {Utilização:Bras}
\end{itemize}
Arma, usada na guerra por alguns selvagens, e feita de pau-roxo ou de pau-ferro, espalmada, com dois gumes, á maneira de espada.
\section{Tamarança}
\begin{itemize}
\item {Grp. gram.:f.}
\end{itemize}
\begin{itemize}
\item {Utilização:Prov.}
\end{itemize}
\begin{itemize}
\item {Utilização:dur.}
\end{itemize}
O mesmo que \textunderscore raposa\textunderscore .
\section{Tamararés}
\begin{itemize}
\item {Grp. gram.:m. pl.}
\end{itemize}
(V.tamarés)
\section{Tamareira}
\begin{itemize}
\item {Grp. gram.:f.}
\end{itemize}
\begin{itemize}
\item {Proveniência:(De \textunderscore tâmara\textunderscore )}
\end{itemize}
Espécie de palmeira.
\section{Tamarés}
\begin{itemize}
\item {Grp. gram.:m. pl.}
\end{itemize}
Tríbo de Índios do Brasil, em Mato Grosso.
\section{Tamarês}
\begin{itemize}
\item {Grp. gram.:m.  e  adj.}
\end{itemize}
\begin{itemize}
\item {Proveniência:(De \textunderscore tâmara\textunderscore )}
\end{itemize}
Diz-se de uma espécie de uva branca.
\section{Tamargal}
\begin{itemize}
\item {Grp. gram.:m.}
\end{itemize}
Terreno, onde crescem tamargueiras.
(Por \textunderscore tamargueiral\textunderscore , de \textunderscore tamargueira\textunderscore )
\section{Tamargueira}
\begin{itemize}
\item {Grp. gram.:f.}
\end{itemize}
Arbusto, de casca adstringente, (\textunderscore tamarix gallica\textunderscore ).
\section{Tamaricáceas}
\begin{itemize}
\item {Grp. gram.:f. pl.}
\end{itemize}
\begin{itemize}
\item {Proveniência:(Do lat. \textunderscore tamarix\textunderscore )}
\end{itemize}
O mesmo ou melhór que \textunderscore tamarináceas\textunderscore .
\section{Tamarináceas}
\begin{itemize}
\item {Grp. gram.:f. pl.}
\end{itemize}
Família de plantas, o mesmo que \textunderscore tamariscíneas\textunderscore .
(Fem. pl. de \textunderscore tamarináceo\textunderscore )
\section{Tamarináceo}
\begin{itemize}
\item {Grp. gram.:adj.}
\end{itemize}
Relativo ou semelhante á tamargueira.
\section{Tamarindal}
\begin{itemize}
\item {Grp. gram.:m.}
\end{itemize}
Lugar, onde crescem tamarindos.
\section{Tamarindeira}
\begin{itemize}
\item {Grp. gram.:f.}
\end{itemize}
O mesmo que \textunderscore tamarindeiro\textunderscore . Cf. Camillo, \textunderscore Estrêll. Prop.\textunderscore , 90.
\section{Tamarindeiro}
\begin{itemize}
\item {Grp. gram.:m.}
\end{itemize}
Árvore, o mesmo que \textunderscore tamarindo\textunderscore .
\section{Tamarindo}
\begin{itemize}
\item {Grp. gram.:m.}
\end{itemize}
\begin{itemize}
\item {Proveniência:(Do ár. \textunderscore tamr-hindi\textunderscore )}
\end{itemize}
Gênero de árvores leguminosas.
Fruto dessas árvores.
\section{Tamarineiro}
\begin{itemize}
\item {Grp. gram.:m.}
\end{itemize}
O mesmo que \textunderscore tamarindo\textunderscore .
\section{Tamarinheiro}
\begin{itemize}
\item {Grp. gram.:m.}
\end{itemize}
O mesmo que \textunderscore tamarindo\textunderscore .
\section{Tamarinho}
\begin{itemize}
\item {Grp. gram.:m.}
\end{itemize}
O mesmo que \textunderscore tamarindo\textunderscore .
\section{Tamarino}
\begin{itemize}
\item {Grp. gram.:m.}
\end{itemize}
Espécie de mammíferos quadrúmanos.
\section{Tamarino}
\begin{itemize}
\item {Grp. gram.:m.}
\end{itemize}
Espécie de cana de açúcar.
\section{Tamariscíneas}
\begin{itemize}
\item {Grp. gram.:f. pl.}
\end{itemize}
\begin{itemize}
\item {Proveniência:(Do lat. \textunderscore tamariscus\textunderscore )}
\end{itemize}
Família de plantas, tiradas das portuláceas e que destas só differe pela ausência do perisperma.
\section{Tamaru}
\begin{itemize}
\item {Grp. gram.:m.}
\end{itemize}
Crustáceo do Brasil.
\section{Tamariz}
\begin{itemize}
\item {Grp. gram.:m.}
\end{itemize}
\begin{itemize}
\item {Proveniência:(Do lat. \textunderscore tamarix\textunderscore )}
\end{itemize}
O mesmo que \textunderscore tamargueira\textunderscore .
\section{Tamatiá}
\begin{itemize}
\item {Grp. gram.:m.}
\end{itemize}
Nome de várias aves trepadoras.
\section{Tamaxeque}
\begin{itemize}
\item {Grp. gram.:m.}
\end{itemize}
A língua berbere.
\section{Tambaca}
\begin{itemize}
\item {Grp. gram.:f.}
\end{itemize}
\begin{itemize}
\item {Proveniência:(Do mal. \textunderscore tambaga\textunderscore )}
\end{itemize}
Metal, composto de cobre e zinco.
Mistura fundida de oiro e prata. Cf. \textunderscore Luz e Calor\textunderscore , 445.
\section{Tambaíba}
\begin{itemize}
\item {Grp. gram.:f.}
\end{itemize}
\begin{itemize}
\item {Utilização:Bras}
\end{itemize}
Árvore silvestre, cuja madeira, listrada de preto e amarelo, se emprega em marcenaria.
\section{Tambaque}
\begin{itemize}
\item {Grp. gram.:m.}
\end{itemize}
\begin{itemize}
\item {Utilização:Bras}
\end{itemize}
O mesmo que \textunderscore tabaque\textunderscore ^2.
\section{Tambaque}
\begin{itemize}
\item {Grp. gram.:m.}
\end{itemize}
Espécie de tambor, o mesmo que \textunderscore tambaca\textunderscore . Cf. M. Feijó, \textunderscore Orthogr.\textunderscore 
\section{Tambaqui}
\begin{itemize}
\item {Grp. gram.:m.}
\end{itemize}
\begin{itemize}
\item {Utilização:Bras}
\end{itemize}
Saboroso peixe dos rios do Pará e do Amazonas.
\section{Tambarane}
\begin{itemize}
\item {Grp. gram.:m.}
\end{itemize}
Pedra branca, espécie de amuleto, que os sacerdotes gentios da Índia trazem ao pescoço.
\section{Tambatajá}
\begin{itemize}
\item {Grp. gram.:m.}
\end{itemize}
\begin{itemize}
\item {Utilização:Bras}
\end{itemize}
Planta, espécie de jarro, (\textunderscore caladium auritum bicolor\textunderscore ).
\section{Tamba-tica}
\begin{itemize}
\item {Grp. gram.:f.}
\end{itemize}
Arbusto de Moçambique.
\section{Tambeira}
\begin{itemize}
\item {Grp. gram.:f.}
\end{itemize}
\begin{itemize}
\item {Utilização:Des.}
\end{itemize}
\begin{itemize}
\item {Proveniência:(De \textunderscore tambo\textunderscore )}
\end{itemize}
Madrinha da noiva, a qual a conduz á cama.
\section{Tambeiro}
\begin{itemize}
\item {Grp. gram.:adj.}
\end{itemize}
\begin{itemize}
\item {Utilização:Bras. do S}
\end{itemize}
\begin{itemize}
\item {Proveniência:(De \textunderscore tambo\textunderscore ?)}
\end{itemize}
Diz-se do gado manso, que vive perto das habitações.
\section{Também}
\begin{itemize}
\item {Grp. gram.:adj. conj.}
\end{itemize}
\begin{itemize}
\item {Utilização:Fam.}
\end{itemize}
\begin{itemize}
\item {Proveniência:(De \textunderscore tam\textunderscore  + \textunderscore bem\textunderscore )}
\end{itemize}
Da mesma fórma; igualmente; outrosim.
Com effeito, realmente.
\section{Tambero}
\begin{itemize}
\item {Grp. gram.:m.}
\end{itemize}
\begin{itemize}
\item {Utilização:Bras}
\end{itemize}
Animal, muito bem domesticado.
(Cp. \textunderscore tambeiro\textunderscore )
\section{Tâmbi}
\begin{itemize}
\item {Grp. gram.:m.}
\end{itemize}
Luto ou nojo, entre os indígenas de Angola.
(Quimbunde \textunderscore tambi\textunderscore )
\section{Tambica}
\begin{itemize}
\item {Grp. gram.:f.}
\end{itemize}
Chumbo da rede.
\section{Tambió}
\begin{itemize}
\item {Grp. gram.:m.}
\end{itemize}
\begin{itemize}
\item {Utilização:T. da Índia Portuguesa}
\end{itemize}
\begin{itemize}
\item {Proveniência:(Do conc. \textunderscore taãbẽ\textunderscore )}
\end{itemize}
Bilha ou cântaro, com que se tira a água da fonte.
Jarro de cobre.
\section{Tambo}
\begin{itemize}
\item {Grp. gram.:m.}
\end{itemize}
\begin{itemize}
\item {Utilização:Des.}
\end{itemize}
O mesmo que \textunderscore thálamo\textunderscore .
Boda de casamento.
Mesa baixa, em que os frades, por castigo, comiam, dentro do refeitório.
(Corr. de \textunderscore thálamo\textunderscore )
\section{Tamboatá}
\begin{itemize}
\item {Grp. gram.:m.}
\end{itemize}
Peixe do Brasil.
\section{Tamboeira}
\begin{itemize}
\item {Grp. gram.:f.}
\end{itemize}
\begin{itemize}
\item {Utilização:Bras}
\end{itemize}
Cana de mandioca, pouco desenvolvida.
Maçaroca do milho.
(Do tupi)
\section{Tamboladeira}
\begin{itemize}
\item {Grp. gram.:f.}
\end{itemize}
\begin{itemize}
\item {Utilização:Prov.}
\end{itemize}
(V.tambuladeira)
\section{Tambono}
\begin{itemize}
\item {Grp. gram.:m.}
\end{itemize}
Planta indiana.
\section{Tambor}
\begin{itemize}
\item {Grp. gram.:m.}
\end{itemize}
\begin{itemize}
\item {Utilização:Prov.}
\end{itemize}
\begin{itemize}
\item {Utilização:minh.}
\end{itemize}
Caixa, de fórma cylíndrica, com dois fundos de pelle tensa, sôbre um dos quaes se toca com baquetas.
Indivíduo, que toca tambor: \textunderscore o João é tambor do regimento\textunderscore .
Týmpano do ouvido.
Nome de vários objectos, de fórma cylíndrica.
Árvore leguminosa brasileira, (\textunderscore mimosa carunculata\textunderscore ).
O mesmo que \textunderscore goraz\textunderscore .
(Talvez do ár. \textunderscore tambur\textunderscore )
\section{Tamborete}
\begin{itemize}
\item {fónica:borê}
\end{itemize}
\begin{itemize}
\item {Grp. gram.:m.}
\end{itemize}
\begin{itemize}
\item {Utilização:Prov.}
\end{itemize}
\begin{itemize}
\item {Grp. gram.:Pl.}
\end{itemize}
\begin{itemize}
\item {Utilização:Náut.}
\end{itemize}
\begin{itemize}
\item {Proveniência:(Do fr. \textunderscore tabouret\textunderscore )}
\end{itemize}
Cadeira de braços, sem costas.
Cadeira com assento de pau.
Peças de madeira, que fortificam as enoras.
\section{Tamboreto}
\begin{itemize}
\item {fónica:borê}
\end{itemize}
\begin{itemize}
\item {Grp. gram.:m.}
\end{itemize}
\begin{itemize}
\item {Utilização:Des.}
\end{itemize}
O mesmo que \textunderscore tamborete\textunderscore . Cf. \textunderscore Hist. Trág. Marit.\textunderscore , 10.
\section{Tamboril}
\begin{itemize}
\item {Grp. gram.:m.}
\end{itemize}
\begin{itemize}
\item {Utilização:Bras. do N}
\end{itemize}
Pequeno tambor.
O mesmo que \textunderscore enxarroco\textunderscore .
Grande árvore, de que se fazem gamelas, pitões, etc.
\section{Tamborilada}
\begin{itemize}
\item {Grp. gram.:f.}
\end{itemize}
Toque de tambor ou de tamboril.
\section{Tamborilar}
\begin{itemize}
\item {Grp. gram.:v. i.}
\end{itemize}
\begin{itemize}
\item {Proveniência:(De \textunderscore tamboril\textunderscore )}
\end{itemize}
Bater com os dedos numa superfície qualquer, imitando o rufar do tambor.
Produzir som análogo noutra coUsa:«\textunderscore a chuva que tamborila nas folhas do tinhorão...\textunderscore »M. Assis, \textunderscore B. Cubas\textunderscore .
\section{Tamborileiro}
\begin{itemize}
\item {Grp. gram.:m.  e  adj.}
\end{itemize}
Diz-se do que toca tambor ou tamboril.
\section{Tamborilete}
\begin{itemize}
\item {fónica:lê}
\end{itemize}
\begin{itemize}
\item {Grp. gram.:m.}
\end{itemize}
Pequeno tamboril.
\section{Tamborim}
\begin{itemize}
\item {Grp. gram.:m.}
\end{itemize}
\begin{itemize}
\item {Proveniência:(De \textunderscore tambor\textunderscore )}
\end{itemize}
O mesmo que \textunderscore tamboril\textunderscore .
Planta leguminosa, o mesmo que \textunderscore timburi\textunderscore .
\section{Tamborinar}
\begin{itemize}
\item {Grp. gram.:v. i.}
\end{itemize}
O mesmo que \textunderscore tamborilar\textunderscore .
\section{Tambueira}
\begin{itemize}
\item {Grp. gram.:f.}
\end{itemize}
\begin{itemize}
\item {Utilização:Bras. do N}
\end{itemize}
(Outra fórma de \textunderscore tamboeira\textunderscore . V. \textunderscore tamboeira\textunderscore )
\section{Tambuera}
\begin{itemize}
\item {Grp. gram.:f.}
\end{itemize}
\begin{itemize}
\item {Utilização:Bras}
\end{itemize}
(Outra fórma de \textunderscore tamboeira\textunderscore . V. \textunderscore tamboeira\textunderscore )
\section{Tambul}
\begin{itemize}
\item {Grp. gram.:m.}
\end{itemize}
Nome, que os Árabes dão ao bétel.
\section{Tambuladeira}
\begin{itemize}
\item {Grp. gram.:f.}
\end{itemize}
\begin{itemize}
\item {Utilização:Prov.}
\end{itemize}
Disco de prata, com a borda e o centro relevados, á semelhança do fundo de garrafa preta, e com que se avalia a grossura do vinho, conforme êlle cobre ou barra o disco. Cf. \textunderscore Techn. Rur.\textunderscore , 37.
Copo, ou utensílio, de prata ou loiça, para se vêr a côr do vinho, ou para se lhe apreciar o cheiro.
(Do cast.?)
\section{Tamearama}
\begin{itemize}
\item {Grp. gram.:f.}
\end{itemize}
Planta euphorbiácea e trepadeira, do Brasil, (\textunderscore dalechampia brasiliensis\textunderscore ).
\section{Tameira}
\begin{itemize}
\item {Grp. gram.:f.}
\end{itemize}
\begin{itemize}
\item {Proveniência:(De \textunderscore tamo\textunderscore ^1)}
\end{itemize}
O mesmo que \textunderscore tambeira\textunderscore .
\section{Tamém}
\begin{itemize}
\item {Grp. gram.:adv.}
\end{itemize}
\begin{itemize}
\item {Utilização:Pop.}
\end{itemize}
O mesmo que \textunderscore também\textunderscore .
\section{Tametara}
\begin{itemize}
\item {Grp. gram.:f.}
\end{itemize}
\begin{itemize}
\item {Utilização:Bras}
\end{itemize}
O mesmo que \textunderscore metara\textunderscore .
\section{Tâmia}
\begin{itemize}
\item {Grp. gram.:f.}
\end{itemize}
Gênero de mammíferos roedores.
\section{Tamiarana}
\begin{itemize}
\item {Grp. gram.:f.}
\end{itemize}
Planta euphorbiácea, (\textunderscore euphorbia urens\textunderscore ).
\section{Tamiça}
\begin{itemize}
\item {Grp. gram.:f.}
\end{itemize}
\begin{itemize}
\item {Utilização:Açor}
\end{itemize}
\begin{itemize}
\item {Proveniência:(Do fr. \textunderscore tamis\textunderscore )}
\end{itemize}
Cordel delgado de esparto.
Corda de feno, e de outras plantas, para segurar o bagaço, que, no meio do lagar, recebe a pressão.
\section{Tamiceira}
\begin{itemize}
\item {Grp. gram.:f.  e  adj.}
\end{itemize}
Fem. de \textunderscore tamiceiro\textunderscore .
\section{Tamiceiro}
\begin{itemize}
\item {Grp. gram.:m.  e  adj.}
\end{itemize}
Diz-se aquelle que fabríca e vende tamiça.
\section{Tâmil}
\begin{itemize}
\item {Grp. gram.:m.}
\end{itemize}
(V.tâmul)
\section{Tamina}
\begin{itemize}
\item {Grp. gram.:f.}
\end{itemize}
\begin{itemize}
\item {Utilização:Bras}
\end{itemize}
Vasilha, com que se mede a ração da farinha para os pretos.
Ração de farinha.
(Do quimbundo \textunderscore ritamina\textunderscore )
\section{Tamis}
\begin{itemize}
\item {Grp. gram.:m.}
\end{itemize}
\begin{itemize}
\item {Proveniência:(Do b. lat. \textunderscore tamisium\textunderscore )}
\end{itemize}
Espécie de peneira, para substâncias pulverizadas ou líquidos espessos.
Tecido de lan, inglês.
\section{Tamisação}
\begin{itemize}
\item {Grp. gram.:f.}
\end{itemize}
Acto ou effeito de tamisar.
\section{Tamisar}
\begin{itemize}
\item {Grp. gram.:v. t.}
\end{itemize}
\begin{itemize}
\item {Utilização:Ext.}
\end{itemize}
Passar pelo tamis.
Peneirar; depurar.
\section{Tamiuá}
\begin{itemize}
\item {Grp. gram.:m.}
\end{itemize}
\begin{itemize}
\item {Utilização:Bras}
\end{itemize}
Espécie de môsca das regiões do Amazonas.
\section{Tam-manho}
\begin{itemize}
\item {Grp. gram.:adj.}
\end{itemize}
\begin{itemize}
\item {Utilização:Ant.}
\end{itemize}
O mesmo que \textunderscore tamanho\textunderscore . Cf. João de Barros, \textunderscore Gramm.\textunderscore , 168.
\section{Tamo}
\begin{itemize}
\item {Grp. gram.:m.}
\end{itemize}
\begin{itemize}
\item {Utilização:Ant.}
\end{itemize}
O mesmo que \textunderscore tambo\textunderscore .
\section{Tamo}
\begin{itemize}
\item {Grp. gram.:m.}
\end{itemize}
Planta diurética e purgativa, (\textunderscore tamus communis\textunderscore , Lin.).
\section{Tamoão}
\begin{itemize}
\item {Grp. gram.:m.}
\end{itemize}
\begin{itemize}
\item {Utilização:T. do Fundão}
\end{itemize}
O mesmo que \textunderscore temão\textunderscore .
\section{Tamoeiro}
\begin{itemize}
\item {Grp. gram.:m.}
\end{itemize}
Peça de coiro, na parte superior do jugo dos bois, e na qual se prende a cabeçalha do carro.
Peça de coiro que, presa á canga, sustém o arado ou tirante; apeiro.
(Cp. \textunderscore temoeiro\textunderscore )
\section{Tamoéla}
\begin{itemize}
\item {Grp. gram.:f.}
\end{itemize}
\begin{itemize}
\item {Utilização:T. de Moncorvo}
\end{itemize}
Temão de grade de lavoira.
\section{Tamóios}
\begin{itemize}
\item {Grp. gram.:m. pl.}
\end{itemize}
Antiga e poderosa Confederação da América do Sul, que lutou tenazmente contra a dominação portuguesa no Brasil.--Também se escreve \textunderscore tamoyos\textunderscore , sem razão sólida.
\section{Tamom}
\begin{itemize}
\item {Grp. gram.:m.}
\end{itemize}
Árvore da Índia Portuguesa.
\section{Tamónea}
\begin{itemize}
\item {Grp. gram.:f.}
\end{itemize}
Gênero de plantas verbenáceas.
\section{Tampa}
\begin{itemize}
\item {Grp. gram.:f.}
\end{itemize}
Peça movediça, com que se tapa um vaso ou caixa, a que está, ou não, ligada por dobradiça.
Prensa de penteeiro.
(Por \textunderscore tapa\textunderscore ^1, de \textunderscore tapar\textunderscore )
\section{Tampadoiro}
\begin{itemize}
\item {Grp. gram.:m.}
\end{itemize}
\begin{itemize}
\item {Utilização:Prov.}
\end{itemize}
O mesmo que \textunderscore tampa\textunderscore .
\section{Tampadouro}
\begin{itemize}
\item {Grp. gram.:m.}
\end{itemize}
\begin{itemize}
\item {Utilização:Prov.}
\end{itemize}
O mesmo que \textunderscore tampa\textunderscore .
\section{Tampam}
\begin{itemize}
\item {Grp. gram.:m.}
\end{itemize}
Grande tampa.
Tampo.
Grande rolha ou buxa.
\section{Tampão}
\begin{itemize}
\item {Grp. gram.:m.}
\end{itemize}
Grande tampa.
Tampo.
Grande rolha ou buxa.
\section{Tampar}
\begin{itemize}
\item {Grp. gram.:v. t.}
\end{itemize}
Pôr tempo ou tampos em.
\section{Tâmpelo}
\begin{itemize}
\item {Grp. gram.:m.}
\end{itemize}
\begin{itemize}
\item {Utilização:Ant.}
\end{itemize}
Ordem militar, o mesmo que \textunderscore temple\textunderscore .
\section{Tampo}
\begin{itemize}
\item {Grp. gram.:m.}
\end{itemize}
\begin{itemize}
\item {Grp. gram.:Pl.}
\end{itemize}
\begin{itemize}
\item {Utilização:Pop.}
\end{itemize}
\begin{itemize}
\item {Proveniência:(De \textunderscore tampa\textunderscore )}
\end{itemize}
Cada uma das peças circulares, em que se entalham as aduelas das pipas, ou que constituem os topes de vasilhas ou objectos análogos.
Cada uma das peças, que constituem a parte anterior e posterior de certos instrumentos de corda.
Face horizontal dos degraus de madeira.
Cabeça; miolos.
\section{Tampo}
\begin{itemize}
\item {Grp. gram.:m.}
\end{itemize}
Pedaço de pelle, tirado de rês que se encontrou morta.
Aquillo que exhala cheiro repugnante.
Fétido.
\section{Tampor}
\begin{itemize}
\item {Grp. gram.:m.}
\end{itemize}
O vinho artificial de Bornéu.
\section{Tamposa}
\begin{itemize}
\item {Grp. gram.:f.}
\end{itemize}
\begin{itemize}
\item {Utilização:Gír.}
\end{itemize}
\begin{itemize}
\item {Proveniência:(De \textunderscore tampa\textunderscore )}
\end{itemize}
Caixa para rapé.
\section{Tam-só}
\begin{itemize}
\item {Grp. gram.:m.}
\end{itemize}
\begin{itemize}
\item {Utilização:Ant.}
\end{itemize}
A única coisa.
Pl. \textunderscore tamsoes\textunderscore , em G. Vicente, I, 221.
\section{Tam-tam}
\begin{itemize}
\item {fónica:tantan}
\end{itemize}
\begin{itemize}
\item {Grp. gram.:m.}
\end{itemize}
Instrumento de percussão, coberto de uma pelle em que se bate.
(É t. onom., corrente, mas mal escrito, porque, pronunciando-se \textunderscore tan-tan\textunderscore , melhor se escreveria \textunderscore tan-tan\textunderscore , ou \textunderscore tantan\textunderscore , ou \textunderscore tantã\textunderscore )
\section{Tamuanas}
\begin{itemize}
\item {Grp. gram.:m. pl.}
\end{itemize}
Tríbo de Índios da margem direita do Amazonas.
\section{Tamuatá}
\begin{itemize}
\item {Grp. gram.:m.}
\end{itemize}
\begin{itemize}
\item {Utilização:Bras. do N}
\end{itemize}
O mesmo que \textunderscore camboatá\textunderscore .
\section{Tamuge}
\begin{itemize}
\item {Grp. gram.:m.}
\end{itemize}
Espécie de sanguinheiro.
(Cast. \textunderscore tamujo\textunderscore )
\section{Tamugões}
\begin{itemize}
\item {Grp. gram.:m. pl.}
\end{itemize}
Segunda classe dos indígenas de Timor.
\section{Tamujo}
\begin{itemize}
\item {Grp. gram.:m.}
\end{itemize}
O mesmo ou melhór que \textunderscore tamuge\textunderscore .
\section{Tâmul}
\begin{itemize}
\item {Grp. gram.:m.}
\end{itemize}
Nome da mais culta das línguas dravídicas, o qual, ás vezes, se applica a toda a família dessas línguas, que são--o malabár, o tâmul, o telinga ou tálugo, o canará ou canarim, o malaiala ou malaialim, o tulo ou túluva, faladas ao sul da Índia.
\section{Tamungo}
\begin{itemize}
\item {Grp. gram.:m.}
\end{itemize}
Espécie de juiz que, em Malaca, e antes da conquista portuguesa, applicava a justiça aos estrangeiros. Cf. \textunderscore Comment. de Aff. de Albuq.\textunderscore 
\section{Tamurupará}
\begin{itemize}
\item {Grp. gram.:m.}
\end{itemize}
\begin{itemize}
\item {Utilização:Bras}
\end{itemize}
Ave das regiões do Amazonas.
\section{Tanaceto}
\begin{itemize}
\item {fónica:cê}
\end{itemize}
\begin{itemize}
\item {Grp. gram.:m.}
\end{itemize}
Planta, da fam. das synanthéreas.
\section{Tanadar}
\begin{itemize}
\item {Grp. gram.:m.}
\end{itemize}
Funccionário português que, na Índia, arrecadava as rendas das gancarias. Cf. Filinto, \textunderscore D. Man.\textunderscore , II, 258.
(Do indostano)
\section{Tanadaria}
\begin{itemize}
\item {Grp. gram.:f.}
\end{itemize}
Cargo de tanadar.
Território da jurisdicção do tanadar.
\section{Tanado}
\begin{itemize}
\item {Grp. gram.:adj.}
\end{itemize}
\begin{itemize}
\item {Proveniência:(Do fr. \textunderscore tan\textunderscore ?)}
\end{itemize}
Que tem côr de castanha; trigueiro.
\section{Tanagrídeas}
\begin{itemize}
\item {Grp. gram.:f. pl.}
\end{itemize}
Família de aves, da ordem dos passaros.
\section{Tanais}
\begin{itemize}
\item {Grp. gram.:m.}
\end{itemize}
\begin{itemize}
\item {Proveniência:(Lat. \textunderscore tanais\textunderscore )}
\end{itemize}
Gênero de crustáceos amphípodes.
\section{Tanajura}
\textunderscore f. Bras. de Minas\textunderscore , \textunderscore Espirito-Santo\textunderscore , etc.
O mesmo que \textunderscore saúba\textunderscore .
\section{Tanalbino}
\begin{itemize}
\item {Grp. gram.:m.}
\end{itemize}
Medicamento adstrigente, que é um albuminato de tanino.
\section{Tanaria}
\begin{itemize}
\item {Grp. gram.:f.}
\end{itemize}
\begin{itemize}
\item {Utilização:Ant.}
\end{itemize}
Fábrica de cortumes.
(Cp. \textunderscore tanino\textunderscore )
\section{Tanas}
\begin{itemize}
\item {Grp. gram.:m.}
\end{itemize}
\begin{itemize}
\item {Utilização:Fam.}
\end{itemize}
Qualquer sujeito, que se não quere, não se sabe ou não se deseja nomear: \textunderscore não foste tu que pregaste a pêta: havia de sêr o tanas\textunderscore .
Sujeito de pouco préstimo.
\section{Tanásia}
\begin{itemize}
\item {Grp. gram.:f.}
\end{itemize}
O mesmo que \textunderscore tanaceto\textunderscore .
\section{Tanasse}
\begin{itemize}
\item {Grp. gram.:f.}
\end{itemize}
Árvore intertropical, (\textunderscore dalbergia oojenensis\textunderscore ), cuja madeira, amarela como a do buxo, é muito apreciada na Índia.
(Do concani)
\section{Tanateiro}
\begin{itemize}
\item {Grp. gram.:adj.}
\end{itemize}
\begin{itemize}
\item {Utilização:Prov.}
\end{itemize}
\begin{itemize}
\item {Utilização:beir.}
\end{itemize}
Mentiroso; impostor.
\section{Tanato}
\begin{itemize}
\item {Grp. gram.:m.}
\end{itemize}
Sal, resultante da combinação do ácido tânico com uma base.
(Cp. \textunderscore tânico\textunderscore )
\section{Tanau}
\begin{itemize}
\item {Grp. gram.:m.}
\end{itemize}
Árvore da Índia Portuguesa.
\section{Tanauanas}
\begin{itemize}
\item {Grp. gram.:m. pl.}
\end{itemize}
Indígenas do norte do Brasil.
\section{Tanazinho}
\begin{itemize}
\item {Grp. gram.:adv.}
\end{itemize}
\begin{itemize}
\item {Utilização:Prov.}
\end{itemize}
\begin{itemize}
\item {Utilização:alg.}
\end{itemize}
\begin{itemize}
\item {Proveniência:(De \textunderscore tão\textunderscore  + \textunderscore azinha\textunderscore )}
\end{itemize}
Depressa.
\section{Tanca}
\begin{itemize}
\item {Grp. gram.:f.}
\end{itemize}
\begin{itemize}
\item {Utilização:T. da Áfr. Or. Port}
\end{itemize}
O mesmo que \textunderscore tanga\textunderscore ^1.
\section{Tancá}
\begin{itemize}
\item {Grp. gram.:m.}
\end{itemize}
Pequeno barco macaense, tripulado por mulheres.
\section{Tancar}
\begin{itemize}
\item {Grp. gram.:m.}
\end{itemize}
O mesmo que \textunderscore tancá\textunderscore .
\section{Tancar}
\begin{itemize}
\item {Grp. gram.:v. t.}
\end{itemize}
\begin{itemize}
\item {Utilização:Prov.}
\end{itemize}
\begin{itemize}
\item {Utilização:alg.}
\end{itemize}
\begin{itemize}
\item {Proveniência:(De \textunderscore tanque\textunderscore )}
\end{itemize}
O mesmo que \textunderscore estancar\textunderscore , não deixar correr.
\section{Tancareira}
\begin{itemize}
\item {Grp. gram.:f.}
\end{itemize}
\begin{itemize}
\item {Proveniência:(De \textunderscore tancar\textunderscore ^1)}
\end{itemize}
Mulhér que tripula o tancar.
\section{Tancha}
\begin{itemize}
\item {Grp. gram.:f.}
\end{itemize}
\begin{itemize}
\item {Proveniência:(De \textunderscore tanchar\textunderscore )}
\end{itemize}
Antigo utensílio de pesca.
\section{Tanchagem}
\begin{itemize}
\item {Grp. gram.:f.}
\end{itemize}
\begin{itemize}
\item {Proveniência:(Do lat. \textunderscore plantago\textunderscore , com metáth. das duas primeiras sýllabas)}
\end{itemize}
Planta vivaz e medicinal, da fam. das plantagíneas.
Planta alismácea.
\section{Tanchão}
\begin{itemize}
\item {Grp. gram.:m.}
\end{itemize}
\begin{itemize}
\item {Utilização:Prov.}
\end{itemize}
\begin{itemize}
\item {Utilização:alent.}
\end{itemize}
\begin{itemize}
\item {Utilização:Prov.}
\end{itemize}
\begin{itemize}
\item {Utilização:alent.}
\end{itemize}
\begin{itemize}
\item {Proveniência:(De \textunderscore tanchar\textunderscore )}
\end{itemize}
Braço ou estaca de árvore, que se planta, para reproducção.
Esteio de parreiras.
Espécie de alvião.
Estaca de azinho, aguçada na parte inferior, e com que se segura a rêde que veda o recinto onde dorme o gado, ao ar livre.
\section{Tanchar}
\begin{itemize}
\item {Grp. gram.:v. t.}
\end{itemize}
\begin{itemize}
\item {Utilização:Prov.}
\end{itemize}
\begin{itemize}
\item {Utilização:minh.}
\end{itemize}
\begin{itemize}
\item {Grp. gram.:V. i.}
\end{itemize}
\begin{itemize}
\item {Utilização:Pesc.}
\end{itemize}
O mesmo que \textunderscore plantar\textunderscore .
Firmar, á maneira de estaca; espetar.
Firmar no leito do rio (a vara), para impellir o barco.
Pescar, fundeando a rêde sardinheira.
(Metáth. de \textunderscore chantar\textunderscore )
\section{Tanchôa}
\begin{itemize}
\item {Grp. gram.:f.}
\end{itemize}
\begin{itemize}
\item {Utilização:Prov.}
\end{itemize}
\begin{itemize}
\item {Utilização:beir.}
\end{itemize}
Acto de tanchoar.
\section{Tanchoada}
\begin{itemize}
\item {Grp. gram.:f.}
\end{itemize}
\begin{itemize}
\item {Utilização:Prov.}
\end{itemize}
\begin{itemize}
\item {Utilização:trasm.}
\end{itemize}
Sebe de tanchões.
\section{Tanchoal}
\begin{itemize}
\item {Grp. gram.:m.}
\end{itemize}
(Metáth. de \textunderscore chantoal\textunderscore )
\section{Tanchoar}
\begin{itemize}
\item {Grp. gram.:v. t.}
\end{itemize}
\begin{itemize}
\item {Utilização:Prov.}
\end{itemize}
\begin{itemize}
\item {Utilização:beir.}
\end{itemize}
Empar na árvore ou no tanchão.
\section{Tanchoeira}
\begin{itemize}
\item {Grp. gram.:f.}
\end{itemize}
O mesmo que \textunderscore tanchão\textunderscore .
\section{Tandel}
\begin{itemize}
\item {Grp. gram.:m.}
\end{itemize}
\begin{itemize}
\item {Utilização:Náut.}
\end{itemize}
Diz o \textunderscore Vocab. Marujo\textunderscore , de Maur. Campos, que é t. asiático, correspondente a \textunderscore guardião de navio\textunderscore .
Para S. R. Dalgado, significa \textunderscore remeiro\textunderscore .
\section{Tândem}
\begin{itemize}
\item {Grp. gram.:m.}
\end{itemize}
\begin{itemize}
\item {Proveniência:(Lat. \textunderscore tandem\textunderscore , afinal)}
\end{itemize}
Espécie de cabriolé descoberto, de origem inglesa.
Velocípede de duas rodas, para duas pessôas.
\section{Taneco}
\begin{itemize}
\item {fónica:nê}
\end{itemize}
\begin{itemize}
\item {Grp. gram.:m.}
\end{itemize}
\begin{itemize}
\item {Utilização:Prov.}
\end{itemize}
O diabo.
\section{Tanga}
\begin{itemize}
\item {Grp. gram.:f.}
\end{itemize}
\begin{itemize}
\item {Utilização:Bras. do N}
\end{itemize}
Espécie de avental, com que os selvagens velam o corpo, desde o ventre ás coxas.
Espécie de franja, que se põe na rêde em que se descansa.
(Quimb. \textunderscore tanga\textunderscore )
\section{Tanga}
\begin{itemize}
\item {Grp. gram.:f.}
\end{itemize}
\begin{itemize}
\item {Proveniência:(T. ind.)}
\end{itemize}
Moéda asiática, do valor aproximado de 32 reis.
\section{Tangalho}
\begin{itemize}
\item {Grp. gram.:m.}
\end{itemize}
\begin{itemize}
\item {Utilização:Prov.}
\end{itemize}
O mesmo que \textunderscore tanganho\textunderscore .
\section{Tanganéu}
\begin{itemize}
\item {Grp. gram.:m.}
\end{itemize}
\begin{itemize}
\item {Utilização:Prov.}
\end{itemize}
\begin{itemize}
\item {Utilização:beir.}
\end{itemize}
Jôgo de rapazes, que consiste em atirar uma moéda de cobre, procurando deslocar outras que, a distância, são sustentadas por uma rôlha de cortiça ou por outra peça de madeira ou de osso.
\section{Tanganhão}
\begin{itemize}
\item {Grp. gram.:m.}
\end{itemize}
Negociante de escravos.
O que enfeita mercadorias, para terem melhor venda.
\section{Tanganhão}
\begin{itemize}
\item {Grp. gram.:m.}
\end{itemize}
\begin{itemize}
\item {Utilização:Pop.}
\end{itemize}
\begin{itemize}
\item {Proveniência:(De \textunderscore tanganho\textunderscore )}
\end{itemize}
Homem de grande estatura.
\section{Tanganheira}
\begin{itemize}
\item {Grp. gram.:f.  e  adj.}
\end{itemize}
\begin{itemize}
\item {Proveniência:(De \textunderscore tanga\textunderscore )}
\end{itemize}
Diz-se das Pretas, que têm os peitos muito pendentes e que por isso têm pouco valor.
\section{Tanganheiro}
\begin{itemize}
\item {Grp. gram.:adj.}
\end{itemize}
\begin{itemize}
\item {Utilização:Ant.}
\end{itemize}
\begin{itemize}
\item {Proveniência:(De \textunderscore tanganho\textunderscore ?)}
\end{itemize}
Preguiçoso.
Vadio.
\section{Tanganho}
\begin{itemize}
\item {Grp. gram.:m.}
\end{itemize}
\begin{itemize}
\item {Utilização:Prov.}
\end{itemize}
\begin{itemize}
\item {Utilização:Prov.}
\end{itemize}
\begin{itemize}
\item {Utilização:Prov.}
\end{itemize}
\begin{itemize}
\item {Utilização:beir.}
\end{itemize}
O mesmo que \textunderscore tranganho\textunderscore .
Ramo, que secou sôbre a árvore e que convém cortar.
Homem lorpa ou desajeitado. (Colhido em Mortágua)
\section{Tanganim}
\begin{itemize}
\item {Grp. gram.:m.}
\end{itemize}
Antiga medida de Cananor, correspondente a um litro e meio proximamente.
\section{Tangão}
\begin{itemize}
\item {Grp. gram.:m.}
\end{itemize}
Viga, com ferros atravessados e posta ao alto, á qual se prendem bastidores de theatro.
\section{Tangapema}
\begin{itemize}
\item {Grp. gram.:f.}
\end{itemize}
Arma dos Índios do centro do Brasil.
\section{Tangar}
\begin{itemize}
\item {Grp. gram.:v. t.}
\end{itemize}
Cobrir com tanga.
\section{Tângara}
\begin{itemize}
\item {Grp. gram.:m.}
\end{itemize}
Pássaro dentirostro do Brasil, (\textunderscore tanagra\textunderscore , Lin.)--Os diccion. dizem, incorrectamente, \textunderscore tangará\textunderscore .
(Cp. cast. \textunderscore tángara\textunderscore )
\section{Tangará-açu}
\begin{itemize}
\item {Grp. gram.:m.}
\end{itemize}
Arbusto plantagíneo do Brasil.
\section{Tangaracá}
\begin{itemize}
\item {Grp. gram.:m.}
\end{itemize}
Planta rubiácea do Brasil.
Planta medicinal brasileira, da fam. das synanthéreas.--Alguns, como L. da Fonseca, \textunderscore Flora Bras.\textunderscore , dizem \textunderscore tangaráca\textunderscore .
\section{Tangarra}
\begin{itemize}
\item {Grp. gram.:f.}
\end{itemize}
\begin{itemize}
\item {Utilização:Prov.}
\end{itemize}
\begin{itemize}
\item {Utilização:alg.}
\end{itemize}
Planta, de que se faz salada.
\section{Tange-asno}
\begin{itemize}
\item {Grp. gram.:m.}
\end{itemize}
O mesmo que \textunderscore tanjasno\textunderscore . Cf. Ed. Sequeira, \textunderscore Ovos e Ninhos\textunderscore .
\section{Tangedoiras}
\begin{itemize}
\item {Grp. gram.:f. Pl.}
\end{itemize}
\begin{itemize}
\item {Proveniência:(De \textunderscore tangedor\textunderscore )}
\end{itemize}
Prumos, que sustentam o folle das forjas de ferreiro.
\section{Tangedoiros}
\begin{itemize}
\item {Grp. gram.:m. Pl.}
\end{itemize}
O mesmo que \textunderscore tangedoiras\textunderscore .
\section{Tangedor}
\begin{itemize}
\item {Grp. gram.:m.  e  adj.}
\end{itemize}
\begin{itemize}
\item {Grp. gram.:Pl.}
\end{itemize}
O que tange ou toca qualquer instrumento.
O que toca as alimárias para as fazer andar.
Boicininga.
O mesmo que \textunderscore tangedoiras\textunderscore .
\section{Tangedouras}
\begin{itemize}
\item {Grp. gram.:f. Pl.}
\end{itemize}
\begin{itemize}
\item {Proveniência:(De \textunderscore tangedor\textunderscore )}
\end{itemize}
Prumos, que sustentam o folle das forjas de ferreiro.
\section{Tangedouros}
\begin{itemize}
\item {Grp. gram.:m. Pl.}
\end{itemize}
O mesmo que \textunderscore tangedouras\textunderscore .
\section{Tangefoles}
\begin{itemize}
\item {Grp. gram.:m.}
\end{itemize}
\begin{itemize}
\item {Utilização:Fig.}
\end{itemize}
\begin{itemize}
\item {Proveniência:(De \textunderscore tanger\textunderscore  + \textunderscore fole\textunderscore )}
\end{itemize}
O que toca os foles, nas forjas.
O que faz falar um tagarela.
\section{Tangefolles}
\begin{itemize}
\item {Grp. gram.:m.}
\end{itemize}
\begin{itemize}
\item {Utilização:Fig.}
\end{itemize}
\begin{itemize}
\item {Proveniência:(De \textunderscore tanger\textunderscore  + \textunderscore folle\textunderscore )}
\end{itemize}
O que toca os foles, nas forjas.
O que faz falar um tagarela.
\section{Tangência}
\begin{itemize}
\item {Grp. gram.:f.}
\end{itemize}
Qualidade do que é tangente.
Ponto, em que se tocam duas linhas ou duas superfícias.
\section{Tangencial}
\begin{itemize}
\item {Grp. gram.:adj.}
\end{itemize}
Relativo á tangência ou á tangente.
\section{Tangencialmente}
\begin{itemize}
\item {Grp. gram.:adv.}
\end{itemize}
De modo tangencial.
\section{Tangendo}
\begin{itemize}
\item {Grp. gram.:adj.}
\end{itemize}
\begin{itemize}
\item {Utilização:Neol.}
\end{itemize}
\begin{itemize}
\item {Proveniência:(Lat. \textunderscore tangendus\textunderscore )}
\end{itemize}
O mesmo que \textunderscore tangível\textunderscore .
\section{Tangente}
\begin{itemize}
\item {Grp. gram.:adj.}
\end{itemize}
\begin{itemize}
\item {Grp. gram.:F.}
\end{itemize}
\begin{itemize}
\item {Utilização:fig.}
\end{itemize}
\begin{itemize}
\item {Utilização:Fam.}
\end{itemize}
\begin{itemize}
\item {Proveniência:(Lat. \textunderscore tangens\textunderscore )}
\end{itemize}
Que tange.
Linha recta, que toca outra linha ou uma superfície num só ponto.
Expediente único.
Tábua de salvação; escapadela.
\section{Tanger}
\begin{itemize}
\item {Grp. gram.:v. t.}
\end{itemize}
\begin{itemize}
\item {Grp. gram.:V. i.}
\end{itemize}
\begin{itemize}
\item {Utilização:Fig.}
\end{itemize}
\begin{itemize}
\item {Grp. gram.:M.}
\end{itemize}
\begin{itemize}
\item {Proveniência:(Lat. \textunderscore tangere\textunderscore )}
\end{itemize}
Tocar (instrumentos músicos).
Tocar (alimárias), para as estimular na marcha.
Tocar (folle de ferreiro).
Soar.
Tocar qualquer instrumento.
Pertencer.
Referir-se.
Acto de tanger.
\section{Tangerina}
\begin{itemize}
\item {Grp. gram.:f.}
\end{itemize}
\begin{itemize}
\item {Proveniência:(De \textunderscore tangerino\textunderscore )}
\end{itemize}
Espécie de laranja pequena, muito aromática.
\section{Tangerineira}
\begin{itemize}
\item {Grp. gram.:f.}
\end{itemize}
Árvore, que produz tangerinas.
\section{Tangerino}
\begin{itemize}
\item {Grp. gram.:adj.}
\end{itemize}
\begin{itemize}
\item {Grp. gram.:M.}
\end{itemize}
Relativo a Tânger.
Habitante de Tânger.
\section{Tange-tange}
\begin{itemize}
\item {Grp. gram.:m.}
\end{itemize}
Arbusto leguminoso do Brasil.
\section{Tangibilidade}
\begin{itemize}
\item {Grp. gram.:f.}
\end{itemize}
Qualidade de tangível.
\section{Tangimento}
\begin{itemize}
\item {Grp. gram.:m.}
\end{itemize}
\begin{itemize}
\item {Utilização:Des.}
\end{itemize}
Acto de tanger ou tactear.
Acto de acariciar ou afagar.
\section{Tangido}
\begin{itemize}
\item {Grp. gram.:adj.}
\end{itemize}
\begin{itemize}
\item {Proveniência:(De \textunderscore tanger\textunderscore )}
\end{itemize}
Que se tangeu; que sôa por o tangerem: \textunderscore sinos tangidos\textunderscore .
\section{Tangir}
\begin{itemize}
\item {Grp. gram.:v. t.  e  i.}
\end{itemize}
\begin{itemize}
\item {Utilização:Ant.}
\end{itemize}
O mesmo que \textunderscore tanger\textunderscore .
\section{Tangível}
\begin{itemize}
\item {Grp. gram.:adj.}
\end{itemize}
\begin{itemize}
\item {Proveniência:(Lat. \textunderscore tangibilis\textunderscore )}
\end{itemize}
Que se pode tanger.
Que se póde tocar ou apalpar.
Palpável; sensível.
\section{Tangivelmente}
\begin{itemize}
\item {Grp. gram.:adv.}
\end{itemize}
De modo tangível.
\section{Tanglomanglo}
\begin{itemize}
\item {Grp. gram.:m.}
\end{itemize}
O mesmo que \textunderscore tangro-mangro\textunderscore .
\section{Tango}
\begin{itemize}
\item {Grp. gram.:m.}
\end{itemize}
Espécie de dança espanhola.
A música que lhe corresponde.
(Cast. \textunderscore tango\textunderscore )
\section{Tangomão}
\begin{itemize}
\item {Grp. gram.:m.}
\end{itemize}
\begin{itemize}
\item {Utilização:Ant.}
\end{itemize}
O mesmo que \textunderscore tanganhão\textunderscore ^1 ou \textunderscore pombeiro\textunderscore ^2.
Aquelle que morre ausente ou desterrado da pátria.
\section{Tângoro-mângoro}
\begin{itemize}
\item {Grp. gram.:m.}
\end{itemize}
\begin{itemize}
\item {Utilização:Bras}
\end{itemize}
O mesmo que \textunderscore tangro-mangro\textunderscore .
\section{Tangomau}
\begin{itemize}
\item {Grp. gram.:m.}
\end{itemize}
\begin{itemize}
\item {Utilização:Ant.}
\end{itemize}
O mesmo que \textunderscore tanganhão\textunderscore ^1 ou \textunderscore pombeiro\textunderscore ^2.
Aquelle que morre ausente ou desterrado da pátria.
\section{Tangos-maus}
\begin{itemize}
\item {Grp. gram.:m. pl.}
\end{itemize}
\begin{itemize}
\item {Utilização:Ant.}
\end{itemize}
Portugueses, que viviam entre os Negros, adoptando os costumes dêstes.
(Relaciona-se naturalmente com \textunderscore tangomau\textunderscore )
\section{Tangro-mangro}
\begin{itemize}
\item {Grp. gram.:m.}
\end{itemize}
\begin{itemize}
\item {Utilização:Pop.}
\end{itemize}
Malefício de bruxas.
Doença, attribuida a feitiçaria.
Doença teimosa.
(Gall. \textunderscore tangano-mangano\textunderscore )
\section{Tanguaraguaçu}
\begin{itemize}
\item {Grp. gram.:m.}
\end{itemize}
Árvore polygonáea da América.
\section{Tangueiro}
\begin{itemize}
\item {Grp. gram.:m.}
\end{itemize}
\begin{itemize}
\item {Grp. gram.:Adj.}
\end{itemize}
O mesmo que \textunderscore tanga\textunderscore ^1.
Relativo a tanga.
\section{Tangul}
\begin{itemize}
\item {Grp. gram.:m.}
\end{itemize}
\begin{itemize}
\item {Utilização:Ant.}
\end{itemize}
Cobre da Berberia.
(Do berbere \textunderscore tungult\textunderscore )
\section{Tanha}
\begin{itemize}
\item {Grp. gram.:f.}
\end{itemize}
\begin{itemize}
\item {Utilização:Prov.}
\end{itemize}
\begin{itemize}
\item {Utilização:trasm.}
\end{itemize}
Talha ou vasilha de barro, para azeite.
\section{Tanhada}
\begin{itemize}
\item {Grp. gram.:f.}
\end{itemize}
\begin{itemize}
\item {Utilização:Ant.}
\end{itemize}
\begin{itemize}
\item {Proveniência:(De \textunderscore tanho\textunderscore )}
\end{itemize}
Despejos de lixo. Cf. Soropita, \textunderscore Poes. e Pros.\textunderscore , 83.
\section{Tanho}
\begin{itemize}
\item {Grp. gram.:m.}
\end{itemize}
\begin{itemize}
\item {Utilização:Prov.}
\end{itemize}
\begin{itemize}
\item {Utilização:alg.}
\end{itemize}
\begin{itemize}
\item {Utilização:Ant.}
\end{itemize}
Grande seirão, próprio para conter cereaes.
Assento de tabúa; esteira. Cf. \textunderscore Anat. Joc.\textunderscore , I, 258.
\section{Tani}
\begin{itemize}
\item {Grp. gram.:m.}
\end{itemize}
Espécie de cipó, com que se enrolam no Brasil as fôlhas de tabaco, depois de sêcas.
\section{Tanibuca}
\begin{itemize}
\item {Grp. gram.:f.}
\end{itemize}
Planta medicinal e fructífera do Brasil.
\section{Taniça}
\begin{itemize}
\item {Grp. gram.:f.}
\end{itemize}
\begin{itemize}
\item {Utilização:Prov.}
\end{itemize}
\begin{itemize}
\item {Utilização:trasm.}
\end{itemize}
O mesmo que \textunderscore tani\textunderscore .
O mesmo que \textunderscore tamiça\textunderscore .
\section{Tânico}
\begin{itemize}
\item {Grp. gram.:adj.}
\end{itemize}
Diz-se de um ácido, extrahido da casca do carvalho.
(Cp. \textunderscore tanino\textunderscore )
\section{Tanigênio}
\begin{itemize}
\item {Grp. gram.:m.}
\end{itemize}
\begin{itemize}
\item {Proveniência:(De \textunderscore tanino\textunderscore  + gr. \textunderscore genos\textunderscore )}
\end{itemize}
Corpo composto e adstringente, derivado do tanino.
\section{Tanimbuca-tapuias}
\begin{itemize}
\item {Grp. gram.:m. pl.}
\end{itemize}
Índios selvagens das margens do Apaporis, no Brasil.
\section{Taninar}
\begin{itemize}
\item {Grp. gram.:v. t.}
\end{itemize}
Misturar com tanino.
\section{Tanino}
\begin{itemize}
\item {Grp. gram.:m.}
\end{itemize}
\begin{itemize}
\item {Proveniência:(Do fr. \textunderscore tanin\textunderscore )}
\end{itemize}
Substância adstringente, que se acha na casca do carvalho e noutros vegetaes, e é o mesmo que ácido tânico.
\section{Taninoso}
\begin{itemize}
\item {Grp. gram.:m. e adj.}
\end{itemize}
Que tem tanino.
\section{Tanjão}
\begin{itemize}
\item {Grp. gram.:m.  e  adj.}
\end{itemize}
\begin{itemize}
\item {Utilização:Pop.}
\end{itemize}
\begin{itemize}
\item {Proveniência:(De \textunderscore tanger\textunderscore )}
\end{itemize}
Indivíduo preguiçoso; e que só move, tocando-o.
\section{Tanjara}
\begin{itemize}
\item {Grp. gram.:f.}
\end{itemize}
\begin{itemize}
\item {Utilização:Prov.}
\end{itemize}
\begin{itemize}
\item {Utilização:minh.}
\end{itemize}
\begin{itemize}
\item {Proveniência:(De \textunderscore tanger\textunderscore ?)}
\end{itemize}
Pancadaria, sova, tunda.
\section{Tanjardo}
\begin{itemize}
\item {Grp. gram.:m.}
\end{itemize}
\begin{itemize}
\item {Utilização:Prov.}
\end{itemize}
O mesmo que \textunderscore tanjasno\textunderscore .
\section{Tanjarra}
\begin{itemize}
\item {Grp. gram.:f.}
\end{itemize}
Pássaro dentirostro, o mesmo que \textunderscore tanjasno\textunderscore .
\section{Tanjarro}
\begin{itemize}
\item {Grp. gram.:m.}
\end{itemize}
Pássaro dentirostro, o mesmo que \textunderscore tanjasno\textunderscore .
\section{Tanjasno}
\begin{itemize}
\item {Grp. gram.:m.}
\end{itemize}
\begin{itemize}
\item {Utilização:Prov.}
\end{itemize}
\begin{itemize}
\item {Proveniência:(De \textunderscore tanger\textunderscore  + \textunderscore asno\textunderscore . Cp. \textunderscore tange-asno\textunderscore  e \textunderscore tingeburro\textunderscore )}
\end{itemize}
Pássaro, pouco maior que o pardal, de bella plumagem, em que predomina o branco, o castanho e o amarelo, (\textunderscore saxicola albicollis\textunderscore , Vieill.).
O mesmo que \textunderscore chasco\textunderscore ^2, (\textunderscore pratincola rubetra\textunderscore ).
\section{Tanjudo}
\begin{itemize}
\item {Grp. gram.:adj.}
\end{itemize}
\begin{itemize}
\item {Utilização:Ant.}
\end{itemize}
\begin{itemize}
\item {Proveniência:(De \textunderscore tanger\textunderscore )}
\end{itemize}
O mesmo que \textunderscore tangido\textunderscore , (falando-se de campa ou sino).
\section{Tanjugo}
\begin{itemize}
\item {Grp. gram.:adj.}
\end{itemize}
\begin{itemize}
\item {Utilização:Ant.}
\end{itemize}
O mesmo que \textunderscore tanjudo\textunderscore .
\section{Tan-leom}
\begin{itemize}
\item {Grp. gram.:m.}
\end{itemize}
Árvore de Timor, espécie de sândalo.
\section{Tanoa}
\begin{itemize}
\item {Grp. gram.:f.}
\end{itemize}
\begin{itemize}
\item {Utilização:T. de Pernes}
\end{itemize}
\begin{itemize}
\item {Proveniência:(Do rad. \textunderscore tan\textunderscore  do b. bret. \textunderscore tanu\textunderscore , carvalho)}
\end{itemize}
Offício de tanoeiro.
Tanoaria.
Reparo ou fabríco de tonéis e pipas, ou de outras vasilhas de adega.
Sova, tareia.
\section{Tanoar}
\begin{itemize}
\item {Grp. gram.:v. i.}
\end{itemize}
\begin{itemize}
\item {Proveniência:(De \textunderscore tanôa\textunderscore )}
\end{itemize}
Exercer o offício de tanoeiro.
\section{Tanoaria}
\begin{itemize}
\item {Grp. gram.:f.}
\end{itemize}
\begin{itemize}
\item {Proveniência:(De \textunderscore tanôa\textunderscore )}
\end{itemize}
Fábrica ou estabelecimento de tanoeiro.
Arruamento de tanoeiros.
Obra de tanoeiro.
Profissão de tanoeiro.
\section{Tanoclarímetro}
\begin{itemize}
\item {Grp. gram.:m.}
\end{itemize}
\begin{itemize}
\item {Proveniência:(De \textunderscore tanino\textunderscore  + \textunderscore claro\textunderscore  + gr. \textunderscore metron\textunderscore )}
\end{itemize}
Instrumento, com que se verifica se os vinhos estão aptos para receber a collagem, e qual a dose, de colla e tanino, que se deve empregar para uma bôa clarificação.
\section{Tanóclero}
\begin{itemize}
\item {Grp. gram.:m.}
\end{itemize}
Gênero de insectos coleópteros pentâmeros.
\section{Tanoco}
\begin{itemize}
\item {fónica:nô}
\end{itemize}
\begin{itemize}
\item {Grp. gram.:m.}
\end{itemize}
\begin{itemize}
\item {Utilização:T. da Bairrada}
\end{itemize}
Pedaço de um tronco vegetal.
Tanganho; pau curto.
\section{Tanoeiro}
\begin{itemize}
\item {Grp. gram.:m.}
\end{itemize}
\begin{itemize}
\item {Proveniência:(De \textunderscore tanôa\textunderscore )}
\end{itemize}
Aquelle que fabrica pipas, dornas e objectos análogos.
\section{Tanofórmio}
\begin{itemize}
\item {Grp. gram.:m.}
\end{itemize}
\begin{itemize}
\item {Utilização:Pharm.}
\end{itemize}
Mistura de tanino e formalina, applicável no tratamento dos suores fétidos.
\section{Tanos}
\begin{itemize}
\item {Grp. gram.:m.}
\end{itemize}
\begin{itemize}
\item {Proveniência:(Lat. \textunderscore tanos\textunderscore )}
\end{itemize}
Pedra preciosa, que se suppõe conter o fluoreto de cálcio.
\section{Tanosal}
\begin{itemize}
\item {fónica:sal}
\end{itemize}
\begin{itemize}
\item {Grp. gram.:m.}
\end{itemize}
\begin{itemize}
\item {Utilização:Pharm.}
\end{itemize}
Tanato de creosota.
\section{Tanossal}
\begin{itemize}
\item {Grp. gram.:m.}
\end{itemize}
\begin{itemize}
\item {Utilização:Pharm.}
\end{itemize}
Tanato de creosota.
\section{Tanque}
\begin{itemize}
\item {Grp. gram.:m.}
\end{itemize}
\begin{itemize}
\item {Utilização:Náut.}
\end{itemize}
Reservatório, mais ou menos extenso, e feito de pedra ou de metal, para conter água ou outros líquidos.
Depósito das tinas de baldeação.
(Talvez do marata)
\section{Tanquia}
\begin{itemize}
\item {Grp. gram.:f.}
\end{itemize}
\begin{itemize}
\item {Utilização:Ant.}
\end{itemize}
Medicamento depilatório, composto de cal e oiro-pigmento.
(Cast. \textunderscore tanquia\textunderscore )
\section{Tanseira}
\begin{itemize}
\item {Grp. gram.:f.}
\end{itemize}
Parte do cano da bota, a que está ligada a presilha.
\section{Tanso}
\begin{itemize}
\item {Grp. gram.:m.  e  adj.}
\end{itemize}
\begin{itemize}
\item {Utilização:Pop.}
\end{itemize}
Pacóvio; palerma; sonso.
\section{Tantalato}
\begin{itemize}
\item {Grp. gram.:m.}
\end{itemize}
\begin{itemize}
\item {Proveniência:(De \textunderscore tântalo\textunderscore )}
\end{itemize}
Sal, produzido pela combinação do ácido tantálico com uma base.
\section{Tantálico}
\begin{itemize}
\item {Grp. gram.:adj.}
\end{itemize}
Relativo a tântalo.
\section{Tantálidos}
\begin{itemize}
\item {Grp. gram.:m. pl.}
\end{itemize}
Família de mineraes, a que pertence o tântalo.
\section{Tantalina}
\begin{itemize}
\item {Grp. gram.:f.}
\end{itemize}
\begin{itemize}
\item {Utilização:Miner.}
\end{itemize}
Deu-se êste nome a uma espécie de terra, em que se julgou haver analogia com o óxydo de tântalo e que depois se verificou sêr sílica.
\section{Tantálio}
\begin{itemize}
\item {Grp. gram.:m.}
\end{itemize}
O mesmo que \textunderscore tântalo\textunderscore .
\section{Tantalito}
\begin{itemize}
\item {Grp. gram.:m.}
\end{itemize}
\begin{itemize}
\item {Utilização:Miner.}
\end{itemize}
Substância, composta de ácido tantálico e de bases protoxydadas.
\section{Tantalizar}
\begin{itemize}
\item {Grp. gram.:v. t.}
\end{itemize}
\begin{itemize}
\item {Utilização:Fig.}
\end{itemize}
Causar o supplício de Tântalo a; produzir desejos irrealizáveis em. Cf. Camillo, \textunderscore Cancion. Al.\textunderscore , 430.
\section{Tântalo}
\begin{itemize}
\item {Grp. gram.:m.}
\end{itemize}
\begin{itemize}
\item {Proveniência:(De \textunderscore Tântalo\textunderscore , n. p.)}
\end{itemize}
Corpo metállico, simples, descoberto em 1801.
\section{Tantanguê}
\begin{itemize}
\item {Grp. gram.:m.}
\end{itemize}
\begin{itemize}
\item {Utilização:Bras}
\end{itemize}
Espécie de brinquedo de crianças.
\section{Tantaraga}
\begin{itemize}
\item {Grp. gram.:f.}
\end{itemize}
Gênero de plantas leguminosas, (\textunderscore cajanus indicus\textunderscore , Sphreng.) da Índia Portuguesa.
\section{Tantaréu}
\begin{itemize}
\item {Grp. gram.:m.}
\end{itemize}
\begin{itemize}
\item {Utilização:Mad}
\end{itemize}
Indivíduo, que exerce noutro a maior influência ou attracção.
(Por \textunderscore tentaréu\textunderscore , de \textunderscore tentar\textunderscore ?)
\section{Tantinho}
\begin{itemize}
\item {Grp. gram.:m.}
\end{itemize}
\begin{itemize}
\item {Utilização:Pop.}
\end{itemize}
\begin{itemize}
\item {Proveniência:(De \textunderscore tanto\textunderscore )}
\end{itemize}
Pequena porção; tudo-nada; bocadinho.
\section{Tantíssimo}
\begin{itemize}
\item {Grp. gram.:adj.}
\end{itemize}
\begin{itemize}
\item {Proveniência:(De \textunderscore tanto\textunderscore )}
\end{itemize}
Que é em grandíssimo número.
Numerosíssimo.
Que está no mais alto grau.
Muitíssimo:«\textunderscore ...em razão de tantíssimos estragos...\textunderscore »Filinto, \textunderscore D. Man.\textunderscore , I, 315.«\textunderscore ...com tantíssima alegria...\textunderscore »\textunderscore Idem\textunderscore , II, 90.
\section{Tantito}
\begin{itemize}
\item {Grp. gram.:adj.}
\end{itemize}
\begin{itemize}
\item {Grp. gram.:M.}
\end{itemize}
\begin{itemize}
\item {Proveniência:(De \textunderscore tanto\textunderscore )}
\end{itemize}
Que existe em pequena porção; pequenino.
Porção pequena.
\section{Tanto}
\begin{itemize}
\item {Grp. gram.:adj.}
\end{itemize}
\begin{itemize}
\item {Grp. gram.:M.}
\end{itemize}
\begin{itemize}
\item {Grp. gram.:Adv.}
\end{itemize}
\begin{itemize}
\item {Proveniência:(Lat. \textunderscore tantus\textunderscore )}
\end{itemize}
Tão numeroso: \textunderscore tanta gente\textunderscore .
Tão grande: \textunderscore tanta bondade\textunderscore .
Quantidade.
Porção indeterminada.
Extensão, volume.
Igual quantidade.
Dôbro.
Tal grau.
Tal número.
Em tão alto grau; em tal quantidade ou número.
Com tal fôrça; de tal modo: \textunderscore soffre tanto como eu\textunderscore .
\section{Tantra}
\begin{itemize}
\item {Grp. gram.:m.}
\end{itemize}
Tratado das fórmulas e ritos, que se observam no culto dos deuses índios.
\section{Tão}
\begin{itemize}
\item {Grp. gram.:adv.}
\end{itemize}
\begin{itemize}
\item {Proveniência:(Lat. \textunderscore tam\textunderscore )}
\end{itemize}
O mesmo que \textunderscore tanto\textunderscore .
\section{Táo}
\begin{itemize}
\item {Grp. gram.:m.}
\end{itemize}
Dialecto peruano.
\section{Taó}
\begin{itemize}
\item {Grp. gram.:m.}
\end{itemize}
Embarcação chinesa. Cf. Castilho, \textunderscore Fastos\textunderscore , II, 453.
\section{Tão-badalão}
\begin{itemize}
\item {Grp. gram.:m.}
\end{itemize}
O mesmo que \textunderscore tão-balalão\textunderscore .
\section{Tão-balalão}
\begin{itemize}
\item {Grp. gram.:m.}
\end{itemize}
\begin{itemize}
\item {Utilização:Fam.}
\end{itemize}
\begin{itemize}
\item {Proveniência:(T. onom.)}
\end{itemize}
Voz imitativa do som do sino.
\section{Taóca}
\begin{itemize}
\item {Grp. gram.:f.}
\end{itemize}
\begin{itemize}
\item {Utilização:Bras}
\end{itemize}
\begin{itemize}
\item {Utilização:Bras. do N}
\end{itemize}
Saboroso peixe marítimo.
Formiga vermelha.
\section{Taoísmo}
\begin{itemize}
\item {Grp. gram.:m.}
\end{itemize}
Seita religiosa e philosóphica da China, fundada no século VI antes de Christo.
\section{Taoronero}
\begin{itemize}
\item {Grp. gram.:m.}
\end{itemize}
Grande árvore da Guiana inglesa.
\section{Taos}
\begin{itemize}
\item {Grp. gram.:m.}
\end{itemize}
\begin{itemize}
\item {Proveniência:(Lat. \textunderscore taos\textunderscore )}
\end{itemize}
Variedade de pedra preciosa, hoje desconhecida.
\section{Tapa}
\begin{itemize}
\item {Grp. gram.:f.}
\end{itemize}
\begin{itemize}
\item {Utilização:Fam.}
\end{itemize}
\begin{itemize}
\item {Utilização:Pleb.}
\end{itemize}
\begin{itemize}
\item {Utilização:Bras}
\end{itemize}
\begin{itemize}
\item {Utilização:Prov.}
\end{itemize}
\begin{itemize}
\item {Utilização:trasm.}
\end{itemize}
\begin{itemize}
\item {Proveniência:(De \textunderscore tapar\textunderscore )}
\end{itemize}
Parte exterior e circular do casco da bêsta.
Rolha de madeira, com que se tapa a boca do canhão, para impedir a entrada de humidade.
Argumento irrespondível, ou que faz calar o adversário.
Bofetão.
Pedaço de pano, com que se vendam os olhos do burro pouco manso, para se deixar arrear.
Tapada, boiça.
\section{Tapa}
\begin{itemize}
\item {Grp. gram.:f.}
\end{itemize}
\begin{itemize}
\item {Utilização:Bras}
\end{itemize}
Peixe do gênero dos pleuronectos.
\section{Tàpabôca}
\begin{itemize}
\item {Grp. gram.:f.}
\end{itemize}
\begin{itemize}
\item {Utilização:Pleb.}
\end{itemize}
\begin{itemize}
\item {Proveniência:(De \textunderscore tapar\textunderscore  + \textunderscore bôca\textunderscore )}
\end{itemize}
Bofetada.
\section{Tapa-boquilha}
\begin{itemize}
\item {Grp. gram.:f.}
\end{itemize}
Pequena peça de madeira ou de metal, com que se cobre a boquilha do clarinete.
\section{Tapaciriba}
\begin{itemize}
\item {Grp. gram.:f.}
\end{itemize}
\begin{itemize}
\item {Utilização:Bras}
\end{itemize}
Gênero de plantas nyctagíneas.
\section{Tapacoás}
\begin{itemize}
\item {Grp. gram.:m. Pl.}
\end{itemize}
Tríbo de Índios do Brasil, ao norte de Goiás.
\section{Tapada}
\begin{itemize}
\item {Grp. gram.:f.}
\end{itemize}
\begin{itemize}
\item {Utilização:Ant.}
\end{itemize}
\begin{itemize}
\item {Proveniência:(De \textunderscore tapado\textunderscore )}
\end{itemize}
Cêrca.
Terreno murado; parque.
Meretriz.
\section{Tapadeiro}
\begin{itemize}
\item {Grp. gram.:m.}
\end{itemize}
\begin{itemize}
\item {Proveniência:(De \textunderscore tapar\textunderscore )}
\end{itemize}
O mesmo que \textunderscore tampa\textunderscore .
\section{Tapado}
\begin{itemize}
\item {Grp. gram.:adj.}
\end{itemize}
\begin{itemize}
\item {Utilização:Fig.}
\end{itemize}
\begin{itemize}
\item {Utilização:Escol.}
\end{itemize}
\begin{itemize}
\item {Grp. gram.:M.}
\end{itemize}
\begin{itemize}
\item {Utilização:Des.}
\end{itemize}
\begin{itemize}
\item {Proveniência:(De \textunderscore tapar\textunderscore )}
\end{itemize}
Que se tapou; vedado.
Estúpido; bronco.
Que faltou á aula todas as vezes que podia faltar, sem perder o anno, (falando-se de estudantes)
Cêrca, tapada.
\section{Tapadoira}
\begin{itemize}
\item {Grp. gram.:f.}
\end{itemize}
O mesmo que \textunderscore tapadoiro\textunderscore .
Postigo de tonel ou de pipa.
\section{Tapadoiro}
\begin{itemize}
\item {Grp. gram.:m.}
\end{itemize}
\begin{itemize}
\item {Proveniência:(De \textunderscore tapar\textunderscore )}
\end{itemize}
Tampa.
Parte do eixo, que sái para fóra da roda, nos coches.
\section{Tapador}
\begin{itemize}
\item {Grp. gram.:m.}
\end{itemize}
\begin{itemize}
\item {Proveniência:(De \textunderscore tapar\textunderscore )}
\end{itemize}
O mesmo que \textunderscore tampa\textunderscore .
\section{Tapadoura}
\begin{itemize}
\item {Grp. gram.:f.}
\end{itemize}
O mesmo que \textunderscore tapadouro\textunderscore .
Postigo de tonel ou de pipa.
\section{Tapadouro}
\begin{itemize}
\item {Grp. gram.:m.}
\end{itemize}
\begin{itemize}
\item {Proveniência:(De \textunderscore tapar\textunderscore )}
\end{itemize}
Tampa.
Parte do eixo, que sái para fóra da roda, nos coches.
\section{Tapadura}
\begin{itemize}
\item {Grp. gram.:f.}
\end{itemize}
\begin{itemize}
\item {Proveniência:(De \textunderscore tapar\textunderscore )}
\end{itemize}
Tapamento.
Tampa.
Tapume.
Cerrado.
\section{Tapa-embornaes}
\begin{itemize}
\item {Grp. gram.:m.}
\end{itemize}
Peça de coiro, que tapa os embornaes.
\section{Tapa-esteiro}
\begin{itemize}
\item {Grp. gram.:m.}
\end{itemize}
Diz-se armadilha ou rêde de \textunderscore tapa-esteiro\textunderscore  certa rêde grande de pesca; e diz-se peixe de \textunderscore tapa-esteiro\textunderscore  o peixe colhido nessa rêde. Cf. \textunderscore Decreto\textunderscore  de 14-I-909; e \textunderscore Port. Ant. e Mod.\textunderscore , II, 209.
\section{Tapagem}
\begin{itemize}
\item {Grp. gram.:f.}
\end{itemize}
\begin{itemize}
\item {Utilização:Prov.}
\end{itemize}
\begin{itemize}
\item {Utilização:minh.}
\end{itemize}
\begin{itemize}
\item {Utilização:Ext.}
\end{itemize}
\begin{itemize}
\item {Proveniência:(De \textunderscore tapar\textunderscore )}
\end{itemize}
Tapamento; sebe.
Tapume, feito com varas, no rio, para apanhar peixe.
Bosta, com que se tapa a porta do forno.
Excremento.
\section{Tapajiba}
\begin{itemize}
\item {Grp. gram.:f.}
\end{itemize}
O mesmo que \textunderscore tatajuba\textunderscore .
\section{Tapajós}
\begin{itemize}
\item {Grp. gram.:m. Pl.}
\end{itemize}
Numerosa tríbo de Índios, nas margens do rio do mesmo nome, no Brasil.
\section{Tafiano}
\begin{itemize}
\item {Grp. gram.:m.}
\end{itemize}
Gênero de mamíferos chirópteros.
\section{Táfio}
\begin{itemize}
\item {Grp. gram.:m.}
\end{itemize}
Gênero de mamíferos chirópteros.
\section{Táfios}
\begin{itemize}
\item {Grp. gram.:m. pl.}
\end{itemize}
\begin{itemize}
\item {Proveniência:(Lat. \textunderscore Taphii\textunderscore )}
\end{itemize}
Povos da Scýthia europeia.
Povos da Acarnânia.
\section{Tafofobia}
\begin{itemize}
\item {Grp. gram.:f.}
\end{itemize}
\begin{itemize}
\item {Proveniência:(Do gr. \textunderscore taphos\textunderscore  + \textunderscore phobos\textunderscore )}
\end{itemize}
Mêdo mórbido de sêr enterrado vivo.
\section{Tafria}
\begin{itemize}
\item {Grp. gram.:f.}
\end{itemize}
\begin{itemize}
\item {Proveniência:(Do gr. \textunderscore taphreia\textunderscore )}
\end{itemize}
Gênero de insectos coleópteros pentâmeros.
\section{Tafrócero}
\begin{itemize}
\item {Grp. gram.:m.}
\end{itemize}
\begin{itemize}
\item {Proveniência:(Do gr. \textunderscore taphros\textunderscore  + \textunderscore keros\textunderscore )}
\end{itemize}
Gênero de insectos coleópteros pentâmeros.
\section{Tafródero}
\begin{itemize}
\item {Grp. gram.:m.}
\end{itemize}
\begin{itemize}
\item {Proveniência:(Do gr. \textunderscore taphros\textunderscore  + \textunderscore dere\textunderscore )}
\end{itemize}
Gênero de insectos pentâmeros.
\section{Tapa-luz}
\begin{itemize}
\item {Grp. gram.:m.}
\end{itemize}
Termo, proposto por Garrett, em substituição do francesismo \textunderscore abatjour\textunderscore .
\section{Tapamento}
\begin{itemize}
\item {Grp. gram.:m.}
\end{itemize}
Acto ou effeito de tapar.
Tapume; cêrca.
\section{Tapa-misérias}
\begin{itemize}
\item {Grp. gram.:m.}
\end{itemize}
\begin{itemize}
\item {Utilização:Pop.}
\end{itemize}
Capa de gente pobre.
\section{Tapanhunas}
\begin{itemize}
\item {Grp. gram.:m. Pl.}
\end{itemize}
Tríbo de Índios do Brasil, em Mato-Grosso.
\section{Tapanhuno}
\begin{itemize}
\item {Grp. gram.:m.}
\end{itemize}
\begin{itemize}
\item {Utilização:Bras}
\end{itemize}
Árvore silvestre, empregada em construcções.
\section{Tapa-ôlho}
\begin{itemize}
\item {Grp. gram.:m.}
\end{itemize}
Grande árvore euphorbiácea de San-Thomé.
\section{Tapa-olhos}
\begin{itemize}
\item {Grp. gram.:m.}
\end{itemize}
\begin{itemize}
\item {Utilização:Pop.}
\end{itemize}
Bofetada.
\section{Tapar}
\begin{itemize}
\item {Grp. gram.:v. t.}
\end{itemize}
\begin{itemize}
\item {Grp. gram.:V. p.}
\end{itemize}
\begin{itemize}
\item {Utilização:T. de Turquel}
\end{itemize}
\begin{itemize}
\item {Proveniência:(Do b. al. \textunderscore tap\textunderscore )}
\end{itemize}
Pôr tampa em.
Cobrir: \textunderscore tapar o corpo com roupa\textunderscore .
Fechar; esconder.
Entupir: \textunderscore tapar uma cova\textunderscore .
Arrolhar: \textunderscore tapar uma garrafa\textunderscore .
Resguardar.
Cobrir-se.
Meter uma das mãos pela outra, (falando-se do cavallo).
Diz-se do toiro, quando levanta ou abaixa demasiadamente a cabeça, para evitar o ferro do toireiro.
Calar-se, por mêdo ou receio.
\section{Tapari}
\begin{itemize}
\item {Grp. gram.:m.}
\end{itemize}
Peixe do Brasil.
\section{Tapaxanas}
\begin{itemize}
\item {Grp. gram.:m. Pl.}
\end{itemize}
Indígenas do norte do Brasil.
\section{Tapeamento}
\begin{itemize}
\item {Grp. gram.:m.}
\end{itemize}
Acto de tapear.
\section{Tapear}
\begin{itemize}
\item {Grp. gram.:V. t.}
\end{itemize}
\begin{itemize}
\item {Utilização:Bras}
\end{itemize}
\begin{itemize}
\item {Proveniência:(Do rad. de \textunderscore tapar\textunderscore ?)}
\end{itemize}
Enganar.
\section{Tapeçar}
\begin{itemize}
\item {Grp. gram.:v. t.}
\end{itemize}
O mesmo que \textunderscore atapetar\textunderscore .
(Alter. de \textunderscore tapizar\textunderscore )
\section{Tapeçaria}
\begin{itemize}
\item {Grp. gram.:f.}
\end{itemize}
\begin{itemize}
\item {Utilização:Fig.}
\end{itemize}
\begin{itemize}
\item {Proveniência:(De \textunderscore tapeçar\textunderscore )}
\end{itemize}
Estôfo, geralmente lavrado, para forrar paredes, móveis, sobrados, etc.; alcatifa.
Conjunto de estôfos e alcatifas: \textunderscore no palácio de Monserrate é riquíssima a tapeçaria\textunderscore .
Terreno, coberto de verdura.
As flôres e a relva, que cobrem um terreno.
\section{Tapeceiro}
\begin{itemize}
\item {Grp. gram.:m.}
\end{itemize}
\begin{itemize}
\item {Proveniência:(De \textunderscore tapeçar\textunderscore )}
\end{itemize}
Fabricante ou vendedor de tapetes.
\section{Tapeiçava}
\begin{itemize}
\item {Grp. gram.:f.}
\end{itemize}
O mesmo que \textunderscore tupeiçava\textunderscore .
\section{Tapeina}
\begin{itemize}
\item {Grp. gram.:f.}
\end{itemize}
\begin{itemize}
\item {Proveniência:(Do gr. \textunderscore tapeinos\textunderscore )}
\end{itemize}
Gênero de insectos coleópteros.
\section{Tapeira}
\begin{itemize}
\item {Grp. gram.:f.}
\end{itemize}
\begin{itemize}
\item {Utilização:Ant.}
\end{itemize}
Tigela de doce.
\section{Tapejara}
\begin{itemize}
\item {Grp. gram.:m.}
\end{itemize}
\begin{itemize}
\item {Utilização:Bras. do S}
\end{itemize}
\begin{itemize}
\item {Proveniência:(T. tupi)}
\end{itemize}
O mesmo que \textunderscore vaqueano\textunderscore .
\section{Tapera}
\begin{itemize}
\item {Grp. gram.:f.}
\end{itemize}
\begin{itemize}
\item {Utilização:Bras}
\end{itemize}
Terreno de poisio.
Casa em ruínas, pardeeiro.--Os diccionaristas mandam lêr \textunderscore tapéra\textunderscore ; mas Vieira, falando a brasileiros, dizia \textunderscore tápera\textunderscore : cf. suas obras, XII, p. 219. O gram. bras. João Ribeiro manda lêr \textunderscore tapéra\textunderscore .
(Do tupi)
\section{Taperá}
\begin{itemize}
\item {Grp. gram.:m.}
\end{itemize}
\begin{itemize}
\item {Utilização:Bras}
\end{itemize}
Espécie de andorinha branca, de cabeça, cauda e asas negras.--B. C. Rubim, \textunderscore Vocab. Bras.\textunderscore , diz \textunderscore tapéra\textunderscore .
(Do tupi)
\section{Taperebá}
\begin{itemize}
\item {Grp. gram.:m.}
\end{itemize}
\begin{itemize}
\item {Utilização:Bras. do N}
\end{itemize}
Árvore, o mesmo que \textunderscore cajazeira\textunderscore .
\section{Taperiba}
\begin{itemize}
\item {Grp. gram.:f.}
\end{itemize}
\begin{itemize}
\item {Utilização:Bras}
\end{itemize}
Planta medicinal, provavelmente a mesma que \textunderscore taperebá\textunderscore , se é que uma das fórmas não é errada. Cf. L. da Fonseca, \textunderscore Flora Bras.\textunderscore 
\section{Taperoá}
\begin{itemize}
\item {Grp. gram.:m.}
\end{itemize}
\begin{itemize}
\item {Utilização:Bras}
\end{itemize}
Árvore silvestre.
\section{Taperu}
\begin{itemize}
\item {Grp. gram.:m.}
\end{itemize}
\begin{itemize}
\item {Utilização:Bras. do N}
\end{itemize}
\begin{itemize}
\item {Proveniência:(T. tupi)}
\end{itemize}
Larva de certos insectos.
\section{Tapes}
\begin{itemize}
\item {Grp. gram.:m. pl.}
\end{itemize}
Antiga nação de Índios do Brasil, hoje civilizada, e espalhada pelo Rio Grande-do-Sul.
\section{Tape-tape}
\begin{itemize}
\item {Grp. gram.:m.}
\end{itemize}
\begin{itemize}
\item {Utilização:Agr.}
\end{itemize}
\begin{itemize}
\item {Proveniência:(De \textunderscore tapar\textunderscore )}
\end{itemize}
Systema de distribuição de águas, segundo o qual a água pertence ao último que a desvia para o seu prédio, não podendo contudo o utente permanecer no lugar em que operou o desvio ou córte. Cf. Assis, \textunderscore Aguas\textunderscore , 193.
\section{Tapetar}
\begin{itemize}
\item {Grp. gram.:v. t.}
\end{itemize}
O mesmo que \textunderscore atapetar\textunderscore .
\section{Tapête}
\begin{itemize}
\item {Grp. gram.:m.}
\end{itemize}
\begin{itemize}
\item {Utilização:Fig.}
\end{itemize}
\begin{itemize}
\item {Proveniência:(Lat. \textunderscore tapete\textunderscore )}
\end{itemize}
Estôfo fixo, com que se revestem sobrados, escadas, etc.
Alcatifa.
Pano forte, para mesa.
Peça móvel de estôfo, com que se cobre uma só parte de um aposento, especialmente o espaço próximo das camas, dos sofás, etc.
Pequena peça de ornato, de pano ou de outra substância, que se sotopõe aos candeeiros portáteis, aos castiçaes, etc.
Relva; campo florido.
\section{Tapeteiro}
\begin{itemize}
\item {Grp. gram.:m.}
\end{itemize}
Fabricante de tapêtes.
\section{Tapetií}
\begin{itemize}
\item {Grp. gram.:m.}
\end{itemize}
\begin{itemize}
\item {Utilização:Bras}
\end{itemize}
\begin{itemize}
\item {Utilização:ant.}
\end{itemize}
Coelho; lebre.
\section{Taphiano}
\begin{itemize}
\item {Grp. gram.:m.}
\end{itemize}
Gênero de mammíferos chirópteros.
\section{Táphio}
\begin{itemize}
\item {Grp. gram.:m.}
\end{itemize}
Gênero de mammíferos chirópteros.
\section{Táphios}
\begin{itemize}
\item {Grp. gram.:m. pl.}
\end{itemize}
\begin{itemize}
\item {Proveniência:(Lat. \textunderscore Taphii\textunderscore )}
\end{itemize}
Povos da Scýthia europeia.
Povos da Acarnânia.
\section{Taphophobia}
\begin{itemize}
\item {Grp. gram.:f.}
\end{itemize}
\begin{itemize}
\item {Proveniência:(Do gr. \textunderscore taphos\textunderscore  + \textunderscore phobos\textunderscore )}
\end{itemize}
Mêdo mórbido de sêr enterrado vivo.
\section{Taphria}
\begin{itemize}
\item {Grp. gram.:f.}
\end{itemize}
\begin{itemize}
\item {Proveniência:(Do gr. \textunderscore taphreia\textunderscore )}
\end{itemize}
Gênero de insectos coleópteros pentâmeros.
\section{Taphrócero}
\begin{itemize}
\item {Grp. gram.:m.}
\end{itemize}
\begin{itemize}
\item {Proveniência:(Do gr. \textunderscore taphros\textunderscore  + \textunderscore keros\textunderscore )}
\end{itemize}
Gênero de insectos coleópteros pentâmeros.
\section{Taphródero}
\begin{itemize}
\item {Grp. gram.:m.}
\end{itemize}
\begin{itemize}
\item {Proveniência:(Do gr. \textunderscore taphros\textunderscore  + \textunderscore dere\textunderscore )}
\end{itemize}
Gênero de insectos pentâmeros.
\section{Tapiá}
\begin{itemize}
\item {Grp. gram.:m.}
\end{itemize}
Árvore silvestre do Brasil.
Planta urticácea brasileira.
\section{Tapiçar}
\begin{itemize}
\item {Grp. gram.:v. t.}
\end{itemize}
O mesmo que \textunderscore tapeçar\textunderscore . Cf. Filinto, IV, 285; IX, 182.
(Cp. \textunderscore tapizar\textunderscore )
\section{Tapicaris}
\begin{itemize}
\item {Grp. gram.:m. pl.}
\end{itemize}
Indígenas do norte do Brasil.
\section{Tapicuri}
\begin{itemize}
\item {Grp. gram.:m.}
\end{itemize}
\begin{itemize}
\item {Utilização:Bras}
\end{itemize}
Vinho de mandioca.
\section{Tapigo}
\begin{itemize}
\item {Grp. gram.:m.}
\end{itemize}
\begin{itemize}
\item {Proveniência:(De \textunderscore tapar\textunderscore )}
\end{itemize}
Tapume; barricada.
\section{Tapiíra}
\begin{itemize}
\item {Grp. gram.:f.}
\end{itemize}
\begin{itemize}
\item {Utilização:Bras}
\end{itemize}
Animal, o mesmo que \textunderscore anta\textunderscore .
(Cp. \textunderscore tapir\textunderscore )
\section{Tapina}
\begin{itemize}
\item {Grp. gram.:f.}
\end{itemize}
Gênero de plantas geraniáceas.
\section{Tapinambaba}
\begin{itemize}
\item {Grp. gram.:f.}
\end{itemize}
\begin{itemize}
\item {Utilização:Bras. do Ceará}
\end{itemize}
Massame de linhas com anzóes, nas jangadas de pesca.
\section{Tapinhoan}
\begin{itemize}
\item {Grp. gram.:m.}
\end{itemize}
\begin{itemize}
\item {Utilização:Bras}
\end{itemize}
Árvore silvestre, própria para construcções.
\section{Tapinócera}
\begin{itemize}
\item {Grp. gram.:f.}
\end{itemize}
Gênero de insectos dípteros.
\section{Tapioca}
\begin{itemize}
\item {Grp. gram.:f.}
\end{itemize}
\begin{itemize}
\item {Utilização:Bras}
\end{itemize}
Fécula da raíz da mandioca.
(Do tupi)
\section{Tapiocano}
\begin{itemize}
\item {Grp. gram.:m.}
\end{itemize}
\begin{itemize}
\item {Utilização:Bras. do Rio}
\end{itemize}
\begin{itemize}
\item {Proveniência:(De \textunderscore tapioca\textunderscore )}
\end{itemize}
O mesmo que \textunderscore caipira\textunderscore .
\section{Tapiocuí}
\begin{itemize}
\item {Grp. gram.:m.}
\end{itemize}
\begin{itemize}
\item {Utilização:Bras}
\end{itemize}
\begin{itemize}
\item {Proveniência:(T. tupi)}
\end{itemize}
Farinha de tapioca.
\section{Tapir}
\begin{itemize}
\item {Grp. gram.:m.}
\end{itemize}
\begin{itemize}
\item {Utilização:Zool.}
\end{itemize}
\begin{itemize}
\item {Proveniência:(T. tupi)}
\end{itemize}
Planta do Brasil.
\section{Tapirá-caiena}
\begin{itemize}
\item {Grp. gram.:m.}
\end{itemize}
\begin{itemize}
\item {Utilização:Bras}
\end{itemize}
O mesmo que \textunderscore canafístula\textunderscore .
\section{Tapirá-coana}
\begin{itemize}
\item {Grp. gram.:m.}
\end{itemize}
O mesmo que \textunderscore tapirá-caiena\textunderscore .
\section{Tapirá-coianna}
\begin{itemize}
\item {Grp. gram.:f.}
\end{itemize}
(V.tapirá-coana)
\section{Tapiranga}
\begin{itemize}
\item {Grp. gram.:f.}
\end{itemize}
Ave brasileira, também conhecida por \textunderscore sangue-de-boi\textunderscore .
\section{Tapirá-pecu}
\begin{itemize}
\item {Grp. gram.:m.}
\end{itemize}
Planta leguminosa do Brasil.
\section{Tapirapes}
\begin{itemize}
\item {Grp. gram.:m. pl.}
\end{itemize}
(V.tapiraques)
\section{Tapiraques}
\begin{itemize}
\item {Grp. gram.:m. pl.}
\end{itemize}
Tríbo de Índios do Brasil, em Mato-Grosso.
\section{Tapirete}
\begin{itemize}
\item {fónica:pirê}
\end{itemize}
\begin{itemize}
\item {Grp. gram.:m.}
\end{itemize}
\begin{itemize}
\item {Utilização:Bras}
\end{itemize}
Pequeno tapir.
\section{Tapiri}
\begin{itemize}
\item {Grp. gram.:m.}
\end{itemize}
\begin{itemize}
\item {Utilização:Bras. do Amazonas}
\end{itemize}
Palhoça provisória, em que se abrigam caminheiros, lavradores, etc.
\section{Tapiriba}
\begin{itemize}
\item {Grp. gram.:f.}
\end{itemize}
O mesmo que \textunderscore cajá\textunderscore .
\section{Tapíria}
\begin{itemize}
\item {Grp. gram.:f.}
\end{itemize}
Gênero de plantas burseráceas.
\section{Tapirotério}
\begin{itemize}
\item {Grp. gram.:m.}
\end{itemize}
Mammífero fóssil, parecido com o tapir.
\section{Tapirothério}
\begin{itemize}
\item {Grp. gram.:m.}
\end{itemize}
Mammífero fóssil, parecido com o tapir.
\section{Tàpisso}
\begin{itemize}
\item {Grp. gram.:m.}
\end{itemize}
\begin{itemize}
\item {Utilização:Prov.}
\end{itemize}
\begin{itemize}
\item {Utilização:minh.}
\end{itemize}
\begin{itemize}
\item {Utilização:Prov.}
\end{itemize}
\begin{itemize}
\item {Utilização:alg.}
\end{itemize}
\begin{itemize}
\item {Utilização:Pop.}
\end{itemize}
\begin{itemize}
\item {Utilização:Des.}
\end{itemize}
Tudo que serve para tapar.
O mesmo que \textunderscore avental\textunderscore .
Barretinho, com que se cobre a calva.
Espécie de toucado.
(Provavelmente da loc. \textunderscore tape isso\textunderscore )
\section{Tapiti}
\begin{itemize}
\item {Grp. gram.:m.}
\end{itemize}
\begin{itemize}
\item {Utilização:Bras}
\end{itemize}
O mesmo que \textunderscore tipiti\textunderscore .
\section{Tapiz}
\begin{itemize}
\item {Grp. gram.:m.}
\end{itemize}
\begin{itemize}
\item {Proveniência:(Do b. lat. \textunderscore tapicium\textunderscore )}
\end{itemize}
O mesmo que \textunderscore tapête\textunderscore .
\section{Tapizar}
\begin{itemize}
\item {Grp. gram.:v. t.}
\end{itemize}
\begin{itemize}
\item {Proveniência:(De \textunderscore tapiz\textunderscore )}
\end{itemize}
O mesmo que \textunderscore atapetar\textunderscore .
\section{Tapo}
\begin{itemize}
\item {Grp. gram.:m.}
\end{itemize}
\begin{itemize}
\item {Utilização:Prov.}
\end{itemize}
\begin{itemize}
\item {Utilização:minh.}
\end{itemize}
\begin{itemize}
\item {Proveniência:(De \textunderscore tapar\textunderscore )}
\end{itemize}
O mesmo que \textunderscore tampa\textunderscore .
\section{Tapona}
\begin{itemize}
\item {Grp. gram.:f.}
\end{itemize}
\begin{itemize}
\item {Utilização:Pleb.}
\end{itemize}
\begin{itemize}
\item {Proveniência:(Do b. al. \textunderscore tappe\textunderscore ?)}
\end{itemize}
Pontapé; sopapo; pancada.
\section{Tapór}
\begin{itemize}
\item {Grp. gram.:f.}
\end{itemize}
\begin{itemize}
\item {Utilização:Gír. de pedreiros.}
\end{itemize}
Porta.
(Metáth. de \textunderscore porta\textunderscore )
\section{Tapotopathia}
\begin{itemize}
\item {Grp. gram.:f.}
\end{itemize}
\begin{itemize}
\item {Utilização:Med.}
\end{itemize}
\begin{itemize}
\item {Proveniência:(Do fr. \textunderscore tapoter\textunderscore  + gr. \textunderscore pathos\textunderscore )}
\end{itemize}
Designação, hoje desusada, da maçagem.
\section{Tapotopatia}
\begin{itemize}
\item {Grp. gram.:f.}
\end{itemize}
\begin{itemize}
\item {Utilização:Med.}
\end{itemize}
\begin{itemize}
\item {Proveniência:(Do fr. \textunderscore tapoter\textunderscore  + gr. \textunderscore pathos\textunderscore )}
\end{itemize}
Designação, hoje desusada, da maçagem.
\section{Tapua}
\begin{itemize}
\item {Grp. gram.:m.}
\end{itemize}
\begin{itemize}
\item {Utilização:Bras}
\end{itemize}
Espécie de macaco.
\section{Tapui}
\begin{itemize}
\item {Grp. gram.:m.}
\end{itemize}
O mesmo que \textunderscore tapuio\textunderscore .
\section{Tapuia}
\begin{itemize}
\item {Grp. gram.:m.  e  f.}
\end{itemize}
\begin{itemize}
\item {Grp. gram.:M. Pl.}
\end{itemize}
Indivíduo indígena do Brasil, mas já sujeito aos brancos.
Antiga nação de Índios do Brasil, tronco de numerosas tríbos, espalhadas principalmente pelo Maranhão e Ceará.
(Do tupi)
\section{Tapuia}
\begin{itemize}
\item {Grp. gram.:f.}
\end{itemize}
\begin{itemize}
\item {Utilização:T. do Fundão}
\end{itemize}
Homem rústico, labrego.
\section{Tapuio}
\begin{itemize}
\item {Grp. gram.:m.}
\end{itemize}
(Cp. \textunderscore tapuia\textunderscore ^1)
\section{Tapuirana}
\begin{itemize}
\item {Grp. gram.:f.}
\end{itemize}
\begin{itemize}
\item {Utilização:Bras}
\end{itemize}
Nome de certo tecido, próprio para rêdes.
\section{Tapula}
\begin{itemize}
\item {Grp. gram.:adj. f.}
\end{itemize}
\begin{itemize}
\item {Proveniência:(Lat. \textunderscore tapulla\textunderscore )}
\end{itemize}
Dizia-se de uma lei romana á cêrca dos banquetes.
\section{Tapulho}
\begin{itemize}
\item {Grp. gram.:m.}
\end{itemize}
\begin{itemize}
\item {Proveniência:(Do cast. \textunderscore tapujo\textunderscore )}
\end{itemize}
Aquillo com que se tapa.
\section{Tapulho}
\begin{itemize}
\item {Grp. gram.:m.}
\end{itemize}
\begin{itemize}
\item {Utilização:Prov.}
\end{itemize}
O mesmo que \textunderscore trabulo\textunderscore .
\section{Tapulla}
\begin{itemize}
\item {Grp. gram.:adj. f.}
\end{itemize}
\begin{itemize}
\item {Proveniência:(Lat. \textunderscore tapulla\textunderscore )}
\end{itemize}
Dizia-se de uma lei romana á cêrca dos banquetes.
\section{Tapume}
\begin{itemize}
\item {Grp. gram.:m.}
\end{itemize}
\begin{itemize}
\item {Proveniência:(De \textunderscore tapar\textunderscore )}
\end{itemize}
Vedação de um terreno, feita com tábuas.
Sebe.
O mesmo que \textunderscore vallado\textunderscore .
\section{Tapurapo}
\begin{itemize}
\item {Grp. gram.:m.}
\end{itemize}
Planta purgativa da Guiana inglesa.
\section{Tapuru}
\begin{itemize}
\item {Grp. gram.:m.}
\end{itemize}
\begin{itemize}
\item {Utilização:Bras}
\end{itemize}
O mesmo que \textunderscore taperu\textunderscore .
\section{Tapuruísse}
\begin{itemize}
\item {Grp. gram.:m.}
\end{itemize}
Árvore do Brasil, cuja madeira se applica em marcenaria.
\section{Taputém}
\begin{itemize}
\item {Grp. gram.:m.}
\end{itemize}
\begin{itemize}
\item {Utilização:Náut.}
\end{itemize}
Válvula de sola nos embornaes dos tanques das pelles.
\section{Taquara}
\begin{itemize}
\item {Grp. gram.:f.}
\end{itemize}
\begin{itemize}
\item {Utilização:Bras}
\end{itemize}
Designação vulgar das várias espécies de bambu.
Nome de um pássaro esverdeado.
\section{Taquaral}
\begin{itemize}
\item {Grp. gram.:f.}
\end{itemize}
\begin{itemize}
\item {Utilização:Bras. do S}
\end{itemize}
Bosque de taquaras.
\section{Taquari}
\begin{itemize}
\item {Grp. gram.:m.}
\end{itemize}
\begin{itemize}
\item {Utilização:Bras. do N}
\end{itemize}
Espécie de taquara, árvore.
Canudo alongado de cachimbo.
\section{Taquaripana}
\begin{itemize}
\item {Grp. gram.:f.}
\end{itemize}
\begin{itemize}
\item {Utilização:Bras}
\end{itemize}
Formosa árvore dicotyledónea. Cf. \textunderscore Jorn.-do-Comm.\textunderscore , do Rio, de 12-II-901.
\section{Taquaruçu}
\begin{itemize}
\item {Grp. gram.:m.}
\end{itemize}
\begin{itemize}
\item {Utilização:Bras}
\end{itemize}
\begin{itemize}
\item {Proveniência:(De \textunderscore taquara\textunderscore )}
\end{itemize}
Árvore de Mato-Grosso.
\section{Taqueira}
\begin{itemize}
\item {Grp. gram.:f.}
\end{itemize}
\begin{itemize}
\item {Utilização:Bras}
\end{itemize}
Espécie de abóbora, pequena e chata.
\section{Taqueira}
\begin{itemize}
\item {Grp. gram.:f.}
\end{itemize}
\begin{itemize}
\item {Proveniência:(De \textunderscore taco\textunderscore ^1)}
\end{itemize}
Utensílio, onde se guardam os tacos, com que se joga o bilhar.
\section{Taqueometria}
\begin{itemize}
\item {fónica:que-o}
\end{itemize}
\begin{itemize}
\item {Grp. gram.:f.}
\end{itemize}
Conjunto de princípios e operações, que constituem o processo mais económico, mais rápido, e menos fatigante, para se obter, sem damno das propriedades, o relêvo de um terreno.
(Cp. \textunderscore taqueómetro\textunderscore )
\section{Taqueómetro}
\begin{itemize}
\item {Grp. gram.:m.}
\end{itemize}
Instrumento, com que se pratíca a taqueometria.
(Palavra mal formada, do gr. \textunderscore takhus\textunderscore  + \textunderscore metron\textunderscore . Cp. \textunderscore taquímetro\textunderscore )
\section{Taque-taque}
\begin{itemize}
\item {Grp. gram.:m.}
\end{itemize}
O mesmo que \textunderscore tique-taque\textunderscore .
\section{Táquia}
\begin{itemize}
\item {Grp. gram.:f.}
\end{itemize}
Gênero de plantas gencianáceas.
\section{Taquiádeno}
\begin{itemize}
\item {Grp. gram.:f.}
\end{itemize}
\begin{itemize}
\item {Proveniência:(Do gr. \textunderscore takhus\textunderscore  + \textunderscore aden\textunderscore )}
\end{itemize}
Gênero de plantas gencianáceas.
\section{Taquicardia}
\begin{itemize}
\item {Grp. gram.:f.}
\end{itemize}
\begin{itemize}
\item {Proveniência:(Do gr. \textunderscore takhus\textunderscore  + \textunderscore kardia\textunderscore )}
\end{itemize}
Rapidez de pulsações.
\section{Táquide}
\begin{itemize}
\item {Grp. gram.:f.}
\end{itemize}
Gênero de insectos coleópteros pentâmeros.
\section{Taquidrito}
\begin{itemize}
\item {Grp. gram.:m.}
\end{itemize}
\begin{itemize}
\item {Utilização:Miner.}
\end{itemize}
\begin{itemize}
\item {Proveniência:(Do gr. \textunderscore takhus\textunderscore  + \textunderscore hudor\textunderscore )}
\end{itemize}
Cloreto hidratado de cálcio e magnésio.
\section{Taquidromia}
\begin{itemize}
\item {Grp. gram.:f.}
\end{itemize}
\begin{itemize}
\item {Proveniência:(Do gr. \textunderscore takhus\textunderscore  + \textunderscore dromos\textunderscore )}
\end{itemize}
Gênero de insectos dípteros.
\section{Taquigalia}
\begin{itemize}
\item {Grp. gram.:f.}
\end{itemize}
\begin{itemize}
\item {Proveniência:(Do gr. \textunderscore takhus\textunderscore  + \textunderscore gala\textunderscore )}
\end{itemize}
Gênero de plantas leguminosas.
\section{Taquígono}
\begin{itemize}
\item {Grp. gram.:m.}
\end{itemize}
\begin{itemize}
\item {Proveniência:(Do gr. \textunderscore takhus\textunderscore  + \textunderscore gonos\textunderscore )}
\end{itemize}
Gênero de insectos coleópteros tetrâmeros.
\section{Taquigrafar}
\begin{itemize}
\item {Grp. gram.:v. t.}
\end{itemize}
\begin{itemize}
\item {Proveniência:(De \textunderscore taquígrafo\textunderscore )}
\end{itemize}
Escrever taquigraficamente.
\section{Taquigrafia}
\begin{itemize}
\item {Grp. gram.:f.}
\end{itemize}
\begin{itemize}
\item {Proveniência:(De \textunderscore taquígrafo\textunderscore )}
\end{itemize}
Sistema de escrita, por meio do qual se escreve quási tão depressa como se fala.
\section{Taquigraficamente}
\begin{itemize}
\item {Grp. gram.:adv.}
\end{itemize}
De modo taquigráfico; segundo os processos da taquigrafia.
\section{Taquigráfico}
\begin{itemize}
\item {Grp. gram.:adj.}
\end{itemize}
Relativo á taquigrafia.
\section{Taquígrafo}
\begin{itemize}
\item {Grp. gram.:m.}
\end{itemize}
\begin{itemize}
\item {Proveniência:(Do gr. \textunderscore takhus\textunderscore  + \textunderscore graphein\textunderscore )}
\end{itemize}
Aquele que escreve taquigraficamente.
Tratadista de taquigrafia.
\section{Taquilha}
\begin{itemize}
\item {Grp. gram.:f.}
\end{itemize}
Utensílio de madeira, onde se collocam e guardam os tacos de bilhar.
(Cast. \textunderscore taquillo\textunderscore )
\section{Taquílita}
\begin{itemize}
\item {Grp. gram.:f.}
\end{itemize}
\begin{itemize}
\item {Proveniência:(Do gr. \textunderscore takhus\textunderscore  + \textunderscore lithos\textunderscore )}
\end{itemize}
Silicato de alumina e bases protoxidadas, que se encontra no basalto.
\section{Taquílito}
\begin{itemize}
\item {Grp. gram.:m.}
\end{itemize}
O mesmo ou melhór que \textunderscore taquílita\textunderscore .
\section{Taquímetro}
\begin{itemize}
\item {Grp. gram.:m.}
\end{itemize}
\begin{itemize}
\item {Proveniência:(Do gr. \textunderscore takhus\textunderscore  + \textunderscore metron\textunderscore )}
\end{itemize}
Instrumento, com que se procura determinar ou avaliar a velocidade de uma máquina.--É preferível \textunderscore tacómetro\textunderscore .
V. \textunderscore tacómetro\textunderscore .
\section{Taquipneia}
\begin{itemize}
\item {Grp. gram.:f.}
\end{itemize}
\begin{itemize}
\item {Utilização:Med.}
\end{itemize}
\begin{itemize}
\item {Proveniência:(Do gr. \textunderscore takhus\textunderscore  + \textunderscore pnein\textunderscore )}
\end{itemize}
Grande aceleração do ritmo respiratório.
\section{Taquiplóteros}
\begin{itemize}
\item {Grp. gram.:m. pl.}
\end{itemize}
\begin{itemize}
\item {Utilização:Zool.}
\end{itemize}
\begin{itemize}
\item {Proveniência:(Do gr. \textunderscore takhus\textunderscore  + \textunderscore ploter\textunderscore )}
\end{itemize}
Uma das divisões da família das anátides.
\section{Táquipo}
\begin{itemize}
\item {Grp. gram.:m.}
\end{itemize}
\begin{itemize}
\item {Proveniência:(Do gr. \textunderscore takhus\textunderscore  + \textunderscore pous\textunderscore )}
\end{itemize}
Gênero de insectos coleópteros pentâmeros.
\section{Taquíporo}
\begin{itemize}
\item {Grp. gram.:m.}
\end{itemize}
\begin{itemize}
\item {Proveniência:(Do gr. \textunderscore takhus\textunderscore  + \textunderscore poros\textunderscore )}
\end{itemize}
Gênero de insectos coleópteros pentâmeros.
\section{Taquirá}
\begin{itemize}
\item {Grp. gram.:m.}
\end{itemize}
\begin{itemize}
\item {Utilização:Bras}
\end{itemize}
Gênero de amaryllídeas.
\section{Tara}
\begin{itemize}
\item {Grp. gram.:f.}
\end{itemize}
\begin{itemize}
\item {Utilização:Fig.}
\end{itemize}
\begin{itemize}
\item {Proveniência:(Do ár. \textunderscore tarha\textunderscore )}
\end{itemize}
Desconto ou abatimento, que se faz no pêso de uma mercadoria, attendendo-se ao carro, caixa ou vaso, em que ella vai metida ou é transportada.
Caixa, vaso ou carro, que contém ou póde conter um gênero ou mercadoria.
Falha, quebra.
Defeito, mácula.
O mesmo que \textunderscore taioba\textunderscore .
\section{Tarabelho}
\begin{itemize}
\item {fónica:bê}
\end{itemize}
\begin{itemize}
\item {Grp. gram.:m.}
\end{itemize}
O mesmo que \textunderscore trabelho\textunderscore .
\section{Tarabos}
\begin{itemize}
\item {Grp. gram.:m. pl.}
\end{itemize}
Antigo povo bravio da ilha de Geilolo, nas Molucas. Cf. Couto, \textunderscore Déc.\textunderscore  VI, l. IX, c. 10.
\section{Taracajá}
\begin{itemize}
\item {Grp. gram.:m.}
\end{itemize}
\begin{itemize}
\item {Utilização:Bras}
\end{itemize}
Espécie de tartaruga das regiões do Amazonas.
\section{Taracena}
\begin{itemize}
\item {Grp. gram.:f.}
\end{itemize}
O mesmo que \textunderscore tercena\textunderscore . Cf. Herculano, \textunderscore Lendas\textunderscore , 94.
\section{Tarado}
\begin{itemize}
\item {Grp. gram.:adj.}
\end{itemize}
\begin{itemize}
\item {Utilização:Fig.}
\end{itemize}
Que tem marcado o pêso da tara: \textunderscore vasilhas taradas\textunderscore .
Que tem falha ou defeito; desequilibrado, (em sentido moral).
\section{Taralhão}
\begin{itemize}
\item {Grp. gram.:m.}
\end{itemize}
\begin{itemize}
\item {Utilização:Pop.}
\end{itemize}
Pequeno pássaro dentirostro.
Homem metediço.
\section{Taralhão-mosqueiro}
\begin{itemize}
\item {Grp. gram.:m.}
\end{itemize}
Espécie de taralhão, o mesmo que \textunderscore papa-môscas\textunderscore .
\section{Taralhar}
\begin{itemize}
\item {Grp. gram.:v. i.}
\end{itemize}
\begin{itemize}
\item {Utilização:Bras}
\end{itemize}
\begin{itemize}
\item {Utilização:Neol.}
\end{itemize}
\begin{itemize}
\item {Proveniência:(De \textunderscore taralhão\textunderscore ?)}
\end{itemize}
O mesmo que \textunderscore pipilar\textunderscore .
\section{Taralheta}
\begin{itemize}
\item {fónica:lhê}
\end{itemize}
\begin{itemize}
\item {Grp. gram.:m.}
\end{itemize}
\begin{itemize}
\item {Utilização:Prov.}
\end{itemize}
\begin{itemize}
\item {Utilização:trasm.}
\end{itemize}
Homem tagarela ou que fala de tudo.
(Cp. \textunderscore taralhão\textunderscore )
\section{Taralhoeira}
\begin{itemize}
\item {Grp. gram.:f.}
\end{itemize}
\begin{itemize}
\item {Utilização:Prov.}
\end{itemize}
Armadilha de rêde, para apanhar taralhões.
\section{Taramá}
\begin{itemize}
\item {Grp. gram.:m.}
\end{itemize}
Planta verbenácea, medicinal, do Brasil.
\section{Tarambecos}
\begin{itemize}
\item {Grp. gram.:m. pl.}
\end{itemize}
\begin{itemize}
\item {Utilização:Prov.}
\end{itemize}
\begin{itemize}
\item {Utilização:trasm.}
\end{itemize}
Trastes caseiros.
Móveis diversos; tarecos.
\section{Tarambelho}
\begin{itemize}
\item {fónica:bê}
\end{itemize}
\begin{itemize}
\item {Grp. gram.:m.}
\end{itemize}
(V. \textunderscore trambelho\textunderscore ^1). Cf. Filinto, III, 152.
\section{Tarambola}
\begin{itemize}
\item {Grp. gram.:f.}
\end{itemize}
Gênero de aves pernaltas, (\textunderscore charadrius pluvialis\textunderscore , Lin.).
Morinelo.
\section{Tarambote}
\begin{itemize}
\item {Grp. gram.:m.}
\end{itemize}
\begin{itemize}
\item {Utilização:Pop.}
\end{itemize}
Concêrto vocal e instrumental.
Antiga canção popular:«\textunderscore os seus cantares eram motetes, taramboles e chácaras.\textunderscore »Camillo, \textunderscore Caveira\textunderscore , 221.
\section{Taramela}
\begin{itemize}
\item {Grp. gram.:f.}
\end{itemize}
\begin{itemize}
\item {Utilização:Fig.}
\end{itemize}
\begin{itemize}
\item {Grp. gram.:M.  e  f.}
\end{itemize}
Peça de madeira que, girando em volta de um prego, cravado no batente de uma porta ou cancella, a fecha.
Cravelho.
Peça de madeira que, batendo na mó do moínho quando esta gira, faz estremecer o quelho, donde vai caindo o grão para debaixo da mó.
Espécie de cunha, para segurar a retranca nos navios.
Língua.
Falatório.
Pessôa tagarela.
\section{Taramelar}
\begin{itemize}
\item {Grp. gram.:v. t.}
\end{itemize}
Dar á taramela; palrar; falar muito.
\section{Taramelear}
\begin{itemize}
\item {Grp. gram.:v. i.}
\end{itemize}
O mesmo que \textunderscore taramelar\textunderscore .
\section{Tarameleiro}
\begin{itemize}
\item {Grp. gram.:m.  e  adj.}
\end{itemize}
\begin{itemize}
\item {Proveniência:(De \textunderscore taramela\textunderscore )}
\end{itemize}
Indivíduo muito falador. Cf. Filinto, XII, 17.
\section{Taramelo}
\begin{itemize}
\item {Grp. gram.:m.}
\end{itemize}
\begin{itemize}
\item {Utilização:Prov.}
\end{itemize}
\begin{itemize}
\item {Utilização:minh.}
\end{itemize}
O mesmo que \textunderscore taramela\textunderscore .
\section{Taramelo}
\begin{itemize}
\item {Grp. gram.:m.}
\end{itemize}
\begin{itemize}
\item {Utilização:Prov.}
\end{itemize}
\begin{itemize}
\item {Utilização:minh.}
\end{itemize}
Embaraço ou impedimento no falar.
(Por \textunderscore entaramelo\textunderscore , de \textunderscore entaramelar\textunderscore )
\section{Taramembés}
\begin{itemize}
\item {Grp. gram.:m. pl.}
\end{itemize}
Tríbo de aborígenas do Maranhão.
\section{Tarampabo}
\begin{itemize}
\item {Grp. gram.:m.}
\end{itemize}
Espécie de palmeira.
\section{Tarampantão}
\begin{itemize}
\item {Grp. gram.:m.}
\end{itemize}
\begin{itemize}
\item {Proveniência:(T. onom.)}
\end{itemize}
Voz imitativa do som do tambor.
\section{Taranta}
\begin{itemize}
\item {Grp. gram.:m.  e  f.}
\end{itemize}
Pessôa aparvalhada, irresoluta.
\section{Tarantela}
\begin{itemize}
\item {Grp. gram.:f.}
\end{itemize}
\begin{itemize}
\item {Proveniência:(It. \textunderscore tarantella\textunderscore )}
\end{itemize}
Música e dança, de movimento rápido, e peculiar aos Napolitanos.
\section{Tarantella}
\begin{itemize}
\item {Grp. gram.:f.}
\end{itemize}
\begin{itemize}
\item {Proveniência:(It. \textunderscore tarantella\textunderscore )}
\end{itemize}
Música e dança, de movimento rápido, e peculiar aos Napolitanos.
\section{Tarantismo}
\begin{itemize}
\item {Grp. gram.:m.}
\end{itemize}
(V.tarentismo)
\section{Tarantona}
\begin{itemize}
\item {Grp. gram.:f.}
\end{itemize}
O mesmo que \textunderscore tarantella\textunderscore . Cp. \textunderscore Fénix Renasc.\textunderscore , IV, 36.
\section{Tarântula}
\begin{itemize}
\item {Grp. gram.:f.}
\end{itemize}
\begin{itemize}
\item {Proveniência:(Lat. \textunderscore tarantula\textunderscore )}
\end{itemize}
Espécie de aranha, cuja mordedura é venenosa.
Medicamento, preparado com o suco daquelle animal.
\section{Tarar}
\begin{itemize}
\item {Grp. gram.:v. t.}
\end{itemize}
Pesar, para abater a tara.
Marcar o pêso da tara sôbre (vasilhas, sacos, etc).
\section{Tarara}
\begin{itemize}
\item {Grp. gram.:f.}
\end{itemize}
\begin{itemize}
\item {Proveniência:(Fr. \textunderscore tarare\textunderscore )}
\end{itemize}
Apparêlho, para limpar o grão do trigo, agitando-o e ventilando-o.
\section{Tarará}
\begin{itemize}
\item {Grp. gram.:m.}
\end{itemize}
Voz onomatopaica, imitativa do som de trombeta.
\section{Tarasca}
\begin{itemize}
\item {Grp. gram.:f.}
\end{itemize}
\begin{itemize}
\item {Utilização:Pop.}
\end{itemize}
\begin{itemize}
\item {Grp. gram.:Adj.}
\end{itemize}
Mulhér feia, mal comportada ou de mau gênio.
Chanfalho.
Diz-se da mulhér feia ou mal comportada. Cf. Castilho, \textunderscore Misantropo\textunderscore , 13.
\section{Tarasco}
\begin{itemize}
\item {Grp. gram.:adj.}
\end{itemize}
\begin{itemize}
\item {Grp. gram.:M.}
\end{itemize}
\begin{itemize}
\item {Utilização:Prov.}
\end{itemize}
\begin{itemize}
\item {Utilização:alg.}
\end{itemize}
\begin{itemize}
\item {Proveniência:(De \textunderscore tarasca\textunderscore )}
\end{itemize}
Desabrido; áspero; esquivo.
Vento cortante.
\section{Tarasquento}
\begin{itemize}
\item {Grp. gram.:adj.}
\end{itemize}
\begin{itemize}
\item {Utilização:Prov.}
\end{itemize}
\begin{itemize}
\item {Utilização:alg.}
\end{itemize}
\begin{itemize}
\item {Proveniência:(De \textunderscore tarasco\textunderscore )}
\end{itemize}
Ventoso.
\section{Tarau}
\begin{itemize}
\item {Grp. gram.:m.}
\end{itemize}
\begin{itemize}
\item {Utilização:Prov.}
\end{itemize}
\begin{itemize}
\item {Utilização:beir.}
\end{itemize}
Rapaz ou rapariga foliona ou leviana.
\section{Taráxaco}
\begin{itemize}
\item {Grp. gram.:m.}
\end{itemize}
\begin{itemize}
\item {Proveniência:(Do lat. \textunderscore taraxacon\textunderscore )}
\end{itemize}
Planta, o mesmo que \textunderscore dente-de-leão\textunderscore .
\section{Tarbuche}
\begin{itemize}
\item {Grp. gram.:m.}
\end{itemize}
O barrete usual dos Turcos. Cf. Benalcanfor, \textunderscore Cartas de Viagem\textunderscore , c. XVI e XVIII.
\section{Tardada}
\begin{itemize}
\item {Grp. gram.:f.}
\end{itemize}
Acto de tardar; demora; delonga.
\section{Tardador}
\begin{itemize}
\item {Grp. gram.:m.  e  adj.}
\end{itemize}
O que tarda; aquelle que é vagaroso.
\section{Tardamente}
\begin{itemize}
\item {Grp. gram.:adj.}
\end{itemize}
\begin{itemize}
\item {Proveniência:(De \textunderscore tardo\textunderscore )}
\end{itemize}
O mesmo que \textunderscore tardiamente\textunderscore .
\section{Tardamento}
\begin{itemize}
\item {Grp. gram.:m.}
\end{itemize}
Acto ou effeito de tardar.
Demora.
\section{Tardança}
\begin{itemize}
\item {Grp. gram.:f.}
\end{itemize}
Acto ou effeito de tardar.
Demora.
\section{Tardão}
\begin{itemize}
\item {Grp. gram.:m.}
\end{itemize}
(V.tardador)
\section{Tardar}
\begin{itemize}
\item {Grp. gram.:v. t.}
\end{itemize}
\begin{itemize}
\item {Grp. gram.:V. i.}
\end{itemize}
\begin{itemize}
\item {Proveniência:(Lat. \textunderscore tardare\textunderscore )}
\end{itemize}
Demorar; adiar; procrastinar.
Demorar-se.
Proceder lentamente.
Não têr pressa; chegar tarde.
Estacionar.
\section{Tardar}
\begin{itemize}
\item {Grp. gram.:m.}
\end{itemize}
\begin{itemize}
\item {Utilização:Ant.}
\end{itemize}
\begin{itemize}
\item {Utilização:Gír.}
\end{itemize}
Vestido de mulhér.
\section{Tarde}
\begin{itemize}
\item {Grp. gram.:adv.}
\end{itemize}
\begin{itemize}
\item {Grp. gram.:F.}
\end{itemize}
\begin{itemize}
\item {Grp. gram.:M.}
\end{itemize}
\begin{itemize}
\item {Proveniência:(Lat. \textunderscore tarde\textunderscore )}
\end{itemize}
Depois de passar o tempo próprio, conveniente ou ajustado: \textunderscore chegou tarde\textunderscore .
A horas adeantadas, perto da noite.
Espaço de tempo entre o meio-dia e o anoitecer.

Us. na loc. adv. \textunderscore no tarde\textunderscore , serodiamente, fóra do tempo próprio.
\section{Tardeco}
\begin{itemize}
\item {Grp. gram.:adj.}
\end{itemize}
\begin{itemize}
\item {Utilização:Prov.}
\end{itemize}
\begin{itemize}
\item {Utilização:trasm.}
\end{itemize}
\begin{itemize}
\item {Proveniência:(De \textunderscore tarde\textunderscore )}
\end{itemize}
Serôdio.
Tardio.
Retardatário.
\section{Tardego}
\begin{itemize}
\item {Grp. gram.:adj.}
\end{itemize}
\begin{itemize}
\item {Utilização:Prov.}
\end{itemize}
\begin{itemize}
\item {Utilização:trasm.}
\end{itemize}
\begin{itemize}
\item {Proveniência:(De \textunderscore tarde\textunderscore )}
\end{itemize}
Serôdio.
Tardio.
Retardatário.
\section{Tardeza}
\begin{itemize}
\item {Grp. gram.:f.}
\end{itemize}
Qualidade do que é tardo.
\section{Tardiamente}
\begin{itemize}
\item {Grp. gram.:adv.}
\end{itemize}
De modo tardio; serodiamente; tarde; com demora.
\section{Tardião}
\begin{itemize}
\item {Grp. gram.:adj.}
\end{itemize}
\begin{itemize}
\item {Grp. gram.:M.}
\end{itemize}
\begin{itemize}
\item {Proveniência:(De \textunderscore tardio\textunderscore )}
\end{itemize}
O mesmo que \textunderscore tardio\textunderscore .
Dança antiga.
\section{Tardígrado}
\begin{itemize}
\item {Grp. gram.:adj.}
\end{itemize}
\begin{itemize}
\item {Utilização:Poét.}
\end{itemize}
\begin{itemize}
\item {Grp. gram.:M. Pl.}
\end{itemize}
\begin{itemize}
\item {Proveniência:(Lat. \textunderscore tardigradus\textunderscore )}
\end{itemize}
Que anda vagarosamente.
Família de mammíferos, a que pertence a preguiça; macrobiotas.
\section{Tardíloquo}
\begin{itemize}
\item {Grp. gram.:adj.}
\end{itemize}
\begin{itemize}
\item {Proveniência:(Lat. \textunderscore tardiloquus\textunderscore )}
\end{itemize}
Que é gago ou tem a fala demorada.
\section{Tardinha}
\begin{itemize}
\item {Grp. gram.:f.}
\end{itemize}
\begin{itemize}
\item {Utilização:Pop.}
\end{itemize}
O fim da tarde.
\section{Tardinheiramente}
\begin{itemize}
\item {Grp. gram.:adv.}
\end{itemize}
\begin{itemize}
\item {Proveniência:(De \textunderscore tardinheiro\textunderscore )}
\end{itemize}
O mesmo que \textunderscore tardiamente\textunderscore .
\section{Tardinheiro}
\begin{itemize}
\item {Grp. gram.:m.  e  adj.}
\end{itemize}
\begin{itemize}
\item {Proveniência:(De \textunderscore tardinha\textunderscore )}
\end{itemize}
O que é preguiçoso ou vagaroso, por hábito.
\section{Tardio}
\begin{itemize}
\item {Grp. gram.:adj.}
\end{itemize}
\begin{itemize}
\item {Proveniência:(Do lat. \textunderscore tardivus\textunderscore )}
\end{itemize}
O mesmo que \textunderscore tardo\textunderscore ^1.
Que chega tarde, ou que vem fóra do tempo próprio; serôdio.
\section{Tardo}
\begin{itemize}
\item {Grp. gram.:adj.}
\end{itemize}
\begin{itemize}
\item {Proveniência:(Lat. \textunderscore tardus\textunderscore )}
\end{itemize}
Que anda lentamente.
Preguiçoso; vagaroso.
Que tem pouca ou nenhuma actividade.
Que entende difficilmente.
Serôdio.
\section{Tardo}
\begin{itemize}
\item {Grp. gram.:m.}
\end{itemize}
\begin{itemize}
\item {Utilização:Prov.}
\end{itemize}
\begin{itemize}
\item {Utilização:Prov.}
\end{itemize}
\begin{itemize}
\item {Utilização:minh.}
\end{itemize}
O mesmo que \textunderscore trasgo\textunderscore .
O mesmo que \textunderscore pesadelo\textunderscore .
\section{Tardonho}
\begin{itemize}
\item {Grp. gram.:adj.}
\end{itemize}
O mesmo que \textunderscore tardinheiro\textunderscore  ou \textunderscore tardo\textunderscore ^1.
\section{Tardós}
\begin{itemize}
\item {Grp. gram.:f.}
\end{itemize}
\begin{itemize}
\item {Utilização:Gír.}
\end{itemize}
Lado tôsco de uma pedra de cantaria, que fica para dentro da parede.
Nádegas.
\section{Tarear}
\begin{itemize}
\item {Grp. gram.:v. t.}
\end{itemize}
O mesmo que \textunderscore tarar\textunderscore .
\section{Tarear}
\begin{itemize}
\item {Grp. gram.:v. t.}
\end{itemize}
Dar tareia em; sovar.
\section{Tarecada}
\begin{itemize}
\item {Grp. gram.:f.}
\end{itemize}
\begin{itemize}
\item {Utilização:Prov.}
\end{itemize}
\begin{itemize}
\item {Utilização:trasm.}
\end{itemize}
\begin{itemize}
\item {Proveniência:(De \textunderscore tareco\textunderscore )}
\end{itemize}
Barulho de tareco; traquinada.
Grande porção de tarecos.
Acção ou dito de mulhér tareca.
\section{Tarecena}
\begin{itemize}
\item {Grp. gram.:f.}
\end{itemize}
\begin{itemize}
\item {Utilização:Ant.}
\end{itemize}
\begin{itemize}
\item {Proveniência:(Do ár. \textunderscore dar-senaha\textunderscore , fábrica)}
\end{itemize}
Estaleiro; arsenal.
O mesmo que \textunderscore tercena\textunderscore .
\section{Tareco}
\begin{itemize}
\item {Grp. gram.:m.  e  adj.}
\end{itemize}
\begin{itemize}
\item {Utilização:Prov.}
\end{itemize}
\begin{itemize}
\item {Utilização:trasm.}
\end{itemize}
\begin{itemize}
\item {Grp. gram.:M. Pl.}
\end{itemize}
\begin{itemize}
\item {Utilização:Prov.}
\end{itemize}
\begin{itemize}
\item {Utilização:minh.}
\end{itemize}
\begin{itemize}
\item {Proveniência:(Do ár. \textunderscore taric\textunderscore )}
\end{itemize}
Indivíduo inquieto; traquinas.
O mesmo que \textunderscore gato\textunderscore . (Colhido em Alijó)
Utensílios ou mobília usada ou meio partida e de pouco valor.
Testículos.
\section{Tarefa}
\begin{itemize}
\item {Grp. gram.:f.}
\end{itemize}
\begin{itemize}
\item {Utilização:Fig.}
\end{itemize}
\begin{itemize}
\item {Utilização:Bras. da Baía}
\end{itemize}
\begin{itemize}
\item {Utilização:Prov.}
\end{itemize}
\begin{itemize}
\item {Utilização:beir.}
\end{itemize}
\begin{itemize}
\item {Proveniência:(Do ár. \textunderscore tareha\textunderscore )}
\end{itemize}
Qualidade ou porção de trabalho, que se deve realizar em determinado tempo.
Empreitada.
Em pyroctechnia, é um misto de carvão e enxôfre ou de carvão e salitre ou das três coisas, o qual se tritura por uma vez.
Trabalho, cuja realização se delibera ou se impõe; encargo.
Medida agrária, igual a 4:356 metros quadrados.
Vaso de barro, usado em lagares de azeite.
\section{Tarefeiro}
\begin{itemize}
\item {Grp. gram.:m.}
\end{itemize}
Aquelle que se encarrega de uma tarefa ou empreitada; empreiteiro.--Nas obras do Estado, o tarefeiro não fornece os materiaes como o empreiteiro em geral.
\section{Taregá}
\begin{itemize}
\item {Grp. gram.:m.}
\end{itemize}
Ferro velho, adelo de tarecos.
(Do télugo \textunderscore taraga\textunderscore )
\section{Taregicagem}
\begin{itemize}
\item {Grp. gram.:f.}
\end{itemize}
Profissão de \textunderscore taregá\textunderscore .
\section{Tareia}
\begin{itemize}
\item {Grp. gram.:f.}
\end{itemize}
Pancadas, tunda, sova.
(Cp. cast. \textunderscore tareia\textunderscore )
\section{Tarelar}
\begin{itemize}
\item {Grp. gram.:v. i.}
\end{itemize}
\begin{itemize}
\item {Utilização:Prov.}
\end{itemize}
\begin{itemize}
\item {Utilização:beir.}
\end{itemize}
O mesmo que \textunderscore tagarelar\textunderscore .
\section{Tarelice}
\begin{itemize}
\item {Grp. gram.:f.}
\end{itemize}
\begin{itemize}
\item {Utilização:Prov.}
\end{itemize}
\begin{itemize}
\item {Utilização:beir.}
\end{itemize}
Acto ou dito de tarelo.
\section{Tarela}
\begin{itemize}
\item {Grp. gram.:m.  e  f.}
\end{itemize}
Pessôa, que fala muito, e sem propósito.
(Cp. \textunderscore tarelo\textunderscore )
\section{Tarelo}
\begin{itemize}
\item {Grp. gram.:m.}
\end{itemize}
\begin{itemize}
\item {Utilização:Prov.}
\end{itemize}
\begin{itemize}
\item {Utilização:beir.}
\end{itemize}
Indivíduo, que tagarela. Cf. Filinto, \textunderscore passim\textunderscore .
Homem intrometido, que fala desapropositadamente e do que não entende.
\section{Tarentino}
\begin{itemize}
\item {Grp. gram.:adj.}
\end{itemize}
\begin{itemize}
\item {Grp. gram.:M.}
\end{itemize}
\begin{itemize}
\item {Proveniência:(Lat. \textunderscore tarentinus\textunderscore )}
\end{itemize}
Relativo a Tarento.
Habitante de Tarento.
\section{Tarentismo}
\begin{itemize}
\item {Grp. gram.:m.}
\end{itemize}
\begin{itemize}
\item {Proveniência:(De \textunderscore tarêntula\textunderscore )}
\end{itemize}
Doença nervosa, que dominou na Itália, nos séculos XV, XVI e XVII, causada especialmente pela mordedura da tarântula e ainda de outros insectos, e bem assim pela vista dos outros doentes feridos por tarântula.
\section{Tarêntula}
\begin{itemize}
\item {Grp. gram.:f.}
\end{itemize}
O mesmo que \textunderscore tarântula\textunderscore .
\section{Tarequice}
\begin{itemize}
\item {Grp. gram.:f.}
\end{itemize}
\begin{itemize}
\item {Utilização:Prov.}
\end{itemize}
\begin{itemize}
\item {Utilização:trasm.}
\end{itemize}
O mesmo que \textunderscore tarecada\textunderscore .
\section{Tarerequi}
\begin{itemize}
\item {Grp. gram.:m.}
\end{itemize}
\begin{itemize}
\item {Utilização:Bras}
\end{itemize}
O mesmo que \textunderscore matapasto\textunderscore .
\section{Tareroqui}
\begin{itemize}
\item {Grp. gram.:m.}
\end{itemize}
\begin{itemize}
\item {Utilização:Bras}
\end{itemize}
O mesmo que \textunderscore matapasto\textunderscore .
\section{Tareto}
\begin{itemize}
\item {Grp. gram.:m.}
\end{itemize}
Gênero de molluscos acéphalos.
\section{Targana}
\begin{itemize}
\item {Grp. gram.:f.}
\end{itemize}
O mesmo que \textunderscore taínha\textunderscore .
(Cp. \textunderscore tagana\textunderscore )
\section{Targiónia}
\begin{itemize}
\item {Grp. gram.:f.}
\end{itemize}
\begin{itemize}
\item {Proveniência:(De \textunderscore Targion\textunderscore , n. p.)}
\end{itemize}
Gênero de plantas phýceas.
\section{Targra}
\begin{itemize}
\item {Grp. gram.:f.}
\end{itemize}
Gênero de insectos hymenópteros.
\section{Tari}
\begin{itemize}
\item {Grp. gram.:m.}
\end{itemize}
Licor alcoólico, resultante do suco fermentado de várias palmeiras.
\section{Tari}
\begin{itemize}
\item {Grp. gram.:adj.}
\end{itemize}
(?)«\textunderscore ...e pagavam do soldo ao cavallo tari com faca, armado á guisa, trinta soldos por dia.\textunderscore »Fern. Lopes, \textunderscore Chrón. de D. Fern.\textunderscore , XXXVI.
\section{Tarianas}
\begin{itemize}
\item {Grp. gram.:m. Pl.}
\end{itemize}
\begin{itemize}
\item {Utilização:Bras}
\end{itemize}
Tríbo de aborígenes do Pará.
\section{Tarifa}
\begin{itemize}
\item {Grp. gram.:f.}
\end{itemize}
\begin{itemize}
\item {Proveniência:(Do ár. \textunderscore talrif\textunderscore )}
\end{itemize}
Pauta de direitos da alfandega.
Registo de valores.
\section{Tarifação}
\begin{itemize}
\item {Grp. gram.:f.}
\end{itemize}
\begin{itemize}
\item {Utilização:Bras}
\end{itemize}
Acto de tarifar.
\section{Tarifar}
\begin{itemize}
\item {Grp. gram.:v. t.}
\end{itemize}
Reduzir á tarifa; applicar a tarifa a.
\section{Tarifário}
\begin{itemize}
\item {Grp. gram.:adj.}
\end{itemize}
\begin{itemize}
\item {Utilização:Neol.}
\end{itemize}
Relativo a tarifas.
\section{Tarima}
\begin{itemize}
\item {Grp. gram.:f.}
\end{itemize}
\begin{itemize}
\item {Proveniência:(Do ár. \textunderscore tarima\textunderscore )}
\end{itemize}
Estrado com alcatifa, coberto com dossel.
Tarimba.
\section{Tarimba}
\begin{itemize}
\item {Grp. gram.:f.}
\end{itemize}
\begin{itemize}
\item {Utilização:Prov.}
\end{itemize}
\begin{itemize}
\item {Utilização:alent.}
\end{itemize}
\begin{itemize}
\item {Utilização:Fig.}
\end{itemize}
Estrado, sôbre que dormem os soldados nos quartéis e postos da guarda.
O mesmo que \textunderscore gironda\textunderscore ^1, porca velha.
Vida dos quartéis, vida do soldado.
(Corr. de \textunderscore tarima\textunderscore )
\section{Tarimbar}
\begin{itemize}
\item {Grp. gram.:v. i.}
\end{itemize}
\begin{itemize}
\item {Utilização:Pop.}
\end{itemize}
\begin{itemize}
\item {Proveniência:(De \textunderscore tarimba\textunderscore )}
\end{itemize}
Sêr soldado.
\section{Tarimbeiro}
\begin{itemize}
\item {Grp. gram.:m.  e  adj.}
\end{itemize}
\begin{itemize}
\item {Utilização:Fig.}
\end{itemize}
\begin{itemize}
\item {Proveniência:(De \textunderscore tarimba\textunderscore )}
\end{itemize}
O que dorme na tarimba.
O que vive ou viveu nos quartéis.
Diz-se do official do exército, que passou pelos postos de soldado, cabo e sargento, sem têr seguido o curso de habilitação para official superior.
Indivíduo grosseiro; incivil.
\section{Tarioba}
\begin{itemize}
\item {Grp. gram.:f.}
\end{itemize}
\begin{itemize}
\item {Utilização:Bras}
\end{itemize}
\begin{itemize}
\item {Proveniência:(T. tupi)}
\end{itemize}
Espécie de mollusco.
\section{Tarira}
\begin{itemize}
\item {Grp. gram.:f.}
\end{itemize}
Peixe do norte do Brasil.
\section{Taririqui}
\begin{itemize}
\item {Grp. gram.:m.}
\end{itemize}
\begin{itemize}
\item {Utilização:Bras}
\end{itemize}
Planta medicinal.
\section{Tarja}
\begin{itemize}
\item {Grp. gram.:f.}
\end{itemize}
Ornato de pintura, desenho ou escultura, na orla ou contôrno de um objecto.
Orla, guarnição.
Traço preto, que indica luto, nas margens do papel.
Escudo antigo.
(Do germ.)
\section{Tarjão}
\begin{itemize}
\item {Grp. gram.:m.}
\end{itemize}
Tarja grande.
Lápide rectangular, que contém inscripção ou letreiro.
\section{Tarjar}
\begin{itemize}
\item {Grp. gram.:v. t.}
\end{itemize}
Pôr tarja em; guarnecer de tarja; orlar.
\section{Tarjeta}
\begin{itemize}
\item {fónica:jê}
\end{itemize}
\begin{itemize}
\item {Grp. gram.:f.}
\end{itemize}
Pequena tarja.
\section{Tarlatana}
\begin{itemize}
\item {Grp. gram.:f.}
\end{itemize}
\begin{itemize}
\item {Proveniência:(Fr. \textunderscore tarlatane\textunderscore )}
\end{itemize}
Tecido transparente, e geralmente encorpado, para forros de vestuário.
\section{Taro}
\begin{itemize}
\item {Grp. gram.:m.}
\end{itemize}
Tubérculo das ilhas de Samôa, empregado na alimentação dos Indígenas. Cf. \textunderscore Jornal de Viagens\textunderscore , IV, 274.
\section{Taró}
\begin{itemize}
\item {Grp. gram.:m.}
\end{itemize}
\begin{itemize}
\item {Utilização:Gír.}
\end{itemize}
Frio.
Vento frio.
\section{Taroca}
\begin{itemize}
\item {Grp. gram.:f.}
\end{itemize}
\begin{itemize}
\item {Utilização:Prov.}
\end{itemize}
O mesmo que \textunderscore tamanco\textunderscore .
\section{Taroco}
\begin{itemize}
\item {fónica:tarô}
\end{itemize}
\begin{itemize}
\item {Grp. gram.:m.}
\end{itemize}
\begin{itemize}
\item {Utilização:Prov.}
\end{itemize}
\begin{itemize}
\item {Utilização:trasm.}
\end{itemize}
\begin{itemize}
\item {Grp. gram.:Pl.}
\end{itemize}
O mesmo que \textunderscore pedaço\textunderscore : \textunderscore um taroco de pão\textunderscore .
Tamancos.
(Cp. \textunderscore tanoco\textunderscore )
\section{Tarol}
\begin{itemize}
\item {Grp. gram.:m.}
\end{itemize}
O mesmo que \textunderscore tarola\textunderscore .
\section{Tarola}
\begin{itemize}
\item {Grp. gram.:f.}
\end{itemize}
Tambor chato de milícia, caixa de guerra.
(Cp. \textunderscore tarolo\textunderscore )
\section{Tarolar}
\begin{itemize}
\item {Grp. gram.:v. i.}
\end{itemize}
(V.tarelar)
\section{Tarole}
\begin{itemize}
\item {Grp. gram.:m.}
\end{itemize}
\begin{itemize}
\item {Utilização:Bras}
\end{itemize}
O mesmo que \textunderscore tarola\textunderscore . Cf. \textunderscore Diário Official\textunderscore , do Brasil, de 22-III-901.
\section{Tarolo}
\begin{itemize}
\item {fónica:tarô}
\end{itemize}
\begin{itemize}
\item {Grp. gram.:m.}
\end{itemize}
Pequeno tôro de lenha.
(Por \textunderscore torolo\textunderscore , de \textunderscore tôro\textunderscore )
\section{Tarono}
\begin{itemize}
\item {Grp. gram.:m.}
\end{itemize}
\begin{itemize}
\item {Utilização:Prov.}
\end{itemize}
\begin{itemize}
\item {Utilização:trasm.}
\end{itemize}
Pião, feito a podão, não torneiro.
\section{Taroque}
\begin{itemize}
\item {Grp. gram.:m.}
\end{itemize}
\begin{itemize}
\item {Utilização:Bras}
\end{itemize}
O mesmo que \textunderscore cornimboque\textunderscore .
\section{Tarós}
\begin{itemize}
\item {Grp. gram.:m.}
\end{itemize}
\begin{itemize}
\item {Utilização:Prov.}
\end{itemize}
\begin{itemize}
\item {Utilização:alg.}
\end{itemize}
Vento do Sueste.
(Cp. \textunderscore taró\textunderscore )
\section{Tarote}
\begin{itemize}
\item {Grp. gram.:m.}
\end{itemize}
\begin{itemize}
\item {Utilização:T. da Bairrada}
\end{itemize}
O mesmo que \textunderscore tarau\textunderscore .
\section{Tarouca}
\begin{itemize}
\item {Grp. gram.:f.}
\end{itemize}
\begin{itemize}
\item {Utilização:Anat.}
\end{itemize}
\begin{itemize}
\item {Utilização:ant.}
\end{itemize}
\begin{itemize}
\item {Proveniência:(Do ár. \textunderscore taruca\textunderscore )}
\end{itemize}
Músculo da coxa da perna.
\section{Tarouco}
\begin{itemize}
\item {Grp. gram.:adj.}
\end{itemize}
\begin{itemize}
\item {Utilização:Prov.}
\end{itemize}
\begin{itemize}
\item {Utilização:beir.}
\end{itemize}
Idiota; apatetado.
Desmemoriado pela idade.
\section{Tarouquice}
\begin{itemize}
\item {Grp. gram.:f.}
\end{itemize}
\begin{itemize}
\item {Utilização:Fam.}
\end{itemize}
\begin{itemize}
\item {Proveniência:(De \textunderscore tarouco\textunderscore )}
\end{itemize}
Parvoíce; estupidez.
\section{Tarpa}
\begin{itemize}
\item {Grp. gram.:f.}
\end{itemize}
Instrumento popular da Índia Portuguesa, formado de três peças, sendo uma de abóbora branca, outra de bambu e a terceira de fôlhas de palmeira brava. Cf. Lopes Mendes, \textunderscore Índ. Port.\textunderscore 
\section{Tarraçada}
\begin{itemize}
\item {Grp. gram.:f.}
\end{itemize}
\begin{itemize}
\item {Utilização:Chul.}
\end{itemize}
\begin{itemize}
\item {Proveniência:(De \textunderscore tarro\textunderscore )}
\end{itemize}
Grande porção de bebida; tigelada.
\section{Tarracha}
\textunderscore f.\textunderscore  (e der.)
(V. \textunderscore tarraxa\textunderscore , etc.)
\section{Tarracho}
\begin{itemize}
\item {Grp. gram.:m.}
\end{itemize}
\begin{itemize}
\item {Utilização:Prov.}
\end{itemize}
\begin{itemize}
\item {Utilização:trasm.}
\end{itemize}
Homem atarracado.
\section{Tarraco}
\begin{itemize}
\item {Grp. gram.:m.}
\end{itemize}
\begin{itemize}
\item {Utilização:Prov.}
\end{itemize}
\begin{itemize}
\item {Utilização:beir.}
\end{itemize}
O mesmo que \textunderscore tarracho\textunderscore .
\section{Tarraço}
\begin{itemize}
\item {Grp. gram.:m.}
\end{itemize}
Tarro grande. Cf. \textunderscore Chiado\textunderscore , 60.
\section{Tarraconense}
\begin{itemize}
\item {Grp. gram.:adj.}
\end{itemize}
\begin{itemize}
\item {Grp. gram.:M.}
\end{itemize}
\begin{itemize}
\item {Proveniência:(Lat. \textunderscore tarraconensis\textunderscore )}
\end{itemize}
Relativo a Tarragona.
Habitante de Tarragona.
\section{Tarracote}
\begin{itemize}
\item {Grp. gram.:adj.}
\end{itemize}
\begin{itemize}
\item {Utilização:T. da Bairrada}
\end{itemize}
\begin{itemize}
\item {Proveniência:(De \textunderscore tarraco\textunderscore )}
\end{itemize}
O mesmo que \textunderscore atarracado\textunderscore .
\section{Tarrada}
\begin{itemize}
\item {Grp. gram.:f.}
\end{itemize}
\begin{itemize}
\item {Proveniência:(De \textunderscore tarro\textunderscore )}
\end{itemize}
Porção de líquido, que um tarro póde conter; tarro cheio.
Antiga embarcação indiana.
\section{Tarrafa}
\begin{itemize}
\item {Grp. gram.:f.}
\end{itemize}
\begin{itemize}
\item {Utilização:Pop.}
\end{itemize}
\begin{itemize}
\item {Utilização:Bras. do N}
\end{itemize}
\begin{itemize}
\item {Proveniência:(Do ár. \textunderscore tarraha\textunderscore )}
\end{itemize}
Rêde de pesca.
Capa ou capote roto.
Espécie de renda.
\section{Tarrafar}
\begin{itemize}
\item {Grp. gram.:v. i.}
\end{itemize}
Pescar com tarrafa.
\section{Tarrafe}
\begin{itemize}
\item {Grp. gram.:m.}
\end{itemize}
O mesmo que \textunderscore tamargueira\textunderscore .
\section{Tarrafear}
\begin{itemize}
\item {Grp. gram.:v. i.}
\end{itemize}
\begin{itemize}
\item {Utilização:Bras. do N}
\end{itemize}
O mesmo que \textunderscore tarrafar\textunderscore .
Pegar na cauda do boi, para o derribar.
\section{Tarrafeira}
\begin{itemize}
\item {Grp. gram.:f.}
\end{itemize}
\begin{itemize}
\item {Utilização:Prov.}
\end{itemize}
\begin{itemize}
\item {Utilização:alent.}
\end{itemize}
O mesmo que \textunderscore tarrafe\textunderscore .
\section{Tarrafia}
\begin{itemize}
\item {Grp. gram.:f.}
\end{itemize}
\begin{itemize}
\item {Utilização:T. do Fundão}
\end{itemize}
O mesmo que \textunderscore pirraça\textunderscore .
\section{Tárraga}
\begin{itemize}
\item {Grp. gram.:f.}
\end{itemize}
\begin{itemize}
\item {Proveniência:(T. cast.)}
\end{itemize}
Espécie de dança, usada no século XVII.
\section{Tarranquém}
\begin{itemize}
\item {Grp. gram.:m.}
\end{itemize}
O mesmo que \textunderscore tarranquim\textunderscore .
\section{Tarranquim}
\begin{itemize}
\item {Grp. gram.:m.}
\end{itemize}
Embarcação asiática.
\section{Tarrantana}
\begin{itemize}
\item {Grp. gram.:f.}
\end{itemize}
Ave aquática (\textunderscore fuligula ferina\textunderscore , Lin.).
\section{Tarrasca}
\begin{itemize}
\item {Grp. gram.:f.}
\end{itemize}
\begin{itemize}
\item {Utilização:Prov.}
\end{itemize}
\begin{itemize}
\item {Utilização:beir.}
\end{itemize}
Espada velha, durindana, chanfalho.
\section{Tarratão}
\begin{itemize}
\item {Grp. gram.:m.}
\end{itemize}
Espécie de adem real.
(Cp. \textunderscore terranhão\textunderscore )
\section{Tarraxa}
\begin{itemize}
\item {Grp. gram.:f.}
\end{itemize}
Parafuso.
Cavilha; cunha.
Utensílio de serralheiro, com que se fazem as roscas dos parafusos.--Os diccionários portugueses, desde os mais antigos, têm \textunderscore tarracha\textunderscore , \textunderscore atarrachar\textunderscore , etc. Succede porém que a pronúncia beirôa e trasmontana diz \textunderscore tarraxa\textunderscore , o que está de acôrdo com o cast. \textunderscore terraja\textunderscore , visto que \textunderscore j\textunderscore  cast. não corresponde ao nosso \textunderscore ch\textunderscore . Compare \textunderscore Quijote\textunderscore  e \textunderscore Quixote\textunderscore .
\section{Tarraxar}
\textunderscore v. t.\textunderscore  (e der.)
O mesmo que \textunderscore atarrachar\textunderscore , etc. Cf. Filinto, V, 147.
\section{Tarraz-borraz}
\begin{itemize}
\item {Grp. gram.:adv.}
\end{itemize}
\begin{itemize}
\item {Utilização:Pleb.}
\end{itemize}
Desordenadamente; em confusão.
\section{Tarre}
\begin{itemize}
\item {Grp. gram.:m.}
\end{itemize}
Antiga moéda do Malabar.
\section{Tarreco}
\begin{itemize}
\item {Grp. gram.:m.}
\end{itemize}
\begin{itemize}
\item {Grp. gram.:Adj.}
\end{itemize}
O mesmo que \textunderscore tarraco\textunderscore .
Que tem pequena estatura.
\section{Tarrelo}
\begin{itemize}
\item {Grp. gram.:m.}
\end{itemize}
\begin{itemize}
\item {Utilização:Prov.}
\end{itemize}
\begin{itemize}
\item {Utilização:trasm.}
\end{itemize}
\begin{itemize}
\item {Proveniência:(De \textunderscore tarro\textunderscore )}
\end{itemize}
Panela pequena.
\section{T'arrenego!}
\begin{itemize}
\item {Grp. gram.:interj.}
\end{itemize}
\begin{itemize}
\item {Utilização:Fam.}
\end{itemize}
(Serve para designar repulsão ou censura)
Some-te! abrenúncio! deixa-me!
\section{Tarrincar}
\begin{itemize}
\item {Grp. gram.:V. i.}
\end{itemize}
\begin{itemize}
\item {Utilização:Prov.}
\end{itemize}
\begin{itemize}
\item {Utilização:trasm.}
\end{itemize}
\textunderscore v. t.\textunderscore  (e der.) \textunderscore Pop.\textunderscore 
O mesmo que \textunderscore trincar\textunderscore , etc.
Fazer trovoada.
Ranger os dentes.
\section{Tarrinco}
\begin{itemize}
\item {Grp. gram.:m.}
\end{itemize}
\begin{itemize}
\item {Utilização:Prov.}
\end{itemize}
\begin{itemize}
\item {Utilização:trasm.}
\end{itemize}
\begin{itemize}
\item {Proveniência:(De \textunderscore tarrincar\textunderscore )}
\end{itemize}
O mesmo que \textunderscore trovão\textunderscore .
\section{Tarrinheira}
\begin{itemize}
\item {Grp. gram.:f.}
\end{itemize}
\begin{itemize}
\item {Utilização:Prov.}
\end{itemize}
\begin{itemize}
\item {Utilização:trasm.}
\end{itemize}
Caminho estreito, entre paredes, de propriedades ruraes; canada, azinhaga.
\section{Tarro}
\begin{itemize}
\item {Grp. gram.:m.}
\end{itemize}
\begin{itemize}
\item {Proveniência:(Do gr. \textunderscore tarros\textunderscore ?)}
\end{itemize}
Vaso em que se recolhe o leite, quando êste se ordenha.
O mesmo que \textunderscore taioba\textunderscore .
\section{Tarro}
\begin{itemize}
\item {Grp. gram.:m.}
\end{itemize}
\begin{itemize}
\item {Utilização:Prov.}
\end{itemize}
\begin{itemize}
\item {Utilização:beir.}
\end{itemize}
O mesmo que \textunderscore sarro\textunderscore .
Depósito, que a urina deixa nos bacios.
Sedimento de qualquer líquido.
\section{Tarseiro}
\begin{itemize}
\item {Grp. gram.:m.}
\end{itemize}
\begin{itemize}
\item {Proveniência:(De \textunderscore tarso\textunderscore )}
\end{itemize}
Gênero de mammíferos quadrúmanos, de tarsos muito compridos.
\section{Tarsiano}
\begin{itemize}
\item {Grp. gram.:adj.}
\end{itemize}
Relativo ao tarso.
\section{Társico}
\begin{itemize}
\item {Grp. gram.:adj.}
\end{itemize}
O mesmo que \textunderscore tarsiano\textunderscore .
\section{Társio}
\begin{itemize}
\item {Grp. gram.:m.}
\end{itemize}
Gênero de mammíferos primatas.
\section{Tarsite}
\begin{itemize}
\item {Grp. gram.:f.}
\end{itemize}
Inflammação do tarso.
\section{Tarso}
\begin{itemize}
\item {Grp. gram.:m.}
\end{itemize}
\begin{itemize}
\item {Grp. gram.:Adj.}
\end{itemize}
\begin{itemize}
\item {Proveniência:(Do gr. \textunderscore tarsos\textunderscore )}
\end{itemize}
Parte posterior do pé, composto de sete ossos, encravados uns nos outros.
Terceiro segmento do pé das aves.
Sexta peça do pé simples dos crustáceos.
A última parte do pé dos insectos.
Diz-se das duas cartilagens que há na espessura do bórdo livre das pálpebras.
\section{Tarsósteno}
\begin{itemize}
\item {Grp. gram.:m.}
\end{itemize}
\begin{itemize}
\item {Proveniência:(De \textunderscore tarso\textunderscore  + gr. \textunderscore stenos\textunderscore )}
\end{itemize}
Gênero de insectos coleópteros pentâmeros.
\section{Tarsotomia}
\begin{itemize}
\item {Grp. gram.:f.}
\end{itemize}
\begin{itemize}
\item {Proveniência:(Do gr. \textunderscore tarsos\textunderscore  + \textunderscore tome\textunderscore )}
\end{itemize}
Córte do tarso.
\section{Tartada}
\begin{itemize}
\item {Grp. gram.:f.}
\end{itemize}
Barco indiano.
(Cp. \textunderscore tartana\textunderscore )
\section{Tartago}
\begin{itemize}
\item {Grp. gram.:m.}
\end{itemize}
Planta euphorbiácea, de sementes purgativas.
\section{Tartalha}
\begin{itemize}
\item {Grp. gram.:m.}
\end{itemize}
\begin{itemize}
\item {Utilização:Ant.}
\end{itemize}
\begin{itemize}
\item {Utilização:Pop.}
\end{itemize}
Homem muito falador; tagarela.
\section{Tartamelear}
\begin{itemize}
\item {Grp. gram.:v. i.}
\end{itemize}
\begin{itemize}
\item {Proveniência:(De \textunderscore tartamelo\textunderscore )}
\end{itemize}
O mesmo que \textunderscore tartamudear\textunderscore .
\section{Tartamelo}
\begin{itemize}
\item {Grp. gram.:m.}
\end{itemize}
O mesmo que \textunderscore tartamudo\textunderscore .
\section{Tartamudear}
\begin{itemize}
\item {Grp. gram.:v. i.}
\end{itemize}
\begin{itemize}
\item {Proveniência:(De \textunderscore tartamudo\textunderscore )}
\end{itemize}
Gaguejar.
Entaramelar-se.
Falar com difficuldade ou com tremura na voz, em consequência de susto ou surpresa.
\section{Tartamudez}
\begin{itemize}
\item {Grp. gram.:f.}
\end{itemize}
Qualidade ou defeito de tartamudo.
\section{Tartamudo}
\begin{itemize}
\item {Grp. gram.:adj.}
\end{itemize}
\begin{itemize}
\item {Proveniência:(De \textunderscore tártaro\textunderscore ^4 + \textunderscore mudo\textunderscore )}
\end{itemize}
Que tartamudeia.
Gago.
Que se exprime mal.
Que tem difficuldade em lêr pronunciando.
Hesitante, entaramelado:«\textunderscore voz tartamuda...\textunderscore ». Camillo, \textunderscore Brasileira\textunderscore , 279.
\section{Tartana}
\begin{itemize}
\item {Grp. gram.:f.}
\end{itemize}
\begin{itemize}
\item {Utilização:Prov.}
\end{itemize}
\begin{itemize}
\item {Utilização:alent.}
\end{itemize}
Barco alongado do Mediterrâneo.
Carroção coberto com toldo, mas aberto nos dois topos.
\section{Tartanha}
\begin{itemize}
\item {Grp. gram.:f.}
\end{itemize}
O mesmo que \textunderscore tartana\textunderscore .
\section{Tartaranha}
\begin{itemize}
\item {Grp. gram.:f.}
\end{itemize}
Fêmea do tartaranhão.
Barco de pesca, no Tejo.
Rêde de arrastar, a reboque de uma embarcação.
(Cast. \textunderscore tartaraña\textunderscore )
\section{Tartaranhão}
\begin{itemize}
\item {Grp. gram.:m.}
\end{itemize}
\begin{itemize}
\item {Proveniência:(De \textunderscore tartaranha\textunderscore )}
\end{itemize}
Nome de várias espécies de falcão.
\section{Tartaranho}
\begin{itemize}
\item {Grp. gram.:m.}
\end{itemize}
\begin{itemize}
\item {Utilização:Des.}
\end{itemize}
Carantonha, côca, papão, (para assustar crianças).
\section{Tartarato}
\begin{itemize}
\item {Grp. gram.:m.}
\end{itemize}
\begin{itemize}
\item {Utilização:Chím.}
\end{itemize}
Sal, resultante da combinação do ácido tartárico com uma base.
\section{Tartarear}
\begin{itemize}
\item {Grp. gram.:v. i.}
\end{itemize}
\begin{itemize}
\item {Grp. gram.:V. i.}
\end{itemize}
\begin{itemize}
\item {Proveniência:(De \textunderscore tártaro\textunderscore ^4)}
\end{itemize}
O mesmo que \textunderscore tartamudear\textunderscore .
Chalrear (a criança)
O mesmo que \textunderscore parolar\textunderscore .
\section{Tartáreo}
\begin{itemize}
\item {Grp. gram.:adj.}
\end{itemize}
\begin{itemize}
\item {Utilização:Poét.}
\end{itemize}
\begin{itemize}
\item {Proveniência:(Lat. \textunderscore tartareus\textunderscore )}
\end{itemize}
Relativo ao inferno.
\section{Tartárico}
\begin{itemize}
\item {Grp. gram.:adj.}
\end{itemize}
\begin{itemize}
\item {Proveniência:(De \textunderscore tártaro\textunderscore ^1)}
\end{itemize}
O mesmo que \textunderscore tartáreo\textunderscore .
\section{Tartárico}
\begin{itemize}
\item {Grp. gram.:adj.}
\end{itemize}
\begin{itemize}
\item {Proveniência:(De \textunderscore tártaro\textunderscore ^2)}
\end{itemize}
Relativo ao tártaro^2 e aos seus compostos.
Diz-se principalmente de um ácido, que se encontra nas uvas e em outros frutos.
\section{Tartárico}
\begin{itemize}
\item {Grp. gram.:adj.}
\end{itemize}
\begin{itemize}
\item {Proveniência:(De \textunderscore tártaro\textunderscore ^3)}
\end{itemize}
Diz-se dos Tártaros e das línguas uralo-altaicas.
\section{Tartarizar}
\begin{itemize}
\item {Grp. gram.:v. t.}
\end{itemize}
Misturar com tártaro^2.
\section{Tártaro}
\begin{itemize}
\item {Grp. gram.:m.}
\end{itemize}
\begin{itemize}
\item {Utilização:Poét.}
\end{itemize}
\begin{itemize}
\item {Proveniência:(Lat. \textunderscore tartarus\textunderscore )}
\end{itemize}
O mesmo que \textunderscore inferno\textunderscore .
\section{Tártaro}
\begin{itemize}
\item {Grp. gram.:m.}
\end{itemize}
Substância que, sob a fórma de crosta, adhere ás paredes das vasilhas de vinho.
Sarro.
Incrustação calcária, que se fórma sôbre os dentes.
(B. lat. \textunderscore tartarum\textunderscore )
\section{Tártaro}
\begin{itemize}
\item {Grp. gram.:adj.}
\end{itemize}
\begin{itemize}
\item {Grp. gram.:M.}
\end{itemize}
Relativo á Tartária.
Habitante da Tartária.
\section{Tártaro}
\begin{itemize}
\item {Grp. gram.:m.  e  adj.}
\end{itemize}
(V.tátaro)
\section{Tartaroso}
\begin{itemize}
\item {Grp. gram.:adj.}
\end{itemize}
Que tem tártaro^2; tartárico.
\section{Tartaruga}
\begin{itemize}
\item {Grp. gram.:f.}
\end{itemize}
\begin{itemize}
\item {Grp. gram.:M.  e  f.}
\end{itemize}
\begin{itemize}
\item {Utilização:Pop.}
\end{itemize}
\begin{itemize}
\item {Proveniência:(Do b. lat. \textunderscore tartuga\textunderscore )}
\end{itemize}
Animal amphíbio, que se move vagarosamente sôbre quatro pés, e cujo corpo é coberto por uma espécie de concha.
Concha de tartaruga.
Pessôa velha e feia.
\section{Tartarugo}
\begin{itemize}
\item {Grp. gram.:m.}
\end{itemize}
\begin{itemize}
\item {Utilização:Ant.}
\end{itemize}
Termo injurioso. Cf. G. Vicente, I, 262.
\section{Tartéssios}
\begin{itemize}
\item {Grp. gram.:m. pl.}
\end{itemize}
\begin{itemize}
\item {Proveniência:(Lat. \textunderscore Tartessii\textunderscore )}
\end{itemize}
Antigos habitadores da costa meridional da Espanha.
\section{Tártrico}
\begin{itemize}
\item {Grp. gram.:adj.}
\end{itemize}
\begin{itemize}
\item {Utilização:Gal}
\end{itemize}
\begin{itemize}
\item {Proveniência:(Fr. \textunderscore tartrique\textunderscore )}
\end{itemize}
O mesmo que \textunderscore tartárico\textunderscore ^1.
\section{Tartufia}
\begin{itemize}
\item {Grp. gram.:f.}
\end{itemize}
O mesmo que \textunderscore tartufice\textunderscore . Cf. José Castilho, \textunderscore Grinalda\textunderscore .
\section{Tartuficar}
\begin{itemize}
\item {Grp. gram.:v. t.}
\end{itemize}
\begin{itemize}
\item {Proveniência:(De \textunderscore tartufo\textunderscore )}
\end{itemize}
Embair com tartufices.
\section{Tartufice}
\begin{itemize}
\item {Grp. gram.:f.}
\end{itemize}
Acto ou dito de tartufo.
\section{Tartufismo}
\begin{itemize}
\item {Grp. gram.:m.}
\end{itemize}
Qualidade de tartufo; tartufice. Cf. Camillo, \textunderscore Sebenta\textunderscore , 21.
\section{Tartufista}
\begin{itemize}
\item {Grp. gram.:adj.}
\end{itemize}
Próprio de tartufo.
\section{Tartufo}
\begin{itemize}
\item {Grp. gram.:m.}
\end{itemize}
\begin{itemize}
\item {Proveniência:(De \textunderscore Tartufo\textunderscore , n. p.)}
\end{itemize}
Homem hypócrita.
Aquelle que tem falsa devoção.
\section{Taruba}
\begin{itemize}
\item {Grp. gram.:m.}
\end{itemize}
Bebida, feita de mandioca ralada, e usada entre os indígenas das margens do Tocantins.
\section{Taruca}
\begin{itemize}
\item {Grp. gram.:f.}
\end{itemize}
O mesmo que \textunderscore vicunha\textunderscore .
\section{Taruga}
\begin{itemize}
\item {Grp. gram.:f.}
\end{itemize}
O mesmo que \textunderscore taruca\textunderscore .
\section{Tarugamento}
\begin{itemize}
\item {Grp. gram.:m.}
\end{itemize}
Acto de tarugar.
\section{Tarugar}
\begin{itemize}
\item {Grp. gram.:v. t.}
\end{itemize}
Prender ou pregar com tarugo.
\section{Tarugo}
\begin{itemize}
\item {Grp. gram.:m.}
\end{itemize}
\begin{itemize}
\item {Utilização:Prov.}
\end{itemize}
\begin{itemize}
\item {Utilização:beir.}
\end{itemize}
\begin{itemize}
\item {Utilização:Bras}
\end{itemize}
Espécie de tôrno, com que se ligam uma a outra duas peças de madeira ou de outra substância.
Prego de madeira.
Pedaço de pau, que se colloca nos tectos, entre caibro e caibro.
Peça transversal de madeira, que se segura entre os barrotes do sobrado, para evitar que estes se desloquem lateralmente.
O mesmo que \textunderscore magnate\textunderscore .
(Cast. \textunderscore tarugo\textunderscore )
\section{Tarumá}
\begin{itemize}
\item {Grp. gram.:m.}
\end{itemize}
Nome de várias árvores fructíferas, da fam. das verbenáceas.
\section{Taruman}
\begin{itemize}
\item {Grp. gram.:m.}
\end{itemize}
\begin{itemize}
\item {Utilização:Bras}
\end{itemize}
Nome de várias árvores fructíferas, da fam. das verbenáceas.
\section{Tarumás}
\begin{itemize}
\item {Grp. gram.:m. pl.}
\end{itemize}
\begin{itemize}
\item {Utilização:Bras}
\end{itemize}
Tríbo de aborígenes do Pará.
\section{Tasca}
\begin{itemize}
\item {Grp. gram.:f.}
\end{itemize}
\begin{itemize}
\item {Utilização:Prov.}
\end{itemize}
\begin{itemize}
\item {Utilização:beir.}
\end{itemize}
Bodega.
Ordinária casa de pasto; baiúca; taberna.
Utensílio, formado por duas tábuas, sôbre o qual se espadela o linho, á falta de um cortiço.
(Do caló espanhol \textunderscore tasca\textunderscore )
\section{Tasca}
\begin{itemize}
\item {Grp. gram.:f.}
\end{itemize}
Acto ou effeito de tascar.
\section{Tascadeira}
\begin{itemize}
\item {Grp. gram.:f.}
\end{itemize}
Mulhér, que tasca o linho.
\section{Tascante}
\begin{itemize}
\item {Grp. gram.:adj.}
\end{itemize}
Que tasca.
\section{Tascante}
\begin{itemize}
\item {Grp. gram.:m.}
\end{itemize}
\begin{itemize}
\item {Utilização:Pop.}
\end{itemize}
\begin{itemize}
\item {Proveniência:(De \textunderscore tasca\textunderscore ^1)}
\end{itemize}
O mesmo que \textunderscore taberneiro\textunderscore .
\section{Tascar}
\begin{itemize}
\item {Grp. gram.:v. t.}
\end{itemize}
Tirar tasco a (o linho).
Espadelar.
Mastigar ou morder (o freio).
Morder; roer.
(Cast. \textunderscore tascar\textunderscore )
\section{Tasco}
\begin{itemize}
\item {Grp. gram.:m.}
\end{itemize}
\begin{itemize}
\item {Proveniência:(De \textunderscore tascar\textunderscore )}
\end{itemize}
Casca das fibras do linho, que se separa com a espadela; tomento.
\section{Tascôa}
\begin{itemize}
\item {Grp. gram.:f.}
\end{itemize}
\begin{itemize}
\item {Proveniência:(De um hyp. \textunderscore tascoar\textunderscore , por \textunderscore tascar\textunderscore )}
\end{itemize}
Acto de tascar o linho. Cf. \textunderscore Port. au point de vue agr.\textunderscore 
\section{Tasmânia}
\begin{itemize}
\item {Grp. gram.:f.}
\end{itemize}
\begin{itemize}
\item {Proveniência:(De \textunderscore Tasman\textunderscore , n. p.)}
\end{itemize}
Espécie de magnólia.
\section{Tasmaniano}
\begin{itemize}
\item {Grp. gram.:m.}
\end{itemize}
Aquelle que é natural da Tasmânia.
\section{Tasna}
\begin{itemize}
\item {Grp. gram.:m.}
\end{itemize}
O mesmo que \textunderscore tasneira\textunderscore .
\section{Tasneira}
\begin{itemize}
\item {Grp. gram.:f.}
\end{itemize}
Gênero de plantas compostas, (\textunderscore senecio jacobaea\textunderscore , Lin.).
\section{Tasneirinha}
\begin{itemize}
\item {Grp. gram.:f.}
\end{itemize}
\begin{itemize}
\item {Proveniência:(De \textunderscore tasneira\textunderscore )}
\end{itemize}
Espécie de tasna, (\textunderscore senecio vulgaris\textunderscore ).
\section{Tasqueiro}
\begin{itemize}
\item {Grp. gram.:m.}
\end{itemize}
\begin{itemize}
\item {Proveniência:(De \textunderscore tasca\textunderscore )}
\end{itemize}
O mesmo que \textunderscore taberneiro\textunderscore .
\section{Tasquinha}
\begin{itemize}
\item {Grp. gram.:f.}
\end{itemize}
\begin{itemize}
\item {Grp. gram.:M.  e  f.}
\end{itemize}
\begin{itemize}
\item {Utilização:Fam.}
\end{itemize}
\begin{itemize}
\item {Proveniência:(De \textunderscore tasco\textunderscore )}
\end{itemize}
Espadela.
Pessôa, que come com pouco appetite, que debica na comida.
\section{Tasquinhar}
\begin{itemize}
\item {Grp. gram.:v. t.}
\end{itemize}
\begin{itemize}
\item {Utilização:Fam.}
\end{itemize}
\begin{itemize}
\item {Grp. gram.:V. i.}
\end{itemize}
\begin{itemize}
\item {Utilização:Fam.}
\end{itemize}
\begin{itemize}
\item {Proveniência:(De \textunderscore tasquinha\textunderscore )}
\end{itemize}
Espadelar.
Comer.
Separar o tasco do linho.
Comer com pouco appetite, debicar.
\section{Tassádia}
\begin{itemize}
\item {Grp. gram.:f.}
\end{itemize}
Gênero de plantas asclepiádeas.
\section{Tassalho}
\begin{itemize}
\item {Grp. gram.:m.}
\end{itemize}
\begin{itemize}
\item {Utilização:Fam.}
\end{itemize}
Grande pedaço; grande fatia, naco.
(Cast. \textunderscore tassajo\textunderscore )
\section{Tassellos}
\begin{itemize}
\item {fónica:sê}
\end{itemize}
\begin{itemize}
\item {Grp. gram.:m. Pl.}
\end{itemize}
\begin{itemize}
\item {Proveniência:(It. \textunderscore tasello\textunderscore )}
\end{itemize}
Peças, de que se compõem as fôrmas, em que se vasa o gêsso líquido, o metal ou a cera, para dellas se extrahirem estátuas, baixos-relevos ou outros objectos de arte.
\section{Tastear}
\begin{itemize}
\item {Grp. gram.:v.}
\end{itemize}
\begin{itemize}
\item {Utilização:t. Mús.}
\end{itemize}
Pôr tastos em; pontear.
\section{Tasto}
\begin{itemize}
\item {Grp. gram.:m.}
\end{itemize}
\begin{itemize}
\item {Utilização:Ant.}
\end{itemize}
\begin{itemize}
\item {Proveniência:(It. \textunderscore tasto\textunderscore )}
\end{itemize}
Cada um dos filetes de metal que atravessam o braço dos instrumentos de corda, servindo de sinal, para que o tocador saiba onde deve apoiar os dedos.
O mesmo que \textunderscore tecla\textunderscore ^1.
\section{Tàtá}
\begin{itemize}
\item {Grp. gram.:m.}
\end{itemize}
(Outra fórma de \textunderscore papá\textunderscore )
(Cp. lat. \textunderscore tata\textunderscore )
\section{Tá-tá!}
\begin{itemize}
\item {Grp. gram.:interj.}
\end{itemize}
(com que alguém designa que alguma coisa lhe vejo repentinamente á ideia ou ao conhecimento)
\section{Tatabu}
\begin{itemize}
\item {Grp. gram.:m.}
\end{itemize}
Grande árvore de madeira escura, na Guiana inglesa.
\section{Tataíba}
\begin{itemize}
\item {Grp. gram.:f.}
\end{itemize}
O mesmo que \textunderscore tatajuba\textunderscore .
\section{Tataíra}
\begin{itemize}
\item {Grp. gram.:f.}
\end{itemize}
\begin{itemize}
\item {Utilização:Bras}
\end{itemize}
\begin{itemize}
\item {Grp. gram.:M.}
\end{itemize}
\begin{itemize}
\item {Utilização:Bras. do N}
\end{itemize}
Espécie de abelha.
Pássaro preto, de malhas vermelhas.
\section{Tatajiba}
\begin{itemize}
\item {Grp. gram.:f.}
\end{itemize}
Planta urticácea, (\textunderscore morus tinctoria\textunderscore ).
\section{Tatajuba}
\begin{itemize}
\item {Grp. gram.:f.}
\end{itemize}
Planta urticácea, (\textunderscore morus tinctoria\textunderscore ).
\section{Tatalar}
\begin{itemize}
\item {Grp. gram.:v. i.}
\end{itemize}
\begin{itemize}
\item {Utilização:bras}
\end{itemize}
\begin{itemize}
\item {Utilização:Neol.}
\end{itemize}
\begin{itemize}
\item {Proveniência:(T. onom.)}
\end{itemize}
Produzir som sêco, como o de ossos que batem uns nos outros.
Rumorejar.
\section{Tatamba}
\begin{itemize}
\item {Grp. gram.:m.  e  f.}
\end{itemize}
\begin{itemize}
\item {Utilização:Bras}
\end{itemize}
Pessôa, que fala mal.
Pessôa rude.
\section{Tatapiririca}
\begin{itemize}
\item {Grp. gram.:f.}
\end{itemize}
Planta terebinthácea do Brasil.
\section{Tataporas}
\begin{itemize}
\item {Grp. gram.:f. pl.}
\end{itemize}
\begin{itemize}
\item {Utilização:Bras}
\end{itemize}
O mesmo que \textunderscore cataporas\textunderscore .
\section{Tataraneta}
\begin{itemize}
\item {Grp. gram.:f.}
\end{itemize}
(Corr. de \textunderscore tetraneta\textunderscore )
\section{Tataraneto}
\begin{itemize}
\item {Grp. gram.:m.}
\end{itemize}
(Corr. de \textunderscore tetraneto\textunderscore )
\section{Tataranha}
\begin{itemize}
\item {Grp. gram.:m. ,  f.  e  adj.}
\end{itemize}
\begin{itemize}
\item {Utilização:Fam.}
\end{itemize}
\begin{itemize}
\item {Proveniência:(De \textunderscore tátaro\textunderscore )}
\end{itemize}
Pessôa acanhada, tímida.
\section{Tataranhão}
\begin{itemize}
\item {Grp. gram.:m.}
\end{itemize}
O mesmo que \textunderscore milhafre\textunderscore .
Nome, que também se dá ao \textunderscore ritaforme\textunderscore . Cf. Filinto, X, 226.
(Cp. \textunderscore tartaranhão\textunderscore )
\section{Tataranhar}
\begin{itemize}
\item {Grp. gram.:v. i.}
\end{itemize}
\begin{itemize}
\item {Utilização:Fam.}
\end{itemize}
\begin{itemize}
\item {Proveniência:(De \textunderscore tataranha\textunderscore )}
\end{itemize}
Tartamudear.
Acanhar-se; atrapalhar-se.
\section{Tataranho}
\begin{itemize}
\item {Grp. gram.:m.  e  adj.}
\end{itemize}
O que tataranha.
\section{Tataravô}
\begin{itemize}
\item {Grp. gram.:m.}
\end{itemize}
(Corr. de \textunderscore tetravô\textunderscore )
\section{Tataravó}
\begin{itemize}
\item {Grp. gram.:f.}
\end{itemize}
(Corr de \textunderscore tetravó\textunderscore )
\section{Tataré}
\begin{itemize}
\item {Grp. gram.:m.}
\end{itemize}
Árvore do Paraguai, de madeira amarelada, e bôa para construcções navaes.
\section{Tatarema}
\begin{itemize}
\item {Grp. gram.:f.}
\end{itemize}
O mesmo que \textunderscore tatajuba\textunderscore .
\section{Tatarez}
\begin{itemize}
\item {Grp. gram.:f.}
\end{itemize}
Qualidade de tátaro; gaguez.
\section{Tataria}
\begin{itemize}
\item {Grp. gram.:f.}
\end{itemize}
\begin{itemize}
\item {Utilização:Prov.}
\end{itemize}
\begin{itemize}
\item {Utilização:trasm.}
\end{itemize}
(Mais us. no pl.)
Palavreado enfadonho; séca; impertinência.
\section{Tátaro}
\begin{itemize}
\item {Grp. gram.:m.  e  adj.}
\end{itemize}
\begin{itemize}
\item {Proveniência:(T. onom.)}
\end{itemize}
O que articula mal ou com difficuldade as palavras.
\section{Tataúba}
\begin{itemize}
\item {Grp. gram.:f.}
\end{itemize}
O mesmo que \textunderscore tatajuba\textunderscore .
\section{Tate!}
\begin{itemize}
\item {Grp. gram.:interj.}
\end{itemize}
Cautela! Veja lá! Oh!
\section{Tateto}
\begin{itemize}
\item {fónica:tê}
\end{itemize}
\begin{itemize}
\item {Grp. gram.:m.}
\end{itemize}
\begin{itemize}
\item {Utilização:Bras. do S}
\end{itemize}
Mammífero, o mesmo que \textunderscore caititu\textunderscore ^1.
\section{Tatibitate}
\begin{itemize}
\item {Grp. gram.:m.  e  adj.}
\end{itemize}
\begin{itemize}
\item {Utilização:Fig.}
\end{itemize}
\begin{itemize}
\item {Proveniência:(T. onom.)}
\end{itemize}
O que é tataranha; gago; tartamudo.
Acanhado, inhenho.
\section{Taticuman}
\begin{itemize}
\item {Grp. gram.:m.}
\end{itemize}
\begin{itemize}
\item {Utilização:Bras. do N}
\end{itemize}
O mesmo que \textunderscore picuman\textunderscore .
\section{Tatu}
\begin{itemize}
\item {Grp. gram.:m.}
\end{itemize}
Nome commum a várias espécies de mammíferos desdentados do Brasil.
\section{Tatu}
\begin{itemize}
\item {Grp. gram.:m.}
\end{itemize}
Árvore americana, própria para construcções.
\section{Tatu}
\begin{itemize}
\item {Grp. gram.:m.}
\end{itemize}
\begin{itemize}
\item {Utilização:Bras}
\end{itemize}
Bailado campestre, espécie de fandango.
\section{Tatua}
\begin{itemize}
\item {Grp. gram.:f.}
\end{itemize}
Espécie de vespa americana.
\section{Tatuadora}
\begin{itemize}
\item {Grp. gram.:f.}
\end{itemize}
Mulhér que, nalguns paises selvagens, é especialmente incumbida de tatuar as raparigas.
\section{Tatuagem}
\begin{itemize}
\item {Grp. gram.:f.}
\end{itemize}
\begin{itemize}
\item {Proveniência:(De \textunderscore tatuar\textunderscore )}
\end{itemize}
Conjunto dos meios, com que se introduzem debaixo da epiderme substâncias còrantes, vegetaes ou mineraes, para produzir desenhos duradoiros e apparentes.
\section{Tatuar}
\begin{itemize}
\item {Grp. gram.:v. t.}
\end{itemize}
\begin{itemize}
\item {Proveniência:(Do fr. \textunderscore tatouer\textunderscore )}
\end{itemize}
Fazer tatuagem em.
\section{Tatu-bola}
\begin{itemize}
\item {Grp. gram.:m.}
\end{itemize}
Variedade de tatu^1.
\section{Tatu-china}
\begin{itemize}
\item {Grp. gram.:m.}
\end{itemize}
Variedade de tatu^1.
\section{Tatuço}
\begin{itemize}
\item {Grp. gram.:m.}
\end{itemize}
\begin{itemize}
\item {Utilização:Prov.}
\end{itemize}
\begin{itemize}
\item {Utilização:alg.}
\end{itemize}
O mesmo que toitiço.
\section{Tatui}
\begin{itemize}
\item {Grp. gram.:m.}
\end{itemize}
(V. \textunderscore tatu\textunderscore ^1)
\section{Tatu-peba}
\begin{itemize}
\item {Grp. gram.:m.}
\end{itemize}
Variedade de tatu^1.
\section{Taturana}
\begin{itemize}
\item {Grp. gram.:f.}
\end{itemize}
\begin{itemize}
\item {Utilização:Bras}
\end{itemize}
Larva, com uma felpa que produz sensação dolorosa em quem a toca.
\section{Tatzé}
\begin{itemize}
\item {Grp. gram.:m.}
\end{itemize}
Planta myrsinácea brasileira.
O fruto sêco dessa planta.
\section{Tau}
\begin{itemize}
\item {Grp. gram.:m.}
\end{itemize}
Cruz branca em fórma de T, que no seu hábito usavam os cónegos de Santo-Antão.
(Da fórma e nome da última letra do alphabeto hebraico)
\section{Tauá}
\begin{itemize}
\item {Grp. gram.:m.}
\end{itemize}
\begin{itemize}
\item {Utilização:Bras}
\end{itemize}
\begin{itemize}
\item {Proveniência:(T. tupi)}
\end{itemize}
Peróxydo de ferro.
\section{Tauaçu}
\begin{itemize}
\item {Grp. gram.:m.}
\end{itemize}
\begin{itemize}
\item {Utilização:Bras. do N}
\end{itemize}
Pedra furada, que serve de âncora ás jangadas.
(Contr. do tupi \textunderscore itaguaçu\textunderscore )
\section{Taunan}
\begin{itemize}
\item {Grp. gram.:m.}
\end{itemize}
\begin{itemize}
\item {Utilização:Bras}
\end{itemize}
Variedade de cigarro.
\section{Tauari}
\begin{itemize}
\item {Grp. gram.:m.}
\end{itemize}
Árvore brasileira, de fibras têxteis.
\section{Taúba}
\begin{itemize}
\item {Grp. gram.:f.}
\end{itemize}
O mesmo que \textunderscore tatajuba\textunderscore .
\section{Taugui}
\begin{itemize}
\item {Grp. gram.:m.}
\end{itemize}
Língua uralo-altaica, do grupo samoiedico.
\section{Tauncho}
\begin{itemize}
\item {Grp. gram.:m.}
\end{itemize}
Planta da serra de Sintra.
\section{Taupla}
\begin{itemize}
\item {Grp. gram.:f.}
\end{itemize}
Espécie de antigo e luxuoso vestuário de mulhér? cobertor precioso? Cf. \textunderscore Provas da Hist. Geneal.\textunderscore , II, na descripção do enxoval de D. Beatriz.
(Cp. fr. ant. \textunderscore tauplis\textunderscore )
\section{Táureo}
\begin{itemize}
\item {Grp. gram.:adj.}
\end{itemize}
\begin{itemize}
\item {Utilização:Poét.}
\end{itemize}
\begin{itemize}
\item {Proveniência:(Lat. \textunderscore taureus\textunderscore )}
\end{itemize}
Relativo a toiro. Cf. Castilho, \textunderscore Fastos\textunderscore , III, 73.
\section{Tauricéfalo}
\begin{itemize}
\item {Grp. gram.:adj.}
\end{itemize}
\begin{itemize}
\item {Proveniência:(Do gr. \textunderscore tauros\textunderscore  + \textunderscore kephale\textunderscore )}
\end{itemize}
Que tem cabeça de toiro.
\section{Tauricéphalo}
\begin{itemize}
\item {Grp. gram.:adj.}
\end{itemize}
\begin{itemize}
\item {Proveniência:(Do gr. \textunderscore tauros\textunderscore  + \textunderscore kephale\textunderscore )}
\end{itemize}
Que tem cabeça de toiro.
\section{Tauricida}
\begin{itemize}
\item {Grp. gram.:m. ,  f.  e  adj.}
\end{itemize}
\begin{itemize}
\item {Proveniência:(Do lat. \textunderscore taurus\textunderscore  + \textunderscore caedere\textunderscore )}
\end{itemize}
Pessôa, que mata toiros.
\section{Tauricídio}
\begin{itemize}
\item {Grp. gram.:m.}
\end{itemize}
Acto de matar toiros.
(Cp. \textunderscore tauricida\textunderscore )
\section{Tauricorne}
\begin{itemize}
\item {Grp. gram.:adj.}
\end{itemize}
O mesmo ou melhór que \textunderscore tauricórneo\textunderscore .
\section{Tauricórneo}
\begin{itemize}
\item {Grp. gram.:adj.}
\end{itemize}
\begin{itemize}
\item {Proveniência:(Lat. \textunderscore tauricornis\textunderscore )}
\end{itemize}
Que tem cornos de toiro.
\section{Tauricorno}
\begin{itemize}
\item {Grp. gram.:adj.}
\end{itemize}
(V.tauricórneo)
\section{Taurífero}
\begin{itemize}
\item {Grp. gram.:adj.}
\end{itemize}
\begin{itemize}
\item {Proveniência:(Lat. \textunderscore taurifer\textunderscore )}
\end{itemize}
Em que se criam toiros; em que pastam toiros.
\section{Tauriforme}
\begin{itemize}
\item {Grp. gram.:adj.}
\end{itemize}
\begin{itemize}
\item {Proveniência:(Lat. \textunderscore tauriformis\textunderscore )}
\end{itemize}
Que tem fórma de toiro.
Semelhante a toiro.
\section{Taurifrônteo}
\begin{itemize}
\item {Grp. gram.:adj.}
\end{itemize}
\begin{itemize}
\item {Proveniência:(Do lat. \textunderscore taurus\textunderscore  + \textunderscore frons\textunderscore , \textunderscore frontis\textunderscore )}
\end{itemize}
Relativo á fronte do toiro:«\textunderscore o taurifrônteo pelo...\textunderscore »Filinto, XIV, 12.
\section{Taurim}
\begin{itemize}
\item {Grp. gram.:m.}
\end{itemize}
Antiga embarcação indiana.
\section{Taurina}
\begin{itemize}
\item {Grp. gram.:f.}
\end{itemize}
\begin{itemize}
\item {Utilização:Chím.}
\end{itemize}
\begin{itemize}
\item {Proveniência:(Lat. \textunderscore taurina\textunderscore )}
\end{itemize}
Substância crystallizável, descoberta no fel do boi.
\section{Taurino}
\begin{itemize}
\item {Grp. gram.:adj.}
\end{itemize}
\begin{itemize}
\item {Proveniência:(Lat. \textunderscore taurinus\textunderscore )}
\end{itemize}
O mesmo que \textunderscore táureo\textunderscore .
\section{Tauro}
\begin{itemize}
\item {Grp. gram.:m.}
\end{itemize}
\begin{itemize}
\item {Proveniência:(Lat. \textunderscore taurus\textunderscore )}
\end{itemize}
Segundo signo do zodíaco, que se representa sob a figura de um boi.
\section{Taurobólio}
\begin{itemize}
\item {Grp. gram.:m.}
\end{itemize}
\begin{itemize}
\item {Proveniência:(Lat. \textunderscore taurobolium\textunderscore )}
\end{itemize}
Sacrifício expiatório, inventado pelos sacerdotes pagãos no século III, em opposição ao baptismo christão e em honra de Cybele, que consistia em immolar um toiro sôbre uma pedra crivada de furos, debaixo da qual estava o peccador, para receber a ablução do sangue da rês.
\section{Tauróbolo}
\begin{itemize}
\item {Grp. gram.:m.}
\end{itemize}
\begin{itemize}
\item {Proveniência:(Lat. \textunderscore taurobolos\textunderscore )}
\end{itemize}
Antigo sacrificador de toiros.
\section{Tauróbolo}
\begin{itemize}
\item {Grp. gram.:m.}
\end{itemize}
(V.taurobólio)
\section{Taurocenta}
\begin{itemize}
\item {Grp. gram.:m.}
\end{itemize}
\begin{itemize}
\item {Proveniência:(Lat. \textunderscore taurocenta\textunderscore )}
\end{itemize}
Aquelle que, entre os antigos, toireava a cavallo.
\section{Taurócera}
\begin{itemize}
\item {Grp. gram.:f.}
\end{itemize}
\begin{itemize}
\item {Proveniência:(Do gr. \textunderscore tauros\textunderscore  + \textunderscore keras\textunderscore )}
\end{itemize}
Gênero de insectos coleópteros heterómeros.
\section{Taurócero}
\begin{itemize}
\item {Grp. gram.:m.}
\end{itemize}
\begin{itemize}
\item {Proveniência:(Do gr. \textunderscore tauros\textunderscore  + \textunderscore keras\textunderscore )}
\end{itemize}
Gênero de insectos hemípteros.
\section{Tauromachia}
\begin{itemize}
\item {fónica:qui}
\end{itemize}
\begin{itemize}
\item {Grp. gram.:f.}
\end{itemize}
\begin{itemize}
\item {Proveniência:(Do gr. \textunderscore tauros\textunderscore  + \textunderscore makhe\textunderscore )}
\end{itemize}
Arte de toirear.
\section{Tauromáchico}
\begin{itemize}
\item {fónica:qui}
\end{itemize}
\begin{itemize}
\item {Grp. gram.:adj.}
\end{itemize}
Relativo á tauromachia.
\section{Tauromaquia}
\begin{itemize}
\item {Grp. gram.:f.}
\end{itemize}
\begin{itemize}
\item {Proveniência:(Do gr. \textunderscore tauros\textunderscore  + \textunderscore makhe\textunderscore )}
\end{itemize}
Arte de toirear.
\section{Tauromáquico}
\begin{itemize}
\item {Grp. gram.:adj.}
\end{itemize}
Relativo á tauromaquia.
\section{Tauros}
\begin{itemize}
\item {Grp. gram.:m. pl.}
\end{itemize}
\begin{itemize}
\item {Proveniência:(Lat. \textunderscore Tauri\textunderscore )}
\end{itemize}
Antigos povos da Thrácia, que se diffundiram na região chamada hoje Crimeia.
\section{Tausar}
\textunderscore v. t.\textunderscore  (e der.) \textunderscore Ant.\textunderscore 
O mesmo que \textunderscore taxar\textunderscore , etc.
\section{Tauschéria}
\begin{itemize}
\item {Grp. gram.:f.}
\end{itemize}
\begin{itemize}
\item {Proveniência:(De \textunderscore Tauscher\textunderscore , n. p.)}
\end{itemize}
Gênero de plantas crucíferas.
\section{Tautear}
\begin{itemize}
\item {Grp. gram.:v. t.  e  i.}
\end{itemize}
(V.trautear)
\section{Tautochronismo}
\begin{itemize}
\item {Grp. gram.:m.}
\end{itemize}
Qualidade ou estado do que é tautóchrono.
\section{Tautóchrono}
\begin{itemize}
\item {Grp. gram.:adj.}
\end{itemize}
\begin{itemize}
\item {Proveniência:(Do gr. \textunderscore tauto\textunderscore  + \textunderscore khronos\textunderscore )}
\end{itemize}
O mesmo que \textunderscore sýnchrono\textunderscore .
\section{Tautocronismo}
\begin{itemize}
\item {Grp. gram.:m.}
\end{itemize}
Qualidade ou estado do que é tautóchrono.
\section{Tautócrono}
\begin{itemize}
\item {Grp. gram.:adj.}
\end{itemize}
\begin{itemize}
\item {Proveniência:(Do gr. \textunderscore tauto\textunderscore  + \textunderscore khronos\textunderscore )}
\end{itemize}
O mesmo que \textunderscore síncrono\textunderscore .
\section{Tautofonia}
\begin{itemize}
\item {Grp. gram.:f.}
\end{itemize}
\begin{itemize}
\item {Proveniência:(Do gr. \textunderscore tauto\textunderscore  + \textunderscore phone\textunderscore )}
\end{itemize}
Repetição excessiva do mesmo som.
Monotonia de som.
\section{Tautograma}
\begin{itemize}
\item {Grp. gram.:m.}
\end{itemize}
\begin{itemize}
\item {Proveniência:(Do gr. \textunderscore tauto\textunderscore  + \textunderscore gramma\textunderscore )}
\end{itemize}
Composição em verso, em que se não empregam senão palavras que comecem todas pela mesma letra.
\section{Tautogramma}
\begin{itemize}
\item {Grp. gram.:m.}
\end{itemize}
\begin{itemize}
\item {Proveniência:(Do gr. \textunderscore tauto\textunderscore  + \textunderscore gramma\textunderscore )}
\end{itemize}
Composição em verso, em que se não empregam senão palavras que comecem todas pela mesma letra.
\section{Tautologia}
\begin{itemize}
\item {Grp. gram.:f.}
\end{itemize}
\begin{itemize}
\item {Utilização:Gram.}
\end{itemize}
\begin{itemize}
\item {Proveniência:(Lat. \textunderscore tautologia\textunderscore )}
\end{itemize}
Vício de linguagem, que consiste em dizer sempre a mesma coisa por fórmas differentes.
\section{Tautológico}
\begin{itemize}
\item {Grp. gram.:adj.}
\end{itemize}
Relativo á tautologia ou que tem o caracter della.
\section{Tautometria}
\begin{itemize}
\item {Grp. gram.:f.}
\end{itemize}
\begin{itemize}
\item {Proveniência:(Do gr. \textunderscore tauto\textunderscore  + \textunderscore metron\textunderscore )}
\end{itemize}
Demasiada symetria; monotonia.
\section{Tautophonia}
\begin{itemize}
\item {Grp. gram.:f.}
\end{itemize}
\begin{itemize}
\item {Proveniência:(Do gr. \textunderscore tauto\textunderscore  + \textunderscore phone\textunderscore )}
\end{itemize}
Repetição excessiva do mesmo som.
Monotonia de som.
\section{Tautossilabismo}
\begin{itemize}
\item {Grp. gram.:m.}
\end{itemize}
\begin{itemize}
\item {Proveniência:(Do gr. \textunderscore tautos\textunderscore  + \textunderscore sullabe\textunderscore )}
\end{itemize}
Repetição de sílabas iguaes, formando vocabulos populares ou familiares: \textunderscore Lulu\textunderscore , \textunderscore Mimi\textunderscore , \textunderscore Titi\textunderscore  (tia), etc.
\section{Tautosyllabismo}
\begin{itemize}
\item {fónica:si}
\end{itemize}
\begin{itemize}
\item {Grp. gram.:m.}
\end{itemize}
\begin{itemize}
\item {Proveniência:(Do gr. \textunderscore tautos\textunderscore  + \textunderscore sullabe\textunderscore )}
\end{itemize}
Repetição de sýllabas iguaes, formando vocabulos populares ou familiares: \textunderscore Lulu\textunderscore , \textunderscore Mimi\textunderscore , \textunderscore Titi\textunderscore  (tia), etc.
\section{Tauxia}
\begin{itemize}
\item {Grp. gram.:f.}
\end{itemize}
\begin{itemize}
\item {Proveniência:(Do ár. \textunderscore tauxia\textunderscore )}
\end{itemize}
Obra de embutidos de metal em aço ou ferro.
\section{Tauxiar}
\begin{itemize}
\item {Grp. gram.:v. t.}
\end{itemize}
Ornar ou lavrar com tauxia.
\section{Tava}
\begin{itemize}
\item {Grp. gram.:f.}
\end{itemize}
\begin{itemize}
\item {Utilização:Bras. do S}
\end{itemize}
\begin{itemize}
\item {Proveniência:(Do cast. \textunderscore taba\textunderscore )}
\end{itemize}
Jôgo, usado pelos gachos.
\section{Tavajiba}
\begin{itemize}
\item {Grp. gram.:f.}
\end{itemize}
O mesmo que \textunderscore tatajuba\textunderscore .
\section{Tavalas}
\begin{itemize}
\item {Grp. gram.:m. pl.}
\end{itemize}
Uma das tribos cafreaes de Tete e Zumbo.
\section{Tavanês}
\begin{itemize}
\item {Grp. gram.:adj.}
\end{itemize}
\begin{itemize}
\item {Proveniência:(De \textunderscore tavão\textunderscore )}
\end{itemize}
Turbulento; estavanado; estouvado.
Activo.
\section{Tavão}
\begin{itemize}
\item {Grp. gram.:m.}
\end{itemize}
\begin{itemize}
\item {Proveniência:(Do lat. \textunderscore tabanus\textunderscore )}
\end{itemize}
Insecto díptero.
\section{Tavarenho}
\begin{itemize}
\item {Grp. gram.:adj.}
\end{itemize}
(?)«\textunderscore ...o arção derradeiro da sella tavarenha.\textunderscore »Fern. Lopes, \textunderscore Chrón. de D. Fern.\textunderscore , c. XCIX.
\section{Tavares}
\begin{itemize}
\item {Grp. gram.:m.}
\end{itemize}
Variedade de pêro.
\section{Táveda}
\begin{itemize}
\item {Grp. gram.:f.}
\end{itemize}
\begin{itemize}
\item {Utilização:Bot.}
\end{itemize}
O mesmo que \textunderscore tádega\textunderscore . Cf. P. Coutinho. \textunderscore Flora\textunderscore , 621.
\section{Taverna}
\textunderscore f.\textunderscore  (e der.)
(V. \textunderscore taberna\textunderscore , etc.)
\section{Taverniéria}
\begin{itemize}
\item {Grp. gram.:f.}
\end{itemize}
\begin{itemize}
\item {Proveniência:(De \textunderscore Tavernier\textunderscore , n. p.)}
\end{itemize}
Gênero de plantas leguminosas.
\section{Távola}
\begin{itemize}
\item {Grp. gram.:f.}
\end{itemize}
(V.tábula)
\section{Tavolageiro}
\begin{itemize}
\item {Grp. gram.:m.}
\end{itemize}
(V.tabulageiro)
\section{Tavolagem}
\begin{itemize}
\item {Grp. gram.:f.}
\end{itemize}
(V.tabulagem)
\section{Tavolachinha}
\begin{itemize}
\item {Grp. gram.:f.}
\end{itemize}
\begin{itemize}
\item {Utilização:Ant.}
\end{itemize}
\begin{itemize}
\item {Proveniência:(De \textunderscore távola\textunderscore )}
\end{itemize}
Talvez arma defensiva, que exhibia superfície larga ou escudete. Cf. \textunderscore Roteiro de Vasco da Gama\textunderscore .
\section{Tavolatura}
\begin{itemize}
\item {Grp. gram.:f}
\end{itemize}
\begin{itemize}
\item {Utilização:Mús.}
\end{itemize}
\begin{itemize}
\item {Utilização:Ant.}
\end{itemize}
\begin{itemize}
\item {Proveniência:(De \textunderscore távola\textunderscore )}
\end{itemize}
Systema de notação, usado nos séculos XVI e XVII, e applicável especialmente nos instrumentos de corda, traçando-se um número de linhas, igual ao das cordas, e escrevendo-se sôbre essas linhas o nome dos dedos que o tocador devia apoiar entre os tastos.
\section{Tax}
\begin{itemize}
\item {Grp. gram.:m.}
\end{itemize}
O mesmo que \textunderscore taz\textunderscore . Cf. Leon, \textunderscore Arte de Ferrar\textunderscore , 61.
\section{Taxa}
\begin{itemize}
\item {Grp. gram.:f.}
\end{itemize}
\begin{itemize}
\item {Utilização:Fig.}
\end{itemize}
\begin{itemize}
\item {Proveniência:(De \textunderscore taxar\textunderscore )}
\end{itemize}
Regulamento sôbre o preço de gêneros ou mercadorias.
Preço, conforme aos regulamentos.
Imposto.
Termo, limite.
\section{Taxação}
\begin{itemize}
\item {Grp. gram.:f.}
\end{itemize}
\begin{itemize}
\item {Proveniência:(Do lat. \textunderscore taxatio\textunderscore )}
\end{itemize}
Acto ou effeito de taxar.
Direito antigo, que se pagava aos administradores da fazenda nacional.
\section{Taxadamente}
\begin{itemize}
\item {Grp. gram.:adv.}
\end{itemize}
\begin{itemize}
\item {Proveniência:(De \textunderscore taxado\textunderscore )}
\end{itemize}
Com moderação; limitadamente.
\section{Taxador}
\begin{itemize}
\item {Grp. gram.:m.  e  adj.}
\end{itemize}
\begin{itemize}
\item {Proveniência:(Lat. \textunderscore taxator\textunderscore )}
\end{itemize}
O que taxa.
\section{Taxâmetro}
\begin{itemize}
\item {Grp. gram.:m.}
\end{itemize}
(V.taxímetro)
\section{Taxar}
\begin{itemize}
\item {Grp. gram.:v. t.}
\end{itemize}
\begin{itemize}
\item {Proveniência:(Lat. \textunderscore taxare\textunderscore )}
\end{itemize}
Estabelecer a taxa ou o preço de.
Limitar.
Regular; moderar.
Fixar a quantidade de.
Avaliar.
\section{Taxativo}
\begin{itemize}
\item {Grp. gram.:adj.}
\end{itemize}
Que taxa; limitativo; restricto: \textunderscore preceitos taxativos\textunderscore .
\section{Taxe}
\begin{itemize}
\item {fónica:cse}
\end{itemize}
\begin{itemize}
\item {Grp. gram.:f.}
\end{itemize}
O mesmo ou melhór que \textunderscore táxis\textunderscore .
\section{Taxi}
\begin{itemize}
\item {Grp. gram.:f.}
\end{itemize}
\begin{itemize}
\item {Utilização:Bras}
\end{itemize}
Formiga vermelha, cuja mordedura queima.
\section{Taxia}
\begin{itemize}
\item {fónica:csi}
\end{itemize}
\begin{itemize}
\item {Grp. gram.:f.}
\end{itemize}
\begin{itemize}
\item {Utilização:Med.}
\end{itemize}
\begin{itemize}
\item {Proveniência:(Do gr. \textunderscore taxis\textunderscore )}
\end{itemize}
Influência attractiva ou repulsiva, exercida por certas substâncias ou certos phenómenos sôbre o protoplasma.
\section{Taxiarca}
\begin{itemize}
\item {fónica:csi}
\end{itemize}
\begin{itemize}
\item {Grp. gram.:m.}
\end{itemize}
\begin{itemize}
\item {Proveniência:(Do gr. \textunderscore taxis\textunderscore  + \textunderscore arkhein\textunderscore )}
\end{itemize}
Antigo oficial ateniense, que commandava a infantaria da sua tríbo.
\section{Taxiarcha}
\begin{itemize}
\item {fónica:csi}
\end{itemize}
\begin{itemize}
\item {Grp. gram.:m.}
\end{itemize}
\begin{itemize}
\item {Proveniência:(Do gr. \textunderscore taxis\textunderscore  + \textunderscore arkhein\textunderscore )}
\end{itemize}
Antigo official atheniense, que commandava a infantaria da sua tríbo.
\section{Taxiarquia}
\begin{itemize}
\item {fónica:csi}
\end{itemize}
\begin{itemize}
\item {Grp. gram.:f.}
\end{itemize}
Subdivisão da infantaria grega, constando de 128 homens.
(Cp. \textunderscore taxiarca\textunderscore )
\section{Taxíneas}
\begin{itemize}
\item {fónica:csi}
\end{itemize}
\begin{itemize}
\item {Grp. gram.:f. pl.}
\end{itemize}
Família de plantas coníferas, que tem por typo o teixo.
(Fem. pl. de \textunderscore taxíneo\textunderscore )
\section{Taxíneo}
\begin{itemize}
\item {fónica:csi}
\end{itemize}
\begin{itemize}
\item {Grp. gram.:adj.}
\end{itemize}
\begin{itemize}
\item {Proveniência:(Do lat. \textunderscore taxus\textunderscore )}
\end{itemize}
Relativo ou semelhante ao teixo.
\section{Taxinomia}
\begin{itemize}
\item {fónica:csi}
\end{itemize}
\textunderscore f.\textunderscore  (e der.)
Fórma preferível a \textunderscore taxonomia\textunderscore , etc.
\section{Táxis}
\begin{itemize}
\item {fónica:csi}
\end{itemize}
\begin{itemize}
\item {Grp. gram.:f.}
\end{itemize}
\begin{itemize}
\item {Utilização:Med.}
\end{itemize}
\begin{itemize}
\item {Proveniência:(Do gr. \textunderscore taxis\textunderscore )}
\end{itemize}
Pressão methódica, que se exerce com a mão sôbre um tumor herniário, para o reduzir.
\section{Taxódio}
\begin{itemize}
\item {fónica:csó}
\end{itemize}
\begin{itemize}
\item {Grp. gram.:m.}
\end{itemize}
\begin{itemize}
\item {Proveniência:(Do lat. \textunderscore taxus\textunderscore  + gr. \textunderscore eidos\textunderscore )}
\end{itemize}
Gênero de plantas coníferas.
\section{Taxologia}
\begin{itemize}
\item {fónica:cso}
\end{itemize}
\begin{itemize}
\item {Grp. gram.:f.}
\end{itemize}
\begin{itemize}
\item {Proveniência:(Do gr. \textunderscore taxis\textunderscore  + \textunderscore logos\textunderscore )}
\end{itemize}
Tratado ou sciência das classificações.--Melhór se dirá \textunderscore taxilogia\textunderscore .
\section{Taxológico}
\begin{itemize}
\item {fónica:cso}
\end{itemize}
\begin{itemize}
\item {Grp. gram.:adj.}
\end{itemize}
Relativo á taxologia.
\section{Taxólogo}
\begin{itemize}
\item {fónica:csó}
\end{itemize}
\begin{itemize}
\item {Grp. gram.:m.}
\end{itemize}
Autor de uma classificação ou de um tratado de classificações.
(Cp. \textunderscore taxologia\textunderscore )
\section{Taxómetro}
\begin{itemize}
\item {Grp. gram.:m.}
\end{itemize}
(V.taxímetro)
\section{Taxonomia}
\begin{itemize}
\item {fónica:cso}
\end{itemize}
\begin{itemize}
\item {Grp. gram.:f.}
\end{itemize}
\begin{itemize}
\item {Proveniência:(Do gr. \textunderscore taxis\textunderscore  + \textunderscore nomos\textunderscore )}
\end{itemize}
Classificação scientífica.
Parte da Botânica, que trata da classificação das Plantas.
Parte da Grammática, que trata da classificação das palavras.--Melhór se dirá \textunderscore taxinomia\textunderscore .
\section{Taxonómico}
\begin{itemize}
\item {fónica:cso}
\end{itemize}
\begin{itemize}
\item {Grp. gram.:adj.}
\end{itemize}
\begin{itemize}
\item {Grp. gram.:M.}
\end{itemize}
Relativo á taxonomia.
Aquelle que trata de taxonomia.
\section{Taz}
\begin{itemize}
\item {Grp. gram.:f.}
\end{itemize}
\begin{itemize}
\item {Proveniência:(Fr. \textunderscore tas\textunderscore )}
\end{itemize}
Pequena bigorna de aço, sem hastes usada nos estabelecimentos de cunhagem da moéda o nas ferradorias. Cf. F. de Mendonça. \textunderscore Vocab. Techn.\textunderscore 
\section{Tazela}
\begin{itemize}
\item {Grp. gram.:f.}
\end{itemize}
\begin{itemize}
\item {Utilização:Prov.}
\end{itemize}
\begin{itemize}
\item {Utilização:alg.}
\end{itemize}
O mesmo que \textunderscore ladeira\textunderscore .
\section{Te}
\begin{itemize}
\item {Grp. gram.:pron.}
\end{itemize}
\begin{itemize}
\item {Proveniência:(Lat. \textunderscore te\textunderscore )}
\end{itemize}
A ti.--É um dos casos do pron. \textunderscore tu\textunderscore , e emprega-se geralmente como complemento directo: \textunderscore amo-te, Maria\textunderscore ; e como terminativo: \textunderscore digo-te que mentes\textunderscore .
\section{Té}
\begin{itemize}
\item {Grp. gram.:prep.}
\end{itemize}
(Aphér. de \textunderscore até\textunderscore )
\section{Téa}
\begin{itemize}
\item {Grp. gram.:f.}
\end{itemize}
(V. \textunderscore teia\textunderscore ^2)
\section{Teada}
\begin{itemize}
\item {Grp. gram.:f.}
\end{itemize}
Teia de pano.
\section{Teagem}
\begin{itemize}
\item {Grp. gram.:f.}
\end{itemize}
\begin{itemize}
\item {Proveniência:(De \textunderscore teia\textunderscore )}
\end{itemize}
O mesmo que \textunderscore teada\textunderscore .
Membrana cellular reticular.
\section{Tear}
\begin{itemize}
\item {Grp. gram.:m.}
\end{itemize}
\begin{itemize}
\item {Proveniência:(De \textunderscore teia\textunderscore )}
\end{itemize}
Apparelho para tecer pano.
Instrumento, com que os encadernadores cosem os livros.
Conjunto das rodas de um relógio.
\section{Tebas}
\begin{itemize}
\item {Grp. gram.:m.}
\end{itemize}
\begin{itemize}
\item {Utilização:Bras}
\end{itemize}
Valentão.
\section{Tebele}
\begin{itemize}
\item {Grp. gram.:m.}
\end{itemize}
Uma das línguas de Angola.
\section{Teca}
\begin{itemize}
\item {Grp. gram.:f.}
\end{itemize}
\begin{itemize}
\item {Utilização:Pop.}
\end{itemize}
Árvore verbenácea da Ásia, (\textunderscore tectona grandis\textunderscore )
Árvore leguminosa do Brasil.
Dinheiro.
(Do malaiala \textunderscore tekka\textunderscore )
\section{Tecale}
\begin{itemize}
\item {Grp. gram.:m.}
\end{itemize}
Qualidade de mármore mexicano, branco e transparente, de que se faz uma espécie de vidros para janelas.
\section{Tecáli}
\begin{itemize}
\item {Grp. gram.:m.}
\end{itemize}
Qualidade de mármore mexicano, branco e transparente, de que se faz uma espécie de vidros para janelas.
\section{Tecedeira}
\begin{itemize}
\item {Grp. gram.:f.}
\end{itemize}
Mulhér, que tece pano.
\section{Tecedor}
\begin{itemize}
\item {Grp. gram.:m.  e  adj.}
\end{itemize}
\begin{itemize}
\item {Utilização:Fig.}
\end{itemize}
\begin{itemize}
\item {Proveniência:(De \textunderscore tecer\textunderscore )}
\end{itemize}
O que tece pano; o que é tecelão.
Intriguista.
\section{Tecedura}
\begin{itemize}
\item {Grp. gram.:f.}
\end{itemize}
\begin{itemize}
\item {Utilização:Fig.}
\end{itemize}
Acto ou effeito de tecer.
Conjunto dos fios, que se cruzam com a urdidura.
Enrêdo, intriga.
\section{Tecelagem}
\begin{itemize}
\item {Grp. gram.:f.}
\end{itemize}
\begin{itemize}
\item {Proveniência:(De \textunderscore tecelão\textunderscore )}
\end{itemize}
Tecedura; offício de tecelão.
\section{Tecelão}
\begin{itemize}
\item {Grp. gram.:m.}
\end{itemize}
\begin{itemize}
\item {Proveniência:(De \textunderscore tecer\textunderscore )}
\end{itemize}
Aquelle que tece pano ou trabalha em teares.
Passarinho conirostro, que, dentro da gaiola, e talvez fóra, entretece linhas ou fibras, que tenha ao seu alcance, produzindo artefactos caprichosos.--Creio que é originário da África, onde também lhe chamam \textunderscore alfaiate\textunderscore .
\section{Tecelinho}
\begin{itemize}
\item {Grp. gram.:m.}
\end{itemize}
O mesmo que \textunderscore maria-fia\textunderscore .
(Dem. de \textunderscore tecelão\textunderscore )
\section{Tecelôa}
\begin{itemize}
\item {Grp. gram.:f.}
\end{itemize}
\begin{itemize}
\item {Proveniência:(De \textunderscore tecelão\textunderscore )}
\end{itemize}
O mesmo que \textunderscore tecedeira\textunderscore .
Mulhér de tecelão.
\section{Tecer}
\begin{itemize}
\item {Grp. gram.:v. t.}
\end{itemize}
\begin{itemize}
\item {Utilização:Fig.}
\end{itemize}
\begin{itemize}
\item {Grp. gram.:V. i.}
\end{itemize}
\begin{itemize}
\item {Utilização:Fam.}
\end{itemize}
\begin{itemize}
\item {Utilização:Des.}
\end{itemize}
\begin{itemize}
\item {Grp. gram.:V. p.}
\end{itemize}
\begin{itemize}
\item {Utilização:Fig.}
\end{itemize}
\begin{itemize}
\item {Proveniência:(Do lat. \textunderscore texere\textunderscore )}
\end{itemize}
Fazer (teia), tramando fios com fios; tramar.
Entrelaçar.
Preparar: \textunderscore tecer intrigas\textunderscore .
Coordenar.
Mesclar.
Adornar.
Fazer teias.
Exercer o offício de tecelão.
Mexer os braços e as pernas automaticamente, (falando-se de crianças de leite)
Fazer mexericos ou intrigas.
Perpassar, cruzando-se:«\textunderscore ...animaes que, saltando, tecião huns pelos outros.\textunderscore »\textunderscore Peregrinação\textunderscore , LXXIII.
Entrelaçar-se; enredar-se.
Preparar-se; engendrar-se.--A fórma exacta sería \textunderscore tesser\textunderscore , mas nunca se usou.
\section{Téchnica}
\begin{itemize}
\item {Grp. gram.:f.}
\end{itemize}
Conjunto dos processos de uma arte ou de uma fabricação.
(Fem. de \textunderscore téchnico\textunderscore )
\section{Technicamente}
\begin{itemize}
\item {Grp. gram.:adv.}
\end{itemize}
De modo téchnico.
Segundo a téchnica.
\section{Technicismo}
\begin{itemize}
\item {Grp. gram.:m.}
\end{itemize}
Qualidade do que é téchnico.
\section{Téchnico}
\begin{itemize}
\item {Grp. gram.:adj.}
\end{itemize}
\begin{itemize}
\item {Utilização:Ext.}
\end{itemize}
\begin{itemize}
\item {Grp. gram.:M.  e  adj.}
\end{itemize}
\begin{itemize}
\item {Proveniência:(Lat. \textunderscore technicus\textunderscore )}
\end{itemize}
Próprio de uma arte: \textunderscore conhecimentos téchnicos\textunderscore .
Relativo a uma sciência.
O que é perito numa arte ou sciência.
\section{Technismo}
\begin{itemize}
\item {Grp. gram.:m.}
\end{itemize}
\begin{itemize}
\item {Proveniência:(Do gr. \textunderscore tekhne\textunderscore )}
\end{itemize}
Influência das artes. Cf. Latino, \textunderscore Camões\textunderscore , 263 e 265.
\section{Technita}
\begin{itemize}
\item {Grp. gram.:m.}
\end{itemize}
\begin{itemize}
\item {Proveniência:(Do gr. \textunderscore tekhnites\textunderscore )}
\end{itemize}
Gênero de insectos coleópteros tetrâmeros.
\section{Technographia}
\begin{itemize}
\item {Grp. gram.:f.}
\end{itemize}
\begin{itemize}
\item {Proveniência:(Do gr. \textunderscore tekhne\textunderscore  + \textunderscore graphein\textunderscore )}
\end{itemize}
Descripção das artes e dos seus processos.
\section{Technográphico}
\begin{itemize}
\item {Grp. gram.:adj.}
\end{itemize}
Relativo á technographia.
\section{Technologia}
\begin{itemize}
\item {Grp. gram.:f.}
\end{itemize}
\begin{itemize}
\item {Proveniência:(De \textunderscore technólogo\textunderscore )}
\end{itemize}
Tratado das artes em geral.
Explicação dos termos peculiares ás artes e offícios.
Linguagem privativa das sciências, artes e indústrias: \textunderscore technologia rural\textunderscore .
\section{Technológico}
\begin{itemize}
\item {Grp. gram.:adj.}
\end{itemize}
\begin{itemize}
\item {Proveniência:(Gr. \textunderscore tekhnologikos\textunderscore )}
\end{itemize}
Relativo á technologia.
\section{Technólogo}
\begin{itemize}
\item {Grp. gram.:m.}
\end{itemize}
\begin{itemize}
\item {Proveniência:(Do gr. \textunderscore tekhne\textunderscore  + \textunderscore logos\textunderscore )}
\end{itemize}
Aquelle que escreve á cêrca de artes e offícios, ou que é perito em technologia.
\section{Tecido}
\begin{itemize}
\item {Grp. gram.:adj.}
\end{itemize}
\begin{itemize}
\item {Utilização:Fig.}
\end{itemize}
\begin{itemize}
\item {Grp. gram.:M.}
\end{itemize}
\begin{itemize}
\item {Utilização:Fig.}
\end{itemize}
Feito no tear.
Preparado, urdido.
Apropriado.
Obra feita no tear.
Pano, estôfo.
Trama.
Parte sólida dos corpos organizados.
Conjunto.
Disposição.
Cerração, negrume.
\section{Tecimento}
\begin{itemize}
\item {Grp. gram.:m.}
\end{itemize}
O mesmo que \textunderscore tecedura\textunderscore .
Enrêdo, mexerico.
\section{Tecla}
\begin{itemize}
\item {Grp. gram.:f.}
\end{itemize}
\begin{itemize}
\item {Utilização:Fig.}
\end{itemize}
\begin{itemize}
\item {Proveniência:(Do lat. \textunderscore tegula\textunderscore )}
\end{itemize}
Peça de marfim ou de outra substância, que, sob a pressão dos dedos, faz soar o piano ou outros instrumentos.
Opportunidade.
Corda sensível: \textunderscore tocaram-lhe na tecla\textunderscore .
Assumpto, que se trata depois de outro: \textunderscore passemos a outra tecla\textunderscore .
\section{Teclado}
\begin{itemize}
\item {Grp. gram.:m.}
\end{itemize}
Conjunto de teclas de um instrumento.
\section{Técnica}
\begin{itemize}
\item {Grp. gram.:f.}
\end{itemize}
Conjunto dos processos de uma arte ou de uma fabricação.
(Fem. de \textunderscore técnico\textunderscore )
\section{Tecnicamente}
\begin{itemize}
\item {Grp. gram.:adv.}
\end{itemize}
De modo técnico.
Segundo a técnica.
\section{Tecnicismo}
\begin{itemize}
\item {Grp. gram.:m.}
\end{itemize}
Qualidade do que é técnico.
\section{Técnico}
\begin{itemize}
\item {Grp. gram.:adj.}
\end{itemize}
\begin{itemize}
\item {Utilização:Ext.}
\end{itemize}
\begin{itemize}
\item {Grp. gram.:M.  e  adj.}
\end{itemize}
\begin{itemize}
\item {Proveniência:(Lat. \textunderscore technicus\textunderscore )}
\end{itemize}
Próprio de uma arte: \textunderscore conhecimentos técnicos\textunderscore .
Relativo a uma ciência.
O que é perito numa arte ou ciência.
\section{Tecnismo}
\begin{itemize}
\item {Grp. gram.:m.}
\end{itemize}
\begin{itemize}
\item {Proveniência:(Do gr. \textunderscore tekhne\textunderscore )}
\end{itemize}
Influência das artes. Cf. Latino, \textunderscore Camões\textunderscore , 263 e 265.
\section{Tecnita}
\begin{itemize}
\item {Grp. gram.:m.}
\end{itemize}
\begin{itemize}
\item {Proveniência:(Do gr. \textunderscore tekhnites\textunderscore )}
\end{itemize}
Gênero de insectos coleópteros tetrâmeros.
\section{Tecnografia}
\begin{itemize}
\item {Grp. gram.:f.}
\end{itemize}
\begin{itemize}
\item {Proveniência:(Do gr. \textunderscore tekhne\textunderscore  + \textunderscore graphein\textunderscore )}
\end{itemize}
Descripção das artes e dos seus processos.
\section{Tecnográfico}
\begin{itemize}
\item {Grp. gram.:adj.}
\end{itemize}
Relativo á tecnografia.
\section{Tecnologia}
\begin{itemize}
\item {Grp. gram.:f.}
\end{itemize}
\begin{itemize}
\item {Proveniência:(De \textunderscore tecnólogo\textunderscore )}
\end{itemize}
Tratado das artes em geral.
Explicação dos termos peculiares ás artes e ofícios.
Linguagem privativa das sciências, artes e indústrias: \textunderscore tecnologia rural\textunderscore .
\section{Tecnológico}
\begin{itemize}
\item {Grp. gram.:adj.}
\end{itemize}
\begin{itemize}
\item {Proveniência:(Gr. \textunderscore tekhnologikos\textunderscore )}
\end{itemize}
Relativo á tecnologia.
\section{Tecnólogo}
\begin{itemize}
\item {Grp. gram.:m.}
\end{itemize}
\begin{itemize}
\item {Proveniência:(Do gr. \textunderscore tekhne\textunderscore  + \textunderscore logos\textunderscore )}
\end{itemize}
Aquele que escreve á cêrca de artes e ofícios, ou que é perito em tecnologia.
\section{Tecó}
\begin{itemize}
\item {Grp. gram.:adv.}
\end{itemize}
\begin{itemize}
\item {Utilização:Bras. do N}
\end{itemize}
Da mesma sorte; na fórma do costume.
\section{Tecoma}
\begin{itemize}
\item {Grp. gram.:f.}
\end{itemize}
\begin{itemize}
\item {Proveniência:(Fr. \textunderscore tecome\textunderscore )}
\end{itemize}
Gênero de plantas bignoniáceas.
\section{Tectipenas}
\begin{itemize}
\item {Grp. gram.:m. pl.}
\end{itemize}
\begin{itemize}
\item {Proveniência:(Do lat. \textunderscore tectum\textunderscore  + \textunderscore penna\textunderscore )}
\end{itemize}
Insectos, de asas reticuladas e bôca saliente.
\section{Tectipenes}
\begin{itemize}
\item {Grp. gram.:m. pl.}
\end{itemize}
\begin{itemize}
\item {Proveniência:(Do lat. \textunderscore tectum\textunderscore  + \textunderscore penna\textunderscore )}
\end{itemize}
Insectos, de asas reticuladas e bôca saliente.
\section{Tectipennas}
\begin{itemize}
\item {Grp. gram.:m. pl.}
\end{itemize}
\begin{itemize}
\item {Proveniência:(Do lat. \textunderscore tectum\textunderscore  + \textunderscore penna\textunderscore )}
\end{itemize}
Insectos, de asas reticuladas e bôca saliente.
\section{Tectipennes}
\begin{itemize}
\item {Grp. gram.:m. pl.}
\end{itemize}
\begin{itemize}
\item {Proveniência:(Do lat. \textunderscore tectum\textunderscore  + \textunderscore penna\textunderscore )}
\end{itemize}
Insectos, de asas reticuladas e bôca saliente.
\section{Tecto}
\begin{itemize}
\item {Grp. gram.:m.}
\end{itemize}
\begin{itemize}
\item {Utilização:Ext.}
\end{itemize}
\begin{itemize}
\item {Utilização:Pop.}
\end{itemize}
\begin{itemize}
\item {Utilização:Geol.}
\end{itemize}
\begin{itemize}
\item {Proveniência:(Lat. \textunderscore tectum\textunderscore )}
\end{itemize}
Parte superior de uma casa ou de um edifício, considerada especialmente do lado interior.
Cobertura de uma sala ou de outro compartimento, em qualquer andar de um edifício.
Cobertura.
Abrigo, habitação.
Cabeça, juízo.
Diz-se o extrato geológico, em relação a outro, sôbre que assenta.
\section{Tectona}
\begin{itemize}
\item {Grp. gram.:f.}
\end{itemize}
\begin{itemize}
\item {Proveniência:(Do gr. \textunderscore tekton\textunderscore )}
\end{itemize}
Gênero de plantas verbenáceas, da tríbo das lantâneas.
\section{Tectónica}
\begin{itemize}
\item {Grp. gram.:f.}
\end{itemize}
\begin{itemize}
\item {Proveniência:(Do gr. \textunderscore tekton\textunderscore )}
\end{itemize}
Arte de carpinteiro.
Arte de construir edifícios; Architectura. Cf. Latino, \textunderscore Humboldt\textunderscore , 523.
\section{Tectónico}
\begin{itemize}
\item {Grp. gram.:adj.}
\end{itemize}
Relativo a edifícios ou á Architectura.
(Cp. \textunderscore tectónica\textunderscore )
\section{Tectriz}
\begin{itemize}
\item {Grp. gram.:adj. f.}
\end{itemize}
\begin{itemize}
\item {Utilização:Anat.}
\end{itemize}
\begin{itemize}
\item {Utilização:Zool.}
\end{itemize}
Diz-se das lâminas, que constituem a parte posterior do osso frontal.
Diz-se das pennas, que cobrem as asas e a cauda das aves.
(Cp. lat. \textunderscore tector\textunderscore )
\section{Tecum}
\begin{itemize}
\item {Grp. gram.:m.}
\end{itemize}
Fibra têxtil, que se extrái da tecuma.
\section{Tecuma}
\begin{itemize}
\item {Grp. gram.:f.}
\end{itemize}
Espécie de palmeira.
\section{Teçume}
\begin{itemize}
\item {Grp. gram.:m.}
\end{itemize}
\begin{itemize}
\item {Utilização:Prov.}
\end{itemize}
\begin{itemize}
\item {Utilização:minh.}
\end{itemize}
\begin{itemize}
\item {Proveniência:(De \textunderscore tecer\textunderscore )}
\end{itemize}
Tecido, que reveste o urdume.
\section{Tecunas}
\begin{itemize}
\item {Grp. gram.:m. pl.}
\end{itemize}
Indígenas do norte do Brasil.--Provavelmente, o mesmo que \textunderscore tacunás\textunderscore , se uma das fórmas não apparece errada nos ethnógraphos brasileiros. Cf. Araujo e Amazonas, \textunderscore Diccion. Topogr.\textunderscore 
\section{Teda}
\begin{itemize}
\item {fónica:tê}
\end{itemize}
\begin{itemize}
\item {Grp. gram.:f.}
\end{itemize}
\begin{itemize}
\item {Utilização:Poét.}
\end{itemize}
\begin{itemize}
\item {Proveniência:(Lat. \textunderscore taeda\textunderscore )}
\end{itemize}
O mesmo que \textunderscore archote\textunderscore . Cf. Quevedo, \textunderscore Aff. Africano\textunderscore , 68; Castilho, \textunderscore Metam.\textunderscore , 141.
\section{Tedesco}
\begin{itemize}
\item {fónica:dês}
\end{itemize}
\begin{itemize}
\item {Grp. gram.:m.  e  adj.}
\end{itemize}
(V.tudesco)Cf. Latino, \textunderscore Humboldt\textunderscore , 344; \textunderscore Elogios\textunderscore , 90.
\section{Tedífero}
\begin{itemize}
\item {Grp. gram.:adj.}
\end{itemize}
\begin{itemize}
\item {Utilização:Poét.}
\end{itemize}
Que leva tocha.
(Lat.\textunderscore  taedifer\textunderscore )
\section{Tédio}
\begin{itemize}
\item {Grp. gram.:m.}
\end{itemize}
\begin{itemize}
\item {Proveniência:(Lat. \textunderscore taedium\textunderscore )}
\end{itemize}
Aborrecimento; enfado; desgôsto.
\section{Tedioso}
\begin{itemize}
\item {Grp. gram.:adj.}
\end{itemize}
\begin{itemize}
\item {Proveniência:(Lat. \textunderscore taediosus\textunderscore )}
\end{itemize}
Que tem tédio; em que há tédio: \textunderscore horas tediosas\textunderscore .
Que produz tédio:«\textunderscore o meu tedioso boticário\textunderscore ». Camillo, \textunderscore Hist. e Sentiment.\textunderscore , 127.
\section{Tedo}
\begin{itemize}
\item {Grp. gram.:adj.}
\end{itemize}
\begin{itemize}
\item {Utilização:Ant.}
\end{itemize}
Obrigado.
(Por \textunderscore teúdo\textunderscore ?)
\section{Tèdor}
\begin{itemize}
\item {Grp. gram.:m.}
\end{itemize}
O mesmo que \textunderscore teedor\textunderscore .
\section{Teédia}
\begin{itemize}
\item {Grp. gram.:f.}
\end{itemize}
\begin{itemize}
\item {Proveniência:(De \textunderscore Teed\textunderscore , n. p.)}
\end{itemize}
Gênero de plantas escrofularíneas.
\section{Teedor}
\begin{itemize}
\item {Grp. gram.:m.}
\end{itemize}
\begin{itemize}
\item {Utilização:Ant.}
\end{itemize}
Senhor, possuidor.
(Contr. de \textunderscore tenedor\textunderscore , do lat. \textunderscore tenere\textunderscore )
\section{Teer}
\begin{itemize}
\item {Grp. gram.:v. t.}
\end{itemize}
\begin{itemize}
\item {Utilização:Ant.}
\end{itemize}
O mesmo que \textunderscore têr\textunderscore .
\section{Tefe}
\begin{itemize}
\item {Grp. gram.:m.}
\end{itemize}
\begin{itemize}
\item {Utilização:Gír.}
\end{itemize}
Ânus.
Partes pudendas da mulhér.
\section{Tefe-tefe}
\begin{itemize}
\item {Grp. gram.:m.}
\end{itemize}
\begin{itemize}
\item {Utilização:Pop.}
\end{itemize}
\begin{itemize}
\item {Utilização:Burl.}
\end{itemize}
\begin{itemize}
\item {Proveniência:(Do ar. \textunderscore tafe-tafe\textunderscore )}
\end{itemize}
O pulsar do coração.
Paixão amorosa.
\section{Tègão}
\begin{itemize}
\item {Grp. gram.:m.}
\end{itemize}
(V.tremonha)
\section{Tegbos}
\begin{itemize}
\item {Grp. gram.:m. pl.}
\end{itemize}
Tríbo da África Central.
\section{Tegelaó}
\textunderscore f.\textunderscore  (e der.)
O mesmo ou melhór que \textunderscore tigela\textunderscore , etc.
(Cp. lat. \textunderscore tegula\textunderscore )
\section{Tegéremo}
\begin{itemize}
\item {Grp. gram.:adj.}
\end{itemize}
\begin{itemize}
\item {Utilização:Ant.}
\end{itemize}
O mesmo que \textunderscore trigésimo\textunderscore .
\section{Tegme}
\begin{itemize}
\item {Grp. gram.:m.}
\end{itemize}
\begin{itemize}
\item {Proveniência:(Lat. \textunderscore tegmen\textunderscore )}
\end{itemize}
Membrana interna da semente.
\section{Tégmen}
\begin{itemize}
\item {Grp. gram.:m.}
\end{itemize}
\begin{itemize}
\item {Utilização:Bot.}
\end{itemize}
\begin{itemize}
\item {Proveniência:(Lat. \textunderscore tegmen\textunderscore )}
\end{itemize}
Membrana interna da semente.
\section{Tás}
\begin{itemize}
\item {Grp. gram.:f.}
\end{itemize}
\begin{itemize}
\item {Proveniência:(Fr. \textunderscore tas\textunderscore )}
\end{itemize}
Pequena bigorna de aço, sem hastes usada nos estabelecimentos de cunhagem da moéda o nas ferradorias. Cf. F. de Mendonça. \textunderscore Vocab. Techn.\textunderscore 
\section{Taxiarcado}
\begin{itemize}
\item {fónica:csi}
\end{itemize}
\begin{itemize}
\item {Grp. gram.:m.}
\end{itemize}
Pôsto ou dignidade de taxiarca.
\section{Taxiarchado}
\begin{itemize}
\item {fónica:csi}
\end{itemize}
\begin{itemize}
\item {Grp. gram.:m.}
\end{itemize}
Pôsto ou dignidade de taxiarcha.
\section{Taxiarchia}
\begin{itemize}
\item {fónica:csi,qui}
\end{itemize}
\begin{itemize}
\item {Grp. gram.:f.}
\end{itemize}
Subdivisão da infantaria grega, constando de 128 homens.
(Cp. \textunderscore taxiarcha\textunderscore )
\section{Taxícola}
\begin{itemize}
\item {fónica:csi}
\end{itemize}
\begin{itemize}
\item {Grp. gram.:adj.}
\end{itemize}
\begin{itemize}
\item {Proveniência:(Do lat. \textunderscore taxus\textunderscore  + \textunderscore colere\textunderscore )}
\end{itemize}
Que vive como parasito nos teixos.
\section{Taxicórneos}
\begin{itemize}
\item {fónica:csi}
\end{itemize}
\begin{itemize}
\item {Grp. gram.:m. pl.}
\end{itemize}
Família de insectos coleópteros heterómeros.
\section{Taxidermia}
\begin{itemize}
\item {fónica:csi}
\end{itemize}
\begin{itemize}
\item {Grp. gram.:f.}
\end{itemize}
\begin{itemize}
\item {Proveniência:(Do gr. \textunderscore taxis\textunderscore  + \textunderscore derma\textunderscore )}
\end{itemize}
Arte de empalhar animaes.
\section{Taxidérmico}
\begin{itemize}
\item {fónica:csi}
\end{itemize}
\begin{itemize}
\item {Grp. gram.:adj.}
\end{itemize}
Relativo á taxidermia.
\section{Taxiforme}
\begin{itemize}
\item {fónica:csi}
\end{itemize}
\begin{itemize}
\item {Grp. gram.:adj.}
\end{itemize}
\begin{itemize}
\item {Utilização:Bot.}
\end{itemize}
\begin{itemize}
\item {Proveniência:(Do lat. \textunderscore taxus\textunderscore  + \textunderscore forma\textunderscore )}
\end{itemize}
Diz-se da planta, cujas fôlhas têm quási a mesma disposição que as do teixo.
\section{Taxiladar}
\begin{itemize}
\item {Grp. gram.:m.}
\end{itemize}
Espécie de cabo de polícia da Índia portuguesa. Cf. L. Mendes, \textunderscore Ind. Port.\textunderscore 
(Do concani)
\section{Taximetria}
\begin{itemize}
\item {fónica:csi}
\end{itemize}
\begin{itemize}
\item {Grp. gram.:f.}
\end{itemize}
Applicação do taxímetro.
\section{Taxímetro}
\begin{itemize}
\item {fónica:csi}
\end{itemize}
\begin{itemize}
\item {Grp. gram.:m.}
\end{itemize}
\begin{itemize}
\item {Utilização:Ext.}
\end{itemize}
\begin{itemize}
\item {Proveniência:(Do gr. \textunderscore taxis\textunderscore  + \textunderscore metron\textunderscore )}
\end{itemize}
Apparelho, para medir a distância percorrida por um vehículo.
Vehículo, que tem taxímetro.
\section{Tego}
\begin{itemize}
\item {fónica:tê}
\end{itemize}
\begin{itemize}
\item {Grp. gram.:m.}
\end{itemize}
\begin{itemize}
\item {Utilização:Gír.}
\end{itemize}
O mesmo que \textunderscore padre\textunderscore .
\section{Tégula}
\begin{itemize}
\item {Grp. gram.:f.}
\end{itemize}
\begin{itemize}
\item {Proveniência:(Lat. \textunderscore tegula\textunderscore )}
\end{itemize}
Gênero de molluscos gasterópodes.
\section{Tegumentar}
\begin{itemize}
\item {Grp. gram.:adj.}
\end{itemize}
O mesmo que \textunderscore tegumentário\textunderscore .
\section{Tegumentário}
\begin{itemize}
\item {Grp. gram.:adj.}
\end{itemize}
Relativo a tegumento.
\section{Tegumento}
\begin{itemize}
\item {Grp. gram.:m.}
\end{itemize}
\begin{itemize}
\item {Utilização:Bot.}
\end{itemize}
\begin{itemize}
\item {Proveniência:(Lat. \textunderscore tegumentum\textunderscore )}
\end{itemize}
Parte externa, tudo que serve para cobrir ou revestir.
Invólucro de uma semente.
Cálice das plantas.
Corolla.
\section{Teia}
\begin{itemize}
\item {Grp. gram.:f.}
\end{itemize}
\begin{itemize}
\item {Utilização:Fig.}
\end{itemize}
\begin{itemize}
\item {Utilização:Ant.}
\end{itemize}
\begin{itemize}
\item {Proveniência:(Do lat. \textunderscore tela\textunderscore )}
\end{itemize}
Tecido de linho, algodão, cânhamo, etc.
Trama.
Organismo.
Intriga.
Entrecho.
Série.
Circo.
Gradeamento, que constitue uma divisória no pavimento de algumas igrejas, dos tribunaes, etc.
Divisão ou estrema territorial.
Espécie de cotão, que se fórma em volta de algumas uvas e botões de certas plantas.
Rêde, que as aranhas tecem, para colher insectos ou outras prêsas.
\section{Teia}
\begin{itemize}
\item {Grp. gram.:f.}
\end{itemize}
\begin{itemize}
\item {Utilização:Poét.}
\end{itemize}
\begin{itemize}
\item {Proveniência:(Do lat. \textunderscore taeda\textunderscore )}
\end{itemize}
Facho, tocha, archote, teda. Cf. C. da Ericeira, \textunderscore Henriqueida\textunderscore , VI, 36.
\section{Teifol}
\begin{itemize}
\item {Grp. gram.:m.}
\end{itemize}
Árvore da Índia Portuguesa.
\section{Teiga}
\begin{itemize}
\item {Grp. gram.:f.}
\end{itemize}
\begin{itemize}
\item {Utilização:T. de Alcanena}
\end{itemize}
Espécie de cesto.
Antiga medida para cereaes.
Açafate de costura.
(Contr. de \textunderscore taleiga\textunderscore )
\section{Teigo}
\begin{itemize}
\item {Grp. gram.:m.}
\end{itemize}
\begin{itemize}
\item {Utilização:T. de Serpa}
\end{itemize}
Prato do barro, em que se costuma comer o caspacho. Cf. Rev. \textunderscore Tradição\textunderscore , II, 11.
\section{Têigula}
\begin{itemize}
\item {Grp. gram.:f.}
\end{itemize}
\begin{itemize}
\item {Utilização:Ant.}
\end{itemize}
Pequena teiga.
\section{Teijão}
\begin{itemize}
\item {Grp. gram.:m.}
\end{itemize}
Ave do Cabo da Boa-Esperança.
\section{Teima}
\begin{itemize}
\item {Grp. gram.:f.}
\end{itemize}
\begin{itemize}
\item {Proveniência:(Do cast. \textunderscore tema\textunderscore )}
\end{itemize}
Acto ou effeito de teimar; teimosia; insistência.
\section{Teimar}
\begin{itemize}
\item {Grp. gram.:v. i.}
\end{itemize}
\begin{itemize}
\item {Grp. gram.:V. t.}
\end{itemize}
\begin{itemize}
\item {Proveniência:(De \textunderscore teima\textunderscore )}
\end{itemize}
Obstinar-se; insistir.
Sarrazinar.
Insistir em.
Pretender com insistência:«\textunderscore ...a sobrinha teimava não enjeitar o filho...\textunderscore »Camillo, \textunderscore Volcoens\textunderscore , 90.
\section{Teimice}
\begin{itemize}
\item {Grp. gram.:f.}
\end{itemize}
\begin{itemize}
\item {Utilização:Fam.}
\end{itemize}
O mesmo que \textunderscore teimosia\textunderscore .
\section{Teimosa}
\begin{itemize}
\item {Grp. gram.:f.}
\end{itemize}
\begin{itemize}
\item {Utilização:Bras. do Ceará}
\end{itemize}
\begin{itemize}
\item {Utilização:Bras}
\end{itemize}
O mesmo que \textunderscore mandureba\textunderscore .
O mesmo que \textunderscore aguardente\textunderscore .
\section{Teimosamente}
\begin{itemize}
\item {Grp. gram.:adv.}
\end{itemize}
De modo teimoso; pertinazmente.
\section{Teimosia}
\begin{itemize}
\item {Grp. gram.:f.}
\end{itemize}
Qualidade do que é teimoso; teima excessiva.
\section{Teimosice}
\begin{itemize}
\item {Grp. gram.:f.}
\end{itemize}
O mesmo que \textunderscore teimosia\textunderscore .
\section{Teimoso}
\begin{itemize}
\item {Grp. gram.:adj.}
\end{itemize}
\begin{itemize}
\item {Proveniência:(De \textunderscore teima\textunderscore )}
\end{itemize}
Que teima; persistente.
Que tem birras.
\section{Teio}
\begin{itemize}
\item {Grp. gram.:m.}
\end{itemize}
\begin{itemize}
\item {Utilização:Prov.}
\end{itemize}
\begin{itemize}
\item {Utilização:T. de Vouzela}
\end{itemize}
O mesmo que \textunderscore cerne\textunderscore .
O mesmo que \textunderscore azevinho\textunderscore .
\section{Teipoca}
\begin{itemize}
\item {Grp. gram.:f.}
\end{itemize}
Árvore apocýnea do Brasil.
\section{Teira}
\begin{itemize}
\item {Grp. gram.:f.}
\end{itemize}
Peixe acanthopterýgio.
\section{Teiró}
\begin{itemize}
\item {Grp. gram.:m.}
\end{itemize}
\begin{itemize}
\item {Utilização:Fig.}
\end{itemize}
Travéssa perpendicular que, cravada na cabeça do vessadoiro, sustenta e trespassa o temão.
Parte da fecharia de algumas armas de fogo.
Teima.
\section{Teiroga}
\begin{itemize}
\item {Grp. gram.:f.}
\end{itemize}
O mesmo que \textunderscore teiró\textunderscore .
\section{Teiroga}
\begin{itemize}
\item {Grp. gram.:f.}
\end{itemize}
\begin{itemize}
\item {Utilização:Pesc.}
\end{itemize}
O mesmo que \textunderscore ourega\textunderscore .
\section{Teité!}
\begin{itemize}
\item {Grp. gram.:interj.}
\end{itemize}
\begin{itemize}
\item {Utilização:Bras. do N}
\end{itemize}
\begin{itemize}
\item {Proveniência:(T. tupi)}
\end{itemize}
(Designativa de compadecimento)
\section{Teito}
\begin{itemize}
\item {Grp. gram.:m.}
\end{itemize}
\begin{itemize}
\item {Utilização:Ant.}
\end{itemize}
O mesmo que \textunderscore tecto\textunderscore . Cf. \textunderscore Eufrosina\textunderscore , 150.
\section{Teiu}
\begin{itemize}
\item {Grp. gram.:m.}
\end{itemize}
Planta euphorbiácea do Brasil.
Grande lagarto do Brasil.
\section{Teixe}
\begin{itemize}
\item {Grp. gram.:m.}
\end{itemize}
Berloque antigo, de oiro.
(Outra fórma de \textunderscore dixe\textunderscore )
\section{Teixo}
\begin{itemize}
\item {Grp. gram.:m.}
\end{itemize}
\begin{itemize}
\item {Proveniência:(Do lat. \textunderscore taxus\textunderscore )}
\end{itemize}
Árvore conífera.
\section{Teixugo}
\begin{itemize}
\item {Grp. gram.:m.}
\end{itemize}
(V.texugo)
\section{Tejadilho}
\begin{itemize}
\item {Grp. gram.:m.}
\end{itemize}
Tecto de vehículos.
(Cast. \textunderscore tejadillo\textunderscore )
\section{Tejo}
\begin{itemize}
\item {Grp. gram.:m.}
\end{itemize}
Nome que, no Algarve, se dá a um dos compartimentos das marinhas. Cf. \textunderscore Museu Techn.\textunderscore , 107.
\section{Tejo}
\begin{itemize}
\item {Grp. gram.:m.}
\end{itemize}
\begin{itemize}
\item {Utilização:Bras. do S}
\end{itemize}
\begin{itemize}
\item {Proveniência:(T. cast.)}
\end{itemize}
Espécie de jôgo, em que se atiram moédas de cobre sôbre uma faca fincada no chão, dentro de um pequeno quadro.
\section{Tejoila}
\begin{itemize}
\item {Grp. gram.:f.}
\end{itemize}
\begin{itemize}
\item {Utilização:Pop.}
\end{itemize}
Um dos ossos do casco do cavallo.
\section{Tejolo}
\begin{itemize}
\item {fónica:jô}
\end{itemize}
\begin{itemize}
\item {Grp. gram.:m.}
\end{itemize}
\begin{itemize}
\item {Grp. gram.:Loc.}
\end{itemize}
\begin{itemize}
\item {Utilização:pop.}
\end{itemize}
\begin{itemize}
\item {Utilização:Bras}
\end{itemize}
\begin{itemize}
\item {Utilização:pop.}
\end{itemize}
Peça de barro cozido, geralmente rectangular, e destinada a construcções.
Instrumento do ourivezaria, para vazar arruelas.
Doce de goiaba, ou de outras frutas, consistente e de fórma semelhante á dos tejolos de barro.
\textunderscore Fazer tejolo\textunderscore , estar sepultado.
\textunderscore Fazer tejolo\textunderscore , namorar.
\textunderscore Tejolo burro\textunderscore , espécie de tijolo grosseiro.
\textunderscore m.\textunderscore  (e der.)
O mesmo ou melhór que \textunderscore tijolo\textunderscore , etc.
(Cp. \textunderscore tégula\textunderscore )
\section{Teju}
\begin{itemize}
\item {Grp. gram.:m.}
\end{itemize}
\begin{itemize}
\item {Utilização:Bras}
\end{itemize}
O mesmo que \textunderscore teiu\textunderscore .
\section{Teju-açu}
\begin{itemize}
\item {Grp. gram.:m.}
\end{itemize}
Grande lagarto do Brasil.
\section{Tejuco}
\begin{itemize}
\item {Grp. gram.:m.}
\end{itemize}
Planta cucurbitácea do Brasil.
\section{Tejupá}
\begin{itemize}
\item {Grp. gram.:m.}
\end{itemize}
(V.tijupá)
\section{Tejupar}
\begin{itemize}
\item {Grp. gram.:m.}
\end{itemize}
(V.tijupar)
\section{Tela}
\begin{itemize}
\item {Grp. gram.:f.}
\end{itemize}
\begin{itemize}
\item {Utilização:Des.}
\end{itemize}
\begin{itemize}
\item {Utilização:Ext.}
\end{itemize}
\begin{itemize}
\item {Proveniência:(Lat. \textunderscore tela\textunderscore )}
\end{itemize}
Teia, tecido.
Pano grosso ou encorpado, revestido de tinta branca ou pardacenta, e sôbre o qual se pintam os quadros.
Quadro: \textunderscore uma tela de Rafael\textunderscore .
Arena para torneios.
Objecto de discussão.
Momento em que se discute.
\section{Telagarça}
\begin{itemize}
\item {Grp. gram.:f.}
\end{itemize}
O mesmo ou talvez melhór que \textunderscore talagarça\textunderscore . Cf. Dom. Vieira, \textunderscore Diccion.\textunderscore , vb. \textunderscore garça\textunderscore .
\section{Telamones}
\begin{itemize}
\item {Grp. gram.:m. pl.}
\end{itemize}
\begin{itemize}
\item {Proveniência:(Lat. \textunderscore telamones\textunderscore )}
\end{itemize}
Figuras de homens, a que, em Architectura, se dá a mesma applicação que ás cariátides.
\section{Telangectasia}
\begin{itemize}
\item {Grp. gram.:f.}
\end{itemize}
\begin{itemize}
\item {Utilização:Physiol.}
\end{itemize}
\begin{itemize}
\item {Proveniência:(Do gr. \textunderscore telos\textunderscore  + \textunderscore angeion\textunderscore  + \textunderscore ektasis\textunderscore )}
\end{itemize}
Dilatação dos vasos orgânicos.
\section{Telão}
\begin{itemize}
\item {Grp. gram.:m.}
\end{itemize}
Pano, que contém annúncios, e pende adeante do pano de bôca, nos theatros.
(Cast. \textunderscore telón\textunderscore )
\section{Telaria}
\begin{itemize}
\item {Grp. gram.:f.}
\end{itemize}
\begin{itemize}
\item {Utilização:Des.}
\end{itemize}
Grande porção de telas. Cf. \textunderscore Viriato Trág.\textunderscore , III. 6.
\section{Telária}
\begin{itemize}
\item {Grp. gram.:f.}
\end{itemize}
Planta da serra da Sintra.
\section{Telautógrafo}
\begin{itemize}
\item {Grp. gram.:m.}
\end{itemize}
\begin{itemize}
\item {Proveniência:(Do gr. \textunderscore tele\textunderscore  + \textunderscore autos\textunderscore  + \textunderscore graphein\textunderscore )}
\end{itemize}
Pequeno aparelho telegráfico, recentemente inventado, e destinado a transmitir pelo fio a escrita em \textunderscore fac-simile\textunderscore , com a mesma facilidade, com que a voz se transmite pelo telefónio.
\section{Telautógrapho}
\begin{itemize}
\item {Grp. gram.:m.}
\end{itemize}
\begin{itemize}
\item {Proveniência:(Do gr. \textunderscore tele\textunderscore  + \textunderscore autos\textunderscore  + \textunderscore graphein\textunderscore )}
\end{itemize}
Pequeno apparelho telegráphico, recentemente inventado, e destinado a transmittir pelo fio a escrita em \textunderscore fac-simile\textunderscore , com a mesma facilidade, com que a voz se transmitte pelo telephónio.
\section{Telearca}
\begin{itemize}
\item {Grp. gram.:m.}
\end{itemize}
Funcionário que, entre os Tebanos, era encarregado da limpeza das ruas, canalizações, etc.
\section{Telearcha}
\begin{itemize}
\item {fónica:ca}
\end{itemize}
\begin{itemize}
\item {Grp. gram.:m.}
\end{itemize}
Funccionário que, entre os Thebanos, era encarregado da limpeza das ruas, canalizações, etc.
\section{Telecriptógrafo}
\begin{itemize}
\item {Grp. gram.:m.}
\end{itemize}
\begin{itemize}
\item {Proveniência:(Do gr. \textunderscore tele\textunderscore  + \textunderscore cruptos\textunderscore  + \textunderscore graphein\textunderscore )}
\end{itemize}
Telégrafo, inventado na Alemanha em 19O4 e que transmite duas mil letras ou sinaes por minuto.
\section{Telecryptógrapho}
\begin{itemize}
\item {Grp. gram.:m.}
\end{itemize}
\begin{itemize}
\item {Proveniência:(Do gr. \textunderscore tele\textunderscore  + \textunderscore cruptos\textunderscore  + \textunderscore graphein\textunderscore )}
\end{itemize}
Telégrapho, inventado na Alemanha em 19O4 e que transmitte duas mil letras ou sinaes por minuto.
\section{Telediágrafo}
\begin{itemize}
\item {Grp. gram.:m.}
\end{itemize}
\begin{itemize}
\item {Proveniência:(Do gr. \textunderscore tele\textunderscore  + \textunderscore dia\textunderscore  + \textunderscore graphein\textunderscore )}
\end{itemize}
Aparelho, com que se transmitem desenhos a distância.
\section{Telediágrapho}
\begin{itemize}
\item {Grp. gram.:m.}
\end{itemize}
\begin{itemize}
\item {Proveniência:(Do gr. \textunderscore tele\textunderscore  + \textunderscore dia\textunderscore  + \textunderscore graphein\textunderscore )}
\end{itemize}
Apparelho, com que se transmittem desenhos a distância.
\section{Teledinâmico}
\begin{itemize}
\item {Grp. gram.:adj.}
\end{itemize}
\begin{itemize}
\item {Utilização:Med.}
\end{itemize}
\begin{itemize}
\item {Proveniência:(Do gr. \textunderscore tele\textunderscore  + \textunderscore dunamis\textunderscore )}
\end{itemize}
Diz-se do medicamento de efeitos geraes mas remotos.
\section{Teledynâmico}
\begin{itemize}
\item {Grp. gram.:adj.}
\end{itemize}
\begin{itemize}
\item {Utilização:Med.}
\end{itemize}
\begin{itemize}
\item {Proveniência:(Do gr. \textunderscore tele\textunderscore  + \textunderscore dunamis\textunderscore )}
\end{itemize}
Diz-se do medicamento de effeitos geraes mas remotos.
\section{Teléfio}
\begin{itemize}
\item {Grp. gram.:m.}
\end{itemize}
\begin{itemize}
\item {Proveniência:(Lat. \textunderscore telephion\textunderscore )}
\end{itemize}
Planta medicinal, o mesmo que \textunderscore favária-maior\textunderscore .
\section{Telefo}
\begin{itemize}
\item {Grp. gram.:m.}
\end{itemize}
\begin{itemize}
\item {Utilização:Prov.}
\end{itemize}
O mesmo que \textunderscore talefe\textunderscore .
\section{Telefonar}
\begin{itemize}
\item {Grp. gram.:v. t.}
\end{itemize}
\begin{itemize}
\item {Grp. gram.:V. i.}
\end{itemize}
Comunicar pelo telefónio.
Fazer communicações pelo telefónio.
\section{Telefonema}
\begin{itemize}
\item {Grp. gram.:m.}
\end{itemize}
\begin{itemize}
\item {Utilização:bras}
\end{itemize}
\begin{itemize}
\item {Utilização:Neol.}
\end{itemize}
Comunicação telefónica.
\section{Telefonia}
\begin{itemize}
\item {Grp. gram.:f.}
\end{itemize}
\begin{itemize}
\item {Proveniência:(De \textunderscore telefónio\textunderscore )}
\end{itemize}
Arte de fazer chegar os sons a grande distância.
\section{Telefónico}
\begin{itemize}
\item {Grp. gram.:adj.}
\end{itemize}
Relativo á telefonia ou ao telefónio.
\section{Telefónio}
\begin{itemize}
\item {Grp. gram.:m.}
\end{itemize}
\begin{itemize}
\item {Proveniência:(Do gr. \textunderscore tele\textunderscore  + \textunderscore phone\textunderscore )}
\end{itemize}
Aparelho, para transmitir sons ou vozes a grande distância. Cf. Gonç. Guimarães, \textunderscore Vocab. Etymol.\textunderscore 
\section{Telefonista}
\begin{itemize}
\item {Grp. gram.:m.  e  f.}
\end{itemize}
Pessôa, empregada nas estações dos telefónios.
\section{Telefonógrafo}
\begin{itemize}
\item {Grp. gram.:m.}
\end{itemize}
\begin{itemize}
\item {Proveniência:(Do gr. \textunderscore tele\textunderscore  + \textunderscore phone\textunderscore  + \textunderscore graphein\textunderscore )}
\end{itemize}
Aparelho, recentemente inventado na Dinamarca e com que se póde telefonar para grandes distâncias.
\section{Teléforo}
\begin{itemize}
\item {Grp. gram.:m.}
\end{itemize}
\begin{itemize}
\item {Proveniência:(Do gr. \textunderscore tele\textunderscore  + \textunderscore phoros\textunderscore )}
\end{itemize}
Gênero de insectos coleópteros.
\section{Telefote}
\begin{itemize}
\item {Grp. gram.:m.}
\end{itemize}
\begin{itemize}
\item {Proveniência:(Do gr. \textunderscore tele\textunderscore  + \textunderscore photos\textunderscore )}
\end{itemize}
Aparelho, de invenção recentíssima, o qual reune ou tenta reunir ás condições do telégrafo e do telefónio a de fazer que vejamos o que se passa no lugar donde nos telefonam, ou telegrafam.
\section{Telefotografia}
\begin{itemize}
\item {Grp. gram.:f.}
\end{itemize}
\begin{itemize}
\item {Proveniência:(De \textunderscore telefotógrafo\textunderscore )}
\end{itemize}
Arte de fotografar, a grande distância.
\section{Telega}
\begin{itemize}
\item {Grp. gram.:f.}
\end{itemize}
Carro de quatro rodas, descoberto e tôsco, usado na Sibéria.
\section{Telegonia}
\begin{itemize}
\item {Grp. gram.:f.}
\end{itemize}
\begin{itemize}
\item {Utilização:Zool.}
\end{itemize}
\begin{itemize}
\item {Proveniência:(Do gr. \textunderscore tele\textunderscore , longe + \textunderscore gonos\textunderscore , semente)}
\end{itemize}
Phenómeno, que se dá, quando uma fêmea, coberta primeiramente por um dado reproductor, tem, nos partos futuros, filhos, resultantes de outros reproductores, mas com os caracteres do primeiro.
\section{Telegrafar}
\begin{itemize}
\item {Grp. gram.:v. i.}
\end{itemize}
\begin{itemize}
\item {Grp. gram.:V. t.}
\end{itemize}
\begin{itemize}
\item {Proveniência:(De \textunderscore telégrafo\textunderscore )}
\end{itemize}
Mandar telegrama ou telegramas.
Comunicar pelo telégrafo.
\section{Telegrafia}
\begin{itemize}
\item {Grp. gram.:f.}
\end{itemize}
\begin{itemize}
\item {Proveniência:(De \textunderscore telégrafo\textunderscore )}
\end{itemize}
Arte de construir telegráfos e fazer uso dêles.
\section{Telegraficamente}
\begin{itemize}
\item {Grp. gram.:adv.}
\end{itemize}
De modo telegráfico.
Por meio de telégrafo: \textunderscore comunicar telegraficamente uma notícia\textunderscore .
\section{Telegráfico}
\begin{itemize}
\item {Grp. gram.:adj.}
\end{itemize}
\begin{itemize}
\item {Utilização:Fig.}
\end{itemize}
Relativo ao telégrafo.
Que se realiza por meio do telégrafo: \textunderscore notícia telegráfica\textunderscore .
Em que funciona o telégrafo: \textunderscore estação telegráfica\textunderscore .
Que se transmite ou se comunica rapidamente.
\section{Telegrafista}
\begin{itemize}
\item {Grp. gram.:m.  e  f.}
\end{itemize}
\begin{itemize}
\item {Proveniência:(De \textunderscore telégrafo\textunderscore )}
\end{itemize}
Empregado ou empregada que, trabalhando no telégrafo, transmite telegramas e decifra os que vêm de outro ponto.
\section{Telégrafo}
\begin{itemize}
\item {Grp. gram.:m.}
\end{itemize}
\begin{itemize}
\item {Proveniência:(Do gr. \textunderscore tele\textunderscore  + \textunderscore graphein\textunderscore )}
\end{itemize}
Aparelho, próprio para transmitir, a distância, quaesquer comunicações.
Lugar ou casa, onde funciona êsse aparelho: \textunderscore João, vai ali ao telégrafo...\textunderscore 
\section{Telegrama}
\begin{itemize}
\item {Grp. gram.:m.}
\end{itemize}
\begin{itemize}
\item {Proveniência:(Do gr. \textunderscore tele\textunderscore  + \textunderscore gramma\textunderscore )}
\end{itemize}
Notícia, informação, pergunta ou resposta, que se transmite pelo telégrafo.
\section{Telegramma}
\begin{itemize}
\item {Grp. gram.:m.}
\end{itemize}
\begin{itemize}
\item {Proveniência:(Do gr. \textunderscore tele\textunderscore  + \textunderscore gramma\textunderscore )}
\end{itemize}
Notícia, informação, pergunta ou resposta, que se transmitte pelo telégrapho.
\section{Telegraphar}
\begin{itemize}
\item {Grp. gram.:v. i.}
\end{itemize}
\begin{itemize}
\item {Grp. gram.:V. t.}
\end{itemize}
\begin{itemize}
\item {Proveniência:(De \textunderscore telégrapho\textunderscore )}
\end{itemize}
Mandar telegramma ou telegrammas.
Communicar pelo telégrapho.
\section{Telegraphia}
\begin{itemize}
\item {Grp. gram.:f.}
\end{itemize}
\begin{itemize}
\item {Proveniência:(De \textunderscore telégrapho\textunderscore )}
\end{itemize}
Arte de construir telegráphos e fazer uso dêlles.
\section{Telegraphicamente}
\begin{itemize}
\item {Grp. gram.:adv.}
\end{itemize}
De modo telegráphico.
Por meio de telégrapho: \textunderscore communicar telegraphicamente uma notícia\textunderscore .
\section{Telegráphico}
\begin{itemize}
\item {Grp. gram.:adj.}
\end{itemize}
\begin{itemize}
\item {Utilização:Fig.}
\end{itemize}
Relativo ao telégrapho.
Que se realiza por meio do telégrapho: \textunderscore notícia telegráphica\textunderscore .
Em que funcciona o telégrapho: \textunderscore estação telegráphica\textunderscore .
Que se transmitte ou se communica rapidamente.
\section{Telegraphista}
\begin{itemize}
\item {Grp. gram.:m.  e  f.}
\end{itemize}
\begin{itemize}
\item {Proveniência:(De \textunderscore telégrapho\textunderscore )}
\end{itemize}
Empregado ou empregada que, trabalhando no telégrapho, transmitte telegrammas e decifra os que vêm de outro ponto.
\section{Telégrapho}
\begin{itemize}
\item {Grp. gram.:m.}
\end{itemize}
\begin{itemize}
\item {Proveniência:(Do gr. \textunderscore tele\textunderscore  + \textunderscore graphein\textunderscore )}
\end{itemize}
Apparelho, próprio para transmittir, a distância, quaesquer communicações.
Lugar ou casa, onde funcciona êsse apparelho: \textunderscore João, vai alli ao telégrapho...\textunderscore 
\section{Télegu}
\begin{itemize}
\item {Grp. gram.:m.}
\end{itemize}
(V.télugo)
\section{Teleiconógrafo}
\begin{itemize}
\item {fónica:le-í}
\end{itemize}
\begin{itemize}
\item {Grp. gram.:m.}
\end{itemize}
\begin{itemize}
\item {Proveniência:(Do gr. \textunderscore tele\textunderscore  + \textunderscore eikon\textunderscore  + \textunderscore graphein\textunderscore )}
\end{itemize}
Aparelho, para reproduzir um desenho a distância, por meio de correntes transmitidas por fins eléctricos. Cf. Mélida, \textunderscore Vocabulário\textunderscore .
\section{Teleiconógrapho}
\begin{itemize}
\item {fónica:le-í}
\end{itemize}
\begin{itemize}
\item {Grp. gram.:m.}
\end{itemize}
\begin{itemize}
\item {Proveniência:(Do gr. \textunderscore tele\textunderscore  + \textunderscore eikon\textunderscore  + \textunderscore graphein\textunderscore )}
\end{itemize}
Apparelho, para reproduzir um desenho a distância, por meio de correntes transmittidas por fins eléctricos. Cf. Mélida, \textunderscore Vocabulário\textunderscore .
\section{Telemetria}
\begin{itemize}
\item {Grp. gram.:f.}
\end{itemize}
\begin{itemize}
\item {Proveniência:(De \textunderscore telêmetro\textunderscore )}
\end{itemize}
Arte de medir distâncias.
\section{Telemétrico}
\begin{itemize}
\item {Grp. gram.:adj.}
\end{itemize}
Relativo á telemetria.
\section{Telêmetro}
\begin{itemize}
\item {Grp. gram.:m.}
\end{itemize}
\begin{itemize}
\item {Proveniência:(Do gr. \textunderscore tele\textunderscore  + \textunderscore metron\textunderscore )}
\end{itemize}
Instrumento para medir distâncias rapidamente.
\section{Teleologia}
\begin{itemize}
\item {Grp. gram.:f.}
\end{itemize}
\begin{itemize}
\item {Proveniência:(Do gr. \textunderscore telos\textunderscore , \textunderscore teleos\textunderscore  + \textunderscore logos\textunderscore )}
\end{itemize}
Doutrina, á cêrca das causas finaes.
Theoria, que explica os seres pelo fim a que apparentemente são destinados.
\section{Teleológico}
\begin{itemize}
\item {Grp. gram.:adj.}
\end{itemize}
Relativo á teleologia.
\section{Teleósteos}
\begin{itemize}
\item {Grp. gram.:m. pl.}
\end{itemize}
\begin{itemize}
\item {Proveniência:(Do gr. \textunderscore telos\textunderscore , \textunderscore teleos\textunderscore  + \textunderscore osteon\textunderscore )}
\end{itemize}
Animaes, de esqueleto ósseo, entre os quaes se comprehendem os peixes de ordem superior.
\section{Telepathia}
\begin{itemize}
\item {Grp. gram.:f.}
\end{itemize}
\begin{itemize}
\item {Utilização:Neol.}
\end{itemize}
\begin{itemize}
\item {Proveniência:(Do gr. \textunderscore tele\textunderscore  + \textunderscore pathos\textunderscore )}
\end{itemize}
Estado mórbido das pessôas, de quem se diz que, sem fazer uso da vista natural, vêem e conhecem o que se passa muito longe dellas.
\section{Telepáthico}
\begin{itemize}
\item {Grp. gram.:adj.}
\end{itemize}
Relativo á telepathia.
\section{Telepatia}
\begin{itemize}
\item {Grp. gram.:f.}
\end{itemize}
\begin{itemize}
\item {Utilização:Neol.}
\end{itemize}
\begin{itemize}
\item {Proveniência:(Do gr. \textunderscore tele\textunderscore  + \textunderscore pathos\textunderscore )}
\end{itemize}
Estado mórbido das pessôas, de quem se diz que, sem fazer uso da vista natural, vêem e conhecem o que se passa muito longe delas.
\section{Telepático}
\begin{itemize}
\item {Grp. gram.:adj.}
\end{itemize}
Relativo á telepatia.
\section{Teléphio}
\begin{itemize}
\item {Grp. gram.:m.}
\end{itemize}
\begin{itemize}
\item {Proveniência:(Lat. \textunderscore telephion\textunderscore )}
\end{itemize}
Planta medicinal, o mesmo que \textunderscore favária-maior\textunderscore .
\section{Telephonar}
\begin{itemize}
\item {Grp. gram.:v. t.}
\end{itemize}
\begin{itemize}
\item {Grp. gram.:V. i.}
\end{itemize}
Communicar pelo telephónio.
Fazer communicações pelo telephónio.
\section{Telephone}
\begin{itemize}
\item {Grp. gram.:m.}
\end{itemize}
\begin{itemize}
\item {Proveniência:(T. francês, desnecessariamente usado em vez de \textunderscore telephónio\textunderscore )}
\end{itemize}
O mesmo que \textunderscore telephónio\textunderscore .
\section{Telephonema}
\begin{itemize}
\item {Grp. gram.:m.}
\end{itemize}
\begin{itemize}
\item {Utilização:bras}
\end{itemize}
\begin{itemize}
\item {Utilização:Neol.}
\end{itemize}
Communicação telephónica.
\section{Telephonia}
\begin{itemize}
\item {Grp. gram.:f.}
\end{itemize}
\begin{itemize}
\item {Proveniência:(De \textunderscore telephónio\textunderscore )}
\end{itemize}
Arte de fazer chegar os sons a grande distância.
\section{Telephónico}
\begin{itemize}
\item {Grp. gram.:adj.}
\end{itemize}
Relativo á telephonia ou ao telephónio.
\section{Telephónio}
\begin{itemize}
\item {Grp. gram.:m.}
\end{itemize}
\begin{itemize}
\item {Proveniência:(Do gr. \textunderscore tele\textunderscore  + \textunderscore phone\textunderscore )}
\end{itemize}
Apparelho, para transmittir sons ou vozes a grande distância. Cf. Gonç. Guimarães, \textunderscore Vocab. Etymol.\textunderscore 
\section{Telephonista}
\begin{itemize}
\item {Grp. gram.:m.  e  f.}
\end{itemize}
Pessôa, empregada nas estações dos telefónios.
\section{Telephonógrapho}
\begin{itemize}
\item {Grp. gram.:m.}
\end{itemize}
\begin{itemize}
\item {Proveniência:(Do gr. \textunderscore tele\textunderscore  + \textunderscore phone\textunderscore  + \textunderscore graphein\textunderscore )}
\end{itemize}
Apparelho, recentemente inventado na Dinamarca e com que se póde telephonar para grandes distâncias.
\section{Teléphoro}
\begin{itemize}
\item {Grp. gram.:m.}
\end{itemize}
\begin{itemize}
\item {Proveniência:(Do gr. \textunderscore tele\textunderscore  + \textunderscore phoros\textunderscore )}
\end{itemize}
Gênero de insectos coleópteros.
\section{Telephote}
\begin{itemize}
\item {Grp. gram.:m.}
\end{itemize}
\begin{itemize}
\item {Proveniência:(Do gr. \textunderscore tele\textunderscore  + \textunderscore photos\textunderscore )}
\end{itemize}
Apparelho, de invenção recentíssima, o qual reune ou tenta reunir ás condições do telégrapho e do telephónio a de fazer que vejamos o que se passa no lugar donde nos telephonam, ou telegrapham.
\section{Telephotographia}
\begin{itemize}
\item {Grp. gram.:f.}
\end{itemize}
\begin{itemize}
\item {Proveniência:(De \textunderscore telephotógrapho\textunderscore )}
\end{itemize}
Arte de photographar, a grande distância.
\section{Telefotográfico}
\begin{itemize}
\item {Grp. gram.:adj.}
\end{itemize}
Relativo á telefotografia.
\section{Telefotógrafo}
\begin{itemize}
\item {Grp. gram.:m.}
\end{itemize}
\begin{itemize}
\item {Proveniência:(Do gr. \textunderscore tele\textunderscore  + \textunderscore photos\textunderscore  + \textunderscore graphein\textunderscore )}
\end{itemize}
Aquele que pratíca a telefotografia.
\section{Telephotográphico}
\begin{itemize}
\item {Grp. gram.:adj.}
\end{itemize}
Relativo á telephotographia.
\section{Telephotógrapho}
\begin{itemize}
\item {Grp. gram.:m.}
\end{itemize}
\begin{itemize}
\item {Proveniência:(Do gr. \textunderscore tele\textunderscore  + \textunderscore photos\textunderscore  + \textunderscore graphein\textunderscore )}
\end{itemize}
Aquelle que pratíca a telephotographia.
\section{Teléquia}
\begin{itemize}
\item {Grp. gram.:f.}
\end{itemize}
Gênero de plantas, da fam. das compostas.
\section{Teles}
\begin{itemize}
\item {Grp. gram.:f. pl.}
\end{itemize}
\begin{itemize}
\item {Utilização:Prov.}
\end{itemize}
\begin{itemize}
\item {Utilização:beir.}
\end{itemize}
Rêde, para apanhar perdizes. (Colhido no Fundão)
\section{Telescópico}
\begin{itemize}
\item {Grp. gram.:adj.}
\end{itemize}
Relativo ao telescópio.
\section{Telescópio}
\begin{itemize}
\item {Grp. gram.:m.}
\end{itemize}
\begin{itemize}
\item {Proveniência:(Do gr. \textunderscore tele\textunderscore  + \textunderscore skopein\textunderscore )}
\end{itemize}
Nome genérico dos instrumentos ópticos, destinados a observar os objectos distantes.
Apparelho, para observar os astros.
Pequena constellação meridional.
\section{Telescritor}
\begin{itemize}
\item {Grp. gram.:m.}
\end{itemize}
\begin{itemize}
\item {Proveniência:(Do gr. \textunderscore tele\textunderscore  + lat. \textunderscore scriptor\textunderscore )}
\end{itemize}
Apparelho, recentemente inventado, (1897), e que, applicado a uma rêde telegráphica ou telephónica, estabelece uma corrente, que põe em movimento um mecanismo destinado a imprimir letras a distância.
\section{Telésia}
\begin{itemize}
\item {Grp. gram.:f.}
\end{itemize}
\begin{itemize}
\item {Utilização:Miner.}
\end{itemize}
Saphira branca, typo dos crystaes hyalinos.
\section{Telga}
\begin{itemize}
\item {Grp. gram.:f.}
\end{itemize}
Antiga medida de capacidade, o mesmo que \textunderscore teiga\textunderscore .
\section{Têlha}
\begin{itemize}
\item {Grp. gram.:f.}
\end{itemize}
\begin{itemize}
\item {Utilização:Fam.}
\end{itemize}
\begin{itemize}
\item {Utilização:T. de Maçonaria}
\end{itemize}
\begin{itemize}
\item {Proveniência:(Do lat. \textunderscore tegula\textunderscore )}
\end{itemize}
Peça, feita geralmente de barro cozido, e que serve especialmente para a cobertura de edifícios.
Mania, bolha.
Chapéu de mulhér, com as abas erguidas ou voltadas para cima lateralmente.
Cada um dos pratos de um banquete.
\section{Telhado}
\begin{itemize}
\item {Grp. gram.:m.}
\end{itemize}
\begin{itemize}
\item {Utilização:Ant.}
\end{itemize}
\begin{itemize}
\item {Utilização:Fam.}
\end{itemize}
Parte, externa da cobertura de um edifício, feita geralmente de têlhas.
Conjunto de têlhas, que cobrem uma construcção.
Cobertura de uma construcção.
Prego de arame.
Coberta de navio.
Grande mania, têlha excessiva: \textunderscore o João não tem têlha, tem telhado\textunderscore .
\section{Telhador}
\begin{itemize}
\item {Grp. gram.:m.}
\end{itemize}
\begin{itemize}
\item {Proveniência:(De \textunderscore telhar\textunderscore )}
\end{itemize}
Aquelle que telha.
Tampa de uma vasilha de barro.
\section{Telhadura}
\begin{itemize}
\item {Grp. gram.:f.}
\end{itemize}
\begin{itemize}
\item {Utilização:T. da Bairrada}
\end{itemize}
Acto ou effeito de telhar.
Lugar, onde se fazem têlhas.
Telheiro, alpendre.
\section{Telhal}
\begin{itemize}
\item {Grp. gram.:m.}
\end{itemize}
Forno, onde se cozem têlhas.
\section{Telhão}
\begin{itemize}
\item {Grp. gram.:m.}
\end{itemize}
\begin{itemize}
\item {Utilização:T. da Bairrada}
\end{itemize}
Grande têlha; têlha prensada.
Pedaço de têlha partida; caco.
\section{Telhar}
\begin{itemize}
\item {Grp. gram.:v. t.}
\end{itemize}
Cobrir com telhas.
\section{Têlha-van}
\begin{itemize}
\item {Grp. gram.:f.}
\end{itemize}
Telhado sem forros. Cf. Filinto, I, 194.
\section{Telheira}
\begin{itemize}
\item {Grp. gram.:f.}
\end{itemize}
Fábrica de têlhas.
Fábrica de olleiro.
\section{Telheiro}
\begin{itemize}
\item {Grp. gram.:m.}
\end{itemize}
Fabricante de têlhas.
Simples cobertura de têlha, para abrigo de pessôas, de animaes, de lenha, utensílios de lavoira, etc.
Alpendre.
\section{Telhice}
\begin{itemize}
\item {Grp. gram.:f.}
\end{itemize}
\begin{itemize}
\item {Utilização:Fam.}
\end{itemize}
Mania, têlha.
\section{Telhinha}
\begin{itemize}
\item {Grp. gram.:f.}
\end{itemize}
O mesmo que \textunderscore telhinhas\textunderscore . Cf. \textunderscore Filodemo\textunderscore , V, 2.
\section{Telhinhas}
\begin{itemize}
\item {Grp. gram.:f. pl.}
\end{itemize}
\begin{itemize}
\item {Proveniência:(De \textunderscore têlha\textunderscore )}
\end{itemize}
Dois fragmentos de loiça, que se fazem soar, percutindo-os um com o outro.
\section{Têlho}
\begin{itemize}
\item {Grp. gram.:m.}
\end{itemize}
\begin{itemize}
\item {Proveniência:(Do lat. \textunderscore tegulum\textunderscore )}
\end{itemize}
Têsto de barro.
Pedaço de barro, que serve de têsto.
\section{Telhudo}
\begin{itemize}
\item {Grp. gram.:adj.}
\end{itemize}
\begin{itemize}
\item {Utilização:Fam.}
\end{itemize}
Maníaco, que tem têlha.
\section{Telilha}
\begin{itemize}
\item {Grp. gram.:f.}
\end{itemize}
Tela fina.
\section{Telim}
\begin{itemize}
\item {Grp. gram.:m.}
\end{itemize}
\begin{itemize}
\item {Proveniência:(T. onom.)}
\end{itemize}
Voz imitativa do sino, da campaínha ou do choque de dinheiro em metal.
\section{Telina}
\begin{itemize}
\item {Grp. gram.:f.}
\end{itemize}
\begin{itemize}
\item {Proveniência:(Do gr. \textunderscore telline\textunderscore )}
\end{itemize}
Molusco acéfalo.
\section{Telinga}
\begin{itemize}
\item {Grp. gram.:m.}
\end{itemize}
Idioma do reino de Telinga.
O mesmo que \textunderscore télugo\textunderscore .
\section{Telintar}
\begin{itemize}
\item {Grp. gram.:v. i.}
\end{itemize}
\begin{itemize}
\item {Proveniência:(De \textunderscore telim\textunderscore )}
\end{itemize}
Fazer telim; soar, como o sino, campaínha ou dinheiro.
\section{Telipote-iba}
\begin{itemize}
\item {Grp. gram.:m.}
\end{itemize}
O mesmo que \textunderscore guirá\textunderscore .
\section{Telitóne}
\begin{itemize}
\item {Grp. gram.:m.}
\end{itemize}
\begin{itemize}
\item {Utilização:Prov.}
\end{itemize}
\begin{itemize}
\item {Utilização:Trasm.}
\end{itemize}
Vestido espaventoso, com muitas fitas e laços.
\section{Teliz}
\begin{itemize}
\item {Grp. gram.:m.}
\end{itemize}
\begin{itemize}
\item {Proveniência:(Do lat. \textunderscore trilix\textunderscore )}
\end{itemize}
Pano, para cobrir a sella do cavallo.
\section{Tellina}
\begin{itemize}
\item {Grp. gram.:f.}
\end{itemize}
\begin{itemize}
\item {Proveniência:(Do gr. \textunderscore telline\textunderscore )}
\end{itemize}
Mollusco acéphalo.
\section{Tellurato}
\begin{itemize}
\item {Grp. gram.:m.}
\end{itemize}
\begin{itemize}
\item {Proveniência:(De \textunderscore tellúrio\textunderscore )}
\end{itemize}
Qualquer sal, proveniente do ácido tellúrico.
\section{Tellureto}
\begin{itemize}
\item {Grp. gram.:m.}
\end{itemize}
Combinação do tellúrio com outro metal.
\section{Tellúrico}
\begin{itemize}
\item {Grp. gram.:adj.}
\end{itemize}
Relativo ao tellúrio.
Relativo á terra.
\section{Tellurífero}
\begin{itemize}
\item {Grp. gram.:adj.}
\end{itemize}
\begin{itemize}
\item {Proveniência:(Do lat. \textunderscore tellus\textunderscore , \textunderscore telluris\textunderscore  + \textunderscore ferre\textunderscore )}
\end{itemize}
Que contém tellúrio.
\section{Tellúrio}
\begin{itemize}
\item {Grp. gram.:m.}
\end{itemize}
\begin{itemize}
\item {Proveniência:(Do lat. \textunderscore tellus\textunderscore , \textunderscore telluris\textunderscore )}
\end{itemize}
Metallóide raro, friável e semelhante ao enxôfre.
\section{Telo}
\begin{itemize}
\item {Grp. gram.:m.}
\end{itemize}
\begin{itemize}
\item {Utilização:Gír.}
\end{itemize}
Jumento.
\section{Telodinâmico}
\begin{itemize}
\item {Grp. gram.:adj.}
\end{itemize}
\begin{itemize}
\item {Proveniência:(Do gr. \textunderscore tele\textunderscore  + \textunderscore dunamis\textunderscore )}
\end{itemize}
Diz-se da transmissão da fôrça de um dado motor a uma grande distância.
\section{Telodynâmico}
\begin{itemize}
\item {Grp. gram.:adj.}
\end{itemize}
\begin{itemize}
\item {Proveniência:(Do gr. \textunderscore tele\textunderscore  + \textunderscore dunamis\textunderscore )}
\end{itemize}
Diz-se da transmissão da fôrça de um dado motor a uma grande distância.
\section{Telometria}
\begin{itemize}
\item {Grp. gram.:f.}
\end{itemize}
O mesmo ou melhor que \textunderscore telemetria\textunderscore .
\section{Telométrico}
\begin{itemize}
\item {Grp. gram.:adj.}
\end{itemize}
Relativo á telometria.
\section{Telómetro}
\begin{itemize}
\item {Grp. gram.:m.}
\end{itemize}
O mesmo ou melhor que \textunderscore telêmetro\textunderscore .
\section{Telónio}
\begin{itemize}
\item {Grp. gram.:m.}
\end{itemize}
\begin{itemize}
\item {Proveniência:(Lat. \textunderscore telonium\textunderscore )}
\end{itemize}
Mesa, onde se recebiam as rendas públicas.
Lugar, em que se mercadeja ou em que se chatina.
Arca de pau-santo, com pregaria amarela. Cf. A. Corvo, \textunderscore Um Anno na Côrte\textunderscore , II, 65.
Penteado alto, que se usou em Lisbôa, no século XVIII.
\section{Télugo}
\begin{itemize}
\item {Grp. gram.:m.}
\end{itemize}
Idioma indiano, falado em Telinga.
\section{Telurato}
\begin{itemize}
\item {Grp. gram.:m.}
\end{itemize}
\begin{itemize}
\item {Proveniência:(De \textunderscore telúrio\textunderscore )}
\end{itemize}
Qualquer sal, proveniente do ácido telúrico.
\section{Telureto}
\begin{itemize}
\item {fónica:lurê}
\end{itemize}
\begin{itemize}
\item {Grp. gram.:m.}
\end{itemize}
Combinação do telúrio com outro metal.
\section{Telurífero}
\begin{itemize}
\item {Grp. gram.:adj.}
\end{itemize}
\begin{itemize}
\item {Proveniência:(Do lat. \textunderscore tellus\textunderscore , \textunderscore telluris\textunderscore  + \textunderscore ferre\textunderscore )}
\end{itemize}
Que contém telúrio.
\section{Telúrio}
\begin{itemize}
\item {Grp. gram.:m.}
\end{itemize}
\begin{itemize}
\item {Proveniência:(Do lat. \textunderscore tellus\textunderscore , \textunderscore telluris\textunderscore )}
\end{itemize}
Metalóide raro, friável e semelhante ao enxôfre.
\section{Temão}
\begin{itemize}
\item {Grp. gram.:m.}
\end{itemize}
\begin{itemize}
\item {Utilização:Ext.}
\end{itemize}
\begin{itemize}
\item {Utilização:Fig.}
\end{itemize}
\begin{itemize}
\item {Proveniência:(Do lat. \textunderscore temo\textunderscore )}
\end{itemize}
Peça comprida do carro ou do arado, a que se atrelam os animaes, que puxam o mesmo arado ou carro.
Lança de carruagem; tirante.
Tiradoira.
Barra do leme.
O leme.
Governação, direcção.
\section{Temba}
\begin{itemize}
\item {Grp. gram.:f.}
\end{itemize}
\begin{itemize}
\item {Utilização:T. da Áfr. Or. Port}
\end{itemize}
O mesmo que \textunderscore povoação\textunderscore .
\section{Temberatu}
\begin{itemize}
\item {Grp. gram.:m.}
\end{itemize}
Planta rutácea.
\section{Tembetara}
\begin{itemize}
\item {Grp. gram.:f.}
\end{itemize}
O mesmo que \textunderscore metara\textunderscore .
\section{Temboíba}
\begin{itemize}
\item {Grp. gram.:f.}
\end{itemize}
\begin{itemize}
\item {Utilização:Bras}
\end{itemize}
Árvore silvestre.
\section{Temedoiro}
\begin{itemize}
\item {Grp. gram.:adj.}
\end{itemize}
\begin{itemize}
\item {Utilização:Des.}
\end{itemize}
O mesmo que \textunderscore temível\textunderscore .
\section{Temedouro}
\begin{itemize}
\item {Grp. gram.:adj.}
\end{itemize}
\begin{itemize}
\item {Utilização:Des.}
\end{itemize}
O mesmo que \textunderscore temível\textunderscore .
\section{Temembós}
\begin{itemize}
\item {Grp. gram.:m. pl.}
\end{itemize}
Tribo de Índios do Brasil, no Maranhão.
\section{Temente}
\begin{itemize}
\item {Grp. gram.:adj.}
\end{itemize}
Que teme: \textunderscore pessôa muito temente a Deus\textunderscore .
\section{Temer}
\begin{itemize}
\item {Grp. gram.:v. t.}
\end{itemize}
\begin{itemize}
\item {Proveniência:(Do lat. \textunderscore timere\textunderscore )}
\end{itemize}
Têr medo ou receio de.
Tributar grande respeito ou reverência a.
\textunderscore Fazer-se temer\textunderscore , tornar-se temido, causar temor.
\section{Temerando}
\begin{itemize}
\item {Grp. gram.:adj.}
\end{itemize}
\begin{itemize}
\item {Utilização:Des.}
\end{itemize}
\begin{itemize}
\item {Proveniência:(De \textunderscore temer\textunderscore )}
\end{itemize}
Que se deve temer; temeroso; temível. Cf. F. Barreto, \textunderscore Eneida\textunderscore , II, 41.
\section{Temerariamente}
\begin{itemize}
\item {Grp. gram.:adv.}
\end{itemize}
De modo temerário.
Com temeridade; de modo arriscado.
\section{Temerário}
\begin{itemize}
\item {Grp. gram.:adj.}
\end{itemize}
\begin{itemize}
\item {Proveniência:(Lat. \textunderscore temerarius\textunderscore )}
\end{itemize}
Precipitado, imprudente.
Arriscado.
Audacioso; atrevido.
Que mostra audácia: \textunderscore acto temerário\textunderscore .
\section{Temeridade}
\begin{itemize}
\item {Grp. gram.:f.}
\end{itemize}
\begin{itemize}
\item {Proveniência:(Lat. \textunderscore temeritas\textunderscore )}
\end{itemize}
Qualidade do que é temerário; imprudência, ousadia.
\section{Temero}
\begin{itemize}
\item {Grp. gram.:adj.}
\end{itemize}
\begin{itemize}
\item {Utilização:Bras. do Ceará}
\end{itemize}
O mesmo que \textunderscore temerário\textunderscore .
(Do port.)
\section{Temerosamente}
\begin{itemize}
\item {Grp. gram.:adv.}
\end{itemize}
De modo temeroso; com pavor.
\section{Temeroso}
\begin{itemize}
\item {Grp. gram.:adj.}
\end{itemize}
\begin{itemize}
\item {Proveniência:(De \textunderscore temer\textunderscore )}
\end{itemize}
Que infunde temor.
Terrível: \textunderscore tempestade temerosa\textunderscore .
Que tem mêdo; medroso.
\section{Temeto}
\begin{itemize}
\item {Grp. gram.:m.}
\end{itemize}
\begin{itemize}
\item {Proveniência:(Lat. \textunderscore temetum\textunderscore )}
\end{itemize}
Bebida capitosa, usada pelos Romanos. Cf. Castilho, \textunderscore Metam.\textunderscore , 302.
\section{Têmia}
\begin{itemize}
\item {Grp. gram.:f.}
\end{itemize}
Gênero de aves, da ordem dos pássares.
(Relaciona-se com \textunderscore têmio\textunderscore )
\section{Temibilidade}
\begin{itemize}
\item {Grp. gram.:f.}
\end{itemize}
Qualidade de temível.
\section{Temido}
\begin{itemize}
\item {Grp. gram.:adj.}
\end{itemize}
\begin{itemize}
\item {Proveniência:(De \textunderscore temer\textunderscore )}
\end{itemize}
Que causa mêdo; assustador.
Destemido; valente.
\section{Temido}
\begin{itemize}
\item {Grp. gram.:adj.}
\end{itemize}
Que tem mêdo; medroso.
Tímido.
(Talvez de \textunderscore tímido\textunderscore , com deslocação de accento)
\section{Temiminós}
\begin{itemize}
\item {Grp. gram.:m. pl.}
\end{itemize}
Tribo selvagem do Brasil.
\section{Têmio}
\begin{itemize}
\item {Grp. gram.:m.}
\end{itemize}
Espécie de corvo de Java.
\section{Temível}
\begin{itemize}
\item {Grp. gram.:adj.}
\end{itemize}
Que se póde ou se deve temer.
Que causa temor.
\section{Temivelmente}
\begin{itemize}
\item {Grp. gram.:adv.}
\end{itemize}
De modo temível.
\section{Temnodonte}
\begin{itemize}
\item {Grp. gram.:m.}
\end{itemize}
Gênero de peixes malacopterýgios.
\section{Temnóptero}
\begin{itemize}
\item {Grp. gram.:m.}
\end{itemize}
Gênero de insectos coleópteros pentâmeros.
\section{Temoeiro}
\begin{itemize}
\item {Grp. gram.:m.}
\end{itemize}
\begin{itemize}
\item {Proveniência:(Do lat. \textunderscore temonarius\textunderscore )}
\end{itemize}
O mesmo ou melhór que \textunderscore tamoeiro\textunderscore . Cf. B. Pereira, \textunderscore Prosódia\textunderscore , vb. \textunderscore zygodesma\textunderscore .
\section{Temonar}
\begin{itemize}
\item {Grp. gram.:v. t.}
\end{itemize}
Servir de temoneiro a (uma embarcação).
\section{Temoneira}
\begin{itemize}
\item {Grp. gram.:f.}
\end{itemize}
\begin{itemize}
\item {Utilização:Náut.}
\end{itemize}
\begin{itemize}
\item {Proveniência:(Do lat. \textunderscore temo\textunderscore , \textunderscore temonis\textunderscore )}
\end{itemize}
Espaço, em que se move o pinçote do leme.
\section{Temoneiro}
\begin{itemize}
\item {Grp. gram.:m.}
\end{itemize}
\begin{itemize}
\item {Utilização:Fig.}
\end{itemize}
\begin{itemize}
\item {Proveniência:(Do lat. \textunderscore temonarius\textunderscore )}
\end{itemize}
Aquella que governa o temão das embarcações.
Aquelle que dirige ou regula qualquer coisa; guia.
\section{Temor}
\begin{itemize}
\item {Grp. gram.:m.}
\end{itemize}
\begin{itemize}
\item {Utilização:Fig.}
\end{itemize}
\begin{itemize}
\item {Proveniência:(Do lat. \textunderscore timor\textunderscore )}
\end{itemize}
Acto ou effeito de temer.
Mêdo, susto.
Sentimento de respeito ou reverência.
Pessôa ou coisa, que causa mêdo.
Pontualidade.
Escrúpulo.
Zêlo.
\section{Temorizar}
\begin{itemize}
\item {Grp. gram.:v. t.}
\end{itemize}
\begin{itemize}
\item {Utilização:Des.}
\end{itemize}
O mesmo que \textunderscore atemorizar\textunderscore . Cf. Pant. de Aveiro, \textunderscore Itiner.\textunderscore , 319 v.^o, (3.^a ed.).
\section{Tempão}
\begin{itemize}
\item {Grp. gram.:m.}
\end{itemize}
\begin{itemize}
\item {Utilização:Bras. de Minas}
\end{itemize}
Grande espaço de tempo.
\section{Têmpera}
\begin{itemize}
\item {Grp. gram.:f.}
\end{itemize}
\begin{itemize}
\item {Utilização:Fig.}
\end{itemize}
Acto ou effeito de temperar.
Temperatura.
Banho, em que se temperam os metaes, introduzindo os candentes em água fria.
Feitio, índole: \textunderscore homem de má têmpera\textunderscore .
Austeridade.
Cunha, em vários apparelhos.
Preparação, que se dava aos falcões e a outras aves, que no dia seguinte se deviam empregar na caça.
(Talvez do lat. \textunderscore tempora\textunderscore , pl. de \textunderscore tempus\textunderscore , se não é alter. de \textunderscore tempra\textunderscore , de \textunderscore temprar\textunderscore , por \textunderscore temperar\textunderscore )
\section{Têmpera}
\begin{itemize}
\item {Grp. gram.:f.}
\end{itemize}
\begin{itemize}
\item {Utilização:Prov.}
\end{itemize}
\begin{itemize}
\item {Utilização:minh.}
\end{itemize}
O mesmo que \textunderscore trempe\textunderscore .
\section{Tempéra}
\begin{itemize}
\item {Grp. gram.:f.}
\end{itemize}
\begin{itemize}
\item {Utilização:Prov.}
\end{itemize}
\begin{itemize}
\item {Utilização:minh.}
\end{itemize}
\begin{itemize}
\item {Utilização:Fig.}
\end{itemize}
\begin{itemize}
\item {Proveniência:(De \textunderscore temperar\textunderscore )}
\end{itemize}
O mesmo que \textunderscore têmpera\textunderscore ^1.
Tareia, sova.
\section{Temperadmente}
\begin{itemize}
\item {Grp. gram.:adv.}
\end{itemize}
De modo temperado.
Com moderação; prudentemente.
\section{Temperado}
\begin{itemize}
\item {Grp. gram.:adj.}
\end{itemize}
Em que se deitou tempêro; adubado.
Moderado; suavizado: \textunderscore clima temperado\textunderscore .
Agradável.
Delicado.
Diz-se da zona terrestre, collocada entre os trópicos e os círculos polares, e em que o clima é mais ou menos moderado.
\section{Temperador}
\begin{itemize}
\item {Grp. gram.:m.  e  adj.}
\end{itemize}
\begin{itemize}
\item {Proveniência:(Do lat. \textunderscore temperator\textunderscore )}
\end{itemize}
O que tempéra; moderador.
\section{Temperamento}
\begin{itemize}
\item {Grp. gram.:m.}
\end{itemize}
\begin{itemize}
\item {Utilização:Mús.}
\end{itemize}
\begin{itemize}
\item {Proveniência:(Lat. \textunderscore temperamentum\textunderscore )}
\end{itemize}
O mesmo que \textunderscore têmpera\textunderscore ^1.
Carácter.
Modo de constituição de um corpo animal; compleição.
Qualidade predominante num organismo: \textunderscore temperamento sanguíneo\textunderscore .
Mistura proporcional de coisas.
Combinação.
Mescla.
Temperança, moderação, ordem.
Formação artificial da escala musical, de maneira que se componha só de doze sons differentes em cada oitava.
\section{Temperança}
\begin{itemize}
\item {Grp. gram.:f.}
\end{itemize}
\begin{itemize}
\item {Proveniência:(Do lat. \textunderscore temperantia\textunderscore )}
\end{itemize}
Qualidade ou virtude de quem é moderado ou de quem modera appetites e paixões.
Sobriedade; parcimónia.
\section{Temperante}
\begin{itemize}
\item {Grp. gram.:adj.}
\end{itemize}
\begin{itemize}
\item {Proveniência:(Lat. \textunderscore temperans\textunderscore )}
\end{itemize}
Que tempéra.
Que tem temperança.
Calmante.
\section{Temperar}
\begin{itemize}
\item {Grp. gram.:v. t.}
\end{itemize}
\begin{itemize}
\item {Utilização:Fig.}
\end{itemize}
\begin{itemize}
\item {Utilização:Prov.}
\end{itemize}
\begin{itemize}
\item {Utilização:minh.}
\end{itemize}
\begin{itemize}
\item {Proveniência:(Lat. \textunderscore temperare\textunderscore )}
\end{itemize}
Misturar proporcionalmente.
Adubar.
Moderar o gôsto ou sabôr de.
Preparar.
Predispor.
Endurecer, tornar consistente (o metal).
Fortificar, dar vigor a.
Amenizar, tornar suave.
Harmonizar.
Governar.
Afinar (instrumentos).
Alliar, addicionar a.
Sovar; castigar.
\section{Temperatura}
\begin{itemize}
\item {Grp. gram.:f.}
\end{itemize}
\begin{itemize}
\item {Utilização:Fig.}
\end{itemize}
\begin{itemize}
\item {Proveniência:(Lat. \textunderscore temperatura\textunderscore )}
\end{itemize}
Estado de frio ou calor, de humidade ou secura do ar, impressionando os nossos órgãos.
Grau apreciável de calor ou frio, num lugar ou corpo.
Situação ou estado moral; acção.
\section{Temperatural}
\begin{itemize}
\item {Grp. gram.:adj.}
\end{itemize}
\begin{itemize}
\item {Utilização:Neol.}
\end{itemize}
Relativo a temperatura.
\section{Temperaturalmente}
\begin{itemize}
\item {Grp. gram.:adv.}
\end{itemize}
\begin{itemize}
\item {Utilização:Neol.}
\end{itemize}
De modo temperatural; por meio de temperatura:«\textunderscore a evolução cýclica da pneumonia franca, aquilatada temperaturalmente por Wunderlich...\textunderscore »R. Jorge, na \textunderscore Luta\textunderscore , de 6-VI-913.
\section{Tempereiro}
\begin{itemize}
\item {Grp. gram.:m.}
\end{itemize}
\begin{itemize}
\item {Proveniência:(De \textunderscore temperar\textunderscore )}
\end{itemize}
Utensílio, com que as tecedeiras esticam o pano no tear.
Cada um dos paus da nora.
\section{Tempérie}
\begin{itemize}
\item {Grp. gram.:f.}
\end{itemize}
\begin{itemize}
\item {Proveniência:(Lat. \textunderscore temperies\textunderscore )}
\end{itemize}
Temperamento; temperatura.
\section{Temperilha}
\begin{itemize}
\item {Grp. gram.:f.}
\end{itemize}
\begin{itemize}
\item {Utilização:Fig.}
\end{itemize}
\begin{itemize}
\item {Proveniência:(De \textunderscore temperar\textunderscore )}
\end{itemize}
Coisa que tempéra.
Meio de moderar a má disposição de alguém.
\section{Temperilho}
\begin{itemize}
\item {Grp. gram.:m.}
\end{itemize}
\begin{itemize}
\item {Utilização:Veter.}
\end{itemize}
\begin{itemize}
\item {Utilização:Des.}
\end{itemize}
\begin{itemize}
\item {Proveniência:(De \textunderscore temperar\textunderscore )}
\end{itemize}
Govêrno das rédeas.
Modo de as governar.
Tempêro ordinário.
Mistura proporcionada de alimentos appetitosos e medicamentos, para se dar a animaes doentes. Cf. Mac. Pinto, \textunderscore Comp. de Veter.\textunderscore , II, 31.
Jeito para fazer qualquer coisa com acêrto.
\section{Tempêro}
\begin{itemize}
\item {Grp. gram.:m.}
\end{itemize}
\begin{itemize}
\item {Proveniência:(De \textunderscore temperar\textunderscore )}
\end{itemize}
Substância, com que se aduba a comida.
Estado da comida adubada.
Meio de dirigir ou effectuar uma negociação.
Remédio.
Palliativo.
Têmpera.
\section{Tempestade}
\begin{itemize}
\item {Grp. gram.:f.}
\end{itemize}
\begin{itemize}
\item {Utilização:Fig.}
\end{itemize}
\begin{itemize}
\item {Proveniência:(Do lat. \textunderscore tempestas\textunderscore )}
\end{itemize}
Violenta agitação atmosphérica, muitas vezes acompanhada de chuva, granizo, trovões.
Grande estrondo.
Agitação do espírito.
Agitação moral; grande perturbação.
\section{Tempestear}
\begin{itemize}
\item {Grp. gram.:v. t.}
\end{itemize}
\begin{itemize}
\item {Grp. gram.:V. i.}
\end{itemize}
\begin{itemize}
\item {Proveniência:(De \textunderscore tempestade\textunderscore )}
\end{itemize}
Agitar.
Maltratar.
Fazer grande estrondo.
\section{Tempestivamente}
\begin{itemize}
\item {Grp. gram.:adv.}
\end{itemize}
De modo tempestivo.
Opportunamente; em occasião própria.
\section{Tempestivo}
\begin{itemize}
\item {Grp. gram.:adj.}
\end{itemize}
\begin{itemize}
\item {Proveniência:(Lat. \textunderscore tempestivus\textunderscore )}
\end{itemize}
Que vem ou succede no tempo próprio.
Opportuno.
\section{Tempestuar}
\begin{itemize}
\item {Grp. gram.:v. i.}
\end{itemize}
\begin{itemize}
\item {Proveniência:(De \textunderscore tempestade\textunderscore )}
\end{itemize}
Tempestear; enfurecer-se. Cf. Camillo, \textunderscore Noites de Insómn.\textunderscore , XII, 18.
\section{Tempestuosamente}
\begin{itemize}
\item {Grp. gram.:adv.}
\end{itemize}
De modo tempestuoso.
\section{Tempestuosidade}
\begin{itemize}
\item {Grp. gram.:f.}
\end{itemize}
Qualidade do que é tempestuoso.
\section{Tempestuoso}
\begin{itemize}
\item {Grp. gram.:adj.}
\end{itemize}
\begin{itemize}
\item {Utilização:Fig.}
\end{itemize}
\begin{itemize}
\item {Proveniência:(Lat. \textunderscore tempestuosus\textunderscore )}
\end{itemize}
Que traz tempestade.
Em que há tempestade; procelloso.
Sujeito a tempestades.
Violento; muito agitado.
\section{Tempíssimo}
\begin{itemize}
\item {Grp. gram.:m.}
\end{itemize}
\begin{itemize}
\item {Utilização:T. da Bairrada}
\end{itemize}
\begin{itemize}
\item {Proveniência:(De \textunderscore tempo\textunderscore . Cp. \textunderscore coisíssima\textunderscore  e \textunderscore verdadíssima\textunderscore )}
\end{itemize}
Muito tempo: \textunderscore já passou tempíssimo, desde que elle me escreveu\textunderscore .
\section{Tempista}
\begin{itemize}
\item {Grp. gram.:m.}
\end{itemize}
\begin{itemize}
\item {Utilização:Mús.}
\end{itemize}
Músico, que toca bem a tempo.
\section{Templário}
\begin{itemize}
\item {Grp. gram.:m.}
\end{itemize}
\begin{itemize}
\item {Proveniência:(De \textunderscore templo\textunderscore )}
\end{itemize}
Cavalleiro da Ordem militar, instituída em Jerusalém em 1118, perto do lugar onde estivera o templo de Salomão.
\section{Temple}
\begin{itemize}
\item {Grp. gram.:m.}
\end{itemize}
\begin{itemize}
\item {Utilização:Ant.}
\end{itemize}
\begin{itemize}
\item {Proveniência:(Fr. \textunderscore temple\textunderscore )}
\end{itemize}
A Ordem militar dos Templários.
\section{Templeiro}
\begin{itemize}
\item {Grp. gram.:m.}
\end{itemize}
\begin{itemize}
\item {Utilização:Des.}
\end{itemize}
O mesmo que \textunderscore templário\textunderscore .
\section{Templetónia}
\begin{itemize}
\item {Grp. gram.:f.}
\end{itemize}
Gênero de plantas leguminosas.
\section{Templo}
\begin{itemize}
\item {Grp. gram.:m.}
\end{itemize}
\begin{itemize}
\item {Utilização:Fig.}
\end{itemize}
\begin{itemize}
\item {Proveniência:(Lat. \textunderscore templum\textunderscore )}
\end{itemize}
Lugar descoberto e elevado, consagrado pelos áugures, entre os Romanos.
Edifício público, consagrado ao culto religioso.
Igreja.
Sala, em que se celebram as sessões da Maçonaria.
A Ordem militar dos Templários.
Lugar mysterioso e respeitável.
Recordação perenne das acções memoráveis.
\section{Tempo}
\begin{itemize}
\item {Grp. gram.:m.}
\end{itemize}
\begin{itemize}
\item {Utilização:Mús.}
\end{itemize}
\begin{itemize}
\item {Utilização:Gram.}
\end{itemize}
\begin{itemize}
\item {Grp. gram.:Loc. adv.}
\end{itemize}
\begin{itemize}
\item {Grp. gram.:Loc. adv.}
\end{itemize}
\begin{itemize}
\item {Utilização:fam.}
\end{itemize}
\begin{itemize}
\item {Utilização:Prov.}
\end{itemize}
\begin{itemize}
\item {Utilização:minh.}
\end{itemize}
\begin{itemize}
\item {Proveniência:(Lat. \textunderscore tempus\textunderscore )}
\end{itemize}
Duração calculável das coisas.
Duração limitada, em opposição á eternidade.
Successão de dias, horas, momentos.
Período: \textunderscore o tempo das Cruzadas\textunderscore .
Época actual.
Estado atmosphérico: \textunderscore o tempo está ennevoado\textunderscore .
Os séculos.
Época notável que se preannunciou.
Ensejo, conjunctura: \textunderscore agora é tempo de se apurar o caso\textunderscore .
Estação própria.
Occasião própria.
Movimento, parte de um movimento.
Cada uma das inflexões, que indicam nos verbos o momento a que se refere o estado ou a acção.
\textunderscore A tempo\textunderscore , opportunamente; nem cedo nem tarde.
\textunderscore No tempo dos Affonsinhos\textunderscore , em época muito remota.
\textunderscore Peixe do tempo\textunderscore , peixe de salmoira.
\section{Temporada}
\begin{itemize}
\item {Grp. gram.:f.}
\end{itemize}
\begin{itemize}
\item {Proveniência:(Do lat. \textunderscore tempus\textunderscore , \textunderscore temporis\textunderscore )}
\end{itemize}
Grande ou certo espaço de tempo: \textunderscore a temporada dos banhos\textunderscore .
\section{Temporal}
\begin{itemize}
\item {Grp. gram.:adj.}
\end{itemize}
\begin{itemize}
\item {Utilização:Anat.}
\end{itemize}
\begin{itemize}
\item {Grp. gram.:M.}
\end{itemize}
\begin{itemize}
\item {Proveniência:(Lat. \textunderscore temporalis\textunderscore )}
\end{itemize}
Temporário, que passa com o tempo.
Que dura certo tempo.
Profano; mundano: \textunderscore interesses temporaes\textunderscore .
Relativo ás fontes da cabeça: \textunderscore região temporal\textunderscore .
O mesmo que \textunderscore tempestade\textunderscore .
Região temporal do crânio, entre o ôlho e a orelha.
\section{Temporalidade}
\begin{itemize}
\item {Grp. gram.:f.}
\end{itemize}
\begin{itemize}
\item {Grp. gram.:Pl.}
\end{itemize}
\begin{itemize}
\item {Proveniência:(Do lat. \textunderscore temporalitas\textunderscore )}
\end{itemize}
Qualidade do que é temporal ou provisório.
Interinidade.
Prebendas; rendimentos ecclesiásticos.
\section{Temporalizar}
\begin{itemize}
\item {Grp. gram.:v. t.}
\end{itemize}
Tornar temporal; secularizar.
\section{Temporalmente}
\begin{itemize}
\item {Grp. gram.:adv.}
\end{itemize}
De modo temporal; temporariamente.
\section{Temporâneo}
\begin{itemize}
\item {Grp. gram.:adj.}
\end{itemize}
\begin{itemize}
\item {Proveniência:(Lat. \textunderscore temporaneus\textunderscore )}
\end{itemize}
O mesmo que \textunderscore temporário\textunderscore .
\section{Temporanmente}
\begin{itemize}
\item {Grp. gram.:adv.}
\end{itemize}
De modo temporão.
Antes do tempo próprio; antecipadamente.
\section{Temporão}
\begin{itemize}
\item {Grp. gram.:adj.}
\end{itemize}
\begin{itemize}
\item {Grp. gram.:M.}
\end{itemize}
\begin{itemize}
\item {Proveniência:(De \textunderscore temporâneo\textunderscore )}
\end{itemize}
Que vem ou succede antes do tempo apropriado.
Que amadurece primeiro que outros, (falando-se de frutos).
Prematuro.
\textunderscore Temporão de Coruche\textunderscore , variedade de trigo molle.
\section{Temporariamente}
\begin{itemize}
\item {Grp. gram.:adv.}
\end{itemize}
De modo temporário.
Provisoriamente; interinamente.
\section{Temporário}
\begin{itemize}
\item {Grp. gram.:adj.}
\end{itemize}
\begin{itemize}
\item {Proveniência:(Lat. \textunderscore temporarius\textunderscore )}
\end{itemize}
Que dura certo tempo.
Transitório; provisório.
Relativo a tempo.
\section{Têmporas}
\begin{itemize}
\item {Grp. gram.:f. pl.}
\end{itemize}
\begin{itemize}
\item {Proveniência:(Lat. \textunderscore tempora\textunderscore )}
\end{itemize}
Os três dias de jejum, que há numa semana, em cada estação do anno, segundo o rito cathólico.
\section{Temporejar}
\begin{itemize}
\item {Grp. gram.:v. i.}
\end{itemize}
\begin{itemize}
\item {Utilização:Prov.}
\end{itemize}
\begin{itemize}
\item {Utilização:trasm.}
\end{itemize}
Diz-se da coisa que apparece ou nasce ao mesmo tempo que outra: \textunderscore êste anno o centeio temporijou com a cevada\textunderscore .
\section{Temporização}
\begin{itemize}
\item {Grp. gram.:f.}
\end{itemize}
Acto ou effeito de temporizar.
\section{Temporizador}
\begin{itemize}
\item {Grp. gram.:m.  e  adj.}
\end{itemize}
O que temporiza.
\section{Temporizamento}
\begin{itemize}
\item {Grp. gram.:m.}
\end{itemize}
O mesmo que \textunderscore temporização\textunderscore .
\section{Temporizar}
\begin{itemize}
\item {Grp. gram.:v. t.}
\end{itemize}
\begin{itemize}
\item {Grp. gram.:V. i.}
\end{itemize}
\begin{itemize}
\item {Proveniência:(Do lat. \textunderscore tempus\textunderscore , \textunderscore temporis\textunderscore )}
\end{itemize}
Adiar, demorar.
Esperar outra occasião.
Contemporizar; condescender.
\section{Temporo-auricular}
\begin{itemize}
\item {Grp. gram.:adj.}
\end{itemize}
\begin{itemize}
\item {Utilização:Anat.}
\end{itemize}
Diz-se de um dos músculos da orelha.
\section{Temporo-maxillar}
\begin{itemize}
\item {Grp. gram.:adj.}
\end{itemize}
\begin{itemize}
\item {Utilização:Anat.}
\end{itemize}
Diz-se de um músculo, que pertence ao osso temporal e ao maxillar.
\section{Temposa}
\begin{itemize}
\item {Grp. gram.:f.}
\end{itemize}
\begin{itemize}
\item {Utilização:Gír.}
\end{itemize}
O mesmo que \textunderscore tamposa\textunderscore .
\section{Tempre}
\begin{itemize}
\item {Grp. gram.:m.}
\end{itemize}
\begin{itemize}
\item {Utilização:Ant.}
\end{itemize}
O mesmo que \textunderscore temple\textunderscore .
\section{Tempreiro}
\begin{itemize}
\item {Grp. gram.:m.}
\end{itemize}
\begin{itemize}
\item {Proveniência:(De \textunderscore tempre\textunderscore )}
\end{itemize}
O mesmo que \textunderscore templeiro\textunderscore .
\section{Tempro}
\begin{itemize}
\item {Grp. gram.:m.}
\end{itemize}
\begin{itemize}
\item {Utilização:Ant.}
\end{itemize}
O mesmo que \textunderscore templo\textunderscore . Cf. \textunderscore Port. Mon. Hist.\textunderscore , \textunderscore Script.\textunderscore , 246.
\section{Tem-te-lá}
\begin{itemize}
\item {Grp. gram.:m.}
\end{itemize}
Nome, que em Penafiel se dá á codorniz.
\section{Tem-tem}
\begin{itemize}
\item {Grp. gram.:m.}
\end{itemize}
\begin{itemize}
\item {Utilização:Fam.}
\end{itemize}
\begin{itemize}
\item {Proveniência:(De \textunderscore têr\textunderscore )}
\end{itemize}
Equilíbrio, em relação ás crianças que dão os primeiros passos.
\section{Tem-te-na-raiz}
\begin{itemize}
\item {Grp. gram.:m.}
\end{itemize}
\begin{itemize}
\item {Proveniência:(T. onom., por imitação do canto dessa ave)}
\end{itemize}
O mesmo que \textunderscore trigueirão\textunderscore .
\section{Temudo}
\begin{itemize}
\item {Grp. gram.:adj.}
\end{itemize}
\begin{itemize}
\item {Utilização:Ant.}
\end{itemize}
O mesmo que \textunderscore temido\textunderscore ^1.
\section{Temulência}
\begin{itemize}
\item {Grp. gram.:f.}
\end{itemize}
\begin{itemize}
\item {Proveniência:(Lat. \textunderscore temulentia\textunderscore )}
\end{itemize}
Estado do que é temulento.
Estado mórbido, semelhante á embriaguez.
\section{Temulento}
\begin{itemize}
\item {Grp. gram.:adj.}
\end{itemize}
\begin{itemize}
\item {Proveniência:(Lat. \textunderscore temulentus\textunderscore )}
\end{itemize}
Bêbedo, ébrio.
Em que há orgias ou scenas de embriaguez.
\section{Tenace}
\begin{itemize}
\item {Grp. gram.:adj.}
\end{itemize}
\begin{itemize}
\item {Utilização:Des.}
\end{itemize}
O mesmo que \textunderscore tenaz\textunderscore .
\section{Tenacidade}
\begin{itemize}
\item {Grp. gram.:f.}
\end{itemize}
\begin{itemize}
\item {Utilização:Fig.}
\end{itemize}
\begin{itemize}
\item {Proveniência:(Lat. \textunderscore tenacitas\textunderscore )}
\end{itemize}
Qualidade do que é tenaz.
Contumácia.
Apêgo.
Avareza.
\section{Tenáculo}
\begin{itemize}
\item {Grp. gram.:m.}
\end{itemize}
\begin{itemize}
\item {Utilização:Med.}
\end{itemize}
\begin{itemize}
\item {Proveniência:(Do lat. \textunderscore tenere\textunderscore )}
\end{itemize}
Instrumento em fórma de agulha curva, com que se tiravam os vasos que se queriam ligar.
\section{Tenador}
\begin{itemize}
\item {Grp. gram.:m.}
\end{itemize}
\begin{itemize}
\item {Utilização:Prov.}
\end{itemize}
\begin{itemize}
\item {Utilização:trasm.}
\end{itemize}
O mesmo que \textunderscore garfo\textunderscore .
(Cp. cast. \textunderscore tenedor\textunderscore )
\section{Tenalgia}
\begin{itemize}
\item {Grp. gram.:f.}
\end{itemize}
\begin{itemize}
\item {Utilização:Med.}
\end{itemize}
\begin{itemize}
\item {Proveniência:(Do gr. \textunderscore tenon\textunderscore  + \textunderscore algos\textunderscore )}
\end{itemize}
Dôr nos tendões.
\section{Tenalha}
\begin{itemize}
\item {Grp. gram.:f.}
\end{itemize}
\begin{itemize}
\item {Proveniência:(Fr. \textunderscore tenaille\textunderscore )}
\end{itemize}
Pequena obra de duas faces, nas fortalezas, apresentando um ângulo reentrante para o lado do campo.
\section{Tenalhão}
\begin{itemize}
\item {Grp. gram.:m.}
\end{itemize}
\begin{itemize}
\item {Proveniência:(De \textunderscore tenalha\textunderscore )}
\end{itemize}
Obra de fortificação, que assenta algumas vezes em cada uma das faces de uma meia-lua.
\section{Tenalina}
\begin{itemize}
\item {Grp. gram.:f.}
\end{itemize}
Medicamento vermífugo.
\section{Tenaria}
\begin{itemize}
\item {Grp. gram.:f.}
\end{itemize}
\begin{itemize}
\item {Utilização:Ant.}
\end{itemize}
Officina de curtidor de pelles.
(Por \textunderscore tanaria\textunderscore , de um hyp. \textunderscore tanar\textunderscore , do fr. \textunderscore tanner\textunderscore )
\section{Tenaz}
\begin{itemize}
\item {Grp. gram.:adj.}
\end{itemize}
\begin{itemize}
\item {Utilização:Fig.}
\end{itemize}
\begin{itemize}
\item {Grp. gram.:F.}
\end{itemize}
\begin{itemize}
\item {Proveniência:(Lat. \textunderscore tenax\textunderscore )}
\end{itemize}
Que está muito adherente.
Que tem grande cohesão.
Viscoso.
Que segura com firmeza.
Aferrado.
Pertinaz.
Firme; constante.
Obstinado.
Sovina, avarento.
Instrumento de metal, composto de duas lâminas, que se alargam e se apertam para agarrar ou arrancar qualquer objecto; pinça.
\section{Tenazmente}
\begin{itemize}
\item {Grp. gram.:adv.}
\end{itemize}
De modo tenaz; com afinco; com pertinácia.
\section{Tenca}
\begin{itemize}
\item {Grp. gram.:f.}
\end{itemize}
\begin{itemize}
\item {Proveniência:(Do lat. \textunderscore tinca\textunderscore )}
\end{itemize}
Peixe cyprinóide de água doce.
Taínha dos rios.
\section{Tença}
\begin{itemize}
\item {Grp. gram.:f.}
\end{itemize}
\begin{itemize}
\item {Utilização:Ant.}
\end{itemize}
Pensão, com que se remuneram serviços.
Acto de têr.
\section{Tenção}
\begin{itemize}
\item {Grp. gram.:f.}
\end{itemize}
\begin{itemize}
\item {Utilização:Jur.}
\end{itemize}
\begin{itemize}
\item {Proveniência:(Do lat. \textunderscore tentus\textunderscore )}
\end{itemize}
Resolução, intento, plano.
Devoção.
Parecer escrito e fundamentado dos juízes de segunda instância, no julgamento de algumas causas.
Divisa de brasão, relativo a feitos gloriosos.
Contenda, briga.
Composição poética, em que dois ou mais trovadores contendiam, injuriando-se por vezes.
\section{Tenceiro}
\begin{itemize}
\item {Grp. gram.:m.}
\end{itemize}
\begin{itemize}
\item {Utilização:Ant.}
\end{itemize}
Cobrador de tenças.
\section{Tencionar}
\begin{itemize}
\item {Grp. gram.:v. t.}
\end{itemize}
\begin{itemize}
\item {Grp. gram.:V. i.}
\end{itemize}
\begin{itemize}
\item {Utilização:Jur.}
\end{itemize}
Fazer tenção de; planear, projectar.
Escrever tenção, em processos forenses.
\section{Tencionário}
\begin{itemize}
\item {Grp. gram.:m.}
\end{itemize}
Aquelle que recebe tença.
\section{Tencioneiro}
\begin{itemize}
\item {Grp. gram.:adj.}
\end{itemize}
\begin{itemize}
\item {Proveniência:(De \textunderscore tenção\textunderscore )}
\end{itemize}
Que anda desavindo com alguém.
Teimoso, pertinaz.
\section{Tençoeiro}
\begin{itemize}
\item {Grp. gram.:adj.}
\end{itemize}
\begin{itemize}
\item {Utilização:Des.}
\end{itemize}
\begin{itemize}
\item {Proveniência:(De \textunderscore tenção\textunderscore )}
\end{itemize}
Que anda desavindo com alguém.
Teimoso, pertinaz.
\section{Tenda}
\begin{itemize}
\item {Grp. gram.:f.}
\end{itemize}
Barraca de campanha.
Pequeno estabelecimento de merceeiro.
Barraca de feira, para venda de quinquilharias.
Caixa com tampa de vidro, suspensa ao pescoço por corda ou correia, e em que o tendeiro ambulante traz quinquilharias.
(B. lat. \textunderscore tenda\textunderscore )
\section{Tendaes}
\begin{itemize}
\item {Grp. gram.:m. pl.}
\end{itemize}
\begin{itemize}
\item {Utilização:Prov.}
\end{itemize}
\begin{itemize}
\item {Utilização:alg.}
\end{itemize}
Varaes, assentes na extremidade dos fuselos.
(Cp. \textunderscore tendal\textunderscore ^1)
\section{Tendais}
\begin{itemize}
\item {Grp. gram.:m. pl.}
\end{itemize}
\begin{itemize}
\item {Utilização:Prov.}
\end{itemize}
\begin{itemize}
\item {Utilização:alg.}
\end{itemize}
Varaes, assentes na extremidade dos fuselos.
(Cp. \textunderscore tendal\textunderscore ^1)
\section{Tendal}
\begin{itemize}
\item {Grp. gram.:m.}
\end{itemize}
\begin{itemize}
\item {Utilização:Prov.}
\end{itemize}
\begin{itemize}
\item {Utilização:minh.}
\end{itemize}
\begin{itemize}
\item {Utilização:Prov.}
\end{itemize}
\begin{itemize}
\item {Utilização:trasm.}
\end{itemize}
Tolda fixa na primeira coberta de um navio.
Lugar, onde se assentam as fôrmas, nos engenhos de açúcar.
Recinto adjunto á casa de habitação.
Peça de madeira, em que se encaixa a parte superior dos fueiros.
Pano branco de linho, com que se cobre a massa do pão, até levedar.
(B. lat. \textunderscore tendalis\textunderscore )
\section{Tendal}
\begin{itemize}
\item {Grp. gram.:m.}
\end{itemize}
Lugar, onde se tosquiam ovelhas.
(Por \textunderscore tondal\textunderscore , que presuppõe \textunderscore tonda\textunderscore , termo existente ainda na chorographia nacional, e que póde sêr o substantivo verbal de um hyp. \textunderscore tonder\textunderscore , do lat. \textunderscore tondere\textunderscore , tosquiar)
\section{Tendão}
\begin{itemize}
\item {Grp. gram.:m.}
\end{itemize}
\begin{itemize}
\item {Utilização:Anat.}
\end{itemize}
\begin{itemize}
\item {Proveniência:(Do lat. \textunderscore tendo\textunderscore , \textunderscore tendonis\textunderscore )}
\end{itemize}
Feixe de fibras, mais ou menos longo, e ordinariamente achatado, situado na extremidade dos músculos, dos quaes se distingue pela natureza das suas fibras e porque não é contráctil.
\section{Tendedeira}
\begin{itemize}
\item {Grp. gram.:f.}
\end{itemize}
\begin{itemize}
\item {Proveniência:(De \textunderscore tender\textunderscore )}
\end{itemize}
Tábua, em que se tende o pão que se há de cozer.
\section{Tendedura}
\begin{itemize}
\item {Grp. gram.:f.}
\end{itemize}
Acto ou effeito de tender. Cf. C. Guerreiro, \textunderscore Diccion. de Cons.\textunderscore , 416.
\section{Tendeira}
\begin{itemize}
\item {Grp. gram.:f.}
\end{itemize}
\begin{itemize}
\item {Proveniência:(De \textunderscore tendeiro\textunderscore )}
\end{itemize}
Mulhér, que vende em tenda.
Dona de tenda.
Mulhér de tendeiro.
\section{Tendeiro}
\begin{itemize}
\item {Grp. gram.:m.}
\end{itemize}
\begin{itemize}
\item {Utilização:Pop.}
\end{itemize}
\begin{itemize}
\item {Proveniência:(Do b. lat. \textunderscore tendarius\textunderscore )}
\end{itemize}
Aquelle que vende em tenda.
Dono de tenda.
O diabo.
\section{Tendência}
\begin{itemize}
\item {Grp. gram.:f.}
\end{itemize}
\begin{itemize}
\item {Proveniência:(De \textunderscore tendente\textunderscore )}
\end{itemize}
Propensão, inclinação.
Fôrça, que determina o movimento de um corpo.
Disposição.
Intenção.
Vocação.
\section{Tendencioso}
\begin{itemize}
\item {Grp. gram.:adj.}
\end{itemize}
\begin{itemize}
\item {Utilização:Neol.}
\end{itemize}
Que mostra tendência ou propósito de desagradar ou prejudicar.
Malévolo: \textunderscore boato tendencioso\textunderscore .
\section{Tendente}
\begin{itemize}
\item {Grp. gram.:adj.}
\end{itemize}
\begin{itemize}
\item {Proveniência:(Lat. \textunderscore tendens\textunderscore )}
\end{itemize}
Que tende.
Que tem vocação; que se inclina.
\section{Tendente}
\begin{itemize}
\item {Grp. gram.:m.}
\end{itemize}
Planta malvácea de Cabo-Verde.
\section{Tender}
\begin{itemize}
\item {Grp. gram.:v. t.}
\end{itemize}
\begin{itemize}
\item {Utilização:Ant.}
\end{itemize}
\begin{itemize}
\item {Grp. gram.:V. i.}
\end{itemize}
\begin{itemize}
\item {Proveniência:(Do lat. \textunderscore tendere\textunderscore )}
\end{itemize}
Estender.
Desfraldar.
Bater e arredondar na masseira ou numa tigela (o pão que se vai cozer).
O mesmo que \textunderscore entender\textunderscore .
Têr vocação; inclinar-se; propender.
Destinar-se.
Dispor-se.
Aspirar.
Aproximar-se.
\section{Tendilha}
\begin{itemize}
\item {Grp. gram.:f.}
\end{itemize}
\begin{itemize}
\item {Utilização:Prov.}
\end{itemize}
\begin{itemize}
\item {Utilização:minh.}
\end{itemize}
Pequena tenda.
Espécie de agulheta de ferro, que prende os arcos de alguns carros de bois á ensogadura do jugo.
\section{Tendilhão}
\begin{itemize}
\item {Grp. gram.:m.}
\end{itemize}
\begin{itemize}
\item {Proveniência:(De \textunderscore tendilha\textunderscore )}
\end{itemize}
Barraca de campanha.
\section{Tendilhão}
\begin{itemize}
\item {Grp. gram.:m.}
\end{itemize}
(Corr. de \textunderscore tentilhão\textunderscore )
\section{Tendinoso}
\begin{itemize}
\item {Grp. gram.:adj.}
\end{itemize}
\begin{itemize}
\item {Proveniência:(Fr. \textunderscore tendineux\textunderscore )}
\end{itemize}
Relativo aos tendões.
\section{Tendola}
\begin{itemize}
\item {Grp. gram.:f.}
\end{itemize}
\begin{itemize}
\item {Utilização:Pop.}
\end{itemize}
Tenda reles.
\section{Tene}
\begin{itemize}
\item {Grp. gram.:m.}
\end{itemize}
Árvore do Congo.
\section{Tênebra}
\begin{itemize}
\item {Grp. gram.:f.}
\end{itemize}
\begin{itemize}
\item {Utilização:Des.}
\end{itemize}
\begin{itemize}
\item {Proveniência:(Lat. \textunderscore tenebrae\textunderscore )}
\end{itemize}
O mesmo que \textunderscore trevas\textunderscore . Cf. Usque, 20, 29 v.^o e 48.
\section{Tenebrário}
\begin{itemize}
\item {Grp. gram.:m.}
\end{itemize}
\begin{itemize}
\item {Proveniência:(Lat. \textunderscore tenebrarius\textunderscore )}
\end{itemize}
Candeeiro, que está acceso durante o offício de trevas, na Semana Santa.
\section{Tenebrião}
\begin{itemize}
\item {Grp. gram.:m.}
\end{itemize}
\begin{itemize}
\item {Proveniência:(Lat. \textunderscore tenebrio\textunderscore )}
\end{itemize}
Gênero de insectos coleópteros.
\section{Tenebricosidade}
\begin{itemize}
\item {Grp. gram.:f.}
\end{itemize}
\begin{itemize}
\item {Proveniência:(Do lat. \textunderscore tenebricositas\textunderscore )}
\end{itemize}
Qualidade do que é tenebricoso.
\section{Tenebricoso}
\begin{itemize}
\item {Grp. gram.:adj.}
\end{itemize}
\begin{itemize}
\item {Utilização:Des.}
\end{itemize}
\begin{itemize}
\item {Proveniência:(Lat. \textunderscore tenebricosus\textunderscore )}
\end{itemize}
Obscuro.
Perturbado de entendimento.
\section{Tenebrizador}
\begin{itemize}
\item {Grp. gram.:m.}
\end{itemize}
\begin{itemize}
\item {Proveniência:(De \textunderscore tênebra\textunderscore )}
\end{itemize}
Amigo das trevas, do obscurantismo. Cf. Castilho, \textunderscore Tosquia\textunderscore .
\section{Tenebrosidade}
\begin{itemize}
\item {Grp. gram.:f.}
\end{itemize}
\begin{itemize}
\item {Proveniência:(Do lat. \textunderscore tenebrositas\textunderscore )}
\end{itemize}
Qualidade do que é tenebroso.
\section{Tenebroso}
\begin{itemize}
\item {Grp. gram.:adj.}
\end{itemize}
\begin{itemize}
\item {Utilização:Fig.}
\end{itemize}
\begin{itemize}
\item {Proveniência:(Lat. \textunderscore tenebrosus\textunderscore )}
\end{itemize}
Cheio ou coberto de trevas.
Escuro; caliginoso.
Terrível.
Indigno, criminoso: \textunderscore procedimento tenebroso\textunderscore .
Afflictivo; pungente.
Medonho.
\section{Tenedeira}
\begin{itemize}
\item {Grp. gram.:f.}
\end{itemize}
Casta de uva branca algarvia.
\section{Tenência}
\begin{itemize}
\item {Grp. gram.:f.}
\end{itemize}
\begin{itemize}
\item {Utilização:Bras}
\end{itemize}
Cargo de tenente.
Habitação de tenente.
Antiga repartição do tenente-general de artilharia.
Vigor; firmeza.
\section{Tenente}
\begin{itemize}
\item {Grp. gram.:m.}
\end{itemize}
\begin{itemize}
\item {Utilização:Prov.}
\end{itemize}
\begin{itemize}
\item {Utilização:trasm.}
\end{itemize}
\begin{itemize}
\item {Grp. gram.:Adj. Loc. adv.}
\end{itemize}
\begin{itemize}
\item {Proveniência:(Lat. \textunderscore tenens\textunderscore )}
\end{itemize}
Aquelle que substitue um chefe, na ausência dêste.
Pôsto militar, immediatamente inferior ao de capitão.
Homem, a quem a mulhér é infiel. Cp. a expressão pop. \textunderscore tenente-cornel\textunderscore .
\textunderscore Á mão tenente\textunderscore , á queima-roupa, muito de perto.
\section{Tenente-coronel}
\begin{itemize}
\item {Grp. gram.:m.}
\end{itemize}
Official do exército, de graduação immediatamente inferior á de coronel.
\section{Tenente-general}
\begin{itemize}
\item {Grp. gram.:m.}
\end{itemize}
\begin{itemize}
\item {Utilização:Des.}
\end{itemize}
Official do exército, que tinha graduação immediatamente inferior á de general.
\section{Tenente-rei}
\begin{itemize}
\item {Grp. gram.:m.}
\end{itemize}
\begin{itemize}
\item {Utilização:Ant.}
\end{itemize}
Governador de praça forte.
\section{Teneótico}
\begin{itemize}
\item {Grp. gram.:adj.}
\end{itemize}
Dizia-se de certa qualidade de papel, fabricado em Alexandria. Cf. Castilho, \textunderscore Fastos\textunderscore , I, 312.
\section{Tenesmo}
\begin{itemize}
\item {fónica:nês}
\end{itemize}
\begin{itemize}
\item {Grp. gram.:m.}
\end{itemize}
\begin{itemize}
\item {Utilização:Med.}
\end{itemize}
\begin{itemize}
\item {Proveniência:(Lat. \textunderscore tenesmus\textunderscore )}
\end{itemize}
Sentimento doloroso, na bexiga ou na região anal, com desejo contínuo, mas quási inútil, de urinar ou evacuar.
\section{Tenesmódico}
\begin{itemize}
\item {Grp. gram.:adj.}
\end{itemize}
\begin{itemize}
\item {Proveniência:(Do gr. \textunderscore teinesmodes\textunderscore )}
\end{itemize}
Acompanhado de tenesmo.
\section{Tengarra}
\begin{itemize}
\item {Grp. gram.:f.}
\end{itemize}
\begin{itemize}
\item {Grp. gram.:f.}
\end{itemize}
\begin{itemize}
\item {Utilização:Prov.}
\end{itemize}
\begin{itemize}
\item {Utilização:alg.}
\end{itemize}
(V.tangarra)
Planta, de que se faz salada.
\section{Tenhosa}
\begin{itemize}
\item {Grp. gram.:f.}
\end{itemize}
Pássaro do mar da Índia. Cf. \textunderscore Hist. Trág. Marit.\textunderscore , 175.
\section{Tênia}
\begin{itemize}
\item {Grp. gram.:f.}
\end{itemize}
\begin{itemize}
\item {Proveniência:(Lat. \textunderscore taenia\textunderscore )}
\end{itemize}
Gênero de vermes intestinaes, de corpo chato e comprido, composto de um grande número de anéis articulados; solitária.
\section{Teníase}
\begin{itemize}
\item {Grp. gram.:f.}
\end{itemize}
Doença, produzida pela tênia.
\section{Tenífugo}
\begin{itemize}
\item {Grp. gram.:adj.}
\end{itemize}
\begin{itemize}
\item {Proveniência:(Do lat. \textunderscore taenia\textunderscore  + \textunderscore fugare\textunderscore )}
\end{itemize}
Diz-se do medicamento, destinado a expulsar a tênia.
\section{Teniobrânchio}
\begin{itemize}
\item {fónica:qui}
\end{itemize}
\begin{itemize}
\item {Grp. gram.:adj.}
\end{itemize}
\begin{itemize}
\item {Utilização:Zool.}
\end{itemize}
\begin{itemize}
\item {Proveniência:(Do gr. \textunderscore tainia\textunderscore  + \textunderscore brankhia\textunderscore )}
\end{itemize}
Que tem as brânchias em fórma de fita.
\section{Teniobrânquio}
\begin{itemize}
\item {Grp. gram.:adj.}
\end{itemize}
\begin{itemize}
\item {Utilização:Zool.}
\end{itemize}
\begin{itemize}
\item {Proveniência:(Do gr. \textunderscore tainia\textunderscore  + \textunderscore brankhia\textunderscore )}
\end{itemize}
Que tem as brânquias em fórma de fita.
\section{Teniocarpo}
\begin{itemize}
\item {Grp. gram.:m.}
\end{itemize}
\begin{itemize}
\item {Proveniência:(Do gr. \textunderscore tainia\textunderscore  + \textunderscore karpos\textunderscore )}
\end{itemize}
Planta trepadeira da América tropical.
\section{Tenióide}
\begin{itemize}
\item {Grp. gram.:adj.}
\end{itemize}
\begin{itemize}
\item {Grp. gram.:M. Pl.}
\end{itemize}
\begin{itemize}
\item {Proveniência:(Do gr. \textunderscore tainia\textunderscore  + \textunderscore eidos\textunderscore )}
\end{itemize}
Semelhante á tênia.
Peixes acanthopterýgios, de feitio análogo ao da tênia.
\section{Teniope}
\begin{itemize}
\item {Grp. gram.:adj.}
\end{itemize}
\begin{itemize}
\item {Utilização:Zool.}
\end{itemize}
\begin{itemize}
\item {Proveniência:(Do gr. \textunderscore tainia\textunderscore  + \textunderscore ops\textunderscore )}
\end{itemize}
Que apresenta nos olhos listras de côr.
\section{Tenióptero}
\begin{itemize}
\item {Grp. gram.:adj.}
\end{itemize}
\begin{itemize}
\item {Utilização:Zool.}
\end{itemize}
\begin{itemize}
\item {Proveniência:(Do gr. \textunderscore tainia\textunderscore  + \textunderscore pteron\textunderscore )}
\end{itemize}
Que tem listras de côr nas asas ou barbatanas.
\section{Teniosomo}
\begin{itemize}
\item {fónica:so}
\end{itemize}
\begin{itemize}
\item {Grp. gram.:adj.}
\end{itemize}
\begin{itemize}
\item {Utilização:Zool.}
\end{itemize}
\begin{itemize}
\item {Proveniência:(Do gr. \textunderscore tainia\textunderscore  + \textunderscore soma\textunderscore )}
\end{itemize}
Que tem o corpo em fórma de fita.
\section{Teniossomo}
\begin{itemize}
\item {Grp. gram.:adj.}
\end{itemize}
\begin{itemize}
\item {Utilização:Zool.}
\end{itemize}
\begin{itemize}
\item {Proveniência:(Do gr. \textunderscore tainia\textunderscore  + \textunderscore soma\textunderscore )}
\end{itemize}
Que tem o corpo em fórma de fita.
\section{Teníoto}
\begin{itemize}
\item {Grp. gram.:adj.}
\end{itemize}
\begin{itemize}
\item {Utilização:Zool.}
\end{itemize}
\begin{itemize}
\item {Proveniência:(Do gr. \textunderscore tainia\textunderscore  + \textunderscore ous\textunderscore , \textunderscore otos\textunderscore )}
\end{itemize}
Que tem orelhas compridas e estreitas.
\section{Tenito}
\begin{itemize}
\item {Grp. gram.:m.}
\end{itemize}
\begin{itemize}
\item {Utilização:Miner.}
\end{itemize}
\begin{itemize}
\item {Proveniência:(Do gr. \textunderscore tainia\textunderscore )}
\end{itemize}
Combinação ferruginosa, que entra no ferro meteórico e que se apresenta em fórma de tiras delgadas.
\section{Tenjarro}
\begin{itemize}
\item {Grp. gram.:m.}
\end{itemize}
Pequeno pássaro, da fam. dos tordos, (\textunderscore saxicola aemanthe\textunderscore , Lin.).
(Cp. \textunderscore tanjasno\textunderscore )
\section{Tenonite}
\begin{itemize}
\item {Grp. gram.:f.}
\end{itemize}
\begin{itemize}
\item {Utilização:Med.}
\end{itemize}
\begin{itemize}
\item {Proveniência:(De \textunderscore Tenon\textunderscore , n. p.)}
\end{itemize}
Inflammação da cápsula de Tenon, no ôlho.
\section{Tenontagra}
\begin{itemize}
\item {Grp. gram.:f.}
\end{itemize}
\begin{itemize}
\item {Utilização:Med.}
\end{itemize}
Espécie de gota, que se localiza nos tendões largos.
\section{Tenoplastia}
\begin{itemize}
\item {Grp. gram.:f.}
\end{itemize}
\begin{itemize}
\item {Utilização:Med.}
\end{itemize}
\begin{itemize}
\item {Proveniência:(Do gr. \textunderscore tenon\textunderscore  + \textunderscore plassein\textunderscore )}
\end{itemize}
Enxêrto tendinoso.
\section{Tenor}
\begin{itemize}
\item {Grp. gram.:m.}
\end{itemize}
\begin{itemize}
\item {Utilização:Mús.}
\end{itemize}
\begin{itemize}
\item {Proveniência:(It. \textunderscore tenore\textunderscore )}
\end{itemize}
Voz de homem, mais alta que a de barýtono.
Pessôa, que tem essa voz.
\section{Tenor}
\begin{itemize}
\item {Grp. gram.:m.}
\end{itemize}
\begin{itemize}
\item {Utilização:t. de Turquel}
\end{itemize}
\begin{itemize}
\item {Utilização:Ant.}
\end{itemize}
\begin{itemize}
\item {Proveniência:(Lat. \textunderscore tenor\textunderscore )}
\end{itemize}
O mesmo que \textunderscore teor\textunderscore .
\section{Tenorhaphia}
\begin{itemize}
\item {Grp. gram.:f.}
\end{itemize}
\begin{itemize}
\item {Utilização:Med.}
\end{itemize}
\begin{itemize}
\item {Proveniência:(Do gr. \textunderscore tenon\textunderscore  + \textunderscore rhaphe\textunderscore )}
\end{itemize}
Sutura dos tendões.
\section{Tenorrafia}
\begin{itemize}
\item {Grp. gram.:f.}
\end{itemize}
\begin{itemize}
\item {Utilização:Med.}
\end{itemize}
\begin{itemize}
\item {Proveniência:(Do gr. \textunderscore tenon\textunderscore  + \textunderscore rhaphe\textunderscore )}
\end{itemize}
Sutura dos tendões.
\section{Tenosinite}
\begin{itemize}
\item {fónica:si}
\end{itemize}
\begin{itemize}
\item {Grp. gram.:f.}
\end{itemize}
\begin{itemize}
\item {Utilização:Med.}
\end{itemize}
\begin{itemize}
\item {Proveniência:(Do gr. \textunderscore tenon\textunderscore  + \textunderscore sinos\textunderscore )}
\end{itemize}
Doença, caracterizada por sensação de fadiga local, dôr e engurgitamento de um tendão.
\section{Tenossinite}
\begin{itemize}
\item {Grp. gram.:f.}
\end{itemize}
\begin{itemize}
\item {Utilização:Med.}
\end{itemize}
\begin{itemize}
\item {Proveniência:(Do gr. \textunderscore tenon\textunderscore  + \textunderscore sinos\textunderscore )}
\end{itemize}
Doença, caracterizada por sensação de fadiga local, dôr e engurgitamento de um tendão.
\section{Tenotomia}
\begin{itemize}
\item {Grp. gram.:f.}
\end{itemize}
\begin{itemize}
\item {Proveniência:(Do gr. \textunderscore tenon\textunderscore  + \textunderscore tome\textunderscore )}
\end{itemize}
Córte de tendões.
Córte de um órgão qualquer.
\section{Tenrada}
\begin{itemize}
\item {Grp. gram.:f.}
\end{itemize}
\begin{itemize}
\item {Utilização:Prov.}
\end{itemize}
\begin{itemize}
\item {Utilização:trasm.}
\end{itemize}
\begin{itemize}
\item {Proveniência:(De \textunderscore tenro\textunderscore )}
\end{itemize}
Milhal tenro e basto, para os bois.
\section{Tenramente}
\begin{itemize}
\item {Grp. gram.:adv.}
\end{itemize}
De modo tenro.
Com ternura, ternamente.
\section{Tenreira}
\begin{itemize}
\item {Grp. gram.:f.}
\end{itemize}
\begin{itemize}
\item {Utilização:Ant.}
\end{itemize}
\begin{itemize}
\item {Proveniência:(De \textunderscore tenreiro\textunderscore )}
\end{itemize}
O mesmo que \textunderscore vitella\textunderscore .
\section{Tenreiro}
\begin{itemize}
\item {Grp. gram.:adj.}
\end{itemize}
\begin{itemize}
\item {Grp. gram.:M.}
\end{itemize}
\begin{itemize}
\item {Utilização:Prov.}
\end{itemize}
\begin{itemize}
\item {Utilização:trasm.}
\end{itemize}
\begin{itemize}
\item {Utilização:Ant.}
\end{itemize}
\begin{itemize}
\item {Proveniência:(Do b. lat. \textunderscore tenrarius\textunderscore )}
\end{itemize}
Tenro.
O mesmo que \textunderscore novilho\textunderscore .
\section{Tenro}
\begin{itemize}
\item {Grp. gram.:adj.}
\end{itemize}
\begin{itemize}
\item {Utilização:Ant.}
\end{itemize}
\begin{itemize}
\item {Proveniência:(Do lat. \textunderscore tener\textunderscore )}
\end{itemize}
Molle; brando.
Fresco, recente.
Mimoso, delicado.
Viçoso.
Que dura há pouco tempo; pouco crescido.
O mesmo que \textunderscore terno\textunderscore ^2.
\section{Tenrura}
\begin{itemize}
\item {Grp. gram.:f.}
\end{itemize}
Qualidade do que é tenro.
\section{Tensa}
\begin{itemize}
\item {Grp. gram.:f.}
\end{itemize}
\begin{itemize}
\item {Proveniência:(Lat. \textunderscore tensa\textunderscore )}
\end{itemize}
Carroça, em que se levavam as imagens dos deuses, nos jogos circenses.
\section{Tensamente}
\begin{itemize}
\item {Grp. gram.:adv.}
\end{itemize}
De modo tenso.
\section{Tensão}
\begin{itemize}
\item {Grp. gram.:f.}
\end{itemize}
\begin{itemize}
\item {Proveniência:(Do lat. \textunderscore tensio\textunderscore )}
\end{itemize}
Qualidade ou estado do que é tenso.
Fôrça expansiva.
Rigidez, em certas partes do organismo.
\section{Tensivo}
\begin{itemize}
\item {Grp. gram.:adj.}
\end{itemize}
\begin{itemize}
\item {Proveniência:(Lat. \textunderscore tensivus\textunderscore )}
\end{itemize}
Que produz tensão.
\section{Tenso}
\begin{itemize}
\item {Grp. gram.:adj.}
\end{itemize}
\begin{itemize}
\item {Utilização:Fig.}
\end{itemize}
\begin{itemize}
\item {Proveniência:(Lat. \textunderscore tensus\textunderscore )}
\end{itemize}
Estendido com fôrça.
Esticado; retesado: \textunderscore corda tensa\textunderscore .
Muito applicado.
\section{Tensor}
\begin{itemize}
\item {Grp. gram.:adj.}
\end{itemize}
\begin{itemize}
\item {Grp. gram.:M.}
\end{itemize}
\begin{itemize}
\item {Utilização:Anat.}
\end{itemize}
\begin{itemize}
\item {Proveniência:(De \textunderscore tenso\textunderscore )}
\end{itemize}
Que estende.
Músculo, que serve para estender qualquer membro ou órgão.
\section{Tenta}
\begin{itemize}
\item {Grp. gram.:f.}
\end{itemize}
\begin{itemize}
\item {Utilização:Cir.}
\end{itemize}
\begin{itemize}
\item {Proveniência:(De \textunderscore tentar\textunderscore )}
\end{itemize}
Instrumento, para sondar feridas.
Corrida de novilhos, logo depois da ferra e da enchocalhação, por diversão ou para lhes experimentar a disposição para as lides tauromáchicas.
\section{Tenta-cânulas}
\begin{itemize}
\item {Grp. gram.:m.}
\end{itemize}
\begin{itemize}
\item {Utilização:Bras}
\end{itemize}
Instrumento cirúrgico. Cf. \textunderscore Tarifa das Alfând.\textunderscore , do Brasil, 109.
\section{Tentação}
\begin{itemize}
\item {Grp. gram.:f.}
\end{itemize}
\begin{itemize}
\item {Proveniência:(Do lat. \textunderscore tentatio\textunderscore )}
\end{itemize}
Acto ou effeito de tentar.
Disposição de ânimo para a prática de coisas indifferentes ou censuráveis.
Desejo vehemente.
\section{Tentaculado}
\begin{itemize}
\item {Grp. gram.:adj.}
\end{itemize}
\begin{itemize}
\item {Utilização:Zool.}
\end{itemize}
Que tem tentáculos.
\section{Tentacular}
\begin{itemize}
\item {Grp. gram.:adj.}
\end{itemize}
Relativo a tentáculo.
\section{Tentaculífero}
\begin{itemize}
\item {Grp. gram.:adj.}
\end{itemize}
\begin{itemize}
\item {Grp. gram.:M. pl.}
\end{itemize}
\begin{itemize}
\item {Proveniência:(Do lat. \textunderscore tentaculum\textunderscore  + \textunderscore ferre\textunderscore )}
\end{itemize}
Que tem tentáculos; tentaculado.
Ordem de molluscos cephalópodes.
\section{Tentaculiforme}
\begin{itemize}
\item {Grp. gram.:adj.}
\end{itemize}
\begin{itemize}
\item {Proveniência:(Do lat. \textunderscore tentaculum\textunderscore  + \textunderscore forma\textunderscore )}
\end{itemize}
Que tem fórma de tentáculo.
\section{Tentáculo}
\begin{itemize}
\item {Grp. gram.:m.}
\end{itemize}
\begin{itemize}
\item {Utilização:Zool.}
\end{itemize}
\begin{itemize}
\item {Utilização:Fig.}
\end{itemize}
\begin{itemize}
\item {Proveniência:(Lat. \textunderscore tentaculum\textunderscore )}
\end{itemize}
Appêndice móvel, não articulado, que, na cabeça ou na parte anterior dos animaes, serve geralmente de órgão tactil.
Cada um dos meios, de que se serve a ambição ou a astúcia, para attingir e apprehender aquillo que a tenta.
\section{Tentadeiro}
\begin{itemize}
\item {Grp. gram.:m.}
\end{itemize}
\begin{itemize}
\item {Utilização:Neol.}
\end{itemize}
\begin{itemize}
\item {Proveniência:(De \textunderscore tentar\textunderscore )}
\end{itemize}
Lugar cercado, onde se tentam novilhos e se ferra o gado.
\section{Tentadiço}
\begin{itemize}
\item {Grp. gram.:adj.}
\end{itemize}
Que se deixa tentar facilmente. Cf. Castilho, \textunderscore Tartufo\textunderscore , 84.
\section{Tentador}
\begin{itemize}
\item {Grp. gram.:m.  e  adj.}
\end{itemize}
\begin{itemize}
\item {Grp. gram.:M.}
\end{itemize}
\begin{itemize}
\item {Utilização:Pop.}
\end{itemize}
\begin{itemize}
\item {Proveniência:(Do lat. \textunderscore tentator\textunderscore )}
\end{itemize}
O que tenta; o que seduz.
O diabo.
\section{Tentame}
\begin{itemize}
\item {Grp. gram.:m.}
\end{itemize}
\begin{itemize}
\item {Proveniência:(Lat. \textunderscore tentamen\textunderscore )}
\end{itemize}
Tentativa; ensaio.
\section{Tentamento}
\begin{itemize}
\item {Grp. gram.:m.}
\end{itemize}
\begin{itemize}
\item {Proveniência:(Lat. \textunderscore tentamentum\textunderscore )}
\end{itemize}
Tentação; tentame.
\section{Tentante}
\begin{itemize}
\item {Grp. gram.:adj.}
\end{itemize}
\begin{itemize}
\item {Proveniência:(Lat. \textunderscore tentans\textunderscore )}
\end{itemize}
O mesmo que \textunderscore tentativo\textunderscore .
\section{Tentar}
\begin{itemize}
\item {Grp. gram.:v. t.}
\end{itemize}
\begin{itemize}
\item {Proveniência:(Lat. \textunderscore tentare\textunderscore )}
\end{itemize}
Empregar meios para obter (o que se deseja ou se emprehende).
Procurar conseguir.
Emprehender.
Aventurar, arriscar: \textunderscore tentar experiências\textunderscore .
Instigar para o mal ou para o peccado: \textunderscore tentou-me o demónio\textunderscore .
Provocar.
Procurar seduzir.
Experimentar a fé ou a paciência de.
Causar desejo a: \textunderscore aquella fruta tenta a gente\textunderscore .
Exercitar.
Instaurar ou apresentar em juízo (acção ou demanda).
Proceder á tenta ou corrida de (novilhos).
\section{Tefrina}
\begin{itemize}
\item {Grp. gram.:f.}
\end{itemize}
\begin{itemize}
\item {Utilização:Geol.}
\end{itemize}
Rocha pardacenta, talvez o mesmo que \textunderscore tefrito\textunderscore .
\section{Tefrite}
\begin{itemize}
\item {Grp. gram.:f.}
\end{itemize}
\begin{itemize}
\item {Utilização:Geol.}
\end{itemize}
\begin{itemize}
\item {Proveniência:(Lat. \textunderscore tephritis\textunderscore )}
\end{itemize}
Typo de rochas, em que a nefelite ou a leucite se associa á plagioclasse.
\section{Tefroíte}
\begin{itemize}
\item {Grp. gram.:m.}
\end{itemize}
\begin{itemize}
\item {Utilização:Miner.}
\end{itemize}
\begin{itemize}
\item {Proveniência:(Do gr. \textunderscore tephros\textunderscore )}
\end{itemize}
Substância compacta, de côr cinzenta e brilho diamantino.
\section{Tefromancia}
\begin{itemize}
\item {Grp. gram.:f.}
\end{itemize}
\begin{itemize}
\item {Proveniência:(Do gr. \textunderscore tephra\textunderscore  + \textunderscore manteia\textunderscore )}
\end{itemize}
Espécie de adivinhação, em que se empregava a cinza dos sacrifícios.
\section{Tefromântico}
\begin{itemize}
\item {Grp. gram.:adj.}
\end{itemize}
Relativo á tefromancia.
\section{Tefrósia}
\begin{itemize}
\item {Grp. gram.:f.}
\end{itemize}
\begin{itemize}
\item {Proveniência:(Do gr. \textunderscore tephros\textunderscore )}
\end{itemize}
Planta leguminosa, espécie de lódão.
\section{Tentativa}
\begin{itemize}
\item {Grp. gram.:f.}
\end{itemize}
\begin{itemize}
\item {Proveniência:(De \textunderscore tentativo\textunderscore )}
\end{itemize}
Tentação; ensaio, experiência.
\section{Tentativo}
\begin{itemize}
\item {Grp. gram.:adj.}
\end{itemize}
\begin{itemize}
\item {Proveniência:(Lat. \textunderscore tentativus\textunderscore )}
\end{itemize}
Que tenta.
Que se póde tentar ou experimentar.
\section{Tente}
\begin{itemize}
\item {Grp. gram.:adj.}
\end{itemize}
\begin{itemize}
\item {Utilização:Des.}
\end{itemize}
(Contr. de \textunderscore tenente\textunderscore )
\section{Tenteador}
\begin{itemize}
\item {Grp. gram.:m.  e  adj.}
\end{itemize}
O que tenteia.
\section{Tentear}
\begin{itemize}
\item {Grp. gram.:v. t.}
\end{itemize}
\begin{itemize}
\item {Utilização:Ext.}
\end{itemize}
\begin{itemize}
\item {Proveniência:(De \textunderscore tenta\textunderscore )}
\end{itemize}
Investigar ou sondar com tenta.
Examinar.
Ensaiar.
Tactear.
\section{Tentear}
\begin{itemize}
\item {Grp. gram.:v. t.}
\end{itemize}
\begin{itemize}
\item {Proveniência:(De \textunderscore tento\textunderscore )}
\end{itemize}
Dirigir com tento.
Calcular com attenção.
Dar attenção a.
Entreter.
Marcar com tentos.
Distribuír ou empregar com tento ou parcimónia: \textunderscore tentear uma despesa\textunderscore .
\section{Tenteio}
\begin{itemize}
\item {Grp. gram.:m.}
\end{itemize}
Acto de tentear^1. Cf. Arn. Gama \textunderscore Motim\textunderscore , 48.
\section{Tenthredém}
\begin{itemize}
\item {Grp. gram.:f.}
\end{itemize}
\begin{itemize}
\item {Proveniência:(Do gr. \textunderscore tenthredon\textunderscore )}
\end{itemize}
Gênero de insectos hymenópteros.
\section{Tenthredo}
\begin{itemize}
\item {Grp. gram.:m.}
\end{itemize}
O mesmo ou melhór que \textunderscore tenthredém\textunderscore .
\section{Tentredém}
\begin{itemize}
\item {Grp. gram.:f.}
\end{itemize}
\begin{itemize}
\item {Proveniência:(Do gr. \textunderscore tenthredon\textunderscore )}
\end{itemize}
Gênero de insectos himenópteros.
\section{Tentredo}
\begin{itemize}
\item {fónica:trê}
\end{itemize}
\begin{itemize}
\item {Grp. gram.:m.}
\end{itemize}
O mesmo ou melhór que \textunderscore tentredém\textunderscore .
\section{Tentilha}
\begin{itemize}
\item {Grp. gram.:f.}
\end{itemize}
\begin{itemize}
\item {Utilização:Prov.}
\end{itemize}
O mesmo que \textunderscore tentilhão\textunderscore .
\section{Tentilhão}
\begin{itemize}
\item {Grp. gram.:m.}
\end{itemize}
Pássaro conirostro.
Peixe labroide, também chamado \textunderscore bodião\textunderscore .
\section{Tentilhão-da-índia}
\begin{itemize}
\item {Grp. gram.:m.}
\end{itemize}
Nome que, em Melres, se dá ao \textunderscore dom-fafe\textunderscore .
\section{Tentilheira}
\begin{itemize}
\item {Grp. gram.:f.}
\end{itemize}
Variedade de azeitona.
\section{Tentim}
\begin{itemize}
\item {Grp. gram.:m.}
\end{itemize}
\begin{itemize}
\item {Utilização:Des.}
\end{itemize}
Pequeno tento.
\section{Tentim-por-tentim}
\begin{itemize}
\item {Grp. gram.:loc. adv.}
\end{itemize}
\begin{itemize}
\item {Proveniência:(De \textunderscore tentim\textunderscore )}
\end{itemize}
Minuciosamente, por miúdo; ponto por ponto; com todos os pormenores.
\section{Tento}
\begin{itemize}
\item {Grp. gram.:m.}
\end{itemize}
\begin{itemize}
\item {Utilização:Fig.}
\end{itemize}
\begin{itemize}
\item {Utilização:Gír.}
\end{itemize}
\begin{itemize}
\item {Utilização:T. de Serpa}
\end{itemize}
\begin{itemize}
\item {Utilização:Ant.}
\end{itemize}
\begin{itemize}
\item {Proveniência:(Lat. \textunderscore tentus\textunderscore )}
\end{itemize}
Attenção.
Precaução, cuidado.
Pauzinho, em que se apoia a mão, para pintar com firmeza.
Cálculo.
Bofetada.
Utensílio de barro, com que se amparam as panelas no fogareiro. Cf. Rev. \textunderscore Tradição\textunderscore , II, 11.
Designação genérica de qualquer peça de música.
\textunderscore Dar tento\textunderscore , dar fé, dar attenção. Cf. Camillo, \textunderscore Enjeitada\textunderscore , 29.
\section{Tento}
\begin{itemize}
\item {Grp. gram.:m.}
\end{itemize}
\begin{itemize}
\item {Proveniência:(Do lat. \textunderscore talentum\textunderscore )}
\end{itemize}
Peça de metal, com que se marcam pontos no jôgo.
\section{Tentória}
\begin{itemize}
\item {Grp. gram.:f.}
\end{itemize}
\begin{itemize}
\item {Proveniência:(Lat. \textunderscore tentoria\textunderscore , pl. de \textunderscore tentorium\textunderscore )}
\end{itemize}
O mesmo que \textunderscore tenta\textunderscore . Cf. Castilho, \textunderscore Fastos\textunderscore , I, 585.
\section{Tentório}
\begin{itemize}
\item {Grp. gram.:m.}
\end{itemize}
\begin{itemize}
\item {Proveniência:(Lat. \textunderscore tentorium\textunderscore )}
\end{itemize}
Barraca de campanha.
\section{Tentos}
\begin{itemize}
\item {Grp. gram.:m. Pl.}
\end{itemize}
\begin{itemize}
\item {Utilização:Bras. do S}
\end{itemize}
Pequenas tiras de coiro, na parte posterior do lombilho, ás quaes se prende qualquer coisa que se queira trazer á garupa.
\section{Tênue}
\begin{itemize}
\item {Grp. gram.:adj.}
\end{itemize}
\begin{itemize}
\item {Proveniência:(Lat. \textunderscore tenuis\textunderscore )}
\end{itemize}
Delgado; subtil.
Grácil; frágil.
Pequeníssimo; débil.
Pouco importante.
\section{Tenuemente}
\begin{itemize}
\item {Grp. gram.:adv.}
\end{itemize}
De modo tênue.
Escassamente; em pequena quantidade.
\section{Tenuicorne}
\begin{itemize}
\item {fónica:nu-i}
\end{itemize}
\begin{itemize}
\item {Grp. gram.:adj.}
\end{itemize}
\begin{itemize}
\item {Proveniência:(Do lat. \textunderscore tenuis\textunderscore  + \textunderscore cornu\textunderscore )}
\end{itemize}
Que tem antenas ou cornos delgados.
\section{Tenuicórneo}
\begin{itemize}
\item {fónica:nu-i}
\end{itemize}
\begin{itemize}
\item {Grp. gram.:adj.}
\end{itemize}
\begin{itemize}
\item {Utilização:Zool.}
\end{itemize}
\begin{itemize}
\item {Proveniência:(Do lat. \textunderscore tenuis\textunderscore  + \textunderscore cornu\textunderscore )}
\end{itemize}
Que tem antenas ou cornos delgados.
\section{Tenuidade}
\begin{itemize}
\item {fónica:nu-i}
\end{itemize}
\begin{itemize}
\item {Grp. gram.:f.}
\end{itemize}
\begin{itemize}
\item {Proveniência:(Do lat. \textunderscore tenuitas\textunderscore )}
\end{itemize}
Qualidade do que é tênue.
\section{Tenuifloro}
\begin{itemize}
\item {fónica:nu-i}
\end{itemize}
\begin{itemize}
\item {Grp. gram.:adj.}
\end{itemize}
\begin{itemize}
\item {Utilização:Bot.}
\end{itemize}
\begin{itemize}
\item {Proveniência:(Do lat. \textunderscore tenuis\textunderscore  + \textunderscore flos\textunderscore , \textunderscore floris\textunderscore )}
\end{itemize}
Que tem flôres pequenas.
\section{Tenuifoliado}
\begin{itemize}
\item {fónica:nu-i}
\end{itemize}
\begin{itemize}
\item {Grp. gram.:adj.}
\end{itemize}
\begin{itemize}
\item {Utilização:Bot.}
\end{itemize}
\begin{itemize}
\item {Proveniência:(De \textunderscore tenue\textunderscore  + \textunderscore foliado\textunderscore )}
\end{itemize}
Que tem fôlhas pequenas.
\section{Tenuípede}
\begin{itemize}
\item {Grp. gram.:adj.}
\end{itemize}
\begin{itemize}
\item {Utilização:Zool.}
\end{itemize}
\begin{itemize}
\item {Proveniência:(Do lat. \textunderscore tenuis\textunderscore  + \textunderscore pes\textunderscore , \textunderscore pedis\textunderscore )}
\end{itemize}
Que tem pequenos pés.
\section{Tenuipene}
\begin{itemize}
\item {fónica:nu-i}
\end{itemize}
\begin{itemize}
\item {Grp. gram.:adj.}
\end{itemize}
\begin{itemize}
\item {Utilização:Zool.}
\end{itemize}
\begin{itemize}
\item {Proveniência:(Do lat. \textunderscore tenuis\textunderscore  + \textunderscore penna\textunderscore )}
\end{itemize}
Que tem pennas pequenas.
\section{Tenuipenne}
\begin{itemize}
\item {Grp. gram.:adj.}
\end{itemize}
\begin{itemize}
\item {Utilização:Zool.}
\end{itemize}
\begin{itemize}
\item {Proveniência:(Do lat. \textunderscore tenuis\textunderscore  + \textunderscore penna\textunderscore )}
\end{itemize}
Que tem pennas pequenas.
\section{Tenuirostro}
\begin{itemize}
\item {fónica:rós}
\end{itemize}
\begin{itemize}
\item {Grp. gram.:adj.}
\end{itemize}
\begin{itemize}
\item {Utilização:Zool.}
\end{itemize}
\begin{itemize}
\item {Grp. gram.:M. Pl.}
\end{itemize}
\begin{itemize}
\item {Proveniência:(Do lat. \textunderscore tenuis\textunderscore  + \textunderscore rostrum\textunderscore )}
\end{itemize}
Que tem bico delgado e longo.
Família de pássaros tenuirostros.
\section{Tenuirrostro}
\begin{itemize}
\item {fónica:nu-i}
\end{itemize}
\begin{itemize}
\item {Grp. gram.:adj.}
\end{itemize}
\begin{itemize}
\item {Utilização:Zool.}
\end{itemize}
\begin{itemize}
\item {Grp. gram.:M. Pl.}
\end{itemize}
\begin{itemize}
\item {Proveniência:(Do lat. \textunderscore tenuis\textunderscore  + \textunderscore rostrum\textunderscore )}
\end{itemize}
Que tem bico delgado e longo.
Família de pássaros tenuirrostros.
\section{Teor}
\begin{itemize}
\item {Grp. gram.:m.}
\end{itemize}
\begin{itemize}
\item {Utilização:Fig.}
\end{itemize}
\begin{itemize}
\item {Utilização:Chím.}
\end{itemize}
\begin{itemize}
\item {Proveniência:(Do lat. \textunderscore tenor\textunderscore )}
\end{itemize}
Conteúdo de uma escrita.
Maneira, modo.
Systema, norma; qualidade.
Proporção, em que está cada um dos elementos de um corpo composto.
\section{Teosinto}
\begin{itemize}
\item {Grp. gram.:m.}
\end{itemize}
\begin{itemize}
\item {Utilização:Bras}
\end{itemize}
Espécie de forragem, o mesmo que \textunderscore riana\textunderscore .
\section{Tepe}
\begin{itemize}
\item {Grp. gram.:m.}
\end{itemize}
Torrão cuneiforme, que se emprega na construcção do muralhas.
(Cast. \textunderscore tepe\textunderscore )
\section{Tepente}
\begin{itemize}
\item {Grp. gram.:adj.}
\end{itemize}
\begin{itemize}
\item {Proveniência:(Lat. \textunderscore tepens\textunderscore )}
\end{itemize}
O mesmo que \textunderscore tépido\textunderscore . Cf. Th. de Carvalho, \textunderscore B. da Seda\textunderscore .
\section{Tepez}
\begin{itemize}
\item {Grp. gram.:adj.}
\end{itemize}
\begin{itemize}
\item {Utilização:Pop.}
\end{itemize}
Teimoso; pertinaz.--O \textunderscore Elucidário\textunderscore  de S. R. Viterbo diz \textunderscore tepés\textunderscore .
\section{Tephrina}
\begin{itemize}
\item {Grp. gram.:f.}
\end{itemize}
\begin{itemize}
\item {Utilização:Geol.}
\end{itemize}
Rocha pardacenta, talvez o mesmo que \textunderscore tephrito\textunderscore .
\section{Tephrite}
\begin{itemize}
\item {Grp. gram.:f.}
\end{itemize}
\begin{itemize}
\item {Utilização:Geol.}
\end{itemize}
\begin{itemize}
\item {Proveniência:(Lat. \textunderscore tephritis\textunderscore )}
\end{itemize}
Typo de rochas, em que a nephelite ou a leucite se associa á plagioclasse.
\section{Tephroíte}
\begin{itemize}
\item {Grp. gram.:m.}
\end{itemize}
\begin{itemize}
\item {Utilização:Miner.}
\end{itemize}
\begin{itemize}
\item {Proveniência:(Do gr. \textunderscore tephros\textunderscore )}
\end{itemize}
Substância compacta, de côr cinzenta e brilho diamantino.
\section{Tephromancia}
\begin{itemize}
\item {Grp. gram.:f.}
\end{itemize}
\begin{itemize}
\item {Proveniência:(Do gr. \textunderscore tephra\textunderscore  + \textunderscore manteia\textunderscore )}
\end{itemize}
Espécie de adivinhação, em que se empregava a cinza dos sacrifícios.
\section{Tephromântico}
\begin{itemize}
\item {Grp. gram.:adj.}
\end{itemize}
Relativo á tephromancia.
\section{Tephrósia}
\begin{itemize}
\item {Grp. gram.:f.}
\end{itemize}
\begin{itemize}
\item {Proveniência:(Do gr. \textunderscore tephros\textunderscore )}
\end{itemize}
Planta leguminosa, espécie de lódão.
\section{Tepidamente}
\begin{itemize}
\item {Grp. gram.:adv.}
\end{itemize}
\begin{itemize}
\item {Utilização:Fig.}
\end{itemize}
Com pouco calor.
De modo tépido.
Tibiamente, com tibieza, com froixidão.
\section{Tepidário}
\begin{itemize}
\item {Grp. gram.:m.}
\end{itemize}
\begin{itemize}
\item {Proveniência:(Lat. \textunderscore tepidarium\textunderscore )}
\end{itemize}
Casa de banhos mornos, entre os Romanos.
Depósito de água morna.
\section{Tepidez}
\begin{itemize}
\item {Grp. gram.:f.}
\end{itemize}
\begin{itemize}
\item {Utilização:Fig.}
\end{itemize}
Estado daquillo que é tépido.
Tibieza; froixidão.
\section{Tépido}
\begin{itemize}
\item {Grp. gram.:adj.}
\end{itemize}
\begin{itemize}
\item {Utilização:Fig.}
\end{itemize}
\begin{itemize}
\item {Proveniência:(Lat. \textunderscore tepidus\textunderscore )}
\end{itemize}
Que tem pouco calor; morno.
Froixo, tibio.
\section{Tepor}
\begin{itemize}
\item {Grp. gram.:m.}
\end{itemize}
\begin{itemize}
\item {Proveniência:(Lat. \textunderscore tepor\textunderscore )}
\end{itemize}
(V.tepidez)
\section{Teque}
\begin{itemize}
\item {Grp. gram.:m.}
\end{itemize}
\begin{itemize}
\item {Utilização:Náut.}
\end{itemize}
Peça de poleame, formada de dois moitões, um superior ao outro.
\section{Teque-teque}
\begin{itemize}
\item {Grp. gram.:m.}
\end{itemize}
\begin{itemize}
\item {Utilização:Bras. do N}
\end{itemize}
Vendedor ambulante de fazendas e objectos de armarinho.
\section{Tèqui}
\begin{itemize}
\item {Grp. gram.:adv.}
\end{itemize}
Fórma pop. da loc. \textunderscore até aqui\textunderscore .«\textunderscore Tèqui o conteúdo da escritura\textunderscore ». Filinto, \textunderscore D. Man.\textunderscore , I, 327.
\section{Têr}
\begin{itemize}
\item {Grp. gram.:v. t.}
\end{itemize}
\begin{itemize}
\item {Grp. gram.:V. i.}
\end{itemize}
\begin{itemize}
\item {Utilização:Des.}
\end{itemize}
\begin{itemize}
\item {Grp. gram.:V. p.}
\end{itemize}
\begin{itemize}
\item {Proveniência:(Do lat. \textunderscore tenere\textunderscore )}
\end{itemize}
Segurar nas mãos ou entre as mãos.
Suster; agarrar.
Possuir; haver: \textunderscore têr propriedades\textunderscore .
Adquirir.
Ocupar.
Gozar, usufruír: \textunderscore têr horas de ócio\textunderscore .
Conter, encerrar: \textunderscore bolsa, que tem dinheiro\textunderscore .
Reputar, julgar: \textunderscore tenho para mim que és tolo\textunderscore .
Alcançar.
Poder dispor.
Deter.
Fazer parar.
Sêr composto de.
Sentir, experimentar: \textunderscore têr dores\textunderscore .
Trazer vestido, trajar.
Dar á luz: \textunderscore minha avó teve seis filhos\textunderscore .
Sêr dotado de: \textunderscore esta menina tem pudor\textunderscore .
Sêr obrigado a: \textunderscore tenho de educar os filhos\textunderscore .
Produzir.
\textunderscore Têr a barba têsa\textunderscore , não se curvar, não se deixar vencer, sêr enérgico, recalcitrar.
\textunderscore Têr de\textunderscore , haver de.
\textunderscore Têr que\textunderscore , têr necessidade ou obrigação de.
\textunderscore Têr fé\textunderscore , sêr digno de crédito.
\textunderscore Têr mão\textunderscore , deter-se, fazer parar.
\textunderscore Têr parte\textunderscore , participar, sêr partícipe.
\textunderscore Têr dedo\textunderscore , têr geito ou habilidade.
\textunderscore Têr pé\textunderscore , andar ligeiro.
\textunderscore Têr a palavra\textunderscore , sêr autorizado para falar numa assembleia.
\textunderscore Têr a palavra de\textunderscore , receber ou guardar a promessa de.
\textunderscore Têr em vista\textunderscore , planear, attender a.
Sêr equivalente.
Manter-se firme, segurar-se para não caír.
Sustar o passo, deter-se.
Ater-se, confiar.
Considerar-se, julgar-se.
Outras e variadas accepções tem êste v., as quaes melhór se conhecerão pelo sentido da phrase, como em \textunderscore ir têr com alguém\textunderscore , \textunderscore vir têr com alguém\textunderscore , ir ou vir encontrar-se com alguém, ir ou vir juntar com alguém, apparecer a alguém.
\textunderscore Têr para si\textunderscore , estar convencido de, opinar: \textunderscore tenho para mim que o Mario Cândido esconde algum desgôsto\textunderscore .
\section{Terarca}
\begin{itemize}
\item {Grp. gram.:m.}
\end{itemize}
\begin{itemize}
\item {Proveniência:(Do gr. \textunderscore teras\textunderscore  + \textunderscore arkhein\textunderscore )}
\end{itemize}
Aquele que, entre os antigos, commandava soldados montados em elefantes.
\section{Terarcha}
\begin{itemize}
\item {fónica:ca}
\end{itemize}
\begin{itemize}
\item {Proveniência:(Do gr. \textunderscore teras\textunderscore  + \textunderscore arkhein\textunderscore )}
\end{itemize}
Aquelle que, entre os antigos, commandava soldados montados em elephantes.
\section{Teratogenesia}
\begin{itemize}
\item {Grp. gram.:f.}
\end{itemize}
O mesmo que \textunderscore teratogenia\textunderscore .
\section{Teratogenia}
\begin{itemize}
\item {Grp. gram.:f.}
\end{itemize}
\begin{itemize}
\item {Proveniência:(Do gr. \textunderscore teras\textunderscore , \textunderscore teratos\textunderscore  + \textunderscore genos\textunderscore )}
\end{itemize}
Producção de monstruosidades.
\section{Teratogênico}
\begin{itemize}
\item {Grp. gram.:adj.}
\end{itemize}
Relativo á teratogenia.
\section{Teratologia}
\begin{itemize}
\item {Grp. gram.:f.}
\end{itemize}
\begin{itemize}
\item {Proveniência:(Do gr. \textunderscore teras\textunderscore , \textunderscore teratos\textunderscore  + \textunderscore logos\textunderscore )}
\end{itemize}
Descripção e classificação das monstruosidades, em Pathologia e em Botânica.
\section{Teratológico}
\begin{itemize}
\item {Grp. gram.:adj.}
\end{itemize}
Relativo á teratologia.
\section{Teratologista}
\begin{itemize}
\item {Grp. gram.:m.}
\end{itemize}
Aquelle que trata de teratologia.
\section{Teratólogo}
\begin{itemize}
\item {Grp. gram.:m.}
\end{itemize}
Aquelle que é perito em teratologia.
(Cp. \textunderscore teratologia\textunderscore )
\section{Teratopagia}
\begin{itemize}
\item {Grp. gram.:f.}
\end{itemize}
Qualidade ou estado de teratópago.
\section{Teratópago}
\begin{itemize}
\item {Grp. gram.:m.}
\end{itemize}
\begin{itemize}
\item {Proveniência:(Do gr. \textunderscore teras\textunderscore , \textunderscore teratos\textunderscore  + \textunderscore pagein\textunderscore )}
\end{itemize}
O mesmo que \textunderscore xiphtópago\textunderscore .
\section{Teratoscopia}
\begin{itemize}
\item {Proveniência:(Do gr. \textunderscore teras\textunderscore , \textunderscore teratos\textunderscore  + \textunderscore skopein\textunderscore )}
\end{itemize}
\textunderscore f.\textunderscore 
Supposta adivinhação, baseada na observação dos phenómenos que se julgavam milagrosos.
\section{Térbio}
\begin{itemize}
\item {Grp. gram.:m.}
\end{itemize}
Metal, descoberto recentemente no ýtrio.
\section{Têrça}
\begin{itemize}
\item {Grp. gram.:adj. f.}
\end{itemize}
\begin{itemize}
\item {Grp. gram.:F.}
\end{itemize}
\begin{itemize}
\item {Utilização:T. de Vouzella}
\end{itemize}
\begin{itemize}
\item {Proveniência:(Do lat. \textunderscore tertia\textunderscore )}
\end{itemize}
O mesmo que \textunderscore terceira\textunderscore , flexão de \textunderscore terceiro\textunderscore .
A têrça parte de um todo.
Uma das horas canónicas.
Têrça parte de uma herança: \textunderscore deixar a têrça a alguém\textunderscore .
Intervallo musical entre duas notas, separadas por outra.
Peça de madeira, sotoposta aos caibros, para não vergarem.
O mesmo que \textunderscore senábria\textunderscore .
\section{Terçado}
\begin{itemize}
\item {Grp. gram.:m.}
\end{itemize}
\begin{itemize}
\item {Grp. gram.:Adj.}
\end{itemize}
\begin{itemize}
\item {Utilização:Prov.}
\end{itemize}
\begin{itemize}
\item {Proveniência:(De \textunderscore terçar\textunderscore )}
\end{itemize}
Espada, de folha larga e curta.
Diz-se do pão, feito de três qualidades de farinha, (milhão, centeio e milho branco) em partes iguaes.
\section{Terçador}
\begin{itemize}
\item {Grp. gram.:m.  e  adj.}
\end{itemize}
O que térça; o que pugna.
O que intercede.
\section{Têrça-feira}
\begin{itemize}
\item {Grp. gram.:f.}
\end{itemize}
O terceiro dia da semana.
\section{Terçan}
\begin{itemize}
\item {Grp. gram.:f.  e  adj.}
\end{itemize}
\begin{itemize}
\item {Proveniência:(Do lat. \textunderscore tertiana\textunderscore )}
\end{itemize}
Diz-se da febre, em que os accessos se repetem, de três em três dias.
\section{Terção}
\begin{itemize}
\item {Grp. gram.:m.}
\end{itemize}
\begin{itemize}
\item {Proveniência:(Do lat. \textunderscore tertianus\textunderscore )}
\end{itemize}
Rebento da cepa, que se não cortou por occasião da poda.
\section{Terçar}
\begin{itemize}
\item {Grp. gram.:v. t.}
\end{itemize}
\begin{itemize}
\item {Grp. gram.:V. i.}
\end{itemize}
\begin{itemize}
\item {Proveniência:(Do lat. \textunderscore tertiare\textunderscore )}
\end{itemize}
Misturar (três coisas).
Dividir em três partes.
Atravessar, cruzar.
Pôr em diagonal.
Intervir.
Combater a favor, pugnar.
Brigar.
\section{Terçaria}
\begin{itemize}
\item {Grp. gram.:f.}
\end{itemize}
\begin{itemize}
\item {Utilização:Des.}
\end{itemize}
\begin{itemize}
\item {Proveniência:(De \textunderscore terçar\textunderscore )}
\end{itemize}
Intervenção.
Caução em poder de terceiro.
\section{Tercedia}
\begin{itemize}
\item {Grp. gram.:f.}
\end{itemize}
\begin{itemize}
\item {Utilização:Des.}
\end{itemize}
Prazo de três annos, três meses, três semanas e três dias, que se concedia a um devedor, para solver a dívida.
(Contr. de \textunderscore terceiro\textunderscore  + \textunderscore dia\textunderscore )
\section{Terceira}
\begin{itemize}
\item {Grp. gram.:f.}
\end{itemize}
\begin{itemize}
\item {Utilização:Mús.}
\end{itemize}
\begin{itemize}
\item {Proveniência:(De \textunderscore terceiro\textunderscore )}
\end{itemize}
Mulhér, que intercede; alcoviteira.
Têrça.
Intervallo de três graus, que tambem se designa por \textunderscore têrça\textunderscore .
\section{Terceiramente}
\begin{itemize}
\item {Grp. gram.:adv.}
\end{itemize}
\begin{itemize}
\item {Proveniência:(De \textunderscore terceiro\textunderscore )}
\end{itemize}
Em terceiro lugar.
\section{Terceiranista}
\begin{itemize}
\item {Grp. gram.:m.}
\end{itemize}
Estudante, que frequenta o terceiro ano de qualquer faculdade universitária ou de outra escola superior.
\section{Terceirannista}
\begin{itemize}
\item {Grp. gram.:m.}
\end{itemize}
Estudante, que frequenta o terceiro anno de qualquer faculdade universitária ou de outra escola superior.
\section{Terceiro}
\begin{itemize}
\item {Grp. gram.:adj.}
\end{itemize}
\begin{itemize}
\item {Grp. gram.:M.}
\end{itemize}
\begin{itemize}
\item {Proveniência:(De \textunderscore terço\textunderscore )}
\end{itemize}
Que numa série de três occupa o último lugar.
Intercessor; alcoviteiro.
\section{Terceiros}
\begin{itemize}
\item {Grp. gram.:m. pl.}
\end{itemize}
\begin{itemize}
\item {Utilização:Ant.}
\end{itemize}
Membros da confraria, instituida sob a invocação de San-Francisco de Assis.
Agua-furtada.
(Pl. de \textunderscore terceiro\textunderscore )
\section{Tercena}
\begin{itemize}
\item {Grp. gram.:f.}
\end{itemize}
\begin{itemize}
\item {Utilização:Des.}
\end{itemize}
\begin{itemize}
\item {Proveniência:(Do ár. \textunderscore dâr-cinâ'a\textunderscore )}
\end{itemize}
Tulha, ou celleiro, á beira do rio ou perto de um caes.
\section{Tercenaria}
\begin{itemize}
\item {Grp. gram.:f.}
\end{itemize}
\begin{itemize}
\item {Utilização:Ant.}
\end{itemize}
Qualidade do beneficiado ecclesiástico, que era tercenário.
\section{Tercenário}
\begin{itemize}
\item {Grp. gram.:m.}
\end{itemize}
Aquelle que recebe a têrça de uma herança.
Beneficiado ecclesiástico, que tinha a têrça parte da prebenda de um cónego.
(Cp. lat. \textunderscore tertianus\textunderscore )
\section{Terceneiro}
\begin{itemize}
\item {Grp. gram.:m.}
\end{itemize}
Aquelle que trabalha ou é empregado em tercenas.
\section{Tercentésimo}
\begin{itemize}
\item {Grp. gram.:adj.}
\end{itemize}
(V.trecentésimo)
\section{Tercetagem}
\begin{itemize}
\item {Grp. gram.:f.}
\end{itemize}
\begin{itemize}
\item {Utilização:Des.}
\end{itemize}
Composição poética em tercetos.
\section{Tercetar}
\begin{itemize}
\item {Grp. gram.:v. i.}
\end{itemize}
\begin{itemize}
\item {Utilização:Des.}
\end{itemize}
Fazer tercetos.
\section{Tercêto}
\begin{itemize}
\item {Grp. gram.:m.}
\end{itemize}
\begin{itemize}
\item {Proveniência:(It. \textunderscore terzetto\textunderscore )}
\end{itemize}
Estrophe de três versos.
Concêrto musical de três vozes ou três instrumentos.
\section{Tércia}
\begin{itemize}
\item {Grp. gram.:f.}
\end{itemize}
\begin{itemize}
\item {Proveniência:(Lat. \textunderscore tertia\textunderscore )}
\end{itemize}
Uma das horas canónicas, têrça.
\section{Terciado}
\begin{itemize}
\item {Grp. gram.:adj.}
\end{itemize}
\begin{itemize}
\item {Utilização:Heráld.}
\end{itemize}
\begin{itemize}
\item {Proveniência:(Do lat. \textunderscore tertia\textunderscore )}
\end{itemize}
Diz-se do escudo, dividido em três secções. Cf. L. Ribeiro, \textunderscore Trat. de Armaria\textunderscore .
\section{Terciarão}
\begin{itemize}
\item {Grp. gram.:m.}
\end{itemize}
\begin{itemize}
\item {Proveniência:(Fr. \textunderscore tierceron\textunderscore )}
\end{itemize}
Arco, cujas extremidades partem dos ângulos de uma abóbada ogival.
\section{Terciário}
\begin{itemize}
\item {Grp. gram.:adj.}
\end{itemize}
\begin{itemize}
\item {Proveniência:(Lat. \textunderscore tertiarius\textunderscore )}
\end{itemize}
Que está ou vem em terceiro lugar ou ordem.
Diz-se dos effeitos posteriores aos que seguem immediatamente certas affecções orgânicas.
Diz-se do terceiro período geológico e dos terrenos immediatamente anteriores ás antigas alluviões.
\section{Tèrcifalange}
\begin{itemize}
\item {Grp. gram.:f.}
\end{itemize}
\begin{itemize}
\item {Utilização:Anat.}
\end{itemize}
Terceira falange do pé.
\section{Tèrcifalangeta}
\begin{itemize}
\item {Grp. gram.:f.}
\end{itemize}
\begin{itemize}
\item {Utilização:Anat.}
\end{itemize}
Terceira falangeta do pé.
\section{Tèrcifalanginha}
\begin{itemize}
\item {Grp. gram.:f.}
\end{itemize}
\begin{itemize}
\item {Utilização:Anat.}
\end{itemize}
Terceira falanginha do pé.
\section{Tercilho}
\begin{itemize}
\item {Grp. gram.:m.}
\end{itemize}
\begin{itemize}
\item {Utilização:Mús.}
\end{itemize}
\begin{itemize}
\item {Proveniência:(Do cast. \textunderscore tercillo\textunderscore )}
\end{itemize}
Medida antiga, talvez a têrça parte do quartilho.
Grupo de três notas, com o valor de duas.
\section{Tercimetatársico}
\begin{itemize}
\item {Grp. gram.:adj.}
\end{itemize}
\begin{itemize}
\item {Utilização:Anat.}
\end{itemize}
Diz-se do terceiro osso metatársico.
\section{Tercina}
\begin{itemize}
\item {Grp. gram.:f.}
\end{itemize}
\begin{itemize}
\item {Utilização:Bot.}
\end{itemize}
\begin{itemize}
\item {Utilização:Mús.}
\end{itemize}
\begin{itemize}
\item {Proveniência:(De \textunderscore têrço\textunderscore )}
\end{itemize}
A terceira ou a mais interna das membranas, que revestem a núcula do ovário de certas plantas.
O mesmo que \textunderscore tercilho\textunderscore .
\section{Terciodécimo}
\begin{itemize}
\item {Grp. gram.:adj.}
\end{itemize}
\begin{itemize}
\item {Proveniência:(Do lat. \textunderscore tertius\textunderscore  + \textunderscore decimus\textunderscore )}
\end{itemize}
Décimo terceiro.
\section{Tercionário}
\begin{itemize}
\item {Grp. gram.:m.  e  adj.}
\end{itemize}
O que tem terçans.
\section{Tèrciopêlo}
\begin{itemize}
\item {Grp. gram.:m.}
\end{itemize}
Velludo de três pêlos; velludo bem coberto de pêlo.
(Cast. \textunderscore terciopelo\textunderscore )
\section{Tèrciopeludo}
\begin{itemize}
\item {Grp. gram.:adj.}
\end{itemize}
\begin{itemize}
\item {Proveniência:(De \textunderscore tèrciopêlo\textunderscore )}
\end{itemize}
Que tem muito pêlo:«\textunderscore ...gato... gordo e nédio, grande e tèrciopeludo\textunderscore ». Filinto, XIII, 12.
\section{Tèrciphalange}
\begin{itemize}
\item {Grp. gram.:f.}
\end{itemize}
\begin{itemize}
\item {Utilização:Anat.}
\end{itemize}
Terceira phalange do pé.
\section{Tèrciphalangeta}
\begin{itemize}
\item {Grp. gram.:f.}
\end{itemize}
\begin{itemize}
\item {Utilização:Anat.}
\end{itemize}
Terceira phalangeta do pé.
\section{Tèrciphalanginha}
\begin{itemize}
\item {Grp. gram.:f.}
\end{itemize}
\begin{itemize}
\item {Utilização:Anat.}
\end{itemize}
Terceira phalanginha do pé.
\section{Terco}
\begin{itemize}
\item {Grp. gram.:adj.}
\end{itemize}
\begin{itemize}
\item {Utilização:Des.}
\end{itemize}
Teimoso, pertinaz.
(Cast. \textunderscore terco\textunderscore )
\section{Têrço}
\begin{itemize}
\item {Grp. gram.:m.}
\end{itemize}
\begin{itemize}
\item {Utilização:Náut.}
\end{itemize}
\begin{itemize}
\item {Proveniência:(Do lat. \textunderscore tertius\textunderscore )}
\end{itemize}
A têrça parte de qualquer coisa.
Têrça parte do rosário.
A têrça parte da espada, mais próxima do punho.
A têrça parte de um fuste.
Antigo corpo de tropas; regimento.
Parte da vêrga, dos paus, etc, igualmente distante dos extremos:«\textunderscore a vêrga partiu pelo têrço...\textunderscore »
\section{Terçô}
\begin{itemize}
\item {Grp. gram.:adj.}
\end{itemize}
\begin{itemize}
\item {Grp. gram.:M.}
\end{itemize}
\begin{itemize}
\item {Proveniência:(Do cast. \textunderscore terzuelo\textunderscore )}
\end{itemize}
Dizia-se do animal que, de uma ninhada, fôra o último a nascer.
Falcão macho, ou açor, ou gerifalte.
\section{Terçó}
\begin{itemize}
\item {Grp. gram.:adj.}
\end{itemize}
\begin{itemize}
\item {Grp. gram.:M.}
\end{itemize}
\begin{itemize}
\item {Proveniência:(Do cast. \textunderscore terzuelo\textunderscore )}
\end{itemize}
Dizia-se do animal que, de uma ninhada, fôra o último a nascer.
Falcão macho, ou açor, ou gerifalte.
\section{Terçogo}
\begin{itemize}
\item {Grp. gram.:m.}
\end{itemize}
\begin{itemize}
\item {Utilização:Prov.}
\end{itemize}
\begin{itemize}
\item {Utilização:minh.}
\end{itemize}
\begin{itemize}
\item {Utilização:Prov.}
\end{itemize}
\begin{itemize}
\item {Utilização:dur.}
\end{itemize}
O mesmo que \textunderscore terçol\textunderscore .
O último bácoro de uma ninhada.
\section{Terçoínho}
\begin{itemize}
\item {Grp. gram.:m.}
\end{itemize}
\begin{itemize}
\item {Utilização:Prov.}
\end{itemize}
\begin{itemize}
\item {Utilização:minh.}
\end{itemize}
\begin{itemize}
\item {Proveniência:(De \textunderscore terçô\textunderscore )}
\end{itemize}
O mais novo de todos, (falando-se de crianças ou de animaes).
\section{Terçol}
\begin{itemize}
\item {Grp. gram.:m.}
\end{itemize}
Pequeno tumor no bôrdo das pálpebras.
(Cp. \textunderscore terçolho\textunderscore )
\section{Terçolho}
\begin{itemize}
\item {fónica:çô}
\end{itemize}
\begin{itemize}
\item {Grp. gram.:m.}
\end{itemize}
\begin{itemize}
\item {Utilização:Pop.}
\end{itemize}
\begin{itemize}
\item {Utilização:Prov.}
\end{itemize}
\begin{itemize}
\item {Utilização:minh.}
\end{itemize}
O mesmo que \textunderscore terçol\textunderscore .
O mesmo que \textunderscore terçol\textunderscore .
(Corr. de \textunderscore terçol\textunderscore , sob a infl. de \textunderscore ôlho\textunderscore ? Ou formação de \textunderscore ôlho\textunderscore  com outro elemento desconhecido?)
\section{Terebelária}
\begin{itemize}
\item {Grp. gram.:f.}
\end{itemize}
\begin{itemize}
\item {Proveniência:(Do lat. \textunderscore terebella\textunderscore )}
\end{itemize}
Gênero de polipeiros fósseis.
\section{Terebellária}
\begin{itemize}
\item {Grp. gram.:f.}
\end{itemize}
\begin{itemize}
\item {Proveniência:(Do lat. \textunderscore terebella\textunderscore )}
\end{itemize}
Gênero de polypeiros fósseis.
\section{Terebênico}
\begin{itemize}
\item {Grp. gram.:adj.}
\end{itemize}
\begin{itemize}
\item {Utilização:Chím.}
\end{itemize}
Diz-se de um grupo de carbonetos.
\section{Terebintáceas}
\begin{itemize}
\item {Grp. gram.:f. pl.}
\end{itemize}
Família de plantas, que têm por tipo o terebinto.
\section{Terebínteas}
\begin{itemize}
\item {Grp. gram.:f. pl.}
\end{itemize}
\begin{itemize}
\item {Proveniência:(De \textunderscore terebinto\textunderscore )}
\end{itemize}
Ordem de plantas, que abrange as terebintáceas e as juglândeas.
\section{Terebintháceas}
\begin{itemize}
\item {Grp. gram.:f. pl.}
\end{itemize}
Família de plantas, que têm por typo o terebintho.
\section{Terebíntheas}
\begin{itemize}
\item {Grp. gram.:f. pl.}
\end{itemize}
\begin{itemize}
\item {Proveniência:(De \textunderscore terebintho\textunderscore )}
\end{itemize}
Ordem de plantas, que abrange as terebintháceas e as juglândeas.
\section{Terebinthina}
\begin{itemize}
\item {Grp. gram.:f.}
\end{itemize}
\begin{itemize}
\item {Proveniência:(Lat. \textunderscore terebinthina\textunderscore )}
\end{itemize}
Nome commum ás resinas líquidas, semi-líquidas e glutinosas, que se extrahem de árvores coníferas e terebintháceas.
\section{Terebinthinar}
\begin{itemize}
\item {Grp. gram.:v. t.}
\end{itemize}
Preparar ou misturar com terebinthina.
\section{Terebintho}
\begin{itemize}
\item {Grp. gram.:m.}
\end{itemize}
\begin{itemize}
\item {Proveniência:(Lat. \textunderscore terebinthus\textunderscore )}
\end{itemize}
O mesmo que \textunderscore lentisco\textunderscore  ou \textunderscore almecegueira\textunderscore .
\section{Terebintina}
\begin{itemize}
\item {Grp. gram.:f.}
\end{itemize}
\begin{itemize}
\item {Proveniência:(Lat. \textunderscore terebinthina\textunderscore )}
\end{itemize}
Nome comum ás resinas líquidas, semi-líquidas e glutinosas, que se extrahem de árvores coníferas e terebintáceas.
\section{Terebintinar}
\begin{itemize}
\item {Grp. gram.:v. t.}
\end{itemize}
Preparar ou misturar com terebintina.
\section{Terebinto}
\begin{itemize}
\item {Grp. gram.:m.}
\end{itemize}
\begin{itemize}
\item {Proveniência:(Lat. \textunderscore terebinthus\textunderscore )}
\end{itemize}
O mesmo que \textunderscore lentisco\textunderscore  ou \textunderscore almecegueira\textunderscore .
\section{Térebra}
\begin{itemize}
\item {Grp. gram.:f.}
\end{itemize}
\begin{itemize}
\item {Proveniência:(Lat. \textunderscore terebra\textunderscore )}
\end{itemize}
Máquina de guerra, com que se furavam as muralhas.
\section{Terebração}
\begin{itemize}
\item {Grp. gram.:f.}
\end{itemize}
\begin{itemize}
\item {Utilização:Med.}
\end{itemize}
\begin{itemize}
\item {Proveniência:(Do lat. \textunderscore terebratio\textunderscore )}
\end{itemize}
Acto de terebrar.
Sensação de dôres, como se fossem produzidas por objecto perfurante.
\section{Terebrante}
\begin{itemize}
\item {Grp. gram.:adj.}
\end{itemize}
\begin{itemize}
\item {Grp. gram.:M. Pl.}
\end{itemize}
\begin{itemize}
\item {Proveniência:(Lat. \textunderscore terebrans\textunderscore )}
\end{itemize}
Que terébra.
Diz-se das dôres, cuja sensação é comparável á que produziria uma verruma, penetrando no corpo.
Família de insectos hymenópteros.
\section{Terebrar}
\begin{itemize}
\item {Grp. gram.:v. t.}
\end{itemize}
\begin{itemize}
\item {Proveniência:(Lat. \textunderscore terebrare\textunderscore )}
\end{itemize}
Furar com verruma; furar; penetrar.
\section{Terebrátula}
\begin{itemize}
\item {Grp. gram.:f.}
\end{itemize}
\begin{itemize}
\item {Proveniência:(Do lat. \textunderscore terebra\textunderscore )}
\end{itemize}
Gênero de molluscos, que comprehende muitas espécies vivas e muitas espécies fósseis.
\section{Terecena}
\begin{itemize}
\item {Grp. gram.:f.}
\end{itemize}
O mesmo que \textunderscore tercena\textunderscore . Cf. Herculano e \textunderscore Peregrinação\textunderscore , LX.
\section{Terédem}
\begin{itemize}
\item {Grp. gram.:m.}
\end{itemize}
\begin{itemize}
\item {Proveniência:(Do lat. \textunderscore teredo\textunderscore , \textunderscore teredinis\textunderscore )}
\end{itemize}
Mollusco acéphalo e tubicolado, que vive debaixo da água, nas fendas dos navios, etc.
\section{Teredilos}
\begin{itemize}
\item {Grp. gram.:m. Pl.}
\end{itemize}
\begin{itemize}
\item {Proveniência:(Do gr. \textunderscore teredon\textunderscore  + \textunderscore ule\textunderscore )}
\end{itemize}
Família de insectos coleópteros pentâmeros.
\section{Teredo}
\begin{itemize}
\item {Grp. gram.:m.}
\end{itemize}
O mesmo que \textunderscore terédem\textunderscore . Cf. Ortigão, \textunderscore Praias\textunderscore , 17.
\section{Teredylos}
\begin{itemize}
\item {Grp. gram.:m. Pl.}
\end{itemize}
\begin{itemize}
\item {Proveniência:(Do gr. \textunderscore teredon\textunderscore  + \textunderscore ule\textunderscore )}
\end{itemize}
Família de insectos coleópteros pentâmeros.
\section{Terém}
\begin{itemize}
\item {Grp. gram.:m.}
\end{itemize}
\begin{itemize}
\item {Utilização:Prov.}
\end{itemize}
Verme, nocivo á raiz do milho nascente.
O mesmo que \textunderscore terédem\textunderscore ?
\section{Tereniabim}
\begin{itemize}
\item {Grp. gram.:m.}
\end{itemize}
\begin{itemize}
\item {Utilização:Pharm.}
\end{itemize}
Manná líquido.
Substância viscosa, branca e doce, que se acha adherente ás fôlhas de certas árvores da Pérsia e que tem propriedades purgativas.
(Cast. \textunderscore tereniabin\textunderscore )
\section{Teres}
\begin{itemize}
\item {fónica:tê}
\end{itemize}
\begin{itemize}
\item {Grp. gram.:m. Pl.}
\end{itemize}
\begin{itemize}
\item {Grp. gram.:Loc.}
\end{itemize}
\begin{itemize}
\item {Utilização:pop.}
\end{itemize}
\begin{itemize}
\item {Proveniência:(De \textunderscore têr\textunderscore )}
\end{itemize}
Posses, haveres, bens.
\textunderscore Fazer teres\textunderscore , diz-se das crianças, que começam a suster-se nas pernas. Cp. \textunderscore tem-tem\textunderscore .
\section{Tereticaude}
\begin{itemize}
\item {Grp. gram.:adj.}
\end{itemize}
\begin{itemize}
\item {Utilização:Zool.}
\end{itemize}
\begin{itemize}
\item {Proveniência:(Do lat. \textunderscore teres\textunderscore  + \textunderscore cauda\textunderscore )}
\end{itemize}
Que tem cauda delgada.
\section{Tereticollo}
\begin{itemize}
\item {Grp. gram.:adj.}
\end{itemize}
\begin{itemize}
\item {Utilização:Zool.}
\end{itemize}
\begin{itemize}
\item {Proveniência:(Do lat. \textunderscore teres\textunderscore  + \textunderscore collum\textunderscore )}
\end{itemize}
Que tem pescoço delgado.
\section{Tereticolo}
\begin{itemize}
\item {Grp. gram.:adj.}
\end{itemize}
\begin{itemize}
\item {Utilização:Zool.}
\end{itemize}
\begin{itemize}
\item {Proveniência:(Do lat. \textunderscore teres\textunderscore  + \textunderscore collum\textunderscore )}
\end{itemize}
Que tem pescoço delgado.
\section{Teretifoliado}
\begin{itemize}
\item {Grp. gram.:adj.}
\end{itemize}
\begin{itemize}
\item {Utilização:Bot.}
\end{itemize}
\begin{itemize}
\item {Proveniência:(Do lat. \textunderscore teres\textunderscore  + \textunderscore folium\textunderscore )}
\end{itemize}
Que tem fôlhas delgadas.
\section{Teretiforme}
\begin{itemize}
\item {Grp. gram.:adj.}
\end{itemize}
\begin{itemize}
\item {Grp. gram.:M. Pl.}
\end{itemize}
\begin{itemize}
\item {Proveniência:(Do lat. \textunderscore teres\textunderscore  + \textunderscore forma\textunderscore )}
\end{itemize}
Cilíndrico.
Insectos coleópteros, de corpo teretiforme e antenas granulosas.
\section{Teretirostro}
\begin{itemize}
\item {fónica:rós}
\end{itemize}
\begin{itemize}
\item {Grp. gram.:adj.}
\end{itemize}
\begin{itemize}
\item {Utilização:Zool.}
\end{itemize}
\begin{itemize}
\item {Proveniência:(Do lat. \textunderscore teres\textunderscore  + \textunderscore rostrum\textunderscore )}
\end{itemize}
Que tem bico delgado.
\section{Teretirrostro}
\begin{itemize}
\item {Grp. gram.:adj.}
\end{itemize}
\begin{itemize}
\item {Utilização:Zool.}
\end{itemize}
\begin{itemize}
\item {Proveniência:(Do lat. \textunderscore teres\textunderscore  + \textunderscore rostrum\textunderscore )}
\end{itemize}
Que tem bico delgado.
\section{Tergal}
\begin{itemize}
\item {Grp. gram.:adj.}
\end{itemize}
\begin{itemize}
\item {Utilização:Zool.}
\end{itemize}
\begin{itemize}
\item {Proveniência:(Do lat. \textunderscore tergum\textunderscore )}
\end{itemize}
Relativo ao dorso dos insectos.
\section{Tergeminado}
\begin{itemize}
\item {Grp. gram.:adj.}
\end{itemize}
\begin{itemize}
\item {Utilização:Bot.}
\end{itemize}
\begin{itemize}
\item {Proveniência:(De \textunderscore tergêmino\textunderscore )}
\end{itemize}
Triplicado, (falando-se das fôlhas das plantas).
\section{Tergêmino}
\begin{itemize}
\item {Grp. gram.:adj.}
\end{itemize}
\begin{itemize}
\item {Proveniência:(Lat. \textunderscore tergeminus\textunderscore )}
\end{itemize}
Tríplice; trigêmeo.
\section{Tergiversação}
\begin{itemize}
\item {Grp. gram.:f.}
\end{itemize}
\begin{itemize}
\item {Proveniência:(Do lat. \textunderscore tergiversatio\textunderscore )}
\end{itemize}
Acto ou effeito de tergiversar.
Evasiva; rodeios.
\section{Tergiversador}
\begin{itemize}
\item {Grp. gram.:m.  e  adj.}
\end{itemize}
\begin{itemize}
\item {Proveniência:(Do lat. \textunderscore tergiversator\textunderscore )}
\end{itemize}
O que tergiversa.
\section{Tergiversante}
\begin{itemize}
\item {Grp. gram.:adj.}
\end{itemize}
\begin{itemize}
\item {Proveniência:(Lat. \textunderscore tergiversans\textunderscore )}
\end{itemize}
Que tergiversa.
\section{Tergiversar}
\begin{itemize}
\item {Grp. gram.:v. i.}
\end{itemize}
\begin{itemize}
\item {Utilização:Fig.}
\end{itemize}
\begin{itemize}
\item {Proveniência:(Lat. \textunderscore tergiversari\textunderscore )}
\end{itemize}
Voltar as costas.
Usar de subterfúgios.
Buscar evasivas.
\section{Tergiversável}
\begin{itemize}
\item {Grp. gram.:adj.}
\end{itemize}
Capaz de tergiversar.
\section{Tergo}
\begin{itemize}
\item {Grp. gram.:m.}
\end{itemize}
\begin{itemize}
\item {Utilização:Poét.}
\end{itemize}
\begin{itemize}
\item {Proveniência:(Lat. \textunderscore tergum\textunderscore )}
\end{itemize}
O dorso, as costas.
\section{Terícia}
\begin{itemize}
\item {Grp. gram.:f.}
\end{itemize}
\begin{itemize}
\item {Utilização:Pop.}
\end{itemize}
O mesmo que \textunderscore icterícia\textunderscore . Cf. B. Pereira, \textunderscore Prosodia\textunderscore , vb. \textunderscore aurigo\textunderscore .
\section{Terjurar}
\begin{itemize}
\item {Grp. gram.:v. i.}
\end{itemize}
O mesmo que \textunderscore trejurar\textunderscore . Cf. F. Manuel, \textunderscore Apólogos\textunderscore .
\section{Terlinta}
\begin{itemize}
\item {Grp. gram.:m.  e  f.}
\end{itemize}
\begin{itemize}
\item {Utilização:Prov.}
\end{itemize}
\begin{itemize}
\item {Proveniência:(De \textunderscore terlintar\textunderscore )}
\end{itemize}
Pessôa chocalheira, tagarela.
\section{Terlintar}
\begin{itemize}
\item {Grp. gram.:v. i.}
\end{itemize}
\begin{itemize}
\item {Utilização:Prov.}
\end{itemize}
O mesmo que \textunderscore telintar\textunderscore .
\section{Terlintim}
\begin{itemize}
\item {Grp. gram.:m.}
\end{itemize}
Som daquillo que terlinta. Cf. Castilho, \textunderscore Fausto\textunderscore , 310.
\section{Terma}
\begin{itemize}
\item {Grp. gram.:f.}
\end{itemize}
O mesmo que \textunderscore terme\textunderscore .
\section{Terme}
\begin{itemize}
\item {Grp. gram.:m.  e  f.}
\end{itemize}
O mesmo que \textunderscore térmite\textunderscore .
\section{Termes}
\begin{itemize}
\item {Grp. gram.:m.  e  f.}
\end{itemize}
O mesmo que \textunderscore térmite\textunderscore .
\section{Termilionesimal}
\begin{itemize}
\item {Grp. gram.:adj.}
\end{itemize}
O mesmo que \textunderscore termilionésimo\textunderscore .
\section{Termilionésimo}
\begin{itemize}
\item {Grp. gram.:adj.}
\end{itemize}
\begin{itemize}
\item {Utilização:Neol.}
\end{itemize}
\begin{itemize}
\item {Proveniência:(De \textunderscore ter\textunderscore  lat. + \textunderscore milionésimo\textunderscore )}
\end{itemize}
Relativo á terça parte de uma milionésima.
\section{Termillionésimo}
\begin{itemize}
\item {Grp. gram.:adj.}
\end{itemize}
\begin{itemize}
\item {Utilização:Neol.}
\end{itemize}
\begin{itemize}
\item {Proveniência:(De \textunderscore ter\textunderscore  lat. + \textunderscore millionésimo\textunderscore )}
\end{itemize}
Relativo á terça parte de uma millionésima.
\section{Termillionesimal}
\begin{itemize}
\item {Grp. gram.:adj.}
\end{itemize}
O mesmo que \textunderscore termillionésimo\textunderscore .
\section{Terminação}
\begin{itemize}
\item {Grp. gram.:f.}
\end{itemize}
\begin{itemize}
\item {Proveniência:(Do lat. \textunderscore terminatio\textunderscore )}
\end{itemize}
Acto ou effeito de terminar.
Conclusão.
Desinência das palavras.
Extremidade, remate.
\section{Terminaes}
\begin{itemize}
\item {Grp. gram.:f. pl.}
\end{itemize}
\begin{itemize}
\item {Proveniência:(Lat. \textunderscore terminalia\textunderscore )}
\end{itemize}
Festas romanas, em honra do deus Termo.
\section{Terminais}
\begin{itemize}
\item {Grp. gram.:f. pl.}
\end{itemize}
\begin{itemize}
\item {Proveniência:(Lat. \textunderscore terminalia\textunderscore )}
\end{itemize}
Festas romanas, em honra do deus Termo.
\section{Terminal}
\begin{itemize}
\item {Grp. gram.:adj.}
\end{itemize}
\begin{itemize}
\item {Proveniência:(Lat. \textunderscore terminalis\textunderscore )}
\end{itemize}
Relativo ao termo ou ao remate.
Que constitue o termo ou extremidade.
Relativo á demarcação dos campos.
\section{Terminália}
\begin{itemize}
\item {Grp. gram.:f.}
\end{itemize}
Gênero de plantas, da fam. das compostas.
\section{Terminante}
\begin{itemize}
\item {Grp. gram.:adj.}
\end{itemize}
\begin{itemize}
\item {Proveniência:(Lat. \textunderscore terminans\textunderscore )}
\end{itemize}
Que termina.
Categórico; decisivo; \textunderscore declarações terminantes\textunderscore .
\section{Terminantemente}
\begin{itemize}
\item {Grp. gram.:adv.}
\end{itemize}
De modo terminante.
Categoricamente.
\section{Terminar}
\begin{itemize}
\item {Grp. gram.:v. t.}
\end{itemize}
\begin{itemize}
\item {Grp. gram.:V. i.  e  p.}
\end{itemize}
\begin{itemize}
\item {Proveniência:(Lat. \textunderscore terminare\textunderscore )}
\end{itemize}
Pôr termo a; concluir.
Estar na extremidade ou occupar a extremidade de.
Lindar, demarcar.
Confinar.
Têr certa desinência, (falando-se de vocábulos).
\section{Terminativamente}
\begin{itemize}
\item {Grp. gram.:adv.}
\end{itemize}
De modo terminativo.
Quanto ao termo ou objecto.
\section{Terminativo}
\begin{itemize}
\item {Grp. gram.:adj.}
\end{itemize}
\begin{itemize}
\item {Utilização:Gram.}
\end{itemize}
\begin{itemize}
\item {Proveniência:(De \textunderscore terminar\textunderscore )}
\end{itemize}
O mesmo que \textunderscore terminante\textunderscore .
Que faz terminar.
O mesmo que \textunderscore indirecto\textunderscore , (falando-se de complementos grammaticaes).
\section{Término}
\begin{itemize}
\item {Grp. gram.:m.}
\end{itemize}
O mesmo que \textunderscore têrmo\textunderscore .
Baliza; limite.
\section{Terminologia}
\begin{itemize}
\item {Grp. gram.:f.}
\end{itemize}
\begin{itemize}
\item {Proveniência:(Do lat. \textunderscore terminus\textunderscore  + gr. \textunderscore logos\textunderscore )}
\end{itemize}
Tratado dos termos téchnicos de uma sciência ou arte.
Conjunto dêsses termos.
Emprêgo de palavras peculiares a um escritor.
\section{Terminologista}
\begin{itemize}
\item {Grp. gram.:m.}
\end{itemize}
Aquelle que se occupa de terminologia.
\section{Termínos}
\begin{itemize}
\item {Grp. gram.:m. pl. Loc. adv.}
\end{itemize}
\begin{itemize}
\item {Utilização:Prov.}
\end{itemize}
\begin{itemize}
\item {Utilização:minh.}
\end{itemize}
\textunderscore Em terminos\textunderscore , quási; a ponto; em risco:«\textunderscore ...em terminos de me levar o diabo.\textunderscore »Camillo, \textunderscore Brasileira\textunderscore , 288.
(Talvez alter. de \textunderscore terminhos\textunderscore , dem. de \textunderscore termos\textunderscore , se não vem do lat. \textunderscore terminus\textunderscore , com hyperbibasmo ou deslocação de accento)
\section{Térmita}
\begin{itemize}
\item {Grp. gram.:f.}
\end{itemize}
\begin{itemize}
\item {Proveniência:(Lat. \textunderscore termes\textunderscore , \textunderscore termitis\textunderscore )}
\end{itemize}
(geralmente dizem \textunderscore termíta\textunderscore )
Gênero de neurópteros ou insectos roedores, também conhecidos por formigas brancas.
\section{Térmite}
\begin{itemize}
\item {Grp. gram.:f.}
\end{itemize}
\begin{itemize}
\item {Proveniência:(Lat. \textunderscore termes\textunderscore , \textunderscore termitis\textunderscore )}
\end{itemize}
Gênero de neurópteros ou insectos roedores, também conhecidos por formigas brancas.
\section{Termiteira}
\begin{itemize}
\item {Grp. gram.:f.}
\end{itemize}
\begin{itemize}
\item {Proveniência:(De \textunderscore térmite\textunderscore )}
\end{itemize}
Habitação de alguns insectos neurópteros, construída de barro em fórma de pyrâmide, chegando a attingir 2 metros de altura.
\section{Termítico}
\begin{itemize}
\item {Grp. gram.:adj.}
\end{itemize}
Relativo á térmite.
\section{Têrmo}
\begin{itemize}
\item {Grp. gram.:m.}
\end{itemize}
\begin{itemize}
\item {Utilização:Gram.}
\end{itemize}
\begin{itemize}
\item {Grp. gram.:Pl.}
\end{itemize}
\begin{itemize}
\item {Grp. gram.:Loc. adv.}
\end{itemize}
\begin{itemize}
\item {Grp. gram.:Loc. conj.}
\end{itemize}
\begin{itemize}
\item {Proveniência:(Do lat. \textunderscore terminus\textunderscore )}
\end{itemize}
Limite, baliza: \textunderscore o termo de uma propriedade\textunderscore .
Fim: \textunderscore o termo da vida\textunderscore .
Prazo, tempo prefixo.
Palavra, expressão: \textunderscore usar termos clássicos\textunderscore .
Expressão peculiar a uma arte ou sciência.
Espaço.
Declaração exarada em processos judiciaes: \textunderscore lavrar um termo\textunderscore .
Maneira, modo: \textunderscore tudo tem meio termo\textunderscore .
Fórma, teor: \textunderscore e em taes termos lhe falou...\textunderscore 
Área: \textunderscore no termo do concelho\textunderscore .
Circunvizinhança.
Praia; confins.
Cada um dos elementos de uma proposição.
Cada um dos objectos, que se comparam ou têm relação entre si.
Cada uma das quantidades mathemáticas, que compõem uma proporção, uma progressão, etc.
Modos, procedimento, acções.
\textunderscore A termo de\textunderscore , a ponto de.
\textunderscore A termo que\textunderscore , de maneira que:«\textunderscore ...a termos que o padre já lhe não dava a hóstia.\textunderscore »Camillo, \textunderscore Volcoens\textunderscore , 52.
\section{Térmo}
\begin{itemize}
\item {Grp. gram.:m.}
\end{itemize}
\begin{itemize}
\item {Proveniência:(Do lat. \textunderscore Terminus\textunderscore , n. p.)}
\end{itemize}
Marco, ou pilastra, que sustinha o busto de Marcúrio e servia, entre os Romanos, para assinalar milhas itinerárias ou para marcar o limite de um território:«\textunderscore todos prometem sêr mais firmes que um térmo.\textunderscore »Filinto, XIII, 162.
\section{Ternado}
\begin{itemize}
\item {Grp. gram.:m.}
\end{itemize}
\begin{itemize}
\item {Utilização:Bot.}
\end{itemize}
\begin{itemize}
\item {Proveniência:(De \textunderscore terno\textunderscore ^1)}
\end{itemize}
Diz-se das partes de uma planta, dispostas em grupos de três.
\section{Ternal}
\begin{itemize}
\item {Grp. gram.:adj.}
\end{itemize}
\begin{itemize}
\item {Utilização:bras}
\end{itemize}
\begin{itemize}
\item {Utilização:Neol.}
\end{itemize}
O mesmo que \textunderscore ternário\textunderscore .
\section{Ternamente}
\begin{itemize}
\item {Grp. gram.:adv.}
\end{itemize}
De modo terno^2.
Com ternura.
Meigamente; suavemente.
\section{Ternário}
\begin{itemize}
\item {Grp. gram.:adj.}
\end{itemize}
\begin{itemize}
\item {Proveniência:(Lat. \textunderscore ternarius\textunderscore )}
\end{itemize}
Formado de três.
Diz-se do compasso musical, dividido em três tempos iguaes.
\section{Ternates}
\begin{itemize}
\item {Grp. gram.:m. pl.}
\end{itemize}
O mesmo que \textunderscore Ternateses\textunderscore . Cf. Couto, \textunderscore Déc.\textunderscore  IX, c. 30.
\section{Ternateses}
\begin{itemize}
\item {fónica:tê}
\end{itemize}
\begin{itemize}
\item {Grp. gram.:m. Pl.}
\end{itemize}
Habitantes de Ternate. Cf. Couto, \textunderscore Déc.\textunderscore  VI, l. XI, c. 10 e \textunderscore Déc.\textunderscore  IV, l. VIII, c. 1; etc.
\section{Terne}
\begin{itemize}
\item {Grp. gram.:m.}
\end{itemize}
\begin{itemize}
\item {Utilização:Ant.}
\end{itemize}
\begin{itemize}
\item {Utilização:Gír.}
\end{itemize}
Costas.
\section{Terneira}
\begin{itemize}
\item {Grp. gram.:f.}
\end{itemize}
O mesmo que \textunderscore tenreira\textunderscore .
(Cp. \textunderscore terneiro\textunderscore )
\section{Terneiro}
\begin{itemize}
\item {Grp. gram.:m.}
\end{itemize}
\begin{itemize}
\item {Utilização:Bras. do S}
\end{itemize}
O mesmo que \textunderscore tenreiro\textunderscore .
(Cast. \textunderscore ternero\textunderscore )
\section{Terneza}
\begin{itemize}
\item {Grp. gram.:f.}
\end{itemize}
\begin{itemize}
\item {Utilização:Des.}
\end{itemize}
O mesmo que \textunderscore ternura\textunderscore .
Expressão terna; carinho:«\textunderscore ...arrazoou muitas ternezas, com que amolgasse seus animos á compaixão.\textunderscore »Filinto, \textunderscore D. Man.\textunderscore , III, 158.
\section{Terno}
\begin{itemize}
\item {Grp. gram.:m.}
\end{itemize}
\begin{itemize}
\item {Utilização:Bras. de Minas}
\end{itemize}
\begin{itemize}
\item {Proveniência:(Do lat. \textunderscore terni\textunderscore )}
\end{itemize}
Grupo de três coisas ou pessôas; trio.
Dado ou carta de jogar, com três pintas.
O mesmo que \textunderscore trindade\textunderscore . Cf. \textunderscore Luz e Calor\textunderscore , 538.
Grupo de pessôas, pouco ou muito numeroso: \textunderscore juntou-se ali um terno de vadios\textunderscore .
\section{Terno}
\begin{itemize}
\item {Grp. gram.:adj.}
\end{itemize}
Meigo; sensível.
Affectuoso.
Que causa dó.
Brando; suave.
(Metáth. de \textunderscore tenro\textunderscore )
\section{Ternstrêmia}
\begin{itemize}
\item {Grp. gram.:f.}
\end{itemize}
\begin{itemize}
\item {Proveniência:(De \textunderscore Ternstroem\textunderscore , n. p.)}
\end{itemize}
Gênero de grandes árvores da América tropical.
\section{Ternstremiáceas}
\begin{itemize}
\item {Grp. gram.:f. pl.}
\end{itemize}
Família de plantas, que tem por typo a ternstrêmia.
\section{Ternura}
\begin{itemize}
\item {Grp. gram.:f.}
\end{itemize}
Qualidade do que é terno^2.
Meiguice, carinho.
Affecto brando ou sem grandes transportes.
\section{Tèroléro}
\begin{itemize}
\item {Grp. gram.:m.}
\end{itemize}
\begin{itemize}
\item {Utilização:Ant.}
\end{itemize}
\begin{itemize}
\item {Utilização:Prov.}
\end{itemize}
\begin{itemize}
\item {Utilização:trasm.}
\end{itemize}
\begin{itemize}
\item {Utilização:beir.}
\end{itemize}
\begin{itemize}
\item {Utilização:Prov.}
\end{itemize}
Espécie de dança popular.
Pessôa leviana, adoidada.
Taramela do moínho.
\section{Terpeno}
\begin{itemize}
\item {Grp. gram.:m.}
\end{itemize}
\begin{itemize}
\item {Utilização:Chím.}
\end{itemize}
Um dos carbonetos do grupo benzênico.
\section{Terpina}
\begin{itemize}
\item {Grp. gram.:f.}
\end{itemize}
Medicamento diurético e antineurálgico.
\section{Terpinol}
\begin{itemize}
\item {Grp. gram.:m.}
\end{itemize}
Hydrato de terebinthina, resultante da combinação de duas moléculas de essência com uma de água.
\section{Terpola}
\begin{itemize}
\item {fónica:pô}
\end{itemize}
\begin{itemize}
\item {Grp. gram.:f.}
\end{itemize}
\begin{itemize}
\item {Utilização:Prov.}
\end{itemize}
\begin{itemize}
\item {Utilização:trasm.}
\end{itemize}
Excrescência nodosa nos troncos ou ramos das árvores.
\section{Terra}
\begin{itemize}
\item {Grp. gram.:f.}
\end{itemize}
\begin{itemize}
\item {Utilização:Mathem.}
\end{itemize}
\begin{itemize}
\item {Utilização:Fig.}
\end{itemize}
\begin{itemize}
\item {Grp. gram.:Loc. adv.}
\end{itemize}
\begin{itemize}
\item {Utilização:Fig.}
\end{itemize}
\begin{itemize}
\item {Grp. gram.:Loc. adv.}
\end{itemize}
\begin{itemize}
\item {Proveniência:(Lat. \textunderscore terra\textunderscore )}
\end{itemize}
Solo, sôbre que se anda.
A parte branda do solo, que produz os vegetaes.
Planeta que habitamos; (nesta accepção, convém inicial maiúscula, como em \textunderscore Sol\textunderscore  e \textunderscore Lua\textunderscore  quando nomes próprios de astros).
Parte sólida, da superfície dêste planeta.
Pó, poeira.
Povoação: \textunderscore Collares é linda terra\textunderscore .
Local.
Pátria.
Habitantes de uma povoação: \textunderscore escandalizou toda aquella terra\textunderscore .
Prédio rústico: \textunderscore possue umas terras na Outra Banda\textunderscore .
Campo, terreno.
Região.
Território.
Argilla, própria para esculpturas.
Vida temporal.
\textunderscore Terra vegetal\textunderscore , o mesmo que \textunderscore terriço\textunderscore .
\textunderscore Linha de terra\textunderscore , traço do geometral sôbre o plano do quadro.
\textunderscore Terra da verdade\textunderscore , a vida eterna; a sepultura.
\textunderscore Comer terra\textunderscore , viver amarguradamente, soffrer grandes misérias.
\textunderscore Terra-terra\textunderscore  ou \textunderscore terra a terra\textunderscore , junto á costa; costeando.
Singelamente; rasteiramente; sem elevação: \textunderscore discorrer terra-terra\textunderscore .
\textunderscore A terra\textunderscore , abaixo; morra.
\textunderscore Terra-terra\textunderscore ! grito de marinheiros que, depois de longa viagem, avistam terra.
\textunderscore Terra franca\textunderscore , terreno argilioso, amarelado e gordo; terra barrenta. Cf. F. Lapa, \textunderscore Chím. Agr.\textunderscore 
\textunderscore Terra de urze\textunderscore , terriço, formado sob as urzes e outros arbustos pela quéda das fôlhas, e que é bom adubo para jardins.
\textunderscore Terra de Siena\textunderscore , espécie de ocre amarelo.
\textunderscore Terra de Sevilha\textunderscore , espécie de caparrosa, empregada para tingir de negro.
\section{Terrabinto}
\begin{itemize}
\item {Grp. gram.:adj.}
\end{itemize}
\begin{itemize}
\item {Utilização:Prov.}
\end{itemize}
Diz-se do rapaz travesso ou traquinas.
\section{Terraço}
\begin{itemize}
\item {Grp. gram.:m.}
\end{itemize}
\begin{itemize}
\item {Proveniência:(Do lat. \textunderscore terraceus\textunderscore )}
\end{itemize}
Cobertura plana de um edifício, feita de pedra ou de argamassa; eirado.
\section{Terra-cotta}
\begin{itemize}
\item {Grp. gram.:f.}
\end{itemize}
\begin{itemize}
\item {Proveniência:(T. it.)}
\end{itemize}
Substância de figurinhas de barro.
Productos cerâmicos e de esculptura, mais ou menos delicados, que foram cozidos em fornos.
\section{Terrada}
\begin{itemize}
\item {Grp. gram.:f.}
\end{itemize}
\begin{itemize}
\item {Proveniência:(Do ár. \textunderscore terrad\textunderscore )}
\end{itemize}
Pequeno navio asiático.
\section{Terrada}
\begin{itemize}
\item {Grp. gram.:f.}
\end{itemize}
Porção de terra?:«\textunderscore ...amontoou uma terrada, em que cavalgou artilharia.\textunderscore »Filinto, \textunderscore D. Man.\textunderscore , II, 91.
\section{Terrádego}
\begin{itemize}
\item {Grp. gram.:m.}
\end{itemize}
\begin{itemize}
\item {Utilização:Ant.}
\end{itemize}
\begin{itemize}
\item {Proveniência:(Do b. lat. \textunderscore terraticus\textunderscore )}
\end{itemize}
Imposto, que se paga pela occupação de um terreno, em que se faz barraca de feira ou se expõem quaesquer productos á venda.
Terreno, occupado para esse fim.
O que se pagava de renda pela posse e cultura de terra alheia.
Laudêmio de quarentena.
\section{Terradegueiro}
\begin{itemize}
\item {Grp. gram.:m.}
\end{itemize}
Cobrador de terrádegos.
\section{Terrádigo}
\begin{itemize}
\item {Grp. gram.:m.}
\end{itemize}
O mesmo ou melhór que \textunderscore terrádego\textunderscore . Cf. Herculano, \textunderscore Hist. de Port.\textunderscore , III, 355 e 356; IV, 410.
\section{Terrado}
\begin{itemize}
\item {Grp. gram.:m.}
\end{itemize}
\begin{itemize}
\item {Utilização:Pop.}
\end{itemize}
\begin{itemize}
\item {Utilização:Ant.}
\end{itemize}
\begin{itemize}
\item {Proveniência:(De \textunderscore terra\textunderscore )}
\end{itemize}
O mesmo que \textunderscore terraço\textunderscore .
O mesmo que \textunderscore terrádego\textunderscore .
Contribuição, que se pagava ao Bispo de Coímbra, de qualquer venda de propriedade que se fizesse no seu bispado.
\section{Terra-inglesa}
\begin{itemize}
\item {Grp. gram.:f.}
\end{itemize}
Designação vulgar do cimento.
\section{Terra-japónica}
\begin{itemize}
\item {Grp. gram.:f.}
\end{itemize}
Substância sêca e friável, que se obtém pela decocção do pau-ferro, e que é de muita applicação na Pintura, na Medicina, etc.
\section{Terral}
\begin{itemize}
\item {Grp. gram.:adj.}
\end{itemize}
\begin{itemize}
\item {Utilização:T. da Índ. Port}
\end{itemize}
Relativo á terra; que sopra da terra:«\textunderscore ...o vento terral...\textunderscore »Filinto, \textunderscore D. Man.\textunderscore , II, 46.
Diz-se da estação, em que predomina o vento que sopra da terra.
\section{Terralina}
\begin{itemize}
\item {Grp. gram.:f.}
\end{itemize}
\begin{itemize}
\item {Utilização:Chím.}
\end{itemize}
Excipiente, há pouco descoberto, amarelado e de cheiro terroso.
\section{Terramoto}
\begin{itemize}
\item {Grp. gram.:m.}
\end{itemize}
O mesmo que \textunderscore terremoto\textunderscore . Cf. Camillo, \textunderscore Perfil do Marquês\textunderscore , 3 e 28.
\section{Terramotada}
\begin{itemize}
\item {Grp. gram.:f.}
\end{itemize}
\begin{itemize}
\item {Utilização:Prov.}
\end{itemize}
\begin{itemize}
\item {Utilização:alg.}
\end{itemize}
\begin{itemize}
\item {Proveniência:(De \textunderscore terramoto\textunderscore )}
\end{itemize}
Grande ruído.
\section{Terra-nova}
\begin{itemize}
\item {Grp. gram.:m.}
\end{itemize}
\begin{itemize}
\item {Proveniência:(De \textunderscore Terra-Nova\textunderscore , n. p.)}
\end{itemize}
Cão, pertencente a uma raça, que dizem procedente da Terra-Nova.
\section{Terranquim}
\begin{itemize}
\item {Grp. gram.:m.}
\end{itemize}
Embarcação indiana.
\section{Terrantês}
\begin{itemize}
\item {Grp. gram.:adj.}
\end{itemize}
\begin{itemize}
\item {Grp. gram.:M.}
\end{itemize}
\begin{itemize}
\item {Proveniência:(De \textunderscore terra\textunderscore )}
\end{itemize}
Natural de uma terra ou povoação.
Espécie de uva branca.
\section{Terrão}
\begin{itemize}
\item {Grp. gram.:m.}
\end{itemize}
\begin{itemize}
\item {Utilização:Prov.}
\end{itemize}
\begin{itemize}
\item {Utilização:minh.}
\end{itemize}
O mesmo ou melhór que \textunderscore torrão\textunderscore .
Cerrado para o gado.
(Cp. cast. \textunderscore terrón\textunderscore )
\section{Terraplanar}
\textunderscore v. t.\textunderscore  (e der.)
Corr. usual de \textunderscore terraplenar\textunderscore , etc.
\section{Terraplenagem}
\begin{itemize}
\item {Grp. gram.:f.}
\end{itemize}
Acto ou effeito de terraplenar.
\section{Terraplenamento}
\begin{itemize}
\item {Grp. gram.:m.}
\end{itemize}
O mesmo que \textunderscore terraplenagem\textunderscore .
\section{Terraplenar}
\begin{itemize}
\item {Grp. gram.:v. t.}
\end{itemize}
Encher de terra.
Formar terrapleno em.
\section{Terrapleno}
\begin{itemize}
\item {Grp. gram.:m.}
\end{itemize}
\begin{itemize}
\item {Proveniência:(De \textunderscore terra\textunderscore  + \textunderscore pleno\textunderscore )}
\end{itemize}
Terreno, em que se encheu uma depressão ou cavidade, ficando plano.
Terreno, plano.
O mesmo que \textunderscore terraço\textunderscore . Cf. \textunderscore Peregrinação\textunderscore , LXXV.
\section{Terráqueo}
\begin{itemize}
\item {Grp. gram.:m.}
\end{itemize}
\begin{itemize}
\item {Proveniência:(De \textunderscore terra\textunderscore )}
\end{itemize}
Relativo ao globo terrestre; terrestre.
\section{Terrar}
\begin{itemize}
\item {Grp. gram.:v. t.}
\end{itemize}
\begin{itemize}
\item {Utilização:Prov.}
\end{itemize}
\begin{itemize}
\item {Utilização:alent.}
\end{itemize}
O mesmo que \textunderscore aterrar\textunderscore ^2, cobrir de terra.
\section{Terreal}
\begin{itemize}
\item {Grp. gram.:adj.}
\end{itemize}
Relativo á Terra; terrestre, mundano.
\section{Terrear}
\begin{itemize}
\item {Grp. gram.:v. i.}
\end{itemize}
\begin{itemize}
\item {Utilização:Pop.}
\end{itemize}
\begin{itemize}
\item {Utilização:Ant.}
\end{itemize}
\begin{itemize}
\item {Proveniência:(De \textunderscore terra\textunderscore )}
\end{itemize}
Mostrar-se sem vegetação (a terra).
Mostrar-se nua (a terra): \textunderscore em janeiro, se vires terrear, põe-te a cantar\textunderscore . (Prolóquio popular)
Aproximar-se da terra, navegando.
\section{Terregosa}
\begin{itemize}
\item {Grp. gram.:f.}
\end{itemize}
\begin{itemize}
\item {Utilização:Ant.}
\end{itemize}
\begin{itemize}
\item {Utilização:Gír.}
\end{itemize}
Epítheto de Lisbôa.
\section{Terreiro}
\begin{itemize}
\item {Grp. gram.:m.}
\end{itemize}
\begin{itemize}
\item {Grp. gram.:Adj.}
\end{itemize}
\begin{itemize}
\item {Proveniência:(Lat. \textunderscore terrarium\textunderscore )}
\end{itemize}
Espaço de terra, plano e largo.
Praça.
Terraço.
Lugar ao ar livro, onde há folguedos ou cantos ao desafio.
Lugar, onde os bèsteiros faziam exercícios.
O mesmo que \textunderscore térreo\textunderscore : \textunderscore casa terreira\textunderscore .
\section{Terrejola}
\begin{itemize}
\item {Grp. gram.:f.}
\end{itemize}
O mesmo que \textunderscore terriola\textunderscore .
\section{Terremoto}
\begin{itemize}
\item {Grp. gram.:m.}
\end{itemize}
\begin{itemize}
\item {Utilização:Fig.}
\end{itemize}
\begin{itemize}
\item {Proveniência:(Lat. \textunderscore terraemotus\textunderscore )}
\end{itemize}
Movimento ou abalo da superfície da terra.
Tremor de terra.
Grande estrondo.
\section{Terrenal}
\begin{itemize}
\item {Grp. gram.:adj.}
\end{itemize}
\begin{itemize}
\item {Proveniência:(De \textunderscore terreno\textunderscore )}
\end{itemize}
O mesmo que \textunderscore terreal\textunderscore .
\section{Terrenamente}
\begin{itemize}
\item {Grp. gram.:adv.}
\end{itemize}
De modo terreno.
\section{Terrenho}
\begin{itemize}
\item {Grp. gram.:adj.}
\end{itemize}
\begin{itemize}
\item {Grp. gram.:M.}
\end{itemize}
\begin{itemize}
\item {Proveniência:(De \textunderscore terra\textunderscore )}
\end{itemize}
Terrestre; mundano.
Vento que sopra do lado da terra para o mar.
\section{Terreno}
\begin{itemize}
\item {Grp. gram.:adj.}
\end{itemize}
\begin{itemize}
\item {Utilização:Ant.}
\end{itemize}
\begin{itemize}
\item {Grp. gram.:M.}
\end{itemize}
\begin{itemize}
\item {Utilização:Ant.}
\end{itemize}
\begin{itemize}
\item {Proveniência:(Lat. \textunderscore terrenus\textunderscore )}
\end{itemize}
Terrestre.
Mundano.
Que tem a côr da terra; terroso.
\textunderscore Mar terreno\textunderscore , mar mediterrâneo. Cf. Pant. de Aveiro, \textunderscore Itiner.\textunderscore , 218 v.^o, (2.^a ed.).
Espaço de terra.
Porção de terra cultivável.
Cada uma das camadas de terra ou de rocha, consideradas quanto á extensão que occupam e quanto ao modo e época da sua formação: \textunderscore terreno terciário\textunderscore .
Vento, que sopra da terra. Cf. Pant. de Aveiro, \textunderscore Itiner.\textunderscore , 8 e 218 v.^o, (2.^a ed.).
\section{Terrento}
\begin{itemize}
\item {Grp. gram.:adj.}
\end{itemize}
O mesmo que \textunderscore terroso\textunderscore .
\section{Térreo}
\begin{itemize}
\item {Grp. gram.:adj.}
\end{itemize}
\begin{itemize}
\item {Proveniência:(Lat. \textunderscore terreus\textunderscore )}
\end{itemize}
Relativo á terra; próprio da terra.
Que se não eleva acima do nível da terra: \textunderscore pavimento térreo\textunderscore .
Terrestre; terroso.
\section{Terreola}
\begin{itemize}
\item {Grp. gram.:f.}
\end{itemize}
O mesmo ou melhór que \textunderscore terriola\textunderscore .
\section{Terrestre}
\begin{itemize}
\item {Grp. gram.:adj.}
\end{itemize}
\begin{itemize}
\item {Proveniência:(Lat. \textunderscore terrestris\textunderscore )}
\end{itemize}
Relativo á terra; que provém da terra ou nasce na terra.
Mundano.
\section{Terréu}
\begin{itemize}
\item {Grp. gram.:m.}
\end{itemize}
\begin{itemize}
\item {Proveniência:(De \textunderscore terra\textunderscore )}
\end{itemize}
O mesmo que \textunderscore baldio\textunderscore .
\section{Terríbel}
\begin{itemize}
\item {Grp. gram.:adj.}
\end{itemize}
\begin{itemize}
\item {Utilização:Des.}
\end{itemize}
O mesmo que \textunderscore terrível\textunderscore . Cf. \textunderscore Peregrinação\textunderscore , CIII; Usque, 18 v.^o; etc.
\section{Terríbil}
\begin{itemize}
\item {Grp. gram.:adj.}
\end{itemize}
\begin{itemize}
\item {Utilização:Ant.}
\end{itemize}
O mesmo que \textunderscore terrível\textunderscore :«\textunderscore Albuquerque terríbil...\textunderscore »\textunderscore Lusíadas\textunderscore .«\textunderscore E as mãys que que o som terríbil escuitaram...\textunderscore »\textunderscore Ibidem\textunderscore .
\section{Terribilidade}
\begin{itemize}
\item {Grp. gram.:f.}
\end{itemize}
Qualidade do que é terrível.
Coisa terrível; grande ameaça. Cf. \textunderscore Peregrinação\textunderscore . CXXXVI; Usque, 8 e 19.
\section{Terriça}
\begin{itemize}
\item {Grp. gram.:f.}
\end{itemize}
\begin{itemize}
\item {Utilização:Prov.}
\end{itemize}
O mesmo que \textunderscore caliça\textunderscore .
(Cp. \textunderscore terriço\textunderscore )
\section{Terriço}
\begin{itemize}
\item {Grp. gram.:m.}
\end{itemize}
\begin{itemize}
\item {Utilização:Prov.}
\end{itemize}
\begin{itemize}
\item {Utilização:trasm.}
\end{itemize}
\begin{itemize}
\item {Proveniência:(De \textunderscore terra\textunderscore )}
\end{itemize}
Adubo, formado de substâncias animaes e vegetaes em decomposição e misturadas com a terra sôbre que se decompuseram.
Cova ou subterrâneo, onde os coêlhos e outros animaes se abrigam durante as nevadas e durante os grandes calores.
\section{Terrícola}
\begin{itemize}
\item {Grp. gram.:m. ,  f.  e  adj.}
\end{itemize}
\begin{itemize}
\item {Proveniência:(Lat. \textunderscore terricola\textunderscore )}
\end{itemize}
Pessôa ou animal, que habita na terra.
\section{Terrificante}
\begin{itemize}
\item {Grp. gram.:adj.}
\end{itemize}
\begin{itemize}
\item {Proveniência:(Lat. \textunderscore terrificans\textunderscore )}
\end{itemize}
Que terrifica.
\section{Terrificar}
\begin{itemize}
\item {Grp. gram.:v. t.}
\end{itemize}
\begin{itemize}
\item {Proveniência:(Lat. \textunderscore terrificare\textunderscore )}
\end{itemize}
Causar terror a.
Assustar.
Apavorar.
\section{Terrífico}
\begin{itemize}
\item {Grp. gram.:adj.}
\end{itemize}
\begin{itemize}
\item {Proveniência:(Lat. \textunderscore terrificus\textunderscore )}
\end{itemize}
O mesmo que \textunderscore terrificante\textunderscore .
\section{Terrígeno}
\begin{itemize}
\item {Grp. gram.:adj.}
\end{itemize}
\begin{itemize}
\item {Proveniência:(Lat. \textunderscore terigenus\textunderscore )}
\end{itemize}
Produzido na terra.
\section{Terrina}
\begin{itemize}
\item {Grp. gram.:f.}
\end{itemize}
\begin{itemize}
\item {Proveniência:(Fr. \textunderscore terrine\textunderscore )}
\end{itemize}
Vaso de loiça ou de metal, geralmente com tampa, no qual se leva a sopa ou caldo á mesa.
\section{Terrincar}
\begin{itemize}
\item {Grp. gram.:v. t.}
\end{itemize}
\begin{itemize}
\item {Utilização:Pop.}
\end{itemize}
Trincar, (objectos duros), produzindo estalidos.
(Alter. de \textunderscore trincar\textunderscore )
\section{Terriola}
\begin{itemize}
\item {Grp. gram.:f.}
\end{itemize}
\begin{itemize}
\item {Proveniência:(De \textunderscore terra\textunderscore )}
\end{itemize}
Pequena terra.
Lugarejo; aldeóla.
\section{Terrísono}
\begin{itemize}
\item {fónica:so}
\end{itemize}
\begin{itemize}
\item {Grp. gram.:adj.}
\end{itemize}
\begin{itemize}
\item {Proveniência:(Lat. \textunderscore terrisonus\textunderscore )}
\end{itemize}
Que aterra com o som ou com o estrondo; horrivelmente estrondoso.
\section{Terríssono}
\begin{itemize}
\item {Grp. gram.:adj.}
\end{itemize}
\begin{itemize}
\item {Proveniência:(Lat. \textunderscore terrisonus\textunderscore )}
\end{itemize}
Que aterra com o som ou com o estrondo; horrivelmente estrondoso.
\section{Territorial}
\begin{itemize}
\item {Grp. gram.:adj.}
\end{itemize}
\begin{itemize}
\item {Proveniência:(Lat. \textunderscore territorialis\textunderscore )}
\end{itemize}
Relativo a território.
\section{Territorialidade}
\begin{itemize}
\item {Grp. gram.:f.}
\end{itemize}
\begin{itemize}
\item {Utilização:Jur.}
\end{itemize}
\begin{itemize}
\item {Proveniência:(De \textunderscore territorial\textunderscore )}
\end{itemize}
Princípio, que domina as disposições relativas ao território de um cidadão.
\section{Território}
\begin{itemize}
\item {Grp. gram.:m.}
\end{itemize}
\begin{itemize}
\item {Proveniência:(Lat. \textunderscore territorium\textunderscore )}
\end{itemize}
Terreno, mais ou menos extenso.
Área de um país, província, cidade, etc.
Jurisdicção.
Área de uma jurisdicção.
Cada uma das regiões que, nos Estados-Unidos, sendo já bastante populosas, ainda não têm todavia a população necessária para constituir um Estado.
\section{Terrível}
\begin{itemize}
\item {Grp. gram.:adj.}
\end{itemize}
\begin{itemize}
\item {Grp. gram.:M.}
\end{itemize}
\begin{itemize}
\item {Proveniência:(Do lat. \textunderscore terribilis\textunderscore )}
\end{itemize}
Que causa ou infunde terror.
Que produz resultados funestos.
Enorme, extraordinário.
Um dos cargos de loja maçonica.
\section{Terrivelmente}
\begin{itemize}
\item {Grp. gram.:adv.}
\end{itemize}
De modo terrível.
\section{Terrívomo}
\begin{itemize}
\item {Grp. gram.:adj.}
\end{itemize}
\begin{itemize}
\item {Proveniência:(Do lat. \textunderscore terra\textunderscore  + \textunderscore vomere\textunderscore )}
\end{itemize}
Que expelle ou lança terra.
\section{Terroada}
\begin{itemize}
\item {Grp. gram.:f.}
\end{itemize}
Pancada com terrão.
\section{Terrola}
\begin{itemize}
\item {Grp. gram.:f.}
\end{itemize}
O mesmo que \textunderscore terriola\textunderscore .
\section{Terror}
\begin{itemize}
\item {Grp. gram.:m.}
\end{itemize}
\begin{itemize}
\item {Proveniência:(Lat. \textunderscore terror\textunderscore )}
\end{itemize}
Grande susto.
Pavor.
Qualidade do que é terrível.
Época da revolução francesa, durante a qual permaneceu o tribunal revolucionário e a guilhotina.
\section{Terrorífico}
\begin{itemize}
\item {Grp. gram.:adj.}
\end{itemize}
\begin{itemize}
\item {Proveniência:(Do lat. \textunderscore terror\textunderscore  + \textunderscore facere\textunderscore )}
\end{itemize}
O mesmo que \textunderscore terrífico\textunderscore . Cf. Júlio Dinís, \textunderscore Morgadinha\textunderscore , 248.
\section{Terrorismo}
\begin{itemize}
\item {Grp. gram.:m.}
\end{itemize}
\begin{itemize}
\item {Proveniência:(De \textunderscore terror\textunderscore )}
\end{itemize}
Systema de governar por meio de terror ou de revoluções violentas.
\section{Terrorista}
\begin{itemize}
\item {Grp. gram.:m. ,  f.  e  adj.}
\end{itemize}
\begin{itemize}
\item {Utilização:Ext.}
\end{itemize}
\begin{itemize}
\item {Proveniência:(De \textunderscore terror\textunderscore )}
\end{itemize}
Pessôa partidária do terrorismo.
Pessôa que infunde terror.
Pessimista.
\section{Terrorizar}
\begin{itemize}
\item {Grp. gram.:v. t.}
\end{itemize}
O mesmo que \textunderscore aterrorizar\textunderscore .
\section{Terroso}
\begin{itemize}
\item {Grp. gram.:adj.}
\end{itemize}
\begin{itemize}
\item {Proveniência:(Lat. \textunderscore terrosus\textunderscore )}
\end{itemize}
Que tem côr, aspecto, mistura ou natureza de terra; baço.
\section{Terrulento}
\begin{itemize}
\item {Grp. gram.:adj.}
\end{itemize}
\begin{itemize}
\item {Proveniência:(Lat. \textunderscore terrulentus\textunderscore )}
\end{itemize}
O mesmo que \textunderscore terroso\textunderscore .
\section{Tersina}
\begin{itemize}
\item {Grp. gram.:f.}
\end{itemize}
Gênero de aves, da ordem dos pásseres.
\section{Terso}
\begin{itemize}
\item {Grp. gram.:adj.}
\end{itemize}
\begin{itemize}
\item {Utilização:Fig.}
\end{itemize}
\begin{itemize}
\item {Proveniência:(Lat. \textunderscore tersus\textunderscore )}
\end{itemize}
Puro; limpo.
Lustroso.
Correcto, vernáculo: \textunderscore linguagem tersa\textunderscore .
\section{Tersol}
\begin{itemize}
\item {Grp. gram.:m.}
\end{itemize}
\begin{itemize}
\item {Utilização:Ant.}
\end{itemize}
\begin{itemize}
\item {Proveniência:(De \textunderscore terso\textunderscore ?)}
\end{itemize}
Toalha de igreja, com que o sacerdote limpa as mãos.
\section{Tertulha}
\begin{itemize}
\item {Grp. gram.:f.}
\end{itemize}
O mesmo que \textunderscore tertúlia\textunderscore . Cf. Filinto, XVIII, 61 e 179.
\section{Tertúlia}
\begin{itemize}
\item {Grp. gram.:f.}
\end{itemize}
Reunião familiar.
Agrupamento de amigos.
Assembleia literária.
Assembleia.
(Cast. \textunderscore tertulia\textunderscore )
\section{Tertulianismo}
\begin{itemize}
\item {Grp. gram.:m.}
\end{itemize}
Seita herética dos Tertulianistas.
\section{Tertulianistas}
\begin{itemize}
\item {Grp. gram.:m. pl.}
\end{itemize}
Os que seguiam a doutrina de Tertuliano, depois que êste se afastou da Igreja romana, para aderir ao montanismo.
\section{Tertullianismo}
\begin{itemize}
\item {Grp. gram.:m.}
\end{itemize}
Seita herética dos Tertullianistas.
\section{Tertullianistas}
\begin{itemize}
\item {Grp. gram.:m. pl.}
\end{itemize}
Os que seguiam a doutrina de Tertulliano, depois que êste se afastou da Igreja romana, para adherir ao montanismo.
\section{Teruelo}
\begin{itemize}
\item {Grp. gram.:m.}
\end{itemize}
\begin{itemize}
\item {Proveniência:(De \textunderscore Teruel\textunderscore , n. p.)}
\end{itemize}
Antigo tecido de sêda.
\section{Teruelita}
\begin{itemize}
\item {Grp. gram.:f.}
\end{itemize}
\begin{itemize}
\item {Utilização:Miner.}
\end{itemize}
\begin{itemize}
\item {Proveniência:(De \textunderscore Teruel\textunderscore , n. p.)}
\end{itemize}
Silicato alcalino de alumina.
\section{Terúncio}
\begin{itemize}
\item {Grp. gram.:m.}
\end{itemize}
\begin{itemize}
\item {Proveniência:(Lat. \textunderscore teruncius\textunderscore )}
\end{itemize}
Pequena moéda romana de prata, equivalente á quarta parte do asse. Cf. Castilho, \textunderscore Fastos\textunderscore , I, 358 e 359.
\section{Têsa}
\begin{itemize}
\item {Grp. gram.:f.}
\end{itemize}
\begin{itemize}
\item {Utilização:Prov.}
\end{itemize}
\begin{itemize}
\item {Utilização:trasm.}
\end{itemize}
Peça de ferro, em fórma de gancho, e que serve para se meter nos buracos da teiró, afim de a segurar e de impedir que o arado se abra de mais.
\section{Tesadeira}
\begin{itemize}
\item {Grp. gram.:f.}
\end{itemize}
\begin{itemize}
\item {Proveniência:(De \textunderscore tesar\textunderscore )}
\end{itemize}
Maquinismo das fábricas de lanifícios, próprio para esticar o tecido. Cf. \textunderscore Inquér. Industr.\textunderscore , p. II, l. III, 37.
\section{Tesamente}
\begin{itemize}
\item {Grp. gram.:adv.}
\end{itemize}
De modo têso.
Com vigor, com energia: \textunderscore portou-se tesamente\textunderscore .
\section{Tesão}
\begin{itemize}
\item {Grp. gram.:m.}
\end{itemize}
\begin{itemize}
\item {Utilização:Prov.}
\end{itemize}
\begin{itemize}
\item {Utilização:beir.}
\end{itemize}
\begin{itemize}
\item {Utilização:Prov.}
\end{itemize}
\begin{itemize}
\item {Utilização:trasm.}
\end{itemize}
\begin{itemize}
\item {Utilização:Pleb.}
\end{itemize}
\begin{itemize}
\item {Proveniência:(Do lat. \textunderscore tensio\textunderscore , \textunderscore tensionis\textunderscore )}
\end{itemize}
O mesmo que \textunderscore tesura\textunderscore .
Ímpeto, embate violento:«\textunderscore ...amainárão os inimigos no tesão da acomettida.\textunderscore »Filinto, \textunderscore D. Man.\textunderscore , I, 280.
Rêde de pesca, de fórma oblonga, retesada em meio arco de vime e muito usada no rio Alva.
Última travessa, que une as chedas.
Orgasmo do pênis.
\section{Tesar}
\begin{itemize}
\item {Grp. gram.:v. t.}
\end{itemize}
O mesmo que \textunderscore entesar\textunderscore .
\section{Tesidão}
\begin{itemize}
\item {Grp. gram.:f.}
\end{itemize}
O mesmo que \textunderscore tesura\textunderscore .
\section{Têso}
\begin{itemize}
\item {Grp. gram.:adj.}
\end{itemize}
\begin{itemize}
\item {Grp. gram.:M.}
\end{itemize}
\begin{itemize}
\item {Utilização:Bras}
\end{itemize}
\begin{itemize}
\item {Grp. gram.:Adv.}
\end{itemize}
\begin{itemize}
\item {Proveniência:(Lat. \textunderscore tensus\textunderscore )}
\end{itemize}
Tenso; esticado.
Immóvel.
Inteiriçado.
Firme.
Intrépido.
Rijo; forte.
Impetuoso.
Áspero; alcantilado.
Monte alcantilado ou íngreme.
Cimo do monte.
Porção de terreno que, numa superfície inundada, fica acima do nível das águas.
O mesmo que \textunderscore tesamente\textunderscore .
\section{Tesoira}
\begin{itemize}
\item {Grp. gram.:f.}
\end{itemize}
\begin{itemize}
\item {Utilização:Fig.}
\end{itemize}
\begin{itemize}
\item {Utilização:Taur.}
\end{itemize}
\begin{itemize}
\item {Utilização:Pop.}
\end{itemize}
\begin{itemize}
\item {Utilização:Bras}
\end{itemize}
\begin{itemize}
\item {Grp. gram.:F. Pl.}
\end{itemize}
\begin{itemize}
\item {Utilização:Cyn.}
\end{itemize}
\begin{itemize}
\item {Proveniência:(Do lat. \textunderscore tonsoria\textunderscore )}
\end{itemize}
Instrumento, formado de duas lâminas cortantes, reunidas por um eixo, sôbre que se movem.
Peça longitudinal, de madeira ou de ferro, nos jogos deanteiros dos carros de quatro rodas.
Cruzamento das rédeas, com que os cocheiros governam uma parelha de tiro.
Pessôa maldizente.
Sorte de capote, em que o toireiro toma a mesma posição que na sorte da verónica, com a differença de que segura o capote com os braços cruzados.
Grandes unhas, muito agudas.
Aquillo que corta ou dilacera.
Aquillo que tem fórma de tesoira aberta ou de aspa.
Ave diurna, de rapina.
Pennas da ponta da asa, menores que as voadeiras. Cf. Fern. Pereira, \textunderscore Caça de Altan.\textunderscore 
\section{Tesoirada}
\begin{itemize}
\item {Grp. gram.:f.}
\end{itemize}
Golpe com tesoira; acto de tesoirar.
\section{Tesoirar}
\begin{itemize}
\item {Grp. gram.:v. t.}
\end{itemize}
\begin{itemize}
\item {Utilização:Ext.}
\end{itemize}
\begin{itemize}
\item {Utilização:Fam.}
\end{itemize}
Cortar com tesoira.
Cortar.
Falar mal de.
\section{Tesoirinha}
\begin{itemize}
\item {Grp. gram.:f.}
\end{itemize}
Pequena tesoira.
Gavinha.
\section{Tesorelho}
\begin{itemize}
\item {fónica:zorê}
\end{itemize}
\begin{itemize}
\item {Grp. gram.:m.}
\end{itemize}
\begin{itemize}
\item {Utilização:Pop.}
\end{itemize}
Inflammação do tecido cellular, que envolve a parótida.
(Por \textunderscore trasorelho\textunderscore , de \textunderscore tras...\textunderscore  + \textunderscore orelha\textunderscore )
\section{Tesoura}
\begin{itemize}
\item {Grp. gram.:f.}
\end{itemize}
\begin{itemize}
\item {Utilização:Fig.}
\end{itemize}
\begin{itemize}
\item {Utilização:Taur.}
\end{itemize}
\begin{itemize}
\item {Utilização:Pop.}
\end{itemize}
\begin{itemize}
\item {Utilização:Bras}
\end{itemize}
\begin{itemize}
\item {Grp. gram.:F. Pl.}
\end{itemize}
\begin{itemize}
\item {Utilização:Cyn.}
\end{itemize}
\begin{itemize}
\item {Proveniência:(Do lat. \textunderscore tonsoria\textunderscore )}
\end{itemize}
Instrumento, formado de duas lâminas cortantes, reunidas por um eixo, sôbre que se movem.
Peça longitudinal, de madeira ou de ferro, nos jogos deanteiros dos carros de quatro rodas.
Cruzamento das rédeas, com que os cocheiros governam uma parelha de tiro.
Pessôa maldizente.
Sorte de capote, em que o toireiro toma a mesma posição que na sorte da verónica, com a differença de que segura o capote com os braços cruzados.
Grandes unhas, muito agudas.
Aquillo que corta ou dilacera.
Aquillo que tem fórma de tesoura aberta ou de aspa.
Ave diurna, de rapina.
Pennas da ponta da asa, menores que as voadeiras. Cf. Fern. Pereira, \textunderscore Caça de Altan.\textunderscore 
\section{Tesourada}
\begin{itemize}
\item {Grp. gram.:f.}
\end{itemize}
Golpe com tesoura; acto de tesourar.
\section{Tesourar}
\begin{itemize}
\item {Grp. gram.:v. t.}
\end{itemize}
\begin{itemize}
\item {Utilização:Ext.}
\end{itemize}
\begin{itemize}
\item {Utilização:Fam.}
\end{itemize}
Cortar com tesoura.
Cortar.
Falar mal de.
\section{Tesourinha}
\begin{itemize}
\item {Grp. gram.:f.}
\end{itemize}
Pequena tesoura.
Gavinha.
\section{Tessária}
\begin{itemize}
\item {Grp. gram.:f.}
\end{itemize}
\begin{itemize}
\item {Proveniência:(Do gr. \textunderscore tessares\textunderscore )}
\end{itemize}
Gênero de plantas, da fam. das compostas.
\section{Tesse}
\begin{itemize}
\item {Grp. gram.:m.}
\end{itemize}
Arbusto violáceo de Angola, (\textunderscore alsodeia dentata\textunderscore , Beauv.).
\section{Tessela}
\begin{itemize}
\item {Grp. gram.:f.}
\end{itemize}
\begin{itemize}
\item {Proveniência:(Lat. \textunderscore tessella\textunderscore )}
\end{itemize}
Pedra quadrada, para lagear compartimentos de um edifício.
Cubo ou peça de mosaico.
\section{Tesselário}
\begin{itemize}
\item {Grp. gram.:m.}
\end{itemize}
\begin{itemize}
\item {Proveniência:(Lat. \textunderscore tessellarius\textunderscore )}
\end{itemize}
Operário, que prepara pedras ou tijolos para pavimentos.
Mosaísta.
Fabricante de dados.
\section{Tessella}
\begin{itemize}
\item {Grp. gram.:f.}
\end{itemize}
\begin{itemize}
\item {Proveniência:(Lat. \textunderscore tessella\textunderscore )}
\end{itemize}
Pedra quadrada, para lagear compartimentos de um edifício.
Cubo ou peça de mosaico.
\section{Tessellário}
\begin{itemize}
\item {Grp. gram.:m.}
\end{itemize}
\begin{itemize}
\item {Proveniência:(Lat. \textunderscore tessellarius\textunderscore )}
\end{itemize}
Operário, que prepara pedras ou tijolos para pavimentos.
Mosaísta.
Fabricante de dados.
\section{Tessemidus}
\begin{itemize}
\item {Grp. gram.:m. pl.}
\end{itemize}
Antiga tríbo de Índios do Brasil, nas margens do Araguaia.
\section{Tesser}
\textunderscore v. t.\textunderscore  (e der.)
Fórma desusada, mas exacta em vez de \textunderscore tecer\textunderscore , etc.
\section{Téssera}
\begin{itemize}
\item {Grp. gram.:f.}
\end{itemize}
\begin{itemize}
\item {Proveniência:(Lat. \textunderscore tessera\textunderscore )}
\end{itemize}
Nome, que se dava aos objectos, que serviam de senha, entre os primitivos christãos.
Cubo ou dado, com marcas em todas as seis faces.
Tabuleta quadrada, em que os chefes militares traçavam as suas ordens, para que um subalterno, o tesserário, as transmitisse ás tropas.
\section{Tesserário}
\begin{itemize}
\item {Grp. gram.:m.}
\end{itemize}
\begin{itemize}
\item {Proveniência:(Lat. \textunderscore tesserarius\textunderscore )}
\end{itemize}
Aquelle que, entre os antigos, transmittia aos soldados as ordens recebidas dos chefes por meio da téssera.
\section{Tessitura}
\begin{itemize}
\item {Grp. gram.:f.}
\end{itemize}
\begin{itemize}
\item {Utilização:Mús.}
\end{itemize}
\begin{itemize}
\item {Utilização:Fig.}
\end{itemize}
\begin{itemize}
\item {Proveniência:(It. \textunderscore tessitura\textunderscore )}
\end{itemize}
Disposição das notas musicaes, para se accommodarem a certa voz ou a certo instrumento.
Contextura; organização.--É preferível \textunderscore textura\textunderscore .
\section{Testa}
\begin{itemize}
\item {Grp. gram.:f.}
\end{itemize}
\begin{itemize}
\item {Utilização:Fig.}
\end{itemize}
\begin{itemize}
\item {Utilização:Pesc.}
\end{itemize}
\begin{itemize}
\item {Utilização:Pesc.}
\end{itemize}
\begin{itemize}
\item {Proveniência:(Lat. \textunderscore testa\textunderscore )}
\end{itemize}
Parte do rosto, entre os olhos e a raíz dos cabellos anteriores da cabeça.
Fronte.
Frente: \textunderscore Pôs-se á testa da revolta\textunderscore .
Invólucro exterior de uma semente.
Extremidade exterior do copo das armações fixas.
Lado das rêdes, perpendicular ás tralhas, onde amarram os cabos, etc.
\section{Testaça}
\begin{itemize}
\item {Grp. gram.:f.}
\end{itemize}
\begin{itemize}
\item {Utilização:Pop.}
\end{itemize}
Grande testa.
\section{Testação}
\begin{itemize}
\item {Grp. gram.:f.}
\end{itemize}
\begin{itemize}
\item {Utilização:Ant.}
\end{itemize}
\begin{itemize}
\item {Proveniência:(Do lat. \textunderscore testatio\textunderscore )}
\end{itemize}
Coima, multa, comminação da pena.
\section{Testaceado}
\begin{itemize}
\item {Grp. gram.:adj.}
\end{itemize}
O mesmo que \textunderscore testáceo\textunderscore . Cf. P. Caldas, \textunderscore Anim. Vertebr.\textunderscore 
\section{Testáceo}
\begin{itemize}
\item {Grp. gram.:adj.}
\end{itemize}
\begin{itemize}
\item {Grp. gram.:M. Pl.}
\end{itemize}
\begin{itemize}
\item {Proveniência:(Lat. \textunderscore testaceus\textunderscore )}
\end{itemize}
Que tem concha.
Molluscos, cujo corpo é coberto de uma substância sólida ou de muitas peças.
\section{Testaceografia}
\begin{itemize}
\item {Grp. gram.:f.}
\end{itemize}
Parte da Zoologia, em que se descrevem os testáceos.
\section{Testaceográfico}
\begin{itemize}
\item {Grp. gram.:adj.}
\end{itemize}
Relativo á testaceografia.
\section{Testaceographia}
\begin{itemize}
\item {Grp. gram.:f.}
\end{itemize}
Parte da Zoologia, em que se descrevem os testáceos.
\section{Testaceográphico}
\begin{itemize}
\item {Grp. gram.:adj.}
\end{itemize}
Relativo á testaceographia.
\section{Testaceologia}
\begin{itemize}
\item {Grp. gram.:f.}
\end{itemize}
Tratado á cêrca dos testáceos.
\section{Testaceológico}
\begin{itemize}
\item {Grp. gram.:adj.}
\end{itemize}
Relativo á testaceologia.
\section{Testaçudo}
\begin{itemize}
\item {Grp. gram.:adj.}
\end{itemize}
\begin{itemize}
\item {Proveniência:(De \textunderscore testaça\textunderscore )}
\end{itemize}
O mesmo que \textunderscore testudo\textunderscore ^1.
\section{Testada}
\begin{itemize}
\item {Grp. gram.:f.}
\end{itemize}
\begin{itemize}
\item {Utilização:Bras. do N}
\end{itemize}
\begin{itemize}
\item {Utilização:Prov.}
\end{itemize}
\begin{itemize}
\item {Utilização:trasm.}
\end{itemize}
\begin{itemize}
\item {Grp. gram.:Loc.}
\end{itemize}
\begin{itemize}
\item {Utilização:fam.}
\end{itemize}
\begin{itemize}
\item {Proveniência:(De \textunderscore testa\textunderscore )}
\end{itemize}
Parte de uma rua ou estrada, que fica á frente de um prédio.
Êrro; tolice; asneira.
Pancada com a testa; marrada.
\textunderscore Varrer a sua testada\textunderscore , desviar de si certa responsabilidade.
\section{Testa-de-boi}
\begin{itemize}
\item {Grp. gram.:f.}
\end{itemize}
\begin{itemize}
\item {Utilização:Bras}
\end{itemize}
Espécie de vinhático. Cf. \textunderscore Jorn.-do-Comm.\textunderscore , do Rio, de 14-V-901.
\section{Testado}
\begin{itemize}
\item {Grp. gram.:m.}
\end{itemize}
\begin{itemize}
\item {Utilização:T. do Pôrto}
\end{itemize}
O mesmo que \textunderscore atestado\textunderscore .
\section{Testador}
\begin{itemize}
\item {Grp. gram.:m.  e  adj.}
\end{itemize}
\begin{itemize}
\item {Proveniência:(Do lat. \textunderscore testator\textunderscore )}
\end{itemize}
O que testa ou faz testamento.
\section{Testamental}
\begin{itemize}
\item {Grp. gram.:adj.}
\end{itemize}
\begin{itemize}
\item {Proveniência:(Lat. \textunderscore testamentalis\textunderscore )}
\end{itemize}
Relativo a testamento ou que tem a natureza dêlle.
\section{Testamentaria}
\begin{itemize}
\item {Grp. gram.:f.}
\end{itemize}
\begin{itemize}
\item {Proveniência:(De \textunderscore testamento\textunderscore )}
\end{itemize}
Cargo de testamenteiro.
\section{Testamentário}
\begin{itemize}
\item {Grp. gram.:adj.}
\end{itemize}
\begin{itemize}
\item {Grp. gram.:M.}
\end{itemize}
\begin{itemize}
\item {Proveniência:(Lat. \textunderscore testamentarius\textunderscore )}
\end{itemize}
Testamental: \textunderscore cláusulas testamentárias\textunderscore .
Herdeiro por testamento.
\section{Testamenteiro}
\begin{itemize}
\item {Grp. gram.:m.  e  adj.}
\end{itemize}
\begin{itemize}
\item {Proveniência:(Lat. \textunderscore testamentarius\textunderscore )}
\end{itemize}
O que cumpre ou faz cumprir um testamento.
Aquelle a quem um testador incumbe expressamente de cumprir o respectivo testamento.
\section{Testamento}
\begin{itemize}
\item {Grp. gram.:m.}
\end{itemize}
\begin{itemize}
\item {Utilização:Fam.}
\end{itemize}
\begin{itemize}
\item {Utilização:Fam.}
\end{itemize}
\begin{itemize}
\item {Grp. gram.:Pl.}
\end{itemize}
\begin{itemize}
\item {Proveniência:(Lat. \textunderscore testamentum\textunderscore )}
\end{itemize}
Declaração authêntica da última vontade; ou acto, com que alguém dispõe de todos ou de parte dos seus haveres, para depois da sua morte.
Conjunto de despachos, que um Ministro de Estado, ao sair do poder, assina em favor dos seus protegidos.
Carta muito extensa.
Conventos ou solares, cujas rendimentos, no todo ou em parte, competiam aos herdeiros de quem os fundou.
\section{Testante}
\begin{itemize}
\item {Grp. gram.:m.  e  adj.}
\end{itemize}
\begin{itemize}
\item {Proveniência:(De \textunderscore testar\textunderscore )}
\end{itemize}
O mesmo que \textunderscore testador\textunderscore .
\section{Testar}
\begin{itemize}
\item {Grp. gram.:v. t.}
\end{itemize}
\begin{itemize}
\item {Utilização:Ant.}
\end{itemize}
\begin{itemize}
\item {Grp. gram.:V. i.}
\end{itemize}
\begin{itemize}
\item {Proveniência:(Lat. \textunderscore testari\textunderscore )}
\end{itemize}
Deixar ou legar em testamento.
Testificar; attestar.
Fazer testamento.
Dar testemunho.
\section{Testar}
\begin{itemize}
\item {Grp. gram.:v. t.}
\end{itemize}
\begin{itemize}
\item {Utilização:Prov.}
\end{itemize}
\begin{itemize}
\item {Utilização:trasm.}
\end{itemize}
Tornar tésto, entesar.
\section{Teste}
\begin{itemize}
\item {Grp. gram.:f.}
\end{itemize}
\begin{itemize}
\item {Utilização:Obsol.}
\end{itemize}
\begin{itemize}
\item {Proveniência:(Lat. \textunderscore testis\textunderscore )}
\end{itemize}
O mesmo que \textunderscore testemunha\textunderscore .
\section{Testear}
\begin{itemize}
\item {Grp. gram.:v. t.}
\end{itemize}
\begin{itemize}
\item {Utilização:Prov.}
\end{itemize}
\begin{itemize}
\item {Utilização:trasm.}
\end{itemize}
Agarrar (alguém)
\section{Testeira}
\begin{itemize}
\item {Grp. gram.:f.}
\end{itemize}
\begin{itemize}
\item {Utilização:Prov.}
\end{itemize}
\begin{itemize}
\item {Utilização:trasm.}
\end{itemize}
\begin{itemize}
\item {Proveniência:(Do b. lat. \textunderscore testaria\textunderscore )}
\end{itemize}
Parte deanteira, testada.
Frente.
Lenço ou tira de pano, que se põe no testa dos recemnascidos.
Pedaço de pano branco, que assenta na testa das religiosas.
A parte da cabeçada, que circunda a cabeça da cavalgadura.
Cabeceira de caixa ou mesa, a que se prendem os lados.
O mesmo que \textunderscore testico\textunderscore .
Portal ou entrada no cerrado, para carros.
\section{Testemóio}
\begin{itemize}
\item {Grp. gram.:m.}
\end{itemize}
O mesmo que \textunderscore testemunho\textunderscore .
Pública-fórma de documento authêntico.
\section{Testemónio}
\begin{itemize}
\item {Grp. gram.:m.}
\end{itemize}
\begin{itemize}
\item {Utilização:Ant.}
\end{itemize}
O mesmo que \textunderscore testemunho\textunderscore .
Pública-fórma de documento authêntico.
\section{Testemunha}
\begin{itemize}
\item {Grp. gram.:f.}
\end{itemize}
\begin{itemize}
\item {Grp. gram.:Pl.}
\end{itemize}
\begin{itemize}
\item {Utilização:Pop.}
\end{itemize}
\begin{itemize}
\item {Proveniência:(De \textunderscore testemunhar\textunderscore )}
\end{itemize}
Pessôa, que é chamada a assistir a certos actos authênticos ou solennes.
Pessôa, que ouviu ou viu alguma coisa.
Pessôa, que dá testemunho do que viu ou ouviu.
Prova.
Montículo, espécie de marco de pedra, que se deixa em meio de uma escavação, para se conhecer depois a profundidade desta.
Pedras, que se collocam ao lado de um marco.
Árvores, que estão ao lado de uma que serve de baliza.
Os testículos.
\section{Testemunhadeira}
\begin{itemize}
\item {Grp. gram.:f.}
\end{itemize}
Mulhér, que levanta falsos testemunhos, que é calumniadora. Cf. Castilho, \textunderscore Doente de Scisma\textunderscore , 42.
(Cp. \textunderscore testemunhador\textunderscore )
\section{Testemunhador}
\begin{itemize}
\item {Grp. gram.:m.  e  adj.}
\end{itemize}
Que testemunha.
O mesmo que \textunderscore calumniador\textunderscore . Cf. Castilho, \textunderscore Escavações\textunderscore , 51.
\section{Testemunhal}
\begin{itemize}
\item {Grp. gram.:adj.}
\end{itemize}
\begin{itemize}
\item {Proveniência:(Do lat. \textunderscore testimonialis\textunderscore )}
\end{itemize}
Relativo a testemunha: \textunderscore prova testemunhal\textunderscore .
\section{Testemunhalmente}
\begin{itemize}
\item {Grp. gram.:adv.}
\end{itemize}
De modo testemunhal, por meio de testemunhas. Cf. Camillo, \textunderscore Perfil do Marquês\textunderscore , 231.
\section{Testemunhar}
\begin{itemize}
\item {Grp. gram.:v. t.}
\end{itemize}
\begin{itemize}
\item {Grp. gram.:V. i.}
\end{itemize}
Dar testemunho á cêrca de.
Testificar, confirmar.
Manifestar.
Dar testemunho; servir de testemunha.
\section{Testemunhável}
\begin{itemize}
\item {Grp. gram.:adj.}
\end{itemize}
\begin{itemize}
\item {Utilização:Jur.}
\end{itemize}
\begin{itemize}
\item {Proveniência:(De \textunderscore testemunhar\textunderscore )}
\end{itemize}
Que confirma; que merece crédito; que testemunha.
Diz-se da cópia de peças de um processo, feita a pedido de quem aggrava de um despacho, não permittindo o juiz que o aggravo se escreva.
\section{Testemunho}
\begin{itemize}
\item {Grp. gram.:m.}
\end{itemize}
\begin{itemize}
\item {Utilização:Pop.}
\end{itemize}
\begin{itemize}
\item {Proveniência:(Do lat. \textunderscore testimonium\textunderscore )}
\end{itemize}
Aquillo que uma testemunha declara ou allega em juízo.
Depoimento.
Demonstração.
Prova; vestígio.
Calúmnia: \textunderscore levantar testemunhos\textunderscore .
\section{Testico}
\begin{itemize}
\item {Grp. gram.:m.}
\end{itemize}
\begin{itemize}
\item {Proveniência:(De \textunderscore testa\textunderscore )}
\end{itemize}
Cada uma das duas peças da serra, a que se prende o cairo e a fôlha.
\section{Testicondo}
\begin{itemize}
\item {Grp. gram.:adj.}
\end{itemize}
\begin{itemize}
\item {Proveniência:(Do lat. \textunderscore testis\textunderscore  + \textunderscore condere\textunderscore )}
\end{itemize}
Diz-se do cavallo, que tem os testículos recolhidos no ventre.
\section{Testicular}
\begin{itemize}
\item {Grp. gram.:adj.}
\end{itemize}
Relativo aos testículos.
\section{Testículo}
\begin{itemize}
\item {Grp. gram.:m.}
\end{itemize}
\begin{itemize}
\item {Proveniência:(Lat. \textunderscore testiculus\textunderscore )}
\end{itemize}
Cada uma das duas glândulas do escroto.
\section{Testículo-de-cão}
\begin{itemize}
\item {Grp. gram.:f.}
\end{itemize}
Variedade de orchídea, portuguesa. Cf. Est. Veiga, \textunderscore Orchíd. de Portugal\textunderscore .
\section{Testículo-de-perro}
\begin{itemize}
\item {Grp. gram.:m.}
\end{itemize}
Variedade de orchídea, differente da chamada \textunderscore testículo-de-cão\textunderscore .
\section{Testículos-de-cão}
\begin{itemize}
\item {Grp. gram.:f.}
\end{itemize}
O mesmo que \textunderscore testículo-de-cão\textunderscore . Cf. Juromenha, \textunderscore Sintra\textunderscore .
\section{Testiculoso}
\begin{itemize}
\item {Grp. gram.:adj.}
\end{itemize}
\begin{itemize}
\item {Utilização:Bot.}
\end{itemize}
\begin{itemize}
\item {Proveniência:(De \textunderscore testículo\textunderscore )}
\end{itemize}
Testicular.
Diz-se dos órgãos, vegetaes bilobados.
\section{Testificação}
\begin{itemize}
\item {Grp. gram.:f.}
\end{itemize}
\begin{itemize}
\item {Proveniência:(Do lat. \textunderscore testificatio\textunderscore )}
\end{itemize}
Acto ou effeito de testificar.
\section{Testificador}
\begin{itemize}
\item {Grp. gram.:m.  e  adj.}
\end{itemize}
\begin{itemize}
\item {Proveniência:(Do lat. \textunderscore testificator\textunderscore )}
\end{itemize}
O que testifica.
\section{Testificante}
\begin{itemize}
\item {Grp. gram.:m.  e  adj.}
\end{itemize}
\begin{itemize}
\item {Proveniência:(De \textunderscore testificar\textunderscore )}
\end{itemize}
O mesmo que \textunderscore testificador\textunderscore .
\section{Testificar}
\begin{itemize}
\item {Grp. gram.:v. t.}
\end{itemize}
Assegurar; comprovar; declarar.
(B. lat. \textunderscore testificar\textunderscore )
\section{Testigo}
\begin{itemize}
\item {Grp. gram.:m.}
\end{itemize}
\begin{itemize}
\item {Utilização:Des.}
\end{itemize}
\begin{itemize}
\item {Utilização:Prov.}
\end{itemize}
\begin{itemize}
\item {Utilização:trasm.}
\end{itemize}
O mesmo que \textunderscore testemunha\textunderscore .
O mesmo que \textunderscore Demónio\textunderscore  ou \textunderscore Satanás\textunderscore .
(Cast. \textunderscore testigo\textunderscore )
\section{Testilha}
\begin{itemize}
\item {Grp. gram.:f.}
\end{itemize}
\begin{itemize}
\item {Proveniência:(De \textunderscore testa\textunderscore )}
\end{itemize}
Luta, briga; disputa.
\section{Testilhar}
\begin{itemize}
\item {Grp. gram.:v. i.}
\end{itemize}
\begin{itemize}
\item {Utilização:Prov.}
\end{itemize}
\begin{itemize}
\item {Utilização:beir.}
\end{itemize}
\begin{itemize}
\item {Proveniência:(De \textunderscore testilha\textunderscore )}
\end{itemize}
Brigar; contender.
Altercar.
\section{Testilho}
\begin{itemize}
\item {Grp. gram.:m.}
\end{itemize}
\begin{itemize}
\item {Proveniência:(De \textunderscore testa\textunderscore )}
\end{itemize}
Testeira de caixa.
Cada uma das duas faces internas e lateraes da chaminé, da vêrga para cima.
\section{Testimunho}
\textunderscore m.\textunderscore  (e der.)
O mesmo que \textunderscore testemunho\textunderscore , etc.
\section{Testinha}
\begin{itemize}
\item {Grp. gram.:f.}
\end{itemize}
\begin{itemize}
\item {Utilização:Pesc.}
\end{itemize}
\begin{itemize}
\item {Proveniência:(De \textunderscore testa\textunderscore )}
\end{itemize}
Extremidade do corpo da armação fixa, opposta á testa.
\section{Tésto}
\begin{itemize}
\item {Grp. gram.:adj.}
\end{itemize}
\begin{itemize}
\item {Utilização:Fam.}
\end{itemize}
\begin{itemize}
\item {Proveniência:(De \textunderscore testa\textunderscore )}
\end{itemize}
Enérgico; resoluto.
Firme.
Sério.
\section{Têsto}
\begin{itemize}
\item {Grp. gram.:m.}
\end{itemize}
\begin{itemize}
\item {Grp. gram.:Pl.}
\end{itemize}
\begin{itemize}
\item {Utilização:Chul.}
\end{itemize}
\begin{itemize}
\item {Proveniência:(De \textunderscore testa\textunderscore )}
\end{itemize}
Tampa de barro, para vasilha da mesma substância.
Tampa de ferro para tacho ou panela do mesmo metal.
Testico.
Testa do boi.
Cabeça; mioleira.
\section{Testudaço}
\begin{itemize}
\item {Grp. gram.:adj.}
\end{itemize}
\begin{itemize}
\item {Utilização:Ant.}
\end{itemize}
\begin{itemize}
\item {Utilização:Fam.}
\end{itemize}
\begin{itemize}
\item {Proveniência:(De \textunderscore testudo\textunderscore )}
\end{itemize}
Muito teimoso, muito cabeçudo.
\section{Testude}
\begin{itemize}
\item {Grp. gram.:f.}
\end{itemize}
O mesmo ou melhór que \textunderscore testudo\textunderscore ^2.
\section{Testudem}
\begin{itemize}
\item {Grp. gram.:f.}
\end{itemize}
\begin{itemize}
\item {Utilização:Ant.}
\end{itemize}
O mesmo ou melhór que \textunderscore testudo\textunderscore ^2.
\section{Testudo}
\begin{itemize}
\item {Grp. gram.:adj.}
\end{itemize}
\begin{itemize}
\item {Utilização:Fig.}
\end{itemize}
\begin{itemize}
\item {Proveniência:(De \textunderscore testa\textunderscore )}
\end{itemize}
Que tem testa ou cabeça grande.
Obstinado, cabeçudo.
\section{Testudo}
\begin{itemize}
\item {Grp. gram.:m.}
\end{itemize}
\begin{itemize}
\item {Proveniência:(Lat. \textunderscore testudo\textunderscore )}
\end{itemize}
Nome scientífico da tartaruga.
Cobertura, que os soldados romanos faziam, juntando os escudos de uns aos de outros, para se resguardarem dos projécteis dos inimigos.
Tumor cystoso, semelhante á casca da tartaruga.
\section{Testugar}
\begin{itemize}
\item {Grp. gram.:v. t.}
\end{itemize}
\begin{itemize}
\item {Utilização:Prov.}
\end{itemize}
\begin{itemize}
\item {Utilização:alg.}
\end{itemize}
O mesmo que \textunderscore torcegar\textunderscore .
\section{Testugem}
\begin{itemize}
\item {Grp. gram.:f.}
\end{itemize}
Antiga máquina de guerra, espécie de catapulta:«\textunderscore Outras testugens arietárias tinham.\textunderscore »\textunderscore Vir. Trág.\textunderscore , II. 17.
(Por \textunderscore testudem\textunderscore , do lat. \textunderscore testudo\textunderscore , \textunderscore testudinis\textunderscore )
\section{Testulária}
\begin{itemize}
\item {Grp. gram.:f.}
\end{itemize}
\begin{itemize}
\item {Utilização:Zool.}
\end{itemize}
\begin{itemize}
\item {Proveniência:(Do lat. \textunderscore testula\textunderscore )}
\end{itemize}
Gênero de foraminíferos.
\section{Tesum}
\begin{itemize}
\item {Grp. gram.:m.}
\end{itemize}
\begin{itemize}
\item {Proveniência:(De \textunderscore teso\textunderscore ?)}
\end{itemize}
Tecido de oiro ou prata.
Estôfo encoscorado de fios de metal precioso. Cf. Camillo, \textunderscore Cav. em Ruínas\textunderscore , 100.
\section{Tesura}
\begin{itemize}
\item {Grp. gram.:f.}
\end{itemize}
\begin{itemize}
\item {Proveniência:(Do lat. \textunderscore tensura\textunderscore )}
\end{itemize}
Estado do que é têso.
Fôrça.
Austeridade.
Vaidade; orgulho.
\section{Têta}
\begin{itemize}
\item {Grp. gram.:f.}
\end{itemize}
\begin{itemize}
\item {Utilização:Fig.}
\end{itemize}
Glândula mamal; úbere.
Manancial.
(Cp. lat. \textunderscore tata\textunderscore )
\section{Têta-de-cabra}
\begin{itemize}
\item {Grp. gram.:f.}
\end{itemize}
Casta de uva branca da ilha de San-Miguel.
\section{Tetania}
\begin{itemize}
\item {Grp. gram.:f.}
\end{itemize}
\begin{itemize}
\item {Utilização:Med.}
\end{itemize}
\begin{itemize}
\item {Proveniência:(De \textunderscore tétano\textunderscore )}
\end{itemize}
Tétano intermittente.
\section{Tetânico}
\begin{itemize}
\item {Grp. gram.:adj.}
\end{itemize}
\begin{itemize}
\item {Proveniência:(Lat. \textunderscore tetanicus\textunderscore )}
\end{itemize}
Relativo a tétano; que soffre tétano.
\section{Tètaniforme}
\begin{itemize}
\item {Grp. gram.:adj.}
\end{itemize}
\begin{itemize}
\item {Proveniência:(Do lat. \textunderscore tetanus\textunderscore  + \textunderscore forma\textunderscore )}
\end{itemize}
Semelhante ao tétano.
\section{Tètanizar}
\begin{itemize}
\item {Grp. gram.:v. t.}
\end{itemize}
\begin{itemize}
\item {Proveniência:(De \textunderscore tétano\textunderscore )}
\end{itemize}
Tornar tetânico.
\section{Tétano}
\begin{itemize}
\item {Grp. gram.:m.}
\end{itemize}
\begin{itemize}
\item {Proveniência:(Lat. \textunderscore tetanus\textunderscore )}
\end{itemize}
Doença, caracterizada pela rigidez convulsiva dos músculos.
\section{Tetanothro}
\begin{itemize}
\item {Grp. gram.:m.}
\end{itemize}
\begin{itemize}
\item {Proveniência:(Lat. \textunderscore tetanothrum\textunderscore )}
\end{itemize}
Cosmético antigo, que se empregava para fazer desapparecer as rugas.
\section{Tetanotro}
\begin{itemize}
\item {Grp. gram.:m.}
\end{itemize}
\begin{itemize}
\item {Proveniência:(Lat. \textunderscore tetanothrum\textunderscore )}
\end{itemize}
Cosmético antigo, que se empregava para fazer desaparecer as rugas.
\section{Tetar}
\begin{itemize}
\item {Grp. gram.:v. t.}
\end{itemize}
\begin{itemize}
\item {Utilização:Ant.}
\end{itemize}
O mesmo que \textunderscore mamar\textunderscore :«\textunderscore estas erão as tetas cheias de leite que com gosto e sabor tetavam das piedosas madres\textunderscore ». Usque, 9 v.^o
\section{Tétara...}
(V.tetra...)
\section{Tètartemório}
\begin{itemize}
\item {Grp. gram.:m.}
\end{itemize}
\begin{itemize}
\item {Proveniência:(Lat. \textunderscore tetartemorion\textunderscore )}
\end{itemize}
A quarta parte do Zodíaco.
\section{Tetartroedia}
\begin{itemize}
\item {fónica:tro-e}
\end{itemize}
\begin{itemize}
\item {Grp. gram.:f.}
\end{itemize}
Estado ou qualidade de tetartoédro.
\section{Tetartoédrico}
\begin{itemize}
\item {Grp. gram.:adj.}
\end{itemize}
Relativo á tetartoedria.
\section{Tetartoédro}
\begin{itemize}
\item {Grp. gram.:m.}
\end{itemize}
\begin{itemize}
\item {Utilização:Miner.}
\end{itemize}
\begin{itemize}
\item {Proveniência:(Do gr. \textunderscore tetartos\textunderscore  + \textunderscore edra\textunderscore )}
\end{itemize}
Crystal, em fórma de pyrâmide quadrada, cujas faces se inclinam de um modo particular sôbre a base.
\section{Tetartopirâmide}
\begin{itemize}
\item {Grp. gram.:f.}
\end{itemize}
\begin{itemize}
\item {Utilização:Miner.}
\end{itemize}
Cada uma das quatro fórmas elementares das protopirâmides ou pirâmides de primeira ordem; quarto de pirâmide.
\section{Tetartopyrâmide}
\begin{itemize}
\item {Grp. gram.:f.}
\end{itemize}
\begin{itemize}
\item {Utilização:Miner.}
\end{itemize}
Cada uma das quatro fórmas elementares das protopyrâmides ou pyrâmides de primeira ordem; quarto de pyrâmide.
\section{Têtas}
\begin{itemize}
\item {Grp. gram.:m.}
\end{itemize}
\begin{itemize}
\item {Utilização:Chul.}
\end{itemize}
\begin{itemize}
\item {Proveniência:(De \textunderscore têta\textunderscore )}
\end{itemize}
O mesmo que \textunderscore maricas\textunderscore .
\section{Té-té}
\begin{itemize}
\item {Grp. gram.:m.}
\end{itemize}
\begin{itemize}
\item {Utilização:Pop.}
\end{itemize}
Espécie de jôgo, o mesmo que \textunderscore escondidas\textunderscore .
Acto de espreitar, brincando.
\section{Tetecuêra}
\begin{itemize}
\item {Grp. gram.:f.}
\end{itemize}
\begin{itemize}
\item {Utilização:Bras}
\end{itemize}
Depressão de terreno, que serviu de leito ao rio Paraíba do Sul e que se acha hoje coberto de vegetação.
\section{Teteia}
\begin{itemize}
\item {Grp. gram.:f.}
\end{itemize}
\begin{itemize}
\item {Utilização:Infant.}
\end{itemize}
Dixe, com que brincam crianças; brinquedo.
\section{Teteira}
\begin{itemize}
\item {Grp. gram.:f.}
\end{itemize}
\begin{itemize}
\item {Utilização:Prov.}
\end{itemize}
Doença, que ataca as têtas das cabras, tornando o leite dellas nocivo á saúde.
\section{Tetérrimo}
\begin{itemize}
\item {Grp. gram.:adj.}
\end{itemize}
Muito feio; hediondo.
(Lat, \textunderscore teterrimus\textunderscore )
\section{Tetilha}
\begin{itemize}
\item {Utilização:Prov.}
\end{itemize}
\begin{itemize}
\item {Utilização:minh.}
\end{itemize}
\textunderscore f.\textunderscore  (e der.)
O mesmo que \textunderscore testilha\textunderscore , etc.
\section{Tetim}
\begin{itemize}
\item {Grp. gram.:m.}
\end{itemize}
Massa pegajosa, feita de pó de tijolo, cal e azeite.
\section{Teti-poteira}
\begin{itemize}
\item {Grp. gram.:f.}
\end{itemize}
Planta ampelídea do Brasil.
\section{Teto}
\begin{itemize}
\item {Grp. gram.:m.}
\end{itemize}
Uma das línguas faladas em Timor.
\section{Tetra...}
\begin{itemize}
\item {Grp. gram.:pref.}
\end{itemize}
\begin{itemize}
\item {Proveniência:(Do gr. \textunderscore tetra\textunderscore )}
\end{itemize}
(designativo de \textunderscore quatro\textunderscore )
\section{Tètrabranchiados}
\begin{itemize}
\item {fónica:qui}
\end{itemize}
\begin{itemize}
\item {Grp. gram.:m. pl.}
\end{itemize}
\begin{itemize}
\item {Utilização:Zool.}
\end{itemize}
\begin{itemize}
\item {Proveniência:(De \textunderscore tetra...\textunderscore  + \textunderscore brânchias\textunderscore )}
\end{itemize}
Ordem zoológica da série paleozóica, da qual apenas resta o náutilo.
\section{Tètrabranchiaes}
\begin{itemize}
\item {fónica:qui}
\end{itemize}
\begin{itemize}
\item {Grp. gram.:m. pl.}
\end{itemize}
O mesmo que \textunderscore tètrabranchiados\textunderscore .
\section{Tètrabranquiados}
\begin{itemize}
\item {Grp. gram.:m. pl.}
\end{itemize}
\begin{itemize}
\item {Utilização:Zool.}
\end{itemize}
\begin{itemize}
\item {Proveniência:(De \textunderscore tetra...\textunderscore  + \textunderscore brânquias\textunderscore )}
\end{itemize}
Ordem zoológica da série paleozóica, da qual apenas resta o náutilo.
\section{Tetrabranquiais}
\begin{itemize}
\item {Grp. gram.:m. pl.}
\end{itemize}
O mesmo que \textunderscore tètrabranquiados\textunderscore .
\section{Tètracâmaro}
\begin{itemize}
\item {Grp. gram.:adj.}
\end{itemize}
\begin{itemize}
\item {Utilização:Bot.}
\end{itemize}
\begin{itemize}
\item {Proveniência:(De \textunderscore tetra...\textunderscore  + \textunderscore câmara\textunderscore )}
\end{itemize}
Que tem quatro câmaras.
\section{Tètracarpo}
\begin{itemize}
\item {Grp. gram.:adj.}
\end{itemize}
\begin{itemize}
\item {Utilização:Bot.}
\end{itemize}
\begin{itemize}
\item {Proveniência:(Do gr. \textunderscore tetra\textunderscore  + \textunderscore karpos\textunderscore )}
\end{itemize}
Que tem quatro frutos.
\section{Tètracentígrado}
\begin{itemize}
\item {Grp. gram.:adj.}
\end{itemize}
\begin{itemize}
\item {Proveniência:(De \textunderscore tetra...\textunderscore  + \textunderscore centígrado\textunderscore )}
\end{itemize}
Diz-se do thermómetro, cujos extremos distam entre si 400°.
\section{Tètrácero}
\begin{itemize}
\item {Grp. gram.:adj.}
\end{itemize}
\begin{itemize}
\item {Utilização:Zool.}
\end{itemize}
\begin{itemize}
\item {Proveniência:(Do gr. \textunderscore tetra\textunderscore  + \textunderscore keras\textunderscore )}
\end{itemize}
Diz-se dos molluscos, que têm quatro antennas ou tentáculos.
\section{Tètrachna}
\begin{itemize}
\item {Grp. gram.:f.}
\end{itemize}
Gênero de plantas gramíneas.
\section{Tètracna}
\begin{itemize}
\item {Grp. gram.:f.}
\end{itemize}
Gênero de plantas gramíneas.
\section{Tètracólon}
\begin{itemize}
\item {Grp. gram.:m.}
\end{itemize}
\begin{itemize}
\item {Utilização:Gram.}
\end{itemize}
\begin{itemize}
\item {Proveniência:(Gr. \textunderscore tetrakolon\textunderscore )}
\end{itemize}
Período de quatro membros.
\section{Tètracorde}
\begin{itemize}
\item {Grp. gram.:m.}
\end{itemize}
\begin{itemize}
\item {Proveniência:(Do gr. \textunderscore tetrakhordon\textunderscore )}
\end{itemize}
Série de quatro sons consecutivos.
\section{Tètracórdio}
\begin{itemize}
\item {Grp. gram.:m.}
\end{itemize}
Antiga lyra de quatro cordas.
(Cp. \textunderscore tètracorde\textunderscore )
\section{Tètracórdo}
\begin{itemize}
\item {Grp. gram.:adj.}
\end{itemize}
\begin{itemize}
\item {Grp. gram.:M.}
\end{itemize}
Que tem quatro cordas.
O mesmo que \textunderscore tètracórdio\textunderscore .
\section{Tetradáctilo}
\begin{itemize}
\item {Grp. gram.:adj.}
\end{itemize}
\begin{itemize}
\item {Proveniência:(Do gr. \textunderscore tetra\textunderscore  + \textunderscore daktulos\textunderscore )}
\end{itemize}
Que tem quatro dedos.
\section{Tètradáctylo}
\begin{itemize}
\item {Grp. gram.:adj.}
\end{itemize}
\begin{itemize}
\item {Proveniência:(Do gr. \textunderscore tetra\textunderscore  + \textunderscore daktulos\textunderscore )}
\end{itemize}
Que tem quatro dedos.
\section{Tètrádia}
\begin{itemize}
\item {Grp. gram.:f.}
\end{itemize}
\begin{itemize}
\item {Proveniência:(Do gr. \textunderscore tettradion\textunderscore )}
\end{itemize}
Gênero de plantas esterculiáceas.
\section{Tètradiapasão}
\begin{itemize}
\item {Grp. gram.:m.}
\end{itemize}
\begin{itemize}
\item {Utilização:Mús.}
\end{itemize}
\begin{itemize}
\item {Utilização:Des.}
\end{itemize}
\begin{itemize}
\item {Proveniência:(De \textunderscore tetra...\textunderscore  + \textunderscore diapasão\textunderscore )}
\end{itemize}
Oitava quádrupla.
\section{Tetrádimos}
\begin{itemize}
\item {Grp. gram.:m. pl.}
\end{itemize}
\begin{itemize}
\item {Utilização:Miner.}
\end{itemize}
\begin{itemize}
\item {Proveniência:(Do gr. \textunderscore tetra\textunderscore  + \textunderscore dumos\textunderscore )}
\end{itemize}
Uma das subdivisões dos grupamentos regulares dos cristaes.
\section{Tètradinamia}
\begin{itemize}
\item {Grp. gram.:f.}
\end{itemize}
\begin{itemize}
\item {Utilização:Bot.}
\end{itemize}
\begin{itemize}
\item {Proveniência:(Do gr. \textunderscore tetra\textunderscore  + \textunderscore dunamis\textunderscore )}
\end{itemize}
Disposição de seis estames na flôr, sendo quatro mais fortes que os outros dois.
\section{Tètradínamo}
\begin{itemize}
\item {Grp. gram.:adj.}
\end{itemize}
\begin{itemize}
\item {Utilização:Bot.}
\end{itemize}
Diz-se dos estames, em que se dá a tetradinamia.
\section{Tètradoro}
\begin{itemize}
\item {Grp. gram.:adj.}
\end{itemize}
\begin{itemize}
\item {Proveniência:(Lat. \textunderscore tetradoros\textunderscore )}
\end{itemize}
Dizia-se, na Architectura antiga, daquillo que tem de largura quatro mãos travessas.
\section{Tètradrachmo}
\begin{itemize}
\item {Grp. gram.:m.}
\end{itemize}
\begin{itemize}
\item {Proveniência:(Gr. \textunderscore tetradrakhmon\textunderscore )}
\end{itemize}
Antiga moéda grega, equivalente a quatro drachmas.
\section{Tetradracmo}
\begin{itemize}
\item {Grp. gram.:m.}
\end{itemize}
\begin{itemize}
\item {Proveniência:(Gr. \textunderscore tetradrakhmon\textunderscore )}
\end{itemize}
Antiga moéda grega, equivalente a quatro drachmas.
\section{Tètrádymos}
\begin{itemize}
\item {Grp. gram.:m. pl.}
\end{itemize}
\begin{itemize}
\item {Utilização:Miner.}
\end{itemize}
\begin{itemize}
\item {Proveniência:(Do gr. \textunderscore tetra\textunderscore  + \textunderscore dumos\textunderscore )}
\end{itemize}
Uma das subdivisões dos grupamentos regulares dos crystaes.
\section{Tètradynamia}
\begin{itemize}
\item {Grp. gram.:f.}
\end{itemize}
\begin{itemize}
\item {Utilização:Bot.}
\end{itemize}
\begin{itemize}
\item {Proveniência:(Do gr. \textunderscore tetra\textunderscore  + \textunderscore dunamis\textunderscore )}
\end{itemize}
Disposição de seis estames na flôr, sendo quatro mais fortes que os outros dois.
\section{Tètradýnamo}
\begin{itemize}
\item {Grp. gram.:adj.}
\end{itemize}
\begin{itemize}
\item {Utilização:Bot.}
\end{itemize}
Diz-se dos estames, em que se dá a tetradynamia.
\section{Tètraédrico}
\begin{itemize}
\item {Grp. gram.:adj.}
\end{itemize}
Relativo ao tetraédro.
Que tem fórma de tetraédro.
\section{Tètraédro}
\begin{itemize}
\item {Grp. gram.:m.}
\end{itemize}
\begin{itemize}
\item {Utilização:Geom.}
\end{itemize}
\begin{itemize}
\item {Proveniência:(Do gr. \textunderscore tetra\textunderscore  + \textunderscore edra\textunderscore )}
\end{itemize}
Sólido, terminado por quatro faces planas.
\section{Tètraedrito}
\begin{itemize}
\item {Grp. gram.:m.}
\end{itemize}
\begin{itemize}
\item {Utilização:Miner.}
\end{itemize}
\begin{itemize}
\item {Proveniência:(De \textunderscore tetraédro\textunderscore )}
\end{itemize}
Variedade de metal, que é um minério de cobre cinzento.
\section{Tètraexaédro}
\begin{itemize}
\item {fónica:eiza}
\end{itemize}
\begin{itemize}
\item {Grp. gram.:m.}
\end{itemize}
\begin{itemize}
\item {Proveniência:(De \textunderscore tetra...\textunderscore  + \textunderscore hexaédro\textunderscore )}
\end{itemize}
Poliedro, limitado por 24 triângulos isósceles iguaes, formando, 4 a 4, ângulos tetraédros regulares, em posições correspondentes ás 6 faces do hexaédro.
\section{Tètrafalangarquia}
\begin{itemize}
\item {Grp. gram.:f.}
\end{itemize}
A grande phalange macedónica, entre os antigos Gregos, ou corpo de exército, que constava de quatro falanges.
\section{Tètrafármaco}
\begin{itemize}
\item {Grp. gram.:m.}
\end{itemize}
\begin{itemize}
\item {Utilização:Pharm.}
\end{itemize}
\begin{itemize}
\item {Utilização:ant.}
\end{itemize}
\begin{itemize}
\item {Proveniência:(Lat. \textunderscore tetrapharmacum\textunderscore )}
\end{itemize}
Medicamento, composto de quatro ingredientes.
\section{Tètráfide}
\begin{itemize}
\item {Grp. gram.:f.}
\end{itemize}
Gênero de musgos.
\section{Tètráfido}
\begin{itemize}
\item {Grp. gram.:adj.}
\end{itemize}
\begin{itemize}
\item {Utilização:Bot.}
\end{itemize}
\begin{itemize}
\item {Proveniência:(Do gr. \textunderscore tetra\textunderscore  + lat. \textunderscore findere\textunderscore )}
\end{itemize}
Diz-se dos órgãos vegetaes divididos em quatro lóbulos.
\section{Tetrafilo}
\begin{itemize}
\item {Grp. gram.:adj.}
\end{itemize}
\begin{itemize}
\item {Proveniência:(Do gr. \textunderscore tetra\textunderscore  + \textunderscore phullon\textunderscore )}
\end{itemize}
Que tem quatro fôlhas.
\section{Tètrafoliado}
\begin{itemize}
\item {Grp. gram.:adj.}
\end{itemize}
\begin{itemize}
\item {Utilização:Bot.}
\end{itemize}
\begin{itemize}
\item {Proveniência:(De \textunderscore tetra...\textunderscore  + lat. \textunderscore folium\textunderscore )}
\end{itemize}
Que tem as fôlhas dispostas a quatro e quatro.
\section{Tètrafonia}
\begin{itemize}
\item {Grp. gram.:f.}
\end{itemize}
\begin{itemize}
\item {Utilização:Mús.}
\end{itemize}
\begin{itemize}
\item {Proveniência:(Do gr. \textunderscore tetra\textunderscore  + \textunderscore phone\textunderscore )}
\end{itemize}
Efeito de quatro sons simultâneos.
Harmonia a quatro partes.
\section{Tètráforos}
\begin{itemize}
\item {Grp. gram.:m. pl.}
\end{itemize}
\begin{itemize}
\item {Utilização:Des.}
\end{itemize}
\begin{itemize}
\item {Proveniência:(Lat. \textunderscore tetraphori\textunderscore )}
\end{itemize}
Dizia-se dos quatro mariolas que transportam a mesma carga.
\section{Tètraginia}
\begin{itemize}
\item {Grp. gram.:f.}
\end{itemize}
\begin{itemize}
\item {Proveniência:(Do gr. \textunderscore tetra\textunderscore  + \textunderscore gune\textunderscore )}
\end{itemize}
Classe de plantas, que compreende as que têm quatro pistilos.
\section{Tètrágino}
\begin{itemize}
\item {Grp. gram.:adj.}
\end{itemize}
\begin{itemize}
\item {Utilização:Bot.}
\end{itemize}
\begin{itemize}
\item {Proveniência:(Do gr. \textunderscore tetra\textunderscore  + \textunderscore gune\textunderscore )}
\end{itemize}
Em que há tetragínia.
Que tem quatro pistilos.
\section{Tètragnáthio}
\begin{itemize}
\item {Grp. gram.:m.}
\end{itemize}
\begin{itemize}
\item {Proveniência:(Lat. \textunderscore tetrognathion\textunderscore )}
\end{itemize}
Espécie de aranha venenosa, talvez só conhecida dos antigos.
\section{Tetragnátio}
\begin{itemize}
\item {Grp. gram.:m.}
\end{itemize}
\begin{itemize}
\item {Proveniência:(Lat. \textunderscore tetrognathion\textunderscore )}
\end{itemize}
Espécie de aranha venenosa, talvez só conhecida dos antigos.
\section{Tètragonal}
\begin{itemize}
\item {Grp. gram.:adj.}
\end{itemize}
\begin{itemize}
\item {Proveniência:(Lat. \textunderscore tetragonalis\textunderscore )}
\end{itemize}
Que tem fórma de tetrágono.
\section{Tètragónia}
\begin{itemize}
\item {Grp. gram.:f.}
\end{itemize}
Gênero de plantas portuláceas.
(Cp. \textunderscore tètrágono\textunderscore )
\section{Tètragónico}
\begin{itemize}
\item {Grp. gram.:adj.}
\end{itemize}
\begin{itemize}
\item {Proveniência:(Lat. \textunderscore tetragonicus\textunderscore )}
\end{itemize}
Que tem quatro lados.
\section{Tètragónio}
\begin{itemize}
\item {Grp. gram.:m.}
\end{itemize}
\begin{itemize}
\item {Proveniência:(Lat. \textunderscore tetragonium\textunderscore )}
\end{itemize}
Manto quadrado ou de quatro pontas, usado pelos antigos.
\section{Tètragonismo}
\begin{itemize}
\item {Grp. gram.:m.}
\end{itemize}
\begin{itemize}
\item {Proveniência:(Lat. \textunderscore tetragonismus\textunderscore )}
\end{itemize}
Qualidade de tetrágono.
\section{Tètrágono}
\begin{itemize}
\item {Grp. gram.:adj.}
\end{itemize}
\begin{itemize}
\item {Grp. gram.:M.}
\end{itemize}
\begin{itemize}
\item {Proveniência:(Lat. \textunderscore tetragonum\textunderscore )}
\end{itemize}
Que tem quatro ângulos e quatro lados.
O mesmo que \textunderscore quadrilátero\textunderscore .
\section{Tètragonocarpo}
\begin{itemize}
\item {Grp. gram.:adj.}
\end{itemize}
\begin{itemize}
\item {Utilização:Bot.}
\end{itemize}
\begin{itemize}
\item {Proveniência:(Do gr. \textunderscore tetragonos\textunderscore  + \textunderscore karpos\textunderscore )}
\end{itemize}
Que tem frutos tetrágonos.
\section{Tètragonocéfalo}
\begin{itemize}
\item {Grp. gram.:adj.}
\end{itemize}
\begin{itemize}
\item {Proveniência:(Do gr. \textunderscore tetragonos\textunderscore  + \textunderscore kephale\textunderscore )}
\end{itemize}
Que tem cabeça quadrangular.
\section{Tètragonocéphalo}
\begin{itemize}
\item {Grp. gram.:adj.}
\end{itemize}
\begin{itemize}
\item {Proveniência:(Do gr. \textunderscore tetragonos\textunderscore  + \textunderscore kephale\textunderscore )}
\end{itemize}
Que tem cabeça quadrangular.
\section{Tètragonolóbio}
\begin{itemize}
\item {Grp. gram.:adj.}
\end{itemize}
\begin{itemize}
\item {Utilização:Bot.}
\end{itemize}
\begin{itemize}
\item {Proveniência:(Do gr. \textunderscore tetragonos\textunderscore  + \textunderscore lobos\textunderscore )}
\end{itemize}
Que tem frutos quadrangulares.
\section{Tètragonóptero}
\begin{itemize}
\item {Grp. gram.:adj.}
\end{itemize}
\begin{itemize}
\item {Utilização:Zool.}
\end{itemize}
\begin{itemize}
\item {Proveniência:(Do gr. \textunderscore tetragonos\textunderscore  + \textunderscore pteron\textunderscore )}
\end{itemize}
Diz-se dos peixes que têm as barbatanas quadradas.
\section{Tètragonoteco}
\begin{itemize}
\item {Grp. gram.:adj.}
\end{itemize}
\begin{itemize}
\item {Utilização:Bot.}
\end{itemize}
\begin{itemize}
\item {Proveniência:(Do gr. \textunderscore tetragonos\textunderscore  + \textunderscore theke\textunderscore )}
\end{itemize}
Que tem invólucro quadrado.
\section{Tètragonotheco}
\begin{itemize}
\item {Grp. gram.:adj.}
\end{itemize}
\begin{itemize}
\item {Utilização:Bot.}
\end{itemize}
\begin{itemize}
\item {Proveniência:(Do gr. \textunderscore tetragonos\textunderscore  + \textunderscore theke\textunderscore )}
\end{itemize}
Que tem invólucro quadrado.
\section{Tètragrama}
\begin{itemize}
\item {Grp. gram.:adj.}
\end{itemize}
\begin{itemize}
\item {Grp. gram.:M.}
\end{itemize}
\begin{itemize}
\item {Proveniência:(Do gr. \textunderscore tetra\textunderscore  + \textunderscore gramma\textunderscore )}
\end{itemize}
Que tem quatro letras.
Conjunto de quatro letras, formando palavra, firma ou sinal.
Pauta de quatro linhas, usada no cantochão.
\section{Tètragramma}
\begin{itemize}
\item {Grp. gram.:adj.}
\end{itemize}
\begin{itemize}
\item {Grp. gram.:M.}
\end{itemize}
\begin{itemize}
\item {Proveniência:(Do gr. \textunderscore tetra\textunderscore  + \textunderscore gramma\textunderscore )}
\end{itemize}
Que tem quatro letras.
Conjunto de quatro letras, formando palavra, firma ou sinal.
Pauta de quatro linhas, usada no cantochão.
\section{Tètragynia}
\begin{itemize}
\item {Grp. gram.:f.}
\end{itemize}
\begin{itemize}
\item {Proveniência:(Do gr. \textunderscore tetra\textunderscore  + \textunderscore gune\textunderscore )}
\end{itemize}
Classe de plantas, que comprehende as que têm quatro pistillos.
\section{Tètrágyno}
\begin{itemize}
\item {Grp. gram.:adj.}
\end{itemize}
\begin{itemize}
\item {Utilização:Bot.}
\end{itemize}
\begin{itemize}
\item {Proveniência:(Do gr. \textunderscore tetra\textunderscore  + \textunderscore gune\textunderscore )}
\end{itemize}
Em que há tetragýnia.
Que tem quatro pistillos.
\section{Tètrahexaédro}
\begin{itemize}
\item {Grp. gram.:m.}
\end{itemize}
\begin{itemize}
\item {Proveniência:(De \textunderscore tetra...\textunderscore  + \textunderscore hexaédro\textunderscore )}
\end{itemize}
Polyedro, limitado por 24 triângulos isósceles iguaes, formando, 4 a 4, ângulos tetraédros regulares, em posições correspondentes ás 6 faces do hexaédro.
\section{Tètrahýdrico}
\begin{itemize}
\item {Grp. gram.:adj.}
\end{itemize}
\begin{itemize}
\item {Utilização:Chím.}
\end{itemize}
Que contém hydrogênio em proporção quádrupla.
\section{Tètraídrico}
\begin{itemize}
\item {Grp. gram.:adj.}
\end{itemize}
\begin{itemize}
\item {Utilização:Chím.}
\end{itemize}
Que contém hidrogênio em proporção quádrupla.
\section{Tètralépido}
\begin{itemize}
\item {Grp. gram.:adj.}
\end{itemize}
\begin{itemize}
\item {Utilização:Bot.}
\end{itemize}
\begin{itemize}
\item {Proveniência:(Do gr. \textunderscore tetra\textunderscore  + \textunderscore lepis\textunderscore )}
\end{itemize}
Que tem quatro estames.
\section{Tètralogia}
\begin{itemize}
\item {Grp. gram.:f.}
\end{itemize}
\begin{itemize}
\item {Proveniência:(Do gr. \textunderscore tetra\textunderscore  + \textunderscore logos\textunderscore )}
\end{itemize}
Conjunto das quatro peças theatraes, que os poetas gregos apresentavam em concurso.
\section{Tètrâmero}
\begin{itemize}
\item {Grp. gram.:adj.}
\end{itemize}
\begin{itemize}
\item {Grp. gram.:M. pl.}
\end{itemize}
\begin{itemize}
\item {Proveniência:(Do gr. \textunderscore tetra\textunderscore  + \textunderscore meros\textunderscore )}
\end{itemize}
Dividido em quatro partes.
Secção da ordem dos insectos coleópteros, á qual pertencem os insectos que têm quatro artículos nos tarsos.
\section{Tètrâmetro}
\begin{itemize}
\item {Grp. gram.:m.}
\end{itemize}
\begin{itemize}
\item {Proveniência:(Do gr. \textunderscore tetra\textunderscore  + \textunderscore metron\textunderscore )}
\end{itemize}
Verso grego ou latino de quatro pés.
\section{Tètraminas}
\begin{itemize}
\item {Grp. gram.:f. pl.}
\end{itemize}
\begin{itemize}
\item {Utilização:Chím.}
\end{itemize}
Aminas, formadas por quatro moléculas de ammoníaco.
\section{Tètrandria}
\begin{itemize}
\item {Grp. gram.:f.}
\end{itemize}
\begin{itemize}
\item {Utilização:Bot.}
\end{itemize}
Qualidade de tetrandro; conjunto dos vegetaes tetrandros.
\section{Tètrandro}
\begin{itemize}
\item {Grp. gram.:adj.}
\end{itemize}
\begin{itemize}
\item {Utilização:Bot.}
\end{itemize}
\begin{itemize}
\item {Proveniência:(Do gr. \textunderscore tetra\textunderscore  + \textunderscore aner\textunderscore , \textunderscore andros\textunderscore )}
\end{itemize}
Que tem quatro estames, livres entre si.
\section{Tètrânemo}
\begin{itemize}
\item {Grp. gram.:adj.}
\end{itemize}
Que tem quatro filamentos ou tentáculos.
\section{Tètraneta}
\begin{itemize}
\item {Grp. gram.:f.}
\end{itemize}
Fem. de \textunderscore tètraneto\textunderscore .
\section{Tètraneto}
\begin{itemize}
\item {Grp. gram.:m.}
\end{itemize}
\begin{itemize}
\item {Proveniência:(De \textunderscore tetra...\textunderscore  + \textunderscore neto\textunderscore )}
\end{itemize}
Filho do trineto ou da trineta.
\section{Tètrantho}
\begin{itemize}
\item {Grp. gram.:m.}
\end{itemize}
\begin{itemize}
\item {Proveniência:(Do gr. \textunderscore tetra\textunderscore  + \textunderscore anthos\textunderscore )}
\end{itemize}
Gênero de plantas, da fam. das compostas.
\section{Tètranto}
\begin{itemize}
\item {Grp. gram.:m.}
\end{itemize}
\begin{itemize}
\item {Proveniência:(Do gr. \textunderscore tetra\textunderscore  + \textunderscore anthos\textunderscore )}
\end{itemize}
Gênero de plantas, da fam. das compostas.
\section{Tètraónice}
\begin{itemize}
\item {Grp. gram.:m.}
\end{itemize}
Gênero de reptís.
\section{Tètraónyce}
\begin{itemize}
\item {Grp. gram.:m.}
\end{itemize}
Gênero de reptís.
\section{Tètrapétalo}
\begin{itemize}
\item {Grp. gram.:adj.}
\end{itemize}
\begin{itemize}
\item {Utilização:Bot.}
\end{itemize}
\begin{itemize}
\item {Proveniência:(De \textunderscore tetra...\textunderscore  + \textunderscore pétala\textunderscore )}
\end{itemize}
Que tem quatro pétalas.
\section{Tètraphalanarchia}
\begin{itemize}
\item {fónica:qui}
\end{itemize}
\begin{itemize}
\item {Grp. gram.:f.}
\end{itemize}
A grande phalange macedónica, entre os antigos Gregos, ou corpo de exército, que constava de quatro phalanges.
\section{Tètraphármaco}
\begin{itemize}
\item {Grp. gram.:m.}
\end{itemize}
\begin{itemize}
\item {Utilização:Pharm.}
\end{itemize}
\begin{itemize}
\item {Utilização:ant.}
\end{itemize}
\begin{itemize}
\item {Proveniência:(Lat. \textunderscore tetrapharmacum\textunderscore )}
\end{itemize}
Medicamento, composto de quatro ingredientes.
\section{Tètraphonia}
\begin{itemize}
\item {Grp. gram.:f.}
\end{itemize}
\begin{itemize}
\item {Utilização:Mús.}
\end{itemize}
\begin{itemize}
\item {Proveniência:(Do gr. \textunderscore tetra\textunderscore  + \textunderscore phone\textunderscore )}
\end{itemize}
Effeito de quatro sons simultâneos.
Harmonia a quatro partes.
\section{Tètráphoros}
\begin{itemize}
\item {Grp. gram.:m. pl.}
\end{itemize}
\begin{itemize}
\item {Utilização:Des.}
\end{itemize}
\begin{itemize}
\item {Proveniência:(Lat. \textunderscore tetraphori\textunderscore )}
\end{itemize}
Dizia-se dos quatro mariolas que transportam a mesma carga.
\section{Tètráphyde}
\begin{itemize}
\item {Grp. gram.:f.}
\end{itemize}
Gênero de musgos.
\section{Tètraphyllo}
\begin{itemize}
\item {Grp. gram.:adj.}
\end{itemize}
\begin{itemize}
\item {Proveniência:(Do gr. \textunderscore tetra\textunderscore  + \textunderscore phullon\textunderscore )}
\end{itemize}
Que tem quatro fôlhas.
\section{Tetrápilo}
\begin{itemize}
\item {Grp. gram.:adj.}
\end{itemize}
\begin{itemize}
\item {Grp. gram.:M.}
\end{itemize}
\begin{itemize}
\item {Utilização:Bot.}
\end{itemize}
\begin{itemize}
\item {Proveniência:(Lat. \textunderscore tetrapylum\textunderscore )}
\end{itemize}
Que tem quatro portas.
Gênero de plantas oleáceas.
\section{Tètraplegia}
\begin{itemize}
\item {Grp. gram.:f.}
\end{itemize}
\begin{itemize}
\item {Utilização:Med.}
\end{itemize}
\begin{itemize}
\item {Proveniência:(Do gr. \textunderscore tetra\textunderscore  + \textunderscore plassein\textunderscore )}
\end{itemize}
Paralysia de todos os quatro membros.
\section{Tètraplo}
\begin{itemize}
\item {Grp. gram.:m.}
\end{itemize}
\begin{itemize}
\item {Utilização:Des.}
\end{itemize}
O mesmo que \textunderscore quádruplo\textunderscore .
\section{Tètraplodonte}
\begin{itemize}
\item {Grp. gram.:m.}
\end{itemize}
\begin{itemize}
\item {Proveniência:(Do lat. \textunderscore tetraplare\textunderscore  + gr. \textunderscore odous\textunderscore , \textunderscore odontos\textunderscore )}
\end{itemize}
Gênero de musgos, de que há três espécies.
\section{Tètrápode}
\begin{itemize}
\item {Grp. gram.:adj.}
\end{itemize}
\begin{itemize}
\item {Proveniência:(Do gr. \textunderscore tetra\textunderscore  + \textunderscore pous\textunderscore , \textunderscore podos\textunderscore )}
\end{itemize}
Que tem quatro pés.
\section{Tètrapodólitho}
\begin{itemize}
\item {Grp. gram.:m.}
\end{itemize}
\begin{itemize}
\item {Proveniência:(Do gr. \textunderscore tetra\textunderscore  + \textunderscore pous\textunderscore  + \textunderscore lithos\textunderscore )}
\end{itemize}
Quadrúpede fóssil.
\section{Tetrapodólito}
\begin{itemize}
\item {Grp. gram.:m.}
\end{itemize}
\begin{itemize}
\item {Proveniência:(Do gr. \textunderscore tetra\textunderscore  + \textunderscore pous\textunderscore  + \textunderscore lithos\textunderscore )}
\end{itemize}
Quadrúpede fóssil.
\section{Tètrapodologia}
\begin{itemize}
\item {Grp. gram.:f.}
\end{itemize}
\begin{itemize}
\item {Proveniência:(Do gr. \textunderscore tetra\textunderscore  + \textunderscore pous\textunderscore  + \textunderscore logos\textunderscore )}
\end{itemize}
Tratado á cêrca dos quadrúpedes.
\section{Tètrapolar}
\begin{itemize}
\item {Grp. gram.:adj.}
\end{itemize}
\begin{itemize}
\item {Proveniência:(De \textunderscore tetra\textunderscore  + \textunderscore polar\textunderscore )}
\end{itemize}
Diz-se de um novo dýnamo, inventado no Brasil em 1901, pelo Dr. Paulo Saboia. Cf. \textunderscore Jorn.-do-Comm.\textunderscore , do Rio, de 21-IV-901.
\section{Tètrápora}
\begin{itemize}
\item {Grp. gram.:f.}
\end{itemize}
\begin{itemize}
\item {Proveniência:(Do gr. \textunderscore tetra\textunderscore  + \textunderscore poros\textunderscore )}
\end{itemize}
Gênero de plantas myrtáceas.
\section{Tètráptero}
\begin{itemize}
\item {Grp. gram.:adj.}
\end{itemize}
\begin{itemize}
\item {Utilização:Bot.}
\end{itemize}
\begin{itemize}
\item {Proveniência:(Do gr. \textunderscore tetra\textunderscore  + \textunderscore pteron\textunderscore )}
\end{itemize}
Que tem quatro asas.
Que tem quatro appêndices, em fórma de asas.
\section{Tètrápylo}
\begin{itemize}
\item {Grp. gram.:adj.}
\end{itemize}
\begin{itemize}
\item {Grp. gram.:M.}
\end{itemize}
\begin{itemize}
\item {Utilização:Bot.}
\end{itemize}
\begin{itemize}
\item {Proveniência:(Lat. \textunderscore tetrapylum\textunderscore )}
\end{itemize}
Que tem quatro portas.
Gênero de plantas oleáceas.
\section{Tetrarca}
\begin{itemize}
\item {Grp. gram.:f.}
\end{itemize}
\begin{itemize}
\item {Proveniência:(Lat. \textunderscore tetrarcha\textunderscore )}
\end{itemize}
Governador de tetrarquia.
\section{Tetrarcado}
\begin{itemize}
\item {Grp. gram.:m.}
\end{itemize}
\begin{itemize}
\item {Proveniência:(De \textunderscore tetrarca\textunderscore )}
\end{itemize}
Cargo ou dignidade de tetrarca.
\section{Tetrarcha}
\begin{itemize}
\item {fónica:ca}
\end{itemize}
\begin{itemize}
\item {Grp. gram.:f.}
\end{itemize}
\begin{itemize}
\item {Proveniência:(Lat. \textunderscore tetrarcha\textunderscore )}
\end{itemize}
Governador de tetrarchia.
\section{Tetrarchado}
\begin{itemize}
\item {fónica:ca}
\end{itemize}
\begin{itemize}
\item {Grp. gram.:m.}
\end{itemize}
\begin{itemize}
\item {Proveniência:(De \textunderscore tetrarcha\textunderscore )}
\end{itemize}
Cargo ou dignidade de tetrarcha.
\section{Tetrarchia}
\begin{itemize}
\item {fónica:qui}
\end{itemize}
\begin{itemize}
\item {Grp. gram.:f.}
\end{itemize}
\begin{itemize}
\item {Proveniência:(De \textunderscore tetrarca\textunderscore )}
\end{itemize}
Cada uma das quatro partes, províncias ou govêrnos, em que se dividiam alguns Estados.
Tetrarchado.
Secção de 64 homens na antiga phalange grega.
\section{Tètrarhythmo}
\begin{itemize}
\item {Grp. gram.:adj.}
\end{itemize}
\begin{itemize}
\item {Proveniência:(Lat. \textunderscore tetrarhythmus\textunderscore )}
\end{itemize}
Que tem quatro rhythmos.
\section{Tetrarquia}
\begin{itemize}
\item {Grp. gram.:f.}
\end{itemize}
\begin{itemize}
\item {Proveniência:(De \textunderscore tetrarca\textunderscore )}
\end{itemize}
Cada uma das quatro partes, províncias ou govêrnos, em que se dividiam alguns Estados.
Tetrarcado.
Secção de 64 homens na antiga falange grega.
\section{Tetrarritmo}
\begin{itemize}
\item {Grp. gram.:adj.}
\end{itemize}
\begin{itemize}
\item {Proveniência:(Lat. \textunderscore tetrarhythmus\textunderscore )}
\end{itemize}
Que tem quatro ritmos.
\section{Tètráscelo}
\begin{itemize}
\item {Grp. gram.:m.}
\end{itemize}
\begin{itemize}
\item {Utilização:Archeol.}
\end{itemize}
\begin{itemize}
\item {Proveniência:(Do gr. \textunderscore tetra\textunderscore  + \textunderscore skelos\textunderscore )}
\end{itemize}
Ornato, que é uma das fórmas da suástica, e que é constituído por quatro linhas curvas divergentes, em cruz, de um centro commum, formando roseta.
\section{Tètrasemo}
\begin{itemize}
\item {fónica:sê}
\end{itemize}
\begin{itemize}
\item {Grp. gram.:adj.}
\end{itemize}
\begin{itemize}
\item {Proveniência:(Lat. \textunderscore tetrasemus\textunderscore )}
\end{itemize}
Dizia-se do pé de quatro sýllabas, na métrica antiga.
\section{Tètrasépalo}
\begin{itemize}
\item {fónica:sé}
\end{itemize}
\begin{itemize}
\item {Grp. gram.:adj.}
\end{itemize}
\begin{itemize}
\item {Proveniência:(De \textunderscore tetra...\textunderscore  + \textunderscore sepala\textunderscore )}
\end{itemize}
Que tem quatro sépalas.
\section{Tètraspermo}
\begin{itemize}
\item {Grp. gram.:adj.}
\end{itemize}
\begin{itemize}
\item {Utilização:Bot.}
\end{itemize}
\begin{itemize}
\item {Proveniência:(Do gr. \textunderscore tetra\textunderscore  + \textunderscore sperma\textunderscore )}
\end{itemize}
Que encerra quatro grãos.
\section{Tètrassemo}
\begin{itemize}
\item {Grp. gram.:adj.}
\end{itemize}
\begin{itemize}
\item {Proveniência:(Lat. \textunderscore tetrasemus\textunderscore )}
\end{itemize}
Dizia-se do pé de quatro sílabas, na métrica antiga.
\section{Tètrassépalo}
\begin{itemize}
\item {Grp. gram.:adj.}
\end{itemize}
\begin{itemize}
\item {Proveniência:(De \textunderscore tetra...\textunderscore  + \textunderscore sepala\textunderscore )}
\end{itemize}
Que tem quatro sépalas.
\section{Tètrassilábico}
\begin{itemize}
\item {Grp. gram.:adj.}
\end{itemize}
\begin{itemize}
\item {Proveniência:(De \textunderscore tetra...\textunderscore  + \textunderscore sílaba\textunderscore )}
\end{itemize}
O mesmo que \textunderscore quadrissílabo\textunderscore .
\section{Tètrassílabo}
\begin{itemize}
\item {Grp. gram.:m.  e  adj.}
\end{itemize}
\begin{itemize}
\item {Proveniência:(De \textunderscore tetra...\textunderscore  + \textunderscore sílaba\textunderscore )}
\end{itemize}
O mesmo que \textunderscore quadrissílabo\textunderscore .
\section{Tètrastêmone}
\begin{itemize}
\item {Grp. gram.:adj.}
\end{itemize}
\begin{itemize}
\item {Utilização:Bot.}
\end{itemize}
\begin{itemize}
\item {Proveniência:(Do gr. \textunderscore tetra\textunderscore  + \textunderscore stemon\textunderscore )}
\end{itemize}
Que tem quatro estames livres.
\section{Tètrásticho}
\begin{itemize}
\item {fónica:co}
\end{itemize}
\begin{itemize}
\item {Grp. gram.:adj.}
\end{itemize}
\begin{itemize}
\item {Grp. gram.:M.}
\end{itemize}
\begin{itemize}
\item {Proveniência:(Lat. \textunderscore tetrastichum\textunderscore )}
\end{itemize}
Que tem quatro fileiras.
Composto de quatro versos.
Estrophe de quatro versos.
\section{Tetrástico}
\begin{itemize}
\item {Grp. gram.:adj.}
\end{itemize}
\begin{itemize}
\item {Grp. gram.:M.}
\end{itemize}
\begin{itemize}
\item {Proveniência:(Lat. \textunderscore tetrastichum\textunderscore )}
\end{itemize}
Que tem quatro fileiras.
Composto de quatro versos.
Estrofe de quatro versos.
\section{Tètrastigma}
\begin{itemize}
\item {Grp. gram.:f.}
\end{itemize}
\begin{itemize}
\item {Proveniência:(Do gr. \textunderscore tetra\textunderscore  + \textunderscore stigma\textunderscore )}
\end{itemize}
Gênero de videiras, em que se dividiu a família das ampelídeas.
\section{Tètrástilo}
\begin{itemize}
\item {Grp. gram.:m.}
\end{itemize}
\begin{itemize}
\item {Proveniência:(Lat. \textunderscore tetrastylum\textunderscore )}
\end{itemize}
Edifício com quatro ordens de colunas.
\section{Tètrástomo}
\begin{itemize}
\item {Grp. gram.:adj.}
\end{itemize}
\begin{itemize}
\item {Utilização:Zool.}
\end{itemize}
\begin{itemize}
\item {Proveniência:(Do gr. \textunderscore tetra\textunderscore  + \textunderscore stoma\textunderscore )}
\end{itemize}
Que tem quatro bôcas ou sugadoiros.
\section{Tetrástrofo}
\begin{itemize}
\item {Grp. gram.:adj.}
\end{itemize}
\begin{itemize}
\item {Proveniência:(Do gr. \textunderscore tetra\textunderscore  + \textunderscore strophe\textunderscore )}
\end{itemize}
Que abrange quatro estrofes.
\section{Tètrástropho}
\begin{itemize}
\item {Grp. gram.:adj.}
\end{itemize}
\begin{itemize}
\item {Proveniência:(Do gr. \textunderscore tetra\textunderscore  + \textunderscore strophe\textunderscore )}
\end{itemize}
Que abrange quatro estrophes.
\section{Tètrástylo}
\begin{itemize}
\item {Grp. gram.:m.}
\end{itemize}
\begin{itemize}
\item {Proveniência:(Lat. \textunderscore tetrastylum\textunderscore )}
\end{itemize}
Edifício com quatro ordens de columnas.
\section{Tètrasyllábico}
\begin{itemize}
\item {fónica:si}
\end{itemize}
\begin{itemize}
\item {Grp. gram.:adj.}
\end{itemize}
\begin{itemize}
\item {Proveniência:(De \textunderscore tetra...\textunderscore  + \textunderscore sýllaba\textunderscore )}
\end{itemize}
O mesmo que \textunderscore quadrisýllabo\textunderscore .
\section{Tètrasýllabo}
\begin{itemize}
\item {fónica:si}
\end{itemize}
\begin{itemize}
\item {Grp. gram.:m.  e  adj.}
\end{itemize}
\begin{itemize}
\item {Proveniência:(De \textunderscore tetra...\textunderscore  + \textunderscore sýllaba\textunderscore )}
\end{itemize}
O mesmo que \textunderscore quadrisýllabo\textunderscore .
\section{Tètratómico}
\begin{itemize}
\item {Grp. gram.:adj.}
\end{itemize}
\begin{itemize}
\item {Utilização:Phýs.}
\end{itemize}
\begin{itemize}
\item {Proveniência:(De \textunderscore tetra...\textunderscore  + \textunderscore atómico\textunderscore )}
\end{itemize}
Diz-se de um átomo, que tem quatro pontos de attracção.
Diz-se dos corpos, que não são saturados senão por quatro átomos de outro corpo.
\section{Tètrátono}
\begin{itemize}
\item {Grp. gram.:m.}
\end{itemize}
\begin{itemize}
\item {Utilização:Mús.}
\end{itemize}
\begin{itemize}
\item {Utilização:Des.}
\end{itemize}
Intervallo de quatro tons.
\section{Tètravó}
\begin{itemize}
\item {Grp. gram.:f.}
\end{itemize}
Fem. de \textunderscore tètravô\textunderscore .
\section{Tètravô}
\begin{itemize}
\item {Grp. gram.:m.}
\end{itemize}
\begin{itemize}
\item {Proveniência:(De \textunderscore tetra...\textunderscore  + \textunderscore avô\textunderscore )}
\end{itemize}
O pai do trisavô ou da trisavó. Cf. Castilho, \textunderscore Fastos\textunderscore , II, 288.
\section{Talamego}
\begin{itemize}
\item {Grp. gram.:m.}
\end{itemize}
\begin{itemize}
\item {Proveniência:(Lat. \textunderscore thalamegus\textunderscore )}
\end{itemize}
Embarcação antiga, com muitos beliches, elegante e adornada; espécie de gôndola, usada outrora no Egipto, especialmente.
\section{Talâmico}
\begin{itemize}
\item {Grp. gram.:adj.}
\end{itemize}
\begin{itemize}
\item {Utilização:Bot.}
\end{itemize}
\begin{itemize}
\item {Proveniência:(De \textunderscore tálamo\textunderscore )}
\end{itemize}
Que tem a inserção sôbre o receptáculo.
\section{Tàlamifloras}
\begin{itemize}
\item {Grp. gram.:f. pl.}
\end{itemize}
\begin{itemize}
\item {Proveniência:(De \textunderscore tálamo\textunderscore  + \textunderscore flôr\textunderscore )}
\end{itemize}
Numerosa classe de plantas, cuja corolla é polipétala e inserta com os estames no alargamento de um pedúnculo.
\section{Tálamo}
\begin{itemize}
\item {Grp. gram.:m.}
\end{itemize}
\begin{itemize}
\item {Utilização:Fig.}
\end{itemize}
\begin{itemize}
\item {Utilização:Bot.}
\end{itemize}
\begin{itemize}
\item {Proveniência:(Lat. \textunderscore thálamus\textunderscore )}
\end{itemize}
Leito conjugal.
Casamento.
Bodas.
Alargamento do pedúnculo de certas plantas.
Cálice das plantas.
Receptáculo das plantas.
\section{Talássero}
\begin{itemize}
\item {Grp. gram.:m.}
\end{itemize}
\begin{itemize}
\item {Proveniência:(Lat. \textunderscore thalasseros\textunderscore )}
\end{itemize}
Remédio antigo para doenças de olhos.
\section{Talassia}
\begin{itemize}
\item {Grp. gram.:f.}
\end{itemize}
\begin{itemize}
\item {Utilização:Neol.}
\end{itemize}
\begin{itemize}
\item {Proveniência:(Do gr. \textunderscore thalassa\textunderscore )}
\end{itemize}
O mesmo que [[enjôo do mar|enjôo]].
\section{Talassiarca}
\begin{itemize}
\item {Grp. gram.:m.}
\end{itemize}
\begin{itemize}
\item {Proveniência:(Do gr. \textunderscore thalassa\textunderscore  + \textunderscore arkhein\textunderscore )}
\end{itemize}
Supremo chefe de armada, espécie de almirante, entre os antigos Gregos e Romanos.
\section{Talassiarquia}
\begin{itemize}
\item {Grp. gram.:f.}
\end{itemize}
Pôsto ou dignidade de talassiarca.
\section{Talássico}
\begin{itemize}
\item {Grp. gram.:adj.}
\end{itemize}
\begin{itemize}
\item {Proveniência:(Lat. \textunderscore thalassicus\textunderscore )}
\end{itemize}
Relativo ao mar; que tem a côr do mar.
\section{Talassina}
\begin{itemize}
\item {Grp. gram.:f.}
\end{itemize}
Gênero de crustáceos decápodes, tipo dos talassíneos.
\section{Talassíneos}
\begin{itemize}
\item {Grp. gram.:m. pl.}
\end{itemize}
\begin{itemize}
\item {Proveniência:(Do gr. \textunderscore thalassa\textunderscore )}
\end{itemize}
Família de crustáceos decápodes macruros.
\section{Talassiofilo}
\begin{itemize}
\item {Grp. gram.:m.}
\end{itemize}
\begin{itemize}
\item {Proveniência:(Do gr. \textunderscore thalassa\textunderscore  + \textunderscore phulon\textunderscore )}
\end{itemize}
Gênero de plantas fíceas.
\section{Talassiófito}
\begin{itemize}
\item {Grp. gram.:m.}
\end{itemize}
\begin{itemize}
\item {Proveniência:(Do gr. \textunderscore thalassa\textunderscore  + \textunderscore phuton\textunderscore )}
\end{itemize}
Designação genérica dos vegetaes, que crescem no fundo do mar ou nas rochas marítimas que acompanham o litoral.
\section{Talassite}
\begin{itemize}
\item {Grp. gram.:m.}
\end{itemize}
\begin{itemize}
\item {Proveniência:(Lat. \textunderscore thalassites\textunderscore )}
\end{itemize}
Vinho que os antigos conservavam envasilhado sob a água do mar.
\section{Talassocracia}
\begin{itemize}
\item {Grp. gram.:f.}
\end{itemize}
\begin{itemize}
\item {Utilização:P. us.}
\end{itemize}
Domínio dos mares.
(Cp. \textunderscore talassócrata\textunderscore )
\section{Talassócrata}
\begin{itemize}
\item {Grp. gram.:adj.}
\end{itemize}
\begin{itemize}
\item {Proveniência:(Do gr. \textunderscore thalassa\textunderscore  + \textunderscore kratos\textunderscore )}
\end{itemize}
Que domina os mares.
\section{Talassografia}
\begin{itemize}
\item {Grp. gram.:f.}
\end{itemize}
\begin{itemize}
\item {Proveniência:(Do gr. \textunderscore thalassa\textunderscore  + \textunderscore graphein\textunderscore )}
\end{itemize}
Descripção dos mares.
\section{Talassofobia}
\begin{itemize}
\item {Grp. gram.:f.}
\end{itemize}
\begin{itemize}
\item {Proveniência:(Do gr. \textunderscore thalassa\textunderscore  + \textunderscore phobein\textunderscore )}
\end{itemize}
Mêdo mórbido do mar.
\section{Talassófobo}
\begin{itemize}
\item {Grp. gram.:m.}
\end{itemize}
Aquele que tem talassofobia.
\section{Talassográfico}
\begin{itemize}
\item {Grp. gram.:adj.}
\end{itemize}
Relativo á talassografia.
\section{Talassomel}
\begin{itemize}
\item {Grp. gram.:m.}
\end{itemize}
\begin{itemize}
\item {Proveniência:(Lat. \textunderscore thalassomelli\textunderscore )}
\end{itemize}
Bebida, feita de mel e água do mar.
\section{Talassómetro}
\begin{itemize}
\item {Grp. gram.:m.}
\end{itemize}
\begin{itemize}
\item {Proveniência:(Do gr. \textunderscore thalassa\textunderscore  + \textunderscore metron\textunderscore )}
\end{itemize}
Sonda marítima.
\section{Talassosfera}
\begin{itemize}
\item {Grp. gram.:f.}
\end{itemize}
\begin{itemize}
\item {Proveniência:(Do gr. \textunderscore thalassa\textunderscore  + \textunderscore sphaira\textunderscore )}
\end{itemize}
A parte líquida do globo terrestre.
\section{Talassoterapia}
\begin{itemize}
\item {Grp. gram.:f.}
\end{itemize}
\begin{itemize}
\item {Proveniência:(Do gr. \textunderscore thalassa\textunderscore  + \textunderscore therapeia\textunderscore )}
\end{itemize}
Tratamento terapêutico pelos banhos de mar e clima marítimo.
\section{Tália}
\begin{itemize}
\item {Grp. gram.:f.}
\end{itemize}
\begin{itemize}
\item {Proveniência:(De \textunderscore Thalia\textunderscore , n. p.)}
\end{itemize}
Planta canácea da América tropical.
Espécie de molusco, o mesmo que \textunderscore salpa\textunderscore .
\section{Taliato}
\begin{itemize}
\item {Grp. gram.:m.}
\end{itemize}
\begin{itemize}
\item {Proveniência:(De \textunderscore tálio\textunderscore )}
\end{itemize}
Nome de vários sáes, de que só um é conhecido, o sal de potássio.
\section{Tálico}
\begin{itemize}
\item {Grp. gram.:adj.}
\end{itemize}
\begin{itemize}
\item {Proveniência:(De \textunderscore tálio\textunderscore )}
\end{itemize}
Diz-se de certos corpos, em cuja composição química entra o tálio na proporção máxima.
\section{Talietro}
\begin{itemize}
\item {Grp. gram.:m.}
\end{itemize}
\begin{itemize}
\item {Proveniência:(Lat. \textunderscore thalietrum\textunderscore )}
\end{itemize}
Planta ranunculácea, de haste semelhante á da papoila.
\section{Talina}
\begin{itemize}
\item {Grp. gram.:f.}
\end{itemize}
Alcalóide da tália, medicinal.
\section{Tálio}
\begin{itemize}
\item {Grp. gram.:m.}
\end{itemize}
\begin{itemize}
\item {Proveniência:(Do gr. \textunderscore thallos\textunderscore )}
\end{itemize}
Metal branco, que se encontra nas pirites.
\section{Talioso}
\begin{itemize}
\item {Grp. gram.:adj.}
\end{itemize}
\begin{itemize}
\item {Proveniência:(De \textunderscore tálio\textunderscore )}
\end{itemize}
Diz-se de certos sáes em cuja combinação o tálio entra em proporção mínima.
\section{Talo}
\begin{itemize}
\item {Grp. gram.:m.}
\end{itemize}
\begin{itemize}
\item {Utilização:Bot.}
\end{itemize}
\begin{itemize}
\item {Proveniência:(Gr. \textunderscore thallos\textunderscore )}
\end{itemize}
Expansão foliácea dos líchens, semelhante a uma haste ou a uma fôlha e que constitue toda a planta, exceptuada a fructificação.
Expansão foliácea das algas.
Designação científica do corpo das plantas, quando destituído de raiz, caule e fôlhas.
\section{Talófitas}
\begin{itemize}
\item {Grp. gram.:f. pl.}
\end{itemize}
\begin{itemize}
\item {Proveniência:(Do gr. \textunderscore thallos\textunderscore  + \textunderscore phuton\textunderscore )}
\end{itemize}
Plantas sem caule.
\section{Talpófila}
\begin{itemize}
\item {Grp. gram.:f.}
\end{itemize}
Gênero de insectos coleópteros heterómeros.
\section{Talúsias}
\begin{itemize}
\item {Grp. gram.:f. pl.}
\end{itemize}
Antigas festas gregas, em que os lavradores agradeciam á divindade a bôa colheita de frutos.
\section{Talvegue}
\begin{itemize}
\item {Grp. gram.:m.}
\end{itemize}
\begin{itemize}
\item {Utilização:Neol.}
\end{itemize}
\begin{itemize}
\item {Proveniência:(Al. \textunderscore thalweg\textunderscore )}
\end{itemize}
Linha mais ou menos sinuosa no fundo de um vale, pela qual se dirigem águas correntes.
Linha de intersecção dos planos de duas encostas.
\section{Tamnocarpo}
\begin{itemize}
\item {Grp. gram.:m.}
\end{itemize}
\begin{itemize}
\item {Proveniência:(Do gr. \textunderscore thamnos\textunderscore  + \textunderscore karpos\textunderscore )}
\end{itemize}
Gênero de plantas fíceas.
\section{Tamnófilo}
\begin{itemize}
\item {Grp. gram.:m.}
\end{itemize}
\begin{itemize}
\item {Proveniência:(Do gr. \textunderscore thamnos\textunderscore  + \textunderscore philos\textunderscore )}
\end{itemize}
Gênero de insectos colópteros.
Gênero de aves insectívoras da América do Sul.
\section{Tamnófora}
\begin{itemize}
\item {Grp. gram.:f.}
\end{itemize}
\begin{itemize}
\item {Proveniência:(Do gr. \textunderscore thamnos\textunderscore  + \textunderscore phoros\textunderscore )}
\end{itemize}
Gênero de plantas fíceas.
\section{Tanatofilia}
\begin{itemize}
\item {Grp. gram.:f.}
\end{itemize}
\begin{itemize}
\item {Proveniência:(Do gr. \textunderscore thanatos\textunderscore  + \textunderscore philos\textunderscore )}
\end{itemize}
Psicose, que se revela pelo amor a tudo que diga respeito á morte.
\section{Tanatologia}
\begin{itemize}
\item {Grp. gram.:f.}
\end{itemize}
\begin{itemize}
\item {Proveniência:(Do gr. \textunderscore thanatos\textunderscore  + \textunderscore logos\textunderscore )}
\end{itemize}
Tratado á cêrca da morte.
\section{Tanatológico}
\begin{itemize}
\item {Grp. gram.:adj.}
\end{itemize}
Relativo á tanatologia.
\section{Tanatómetro}
\begin{itemize}
\item {Grp. gram.:m.}
\end{itemize}
\begin{itemize}
\item {Proveniência:(Do gr. \textunderscore thanatos\textunderscore  + \textunderscore metron\textunderscore )}
\end{itemize}
Termómetro, destinado a introduzir-se no estômago ou no recto, onde a temperatura desce rapidamente a 20° depois da morte real, o que nunca sucede depois da morte aparente.
\section{Tetigónia}
\begin{itemize}
\item {Grp. gram.:f.}
\end{itemize}
\begin{itemize}
\item {Proveniência:(Do gr. \textunderscore tettix\textunderscore , \textunderscore tettigos\textunderscore )}
\end{itemize}
Gênero de insectos hemítperos.
\section{Tetraz}
\begin{itemize}
\item {Grp. gram.:m.}
\end{itemize}
\begin{itemize}
\item {Proveniência:(Lat. \textunderscore tetrax\textunderscore )}
\end{itemize}
Gênero de aves gallináceas.
\section{Tetrazígia}
\begin{itemize}
\item {Grp. gram.:f.}
\end{itemize}
\begin{itemize}
\item {Proveniência:(Do gr. \textunderscore tetra\textunderscore  + \textunderscore zugos\textunderscore )}
\end{itemize}
Gênero de plantas melastomáceas.
\section{Tetrazýgia}
\begin{itemize}
\item {Grp. gram.:f.}
\end{itemize}
\begin{itemize}
\item {Proveniência:(Do gr. \textunderscore tetra\textunderscore  + \textunderscore zugos\textunderscore )}
\end{itemize}
Gênero de plantas melastomáceas.
\section{Tetrícia}
\begin{itemize}
\item {Grp. gram.:f.}
\end{itemize}
\begin{itemize}
\item {Proveniência:(Do lat. \textunderscore tetricus\textunderscore ?)}
\end{itemize}
Gênero de insectos orthópteros.
\section{Tetricidade}
\begin{itemize}
\item {Grp. gram.:f.}
\end{itemize}
\begin{itemize}
\item {Proveniência:(Lat. \textunderscore tetricitas\textunderscore )}
\end{itemize}
Qualidade de tétrico.
\section{Tétrico}
\begin{itemize}
\item {Grp. gram.:adj.}
\end{itemize}
\begin{itemize}
\item {Utilização:Fig.}
\end{itemize}
\begin{itemize}
\item {Proveniência:(Lat. \textunderscore tetricus\textunderscore )}
\end{itemize}
Muito triste.
Fúnebre.
Medonho.
Rígido, severo.
\section{Tetríneto}
\begin{itemize}
\item {Grp. gram.:m.}
\end{itemize}
O mesmo que \textunderscore tètraneto\textunderscore :«\textunderscore há de enterrar bisnetos e tetrinetos\textunderscore ». Castilho, \textunderscore Avarento\textunderscore , II, 7.
\section{Tetro}
\begin{itemize}
\item {Grp. gram.:adj.}
\end{itemize}
\begin{itemize}
\item {Proveniência:(Do lat. \textunderscore teter\textunderscore )}
\end{itemize}
Escuro, negro.
Feio.
Sombrio.
Horrível.
\section{Tètróbolo}
\begin{itemize}
\item {Grp. gram.:m.}
\end{itemize}
\begin{itemize}
\item {Proveniência:(Gr. \textunderscore tetrobolon\textunderscore )}
\end{itemize}
Antiga moéda grega, que valia quatro óbolos.
\section{Tetrócio}
\begin{itemize}
\item {Grp. gram.:m.}
\end{itemize}
Gênero de plantas alismáceas.
\section{Tètrodão}
\begin{itemize}
\item {Grp. gram.:m.}
\end{itemize}
Gênero de peixes, considerado como typo dos plectógnathos.
(Cp. \textunderscore tètrodonte\textunderscore )
\section{Tètrodonte}
\begin{itemize}
\item {Grp. gram.:m.}
\end{itemize}
\begin{itemize}
\item {Proveniência:(Do gr. \textunderscore tetra\textunderscore  + \textunderscore odous\textunderscore , \textunderscore odontos\textunderscore )}
\end{itemize}
O mesmo ou melhór que \textunderscore tetrodão\textunderscore .
\section{Tètroftalmo}
\begin{itemize}
\item {Grp. gram.:adj.}
\end{itemize}
\begin{itemize}
\item {Utilização:Zool.}
\end{itemize}
\begin{itemize}
\item {Proveniência:(Do gr. \textunderscore tetra\textunderscore  + \textunderscore ophthalmos\textunderscore )}
\end{itemize}
Que tem quatro olhos.
\section{Tètronal}
\begin{itemize}
\item {Grp. gram.:m.}
\end{itemize}
\begin{itemize}
\item {Utilização:Pharm.}
\end{itemize}
Medicamento hypnótico e sedativo.
\section{Tètrophthalmo}
\begin{itemize}
\item {Grp. gram.:adj.}
\end{itemize}
\begin{itemize}
\item {Utilização:Zool.}
\end{itemize}
\begin{itemize}
\item {Proveniência:(Do gr. \textunderscore tetra\textunderscore  + \textunderscore ophthalmos\textunderscore )}
\end{itemize}
Que tem quatro olhos.
\section{Tètrorchídeo}
\begin{itemize}
\item {fónica:qui}
\end{itemize}
\begin{itemize}
\item {Grp. gram.:m.}
\end{itemize}
\begin{itemize}
\item {Proveniência:(Do gr. \textunderscore tetra\textunderscore  + \textunderscore orkhis\textunderscore )}
\end{itemize}
Gênero de plantas euphorbiáceas.
\section{Tètrorquídeo}
\begin{itemize}
\item {Grp. gram.:m.}
\end{itemize}
\begin{itemize}
\item {Proveniência:(Do gr. \textunderscore tetra\textunderscore  + \textunderscore orkhis\textunderscore )}
\end{itemize}
Gênero de plantas euforbiáceas.
\section{Tettigónia}
\begin{itemize}
\item {Grp. gram.:f.}
\end{itemize}
\begin{itemize}
\item {Proveniência:(Do gr. \textunderscore tettix\textunderscore , \textunderscore tettigos\textunderscore )}
\end{itemize}
Gênero de insectos hemítperos.
\section{Tetudo}
\begin{itemize}
\item {Grp. gram.:adj.}
\end{itemize}
Que tem têtas grandes.
\section{Teu}
\begin{itemize}
\item {Grp. gram.:adj.}
\end{itemize}
\begin{itemize}
\item {Proveniência:(Do lat. \textunderscore tuus\textunderscore )}
\end{itemize}
(designativo da \textunderscore posse\textunderscore  que tem a pessôa com quem se fala)
Relativo a ti; próprio de ti: \textunderscore o teu talento\textunderscore .
\section{Teúba}
\begin{itemize}
\item {Grp. gram.:f.}
\end{itemize}
\begin{itemize}
\item {Utilização:Bras}
\end{itemize}
Abelha pequena e amarelada.
\section{Teucrieta}
\begin{itemize}
\item {fónica:ê}
\end{itemize}
\begin{itemize}
\item {Grp. gram.:f.}
\end{itemize}
\begin{itemize}
\item {Proveniência:(De \textunderscore têucrio\textunderscore )}
\end{itemize}
Planta, o mesmo que \textunderscore verónica\textunderscore .
\section{Têucrina}
\begin{itemize}
\item {Grp. gram.:f.}
\end{itemize}
Extracto medicinal de uma espécie de têucrio.
\section{Têucrio}
\begin{itemize}
\item {Grp. gram.:m.}
\end{itemize}
\begin{itemize}
\item {Proveniência:(Lat. \textunderscore teucrion\textunderscore )}
\end{itemize}
Planta, o mesmo que \textunderscore carvalhinha\textunderscore .
\section{Teucro}
\begin{itemize}
\item {Grp. gram.:m.  e  adj.}
\end{itemize}
\begin{itemize}
\item {Proveniência:(Lat. \textunderscore teucrus\textunderscore )}
\end{itemize}
O mesmo que \textunderscore troiano\textunderscore .
\section{Teúdo}
\begin{itemize}
\item {Grp. gram.:adj.}
\end{itemize}
\begin{itemize}
\item {Utilização:Ant.}
\end{itemize}
Que tem obrigação.
Que se teve ou se tem conservado: \textunderscore amásia teúda e manteúda\textunderscore .
\section{Teune}
\begin{itemize}
\item {Grp. gram.:adj.}
\end{itemize}
\begin{itemize}
\item {Utilização:Gír. de pedreiros.}
\end{itemize}
Teu.
\section{Tentéu}
\begin{itemize}
\item {Grp. gram.:m.}
\end{itemize}
\begin{itemize}
\item {Utilização:Bras}
\end{itemize}
Ave das regiões do Amazonas.
\section{Teutões}
\begin{itemize}
\item {Grp. gram.:m. pl.}
\end{itemize}
\begin{itemize}
\item {Proveniência:(Lat. \textunderscore teutones\textunderscore )}
\end{itemize}
Povo germânico, procedente das margens do Báltico.
\section{Teutónico}
\begin{itemize}
\item {Grp. gram.:adj.}
\end{itemize}
\begin{itemize}
\item {Grp. gram.:M.}
\end{itemize}
\begin{itemize}
\item {Proveniência:(Lat. \textunderscore teutonicus\textunderscore )}
\end{itemize}
Relativo aos Teutões.
Relativo aos Germanos.
Diz-se de uma Ordem militar, fundada em Acre, no tempo das Cruzadas.
Diz-se de uma espécie de letra gótica.
Antiga língua germânica, que se falou na côrte dos Reis francos até ao tratado de Verdun, em cuja época se formou o antigo francês.
\section{Texilheiro}
\begin{itemize}
\item {Grp. gram.:m.}
\end{itemize}
\begin{itemize}
\item {Utilização:Ant.}
\end{itemize}
Aquelle que faz tecidos de oiro, prata, etc. Cf. \textunderscore Instituto\textunderscore , LI, 643.
\section{Têxtil}
\begin{itemize}
\item {Grp. gram.:adj.}
\end{itemize}
\begin{itemize}
\item {Utilização:Neol.}
\end{itemize}
\begin{itemize}
\item {Proveniência:(Lat. \textunderscore textilis\textunderscore )}
\end{itemize}
Que se póde tecer; próprio para sêr tecido: \textunderscore fibras têxteis\textunderscore .
Relativo a tecelões: \textunderscore associação têxtil\textunderscore .
\section{Texto}
\begin{itemize}
\item {Grp. gram.:m.}
\end{itemize}
\begin{itemize}
\item {Proveniência:(Lat. \textunderscore textus\textunderscore )}
\end{itemize}
As próprias palavras de um autor, de um livro ou de um escrito, em relação a commentários ou additamentos que se fizeram ou se fazem sôbre ellas.
Palavras, que se citam, para demonstrar alguma coisa.
Palavras bíblicas, que o orador sagrado cita, tornando-as assumpto do sermão.
\section{Textório}
\begin{itemize}
\item {Grp. gram.:adj.}
\end{itemize}
\begin{itemize}
\item {Proveniência:(Do lat. \textunderscore textum\textunderscore )}
\end{itemize}
Relativo á arte de tecelão.
\section{Textual}
\begin{itemize}
\item {Grp. gram.:adj.}
\end{itemize}
Relativo ao texto.
Que está num texto.
Fielmente reproduzido, transcrito ou citado: \textunderscore palavras textuaes\textunderscore .
\section{Textualista}
\begin{itemize}
\item {Grp. gram.:m.}
\end{itemize}
\begin{itemize}
\item {Proveniência:(De \textunderscore textual\textunderscore )}
\end{itemize}
Aquelle que só attende á letra do texto, desprezando os intuitos do autor respectivo ou quaesquer glosas e commentários.
\section{Textualmente}
\begin{itemize}
\item {Grp. gram.:adv.}
\end{itemize}
De modo textual.
Conforme ao texto.
Segundo as palavras do texto.
\section{Textuário}
\begin{itemize}
\item {Grp. gram.:m.}
\end{itemize}
\begin{itemize}
\item {Proveniência:(Do lat. \textunderscore textus\textunderscore )}
\end{itemize}
Livro, que contém texto ou textos, sem commentários nem annotações.
\section{Textura}
\begin{itemize}
\item {Grp. gram.:f.}
\end{itemize}
\begin{itemize}
\item {Proveniência:(Lat. \textunderscore textura\textunderscore )}
\end{itemize}
Acto ou effeito de tecer.
Contextura; trama.
\section{Texugo}
\begin{itemize}
\item {Grp. gram.:m.}
\end{itemize}
\begin{itemize}
\item {Utilização:Pop.}
\end{itemize}
Mammífero plantígrado.
Pessôa muito nutrida.
\section{Tez}
\begin{itemize}
\item {Grp. gram.:f.}
\end{itemize}
Epiderme do rosto.
Cútis; epiderme.
(Cp. cast. \textunderscore tez\textunderscore )
\section{Tezime}
\begin{itemize}
\item {Grp. gram.:m.}
\end{itemize}
\begin{itemize}
\item {Utilização:Pesc.}
\end{itemize}
Designação antiga da rêde de sacada.
\section{Thalamego}
\begin{itemize}
\item {Grp. gram.:m.}
\end{itemize}
\begin{itemize}
\item {Proveniência:(Lat. \textunderscore thalamegus\textunderscore )}
\end{itemize}
Embarcação antiga, com muitos beliches, elegante e adornada; espécie de gôndola, usada outrora no Egypto, especialmente.
\section{Thalâmico}
\begin{itemize}
\item {Grp. gram.:adj.}
\end{itemize}
\begin{itemize}
\item {Utilização:Bot.}
\end{itemize}
\begin{itemize}
\item {Proveniência:(De \textunderscore thálamo\textunderscore )}
\end{itemize}
Que tem a inserção sôbre o receptáculo.
\section{Thàlamifloras}
\begin{itemize}
\item {Grp. gram.:f. pl.}
\end{itemize}
\begin{itemize}
\item {Proveniência:(De \textunderscore thálamo\textunderscore  + \textunderscore flôr\textunderscore )}
\end{itemize}
Numerosa classe de plantas, cuja corolla é polypétala e inserta com os estames no alargamento de um pedúnculo.
\section{Thálamo}
\begin{itemize}
\item {Grp. gram.:m.}
\end{itemize}
\begin{itemize}
\item {Utilização:Fig.}
\end{itemize}
\begin{itemize}
\item {Utilização:Bot.}
\end{itemize}
\begin{itemize}
\item {Proveniência:(Lat. \textunderscore thálamus\textunderscore )}
\end{itemize}
Leito conjugal.
Casamento.
Bodas.
Alargamento do pedúnculo de certas plantas.
Cálice das plantas.
Receptáculo das plantas.
\section{Thalássero}
\begin{itemize}
\item {Grp. gram.:m.}
\end{itemize}
\begin{itemize}
\item {Proveniência:(Lat. \textunderscore thalasseros\textunderscore )}
\end{itemize}
Remédio antigo para doenças de olhos.
\section{Thalassia}
\begin{itemize}
\item {Grp. gram.:f.}
\end{itemize}
\begin{itemize}
\item {Utilização:Neol.}
\end{itemize}
\begin{itemize}
\item {Proveniência:(Do gr. \textunderscore thalassa\textunderscore )}
\end{itemize}
O mesmo que [[enjôo do mar|enjôo]].
\section{Thalassiarcha}
\begin{itemize}
\item {fónica:ca}
\end{itemize}
\begin{itemize}
\item {Grp. gram.:m.}
\end{itemize}
\begin{itemize}
\item {Proveniência:(Do gr. \textunderscore thalassa\textunderscore  + \textunderscore arkhein\textunderscore )}
\end{itemize}
Supremo chefe de armada, espécie de almirante, entre os antigos Gregos e Romanos.
\section{Thalassiarchia}
\begin{itemize}
\item {fónica:qui}
\end{itemize}
\begin{itemize}
\item {Grp. gram.:f.}
\end{itemize}
Pôsto ou dignidade de thalassiarcha.
\section{Thalássico}
\begin{itemize}
\item {Grp. gram.:adj.}
\end{itemize}
\begin{itemize}
\item {Proveniência:(Lat. \textunderscore thalassicus\textunderscore )}
\end{itemize}
Relativo ao mar; que tem a côr do mar.
\section{Thalassina}
\begin{itemize}
\item {Grp. gram.:f.}
\end{itemize}
Gênero de crustáceos decápodes, typo dos thalassíneos.
\section{Thalassíneos}
\begin{itemize}
\item {Grp. gram.:m. pl.}
\end{itemize}
\begin{itemize}
\item {Proveniência:(Do gr. \textunderscore thalassa\textunderscore )}
\end{itemize}
Família de crustáceos decápodes macruros.
\section{Thalassiophyllo}
\begin{itemize}
\item {Grp. gram.:m.}
\end{itemize}
\begin{itemize}
\item {Proveniência:(Do gr. \textunderscore thalassa\textunderscore  + \textunderscore phulon\textunderscore )}
\end{itemize}
Gênero de plantas phýceas.
\section{Thalassióphyto}
\begin{itemize}
\item {Grp. gram.:m.}
\end{itemize}
\begin{itemize}
\item {Proveniência:(Do gr. \textunderscore thalassa\textunderscore  + \textunderscore phuton\textunderscore )}
\end{itemize}
Designação genérica dos vegetaes, que crescem no fundo do mar ou nas rochas marítimas que acompanham o littoral.
\section{Thalassite}
\begin{itemize}
\item {Grp. gram.:m.}
\end{itemize}
\begin{itemize}
\item {Proveniência:(Lat. \textunderscore thalassites\textunderscore )}
\end{itemize}
Vinho que os antigos conservavam envasilhado sob a água do mar.
\section{Thalassocracia}
\begin{itemize}
\item {Grp. gram.:f.}
\end{itemize}
\begin{itemize}
\item {Utilização:P. us.}
\end{itemize}
Domínio dos mares.
(Cp. \textunderscore thalassócrata\textunderscore )
\section{Thalassócrata}
\begin{itemize}
\item {Grp. gram.:adj.}
\end{itemize}
\begin{itemize}
\item {Proveniência:(Do gr. \textunderscore thalassa\textunderscore  + \textunderscore kratos\textunderscore )}
\end{itemize}
Que domina os mares.
\section{Thalassographia}
\begin{itemize}
\item {Grp. gram.:f.}
\end{itemize}
\begin{itemize}
\item {Proveniência:(Do gr. \textunderscore thalassa\textunderscore  + \textunderscore graphein\textunderscore )}
\end{itemize}
Descripção dos mares.
\section{Thalassográphico}
\begin{itemize}
\item {Grp. gram.:adj.}
\end{itemize}
Relativo á thalassographia.
\section{Thalassomel}
\begin{itemize}
\item {Grp. gram.:m.}
\end{itemize}
\begin{itemize}
\item {Proveniência:(Lat. \textunderscore thalassomelli\textunderscore )}
\end{itemize}
Bebida, feita de mel e água do mar.
\section{Thalassómetro}
\begin{itemize}
\item {Grp. gram.:m.}
\end{itemize}
\begin{itemize}
\item {Proveniência:(Do gr. \textunderscore thalassa\textunderscore  + \textunderscore metron\textunderscore )}
\end{itemize}
Sonda marítima.
\section{Thalassophobia}
\begin{itemize}
\item {Grp. gram.:f.}
\end{itemize}
\begin{itemize}
\item {Proveniência:(Do gr. \textunderscore thalassa\textunderscore  + \textunderscore phobein\textunderscore )}
\end{itemize}
Mêdo mórbido do mar.
\section{Thalassóphobo}
\begin{itemize}
\item {Grp. gram.:m.}
\end{itemize}
Aquelle que tem thalassophobia.
\section{Thalassosphera}
\begin{itemize}
\item {Grp. gram.:f.}
\end{itemize}
\begin{itemize}
\item {Proveniência:(Do gr. \textunderscore thalassa\textunderscore  + \textunderscore sphaira\textunderscore )}
\end{itemize}
A parte líquida do globo terrestre.
\section{Thalassotherapia}
\begin{itemize}
\item {Grp. gram.:f.}
\end{itemize}
\begin{itemize}
\item {Proveniência:(Do gr. \textunderscore thalassa\textunderscore  + \textunderscore therapeia\textunderscore )}
\end{itemize}
Tratamento therapêutico pelos banhos de mar e clima marítimo.
\section{Thália}
\begin{itemize}
\item {Grp. gram.:f.}
\end{itemize}
\begin{itemize}
\item {Proveniência:(De \textunderscore Thalia\textunderscore , n. p.)}
\end{itemize}
Planta canácea da América tropical.
Espécie de mollusco, o mesmo que \textunderscore salpa\textunderscore .
\section{Thalietro}
\begin{itemize}
\item {Grp. gram.:m.}
\end{itemize}
\begin{itemize}
\item {Proveniência:(Lat. \textunderscore thalietrum\textunderscore )}
\end{itemize}
Planta ranunculácea, de haste semelhante á da papoila.
\section{Thalina}
\begin{itemize}
\item {Grp. gram.:f.}
\end{itemize}
Alcalóide da thália, medicinal.
\section{Thalliato}
\begin{itemize}
\item {Grp. gram.:m.}
\end{itemize}
\begin{itemize}
\item {Proveniência:(De \textunderscore thállio\textunderscore )}
\end{itemize}
Nome de vários sáes, de que só um é conhecido, o sal de potássio.
\section{Thállico}
\begin{itemize}
\item {Grp. gram.:adj.}
\end{itemize}
\begin{itemize}
\item {Proveniência:(De \textunderscore thállio\textunderscore )}
\end{itemize}
Diz-se de certos corpos, em cuja composição chímica entra o thállio na proporção máxima.
\section{Thállio}
\begin{itemize}
\item {Grp. gram.:m.}
\end{itemize}
\begin{itemize}
\item {Proveniência:(Do gr. \textunderscore thallos\textunderscore )}
\end{itemize}
Metal branco, que se encontra nas pyrites.
\section{Thallioso}
\begin{itemize}
\item {Grp. gram.:adj.}
\end{itemize}
\begin{itemize}
\item {Proveniência:(De \textunderscore thállio\textunderscore )}
\end{itemize}
Diz-se de certos sáes em cuja combinação o thálio entra em porpoção mínima.
\section{Thallo}
\begin{itemize}
\item {Grp. gram.:m.}
\end{itemize}
\begin{itemize}
\item {Utilização:Bot.}
\end{itemize}
\begin{itemize}
\item {Proveniência:(Gr. \textunderscore thallos\textunderscore )}
\end{itemize}
Expansão foliácea dos líchens, semelhante a uma haste ou a uma fôlha e que constitue toda a planta, exceptuada a fructificação.
Expansão foliácea das algas.
Designação scientífica do corpo das plantas, quando destituído de raiz, caule e fôlhas.
\section{Thallóphytas}
\begin{itemize}
\item {Grp. gram.:f. pl.}
\end{itemize}
\begin{itemize}
\item {Proveniência:(Do gr. \textunderscore thallos\textunderscore  + \textunderscore phuton\textunderscore )}
\end{itemize}
Plantas sem caule.
\section{Thalpóphila}
\begin{itemize}
\item {Grp. gram.:f.}
\end{itemize}
Gênero de insectos coleópteros heterómeros.
\section{Thalúsias}
\begin{itemize}
\item {Grp. gram.:f. pl.}
\end{itemize}
Antigas festas gregas, em que os lavradores agradeciam á divindade a bôa colheita de frutos.
\section{Thalvegue}
\begin{itemize}
\item {Grp. gram.:m.}
\end{itemize}
\begin{itemize}
\item {Utilização:Neol.}
\end{itemize}
\begin{itemize}
\item {Proveniência:(Al. \textunderscore thalweg\textunderscore )}
\end{itemize}
Linha mais ou menos sinuosa no fundo de um valle, pela qual se dirigem águas correntes.
Linha de intersecção dos planos de duas encostas.
\section{Thamnocarpo}
\begin{itemize}
\item {Grp. gram.:m.}
\end{itemize}
\begin{itemize}
\item {Proveniência:(Do gr. \textunderscore thamnos\textunderscore  + \textunderscore karpos\textunderscore )}
\end{itemize}
Gênero de plantas phýceas.
\section{Thamnóphilo}
\begin{itemize}
\item {Grp. gram.:m.}
\end{itemize}
\begin{itemize}
\item {Proveniência:(Do gr. \textunderscore thamnos\textunderscore  + \textunderscore philos\textunderscore )}
\end{itemize}
Gênero de insectos colópteros.
Gênero de aves insectívoras da América do Sul.
\section{Thamnóphora}
\begin{itemize}
\item {Grp. gram.:f.}
\end{itemize}
\begin{itemize}
\item {Proveniência:(Do gr. \textunderscore thamnos\textunderscore  + \textunderscore phoros\textunderscore )}
\end{itemize}
Gênero de plantas phýceas.
\section{Thanatologia}
\begin{itemize}
\item {Grp. gram.:f.}
\end{itemize}
\begin{itemize}
\item {Proveniência:(Do gr. \textunderscore thanatos\textunderscore  + \textunderscore logos\textunderscore )}
\end{itemize}
Tratado á cêrca da morte.
\section{Thanatológico}
\begin{itemize}
\item {Grp. gram.:adj.}
\end{itemize}
Relativo á thanatologia
\section{Thanatómetro}
\begin{itemize}
\item {Grp. gram.:m.}
\end{itemize}
\begin{itemize}
\item {Proveniência:(Do gr. \textunderscore thanatos\textunderscore  + \textunderscore metron\textunderscore )}
\end{itemize}
Thermómetro, destinado a introduzir-se no estômago ou no recto, onde a temperatura desce rapidamente a 20° depois da morte real, o que nunca succede depois da morte apparente.
\section{Thanatofilia}
\begin{itemize}
\item {Grp. gram.:f.}
\end{itemize}
\begin{itemize}
\item {Proveniência:(Do gr. \textunderscore thanatos\textunderscore  + \textunderscore philos\textunderscore )}
\end{itemize}
Psychose, que se revela pelo amor a tudo que diga respeito á morte.
\section{Tanatófilo}
\begin{itemize}
\item {Grp. gram.:m.}
\end{itemize}
\begin{itemize}
\item {Proveniência:(Do gr. \textunderscore thanatos\textunderscore  + \textunderscore philos\textunderscore )}
\end{itemize}
Gênero de insectos coleópteros pentâmeros.
\section{Tanatofobia}
\begin{itemize}
\item {Grp. gram.:f.}
\end{itemize}
\begin{itemize}
\item {Proveniência:(Do gr. \textunderscore thanatos\textunderscore  + \textunderscore phobos\textunderscore )}
\end{itemize}
Exagerado temor da morte, temor que é sintoma de hipocondria.
\section{Tápsia}
\begin{itemize}
\item {Grp. gram.:f.}
\end{itemize}
\begin{itemize}
\item {Proveniência:(Gr. \textunderscore thapsia\textunderscore )}
\end{itemize}
Planta umbelífera, medicinal.
\section{Targélias}
\begin{itemize}
\item {Grp. gram.:f.}
\end{itemize}
\begin{itemize}
\item {Proveniência:(Do gr. \textunderscore thargelon\textunderscore , caldeira)}
\end{itemize}
Antigas festas áticas, em honra de Diana.
\section{Targélio}
\begin{itemize}
\item {Grp. gram.:m.}
\end{itemize}
Mês, em que se celebravam as targélias e que era o undécimo do ano ático.
\section{Tármico}
\begin{itemize}
\item {Grp. gram.:adj.}
\end{itemize}
O mesmo que \textunderscore esternutatório\textunderscore .
\section{Taumásia}
\begin{itemize}
\item {Grp. gram.:f.}
\end{itemize}
\begin{itemize}
\item {Proveniência:(Do gr. \textunderscore thaumasios\textunderscore )}
\end{itemize}
Gênero de plantas fíceas.
\section{Taumaturgia}
\begin{itemize}
\item {Grp. gram.:f.}
\end{itemize}
Obra de taumaturgo.
\section{Taumatúrgico}
\begin{itemize}
\item {Grp. gram.:adj.}
\end{itemize}
Relativo á taumaturgia.
\section{Taumaturgo}
\begin{itemize}
\item {Grp. gram.:m.  e  adj.}
\end{itemize}
\begin{itemize}
\item {Proveniência:(Gr. \textunderscore thaumatourgos\textunderscore )}
\end{itemize}
O que faz milagres.
\section{Teáceas}
\begin{itemize}
\item {Grp. gram.:f. pl.}
\end{itemize}
\begin{itemize}
\item {Proveniência:(De \textunderscore teáceo\textunderscore )}
\end{itemize}
Família de plantas dicotiledóneas.
\section{Teáceo}
\begin{itemize}
\item {Grp. gram.:adj.}
\end{itemize}
\begin{itemize}
\item {Utilização:Gal}
\end{itemize}
\begin{itemize}
\item {Proveniência:(Fr. \textunderscore théace\textunderscore )}
\end{itemize}
Relativo ou semelhante ao chá.
\section{Teália}
\begin{itemize}
\item {Grp. gram.:f.}
\end{itemize}
Gênero de crustáceos decápodes.
\section{Teame}
\begin{itemize}
\item {Grp. gram.:m.}
\end{itemize}
\begin{itemize}
\item {Utilização:Miner.}
\end{itemize}
\begin{itemize}
\item {Proveniência:(De \textunderscore Theame\textunderscore , n. p.?)}
\end{itemize}
Pedra da Etiópia, da qual se dizia que tinha a propriedade de repelir o ferro, ao contrário do íman.
\section{Teandria}
\begin{itemize}
\item {Grp. gram.:f.}
\end{itemize}
\begin{itemize}
\item {Proveniência:(Do gr. \textunderscore theos\textunderscore  + \textunderscore aner\textunderscore )}
\end{itemize}
O mesmo que \textunderscore teantropia\textunderscore .
\section{Teândrico}
\begin{itemize}
\item {Grp. gram.:adj.}
\end{itemize}
Relativo á teandria.
\section{Teangélide}
\begin{itemize}
\item {Grp. gram.:f.}
\end{itemize}
\begin{itemize}
\item {Proveniência:(Lat. \textunderscore theangelis\textunderscore )}
\end{itemize}
Planta do Libano, da qual se dizia que despertava o entusiasmo profético.
\section{Teantropia}
\begin{itemize}
\item {Grp. gram.:f.}
\end{itemize}
\begin{itemize}
\item {Proveniência:(De \textunderscore teantropo\textunderscore )}
\end{itemize}
Tratado á cêrca de Deus feito homem.
\section{Teantropista}
\begin{itemize}
\item {Grp. gram.:m.  e  f.}
\end{itemize}
\begin{itemize}
\item {Proveniência:(De \textunderscore teantropo\textunderscore )}
\end{itemize}
Pessôa, que atribue a Deus qualidades ou paixões humanas.
\section{Teantropo}
\begin{itemize}
\item {Grp. gram.:m.}
\end{itemize}
\begin{itemize}
\item {Proveniência:(Do gr. \textunderscore theos\textunderscore  + \textunderscore anthropos\textunderscore )}
\end{itemize}
Cristo, considerado como Deus e homem.
\section{Teatino}
\begin{itemize}
\item {Grp. gram.:m.}
\end{itemize}
\begin{itemize}
\item {Proveniência:(De \textunderscore Theato\textunderscore , n. p.)}
\end{itemize}
Membro da Ordem religiosa de San-Caetano, fundada por um bispo de Theato.
\section{Teatrada}
\begin{itemize}
\item {Grp. gram.:f.}
\end{itemize}
\begin{itemize}
\item {Utilização:Fam.}
\end{itemize}
Função de teatro.
\section{Teatral}
\begin{itemize}
\item {Grp. gram.:adj.}
\end{itemize}
\begin{itemize}
\item {Utilização:Fig.}
\end{itemize}
\begin{itemize}
\item {Proveniência:(Lat. \textunderscore theatralis\textunderscore )}
\end{itemize}
Relativo a teatro.
Ostentoso; espectaculoso.
\section{Teatralidade}
\begin{itemize}
\item {Grp. gram.:f.}
\end{itemize}
\begin{itemize}
\item {Utilização:Neol.}
\end{itemize}
\begin{itemize}
\item {Proveniência:(De \textunderscore teatral\textunderscore )}
\end{itemize}
Qualidade daquilo que tem condições cênicas, para se representar.
\section{Teatralização}
\begin{itemize}
\item {Grp. gram.:f.}
\end{itemize}
Acto de teatralizar.
\section{Teatralizar}
\begin{itemize}
\item {Grp. gram.:v. t.}
\end{itemize}
\begin{itemize}
\item {Proveniência:(De \textunderscore teatral\textunderscore )}
\end{itemize}
Adaptar ás exigências do teatro; tornar representável em teatro.
\section{Teatralmente}
\begin{itemize}
\item {Grp. gram.:adv.}
\end{itemize}
De modo teatral.
Espectaculosamente.
De fórma dramática, trágica ou comovente.
\section{Teatrelho}
\begin{itemize}
\item {fónica:trê}
\end{itemize}
\begin{itemize}
\item {Grp. gram.:m.}
\end{itemize}
\begin{itemize}
\item {Utilização:Deprec.}
\end{itemize}
Teatro ordinário. Cf. Trind. Coelho, \textunderscore In Illo Tempore\textunderscore .
\section{Thanatóphilo}
\begin{itemize}
\item {Grp. gram.:m.}
\end{itemize}
\begin{itemize}
\item {Proveniência:(Do gr. \textunderscore thanatos\textunderscore  + \textunderscore philos\textunderscore )}
\end{itemize}
Gênero de insectos coleópteros pentâmeros.
\section{Thanatophobia}
\begin{itemize}
\item {Grp. gram.:f.}
\end{itemize}
\begin{itemize}
\item {Proveniência:(Do gr. \textunderscore thanatos\textunderscore  + \textunderscore phobos\textunderscore )}
\end{itemize}
Exaggerado temor da morte, temor que é symptoma de hypocondria.
\section{Thápsia}
\begin{itemize}
\item {Grp. gram.:f.}
\end{itemize}
\begin{itemize}
\item {Proveniência:(Gr. \textunderscore thapsia\textunderscore )}
\end{itemize}
Planta umbellífera, medicinal.
\section{Thargélias}
\begin{itemize}
\item {Grp. gram.:f.}
\end{itemize}
\begin{itemize}
\item {Proveniência:(Do gr. \textunderscore thargelon\textunderscore , caldeira)}
\end{itemize}
Antigas festas átticas, em honra de Diana.
\section{Thargélio}
\begin{itemize}
\item {Grp. gram.:m.}
\end{itemize}
Mês, em que se celebravam as thargélias e que era o undécimo do anno áttico.
\section{Thármico}
\begin{itemize}
\item {Grp. gram.:adj.}
\end{itemize}
O mesmo que \textunderscore esternutatório\textunderscore .
\section{Thaumásia}
\begin{itemize}
\item {Grp. gram.:f.}
\end{itemize}
\begin{itemize}
\item {Proveniência:(Do gr. \textunderscore thaumasios\textunderscore )}
\end{itemize}
Gênero de plantas phýceas.
\section{Thaumaturgia}
\begin{itemize}
\item {Grp. gram.:f.}
\end{itemize}
Obra de thaumaturgo.
\section{Thaumatúrgico}
\begin{itemize}
\item {Grp. gram.:adj.}
\end{itemize}
Relativo á thaumaturgia.
\section{Thaumaturgo}
\begin{itemize}
\item {Grp. gram.:m.  e  adj.}
\end{itemize}
\begin{itemize}
\item {Proveniência:(Gr. \textunderscore thaumatourgos\textunderscore )}
\end{itemize}
O que faz milagres.
\section{Theáceas}
\begin{itemize}
\item {Grp. gram.:f. pl.}
\end{itemize}
\begin{itemize}
\item {Proveniência:(De \textunderscore theáceo\textunderscore )}
\end{itemize}
Família de plantas dycotyledóneas.
\section{Theáceo}
\begin{itemize}
\item {Grp. gram.:adj.}
\end{itemize}
\begin{itemize}
\item {Utilização:Gal}
\end{itemize}
\begin{itemize}
\item {Proveniência:(Fr. \textunderscore théace\textunderscore )}
\end{itemize}
Relativo ou semelhante ao chá.
\section{Theália}
\begin{itemize}
\item {Grp. gram.:f.}
\end{itemize}
Gênero de crustáceos decápodes.
\section{Theame}
\begin{itemize}
\item {Grp. gram.:m.}
\end{itemize}
\begin{itemize}
\item {Utilização:Miner.}
\end{itemize}
\begin{itemize}
\item {Proveniência:(De \textunderscore Theame\textunderscore , n. p.?)}
\end{itemize}
Pedra da Ethiópia, da qual se dizia que tinha a propriedade de repellir o ferro, ao contrário do íman.
\section{Theandria}
\begin{itemize}
\item {Grp. gram.:f.}
\end{itemize}
\begin{itemize}
\item {Proveniência:(Do gr. \textunderscore theos\textunderscore  + \textunderscore aner\textunderscore )}
\end{itemize}
O mesmo que \textunderscore theanthropia\textunderscore .
\section{Theândrico}
\begin{itemize}
\item {Grp. gram.:adj.}
\end{itemize}
Relativo á theandria.
\section{Theangélide}
\begin{itemize}
\item {Grp. gram.:f.}
\end{itemize}
\begin{itemize}
\item {Proveniência:(Lat. \textunderscore theangelis\textunderscore )}
\end{itemize}
Planta do Libano, da qual se dizia que despertava o enthusiasmo prophético.
\section{Theanthropia}
\begin{itemize}
\item {Grp. gram.:f.}
\end{itemize}
\begin{itemize}
\item {Proveniência:(De \textunderscore theanthropo\textunderscore )}
\end{itemize}
Tratado á cêrca de Deus feito homem.
\section{Theanthropista}
\begin{itemize}
\item {Grp. gram.:m.  e  f.}
\end{itemize}
\begin{itemize}
\item {Proveniência:(De \textunderscore theanthropo\textunderscore )}
\end{itemize}
Pessôa, que attribue a Deus qualidades ou paixões humanas.
\section{Theanthropo}
\begin{itemize}
\item {Grp. gram.:m.}
\end{itemize}
\begin{itemize}
\item {Proveniência:(Do gr. \textunderscore theos\textunderscore  + \textunderscore anthropos\textunderscore )}
\end{itemize}
Christo, considerado como Deus e homem.
\section{Theatino}
\begin{itemize}
\item {Grp. gram.:m.}
\end{itemize}
\begin{itemize}
\item {Proveniência:(De \textunderscore Theato\textunderscore , n. p.)}
\end{itemize}
Membro da Ordem religiosa de San-Caetano, fundada por um bispo de Theato.
\section{Theatrada}
\begin{itemize}
\item {Grp. gram.:f.}
\end{itemize}
\begin{itemize}
\item {Utilização:Fam.}
\end{itemize}
Funcção de theatro.
\section{Theatral}
\begin{itemize}
\item {Grp. gram.:adj.}
\end{itemize}
\begin{itemize}
\item {Utilização:Fig.}
\end{itemize}
\begin{itemize}
\item {Proveniência:(Lat. \textunderscore theatralis\textunderscore )}
\end{itemize}
Relativo a theatro.
Ostentoso; espectaculoso.
\section{Theatralidade}
\begin{itemize}
\item {Grp. gram.:f.}
\end{itemize}
\begin{itemize}
\item {Utilização:Neol.}
\end{itemize}
\begin{itemize}
\item {Proveniência:(De \textunderscore theatral\textunderscore )}
\end{itemize}
Qualidade daquillo que tem condições scênicas, para se representar.
\section{Theatralização}
\begin{itemize}
\item {Grp. gram.:f.}
\end{itemize}
Acto de theatralizar.
\section{Theatralizar}
\begin{itemize}
\item {Grp. gram.:v. t.}
\end{itemize}
\begin{itemize}
\item {Proveniência:(De \textunderscore theatral\textunderscore )}
\end{itemize}
Adaptar ás exigências do theatro; tornar representável em theatro.
\section{Theatralmente}
\begin{itemize}
\item {Grp. gram.:adv.}
\end{itemize}
De modo theatral.
Espectaculosamente.
De fórma dramática, trágica ou commovente.
\section{Theatrelho}
\begin{itemize}
\item {fónica:trê}
\end{itemize}
\begin{itemize}
\item {Grp. gram.:m.}
\end{itemize}
\begin{itemize}
\item {Utilização:Deprec.}
\end{itemize}
Theatro ordinário. Cf. Trind. Coelho, \textunderscore In Illo Tempore\textunderscore .
\section{Teátrico}
\begin{itemize}
\item {Grp. gram.:adj.}
\end{itemize}
\begin{itemize}
\item {Utilização:Des.}
\end{itemize}
\begin{itemize}
\item {Proveniência:(Lat. \textunderscore theatricus\textunderscore )}
\end{itemize}
O mesmo que \textunderscore teatral\textunderscore .
\section{Teatrista}
\begin{itemize}
\item {Grp. gram.:m.  e  f.}
\end{itemize}
Pessôa, que frequenta habitualmente teatros.
\section{Teatro}
\begin{itemize}
\item {Grp. gram.:m.}
\end{itemize}
\begin{itemize}
\item {Utilização:Fig.}
\end{itemize}
\begin{itemize}
\item {Proveniência:(Lat. \textunderscore theatrum\textunderscore )}
\end{itemize}
Lugar, onde se representam peças dramáticas.
Casa ou edifício, apropriado para representações dramáticas ou espectáculos.
Circo.
Conjunto das obras dramáticas de uma nação, de uma época ou de um autor: \textunderscore o teatro de Gil Vicente\textunderscore .
Literatura dramática.
Arte de representar.
Profissão de actor: \textunderscore vocação para o teatro\textunderscore .
Obra instructiva ou educativa.
Lugar, onde se realiza um acontecimento.
Ilusão, aparência.
\textunderscore Teatro anatómico\textunderscore , casa para autopsias e outras operações anatómicas, nas escolas de Medicina.
\section{Tebaico}
\begin{itemize}
\item {Grp. gram.:adj.}
\end{itemize}
\begin{itemize}
\item {Utilização:Pharm.}
\end{itemize}
\begin{itemize}
\item {Proveniência:(Lat. \textunderscore thebaicus\textunderscore )}
\end{itemize}
Relativo á cidade de Thebas.
Diz-se do extracto aquoso do ópio.
\section{Tebaida}
\begin{itemize}
\item {Grp. gram.:f.}
\end{itemize}
\begin{itemize}
\item {Utilização:Fig.}
\end{itemize}
\begin{itemize}
\item {Proveniência:(De \textunderscore Thebaida\textunderscore , n. p.)}
\end{itemize}
Solidão; retiro; insulamento.
\section{Tebano}
\begin{itemize}
\item {Grp. gram.:adj.}
\end{itemize}
\begin{itemize}
\item {Grp. gram.:M.}
\end{itemize}
\begin{itemize}
\item {Proveniência:(Lat. \textunderscore thebanus\textunderscore )}
\end{itemize}
Relativo a Thebas.
Habitante de Thebas.
\section{Teca}
\begin{itemize}
\item {Grp. gram.:f.}
\end{itemize}
\begin{itemize}
\item {Utilização:Bot.}
\end{itemize}
\begin{itemize}
\item {Proveniência:(Lat. \textunderscore theca\textunderscore )}
\end{itemize}
Célula mãe.
Urnário dos musgos.
\section{Tecamonádeos}
\begin{itemize}
\item {Grp. gram.:m. pl.}
\end{itemize}
\begin{itemize}
\item {Proveniência:(Do gr. \textunderscore theke\textunderscore  + \textunderscore monas\textunderscore )}
\end{itemize}
Família de infusórios.
\section{Tecápode}
\begin{itemize}
\item {Grp. gram.:m.}
\end{itemize}
\begin{itemize}
\item {Utilização:Bot.}
\end{itemize}
\begin{itemize}
\item {Proveniência:(Do gr. \textunderscore theke\textunderscore  + \textunderscore pous\textunderscore , \textunderscore podos\textunderscore )}
\end{itemize}
Suporte dos frutos, nas plantas cariofiláceas.
\section{Tecla}
\begin{itemize}
\item {Grp. gram.:f.}
\end{itemize}
Gênero de insectos lepidópteros diurnos.
\section{Tecodontes}
\begin{itemize}
\item {Grp. gram.:m. pl.}
\end{itemize}
\begin{itemize}
\item {Utilização:Zool.}
\end{itemize}
\begin{itemize}
\item {Proveniência:(Do gr. \textunderscore theke\textunderscore  + \textunderscore odous\textunderscore , \textunderscore odontos\textunderscore )}
\end{itemize}
Animaes, que têm os dentes implantados nos alvéolos.
\section{Tecodontossauro}
\begin{itemize}
\item {Grp. gram.:m.}
\end{itemize}
\begin{itemize}
\item {Proveniência:(Do gr. \textunderscore theke\textunderscore  + \textunderscore odous\textunderscore , \textunderscore odontos\textunderscore  + \textunderscore saura\textunderscore )}
\end{itemize}
Gênero de reptís sáurios.
\section{Tecólito}
\begin{itemize}
\item {Grp. gram.:m.}
\end{itemize}
\begin{itemize}
\item {Proveniência:(Do gr. \textunderscore theke\textunderscore  + \textunderscore lithos\textunderscore )}
\end{itemize}
Pedra, que se encontra dentro das esponjas, e a que os antigos atribuíram a propriedade de dissolver os cálculos urinários.
\section{Teiforme}
\begin{itemize}
\item {fónica:te-i}
\end{itemize}
\begin{itemize}
\item {Grp. gram.:adj.}
\end{itemize}
\begin{itemize}
\item {Utilização:Gal}
\end{itemize}
\begin{itemize}
\item {Proveniência:(Fr. \textunderscore theiforme\textunderscore )}
\end{itemize}
Que se prepara como chá.--Emprega-se ás vezes em Medicina.
\section{Teína}
\begin{itemize}
\item {Grp. gram.:f.}
\end{itemize}
\begin{itemize}
\item {Utilização:Gal}
\end{itemize}
\begin{itemize}
\item {Proveniência:(Do fr. \textunderscore thé\textunderscore )}
\end{itemize}
Princípio activo do chá. Cf. E. Monin, \textunderscore Hygiene do Estômago\textunderscore , 333.
\section{Teísmo}
\begin{itemize}
\item {Grp. gram.:m.}
\end{itemize}
\begin{itemize}
\item {Proveniência:(Do gr. \textunderscore theos\textunderscore )}
\end{itemize}
Crença na existência de Deus.
\section{Teísta}
\begin{itemize}
\item {Grp. gram.:m. ,  f.  e  adj.}
\end{itemize}
\begin{itemize}
\item {Proveniência:(Do gr. \textunderscore theos\textunderscore )}
\end{itemize}
Pessôa, que acredita na existência de Deus.
\section{Telalgia}
\begin{itemize}
\item {Grp. gram.:f.}
\end{itemize}
\begin{itemize}
\item {Proveniência:(Do gr. \textunderscore thele\textunderscore  + \textunderscore algos\textunderscore )}
\end{itemize}
Dôr na glândula mamal.
\section{Telite}
\begin{itemize}
\item {Grp. gram.:f.}
\end{itemize}
\begin{itemize}
\item {Proveniência:(Do gr. \textunderscore thele\textunderscore )}
\end{itemize}
Inflamação do bico do peito.
\section{Tema}
\begin{itemize}
\item {Grp. gram.:m.}
\end{itemize}
\begin{itemize}
\item {Utilização:Prov.}
\end{itemize}
\begin{itemize}
\item {Utilização:minh.}
\end{itemize}
\begin{itemize}
\item {Proveniência:(Lat. \textunderscore thema\textunderscore )}
\end{itemize}
Assunto.
Proposição, de que se vai tratar ou que se vai provar.
Trecho, dado ou indicado pelo professor aos seus alunos, para exercício de tradução ou de análise.
Texto, em que se baseia um sermão.
Radical, base ou elemento primitivo de uma palavra, a que se junta uma desinência ou sufixo.
Primeira pessôa do presente do indicativo, na gramática grega.
\textunderscore Tomar tema\textunderscore , fixar na memória qualquer coisa, aprendê-la.
\section{Tema}
\begin{itemize}
\item {Grp. gram.:m.}
\end{itemize}
Espécie de melro do Chile.
\section{Temático}
\begin{itemize}
\item {Grp. gram.:adj.}
\end{itemize}
\begin{itemize}
\item {Utilização:Gram.}
\end{itemize}
\begin{itemize}
\item {Proveniência:(Do gr. \textunderscore thematikos\textunderscore )}
\end{itemize}
Relativo ao tema das palavras.
\section{Tematologia}
\begin{itemize}
\item {Grp. gram.:f.}
\end{itemize}
\begin{itemize}
\item {Utilização:Gram.}
\end{itemize}
\begin{itemize}
\item {Proveniência:(Do gr. \textunderscore thema\textunderscore  + \textunderscore logos\textunderscore )}
\end{itemize}
Parte da Morfologia, que estuda a constituição das fórmas específicas ou temas de cada uma das categorias gramaticaes que entram no discurso e que fôram classificadas na lexiologia. Cf. A. G. R. de Vasconcélloz, \textunderscore Gram. Port.\textunderscore , 78.
\section{Theátrico}
\begin{itemize}
\item {Grp. gram.:adj.}
\end{itemize}
\begin{itemize}
\item {Utilização:Des.}
\end{itemize}
\begin{itemize}
\item {Proveniência:(Lat. \textunderscore theatricus\textunderscore )}
\end{itemize}
O mesmo que \textunderscore theatral\textunderscore .
\section{Theatrista}
\begin{itemize}
\item {Grp. gram.:m.  e  f.}
\end{itemize}
Pessôa, que frequenta habitualmente theatros.
\section{Theatro}
\begin{itemize}
\item {Grp. gram.:m.}
\end{itemize}
\begin{itemize}
\item {Utilização:Fig.}
\end{itemize}
\begin{itemize}
\item {Proveniência:(Lat. \textunderscore theatrum\textunderscore )}
\end{itemize}
Lugar, onde se representam peças dramáticas.
Casa ou edifício, apropriado para representações dramáticas ou espectáculos.
Circo.
Conjunto das obras dramáticas de uma nação, de uma época ou de um autor: \textunderscore o theatro de Gil Vicente\textunderscore .
Literatura dramática.
Arte de representar.
Profissão de actor: \textunderscore vocação para o theatro\textunderscore .
Obra instructiva ou educativa.
Lugar, onde se realiza um acontecimento.
Illusão, apparência.
\textunderscore Theatro anatómico\textunderscore , casa para autopsias e outras operações anatómicas, nas escolas de Medicina.
\section{Thebaico}
\begin{itemize}
\item {Grp. gram.:adj.}
\end{itemize}
\begin{itemize}
\item {Utilização:Pharm.}
\end{itemize}
\begin{itemize}
\item {Proveniência:(Lat. \textunderscore thebaicus\textunderscore )}
\end{itemize}
Relativo á cidade de Thebas.
Diz-se do extracto aquoso do ópio.
\section{Thebaida}
\begin{itemize}
\item {Grp. gram.:f.}
\end{itemize}
\begin{itemize}
\item {Utilização:Fig.}
\end{itemize}
\begin{itemize}
\item {Proveniência:(De \textunderscore Thebaida\textunderscore , n. p.)}
\end{itemize}
Solidão; retiro; insulamento.
\section{Thebano}
\begin{itemize}
\item {Grp. gram.:adj.}
\end{itemize}
\begin{itemize}
\item {Grp. gram.:M.}
\end{itemize}
\begin{itemize}
\item {Proveniência:(Lat. \textunderscore thebanus\textunderscore )}
\end{itemize}
Relativo a Thebas.
Habitante de Thebas.
\section{Theca}
\begin{itemize}
\item {Grp. gram.:f.}
\end{itemize}
\begin{itemize}
\item {Utilização:Bot.}
\end{itemize}
\begin{itemize}
\item {Proveniência:(Lat. \textunderscore theca\textunderscore )}
\end{itemize}
Céllula mãe.
Urnário dos musgos.
\section{Thecamonádeos}
\begin{itemize}
\item {Grp. gram.:m. pl.}
\end{itemize}
\begin{itemize}
\item {Proveniência:(Do gr. \textunderscore theke\textunderscore  + \textunderscore monas\textunderscore )}
\end{itemize}
Família de infusórios.
\section{Thecápode}
\begin{itemize}
\item {Grp. gram.:m.}
\end{itemize}
\begin{itemize}
\item {Utilização:Bot.}
\end{itemize}
\begin{itemize}
\item {Proveniência:(Do gr. \textunderscore theke\textunderscore  + \textunderscore pous\textunderscore , \textunderscore podos\textunderscore )}
\end{itemize}
Supporte dos frutos, nas plantas caryophylláceas.
\section{Thecla}
\begin{itemize}
\item {Grp. gram.:f.}
\end{itemize}
Gênero de insectos lepidópteros diurnos.
\section{Thecodontes}
\begin{itemize}
\item {Grp. gram.:m. pl.}
\end{itemize}
\begin{itemize}
\item {Utilização:Zool.}
\end{itemize}
\begin{itemize}
\item {Proveniência:(Do gr. \textunderscore theke\textunderscore  + \textunderscore odous\textunderscore , \textunderscore odontos\textunderscore )}
\end{itemize}
Animaes, que têm os dentes implantados nos alvéolos.
\section{Thecodontosauro}
\begin{itemize}
\item {fónica:sau}
\end{itemize}
\begin{itemize}
\item {Grp. gram.:m.}
\end{itemize}
\begin{itemize}
\item {Proveniência:(Do gr. \textunderscore theke\textunderscore  + \textunderscore odous\textunderscore , \textunderscore odontos\textunderscore  + \textunderscore saura\textunderscore )}
\end{itemize}
Gênero de reptís sáurios.
\section{Thecólitho}
\begin{itemize}
\item {Grp. gram.:m.}
\end{itemize}
\begin{itemize}
\item {Proveniência:(Do gr. \textunderscore theke\textunderscore  + \textunderscore lithos\textunderscore )}
\end{itemize}
Pedra, que se encontra dentro das esponjas, e a que os antigos attribuíram a propriedade de dissolver os cálculos urinários.
\section{Theiforme}
\begin{itemize}
\item {Grp. gram.:adj.}
\end{itemize}
\begin{itemize}
\item {Utilização:Gal}
\end{itemize}
\begin{itemize}
\item {Proveniência:(Fr. \textunderscore theiforme\textunderscore )}
\end{itemize}
Que se prepara como chá.--Emprega-se ás vezes em Medicina.
\section{Theína}
\begin{itemize}
\item {Grp. gram.:f.}
\end{itemize}
\begin{itemize}
\item {Utilização:Gal}
\end{itemize}
\begin{itemize}
\item {Proveniência:(Do fr. \textunderscore thé\textunderscore )}
\end{itemize}
Princípio activo do chá. Cf. E. Monin, \textunderscore Hygiene do Estômago\textunderscore , 333.
\section{Theísmo}
\begin{itemize}
\item {Grp. gram.:m.}
\end{itemize}
\begin{itemize}
\item {Proveniência:(Do gr. \textunderscore theos\textunderscore )}
\end{itemize}
Crença na existência de Deus.
\section{Theísta}
\begin{itemize}
\item {Grp. gram.:m. ,  f.  e  adj.}
\end{itemize}
\begin{itemize}
\item {Proveniência:(Do gr. \textunderscore theos\textunderscore )}
\end{itemize}
Pessôa, que acredita na existência de Deus.
\section{Thelalgia}
\begin{itemize}
\item {Grp. gram.:f.}
\end{itemize}
\begin{itemize}
\item {Proveniência:(Do gr. \textunderscore thele\textunderscore  + \textunderscore algos\textunderscore )}
\end{itemize}
Dôr na glândula mamal.
\section{Thelite}
\begin{itemize}
\item {Grp. gram.:f.}
\end{itemize}
\begin{itemize}
\item {Proveniência:(Do gr. \textunderscore thele\textunderscore )}
\end{itemize}
Inflammação do bico do peito.
\section{Thema}
\begin{itemize}
\item {Grp. gram.:m.}
\end{itemize}
\begin{itemize}
\item {Utilização:Prov.}
\end{itemize}
\begin{itemize}
\item {Utilização:minh.}
\end{itemize}
\begin{itemize}
\item {Proveniência:(Lat. \textunderscore thema\textunderscore )}
\end{itemize}
Assumpto.
Proposição, de que se vai tratar ou que se vai provar.
Trecho, dado ou indicado pelo professor aos seus alumnos, para exercício de traducção ou de anályse.
Texto, em que se baseia um sermão.
Radical, base ou elemento primitivo de uma palavra, a que se junta uma desinência ou suffixo.
Primeira pessôa do presente do indicativo, na grammática grega.
\textunderscore Tomar thema\textunderscore , fixar na memória qualquer coisa, aprendê-la.
\section{Thema}
\begin{itemize}
\item {Grp. gram.:m.}
\end{itemize}
Espécie de melro do Chile.
\section{Themático}
\begin{itemize}
\item {Grp. gram.:adj.}
\end{itemize}
\begin{itemize}
\item {Utilização:Gram.}
\end{itemize}
\begin{itemize}
\item {Proveniência:(Do gr. \textunderscore thematikos\textunderscore )}
\end{itemize}
Relativo ao thema das palavras.
\section{Thematologia}
\begin{itemize}
\item {Grp. gram.:f.}
\end{itemize}
\begin{itemize}
\item {Utilização:Gram.}
\end{itemize}
\begin{itemize}
\item {Proveniência:(Do gr. \textunderscore thema\textunderscore  + \textunderscore logos\textunderscore )}
\end{itemize}
Parte da Morphologia, que estuda a constituição das fórmas específicas ou themas de cada uma das categorias grammaticaes que entram no discurso e que fôram classificadas na lexiologia. Cf. A. G. R. de Vasconcélloz, \textunderscore Gram. Port.\textunderscore , 78.
\section{Tematológico}
\begin{itemize}
\item {Grp. gram.:adj.}
\end{itemize}
Relativo á tematologia.
\section{Temisto}
\begin{itemize}
\item {Grp. gram.:m.}
\end{itemize}
Gênero de crustáceos anfípodes.
\section{Tenar}
\begin{itemize}
\item {Grp. gram.:m.}
\end{itemize}
\begin{itemize}
\item {Utilização:Anat.}
\end{itemize}
\begin{itemize}
\item {Proveniência:(Gr. \textunderscore thenar\textunderscore , palma da mão)}
\end{itemize}
Eminência da parte anterior e externa da mão, formada por certos músculos do polegar.
\section{Teobromina}
\begin{itemize}
\item {Grp. gram.:f.}
\end{itemize}
Princípio activo do chocolate. Cf. E. Monin, \textunderscore Hygiene do Estômago\textunderscore , 333.
\section{Teocal}
\begin{itemize}
\item {Grp. gram.:m.}
\end{itemize}
Templo, em fórma de pirâmide, usado pelos antigos Mexicanos.
\section{Teocina}
\begin{itemize}
\item {Grp. gram.:f.}
\end{itemize}
Medicamento diurético.
\section{Teocinto}
\begin{itemize}
\item {Grp. gram.:m.}
\end{itemize}
Planta forraginosa do Brasil. Cf. \textunderscore País\textunderscore , do Rio, de 22-XI-900.
\section{Teocracia}
\begin{itemize}
\item {Grp. gram.:f.}
\end{itemize}
\begin{itemize}
\item {Utilização:Fig.}
\end{itemize}
\begin{itemize}
\item {Proveniência:(Gr. \textunderscore theokratia\textunderscore )}
\end{itemize}
Govêrno, em que os chefes da nação pertencem á classe sacerdotal ou são considerados ministros de Deus ou dos deuses.
Sistema, escola ou seita, em que procuram predominar os indivíduos que o vulgo tem como invioláveis ou consagrados: \textunderscore as teocracias literárias...\textunderscore 
\section{Teócrata}
\begin{itemize}
\item {Grp. gram.:m.}
\end{itemize}
Membro da teocracia.
O que exerce poder teocrático.--Usa-se \textunderscore teocráta\textunderscore , mas é pronúncia errónea.
(Cp. \textunderscore teocracia\textunderscore )
\section{Teocraticamente}
\begin{itemize}
\item {Grp. gram.:adv.}
\end{itemize}
De modo teocrático.
Segundo o sistema da teocracia.
\section{Teocrático}
\begin{itemize}
\item {Grp. gram.:adj.}
\end{itemize}
Relativo á teocracia.
\section{Teocratizar}
\begin{itemize}
\item {Grp. gram.:v. t.}
\end{itemize}
\begin{itemize}
\item {Utilização:P. us.}
\end{itemize}
\begin{itemize}
\item {Proveniência:(De \textunderscore teocrata\textunderscore )}
\end{itemize}
Sujeitar a um poder teocrático.
\section{Teodicéa}
\begin{itemize}
\item {Grp. gram.:f.}
\end{itemize}
O mesmo que \textunderscore teodiceia\textunderscore .
\section{Teodiceia}
\begin{itemize}
\item {Grp. gram.:f.}
\end{itemize}
\begin{itemize}
\item {Utilização:Improp.}
\end{itemize}
\begin{itemize}
\item {Proveniência:(Do gr. \textunderscore theos\textunderscore  + \textunderscore dike\textunderscore )}
\end{itemize}
Parte da Teologia natural, que trata da justiça de Deus.
Parte da Filosophia, que trata da existência e dos atributos de Deus.
\section{Teodólito}
\begin{itemize}
\item {Grp. gram.:m.}
\end{itemize}
\begin{itemize}
\item {Proveniência:(Do gr. \textunderscore theamai\textunderscore  + \textunderscore dolitos\textunderscore )}
\end{itemize}
Instrumento astronómico e geodésico, para medir directamente as alturas zenitaes e os ângulos reduzidos ao horizonte.
\section{Teodosiano}
\begin{itemize}
\item {Grp. gram.:adj.}
\end{itemize}
\begin{itemize}
\item {Proveniência:(Lat. \textunderscore theodosíanus\textunderscore )}
\end{itemize}
Relativo ao imperador romano Teodósio I, o \textunderscore Grande\textunderscore , ou a seu neto Teodósio II.
\section{Teogenesia}
\begin{itemize}
\item {Grp. gram.:f.}
\end{itemize}
O mesmo que \textunderscore teogonia\textunderscore . Cf. Castilho, \textunderscore Fastos\textunderscore , I, 528.
\section{Teogonia}
\begin{itemize}
\item {Grp. gram.:f.}
\end{itemize}
\begin{itemize}
\item {Proveniência:(Lat. \textunderscore theogonia\textunderscore )}
\end{itemize}
Genealogia dos deuses.
Qualquer sistema religioso da antiguidade, á cêrca das relações dos deuses entre si e entre êles e os homens.
\section{Teogónico}
\begin{itemize}
\item {Grp. gram.:adj.}
\end{itemize}
Relativo á teogonia.
\section{Teogonista}
\begin{itemize}
\item {Grp. gram.:m.}
\end{itemize}
Aquelle que trata de teogonia.
\section{Teologal}
\begin{itemize}
\item {Grp. gram.:adj.}
\end{itemize}
\begin{itemize}
\item {Proveniência:(De \textunderscore teólogo\textunderscore )}
\end{itemize}
Relativo á Teologia.
\section{Teologalmente}
\begin{itemize}
\item {Grp. gram.:adv.}
\end{itemize}
De modo teologal.
\section{Teologastro}
\begin{itemize}
\item {Grp. gram.:m.}
\end{itemize}
\begin{itemize}
\item {Utilização:Deprec.}
\end{itemize}
Mau teólogo.
\section{Teologia}
\begin{itemize}
\item {Grp. gram.:f.}
\end{itemize}
\begin{itemize}
\item {Proveniência:(Lat. \textunderscore theologia\textunderscore )}
\end{itemize}
Doutrina, á cêrca das coisas divinas.
Doutrina da religião christan.
Doutrina.
Colecção de obras teológicas de um escritor.
Os teólogos: \textunderscore a Teologia defende os dogmas\textunderscore .
\section{Teófago}
\begin{itemize}
\item {Grp. gram.:adj.}
\end{itemize}
\begin{itemize}
\item {Proveniência:(Do gr. \textunderscore theos\textunderscore  + \textunderscore phagein\textunderscore )}
\end{itemize}
Que come Deus.
É epíteto injurioso que se deu aos Católicos.
\section{Teofania}
\begin{itemize}
\item {Grp. gram.:f.}
\end{itemize}
\begin{itemize}
\item {Proveniência:(Gr. \textunderscore theophania\textunderscore )}
\end{itemize}
Aparição ou revelação da divindade.
\section{Teofilantropia}
\begin{itemize}
\item {Grp. gram.:f.}
\end{itemize}
\begin{itemize}
\item {Proveniência:(Do gr. \textunderscore theos\textunderscore  + \textunderscore philos\textunderscore  + \textunderscore anthropos\textunderscore )}
\end{itemize}
Seita religiosa, que professava o teísmo e se estabeleceu em França, vigorando desde 1796 a 1801 em que foi suprimida.--O seu culto cifrava-se em discursos de moral e hinos a Deus e a todas as virtudes.
\section{Teofilantrópico}
\begin{itemize}
\item {Grp. gram.:adj.}
\end{itemize}
Relativo á teofilantropia.
\section{Teofilantropo}
\begin{itemize}
\item {Grp. gram.:m.}
\end{itemize}
Sectário da teofilantropia.
\section{Teofobia}
\begin{itemize}
\item {Grp. gram.:f.}
\end{itemize}
Estado ou qualidade de teófobo.
\section{Teófobo}
\begin{itemize}
\item {Grp. gram.:m.}
\end{itemize}
\begin{itemize}
\item {Proveniência:(Do gr. \textunderscore theos\textunderscore  + \textunderscore phobein\textunderscore )}
\end{itemize}
Aquele que tem aversão a Deus ou ás coisas divinas.
\section{Teofrasta}
\begin{itemize}
\item {Grp. gram.:f.}
\end{itemize}
Gênero de plantas mirsíneas.
\section{Teologicamente}
\begin{itemize}
\item {Grp. gram.:adv.}
\end{itemize}
De modo teológico.
Segundo as regras da Teologia.
\section{Teológico}
\begin{itemize}
\item {Grp. gram.:adj.}
\end{itemize}
\begin{itemize}
\item {Proveniência:(Lat. \textunderscore theologicus\textunderscore )}
\end{itemize}
Relativo á Teologia.
\section{Teologismo}
\begin{itemize}
\item {Grp. gram.:m.}
\end{itemize}
\begin{itemize}
\item {Proveniência:(De \textunderscore teologia\textunderscore )}
\end{itemize}
Abuso dos princípios teológicos.
\section{Teologizar}
\begin{itemize}
\item {Grp. gram.:v. i.}
\end{itemize}
Discorrer sôbre Teologia. Cf. Camillo, \textunderscore Mulhér Fatal\textunderscore , 225.
\section{Teólogo}
\begin{itemize}
\item {Grp. gram.:m.}
\end{itemize}
\begin{itemize}
\item {Grp. gram.:Adj.}
\end{itemize}
\begin{itemize}
\item {Proveniência:(Lat. \textunderscore theologus\textunderscore )}
\end{itemize}
Aquele que é perito em Teologia.
O que escreve á cêrca de Teologia.
O que estuda Teologia.
O mesmo que \textunderscore teológico\textunderscore . Cf. Filinto, XI, 153.
\section{Teomancia}
\begin{itemize}
\item {Grp. gram.:f.}
\end{itemize}
\begin{itemize}
\item {Proveniência:(Do gr. \textunderscore theos\textunderscore  + \textunderscore manteia\textunderscore )}
\end{itemize}
Sistema de adivinhação, por suposta inspiração divina.
\section{Teomania}
\begin{itemize}
\item {Grp. gram.:f.}
\end{itemize}
\begin{itemize}
\item {Proveniência:(Do gr. \textunderscore theos\textunderscore  + \textunderscore mania\textunderscore )}
\end{itemize}
Espécie de loucura, em que o doente se considera Deus ou por êle inspirado.
\section{Teomaníaco}
\begin{itemize}
\item {Grp. gram.:m.  e  adj.}
\end{itemize}
\begin{itemize}
\item {Proveniência:(De \textunderscore teomania\textunderscore )}
\end{itemize}
O que sofre teomania.
\section{Teomante}
\begin{itemize}
\item {Grp. gram.:m.}
\end{itemize}
Aquele que pratíca a teomancia.
\section{Teomitia}
\begin{itemize}
\item {Grp. gram.:f.}
\end{itemize}
\begin{itemize}
\item {Proveniência:(Do gr. \textunderscore theos\textunderscore  + \textunderscore muthos\textunderscore )}
\end{itemize}
Conjunto ou sistema dos dogmas antigos, que se conservaram por tradição.
\section{Teomítico}
\begin{itemize}
\item {Grp. gram.:adj.}
\end{itemize}
Relativo á teomitia.
\section{Teomitologia}
\begin{itemize}
\item {Grp. gram.:f.}
\end{itemize}
\begin{itemize}
\item {Proveniência:(Do gr. \textunderscore theos\textunderscore  + \textunderscore muthos\textunderscore  + \textunderscore logos\textunderscore )}
\end{itemize}
Tratado á cêrca dos deuses do Paganismo.
\section{Teomitológico}
\begin{itemize}
\item {Grp. gram.:adj.}
\end{itemize}
Relativo á teomitologia.
\section{Teopsia}
\begin{itemize}
\item {Grp. gram.:f.}
\end{itemize}
\begin{itemize}
\item {Proveniência:(Do gr. \textunderscore theos\textunderscore  + \textunderscore ops\textunderscore )}
\end{itemize}
Suposta aparição súbita de uma divindade.
\section{Teorema}
\begin{itemize}
\item {Grp. gram.:m.}
\end{itemize}
\begin{itemize}
\item {Proveniência:(Lat. \textunderscore theorema\textunderscore )}
\end{itemize}
Qualquer proposição que, para ser evidente, precisa de demonstração.
\section{Teoreticamente}
\begin{itemize}
\item {Grp. gram.:adv.}
\end{itemize}
\begin{itemize}
\item {Proveniência:(De \textunderscore teorético\textunderscore )}
\end{itemize}
O mesmo que \textunderscore teoricamente\textunderscore .
\section{Teorético}
\begin{itemize}
\item {Grp. gram.:adj.}
\end{itemize}
\begin{itemize}
\item {Proveniência:(Lat. \textunderscore theoreticus\textunderscore )}
\end{itemize}
O mesmo que \textunderscore teórico\textunderscore .
\section{Teoria}
\begin{itemize}
\item {Grp. gram.:f.}
\end{itemize}
\begin{itemize}
\item {Proveniência:(Lat. \textunderscore theoria\textunderscore )}
\end{itemize}
Conhecimento, limitado a princípios ou á especulação, sem passar á prática.
Conjunto de princípios fundamentaes de uma arte ou ciência.
Sistema ou doutrina á cêrca dêsses princípios.
Hipótese; noções geraes; utopia.
\section{Teoria}
\begin{itemize}
\item {Grp. gram.:f.}
\end{itemize}
\begin{itemize}
\item {Proveniência:(Gr. \textunderscore theoria\textunderscore )}
\end{itemize}
Deputação ou comissão que, na antiguidade, era enviada ou mandada em nome de uma cidade, para ir fazer sacrifícios aos deuses ou consultar um oráculo.--Há quem diga \textunderscore theória\textunderscore , como Filinto, XIV, 134, e XVI, 218.
\section{Thematológico}
\begin{itemize}
\item {Grp. gram.:adj.}
\end{itemize}
Relativo á thematologia.
\section{Themisto}
\begin{itemize}
\item {Grp. gram.:m.}
\end{itemize}
Gênero de crustáceos amphípodes.
\section{Thenar}
\begin{itemize}
\item {Grp. gram.:m.}
\end{itemize}
\begin{itemize}
\item {Utilização:Anat.}
\end{itemize}
\begin{itemize}
\item {Proveniência:(Gr. \textunderscore thenar\textunderscore , palma da mão)}
\end{itemize}
Eminência da parte anterior e externa da mão, formada por certos músculos do pollegar.
\section{Theobromina}
\begin{itemize}
\item {Grp. gram.:f.}
\end{itemize}
Princípio activo do chocolate. Cf. E. Monin, \textunderscore Hygiene do Estômago\textunderscore , 333.
\section{Theocal}
\begin{itemize}
\item {Grp. gram.:m.}
\end{itemize}
Templo, em fórma de pyrâmide, usado pelos antigos Mexicanos.
\section{Theocina}
\begin{itemize}
\item {Grp. gram.:f.}
\end{itemize}
Medicamento diurético.
\section{Theocintho}
\begin{itemize}
\item {Grp. gram.:m.}
\end{itemize}
Planta forraginosa do Brasil. Cf. \textunderscore País\textunderscore , do Rio, de 22-XI-900.
\section{Theocracia}
\begin{itemize}
\item {Grp. gram.:f.}
\end{itemize}
\begin{itemize}
\item {Utilização:Fig.}
\end{itemize}
\begin{itemize}
\item {Proveniência:(Gr. \textunderscore theokratia\textunderscore )}
\end{itemize}
Govêrno, em que os chefes da nação pertencem á classe sacerdotal ou são considerados ministros de Deus ou dos deuses.
Systema, escola ou seita, em que procuram predominar os indivíduos que o vulgo tem como invioláveis ou consagrados: \textunderscore as theocracias literárias...\textunderscore 
\section{Theócrata}
\begin{itemize}
\item {Grp. gram.:m.}
\end{itemize}
Membro da theocracia.
O que exerce poder theocrático.--Usa-se \textunderscore teocráta\textunderscore , mas é pronúncia errónea.
(Cp. \textunderscore theocracia\textunderscore )
\section{Theocraticamente}
\begin{itemize}
\item {Grp. gram.:adv.}
\end{itemize}
De modo theocrático.
Segundo o systema da theocracia.
\section{Theocrático}
\begin{itemize}
\item {Grp. gram.:adj.}
\end{itemize}
Relativo á theocracia.
\section{Theocratizar}
\begin{itemize}
\item {Grp. gram.:v. t.}
\end{itemize}
\begin{itemize}
\item {Utilização:P. us.}
\end{itemize}
\begin{itemize}
\item {Proveniência:(De \textunderscore theocrata\textunderscore )}
\end{itemize}
Sujeitar a um poder theocrático.
\section{Theodicéa}
\begin{itemize}
\item {Grp. gram.:f.}
\end{itemize}
O mesmo que \textunderscore theodiceia\textunderscore .
\section{Theodiceia}
\begin{itemize}
\item {Grp. gram.:f.}
\end{itemize}
\begin{itemize}
\item {Utilização:Improp.}
\end{itemize}
\begin{itemize}
\item {Proveniência:(Do gr. \textunderscore theos\textunderscore  + \textunderscore dike\textunderscore )}
\end{itemize}
Parte da Theologia natural, que trata da justiça de Deus.
Parte da Philosophia, que trata da existência e dos attributos de Deus.
\section{Theodólito}
\begin{itemize}
\item {Grp. gram.:m.}
\end{itemize}
\begin{itemize}
\item {Proveniência:(Do gr. \textunderscore theamai\textunderscore  + \textunderscore dolitos\textunderscore )}
\end{itemize}
Instrumento astronómico e geodésico, para medir directamente as alturas zenithaes e os ângulos reduzidos ao horizonte.
\section{Theodosiano}
\begin{itemize}
\item {Grp. gram.:adj.}
\end{itemize}
\begin{itemize}
\item {Proveniência:(Lat. \textunderscore theodosíanus\textunderscore )}
\end{itemize}
Relativo ao imperador romano Theodósio I, o \textunderscore Grande\textunderscore , ou a seu neto Theodósio II.
\section{Theogenesia}
\begin{itemize}
\item {Grp. gram.:f.}
\end{itemize}
O mesmo que \textunderscore theogonia\textunderscore . Cf. Castilho, \textunderscore Fastos\textunderscore , I, 528.
\section{Theogonia}
\begin{itemize}
\item {Grp. gram.:f.}
\end{itemize}
\begin{itemize}
\item {Proveniência:(Lat. \textunderscore theogonia\textunderscore )}
\end{itemize}
Genealogia dos deuses.
Qualquer systema religioso da antiguidade, á cêrca das relações dos deuses entre si e entre êlles e os homens.
\section{Theogónico}
\begin{itemize}
\item {Grp. gram.:adj.}
\end{itemize}
Relativo á theogonia.
\section{Theogonista}
\begin{itemize}
\item {Grp. gram.:m.}
\end{itemize}
Aquelle que trata de theogonia.
\section{Theologal}
\begin{itemize}
\item {Grp. gram.:adj.}
\end{itemize}
\begin{itemize}
\item {Proveniência:(De \textunderscore theólogo\textunderscore )}
\end{itemize}
Relativo á Theologia.
\section{Theologalmente}
\begin{itemize}
\item {Grp. gram.:adv.}
\end{itemize}
De modo theologal.
\section{Theologastro}
\begin{itemize}
\item {Grp. gram.:m.}
\end{itemize}
\begin{itemize}
\item {Utilização:Deprec.}
\end{itemize}
Mau theólogo.
\section{Theologia}
\begin{itemize}
\item {Grp. gram.:f.}
\end{itemize}
\begin{itemize}
\item {Proveniência:(Lat. \textunderscore theologia\textunderscore )}
\end{itemize}
Doutrina, á cêrca das coisas divinas.
Doutrina da religião christan.
Doutrina.
Collecção de obras theológicas de um escritor.
Os theólogos: \textunderscore a Theologia defende os dogmas\textunderscore .
\section{Theologicamente}
\begin{itemize}
\item {Grp. gram.:adv.}
\end{itemize}
De modo teológico.
Segundo as regras da Theologia.
\section{Theológico}
\begin{itemize}
\item {Grp. gram.:adj.}
\end{itemize}
\begin{itemize}
\item {Proveniência:(Lat. \textunderscore theologicus\textunderscore )}
\end{itemize}
Relativo á Theologia.
\section{Theologismo}
\begin{itemize}
\item {Grp. gram.:m.}
\end{itemize}
\begin{itemize}
\item {Proveniência:(De \textunderscore theologia\textunderscore )}
\end{itemize}
Abuso dos princípios theológicos.
\section{Theologizar}
\begin{itemize}
\item {Grp. gram.:v. i.}
\end{itemize}
Discorrer sôbre Theologia. Cf. Camillo, \textunderscore Mulhér Fatal\textunderscore , 225.
\section{Theólogo}
\begin{itemize}
\item {Grp. gram.:m.}
\end{itemize}
\begin{itemize}
\item {Grp. gram.:Adj.}
\end{itemize}
\begin{itemize}
\item {Proveniência:(Lat. \textunderscore theologus\textunderscore )}
\end{itemize}
Aquelle que é perito em Theologia.
O que escreve á cêrca de Theologia.
O que estuda Theologia.
O mesmo que \textunderscore theológico\textunderscore . Cf. Filinto, XI, 153.
\section{Theomancia}
\begin{itemize}
\item {Grp. gram.:f.}
\end{itemize}
\begin{itemize}
\item {Proveniência:(Do gr. \textunderscore theos\textunderscore  + \textunderscore manteia\textunderscore )}
\end{itemize}
Systema de adivinhação, por supposta inspiração divina.
\section{Theomania}
\begin{itemize}
\item {Grp. gram.:f.}
\end{itemize}
\begin{itemize}
\item {Proveniência:(Do gr. \textunderscore theos\textunderscore  + \textunderscore mania\textunderscore )}
\end{itemize}
Espécie de loucura, em que o doente se considera Deus ou por êlle inspirado.
\section{Theomaníaco}
\begin{itemize}
\item {Grp. gram.:m.  e  adj.}
\end{itemize}
\begin{itemize}
\item {Proveniência:(De \textunderscore theomania\textunderscore )}
\end{itemize}
O que soffre theomania.
\section{Theomante}
\begin{itemize}
\item {Grp. gram.:m.}
\end{itemize}
Aquelle que pratica a theomancia.
\section{Theomythia}
\begin{itemize}
\item {Grp. gram.:f.}
\end{itemize}
\begin{itemize}
\item {Proveniência:(Do gr. \textunderscore theos\textunderscore  + \textunderscore muthos\textunderscore )}
\end{itemize}
Conjunto ou systema dos dogmas antigos, que se conservaram por tradição.
\section{Theomýthico}
\begin{itemize}
\item {Grp. gram.:adj.}
\end{itemize}
Relativo á theomythia.
\section{Theomythologia}
\begin{itemize}
\item {Grp. gram.:f.}
\end{itemize}
\begin{itemize}
\item {Proveniência:(Do gr. \textunderscore theos\textunderscore  + \textunderscore muthos\textunderscore  + \textunderscore logos\textunderscore )}
\end{itemize}
Tratado á cêrca dos deuses do Paganismo.
\section{Theomythológico}
\begin{itemize}
\item {Grp. gram.:adj.}
\end{itemize}
Relativo á theomythologia.
\section{Theóphago}
\begin{itemize}
\item {Grp. gram.:adj.}
\end{itemize}
\begin{itemize}
\item {Proveniência:(Do gr. \textunderscore theos\textunderscore  + \textunderscore phagein\textunderscore )}
\end{itemize}
Que come Deus.
É epítheto injurioso que se deu aos Cathólicos.
\section{Theophania}
\begin{itemize}
\item {Grp. gram.:f.}
\end{itemize}
\begin{itemize}
\item {Proveniência:(Gr. \textunderscore theophania\textunderscore )}
\end{itemize}
Apparição ou revelação da divindade.
\section{Theophilanthropia}
\begin{itemize}
\item {Grp. gram.:f.}
\end{itemize}
\begin{itemize}
\item {Proveniência:(Do gr. \textunderscore theos\textunderscore  + \textunderscore philos\textunderscore  + \textunderscore anthropos\textunderscore )}
\end{itemize}
Seita religiosa, que professava o theísmo e se estabeleceu em França, vigorando desde 1796 a 1801 em que foi supprimida.--O seu culto cifrava-se em discursos de moral e hymnos a Deus e a todas as virtudes.
\section{Theophilanthrópico}
\begin{itemize}
\item {Grp. gram.:adj.}
\end{itemize}
Relativo á theophilanthropia.
\section{Theophilanthropo}
\begin{itemize}
\item {Grp. gram.:m.}
\end{itemize}
Sectário da theophilanthropia.
\section{Theophobia}
\begin{itemize}
\item {Grp. gram.:f.}
\end{itemize}
Estado ou qualidade de theóphobo.
\section{Theóphobo}
\begin{itemize}
\item {Grp. gram.:m.}
\end{itemize}
\begin{itemize}
\item {Proveniência:(Do gr. \textunderscore theos\textunderscore  + \textunderscore phobein\textunderscore )}
\end{itemize}
Aquelle que tem aversão a Deus ou ás coisas divinas.
\section{Theophrasta}
\begin{itemize}
\item {Grp. gram.:f.}
\end{itemize}
Gênero de plantas myrsíneas.
\section{Theopsia}
\begin{itemize}
\item {Grp. gram.:f.}
\end{itemize}
\begin{itemize}
\item {Proveniência:(Do gr. \textunderscore theos\textunderscore  + \textunderscore ops\textunderscore )}
\end{itemize}
Supposta apparição súbita de uma divindade.
\section{Theorema}
\begin{itemize}
\item {Proveniência:(Lat. \textunderscore theorema\textunderscore )}
\end{itemize}
Qualquer proposição que, para ser evidente, precisa de demonstração.
\section{Theoreticamente}
\begin{itemize}
\item {Grp. gram.:adv.}
\end{itemize}
\begin{itemize}
\item {Proveniência:(De \textunderscore theorético\textunderscore )}
\end{itemize}
O mesmo que \textunderscore theoricamente\textunderscore .
\section{Theorético}
\begin{itemize}
\item {Grp. gram.:adj.}
\end{itemize}
\begin{itemize}
\item {Proveniência:(Lat. \textunderscore theoreticus\textunderscore )}
\end{itemize}
O mesmo que \textunderscore theórico\textunderscore .
\section{Theoria}
\begin{itemize}
\item {Grp. gram.:f.}
\end{itemize}
\begin{itemize}
\item {Proveniência:(Lat. \textunderscore theoria\textunderscore )}
\end{itemize}
Conhecimento, limitado a princípios ou á especulação, sem passar á prática.
Conjunto de princípios fundamentaes de uma arte ou sciência.
Systema ou doutrina á cêrca dêsses princípios.
Hypóthese; noções geraes; utopia.
\section{Theoria}
\begin{itemize}
\item {Grp. gram.:f.}
\end{itemize}
\begin{itemize}
\item {Proveniência:(Gr. \textunderscore theoria\textunderscore )}
\end{itemize}
Deputação ou commissão que, na antiguidade, era enviada ou mandada em nome de uma cidade, para ir fazer sacrifícios aos deuses ou consultar um oráculo.--Há quem diga \textunderscore theória\textunderscore , como Filinto, XIV, 134, e XVI, 218.
\section{Teórica}
\begin{itemize}
\item {Grp. gram.:f.}
\end{itemize}
\begin{itemize}
\item {Proveniência:(De \textunderscore teórico\textunderscore )}
\end{itemize}
O mesmo que \textunderscore teoria\textunderscore ^1. Cf. \textunderscore Eufrosina\textunderscore , 143.
\section{Teoricamente}
\begin{itemize}
\item {Grp. gram.:adv.}
\end{itemize}
De modo teórico.
Segundo a teoria; abstraindo-se da prática.
\section{Teórico}
\begin{itemize}
\item {Grp. gram.:adj.}
\end{itemize}
\begin{itemize}
\item {Grp. gram.:M.}
\end{itemize}
\begin{itemize}
\item {Utilização:Fam.}
\end{itemize}
\begin{itemize}
\item {Proveniência:(Lat. \textunderscore theoricus\textunderscore )}
\end{itemize}
Relativo á teoria.
Indivíduo, que conhece cientificamente os princípios de uma arte.
Utopista.
\section{Teórida}
\begin{itemize}
\item {Grp. gram.:f.}
\end{itemize}
Nau, em que os teóros iam a Delphos.
(Cp. \textunderscore teóro\textunderscore )
\section{Teorismo}
\begin{itemize}
\item {Grp. gram.:m.}
\end{itemize}
Amor ás teorias.
\section{Teorista}
\begin{itemize}
\item {Grp. gram.:m.}
\end{itemize}
\begin{itemize}
\item {Proveniência:(De \textunderscore teoria\textunderscore )}
\end{itemize}
Aquele que conhece os princípios de uma ciência, mas que a não pratíca ou a não sabe praticar.
\section{Teorização}
\begin{itemize}
\item {Grp. gram.:f.}
\end{itemize}
Acto ou efeito de teorizar.
\section{Teorizar}
\begin{itemize}
\item {Grp. gram.:v. t.}
\end{itemize}
Expor teorias sôbre.
Reduzir a teorias; metodizar.
\section{Teóro}
\begin{itemize}
\item {Grp. gram.:m.}
\end{itemize}
\begin{itemize}
\item {Proveniência:(Gr. \textunderscore theoros\textunderscore )}
\end{itemize}
Cada um dos sacrificadores, que os Atenienses mandavam a Delphos, para que, em nome de Atenas, oferecessem a Apolo sacrifícios solenes.
Cada um daqueles que uma cidade enviava a Delphos, para consultarem o oráculo.
\section{Teose}
\begin{itemize}
\item {Grp. gram.:f.}
\end{itemize}
\begin{itemize}
\item {Proveniência:(Do gr. \textunderscore theos\textunderscore )}
\end{itemize}
Deificação; divinização.
\section{Teossofia}
\begin{itemize}
\item {Grp. gram.:f.}
\end{itemize}
\begin{itemize}
\item {Proveniência:(De \textunderscore teósofo\textunderscore )}
\end{itemize}
Suposta comunicação com a divindade, recebendo-se dela dons particulares, ou combatendo-lhe a influência por intermédio de gênios ou demónios.
\section{Teossófico}
\begin{itemize}
\item {Grp. gram.:adj.}
\end{itemize}
Relativo á teosofia.
\section{Teossofismo}
\begin{itemize}
\item {Grp. gram.:m.}
\end{itemize}
\begin{itemize}
\item {Proveniência:(De \textunderscore teosofia\textunderscore )}
\end{itemize}
Carácter das especulações teosóficas.
\section{Teóssofo}
\begin{itemize}
\item {Grp. gram.:m.}
\end{itemize}
\begin{itemize}
\item {Proveniência:(Gr. \textunderscore theosophos\textunderscore )}
\end{itemize}
Aquele que segue ou ensina as teorias teosóficas.
\section{Teotismo}
\begin{itemize}
\item {Grp. gram.:m.}
\end{itemize}
Doutrina filosófica, preconizada por Catharina Theot.
\section{Terapeuta}
\begin{itemize}
\item {Grp. gram.:m.}
\end{itemize}
\begin{itemize}
\item {Proveniência:(Gr. \textunderscore therapeutes\textunderscore )}
\end{itemize}
Aquele que escreve sôbre terapêutica.
O que exerce a Medicina.
Médico, clínico.
Designação dos monjes, entre os Judeus.
\section{Terapêutica}
\begin{itemize}
\item {Grp. gram.:f.}
\end{itemize}
\begin{itemize}
\item {Proveniência:(Lat. \textunderscore therapeutice\textunderscore )}
\end{itemize}
Parte da Medicina, que trata da escolha e administração dos meios de curar doenças, e da natureza dos remédios.
Tratamento das doenças.
\section{Terapêutico}
\begin{itemize}
\item {Grp. gram.:adj.}
\end{itemize}
\begin{itemize}
\item {Proveniência:(Gr. \textunderscore therapeutikos\textunderscore )}
\end{itemize}
Relativo á terapêutica.
\section{Terapia}
\begin{itemize}
\item {Grp. gram.:f.}
\end{itemize}
\begin{itemize}
\item {Proveniência:(Gr. \textunderscore therapeia\textunderscore )}
\end{itemize}
O mesmo que \textunderscore terapêutica\textunderscore .
\section{Teratólito}
\begin{itemize}
\item {Grp. gram.:m.}
\end{itemize}
\begin{itemize}
\item {Utilização:Miner.}
\end{itemize}
Silicato hidratado de alumina e cal.
\section{Teresa}
\begin{itemize}
\item {Grp. gram.:f.}
\end{itemize}
Pássaro do México, espécie de verdelhão.
\section{Téria}
\begin{itemize}
\item {Grp. gram.:f.}
\end{itemize}
\begin{itemize}
\item {Proveniência:(Do gr. \textunderscore ther\textunderscore )}
\end{itemize}
Gênero de insectos lepidópteros nocturnos.
\section{Teriacal}
\begin{itemize}
\item {Grp. gram.:adj.}
\end{itemize}
\begin{itemize}
\item {Proveniência:(Do lat. \textunderscore theriaca\textunderscore )}
\end{itemize}
Relativo á teriaga; que tem a virtude da teriaga.
\section{Teríaco}
\begin{itemize}
\item {Grp. gram.:m.}
\end{itemize}
Designação científica da teriaga.(V.teriaga)
\section{Teriacologia}
\begin{itemize}
\item {Grp. gram.:f.}
\end{itemize}
\begin{itemize}
\item {Proveniência:(Do gr. \textunderscore theriake\textunderscore  + \textunderscore logos\textunderscore )}
\end{itemize}
Tratado dos animaes venenosos.
\section{Teriaga}
\begin{itemize}
\item {Grp. gram.:f.}
\end{itemize}
\begin{itemize}
\item {Utilização:Fam.}
\end{itemize}
\begin{itemize}
\item {Utilização:Fig.}
\end{itemize}
\begin{itemize}
\item {Proveniência:(Lat. \textunderscore theriaca\textunderscore )}
\end{itemize}
Electuário antigo, que se supunha eficaz contra a mordedura de animaes venenosos.
Remédio caseiro.
Coisa muito amarga.
\section{Terionarca}
\begin{itemize}
\item {Grp. gram.:f.}
\end{itemize}
\begin{itemize}
\item {Utilização:Bot.}
\end{itemize}
\begin{itemize}
\item {Utilização:Ant.}
\end{itemize}
\begin{itemize}
\item {Proveniência:(Lat. \textunderscore therionarca\textunderscore )}
\end{itemize}
Espécie de erva, da qual se dizia têr a virtude de adormentar as serpentes.
\section{Teristro}
\begin{itemize}
\item {Grp. gram.:m.}
\end{itemize}
\begin{itemize}
\item {Utilização:Ant.}
\end{itemize}
\begin{itemize}
\item {Proveniência:(Lat. \textunderscore theristrum\textunderscore )}
\end{itemize}
Véu ligeiro, que as mulhéres traziam de verão.
\section{Termal}
\begin{itemize}
\item {Grp. gram.:adj.}
\end{itemize}
\begin{itemize}
\item {Proveniência:(De \textunderscore termas\textunderscore )}
\end{itemize}
Diz-se das águas medicinaes, cuja temperatura habitual excede 25° centígrados.
\section{Termalidade}
\begin{itemize}
\item {Grp. gram.:f.}
\end{itemize}
\begin{itemize}
\item {Proveniência:(De \textunderscore termal\textunderscore )}
\end{itemize}
Qualidade ou natureza das águas termaes.
\section{Termântico}
\begin{itemize}
\item {Grp. gram.:adj.}
\end{itemize}
\begin{itemize}
\item {Proveniência:(Gr. \textunderscore thermantikos\textunderscore )}
\end{itemize}
Excitante.
Que faz calor.
\section{Termas}
\begin{itemize}
\item {Grp. gram.:f. pl.}
\end{itemize}
\begin{itemize}
\item {Utilização:Ant.}
\end{itemize}
\begin{itemize}
\item {Proveniência:(Lat. \textunderscore thermae\textunderscore )}
\end{itemize}
Estabelecimento, apropriado para uso terapêutico das águas medicinaes quentes.
Águas termaes.
Edifício, entre os povos antigos, destinado ao uso dos banhos públicos.
\section{Termiatria}
\begin{itemize}
\item {Grp. gram.:f.}
\end{itemize}
\begin{itemize}
\item {Proveniência:(Do gr. \textunderscore therme\textunderscore  + \textunderscore iatreia\textunderscore )}
\end{itemize}
Parte da terapêutica, que se ocupa das águas termaes.
\section{Térmico}
\begin{itemize}
\item {Grp. gram.:adj.}
\end{itemize}
Relativo ás termas ou ao calor.
\section{Termidor}
\begin{itemize}
\item {Grp. gram.:m.}
\end{itemize}
\begin{itemize}
\item {Proveniência:(Do gr. \textunderscore therme\textunderscore  + \textunderscore doron\textunderscore )}
\end{itemize}
Undécimo mês do calendário da primeira república francesa, o qual começava a 19 de julho e acabava a 17 de agosto.
\section{Termo...}
\begin{itemize}
\item {Grp. gram.:pref.}
\end{itemize}
\begin{itemize}
\item {Proveniência:(Do gr. \textunderscore therme\textunderscore )}
\end{itemize}
(designativo de \textunderscore calor\textunderscore )
\section{Termobarométrico}
\begin{itemize}
\item {Grp. gram.:adj.}
\end{itemize}
Relativo ao termobarómetro.
\section{Termobarómetro}
\begin{itemize}
\item {Grp. gram.:m.}
\end{itemize}
Instrumento, em que se reúnem as propriedades do termómetro e do barómetro.
\section{Termocautério}
\begin{itemize}
\item {Grp. gram.:m.}
\end{itemize}
Acto de cauterizar, por meio do calor ou do fogo.(V.cautério)
\section{Termódota}
\begin{itemize}
\item {Grp. gram.:m.}
\end{itemize}
Aquele que distribuía água quente nos banhos públicos da Grécia.
\section{Termogêneo}
\begin{itemize}
\item {Grp. gram.:adj.}
\end{itemize}
\begin{itemize}
\item {Proveniência:(Do gr. \textunderscore therme\textunderscore  + \textunderscore genos\textunderscore )}
\end{itemize}
Que produz calor.
Diz-se especialmente do algodão, embebido em tintura de cápsico.
\section{Termogenia}
\begin{itemize}
\item {Grp. gram.:f.}
\end{itemize}
Qualidade de termogêneo.
\section{Termógrafo}
\begin{itemize}
\item {Grp. gram.:m.}
\end{itemize}
\begin{itemize}
\item {Proveniência:(Do gr. \textunderscore therme\textunderscore  + \textunderscore graphein\textunderscore )}
\end{itemize}
Maquinista, que regista as temperaturas.
\section{Termologia}
\begin{itemize}
\item {Grp. gram.:f.}
\end{itemize}
\begin{itemize}
\item {Proveniência:(Do gr. \textunderscore therme\textunderscore  + \textunderscore logos\textunderscore )}
\end{itemize}
Tratado á cêrca do calor.
\section{Termológico}
\begin{itemize}
\item {Grp. gram.:adj.}
\end{itemize}
Relativo á termologia.
\section{Theórica}
\begin{itemize}
\item {Grp. gram.:f.}
\end{itemize}
\begin{itemize}
\item {Proveniência:(De \textunderscore theórico\textunderscore )}
\end{itemize}
O mesmo que \textunderscore theoria\textunderscore ^1. Cf. \textunderscore Eufrosina\textunderscore , 143.
\section{Theoricamente}
\begin{itemize}
\item {Grp. gram.:adv.}
\end{itemize}
De modo theórico.
Segundo a theoria; abstrahindo-se da prática.
\section{Theórico}
\begin{itemize}
\item {Grp. gram.:adj.}
\end{itemize}
\begin{itemize}
\item {Grp. gram.:M.}
\end{itemize}
\begin{itemize}
\item {Utilização:Fam.}
\end{itemize}
\begin{itemize}
\item {Proveniência:(Lat. \textunderscore theoricus\textunderscore )}
\end{itemize}
Relativo á theoria.
Indivíduo, que conhece scientificamente os princípios de uma arte.
Utopista.
\section{Theórida}
\begin{itemize}
\item {Grp. gram.:f.}
\end{itemize}
Nau, em que os theóros iam a Delphos.
(Cp. \textunderscore theóro\textunderscore )
\section{Theorismo}
\begin{itemize}
\item {Grp. gram.:m.}
\end{itemize}
Amor ás theorias.
\section{Theorista}
\begin{itemize}
\item {Grp. gram.:m.}
\end{itemize}
\begin{itemize}
\item {Proveniência:(De \textunderscore theoria\textunderscore )}
\end{itemize}
Aquelle que conhece os princípios de uma sciência, mas que a não pratíca ou a não sabe praticar.
\section{Theorização}
\begin{itemize}
\item {Grp. gram.:f.}
\end{itemize}
Acto ou effeito de theorizar.
\section{Theorizar}
\begin{itemize}
\item {Grp. gram.:v. t.}
\end{itemize}
Expor theorias sôbre.
Reduzir a theorias; methodizar.
\section{Theóro}
\begin{itemize}
\item {Grp. gram.:m.}
\end{itemize}
\begin{itemize}
\item {Proveniência:(Gr. \textunderscore theoros\textunderscore )}
\end{itemize}
Cada um dos sacrificadores, que os Athenienses mandavam a Delphos, para que, em nome de Athenas, offerecessem a Apollo sacrifícios solennes.
Cada um daquelles que uma cidade enviava a Delphos, para consultarem o oráculo.
\section{Theose}
\begin{itemize}
\item {Grp. gram.:f.}
\end{itemize}
\begin{itemize}
\item {Proveniência:(Do gr. \textunderscore theos\textunderscore )}
\end{itemize}
Deificação; divinização.
\section{Theosophia}
\begin{itemize}
\item {fónica:so}
\end{itemize}
\begin{itemize}
\item {Grp. gram.:f.}
\end{itemize}
\begin{itemize}
\item {Proveniência:(De \textunderscore teósopho\textunderscore )}
\end{itemize}
Supposta communicação com a divindade, recebendo-se della dons particulares, ou combatendo-lhe a influência por intermédio de gênios ou demónios.
\section{Theosóphico}
\begin{itemize}
\item {fónica:só}
\end{itemize}
\begin{itemize}
\item {Grp. gram.:adj.}
\end{itemize}
Relativo á theosophia.
\section{Theosophismo}
\begin{itemize}
\item {fónica:so}
\end{itemize}
\begin{itemize}
\item {Grp. gram.:m.}
\end{itemize}
\begin{itemize}
\item {Proveniência:(De \textunderscore theosophia\textunderscore )}
\end{itemize}
Carácter das especulações theosóphicas.
\section{Theósopho}
\begin{itemize}
\item {fónica:so}
\end{itemize}
\begin{itemize}
\item {Grp. gram.:m.}
\end{itemize}
\begin{itemize}
\item {Proveniência:(Gr. \textunderscore theosophos\textunderscore )}
\end{itemize}
Aquelle que segue ou ensina as theorias theosóphicas.
\section{Theotismo}
\begin{itemize}
\item {Grp. gram.:m.}
\end{itemize}
Doutrina philosóphica, preconizada por Catharina Theot.
\section{Therapeuta}
\begin{itemize}
\item {Grp. gram.:m.}
\end{itemize}
\begin{itemize}
\item {Proveniência:(Gr. \textunderscore therapeutes\textunderscore )}
\end{itemize}
Aquelle que escreve sôbre therapêutica.
O que exerce a Medicina.
Médico, clínico.
Designação dos monjes, entre os Judeus.
\section{Therapêutica}
\begin{itemize}
\item {Grp. gram.:f.}
\end{itemize}
\begin{itemize}
\item {Proveniência:(Lat. \textunderscore therapeutice\textunderscore )}
\end{itemize}
Parte da Medicina, que trata da escolha e administração dos meios de curar doenças, e da natureza dos remédios.
Tratamento das doenças.
\section{Therapêutico}
\begin{itemize}
\item {Grp. gram.:adj.}
\end{itemize}
\begin{itemize}
\item {Proveniência:(Gr. \textunderscore therapeutikos\textunderscore )}
\end{itemize}
Relativo á therapêutica.
\section{Therapia}
\begin{itemize}
\item {Grp. gram.:f.}
\end{itemize}
\begin{itemize}
\item {Proveniência:(Gr. \textunderscore therapeia\textunderscore )}
\end{itemize}
O mesmo que \textunderscore therapêutica\textunderscore .
\section{Theratólitho}
\begin{itemize}
\item {Grp. gram.:m.}
\end{itemize}
\begin{itemize}
\item {Utilização:Miner.}
\end{itemize}
Silicato hydratado de alumina e cal.
\section{Theresa}
\begin{itemize}
\item {Grp. gram.:f.}
\end{itemize}
Pássaro do México, espécie de verdelhão.
\section{Théria}
\begin{itemize}
\item {Grp. gram.:f.}
\end{itemize}
\begin{itemize}
\item {Proveniência:(Do gr. \textunderscore ther\textunderscore )}
\end{itemize}
Gênero de insectos lepidópteros nocturnos.
\section{Theriacal}
\begin{itemize}
\item {Grp. gram.:adj.}
\end{itemize}
\begin{itemize}
\item {Proveniência:(Do lat. \textunderscore theriaca\textunderscore )}
\end{itemize}
Relativo á theriaga; que tem a virtude da theriaga.
\section{Theríaco}
\begin{itemize}
\item {Grp. gram.:m.}
\end{itemize}
Designação scientífica da theriaga.(V.theriaga)
\section{Theriacologia}
\begin{itemize}
\item {Grp. gram.:f.}
\end{itemize}
\begin{itemize}
\item {Proveniência:(Do gr. \textunderscore theriake\textunderscore  + \textunderscore logos\textunderscore )}
\end{itemize}
Tratado dos animaes venenosos.
\section{Theriaga}
\begin{itemize}
\item {Grp. gram.:f.}
\end{itemize}
\begin{itemize}
\item {Utilização:Fam.}
\end{itemize}
\begin{itemize}
\item {Utilização:Fig.}
\end{itemize}
\begin{itemize}
\item {Proveniência:(Lat. \textunderscore theriaca\textunderscore )}
\end{itemize}
Electuário antigo, que se suppunha efficaz contra a mordedura de animaes venenosos.
Remédio caseiro.
Coisa muito amarga.
\section{Therionarca}
\begin{itemize}
\item {Grp. gram.:f.}
\end{itemize}
\begin{itemize}
\item {Utilização:Bot.}
\end{itemize}
\begin{itemize}
\item {Utilização:Ant.}
\end{itemize}
\begin{itemize}
\item {Proveniência:(Lat. \textunderscore therionarca\textunderscore )}
\end{itemize}
Espécie de erva, da qual se dizia têr a virtude de adormentar as serpentes.
\section{Theristro}
\begin{itemize}
\item {Grp. gram.:m.}
\end{itemize}
\begin{itemize}
\item {Utilização:Ant.}
\end{itemize}
\begin{itemize}
\item {Proveniência:(Lat. \textunderscore theristrum\textunderscore )}
\end{itemize}
Véu ligeiro, que as mulhéres traziam de verão.
\section{Therma}
\begin{itemize}
\item {Grp. gram.:f.}
\end{itemize}
(V.thermas)
\section{Thermal}
\begin{itemize}
\item {Grp. gram.:adj.}
\end{itemize}
\begin{itemize}
\item {Proveniência:(De \textunderscore thermas\textunderscore )}
\end{itemize}
Diz-se das águas medicinaes, cuja temperatura habitual excede 25° centígrados.
\section{Thermalidade}
\begin{itemize}
\item {Grp. gram.:f.}
\end{itemize}
\begin{itemize}
\item {Proveniência:(De \textunderscore termal\textunderscore )}
\end{itemize}
Qualidade ou natureza das águas thermaes.
\section{Thermântico}
\begin{itemize}
\item {Grp. gram.:adj.}
\end{itemize}
\begin{itemize}
\item {Proveniência:(Gr. \textunderscore thermantikos\textunderscore )}
\end{itemize}
Excitante.
Que faz calor.
\section{Thermas}
\begin{itemize}
\item {Grp. gram.:f. pl.}
\end{itemize}
\begin{itemize}
\item {Utilização:Ant.}
\end{itemize}
\begin{itemize}
\item {Proveniência:(Lat. \textunderscore thermæ\textunderscore )}
\end{itemize}
Estabelecimento, apropriado para uso therapêutico das águas medicinaes quentes.
Águas thermaes.
Edifício, entre os povos antigos, destinado ao uso dos banhos públicos.
\section{Thermiatria}
\begin{itemize}
\item {Grp. gram.:f.}
\end{itemize}
\begin{itemize}
\item {Proveniência:(Do gr. \textunderscore therme\textunderscore  + \textunderscore iatreia\textunderscore )}
\end{itemize}
Parte da therapêutica, que se occupa das águas thermaes.
\section{Thérmico}
\begin{itemize}
\item {Grp. gram.:adj.}
\end{itemize}
Relativo ás thermas ou ao calor.
\section{Thermidor}
\begin{itemize}
\item {Grp. gram.:m.}
\end{itemize}
\begin{itemize}
\item {Proveniência:(Do gr. \textunderscore therme\textunderscore  + \textunderscore doron\textunderscore )}
\end{itemize}
Undécimo mês do calendário da primeira república francesa, o qual começava a 19 de julho e acabava a 17 de agosto.
\section{Thermo...}
\begin{itemize}
\item {Grp. gram.:pref.}
\end{itemize}
\begin{itemize}
\item {Proveniência:(Do gr. \textunderscore therme\textunderscore )}
\end{itemize}
(designativo de \textunderscore calor\textunderscore )
\section{Thermobarométrico}
\begin{itemize}
\item {Grp. gram.:adj.}
\end{itemize}
Relativo ao thermobarómetro.
\section{Thermobarómetro}
\begin{itemize}
\item {Grp. gram.:m.}
\end{itemize}
Instrumento, em que se reúnem as propriedades do thermómetro e do barómetro.
\section{Thermocautério}
\begin{itemize}
\item {Grp. gram.:m.}
\end{itemize}
Acto de cauterizar, por meio do calor ou do fogo.(V.cautério)
\section{Thermo-chímica}
\begin{itemize}
\item {Grp. gram.:f.}
\end{itemize}
Theoria dos phenómenos caloríficos, que acompanham os phenómenos chímicos.
\section{Thermo-chímico}
\begin{itemize}
\item {Grp. gram.:adj.}
\end{itemize}
Relativo á thermo-chímica.
\section{Thermódota}
\begin{itemize}
\item {Grp. gram.:m.}
\end{itemize}
Aquelle que distribuía água quente nos banhos públicos da Grécia.
\section{Thermo-dynâmica}
\begin{itemize}
\item {Grp. gram.:f.}
\end{itemize}
Sciência da fôrça produzida pelo calor.
\section{Thermo-dynâmico}
\begin{itemize}
\item {Grp. gram.:adj.}
\end{itemize}
Relativo á thermo-dynâmica.
\section{Thermo-electricidade}
\begin{itemize}
\item {Grp. gram.:f.}
\end{itemize}
Electricidade desenvolvida por mudança de temperatura.
\section{Thermo-eléctrico}
\begin{itemize}
\item {Grp. gram.:adj.}
\end{itemize}
Relativo á thermo-electricidade.
\section{Thermogêneo}
\begin{itemize}
\item {Grp. gram.:adj.}
\end{itemize}
\begin{itemize}
\item {Proveniência:(Do gr. \textunderscore therme\textunderscore  + \textunderscore genos\textunderscore )}
\end{itemize}
Que produz calor.
Diz-se especialmente do algodão, embebido em tintura de cápsico.
\section{Thermogenia}
\begin{itemize}
\item {Grp. gram.:f.}
\end{itemize}
Qualidade de thermogêneo.
\section{Thermógrapho}
\begin{itemize}
\item {Grp. gram.:m.}
\end{itemize}
\begin{itemize}
\item {Proveniência:(Do gr. \textunderscore therme\textunderscore  + \textunderscore graphein\textunderscore )}
\end{itemize}
Maquinista, que regista as temperaturas.
\section{Thermologia}
\begin{itemize}
\item {Grp. gram.:f.}
\end{itemize}
\begin{itemize}
\item {Proveniência:(Do gr. \textunderscore therme\textunderscore  + \textunderscore logos\textunderscore )}
\end{itemize}
Tratado á cêrca do calor.
\section{Thermológico}
\begin{itemize}
\item {Grp. gram.:adj.}
\end{itemize}
Relativo á thermologia.
\section{Termóforo}
\begin{itemize}
\item {Grp. gram.:m.}
\end{itemize}
\begin{itemize}
\item {Proveniência:(Do gr. \textunderscore therme\textunderscore  + \textunderscore phoros\textunderscore )}
\end{itemize}
Aparelho moderno, para a producção de calor nas maiores altitudes e latitudes. Cf. \textunderscore Jorn.-do-Comm.\textunderscore , do Rio, de 7-IX-901.
\section{Termometria}
\begin{itemize}
\item {Grp. gram.:f.}
\end{itemize}
\begin{itemize}
\item {Proveniência:(De \textunderscore termómetro\textunderscore )}
\end{itemize}
Medição do calor.
\section{Termométrico}
\begin{itemize}
\item {Grp. gram.:adj.}
\end{itemize}
Relativo á termometria.
\section{Termómetro}
\begin{itemize}
\item {Grp. gram.:m.}
\end{itemize}
\begin{itemize}
\item {Utilização:Fig.}
\end{itemize}
\begin{itemize}
\item {Proveniência:(Do gr. \textunderscore therme\textunderscore  + \textunderscore metron\textunderscore )}
\end{itemize}
Instrumento, para indicar os graus do calor ou do frio, num momento dado.
Indicação de um estado ou de certas condições físicas ou moraes.
\section{Termometógrafo}
\begin{itemize}
\item {Grp. gram.:m.}
\end{itemize}
\begin{itemize}
\item {Proveniência:(De \textunderscore termómetro\textunderscore  + gr. \textunderscore graphein\textunderscore )}
\end{itemize}
O mesmo que \textunderscore termógrafo\textunderscore .
\section{Termoscopia}
\begin{itemize}
\item {Grp. gram.:f.}
\end{itemize}
Medição do calor atmosférico.
(Cp. \textunderscore termoscópio\textunderscore )
\section{Termoscópico}
\begin{itemize}
\item {Grp. gram.:adj.}
\end{itemize}
Relativo á termoscopia.
\section{Termoscópio}
\begin{itemize}
\item {Grp. gram.:m.}
\end{itemize}
\begin{itemize}
\item {Proveniência:(Do gr. \textunderscore therme\textunderscore  + \textunderscore skopein\textunderscore )}
\end{itemize}
Instrumento, com que se avaliam as mais pequenas mudanças de temperatura.
\section{Termosifão}
\begin{itemize}
\item {Grp. gram.:m.}
\end{itemize}
\begin{itemize}
\item {Proveniência:(De \textunderscore termo...\textunderscore  + \textunderscore sifão\textunderscore )}
\end{itemize}
Sifão, com que se conduz calor para uma estufa ou para outro lugar.
\section{Termosfera}
\begin{itemize}
\item {Grp. gram.:f.}
\end{itemize}
\begin{itemize}
\item {Proveniência:(Do gr. \textunderscore therme\textunderscore  + \textunderscore sphaira\textunderscore )}
\end{itemize}
Aparelho aerostático, com que se aproveita a rarefacção do ar pelo aquecimento, e a produção de algum vapor de água, que, unido áquele ar, produz um gás que póde elevar o aparelho. Cf. \textunderscore Jorn.-do-Comm.\textunderscore , do Rio, de 18-VI-902.
\section{Termoterapia}
\begin{itemize}
\item {Grp. gram.:f.}
\end{itemize}
\begin{itemize}
\item {Utilização:Med.}
\end{itemize}
\begin{itemize}
\item {Proveniência:(Do gr. \textunderscore therme\textunderscore  + \textunderscore therapeia\textunderscore )}
\end{itemize}
Emprêgo terapêutico do calor.
\section{Tero}
\begin{itemize}
\item {Grp. gram.:m.}
\end{itemize}
\begin{itemize}
\item {Proveniência:(Do gr. \textunderscore thairos\textunderscore )}
\end{itemize}
Gênero de moluscos gasterópodes.--A fórma \textunderscore taíra\textunderscore , adoptada por alguns zoologistas, é incorrecta.
\section{Termotropismo}
\begin{itemize}
\item {Grp. gram.:m.}
\end{itemize}
\begin{itemize}
\item {Utilização:Med.}
\end{itemize}
\begin{itemize}
\item {Proveniência:(Do gr. \textunderscore therme\textunderscore  + \textunderscore trepein\textunderscore )}
\end{itemize}
Propriedade, que o protoplasma tem, de reagir ao calor.
\section{Tese}
\begin{itemize}
\item {Grp. gram.:f.}
\end{itemize}
\begin{itemize}
\item {Utilização:Ext.}
\end{itemize}
\begin{itemize}
\item {Proveniência:(Lat. \textunderscore thesis\textunderscore )}
\end{itemize}
Proposição, que se apresenta para sêr defendida, se fôr impugnada.
Proposição que nas escolas superiores, é formulada para sêr defendida em público.
A discussão da própria tese.
\section{Teseias}
\begin{itemize}
\item {Grp. gram.:f. pl.}
\end{itemize}
Antigas festas gregas em honra de Theseu.
\section{Tésio}
\begin{itemize}
\item {Grp. gram.:m.}
\end{itemize}
\begin{itemize}
\item {Proveniência:(Lat. \textunderscore thesium\textunderscore )}
\end{itemize}
Gênero de plantas santaláceas.
\section{Tesmófora}
\begin{itemize}
\item {Grp. gram.:adj. f.}
\end{itemize}
\begin{itemize}
\item {Proveniência:(Lat. \textunderscore thesmophora\textunderscore )}
\end{itemize}
Dizia-se de Ceres, considerada como legisladora.
\section{Tesmofórias}
\begin{itemize}
\item {Grp. gram.:f. pl.}
\end{itemize}
Antigas festas gregas em honra da Ceres tesmófora.
\section{Tesmóteta}
\begin{itemize}
\item {Grp. gram.:m.}
\end{itemize}
Cada um dos magistrados que, em Atenas, estavam encarregados da guarda e observação das leis. Cf. Latino, \textunderscore Or. da Corôa\textunderscore , 25.
\section{Tesoirado}
\begin{itemize}
\item {Grp. gram.:m.}
\end{itemize}
\begin{itemize}
\item {Proveniência:(De \textunderscore tesoiro\textunderscore )}
\end{itemize}
Cargo de tesoireiro.
\section{Tesoiraria}
\begin{itemize}
\item {Grp. gram.:f.}
\end{itemize}
\begin{itemize}
\item {Proveniência:(De \textunderscore tesoiro\textunderscore )}
\end{itemize}
O mesmo que \textunderscore tesoirado\textunderscore .
Casa ou lugar, onde se guarda ou administra o tesoiro público.
Repartição, onde funciona o tesoireiro.
Escritório de companhia ou casa bancária, em que se realizam transacções monetárias.
\section{Tesoireiro}
\begin{itemize}
\item {Grp. gram.:m.}
\end{itemize}
\begin{itemize}
\item {Proveniência:(Do lat. \textunderscore thesaurarius\textunderscore )}
\end{itemize}
Guarda de tesoiro.
Empregado superior da administração do tesoiro público.
Aquele que é encarregado das operações monetárias numa casa bancária, companhia, associação, etc.
\section{Tesoiro}
\begin{itemize}
\item {Grp. gram.:m.}
\end{itemize}
\begin{itemize}
\item {Utilização:Fig.}
\end{itemize}
\begin{itemize}
\item {Proveniência:(Do lat. \textunderscore thesaurus\textunderscore )}
\end{itemize}
Grande porção de dinheiro ou de objectos preciosos.
Lugar, onde se guardam os rendimentos do Estado.
Erário.
Lugar, onde se guardam objectos preciosos.
Objecto ou objectos preciosos, que estavam escondidos, e que se descobriram casualmente.
Ministério das Finanças.
Repositório.
Objecto de grande estimação.
Coisa ou pessôa de grande valia.
Riqueza.
\section{Tesourado}
\begin{itemize}
\item {Grp. gram.:m.}
\end{itemize}
\begin{itemize}
\item {Proveniência:(De \textunderscore tesouro\textunderscore )}
\end{itemize}
Cargo de tesoureiro.
\section{Tesouraria}
\begin{itemize}
\item {Grp. gram.:f.}
\end{itemize}
\begin{itemize}
\item {Proveniência:(De \textunderscore tesouro\textunderscore )}
\end{itemize}
O mesmo que \textunderscore tesourado\textunderscore .
Casa ou lugar, onde se guarda ou administra o tesouro público.
Repartição, onde funciona o tesoureiro.
Escritório de companhia ou casa bancária, em que se realizam transacções monetárias.
\section{Tesoureiro}
\begin{itemize}
\item {Grp. gram.:m.}
\end{itemize}
\begin{itemize}
\item {Proveniência:(Do lat. \textunderscore thesaurarius\textunderscore )}
\end{itemize}
Guarda de tesouro.
Empregado superior da administração do tesouro público.
Aquele que é encarregado das operações monetárias numa casa bancária, companhia, associação, etc.
\section{Tesouro}
\begin{itemize}
\item {Grp. gram.:m.}
\end{itemize}
\begin{itemize}
\item {Utilização:Fig.}
\end{itemize}
\begin{itemize}
\item {Proveniência:(Do lat. \textunderscore thesaurus\textunderscore )}
\end{itemize}
Grande porção de dinheiro ou de objectos preciosos.
Lugar, onde se guardam os rendimentos do Estado.
Erário.
Lugar, onde se guardam objectos preciosos.
Objecto ou objectos preciosos, que estavam escondidos, e que se descobriram casualmente.
Ministério das Finanças.
Repositório.
Objecto de grande estimação.
Coisa ou pessôa de grande valia.
Riqueza.
\section{Tessálico}
\begin{itemize}
\item {Grp. gram.:adj.}
\end{itemize}
\begin{itemize}
\item {Proveniência:(Lat. \textunderscore thessalicus\textunderscore )}
\end{itemize}
Relativo á Thessália.
\section{Tessálio}
\begin{itemize}
\item {Grp. gram.:adj.}
\end{itemize}
\begin{itemize}
\item {Grp. gram.:M.}
\end{itemize}
\begin{itemize}
\item {Proveniência:(Lat. \textunderscore thessalius\textunderscore )}
\end{itemize}
O mesmo que \textunderscore tessálico\textunderscore .
Habitante de Thessália.
\section{Téssalo}
\begin{itemize}
\item {Grp. gram.:adj.}
\end{itemize}
\begin{itemize}
\item {Grp. gram.:M.}
\end{itemize}
\begin{itemize}
\item {Proveniência:(Lat. \textunderscore thessalus\textunderscore )}
\end{itemize}
O mesmo que \textunderscore tessálico\textunderscore .
Habitante da Thessália.
\section{Tessalonicense}
\begin{itemize}
\item {Grp. gram.:adj.}
\end{itemize}
\begin{itemize}
\item {Grp. gram.:M.}
\end{itemize}
\begin{itemize}
\item {Proveniência:(Lat. \textunderscore thessalonicensis\textunderscore )}
\end{itemize}
Relativo a Thessalonica.
Habitante de Thessalonica.
\section{Tétis}
\begin{itemize}
\item {Grp. gram.:f.}
\end{itemize}
\begin{itemize}
\item {Proveniência:(De \textunderscore Thetis\textunderscore , n. p.)}
\end{itemize}
Gênero de moluscos bivalves.
Nome de um pequeno planeta, entre Marte e Júpiter.
\section{Teurgia}
\begin{itemize}
\item {fónica:te-ur}
\end{itemize}
\begin{itemize}
\item {Grp. gram.:f.}
\end{itemize}
\begin{itemize}
\item {Proveniência:(Lat. \textunderscore theurgia\textunderscore )}
\end{itemize}
Espécie de magia.
Arte de fazer milagres.
\section{Teúrgico}
\begin{itemize}
\item {Grp. gram.:adj.}
\end{itemize}
\begin{itemize}
\item {Proveniência:(Lat. \textunderscore theurgicus\textunderscore )}
\end{itemize}
Relativo á teurgia.
\section{Teurgismo}
\begin{itemize}
\item {fónica:te-ur}
\end{itemize}
\begin{itemize}
\item {Grp. gram.:m.}
\end{itemize}
Instituição e doutrina dos Teúrgos.
\section{Teurgista}
\begin{itemize}
\item {Grp. gram.:m.  e  f.}
\end{itemize}
Pessôa, que trata ou se ocupa de teurgia.
\section{Teúrgo}
\begin{itemize}
\item {Grp. gram.:m.}
\end{itemize}
\begin{itemize}
\item {Proveniência:(Lat. \textunderscore theurgus\textunderscore )}
\end{itemize}
Aquele que pratíca a teurgia.
\section{Tevécia}
\begin{itemize}
\item {Grp. gram.:f.}
\end{itemize}
\begin{itemize}
\item {Proveniência:(De \textunderscore Thevet\textunderscore , n. p.)}
\end{itemize}
Gênero de plantas apocíneas.
\section{Tevenótia}
\begin{itemize}
\item {Grp. gram.:f.}
\end{itemize}
\begin{itemize}
\item {Proveniência:(De \textunderscore Thevenot\textunderscore  n. p.)}
\end{itemize}
Gênero de plantas, da fam. das compostas.
\section{Thermo-magnético}
\begin{itemize}
\item {Grp. gram.:adj.}
\end{itemize}
Relativo ao thermo-magnetismo.
\section{Thermo-magnetismo}
\begin{itemize}
\item {Grp. gram.:m.}
\end{itemize}
Magnetismo desenvolvido pelo calor.
\section{Thermo-manómetro}
\begin{itemize}
\item {Grp. gram.:m.}
\end{itemize}
Espécie de thermómetro, para medir temperaturas elevadas.
\section{Thermo-mecânica}
\begin{itemize}
\item {Grp. gram.:f.}
\end{itemize}
Mecânica do calórico.
\section{Thermo-mecânico}
\begin{itemize}
\item {Grp. gram.:adj.}
\end{itemize}
Relativo á thermo-mecânica.
\section{Thermometria}
\begin{itemize}
\item {Grp. gram.:f.}
\end{itemize}
\begin{itemize}
\item {Proveniência:(De \textunderscore thermómetro\textunderscore )}
\end{itemize}
Medição do calor.
\section{Thermométrico}
\begin{itemize}
\item {Grp. gram.:adj.}
\end{itemize}
Relativo á thermometria.
\section{Thermometrização}
\begin{itemize}
\item {Grp. gram.:f.}
\end{itemize}
Observação dos graus de calor pelo thermómetro.
\section{Thermómetro}
\begin{itemize}
\item {Grp. gram.:m.}
\end{itemize}
\begin{itemize}
\item {Utilização:Fig.}
\end{itemize}
\begin{itemize}
\item {Proveniência:(Do gr. \textunderscore therme\textunderscore  + \textunderscore metron\textunderscore )}
\end{itemize}
Instrumento, para indicar os graus do calor ou do frio, num momento dado.
Indicação de um estado ou de certas condições phýsicas ou moraes.
\section{Thermometógrapho}
\begin{itemize}
\item {Grp. gram.:m.}
\end{itemize}
\begin{itemize}
\item {Proveniência:(De \textunderscore thermómetro\textunderscore  + gr. \textunderscore graphein\textunderscore )}
\end{itemize}
O mesmo que \textunderscore thermógrapho\textunderscore .
\section{Thermo-mineral}
\begin{itemize}
\item {Grp. gram.:adj.}
\end{itemize}
Relativo ás águas mineraes quentes.
\section{Thermo-multiplicador}
\begin{itemize}
\item {Grp. gram.:m.}
\end{itemize}
Maquinismo termométrico muito sensível.
\section{Thermóphoro}
\begin{itemize}
\item {Grp. gram.:m.}
\end{itemize}
\begin{itemize}
\item {Proveniência:(Do gr. \textunderscore therme\textunderscore  + \textunderscore phoros\textunderscore )}
\end{itemize}
Apparelho moderno, para a producção de calor nas maiores altitudes e latitudes. Cf. \textunderscore Jorn.-do-Comm.\textunderscore , do Rio, de 7-IX-901.
\section{Thermoscopia}
\begin{itemize}
\item {Grp. gram.:f.}
\end{itemize}
Medição do calor atmosphérico.
(Cp. \textunderscore thermoscópio\textunderscore )
\section{Thermoscópico}
\begin{itemize}
\item {Grp. gram.:adj.}
\end{itemize}
Relativo á thermoscopia.
\section{Thermoscópio}
\begin{itemize}
\item {Grp. gram.:m.}
\end{itemize}
\begin{itemize}
\item {Proveniência:(Do gr. \textunderscore therme\textunderscore  + \textunderscore skopein\textunderscore )}
\end{itemize}
Instrumento, com que se avaliam as mais pequenas mudanças de temperatura.
\section{Thermosiphão}
\begin{itemize}
\item {Grp. gram.:m.}
\end{itemize}
\begin{itemize}
\item {Proveniência:(De \textunderscore thermo...\textunderscore  + \textunderscore siphão\textunderscore )}
\end{itemize}
Siphão, com que se conduz calor para uma estufa ou para outro lugar.
\section{Thermosphera}
\begin{itemize}
\item {Grp. gram.:f.}
\end{itemize}
\begin{itemize}
\item {Proveniência:(Do gr. \textunderscore therme\textunderscore  + \textunderscore sphaira\textunderscore )}
\end{itemize}
Apparelho aerostático, com que se aproveita a rarefacção do ar pelo aquecimento, e a producção de algum vapor de água, que, unido áquelle ar, produz um gás que póde elevar o apparelho. Cf. \textunderscore Jorn.-do-Comm.\textunderscore , do Rio, de 18-VI-902.
\section{Thermotherapia}
\begin{itemize}
\item {Grp. gram.:f.}
\end{itemize}
\begin{itemize}
\item {Utilização:Med.}
\end{itemize}
\begin{itemize}
\item {Proveniência:(Do gr. \textunderscore therme\textunderscore  + \textunderscore therapeia\textunderscore )}
\end{itemize}
Emprêgo therapêutico do calor.
\section{Thero}
\begin{itemize}
\item {Grp. gram.:m.}
\end{itemize}
\begin{itemize}
\item {Proveniência:(Do gr. \textunderscore thairos\textunderscore )}
\end{itemize}
Gênero de molluscos gasterópodes.--A fórma \textunderscore thaíra\textunderscore , adoptada por alguns zoologistas, é incorrecta.
\section{Thermotropismo}
\begin{itemize}
\item {Grp. gram.:m.}
\end{itemize}
\begin{itemize}
\item {Utilização:Med.}
\end{itemize}
\begin{itemize}
\item {Proveniência:(Do gr. \textunderscore therme\textunderscore  + \textunderscore trepein\textunderscore )}
\end{itemize}
Propriedade, que o protoplasma tem, de reagir ao calor.
\section{These}
\begin{itemize}
\item {Grp. gram.:f.}
\end{itemize}
\begin{itemize}
\item {Utilização:Ext.}
\end{itemize}
\begin{itemize}
\item {Proveniência:(Lat. \textunderscore thesis\textunderscore )}
\end{itemize}
Proposição, que se apresenta para sêr defendida, se fôr impugnada.
Proposição que nas escolas superiores, é formulada para sêr defendida em público.
A discussão da própria these.
\section{Theseias}
\begin{itemize}
\item {Grp. gram.:f. pl.}
\end{itemize}
Antigas festas gregas em honra de Theseu.
\section{Thésio}
\begin{itemize}
\item {Grp. gram.:m.}
\end{itemize}
\begin{itemize}
\item {Proveniência:(Lat. \textunderscore thesium\textunderscore )}
\end{itemize}
Gênero de plantas santaláceas.
\section{Thesmóphora}
\begin{itemize}
\item {Grp. gram.:adj. f.}
\end{itemize}
\begin{itemize}
\item {Proveniência:(Lat. \textunderscore thesmophora\textunderscore )}
\end{itemize}
Dizia-se de Ceres, considerada como legisladora.
\section{Tesmophórias}
\begin{itemize}
\item {Grp. gram.:f. pl.}
\end{itemize}
Antigas festas gregas em honra da Ceres thesmóphora.
\section{Thesmótheta}
\begin{itemize}
\item {Grp. gram.:m.}
\end{itemize}
Cada um dos magistrados que, em Athenas, estavam encarregados da guarda e observação das leis. Cf. Latino, \textunderscore Or. da Corôa\textunderscore , 25.
\section{Thesoirado}
\begin{itemize}
\item {Grp. gram.:m.}
\end{itemize}
\begin{itemize}
\item {Proveniência:(De \textunderscore thesoiro\textunderscore )}
\end{itemize}
Cargo de thesoireiro.
\section{Thesoiraria}
\begin{itemize}
\item {Grp. gram.:f.}
\end{itemize}
\begin{itemize}
\item {Proveniência:(De \textunderscore thesoiro\textunderscore )}
\end{itemize}
O mesmo que \textunderscore thesoirado\textunderscore .
Casa ou lugar, onde se guarda ou administra o thesoiro público.
Repartição, onde funcciona o thesoireiro.
Escritório de companhia ou casa bancária, em que se realizam transacções monetárias.
\section{Thesoireiro}
\begin{itemize}
\item {Grp. gram.:m.}
\end{itemize}
\begin{itemize}
\item {Proveniência:(Do lat. \textunderscore thesaurarius\textunderscore )}
\end{itemize}
Guarda de thesoiro.
Empregado superior da administração do thesoiro público.
Aquelle que é encarregado das operações monetárias numa casa bancária, companhia, associação, etc.
\section{Thesoiro}
\begin{itemize}
\item {Grp. gram.:m.}
\end{itemize}
\begin{itemize}
\item {Utilização:Fig.}
\end{itemize}
\begin{itemize}
\item {Proveniência:(Do lat. \textunderscore thesaurus\textunderscore )}
\end{itemize}
Grande porção de dinheiro ou de objectos preciosos.
Lugar, onde se guardam os rendimentos do Estado.
Erário.
Lugar, onde se guardam objectos preciosos.
Objecto ou objectos preciosos, que estavam escondidos, e que se descobriram casualmente.
Ministério das Finanças.
Repositório.
Objecto de grande estimação.
Coisa ou pessôa de grande valia.
Riqueza.
\section{Thessálico}
\begin{itemize}
\item {Grp. gram.:adj.}
\end{itemize}
\begin{itemize}
\item {Proveniência:(Lat. \textunderscore thessalicus\textunderscore )}
\end{itemize}
Relativo á Thessália.
\section{Thessálio}
\begin{itemize}
\item {Grp. gram.:adj.}
\end{itemize}
\begin{itemize}
\item {Grp. gram.:M.}
\end{itemize}
\begin{itemize}
\item {Proveniência:(Lat. \textunderscore thessalius\textunderscore )}
\end{itemize}
O mesmo que \textunderscore thessálico\textunderscore .
Habitante de Thessália.
\section{Théssalo}
\begin{itemize}
\item {Grp. gram.:adj.}
\end{itemize}
\begin{itemize}
\item {Grp. gram.:M.}
\end{itemize}
\begin{itemize}
\item {Proveniência:(Lat. \textunderscore thessalus\textunderscore )}
\end{itemize}
O mesmo que \textunderscore thessálico\textunderscore .
Habitante da Thessália.
\section{Thessalonicense}
\begin{itemize}
\item {Grp. gram.:adj.}
\end{itemize}
\begin{itemize}
\item {Grp. gram.:M.}
\end{itemize}
\begin{itemize}
\item {Proveniência:(Lat. \textunderscore thessalonicensis\textunderscore )}
\end{itemize}
Relativo a Thessalonica.
Habitante de Thessalonica.
\section{Thétis}
\begin{itemize}
\item {Grp. gram.:f.}
\end{itemize}
\begin{itemize}
\item {Proveniência:(De \textunderscore Thetis\textunderscore , n. p.)}
\end{itemize}
Gênero de molluscos bivalves.
Nome de um pequeno planeta, entre Marte e Júpiter.
\section{Theurgia}
\begin{itemize}
\item {Grp. gram.:f.}
\end{itemize}
\begin{itemize}
\item {Proveniência:(Lat. \textunderscore theurgia\textunderscore )}
\end{itemize}
Espécie de magia.
Arte de fazer milagres.
\section{Theúrgico}
\begin{itemize}
\item {Grp. gram.:adj.}
\end{itemize}
\begin{itemize}
\item {Proveniência:(Lat. \textunderscore theurgicus\textunderscore )}
\end{itemize}
Relativo á theurgia.
\section{Theurgismo}
\begin{itemize}
\item {Grp. gram.:m.}
\end{itemize}
Instituição e doutrina dos Theúrgos.
\section{Theurgista}
\begin{itemize}
\item {Grp. gram.:m.  e  f.}
\end{itemize}
Pessôa, que trata ou se occupa de theurgia.
\section{Theúrgo}
\begin{itemize}
\item {Grp. gram.:m.}
\end{itemize}
\begin{itemize}
\item {Proveniência:(Lat. \textunderscore theurgus\textunderscore )}
\end{itemize}
Aquelle que pratíca a theurgia.
\section{Thevécia}
\begin{itemize}
\item {Grp. gram.:f.}
\end{itemize}
\begin{itemize}
\item {Proveniência:(De \textunderscore Thevet\textunderscore , n. p.)}
\end{itemize}
Gênero de plantas apocýneas.
\section{Thevenótia}
\begin{itemize}
\item {Grp. gram.:f.}
\end{itemize}
\begin{itemize}
\item {Proveniência:(De \textunderscore Thevenot\textunderscore  n. p.)}
\end{itemize}
Gênero de plantas, da fam. das compostas.
\section{Thibáudia}
\begin{itemize}
\item {Grp. gram.:f.}
\end{itemize}
\begin{itemize}
\item {Proveniência:(De \textunderscore Thibaud\textunderscore , n. p.)}
\end{itemize}
Gênero de plantas ericáceas.
\section{Thigenol}
\begin{itemize}
\item {Grp. gram.:m.}
\end{itemize}
Medicamento, succedâneo do ichthiol.
\section{Thimbérgia}
\begin{itemize}
\item {Grp. gram.:f.}
\end{itemize}
\begin{itemize}
\item {Proveniência:(De \textunderscore Thimberg\textunderscore , n. p.)}
\end{itemize}
Gênero de plantas de jardim.
\section{Thinócoro}
\begin{itemize}
\item {Grp. gram.:m.}
\end{itemize}
Gênero de aves gallináceas.
\section{Thiocole}
\begin{itemize}
\item {Grp. gram.:m.}
\end{itemize}
Medicamento, empregado especialmente contra a tuberculose.
\section{Thiónico}
\begin{itemize}
\item {Grp. gram.:adj.}
\end{itemize}
\begin{itemize}
\item {Proveniência:(Do gr. \textunderscore theion\textunderscore )}
\end{itemize}
Relativo ao enxôfre e aos seus compostos.
\section{Thlaspídeas}
\begin{itemize}
\item {Grp. gram.:f. pl.}
\end{itemize}
Tríbo de plantas crucíferas, que tem por typo o thláspio.
\section{Thláspio}
\begin{itemize}
\item {Grp. gram.:m.}
\end{itemize}
\begin{itemize}
\item {Proveniência:(Do gr. \textunderscore thlaspis\textunderscore )}
\end{itemize}
Gênero de plantas crucíferas.
\section{Thlipsencephalia}
\begin{itemize}
\item {Grp. gram.:f.}
\end{itemize}
Estado ou qualidade de thlipsencéphalo.
\section{Thlipsencéphalo}
\begin{itemize}
\item {Grp. gram.:m.}
\end{itemize}
\begin{itemize}
\item {Proveniência:(Do gr. \textunderscore thlipsis\textunderscore  + \textunderscore enkephalon\textunderscore )}
\end{itemize}
Monstro, cujo cérebro está desfigurado por effeito de compressão.
\section{Thlipsia}
\begin{itemize}
\item {Grp. gram.:f.}
\end{itemize}
\begin{itemize}
\item {Utilização:Med.}
\end{itemize}
\begin{itemize}
\item {Proveniência:(Do gr. \textunderscore thlipsis\textunderscore )}
\end{itemize}
Compressão dos vasos orgânicos por uma causa externa.
\section{Thomismo}
\begin{itemize}
\item {Grp. gram.:m.}
\end{itemize}
\begin{itemize}
\item {Proveniência:(De \textunderscore Thomás\textunderscore , n. p.)}
\end{itemize}
Doutrina theológica e philosóphica de San-Thomás de Aquino.
\section{Thomista}
\begin{itemize}
\item {Grp. gram.:adj.}
\end{itemize}
\begin{itemize}
\item {Grp. gram.:M.}
\end{itemize}
Relativo ao thomismo.
Sectário do thomismo.
\section{Thomístico}
\begin{itemize}
\item {Grp. gram.:adj.}
\end{itemize}
\begin{itemize}
\item {Proveniência:(De \textunderscore thomista\textunderscore )}
\end{itemize}
Relativo a San-Thomás ou á sua doutrina.
\section{Thoracêntese}
\begin{itemize}
\item {Grp. gram.:f.}
\end{itemize}
(V.thoracocêntese)
\section{Thoracete}
\begin{itemize}
\item {Grp. gram.:m.}
\end{itemize}
Pequeno thórax.
(Dem. de \textunderscore thórax\textunderscore )
\section{Thorácico}
\begin{itemize}
\item {Grp. gram.:adj.}
\end{itemize}
\begin{itemize}
\item {Grp. gram.:M. pl.}
\end{itemize}
\begin{itemize}
\item {Proveniência:(Gr. \textunderscore thorakikos\textunderscore )}
\end{itemize}
Relativo ao thórax.
Ordem de peixes ósseos.
Família de coleópteros.
\section{Thoracocêntese}
\begin{itemize}
\item {Grp. gram.:f.}
\end{itemize}
\begin{itemize}
\item {Proveniência:(Do gr. \textunderscore thorax\textunderscore , \textunderscore torakos\textunderscore  + \textunderscore kentesis\textunderscore )}
\end{itemize}
Operação cirúrgica, em que se abrem as paredes do thórax, para dar saída a um líquido accumulado na cavidade pleural.
\section{Thóraco-facial}
\begin{itemize}
\item {Grp. gram.:adj.}
\end{itemize}
\begin{itemize}
\item {Utilização:Anat.}
\end{itemize}
Diz-se de um músculo, que se estende do peito á cara.
\section{Thoracometria}
\begin{itemize}
\item {Grp. gram.:f.}
\end{itemize}
\begin{itemize}
\item {Proveniência:(Do gr. \textunderscore thorax\textunderscore  + \textunderscore metron\textunderscore )}
\end{itemize}
Mensuração do thórax.
\section{Thoracométrico}
\begin{itemize}
\item {Grp. gram.:adj.}
\end{itemize}
Relativo á thoracometria.
\section{Thoracópago}
\begin{itemize}
\item {Grp. gram.:m.}
\end{itemize}
\begin{itemize}
\item {Utilização:Terat.}
\end{itemize}
\begin{itemize}
\item {Proveniência:(Do gr. \textunderscore thorax\textunderscore  + \textunderscore pagein\textunderscore )}
\end{itemize}
Monstro duplo, formado de dois indivíduos, ligados pelo thórax.
\section{Thoracóphoro}
\begin{itemize}
\item {Grp. gram.:m.}
\end{itemize}
\begin{itemize}
\item {Proveniência:(Do gr. \textunderscore thorax\textunderscore  + \textunderscore phoros\textunderscore )}
\end{itemize}
Gênero de insectos coleópteros heterómeros.
\section{Thoracoplastia}
\begin{itemize}
\item {Grp. gram.:f.}
\end{itemize}
\begin{itemize}
\item {Utilização:Med.}
\end{itemize}
\begin{itemize}
\item {Proveniência:(Do gr. \textunderscore thorax\textunderscore  + \textunderscore plassein\textunderscore )}
\end{itemize}
Modificação cirúrgica da conformação do thórax.
\section{Thoracoscopia}
\begin{itemize}
\item {Grp. gram.:f.}
\end{itemize}
\begin{itemize}
\item {Utilização:Med.}
\end{itemize}
\begin{itemize}
\item {Proveniência:(Do gr. \textunderscore thorax\textunderscore  + \textunderscore skopein\textunderscore )}
\end{itemize}
Observação do peito.
\section{Thoracotomia}
\begin{itemize}
\item {Grp. gram.:f.}
\end{itemize}
\begin{itemize}
\item {Utilização:Med.}
\end{itemize}
\begin{itemize}
\item {Proveniência:(Do gr. \textunderscore thorax\textunderscore  + \textunderscore tome\textunderscore )}
\end{itemize}
Acto cirúrgico de abrir o thórax.
\section{Thoracozoário}
\begin{itemize}
\item {Grp. gram.:adj.}
\end{itemize}
\begin{itemize}
\item {Utilização:Zool.}
\end{itemize}
\begin{itemize}
\item {Proveniência:(Do gr. \textunderscore thorax\textunderscore  + \textunderscore zoon\textunderscore )}
\end{itemize}
Diz-se dos animaes, cujos órgãos respiratórios adquiriram grande desenvolvimento.
\section{Thoradelpho}
\begin{itemize}
\item {Grp. gram.:m.}
\end{itemize}
\begin{itemize}
\item {Utilização:Terat.}
\end{itemize}
\begin{itemize}
\item {Proveniência:(Do gr. \textunderscore thorax\textunderscore  + \textunderscore adelphos\textunderscore )}
\end{itemize}
Monstro duplo monocéphalo, formado de dois indivíduos, separados, do umbigo para baixo, e confundidos daí para cima.
\section{Thórax}
\begin{itemize}
\item {Grp. gram.:m.}
\end{itemize}
\begin{itemize}
\item {Proveniência:(Lat. \textunderscore thorax\textunderscore )}
\end{itemize}
Peito; cavidade do peito.
Segmento intermédio do corpo dos insectos.
\section{Thórea}
\begin{itemize}
\item {Grp. gram.:f.}
\end{itemize}
Gênero de plantas aristolóchias.
\section{Thória}
\begin{itemize}
\item {Grp. gram.:adj. f.}
\end{itemize}
\begin{itemize}
\item {Proveniência:(Lat. \textunderscore thoria\textunderscore )}
\end{itemize}
Diz-se de uma lei agrária, de que foi autor o tribuno Thório Balbo, em Roma.
\section{Thorianito}
\begin{itemize}
\item {Grp. gram.:m.}
\end{itemize}
Mineral, descoberto há pouco em Ceilão e que é uma fonte importante de rádio.
\section{Thorínio}
\begin{itemize}
\item {Grp. gram.:m.}
\end{itemize}
\begin{itemize}
\item {Proveniência:(De \textunderscore Thor\textunderscore , n. p.)}
\end{itemize}
Metal em pó, escuro ou terroso.
\section{Thório}
\begin{itemize}
\item {Grp. gram.:m.}
\end{itemize}
\begin{itemize}
\item {Proveniência:(De \textunderscore Thor\textunderscore , n. p.)}
\end{itemize}
Metal em pó, escuro ou terroso.
\section{Thorita}
\begin{itemize}
\item {Grp. gram.:f.}
\end{itemize}
O mesmo que \textunderscore thorite\textunderscore .
\section{Thorite}
\begin{itemize}
\item {Grp. gram.:f.}
\end{itemize}
Mineral, de que se extrahiu o thório.
\section{Thorito}
\begin{itemize}
\item {Grp. gram.:m.}
\end{itemize}
O mesmo ou melhor que \textunderscore thorite\textunderscore .
\section{Thoro}
\begin{itemize}
\item {Grp. gram.:m.}
\end{itemize}
\begin{itemize}
\item {Utilização:Poét.}
\end{itemize}
\begin{itemize}
\item {Proveniência:(Lat. \textunderscore thorum\textunderscore )}
\end{itemize}
O leito conjugal.
\section{Thrace}
\begin{itemize}
\item {Grp. gram.:m.}
\end{itemize}
\begin{itemize}
\item {Proveniência:(Lat. \textunderscore Thraces\textunderscore )}
\end{itemize}
O mesmo ou melhor que \textunderscore thrácio\textunderscore :«\textunderscore Gregos, Thraces, Armenios...\textunderscore »\textunderscore Lusíadas\textunderscore .
\section{Thrácio}
\begin{itemize}
\item {Grp. gram.:adj.}
\end{itemize}
\begin{itemize}
\item {Grp. gram.:M. pl.}
\end{itemize}
\begin{itemize}
\item {Proveniência:(Lat. \textunderscore thracius\textunderscore )}
\end{itemize}
Relativo á Thrácia.
Habitantes da Thrácia.
\section{Thracónico}
\begin{itemize}
\item {Grp. gram.:adj.}
\end{itemize}
\begin{itemize}
\item {Utilização:Des.}
\end{itemize}
\begin{itemize}
\item {Proveniência:(Do lat. \textunderscore thracus\textunderscore )}
\end{itemize}
Traidor; velhaco.
\section{Thraconismo}
\begin{itemize}
\item {Grp. gram.:m.}
\end{itemize}
\begin{itemize}
\item {Utilização:Des.}
\end{itemize}
Perfídia; velhacaria.--Moraes traz \textunderscore thrasonismo\textunderscore , certamente por êrro typográphico.
(Cp. \textunderscore thracónico\textunderscore )
\section{Threno}
\begin{itemize}
\item {Grp. gram.:m.}
\end{itemize}
\begin{itemize}
\item {Proveniência:(Lat. \textunderscore threnus\textunderscore )}
\end{itemize}
Canto plangente; lamentação; elegia.
\section{Thridácio}
\begin{itemize}
\item {Grp. gram.:m.}
\end{itemize}
\begin{itemize}
\item {Proveniência:(Do gr. \textunderscore thridax\textunderscore )}
\end{itemize}
Substância medicamentosa, que se prepara com o suco de alface.
\section{Thríncia}
\begin{itemize}
\item {Grp. gram.:f.}
\end{itemize}
Gênero de plantas, da fam. das compostas.
\section{Thripes}
\begin{itemize}
\item {Grp. gram.:m. Pl.}
\end{itemize}
\begin{itemize}
\item {Proveniência:(Lat. \textunderscore thripes\textunderscore )}
\end{itemize}
Gênero de pequenos insectos, que vivem nas fôlhas e nas flôres das plantas.
\section{Thripóphago}
\begin{itemize}
\item {Grp. gram.:adj.}
\end{itemize}
\begin{itemize}
\item {Utilização:Zool.}
\end{itemize}
\begin{itemize}
\item {Proveniência:(Do gr. \textunderscore thrips\textunderscore  + \textunderscore phagein\textunderscore )}
\end{itemize}
Que se alimenta de insectos e de pequenos vermes.
\section{Thriptomena}
\begin{itemize}
\item {Grp. gram.:f.}
\end{itemize}
Gênero de plantas myrtáceas.
\section{Thrombo}
\begin{itemize}
\item {Grp. gram.:m.}
\end{itemize}
\begin{itemize}
\item {Utilização:Med.}
\end{itemize}
\begin{itemize}
\item {Proveniência:(Gr. \textunderscore thrombos\textunderscore )}
\end{itemize}
Pequeno tumor duro e violáceo, que se fórma em tôrno da abertura de uma veia, depois da sangria.
Tumor, constituido por sangue infiltrado; etc.
\section{Thrombose}
\begin{itemize}
\item {Grp. gram.:f.}
\end{itemize}
\begin{itemize}
\item {Utilização:Med.}
\end{itemize}
\begin{itemize}
\item {Proveniência:(Do gr. \textunderscore thrombos\textunderscore )}
\end{itemize}
Coagulação do sangue em qualquer ponto do systema circulatório, no corpo vivo.
\section{Throneto}
\begin{itemize}
\item {fónica:nê}
\end{itemize}
\begin{itemize}
\item {Grp. gram.:m.}
\end{itemize}
Pequeno trono.
\section{Throno}
\begin{itemize}
\item {Grp. gram.:m.}
\end{itemize}
\begin{itemize}
\item {Utilização:Fig.}
\end{itemize}
\begin{itemize}
\item {Grp. gram.:Pl.}
\end{itemize}
\begin{itemize}
\item {Utilização:Theol.}
\end{itemize}
\begin{itemize}
\item {Proveniência:(Lat. \textunderscore thronus\textunderscore )}
\end{itemize}
Assento elevado ou sólio, que os Soberanos occupam nas occasiões solennes.
Poder ou autoridade soberana; soberania.
Soberano.
Um dos nove coros dos anjos.
\section{Thrypsina}
\begin{itemize}
\item {Grp. gram.:f.}
\end{itemize}
\begin{itemize}
\item {Proveniência:(Do gr. \textunderscore thrupsis\textunderscore )}
\end{itemize}
Um dos princípios do suco pancreático.
\section{Thuia}
\begin{itemize}
\item {Grp. gram.:f.}
\end{itemize}
\begin{itemize}
\item {Proveniência:(Do gr. \textunderscore thuia\textunderscore )}
\end{itemize}
Gênero de árvores coníferas.
\section{Thuja}
\begin{itemize}
\item {Grp. gram.:f.}
\end{itemize}
\begin{itemize}
\item {Utilização:Des.}
\end{itemize}
O mesmo que \textunderscore thuia\textunderscore .
\section{Thuribular}
\begin{itemize}
\item {Grp. gram.:v. t.}
\end{itemize}
\begin{itemize}
\item {Utilização:Fig.}
\end{itemize}
\begin{itemize}
\item {Proveniência:(De \textunderscore thuríbulo\textunderscore )}
\end{itemize}
Queimar incenso, em honra de.
Lisonjear, adular:«\textunderscore ...é thuribular a ignorância dos povos.\textunderscore »D. Ant. da Costa, \textunderscore Três Mundos\textunderscore , 16.
\section{Thuribulário}
\begin{itemize}
\item {Grp. gram.:m.  e  adj.}
\end{itemize}
\begin{itemize}
\item {Utilização:Fig.}
\end{itemize}
\begin{itemize}
\item {Proveniência:(De \textunderscore thuríbulo\textunderscore )}
\end{itemize}
O que agita o thuríbulo para incensar.
Adulador.
\section{Thuríbulo}
\begin{itemize}
\item {Grp. gram.:m.}
\end{itemize}
\begin{itemize}
\item {Proveniência:(Lat. \textunderscore thuribulum\textunderscore )}
\end{itemize}
Vaso, em que se queima incenso.
\section{Thurícremo}
\begin{itemize}
\item {Grp. gram.:adj.}
\end{itemize}
\begin{itemize}
\item {Utilização:Poét.}
\end{itemize}
\begin{itemize}
\item {Proveniência:(Lat. \textunderscore thuricremus\textunderscore )}
\end{itemize}
Em que se queima incenso. Cf. F. Barreto, \textunderscore Eneida\textunderscore , IV, 103.
\section{Thuriferário}
\begin{itemize}
\item {Grp. gram.:m.  e  adj.}
\end{itemize}
\begin{itemize}
\item {Proveniência:(De \textunderscore thurífero\textunderscore )}
\end{itemize}
O que leva o thuríbulo.
\section{Thurífero}
\begin{itemize}
\item {Grp. gram.:adj.}
\end{itemize}
\begin{itemize}
\item {Proveniência:(Lat. \textunderscore thuriferus\textunderscore )}
\end{itemize}
Que produz incenso.
\section{Thurificação}
\begin{itemize}
\item {Grp. gram.:f.}
\end{itemize}
\begin{itemize}
\item {Proveniência:(Lat. \textunderscore thurificatio\textunderscore )}
\end{itemize}
Acto ou effeito de thurificar.
\section{Thurificador}
\begin{itemize}
\item {Grp. gram.:m.  e  adj.}
\end{itemize}
\begin{itemize}
\item {Proveniência:(Do lat. \textunderscore thurificator\textunderscore )}
\end{itemize}
O que thurifica.
\section{Thurificante}
\begin{itemize}
\item {Grp. gram.:adj.}
\end{itemize}
\begin{itemize}
\item {Proveniência:(Lat. \textunderscore thurificans\textunderscore )}
\end{itemize}
Que thurifica.
\section{Thurificar}
\begin{itemize}
\item {Grp. gram.:v. t.}
\end{itemize}
\begin{itemize}
\item {Proveniência:(Lat. \textunderscore thurificare\textunderscore )}
\end{itemize}
O mesmo que \textunderscore incensar\textunderscore , em sentido próprio e figurado.
\section{Thuríngia}
\begin{itemize}
\item {Grp. gram.:f.}
\end{itemize}
O mesmo que \textunderscore toronja\textunderscore .
\section{Thuríngios}
\begin{itemize}
\item {Grp. gram.:m. Pl.}
\end{itemize}
\begin{itemize}
\item {Proveniência:(Lat. \textunderscore thuringi\textunderscore )}
\end{itemize}
Tríbo combatente nas guerras góticas da Espanha. Cf. Herculano, \textunderscore Eurico\textunderscore , c. IV.
\section{Thýade}
\begin{itemize}
\item {Grp. gram.:f.}
\end{itemize}
\begin{itemize}
\item {Proveniência:(Lat. \textunderscore thyas\textunderscore )}
\end{itemize}
Sacerdotisa de Baccho; bacchante.
\section{Thylacino}
\begin{itemize}
\item {Grp. gram.:m.}
\end{itemize}
\begin{itemize}
\item {Proveniência:(Do gr. \textunderscore thulax\textunderscore )}
\end{itemize}
Gênero de mammíferos marsupiaes.
\section{Thymallo}
\begin{itemize}
\item {Grp. gram.:m.}
\end{itemize}
\begin{itemize}
\item {Proveniência:(Lat. \textunderscore thymallus\textunderscore )}
\end{itemize}
Gênero de peixes malacopterýgios.
\section{Thymbra}
\begin{itemize}
\item {Grp. gram.:f.}
\end{itemize}
\begin{itemize}
\item {Proveniência:(Lat. \textunderscore thymbra\textunderscore )}
\end{itemize}
Gênero de plantas aromáticas, da fam. das labiadas.
\section{Thymbreira}
\begin{itemize}
\item {Grp. gram.:f.}
\end{itemize}
O mesmo que \textunderscore thymbra\textunderscore .
\section{Thymeláceas}
\begin{itemize}
\item {Grp. gram.:f. pl.}
\end{itemize}
Família de plantas, que tem por typo a thymeleia.
\section{Thýmele}
\begin{itemize}
\item {Grp. gram.:f.}
\end{itemize}
\begin{itemize}
\item {Proveniência:(Lat. \textunderscore thymele\textunderscore )}
\end{itemize}
Estrado, adeante do proscênio, nos theatros gregos, donde os músicos dirigiam as evoluções dos coros.
O altar dos sacrifícios, na tragédia grega.
\section{Thymeleia}
\begin{itemize}
\item {Grp. gram.:f.}
\end{itemize}
\begin{itemize}
\item {Proveniência:(Lat. \textunderscore thymelaea\textunderscore )}
\end{itemize}
Gênero de plantas, conhecido scientificamente por \textunderscore daphne thymelaea\textunderscore .
\section{Thymiama}
\begin{itemize}
\item {Grp. gram.:f.}
\end{itemize}
\begin{itemize}
\item {Proveniência:(Lat. \textunderscore thymiama\textunderscore )}
\end{itemize}
Certa droga medicinal. Cf. Garc. da Orta, 7, v.^o.
\section{Thymiatechnia}
\begin{itemize}
\item {Grp. gram.:f.}
\end{itemize}
\begin{itemize}
\item {Proveniência:(Do gr. \textunderscore thumos\textunderscore  + \textunderscore tekne\textunderscore )}
\end{itemize}
Arte de fabricar perfumes.
\section{Thýmico}
\begin{itemize}
\item {Grp. gram.:adj.}
\end{itemize}
\begin{itemize}
\item {Utilização:Anat.}
\end{itemize}
Relativo ao \textunderscore thymo\textunderscore ^2.
\section{Thymo}
\begin{itemize}
\item {Grp. gram.:m.}
\end{itemize}
\begin{itemize}
\item {Proveniência:(Lat. \textunderscore thymum\textunderscore )}
\end{itemize}
O mesmo que \textunderscore tomilho\textunderscore .
\section{Thymo}
\begin{itemize}
\item {Grp. gram.:m.}
\end{itemize}
\begin{itemize}
\item {Utilização:Anat.}
\end{itemize}
\begin{itemize}
\item {Proveniência:(Do gr. \textunderscore thumos\textunderscore )}
\end{itemize}
Corpo carnoso ou glandular, no thórax do feto.
\section{Thymocracia}
\begin{itemize}
\item {Grp. gram.:f.}
\end{itemize}
Systema de governação, em que preponderam os ricos.
(Cp. \textunderscore thymócrata\textunderscore )
\section{Thymócrata}
\begin{itemize}
\item {Grp. gram.:m.}
\end{itemize}
\begin{itemize}
\item {Proveniência:(Do gr. \textunderscore thumos\textunderscore  + \textunderscore kratos\textunderscore )}
\end{itemize}
Partidário da thymocracia.
\section{Thymocrático}
\begin{itemize}
\item {Grp. gram.:adj.}
\end{itemize}
Relativo á thymocracia.
\section{Thymolina}
\begin{itemize}
\item {Grp. gram.:f.}
\end{itemize}
\begin{itemize}
\item {Proveniência:(De \textunderscore thymo\textunderscore ^1 + \textunderscore óleo\textunderscore ?)}
\end{itemize}
Substância pharmacêutica, de que se faz uma pasta dentífrica.
\section{Thyreóide}
\begin{itemize}
\item {Grp. gram.:f.}
\end{itemize}
\begin{itemize}
\item {Utilização:Anat.}
\end{itemize}
\begin{itemize}
\item {Grp. gram.:Adj.}
\end{itemize}
A maior cartilagem da larynge, na parte antero-superior.
O mesmo que \textunderscore thyreoídeo\textunderscore .
\section{Thyreoidectomia}
\begin{itemize}
\item {Grp. gram.:f.}
\end{itemize}
\begin{itemize}
\item {Utilização:Cir.}
\end{itemize}
Operação, que consiste na extirpação total da glândula thyreoídea, no tratamento do bócio exophthálmico.
\section{Thyreoídeo}
\begin{itemize}
\item {Grp. gram.:adj.}
\end{itemize}
\begin{itemize}
\item {Proveniência:(Do gr. \textunderscore thureos\textunderscore  + \textunderscore eidos\textunderscore )}
\end{itemize}
Diz-se da maior cartilagem da larynge e de um corpo glandular, situado na parte antero-superior da larynge.
\section{Thyrídia}
\begin{itemize}
\item {Grp. gram.:f.}
\end{itemize}
\begin{itemize}
\item {Proveniência:(Do gr. \textunderscore thuridion\textunderscore )}
\end{itemize}
Gênero de insectos lepidópteros.
\section{Thyròhyal}
\begin{itemize}
\item {Grp. gram.:m.}
\end{itemize}
Ponta maior do osso hyóide.
(Por \textunderscore thyreóhyal\textunderscore , do gr. \textunderscore thureos\textunderscore )
\section{Thyrsígero}
\begin{itemize}
\item {Grp. gram.:adj.}
\end{itemize}
\begin{itemize}
\item {Proveniência:(Lat. \textunderscore thyrsiger\textunderscore )}
\end{itemize}
Que tem thyrso.
\section{Thyrso}
\begin{itemize}
\item {Grp. gram.:m.}
\end{itemize}
\begin{itemize}
\item {Proveniência:(Lat. \textunderscore thyrsus\textunderscore )}
\end{itemize}
Bastão, enfeitado de hera e pâmpanos e terminado em fórma de pinha.
Espécie de panícula, semelhante a um ramalhete comprido.
\section{Thyrsoso}
\begin{itemize}
\item {Grp. gram.:adj.}
\end{itemize}
Que tem flôres em fórma de thyrso.
\section{Thysanópode}
\begin{itemize}
\item {Grp. gram.:m.}
\end{itemize}
\begin{itemize}
\item {Proveniência:(Do gr. \textunderscore thusanos\textunderscore  + \textunderscore pous\textunderscore )}
\end{itemize}
Gênero de crustáceos.
\section{Thysanópteros}
\begin{itemize}
\item {Grp. gram.:m. pl.}
\end{itemize}
\begin{itemize}
\item {Proveniência:(Do gr. \textunderscore thusanos\textunderscore  + \textunderscore pteron\textunderscore )}
\end{itemize}
Ordem de insectos, que vivem nos vegetaes, damnificando-os.
\section{Tegela}
\begin{itemize}
\item {Grp. gram.:f.}
\end{itemize}
Espécie de chícara grande, sem asa.
Vaso de barro, mais ou menos tôsco, sem gargalo, e sem asas ou com asas pequenas.
Disco de barro, em que se levam ao forno certas qualidades de doce.
(Dem. do lat. \textunderscore tegula\textunderscore )
\section{Tegelada}
\begin{itemize}
\item {Grp. gram.:f.}
\end{itemize}
\begin{itemize}
\item {Utilização:Ant.}
\end{itemize}
\begin{itemize}
\item {Proveniência:(De \textunderscore tegela\textunderscore )}
\end{itemize}
Conteúdo de uma tegela.
O que uma tegela póde conter.
Caldeirada.
Variedade de pudim.
Guisado de vinho branco, leite, ovos, etc., que entrava nas foragens dos antigos prazos.
\section{Tegelinha}
\begin{itemize}
\item {Grp. gram.:f.}
\end{itemize}
Pequena tegela, para illuminações e outros usos.
\section{Tejoleira}
\begin{itemize}
\item {Grp. gram.:f.}
\end{itemize}
Fragmento de tejolo, para ladrilhos.
Grande tejolo.
\section{Tejoleiro}
\begin{itemize}
\item {Grp. gram.:m.}
\end{itemize}
Fabricante de tejolos.
\section{Thysanuros}
\begin{itemize}
\item {Grp. gram.:m. pl.}
\end{itemize}
\begin{itemize}
\item {Proveniência:(Gr. \textunderscore thusanouros\textunderscore )}
\end{itemize}
Ordem de insectos neurópteros.
\section{Thysito}
\begin{itemize}
\item {Grp. gram.:m.}
\end{itemize}
Variedade de mármore verde.
\section{Ti}
\begin{itemize}
\item {Proveniência:(Do lat. \textunderscore tibi\textunderscore )}
\end{itemize}
Flexão do pron. \textunderscore teu\textunderscore , quando é precedido de prep.: \textunderscore gósto de ti\textunderscore .
\section{Ti}
\begin{itemize}
\item {Grp. gram.:m.}
\end{itemize}
Planta liliácea, procedente da China.
\section{Tia}
\begin{itemize}
\item {Grp. gram.:f.}
\end{itemize}
(Fem. de \textunderscore tio\textunderscore )
\section{Tíade}
\begin{itemize}
\item {Grp. gram.:f.}
\end{itemize}
\begin{itemize}
\item {Proveniência:(Lat. \textunderscore thyas\textunderscore )}
\end{itemize}
Sacerdotisa de Baco; bacante.
\section{Tiambó}
\begin{itemize}
\item {Grp. gram.:m.}
\end{itemize}
\begin{itemize}
\item {Utilização:Bras}
\end{itemize}
Espécie de cana de açúcar.
\section{Tiara}
\begin{itemize}
\item {Grp. gram.:f.}
\end{itemize}
\begin{itemize}
\item {Utilização:Fig.}
\end{itemize}
\begin{itemize}
\item {Proveniência:(Lat. \textunderscore tiara\textunderscore )}
\end{itemize}
Ornato para a cabeça, usado outrora entre vários povos do Oriente.
Barrete, de fórma cónica, usado pelo Papa nas grandes ceremónias, e que é rodeado de três corôas de oiro e rematado por um globo que sustenta uma cruz.
Dignidade pontifícia.
\section{Tiarela}
\begin{itemize}
\item {Grp. gram.:f.}
\end{itemize}
\begin{itemize}
\item {Proveniência:(De \textunderscore tiara\textunderscore )}
\end{itemize}
Gênero de molluscos gasterópodes.
Espécie de saxífraga.
\section{Tiarode}
\begin{itemize}
\item {Grp. gram.:m.}
\end{itemize}
Gênero de insectos hemípteros.
\section{Tiba}
\begin{itemize}
\item {Grp. gram.:f.}
\end{itemize}
\begin{itemize}
\item {Utilização:Gír. ant. de marujos.}
\end{itemize}
Navalha, faca.
\section{Tiba}
\begin{itemize}
\item {Grp. gram.:f.}
\end{itemize}
\begin{itemize}
\item {Utilização:Bras}
\end{itemize}
Lugar, onde há muitas pessôas ou coisas reunidas.
\section{Tibaca}
\begin{itemize}
\item {Grp. gram.:f.}
\end{itemize}
\begin{itemize}
\item {Utilização:Bras}
\end{itemize}
Bráctea floral das palmeiras.
\section{Tibada}
\begin{itemize}
\item {Grp. gram.:f.}
\end{itemize}
\begin{itemize}
\item {Utilização:Gír. ant. de marujos.}
\end{itemize}
\begin{itemize}
\item {Proveniência:(De \textunderscore tiba\textunderscore ^1)}
\end{itemize}
Facada, navalhada.
\section{Tibáudia}
\begin{itemize}
\item {Grp. gram.:f.}
\end{itemize}
\begin{itemize}
\item {Proveniência:(De \textunderscore Thibaud\textunderscore , n. p.)}
\end{itemize}
Gênero de plantas ericáceas.
\section{Tibel}
\begin{itemize}
\item {Grp. gram.:m.}
\end{itemize}
Árvore de Damão, (\textunderscore caesalpinea ferrea\textunderscore ).
\section{Tiberino}
\begin{itemize}
\item {Grp. gram.:adj.}
\end{itemize}
\begin{itemize}
\item {Proveniência:(Lat. \textunderscore tiberinus\textunderscore )}
\end{itemize}
Relativo ao rio Tibre ou á região banhada por elle.
\section{Tibetano}
\begin{itemize}
\item {Grp. gram.:adj.}
\end{itemize}
\begin{itemize}
\item {Grp. gram.:M.}
\end{itemize}
Relativo ao Tibet.
Habitante do Tibet.
Língua, falada no Tibet.
\section{Tíbi!}
\begin{itemize}
\item {Grp. gram.:interj.}
\end{itemize}
\begin{itemize}
\item {Utilização:Bras}
\end{itemize}
(Designa \textunderscore espanto\textunderscore )
\section{Tíbia}
\begin{itemize}
\item {Grp. gram.:f.}
\end{itemize}
\begin{itemize}
\item {Proveniência:(Lat. \textunderscore tibia\textunderscore )}
\end{itemize}
O mais grosso dos dois ossos da perna.
Canela da perna.
Terceira articulação das pernas dos insectos.
Pífaro.
Frauta de pastor.
Trombeta.
\section{Tibial}
\begin{itemize}
\item {Grp. gram.:adj.}
\end{itemize}
\begin{itemize}
\item {Utilização:Anat.}
\end{itemize}
\begin{itemize}
\item {Grp. gram.:M.}
\end{itemize}
\begin{itemize}
\item {Proveniência:(Lat. \textunderscore tibialis\textunderscore )}
\end{itemize}
Relativo á tíbia.
Cada um dos músculos da perna.
\section{Tibiamente}
\begin{itemize}
\item {Grp. gram.:adv.}
\end{itemize}
De modo tíbio; com froixidão.
\section{Tibiez}
\begin{itemize}
\item {Grp. gram.:f.}
\end{itemize}
O mesmo que \textunderscore tibieza\textunderscore .
Cf. Filinto, XI, 10.
\section{Tibieza}
\begin{itemize}
\item {fónica:ê}
\end{itemize}
\begin{itemize}
\item {Grp. gram.:f.}
\end{itemize}
Qualidade do que é tíbio; froixidão; indolência.
\section{Tíbio}
\begin{itemize}
\item {Grp. gram.:adj.}
\end{itemize}
\begin{itemize}
\item {Utilização:Fig.}
\end{itemize}
\begin{itemize}
\item {Proveniência:(Do lat. \textunderscore tepidus\textunderscore )}
\end{itemize}
Tépido.
Froixo; indolente.
Escasso.
\section{Tibio-calcaneano}
\begin{itemize}
\item {Grp. gram.:adj.}
\end{itemize}
\begin{itemize}
\item {Utilização:Anat.}
\end{itemize}
Diz-se de um dos músculos da perna.
\section{Tibio-tarsiano}
\begin{itemize}
\item {Grp. gram.:adj.}
\end{itemize}
\begin{itemize}
\item {Utilização:Anat.}
\end{itemize}
Relativo á tíbia e ao tarso.
\section{Tibira}
\begin{itemize}
\item {Grp. gram.:f.}
\end{itemize}
\begin{itemize}
\item {Utilização:Bras. do N}
\end{itemize}
\begin{itemize}
\item {Grp. gram.:M.}
\end{itemize}
Vaca, que dá pouco leite.
Mau tirador do leite.
\section{Tibirar}
\begin{itemize}
\item {Grp. gram.:v. i.}
\end{itemize}
\begin{itemize}
\item {Utilização:Bras. do N}
\end{itemize}
\begin{itemize}
\item {Proveniência:(De \textunderscore tibira\textunderscore )}
\end{itemize}
Não formar espuma (o leite).
\section{Tiborna}
\begin{itemize}
\item {Grp. gram.:f.}
\end{itemize}
\begin{itemize}
\item {Utilização:Fig.}
\end{itemize}
Pão quente embebido, em azeite novo.
Mixórdia.
Líquido entornado.
Aguapé muito ordinária.
Planta apocýnea do Brasil.
\section{Tibórnea}
\begin{itemize}
\item {Grp. gram.:f.}
\end{itemize}
\begin{itemize}
\item {Utilização:Prov.}
\end{itemize}
\begin{itemize}
\item {Utilização:beir.}
\end{itemize}
O mesmo que \textunderscore tiborna\textunderscore .
Refeição, que se usa nos lagares de azeite, e que consta de bacalhau cozido, batatas e couves, tudo temperado com azeite novo.
\section{Tiborneira}
\begin{itemize}
\item {Grp. gram.:f.}
\end{itemize}
\begin{itemize}
\item {Utilização:T. de Serpa}
\end{itemize}
\begin{itemize}
\item {Proveniência:(De \textunderscore tiborna\textunderscore )}
\end{itemize}
Espécie de vasilha de barro.
\section{Tibornice}
\begin{itemize}
\item {Grp. gram.:f.}
\end{itemize}
\begin{itemize}
\item {Utilização:Fam.}
\end{itemize}
\begin{itemize}
\item {Proveniência:(De \textunderscore tiborna\textunderscore )}
\end{itemize}
Mixórdia; salgalhada.
\section{Tibrino}
\begin{itemize}
\item {Grp. gram.:adj.}
\end{itemize}
\begin{itemize}
\item {Proveniência:(De \textunderscore Tibre\textunderscore , n. p.)}
\end{itemize}
(V.tiberino)
\section{Tibungar}
\begin{itemize}
\item {Grp. gram.:v. i.}
\end{itemize}
\begin{itemize}
\item {Utilização:Bras. do N}
\end{itemize}
\begin{itemize}
\item {Proveniência:(De \textunderscore tibungo\textunderscore )}
\end{itemize}
Cair na água; afundar-se.
\section{Tibungo!}
\begin{itemize}
\item {Grp. gram.:interj.}
\end{itemize}
\begin{itemize}
\item {Utilização:Bras. do N}
\end{itemize}
\begin{itemize}
\item {Proveniência:(T. onom.)}
\end{itemize}
Voz imitativa do som, que um corpo produz, caindo na água.
\section{Tiburtino}
\begin{itemize}
\item {Grp. gram.:adj.}
\end{itemize}
\begin{itemize}
\item {Proveniência:(Lat. \textunderscore tiburtinus\textunderscore )}
\end{itemize}
Relativo á antiga cidade de Tibur, vizinha de Roma. Cf. Júl. Castilho, \textunderscore Ermitério\textunderscore , 163.
\section{Tiçalho}
\begin{itemize}
\item {Grp. gram.:m.}
\end{itemize}
\begin{itemize}
\item {Utilização:Prov.}
\end{itemize}
\begin{itemize}
\item {Utilização:minh.}
\end{itemize}
Espevitador de candeeiros de azeite.
(Por \textunderscore atiçalho\textunderscore , de \textunderscore atiçar\textunderscore )
\section{Tical}
\begin{itemize}
\item {Grp. gram.:m.}
\end{itemize}
Antigo e pequeno pêso indiano.
Moéda siamesa. Cf. \textunderscore Século\textunderscore , de 30-VI-913.
\section{Tição}
\begin{itemize}
\item {Grp. gram.:m.}
\end{itemize}
\begin{itemize}
\item {Utilização:Fig.}
\end{itemize}
\begin{itemize}
\item {Utilização:T. de Turquel}
\end{itemize}
\begin{itemize}
\item {Proveniência:(Do lat. \textunderscore titio\textunderscore )}
\end{itemize}
Pedaço de lenha accesa ou meio queimada.
Pessôa suja ou muito trigueira.
Diabo.
Pessôa ruím.
\section{Tico}
\begin{itemize}
\item {Grp. gram.:m.}
\end{itemize}
\begin{itemize}
\item {Utilização:Bras. do Rio}
\end{itemize}
O mesmo que \textunderscore taco\textunderscore ^2.
\section{Tico}
\begin{itemize}
\item {Grp. gram.:m.}
\end{itemize}
\begin{itemize}
\item {Utilização:Veter.}
\end{itemize}
Vício dos equídeos, que poisam os dentes superiores na mangedoira ou noutro objecto, parecendo que tomam ar. Cf. Mac. Pinto, \textunderscore Comp. de Veter.\textunderscore 
(Por \textunderscore tique\textunderscore ?)
\section{Tiçoada}
\begin{itemize}
\item {Grp. gram.:f.}
\end{itemize}
Pancada com tição.
\section{Tiçoeiro}
\begin{itemize}
\item {Grp. gram.:m.}
\end{itemize}
\begin{itemize}
\item {Proveniência:(De \textunderscore tição\textunderscore )}
\end{itemize}
Utensílio de ferro, com que se atiça o lume ou se revolve e brasido para o avivar.
\section{Tiçonado}
\begin{itemize}
\item {Grp. gram.:adj.}
\end{itemize}
\begin{itemize}
\item {Proveniência:(De \textunderscore tição\textunderscore )}
\end{itemize}
Chamuscado.
Malhado de negro.
\section{Ticórea}
\begin{itemize}
\item {Grp. gram.:f.}
\end{itemize}
Gênero de plantas diósmeas.
\section{Tico-tico}
\begin{itemize}
\item {Grp. gram.:m.}
\end{itemize}
\begin{itemize}
\item {Utilização:Bras}
\end{itemize}
\begin{itemize}
\item {Proveniência:(T. onom.)}
\end{itemize}
O pipilar de algumas aves.
Passarinho, de papo amarelo.
\section{Tic-tac}
\begin{itemize}
\item {Grp. gram.:m.}
\end{itemize}
(V.tique-taque)
\section{Ticum}
\begin{itemize}
\item {Grp. gram.:m.}
\end{itemize}
\begin{itemize}
\item {Utilização:Prov.}
\end{itemize}
\begin{itemize}
\item {Utilização:minh.}
\end{itemize}
O mesmo que \textunderscore tucum\textunderscore .
Linho especial para rêdes.
\section{Ticuma}
\begin{itemize}
\item {Grp. gram.:f.}
\end{itemize}
O mesmo que \textunderscore curare\textunderscore .
\section{Ticunas}
\begin{itemize}
\item {Grp. gram.:m. pl.}
\end{itemize}
Tríbo de Índios do Peru.
\section{Tido}
\begin{itemize}
\item {Grp. gram.:adj.}
\end{itemize}
\begin{itemize}
\item {Proveniência:(De \textunderscore têr\textunderscore )}
\end{itemize}
Possuído.
Julgado, considerado: \textunderscore o Viana é tido como sábio\textunderscore .
\section{Tidores}
\begin{itemize}
\item {Grp. gram.:m. pl.}
\end{itemize}
Habitantes de Tidore. Cf. Couto, \textunderscore Déc.\textunderscore , IX, c. 29.
\section{Tiê}
\begin{itemize}
\item {Grp. gram.:m.}
\end{itemize}
\begin{itemize}
\item {Utilização:Bras}
\end{itemize}
Gênero de pássaros.
\section{Tien}
\begin{itemize}
\item {Grp. gram.:m.}
\end{itemize}
\begin{itemize}
\item {Proveniência:(T. chinês)}
\end{itemize}
O Sêr Supremo, segundo a doutrina de Confúcio.
O céu, considerado como emblema da divindade, segundo a mesma doutrina.
\section{Tietê}
\begin{itemize}
\item {Grp. gram.:m.}
\end{itemize}
Gênero de aves brasileiras, nocivas aos frutos.
\section{Tigela}
\begin{itemize}
\item {Grp. gram.:f.}
\end{itemize}
Espécie de chícara grande, sem asa.
Vaso de barro, mais ou menos tôsco, sem gargalo, e sem asas ou com asas pequenas.
Disco de barro, em que se levam ao forno certas qualidades de doce.
(Dem. do lat. \textunderscore tegula\textunderscore )
\section{Tigelada}
\begin{itemize}
\item {Grp. gram.:f.}
\end{itemize}
\begin{itemize}
\item {Utilização:Ant.}
\end{itemize}
\begin{itemize}
\item {Proveniência:(De \textunderscore tigela\textunderscore )}
\end{itemize}
Conteúdo de uma tigela.
O que uma tigela póde conter.
Caldeirada.
Variedade de pudim.
Guisado de vinho branco, leite, ovos, etc., que entrava nas foragens dos antigos prazos.
\section{Tigelinha}
\begin{itemize}
\item {Grp. gram.:f.}
\end{itemize}
Pequena tigela, para illuminações e outros usos.
\section{Tigello}
\begin{itemize}
\item {Grp. gram.:m.}
\end{itemize}
\begin{itemize}
\item {Utilização:Ant.}
\end{itemize}
\begin{itemize}
\item {Proveniência:(Fr. \textunderscore tigelle\textunderscore )}
\end{itemize}
Vara, que se empregava em certos jogos. Cf. \textunderscore Port. Mon. Hist.\textunderscore , \textunderscore Script.\textunderscore , 250.
\section{Tigelo}
\begin{itemize}
\item {Grp. gram.:m.}
\end{itemize}
\begin{itemize}
\item {Utilização:Ant.}
\end{itemize}
\begin{itemize}
\item {Proveniência:(Fr. \textunderscore tigelle\textunderscore )}
\end{itemize}
Vara, que se empregava em certos jogos. Cf. \textunderscore Port. Mon. Hist.\textunderscore , \textunderscore Script.\textunderscore , 250.
\section{Tigêlo}
\begin{itemize}
\item {Grp. gram.:m.}
\end{itemize}
\begin{itemize}
\item {Utilização:Ant.}
\end{itemize}
O mesmo que \textunderscore tijolo\textunderscore .
\section{Tigenol}
\begin{itemize}
\item {Grp. gram.:m.}
\end{itemize}
Medicamento, sucedâneo do ichthiol.
\section{Tiglina}
\begin{itemize}
\item {Grp. gram.:f.}
\end{itemize}
\begin{itemize}
\item {Utilização:Chím.}
\end{itemize}
Substância resinosa, que se extrái do \textunderscore croton tiglium\textunderscore .
\section{Tigrado}
\begin{itemize}
\item {Grp. gram.:adj.}
\end{itemize}
Mosqueado como a pelle do tigre.
\section{Tigre}
\begin{itemize}
\item {Grp. gram.:m.}
\end{itemize}
\begin{itemize}
\item {Utilização:Fig.}
\end{itemize}
\begin{itemize}
\item {Grp. gram.:F.}
\end{itemize}
\begin{itemize}
\item {Proveniência:(Lat. \textunderscore tigris\textunderscore )}
\end{itemize}
Animal feroz, da fam. dos felinos.
Homem sanguinário, cruel.
Insecto, que ataca as macieiras e pereiras, devorando as fôlhas, até deixar só as fibras e lenhos. Cf \textunderscore Bibl. da Gente do Campo\textunderscore , 382.
«\textunderscore ver uma tigre devorar um filho\textunderscore ». Gonç. Dias.
\section{Tigré}
\begin{itemize}
\item {Grp. gram.:m.}
\end{itemize}
Uma das línguas da África oriental, ao norte do Equador.
\section{Tigrídio}
\begin{itemize}
\item {Grp. gram.:f.}
\end{itemize}
Gênero de plantas irídeas.
\section{Tígrido}
\begin{itemize}
\item {Grp. gram.:adj.}
\end{itemize}
\begin{itemize}
\item {Proveniência:(Do lat. \textunderscore tigris\textunderscore , \textunderscore tigridis\textunderscore )}
\end{itemize}
Vestido da pelle de tigre:«\textunderscore as tígridas Bacchantes...\textunderscore »Filinto, II, 8.
\section{Tigrinho}
\begin{itemize}
\item {Grp. gram.:m.}
\end{itemize}
O mesmo que \textunderscore tigré\textunderscore .
\section{Tigrino}
\begin{itemize}
\item {Grp. gram.:adj.}
\end{itemize}
\begin{itemize}
\item {Proveniência:(Lat. \textunderscore trigrinus\textunderscore )}
\end{itemize}
Relativo a tigre.
Sanguinário como o tigre.
\section{Tigrino}
\begin{itemize}
\item {Grp. gram.:m.}
\end{itemize}
O mesmo que \textunderscore tigré\textunderscore .
\section{Tiguera}
\begin{itemize}
\item {fónica:gu-é}
\end{itemize}
\begin{itemize}
\item {Grp. gram.:f.}
\end{itemize}
\begin{itemize}
\item {Utilização:Bras}
\end{itemize}
Roça de milho ou de outras plantações annuaes, depois de effectuada a colheita.
Restolho.
(Do tupi)
\section{Tiitho}
\begin{itemize}
\item {Grp. gram.:m.}
\end{itemize}
Nome, dado por Dioscórides a uma argilla esverdeada e dura.
\section{Tiito}
\begin{itemize}
\item {Grp. gram.:m.}
\end{itemize}
Nome, dado por Dioscórides a uma argilla esverdeada e dura.
\section{Tijeguacu}
\begin{itemize}
\item {Grp. gram.:m.}
\end{itemize}
Pássaro dentirostro da América.
\section{Tijoleira}
\begin{itemize}
\item {Grp. gram.:f.}
\end{itemize}
Fragmento de tijolo, para ladrilhos.
Grande tijolo.
\section{Tijoleiro}
\begin{itemize}
\item {Grp. gram.:m.}
\end{itemize}
Fabricante de tijolos.
\section{Tijolo}
\begin{itemize}
\item {fónica:jô}
\end{itemize}
\begin{itemize}
\item {Grp. gram.:m.}
\end{itemize}
\begin{itemize}
\item {Grp. gram.:Loc.}
\end{itemize}
\begin{itemize}
\item {Utilização:pop.}
\end{itemize}
\begin{itemize}
\item {Utilização:Bras}
\end{itemize}
\begin{itemize}
\item {Utilização:pop.}
\end{itemize}
Peça de barro cozido, geralmente rectangular, e destinada a construcções.
Instrumento do ourivezaria, para vazar arruelas.
Doce de goiaba, ou de outras frutas, consistente e de fórma semelhante á dos tijolos de barro.
\textunderscore Fazer tijolo\textunderscore , estar sepultado.
\textunderscore Fazer tijolo\textunderscore , namorar.
\textunderscore Tijolo burro\textunderscore , espécie de tijolo grosseiro.
(Cp. \textunderscore tégula\textunderscore )
\section{Tijuca}
\begin{itemize}
\item {Grp. gram.:f.}
\end{itemize}
(V.tijuco)
\section{Tijucal}
\begin{itemize}
\item {Grp. gram.:m.}
\end{itemize}
\begin{itemize}
\item {Utilização:Bras}
\end{itemize}
\begin{itemize}
\item {Proveniência:(De \textunderscore tijuco\textunderscore )}
\end{itemize}
Lameiro; pântano.
\section{Tijucano}
\begin{itemize}
\item {Grp. gram.:adj.}
\end{itemize}
Relativo a tijuco: \textunderscore solo tijucano\textunderscore .
\section{Tijuco}
\begin{itemize}
\item {Grp. gram.:m.}
\end{itemize}
\begin{itemize}
\item {Utilização:Bras}
\end{itemize}
Lameiro; atoleiro.
(Do tupi \textunderscore tuiuca\textunderscore )
\section{Tijucopáua}
\begin{itemize}
\item {Grp. gram.:f.}
\end{itemize}
\begin{itemize}
\item {Utilização:Bras. do N}
\end{itemize}
Praia de tijuco; tremedal.
O mesmo que \textunderscore tijucal\textunderscore .
(Do tupi \textunderscore tuiuca\textunderscore  + \textunderscore páua\textunderscore )
\section{Tijupá}
\begin{itemize}
\item {Grp. gram.:m.}
\end{itemize}
\begin{itemize}
\item {Utilização:Bras. do N}
\end{itemize}
Palhoça com duas vertentes que tocam no chão, e na qual se abrigam trabalhadores.
(Do tupi)
\section{Tijupar}
\begin{itemize}
\item {Grp. gram.:m.}
\end{itemize}
\begin{itemize}
\item {Utilização:Bras}
\end{itemize}
O mesmo que \textunderscore tijupá\textunderscore .
\section{Til}
\begin{itemize}
\item {Grp. gram.:m.}
\end{itemize}
\begin{itemize}
\item {Utilização:Fig.}
\end{itemize}
Sinal que, em Orthographia, serve para nasalar a vogal a que se sobrepõe.
Insignificância, bagatela.
(Cast. \textunderscore tilde\textunderscore , do lat. \textunderscore titulum\textunderscore )
\section{Til}
\begin{itemize}
\item {Grp. gram.:m.}
\end{itemize}
\begin{itemize}
\item {Utilização:Poét.}
\end{itemize}
O mesmo que \textunderscore tília\textunderscore .
\section{Tilápia}
\begin{itemize}
\item {Grp. gram.:f.}
\end{itemize}
Peixe do Sul da África.
\section{Tilar}
\begin{itemize}
\item {Grp. gram.:v.}
\end{itemize}
\begin{itemize}
\item {Utilização:t. Gram.}
\end{itemize}
O mesmo que \textunderscore tildar\textunderscore . Cf. J. de Deus. \textunderscore Cart. Maternal\textunderscore , 93.
\section{Tildar}
\begin{itemize}
\item {Grp. gram.:v.}
\end{itemize}
\begin{itemize}
\item {Utilização:t. Gram.}
\end{itemize}
\begin{itemize}
\item {Proveniência:(Do cast. \textunderscore tilde\textunderscore )}
\end{itemize}
Pôr til em.
\section{Tílea}
\begin{itemize}
\item {Grp. gram.:f.}
\end{itemize}
Gênero de plantas crassuláceas.
\section{Tilha}
\begin{itemize}
\item {Grp. gram.:f.}
\end{itemize}
\begin{itemize}
\item {Utilização:Náut.}
\end{itemize}
Coberta de navio.
(Do germ.?)
\section{Tilhado}
\begin{itemize}
\item {Grp. gram.:adj.}
\end{itemize}
Que tem tilha.
\section{Tilho}
\begin{itemize}
\item {Grp. gram.:m.}
\end{itemize}
\begin{itemize}
\item {Proveniência:(De \textunderscore Tilly\textunderscore , n. p.)}
\end{itemize}
Planta exótica, purgativa.
\section{Tília}
\begin{itemize}
\item {Grp. gram.:f.}
\end{itemize}
\begin{itemize}
\item {Proveniência:(Lat. \textunderscore tilia\textunderscore )}
\end{itemize}
Gênero de plantas ornamentaes, (\textunderscore tilia silvestris\textunderscore ), cujas fôlhas são medicinaes.
\section{Tiliáceas}
\begin{itemize}
\item {Grp. gram.:f. pl.}
\end{itemize}
\begin{itemize}
\item {Proveniência:(Do lat. \textunderscore tiliaceus\textunderscore )}
\end{itemize}
Família de plantas, que tem por typo a tília.
\section{Tilintar}
\begin{itemize}
\item {Grp. gram.:v. i.}
\end{itemize}
O mesmo que \textunderscore telintar\textunderscore .
\section{Tíllea}
\begin{itemize}
\item {Grp. gram.:f.}
\end{itemize}
Gênero de plantas crassuláceas.
\section{Tillo}
\begin{itemize}
\item {Grp. gram.:m.}
\end{itemize}
Gênero de insectos coleópteros.
(Do germ.)
\section{Tillo}
\begin{itemize}
\item {Grp. gram.:m.}
\end{itemize}
Deus dos Amatongas, em Lourenço Marques.
\section{Tilo}
\begin{itemize}
\item {Grp. gram.:m.}
\end{itemize}
Gênero de insectos coleópteros.
(Do germ.)
\section{Tilo}
\begin{itemize}
\item {Grp. gram.:m.}
\end{itemize}
Deus dos Amatongas, em Lourenço Marques.
\section{Timanaras}
\begin{itemize}
\item {Grp. gram.:m. pl.}
\end{itemize}
\begin{itemize}
\item {Utilização:Bras}
\end{itemize}
Tríbo de aborígenes do Pará.
\section{Timão}
\begin{itemize}
\item {Grp. gram.:m.}
\end{itemize}
O mesmo que \textunderscore temão\textunderscore .
\section{Timão}
\begin{itemize}
\item {Grp. gram.:m.}
\end{itemize}
\begin{itemize}
\item {Utilização:Bras}
\end{itemize}
Camisola de menino.
(Talvez alter. de \textunderscore quimão\textunderscore )
\section{Timaquete}
\begin{itemize}
\item {fónica:quê}
\end{itemize}
\begin{itemize}
\item {Grp. gram.:m.}
\end{itemize}
\begin{itemize}
\item {Utilização:Ant.}
\end{itemize}
Qualquer artefacto que, movendo-se, faz estrondo, como são os cascavéis, as soalhas, etc.
\section{Timbalão}
\begin{itemize}
\item {Grp. gram.:m.}
\end{itemize}
\begin{itemize}
\item {Proveniência:(De \textunderscore timbale\textunderscore )}
\end{itemize}
Caixa de rufo.
\section{Timbale}
\begin{itemize}
\item {Grp. gram.:m.}
\end{itemize}
Espécie de tambor de metal, em fórma de meio globo e coberto de uma pelle tensa, sôbre que se toca.
Tambor de cavallaria.
Espécie de empada.
(Por \textunderscore atabale\textunderscore , do ár. \textunderscore tabal\textunderscore )
\section{Timbaleiro}
\begin{itemize}
\item {Grp. gram.:m.}
\end{itemize}
Aquelle que toca timbales.
\section{Timbatu}
\begin{itemize}
\item {Grp. gram.:m.}
\end{itemize}
Instrumento de teclas, entre os indígenas do Amazonas.
\section{Timbaúba}
\begin{itemize}
\item {Grp. gram.:f.}
\end{itemize}
Árvore leguminosa das florestas brasileiras.
\section{Timbérgia}
\begin{itemize}
\item {Grp. gram.:f.}
\end{itemize}
\begin{itemize}
\item {Proveniência:(De \textunderscore Thimberg\textunderscore , n. p.)}
\end{itemize}
Gênero de plantas de jardim.
\section{Timbiras}
\begin{itemize}
\item {Grp. gram.:f. pl.}
\end{itemize}
Tribo selvagem, que vive ao sudoéste do Maranhão.
\section{Timbó}
\begin{itemize}
\item {Grp. gram.:m.}
\end{itemize}
\begin{itemize}
\item {Utilização:Bras}
\end{itemize}
Arbusto sapindácio.
Planta solânea, também conhecida por canapu, no Brasil.
Processo de pescar, narcotizando-se os peixes, por meio de plantas venenosas, maceradas, cuja seiva se mistura com a água. Cf. \textunderscore Jorn.-do-Com.\textunderscore , de 12-XI-901.
\section{Timbopeba}
\begin{itemize}
\item {Grp. gram.:f.}
\end{itemize}
\begin{itemize}
\item {Utilização:Bras}
\end{itemize}
Espécie de cipó.
\section{Timbragem}
\begin{itemize}
\item {Grp. gram.:f.}
\end{itemize}
Acto de timbrar.
\section{Timbrar}
\begin{itemize}
\item {Grp. gram.:v. t.}
\end{itemize}
\begin{itemize}
\item {Grp. gram.:V. i.}
\end{itemize}
Fazer timbre a, marcar com timbre.
Apodar, qualificar.
Caprichar; ufanar-se: \textunderscore timbrar de honrado\textunderscore .
\section{Timbre}
\begin{itemize}
\item {Grp. gram.:m.}
\end{itemize}
\begin{itemize}
\item {Utilização:Heráld.}
\end{itemize}
\begin{itemize}
\item {Utilização:Ext.}
\end{itemize}
\begin{itemize}
\item {Utilização:Fig.}
\end{itemize}
\begin{itemize}
\item {Grp. gram.:Pl.}
\end{itemize}
\begin{itemize}
\item {Utilização:Prov.}
\end{itemize}
\begin{itemize}
\item {Proveniência:(Fr. \textunderscore timbre\textunderscore )}
\end{itemize}
Insígnia, appensa exteriormente aos escudos.
Instrumento metállico de percussão, que se percute com martelo, também de metal, e que tem a fórma de um pequeno hemisphério, com a parte convexa voltada para cima.
Insígnia, marca.
Divisa.
Sêllo, carimbo.
Divisa honrosa.
Honra.
Capricho.
Orgulho legítimo.
Remate.
Cúmulo.
Qualidade sonora de uma voz ou de um instrumento.
Birras; caprichos; teimas de criança.
\section{Timbri}
\begin{itemize}
\item {Grp. gram.:m.}
\end{itemize}
Árvore intertropical, de pequenas dimensões, (\textunderscore diospyrus montana\textunderscore ), cuja madeira é preta como ébano.
(Do concani)
\section{Timbroso}
\begin{itemize}
\item {Grp. gram.:adj.}
\end{itemize}
\begin{itemize}
\item {Proveniência:(De \textunderscore timbre\textunderscore )}
\end{itemize}
Que tem timbre ou capricho.
Meticuloso; escrupuloso.
\section{Timbu}
\begin{itemize}
\item {Grp. gram.:m.}
\end{itemize}
\begin{itemize}
\item {Utilização:Bras. do N}
\end{itemize}
O mesmo que \textunderscore sarigueia\textunderscore .
\section{Timbuíba}
\begin{itemize}
\item {Grp. gram.:f.}
\end{itemize}
\begin{itemize}
\item {Utilização:Bras}
\end{itemize}
Gênero de árvores silvestres.
\section{Timburé}
\begin{itemize}
\item {Grp. gram.:m.}
\end{itemize}
\begin{itemize}
\item {Utilização:Bras}
\end{itemize}
Peixe de água doce.
\section{Timburi}
\begin{itemize}
\item {Grp. gram.:m.}
\end{itemize}
Planta leguminosa.
\section{Timena}
\begin{itemize}
\item {Grp. gram.:f.}
\end{itemize}
Gênero de insectos lepidópteros.
\section{Timidamente}
\begin{itemize}
\item {Grp. gram.:adv.}
\end{itemize}
De modo tímido; com timidez; com acanhamento.
\section{Timidez}
\begin{itemize}
\item {Grp. gram.:f.}
\end{itemize}
Qualidade do que é tímido; fraqueza; acanhamento.
\section{Tímido}
\begin{itemize}
\item {Grp. gram.:adj.}
\end{itemize}
\begin{itemize}
\item {Utilização:Fig.}
\end{itemize}
\begin{itemize}
\item {Grp. gram.:M.}
\end{itemize}
\begin{itemize}
\item {Proveniência:(Lat. \textunderscore timidus\textunderscore )}
\end{itemize}
Que tem temor ou mêdo.
Receoso; acanhado.
Froixo, débil.
Indivíduo acanhado ou cobarde.
\section{Timoneiro}
\begin{itemize}
\item {Grp. gram.:m.}
\end{itemize}
O mesmo que \textunderscore temoneiro\textunderscore .
\section{Timor}
\begin{itemize}
\item {Grp. gram.:m.  e  f.}
\end{itemize}
Habitante de Timor. Cf. Herculano, \textunderscore Reacção Ultram.\textunderscore , 23.
\section{Timoratamente}
\begin{itemize}
\item {Grp. gram.:adv.}
\end{itemize}
De modo timorato; timidamente.
\section{Timorato}
\begin{itemize}
\item {Grp. gram.:adj.}
\end{itemize}
\begin{itemize}
\item {Proveniência:(Lat. \textunderscore timoratus\textunderscore )}
\end{itemize}
Tímido.
Que receia errar.
Escrupuloso.
\section{Timorense}
\begin{itemize}
\item {Grp. gram.:adj.}
\end{itemize}
\begin{itemize}
\item {Grp. gram.:M.  e  f.}
\end{itemize}
Relativo a Timor.
O mesmo que \textunderscore timor\textunderscore .
\section{Tímoro}
\begin{itemize}
\item {Grp. gram.:m.}
\end{itemize}
Gênero de insectos coleópteros tetrâmeros.
\section{Timpabeba}
\begin{itemize}
\item {Grp. gram.:f.}
\end{itemize}
Planta berberídea.
\section{Tim-tim}
(V.tentim)
\section{Timucu}
\begin{itemize}
\item {Grp. gram.:m.}
\end{itemize}
Peixe malacopterýgio do Brasil.
\section{Tina}
\begin{itemize}
\item {Grp. gram.:f.}
\end{itemize}
\begin{itemize}
\item {Proveniência:(Do lat. \textunderscore tinna\textunderscore , por \textunderscore tina\textunderscore ?)}
\end{itemize}
Vasilha de aduelas, espécie de dorna.
Vaso de pedra ou de metal, em que se tomam banhos; banheira.
\section{Tinada}
\begin{itemize}
\item {Grp. gram.:f.}
\end{itemize}
Conteúdo de uma tina.
\section{Tinalha}
\begin{itemize}
\item {Grp. gram.:f.}
\end{itemize}
Tina pequena para vinho; dorna.
(Cp. cast. \textunderscore tinaja\textunderscore )
\section{Tinamu}
\begin{itemize}
\item {Grp. gram.:m.}
\end{itemize}
Ave dos climas quentes da América, um pouco semelhante á betarda e á perdiz, mas não gallinácea, como o consideram diccionários.
\section{Tinca}
\begin{itemize}
\item {Grp. gram.:f.}
\end{itemize}
(V.tenca)
\section{Tincal}
\begin{itemize}
\item {Grp. gram.:m.}
\end{itemize}
\begin{itemize}
\item {Proveniência:(Do ár. \textunderscore at-tencar\textunderscore )}
\end{itemize}
Borato de soda, empregado em certas indústrias, para soldar metaes.
\section{Tincaleira}
\begin{itemize}
\item {Grp. gram.:f.}
\end{itemize}
\begin{itemize}
\item {Proveniência:(De \textunderscore tincal\textunderscore )}
\end{itemize}
Vasilha, em que os ourives deitam o tincal.
\section{Tinção}
\begin{itemize}
\item {Grp. gram.:f.}
\end{itemize}
\begin{itemize}
\item {Proveniência:(Do lat. \textunderscore tinctio\textunderscore )}
\end{itemize}
Acto ou effeito de tingir; tintura.
\section{Tincar}
\begin{itemize}
\item {Grp. gram.:m.}
\end{itemize}
O mesmo que \textunderscore tincal\textunderscore .
\section{Tincção}
\begin{itemize}
\item {Grp. gram.:f.}
\end{itemize}
\begin{itemize}
\item {Proveniência:(Do lat. \textunderscore tinctio\textunderscore )}
\end{itemize}
Acto ou effeito de tingir; tintura.
\section{Tinctorial}
\begin{itemize}
\item {Grp. gram.:adj.}
\end{itemize}
\begin{itemize}
\item {Proveniência:(Do lat. \textunderscore tinctor\textunderscore )}
\end{itemize}
Que serve para tingir.
Relativo a tinturaria.
\section{Tinctório}
\begin{itemize}
\item {Grp. gram.:adj.}
\end{itemize}
\begin{itemize}
\item {Proveniência:(Lat. \textunderscore tinctorius\textunderscore )}
\end{itemize}
Que serve para tingir.
Que produz substância usada em tinturaria.
O mesmo que \textunderscore tinctorial\textunderscore .
\section{Tinebra}
\begin{itemize}
\item {Grp. gram.:f.}
\end{itemize}
\begin{itemize}
\item {Utilização:Prov.}
\end{itemize}
\begin{itemize}
\item {Utilização:alent.}
\end{itemize}
Temeridade; brincadeira arriscada.
\section{Tineira}
\begin{itemize}
\item {Grp. gram.:f.}
\end{itemize}
\begin{itemize}
\item {Utilização:Prov.}
\end{itemize}
\begin{itemize}
\item {Utilização:trasm.}
\end{itemize}
Fôrça; intensidade.
(Por \textunderscore teneira\textunderscore , do lat. \textunderscore tenor\textunderscore ?)
\section{Tineleiro}
\begin{itemize}
\item {Grp. gram.:m.}
\end{itemize}
\begin{itemize}
\item {Grp. gram.:Adj.}
\end{itemize}
Aquele que trata do tinelo.
Relativo ao tinelo.
\section{Tinelleiro}
\begin{itemize}
\item {Grp. gram.:m.}
\end{itemize}
\begin{itemize}
\item {Grp. gram.:Adj.}
\end{itemize}
Aquelle que trata do tinello.
Relativo ao tinello.
\section{Tinello}
\begin{itemize}
\item {Grp. gram.:m.}
\end{itemize}
\begin{itemize}
\item {Proveniência:(It. \textunderscore tinello\textunderscore )}
\end{itemize}
O mesmo que \textunderscore refeitório\textunderscore .
Sala, em que os criados de uma casa comem em commum.
\section{Tinelo}
\begin{itemize}
\item {Grp. gram.:m.}
\end{itemize}
\begin{itemize}
\item {Proveniência:(It. \textunderscore tinello\textunderscore )}
\end{itemize}
O mesmo que \textunderscore refeitório\textunderscore .
Sala, em que os criados de uma casa comem em comum.
\section{Tinente}
\begin{itemize}
\item {Grp. gram.:adj.}
\end{itemize}
\begin{itemize}
\item {Utilização:Gír.}
\end{itemize}
Que tem tino, que é esperto.
\section{Tineta}
\begin{itemize}
\item {fónica:nê}
\end{itemize}
\begin{itemize}
\item {Grp. gram.:f.}
\end{itemize}
\begin{itemize}
\item {Utilização:Fam.}
\end{itemize}
\begin{itemize}
\item {Proveniência:(De \textunderscore tino\textunderscore ?)}
\end{itemize}
Mania; veneta.
Tendência.
Habilidade.
Teimosia.
\section{Tinga}
\begin{itemize}
\item {Grp. gram.:f.}
\end{itemize}
\begin{itemize}
\item {Utilização:Bras}
\end{itemize}
Espécie de cágado das regiões do Amazonas.
\section{...tinga}
\begin{itemize}
\item {Grp. gram.:suf.}
\end{itemize}
(us. em vocab. brasileiros e designativo de \textunderscore branco\textunderscore )
\section{Tingará}
\begin{itemize}
\item {Grp. gram.:m.}
\end{itemize}
\begin{itemize}
\item {Utilização:Bras}
\end{itemize}
Passarinho verde, de cabeça vermelha.
\section{Tingarra}
\begin{itemize}
\item {Grp. gram.:f.}
\end{itemize}
\begin{itemize}
\item {Grp. gram.:f.}
\end{itemize}
\begin{itemize}
\item {Utilização:Prov.}
\end{itemize}
\begin{itemize}
\item {Utilização:alg.}
\end{itemize}
O mesmo que \textunderscore tangarra\textunderscore .
Planta, de que se faz salada.
\section{Tingar-se}
\begin{itemize}
\item {Grp. gram.:v. p.}
\end{itemize}
\begin{itemize}
\item {Utilização:Pop.}
\end{itemize}
Escapar-se, fugir.
\section{Tinge-burro}
\begin{itemize}
\item {Grp. gram.:m.}
\end{itemize}
\begin{itemize}
\item {Utilização:Mad}
\end{itemize}
\begin{itemize}
\item {Proveniência:(De \textunderscore tanger\textunderscore  + \textunderscore burro\textunderscore ? Cp. \textunderscore tanjasno\textunderscore )}
\end{itemize}
Espécie de ave, (\textunderscore sylvia conspicillata\textunderscore , Marm.).
\section{Tinge-cuia}
\begin{itemize}
\item {Grp. gram.:m.}
\end{itemize}
(V. \textunderscore papeira\textunderscore , planta)
\section{Tingidor}
\begin{itemize}
\item {Grp. gram.:m.  e  adj.}
\end{itemize}
O que tinge.
\section{Tingidura}
\begin{itemize}
\item {Grp. gram.:f.}
\end{itemize}
O mesmo que \textunderscore tintura\textunderscore .
\section{Tingir}
\begin{itemize}
\item {Grp. gram.:v. t.}
\end{itemize}
\begin{itemize}
\item {Grp. gram.:V. p.}
\end{itemize}
\begin{itemize}
\item {Proveniência:(Lat. \textunderscore tingere\textunderscore )}
\end{itemize}
Meter em tinta ou molhar com tinta, alterando a côr primitiva.
Dar certa côr a.
Colorir.
Tornar preto.
Tomar certa côr.
\section{Tingitano}
\begin{itemize}
\item {Grp. gram.:m.}
\end{itemize}
\begin{itemize}
\item {Proveniência:(Lat. \textunderscore tingitanus\textunderscore )}
\end{itemize}
Relativo a Tânger.
\section{Tinguaci}
\begin{itemize}
\item {Grp. gram.:m.}
\end{itemize}
\begin{itemize}
\item {Utilização:Bras}
\end{itemize}
Árvore rutácea.
\section{Tingueiro}
\begin{itemize}
\item {Grp. gram.:m.}
\end{itemize}
\begin{itemize}
\item {Grp. gram.:M.}
\end{itemize}
\begin{itemize}
\item {Grp. gram.:Adj.}
\end{itemize}
Pequena embarcação do Tejo.
O homem que tripula essa embarcação.
Relativo á mesma embarcação.
\section{Tingui}
\begin{itemize}
\item {Grp. gram.:m.}
\end{itemize}
\begin{itemize}
\item {Proveniência:(T. tupi)}
\end{itemize}
Arbusto leguminoso.
Arbusto silvestre do norte do Brasil.
Nome commum a vários vegetaes que, á semelhança do trovisco, e lançados á água, matam o peixe.
\section{Tinguijada}
\begin{itemize}
\item {Grp. gram.:f.}
\end{itemize}
\begin{itemize}
\item {Utilização:Bras}
\end{itemize}
\begin{itemize}
\item {Proveniência:(De \textunderscore tinguijar\textunderscore )}
\end{itemize}
Pescaria, feita com o emprêgo de tinguí.
\section{Tinguijar}
\begin{itemize}
\item {Grp. gram.:v. t.}
\end{itemize}
\begin{itemize}
\item {Utilização:Bras}
\end{itemize}
\begin{itemize}
\item {Grp. gram.:V. i.}
\end{itemize}
\begin{itemize}
\item {Proveniência:(De \textunderscore tinguí\textunderscore )}
\end{itemize}
Pescar (peixe), envenenando-o com tinguí.
Deitar tinguí em (os rios), para envenenar e pescar peixe.
Sêr envenenado com tinguí, (falando-se de peixes).
\section{Tinha}
\begin{itemize}
\item {Grp. gram.:f.}
\end{itemize}
\begin{itemize}
\item {Utilização:Fig.}
\end{itemize}
\begin{itemize}
\item {Proveniência:(Do lat. \textunderscore tinea\textunderscore )}
\end{itemize}
Moléstia cutânea da cabeça.
Defeito, mácula.
\section{Tinha}
\begin{itemize}
\item {Grp. gram.:f.}
\end{itemize}
O mesmo que \textunderscore tina\textunderscore .
\section{Tinhó}
\begin{itemize}
\item {Grp. gram.:m.}
\end{itemize}
\begin{itemize}
\item {Proveniência:(Do lat. hyp. \textunderscore tineola\textunderscore , de \textunderscore tinea\textunderscore )}
\end{itemize}
Moléstia cutânea.
\section{Tinhorão}
\begin{itemize}
\item {Grp. gram.:m.}
\end{itemize}
Planta arácea do Brasil.
\section{Tinhosa}
\begin{itemize}
\item {Grp. gram.:f.}
\end{itemize}
\begin{itemize}
\item {Utilização:Prov.}
\end{itemize}
\begin{itemize}
\item {Utilização:trasm.}
\end{itemize}
\begin{itemize}
\item {Proveniência:(De \textunderscore tinhoso\textunderscore )}
\end{itemize}
Espécie de tortulho venenoso.
\section{Tinhoso}
\begin{itemize}
\item {Grp. gram.:adj.}
\end{itemize}
\begin{itemize}
\item {Utilização:Fig.}
\end{itemize}
\begin{itemize}
\item {Grp. gram.:M.}
\end{itemize}
\begin{itemize}
\item {Utilização:Pop.}
\end{itemize}
\begin{itemize}
\item {Utilização:Pop.}
\end{itemize}
Que tem tinha.
Que mete nojo.
Repugnante.
Aquelle que soffre tinha.
O Diabo.
\textunderscore Cão tinhoso\textunderscore , o Diabo.
(Do lat. \textunderscore tineosus\textunderscore )
\section{Tinido}
\begin{itemize}
\item {Grp. gram.:m.}
\end{itemize}
Acto ou effeito de tinir; som vibrante de vidro ou metal.
\section{Tinidor}
\begin{itemize}
\item {Grp. gram.:m.  e  adj.}
\end{itemize}
O que tine.
\section{Tinilho}
\begin{itemize}
\item {Grp. gram.:m.}
\end{itemize}
\begin{itemize}
\item {Proveniência:(Do lat. \textunderscore tinus\textunderscore )}
\end{itemize}
Espécie de loireiro silvestre.
\section{Tininte}
\begin{itemize}
\item {Grp. gram.:adj.}
\end{itemize}
Que tine.
\section{Tinir}
\begin{itemize}
\item {Grp. gram.:v. i.}
\end{itemize}
\begin{itemize}
\item {Utilização:Pop.}
\end{itemize}
\begin{itemize}
\item {Utilização:Chul.}
\end{itemize}
\begin{itemize}
\item {Proveniência:(Do lat. \textunderscore tinnire\textunderscore )}
\end{itemize}
Soar aguda ou vibrantemente, (falando-se do vidro ou do metal).
Zunir, (falando-se dos ouvidos).
Tremer com frio ou mêdo.
Não têr dinheiro: \textunderscore aquelle anda a tinir\textunderscore .
Fazer ouvir a sua voz (a milheira).
\section{Tinjema}
\begin{itemize}
\item {Grp. gram.:f.}
\end{itemize}
Árvore moçambicana, cuja madeira é excellente para construcções.
\section{Tino}
\begin{itemize}
\item {Grp. gram.:m.}
\end{itemize}
Juizo natural.
Discrição.
Prudência.
Circunspecção.
Tacto.
Sentido.
Attenção.
Conhecimento, ideia.
Facilidade em andar ás escuras.
\textunderscore Dar tino\textunderscore , dar fé, notar:«\textunderscore assim que deu tino do olhar petulante do desconhecido...\textunderscore »Camillo, \textunderscore Retrat. de Ricard.\textunderscore , 219.
\section{Tino}
\begin{itemize}
\item {Grp. gram.:m.}
\end{itemize}
\begin{itemize}
\item {Utilização:Des.}
\end{itemize}
\begin{itemize}
\item {Proveniência:(De \textunderscore tinir\textunderscore )}
\end{itemize}
O mesmo que \textunderscore tinido\textunderscore .
\section{Tino}
\begin{itemize}
\item {Grp. gram.:m.}
\end{itemize}
O mesmo que \textunderscore tina\textunderscore .
\section{Tinoca}
\begin{itemize}
\item {Grp. gram.:f.}
\end{itemize}
\begin{itemize}
\item {Utilização:Prov.}
\end{itemize}
Tino, bom senso.
\section{Tinócoro}
\begin{itemize}
\item {Grp. gram.:m.}
\end{itemize}
Gênero de aves galináceas.
\section{Tinor}
\begin{itemize}
\item {Grp. gram.:m.}
\end{itemize}
\begin{itemize}
\item {Utilização:Prov.}
\end{itemize}
Tino, tento.
\section{Tinor}
\begin{itemize}
\item {Grp. gram.:m.}
\end{itemize}
\begin{itemize}
\item {Utilização:T. de Serpa}
\end{itemize}
Vasilha de barro, em que se guarda o mel ou o pingo. Cf. Rev. \textunderscore Tradição\textunderscore , II, 11.
\section{Tinote}
\begin{itemize}
\item {Grp. gram.:m.}
\end{itemize}
\begin{itemize}
\item {Utilização:Pop.}
\end{itemize}
\begin{itemize}
\item {Proveniência:(De \textunderscore tino\textunderscore ^1)}
\end{itemize}
O cérebro.
\section{Tinote}
\begin{itemize}
\item {Grp. gram.:m.}
\end{itemize}
\begin{itemize}
\item {Proveniência:(De \textunderscore tina\textunderscore )}
\end{itemize}
Pequena tina; cuba ou selha.
\section{Tinqui}
\begin{itemize}
\item {Grp. gram.:m.}
\end{itemize}
\begin{itemize}
\item {Utilização:Bras}
\end{itemize}
(V.tingui)
\section{Tinta}
\begin{itemize}
\item {Grp. gram.:f.}
\end{itemize}
\begin{itemize}
\item {Utilização:Açor}
\end{itemize}
\begin{itemize}
\item {Utilização:Fam.}
\end{itemize}
\begin{itemize}
\item {Proveniência:(De \textunderscore tinto\textunderscore )}
\end{itemize}
Líquido de qualquer côr, para escrever, imprimir ou tingir.
Vestígio.
Pequena dose.
Tintura ou laivo.
Matiz.
Casta de uva, o mesmo que \textunderscore sousão\textunderscore .
As partes pudendas da mulhér.
\textunderscore Estar na tinta\textunderscore  ou \textunderscore estar nas tintas\textunderscore , não se importar, sêr indifferente.
\section{Tinta-albanesa}
\begin{itemize}
\item {Grp. gram.:f.}
\end{itemize}
Casta de uva alentejana.
\section{Tinta-amarela}
\begin{itemize}
\item {Grp. gram.:f.}
\end{itemize}
Casta de uva preta, beirôa e duriense.
\section{Tinta-aragonesa}
\begin{itemize}
\item {Grp. gram.:f.}
\end{itemize}
Casta de uva, também conhecida por \textunderscore aragonês\textunderscore .
\section{Tinta-bastarda}
\begin{itemize}
\item {Grp. gram.:f.}
\end{itemize}
Casta de uva, o mesmo que \textunderscore bastardeira\textunderscore .
\section{Tinta-cachuda}
\begin{itemize}
\item {Grp. gram.:f.}
\end{itemize}
Casta de uva de Azeitão.
\section{Tinta-carvalha}
\begin{itemize}
\item {Grp. gram.:f.}
\end{itemize}
Casta de uva preta da região do Doiro.
\section{Tinta-castellan}
\begin{itemize}
\item {Grp. gram.:f.}
\end{itemize}
Casta de uva.
\section{Tinta-castellôa}
\begin{itemize}
\item {Grp. gram.:f.}
\end{itemize}
Casta de uva.
\section{Tinta-commum}
\begin{itemize}
\item {Grp. gram.:f.}
\end{itemize}
Casta de uva do Cartaxo.
\section{Tinta-consoeira}
\begin{itemize}
\item {Grp. gram.:f.}
\end{itemize}
Casta de uva preta.
\section{Tinta-de-castella}
\begin{itemize}
\item {Grp. gram.:f.}
\end{itemize}
O mesmo que \textunderscore tinta-castellan\textunderscore .
\section{Tinta-de-manuel-pereira}
\begin{itemize}
\item {Grp. gram.:f.}
\end{itemize}
Casta de uva preta.
\section{Tinta-de-murteira}
\begin{itemize}
\item {Grp. gram.:f.}
\end{itemize}
Casta de uva de Azeitão.
\section{Tinta-de-pé-curto}
\begin{itemize}
\item {Grp. gram.:f.}
\end{itemize}
Casta de uva preta do Cartaxo.
\section{Tinta-de-santiago}
\begin{itemize}
\item {Grp. gram.:f.}
\end{itemize}
Casta de uva de Azeitão.
\section{Tinta-do-gregório}
\begin{itemize}
\item {Grp. gram.:f.}
\end{itemize}
Casta de uva de Alcácer.
\section{Tinta-do-lameiro}
\begin{itemize}
\item {Grp. gram.:f.}
\end{itemize}
O mesmo que \textunderscore tinta-lameira\textunderscore .
\section{Tinta-do-languedoque}
\begin{itemize}
\item {Grp. gram.:f.}
\end{itemize}
Casta de uva preta de Azeitão.
\section{Tinta-do-padre-antónio}
\begin{itemize}
\item {Grp. gram.:f.}
\end{itemize}
Casta de uva preta da Arruda e de outros sítios.
\section{Tinta-do-pinhão}
\begin{itemize}
\item {Grp. gram.:f.}
\end{itemize}
\begin{itemize}
\item {Utilização:T. da Bairrada}
\end{itemize}
O mesmo que \textunderscore tinta-pinheira\textunderscore .
\section{Tinta-dos-pobres}
\begin{itemize}
\item {Grp. gram.:f.}
\end{itemize}
Nome, que no Doiro se dá á nevoeira.
\section{Tinta-espadeira}
\begin{itemize}
\item {Grp. gram.:f.}
\end{itemize}
Bôa casta de uva preta.
\section{Tinta-fina}
\begin{itemize}
\item {Grp. gram.:f.}
\end{itemize}
Casta de uva preta do Alentejo.
\section{Tinta-francesa}
\begin{itemize}
\item {Grp. gram.:f.}
\end{itemize}
Casta de uva preta de Tôrres-Novas, talvez o mesmo que \textunderscore negrão-francês\textunderscore .
\section{Tinta-francisca}
\begin{itemize}
\item {Grp. gram.:f.}
\end{itemize}
Casta de uva preta beirôa e duriense.
\section{Tinta-franciscana}
\begin{itemize}
\item {Grp. gram.:f.}
\end{itemize}
O mesmo que \textunderscore tinta-francisca\textunderscore .
\section{Tinta-geral}
\begin{itemize}
\item {Grp. gram.:f.}
\end{itemize}
Casta de uva do Cartaxo.
\section{Tinta-gorda}
\begin{itemize}
\item {Grp. gram.:f.}
\end{itemize}
Casta de uva preta do Cartaxo.
\section{Tinta-grossa}
\begin{itemize}
\item {Grp. gram.:f.}
\end{itemize}
Casta de uva algarvia.
\section{Tinta-imperial}
\begin{itemize}
\item {Grp. gram.:f.}
\end{itemize}
Casta de uva preta de Azeitão.
\section{Tinta-índica}
\begin{itemize}
\item {Grp. gram.:f.}
\end{itemize}
O mesmo que \textunderscore erva-impigem\textunderscore .
\section{Tinta-lameira}
\begin{itemize}
\item {Grp. gram.:f.}
\end{itemize}
Casta de uva preta da região do Doiro.
\section{Tinta-miúda}
\begin{itemize}
\item {Grp. gram.:f.}
\end{itemize}
Casta de uva de Tôrres-Vedras.
\section{Tinta-mollar}
\begin{itemize}
\item {Grp. gram.:f.}
\end{itemize}
Uma das melhores castas de uva preta.
\section{Tinta-molle}
\begin{itemize}
\item {Grp. gram.:f.}
\end{itemize}
O mesmo que \textunderscore tinta-sobreirinha\textunderscore .
\section{Tinta-morela}
\begin{itemize}
\item {Grp. gram.:f.}
\end{itemize}
Casta de uva preta do Alto Doiro.
\section{Tinta-musguenta}
\begin{itemize}
\item {Grp. gram.:f.}
\end{itemize}
Casta de uva preta.
\section{Tinta-patorra}
\begin{itemize}
\item {Grp. gram.:f.}
\end{itemize}
Casta de uva preta do Alto Doiro.
\section{Tinta-peral}
\begin{itemize}
\item {Grp. gram.:f.}
\end{itemize}
Casta de uva preta do Cartaxo.
\section{Tinta-pinheira}
\begin{itemize}
\item {Grp. gram.:f.}
\end{itemize}
Casta de uva preta da região do Doiro.
\section{Tintar}
\begin{itemize}
\item {Grp. gram.:v. t.}
\end{itemize}
\begin{itemize}
\item {Utilização:Prov.}
\end{itemize}
\begin{itemize}
\item {Proveniência:(De \textunderscore tinta\textunderscore )}
\end{itemize}
O mesmo que \textunderscore tingir\textunderscore .
\section{Tinta-rija}
\begin{itemize}
\item {Grp. gram.:f.}
\end{itemize}
Variedade de uva.
\section{Tinta-sobreirinha}
\begin{itemize}
\item {Grp. gram.:f.}
\end{itemize}
\begin{itemize}
\item {Utilização:Prov.}
\end{itemize}
\begin{itemize}
\item {Utilização:beir.}
\end{itemize}
Variedade de uva tinta.
\section{Tinta-vianesa}
\begin{itemize}
\item {Grp. gram.:f.}
\end{itemize}
Casta de uva preta da região do Doiro.
\section{Tinta-vigária}
\begin{itemize}
\item {Grp. gram.:f.}
\end{itemize}
O mesmo que \textunderscore bôca-de-mina\textunderscore .
\section{Tinte}
\begin{itemize}
\item {Grp. gram.:m.}
\end{itemize}
\begin{itemize}
\item {Utilização:Des.}
\end{itemize}
\begin{itemize}
\item {Proveniência:(T. cast.)}
\end{itemize}
O mesmo que \textunderscore tinturaria\textunderscore .
\section{Tinteiro}
\begin{itemize}
\item {Grp. gram.:m.}
\end{itemize}
\begin{itemize}
\item {Proveniência:(De \textunderscore tinta\textunderscore )}
\end{itemize}
Pequeno vaso, para conter tinta de escrever.
Utensílio de escritório, com um ou mais vasos para tinta de escrever.
\section{Tintenanim}
\begin{itemize}
\item {Grp. gram.:m.}
\end{itemize}
Jôgo antigo, o mesmo que \textunderscore tintinini\textunderscore . Cf. \textunderscore Aulegrafia\textunderscore , 89.
\section{Tintilão}
\begin{itemize}
\item {Grp. gram.:m.}
\end{itemize}
\begin{itemize}
\item {Utilização:Mad}
\end{itemize}
O mesmo que \textunderscore tentilhão\textunderscore .
\section{Tintilhão}
\begin{itemize}
\item {Grp. gram.:m.}
\end{itemize}
O mesmo que \textunderscore tentilhão\textunderscore . Cf. B. Pereira, \textunderscore Prosódia\textunderscore , vb. \textunderscore sparasion\textunderscore .
\section{Tintim}
\begin{itemize}
\item {Grp. gram.:m.}
\end{itemize}
\begin{itemize}
\item {Utilização:Prov.}
\end{itemize}
\begin{itemize}
\item {Utilização:trasm.}
\end{itemize}
Passarinho do campo, talvez o mesmo que \textunderscore chinchão\textunderscore .
\section{Tintim}
\begin{itemize}
\item {Grp. gram.:m.}
\end{itemize}
O mesmo que \textunderscore tlim\textunderscore .
\section{Tintim-por-tintim}
\begin{itemize}
\item {Grp. gram.:loc. adv.}
\end{itemize}
Por miúdo, minuciosamente; especificadamente.(V.tentim-por-tentim)
\section{Tintinabular}
\begin{itemize}
\item {Grp. gram.:v. i.}
\end{itemize}
\begin{itemize}
\item {Proveniência:(De \textunderscore tintinábulo\textunderscore )}
\end{itemize}
Soar como campaínha. Cf. C. Neto, \textunderscore Baladilhas\textunderscore , 85.
\section{Tintinante}
\begin{itemize}
\item {Grp. gram.:adj.}
\end{itemize}
Que tintina. Cf. Castilho, \textunderscore Metam.\textunderscore , 72.
\section{Tintinar}
\begin{itemize}
\item {Grp. gram.:v. i.}
\end{itemize}
\begin{itemize}
\item {Proveniência:(De \textunderscore tintim\textunderscore ^2)}
\end{itemize}
O mesmo que \textunderscore telintar\textunderscore .
\section{Tintinini}
\begin{itemize}
\item {Grp. gram.:m.}
\end{itemize}
Espécie de jôgo antigo:«\textunderscore cincoenta cruzados ganhais vós aí num só lanço da péla ou do tintinini\textunderscore ». Castilho, \textunderscore Camões\textunderscore , 118.--Era prohibido nos paços, aos Domingos e dias santos, por alvará de 8 de Julho de 1521.
\section{Tintinir}
\begin{itemize}
\item {Grp. gram.:v. i.}
\end{itemize}
O mesmo que \textunderscore tintinar\textunderscore .
\section{Tintinnabular}
\begin{itemize}
\item {Grp. gram.:v. i.}
\end{itemize}
\begin{itemize}
\item {Proveniência:(De \textunderscore tintinnábulo\textunderscore )}
\end{itemize}
Soar como campaínha. Cf. C. Neto, \textunderscore Baladilhas\textunderscore , 85.
\section{Tífia}
\begin{itemize}
\item {Grp. gram.:f.}
\end{itemize}
\begin{itemize}
\item {Proveniência:(Do gr. \textunderscore tiphos\textunderscore )}
\end{itemize}
Gênero de insectos himenópteros, de corpo escuro e pubescente.
\section{Tilacino}
\begin{itemize}
\item {Grp. gram.:m.}
\end{itemize}
\begin{itemize}
\item {Proveniência:(Do gr. \textunderscore thulax\textunderscore )}
\end{itemize}
Gênero de mamíferos marsupiaes.
\section{Timalo}
\begin{itemize}
\item {Grp. gram.:m.}
\end{itemize}
\begin{itemize}
\item {Proveniência:(Lat. \textunderscore thymallus\textunderscore )}
\end{itemize}
Gênero de peixes malacopterígios.
\section{Timbra}
\begin{itemize}
\item {Grp. gram.:f.}
\end{itemize}
\begin{itemize}
\item {Proveniência:(Lat. \textunderscore thymbra\textunderscore )}
\end{itemize}
Gênero de plantas aromáticas, da fam. das labiadas.
\section{Timbreira}
\begin{itemize}
\item {Grp. gram.:f.}
\end{itemize}
O mesmo que \textunderscore timbra\textunderscore .
\section{Tintinábulo}
\begin{itemize}
\item {Grp. gram.:m.}
\end{itemize}
\begin{itemize}
\item {Proveniência:(Lat. \textunderscore tintinnabulum\textunderscore )}
\end{itemize}
Campaínha; sineta. Cf. Herculano, \textunderscore Bobo\textunderscore , 29.
\section{Tintinnábulo}
\begin{itemize}
\item {Grp. gram.:m.}
\end{itemize}
\begin{itemize}
\item {Proveniência:(Lat. \textunderscore tintinnabulum\textunderscore )}
\end{itemize}
Campaínha; sineta. Cf. Herculano, \textunderscore Bobo\textunderscore , 29.
\section{Tinto}
\begin{itemize}
\item {Grp. gram.:adj.}
\end{itemize}
\begin{itemize}
\item {Utilização:Fig.}
\end{itemize}
\begin{itemize}
\item {Grp. gram.:M.}
\end{itemize}
\begin{itemize}
\item {Utilização:Prov.}
\end{itemize}
\begin{itemize}
\item {Utilização:minh.}
\end{itemize}
\begin{itemize}
\item {Proveniência:(Lat. \textunderscore tinctus\textunderscore )}
\end{itemize}
Que se tingiu: \textunderscore um pano tinto\textunderscore .
Sujo, ennodoado.
Diz-se do vinho ou da uva de côr mais ou menos escura.
Medida de pano grosso, correspondente a 6 varas ou 6^m,60.
\section{Tinto-cão}
\begin{itemize}
\item {Grp. gram.:m.}
\end{itemize}
Casta de uva preta, na região do Doiro e na Extremadura.
\section{Tintojarra}
\begin{itemize}
\item {Grp. gram.:f.}
\end{itemize}
\begin{itemize}
\item {Utilização:Mad}
\end{itemize}
O mesmo que \textunderscore tintonegro\textunderscore .
\section{Tinto-macho}
\begin{itemize}
\item {Grp. gram.:m.}
\end{itemize}
Casta de uva trasmontana.
\section{Tintonegro}
\begin{itemize}
\item {fónica:nê}
\end{itemize}
\begin{itemize}
\item {Grp. gram.:m.}
\end{itemize}
\begin{itemize}
\item {Utilização:Mad}
\end{itemize}
O mesmo que \textunderscore toutinegra\textunderscore . Cf. Schmitz, \textunderscore Die Vögel Madeira's\textunderscore .
\section{Tintor}
\begin{itemize}
\item {Grp. gram.:m.  e  adj.}
\end{itemize}
\begin{itemize}
\item {Proveniência:(Lat. \textunderscore tinctor\textunderscore )}
\end{itemize}
O que tinge.
O mesmo que \textunderscore tintureiro\textunderscore .
\section{Tintorial}
\begin{itemize}
\item {Grp. gram.:adj.}
\end{itemize}
\begin{itemize}
\item {Proveniência:(Do lat. \textunderscore tinctor\textunderscore )}
\end{itemize}
Que serve para tingir.
Relativo a tinturaria.
\section{Tintório}
\begin{itemize}
\item {Grp. gram.:adj.}
\end{itemize}
\begin{itemize}
\item {Proveniência:(Lat. \textunderscore tinctorius\textunderscore )}
\end{itemize}
Que serve para tingir.
Que produz substância usada em tinturania.
O mesmo que \textunderscore tintorial\textunderscore .
\section{Tintorro}
\begin{itemize}
\item {fónica:tô}
\end{itemize}
\begin{itemize}
\item {Grp. gram.:m.}
\end{itemize}
Casta de uva tinta, chamada também \textunderscore tintureiro\textunderscore .
\section{Tintorroxa}
\begin{itemize}
\item {Grp. gram.:f.}
\end{itemize}
O mesmo que \textunderscore tintorroxo\textunderscore .
\section{Tintorroxo}
\begin{itemize}
\item {Grp. gram.:m.}
\end{itemize}
\begin{itemize}
\item {Utilização:Mad}
\end{itemize}
O mesmo que \textunderscore pintarroxo\textunderscore .
\section{Tintura}
\begin{itemize}
\item {Grp. gram.:f.}
\end{itemize}
\begin{itemize}
\item {Utilização:Fig.}
\end{itemize}
\begin{itemize}
\item {Proveniência:(Lat. \textunderscore tinctura\textunderscore )}
\end{itemize}
Acto ou effeito de tingir.
Tinta.
Solução de substância ou substâncias mais ou menos colorida.
Laivo, vestígios.
Conhecimentos rudimentares, noção superficial: \textunderscore têr umas tinturas de Grammática\textunderscore .
\section{Tinturão}
\begin{itemize}
\item {Grp. gram.:m.  e  adj.}
\end{itemize}
\begin{itemize}
\item {Proveniência:(De \textunderscore tintura\textunderscore )}
\end{itemize}
Variedade de uva.
\section{Tinturaria}
\begin{itemize}
\item {Grp. gram.:f.}
\end{itemize}
\begin{itemize}
\item {Proveniência:(De \textunderscore tintura\textunderscore )}
\end{itemize}
Estabelecimento ou fábrica, onde se tingem panos.
Offício ou arte de tintureiro.
\section{Tintureira}
\begin{itemize}
\item {Grp. gram.:f.}
\end{itemize}
\begin{itemize}
\item {Proveniência:(De \textunderscore tintureiro\textunderscore )}
\end{itemize}
Mulhér, que exerce a arte de tingir panos.
Dona de tinturaria.
Peixe esqualo.
Planta phytolácea.
Variedade de uva, o mesmo que \textunderscore tintureiro\textunderscore .
\section{Tintureiro}
\begin{itemize}
\item {Grp. gram.:adj.}
\end{itemize}
\begin{itemize}
\item {Grp. gram.:M.}
\end{itemize}
\begin{itemize}
\item {Proveniência:(De \textunderscore tintura\textunderscore )}
\end{itemize}
Que tinge.
Aquelle que tinge panos.
Dono de tinturaria.
Espécie de uva tinta.
\section{Tinturial}
\begin{itemize}
\item {Grp. gram.:adj.}
\end{itemize}
(V.tinctorial)
\section{Tio}
\begin{itemize}
\item {Grp. gram.:m.}
\end{itemize}
\begin{itemize}
\item {Utilização:Ext.}
\end{itemize}
\begin{itemize}
\item {Utilização:Pop.}
\end{itemize}
\begin{itemize}
\item {Proveniência:(Do gr. \textunderscore theios\textunderscore )}
\end{itemize}
Diz-se o indivíduo em relação a outrem, de cujo pai ou mãe é irmão.
Marido da tia.
Tratamento, que se dá aos homens de idade e àquelles de que se não sabe o nome.
\section{Tioco}
\begin{itemize}
\item {Grp. gram.:m.}
\end{itemize}
Nome de vários pássaros conirostros da África.
\section{Tiocole}
\begin{itemize}
\item {Grp. gram.:m.}
\end{itemize}
Medicamento, empregado especialmente contra a tuberculose.
\section{Tio-lio}
\begin{itemize}
\item {Grp. gram.:m.}
\end{itemize}
\begin{itemize}
\item {Utilização:T. de Macau}
\end{itemize}
Remo, com que as tancareiras governam os tancares.
\section{Tiónico}
\begin{itemize}
\item {Grp. gram.:adj.}
\end{itemize}
\begin{itemize}
\item {Proveniência:(Do gr. \textunderscore theion\textunderscore )}
\end{itemize}
Relativo ao enxôfre e aos seus compostos.
\section{Tiorba}
\begin{itemize}
\item {Grp. gram.:f.}
\end{itemize}
\begin{itemize}
\item {Proveniência:(It. \textunderscore tiorba\textunderscore )}
\end{itemize}
Instrumento de cordas, semelhante ao alaúde, mas maior do que êlle.
\section{Tiorga}
\begin{itemize}
\item {Grp. gram.:f.}
\end{itemize}
\begin{itemize}
\item {Utilização:Pop.}
\end{itemize}
Bebedeira.
\section{Típhia}
\begin{itemize}
\item {Grp. gram.:f.}
\end{itemize}
\begin{itemize}
\item {Proveniência:(Do gr. \textunderscore tiphos\textunderscore )}
\end{itemize}
Gênero de insectos hymenópteros, de corpo escuro e pubescente.
\section{Tipi}
\begin{itemize}
\item {Grp. gram.:m.}
\end{itemize}
Nome de várias plantas do Brasil.
\section{Tipiti}
\begin{itemize}
\item {Grp. gram.:m.}
\end{itemize}
\begin{itemize}
\item {Utilização:Bras}
\end{itemize}
Peça cylíndrica, tecida de talas de palmeira, em que se mete a mandioca ou outra substância, de que se quere extrahir caldo.
\section{Tiple}
\begin{itemize}
\item {Grp. gram.:m.  e  f.}
\end{itemize}
\begin{itemize}
\item {Proveniência:(It. \textunderscore tiple\textunderscore )}
\end{itemize}
O mesmo que \textunderscore soprano\textunderscore .
\section{Tipóia}
\begin{itemize}
\item {Grp. gram.:f.}
\end{itemize}
\begin{itemize}
\item {Utilização:Pop.}
\end{itemize}
\begin{itemize}
\item {Proveniência:(Do ingl. \textunderscore teapoy\textunderscore )}
\end{itemize}
Palanquim de rêde.
Carruagem reles ou velha.
\section{Tipóia}
\begin{itemize}
\item {Grp. gram.:f.}
\end{itemize}
\begin{itemize}
\item {Utilização:Bras}
\end{itemize}
Charpa, para sustentar um braço doente.
Camisa sem mangas, feita do entre-casco de certas árvores.
Rêde velha, fiango.
\section{Tipre}
\begin{itemize}
\item {Grp. gram.:m.}
\end{itemize}
Fórma pop. de \textunderscore tiple\textunderscore . Cf. E. Vieira, \textunderscore Diccion. Mús.\textunderscore 
\section{Tipu}
\begin{itemize}
\item {Grp. gram.:m.}
\end{itemize}
Planta leguminosa.
\section{Tipuca}
\begin{itemize}
\item {Grp. gram.:f.}
\end{itemize}
\begin{itemize}
\item {Utilização:Bras. do N}
\end{itemize}
\begin{itemize}
\item {Proveniência:(Do guar. \textunderscore tipig\textunderscore )}
\end{itemize}
O último leite que sai da têta das vaccas.
\section{Típula}
\begin{itemize}
\item {Grp. gram.:f.}
\end{itemize}
\begin{itemize}
\item {Proveniência:(Lat. \textunderscore tippula\textunderscore )}
\end{itemize}
Gênero de insectos dípteros, cujas larvas se alimentam exclusivamente de terra.
\section{Tipulárias}
\begin{itemize}
\item {Grp. gram.:f. pl.}
\end{itemize}
Família de insectos, que tem por typo a típula.
(Fem. pl. de \textunderscore tipulário\textunderscore )
\section{Tipulário}
\begin{itemize}
\item {Grp. gram.:adj.}
\end{itemize}
Relativo ou semelhante á típula.
\section{Tipuliforme}
\begin{itemize}
\item {Grp. gram.:adj.}
\end{itemize}
\begin{itemize}
\item {Proveniência:(De \textunderscore típula\textunderscore  + \textunderscore fórma\textunderscore )}
\end{itemize}
Que tem fórma de típula.
\section{Tiquara}
\begin{itemize}
\item {Grp. gram.:f.}
\end{itemize}
\begin{itemize}
\item {Utilização:Bras. do N}
\end{itemize}
O mesmo que \textunderscore jacuba\textunderscore .
Qualquer bebida refrigerante.
(Do tupi)
\section{Tique}
\begin{itemize}
\item {Grp. gram.:m.}
\end{itemize}
\begin{itemize}
\item {Utilização:Fig.}
\end{itemize}
\begin{itemize}
\item {Proveniência:(Fr. \textunderscore tic\textunderscore )}
\end{itemize}
Contracção espasmódica dos músculos faciaes.
Neuralgia facial.
Modos característicos; feitio.
\section{Tique-taque}
\begin{itemize}
\item {Grp. gram.:m.}
\end{itemize}
\begin{itemize}
\item {Utilização:Fam.}
\end{itemize}
\begin{itemize}
\item {Proveniência:(T. onom.)}
\end{itemize}
Voz imitativa de um som regular e cadenciado.
O bater do coração.
Palpite.
\section{Tique-tique}
\begin{itemize}
\item {Grp. gram.:m.}
\end{itemize}
\begin{itemize}
\item {Utilização:Fam.}
\end{itemize}
Voz imitativa de um som que se repete, mais ou menos regularmente, e se prolonga.
O mesmo que \textunderscore tique-taque\textunderscore .
\section{Tiquira}
\begin{itemize}
\item {Grp. gram.:m.}
\end{itemize}
\begin{itemize}
\item {Utilização:Bras}
\end{itemize}
Aguardente de mandioca.
\section{Tira}
\begin{itemize}
\item {Grp. gram.:f.}
\end{itemize}
\begin{itemize}
\item {Proveniência:(De \textunderscore tirar\textunderscore )}
\end{itemize}
Pedaço de pano, papel, etc., mais comprido que largo.
Ourela.
Lista.
Fita.
Filete.
\section{Tirabragal}
\begin{itemize}
\item {Grp. gram.:m.}
\end{itemize}
\begin{itemize}
\item {Proveniência:(De \textunderscore tirar\textunderscore  + \textunderscore bragal\textunderscore )}
\end{itemize}
Funda, usada por quem padece quebradura ou hérnia intestinal.
Correia, que faz parte do apparelho do cavallo, sôbre as ancas.
\section{Tirachumbo}
\begin{itemize}
\item {Grp. gram.:m.}
\end{itemize}
\begin{itemize}
\item {Proveniência:(De \textunderscore tirar\textunderscore  + \textunderscore chumbo\textunderscore )}
\end{itemize}
Apparelho, para fazer laminas de chumbo.
\section{Tiracollo}
\begin{itemize}
\item {Grp. gram.:m.}
\end{itemize}
\begin{itemize}
\item {Grp. gram.:Loc. adv.}
\end{itemize}
Correia, que cinge o corpo, passando por cima de um ombro e por baixo do braço opposto a êsse ombro.
Boldrié.
\textunderscore A tiracollo\textunderscore , indo de um ombro para o lado opposto, na cintura ou debaixo do braço opposto a êsse ombro.
(Cast. \textunderscore tiracuello\textunderscore )
\section{Tiracolo}
\begin{itemize}
\item {Grp. gram.:m.}
\end{itemize}
\begin{itemize}
\item {Grp. gram.:Loc. adv.}
\end{itemize}
Correia, que cinge o corpo, passando por cima de um ombro e por baixo do braço oposto a êsse ombro.
Boldrié.
\textunderscore A tiracolo\textunderscore , indo de um ombro para o lado oposto, na cintura ou debaixo do braço oposto a êsse ombro.
(Cast. \textunderscore tiracuello\textunderscore )
\section{Tirada}
\begin{itemize}
\item {Grp. gram.:f.}
\end{itemize}
\begin{itemize}
\item {Proveniência:(De \textunderscore tirar\textunderscore )}
\end{itemize}
Acto de tirar.
Tiradura.
Exportação de gêneros.
Grande espaço de tempo.
Caminhada.
Grande extensão de caminho.
Disurso extenso; trecho longo.
\section{Tiradeira}
\begin{itemize}
\item {Grp. gram.:f.}
\end{itemize}
O mesmo que \textunderscore estrovenga\textunderscore .
\section{Tiradeiras}
\begin{itemize}
\item {Grp. gram.:f. pl.}
\end{itemize}
\begin{itemize}
\item {Proveniência:(De \textunderscore tirar\textunderscore )}
\end{itemize}
Espécie de tirantes, entre os quaes vão as bêstas, nos engenhos de açúcar.
\section{Tiradela}
\begin{itemize}
\item {Grp. gram.:f.}
\end{itemize}
O mesmo que \textunderscore tiradura\textunderscore . Cf. \textunderscore Museu Technol.\textunderscore , 97 e 99.
\section{Tiradoira}
\begin{itemize}
\item {Grp. gram.:f.}
\end{itemize}
\begin{itemize}
\item {Proveniência:(De \textunderscore tirar\textunderscore )}
\end{itemize}
Temão do carro ou do arado.
\section{Tirador}
\begin{itemize}
\item {Grp. gram.:m.  e  adj.}
\end{itemize}
\begin{itemize}
\item {Grp. gram.:M.}
\end{itemize}
\begin{itemize}
\item {Utilização:Bras. do S}
\end{itemize}
\begin{itemize}
\item {Proveniência:(De \textunderscore tirar\textunderscore )}
\end{itemize}
O que tira.
Chicote do cabo de qualquer apparelho náutico.
Pedaço de coiro cru, que os laçadores põem em volta da cintura, para amparar as ilhargas quando laçam o animal.
\section{Tiradoura}
\begin{itemize}
\item {Grp. gram.:f.}
\end{itemize}
\begin{itemize}
\item {Proveniência:(De \textunderscore tirar\textunderscore )}
\end{itemize}
Temão do carro ou do arado.
\section{Tiradura}
\begin{itemize}
\item {Grp. gram.:f.}
\end{itemize}
Acto ou effeito de tirar.
\section{Tira-flôr}
\begin{itemize}
\item {Grp. gram.:m.}
\end{itemize}
Instrumento para tirar a flôr do vinho.
\section{Tirafundo}
\begin{itemize}
\item {Grp. gram.:m.}
\end{itemize}
\begin{itemize}
\item {Proveniência:(Do fr. \textunderscore tire-fond\textunderscore )}
\end{itemize}
Espécie de verruma de torneiro.
Parafuso, com que se fixam os carris de caminho de ferro nas travéssas.
\section{Tiragem}
\begin{itemize}
\item {Grp. gram.:f.}
\end{itemize}
\begin{itemize}
\item {Proveniência:(De \textunderscore tirar\textunderscore )}
\end{itemize}
Tiradura.
Passagem dos metaes pela fieira.
Impressão typográphica.
Número de exemplares impressos.
Corrente do ar que sái quente de uma chaminé e entrada do ar frio que o substitue.
\section{Tira-leite}
\begin{itemize}
\item {Grp. gram.:m.}
\end{itemize}
\begin{itemize}
\item {Utilização:Bras}
\end{itemize}
Instrumento para extrair da mama o leite. Cf. \textunderscore Tarifa das Alfand.\textunderscore , do Brasil.
\section{Tira-linhas}
\begin{itemize}
\item {Grp. gram.:m.}
\end{itemize}
Instrumento de metal, que traça linhas com tinta no papel.
\section{Tiramento}
\begin{itemize}
\item {Grp. gram.:m.}
\end{itemize}
\begin{itemize}
\item {Utilização:Ant.}
\end{itemize}
O mesmo que \textunderscore tiradura\textunderscore .
Arrecadação de tributos ou rendas.
\section{Tiramola}
\begin{itemize}
\item {Grp. gram.:f.}
\end{itemize}
\begin{itemize}
\item {Utilização:Náut.}
\end{itemize}
\begin{itemize}
\item {Proveniência:(De \textunderscore tirar\textunderscore  + \textunderscore mola\textunderscore )}
\end{itemize}
Acto de tocar qualquer apparelho náutico.
\section{Tiramolar}
\begin{itemize}
\item {Grp. gram.:v.}
\end{itemize}
\begin{itemize}
\item {Utilização:t. Náut.}
\end{itemize}
\begin{itemize}
\item {Proveniência:(De \textunderscore tiramola\textunderscore )}
\end{itemize}
Amainar ou arrear (uma talha, a bordo).
\section{Tirana}
\begin{itemize}
\item {Grp. gram.:f.}
\end{itemize}
\begin{itemize}
\item {Utilização:Bras. do S}
\end{itemize}
Bailado campestre, espécie de fandango.
\section{Tira-nódoas}
\begin{itemize}
\item {Grp. gram.:m.}
\end{itemize}
Substância, com que se tiram nódoas de roupas, soalhos, etc. Cf. Castilho, \textunderscore Fastos\textunderscore , II, 218 e 319.
\section{Tirante}
\begin{itemize}
\item {Grp. gram.:adj.}
\end{itemize}
\begin{itemize}
\item {Utilização:Prov.}
\end{itemize}
\begin{itemize}
\item {Utilização:beir.}
\end{itemize}
\begin{itemize}
\item {Grp. gram.:M.}
\end{itemize}
\begin{itemize}
\item {Grp. gram.:Prep.}
\end{itemize}
\begin{itemize}
\item {Grp. gram.:Pl.}
\end{itemize}
\begin{itemize}
\item {Utilização:Ant.}
\end{itemize}
\begin{itemize}
\item {Utilização:Gír.}
\end{itemize}
\begin{itemize}
\item {Proveniência:(De \textunderscore tirar\textunderscore )}
\end{itemize}
Que tira.
Exceptuado.
Que dá apparências, que se aproxima: \textunderscore um vestido, tirante a verde\textunderscore .
Inclinado, desnivelado, (tratando-se de uma parede).
Cada uma das correias, que ligam o carro ás bêstas que o puxam.
Barra de ferro ou viga, que sustenta o madeiramento de um tecto.
Barra de ferro, que vai de uma parede de um edificio á parede opposta, e que serve para nella se pendurarem quaesquer objectos.
Cada uma das cordas, com que se puxam os reparos, em artilharia.
Excepto:«\textunderscore tirante as horas...\textunderscore »Camillo, \textunderscore Filha do Reg.\textunderscore 
Calções.
\section{Tiranteza}
\begin{itemize}
\item {Grp. gram.:f.}
\end{itemize}
\begin{itemize}
\item {Utilização:Prov.}
\end{itemize}
\begin{itemize}
\item {Utilização:beir.}
\end{itemize}
Qualidade de tirante ou inclinação, (falando-se de uma construcção ou parede).
\section{Tirão}
\begin{itemize}
\item {Grp. gram.:m.}
\end{itemize}
Acto ou effeito de tirar com fôrça.
Estirão; grande caminhada.
\section{Tira-olhos}
\begin{itemize}
\item {Grp. gram.:m.}
\end{itemize}
Designação vulgar da libellinha.
\section{Tirapé}
\begin{itemize}
\item {Grp. gram.:m.}
\end{itemize}
\begin{itemize}
\item {Proveniência:(De \textunderscore tirar\textunderscore  + \textunderscore pé\textunderscore )}
\end{itemize}
Correia de coiro, com que os sapateiros seguram a obra sôbre a fôrma.
\section{Tira-peixe}
\begin{itemize}
\item {Grp. gram.:m.}
\end{itemize}
\begin{itemize}
\item {Utilização:T. da Bairrada}
\end{itemize}
O mesmo que \textunderscore pica-peixe\textunderscore .
\section{Tira-puxa}
\begin{itemize}
\item {Grp. gram.:f.}
\end{itemize}
\begin{itemize}
\item {Utilização:Prov.}
\end{itemize}
Questão, disputa.
\section{Tira-que-tira}
\begin{itemize}
\item {Grp. gram.:loc. adv.}
\end{itemize}
\begin{itemize}
\item {Proveniência:(De \textunderscore tirar\textunderscore )}
\end{itemize}
(designativa de \textunderscore movimento rápido e prolongado\textunderscore )
\section{Tirar}
\begin{itemize}
\item {Grp. gram.:v. t.}
\end{itemize}
\begin{itemize}
\item {Utilização:Bras}
\end{itemize}
\begin{itemize}
\item {Grp. gram.:V. i.}
\end{itemize}
\begin{itemize}
\item {Grp. gram.:V. p.}
\end{itemize}
\begin{itemize}
\item {Proveniência:(Do germ. \textunderscore tairan\textunderscore )}
\end{itemize}
Fazer saír de um ponto ou lugar.
Dar comêço a, para que outros acompanhem: \textunderscore tirar a reza\textunderscore ; \textunderscore tirar a modinha\textunderscore .
Extrahir, arrancar: \textunderscore tirar um dente\textunderscore .
Puxar.
Attrahir.
Extractar.
Auferir, receber: \textunderscore tirar lucros\textunderscore .
Libertar.
Dissuadir: \textunderscore ninguém me tira disto\textunderscore .
Apartar.
Separar.
Privar de.
Tolher.
Derivar: \textunderscore tirar conclusões\textunderscore .
Usurpar.
Furtar.
Arrecadar.
Lucrar.
Atirar, arremessar.
Eliminar.
Reproduzir.
Exceptuar.
Exportar.
Abolir; extinguir.
Estampar, imprimir.
Despir.
Puxar.
Dar ares, assemelhar-se, (falando-se de côres).
Dar tiros.
Visar, attender ou têr em mira.
\textunderscore Tirar atrás\textunderscore , recuar.
\textunderscore Não tirar\textunderscore , não valer nada, sêr indifferente.
Saír; libertar-se: \textunderscore tirou-se de difficuldades\textunderscore .
\section{Tira-teimas}
\begin{itemize}
\item {Grp. gram.:m.}
\end{itemize}
\begin{itemize}
\item {Utilização:Pleb.}
\end{itemize}
\begin{itemize}
\item {Utilização:Fam.}
\end{itemize}
Objecto, com que se castigam crianças ou teimosos.
Prova categórica, argumento decisivo.
O diccionário.
\section{Tira-testa}
\begin{itemize}
\item {Grp. gram.:m.}
\end{itemize}
Parte do arreio, correspondente á testa da cavalgadura.
\section{Tira-tira}
\begin{itemize}
\item {Grp. gram.:loc. adv.}
\end{itemize}
O mesmo que \textunderscore tira-que-tira\textunderscore .
\section{Tira-vergal}
\begin{itemize}
\item {Grp. gram.:m.}
\end{itemize}
Tira de coiro, que prendia os machos á liteira.
\section{Tiravira}
\begin{itemize}
\item {Grp. gram.:f.}
\end{itemize}
Cabo duplo, com que se embarcam pipas e tonéis.
Nome de um peixe, que também se chama \textunderscore agulhão\textunderscore , (\textunderscore scombrecox saurus\textunderscore , Lin.).
(Cp. fr. \textunderscore trevire\textunderscore )
\section{Tiraz}
\begin{itemize}
\item {Grp. gram.:m.}
\end{itemize}
\begin{itemize}
\item {Utilização:Ant.}
\end{itemize}
\begin{itemize}
\item {Grp. gram.:Adj.}
\end{itemize}
Pano de linho, com ramagens e, ás vezes, tecido de oiro.
Diz-se dêsse pano. Cf. Herculano, \textunderscore Bobo\textunderscore , 336.
\section{Tirefão}
\begin{itemize}
\item {Grp. gram.:m.}
\end{itemize}
\begin{itemize}
\item {Utilização:Neol.}
\end{itemize}
\begin{itemize}
\item {Proveniência:(Do fr. \textunderscore tire-fond\textunderscore )}
\end{itemize}
Parafuso, que prende o carril á travéssa, nos caminhos de ferro; tirafundo.
Pl. \textunderscore tirefões\textunderscore .
\section{Tirésias}
\begin{itemize}
\item {Grp. gram.:m.}
\end{itemize}
\begin{itemize}
\item {Proveniência:(De \textunderscore Tirésias\textunderscore , n. p.)}
\end{itemize}
Gênero de insectos coleópteros pentâmeros.
\section{Tirete}
\begin{itemize}
\item {fónica:tirê}
\end{itemize}
\begin{itemize}
\item {Grp. gram.:m.}
\end{itemize}
\begin{itemize}
\item {Proveniência:(De \textunderscore tira\textunderscore )}
\end{itemize}
O mesmo que \textunderscore hýphen\textunderscore .
\section{Timeláceas}
\begin{itemize}
\item {Grp. gram.:f. pl.}
\end{itemize}
Família de plantas, que tem por tipo a timeleia.
\section{Tímele}
\begin{itemize}
\item {Grp. gram.:f.}
\end{itemize}
\begin{itemize}
\item {Proveniência:(Lat. \textunderscore thymele\textunderscore )}
\end{itemize}
Estrado, adeante do proscénio, nos teatros gregos, donde os músicos dirigiam as evoluções dos coros.
O altar dos sacrifícios, na tragédia grega.
\section{Timeleia}
\begin{itemize}
\item {Grp. gram.:f.}
\end{itemize}
\begin{itemize}
\item {Proveniência:(Lat. \textunderscore thymelaea\textunderscore )}
\end{itemize}
Gênero de plantas, conhecido cientificamente por \textunderscore daphne thymelaea\textunderscore .
\section{Timiama}
\begin{itemize}
\item {Grp. gram.:f.}
\end{itemize}
\begin{itemize}
\item {Proveniência:(Lat. \textunderscore thymiama\textunderscore )}
\end{itemize}
Certa droga medicinal. Cf. Garc. da Orta, 7, v.^o.
\section{Timiatecnia}
\begin{itemize}
\item {Grp. gram.:f.}
\end{itemize}
\begin{itemize}
\item {Proveniência:(Do gr. \textunderscore thumos\textunderscore  + \textunderscore tekne\textunderscore )}
\end{itemize}
Arte de fabricar perfumes.
\section{Tímico}
\begin{itemize}
\item {Grp. gram.:adj.}
\end{itemize}
\begin{itemize}
\item {Utilização:Anat.}
\end{itemize}
Relativo ao \textunderscore timo\textunderscore ^2.
\section{Timo}
\begin{itemize}
\item {Grp. gram.:m.}
\end{itemize}
\begin{itemize}
\item {Proveniência:(Lat. \textunderscore thymum\textunderscore )}
\end{itemize}
O mesmo que \textunderscore tomilho\textunderscore .
\section{Timo}
\begin{itemize}
\item {Grp. gram.:m.}
\end{itemize}
\begin{itemize}
\item {Utilização:Anat.}
\end{itemize}
\begin{itemize}
\item {Proveniência:(Do gr. \textunderscore thumos\textunderscore )}
\end{itemize}
Corpo carnoso ou glandular, no tórax do feto.
\section{Timocracia}
\begin{itemize}
\item {Grp. gram.:f.}
\end{itemize}
Sistema de governação, em que preponderam os ricos.
(Cp. \textunderscore timócrata\textunderscore )
\section{Timócrata}
\begin{itemize}
\item {Grp. gram.:m.}
\end{itemize}
\begin{itemize}
\item {Proveniência:(Do gr. \textunderscore thumos\textunderscore  + \textunderscore kratos\textunderscore )}
\end{itemize}
Partidário da timocracia.
\section{Timocrático}
\begin{itemize}
\item {Grp. gram.:adj.}
\end{itemize}
Relativo á timocracia.
\section{Timolina}
\begin{itemize}
\item {Grp. gram.:f.}
\end{itemize}
\begin{itemize}
\item {Proveniência:(De \textunderscore thymo\textunderscore ^1 + \textunderscore óleo\textunderscore ?)}
\end{itemize}
Substância farmacêutica, de que se faz uma pasta dentífrica.
\section{Tireóide}
\begin{itemize}
\item {Grp. gram.:f.}
\end{itemize}
\begin{itemize}
\item {Utilização:Anat.}
\end{itemize}
\begin{itemize}
\item {Grp. gram.:Adj.}
\end{itemize}
A maior cartilagem da laringe, na parte antero-superior.
O mesmo que \textunderscore tireoídeo\textunderscore .
\section{Tireoidectomia}
\begin{itemize}
\item {Grp. gram.:f.}
\end{itemize}
\begin{itemize}
\item {Utilização:Cir.}
\end{itemize}
Operação, que consiste na extirpação total da glândula tireoídea, no tratamento do bócio exoftálmico.
\section{Tireoídeo}
\begin{itemize}
\item {Grp. gram.:adj.}
\end{itemize}
\begin{itemize}
\item {Proveniência:(Do gr. \textunderscore thureos\textunderscore  + \textunderscore eidos\textunderscore )}
\end{itemize}
Diz-se da maior cartilagem da laringe e de um corpo glandular, situado na parte antero-superior da laringe.
\section{Tirícia}
\begin{itemize}
\item {Grp. gram.:f.}
\end{itemize}
\begin{itemize}
\item {Utilização:Pop.}
\end{itemize}
O mesmo que \textunderscore icterícia\textunderscore .
\section{Tiriba-pequeno}
\begin{itemize}
\item {Grp. gram.:m.}
\end{itemize}
Ave americana, semelhante á arara, mas mais pequena.
\section{Tiriciado}
\begin{itemize}
\item {Grp. gram.:adj.}
\end{itemize}
\begin{itemize}
\item {Utilização:Pop.}
\end{itemize}
\begin{itemize}
\item {Proveniência:(De \textunderscore tirícia\textunderscore )}
\end{itemize}
Que tem icterícia.
\section{Tirídia}
\begin{itemize}
\item {Grp. gram.:f.}
\end{itemize}
\begin{itemize}
\item {Proveniência:(Do gr. \textunderscore thuridion\textunderscore )}
\end{itemize}
Gênero de insectos lepidópteros.
\section{Tirintintim}
\begin{itemize}
\item {Grp. gram.:m.}
\end{itemize}
\begin{itemize}
\item {Proveniência:(T. onom.)}
\end{itemize}
Voz imitativa do som da trombeta.
\section{Tirió}
\begin{itemize}
\item {Grp. gram.:m.}
\end{itemize}
Bicho venenoso da Índia Portuguesa.
\section{Tiririca}
\begin{itemize}
\item {Grp. gram.:f.}
\end{itemize}
Arbusto cyperáceo do Brasil.
Nome de outras plantas.
\section{Tiritana}
\begin{itemize}
\item {Grp. gram.:f.}
\end{itemize}
\begin{itemize}
\item {Utilização:Bot.}
\end{itemize}
Mantéu de seriguilha, que as camponesas usam sôbre outro mantéu.
O mesmo que \textunderscore parietária\textunderscore .
\section{Tiritante}
\begin{itemize}
\item {Grp. gram.:adj.}
\end{itemize}
Que tirita.
\section{Tiritar}
\begin{itemize}
\item {Grp. gram.:v. i.}
\end{itemize}
\begin{itemize}
\item {Proveniência:(T. onom.)}
\end{itemize}
Tremer com frio.
\section{Tiritir}
\begin{itemize}
\item {Grp. gram.:v. i.}
\end{itemize}
\begin{itemize}
\item {Utilização:Bras}
\end{itemize}
Retinir; telintar:«\textunderscore tiritiu por vezes a campaínha presidencial\textunderscore ». \textunderscore Jorn.-do-Comm.\textunderscore , do Rio, de 31-III-913.
(Onom.)
\section{Tiriúma}
\begin{itemize}
\item {Grp. gram.:adj.}
\end{itemize}
\begin{itemize}
\item {Utilização:Bras}
\end{itemize}
Desacompanhado; solitário.
(Do tupi \textunderscore itirama\textunderscore )
\section{Tiriva}
\begin{itemize}
\item {Grp. gram.:m.}
\end{itemize}
\begin{itemize}
\item {Utilização:Bras}
\end{itemize}
O mesmo que \textunderscore tiriba-pequeno\textunderscore .
\section{Tirlintar}
\begin{itemize}
\item {Grp. gram.:v. i.}
\end{itemize}
O mesmo que \textunderscore telintar\textunderscore . Cf. Arn. Gama, \textunderscore Última Dona\textunderscore , 99.
\section{Tiro}
\begin{itemize}
\item {Grp. gram.:m.}
\end{itemize}
\begin{itemize}
\item {Utilização:Fig.}
\end{itemize}
\begin{itemize}
\item {Proveniência:(De \textunderscore tirar\textunderscore )}
\end{itemize}
Acto ou effeito de atirar ou arremessar.
O disparar de uma arma de fogo.
Carga, disparada por arma de fogo.
Explosão.
Distância, que a carga de uma arma de fogo póde vencer: \textunderscore o inimigo ficava-nos a dois tiros de mosquete\textunderscore .
Lugar, onde se fazem exercícios com arma de fogo.
Referência picante.
Remoque.
Expansão; ímpeto.
Tirante, corda com que se atrela um animal a um vehículo.
Acto de puxar carros, (falando-se de cavalgaduras).
Animaes, que puxam um carro.
\textunderscore Tiro de coxia\textunderscore , tiro de peça, que, tendo por alvo um navio, o atravessa de prôa á pôpa ou vice versa.
\section{Tirocínio}
\begin{itemize}
\item {Grp. gram.:m.}
\end{itemize}
\begin{itemize}
\item {Proveniência:(Lat. \textunderscore tirocinium\textunderscore )}
\end{itemize}
Primeiro ensino.
Aprendizagem.
Prática ou exercício militar, para subir de pôsto.
\section{Tiroial}
\begin{itemize}
\item {Grp. gram.:m.}
\end{itemize}
Ponta maior do osso hióide.
(Por \textunderscore thyreóhyal\textunderscore , do gr. \textunderscore thureos\textunderscore )
\section{Tirolico-tico}
\begin{itemize}
\item {Grp. gram.:m.}
\end{itemize}
Expressão, empregada por crianças num jôgo.
\section{Tiroliro}
\begin{itemize}
\item {Grp. gram.:m.}
\end{itemize}
\begin{itemize}
\item {Utilização:Prov.}
\end{itemize}
\begin{itemize}
\item {Proveniência:(T. onom.)}
\end{itemize}
Toque de pífaro.
O mesmo que \textunderscore pifaro\textunderscore .
\section{Tiromancia}
\begin{itemize}
\item {Grp. gram.:f.}
\end{itemize}
\begin{itemize}
\item {Grp. gram.:f.}
\end{itemize}
\begin{itemize}
\item {Proveniência:(Do gr. \textunderscore turos\textunderscore  + \textunderscore manteia\textunderscore )}
\end{itemize}
Supposta adivinhação, em que se empregava o queijo.
Adivinhação por meio do queijo.
\section{Tironear}
\begin{itemize}
\item {Grp. gram.:v. t.}
\end{itemize}
\begin{itemize}
\item {Utilização:Bras. do S}
\end{itemize}
\begin{itemize}
\item {Proveniência:(De \textunderscore tirão\textunderscore )}
\end{itemize}
Dar tirões ou puxões pela rédea a (o cavallo), para que êste obedeça.
\section{Tirotear}
\begin{itemize}
\item {Grp. gram.:v. i.}
\end{itemize}
Fazer tiroteio. Cf. F. Recreio, \textunderscore Bat. de Ourique\textunderscore .
\section{Tiroteio}
\begin{itemize}
\item {Grp. gram.:m.}
\end{itemize}
\begin{itemize}
\item {Utilização:Fig.}
\end{itemize}
Fogo de fuzilaria, em que os tiros são muitos e successivos.
Fogo de guerrilhas ou de atiradores dispersos.
Troca ininterrupta de palavras entre pessôas que discutem ou ralham.
(Cast. \textunderscore tiroteo\textunderscore )
\section{Tirsígero}
\begin{itemize}
\item {Grp. gram.:adj.}
\end{itemize}
\begin{itemize}
\item {Proveniência:(Lat. \textunderscore thyrsiger\textunderscore )}
\end{itemize}
Que tem tirso.
\section{Tirso}
\begin{itemize}
\item {Grp. gram.:m.}
\end{itemize}
\begin{itemize}
\item {Proveniência:(Lat. \textunderscore thyrsus\textunderscore )}
\end{itemize}
Bastão, enfeitado de hera e pâmpanos e terminado em fórma de pinha.
Espécie de panícula, semelhante a um ramalhete comprido.
\section{Tirsoso}
\begin{itemize}
\item {Grp. gram.:adj.}
\end{itemize}
Que tem flôres em fórma de tirso.
\section{Tir-te}
(Cp. \textunderscore sem-tirte-nem-guarte\textunderscore )
\section{Tisana}
\begin{itemize}
\item {Grp. gram.:f.}
\end{itemize}
\begin{itemize}
\item {Proveniência:(Lat. \textunderscore ptisana\textunderscore )}
\end{itemize}
Cozimento de cevada.
Medicamento líquido, que constitue a bebida ordinária de um doente.
\section{Tisanópode}
\begin{itemize}
\item {Grp. gram.:m.}
\end{itemize}
\begin{itemize}
\item {Proveniência:(Do gr. \textunderscore thusanos\textunderscore  + \textunderscore pous\textunderscore )}
\end{itemize}
Gênero de crustáceos.
\section{Tisanópteros}
\begin{itemize}
\item {Grp. gram.:m. pl.}
\end{itemize}
\begin{itemize}
\item {Proveniência:(Do gr. \textunderscore thusanos\textunderscore  + \textunderscore pteron\textunderscore )}
\end{itemize}
Ordem de insectos, que vivem nos vegetaes, danificando-os.
\section{Tisanuros}
\begin{itemize}
\item {Grp. gram.:m. pl.}
\end{itemize}
\begin{itemize}
\item {Proveniência:(Gr. \textunderscore thusanouros\textunderscore )}
\end{itemize}
Ordem de insectos neurópteros.
\section{Tisco}
\begin{itemize}
\item {Grp. gram.:m.}
\end{itemize}
\begin{itemize}
\item {Utilização:Bras}
\end{itemize}
Pedacinho de qualquer coisa.
(Cp. \textunderscore tico\textunderscore ^1)
\section{Tísica}
\begin{itemize}
\item {Grp. gram.:f.}
\end{itemize}
\begin{itemize}
\item {Utilização:Med.}
\end{itemize}
\begin{itemize}
\item {Proveniência:(De \textunderscore tísico\textunderscore )}
\end{itemize}
Consumpção lenta.
Doença tuberculosa dos pulmões.
Laryngite chrónica.
Tuberculose.
\section{Tísico}
\begin{itemize}
\item {Grp. gram.:m.  e  adj.}
\end{itemize}
\begin{itemize}
\item {Utilização:Fig.}
\end{itemize}
\begin{itemize}
\item {Proveniência:(Lat. \textunderscore phthisicus\textunderscore )}
\end{itemize}
O que soffre tísica.
O que está muito magro.
\section{Tisiologia}
\begin{itemize}
\item {Grp. gram.:f.}
\end{itemize}
\begin{itemize}
\item {Proveniência:(Do gr. \textunderscore phthisis\textunderscore  + \textunderscore logos\textunderscore )}
\end{itemize}
Tratado á cêrca da tísica.
\section{Tisiológico}
\begin{itemize}
\item {Grp. gram.:adj.}
\end{itemize}
Relativo á tisiologia.
\section{Tisito}
\begin{itemize}
\item {Grp. gram.:m.}
\end{itemize}
Variedade de mármore verde.
\section{Tisna}
\begin{itemize}
\item {Grp. gram.:f.}
\end{itemize}
Acto ou effeito de tisnar.
Substância, preparada para ennegrecer qualquer coisa.
\section{Tisnadura}
\begin{itemize}
\item {Grp. gram.:f.}
\end{itemize}
O mesmo que \textunderscore tisna\textunderscore .
\section{Tisnar}
\begin{itemize}
\item {Grp. gram.:v. t.}
\end{itemize}
\begin{itemize}
\item {Utilização:Fig.}
\end{itemize}
Tornar negro, com fumo, carvão, etc.
Requeimar; tostar.
Macular.
(Cp. cast. \textunderscore tiznar\textunderscore )
\section{Tisne}
\begin{itemize}
\item {Grp. gram.:m.}
\end{itemize}
\begin{itemize}
\item {Proveniência:(De \textunderscore tisnar\textunderscore )}
\end{itemize}
Côr, produzida na pelle pelo fogo ou pelo fumo.
O mesmo que \textunderscore fuligem\textunderscore .
\section{Tisneira}
\begin{itemize}
\item {Grp. gram.:f.}
\end{itemize}
\begin{itemize}
\item {Utilização:Prov.}
\end{itemize}
\begin{itemize}
\item {Utilização:dur.}
\end{itemize}
\begin{itemize}
\item {Proveniência:(De \textunderscore tisnar\textunderscore )}
\end{itemize}
Acção do sol, soalheira.
\section{Tisso}
\begin{itemize}
\item {Grp. gram.:m.}
\end{itemize}
Tecido leve e ralo. Cf. Rebello, \textunderscore Contos e Lendas\textunderscore , 33; Filinto, VII, 102.
(Cp. fr. \textunderscore tissu\textunderscore )
\section{Titã}
\begin{itemize}
\item {Grp. gram.:m.}
\end{itemize}
\begin{itemize}
\item {Utilização:Fig.}
\end{itemize}
\begin{itemize}
\item {Proveniência:(Lat. \textunderscore titan\textunderscore )}
\end{itemize}
Nome de cada um dos gigantes que, segundo a Mythologia, quiseram escalar o céu e desthronar Júpiter.
O mesmo que \textunderscore gigante\textunderscore . Cf. Crespo, \textunderscore Min.\textunderscore , 48.
Guindaste de grande poder.
\section{Titan}
\begin{itemize}
\item {Grp. gram.:m.}
\end{itemize}
\begin{itemize}
\item {Utilização:Fig.}
\end{itemize}
\begin{itemize}
\item {Proveniência:(Lat. \textunderscore titan\textunderscore )}
\end{itemize}
Nome de cada um dos gigantes que, segundo a Mythologia, quiseram escalar o céu e desthronar Júpiter.
O mesmo que \textunderscore gigante\textunderscore . Cf. Crespo, \textunderscore Min.\textunderscore , 48.
Guindaste de grande poder.
\section{Titanado}
\begin{itemize}
\item {Grp. gram.:adj.}
\end{itemize}
Diz-se do ferro mineral, que contém titânio.
\section{Titanato}
\begin{itemize}
\item {Grp. gram.:m.}
\end{itemize}
\begin{itemize}
\item {Utilização:Chím.}
\end{itemize}
\begin{itemize}
\item {Proveniência:(De \textunderscore titânio\textunderscore )}
\end{itemize}
Sal, produzido pela combinação do ácido titânico com uma base.
\section{Titânico}
\begin{itemize}
\item {Grp. gram.:adj.}
\end{itemize}
\begin{itemize}
\item {Utilização:Fig.}
\end{itemize}
Relativo aos titans.
Que revela grande fôrça.
\section{Titânico}
\begin{itemize}
\item {Grp. gram.:adj.}
\end{itemize}
Relativo ao titânio.
\section{Titânio}
\begin{itemize}
\item {Grp. gram.:m.}
\end{itemize}
\begin{itemize}
\item {Proveniência:(Do gr. \textunderscore titanos\textunderscore )}
\end{itemize}
Metal raro, descoberto em 1791.
\section{Titano}
\begin{itemize}
\item {Grp. gram.:m.}
\end{itemize}
O mesmo que \textunderscore titânio\textunderscore .
\section{Titão}
\begin{itemize}
\item {Grp. gram.:m.}
\end{itemize}
O mesmo ou melhór que \textunderscore titan\textunderscore . Cf. Rebello, \textunderscore Contos e Lendas\textunderscore , 73.
\section{Titara}
\begin{itemize}
\item {Grp. gram.:f.}
\end{itemize}
Planta trepadeira, da fam. das palmeiras, (\textunderscore euterpe sarmentosa\textunderscore ).
\section{Titela}
\begin{itemize}
\item {Grp. gram.:f.}
\end{itemize}
\begin{itemize}
\item {Utilização:Fig.}
\end{itemize}
\begin{itemize}
\item {Grp. gram.:Pl.}
\end{itemize}
\begin{itemize}
\item {Utilização:Prov.}
\end{itemize}
\begin{itemize}
\item {Utilização:trasm.}
\end{itemize}
A parte carnuda do peito da ave.
Coisa preciosa.
Parte deanteira das chedas, onde estas se curvam para se unirem á cabeçalha.
\section{Títere}
\begin{itemize}
\item {Grp. gram.:m.}
\end{itemize}
\begin{itemize}
\item {Utilização:Pop.}
\end{itemize}
\begin{itemize}
\item {Proveniência:(Do lat. \textunderscore titulum\textunderscore , por intermédio do fr. \textunderscore titre\textunderscore ?)}
\end{itemize}
Boneco, que se faz mexer e gesticular por meio de cordéis e engonços; fantoche.
Palhaço.
Janota.
\section{Titerear}
\begin{itemize}
\item {Grp. gram.:v. i.}
\end{itemize}
Fazer mover títeres.
Mover-se como um títere.
\section{Titereiro}
\begin{itemize}
\item {Grp. gram.:m.  e  adj.}
\end{itemize}
\begin{itemize}
\item {Proveniência:(De \textunderscore títere\textunderscore )}
\end{itemize}
O que titereia.
\section{Titeriteiro}
\begin{itemize}
\item {Grp. gram.:m.  e  adj.}
\end{itemize}
O mesmo que \textunderscore titereiro\textunderscore .
\section{Tithónia}
\begin{itemize}
\item {Grp. gram.:f.}
\end{itemize}
\begin{itemize}
\item {Utilização:Poét.}
\end{itemize}
\begin{itemize}
\item {Proveniência:(Lat. \textunderscore Tithonia\textunderscore , n. p.)}
\end{itemize}
Aurora.
\section{Tithtýmalo}
\begin{itemize}
\item {Grp. gram.:m.}
\end{itemize}
\begin{itemize}
\item {Proveniência:(Lat. \textunderscore tithymalus\textunderscore )}
\end{itemize}
Planta euphorbiácea.
\section{Titi}
\begin{itemize}
\item {Grp. gram.:m.  e  f.}
\end{itemize}
\begin{itemize}
\item {Utilização:Infant.}
\end{itemize}
O mesmo que \textunderscore tio\textunderscore  ou \textunderscore tia\textunderscore .
\section{Titi}
\begin{itemize}
\item {Grp. gram.:m.}
\end{itemize}
Pássaro conirostro.
\section{Titia}
\begin{itemize}
\item {Grp. gram.:f.}
\end{itemize}
\begin{itemize}
\item {Utilização:Bras}
\end{itemize}
\begin{itemize}
\item {Utilização:Infant.}
\end{itemize}
Tia.
\section{Titica}
\begin{itemize}
\item {Grp. gram.:f.}
\end{itemize}
\begin{itemize}
\item {Utilização:Bras}
\end{itemize}
\begin{itemize}
\item {Utilização:Bras. do N}
\end{itemize}
Caça.
Excremento de gallinha.
\section{Titicar}
\begin{itemize}
\item {Grp. gram.:v. i.}
\end{itemize}
\begin{itemize}
\item {Utilização:Bras. do N}
\end{itemize}
Bater ou tocar de leve.
(Cp. \textunderscore tutucar\textunderscore )
\section{Titilação}
\begin{itemize}
\item {Grp. gram.:f.}
\end{itemize}
\begin{itemize}
\item {Proveniência:(Do lat. \textunderscore titillatio\textunderscore )}
\end{itemize}
Acto ou efeito de titilar.
\section{Titilamento}
\begin{itemize}
\item {Grp. gram.:m.}
\end{itemize}
\begin{itemize}
\item {Proveniência:(Lat. \textunderscore titillamentum\textunderscore )}
\end{itemize}
O mesmo que \textunderscore titilação\textunderscore .
\section{Titilante}
\begin{itemize}
\item {Grp. gram.:adj.}
\end{itemize}
\begin{itemize}
\item {Proveniência:(Lat. \textunderscore titillans\textunderscore )}
\end{itemize}
Que titila.
\section{Titilar}
\begin{itemize}
\item {Grp. gram.:v. t.}
\end{itemize}
\begin{itemize}
\item {Utilização:Fig.}
\end{itemize}
\begin{itemize}
\item {Grp. gram.:V. i.}
\end{itemize}
\begin{itemize}
\item {Proveniência:(Lat. \textunderscore titillare\textunderscore )}
\end{itemize}
Causar cócegas a.
Afagar; lisonjear.
Palpitar.
Têr estremecimentos.
\section{Titilar}
\begin{itemize}
\item {Grp. gram.:adj.}
\end{itemize}
\begin{itemize}
\item {Utilização:Anat.}
\end{itemize}
\begin{itemize}
\item {Proveniência:(Do lat. \textunderscore titillus\textunderscore )}
\end{itemize}
Diz-se das veias, que estão por baixo dos sovacos.
\section{Titillação}
\begin{itemize}
\item {Grp. gram.:f.}
\end{itemize}
\begin{itemize}
\item {Proveniência:(Do lat. \textunderscore titillatio\textunderscore )}
\end{itemize}
Acto ou effeito de titillar.
\section{Titillamento}
\begin{itemize}
\item {Grp. gram.:m.}
\end{itemize}
\begin{itemize}
\item {Proveniência:(Lat. \textunderscore titillamentum\textunderscore )}
\end{itemize}
O mesmo que \textunderscore titillação\textunderscore .
\section{Titillante}
\begin{itemize}
\item {Grp. gram.:adj.}
\end{itemize}
\begin{itemize}
\item {Proveniência:(Lat. \textunderscore titillans\textunderscore )}
\end{itemize}
Que titilla.
\section{Titillar}
\begin{itemize}
\item {Grp. gram.:v. t.}
\end{itemize}
\begin{itemize}
\item {Utilização:Fig.}
\end{itemize}
\begin{itemize}
\item {Grp. gram.:V. i.}
\end{itemize}
\begin{itemize}
\item {Proveniência:(Lat. \textunderscore titillare\textunderscore )}
\end{itemize}
Causar cócegas a.
Afagar; lisonjear.
Palpitar.
Têr estremecimentos.
\section{Titillar}
\begin{itemize}
\item {Grp. gram.:adj.}
\end{itemize}
\begin{itemize}
\item {Utilização:Anat.}
\end{itemize}
\begin{itemize}
\item {Proveniência:(Do lat. \textunderscore titillus\textunderscore )}
\end{itemize}
Diz-se das veias, que estão por baixo dos sovacos.
\section{Titilloso}
\begin{itemize}
\item {Grp. gram.:adj.}
\end{itemize}
\begin{itemize}
\item {Proveniência:(Lat. \textunderscore titillosus\textunderscore )}
\end{itemize}
O mesmo que \textunderscore titillante\textunderscore .
\section{Titim}
\begin{itemize}
\item {Grp. gram.:m.}
\end{itemize}
O mesmo que \textunderscore tingui\textunderscore .
\section{Titímalo}
\begin{itemize}
\item {Grp. gram.:m.}
\end{itemize}
\begin{itemize}
\item {Proveniência:(Lat. \textunderscore tithymalus\textunderscore )}
\end{itemize}
Planta euforbiácea.
\section{Titina}
\begin{itemize}
\item {Grp. gram.:f.}
\end{itemize}
Avezinha, de pennas cinzentas, salpicadas de branco.
\section{Titinga}
\begin{itemize}
\item {Grp. gram.:f.}
\end{itemize}
\begin{itemize}
\item {Utilização:Bras. do N}
\end{itemize}
\begin{itemize}
\item {Proveniência:(T. tupi)}
\end{itemize}
Manchas brancas no rosto ou em outras partes do corpo.
\section{Titio}
\begin{itemize}
\item {Grp. gram.:m.}
\end{itemize}
\begin{itemize}
\item {Utilização:Bras}
\end{itemize}
\begin{itemize}
\item {Utilização:Infant.}
\end{itemize}
Tio.
\section{Titónia}
\begin{itemize}
\item {Grp. gram.:f.}
\end{itemize}
\begin{itemize}
\item {Utilização:Poét.}
\end{itemize}
\begin{itemize}
\item {Proveniência:(Lat. \textunderscore Tithonia\textunderscore , n. p.)}
\end{itemize}
Aurora.
\section{Titónia}
\begin{itemize}
\item {Grp. gram.:f.}
\end{itemize}
Gênero de plantas, da fam. das compostas.
\section{Titor}
\begin{itemize}
\item {Grp. gram.:m.}
\end{itemize}
\begin{itemize}
\item {Utilização:ant.}
\end{itemize}
\begin{itemize}
\item {Utilização:Pop.}
\end{itemize}
O mesmo que \textunderscore tutor\textunderscore :«\textunderscore dos Mininhos Orfãos a que dam titores...\textunderscore »Côrtes de Santarém, art. 51. Cf. \textunderscore Eufrosina\textunderscore , 263.
\section{Titubar}
\begin{itemize}
\item {Proveniência:(Lat. \textunderscore titubare\textunderscore )}
\end{itemize}
\textunderscore v. i.\textunderscore  (e der.)
O mesmo que \textunderscore titubear\textunderscore . Cf. Pant. do Aveiro, \textunderscore Itiner.\textunderscore , 24, (2.^a ed.), etc.
\section{Titubeação}
\begin{itemize}
\item {Grp. gram.:f.}
\end{itemize}
Acto ou effeito de titubear.
\section{Titubeante}
\begin{itemize}
\item {Grp. gram.:adj.}
\end{itemize}
Que titubeia.
\section{Titubear}
\begin{itemize}
\item {Grp. gram.:v. i.}
\end{itemize}
\begin{itemize}
\item {Proveniência:(Do lat. \textunderscore titubare\textunderscore )}
\end{itemize}
Não poder estar firme.
Cambalear; vacillar.
Falar, hesitando ou trepidando.
Exprimir-se com difficuldade.
\section{Titué}
\begin{itemize}
\item {Grp. gram.:m.}
\end{itemize}
O mesmo que \textunderscore quitué\textunderscore .
\section{Titular}
\begin{itemize}
\item {Grp. gram.:v. t.}
\end{itemize}
\begin{itemize}
\item {Proveniência:(Lat. \textunderscore titulare\textunderscore )}
\end{itemize}
O mesmo que \textunderscore intitular\textunderscore .
Dar título a.
Basear em título.
Registar em títulos authênticos.
Registar.
\section{Titular}
\begin{itemize}
\item {Grp. gram.:adj.}
\end{itemize}
\begin{itemize}
\item {Grp. gram.:M.  e  f.}
\end{itemize}
\begin{itemize}
\item {Proveniência:(De \textunderscore título\textunderscore )}
\end{itemize}
Que tem título honorífico.
Honorário simplesmente, sem domínio real.
Nominal.
Pessôa, que tem título honorífico; pessôa titular.
Pessôa nobre.
Cada um dos membros de um Ministério: \textunderscore o titular da pasta da Justiça\textunderscore .
\section{Tituleiro}
\begin{itemize}
\item {Grp. gram.:m.}
\end{itemize}
\begin{itemize}
\item {Utilização:Ant.}
\end{itemize}
\begin{itemize}
\item {Proveniência:(De \textunderscore título\textunderscore )}
\end{itemize}
Letreiro; epitáphio.
\section{Título}
\begin{itemize}
\item {Grp. gram.:m.}
\end{itemize}
\begin{itemize}
\item {Proveniência:(Lat. \textunderscore titulus\textunderscore )}
\end{itemize}
Inscripção no frontespício de um livro, indicando a matéria que nêlle se trata.
Palavra ou palavras que, no princípio de um capítulo ou de qualquer escrito, indicam o assumpto dêste.
Letreiro.
Subdivisão, num código de leis ou numa obra de jurisprudência.
Denominação honorifica: \textunderscore o título de Marquês\textunderscore .
Nome, que exprime uma qualidade honrosa.
Reputação.
Pretexo.
Intuito.
Fundamento; causa: \textunderscore illustre, por muitos títulos\textunderscore .
Documento, que torna authêntico um direito.
\section{Titupururui}
\begin{itemize}
\item {Grp. gram.:m.}
\end{itemize}
\begin{itemize}
\item {Utilização:Bras}
\end{itemize}
Ave canora, cuju canto lembra o seu nome.
\section{Tiú}
\begin{itemize}
\item {Grp. gram.:m.}
\end{itemize}
Planta euphorbiácea do Brasil.
\section{Tiúba}
\begin{itemize}
\item {Grp. gram.:f.}
\end{itemize}
\begin{itemize}
\item {Utilização:Bras. do N}
\end{itemize}
Aguardente de cana; cachaça.
\section{Tiufadia}
\begin{itemize}
\item {Grp. gram.:f.}
\end{itemize}
\begin{itemize}
\item {Utilização:Ant.}
\end{itemize}
\begin{itemize}
\item {Proveniência:(De \textunderscore tiufado\textunderscore )}
\end{itemize}
Corpo de mil soldados, no exército godo.
\section{Tiufado}
\begin{itemize}
\item {Grp. gram.:m.}
\end{itemize}
\begin{itemize}
\item {Utilização:Ant.}
\end{itemize}
\begin{itemize}
\item {Proveniência:(Do gót. \textunderscore taihunda\textunderscore  + \textunderscore fath\textunderscore )}
\end{itemize}
Commandante de uma tiufadia.
\section{Tiza}
\begin{itemize}
\item {Grp. gram.:f.}
\end{itemize}
\begin{itemize}
\item {Utilização:Gír.}
\end{itemize}
Criança do sexo feminino; pequena.
(Por \textunderscore petiza\textunderscore , de \textunderscore petiz\textunderscore )
\section{Tlaspídeas}
\begin{itemize}
\item {Grp. gram.:f. pl.}
\end{itemize}
Tríbo de plantas crucíferas, que tem por tipo o tláspio.
\section{Tláspio}
\begin{itemize}
\item {Grp. gram.:m.}
\end{itemize}
\begin{itemize}
\item {Proveniência:(Do gr. \textunderscore thlaspis\textunderscore )}
\end{itemize}
Gênero de plantas crucíferas.
\section{Tleua}
\begin{itemize}
\item {Grp. gram.:f.}
\end{itemize}
Espécie de serpente do Brasil.
\section{Tlim}
\begin{itemize}
\item {Grp. gram.:m.}
\end{itemize}
\begin{itemize}
\item {Proveniência:(T. onom.)}
\end{itemize}
\textunderscore m.\textunderscore  (e der.)
O mesme que \textunderscore telim\textunderscore , etc.
Voz imitativa do sino, da campaínha ou do choque de dinheiro em metal.
\section{Tlipsencefalia}
\begin{itemize}
\item {Grp. gram.:f.}
\end{itemize}
Estado ou qualidade de tlipsencéfalo.
\section{Tlipsencéfalo}
\begin{itemize}
\item {Grp. gram.:m.}
\end{itemize}
\begin{itemize}
\item {Proveniência:(Do gr. \textunderscore thlipsis\textunderscore  + \textunderscore enkephalon\textunderscore )}
\end{itemize}
Monstro, cujo cérebro está desfigurado por efeito de compressão.
\section{Tlipsia}
\begin{itemize}
\item {Grp. gram.:f.}
\end{itemize}
\begin{itemize}
\item {Utilização:Med.}
\end{itemize}
\begin{itemize}
\item {Proveniência:(Do gr. \textunderscore thlipsis\textunderscore )}
\end{itemize}
Compressão dos vasos orgânicos por uma causa externa.
\section{Tmese}
\begin{itemize}
\item {Grp. gram.:f.}
\end{itemize}
\begin{itemize}
\item {Utilização:Gram.}
\end{itemize}
\begin{itemize}
\item {Proveniência:(Lat. \textunderscore tmesis\textunderscore )}
\end{itemize}
Divisão das partes de uma palavra composta, para nella se intercalar outra ou outras.
\section{Tó!}
\begin{itemize}
\item {Grp. gram.:interj.}
\end{itemize}
\begin{itemize}
\item {Utilização:T. da Bairrada}
\end{itemize}
\begin{itemize}
\item {Utilização:Prov.}
\end{itemize}
\begin{itemize}
\item {Utilização:trasm.}
\end{itemize}
Usa-se para chamar porcos.
Usa-se para afastar cães ou outros animaes.
(Cp. \textunderscore tô\textunderscore )
\section{Tô}
\begin{itemize}
\item {Grp. gram.:m.}
\end{itemize}
\begin{itemize}
\item {Utilização:Gír.}
\end{itemize}
Porco.
\section{Tôa}
\begin{itemize}
\item {Grp. gram.:f.}
\end{itemize}
\begin{itemize}
\item {Grp. gram.:Loc. adv.}
\end{itemize}
\begin{itemize}
\item {Proveniência:(Do ingl. \textunderscore tow\textunderscore )}
\end{itemize}
Corda, com que uma embarcação reboca outra.
Sirga.
Reboque.
\textunderscore Á tôa\textunderscore , ao acaso, impensadamente.
\section{Toada}
\begin{itemize}
\item {Grp. gram.:f.}
\end{itemize}
Acto ou effeito de toar.
Rumor; ruído.
Atoarda; boato.
Entoação.
Canto.
Maneira, systema; gôsto.
\section{Toadilha}
\begin{itemize}
\item {Grp. gram.:f.}
\end{itemize}
\begin{itemize}
\item {Proveniência:(Do cast. \textunderscore tonadilha\textunderscore )}
\end{itemize}
Pequena toada; cantiguinha.
\section{Toalha}
\begin{itemize}
\item {Grp. gram.:f.}
\end{itemize}
Peça de linho ou algodão, para cobrir mesas em que se come, ou para enxugar qualquer parte do corpo, quando se lava.
Peça análoga, com fôlhas ou rendas, para cobrir a parte superior do altar.
Tudo aquillo que tem fórma ou apparência de toalha.
Camada extensa: \textunderscore uma toalha de verdura\textunderscore .
(Talvez do lat. hyp. \textunderscore togadia\textunderscore , de \textunderscore toga\textunderscore )
\section{Toalhete}
\begin{itemize}
\item {fónica:lhê}
\end{itemize}
\begin{itemize}
\item {Grp. gram.:m.}
\end{itemize}
\begin{itemize}
\item {Utilização:P. us.}
\end{itemize}
\begin{itemize}
\item {Proveniência:(De \textunderscore toalha\textunderscore )}
\end{itemize}
O mesmo que \textunderscore guardanapo\textunderscore .
Pequena toalha de mãos.
\section{Toalhinha}
\begin{itemize}
\item {Grp. gram.:f.}
\end{itemize}
Pequena toalha.
Touca de freira.
\section{Toante}
\begin{itemize}
\item {Grp. gram.:adj.}
\end{itemize}
\begin{itemize}
\item {Proveniência:(Do lat. \textunderscore tonans\textunderscore , \textunderscore tonantis\textunderscore )}
\end{itemize}
Que tôa.
Diz-se das rimas, em que só coincidem as vogaes tónicas.
\section{Toanteiro}
\begin{itemize}
\item {Grp. gram.:adj.}
\end{itemize}
Que faz rimas toantes:«\textunderscore poétas toanteiros.\textunderscore »Camillo, \textunderscore Perfil do Marquês\textunderscore , 7.
\section{Toar}
\begin{itemize}
\item {Grp. gram.:v. i.}
\end{itemize}
\begin{itemize}
\item {Utilização:Fig.}
\end{itemize}
\begin{itemize}
\item {Grp. gram.:V. t.}
\end{itemize}
\begin{itemize}
\item {Utilização:Ant.}
\end{itemize}
\begin{itemize}
\item {Proveniência:(Do lat. \textunderscore tonare\textunderscore )}
\end{itemize}
Emittir som; soar.
Fazer estrondo.
Trovejar.
Adaptar-se.
Convir.
Sêr semelhante.
Agradar, aprazer: \textunderscore a tua proposta não me tôa\textunderscore .
Regular, guiar bem.
\section{Toarda}
\begin{itemize}
\item {Grp. gram.:f.}
\end{itemize}
\begin{itemize}
\item {Utilização:Des.}
\end{itemize}
O mesmo que \textunderscore atoarda\textunderscore . Cf. \textunderscore Peregrinação\textunderscore , XLII.
\section{Tobatinga}
\begin{itemize}
\item {Grp. gram.:f.}
\end{itemize}
\begin{itemize}
\item {Utilização:Bras}
\end{itemize}
O mesmo que \textunderscore tabatinga\textunderscore .
\section{Tobiano}
\begin{itemize}
\item {Grp. gram.:m.  e  adj.}
\end{itemize}
\begin{itemize}
\item {Utilização:Bras}
\end{itemize}
Diz-se de um cavallo, de certa raça e manchas grandes e compridas.
\section{Toca}
\begin{itemize}
\item {Grp. gram.:f.}
\end{itemize}
\begin{itemize}
\item {Utilização:Fig.}
\end{itemize}
\begin{itemize}
\item {Proveniência:(Do cast. \textunderscore tueca\textunderscore )}
\end{itemize}
Buraco, onde se abrigam coêlhos ou outros animaes; covil.
Habitação pequena e miserável.
\section{Toca}
\begin{itemize}
\item {Grp. gram.:f.}
\end{itemize}
\begin{itemize}
\item {Utilização:Açor}
\end{itemize}
Parte da planta, que mergulha na terra; raíz.
(Cp. \textunderscore tôco\textunderscore )
\section{Toça}
\begin{itemize}
\item {Grp. gram.:f.}
\end{itemize}
\begin{itemize}
\item {Utilização:Prov.}
\end{itemize}
\begin{itemize}
\item {Utilização:beir.}
\end{itemize}
Outra fórma de \textunderscore torça\textunderscore , padieira, vêrga de porta.
\section{Tocada}
\begin{itemize}
\item {Grp. gram.:f.}
\end{itemize}
\begin{itemize}
\item {Utilização:Bras. do S}
\end{itemize}
\begin{itemize}
\item {Proveniência:(De \textunderscore tocar\textunderscore )}
\end{itemize}
Acto de chicotear na corrida (um cavallo), para o governar.
\section{Tocadela}
\begin{itemize}
\item {Grp. gram.:f.}
\end{itemize}
\begin{itemize}
\item {Utilização:Fam.}
\end{itemize}
Acto ou effeito de tocar.
Tocata.
\section{Tocadilho}
\begin{itemize}
\item {Grp. gram.:m.}
\end{itemize}
\begin{itemize}
\item {Proveniência:(De \textunderscore tocar\textunderscore )}
\end{itemize}
Jôgo, semelhante ao do gamão.
\section{Tocado}
\begin{itemize}
\item {Grp. gram.:adj.}
\end{itemize}
\begin{itemize}
\item {Utilização:Fam.}
\end{itemize}
\begin{itemize}
\item {Proveniência:(De \textunderscore tocar\textunderscore )}
\end{itemize}
Um tanto ébrio.
\section{Tocador}
\begin{itemize}
\item {Grp. gram.:m.  e  adj.}
\end{itemize}
\begin{itemize}
\item {Utilização:Bras. de Minas}
\end{itemize}
\begin{itemize}
\item {Utilização:Gír.}
\end{itemize}
O que toca.
Almocreve, que guia um lote de animaes de carga.
Bebedor.
\section{Tocadura}
\begin{itemize}
\item {Grp. gram.:f.}
\end{itemize}
O mesmo que \textunderscore tocadela\textunderscore .
Contusão resultante de um pé do animal tocar no outro pé, do lado interior. Cf. Mac. Pinto, \textunderscore Comp. de Veter.\textunderscore , 414 e 417.
\section{Tocaia}
\begin{itemize}
\item {Grp. gram.:f.}
\end{itemize}
\begin{itemize}
\item {Utilização:Bras}
\end{itemize}
\begin{itemize}
\item {Utilização:Bras. do N}
\end{itemize}
\begin{itemize}
\item {Proveniência:(T. tupi)}
\end{itemize}
Emboscada, em que se occulta alguém que quere matar outrem ou que quere caçar.
Poleiro de gallinhas.
\section{Tocaia}
\begin{itemize}
\item {Grp. gram.:f.}
\end{itemize}
(Fem. de \textunderscore tocaio\textunderscore )
\section{Tocaiar}
\begin{itemize}
\item {Grp. gram.:v. t.}
\end{itemize}
\begin{itemize}
\item {Proveniência:(De \textunderscore tocaia\textunderscore ^1)}
\end{itemize}
Emboscar-se para matar ou para caçar.
\section{Tocaio}
\begin{itemize}
\item {Grp. gram.:adj.}
\end{itemize}
\begin{itemize}
\item {Utilização:Prov.}
\end{itemize}
\begin{itemize}
\item {Utilização:Bras. do N}
\end{itemize}
\begin{itemize}
\item {Utilização:trasm.}
\end{itemize}
O mesmo que \textunderscore homónymo\textunderscore .
(Cast. \textunderscore tocayo\textunderscore )
\section{Tocajé}
\begin{itemize}
\item {Grp. gram.:m.}
\end{itemize}
Arbusto brasileiro, (\textunderscore rupala\textunderscore ).
\section{Toca-lápis}
\begin{itemize}
\item {Grp. gram.:m.}
\end{itemize}
Uma das pernas do compasso, na qual se encaixa o lápis, para descrever círculos ou arcos.
\section{Tocamento}
\begin{itemize}
\item {Grp. gram.:m.}
\end{itemize}
O mesmo que \textunderscore toque\textunderscore ^1.
\section{Tocandera}
\begin{itemize}
\item {Grp. gram.:f.}
\end{itemize}
\begin{itemize}
\item {Utilização:Bras}
\end{itemize}
Espécie de formiga das regiões do Amazonas.
\section{Tocandiras}
\begin{itemize}
\item {Grp. gram.:m. Pl.}
\end{itemize}
Índios selvagens das margens do Apaporis, no Brasil.
\section{Tocanos}
\begin{itemize}
\item {Grp. gram.:m. Pl.}
\end{itemize}
\begin{itemize}
\item {Utilização:Bras}
\end{itemize}
Tríbo de indígenas do Pará.
\section{Tocante}
\begin{itemize}
\item {Grp. gram.:adj.}
\end{itemize}
\begin{itemize}
\item {Grp. gram.:Loc. prep.}
\end{itemize}
Que toca.
Relativo.
\textunderscore No tocante a\textunderscore , a respeito de; relativamente a.
\section{Tocantins}
\begin{itemize}
\item {Grp. gram.:m. Pl.}
\end{itemize}
\begin{itemize}
\item {Utilização:Bras}
\end{itemize}
Tríbo de aborígenes do Pará.
\section{Tocar}
\begin{itemize}
\item {Grp. gram.:v. t.}
\end{itemize}
\begin{itemize}
\item {Utilização:Gír.}
\end{itemize}
\begin{itemize}
\item {Grp. gram.:V. i.}
\end{itemize}
\begin{itemize}
\item {Utilização:Ant.}
\end{itemize}
\begin{itemize}
\item {Grp. gram.:V. p.}
\end{itemize}
\begin{itemize}
\item {Utilização:Gír.}
\end{itemize}
Pôr a mão em.
Têr contacto com.
Roçar.
Bater.
Tanger, fazer soar: \textunderscore tocar viola\textunderscore .
Attingir com um golpe, na esgrima.
Dizer respeito a.
Commover, sensibilizar: \textunderscore tocaram-me aquelles lamentos\textunderscore .
Influir em; excitar.
Mencionar.
Aproximar-se de.
Retocar; aprimorar.
Beber.
Pertencer.
Dizer respeito.
Estar em contacto ou próximo.
Fazer soar um instrumento: \textunderscore o pequeno toca bem\textunderscore .
Raiar, orçar:«\textunderscore tocava por sete annos.\textunderscore »Camillo, \textunderscore Enjeitada\textunderscore , 113.
Dar sinal, por meio de instrumento.
Referir-se desfavoravelmente.
Bater em baixio (uma embarcação).
Lançar âncora, fundear, de passagem ou fazendo escala: \textunderscore o navio tocou em Lisbôa\textunderscore .
\textunderscore Tocar rijamente\textunderscore , andar a toda a pressa. Cf. Pant. de Aveiro, \textunderscore Itiner.\textunderscore , 254, (2.^a ed.).
Têr contacto.
Pôr-se em contacto.
Impressionar-se, commover-se.
Melindrar-se.
Começar a apodrecer, (falando-se de fruta).
Embriagar-se um pouco.
(Cp. it. \textunderscore toccare\textunderscore )
\section{Tocari}
\begin{itemize}
\item {Grp. gram.:m.}
\end{itemize}
\begin{itemize}
\item {Utilização:Bras}
\end{itemize}
Árvore fructífera dos sertões.
\section{Tó-carocha!}
\begin{itemize}
\item {Grp. gram.:interj.}
\end{itemize}
(designativa de \textunderscore negação\textunderscore  ou \textunderscore recusa\textunderscore :«\textunderscore isso de amigos, replicou o canastreiro, tó-carocha!\textunderscore »Camillo, \textunderscore Volcoens\textunderscore , 54)
\section{Tó-carocho!}
\begin{itemize}
\item {Grp. gram.:interj.}
\end{itemize}
O mesmo que \textunderscore tó-carocha!\textunderscore 
\section{Tocarola}
\begin{itemize}
\item {Grp. gram.:f.}
\end{itemize}
\begin{itemize}
\item {Utilização:Fam.}
\end{itemize}
\begin{itemize}
\item {Proveniência:(De \textunderscore tocar\textunderscore )}
\end{itemize}
Apêrto de mão, por cumprimento.
Tocata desafinada.
\section{Tocata}
\begin{itemize}
\item {Grp. gram.:f.}
\end{itemize}
\begin{itemize}
\item {Utilização:Pop.}
\end{itemize}
\begin{itemize}
\item {Proveniência:(De \textunderscore tocar\textunderscore )}
\end{itemize}
Toque de instrumentos; musicata.
\section{Toca-teclas}
\begin{itemize}
\item {Grp. gram.:m.}
\end{itemize}
\begin{itemize}
\item {Utilização:Deprec.}
\end{itemize}
Mau tocador de piano. Cf. Ortigão, \textunderscore Praias\textunderscore , 90 e 92.
\section{Tocear}
\begin{itemize}
\item {Grp. gram.:v. t.}
\end{itemize}
\begin{itemize}
\item {Utilização:Prov.}
\end{itemize}
\begin{itemize}
\item {Utilização:beir.}
\end{itemize}
Pôr toça em (porta ou janela).
\section{Tocha}
\begin{itemize}
\item {Grp. gram.:f.}
\end{itemize}
\begin{itemize}
\item {Proveniência:(Do it. \textunderscore torcia\textunderscore )}
\end{itemize}
Grande vela de cera.
Brandão.
Facho.
Brilho.
\section{Tocheira}
\begin{itemize}
\item {Grp. gram.:f.}
\end{itemize}
Castiçal para tocha.
\section{Tocheiro}
\begin{itemize}
\item {Grp. gram.:m.}
\end{itemize}
O mesmo que \textunderscore tocheira\textunderscore .
\section{Tocho}
\begin{itemize}
\item {Grp. gram.:m.}
\end{itemize}
\begin{itemize}
\item {Utilização:Des.}
\end{itemize}
Moca, pau, cacete.
(Cp. cast. \textunderscore tocho\textunderscore )
\section{Tôco}
\begin{itemize}
\item {Grp. gram.:m.}
\end{itemize}
\begin{itemize}
\item {Proveniência:(It. \textunderscore tocco\textunderscore )}
\end{itemize}
Parte de um tronco vegetal, que fica ligada á terra, depois de cortada a árvore.
Cacete.
Pedaço de vela ou tocha; coto.
Resto de um mastro que se desarvorou.
\section{Tocó}
\begin{itemize}
\item {Grp. gram.:adj.}
\end{itemize}
\begin{itemize}
\item {Utilização:Bras. do N}
\end{itemize}
Diz-se do animal que não tem cauda; suro.
(Cp. \textunderscore cotó\textunderscore ^2)
\section{Tocografia}
\begin{itemize}
\item {Grp. gram.:f.}
\end{itemize}
\begin{itemize}
\item {Utilização:Med.}
\end{itemize}
Descripção dos partos.
(Cp. \textunderscore tocógrafo\textunderscore )
\section{Tocográfico}
\begin{itemize}
\item {Grp. gram.:adj.}
\end{itemize}
Relativo á tocografia.
\section{Tocógrafo}
\begin{itemize}
\item {Grp. gram.:m.}
\end{itemize}
\begin{itemize}
\item {Proveniência:(Do gr. \textunderscore tokos\textunderscore  + \textunderscore graphein\textunderscore )}
\end{itemize}
Autor de qualquer tocografia.
\section{Tocographia}
\begin{itemize}
\item {Grp. gram.:f.}
\end{itemize}
\begin{itemize}
\item {Utilização:Med.}
\end{itemize}
Descripção dos partos.
(Cp. \textunderscore tocógrapho\textunderscore )
\section{Tocográphico}
\begin{itemize}
\item {Grp. gram.:adj.}
\end{itemize}
Relativo á tocographia.
\section{Tocógrapho}
\begin{itemize}
\item {Grp. gram.:m.}
\end{itemize}
\begin{itemize}
\item {Proveniência:(Do gr. \textunderscore tokos\textunderscore  + \textunderscore graphein\textunderscore )}
\end{itemize}
Autor de qualquer tocographia.
\section{Tocoiena}
\begin{itemize}
\item {Grp. gram.:f.}
\end{itemize}
Gênero de plantas rubiáceas.
\section{Tocologia}
\begin{itemize}
\item {Grp. gram.:f.}
\end{itemize}
\begin{itemize}
\item {Utilização:Med.}
\end{itemize}
\begin{itemize}
\item {Proveniência:(Do gr. \textunderscore tokos\textunderscore  + \textunderscore logos\textunderscore )}
\end{itemize}
Tratado dos partos.
\section{Tocológico}
\begin{itemize}
\item {Grp. gram.:adj.}
\end{itemize}
Relativo á tocologia.
\section{Tocomático}
\begin{itemize}
\item {Grp. gram.:m.}
\end{itemize}
\begin{itemize}
\item {Proveniência:(Do gr. \textunderscore tokos\textunderscore )}
\end{itemize}
Espécie de manequim, que dá aos estudantes de cirurgia ideia da fórma e profundidade do útero, para que elles se exercitem nas operações dos partos.
\section{Toconomia}
\begin{itemize}
\item {Grp. gram.:f.}
\end{itemize}
\begin{itemize}
\item {Proveniência:(Do gr. \textunderscore tokos\textunderscore  + \textunderscore nomos\textunderscore )}
\end{itemize}
Conjunto das regras, que formam a arte dos partos.
\section{Toconómico}
\begin{itemize}
\item {Grp. gram.:adj.}
\end{itemize}
Relativo á toconomia.
\section{Tocotechnia}
\begin{itemize}
\item {Grp. gram.:f.}
\end{itemize}
\begin{itemize}
\item {Proveniência:(Do gr. \textunderscore tokos\textunderscore  + \textunderscore tekne\textunderscore )}
\end{itemize}
Arte de partejar.
\section{Tocotéchnico}
\begin{itemize}
\item {Grp. gram.:adj.}
\end{itemize}
Relativo á tocotechnia.
\section{Tocotecnia}
\begin{itemize}
\item {Grp. gram.:f.}
\end{itemize}
\begin{itemize}
\item {Proveniência:(Do gr. \textunderscore tokos\textunderscore  + \textunderscore tekne\textunderscore )}
\end{itemize}
Arte de partejar.
\section{Tocotécnico}
\begin{itemize}
\item {Grp. gram.:adj.}
\end{itemize}
Relativo á tocotechnia.
\section{Tóda}
\begin{itemize}
\item {Grp. gram.:f.}
\end{itemize}
(V.todeiro)
\section{Todároa}
\begin{itemize}
\item {Grp. gram.:f.}
\end{itemize}
Gênero de plantas umbellíferas.
\section{Todavia}
\begin{itemize}
\item {Grp. gram.:adv.  e  conj.}
\end{itemize}
\begin{itemize}
\item {Proveniência:(De \textunderscore todo\textunderscore  + \textunderscore via\textunderscore )}
\end{itemize}
Contudo; porém; entretanto; ainda assim.
\section{Todeiro}
\begin{itemize}
\item {Grp. gram.:m.}
\end{itemize}
\begin{itemize}
\item {Proveniência:(Do lat. \textunderscore todus\textunderscore )}
\end{itemize}
Pássaro fissirostro.
\section{Todirostro}
\begin{itemize}
\item {fónica:rós}
\end{itemize}
\begin{itemize}
\item {Grp. gram.:m.}
\end{itemize}
\begin{itemize}
\item {Proveniência:(Do lat. \textunderscore todus\textunderscore  + \textunderscore rostrum\textunderscore )}
\end{itemize}
Gênero de aves muscívoras.
\section{Todirrostro}
\begin{itemize}
\item {Grp. gram.:m.}
\end{itemize}
\begin{itemize}
\item {Proveniência:(Do lat. \textunderscore todus\textunderscore  + \textunderscore rostrum\textunderscore )}
\end{itemize}
Gênero de aves muscívoras.
\section{Todo}
\begin{itemize}
\item {Grp. gram.:adj.}
\end{itemize}
\begin{itemize}
\item {Grp. gram.:M.}
\end{itemize}
\begin{itemize}
\item {Grp. gram.:Loc. adv.}
\end{itemize}
\begin{itemize}
\item {Grp. gram.:Pron.}
\end{itemize}
\begin{itemize}
\item {Utilização:Ant.}
\end{itemize}
\begin{itemize}
\item {Grp. gram.:Pl.}
\end{itemize}
\begin{itemize}
\item {Utilização:Prov.}
\end{itemize}
\begin{itemize}
\item {Utilização:alg.}
\end{itemize}
\begin{itemize}
\item {Proveniência:(Do lat. \textunderscore totus\textunderscore )}
\end{itemize}
Completo, íntegro.
Que não deixa nada de fóra.
A que não falta parte alguma.
Qualquer; cada: \textunderscore todo homem que pensa\textunderscore .
Conjunto.
Massa.
Generalidade.
\textunderscore De todo\textunderscore , ou \textunderscore de todo em todo\textunderscore , completamente. Cf. Camillo, \textunderscore Enjeitada\textunderscore , 168.
O mesmo que \textunderscore tudo\textunderscore ^1:«\textunderscore a ilha e todo o mais desamparado\textunderscore ». \textunderscore Lusíadas\textunderscore , I, 91.
Toda a gente; a humanidade: \textunderscore todos soffrem\textunderscore .
\textunderscore Todos ambos\textunderscore , o mesmo que \textunderscore ambos\textunderscore .
\section{Todo-nada}
\begin{itemize}
\item {Grp. gram.:m.}
\end{itemize}
O mesmo que \textunderscore tudo-nada\textunderscore :«\textunderscore dá cá um todo-nada de aguardente\textunderscore ». Camillo, \textunderscore Brasileira\textunderscore , 310.
\section{Todonada}
\begin{itemize}
\item {Grp. gram.:m.}
\end{itemize}
O mesmo que \textunderscore tudo-nada\textunderscore :«\textunderscore dá cá um todonada de aguardente\textunderscore ». Camillo, \textunderscore Brasileira\textunderscore , 310.
\section{Todo-poderoso}
\begin{itemize}
\item {Grp. gram.:m.  e  adj.}
\end{itemize}
O que póde tudo.
Omnipotente; Deus.
\section{Toeira}
\begin{itemize}
\item {Grp. gram.:f.}
\end{itemize}
\begin{itemize}
\item {Utilização:Prov.}
\end{itemize}
\begin{itemize}
\item {Utilização:trasm.}
\end{itemize}
\begin{itemize}
\item {Proveniência:(De \textunderscore toar\textunderscore )}
\end{itemize}
Cada uma das duas cordas immediatas aos dois bordões da guitarra. Cf. Camillo, \textunderscore Corja\textunderscore , 165.
O mesmo que \textunderscore trovoada\textunderscore .
\section{Toeiro}
\begin{itemize}
\item {Grp. gram.:adj.}
\end{itemize}
\begin{itemize}
\item {Utilização:Prov.}
\end{itemize}
\begin{itemize}
\item {Utilização:beir.}
\end{itemize}
\begin{itemize}
\item {Proveniência:(De \textunderscore toar\textunderscore )}
\end{itemize}
Que tem som forte. (Colhido no Fundão)
\section{Toêsa}
\begin{itemize}
\item {Grp. gram.:f.}
\end{itemize}
\begin{itemize}
\item {Utilização:Pop.}
\end{itemize}
\begin{itemize}
\item {Proveniência:(Do fr. \textunderscore toise\textunderscore )}
\end{itemize}
Medida francesa de seis pés.
Pé muito grande.
\section{Tofel}
\begin{itemize}
\item {Grp. gram.:m.}
\end{itemize}
Espécie de pandeiro antigo.
(Cp. \textunderscore adufe\textunderscore )
\section{Tofes}
\begin{itemize}
\item {Grp. gram.:m. pl.}
\end{itemize}
\begin{itemize}
\item {Utilização:Prov.}
\end{itemize}
\begin{itemize}
\item {Utilização:trasm.}
\end{itemize}
Laçarada, muitas fitas nos vestidos.
(Relaciona-se com \textunderscore tufo\textunderscore ?)
\section{Toga}
\begin{itemize}
\item {Grp. gram.:f.}
\end{itemize}
\begin{itemize}
\item {Utilização:Fig.}
\end{itemize}
\begin{itemize}
\item {Proveniência:(Lat. \textunderscore toga\textunderscore )}
\end{itemize}
Traje antigo dos Romanos, espécie de capa.
Vestuário de magistrado; beca.
A magistratura.
\section{Togado}
\begin{itemize}
\item {Grp. gram.:adj.}
\end{itemize}
\begin{itemize}
\item {Grp. gram.:M.}
\end{itemize}
\begin{itemize}
\item {Proveniência:(Do lat. \textunderscore togatus\textunderscore )}
\end{itemize}
Que usa toga.
Magistrado judicial.
\section{Togata}
\begin{itemize}
\item {Grp. gram.:f.}
\end{itemize}
\begin{itemize}
\item {Proveniência:(Lat. \textunderscore togata\textunderscore )}
\end{itemize}
Entre os antigos Romanos, peça theatral, patriótica ou de assumpto nacional.
\section{Toiça}
\begin{itemize}
\item {Grp. gram.:f.}
\end{itemize}
\begin{itemize}
\item {Utilização:Prov.}
\end{itemize}
\begin{itemize}
\item {Utilização:minh.}
\end{itemize}
\begin{itemize}
\item {Utilização:Prov.}
\end{itemize}
\begin{itemize}
\item {Utilização:trasm.}
\end{itemize}
\begin{itemize}
\item {Proveniência:(Do b. lat. \textunderscore toussa\textunderscore ?)}
\end{itemize}
Grande vergôntea de castanheiro, de que se fazem arcos para pipas.
Vara ou pernada alta e grossa de qualquer árvore.
O pé da cana de açúcar.
Moita de feno grosseiro.
O mesmo que \textunderscore toiço\textunderscore .
Qualquer moita:«\textunderscore toiça de carvalhos\textunderscore ». Camillo, \textunderscore Mem. do Cárcere\textunderscore .
\section{Toiceira}
\begin{itemize}
\item {Grp. gram.:f.}
\end{itemize}
\begin{itemize}
\item {Utilização:T. do Fundão}
\end{itemize}
Grande toiça.
Pé de uma planta, com raízes.
\section{Toiceiral}
\begin{itemize}
\item {Grp. gram.:m.}
\end{itemize}
O mesmo que \textunderscore moitedo\textunderscore . Cf. Neto, \textunderscore Baladilhas\textunderscore , 271.
\section{Toicinheira}
\begin{itemize}
\item {Grp. gram.:f.}
\end{itemize}
\begin{itemize}
\item {Utilização:Prov.}
\end{itemize}
\begin{itemize}
\item {Utilização:minh.}
\end{itemize}
O mesmo que \textunderscore matadeira\textunderscore .
\section{Toicinheiro}
\begin{itemize}
\item {Grp. gram.:m.}
\end{itemize}
Aquelle que vende toicinho ou qualquer carne de porco.
\section{Toicinho}
\begin{itemize}
\item {Grp. gram.:m.}
\end{itemize}
\begin{itemize}
\item {Proveniência:(Do cast. \textunderscore tocino\textunderscore )}
\end{itemize}
Gordura dos porcos, subjacente á pelle.
\section{Toiço}
\begin{itemize}
\item {Grp. gram.:m.}
\end{itemize}
\begin{itemize}
\item {Utilização:Prov.}
\end{itemize}
\begin{itemize}
\item {Utilização:minh.}
\end{itemize}
O mesmo que \textunderscore temão\textunderscore  (do carro).
Parte do carro, donde sái o cabeçalho.
\textunderscore Pôr-se ao toiço\textunderscore , resistir.
(Cp. \textunderscore toiça\textunderscore )
\section{Toiene}
\begin{itemize}
\item {Grp. gram.:adj.}
\end{itemize}
\begin{itemize}
\item {Utilização:Gír. de pedreiros.}
\end{itemize}
Teu.
\section{Tóino}
\begin{itemize}
\item {Grp. gram.:m.}
\end{itemize}
\begin{itemize}
\item {Utilização:T. da Covilhan}
\end{itemize}
Vadio, tunante.
(Cp. \textunderscore tuno\textunderscore )
\section{Toira}
\begin{itemize}
\item {Grp. gram.:f.}
\end{itemize}
\begin{itemize}
\item {Utilização:Fam.}
\end{itemize}
\begin{itemize}
\item {Proveniência:(Do lat. \textunderscore taura\textunderscore )}
\end{itemize}
Vaca estéril.
Mulhér irascível, bravia.
\section{Toira}
\begin{itemize}
\item {Grp. gram.:f.}
\end{itemize}
\begin{itemize}
\item {Utilização:Prov.}
\end{itemize}
\begin{itemize}
\item {Utilização:alg.}
\end{itemize}
O mesmo que \textunderscore tacho\textunderscore .
\section{Toirada}
\begin{itemize}
\item {Grp. gram.:f.}
\end{itemize}
Bando de toiros.
Corrida de toiros, em circos.
\section{Toiral}
\begin{itemize}
\item {Grp. gram.:m.}
\end{itemize}
\begin{itemize}
\item {Grp. gram.:Adj.}
\end{itemize}
Lugar, onde um coêlho costuma estercar, e onde os caçadores o esperam.
Diz-se de uma variedade de azeitona, também chamada \textunderscore madural\textunderscore .
\section{Toiralho}
\begin{itemize}
\item {Grp. gram.:m.}
\end{itemize}
\begin{itemize}
\item {Utilização:T. de Turquel}
\end{itemize}
Estêrco de coêlho.
(Cp. \textunderscore toiral\textunderscore )
\section{Toirão}
\begin{itemize}
\item {Grp. gram.:m.}
\end{itemize}
\begin{itemize}
\item {Utilização:Fam.}
\end{itemize}
\begin{itemize}
\item {Proveniência:(De \textunderscore toiro\textunderscore )}
\end{itemize}
Furão bravo.
Criança traquina.
\section{Toirão-de-mato}
\begin{itemize}
\item {Grp. gram.:m.}
\end{itemize}
Ave gallinácea, semelhante á cordoniz.
\section{Toiraria}
\begin{itemize}
\item {Grp. gram.:f.}
\end{itemize}
\begin{itemize}
\item {Utilização:Fam.}
\end{itemize}
\begin{itemize}
\item {Proveniência:(De \textunderscore toiro\textunderscore )}
\end{itemize}
Barulho, desordem.
Fúria.
\section{Toireador}
\begin{itemize}
\item {Grp. gram.:m.  e  adj.}
\end{itemize}
O que toirea; toireiro.
\section{Toirear}
\begin{itemize}
\item {Grp. gram.:v. t.}
\end{itemize}
\begin{itemize}
\item {Utilização:Fig.}
\end{itemize}
\begin{itemize}
\item {Utilização:Bras}
\end{itemize}
\begin{itemize}
\item {Grp. gram.:V. i.}
\end{itemize}
Correr ou lidar (toiros) num circo ou praça.
Perseguir, atacar.
Namorar.
Correr toiros.
\section{Toireio}
\begin{itemize}
\item {Grp. gram.:m.}
\end{itemize}
Acto, effeito ou arte de toirear.
\section{Toireiro}
\begin{itemize}
\item {Grp. gram.:m.}
\end{itemize}
\begin{itemize}
\item {Grp. gram.:Adj.}
\end{itemize}
Aquelle que toireia, especialmente o que toireia por hábito ou profissão.
Relativo a toiro.
\section{Toirejão}
\begin{itemize}
\item {Grp. gram.:m.}
\end{itemize}
Cavilha, que ampara as rodas da carreta, nas extremidades do eixo.
\section{Toirejar}
\begin{itemize}
\item {Grp. gram.:v. i.  e  t.}
\end{itemize}
(V.toirear)
\section{Toiril}
\begin{itemize}
\item {Grp. gram.:m.}
\end{itemize}
Curral de gado vaccum.
Lugar, annexo á praça de toiros, em que êstes se guardam antes da corrida.
(Cp. cast. \textunderscore toril\textunderscore )
\section{Toirinha}
\begin{itemize}
\item {Grp. gram.:f.}
\end{itemize}
\begin{itemize}
\item {Utilização:Fam.}
\end{itemize}
\begin{itemize}
\item {Proveniência:(De \textunderscore toiro\textunderscore )}
\end{itemize}
Corrida de novilhas mansas.
Imitação de uma corrida de toiros, sendo êstes representados por canastras, etc.
Objecto de troça.
\section{Toirinha}
\begin{itemize}
\item {Grp. gram.:f.}
\end{itemize}
Peixe pletógnatho.
\section{Toirista}
\begin{itemize}
\item {Grp. gram.:m.}
\end{itemize}
O mesmo que \textunderscore toireiro\textunderscore . Cf. Herculano, \textunderscore Lendas\textunderscore , II, 183.
\section{Toiro}
\begin{itemize}
\item {Grp. gram.:m.}
\end{itemize}
\begin{itemize}
\item {Utilização:Fig.}
\end{itemize}
\begin{itemize}
\item {Utilização:Prov.}
\end{itemize}
\begin{itemize}
\item {Utilização:alg.}
\end{itemize}
\begin{itemize}
\item {Grp. gram.:Pl.}
\end{itemize}
\begin{itemize}
\item {Proveniência:(Do lat. \textunderscore taurus\textunderscore )}
\end{itemize}
Boi, que não é castrado; boi bravo.
Homem robusto e fogoso.
Um dos signos de Zodíaco.
Tacho de papas.
O mesmo que \textunderscore toirada\textunderscore : \textunderscore hoje, assistiu pouca gente aos toiros\textunderscore .
\section{Toiro-gallego}
\begin{itemize}
\item {Grp. gram.:m.}
\end{itemize}
O mesmo que \textunderscore garcenho\textunderscore .
\section{Toiro-paul}
\begin{itemize}
\item {Grp. gram.:m.}
\end{itemize}
O mesmo que \textunderscore abetoiro\textunderscore .
\section{Toiruno}
\begin{itemize}
\item {Grp. gram.:adj.}
\end{itemize}
\begin{itemize}
\item {Proveniência:(De \textunderscore toiro\textunderscore )}
\end{itemize}
Mal castrado, (falando-se do boi).
\section{Toita}
\begin{itemize}
\item {Grp. gram.:f.}
\end{itemize}
\begin{itemize}
\item {Utilização:Prov.}
\end{itemize}
\begin{itemize}
\item {Utilização:beir.}
\end{itemize}
\begin{itemize}
\item {Utilização:Ant.}
\end{itemize}
Mata de castanheiros.
(Cp. \textunderscore toiça\textunderscore )
\section{Tojal}
\begin{itemize}
\item {Grp. gram.:m.}
\end{itemize}
Terreno, onde crescem tojos.
\section{Tojeira}
\begin{itemize}
\item {Grp. gram.:f.}
\end{itemize}
Mulhér, que conduz tojo para os fornos.
Tojo.
Tojal.
\section{Tojeiro}
\begin{itemize}
\item {Grp. gram.:m.}
\end{itemize}
Aquelle que, conduz tojo para os fornos.
Tojo grande.
\section{Tojo}
\begin{itemize}
\item {fónica:tô}
\end{itemize}
\begin{itemize}
\item {Grp. gram.:m.}
\end{itemize}
Planta espinhosa, que vegeta de ordinário em sítios áridos, dando flôres amarelas, e de que há várias espécies, como: \textunderscore tojo-arnal\textunderscore , (\textunderscore ulex europaeus\textunderscore , Lin.); \textunderscore tojo durázio\textunderscore , (\textunderscore ulex scaber\textunderscore , Kze.); \textunderscore tojo-gatenho\textunderscore , (\textunderscore ulex lusit anicus\textunderscore , Mariz); \textunderscore tojo-gatum\textunderscore , (\textunderscore ulex vaillanti\textunderscore , Web.); \textunderscore tojo molar\textunderscore (\textunderscore ulex nanus\textunderscore , Forst)
(Talvez do lat. \textunderscore toxicum\textunderscore )
\section{Tolã}
\begin{itemize}
\item {Grp. gram.:f.}
\end{itemize}
\begin{itemize}
\item {Utilização:Pop.}
\end{itemize}
\begin{itemize}
\item {Proveniência:(De \textunderscore tolo\textunderscore )}
\end{itemize}
Lôgro; burla.
\section{Tóla}
\begin{itemize}
\item {Grp. gram.:f.}
\end{itemize}
\begin{itemize}
\item {Utilização:Chul.}
\end{itemize}
\begin{itemize}
\item {Proveniência:(De \textunderscore tolo\textunderscore )}
\end{itemize}
Cabeça; mioleira.
\section{Tóla}
\begin{itemize}
\item {Grp. gram.:f.}
\end{itemize}
Torquez de madeira, usada por penteeiros.
\section{Tóla}
\begin{itemize}
\item {Grp. gram.:f.}
\end{itemize}
\begin{itemize}
\item {Utilização:Prov.}
\end{itemize}
\begin{itemize}
\item {Utilização:minh.}
\end{itemize}
Parte do rêgo, onde há roturas, pelas quaes se perde a água.
\section{Tolaço}
\begin{itemize}
\item {Grp. gram.:m.  e  adj.}
\end{itemize}
\begin{itemize}
\item {Utilização:Des.}
\end{itemize}
Grande tolo. Cf. G. Vicente, I, 224.
\section{Tolamente}
\begin{itemize}
\item {Grp. gram.:adv.}
\end{itemize}
De modo tolo.
Insensatamente.
Sem prudência; sem juízo.
\section{Tolan}
\begin{itemize}
\item {Grp. gram.:f.}
\end{itemize}
\begin{itemize}
\item {Utilização:Pop.}
\end{itemize}
\begin{itemize}
\item {Proveniência:(De \textunderscore tolo\textunderscore )}
\end{itemize}
Lôgro; burla.
\section{Tolanga}
\begin{itemize}
\item {Grp. gram.:f.}
\end{itemize}
\begin{itemize}
\item {Utilização:Bras}
\end{itemize}
Planta medicinal.
\section{Tolano}
\begin{itemize}
\item {Grp. gram.:m.}
\end{itemize}
\begin{itemize}
\item {Proveniência:(De \textunderscore tôla\textunderscore ^1?)}
\end{itemize}
Sulco, no paladar das cavalgaduras.
\section{Tolaria}
\begin{itemize}
\item {Grp. gram.:f.}
\end{itemize}
\begin{itemize}
\item {Utilização:Prov.}
\end{itemize}
Tolice.
Impostura, vaidade. Cf. Júl. Dinis, \textunderscore Fidalgos\textunderscore , I, 47.
\section{Tolaz}
\begin{itemize}
\item {Grp. gram.:adj.}
\end{itemize}
Muito tolo, pacóvio.
Que se deixa enganar facilmente:«\textunderscore o tolaz marido.\textunderscore »Filinto, V, 131.
\section{Tolda}
\begin{itemize}
\item {Grp. gram.:f.}
\end{itemize}
\begin{itemize}
\item {Utilização:Prov.}
\end{itemize}
\begin{itemize}
\item {Utilização:alg.}
\end{itemize}
\begin{itemize}
\item {Proveniência:(Do ár. \textunderscore dholla\textunderscore )}
\end{itemize}
O mesmo que \textunderscore tôldo\textunderscore .
Primeira coberta de uma embarcação.
Armação de madeira, dentro da qual se empilham as maçarocas de milho, por fórma que recebam o ar e se conservem sans; espigueiro.
O mesmo que \textunderscore tremonha\textunderscore .
\section{Tolda}
\begin{itemize}
\item {Grp. gram.:f.}
\end{itemize}
Acto ou effeito de toldar.
Turvação do vinho.
\section{Toldar}
\begin{itemize}
\item {Grp. gram.:v. t.}
\end{itemize}
\begin{itemize}
\item {Utilização:Fig.}
\end{itemize}
\begin{itemize}
\item {Grp. gram.:V. p.}
\end{itemize}
\begin{itemize}
\item {Utilização:Fig.}
\end{itemize}
Cobrir com tôldo.
Encobrir: \textunderscore as nuvens toldam o Sol\textunderscore .
Anuvear.
Obscurecer: \textunderscore a embriaguez tolda a razão\textunderscore .
Entristecer; turbar.
Embriagar-se.
\section{Toldaria}
\begin{itemize}
\item {Grp. gram.:f.}
\end{itemize}
\begin{itemize}
\item {Proveniência:(De \textunderscore tôldo\textunderscore )}
\end{itemize}
Povoação de Índios americanos, formada de tendas ou barracas, mais ou menos persistentes, e cobertas de pelles de cavallos selvagens ou de ramos de palmeira. Cf. Mayne-Reid.
\section{Tólde}
\begin{itemize}
\item {Grp. gram.:m.}
\end{itemize}
\begin{itemize}
\item {Utilização:Prov.}
\end{itemize}
\begin{itemize}
\item {Utilização:trasm.}
\end{itemize}
Vestido espaventoso, o mesmo que \textunderscore telitône\textunderscore .
\section{Toldeira}
\begin{itemize}
\item {Grp. gram.:f.}
\end{itemize}
\begin{itemize}
\item {Utilização:Prov.}
\end{itemize}
\begin{itemize}
\item {Utilização:trasm.}
\end{itemize}
Mulhér, que gosta de tóldes.
\section{Tôldo}
\begin{itemize}
\item {Grp. gram.:m.}
\end{itemize}
\begin{itemize}
\item {Utilização:Bras}
\end{itemize}
\begin{itemize}
\item {Proveniência:(De \textunderscore tolda\textunderscore ^2)}
\end{itemize}
Coberta ou peça de lona ou de outra substância, e destinada principalmente a abrigar do sol e da chuva uma porta, uma praça, etc.
Aldeia ou povoação de indígenas.
\section{Tole}
\begin{itemize}
\item {Grp. gram.:m.}
\end{itemize}
\begin{itemize}
\item {Utilização:Ant.}
\end{itemize}
\begin{itemize}
\item {Proveniência:(Lat. \textunderscore tolle\textunderscore , imper.)}
\end{itemize}
(Us. só na loc. \textunderscore tomar o tole\textunderscore , safar-se, despedir-se; dar ás de villa-diogo)
\section{Toledana}
\begin{itemize}
\item {Grp. gram.:f.}
\end{itemize}
\begin{itemize}
\item {Proveniência:(De \textunderscore toledano\textunderscore )}
\end{itemize}
Espada, fabricada em Toledo. Cf. Herculano, \textunderscore Lendas e Narr.\textunderscore , II, 75.
\section{Toledano}
\begin{itemize}
\item {Grp. gram.:adj.}
\end{itemize}
\begin{itemize}
\item {Grp. gram.:M.}
\end{itemize}
Relativo a Toledo.
Habitante de Toledo.
\section{Toledo}
\begin{itemize}
\item {fónica:lê}
\end{itemize}
\begin{itemize}
\item {Grp. gram.:m.}
\end{itemize}
\begin{itemize}
\item {Utilização:Pop.}
\end{itemize}
\begin{itemize}
\item {Proveniência:(De \textunderscore tolo\textunderscore )}
\end{itemize}
Acto ou dito de pessôa desassisada; toleima.
\section{Toleima}
\begin{itemize}
\item {Grp. gram.:f.}
\end{itemize}
O mesmo que \textunderscore tolice\textunderscore .
\section{Toleirão}
\begin{itemize}
\item {Grp. gram.:m.  e  adj.}
\end{itemize}
O que é tolo ou muito tolo; pateta.
\section{Tolejar}
\begin{itemize}
\item {Grp. gram.:v. i.}
\end{itemize}
\begin{itemize}
\item {Utilização:Bras de Baía}
\end{itemize}
\begin{itemize}
\item {Proveniência:(De \textunderscore tolo\textunderscore )}
\end{itemize}
Dizer ou praticar tolices.
Andar á tôa; vadiar.
\section{Tolentiniano}
\begin{itemize}
\item {Grp. gram.:adj.}
\end{itemize}
Relativo ao poéta português Nicolau Tolentino.
Semelhante, na fórma ou no estilo, ás poesias de Tolentino. Cf. Camillo, \textunderscore Cancion. Al.\textunderscore , 154 e 255.
\section{Toler}
\begin{itemize}
\item {Grp. gram.:v. t.}
\end{itemize}
\begin{itemize}
\item {Utilização:Ant.}
\end{itemize}
\begin{itemize}
\item {Proveniência:(Do lat. \textunderscore tollere\textunderscore )}
\end{itemize}
Tirar; subtrair.
\section{Tolerada}
\begin{itemize}
\item {Grp. gram.:f.}
\end{itemize}
\begin{itemize}
\item {Proveniência:(De \textunderscore tolerado\textunderscore )}
\end{itemize}
Prostituta, que tem o nome inscrito nos registos administrativos e está sujeita á inspecção e regulamentação policial.
Mulhér pública.
\section{Toleradamente}
\begin{itemize}
\item {Grp. gram.:adv.}
\end{itemize}
\begin{itemize}
\item {Proveniência:(De \textunderscore tolerado\textunderscore )}
\end{itemize}
Com tolerância.
\section{Tolerado}
\begin{itemize}
\item {Grp. gram.:adj.}
\end{itemize}
\begin{itemize}
\item {Proveniência:(De \textunderscore tolerar\textunderscore )}
\end{itemize}
Que se tolera.
Consentido.
Apreciado com indulgência: \textunderscore faltas toleradas\textunderscore .
\section{Tolerância}
\begin{itemize}
\item {Grp. gram.:f.}
\end{itemize}
\begin{itemize}
\item {Proveniência:(Lat. \textunderscore tolerantia\textunderscore )}
\end{itemize}
Qualidade do que é tolerante.
Acto ou effeito de tolerar.
\section{Tolerante}
\begin{itemize}
\item {Grp. gram.:adj.}
\end{itemize}
\begin{itemize}
\item {Proveniência:(Lat. \textunderscore tolerans\textunderscore )}
\end{itemize}
Que tolera.
Dotado de tolerância.
Indulgente.
Que desculpa certas faltas ou êrros.
Que admitte ou respeita opiniões contrárias á sua.
\section{Tolerantismo}
\begin{itemize}
\item {Grp. gram.:m.}
\end{itemize}
\begin{itemize}
\item {Proveniência:(De \textunderscore tolerante\textunderscore )}
\end{itemize}
Opinião dos que defendem a tolerância religiosa.
Systema dos que entendem que se devem tolerar num Estado todas as espécies de religiões.
\section{Tolerar}
\begin{itemize}
\item {Grp. gram.:v. t.}
\end{itemize}
\begin{itemize}
\item {Proveniência:(Lat. \textunderscore tolerare\textunderscore )}
\end{itemize}
Sêr indulgente para com.
Consentir tacitamente.
Supportar.
\section{Tolerável}
\begin{itemize}
\item {Grp. gram.:adj.}
\end{itemize}
\begin{itemize}
\item {Proveniência:(Do lat. \textunderscore tolerabilis\textunderscore )}
\end{itemize}
Que se póde tolerar; soffrível.
Que não tem grandes defeitos.
Merecedor de indulgência.
\section{Toleravelmente}
\begin{itemize}
\item {Grp. gram.:adv.}
\end{itemize}
De modo tolerável.
\section{Toletano}
\begin{itemize}
\item {Grp. gram.:adj.}
\end{itemize}
\begin{itemize}
\item {Grp. gram.:M.}
\end{itemize}
\begin{itemize}
\item {Proveniência:(Lat. \textunderscore toletanus\textunderscore )}
\end{itemize}
Relativo a Toledo.
Habitante de Toledo; toledano.
\section{Tolete}
\begin{itemize}
\item {fónica:lê}
\end{itemize}
\begin{itemize}
\item {Grp. gram.:m.}
\end{itemize}
\begin{itemize}
\item {Proveniência:(Do fr. \textunderscore tolet\textunderscore )}
\end{itemize}
Cada uma das cavilhas, na borda do barco, ás quaes se encosta o remo, para jogar.
Pau aguçado, com que os Índios da América apanham os crocodilos.
\section{Toleteira}
\begin{itemize}
\item {Grp. gram.:f.}
\end{itemize}
Pequena elevação na borda do barco, em que se cravam os toletes.
\section{Tolhedoiro}
\begin{itemize}
\item {Grp. gram.:m.}
\end{itemize}
\begin{itemize}
\item {Utilização:Prov.}
\end{itemize}
\begin{itemize}
\item {Utilização:dur.}
\end{itemize}
\begin{itemize}
\item {Proveniência:(De \textunderscore tolher\textunderscore )}
\end{itemize}
Tábua, suspensa sôbre o rodízio, para o fazer parar, tolhendo a água da cale.
\section{Tolhedouro}
\begin{itemize}
\item {Grp. gram.:m.}
\end{itemize}
\begin{itemize}
\item {Utilização:Prov.}
\end{itemize}
\begin{itemize}
\item {Utilização:dur.}
\end{itemize}
\begin{itemize}
\item {Proveniência:(De \textunderscore tolher\textunderscore )}
\end{itemize}
Tábua, suspensa sôbre o rodízio, para o fazer parar, tolhendo a água da cale.
\section{Tolhedura}
\begin{itemize}
\item {Grp. gram.:f.}
\end{itemize}
Excremento de ave de rapina.
\section{Tolheita}
\begin{itemize}
\item {Grp. gram.:f.}
\end{itemize}
\begin{itemize}
\item {Utilização:Des.}
\end{itemize}
\begin{itemize}
\item {Proveniência:(De \textunderscore tolheito\textunderscore )}
\end{itemize}
Embaraço, tolhimento. Cf. Filinto, VI, 189.
\section{Tolheito}
\begin{itemize}
\item {Grp. gram.:adj.}
\end{itemize}
\begin{itemize}
\item {Utilização:Prov.}
\end{itemize}
\begin{itemize}
\item {Utilização:minh.}
\end{itemize}
\begin{itemize}
\item {Utilização:Ant.}
\end{itemize}
O mesmo que \textunderscore tolhido\textunderscore . Cf. Frei Fortun., \textunderscore Inéd.\textunderscore , 315.
\section{Tolher}
\begin{itemize}
\item {Grp. gram.:v. t.}
\end{itemize}
\begin{itemize}
\item {Grp. gram.:V. p.}
\end{itemize}
\begin{itemize}
\item {Utilização:Fig.}
\end{itemize}
\begin{itemize}
\item {Proveniência:(Do lat. \textunderscore tollere\textunderscore )}
\end{itemize}
Embaraçar, impedir.
Prohibir.
Oppor-se a.
Privar.
Têr paralysia.
Tornar-se immóvel.
\section{Tolhiço}
\begin{itemize}
\item {Grp. gram.:m.}
\end{itemize}
\begin{itemize}
\item {Proveniência:(De \textunderscore tolher\textunderscore )}
\end{itemize}
Coisa tolhida ou defeituosa; monstruosidade. Cf. Camillo, \textunderscore Mem. do Cárcere\textunderscore .
\section{Tolhido}
\begin{itemize}
\item {Grp. gram.:adj.}
\end{itemize}
\begin{itemize}
\item {Proveniência:(De \textunderscore tolher\textunderscore )}
\end{itemize}
Impedido.
Prohibido.
Entrèvado; paralýtico.
\section{Tolhimento}
\begin{itemize}
\item {Grp. gram.:m.}
\end{itemize}
Acto ou effeito de tolher ou de tolher-se.
\section{Tolho}
\begin{itemize}
\item {fónica:tô}
\end{itemize}
\begin{itemize}
\item {Grp. gram.:m.}
\end{itemize}
Peixe da costa do Algarve.
\section{Tólho}
\begin{itemize}
\item {Grp. gram.:m.}
\end{itemize}
\begin{itemize}
\item {Utilização:Prov.}
\end{itemize}
\begin{itemize}
\item {Utilização:beir.}
\end{itemize}
Rapazote vadio. (Colhido na Guarda)
\section{Tolice}
\begin{itemize}
\item {Grp. gram.:f.}
\end{itemize}
Qualidade do que é tolo.
Acto ou dito de tolo.
Parvoíce; asneira; desconchavo.
\section{Tolidade}
\begin{itemize}
\item {Grp. gram.:f.}
\end{itemize}
\begin{itemize}
\item {Utilização:Prov.}
\end{itemize}
\begin{itemize}
\item {Utilização:trasm.}
\end{itemize}
O mesmo que \textunderscore tolice\textunderscore .
\section{Tolina}
\begin{itemize}
\item {Grp. gram.:f.}
\end{itemize}
\begin{itemize}
\item {Utilização:Chul.}
\end{itemize}
\begin{itemize}
\item {Proveniência:(De \textunderscore tolo\textunderscore )}
\end{itemize}
O mesmo que \textunderscore toledo\textunderscore .
Lôgro, que se faz a um tolo.
\section{Tolinar}
\begin{itemize}
\item {Grp. gram.:v. t.}
\end{itemize}
\begin{itemize}
\item {Utilização:Chul.}
\end{itemize}
Fazer tolina a.
\section{Tolineiro}
\begin{itemize}
\item {Grp. gram.:m.}
\end{itemize}
\begin{itemize}
\item {Utilização:Chul.}
\end{itemize}
Aquelle que tolina.
\section{Tolitates}
\begin{itemize}
\item {Grp. gram.:m.}
\end{itemize}
\begin{itemize}
\item {Utilização:T. de Turquel}
\end{itemize}
Homem vaidoso e ridículo.
(Cp. \textunderscore tolidade\textunderscore )
\section{Tolle}
\begin{itemize}
\item {Grp. gram.:m.}
\end{itemize}
\begin{itemize}
\item {Utilização:Ant.}
\end{itemize}
\begin{itemize}
\item {Proveniência:(Lat. \textunderscore tolle\textunderscore , imper.)}
\end{itemize}
(Us. só na loc. \textunderscore tomar o tolle\textunderscore , safar-se, despedir-se; dar ás de villa-diogo)
\section{Toller}
\begin{itemize}
\item {Grp. gram.:v. t.}
\end{itemize}
\begin{itemize}
\item {Utilização:Ant.}
\end{itemize}
\begin{itemize}
\item {Proveniência:(Do lat. \textunderscore tollere\textunderscore )}
\end{itemize}
Tirar; subtrahir.
\section{Tolo}
\begin{itemize}
\item {fónica:tô}
\end{itemize}
\begin{itemize}
\item {Grp. gram.:adj.}
\end{itemize}
\begin{itemize}
\item {Grp. gram.:M.}
\end{itemize}
Louco; doido.
Que não tem juízo ou intelligência.
Que não faz sentido ou que nada significa; disparatado: \textunderscore argumento tolo\textunderscore .
Ridículo.
Vaidoso.
Cheio de espanto; boquiaberto.
Aquelle que não tem juízo.
Idiota; pàteta; pacóvio.
(Relaciona-se com o lat. \textunderscore stolidus\textunderscore ?)
\section{Tolontro}
\begin{itemize}
\item {Grp. gram.:m.}
\end{itemize}
\begin{itemize}
\item {Proveniência:(Do cast. \textunderscore tolondro\textunderscore )}
\end{itemize}
Tumor, produzido por contusão.
Caroço.
Túbera.
\section{Tólpide}
\begin{itemize}
\item {Grp. gram.:f.}
\end{itemize}
Gênero de plantas, da fam. das compostas.
\section{Tolstoísmo}
\begin{itemize}
\item {Grp. gram.:m.}
\end{itemize}
Conjunto das theorias ou o systema de Tolstoi.
\section{Tolstoísta}
\begin{itemize}
\item {Grp. gram.:m.}
\end{itemize}
Sectário do tolstoísmo.
\section{Toltecas}
\begin{itemize}
\item {Grp. gram.:m. pl.}
\end{itemize}
Antiga nação mexicana.
\section{Tolu}
\begin{itemize}
\item {Grp. gram.:m.}
\end{itemize}
\begin{itemize}
\item {Proveniência:(De \textunderscore Tolu\textunderscore , n. p.)}
\end{itemize}
Bálsamo, produzido por uma árvore da Colômbia.
\section{Toluato}
\begin{itemize}
\item {Grp. gram.:m.}
\end{itemize}
\begin{itemize}
\item {Proveniência:(De \textunderscore tolu\textunderscore )}
\end{itemize}
Sal do ácido toluico.
\section{Toluena}
\begin{itemize}
\item {Grp. gram.:f.}
\end{itemize}
\begin{itemize}
\item {Proveniência:(De \textunderscore tolu\textunderscore )}
\end{itemize}
Combinação de carbóne e hydrogênio.
\section{Tolueno}
\begin{itemize}
\item {Grp. gram.:m.}
\end{itemize}
\begin{itemize}
\item {Utilização:Chím.}
\end{itemize}
Um dos carbonetos do grupo benzênico.
\section{Toluico}
\begin{itemize}
\item {Grp. gram.:adj.}
\end{itemize}
Relativo ao bálsamo tolu.
\section{Toluífero}
\begin{itemize}
\item {Grp. gram.:adj.}
\end{itemize}
Que produz o bálsamo tolu.
\section{Toluína}
\begin{itemize}
\item {Grp. gram.:f.}
\end{itemize}
Princípio estimulante, contido no bálsamo tolu.
\section{Tom}
\begin{itemize}
\item {Grp. gram.:m.}
\end{itemize}
\begin{itemize}
\item {Grp. gram.:Loc. adv.}
\end{itemize}
\begin{itemize}
\item {Proveniência:(Do lat. \textunderscore tonus\textunderscore )}
\end{itemize}
Tensão, estado de elasticidade de cada tecido orgânico.
Effeito de tonificar.
Certo grau de abaixamento ou elevação da voz.
Elevação da voz sôbre uma sýllaba de uma palavra.
Carácter da voz, relativamente á natureza do discurso.
Em música, o som, relativamente ao seu grau de gravidade ou de acuidade.
Intervallo, entre duas notas da escala musical, excepto o de mi para fá, e o de si para dó.
Côr, que predomina num quadro.
Colorido.
Carácter.
Modo geral.
Moda.
Semelhança.
Teor.
Maneira.
Gamma, que se adopta para a composição ou execução de uma peça musical.
\textunderscore Sem tom nem som\textunderscore , disparatadamente.
\section{Tóma}
\begin{itemize}
\item {Grp. gram.:f.}
\end{itemize}
Acto de tomar; tomada. Cf. Filinto, VI, 32.
\section{Tóma!}
\begin{itemize}
\item {Grp. gram.:interj.}
\end{itemize}
\begin{itemize}
\item {Utilização:fam.}
\end{itemize}
\begin{itemize}
\item {Proveniência:(De \textunderscore tomar\textunderscore )}
\end{itemize}
(designativa de congratulação: \textunderscore então saiu-te a sorte grande? toma\textunderscore !)
\section{Tomada}
\begin{itemize}
\item {Grp. gram.:f.}
\end{itemize}
Acto ou effeito de tomar.
Conquista: \textunderscore a tomada de Santarém\textunderscore .
\section{Tomadeira}
\begin{itemize}
\item {Grp. gram.:f.}
\end{itemize}
\begin{itemize}
\item {Utilização:Prov.}
\end{itemize}
\begin{itemize}
\item {Utilização:beir.}
\end{itemize}
\begin{itemize}
\item {Proveniência:(De \textunderscore tomar\textunderscore )}
\end{itemize}
Forquilha, feita de um galho de árvore, que já de si tem a fórma de forcado.
\section{Tomadete}
\begin{itemize}
\item {fónica:dê}
\end{itemize}
\begin{itemize}
\item {Grp. gram.:adj.}
\end{itemize}
\begin{itemize}
\item {Utilização:Chul.}
\end{itemize}
\begin{itemize}
\item {Proveniência:(De \textunderscore tomado\textunderscore )}
\end{itemize}
Tocado (de vinho)
Um tanto ébrio.
\section{Tomadia}
\begin{itemize}
\item {Grp. gram.:f.}
\end{itemize}
\begin{itemize}
\item {Proveniência:(De \textunderscore tomar\textunderscore )}
\end{itemize}
O mesmo que \textunderscore tomada\textunderscore .
Coisa apprehendida; apprehensão.
\section{Tomadiço}
\begin{itemize}
\item {Grp. gram.:adj.}
\end{itemize}
\begin{itemize}
\item {Proveniência:(De \textunderscore tomar\textunderscore )}
\end{itemize}
Que se enfada facilmente.
\section{Tomado}
\begin{itemize}
\item {Grp. gram.:adj.}
\end{itemize}
\begin{itemize}
\item {Grp. gram.:M. pl.}
\end{itemize}
Agarrado.
Dominado: \textunderscore tomado de espanto\textunderscore .
Avaliado, julgado: \textunderscore palavras, tomadas em mau sentido\textunderscore .
Refegos em vestidos de mulhéres.
Pontos, com que se consertam as roupas.
\section{Tomadoiro}
\begin{itemize}
\item {Grp. gram.:m.}
\end{itemize}
\begin{itemize}
\item {Proveniência:(De \textunderscore tomar\textunderscore )}
\end{itemize}
Pedaço da cinta das vêrgas, a que se ferra a vela do navio.
Bomba de tomar água.
Tubo parallelepipedal, que toma água para os viveiros das salinas, em Aveiro.
\section{Tomador}
\begin{itemize}
\item {Grp. gram.:m.  e  adj.}
\end{itemize}
O que toma.
\section{Tomadote}
\begin{itemize}
\item {Grp. gram.:adj.}
\end{itemize}
O mesmo que \textunderscore tomadete\textunderscore .
\section{Tomadouro}
\begin{itemize}
\item {Grp. gram.:m.}
\end{itemize}
\begin{itemize}
\item {Proveniência:(De \textunderscore tomar\textunderscore )}
\end{itemize}
Pedaço da cinta das vêrgas, a que se ferra a vela do navio.
Bomba de tomar água.
Tubo parallelepipedal, que toma água para os viveiros das salinas, em Aveiro.
\section{Tomadura}
\begin{itemize}
\item {Grp. gram.:f.}
\end{itemize}
\begin{itemize}
\item {Proveniência:(De \textunderscore tomar\textunderscore )}
\end{itemize}
Ferimento, produzido na cavalgadura pelo roçar da sella ou da albarda.
\section{Toma-larguras}
\begin{itemize}
\item {Grp. gram.:m.}
\end{itemize}
\begin{itemize}
\item {Utilização:Bras}
\end{itemize}
O mesmo que \textunderscore talaveira\textunderscore .
\section{Tomamento}
\begin{itemize}
\item {Grp. gram.:m.}
\end{itemize}
O mesmo que \textunderscore tomada\textunderscore .
\section{Tomão}
\begin{itemize}
\item {Grp. gram.:m.}
\end{itemize}
\begin{itemize}
\item {Utilização:Prov.}
\end{itemize}
\begin{itemize}
\item {Utilização:minh.}
\end{itemize}
O mesmo que \textunderscore temão\textunderscore .
\section{Tomar}
\begin{itemize}
\item {Grp. gram.:v. t.}
\end{itemize}
\begin{itemize}
\item {Utilização:Ant.}
\end{itemize}
\begin{itemize}
\item {Grp. gram.:V. p.}
\end{itemize}
\begin{itemize}
\item {Utilização:Pleb.}
\end{itemize}
\begin{itemize}
\item {Proveniência:(Do sax. \textunderscore tômian\textunderscore )}
\end{itemize}
Agarrar; pegar em; segurar.
Suspender.
Conquistar.
Roubar.
Capturar.
Servir-se de: \textunderscore tomei-o para meu criado\textunderscore .
Recolher.
Absorver; beber; engulir: \textunderscore tomar uma limonada\textunderscore .
Occupar, preencher: \textunderscore tomar lugar\textunderscore .
Alcançar.
Abranger: \textunderscore tomar muito espaço\textunderscore .
Impedir.
Acceitar.
Dirigir-se, seguir por: \textunderscore tomar a estrada real\textunderscore .
Adoptar.
Assumir: \textunderscore tomar encargos\textunderscore .
Mostrar que tem.
Puxar para si.
Escolher.
Desejar: \textunderscore tomara êlle que assim fôsse\textunderscore !
Interpretar.
Avaliar: \textunderscore tomar em mau sentido\textunderscore .
\textunderscore Tomar o Sol\textunderscore , tomar a altura do Sol, com o astrolábio ou com instrumento idêntico. Cf. \textunderscore Rot. do Mar Verm.\textunderscore , 11.
Deixar-se possuir, deixar-se dominar; sêr invadido: \textunderscore tomar-se de susto\textunderscore .
Impregnar-se.
Embebedar-se um pouco; embriagar-se.
\section{Tomarense}
\begin{itemize}
\item {Grp. gram.:adj.}
\end{itemize}
\begin{itemize}
\item {Grp. gram.:M.}
\end{itemize}
Relativo á cidade de Tomar.
Habitante de Tomar.
\section{Tomares}
\begin{itemize}
\item {Grp. gram.:m. pl.}
\end{itemize}
(Cp. \textunderscore dares\textunderscore )
\section{Tomarista}
\begin{itemize}
\item {Grp. gram.:m.}
\end{itemize}
Habitante de Tomar.
Freire da Ordem de Christo.
\section{Tomasínia}
\begin{itemize}
\item {Grp. gram.:f.}
\end{itemize}
\begin{itemize}
\item {Proveniência:(De \textunderscore Tommasini\textunderscore , n. p.)}
\end{itemize}
Planta umbelífera, cuja espécie tipo cresce no Piemonte.
\section{Tomata}
\begin{itemize}
\item {Grp. gram.:f.}
\end{itemize}
\begin{itemize}
\item {Utilização:Prov.}
\end{itemize}
\begin{itemize}
\item {Utilização:beir.}
\end{itemize}
O mesmo que \textunderscore tomate\textunderscore . (Colhido em Sátão, Pinhel, etc.)
\section{Tomatada}
\begin{itemize}
\item {Grp. gram.:f.}
\end{itemize}
Massa de tomate, para tempêro.
\section{Tomate}
\begin{itemize}
\item {Grp. gram.:m.}
\end{itemize}
\begin{itemize}
\item {Utilização:Chul.}
\end{itemize}
Fruto de tomateiro; tomateiro.
O mesmo que \textunderscore testículo\textunderscore .
(Cast. \textunderscore tomate\textunderscore )
\section{Tomateira}
\begin{itemize}
\item {Grp. gram.:f.}
\end{itemize}
\begin{itemize}
\item {Utilização:Prov.}
\end{itemize}
\begin{itemize}
\item {Utilização:beir.}
\end{itemize}
O mesmo que \textunderscore tomateiro\textunderscore .
(Cast. \textunderscore tomatera\textunderscore )
\section{Tomateiro}
\begin{itemize}
\item {Grp. gram.:m.}
\end{itemize}
\begin{itemize}
\item {Proveniência:(De \textunderscore tomate\textunderscore )}
\end{itemize}
Gênero de plantas solâneas, procedente da América tropical e cujo fruto vermelho tem applicações culinárias.
\section{Tomba}
\begin{itemize}
\item {Grp. gram.:f.}
\end{itemize}
Remendo no calçado.
\section{Tomba}
\begin{itemize}
\item {Grp. gram.:f.}
\end{itemize}
Planta, o mesmo que \textunderscore espelina\textunderscore .
\section{Tombadilho}
\begin{itemize}
\item {Grp. gram.:m.}
\end{itemize}
\begin{itemize}
\item {Proveniência:(De \textunderscore tombar\textunderscore ^1)}
\end{itemize}
A parte mais alta de um navio, entre a popa e o mastro de mezena.
\section{Tombador}
\begin{itemize}
\item {Grp. gram.:m.}
\end{itemize}
\begin{itemize}
\item {Utilização:Bras. do N}
\end{itemize}
\begin{itemize}
\item {Proveniência:(De \textunderscore tombar\textunderscore ^1)}
\end{itemize}
Encosta íngreme, terreno escarpado, cheio de barrancos.
\section{Tombador}
\begin{itemize}
\item {Grp. gram.:m.  e  adj.}
\end{itemize}
\begin{itemize}
\item {Proveniência:(De \textunderscore tombar\textunderscore ^2)}
\end{itemize}
O que tomba ou faz tombar.
\section{Tomba-ladeiras}
\begin{itemize}
\item {Grp. gram.:m.}
\end{itemize}
\begin{itemize}
\item {Utilização:Prov.}
\end{itemize}
\begin{itemize}
\item {Utilização:trasm.}
\end{itemize}
O mesmo que \textunderscore trangalhadanças\textunderscore .
\section{Tomba-las-águas}
\begin{itemize}
\item {Grp. gram.:m.}
\end{itemize}
\begin{itemize}
\item {Utilização:Bras}
\end{itemize}
O mesmo que \textunderscore tromba-las-águas\textunderscore .
\section{Tomba-lobos}
\begin{itemize}
\item {Grp. gram.:m.}
\end{itemize}
\begin{itemize}
\item {Utilização:Fam.}
\end{itemize}
Homem robusto e desajeitado.
Brutamontes; trangalhadanças.
\section{Tombamento}
\begin{itemize}
\item {Grp. gram.:m.}
\end{itemize}
Acto ou effeito de tombar.
\section{Tombante}
\begin{itemize}
\item {Grp. gram.:adj.}
\end{itemize}
\begin{itemize}
\item {Proveniência:(De \textunderscore tombar\textunderscore ^1)}
\end{itemize}
Que tomba. Cf. Arn. Gama, \textunderscore Última Dona\textunderscore , 291.
\section{Tombar}
\begin{itemize}
\item {Grp. gram.:v. t.}
\end{itemize}
\begin{itemize}
\item {Grp. gram.:V. i.}
\end{itemize}
\begin{itemize}
\item {Grp. gram.:V. p.}
\end{itemize}
\begin{itemize}
\item {Proveniência:(Do ant. alt. al. \textunderscore tumon\textunderscore )}
\end{itemize}
Deitar no chão; fazer caír; derrubar.
Caír no chão.
Caír.
Declinar; deslizar.
Caír para o lado; virar-se.
\section{Tombar}
\begin{itemize}
\item {Grp. gram.:v. t.}
\end{itemize}
\begin{itemize}
\item {Proveniência:(De \textunderscore tombo\textunderscore ^2)}
\end{itemize}
Fazer o tombo de, inventariar; arrolar.
\section{Tombar}
\begin{itemize}
\item {Grp. gram.:v. i.}
\end{itemize}
\begin{itemize}
\item {Utilização:Ant.}
\end{itemize}
Resoar, retumbar.
(Cp. \textunderscore retumbar\textunderscore )
\section{Tombear}
\begin{itemize}
\item {Grp. gram.:v. i.}
\end{itemize}
\begin{itemize}
\item {Utilização:Prov.}
\end{itemize}
\begin{itemize}
\item {Utilização:trasm.}
\end{itemize}
\begin{itemize}
\item {Proveniência:(De \textunderscore tombo\textunderscore ^1)}
\end{itemize}
Dar tombos; caír.
\section{Tombeirinho}
\begin{itemize}
\item {Grp. gram.:m.}
\end{itemize}
\begin{itemize}
\item {Utilização:Prov.}
\end{itemize}
\begin{itemize}
\item {Utilização:trasm.}
\end{itemize}
O mesmo que \textunderscore mamôa\textunderscore .
\section{Tombeiro}
\begin{itemize}
\item {Grp. gram.:adj.}
\end{itemize}
\begin{itemize}
\item {Utilização:Bras. do S}
\end{itemize}
Manso, (falando-se de gado).
\section{Tombo}
\begin{itemize}
\item {Grp. gram.:m.}
\end{itemize}
\begin{itemize}
\item {Utilização:Prov.}
\end{itemize}
Acto ou effeito de tombar^1.
Espécie de armadilha para caçar.
\section{Tombo}
\begin{itemize}
\item {Grp. gram.:m.}
\end{itemize}
Inventário de terrenos demarcados.
Archivo.
Registo ou relação de coisas ou factos, relativos a uma especialidade ou a uma região: \textunderscore no Tombo do Estado da Índia há noticias muito interessantes...\textunderscore 
(Relaciona-se talvez com o lat. \textunderscore tumulus\textunderscore )
\section{Tômbola}
\begin{itemize}
\item {Grp. gram.:f.}
\end{itemize}
\begin{itemize}
\item {Proveniência:(It. \textunderscore tombola\textunderscore )}
\end{itemize}
Espécie de lôto, em que é preciso encher-se um cartão, para ganhar.
Jôgo de asar, que se pratíca sôbre um tabuleiro, em que há várias cavidades de côres diversas.
\section{Tombolar}
\begin{itemize}
\item {Grp. gram.:v. i.}
\end{itemize}
Ganhar no jôgo da tômbola.
\section{Tômboro}
\begin{itemize}
\item {Grp. gram.:m.}
\end{itemize}
\begin{itemize}
\item {Utilização:Prov.}
\end{itemize}
\begin{itemize}
\item {Utilização:trasm.}
\end{itemize}
\begin{itemize}
\item {Utilização:Ant.}
\end{itemize}
O mesmo que \textunderscore cômoro\textunderscore .
(Cp. \textunderscore tômoro\textunderscore )
\section{Tomelo}
\begin{itemize}
\item {Grp. gram.:m.}
\end{itemize}
\begin{itemize}
\item {Utilização:Prov.}
\end{itemize}
O mesmo que \textunderscore rosmaninho\textunderscore . (Colhido em Barca-de-Alva)
(Cp. \textunderscore tomilho\textunderscore )
\section{Tomentelo}
\begin{itemize}
\item {fónica:tê}
\end{itemize}
\begin{itemize}
\item {Grp. gram.:m.}
\end{itemize}
(Dem. de \textunderscore tomento\textunderscore )
\section{Tomentina}
\begin{itemize}
\item {Grp. gram.:f.}
\end{itemize}
O mesmo que \textunderscore tormentina\textunderscore ; sete-em-rama.
\section{Tomento}
\begin{itemize}
\item {Grp. gram.:m.}
\end{itemize}
\begin{itemize}
\item {Proveniência:(Lat. \textunderscore tomentum\textunderscore )}
\end{itemize}
A fibra mais áspera do linho.
Estopa grossa.
Lanugem, que reveste certos órgãos vegetaes.
\section{Tomentoso}
\begin{itemize}
\item {Grp. gram.:adj.}
\end{itemize}
\begin{itemize}
\item {Utilização:Bot.}
\end{itemize}
\begin{itemize}
\item {Proveniência:(De \textunderscore tomento\textunderscore )}
\end{itemize}
Coberto de lanugem.
\section{Tomilhal}
\begin{itemize}
\item {Grp. gram.:m.}
\end{itemize}
Lugar, onde crescem tomilhos.
\section{Tomilhinha}
\begin{itemize}
\item {Grp. gram.:f.}
\end{itemize}
\begin{itemize}
\item {Utilização:Prov.}
\end{itemize}
\begin{itemize}
\item {Utilização:trasm.}
\end{itemize}
\begin{itemize}
\item {Proveniência:(De \textunderscore tomilho\textunderscore )}
\end{itemize}
Erva, de cheiro activo e agradável, a qual se emprega na cura de azeitonas.
\section{Tomilho}
\begin{itemize}
\item {Grp. gram.:m.}
\end{itemize}
Planta labiada, (\textunderscore thymus vulgaris\textunderscore ).
\section{Tomíparo}
\begin{itemize}
\item {Grp. gram.:adj.}
\end{itemize}
\begin{itemize}
\item {Utilização:Hist. Nat.}
\end{itemize}
\begin{itemize}
\item {Proveniência:(Do gr. \textunderscore tome\textunderscore  + lat. \textunderscore p[-a]rere\textunderscore )}
\end{itemize}
Que se multiplica por incisão ou córte, (falando-se de algumas plantas e de certos animaes).
\section{Tomismo}
\begin{itemize}
\item {Grp. gram.:m.}
\end{itemize}
\begin{itemize}
\item {Proveniência:(De \textunderscore Thomás\textunderscore , n. p.)}
\end{itemize}
Doutrina teológica e filosófica de San-Thomás de Aquino.
\section{Tomista}
\begin{itemize}
\item {Grp. gram.:adj.}
\end{itemize}
\begin{itemize}
\item {Grp. gram.:M.}
\end{itemize}
Relativo ao tomismo.
Sectário do tomismo.
\section{Tomístico}
\begin{itemize}
\item {Grp. gram.:adj.}
\end{itemize}
\begin{itemize}
\item {Proveniência:(De \textunderscore tomista\textunderscore )}
\end{itemize}
Relativo a San-Thomás ou á sua doutrina.
\section{Tommasínia}
\begin{itemize}
\item {Grp. gram.:f.}
\end{itemize}
\begin{itemize}
\item {Proveniência:(De \textunderscore Tommasini\textunderscore , n. p.)}
\end{itemize}
Planta umbellífera, cuja espécie typo cresce no Piemonte.
\section{Tomo}
\begin{itemize}
\item {Grp. gram.:m.}
\end{itemize}
\begin{itemize}
\item {Utilização:Fig.}
\end{itemize}
\begin{itemize}
\item {Proveniência:(Lat. \textunderscore tomus\textunderscore )}
\end{itemize}
Volume de obra impressa ou manuscrita.
Cada uma das partes de uma obra scientífica ou literária, que se brocharam ou encadernaram separadamente.
Divisão, parte.
Valia; importância; alcance: \textunderscore incidentes de pequeno tomo\textunderscore .
\section{Tômolo}
\begin{itemize}
\item {Grp. gram.:m.}
\end{itemize}
Medida de capacidade, na Itália do Sul.
\section{Tomóptero}
\begin{itemize}
\item {Grp. gram.:m.}
\end{itemize}
\begin{itemize}
\item {Proveniência:(Do gr. \textunderscore tome\textunderscore  + \textunderscore pteron\textunderscore )}
\end{itemize}
Gênero de insectos coleópteros tetrâmeros.
\section{Tômoro}
\begin{itemize}
\item {Grp. gram.:m.}
\end{itemize}
\begin{itemize}
\item {Utilização:Prov.}
\end{itemize}
(V.cômoro)
\section{Tomotocia}
\begin{itemize}
\item {Grp. gram.:f.}
\end{itemize}
\begin{itemize}
\item {Utilização:Med.}
\end{itemize}
\begin{itemize}
\item {Proveniência:(Do gr. \textunderscore tome\textunderscore  + \textunderscore tokos\textunderscore )}
\end{itemize}
A operação cesariana.
\section{Tôna}
\begin{itemize}
\item {Grp. gram.:f.}
\end{itemize}
\begin{itemize}
\item {Utilização:Fig.}
\end{itemize}
Casca tênue; pellícula.
Alburno.
Superfície: \textunderscore á tona da água\textunderscore .
(Do câmbrico \textunderscore ton\textunderscore , casca)
\section{Tona}
\begin{itemize}
\item {Grp. gram.:f.}
\end{itemize}
\begin{itemize}
\item {Utilização:Bras. do N}
\end{itemize}
Bella e grande ave cinzenta.
\section{Tóna}
\begin{itemize}
\item {Grp. gram.:f.}
\end{itemize}
Barco de transporte, nos rios de Gôa.
\section{Tonadilha}
\begin{itemize}
\item {Grp. gram.:f.}
\end{itemize}
Toada, canção ligeira ou rústica; o mesmo que \textunderscore toadilha\textunderscore .
(Cast. \textunderscore tonadilla\textunderscore )
\section{Tonal}
\begin{itemize}
\item {Grp. gram.:adj.}
\end{itemize}
\begin{itemize}
\item {Proveniência:(Do lat. \textunderscore tonus\textunderscore )}
\end{itemize}
Relativo ao tom ou á tonalidade.
\section{Tonalidade}
\begin{itemize}
\item {Grp. gram.:f.}
\end{itemize}
\begin{itemize}
\item {Proveniência:(De \textunderscore tonal\textunderscore )}
\end{itemize}
Propriedade que caracteriza um tom.
Qualidade de um escrito ou de uma peça musical, em que predomina um determinado tom.
\section{Tonalmente}
\begin{itemize}
\item {Grp. gram.:adv.}
\end{itemize}
De modo tonal; conforme ao tom.
\section{Tonante}
\begin{itemize}
\item {Grp. gram.:adj.}
\end{itemize}
\begin{itemize}
\item {Proveniência:(Lat. \textunderscore tonans\textunderscore )}
\end{itemize}
Que troveja; que atrôa.
Forte.
\section{Tonar}
\begin{itemize}
\item {Grp. gram.:v. i.}
\end{itemize}
\begin{itemize}
\item {Utilização:Ant.}
\end{itemize}
\begin{itemize}
\item {Proveniência:(Lat. \textunderscore tonare\textunderscore )}
\end{itemize}
Trovejar.
\section{Tonário}
\begin{itemize}
\item {Grp. gram.:m.}
\end{itemize}
\begin{itemize}
\item {Proveniência:(Lat. \textunderscore tonarium\textunderscore )}
\end{itemize}
Espécie de frauta, com que se dava o tom aos oradores.
\section{Tonca}
\begin{itemize}
\item {Grp. gram.:f.}
\end{itemize}
Árvore leguminosa da América, que produz uma espécie de fava, com que se aromatiza o tabaco.
\section{Toncai}
\begin{itemize}
\item {Grp. gram.:m.  e  adj.}
\end{itemize}
Diz-se da espécie mais ordinária de chá verde.
\section{Tondinho}
\begin{itemize}
\item {Grp. gram.:m.}
\end{itemize}
\begin{itemize}
\item {Utilização:Anat.}
\end{itemize}
\begin{itemize}
\item {Proveniência:(Do it. \textunderscore tondino\textunderscore )}
\end{itemize}
Moldura pequena e redonda, na base das columnas.
O mesmo que \textunderscore tarso\textunderscore .
\section{Tone}
\begin{itemize}
\item {Grp. gram.:m.}
\end{itemize}
O mesmo que \textunderscore almadia\textunderscore .
Talvez o mesmo que \textunderscore tóna\textunderscore .
\section{Tonel}
\begin{itemize}
\item {Grp. gram.:m.}
\end{itemize}
\begin{itemize}
\item {Utilização:Fig.}
\end{itemize}
\begin{itemize}
\item {Utilização:Ant.}
\end{itemize}
\begin{itemize}
\item {Proveniência:(Do germ. \textunderscore tonne\textunderscore )}
\end{itemize}
Grande vasilha para líquidos, formada de aduelas, arcos e tampos, e de capacidade igual ou superior á de duas pipas.
Beberrão.
Tonelada.
\section{Tonelada}
\begin{itemize}
\item {Grp. gram.:f.}
\end{itemize}
\begin{itemize}
\item {Utilização:Ant.}
\end{itemize}
O que um tonel póde conter.
Pêso de treze quintaes e meio.
Medida, com que se calcula o carregamento dos navios ou o que os navios pódem transportar.
Cincoenta almudes ou duas pipas de vinho.
\textunderscore Tonelada métrica\textunderscore , mil quilogrammas.
\section{Tonelagem}
\begin{itemize}
\item {Grp. gram.:f.}
\end{itemize}
\begin{itemize}
\item {Proveniência:(De \textunderscore tonel\textunderscore )}
\end{itemize}
Capacidade de um navio; medida dessa capacidade.
\section{Tonelaria}
\begin{itemize}
\item {Grp. gram.:f.}
\end{itemize}
\begin{itemize}
\item {Proveniência:(De \textunderscore tonel\textunderscore )}
\end{itemize}
O mesmo que \textunderscore tanoaria\textunderscore .
\section{Toneletes}
\begin{itemize}
\item {fónica:lê}
\end{itemize}
\begin{itemize}
\item {Grp. gram.:m. pl.}
\end{itemize}
\begin{itemize}
\item {Proveniência:(Fr. \textunderscore tonnelet\textunderscore )}
\end{itemize}
Parte da antiga armadura, que descia da cintura ao joêlho.
\section{Tongas}
\begin{itemize}
\item {Grp. gram.:m. pl.}
\end{itemize}
Indígenas, que povoaram a região de Gaza, na África Oriental, antes da invasão dos Vátuas.
\section{Tónha}
\begin{itemize}
\item {Grp. gram.:f.}
\end{itemize}
\begin{itemize}
\item {Utilização:T. do Fundão}
\end{itemize}
Mulhér pública; rameira.
\section{Tónho}
\begin{itemize}
\item {Grp. gram.:adj.}
\end{itemize}
\begin{itemize}
\item {Utilização:Prov.}
\end{itemize}
Apalermado, idiota.
O mesmo que \textunderscore tólho\textunderscore .
\section{Tonia}
\begin{itemize}
\item {Grp. gram.:f.}
\end{itemize}
\begin{itemize}
\item {Proveniência:(Do lat. \textunderscore tonus\textunderscore )}
\end{itemize}
O mesmo que \textunderscore tonicidade\textunderscore .
\section{Tónica}
\begin{itemize}
\item {Grp. gram.:f.}
\end{itemize}
\begin{itemize}
\item {Utilização:Mús.}
\end{itemize}
\begin{itemize}
\item {Proveniência:(De \textunderscore tónico\textunderscore )}
\end{itemize}
Nota tónica.
\section{Tonicidade}
\begin{itemize}
\item {Grp. gram.:f.}
\end{itemize}
Qualidade ou estado do que é tónico.
Estado, em que os tecidos orgânicos revelam vigor ou energia.
\section{Tónico}
\begin{itemize}
\item {Grp. gram.:adj.}
\end{itemize}
\begin{itemize}
\item {Utilização:Mús.}
\end{itemize}
\begin{itemize}
\item {Grp. gram.:M.}
\end{itemize}
\begin{itemize}
\item {Proveniência:(Do lat. \textunderscore tonus\textunderscore )}
\end{itemize}
Relativo a tom.
Que tonifica ou dá energia a certos tecidos.
Diz-se da primeira nota de uma escala ou gamma.
Diz-se da elevação ou pausa da voz, numa sýllaba de uma palavra: \textunderscore accento tónico\textunderscore .
Diz-se da vogal ou da sýllaba, em que recai a accentuação tónica de uma palavra.
Remédio que tonifica.
\section{Tofo}
\begin{itemize}
\item {Grp. gram.:m.}
\end{itemize}
\begin{itemize}
\item {Proveniência:(Lat. \textunderscore tophus\textunderscore )}
\end{itemize}
Substância dura, que se fórma no interior dos órgãos ou junto das articulações dos seres animados.
\section{Tonificação}
\begin{itemize}
\item {Grp. gram.:f.}
\end{itemize}
Acto ou effeito de tonificar.
\section{Tonificante}
\begin{itemize}
\item {Grp. gram.:adj.}
\end{itemize}
Que tonifica.
\section{Tonificar}
\begin{itemize}
\item {Grp. gram.:v. t.}
\end{itemize}
\begin{itemize}
\item {Proveniência:(Do lat. \textunderscore tonus\textunderscore  + \textunderscore facere\textunderscore )}
\end{itemize}
Dar tom a, dar vigor a, fortalecer.
\section{Tonilho}
\begin{itemize}
\item {Grp. gram.:m.}
\end{itemize}
\begin{itemize}
\item {Utilização:Prov.}
\end{itemize}
\begin{itemize}
\item {Proveniência:(Do lat. \textunderscore tonus\textunderscore )}
\end{itemize}
Tom débil; tonadilha.
Palavreado embusteiro.
\section{Tonina}
\begin{itemize}
\item {Grp. gram.:f.}
\end{itemize}
O mesmo que \textunderscore toninha\textunderscore .
\section{Toninha}
\begin{itemize}
\item {Grp. gram.:f.}
\end{itemize}
\begin{itemize}
\item {Proveniência:(Do lat. \textunderscore thunnus\textunderscore )}
\end{itemize}
Atum de pouca idade.
Espécie de cetáceo, porco marinho.
\section{Toninho}
\begin{itemize}
\item {Grp. gram.:m.}
\end{itemize}
O mesmo que \textunderscore toninha\textunderscore .
\section{Tonioneia}
\begin{itemize}
\item {Grp. gram.:f.}
\end{itemize}
Ave do Brasil.
\section{Tonismo}
\begin{itemize}
\item {Grp. gram.:m.}
\end{itemize}
\begin{itemize}
\item {Proveniência:(Do gr. \textunderscore tonos\textunderscore )}
\end{itemize}
O mesmo que \textunderscore tétano\textunderscore .
\section{Tonitruante}
\begin{itemize}
\item {Grp. gram.:adj.}
\end{itemize}
\begin{itemize}
\item {Proveniência:(Lat. \textunderscore tonitruans\textunderscore )}
\end{itemize}
Que troveja.
Atroador.
\section{Tonítruo}
\begin{itemize}
\item {Grp. gram.:adj.}
\end{itemize}
\begin{itemize}
\item {Utilização:Poét.}
\end{itemize}
\begin{itemize}
\item {Proveniência:(Do lat. \textunderscore tonitruus\textunderscore )}
\end{itemize}
O mesmo que \textunderscore tonitruante\textunderscore .
\section{Tonitruoso}
\begin{itemize}
\item {Grp. gram.:adj.}
\end{itemize}
\begin{itemize}
\item {Proveniência:(De \textunderscore tonítruo\textunderscore )}
\end{itemize}
Tonitruante.
Sujeito a trovoadas.
\section{Tonizar}
\begin{itemize}
\item {Grp. gram.:v. t.}
\end{itemize}
O mesmo que \textunderscore tonificar\textunderscore :«\textunderscore ...tonizar a arca do peito de ar bem oxygenado.\textunderscore »Camillo, \textunderscore Cancion. Al.\textunderscore , XIX.
\section{Tono}
\begin{itemize}
\item {Grp. gram.:m.}
\end{itemize}
\begin{itemize}
\item {Proveniência:(Lat. \textunderscore tonus\textunderscore )}
\end{itemize}
Tom.
Ária; tonadilha.
Attitude.
\section{Tonôa}
\begin{itemize}
\item {Grp. gram.:f.}
\end{itemize}
Consêrto em tonéis, ou em vasilhas análogas.
(Talvez por \textunderscore tanôa\textunderscore , sob a infl. de \textunderscore tonel\textunderscore )
\section{Tonómetro}
\begin{itemize}
\item {Grp. gram.:m.}
\end{itemize}
\begin{itemize}
\item {Proveniência:(Do gr. \textunderscore tonos\textunderscore  + \textunderscore metron\textunderscore )}
\end{itemize}
Apparelho de acústica, para conferir, com a maior exactidão, a altura de qualquer nota musical.
\section{Tonos}
\begin{itemize}
\item {Grp. gram.:m.}
\end{itemize}
\begin{itemize}
\item {Proveniência:(Gr. \textunderscore tonos\textunderscore )}
\end{itemize}
Pêso grego, equivalente a 1:500 quilogrammas.
\section{Tonquim}
\begin{itemize}
\item {Grp. gram.:m.}
\end{itemize}
\begin{itemize}
\item {Grp. gram.:Pl.}
\end{itemize}
Língua dos indígenas do Tonquim, na Ásia oriental. Cf. Filinto, XI, 195.
\textunderscore Chale de Tonquim\textunderscore , chale de sêda, bordado, que vinha de Tonquim, e que os Espanhóis chamam \textunderscore mantón de Manila\textunderscore , porque de Manila lhes vem.
Habitantes do [[Tonquim]].
\section{Tonsar}
\begin{itemize}
\item {Grp. gram.:v. t.}
\end{itemize}
\begin{itemize}
\item {Utilização:Ant.}
\end{itemize}
\begin{itemize}
\item {Proveniência:(Lat. \textunderscore tonsare\textunderscore )}
\end{itemize}
O mesmo que \textunderscore tosquiar\textunderscore .
\section{Tonsila}
\begin{itemize}
\item {Grp. gram.:f.}
\end{itemize}
\begin{itemize}
\item {Proveniência:(Lat. \textunderscore tonsilla\textunderscore )}
\end{itemize}
O mesmo que \textunderscore amígdala\textunderscore .
\section{Tonsilar}
\begin{itemize}
\item {Grp. gram.:adj.}
\end{itemize}
Relativo á tonsila.
\section{Tonsilite}
\begin{itemize}
\item {Grp. gram.:f.}
\end{itemize}
Inflamação da tonsila.
\section{Tonsilla}
\begin{itemize}
\item {Grp. gram.:f.}
\end{itemize}
\begin{itemize}
\item {Proveniência:(Lat. \textunderscore tonsilla\textunderscore )}
\end{itemize}
O mesmo que \textunderscore amýgdala\textunderscore .
\section{Tonsillar}
\begin{itemize}
\item {Grp. gram.:adj.}
\end{itemize}
Relativo á tonsilla.
\section{Tonsillite}
\begin{itemize}
\item {Grp. gram.:f.}
\end{itemize}
Inflammação da tonsilla.
\section{Tonsonito}
\begin{itemize}
\item {Grp. gram.:m.}
\end{itemize}
\begin{itemize}
\item {Utilização:Miner.}
\end{itemize}
Silicato hydratado de alumina e cal.
\section{Tonsura}
\begin{itemize}
\item {Grp. gram.:f.}
\end{itemize}
Acto ou effeito de tonsurar.
Ceremonia ecclesiástica, em que o Bispo dá um pequeno córte no cabello do ordinando, ao conferir-lhe Ordens menores.
Corôa do clérigo.
\textunderscore Prima tonsura\textunderscore , ceremónia religiosa, em que o Prelado dá um córte no cabello do ordinando, ao conferir-lhe o primeiro grau do clericato.
\section{Tonsurado}
\begin{itemize}
\item {Grp. gram.:m.}
\end{itemize}
\begin{itemize}
\item {Proveniência:(De \textunderscore tonsurar\textunderscore )}
\end{itemize}
O mesmo que \textunderscore clérigo\textunderscore .
\section{Tonsurar}
\begin{itemize}
\item {Grp. gram.:v. t.}
\end{itemize}
\begin{itemize}
\item {Proveniência:(Lat. \textunderscore tonsurare\textunderscore )}
\end{itemize}
Tosquiar.
Praticar a ceremónia da tonsura em.
\section{Tonta}
\begin{itemize}
\item {Grp. gram.:f.}
\end{itemize}
\begin{itemize}
\item {Utilização:Pop.}
\end{itemize}
\begin{itemize}
\item {Proveniência:(De \textunderscore tonta\textunderscore )}
\end{itemize}
Mulhér tonta; mulhér idiota.
A cabeça.
\section{Tontaria}
\begin{itemize}
\item {Grp. gram.:f.}
\end{itemize}
\begin{itemize}
\item {Proveniência:(De \textunderscore tonto\textunderscore )}
\end{itemize}
Acto ou dito de pessôa tonta.
Disparate; tolice.
\section{Tontear}
\begin{itemize}
\item {Grp. gram.:v. i.}
\end{itemize}
Dizer ou fazer tolices.
Proceder como tonto.
Disparatar.
Estar tonto; perturbar-se.
Escabecear.
\section{Tonteira}
\begin{itemize}
\item {Grp. gram.:f.}
\end{itemize}
\begin{itemize}
\item {Proveniência:(De \textunderscore tonto\textunderscore )}
\end{itemize}
Tontice; tontura.
\section{Tontice}
\begin{itemize}
\item {Grp. gram.:f.}
\end{itemize}
\begin{itemize}
\item {Proveniência:(De \textunderscore tonto\textunderscore )}
\end{itemize}
Acto ou dito de tonto; tolice.
Demência.
\section{Tontina}
\begin{itemize}
\item {Grp. gram.:f.}
\end{itemize}
\begin{itemize}
\item {Proveniência:(De \textunderscore Tonti\textunderscore , n. p.)}
\end{itemize}
Associação, em que o capital dos membros fallecidos passa para os sobreviventes.
Qualquer operação financeira, baseada na duração da vida humana.
\section{Tonto}
\begin{itemize}
\item {Grp. gram.:m.}
\end{itemize}
\begin{itemize}
\item {Grp. gram.:M.}
\end{itemize}
Attónito.
Perturbado.
Que tem tonturas.
Idiota.
Demente.
Indivíduo tonto; pàteta.
(Contr. do \textunderscore attónito\textunderscore )
\section{Tontura}
\begin{itemize}
\item {Grp. gram.:f.}
\end{itemize}
\begin{itemize}
\item {Proveniência:(De \textunderscore tonto\textunderscore )}
\end{itemize}
Perturbação de cabeça; vertigem; estonteamento.
\section{Tôo}
\begin{itemize}
\item {Grp. gram.:m.}
\end{itemize}
\begin{itemize}
\item {Utilização:Prov.}
\end{itemize}
\begin{itemize}
\item {Utilização:trasm.}
\end{itemize}
\begin{itemize}
\item {Proveniência:(De \textunderscore toar\textunderscore )}
\end{itemize}
O mesmo que \textunderscore trovão\textunderscore .
\section{Topa}
\begin{itemize}
\item {Grp. gram.:f.}
\end{itemize}
\begin{itemize}
\item {Proveniência:(De \textunderscore topar\textunderscore )}
\end{itemize}
Jôgo de crianças.
\section{Topa-a-tudo}
\begin{itemize}
\item {Grp. gram.:m.}
\end{itemize}
\begin{itemize}
\item {Utilização:Fam.}
\end{itemize}
Fura-vidas, indivíduo que em tudo procura achar interesse ou vantagem.
\section{Topa-carneiro}
\begin{itemize}
\item {Grp. gram.:m.}
\end{itemize}
Sorte de bandarilhas, em que o toireiro cita o animal de frente e a pé firme, fazendo quebro quando o toiro humilha e pregando-lhe então as bandarilhas.
\section{Topada}
\begin{itemize}
\item {Grp. gram.:f.}
\end{itemize}
\begin{itemize}
\item {Proveniência:(De \textunderscore topar\textunderscore )}
\end{itemize}
Acto ou effeito de bater involuntariamente com a ponta do pé.
Choque.
\section{Topar}
\begin{itemize}
\item {Grp. gram.:v. t.}
\end{itemize}
\begin{itemize}
\item {Grp. gram.:V. i.}
\end{itemize}
\begin{itemize}
\item {Proveniência:(De \textunderscore tope\textunderscore )}
\end{itemize}
Encontrar; achar: \textunderscore topar alguns conhecidos\textunderscore .
Jogar contra (todo o dinheiro que está na banca do jôgo): \textunderscore topar a banca\textunderscore .
Ir de encontro; encontrar-se.
Dar com o pé.
Bater.
Chegar.
\section{Toparca}
\begin{itemize}
\item {Grp. gram.:m.}
\end{itemize}
\begin{itemize}
\item {Proveniência:(Lat. \textunderscore toparcha\textunderscore )}
\end{itemize}
Chefe de uma toparquia.
\section{Toparcha}
\begin{itemize}
\item {fónica:ca}
\end{itemize}
\begin{itemize}
\item {Grp. gram.:m.}
\end{itemize}
\begin{itemize}
\item {Proveniência:(Lat. \textunderscore toparcha\textunderscore )}
\end{itemize}
Chefe de uma toparchia.
\section{Toparchia}
\begin{itemize}
\item {fónica:qui}
\end{itemize}
\begin{itemize}
\item {Grp. gram.:f.}
\end{itemize}
\begin{itemize}
\item {Proveniência:(Lat. \textunderscore toparchia\textunderscore )}
\end{itemize}
Espécie de principado independente, na antiguidade.
\section{Toparquia}
\begin{itemize}
\item {Grp. gram.:f.}
\end{itemize}
\begin{itemize}
\item {Proveniência:(Lat. \textunderscore toparchia\textunderscore )}
\end{itemize}
Espécie de principado independente, na antiguidade.
\section{Topaz}
\begin{itemize}
\item {Grp. gram.:m.}
\end{itemize}
\begin{itemize}
\item {Utilização:Ant.}
\end{itemize}
Christão mestiço, no Oriente.
Intérprete chinês. Cf. \textunderscore Cartas do Japão\textunderscore , ed. de Evora, 43.
\section{Topázio}
\begin{itemize}
\item {Grp. gram.:m.}
\end{itemize}
\begin{itemize}
\item {Proveniência:(Lat. \textunderscore topazius\textunderscore )}
\end{itemize}
Pedra preciosa, de côr amarela.
\textunderscore Falso topázio\textunderscore , espécie de quartzo amarelado.
\section{Tope}
\begin{itemize}
\item {Grp. gram.:m.}
\end{itemize}
\begin{itemize}
\item {Utilização:Fig.}
\end{itemize}
\begin{itemize}
\item {Utilização:Náut.}
\end{itemize}
\begin{itemize}
\item {Proveniência:(Ingl. \textunderscore top\textunderscore )}
\end{itemize}
Encontro ou choque de dois objectos.
Tôpo, cimo.
Planta amaryllídea.
O mais alto grau, cúmulo.
Laço de fita em chapéu, toucado, etc.
\textunderscore A tope\textunderscore , a topetar.
\section{Topejar}
\begin{itemize}
\item {Grp. gram.:v. t.}
\end{itemize}
\begin{itemize}
\item {Utilização:T. de serralh}
\end{itemize}
\begin{itemize}
\item {Proveniência:(De \textunderscore tôpo\textunderscore )}
\end{itemize}
Aproximar, unindo-os, os topos de (duas peças).
\section{Topetada}
\begin{itemize}
\item {Grp. gram.:f.}
\end{itemize}
\begin{itemize}
\item {Proveniência:(De \textunderscore topetar\textunderscore )}
\end{itemize}
Pancada com a cabeça; marrada.
\section{Topetar}
\begin{itemize}
\item {Grp. gram.:v. t.}
\end{itemize}
\begin{itemize}
\item {Grp. gram.:V. i.}
\end{itemize}
\begin{itemize}
\item {Proveniência:(De \textunderscore topête\textunderscore )}
\end{itemize}
Tocar o ponto mais alto de.
Subir ás alturas de.
Elevar-se a.
Bater com topete ou com a cabeça.
Dar marradas.
Elevar-se.
Tocar no ponto mais alto.
\section{Topête}
\begin{itemize}
\item {Grp. gram.:m.}
\end{itemize}
\begin{itemize}
\item {Utilização:Pop.}
\end{itemize}
\begin{itemize}
\item {Proveniência:(De \textunderscore tope\textunderscore )}
\end{itemize}
Cabello, levantado á frente da cabeça.
Parte anterior e elevada da cabelleira de palhaço.
Parte da crina do cavallo, pendente sôbre a testa.
Pennas alongadas, que se levantam na cabeça de algumas aves, como o \textunderscore cardial\textunderscore .
Cabeça.
\section{Topeteira}
\begin{itemize}
\item {Grp. gram.:f.}
\end{itemize}
\begin{itemize}
\item {Proveniência:(De \textunderscore topete\textunderscore )}
\end{itemize}
O mesmo que \textunderscore testeira\textunderscore .
\section{Topetudo}
\begin{itemize}
\item {Grp. gram.:adj.}
\end{itemize}
\begin{itemize}
\item {Utilização:Bras}
\end{itemize}
Que traz topete.
Valente, destemido.
\section{Topho}
\begin{itemize}
\item {Grp. gram.:m.}
\end{itemize}
\begin{itemize}
\item {Proveniência:(Lat. \textunderscore tophus\textunderscore )}
\end{itemize}
Substância dura, que se fórma no interior dos órgãos ou junto das articulações dos seres animados.
\section{Topiaria}
\begin{itemize}
\item {Grp. gram.:f.}
\end{itemize}
\begin{itemize}
\item {Proveniência:(Lat. \textunderscore topiaria\textunderscore )}
\end{itemize}
Arte de adornar os jardins, dando a grupos de plantas configurações diversas.
\section{Topiário}
\begin{itemize}
\item {Grp. gram.:m.}
\end{itemize}
\begin{itemize}
\item {Proveniência:(Lat. \textunderscore topiarius\textunderscore )}
\end{itemize}
Jardineiro, que pratíca a topiaria.
\section{Tópica}
\begin{itemize}
\item {Grp. gram.:f.}
\end{itemize}
\begin{itemize}
\item {Proveniência:(De \textunderscore tópico\textunderscore )}
\end{itemize}
Doutrina dos tópicos ou remédios tópicos.
\section{Tópico}
\begin{itemize}
\item {Grp. gram.:adj.}
\end{itemize}
\begin{itemize}
\item {Grp. gram.:M.}
\end{itemize}
\begin{itemize}
\item {Grp. gram.:Pl.}
\end{itemize}
\begin{itemize}
\item {Proveniência:(Gr. \textunderscore topikos\textunderscore )}
\end{itemize}
Relativo a lugar.
Relativo exactamente àquillo de que se trata.
Externo, (falando-se de medicamentos).
E diz-se dos lugares communs, em Rethórica.
Remédio tópico.
Remédio.
Ponto principal.
Thema.
Lugares communs.
Sýnthese, generalidade.
\section{Topinambo}
\begin{itemize}
\item {Grp. gram.:m.}
\end{itemize}
Planta, originária do Chile, também conhecida por \textunderscore girasol batateiro\textunderscore , que produz tubérculos comestíveis, de que há duas variedades, vermelha e amarela.
O tubérculo do topinambo.
\section{Topinambor}
\begin{itemize}
\item {Grp. gram.:m.}
\end{itemize}
O mesmo que \textunderscore topinambo\textunderscore .
\section{Topinho}
\begin{itemize}
\item {Grp. gram.:adj.}
\end{itemize}
\begin{itemize}
\item {Utilização:Pop.}
\end{itemize}
\begin{itemize}
\item {Proveniência:(De \textunderscore tôpo\textunderscore )}
\end{itemize}
Diz-se da cavalgadura que tem os talões e quartos muito altos.
Que tem os pés cambados ou com as pontas inclinadas para dentro.
\section{Topino}
\begin{itemize}
\item {Grp. gram.:adj.}
\end{itemize}
\begin{itemize}
\item {Utilização:Prov.}
\end{itemize}
\begin{itemize}
\item {Utilização:trasm.}
\end{itemize}
Cambaio, torto das pernas.
(Cp. \textunderscore topinho\textunderscore )
\section{Tôpo}
\begin{itemize}
\item {Grp. gram.:m.}
\end{itemize}
Tope, cume; extremidade.
(Cp. \textunderscore tope\textunderscore )
\section{Tópo}
\begin{itemize}
\item {Grp. gram.:m.}
\end{itemize}
\begin{itemize}
\item {Utilização:Des.}
\end{itemize}
Encontro, acto de topar; tope.
(Cp. \textunderscore tope\textunderscore )
\section{Topofobia}
\begin{itemize}
\item {Grp. gram.:f.}
\end{itemize}
\begin{itemize}
\item {Proveniência:(Do gr. \textunderscore topos\textunderscore  + \textunderscore phobein\textunderscore )}
\end{itemize}
Mêdo mórbido a lugares.
\section{Topófobo}
\begin{itemize}
\item {Grp. gram.:m.}
\end{itemize}
Aquele que sofre topofobia.
\section{Topofotografia}
\begin{itemize}
\item {Grp. gram.:f.}
\end{itemize}
O mesmo que \textunderscore fototopografia\textunderscore .
\section{Topografia}
\begin{itemize}
\item {Grp. gram.:f.}
\end{itemize}
\begin{itemize}
\item {Utilização:Neol.}
\end{itemize}
\begin{itemize}
\item {Proveniência:(Lat. \textunderscore topographia\textunderscore )}
\end{itemize}
Descripção minuciosa de uma localidade.
Arte de representar no papel a configuração de uma porção de terreno, com todos os objectos que estão á superfície dêste.
Descripção anatómica e minuciosa de qualquer parte do organismo humano: \textunderscore a topografia craniana...\textunderscore 
\section{Topograficamente}
\begin{itemize}
\item {Grp. gram.:adv.}
\end{itemize}
De modo topográfico.
Segundo os processos da topografia.
\section{Topográfico}
\begin{itemize}
\item {Grp. gram.:adj.}
\end{itemize}
Relativo á topografia.
\section{Topógrafo}
\begin{itemize}
\item {Grp. gram.:m.}
\end{itemize}
\begin{itemize}
\item {Proveniência:(Gr. \textunderscore topographos\textunderscore )}
\end{itemize}
Aquele que se ocupa de topografia.
\section{Topographia}
\begin{itemize}
\item {Grp. gram.:f.}
\end{itemize}
\begin{itemize}
\item {Utilização:Neol.}
\end{itemize}
\begin{itemize}
\item {Proveniência:(Lat. \textunderscore topographia\textunderscore )}
\end{itemize}
Descripção minuciosa de uma localidade.
Arte de representar no papel a configuração de uma porção de terreno, com todos os objectos que estão á superfície dêste.
Descripção anatómica e minuciosa de qualquer parte do organismo humano: \textunderscore a topographia craniana...\textunderscore 
\section{Topographicamente}
\begin{itemize}
\item {Grp. gram.:adv.}
\end{itemize}
De modo topográphico.
Segundo os processos da topographia.
\section{Topográphico}
\begin{itemize}
\item {Grp. gram.:adj.}
\end{itemize}
Relativo á topographia.
\section{Topógrapho}
\begin{itemize}
\item {Grp. gram.:m.}
\end{itemize}
\begin{itemize}
\item {Proveniência:(Gr. \textunderscore topographos\textunderscore )}
\end{itemize}
Aquelle que se occupa de topographia.
\section{Topologia}
\begin{itemize}
\item {Grp. gram.:f.}
\end{itemize}
\begin{itemize}
\item {Utilização:Gram.}
\end{itemize}
\begin{itemize}
\item {Proveniência:(Do gr. \textunderscore topos\textunderscore  + \textunderscore logos\textunderscore )}
\end{itemize}
O mesmo que \textunderscore topographia\textunderscore .
Tratado da collocação ou disposição de certas espécies de palavras: \textunderscore topologia pronominal\textunderscore .
\section{Topológico}
\begin{itemize}
\item {Grp. gram.:adj.}
\end{itemize}
Relativo á topologia.
\section{Toponímia}
\begin{itemize}
\item {Grp. gram.:f.}
\end{itemize}
\begin{itemize}
\item {Proveniência:(Do gr. \textunderscore topos\textunderscore  + \textunderscore onuma\textunderscore )}
\end{itemize}
Designação dos lugares pelos seus nomes.
\section{Toponímico}
\begin{itemize}
\item {Grp. gram.:adj.}
\end{itemize}
Relativo á toponímia.
\section{Toponomástica}
\begin{itemize}
\item {Grp. gram.:f.}
\end{itemize}
\begin{itemize}
\item {Proveniência:(Do gr. \textunderscore topos\textunderscore  + \textunderscore onoma\textunderscore )}
\end{itemize}
Onomástica dos lugares.
\section{Toponomástico}
\begin{itemize}
\item {Grp. gram.:adj.}
\end{itemize}
\begin{itemize}
\item {Grp. gram.:M.}
\end{itemize}
Relativo a toponomástica.
O mesmo que \textunderscore toponomástica\textunderscore .
\section{Toponýmia}
\begin{itemize}
\item {Grp. gram.:f.}
\end{itemize}
\begin{itemize}
\item {Proveniência:(Do gr. \textunderscore topos\textunderscore  + \textunderscore onuma\textunderscore )}
\end{itemize}
Designação dos lugares pelos seus nomes.
\section{Toponýmico}
\begin{itemize}
\item {Grp. gram.:adj.}
\end{itemize}
Relativo á toponýmia.
\section{Topophobia}
\begin{itemize}
\item {Grp. gram.:f.}
\end{itemize}
\begin{itemize}
\item {Proveniência:(Do gr. \textunderscore topos\textunderscore  + \textunderscore phobein\textunderscore )}
\end{itemize}
Mêdo mórbido a lugares.
\section{Topóphobo}
\begin{itemize}
\item {Grp. gram.:m.}
\end{itemize}
Aquelle que soffre topophobia.
\section{Topophotographia}
\begin{itemize}
\item {Grp. gram.:f.}
\end{itemize}
O mesmo que \textunderscore phototopographia\textunderscore .
\section{Toporama}
\begin{itemize}
\item {Grp. gram.:m.}
\end{itemize}
\begin{itemize}
\item {Proveniência:(Do gr. \textunderscore topos\textunderscore  + \textunderscore oraein\textunderscore )}
\end{itemize}
Panorama de um determinado lugar.
\section{Topotesia}
\begin{itemize}
\item {Grp. gram.:f.}
\end{itemize}
\begin{itemize}
\item {Proveniência:(Lat. \textunderscore topothesia\textunderscore )}
\end{itemize}
Descripção de um lugar imaginário.
\section{Topothesia}
\begin{itemize}
\item {Grp. gram.:f.}
\end{itemize}
\begin{itemize}
\item {Proveniência:(Lat. \textunderscore topothesia\textunderscore )}
\end{itemize}
Descripção de um lugar imaginário.
\section{Toque}
\begin{itemize}
\item {Grp. gram.:m.}
\end{itemize}
\begin{itemize}
\item {Utilização:Fig.}
\end{itemize}
Acto ou effeito de tocar.
Contacto.
Som, produzido pelo contacto: \textunderscore toque de campaínha\textunderscore .
Pancada.
Som, produzido por pancada.
Acto de tocar instrumentos músicos.
Som de instrumentos músicos.
Aperto de mão, como sinal de cortesia.
Retoque, em pintura.
Remoque.
Sabor ou cheiro peculiar de certos vinhos.
Sinal, vestígio.
Mancha, que, na fruta, indica princípio de putrefacção.
Inspiração.
Esmêro num trabalho artístico.
Meio de conhecer ou de avaliar: \textunderscore pedra de toque\textunderscore .
\section{Toque}
\begin{itemize}
\item {Grp. gram.:m.}
\end{itemize}
\begin{itemize}
\item {Utilização:Des.}
\end{itemize}
Unidade pecuniária que, na costa oriental da África, correspondia a quarenta caurins.
\section{Toquedás}
\begin{itemize}
\item {Grp. gram.:m. pl.}
\end{itemize}
Indígenas do norte do Brasil.
\section{Toque-emboque}
\begin{itemize}
\item {Grp. gram.:m.}
\end{itemize}
\begin{itemize}
\item {Proveniência:(De \textunderscore tocar\textunderscore  + \textunderscore embocar\textunderscore )}
\end{itemize}
Jôgo, com bola e arco.
\section{Toqueixo}
\begin{itemize}
\item {Grp. gram.:m.}
\end{itemize}
\begin{itemize}
\item {Utilização:Ant.}
\end{itemize}
Espécie de touca.
(Por \textunderscore touqueixo\textunderscore , de \textunderscore touca\textunderscore )
\section{Toque-remoque}
\begin{itemize}
\item {Grp. gram.:m.}
\end{itemize}
\begin{itemize}
\item {Utilização:T. da Bairrada}
\end{itemize}
Jôgo de rapazes.
\section{Tora}
\begin{itemize}
\item {Grp. gram.:f.}
\end{itemize}
Nome, que os Judeus portugueses davam ao livro da sua lei.
Tributo, que os Judeus pagavam por família.
\section{Tora}
\begin{itemize}
\item {Grp. gram.:f.}
\end{itemize}
\begin{itemize}
\item {Utilização:mil.}
\end{itemize}
\begin{itemize}
\item {Utilização:Gír.}
\end{itemize}
\begin{itemize}
\item {Utilização:Bras. do N}
\end{itemize}
Carne do rancho, correspondente a cada marmita.
Pedaço de alguma coisa; fatia.
(Relaciona-se com \textunderscore tôro\textunderscore ?)
\section{Toracete}
\begin{itemize}
\item {fónica:cê}
\end{itemize}
\begin{itemize}
\item {Grp. gram.:m.}
\end{itemize}
Pequeno tórax.
(Dem. de \textunderscore thórax\textunderscore )
\section{Torácico}
\begin{itemize}
\item {Grp. gram.:adj.}
\end{itemize}
\begin{itemize}
\item {Grp. gram.:M. pl.}
\end{itemize}
\begin{itemize}
\item {Proveniência:(Gr. \textunderscore thorakikos\textunderscore )}
\end{itemize}
Relativo ao tórax.
Ordem de peixes ósseos.
Família de coleópteros.
\section{Toracocêntese}
\begin{itemize}
\item {Grp. gram.:f.}
\end{itemize}
\begin{itemize}
\item {Proveniência:(Do gr. \textunderscore thorax\textunderscore , \textunderscore torakos\textunderscore  + \textunderscore kentesis\textunderscore )}
\end{itemize}
Operação cirúrgica, em que se abrem as paredes do tórax, para dar saída a um líquido acumulado na cavidade pleural.
\section{Toracóforo}
\begin{itemize}
\item {Grp. gram.:m.}
\end{itemize}
\begin{itemize}
\item {Proveniência:(Do gr. \textunderscore thorax\textunderscore  + \textunderscore phoros\textunderscore )}
\end{itemize}
Gênero de insectos coleópteros heterómeros.
\section{Toracometria}
\begin{itemize}
\item {Grp. gram.:f.}
\end{itemize}
\begin{itemize}
\item {Proveniência:(Do gr. \textunderscore thorax\textunderscore  + \textunderscore metron\textunderscore )}
\end{itemize}
Mensuração do tórax.
\section{Toracométrico}
\begin{itemize}
\item {Grp. gram.:adj.}
\end{itemize}
Relativo á toracometria.
\section{Toracópago}
\begin{itemize}
\item {Grp. gram.:m.}
\end{itemize}
\begin{itemize}
\item {Utilização:Terat.}
\end{itemize}
\begin{itemize}
\item {Proveniência:(Do gr. \textunderscore thorax\textunderscore  + \textunderscore pagein\textunderscore )}
\end{itemize}
Monstro duplo, formado de dois indivíduos, ligados pelo tórax.
\section{Toracoplastia}
\begin{itemize}
\item {Grp. gram.:f.}
\end{itemize}
\begin{itemize}
\item {Utilização:Med.}
\end{itemize}
\begin{itemize}
\item {Proveniência:(Do gr. \textunderscore thorax\textunderscore  + \textunderscore plassein\textunderscore )}
\end{itemize}
Modificação cirúrgica da conformação do tórax.
\section{Toracoscopia}
\begin{itemize}
\item {Grp. gram.:f.}
\end{itemize}
\begin{itemize}
\item {Utilização:Med.}
\end{itemize}
\begin{itemize}
\item {Proveniência:(Do gr. \textunderscore thorax\textunderscore  + \textunderscore skopein\textunderscore )}
\end{itemize}
Observação do peito.
\section{Toracotomia}
\begin{itemize}
\item {Grp. gram.:f.}
\end{itemize}
\begin{itemize}
\item {Utilização:Med.}
\end{itemize}
\begin{itemize}
\item {Proveniência:(Do gr. \textunderscore thorax\textunderscore  + \textunderscore tome\textunderscore )}
\end{itemize}
Acto cirúrgico de abrir o tórax.
\section{Toracozoário}
\begin{itemize}
\item {Grp. gram.:adj.}
\end{itemize}
\begin{itemize}
\item {Utilização:Zool.}
\end{itemize}
\begin{itemize}
\item {Proveniência:(Do gr. \textunderscore thorax\textunderscore  + \textunderscore zoon\textunderscore )}
\end{itemize}
Diz-se dos animaes, cujos órgãos respiratórios adquiriram grande desenvolvimento.
\section{Toradelfo}
\begin{itemize}
\item {Grp. gram.:m.}
\end{itemize}
\begin{itemize}
\item {Utilização:Terat.}
\end{itemize}
\begin{itemize}
\item {Proveniência:(Do gr. \textunderscore thorax\textunderscore  + \textunderscore adelphos\textunderscore )}
\end{itemize}
Monstro duplo monocéfalo, formado de dois indivíduos, separados, do umbigo para baixo, e confundidos daí para cima.
\section{Toragem}
\begin{itemize}
\item {Grp. gram.:f.}
\end{itemize}
\begin{itemize}
\item {Utilização:Prov.}
\end{itemize}
\begin{itemize}
\item {Utilização:alent.}
\end{itemize}
Acto de torar.
\section{Toral}
\begin{itemize}
\item {Grp. gram.:m.}
\end{itemize}
\begin{itemize}
\item {Utilização:Des.}
\end{itemize}
\begin{itemize}
\item {Grp. gram.:Pl.}
\end{itemize}
\begin{itemize}
\item {Utilização:Prov.}
\end{itemize}
\begin{itemize}
\item {Utilização:minh.}
\end{itemize}
\begin{itemize}
\item {Proveniência:(De \textunderscore tôro\textunderscore )}
\end{itemize}
A arte mais grossa ou forte da lança.
Cabeção, em camisa de mulhér.
As duas peças do peito da camisa.
\section{Toralho}
\begin{itemize}
\item {Grp. gram.:m.}
\end{itemize}
\begin{itemize}
\item {Utilização:T. da Bairrada}
\end{itemize}
O mesmo que \textunderscore toiral\textunderscore .
(Cp. \textunderscore toiralho\textunderscore )
\section{Toranja}
\begin{itemize}
\item {Grp. gram.:f.}
\end{itemize}
O mesmo que \textunderscore toronja\textunderscore .
\section{Torão}
\begin{itemize}
\item {Grp. gram.:m.}
\end{itemize}
\begin{itemize}
\item {Utilização:Mad}
\end{itemize}
Belga, coirela.
\section{Torar}
\begin{itemize}
\item {Grp. gram.:v. t.}
\end{itemize}
\begin{itemize}
\item {Utilização:Bras}
\end{itemize}
\begin{itemize}
\item {Utilização:Bras. do N}
\end{itemize}
\begin{itemize}
\item {Proveniência:(De \textunderscore tôro\textunderscore )}
\end{itemize}
Partir em toros.
Atravessar.
Cortar, fazer em pedaços.
Cp. \textunderscore tora\textunderscore ^2.
\section{Torás}
\begin{itemize}
\item {Grp. gram.:m. pl.}
\end{itemize}
Indígenas do norte do Brasil.
\section{Tórax}
\begin{itemize}
\item {Grp. gram.:m.}
\end{itemize}
\begin{itemize}
\item {Proveniência:(Lat. \textunderscore thorax\textunderscore )}
\end{itemize}
Peito; cavidade do peito.
Segmento intermédio do corpo dos insectos.
\section{Torba}
\begin{itemize}
\item {fónica:tôr}
\end{itemize}
\begin{itemize}
\item {Grp. gram.:f.}
\end{itemize}
\begin{itemize}
\item {Utilização:Prov.}
\end{itemize}
\begin{itemize}
\item {Utilização:beir.}
\end{itemize}
\begin{itemize}
\item {Proveniência:(T. gall.)}
\end{itemize}
O mesmo que \textunderscore moéga\textunderscore . (Colhido na Guarda)
\section{Torça}
\begin{itemize}
\item {Grp. gram.:f.}
\end{itemize}
Pedra quadrilonga e esquadriada.
Vêrga de porta; padieira.
\section{Torçado}
\begin{itemize}
\item {Grp. gram.:m.}
\end{itemize}
\begin{itemize}
\item {Proveniência:(De \textunderscore torça\textunderscore )}
\end{itemize}
Vêrga de porta, torça.
\section{Torçal}
\begin{itemize}
\item {Grp. gram.:m.}
\end{itemize}
\begin{itemize}
\item {Utilização:Bras. do S}
\end{itemize}
\begin{itemize}
\item {Proveniência:(De \textunderscore torcer\textunderscore )}
\end{itemize}
Cordão, feito de fios de retrós.
Cordão de sêda, com fios de oiro.
Espécie de cabresto, para conter animaes ariscos.
\section{Torçalado}
\begin{itemize}
\item {Grp. gram.:adj.}
\end{itemize}
Guarnecido com torçal.
\section{Torção}
\begin{itemize}
\item {Grp. gram.:f.}
\end{itemize}
\begin{itemize}
\item {Proveniência:(Do lat. \textunderscore tortio\textunderscore )}
\end{itemize}
Torcedura.
Cólica de certos animaes, especialmente do cavallo.
\section{Torcaz}
\begin{itemize}
\item {Grp. gram.:m.  e  adj.}
\end{itemize}
\begin{itemize}
\item {Proveniência:(Do lat. \textunderscore torquax\textunderscore )}
\end{itemize}
Diz-se de uma espécie de pombo, (\textunderscore columba palumbus\textunderscore , Lin.), cujo pescoço tem várias côres.
\section{Torcear}
\begin{itemize}
\item {Grp. gram.:v. t.}
\end{itemize}
Pôr torça em.
\section{Torcedeira}
\begin{itemize}
\item {Grp. gram.:f.}
\end{itemize}
O mesmo que \textunderscore torcedoira\textunderscore .
O mesmo que \textunderscore torcedora\textunderscore .
\section{Torcedela}
\begin{itemize}
\item {Grp. gram.:f.}
\end{itemize}
O mesmo que \textunderscore torcedura\textunderscore .
\section{Torcedoira}
\begin{itemize}
\item {Grp. gram.:f.}
\end{itemize}
\begin{itemize}
\item {Proveniência:(De \textunderscore torcer\textunderscore )}
\end{itemize}
Apparelho de fiação; torcedor. Cf. \textunderscore Inquér. Industr.\textunderscore , II, l. II, 123.
\section{Torcedor}
\begin{itemize}
\item {Grp. gram.:adj.}
\end{itemize}
\begin{itemize}
\item {Grp. gram.:M.}
\end{itemize}
\begin{itemize}
\item {Proveniência:(De \textunderscore torcer\textunderscore )}
\end{itemize}
Que torce.
Instrumento para torcer.
Fuso.
Arrôcho.
\section{Torcedora}
\begin{itemize}
\item {Grp. gram.:f.}
\end{itemize}
\begin{itemize}
\item {Proveniência:(De \textunderscore torcer\textunderscore )}
\end{itemize}
Parasito do pinheiro e de outras árvores, cujas larvas atacam os rebentos, tornando-os tortuosos. Cf. P. Moraes, \textunderscore Zool. Elem.\textunderscore , 678.
\section{Torcedoura}
\begin{itemize}
\item {Grp. gram.:f.}
\end{itemize}
\begin{itemize}
\item {Proveniência:(De \textunderscore torcer\textunderscore )}
\end{itemize}
Apparelho de fiação; torcedor. Cf. \textunderscore Inquér. Industr.\textunderscore , II, l. II, 123.
\section{Torcedura}
\begin{itemize}
\item {Grp. gram.:f.}
\end{itemize}
\begin{itemize}
\item {Utilização:Fig.}
\end{itemize}
Acto ou effeito de torcer.
Sinuosidade.
Evasiva; sophisma.
\section{Torcegar}
\begin{itemize}
\item {Grp. gram.:v. t.}
\end{itemize}
O mesmo que \textunderscore estorcegar\textunderscore .
\section{Torcer}
\begin{itemize}
\item {Grp. gram.:v. t.}
\end{itemize}
\begin{itemize}
\item {Grp. gram.:V. i.}
\end{itemize}
\begin{itemize}
\item {Proveniência:(Do lat. \textunderscore torquere\textunderscore )}
\end{itemize}
Fazer volver (objecto mais ou menos longo) pelas duas extremidades em sentido contrário, ou por uma só, estando a outra fixa.
Tornar torto.
Encurvar.
Desviar.
Deslocar.
Fazer mudar de rumo ou de tenção.
Induzir.
Perverter; alterar: \textunderscore torcer o sentido\textunderscore .
Encaracolar.
Dar volta.
Desistir de um plano.
Vergar-se.
Annuir; sujeitar-se.
\section{Torcicollo}
\begin{itemize}
\item {Grp. gram.:m.}
\end{itemize}
\begin{itemize}
\item {Utilização:Fig.}
\end{itemize}
\begin{itemize}
\item {Proveniência:(De \textunderscore torcer\textunderscore  + \textunderscore collo\textunderscore )}
\end{itemize}
Rodeio; sinuosidade.
Inclinação involuntária da cabeça, em virtude de inflammação ou dores nos músculos do pescoço.
Ave trepadeira, papa-formigas.
Ambiguidade.
\section{Torcicolo}
\begin{itemize}
\item {Grp. gram.:m.}
\end{itemize}
\begin{itemize}
\item {Utilização:Fig.}
\end{itemize}
\begin{itemize}
\item {Proveniência:(De \textunderscore torcer\textunderscore  + \textunderscore collo\textunderscore )}
\end{itemize}
Rodeio; sinuosidade.
Inclinação involuntária da cabeça, em virtude de inflamação ou dores nos músculos do pescoço.
Ave trepadeira, papa-formigas.
Ambiguidade.
\section{Torcida}
\begin{itemize}
\item {Grp. gram.:f.}
\end{itemize}
\begin{itemize}
\item {Utilização:Gír.}
\end{itemize}
\begin{itemize}
\item {Proveniência:(De \textunderscore torcido\textunderscore )}
\end{itemize}
Mecha de candeeiro ou de vela.
Pavio.
Objecto, semelhante a uma torcida.
Objecto torcido.
\textunderscore Torcida grossa\textunderscore , pechincha.
\section{Torcidamente}
\begin{itemize}
\item {Grp. gram.:adv.}
\end{itemize}
\begin{itemize}
\item {Proveniência:(De \textunderscore torcido\textunderscore )}
\end{itemize}
De modo constrangido; tortuosamente.
\section{Torcilhão}
\begin{itemize}
\item {Grp. gram.:m.}
\end{itemize}
O mesmo que \textunderscore torção\textunderscore .
\section{Torcimento}
\begin{itemize}
\item {Grp. gram.:m.}
\end{itemize}
O mesmo que \textunderscore torcedura\textunderscore .
\section{Torcisco}
\begin{itemize}
\item {Grp. gram.:m.}
\end{itemize}
\begin{itemize}
\item {Utilização:Ant.}
\end{itemize}
O mesmo que \textunderscore trocisco\textunderscore ^1.
\section{Torço}
\begin{itemize}
\item {fónica:tôr}
\end{itemize}
\begin{itemize}
\item {Grp. gram.:m.}
\end{itemize}
\begin{itemize}
\item {Utilização:Bras}
\end{itemize}
O mesmo que \textunderscore torcedura\textunderscore . Cf. \textunderscore Inquér. Industr.\textunderscore , III, p. III, 237.
Chale ou manta, que os Baïanos enrolam na cabeça, como turbante.
\section{Torçol}
\begin{itemize}
\item {Grp. gram.:m.}
\end{itemize}
(V.terçol)
\section{Torcular}
\begin{itemize}
\item {Grp. gram.:adj.}
\end{itemize}
\begin{itemize}
\item {Grp. gram.:M.}
\end{itemize}
\begin{itemize}
\item {Utilização:Anat.}
\end{itemize}
\begin{itemize}
\item {Proveniência:(Lat. \textunderscore torcularis\textunderscore )}
\end{itemize}
Em fórma de tórculo.
Confluência de alguns seios cerebraes, conhecida também por \textunderscore lagar de Heróphilo\textunderscore . Cf. J. A. Serrano, \textunderscore Osteologia\textunderscore .
\section{Torcular}
\begin{itemize}
\item {Grp. gram.:v. t.}
\end{itemize}
Alisar ou polir com tórculo.
\section{Tórculo}
\begin{itemize}
\item {Grp. gram.:m.}
\end{itemize}
\begin{itemize}
\item {Proveniência:(Lat. \textunderscore torculum\textunderscore )}
\end{itemize}
Pequena prensa; prelo.
Apparelho para polir metaes.
\section{Torda}
\begin{itemize}
\item {Grp. gram.:f.}
\end{itemize}
\begin{itemize}
\item {Utilização:Prov.}
\end{itemize}
\begin{itemize}
\item {Utilização:trasm.}
\end{itemize}
\begin{itemize}
\item {Proveniência:(Do lat. \textunderscore turda\textunderscore )}
\end{itemize}
Fêmea do tordo.
O mesmo que \textunderscore bebedeira\textunderscore . (Colhido em Sabrosa)
\section{Torda-mergulheira}
\begin{itemize}
\item {Grp. gram.:f.}
\end{itemize}
Ave aquática, (\textunderscore alca torda\textunderscore , Lin.).
\section{Tordeia}
\begin{itemize}
\item {Grp. gram.:f.}
\end{itemize}
\begin{itemize}
\item {Utilização:Prov.}
\end{itemize}
\begin{itemize}
\item {Utilização:minh.}
\end{itemize}
O mesmo que \textunderscore tordeira\textunderscore .
\section{Tordeira}
\begin{itemize}
\item {Grp. gram.:f.}
\end{itemize}
Ave, semelhante ao tordo, mas maior.
\section{Tordeiro}
\begin{itemize}
\item {Grp. gram.:m.}
\end{itemize}
Ave, o mesmo que \textunderscore tarambola\textunderscore .
\section{Tordião}
\begin{itemize}
\item {Grp. gram.:m.}
\end{itemize}
\begin{itemize}
\item {Utilização:Ant.}
\end{itemize}
Baile cantado.
Canto, que acompanhava o bailado. Cf. G. Vicente, I, 227.
\section{Tordilho}
\begin{itemize}
\item {Grp. gram.:adj.}
\end{itemize}
Que tem côr de tordo.
Diz-se do cavallo, cujo pêlo tem a apparência da côr do tôrdo.
(Cast. \textunderscore tordillo\textunderscore )
\section{Tordílio}
\begin{itemize}
\item {Grp. gram.:m.}
\end{itemize}
\begin{itemize}
\item {Proveniência:(Lat. \textunderscore tordylion\textunderscore )}
\end{itemize}
Gênero de plantas umbelíferas.
\section{Tordo}
\begin{itemize}
\item {fónica:tor}
\end{itemize}
\begin{itemize}
\item {Grp. gram.:m.}
\end{itemize}
\begin{itemize}
\item {Proveniência:(Do lat. \textunderscore turdus\textunderscore )}
\end{itemize}
Gênero de pássaros dentirostros.
Peixe labroide do Mediterrâneo.
\section{Tordo-branco}
\begin{itemize}
\item {Grp. gram.:m.}
\end{itemize}
Espécie de tordo, (\textunderscore turdus musicus\textunderscore , Lin.).
\section{Tordo-marinho}
\begin{itemize}
\item {Grp. gram.:m.}
\end{itemize}
O mesmo que \textunderscore pica-peixe\textunderscore .
\section{Tordo-pisco}
\begin{itemize}
\item {Grp. gram.:m.}
\end{itemize}
Espécie de melro, se bem que as asas, a fórma do vôo e outras condições não se confundem com as do melro.--É o \textunderscore hydrobata cinclus\textunderscore , a que também se deu o nome de \textunderscore melro da água\textunderscore , ou \textunderscore melro peixeiro\textunderscore .
\section{Tordoveia}
\begin{itemize}
\item {Grp. gram.:f.}
\end{itemize}
O mesmo que \textunderscore tordeira\textunderscore .
\section{Tordo-visgueiro}
\begin{itemize}
\item {Grp. gram.:m.}
\end{itemize}
O mesmo que \textunderscore tordeira\textunderscore .
\section{Tordo-zornal}
\begin{itemize}
\item {Grp. gram.:m.}
\end{itemize}
Espécie de tordo de arribação; zorral. Cf. P. Moraes, \textunderscore Zool. Elem.\textunderscore , 335.
\section{Tordýlio}
\begin{itemize}
\item {Grp. gram.:m.}
\end{itemize}
\begin{itemize}
\item {Proveniência:(Lat. \textunderscore tordylion\textunderscore )}
\end{itemize}
Gênero de plantas umbellíferas.
\section{Toré}
\begin{itemize}
\item {Grp. gram.:m.}
\end{itemize}
\begin{itemize}
\item {Utilização:Bras}
\end{itemize}
Frauta, feita de taboca.
\section{Tórea}
\begin{itemize}
\item {Grp. gram.:f.}
\end{itemize}
Gênero de plantas aristolóquias.
\section{Torém}
\begin{itemize}
\item {Grp. gram.:m.}
\end{itemize}
\begin{itemize}
\item {Utilização:Bras}
\end{itemize}
Instrumento indígena.
O mesmo que \textunderscore toré\textunderscore ?
\section{Torena}
\begin{itemize}
\item {Grp. gram.:m.}
\end{itemize}
\begin{itemize}
\item {Utilização:Bras. do S}
\end{itemize}
Homem elegante, guapo.
\section{Torênia}
\begin{itemize}
\item {Grp. gram.:f.}
\end{itemize}
Espécie de violeta inglesa.
\section{Toreumatografia}
\begin{itemize}
\item {Grp. gram.:f.}
\end{itemize}
Arte de toreumatógrafo.
(Cp. \textunderscore toreumatógrafo\textunderscore )
\section{Toreumatógrafo}
\begin{itemize}
\item {Grp. gram.:m.}
\end{itemize}
\begin{itemize}
\item {Proveniência:(Do gr. \textunderscore toreuma\textunderscore , \textunderscore toreumatos\textunderscore  + \textunderscore graphein\textunderscore )}
\end{itemize}
Aquele que se ocupa dos monumentos esculpidos, especialmente dos antigos baixos-relevos.
\section{Toreumatographia}
\begin{itemize}
\item {Grp. gram.:f.}
\end{itemize}
Arte de toreumatógrapho.
(Cp. \textunderscore toreumatógrapho\textunderscore )
\section{Toreumatógrapho}
\begin{itemize}
\item {Grp. gram.:m.}
\end{itemize}
\begin{itemize}
\item {Proveniência:(Do gr. \textunderscore toreuma\textunderscore , \textunderscore toreumatos\textunderscore  + \textunderscore graphein\textunderscore )}
\end{itemize}
Aquelle que se occupa dos monumentos esculpidos, especialmente dos antigos baixos-relevos.
\section{Toreuta}
\begin{itemize}
\item {Grp. gram.:m.}
\end{itemize}
\begin{itemize}
\item {Proveniência:(Lat. \textunderscore toreuta\textunderscore )}
\end{itemize}
Aquelle que exerce a torêutica. Cf. Latino, \textunderscore Or. da Corôa\textunderscore , CCX.
\section{Torêutica}
\begin{itemize}
\item {Grp. gram.:f.}
\end{itemize}
\begin{itemize}
\item {Proveniência:(Lat. \textunderscore toreutice\textunderscore )}
\end{itemize}
Arte de cinzelar ou esculpir sôbre metaes, madeira ou marfim.
\section{Torêutico}
\begin{itemize}
\item {Grp. gram.:adj.}
\end{itemize}
Relativo á torêutica.
\section{Torga}
\begin{itemize}
\item {Grp. gram.:f.}
\end{itemize}
\begin{itemize}
\item {Utilização:Pop.}
\end{itemize}
\begin{itemize}
\item {Utilização:Fig.}
\end{itemize}
Raíz de urze, de que se faz carvão.
Cabeça grande.
(Talvez do lat. hyp. \textunderscore torica\textunderscore , de \textunderscore torus\textunderscore )
\section{Torgalho}
\begin{itemize}
\item {Grp. gram.:m.}
\end{itemize}
Outra fórma, talvez preferível, de \textunderscore trogalho\textunderscore .
\section{Torgo}
\begin{itemize}
\item {Grp. gram.:m.}
\end{itemize}
O mesmo que \textunderscore torga\textunderscore :«\textunderscore carvão de torgos\textunderscore ». Camillo, \textunderscore Anáthema\textunderscore , 108.
\section{Torgueira}
\begin{itemize}
\item {Grp. gram.:f.}
\end{itemize}
\begin{itemize}
\item {Utilização:Prov.}
\end{itemize}
\begin{itemize}
\item {Utilização:trasm.}
\end{itemize}
\begin{itemize}
\item {Proveniência:(De \textunderscore torga\textunderscore )}
\end{itemize}
Espécie de urze, de cujas raízes se faz carvão.
\section{Tória}
\begin{itemize}
\item {Grp. gram.:adj. f.}
\end{itemize}
\begin{itemize}
\item {Proveniência:(Lat. \textunderscore thoria\textunderscore )}
\end{itemize}
Diz-se de uma lei agrária, de que foi autor o tribuno Thório Balbo, em Roma.
\section{Torianito}
\begin{itemize}
\item {Grp. gram.:m.}
\end{itemize}
Mineral, descoberto há pouco em Ceilão e que é uma fonte importante de rádio.
\section{Toríbios}
\begin{itemize}
\item {Grp. gram.:m. pl.}
\end{itemize}
Avelórios de crystal, procedentes da Índia.
\section{Torilo}
\begin{itemize}
\item {Grp. gram.:m.}
\end{itemize}
\begin{itemize}
\item {Utilização:Bot.}
\end{itemize}
\begin{itemize}
\item {Proveniência:(De \textunderscore tôro\textunderscore )}
\end{itemize}
Ponto, donde nasce a flôr, no pedúnculo.
\section{Torínio}
\begin{itemize}
\item {Grp. gram.:m.}
\end{itemize}
\begin{itemize}
\item {Proveniência:(De \textunderscore Thor\textunderscore , n. p.)}
\end{itemize}
Metal em pó, escuro ou terroso.
\section{Tório}
\begin{itemize}
\item {Grp. gram.:m.}
\end{itemize}
\begin{itemize}
\item {Proveniência:(De \textunderscore Thor\textunderscore , n. p.)}
\end{itemize}
Metal em pó, escuro ou terroso.
\section{Torita}
\begin{itemize}
\item {Grp. gram.:f.}
\end{itemize}
O mesmo que \textunderscore torite\textunderscore .
\section{Torite}
\begin{itemize}
\item {Grp. gram.:f.}
\end{itemize}
Mineral, de que se extraiu o tório.
\section{Torito}
\begin{itemize}
\item {Grp. gram.:m.}
\end{itemize}
O mesmo ou melhor que \textunderscore torite\textunderscore .
\section{Torma}
\begin{itemize}
\item {Grp. gram.:f.}
\end{itemize}
\begin{itemize}
\item {Utilização:Ant.}
\end{itemize}
O mesmo que \textunderscore turma\textunderscore ^1. Cf. \textunderscore Viriat. Trág.\textunderscore , 119.
\section{Tormenta}
\begin{itemize}
\item {Grp. gram.:f.}
\end{itemize}
\begin{itemize}
\item {Utilização:Fig.}
\end{itemize}
\begin{itemize}
\item {Proveniência:(Lat. \textunderscore tormenta\textunderscore )}
\end{itemize}
Tempestade violenta.
Grande barulho; desordem; agitação.
\section{Tormentaria}
\begin{itemize}
\item {Grp. gram.:f.}
\end{itemize}
Designação antiga da artilharia. Cf. C. Aires, \textunderscore Hist. do Exer. Port.\textunderscore 
\section{Tormentelho}
\begin{itemize}
\item {fónica:tê}
\end{itemize}
\begin{itemize}
\item {Grp. gram.:m.}
\end{itemize}
O mesmo que \textunderscore tremontelo\textunderscore .
\section{Tormentila}
\begin{itemize}
\item {Grp. gram.:f.}
\end{itemize}
O mesmo que \textunderscore tormentilha\textunderscore .
(Cp. cast. \textunderscore tormentila\textunderscore )
\section{Tormentilha}
\begin{itemize}
\item {Grp. gram.:f.}
\end{itemize}
O mesmo que \textunderscore sete-em-rama\textunderscore .
\section{Tormentina}
\begin{itemize}
\item {Grp. gram.:f.}
\end{itemize}
Designação vulgar de uma planta medicinal, o mesmo que \textunderscore tormentilha\textunderscore  e \textunderscore terebinthina\textunderscore . Cf. \textunderscore Desengano da Medicina\textunderscore , 246.
(Corr. de \textunderscore terebinthina\textunderscore )
\section{Tormento}
\begin{itemize}
\item {Grp. gram.:m.}
\end{itemize}
\begin{itemize}
\item {Proveniência:(Lat. \textunderscore tormentum\textunderscore )}
\end{itemize}
Acto ou effeito de atormentar.
Afflicção.
Tortura.
Desgraça.
\section{Tormentório}
\begin{itemize}
\item {Grp. gram.:f.}
\end{itemize}
Relativo a tormenta; em que há tormentas: \textunderscore passou o Cabo tormentório\textunderscore .
\section{Tormentoso}
\begin{itemize}
\item {Grp. gram.:adj.}
\end{itemize}
\begin{itemize}
\item {Utilização:Fig.}
\end{itemize}
Relativo a tormenta.
Tormentório.
Que causa tormento; trabalhoso.
\section{Torna}
\begin{itemize}
\item {Grp. gram.:f.}
\end{itemize}
\begin{itemize}
\item {Utilização:Prov.}
\end{itemize}
\begin{itemize}
\item {Utilização:trasm.}
\end{itemize}
\begin{itemize}
\item {Proveniência:(De \textunderscore tornar\textunderscore )}
\end{itemize}
Aquillo que, além do objecto que se troca por outro, se dá para igualar o valor dêste.
Compensação, que um co-herdeiro, mais favorecido na partilha, dá a outro ou outros, para igualar os quinhões.
Nesga de terreno.
\section{Torna}
\begin{itemize}
\item {Grp. gram.:f.}
\end{itemize}
Espécie de pállio indiano. Cf. Th. Ribeiro, \textunderscore Jornadas\textunderscore , II, 113.
\section{Tornaboda}
\begin{itemize}
\item {fónica:bô}
\end{itemize}
\begin{itemize}
\item {Grp. gram.:f.}
\end{itemize}
\begin{itemize}
\item {Proveniência:(De \textunderscore tornar\textunderscore  + \textunderscore boda\textunderscore )}
\end{itemize}
Segunda celebração de uma boda de núpcias.
Festa, no dia immediato ao das núpcias.
\section{Tornada}
\begin{itemize}
\item {Grp. gram.:f.}
\end{itemize}
Acto ou effeito de tornar.
Volta, regresso:«\textunderscore quando já vinha de tornada...\textunderscore »Filinto, \textunderscore D. Man.\textunderscore , III, 249.
Banco de areia, no fim dos cabedelos.
\section{Tornada}
\begin{itemize}
\item {Grp. gram.:f.}
\end{itemize}
\begin{itemize}
\item {Proveniência:(De \textunderscore tôrno\textunderscore )}
\end{itemize}
Líquido, que sái de uma vasilha, tirada a chave da torneira.
\section{Tornadiço}
\begin{itemize}
\item {Grp. gram.:adj.}
\end{itemize}
\begin{itemize}
\item {Proveniência:(De \textunderscore tornar\textunderscore )}
\end{itemize}
Apóstata.
Renegado.
Desertor.
Que volta ao gremio ou religião donde tinha saído.
Que volta.
\section{Tornado}
\begin{itemize}
\item {Grp. gram.:m.}
\end{itemize}
Aguaceiro forte, com grandes ondas, vento e trovoada: \textunderscore na serra Leôa a época dos tornados violentos termina em fins de maio\textunderscore . Cf. Fil. Simões, \textunderscore C. da Beiramar\textunderscore , 89.
\section{Tornadoira}
\begin{itemize}
\item {Grp. gram.:f.}
\end{itemize}
\begin{itemize}
\item {Utilização:T. de Arganil}
\end{itemize}
O mesmo que \textunderscore tornadura\textunderscore .
Forcado de pau, para levantar a palha nas eiras e separar o grão.
\section{Tornadoiro}
\begin{itemize}
\item {Grp. gram.:m.}
\end{itemize}
\begin{itemize}
\item {Utilização:Prov.}
\end{itemize}
\begin{itemize}
\item {Utilização:minh.}
\end{itemize}
\begin{itemize}
\item {Proveniência:(De \textunderscore tornar\textunderscore )}
\end{itemize}
Pegadoira, na retaguarda da grade, para a guiar e ajudá-la a dar volta no fim do campo.
\section{Tornador}
\begin{itemize}
\item {Grp. gram.:m.  e  adj.}
\end{itemize}
O mesmo que \textunderscore torneador\textunderscore .
\section{Tornadoura}
\begin{itemize}
\item {Grp. gram.:f.}
\end{itemize}
\begin{itemize}
\item {Utilização:T. de Arganil}
\end{itemize}
O mesmo que \textunderscore tornadura\textunderscore .
Forcado de pau, para levantar a palha nas eiras e separar o grão.
\section{Tornadouro}
\begin{itemize}
\item {Grp. gram.:m.}
\end{itemize}
\begin{itemize}
\item {Utilização:Prov.}
\end{itemize}
\begin{itemize}
\item {Utilização:minh.}
\end{itemize}
\begin{itemize}
\item {Proveniência:(De \textunderscore tornar\textunderscore )}
\end{itemize}
Pegadoira, na retaguarda da grade, para a guiar e ajudá-la a dar volta no fim do campo.
\section{Tornadura}
\begin{itemize}
\item {Grp. gram.:f.}
\end{itemize}
\begin{itemize}
\item {Proveniência:(De \textunderscore tornar\textunderscore )}
\end{itemize}
Instrumento, para torcer vimes e arcos.
\section{Torna-fio}
\begin{itemize}
\item {Grp. gram.:m.}
\end{itemize}
Peça de ferro, em que os penteeiros afiam as ferramentas.
\section{Tornamento}
\begin{itemize}
\item {Grp. gram.:m.}
\end{itemize}
\begin{itemize}
\item {Utilização:Des.}
\end{itemize}
O mesmo que \textunderscore tornada\textunderscore ^2.
\section{Tornar}
\begin{itemize}
\item {Grp. gram.:v. t. ,  i.  e  p.}
\end{itemize}
\begin{itemize}
\item {Grp. gram.:V. t.}
\end{itemize}
\begin{itemize}
\item {Utilização:Prov.}
\end{itemize}
\begin{itemize}
\item {Utilização:Prov.}
\end{itemize}
\begin{itemize}
\item {Utilização:minh.}
\end{itemize}
\begin{itemize}
\item {Utilização:Prov.}
\end{itemize}
\begin{itemize}
\item {Utilização:minh.}
\end{itemize}
\begin{itemize}
\item {Grp. gram.:V. i.}
\end{itemize}
\begin{itemize}
\item {Grp. gram.:V. p.}
\end{itemize}
\begin{itemize}
\item {Grp. gram.:Loc.}
\end{itemize}
\begin{itemize}
\item {Utilização:Des.}
\end{itemize}
\begin{itemize}
\item {Proveniência:(Lat. \textunderscore tornare\textunderscore )}
\end{itemize}
Voltar, virar, volver.
Fazer, transformar em.
Retirar (o gado) donde faz damno.
Desviar (a água) de um terreno para outro.
Traduzir.
Restituir.
Reconduzir.
Responder, replicar: \textunderscore não te acredito,--respondi eu\textunderscore .
Acercar-se de, tolher o passo a.
Mudar de intento.
Tomar defesa.
Regressar.
Voltar ao ponto de partida.
Recorrer: \textunderscore já não tinha a que se tornar\textunderscore .
Transformar-se, converter-se: \textunderscore o amo tornou-se em servo\textunderscore .
\textunderscore Tornar mão\textunderscore , defender-se á mão armada; resistir á justiça.
\section{Tornasol}
\begin{itemize}
\item {fónica:sol}
\end{itemize}
\begin{itemize}
\item {Grp. gram.:m.}
\end{itemize}
\begin{itemize}
\item {Proveniência:(De \textunderscore tornar\textunderscore  + \textunderscore sol\textunderscore )}
\end{itemize}
Planta, o mesmo que \textunderscore girasol\textunderscore .
Heliotrópio.
\section{Tornassol}
\begin{itemize}
\item {Grp. gram.:m.}
\end{itemize}
\begin{itemize}
\item {Proveniência:(De \textunderscore tornar\textunderscore  + \textunderscore sol\textunderscore )}
\end{itemize}
Planta, o mesmo que \textunderscore girasol\textunderscore .
Heliotrópio.
\section{Torna-torna}
\begin{itemize}
\item {Grp. gram.:f.}
\end{itemize}
\begin{itemize}
\item {Utilização:Prov.}
\end{itemize}
\begin{itemize}
\item {Utilização:minh.}
\end{itemize}
\textunderscore Água de torna-torna\textunderscore , a que cada lavrador póde guiar para o seu campo, antes de chegar a época em que ella se reparte. Cf. Assis, \textunderscore Águas\textunderscore , 193.
\section{Torna-tornarás}
\begin{itemize}
\item {Grp. gram.:f.}
\end{itemize}
O mesmo que \textunderscore torna-torna\textunderscore . Cf. Assis, \textunderscore Águas\textunderscore , 193.
\section{Torna-viagem}
\begin{itemize}
\item {Grp. gram.:f.}
\end{itemize}
\begin{itemize}
\item {Utilização:Fig.}
\end{itemize}
Volta de uma viagem por mar.
Regresso.
Refúgio.
Resto.
\section{Torneador}
\begin{itemize}
\item {Grp. gram.:m.  e  adj.}
\end{itemize}
\begin{itemize}
\item {Grp. gram.:M.}
\end{itemize}
\begin{itemize}
\item {Proveniência:(De \textunderscore tornear\textunderscore )}
\end{itemize}
O que torneia.
Banco em que se fazem as rodas das seges.
Instrumento com que os espingardeiros abrem as escorvas.
\section{Torneamento}
\begin{itemize}
\item {Grp. gram.:m.}
\end{itemize}
Acto ou effeito de tornear.
\section{Torneante}
\begin{itemize}
\item {Grp. gram.:adj.}
\end{itemize}
Que torneia. Cf. Castilho, \textunderscore Fastos\textunderscore , III, 131.
\section{Tornear}
\begin{itemize}
\item {Grp. gram.:v. t.}
\end{itemize}
Fabricar com o tôrno; desbastar ou lavrar ao tôrno: \textunderscore tornear piões\textunderscore .
Arredondar; tornar roliço.
Circundar.
\section{Tornear}
\begin{itemize}
\item {Grp. gram.:v. i.}
\end{itemize}
Andar em torneio ou justa.
Fazer exercícios de torneio.
\section{Tornearia}
\begin{itemize}
\item {Grp. gram.:f.}
\end{itemize}
\begin{itemize}
\item {Proveniência:(De \textunderscore tornear\textunderscore )}
\end{itemize}
Arte ou officina de torneiro.
\section{Torneável}
\begin{itemize}
\item {Grp. gram.:adj.}
\end{itemize}
Que se póde tornear^1.
\section{Torneio}
\begin{itemize}
\item {Grp. gram.:m.}
\end{itemize}
Acto ou effeito de tornear^1.
Flexibilidade ou elegância de fórmas.
Elegância de phrase.
\section{Torneio}
\begin{itemize}
\item {Grp. gram.:m.}
\end{itemize}
\begin{itemize}
\item {Utilização:Fig.}
\end{itemize}
\begin{itemize}
\item {Proveniência:(Do fr. \textunderscore tournoi\textunderscore )}
\end{itemize}
Jogos públicos de cavalleiros, na Idade-Média.
Justa.
Discussão; polêmica.
\section{Torneira}
\begin{itemize}
\item {Grp. gram.:f.}
\end{itemize}
\begin{itemize}
\item {Proveniência:(De \textunderscore tôrno\textunderscore )}
\end{itemize}
Tubo, com uma espécie de chave, que se adapta a qualquer vasilha, para della se extrahir o líquido que contém, quando se queira.
Casta de uva de Cascaes. Cf. \textunderscore Rev. Agron.\textunderscore , I, 18.
\section{Torneiro}
\begin{itemize}
\item {Grp. gram.:m.}
\end{itemize}
\begin{itemize}
\item {Utilização:T. de Turquel}
\end{itemize}
\begin{itemize}
\item {Proveniência:(De \textunderscore tôrno\textunderscore )}
\end{itemize}
Aquelle que trabalha ao tôrno.
Aquelle que torneia.
Casta de uva.
Indivíduo teimoso, renitente.
\section{Torneja}
\begin{itemize}
\item {Grp. gram.:f.}
\end{itemize}
Cada uma das cavilhas que, na extremidade do eixo do carro, impedem que as rodas sáiam ou que saia a maça, onde entram os raios das rodas.--Os diccionaristas portugueses, desde Bento Pereira e Moraes até os modernos, dão de \textunderscore torneja\textunderscore  uma definição que supponho errada.
\section{Tornejamento}
\begin{itemize}
\item {Grp. gram.:m.}
\end{itemize}
Acto ou effeito de tornejar.
\section{Tornejão}
\begin{itemize}
\item {Grp. gram.:m.}
\end{itemize}
\begin{itemize}
\item {Proveniência:(De \textunderscore torneja\textunderscore )}
\end{itemize}
Peça de ferro, que atravessa verticalmente os limões e eixo do carro alentejano ligando o eixo ao leito.
\section{Tornejar}
\begin{itemize}
\item {Grp. gram.:v. t.}
\end{itemize}
\begin{itemize}
\item {Grp. gram.:V. i.}
\end{itemize}
\begin{itemize}
\item {Proveniência:(De \textunderscore tôrno\textunderscore )}
\end{itemize}
Encurvar.
Dar volta a, andar á roda de: \textunderscore tornejar uma praça\textunderscore .
Encurvar-se, recurvar-se.
Dar volta.
\section{Tornel}
\begin{itemize}
\item {Grp. gram.:m.}
\end{itemize}
\begin{itemize}
\item {Proveniência:(De \textunderscore tôrno\textunderscore )}
\end{itemize}
Argola, cravada na extremidade de uma haste, sôbre a qual se revolve para todos os lados.
Cada uma das duas peças móveis de madeira, que atravessam a extremidade das testeiras de uma serra, e nas quaes se fixam os extremos da lâmina da mesma serra.
\section{Tornês}
\begin{itemize}
\item {Grp. gram.:m.}
\end{itemize}
\begin{itemize}
\item {Proveniência:(Do fr. \textunderscore tournois\textunderscore )}
\end{itemize}
Moéda antiga de prata.
\section{Tornete}
\begin{itemize}
\item {fónica:nê}
\end{itemize}
\begin{itemize}
\item {Grp. gram.:m.}
\end{itemize}
Pequeno tôrno.
\section{Tornilheiro}
\begin{itemize}
\item {Grp. gram.:adj.}
\end{itemize}
\begin{itemize}
\item {Proveniência:(De \textunderscore tornar\textunderscore )}
\end{itemize}
Que deixa o regimento, voltando para casa, (falando-se do soldado).
\section{Tornilho}
\begin{itemize}
\item {Grp. gram.:m.}
\end{itemize}
\begin{itemize}
\item {Utilização:Fig.}
\end{itemize}
\begin{itemize}
\item {Proveniência:(De \textunderscore tôrno\textunderscore )}
\end{itemize}
Castigo que se dava aos militares, apertando-lhes uma espingarda sôbre o pescoço e outra nas curvas das pernas, o que os obrigava a curvar-se, difficultando-lhes os movimentos.
Pequeno tôrno.
Lance apertado; apertos.
\section{Torninho}
\begin{itemize}
\item {Grp. gram.:m.}
\end{itemize}
\begin{itemize}
\item {Proveniência:(De \textunderscore tôrno\textunderscore )}
\end{itemize}
Tôrno pequeno, em que os serralheiros ou ferreiros apertam as peças que querem limar.
\section{Torninho}
\begin{itemize}
\item {Grp. gram.:m.}
\end{itemize}
\begin{itemize}
\item {Utilização:T. da Bairrada}
\end{itemize}
O mesmo que \textunderscore estorninho\textunderscore .
\section{Torniquete}
\begin{itemize}
\item {fónica:quê}
\end{itemize}
\begin{itemize}
\item {Grp. gram.:m.}
\end{itemize}
\begin{itemize}
\item {Proveniência:(Fr. \textunderscore tourniquet\textunderscore )}
\end{itemize}
Cruz móvel, collocada horizontalmente á entreda de uma rua ou estrada, para só deixar passar peões.
Apparelho, para demonstrar a reacção causada pelo escoamento dos gases.
Trapézio fixo.
Tôrno.
Instrumento cirúrgico, para comprimir as artérias, suspendendo uma hemorragia.
Antigo instrumento de tortura inquisitorial.
\section{Tôrno}
\begin{itemize}
\item {Grp. gram.:m.}
\end{itemize}
\begin{itemize}
\item {Utilização:Ant.}
\end{itemize}
\begin{itemize}
\item {Utilização:Prov.}
\end{itemize}
\begin{itemize}
\item {Utilização:dur.}
\end{itemize}
\begin{itemize}
\item {Proveniência:(Lat. \textunderscore tornus\textunderscore )}
\end{itemize}
Engenho, em que se faz girar uma peça de madeira, marfim, etc., que se quere lavrar ou arredondar.
Torninho.
Chave da torneira.
A Roda, nos conventos.
Volta.
Prego de madeira.
Pua.
Cavilha.
Borbotão, jôrro.
O mesmo que \textunderscore espera\textunderscore ^1 e \textunderscore talão\textunderscore  nas videiras; cp. \textunderscore espera\textunderscore .
\section{Tórno}
\begin{itemize}
\item {Grp. gram.:m.}
\end{itemize}
Acto de tornar. Cf. Lapa, \textunderscore Proc. de Vin.\textunderscore , 99.
\section{Tornozelo}
\begin{itemize}
\item {fónica:zê}
\end{itemize}
\begin{itemize}
\item {Grp. gram.:m.}
\end{itemize}
\begin{itemize}
\item {Proveniência:(De \textunderscore tôrno\textunderscore )}
\end{itemize}
Saliêncea óssea, na articulação do pé com a perna; artelho.
\section{Toro}
\begin{itemize}
\item {Grp. gram.:m.}
\end{itemize}
\begin{itemize}
\item {Utilização:Poét.}
\end{itemize}
\begin{itemize}
\item {Proveniência:(Lat. \textunderscore thorum\textunderscore )}
\end{itemize}
O leito conjugal.
\section{Tôro}
\begin{itemize}
\item {Grp. gram.:m.}
\end{itemize}
\begin{itemize}
\item {Utilização:Ant.}
\end{itemize}
\begin{itemize}
\item {Proveniência:(Lat. \textunderscore torus\textunderscore )}
\end{itemize}
Tronco de uma árvore, sem rama.
Cepo.
Tronco do corpo.
Fragmento de um braço de árvore.
Moldura circular, na base da columna.
Pedaço de cabo náutico, para se desfiar em fio de carreta.
Receptáculo de alguns frutos.
Leito.
\section{Tóro}
\begin{itemize}
\item {Grp. gram.:m.}
\end{itemize}
\begin{itemize}
\item {Utilização:Ant.}
\end{itemize}
\begin{itemize}
\item {Proveniência:(Lat. \textunderscore torus\textunderscore )}
\end{itemize}
Tronco de uma árvore, sem rama.
Cepo.
Tronco do corpo.
Fragmento de um braço de árvore.
Moldura circular, na base da columna.
Pedaço de cabo náutico, para se desfiar em fio de carreta.
Receptáculo de alguns frutos.
Leito.
\section{Toró}
\begin{itemize}
\item {Grp. gram.:adj.}
\end{itemize}
\begin{itemize}
\item {Utilização:Bras}
\end{itemize}
Diz-se da pessôa, que perdeu a phalange de qualquer dedo da mão.
(Talvez de \textunderscore torar\textunderscore )
\section{Toró}
\begin{itemize}
\item {Grp. gram.:m.}
\end{itemize}
\begin{itemize}
\item {Utilização:Bras}
\end{itemize}
Ave gallinácea das regiões do Amazonas, espécie de inambu.
\section{Tó-rôla!}
\begin{itemize}
\item {Grp. gram.:interj.}
\end{itemize}
Isso sim! não me enganas! Cf. Eça, \textunderscore P. Basílio\textunderscore , 221.
\section{Toronja}
\begin{itemize}
\item {Grp. gram.:f.}
\end{itemize}
\begin{itemize}
\item {Proveniência:(Do ár. \textunderscore toronja\textunderscore )}
\end{itemize}
Variedade de laranja pouco doce, ou antes, variedade de limão, (\textunderscore citrus decumana\textunderscore , Lin.).
\section{Tororoma}
\begin{itemize}
\item {Grp. gram.:f.}
\end{itemize}
\begin{itemize}
\item {Utilização:Bras. do N}
\end{itemize}
Corrente fluvial, ruidosa e forte.
\section{Toroso}
\begin{itemize}
\item {Grp. gram.:adj.}
\end{itemize}
\begin{itemize}
\item {Proveniência:(Lat. \textunderscore torosus\textunderscore )}
\end{itemize}
Vigoroso.
Polpudo, carnudo.
\section{Tôrpe}
\begin{itemize}
\item {Grp. gram.:adj.}
\end{itemize}
\begin{itemize}
\item {Proveniência:(Do lat. \textunderscore turpis\textunderscore )}
\end{itemize}
Disforme.
Repugnante.
Deshonesto; infame; ignóbil.
Nojento.
Manchado.
\section{Tôrpe}
\begin{itemize}
\item {Grp. gram.:adj.}
\end{itemize}
\begin{itemize}
\item {Proveniência:(Do lat. \textunderscore torpidus\textunderscore )}
\end{itemize}
Que entorpece; entorpecido.
Embaraçado, acanhado.
\section{Tórpe}
\begin{itemize}
\item {Grp. gram.:adj.}
\end{itemize}
\begin{itemize}
\item {Utilização:Prov.}
\end{itemize}
\begin{itemize}
\item {Utilização:trasm.}
\end{itemize}
O mesmo que \textunderscore tôrpe\textunderscore ^2.
\section{Torpecer}
\begin{itemize}
\item {Grp. gram.:v. i.}
\end{itemize}
O mesmo que [[entorpecer-se|entorpecer]].
\section{Torpecido}
\begin{itemize}
\item {Grp. gram.:adj.}
\end{itemize}
O mesmo que [[entorpecido|entorpecer]]. Cf. Filinto, VI, 263.
\section{Torpedeira}
\begin{itemize}
\item {Grp. gram.:f.}
\end{itemize}
O mesmo que \textunderscore torpedeiro\textunderscore .
\section{Torpedeiro}
\begin{itemize}
\item {Grp. gram.:m.}
\end{itemize}
Barco com torpedo, para uso de guerra marítima.
Collocador de torpedos.
\section{Torpedo}
\begin{itemize}
\item {fónica:pê}
\end{itemize}
\begin{itemize}
\item {Grp. gram.:m.}
\end{itemize}
\begin{itemize}
\item {Proveniência:(Lat. \textunderscore torpedo\textunderscore )}
\end{itemize}
Gênero de peixes cartilaginosos, cuja cauda produz commoção eléctrica em quem a toca.
Máquina de guerra, que serve debaixo da água, e faz explosão pelo choque de um navio ou de outro apparelho.
\section{Torpemente}
\begin{itemize}
\item {Grp. gram.:adj.}
\end{itemize}
De modo torpe; ignobilmente.
\section{Torpente}
\begin{itemize}
\item {Grp. gram.:adj.}
\end{itemize}
\begin{itemize}
\item {Proveniência:(Lat. \textunderscore torpens\textunderscore )}
\end{itemize}
Que entorpece.
O mesmo que [[entorpecido|entorpecer]]:«\textunderscore mal que a torpente cobra o calor sente...\textunderscore »Filinto, XII, 285.
\section{Torpeza}
\begin{itemize}
\item {Grp. gram.:f.}
\end{itemize}
Qualidade do que é tôrpe^1.
Procedimento índigno ou ignóbil.
Desvergonhada; deshonestidade.
\section{Torpidade}
\begin{itemize}
\item {Grp. gram.:f.}
\end{itemize}
(V.torpeza)
\section{Tórpido}
\begin{itemize}
\item {Grp. gram.:adj.}
\end{itemize}
\begin{itemize}
\item {Proveniência:(Lat. \textunderscore torpidus\textunderscore )}
\end{itemize}
Entorpecido.
\section{Torpilha}
\begin{itemize}
\item {Grp. gram.:f.}
\end{itemize}
\begin{itemize}
\item {Proveniência:(Fr. \textunderscore torpille\textunderscore )}
\end{itemize}
Apparelho para enxofrar vinhas, composto de um depósito de enxôfre e de um folle que expulsa o mesmo enxôfre por meio de uma agulheta.
\section{Torpitude}
\begin{itemize}
\item {Grp. gram.:f.}
\end{itemize}
O mesmo que \textunderscore torpeza\textunderscore .
\section{Torpor}
\begin{itemize}
\item {Grp. gram.:m.}
\end{itemize}
\begin{itemize}
\item {Proveniência:(Lat. \textunderscore torpor\textunderscore )}
\end{itemize}
Entorpecimento.
Indifferênça ou inércia moral.
\section{Torquaz}
\begin{itemize}
\item {Grp. gram.:m.  e  adj.}
\end{itemize}
(V.torcaz)
\section{Torquez}
\begin{itemize}
\item {Grp. gram.:f.}
\end{itemize}
\begin{itemize}
\item {Proveniência:(Do lat. \textunderscore torquere\textunderscore )}
\end{itemize}
Instrumento de ferro ou de outros metaes, formado de duas peças, á maneira de tenaz ou de tesoira, e com as quaes se póde apertar e arrancar um objecto.
\section{Torquezada}
\begin{itemize}
\item {Grp. gram.:f.}
\end{itemize}
\begin{itemize}
\item {Utilização:Ext.}
\end{itemize}
Pancada com torquez.
Pancada.
\section{Torquisco}
\begin{itemize}
\item {Grp. gram.:m.}
\end{itemize}
Espécie de arma antiga. Cf. Tenreiro, \textunderscore Itiner.\textunderscore , (ed. de 1829), 21.
\section{Torra}
\begin{itemize}
\item {Grp. gram.:f.}
\end{itemize}
Acto ou effeito de torrar.
\section{Torração}
\begin{itemize}
\item {Grp. gram.:f.}
\end{itemize}
Acto de torrar: \textunderscore a torração do café\textunderscore .
\section{Torrada}
\begin{itemize}
\item {Grp. gram.:f.}
\end{itemize}
\begin{itemize}
\item {Proveniência:(De \textunderscore torrado\textunderscore )}
\end{itemize}
Fatia de pão torrado.
\section{Torrado}
\begin{itemize}
\item {Grp. gram.:adj.}
\end{itemize}
\begin{itemize}
\item {Grp. gram.:M.}
\end{itemize}
\begin{itemize}
\item {Utilização:Bras. do N}
\end{itemize}
\begin{itemize}
\item {Proveniência:(De \textunderscore torrar\textunderscore )}
\end{itemize}
Que se torrou: \textunderscore fatias torradas\textunderscore .
Murcho; sêco.
Diz-se do toiro retinto, que tem o pêlo negro, desde meio corpo para baixo.
O mesmo que \textunderscore tórrido\textunderscore :«\textunderscore zona torrada.\textunderscore »\textunderscore Eufrosina\textunderscore , 252.
O mesmo que \textunderscore rapé\textunderscore .
\section{Torrador}
\begin{itemize}
\item {Grp. gram.:m.}
\end{itemize}
Apparelho para torrar café.
\section{Torragem}
\begin{itemize}
\item {Grp. gram.:f.}
\end{itemize}
O mesmo que \textunderscore torração\textunderscore .
\section{Torrão}
\begin{itemize}
\item {Grp. gram.:m.}
\end{itemize}
\begin{itemize}
\item {Utilização:Ext.}
\end{itemize}
\begin{itemize}
\item {Grp. gram.:Pl.}
\end{itemize}
Pedaço de terra, de fórma análoga á de um rebo.
Gleba.
Terreno.
Território.
Pedaço, bocado, fragmento: \textunderscore torrão de açúcar\textunderscore .
Solo.
Propriedades rústicas: \textunderscore possuidor de alguns torrões\textunderscore .
(Por \textunderscore terrão\textunderscore , de \textunderscore terra\textunderscore )
\section{Torrão-de-açúcar}
\begin{itemize}
\item {Grp. gram.:f.}
\end{itemize}
Variedade de pêra.
\section{Torrar}
\begin{itemize}
\item {Grp. gram.:v. t.}
\end{itemize}
\begin{itemize}
\item {Utilização:Bras}
\end{itemize}
\begin{itemize}
\item {Proveniência:(Do lat. \textunderscore torrere\textunderscore )}
\end{itemize}
Tornar muito sêco, por meio do calor: \textunderscore torrar café\textunderscore .
Tostar.
Tornar murcho.
Vender pelo mais baixo preço ou por todo o preço: \textunderscore o leiloeiro torrou tudo quanto vendeu\textunderscore .
\section{Tôrre}
\begin{itemize}
\item {Grp. gram.:f.}
\end{itemize}
\begin{itemize}
\item {Utilização:Fig.}
\end{itemize}
\begin{itemize}
\item {Proveniência:(Do lat. \textunderscore turris\textunderscore )}
\end{itemize}
Edifício alto, que se construía especialmente para defesa em caso de guerra.
Fortaleza.
Construcção redonda ou de várias faces, geralmente estreita e alta, insulada ou annexa a uma igreja ou a outro edifício, e que serve para têr os sinos ou para communicar certos sinaes a distância.
Campanário.
Peça do jôgo do xadrez.
Pessôa muito corpulenta.
\section{Torreano}
\begin{itemize}
\item {Grp. gram.:m.  e  adj.}
\end{itemize}
(V.torresão)
\section{Torreante}
\begin{itemize}
\item {Grp. gram.:adj.}
\end{itemize}
\begin{itemize}
\item {Proveniência:(De \textunderscore torrear\textunderscore )}
\end{itemize}
Que se eleva como uma tôrre.
\section{Torreão}
\begin{itemize}
\item {Grp. gram.:m.}
\end{itemize}
Tôrre larga e com ameias, sôbre um castello.
Espécie de tôrre, pavilhão ou eirado, no ângulo ou no alto de um edifício.
\section{Torrear}
\begin{itemize}
\item {Grp. gram.:v. t.}
\end{itemize}
\begin{itemize}
\item {Grp. gram.:V. i.}
\end{itemize}
Fortificar com tôrres.
Elevar-se á maneira de tôrre.
\section{Torrefacção}
\begin{itemize}
\item {Grp. gram.:f.}
\end{itemize}
\begin{itemize}
\item {Proveniência:(Do lat. \textunderscore torrefactus\textunderscore )}
\end{itemize}
Acto ou effeito de torrificar.
\section{Torrefacto}
\begin{itemize}
\item {Grp. gram.:adj.}
\end{itemize}
\begin{itemize}
\item {Proveniência:(Lat. \textunderscore torrefactus\textunderscore )}
\end{itemize}
Que se torrificou.
Torrado.
\section{Torrefactor}
\begin{itemize}
\item {Grp. gram.:adj.}
\end{itemize}
\begin{itemize}
\item {Grp. gram.:M.}
\end{itemize}
Que torrefaz.
Apparelho, para torrefazer. Cf. \textunderscore Inquér. Industr.\textunderscore , p. II, l. III, 319.
\section{Torrefeito}
\begin{itemize}
\item {Grp. gram.:adj.}
\end{itemize}
O mesmo que \textunderscore torrefacto\textunderscore .
\section{Torrefazer}
\begin{itemize}
\item {Grp. gram.:v. t.}
\end{itemize}
O mesmo que \textunderscore torrificar\textunderscore .
\section{Torreira}
\begin{itemize}
\item {Grp. gram.:v. t.}
\end{itemize}
\begin{itemize}
\item {Proveniência:(De \textunderscore torrar\textunderscore )}
\end{itemize}
Calor excessivo.
O pino da calma.
Lugar, onde é mais intenso o calor do sol; soalheira.
\section{Torrejano}
\begin{itemize}
\item {Grp. gram.:adj.}
\end{itemize}
\begin{itemize}
\item {Grp. gram.:M.}
\end{itemize}
Relativo a Tôrres-Novas.
Habitante de Tôrres-Novas.
\section{Torrejar}
\begin{itemize}
\item {Grp. gram.:v. t.}
\end{itemize}
O mesmo que \textunderscore torrear\textunderscore .
\section{Torrelhas}
\begin{itemize}
\item {fónica:rê}
\end{itemize}
\begin{itemize}
\item {Grp. gram.:f. pl.}
\end{itemize}
\begin{itemize}
\item {Proveniência:(De \textunderscore tôrre\textunderscore )}
\end{itemize}
Jôgo vulgar antigo, prohibido por D. Affonso V.--O \textunderscore Elucidário\textunderscore  de Viterbo diz \textunderscore torrélhas\textunderscore .
\section{Torrencial}
\begin{itemize}
\item {Grp. gram.:adj.}
\end{itemize}
Relativo a torrente.
Caudaloso; semelhante a torrente.
\section{Torrencialmente}
\begin{itemize}
\item {Grp. gram.:adv.}
\end{itemize}
De modo torrencial.
Em grande abundância; caudalosamente.
\section{Torrente}
\begin{itemize}
\item {Grp. gram.:f.}
\end{itemize}
\begin{itemize}
\item {Proveniência:(Lat. \textunderscore torrens\textunderscore )}
\end{itemize}
Corrente de água, muito rápida e impetuosa.
Grande abundância; grande fluência: \textunderscore torrente de eloquência\textunderscore .
Multidão, que se precipita impetuosamente.
Influência do exemplo ou da moda.
O maior número, a maioria.
Fôrça das coisas.--É voc. masculino em escritos antigos. Cf. Pant. de Aveiro, \textunderscore Itiner.\textunderscore , 40, (2.^a ed.)
\section{Torrentoso}
\begin{itemize}
\item {Grp. gram.:adj.}
\end{itemize}
O mesmo que \textunderscore torrencial\textunderscore .
\section{Torresão}
\begin{itemize}
\item {Grp. gram.:adj.}
\end{itemize}
\begin{itemize}
\item {Grp. gram.:M.}
\end{itemize}
\begin{itemize}
\item {Proveniência:(De \textunderscore Tôrres\textunderscore , n. p.)}
\end{itemize}
Relativo a Tôrres-Vedras.
Habitante de Tôrres-Vedras.
\section{Torresmada}
\begin{itemize}
\item {Grp. gram.:f.}
\end{itemize}
\begin{itemize}
\item {Utilização:T. do Faial}
\end{itemize}
Parvoíce; disparate.
\section{Torresmo}
\begin{itemize}
\item {fónica:rês}
\end{itemize}
\begin{itemize}
\item {Grp. gram.:m.}
\end{itemize}
\begin{itemize}
\item {Utilização:Ext.}
\end{itemize}
\begin{itemize}
\item {Proveniência:(De \textunderscore torrar\textunderscore )}
\end{itemize}
Parte consistente e membranosa, que resta do toicinho frito.
Cada um dos pedaços do sarrabulho.
\section{Tórrido}
\begin{itemize}
\item {Grp. gram.:adj.}
\end{itemize}
\begin{itemize}
\item {Proveniência:(Lat. \textunderscore torridus\textunderscore )}
\end{itemize}
Muito quente, ardente.
Em Geographia, diz-se da zona, comprehendida entre os Trópicos.
\section{Torrificação}
\begin{itemize}
\item {Grp. gram.:f.}
\end{itemize}
Acto de torrificar. Cf. Castilho, \textunderscore Fastos\textunderscore , III, 476.
\section{Torrificado}
\begin{itemize}
\item {Grp. gram.:adj.}
\end{itemize}
\begin{itemize}
\item {Proveniência:(De \textunderscore torrificar\textunderscore )}
\end{itemize}
O mesmo que \textunderscore torrefacto\textunderscore .
\section{Torrificador}
\begin{itemize}
\item {Grp. gram.:adj.}
\end{itemize}
Que torrifica.
\section{Torrificar}
\begin{itemize}
\item {Grp. gram.:v. t.}
\end{itemize}
\begin{itemize}
\item {Proveniência:(Do lat. \textunderscore torrefacere\textunderscore )}
\end{itemize}
Tornar tórrido; tostar; fazer torrar.
\section{Torrija}
\begin{itemize}
\item {Grp. gram.:f.}
\end{itemize}
Fatia torrada, embebida em vinho e coberta de ovos batidos e açúcar.
(Cast. \textunderscore torrija\textunderscore )
\section{Torrinha}
\begin{itemize}
\item {Grp. gram.:f.}
\end{itemize}
Pequena tôrre.
Camarote ou galeria da última ordem, nos theatros.
\section{Torrinheira}
\begin{itemize}
\item {Grp. gram.:f.}
\end{itemize}
\begin{itemize}
\item {Utilização:Prov.}
\end{itemize}
\begin{itemize}
\item {Utilização:trasm.}
\end{itemize}
\begin{itemize}
\item {Proveniência:(De \textunderscore torrinha\textunderscore )}
\end{itemize}
Pequeno monte de pedras soltas.
\section{Torriqueiro}
\begin{itemize}
\item {Grp. gram.:m.}
\end{itemize}
\begin{itemize}
\item {Utilização:Prov.}
\end{itemize}
\begin{itemize}
\item {Utilização:beir.}
\end{itemize}
O mesmo que \textunderscore torreira\textunderscore .
\section{Torriscado}
\begin{itemize}
\item {Grp. gram.:adj.}
\end{itemize}
\begin{itemize}
\item {Proveniência:(De \textunderscore torrar\textunderscore )}
\end{itemize}
Excessivamente torrado, (falando-se da fatias de pão):«\textunderscore muito torriscadas, davam rangidos numa trincadeira voluptuosa\textunderscore ». Camillo, \textunderscore Brasileira\textunderscore , 123.
\section{Torroada}
\begin{itemize}
\item {Grp. gram.:f.}
\end{itemize}
\begin{itemize}
\item {Utilização:Bras. do Maranhão}
\end{itemize}
\begin{itemize}
\item {Proveniência:(De \textunderscore torrão\textunderscore )}
\end{itemize}
Porção de torrões.
Pancada com torrão.
Fenda nos terrenos argillosos, depois de sêcos.
\section{Torroeira}
\begin{itemize}
\item {Grp. gram.:f.}
\end{itemize}
\begin{itemize}
\item {Proveniência:(De \textunderscore torrão\textunderscore )}
\end{itemize}
Parte da praia, donde se tira o torrão para construir a vedação das marinhas. Cf. \textunderscore Museu Techn.\textunderscore , 55.
\section{Torso}
\begin{itemize}
\item {fónica:ô}
\end{itemize}
\begin{itemize}
\item {Grp. gram.:m.}
\end{itemize}
\begin{itemize}
\item {Proveniência:(It. \textunderscore torso\textunderscore )}
\end{itemize}
Busto de pessôa ou de estátua.
\section{Torso}
\begin{itemize}
\item {fónica:tôr}
\end{itemize}
\begin{itemize}
\item {Grp. gram.:adj.}
\end{itemize}
\begin{itemize}
\item {Grp. gram.:M.}
\end{itemize}
\begin{itemize}
\item {Utilização:Ant.}
\end{itemize}
\begin{itemize}
\item {Proveniência:(Lat. \textunderscore torsus\textunderscore )}
\end{itemize}
O mesmo que [[torcido|torcer]].
Columna torcida em espiral.
\section{Torta}
\begin{itemize}
\item {Grp. gram.:f.}
\end{itemize}
\begin{itemize}
\item {Proveniência:(Lat. \textunderscore torta\textunderscore )}
\end{itemize}
Espécie de pastelão.
\section{Tortão}
\begin{itemize}
\item {Grp. gram.:m.}
\end{itemize}
\begin{itemize}
\item {Proveniência:(De \textunderscore torta\textunderscore )}
\end{itemize}
Arruela em fórma de torta, nos brasões.
\section{Torteira}
\begin{itemize}
\item {Grp. gram.:f.}
\end{itemize}
Vaso, para fazer tortas.
\section{Tortela}
\begin{itemize}
\item {Grp. gram.:f.}
\end{itemize}
\begin{itemize}
\item {Utilização:T. de Resende}
\end{itemize}
Fenda ou frincha de uma porta.
\section{Tortelia}
\begin{itemize}
\item {Grp. gram.:f.}
\end{itemize}
\begin{itemize}
\item {Utilização:Chul.}
\end{itemize}
\begin{itemize}
\item {Proveniência:(De \textunderscore tortelos\textunderscore )}
\end{itemize}
O mesmo que \textunderscore bebedeira\textunderscore .
\section{Tortelos}
\begin{itemize}
\item {Grp. gram.:m.  e  adj.}
\end{itemize}
\begin{itemize}
\item {Utilização:Chul.}
\end{itemize}
\begin{itemize}
\item {Proveniência:(De \textunderscore torto\textunderscore )}
\end{itemize}
O que é zarolho; vesgo.
\section{Torticeiro}
\begin{itemize}
\item {Grp. gram.:adj.}
\end{itemize}
\begin{itemize}
\item {Utilização:Ant.}
\end{itemize}
Que torce a lei ou o direito.
Injusto.
Perverso; malfeitor. Cf. S. R. Viterbo, \textunderscore Elucidário\textunderscore .
(Cast. \textunderscore torticero\textunderscore )
\section{Torticollo}
\begin{itemize}
\item {Grp. gram.:m.}
\end{itemize}
O mesmo que \textunderscore torcicollo\textunderscore .
\section{Torticolo}
\begin{itemize}
\item {Grp. gram.:m.}
\end{itemize}
O mesmo que \textunderscore torcicolo\textunderscore .
\section{Tortilha}
\begin{itemize}
\item {Grp. gram.:f.}
\end{itemize}
Pequena torta.
(Cast. \textunderscore tortilla\textunderscore )
\section{Torto}
\begin{itemize}
\item {fónica:tôr}
\end{itemize}
\begin{itemize}
\item {Grp. gram.:adj.}
\end{itemize}
\begin{itemize}
\item {Utilização:Fig.}
\end{itemize}
\begin{itemize}
\item {Utilização:T. de Turquel}
\end{itemize}
\begin{itemize}
\item {Grp. gram.:M.}
\end{itemize}
\begin{itemize}
\item {Utilização:Ant.}
\end{itemize}
\begin{itemize}
\item {Grp. gram.:Adv.}
\end{itemize}
\begin{itemize}
\item {Proveniência:(Lat. \textunderscore tortus\textunderscore )}
\end{itemize}
Torcido; retorcido.
Curvo; oblíquo.
Vêsgo.
Errado.
Que não tem lealdade.
Que gosta de pirraças.
Bravio, sanhudo.
Offensa, injúria.
Damno; aggravo.
Mal.
De modo errado.
Com desprêzo ou falta de respeito: \textunderscore responder torto\textunderscore .
\section{Tortoles}
\begin{itemize}
\item {Grp. gram.:m.}
\end{itemize}
\begin{itemize}
\item {Utilização:Ant.}
\end{itemize}
Zarolho, zanaga.
O mesmo que \textunderscore tortelos\textunderscore . Cf. Soropita, \textunderscore Poes. e Pros.\textunderscore , 95.
\section{Tortolho}
\begin{itemize}
\item {fónica:tô}
\end{itemize}
\begin{itemize}
\item {Grp. gram.:m.}
\end{itemize}
Arbusto euphorbiáceo de Cabo Verde.
\section{Tortor}
\begin{itemize}
\item {Grp. gram.:m.}
\end{itemize}
Cada um dos cabos náuticos, que ligam as bordas do navio, retesando-se, para que o navio se não abra.
(Cast. \textunderscore tortor\textunderscore )
\section{Tortos}
\begin{itemize}
\item {Grp. gram.:m. pl.}
\end{itemize}
\begin{itemize}
\item {Proveniência:(De \textunderscore torto\textunderscore )}
\end{itemize}
Us. na loc. \textunderscore dôr de tortos\textunderscore , dôr no ventre das parturientes, depois do parto.
\section{Tortosa}
\begin{itemize}
\item {Grp. gram.:f.}
\end{itemize}
\begin{itemize}
\item {Utilização:Gír.}
\end{itemize}
O mesmo que \textunderscore corda\textunderscore .
\section{Tortóz}
\begin{itemize}
\item {Grp. gram.:f.}
\end{itemize}
\begin{itemize}
\item {Utilização:Ant.}
\end{itemize}
O mesmo que \textunderscore rôla\textunderscore ^1, ave.
\section{Tortual}
\begin{itemize}
\item {Grp. gram.:m.}
\end{itemize}
\begin{itemize}
\item {Proveniência:(De \textunderscore torto\textunderscore )}
\end{itemize}
Tranca de ferro ou madeira, que se atravessa no fuso do lagar, para o fazer girar.
Disco, que se adapta ao fuso da roca, para lhe facilitar o giro.
\section{Tortueiral}
\begin{itemize}
\item {Grp. gram.:m.}
\end{itemize}
\begin{itemize}
\item {Proveniência:(Do cast. \textunderscore tortuera\textunderscore )}
\end{itemize}
O mesmo que \textunderscore tortual\textunderscore .
\section{Tortulheira}
\begin{itemize}
\item {Grp. gram.:f.}
\end{itemize}
\begin{itemize}
\item {Utilização:T. da Bairrada}
\end{itemize}
\begin{itemize}
\item {Proveniência:(De \textunderscore tortulho\textunderscore )}
\end{itemize}
Conjunto de hastes ou rebentos saídos da raíz de uma só planta.
\section{Tortulho}
\begin{itemize}
\item {Grp. gram.:m.}
\end{itemize}
\begin{itemize}
\item {Utilização:Fig.}
\end{itemize}
\begin{itemize}
\item {Proveniência:(De \textunderscore torto\textunderscore ?)}
\end{itemize}
Designação genérica do cogumelo.
Feixe de tripas, sêcas e atadas, para se exporem á venda.
Pessôa atarracada.
\section{Tortumelo}
\begin{itemize}
\item {fónica:mê}
\end{itemize}
\begin{itemize}
\item {Grp. gram.:m.}
\end{itemize}
\begin{itemize}
\item {Utilização:T. do Fundão}
\end{itemize}
Tumor, inchaço.
\section{Tortuosamente}
\begin{itemize}
\item {Grp. gram.:adv.}
\end{itemize}
De modo tortuoso, enviesado ou torcido.
Erradamente.
\section{Tortuosidade}
\begin{itemize}
\item {Grp. gram.:f.}
\end{itemize}
\begin{itemize}
\item {Proveniência:(Do lat. \textunderscore tortuositas\textunderscore )}
\end{itemize}
Qualidade do que é tortuoso.
\section{Tortuoso}
\begin{itemize}
\item {Grp. gram.:adj.}
\end{itemize}
\begin{itemize}
\item {Utilização:Fig.}
\end{itemize}
\begin{itemize}
\item {Proveniência:(Lat. \textunderscore tortuosus\textunderscore )}
\end{itemize}
Torto.
Sinuoso.
Opposto á verdade e á justiça.
\section{Tortura}
\begin{itemize}
\item {Grp. gram.:f.}
\end{itemize}
\begin{itemize}
\item {Utilização:Fig.}
\end{itemize}
\begin{itemize}
\item {Utilização:Prov.}
\end{itemize}
\begin{itemize}
\item {Proveniência:(Lat. \textunderscore tortura\textunderscore )}
\end{itemize}
Qualidade ou estado do que é torto.
Curvatura; tortuosidade.
Grande mágua.
Supplício.
Tormento, que se applicava a um accusado, para que êste fizesse revelações.
Lance diffícil; apertos.
Simples transtôrno ou incômmodo:«\textunderscore se isso lhe causa tortura, fique-me com esta bilha maior.\textunderscore »(Colhido em Caneças)
\section{Torturado}
\begin{itemize}
\item {Grp. gram.:adj.}
\end{itemize}
\begin{itemize}
\item {Proveniência:(De \textunderscore torturar\textunderscore )}
\end{itemize}
Compungido; amargurado.
\section{Torturante}
\begin{itemize}
\item {Grp. gram.:adj.}
\end{itemize}
Que tortura; afflictivo.
\section{Torturar}
\begin{itemize}
\item {Grp. gram.:v. t.}
\end{itemize}
Atormentar; affligir muito.
Sujeitar a tortura.
\section{Tórulo}
\begin{itemize}
\item {Grp. gram.:m.}
\end{itemize}
\begin{itemize}
\item {Utilização:Bot.}
\end{itemize}
\begin{itemize}
\item {Proveniência:(Lat. \textunderscore torulus\textunderscore )}
\end{itemize}
Saliência circular nas vagens de algumas plantas.
\section{Toruloso}
\begin{itemize}
\item {Grp. gram.:adj.}
\end{itemize}
Que tem tórulos.
\section{Toruman}
\begin{itemize}
\item {Grp. gram.:m.}
\end{itemize}
\begin{itemize}
\item {Utilização:Bras}
\end{itemize}
Árvore silvestre.
\section{Tôrva}
\begin{itemize}
\item {Grp. gram.:f.}
\end{itemize}
\begin{itemize}
\item {Utilização:Ant.}
\end{itemize}
Acto de torvar^1.
Embaraço, empecilho.
Tremonha, num dos extremos do banco, que constitui a máquina de talhar azeitona.
\section{Torvação}
\begin{itemize}
\item {Grp. gram.:f.}
\end{itemize}
\begin{itemize}
\item {Utilização:Ant.}
\end{itemize}
Acto ou effeito de torvar^1.
Impedimento, obstáculo.
\section{Torvamente}
\begin{itemize}
\item {Grp. gram.:adv.}
\end{itemize}
De modo tôrvo.
Sombriamente; com o sobrecenho carregado.
\section{Torvamento}
\begin{itemize}
\item {Grp. gram.:m.}
\end{itemize}
(V.torvação)
\section{Torvar}
\begin{itemize}
\item {Grp. gram.:v. i.  e  p.}
\end{itemize}
\begin{itemize}
\item {Proveniência:(De \textunderscore tôrvo\textunderscore )}
\end{itemize}
Perturbar-se; irritar-se.
Tornar-se sombrio ou carrancudo.
\section{Torvar}
\begin{itemize}
\item {Grp. gram.:v. t.}
\end{itemize}
(V.turvar)
\section{Torvelinhar}
\begin{itemize}
\item {Grp. gram.:v. i.}
\end{itemize}
Fazer torvelinho; agitar-se; redemoinhar.
\section{Torvelinho}
\begin{itemize}
\item {Grp. gram.:m.}
\end{itemize}
O mesmo que \textunderscore redemoínho\textunderscore .
\section{Torvelino}
\begin{itemize}
\item {Grp. gram.:m.}
\end{itemize}
O mesmo que \textunderscore redemoínho\textunderscore .
\section{Torvento}
\begin{itemize}
\item {Grp. gram.:adj.}
\end{itemize}
\begin{itemize}
\item {Utilização:Ant.}
\end{itemize}
O mesmo que \textunderscore turbulento\textunderscore . Cf. Frei Fortun., \textunderscore Inéd.\textunderscore , 316.
\section{Tôrvo}
\begin{itemize}
\item {Grp. gram.:adj.}
\end{itemize}
\begin{itemize}
\item {Grp. gram.:M.}
\end{itemize}
\begin{itemize}
\item {Utilização:Ant.}
\end{itemize}
\begin{itemize}
\item {Proveniência:(Lat. \textunderscore torvus\textunderscore )}
\end{itemize}
Que causa terror; terrível.
Iracundo.
Que tem aspecto carregado ou carrancudo.
Pavoroso.
Qualidade do que é tôrvo.
O mesmo que \textunderscore tôrva\textunderscore .
\section{Toira}
\begin{itemize}
\item {Grp. gram.:f.}
\end{itemize}
O mesmo que \textunderscore toirinha\textunderscore ^1:«\textunderscore entremezes, toiras e guinolas...\textunderscore »Garrett, \textunderscore Theatro\textunderscore , II, 158.
\section{Tosa}
\begin{itemize}
\item {Grp. gram.:f.}
\end{itemize}
\begin{itemize}
\item {Proveniência:(De \textunderscore tosar\textunderscore ^1)}
\end{itemize}
Operação de tosar a lan ou aparar-lhe a felpa. Cf. \textunderscore Inquér. Industr.\textunderscore , p. II, l. III, 31.
\section{Tosa}
\begin{itemize}
\item {Grp. gram.:f.}
\end{itemize}
\begin{itemize}
\item {Utilização:Fam.}
\end{itemize}
\begin{itemize}
\item {Utilização:Fig.}
\end{itemize}
\begin{itemize}
\item {Proveniência:(De \textunderscore tosar\textunderscore ^2)}
\end{itemize}
Pancadaria, tunda.
Reprehensão.
\section{Tosador}
\begin{itemize}
\item {Grp. gram.:m.  e  adj.}
\end{itemize}
O que tosa.
\section{Tosadura}
\begin{itemize}
\item {Grp. gram.:f.}
\end{itemize}
Acto ou effeito de tosar^1.
\section{Tosão}
\begin{itemize}
\item {Grp. gram.:m.}
\end{itemize}
\begin{itemize}
\item {Proveniência:(Do lat. \textunderscore tonsio\textunderscore , de \textunderscore tondere\textunderscore )}
\end{itemize}
Vello de carneiro.
Ordem militar, na Espanha.
Rêde, para pescar trutas.
\section{Tosar}
\begin{itemize}
\item {Grp. gram.:v. t.}
\end{itemize}
\begin{itemize}
\item {Utilização:Fig.}
\end{itemize}
\begin{itemize}
\item {Proveniência:(Do lat. \textunderscore tonsare\textunderscore )}
\end{itemize}
Tosquiar; aparar a felpa de.
Roer, comer, (falando-se do gado que pasta).
\section{Tosar}
\begin{itemize}
\item {Grp. gram.:v. t.}
\end{itemize}
\begin{itemize}
\item {Proveniência:(Do lat. \textunderscore tunsus\textunderscore )}
\end{itemize}
Dar tosa em.
Bater, sovar.
\section{Toscamente}
\begin{itemize}
\item {Grp. gram.:adv.}
\end{itemize}
De modo tôsco.
\section{Toscanejar}
\begin{itemize}
\item {Grp. gram.:v. i.}
\end{itemize}
\begin{itemize}
\item {Proveniência:(De \textunderscore toscar\textunderscore ?)}
\end{itemize}
Escabecear com somno, abrindo e fechando os olhos muitas vezes.
\section{Toscano}
\begin{itemize}
\item {Grp. gram.:adj.}
\end{itemize}
\begin{itemize}
\item {Grp. gram.:M.}
\end{itemize}
\begin{itemize}
\item {Proveniência:(It. \textunderscore toscano\textunderscore )}
\end{itemize}
Relativo á Toscana.
Diz-se da mais simples das ordens de Architectura entre os Romanos.
Dialecto italiano, falado na Toscana.
Habitante da Toscana.
Ordem toscana de Architectura.
\section{Toscano}
\begin{itemize}
\item {Grp. gram.:m.  e  adj.}
\end{itemize}
\begin{itemize}
\item {Utilização:Pop.}
\end{itemize}
\begin{itemize}
\item {Proveniência:(De \textunderscore tôsco\textunderscore )}
\end{itemize}
Diz-se do carpinteiro, que executa as obras mais tôscas, como desbastar, serrar, etc.
\section{Toscano}
\begin{itemize}
\item {Grp. gram.:adj.}
\end{itemize}
\begin{itemize}
\item {Utilização:Bras. riograndense}
\end{itemize}
O mesmo que \textunderscore narigudo\textunderscore .
\section{Toscar}
\begin{itemize}
\item {Grp. gram.:v. t.}
\end{itemize}
\begin{itemize}
\item {Utilização:Gír.}
\end{itemize}
Avistar; vêr.
Entender, perceber.
\section{Tôsco}
\begin{itemize}
\item {Grp. gram.:adj.}
\end{itemize}
Tal como veio da natureza.
Que não é lapidado nem polido.
Informe; bronco; mal feito.
Rude; inculto.
(Talvez do lat. \textunderscore thyrsicus\textunderscore )
\section{Tôso}
\begin{itemize}
\item {Grp. gram.:m.}
\end{itemize}
\begin{itemize}
\item {Utilização:Bras. do S}
\end{itemize}
\begin{itemize}
\item {Proveniência:(De \textunderscore tosar\textunderscore ^1)}
\end{itemize}
Certo modo de cortar a crina ao cavallo.
\section{Tosquenejamento}
\begin{itemize}
\item {Grp. gram.:m.}
\end{itemize}
Acto de tosquenejar. Cf. B. Pereira, \textunderscore Prosódia\textunderscore , vb. \textunderscore nutatio\textunderscore .
\section{Tosquenejar}
\begin{itemize}
\item {Grp. gram.:v. i.}
\end{itemize}
O mesmo que \textunderscore toscanejar\textunderscore . Cf. Camillo, \textunderscore Volcoens\textunderscore , 157.
\section{Tosquia}
\begin{itemize}
\item {Grp. gram.:f.}
\end{itemize}
\begin{itemize}
\item {Utilização:Fig.}
\end{itemize}
Acto ou effeito de tosquiar.
Época própria para a tosquia dos animaes.
Crítica severa.
\section{Tosquiadela}
\begin{itemize}
\item {Grp. gram.:f.}
\end{itemize}
\begin{itemize}
\item {Utilização:Fig.}
\end{itemize}
\begin{itemize}
\item {Proveniência:(De \textunderscore tosquiar\textunderscore )}
\end{itemize}
O mesmo que \textunderscore tosquia\textunderscore .
Reprehensão; censura.
Tunda.
\section{Tosquiador}
\begin{itemize}
\item {Grp. gram.:m.  e  adj.}
\end{itemize}
O que tosquia.
\section{Tosquiadura}
\begin{itemize}
\item {Grp. gram.:f.}
\end{itemize}
O mesmo que \textunderscore tosquia\textunderscore .
\section{Tosquiar}
\begin{itemize}
\item {Grp. gram.:v. t.}
\end{itemize}
\begin{itemize}
\item {Utilização:Fig.}
\end{itemize}
\begin{itemize}
\item {Proveniência:(Do cast. \textunderscore esquilar\textunderscore )}
\end{itemize}
Cortar rente (pêlo ou cabello).
Cortar rente o pêlo ou o cabello de: \textunderscore tosquiar um rapaz\textunderscore ; \textunderscore tosquiar um burro\textunderscore .
Aparar ou cortar as extremidades da rama de (plantas).
Despojar.
\section{Tosse}
\begin{itemize}
\item {Grp. gram.:f.}
\end{itemize}
\begin{itemize}
\item {Utilização:Gír.}
\end{itemize}
\begin{itemize}
\item {Proveniência:(Do lat. \textunderscore tussis\textunderscore )}
\end{itemize}
Expiração súbita e mais ou menos frequente, pela qual o ar, atravessando os brônchios e a tracheia, produz ruído especial.
Fome.
\textunderscore Tosse sêca\textunderscore , a tosse que não é acompanhada de expectoração.
\textunderscore Tosse convulsa\textunderscore , o mesmo que \textunderscore coqueluche\textunderscore .
\section{Tosse-comprida}
\begin{itemize}
\item {Grp. gram.:f.}
\end{itemize}
\begin{itemize}
\item {Utilização:Bras}
\end{itemize}
O mesmo que \textunderscore coqueluche\textunderscore .
\section{Tosse-da-guariba}
\begin{itemize}
\item {Grp. gram.:f.}
\end{itemize}
\begin{itemize}
\item {Utilização:Bras. do N}
\end{itemize}
O mesmo que \textunderscore tosse-comprida\textunderscore .
\section{Tossegoso}
\begin{itemize}
\item {Grp. gram.:adj.}
\end{itemize}
Que tem tosse.
\section{Tossicar}
\begin{itemize}
\item {Grp. gram.:v. i.}
\end{itemize}
\begin{itemize}
\item {Utilização:Neol.}
\end{itemize}
Têr tosse fraca mas repetida. Cf. Eça, \textunderscore P. Amaro\textunderscore , 346.
\section{Tossidela}
\begin{itemize}
\item {Grp. gram.:f.}
\end{itemize}
\begin{itemize}
\item {Utilização:Pop.}
\end{itemize}
Acto de tossir.
\section{Tossido}
\begin{itemize}
\item {Grp. gram.:m.}
\end{itemize}
\begin{itemize}
\item {Proveniência:(De \textunderscore tossir\textunderscore )}
\end{itemize}
Acto de tossir voluntariamente, para dar qualquer sinal ou exprimir algum sentimento.
\section{Tossir}
\begin{itemize}
\item {Grp. gram.:v. i.}
\end{itemize}
\begin{itemize}
\item {Grp. gram.:V. t.}
\end{itemize}
\begin{itemize}
\item {Utilização:Fig.}
\end{itemize}
\begin{itemize}
\item {Proveniência:(Do lat. \textunderscore tussire\textunderscore )}
\end{itemize}
Têr tosse.
Provocar a tosse artificialmente.
Expellir da garganta.
\section{Tôsso}
\begin{itemize}
\item {Grp. gram.:adj.}
\end{itemize}
\begin{itemize}
\item {Utilização:Des.}
\end{itemize}
\begin{itemize}
\item {Proveniência:(It. \textunderscore tozzo\textunderscore )}
\end{itemize}
Desgracioso, desproporcionado.
\section{Tosta}
\begin{itemize}
\item {Grp. gram.:f.}
\end{itemize}
\begin{itemize}
\item {Proveniência:(De \textunderscore tostar\textunderscore )}
\end{itemize}
O mesmo que \textunderscore torrada\textunderscore .
Bôlo em fórma de torrada.
\section{Tostadela}
\begin{itemize}
\item {Grp. gram.:f.}
\end{itemize}
O mesmo que \textunderscore tostadura\textunderscore .
\section{Tostadura}
\begin{itemize}
\item {Grp. gram.:f.}
\end{itemize}
Acto ou effeito de tostar.
\section{Tostamento}
\begin{itemize}
\item {Grp. gram.:m.}
\end{itemize}
O mesmo que \textunderscore tostadura\textunderscore . Cf. Usque, 22.
\section{Tostão}
\begin{itemize}
\item {Grp. gram.:m.}
\end{itemize}
\begin{itemize}
\item {Utilização:Ant.}
\end{itemize}
\begin{itemize}
\item {Proveniência:(Do it. \textunderscore testone\textunderscore )}
\end{itemize}
Moéda portuguesa de prata, do valor de 100 reis.
Quantia de 100 reis.
Cédula monetária do valor de 100 reis.
Moéda de oiro do valor de 1:200 reis, cunhada no tempo D. Manuel I.
\section{Tostão}
\begin{itemize}
\item {Grp. gram.:m.}
\end{itemize}
Planta herbácea, (\textunderscore baeheravia hirsuta\textunderscore ).
\section{Tostar}
\begin{itemize}
\item {Grp. gram.:v. t.}
\end{itemize}
\begin{itemize}
\item {Proveniência:(Lat. \textunderscore tostare\textunderscore )}
\end{itemize}
Queimar superficialmente.
Torrar.
Tisnar; crestar.
\section{Toste}
\begin{itemize}
\item {Grp. gram.:m.}
\end{itemize}
\begin{itemize}
\item {Utilização:Ant.}
\end{itemize}
Banco, a que se prendiam os forçados, na galé.
\section{Toste}
\begin{itemize}
\item {Grp. gram.:adj.}
\end{itemize}
\begin{itemize}
\item {Utilização:Ant.}
\end{itemize}
\begin{itemize}
\item {Grp. gram.:Adv.}
\end{itemize}
\begin{itemize}
\item {Proveniência:(It. \textunderscore tosto\textunderscore )}
\end{itemize}
Rápido; breve.
Depressa.
\section{Toste}
\begin{itemize}
\item {Grp. gram.:m.}
\end{itemize}
\begin{itemize}
\item {Utilização:Neol.}
\end{itemize}
\begin{itemize}
\item {Proveniência:(Ingl. \textunderscore toast\textunderscore )}
\end{itemize}
Saudação ou brinde, num banquete.
Acto de beber á saúde de alguém.
\section{Tostegar}
\begin{itemize}
\item {Grp. gram.:v. t.}
\end{itemize}
\begin{itemize}
\item {Utilização:Prov.}
\end{itemize}
\begin{itemize}
\item {Utilização:alg.}
\end{itemize}
O mesmo que \textunderscore torcegar\textunderscore .
\section{Tostemente}
\begin{itemize}
\item {Grp. gram.:adv.}
\end{itemize}
\begin{itemize}
\item {Utilização:Ant.}
\end{itemize}
\begin{itemize}
\item {Proveniência:(De \textunderscore toste\textunderscore ^2)}
\end{itemize}
De modo apressado; depressa.
\section{Tóstia}
\begin{itemize}
\item {Grp. gram.:f.}
\end{itemize}
\begin{itemize}
\item {Utilização:Prov.}
\end{itemize}
\begin{itemize}
\item {Utilização:dur.}
\end{itemize}
Tábua transversal para assento, nos barcos. (Colhido em Aveiro)
\section{Total}
\begin{itemize}
\item {Grp. gram.:adj.}
\end{itemize}
\begin{itemize}
\item {Grp. gram.:M.}
\end{itemize}
\begin{itemize}
\item {Proveniência:(Lat. des. \textunderscore totalis\textunderscore )}
\end{itemize}
Que fórma ou abrange um todo.
Completo: \textunderscore ruína total\textunderscore .
O todo, a somma.
\section{Totalidade}
\begin{itemize}
\item {Grp. gram.:f.}
\end{itemize}
\begin{itemize}
\item {Proveniência:(De \textunderscore total\textunderscore )}
\end{itemize}
Somma, conjunto das partes que formam um todo.
\section{Totalização}
\begin{itemize}
\item {Grp. gram.:f.}
\end{itemize}
Acto ou effeito de totalizar.
\section{Totalizador}
\begin{itemize}
\item {Grp. gram.:m.  e  adj.}
\end{itemize}
O que totaliza.
\section{Totalizar}
\begin{itemize}
\item {Grp. gram.:v. t.}
\end{itemize}
\begin{itemize}
\item {Proveniência:(De \textunderscore total\textunderscore )}
\end{itemize}
Avaliar no todo.
Apreciar conjuntamente.
\section{Totalmente}
\begin{itemize}
\item {Grp. gram.:adv.}
\end{itemize}
De modo total; completamente; inteiramente.
\section{Totanga}
\begin{itemize}
\item {Grp. gram.:f.}
\end{itemize}
\begin{itemize}
\item {Utilização:Bras}
\end{itemize}
Planta labiada, o mesmo que \textunderscore cardíaca\textunderscore .
\section{Totilimúndi}
\begin{itemize}
\item {Grp. gram.:m.}
\end{itemize}
\begin{itemize}
\item {Utilização:Fam.}
\end{itemize}
\begin{itemize}
\item {Proveniência:(Do it. \textunderscore totto\textunderscore  + \textunderscore il\textunderscore  + \textunderscore mondo\textunderscore )}
\end{itemize}
Cosmorama.
Salgalhada; mistura de várias coisas.
\section{Totipalmas}
\begin{itemize}
\item {Grp. gram.:f. pl.}
\end{itemize}
\begin{itemize}
\item {Proveniência:(Do lat. \textunderscore totus\textunderscore  + \textunderscore palma\textunderscore )}
\end{itemize}
Família de aves palmípedes, a que pertence o pelicano.
\section{Totipálmeas}
\begin{itemize}
\item {Grp. gram.:f. pl.}
\end{itemize}
O mesmo que \textunderscore totipalmas\textunderscore .
\section{Tòtó}
\begin{itemize}
\item {Grp. gram.:m.}
\end{itemize}
\begin{itemize}
\item {Utilização:Fam.}
\end{itemize}
O mesmo que \textunderscore cãozinho\textunderscore .
\section{Toto}
\begin{itemize}
\item {Grp. gram.:m.}
\end{itemize}
Ave da América setentrional.
\section{Totolito}
\begin{itemize}
\item {Grp. gram.:m.}
\end{itemize}
\begin{itemize}
\item {Utilização:Miner.}
\end{itemize}
Variedade de perídoto.
\section{Totoloque}
\begin{itemize}
\item {Grp. gram.:m.}
\end{itemize}
Espécie de jôgo, usado pelos antigos Mexicanos, e que consistia em deslocar, a distância, bólas de oiro com bólas do mesmo metal.
\section{Totó-piruleta}
\begin{itemize}
\item {Grp. gram.:m.}
\end{itemize}
\begin{itemize}
\item {Utilização:Fam.}
\end{itemize}
Homem ridículo, pedante.
\section{Totuma}
\begin{itemize}
\item {Grp. gram.:f.}
\end{itemize}
Abóbora americana, que os Indígenas comem depois de cozida, aproveitando a casca para vasilha.
\section{Totumo}
\begin{itemize}
\item {Grp. gram.:m.}
\end{itemize}
Abóbora americana, que os Indígenas comem depois de cozida, aproveitando a casca para vasilha.
\section{Toturubá}
\begin{itemize}
\item {Grp. gram.:m.}
\end{itemize}
\begin{itemize}
\item {Utilização:Bras}
\end{itemize}
Árvore fructífera dos sertões.
\section{Touca}
\begin{itemize}
\item {Grp. gram.:f.}
\end{itemize}
\begin{itemize}
\item {Utilização:Pop.}
\end{itemize}
\begin{itemize}
\item {Proveniência:(Do b. bret. \textunderscore tok\textunderscore )}
\end{itemize}
Adôrno de cambraia ou de outro tecido, com que se cobre toda a parte cabelluda da cabeça e é usado por crianças e mulheres.
Peça de vestuário, usado por freiras, e que lhes cobre a cabeça, o pescoço e os ombros.
Turbante.
Faixa enrolada, que se usava como turbante.
O mesmo que \textunderscore bebedeira\textunderscore .
\section{Touça}
\begin{itemize}
\item {Grp. gram.:f.}
\end{itemize}
\begin{itemize}
\item {Utilização:Prov.}
\end{itemize}
\begin{itemize}
\item {Utilização:minh.}
\end{itemize}
\begin{itemize}
\item {Utilização:Prov.}
\end{itemize}
\begin{itemize}
\item {Utilização:trasm.}
\end{itemize}
\begin{itemize}
\item {Proveniência:(Do b. lat. \textunderscore toussa\textunderscore ?)}
\end{itemize}
Grande vergôntea de castanheiro, de que se fazem arcos para pipas.
Vara ou pernada alta e grossa de qualquer árvore.
O pé da cana de açúcar.
Moita de feno grosseiro.
O mesmo que \textunderscore touço\textunderscore .
Qualquer moita:«\textunderscore touça de carvalhos\textunderscore ». Camillo, \textunderscore Mem. do Cárcere\textunderscore .
\section{Toucado}
\begin{itemize}
\item {Grp. gram.:m.}
\end{itemize}
\begin{itemize}
\item {Proveniência:(De \textunderscore toucar\textunderscore )}
\end{itemize}
Conjunto dos adornos da cabeça das mulheres.
\section{Toucador}
\begin{itemize}
\item {Grp. gram.:m.  e  adj.}
\end{itemize}
\begin{itemize}
\item {Grp. gram.:M.}
\end{itemize}
\begin{itemize}
\item {Proveniência:(De \textunderscore toucar\textunderscore )}
\end{itemize}
O que touca.
Espécie de mesa, encimada por um espelho, para servir a quem se touca ou penteia.
Casa ou gabinete, destinado especialmente para alguém se toucar, pentear ou vestir.
Touca, em que as mulheres envolvem o cabello, ao deitar-se.
\section{Toucar}
\begin{itemize}
\item {Grp. gram.:v. t.}
\end{itemize}
\begin{itemize}
\item {Utilização:Fig.}
\end{itemize}
Cingir ou cobrir com touca.
Pentear e dispor convenientemente o cabello de.
Enfeitar.
Encimar; circundar; aureolar.
\section{Touceira}
\begin{itemize}
\item {Grp. gram.:f.}
\end{itemize}
\begin{itemize}
\item {Utilização:T. do Fundão}
\end{itemize}
Grande touça.
Pé de uma planta, com raízes.
\section{Touceiral}
\begin{itemize}
\item {Grp. gram.:m.}
\end{itemize}
O mesmo que \textunderscore moitedo\textunderscore . Cf. Neto, \textunderscore Baladilhas\textunderscore , 271.
\section{Toucinheira}
\begin{itemize}
\item {Grp. gram.:f.}
\end{itemize}
\begin{itemize}
\item {Utilização:Prov.}
\end{itemize}
\begin{itemize}
\item {Utilização:minh.}
\end{itemize}
O mesmo que \textunderscore matadeira\textunderscore .
\section{Toucinheiro}
\begin{itemize}
\item {Grp. gram.:m.}
\end{itemize}
Aquele que vende toucinho ou qualquer carne de porco.
\section{Toucinho}
\begin{itemize}
\item {Grp. gram.:m.}
\end{itemize}
\begin{itemize}
\item {Proveniência:(Do cast. \textunderscore tocino\textunderscore )}
\end{itemize}
Gordura dos porcos, subjacente á pele.
\section{Touço}
\begin{itemize}
\item {Grp. gram.:m.}
\end{itemize}
\begin{itemize}
\item {Utilização:Prov.}
\end{itemize}
\begin{itemize}
\item {Utilização:minh.}
\end{itemize}
O mesmo que \textunderscore temão\textunderscore  (do carro).
Parte do carro, donde sái o cabeçalho.
\textunderscore Pôr-se ao touço\textunderscore , resistir.
(Cp. \textunderscore touça\textunderscore )
\section{Toufão}
\begin{itemize}
\item {Grp. gram.:m.}
\end{itemize}
\begin{itemize}
\item {Utilização:Prov.}
\end{itemize}
Buraco, onde os peixes se abrigam, nos rios, entre pedras ou debaixo dellas, e em que os pescadores os pescam com a mão.
\section{Touga}
\begin{itemize}
\item {Grp. gram.:f.}
\end{itemize}
\begin{itemize}
\item {Utilização:Ant.}
\end{itemize}
O mesmo que \textunderscore touca\textunderscore .
\section{Tougue}
\begin{itemize}
\item {Grp. gram.:m.}
\end{itemize}
Espécie de estandarte turco, formado de meia lança, na extremidade da qual se prende uma cauda de cavalo com botão de oiro.
(Do turco \textunderscore tug\textunderscore )
\section{Touguinho}
\begin{itemize}
\item {Grp. gram.:adj.}
\end{itemize}
\begin{itemize}
\item {Utilização:T. de Avintes}
\end{itemize}
Imbecil, idiota.
\section{Toupeira}
\begin{itemize}
\item {Grp. gram.:f.}
\end{itemize}
\begin{itemize}
\item {Utilização:Fig.}
\end{itemize}
\begin{itemize}
\item {Utilização:Fam.}
\end{itemize}
\begin{itemize}
\item {Proveniência:(Do lat. hyp. \textunderscore talparia\textunderscore , de \textunderscore talpa\textunderscore )}
\end{itemize}
Mammífero insectívoro, que vive debaixo da terra, minando-a.
Peixe acanthopterýgio.
Pessôa de olhos pequenos e piscos.
Pessôa estúpida.
Mulhér, mal vestida e velha.
Pessôa, que conspira a occultas, procurando minar ou subverter instituições ou systemas.
Pessôa intriguista ou mexeriqueira.
\section{Toupeirinho}
\begin{itemize}
\item {Grp. gram.:adj.}
\end{itemize}
Diz-se de uma variedade de grillo, também conhecido pelo nome de \textunderscore rallo\textunderscore .
(Cp. \textunderscore toupeira\textunderscore  e fr. \textunderscore taupe-grillon\textunderscore )
\section{Touqueixo}
\begin{itemize}
\item {Grp. gram.:m.}
\end{itemize}
\begin{itemize}
\item {Utilização:Ant.}
\end{itemize}
O mesmo ou melhór que \textunderscore toqueixo\textunderscore .
\section{Toura}
\begin{itemize}
\item {Grp. gram.:f.}
\end{itemize}
\begin{itemize}
\item {Utilização:Fam.}
\end{itemize}
\begin{itemize}
\item {Proveniência:(Do lat. \textunderscore taura\textunderscore )}
\end{itemize}
Vaca estéril.
Mulhér irascível, bravia.
\section{Toura}
\begin{itemize}
\item {Grp. gram.:f.}
\end{itemize}
\begin{itemize}
\item {Utilização:Prov.}
\end{itemize}
\begin{itemize}
\item {Utilização:alg.}
\end{itemize}
O mesmo que \textunderscore tacho\textunderscore .
\section{Toura}
\begin{itemize}
\item {Grp. gram.:f.}
\end{itemize}
O mesmo que \textunderscore toirinha\textunderscore ^1:«\textunderscore entremezes, touras e guinolas...\textunderscore »Garrett, \textunderscore Theatro\textunderscore , II, 158.
\section{Tourada}
\begin{itemize}
\item {Grp. gram.:f.}
\end{itemize}
Bando de touros.
Corrida de touros, em circos.
\section{Toural}
\begin{itemize}
\item {Grp. gram.:m.}
\end{itemize}
\begin{itemize}
\item {Grp. gram.:Adj.}
\end{itemize}
Lugar, onde um coêlho costuma estercar, e onde os caçadores o esperam.
Diz-se de uma variedade de azeitona, também chamada \textunderscore madural\textunderscore .
\section{Touralho}
\begin{itemize}
\item {Grp. gram.:m.}
\end{itemize}
\begin{itemize}
\item {Utilização:T. de Turquel}
\end{itemize}
Estêrco de coêlho.
(Cp. \textunderscore toural\textunderscore )
\section{Tourão}
\begin{itemize}
\item {Grp. gram.:m.}
\end{itemize}
\begin{itemize}
\item {Utilização:Fam.}
\end{itemize}
\begin{itemize}
\item {Proveniência:(De \textunderscore touro\textunderscore )}
\end{itemize}
Furão bravo.
Criança traquina.
\section{Touraria}
\begin{itemize}
\item {Grp. gram.:f.}
\end{itemize}
\begin{itemize}
\item {Utilização:Fam.}
\end{itemize}
\begin{itemize}
\item {Proveniência:(De \textunderscore touro\textunderscore )}
\end{itemize}
Barulho, desordem.
Fúria.
\section{Toureador}
\begin{itemize}
\item {Grp. gram.:m.  e  adj.}
\end{itemize}
O que tourea; toureiro.
\section{Tourear}
\begin{itemize}
\item {Grp. gram.:v. t.}
\end{itemize}
\begin{itemize}
\item {Utilização:Fig.}
\end{itemize}
\begin{itemize}
\item {Utilização:Bras}
\end{itemize}
\begin{itemize}
\item {Grp. gram.:V. i.}
\end{itemize}
Correr ou lidar (touros) num circo ou praça.
Perseguir, atacar.
Namorar.
Correr touros.
\section{Toureio}
\begin{itemize}
\item {Grp. gram.:m.}
\end{itemize}
Acto, efeito ou arte de tourear.
\section{Toureiro}
\begin{itemize}
\item {Grp. gram.:m.}
\end{itemize}
\begin{itemize}
\item {Grp. gram.:Adj.}
\end{itemize}
Aquele que toureia, especialmente o que toureia por hábito ou profissão.
Relativo a touro.
\section{Tourejão}
\begin{itemize}
\item {Grp. gram.:m.}
\end{itemize}
Cavilha, que ampara as rodas da carreta, nas extremidades do eixo.
\section{Tourejar}
\begin{itemize}
\item {Grp. gram.:v. i.  e  t.}
\end{itemize}
(V.tourear)
\section{Touri}
\begin{itemize}
\item {Grp. gram.:m.}
\end{itemize}
O mesmo que \textunderscore umari\textunderscore .
\section{Touriga}
\begin{itemize}
\item {Grp. gram.:f.}
\end{itemize}
\begin{itemize}
\item {Proveniência:(De \textunderscore tourigo\textunderscore )}
\end{itemize}
Nome de três variedades de uva.
\section{Tourigão}
\begin{itemize}
\item {Grp. gram.:m.}
\end{itemize}
\begin{itemize}
\item {Proveniência:(De \textunderscore tourigo\textunderscore )}
\end{itemize}
Casta de uva.
\section{Tourigão-foufeiro}
\begin{itemize}
\item {Grp. gram.:m.}
\end{itemize}
Casta de uva, na região do Doiro.
\section{Tourigo}
\begin{itemize}
\item {Grp. gram.:m.}
\end{itemize}
\begin{itemize}
\item {Proveniência:(De \textunderscore Tourigo\textunderscore , n. p.)}
\end{itemize}
O mesmo que \textunderscore touriga\textunderscore .
\section{Touril}
\begin{itemize}
\item {Grp. gram.:m.}
\end{itemize}
Curral de gado vaccum.
Lugar, annexo á praça de toiros, em que êstes se guardam antes da corrida.
(Cp. cast. \textunderscore toril\textunderscore )
\section{Tourinha}
\begin{itemize}
\item {Grp. gram.:f.}
\end{itemize}
\begin{itemize}
\item {Utilização:Fam.}
\end{itemize}
\begin{itemize}
\item {Proveniência:(De \textunderscore touro\textunderscore )}
\end{itemize}
Corrida de novilhas mansas.
Imitação de uma corrida de touros, sendo êstes representados por canastras, etc.
Objecto de troça.
\section{Tourinha}
\begin{itemize}
\item {Grp. gram.:f.}
\end{itemize}
Peixe pletógnato.
\section{Tourista}
\begin{itemize}
\item {Grp. gram.:m.}
\end{itemize}
O mesmo que \textunderscore toureiro\textunderscore . Cf. Herculano, \textunderscore Lendas\textunderscore , II, 183.
\section{Touro}
\begin{itemize}
\item {Grp. gram.:m.}
\end{itemize}
\begin{itemize}
\item {Utilização:Fig.}
\end{itemize}
\begin{itemize}
\item {Utilização:Prov.}
\end{itemize}
\begin{itemize}
\item {Utilização:alg.}
\end{itemize}
\begin{itemize}
\item {Grp. gram.:Pl.}
\end{itemize}
\begin{itemize}
\item {Proveniência:(Do lat. \textunderscore taurus\textunderscore )}
\end{itemize}
Boi, que não é castrado; boi bravo.
Homem robusto e fogoso.
Um dos signos de Zodíaco.
Tacho de papas.
O mesmo que \textunderscore tourada\textunderscore : \textunderscore hoje, assistiu pouca gente aos touros\textunderscore .
\section{Tousar}
\textunderscore v. t.\textunderscore  (e der.)
Fórma ant. de \textunderscore taxar\textunderscore , etc.
\section{Touta}
\begin{itemize}
\item {Grp. gram.:f.}
\end{itemize}
\begin{itemize}
\item {Utilização:Pop.}
\end{itemize}
\begin{itemize}
\item {Proveniência:(Do lat. \textunderscore capita\textunderscore , pl. de \textunderscore caput\textunderscore )}
\end{itemize}
Topête; toutiço; cabeça.
\section{Touteador}
\begin{itemize}
\item {Grp. gram.:m.  e  adj.}
\end{itemize}
O que touteia.
\section{Toutear}
\begin{itemize}
\item {Grp. gram.:v. i.}
\end{itemize}
\begin{itemize}
\item {Proveniência:(De \textunderscore touta\textunderscore )}
\end{itemize}
Fazer ou dizer tolices.
\section{Toutiçada}
\begin{itemize}
\item {Grp. gram.:f.}
\end{itemize}
Pancada no toutiço.
\section{Toutiço}
\begin{itemize}
\item {Grp. gram.:m.}
\end{itemize}
\begin{itemize}
\item {Proveniência:(De \textunderscore touta\textunderscore )}
\end{itemize}
A parte posterior da cabeça.
Nuca.
Cachaço.
A cabeça.
\section{Toutinegra}
\begin{itemize}
\item {Grp. gram.:f.}
\end{itemize}
\begin{itemize}
\item {Proveniência:(Do lat. \textunderscore capite nigra\textunderscore , seg. Cornu)}
\end{itemize}
Nome de várias espécies de pássaros dentirostros.
O mesmo que \textunderscore tutinegro\textunderscore .
\section{Toutivanas}
\begin{itemize}
\item {Grp. gram.:m.}
\end{itemize}
\begin{itemize}
\item {Utilização:Des.}
\end{itemize}
\begin{itemize}
\item {Proveniência:(De \textunderscore toutear\textunderscore )}
\end{itemize}
O mesmo que \textunderscore doidivanas\textunderscore .
\section{Továria}
\begin{itemize}
\item {Grp. gram.:f.}
\end{itemize}
\begin{itemize}
\item {Proveniência:(De \textunderscore Tovar\textunderscore , n. p.)}
\end{itemize}
Gênero de plantas capparidáceas.
\section{Tovas}
\begin{itemize}
\item {Grp. gram.:m. pl.}
\end{itemize}
Tríbo selvagem da América do Sul, na região do Chaco.
\section{Tovetove}
\begin{itemize}
\item {Grp. gram.:m.}
\end{itemize}
Árvore do Congo.
\section{Toxemia}
\begin{itemize}
\item {fónica:cse}
\end{itemize}
\begin{itemize}
\item {Grp. gram.:f.}
\end{itemize}
\begin{itemize}
\item {Proveniência:(Do gr. \textunderscore toxon\textunderscore  + \textunderscore haima\textunderscore )}
\end{itemize}
Natureza séptica do sangue.
Entoxicação do sangue.
\section{Toxicar}
\begin{itemize}
\item {fónica:csi}
\end{itemize}
\textunderscore v. t.\textunderscore  (e der.)
O mesmo que \textunderscore entoxicar\textunderscore , etc.
\section{Toxicidade}
\begin{itemize}
\item {fónica:csi}
\end{itemize}
\begin{itemize}
\item {Grp. gram.:f.}
\end{itemize}
Qualidade daquillo que é tóxico.
\section{Tóxico}
\begin{itemize}
\item {fónica:csi}
\end{itemize}
\begin{itemize}
\item {Grp. gram.:adj.}
\end{itemize}
\begin{itemize}
\item {Grp. gram.:M.}
\end{itemize}
\begin{itemize}
\item {Proveniência:(Lat. \textunderscore toxicum\textunderscore )}
\end{itemize}
Que envenena.
Que tem a propriedade de envenenar.
O mesmo que \textunderscore veneno\textunderscore .
\section{Toxicoemia}
\begin{itemize}
\item {fónica:csi}
\end{itemize}
\begin{itemize}
\item {Grp. gram.:f.}
\end{itemize}
\begin{itemize}
\item {Utilização:Med.}
\end{itemize}
\begin{itemize}
\item {Proveniência:(Do gr. \textunderscore toxikon\textunderscore  + \textunderscore haima\textunderscore )}
\end{itemize}
Estado do sangue, que contém substância venenosa; toxemía.
\section{Toxicoêmico}
\begin{itemize}
\item {fónica:csi}
\end{itemize}
\begin{itemize}
\item {Grp. gram.:adj.}
\end{itemize}
Relativo á toxicoemía.
\section{Toxicófago}
\begin{itemize}
\item {fónica:csi}
\end{itemize}
\begin{itemize}
\item {Grp. gram.:m.  e  adj.}
\end{itemize}
\begin{itemize}
\item {Proveniência:(Do gr. \textunderscore toxikon\textunderscore  + \textunderscore phagein\textunderscore )}
\end{itemize}
O que mistura com os seus alimentos substâncias venenosas. Cf. Júl. Dinis, \textunderscore Pupillas\textunderscore , 122.
\section{Toxicóforo}
\begin{itemize}
\item {fónica:csi}
\end{itemize}
\begin{itemize}
\item {Grp. gram.:adj.}
\end{itemize}
\begin{itemize}
\item {Proveniência:(Do gr. \textunderscore toxikon\textunderscore  + \textunderscore phoros\textunderscore )}
\end{itemize}
Que produz veneno.
\section{Toxicografia}
\begin{itemize}
\item {fónica:csi}
\end{itemize}
\begin{itemize}
\item {Grp. gram.:f.}
\end{itemize}
\begin{itemize}
\item {Proveniência:(Do gr. \textunderscore toxikon\textunderscore  + \textunderscore graphein\textunderscore )}
\end{itemize}
Descripção dos tóxicos.
\section{Toxicográfico}
\begin{itemize}
\item {fónica:csi}
\end{itemize}
\begin{itemize}
\item {Grp. gram.:adj.}
\end{itemize}
Relativo á toxicografia.
\section{Toxicographia}
\begin{itemize}
\item {fónica:csi}
\end{itemize}
\begin{itemize}
\item {Grp. gram.:f.}
\end{itemize}
\begin{itemize}
\item {Proveniência:(Do gr. \textunderscore toxikon\textunderscore  + \textunderscore graphein\textunderscore )}
\end{itemize}
Descripção dos tóxicos.
\section{Toxicográphico}
\begin{itemize}
\item {fónica:csi}
\end{itemize}
\begin{itemize}
\item {Grp. gram.:adj.}
\end{itemize}
Relativo á toxicographia.
\section{Toxicohemia}
\begin{itemize}
\item {fónica:csi}
\end{itemize}
\begin{itemize}
\item {Grp. gram.:f.}
\end{itemize}
\begin{itemize}
\item {Utilização:Med.}
\end{itemize}
\begin{itemize}
\item {Proveniência:(Do gr. \textunderscore toxikon\textunderscore  + \textunderscore haima\textunderscore )}
\end{itemize}
Estado do sangue, que contém substância venenosa; toxemía.
\section{Toxicohêmico}
\begin{itemize}
\item {fónica:csi}
\end{itemize}
\begin{itemize}
\item {Grp. gram.:adj.}
\end{itemize}
Relativo á toxicohemía.
\section{Toxicologia}
\begin{itemize}
\item {fónica:csi}
\end{itemize}
\begin{itemize}
\item {Grp. gram.:f.}
\end{itemize}
\begin{itemize}
\item {Proveniência:(De \textunderscore toxicólogo\textunderscore )}
\end{itemize}
Sciência, que se occupa dos tóxicos.
\section{Toxicológico}
\begin{itemize}
\item {fónica:csi}
\end{itemize}
\begin{itemize}
\item {Grp. gram.:adj.}
\end{itemize}
Relativo á toxicologia.
\section{Toxicologista}
\begin{itemize}
\item {fónica:csi}
\end{itemize}
\begin{itemize}
\item {Grp. gram.:m.}
\end{itemize}
Tratadista de toxicologia. Cf. Júl. Dinis, \textunderscore Serões\textunderscore , 157.
\section{Toxicólogo}
\begin{itemize}
\item {Grp. gram.:m.}
\end{itemize}
\begin{itemize}
\item {Proveniência:(Do gr. \textunderscore toxikon\textunderscore  + \textunderscore logos\textunderscore )}
\end{itemize}
Aquelle que trata de toxicologia.
\section{Toxicómetro}
\begin{itemize}
\item {fónica:csi}
\end{itemize}
\begin{itemize}
\item {Grp. gram.:m.}
\end{itemize}
\begin{itemize}
\item {Proveniência:(Do gr. \textunderscore toxikon\textunderscore  + \textunderscore metron\textunderscore )}
\end{itemize}
Instrumento, para avaliar a intensidade dos venenos.
\section{Toxicóphago}
\begin{itemize}
\item {fónica:csi}
\end{itemize}
\begin{itemize}
\item {Grp. gram.:m.  e  adj.}
\end{itemize}
\begin{itemize}
\item {Proveniência:(Do gr. \textunderscore toxikon\textunderscore  + \textunderscore phagein\textunderscore )}
\end{itemize}
O que mistura com os seus alimentos substâncias venenosas. Cf. Júl. Dinis, \textunderscore Pupillas\textunderscore , 122.
\section{Toxicóphoro}
\begin{itemize}
\item {fónica:csi}
\end{itemize}
\begin{itemize}
\item {Grp. gram.:adj.}
\end{itemize}
\begin{itemize}
\item {Proveniência:(Do gr. \textunderscore toxikon\textunderscore  + \textunderscore phoros\textunderscore )}
\end{itemize}
Que produz veneno.
\section{Toxidendro}
\begin{itemize}
\item {fónica:csi}
\end{itemize}
\begin{itemize}
\item {Grp. gram.:m.}
\end{itemize}
Nome de várias plantas, do gênero sumagre.
\section{Toxidermia}
\begin{itemize}
\item {fónica:csi}
\end{itemize}
\begin{itemize}
\item {Grp. gram.:f.}
\end{itemize}
\begin{itemize}
\item {Utilização:Med.}
\end{itemize}
\begin{itemize}
\item {Proveniência:(Do gr. \textunderscore toxikon\textunderscore  + \textunderscore derma\textunderscore )}
\end{itemize}
Dermatose, de origem tóxica.
\section{Toxina}
\begin{itemize}
\item {fónica:csi}
\end{itemize}
\begin{itemize}
\item {Grp. gram.:f.}
\end{itemize}
\begin{itemize}
\item {Utilização:Med.}
\end{itemize}
\begin{itemize}
\item {Proveniência:(Fr. \textunderscore toxine\textunderscore )}
\end{itemize}
Substância venenosa, segregada por bactérias.
\section{Toxocarpo}
\begin{itemize}
\item {fónica:cso}
\end{itemize}
\begin{itemize}
\item {Grp. gram.:m.}
\end{itemize}
\begin{itemize}
\item {Proveniência:(Do gr. \textunderscore toxon\textunderscore  + \textunderscore karpos\textunderscore )}
\end{itemize}
Gênero de plantas asclepiadáceas.
\section{Toxócera}
\begin{itemize}
\item {fónica:csó}
\end{itemize}
\begin{itemize}
\item {Grp. gram.:f.}
\end{itemize}
\begin{itemize}
\item {Proveniência:(Do gr. \textunderscore toxon\textunderscore  + \textunderscore keras\textunderscore )}
\end{itemize}
Gênero de molluscos cephalópodes.
\section{Toxodonte}
\begin{itemize}
\item {fónica:cso}
\end{itemize}
\begin{itemize}
\item {Grp. gram.:m.}
\end{itemize}
\begin{itemize}
\item {Proveniência:(Do gr. \textunderscore toxon\textunderscore  + \textunderscore odous\textunderscore )}
\end{itemize}
Gênero de mammíferos pachydermes.
\section{Toxofilo}
\begin{itemize}
\item {fónica:cso}
\end{itemize}
\begin{itemize}
\item {Grp. gram.:adj.}
\end{itemize}
\begin{itemize}
\item {Utilização:Bot.}
\end{itemize}
\begin{itemize}
\item {Proveniência:(Do gr. \textunderscore toxon\textunderscore  + \textunderscore phullon\textunderscore )}
\end{itemize}
Que tem fôlhas em fórma de frecha.
\section{Toxóforo}
\begin{itemize}
\item {fónica:csó}
\end{itemize}
\begin{itemize}
\item {Grp. gram.:m.}
\end{itemize}
\begin{itemize}
\item {Proveniência:(Do gr. \textunderscore toxon\textunderscore  + \textunderscore phoros\textunderscore )}
\end{itemize}
Gênero de insectos coleópteros tetrâmeros.
\section{Toxoneura}
\begin{itemize}
\item {fónica:cso}
\end{itemize}
\begin{itemize}
\item {Grp. gram.:f.}
\end{itemize}
\begin{itemize}
\item {Proveniência:(Do gr. \textunderscore toxon\textunderscore  + \textunderscore neuron\textunderscore )}
\end{itemize}
Gênero de insectos dípteros.
\section{Toxóphoro}
\begin{itemize}
\item {fónica:csó}
\end{itemize}
\begin{itemize}
\item {Grp. gram.:m.}
\end{itemize}
\begin{itemize}
\item {Proveniência:(Do gr. \textunderscore toxon\textunderscore  + \textunderscore phoros\textunderscore )}
\end{itemize}
Gênero de insectos coleópteros tetrâmeros.
\section{Toxophyllo}
\begin{itemize}
\item {fónica:cso}
\end{itemize}
\begin{itemize}
\item {Grp. gram.:adj.}
\end{itemize}
\begin{itemize}
\item {Utilização:Bot.}
\end{itemize}
\begin{itemize}
\item {Proveniência:(Do gr. \textunderscore toxon\textunderscore  + \textunderscore phullon\textunderscore )}
\end{itemize}
Que tem fôlhas em fórma de frecha.
\section{Tóxotea}
\begin{itemize}
\item {fónica:cso}
\end{itemize}
\begin{itemize}
\item {Grp. gram.:m.}
\end{itemize}
\begin{itemize}
\item {Proveniência:(Gr. \textunderscore toxotes\textunderscore )}
\end{itemize}
Soldado, dos que velavam pela ordem pública nas ruas de Athenas.
\section{Tóxote}
\begin{itemize}
\item {fónica:cso}
\end{itemize}
\begin{itemize}
\item {Grp. gram.:f.}
\end{itemize}
\begin{itemize}
\item {Utilização:Bot.}
\end{itemize}
\begin{itemize}
\item {Proveniência:(Gr. \textunderscore toxotis\textunderscore )}
\end{itemize}
Designação antiga da artemísia.
\section{Tozamento}
\begin{itemize}
\item {Grp. gram.:m.}
\end{itemize}
Curva, descrita pelos madeiros de um navio, collocados de prôa á prôa.
\section{Tra...}
\begin{itemize}
\item {Grp. gram.:pref.}
\end{itemize}
O mesmo que \textunderscore trans...\textunderscore 
\section{Trabal}
\begin{itemize}
\item {Grp. gram.:adj.}
\end{itemize}
\begin{itemize}
\item {Proveniência:(Lat. \textunderscore trabalis\textunderscore )}
\end{itemize}
Diz-se do prego, próprio para pregar traves.
\section{Trabalhadamente}
\begin{itemize}
\item {Grp. gram.:adv.}
\end{itemize}
\begin{itemize}
\item {Proveniência:(De \textunderscore trabalhado\textunderscore )}
\end{itemize}
Com trabalho ou com cuidado.
\section{Trabalhadeira}
\begin{itemize}
\item {Grp. gram.:f.  e  adj.}
\end{itemize}
\begin{itemize}
\item {Proveniência:(De \textunderscore trabalhar\textunderscore )}
\end{itemize}
Diz-se da mulhér, que gosta de trabalhar, que é diligente e cuidadosa.
\section{Trabalhado}
\begin{itemize}
\item {Grp. gram.:adj.}
\end{itemize}
\begin{itemize}
\item {Proveniência:(De \textunderscore trabalhar\textunderscore )}
\end{itemize}
Pôsto em obra; lavrado.
Trabalhoso.
\section{Trabalhador}
\begin{itemize}
\item {Grp. gram.:m.}
\end{itemize}
\begin{itemize}
\item {Grp. gram.:Adj.}
\end{itemize}
\begin{itemize}
\item {Proveniência:(De \textunderscore trabalhar\textunderscore )}
\end{itemize}
Aquelle que trabalha.
Jornaleiro.
Aquelle que se occupa nos trabalhos mais rudes do campo.
Dado ao trabalho, que gosta de trabalho.
Laborioso; activo: \textunderscore aquelle Ministro é muito trabalhador\textunderscore .
\section{Trabalhão}
\begin{itemize}
\item {Grp. gram.:m.}
\end{itemize}
Grande trabalho ou grande fadiga.
\section{Trabalhar}
\begin{itemize}
\item {Grp. gram.:v. t.}
\end{itemize}
\begin{itemize}
\item {Grp. gram.:V. i.}
\end{itemize}
\begin{itemize}
\item {Grp. gram.:V. p.}
\end{itemize}
\begin{itemize}
\item {Proveniência:(De \textunderscore trabalho\textunderscore )}
\end{itemize}
Applicar trabalho a.
Executar cuidadosamente.
Fazer artisticamente.
Apurar-se na feitura de.
Lavrar, pôr em obra.
Preoccupar; affligir, atormentar.
Applicar a actividade própria.
Exercer o seu offício.
Fazer diligência; esforçar-se.
Cogitar.
Regular, mover-se, (falando-se de certos maquinismos).
(seguido da prep. \textunderscore de\textunderscore ), esforçar-se por. Cf. Sous. Monteiro, \textunderscore Elog. de Lat.\textunderscore 
\section{Trabalheira}
\begin{itemize}
\item {Grp. gram.:f.}
\end{itemize}
\begin{itemize}
\item {Utilização:Fam.}
\end{itemize}
\begin{itemize}
\item {Proveniência:(De \textunderscore trabalhar\textunderscore )}
\end{itemize}
Trabalhão; azáfama.
\section{Trabalho}
\begin{itemize}
\item {Grp. gram.:m.}
\end{itemize}
\begin{itemize}
\item {Utilização:Physiol.}
\end{itemize}
\begin{itemize}
\item {Grp. gram.:Pl.}
\end{itemize}
\begin{itemize}
\item {Proveniência:(Do lat. hyp. \textunderscore trabaculum\textunderscore  ou \textunderscore trepalium\textunderscore )}
\end{itemize}
Applicação da actividade intellectual ou phýsica.
Serviço.
Fadiga.
Acção de um maquinismo.
Resultado dessa acção.
Resultado de um serviço ou da actividade phýsica ou moral do homem.
Labutação.
Cuidado ou esmêro em qualquer serviço.
Afflicção, inquietação.
Exercício.
Obra, que está para fazer-se ou em via de execução.
Maneira, com que se exerce a actividade intellectual ou material.
Acção mechânica dos agentes naturaes.
Phenómeno orgânico no interior dos tecidos.
Discussões ou deliberações, (falando-se de uma corporação).
Emprehendimentos gloriosos e fatigantes.
Afflicções, cuidados:«\textunderscore acabaram-se as penas e os trabalhos.\textunderscore »Junqueiro, \textunderscore Melro\textunderscore .
\section{Trabalhosamente}
\begin{itemize}
\item {Grp. gram.:adv.}
\end{itemize}
De modo trabalhoso.
Com trabalho; com fadiga.
Á custa de sacrifícios.
\section{Trabalhoso}
\begin{itemize}
\item {Grp. gram.:adj.}
\end{itemize}
Que dá trabalho ou fadiga.
Custoso; diffícil.
\section{Trabalhucar}
\begin{itemize}
\item {Grp. gram.:v. i.}
\end{itemize}
\begin{itemize}
\item {Utilização:Prov.}
\end{itemize}
\begin{itemize}
\item {Utilização:trasm.}
\end{itemize}
O mesmo que \textunderscore trabalhar\textunderscore .
\section{Trábea}
\begin{itemize}
\item {Grp. gram.:f.}
\end{itemize}
\begin{itemize}
\item {Proveniência:(Lat. \textunderscore trabea\textunderscore )}
\end{itemize}
Espécie de toga branca com listras encarnadas, usada por alguns dos antigos Romanos. Cf. C. Lobo, \textunderscore Sát. de Jud.\textunderscore , I, 48.
\section{Trabécula}
\begin{itemize}
\item {Grp. gram.:f.}
\end{itemize}
\begin{itemize}
\item {Utilização:Anat.}
\end{itemize}
\begin{itemize}
\item {Proveniência:(Lat. \textunderscore trabécula\textunderscore )}
\end{itemize}
Pequena trave.
Cada um dos filamentos cruzados, de que se compõe a substância esponjosa, areolar ou reticular, do interior dos ossos.
\section{Trabela}
\begin{itemize}
\item {Grp. gram.:f.}
\end{itemize}
\begin{itemize}
\item {Utilização:Prov.}
\end{itemize}
\begin{itemize}
\item {Utilização:trasm.}
\end{itemize}
Bicho, que dá no milhão.
(Relaciona-se com \textunderscore trabécula\textunderscore )
\section{Trabelho}
\begin{itemize}
\item {fónica:bê}
\end{itemize}
\begin{itemize}
\item {Grp. gram.:m.}
\end{itemize}
\begin{itemize}
\item {Proveniência:(Do lat. \textunderscore trabécula\textunderscore )}
\end{itemize}
Peça de madeira, com que se torce a corda da serra, para a retesar.
Peça de xadrez.
Peia.
\section{Trabola}
\begin{itemize}
\item {Grp. gram.:m.}
\end{itemize}
\begin{itemize}
\item {Utilização:Prov.}
\end{itemize}
\begin{itemize}
\item {Utilização:beir.}
\end{itemize}
O mesmo que \textunderscore trapóla\textunderscore .
\section{Trabucada}
\begin{itemize}
\item {Grp. gram.:f.}
\end{itemize}
\begin{itemize}
\item {Utilização:Ext.}
\end{itemize}
\begin{itemize}
\item {Proveniência:(De \textunderscore trabuco\textunderscore )}
\end{itemize}
Ruído, produzido pelo rodar da antiga máquina chamada trabuco.
Ruído, estrondo.
\section{Trabucador}
\begin{itemize}
\item {Grp. gram.:m.  e  adj.}
\end{itemize}
O que trabuca.
\section{Trabucar}
\begin{itemize}
\item {Grp. gram.:v. t.}
\end{itemize}
\begin{itemize}
\item {Grp. gram.:V. i.}
\end{itemize}
Atacar, lançando pedras.
Desmoronar.
Fazer ir a pique (um navio).
Agitar.
Trabalhar insistentemente.
Ir a pique.
Fazer barulho; fazer estrondo, martelando, ou dando pancadas repetidas em substância rija. Cf. Bernárdez, \textunderscore Luz e Calor\textunderscore , 12.
(Cast. \textunderscore trabucar\textunderscore )
\section{Trabuco}
\begin{itemize}
\item {Grp. gram.:m.}
\end{itemize}
Antiga máquina de guerra, com que se atiravam pedras.
Espécie de bacamarte.
(Cast. \textunderscore trabuco\textunderscore )
\section{Trabul}
\begin{itemize}
\item {Grp. gram.:m.}
\end{itemize}
Estrado rectangular, donde se ergue o eixo da roda em que trabalha o olleiro.
(Cp. \textunderscore trabulo\textunderscore )
\section{Trabula}
\begin{itemize}
\item {Grp. gram.:f.}
\end{itemize}
O mesmo que \textunderscore trabulo\textunderscore .
\section{Trabulo}
\begin{itemize}
\item {Grp. gram.:m.}
\end{itemize}
\begin{itemize}
\item {Utilização:Prov.}
\end{itemize}
\begin{itemize}
\item {Utilização:Prov.}
\end{itemize}
\begin{itemize}
\item {Utilização:beir.}
\end{itemize}
O pecíolo da couve ou de outra planta herbácea.
O mesmo que \textunderscore trabul\textunderscore .
Cana de milho, depois de separada da espiga.
(Dem. do lat. \textunderscore trabs\textunderscore ?)
\section{Trabuqueiro}
\begin{itemize}
\item {Grp. gram.:m.}
\end{itemize}
\begin{itemize}
\item {Proveniência:(De \textunderscore trabuco\textunderscore )}
\end{itemize}
Salteador, armado de trabuco.
\section{Trabuquete}
\begin{itemize}
\item {fónica:quê}
\end{itemize}
\begin{itemize}
\item {Grp. gram.:m.}
\end{itemize}
Pequeno trabuco.
\section{Trabuquete}
\begin{itemize}
\item {fónica:quê}
\end{itemize}
\begin{itemize}
\item {Grp. gram.:m.}
\end{itemize}
\begin{itemize}
\item {Utilização:Ant.}
\end{itemize}
\begin{itemize}
\item {Proveniência:(Fr. ant. \textunderscore trabuquet\textunderscore , mod. \textunderscore trebouchet\textunderscore )}
\end{itemize}
Casa de câmbio, onde se trocava ou rebatia qualquer qualidade de moéda, com desconto ou ágio em favor do thesoiro público.
\section{Trabuzana}
\begin{itemize}
\item {Grp. gram.:f.}
\end{itemize}
\begin{itemize}
\item {Utilização:Pop.}
\end{itemize}
Tempestade.
Incômmodo ou doença.
Melancolia.
Indigestão.
Bebedeira.
\section{Traça}
\begin{itemize}
\item {Grp. gram.:m.}
\end{itemize}
\begin{itemize}
\item {Utilização:Fig.}
\end{itemize}
Acto ou effeito de traçar^1.
Esbôço.
Plano; desenho.
Maneira.
Organização.
Manha, ardil, astúcia.
\section{Traça}
\begin{itemize}
\item {Utilização:Fig.}
\end{itemize}
\begin{itemize}
\item {Utilização:Fam.}
\end{itemize}
\textunderscore f.\textunderscore 
Pequeno insecto roedor.
Aquillo que destrói a pouco e pouco.
Pessôa maçadora.
\section{Traçado}
\begin{itemize}
\item {Grp. gram.:m.}
\end{itemize}
\begin{itemize}
\item {Proveniência:(De \textunderscore traçar\textunderscore ^1)}
\end{itemize}
Traçamento.
Relatório escrito, em Maçonaria.
\section{Traçado}
\begin{itemize}
\item {Grp. gram.:m.}
\end{itemize}
\begin{itemize}
\item {Utilização:Bras}
\end{itemize}
\begin{itemize}
\item {Proveniência:(De \textunderscore traçar\textunderscore ^2)}
\end{itemize}
Lona estreita, para velame.
\section{Traçador}
\begin{itemize}
\item {Grp. gram.:m.  e  adj.}
\end{itemize}
\begin{itemize}
\item {Proveniência:(De \textunderscore traçar\textunderscore ^1)}
\end{itemize}
Aquelle que traça.
\section{Tracajá}
\begin{itemize}
\item {Grp. gram.:m.}
\end{itemize}
\begin{itemize}
\item {Utilização:Bras}
\end{itemize}
Espécie de reptil, congênere da tartaruga.
(Do tupi)
\section{Tracalhaz}
\begin{itemize}
\item {Grp. gram.:m.}
\end{itemize}
\begin{itemize}
\item {Utilização:Pop.}
\end{itemize}
Grande fatia ou grande porção; naco.
(Por \textunderscore trancalhaz\textunderscore , de \textunderscore tranca\textunderscore )
\section{Tracalheiro}
\begin{itemize}
\item {Grp. gram.:m.  e  adj.}
\end{itemize}
\begin{itemize}
\item {Utilização:Prov.}
\end{itemize}
\begin{itemize}
\item {Utilização:trasm.}
\end{itemize}
O que faz intrigas ou mexericos.
\section{Tracalhice}
\begin{itemize}
\item {Grp. gram.:f.}
\end{itemize}
\begin{itemize}
\item {Utilização:Prov.}
\end{itemize}
\begin{itemize}
\item {Utilização:trasm.}
\end{itemize}
Mexerico, intriga.
\section{Traçalho}
\begin{itemize}
\item {Grp. gram.:m.}
\end{itemize}
\begin{itemize}
\item {Utilização:Prov.}
\end{itemize}
\begin{itemize}
\item {Utilização:bras}
\end{itemize}
\begin{itemize}
\item {Utilização:Bras}
\end{itemize}
\begin{itemize}
\item {Proveniência:(De \textunderscore traçar\textunderscore ^2)}
\end{itemize}
Naco, pedaço de pão ou carne.
Pedaço de carne retalhada, já sêca ou que está secando ao sol.
\section{Tracambista}
\begin{itemize}
\item {Grp. gram.:m.}
\end{itemize}
\begin{itemize}
\item {Utilização:Bras}
\end{itemize}
\begin{itemize}
\item {Proveniência:(De \textunderscore tra...\textunderscore  + \textunderscore câmbio\textunderscore )}
\end{itemize}
Tratante, biltre, trocatintas.
\section{Traçamento}
\begin{itemize}
\item {Grp. gram.:m.}
\end{itemize}
Acto ou effeito de traçar^1.
\section{Tracanaz}
\begin{itemize}
\item {Grp. gram.:m.}
\end{itemize}
O mesmo que \textunderscore tracalhaz\textunderscore .
\section{Tração}
\begin{itemize}
\item {Grp. gram.:m.}
\end{itemize}
\begin{itemize}
\item {Utilização:Ant.}
\end{itemize}
\begin{itemize}
\item {Proveniência:(De \textunderscore traçar\textunderscore ^1)}
\end{itemize}
Lineamento, perfil.
\section{Tração}
\begin{itemize}
\item {Grp. gram.:adj.}
\end{itemize}
\begin{itemize}
\item {Utilização:Açor}
\end{itemize}
Intrigante, mexeriqueiro.
\section{Tração}
\begin{itemize}
\item {Grp. gram.:m.}
\end{itemize}
\begin{itemize}
\item {Utilização:Ant.}
\end{itemize}
\begin{itemize}
\item {Proveniência:(De \textunderscore traçar\textunderscore ^2)}
\end{itemize}
Bocado, fragmento.
\section{Traçar}
\begin{itemize}
\item {Grp. gram.:v. t.}
\end{itemize}
\begin{itemize}
\item {Proveniência:(Do lat. hyp. \textunderscore tractiare\textunderscore )}
\end{itemize}
Representar ou fazer por meio de traços: \textunderscore traçar uma caricatura\textunderscore .
Dar traços em.
Descrever: \textunderscore traçar um projecto\textunderscore .
Escrever.
Delinear.
Projectar.
Suppor.
Assinalar.
Tramar; urdir, planear.
Pôr a tiracollo.
Decidir.
Desfazer com traços ou golpes; espatifar, partir em pedaços, partir ao meio: \textunderscore as rodas do combóio traçaram-no\textunderscore .
\section{Traçar}
\begin{itemize}
\item {Grp. gram.:v. t.}
\end{itemize}
\begin{itemize}
\item {Utilização:Fig.}
\end{itemize}
\begin{itemize}
\item {Utilização:Açor}
\end{itemize}
\begin{itemize}
\item {Grp. gram.:V. i.  e  p.}
\end{itemize}
\begin{itemize}
\item {Proveniência:(De \textunderscore traça\textunderscore ^2)}
\end{itemize}
Roer, corroer, (falando-se da traça^2).
Cortar.
Gastar.
Affligir.
Misturar.
Sêr corroído pela traça.
\section{Tracção}
\begin{itemize}
\item {Grp. gram.:f.}
\end{itemize}
\begin{itemize}
\item {Proveniência:(Do lat. \textunderscore tractio\textunderscore )}
\end{itemize}
Acção de uma fôrça que desloca um objecto móvel, por meio de corda ou de outra coisa intermediária.
Acto de deslocar.
\section{Trace}
\begin{itemize}
\item {Grp. gram.:m.}
\end{itemize}
\begin{itemize}
\item {Proveniência:(Lat. \textunderscore Thraces\textunderscore )}
\end{itemize}
O mesmo ou melhor que \textunderscore trácio\textunderscore :«\textunderscore Gregos, Traces, Armenios...\textunderscore »\textunderscore Lusíadas\textunderscore .
\section{Tracejamento}
\begin{itemize}
\item {Grp. gram.:m.}
\end{itemize}
Acto de tracejar.
\section{Tracejar}
\begin{itemize}
\item {Grp. gram.:v. i.}
\end{itemize}
\begin{itemize}
\item {Grp. gram.:V. t.}
\end{itemize}
\begin{itemize}
\item {Proveniência:(De \textunderscore traço\textunderscore )}
\end{itemize}
Fazer traços.
Formar com pequenos traços, postos uns adeante dos outros.
Descrever ligeiramente.
\section{Tracelete}
\begin{itemize}
\item {fónica:lê}
\end{itemize}
\begin{itemize}
\item {Grp. gram.:m.}
\end{itemize}
\begin{itemize}
\item {Proveniência:(Fr. \textunderscore tracelet\textunderscore )}
\end{itemize}
Espécie de puncção, com que se traçam as divisões dos instrumentos de Mathemática.
\section{Tracheal}
\begin{itemize}
\item {fónica:quê}
\end{itemize}
\begin{itemize}
\item {Grp. gram.:adj.}
\end{itemize}
Relativo á tracheia.
\section{Tracheano}
\begin{itemize}
\item {fónica:que}
\end{itemize}
\begin{itemize}
\item {Grp. gram.:adj.}
\end{itemize}
Que tem tracheias.
Relativo a tracheia.
\section{Tracheia}
\begin{itemize}
\item {fónica:quei}
\end{itemize}
\begin{itemize}
\item {Grp. gram.:f.}
\end{itemize}
\begin{itemize}
\item {Utilização:Anat.}
\end{itemize}
\begin{itemize}
\item {Utilização:Bot.}
\end{itemize}
\begin{itemize}
\item {Proveniência:(Gr. \textunderscore trakheia\textunderscore )}
\end{itemize}
Canal, que estabelece communicação entre a larynge e os brônchios, e dá passagem ao ar.
Cada um dos canaes que, nos insectos, levam o ar a todas as partes do corpo.
Cada um dos vasos, que são compostos de céllulas sobrepostas, ligadas por extremidades cónicas.
\section{Tracheia-artéria}
\begin{itemize}
\item {fónica:quei}
\end{itemize}
\begin{itemize}
\item {Grp. gram.:f.}
\end{itemize}
O mesmo que a tracheia do corpo humano.
\section{Tracheíte}
\begin{itemize}
\item {fónica:que}
\end{itemize}
\begin{itemize}
\item {Grp. gram.:f.}
\end{itemize}
Inflammação da tracheia.
\section{Trachélia}
\begin{itemize}
\item {fónica:qué}
\end{itemize}
\begin{itemize}
\item {Grp. gram.:f.}
\end{itemize}
\begin{itemize}
\item {Proveniência:(Do gr. \textunderscore trakhelos\textunderscore )}
\end{itemize}
Gênero de insectos coleópteros, a que pertence a canthárida.
Gênero de plantas campanuláceas.
\section{Tracheliano}
\begin{itemize}
\item {fónica:que}
\end{itemize}
\begin{itemize}
\item {Grp. gram.:adj.}
\end{itemize}
\begin{itemize}
\item {Utilização:Anat.}
\end{itemize}
\begin{itemize}
\item {Proveniência:(Do gr. \textunderscore trakhelos\textunderscore )}
\end{itemize}
Relativo á parte posterior do pescoço.
\section{Trachelíneo}
\begin{itemize}
\item {fónica:que}
\end{itemize}
\begin{itemize}
\item {Grp. gram.:adj.}
\end{itemize}
\begin{itemize}
\item {Grp. gram.:M. pl.}
\end{itemize}
Relativo ou semelhante á trachélia, insecto.
Família de insectos, que tem por typo a trachélia.
\section{Trachélio}
\begin{itemize}
\item {fónica:qué}
\end{itemize}
\begin{itemize}
\item {Grp. gram.:m.}
\end{itemize}
Gênero de plantas campanuláceas, também conhecido por trachélia.
Gênero de insectos, o mesmo que \textunderscore trachélia\textunderscore .
\section{Trachelípode}
\begin{itemize}
\item {fónica:que}
\end{itemize}
\begin{itemize}
\item {Grp. gram.:adj.}
\end{itemize}
\begin{itemize}
\item {Utilização:Zool.}
\end{itemize}
\begin{itemize}
\item {Proveniência:(Do gr. \textunderscore trakhelos\textunderscore  + \textunderscore pous\textunderscore )}
\end{itemize}
Que tem os pés adherentes á base do pescoço.
\section{Trachelismo}
\begin{itemize}
\item {fónica:que}
\end{itemize}
\begin{itemize}
\item {Grp. gram.:m.}
\end{itemize}
\begin{itemize}
\item {Proveniência:(Do gr. \textunderscore trakhelos\textunderscore )}
\end{itemize}
Contracção espasmódica dos músculos do pescoço.
\section{Tracheobronchite}
\begin{itemize}
\item {fónica:quei,qui}
\end{itemize}
\begin{itemize}
\item {Grp. gram.:f.}
\end{itemize}
\begin{itemize}
\item {Utilização:Med.}
\end{itemize}
\begin{itemize}
\item {Proveniência:(De \textunderscore tracheia\textunderscore  + \textunderscore brônchio\textunderscore )}
\end{itemize}
Inflammação simultânea da tracheia e dos bronchios.
\section{Tracheocele}
\begin{itemize}
\item {fónica:que}
\end{itemize}
\begin{itemize}
\item {Grp. gram.:f.}
\end{itemize}
Tumor na tracheia.
\section{Tracheorrhagia}
\begin{itemize}
\item {fónica:que}
\end{itemize}
\begin{itemize}
\item {Grp. gram.:f.}
\end{itemize}
\begin{itemize}
\item {Proveniência:(Do gr. \textunderscore trakheia\textunderscore  + \textunderscore ragumni\textunderscore )}
\end{itemize}
Derramamento de sangue pela tracheia.
\section{Tracheorrhágico}
\begin{itemize}
\item {fónica:que}
\end{itemize}
\begin{itemize}
\item {Grp. gram.:adj.}
\end{itemize}
Relativo á tracheorrhagia.
\section{Tracheoscopia}
\begin{itemize}
\item {fónica:que}
\end{itemize}
\begin{itemize}
\item {Grp. gram.:f.}
\end{itemize}
\begin{itemize}
\item {Utilização:Med.}
\end{itemize}
\begin{itemize}
\item {Proveniência:(Do gr. \textunderscore trakheia\textunderscore  + \textunderscore skopein\textunderscore )}
\end{itemize}
Exame da cavidade da tracheia.
\section{Tracheostenose}
\begin{itemize}
\item {fónica:que}
\end{itemize}
\begin{itemize}
\item {Grp. gram.:f.}
\end{itemize}
\begin{itemize}
\item {Utilização:Med.}
\end{itemize}
\begin{itemize}
\item {Proveniência:(Do gr. \textunderscore trakheia\textunderscore  + \textunderscore stenos\textunderscore )}
\end{itemize}
Contracção da tracheia.
\section{Tracheotomia}
\begin{itemize}
\item {fónica:que}
\end{itemize}
\begin{itemize}
\item {Grp. gram.:f.}
\end{itemize}
\begin{itemize}
\item {Proveniência:(Do gr. \textunderscore trakheia\textunderscore  + \textunderscore tome\textunderscore )}
\end{itemize}
Operação cirúrgica, com que se estabelece communicação entre a tracheia e o exterior.
\section{Tracheotómico}
\begin{itemize}
\item {fónica:que}
\end{itemize}
\begin{itemize}
\item {Grp. gram.:adj.}
\end{itemize}
Relativo á tracheotomia.
\section{Trachinídeo}
\begin{itemize}
\item {fónica:qui}
\end{itemize}
\begin{itemize}
\item {Grp. gram.:adj.}
\end{itemize}
\begin{itemize}
\item {Grp. gram.:M. Pl.}
\end{itemize}
\begin{itemize}
\item {Proveniência:(De \textunderscore trachino\textunderscore )}
\end{itemize}
Relativo ou semelhante ao dragão-marinho.
Grupo de peixes, que tem por typo o dragão-marinho.
\section{Trachino}
\begin{itemize}
\item {fónica:qui}
\end{itemize}
\begin{itemize}
\item {Grp. gram.:m.}
\end{itemize}
\begin{itemize}
\item {Proveniência:(Do gr. \textunderscore trakhus\textunderscore )}
\end{itemize}
Nome scientífico do dragão-marinho.
\section{Trachoma}
\begin{itemize}
\item {fónica:cô}
\end{itemize}
\begin{itemize}
\item {Grp. gram.:m.}
\end{itemize}
\begin{itemize}
\item {Utilização:Med.}
\end{itemize}
\begin{itemize}
\item {Proveniência:(Gr. \textunderscore trakhoma\textunderscore )}
\end{itemize}
Ophtalmia, acompanhada de aspereza na parte interior das pálpebras.
\section{Trachycardia}
\begin{itemize}
\item {fónica:qui}
\end{itemize}
\begin{itemize}
\item {Grp. gram.:f.}
\end{itemize}
\begin{itemize}
\item {Proveniência:(Do gr. \textunderscore trakhus\textunderscore  + \textunderscore kardia\textunderscore )}
\end{itemize}
Pulsação rápida do coração.
\section{Trachycardíaco}
\begin{itemize}
\item {fónica:qui}
\end{itemize}
\begin{itemize}
\item {Grp. gram.:adj.}
\end{itemize}
Relativo á trachycardia.
Que soffre trachycardia.
\section{Trachýtico}
\begin{itemize}
\item {fónica:quí}
\end{itemize}
\begin{itemize}
\item {Grp. gram.:adj.}
\end{itemize}
Relativo ao trachyto.
\section{Trachyto}
\begin{itemize}
\item {fónica:qui}
\end{itemize}
\begin{itemize}
\item {Grp. gram.:m.}
\end{itemize}
\begin{itemize}
\item {Utilização:Miner.}
\end{itemize}
\begin{itemize}
\item {Proveniência:(Do gr. \textunderscore trakhus\textunderscore )}
\end{itemize}
Feldspatho de rochas vulcânicas.
\section{Trachytóide}
\begin{itemize}
\item {fónica:qui}
\end{itemize}
\begin{itemize}
\item {Grp. gram.:adj.}
\end{itemize}
\begin{itemize}
\item {Utilização:Miner.}
\end{itemize}
Semelhante ou parecido com o trachyto.
\section{Trachytoporphýrico}
\begin{itemize}
\item {fónica:qui}
\end{itemize}
\begin{itemize}
\item {Grp. gram.:adj.}
\end{itemize}
Que participa da natureza do trachyto e do pórphyro.
\section{Tracicomido}
\begin{itemize}
\item {Grp. gram.:adj.}
\end{itemize}
Comido ou roído pela traça. Cf. Filinto, V, 18.
\section{Trácio}
\begin{itemize}
\item {Grp. gram.:adj.}
\end{itemize}
\begin{itemize}
\item {Grp. gram.:M. pl.}
\end{itemize}
\begin{itemize}
\item {Proveniência:(Lat. \textunderscore thracius\textunderscore )}
\end{itemize}
Relativo á Thrácia,
Habitantes da Thrácia.
\section{Tracista}
\begin{itemize}
\item {Grp. gram.:m. ,  f.  e  adj.}
\end{itemize}
\begin{itemize}
\item {Proveniência:(De \textunderscore traçar\textunderscore ^1)}
\end{itemize}
Pessôa, que faz traços.
Pessôa que faz planos ou dá alvitres.
\section{Traço}
\begin{itemize}
\item {Grp. gram.:m.}
\end{itemize}
\begin{itemize}
\item {Utilização:Fig.}
\end{itemize}
\begin{itemize}
\item {Utilização:Prov.}
\end{itemize}
\begin{itemize}
\item {Utilização:minh.}
\end{itemize}
Acto ou effeito de traçar^1.
Linha, que se descreve ou se traça, por meio de lápis, pincel, etc.
Lineamento; feição.
Trecho, excerpto.
Vestígio; rasto.
Carácter.
Parte de qualquer coisa, cortada em sentido transversal: \textunderscore um traço de pescada\textunderscore ; \textunderscore um traço de madeira\textunderscore .
\section{Tracoma}
\begin{itemize}
\item {Grp. gram.:m.}
\end{itemize}
\begin{itemize}
\item {Utilização:Med.}
\end{itemize}
\begin{itemize}
\item {Proveniência:(Gr. \textunderscore trakhoma\textunderscore )}
\end{itemize}
Oftalmia, acompanhada de aspereza na parte interior das pálpebras.
\section{Tracónico}
\begin{itemize}
\item {Grp. gram.:adj.}
\end{itemize}
\begin{itemize}
\item {Utilização:Des.}
\end{itemize}
\begin{itemize}
\item {Proveniência:(Do lat. \textunderscore thracus\textunderscore )}
\end{itemize}
Traidor; velhaco.
\section{Traconismo}
\begin{itemize}
\item {Grp. gram.:m.}
\end{itemize}
\begin{itemize}
\item {Utilização:Des.}
\end{itemize}
Perfídia; velhacaria.--Moraes traz \textunderscore thrasonismo\textunderscore , certamente por êrro tipográfico.
(Cp. \textunderscore tracónico\textunderscore )
\section{Tracto}
\begin{itemize}
\item {Grp. gram.:m.}
\end{itemize}
\begin{itemize}
\item {Proveniência:(Lat. \textunderscore tractus\textunderscore )}
\end{itemize}
Espaço (de terreno).
Região.
Intervallo; decurso.
\section{Tractório}
\begin{itemize}
\item {Grp. gram.:adj.}
\end{itemize}
\begin{itemize}
\item {Proveniência:(Do lat. \textunderscore tractus\textunderscore )}
\end{itemize}
Relativo a tracção.
\section{Tracuans}
\begin{itemize}
\item {Grp. gram.:m. pl.}
\end{itemize}
O mesmo que \textunderscore imbé\textunderscore .
\section{Tradear}
\begin{itemize}
\item {Grp. gram.:v. t.}
\end{itemize}
Furar com trado.
\section{Tradela}
\begin{itemize}
\item {Grp. gram.:f.}
\end{itemize}
\begin{itemize}
\item {Utilização:Prov.}
\end{itemize}
\begin{itemize}
\item {Utilização:beir.}
\end{itemize}
\begin{itemize}
\item {Proveniência:(De \textunderscore trado\textunderscore )}
\end{itemize}
O mesmo que \textunderscore verruma\textunderscore .
\section{Tradescância}
\begin{itemize}
\item {Grp. gram.:f.}
\end{itemize}
\begin{itemize}
\item {Proveniência:(De \textunderscore Tradescant\textunderscore , n. p.)}
\end{itemize}
Planta commelínea, o mesmo que \textunderscore ephemerina\textunderscore .
\section{Tradição}
\begin{itemize}
\item {Grp. gram.:f.}
\end{itemize}
\begin{itemize}
\item {Proveniência:(Do lat. \textunderscore traditio\textunderscore )}
\end{itemize}
Acto de entregar ou transmittir.
Entrega.
Transmissão de factos históricos, systemas, lendas, etc., de idade em idade, sem prova authêntica ou escrita.
Tudo que se sabe ou se pratíca, provindo da transmissão oral ou de hábitos inveterados.
Recordação, memória: \textunderscore acatar as tradições de família\textunderscore .
\section{Tradicional}
\begin{itemize}
\item {Grp. gram.:adj.}
\end{itemize}
\begin{itemize}
\item {Proveniência:(Do lat. \textunderscore traditio\textunderscore )}
\end{itemize}
Relativo á tradição: \textunderscore narrativas tradicionaes\textunderscore .
\section{Tradicionalismo}
\begin{itemize}
\item {Grp. gram.:m.}
\end{itemize}
\begin{itemize}
\item {Proveniência:(De \textunderscore tradicional\textunderscore )}
\end{itemize}
Afêrro ás tradições ou usos antigos.
Qualidade de quem é desaffeiçoado ás ideias de progresso.
\section{Tradicionalista}
\begin{itemize}
\item {Grp. gram.:m.  e  f.}
\end{itemize}
\begin{itemize}
\item {Proveniência:(De \textunderscore tradicional\textunderscore )}
\end{itemize}
Pessôa, partidária dos systemas oppostos ás ideias de progresso.
Pessôa, que preza muito as tradições.
\section{Tradicionalmente}
\begin{itemize}
\item {Grp. gram.:adv.}
\end{itemize}
De modo tradicional.
\section{Tradicionário}
\begin{itemize}
\item {Grp. gram.:m.  e  adj.}
\end{itemize}
\begin{itemize}
\item {Proveniência:(Do lat. \textunderscore traditio\textunderscore , \textunderscore traditionis\textunderscore )}
\end{itemize}
O mesmo que \textunderscore tradicionalista\textunderscore .
\section{Tràdinha}
\begin{itemize}
\item {Grp. gram.:f.}
\end{itemize}
\begin{itemize}
\item {Utilização:Prov.}
\end{itemize}
\begin{itemize}
\item {Utilização:trasm.}
\end{itemize}
\begin{itemize}
\item {Proveniência:(De \textunderscore trado\textunderscore )}
\end{itemize}
Pequena verruma, tradela.
\section{Trado}
\begin{itemize}
\item {Grp. gram.:m.}
\end{itemize}
\begin{itemize}
\item {Proveniência:(Do lat. \textunderscore taratrum\textunderscore )}
\end{itemize}
Utensílio, usado especialmente por carpinteiros e tanoeiros, o qual tem a fórma de grande verruma.
Furo, aberto por êsse instrumento.
Mollusco gasterópode.
\section{Tradução}
\begin{itemize}
\item {Grp. gram.:f.}
\end{itemize}
\begin{itemize}
\item {Utilização:Fig.}
\end{itemize}
\begin{itemize}
\item {Proveniência:(Do lat. \textunderscore traductio\textunderscore )}
\end{itemize}
Acto ou efeito de traduzir.
Obra traduzida.
Reflexo, repercussão, imagem.
\section{Traducção}
\begin{itemize}
\item {Grp. gram.:f.}
\end{itemize}
\begin{itemize}
\item {Utilização:Fig.}
\end{itemize}
\begin{itemize}
\item {Proveniência:(Do lat. \textunderscore traductio\textunderscore )}
\end{itemize}
Acto ou effeito de traduzir.
Obra traduzida.
Reflexo, repercussão, imagem.
\section{Traductor}
\begin{itemize}
\item {Grp. gram.:m.  e  adj.}
\end{itemize}
\begin{itemize}
\item {Proveniência:(Lat. \textunderscore traductor\textunderscore )}
\end{itemize}
O que traduz.
\section{Tradutor}
\begin{itemize}
\item {Grp. gram.:m.  e  adj.}
\end{itemize}
\begin{itemize}
\item {Proveniência:(Lat. \textunderscore traductor\textunderscore )}
\end{itemize}
O que traduz.
\section{Traduzideiro}
\begin{itemize}
\item {Grp. gram.:m.  e  adj.}
\end{itemize}
O mesmo que \textunderscore traduzidor\textunderscore :«\textunderscore ...por entre o escarcéu e o vagalhão da imprensa traduzideira.\textunderscore »Castilho.
\section{Traduzidor}
\begin{itemize}
\item {Grp. gram.:m.  e  adj.}
\end{itemize}
\begin{itemize}
\item {Utilização:Deprec.}
\end{itemize}
\begin{itemize}
\item {Proveniência:(De \textunderscore traduzir\textunderscore )}
\end{itemize}
Mau traductor.
\section{Traduzir}
\begin{itemize}
\item {Grp. gram.:v. t.}
\end{itemize}
\begin{itemize}
\item {Proveniência:(Do lat. \textunderscore traducere\textunderscore )}
\end{itemize}
Fazer passar de uma língua para outra.
Trasladar, verter.
Interpretar: \textunderscore traduzir a intenção de outrem\textunderscore .
Revelar.
Representar.
Sêr o reflexo ou a imagem de.
\section{Traduzível}
\begin{itemize}
\item {Grp. gram.:adj.}
\end{itemize}
Que se póde traduzir.
\section{Traer}
\begin{itemize}
\item {Grp. gram.:v. t.}
\end{itemize}
\begin{itemize}
\item {Utilização:Ant.}
\end{itemize}
\begin{itemize}
\item {Proveniência:(Do lat. \textunderscore tradere\textunderscore )}
\end{itemize}
Entregar traiçoeiramente; atraiçoar.
\section{Tráfega}
\begin{itemize}
\item {Grp. gram.:f.}
\end{itemize}
\begin{itemize}
\item {Utilização:Prov.}
\end{itemize}
\begin{itemize}
\item {Utilização:trasm.}
\end{itemize}
Azáfama; afan.
(Cp. \textunderscore tráfego\textunderscore )
\section{Trafegante}
\begin{itemize}
\item {Grp. gram.:m.}
\end{itemize}
\begin{itemize}
\item {Utilização:T. de Turquel}
\end{itemize}
O mesmo que \textunderscore traficante\textunderscore .
\section{Trafegar}
\begin{itemize}
\item {Grp. gram.:v. i.}
\end{itemize}
\begin{itemize}
\item {Grp. gram.:V. t.}
\end{itemize}
\begin{itemize}
\item {Utilização:Des.}
\end{itemize}
\begin{itemize}
\item {Proveniência:(De \textunderscore tráfego\textunderscore )}
\end{itemize}
Praticar tráfego.
Afadigar-se; lidar.
Percorrer apressadamente.
\section{Tráfego}
\begin{itemize}
\item {Grp. gram.:m.}
\end{itemize}
O mesmo que \textunderscore tráfico\textunderscore .
Afan; trabalho.
Convivência.
Transporte de mercadorias nas linhas férreas.
Repartição ou pessoal, que se occupa dêsse transporte.
(Alter. de \textunderscore tráfico\textunderscore )
\section{Trafêgo}
\begin{itemize}
\item {Grp. gram.:m.}
\end{itemize}
\begin{itemize}
\item {Utilização:Pop.}
\end{itemize}
O mesmo que \textunderscore tráfego\textunderscore .
\section{Trafeguear}
\begin{itemize}
\item {Grp. gram.:v. i.}
\end{itemize}
O mesmo que \textunderscore trafegar\textunderscore .
\section{Trafegueiro}
\begin{itemize}
\item {Grp. gram.:m.}
\end{itemize}
\begin{itemize}
\item {Utilização:Prov.}
\end{itemize}
\begin{itemize}
\item {Utilização:minh.}
\end{itemize}
Cabeceira do lar, atrás da qual fica a borralheira.
(Alter. de \textunderscore trasfogueiro\textunderscore )
\section{Trafegueiro}
\begin{itemize}
\item {Grp. gram.:m.}
\end{itemize}
\begin{itemize}
\item {Utilização:Prov.}
\end{itemize}
\begin{itemize}
\item {Utilização:dur.}
\end{itemize}
\begin{itemize}
\item {Proveniência:(De \textunderscore trafegar\textunderscore )}
\end{itemize}
Pequeno barco, que, através de baixios, conduz parte da carga de barcos maiores, para que êstes possam vencer os pontos ou rápidos.
\section{Traficância}
\begin{itemize}
\item {Grp. gram.:f.}
\end{itemize}
\begin{itemize}
\item {Utilização:Pop.}
\end{itemize}
\begin{itemize}
\item {Proveniência:(De \textunderscore traficante\textunderscore )}
\end{itemize}
Acto ou effeito de traficar.
Tratantada.
Negócio fraudulento.
\section{Traficante}
\begin{itemize}
\item {Grp. gram.:m.  e  adj.}
\end{itemize}
\begin{itemize}
\item {Utilização:Pop.}
\end{itemize}
\begin{itemize}
\item {Utilização:Des.}
\end{itemize}
\begin{itemize}
\item {Proveniência:(De \textunderscore traficar\textunderscore )}
\end{itemize}
O que pratíca fraudes em negócios; tratante.
Negociante.
\section{Traficar}
\begin{itemize}
\item {Grp. gram.:v. t.  e  i.}
\end{itemize}
\begin{itemize}
\item {Utilização:Fam.}
\end{itemize}
\begin{itemize}
\item {Proveniência:(De \textunderscore tráfico\textunderscore )}
\end{itemize}
Commerciar.
Fazer negócios fraudulentos.
\section{Tráfico}
\begin{itemize}
\item {Grp. gram.:m.}
\end{itemize}
\begin{itemize}
\item {Utilização:Fam.}
\end{itemize}
Commércio; negociação.
Negócio indecoroso: \textunderscore o tráfico da escravatura\textunderscore .
\section{Trafoixo}
\begin{itemize}
\item {Grp. gram.:m.}
\end{itemize}
\begin{itemize}
\item {Utilização:T. da Bairrada}
\end{itemize}
Troixa pesada.
Inchaço.
Chumaço.
\section{Trafulha}
\begin{itemize}
\item {Grp. gram.:f.}
\end{itemize}
\begin{itemize}
\item {Utilização:Prov.}
\end{itemize}
\begin{itemize}
\item {Utilização:alg.}
\end{itemize}
O mesmo que \textunderscore trapaça\textunderscore .
\section{Tragacanta}
\begin{itemize}
\item {Grp. gram.:f.}
\end{itemize}
\begin{itemize}
\item {Proveniência:(Lat. \textunderscore tragacantha\textunderscore )}
\end{itemize}
Goma de tragacantho.
\section{Tragacantha}
\begin{itemize}
\item {Grp. gram.:f.}
\end{itemize}
\begin{itemize}
\item {Proveniência:(Lat. \textunderscore tragacantha\textunderscore )}
\end{itemize}
Goma de tragacantho.
\section{Tragacantho}
\begin{itemize}
\item {Grp. gram.:m.}
\end{itemize}
\begin{itemize}
\item {Proveniência:(Lat. \textunderscore tragacanthum\textunderscore )}
\end{itemize}
\begin{itemize}
\item {Proveniência:(Note-se que, ao invés do português, o lat. \textunderscore tragacantha\textunderscore  é planta, e \textunderscore tragacanthum\textunderscore  a goma da planta)
}
\end{itemize}
Nome de várias plantas do gênero astrálago.
\section{Tragacanto}
\begin{itemize}
\item {Grp. gram.:m.}
\end{itemize}
\begin{itemize}
\item {Proveniência:(Lat. \textunderscore tragacanthum\textunderscore )}
\end{itemize}
\begin{itemize}
\item {Proveniência:(Note-se que, ao invés do português, o lat. \textunderscore tragacantha\textunderscore  é planta, e \textunderscore tragacanthum\textunderscore  a goma da planta)
}
\end{itemize}
Nome de várias plantas do gênero astrálago.
\section{Tragadeiro}
\begin{itemize}
\item {Grp. gram.:m.}
\end{itemize}
\begin{itemize}
\item {Utilização:Pop.}
\end{itemize}
\begin{itemize}
\item {Utilização:Fig.}
\end{itemize}
\begin{itemize}
\item {Proveniência:(De \textunderscore tragar\textunderscore )}
\end{itemize}
Goelas.
Voragem.
\section{Tragadoiro}
\begin{itemize}
\item {Grp. gram.:m.}
\end{itemize}
\begin{itemize}
\item {Proveniência:(De \textunderscore tragar\textunderscore )}
\end{itemize}
O mesmo que \textunderscore sorvedoiro\textunderscore .
Voragem, abysmo.
\section{Tragador}
\begin{itemize}
\item {Grp. gram.:m.  e  adj.}
\end{itemize}
O que traga.
\section{Tragadouro}
\begin{itemize}
\item {Grp. gram.:m.}
\end{itemize}
\begin{itemize}
\item {Proveniência:(De \textunderscore tragar\textunderscore )}
\end{itemize}
O mesmo que \textunderscore sorvedouro\textunderscore .
Voragem, abysmo.
\section{Tragamalha}
\begin{itemize}
\item {Grp. gram.:m.}
\end{itemize}
\begin{itemize}
\item {Utilização:Ant.}
\end{itemize}
\begin{itemize}
\item {Utilização:Ant.}
\end{itemize}
\begin{itemize}
\item {Utilização:Fam.}
\end{itemize}
Pescador do mar alto, que pagava o tragamalho.
Homem muito falador, tagarela.
\section{Tragamalho}
\begin{itemize}
\item {Grp. gram.:m.}
\end{itemize}
Imposto, que ao desembarque pagavam á Camara Municipal os pescadores de Lisbôa. (F. Borges, \textunderscore Diccion. Jur.\textunderscore , inclina-se a que o nome viria do aluguel de um malho, para enterrar a estaca, a que os barcos se amarravam. Acrescente-se que, segundo outras informações, os barqueiros, para não alugar o malho, \textunderscore traziam-no\textunderscore ; donde o imperativo \textunderscore traga malho\textunderscore , de \textunderscore trazer\textunderscore  + \textunderscore malho\textunderscore . Cf. G. Viana, \textunderscore Apostillas\textunderscore ).
\section{Tragamento}
\begin{itemize}
\item {Grp. gram.:m.}
\end{itemize}
Acto ou effeito de tragar.
\section{Traga-moiros}
\begin{itemize}
\item {Grp. gram.:m.}
\end{itemize}
Valentão, fanfarrão; trancarruas.
\section{Tragano}
\begin{itemize}
\item {Grp. gram.:m.}
\end{itemize}
Gênero de plantas chenopódias.
\section{Tragantho}
\begin{itemize}
\item {Grp. gram.:m.}
\end{itemize}
\begin{itemize}
\item {Proveniência:(Do gr. \textunderscore tragos\textunderscore  + \textunderscore anthos\textunderscore )}
\end{itemize}
Gênero de plantas euphorbiáceas.
\section{Traganto}
\begin{itemize}
\item {Grp. gram.:m.}
\end{itemize}
\begin{itemize}
\item {Proveniência:(Do gr. \textunderscore tragos\textunderscore  + \textunderscore anthos\textunderscore )}
\end{itemize}
Gênero de plantas euforbiáceas.
\section{Tragar}
\begin{itemize}
\item {Grp. gram.:v. t.}
\end{itemize}
\begin{itemize}
\item {Utilização:Fig.}
\end{itemize}
\begin{itemize}
\item {Proveniência:(Do rad. do lat. \textunderscore tractus\textunderscore )}
\end{itemize}
Engulir com avidez ou sem mastigar; devorar.
Aguentar, tolerar.
Desejar com soffreguidão.
Ambicionar vivamente.
Absorver.
Derruír.
\section{Tragável}
\begin{itemize}
\item {Grp. gram.:adj.}
\end{itemize}
Que se póde tragar. Cf. Castilho, \textunderscore Fastos\textunderscore , II, 484.
\section{Tragédia}
\begin{itemize}
\item {Grp. gram.:f.}
\end{itemize}
\begin{itemize}
\item {Utilização:Fig.}
\end{itemize}
\begin{itemize}
\item {Proveniência:(Lat. \textunderscore tragaedia\textunderscore )}
\end{itemize}
Peça theatral, geralmente em verso, e que termina ordinariamente por um acontecimento funesto.
Arte de fazer ou representar tragédias.
Acontecimento que desperta piedade ou terror: \textunderscore actor, insignie na tragédia\textunderscore .
\section{Trager}
\textunderscore v. t.\textunderscore  (e der.)
Fórma ant. de \textunderscore trazer\textunderscore , etc.
\section{Trágia}
\begin{itemize}
\item {Grp. gram.:f.}
\end{itemize}
\begin{itemize}
\item {Proveniência:(Do gr. \textunderscore tragos\textunderscore )}
\end{itemize}
Gênero de plantas euphorbiáceas.
\section{Trágica}
\begin{itemize}
\item {Grp. gram.:f.}
\end{itemize}
\begin{itemize}
\item {Proveniência:(De \textunderscore trágico\textunderscore )}
\end{itemize}
Mulhér, que representa tragédias: \textunderscore a Ristori, grande trágica\textunderscore .
\section{Tragicamente}
\begin{itemize}
\item {Grp. gram.:adv.}
\end{itemize}
De modo trágico.
Funestamente.
A modo de catástrophe.
\section{Trágico}
\begin{itemize}
\item {Grp. gram.:adj.}
\end{itemize}
\begin{itemize}
\item {Utilização:Fig.}
\end{itemize}
\begin{itemize}
\item {Grp. gram.:M.}
\end{itemize}
\begin{itemize}
\item {Proveniência:(Lat. \textunderscore tragicus\textunderscore )}
\end{itemize}
Relativo a tragédia.
Sinistro; funesto: \textunderscore desenlace trágico\textunderscore .
Aquelle que faz ou representa tragédias.
\section{Tragicomédia}
\begin{itemize}
\item {Grp. gram.:f.}
\end{itemize}
\begin{itemize}
\item {Proveniência:(Lat. \textunderscore tragicomaedia\textunderscore )}
\end{itemize}
Peça theatral, que participa da tragédia pelo assumpto e personagens, e da comédia pelos incidentes e desenlace.
\section{Tragicómico}
\begin{itemize}
\item {Grp. gram.:adj.}
\end{itemize}
\begin{itemize}
\item {Proveniência:(De \textunderscore trágico\textunderscore  + \textunderscore cómico\textunderscore )}
\end{itemize}
Relativo á tragicomédia.
Funesto, mas acompanhado de incidentes cómicos.
\section{Tragimentos}
\begin{itemize}
\item {Grp. gram.:m. pl.}
\end{itemize}
\begin{itemize}
\item {Utilização:Ant.}
\end{itemize}
\begin{itemize}
\item {Proveniência:(De \textunderscore trager\textunderscore )}
\end{itemize}
Apontamentos, que os procuradores dos povos levavam ás Côrtes, para que o Rei tomasse dêlles conhecimento e provesse de justiça. Cf. S. R. Viterbo, \textunderscore Elucidário\textunderscore .
\section{Traginante}
\begin{itemize}
\item {Grp. gram.:m.}
\end{itemize}
\begin{itemize}
\item {Utilização:Des.}
\end{itemize}
\begin{itemize}
\item {Proveniência:(T. cast.)}
\end{itemize}
Aquelle que acarreta mercadorias; carreiro:«\textunderscore na estrada, o traginante cauteloso...\textunderscore »\textunderscore Viriato Trág.\textunderscore , 336.
\section{Trago}
\begin{itemize}
\item {Grp. gram.:m.}
\end{itemize}
\begin{itemize}
\item {Utilização:Fig.}
\end{itemize}
\begin{itemize}
\item {Proveniência:(De \textunderscore tragar\textunderscore )}
\end{itemize}
Sôrvo, gole.
O que se bebe de uma vez.
Afflicção.
Adversidade.
\section{Trago}
\begin{itemize}
\item {Grp. gram.:m.}
\end{itemize}
\begin{itemize}
\item {Utilização:Anat.}
\end{itemize}
\begin{itemize}
\item {Proveniência:(Do gr. \textunderscore tragos\textunderscore )}
\end{itemize}
Pequena saliência, que há á entrada do ouvido externo, e que se cobre de pêlos quando se chega a certa idade.
\section{Tragócero}
\begin{itemize}
\item {Grp. gram.:m.}
\end{itemize}
\begin{itemize}
\item {Proveniência:(Lat. \textunderscore tragoceros\textunderscore )}
\end{itemize}
Gênero de plantas, da fam. das compostas.
Gênero de insectos coleópteros.
\section{Tragopana}
\begin{itemize}
\item {Grp. gram.:f.}
\end{itemize}
\begin{itemize}
\item {Proveniência:(Do cast. \textunderscore tragopan\textunderscore )}
\end{itemize}
Gênero de áves gallináceas.
\section{Tragor}
\begin{itemize}
\item {Grp. gram.:m.}
\end{itemize}
\begin{itemize}
\item {Utilização:Prov.}
\end{itemize}
\begin{itemize}
\item {Utilização:minh.}
\end{itemize}
O mesmo que \textunderscore travor\textunderscore .
\section{Traguer}
\begin{itemize}
\item {Grp. gram.:v. t.}
\end{itemize}
\begin{itemize}
\item {Utilização:Prov.}
\end{itemize}
\begin{itemize}
\item {Utilização:beir.}
\end{itemize}
\begin{itemize}
\item {Utilização:Ant.}
\end{itemize}
O mesmo que \textunderscore trazer\textunderscore .--Na Beira-Baixa, ouve-se: \textunderscore eu trago\textunderscore , \textunderscore tu tragues\textunderscore , \textunderscore êlle trague...\textunderscore 
\section{Trágula}
\begin{itemize}
\item {Grp. gram.:f.}
\end{itemize}
\begin{itemize}
\item {Proveniência:(Lat. \textunderscore tragula\textunderscore )}
\end{itemize}
Espécie de dardo comprido, usado na antiguidade. Cf. Filinto, VI, 235.
\section{Trágus}
\begin{itemize}
\item {Grp. gram.:m.}
\end{itemize}
(V. \textunderscore trago\textunderscore ^2)
\section{Trahir}
\textunderscore v. t.\textunderscore  (e der.)
Fórma usual, em vez de \textunderscore traír\textunderscore , etc, mas insustentável, desde que se vê que a etym. não é o lat. \textunderscore trahere\textunderscore , mas, sim, \textunderscore tradere\textunderscore .
(Cp. \textunderscore traer\textunderscore )
\section{Traição}
\begin{itemize}
\item {Grp. gram.:f.}
\end{itemize}
\begin{itemize}
\item {Grp. gram.:Loc. adv.}
\end{itemize}
\begin{itemize}
\item {Proveniência:(Do lat. \textunderscore traditio\textunderscore )}
\end{itemize}
Acto ou effeito de traír; perfídia; infidelidade.
\textunderscore Á traição\textunderscore , traiçoeiramente.--A pronúncia primitiva seria \textunderscore tra-i-ção\textunderscore .
\section{Traiçoeiramente}
\begin{itemize}
\item {Grp. gram.:adv.}
\end{itemize}
De modo traiçoeiro; com traição; cobardemente.
\section{Traiçoeiro}
\begin{itemize}
\item {Grp. gram.:adj.}
\end{itemize}
Que atraiçoa: \textunderscore homem traiçoeiro\textunderscore .
Em que há traição: \textunderscore procedimento traiçoeiro\textunderscore .
Relativo a traição.
Pérfido; infiel.
\section{Traidor}
\begin{itemize}
\item {Grp. gram.:adj.}
\end{itemize}
\begin{itemize}
\item {Grp. gram.:M.}
\end{itemize}
\begin{itemize}
\item {Utilização:Gír.}
\end{itemize}
\begin{itemize}
\item {Proveniência:(Do lat. \textunderscore traditor\textunderscore )}
\end{itemize}
Traiçoeiro.
Perigoso.
Indivíduo, que atraiçôa.
Sapato.
\section{Traidora}
\begin{itemize}
\item {Grp. gram.:f.}
\end{itemize}
(Fem. de \textunderscore traidor\textunderscore )
\section{Traidoramente}
\begin{itemize}
\item {Grp. gram.:adv.}
\end{itemize}
\begin{itemize}
\item {Proveniência:(De \textunderscore traidor\textunderscore )}
\end{itemize}
O mesmo que \textunderscore traiçoeiramente\textunderscore . Cf. Filinto, \textunderscore D. Man.\textunderscore , II, 178.
\section{Traimento}
\begin{itemize}
\item {fónica:tra-i}
\end{itemize}
\begin{itemize}
\item {Grp. gram.:m.}
\end{itemize}
O mesmo que \textunderscore traição\textunderscore .
\section{Traina}
\begin{itemize}
\item {Grp. gram.:f.}
\end{itemize}
Rêde, de cincoenta braças de comprimento e oito de largura, com que na costa setentrional da Espanha se pesca sardinha e outros peixes.
Nome de outras rêdes, usadas por pescadores espanhóes.
(Cast. \textunderscore traina\textunderscore )
\section{Trainel}
\begin{itemize}
\item {Grp. gram.:m.}
\end{itemize}
\begin{itemize}
\item {Utilização:Prov.}
\end{itemize}
\begin{itemize}
\item {Utilização:beir.}
\end{itemize}
\begin{itemize}
\item {Utilização:Prov.}
\end{itemize}
\begin{itemize}
\item {Utilização:alent.}
\end{itemize}
Diz-se, em Náutica, \textunderscore costado do trainel\textunderscore , o costado atravessado ao correr da testa do pano.
Espécie de nó, com que se amarra provisoriamente um cabo.
Lanço ou seguimento de estrada.
Declive escoante, como no telhado.
(Cast. \textunderscore trainel\textunderscore )
\section{Trair}
\begin{itemize}
\item {Grp. gram.:v. t.}
\end{itemize}
\begin{itemize}
\item {Grp. gram.:V. p.}
\end{itemize}
\begin{itemize}
\item {Proveniência:(Do lat. \textunderscore tradere\textunderscore )}
\end{itemize}
Atraiçoar.
Não cumprir: \textunderscore trair os seus deveres\textunderscore .
Sêr infiel a.
Manifestar.
Denunciar.
Dar a entender involuntariamente: \textunderscore trair a sua intenção\textunderscore .
Falsear.
Descobrir involuntariamente (o que se devia ou se desejava occultar).
Comprometer-se.
\section{Traíra}
\begin{itemize}
\item {Grp. gram.:f.}
\end{itemize}
\begin{itemize}
\item {Utilização:Bras}
\end{itemize}
Peixe fluvial.
Variedade de reptil.
\section{Traita}
\begin{itemize}
\item {Grp. gram.:f.}
\end{itemize}
\begin{itemize}
\item {Utilização:Des.}
\end{itemize}
\begin{itemize}
\item {Utilização:Prov.}
\end{itemize}
\begin{itemize}
\item {Utilização:alg.}
\end{itemize}
Direcção do vôo de uma ave.
Abalada.
Vereda.
(Cp. \textunderscore traite\textunderscore )
\section{Traite}
\begin{itemize}
\item {Grp. gram.:m.}
\end{itemize}
\begin{itemize}
\item {Proveniência:(Fr. \textunderscore trait\textunderscore )}
\end{itemize}
Acto de cardar lan.
\section{Traites}
\begin{itemize}
\item {Grp. gram.:m. pl.}
\end{itemize}
\begin{itemize}
\item {Utilização:T. de Vimioso}
\end{itemize}
O jôgo das nécaras.
\section{Trajadura}
\begin{itemize}
\item {Grp. gram.:f.}
\end{itemize}
Variedade de uva branca do Minho.
\section{Trajar}
\begin{itemize}
\item {Grp. gram.:v. t.}
\end{itemize}
\begin{itemize}
\item {Grp. gram.:V. i.}
\end{itemize}
\begin{itemize}
\item {Grp. gram.:M.}
\end{itemize}
Empregar ou applicar como vestuário.
Vestir: \textunderscore trajar casaca\textunderscore .
Vestir-se; usar como vestuário; adornar-se: \textunderscore trajar de amazona\textunderscore .
Traje.
(B. lat. \textunderscore tragere\textunderscore )
\section{Traje}
\begin{itemize}
\item {Grp. gram.:f.}
\end{itemize}
\begin{itemize}
\item {Proveniência:(De \textunderscore trajar\textunderscore )}
\end{itemize}
Vestuário habitual.
Vestuário próprio de uma profissão.
Vestes.
Aquillo que se veste; fato. Cf. Filinto, \textunderscore D. Man.\textunderscore , I, 196.
\section{Trajecto}
\begin{itemize}
\item {Grp. gram.:m.}
\end{itemize}
\begin{itemize}
\item {Proveniência:(Lat. \textunderscore trajectus\textunderscore )}
\end{itemize}
Espaço, que alguém ou alguma coisa tem de percorrer, para passar de um lugar para outro.
\section{Trajectória}
\begin{itemize}
\item {Grp. gram.:f.}
\end{itemize}
\begin{itemize}
\item {Utilização:Fig.}
\end{itemize}
\begin{itemize}
\item {Proveniência:(De \textunderscore trajecto\textunderscore )}
\end{itemize}
Linha, descrita ou percorrida pelo centro de gravidade de um corpo em movimento.
Trajecto; meio; via.
\section{Trajo}
\begin{itemize}
\item {Grp. gram.:m.}
\end{itemize}
(V.traje)
\section{Traladar}
\textunderscore v. t. Ant.\textunderscore  (e der.)
O mesmo que \textunderscore trasladar\textunderscore , etc. Cf. \textunderscore Rev. Lus.\textunderscore , XVI, 11.
\section{Tralha}
\begin{itemize}
\item {Grp. gram.:f.}
\end{itemize}
\begin{itemize}
\item {Utilização:T. de Turquel}
\end{itemize}
\begin{itemize}
\item {Utilização:Gír.}
\end{itemize}
\begin{itemize}
\item {Proveniência:(Do lat. \textunderscore tragula\textunderscore )}
\end{itemize}
Pequena rêde, que póde sêr lançada ou armada por um homem só.
Malha de rêde.
Cabo, que guarnece as orlas do pano das velas.
Qualquer utensílio de trabalho.
Capote.
\section{Tralhado}
\begin{itemize}
\item {Grp. gram.:m.}
\end{itemize}
\begin{itemize}
\item {Utilização:Ant.}
\end{itemize}
O mesmo que \textunderscore traslado\textunderscore , cópia, exemplar.
\section{Tralhão}
\begin{itemize}
\item {Grp. gram.:m.}
\end{itemize}
\begin{itemize}
\item {Grp. gram.:Loc.}
\end{itemize}
\begin{itemize}
\item {Utilização:pop.}
\end{itemize}
O mesmo que \textunderscore taralhão\textunderscore .
\textunderscore Meter-se a tralhão\textunderscore , atrever-se, tomar confiança; sêr metediço.
\section{Tralhar}
\begin{itemize}
\item {Grp. gram.:v. t.}
\end{itemize}
Lançar tralha em.
\section{Tralhar}
\begin{itemize}
\item {Utilização:Prov.}
\end{itemize}
\begin{itemize}
\item {Utilização:trasm.}
\end{itemize}
Coagular, solidificar.
\section{Tralhas-malhas}
\begin{itemize}
\item {Grp. gram.:f. pl.}
\end{itemize}
\begin{itemize}
\item {Utilização:Prov.}
\end{itemize}
\textunderscore Por tralhas-malhas\textunderscore , manhosamente, astutamente.--Camillo, escreveu \textunderscore por tralhas ou malhas\textunderscore . Cf. \textunderscore Onde está a Felic.\textunderscore , 33.
\section{Tralheta}
\begin{itemize}
\item {fónica:lhê}
\end{itemize}
\begin{itemize}
\item {Grp. gram.:f.}
\end{itemize}
\begin{itemize}
\item {Utilização:Prov.}
\end{itemize}
\begin{itemize}
\item {Utilização:trasm.}
\end{itemize}
Rapariga tagarela e leviana.
(Cp. \textunderscore tralhão\textunderscore )
\section{Tralho}
\begin{itemize}
\item {Grp. gram.:m.}
\end{itemize}
O mesmo que \textunderscore tralha\textunderscore , rêde.
\section{Tralhoada}
\begin{itemize}
\item {Grp. gram.:f.}
\end{itemize}
\begin{itemize}
\item {Utilização:T. de Ribatejo}
\end{itemize}
\begin{itemize}
\item {Utilização:Prov.}
\end{itemize}
\begin{itemize}
\item {Utilização:alg.}
\end{itemize}
Grande porção de miudezas; trapalhada; salgalhada.
Três juntas de bois, ou três cingéis, que puxam a mesma carrêta, zorra, etc.
Sova, pancadaria.
\section{Tralhoto}
\begin{itemize}
\item {fónica:lhô}
\end{itemize}
\begin{itemize}
\item {Grp. gram.:m.}
\end{itemize}
\begin{itemize}
\item {Utilização:Bras. do N}
\end{itemize}
Espécie de peixe marítimo, que vive habitualmente á tona da água.
\section{Trama}
\begin{itemize}
\item {Grp. gram.:f.}
\end{itemize}
\begin{itemize}
\item {Grp. gram.:M.  e  f.}
\end{itemize}
\begin{itemize}
\item {Utilização:Fig.}
\end{itemize}
\begin{itemize}
\item {Utilização:Bras. do N}
\end{itemize}
\begin{itemize}
\item {Proveniência:(Lat. \textunderscore trama\textunderscore )}
\end{itemize}
Fio, que se conduz com a lançadeira através do urdume da teia.
Fios de seda grossa.
Fio grosso.
Tecido.
Intriga; procedimento ardiloso.
Ladroeira; lôgro.
\section{Trama}
\begin{itemize}
\item {Grp. gram.:m.}
\end{itemize}
\begin{itemize}
\item {Utilização:Ant.}
\end{itemize}
Peste.
Inchaço.
Doença. Cf. \textunderscore Leal Conselheiro\textunderscore .
(Talvez se relacione com o lat. \textunderscore struma\textunderscore )
\section{Tramador}
\begin{itemize}
\item {Grp. gram.:m.  e  adj.}
\end{itemize}
O que trama.
\section{Tramaga}
\begin{itemize}
\item {Grp. gram.:f.}
\end{itemize}
O mesmo que \textunderscore tramagueira\textunderscore .
\section{Tramagal}
\begin{itemize}
\item {Grp. gram.:m.}
\end{itemize}
\begin{itemize}
\item {Utilização:Prov.}
\end{itemize}
\begin{itemize}
\item {Utilização:beir.}
\end{itemize}
Campo de tramargas.
Rapaz ou rapariga de pouco juízo; tarau.
\section{Tramagueira}
\begin{itemize}
\item {Grp. gram.:f.}
\end{itemize}
\begin{itemize}
\item {Utilização:Pop.}
\end{itemize}
Planta, o mesmo que \textunderscore tamargueira\textunderscore .
(Methát. de \textunderscore tamargueira\textunderscore )
\section{Tramar}
\begin{itemize}
\item {Grp. gram.:v. t.}
\end{itemize}
\begin{itemize}
\item {Utilização:Fig.}
\end{itemize}
\begin{itemize}
\item {Proveniência:(De \textunderscore trama\textunderscore ^1)}
\end{itemize}
Passar (a trama) por entre os fios da urdidura.
Tecer, entretecer.
Maquinar; enredar, intrigar.
\section{Tramazeira}
\begin{itemize}
\item {Grp. gram.:f.}
\end{itemize}
(V.corno-godinho)
\section{Tramba-las-águas}
\begin{itemize}
\item {Grp. gram.:m.}
\end{itemize}
\begin{itemize}
\item {Utilização:Bras}
\end{itemize}
Lugar, onde se encontram duas marés, num canal que tenha duas saídas para o mar.
(Cp. \textunderscore entre-amba-las-águas\textunderscore )
\section{Trambecar}
\begin{itemize}
\item {Grp. gram.:v. i.}
\end{itemize}
\begin{itemize}
\item {Utilização:Bras}
\end{itemize}
Andar aos bordos, como ébrio.
\section{Trambelho}
\begin{itemize}
\item {fónica:bê}
\end{itemize}
\begin{itemize}
\item {Grp. gram.:m.}
\end{itemize}
\begin{itemize}
\item {Utilização:Prov.}
\end{itemize}
\begin{itemize}
\item {Utilização:alg.}
\end{itemize}
\begin{itemize}
\item {Utilização:Náut.}
\end{itemize}
O mesmo que \textunderscore trabelho\textunderscore .
Acêrto, juízo.
Pequeno petrecho, usado nas adriças das bandeiras, nas linhas de prumo, etc.
\section{Trambelho}
\begin{itemize}
\item {fónica:bê}
\end{itemize}
\begin{itemize}
\item {Grp. gram.:m.}
\end{itemize}
\begin{itemize}
\item {Utilização:T. da Bairrada}
\end{itemize}
O mesmo que \textunderscore tramelo\textunderscore ^2.
\section{Trambola}
\begin{itemize}
\item {Grp. gram.:f.}
\end{itemize}
Ave, o mesmo que \textunderscore tarambola\textunderscore .
\section{Trambolhada}
\begin{itemize}
\item {Grp. gram.:f.}
\end{itemize}
\begin{itemize}
\item {Proveniência:(De \textunderscore trambolho\textunderscore )}
\end{itemize}
Porção de coisas, atadas ou enfiadas.
\section{Trambolhão}
\begin{itemize}
\item {Grp. gram.:m.}
\end{itemize}
\begin{itemize}
\item {Utilização:Pop.}
\end{itemize}
\begin{itemize}
\item {Utilização:Fam.}
\end{itemize}
\begin{itemize}
\item {Proveniência:(De \textunderscore trambolho\textunderscore )}
\end{itemize}
Quéda com estrondo.
Acto de caír, rebolando.
Decadência.
Contratempo inesperado.
\section{Trambolhar}
\begin{itemize}
\item {Grp. gram.:v. i.}
\end{itemize}
\begin{itemize}
\item {Proveniência:(De \textunderscore trambolho\textunderscore )}
\end{itemize}
Andar ou ir aos trambolhões.
Falar com embaraço ou confusão.
\section{Trambolhia}
\begin{itemize}
\item {Grp. gram.:f.}
\end{itemize}
\begin{itemize}
\item {Utilização:Prov.}
\end{itemize}
\begin{itemize}
\item {Utilização:alent.}
\end{itemize}
\begin{itemize}
\item {Proveniência:(De \textunderscore trambolho\textunderscore )}
\end{itemize}
Lenha de pernadas.
\section{Trambolho}
\begin{itemize}
\item {fónica:bô}
\end{itemize}
\begin{itemize}
\item {Grp. gram.:m.}
\end{itemize}
\begin{itemize}
\item {Utilização:Fig.}
\end{itemize}
\begin{itemize}
\item {Utilização:Fam.}
\end{itemize}
\begin{itemize}
\item {Proveniência:(Do lat. hyp. \textunderscore trabuculum\textunderscore )}
\end{itemize}
Qualquer corpo, que se prende aos pés dos animaes domésticos, para que se não afastem para longe.
Mólho grande.
Enfiada.
Embaraço, empecilho.
Pessôa muito nutrida, que anda com difficuldade.
\section{Trambuzana}
\begin{itemize}
\item {Grp. gram.:f.}
\end{itemize}
\begin{itemize}
\item {Utilização:Prov.}
\end{itemize}
\begin{itemize}
\item {Utilização:trasm.}
\end{itemize}
O mesmo que \textunderscore trabuzana\textunderscore .
\section{Tramela}
\begin{itemize}
\item {Grp. gram.:f.}
\end{itemize}
O mesmo que \textunderscore taramela\textunderscore .
\section{Tramelo}
\begin{itemize}
\item {fónica:mê}
\end{itemize}
\begin{itemize}
\item {Grp. gram.:m.}
\end{itemize}
Ratinho caseiro.
\section{Tramelo}
\begin{itemize}
\item {fónica:mê}
\end{itemize}
\begin{itemize}
\item {Grp. gram.:m.}
\end{itemize}
\begin{itemize}
\item {Utilização:Prov.}
\end{itemize}
\begin{itemize}
\item {Utilização:trasm.}
\end{itemize}
\begin{itemize}
\item {Utilização:Prov.}
\end{itemize}
O mesmo que \textunderscore taramelo\textunderscore ^2.
Rapaz traquina.
\section{Tramembés}
\begin{itemize}
\item {Grp. gram.:m. pl.}
\end{itemize}
Tríbo de Índios do Ceará.
\section{Tramista}
\begin{itemize}
\item {Grp. gram.:m.}
\end{itemize}
\begin{itemize}
\item {Utilização:Bras. do N}
\end{itemize}
\begin{itemize}
\item {Proveniência:(De \textunderscore trama\textunderscore )}
\end{itemize}
Caloteiro.
Velhaco.
\section{Trâmite}
\begin{itemize}
\item {Grp. gram.:m.}
\end{itemize}
\begin{itemize}
\item {Utilização:Fig.}
\end{itemize}
\begin{itemize}
\item {Proveniência:(Lat. \textunderscore tramis\textunderscore )}
\end{itemize}
Caminho ou atalho determinado.
Direcção; meio apropriado: \textunderscore a questão vai seguindo os seus trâmites\textunderscore .
\section{Tramo}
\begin{itemize}
\item {Grp. gram.:m.}
\end{itemize}
\begin{itemize}
\item {Proveniência:(De \textunderscore tramar\textunderscore )}
\end{itemize}
Espaço entre duas ou mais asnas.
\section{Tramo}
\begin{itemize}
\item {Grp. gram.:m.}
\end{itemize}
\begin{itemize}
\item {Utilização:Ant.}
\end{itemize}
Peste, o mesmo que \textunderscore trama\textunderscore ^2. Cf. G. Vicente, \textunderscore M. Parda\textunderscore .
(Cp. lat. \textunderscore strumus\textunderscore )
\section{Tramoço}
\begin{itemize}
\item {fónica:mô}
\end{itemize}
\begin{itemize}
\item {Grp. gram.:m.}
\end{itemize}
\begin{itemize}
\item {Utilização:ant.}
\end{itemize}
\begin{itemize}
\item {Utilização:Pop.}
\end{itemize}
(V.tremoço)
\section{Tramóia}
\begin{itemize}
\item {Grp. gram.:f.}
\end{itemize}
\begin{itemize}
\item {Utilização:Fam.}
\end{itemize}
Intriga, enrêdo.
Trampolinice.
(Cp. cast. \textunderscore tramoya\textunderscore , de \textunderscore trama\textunderscore )
\section{Tramóia}
\begin{itemize}
\item {Grp. gram.:f.}
\end{itemize}
\begin{itemize}
\item {Utilização:Prov.}
\end{itemize}
O mesmo que \textunderscore tremóia\textunderscore .
\section{Tramóia}
\begin{itemize}
\item {Grp. gram.:f.}
\end{itemize}
(?):«\textunderscore ...determinavão lançar no rio mui grandes jangadas, e tramoias untadas de breo...\textunderscore »Filinto, \textunderscore D. Man.\textunderscore , III, 21.
\section{Tramolhada}
\begin{itemize}
\item {Grp. gram.:f.}
\end{itemize}
Terra húmida, lameiro.
(Contr. de \textunderscore terra\textunderscore  + \textunderscore molhada\textunderscore )
\section{Tramontana}
\begin{itemize}
\item {Grp. gram.:f.}
\end{itemize}
\begin{itemize}
\item {Utilização:Fig.}
\end{itemize}
\begin{itemize}
\item {Grp. gram.:Loc.}
\end{itemize}
\begin{itemize}
\item {Utilização:fam.}
\end{itemize}
\begin{itemize}
\item {Proveniência:(Do lat. \textunderscore transmontana\textunderscore )}
\end{itemize}
A Estrêlla Polar.
Vento do Norte.
Lado do Norte.
Rumo, direcção.
\textunderscore Perder a tramontana\textunderscore , desnortear-se, perder o tino.
\section{Tramontar}
\begin{itemize}
\item {Grp. gram.:v. i.}
\end{itemize}
\begin{itemize}
\item {Grp. gram.:M.}
\end{itemize}
\begin{itemize}
\item {Proveniência:(De \textunderscore tra...\textunderscore  + \textunderscore monte\textunderscore )}
\end{itemize}
Esconder-se além dos montes, (falando-se do Sol).
Acto de tramontar.
\section{Trampa}
\begin{itemize}
\item {Grp. gram.:f.}
\end{itemize}
\begin{itemize}
\item {Utilização:Ant.}
\end{itemize}
Trama, enredo.
(Cast. \textunderscore trampa\textunderscore )
\section{Trampa}
\begin{itemize}
\item {Grp. gram.:f.}
\end{itemize}
\begin{itemize}
\item {Utilização:Chul.}
\end{itemize}
\begin{itemize}
\item {Utilização:Fig.}
\end{itemize}
Excremento.
Insignificância.
\section{Trampalho}
\begin{itemize}
\item {Grp. gram.:m.}
\end{itemize}
\begin{itemize}
\item {Utilização:Prov.}
\end{itemize}
\begin{itemize}
\item {Utilização:alg.}
\end{itemize}
\begin{itemize}
\item {Utilização:T. da Bairrada}
\end{itemize}
\begin{itemize}
\item {Utilização:Fig.}
\end{itemize}
Pau sêco.
Obstáculo; embaraço.
Peça de roupa suja; farrapo sujo.
O mesmo que \textunderscore estafermo\textunderscore .
(Talvez por \textunderscore trapalho\textunderscore , de \textunderscore trapo\textunderscore )
\section{Trampão}
\begin{itemize}
\item {Grp. gram.:adj.}
\end{itemize}
\begin{itemize}
\item {Utilização:Ant.}
\end{itemize}
\begin{itemize}
\item {Proveniência:(De \textunderscore trampa\textunderscore ^1)}
\end{itemize}
Trampolineiro.
Que faz tramóias.
\section{Trampear}
\begin{itemize}
\item {Grp. gram.:v. i.}
\end{itemize}
\begin{itemize}
\item {Utilização:Ant.}
\end{itemize}
\begin{itemize}
\item {Proveniência:(De \textunderscore trampa\textunderscore ^1)}
\end{itemize}
Fazer tramóias ou trampolinas.
\section{Trampesco}
\begin{itemize}
\item {Grp. gram.:m.}
\end{itemize}
\begin{itemize}
\item {Utilização:Prov.}
\end{itemize}
Sôco, bofetada.
\section{Trampista}
\begin{itemize}
\item {Grp. gram.:m.}
\end{itemize}
\begin{itemize}
\item {Utilização:Ant.}
\end{itemize}
O mesmo que \textunderscore trampão\textunderscore .
\section{Trampo}
\begin{itemize}
\item {Grp. gram.:m.}
\end{itemize}
\begin{itemize}
\item {Utilização:Prov.}
\end{itemize}
\begin{itemize}
\item {Utilização:trasm.}
\end{itemize}
Tôro de lenha, bastante grosso.
(Cp. \textunderscore trampalho\textunderscore )
\section{Trampolim}
\begin{itemize}
\item {Grp. gram.:m.}
\end{itemize}
\begin{itemize}
\item {Proveniência:(Do it. \textunderscore trampellino\textunderscore )}
\end{itemize}
Prancha, donde os acrobatas formam o salto.
\section{Trampolina}
\begin{itemize}
\item {Grp. gram.:f.}
\end{itemize}
\begin{itemize}
\item {Utilização:Pop.}
\end{itemize}
\begin{itemize}
\item {Proveniência:(De \textunderscore trampolim\textunderscore )}
\end{itemize}
Dito ou acto de trampolineiro.
\section{Trampolinar}
\begin{itemize}
\item {Grp. gram.:v. i.}
\end{itemize}
\begin{itemize}
\item {Utilização:Pop.}
\end{itemize}
Fazer trampolinas.
\section{Trampolineiro}
\begin{itemize}
\item {Grp. gram.:m.  e  adj.}
\end{itemize}
\begin{itemize}
\item {Proveniência:(De \textunderscore trampolina\textunderscore )}
\end{itemize}
Velhaco; solerte; embusteiro.
\section{Trampolinice}
\begin{itemize}
\item {Grp. gram.:f.}
\end{itemize}
O mesmo que \textunderscore trampolina\textunderscore .
\section{Trampolinista}
\begin{itemize}
\item {Grp. gram.:m.}
\end{itemize}
O mesmo que \textunderscore trampolineiro\textunderscore .
\section{Tramposo}
\begin{itemize}
\item {Grp. gram.:adj.}
\end{itemize}
\begin{itemize}
\item {Utilização:Ant.}
\end{itemize}
\begin{itemize}
\item {Proveniência:(De \textunderscore trampa\textunderscore ^1)}
\end{itemize}
Intriguista; trapaceiro; velhaco.
\section{Tramposo}
\begin{itemize}
\item {Grp. gram.:adj.}
\end{itemize}
\begin{itemize}
\item {Utilização:Chul.}
\end{itemize}
\begin{itemize}
\item {Proveniência:(De \textunderscore trampa\textunderscore ^2)}
\end{itemize}
Nojento, porco.
Cheio de immundície; immundo.
\section{Trâmuei}
\begin{itemize}
\item {Grp. gram.:m.}
\end{itemize}
\begin{itemize}
\item {Proveniência:(Do ingl. \textunderscore tramway\textunderscore )}
\end{itemize}
Comboio de serviço restrito ás linhas próximas de Lisbôa.(V.tranvia)
\section{Tramuínha}
\begin{itemize}
\item {Grp. gram.:f.}
\end{itemize}
\begin{itemize}
\item {Utilização:T. da Chamusca}
\end{itemize}
Rato pequeno.
(Cp. \textunderscore tramelo\textunderscore ^1)
\section{Tranar}
\begin{itemize}
\item {Grp. gram.:v. i.}
\end{itemize}
\begin{itemize}
\item {Proveniência:(Lat. \textunderscore tranare\textunderscore )}
\end{itemize}
Passar a nado.
\section{Tranca}
\begin{itemize}
\item {Grp. gram.:f.}
\end{itemize}
\begin{itemize}
\item {Utilização:T. de Turquel}
\end{itemize}
\begin{itemize}
\item {Utilização:Prov.}
\end{itemize}
\begin{itemize}
\item {Utilização:minh.}
\end{itemize}
\begin{itemize}
\item {Utilização:Ext.}
\end{itemize}
\begin{itemize}
\item {Utilização:Minh}
\end{itemize}
\begin{itemize}
\item {Utilização:Pesc.}
\end{itemize}
\begin{itemize}
\item {Grp. gram.:Pl.}
\end{itemize}
\begin{itemize}
\item {Utilização:Chul.}
\end{itemize}
Barra de ferro ou madeira que, collocada transversalmente, serve para segurar as portas do lado interior.
Braço de árvore; pernada.
Vide, que se não repoda.
Obstáculo; peia; travanca.
Salmão magro, depois de desovar.
As pernas.
(Contr. de \textunderscore travanca\textunderscore )
\section{Trança}
\begin{itemize}
\item {Grp. gram.:f.}
\end{itemize}
Conjunto de fios ou cabellos entrelaçados.
Madeixa.
Galão estreito, para guarnições ou bordados.
(Talvez, seg. Körling, do lat. hyp. \textunderscore trinitia\textunderscore , do hyp. \textunderscore trinitiare\textunderscore , de \textunderscore trinitas\textunderscore , conjunto de três)
\section{Trancada}
\begin{itemize}
\item {Grp. gram.:f.}
\end{itemize}
\begin{itemize}
\item {Utilização:Pesc.}
\end{itemize}
\begin{itemize}
\item {Proveniência:(De \textunderscore trancar\textunderscore )}
\end{itemize}
Estacada, que atravessa um rio, de um lado ao outro.
\section{Trançadeira}
\begin{itemize}
\item {Grp. gram.:f.}
\end{itemize}
\begin{itemize}
\item {Proveniência:(De \textunderscore trançar\textunderscore )}
\end{itemize}
Fita, com que se prende o cabello.
\section{Trançado}
\begin{itemize}
\item {Grp. gram.:m.}
\end{itemize}
\begin{itemize}
\item {Proveniência:(De \textunderscore trançar\textunderscore )}
\end{itemize}
Trança; trançadeira.
\section{Trancador}
\begin{itemize}
\item {Grp. gram.:adj.}
\end{itemize}
\begin{itemize}
\item {Utilização:Açor}
\end{itemize}
\begin{itemize}
\item {Proveniência:(De \textunderscore trancar\textunderscore )}
\end{itemize}
Diz-se do tripulante que, numa barca baleeira, tem a seu cargo a linha do arpão. Cf. \textunderscore Diár.-de-Not.\textunderscore , de 26-IX-97.
\section{Trancafilar}
\begin{itemize}
\item {Grp. gram.:v. t.}
\end{itemize}
\begin{itemize}
\item {Utilização:Pop.}
\end{itemize}
(Outra fórma de \textunderscore catrafilar\textunderscore )
\section{Trancafio}
\textunderscore m.\textunderscore  (e der.)
O mesmo que \textunderscore trincafio\textunderscore , etc.
\section{Trancalho}
\begin{itemize}
\item {Grp. gram.:m.}
\end{itemize}
\begin{itemize}
\item {Utilização:Prov.}
\end{itemize}
\begin{itemize}
\item {Utilização:beir.}
\end{itemize}
\begin{itemize}
\item {Proveniência:(De \textunderscore tranca\textunderscore )}
\end{itemize}
Ramo de árvore, pôla, pernada.
\section{Trançante}
\begin{itemize}
\item {Grp. gram.:adj.}
\end{itemize}
Que trança. Cf. Castilho, \textunderscore Fastos\textunderscore , III, 551.
\section{Trancar}
\begin{itemize}
\item {Grp. gram.:v. t.}
\end{itemize}
\begin{itemize}
\item {Utilização:Fig.}
\end{itemize}
Segurar ou fechar com tranca.
Pôr fim a.
Rematar.
Riscar ou tornar sem effeito (um documento escrito).
\section{Trançar}
\begin{itemize}
\item {Grp. gram.:v. t.}
\end{itemize}
O mesmo que \textunderscore entrançar\textunderscore .
\section{Trancaria}
\begin{itemize}
\item {Grp. gram.:f.}
\end{itemize}
\begin{itemize}
\item {Proveniência:(De \textunderscore tranca\textunderscore )}
\end{itemize}
Grande porção de toros de madeira ou lenha.
\section{Trancarruas}
\begin{itemize}
\item {Grp. gram.:m.}
\end{itemize}
\begin{itemize}
\item {Proveniência:(De \textunderscore trancar\textunderscore  + \textunderscore rua\textunderscore )}
\end{itemize}
Fanfarrão; valentão; arruador.
\section{Trancaço}
\begin{itemize}
\item {Grp. gram.:m.}
\end{itemize}
\begin{itemize}
\item {Utilização:Prov.}
\end{itemize}
\begin{itemize}
\item {Utilização:trasm.}
\end{itemize}
Tosse violenta, como esgana.
Andaço de doença, que mais ou menos obstrue a garganta. (Colhido em Caçarelhos)
(Cast. \textunderscore trancazo\textunderscore )
\section{Trance}
\begin{itemize}
\item {Grp. gram.:m.}
\end{itemize}
(V.transe):«\textunderscore tinha passado trances amargos.\textunderscore »Camillo, \textunderscore Brasileira\textunderscore , 272.
(Cp. cast. \textunderscore trance\textunderscore )
\section{Trancelim}
\begin{itemize}
\item {Grp. gram.:m.}
\end{itemize}
\begin{itemize}
\item {Proveniência:(De \textunderscore trança\textunderscore )}
\end{itemize}
Trancinha.
Cordão delgado de oiro.
\section{Tranchefilas}
\begin{itemize}
\item {Grp. gram.:m.}
\end{itemize}
\begin{itemize}
\item {Proveniência:(Fr. \textunderscore tranchefile\textunderscore )}
\end{itemize}
Pedaço quadrilongo de papel ou de pellica, que os encadernadores adaptam á parte superior e á inferior da lombada de um livro, para se conservarem presos os cadernos e não se desmancharem ao manusear-se o livro.
\section{Trancho}
\begin{itemize}
\item {Grp. gram.:m.}
\end{itemize}
\begin{itemize}
\item {Utilização:T. de Viana}
\end{itemize}
Sardinha, que a rêde partiu, ou que ficou partida pelas más condições do transporte: \textunderscore a peixeira apregoava tranchos\textunderscore .
(Cp. cast. \textunderscore trancho\textunderscore  e fr. \textunderscore tranche\textunderscore )
\section{Trancinha}
\begin{itemize}
\item {Grp. gram.:f.}
\end{itemize}
\begin{itemize}
\item {Grp. gram.:Pl.}
\end{itemize}
\begin{itemize}
\item {Utilização:Ant.}
\end{itemize}
Pequena trança.
Galão estreito ou trança estreita de fios, para guarnições e bordados.
Meios astuciosos, para se saber indirectamente o que se deseja saber.
\section{Tranco}
\begin{itemize}
\item {Grp. gram.:m.}
\end{itemize}
Salto largo das cavalgaduras.
Solavanco.
Abalo, commoção:«\textunderscore encarregava-o, com o coração a trancos dolorosos\textunderscore ». Camillo, \textunderscore Filha do Reg.\textunderscore 
(Cast. \textunderscore tranco\textunderscore )
\section{Tranco}
\begin{itemize}
\item {Grp. gram.:m.}
\end{itemize}
\begin{itemize}
\item {Utilização:T. de Turquel}
\end{itemize}
Tranca pequena.
Pernada de árvore.
\section{Trancucho}
\begin{itemize}
\item {Grp. gram.:m.}
\end{itemize}
\begin{itemize}
\item {Utilização:Bras. do S}
\end{itemize}
O mesmo que \textunderscore bêbedo\textunderscore .
\section{Trangalhadanças}
\begin{itemize}
\item {Grp. gram.:m.  e  f.}
\end{itemize}
\begin{itemize}
\item {Utilização:Burl.}
\end{itemize}
\begin{itemize}
\item {Proveniência:(De \textunderscore tranca\textunderscore )}
\end{itemize}
Pessôa alta e desajeitada.
\section{Trangalho}
\begin{itemize}
\item {Grp. gram.:m.}
\end{itemize}
Trambolho.
O mesmo que \textunderscore tranganho\textunderscore .
(Cast. \textunderscore trangallo\textunderscore )
\section{Tranganho}
\begin{itemize}
\item {Grp. gram.:m.}
\end{itemize}
Tôro de madeira.
Ramo cortado, para lenha.
Cacete.
(Por \textunderscore trancanho\textunderscore , de \textunderscore tranca\textunderscore , se não alter. de \textunderscore trangalho\textunderscore )
\section{Trangola}
\begin{itemize}
\item {Grp. gram.:m.}
\end{itemize}
\begin{itemize}
\item {Utilização:Burl.}
\end{itemize}
Homem alto, feio e magrizela.
(Por \textunderscore trancola\textunderscore , de \textunderscore tranca\textunderscore )
\section{Trangolho}
\begin{itemize}
\item {fónica:gô}
\end{itemize}
\begin{itemize}
\item {Grp. gram.:m.}
\end{itemize}
\begin{itemize}
\item {Utilização:Chul.}
\end{itemize}
O mesmo que \textunderscore tranganho\textunderscore .
O mesmo que \textunderscore pênis\textunderscore .
\section{Trango-mango}
\begin{itemize}
\item {Grp. gram.:m.}
\end{itemize}
O mesmo que \textunderscore tangro-mangro\textunderscore .
\section{Tranqueira}
\begin{itemize}
\item {Grp. gram.:f.}
\end{itemize}
\begin{itemize}
\item {Utilização:Prov.}
\end{itemize}
\begin{itemize}
\item {Utilização:Prov.}
\end{itemize}
\begin{itemize}
\item {Utilização:minh.}
\end{itemize}
\begin{itemize}
\item {Utilização:Prov.}
\end{itemize}
\begin{itemize}
\item {Utilização:minh.}
\end{itemize}
\begin{itemize}
\item {Proveniência:(De \textunderscore tranca\textunderscore )}
\end{itemize}
Estacada, para cercar ou fortificar; trincheira.
Abertura nas paredes lateraes da porta da rua, para ali se meterem as extremidades da tranca de madeira, com que interiormente se mantém fechada e segura a porta.
Pedra alta.
Ombreira da porta.
\section{Tranqueirar}
\begin{itemize}
\item {Grp. gram.:v. t.}
\end{itemize}
\begin{itemize}
\item {Utilização:Des.}
\end{itemize}
Pôr tranqueira em; atravancar. Cf. \textunderscore Hist. Insulana\textunderscore , II, 44.
\section{Tranqueiro}
\begin{itemize}
\item {Grp. gram.:m.}
\end{itemize}
\begin{itemize}
\item {Utilização:Prov.}
\end{itemize}
\begin{itemize}
\item {Utilização:trasm.}
\end{itemize}
\begin{itemize}
\item {Proveniência:(De \textunderscore tranca\textunderscore )}
\end{itemize}
Cada um dos paus ou escoras, que sustentam um madeiro que se vai serrar com serra braçal.
O mesmo que \textunderscore ombreira\textunderscore .
\section{Tranqueta}
\begin{itemize}
\item {fónica:quê}
\end{itemize}
\begin{itemize}
\item {Grp. gram.:f.}
\end{itemize}
\begin{itemize}
\item {Proveniência:(De \textunderscore tranca\textunderscore )}
\end{itemize}
Pequena tranca.
Peça de ferro que, collocada verticalmente no lado interior das portas ou janelas, serve para as fechar.
\section{Tranquia}
\begin{itemize}
\item {Grp. gram.:f.}
\end{itemize}
O mesmo que \textunderscore tranqueira\textunderscore .
\section{Tranquibernar}
\begin{itemize}
\item {Grp. gram.:v. i.}
\end{itemize}
\begin{itemize}
\item {Utilização:Pop.}
\end{itemize}
Fazer tranquibérnias.
\section{Tranquiberneiro}
\begin{itemize}
\item {Grp. gram.:m.  e  adj.}
\end{itemize}
\begin{itemize}
\item {Proveniência:(De \textunderscore tranquibérnia\textunderscore )}
\end{itemize}
O que tranquiberna.
\section{Tranquibérnia}
\begin{itemize}
\item {Grp. gram.:f.}
\end{itemize}
\begin{itemize}
\item {Utilização:Pop.}
\end{itemize}
Tramóia; fraude; burla, trapaça.
\section{Tranquibernice}
\begin{itemize}
\item {Grp. gram.:f.}
\end{itemize}
O mesmo que \textunderscore tranquibérnia\textunderscore .
\section{Tranquilamente}
\begin{itemize}
\item {fónica:cu-i}
\end{itemize}
\begin{itemize}
\item {Grp. gram.:adv.}
\end{itemize}
De modo tranquilo; pacificamente; com sossêgo.
\section{Tranquilha}
\begin{itemize}
\item {Grp. gram.:f.}
\end{itemize}
Peça de madeira, com que se aperta o cavallo, no manejo.
O pau que está de esguelha, no jôgo da bola.
(Cast. \textunderscore tranquilla\textunderscore )
\section{Tranquilheiro}
\begin{itemize}
\item {Grp. gram.:m.}
\end{itemize}
\begin{itemize}
\item {Utilização:T. da Bairrada}
\end{itemize}
\begin{itemize}
\item {Proveniência:(De \textunderscore tranquilha\textunderscore )}
\end{itemize}
Mexeriqueiro.
\section{Tranquilidade}
\begin{itemize}
\item {fónica:cu-i}
\end{itemize}
\begin{itemize}
\item {Grp. gram.:f.}
\end{itemize}
\begin{itemize}
\item {Proveniência:(Lat. \textunderscore tranquillitas\textunderscore )}
\end{itemize}
Estado do que é tranquilo.
Paz.
Quietação; serenidade.
\section{Tranquilizador}
\begin{itemize}
\item {fónica:cu-i}
\end{itemize}
\begin{itemize}
\item {Grp. gram.:adj.}
\end{itemize}
Que tranquiliza.
\section{Tranquilizar}
\begin{itemize}
\item {fónica:cu-i}
\end{itemize}
\begin{itemize}
\item {Grp. gram.:v. t.}
\end{itemize}
Tornar tranquilo; acalmar; pacificar.
\section{Tranquillamente}
\begin{itemize}
\item {fónica:cu-i}
\end{itemize}
\begin{itemize}
\item {Grp. gram.:adv.}
\end{itemize}
De modo tranquillo; pacificamente; com sossêgo.
\section{Tranquillidade}
\begin{itemize}
\item {fónica:cu-i}
\end{itemize}
\begin{itemize}
\item {Grp. gram.:f.}
\end{itemize}
\begin{itemize}
\item {Proveniência:(Lat. \textunderscore tranquillitas\textunderscore )}
\end{itemize}
Estado do que é tranquillo.
Paz.
Quietação; serenidade.
\section{Tranquillizador}
\begin{itemize}
\item {fónica:cu-i}
\end{itemize}
\begin{itemize}
\item {Grp. gram.:adj.}
\end{itemize}
Que tranquilliza.
\section{Tranquillizar}
\begin{itemize}
\item {fónica:cu-i}
\end{itemize}
\begin{itemize}
\item {Grp. gram.:v. t.}
\end{itemize}
Tornar tranquilo; acalmar; pacificar.
\section{Tranquillo}
\begin{itemize}
\item {fónica:cu-i}
\end{itemize}
\begin{itemize}
\item {Grp. gram.:adj.}
\end{itemize}
\begin{itemize}
\item {Proveniência:(Lat. \textunderscore tranquillus\textunderscore )}
\end{itemize}
Que não tem agitação; que está em paz; sossegado; calmo; sereno.
\section{Tranquilo}
\begin{itemize}
\item {fónica:cu-i}
\end{itemize}
\begin{itemize}
\item {Grp. gram.:adj.}
\end{itemize}
\begin{itemize}
\item {Proveniência:(Lat. \textunderscore tranquillus\textunderscore )}
\end{itemize}
Que não tem agitação; que está em paz; sossegado; calmo; sereno.
\section{Tranquitana}
\begin{itemize}
\item {Grp. gram.:f.}
\end{itemize}
O mesmo que \textunderscore traquitana\textunderscore . Cf. C. Guerreiro, \textunderscore Diccion. de Cons.\textunderscore , 107.
\section{Tranquito}
\begin{itemize}
\item {Grp. gram.:m.}
\end{itemize}
\begin{itemize}
\item {Utilização:Bras. do S}
\end{itemize}
\begin{itemize}
\item {Proveniência:(De \textunderscore tranco\textunderscore ^1)}
\end{itemize}
Cavallo, que anda bem, que é estradeiro.
\section{Trans...}
\begin{itemize}
\item {Grp. gram.:pref.}
\end{itemize}
\begin{itemize}
\item {Proveniência:(Lat. \textunderscore trans\textunderscore )}
\end{itemize}
(designativo de \textunderscore além de\textunderscore , \textunderscore através\textunderscore , \textunderscore para trás\textunderscore , etc.)
\section{Transacção}
\begin{itemize}
\item {Grp. gram.:f.}
\end{itemize}
\begin{itemize}
\item {Proveniência:(Do lat. \textunderscore transactio\textunderscore )}
\end{itemize}
Acto ou effeito de transigir.
Combinação, ajuste; convênio.
Tudo que se faz por acôrdo.
\section{Transaccionar}
\begin{itemize}
\item {Grp. gram.:v. i.}
\end{itemize}
Fazer transacções ou negócios.
Fazer contrato.
\section{Transacto}
\begin{itemize}
\item {Grp. gram.:adj.}
\end{itemize}
\begin{itemize}
\item {Proveniência:(Lat. \textunderscore transactus\textunderscore )}
\end{itemize}
Que já passou; passado; anterior: \textunderscore no anno transacto\textunderscore .
\section{Transactor}
\begin{itemize}
\item {Grp. gram.:m.  e  adj.}
\end{itemize}
\begin{itemize}
\item {Proveniência:(Lat. \textunderscore transactor\textunderscore )}
\end{itemize}
O que faz transacção.
\section{Transalpino}
\begin{itemize}
\item {Grp. gram.:adj.}
\end{itemize}
\begin{itemize}
\item {Proveniência:(Lat. \textunderscore transalpinus\textunderscore )}
\end{itemize}
Situado além dos Alpes.
\section{Transandino}
\begin{itemize}
\item {Grp. gram.:adj.}
\end{itemize}
Que é de além dos Andes.
\section{Transatlântico}
\begin{itemize}
\item {Grp. gram.:adj.}
\end{itemize}
\begin{itemize}
\item {Grp. gram.:M.}
\end{itemize}
\begin{itemize}
\item {Proveniência:(De \textunderscore trans...\textunderscore  + \textunderscore Atlântico\textunderscore , n. p.)}
\end{itemize}
Situado além do Atlântico.
Que atravessa o Atlântico: \textunderscore navio transatlântico\textunderscore .
Navio, que faz carreira, da Europa para a América.
\section{Transbordar}
\begin{itemize}
\item {Grp. gram.:v. i.}
\end{itemize}
O mesmo que \textunderscore trasbordar\textunderscore .
\section{Transbôrdo}
\begin{itemize}
\item {Grp. gram.:m.}
\end{itemize}
\begin{itemize}
\item {Proveniência:(De \textunderscore trans...\textunderscore  + \textunderscore bôrdo\textunderscore )}
\end{itemize}
Baldeação.
Passagem de mercadorias ou de mercadorias e passageiros, de um para outro navio, de um para outro combóio.
\section{Transcendência}
\begin{itemize}
\item {Grp. gram.:f.}
\end{itemize}
\begin{itemize}
\item {Proveniência:(Lat. \textunderscore transcendentia\textunderscore )}
\end{itemize}
Qualidade do que é transcendente.
\section{Transcendental}
\begin{itemize}
\item {Grp. gram.:adj.}
\end{itemize}
O mesmo que \textunderscore transcendente\textunderscore .
\section{Transcendentalismo}
\begin{itemize}
\item {Grp. gram.:m.}
\end{itemize}
\begin{itemize}
\item {Proveniência:(De \textunderscore transcendental\textunderscore )}
\end{itemize}
Systema philosóphico, que não parte da observação nem da anályse.
Estudo do subjectivo.
\section{Transcendentalista}
\begin{itemize}
\item {Grp. gram.:m.  e  f.}
\end{itemize}
\begin{itemize}
\item {Proveniência:(De \textunderscore transcendental\textunderscore )}
\end{itemize}
Pessôa, sectária do transcendentalismo.
\section{Transcendentalmente}
\begin{itemize}
\item {Grp. gram.:adv.}
\end{itemize}
De modo transcendente.
\section{Transcendente}
\begin{itemize}
\item {Grp. gram.:adj.}
\end{itemize}
\begin{itemize}
\item {Proveniência:(Lat. \textunderscore transcendens\textunderscore )}
\end{itemize}
Que transcede.
Superior.
Muito elevado.
Que dimana immediatamente da razão.
Perspicaz.
Que ultrapassa os limites ordinários; metaphýsico.
\section{Transcender}
\begin{itemize}
\item {Grp. gram.:v. t.}
\end{itemize}
\begin{itemize}
\item {Grp. gram.:V. i.}
\end{itemize}
\begin{itemize}
\item {Proveniência:(Lat. \textunderscore transcendere\textunderscore )}
\end{itemize}
Passar além de.
Elevar-se acima de.
Exceder.
Passar além do que é ordinário.
Distinguir-se; sêr superior aos outros ou a outra coisa.
\section{Transcensão}
\begin{itemize}
\item {Grp. gram.:f.}
\end{itemize}
\begin{itemize}
\item {Proveniência:(Lat. \textunderscore transcensio\textunderscore )}
\end{itemize}
Acto ou effeito de transcender.
Transmigração. Cf. \textunderscore Luz e Calor\textunderscore , 514.
\section{Transcoação}
\begin{itemize}
\item {Grp. gram.:f.}
\end{itemize}
Acto ou effeito de transcoar.
\section{Transcoar}
\begin{itemize}
\item {Grp. gram.:v. t.  e  i.}
\end{itemize}
\begin{itemize}
\item {Proveniência:(Do lat. \textunderscore transcolare\textunderscore )}
\end{itemize}
Coar.
Destillar.
Transpirar.
\section{Transcolar}
\textunderscore v. t.\textunderscore  e \textunderscore i.\textunderscore (e der.)
O mesmo que \textunderscore transcoar\textunderscore , etc.
\section{Transcontinental}
\begin{itemize}
\item {Grp. gram.:adj.}
\end{itemize}
\begin{itemize}
\item {Proveniência:(De \textunderscore trans...\textunderscore  + \textunderscore continental\textunderscore )}
\end{itemize}
Que atravessa um continente: \textunderscore linha férrea transcontinental\textunderscore .
\section{Transcorno}
\begin{itemize}
\item {Grp. gram.:m.}
\end{itemize}
\begin{itemize}
\item {Proveniência:(De \textunderscore trans...\textunderscore  + \textunderscore corno\textunderscore )}
\end{itemize}
Sorte de toireiro, em que êste salta sôbre as hastes do toiro.
\section{Transcorrer}
\begin{itemize}
\item {Grp. gram.:v. i.}
\end{itemize}
\begin{itemize}
\item {Proveniência:(Lat. \textunderscore transcurrere\textunderscore )}
\end{itemize}
Passar além.
Decorrer: \textunderscore transcorreram séculos\textunderscore .
\section{Transcorrido}
\begin{itemize}
\item {Grp. gram.:adj.}
\end{itemize}
\begin{itemize}
\item {Proveniência:(De \textunderscore transcorrer\textunderscore )}
\end{itemize}
O mesmo que \textunderscore decorrido\textunderscore .
\section{Transcorvo}
\begin{itemize}
\item {fónica:côr}
\end{itemize}
\begin{itemize}
\item {Grp. gram.:adj.}
\end{itemize}
Diz-se do cavallo que, observado de lado, não é bem aprumado das mãos.
(Por \textunderscore transcurvo\textunderscore , de \textunderscore trans...\textunderscore  + \textunderscore curvo\textunderscore )
\section{Transcrever}
\begin{itemize}
\item {Grp. gram.:v. t.}
\end{itemize}
\begin{itemize}
\item {Proveniência:(Do lat. \textunderscore transcribere\textunderscore )}
\end{itemize}
Reproduzir, copiando; copiar.
\section{Transcrição}
\begin{itemize}
\item {Grp. gram.:f.}
\end{itemize}
\begin{itemize}
\item {Utilização:Philol.}
\end{itemize}
\begin{itemize}
\item {Proveniência:(Do lat. \textunderscore transcriptio\textunderscore )}
\end{itemize}
Acto ou effeito de transcrever.
Trêcho que se transcreveu.
Reducção de um systema de escrita a outro, como, por ex., do syllabário devanágrico a caracteres romanos.
\section{Transcripção}
\begin{itemize}
\item {Grp. gram.:f.}
\end{itemize}
O mesmo que \textunderscore transcrição\textunderscore .
\section{Transcrito}
\begin{itemize}
\item {Grp. gram.:adj.}
\end{itemize}
\begin{itemize}
\item {Grp. gram.:M.}
\end{itemize}
\begin{itemize}
\item {Proveniência:(Lat. \textunderscore transcriptus\textunderscore )}
\end{itemize}
Que se transcreveu.
Cópia, traslado.
\section{Transcritor}
\begin{itemize}
\item {Grp. gram.:m.  e  adj.}
\end{itemize}
\begin{itemize}
\item {Proveniência:(De \textunderscore transcrito\textunderscore )}
\end{itemize}
O que transcreve.
\section{Transcurar}
\begin{itemize}
\item {Grp. gram.:v. t.}
\end{itemize}
\begin{itemize}
\item {Proveniência:(De \textunderscore trans...\textunderscore  + \textunderscore curar\textunderscore )}
\end{itemize}
Descurar; preterir; esquecer-se de.
\section{Transcursão}
\begin{itemize}
\item {Grp. gram.:f.}
\end{itemize}
\begin{itemize}
\item {Proveniência:(Do lat. \textunderscore transcursio\textunderscore )}
\end{itemize}
O mesmo que \textunderscore transcurso\textunderscore .
\section{Transcursar}
\begin{itemize}
\item {Grp. gram.:v. t.  e  i.}
\end{itemize}
\begin{itemize}
\item {Proveniência:(De \textunderscore transcurso\textunderscore )}
\end{itemize}
Passar além de.
Decorrer.
\section{Transcurso}
\begin{itemize}
\item {Grp. gram.:m.}
\end{itemize}
\begin{itemize}
\item {Proveniência:(Lat. \textunderscore transcursus\textunderscore )}
\end{itemize}
Acto ou effeito de transcorrer.
Decurso.
\section{Transcurvo}
\begin{itemize}
\item {Grp. gram.:adj.}
\end{itemize}
O mesmo ou melhór que \textunderscore transcorvo\textunderscore . Cf. Leon, \textunderscore Arte de Ferrar\textunderscore , 142.
\section{Transe}
\begin{itemize}
\item {Grp. gram.:m.}
\end{itemize}
\begin{itemize}
\item {Grp. gram.:Loc. adv.}
\end{itemize}
\begin{itemize}
\item {Proveniência:(De \textunderscore transir\textunderscore )}
\end{itemize}
Occasião perigosa.
Perigo.
Momento afflictivo.
Lance.
Fallecimento.
Combate.
\textunderscore A todo o transe\textunderscore , a todo o custo; porfiadamente.
Apesar de tudo.
\section{Transeffusão}
\begin{itemize}
\item {Grp. gram.:f.}
\end{itemize}
O mesmo que \textunderscore transfusão\textunderscore :«\textunderscore ...transeffusão com que o mesmo Senhor se infundiu no pobre.\textunderscore »Vieira, VI, 169.
\section{Transefusão}
\begin{itemize}
\item {Grp. gram.:f.}
\end{itemize}
O mesmo que \textunderscore transfusão\textunderscore :«\textunderscore ...transefusão com que o mesmo Senhor se infundiu no pobre.\textunderscore »Vieira, VI, 169.
\section{Transena}
\begin{itemize}
\item {Grp. gram.:f.}
\end{itemize}
\begin{itemize}
\item {Proveniência:(Lat. \textunderscore transenna\textunderscore )}
\end{itemize}
Grade de ferro ou de madeira, com que se fechavam as capelas nas catacumbas de Roma.
\section{Transenna}
\begin{itemize}
\item {Grp. gram.:f.}
\end{itemize}
\begin{itemize}
\item {Proveniência:(Lat. \textunderscore transenna\textunderscore )}
\end{itemize}
Grade de ferro ou de madeira, com que se fechavam as capellas nas catacumbas de Roma.
\section{Transepto}
\begin{itemize}
\item {Grp. gram.:m.}
\end{itemize}
\begin{itemize}
\item {Proveniência:(Fr. \textunderscore transept\textunderscore , do lat. \textunderscore trans\textunderscore  + \textunderscore septum\textunderscore )}
\end{itemize}
Parte transversal da igreja, que se estende para fóra da nave, formando com esta uma cruz. Cf. Alves Mendes, \textunderscore Itália\textunderscore , XV.
\section{Transeunte}
\begin{itemize}
\item {Grp. gram.:adj.}
\end{itemize}
\begin{itemize}
\item {Grp. gram.:M.}
\end{itemize}
\begin{itemize}
\item {Proveniência:(Lat. \textunderscore transeuns\textunderscore )}
\end{itemize}
Que passa.
Que vai passando ou andando.
Que é transitório: \textunderscore prazeres transeuntes\textunderscore .
Que não deixa vestígio.
Indivíduo, que vai passando; viandante.
\section{Transferência}
\begin{itemize}
\item {Grp. gram.:f.}
\end{itemize}
Acto ou effeito de transferir.
\section{Transferidor}
\begin{itemize}
\item {Grp. gram.:m.  e  adj.}
\end{itemize}
\begin{itemize}
\item {Grp. gram.:M.}
\end{itemize}
O que transfere.
Instrumento semi-circular, dividido em 180°, próprio para a medição dos ângulos.
\section{Transferir}
\begin{itemize}
\item {Grp. gram.:v. t.}
\end{itemize}
\begin{itemize}
\item {Proveniência:(Lat. \textunderscore transferre\textunderscore )}
\end{itemize}
Transportar.
Deslocar, mudar de um lugar para outro: \textunderscore transferir domicílio\textunderscore .
Transmittir; ceder: \textunderscore transferir direitos\textunderscore .
Adiar.
\section{Transferível}
\begin{itemize}
\item {Grp. gram.:adj.}
\end{itemize}
Que se póde transferir.
\section{Transfiguração}
\begin{itemize}
\item {Grp. gram.:f.}
\end{itemize}
\begin{itemize}
\item {Proveniência:(Do lat. \textunderscore transfiguratio\textunderscore )}
\end{itemize}
Acto ou effeito de transfigurar.
Estado glorioso, em que Jesus appareceu sôbre o Thabor.
\section{Transfiguradamente}
\begin{itemize}
\item {Grp. gram.:adv.}
\end{itemize}
De modo transfigurado.
Com transfiguração.
\section{Transfigurado}
\begin{itemize}
\item {Grp. gram.:adj.}
\end{itemize}
\begin{itemize}
\item {Grp. gram.:M.}
\end{itemize}
\begin{itemize}
\item {Proveniência:(De \textunderscore transfigurar\textunderscore )}
\end{itemize}
Transformado; alterado.
Transformação, alteração.
\section{Transfigurador}
\begin{itemize}
\item {Grp. gram.:m.  e  adj.}
\end{itemize}
\begin{itemize}
\item {Proveniência:(Do lat. \textunderscore transfigurator\textunderscore )}
\end{itemize}
O que transfigura.
\section{Transfigurar}
\begin{itemize}
\item {Grp. gram.:v. t.}
\end{itemize}
\begin{itemize}
\item {Proveniência:(Lat. \textunderscore transfigurare\textunderscore )}
\end{itemize}
Mudar a figura ou carácter de; transformar.
\section{Transfigurável}
\begin{itemize}
\item {Grp. gram.:adj.}
\end{itemize}
\begin{itemize}
\item {Proveniência:(Do lat. \textunderscore transfigurabilis\textunderscore )}
\end{itemize}
Que se póde transfigurar.
\section{Transfiltrar}
\begin{itemize}
\item {Grp. gram.:v. t.}
\end{itemize}
Fazer passar através de alguma coisa.
Transcoar.
\section{Transfixão}
\begin{itemize}
\item {fónica:csão}
\end{itemize}
\begin{itemize}
\item {Grp. gram.:f.}
\end{itemize}
\begin{itemize}
\item {Proveniência:(Do lat. \textunderscore transfixus\textunderscore )}
\end{itemize}
Processo de amputação cirúrgica, que consiste em atravessar com um golpe a parte que se quere amputar, e em cortar a carne, de dentro para fóra.
Perfuração.
\section{Transfixar}
\begin{itemize}
\item {fónica:csar}
\end{itemize}
\begin{itemize}
\item {Grp. gram.:v. t.}
\end{itemize}
\begin{itemize}
\item {Proveniência:(Do lat. \textunderscore transfixus\textunderscore )}
\end{itemize}
Atravessar de lado a lado. Cf. C. Lobo, \textunderscore Sat. de Juv.\textunderscore , I, 193.
\section{Transfolado}
\begin{itemize}
\item {Grp. gram.:adj.}
\end{itemize}
\begin{itemize}
\item {Proveniência:(De \textunderscore trans...\textunderscore  + \textunderscore esfolado\textunderscore )}
\end{itemize}
Que chega até á dobra do jarrete, (falando-se do alifafe).
\section{Transforação}
\begin{itemize}
\item {Grp. gram.:f.}
\end{itemize}
\begin{itemize}
\item {Utilização:Med.}
\end{itemize}
\begin{itemize}
\item {Proveniência:(Do lat. \textunderscore transforatio\textunderscore )}
\end{itemize}
Operação, que consiste em trespassar o crânio do feto com o transforador.
\section{Transforador}
\begin{itemize}
\item {Grp. gram.:m.}
\end{itemize}
\begin{itemize}
\item {Proveniência:(Do lat. \textunderscore transforare\textunderscore )}
\end{itemize}
Instrumento cirúrgico, com que se trespassa o crânio do feto.
\section{Transformação}
\begin{itemize}
\item {Grp. gram.:f.}
\end{itemize}
\begin{itemize}
\item {Proveniência:(Do lat. \textunderscore transformatio\textunderscore )}
\end{itemize}
Acto ou effeito de transformar.
\section{Transformada}
\begin{itemize}
\item {Grp. gram.:f.}
\end{itemize}
\begin{itemize}
\item {Proveniência:(De \textunderscore transformado\textunderscore )}
\end{itemize}
Curva geométrica, deduzida de outra, segundo determinada lei.
\section{Transformadamente}
\begin{itemize}
\item {Grp. gram.:adv.}
\end{itemize}
De modo transformado.
Com transformação.
Com alteração ou mudança de fórma ou de aspecto.
\section{Transformado}
\begin{itemize}
\item {Grp. gram.:adj.}
\end{itemize}
Que se transformou; que tomou nova fórma.
Desfigurado.
\section{Transformador}
\begin{itemize}
\item {Grp. gram.:m.  e  adj.}
\end{itemize}
O que transforma.
\section{Transformante}
\begin{itemize}
\item {Grp. gram.:adj.}
\end{itemize}
\begin{itemize}
\item {Proveniência:(Lat. \textunderscore transformans\textunderscore )}
\end{itemize}
Que transforma.
\section{Transformar}
\begin{itemize}
\item {Grp. gram.:v. t.}
\end{itemize}
\begin{itemize}
\item {Proveniência:(Lat. \textunderscore transformare\textunderscore )}
\end{itemize}
Dar fórma nova a.
Metamorphosear.
Transfigurar.
Converter; modificar.
Alterar; disfarçar.
\section{Transformativo}
\begin{itemize}
\item {Grp. gram.:adj.}
\end{itemize}
Que póde transformar.
\section{Transformismo}
\begin{itemize}
\item {Grp. gram.:m.}
\end{itemize}
\begin{itemize}
\item {Proveniência:(De \textunderscore transformar\textunderscore )}
\end{itemize}
Systema biológico, segundo o qual se admitte que as espécies derivam umas das outras por uma série de transformações, determinadas pelas condições de vida, clima, etc.
\section{Transformista}
\begin{itemize}
\item {Grp. gram.:adj.}
\end{itemize}
\begin{itemize}
\item {Grp. gram.:M.  e  f.}
\end{itemize}
\begin{itemize}
\item {Proveniência:(De \textunderscore transformar\textunderscore )}
\end{itemize}
Relativo a transformismo.
Pessôa, que é partidária do transformismo.
\section{Transfretano}
\begin{itemize}
\item {Grp. gram.:adj.}
\end{itemize}
\begin{itemize}
\item {Utilização:P. us.}
\end{itemize}
\begin{itemize}
\item {Proveniência:(Lat. \textunderscore transfretanus\textunderscore )}
\end{itemize}
Situado para lá do estreito de Gibraltar.
Ultramarino.
\section{Transfretar}
\begin{itemize}
\item {Grp. gram.:v. t.}
\end{itemize}
\begin{itemize}
\item {Proveniência:(Lat. \textunderscore transfretare\textunderscore )}
\end{itemize}
Transportar em navio, de uma á outra banda do mar:«\textunderscore ...vendendo mercadorias transfretadas da Ásia, da América e África.\textunderscore »\textunderscore Mandamento\textunderscore  de 15 de Maio de 1758, do Cardeal Saldanha, visitador e reformador da Companhia de Jesus.
\section{Trânsfuga}
\begin{itemize}
\item {Grp. gram.:m.}
\end{itemize}
\begin{itemize}
\item {Utilização:Ext.}
\end{itemize}
\begin{itemize}
\item {Proveniência:(Lat. \textunderscore transfuga\textunderscore )}
\end{itemize}
Desertor.
Aquelle que, em tempo de guerra, abandona a sua bandeira, passando ás fileiras inimigas.
Aquelle que abandona os seus deveres.
Aquelle que abandona o partido político em que estava filiado e passa para outro.
Aquelle que muda de religião; apóstata.
\section{Transfúgio}
\begin{itemize}
\item {Grp. gram.:m.}
\end{itemize}
\begin{itemize}
\item {Proveniência:(Lat. \textunderscore transfugium\textunderscore )}
\end{itemize}
Acto de transfugir; deserção.
\section{Transfugir}
\begin{itemize}
\item {Grp. gram.:v. i.}
\end{itemize}
\begin{itemize}
\item {Proveniência:(Lat. \textunderscore transfugere\textunderscore )}
\end{itemize}
Fugir de um lugar para outro; desertar.
\section{Transfundido}
\begin{itemize}
\item {Grp. gram.:adj.}
\end{itemize}
Que se transfundiu.
Espalhado, diffundido.
\section{Transfundir}
\begin{itemize}
\item {Grp. gram.:v. t.}
\end{itemize}
\begin{itemize}
\item {Grp. gram.:V. p.}
\end{itemize}
\begin{itemize}
\item {Proveniência:(Lat. \textunderscore transfundere\textunderscore )}
\end{itemize}
Fazer passar (um líquido), de um recipiente para outro.
Derramar, diffundir.
Transformar-se.
\section{Transfusão}
\begin{itemize}
\item {Grp. gram.:f.}
\end{itemize}
\begin{itemize}
\item {Proveniência:(Do lat. \textunderscore transfusio\textunderscore )}
\end{itemize}
Acto ou effeito de transfundir.
\section{Transfuso}
\begin{itemize}
\item {Grp. gram.:adj.}
\end{itemize}
\begin{itemize}
\item {Proveniência:(Lat. \textunderscore transfusus\textunderscore )}
\end{itemize}
O mesmo que transfundido. Cf. Herculano, \textunderscore Quest. Públ.\textunderscore , II, 73.
\section{Transgangético}
\begin{itemize}
\item {Grp. gram.:adj.}
\end{itemize}
Situado além dos Ganges. Cf. Herculano, \textunderscore Quest. Públ.\textunderscore , II, 287.
\section{Transgredir}
\begin{itemize}
\item {Grp. gram.:v. t.}
\end{itemize}
\begin{itemize}
\item {Proveniência:(Lat. \textunderscore transgredi\textunderscore )}
\end{itemize}
Ir além de.
Infringir: \textunderscore transgredir a lei\textunderscore .
Postergar.
Deixar de cumprir: \textunderscore transgredir deveres\textunderscore .
Desobedecer a (preceito ou lei)
\section{Transgressão}
\begin{itemize}
\item {Grp. gram.:f.}
\end{itemize}
\begin{itemize}
\item {Proveniência:(Do lat. \textunderscore transgressio\textunderscore )}
\end{itemize}
Acto ou effeito de transgredir.
Infracção.
\section{Transgressivo}
\begin{itemize}
\item {Grp. gram.:adj.}
\end{itemize}
\begin{itemize}
\item {Proveniência:(Lat. \textunderscore transgressivus\textunderscore )}
\end{itemize}
Que transgride.
Que envolve transgressão.
\section{Transgressor}
\begin{itemize}
\item {Grp. gram.:m.  e  adj.}
\end{itemize}
\begin{itemize}
\item {Proveniência:(Lat. \textunderscore transgressor\textunderscore )}
\end{itemize}
O que transgride.
\section{Transhumanar}
\begin{itemize}
\item {Grp. gram.:v. t.}
\end{itemize}
\begin{itemize}
\item {Proveniência:(De \textunderscore trans...\textunderscore  + \textunderscore humanar\textunderscore )}
\end{itemize}
Dar natureza humana a; humanizar.
\section{Transhumância}
\begin{itemize}
\item {Grp. gram.:f.}
\end{itemize}
\begin{itemize}
\item {Proveniência:(De \textunderscore transhumante\textunderscore )}
\end{itemize}
Passagem periódica, que os rebanhos fazem, das planícies para os montes e vice versa.
\section{Transhumante}
\begin{itemize}
\item {Grp. gram.:adj.}
\end{itemize}
\begin{itemize}
\item {Proveniência:(De \textunderscore transhumar\textunderscore )}
\end{itemize}
Diz-se do rebanho que transhuma.
\section{Transhumar}
\begin{itemize}
\item {Grp. gram.:v. t.}
\end{itemize}
\begin{itemize}
\item {Grp. gram.:V. i.}
\end{itemize}
\begin{itemize}
\item {Proveniência:(Do lat. \textunderscore trans\textunderscore  + \textunderscore humus\textunderscore )}
\end{itemize}
Fazer mudar de pasto (os rebanhos).
Realizar a transhumância.
\section{Transição}
\begin{itemize}
\item {fónica:zi}
\end{itemize}
\begin{itemize}
\item {Grp. gram.:f.}
\end{itemize}
\begin{itemize}
\item {Proveniência:(Lat. \textunderscore trasitio\textunderscore )}
\end{itemize}
Acto ou effeito de passar de um lugar para outro.
Trajecto.
Modo de passar de um assumpto para outro.
Maneira de ligar as partes de um discurso ou de uma obra.
Passagem, de um estado para outro.
\section{Transido}
\begin{itemize}
\item {fónica:zi}
\end{itemize}
\begin{itemize}
\item {Grp. gram.:adj.}
\end{itemize}
\begin{itemize}
\item {Utilização:Ant.}
\end{itemize}
\begin{itemize}
\item {Proveniência:(De \textunderscore transir\textunderscore )}
\end{itemize}
Impregnado.
Repassado: \textunderscore transido de frio\textunderscore .
Passado; que já foi:«\textunderscore ...que o tempo de mantenhanos Deos... he transido...\textunderscore »\textunderscore Eufrosina\textunderscore , 9.
\section{Transigência}
\begin{itemize}
\item {fónica:zi}
\end{itemize}
\begin{itemize}
\item {Grp. gram.:f.}
\end{itemize}
\begin{itemize}
\item {Proveniência:(De \textunderscore transigente\textunderscore )}
\end{itemize}
Acto ou effeito de transigir.
Tolerância; indulgência.
\section{Transigente}
\begin{itemize}
\item {fónica:zi}
\end{itemize}
\begin{itemize}
\item {Grp. gram.:adj.}
\end{itemize}
\begin{itemize}
\item {Grp. gram.:M.  e  f.}
\end{itemize}
\begin{itemize}
\item {Proveniência:(Lat. \textunderscore transigens\textunderscore )}
\end{itemize}
Que transige.
Condescendente.
Pessôa que transige.
\section{Transigir}
\begin{itemize}
\item {Grp. gram.:v. t.}
\end{itemize}
\begin{itemize}
\item {Grp. gram.:V. i.}
\end{itemize}
\begin{itemize}
\item {Proveniência:(Lat. \textunderscore transigere\textunderscore )}
\end{itemize}
Conciliar, fazer chegar a um acôrdo.
Chegar a acôrdo.
Fazer contrato ou combinação, com que se previne ou se termina uma contestação ou um pleito.
Condescender.
\section{Transigível}
\begin{itemize}
\item {fónica:zi}
\end{itemize}
\begin{itemize}
\item {Grp. gram.:adj.}
\end{itemize}
Sôbre que se póde transigir; que póde sêr objecto de transacção.
\section{Transir}
\begin{itemize}
\item {fónica:zi}
\end{itemize}
\begin{itemize}
\item {Grp. gram.:v. t.}
\end{itemize}
\begin{itemize}
\item {Proveniência:(Lat. \textunderscore transire\textunderscore )}
\end{itemize}
Passar através de; repassar; penetrar.
\section{Transitado}
\begin{itemize}
\item {fónica:zi}
\end{itemize}
\begin{itemize}
\item {Grp. gram.:adj.}
\end{itemize}
\begin{itemize}
\item {Proveniência:(De \textunderscore transitar\textunderscore )}
\end{itemize}
Por onde se transitou: \textunderscore caminhos transitados\textunderscore .
\section{Transitar}
\begin{itemize}
\item {fónica:zi}
\end{itemize}
\begin{itemize}
\item {Grp. gram.:v. i.}
\end{itemize}
\begin{itemize}
\item {Proveniência:(De \textunderscore trânsito\textunderscore )}
\end{itemize}
Fazer caminho, passar: \textunderscore por aqui, transita muita gente\textunderscore .
Andar.
Mudar de lugar ou de estado: \textunderscore Portugal transitou para a República\textunderscore .
\section{Transitável}
\begin{itemize}
\item {fónica:zi}
\end{itemize}
\begin{itemize}
\item {Grp. gram.:adj.}
\end{itemize}
\begin{itemize}
\item {Proveniência:(De \textunderscore transitar\textunderscore )}
\end{itemize}
Que póde sêr percorrido.
\section{Transitivamente}
\begin{itemize}
\item {fónica:zi}
\end{itemize}
\begin{itemize}
\item {Grp. gram.:adv.}
\end{itemize}
\begin{itemize}
\item {Utilização:Gram.}
\end{itemize}
De modo transitivo: \textunderscore empregar um verbo transitivamente\textunderscore .
De modo transitório; de modo passageiro.
\section{Transitivo}
\begin{itemize}
\item {fónica:zi}
\end{itemize}
\begin{itemize}
\item {Grp. gram.:adv.}
\end{itemize}
\begin{itemize}
\item {Utilização:Gram.}
\end{itemize}
\begin{itemize}
\item {Utilização:Geol.}
\end{itemize}
\begin{itemize}
\item {Proveniência:(Lat. \textunderscore transitivus\textunderscore )}
\end{itemize}
Que passa; passageiro; transitório.
Diz-se, dos verbos, que exprimem uma acção transmittida directamente do sujeito para o complemento.
Diz-se dos terrenos, que formam a transição, de uma camada para outra, de formação mais recente.
\section{Trânsito}
\begin{itemize}
\item {Grp. gram.:m.}
\end{itemize}
\begin{itemize}
\item {Utilização:Bras}
\end{itemize}
\begin{itemize}
\item {Proveniência:(Lat. \textunderscore transitus\textunderscore )}
\end{itemize}
Acto ou effeito de caminhar; marcha.
Trajecto; passagem: \textunderscore prohibiram o trânsito naquella rua\textunderscore .
Affluência de viandantes.
Morte: \textunderscore Castilho dedicou uma elegia ao trânsito daquelle príncipe\textunderscore .
Instrumento náutico. Cf. \textunderscore Tarifa das Alfând.\textunderscore , do Brasil, 106.
\section{Transitoriamente}
\begin{itemize}
\item {fónica:zi}
\end{itemize}
\begin{itemize}
\item {Grp. gram.:adv.}
\end{itemize}
De modo transitório.
Provisoriamente; temporariamente.
\section{Transitoriedade}
\begin{itemize}
\item {fónica:zi}
\end{itemize}
\begin{itemize}
\item {Grp. gram.:f.}
\end{itemize}
Qualidade de transitório.
\section{Transitório}
\begin{itemize}
\item {fónica:zi}
\end{itemize}
\begin{itemize}
\item {Grp. gram.:adj.}
\end{itemize}
\begin{itemize}
\item {Proveniência:(Lat. \textunderscore transitorius\textunderscore )}
\end{itemize}
Que passa rapidamente; que tem pouca duração; passageiro; mortal.
\section{Translação}
\begin{itemize}
\item {Grp. gram.:f.}
\end{itemize}
\begin{itemize}
\item {Proveniência:(Do lat. \textunderscore translatio\textunderscore )}
\end{itemize}
O mesmo que \textunderscore traslação\textunderscore .
Transporte.
Metáphora.
Movimento de um corpo, que muda de posição num espaço: \textunderscore a translação da Terra\textunderscore .
\section{Transladação}
\begin{itemize}
\item {Grp. gram.:f.}
\end{itemize}
O mesmo que \textunderscore trasladação\textunderscore .
\section{Transladar}
\begin{itemize}
\item {Grp. gram.:v. t.}
\end{itemize}
O mesmo que \textunderscore trasladar\textunderscore .
\section{Translatamente}
\begin{itemize}
\item {Grp. gram.:adv.}
\end{itemize}
\begin{itemize}
\item {Utilização:Gram.}
\end{itemize}
De modo translato.
De modo figurado.
\section{Translatício}
\begin{itemize}
\item {Grp. gram.:adj.}
\end{itemize}
\begin{itemize}
\item {Proveniência:(Lat. \textunderscore translaticius\textunderscore )}
\end{itemize}
O mesmo que \textunderscore translato\textunderscore .
\section{Translato}
\begin{itemize}
\item {Grp. gram.:adj.}
\end{itemize}
\begin{itemize}
\item {Utilização:Gram.}
\end{itemize}
\begin{itemize}
\item {Proveniência:(Lat. \textunderscore translatus\textunderscore )}
\end{itemize}
Trasladado.
Metaphórico, figurado: \textunderscore sentido translato\textunderscore .
\section{Transliteração}
\begin{itemize}
\item {Grp. gram.:f.}
\end{itemize}
Acto ou effeito de transliterar.
\section{Transliterar}
\begin{itemize}
\item {Grp. gram.:v. t.}
\end{itemize}
\begin{itemize}
\item {Proveniência:(Do lat. \textunderscore trans\textunderscore  + \textunderscore litera\textunderscore )}
\end{itemize}
Representar uma letra de (um vocábulo) por letra differente, no correspondente vocábulo de outra língua.
\section{Translucidez}
\begin{itemize}
\item {Grp. gram.:f.}
\end{itemize}
Qualidade ou estado do que é translúcido.
\section{Translúcido}
\begin{itemize}
\item {Grp. gram.:adj.}
\end{itemize}
\begin{itemize}
\item {Proveniência:(Lat. \textunderscore translucidus\textunderscore )}
\end{itemize}
Que deixa passar a luz; diáphano; transparente.
\section{Translumbrar}
\begin{itemize}
\item {Grp. gram.:v. t.}
\end{itemize}
(V.deslumbrar)
\section{Transluzente}
\begin{itemize}
\item {Grp. gram.:adj.}
\end{itemize}
\begin{itemize}
\item {Proveniência:(Lat. \textunderscore translucens\textunderscore )}
\end{itemize}
Que transluz.
\section{Transluzimento}
\begin{itemize}
\item {Grp. gram.:m.}
\end{itemize}
\begin{itemize}
\item {Proveniência:(De \textunderscore transluzir\textunderscore )}
\end{itemize}
O mesmo que \textunderscore transparência\textunderscore .
\section{Transluzir}
\begin{itemize}
\item {Grp. gram.:v. i.}
\end{itemize}
\begin{itemize}
\item {Grp. gram.:V. p.}
\end{itemize}
\begin{itemize}
\item {Proveniência:(Lat. \textunderscore translucere\textunderscore )}
\end{itemize}
Luzir através de alguma coisa; transparecer.
Reflectir-se.
Reflectir-se; revelar-se.
\section{Transmalhar}
\begin{itemize}
\item {Grp. gram.:v. t.}
\end{itemize}
O mesmo que \textunderscore tresmalhar\textunderscore . Cf. Filinto, \textunderscore D. Man.\textunderscore , II, 140 e 262.
\section{Transmanchuriano}
\begin{itemize}
\item {Grp. gram.:adj.}
\end{itemize}
Situado além da Manchúria.
\section{Transmarino}
\begin{itemize}
\item {Grp. gram.:adj.}
\end{itemize}
\begin{itemize}
\item {Proveniência:(Lat. \textunderscore transmarinus\textunderscore )}
\end{itemize}
O mesmo que \textunderscore ultramarino\textunderscore .
\section{Transmeável}
\begin{itemize}
\item {Grp. gram.:adj.}
\end{itemize}
\begin{itemize}
\item {Proveniência:(Lat. \textunderscore transmeabílis\textunderscore )}
\end{itemize}
Que se póde atravessar; permeável.
Que póde transpirar.
\section{Transmigração}
\begin{itemize}
\item {Grp. gram.:f.}
\end{itemize}
\begin{itemize}
\item {Proveniência:(Do lat. \textunderscore transmigratio\textunderscore )}
\end{itemize}
Acto ou effeito de transmigrar: \textunderscore o dogma indiano da transmigração das almas\textunderscore .
\section{Transmigrador}
\begin{itemize}
\item {Grp. gram.:m.  e  adj.}
\end{itemize}
O que transmigra.
\section{Transmigrante}
\begin{itemize}
\item {Grp. gram.:adj.}
\end{itemize}
\begin{itemize}
\item {Proveniência:(Lat. \textunderscore transmigrans\textunderscore )}
\end{itemize}
Que transmigra.
\section{Transmigrar}
\begin{itemize}
\item {Grp. gram.:v. i.}
\end{itemize}
\begin{itemize}
\item {Grp. gram.:V. t.}
\end{itemize}
\begin{itemize}
\item {Utilização:Des.}
\end{itemize}
\begin{itemize}
\item {Proveniência:(Lat. \textunderscore transmigrare\textunderscore )}
\end{itemize}
Passar de uma região para outra.
Passar de um corpo para outro, (falando-se da alma).
Fazer mudar de residência ou de país.
\section{Transmissão}
\begin{itemize}
\item {Grp. gram.:f.}
\end{itemize}
\begin{itemize}
\item {Proveniência:(Do lat. \textunderscore transmissio\textunderscore )}
\end{itemize}
Acto ou effeito de transmittir: \textunderscore transmissão de uma herança\textunderscore .
Instrumento, para transmittir movimento.
\section{Transmissibilidade}
\begin{itemize}
\item {Grp. gram.:f.}
\end{itemize}
\begin{itemize}
\item {Proveniência:(Do lat. \textunderscore transmissibilis\textunderscore )}
\end{itemize}
Qualidade do que é transmissível.
\section{Transmissível}
\begin{itemize}
\item {Grp. gram.:adj.}
\end{itemize}
\begin{itemize}
\item {Proveniência:(Lat. \textunderscore transmissibilis\textunderscore )}
\end{itemize}
Que se póde transmittir.
\section{Transmissivelmente}
\begin{itemize}
\item {Grp. gram.:adv.}
\end{itemize}
De modo transmissível.
\section{Transmissivo}
\begin{itemize}
\item {Grp. gram.:adj.}
\end{itemize}
\begin{itemize}
\item {Proveniência:(Do lat. \textunderscore transmissus\textunderscore )}
\end{itemize}
Que transmitte: \textunderscore título transmissivo de propriedades\textunderscore .
\section{Transmissor}
\begin{itemize}
\item {Grp. gram.:adj.}
\end{itemize}
\begin{itemize}
\item {Grp. gram.:M.}
\end{itemize}
\begin{itemize}
\item {Proveniência:(Lat. \textunderscore transmissor\textunderscore )}
\end{itemize}
Que transmitte.
Manipulador.
\section{Transmissório}
\begin{itemize}
\item {Grp. gram.:adj.}
\end{itemize}
O mesmo que \textunderscore transmissor\textunderscore .
\section{Transmitir}
\begin{itemize}
\item {Grp. gram.:v. t.}
\end{itemize}
\begin{itemize}
\item {Proveniência:(Lat. \textunderscore transmittere\textunderscore )}
\end{itemize}
Mandar de um lugar para outro: \textunderscore transmitir notícias\textunderscore .
Fazer passar de um ponto para outro ou do poder de alguém para o poder de outrem; transferir; \textunderscore transmitir um direito\textunderscore .
Deferir.
Expedir; enviar.
Comunicar por contágio: \textunderscore transmitir um typho\textunderscore .
Fazer chegar.
Propagar.
\section{Transmittir}
\begin{itemize}
\item {Grp. gram.:v. t.}
\end{itemize}
\begin{itemize}
\item {Proveniência:(Lat. \textunderscore transmittere\textunderscore )}
\end{itemize}
Mandar de um lugar para outro: \textunderscore transmittir notícias\textunderscore .
Fazer passar de um ponto para outro ou do poder de alguém para o poder de outrem; transferir; \textunderscore transmittir um direito\textunderscore .
Deferir.
Expedir; enviar.
Communicar por contágio: \textunderscore transmittir um typho\textunderscore .
Fazer chegar.
Propagar.
\section{Transmontano}
\begin{itemize}
\item {Grp. gram.:m.  e  adj.}
\end{itemize}
(V.trasmontano)
\section{Transmontar}
\begin{itemize}
\item {Grp. gram.:v. t.}
\end{itemize}
\begin{itemize}
\item {Utilização:Fig.}
\end{itemize}
\begin{itemize}
\item {Grp. gram.:V. i.}
\end{itemize}
\begin{itemize}
\item {Grp. gram.:V. p.}
\end{itemize}
\begin{itemize}
\item {Proveniência:(Do lat. \textunderscore trans\textunderscore  + \textunderscore mons\textunderscore , \textunderscore montis\textunderscore )}
\end{itemize}
Passar por cima de; ultrapassar: \textunderscore a águia transmonta a cordilheira\textunderscore .
Sêr superior a.
O mesmo que \textunderscore tramontar\textunderscore .
Passar além:«\textunderscore essência, que se trasmonta sôbre toda a contemplação\textunderscore ». \textunderscore Luz e Calor\textunderscore , 561.
\section{Transmonto}
\begin{itemize}
\item {Grp. gram.:m.}
\end{itemize}
\begin{itemize}
\item {Utilização:bras}
\end{itemize}
\begin{itemize}
\item {Utilização:Neol.}
\end{itemize}
Acto de transmontar.
\section{Transmudação}
\begin{itemize}
\item {Grp. gram.:f.}
\end{itemize}
O mesmo que \textunderscore transmutação\textunderscore .
\section{Transmudamento}
\begin{itemize}
\item {Grp. gram.:m.}
\end{itemize}
O mesmo que \textunderscore transmutação\textunderscore .
\section{Transmudar}
\begin{itemize}
\item {Grp. gram.:v. t.}
\end{itemize}
\begin{itemize}
\item {Proveniência:(Lat. \textunderscore transmutare\textunderscore )}
\end{itemize}
Fazer mudar de lugar ou domínio.
Transferir.
Transformar.
\section{Transmutabilidade}
\begin{itemize}
\item {Grp. gram.:f.}
\end{itemize}
Qualidade do que é transmutável.
\section{Transmutação}
\begin{itemize}
\item {Grp. gram.:f.}
\end{itemize}
\begin{itemize}
\item {Proveniência:(Do lat. \textunderscore transmutatio\textunderscore )}
\end{itemize}
Acto ou effeito de transmudar.
\section{Transmutar}
\begin{itemize}
\item {Grp. gram.:v. t.}
\end{itemize}
O mesmo que \textunderscore transmudar\textunderscore .
\section{Transmutativo}
\begin{itemize}
\item {Grp. gram.:adj.}
\end{itemize}
\begin{itemize}
\item {Proveniência:(Do lat. \textunderscore transmutatus\textunderscore )}
\end{itemize}
Que transmuda ou póde transmudar.
\section{Transmutável}
\begin{itemize}
\item {Grp. gram.:adj.}
\end{itemize}
\begin{itemize}
\item {Proveniência:(De \textunderscore transmutar\textunderscore )}
\end{itemize}
Que se póde transmudar.
\section{Transnadar}
\begin{itemize}
\item {Grp. gram.:v. t.}
\end{itemize}
\begin{itemize}
\item {Proveniência:(Do lat. \textunderscore transnatare\textunderscore )}
\end{itemize}
O mesmo que \textunderscore tranar\textunderscore .
Conduzir, nadando.
\section{Transnoitar}
\begin{itemize}
\item {Grp. gram.:v. i.}
\end{itemize}
\begin{itemize}
\item {Proveniência:(De \textunderscore trans...\textunderscore  + \textunderscore noite\textunderscore )}
\end{itemize}
Passar a noite sem dormir, tresnoitar. Cf. Filinto, XVI, 321.
\section{Transnominação}
\begin{itemize}
\item {Grp. gram.:f.}
\end{itemize}
\begin{itemize}
\item {Proveniência:(Do lat. \textunderscore transnominato\textunderscore )}
\end{itemize}
O mesmo que \textunderscore metonýmia\textunderscore .
\section{Transoceânico}
\begin{itemize}
\item {Grp. gram.:adj.}
\end{itemize}
\begin{itemize}
\item {Proveniência:(De \textunderscore trans...\textunderscore  + \textunderscore oceânico\textunderscore )}
\end{itemize}
O mesmo que \textunderscore ultramarino\textunderscore .
\section{Transordinariamente}
\begin{itemize}
\item {Grp. gram.:adv.}
\end{itemize}
De modo transordinário; extraordinariamente.
\section{Transordinário}
\begin{itemize}
\item {Grp. gram.:adj.}
\end{itemize}
\begin{itemize}
\item {Proveniência:(De \textunderscore trans...\textunderscore  + \textunderscore ordinário\textunderscore )}
\end{itemize}
O mesmo que \textunderscore extraordinário\textunderscore .
\section{Transparecer}
\begin{itemize}
\item {Grp. gram.:v. i.}
\end{itemize}
Apparecer ou avistar-se através de alguma coisa.
Transluzir.
Mostrar-se em parte; manifestar-se.
(Contr. de \textunderscore trans...\textunderscore  + \textunderscore apparecer\textunderscore )
\section{Transparência}
\begin{itemize}
\item {Grp. gram.:f.}
\end{itemize}
Qualidade do que é transparente.
\section{Transparentar}
\begin{itemize}
\item {Grp. gram.:v. t.}
\end{itemize}
Fazer transparente; tornar evidente.
\section{Transparente}
\begin{itemize}
\item {Grp. gram.:adj.}
\end{itemize}
\begin{itemize}
\item {Utilização:Fig.}
\end{itemize}
\begin{itemize}
\item {Grp. gram.:M.}
\end{itemize}
Que se deixa penetrar pela luz.
Que através da sua espessura deixa distinguir os objectos; diáphano.
Que deixa perceber um sentido occulto, qualquer coisa occulta.
Porção de tela, de papel ou de outra substância, com que se afroixa a acção da luz.
Pedaço de tela branca, para fazer experiências ópticas.
(Contr. de \textunderscore trans\textunderscore  + \textunderscore apparente\textunderscore )
\section{Transparentemente}
\begin{itemize}
\item {Grp. gram.:adv.}
\end{itemize}
De modo transparente.
\section{Transpassar}
\begin{itemize}
\item {Grp. gram.:v. t.}
\end{itemize}
O mesmo que \textunderscore traspassar\textunderscore .
\section{Transpiração}
\begin{itemize}
\item {Grp. gram.:f.}
\end{itemize}
Acto ou effeito de transpirar.
Formoso arbusto africano, de frutos em corymbos com pedúnculo carmesim.
\section{Transpiradeiro}
\begin{itemize}
\item {Grp. gram.:m.}
\end{itemize}
\begin{itemize}
\item {Utilização:Des.}
\end{itemize}
\begin{itemize}
\item {Proveniência:(De \textunderscore transpirar\textunderscore )}
\end{itemize}
Cada um dos poros, por onde se transpira.
\section{Transpirar}
\begin{itemize}
\item {Grp. gram.:v. t.}
\end{itemize}
\begin{itemize}
\item {Grp. gram.:V. i.}
\end{itemize}
\begin{itemize}
\item {Utilização:Fig.}
\end{itemize}
\begin{itemize}
\item {Proveniência:(Do lat. \textunderscore trans\textunderscore  + \textunderscore spirare\textunderscore )}
\end{itemize}
Fazer saír pelos poros.
Saír do corpo, exhalando-se pelos poros.
Exhalar suor.
Saír.
Constar.
Divulgar-se; transluzir.
\section{Transpirável}
\begin{itemize}
\item {Grp. gram.:adj.}
\end{itemize}
Que póde transpirar ou que se póde transpirar.
Que dá lugar á transpiração ou pelo qual se póde transpirar.
\section{Transpirenaico}
\begin{itemize}
\item {Grp. gram.:adj.}
\end{itemize}
Situado além dos Pirenéus.
\section{Transplantação}
\begin{itemize}
\item {Grp. gram.:f.}
\end{itemize}
\begin{itemize}
\item {Proveniência:(Lat. \textunderscore transplantatio\textunderscore )}
\end{itemize}
Acto ou effeito de transplantar.
\section{Transplantador}
\begin{itemize}
\item {Grp. gram.:m.  e  adj.}
\end{itemize}
\begin{itemize}
\item {Grp. gram.:M.}
\end{itemize}
\begin{itemize}
\item {Proveniência:(De \textunderscore transplantar\textunderscore )}
\end{itemize}
O que transplanta.
Apparelho para transplantação de vegetaes.
\section{Transplantar}
\begin{itemize}
\item {Grp. gram.:v. t.}
\end{itemize}
\begin{itemize}
\item {Utilização:Fig.}
\end{itemize}
\begin{itemize}
\item {Proveniência:(Lat. \textunderscore transplantare\textunderscore )}
\end{itemize}
Arrancar de um lugar e replantar noutro (uma planta, uma árvore).
Fazer passar de um país para outro.
Traduzir, trasladar, verter.
\section{Transplantatório}
\begin{itemize}
\item {Grp. gram.:adj.}
\end{itemize}
Que se póde transplantar.
\section{Transplante}
\begin{itemize}
\item {Grp. gram.:m.}
\end{itemize}
O mesmo que \textunderscore transplantação\textunderscore .
\section{Transpor}
\begin{itemize}
\item {Grp. gram.:v. t.}
\end{itemize}
\begin{itemize}
\item {Proveniência:(Do lat. \textunderscore transponere\textunderscore )}
\end{itemize}
Pôr (alguma coisa) em lugar differente daquelle onde ella estava ou devia estar.
Inverter a ordem de.
Passar além de; galgar.
\section{Transportação}
\begin{itemize}
\item {Grp. gram.:f.}
\end{itemize}
\begin{itemize}
\item {Proveniência:(Do lat. \textunderscore transportatio\textunderscore )}
\end{itemize}
Acto ou effeito de transportar.
\section{Transportamento}
\begin{itemize}
\item {Grp. gram.:m.}
\end{itemize}
O mesmo que \textunderscore transportação\textunderscore .
\section{Transportar}
\begin{itemize}
\item {Grp. gram.:v. t.}
\end{itemize}
\begin{itemize}
\item {Proveniência:(Lat. \textunderscore transportare\textunderscore )}
\end{itemize}
Levar de um lugar para outro: \textunderscore um navio que transporta degredados\textunderscore .
Transmittir.
Inverter o sentido de, transpor.
Traduzir.
Enthusiasmar, arrebatar.
Passar de um tom para outro (um trecho ou peça musical).
\section{Transportável}
\begin{itemize}
\item {Grp. gram.:adj.}
\end{itemize}
Que se póde transportar.
\section{Transporte}
\begin{itemize}
\item {Grp. gram.:m.}
\end{itemize}
\begin{itemize}
\item {Utilização:Fig.}
\end{itemize}
\begin{itemize}
\item {Proveniência:(De \textunderscore transportar\textunderscore )}
\end{itemize}
O mesmo que \textunderscore transportação\textunderscore .
Conducção.
Vehículo de provisões para um exército em campanha.
Somma, que de uma página passa para outra, juntando-se a novas parcellas.
Passagem dessa somma, de uma página para outra.
Mudança de tom, num trecho ou peça musical.
Êxtase.
Grande enthusiasmo.
\section{Transposição}
\begin{itemize}
\item {Grp. gram.:f.}
\end{itemize}
\begin{itemize}
\item {Proveniência:(Do lat. \textunderscore trans\textunderscore  + \textunderscore positio\textunderscore )}
\end{itemize}
Acto ou effeito de transpor: \textunderscore a transposição de letras de uma palavra\textunderscore .
\section{Transpositor}
\begin{itemize}
\item {Grp. gram.:adj.}
\end{itemize}
\begin{itemize}
\item {Utilização:Mús.}
\end{itemize}
\begin{itemize}
\item {Proveniência:(Do rad. do lat. \textunderscore transpositus\textunderscore )}
\end{itemize}
Que transporta, que opéra a transposição.
\section{Transpyrenaico}
\begin{itemize}
\item {Grp. gram.:adj.}
\end{itemize}
Situado além dos Pyrenéus.
\section{Transrenano}
\begin{itemize}
\item {Grp. gram.:adj.}
\end{itemize}
\begin{itemize}
\item {Proveniência:(Lat. \textunderscore transrhenanus\textunderscore )}
\end{itemize}
Que é de além do Reno.
\section{Transrhenano}
\begin{itemize}
\item {Grp. gram.:adj.}
\end{itemize}
\begin{itemize}
\item {Proveniência:(Lat. \textunderscore transrhenanus\textunderscore )}
\end{itemize}
Que é de além do Rheno.
\section{Transsecular}
\begin{itemize}
\item {Grp. gram.:adj.}
\end{itemize}
\begin{itemize}
\item {Proveniência:(De \textunderscore trans...\textunderscore  + \textunderscore secular\textunderscore )}
\end{itemize}
Que se realiza através dos séculos:«\textunderscore evolução transsecular.\textunderscore »R. Jorge.
\section{Transtagano}
\begin{itemize}
\item {Grp. gram.:adj.}
\end{itemize}
\begin{itemize}
\item {Proveniência:(Do lat. \textunderscore trans\textunderscore  + \textunderscore Tagus\textunderscore , n. p.)}
\end{itemize}
Situado além do Tejo.
Que é de além do Tejo; alentejano. Cf. \textunderscore Lusíadas\textunderscore , IV, 28.
\section{Transtiberino}
\begin{itemize}
\item {Grp. gram.:adj.}
\end{itemize}
\begin{itemize}
\item {Proveniência:(Lat. \textunderscore transtiberinus\textunderscore )}
\end{itemize}
Situado além do Tibre.
Que é de além do Tibre.
\section{Transtornadamente}
\begin{itemize}
\item {Grp. gram.:adv.}
\end{itemize}
De modo transtornado.
Em desordem; em confusão.
Com perturbação.
\section{Transtornado}
\begin{itemize}
\item {Grp. gram.:adj.}
\end{itemize}
\begin{itemize}
\item {Proveniência:(De \textunderscore transtornar\textunderscore )}
\end{itemize}
Perturbado; atordoado.
Confundido.
\section{Transtornador}
\begin{itemize}
\item {Grp. gram.:adj.}
\end{itemize}
Que transtorna. Cf. F. Recreio, \textunderscore Bat. de Ourique\textunderscore .
\section{Transtornamento}
\begin{itemize}
\item {Grp. gram.:m.}
\end{itemize}
O mesmo que \textunderscore transtôrno\textunderscore .
\section{Transtornar}
\begin{itemize}
\item {Grp. gram.:v. t.}
\end{itemize}
\begin{itemize}
\item {Utilização:Fig.}
\end{itemize}
\begin{itemize}
\item {Proveniência:(De \textunderscore trans...\textunderscore  + \textunderscore tornar\textunderscore )}
\end{itemize}
Pôr em desordem.
Perturbar, alterar a ordem de.
Alterar o viver de.
Fazer mudar de opinião.
Alterar.
Adulterar.
Derrubar.
Desfigurar.
Desorganizar.
Dementar.
Perturbar o juízo de.
Atordoar.
Turvar.
\section{Transtôrno}
Acto ou effeito de transtornar.
Contrariedade; decepção; contratempo.
\section{Transtavado}
\begin{itemize}
\item {Grp. gram.:adj.}
\end{itemize}
\begin{itemize}
\item {Proveniência:(De \textunderscore trans...\textunderscore  + \textunderscore travado\textunderscore )}
\end{itemize}
Que tem brancas as mãos e o pé direito, (falando-se do cavallo).
Diz-se do equídeo, que tem a mão direita e o pé esquerdo descalços.
\section{Transtrocar}
\begin{itemize}
\item {Grp. gram.:v. t.}
\end{itemize}
\begin{itemize}
\item {Proveniência:(De \textunderscore trans...\textunderscore  + \textunderscore trocar\textunderscore )}
\end{itemize}
Inverter; confundir.
\section{Transubstanciação}
\begin{itemize}
\item {Grp. gram.:f.}
\end{itemize}
\begin{itemize}
\item {Proveniência:(De \textunderscore transubstanciar\textunderscore )}
\end{itemize}
Mudança de uma substância noutra.
Transformação da substância do pão e do vinho na substância do corpo e sangue de Christo.
\section{Transubstancial}
\begin{itemize}
\item {Grp. gram.:adj.}
\end{itemize}
Que se transubstancia.
\section{Transubstanciar}
\begin{itemize}
\item {Grp. gram.:v. t.}
\end{itemize}
\begin{itemize}
\item {Proveniência:(De \textunderscore trans...\textunderscore  + \textunderscore substância\textunderscore )}
\end{itemize}
Mudar a substância de.
Realizar a transubstanciação de.
Transformar.
\section{Transudação}
\begin{itemize}
\item {Grp. gram.:f.}
\end{itemize}
Acto ou effeito de transudar.
\section{Transudado}
\begin{itemize}
\item {Grp. gram.:m.}
\end{itemize}
\begin{itemize}
\item {Utilização:Med.}
\end{itemize}
O mesmo que \textunderscore transudato\textunderscore .
\section{Transudar}
\begin{itemize}
\item {Grp. gram.:v. i.}
\end{itemize}
\begin{itemize}
\item {Utilização:Fig.}
\end{itemize}
\begin{itemize}
\item {Grp. gram.:V. t.}
\end{itemize}
\begin{itemize}
\item {Proveniência:(Do lat. \textunderscore trans\textunderscore  + \textunderscore sudare\textunderscore )}
\end{itemize}
Transpirar.
Transparecer.
Coar-se.
Resumar; verter.
\section{Transudato}
\begin{itemize}
\item {Grp. gram.:m.}
\end{itemize}
\begin{itemize}
\item {Proveniência:(De \textunderscore transudar\textunderscore )}
\end{itemize}
Serosidade, de origem não inflammatória, infiltrada no tecido conjunctivo, ou derramada em alguma cavidade do corpo.
\section{Transumanar}
\begin{itemize}
\item {fónica:zu}
\end{itemize}
\begin{itemize}
\item {Grp. gram.:v. t.}
\end{itemize}
\begin{itemize}
\item {Proveniência:(De \textunderscore trans...\textunderscore  + \textunderscore humanar\textunderscore )}
\end{itemize}
Dar natureza humana a; humanizar.
\section{Transumância}
\begin{itemize}
\item {fónica:zu}
\end{itemize}
\begin{itemize}
\item {Grp. gram.:f.}
\end{itemize}
\begin{itemize}
\item {Proveniência:(De \textunderscore transumante\textunderscore )}
\end{itemize}
Passagem periódica, que os rebanhos fazem, das planícies para os montes e vice versa.
\section{Transumante}
\begin{itemize}
\item {fónica:zu}
\end{itemize}
\begin{itemize}
\item {Grp. gram.:adj.}
\end{itemize}
\begin{itemize}
\item {Proveniência:(De \textunderscore transumar\textunderscore )}
\end{itemize}
Diz-se do rebanho que transuma.
\section{Transumar}
\begin{itemize}
\item {fónica:zu}
\end{itemize}
\begin{itemize}
\item {Grp. gram.:v. t.}
\end{itemize}
\begin{itemize}
\item {Grp. gram.:V. i.}
\end{itemize}
\begin{itemize}
\item {Proveniência:(Do lat. \textunderscore trans\textunderscore  + \textunderscore humus\textunderscore )}
\end{itemize}
Fazer mudar de pasto (os rebanhos).
Realizar a transumância.
\section{Transumir}
\begin{itemize}
\item {Grp. gram.:v. i.}
\end{itemize}
\begin{itemize}
\item {Proveniência:(Lat. \textunderscore transumere\textunderscore )}
\end{itemize}
Tomar ou receber de outrem ou de outra coisa.
\section{Transumpto}
\begin{itemize}
\item {Grp. gram.:m.}
\end{itemize}
\begin{itemize}
\item {Proveniência:(Lat. \textunderscore transumptus\textunderscore )}
\end{itemize}
Traslado; cópia.
Imagem; reflexo.
Exemplo.
\section{Transunto}
\begin{itemize}
\item {Grp. gram.:m.}
\end{itemize}
\begin{itemize}
\item {Proveniência:(Lat. \textunderscore transumptus\textunderscore )}
\end{itemize}
Traslado; cópia.
Imagem; reflexo.
Exemplo.
\section{Transvaaliano}
\begin{itemize}
\item {Grp. gram.:adj.}
\end{itemize}
\begin{itemize}
\item {Grp. gram.:M.}
\end{itemize}
Relativo ao Transvaal.
Habitante do Transvaal.
\section{Transvasar}
\begin{itemize}
\item {Grp. gram.:v. t.}
\end{itemize}
\begin{itemize}
\item {Proveniência:(It. \textunderscore transvasare\textunderscore )}
\end{itemize}
Passar de um vaso para outro.
Trasfegar.
\section{Transvazar}
\begin{itemize}
\item {Grp. gram.:v. t.}
\end{itemize}
\begin{itemize}
\item {Proveniência:(De \textunderscore trans\textunderscore  + \textunderscore vazar\textunderscore )}
\end{itemize}
Verter, entornar; esvaziar.
\section{Transverberar}
\begin{itemize}
\item {Grp. gram.:v. t.}
\end{itemize}
\begin{itemize}
\item {Utilização:Fig.}
\end{itemize}
\begin{itemize}
\item {Grp. gram.:V. i.}
\end{itemize}
\begin{itemize}
\item {Proveniência:(Lat. \textunderscore transverberare\textunderscore )}
\end{itemize}
Fazer transparecer.
Deixar passar (luz, côr, etc.).
Manifestar.
Mostrar.
Transluzir; manifestar-se.
Dimanar, brilhando.
\section{Transversal}
\begin{itemize}
\item {Grp. gram.:adj.}
\end{itemize}
\begin{itemize}
\item {Grp. gram.:F.}
\end{itemize}
\begin{itemize}
\item {Grp. gram.:M.}
\end{itemize}
\begin{itemize}
\item {Utilização:Anat.}
\end{itemize}
\begin{itemize}
\item {Proveniência:(Lat. \textunderscore transversalis\textunderscore )}
\end{itemize}
Que passa ou está obliquamente ou de través.
Collateral: \textunderscore parente, na linha transversal\textunderscore .
Linha transversal, série de parentes, que não são da linha ascendente nem da descendente.
Músculo transversal.
\section{Transversalidade}
\begin{itemize}
\item {Grp. gram.:f.}
\end{itemize}
Qualidade do que é transversal.
\section{Transversalmente}
\begin{itemize}
\item {Grp. gram.:adv.}
\end{itemize}
De modo transversal.
\section{Transversão}
\begin{itemize}
\item {Grp. gram.:f.}
\end{itemize}
\begin{itemize}
\item {Proveniência:(Do lat. \textunderscore transversio\textunderscore )}
\end{itemize}
O mesmo que \textunderscore transformação\textunderscore .
\section{Transversário}
\begin{itemize}
\item {Grp. gram.:adj.}
\end{itemize}
\begin{itemize}
\item {Utilização:Anat.}
\end{itemize}
\begin{itemize}
\item {Grp. gram.:M.}
\end{itemize}
\begin{itemize}
\item {Proveniência:(Lat. \textunderscore transversarius\textunderscore )}
\end{itemize}
Diz-se especialmente de certos órgãos, relacionados com as apóphyses transversas das vértebras.
Travéssa, que se adaptava ao virote da balestilha, (instrumento náutico).
\section{Transverso}
\begin{itemize}
\item {Grp. gram.:adj.}
\end{itemize}
\begin{itemize}
\item {Grp. gram.:M.}
\end{itemize}
\begin{itemize}
\item {Utilização:Anat.}
\end{itemize}
\begin{itemize}
\item {Proveniência:(Lat. \textunderscore transversus\textunderscore )}
\end{itemize}
Situado de través.
Atravessado; oblíquo.
Músculo transverso.
\section{Transverter}
\begin{itemize}
\item {Grp. gram.:v. t.}
\end{itemize}
\begin{itemize}
\item {Proveniência:(Lat. \textunderscore transvertere\textunderscore )}
\end{itemize}
Transtornar.
Transformar, converter.
Traduzir.
\section{Transviamento}
\begin{itemize}
\item {Grp. gram.:m.}
\end{itemize}
O mesmo que \textunderscore transvio\textunderscore . Cf. Camillo, \textunderscore Guilh. Amaral\textunderscore , (prefácio).
\section{Transviar}
\begin{itemize}
\item {Grp. gram.:v. t.}
\end{itemize}
\begin{itemize}
\item {Utilização:Fig.}
\end{itemize}
\begin{itemize}
\item {Proveniência:(De \textunderscore trans...\textunderscore  + \textunderscore via\textunderscore )}
\end{itemize}
Extraviar; desencaminhar.
Seduzir, desviar do dever.
Tornar vagabundo, erradio.
\section{Transvio}
\begin{itemize}
\item {Grp. gram.:m.}
\end{itemize}
Acto ou effeito de transviar.
\section{Transvoar}
\begin{itemize}
\item {Grp. gram.:v. t.}
\end{itemize}
\begin{itemize}
\item {Proveniência:(De \textunderscore trans...\textunderscore  + \textunderscore voar\textunderscore )}
\end{itemize}
Transpor, voando. Cf. B. Pato, \textunderscore Livro do Monte\textunderscore ; Castilho, \textunderscore Metam.\textunderscore , 103.
\section{Tranvia}
\begin{itemize}
\item {Grp. gram.:f.}
\end{itemize}
Caminho de ferro, de carris chatos, pelo systema americano.
O mesmo que \textunderscore trâmuei\textunderscore .
(Cast. \textunderscore tranvia\textunderscore , do ingl. \textunderscore tramway\textunderscore )
\section{Trapa}
\begin{itemize}
\item {Grp. gram.:f.}
\end{itemize}
Cova, preparada para nella caírem feras.
Espécie de cabo, com que se arreiam pesos para dentro de uma embarcação.
(B. lat. \textunderscore trappa\textunderscore )
\section{Trapa}
\begin{itemize}
\item {Grp. gram.:f.}
\end{itemize}
\begin{itemize}
\item {Proveniência:(De \textunderscore Trappe\textunderscore , n. p.)}
\end{itemize}
Ordem religiosa, cuja séde era em Trappe.
Designação geral dos conventos dessa Ordem.
\section{Trapaça}
\begin{itemize}
\item {Grp. gram.:f.}
\end{itemize}
Contrato fraudulento; burla; dolo.
(Cast. \textunderscore trapaza\textunderscore )
\section{Trapaçador}
\begin{itemize}
\item {Grp. gram.:m.  e  adj.}
\end{itemize}
\begin{itemize}
\item {Utilização:P. us.}
\end{itemize}
O mesmo que \textunderscore trapaceiro\textunderscore .
\section{Trapaçaria}
\begin{itemize}
\item {Grp. gram.:f.}
\end{itemize}
O mesmo que \textunderscore trapaça\textunderscore .
\section{Trapacear}
\begin{itemize}
\item {Grp. gram.:v. t.}
\end{itemize}
\begin{itemize}
\item {Grp. gram.:V. i.}
\end{itemize}
Tratar (de alguma coisa) fraudulentamente.
Fazer trapaças.
\section{Trapaceiro}
\begin{itemize}
\item {Grp. gram.:m.  e  adj.}
\end{itemize}
Indivíduo, que tem o costume de fazer trapaças.
Trapalhão; trampolineiro.
\section{Trapacento}
\begin{itemize}
\item {Grp. gram.:adj.}
\end{itemize}
(V.trapaceiro)
\section{Trapacice}
\begin{itemize}
\item {Grp. gram.:f.}
\end{itemize}
Acto ou dito de trapaceiro. Cf. Camillo, \textunderscore Noites de Insómn.\textunderscore , VII, 37.
\section{Trapagem}
\begin{itemize}
\item {Grp. gram.:f.}
\end{itemize}
Montão de trapos; porção de trapos.
\section{Trápala}
\begin{itemize}
\item {Grp. gram.:f.}
\end{itemize}
\begin{itemize}
\item {Utilização:Ant.}
\end{itemize}
\begin{itemize}
\item {Proveniência:(T. cast.)}
\end{itemize}
Barulho; estrondo.
\section{Trapalhada}
\begin{itemize}
\item {Grp. gram.:f.}
\end{itemize}
O mesmo que \textunderscore trapagem\textunderscore .
\section{Trapalhada}
\begin{itemize}
\item {Grp. gram.:f.}
\end{itemize}
\begin{itemize}
\item {Proveniência:(De \textunderscore trapa\textunderscore ^1.)}
\end{itemize}
Confusão; enrêdo; trampolina.
\section{Trapalhado}
\begin{itemize}
\item {Grp. gram.:adj.}
\end{itemize}
\begin{itemize}
\item {Proveniência:(De \textunderscore trapalho\textunderscore )}
\end{itemize}
Que ficou mal coalhado, (falando-se de leite).
\section{Trapalhão}
\begin{itemize}
\item {Grp. gram.:m.}
\end{itemize}
\begin{itemize}
\item {Grp. gram.:Adj.}
\end{itemize}
Trapo grande.
Frangalho.
Indivíduo mal vestido, esfrangalhado.
Andrajoso, mal vestido.
\section{Trapalhão}
\begin{itemize}
\item {Grp. gram.:m.  e  adj.}
\end{itemize}
Trapaceiro.
Que se atrapalha facilmente ou que atrapalha tudo.
(Cast. \textunderscore trapalón\textunderscore )
\section{Trapalhice}
\begin{itemize}
\item {Grp. gram.:f.}
\end{itemize}
\begin{itemize}
\item {Proveniência:(De \textunderscore trapo\textunderscore )}
\end{itemize}
Trapagem; vestuário roto ou ridículo.
\section{Trapalhice}
\begin{itemize}
\item {Grp. gram.:f.}
\end{itemize}
O mesmo que \textunderscore trapaça\textunderscore .
\section{Trapalho}
\begin{itemize}
\item {Grp. gram.:m.}
\end{itemize}
\begin{itemize}
\item {Utilização:Prov.}
\end{itemize}
\begin{itemize}
\item {Utilização:minh.}
\end{itemize}
\begin{itemize}
\item {Proveniência:(De \textunderscore trapo\textunderscore )}
\end{itemize}
Rodilha de cozinha.
\section{Trapalhona}
\begin{itemize}
\item {Grp. gram.:f.  e  adj.}
\end{itemize}
(Fem. de \textunderscore trapalhão\textunderscore ^2)
\section{Traparia}
\begin{itemize}
\item {Grp. gram.:f.}
\end{itemize}
O mesmo que \textunderscore trapagem\textunderscore . Cf. Herculano, \textunderscore Hist. de Port.\textunderscore , IV, 310.
\section{Trapassado}
\begin{itemize}
\item {Grp. gram.:m.}
\end{itemize}
\begin{itemize}
\item {Utilização:Ant.}
\end{itemize}
\begin{itemize}
\item {Proveniência:(De \textunderscore tra...\textunderscore  + \textunderscore passado\textunderscore )}
\end{itemize}
Tempo passado; tempo decorrido.
\section{Trapaz}
\begin{itemize}
\item {Grp. gram.:adj.}
\end{itemize}
\begin{itemize}
\item {Utilização:Ant.}
\end{itemize}
O mesmo que \textunderscore trapaceiro\textunderscore .
\section{Trape!}
\begin{itemize}
\item {Grp. gram.:interj.}
\end{itemize}
\begin{itemize}
\item {Proveniência:(T. onom.)}
\end{itemize}
(designativa do som, produzido por pancada ou golpe)
\section{Trapear}
\begin{itemize}
\item {Grp. gram.:v. i.}
\end{itemize}
\begin{itemize}
\item {Utilização:Náut.}
\end{itemize}
\begin{itemize}
\item {Proveniência:(De \textunderscore trape\textunderscore )}
\end{itemize}
Bater contra o mastro, (falando-se das velas de navio).
Trapejar.
Diz-se da vela do moínho, quando agitada pelo vento. Cf. Camillo, \textunderscore Caveira\textunderscore , 321.
\section{Trapeira}
\textunderscore fem.\textunderscore  de \textunderscore trapeiro\textunderscore ^1.
\section{Trapeira}
\begin{itemize}
\item {Grp. gram.:f.}
\end{itemize}
\begin{itemize}
\item {Proveniência:(De \textunderscore trapa\textunderscore ^1, se não de \textunderscore trapo\textunderscore , por sêr na referida janela que os moradores pobres estendem as suas roupas ou trapos, para se enxugarem)}
\end{itemize}
Armadilha para caça.
Abertura ou janela sôbre o telhado.
Água-furtada.
\section{Trapeiro}
\begin{itemize}
\item {Grp. gram.:m.}
\end{itemize}
\begin{itemize}
\item {Proveniência:(De \textunderscore trapo\textunderscore )}
\end{itemize}
Aquelle que negocía em trapos ou os apanha nas ruas para os vender.
Gandaieiro.
\section{Trapeiro}
\begin{itemize}
\item {Grp. gram.:m.}
\end{itemize}
\begin{itemize}
\item {Utilização:T. de Moncorvo}
\end{itemize}
Leito do carro.
\section{Trapejar}
\begin{itemize}
\item {Grp. gram.:v. i.}
\end{itemize}
Fazer trape; trapear.
\section{Trape-zape}
\begin{itemize}
\item {Grp. gram.:m.}
\end{itemize}
\begin{itemize}
\item {Utilização:Des.}
\end{itemize}
\begin{itemize}
\item {Proveniência:(T. onom.)}
\end{itemize}
O tinir de espadas que se chocam.
Rumor de carruagens andando:«\textunderscore ...sem trapezape de seges...\textunderscore »\textunderscore Anat. Joc.\textunderscore , I, 270.
\section{Trapeziforme}
\begin{itemize}
\item {Grp. gram.:adj.}
\end{itemize}
\begin{itemize}
\item {Proveniência:(De \textunderscore trapézio\textunderscore  + \textunderscore fórma\textunderscore )}
\end{itemize}
Que tem fórma de trapézio.
\section{Trapézio}
\begin{itemize}
\item {Grp. gram.:m.}
\end{itemize}
\begin{itemize}
\item {Proveniência:(Lat. \textunderscore trapezium\textunderscore )}
\end{itemize}
Quadrilátero, que tem dois lados parallelos e desiguaes.
Apparelho gymnástico, formado de uma barra de madeira, firmada em duas peças verticaes, ou suspensa por duas cordas.
\section{Trapezista}
\begin{itemize}
\item {Grp. gram.:m.  e  f.}
\end{itemize}
Pessôa, que trabalha em trapézio.
\section{Trapezoédro}
\begin{itemize}
\item {Grp. gram.:m.}
\end{itemize}
\begin{itemize}
\item {Utilização:Mathem.}
\end{itemize}
\begin{itemize}
\item {Proveniência:(Do gr. \textunderscore trapezion\textunderscore  + \textunderscore edra\textunderscore )}
\end{itemize}
Sólido de 24 faces, 48 arestas e 26 ângulos.
\section{Trapezoidal}
\begin{itemize}
\item {Grp. gram.:adj.}
\end{itemize}
\begin{itemize}
\item {Grp. gram.:M.}
\end{itemize}
\begin{itemize}
\item {Proveniência:(Do gr. \textunderscore trapezion\textunderscore  + \textunderscore eidos\textunderscore )}
\end{itemize}
O mesmo que \textunderscore trapeziforme\textunderscore .
Quadrilátero, com os lados todos oblíquos entre si.
\section{Trapezóide}
\begin{itemize}
\item {Grp. gram.:adj.}
\end{itemize}
\begin{itemize}
\item {Grp. gram.:M.}
\end{itemize}
\begin{itemize}
\item {Proveniência:(Do gr. \textunderscore trapezion\textunderscore  + \textunderscore eidos\textunderscore )}
\end{itemize}
O mesmo que \textunderscore trapeziforme\textunderscore .
Quadrilátero, com os lados todos oblíquos entre si.
\section{Trapiá}
\begin{itemize}
\item {Grp. gram.:m.}
\end{itemize}
O mesmo que \textunderscore tapiá\textunderscore .
\section{Trapicalho}
\begin{itemize}
\item {Grp. gram.:m.}
\end{itemize}
\begin{itemize}
\item {Utilização:Prov.}
\end{itemize}
\begin{itemize}
\item {Utilização:Fig.}
\end{itemize}
\begin{itemize}
\item {Proveniência:(De \textunderscore trapo\textunderscore )}
\end{itemize}
Trapo, farrapo.
Pessôa andrajosa, ou desmazelada no vestir.
\section{Trapiche}
\begin{itemize}
\item {Grp. gram.:m.}
\end{itemize}
\begin{itemize}
\item {Utilização:Bras}
\end{itemize}
\begin{itemize}
\item {Proveniência:(T. cast.)}
\end{itemize}
Depósito de mercadorias para embarque, junto ao caes.
Casa ou alpendre, onde se guardam essas mercadorias.
Pequeno engenho de açúcar.
\section{Trapicheiro}
\begin{itemize}
\item {Grp. gram.:m.  e  adj.}
\end{itemize}
O que dirige ou possue trapiches.
\section{Trapilho}
\begin{itemize}
\item {Grp. gram.:m.}
\end{itemize}
Pequeno trapo.
\section{Trapisonda}
\begin{itemize}
\item {Grp. gram.:f.}
\end{itemize}
\begin{itemize}
\item {Utilização:Prov.}
\end{itemize}
\begin{itemize}
\item {Utilização:trasm.}
\end{itemize}
\begin{itemize}
\item {Proveniência:(T. cast.)}
\end{itemize}
Bebedeira.
\section{Trapista}
\begin{itemize}
\item {Grp. gram.:adj.}
\end{itemize}
\begin{itemize}
\item {Grp. gram.:M.}
\end{itemize}
\begin{itemize}
\item {Proveniência:(De \textunderscore trapa\textunderscore ^2)}
\end{itemize}
Relativo á Ordem religiosa da Trapa.
Religioso dessa Ordem.
\section{Trapizarga}
\begin{itemize}
\item {Grp. gram.:f.}
\end{itemize}
\begin{itemize}
\item {Utilização:Pop.}
\end{itemize}
Embrulhada; enrêdo; trapalhice.
\section{Trapo}
\begin{itemize}
\item {Grp. gram.:m.}
\end{itemize}
\begin{itemize}
\item {Utilização:Ext.}
\end{itemize}
\begin{itemize}
\item {Grp. gram.:Loc.}
\end{itemize}
\begin{itemize}
\item {Utilização:fam.}
\end{itemize}
\begin{itemize}
\item {Grp. gram.:Pl.}
\end{itemize}
\begin{itemize}
\item {Utilização:Fam.}
\end{itemize}
Pedaço de pano, usado ou velho.
Fato velho.
Rodilha.
Espécie de floco que, com a apparência de trapo, se fórma em certos líquidos: \textunderscore a urina do doente apresentava uns trapos\textunderscore .
Sedimento de vinho ou vinagre nas vasilhas.
Arbusto celastríneo, (\textunderscore evonymus agglomeratus\textunderscore ).
\textunderscore Pegar-lhe com um trapo quente\textunderscore , tentar remediar o irremediável.
\textunderscore Língua de trapos\textunderscore , língua de quem tem má pronúncia.
Pessôa, que fala com difficuldade ou com pronúncia defeituosa.
Pessôa linguareira, maldizente.
(Talvez do lat. \textunderscore drappum\textunderscore )
\section{Trapoeraba}
\begin{itemize}
\item {Grp. gram.:f.}
\end{itemize}
Gênero de plantas commelíneas e medicinaes do Brasil.
\section{Trápola}
\begin{itemize}
\item {Grp. gram.:f.}
\end{itemize}
\begin{itemize}
\item {Proveniência:(De \textunderscore trapa\textunderscore ^1)}
\end{itemize}
Armadilha para caça.
\section{Trapóla}
\begin{itemize}
\item {Grp. gram.:m. ,  f.  e  adj.}
\end{itemize}
\begin{itemize}
\item {Utilização:Pop.}
\end{itemize}
\begin{itemize}
\item {Proveniência:(De \textunderscore trapa\textunderscore ^1)}
\end{itemize}
Pessôa trapaceira.
\section{Trapólas}
\begin{itemize}
\item {Grp. gram.:m. ,  f.  e  adj.}
\end{itemize}
O mesmo que \textunderscore trapóla\textunderscore .
\section{Trapomonga}
\begin{itemize}
\item {Grp. gram.:f.}
\end{itemize}
\begin{itemize}
\item {Utilização:Bras}
\end{itemize}
Planta medicinal.
\section{Trapuz}
\begin{itemize}
\item {Grp. gram.:m.  e  interj.}
\end{itemize}
O mesmo que \textunderscore catrapus!\textunderscore .
\section{Traque}
\begin{itemize}
\item {Grp. gram.:m.}
\end{itemize}
\begin{itemize}
\item {Utilização:Chul.}
\end{itemize}
\begin{itemize}
\item {Utilização:Bras. do N}
\end{itemize}
\begin{itemize}
\item {Proveniência:(T. onom.)}
\end{itemize}
Estrépido, estoiro; ventosidade.
Designação do artefacto pyrotechnico, mais conhecido por \textunderscore bicha de rabear\textunderscore .
\section{Traqueal}
\begin{itemize}
\item {Grp. gram.:adj.}
\end{itemize}
Relativo á traqueia.
\section{Traqueano}
\begin{itemize}
\item {Grp. gram.:adj.}
\end{itemize}
Que tem traqueias.
Relativo a traqueia.
\section{Traquear}
\begin{itemize}
\item {Grp. gram.:v. t.  e  i.}
\end{itemize}
O mesmo que \textunderscore traquejar\textunderscore ^2.
\section{Traqueia}
\begin{itemize}
\item {Grp. gram.:f.}
\end{itemize}
\begin{itemize}
\item {Utilização:Anat.}
\end{itemize}
\begin{itemize}
\item {Utilização:Bot.}
\end{itemize}
\begin{itemize}
\item {Proveniência:(Gr. \textunderscore trakheia\textunderscore )}
\end{itemize}
Canal, que estabelece comunicação entre a laringe e os brônquios, e dá passagem ao ar.
Cada um dos canaes que, nos insectos, levam o ar a todas as partes do corpo.
Cada um dos vasos, que são compostos de céllulas sobrepostas, ligadas por extremidades cónicas.
\section{Traqueia-artéria}
\begin{itemize}
\item {Grp. gram.:f.}
\end{itemize}
O mesmo que a traqueia do corpo humano.
\section{Traqueiro}
\begin{itemize}
\item {Grp. gram.:adj.}
\end{itemize}
\begin{itemize}
\item {Utilização:Chul.}
\end{itemize}
\begin{itemize}
\item {Proveniência:(De \textunderscore traque\textunderscore )}
\end{itemize}
Que estoira ou dá traque.
Diz-se de uma planta caryophyllácea.
\section{Traqueíte}
\begin{itemize}
\item {Grp. gram.:f.}
\end{itemize}
Inflamação da traqueia.
\section{Traquejado}
\begin{itemize}
\item {Grp. gram.:adj.}
\end{itemize}
\begin{itemize}
\item {Utilização:bras}
\end{itemize}
\begin{itemize}
\item {Utilização:Ant.}
\end{itemize}
Perseguido.
Exercitado, experiente.
\section{Traquejar}
\begin{itemize}
\item {Grp. gram.:v. t.}
\end{itemize}
\begin{itemize}
\item {Utilização:Ant.}
\end{itemize}
\begin{itemize}
\item {Proveniência:(Do fr. \textunderscore traquer\textunderscore )}
\end{itemize}
Perseguir.
Exercitar, tornar apto.
Bater (mato), para fazer sair a caça.
\section{Traquejar}
\begin{itemize}
\item {Grp. gram.:v. i.}
\end{itemize}
\begin{itemize}
\item {Utilização:Chul.}
\end{itemize}
Dar traques.
\section{Traquejo}
\begin{itemize}
\item {Grp. gram.:m.}
\end{itemize}
\begin{itemize}
\item {Utilização:Bras}
\end{itemize}
\begin{itemize}
\item {Proveniência:(De \textunderscore traquejar\textunderscore ^1)}
\end{itemize}
Muita prática ou experiência em qualquer serviço.
\section{Traquélia}
\begin{itemize}
\item {Grp. gram.:f.}
\end{itemize}
\begin{itemize}
\item {Proveniência:(Do gr. \textunderscore trakhelos\textunderscore )}
\end{itemize}
Gênero de insectos coleópteros, a que pertence a cantárida.
Gênero de plantas campanuláceas.
\section{Traqueliano}
\begin{itemize}
\item {Grp. gram.:adj.}
\end{itemize}
\begin{itemize}
\item {Utilização:Anat.}
\end{itemize}
\begin{itemize}
\item {Proveniência:(Do gr. \textunderscore trakhelos\textunderscore )}
\end{itemize}
Relativo á parte posterior do pescoço.
\section{Traquelíneo}
\begin{itemize}
\item {Grp. gram.:adj.}
\end{itemize}
\begin{itemize}
\item {Grp. gram.:M. pl.}
\end{itemize}
Relativo ou semelhante á traquélia, insecto.
Família de insectos, que tem por tipo a traquélia.
\section{Traquélio}
\begin{itemize}
\item {Grp. gram.:m.}
\end{itemize}
Gênero de plantas campanuláceas, também conhecido por traquélia.
Gênero de insectos, o mesmo que \textunderscore traquélia\textunderscore .
\section{Traquelípode}
\begin{itemize}
\item {Grp. gram.:adj.}
\end{itemize}
\begin{itemize}
\item {Utilização:Zool.}
\end{itemize}
\begin{itemize}
\item {Proveniência:(Do gr. \textunderscore trakhelos\textunderscore  + \textunderscore pous\textunderscore )}
\end{itemize}
Que tem os pés aderentes á base do pescoço.
\section{Traquelismo}
\begin{itemize}
\item {Grp. gram.:m.}
\end{itemize}
\begin{itemize}
\item {Proveniência:(Do gr. \textunderscore trakhelos\textunderscore )}
\end{itemize}
Contracção espasmódica dos músculos do pescoço.
\section{Traqueobronquite}
\begin{itemize}
\item {Grp. gram.:f.}
\end{itemize}
\begin{itemize}
\item {Utilização:Med.}
\end{itemize}
\begin{itemize}
\item {Proveniência:(De \textunderscore traqueia\textunderscore  + \textunderscore brônquio\textunderscore )}
\end{itemize}
Inflamação simultânea da traqueia e dos bronquios.
\section{Traqueocele}
\begin{itemize}
\item {Grp. gram.:f.}
\end{itemize}
Tumor na traqueia.
\section{Traqueorragia}
\begin{itemize}
\item {Grp. gram.:f.}
\end{itemize}
\begin{itemize}
\item {Proveniência:(Do gr. \textunderscore trakheia\textunderscore  + \textunderscore ragumni\textunderscore )}
\end{itemize}
Derramamento de sangue pela traqueia.
\section{Traqueorrágico}
\begin{itemize}
\item {Grp. gram.:adj.}
\end{itemize}
Relativo á traqueorragia.
\section{Traqueoscopia}
\begin{itemize}
\item {Grp. gram.:f.}
\end{itemize}
\begin{itemize}
\item {Utilização:Med.}
\end{itemize}
\begin{itemize}
\item {Proveniência:(Do gr. \textunderscore trakheia\textunderscore  + \textunderscore skopein\textunderscore )}
\end{itemize}
Exame da cavidade da traqueia.
\section{Traqueostesose}
\begin{itemize}
\item {Grp. gram.:f.}
\end{itemize}
\begin{itemize}
\item {Utilização:Med.}
\end{itemize}
\begin{itemize}
\item {Proveniência:(Do gr. \textunderscore trakheia\textunderscore  + \textunderscore stenos\textunderscore )}
\end{itemize}
Contracção da traqueia.
\section{Traqueotomia}
\begin{itemize}
\item {Grp. gram.:f.}
\end{itemize}
\begin{itemize}
\item {Proveniência:(Do gr. \textunderscore trakheia\textunderscore  + \textunderscore tome\textunderscore )}
\end{itemize}
Operação cirúrgica, com que se estabelece comunicação entre a traqueia e o exterior.
\section{Traqueotómico}
\begin{itemize}
\item {Grp. gram.:adj.}
\end{itemize}
Relativo á traqueotomia.
\section{Traquete}
\begin{itemize}
\item {fónica:quê}
\end{itemize}
\begin{itemize}
\item {Grp. gram.:m.}
\end{itemize}
\begin{itemize}
\item {Utilização:Náut.}
\end{itemize}
\begin{itemize}
\item {Utilização:Des.}
\end{itemize}
\begin{itemize}
\item {Utilização:Gír.}
\end{itemize}
\begin{itemize}
\item {Proveniência:(Do lat. \textunderscore triquetrus\textunderscore )}
\end{itemize}
Vela grande do mastro da prôa.
O mesmo que \textunderscore gravata\textunderscore .
\section{Traquicardia}
\begin{itemize}
\item {Grp. gram.:f.}
\end{itemize}
\begin{itemize}
\item {Proveniência:(Do gr. \textunderscore trakhus\textunderscore  + \textunderscore kardia\textunderscore )}
\end{itemize}
Pulsação rápida do coração.
\section{Traquicardíaco}
\begin{itemize}
\item {Grp. gram.:adj.}
\end{itemize}
Relativo á traquicardia.
Que sofre traquicardia.
\section{Traquina}
\begin{itemize}
\item {Grp. gram.:m. ,  f.  e  adj.}
\end{itemize}
O mesmo que \textunderscore traquinas\textunderscore .
\section{Traquinada}
\begin{itemize}
\item {Grp. gram.:f.}
\end{itemize}
\begin{itemize}
\item {Proveniência:(De \textunderscore traquinar\textunderscore )}
\end{itemize}
Barulho, estrondo.
Travessura de criança.
Enrêdo, intriga. Cf. Ficalho, \textunderscore Pero da Cov.\textunderscore , 212.
\section{Traquinar}
\begin{itemize}
\item {Grp. gram.:v. i.}
\end{itemize}
Estar inquieto; fazer travessuras.
(Cp. it. \textunderscore trascinare\textunderscore )
\section{Traquinas}
\begin{itemize}
\item {Grp. gram.:adj.}
\end{itemize}
\begin{itemize}
\item {Grp. gram.:M.  e  f.}
\end{itemize}
\begin{itemize}
\item {Proveniência:(De \textunderscore traquinar\textunderscore )}
\end{itemize}
Inquieto; buliçoso; travesso; turbulento.
Criança ou pessôa traquinas.
\section{Traquinice}
\begin{itemize}
\item {Grp. gram.:f.}
\end{itemize}
\begin{itemize}
\item {Proveniência:(De \textunderscore traquinas\textunderscore )}
\end{itemize}
Acto ou effeito de traquinar.
\section{Traquinídeo}
\begin{itemize}
\item {Grp. gram.:adj.}
\end{itemize}
\begin{itemize}
\item {Grp. gram.:M. Pl.}
\end{itemize}
\begin{itemize}
\item {Proveniência:(De \textunderscore traquino\textunderscore )}
\end{itemize}
Relativo ou semelhante ao dragão-marinho.
Grupo de peixes, que tem por tipo o dragão-marinho.
\section{Traquino}
\begin{itemize}
\item {Grp. gram.:m.}
\end{itemize}
\begin{itemize}
\item {Proveniência:(Do gr. \textunderscore trakhus\textunderscore )}
\end{itemize}
Nome científico do dragão-marinho.
\section{Traquinote}
\begin{itemize}
\item {Grp. gram.:m.}
\end{itemize}
Espécie de jôgo popular.
\section{Traquitana}
\begin{itemize}
\item {Grp. gram.:f.}
\end{itemize}
Coche de quatro rodas, para duas pessôas.
\section{Traquítico}
\begin{itemize}
\item {Grp. gram.:adj.}
\end{itemize}
Relativo ao traquito.
\section{Traquito}
\begin{itemize}
\item {Grp. gram.:m.}
\end{itemize}
\begin{itemize}
\item {Utilização:Miner.}
\end{itemize}
\begin{itemize}
\item {Proveniência:(Do gr. \textunderscore trakhus\textunderscore )}
\end{itemize}
Feldspatho de rochas vulcânicas.
\section{Traquitóide}
\begin{itemize}
\item {Grp. gram.:adj.}
\end{itemize}
\begin{itemize}
\item {Utilização:Miner.}
\end{itemize}
Semelhante ou parecido com o traquito.
\section{Traquitoporfírico}
\begin{itemize}
\item {Grp. gram.:adj.}
\end{itemize}
Que participa da natureza do traquito e do pórfiro.
\section{Traquiúa}
\begin{itemize}
\item {Grp. gram.:f.}
\end{itemize}
\begin{itemize}
\item {Utilização:Ant.}
\end{itemize}
Moéda de liga de cobre e prata, em Cambaia.
\section{Tras...}
\begin{itemize}
\item {Grp. gram.:pref.}
\end{itemize}
O mesmo que \textunderscore trans...\textunderscore 
\section{Trás}
\begin{itemize}
\item {Grp. gram.:prep.  e  adv.}
\end{itemize}
\begin{itemize}
\item {Proveniência:(Do lat. \textunderscore trans\textunderscore )}
\end{itemize}
O mesmo que \textunderscore atrás\textunderscore ; após:«\textunderscore trás de uma pulga andará três dias\textunderscore ». G. Vicente.«\textunderscore Trás o som correi, cavallos\textunderscore ». Castilho, \textunderscore Outono\textunderscore , 166.
\section{Tràsanteontem}
\begin{itemize}
\item {Grp. gram.:adv.}
\end{itemize}
\begin{itemize}
\item {Proveniência:(De \textunderscore trás\textunderscore  + \textunderscore ante\textunderscore  + \textunderscore ontem\textunderscore )}
\end{itemize}
No dia antecedente ao de ante-ontem.
\section{Tràsantontem}
\begin{itemize}
\item {Grp. gram.:adv.}
\end{itemize}
O mesmo que \textunderscore tràsanteontem\textunderscore .
\section{Transbordamento}
\begin{itemize}
\item {Grp. gram.:m.}
\end{itemize}
Acto ou effeito de trasbordar.
\section{Trasbordante}
\begin{itemize}
\item {Grp. gram.:adj.}
\end{itemize}
Que trasborda.
\section{Trasbordar}
\begin{itemize}
\item {Grp. gram.:v. t.}
\end{itemize}
\begin{itemize}
\item {Utilização:Ext.}
\end{itemize}
\begin{itemize}
\item {Grp. gram.:V. i.}
\end{itemize}
\begin{itemize}
\item {Utilização:Fig.}
\end{itemize}
\begin{itemize}
\item {Proveniência:(De \textunderscore tras...\textunderscore  + \textunderscore borda\textunderscore )}
\end{itemize}
Sair fóra das bordas de.
Derramar, entornar, verter.
Sair fóra das bordas.
Extravasar-se.
Sobejar.
Manifestar-se impetuosamente.
Espalhar-se.
Estar possuido (de um sentimento violento).
\section{Trasbôrdo}
\begin{itemize}
\item {Grp. gram.:m.}
\end{itemize}
O mesmo que \textunderscore transbordamento\textunderscore .
\section{Trascâmara}
\begin{itemize}
\item {Grp. gram.:f.}
\end{itemize}
\begin{itemize}
\item {Proveniência:(De \textunderscore tras...\textunderscore  + \textunderscore câmara\textunderscore )}
\end{itemize}
Quarto esconso, ou mais interior que a câmara. Cf. \textunderscore Port. Mon. Hist.\textunderscore , \textunderscore Script.\textunderscore , 276.
\section{Trascurar}
\begin{itemize}
\item {Grp. gram.:v. t.}
\end{itemize}
(V.transcurar). Cf. Filinto, XXII, 156.
\section{Traseira}
\begin{itemize}
\item {Grp. gram.:f.}
\end{itemize}
\begin{itemize}
\item {Proveniência:(De \textunderscore traseiro\textunderscore )}
\end{itemize}
A parte posterior; rètaguarda: \textunderscore na traseira do carro\textunderscore .
\section{Traseiro}
\begin{itemize}
\item {Grp. gram.:adj.}
\end{itemize}
\begin{itemize}
\item {Grp. gram.:M.}
\end{itemize}
\begin{itemize}
\item {Proveniência:(De \textunderscore trás\textunderscore )}
\end{itemize}
Que está detrás.
Situado na parte posterior.
O mesmo que [[nádegas|nádega]].
\section{Trasféga}
\begin{itemize}
\item {Grp. gram.:f.}
\end{itemize}
O mesmo que \textunderscore trasfêgo\textunderscore .
\section{Trasfegador}
\begin{itemize}
\item {Grp. gram.:m.  e  adj.}
\end{itemize}
O que trasfega.
\section{Trasfegadura}
\begin{itemize}
\item {Grp. gram.:f.}
\end{itemize}
O mesmo que \textunderscore trasfêgo\textunderscore .
\section{Trasfegar}
\begin{itemize}
\item {Grp. gram.:v. t.}
\end{itemize}
\begin{itemize}
\item {Grp. gram.:V. i.}
\end{itemize}
\begin{itemize}
\item {Utilização:Ant.}
\end{itemize}
Passar de uma vasilha para outra, limpando do sedimento.
Lidar; têr negócios; azafamar-se.
(Cp. \textunderscore trafegar\textunderscore )
\section{Trasfêgo}
\begin{itemize}
\item {Grp. gram.:m.}
\end{itemize}
Acto ou effeito de trasfegar.
\section{Trasfegueiro}
\begin{itemize}
\item {Grp. gram.:m.}
\end{itemize}
\begin{itemize}
\item {Proveniência:(De \textunderscore trasfegar\textunderscore )}
\end{itemize}
Pequeno barco do Doiro.
\section{Trasfegueiro}
\begin{itemize}
\item {Grp. gram.:m.}
\end{itemize}
\begin{itemize}
\item {Utilização:Prov.}
\end{itemize}
\begin{itemize}
\item {Utilização:minh.}
\end{itemize}
(Corr. de \textunderscore trasfogueiro\textunderscore )
\section{Trasflor}
\begin{itemize}
\item {Grp. gram.:m.}
\end{itemize}
\begin{itemize}
\item {Proveniência:(De \textunderscore tras...\textunderscore  + \textunderscore flôr\textunderscore )}
\end{itemize}
Lavor de oiro, sôbre esmalte.
\section{Trasfogueiro}
\begin{itemize}
\item {Grp. gram.:m.}
\end{itemize}
\begin{itemize}
\item {Proveniência:(De \textunderscore trás\textunderscore  + \textunderscore fogo\textunderscore )}
\end{itemize}
Tôro de lenha, ou travessão de ferro, em que se apoiam as achas, na lareira.
O mesmo que \textunderscore morilho\textunderscore .
Utensílio de ferro, formado de duas peças verticaes, travadas por uma barra, donde pende a grammalheira.
\section{Trasfoliar}
\begin{itemize}
\item {Grp. gram.:v. t.}
\end{itemize}
\begin{itemize}
\item {Proveniência:(Do lat. \textunderscore trans\textunderscore  + \textunderscore folium\textunderscore )}
\end{itemize}
Copiar em papel transparente, collocando-o sôbre outro, de que se quere extrahir a cópia.
\section{Trasga}
\begin{itemize}
\item {Grp. gram.:f.}
\end{itemize}
\begin{itemize}
\item {Utilização:Prov.}
\end{itemize}
\begin{itemize}
\item {Utilização:trasm.}
\end{itemize}
Espécie de argola de pau, pendente do jugo dos bois, e que serve para segurar o temão, por meio de uma cavilha.
\section{Trasgo}
\begin{itemize}
\item {Grp. gram.:m.}
\end{itemize}
\begin{itemize}
\item {Proveniência:(Do gr. \textunderscore tragos\textunderscore ?)}
\end{itemize}
Apparição phantástica; diabrete; duende.
Pessôa traquinas.
\section{Trasguear}
\begin{itemize}
\item {Grp. gram.:v. i.}
\end{itemize}
\begin{itemize}
\item {Proveniência:(De \textunderscore trasgo\textunderscore )}
\end{itemize}
Traquinar.
\section{Trasgueiro}
\begin{itemize}
\item {Grp. gram.:m.}
\end{itemize}
\begin{itemize}
\item {Utilização:Prov.}
\end{itemize}
\begin{itemize}
\item {Utilização:trasm.}
\end{itemize}
A correia especial que prende a trasga ao jugo.
\section{Traslação}
\begin{itemize}
\item {Grp. gram.:f.}
\end{itemize}
O mesmo que \textunderscore translação\textunderscore .
\section{Trasladação}
\begin{itemize}
\item {Grp. gram.:f.}
\end{itemize}
Acto ou effeito de trasladar.
\section{Trasladador}
\begin{itemize}
\item {Grp. gram.:m.  e  adj.}
\end{itemize}
\begin{itemize}
\item {Proveniência:(Do lat. \textunderscore translator\textunderscore )}
\end{itemize}
O que traslada.
\section{Trasladar}
\begin{itemize}
\item {Grp. gram.:v. t.}
\end{itemize}
\begin{itemize}
\item {Proveniência:(De \textunderscore traslado\textunderscore )}
\end{itemize}
Transferir.
Mudar de um lugar para outro.
Adiar.
Transcrever.
Traduzir.
Dar significação translata a.
Copiar; debuxar.
\section{Traslado}
\begin{itemize}
\item {Grp. gram.:m.}
\end{itemize}
\begin{itemize}
\item {Proveniência:(Do lat. \textunderscore translatus\textunderscore )}
\end{itemize}
Acto ou effeito de trasladar.
Aquillo que se trasladou ou se copiou.
Imagem, retrato; modêlo.
\section{Traslar}
\begin{itemize}
\item {Grp. gram.:m.}
\end{itemize}
\begin{itemize}
\item {Proveniência:(De \textunderscore tras...\textunderscore  + \textunderscore lar\textunderscore )}
\end{itemize}
A parte posterior da lareira ou do fogão.
\section{Trasmontanismo}
\begin{itemize}
\item {Grp. gram.:m.}
\end{itemize}
\begin{itemize}
\item {Proveniência:(De \textunderscore trasmontano\textunderscore )}
\end{itemize}
Vocábulo ou locução privativa de Trás-os-Montes.
\section{Trasmontano}
\begin{itemize}
\item {Grp. gram.:adj.}
\end{itemize}
\begin{itemize}
\item {Grp. gram.:M.}
\end{itemize}
\begin{itemize}
\item {Proveniência:(De \textunderscore tras...\textunderscore  + \textunderscore monte\textunderscore )}
\end{itemize}
Situado além dos montes.
Ultramontano.
Relativo á província de Trás-os-Montes.
Habitante de Trás-os-Montes.
\section{Trasmontar}
\begin{itemize}
\item {Grp. gram.:v. t.}
\end{itemize}
O mesmo que \textunderscore transmontar\textunderscore .
\section{Trasmudar}
\textunderscore v. t.\textunderscore  (e der.)
O mesmo que \textunderscore transmudar\textunderscore , etc.
\section{Trasnoitar}
\begin{itemize}
\item {Grp. gram.:v. i.}
\end{itemize}
O mesmo que \textunderscore tresnoitar\textunderscore .
\section{Trasordinário}
\begin{itemize}
\item {Grp. gram.:adj.}
\end{itemize}
\begin{itemize}
\item {Utilização:Ant.}
\end{itemize}
\begin{itemize}
\item {Proveniência:(De \textunderscore tras...\textunderscore  + \textunderscore ordinário\textunderscore )}
\end{itemize}
O mesmo que \textunderscore extraordinário\textunderscore . Cf. S. R. Viterbo, \textunderscore Elucidário\textunderscore .
\section{Trasorelho}
\begin{itemize}
\item {fónica:zorê}
\end{itemize}
\begin{itemize}
\item {Grp. gram.:m.}
\end{itemize}
\begin{itemize}
\item {Proveniência:(De \textunderscore tras...\textunderscore  + \textunderscore orelha\textunderscore )}
\end{itemize}
Parotidite epidêmica; papeira; caxumba.
\section{Traspassação}
\begin{itemize}
\item {Grp. gram.:f.}
\end{itemize}
Acto ou effeito de traspassar.
\section{Traspassamento}
\begin{itemize}
\item {Grp. gram.:m.}
\end{itemize}
O mesmo que \textunderscore traspassação\textunderscore .
\section{Traspassar}
\begin{itemize}
\item {Grp. gram.:v. t.}
\end{itemize}
\begin{itemize}
\item {Grp. gram.:V. i.}
\end{itemize}
\begin{itemize}
\item {Proveniência:(De \textunderscore tras...\textunderscore  + \textunderscore passar\textunderscore )}
\end{itemize}
Passar através de.
Passar além de.
Furar de lado a lado.
Affligir.
Magoar.
Transgredir.
Demorar, adiar.
Copiar.
Transmittir.
Causar desfallecimento a.
Desanimar.
Transportar-se, transferir-se.
Desmaiar.
Definhar-se.
Morrer.
\section{Traspasse}
\begin{itemize}
\item {Grp. gram.:m.}
\end{itemize}
Acto ou effeito de traspassar.
Subarrendamento.
O mesmo que \textunderscore morte\textunderscore .
\section{Traspasso}
\begin{itemize}
\item {Grp. gram.:m.}
\end{itemize}
\begin{itemize}
\item {Proveniência:(De \textunderscore traspassar\textunderscore )}
\end{itemize}
O mesmo que \textunderscore traspasse\textunderscore .
Dôr penetrante.
Demora.
\section{Traspés}
\begin{itemize}
\item {Grp. gram.:m. pl.}
\end{itemize}
\begin{itemize}
\item {Utilização:Pop.}
\end{itemize}
\begin{itemize}
\item {Proveniência:(De \textunderscore tras...\textunderscore  + \textunderscore pé\textunderscore )}
\end{itemize}
O mesmo que \textunderscore cambapé\textunderscore .
Estado de quem cambaleia.
\section{Traspilar}
\begin{itemize}
\item {Grp. gram.:m.}
\end{itemize}
\begin{itemize}
\item {Proveniência:(De \textunderscore tras...\textunderscore  + \textunderscore pilar\textunderscore )}
\end{itemize}
Pilar, que está atrás de outro.
\section{Traspor}
\textunderscore v. t.\textunderscore  (e der.)
O mesmo que \textunderscore transpor\textunderscore , etc.
\section{Trasportalecer}
\begin{itemize}
\item {Grp. gram.:v. t.}
\end{itemize}
\begin{itemize}
\item {Utilização:Ant.}
\end{itemize}
\begin{itemize}
\item {Proveniência:(De \textunderscore tras...\textunderscore  + \textunderscore portal\textunderscore )}
\end{itemize}
Pôr fóra da porta, expulsar de casa.
\section{Trastalhão}
\begin{itemize}
\item {Grp. gram.:m.}
\end{itemize}
\begin{itemize}
\item {Utilização:Pop.}
\end{itemize}
Grande traste; grande velhaco.
\section{Traste}
\begin{itemize}
\item {Grp. gram.:m.}
\end{itemize}
\begin{itemize}
\item {Utilização:Pop.}
\end{itemize}
\begin{itemize}
\item {Utilização:Prov.}
\end{itemize}
\begin{itemize}
\item {Utilização:dur.}
\end{itemize}
\begin{itemize}
\item {Proveniência:(Do lat. \textunderscore transtrum\textunderscore )}
\end{itemize}
Qualquer móvel de uma casa.
Alfaia.
Velhaco, biltre, tratante.
O mesmo que \textunderscore casco\textunderscore ^1, nas marinhas.
Tabuão, firmado nas dragas e cavernas dos barcos rabelos, e que serve para amparar o mastro, encostando-se a êste, do lado da prôa.
O mesmo que \textunderscore tasto\textunderscore .
\section{Trastear}
\begin{itemize}
\item {Grp. gram.:v.}
\end{itemize}
\begin{itemize}
\item {Utilização:t. Taur.}
\end{itemize}
\begin{itemize}
\item {Proveniência:(De \textunderscore traste\textunderscore ?)}
\end{itemize}
Preparar com a muleta (o toiro) para a sorte de morte.
\section{Trastear}
\begin{itemize}
\item {Grp. gram.:v.}
\end{itemize}
\begin{itemize}
\item {Utilização:t. Mús.}
\end{itemize}
\begin{itemize}
\item {Proveniência:(De \textunderscore traste\textunderscore )}
\end{itemize}
O mesmo que \textunderscore tastear\textunderscore . Cf. E. Vieira, \textunderscore Diccion. Mus.\textunderscore 
\section{Trasteio}
\begin{itemize}
\item {Grp. gram.:m.}
\end{itemize}
Arte de trastear.
\section{Trastejado}
\begin{itemize}
\item {Grp. gram.:adj.}
\end{itemize}
\begin{itemize}
\item {Utilização:Bras}
\end{itemize}
\begin{itemize}
\item {Proveniência:(De \textunderscore trastejar\textunderscore )}
\end{itemize}
O mesmo que [[mobilado|mobilar]].
\section{Trastejar}
\begin{itemize}
\item {Grp. gram.:v. i.}
\end{itemize}
\begin{itemize}
\item {Utilização:Pop.}
\end{itemize}
\begin{itemize}
\item {Grp. gram.:V. t.}
\end{itemize}
\begin{itemize}
\item {Utilização:Bras}
\end{itemize}
\begin{itemize}
\item {Proveniência:(De \textunderscore traste\textunderscore )}
\end{itemize}
Negociar em trastes.
Negociar em coisas de pouco valor.
Têr acções de velhaco.
Cuidar dos trastes ou objectos de casa; fiscalizar serviços domésticos.
Andar de um lado para o outro.
O mesmo que \textunderscore mobilar\textunderscore .
\section{Trastempar}
\begin{itemize}
\item {Grp. gram.:v. i.}
\end{itemize}
\begin{itemize}
\item {Utilização:Ant.}
\end{itemize}
\begin{itemize}
\item {Proveniência:(De \textunderscore trastempo\textunderscore )}
\end{itemize}
Passar além do tempo; prescrever.
\section{Trastempo}
\begin{itemize}
\item {Grp. gram.:m.}
\end{itemize}
\begin{itemize}
\item {Utilização:Ant.}
\end{itemize}
\begin{itemize}
\item {Proveniência:(De \textunderscore tras...\textunderscore  + \textunderscore tempo\textunderscore )}
\end{itemize}
Tempo decorrido.
Prescripção de um direito.
\section{Trasto}
\begin{itemize}
\item {Grp. gram.:m.}
\end{itemize}
\begin{itemize}
\item {Proveniência:(Do lat. \textunderscore transtrum\textunderscore )}
\end{itemize}
Corda ou arame, que se atravessa no braço de alguns instrumentos de corda.
O mesmo que \textunderscore tasto\textunderscore .
\section{Trastornamento}
\begin{itemize}
\item {Grp. gram.:m.}
\end{itemize}
O mesmo que \textunderscore transtôrno\textunderscore .
\section{Trastornar}
\textunderscore v. t.\textunderscore  (e der.)
O mesmo que \textunderscore transtornar\textunderscore , etc. Cf. S. R. Viterbo, \textunderscore Elucidário\textunderscore .
\section{Trasvestido}
\begin{itemize}
\item {Grp. gram.:adj.}
\end{itemize}
Que se trasvestiu.
Mascarado.
Disfarçado.
\section{Trasvestir-se}
\begin{itemize}
\item {Grp. gram.:v. p.}
\end{itemize}
\begin{itemize}
\item {Proveniência:(De \textunderscore tras...\textunderscore  + \textunderscore vestir\textunderscore )}
\end{itemize}
Mudar de traje.
Mascarar-se. Cf. Pina Leitão, \textunderscore Affonsíada\textunderscore , VII, 43.
\section{Trasviar}
\begin{itemize}
\item {Grp. gram.:v. i.}
\end{itemize}
O mesmo que \textunderscore transviar\textunderscore . Cf. Filinto, V, I; XIII, 88; XIV, 141.
\section{Trasvio}
\begin{itemize}
\item {Grp. gram.:m.}
\end{itemize}
O mesmo que \textunderscore transvio\textunderscore . Cf. Filinto, XIX, 34 e 127.
\section{Trasvisto}
\begin{itemize}
\item {Grp. gram.:adj.}
\end{itemize}
\begin{itemize}
\item {Utilização:Fig.}
\end{itemize}
\begin{itemize}
\item {Proveniência:(De \textunderscore tras\textunderscore  + \textunderscore visto\textunderscore )}
\end{itemize}
Visto de lado ou de través.
Mal visto; odioso.
\section{Tratada}
\begin{itemize}
\item {Grp. gram.:f.}
\end{itemize}
\begin{itemize}
\item {Utilização:Pop.}
\end{itemize}
\begin{itemize}
\item {Utilização:P. us.}
\end{itemize}
\begin{itemize}
\item {Proveniência:(De \textunderscore tratar\textunderscore )}
\end{itemize}
Tratantada.
Fraude; enrêdo.
Tratamento.
\section{Tratadista}
\begin{itemize}
\item {Grp. gram.:f.}
\end{itemize}
\begin{itemize}
\item {Proveniência:(De \textunderscore tratado\textunderscore )}
\end{itemize}
Aquelle que escreveu um tratado ou tratados sôbre pontos scientíficos.
\section{Tratado}
\begin{itemize}
\item {Grp. gram.:m.}
\end{itemize}
Contrato internacional, relativo a commércio, paz, etc.
Convênio.
Estudo ou obra, á cêrca de uma sciência, arte, etc.
\section{Tratador}
\begin{itemize}
\item {Grp. gram.:m.  e  adj.}
\end{itemize}
\begin{itemize}
\item {Proveniência:(Do lat. \textunderscore tractator\textunderscore )}
\end{itemize}
O que trata de alguma coisa, especialmente de cavallos ou de outros animaes.
\section{Tratamento}
\begin{itemize}
\item {Grp. gram.:m.}
\end{itemize}
Trato.
Acolhimento.
Processo de curar.
Título de honra ou de graduação.
Passadio.
\section{Tratantada}
\begin{itemize}
\item {Grp. gram.:f.}
\end{itemize}
Acto ou effeito de tratante; velhacada; burla.
\section{Tratante}
\begin{itemize}
\item {Grp. gram.:m. ,  f.  e  adj.}
\end{itemize}
\begin{itemize}
\item {Utilização:Ant.}
\end{itemize}
\begin{itemize}
\item {Proveniência:(De \textunderscore tratar\textunderscore )}
\end{itemize}
Pessôa, que trata ardilosamente de qualquer coisa, ou que procede com velhacaria.
Pessôa, que trafica ou faz negócios.
\section{Tratantice}
\begin{itemize}
\item {Grp. gram.:f.}
\end{itemize}
(V.tratantada)
\section{Tratantório}
\begin{itemize}
\item {Grp. gram.:m.}
\end{itemize}
\begin{itemize}
\item {Utilização:Fam.}
\end{itemize}
Grande tratante, grande biltre. Cf. Camillo, \textunderscore Demónio do Oiro\textunderscore , I, 21.
\section{Tratar}
\begin{itemize}
\item {Grp. gram.:v. t.}
\end{itemize}
\begin{itemize}
\item {Utilização:Fig.}
\end{itemize}
\begin{itemize}
\item {Grp. gram.:V. i.}
\end{itemize}
\begin{itemize}
\item {Proveniência:(Lat. \textunderscore tractare\textunderscore )}
\end{itemize}
Fazer uso de; manusear.
Praticar.
Portar-se ou proceder para com: \textunderscore tratar bem os filhos\textunderscore .
Têr relações com.
Dar certo título, cognome ou alcunha a: \textunderscore tratavam-no por Pé-Leve\textunderscore .
Discorrer á cêrca de: \textunderscore tratar um assumpto\textunderscore .
Debater.
Curar.
Procurar curar, medicar: \textunderscore tratar enfermos\textunderscore .
Dedicar cuidados a.
Combinar: \textunderscore tratar uma compra\textunderscore .
Contratar.
Executar.
Occupar-se de.
Dedicar-se a.
Alimentar.
Dar agasalho a.
Occupar-se: \textunderscore tratar de negócios\textunderscore .
Discorrer.
Cuidar: \textunderscore tratar de crianças\textunderscore .
Formar plano.
Fazer preparativo.
Portar-se.
Negociar.
Applicar curativo: \textunderscore tratar de um doente\textunderscore .
\section{Tratável}
\begin{itemize}
\item {Grp. gram.:adj.}
\end{itemize}
\begin{itemize}
\item {Proveniência:(Do lat. \textunderscore tractabilis\textunderscore )}
\end{itemize}
Que se póde tratar.
Lhano, affável; benévolo.
\section{Tratavelmente}
\begin{itemize}
\item {Grp. gram.:adv.}
\end{itemize}
De modo tratável; com cortesia, com urbanidade.
\section{Trateador}
\begin{itemize}
\item {Grp. gram.:m.}
\end{itemize}
\begin{itemize}
\item {Proveniência:(De \textunderscore tratear\textunderscore )}
\end{itemize}
O que dá maus tratos; algoz. Cf. B. Pereira, \textunderscore Prosódia\textunderscore , vb. \textunderscore extortor\textunderscore .
\section{Tratear}
\begin{itemize}
\item {Grp. gram.:v. t.}
\end{itemize}
Dar tratos a; affligir; atormentar.
\section{Trato}
\begin{itemize}
\item {Grp. gram.:m.}
\end{itemize}
\begin{itemize}
\item {Grp. gram.:Pl.}
\end{itemize}
Acto ou effeito de tratar.
Contracto; ajuste.
Convivência: \textunderscore no trato de gente honesta\textunderscore .
Conversação.
Alimentação; passadio: \textunderscore ganha o bastante para o trato da família\textunderscore .
Mau tratamento; torturas.
\section{Trauma}
\begin{itemize}
\item {Grp. gram.:m.}
\end{itemize}
O mesmo que \textunderscore traumatismo\textunderscore .
\section{Traumaticamente}
\begin{itemize}
\item {Grp. gram.:adv.}
\end{itemize}
De modo traumático.
Com traumatismo.
\section{Traumaticina}
\begin{itemize}
\item {Grp. gram.:f.}
\end{itemize}
\begin{itemize}
\item {Utilização:Pharm.}
\end{itemize}
\begin{itemize}
\item {Proveniência:(De \textunderscore traumático\textunderscore )}
\end{itemize}
Solução de guta-percha em chlorofórmio, para uso externo e para vehículo de várias substâncias medicamentosas.
\section{Traumático}
\begin{itemize}
\item {Grp. gram.:adj.}
\end{itemize}
\begin{itemize}
\item {Proveniência:(Lat. \textunderscore traumaticus\textunderscore )}
\end{itemize}
Relativo a feridas ou contusões.
\section{Traumatismo}
\begin{itemize}
\item {Grp. gram.:m.}
\end{itemize}
\begin{itemize}
\item {Proveniência:(Do gr. \textunderscore trauma\textunderscore , \textunderscore traumatos\textunderscore )}
\end{itemize}
Estado mórbido, resultante de um ferimento grave.
\section{Trausar}
\textunderscore v. t.\textunderscore  (e der.)
Fórma ant. de \textunderscore taxar\textunderscore , etc.
\section{Traussação}
\begin{itemize}
\item {Grp. gram.:f.}
\end{itemize}
\begin{itemize}
\item {Utilização:Ant.}
\end{itemize}
\begin{itemize}
\item {Proveniência:(De \textunderscore traussar\textunderscore )}
\end{itemize}
Comedorias, jantares ou casamentos, que as igrejas ou mosteiros pagavam, em dinheiro taxado, aos herdeiros dos seus instituidores ou doadores.
\section{Traussar}
\textunderscore v. t.\textunderscore  (e der.)
Fórma ant. de \textunderscore taxar\textunderscore , etc.
\section{Trauta}
\begin{itemize}
\item {Grp. gram.:f.}
\end{itemize}
\begin{itemize}
\item {Utilização:P. us.}
\end{itemize}
\begin{itemize}
\item {Proveniência:(Do lat. \textunderscore tractus\textunderscore )}
\end{itemize}
O rasto de caça.
\section{Trautar}
\begin{itemize}
\item {Grp. gram.:v. t.}
\end{itemize}
\begin{itemize}
\item {Utilização:Ant.}
\end{itemize}
\begin{itemize}
\item {Proveniência:(Do lat. \textunderscore tractare\textunderscore )}
\end{itemize}
O mesmo que \textunderscore contratar\textunderscore .
\section{Trautear}
\begin{itemize}
\item {Grp. gram.:v. t.  e  i.}
\end{itemize}
\begin{itemize}
\item {Utilização:Pop.}
\end{itemize}
\begin{itemize}
\item {Grp. gram.:V. t.}
\end{itemize}
\begin{itemize}
\item {Utilização:Pop.}
\end{itemize}
\begin{itemize}
\item {Proveniência:(Do ant. \textunderscore trauto\textunderscore , por \textunderscore trato\textunderscore ?)}
\end{itemize}
Cantarolar.
Importunar, burlar.
Repreender.
Dar pancadas em. Cf. G. Braga, \textunderscore Mal da Delfina\textunderscore , 138.
\section{Trauto}
\begin{itemize}
\item {Grp. gram.:m.}
\end{itemize}
\begin{itemize}
\item {Utilização:Ant.}
\end{itemize}
O mesmo que \textunderscore trato\textunderscore .
Convenção.
Tracto de terreno.
A têrça parte de uma légua. Cf. S. R. Viterbo, \textunderscore Elucidário\textunderscore .
\section{Trava}
\begin{itemize}
\item {Grp. gram.:f.}
\end{itemize}
\begin{itemize}
\item {Utilização:Des.}
\end{itemize}
\begin{itemize}
\item {Proveniência:(Do lat. \textunderscore trabs\textunderscore )}
\end{itemize}
Pequena trave.
\section{Trava}
\begin{itemize}
\item {Grp. gram.:f.}
\end{itemize}
\begin{itemize}
\item {Utilização:Prov.}
\end{itemize}
\begin{itemize}
\item {Utilização:trasm.}
\end{itemize}
Acto de travar.
Peia.
Inclinação alternada dos dentes da serra.
\section{Travação}
\begin{itemize}
\item {Grp. gram.:f.}
\end{itemize}
Acto ou effeito de travar.
Connexão.
Nexo.
Ligação de traves.
\section{Tràvacontas}
\begin{itemize}
\item {Grp. gram.:m.}
\end{itemize}
\begin{itemize}
\item {Proveniência:(De \textunderscore travar\textunderscore  + \textunderscore conta\textunderscore )}
\end{itemize}
Disputa.
Altercação, principalmente em ajustes de contas.
\section{Travadamente}
\begin{itemize}
\item {Grp. gram.:adv.}
\end{itemize}
De modo travado; com travação.
\section{Travadeira}
\begin{itemize}
\item {Grp. gram.:f.}
\end{itemize}
O mesmo que \textunderscore travadoira\textunderscore .
\section{Travado}
\begin{itemize}
\item {Grp. gram.:adj.}
\end{itemize}
\begin{itemize}
\item {Grp. gram.:Pl.  e  m. pl.}
\end{itemize}
\begin{itemize}
\item {Proveniência:(De \textunderscore travar\textunderscore )}
\end{itemize}
Ligado estreitamente.
Íntimo.
Peado.
Represado.
Iniciado, principiado.
Que não tem a língua desembaraçada; tartamudo.
Renhido, encarniçado: \textunderscore combate travado\textunderscore .
Impedido, atravancado.
Moderado, (falando-se do passo das cavalgaduras).
Refreado e andando a passo, (falando-se de cavalgaduras):«\textunderscore tinham éguas travadas, que entravam pelas feiras...\textunderscore »Camillo, \textunderscore Brasileira\textunderscore , 92.
Diz-se do equideo, que tem a mão e o pé esquerdos calçados.
Diz-se dos ventos fortes da costa da Guiné.
\section{Travadoira}
\begin{itemize}
\item {Grp. gram.:f.}
\end{itemize}
\begin{itemize}
\item {Utilização:Marn.}
\end{itemize}
\begin{itemize}
\item {Proveniência:(De \textunderscore travar\textunderscore )}
\end{itemize}
Utensilio de ferro, com que os serradores travam ou inclinam alternadamente os bicos da serra.
Peça de madeira ou de pedra, com que se impede a passagem da água, do corredor para as peças das salinas.
Pedra apparelhada, que se põe nas paredes de pedra miuda, para segurança da construcção, ou para receber as pontas das vigas, da cantaria, etc.
\section{Travadoiro}
\begin{itemize}
\item {Grp. gram.:m.}
\end{itemize}
\begin{itemize}
\item {Grp. gram.:Pl.}
\end{itemize}
\begin{itemize}
\item {Proveniência:(De \textunderscore travar\textunderscore )}
\end{itemize}
Lugar, a que se prende a trava ou peia, na perna dos animaes.
Rêgo, que rodeia os talhos, nas marinhas do Guadiana.
Botões de zinco, adaptados a um projéctil, para o guiarem pelas estrias do canhão.
\section{Travador}
\begin{itemize}
\item {Grp. gram.:m.  e  adj.}
\end{itemize}
O que trava.
O mesmo que \textunderscore travadoira\textunderscore .
\section{Travadoura}
\begin{itemize}
\item {Grp. gram.:f.}
\end{itemize}
\begin{itemize}
\item {Utilização:Marn.}
\end{itemize}
\begin{itemize}
\item {Proveniência:(De \textunderscore travar\textunderscore )}
\end{itemize}
Utensílio de ferro, com que os serradores travam ou inclinam alternadamente os bicos da serra.
Peça de madeira ou de pedra, com que se impede a passagem da água, do corredor para as peças das salinas.
Pedra apparelhada, que se põe nas paredes de pedra miuda, para segurança da construcção, ou para receber as pontas das vigas, da cantaria, etc.
\section{Travadouro}
\begin{itemize}
\item {Grp. gram.:m.}
\end{itemize}
\begin{itemize}
\item {Grp. gram.:Pl.}
\end{itemize}
\begin{itemize}
\item {Proveniência:(De \textunderscore travar\textunderscore )}
\end{itemize}
Lugar, a que se prende a trava ou peia, na perna dos animaes.
Rêgo, que rodeia os talhos, nas marinhas do Guadiana.
Botões de zinco, adaptados a um projéctil, para o guiarem pelas estrias do canhão.
\section{Travadura}
\begin{itemize}
\item {Grp. gram.:f.}
\end{itemize}
O mesmo que \textunderscore travação\textunderscore .
\section{Travagem}
\begin{itemize}
\item {Grp. gram.:f.}
\end{itemize}
\begin{itemize}
\item {Utilização:Bras}
\end{itemize}
\begin{itemize}
\item {Proveniência:(De \textunderscore travar\textunderscore ?)}
\end{itemize}
Inflammação nas gengivas dos cavallos.
\section{Traval}
\begin{itemize}
\item {Grp. gram.:adj.}
\end{itemize}
Relativo a trave.
\section{Travalho}
\textunderscore m.\textunderscore  (e der.) \textunderscore Prov. beir.\textunderscore 
O mesmo que \textunderscore trabalho\textunderscore , etc.
\section{Travamento}
\begin{itemize}
\item {Grp. gram.:m.}
\end{itemize}
O mesmo que \textunderscore travação\textunderscore .
\section{Travanca}
\begin{itemize}
\item {Grp. gram.:f.}
\end{itemize}
\begin{itemize}
\item {Proveniência:(De \textunderscore trave\textunderscore )}
\end{itemize}
Embaraço, obstáculo.
\section{Travão}
\begin{itemize}
\item {Grp. gram.:m.}
\end{itemize}
\begin{itemize}
\item {Proveniência:(De \textunderscore trave\textunderscore )}
\end{itemize}
Cadeia ou trava com que se peiam as bêstas.
Espécie de alavanca, que faz sustar o movimento de um vehiculo, de um maquinismo, etc.
\section{Travar}
\begin{itemize}
\item {Grp. gram.:v. t.}
\end{itemize}
\begin{itemize}
\item {Utilização:Ant.}
\end{itemize}
\begin{itemize}
\item {Grp. gram.:V. i.}
\end{itemize}
\begin{itemize}
\item {Utilização:Fig.}
\end{itemize}
\begin{itemize}
\item {Proveniência:(De \textunderscore trave\textunderscore )}
\end{itemize}
Prender.
Encadear.
Fazer parar com travão.
Pear.
Agarrar.
Tramar, entretecer.
Entabular: \textunderscore travar relações\textunderscore .
Causar travo.
Cruzar.
Voltar alternadamente para um e outro lado os bicos de, (uma serra).
Refrear, metendo a passo (uma cavalgadura).
Censurar.
Têr sabor amargo, \textunderscore têr travo\textunderscore .
Dar desgostos.
Lançar a mão: \textunderscore travou da espada\textunderscore .
\section{Trave}
\begin{itemize}
\item {Grp. gram.:f.}
\end{itemize}
\begin{itemize}
\item {Utilização:Prov.}
\end{itemize}
\begin{itemize}
\item {Utilização:dur.}
\end{itemize}
\begin{itemize}
\item {Utilização:Prov.}
\end{itemize}
\begin{itemize}
\item {Utilização:minh.}
\end{itemize}
\begin{itemize}
\item {Proveniência:(Do lat. \textunderscore trabs\textunderscore )}
\end{itemize}
Grande tronco de árvore, empregado em construcções.
Viga.
Trava, peia.
Arame, que liga a charneira da fivela ao arco.
Membrana sublingual: \textunderscore cortar a trave a uma criança\textunderscore .
Cada uma das pedras compridas, em que assenta a mó inferior do moínho.
\section{Travejamento}
\begin{itemize}
\item {Grp. gram.:m.}
\end{itemize}
\begin{itemize}
\item {Proveniência:(De \textunderscore travejar\textunderscore )}
\end{itemize}
O mesmo que \textunderscore vigamento\textunderscore .
\section{Travejar}
\begin{itemize}
\item {Grp. gram.:v. t.}
\end{itemize}
Pôr traves em; vigar.
\section{Travela}
\begin{itemize}
\item {Grp. gram.:f.}
\end{itemize}
\begin{itemize}
\item {Utilização:Prov.}
\end{itemize}
\begin{itemize}
\item {Utilização:trasm.}
\end{itemize}
\begin{itemize}
\item {Proveniência:(Do lat. \textunderscore trabs\textunderscore )}
\end{itemize}
Espécie de aldrava de madeira, em portas interiores.
Insecto coleóptero, (\textunderscore agriotes lineatus\textunderscore ), nocivo aos milhaes. Cf. \textunderscore Gaz. das Aldeias\textunderscore , V, 102.
\section{Travento}
\begin{itemize}
\item {Grp. gram.:adj.}
\end{itemize}
Que tem travo.
\section{Travertino}
\begin{itemize}
\item {Grp. gram.:m.}
\end{itemize}
\begin{itemize}
\item {Proveniência:(It. \textunderscore travertino\textunderscore )}
\end{itemize}
Variedade de pedra pardacenta, empregada geralmente nos edificios de Roma.
\section{Través}
\begin{itemize}
\item {Grp. gram.:m.}
\end{itemize}
\begin{itemize}
\item {Grp. gram.:Loc. adv.}
\end{itemize}
\begin{itemize}
\item {Proveniência:(Do lat. \textunderscore traverse\textunderscore )}
\end{itemize}
Esguelha, soslaio.
Flanco.
O mesmo que \textunderscore travéssa\textunderscore , ou peça de madeira atravessada:«\textunderscore ...os traveses que a nossa trincheira já tinha.\textunderscore »Vieira, VI, 119.
\textunderscore Ao través\textunderscore , o mesmo que \textunderscore através\textunderscore .
\section{Travéssa}
\begin{itemize}
\item {Grp. gram.:f.}
\end{itemize}
\begin{itemize}
\item {Utilização:Ant.}
\end{itemize}
\begin{itemize}
\item {Proveniência:(Do b. lat. \textunderscore traversa\textunderscore )}
\end{itemize}
Peça de madeira, atravessada sôbre outras, ou posta horizontalmente entre duas outras peças verticaes.
Vêrga de porta ou janela; padieira.
Viga.
Dormente, em que assentam os carris das linhas férreas.
Rua transversal, entre duas ruas mais importantes.
Galeria subterrânea, que estabelece communicação entre duas outras galerias.
Prato oblongo.
Travessia.
Cambapé.
Pente estreito e curvo, com que as mulhéres ou as crianças seguram o cabello.
Espécie de tributo, direito de portagem.
\section{Travessamente}
\begin{itemize}
\item {Grp. gram.:adv.}
\end{itemize}
De modo travêsso; com travessura; maliciosamente.
\section{Travessanho}
\begin{itemize}
\item {Grp. gram.:m.}
\end{itemize}
\begin{itemize}
\item {Proveniência:(De \textunderscore travéssa\textunderscore )}
\end{itemize}
Pequena viga, com que se arma a parte do frontal da janela, correspondente ao peitoril.
\section{Travessão}
\begin{itemize}
\item {Grp. gram.:adj.}
\end{itemize}
\begin{itemize}
\item {Grp. gram.:M.}
\end{itemize}
\begin{itemize}
\item {Proveniência:(De \textunderscore travêsso\textunderscore )}
\end{itemize}
Muito travêsso.
Atravessado.
Diz-se do vento contrário e forte.
Vento travessão. Cf. \textunderscore Lendas da Índia\textunderscore , c. VIII.
\section{Travessão}
\begin{itemize}
\item {Grp. gram.:m.}
\end{itemize}
\begin{itemize}
\item {Utilização:Bras. do S}
\end{itemize}
\begin{itemize}
\item {Utilização:Mús.}
\end{itemize}
Grande travéssa.
Risco ou traço, usado na escrita, para separar phrases, substituír parêntheses e para evitar a repetição de um termo de que se trata.
Braço da balança.
A parte larga da cincha.
Linha perpendicular, que atravessa a pauta, para separar as notas, que cada compasso contém.
\section{Travessar}
\begin{itemize}
\item {Grp. gram.:v. t.}
\end{itemize}
O mesmo que \textunderscore atravessar\textunderscore . Cf. Filinto, V, 273; VI, 275; VII, 21.
\section{Travessear}
\begin{itemize}
\item {Grp. gram.:v. i.}
\end{itemize}
\begin{itemize}
\item {Proveniência:(De \textunderscore travêsso\textunderscore )}
\end{itemize}
Fazer travessuras; traquinar.
\section{Travesseira}
\begin{itemize}
\item {Grp. gram.:f.}
\end{itemize}
\begin{itemize}
\item {Proveniência:(De \textunderscore travesseiro\textunderscore )}
\end{itemize}
O mesmo que \textunderscore fronha\textunderscore .
\section{Travesseiro}
\begin{itemize}
\item {Grp. gram.:m.}
\end{itemize}
\begin{itemize}
\item {Proveniência:(Do lat. \textunderscore traversarius\textunderscore )}
\end{itemize}
Almofada comprida, que se estende ao longo da testeira do leito e que serve para apoio da cabeça de quem se deita, ou para sustentar uma almofada mais pequena em que se deita a cabeça.
Pano, geralmente branco, com que se reveste ou se enfeita aquella almofada comprida.
Fronha.
Cabeçal.
Face do lado das volutas, num capitel de ordem jónica.
\section{Travessia}
\begin{itemize}
\item {Grp. gram.:f.}
\end{itemize}
\begin{itemize}
\item {Utilização:Ant.}
\end{itemize}
\begin{itemize}
\item {Utilização:Náut.}
\end{itemize}
\begin{itemize}
\item {Utilização:Náut.}
\end{itemize}
Vento forte e contrário á navegação.
Acto ou effeito de atravessar uma região, um continente, um mar, etc.
O Poente, ou a parte que fica para o Poente.
\textunderscore Travessia alta\textunderscore , vento de Noroéste.
\textunderscore Travessia baixa\textunderscore , vento de Sudoéste.
(Cp. cast. \textunderscore travesia\textunderscore )
\section{Travêsso}
\begin{itemize}
\item {Grp. gram.:adj.}
\end{itemize}
\begin{itemize}
\item {Grp. gram.:M.}
\end{itemize}
\begin{itemize}
\item {Utilização:Prov.}
\end{itemize}
\begin{itemize}
\item {Proveniência:(Do lat. \textunderscore traversus\textunderscore )}
\end{itemize}
Collocado de través.
Atravessado.
Lateral.
Collateral.
Contrário, oppôsto.
Turbulento; irrequieto: \textunderscore criança travêssa\textunderscore .
Malicioso.
Engraçado.
Que tem vivacidade.
O mesmo que \textunderscore travéssa\textunderscore  (de madeira).
Degrau de uma escada de mão.
\section{Travessura}
\begin{itemize}
\item {Grp. gram.:f.}
\end{itemize}
\begin{itemize}
\item {Proveniência:(De \textunderscore travêsso\textunderscore )}
\end{itemize}
Acto de pessôa travêssa.
Maldade infantil.
Malícia; desenvoltura.
\section{Traveta}
\begin{itemize}
\item {fónica:vê}
\end{itemize}
\begin{itemize}
\item {Grp. gram.:f.}
\end{itemize}
Trave pequena.
\section{Travia}
\begin{itemize}
\item {Grp. gram.:f.}
\end{itemize}
\begin{itemize}
\item {Utilização:Prov.}
\end{itemize}
\begin{itemize}
\item {Utilização:alg.}
\end{itemize}
\begin{itemize}
\item {Proveniência:(Do cast. \textunderscore travía\textunderscore ?)}
\end{itemize}
\textunderscore Perder a travía\textunderscore , perder a tramontana, desorientar-se.
\section{Travia}
\begin{itemize}
\item {Grp. gram.:f.}
\end{itemize}
\begin{itemize}
\item {Utilização:Prov.}
\end{itemize}
\begin{itemize}
\item {Utilização:alent.}
\end{itemize}
\begin{itemize}
\item {Proveniência:(De \textunderscore travo\textunderscore ?)}
\end{itemize}
Requeijão com soro.
\section{Travia}
\begin{itemize}
\item {Grp. gram.:f.}
\end{itemize}
\begin{itemize}
\item {Utilização:Prov.}
\end{itemize}
\begin{itemize}
\item {Utilização:alent.}
\end{itemize}
Massa de farelo e bagaço, para os porcos.
(Relaciona-se com \textunderscore travia\textunderscore ^2?)
\section{Traviata}
\begin{itemize}
\item {Grp. gram.:f.}
\end{itemize}
\begin{itemize}
\item {Utilização:P. us.}
\end{itemize}
\begin{itemize}
\item {Proveniência:(Do n. p. da protagonista de uma ópera. Cp. it. \textunderscore traviata\textunderscore , mulhér perdida)}
\end{itemize}
Cortesan infeliz e sympáthica.
\section{Travinca}
\begin{itemize}
\item {Grp. gram.:f.}
\end{itemize}
\begin{itemize}
\item {Utilização:Pop.}
\end{itemize}
\begin{itemize}
\item {Utilização:Prov.}
\end{itemize}
\begin{itemize}
\item {Utilização:trasm.}
\end{itemize}
\begin{itemize}
\item {Proveniência:(De \textunderscore trave\textunderscore )}
\end{itemize}
Trave pequena.
Cravelha.
Pequena travéssa de metal, na extremidade de algumas cadeias de relógio para as prender á botoeira.
Clavícula.
Pequena peça de pau, em fórma de ângulo obtuso, servindo de argola grosseira nas cilhas e sôbre-cargas.
\section{Travitéu}
\begin{itemize}
\item {Grp. gram.:m.}
\end{itemize}
\begin{itemize}
\item {Utilização:Prov.}
\end{itemize}
O mesmo que \textunderscore traveta\textunderscore .
\section{Travo}
\begin{itemize}
\item {Grp. gram.:m.}
\end{itemize}
\begin{itemize}
\item {Proveniência:(De \textunderscore travar\textunderscore )}
\end{itemize}
Saibo adstringente de comida ou bebida.
Amargor.
Impressão desagradável.
\section{Travoada}
\begin{itemize}
\item {Grp. gram.:f.}
\end{itemize}
\begin{itemize}
\item {Utilização:pop.}
\end{itemize}
\begin{itemize}
\item {Utilização:Ant.}
\end{itemize}
O mesmo que \textunderscore trovoada\textunderscore . Cf. \textunderscore Rot. do Mar Verm.\textunderscore , 66.
\section{Travoela}
\begin{itemize}
\item {Grp. gram.:f.}
\end{itemize}
Espécie de pequeno trado.
(Por \textunderscore tradoela\textunderscore , de \textunderscore trado\textunderscore ?)
\section{Travor}
\begin{itemize}
\item {Grp. gram.:m.}
\end{itemize}
\begin{itemize}
\item {Utilização:Pop.}
\end{itemize}
O mesmo que \textunderscore travo\textunderscore . Cf. Camillo, \textunderscore Estrêll. Fun.\textunderscore , 205.
\section{Travoso}
\begin{itemize}
\item {Grp. gram.:adj.}
\end{itemize}
O mesmo que \textunderscore travento\textunderscore .
\section{Travota}
\begin{itemize}
\item {Grp. gram.:f.}
\end{itemize}
\begin{itemize}
\item {Utilização:Prov.}
\end{itemize}
\begin{itemize}
\item {Utilização:trasm.}
\end{itemize}
\begin{itemize}
\item {Proveniência:(De \textunderscore trave\textunderscore )}
\end{itemize}
Castanheiro, delgado e direito.
\section{Traz!}
\begin{itemize}
\item {Grp. gram.:interj.}
\end{itemize}
\begin{itemize}
\item {Proveniência:(T. onom.)}
\end{itemize}
Voz imitativa de pancada ou quéda.
\section{Trazedeiro}
\begin{itemize}
\item {Grp. gram.:adj.}
\end{itemize}
\begin{itemize}
\item {Utilização:T. de Leiria}
\end{itemize}
Que se costuma trazer: \textunderscore um fato trazedeiro\textunderscore .
\section{Trazedor}
\begin{itemize}
\item {Grp. gram.:m.  e  adj.}
\end{itemize}
O que traz.
\section{Trazer}
\begin{itemize}
\item {Grp. gram.:v. t.}
\end{itemize}
Conduzir.
Transportar para cá.
Dirigir, commandar.
Causar, importar: \textunderscore trazer damno\textunderscore .
Vestir, usar: \textunderscore trazer calças brancas\textunderscore .
Têr, exhibir: \textunderscore traz aspecto de doente\textunderscore .
Communicar.
Derivar.
Herdar; receber de antepassados.
Manusear.
Fazer referência a.
Offertar: \textunderscore trago-lhe aqui uma lembrança\textunderscore .
Tratar; conter: \textunderscore êste livro traz uma história\textunderscore .
Fazer menção de, (falando-se de escritores ou das suas obras): \textunderscore o Vieira traz muitos termos, de origem brasílica. Os«Lusíadas»trazem referências à flora africana\textunderscore .
(Ant. \textunderscore trager\textunderscore , do lat. \textunderscore trahere\textunderscore )
\section{Trazida}
\begin{itemize}
\item {Grp. gram.:f.}
\end{itemize}
Acto ou effeito de trazer.
\section{Trazimento}
\begin{itemize}
\item {Grp. gram.:m.}
\end{itemize}
O mesmo que \textunderscore trazida\textunderscore .
\section{Tré}
\begin{itemize}
\item {Grp. gram.:m.}
\end{itemize}
Espécie de tecido antigo. Cf. \textunderscore Archeólogo Port.\textunderscore , VI, 1.
\section{Tre...}
\begin{itemize}
\item {Grp. gram.:pref.}
\end{itemize}
O mesmo que \textunderscore tres...\textunderscore ^2
\section{Trebelhar}
\begin{itemize}
\item {Grp. gram.:v. i.}
\end{itemize}
\begin{itemize}
\item {Utilização:Des.}
\end{itemize}
\begin{itemize}
\item {Proveniência:(De \textunderscore trebelho\textunderscore )}
\end{itemize}
Mover os trabelhos no xadrez.
Brincar, folgar.
\section{Trebelho}
\begin{itemize}
\item {fónica:bê}
\end{itemize}
\begin{itemize}
\item {Grp. gram.:m.}
\end{itemize}
\begin{itemize}
\item {Utilização:Ant.}
\end{itemize}
\begin{itemize}
\item {Utilização:Ant.}
\end{itemize}
Dança; folguedo; trabelho.
Fôro ou pensão que pagavam os vendedores de vinho a retalho.
Vaso pequeno, para medir vinho.
(Cp. \textunderscore trabelho\textunderscore )
\section{Trebilongo}
\begin{itemize}
\item {Grp. gram.:m.}
\end{itemize}
O mesmo que \textunderscore pernilongo\textunderscore , ave.
\section{Trebol}
\begin{itemize}
\item {Grp. gram.:m.}
\end{itemize}
\begin{itemize}
\item {Utilização:Bras}
\end{itemize}
Planta papilionácea, forraginosa.
\section{Trebola}
\begin{itemize}
\item {Grp. gram.:f.}
\end{itemize}
Cachalote.
O mesmo que \textunderscore trebolha\textunderscore .
\section{Trebolha}
\begin{itemize}
\item {fónica:bô}
\end{itemize}
\begin{itemize}
\item {Grp. gram.:f.}
\end{itemize}
\begin{itemize}
\item {Utilização:Ant.}
\end{itemize}
\begin{itemize}
\item {Proveniência:(Do b. lat. \textunderscore trepolia\textunderscore ?)}
\end{itemize}
Odre grande para vinho.
Embolha.
\section{Trecentésimo}
\begin{itemize}
\item {Grp. gram.:adj.}
\end{itemize}
\begin{itemize}
\item {Grp. gram.:M.}
\end{itemize}
\begin{itemize}
\item {Proveniência:(Lat. \textunderscore trecentesímus\textunderscore )}
\end{itemize}
Que occupa o último lugar numa série de trezentos.
Cada uma das trezentas partes em que se divide um todo.
\section{Trecentista}
\begin{itemize}
\item {Grp. gram.:m.}
\end{itemize}
\begin{itemize}
\item {Proveniência:(It. \textunderscore trecentista\textunderscore , de \textunderscore trecento\textunderscore )}
\end{itemize}
Nome, que se dá aos poétas italianos do século XIV.
\section{Trecha}
\begin{itemize}
\item {fónica:trê}
\end{itemize}
\begin{itemize}
\item {Grp. gram.:f.}
\end{itemize}
\begin{itemize}
\item {Utilização:Prov.}
\end{itemize}
\begin{itemize}
\item {Utilização:minh.}
\end{itemize}
Bátega de água.
\section{Trecheio}
\begin{itemize}
\item {Grp. gram.:adj.}
\end{itemize}
\begin{itemize}
\item {Utilização:Des.}
\end{itemize}
\begin{itemize}
\item {Proveniência:(De \textunderscore tre...\textunderscore  + \textunderscore cheio\textunderscore )}
\end{itemize}
Muito cheio.
\section{Trecho}
\begin{itemize}
\item {fónica:trê}
\end{itemize}
\begin{itemize}
\item {Grp. gram.:m.}
\end{itemize}
\begin{itemize}
\item {Utilização:Prov.}
\end{itemize}
\begin{itemize}
\item {Utilização:dur.}
\end{itemize}
\begin{itemize}
\item {Grp. gram.:Loc. adv.}
\end{itemize}
\begin{itemize}
\item {Grp. gram.:Loc. adv.}
\end{itemize}
Espaço de tempo ou lugar.
Intervallo.
Extracto.
Excerpto de uma obra literária ou musical.
Fragmento.
Pequena pedra, geralmente vermelha, com que os pedreiros riscam, á régua, as pedras que querem apparelhar.
\textunderscore A trecho\textunderscore  ou \textunderscore a trechos\textunderscore , de quando em quando.
\textunderscore A pouco trecho\textunderscore , dentro de pouco tempo, logo.
(Cast. \textunderscore trecho\textunderscore , do lat. \textunderscore tractus\textunderscore )
\section{Treçó}
\begin{itemize}
\item {Grp. gram.:m.}
\end{itemize}
Última ave ou a mais inferior, de uma ninhada de falcões ou açores. Cf. Fil. Simões, \textunderscore Escritos Diversos\textunderscore , 56.
(Por \textunderscore terçó\textunderscore . V. \textunderscore terçó\textunderscore )
\section{Treçolho}
\begin{itemize}
\item {fónica:çô}
\end{itemize}
\begin{itemize}
\item {Grp. gram.:m.}
\end{itemize}
(V.terçolho)
\section{Trécula}
\begin{itemize}
\item {Grp. gram.:f.}
\end{itemize}
Espécie de matraca, para afugentar aves das plantas ou das árvores fructíferas. Cf. Filinto, VI, 235.
(Corr. de \textunderscore tecla\textunderscore ?)
\section{Tredecimal}
\begin{itemize}
\item {Grp. gram.:adj.}
\end{itemize}
\begin{itemize}
\item {Utilização:Miner.}
\end{itemize}
\begin{itemize}
\item {Proveniência:(Do lat. \textunderscore tredecim\textunderscore )}
\end{itemize}
Diz-se da substância, cujos crystaes têm treze faces.
\section{Tredécimo}
\begin{itemize}
\item {Grp. gram.:adj.}
\end{itemize}
\begin{itemize}
\item {Proveniência:(Do lat. \textunderscore tredecim\textunderscore )}
\end{itemize}
O mesmo que [[décimo]]-[[terceiro]]:«\textunderscore D. Joam da gloriosa memoria, dos reys o tredécimo\textunderscore ». R. Pina, \textunderscore João II\textunderscore .
\section{Tredice}
\begin{itemize}
\item {Grp. gram.:f.}
\end{itemize}
Qualidade de tredo. Cf. \textunderscore Aulegrafia\textunderscore , 45.
\section{Tredo}
\begin{itemize}
\item {fónica:trê}
\end{itemize}
\begin{itemize}
\item {Grp. gram.:adj.}
\end{itemize}
\begin{itemize}
\item {Proveniência:(Do rad. do lat. \textunderscore tradere\textunderscore . Cp. borgonhês \textunderscore treite\textunderscore , fr. \textunderscore traîte\textunderscore )}
\end{itemize}
Traiçoeiro; falso.
\section{Tredor}
\begin{itemize}
\item {Grp. gram.:m.}
\end{itemize}
\begin{itemize}
\item {Utilização:Ant.}
\end{itemize}
O mesmo que \textunderscore traidor\textunderscore . Cf. Pant. de Aveiro, \textunderscore Itiner.\textunderscore , 155 v.^o, (2.^a ed.).
(Cp. \textunderscore tredo\textunderscore )
\section{Trêfego}
\begin{itemize}
\item {Grp. gram.:adj.}
\end{itemize}
\begin{itemize}
\item {Proveniência:(De \textunderscore tráfico\textunderscore ?)}
\end{itemize}
Turbulento; traquinas.
Manhoso; astuto.
\section{Trefo}
\begin{itemize}
\item {fónica:trê}
\end{itemize}
\begin{itemize}
\item {Grp. gram.:adj.}
\end{itemize}
\begin{itemize}
\item {Proveniência:(Do cast. \textunderscore trefe\textunderscore )}
\end{itemize}
O mesmo que \textunderscore trêfego\textunderscore .
\section{Tregeito}
\textunderscore m.\textunderscore  (e der.)
(V. \textunderscore trejeito\textunderscore , etc.)
\section{Trégua}
\begin{itemize}
\item {Grp. gram.:f.}
\end{itemize}
\begin{itemize}
\item {Proveniência:(Do got. \textunderscore triggua\textunderscore )}
\end{itemize}
Suspensão temporária de hostilidades.
Descanso; férias.
\section{Treição}
\begin{itemize}
\item {Grp. gram.:f.}
\end{itemize}
\begin{itemize}
\item {Utilização:Ant.}
\end{itemize}
O mesmo que \textunderscore traição\textunderscore . Cf. G. Vicente, I, (?).
(Cp. \textunderscore tredo\textunderscore )
\section{Treicha}
\begin{itemize}
\item {Grp. gram.:f.}
\end{itemize}
O mesmo que \textunderscore treichada\textunderscore .
\section{Treichada}
\begin{itemize}
\item {Grp. gram.:f.}
\end{itemize}
\begin{itemize}
\item {Utilização:Prov.}
\end{itemize}
\begin{itemize}
\item {Utilização:minh.}
\end{itemize}
Bátega de água.
(Cp. \textunderscore trecha\textunderscore )
\section{Treina}
\begin{itemize}
\item {Grp. gram.:f.}
\end{itemize}
\begin{itemize}
\item {Utilização:Fig.}
\end{itemize}
\begin{itemize}
\item {Proveniência:(Fr. \textunderscore traine\textunderscore )}
\end{itemize}
Animal, sôbre que os caçadores davam de comer ao falcão, para o adestrar na caça.
Cevo.
\section{Treinado}
\begin{itemize}
\item {Grp. gram.:adj.}
\end{itemize}
\begin{itemize}
\item {Grp. gram.:M.}
\end{itemize}
\begin{itemize}
\item {Proveniência:(De \textunderscore treinar\textunderscore )}
\end{itemize}
Acostumado; exercitado.
Falcão ou açôr, já adestrado para a caça. Cf. Fil. Simões, \textunderscore Escritos Diversos\textunderscore , 56.
\section{Treinagem}
\begin{itemize}
\item {Grp. gram.:f.}
\end{itemize}
O mesmo que \textunderscore treinamento\textunderscore .
\section{Treinamento}
\begin{itemize}
\item {Grp. gram.:m.}
\end{itemize}
Acto de treinar.
\section{Treinar}
\begin{itemize}
\item {Grp. gram.:v. t.}
\end{itemize}
\begin{itemize}
\item {Utilização:Fig.}
\end{itemize}
\begin{itemize}
\item {Grp. gram.:V. p.}
\end{itemize}
\begin{itemize}
\item {Proveniência:(De \textunderscore treina\textunderscore )}
\end{itemize}
Dar cevo a (aves).
Acostumar; adestrar.
Exercitar-se para corridas ou festas de desporte.
\section{Treino}
\begin{itemize}
\item {Grp. gram.:m.}
\end{itemize}
\begin{itemize}
\item {Proveniência:(De \textunderscore treinar\textunderscore )}
\end{itemize}
Acto de se treinarem ou adestrarem pessôas ou animaes para torneios ou festas de exercícios phýsicos.
\section{Treita}
\begin{itemize}
\item {Grp. gram.:f.}
\end{itemize}
\begin{itemize}
\item {Utilização:T. da Bairrada}
\end{itemize}
\begin{itemize}
\item {Utilização:Prov.}
\end{itemize}
\begin{itemize}
\item {Utilização:minh.}
\end{itemize}
\begin{itemize}
\item {Proveniência:(Do lat. \textunderscore tracta\textunderscore )}
\end{itemize}
Vestígio, pègada.
Belga, nesga de terra.
Cada uma das tiras de terreno lavrado, separadas por meio de ramos, para facilitar a distribuição da semente.
\section{Treitar}
\begin{itemize}
\item {Grp. gram.:v. t.}
\end{itemize}
\begin{itemize}
\item {Utilização:Prov.}
\end{itemize}
\begin{itemize}
\item {Utilização:minh.}
\end{itemize}
Dividir em treitas (um terreno).
\section{Treitento}
\begin{itemize}
\item {Grp. gram.:adj.}
\end{itemize}
Que usa tretas; manhoso.
(Por \textunderscore tretento\textunderscore , de \textunderscore treta\textunderscore )
\section{Treito}
\begin{itemize}
\item {Grp. gram.:adj.}
\end{itemize}
(V.atreito)
\section{Treito}
\begin{itemize}
\item {Grp. gram.:m.}
\end{itemize}
\begin{itemize}
\item {Utilização:Ant.}
\end{itemize}
Contrato, ajuste.
(Cp. \textunderscore trato\textunderscore )
\section{Treitoeira}
\begin{itemize}
\item {Grp. gram.:f.}
\end{itemize}
\begin{itemize}
\item {Utilização:Prov.}
\end{itemize}
\begin{itemize}
\item {Proveniência:(Do lat. \textunderscore tractus\textunderscore ?)}
\end{itemize}
Cada um dos paus, que descem das chedas, e entre os quaes se move o eixo do carro.
\section{Treitoira}
\begin{itemize}
\item {Grp. gram.:f.}
\end{itemize}
\begin{itemize}
\item {Utilização:Prov.}
\end{itemize}
\begin{itemize}
\item {Utilização:trasm.}
\end{itemize}
O mesmo que \textunderscore treitoeira\textunderscore .
\section{Treixa}
\begin{itemize}
\item {Grp. gram.:f.}
\end{itemize}
\begin{itemize}
\item {Utilização:Prov.}
\end{itemize}
\begin{itemize}
\item {Utilização:minh.}
\end{itemize}
Bátega de água.
(Cp. \textunderscore trecha\textunderscore )
\section{Trejeitador}
\begin{itemize}
\item {Grp. gram.:m.  e  adj.}
\end{itemize}
\begin{itemize}
\item {Utilização:Ant.}
\end{itemize}
O que trejeita.
Espécie de bobo.
\section{Trejeitar}
\begin{itemize}
\item {Grp. gram.:v. i.}
\end{itemize}
Fazer trejeitos.
\section{Trejeitear}
\begin{itemize}
\item {Grp. gram.:v. i.}
\end{itemize}
O mesmo que \textunderscore trejeitar\textunderscore . Cf. Arn. Gama, \textunderscore Segr. do Abb.\textunderscore , 136.
\section{Trejeito}
\begin{itemize}
\item {Grp. gram.:m.}
\end{itemize}
\begin{itemize}
\item {Proveniência:(De \textunderscore tre...\textunderscore  + \textunderscore jeito\textunderscore )}
\end{itemize}
Gesto.
Esgares.
Careta.
Prestidigitação.
\section{Trejeitoso}
\begin{itemize}
\item {Grp. gram.:adj.}
\end{itemize}
Que faz trejeitos, trejeitador:«\textunderscore ...garganteou mui affectada e trejeitosa...\textunderscore »Camillo, \textunderscore Caveira\textunderscore , 221.
\section{Trejugado}
\begin{itemize}
\item {Grp. gram.:adj.}
\end{itemize}
\begin{itemize}
\item {Utilização:Prov.}
\end{itemize}
\begin{itemize}
\item {Utilização:minh.}
\end{itemize}
\begin{itemize}
\item {Proveniência:(De \textunderscore tre...\textunderscore  + \textunderscore jugo\textunderscore )}
\end{itemize}
Diz-se do boi, quando cái de maneira que, virando-se o arco do jugo, corre o perigo de ficar esganado, se lhe não acudirem.
\section{Trejurar}
\begin{itemize}
\item {Grp. gram.:v. i.}
\end{itemize}
\begin{itemize}
\item {Grp. gram.:V. i.}
\end{itemize}
\begin{itemize}
\item {Proveniência:(De \textunderscore tre...\textunderscore  + \textunderscore jurar\textunderscore )}
\end{itemize}
Affirmar, jurando muitas vezes.
Jurar três vezes, jurar muitas vezes:«\textunderscore pois se lhe juro, rejuro e trejuro...\textunderscore »Castilho, \textunderscore Sabichonas\textunderscore , 26.
\section{Trela}
\begin{itemize}
\item {Grp. gram.:f.}
\end{itemize}
\begin{itemize}
\item {Utilização:Pop.}
\end{itemize}
\begin{itemize}
\item {Utilização:Fig.}
\end{itemize}
Tira de coiro, com que se prende o cão de caça.
Tagarelice; conversa, cavaco.
Licença, liberdade.
(Talvez contr. de \textunderscore tarela\textunderscore )
\section{Treladar}
\begin{itemize}
\item {Grp. gram.:v. i.}
\end{itemize}
\begin{itemize}
\item {Utilização:Prov.}
\end{itemize}
\begin{itemize}
\item {Utilização:trasm.}
\end{itemize}
Desenvolver-se (uma planta).
Correr bem (um negócio).
\section{Treladar}
\begin{itemize}
\item {Grp. gram.:v. t.}
\end{itemize}
\begin{itemize}
\item {Utilização:Ant.}
\end{itemize}
O mesmo que \textunderscore trasladar\textunderscore . Cf. \textunderscore Églogas de Crisfal\textunderscore .
\section{Trelado}
\begin{itemize}
\item {Grp. gram.:m.}
\end{itemize}
\begin{itemize}
\item {Utilização:Ant.}
\end{itemize}
(V.traslado)
\section{Trelente}
\begin{itemize}
\item {Grp. gram.:m.  e  f.}
\end{itemize}
\begin{itemize}
\item {Utilização:Bras}
\end{itemize}
Pessôa que trelê.
Tagarela.
\section{Treler}
\begin{itemize}
\item {Grp. gram.:v. i.}
\end{itemize}
\begin{itemize}
\item {Utilização:Bras}
\end{itemize}
\begin{itemize}
\item {Proveniência:(De \textunderscore trela\textunderscore ?)}
\end{itemize}
Tagarelar; dar trela.
\section{Trelho}
\begin{itemize}
\item {fónica:trê}
\end{itemize}
\begin{itemize}
\item {Grp. gram.:m.}
\end{itemize}
\begin{itemize}
\item {Grp. gram.:Loc. adv.}
\end{itemize}
\begin{itemize}
\item {Proveniência:(Do lat. \textunderscore tribulum\textunderscore )}
\end{itemize}
Utensílio, com que se bate a manteiga ao fabricar-se.
\textunderscore Sem trelho nem trabelho\textunderscore , sem tom nem som; disparatadamente.
\section{Treliça}
\begin{itemize}
\item {Grp. gram.:f.}
\end{itemize}
\begin{itemize}
\item {Utilização:Bras}
\end{itemize}
\begin{itemize}
\item {Proveniência:(Do lat. \textunderscore trilix\textunderscore )}
\end{itemize}
Rêde metállica, para resguardo: \textunderscore uma ponte com treliça\textunderscore .
\section{Trem}
\begin{itemize}
\item {Grp. gram.:m.}
\end{itemize}
\begin{itemize}
\item {Grp. gram.:Pl.}
\end{itemize}
\begin{itemize}
\item {Utilização:Bras. de Minas}
\end{itemize}
\begin{itemize}
\item {Proveniência:(Fr. \textunderscore train\textunderscore )}
\end{itemize}
Conjunto das malas ou de quaesquer outros objectos, que constituem a bagagem de um viajante.
Comitiva.
Mobília de uma casa.
Conjunto de utensílios e dos mais objectos, próprios para um certo serviço: \textunderscore o trem da cozinha\textunderscore .
Qualquer carruagem: \textunderscore meti-me num trem\textunderscore .
Traje.
Objectos, coisas: \textunderscore um armário cheio de trens\textunderscore .
\section{Trema}
\begin{itemize}
\item {Grp. gram.:m.}
\end{itemize}
\begin{itemize}
\item {Proveniência:(Gr. \textunderscore trema\textunderscore )}
\end{itemize}
Sinal ortográphico que, collocado sôbre uma vogal, indica que ella não fórma ditongo com a que lhe está próxima: \textunderscore saüdade\textunderscore , \textunderscore baïano\textunderscore , \textunderscore saïmento\textunderscore .
\section{Tremândreas}
\begin{itemize}
\item {Grp. gram.:f. pl.}
\end{itemize}
Família de plantas que tem por typo o tremandro.
(Fem. pl. de \textunderscore tremândreo\textunderscore )
\section{Tremândreo}
\begin{itemize}
\item {Grp. gram.:adj.}
\end{itemize}
Relativo ou semelhante ao tremandro.
\section{Tremandro}
\begin{itemize}
\item {Grp. gram.:m.}
\end{itemize}
\begin{itemize}
\item {Proveniência:(Do gr. \textunderscore trema\textunderscore  + \textunderscore aner\textunderscore , \textunderscore andros\textunderscore )}
\end{itemize}
Gênero de arbustos australianos, de cujas espécies há algumas que são cultivadas nas estufas europeias.
\section{Tremar}
\begin{itemize}
\item {Grp. gram.:v. t.}
\end{itemize}
Pôr trema em.
\section{Tremar}
\begin{itemize}
\item {Grp. gram.:v. t.}
\end{itemize}
\begin{itemize}
\item {Proveniência:(Do ant. fr. \textunderscore tremuer\textunderscore )}
\end{itemize}
Descompor os fios de; destramar.
\section{Tremate}
\begin{itemize}
\item {Grp. gram.:m.}
\end{itemize}
Planta brasileira, da fam. das synanthéreas.
\section{Trematódeos}
\begin{itemize}
\item {Grp. gram.:m. pl.}
\end{itemize}
\begin{itemize}
\item {Proveniência:(Do gr. \textunderscore trematodes\textunderscore )}
\end{itemize}
Classe de vermes parasitos, munidos de ventosas.
\section{Tremebundo}
\begin{itemize}
\item {Grp. gram.:adj.}
\end{itemize}
\begin{itemize}
\item {Utilização:Poét.}
\end{itemize}
\begin{itemize}
\item {Proveniência:(Lat. \textunderscore tremebundus\textunderscore )}
\end{itemize}
Que treme; que faz tremer.
\section{Tremecém}
\begin{itemize}
\item {Grp. gram.:adj.}
\end{itemize}
O mesmo que \textunderscore tremês\textunderscore .
\section{Tremedal}
\begin{itemize}
\item {Grp. gram.:m.}
\end{itemize}
\begin{itemize}
\item {Utilização:Fig.}
\end{itemize}
\begin{itemize}
\item {Proveniência:(Do lat. \textunderscore tremere\textunderscore )}
\end{itemize}
Pântano; lameiro; lodaçal.
Degradação moral, torpeza.
\section{Tremedeira}
\begin{itemize}
\item {Grp. gram.:f.}
\end{itemize}
\begin{itemize}
\item {Utilização:Bras}
\end{itemize}
\begin{itemize}
\item {Utilização:Pop.}
\end{itemize}
Peixe da Póvoa-de-Varzim, (\textunderscore torpedo marmorata\textunderscore ).
Tremor, tremura.
\section{Tremedor}
\begin{itemize}
\item {Grp. gram.:adj.}
\end{itemize}
\begin{itemize}
\item {Grp. gram.:M.}
\end{itemize}
\begin{itemize}
\item {Proveniência:(De \textunderscore tremer\textunderscore )}
\end{itemize}
Que treme.
O mesmo que \textunderscore tremelga\textunderscore .
\section{Tremedura}
\begin{itemize}
\item {Grp. gram.:f.}
\end{itemize}
O mesmo que \textunderscore tremura\textunderscore .
\section{Tremela}
\begin{itemize}
\item {Grp. gram.:f.}
\end{itemize}
Gênero de cogumelos.
\section{Tremelear}
\begin{itemize}
\item {Grp. gram.:v. i.}
\end{itemize}
Tremelicar.
Estar perplexo.
Tartamudear.
(Por \textunderscore tremulear\textunderscore , de \textunderscore trêmulo\textunderscore )
\section{Tremelga}
\begin{itemize}
\item {Grp. gram.:f.}
\end{itemize}
\begin{itemize}
\item {Proveniência:(De \textunderscore tremer\textunderscore )}
\end{itemize}
Gênero de peixes pércidas, o mesmo que \textunderscore torpedo\textunderscore .
\section{Tremelhicar}
\begin{itemize}
\item {Grp. gram.:v. i.}
\end{itemize}
O mesmo que \textunderscore tremelicar\textunderscore . Cf. Filinto, IV, 103.
\section{Tremelhique}
\begin{itemize}
\item {Grp. gram.:m.}
\end{itemize}
O mesmo que \textunderscore tremelique\textunderscore .
\section{Tremelica}
\begin{itemize}
\item {Grp. gram.:m. ,  f.  e  adj.}
\end{itemize}
\begin{itemize}
\item {Proveniência:(De \textunderscore tremelicar\textunderscore )}
\end{itemize}
Pessôa, que se assusta facilmente; pusillânime.
\section{Tremelicar}
\begin{itemize}
\item {Grp. gram.:v. i.}
\end{itemize}
Tremer, de susto.
Tremer muitas vezes.
\section{Tremelicoso}
\begin{itemize}
\item {Grp. gram.:adj.}
\end{itemize}
\begin{itemize}
\item {Proveniência:(De \textunderscore tremelicar\textunderscore )}
\end{itemize}
O mesmo que \textunderscore trémulo\textunderscore .
\section{Tremelique}
\begin{itemize}
\item {Grp. gram.:m.}
\end{itemize}
Acto de tremelicar.
\section{Temeluzente}
\begin{itemize}
\item {Grp. gram.:adj.}
\end{itemize}
Que tremeluz.
\section{Tremeluzir}
\begin{itemize}
\item {Grp. gram.:v. i.}
\end{itemize}
\begin{itemize}
\item {Proveniência:(De \textunderscore tremer\textunderscore  + \textunderscore luzir\textunderscore )}
\end{itemize}
Brilhar, tremendo; scintillar.--É t. inventado por Filinto.
\section{Tremenda}
\begin{itemize}
\item {Grp. gram.:f.}
\end{itemize}
\begin{itemize}
\item {Proveniência:(De \textunderscore tremendo\textunderscore )}
\end{itemize}
Pedaço de toicinho, que os frades de San-Bento costumavam comer a certas horas da noite. Cf. Garrett, \textunderscore D. Branca\textunderscore , 11, 13, 35 e 229.
\section{Tremendamente}
\begin{itemize}
\item {Grp. gram.:adv.}
\end{itemize}
De modo tremendo.
Extraordinariamente.
De modo pavoroso.
\section{Tremendo}
\begin{itemize}
\item {Grp. gram.:adj.}
\end{itemize}
\begin{itemize}
\item {Proveniência:(Lat. \textunderscore tremendus\textunderscore )}
\end{itemize}
Que causa temor.
Que faz tremer; horroroso: \textunderscore crime tremendo\textunderscore .
Extraordinário.
Respeitável.
\section{Tremente}
\begin{itemize}
\item {Grp. gram.:adj.}
\end{itemize}
\begin{itemize}
\item {Proveniência:(Lat. \textunderscore tremens\textunderscore )}
\end{itemize}
Que treme.
\section{Trementes}
\begin{itemize}
\item {Grp. gram.:adv.}
\end{itemize}
\begin{itemize}
\item {Utilização:Ant.}
\end{itemize}
O mesmo que \textunderscore entrementes\textunderscore .
\section{Trementina}
\begin{itemize}
\item {Grp. gram.:f.}
\end{itemize}
Designação vulgar da terebinthina.
Entretanto, cp. \textunderscore tormentina\textunderscore .
\section{Tremer}
\begin{itemize}
\item {Grp. gram.:v. t.}
\end{itemize}
\begin{itemize}
\item {Grp. gram.:V. i.}
\end{itemize}
\begin{itemize}
\item {Proveniência:(Lat. \textunderscore tremere\textunderscore )}
\end{itemize}
Têr mêdo de; recear.
Agitar.
Fazer tremer.
Tiritar por causa de.
Agitar-se; ser abalado pelo temor.
Assustar-se.
Ondular: \textunderscore tremem as franças do arvoredo\textunderscore .
Tiritar de frio, de susto, ou por doença.
Tremeluzir.
\section{Tremês}
\begin{itemize}
\item {Grp. gram.:adj.}
\end{itemize}
\begin{itemize}
\item {Proveniência:(Do lat. \textunderscore trimensis\textunderscore )}
\end{itemize}
Que dura três meses.
Que nasce e amadurece em três meses: \textunderscore trigo tremês\textunderscore .
\section{Tremesinho}
\begin{itemize}
\item {Grp. gram.:adj.}
\end{itemize}
O mesmo que \textunderscore tremês\textunderscore .
\section{Tremeter-se}
\begin{itemize}
\item {Grp. gram.:v. p.}
\end{itemize}
\begin{itemize}
\item {Utilização:Ant.}
\end{itemize}
O mesmo que [[intrometer-se|intrometer]]. Us. por Fernão Lopes.
(Por \textunderscore entremeter-se\textunderscore , de \textunderscore entre\textunderscore  + \textunderscore meter\textunderscore )
\section{Treme-treme}
\begin{itemize}
\item {Grp. gram.:m.}
\end{itemize}
\begin{itemize}
\item {Utilização:Bras}
\end{itemize}
Peixe, o mesmo que \textunderscore tremelga\textunderscore . Cf. \textunderscore Jorn-do-Comm.\textunderscore , do Rio de 22-III-902.
\section{Tremidamente}
\begin{itemize}
\item {Grp. gram.:adv.}
\end{itemize}
\begin{itemize}
\item {Proveniência:(De \textunderscore tremido\textunderscore )}
\end{itemize}
A tremer; com tremor.
\section{Tremulento}
\begin{itemize}
\item {Grp. gram.:adj.}
\end{itemize}
\begin{itemize}
\item {Utilização:Neol.}
\end{itemize}
\begin{itemize}
\item {Proveniência:(De \textunderscore trêmulo\textunderscore )}
\end{itemize}
O mesmo que \textunderscore tremulante\textunderscore . Cf. Deusdado, \textunderscore Escorços\textunderscore , 53.
\section{Tremulina}
\begin{itemize}
\item {Grp. gram.:f.}
\end{itemize}
\begin{itemize}
\item {Proveniência:(De \textunderscore trêmulo\textunderscore )}
\end{itemize}
Tremor superficial.
Reflexo trêmulo da luz, na superfície das águas levemente agitadas: \textunderscore a tremulina do Tejo\textunderscore .
\section{Trêmulo}
\begin{itemize}
\item {Grp. gram.:adj.}
\end{itemize}
\begin{itemize}
\item {Grp. gram.:M.}
\end{itemize}
\begin{itemize}
\item {Grp. gram.:Pl.}
\end{itemize}
\begin{itemize}
\item {Proveniência:(Lat. \textunderscore tremulus\textunderscore )}
\end{itemize}
Que treme.
Hesitante.
Indeciso.
Scintillante.
Tímido.
Tremido na voz ou no canto.
Multiplicação rápida das vibrações de um instrumento sôbre a mesma nota.
Conjunto de pedras preciosas, formando flôres e oscilando na extremidade de pequenos arames.
\section{Trémulo}
\begin{itemize}
\item {Grp. gram.:adj.}
\end{itemize}
\begin{itemize}
\item {Grp. gram.:M.}
\end{itemize}
\begin{itemize}
\item {Grp. gram.:Pl.}
\end{itemize}
\begin{itemize}
\item {Proveniência:(Lat. \textunderscore tremulus\textunderscore )}
\end{itemize}
Que treme.
Hesitante.
Indeciso.
Scintillante.
Tímido.
Tremido na voz ou no canto.
Multiplicação rápida das vibrações de um instrumento sôbre a mesma nota.
Conjunto de pedras preciosas, formando flôres e oscilando na extremidade de pequenos arames.
\section{Tremuloso}
\begin{itemize}
\item {Grp. gram.:adj.}
\end{itemize}
(V.trêmulo)
\section{Tremunido}
\begin{itemize}
\item {Grp. gram.:m.}
\end{itemize}
\begin{itemize}
\item {Utilização:Prov.}
\end{itemize}
\begin{itemize}
\item {Utilização:alg.}
\end{itemize}
\begin{itemize}
\item {Proveniência:(De \textunderscore tremer\textunderscore )}
\end{itemize}
Rumor trêmulo, como o da carruagem que roda sobre a calçada.
\section{Tremura}
\begin{itemize}
\item {Grp. gram.:f.}
\end{itemize}
O mesmo que \textunderscore tremor\textunderscore .
\section{Trena}
\begin{itemize}
\item {Grp. gram.:f.}
\end{itemize}
Fita preciosa para atar o cabello.
Baraço de pião.
(Provn. \textunderscore trena\textunderscore , do lat. \textunderscore trinas\textunderscore ?)
\section{Trenar}
\begin{itemize}
\item {Grp. gram.:v. t.}
\end{itemize}
O mesmo que \textunderscore treinar\textunderscore .
Sujeitar (homem ou animal) a exercicios, que, rápida e completamente, o tornam apto para certos trabalhos.
\section{Trença}
\begin{itemize}
\item {Grp. gram.:f.}
\end{itemize}
O mesmo que \textunderscore trança\textunderscore ? Cf. \textunderscore Viriato Trág\textunderscore ., XIV, 49.
\section{Trenel}
\begin{itemize}
\item {Grp. gram.:m.}
\end{itemize}
O mesmo que \textunderscore trenó\textunderscore . Cf. Filinto, IV, 268; IX, 183.
\section{Trengo}
\begin{itemize}
\item {Grp. gram.:m.}
\end{itemize}
\begin{itemize}
\item {Utilização:Prov.}
\end{itemize}
\begin{itemize}
\item {Utilização:minh.}
\end{itemize}
Homem acanhado, sem préstimo. (Colhido em Barcelos)
\section{Treno}
\begin{itemize}
\item {Grp. gram.:m.}
\end{itemize}
\begin{itemize}
\item {Proveniência:(Lat. \textunderscore threnus\textunderscore )}
\end{itemize}
Canto plangente; lamentação; elegia.
\section{Treno}
\begin{itemize}
\item {Grp. gram.:m.}
\end{itemize}
Acto de trenar; o mesmo que \textunderscore treinagem\textunderscore .
\section{Trenó}
\begin{itemize}
\item {Grp. gram.:m.}
\end{itemize}
\begin{itemize}
\item {Proveniência:(Fr. \textunderscore traineau\textunderscore )}
\end{itemize}
Espécie de vehículo sem rodas, próprio para andar sôbre o gêlo e usado nos países do Norte.
\section{Trentoira}
\begin{itemize}
\item {Grp. gram.:f.}
\end{itemize}
\begin{itemize}
\item {Utilização:Prov.}
\end{itemize}
\begin{itemize}
\item {Utilização:beir.}
\end{itemize}
\begin{itemize}
\item {Utilização:minh.}
\end{itemize}
Parte do vessadoiro, encaixada por meio da chavelha no tamoeiro.
(Cp. \textunderscore treitoeira\textunderscore )
\section{Trépa}
\begin{itemize}
\item {Grp. gram.:f.}
\end{itemize}
\begin{itemize}
\item {Utilização:Pop.}
\end{itemize}
Sova, tunda.
Reprehensão; crítica.
\section{Trépa}
\begin{itemize}
\item {Grp. gram.:f.}
\end{itemize}
\begin{itemize}
\item {Utilização:Ant.}
\end{itemize}
Folho de vestido. Cf. G. Vicente.
\section{Trêpa}
\begin{itemize}
\item {Grp. gram.:f.}
\end{itemize}
\begin{itemize}
\item {Utilização:Prov.}
\end{itemize}
\begin{itemize}
\item {Utilização:Prov.}
\end{itemize}
\begin{itemize}
\item {Utilização:minh.}
\end{itemize}
\begin{itemize}
\item {Proveniência:(De \textunderscore trepar\textunderscore ^1)}
\end{itemize}
Galho, que facilita o trepar-se á árvore.
Ramo da árvore.
Vara ou rebento da árvore, junto ao chão.--Em Baião, \textunderscore trépa\textunderscore .
\section{Trepadeira}
\begin{itemize}
\item {Grp. gram.:adj.}
\end{itemize}
\begin{itemize}
\item {Grp. gram.:F.}
\end{itemize}
\begin{itemize}
\item {Utilização:Prov.}
\end{itemize}
\begin{itemize}
\item {Utilização:beir.}
\end{itemize}
\begin{itemize}
\item {Proveniência:(De \textunderscore trepar\textunderscore ^1)}
\end{itemize}
Que trepa, (falando-se de plantas ou de árvores).
Planta que trepa.
Espécie de picapau.
O mesmo que \textunderscore subideira\textunderscore .
\section{Trepadeira-sirigaita}
\begin{itemize}
\item {Grp. gram.:f.}
\end{itemize}
Nome que, em Penafiel, se dá á \textunderscore subideira\textunderscore , ave.
\section{Trepadoiro}
\begin{itemize}
\item {Grp. gram.:m.}
\end{itemize}
Lugar, por onde se trepa.
\section{Trepador}
\begin{itemize}
\item {Grp. gram.:m.  e  adj.}
\end{itemize}
\begin{itemize}
\item {Grp. gram.:M. pl.}
\end{itemize}
O que trepa.
Ordem de pássaros que trepam.
\section{Trepadouro}
\begin{itemize}
\item {Grp. gram.:m.}
\end{itemize}
Lugar, por onde se trepa.
\section{Trepa-gato}
\begin{itemize}
\item {Grp. gram.:m.}
\end{itemize}
\begin{itemize}
\item {Utilização:T. da Bairrada}
\end{itemize}
Ave, o mesmo que \textunderscore subideira\textunderscore .
\section{Trepa-moleque}
\begin{itemize}
\item {Grp. gram.:m.}
\end{itemize}
\begin{itemize}
\item {Utilização:Bras}
\end{itemize}
Penteado alto, hoje desusado.
\section{Trepanação}
\begin{itemize}
\item {Grp. gram.:f.}
\end{itemize}
Acto ou effeito de trepanar.
\section{Trepanar}
\begin{itemize}
\item {Grp. gram.:v. t.}
\end{itemize}
Cortar ou compor com o trépano.
\section{Trépano}
\begin{itemize}
\item {Grp. gram.:m.}
\end{itemize}
\begin{itemize}
\item {Proveniência:(Do gr. \textunderscore trupanon\textunderscore )}
\end{itemize}
Instrumento cirúrgico, com que se perfuram os ossos, especialmente os do crânio.
Trepanação.
\section{Trepante}
\begin{itemize}
\item {Grp. gram.:adj.}
\end{itemize}
\begin{itemize}
\item {Utilização:Heráld.}
\end{itemize}
Que trepa.
Diz-se do animal, que no escudo se representa, trepando.
\section{Trepar}
\begin{itemize}
\item {Grp. gram.:v. t.}
\end{itemize}
\begin{itemize}
\item {Grp. gram.:V. i.}
\end{itemize}
\begin{itemize}
\item {Proveniência:(Do al. \textunderscore treppe\textunderscore )}
\end{itemize}
Subir a (servindo-se das mãos e dos pés).
Elevar-se.
Alçar-se, segurando-se com as mãos e os pés.
\section{Tresfolegar}
\begin{itemize}
\item {Grp. gram.:v. i.}
\end{itemize}
\begin{itemize}
\item {Proveniência:(De \textunderscore tres...\textunderscore ^2 + \textunderscore fôlego\textunderscore )}
\end{itemize}
Respirar difficilmente; offegar.
\section{Tresfolgar}
\begin{itemize}
\item {Grp. gram.:v. i.}
\end{itemize}
O mesmo que \textunderscore tresfolegar\textunderscore . Cf. Herculano, \textunderscore Eurico\textunderscore , 116.
\section{Três-fôlhas}
\begin{itemize}
\item {Grp. gram.:f. pl.}
\end{itemize}
O mesmo que \textunderscore três-fôlhas-brancas\textunderscore .
\section{Três-fôlhas-brancas}
\begin{itemize}
\item {Grp. gram.:f. pl.}
\end{itemize}
Planta rutácea, (\textunderscore ticorea febrifuga\textunderscore ).
\section{Três-fôlhas-do-mato}
\begin{itemize}
\item {Grp. gram.:f. pl.}
\end{itemize}
O mesmo que \textunderscore três-fôlhas-brancas\textunderscore .
\section{Três-fôlhas-vermelhas}
\begin{itemize}
\item {Grp. gram.:f. pl.}
\end{itemize}
Planta rutácea, (\textunderscore vodia febrifuga\textunderscore ).
\section{Tresfoliar}
\begin{itemize}
\item {Grp. gram.:v. i.}
\end{itemize}
\begin{itemize}
\item {Proveniência:(De \textunderscore tres...\textunderscore ^2 + \textunderscore foliar\textunderscore )}
\end{itemize}
Foliar muito; divertir-se á larga. Cf. \textunderscore Alvará\textunderscore  de D. Sebast., in \textunderscore Rev. Lus.\textunderscore , XV, 137.
\section{Tresgastar}
\begin{itemize}
\item {Grp. gram.:v. t.}
\end{itemize}
\begin{itemize}
\item {Proveniência:(De \textunderscore tres...\textunderscore ^2 + \textunderscore gastar\textunderscore )}
\end{itemize}
Gastar demasiadamente; prodigalizar.
\section{Três-irmãos}
\begin{itemize}
\item {Grp. gram.:m. pl.}
\end{itemize}
Planta sapindácea do Brasil.
\section{Tresjurar}
\begin{itemize}
\item {Grp. gram.:v. t.  e  i.}
\end{itemize}
O mesmo que \textunderscore trejurar\textunderscore .
\section{Tresler}
\begin{itemize}
\item {Grp. gram.:v. i.}
\end{itemize}
\begin{itemize}
\item {Utilização:Fam.}
\end{itemize}
\begin{itemize}
\item {Proveniência:(De \textunderscore tres...\textunderscore ^1 + \textunderscore lêr\textunderscore )}
\end{itemize}
Lêr ás avessas.
Perder o juizo por lêr muito.
Dizer ou fazer tolices.
\section{Treslida}
\begin{itemize}
\item {Grp. gram.:adj. f.}
\end{itemize}
\begin{itemize}
\item {Utilização:Prov.}
\end{itemize}
\begin{itemize}
\item {Proveniência:(De \textunderscore tresler\textunderscore )}
\end{itemize}
Diz-se da mulhér muito faladora e sentenciosa.
\section{Tresloucadamente}
\begin{itemize}
\item {Grp. gram.:adv.}
\end{itemize}
De modo tresloucado; loucamente; desvairadamente.
\section{Tresloucar}
\begin{itemize}
\item {Grp. gram.:v. t.}
\end{itemize}
\begin{itemize}
\item {Grp. gram.:V. i.}
\end{itemize}
\begin{itemize}
\item {Proveniência:(De \textunderscore tres...\textunderscore ^2 + \textunderscore louco\textunderscore )}
\end{itemize}
Tornar louco.
Endoidecer.
Perder o juizo.
Tornar-se imprudente.
\section{Tresmalhar}
\begin{itemize}
\item {Grp. gram.:v. t.}
\end{itemize}
\begin{itemize}
\item {Grp. gram.:V. i.}
\end{itemize}
\begin{itemize}
\item {Proveniência:(De \textunderscore tres...\textunderscore ^1 + \textunderscore malha\textunderscore )}
\end{itemize}
Trocar as malhas de.
Deixar caír as malhas de.
Dispersar.
Deixar fugir.
Fazer fugir: \textunderscore tresmalhar o gado\textunderscore .
Dispersar-se.
Afastar-se do bando.
Extraviar-se.
Fugir, dispersando-se.
\section{Tresmalheiro}
\begin{itemize}
\item {Grp. gram.:m.}
\end{itemize}
Pescador, que se serve de tresmalho^1. Cf. \textunderscore Lei\textunderscore  de D. Sebast., sôbre caça e pesca.
\section{Tresmalho}
\begin{itemize}
\item {Grp. gram.:m.}
\end{itemize}
\begin{itemize}
\item {Proveniência:(De \textunderscore três\textunderscore  + \textunderscore malha\textunderscore )}
\end{itemize}
Rêde de três panos, sendo o do meio mais largo e de malha mais cerrada que a dos exteriores, chamados alvitanas, por onde entra o peixe, emmalhando-se nas dobras ou bolsos do pano interior.
\section{Tresmalho}
\begin{itemize}
\item {Grp. gram.:m.}
\end{itemize}
Acto ou effeito de tresmalhar.
\section{Três-marias}
\begin{itemize}
\item {Grp. gram.:f. pl.}
\end{itemize}
\begin{itemize}
\item {Utilização:Pop.}
\end{itemize}
As estrêlas, que formam o cinto de Órion.
\section{Tresmudar}
\begin{itemize}
\item {Grp. gram.:v. t.}
\end{itemize}
(V.transmudar)
\section{Tresneta}
\begin{itemize}
\item {Grp. gram.:f.}
\end{itemize}
(V.trineta)
\section{Tresneto}
\begin{itemize}
\item {Grp. gram.:m.}
\end{itemize}
(V.trineto)
\section{Tresnoitar}
\begin{itemize}
\item {Grp. gram.:v. t.}
\end{itemize}
\begin{itemize}
\item {Grp. gram.:V. i.}
\end{itemize}
\begin{itemize}
\item {Proveniência:(De \textunderscore tres...\textunderscore ^1 + \textunderscore noite\textunderscore )}
\end{itemize}
Tirar o somno a; não deixar dormir.
Passar a noite sem dormir.
\section{Três-novidades}
\begin{itemize}
\item {Grp. gram.:f. pl.}
\end{itemize}
Casta de uva brasileira.
\section{Treso}
\begin{itemize}
\item {fónica:trê}
\end{itemize}
\begin{itemize}
\item {Grp. gram.:adj.}
\end{itemize}
\begin{itemize}
\item {Utilização:P. us.}
\end{itemize}
Que tem má índole; manhoso.
O \textunderscore Elucidário\textunderscore  de S. R. Viterbo diz \textunderscore tréso\textunderscore .
\section{Trespano}
\begin{itemize}
\item {Grp. gram.:m.}
\end{itemize}
\begin{itemize}
\item {Proveniência:(De \textunderscore três\textunderscore  + \textunderscore pano\textunderscore )}
\end{itemize}
Tecido de três liços.
\section{Trespassação}
\begin{itemize}
\item {Grp. gram.:f.}
\end{itemize}
\begin{itemize}
\item {Utilização:Ant.}
\end{itemize}
\begin{itemize}
\item {Proveniência:(De \textunderscore trespassar\textunderscore )}
\end{itemize}
Trespasse de um direito ou domínio, de uma pessôa ou collectividade para outra.
\section{Trespassamento}
\begin{itemize}
\item {Grp. gram.:m.}
\end{itemize}
O mesmo que \textunderscore trespasse\textunderscore . Cf, Usque, 50, v.^o.
\section{Trespassar}
\begin{itemize}
\item {Grp. gram.:v. t.}
\end{itemize}
O mesmo que \textunderscore traspassar\textunderscore , etc.
\section{Trespasse}
\begin{itemize}
\item {Grp. gram.:m.}
\end{itemize}
Acto ou effeito de trespassar.
O mesmo que \textunderscore traspasso\textunderscore , (mas preferido por alguns, no sentido de morte de alguém, talvez pela semelhança do fr. \textunderscore trépas\textunderscore ).
\section{Trespasso}
\begin{itemize}
\item {Grp. gram.:m.}
\end{itemize}
O mesmo que \textunderscore trespasse\textunderscore .
Jejum, observado por cathólicos, desde quinta-feira Santa ao Domingo de Páscoa, em memória do trespasse de Christo. Cf. B. Branco, \textunderscore Hist. das Ord. Monást.\textunderscore , III, 696.
\section{Trevoejar}
\begin{itemize}
\item {Grp. gram.:v. i.}
\end{itemize}
\begin{itemize}
\item {Utilização:ant.}
\end{itemize}
\begin{itemize}
\item {Utilização:Pop.}
\end{itemize}
O mesmo que \textunderscore trovejar\textunderscore . Cf. \textunderscore Rot. do Mar Verm.\textunderscore , 127, 204 e 207.
\section{Trévoas}
\begin{itemize}
\item {Grp. gram.:f. pl.}
\end{itemize}
\begin{itemize}
\item {Utilização:Des.}
\end{itemize}
O mesmo que \textunderscore trevas\textunderscore . Cf. B. Pereira, \textunderscore Prosódia\textunderscore , vb. \textunderscore lyge\textunderscore .
\section{Trevo-cervino}
\begin{itemize}
\item {Grp. gram.:m.}
\end{itemize}
Planta medicinal, (\textunderscore herba kunigundis\textunderscore , Avicena). Cf. \textunderscore Desengano da Medicina\textunderscore , 46.
\section{Trevo-de-cheiro}
\begin{itemize}
\item {Grp. gram.:m.}
\end{itemize}
O mesmo que \textunderscore trevo-real\textunderscore .
\section{Trevo-de-seara}
\begin{itemize}
\item {Grp. gram.:m.}
\end{itemize}
Planta leguminosa, (\textunderscore melilotus parviflora\textunderscore , Desf.).
\section{Trevo-do-egypto}
\begin{itemize}
\item {Grp. gram.:m.}
\end{itemize}
Espécie de trevo, (\textunderscore trifolium alexandrium\textunderscore ).
\section{Trevo-namorado}
\begin{itemize}
\item {Grp. gram.:m.}
\end{itemize}
Planta leguminosa, (\textunderscore trifolium procumbens\textunderscore , Lin.).
\section{Trevo-pé-de-pássaro}
\begin{itemize}
\item {Grp. gram.:m.}
\end{itemize}
Planta leguminosa, (\textunderscore ornithopos compressus\textunderscore , Lin.).
\section{Trevo-real}
\begin{itemize}
\item {Grp. gram.:m.}
\end{itemize}
O mesmo que \textunderscore meliloto\textunderscore .
\section{Trevoso}
\begin{itemize}
\item {Grp. gram.:adj.}
\end{itemize}
\begin{itemize}
\item {Proveniência:(De \textunderscore trevas\textunderscore )}
\end{itemize}
O mesmo que \textunderscore tenebroso\textunderscore .
\section{Treze}
\begin{itemize}
\item {fónica:trê}
\end{itemize}
\begin{itemize}
\item {Grp. gram.:adj.}
\end{itemize}
\begin{itemize}
\item {Grp. gram.:M.  e  adj.}
\end{itemize}
\begin{itemize}
\item {Proveniência:(Do lat. \textunderscore tredecim\textunderscore )}
\end{itemize}
Diz-se do número cardinal, formado de déz e mais três.
Décimo terceiro.
O que numa série de treze occupa o último lugar.
\section{Trezena}
\begin{itemize}
\item {Grp. gram.:f.}
\end{itemize}
\begin{itemize}
\item {Proveniência:(De \textunderscore trezeno\textunderscore )}
\end{itemize}
Conjunto de treze.
Espaço de treze dias.
Reza, própria dos treze dias, que antecedem a festa de um santo.
\section{Trezênio}
\begin{itemize}
\item {Grp. gram.:m.}
\end{itemize}
\begin{itemize}
\item {Proveniência:(De \textunderscore treze\textunderscore , sob infl. de \textunderscore triênnio\textunderscore , \textunderscore biênnio\textunderscore , etc.)}
\end{itemize}
Espaço de treze annos. Cf. Castilho, \textunderscore Fastos\textunderscore , I, 541.
\section{Trezeno}
\begin{itemize}
\item {Grp. gram.:adj.}
\end{itemize}
\begin{itemize}
\item {Proveniência:(De \textunderscore treze\textunderscore )}
\end{itemize}
Décimo terceiro.
\section{Trezentista}
\begin{itemize}
\item {Grp. gram.:m.}
\end{itemize}
O mesmo que \textunderscore trecentista\textunderscore . Cf. Filinto, XVIII, 77.
\section{Trezentos}
\begin{itemize}
\item {Grp. gram.:adj. pl.}
\end{itemize}
\begin{itemize}
\item {Proveniência:(Lat. \textunderscore trecenti\textunderscore )}
\end{itemize}
Três vezes cem.
\section{Tri...}
\begin{itemize}
\item {Grp. gram.:pref.}
\end{itemize}
O mesmo que \textunderscore tris...\textunderscore 
\section{Triacantho}
\begin{itemize}
\item {Grp. gram.:adj.}
\end{itemize}
\begin{itemize}
\item {Proveniência:(Do gr. \textunderscore tri\textunderscore  + \textunderscore acantha\textunderscore )}
\end{itemize}
Que tem três espinhas.
\section{Triacanto}
\begin{itemize}
\item {Grp. gram.:adj.}
\end{itemize}
\begin{itemize}
\item {Proveniência:(Do gr. \textunderscore tri\textunderscore  + \textunderscore acantha\textunderscore )}
\end{itemize}
Que tem três espinhas.
\section{Triácido}
\begin{itemize}
\item {Grp. gram.:adj.}
\end{itemize}
\begin{itemize}
\item {Utilização:Chím.}
\end{itemize}
\begin{itemize}
\item {Proveniência:(De \textunderscore tri...\textunderscore  + \textunderscore ácido\textunderscore )}
\end{itemize}
Diz-se de uma base que, em combinação, não póde sêr neutralizada senão por três equivalentes de um ácido.
\section{Triacontaédro}
\begin{itemize}
\item {Grp. gram.:adj.}
\end{itemize}
\begin{itemize}
\item {Utilização:Miner.}
\end{itemize}
\begin{itemize}
\item {Proveniência:(Do gr. \textunderscore triakonta\textunderscore  + \textunderscore edra\textunderscore )}
\end{itemize}
Diz-se dos crystaes, que têm trinta faces.
\section{Tríada}
\begin{itemize}
\item {Grp. gram.:f.}
\end{itemize}
\begin{itemize}
\item {Utilização:Mús.}
\end{itemize}
\begin{itemize}
\item {Proveniência:(Lat. \textunderscore trias\textunderscore , \textunderscore triadis\textunderscore )}
\end{itemize}
Conjunto de três pessôas ou três coisas; trindade.
Acorde de três sons.
\section{Tríade}
\begin{itemize}
\item {Grp. gram.:f.}
\end{itemize}
\begin{itemize}
\item {Utilização:Mús.}
\end{itemize}
\begin{itemize}
\item {Proveniência:(Lat. \textunderscore trias\textunderscore , \textunderscore triadis\textunderscore )}
\end{itemize}
Conjunto de três pessôas ou três coisas; trindade.
Acorde de três sons.
\section{Triadelfo}
\begin{itemize}
\item {Grp. gram.:adj.}
\end{itemize}
\begin{itemize}
\item {Utilização:Bot.}
\end{itemize}
\begin{itemize}
\item {Proveniência:(Do gr. \textunderscore tri\textunderscore  + \textunderscore adelphos\textunderscore )}
\end{itemize}
Diz-se dos estames, cujos filetes estão soldados em três andróphoros.
\section{Triadelpho}
\begin{itemize}
\item {Grp. gram.:adj.}
\end{itemize}
\begin{itemize}
\item {Utilização:Bot.}
\end{itemize}
\begin{itemize}
\item {Proveniência:(Do gr. \textunderscore tri\textunderscore  + \textunderscore adelphos\textunderscore )}
\end{itemize}
Diz-se dos estames, cujos filetes estão soldados em três andróphoros.
\section{Triádico}
\begin{itemize}
\item {Grp. gram.:adj.}
\end{itemize}
\begin{itemize}
\item {Utilização:Geol.}
\end{itemize}
\begin{itemize}
\item {Proveniência:(Do gr. \textunderscore trias\textunderscore , \textunderscore triados\textunderscore )}
\end{itemize}
Diz-se de um dos terrenos da série mesozóica.
\section{Triaga}
\begin{itemize}
\item {Grp. gram.:f.}
\end{itemize}
\begin{itemize}
\item {Utilização:Pop.}
\end{itemize}
O mesmo que \textunderscore theriaga\textunderscore .
\section{Triagueiro}
\begin{itemize}
\item {Grp. gram.:m.}
\end{itemize}
Aquelle que prepara triagas.
\section{Trialado}
\begin{itemize}
\item {Grp. gram.:adj.}
\end{itemize}
\begin{itemize}
\item {Utilização:Bot.}
\end{itemize}
\begin{itemize}
\item {Proveniência:(De \textunderscore tri...\textunderscore  + \textunderscore alado\textunderscore )}
\end{itemize}
Que tem três asas.
\section{Trialumínico}
\begin{itemize}
\item {Grp. gram.:adj.}
\end{itemize}
\begin{itemize}
\item {Utilização:Chím.}
\end{itemize}
\begin{itemize}
\item {Proveniência:(De \textunderscore tri...\textunderscore  + \textunderscore alumina\textunderscore )}
\end{itemize}
Diz-se de um sal, que contém três vezes tanta alumina como o sal neutro correspondente.
\section{Triaminas}
\begin{itemize}
\item {Grp. gram.:f. pl.}
\end{itemize}
\begin{itemize}
\item {Utilização:Chím.}
\end{itemize}
\begin{itemize}
\item {Proveniência:(De \textunderscore tri...\textunderscore  + \textunderscore aminas\textunderscore )}
\end{itemize}
Aminas, derivadas de três moléculas de amoníaco, condensadas.
\section{Triammónico}
\begin{itemize}
\item {Grp. gram.:adj.}
\end{itemize}
\begin{itemize}
\item {Utilização:Chím.}
\end{itemize}
Diz-se de um sal, que contém três vezes tanto ammoníaco, como o sal neutro correspondente.
\section{Triamónico}
\begin{itemize}
\item {Grp. gram.:adj.}
\end{itemize}
\begin{itemize}
\item {Utilização:Chím.}
\end{itemize}
Diz-se de um sal, que contém três vezes tanto ammoníaco, como o sal neutro correspondente.
\section{Triandria}
\begin{itemize}
\item {Grp. gram.:f.}
\end{itemize}
Qualidade de triandro.
Conjunto dos vegetaes triandros.
\section{Triândrico}
\begin{itemize}
\item {Grp. gram.:adj.}
\end{itemize}
O mesmo que \textunderscore triândrio\textunderscore .
\section{Triândrio}
\begin{itemize}
\item {Grp. gram.:adj.}
\end{itemize}
O mesmo que \textunderscore triandro\textunderscore .
\section{Triandro}
\begin{itemize}
\item {Grp. gram.:adj.}
\end{itemize}
\begin{itemize}
\item {Utilização:Bot.}
\end{itemize}
\begin{itemize}
\item {Proveniência:(Do gr. \textunderscore treis\textunderscore  + \textunderscore aner\textunderscore , \textunderscore andros\textunderscore )}
\end{itemize}
Que tem três estames, livres entre si.
\section{Tribómetro}
\begin{itemize}
\item {Grp. gram.:m.}
\end{itemize}
\begin{itemize}
\item {Proveniência:(Do gr. \textunderscore tribein\textunderscore  + \textunderscore metron\textunderscore )}
\end{itemize}
Instrumento, para medir a fôrça do attrito.
\section{Tríbracho}
\begin{itemize}
\item {fónica:co}
\end{itemize}
\begin{itemize}
\item {Grp. gram.:m.}
\end{itemize}
\begin{itemize}
\item {Proveniência:(Lat. \textunderscore tribrachus\textunderscore )}
\end{itemize}
Pé de verso, grego ou latino, composto de três sýllabas breves.
\section{Tríbraco}
\begin{itemize}
\item {Grp. gram.:m.}
\end{itemize}
\begin{itemize}
\item {Proveniência:(Lat. \textunderscore tribrachus\textunderscore )}
\end{itemize}
Pé de verso, grego ou latino, composto de três sílabas breves.
\section{Tribracteado}
\begin{itemize}
\item {Grp. gram.:adj.}
\end{itemize}
\begin{itemize}
\item {Proveniência:(De \textunderscore tri...\textunderscore  + \textunderscore bráctea\textunderscore )}
\end{itemize}
Que tem três brácteas.
\section{Tribracteolado}
\begin{itemize}
\item {Grp. gram.:adj.}
\end{itemize}
\begin{itemize}
\item {Proveniência:(De \textunderscore tri...\textunderscore  + \textunderscore bractéola\textunderscore )}
\end{itemize}
Que tem três bractéolas.
\section{Tribreve}
\begin{itemize}
\item {Grp. gram.:m.}
\end{itemize}
\begin{itemize}
\item {Proveniência:(Lat. \textunderscore tribrevis\textunderscore )}
\end{itemize}
O mesmo que \textunderscore tríbracho\textunderscore .
\section{Tríbu}
\begin{itemize}
\item {Grp. gram.:f.}
\end{itemize}
(V.tribo)
\section{Tribuir}
\begin{itemize}
\item {Grp. gram.:v. i.}
\end{itemize}
\begin{itemize}
\item {Utilização:Ant.}
\end{itemize}
O mesmo que \textunderscore contribuir\textunderscore :«\textunderscore que todos tribuissem para se pagar.\textunderscore »\textunderscore Alvará\textunderscore  de Aff. V, in \textunderscore Rev. Lus.\textunderscore , XV, 121.
\section{Tribul}
\begin{itemize}
\item {Grp. gram.:adj.}
\end{itemize}
\begin{itemize}
\item {Proveniência:(Lat. \textunderscore tribulis\textunderscore )}
\end{itemize}
Pertencente á mesma tríbo.
\section{Tribulação}
\begin{itemize}
\item {Grp. gram.:f.}
\end{itemize}
\begin{itemize}
\item {Proveniência:(Do lat. \textunderscore tribulatio\textunderscore )}
\end{itemize}
Adversidade; amargura.
Trabalho.
\section{Tríbulo}
\begin{itemize}
\item {Grp. gram.:m.}
\end{itemize}
\begin{itemize}
\item {Proveniência:(Gr. \textunderscore tribolos\textunderscore )}
\end{itemize}
Planta zygophylácea.
Planta aquática, (\textunderscore trapa natans\textunderscore ).
\section{Tribuna}
\begin{itemize}
\item {Grp. gram.:f.}
\end{itemize}
\begin{itemize}
\item {Utilização:Fig.}
\end{itemize}
Lugar elevado, donde falavam ao povo os oradores gregos e romanos.
Lugar elevado, donde falam os oradores.
Varanda ou palanque, donde se assiste a certas ceremónias ou assembleias.
A arte de falar em público, no parlamento ou no púlpito.
Eloquência.
(B. lat. \textunderscore tribuna\textunderscore )
\section{Tribunado}
\begin{itemize}
\item {Grp. gram.:m.}
\end{itemize}
\begin{itemize}
\item {Proveniência:(Do lat. \textunderscore tribunatus\textunderscore )}
\end{itemize}
Cargo de tribuno.
Tempo, durante o qual o tribuno exercia o seu cargo.
\section{Tribunal}
\begin{itemize}
\item {Grp. gram.:m.}
\end{itemize}
\begin{itemize}
\item {Proveniência:(Lat. \textunderscore tribunal\textunderscore )}
\end{itemize}
Cadeira de juiz ou magistrado.
Casa, onde se debatem e se julgam as questões judiciaes.
Qualquer entidade moral, que póde formar juízo ou considerar-se como juíz.
Tudo que julga.
Lugar onde se é julgado.
\section{Tribunato}
\begin{itemize}
\item {Grp. gram.:m.}
\end{itemize}
O mesmo que \textunderscore tribunado\textunderscore .
\section{Tribuneca}
\begin{itemize}
\item {Grp. gram.:f.}
\end{itemize}
\begin{itemize}
\item {Utilização:Deprec.}
\end{itemize}
\begin{itemize}
\item {Utilização:Pop.}
\end{itemize}
\begin{itemize}
\item {Proveniência:(De \textunderscore tribuna\textunderscore )}
\end{itemize}
Tribunal.
Emprêgo rendoso e de pouco trabalho; sinecura.
\section{Tribunício}
\begin{itemize}
\item {Grp. gram.:adj.}
\end{itemize}
\begin{itemize}
\item {Proveniência:(Lat. \textunderscore tribunicius\textunderscore )}
\end{itemize}
Relativo a tribuno.
\section{Tribuno}
\begin{itemize}
\item {Grp. gram.:m.}
\end{itemize}
\begin{itemize}
\item {Utilização:Ant.}
\end{itemize}
\begin{itemize}
\item {Proveniência:(Lat. \textunderscore tribunus\textunderscore )}
\end{itemize}
Magistrado que, em Roma, estava encarregado de defender os direitos e interesses do povo.
Orador de assembleias políticas.
Orador, que pugna pelas regalias populares.
Orador revolucionário.
O mesmo que \textunderscore almoxarife\textunderscore .
\section{Tribunocracia}
\begin{itemize}
\item {Grp. gram.:f.}
\end{itemize}
\begin{itemize}
\item {Proveniência:(De \textunderscore tribunal\textunderscore  + gr. \textunderscore kratos\textunderscore )}
\end{itemize}
Preponderância do poder judicial.
Influência dos agentes subalternos dos tribunaes.
\section{Tributação}
\begin{itemize}
\item {Grp. gram.:f.}
\end{itemize}
Acto ou effeito de tributar.
\section{Tributairo}
\begin{itemize}
\item {Grp. gram.:adj.}
\end{itemize}
\begin{itemize}
\item {Utilização:Ant.}
\end{itemize}
O mesmo que \textunderscore tributário\textunderscore . Cf. Usque, 43 v.^o.
\section{Tributal}
\begin{itemize}
\item {Grp. gram.:adj.}
\end{itemize}
Relativo a tributo.
\section{Tributando}
\begin{itemize}
\item {Grp. gram.:adj.}
\end{itemize}
\begin{itemize}
\item {Proveniência:(De \textunderscore tributar\textunderscore )}
\end{itemize}
Que deve sêr tributado; sujeitado a impostos.
\section{Tributar}
\begin{itemize}
\item {Grp. gram.:v. t.}
\end{itemize}
\begin{itemize}
\item {Grp. gram.:V. p.}
\end{itemize}
\begin{itemize}
\item {Proveniência:(De \textunderscore tributo\textunderscore )}
\end{itemize}
Impôr tributo a.
Prestar ou dedicar a alguém ou a alguma coisa, como tributo: \textunderscore tributar homenagens\textunderscore .
Tornar-se tributário.
Contribuir; quotizar-se.
\section{Tributário}
\begin{itemize}
\item {Grp. gram.:m.  e  adj.}
\end{itemize}
\begin{itemize}
\item {Proveniência:(Lat. \textunderscore tributarius\textunderscore )}
\end{itemize}
O que paga tributo.
O que é sujeito a pagamento de tributo; contribuínte.
\section{Tributável}
\begin{itemize}
\item {Grp. gram.:adj.}
\end{itemize}
\begin{itemize}
\item {Proveniência:(De \textunderscore tributar\textunderscore )}
\end{itemize}
Que póde ou deve sêr tributado. Cf. Herculano, \textunderscore Hist. de Port.\textunderscore , IV, 430.
\section{Tributear}
\begin{itemize}
\item {Grp. gram.:v. i.}
\end{itemize}
\begin{itemize}
\item {Utilização:Des.}
\end{itemize}
Pagar tributo. Cf. \textunderscore Viriato Trág.\textunderscore , X, 48; XIV, 100.
\section{Tributeiro}
\begin{itemize}
\item {Grp. gram.:m.}
\end{itemize}
\begin{itemize}
\item {Utilização:Ant.}
\end{itemize}
\begin{itemize}
\item {Proveniência:(Do lat. \textunderscore tributarius\textunderscore )}
\end{itemize}
Cobrador de tributos.
\section{Trichismo}
\begin{itemize}
\item {fónica:quis}
\end{itemize}
\begin{itemize}
\item {Grp. gram.:m.}
\end{itemize}
\begin{itemize}
\item {Proveniência:(Gr. \textunderscore trikhismos\textunderscore )}
\end{itemize}
Fractura filiforme de um osso.
\section{Trichocéphalo}
\begin{itemize}
\item {fónica:có}
\end{itemize}
\begin{itemize}
\item {Grp. gram.:m.}
\end{itemize}
\begin{itemize}
\item {Proveniência:(Do gr. \textunderscore thrix\textunderscore , \textunderscore trikhos\textunderscore  + \textunderscore kephale\textunderscore )}
\end{itemize}
Gênero de vermes parasitos, que vivem no corpo do homem e de diversos mammíferos.
\section{Trichocystos}
\begin{itemize}
\item {fónica:có}
\end{itemize}
\begin{itemize}
\item {Grp. gram.:m. pl.}
\end{itemize}
\begin{itemize}
\item {Proveniência:(Do gr. \textunderscore trikkos\textunderscore  + \textunderscore kustis\textunderscore )}
\end{itemize}
Órgãos urticantes nos infusórios, e análogos aos nematocystos.
\section{Trichodáctylo}
\begin{itemize}
\item {fónica:co}
\end{itemize}
\begin{itemize}
\item {Grp. gram.:m.}
\end{itemize}
\begin{itemize}
\item {Proveniência:(Do gr. \textunderscore thrix\textunderscore , \textunderscore trikhos\textunderscore  + \textunderscore dáktulos\textunderscore )}
\end{itemize}
Gênero de arachnídeos, da ordem dos ácaros.
\section{Trichodesma}
\begin{itemize}
\item {fónica:co}
\end{itemize}
\begin{itemize}
\item {Grp. gram.:f.}
\end{itemize}
Gênero de plantas borragíneas.
\section{Trichodesmo}
\begin{itemize}
\item {fónica:co}
\end{itemize}
\begin{itemize}
\item {Grp. gram.:m.}
\end{itemize}
Gênero de plantas phýceas.
\section{Trichóglea}
\begin{itemize}
\item {fónica:có}
\end{itemize}
\begin{itemize}
\item {Grp. gram.:f.}
\end{itemize}
Gênero de plantas phýceas.
\section{Trichoglossia}
\begin{itemize}
\item {fónica:có}
\end{itemize}
\begin{itemize}
\item {Grp. gram.:f.}
\end{itemize}
\begin{itemize}
\item {Utilização:Med.}
\end{itemize}
\begin{itemize}
\item {Proveniência:(Do gr. \textunderscore thrix\textunderscore , \textunderscore trikhos\textunderscore  + \textunderscore glossa\textunderscore )}
\end{itemize}
Estado da língua, quando apparece coberta de pêlos.
\section{Trichogónia}
\begin{itemize}
\item {fónica:co}
\end{itemize}
\begin{itemize}
\item {Grp. gram.:f.}
\end{itemize}
\begin{itemize}
\item {Proveniência:(Do gr. \textunderscore thrix\textunderscore , \textunderscore trikhos\textunderscore  + \textunderscore gonia\textunderscore )}
\end{itemize}
Gênero de plantas, da fam. das compostas.
\section{Trichógyno}
\begin{itemize}
\item {fónica:có}
\end{itemize}
\begin{itemize}
\item {Grp. gram.:m.}
\end{itemize}
\begin{itemize}
\item {Proveniência:(Do gr. \textunderscore thrix\textunderscore , \textunderscore trikhos\textunderscore  + \textunderscore gune\textunderscore )}
\end{itemize}
Gênero de plantas, da fam. das compostas.
\section{Trichóide}
\begin{itemize}
\item {fónica:coi}
\end{itemize}
\begin{itemize}
\item {Grp. gram.:adj.}
\end{itemize}
\begin{itemize}
\item {Proveniência:(Do gr. \textunderscore thrix\textunderscore , \textunderscore trikhos\textunderscore  + \textunderscore eidos\textunderscore )}
\end{itemize}
Semelhante a um cabello.
\section{Trichologia}
\begin{itemize}
\item {fónica:co}
\end{itemize}
\begin{itemize}
\item {Grp. gram.:f.}
\end{itemize}
\begin{itemize}
\item {Proveniência:(Do gr. \textunderscore thrix\textunderscore , \textunderscore trikhos\textunderscore  + \textunderscore logos\textunderscore )}
\end{itemize}
Tratado á cêrca dos pêlos ou dos cabellos.
\section{Tricholoma}
\begin{itemize}
\item {fónica:co}
\end{itemize}
\begin{itemize}
\item {Grp. gram.:f.}
\end{itemize}
Gênero de plantas escrofularíneas.
\section{Trichoma}
\begin{itemize}
\item {fónica:cô}
\end{itemize}
\begin{itemize}
\item {Grp. gram.:m.}
\end{itemize}
\begin{itemize}
\item {Proveniência:(Gr. \textunderscore trichhoma\textunderscore )}
\end{itemize}
Doença, que ataca os cabellos, enredando-se êstes por fórma, que se não desembaraçam nem se cortam sem derramamento de sangue.
Denomina-se também \textunderscore plica-polónica\textunderscore .
\section{Trichomária}
\begin{itemize}
\item {fónica:co}
\end{itemize}
\begin{itemize}
\item {Grp. gram.:f.}
\end{itemize}
\begin{itemize}
\item {Proveniência:(Do gr. \textunderscore trikhoma\textunderscore )}
\end{itemize}
Gênero de plantas malpigiáceas.
\section{Trichomático}
\begin{itemize}
\item {fónica:co}
\end{itemize}
\begin{itemize}
\item {Grp. gram.:adj.}
\end{itemize}
Relativo ao trichoma.
Que padece trichoma.
\section{Trichomatoso}
\begin{itemize}
\item {fónica:co}
\end{itemize}
\begin{itemize}
\item {Grp. gram.:adj.}
\end{itemize}
Relativo ao trichoma.
Que padece trichoma.
\section{Trichomisco}
\begin{itemize}
\item {fónica:co}
\end{itemize}
\begin{itemize}
\item {Grp. gram.:m.}
\end{itemize}
Gênero de crustáceos.
\section{Trichonoto}
\begin{itemize}
\item {fónica:co}
\end{itemize}
\begin{itemize}
\item {Grp. gram.:m.}
\end{itemize}
Gênero de insectos coleópteros.
\section{Trichophytia}
\begin{itemize}
\item {fónica:co}
\end{itemize}
\begin{itemize}
\item {Grp. gram.:f.}
\end{itemize}
\begin{itemize}
\item {Utilização:Med.}
\end{itemize}
\begin{itemize}
\item {Proveniência:(Do gr. \textunderscore trix\textunderscore  + \textunderscore phuton\textunderscore )}
\end{itemize}
Moléstia cutânea, espécie de tinha, determinada pela presença de um vegetal microscópico. Cf. G. H. Fox, \textunderscore Iconographie Photogr. des Maladies de la Peau\textunderscore .
\section{Trichóphyto}
\begin{itemize}
\item {fónica:có}
\end{itemize}
\begin{itemize}
\item {Grp. gram.:m.}
\end{itemize}
\begin{itemize}
\item {Proveniência:(Do gr. \textunderscore trix\textunderscore  + \textunderscore phuton\textunderscore )}
\end{itemize}
Vegetal microscópico, que determina a trichophytía.
\section{Trichopíllia}
\begin{itemize}
\item {fónica:co}
\end{itemize}
\begin{itemize}
\item {Grp. gram.:f.}
\end{itemize}
Gênero de orchídeas.
\section{Trichopódio}
\begin{itemize}
\item {fónica:co}
\end{itemize}
\begin{itemize}
\item {Grp. gram.:m.}
\end{itemize}
\begin{itemize}
\item {Proveniência:(Do gr. \textunderscore trix\textunderscore , \textunderscore trikhos\textunderscore  + \textunderscore pous\textunderscore , \textunderscore podos\textunderscore )}
\end{itemize}
Gênero do plantas aristolóchias.
\section{Trichópode}
\begin{itemize}
\item {fónica:có}
\end{itemize}
\begin{itemize}
\item {Grp. gram.:m.}
\end{itemize}
\begin{itemize}
\item {Proveniência:(Do gr. \textunderscore trix\textunderscore , \textunderscore trikhos\textunderscore  + \textunderscore pous\textunderscore , \textunderscore podos\textunderscore )}
\end{itemize}
Gênero de peixes.
\section{Trichosandra}
\begin{itemize}
\item {fónica:co}
\end{itemize}
\begin{itemize}
\item {Grp. gram.:f.}
\end{itemize}
Gênero de plantas asclepiadáceas.
\section{Trichosantho}
\begin{itemize}
\item {fónica:co}
\end{itemize}
\begin{itemize}
\item {Grp. gram.:m.}
\end{itemize}
Gênero de plantas cucurbitáceas.
\section{Trichose}
\begin{itemize}
\item {fónica:có}
\end{itemize}
\begin{itemize}
\item {Grp. gram.:f.}
\end{itemize}
\begin{itemize}
\item {Utilização:Med.}
\end{itemize}
\begin{itemize}
\item {Proveniência:(Do gr. \textunderscore trix\textunderscore , \textunderscore trikhos\textunderscore )}
\end{itemize}
Desenvolvimento anormal de pêlos na mucosa da bexiga ou da uretra.
\section{Tricicla}
\begin{itemize}
\item {Grp. gram.:f.}
\end{itemize}
\begin{itemize}
\item {Proveniência:(Do gr. \textunderscore treis\textunderscore  + \textunderscore kuklos\textunderscore )}
\end{itemize}
Gênero de plantas nictagíneas.
\section{Tricicleta}
\begin{itemize}
\item {Grp. gram.:f.}
\end{itemize}
Pequeno triciclo.
\section{Triciclo}
\begin{itemize}
\item {Grp. gram.:m.}
\end{itemize}
\begin{itemize}
\item {Proveniência:(Do gr. \textunderscore treis\textunderscore  + \textunderscore kuklos\textunderscore )}
\end{itemize}
Velocípede de três rodas.
Antiga carruagem de três rodas.
\section{Tricocéfalo}
\begin{itemize}
\item {Grp. gram.:m.}
\end{itemize}
\begin{itemize}
\item {Proveniência:(Do gr. \textunderscore thrix\textunderscore , \textunderscore trikhos\textunderscore  + \textunderscore kephale\textunderscore )}
\end{itemize}
Gênero de vermes parasitos, que vivem no corpo do homem e de diversos mamíferos.
\section{Tricocistos}
\begin{itemize}
\item {Grp. gram.:m. pl.}
\end{itemize}
\begin{itemize}
\item {Proveniência:(Do gr. \textunderscore trikkos\textunderscore  + \textunderscore kustis\textunderscore )}
\end{itemize}
Órgãos urticantes nos infusórios, e análogos aos nematocistos.
\section{Tricodáctilo}
\begin{itemize}
\item {Grp. gram.:m.}
\end{itemize}
\begin{itemize}
\item {Proveniência:(Do gr. \textunderscore thrix\textunderscore , \textunderscore trikhos\textunderscore  + \textunderscore dáktulos\textunderscore )}
\end{itemize}
Gênero de aracnídeos, da ordem dos ácaros.
\section{Tricodesma}
\begin{itemize}
\item {Grp. gram.:f.}
\end{itemize}
Gênero de plantas borragíneas.
\section{Tricodesmo}
\begin{itemize}
\item {Grp. gram.:m.}
\end{itemize}
Gênero de plantas fíceas.
\section{Tricofitia}
\begin{itemize}
\item {Grp. gram.:f.}
\end{itemize}
\begin{itemize}
\item {Utilização:Med.}
\end{itemize}
\begin{itemize}
\item {Proveniência:(Do gr. \textunderscore trix\textunderscore  + \textunderscore phuton\textunderscore )}
\end{itemize}
Moléstia cutânea, espécie de tinha, determinada pela presença de um vegetal microscópico. Cf. G. H. Fox, \textunderscore Iconographie Photogr. des Maladies de la Peau\textunderscore .
\section{Tricófito}
\begin{itemize}
\item {Grp. gram.:m.}
\end{itemize}
\begin{itemize}
\item {Proveniência:(Do gr. \textunderscore trix\textunderscore  + \textunderscore phuton\textunderscore )}
\end{itemize}
Vegetal microscópico, que determina a tricofitía.
\section{Tricógino}
\begin{itemize}
\item {Grp. gram.:m.}
\end{itemize}
\begin{itemize}
\item {Proveniência:(Do gr. \textunderscore thrix\textunderscore , \textunderscore trikhos\textunderscore  + \textunderscore gune\textunderscore )}
\end{itemize}
Gênero de plantas, da fam. das compostas.
\section{Tricóglea}
\begin{itemize}
\item {Grp. gram.:f.}
\end{itemize}
Gênero de plantas fíceas.
\section{Tricoglossia}
\begin{itemize}
\item {Grp. gram.:f.}
\end{itemize}
\begin{itemize}
\item {Utilização:Med.}
\end{itemize}
\begin{itemize}
\item {Proveniência:(Do gr. \textunderscore thrix\textunderscore , \textunderscore trikhos\textunderscore  + \textunderscore glossa\textunderscore )}
\end{itemize}
Estado da língua, quando apparece coberta de pêlos.
\section{Tricogónia}
\begin{itemize}
\item {Grp. gram.:f.}
\end{itemize}
\begin{itemize}
\item {Proveniência:(Do gr. \textunderscore thrix\textunderscore , \textunderscore trikhos\textunderscore  + \textunderscore gonia\textunderscore )}
\end{itemize}
Gênero de plantas, da fam. das compostas.
\section{Tricóide}
\begin{itemize}
\item {Grp. gram.:adj.}
\end{itemize}
\begin{itemize}
\item {Proveniência:(Do gr. \textunderscore thrix\textunderscore , \textunderscore trikhos\textunderscore  + \textunderscore eidos\textunderscore )}
\end{itemize}
Semelhante a um cabelo.
\section{Tricologia}
\begin{itemize}
\item {Grp. gram.:f.}
\end{itemize}
\begin{itemize}
\item {Proveniência:(Do gr. \textunderscore thrix\textunderscore , \textunderscore trikhos\textunderscore  + \textunderscore logos\textunderscore )}
\end{itemize}
Tratado á cêrca dos pêlos ou dos cabelos.
\section{Tricoloma}
\begin{itemize}
\item {Grp. gram.:f.}
\end{itemize}
Gênero de plantas escrofularíneas.
\section{Tricoma}
\begin{itemize}
\item {Grp. gram.:m.}
\end{itemize}
\begin{itemize}
\item {Proveniência:(Gr. \textunderscore trichhoma\textunderscore )}
\end{itemize}
Doença, que ataca os cabelos, enredando-se êstes por fórma, que se não desembaraçam nem se cortam sem derramamento de sangue.
Denomina-se também \textunderscore plica-polónica\textunderscore .
\section{Tricomária}
\begin{itemize}
\item {Grp. gram.:f.}
\end{itemize}
\begin{itemize}
\item {Proveniência:(Do gr. \textunderscore trikhoma\textunderscore )}
\end{itemize}
Gênero de plantas malpigiáceas.
\section{Tricomático}
\begin{itemize}
\item {Grp. gram.:adj.}
\end{itemize}
Relativo ao tricoma.
Que padece tricoma.
\section{Tricomatoso}
\begin{itemize}
\item {Grp. gram.:adj.}
\end{itemize}
Relativo ao tricoma.
Que padece tricoma.
\section{Tricomisco}
\begin{itemize}
\item {Grp. gram.:m.}
\end{itemize}
Gênero de crustáceos.
\section{Triconoto}
\begin{itemize}
\item {Grp. gram.:m.}
\end{itemize}
Gênero de insectos coleópteros.
\section{Tricopília}
\begin{itemize}
\item {Grp. gram.:f.}
\end{itemize}
Gênero de orquídeas.
\section{Tricopódio}
\begin{itemize}
\item {Grp. gram.:m.}
\end{itemize}
\begin{itemize}
\item {Proveniência:(Do gr. \textunderscore trix\textunderscore , \textunderscore trikhos\textunderscore  + \textunderscore pous\textunderscore , \textunderscore podos\textunderscore )}
\end{itemize}
Gênero do plantas aristolóquias.
\section{Tricópode}
\begin{itemize}
\item {Grp. gram.:m.}
\end{itemize}
\begin{itemize}
\item {Proveniência:(Do gr. \textunderscore trix\textunderscore , \textunderscore trikhos\textunderscore  + \textunderscore pous\textunderscore , \textunderscore podos\textunderscore )}
\end{itemize}
Gênero de peixes.
\section{Tricosandra}
\begin{itemize}
\item {Grp. gram.:f.}
\end{itemize}
Gênero de plantas asclepiadáceas.
\section{Tricosanto}
\begin{itemize}
\item {Grp. gram.:m.}
\end{itemize}
Gênero de plantas cucurbitáceas.
\section{Tricose}
\begin{itemize}
\item {Grp. gram.:f.}
\end{itemize}
\begin{itemize}
\item {Utilização:Med.}
\end{itemize}
\begin{itemize}
\item {Proveniência:(Do gr. \textunderscore trix\textunderscore , \textunderscore trikhos\textunderscore )}
\end{itemize}
Desenvolvimento anormal de pêlos na mucosa da bexiga ou da uretra.
\section{Tricuspidal}
\begin{itemize}
\item {Grp. gram.:adj.}
\end{itemize}
O mesmo que \textunderscore tricúspide\textunderscore .
\section{Tricúspide}
\begin{itemize}
\item {Grp. gram.:adj.}
\end{itemize}
\begin{itemize}
\item {Proveniência:(Lat. \textunderscore tricuspis\textunderscore )}
\end{itemize}
Que tem três pontas.
\section{Tricuspídeo}
\begin{itemize}
\item {Grp. gram.:adj.}
\end{itemize}
O mesmo que \textunderscore tricúspide\textunderscore .
\section{Tricycla}
\begin{itemize}
\item {Grp. gram.:f.}
\end{itemize}
\begin{itemize}
\item {Proveniência:(Do gr. \textunderscore treis\textunderscore  + \textunderscore kuklos\textunderscore )}
\end{itemize}
Gênero de plantas nyctagíneas.
\section{Tricycleta}
\begin{itemize}
\item {Grp. gram.:f.}
\end{itemize}
Pequeno tricyclo.
\section{Tricyclo}
\begin{itemize}
\item {Grp. gram.:m.}
\end{itemize}
\begin{itemize}
\item {Proveniência:(Do gr. \textunderscore treis\textunderscore  + \textunderscore kuklos\textunderscore )}
\end{itemize}
Velocípede de três rodas.
Antiga carruagem de três rodas.
\section{Tridácio}
\begin{itemize}
\item {Grp. gram.:m.}
\end{itemize}
\begin{itemize}
\item {Proveniência:(Do gr. \textunderscore thridax\textunderscore )}
\end{itemize}
Substância medicamentosa, que se prepara com o suco de alface.
\section{Tridacna}
\begin{itemize}
\item {Grp. gram.:f.}
\end{itemize}
\begin{itemize}
\item {Proveniência:(Do gr. \textunderscore tridaknein\textunderscore )}
\end{itemize}
Gênero do molluscos acéphalos.
\section{Tridacofila}
\begin{itemize}
\item {Grp. gram.:f.}
\end{itemize}
Gênero de polipeiros pétreos.
\section{Tridacophylla}
\begin{itemize}
\item {Grp. gram.:f.}
\end{itemize}
Gênero de polypeiros pétreos.
\section{Tridáctilo}
\begin{itemize}
\item {Grp. gram.:adj.}
\end{itemize}
\begin{itemize}
\item {Grp. gram.:M.}
\end{itemize}
\begin{itemize}
\item {Proveniência:(Do gr. \textunderscore tri\textunderscore  + \textunderscore daktulos\textunderscore )}
\end{itemize}
Que tem três dedos.
Gênero de reptís sáurios.
Gênero de insectos ortópteros.
\section{Tridáctylo}
\begin{itemize}
\item {Grp. gram.:adj.}
\end{itemize}
\begin{itemize}
\item {Grp. gram.:M.}
\end{itemize}
\begin{itemize}
\item {Proveniência:(Do gr. \textunderscore tri\textunderscore  + \textunderscore daktulos\textunderscore )}
\end{itemize}
Que tem três dedos.
Gênero de reptís sáurios.
Gênero de insectos orthópteros.
\section{Tridentado}
\begin{itemize}
\item {Grp. gram.:adj.}
\end{itemize}
\begin{itemize}
\item {Utilização:Bot.}
\end{itemize}
\begin{itemize}
\item {Proveniência:(De \textunderscore tridente\textunderscore )}
\end{itemize}
Que tem três dentes ou três divisões em fórma de dentes.
\section{Tridente}
\begin{itemize}
\item {Grp. gram.:adj.}
\end{itemize}
\begin{itemize}
\item {Grp. gram.:M.}
\end{itemize}
\begin{itemize}
\item {Utilização:Fig.}
\end{itemize}
\begin{itemize}
\item {Utilização:Bras}
\end{itemize}
\begin{itemize}
\item {Proveniência:(Lat. \textunderscore tridens\textunderscore )}
\end{itemize}
Que tem três dentes.
Sceptro mythológico de Neptuno.
Domínio dos mares.
O mar.
Apparelho de pesca, em fórma de garfo, com vários dentes, terminados em anzóes, para a pesca do peixe-agulha.
\section{Tridênteo}
\begin{itemize}
\item {Grp. gram.:adj.}
\end{itemize}
Relativo a tridente.
\section{Tridentífero}
\begin{itemize}
\item {Grp. gram.:adj.}
\end{itemize}
\begin{itemize}
\item {Proveniência:(Lat. \textunderscore tridentifer\textunderscore )}
\end{itemize}
O mesmo que \textunderscore tridentígero\textunderscore .
\section{Tridentígero}
\begin{itemize}
\item {Grp. gram.:adj.}
\end{itemize}
\begin{itemize}
\item {Utilização:Poét.}
\end{itemize}
\begin{itemize}
\item {Proveniência:(Lat. \textunderscore tridentiger\textunderscore )}
\end{itemize}
Que tem tridente; que leva o tridente.
\section{Tridentino}
\begin{itemize}
\item {Grp. gram.:adj.}
\end{itemize}
\begin{itemize}
\item {Proveniência:(Lat. \textunderscore tridentinus\textunderscore )}
\end{itemize}
Relativo a Trento.
\section{Tridésmide}
\begin{itemize}
\item {Grp. gram.:f.}
\end{itemize}
Gênero de plantas hypericáceas.
\section{Tridigitado}
\begin{itemize}
\item {Grp. gram.:adj.}
\end{itemize}
\begin{itemize}
\item {Proveniência:(De \textunderscore tri...\textunderscore  + lat. \textunderscore digitatus\textunderscore )}
\end{itemize}
O mesmo que \textunderscore tridáctylo\textunderscore .
\section{Tridimito}
\begin{itemize}
\item {Grp. gram.:m.}
\end{itemize}
\begin{itemize}
\item {Utilização:Miner.}
\end{itemize}
\begin{itemize}
\item {Proveniência:(Do gr. \textunderscore tridumos\textunderscore )}
\end{itemize}
Espécie de quartzo, que só se encontra em rochas vulcânicas.
\section{Trídimos}
\begin{itemize}
\item {Grp. gram.:m. pl.}
\end{itemize}
\begin{itemize}
\item {Utilização:Miner.}
\end{itemize}
\begin{itemize}
\item {Proveniência:(Do gr. \textunderscore tridumos\textunderscore )}
\end{itemize}
Uma das divisões das fórmas macias dos crystaes homomorphos.
\section{Tridodecaédro}
\begin{itemize}
\item {Grp. gram.:adj.}
\end{itemize}
\begin{itemize}
\item {Utilização:Miner.}
\end{itemize}
Diz-se do mineral, que apresenta três dodecaédros.
\section{Tridrachmo}
\begin{itemize}
\item {Grp. gram.:m.}
\end{itemize}
\begin{itemize}
\item {Proveniência:(Gr. \textunderscore tridrakhmon\textunderscore )}
\end{itemize}
Moéda grega, do valor de três drachmas.
\section{Triduano}
\begin{itemize}
\item {Grp. gram.:adj.}
\end{itemize}
\begin{itemize}
\item {Proveniência:(Lat. \textunderscore triduanus\textunderscore )}
\end{itemize}
Que dura três dias.
\section{Tríduo}
\begin{itemize}
\item {Grp. gram.:m.}
\end{itemize}
\begin{itemize}
\item {Proveniência:(Lat. \textunderscore triduus\textunderscore )}
\end{itemize}
Espaço de três dias successivos.
Festa ecclesiástica, que dura três dias.
\section{Tridymito}
\begin{itemize}
\item {Grp. gram.:m.}
\end{itemize}
\begin{itemize}
\item {Utilização:Miner.}
\end{itemize}
\begin{itemize}
\item {Proveniência:(Do gr. \textunderscore tridumos\textunderscore )}
\end{itemize}
Espécie de quartzo, que só se encontra em rochas vulcânicas.
\section{Trídymos}
\begin{itemize}
\item {Grp. gram.:m. pl.}
\end{itemize}
\begin{itemize}
\item {Utilização:Miner.}
\end{itemize}
\begin{itemize}
\item {Proveniência:(Do gr. \textunderscore tridumos\textunderscore )}
\end{itemize}
Uma das divisões das fórmas macias dos crystaes homomorphos.
\section{Triecia}
\begin{itemize}
\item {Grp. gram.:f.}
\end{itemize}
\begin{itemize}
\item {Utilização:Bot.}
\end{itemize}
\begin{itemize}
\item {Proveniência:(Do gr. \textunderscore treis\textunderscore  + \textunderscore oikia\textunderscore )}
\end{itemize}
Conjunto das plantas que comprehendem flôres hermaphroditas, masculinas e femininas.
\section{Triécico}
Relativo á triecía.
\section{Triedro}
\begin{itemize}
\item {Grp. gram.:adj.}
\end{itemize}
\begin{itemize}
\item {Proveniência:(Do gr. \textunderscore treis\textunderscore  + \textunderscore edra\textunderscore )}
\end{itemize}
Que tem três faces, ou que é formado por três planos.
\section{Trienado}
\begin{itemize}
\item {Grp. gram.:m.}
\end{itemize}
O mesmo que \textunderscore triênio\textunderscore .
\section{Trienal}
\begin{itemize}
\item {Grp. gram.:adj.}
\end{itemize}
\begin{itemize}
\item {Proveniência:(De \textunderscore triênio\textunderscore )}
\end{itemize}
Que dura três anos.
Que serve por três anos.
Que dá fruto, de três em três anos.
\section{Triênio}
\begin{itemize}
\item {Grp. gram.:m.}
\end{itemize}
\begin{itemize}
\item {Proveniência:(Lat. \textunderscore triennium\textunderscore )}
\end{itemize}
Espaço de três anos.
Exercício de um cargo por três anos.
\section{Triennado}
\begin{itemize}
\item {Grp. gram.:m.}
\end{itemize}
O mesmo que \textunderscore triênnio\textunderscore .
\section{Triennal}
\begin{itemize}
\item {Grp. gram.:adj.}
\end{itemize}
\begin{itemize}
\item {Proveniência:(De \textunderscore triênnio\textunderscore )}
\end{itemize}
Que dura três annos.
Que serve por três annos.
Que dá fruto, de três em três annos.
\section{Triênnio}
\begin{itemize}
\item {Grp. gram.:m.}
\end{itemize}
\begin{itemize}
\item {Proveniência:(Lat. \textunderscore triennium\textunderscore )}
\end{itemize}
Espaço de três annos.
Exercício de um cargo por três annos.
\section{Triental}
\begin{itemize}
\item {Grp. gram.:f.}
\end{itemize}
\begin{itemize}
\item {Proveniência:(Lat. \textunderscore trientalis\textunderscore )}
\end{itemize}
Gênero de plantas primuláceas.
\section{Triente}
\begin{itemize}
\item {Grp. gram.:m.}
\end{itemize}
\begin{itemize}
\item {Proveniência:(Lat. \textunderscore triens\textunderscore )}
\end{itemize}
Antiga moéda romana, equivalente á têrça parte de um asse. Cf. C. Lobo, \textunderscore Sát. de Juv.\textunderscore , I, 223.
\section{Trierarca}
\begin{itemize}
\item {Grp. gram.:m.}
\end{itemize}
O mesmo que \textunderscore trierarco\textunderscore .
\section{Trígamo}
\begin{itemize}
\item {Grp. gram.:m.}
\end{itemize}
\begin{itemize}
\item {Proveniência:(Lat. \textunderscore trigamus\textunderscore )}
\end{itemize}
Aquelle que é casado com três mulheres ao mesmo tempo.
Aquelle que casou três vezes.
\section{Trigança}
\begin{itemize}
\item {Grp. gram.:f.}
\end{itemize}
\begin{itemize}
\item {Utilização:Des.}
\end{itemize}
O mesmo que \textunderscore triga\textunderscore ^1.
\section{Trigar-se}
\begin{itemize}
\item {Grp. gram.:v. p.}
\end{itemize}
\begin{itemize}
\item {Utilização:Des.}
\end{itemize}
\begin{itemize}
\item {Proveniência:(Do lat. \textunderscore tricare\textunderscore )}
\end{itemize}
Apressar-se.
Andar com pressa.
Proceder apressadamente:«\textunderscore ...e por esta razom se trigou El-Rei de mandar...\textunderscore »Fern. Lopes, \textunderscore Chrón. de D. João I\textunderscore , c. XLV.
\section{Trigêmeo}
\begin{itemize}
\item {Grp. gram.:m.  e  adj.}
\end{itemize}
\begin{itemize}
\item {Utilização:Anat.}
\end{itemize}
\begin{itemize}
\item {Proveniência:(Do lat. \textunderscore trigeminus\textunderscore )}
\end{itemize}
Cada um dos três indivíduos, que nasceram de um só parto.
Diz-se do nervo trifacial.
\section{Trigeminada}
\begin{itemize}
\item {Grp. gram.:adj. f.}
\end{itemize}
\begin{itemize}
\item {Proveniência:(De \textunderscore trigêmino\textunderscore )}
\end{itemize}
Diz-se da janela, dividida em seis vãos.
\section{Trigêmino}
\begin{itemize}
\item {Grp. gram.:adj.}
\end{itemize}
\begin{itemize}
\item {Proveniência:(Lat. \textunderscore trigeminus\textunderscore )}
\end{itemize}
O mesmo que \textunderscore trífido\textunderscore .
\section{Trigênea}
\begin{itemize}
\item {Grp. gram.:f.}
\end{itemize}
\begin{itemize}
\item {Proveniência:(Do gr. \textunderscore treis\textunderscore  + \textunderscore genos\textunderscore )}
\end{itemize}
Gênero de plantas anonáceas.
\section{Trigésimo}
\begin{itemize}
\item {Grp. gram.:adj.}
\end{itemize}
\begin{itemize}
\item {Grp. gram.:M.}
\end{itemize}
\begin{itemize}
\item {Proveniência:(Lat. \textunderscore trigesimus\textunderscore )}
\end{itemize}
Que numa série de trinta occupa o último lugar.
Cada uma das trinta partes, em que se divide um todo.
\section{Trigla}
\begin{itemize}
\item {Grp. gram.:f.}
\end{itemize}
O mesmo que \textunderscore triglo\textunderscore ^1.
Espécie de salmonete.
\section{Triglídeo}
\begin{itemize}
\item {Grp. gram.:adj.}
\end{itemize}
\begin{itemize}
\item {Grp. gram.:M. pl.}
\end{itemize}
\begin{itemize}
\item {Proveniência:(Do gr. \textunderscore trigla\textunderscore  + \textunderscore eidos\textunderscore )}
\end{itemize}
Relativo ou semelhante ao triglo.
Grupo de peixes, que tem por typo o triglo.
\section{Tríglifo}
\begin{itemize}
\item {Grp. gram.:m.}
\end{itemize}
\begin{itemize}
\item {Utilização:Miner.}
\end{itemize}
\begin{itemize}
\item {Proveniência:(Lat. \textunderscore triglyphus\textunderscore )}
\end{itemize}
Ornato arquitectónico num friso de ordem dórica, constando de três sulcos.
Cristal, cujas faces são cobertas de estrias, dispostas em três sentidos perpendiculares.
\section{Triglo}
\begin{itemize}
\item {Grp. gram.:m.}
\end{itemize}
\begin{itemize}
\item {Proveniência:(Do gr. \textunderscore trigla\textunderscore )}
\end{itemize}
Gênero de peixes acanthopterýgios, de corpo alongado, ligeiramente comprimido.
\section{Triglota}
\begin{itemize}
\item {Grp. gram.:adj.}
\end{itemize}
\begin{itemize}
\item {Grp. gram.:M.  e  adj.}
\end{itemize}
\begin{itemize}
\item {Proveniência:(Do gr. \textunderscore tri\textunderscore  + \textunderscore glotta\textunderscore )}
\end{itemize}
Escrito ou composto em três línguas.
O que conhece ou fala três línguas.
\section{Triglotismo}
\begin{itemize}
\item {Grp. gram.:m.}
\end{itemize}
Qualquer palavra híbrida, composta de três palavras ou três elementos, tirados de três línguas diferentes.
Frase, formada de três palavras, tirada cada uma de língua diferente.
(Cp. \textunderscore triglota\textunderscore )
\section{Trigloto}
\begin{itemize}
\item {Grp. gram.:m.  e  adj.}
\end{itemize}
O mesmo que \textunderscore triglota\textunderscore . Cf. R. Galvão, \textunderscore Vocab.\textunderscore 
\section{Triglotta}
\begin{itemize}
\item {Grp. gram.:adj.}
\end{itemize}
\begin{itemize}
\item {Grp. gram.:M.  e  adj.}
\end{itemize}
\begin{itemize}
\item {Proveniência:(Do gr. \textunderscore tri\textunderscore  + \textunderscore glotta\textunderscore )}
\end{itemize}
Escrito ou composto em três línguas.
O que conhece ou fala três línguas.
\section{Triglottismo}
\begin{itemize}
\item {Grp. gram.:m.}
\end{itemize}
Qualquer palavra hýbrida, composta de três palavras ou três elementos, tirados de três línguas differentes.
Frase, formada de três palavras, tirada cada uma de língua differente.
(Cp. \textunderscore triglotta\textunderscore )
\section{Triglotto}
\begin{itemize}
\item {Grp. gram.:m.  e  adj.}
\end{itemize}
O mesmo que \textunderscore triglotta\textunderscore . Cf. R. Galvão, \textunderscore Vocab.\textunderscore .
\section{Tríglumo}
\begin{itemize}
\item {Grp. gram.:adj.}
\end{itemize}
\begin{itemize}
\item {Utilização:Bot.}
\end{itemize}
\begin{itemize}
\item {Proveniência:(De \textunderscore tri...\textunderscore  + \textunderscore gluma\textunderscore )}
\end{itemize}
Que tem três glumas.
\section{Tríglypho}
\begin{itemize}
\item {Grp. gram.:m.}
\end{itemize}
\begin{itemize}
\item {Utilização:Miner.}
\end{itemize}
\begin{itemize}
\item {Proveniência:(Lat. \textunderscore triglyphus\textunderscore )}
\end{itemize}
Ornato architectónico num friso de ordem dórica, constando de três sulcos.
Crystal, cujas faces são cobertas de estrias, dispostas em três sentidos perpendiculares.
\section{Trigo}
\begin{itemize}
\item {Grp. gram.:m.}
\end{itemize}
\begin{itemize}
\item {Utilização:Prov.}
\end{itemize}
\begin{itemize}
\item {Grp. gram.:Adj.}
\end{itemize}
\begin{itemize}
\item {Proveniência:(Do lat. \textunderscore triticum\textunderscore )}
\end{itemize}
Gênero de plantas gramíneas.
Fruto dessas plantas, applicado especialmente ao fabríco do pão.
Pão de trigo.
Feito de trigo: \textunderscore farinha triga\textunderscore ; \textunderscore fogaça triga\textunderscore ; \textunderscore pão trigo\textunderscore .
\section{Trigo-da-terra}
\begin{itemize}
\item {Grp. gram.:m.}
\end{itemize}
Variedade de trigo molle.
\section{Trigo-de-barba-preta}
\begin{itemize}
\item {Grp. gram.:m.}
\end{itemize}
\begin{itemize}
\item {Utilização:Prov.}
\end{itemize}
Trigo rijo.
\section{Trigo-de-milagre}
\begin{itemize}
\item {Grp. gram.:m.}
\end{itemize}
\begin{itemize}
\item {Utilização:Bras}
\end{itemize}
Planta gramínea brasileira, (\textunderscore triticum compositum\textunderscore , Lin.).
\section{Trigolpe}
\begin{itemize}
\item {Grp. gram.:m.}
\end{itemize}
\begin{itemize}
\item {Proveniência:(De \textunderscore tri...\textunderscore  + \textunderscore golpe\textunderscore )}
\end{itemize}
Golpe tríplice. Cf. Filinto, XV, 29.
\section{Trígone}
\begin{itemize}
\item {Grp. gram.:f.}
\end{itemize}
Lyra grega, quási em fórma de triângulo. Cf. Castilho, \textunderscore Fastos\textunderscore , III, 204.
(Cp. \textunderscore trígono\textunderscore )
\section{Trigonela}
\begin{itemize}
\item {Grp. gram.:f.}
\end{itemize}
Gênero de plantas leguminosas.
(Cp. \textunderscore trígono\textunderscore )
\section{Trigónia}
\begin{itemize}
\item {Grp. gram.:f.}
\end{itemize}
\begin{itemize}
\item {Proveniência:(Do gr. \textunderscore trigonon\textunderscore )}
\end{itemize}
Gênero de plantas polygáleas.
\section{Trigonicórneo}
\begin{itemize}
\item {Grp. gram.:adj.}
\end{itemize}
\begin{itemize}
\item {Utilização:Zool.}
\end{itemize}
\begin{itemize}
\item {Proveniência:(De \textunderscore trígono\textunderscore  + \textunderscore córneo\textunderscore )}
\end{itemize}
Diz-se do insecto, que tem as antennas triangulares.
\section{Trigonídio}
\begin{itemize}
\item {Grp. gram.:m.}
\end{itemize}
\begin{itemize}
\item {Proveniência:(Do gr. \textunderscore trigonon\textunderscore )}
\end{itemize}
Gênero de orchídeas.
Gênero de insectos orthópteros.
\section{Trígono}
\begin{itemize}
\item {Grp. gram.:adj.}
\end{itemize}
\begin{itemize}
\item {Grp. gram.:M.}
\end{itemize}
\begin{itemize}
\item {Proveniência:(Lat. \textunderscore trigonus\textunderscore )}
\end{itemize}
Triangular.
Aspecto de dois planetas que distam um do outro 120°.
Gênero de molluscos.
\section{Trilateral}
\begin{itemize}
\item {Grp. gram.:adj.}
\end{itemize}
\begin{itemize}
\item {Proveniência:(Lat. \textunderscore trilaterus\textunderscore )}
\end{itemize}
Que tem três lados.
\section{Trilátero}
\begin{itemize}
\item {Grp. gram.:adj.}
\end{itemize}
\begin{itemize}
\item {Proveniência:(Lat. \textunderscore trilaterus\textunderscore )}
\end{itemize}
Que tem três lados.
\section{Trilema}
\begin{itemize}
\item {Grp. gram.:m.}
\end{itemize}
\begin{itemize}
\item {Utilização:Neol.}
\end{itemize}
\begin{itemize}
\item {Proveniência:(De \textunderscore tri...\textunderscore  + \textunderscore lema\textunderscore )}
\end{itemize}
Situação embaraçosa, de que não há saída, senão por um de três modos, todos difíceis ou penosos.
\section{Trilemma}
\begin{itemize}
\item {Grp. gram.:m.}
\end{itemize}
\begin{itemize}
\item {Utilização:Neol.}
\end{itemize}
\begin{itemize}
\item {Proveniência:(De \textunderscore tri...\textunderscore  + \textunderscore lemma\textunderscore )}
\end{itemize}
Situação embaraçosa, de que não há saída, senão por um de três modos, todos diffíceis ou penosos.
\section{Trilépide}
\begin{itemize}
\item {Grp. gram.:f.}
\end{itemize}
Gênero de plantas cyperáceas.
\section{Trilepísio}
\begin{itemize}
\item {Grp. gram.:m.}
\end{itemize}
\begin{itemize}
\item {Proveniência:(Do gr. \textunderscore treis\textunderscore  + \textunderscore lepis\textunderscore )}
\end{itemize}
Gênero de plantas.
\section{Trileucos}
\begin{itemize}
\item {Grp. gram.:adj. Pl.}
\end{itemize}
Dizia-se de uns rochedos ou ilhas desertas, no mar Cantábrico. Cf. \textunderscore Viriato Trág.\textunderscore , X, 44.
\section{Trilha}
\begin{itemize}
\item {Grp. gram.:f.}
\end{itemize}
Acto ou effeito de trilhar.
Acto de debulhar cereaes na eira.
Rasto, peugada.
Vereda, trilho.
\section{Trilhada}
\begin{itemize}
\item {Grp. gram.:f.}
\end{itemize}
O mesmo que \textunderscore trilha\textunderscore .
\section{Trilhado}
\begin{itemize}
\item {Grp. gram.:Adj.}
\end{itemize}
\begin{itemize}
\item {Utilização:Fig.}
\end{itemize}
\begin{itemize}
\item {Proveniência:(De \textunderscore trilhar\textunderscore )}
\end{itemize}
Calcado.
Esbagoado.
Conhecido, usado, trivial.
\section{Trilhador}
\begin{itemize}
\item {Grp. gram.:m.  e  adj.}
\end{itemize}
O que trilha.
\section{Trilhadura}
\begin{itemize}
\item {Grp. gram.:f.}
\end{itemize}
Acto ou effeito de trilhar.
Trilha, caminho seguido. Cf. Usque, 22, v.^o.
\section{Trilhamento}
\begin{itemize}
\item {Grp. gram.:f.}
\end{itemize}
Acto ou effeito de trilhar.
Trilha, caminho seguido. Cf. Usque, 22, v.^o.
\section{Trilhão}
\begin{itemize}
\item {Grp. gram.:m.}
\end{itemize}
\begin{itemize}
\item {Proveniência:(Fr. \textunderscore trillion\textunderscore . Cp. \textunderscore milhão\textunderscore ^1)}
\end{itemize}
Mil biliões, segundo o sistema francês; ou um milhão de biliões, segundo o sistema inglês.
\section{Trilhão}
\begin{itemize}
\item {Grp. gram.:m.}
\end{itemize}
O mesmo ou melhor que \textunderscore trillião\textunderscore . Cf. \textunderscore Notícia\textunderscore , do Rio, de 27-VI-902.
\section{Trilhar}
\begin{itemize}
\item {Grp. gram.:v. t.}
\end{itemize}
\begin{itemize}
\item {Proveniência:(Do lat. \textunderscore tribulare\textunderscore )}
\end{itemize}
Esbagoar ou debulhar (cereaes).
Pisar, moer.
Reduzir a pequenas partes.
Marcar com pègadas, marcar com o rasto.
Esmagar.
Bater.
Abrir caminho por.
Seguir (determinada direcção).
Magoar, contundir.
\section{Trilho}
\begin{itemize}
\item {Grp. gram.:m.}
\end{itemize}
\begin{itemize}
\item {Utilização:Bras. do S}
\end{itemize}
\begin{itemize}
\item {Proveniência:(Do lat. \textunderscore tribulum\textunderscore )}
\end{itemize}
Utensílio de lavoira, próprio para debulhar trigo e outros cereaes.
Utensílio, com que se bate o leite para fabricar queijo.
Trilha.
Caminho, vereda; direcção.
Modo de proceder ou de pensar.
Carril de ferro, sôbre que andam combóios ou outro vehículos.
\textunderscore Chapa de trilho\textunderscore , círculo de ferro, que reforça os aros das rodas dos combóios e toca directamente nos carris.
\textunderscore Prego de trilho\textunderscore , cada um dos pregos, que seguram a chapa que reveste as rodas dos vehículos.
\section{Trilhoada}
\begin{itemize}
\item {Grp. gram.:f.}
\end{itemize}
\begin{itemize}
\item {Proveniência:(De \textunderscore trilho\textunderscore )}
\end{itemize}
Carroça, com que dantes se debulhava trigo.
Parelha de bêstas, com que se debulham cereaes na eira.
\section{Trilião}
\begin{itemize}
\item {Grp. gram.:m.}
\end{itemize}
\begin{itemize}
\item {Proveniência:(Fr. \textunderscore trillion\textunderscore . Cp. \textunderscore milhão\textunderscore ^1)}
\end{itemize}
Mil biliões, segundo o sistema francês; ou um milhão de biliões, segundo o sistema inglês.
\section{Trílice}
\begin{itemize}
\item {Grp. gram.:adj.}
\end{itemize}
\begin{itemize}
\item {Grp. gram.:M.}
\end{itemize}
\begin{itemize}
\item {Proveniência:(Lat. \textunderscore trilix\textunderscore )}
\end{itemize}
Que tem três fios.
Gênero de plantas tiliáceas.
\section{Trilingue}
\begin{itemize}
\item {Grp. gram.:m.  e  adj.}
\end{itemize}
\begin{itemize}
\item {Proveniência:(Lat. \textunderscore trilinguis\textunderscore )}
\end{itemize}
O mesmo que \textunderscore triglotta\textunderscore .
\section{Trílio}
\begin{itemize}
\item {Grp. gram.:m.}
\end{itemize}
Gênero de plantas esmiláceas.
\section{Triliteral}
\begin{itemize}
\item {Grp. gram.:adj.}
\end{itemize}
\begin{itemize}
\item {Proveniência:(Do lat. \textunderscore tres\textunderscore  + \textunderscore litera\textunderscore )}
\end{itemize}
Que é composto de três letras.
\section{Trilítero}
\begin{itemize}
\item {Grp. gram.:adj.}
\end{itemize}
\begin{itemize}
\item {Proveniência:(Do lat. \textunderscore tres\textunderscore  + \textunderscore litera\textunderscore )}
\end{itemize}
Que é composto de três letras.
\section{Trillião}
\begin{itemize}
\item {Grp. gram.:m.}
\end{itemize}
\begin{itemize}
\item {Proveniência:(Fr. \textunderscore trillion\textunderscore . Cp. \textunderscore milhão\textunderscore ^1)}
\end{itemize}
Mil billiões, segundo o systema francês; ou um milhão de biliões, segundo o systema inglês.
\section{Tríllio}
\begin{itemize}
\item {Grp. gram.:m.}
\end{itemize}
Gênero de plantas esmiláceas.
\section{Trilo}
\begin{itemize}
\item {Grp. gram.:m.}
\end{itemize}
\begin{itemize}
\item {Proveniência:(T. onom.)}
\end{itemize}
Movimento alternado e rápido de duas notas musicaes, que distam entre si um tom ou semitom.
Trinado, gorgeio.
\section{Trilobado}
\begin{itemize}
\item {Grp. gram.:adj.}
\end{itemize}
\begin{itemize}
\item {Proveniência:(De \textunderscore tri...\textunderscore  + \textunderscore lobado\textunderscore )}
\end{itemize}
Que tem três lóbulos.
\section{Trilóbeo}
\begin{itemize}
\item {Grp. gram.:m.}
\end{itemize}
O mesmo que \textunderscore trilobite\textunderscore .
\section{Trilobite}
\begin{itemize}
\item {Grp. gram.:f.}
\end{itemize}
Crustáceo fóssil, da série paleozóica, o qual constitue uma ordem ou família de branchiópodes.
\section{Trilobito}
\begin{itemize}
\item {Grp. gram.:m.}
\end{itemize}
O mesmo que \textunderscore trilobite\textunderscore .
\section{Trilocular}
\begin{itemize}
\item {Grp. gram.:adj.}
\end{itemize}
\begin{itemize}
\item {Proveniência:(De \textunderscore tri...\textunderscore  + \textunderscore locular\textunderscore )}
\end{itemize}
Que tem três lóculos.
\section{Trinca-espinhas}
\begin{itemize}
\item {Grp. gram.:m.}
\end{itemize}
\begin{itemize}
\item {Utilização:Burl.}
\end{itemize}
Homem alto e magro.
\section{Trincafiar}
\begin{itemize}
\item {Grp. gram.:v. t.}
\end{itemize}
\begin{itemize}
\item {Utilização:Pop.}
\end{itemize}
Prender com trincafio.
Amarrar.
Prender.
Encarcerar.
\section{Trincafio}
\begin{itemize}
\item {Grp. gram.:m.}
\end{itemize}
\begin{itemize}
\item {Utilização:Náut.}
\end{itemize}
\begin{itemize}
\item {Utilização:Fig.}
\end{itemize}
Linha de sapateiro.
Espécie de cabo delgado.
Manha, astúcia.
Porção de estopa, que se enrola nas roscas do parafuso, para se apertarem bem as respectivas porcas.
(Cast. \textunderscore trincafia\textunderscore )
\section{Trincal}
\begin{itemize}
\item {Grp. gram.:m. ,  f.  e  adj.}
\end{itemize}
\begin{itemize}
\item {Utilização:Prov.}
\end{itemize}
\begin{itemize}
\item {Proveniência:(De \textunderscore trincar\textunderscore )}
\end{itemize}
Diz-se de uma variedade de uva preta.
\section{Trincalhos}
\begin{itemize}
\item {Grp. gram.:m.}
\end{itemize}
\begin{itemize}
\item {Utilização:Açor}
\end{itemize}
Sino.
\section{Trincalhós}
\begin{itemize}
\item {Grp. gram.:m.}
\end{itemize}
\begin{itemize}
\item {Utilização:Açor}
\end{itemize}
Indivíduo adoentado, fraco, acanaveado.
\section{Trincanizes}
\begin{itemize}
\item {Grp. gram.:m. pl.}
\end{itemize}
\begin{itemize}
\item {Utilização:Náut.}
\end{itemize}
Tabuões, em que se abrem os embornaes, e cujos cantos assentam na amurada do navio.
\section{Trinca-nozes}
\begin{itemize}
\item {Grp. gram.:m.}
\end{itemize}
O mesmo que \textunderscore trinca-pinhas\textunderscore .
Quebra-nozes.
\section{Trinca-pau}
\begin{itemize}
\item {Grp. gram.:m.}
\end{itemize}
Insecto lepidóptero.
\section{Trinca-pinhas}
\begin{itemize}
\item {Grp. gram.:m.}
\end{itemize}
Pássaro, o mesmo que \textunderscore cruza-bico\textunderscore .
\section{Trinca-pintos}
\begin{itemize}
\item {Grp. gram.:m.}
\end{itemize}
\begin{itemize}
\item {Grp. gram.:F.}
\end{itemize}
O mesmo que \textunderscore raposo\textunderscore .
O mesmo que \textunderscore raposa\textunderscore .
\section{Trincar}
\begin{itemize}
\item {Grp. gram.:v. t.}
\end{itemize}
\begin{itemize}
\item {Utilização:Pop.}
\end{itemize}
\begin{itemize}
\item {Utilização:Náut.}
\end{itemize}
\begin{itemize}
\item {Grp. gram.:V. i.}
\end{itemize}
\begin{itemize}
\item {Utilização:Bras. do N}
\end{itemize}
Partir com os dentes.
Morder.
Comprimir com os dentes.
Comer; mastigar.
Prender com trinca.
Fazer ruído alguma coisa, quando se parte com os dentes.
Produzir som metállico; tinir.
(Cp. cast. \textunderscore trincar\textunderscore )
\section{Trincha}
\begin{itemize}
\item {Grp. gram.:f.}
\end{itemize}
\begin{itemize}
\item {Utilização:Ant.}
\end{itemize}
\begin{itemize}
\item {Proveniência:(Do lat. hyp. \textunderscore trinicus\textunderscore , seg. Körting)}
\end{itemize}
Instrumento de carpinteiro, de fórma semelhante á da enxó.
Apara.
Casca.
Posta.
Pincel espalmado, com que se humedecem as fôlhas do copiador, para que neste se fixe uma escrita.
Ferramenta para despregar, formada de uma haste de ferro forjado, adelgaçada na ponta, e que se mete entre prego e prego, para servir de alavanca.
O mesmo que \textunderscore trincheira\textunderscore .
\section{Trincha}
\begin{itemize}
\item {Grp. gram.:f.}
\end{itemize}
\begin{itemize}
\item {Utilização:Prov.}
\end{itemize}
\begin{itemize}
\item {Utilização:minh.}
\end{itemize}
\begin{itemize}
\item {Utilização:beir.}
\end{itemize}
\begin{itemize}
\item {Utilização:Prov.}
\end{itemize}
\begin{itemize}
\item {Utilização:alg.}
\end{itemize}
Cós da sáia.
Trança de palma ou de esparto, com que se fazem esteiras.
(Talvez da mesma or. que \textunderscore trança\textunderscore . Cp. cast. \textunderscore trenza\textunderscore )
\section{Trinchador}
\begin{itemize}
\item {Grp. gram.:m.  e  adj.}
\end{itemize}
O que trincha.
\section{Trinchante}
\begin{itemize}
\item {Grp. gram.:adj.}
\end{itemize}
\begin{itemize}
\item {Grp. gram.:M.}
\end{itemize}
Que trincha ou serve para trinchar.
Aquelle que trincha.
Grande faca, própria para trinchar.
Mesa, aparador, sôbre que se trincha.
\section{Trinchão}
\begin{itemize}
\item {Grp. gram.:m.}
\end{itemize}
\begin{itemize}
\item {Utilização:Ant.}
\end{itemize}
\begin{itemize}
\item {Proveniência:(De \textunderscore trinchar\textunderscore )}
\end{itemize}
Aquelle que trincha.
Bom bocado, grossa fatia. Cf. G. Vicente, I, 230.
\section{Trinchar}
\begin{itemize}
\item {Grp. gram.:v. t.}
\end{itemize}
\begin{itemize}
\item {Grp. gram.:V. i.}
\end{itemize}
Cortar em pedaços, (as viandas das mesas).
Recortar baínhas no fato, para êste assentar bem.
(Cp. cast. \textunderscore trinchar\textunderscore )
\section{Trincheira}
\begin{itemize}
\item {Grp. gram.:f.}
\end{itemize}
\begin{itemize}
\item {Utilização:Náut.}
\end{itemize}
\begin{itemize}
\item {Grp. gram.:Pl.}
\end{itemize}
\begin{itemize}
\item {Utilização:Ant.}
\end{itemize}
Escavação, feita para que a terra escavada sirva de parapeito aos sitiados de uma praça.
Parapeito.
Tapume de madeira, em volta de uma praça de toiros, de um circo, etc.
Tabique parallelo a êsse tapume, mas mais baixo.
Intervallo entre os mesmos tapumes.
Série de assentos ou bancadas, em volta de um circo ou praça de toiros.
Corda, estendida sôbre um terreno, para formação ou nivelamento de uma estrada.
Caixa, sôbre as amuradas, onde se arrecadam as macas da marinhagem.
Queixos.
(Cp. \textunderscore trincha\textunderscore ^1)
\section{Trifilo}
\begin{itemize}
\item {Grp. gram.:adj.}
\end{itemize}
\begin{itemize}
\item {Utilização:Bot.}
\end{itemize}
\begin{itemize}
\item {Proveniência:(Do gr. \textunderscore treis\textunderscore  + \textunderscore phullon\textunderscore )}
\end{itemize}
Diz-se do cálice formado de três peças.
\section{Trifisária}
\begin{itemize}
\item {Grp. gram.:f.}
\end{itemize}
Gênero de plantas escrofularíneas.
\section{Triftongo}
\begin{itemize}
\item {Grp. gram.:m.}
\end{itemize}
(V.tritongo)
\section{Tríncia}
\begin{itemize}
\item {Grp. gram.:f.}
\end{itemize}
Gênero de plantas, da fam. das compostas.
\section{Trinoto}
\begin{itemize}
\item {Grp. gram.:m.}
\end{itemize}
Gênero de insectos.
\section{Trinque}
\begin{itemize}
\item {Grp. gram.:m.}
\end{itemize}
\begin{itemize}
\item {Utilização:Fig.}
\end{itemize}
Cabide de algibebe.
Elegância, esmêro.
Qualidade do que ainda não serviu, do que é novo em fôlha.
\section{Trinquebale}
\begin{itemize}
\item {Grp. gram.:m.}
\end{itemize}
(V.trinqueval)
\section{Trinqueval}
\begin{itemize}
\item {Grp. gram.:m.}
\end{itemize}
Carrêta, para transportar peças de artilharia; zorra.
\section{Trinta}
\begin{itemize}
\item {Grp. gram.:adj.}
\end{itemize}
\begin{itemize}
\item {Grp. gram.:M.}
\end{itemize}
\begin{itemize}
\item {Proveniência:(Do lat. \textunderscore triginta\textunderscore )}
\end{itemize}
Déz vezes três.
Representação dêsse número em algarismos ou conta romana.
Aquelle ou aquillo que numa série de trinta occupa o último lugar.
\section{Trinta-botões}
\begin{itemize}
\item {Grp. gram.:m.}
\end{itemize}
\begin{itemize}
\item {Utilização:Bras. do Rio}
\end{itemize}
Rapaz português, que chegou recentemente ao Brasil.
\section{Trintada}
\begin{itemize}
\item {Grp. gram.:f.}
\end{itemize}
\begin{itemize}
\item {Utilização:T. de Setúbal}
\end{itemize}
Porção de trinta moios de sal, que os compradores de fóra têm de comprar á Misericordia de Setúbal.
\section{Trintadozeno}
\begin{itemize}
\item {Grp. gram.:adj.}
\end{itemize}
\begin{itemize}
\item {Utilização:Ant.}
\end{itemize}
Dizia-se do pano que tinha 3:200 fios de urdidura.
\section{Trinta-e-um}
\begin{itemize}
\item {Grp. gram.:m.}
\end{itemize}
Jôgo, em que, distribuidas três cartas a cada um dos parceiros, êstes pedem as que julgam precisas para se aproximarem de trinta e um pontos, sem excederem êste número.
\section{Trintairo}
\begin{itemize}
\item {Grp. gram.:m.}
\end{itemize}
\begin{itemize}
\item {Utilização:Ant.}
\end{itemize}
O mesmo que \textunderscore trintário\textunderscore .
\section{Trintanário}
\begin{itemize}
\item {Grp. gram.:m.}
\end{itemize}
\begin{itemize}
\item {Proveniência:(Do ant. fr. \textunderscore trantraner\textunderscore )}
\end{itemize}
Criado, que vai ao lado do cocheiro na almofada do trem, e que abre a portinhola, faz recados, etc.
\section{Trintar}
\begin{itemize}
\item {Grp. gram.:v. i.}
\end{itemize}
\begin{itemize}
\item {Utilização:Fam.}
\end{itemize}
Completar trinta annos de idade. Cf. Castilho, \textunderscore Sabichonas\textunderscore , 36.
\section{Trinta-raízes}
\begin{itemize}
\item {Grp. gram.:m.}
\end{itemize}
\begin{itemize}
\item {Utilização:T. de Turquel}
\end{itemize}
Espécie de ave; o mesmo que \textunderscore tem-te-na-raiz\textunderscore ?
\section{Trintário}
\begin{itemize}
\item {Grp. gram.:m.}
\end{itemize}
\begin{itemize}
\item {Proveniência:(De \textunderscore trinta\textunderscore )}
\end{itemize}
Exéquias, no trigésimo dia depois do respectivo fallecimento.
\section{Trintaro}
\begin{itemize}
\item {Grp. gram.:m.}
\end{itemize}
\begin{itemize}
\item {Utilização:Ant.}
\end{itemize}
O mesmo que \textunderscore trintário\textunderscore .
\section{Trintasque}
\begin{itemize}
\item {Grp. gram.:f.}
\end{itemize}
\begin{itemize}
\item {Utilização:Prov.}
\end{itemize}
\begin{itemize}
\item {Utilização:trasm.}
\end{itemize}
Rapariga leviana.
\section{Trintena}
\begin{itemize}
\item {Grp. gram.:f.}
\end{itemize}
A trigésima parte.
Grupo de trinta.
Conjunto de trinta pessôas ou coisas.
\section{Trintenário}
\begin{itemize}
\item {Grp. gram.:m.  e  adj.}
\end{itemize}
\begin{itemize}
\item {Proveniência:(De \textunderscore trintena\textunderscore )}
\end{itemize}
O que tem trinta annos.
\section{Trinteno}
\begin{itemize}
\item {Grp. gram.:m.}
\end{itemize}
\begin{itemize}
\item {Utilização:Ant.}
\end{itemize}
\begin{itemize}
\item {Proveniência:(De \textunderscore trinta\textunderscore )}
\end{itemize}
Pano de 3:000 fios de urdidura.
\section{Trinundino}
\begin{itemize}
\item {Grp. gram.:m.}
\end{itemize}
\begin{itemize}
\item {Proveniência:(Lat. \textunderscore trinundinum\textunderscore )}
\end{itemize}
Intervallo de vinte e sete dias, durante os quaes havia três mercados em Roma. Cf. Castilho, \textunderscore Fastos\textunderscore , I, 243.
\section{Trio}
\begin{itemize}
\item {Grp. gram.:m.}
\end{itemize}
\begin{itemize}
\item {Proveniência:(It. \textunderscore trio\textunderscore )}
\end{itemize}
Trecho musical, próprio para sêr executado por três vozes ou três instrumentos.
Grupo ou conjunto de três pessôas: \textunderscore um trio de poétas\textunderscore . G. Crespo.
\section{Trióbolo}
\begin{itemize}
\item {Grp. gram.:m.}
\end{itemize}
\begin{itemize}
\item {Proveniência:(Lat. \textunderscore triobolus\textunderscore )}
\end{itemize}
Antiga moéda grega, equivalente a três óbolos.
\section{Trioctaédro}
\begin{itemize}
\item {Grp. gram.:m.}
\end{itemize}
\begin{itemize}
\item {Utilização:Miner.}
\end{itemize}
\begin{itemize}
\item {Proveniência:(De \textunderscore tri...\textunderscore  + \textunderscore octaédro\textunderscore )}
\end{itemize}
Polyedro, limitado por 24 triângulos isósceles iguaes, formando, 3 a 3, ângulos triedros regulares, em posições correspondentes ás 8 faces do octaédro.
\section{Trióctil}
\begin{itemize}
\item {Grp. gram.:adj.}
\end{itemize}
\begin{itemize}
\item {Utilização:Astron.}
\end{itemize}
\begin{itemize}
\item {Proveniência:(Do lat. \textunderscore tres\textunderscore , \textunderscore tria\textunderscore  + \textunderscore octo\textunderscore )}
\end{itemize}
Diz-se do aspecto de dois planetas, que distam entre si três oitavos do círculo ou 135°.
\section{Triodonte}
\begin{itemize}
\item {Grp. gram.:m.}
\end{itemize}
\begin{itemize}
\item {Proveniência:(Do gr. \textunderscore treis\textunderscore  + \textunderscore odous\textunderscore )}
\end{itemize}
Gênero de plantas rubiáceas.
Gênero de peixes plectógnathos.
\section{Triões}
\begin{itemize}
\item {Grp. gram.:m. Pl.}
\end{itemize}
\begin{itemize}
\item {Utilização:Des.}
\end{itemize}
\begin{itemize}
\item {Proveniência:(Do lat. \textunderscore triones\textunderscore )}
\end{itemize}
Bois, que puxam um carro ou uma charrua.
Nome antigo das estrêllas da Ursa Maior e da Ursa Menor.
\section{Trioicía}
\begin{itemize}
\item {Grp. gram.:f.}
\end{itemize}
\begin{itemize}
\item {Utilização:Bot.}
\end{itemize}
O mesmo que \textunderscore triecia\textunderscore .
Qualidade de trióico.
\section{Tripes}
\begin{itemize}
\item {Grp. gram.:m. Pl.}
\end{itemize}
\begin{itemize}
\item {Proveniência:(Lat. \textunderscore thripes\textunderscore )}
\end{itemize}
Gênero de pequenos insectos, que vivem nas fôlhas e nas flôres das plantas.
\section{Triphthongo}
\begin{itemize}
\item {Grp. gram.:m.}
\end{itemize}
(V.tritongo)
\section{Triphyllo}
\begin{itemize}
\item {Grp. gram.:adj.}
\end{itemize}
\begin{itemize}
\item {Utilização:Bot.}
\end{itemize}
\begin{itemize}
\item {Proveniência:(Do gr. \textunderscore treis\textunderscore  + \textunderscore phullon\textunderscore )}
\end{itemize}
Diz-se do cálice formado de três peças.
\section{Triphysária}
\begin{itemize}
\item {Grp. gram.:f.}
\end{itemize}
Gênero de plantas escrofularíneas.
\section{Triplamente}
\begin{itemize}
\item {Grp. gram.:adv.}
\end{itemize}
De modo triplo.
\section{Triple}
\begin{itemize}
\item {Grp. gram.:adj.}
\end{itemize}
O mesmo que \textunderscore triplo\textunderscore .
\section{Triplegia}
\begin{itemize}
\item {Grp. gram.:f.}
\end{itemize}
\begin{itemize}
\item {Utilização:Med.}
\end{itemize}
\begin{itemize}
\item {Proveniência:(Do gr. \textunderscore treis\textunderscore  + \textunderscore plessein\textunderscore )}
\end{itemize}
Hemiplegia, acompanhada da paralysia de um membro do lado opposto.
\section{Triplégico}
\begin{itemize}
\item {Grp. gram.:adj.}
\end{itemize}
Relativo á triplegia.
Que soffre triplegia.
\section{Tripleta}
\begin{itemize}
\item {fónica:plê}
\end{itemize}
\begin{itemize}
\item {Grp. gram.:f.}
\end{itemize}
\begin{itemize}
\item {Utilização:Veloc.}
\end{itemize}
\begin{itemize}
\item {Proveniência:(De \textunderscore triplo\textunderscore )}
\end{itemize}
Velocípede de duas rodas, para três pessôas.
\section{Triplicação}
\begin{itemize}
\item {Grp. gram.:f.}
\end{itemize}
\begin{itemize}
\item {Proveniência:(Lat. \textunderscore triplicatio\textunderscore )}
\end{itemize}
Acto ou effeito de triplicar.
\section{Triplicadamente}
\begin{itemize}
\item {Grp. gram.:adv.}
\end{itemize}
De modo triplicado.
De três maneiras.
Três vezes.
\section{Triplicar}
\begin{itemize}
\item {Grp. gram.:v. t.}
\end{itemize}
\begin{itemize}
\item {Utilização:Ext.}
\end{itemize}
\begin{itemize}
\item {Proveniência:(Lat. \textunderscore triplicare\textunderscore )}
\end{itemize}
Tornar triplo.
Multiplicar.
\section{Triplicata}
\begin{itemize}
\item {Grp. gram.:f.}
\end{itemize}
\begin{itemize}
\item {Proveniência:(Lat. \textunderscore triplicata\textunderscore )}
\end{itemize}
Terceira cópia.
\section{Tríplice}
\begin{itemize}
\item {Grp. gram.:f.}
\end{itemize}
\begin{itemize}
\item {Proveniência:(Do lat. \textunderscore triplex\textunderscore )}
\end{itemize}
O mesmo que \textunderscore triplo\textunderscore .
\section{Triplicidade}
\begin{itemize}
\item {Grp. gram.:f.}
\end{itemize}
\begin{itemize}
\item {Proveniência:(Lat. \textunderscore triplicitas\textunderscore )}
\end{itemize}
Qualidade do que é tríplice.
\section{Triplinervado}
\begin{itemize}
\item {Grp. gram.:adj.}
\end{itemize}
O mesmo que \textunderscore triplinérveo\textunderscore .
\section{Triplinérveo}
\begin{itemize}
\item {Grp. gram.:adj.}
\end{itemize}
\begin{itemize}
\item {Utilização:Bot.}
\end{itemize}
\begin{itemize}
\item {Proveniência:(De \textunderscore triplo\textunderscore  + \textunderscore nervo\textunderscore )}
\end{itemize}
Diz-se da fôlha, que tem três nervuras.
\section{Triplenervoso}
\begin{itemize}
\item {Grp. gram.:adj.}
\end{itemize}
O mesmo que \textunderscore triplinérveo\textunderscore .
\section{Triplito}
\begin{itemize}
\item {Grp. gram.:m.}
\end{itemize}
\begin{itemize}
\item {Utilização:Miner.}
\end{itemize}
\begin{itemize}
\item {Proveniência:(Do gr. \textunderscore triploos\textunderscore )}
\end{itemize}
Phosphato de manganés e de ferro, que se encontra nas vizinhanças de Limoges.
\section{Triplo}
\begin{itemize}
\item {Grp. gram.:adj.}
\end{itemize}
\begin{itemize}
\item {Grp. gram.:M.}
\end{itemize}
\begin{itemize}
\item {Proveniência:(Lat. \textunderscore triplus\textunderscore )}
\end{itemize}
Que contém três vezes uma quantidade.
Que contém três partes.
Multiplicado por três.
Relativo a três.
Coisa triplicada; tresdôbro.
\section{Triploédrico}
\begin{itemize}
\item {Grp. gram.:adj.}
\end{itemize}
\begin{itemize}
\item {Utilização:Miner.}
\end{itemize}
\begin{itemize}
\item {Proveniência:(De \textunderscore triploédro\textunderscore )}
\end{itemize}
Diz-se do crystal, cuja superfície apresenta três ordens de faces, sendo cada uma tríplice da immediata.
\section{Triploédro}
\begin{itemize}
\item {Grp. gram.:m.}
\end{itemize}
\begin{itemize}
\item {Utilização:Miner.}
\end{itemize}
\begin{itemize}
\item {Proveniência:(Do gr. \textunderscore triploos\textunderscore  + \textunderscore edra\textunderscore )}
\end{itemize}
Fórma crystallina, produzida pela combinação de três rhomboédros.
\section{Triplóptero}
\begin{itemize}
\item {Grp. gram.:adj.}
\end{itemize}
\begin{itemize}
\item {Utilização:Zool.}
\end{itemize}
\begin{itemize}
\item {Proveniência:(Do gr. \textunderscore triploos\textunderscore  + \textunderscore pteron\textunderscore )}
\end{itemize}
Que tem asas ou barbatanas tripartidas.
\section{Triplostêmono}
\begin{itemize}
\item {Grp. gram.:adj.}
\end{itemize}
\begin{itemize}
\item {Utilização:Bot.}
\end{itemize}
\begin{itemize}
\item {Proveniência:(Do gr. \textunderscore triploos\textunderscore  + \textunderscore stemon\textunderscore )}
\end{itemize}
Diz-se da flôr, cujos estames são três vezes mais numerosos que as divisões da corolla.
\section{Tripo}
\begin{itemize}
\item {Grp. gram.:m.}
\end{itemize}
\begin{itemize}
\item {Utilização:Prov.}
\end{itemize}
\begin{itemize}
\item {Utilização:beir.}
\end{itemize}
O mesmo que \textunderscore tripa\textunderscore . (Colhido em Sátão)
\section{Tripó}
\begin{itemize}
\item {Grp. gram.:m.}
\end{itemize}
Espécie de tripeça.
(Outra fórma de \textunderscore tripé\textunderscore )
\section{Trípoda}
\begin{itemize}
\item {Grp. gram.:f.}
\end{itemize}
O mesmo que \textunderscore trípode\textunderscore .
\section{Trípode}
\begin{itemize}
\item {Grp. gram.:f.}
\end{itemize}
\begin{itemize}
\item {Utilização:Poét.}
\end{itemize}
\begin{itemize}
\item {Grp. gram.:Adj.}
\end{itemize}
\begin{itemize}
\item {Proveniência:(Lat. \textunderscore tripus\textunderscore , \textunderscore tripodis\textunderscore )}
\end{itemize}
Tripeça, em que a pythonisa pronunciava os seus oráculos.
Antigo vaso de três pés.
Tripeça.
Que tem três pés.
\section{Tripófago}
\begin{itemize}
\item {Grp. gram.:adj.}
\end{itemize}
\begin{itemize}
\item {Utilização:Zool.}
\end{itemize}
\begin{itemize}
\item {Proveniência:(Do gr. \textunderscore thrips\textunderscore  + \textunderscore phagein\textunderscore )}
\end{itemize}
Que se alimenta de insectos e de pequenos vermes.
\section{Tripóleo}
\begin{itemize}
\item {Grp. gram.:adj.}
\end{itemize}
\begin{itemize}
\item {Utilização:Miner.}
\end{itemize}
\begin{itemize}
\item {Proveniência:(De \textunderscore trípoli\textunderscore )}
\end{itemize}
Diz-se dos mineraes, que são ásperos ao tacto, como o trípoli.
\section{Trípoli}
\begin{itemize}
\item {Grp. gram.:m.}
\end{itemize}
\begin{itemize}
\item {Proveniência:(De \textunderscore Trípoli\textunderscore , n. p.)}
\end{itemize}
Terra silicosa, pulverulenta, que se emprega para polir objectos de metal, e que resulta da accumulação das carapaças de algas fósseis, microscópicas.
\section{Tripolino}
\begin{itemize}
\item {Grp. gram.:m.  e  adj.}
\end{itemize}
O mesmo que \textunderscore tripolitano\textunderscore .
\section{Tripolitano}
\begin{itemize}
\item {Grp. gram.:adj.}
\end{itemize}
\begin{itemize}
\item {Grp. gram.:M.}
\end{itemize}
\begin{itemize}
\item {Proveniência:(Lat. \textunderscore tripolitanus\textunderscore )}
\end{itemize}
Relativo a Trípoli.
Habitante de Trípoli.
\section{Tripôndio}
\begin{itemize}
\item {Grp. gram.:m.}
\end{itemize}
\begin{itemize}
\item {Proveniência:(Lat. \textunderscore tripondius\textunderscore )}
\end{itemize}
Antigo pêso romano, equivalente a três libras.
\section{Tripsaco}
\begin{itemize}
\item {Grp. gram.:m.}
\end{itemize}
\begin{itemize}
\item {Proveniência:(Do gr. \textunderscore tripsis\textunderscore )}
\end{itemize}
Gênero de plantas gramíneas.
\section{Tripsina}
\begin{itemize}
\item {Grp. gram.:f.}
\end{itemize}
\begin{itemize}
\item {Proveniência:(Do gr. \textunderscore thrupsis\textunderscore )}
\end{itemize}
Um dos princípios do suco pancreático.
\section{Triptomena}
\begin{itemize}
\item {Grp. gram.:f.}
\end{itemize}
Gênero de plantas mirtáceas.
\section{Triquismo}
\begin{itemize}
\item {Grp. gram.:m.}
\end{itemize}
\begin{itemize}
\item {Proveniência:(Gr. \textunderscore trikhismos\textunderscore )}
\end{itemize}
Fractura filiforme de um osso.
\section{Trisal}
\begin{itemize}
\item {fónica:sal}
\end{itemize}
\begin{itemize}
\item {Grp. gram.:m.}
\end{itemize}
\begin{itemize}
\item {Utilização:Chím.}
\end{itemize}
\begin{itemize}
\item {Proveniência:(De \textunderscore tri...\textunderscore  + \textunderscore sal\textunderscore )}
\end{itemize}
Diz-se o sal que encerra três vezes tanto ácido como base, ou três vezes tanta base como ácido, em relação ao sal neutro correspondente.
\section{Trisannual}
\begin{itemize}
\item {Grp. gram.:adj.}
\end{itemize}
\begin{itemize}
\item {Proveniência:(De \textunderscore tris...\textunderscore  + \textunderscore annual\textunderscore )}
\end{itemize}
Que dura três annos.
Que se realiza de três em três annos.
\section{Trisanual}
\begin{itemize}
\item {Grp. gram.:adj.}
\end{itemize}
\begin{itemize}
\item {Proveniência:(De \textunderscore tris...\textunderscore  + \textunderscore anual\textunderscore )}
\end{itemize}
Que dura três anos.
Que se realiza de três em três anos.
\section{Trisarchia}
\begin{itemize}
\item {fónica:qui}
\end{itemize}
\begin{itemize}
\item {Grp. gram.:f.}
\end{itemize}
\begin{itemize}
\item {Proveniência:(Do gr. \textunderscore treis\textunderscore  + \textunderscore arkhe\textunderscore )}
\end{itemize}
Govêrno constituído por três chefes.
\section{Trisarquia}
\begin{itemize}
\item {Grp. gram.:f.}
\end{itemize}
\begin{itemize}
\item {Proveniência:(Do gr. \textunderscore treis\textunderscore  + \textunderscore arkhe\textunderscore )}
\end{itemize}
Govêrno constituído por três chefes.
\section{Trisavó}
\begin{itemize}
\item {Grp. gram.:f.}
\end{itemize}
\begin{itemize}
\item {Proveniência:(De \textunderscore tris...\textunderscore  + \textunderscore avó\textunderscore )}
\end{itemize}
Mãe do bisavô ou da bisavó.
\section{Trisavô}
\begin{itemize}
\item {Grp. gram.:m.}
\end{itemize}
\begin{itemize}
\item {Proveniência:(De \textunderscore tris...\textunderscore  + \textunderscore avô\textunderscore )}
\end{itemize}
Pai do bisavô, ou da bisavó.
\section{Trisca}
\begin{itemize}
\item {Grp. gram.:f.}
\end{itemize}
\begin{itemize}
\item {Utilização:Pop.}
\end{itemize}
Acto ou effeito de triscar:«\textunderscore aqui vay a trisca dos sospiros\textunderscore »\textunderscore Aulegrafia\textunderscore , 123.
\section{Triscar}
\begin{itemize}
\item {Grp. gram.:v. i.}
\end{itemize}
\begin{itemize}
\item {Proveniência:(Do gót. \textunderscore thriskan\textunderscore )}
\end{itemize}
Fazer bulha.
Armar desordem; altercar.
Intrigar.
\section{Tríscelo}
\begin{itemize}
\item {Grp. gram.:m.}
\end{itemize}
\begin{itemize}
\item {Utilização:Archeol.}
\end{itemize}
\begin{itemize}
\item {Proveniência:(Lat. \textunderscore triscelum\textunderscore )}
\end{itemize}
É uma das variantes da suástica e consta de três linhas curvas, que, divergentes de um centro commum, se enroscam em espiral, formando roseta.
\section{Trisecar}
\begin{itemize}
\item {fónica:se}
\end{itemize}
\begin{itemize}
\item {Grp. gram.:v.}
\end{itemize}
\begin{itemize}
\item {Utilização:t. Mathem.}
\end{itemize}
\begin{itemize}
\item {Proveniência:(De \textunderscore tri...\textunderscore  + lat. \textunderscore secare\textunderscore )}
\end{itemize}
Dividir em três partes, especialmente falando-se do ângulo.
\section{Trisecção}
\begin{itemize}
\item {fónica:sé}
\end{itemize}
\begin{itemize}
\item {Grp. gram.:f.}
\end{itemize}
\begin{itemize}
\item {Proveniência:(De \textunderscore tri...\textunderscore  + \textunderscore secção\textunderscore )}
\end{itemize}
Divisão de uma coisa em três partes.
\section{Trisector}
\begin{itemize}
\item {fónica:sé}
\end{itemize}
\begin{itemize}
\item {Grp. gram.:adj.}
\end{itemize}
\begin{itemize}
\item {Grp. gram.:M.}
\end{itemize}
\begin{itemize}
\item {Proveniência:(De \textunderscore tri...\textunderscore  + \textunderscore sector\textunderscore )}
\end{itemize}
Que corta em três partes.
Instrumento, para dividir ângulos em três partes iguaes.
\section{Trisectriz}
\begin{itemize}
\item {fónica:sé}
\end{itemize}
\begin{itemize}
\item {Grp. gram.:f.  e  adj.}
\end{itemize}
Diz-se da linha, que dá a trisecção do ângulo.
(Fem. de \textunderscore trisector\textunderscore )
\section{Trisecular}
\begin{itemize}
\item {fónica:se}
\end{itemize}
\begin{itemize}
\item {Grp. gram.:adj.}
\end{itemize}
\begin{itemize}
\item {Proveniência:(De \textunderscore tri...\textunderscore  + \textunderscore secular\textunderscore )}
\end{itemize}
Que tem três séculos.
\section{Trisépalo}
\begin{itemize}
\item {fónica:sé}
\end{itemize}
\begin{itemize}
\item {Grp. gram.:adj.}
\end{itemize}
\begin{itemize}
\item {Utilização:Bot.}
\end{itemize}
\begin{itemize}
\item {Proveniência:(De \textunderscore tri...\textunderscore  + \textunderscore sépala\textunderscore )}
\end{itemize}
Que tem três sépalas.
\section{Triseto}
\begin{itemize}
\item {fónica:sé}
\end{itemize}
\begin{itemize}
\item {Grp. gram.:m.}
\end{itemize}
\begin{itemize}
\item {Proveniência:(Do gr. \textunderscore treis\textunderscore  + lat. \textunderscore seta\textunderscore )}
\end{itemize}
Gênero de plantas gramíneas.
\section{Trisilicato}
\begin{itemize}
\item {fónica:si}
\end{itemize}
\begin{itemize}
\item {Grp. gram.:m.}
\end{itemize}
\begin{itemize}
\item {Proveniência:(De \textunderscore tri...\textunderscore  + \textunderscore silicato\textunderscore )}
\end{itemize}
Ácido trisilícico.
\section{Trisilícico}
\begin{itemize}
\item {fónica:si}
\end{itemize}
\begin{itemize}
\item {Grp. gram.:adj.}
\end{itemize}
\begin{itemize}
\item {Proveniência:(De \textunderscore tri...\textunderscore  + \textunderscore silícico\textunderscore )}
\end{itemize}
Diz-se de um hydrato, resultante do ácido silícico pela condensação de três moléculas dêste ácido com eliminação de duas moléculas de água.
\section{Trismegisto}
\begin{itemize}
\item {Grp. gram.:adj.}
\end{itemize}
\begin{itemize}
\item {Proveniência:(Gr. \textunderscore trismegistos\textunderscore )}
\end{itemize}
Três vezes máximo, (título dado pelos Gregos ao Mercúrio egýpcio).
\section{Trismo}
\begin{itemize}
\item {Grp. gram.:m.}
\end{itemize}
\begin{itemize}
\item {Utilização:Med.}
\end{itemize}
\begin{itemize}
\item {Proveniência:(Gr. \textunderscore trismos\textunderscore )}
\end{itemize}
Cerração involuntária da bôca, em resultado da contracção espasmódica dos músculos elevadores da maxilla inferior.
\section{Trisperma}
\begin{itemize}
\item {Grp. gram.:adj.}
\end{itemize}
\begin{itemize}
\item {Proveniência:(Do gr. \textunderscore treis\textunderscore  + \textunderscore sperma\textunderscore )}
\end{itemize}
Que tem três sementes.
\section{Trispermo}
\begin{itemize}
\item {Grp. gram.:adj.}
\end{itemize}
\begin{itemize}
\item {Utilização:Bot.}
\end{itemize}
\begin{itemize}
\item {Proveniência:(Do gr. \textunderscore treis\textunderscore  + \textunderscore sperma\textunderscore )}
\end{itemize}
Que tem três sementes.
\section{Trisplânchnico}
\begin{itemize}
\item {Grp. gram.:adj.}
\end{itemize}
\begin{itemize}
\item {Utilização:Anat.}
\end{itemize}
\begin{itemize}
\item {Proveniência:(Do gr. \textunderscore treis\textunderscore  + \textunderscore splankhnon\textunderscore )}
\end{itemize}
Diz-se de um nervo, que pertence ás três cavidades principaes do corpo, isto é, á cabeça, ao peito e ao abdome.
\section{Trisplâncnico}
\begin{itemize}
\item {Grp. gram.:adj.}
\end{itemize}
\begin{itemize}
\item {Utilização:Anat.}
\end{itemize}
\begin{itemize}
\item {Proveniência:(Do gr. \textunderscore treis\textunderscore  + \textunderscore splankhnon\textunderscore )}
\end{itemize}
Diz-se de um nervo, que pertence ás três cavidades principaes do corpo, isto é, á cabeça, ao peito e ao abdome.
\section{Trissal}
\begin{itemize}
\item {Grp. gram.:m.}
\end{itemize}
\begin{itemize}
\item {Utilização:Chím.}
\end{itemize}
\begin{itemize}
\item {Proveniência:(De \textunderscore tri...\textunderscore  + \textunderscore sal\textunderscore )}
\end{itemize}
Diz-se o sal que encerra três vezes tanto ácido como base, ou três vezes tanta base como ácido, em relação ao sal neutro correspondente.
\section{Trissar}
\begin{itemize}
\item {Grp. gram.:v. i.}
\end{itemize}
\begin{itemize}
\item {Proveniência:(Lat. \textunderscore trissare\textunderscore )}
\end{itemize}
Diz-se do canto ou voz da calhandra e da andorinha:«\textunderscore ...a andorinha nos telhados trissa.\textunderscore »C. Neto, \textunderscore Saldunes\textunderscore .
\section{Trissecar}
\begin{itemize}
\item {Grp. gram.:v.}
\end{itemize}
\begin{itemize}
\item {Utilização:t. Mathem.}
\end{itemize}
\begin{itemize}
\item {Proveniência:(De \textunderscore tri...\textunderscore  + lat. \textunderscore secare\textunderscore )}
\end{itemize}
Dividir em três partes, especialmente falando-se do ângulo.
\section{Trissecção}
\begin{itemize}
\item {Grp. gram.:f.}
\end{itemize}
\begin{itemize}
\item {Proveniência:(De \textunderscore tri...\textunderscore  + \textunderscore secção\textunderscore )}
\end{itemize}
Divisão de uma coisa em três partes.
\section{Trissector}
\begin{itemize}
\item {Grp. gram.:adj.}
\end{itemize}
\begin{itemize}
\item {Grp. gram.:M.}
\end{itemize}
\begin{itemize}
\item {Proveniência:(De \textunderscore tri...\textunderscore  + \textunderscore sector\textunderscore )}
\end{itemize}
Que corta em três partes.
Instrumento, para dividir ângulos em três partes iguaes.
\section{Trissectriz}
\begin{itemize}
\item {Grp. gram.:f.  e  adj.}
\end{itemize}
Diz-se da linha, que dá a trissecção do ângulo.
(Fem. de \textunderscore trisector\textunderscore )
\section{Trissecular}
\begin{itemize}
\item {Grp. gram.:adj.}
\end{itemize}
\begin{itemize}
\item {Proveniência:(De \textunderscore tri...\textunderscore  + \textunderscore secular\textunderscore )}
\end{itemize}
Que tem três séculos.
\section{Trissépalo}
\begin{itemize}
\item {Grp. gram.:adj.}
\end{itemize}
\begin{itemize}
\item {Utilização:Bot.}
\end{itemize}
\begin{itemize}
\item {Proveniência:(De \textunderscore tri...\textunderscore  + \textunderscore sépala\textunderscore )}
\end{itemize}
Que tem três sépalas.
\section{Trisseto}
\begin{itemize}
\item {Grp. gram.:m.}
\end{itemize}
\begin{itemize}
\item {Proveniência:(Do gr. \textunderscore treis\textunderscore  + lat. \textunderscore seta\textunderscore )}
\end{itemize}
Gênero de plantas gramíneas.
\section{Trissilicato}
\begin{itemize}
\item {Grp. gram.:m.}
\end{itemize}
\begin{itemize}
\item {Proveniência:(De \textunderscore tri...\textunderscore  + \textunderscore silicato\textunderscore )}
\end{itemize}
Ácido trissilícico.
\section{Trissilícico}
\begin{itemize}
\item {Grp. gram.:adj.}
\end{itemize}
\begin{itemize}
\item {Proveniência:(De \textunderscore tri...\textunderscore  + \textunderscore silícico\textunderscore )}
\end{itemize}
Diz-se de um hidrato, resultante do ácido silícico pela condensação de três moléculas dêste ácido com eliminação de duas moléculas de água.
\section{Trisso}
\begin{itemize}
\item {Grp. gram.:m.}
\end{itemize}
Acto de trissar.
\section{Tristáchya}
\begin{itemize}
\item {fónica:qui}
\end{itemize}
\begin{itemize}
\item {Grp. gram.:f.}
\end{itemize}
\begin{itemize}
\item {Proveniência:(Do gr. \textunderscore treis\textunderscore  + \textunderscore stakhus\textunderscore )}
\end{itemize}
Gênero de plantas gramíneas.
\section{Tristagma}
\begin{itemize}
\item {Grp. gram.:m.}
\end{itemize}
Gênero de plantas liliáceas.
\section{Tristaminífero}
\begin{itemize}
\item {Grp. gram.:adj.}
\end{itemize}
\begin{itemize}
\item {Utilização:Bot.}
\end{itemize}
\begin{itemize}
\item {Proveniência:(De \textunderscore tri...\textunderscore  + \textunderscore estaminífero\textunderscore )}
\end{itemize}
Que tem três estames.
\section{Tristânia}
\begin{itemize}
\item {Grp. gram.:f.}
\end{itemize}
\begin{itemize}
\item {Proveniência:(De \textunderscore Tristan\textunderscore , n. p.)}
\end{itemize}
Gênero de plantas myrtáceas.
\section{Tristáquia}
\begin{itemize}
\item {Grp. gram.:f.}
\end{itemize}
\begin{itemize}
\item {Proveniência:(Do gr. \textunderscore treis\textunderscore  + \textunderscore stakhus\textunderscore )}
\end{itemize}
Gênero de plantas gramíneas.
\section{Trito}
\begin{itemize}
\item {Grp. gram.:adj.}
\end{itemize}
\begin{itemize}
\item {Utilização:Des.}
\end{itemize}
\begin{itemize}
\item {Proveniência:(Lat. \textunderscore tritus\textunderscore )}
\end{itemize}
O mesmo que \textunderscore triturado\textunderscore .
\section{Tritongo}
\begin{itemize}
\item {Grp. gram.:m.}
\end{itemize}
\begin{itemize}
\item {Utilização:Gram.}
\end{itemize}
\begin{itemize}
\item {Proveniência:(Do gr. \textunderscore treis\textunderscore  + \textunderscore phthongos\textunderscore )}
\end{itemize}
Grupo de três vogaes, que se pronuncía com uma só emissão de voz.
\section{Tritoniano}
\begin{itemize}
\item {Grp. gram.:adj.}
\end{itemize}
\begin{itemize}
\item {Utilização:Geol.}
\end{itemize}
\begin{itemize}
\item {Proveniência:(De \textunderscore tritão\textunderscore )}
\end{itemize}
Diz-se do terreno, em que há restos fósseis de animaes marinhos.
\section{Trítono}
\begin{itemize}
\item {Grp. gram.:m.}
\end{itemize}
\begin{itemize}
\item {Utilização:Mús.}
\end{itemize}
\begin{itemize}
\item {Proveniência:(Gr. \textunderscore tritonos\textunderscore )}
\end{itemize}
Intervallo dissonante composto de três tons.
\section{Trituberculado}
\begin{itemize}
\item {Grp. gram.:adj.}
\end{itemize}
\begin{itemize}
\item {Proveniência:(De \textunderscore tri...\textunderscore  + \textunderscore tuberculado\textunderscore )}
\end{itemize}
Que tem três tubérculos.
\section{Tritura}
\begin{itemize}
\item {Grp. gram.:f.}
\end{itemize}
\begin{itemize}
\item {Proveniência:(Lat. \textunderscore tritura\textunderscore )}
\end{itemize}
O mesmo que \textunderscore trituração\textunderscore .
\section{Trituração}
\begin{itemize}
\item {Grp. gram.:f.}
\end{itemize}
\begin{itemize}
\item {Proveniência:(Do lat. \textunderscore trituratio\textunderscore )}
\end{itemize}
Acto ou effeito de triturar.
\section{Triturado}
\begin{itemize}
\item {Grp. gram.:adj.}
\end{itemize}
\begin{itemize}
\item {Utilização:Fig.}
\end{itemize}
\begin{itemize}
\item {Proveniência:(De \textunderscore triturar\textunderscore )}
\end{itemize}
Que se triturou.
Moído.
Magoado; atormentado.
\section{Triturador}
\begin{itemize}
\item {Grp. gram.:m.}
\end{itemize}
\begin{itemize}
\item {Proveniência:(De \textunderscore triturar\textunderscore )}
\end{itemize}
Máquina, para moêr os ingredientes da pólvora.
\section{Trituramento}
\begin{itemize}
\item {Grp. gram.:m.}
\end{itemize}
O mesmo que \textunderscore trituração\textunderscore .
\section{Triturar}
\begin{itemize}
\item {Grp. gram.:v. t.}
\end{itemize}
\begin{itemize}
\item {Utilização:Ext.}
\end{itemize}
\begin{itemize}
\item {Utilização:Fig.}
\end{itemize}
\begin{itemize}
\item {Proveniência:(Lat. \textunderscore triturare\textunderscore )}
\end{itemize}
Reduzir a pequenas parcellas.
Reduzir a pó.
Converter em massa.
Bater, sovar.
Atormentar.
\section{Triturável}
\begin{itemize}
\item {Grp. gram.:adj.}
\end{itemize}
Que se póde triturar.
\section{Triumfeminato}
\begin{itemize}
\item {Grp. gram.:m.}
\end{itemize}
\begin{itemize}
\item {Proveniência:(Do lat. \textunderscore tres\textunderscore  + \textunderscore femina\textunderscore , á semelhança de \textunderscore triumvirato\textunderscore , \textunderscore duumvirato\textunderscore , etc.)}
\end{itemize}
Domínio, exercido por três mulheres. Cf. Castilho, \textunderscore Sabichonas\textunderscore , 130.
\section{Triumphador}
\begin{itemize}
\item {Grp. gram.:m.  e  adj.}
\end{itemize}
\begin{itemize}
\item {Proveniência:(Do lat. \textunderscore triumphator\textunderscore )}
\end{itemize}
O que triumpha.
\section{Triumphal}
\begin{itemize}
\item {Grp. gram.:adj.}
\end{itemize}
\begin{itemize}
\item {Proveniência:(Lat. \textunderscore triumphalis\textunderscore )}
\end{itemize}
Relativo a triumpho.
\section{Triumphante}
\begin{itemize}
\item {Grp. gram.:adj.}
\end{itemize}
\begin{itemize}
\item {Proveniência:(Lat. \textunderscore triumphans\textunderscore )}
\end{itemize}
Que triumpha.
Ostentoso.
Radiante de alegria.
\section{Triumphantemente}
\begin{itemize}
\item {Grp. gram.:adv.}
\end{itemize}
De modo triumphante; gloriosamente.
Com bizarria, com altivez.
\section{Triumphar}
\begin{itemize}
\item {Grp. gram.:v. i.}
\end{itemize}
\begin{itemize}
\item {Grp. gram.:V. t.}
\end{itemize}
\begin{itemize}
\item {Utilização:Des.}
\end{itemize}
\begin{itemize}
\item {Proveniência:(Lat. \textunderscore triumphare\textunderscore )}
\end{itemize}
Conseguir triumpho; alcançar victória.
Levar vantagem.
Vencer qualquer resistência.
Estar muito alegre.
Vencer.
Encher de triumphos, tornar triumphante: \textunderscore mulher que triumphou a vida com as magias da formosura\textunderscore . Camillo, \textunderscore Caveira\textunderscore , I, XIII.
\section{Triumpho}
\begin{itemize}
\item {Grp. gram.:m.}
\end{itemize}
\begin{itemize}
\item {Proveniência:(Lat. \textunderscore triumphus\textunderscore )}
\end{itemize}
Entrada solenne e pomposa dos generaes victoriosos na antiga Roma.
Acto ou effeito de triumphar.
Victória notável.
Êxito brilhante.
Satisfação plena.
Grande alegria.
Esplendor.
Dominação das paixões.
Superioridade.
Manifestação de aplausos, com grande ruído; acclamação.
Espécie de jôgo de cartas.
Ornato central da mesa de banquete. Cf. Rebello, \textunderscore Contos e Lendas\textunderscore , III, 40.
\section{Triumvirado}
\begin{itemize}
\item {Grp. gram.:m.}
\end{itemize}
\begin{itemize}
\item {Utilização:Ext.}
\end{itemize}
\begin{itemize}
\item {Proveniência:(Lat. \textunderscore triumviratus\textunderscore )}
\end{itemize}
Magistratura dos triúmviros.
Associação de três cidadãos, que reúnem em si toda a autoridade.
Govêrno de três indivíduos.
Uma das classes da Ordem de Malta.
\section{Triumviral}
\begin{itemize}
\item {Grp. gram.:adj.}
\end{itemize}
\begin{itemize}
\item {Proveniência:(Lat. \textunderscore triumviralis\textunderscore )}
\end{itemize}
Relativo a triúmviro.
\section{Triumvirato}
\begin{itemize}
\item {Grp. gram.:m.}
\end{itemize}
O mesmo que \textunderscore triumvirado\textunderscore .
\section{Triúmviro}
\begin{itemize}
\item {Grp. gram.:m.}
\end{itemize}
\begin{itemize}
\item {Proveniência:(Lat. \textunderscore triumvir\textunderscore )}
\end{itemize}
Magistrado romano, encarregado de uma parte da administração pública, juntamente com dois collegas.
Membro de qualquer triumvirado.
\section{Triunfador}
\begin{itemize}
\item {Grp. gram.:m.  e  adj.}
\end{itemize}
\begin{itemize}
\item {Proveniência:(Do lat. \textunderscore triumphator\textunderscore )}
\end{itemize}
O que triunfa.
\section{Triunfal}
\begin{itemize}
\item {Grp. gram.:adj.}
\end{itemize}
\begin{itemize}
\item {Proveniência:(Lat. \textunderscore triumphalis\textunderscore )}
\end{itemize}
Relativo a triunfo.
\section{Triunfante}
\begin{itemize}
\item {Grp. gram.:adj.}
\end{itemize}
\begin{itemize}
\item {Proveniência:(Lat. \textunderscore triumphans\textunderscore )}
\end{itemize}
Que triunfa.
Ostentoso.
Radiante de alegria.
\section{Triunfantemente}
\begin{itemize}
\item {Grp. gram.:adv.}
\end{itemize}
De modo triunfante; gloriosamente.
Com bizarria, com altivez.
\section{Triunfar}
\begin{itemize}
\item {Grp. gram.:v. i.}
\end{itemize}
\begin{itemize}
\item {Grp. gram.:V. t.}
\end{itemize}
\begin{itemize}
\item {Utilização:Des.}
\end{itemize}
\begin{itemize}
\item {Proveniência:(Lat. \textunderscore triumphare\textunderscore )}
\end{itemize}
Conseguir triunfo; alcançar victória.
Levar vantagem.
Vencer qualquer resistência.
Estar muito alegre.
Vencer.
Encher de triunfos, tornar triunfante: \textunderscore mulher que triunfou a vida com as magias da formosura\textunderscore . Camillo, \textunderscore Caveira\textunderscore , I, XIII.
\section{Triunfeminato}
\begin{itemize}
\item {Grp. gram.:m.}
\end{itemize}
\begin{itemize}
\item {Proveniência:(Do lat. \textunderscore tres\textunderscore  + \textunderscore femina\textunderscore , á semelhança de \textunderscore triumvirato\textunderscore , \textunderscore duumvirato\textunderscore , etc.)}
\end{itemize}
Domínio, exercido por três mulheres. Cf. Castilho, \textunderscore Sabichonas\textunderscore , 130.
\section{Triunfo}
\begin{itemize}
\item {Grp. gram.:m.}
\end{itemize}
\begin{itemize}
\item {Proveniência:(Lat. \textunderscore triumphus\textunderscore )}
\end{itemize}
Entrada solenne e pomposa dos generaes victoriosos na antiga Roma.
Acto ou efeito de triunfar.
Victória notável.
Êxito brilhante.
Satisfação plena.
Grande alegria.
Esplendor.
Dominação das paixões.
Superioridade.
Manifestação de aplausos, com grande ruído; acclamação.
Espécie de jôgo de cartas.
Ornato central da mesa de banquete. Cf. Rebello, \textunderscore Contos e Lendas\textunderscore , III, 40.
\section{Triunvirado}
\begin{itemize}
\item {Grp. gram.:m.}
\end{itemize}
\begin{itemize}
\item {Utilização:Ext.}
\end{itemize}
\begin{itemize}
\item {Proveniência:(Lat. \textunderscore triumviratus\textunderscore )}
\end{itemize}
Magistratura dos triúnviros.
Associação de três cidadãos, que reúnem em si toda a autoridade.
Govêrno de três indivíduos.
Uma das classes da Ordem de Malta.
\section{Triunviral}
\begin{itemize}
\item {Grp. gram.:adj.}
\end{itemize}
\begin{itemize}
\item {Proveniência:(Lat. \textunderscore triumviralis\textunderscore )}
\end{itemize}
Relativo a triúnviro.
\section{Triunvirato}
\begin{itemize}
\item {Grp. gram.:m.}
\end{itemize}
O mesmo que \textunderscore triunvirado\textunderscore .
\section{Triúnviro}
\begin{itemize}
\item {Grp. gram.:m.}
\end{itemize}
\begin{itemize}
\item {Proveniência:(Lat. \textunderscore triumvir\textunderscore )}
\end{itemize}
Magistrado romano, encarregado de uma parte da administração pública, juntamente com dois colegas.
Membro de qualquer triunvirado.
\section{Trivalência}
\begin{itemize}
\item {Grp. gram.:f.}
\end{itemize}
Propriedade daquillo que é trivalente.
\section{Trocaico}
\begin{itemize}
\item {Grp. gram.:adj.}
\end{itemize}
\begin{itemize}
\item {Grp. gram.:M.}
\end{itemize}
\begin{itemize}
\item {Proveniência:(Lat. \textunderscore trochaicus\textunderscore )}
\end{itemize}
Composto de troqueus.
Verso trocaico.
\section{Trocânter}
\begin{itemize}
\item {Grp. gram.:m.}
\end{itemize}
\begin{itemize}
\item {Utilização:Anat.}
\end{itemize}
\begin{itemize}
\item {Proveniência:(Gr. \textunderscore trokhanter\textunderscore )}
\end{itemize}
Cada uma das duas tuberosidades na parte superior do fêmur.
\section{Trocanteriano}
\begin{itemize}
\item {Grp. gram.:adj.}
\end{itemize}
\begin{itemize}
\item {Proveniência:(De \textunderscore trocâncher\textunderscore )}
\end{itemize}
Diz-se das apófises, que operam movimentos de rotação.
\section{Trocantiniano}
\begin{itemize}
\item {Grp. gram.:adj.}
\end{itemize}
Relativo ao \textunderscore trocantino\textunderscore .
\section{Trocantino}
\begin{itemize}
\item {Grp. gram.:m.}
\end{itemize}
Tuberosidade menor da parte superior do fêmur.
(Por \textunderscore trochanterino\textunderscore , de \textunderscore trochânter\textunderscore )
\section{Trocate}
\begin{itemize}
\item {Grp. gram.:m.}
\end{itemize}
(V.trocarte)
\section{Troca-tintas}
\begin{itemize}
\item {Grp. gram.:m.}
\end{itemize}
\begin{itemize}
\item {Utilização:Ext.}
\end{itemize}
Pintor reles.
Pandilha, bisbórria; trapalhão.
\section{Trocavel}
\begin{itemize}
\item {Grp. gram.:adj.}
\end{itemize}
Que se póde trocar.
\section{Trocaz}
\begin{itemize}
\item {Grp. gram.:m.  e  adj.}
\end{itemize}
O mesmo que \textunderscore torcaz\textunderscore , (\textunderscore columba trocaz\textunderscore , Keinch.). Cf. B. Pato, \textunderscore Liv. do Monte\textunderscore ; Schmitz, \textunderscore Die Vogel Madeira's\textunderscore .
\section{Trocázio}
\begin{itemize}
\item {Grp. gram.:m.}
\end{itemize}
\begin{itemize}
\item {Utilização:Mad}
\end{itemize}
O mesmo que \textunderscore trocaz\textunderscore . Cf. Schmitz, \textunderscore Die Vogel Madeira's\textunderscore .
\section{Trocha}
\begin{itemize}
\item {Grp. gram.:f.}
\end{itemize}
\begin{itemize}
\item {Utilização:Ant.}
\end{itemize}
Caminho sinuoso.
Atalho, que por desvios conduz a um lugar.
(Por \textunderscore torcha\textunderscore , do lat. \textunderscore torsa\textunderscore )
\section{Trocha}
\begin{itemize}
\item {Grp. gram.:f.}
\end{itemize}
Linha de fortificação militar, nas Antilhas.
(Or. caraíba?)
\section{Trochada}
\begin{itemize}
\item {Grp. gram.:f.}
\end{itemize}
\begin{itemize}
\item {Utilização:Prov.}
\end{itemize}
Pancada com trôcho.
Qualquer pancada.
O mesmo que \textunderscore bofetada\textunderscore .
\section{Trochado}
\begin{itemize}
\item {Grp. gram.:m.}
\end{itemize}
Antigo lavor, em sêda ou tecidos.
\section{Trochaico}
\begin{itemize}
\item {fónica:cai}
\end{itemize}
\begin{itemize}
\item {Grp. gram.:adj.}
\end{itemize}
\begin{itemize}
\item {Grp. gram.:M.}
\end{itemize}
\begin{itemize}
\item {Proveniência:(Lat. \textunderscore trochaicus\textunderscore )}
\end{itemize}
Composto de trocheus.
Verso trochaico.
\section{Trochânter}
\begin{itemize}
\item {fónica:can}
\end{itemize}
\begin{itemize}
\item {Grp. gram.:m.}
\end{itemize}
\begin{itemize}
\item {Utilização:Anat.}
\end{itemize}
\begin{itemize}
\item {Proveniência:(Gr. \textunderscore trokhanter\textunderscore )}
\end{itemize}
Cada uma das duas tuberosidades na parte superior do fêmur.
\section{Trochanteriano}
\begin{itemize}
\item {fónica:can}
\end{itemize}
\begin{itemize}
\item {Grp. gram.:adj.}
\end{itemize}
\begin{itemize}
\item {Proveniência:(De \textunderscore trochâncher\textunderscore )}
\end{itemize}
Diz-se das apóphyses, que operam movimentos de rotação.
\section{Trochantiniano}
\begin{itemize}
\item {fónica:can}
\end{itemize}
\begin{itemize}
\item {Grp. gram.:adj.}
\end{itemize}
Relativo ao \textunderscore trochantino\textunderscore .
\section{Trochantino}
\begin{itemize}
\item {fónica:can}
\end{itemize}
\begin{itemize}
\item {Grp. gram.:m.}
\end{itemize}
Tuberosidade menor da parte superior do fêmur.
(Por \textunderscore trochanterino\textunderscore , de \textunderscore trochânter\textunderscore )
\section{Trochar}
\begin{itemize}
\item {Grp. gram.:v. t.}
\end{itemize}
Torcer para reforçar (cano de espingarda).
(Por \textunderscore torchar\textunderscore . Cp. provn. \textunderscore torcha\textunderscore , do lat. \textunderscore tortus\textunderscore )
\section{Trocheu}
\begin{itemize}
\item {fónica:queu}
\end{itemize}
\begin{itemize}
\item {Grp. gram.:m.}
\end{itemize}
\begin{itemize}
\item {Proveniência:(Lat. \textunderscore trochaeus\textunderscore )}
\end{itemize}
Pé de verso grego ou latino, composto de uma sýllaba longa e outra breve.
\section{Tróchilo}
\begin{itemize}
\item {fónica:qui}
\end{itemize}
\begin{itemize}
\item {Grp. gram.:m.}
\end{itemize}
\begin{itemize}
\item {Proveniência:(Lat. \textunderscore trochilus\textunderscore )}
\end{itemize}
Moldura côncava.
\section{Trochiniano}
\begin{itemize}
\item {fónica:qui}
\end{itemize}
\begin{itemize}
\item {Grp. gram.:adj.}
\end{itemize}
Relativo ao trochino.
\section{Trochino}
\begin{itemize}
\item {fónica:qui}
\end{itemize}
\begin{itemize}
\item {Grp. gram.:m.}
\end{itemize}
\begin{itemize}
\item {Utilização:Anat.}
\end{itemize}
\begin{itemize}
\item {Proveniência:(Do gr. \textunderscore trokhos\textunderscore , roda)}
\end{itemize}
Tuberosidade menor da extremidade superior do húmero.
\section{Tróchio}
\begin{itemize}
\item {fónica:qui}
\end{itemize}
\begin{itemize}
\item {Grp. gram.:m.}
\end{itemize}
\begin{itemize}
\item {Proveniência:(Do gr. \textunderscore trokhos\textunderscore )}
\end{itemize}
Gênero de molluscos gasterópodes.
\section{Trochisco}
\begin{itemize}
\item {fónica:quis}
\end{itemize}
\begin{itemize}
\item {Grp. gram.:m.}
\end{itemize}
\begin{itemize}
\item {Proveniência:(Lat. \textunderscore trochiscus\textunderscore )}
\end{itemize}
Entre os Gregos, bolo de sêmeas finas, amassadas com vinho branco. Cf. Castilho, \textunderscore Fastos\textunderscore , 477.
O mesmo que \textunderscore trocisco\textunderscore ^1.
\section{Trochitér}
\begin{itemize}
\item {fónica:qui}
\end{itemize}
\begin{itemize}
\item {Grp. gram.:m.}
\end{itemize}
\begin{itemize}
\item {Utilização:Anat.}
\end{itemize}
\begin{itemize}
\item {Proveniência:(Lat. scient. \textunderscore trochiter\textunderscore , do gr. \textunderscore trokhos\textunderscore )}
\end{itemize}
A maior tuberosidade da extremidade superior do húmero.
\section{Trochiteriano}
\begin{itemize}
\item {fónica:qui}
\end{itemize}
\begin{itemize}
\item {Grp. gram.:adj.}
\end{itemize}
Relativo ao trochitér.
\section{Tróchlea}
\begin{itemize}
\item {Grp. gram.:f.}
\end{itemize}
\begin{itemize}
\item {Proveniência:(Lat. \textunderscore trochlea\textunderscore )}
\end{itemize}
Proeminência articular da extremidade inferior do húmero.
\section{Trochleador}
\begin{itemize}
\item {Grp. gram.:adj.}
\end{itemize}
\begin{itemize}
\item {Utilização:Anat.}
\end{itemize}
\begin{itemize}
\item {Proveniência:(De \textunderscore tróchlea\textunderscore )}
\end{itemize}
Diz-se de um dos músculos do ôlho.
\section{Trochlear}
\begin{itemize}
\item {Grp. gram.:adj.}
\end{itemize}
Relativo á tróchlea.
\section{Trôcho}
\begin{itemize}
\item {Grp. gram.:m.}
\end{itemize}
Pau tôsco; bordão.
(Cp. \textunderscore troncho\textunderscore )
\section{Trochobolista}
\begin{itemize}
\item {fónica:co}
\end{itemize}
\begin{itemize}
\item {Grp. gram.:f.}
\end{itemize}
\begin{itemize}
\item {Proveniência:(Do lat. \textunderscore trochus\textunderscore  + \textunderscore balista\textunderscore )}
\end{itemize}
Balista, que era conduzida sôbre rodas.
\section{Trochocarpo}
\begin{itemize}
\item {fónica:co}
\end{itemize}
\begin{itemize}
\item {Grp. gram.:m.}
\end{itemize}
\begin{itemize}
\item {Proveniência:(Do gr. \textunderscore trokhus\textunderscore  + \textunderscore karpos\textunderscore )}
\end{itemize}
Gênero de plantas epacrídeas.
\section{Trochocephalia}
\begin{itemize}
\item {fónica:co}
\end{itemize}
\begin{itemize}
\item {Grp. gram.:f.}
\end{itemize}
Estado ou qualidade de trococéphalo.
\section{Tróclea}
\begin{itemize}
\item {Grp. gram.:f.}
\end{itemize}
\begin{itemize}
\item {Proveniência:(Lat. \textunderscore trochlea\textunderscore )}
\end{itemize}
Proeminência articular da extremidade inferior do húmero.
\section{Trocleador}
\begin{itemize}
\item {Grp. gram.:adj.}
\end{itemize}
\begin{itemize}
\item {Utilização:Anat.}
\end{itemize}
\begin{itemize}
\item {Proveniência:(De \textunderscore tróclea\textunderscore )}
\end{itemize}
Diz-se de um dos músculos do ôlho.
\section{Troclear}
\begin{itemize}
\item {Grp. gram.:adj.}
\end{itemize}
Relativo á tróclea.
\section{Trocobolista}
\begin{itemize}
\item {Grp. gram.:f.}
\end{itemize}
\begin{itemize}
\item {Proveniência:(Do lat. \textunderscore trochus\textunderscore  + \textunderscore balista\textunderscore )}
\end{itemize}
Balista, que era conduzida sôbre rodas.
\section{Trococarpo}
\begin{itemize}
\item {Grp. gram.:m.}
\end{itemize}
\begin{itemize}
\item {Proveniência:(Do gr. \textunderscore trokhus\textunderscore  + \textunderscore karpos\textunderscore )}
\end{itemize}
Gênero de plantas epacrídeas.
\section{Trococefalia}
\begin{itemize}
\item {Grp. gram.:f.}
\end{itemize}
Estado ou qualidade de trococéfalo.
\section{Troglodítico}
\begin{itemize}
\item {Grp. gram.:adj.}
\end{itemize}
\begin{itemize}
\item {Proveniência:(Lat. \textunderscore troglodyticus\textunderscore )}
\end{itemize}
Relativo a troglodita.
\section{Troglodýtico}
\begin{itemize}
\item {Grp. gram.:adj.}
\end{itemize}
\begin{itemize}
\item {Proveniência:(Lat. \textunderscore troglodyticus\textunderscore )}
\end{itemize}
Relativo a troglodyta.
\section{Trognófido}
\begin{itemize}
\item {Grp. gram.:m.}
\end{itemize}
Gênero de reptis.
\section{Trognóphido}
\begin{itemize}
\item {Grp. gram.:m.}
\end{itemize}
Gênero de reptis.
\section{Trogosito}
\begin{itemize}
\item {fónica:si}
\end{itemize}
\begin{itemize}
\item {Grp. gram.:m.}
\end{itemize}
\begin{itemize}
\item {Proveniência:(Do gr. \textunderscore trogein\textunderscore  + \textunderscore sitos\textunderscore )}
\end{itemize}
Gênero de insectos coleópteros heterómeros.
\section{Trogossito}
\begin{itemize}
\item {Grp. gram.:m.}
\end{itemize}
\begin{itemize}
\item {Proveniência:(Do gr. \textunderscore trogein\textunderscore  + \textunderscore sitos\textunderscore )}
\end{itemize}
Gênero de insectos coleópteros heterómeros.
\section{Trógulo}
\begin{itemize}
\item {Grp. gram.:m.}
\end{itemize}
Gênero de arachnídeos.
\section{Tróia}
\begin{itemize}
\item {Grp. gram.:f.}
\end{itemize}
\begin{itemize}
\item {Proveniência:(De \textunderscore Tróia\textunderscore , n. p.)}
\end{itemize}
Jôgo antigo, simulando combate.
\section{Troiano}
\begin{itemize}
\item {Grp. gram.:adj.}
\end{itemize}
\begin{itemize}
\item {Grp. gram.:M.}
\end{itemize}
\begin{itemize}
\item {Proveniência:(Lat. \textunderscore trojanus\textunderscore )}
\end{itemize}
Relativo a Tróia.
O que era natural de Tróia.
\section{Tróico}
\begin{itemize}
\item {Grp. gram.:adj.}
\end{itemize}
\begin{itemize}
\item {Proveniência:(Lat. \textunderscore troicus\textunderscore )}
\end{itemize}
Troiano. Cf. Filinto. XV, 103.
\section{Troile}
\begin{itemize}
\item {Grp. gram.:m.}
\end{itemize}
Ave palmípede.
\section{Troixa}
\begin{itemize}
\item {Grp. gram.:f.}
\end{itemize}
\begin{itemize}
\item {Utilização:Prov.}
\end{itemize}
\begin{itemize}
\item {Utilização:beir.}
\end{itemize}
\begin{itemize}
\item {Grp. gram.:M.}
\end{itemize}
Fardo de roupa.
Grande pacote.
Mulhér desajeitada e mal procedida. (Colhido na Guarda)
Trampolineiro, pulha. Cf. Camillo, \textunderscore Hist. e Sentim.\textunderscore , 164.
\section{Troixada}
\begin{itemize}
\item {Grp. gram.:f.}
\end{itemize}
Grande troixa.
\section{Troixel}
\begin{itemize}
\item {Grp. gram.:m.}
\end{itemize}
\begin{itemize}
\item {Utilização:Ant.}
\end{itemize}
O mesmo que \textunderscore troixa\textunderscore .
\section{Troixelo}
\begin{itemize}
\item {Grp. gram.:m.}
\end{itemize}
(V.troixel)
\section{Troixe-moixe, a}
\begin{itemize}
\item {Grp. gram.:loc. adv.}
\end{itemize}
\begin{itemize}
\item {Proveniência:(De \textunderscore troixa\textunderscore )}
\end{itemize}
Em desordem; atabalhoadamente.
\section{Trol}
\begin{itemize}
\item {Grp. gram.:m.}
\end{itemize}
\begin{itemize}
\item {Utilização:Bras. do Rio}
\end{itemize}
\begin{itemize}
\item {Proveniência:(Do ingl. \textunderscore trolley\textunderscore ?)}
\end{itemize}
Carruagem, trem.
\section{Troles-boles}
\begin{itemize}
\item {Grp. gram.:m.}
\end{itemize}
\begin{itemize}
\item {Utilização:Prov.}
\end{itemize}
\begin{itemize}
\item {Utilização:trasm.}
\end{itemize}
\begin{itemize}
\item {Utilização:Fam.}
\end{itemize}
Indivíduo pouco sério e pouco constante, que ora diz uma coisa, ora outra, sem saber no que há de ficar; troca-tintas.
\section{Trolha}
\begin{itemize}
\item {fónica:trô}
\end{itemize}
\begin{itemize}
\item {Grp. gram.:f.}
\end{itemize}
\begin{itemize}
\item {Grp. gram.:M.}
\end{itemize}
\begin{itemize}
\item {Utilização:Pop.}
\end{itemize}
\begin{itemize}
\item {Proveniência:(Do lat. \textunderscore trullea\textunderscore )}
\end{itemize}
Espécie de pá, em que o pedreiro tem a cal de que se vai servindo.
Pedreiro ordinário ou servente de pedreiro.
Homem ordinário; maltrapilho.
\section{Trólha}
\begin{itemize}
\item {Grp. gram.:f.}
\end{itemize}
\begin{itemize}
\item {Utilização:Prov.}
\end{itemize}
\begin{itemize}
\item {Utilização:trasm.}
\end{itemize}
Canudo de lata, afunilado, por onde se mete carne em tripas delgadas, para fazer chouriças ou salsichas.
\section{Trolho}
\begin{itemize}
\item {fónica:trô}
\end{itemize}
\begin{itemize}
\item {Grp. gram.:m.}
\end{itemize}
\begin{itemize}
\item {Proveniência:(Do lat. \textunderscore trulleum\textunderscore )}
\end{itemize}
Antiga medida de capacidade, correspondente a meio celamim.
\section{Trolho}
\begin{itemize}
\item {fónica:trô}
\end{itemize}
\begin{itemize}
\item {Grp. gram.:m.}
\end{itemize}
\begin{itemize}
\item {Utilização:Fam.}
\end{itemize}
\begin{itemize}
\item {Utilização:Prov.}
\end{itemize}
\begin{itemize}
\item {Utilização:minh.}
\end{itemize}
\begin{itemize}
\item {Utilização:Prov.}
\end{itemize}
\begin{itemize}
\item {Utilização:minh.}
\end{itemize}
\begin{itemize}
\item {Utilização:Prov.}
\end{itemize}
\begin{itemize}
\item {Utilização:minh.}
\end{itemize}
\begin{itemize}
\item {Grp. gram.:Pl.}
\end{itemize}
\begin{itemize}
\item {Utilização:Prov.}
\end{itemize}
\begin{itemize}
\item {Utilização:minh.}
\end{itemize}
Homem gordo e baixo.
Carrapito no alto da cabeça.
Obra mal feita.
O mesmo que \textunderscore excremento\textunderscore .
Fios, que, repuxados com agulha ou gancho, resaem do corpo das tralhas, no tear.
(Cp. \textunderscore trólha\textunderscore )
\section{Trólio}
\begin{itemize}
\item {Grp. gram.:m.}
\end{itemize}
Gênero de plantas ranunculáceas.
\section{Tróllio}
\begin{itemize}
\item {Grp. gram.:m.}
\end{itemize}
Gênero de plantas ranunculáceas.
\section{Trom}
\begin{itemize}
\item {Grp. gram.:m.}
\end{itemize}
\begin{itemize}
\item {Utilização:Ant.}
\end{itemize}
\begin{itemize}
\item {Proveniência:(T. onom.)}
\end{itemize}
Som do canhão.
Espécie de catapulta, para arremessar pedras.
Trovão.
Canhão.
\section{Tromba}
\begin{itemize}
\item {Grp. gram.:f.}
\end{itemize}
\begin{itemize}
\item {Utilização:Pleb.}
\end{itemize}
\begin{itemize}
\item {Utilização:Ant.}
\end{itemize}
\begin{itemize}
\item {Proveniência:(It. \textunderscore tromba\textunderscore )}
\end{itemize}
Órgão do olfacto e apparelho de apprehensão, alongado, flexível, collocado na parte superior da bôca do elephante e do tapir.
Sugadoiro de insecto.
Focinho.
Meteóro, que consiste numa columna do água, agitada por turbilhões de vento, e girando rapidamente em volta de si própria.
O mesmo que \textunderscore cara\textunderscore ^1.
O mesmo que \textunderscore trombeta\textunderscore ^1:«\textunderscore ...tangendo aquellas celestiaes trombas, os muros das cidades cahyam por terra.\textunderscore »Barros, \textunderscore Rópia\textunderscore , 121.
\section{Trombada}
\begin{itemize}
\item {Grp. gram.:f.}
\end{itemize}
Pancada com a tromba ou com o focinho.
\section{Tromba-de-boi}
\begin{itemize}
\item {Grp. gram.:f.}
\end{itemize}
\begin{itemize}
\item {Grp. gram.:M.}
\end{itemize}
\begin{itemize}
\item {Utilização:Pop.}
\end{itemize}
O mesmo que \textunderscore baionesa\textunderscore .
Homem trombudo, antipáthico, de má catadura.
\section{Trombão}
\begin{itemize}
\item {Grp. gram.:m.}
\end{itemize}
\begin{itemize}
\item {Utilização:Prov.}
\end{itemize}
\begin{itemize}
\item {Utilização:alent.}
\end{itemize}
\begin{itemize}
\item {Proveniência:(De \textunderscore tromba\textunderscore )}
\end{itemize}
O mesmo que \textunderscore trombóne\textunderscore .
A parte mais grossa da cana de pescar, quando esta é composta de duas partes.
\section{Trombo}
\begin{itemize}
\item {Grp. gram.:m.}
\end{itemize}
\begin{itemize}
\item {Utilização:Med.}
\end{itemize}
\begin{itemize}
\item {Proveniência:(Gr. \textunderscore thrombos\textunderscore )}
\end{itemize}
Pequeno tumor duro e violáceo, que se fórma em tôrno da abertura de uma veia, depois da sangria.
Tumor, constituído por sangue infiltrado; etc.
\section{Trombose}
\begin{itemize}
\item {Grp. gram.:f.}
\end{itemize}
\begin{itemize}
\item {Utilização:Med.}
\end{itemize}
\begin{itemize}
\item {Proveniência:(Do gr. \textunderscore thrombos\textunderscore )}
\end{itemize}
Coagulação do sangue em qualquer ponto do sistema circulatório, no corpo vivo.
\section{Troncar}
\textunderscore v. t.\textunderscore  (e der.)
O mesmo que \textunderscore truncar\textunderscore , etc.
\section{Troncaria}
\begin{itemize}
\item {Grp. gram.:f.}
\end{itemize}
\begin{itemize}
\item {Proveniência:(De \textunderscore tronco\textunderscore ^1)}
\end{itemize}
Ornato architectónico, feito com troncos.
\section{Tronchar}
\begin{itemize}
\item {Grp. gram.:v. t.}
\end{itemize}
\begin{itemize}
\item {Proveniência:(Do lat. \textunderscore trunculare\textunderscore )}
\end{itemize}
Cortar cerce; mutilar.
\section{Troncho}
\begin{itemize}
\item {Grp. gram.:adj.}
\end{itemize}
\begin{itemize}
\item {Grp. gram.:M.}
\end{itemize}
\begin{itemize}
\item {Proveniência:(De \textunderscore tronchar\textunderscore )}
\end{itemize}
O mesmo que \textunderscore mutilado\textunderscore .
A que se cortou algum ramo ou membro.
Membro cortado.
\section{Tronchuda}
\begin{itemize}
\item {Grp. gram.:f.}
\end{itemize}
Couve tronchuda:«\textunderscore ...uma enorme tigela de caldo de tronchuda e vagens.\textunderscore »Júl. Dinis, \textunderscore Pupillas\textunderscore , 95.
\section{Tronchudo}
\begin{itemize}
\item {Grp. gram.:adj.}
\end{itemize}
\begin{itemize}
\item {Proveniência:(De \textunderscore troncho\textunderscore )}
\end{itemize}
Que tem talos grossos.
Diz-se especialmente de uma variedade de couve.
\section{Tronco}
\begin{itemize}
\item {Grp. gram.:m.}
\end{itemize}
\begin{itemize}
\item {Utilização:Bras}
\end{itemize}
\begin{itemize}
\item {Utilização:Bras}
\end{itemize}
\begin{itemize}
\item {Utilização:Poét.}
\end{itemize}
\begin{itemize}
\item {Utilização:Gír.}
\end{itemize}
\begin{itemize}
\item {Proveniência:(Do lat. \textunderscore truncus\textunderscore )}
\end{itemize}
O mesmo que \textunderscore caule\textunderscore .
Parte da árvore, entre a raíz e a rama.
Pernada de árvore.
O corpo humano, exceptuando a cabeça e os membros.
Cepo, com olhaes, próprio para nelle se prender o pé ou o pescoço, e que se usava em supplícios inquisitoriaes.
Apparelho, feito de dois tapumes, entre os quaes se prende o gado para o ferrar ou pensar.
Espaço, separado por tapumes, em trabalhos de mineração.
Prisão; cárcere.
Encargo, obrigação.
Mastro de navio.
Fragmento de fuste.
Origem de uma família, raça, estirpe.
Parte de um sólido, separado por um córte perpendicular ou oblíquo ao eixo do mesmo sólido.
O mesmo que \textunderscore navio\textunderscore .
O mesmo que \textunderscore homem\textunderscore .
\section{Tronco}
\begin{itemize}
\item {Grp. gram.:adj.}
\end{itemize}
O mesmo que \textunderscore truncado\textunderscore  ou \textunderscore troncho\textunderscore .
\section{Troncudo}
\begin{itemize}
\item {Grp. gram.:adj.}
\end{itemize}
\begin{itemize}
\item {Utilização:Bras. do N}
\end{itemize}
Que tem o tronco desenvolvido.
\section{Tronda}
\begin{itemize}
\item {Grp. gram.:f.}
\end{itemize}
\begin{itemize}
\item {Utilização:Prov.}
\end{itemize}
\begin{itemize}
\item {Utilização:minh.}
\end{itemize}
Pau queimado, que serve para acender o lume.
\section{Trondão}
\begin{itemize}
\item {Grp. gram.:m.}
\end{itemize}
\begin{itemize}
\item {Utilização:T. de Bragança}
\end{itemize}
Mulhér gorda, feia e desajeitada; estafermo.
\section{Troneira}
\begin{itemize}
\item {Grp. gram.:f.}
\end{itemize}
Intervallo dos merlões, por onde se enfia a bôca do canhão ou bombarda.
(Cast. \textunderscore tronera\textunderscore )
\section{Troneto}
\begin{itemize}
\item {fónica:nê}
\end{itemize}
\begin{itemize}
\item {Grp. gram.:m.}
\end{itemize}
Pequeno trono.
\section{Tronga}
\begin{itemize}
\item {Grp. gram.:f.}
\end{itemize}
\begin{itemize}
\item {Utilização:Gír.}
\end{itemize}
\begin{itemize}
\item {Proveniência:(T. cast.)}
\end{itemize}
Barregan; prostituta.
\section{Trónio}
\begin{itemize}
\item {Grp. gram.:m.}
\end{itemize}
Massa principal de montanha ou de cordilheira, dominando as suas ramificações.
(Relaciona-se com \textunderscore throno\textunderscore ?)
\section{Trono}
\begin{itemize}
\item {Grp. gram.:m.}
\end{itemize}
\begin{itemize}
\item {Utilização:Fig.}
\end{itemize}
\begin{itemize}
\item {Grp. gram.:Pl.}
\end{itemize}
\begin{itemize}
\item {Utilização:Theol.}
\end{itemize}
\begin{itemize}
\item {Proveniência:(Lat. \textunderscore thronus\textunderscore )}
\end{itemize}
Assento elevado ou sólio, que os Soberanos ocupam nas ocasiões solenes.
Poder ou autoridade soberana; soberania.
Soberano.
Um dos nove coros dos anjos.
\section{Trono}
\begin{itemize}
\item {Grp. gram.:m.}
\end{itemize}
\begin{itemize}
\item {Utilização:Prov.}
\end{itemize}
\begin{itemize}
\item {Utilização:minh.}
\end{itemize}
Acto de tronar.
Trovão.
\section{Tronqueira}
\begin{itemize}
\item {Grp. gram.:f.}
\end{itemize}
\begin{itemize}
\item {Utilização:Açor}
\end{itemize}
\begin{itemize}
\item {Utilização:Bras. do S}
\end{itemize}
\begin{itemize}
\item {Proveniência:(De \textunderscore tronco\textunderscore )}
\end{itemize}
Passagem estreita ordinária, onde ficaram os madeiros lateraes de uma portada ou cancella.
Cada um dos madeiros verticaes, em que se introduzem as extremidades das varas de uma cancella.
\section{Tronqueiro}
\begin{itemize}
\item {Grp. gram.:m.}
\end{itemize}
\begin{itemize}
\item {Utilização:Des.}
\end{itemize}
\begin{itemize}
\item {Utilização:Prov.}
\end{itemize}
\begin{itemize}
\item {Utilização:alent.}
\end{itemize}
\begin{itemize}
\item {Proveniência:(De \textunderscore tronco\textunderscore )}
\end{itemize}
Carcereiro, guarda do tronco.
Aquelle que corta árvores. Cf. B. Pato. \textunderscore Livro do Monte\textunderscore , 121.
\section{Tropa}
\begin{itemize}
\item {Grp. gram.:f.}
\end{itemize}
\begin{itemize}
\item {Utilização:Bras}
\end{itemize}
\begin{itemize}
\item {Utilização:Pop.}
\end{itemize}
\begin{itemize}
\item {Grp. gram.:M.}
\end{itemize}
\begin{itemize}
\item {Utilização:pop.}
\end{itemize}
\begin{itemize}
\item {Grp. gram.:F. Pl.}
\end{itemize}
\begin{itemize}
\item {Proveniência:(Do it. \textunderscore trupa\textunderscore )}
\end{itemize}
Multidão de pessôas reunidas.
Conjunto de soldados.
Os soldades.
Exército.
Caravana de bêstas de carga, ou manada de gado grosso.
\textunderscore Tropa fandanga\textunderscore , grupo de gente indisciplinada ou desprezível.
O mesmo que \textunderscore soldado\textunderscore . Cf. Camillo, \textunderscore Maria da Fonte\textunderscore , 258.
O mesmo que \textunderscore exército\textunderscore .
\section{Tropar}
\begin{itemize}
\item {Grp. gram.:v. i.}
\end{itemize}
\begin{itemize}
\item {Utilização:Ant.}
\end{itemize}
O mesmo que \textunderscore tropear\textunderscore . Cf. M. Bernárdez, (excerptos por Castilho, 49).
\section{Tropeada}
\begin{itemize}
\item {Grp. gram.:f.}
\end{itemize}
Acto ou effeito de tropear; tropel.
\section{Tropear}
\begin{itemize}
\item {Grp. gram.:v. i.}
\end{itemize}
\begin{itemize}
\item {Proveniência:(De \textunderscore tropel\textunderscore )}
\end{itemize}
Fazer barulho com os pés, andando; fazer tropel; estropear^2.
\section{Tropido}
\begin{itemize}
\item {Grp. gram.:m.}
\end{itemize}
\begin{itemize}
\item {Utilização:Prov.}
\end{itemize}
\begin{itemize}
\item {Utilização:trasm.}
\end{itemize}
Acto de tropear; tropel. Cf. Deusdado, \textunderscore Escorços Trasm.\textunderscore , 141.
\section{Tropidocarpo}
\begin{itemize}
\item {Grp. gram.:m.}
\end{itemize}
\begin{itemize}
\item {Proveniência:(Do gr. \textunderscore tropis\textunderscore , \textunderscore tropidos\textunderscore  + \textunderscore karpos\textunderscore )}
\end{itemize}
Gênero de plantas crucíferas.
\section{Tropidófide}
\begin{itemize}
\item {Grp. gram.:m.}
\end{itemize}
\begin{itemize}
\item {Proveniência:(Do gr. \textunderscore tropis\textunderscore , \textunderscore tropidos\textunderscore  + \textunderscore ophis\textunderscore )}
\end{itemize}
Gênero de reptís ofídios.
\section{Tropidóphide}
\begin{itemize}
\item {Grp. gram.:m.}
\end{itemize}
\begin{itemize}
\item {Proveniência:(Do gr. \textunderscore tropis\textunderscore , \textunderscore tropidos\textunderscore  + \textunderscore ophis\textunderscore )}
\end{itemize}
Gênero de reptís ophídios.
\section{Tropidorhynco}
\begin{itemize}
\item {fónica:rin}
\end{itemize}
\begin{itemize}
\item {Grp. gram.:m.}
\end{itemize}
\begin{itemize}
\item {Proveniência:(Do gr. \textunderscore tropis\textunderscore , \textunderscore tropidos\textunderscore  + \textunderscore rhunkhos\textunderscore )}
\end{itemize}
Gênero de aves da ordem dos pássaros.
\section{Tropidorrinco}
\begin{itemize}
\item {Grp. gram.:m.}
\end{itemize}
\begin{itemize}
\item {Proveniência:(Do gr. \textunderscore tropis\textunderscore , \textunderscore tropidos\textunderscore  + \textunderscore rhunkhos\textunderscore )}
\end{itemize}
Gênero de aves da ordem dos pássaros.
\section{Tropilha}
\begin{itemize}
\item {Grp. gram.:f.}
\end{itemize}
\begin{itemize}
\item {Utilização:Bras. do S}
\end{itemize}
\begin{itemize}
\item {Proveniência:(De \textunderscore tropa\textunderscore )}
\end{itemize}
Porção de cavallos, que têm o mesmo pelame.
\section{Tropismo}
\begin{itemize}
\item {Grp. gram.:m.}
\end{itemize}
\begin{itemize}
\item {Utilização:Med.}
\end{itemize}
\begin{itemize}
\item {Proveniência:(Do gr. \textunderscore trepein\textunderscore )}
\end{itemize}
O mesmo que \textunderscore taxia\textunderscore .
\section{Tropitas}
\begin{itemize}
\item {Grp. gram.:m. pl.}
\end{itemize}
Herejes, segundo os quaes o Verbo, ao encarnar, deixou de ser Deus.
\section{Tropo}
\begin{itemize}
\item {fónica:trô}
\end{itemize}
\begin{itemize}
\item {Grp. gram.:m.}
\end{itemize}
\begin{itemize}
\item {Utilização:Gram.}
\end{itemize}
\begin{itemize}
\item {Proveniência:(Lat. \textunderscore tropus\textunderscore )}
\end{itemize}
Emprêgo de uma palavra em sentido figurado.
\section{Tropo}
\begin{itemize}
\item {fónica:trô}
\end{itemize}
\begin{itemize}
\item {Grp. gram.:m.}
\end{itemize}
\begin{itemize}
\item {Utilização:Pop.}
\end{itemize}
O mesmo que \textunderscore trôpego\textunderscore .
\section{Tropologia}
\begin{itemize}
\item {Grp. gram.:f.}
\end{itemize}
\begin{itemize}
\item {Proveniência:(Do gr. \textunderscore tropos\textunderscore  + \textunderscore logos\textunderscore )}
\end{itemize}
Emprêgo de linguagem figurada.
Tratado, á cêrca dos tropos.
\section{Tropológico}
\begin{itemize}
\item {Grp. gram.:adj.}
\end{itemize}
Relativo á tropologia.
\section{Tropona}
\begin{itemize}
\item {Grp. gram.:f.}
\end{itemize}
\begin{itemize}
\item {Utilização:Pharm.}
\end{itemize}
Nova preparação alimentícia, em que entra alumina, em a proporção de 89 por 100.
\section{Troponómico}
\begin{itemize}
\item {Grp. gram.:adj.}
\end{itemize}
\begin{itemize}
\item {Proveniência:(Do gr. \textunderscore tropos\textunderscore  + \textunderscore nomos\textunderscore )}
\end{itemize}
Diz-se das mudanças, que um dado objecto experimenta, segundo os diversos tempos e logares.
\section{Tropos-galhopos}
\begin{itemize}
\item {Grp. gram.:m. pl.}
\end{itemize}
\begin{itemize}
\item {Utilização:Prov.}
\end{itemize}
\begin{itemize}
\item {Utilização:alent.}
\end{itemize}
Us. na \textunderscore loc. adv.\textunderscore  \textunderscore aos tropos-galhopos\textunderscore , tropeçando aqui, caíndo acolá, etc.
\section{Troque}
\begin{itemize}
\item {Grp. gram.:m.}
\end{itemize}
\begin{itemize}
\item {Utilização:Des.}
\end{itemize}
O mesmo que \textunderscore troca\textunderscore . Cf. Filinto, VI, 204.
\section{Troquel}
\begin{itemize}
\item {Grp. gram.:m.}
\end{itemize}
Fôrma, para a cunhagem de moédas.
(Cast. \textunderscore troquel\textunderscore )
\section{Troques}
\begin{itemize}
\item {Grp. gram.:m. Pl.}
\end{itemize}
\begin{itemize}
\item {Utilização:Prov.}
\end{itemize}
\begin{itemize}
\item {Utilização:trasm.}
\end{itemize}
\begin{itemize}
\item {Proveniência:(T. onom.)}
\end{itemize}
Estalos, que se dão com o dedo pollegar e o maior, em certas danças e quando se afagam cães.
Flôres da dedaleira, que os rapazes fazem estalar, fechando-lhes a abertura, e batendo com ellas na mão ou na testa.
\section{Troqueu}
\begin{itemize}
\item {Grp. gram.:m.}
\end{itemize}
\begin{itemize}
\item {Proveniência:(Lat. \textunderscore trochaeus\textunderscore )}
\end{itemize}
Pé de verso grego ou latino, composto de uma sílaba longa e outra breve.
\section{Troquilha}
\begin{itemize}
\item {Grp. gram.:m.}
\end{itemize}
\begin{itemize}
\item {Utilização:Pop.}
\end{itemize}
\begin{itemize}
\item {Proveniência:(De \textunderscore troca\textunderscore )}
\end{itemize}
Indivíduo, especialmente cigano, que cifra os seus negócios em fazer trocas successivas de animaes, nas feiras:«\textunderscore e puxava pela saquita de missanga com gestos de troquilha de burros em feira.\textunderscore »Camillo, \textunderscore Brasileira\textunderscore , 72.
\section{Troquilheira}
\begin{itemize}
\item {Grp. gram.:f.}
\end{itemize}
\begin{itemize}
\item {Proveniência:(De \textunderscore troquilha\textunderscore )}
\end{itemize}
Mulhér de má nota, que illude incautos:«\textunderscore tenho vergonha, sua troquilheira.\textunderscore »Camillo, \textunderscore Filha do Arced.\textunderscore , 167.
\section{Tróquilo}
\begin{itemize}
\item {Grp. gram.:m.}
\end{itemize}
\begin{itemize}
\item {Proveniência:(Lat. \textunderscore trochilus\textunderscore )}
\end{itemize}
Moldura côncava.
\section{Troquiniano}
\begin{itemize}
\item {Grp. gram.:adj.}
\end{itemize}
Relativo ao troquino.
\section{Troquino}
\begin{itemize}
\item {Grp. gram.:m.}
\end{itemize}
\begin{itemize}
\item {Utilização:Anat.}
\end{itemize}
\begin{itemize}
\item {Proveniência:(Do gr. \textunderscore trokhos\textunderscore , roda)}
\end{itemize}
Tuberosidade menor da extremidade superior do húmero.
\section{Tróquio}
\begin{itemize}
\item {Grp. gram.:m.}
\end{itemize}
\begin{itemize}
\item {Proveniência:(Do gr. \textunderscore trokhos\textunderscore )}
\end{itemize}
Gênero de moluscos gasterópodes.
\section{Troquisco}
\begin{itemize}
\item {Grp. gram.:m.}
\end{itemize}
\begin{itemize}
\item {Proveniência:(Lat. \textunderscore trochiscus\textunderscore )}
\end{itemize}
Entre os Gregos, bolo de sêmeas finas, amassadas com vinho branco. Cf. Castilho, \textunderscore Fastos\textunderscore , 477.
O mesmo que \textunderscore trocisco\textunderscore ^1.
\section{Troquitér}
\begin{itemize}
\item {Grp. gram.:m.}
\end{itemize}
\begin{itemize}
\item {Utilização:Anat.}
\end{itemize}
\begin{itemize}
\item {Proveniência:(Lat. scient. \textunderscore trochiter\textunderscore , do gr. \textunderscore trokhos\textunderscore )}
\end{itemize}
A maior tuberosidade da extremidade superior do húmero.
\section{Troquiteriano}
\begin{itemize}
\item {Grp. gram.:adj.}
\end{itemize}
Relativo ao troquitér.
\section{Troses}
\begin{itemize}
\item {Grp. gram.:m. Pl.}
\end{itemize}
\begin{itemize}
\item {Utilização:Gír.}
\end{itemize}
Calças.
\section{Trosquiar}
\textunderscore v. t.\textunderscore  (e der.)
Fórma antiga de \textunderscore tosquiar\textunderscore , etc. Cf. \textunderscore Eufrosina\textunderscore , 42.--O \textunderscore Elucidário\textunderscore  de S. R. Viterbo regista \textunderscore trusquiar\textunderscore .
\section{Trotador}
\begin{itemize}
\item {Grp. gram.:m.  e  adj.}
\end{itemize}
O que trota.
\section{Trotão}
\begin{itemize}
\item {Grp. gram.:m.}
\end{itemize}
Cavallo que trota.
\section{Trotar}
\begin{itemize}
\item {Grp. gram.:v. i.}
\end{itemize}
\begin{itemize}
\item {Proveniência:(Do lat. \textunderscore tolutare\textunderscore ?)}
\end{itemize}
Andar a trote.
\section{Trote}
\begin{itemize}
\item {Grp. gram.:m.}
\end{itemize}
\begin{itemize}
\item {Utilização:Pop.}
\end{itemize}
\begin{itemize}
\item {Utilização:Bras}
\end{itemize}
\begin{itemize}
\item {Proveniência:(De \textunderscore trotar\textunderscore )}
\end{itemize}
Andamento natural dos cavallos e de outros quadrúpedes, mais ligeiro que o passo ordinário e menos rápido que o galope.
O mesmo que \textunderscore cotio\textunderscore ^1, uso quotidiano: \textunderscore ainda há dias te mandei fazer essas calças e já as meteste a trote\textunderscore .
Troça, vaia.
\section{Trote}
\begin{itemize}
\item {Grp. gram.:m.}
\end{itemize}
\begin{itemize}
\item {Utilização:T. da Bairrada}
\end{itemize}
Conjunto das carumas, mato miúdo, musgo, etc., que cobre o solo dos pinhaes e que se aproveita para estrume.
\section{Trouxa}
\begin{itemize}
\item {Grp. gram.:f.}
\end{itemize}
\begin{itemize}
\item {Utilização:Prov.}
\end{itemize}
\begin{itemize}
\item {Utilização:beir.}
\end{itemize}
\begin{itemize}
\item {Grp. gram.:M.}
\end{itemize}
Fardo de roupa.
Grande pacote.
Mulhér desajeitada e mal procedida. (Colhido na Guarda)
Trampolineiro, pulha. Cf. Camillo, \textunderscore Hist. e Sentim.\textunderscore , 164.
\section{Trouxada}
\begin{itemize}
\item {Grp. gram.:f.}
\end{itemize}
Grande trouxa.
\section{Trouxel}
\begin{itemize}
\item {Grp. gram.:m.}
\end{itemize}
\begin{itemize}
\item {Utilização:Ant.}
\end{itemize}
O mesmo que \textunderscore trouxa\textunderscore .
\section{Trouxelo}
\begin{itemize}
\item {Grp. gram.:m.}
\end{itemize}
(V.trouxel)
\section{Truco}
\begin{itemize}
\item {Grp. gram.:m.}
\end{itemize}
\begin{itemize}
\item {Utilização:Prov.}
\end{itemize}
Rôlo de madeira, sôbre que se facilita a remoção de caixas e outros fardos.
\section{Trucuás}
\begin{itemize}
\item {Grp. gram.:m.}
\end{itemize}
O mesmo que \textunderscore imbé\textunderscore .
\section{Truculência}
\begin{itemize}
\item {Grp. gram.:f.}
\end{itemize}
\begin{itemize}
\item {Proveniência:(Lat. \textunderscore truculentia\textunderscore )}
\end{itemize}
Qualidade do que é truculento; acto cruel.
\section{Truculento}
\begin{itemize}
\item {Grp. gram.:adj.}
\end{itemize}
\begin{itemize}
\item {Proveniência:(Lat. \textunderscore truculentus\textunderscore )}
\end{itemize}
Atroz; terrível; cruel; bárbaro.
\section{Trude}
\begin{itemize}
\item {Grp. gram.:f.}
\end{itemize}
\begin{itemize}
\item {Proveniência:(Lat. \textunderscore trudis\textunderscore )}
\end{itemize}
Espécie de lança antiga, cujo ferro tinha a fórma de meia-lua. Cf. C. Aires, \textunderscore Hist. do Exérc. Port.\textunderscore 
\section{Trudo}
\begin{itemize}
\item {Grp. gram.:m.}
\end{itemize}
\begin{itemize}
\item {Utilização:Bras}
\end{itemize}
Binóculo de campo.
\section{Trufa}
\begin{itemize}
\item {Grp. gram.:f.}
\end{itemize}
\begin{itemize}
\item {Utilização:Gal}
\end{itemize}
\begin{itemize}
\item {Proveniência:(Fr. \textunderscore truffe\textunderscore )}
\end{itemize}
Gênero de cogumelos.
Cogumelo subterrâneo, carnudo, aromático e comestível.--(Em bom português, \textunderscore túbera\textunderscore )
\section{Trufar}
\begin{itemize}
\item {Grp. gram.:v. t.}
\end{itemize}
\begin{itemize}
\item {Grp. gram.:V. i.}
\end{itemize}
\begin{itemize}
\item {Utilização:Ant.}
\end{itemize}
Rechear ou guarnecer com trufas.
Escarnecer; zombar.
\section{Trufeira}
\begin{itemize}
\item {Grp. gram.:f.}
\end{itemize}
Terreno, onde há trufas.
\section{Trufeiro}
\begin{itemize}
\item {Grp. gram.:adj.}
\end{itemize}
\begin{itemize}
\item {Grp. gram.:M.}
\end{itemize}
Relativo ás trufas.
Aquelle que se occupa em apanhar trufas.
\section{Trugimão}
\begin{itemize}
\item {Grp. gram.:m.}
\end{itemize}
(V.turgimão)
\section{Truita}
\begin{itemize}
\item {Grp. gram.:f.}
\end{itemize}
(V.truta)
\section{Truncadamente}
\begin{itemize}
\item {Grp. gram.:adv.}
\end{itemize}
De modo truncado; incompletamente.
\section{Truncado}
\begin{itemize}
\item {Grp. gram.:adj.}
\end{itemize}
\begin{itemize}
\item {Proveniência:(De \textunderscore truncar\textunderscore )}
\end{itemize}
Incompleto.
Mutilado.
\section{Truncamento}
\begin{itemize}
\item {Grp. gram.:m.}
\end{itemize}
Acto ou effeito de truncar.
\section{Truncar}
\begin{itemize}
\item {Grp. gram.:v. t.}
\end{itemize}
\begin{itemize}
\item {Utilização:Geom.}
\end{itemize}
\begin{itemize}
\item {Proveniência:(Lat. \textunderscore truncare\textunderscore )}
\end{itemize}
Separar do tronco; mutilar.
Cortar com um plano secante (um sólido geométrico).
Omittir parte importante de: \textunderscore truncar um texto\textunderscore .
\section{Truncária}
\begin{itemize}
\item {Grp. gram.:f.}
\end{itemize}
Gênero de plantas melastomáceas.
\section{Truncatura}
\begin{itemize}
\item {Grp. gram.:f.}
\end{itemize}
\begin{itemize}
\item {Utilização:Miner.}
\end{itemize}
\begin{itemize}
\item {Proveniência:(De \textunderscore truncar\textunderscore )}
\end{itemize}
O mesmo que \textunderscore truncamento\textunderscore .
Substituição de uma aresta por uma facêta, num corpo mineral.
\section{Trunfa}
\begin{itemize}
\item {Grp. gram.:f.}
\end{itemize}
Toucado antigo.
Turbante.
Cabello em desalinho; grenha.
\section{Trunfa}
\begin{itemize}
\item {Grp. gram.:f.}
\end{itemize}
\begin{itemize}
\item {Utilização:T. de Pinhel}
\end{itemize}
Troixa ou mala, usada por carrejões.
\section{Trunfada}
\begin{itemize}
\item {Grp. gram.:f.}
\end{itemize}
Acto de trunfar.
Grande porção de trunfos.
\section{Trunfado}
\begin{itemize}
\item {Grp. gram.:adj.}
\end{itemize}
\begin{itemize}
\item {Proveniência:(De \textunderscore trunfar\textunderscore )}
\end{itemize}
Que tem trunfa^1. Cf. Castilho, \textunderscore Metam.\textunderscore , 216.
\section{Trunfar}
\begin{itemize}
\item {Grp. gram.:v. i.}
\end{itemize}
\begin{itemize}
\item {Utilização:Fig.}
\end{itemize}
Jogar trunfo.
Sêr importante, socialmente.
\section{Trunfeira}
\begin{itemize}
\item {Grp. gram.:f.}
\end{itemize}
\begin{itemize}
\item {Utilização:Fam.}
\end{itemize}
Muitos trunfos; trunfada.
\section{Trunfo}
\begin{itemize}
\item {Grp. gram.:m.}
\end{itemize}
\begin{itemize}
\item {Utilização:Fig.}
\end{itemize}
Espécie de jôgo de cartas, com dois, quatro ou seis parceiros.
Naipe que, em jogos de cartas, prevalece aos outros naipes.
Cada uma das cartas dêsse naipe.
Indivíduo, que tem influência ou importância social.
(Corr. de \textunderscore triumpho\textunderscore )
\section{Trupar}
\begin{itemize}
\item {Grp. gram.:v. i.}
\end{itemize}
\begin{itemize}
\item {Utilização:Prov.}
\end{itemize}
Bater (á porta de aguém). Cf. Camillo, \textunderscore Narcót.\textunderscore , I, 233.
(Cp. \textunderscore trupitar\textunderscore )
\section{Trupe!}
\begin{itemize}
\item {Grp. gram.:interj.}
\end{itemize}
\begin{itemize}
\item {Utilização:T. da Bairrada}
\end{itemize}
Voz imitativa de pancada ou estrondo súbito.
\section{Trúpia}
\begin{itemize}
\item {Grp. gram.:f.}
\end{itemize}
O mesmo que \textunderscore breca\textunderscore ^1?:«\textunderscore levadinho da trúpia.\textunderscore »Camillo, \textunderscore Eusébio\textunderscore , 147. Cf Castilho, \textunderscore Méd. á Fôrça\textunderscore , 18.
\section{Trupitar}
\begin{itemize}
\item {Grp. gram.:v. i.}
\end{itemize}
\begin{itemize}
\item {Utilização:Des.}
\end{itemize}
O mesmo que \textunderscore estrepitar\textunderscore .
\section{Truque}
\begin{itemize}
\item {Grp. gram.:m.}
\end{itemize}
\begin{itemize}
\item {Utilização:Fam.}
\end{itemize}
Espécie de bilhar comprido.
Designação de vários processos ou incidentes, no jôgo de bilhar.
Espécie de jôgo de cartas.
Ardil; tramóia.
Em caminhos de ferro, plataforma sôbre rodas ou vagão sem caixa.
(Provn. \textunderscore truc\textunderscore )
\section{Truque-mandruque}
\begin{itemize}
\item {Grp. gram.:m.}
\end{itemize}
\begin{itemize}
\item {Utilização:Prov.}
\end{itemize}
\begin{itemize}
\item {Utilização:alent.}
\end{itemize}
Jôgo de rapazes, mais conhecido por \textunderscore jôgo do homem\textunderscore .
\section{Truqui}
\begin{itemize}
\item {Grp. gram.:m.}
\end{itemize}
Pequena ave canora da ilha de San-Thomé.
\section{Trusquiado}
\begin{itemize}
\item {Grp. gram.:adj.}
\end{itemize}
\begin{itemize}
\item {Proveniência:(De \textunderscore trusquiar\textunderscore )}
\end{itemize}
O mesmo que [[tosquiado|tosquiar]]. Cf. F. Manuel, \textunderscore Carta de Guia\textunderscore , 101.
\section{Tuberáceas}
\begin{itemize}
\item {Grp. gram.:f. pl.}
\end{itemize}
Família de plantas, que tem por typo a túbera.
\section{Tuberculado}
\begin{itemize}
\item {Grp. gram.:adj.}
\end{itemize}
Que tem tubérculos.
\section{Tubercular}
\begin{itemize}
\item {Grp. gram.:adj.}
\end{itemize}
O mesmo que \textunderscore tuberculado\textunderscore .
\section{Tuberculária}
\begin{itemize}
\item {Grp. gram.:f.}
\end{itemize}
\begin{itemize}
\item {Proveniência:(De \textunderscore tubérculo\textunderscore )}
\end{itemize}
Gênero de cogumelos.
\section{Tuberculemia}
\begin{itemize}
\item {Grp. gram.:f.}
\end{itemize}
\begin{itemize}
\item {Utilização:Med.}
\end{itemize}
\begin{itemize}
\item {Proveniência:(De \textunderscore tuberculose\textunderscore  + gr. \textunderscore haima\textunderscore )}
\end{itemize}
Conjunto de accidentes tóxicos, devidos á reabsorpção do veneno tuberculoso.
\section{Tubercúlide}
\begin{itemize}
\item {Grp. gram.:f.}
\end{itemize}
\begin{itemize}
\item {Utilização:Med.}
\end{itemize}
Affecção cutânea, de origem tuberculosa, mas sem bacillos de Koch.
\section{Tuberculífero}
\begin{itemize}
\item {Grp. gram.:adj.}
\end{itemize}
\begin{itemize}
\item {Proveniência:(Do lat. \textunderscore tuberculum\textunderscore  + \textunderscore ferre\textunderscore )}
\end{itemize}
Que tem ou produz tubérculos.
\section{Tuberculiforme}
\begin{itemize}
\item {Grp. gram.:adj.}
\end{itemize}
\begin{itemize}
\item {Proveniência:(De \textunderscore tubérculo\textunderscore  + \textunderscore fórma\textunderscore )}
\end{itemize}
Que tem fórma de tubérculo.
\section{Tuberculina}
\begin{itemize}
\item {Grp. gram.:f.}
\end{itemize}
\begin{itemize}
\item {Proveniência:(De \textunderscore tubérculo\textunderscore )}
\end{itemize}
Vírus artificial, preparado por Koch contra a tuberculose.
\section{Tuberculinização}
\begin{itemize}
\item {Grp. gram.:f.}
\end{itemize}
Acto de tuberculinizar.
\section{Tuberculinizar}
\begin{itemize}
\item {Grp. gram.:v. t.}
\end{itemize}
Applicar a tuberculina a.
\section{Tuberculização}
\begin{itemize}
\item {Grp. gram.:f.}
\end{itemize}
Acto ou effeito de tuberculízar.
\section{Tuberculízar}
\begin{itemize}
\item {Grp. gram.:v. t.}
\end{itemize}
\begin{itemize}
\item {Grp. gram.:V. p.}
\end{itemize}
Causar tubérculos em.
Tornar-se tuberculoso.
\section{Tubérculo}
\begin{itemize}
\item {Grp. gram.:m.}
\end{itemize}
\begin{itemize}
\item {Grp. gram.:Pl.}
\end{itemize}
\begin{itemize}
\item {Proveniência:(Lat. \textunderscore tuberculum\textunderscore )}
\end{itemize}
Massa feculenta e cellular, na parte subterrânea de algumas plantas.
Saliência natural, pouco considerável, em qualquer parte do corpo.
Alteração mórbida, na pelle ou em qualquer outro tecido.
O mesmo que \textunderscore tuberculose\textunderscore .
\section{Tuberculoma}
\begin{itemize}
\item {Grp. gram.:f.}
\end{itemize}
\begin{itemize}
\item {Utilização:Med.}
\end{itemize}
Abcesso tuberculoso.
\section{Tuberculose}
\begin{itemize}
\item {Grp. gram.:f.}
\end{itemize}
\begin{itemize}
\item {Proveniência:(De \textunderscore tubérculo\textunderscore )}
\end{itemize}
Disposição orgânica para a formação de tubérculos.
Tísica, resultante da formação de tubérculos no pulmão.
\section{Tuberculosidade}
\begin{itemize}
\item {Grp. gram.:f.}
\end{itemize}
\begin{itemize}
\item {Utilização:Neol.}
\end{itemize}
Relação, entre os casos de tuberculose, numa região, e o número dos habitantes dessa região.
\section{Tuberculoso}
\begin{itemize}
\item {Grp. gram.:adj.}
\end{itemize}
\begin{itemize}
\item {Grp. gram.:M.}
\end{itemize}
O mesmo que \textunderscore tuberculado\textunderscore .
Relativo a tubérculos.
Que tem tubérculos.
Indivíduo, atacado de tuberculose.
\section{Tuberiforme}
\begin{itemize}
\item {Grp. gram.:adj.}
\end{itemize}
\begin{itemize}
\item {Proveniência:(De \textunderscore túbera\textunderscore  + \textunderscore fórma\textunderscore )}
\end{itemize}
Que tem fórma de túbera.
\section{Tuberóide}
\begin{itemize}
\item {Grp. gram.:adj.}
\end{itemize}
\begin{itemize}
\item {Proveniência:(De \textunderscore túbera\textunderscore  + gr. \textunderscore eidos\textunderscore )}
\end{itemize}
O mesmo que \textunderscore tuberiforme\textunderscore .
\section{Tuberosa}
\begin{itemize}
\item {Grp. gram.:f.}
\end{itemize}
\begin{itemize}
\item {Proveniência:(De \textunderscore tuberoso\textunderscore )}
\end{itemize}
Planta liliácea, de flôr branca e odorífera.
\section{Tuberosidade}
\begin{itemize}
\item {Grp. gram.:f.}
\end{itemize}
\begin{itemize}
\item {Utilização:Bot.}
\end{itemize}
\begin{itemize}
\item {Proveniência:(De \textunderscore tuberoso\textunderscore )}
\end{itemize}
Saliência, em fórma de tubérculo.
Excrescência carnuda.
\section{Tuberositário}
\begin{itemize}
\item {Grp. gram.:adj.}
\end{itemize}
\begin{itemize}
\item {Utilização:Anat.}
\end{itemize}
Em que há tuberosidade.
\section{Tuberoso}
\begin{itemize}
\item {Grp. gram.:adj.}
\end{itemize}
\begin{itemize}
\item {Proveniência:(Lat. \textunderscore tuberosus\textunderscore )}
\end{itemize}
Que tem tuberosidades.
\section{Tubi}
\begin{itemize}
\item {Grp. gram.:m.}
\end{itemize}
\begin{itemize}
\item {Utilização:Bras}
\end{itemize}
\begin{itemize}
\item {Utilização:Bras. do N}
\end{itemize}
O mesmo que \textunderscore tubiba\textunderscore .
Nome de um peixe fluvial.
\section{Tubiba}
\begin{itemize}
\item {Grp. gram.:f.}
\end{itemize}
\begin{itemize}
\item {Utilização:Bras}
\end{itemize}
Espécie de abelha pequena.
\section{Tubícola}
\begin{itemize}
\item {Grp. gram.:adj.}
\end{itemize}
\begin{itemize}
\item {Grp. gram.:Pl.}
\end{itemize}
\begin{itemize}
\item {Proveniência:(Do lat. \textunderscore tubus\textunderscore  + \textunderscore colere\textunderscore )}
\end{itemize}
O mesmo que \textunderscore tubicolado\textunderscore .
Ordem de anélidos, que vivem nas cavidades tubulares da pedra, da madeira, etc.
\section{Tubicolado}
\begin{itemize}
\item {Grp. gram.:adj.}
\end{itemize}
\begin{itemize}
\item {Grp. gram.:M. pl.}
\end{itemize}
\begin{itemize}
\item {Proveniência:(De \textunderscore tubícola\textunderscore )}
\end{itemize}
Que vive no interior de um tubo.
Família de molluscos acéphalos, que segregam um tubo calcário.
\section{Tubicolar}
\begin{itemize}
\item {Grp. gram.:m.}
\end{itemize}
Gênero de infusórios.
(Cp. \textunderscore tubícola\textunderscore )
\section{Tubicolário}
\begin{itemize}
\item {Grp. gram.:adj.}
\end{itemize}
\begin{itemize}
\item {Utilização:Zool.}
\end{itemize}
O mesmo que \textunderscore tubicolado\textunderscore .
\section{Tubicóleos}
\begin{itemize}
\item {Grp. gram.:m. pl.}
\end{itemize}
Molluscos acéphalos, também conhecidos por \textunderscore tubicolados\textunderscore .(V.tubicolado)
\section{Tubicórneos}
\begin{itemize}
\item {Grp. gram.:m. pl.}
\end{itemize}
\begin{itemize}
\item {Proveniência:(De \textunderscore tubo\textunderscore  + \textunderscore corno\textunderscore )}
\end{itemize}
Mammíferos ruminantes, cujos cornos são formados de um eixo ósseo coberto de substância córnea.
\section{Tuchaúa}
\begin{itemize}
\item {Grp. gram.:m.}
\end{itemize}
(V.cacique)
\section{Tuciorista}
\begin{itemize}
\item {Grp. gram.:adj.}
\end{itemize}
\begin{itemize}
\item {Utilização:P. us.}
\end{itemize}
\begin{itemize}
\item {Proveniência:(Do lat. \textunderscore tutior\textunderscore )}
\end{itemize}
Que, entre várias doutrinas ou opiniões, segue a mais segura.
\section{Tucujus}
\begin{itemize}
\item {Grp. gram.:m. pl.}
\end{itemize}
\begin{itemize}
\item {Utilização:Bras}
\end{itemize}
Tríbo de aborígenes do Pará.
\section{Tucum}
\begin{itemize}
\item {Grp. gram.:m.}
\end{itemize}
\begin{itemize}
\item {Utilização:Bras}
\end{itemize}
O mesmo que \textunderscore tucuman\textunderscore .
Pequeno côco, fruto do tucuman.
\section{Tucuma}
\begin{itemize}
\item {Grp. gram.:f.}
\end{itemize}
\begin{itemize}
\item {Utilização:Bras}
\end{itemize}
Palmeira fructífera dos sertões, de cujo fruto se faz vinho.
\section{Tucumá}
\begin{itemize}
\item {Grp. gram.:m.}
\end{itemize}
\begin{itemize}
\item {Utilização:Bras}
\end{itemize}
Palmeida fructífera dos sertões, de cujo fruto se faz vinho.
\section{Tucuman}
\begin{itemize}
\item {Grp. gram.:m.}
\end{itemize}
\begin{itemize}
\item {Utilização:Bras}
\end{itemize}
Palmeida fructífera dos sertões, de cujo fruto se faz vinho.
\section{Tucunaré}
\begin{itemize}
\item {Grp. gram.:m.}
\end{itemize}
\begin{itemize}
\item {Utilização:Bras}
\end{itemize}
Um dos peixes mais estimados do Amazonas.
\section{Tucupi}
\begin{itemize}
\item {Grp. gram.:m.}
\end{itemize}
\begin{itemize}
\item {Utilização:Bras. do N}
\end{itemize}
Môlho de manipueira.
Suco da raíz da mandioca.
(Do tupi)
\section{Tucupim}
\begin{itemize}
\item {Grp. gram.:m.}
\end{itemize}
\begin{itemize}
\item {Utilização:Bras}
\end{itemize}
O mesmo que \textunderscore tucupi\textunderscore .
\section{Tucurujus}
\begin{itemize}
\item {Grp. gram.:m. pl.}
\end{itemize}
Indígenas do norte do Brasil.
O mesmo que \textunderscore tucujus\textunderscore ?
\section{Tudel}
\begin{itemize}
\item {Grp. gram.:m.}
\end{itemize}
Tubo de metal, em que se colloca a palheta de alguns instrumentos músicos.
(Cast. \textunderscore tudel\textunderscore )
\section{Tudense}
\begin{itemize}
\item {Grp. gram.:adj.}
\end{itemize}
\begin{itemize}
\item {Grp. gram.:M.}
\end{itemize}
Relativo a Tuy.
Habitante de Tuy. Cf. Herculano, \textunderscore Hist. de Port.\textunderscore , I, 406.
\section{Tudesco}
\begin{itemize}
\item {fónica:dês}
\end{itemize}
\begin{itemize}
\item {Grp. gram.:adj.}
\end{itemize}
\begin{itemize}
\item {Grp. gram.:M.}
\end{itemize}
\begin{itemize}
\item {Utilização:Ant.}
\end{itemize}
\begin{itemize}
\item {Proveniência:(Do ant. alto al. \textunderscore diutisk\textunderscore )}
\end{itemize}
Relativo aos antigos Germanos.
Alemão.
Língua, o mesmo que \textunderscore alto alemão\textunderscore .
Indivíduo alemão.
Espécie de barrete?«\textunderscore ...leuo do tudesco para trás, como cortesão soldadesco...\textunderscore »\textunderscore Eufrosina\textunderscore , 21.
\section{Tudista}
\begin{itemize}
\item {Grp. gram.:m.}
\end{itemize}
\begin{itemize}
\item {Utilização:Fam.}
\end{itemize}
O mesmo que \textunderscore topa-a-tudo\textunderscore .
\section{Tudo}
\begin{itemize}
\item {Grp. gram.:pron.}
\end{itemize}
\begin{itemize}
\item {Proveniência:(Do lat. \textunderscore totus\textunderscore )}
\end{itemize}
A totalidade ou universalidade das pessôas e coisas que existem.
A totalidade das coisas ou pessôas, de que se está tratando.
Qualquer coisa, considerada na sua totalidade.
Aquillo que é essencial.
Qualquer coisa.
\section{Tudo}
\begin{itemize}
\item {Grp. gram.:adj.}
\end{itemize}
\begin{itemize}
\item {Utilização:Ant.}
\end{itemize}
O mesmo que \textunderscore teúdo\textunderscore  ou \textunderscore tido\textunderscore . cf. S. R. Viterbo, \textunderscore Elucidário\textunderscore .
\section{Tudo-bem}
\begin{itemize}
\item {Grp. gram.:m.}
\end{itemize}
\begin{itemize}
\item {Utilização:Prov.}
\end{itemize}
Ave, o mesmo que \textunderscore ferreirinho\textunderscore .
\section{Tudo-nada}
\begin{itemize}
\item {Grp. gram.:m.}
\end{itemize}
Insignificância; bagatela, quási nada.
Pedacinho; pequeníssima porção.
\section{Tudum}
\begin{itemize}
\item {Grp. gram.:m.}
\end{itemize}
\begin{itemize}
\item {Utilização:T. de Macau}
\end{itemize}
Capucha preta, usada por senhoras.
\section{Tufácio}
\begin{itemize}
\item {Grp. gram.:m.}
\end{itemize}
\begin{itemize}
\item {Proveniência:(Do lat. \textunderscore tofacius\textunderscore )}
\end{itemize}
Pedra ou pedreira areenta ou porosa. Cf. Castilho, \textunderscore Fastos\textunderscore , I, 111.
Cp. \textunderscore tufo\textunderscore ^2.
\section{Tufão}
\begin{itemize}
\item {Grp. gram.:m.}
\end{itemize}
\begin{itemize}
\item {Proveniência:(Gr. \textunderscore tuphon\textunderscore )}
\end{itemize}
Vento muito forte, tempestuoso; vendaval.
\section{Tufão}
\begin{itemize}
\item {Grp. gram.:m.}
\end{itemize}
\begin{itemize}
\item {Utilização:Prov.}
\end{itemize}
\begin{itemize}
\item {Utilização:alg.}
\end{itemize}
\begin{itemize}
\item {Proveniência:(De \textunderscore tufo\textunderscore ^2)}
\end{itemize}
Variedade de calcário tufoso.
\section{Tufar}
\begin{itemize}
\item {Grp. gram.:v. t.}
\end{itemize}
\begin{itemize}
\item {Grp. gram.:V. i.}
\end{itemize}
\begin{itemize}
\item {Utilização:Bras. do N}
\end{itemize}
\begin{itemize}
\item {Proveniência:(De \textunderscore tufo\textunderscore ^1)}
\end{itemize}
Dar fórma de tufo^1 a.
Aumentar o volume a.
Tomar a fórma de tufo^1.
Aumentar de volume.
Enfunar.
O mesmo que \textunderscore amuar\textunderscore .
\section{Tufo}
\begin{itemize}
\item {Grp. gram.:m.}
\end{itemize}
\begin{itemize}
\item {Utilização:Prov.}
\end{itemize}
\begin{itemize}
\item {Utilização:dur.}
\end{itemize}
\begin{itemize}
\item {Proveniência:(Do lat. \textunderscore tufa\textunderscore )}
\end{itemize}
Porção de plantas, ou de flôres, ou de pennas, etc., muito aproximadas.
Vello aberto.
Proeminência.
Montículo.
Saliência, formada pelos tecidos de um vestuário.
Utensílio de espingardeiro.
Válvula de ferro, nos fornos de fundição.
Utensílio de ferreiro, com que se aperfeiçoam os olhos das enxós, dos machados, etc.
Peça de metal, que se introduz na fêmea do leme.
Orifício no lastro de um tanque, para o esvaziar.
\section{Tufo}
\begin{itemize}
\item {Grp. gram.:m.}
\end{itemize}
\begin{itemize}
\item {Proveniência:(Do lat. \textunderscore tofus\textunderscore )}
\end{itemize}
Espécie de pedra porosa, que se esborôa facilmente.
Pedra esbranquiçada e branda, applicada em construcções.
\section{Tufo}
\begin{itemize}
\item {Grp. gram.:m.}
\end{itemize}
O mesmo que \textunderscore alcatruzada\textunderscore .
\section{Tufoso}
\begin{itemize}
\item {Grp. gram.:adj.}
\end{itemize}
Entufado; que tem fórma de tufo^1.
\section{Tuia}
\begin{itemize}
\item {Grp. gram.:f.}
\end{itemize}
\begin{itemize}
\item {Proveniência:(Do gr. \textunderscore thuia\textunderscore )}
\end{itemize}
Gênero de árvores coníferas.
\section{Tuja}
\begin{itemize}
\item {Grp. gram.:f.}
\end{itemize}
\begin{itemize}
\item {Utilização:Des.}
\end{itemize}
O mesmo que \textunderscore tuia\textunderscore .
\section{Tule}
\begin{itemize}
\item {Proveniência:(De \textunderscore Tullo\textunderscore , n. p.)}
\end{itemize}
Tecido transparente de seda ou algodão.
\section{Tuliano}
\begin{itemize}
\item {Grp. gram.:adj.}
\end{itemize}
\begin{itemize}
\item {Proveniência:(Lat. \textunderscore tullianus\textunderscore )}
\end{itemize}
Relativo a Túllio.
Dizia-se de um célebre cárcere de Roma.
\section{Tulle}
\begin{itemize}
\item {Proveniência:(De \textunderscore Tullo\textunderscore , n. p.)}
\end{itemize}
Tecido transparente de seda ou algodão.
\section{Tulliano}
\begin{itemize}
\item {Grp. gram.:adj.}
\end{itemize}
\begin{itemize}
\item {Proveniência:(Lat. \textunderscore tullianus\textunderscore )}
\end{itemize}
Relativo a Túllio.
Dizia-se de um célebre cárcere de Roma.
\section{Tulo}
\begin{itemize}
\item {Grp. gram.:m.}
\end{itemize}
O mesmo que \textunderscore túluva\textunderscore .
\section{Tulocarpo}
\begin{itemize}
\item {Grp. gram.:m.}
\end{itemize}
Gênero de plantas, da fam. das synanthéreas.
\section{Túluva}
\begin{itemize}
\item {Grp. gram.:m.}
\end{itemize}
Uma das línguas dravídicas.
\section{Tum!}
\begin{itemize}
\item {Grp. gram.:interj.}
\end{itemize}
\begin{itemize}
\item {Proveniência:(T. onom.)}
\end{itemize}
(com que se imita a explosão de um tiro)
\section{Tumba}
\begin{itemize}
\item {Grp. gram.:f.}
\end{itemize}
\begin{itemize}
\item {Utilização:Constr.}
\end{itemize}
\begin{itemize}
\item {Grp. gram.:M.  e  f.}
\end{itemize}
\begin{itemize}
\item {Utilização:Fam.}
\end{itemize}
\begin{itemize}
\item {Proveniência:(Lat. \textunderscore tumba\textunderscore )}
\end{itemize}
Lápide sepulcral.
Sepultura.
Maca em que se conduzem cadáveres á sepultura; esquife.
Almofada abahulada, sôbre que os encardenadores põem as capas dos livros, para os doirar ou ornamentar.
\textunderscore Abóbada de tumba\textunderscore , abóbada cylíndrica.
Pessôa infeliz ou desastrada, principalmente ao jôgo.
\section{Tumba!}
\begin{itemize}
\item {Grp. gram.:interj.}
\end{itemize}
\begin{itemize}
\item {Proveniência:(T. onom.)}
\end{itemize}
(designativa do som, produzido por pancada ou quéda); zás-trás.
\section{Tumba}
\begin{itemize}
\item {Grp. gram.:f.}
\end{itemize}
Acto de fazer as três quinas de um cartão, no jôgo do quino.
\section{Tumba-catatumba!}
\begin{itemize}
\item {Grp. gram.:interj.}
\end{itemize}
\begin{itemize}
\item {Utilização:Pop.}
\end{itemize}
Voz, em que se designam pancadas repetidas: \textunderscore agarrou o rapaz, e tumba-catatumba\textunderscore !
\section{Tumbal}
\begin{itemize}
\item {Grp. gram.:adj.}
\end{itemize}
\begin{itemize}
\item {Utilização:Neol.}
\end{itemize}
Relativo a tumba^1.
\section{Tumbansa}
\begin{itemize}
\item {Grp. gram.:f.}
\end{itemize}
\begin{itemize}
\item {Utilização:Bras. do Ceará}
\end{itemize}
Iguaria, feita de castanha de caju, sumo da mesma fruta e açúcar.
\section{Tumbar}
\begin{itemize}
\item {Grp. gram.:v. i.}
\end{itemize}
\begin{itemize}
\item {Proveniência:(De \textunderscore tumba\textunderscore ^3)}
\end{itemize}
Encher todas as casas do cartão, primeiro que os outros parceiros, no jôgo do quino.
\section{Tumbeiro}
\begin{itemize}
\item {Grp. gram.:m.}
\end{itemize}
\begin{itemize}
\item {Grp. gram.:Adj.}
\end{itemize}
Conductor de tumba^1.
Relativo a tumba^1. Cf. Filinto, IX, 107.
\section{Tumbice}
\begin{itemize}
\item {Grp. gram.:f.}
\end{itemize}
\begin{itemize}
\item {Utilização:Fam.}
\end{itemize}
Qualidade de quem é tumba^1 ou infeliz.
\section{Tumbiras}
\begin{itemize}
\item {Grp. gram.:m. pl.}
\end{itemize}
\begin{itemize}
\item {Utilização:Bras}
\end{itemize}
O mesmo que \textunderscore timbiras\textunderscore .
\section{Tumbo}
\begin{itemize}
\item {Grp. gram.:m.}
\end{itemize}
Planta gymnospérmea de Angola.
\section{Tumecer}
\textunderscore v. i.\textunderscore  e \textunderscore p.\textunderscore  (e der.)
O mesmo que \textunderscore entumecer\textunderscore , etc. Cf. Filinto, VI, 152.
\section{Tumefacção}
\begin{itemize}
\item {Grp. gram.:f.}
\end{itemize}
Acto ou effeito de tumefazer.
\section{Tumefaciente}
\begin{itemize}
\item {Grp. gram.:adj.}
\end{itemize}
\begin{itemize}
\item {Proveniência:(Lat. \textunderscore tumefaciens\textunderscore )}
\end{itemize}
Que tumefaz.
\section{Tumefazer}
\begin{itemize}
\item {Grp. gram.:v. t.  e  p.}
\end{itemize}
O mesmo que \textunderscore tumeficar\textunderscore .
\section{Tumeficante}
\begin{itemize}
\item {Grp. gram.:adj.}
\end{itemize}
Que tumefica.
\section{Tumeficar}
\begin{itemize}
\item {Grp. gram.:v. t.}
\end{itemize}
\begin{itemize}
\item {Proveniência:(Lat. \textunderscore tumefacere\textunderscore )}
\end{itemize}
Tornar túmido.
Fazer inchar, entumescer.
\section{Tumente}
\begin{itemize}
\item {Grp. gram.:adj.}
\end{itemize}
\begin{itemize}
\item {Proveniência:(Lat. \textunderscore tumens\textunderscore )}
\end{itemize}
O mesmo que [[tumefacto|tumefacção]].
\section{Tumescer}
\textunderscore v. t.\textunderscore  e \textunderscore i.\textunderscore  (e der.)
O mesmo que \textunderscore entumecer\textunderscore , etc.
\section{Tumidamente}
\begin{itemize}
\item {Grp. gram.:adv.}
\end{itemize}
De modo túmido.
\section{Tumidez}
\begin{itemize}
\item {Grp. gram.:f.}
\end{itemize}
Qualidade do que é túmido; proeminência; inchação.
\section{Tumidificar}
\begin{itemize}
\item {Grp. gram.:v. t.}
\end{itemize}
Tornar túmido; engrossar.
\section{Túmido}
\begin{itemize}
\item {Grp. gram.:adj.}
\end{itemize}
\begin{itemize}
\item {Utilização:Fig.}
\end{itemize}
\begin{itemize}
\item {Proveniência:(Lat. \textunderscore tumidus\textunderscore )}
\end{itemize}
Que aumentou em volume.
Inchado; proeminente; dilatado.
Grosso.
Vaidoso; arrogante.
\section{Tumor}
\begin{itemize}
\item {Grp. gram.:m.}
\end{itemize}
\begin{itemize}
\item {Proveniência:(Lat. \textunderscore tumor\textunderscore )}
\end{itemize}
Saliência circunscrita, desenvolvida em qualquer parte do corpo.
\section{Tumoroso}
\begin{itemize}
\item {Grp. gram.:adj.}
\end{itemize}
Que tem tumor.
\section{Tumular}
\begin{itemize}
\item {Grp. gram.:v. t.}
\end{itemize}
\begin{itemize}
\item {Proveniência:(Lat. \textunderscore tumulare\textunderscore )}
\end{itemize}
Pôr no túmulo; sepultar.
\section{Tumular}
\begin{itemize}
\item {Grp. gram.:adj.}
\end{itemize}
Relativo ao túmulo.
\section{Tumulário}
\begin{itemize}
\item {Grp. gram.:adj.}
\end{itemize}
Relativo a túmulo. Cf. Garrett, \textunderscore Camões\textunderscore , 312.
\section{Tumulização}
\begin{itemize}
\item {Grp. gram.:f.}
\end{itemize}
Acto de \textunderscore tumulizar\textunderscore .
\section{Tumulizar}
\begin{itemize}
\item {Grp. gram.:v. t.}
\end{itemize}
\begin{itemize}
\item {Utilização:Neol.}
\end{itemize}
Encerrar em túmulo.
\section{Túmulo}
\begin{itemize}
\item {Grp. gram.:m.}
\end{itemize}
\begin{itemize}
\item {Utilização:Fig.}
\end{itemize}
\begin{itemize}
\item {Proveniência:(Lat. \textunderscore tumulus\textunderscore )}
\end{itemize}
Monumento em memória de alguém, no lugar onde está sepultado o indivíduo commemorado.
Eça.
Sepulcro.
Morte.
\section{Túnel}
\begin{itemize}
\item {Grp. gram.:m.}
\end{itemize}
\begin{itemize}
\item {Proveniência:(Ingl. \textunderscore tunnel\textunderscore )}
\end{itemize}
Caminho ou passagem subterrânea.
\section{Túnica}
\begin{itemize}
\item {Grp. gram.:f.}
\end{itemize}
\begin{itemize}
\item {Utilização:Anat.}
\end{itemize}
\begin{itemize}
\item {Utilização:Bot.}
\end{itemize}
\begin{itemize}
\item {Proveniência:(Lat. \textunderscore tunica\textunderscore )}
\end{itemize}
Vestuário antigo, comprido e ajustado ao corpo.
Dalmática.
Qualquer membrana, que fórma as paredes de um órgão ou contribue para a sua formação.
Membrana ou invólucro de certos órgãos vegetaes.
\section{Túnica}
\begin{itemize}
\item {Proveniência:(De \textunderscore tuno\textunderscore ?)}
\end{itemize}
\textunderscore m.\textunderscore  e \textunderscore f.\textunderscore  (?)
O mesmo que \textunderscore tunante\textunderscore ?:«\textunderscore ...só por pilhérica se diz entre túnicas.\textunderscore »Filinto, XIII, 304.
\section{Tunicado}
\begin{itemize}
\item {Grp. gram.:m.}
\end{itemize}
Espécie de mollusco.
\section{Tunicários}
\begin{itemize}
\item {Grp. gram.:m. pl.}
\end{itemize}
\begin{itemize}
\item {Proveniência:(De \textunderscore túnica\textunderscore )}
\end{itemize}
Ordem de molluscos muito pequenos, de invólucro coriáceo.
\section{Tuniceiros}
\begin{itemize}
\item {Grp. gram.:m. pl.}
\end{itemize}
\begin{itemize}
\item {Utilização:Zool.}
\end{itemize}
\begin{itemize}
\item {Proveniência:(De \textunderscore túnica\textunderscore )}
\end{itemize}
Ordem de molluscos muito pequenos, de invólucro coriáceo.
\section{Tunicela}
\begin{itemize}
\item {Grp. gram.:f.}
\end{itemize}
Túnica pequena.
Espécie de casula, que os Bispos usavam entre a vestimenta e a alva.
(Dem. de \textunderscore túnica\textunderscore )
\section{Tunicella}
\begin{itemize}
\item {Grp. gram.:f.}
\end{itemize}
Túnica pequena.
Espécie de casula, que os Bispos usavam entre a vestimenta e a alva.
(Dem. de \textunderscore túnica\textunderscore )
\section{Tuniciários}
\begin{itemize}
\item {Grp. gram.:m. pl.}
\end{itemize}
O mesmo que \textunderscore tuniceiros\textunderscore .
\section{Tunicíferos}
\begin{itemize}
\item {Grp. gram.:m. pl.}
\end{itemize}
O mesmo que \textunderscore tuniceiros\textunderscore .
\section{Tunicina}
\begin{itemize}
\item {Grp. gram.:f.}
\end{itemize}
\begin{itemize}
\item {Utilização:Chím.}
\end{itemize}
\begin{itemize}
\item {Proveniência:(De \textunderscore túnica\textunderscore )}
\end{itemize}
Princípio immediato do invólucro dos tuniceiros.
\section{Tuniquete}
\begin{itemize}
\item {fónica:quê}
\end{itemize}
\begin{itemize}
\item {Grp. gram.:m.}
\end{itemize}
O mesmo que \textunderscore tunicella\textunderscore .
\section{Tunisil}
\begin{itemize}
\item {Grp. gram.:m.  e  adj.}
\end{itemize}
\begin{itemize}
\item {Utilização:Ant.}
\end{itemize}
O mesmo que \textunderscore tunisino\textunderscore .
\section{Tunisino}
\begin{itemize}
\item {Grp. gram.:m.  e  adj.}
\end{itemize}
\begin{itemize}
\item {Proveniência:(De \textunderscore Túnis\textunderscore , n. p.)}
\end{itemize}
O mesmo que \textunderscore tunesino\textunderscore .
\section{Túnnel}
\begin{itemize}
\item {Grp. gram.:m.}
\end{itemize}
\begin{itemize}
\item {Proveniência:(Ingl. \textunderscore tunnel\textunderscore )}
\end{itemize}
Caminho ou passagem subterrânea.
\section{Tuno}
\begin{itemize}
\item {Grp. gram.:m.  e  adj.}
\end{itemize}
O mesmo que \textunderscore tunante\textunderscore :«\textunderscore não há naquella feira um tuno, um rato, um pilho\textunderscore ». Castilho, \textunderscore Sabichonas\textunderscore , 182.
\section{Tuo}
\begin{itemize}
\item {Grp. gram.:pron.}
\end{itemize}
\begin{itemize}
\item {Utilização:T. de Ceilão}
\end{itemize}
O mesmo que \textunderscore teu\textunderscore .
\section{Tuongonhe}
\begin{itemize}
\item {Grp. gram.:f.}
\end{itemize}
Grande mammífero africano, de chifres como os do boi e patas como as do veado.
\section{Tupáideos}
\begin{itemize}
\item {Grp. gram.:m. pl.}
\end{itemize}
Família de mammíferos insectívoros.
\section{Tuparubo}
\begin{itemize}
\item {Grp. gram.:m.}
\end{itemize}
\begin{itemize}
\item {Utilização:Bras}
\end{itemize}
O mesmo que \textunderscore caferana\textunderscore .
\section{Tupé}
\begin{itemize}
\item {Grp. gram.:m.}
\end{itemize}
\begin{itemize}
\item {Utilização:Bras. do N}
\end{itemize}
Esteira grande, em que se secam ao sol alguns productos de lavoira.
\section{Tupeia}
\begin{itemize}
\item {Grp. gram.:f.}
\end{itemize}
Gênero de plantas lorantháceas.
\section{Tupeiçava}
\begin{itemize}
\item {fónica:pe-i}
\end{itemize}
\begin{itemize}
\item {Grp. gram.:f.}
\end{itemize}
O mesmo que \textunderscore vassourinha-de-varrer\textunderscore .
\section{Tuperiba}
\begin{itemize}
\item {Grp. gram.:f.}
\end{itemize}
Árvore terebinthácea de Madagáscar.
\section{Tupi}
\begin{itemize}
\item {Grp. gram.:m.}
\end{itemize}
\begin{itemize}
\item {Grp. gram.:M. pl.}
\end{itemize}
Língua dos Tupis e dos Tupinambás.
Grande nação de Índios, que dominava nas costas da Guiana francesa e da Guiana brasileira, e se derramou depois por vários pontos do Brasil.
\section{Tupia}
\begin{itemize}
\item {Grp. gram.:f.}
\end{itemize}
\begin{itemize}
\item {Utilização:Bras}
\end{itemize}
Máquina, para fazer molduras.
\section{Tupi-guarani}
\begin{itemize}
\item {Grp. gram.:m.}
\end{itemize}
Língua, a mais espalhada entre os Índios do Brasil.
\section{Tupinamba}
\begin{itemize}
\item {Grp. gram.:f.}
\end{itemize}
\begin{itemize}
\item {Grp. gram.:Adj.}
\end{itemize}
Espécie de batata americana comprida.
Diz-se da língua dos Tupinambás. Cf. Herculano, \textunderscore Opúsc.\textunderscore , III, 192.
\section{Tupinambaranas}
\begin{itemize}
\item {Grp. gram.:m. pl.}
\end{itemize}
Tríbo de Índios, da nação dos Tupinambás.
\section{Tupinambás}
\begin{itemize}
\item {Grp. gram.:m. pl.}
\end{itemize}
Grande nação de Índios, que dominava ao norte do Brasil, quando os Portugueses alli entravam.--Os diccion. port. dizem erradamente \textunderscore tupinambas\textunderscore .
\section{Tupinambo}
\begin{itemize}
\item {Grp. gram.:m.}
\end{itemize}
Planta, da fam. das compostas, (\textunderscore helianthus tuberosus\textunderscore , Lin.), também conhecida por \textunderscore girasol batateiro\textunderscore .
(Cp. \textunderscore tupinamba\textunderscore )
\section{Tupinambor}
\begin{itemize}
\item {Grp. gram.:m.}
\end{itemize}
O mesmo que \textunderscore tupinamba\textunderscore .
\section{Tupinás}
\begin{itemize}
\item {Grp. gram.:m. pl.}
\end{itemize}
\begin{itemize}
\item {Utilização:Bras}
\end{itemize}
Tríbo de aborígenes da Baía e do Pará.
\section{Tupininquins}
\begin{itemize}
\item {Grp. gram.:m. pl.}
\end{itemize}
Antiga nação de Índios do Brasil, no território da Baía.
\section{Tupir}
\begin{itemize}
\item {Grp. gram.:v. t.}
\end{itemize}
\begin{itemize}
\item {Utilização:Prov.}
\end{itemize}
Entupir.
Tapar os poros de.
(Aphér. de \textunderscore entupir\textunderscore . Cp. cast. \textunderscore tupir\textunderscore )
\section{Tupistra}
\begin{itemize}
\item {Grp. gram.:f.}
\end{itemize}
Gênero de plantas esmiláceas.
\section{Turbinária}
\begin{itemize}
\item {Grp. gram.:f.}
\end{itemize}
\begin{itemize}
\item {Proveniência:(De \textunderscore turbina\textunderscore )}
\end{itemize}
Gênero de plantas phýceas.
\section{Turbinela}
\begin{itemize}
\item {Grp. gram.:f.}
\end{itemize}
\begin{itemize}
\item {Proveniência:(De \textunderscore turbina\textunderscore )}
\end{itemize}
Gênero de infusórios.
Gênero de molluscos pectinibrânchios.
\section{Turbiniforme}
\begin{itemize}
\item {Grp. gram.:adj.}
\end{itemize}
\begin{itemize}
\item {Proveniência:(Do lat. \textunderscore turbo\textunderscore  + \textunderscore forma\textunderscore )}
\end{itemize}
Que tem fórma cónica ou fórma de pião.
\section{Turbinólia}
\begin{itemize}
\item {Grp. gram.:f.}
\end{itemize}
Gênero de pólypos anthozoários.
\section{Turbinolopse}
\begin{itemize}
\item {Grp. gram.:m.}
\end{itemize}
Gênero de polypeiros fósseis.
\section{Turbinoso}
\begin{itemize}
\item {Grp. gram.:adj.}
\end{itemize}
\begin{itemize}
\item {Proveniência:(Do lat. \textunderscore turbo\textunderscore , \textunderscore turbinis\textunderscore )}
\end{itemize}
Semelhante a um turbilhão.
Que gira em volta de um eixo ou centro.
\section{Turbito}
\begin{itemize}
\item {Grp. gram.:m.}
\end{itemize}
\begin{itemize}
\item {Proveniência:(Do ár. \textunderscore turbia\textunderscore )}
\end{itemize}
Planta convolvulácea, da raíz purgativa.
\section{Turbulência}
\begin{itemize}
\item {Grp. gram.:f.}
\end{itemize}
\begin{itemize}
\item {Proveniência:(Lat. \textunderscore turbulentia\textunderscore )}
\end{itemize}
Qualidade do que é turbulento.
Acto turbulento.
Grande desordem, motim.
\section{Turbulento}
\begin{itemize}
\item {Grp. gram.:adj.}
\end{itemize}
\begin{itemize}
\item {Grp. gram.:M.}
\end{itemize}
\begin{itemize}
\item {Proveniência:(Lat. \textunderscore turbulentus\textunderscore )}
\end{itemize}
Que tem disposição para desordem ou motim.
Que se compraz na desordem ou na sedição.
Em que há tumulto ou perturbação da ordem: \textunderscore assembleia turbulenta\textunderscore .
Tumultuoso; buliçoso; agitado.
Indivíduo turbulento.
\section{Turca}
\begin{itemize}
\item {Grp. gram.:f.}
\end{itemize}
\begin{itemize}
\item {Utilização:Pop.}
\end{itemize}
\begin{itemize}
\item {Proveniência:(De \textunderscore turco\textunderscore )}
\end{itemize}
O mesmo que \textunderscore embriaguez\textunderscore .
\section{Turcamente}
\begin{itemize}
\item {Grp. gram.:adv.}
\end{itemize}
Á moda dos Turcos; á maneira de turco.
\section{Turchimão}
\begin{itemize}
\item {Grp. gram.:m.}
\end{itemize}
\begin{itemize}
\item {Utilização:Des.}
\end{itemize}
O mesmo que \textunderscore turgimão\textunderscore .
\section{Túrcica}
\begin{itemize}
\item {Grp. gram.:adj.}
\end{itemize}
\begin{itemize}
\item {Utilização:Anat.}
\end{itemize}
Chama-se \textunderscore sella túrcica\textunderscore  a cavidade ou fossa, onde assenta a glândula pituitária.
\section{Turco}
\begin{itemize}
\item {Grp. gram.:adj.}
\end{itemize}
\begin{itemize}
\item {Grp. gram.:M.}
\end{itemize}
\begin{itemize}
\item {Utilização:Náut.}
\end{itemize}
Relativo á Turquia.
Habitante da Turquia.
Língua, falada pelos Turcos.
Peça de madeira ou de ferro, saliente do costado do navio, e que serve para suspender escaleres e as âncoras.--Se esta peça é de madeira chama-se turco de lambareiro e é o mesmo que \textunderscore serviola\textunderscore .
\section{Turcófilo}
\begin{itemize}
\item {Grp. gram.:adj.}
\end{itemize}
\begin{itemize}
\item {Proveniência:(De \textunderscore turco\textunderscore  + gr. \textunderscore philos\textunderscore )}
\end{itemize}
Que é amigo dos Turcos; afeiçoado á Turquia.
\section{Turcomano}
\begin{itemize}
\item {Grp. gram.:m.}
\end{itemize}
\begin{itemize}
\item {Grp. gram.:M. Pl.}
\end{itemize}
Língua uralo-altaica, vernácula na Tartária e na Turquia.
Povo de raça turca, espalhado no Turquestão e nas vizinhanças do Cáucaso.
(Do pers.)
\section{Turcóphilo}
\begin{itemize}
\item {Grp. gram.:adj.}
\end{itemize}
\begin{itemize}
\item {Proveniência:(De \textunderscore turco\textunderscore  + gr. \textunderscore philos\textunderscore )}
\end{itemize}
Que é amigo dos Turcos; affeiçoado á Turquia.
\section{Turcópolo}
\begin{itemize}
\item {Grp. gram.:m.}
\end{itemize}
Nome que se dá no Levante ao filho de pai turco e mãe grega.
Cada um dos soldados de um corpo de cavallaria turca, que se formava antigamente com os filhos de mulheres gregas.
\section{Turco-tártaro}
\begin{itemize}
\item {Grp. gram.:adj.}
\end{itemize}
Relativo a Turcos e Tártaros.
\section{Turdetano}
\begin{itemize}
\item {Grp. gram.:adj.}
\end{itemize}
\begin{itemize}
\item {Grp. gram.:M.}
\end{itemize}
\begin{itemize}
\item {Proveniência:(Lat. \textunderscore turdetani\textunderscore )}
\end{itemize}
Relativo á Turdetânia, antiga província da Península Hispânica.
Habitante da Turdetânia. Cf. Herculano, \textunderscore Hist. de Port.\textunderscore , I, 14 e 42.
\section{Turdídeo}
\begin{itemize}
\item {Grp. gram.:adj.}
\end{itemize}
\begin{itemize}
\item {Grp. gram.:M. Pl.}
\end{itemize}
\begin{itemize}
\item {Proveniência:(Do lat. \textunderscore turdus\textunderscore )}
\end{itemize}
Relativo ou semelhante ao tordo.
Família de pássaros, que tem por typo o melro.
\section{Turdilho}
\begin{itemize}
\item {Grp. gram.:adj.}
\end{itemize}
(V.tordilho)
\section{Túrdulos}
\begin{itemize}
\item {Grp. gram.:m. Pl.}
\end{itemize}
\begin{itemize}
\item {Proveniência:(Lat. \textunderscore turduli\textunderscore )}
\end{itemize}
Antigo povo da Bética, a léste dos Turdetanos. Cf. Herculano, \textunderscore Hist. de Port.\textunderscore , I, 14 e 42.
\section{Tureba}
\begin{itemize}
\item {Grp. gram.:m.}
\end{itemize}
\begin{itemize}
\item {Utilização:Bras}
\end{itemize}
Valentão.
\section{Turfa}
\begin{itemize}
\item {Grp. gram.:f.}
\end{itemize}
\begin{itemize}
\item {Proveniência:(Do ingl. \textunderscore turf\textunderscore )}
\end{itemize}
Espécie de carvão, formado pela decomposição de substâncias vegetaes, análogo á madeira fóssil carbonizada.
\section{Turfeira}
\begin{itemize}
\item {Grp. gram.:f.}
\end{itemize}
Jazigo de turfa.
\section{Turfoso}
\begin{itemize}
\item {Grp. gram.:adj.}
\end{itemize}
Em que há turfa: \textunderscore terreno turfoso\textunderscore .
\section{Turgência}
\begin{itemize}
\item {Grp. gram.:f.}
\end{itemize}
\begin{itemize}
\item {Proveniência:(De \textunderscore turgente\textunderscore )}
\end{itemize}
O mesmo que \textunderscore turgidez\textunderscore .
\section{Turibular}
\begin{itemize}
\item {Grp. gram.:v. t.}
\end{itemize}
\begin{itemize}
\item {Utilização:Fig.}
\end{itemize}
\begin{itemize}
\item {Proveniência:(De \textunderscore turíbulo\textunderscore )}
\end{itemize}
Queimar incenso, em honra de.
Lisonjear, adular:«\textunderscore ...é turibular a ignorância dos povos.\textunderscore »D. Ant. da Costa, \textunderscore Três Mundos\textunderscore , 16.
\section{Turibulário}
\begin{itemize}
\item {Grp. gram.:m.  e  adj.}
\end{itemize}
\begin{itemize}
\item {Utilização:Fig.}
\end{itemize}
\begin{itemize}
\item {Proveniência:(De \textunderscore turíbulo\textunderscore )}
\end{itemize}
O que agita o turíbulo para incensar.
Adulador.
\section{Turíbulo}
\begin{itemize}
\item {Grp. gram.:m.}
\end{itemize}
\begin{itemize}
\item {Proveniência:(Lat. \textunderscore thuribulum\textunderscore )}
\end{itemize}
Vaso, em que se queima incenso.
\section{Turícremo}
\begin{itemize}
\item {Grp. gram.:adj.}
\end{itemize}
\begin{itemize}
\item {Utilização:Poét.}
\end{itemize}
\begin{itemize}
\item {Proveniência:(Lat. \textunderscore thuricremus\textunderscore )}
\end{itemize}
Em que se queima incenso. Cf. F. Barreto, \textunderscore Eneida\textunderscore , IV, 103.
\section{Turiferário}
\begin{itemize}
\item {Grp. gram.:m.  e  adj.}
\end{itemize}
\begin{itemize}
\item {Proveniência:(De \textunderscore turífero\textunderscore )}
\end{itemize}
O que leva o turíbulo.
\section{Turífero}
\begin{itemize}
\item {Grp. gram.:adj.}
\end{itemize}
\begin{itemize}
\item {Proveniência:(Lat. \textunderscore thuriferus\textunderscore )}
\end{itemize}
Que produz incenso.
\section{Turificação}
\begin{itemize}
\item {Grp. gram.:f.}
\end{itemize}
\begin{itemize}
\item {Proveniência:(Lat. \textunderscore thurificatio\textunderscore )}
\end{itemize}
Acto ou efeito de turificar.
\section{Turificador}
\begin{itemize}
\item {Grp. gram.:m.  e  adj.}
\end{itemize}
\begin{itemize}
\item {Proveniência:(Do lat. \textunderscore thurificator\textunderscore )}
\end{itemize}
O que turifica.
\section{Turificante}
\begin{itemize}
\item {Grp. gram.:adj.}
\end{itemize}
\begin{itemize}
\item {Proveniência:(Lat. \textunderscore thurificans\textunderscore )}
\end{itemize}
Que turifica.
\section{Turificar}
\begin{itemize}
\item {Grp. gram.:v. t.}
\end{itemize}
\begin{itemize}
\item {Proveniência:(Lat. \textunderscore thurificare\textunderscore )}
\end{itemize}
O mesmo que \textunderscore incensar\textunderscore , em sentido próprio e figurado.
\section{Turíngia}
\begin{itemize}
\item {Grp. gram.:f.}
\end{itemize}
O mesmo que \textunderscore toronja\textunderscore .
\section{Turíngios}
\begin{itemize}
\item {Grp. gram.:m. Pl.}
\end{itemize}
\begin{itemize}
\item {Proveniência:(Lat. \textunderscore thuringi\textunderscore )}
\end{itemize}
Tríbo combatente nas guerras góticas da Espanha. Cf. Herculano, \textunderscore Eurico\textunderscore , c. IV.
\section{Turpitude}
\begin{itemize}
\item {Grp. gram.:f.}
\end{itemize}
\begin{itemize}
\item {Proveniência:(Lat. \textunderscore turpitudo\textunderscore )}
\end{itemize}
O mesmo que \textunderscore torpeza\textunderscore . Cf. Camillo, \textunderscore Doze Casam.\textunderscore , 234.
\section{Turquesa}
\begin{itemize}
\item {Grp. gram.:f.}
\end{itemize}
\begin{itemize}
\item {Proveniência:(De \textunderscore turco\textunderscore )}
\end{itemize}
Pedra preciosa, de côr azul, sem transparência.
\section{Turquesado}
\begin{itemize}
\item {Grp. gram.:adj.}
\end{itemize}
Que tem côr de turquesa.
\section{Turquesco}
\begin{itemize}
\item {fónica:quês}
\end{itemize}
\begin{itemize}
\item {Grp. gram.:adj.}
\end{itemize}
Relativo aos Turcos.
Feito á maneira dos Turcos.
\section{Turqui}
\begin{itemize}
\item {Grp. gram.:adj.}
\end{itemize}
\begin{itemize}
\item {Proveniência:(Do it. \textunderscore turchino\textunderscore )}
\end{itemize}
Diz-se do azul carregado e sem brilho.
\section{Turquimão}
\begin{itemize}
\item {Grp. gram.:m.}
\end{itemize}
\begin{itemize}
\item {Utilização:Ant.}
\end{itemize}
O mesmo que \textunderscore turcomano\textunderscore .
Pl. \textunderscore turquimães\textunderscore :«\textunderscore esta cidade é habitada de persianos e alguns turquimães, gente alva...\textunderscore »Tenreiro, \textunderscore Itiner.\textunderscore , c. XV.
\section{Turquina}
\begin{itemize}
\item {Grp. gram.:f.}
\end{itemize}
\begin{itemize}
\item {Proveniência:(De \textunderscore turqui\textunderscore )}
\end{itemize}
Espécie de turquesa ordinária.
\section{Turra}
\begin{itemize}
\item {Grp. gram.:f.}
\end{itemize}
\begin{itemize}
\item {Utilização:Fam.}
\end{itemize}
Pancada com a testa.
Altercação.
Teima, caturrice.
\section{Turrante}
\begin{itemize}
\item {Grp. gram.:adj.}
\end{itemize}
Que turra, que teima. Cf. F. Recreio, \textunderscore Bat. de Ourique\textunderscore .
\section{Turrão}
\begin{itemize}
\item {Grp. gram.:adj.}
\end{itemize}
\begin{itemize}
\item {Utilização:Pop.}
\end{itemize}
\begin{itemize}
\item {Proveniência:(De \textunderscore turra\textunderscore )}
\end{itemize}
Teimoso; pertinaz.
\section{Turrar}
\begin{itemize}
\item {Grp. gram.:v. i.}
\end{itemize}
\begin{itemize}
\item {Utilização:Fam.}
\end{itemize}
\begin{itemize}
\item {Proveniência:(De \textunderscore turra\textunderscore )}
\end{itemize}
Bater com a testa.
Altercar; caturrar.
\section{Túrrea}
\begin{itemize}
\item {Grp. gram.:f.}
\end{itemize}
Gênero de plantas meliáceas.
\section{Turriculado}
\begin{itemize}
\item {Grp. gram.:adj.}
\end{itemize}
\begin{itemize}
\item {Proveniência:(Do lat. \textunderscore turricula\textunderscore )}
\end{itemize}
Que tem a espiral muito alongada, (falando-se de certas conchas univalves).
\section{Turrífrago}
\begin{itemize}
\item {Grp. gram.:adj.}
\end{itemize}
\begin{itemize}
\item {Utilização:Poét.}
\end{itemize}
\begin{itemize}
\item {Proveniência:(Do lat. \textunderscore turris\textunderscore  + \textunderscore frangere\textunderscore )}
\end{itemize}
Que destrói tôrres.
\section{Turrígera}
\begin{itemize}
\item {Grp. gram.:f.}
\end{itemize}
Gênero de plantas asclepiadáceas.
(Cp. \textunderscore turrígero\textunderscore )
\section{Turrígero}
\begin{itemize}
\item {Grp. gram.:adj.}
\end{itemize}
\begin{itemize}
\item {Utilização:Poét.}
\end{itemize}
\begin{itemize}
\item {Proveniência:(Lat. \textunderscore turriger\textunderscore )}
\end{itemize}
Que tem tôrre ou castello.
\section{Túrrio}
\begin{itemize}
\item {Grp. gram.:m.  e  adj.}
\end{itemize}
\begin{itemize}
\item {Utilização:Prov.}
\end{itemize}
\begin{itemize}
\item {Utilização:beir.}
\end{itemize}
Homem teimoso, caturra.
(Cp. \textunderscore turra\textunderscore )
\section{Turrista}
\begin{itemize}
\item {Grp. gram.:m.  e  f.}
\end{itemize}
Pessôa, que turra muito.
\section{Turritela}
\begin{itemize}
\item {Grp. gram.:f.}
\end{itemize}
Gênero de molluscos pectinibrânchios.
\section{Turturila}
\begin{itemize}
\item {Grp. gram.:f.}
\end{itemize}
O mesmo que \textunderscore turturilha\textunderscore .
\section{Turturilha}
\begin{itemize}
\item {Grp. gram.:f.}
\end{itemize}
\begin{itemize}
\item {Utilização:Ant.}
\end{itemize}
\begin{itemize}
\item {Proveniência:(Lat. \textunderscore turturilla\textunderscore )}
\end{itemize}
Pequena rôla, rolinha:«\textunderscore eu serei a turturilha que lhe morre companheira\textunderscore ». Canção de D. Pedro I a Inês de Castro, no \textunderscore Cancion.\textunderscore  de Resende.
\section{Turturilla}
\begin{itemize}
\item {Grp. gram.:f.}
\end{itemize}
O mesmo que \textunderscore turturilha\textunderscore .
\section{Turturinar}
\begin{itemize}
\item {Grp. gram.:v. i.}
\end{itemize}
\begin{itemize}
\item {Utilização:bras}
\end{itemize}
\begin{itemize}
\item {Utilização:Neol.}
\end{itemize}
\begin{itemize}
\item {Proveniência:(De \textunderscore turturino\textunderscore )}
\end{itemize}
O mesmo que \textunderscore arrulhar\textunderscore .
\section{Turturino}
\begin{itemize}
\item {Grp. gram.:adj.}
\end{itemize}
\begin{itemize}
\item {Utilização:Poét.}
\end{itemize}
\begin{itemize}
\item {Proveniência:(Do lat. \textunderscore turtur\textunderscore )}
\end{itemize}
Relativo á rôla.
\section{Turulangila}
\begin{itemize}
\item {Grp. gram.:m.}
\end{itemize}
Espécie de reptis angolenses, (\textunderscore naja Anchietae\textunderscore ).
\section{Turumbamba}
\begin{itemize}
\item {Grp. gram.:m.}
\end{itemize}
\begin{itemize}
\item {Utilização:Bras. do N}
\end{itemize}
Altercação.
Desordem, balbúrdia.
\section{Turuna}
\begin{itemize}
\item {Grp. gram.:m.  e  adj.}
\end{itemize}
\begin{itemize}
\item {Utilização:Bras}
\end{itemize}
Homem valente.
\section{Tururi}
\begin{itemize}
\item {Grp. gram.:m.}
\end{itemize}
\begin{itemize}
\item {Utilização:Bras. do N}
\end{itemize}
Grande árvore myrtácea do Brasil.
Empatha de uma espécie de palmeira.
\section{Tururis}
\begin{itemize}
\item {Grp. gram.:m. pl.}
\end{itemize}
Indígenas do norte do Brasil.
\section{Turvação}
\begin{itemize}
\item {Grp. gram.:f.}
\end{itemize}
\begin{itemize}
\item {Proveniência:(Do lat. \textunderscore turbatio\textunderscore )}
\end{itemize}
Acto ou effeito de turvar.
Doença dos vinhos.
\section{Turvador}
\begin{itemize}
\item {Grp. gram.:adj.}
\end{itemize}
Que turva, que causa turvação.
\section{Turvamento}
\begin{itemize}
\item {Grp. gram.:m.}
\end{itemize}
\begin{itemize}
\item {Proveniência:(Lat. \textunderscore turbamentum\textunderscore )}
\end{itemize}
O mesmo que \textunderscore turvação\textunderscore .
\section{Turvar}
\begin{itemize}
\item {Grp. gram.:v. t.}
\end{itemize}
\begin{itemize}
\item {Grp. gram.:V. i.}
\end{itemize}
\begin{itemize}
\item {Proveniência:(Lat. \textunderscore turbare\textunderscore )}
\end{itemize}
Tornar turvo ou opaco.
Embaciar.
Escurecer.
Perturbar.
Transtornar.
Embriagar.
Tornar-se turvo ou tôrvo.
Tornar-se carrancudo.
\section{Turvejar}
\begin{itemize}
\item {Grp. gram.:v. i.  e  p.}
\end{itemize}
Tornar-se turvo.
\section{Turvi}
\begin{itemize}
\item {Grp. gram.:m.}
\end{itemize}
Planta da serra de Sintra.
\section{Turvo}
\begin{itemize}
\item {Grp. gram.:adj.}
\end{itemize}
\begin{itemize}
\item {Grp. gram.:M.}
\end{itemize}
\begin{itemize}
\item {Proveniência:(Lat. \textunderscore turbidus\textunderscore )}
\end{itemize}
Que não é transparente; opaco; toldado.
Agitado; confuso.
O mesmo que \textunderscore turvação\textunderscore .
\section{Tilanto}
\begin{itemize}
\item {Grp. gram.:m.}
\end{itemize}
\begin{itemize}
\item {Proveniência:(Do gr. \textunderscore tulos\textunderscore  + \textunderscore anthos\textunderscore )}
\end{itemize}
Arbusto ramnáceo da África do Sul.
\section{Tilófora}
\begin{itemize}
\item {Grp. gram.:f.}
\end{itemize}
\begin{itemize}
\item {Proveniência:(Do gr. \textunderscore tulos\textunderscore  + \textunderscore phoros\textunderscore )}
\end{itemize}
Gênero de plantas asclepiadáceas.
\section{Tiloma}
\begin{itemize}
\item {Grp. gram.:f.}
\end{itemize}
\begin{itemize}
\item {Utilização:Med.}
\end{itemize}
\begin{itemize}
\item {Proveniência:(Gr. \textunderscore tuloma\textunderscore )}
\end{itemize}
O mesmo que \textunderscore calo\textunderscore .
\section{Tilose}
\begin{itemize}
\item {Grp. gram.:f.}
\end{itemize}
\begin{itemize}
\item {Utilização:Med.}
\end{itemize}
\begin{itemize}
\item {Proveniência:(Do gr. \textunderscore tulosis\textunderscore )}
\end{itemize}
Pequeno calo nos pés, também conhecido por \textunderscore ôlho de perdiz\textunderscore .
Calosidade em geral.
\section{Timpanal}
\begin{itemize}
\item {Grp. gram.:adj.}
\end{itemize}
\begin{itemize}
\item {Grp. gram.:M.}
\end{itemize}
Relativo ao tímpano.
Osso timpanal.
\section{Timpanicida}
\begin{itemize}
\item {Grp. gram.:adj.}
\end{itemize}
\begin{itemize}
\item {Utilização:Neol.}
\end{itemize}
\begin{itemize}
\item {Proveniência:(Do lat. \textunderscore tympanum\textunderscore  + \textunderscore caedere\textunderscore )}
\end{itemize}
Que sôa mal ao ouvido, que o fere desagradavelmente. Cf. Camillo, \textunderscore Narcót.\textunderscore , I, 284.
\section{Timpânico}
\begin{itemize}
\item {Grp. gram.:adj.}
\end{itemize}
\begin{itemize}
\item {Proveniência:(De \textunderscore tímpano\textunderscore )}
\end{itemize}
O mesmo que \textunderscore timpanal\textunderscore ; timpanítico.
\section{Timpanilho}
\begin{itemize}
\item {Grp. gram.:m.}
\end{itemize}
Caixilho de ferro, recoberto de estôfo, que se encaixa na parte postero-interior do tímpano do prelo, para segurar a almofada.
(Cast. \textunderscore timpanillo\textunderscore )
\section{Timpanismo}
\begin{itemize}
\item {Grp. gram.:m.}
\end{itemize}
O mesmo que \textunderscore timpanite\textunderscore .
\section{Timpanista}
\begin{itemize}
\item {Grp. gram.:m.}
\end{itemize}
\begin{itemize}
\item {Proveniência:(Lat. \textunderscore tympanista\textunderscore )}
\end{itemize}
Tocador de tambor, entre os antigos Gregos e Romanos.
\section{Timpanite}
\begin{itemize}
\item {Grp. gram.:f.}
\end{itemize}
\begin{itemize}
\item {Proveniência:(Lat. \textunderscore tympanites\textunderscore )}
\end{itemize}
Entumecência no ventre, produzida por excessiva acumulação de gases no canal digestivo.
\section{Timpanítico}
\begin{itemize}
\item {Grp. gram.:adj.}
\end{itemize}
\begin{itemize}
\item {Proveniência:(Lat. \textunderscore tympaniticus\textunderscore )}
\end{itemize}
Relativo á timpanite.
\section{Timpanização}
\begin{itemize}
\item {Grp. gram.:f.}
\end{itemize}
\begin{itemize}
\item {Proveniência:(De \textunderscore timpanizar\textunderscore )}
\end{itemize}
O mesmo que \textunderscore timpanite\textunderscore .
\section{Timpanizar}
\begin{itemize}
\item {Grp. gram.:v. t.}
\end{itemize}
\begin{itemize}
\item {Proveniência:(Lat. \textunderscore tympanizare\textunderscore )}
\end{itemize}
Produzir timpanite em.
\section{Tipocromia}
\begin{itemize}
\item {Grp. gram.:f.}
\end{itemize}
\begin{itemize}
\item {Proveniência:(Do gr. \textunderscore tupos\textunderscore  + \textunderscore khroma\textunderscore )}
\end{itemize}
Impressão tipográfica, a côres.
\section{Tipofonia}
\begin{itemize}
\item {Grp. gram.:f.}
\end{itemize}
\begin{itemize}
\item {Utilização:Mús.}
\end{itemize}
Arte ou modo de marcar a voz ou o compasso, batendo.
(Cp. \textunderscore tipofónio\textunderscore )
\section{Tipofónio}
\begin{itemize}
\item {Grp. gram.:m.}
\end{itemize}
\begin{itemize}
\item {Proveniência:(Do gr. \textunderscore tupos\textunderscore  + \textunderscore phone\textunderscore )}
\end{itemize}
Instrumento músico que produz sons simples e invariáveis.
\section{Tipógrafa}
\begin{itemize}
\item {Grp. gram.:f.}
\end{itemize}
Máquina de composição tipográfica, inventada em 1898, na América do Norte.
\section{Tíquio}
\begin{itemize}
\item {Grp. gram.:m.}
\end{itemize}
Gênero de insectos coleópteros tetrâmeros.
\section{Tutu}
\begin{itemize}
\item {Grp. gram.:m.}
\end{itemize}
\begin{itemize}
\item {Utilização:Bras}
\end{itemize}
Ente imaginário, com que se põe medo ás crianças.
Chefe local; influente; magnate.
\section{Tutu}
\begin{itemize}
\item {Grp. gram.:m.}
\end{itemize}
\begin{itemize}
\item {Utilização:Bras. do Rio}
\end{itemize}
Iguaria, que consta de feijão cozido, misturado com farinha de mandioca ou de milho.
\section{Tutucar}
\begin{itemize}
\item {Grp. gram.:v. i.}
\end{itemize}
\begin{itemize}
\item {Utilização:Bras}
\end{itemize}
Produzir som surdo: \textunderscore tutucar em tamborete\textunderscore ; \textunderscore o tutucar dos atabales\textunderscore . Cf. Júl. Ribeiro, \textunderscore Carne\textunderscore .
\section{Tútulo}
\begin{itemize}
\item {Grp. gram.:m.}
\end{itemize}
\begin{itemize}
\item {Proveniência:(Lat. \textunderscore tutulus\textunderscore )}
\end{itemize}
Barrete, que, entre os Romanos, era usado pelos flâmines. Cf. Castilho, \textunderscore Fastos\textunderscore , II, 602.
\section{Tutuque}
\begin{itemize}
\item {Grp. gram.:m.}
\end{itemize}
Acto de tutucar.
\section{Tuturubá}
\begin{itemize}
\item {Grp. gram.:m.}
\end{itemize}
Árvore, o mesmo que \textunderscore cutitiribá\textunderscore .
\section{Tuxaua}
\begin{itemize}
\item {fónica:tuxáua,tuxaúa}
\end{itemize}
\begin{itemize}
\item {Grp. gram.:m.}
\end{itemize}
\begin{itemize}
\item {Utilização:Bras}
\end{itemize}
\begin{itemize}
\item {Proveniência:(T. tupi)}
\end{itemize}
Chefe de uma tribo de Aborígenes.
\section{Tuxi}
\begin{itemize}
\item {Grp. gram.:m.}
\end{itemize}
Ornato, usado ao pescoço por bailadeiras indianas. Cf. Th. Ribeiro, \textunderscore Jornadas\textunderscore , II, 104.
\section{Tuz}
\begin{itemize}
\item {Grp. gram.:m.}
\end{itemize}
O mesmo que \textunderscore chus\textunderscore ^3:«\textunderscore nem tuz nem buz.\textunderscore »Filinto, XIII, 314.
\section{Tuza}
\begin{itemize}
\item {Grp. gram.:f.}
\end{itemize}
Arbusto de Moçambique.
\section{Týchio}
\begin{itemize}
\item {fónica:qui}
\end{itemize}
\begin{itemize}
\item {Grp. gram.:m.}
\end{itemize}
Gênero de insectos coleópteros tetrâmeros.
\section{Tylantho}
\begin{itemize}
\item {Grp. gram.:m.}
\end{itemize}
\begin{itemize}
\item {Proveniência:(Do gr. \textunderscore tulos\textunderscore  + \textunderscore anthos\textunderscore )}
\end{itemize}
Arbusto rhamnáceo da África do Sul.
\section{Tyloma}
\begin{itemize}
\item {Grp. gram.:f.}
\end{itemize}
\begin{itemize}
\item {Utilização:Med.}
\end{itemize}
\begin{itemize}
\item {Proveniência:(Gr. \textunderscore tuloma\textunderscore )}
\end{itemize}
O mesmo que \textunderscore callo\textunderscore .
\section{Tylóphora}
\begin{itemize}
\item {Grp. gram.:f.}
\end{itemize}
\begin{itemize}
\item {Proveniência:(Do gr. \textunderscore tulos\textunderscore  + \textunderscore phoros\textunderscore )}
\end{itemize}
Gênero de plantas asclepiadáceas.
\section{Tylose}
\begin{itemize}
\item {Grp. gram.:f.}
\end{itemize}
\begin{itemize}
\item {Utilização:Med.}
\end{itemize}
\begin{itemize}
\item {Proveniência:(Do gr. \textunderscore tulosis\textunderscore )}
\end{itemize}
Pequeno callo nos pés, também conhecido por \textunderscore ôlho de perdiz\textunderscore .
Callosidade em geral.
\section{Tympanal}
\begin{itemize}
\item {Grp. gram.:adj.}
\end{itemize}
\begin{itemize}
\item {Grp. gram.:M.}
\end{itemize}
Relativo ao týmpano.
Osso tympanal.
\section{Tympanicida}
\begin{itemize}
\item {Grp. gram.:adj.}
\end{itemize}
\begin{itemize}
\item {Utilização:Neol.}
\end{itemize}
\begin{itemize}
\item {Proveniência:(Do lat. \textunderscore tympanum\textunderscore  + \textunderscore caedere\textunderscore )}
\end{itemize}
Que sôa mal ao ouvido, que o fere desagradavelmente. Cf. Camillo, \textunderscore Narcót.\textunderscore , I, 284.
\section{Tympânico}
\begin{itemize}
\item {Grp. gram.:adj.}
\end{itemize}
\begin{itemize}
\item {Proveniência:(De \textunderscore týmpano\textunderscore )}
\end{itemize}
O mesmo que \textunderscore tympanal\textunderscore ; tympanítico.
\section{Tympanilho}
\begin{itemize}
\item {Grp. gram.:m.}
\end{itemize}
Caixilho de ferro, recoberto de estôfo, que se encaixa na parte postero-interior do týmpano do prelo, para segurar a almofada.
(Cast. \textunderscore timpanillo\textunderscore )
\section{Tympanismo}
\begin{itemize}
\item {Grp. gram.:m.}
\end{itemize}
O mesmo que \textunderscore tympanite\textunderscore .
\section{Tympanista}
\begin{itemize}
\item {Grp. gram.:m.}
\end{itemize}
\begin{itemize}
\item {Proveniência:(Lat. \textunderscore tympanista\textunderscore )}
\end{itemize}
Tocador de tambor, entre os antigos Gregos e Romanos.
\section{Tympanite}
\begin{itemize}
\item {Grp. gram.:f.}
\end{itemize}
\begin{itemize}
\item {Proveniência:(Lat. \textunderscore tympanites\textunderscore )}
\end{itemize}
Entumecência no ventre, produzida por excessiva accumulação de gases no canal digestivo.
\section{Tympanítico}
\begin{itemize}
\item {Grp. gram.:adj.}
\end{itemize}
\begin{itemize}
\item {Proveniência:(Lat. \textunderscore tympaniticus\textunderscore )}
\end{itemize}
Relativo á tympanite.
\section{Tympanização}
\begin{itemize}
\item {Grp. gram.:f.}
\end{itemize}
\begin{itemize}
\item {Proveniência:(De \textunderscore tympanizar\textunderscore )}
\end{itemize}
O mesmo que \textunderscore tympanite\textunderscore .
\section{Tympanizar}
\begin{itemize}
\item {Grp. gram.:v. t.}
\end{itemize}
\begin{itemize}
\item {Proveniência:(Lat. \textunderscore tympanizare\textunderscore )}
\end{itemize}
Produzir tympanite em.
\section{Tipografia}
\begin{itemize}
\item {Grp. gram.:f.}
\end{itemize}
Arte de imprimir.
Estabelecimento tipografico.
(Cp. \textunderscore tipógrafo\textunderscore )
\section{Tipografar}
\begin{itemize}
\item {Grp. gram.:v. t.}
\end{itemize}
\begin{itemize}
\item {Proveniência:(De \textunderscore tipógrafo\textunderscore )}
\end{itemize}
Reproduzir pela tipografia; imprimir.
\section{Tipográfico}
\begin{itemize}
\item {Grp. gram.:adj.}
\end{itemize}
Relativo á tipografia: \textunderscore bôa execução tipográfica\textunderscore .
\section{Tipógrafo}
\begin{itemize}
\item {Grp. gram.:m.}
\end{itemize}
\begin{itemize}
\item {Proveniência:(Do gr. \textunderscore tupos\textunderscore  + \textunderscore graphein\textunderscore )}
\end{itemize}
Aquele que sabe ou exerce a arte tipográfica.
\section{Tipograficamente}
\begin{itemize}
\item {Grp. gram.:adv.}
\end{itemize}
De modo tipográfico; por meio da tipografia.
\section{Tipóia}
\begin{itemize}
\item {Grp. gram.:f.}
\end{itemize}
\begin{itemize}
\item {Utilização:Chul.}
\end{itemize}
\begin{itemize}
\item {Proveniência:(De \textunderscore tipo\textunderscore )}
\end{itemize}
Mulhér ordinária, desprezível.
\section{Tipólito}
\begin{itemize}
\item {Grp. gram.:m.}
\end{itemize}
\begin{itemize}
\item {Utilização:Miner.}
\end{itemize}
\begin{itemize}
\item {Proveniência:(Do gr. \textunderscore tupos\textunderscore  + \textunderscore lithos\textunderscore )}
\end{itemize}
Pedra, que tem impressa a fórma de algumas plantas ou animaes.
\section{Tipolitografia}
\begin{itemize}
\item {Grp. gram.:f.}
\end{itemize}
\begin{itemize}
\item {Proveniência:(De \textunderscore tipografia\textunderscore  e \textunderscore litografia\textunderscore )}
\end{itemize}
Arte de imprimir na mesma fôlha desenhos litográficos e caracteres tipográficos.
\section{Tipometria}
\begin{itemize}
\item {Grp. gram.:f.}
\end{itemize}
\begin{itemize}
\item {Proveniência:(Do gr. \textunderscore tupos\textunderscore  + \textunderscore metron\textunderscore )}
\end{itemize}
Arte de compor, por meio de filetes recortados e caracteres móveis, certos desenhos que se imprimem tipograficamente.
\section{Tipómetro}
\begin{itemize}
\item {Grp. gram.:m.}
\end{itemize}
\begin{itemize}
\item {Proveniência:(Do gr. \textunderscore tupos\textunderscore  + \textunderscore metron\textunderscore )}
\end{itemize}
Instrumento de fundição tipográfica, que serve para verificar se as letras estão na sua altura, e se têm o corpo desejado.
\section{Tiptologia}
\begin{itemize}
\item {Grp. gram.:f.}
\end{itemize}
\begin{itemize}
\item {Utilização:Espir.}
\end{itemize}
\begin{itemize}
\item {Proveniência:(Do gr. \textunderscore tuptein\textunderscore  + \textunderscore logos\textunderscore )}
\end{itemize}
Experiência, a que procedem os espiritistas com mesas girantes, chapéus, peneiras, etc.
Comunicação dos espíritos por meio de pancadas.
\section{Tiptológico}
\begin{itemize}
\item {Grp. gram.:adj.}
\end{itemize}
Relativo á tiptologia.
\section{Tiptólogo}
\begin{itemize}
\item {Grp. gram.:m.}
\end{itemize}
\begin{itemize}
\item {Utilização:Espir.}
\end{itemize}
Médium, que está apto para praticar a tiptologia.
(Cp. \textunderscore tiptologia\textunderscore )
\section{Tirana}
\begin{itemize}
\item {Grp. gram.:f.}
\end{itemize}
\begin{itemize}
\item {Utilização:Fam.}
\end{itemize}
\begin{itemize}
\item {Proveniência:(Lat. \textunderscore tyranna\textunderscore )}
\end{itemize}
Mulhér esquiva ou má.
Mulhér cruel.
Dança brasileira.
\section{Tiranete}
\begin{itemize}
\item {fónica:nê}
\end{itemize}
\begin{itemize}
\item {Grp. gram.:m.}
\end{itemize}
\begin{itemize}
\item {Utilização:Burl.}
\end{itemize}
\begin{itemize}
\item {Proveniência:(De \textunderscore tirano\textunderscore )}
\end{itemize}
Pessôa, que vexa ou oprime as pessôas que dela dependem.
\section{Tirania}
\begin{itemize}
\item {Grp. gram.:f.}
\end{itemize}
\begin{itemize}
\item {Proveniência:(Gr. \textunderscore turannia\textunderscore )}
\end{itemize}
Domínio ou poder de tirano.
Govêrno injusto e cruel.
Violência; opressão.
\section{Tiranicamente}
\begin{itemize}
\item {Grp. gram.:adv.}
\end{itemize}
De modo tirânico; com opressão; com crueldade.
\section{Tiranicida}
\begin{itemize}
\item {Grp. gram.:m.}
\end{itemize}
\begin{itemize}
\item {Proveniência:(Lat. \textunderscore tyrannicida\textunderscore )}
\end{itemize}
Assassino de um tirano.
\section{Tiranicídio}
\begin{itemize}
\item {Grp. gram.:m.}
\end{itemize}
\begin{itemize}
\item {Proveniência:(Lat. \textunderscore tyrannicidium\textunderscore )}
\end{itemize}
Assassínio de um tirano.
\section{Tirânico}
\begin{itemize}
\item {Grp. gram.:adj.}
\end{itemize}
\begin{itemize}
\item {Proveniência:(Lat. \textunderscore tyrannicus\textunderscore )}
\end{itemize}
Relativo a tirano.
Próprio de tirano.
Injusto.
Violento; que tiraniza.
\section{Tiranizador}
\begin{itemize}
\item {Grp. gram.:m.  e  adj.}
\end{itemize}
O que tiraniza.
\section{Tiranizar}
\begin{itemize}
\item {Grp. gram.:v. t.}
\end{itemize}
\begin{itemize}
\item {Utilização:Fig.}
\end{itemize}
\begin{itemize}
\item {Proveniência:(De \textunderscore tirano\textunderscore )}
\end{itemize}
Tratar com tirania.
Vèxar, oprimir.
Influir cruelmente em.
Sêr rigoroso para com.
Causar obstáculos a.
\section{Tirano}
\begin{itemize}
\item {Grp. gram.:m.}
\end{itemize}
\begin{itemize}
\item {Grp. gram.:Adj.}
\end{itemize}
\begin{itemize}
\item {Proveniência:(Lat. \textunderscore tyrannus\textunderscore )}
\end{itemize}
Antigamente, o que assumia autoridade soberana sôbre uma comunidade republicana.
Soberano cruel ou injusto, que governa, desprezando as leis.
Aquele que tiraniza.
Pessôa cruel.
Ave da Guiana inglesa.
O mesmo que \textunderscore tirânico\textunderscore .
\section{Tirina}
\begin{itemize}
\item {Grp. gram.:f.}
\end{itemize}
\begin{itemize}
\item {Utilização:Chím.}
\end{itemize}
\begin{itemize}
\item {Proveniência:(Do gr. \textunderscore turos\textunderscore , queijo)}
\end{itemize}
O mesmo que \textunderscore caseína\textunderscore .
\section{Tírio}
\begin{itemize}
\item {Grp. gram.:adj.}
\end{itemize}
\begin{itemize}
\item {Utilização:Poét.}
\end{itemize}
\begin{itemize}
\item {Grp. gram.:M.}
\end{itemize}
\begin{itemize}
\item {Proveniência:(Lat. \textunderscore tyrius\textunderscore )}
\end{itemize}
Purpúreo.
Relativo a Tyro.
Habitante de Tyro.
\section{Tiro}
\begin{itemize}
\item {Grp. gram.:m.}
\end{itemize}
\begin{itemize}
\item {Utilização:Poét.}
\end{itemize}
\begin{itemize}
\item {Proveniência:(Lat. \textunderscore tyrus\textunderscore )}
\end{itemize}
O mesmo que \textunderscore púrpura\textunderscore .
\section{Tiróglifos}
\begin{itemize}
\item {Grp. gram.:m. pl.}
\end{itemize}
\begin{itemize}
\item {Proveniência:(Do gr. \textunderscore turos\textunderscore  + \textunderscore gluphein\textunderscore )}
\end{itemize}
Gênero de aracnídeos ácaros.
\section{Tirolês}
\begin{itemize}
\item {Grp. gram.:adj.}
\end{itemize}
\begin{itemize}
\item {Grp. gram.:M.}
\end{itemize}
Relativo ao Tyrol.
Habitante do Tyrol.
\section{Tirolesa}
\begin{itemize}
\item {Grp. gram.:f.}
\end{itemize}
\begin{itemize}
\item {Proveniência:(De \textunderscore tirolês\textunderscore )}
\end{itemize}
Espécie de ária, semelhante ás canções populares do Tyrol.
Dança do Tyrol.
\section{Tirólito}
\begin{itemize}
\item {Grp. gram.:m.}
\end{itemize}
\begin{itemize}
\item {Proveniência:(Do gr. \textunderscore turos\textunderscore  + \textunderscore lithos\textunderscore )}
\end{itemize}
Arseniato de cobre natural.
\section{Tiromântico}
\begin{itemize}
\item {Grp. gram.:adj.}
\end{itemize}
Relativo á tiromancia.
\section{Tirosina}
\begin{itemize}
\item {Grp. gram.:f.}
\end{itemize}
\begin{itemize}
\item {Proveniência:(It. e cast. \textunderscore tirosina\textunderscore , do gr. \textunderscore turos\textunderscore , queijo)}
\end{itemize}
Substância cristalizável em agulhas brancas, e resultante da acção da potassa sôbre a caseína, a fibrina, etc.
\section{Tirrênio}
\begin{itemize}
\item {Grp. gram.:m.}
\end{itemize}
O mesmo que \textunderscore etrusco\textunderscore .
\section{Typochromia}
\begin{itemize}
\item {Grp. gram.:f.}
\end{itemize}
\begin{itemize}
\item {Proveniência:(Do gr. \textunderscore tupos\textunderscore  + \textunderscore khroma\textunderscore )}
\end{itemize}
Impressão typográphica, a côres.
\section{Typógrapha}
\begin{itemize}
\item {Grp. gram.:f.}
\end{itemize}
Máquina de composição typográphica, inventada em 1898, na América do Norte.
\section{Typographia}
\begin{itemize}
\item {Grp. gram.:f.}
\end{itemize}
Arte de imprimir.
Estabelecimento typographico.
(Cp. \textunderscore typógrapho\textunderscore )
\section{Typographar}
\begin{itemize}
\item {Grp. gram.:v. t.}
\end{itemize}
\begin{itemize}
\item {Proveniência:(De \textunderscore typógrapho\textunderscore )}
\end{itemize}
Reproduzir pela typographia; imprimir.
\section{Typographicamente}
\begin{itemize}
\item {Grp. gram.:adv.}
\end{itemize}
De modo typográphico; por meio da typographia.
\section{Typográphico}
\begin{itemize}
\item {Grp. gram.:adj.}
\end{itemize}
Relativo á typographia: \textunderscore bôa execução typográphica\textunderscore .
\section{Typógrapho}
\begin{itemize}
\item {Grp. gram.:m.}
\end{itemize}
\begin{itemize}
\item {Proveniência:(Do gr. \textunderscore tupos\textunderscore  + \textunderscore graphein\textunderscore )}
\end{itemize}
Aquelle que sabe ou exerce a arte typográphica.
\section{Typóia}
\begin{itemize}
\item {Grp. gram.:f.}
\end{itemize}
\begin{itemize}
\item {Utilização:Chul.}
\end{itemize}
\begin{itemize}
\item {Proveniência:(De \textunderscore typo\textunderscore )}
\end{itemize}
Mulhér ordinária, desprezível.
\section{Typólitho}
\begin{itemize}
\item {Grp. gram.:m.}
\end{itemize}
\begin{itemize}
\item {Utilização:Miner.}
\end{itemize}
\begin{itemize}
\item {Proveniência:(Do gr. \textunderscore tupos\textunderscore  + \textunderscore lithos\textunderscore )}
\end{itemize}
Pedra, que tem impressa a fórma de algumas plantas ou animaes.
\section{Typolithographia}
\begin{itemize}
\item {Grp. gram.:f.}
\end{itemize}
\begin{itemize}
\item {Proveniência:(De \textunderscore typographia\textunderscore  e \textunderscore lithographia\textunderscore )}
\end{itemize}
Arte de imprimir na mesma fôlha desenhos lithográphicos e caracteres typográphicos.
\section{Typometria}
\begin{itemize}
\item {Grp. gram.:f.}
\end{itemize}
\begin{itemize}
\item {Proveniência:(Do gr. \textunderscore tupos\textunderscore  + \textunderscore metron\textunderscore )}
\end{itemize}
Arte de compor, por meio de filetes recortados e caracteres móveis, certos desenhos que se imprimem typographicamente.
\section{Typómetro}
\begin{itemize}
\item {Grp. gram.:m.}
\end{itemize}
\begin{itemize}
\item {Proveniência:(Do gr. \textunderscore tupos\textunderscore  + \textunderscore metron\textunderscore )}
\end{itemize}
Instrumento de fundição typográphica, que serve para verificar se as letras estão na sua altura, e se têm o corpo desejado.
\section{Typophonia}
\begin{itemize}
\item {Grp. gram.:f.}
\end{itemize}
\begin{itemize}
\item {Utilização:Mús.}
\end{itemize}
Arte ou modo de marcar a voz ou o compasso, batendo.
(Cp. \textunderscore typophónio\textunderscore )
\section{Typophónio}
\begin{itemize}
\item {Grp. gram.:m.}
\end{itemize}
\begin{itemize}
\item {Proveniência:(Do gr. \textunderscore tupos\textunderscore  + \textunderscore phone\textunderscore )}
\end{itemize}
Instrumento músico que produz sons simples e invariáveis.
\section{Typtologia}
\begin{itemize}
\item {Grp. gram.:f.}
\end{itemize}
\begin{itemize}
\item {Utilização:Espir.}
\end{itemize}
\begin{itemize}
\item {Proveniência:(Do gr. \textunderscore tuptein\textunderscore  + \textunderscore logos\textunderscore )}
\end{itemize}
Experiência, a que procedem os espiritistas com mesas girantes, chapéus, peneiras, etc.
Communicação dos espíritos por meio de pancadas.
\section{Typtológico}
\begin{itemize}
\item {Grp. gram.:adj.}
\end{itemize}
Relativo á typtologia.
\section{Typtólogo}
\begin{itemize}
\item {Grp. gram.:m.}
\end{itemize}
\begin{itemize}
\item {Utilização:Espir.}
\end{itemize}
Médium, que está apto para praticar a typtologia.
(Cp. \textunderscore typtologia\textunderscore )
\section{Tyranna}
\begin{itemize}
\item {Grp. gram.:f.}
\end{itemize}
\begin{itemize}
\item {Utilização:Fam.}
\end{itemize}
\begin{itemize}
\item {Proveniência:(Lat. \textunderscore tyranna\textunderscore )}
\end{itemize}
Mulhér esquiva ou má.
Mulhér cruel.
Dança brasileira.
\section{Tyrannete}
\begin{itemize}
\item {fónica:nê}
\end{itemize}
\begin{itemize}
\item {Grp. gram.:m.}
\end{itemize}
\begin{itemize}
\item {Utilização:Burl.}
\end{itemize}
\begin{itemize}
\item {Proveniência:(De \textunderscore tyranno\textunderscore )}
\end{itemize}
Pessôa, que vexa ou opprime as pessôas que della dependem.
\section{Tyrannia}
\begin{itemize}
\item {Grp. gram.:f.}
\end{itemize}
\begin{itemize}
\item {Proveniência:(Gr. \textunderscore turannia\textunderscore )}
\end{itemize}
Domínio ou poder de tyranno.
Govêrno injusto e cruel.
Violência; oppressão.
\section{Tyrannicamente}
\begin{itemize}
\item {Grp. gram.:adv.}
\end{itemize}
De modo tyrânnico; com oppressão; com crueldade.
\section{Tyrannicida}
\begin{itemize}
\item {Grp. gram.:m.}
\end{itemize}
\begin{itemize}
\item {Proveniência:(Lat. \textunderscore tyrannicida\textunderscore )}
\end{itemize}
Assassino de um tyranno.
\section{Tyrannicídio}
\begin{itemize}
\item {Grp. gram.:m.}
\end{itemize}
\begin{itemize}
\item {Proveniência:(Lat. \textunderscore tyrannicidium\textunderscore )}
\end{itemize}
Assassínio de um tyranno.
\section{Tyrânnico}
\begin{itemize}
\item {Grp. gram.:adj.}
\end{itemize}
\begin{itemize}
\item {Proveniência:(Lat. \textunderscore tyrannicus\textunderscore )}
\end{itemize}
Relativo a tyranno.
Próprio de tyranno.
Injusto.
Violento; que tyranniza.
\section{Tyrannizador}
\begin{itemize}
\item {Grp. gram.:m.  e  adj.}
\end{itemize}
O que \textunderscore tyranniza\textunderscore .
\section{Tyrannizar}
\begin{itemize}
\item {Grp. gram.:v. t.}
\end{itemize}
\begin{itemize}
\item {Utilização:Fig.}
\end{itemize}
\begin{itemize}
\item {Proveniência:(De \textunderscore tyranno\textunderscore )}
\end{itemize}
Tratar com tyrannia.
Vèxar, opprimir.
Influir cruelmente em.
Sêr rigoroso para com.
Causar obstáculos a.
\section{Tyranno}
\begin{itemize}
\item {Grp. gram.:m.}
\end{itemize}
\begin{itemize}
\item {Grp. gram.:Adj.}
\end{itemize}
\begin{itemize}
\item {Proveniência:(Lat. \textunderscore tyrannus\textunderscore )}
\end{itemize}
Antigamente, o que assumia autoridade soberana sôbre uma communidade republicana.
Soberano cruel ou injusto, que governa, desprezando as leis.
Aquelle que tyranniza.
Pessôa cruel.
Ave da Guiana inglesa.
O mesmo que \textunderscore tyrânnico\textunderscore .
\section{Tyrina}
\begin{itemize}
\item {Grp. gram.:f.}
\end{itemize}
\begin{itemize}
\item {Utilização:Chím.}
\end{itemize}
\begin{itemize}
\item {Proveniência:(Do gr. \textunderscore turos\textunderscore , queijo)}
\end{itemize}
O mesmo que \textunderscore caseína\textunderscore .
\section{Týrio}
\begin{itemize}
\item {Grp. gram.:adj.}
\end{itemize}
\begin{itemize}
\item {Utilização:Poét.}
\end{itemize}
\begin{itemize}
\item {Grp. gram.:M.}
\end{itemize}
\begin{itemize}
\item {Proveniência:(Lat. \textunderscore tyrius\textunderscore )}
\end{itemize}
Purpúreo.
Relativo a Tyro.
Habitante de Tyro.
\section{Tyro}
\begin{itemize}
\item {Grp. gram.:m.}
\end{itemize}
\begin{itemize}
\item {Utilização:Poét.}
\end{itemize}
\begin{itemize}
\item {Proveniência:(Lat. \textunderscore tyrus\textunderscore )}
\end{itemize}
O mesmo que \textunderscore púrpura\textunderscore .
\section{Tyróglyphos}
\begin{itemize}
\item {Grp. gram.:m. pl.}
\end{itemize}
\begin{itemize}
\item {Proveniência:(Do gr. \textunderscore turos\textunderscore  + \textunderscore gluphein\textunderscore )}
\end{itemize}
Gênero de arachnídeos ácaros.
\section{Tyrolês}
\begin{itemize}
\item {Grp. gram.:adj.}
\end{itemize}
\begin{itemize}
\item {Grp. gram.:M.}
\end{itemize}
Relativo ao Tyrol.
Habitante do Tyrol.
\section{Tyrolesa}
\begin{itemize}
\item {Grp. gram.:f.}
\end{itemize}
\begin{itemize}
\item {Proveniência:(De \textunderscore tyrolês\textunderscore )}
\end{itemize}
Espécie de ária, semelhante ás canções populares do Tyrol.
Dança do Tyrol.
\section{Tyrólitho}
\begin{itemize}
\item {Grp. gram.:m.}
\end{itemize}
\begin{itemize}
\item {Proveniência:(Do gr. \textunderscore turos\textunderscore  + \textunderscore lithos\textunderscore )}
\end{itemize}
Arseniato de cobre natural.
\section{Tyromancia}
\begin{itemize}
\item {Grp. gram.:f.}
\end{itemize}
\begin{itemize}
\item {Proveniência:(Do gr. \textunderscore turos\textunderscore  + \textunderscore manteia\textunderscore )}
\end{itemize}
Adivinhação por meio do queijo.
\section{Tyromântico}
\begin{itemize}
\item {Grp. gram.:adj.}
\end{itemize}
Relativo á tyromancia.
\section{Tyrosina}
\begin{itemize}
\item {Grp. gram.:f.}
\end{itemize}
\begin{itemize}
\item {Proveniência:(It. e cast. \textunderscore tirosina\textunderscore , do gr. \textunderscore turos\textunderscore , queijo)}
\end{itemize}
Substância crystallizável em agulhas brancas, e resultante da acção da potassa sôbre a caseína, a fibrina, etc.
\section{Tyrrhênio}
\begin{itemize}
\item {Grp. gram.:m.}
\end{itemize}
\end{document}