
\begin{itemize}
\item {Proveniência: }
\end{itemize}\documentclass{article}
\usepackage[portuguese]{babel}
\title{G}
\begin{document}
Bico de fivela, para segurar a presilha.
\section{Gaimilo}
\begin{itemize}
\item {Grp. gram.:m.}
\end{itemize}
\begin{itemize}
\item {Utilização:Prov.}
\end{itemize}
\begin{itemize}
\item {Utilização:dur.}
\end{itemize}
Peixe miúdo, com que se iscam canas e cordas. (Colhido em Amarante)
\section{Gaja}
\begin{itemize}
\item {Grp. gram.:f.}
\end{itemize}
\begin{itemize}
\item {Utilização:Chul.}
\end{itemize}
Qualquer mulhér: \textunderscore não fales àquela gaja\textunderscore .
(Cp. \textunderscore gajo\textunderscore )
\section{Gajão}
\begin{itemize}
\item {Grp. gram.:m.}
\end{itemize}
\begin{itemize}
\item {Utilização:Chul.}
\end{itemize}
\begin{itemize}
\item {Proveniência:(De \textunderscore gajo\textunderscore )}
\end{itemize}
Sujeito finório, arteiro: \textunderscore aquilo é que é um gajão\textunderscore !
\section{Galvanocautério}
\begin{itemize}
\item {Grp. gram.:m.}
\end{itemize}
\begin{itemize}
\item {Utilização:Med.}
\end{itemize}
\begin{itemize}
\item {Proveniência:(De \textunderscore galvanismo\textunderscore  + \textunderscore cautério\textunderscore )}
\end{itemize}
Cautério, cuja encandescência é devida á passagem de uma corrente eléctrica.
\section{Gamacismo}
\begin{itemize}
\item {Grp. gram.:m.}
\end{itemize}
\begin{itemize}
\item {Utilização:Med.}
\end{itemize}
\begin{itemize}
\item {Proveniência:(De \textunderscore gamma\textunderscore , nome grego da letra \textunderscore g\textunderscore )}
\end{itemize}
Vício de pronunciação, causado pela dificuldade ou impossibilidade de pronunciar a letra \textunderscore g\textunderscore .
\section{Gammacismo}
\begin{itemize}
\item {Grp. gram.:m.}
\end{itemize}
\begin{itemize}
\item {Utilização:Med.}
\end{itemize}
\begin{itemize}
\item {Proveniência:(De \textunderscore gamma\textunderscore , nome grego da letra \textunderscore g\textunderscore )}
\end{itemize}
Vício de pronunciação, causado pela difficuldade ou impossibilidade de pronunciar a letra \textunderscore g\textunderscore .
\section{Garrôa}
\begin{itemize}
\item {Grp. gram.:f.}
\end{itemize}
\begin{itemize}
\item {Utilização:T. de marinheiros}
\end{itemize}
Vento forte do Noroéste, na costa occidental de Portugal.
(Relaciona-se com \textunderscore garrar\textunderscore ?)
\section{Gastroenterostomia}
\begin{itemize}
\item {Grp. gram.:f.}
\end{itemize}
\begin{itemize}
\item {Utilização:Med.}
\end{itemize}
\begin{itemize}
\item {Proveniência:(Do gr. \textunderscore gaster\textunderscore  + \textunderscore enteron\textunderscore  + \textunderscore stoma\textunderscore )}
\end{itemize}
Operação, que consiste em fazer communicar o estômago com uma asa intestinal.
\section{Gatunha}
\begin{itemize}
\item {Grp. gram.:f.}
\end{itemize}
\begin{itemize}
\item {Utilização:Bot.}
\end{itemize}
O mesmo que \textunderscore unhagata\textunderscore .
\section{Glicemia}
\begin{itemize}
\item {Grp. gram.:f.}
\end{itemize}
\begin{itemize}
\item {Utilização:Med.}
\end{itemize}
\begin{itemize}
\item {Proveniência:(Do gr. \textunderscore glukos\textunderscore  + \textunderscore haima\textunderscore )}
\end{itemize}
Existência normal da glicose no sangue.
\section{Glycemia}
\begin{itemize}
\item {Grp. gram.:f.}
\end{itemize}
\begin{itemize}
\item {Utilização:Med.}
\end{itemize}
\begin{itemize}
\item {Proveniência:(Do gr. \textunderscore glukos\textunderscore  + \textunderscore haima\textunderscore )}
\end{itemize}
Existência normal da glycose no sangue.
\section{Gonalgia}
\begin{itemize}
\item {Grp. gram.:f.}
\end{itemize}
\begin{itemize}
\item {Utilização:Med.}
\end{itemize}
\begin{itemize}
\item {Proveniência:(Do gr. \textunderscore gonu\textunderscore  + \textunderscore algos\textunderscore )}
\end{itemize}
Dôr no joêlho.
\section{Gonococcia}
\begin{itemize}
\item {Grp. gram.:f.}
\end{itemize}
\begin{itemize}
\item {Utilização:Med.}
\end{itemize}
Infecção do organismo pelo gonococco.
\section{Gonococia}
\begin{itemize}
\item {Grp. gram.:f.}
\end{itemize}
\begin{itemize}
\item {Utilização:Med.}
\end{itemize}
Infecção do organismo pelo gonococo.
\section{Gorilhóide}
\begin{itemize}
\item {Grp. gram.:adj.}
\end{itemize}
Semelhante ao gorilha ou privativo do gorilha:«\textunderscore uma cara gorilhóide\textunderscore ». R. Jorge, \textunderscore El Greco\textunderscore , 43.
\section{Graforreia}
\begin{itemize}
\item {Grp. gram.:f.}
\end{itemize}
\begin{itemize}
\item {Utilização:Med.}
\end{itemize}
\begin{itemize}
\item {Proveniência:(Do gr. \textunderscore graphein\textunderscore  + \textunderscore rhein\textunderscore )}
\end{itemize}
Necessidade irresistível, que alguns maníacos têm, de escrever.
\section{Graphorrhéa}
\begin{itemize}
\item {Grp. gram.:f.}
\end{itemize}
\begin{itemize}
\item {Utilização:Med.}
\end{itemize}
\begin{itemize}
\item {Proveniência:(Do gr. \textunderscore graphein\textunderscore  + \textunderscore rhein\textunderscore )}
\end{itemize}
Necessidade irresistível, que alguns maníacos têm, de escrever.
\section{Gravídico}
\begin{itemize}
\item {Grp. gram.:adj.}
\end{itemize}
\begin{itemize}
\item {Utilização:Med.}
\end{itemize}
\begin{itemize}
\item {Proveniência:(De \textunderscore grávido\textunderscore )}
\end{itemize}
Que depende da gravidez; relativo á gravidez: \textunderscore accidentes gravídicos\textunderscore .
\section{Grequismo}
\begin{itemize}
\item {Grp. gram.:m.}
\end{itemize}
Processo de pintura, que se observa nas obras de Greco.
Admiração, produzida pelos quadros de Greco. Cf. R. Jorge, \textunderscore El Greco\textunderscore , 35.
\section{Grequista}
\begin{itemize}
\item {Grp. gram.:m.}
\end{itemize}
Aquelle que admira o gôsto artístico de Greco. Cf. R. Jorge, \textunderscore El Greco\textunderscore , 44.
\section{Guelricho}
\begin{itemize}
\item {Grp. gram.:m.}
\end{itemize}
\begin{itemize}
\item {Utilização:Prov.}
\end{itemize}
O mesmo que \textunderscore galracho\textunderscore . (Colhido em Arganil)
\section{Gúttulo}
\begin{itemize}
\item {Grp. gram.:m.}
\end{itemize}
Pequeno vaso, o mesmo que \textunderscore apisteiro\textunderscore . Cf. J. R. Mazarém, \textunderscore Compilação de Doutrinas Obstetrícias\textunderscore , 390, (2.^a ed.)
(Cp. lat. \textunderscore guttula\textunderscore )
\section{Gútulo}
\begin{itemize}
\item {Grp. gram.:m.}
\end{itemize}
Pequeno vaso, o mesmo que \textunderscore apisteiro\textunderscore . Cf. J. R. Mazarém, \textunderscore Compilação de Doutrinas Obstetrícias\textunderscore , 390, (2.^a ed.)
(Cp. lat. \textunderscore guttula\textunderscore )
\section{G}
\begin{itemize}
\item {fónica:gê}
\end{itemize}
\begin{itemize}
\item {Grp. gram.:m.}
\end{itemize}
\begin{itemize}
\item {Grp. gram.:Adj.}
\end{itemize}
\begin{itemize}
\item {Utilização:Mús.}
\end{itemize}
Sétima letra do alphabeto português.
Que numa série occupa o sétimo lugar: \textunderscore livro G\textunderscore , \textunderscore fôlha G\textunderscore , etc.
A sétima nota, em a notação alphabética.
\section{Gaaira}
\begin{itemize}
\item {Grp. gram.:f.}
\end{itemize}
Insecto indiano, (\textunderscore mantis gongylodes\textunderscore ).
\section{Gaança}
\begin{itemize}
\item {Grp. gram.:f.}
\end{itemize}
\begin{itemize}
\item {Utilização:Ant.}
\end{itemize}
O mesmo que \textunderscore ganância\textunderscore .
\textunderscore Filho de gaança\textunderscore , filho bastardo, filho adulterino. Cf. \textunderscore Port. Mon. Hist.\textunderscore , \textunderscore Script.\textunderscore , 260.
\section{Gabação}
\begin{itemize}
\item {Grp. gram.:f.}
\end{itemize}
Acto ou effeito de gabar.
\section{Gabachista}
\begin{itemize}
\item {Grp. gram.:m.  e  f.}
\end{itemize}
\begin{itemize}
\item {Utilização:Prov.}
\end{itemize}
O mesmo que \textunderscore gabarola\textunderscore .
\section{Gabaço}
\begin{itemize}
\item {Grp. gram.:m.}
\end{itemize}
\begin{itemize}
\item {Utilização:Prov.}
\end{itemize}
\begin{itemize}
\item {Utilização:trasm.}
\end{itemize}
\begin{itemize}
\item {Proveniência:(De \textunderscore gabar\textunderscore )}
\end{itemize}
Grande elogio.
\section{Gabadela}
\begin{itemize}
\item {Grp. gram.:f.}
\end{itemize}
\begin{itemize}
\item {Utilização:Pop.}
\end{itemize}
O mesmo que \textunderscore gabação\textunderscore .
\section{Gabadinha}
\begin{itemize}
\item {Grp. gram.:f.}
\end{itemize}
\begin{itemize}
\item {Utilização:Prov.}
\end{itemize}
\begin{itemize}
\item {Utilização:dur.}
\end{itemize}
\begin{itemize}
\item {Proveniência:(De \textunderscore gabadinho\textunderscore )}
\end{itemize}
Predilecção.
Mania.
\section{Gabadinho}
\begin{itemize}
\item {Grp. gram.:adj.}
\end{itemize}
\begin{itemize}
\item {Utilização:Fam.}
\end{itemize}
\begin{itemize}
\item {Proveniência:(De \textunderscore gabar\textunderscore )}
\end{itemize}
Que anda na berra.
Que é muito falado.
\section{Gabador}
\begin{itemize}
\item {Grp. gram.:m.  e  adj.}
\end{itemize}
\begin{itemize}
\item {Proveniência:(Lat. \textunderscore gabator\textunderscore )}
\end{itemize}
O que gaba.
\section{Gabamento}
\begin{itemize}
\item {Grp. gram.:m.}
\end{itemize}
O mesmo que \textunderscore gabação\textunderscore .
\section{Gabança}
\begin{itemize}
\item {Grp. gram.:f.}
\end{itemize}
\begin{itemize}
\item {Utilização:Ant.}
\end{itemize}
O mesmo que \textunderscore gabação\textunderscore .
\section{Gabanista}
\begin{itemize}
\item {Grp. gram.:m.}
\end{itemize}
\begin{itemize}
\item {Utilização:Prov.}
\end{itemize}
O mesmo que \textunderscore gabarola\textunderscore . (Colhido em San-Pedro do Sul)
\section{Gabão}
\begin{itemize}
\item {Grp. gram.:m.}
\end{itemize}
Espécie de capote, com capuz e mangas.
(Talvez do it. \textunderscore gabbano\textunderscore , do lat. \textunderscore cappa\textunderscore )
\section{Gabão}
\begin{itemize}
\item {Grp. gram.:m.}
\end{itemize}
Aquelle que gaba muito. Cf. Arráez, \textunderscore Dial.\textunderscore , II.
Grande louvor ou elogio. Cf. \textunderscore Eufrosina\textunderscore , I, 3.
\section{Gabaonita}
\begin{itemize}
\item {Grp. gram.:m.  e  adj.}
\end{itemize}
O que é de Gabaon, na Palestina.
\section{Gabar}
\begin{itemize}
\item {Grp. gram.:v. t.}
\end{itemize}
\begin{itemize}
\item {Proveniência:(Do norr. ant. \textunderscore gabb\textunderscore , zombaria)}
\end{itemize}
Fazer o elogio de.
Louvar; lisonjear.
\section{Gabardina}
\begin{itemize}
\item {Grp. gram.:f.}
\end{itemize}
Espécie de gabão; gabinardo. Cf. Silveira da Mota, \textunderscore Viagens\textunderscore , 13.
\section{Gabardo}
\begin{itemize}
\item {Grp. gram.:m.}
\end{itemize}
Capote de cabeção e mangas. Cf. Camillo, \textunderscore Filha do Reg.\textunderscore , 121.
\section{Gabari}
\begin{itemize}
\item {Grp. gram.:m.}
\end{itemize}
\begin{itemize}
\item {Proveniência:(Fr. \textunderscore gabarit\textunderscore )}
\end{itemize}
Modelo de um navio, em tamanho natural.
Medida de ferro para verificar as dimensões exteriores de uma bôca de fogo.
Cércea de carga, nos caminhos de ferro.
\section{Gabarola}
\begin{itemize}
\item {Grp. gram.:m.  e  f.}
\end{itemize}
\begin{itemize}
\item {Utilização:Pop.}
\end{itemize}
\begin{itemize}
\item {Proveniência:(De \textunderscore gabar\textunderscore )}
\end{itemize}
Pessôa, que tem bazófia ou que faz ostentação e elogio dos próprios actos.
Indivíduo jactancioso.
\section{Gabarolas}
\begin{itemize}
\item {Grp. gram.:m. pl.}
\end{itemize}
O mesmo que \textunderscore gabarola\textunderscore .
\section{Gabarolice}
\begin{itemize}
\item {Grp. gram.:f.}
\end{itemize}
Acto ou dito de gabarola.
\section{Gabarote}
\begin{itemize}
\item {Grp. gram.:m.}
\end{itemize}
Gabarra pequena, sem coberta.
(Por \textunderscore gabarrote\textunderscore , de \textunderscore gabarra\textunderscore )
\section{Gabarra}
\begin{itemize}
\item {Grp. gram.:f.}
\end{itemize}
Embarcação de vela e remos e de fundo chato.
Rêde de arrastar.
(Cast. \textunderscore gabarra\textunderscore )
\section{Gabarreiro}
\begin{itemize}
\item {Grp. gram.:m.}
\end{itemize}
Arraes de gabarra.
\section{Gabarrice}
\begin{itemize}
\item {Grp. gram.:f.}
\end{itemize}
\begin{itemize}
\item {Utilização:Prov.}
\end{itemize}
\begin{itemize}
\item {Utilização:minh.}
\end{itemize}
Acto ou dito de gabarola.
Bazófia.
(Por \textunderscore gabarice\textunderscore , de \textunderscore gabar\textunderscore )
\section{Gabarro}
\begin{itemize}
\item {Grp. gram.:m.}
\end{itemize}
Apostema, que ataca os pés dos cavallos e dos bois.
(Cast. \textunderscore gabarro\textunderscore )
\section{Gabatório}
\begin{itemize}
\item {Grp. gram.:m.}
\end{itemize}
Grande gabação pública ou feita pelo público. Cf. Camillo, \textunderscore Cancion. Al.\textunderscore , 188.
\section{Gabazola}
\begin{itemize}
\item {Grp. gram.:m.}
\end{itemize}
O mesmo que \textunderscore gabarola\textunderscore .
\section{Gabeia}
\begin{itemize}
\item {Grp. gram.:f.}
\end{itemize}
\begin{itemize}
\item {Utilização:Ant.}
\end{itemize}
Mercado, feira.
\section{Gabela}
\begin{itemize}
\item {Grp. gram.:f.}
\end{itemize}
O mesmo que \textunderscore gavela\textunderscore .
\section{Gabela}
\begin{itemize}
\item {Grp. gram.:f.}
\end{itemize}
\begin{itemize}
\item {Utilização:Ant.}
\end{itemize}
\begin{itemize}
\item {Proveniência:(Do lat. \textunderscore gabella\textunderscore )}
\end{itemize}
Imposto sôbre o sal.
Imposto.
\section{Gabelo}
\begin{itemize}
\item {Grp. gram.:m.}
\end{itemize}
\begin{itemize}
\item {Utilização:Ant.}
\end{itemize}
O mesmo que \textunderscore gabela\textunderscore ^2. Cf. Arráez, \textunderscore Diál.\textunderscore , X.
\section{Gabéu}
\begin{itemize}
\item {Grp. gram.:m.}
\end{itemize}
\begin{itemize}
\item {Utilização:ant.}
\end{itemize}
\begin{itemize}
\item {Utilização:Gír.}
\end{itemize}
Chapéu.
\section{Gábia}
\begin{itemize}
\item {Grp. gram.:f.}
\end{itemize}
\begin{itemize}
\item {Utilização:Prov.}
\end{itemize}
\begin{itemize}
\item {Utilização:trasm.}
\end{itemize}
Escavação em volta da videira, para a estrumar ou para fazer mergulhia. (Colhido no Mogadoiro)
(Cast. \textunderscore gavia\textunderscore )
\section{Gabiagem}
\begin{itemize}
\item {Grp. gram.:f.}
\end{itemize}
\begin{itemize}
\item {Utilização:Náut.}
\end{itemize}
Serviço, relativo aos cestos de gávea.
(Por \textunderscore gaveagem\textunderscore , de \textunderscore gávea\textunderscore . Cp. \textunderscore gabião\textunderscore )
\section{Gabião}
\begin{itemize}
\item {Grp. gram.:m.}
\end{itemize}
\begin{itemize}
\item {Proveniência:(It. \textunderscore gabbione\textunderscore )}
\end{itemize}
Cesto grande, para transporte de terra, adubos, etc.
Cestão.
\section{Gabiar}
\begin{itemize}
\item {Grp. gram.:v. i.}
\end{itemize}
\begin{itemize}
\item {Utilização:Prov.}
\end{itemize}
\begin{itemize}
\item {Utilização:trasm.}
\end{itemize}
Abrir gábias.
\section{Gabilami}
\begin{itemize}
\item {Grp. gram.:m.}
\end{itemize}
Espécie de bacalhau islandês.
\section{Gabinarda}
\begin{itemize}
\item {Grp. gram.:f.}
\end{itemize}
\begin{itemize}
\item {Proveniência:(Do rad. de \textunderscore gabão\textunderscore )}
\end{itemize}
Espécie de gabão; varino.
\section{Gabinardo}
\begin{itemize}
\item {Grp. gram.:m.}
\end{itemize}
\begin{itemize}
\item {Proveniência:(Do rad. de \textunderscore gabão\textunderscore )}
\end{itemize}
Espécie de gabão; varino.
\section{Gabinardo}
\begin{itemize}
\item {Grp. gram.:m.}
\end{itemize}
\begin{itemize}
\item {Utilização:Prov.}
\end{itemize}
\begin{itemize}
\item {Utilização:trasm.}
\end{itemize}
O mesmo que \textunderscore gabiru\textunderscore .
\section{Gabinete}
\begin{itemize}
\item {fónica:nê}
\end{itemize}
\begin{itemize}
\item {Grp. gram.:m.}
\end{itemize}
\begin{itemize}
\item {Proveniência:(It. \textunderscore gabinetto\textunderscore , fr. \textunderscore cabinet\textunderscore , de \textunderscore cabine\textunderscore , fórma ant. de \textunderscore cabane\textunderscore )}
\end{itemize}
Aposento ou compartimento, um pouco insulado do serviço geral de outros compartimentos do mesmo edifício, destinado geralmente a trabalhos particulares.
Escritório.
Camarim.
\section{Gábio}
\begin{itemize}
\item {Grp. gram.:m.}
\end{itemize}
\begin{itemize}
\item {Utilização:ant.}
\end{itemize}
\begin{itemize}
\item {Utilização:Gír.}
\end{itemize}
Chapéu.
(Cp. \textunderscore gabéu\textunderscore )
\section{Gabionada}
\begin{itemize}
\item {Grp. gram.:f.}
\end{itemize}
Trabalho com os gabiões.
Fileira de gabiões.
\section{Gabionador}
\begin{itemize}
\item {Grp. gram.:m.  e  adj.}
\end{itemize}
O que gabiona.
\section{Gabionar}
\begin{itemize}
\item {Grp. gram.:v. t.}
\end{itemize}
Trabalhar com gabiões.
\section{Gabiroba}
\begin{itemize}
\item {Grp. gram.:f.}
\end{itemize}
\begin{itemize}
\item {Utilização:Bras}
\end{itemize}
O mesmo que \textunderscore guabiroba\textunderscore .
\section{Gabiru}
\begin{itemize}
\item {Grp. gram.:m.  e  adj.}
\end{itemize}
\begin{itemize}
\item {Utilização:Prov.}
\end{itemize}
Velhaco, patife.
Garoto.
\section{Gabo}
\begin{itemize}
\item {Grp. gram.:m.}
\end{itemize}
\begin{itemize}
\item {Proveniência:(De \textunderscore gabar\textunderscore )}
\end{itemize}
Acto ou effeito de gabar.
Vaidade; jactância.
\section{Gabolas}
\begin{itemize}
\item {Grp. gram.:m.}
\end{itemize}
\begin{itemize}
\item {Utilização:Chul.}
\end{itemize}
O mesmo que \textunderscore gabarola\textunderscore .
\section{Gaboleia}
\begin{itemize}
\item {Grp. gram.:f.}
\end{itemize}
\begin{itemize}
\item {Utilização:T. de Turquel}
\end{itemize}
\begin{itemize}
\item {Proveniência:(De \textunderscore gabolas\textunderscore )}
\end{itemize}
Louvor próprio, exaggerado.
\section{Gabolice}
\begin{itemize}
\item {Grp. gram.:f.}
\end{itemize}
O mesmo que \textunderscore gaboleia\textunderscore .
\section{Gabonense}
\begin{itemize}
\item {Grp. gram.:adj.}
\end{itemize}
O mesmo que \textunderscore gabonês\textunderscore .
\section{Gabonês}
\begin{itemize}
\item {Grp. gram.:adj.}
\end{itemize}
Relativo ao rio Gabão, em África.
\section{Gabordo}
\begin{itemize}
\item {fónica:bôr}
\end{itemize}
\begin{itemize}
\item {Grp. gram.:m.}
\end{itemize}
\begin{itemize}
\item {Proveniência:(Fr. \textunderscore gabord\textunderscore )}
\end{itemize}
Prancha inferior no bordo exterior da embarcação.
\section{Gabrinaldo}
\begin{itemize}
\item {Grp. gram.:m.}
\end{itemize}
\begin{itemize}
\item {Utilização:Ant.}
\end{itemize}
\begin{itemize}
\item {Utilização:Gír.}
\end{itemize}
O mesmo que \textunderscore gabinardo\textunderscore ^1.
\section{Gabrito}
\begin{itemize}
\item {Grp. gram.:m.}
\end{itemize}
\begin{itemize}
\item {Utilização:Des.}
\end{itemize}
Espécie de rêde para pesca.
(Provavelmente, do fr. \textunderscore gabarit\textunderscore . Cp. \textunderscore gabari\textunderscore )
\section{Gaçapo}
\begin{itemize}
\item {Grp. gram.:m.}
\end{itemize}
\begin{itemize}
\item {Utilização:Ant.}
\end{itemize}
O mesmo que \textunderscore caçapo\textunderscore ^1, coêlho.
\section{Gacha}
\begin{itemize}
\item {Grp. gram.:f.}
\end{itemize}
Rêde, que forra lateralmante o copo das armações de pesca.
(Relaciona-se com \textunderscore cacha\textunderscore ?)
\section{Gacha}
\begin{itemize}
\item {Grp. gram.:f.}
\end{itemize}
\begin{itemize}
\item {Utilização:Prov.}
\end{itemize}
\begin{itemize}
\item {Utilização:trasm.}
\end{itemize}
\begin{itemize}
\item {Utilização:fam.}
\end{itemize}
O mesmo que \textunderscore mão\textunderscore .
\section{Gacheta}
\begin{itemize}
\item {fónica:chê}
\end{itemize}
\begin{itemize}
\item {Grp. gram.:f.}
\end{itemize}
(\textunderscore gaxeta\textunderscore )
\section{Gacho}
\begin{itemize}
\item {Grp. gram.:m.}
\end{itemize}
Parte posterior do pescoço do boi, sôbre que assenta a canga.
(Cast. \textunderscore gacho\textunderscore )
\section{Gacho}
\begin{itemize}
\item {Grp. gram.:m.}
\end{itemize}
\begin{itemize}
\item {Utilização:Prov.}
\end{itemize}
O mesmo que \textunderscore cacho\textunderscore ^1.
\section{Gachumbo}
\begin{itemize}
\item {Grp. gram.:m.}
\end{itemize}
Casca lenhosa e dura de alguns frutos americanos, da qual se fazem vasilhas.
\section{Gaci}
\begin{itemize}
\item {Grp. gram.:m.}
\end{itemize}
\begin{itemize}
\item {Utilização:Ant.}
\end{itemize}
Moiro convertido.
\section{Gadachim}
\begin{itemize}
\item {Grp. gram.:m.}
\end{itemize}
\begin{itemize}
\item {Utilização:Gír.}
\end{itemize}
Unha.
(Cp. \textunderscore gadanho\textunderscore )
\section{Gadachos}
\begin{itemize}
\item {Grp. gram.:m. pl.}
\end{itemize}
\begin{itemize}
\item {Utilização:Fam.}
\end{itemize}
\begin{itemize}
\item {Utilização:Ant.}
\end{itemize}
Os dedos.
\section{Gadamexil}
\begin{itemize}
\item {Grp. gram.:m.}
\end{itemize}
\begin{itemize}
\item {Utilização:Ant.}
\end{itemize}
O mesmo que \textunderscore guadamecim\textunderscore .
\section{Gadamo}
\begin{itemize}
\item {Grp. gram.:m.}
\end{itemize}
Escora, esteio.
\section{Gadanha}
\begin{itemize}
\item {Grp. gram.:f.}
\end{itemize}
\begin{itemize}
\item {Utilização:Pop.}
\end{itemize}
Caço.
Grande colhér para tirar sopa.
Espécie de foice.
Gadanho.
Acto de gadanhar.
O mesmo que \textunderscore mão\textunderscore .
(Cast. \textunderscore guadaña\textunderscore )
\section{Gadanhada}
\begin{itemize}
\item {Grp. gram.:f.}
\end{itemize}
Golpe de gadanho, (ancinho).
\section{Gadanhar}
\begin{itemize}
\item {Grp. gram.:v. t.}
\end{itemize}
Cortar (feno, etc.) com a foice chamada \textunderscore gadanha\textunderscore . Cf. Filinto, XIII, 26.
\section{Gadanheira}
\begin{itemize}
\item {Grp. gram.:f.}
\end{itemize}
\begin{itemize}
\item {Proveniência:(De \textunderscore gadanhar\textunderscore )}
\end{itemize}
Segadeira mechânica, máquina para cortar erva.
\section{Gadanheiro}
\begin{itemize}
\item {Grp. gram.:m.}
\end{itemize}
Aquelle que se emprega em gadanhar.
\section{Gadanho}
\begin{itemize}
\item {Grp. gram.:m.}
\end{itemize}
\begin{itemize}
\item {Utilização:Pop.}
\end{itemize}
Garra.
Unha.
Espécie de ancinho, com grandes dentes de ferro, para arrastar estrume, e para outros serviços agrícolas.
(Cp. \textunderscore gadanha\textunderscore )
\section{Gadaria}
\begin{itemize}
\item {Grp. gram.:f.}
\end{itemize}
\begin{itemize}
\item {Utilização:Prov.}
\end{itemize}
\begin{itemize}
\item {Utilização:trasm.}
\end{itemize}
\begin{itemize}
\item {Proveniência:(De \textunderscore gado\textunderscore . Cf. cast. \textunderscore ganaderia\textunderscore , de \textunderscore ganado\textunderscore , gado)}
\end{itemize}
Porção de gados. Cf. Visconde de Rio Sêco, \textunderscore Exposição Analýtica\textunderscore .
\section{Gadavanho}
\begin{itemize}
\item {Grp. gram.:m.}
\end{itemize}
\begin{itemize}
\item {Utilização:T. do Fundão}
\end{itemize}
Unha.
Gadanho.
Mão.
\section{Gadé}
\begin{itemize}
\item {Grp. gram.:m.}
\end{itemize}
\begin{itemize}
\item {Utilização:Gír.}
\end{itemize}
Dinheiro.
\section{Gadeiro}
\begin{itemize}
\item {Grp. gram.:m.}
\end{itemize}
\begin{itemize}
\item {Utilização:T. de Miranda}
\end{itemize}
Guardador de gado.
\section{Gadelha}
\begin{itemize}
\item {fónica:dê}
\end{itemize}
\begin{itemize}
\item {Grp. gram.:f.}
\end{itemize}
\begin{itemize}
\item {Grp. gram.:M.}
\end{itemize}
\begin{itemize}
\item {Utilização:Prov.}
\end{itemize}
\begin{itemize}
\item {Proveniência:(Do ant. fr. \textunderscore gade\textunderscore  + \textunderscore lain\textunderscore ?)}
\end{itemize}
Cabello desgrenhado e comprido.
Grenha; melena.
Madeixa de quaesquer fios.
Diabo.
\section{Gadelhado}
\begin{itemize}
\item {Grp. gram.:adj.}
\end{itemize}
O mesmo que \textunderscore gadelhudo\textunderscore .
\section{Gadelheira}
\begin{itemize}
\item {Grp. gram.:f.}
\end{itemize}
Grande gadelha.
\section{Gadelhudo}
\begin{itemize}
\item {Grp. gram.:adj.}
\end{itemize}
Que tem gadelhas.
Cabelludo.
\section{Gadidas}
\begin{itemize}
\item {Grp. gram.:m. pl.}
\end{itemize}
\begin{itemize}
\item {Proveniência:(Do gr. \textunderscore gados\textunderscore  + \textunderscore eidos\textunderscore )}
\end{itemize}
Família de peixes, a que pertence a pescada, o bacalhau, etc.
\section{Gadídeos}
\begin{itemize}
\item {Grp. gram.:m. pl.}
\end{itemize}
\begin{itemize}
\item {Proveniência:(Do gr. \textunderscore gados\textunderscore  + \textunderscore eidos\textunderscore )}
\end{itemize}
Família de peixes, a que pertence a pescada, o bacalhau, etc.
\section{Gadidos}
\begin{itemize}
\item {Grp. gram.:m. pl.}
\end{itemize}
\begin{itemize}
\item {Proveniência:(Do gr. \textunderscore gados\textunderscore  + \textunderscore eidos\textunderscore )}
\end{itemize}
Família de peixes, a que pertence a pescada, o bacalhau, etc.
\section{Gadínico}
\begin{itemize}
\item {Grp. gram.:adj.}
\end{itemize}
Relativo a óleo de figado de bacalhau.
(Cp. \textunderscore gadidos\textunderscore )
\section{Gadinina}
\begin{itemize}
\item {Grp. gram.:f.}
\end{itemize}
Substância, que há no óleo de fígado de bacalhau.
(Cp. \textunderscore gadidos\textunderscore )
\section{Gaditano}
\begin{itemize}
\item {Grp. gram.:adj.}
\end{itemize}
\begin{itemize}
\item {Grp. gram.:M.}
\end{itemize}
\begin{itemize}
\item {Proveniência:(Lat. \textunderscore gaditanus\textunderscore , de \textunderscore Gades\textunderscore , n. p.)}
\end{itemize}
Relativo a Cádiz.
Habitante de Cádiz.
\section{Gado}
\begin{itemize}
\item {Grp. gram.:m.}
\end{itemize}
\begin{itemize}
\item {Utilização:Fam.}
\end{itemize}
\begin{itemize}
\item {Proveniência:(Do cast. \textunderscore ganado\textunderscore )}
\end{itemize}
Animaes, geralmente criados no campo, para serviços de lavoira, para consumo doméstico ou para fins industriaes e commerciaes.
Rebanho, armento, vara, fato.
Classe ou conjunto de pessôas descommedidas ou indisciplinadas.
\section{Gado}
\begin{itemize}
\item {Grp. gram.:m.}
\end{itemize}
\begin{itemize}
\item {Proveniência:(Gr. \textunderscore gados\textunderscore )}
\end{itemize}
Peixe, que dá o seu nome aos gadidos.
\section{Gadoides}
\begin{itemize}
\item {Grp. gram.:m. pl.}
\end{itemize}
O mesmo que \textunderscore gadidos\textunderscore .
\section{Gadolinite}
\begin{itemize}
\item {Grp. gram.:f.}
\end{itemize}
\begin{itemize}
\item {Proveniência:(De \textunderscore Gadolin\textunderscore , n. p.)}
\end{itemize}
Silicato de cério.
\section{Gaduína}
\begin{itemize}
\item {Grp. gram.:f.}
\end{itemize}
\begin{itemize}
\item {Proveniência:(De \textunderscore gado\textunderscore ^2)}
\end{itemize}
Substância escura, inodora e insípida, que se extrai do óleo de fígado de bacalhau.
\section{Gadunha}
\begin{itemize}
\item {Grp. gram.:f.}
\end{itemize}
O mesmo que \textunderscore gadunho\textunderscore .
\section{Gadunho}
\begin{itemize}
\item {Grp. gram.:m.}
\end{itemize}
\begin{itemize}
\item {Utilização:Prov.}
\end{itemize}
\begin{itemize}
\item {Utilização:minh.}
\end{itemize}
\begin{itemize}
\item {Proveniência:(De \textunderscore grande\textunderscore  + \textunderscore unha\textunderscore ?)}
\end{itemize}
Unha crescida.
\section{Gaél}
\begin{itemize}
\item {Grp. gram.:m.}
\end{itemize}
Idioma da alta Escócia, o mesmo que \textunderscore gaélico\textunderscore .
\section{Gaélico}
\begin{itemize}
\item {Grp. gram.:adj.}
\end{itemize}
\begin{itemize}
\item {Grp. gram.:M.}
\end{itemize}
\begin{itemize}
\item {Proveniência:(De \textunderscore Gaël\textunderscore , n. p.)}
\end{itemize}
Relativo aos primitivos habitantes da Gállia e da Britânnia.
Língua, falada ao norte da Escócia e procedente do celta.
\section{Gafa}
\begin{itemize}
\item {Grp. gram.:f.}
\end{itemize}
\begin{itemize}
\item {Utilização:Ant.}
\end{itemize}
\begin{itemize}
\item {Utilização:Prov.}
\end{itemize}
\begin{itemize}
\item {Utilização:trasm.}
\end{itemize}
Gancho, com que se puxava a corda da bésta para a armar.
Fungo parasito.
Moléstia das azeitonas, que as engelha e faz caír.
Lepra.
Sarna leprosa de certos animaes.
Fome.
Pequeno caranguejo escuro.
(Do baixo al. \textunderscore gaffel\textunderscore , garfo)
\section{Gafa}
\begin{itemize}
\item {Grp. gram.:f.}
\end{itemize}
\begin{itemize}
\item {Utilização:Marn.}
\end{itemize}
Vaso, com que se transporta o sal nas marinhas.
\section{Gafador}
\begin{itemize}
\item {Grp. gram.:m.}
\end{itemize}
\begin{itemize}
\item {Utilização:Des.}
\end{itemize}
\begin{itemize}
\item {Proveniência:(De \textunderscore gafa\textunderscore ^1?)}
\end{itemize}
(?):«\textunderscore ...fazendo mão de gafador de pela\textunderscore ». \textunderscore Anat. Joc.\textunderscore , I, 27.
\section{Gafanhão}
\begin{itemize}
\item {Grp. gram.:m.}
\end{itemize}
\begin{itemize}
\item {Proveniência:(Do rad. de \textunderscore gafanhoto\textunderscore )}
\end{itemize}
Espécie de gafanhoto grande.
\section{Gafanhão}
\begin{itemize}
\item {Grp. gram.:m.}
\end{itemize}
Habitante ou pescador da Gafanha, no concelho de Ílhavo.
\section{Gafanhota}
\begin{itemize}
\item {Grp. gram.:f.}
\end{itemize}
\begin{itemize}
\item {Utilização:Ant.}
\end{itemize}
O mesmo que \textunderscore gafanhotada\textunderscore .
\section{Gafanhotada}
\begin{itemize}
\item {Grp. gram.:f.}
\end{itemize}
Porção de gafanhotos.
\section{Gafanhoto}
\begin{itemize}
\item {fónica:nhô}
\end{itemize}
\begin{itemize}
\item {Grp. gram.:m.}
\end{itemize}
\begin{itemize}
\item {Utilização:Prov.}
\end{itemize}
\begin{itemize}
\item {Proveniência:(Do rad. de \textunderscore gafa\textunderscore ^1, por allusão ao feitio do gancho da bésta)}
\end{itemize}
Insecto verde-amarelado, da ordem dos orthópteros saltadores.
Planta, conhecida também pelo nome de \textunderscore raiz de cobra\textunderscore .
O mesmo que \textunderscore gavião\textunderscore .
\section{Gafar}
\begin{itemize}
\item {Grp. gram.:v. t.}
\end{itemize}
\begin{itemize}
\item {Utilização:Fig.}
\end{itemize}
\begin{itemize}
\item {Grp. gram.:V. i.  e  p.}
\end{itemize}
\begin{itemize}
\item {Utilização:Fig.}
\end{itemize}
\begin{itemize}
\item {Proveniência:(De \textunderscore gafa\textunderscore ^1)}
\end{itemize}
Contagiar com gafa.
Contaminar.
Encher-se de gafa.
Contaminar-se.
\section{Gafar}
\begin{itemize}
\item {Grp. gram.:m.}
\end{itemize}
\begin{itemize}
\item {Proveniência:(T. ár.)}
\end{itemize}
Pequeno tributo, que os Christãos e Judeus pagavam aos Turcos, sob cujo domínio viviam.
\section{Gafaria}
\begin{itemize}
\item {Grp. gram.:f.}
\end{itemize}
\begin{itemize}
\item {Utilização:Ant.}
\end{itemize}
\begin{itemize}
\item {Proveniência:(De \textunderscore gafa\textunderscore ^1)}
\end{itemize}
Hospital para leprosos.
O mesmo que \textunderscore gafeira\textunderscore :«\textunderscore ...a gafaria espiritual das confessadas\textunderscore ». Camillo, \textunderscore Brasileira\textunderscore , 341.
\section{Gafeira}
\begin{itemize}
\item {Grp. gram.:f.}
\end{itemize}
\begin{itemize}
\item {Proveniência:(De \textunderscore gafa\textunderscore ^1)}
\end{itemize}
Lepra.
Sarna leprosa de certos animaes; morrinha.
Doença dos olhos dos bois, com inchação das pálpebras.
\section{Gafeirento}
\begin{itemize}
\item {Grp. gram.:adj.}
\end{itemize}
Que tem gafeira.
\section{Gafeiroso}
\begin{itemize}
\item {Grp. gram.:adj.}
\end{itemize}
O mesmo que \textunderscore gafeirento\textunderscore .
\section{Gafém}
\begin{itemize}
\item {Grp. gram.:m.}
\end{itemize}
\begin{itemize}
\item {Utilização:Ant.}
\end{itemize}
(V.gafeira)
\section{Gafento}
\begin{itemize}
\item {Grp. gram.:adj.}
\end{itemize}
O mesmo que \textunderscore gafeirento\textunderscore .
\section{Gafetope}
\begin{itemize}
\item {Grp. gram.:f.}
\end{itemize}
\begin{itemize}
\item {Utilização:Náut.}
\end{itemize}
\begin{itemize}
\item {Proveniência:(Ingl. \textunderscore gaff-top\textunderscore )}
\end{itemize}
Vela triangular, que se prende aos mastaréus. Cf. Camillo, \textunderscore Myst. de Lisb.\textunderscore , II, 225.
\section{Gafidade}
\begin{itemize}
\item {Grp. gram.:f.}
\end{itemize}
\begin{itemize}
\item {Utilização:Ant.}
\end{itemize}
O mesmo que \textunderscore gafeira\textunderscore .
\section{Gáfio}
\begin{itemize}
\item {Grp. gram.:m.}
\end{itemize}
(V.mandioca)
\section{Gafo}
\begin{itemize}
\item {Grp. gram.:adj.}
\end{itemize}
\begin{itemize}
\item {Utilização:Fig.}
\end{itemize}
\begin{itemize}
\item {Grp. gram.:M.}
\end{itemize}
O mesmo que \textunderscore gafeirento\textunderscore .
Leproso.
Corrompido; desmoralizado.
Gafeira.
(Cp. \textunderscore gafa\textunderscore ^1)
\section{Gafo}
\begin{itemize}
\item {Grp. gram.:adj.}
\end{itemize}
\begin{itemize}
\item {Utilização:Prov.}
\end{itemize}
\begin{itemize}
\item {Utilização:alg.}
\end{itemize}
O mesmo que \textunderscore cheio\textunderscore .
\section{Gaforina}
\begin{itemize}
\item {Grp. gram.:f.}
\end{itemize}
\begin{itemize}
\item {Utilização:Fam.}
\end{itemize}
\begin{itemize}
\item {Proveniência:(De \textunderscore Gafforini\textunderscore , n. p.)}
\end{itemize}
Grenha; cabello em desalinho.
Topête:«\textunderscore ...dar lustro a caracóes e a gaforinas\textunderscore ». Macedo, \textunderscore Burros\textunderscore , 305.
\section{Gagaísta}
\begin{itemize}
\item {Grp. gram.:m.}
\end{itemize}
Espécie de feiticeiro ou sacerdote preto, que consulta o gagau.
\section{Gaganho}
\begin{itemize}
\item {Grp. gram.:m.  e  adj.}
\end{itemize}
\begin{itemize}
\item {Utilização:Prov.}
\end{itemize}
\begin{itemize}
\item {Utilização:beir.}
\end{itemize}
O mesmo que \textunderscore gago\textunderscore . (Colhido na Guarda)
\section{Gagão}
\begin{itemize}
\item {Grp. gram.:m.}
\end{itemize}
\begin{itemize}
\item {Utilização:Des.}
\end{itemize}
Jôgo de rapazes, com tambores ou com objectos que imitassem tambores.
\section{Gagau}
\begin{itemize}
\item {Grp. gram.:m.}
\end{itemize}
Conjunto de ossos de cabrito e de hyena, misturados com seixos brancos e pretos, e que constituem o oráculo dos Negros das vizinhanças de Lourenço-Marques.
\section{Gage}
\begin{itemize}
\item {Grp. gram.:m.}
\end{itemize}
\begin{itemize}
\item {Utilização:Ant.}
\end{itemize}
\begin{itemize}
\item {Grp. gram.:F.}
\end{itemize}
\begin{itemize}
\item {Utilização:Ant.}
\end{itemize}
\begin{itemize}
\item {Proveniência:(Fr. \textunderscore gaje\textunderscore )}
\end{itemize}
Objecto, que se deu em penhor.
Lucro.
Percalço.
\section{Gagé}
\begin{itemize}
\item {Grp. gram.:m.}
\end{itemize}
\begin{itemize}
\item {Utilização:Pop.}
\end{itemize}
\begin{itemize}
\item {Proveniência:(Do fr. \textunderscore degagé\textunderscore ?)}
\end{itemize}
Donaire; garbo; elegância.
\section{Gageiro}
\begin{itemize}
\item {Grp. gram.:m.}
\end{itemize}
\begin{itemize}
\item {Grp. gram.:Adj.}
\end{itemize}
\begin{itemize}
\item {Proveniência:(Do it. \textunderscore gaggia\textunderscore )}
\end{itemize}
Marinheiro, que vigia o mastro, e que da gávea observa e vigia as embarcações ou a terra.
Que trepa facilmente.
\section{Gageru}
\begin{itemize}
\item {Grp. gram.:m.}
\end{itemize}
Arbusto rosáceo do Brasil.
\section{Gago}
\begin{itemize}
\item {Grp. gram.:m.  e  adj.}
\end{itemize}
\begin{itemize}
\item {Proveniência:(T. onom.)}
\end{itemize}
Aquelle que gagueja.
\section{Gagosa}
\begin{itemize}
\item {Grp. gram.:f.}
\end{itemize}
Us. na loc. adv. \textunderscore á gagosa\textunderscore , sem custo, sem trabalho.
Á socapa, sorrateiramente.
\section{Gagosa}
\begin{itemize}
\item {Grp. gram.:f.}
\end{itemize}
Ave, o mesmo que \textunderscore chapalheta\textunderscore .
\section{Gaguear}
\begin{itemize}
\item {Grp. gram.:v. i.}
\end{itemize}
\begin{itemize}
\item {Utilização:Prov.}
\end{itemize}
\begin{itemize}
\item {Utilização:minh.}
\end{itemize}
\begin{itemize}
\item {Proveniência:(De \textunderscore gago\textunderscore ? Ou t. onom.?)}
\end{itemize}
Diz-se da gallinha, quando canta, a chamar o gallo.
\section{Gagueio}
\begin{itemize}
\item {Grp. gram.:m.}
\end{itemize}
Acto de gaguear ou balbuciar.
\section{Gagueira}
\begin{itemize}
\item {Grp. gram.:f.}
\end{itemize}
O mesmo que \textunderscore gaguez\textunderscore .
\section{Gaguejador}
\begin{itemize}
\item {Grp. gram.:adj.}
\end{itemize}
Que gagueja.
\section{Gaguejar}
\begin{itemize}
\item {Grp. gram.:v. t.}
\end{itemize}
\begin{itemize}
\item {Grp. gram.:V. i.}
\end{itemize}
\begin{itemize}
\item {Proveniência:(De \textunderscore gago\textunderscore )}
\end{itemize}
Pronunciar com hesitação, tartamudeando: \textunderscore gaguejar desculpas\textunderscore .
Falar com embaraço, repetindo involuntariamente, com maior ou menor intervallo, as sýllabas ou palavras.
Tartamudear.
\section{Gaguez}
\begin{itemize}
\item {Grp. gram.:f.}
\end{itemize}
Qualidade de quem é gago.
\section{Gaguice}
\begin{itemize}
\item {Grp. gram.:f.}
\end{itemize}
O mesmo que \textunderscore gaguez\textunderscore .
\section{Gahnite}
\begin{itemize}
\item {Grp. gram.:f.}
\end{itemize}
\begin{itemize}
\item {Proveniência:(De \textunderscore Gahn\textunderscore , n. p.)}
\end{itemize}
Uma das espécies da espinella.
\section{Gaiacena}
\begin{itemize}
\item {Grp. gram.:f.}
\end{itemize}
Essência da resina de gaiaco.
\section{Gaiacetina}
\begin{itemize}
\item {Grp. gram.:f.}
\end{itemize}
\begin{itemize}
\item {Proveniência:(De \textunderscore gaiaco\textunderscore )}
\end{itemize}
Medicamento, com applicações análogas ás do gaiacol.
\section{Gaiácico}
\begin{itemize}
\item {Grp. gram.:adj.}
\end{itemize}
Diz-se de um ácido, extrahido do gaiaco.
\section{Gaiacina}
\begin{itemize}
\item {Grp. gram.:f.}
\end{itemize}
O mesmo que \textunderscore gaiacena\textunderscore .
\section{Gaiaco}
\begin{itemize}
\item {Grp. gram.:m.}
\end{itemize}
Árvore resinosa, americana, (\textunderscore gaiacum officinale\textunderscore , Lin.).
\section{Gaiacol}
\begin{itemize}
\item {Grp. gram.:m.}
\end{itemize}
\begin{itemize}
\item {Proveniência:(De \textunderscore gaiaco\textunderscore  + \textunderscore álcool\textunderscore )}
\end{itemize}
Tintura de gaiaco ou solução de glacina em álcool.
\section{Gaiado}
\begin{itemize}
\item {Grp. gram.:m.}
\end{itemize}
\begin{itemize}
\item {Grp. gram.:Adj.}
\end{itemize}
\begin{itemize}
\item {Proveniência:(De \textunderscore gaias\textunderscore )}
\end{itemize}
Peixe escômbrida.
Diz-se do cavallo, que tem redemoínho nos pêlos do peito.
\section{Gaiar}
\begin{itemize}
\item {Grp. gram.:v. i.}
\end{itemize}
\begin{itemize}
\item {Utilização:Prov.}
\end{itemize}
\begin{itemize}
\item {Utilização:beir.}
\end{itemize}
O mesmo que \textunderscore guaiar\textunderscore .
\section{Gaiar}
\begin{itemize}
\item {Grp. gram.:v. i.}
\end{itemize}
\begin{itemize}
\item {Utilização:Prov.}
\end{itemize}
\begin{itemize}
\item {Utilização:trasm.}
\end{itemize}
Não ir á escola, fazer parede (o estudante). (Colhido em Murça)
(Relaciona-se com \textunderscore gaio\textunderscore ? ou com \textunderscore gandaiar\textunderscore ?)
\section{Gaias}
\begin{itemize}
\item {Grp. gram.:m. pl.}
\end{itemize}
Redemoínho de pêlos, no peito do cavallo, ou nos quartos da base da cauda.
\section{Gaia-sciência}
\begin{itemize}
\item {Grp. gram.:f.}
\end{itemize}
Arte de poetar, entre os Provençaes da Idade-Média.
\section{Gaiatada}
\begin{itemize}
\item {Grp. gram.:f.}
\end{itemize}
Agrupamento de gaiatos.
Acção ou palavras de gaiato.
\section{Gaiatar}
\begin{itemize}
\item {Grp. gram.:v. i.}
\end{itemize}
Proceder como gaiato; garotar.
\section{Gaiatete}
\begin{itemize}
\item {fónica:tê}
\end{itemize}
\begin{itemize}
\item {Grp. gram.:m.}
\end{itemize}
Pequeno gaiato.
\section{Gaiatice}
\begin{itemize}
\item {Grp. gram.:f.}
\end{itemize}
Acção ou palavras próprias de gaiato.
\section{Gaiato}
\begin{itemize}
\item {Grp. gram.:m.}
\end{itemize}
\begin{itemize}
\item {Grp. gram.:Adj.}
\end{itemize}
\begin{itemize}
\item {Proveniência:(De \textunderscore gaio\textunderscore ^1)}
\end{itemize}
Rapaz travesso e vadio; garoto.
Travesso; alegre; malicioso: \textunderscore olhares gaiatos\textunderscore .
\section{Gaibeia}
\begin{itemize}
\item {Grp. gram.:f.}
\end{itemize}
\begin{itemize}
\item {Proveniência:(De \textunderscore gaibéu\textunderscore )}
\end{itemize}
Mulher, que trabalha na monda de searas, no Ribatejo.
\section{Gaibéu}
\begin{itemize}
\item {Grp. gram.:m.}
\end{itemize}
Mondador, no Ribatejo.(V.gaivéu)
\section{Gaifona}
\begin{itemize}
\item {Grp. gram.:f.}
\end{itemize}
\begin{itemize}
\item {Utilização:Pop.}
\end{itemize}
Trejeito; esgar; momice.
\section{Gaifonar}
\begin{itemize}
\item {Grp. gram.:v. i.}
\end{itemize}
Fazer gaifonas.
\section{Gaifonice}
\begin{itemize}
\item {Grp. gram.:f.}
\end{itemize}
O mesmo que \textunderscore gaifona\textunderscore .
\section{Gaillárdia}
\begin{itemize}
\item {Grp. gram.:f.}
\end{itemize}
\begin{itemize}
\item {Proveniência:(De \textunderscore Gaillard\textunderscore , n. p.)}
\end{itemize}
Planta annual brasileira.
\section{Gaimão}
\begin{itemize}
\item {Grp. gram.:m.}
\end{itemize}
\begin{itemize}
\item {Utilização:Prov.}
\end{itemize}
\begin{itemize}
\item {Utilização:trasm.}
\end{itemize}
Haste florida das abróteas.
\section{Gaimenho}
\begin{itemize}
\item {Grp. gram.:adj.}
\end{itemize}
\begin{itemize}
\item {Utilização:Prov.}
\end{itemize}
\begin{itemize}
\item {Utilização:trasm.}
\end{itemize}
Despreoccupado; confiado em si.
(Provavelmente, corr. de \textunderscore gamenho\textunderscore )
\section{Gaínha}
\begin{itemize}
\item {Grp. gram.:adj. f.}
\end{itemize}
\begin{itemize}
\item {Utilização:Prov.}
\end{itemize}
\begin{itemize}
\item {Utilização:trasm.}
\end{itemize}
Diz-se da fala muito fina ou effeminada num homem.
(Relaciona-se com \textunderscore gaio\textunderscore ^3?)
\section{Gaíncha}
\begin{itemize}
\item {Grp. gram.:f.}
\end{itemize}
\begin{itemize}
\item {Utilização:Ant.}
\end{itemize}
Uma das pertenças da bésta.
(Relaciona-se com \textunderscore gancho\textunderscore ? Cp. \textunderscore gafa\textunderscore ^1)
\section{Gaio}
\begin{itemize}
\item {Grp. gram.:adj.}
\end{itemize}
\begin{itemize}
\item {Grp. gram.:M.}
\end{itemize}
\begin{itemize}
\item {Utilização:Mad}
\end{itemize}
\begin{itemize}
\item {Utilização:Gír.}
\end{itemize}
\begin{itemize}
\item {Proveniência:(Do fr. \textunderscore gai\textunderscore )}
\end{itemize}
Jovial; alegre.
Ave, de pennas mosqueadas, e do tamanho da pêga.
Nome da gaivota que não tem mais de um anno.
Cavallo.
\section{Gaio}
\begin{itemize}
\item {Grp. gram.:m.}
\end{itemize}
\begin{itemize}
\item {Utilização:Bras}
\end{itemize}
Braço de uma espécie de antenna, que serve para amarrar a embarcação. Cf. M. de Aguiar, \textunderscore Diccion. de Marinha\textunderscore .
\section{Gaio}
\begin{itemize}
\item {Grp. gram.:m.}
\end{itemize}
\begin{itemize}
\item {Utilização:Prov.}
\end{itemize}
Varinha muito flexível, terminada na parte superior por umas laçadas, feitas da própria vara, vergada.
\section{Gaiola}
\begin{itemize}
\item {Grp. gram.:f.}
\end{itemize}
\begin{itemize}
\item {Utilização:Fig.}
\end{itemize}
\begin{itemize}
\item {Utilização:Fam.}
\end{itemize}
\begin{itemize}
\item {Utilização:Bras. do N}
\end{itemize}
\begin{itemize}
\item {Proveniência:(Do lat. \textunderscore caveola\textunderscore , de \textunderscore cavea\textunderscore )}
\end{itemize}
Pequena clausura móvel, feita de canas, junco, arame, etc., para aves.
Jaula.
Cárcere.
Espaço, comprehendido pelo madeiramento e paredes de uma casa.
Armação de ripas ou tábuas estreitas, para transporte de móveis.
Toiril.
Casinhola.
Pequeno navio do Pará e do Amazonas. (Mais us. no gen. masculino)
\section{Gaioleiro}
\begin{itemize}
\item {Grp. gram.:m.}
\end{itemize}
Fabricante ou vendedor de gaiolas.
\section{Gaiolim}
\begin{itemize}
\item {Grp. gram.:m.}
\end{itemize}
Pequena gaiola.
\section{Gaiolo}
\begin{itemize}
\item {fónica:ô}
\end{itemize}
\begin{itemize}
\item {Grp. gram.:adj.}
\end{itemize}
\begin{itemize}
\item {Grp. gram.:M.}
\end{itemize}
\begin{itemize}
\item {Utilização:Prov.}
\end{itemize}
\begin{itemize}
\item {Proveniência:(De \textunderscore gaiola\textunderscore )}
\end{itemize}
Diz-se do toiro, que tem os cornos em fórma de meia lua e muito próximos nas pontas.
Armadilha para caçar pássaros, feita de varas encruzadas, formando pyrâmide regular de base quadrada.
\section{Gaiorros}
\begin{itemize}
\item {fónica:ô}
\end{itemize}
\begin{itemize}
\item {Grp. gram.:m. pl.}
\end{itemize}
\begin{itemize}
\item {Utilização:Prov.}
\end{itemize}
\begin{itemize}
\item {Utilização:dur.}
\end{itemize}
Feijões, também chamados \textunderscore fradinhos\textunderscore .
\section{Gaiosa}
\begin{itemize}
\item {Grp. gram.:f.}
\end{itemize}
\begin{itemize}
\item {Utilização:Prov.}
\end{itemize}
\begin{itemize}
\item {Utilização:trasm.}
\end{itemize}
\begin{itemize}
\item {Proveniência:(De \textunderscore gaio\textunderscore ^1)}
\end{itemize}
Presente, que os emphyteutas davam aos senhorios, em certos dias festivos. Cf. Herculano, \textunderscore Hist. de Port.\textunderscore , III, 442.
\textunderscore Viver á gaiosa\textunderscore , andar á tuna, gandaiar.
\section{Gaioto}
\begin{itemize}
\item {fónica:ô}
\end{itemize}
\begin{itemize}
\item {Grp. gram.:m.}
\end{itemize}
\begin{itemize}
\item {Utilização:Prov.}
\end{itemize}
\begin{itemize}
\item {Utilização:trasm.}
\end{itemize}
Ave, o gaio macho. (Colhido na Régua)
\section{Gaipa}
\begin{itemize}
\item {Grp. gram.:f.}
\end{itemize}
\begin{itemize}
\item {Utilização:Prov.}
\end{itemize}
\begin{itemize}
\item {Utilização:minh.}
\end{itemize}
Cacho de uvas.
\section{Gaipeiro}
\begin{itemize}
\item {Grp. gram.:adj.}
\end{itemize}
\begin{itemize}
\item {Utilização:Prov.}
\end{itemize}
\begin{itemize}
\item {Utilização:minh.}
\end{itemize}
\begin{itemize}
\item {Proveniência:(De \textunderscore gaipa\textunderscore )}
\end{itemize}
Aquelle que furta uvas.
Aquelle que gosta muito de uvas.
\section{Gaipelo}
\begin{itemize}
\item {fónica:pê}
\end{itemize}
\begin{itemize}
\item {Grp. gram.:m.}
\end{itemize}
\begin{itemize}
\item {Utilização:Prov.}
\end{itemize}
\begin{itemize}
\item {Utilização:minh.}
\end{itemize}
\begin{itemize}
\item {Proveniência:(De \textunderscore gaipa\textunderscore )}
\end{itemize}
Uma das ramificações do eixo central do cacho de uvas.
\section{Gaipilha}
\begin{itemize}
\item {Grp. gram.:m.}
\end{itemize}
\begin{itemize}
\item {Utilização:Prov.}
\end{itemize}
\begin{itemize}
\item {Utilização:minh.}
\end{itemize}
Aquelle que gosta muito dos gaipos ou cachos de uvas.
Aquelle que vai roubar uvas á vinha.
Larápio.
\section{Gaipo}
\begin{itemize}
\item {Grp. gram.:m.}
\end{itemize}
\begin{itemize}
\item {Utilização:Prov.}
\end{itemize}
\begin{itemize}
\item {Utilização:Prov.}
\end{itemize}
\begin{itemize}
\item {Utilização:beir.}
\end{itemize}
O mesmo que \textunderscore gaipa\textunderscore .
O mesmo que \textunderscore chifre\textunderscore .
Rebento da videira, sarmento.
\section{Gairo}
\begin{itemize}
\item {Grp. gram.:m.}
\end{itemize}
\begin{itemize}
\item {Proveniência:(Do conc. \textunderscore garial\textunderscore )}
\end{itemize}
Árvore mimósea da Índia, de vagens tenras e comestíveis.
\section{Gaita}
\begin{itemize}
\item {Grp. gram.:f.}
\end{itemize}
\begin{itemize}
\item {Utilização:Bras. de Minas}
\end{itemize}
\begin{itemize}
\item {Utilização:Escol.}
\end{itemize}
\begin{itemize}
\item {Utilização:Chul.}
\end{itemize}
\begin{itemize}
\item {Grp. gram.:Loc.}
\end{itemize}
\begin{itemize}
\item {Utilização:pop.}
\end{itemize}
\begin{itemize}
\item {Utilização:Pop.}
\end{itemize}
\begin{itemize}
\item {Utilização:Prov.}
\end{itemize}
\begin{itemize}
\item {Utilização:alg.}
\end{itemize}
\begin{itemize}
\item {Grp. gram.:Pl.}
\end{itemize}
Rude instrumento de sopro, formado por um canudo com vários buracos; pífaro.
Qualquer pequeno instrumento, de metal ou madeira, para crianças.
Troça.
Reprovação.
Pênis.
\textunderscore Gaita de capador\textunderscore , instrumento, composto de uma fileira de pequenos canudos de diversas dimensões, sôbre os quaes se sopra, produzindo som differente em cada um.
\textunderscore Ir-se á gaita\textunderscore , mallograr-se.
\textunderscore Gaita de folles\textunderscore , ou \textunderscore gaita gallega\textunderscore , instrumento, formado por um saco de coiro cheio de ar, e por dois tubos.
Côrno.
Mulher gaiteira.
Nome, que se dá a uns furos, que as lampreias têm por baixo da bôca.
\textunderscore Saber a gaitas\textunderscore , ou \textunderscore saber que nem gaitas\textunderscore , têr bom sabor, sêr delicioso.--Li nuns apontamentos de Castilho que na Beira se chamam \textunderscore gaitas\textunderscore  as couves, e que daí virá talvez o prolóquio \textunderscore sabe que nem gaitas\textunderscore . Nunca lá ouvi o termo nessa accepção, o que aliás não prova que êlle não exista. Em apontamentos de um illustre official de marinha, vi também que antigamente se chamou \textunderscore gaita\textunderscore  á lampreia, e que dalli veio o alludido prolóquio. Ignoro o fundamento desta allegação, se bem que, como vimos, se chamam \textunderscore gaitas\textunderscore  uns furos que a lampreia tem debaixo da bôca.--Mais uma explicação conjectural da loc. fam. \textunderscore sabe que nem gaitas\textunderscore .--Dantes, ás portas do antigo Passeio Público de Lisbôa, vendiam-se uns canudinhos de doce, chamados \textunderscore gaitas\textunderscore , que o povo apreciava e saboreava. Viria daqui a loc. \textunderscore sabe que nem gaitas\textunderscore ? Cf. Pinto de Carvalho, \textunderscore Lisbôa de Outros Tempos\textunderscore , II, 195.
\section{Gaitada}
\begin{itemize}
\item {Grp. gram.:f.}
\end{itemize}
\begin{itemize}
\item {Utilização:Prov.}
\end{itemize}
\begin{itemize}
\item {Utilização:Chul.}
\end{itemize}
\begin{itemize}
\item {Utilização:Deprec.}
\end{itemize}
\begin{itemize}
\item {Utilização:Pop.}
\end{itemize}
\begin{itemize}
\item {Utilização:bras. do N}
\end{itemize}
\begin{itemize}
\item {Utilização:Açor}
\end{itemize}
\begin{itemize}
\item {Proveniência:(De \textunderscore gaita\textunderscore )}
\end{itemize}
Toque de gaita.
Pancada com gaita.
Censura, reprehensão.
Trecho de música instrumental.
Marrada.
Gargalhada.
\section{Gaitado}
\begin{itemize}
\item {Grp. gram.:adj.}
\end{itemize}
\begin{itemize}
\item {Utilização:Escol.}
\end{itemize}
\begin{itemize}
\item {Proveniência:(De \textunderscore gaita\textunderscore ^1)}
\end{itemize}
Reprovado.
\section{Gaitar}
\begin{itemize}
\item {Grp. gram.:v. i.}
\end{itemize}
\begin{itemize}
\item {Utilização:Prov.}
\end{itemize}
\begin{itemize}
\item {Utilização:trasm.}
\end{itemize}
\begin{itemize}
\item {Proveniência:(De \textunderscore gaita\textunderscore )}
\end{itemize}
Chorar (a criança)
\section{Gaitear}
\begin{itemize}
\item {Grp. gram.:v. i.}
\end{itemize}
\begin{itemize}
\item {Utilização:Fig.}
\end{itemize}
\begin{itemize}
\item {Utilização:Bras. do N}
\end{itemize}
\begin{itemize}
\item {Utilização:Bras. do N}
\end{itemize}
Tocar gaita.
Foliar.
Urrar (o toiro).
Imitar o som da gaita.
\section{Gaiteiro}
\begin{itemize}
\item {Grp. gram.:m.}
\end{itemize}
\begin{itemize}
\item {Grp. gram.:Adj.}
\end{itemize}
Tocador de gaita.
Que gaiteia.
Que é peralta; garrido; folião.
\section{Gaitona}
\begin{itemize}
\item {Grp. gram.:f.}
\end{itemize}
\begin{itemize}
\item {Utilização:Prov.}
\end{itemize}
\begin{itemize}
\item {Utilização:alg.}
\end{itemize}
\begin{itemize}
\item {Proveniência:(De \textunderscore gaita\textunderscore )}
\end{itemize}
Mulher foliona, mal comportada.
\section{Gaiuta}
\begin{itemize}
\item {Grp. gram.:f.}
\end{itemize}
\begin{itemize}
\item {Utilização:Náut.}
\end{itemize}
\begin{itemize}
\item {Utilização:Ext.}
\end{itemize}
\begin{itemize}
\item {Proveniência:(Fr. \textunderscore cahute\textunderscore )}
\end{itemize}
Cúpula, que cobre uma escotilha redonda. Cf. M. de Aguiar, \textunderscore Diccion. de Marinha\textunderscore .
Parte externa e accessória dos edifícios modernos, com a qual se resguardam as latrinas ou urinoes.
\section{Gaiva}
\begin{itemize}
\item {Grp. gram.:f.}
\end{itemize}
\begin{itemize}
\item {Utilização:Prov.}
\end{itemize}
\begin{itemize}
\item {Utilização:dur.}
\end{itemize}
\begin{itemize}
\item {Utilização:Gír.}
\end{itemize}
\begin{itemize}
\item {Utilização:Ant.}
\end{itemize}
\begin{itemize}
\item {Proveniência:(Do lat. \textunderscore cavea\textunderscore )}
\end{itemize}
O mesmo que \textunderscore goivadura\textunderscore .
Escavação ou fenda, feita na terra por águas pluviaes.
Gaveta.
Cava ou fôsso no castello.
\section{Gaivação}
\begin{itemize}
\item {Grp. gram.:f.}
\end{itemize}
Acto de gaivar.
\section{Gaivagem}
\begin{itemize}
\item {Grp. gram.:f.}
\end{itemize}
\begin{itemize}
\item {Proveniência:(De \textunderscore gaivar\textunderscore )}
\end{itemize}
Rêgo fundo ou valla estreita, para esgôto ou derivação de águas.
Drainagem. Cf. \textunderscore Código Civil\textunderscore , art. 462.
\section{Gaivão}
\begin{itemize}
\item {Grp. gram.:m.}
\end{itemize}
\begin{itemize}
\item {Proveniência:(Do lat. \textunderscore gavia\textunderscore )}
\end{itemize}
Ave, da fam. das andorinhas, (\textunderscore sterna hirundo\textunderscore , Lin.).
\section{Gaivão}
\begin{itemize}
\item {Grp. gram.:m.}
\end{itemize}
\begin{itemize}
\item {Utilização:Bras}
\end{itemize}
Apparelho de pesca, de fórma cónica.
\section{Gaivar}
\begin{itemize}
\item {Grp. gram.:v. t.}
\end{itemize}
\begin{itemize}
\item {Proveniência:(De \textunderscore gaiva\textunderscore )}
\end{itemize}
Fazer gaivagem em.
\section{Gaivel}
\begin{itemize}
\item {Grp. gram.:m.}
\end{itemize}
\begin{itemize}
\item {Proveniência:(De \textunderscore gaiva\textunderscore )}
\end{itemize}
Parede, que vai deminuindo de espessura, da base para cima.
\section{Gaivéu}
\begin{itemize}
\item {Grp. gram.:m.}
\end{itemize}
\begin{itemize}
\item {Utilização:Prov.}
\end{itemize}
\begin{itemize}
\item {Utilização:alent.}
\end{itemize}
\begin{itemize}
\item {Proveniência:(De \textunderscore gaivar\textunderscore , de \textunderscore gaiva\textunderscore , fôsso, escavação, drainagem)}
\end{itemize}
O mesmo que \textunderscore ratinho\textunderscore ^1, trabalhador do norte, que foi empregar-se na monda e noutros trabalhos agrícolas do Alentejo.
\section{Gaivina}
\begin{itemize}
\item {Grp. gram.:f.}
\end{itemize}
Ave palmípede, conhecida também por \textunderscore andorinha do mar\textunderscore , (\textunderscore sterna minuta\textunderscore , Lin.).
(Cp. \textunderscore gaivota\textunderscore )
\section{Gaivinha}
\begin{itemize}
\item {Grp. gram.:f.}
\end{itemize}
O mesmo que \textunderscore gaivina\textunderscore .
\section{Gaivota}
\begin{itemize}
\item {Grp. gram.:f.}
\end{itemize}
Ave palmípede e aquática.
(Cp. cast. \textunderscore gaviota\textunderscore , do lat. \textunderscore gavia\textunderscore )
\section{Gaivotão}
\begin{itemize}
\item {Grp. gram.:m.}
\end{itemize}
Ave do gênero da gaivota.
\section{Gaivotear}
\begin{itemize}
\item {Grp. gram.:v. i.}
\end{itemize}
\begin{itemize}
\item {Utilização:des.}
\end{itemize}
\begin{itemize}
\item {Utilização:Fam.}
\end{itemize}
\begin{itemize}
\item {Proveniência:(De \textunderscore gaivota\textunderscore )}
\end{itemize}
Afagar ironicamente ou com acinte; zombar, afagando.
\section{Gaivoto}
\begin{itemize}
\item {fónica:vô}
\end{itemize}
\begin{itemize}
\item {Grp. gram.:m.}
\end{itemize}
\begin{itemize}
\item {Utilização:T. de Dio}
\end{itemize}
O mesmo que \textunderscore gaivota\textunderscore .
\section{Gajaderopa}
\begin{itemize}
\item {Grp. gram.:f.}
\end{itemize}
Espécie de marisco, conhecido também por \textunderscore pé de burro\textunderscore .
\section{Gajandumbo}
\begin{itemize}
\item {Grp. gram.:m.}
\end{itemize}
Pássaro dentirostro da África occidental.
\section{Gajão}
\begin{itemize}
\item {Grp. gram.:m.}
\end{itemize}
Título obsequioso, com que os ciganos, no Brasil, tratam as pessôas estranhas á sua raça.
(Cp. \textunderscore gajo\textunderscore )
\section{Gajar}
\begin{itemize}
\item {Grp. gram.:v. i.}
\end{itemize}
\begin{itemize}
\item {Utilização:Prov.}
\end{itemize}
\begin{itemize}
\item {Utilização:trasm.}
\end{itemize}
Fazer barulho.
\section{Gájara}
\begin{itemize}
\item {Grp. gram.:f. pl.}
\end{itemize}
\begin{itemize}
\item {Utilização:Prov.}
\end{itemize}
\begin{itemize}
\item {Utilização:trasm.}
\end{itemize}
Comestíveis, que se dão aos ceifadores de empreitada, além da paga em dinheiro.
(Cp. \textunderscore gajas\textunderscore )
\section{Gajas}
\begin{itemize}
\item {Grp. gram.:f. pl.}
\end{itemize}
\begin{itemize}
\item {Utilização:ant.}
\end{itemize}
Soldada.
Expensas; custa.
(Cp. \textunderscore gage\textunderscore )
\section{Gajata}
\begin{itemize}
\item {Grp. gram.:f.}
\end{itemize}
\begin{itemize}
\item {Utilização:Prov.}
\end{itemize}
\begin{itemize}
\item {Utilização:trasm.}
\end{itemize}
O mesmo que \textunderscore cajado\textunderscore .
\section{Gajato}
\begin{itemize}
\item {Grp. gram.:f.}
\end{itemize}
\begin{itemize}
\item {Utilização:Prov.}
\end{itemize}
\begin{itemize}
\item {Utilização:trasm.}
\end{itemize}
Gajata.
Qualquer coisa torta.
Rabisco, feito com a penna por quem começa a escrever.
(Corr. de \textunderscore cajado\textunderscore )
\section{Gajavato}
\begin{itemize}
\item {Grp. gram.:m.}
\end{itemize}
\begin{itemize}
\item {Utilização:Prov.}
\end{itemize}
\begin{itemize}
\item {Utilização:trasm.}
\end{itemize}
O mesmo que \textunderscore gajato\textunderscore .
\section{Gajo}
\begin{itemize}
\item {Grp. gram.:m.}
\end{itemize}
\begin{itemize}
\item {Utilização:Chul.}
\end{itemize}
\begin{itemize}
\item {Grp. gram.:Adj.}
\end{itemize}
Matulão.
Súcio; typo.
Finório, velhaco.
(Do cigano de Espanha, \textunderscore gachó\textunderscore )
\section{Gala}
\begin{itemize}
\item {Grp. gram.:f.}
\end{itemize}
\begin{itemize}
\item {Proveniência:(Do ant. alt. al. \textunderscore geili\textunderscore , brio)}
\end{itemize}
Traje para actos solemnes:«\textunderscore deixolhe 200:000 reis, para hũa gala, advertindolhe que não seja de luto\textunderscore ». (De um testamento de 1691)
Pompa; ornamentos preciosos.
Festa nacional.
Solemnidade.
Regozijo.
Ostentação, jactância.
\section{Galacé}
\begin{itemize}
\item {Grp. gram.:m.}
\end{itemize}
\begin{itemize}
\item {Utilização:Des.}
\end{itemize}
Galão estreito.
\section{Galactagogo}
\begin{itemize}
\item {Grp. gram.:adj.}
\end{itemize}
\begin{itemize}
\item {Grp. gram.:M.}
\end{itemize}
\begin{itemize}
\item {Proveniência:(Do gr. \textunderscore gala\textunderscore  + \textunderscore agogos\textunderscore )}
\end{itemize}
Que faz aumentar a secreção do leite.
Meio ou substância, que se emprega para aumentar a secreção do leite.
\section{Galactífero}
\begin{itemize}
\item {Grp. gram.:adj.}
\end{itemize}
O mesmo que \textunderscore galactóphoro\textunderscore .
\section{Galactite}
\begin{itemize}
\item {Grp. gram.:f.}
\end{itemize}
Pedra preciosa, da côr do leite.
(Cp. \textunderscore galactites\textunderscore )
\section{Galactites}
\begin{itemize}
\item {Grp. gram.:f. pl.}
\end{itemize}
\begin{itemize}
\item {Proveniência:(Gr. \textunderscore galaktites\textunderscore )}
\end{itemize}
Gênero de plantas compostas, que contém um suco branco.
O mesmo que \textunderscore galactite\textunderscore .
\section{Galactocele}
\begin{itemize}
\item {Grp. gram.:f.}
\end{itemize}
\begin{itemize}
\item {Utilização:Med.}
\end{itemize}
\begin{itemize}
\item {Proveniência:(Do gr. \textunderscore gala\textunderscore , \textunderscore galaktos\textunderscore  + \textunderscore kele\textunderscore )}
\end{itemize}
Engorgitamento da mama, por causa do leite.
\section{Galactófago}
\begin{itemize}
\item {Grp. gram.:adj.}
\end{itemize}
\begin{itemize}
\item {Proveniência:(Do gr. \textunderscore gala\textunderscore , \textunderscore galaktos\textunderscore  + \textunderscore phagein\textunderscore )}
\end{itemize}
Diz-se de todos os animaes mamíferos ou que se alimentaram de leite, antes de comer.
\section{Galactoforite}
\begin{itemize}
\item {Grp. gram.:f.}
\end{itemize}
\begin{itemize}
\item {Utilização:Med.}
\end{itemize}
Inflamação dos conductos galactóforos.
\section{Galactóforo}
\begin{itemize}
\item {Grp. gram.:adj.}
\end{itemize}
\begin{itemize}
\item {Proveniência:(Do gr. \textunderscore gala\textunderscore , \textunderscore galaktos\textunderscore  + \textunderscore phoros\textunderscore )}
\end{itemize}
Que produz leite.
\section{Galactogênio}
\begin{itemize}
\item {Grp. gram.:m.}
\end{itemize}
O mesmo que \textunderscore galactagogo\textunderscore , m.
\section{Galactógeno}
\begin{itemize}
\item {Grp. gram.:adj.}
\end{itemize}
O mesmo que \textunderscore galactagogo\textunderscore , adj.
\section{Galactografia}
\begin{itemize}
\item {Grp. gram.:f.}
\end{itemize}
\begin{itemize}
\item {Proveniência:(Do gr. \textunderscore gala\textunderscore , \textunderscore galaktos\textunderscore  + \textunderscore graphein\textunderscore )}
\end{itemize}
Parte da Anatomia, que tem por objecto a descripção dos sucos lácteos.
\section{Galactographia}
\begin{itemize}
\item {Grp. gram.:f.}
\end{itemize}
\begin{itemize}
\item {Proveniência:(Do gr. \textunderscore gala\textunderscore , \textunderscore galaktos\textunderscore  + \textunderscore graphein\textunderscore )}
\end{itemize}
Parte da Anatomia, que tem por objecto a descripção dos sucos lácteos.
\section{Galactologia}
\begin{itemize}
\item {Grp. gram.:f.}
\end{itemize}
\begin{itemize}
\item {Proveniência:(Do gr. \textunderscore gala\textunderscore , \textunderscore galaktos\textunderscore  + \textunderscore logos\textunderscore )}
\end{itemize}
Parte da Medicina, que trata dos sucos lácteos.
\section{Galactológico}
\begin{itemize}
\item {Grp. gram.:adj.}
\end{itemize}
Relativo á galactologia.
\section{Galactólogo}
\begin{itemize}
\item {Grp. gram.:m.}
\end{itemize}
Médico, que se dedica ao estudo das enfermidades, relativas aos sucos lácteos.
(Cp. \textunderscore galactologia\textunderscore )
\section{Galactómetro}
\begin{itemize}
\item {Grp. gram.:m.}
\end{itemize}
\begin{itemize}
\item {Proveniência:(Do gr. \textunderscore gala\textunderscore , \textunderscore galaktos\textunderscore  + \textunderscore metron\textunderscore )}
\end{itemize}
Instrumento, com que se avalia a pureza do leite.
\section{Galactóphago}
\begin{itemize}
\item {Grp. gram.:adj.}
\end{itemize}
\begin{itemize}
\item {Proveniência:(Do gr. \textunderscore gala\textunderscore , \textunderscore galaktos\textunderscore  + \textunderscore phagein\textunderscore )}
\end{itemize}
Diz-se de todos os animaes mammíferos ou que se alimentaram de leite, antes de comer.
\section{Galactophorite}
\begin{itemize}
\item {Grp. gram.:f.}
\end{itemize}
\begin{itemize}
\item {Utilização:Med.}
\end{itemize}
Inflammação dos conductos galactóphoros.
\section{Galactóphoro}
\begin{itemize}
\item {Grp. gram.:adj.}
\end{itemize}
\begin{itemize}
\item {Proveniência:(Do gr. \textunderscore gala\textunderscore , \textunderscore galaktos\textunderscore  + \textunderscore phoros\textunderscore )}
\end{itemize}
Que produz leite.
\section{Galactophthísica}
\begin{itemize}
\item {Grp. gram.:f.}
\end{itemize}
\begin{itemize}
\item {Proveniência:(Do gr. \textunderscore gala\textunderscore , \textunderscore galaktos\textunderscore  + \textunderscore phthisis\textunderscore )}
\end{itemize}
Consumpção ou tísica das mães ou amas, que despenderam demasiado leite na amamentação de uma criança ou de crianças.
\section{Galactopoése}
\begin{itemize}
\item {Grp. gram.:f.}
\end{itemize}
\begin{itemize}
\item {Proveniência:(Do gr. \textunderscore gala\textunderscore , \textunderscore galaktos\textunderscore  + \textunderscore poiein\textunderscore )}
\end{itemize}
Secreção láctea ou formação do leite.
\section{Galactoposia}
\begin{itemize}
\item {Grp. gram.:f.}
\end{itemize}
\begin{itemize}
\item {Proveniência:(Gr. \textunderscore galaktoposia\textunderscore )}
\end{itemize}
Uso do leite, como bebida habitual; tratamento médico, em que o doente se alimenta só de leite.
\section{Galactorreia}
\begin{itemize}
\item {Grp. gram.:f.}
\end{itemize}
\begin{itemize}
\item {Proveniência:(Do gr. \textunderscore gala\textunderscore , \textunderscore galaktos\textunderscore  + \textunderscore rhein\textunderscore )}
\end{itemize}
Abundante secreção de leite.
\section{Galactorrhéa}
\begin{itemize}
\item {Grp. gram.:f.}
\end{itemize}
\begin{itemize}
\item {Proveniência:(Do gr. \textunderscore gala\textunderscore , \textunderscore galaktos\textunderscore  + \textunderscore rhein\textunderscore )}
\end{itemize}
Abundante secreção de leite.
\section{Galactorrheia}
\begin{itemize}
\item {Grp. gram.:f.}
\end{itemize}
\begin{itemize}
\item {Proveniência:(Do gr. \textunderscore gala\textunderscore , \textunderscore galaktos\textunderscore  + \textunderscore rhein\textunderscore )}
\end{itemize}
Abundante secreção de leite.
\section{Galactoscópio}
\begin{itemize}
\item {Grp. gram.:m.}
\end{itemize}
O mesmo que \textunderscore galactómetro\textunderscore .
\section{Galactose}
\begin{itemize}
\item {Grp. gram.:f.}
\end{itemize}
\begin{itemize}
\item {Proveniência:(Gr. \textunderscore galaktosis\textunderscore )}
\end{itemize}
Acção vital, que faz mudar em leite o sangue, o chylo e a lympha.
\section{Galactotísica}
\begin{itemize}
\item {Grp. gram.:f.}
\end{itemize}
\begin{itemize}
\item {Proveniência:(Do gr. \textunderscore gala\textunderscore , \textunderscore galaktos\textunderscore  + \textunderscore phthisis\textunderscore )}
\end{itemize}
Consumpção ou tísica das mães ou amas, que despenderam demasiado leite na amamentação de uma criança ou de crianças.
\section{Galacturia}
\begin{itemize}
\item {Grp. gram.:f.}
\end{itemize}
O mesmo que \textunderscore chyluria\textunderscore .
\section{Gala-gala}
\begin{itemize}
\item {Grp. gram.:m.}
\end{itemize}
Espécie de betume, sôbre que assenta o ferro e o cobre no fundo das embarcações.
(Do mal.)
\section{Galaio}
\begin{itemize}
\item {Grp. gram.:m.}
\end{itemize}
\begin{itemize}
\item {Utilização:Prov.}
\end{itemize}
\begin{itemize}
\item {Utilização:alent.}
\end{itemize}
Oiteirinho.
Espinhaço de um monte.
\section{Galalau}
\begin{itemize}
\item {Grp. gram.:m.}
\end{itemize}
\begin{itemize}
\item {Utilização:Bras}
\end{itemize}
Homem de elevada estatura.
\section{Galamatias}
\begin{itemize}
\item {Grp. gram.:m.}
\end{itemize}
O mesmo que \textunderscore galimatias\textunderscore . Cf. Cand. Lusitano, \textunderscore Diccion. Poét.\textunderscore , (no discurso prelim., § IV).
\section{Galan}
\begin{itemize}
\item {Grp. gram.:m.}
\end{itemize}
\begin{itemize}
\item {Utilização:Fig.}
\end{itemize}
\begin{itemize}
\item {Proveniência:(Do germ. \textunderscore gal\textunderscore )}
\end{itemize}
Actor, que numa peça representa o principal papel de namorado.
Namorado; galanteador.
\section{Galana}
\begin{itemize}
\item {Grp. gram.:f.}
\end{itemize}
\begin{itemize}
\item {Utilização:Des.}
\end{itemize}
\begin{itemize}
\item {Proveniência:(T. da Índia portuguesa)}
\end{itemize}
Briga.
\section{Galane}
\begin{itemize}
\item {Grp. gram.:adj.}
\end{itemize}
Galante, cortês:«\textunderscore entre incivis, pouco galanes Gallos\textunderscore ». Filinto, XIII, 187.
(Extravagância filintiana. Cp. \textunderscore galan\textunderscore  e \textunderscore galante\textunderscore )
\section{Galanear}
\begin{itemize}
\item {Grp. gram.:v. i.}
\end{itemize}
Trajar com garridice.
Trajar fidalgamente.
(Cp. \textunderscore engalanar\textunderscore )
\section{Galanga}
\begin{itemize}
\item {Grp. gram.:f.}
\end{itemize}
\begin{itemize}
\item {Proveniência:(Do malab. \textunderscore kelengu\textunderscore , ou do ár. \textunderscore chalan\textunderscore , que é de or. persa)}
\end{itemize}
Planta amómea, (\textunderscore alpina galanga\textunderscore ).
\section{Galangômbia}
\begin{itemize}
\item {Grp. gram.:f.}
\end{itemize}
Pássaro dentirostro de Benguela, (\textunderscore dryoscopus guttatus\textunderscore ).
\section{Galangundo}
\begin{itemize}
\item {Grp. gram.:m.}
\end{itemize}
Ave pernalta de Benguela, (\textunderscore ardea cinerea\textunderscore ).
\section{Galanice}
\begin{itemize}
\item {Grp. gram.:f.}
\end{itemize}
Qualidade de galan; donaire.
Gentileza; galantaria.
\section{Galantaria}
\begin{itemize}
\item {Grp. gram.:f.}
\end{itemize}
\begin{itemize}
\item {Proveniência:(De \textunderscore galante\textunderscore )}
\end{itemize}
O mesmo que \textunderscore galanice\textunderscore .
Arte de galantear.
Graça; delicadeza.
Coisa ou pessôa galante.
\section{Galante}
\begin{itemize}
\item {Grp. gram.:adj.}
\end{itemize}
\begin{itemize}
\item {Grp. gram.:M.}
\end{itemize}
\begin{itemize}
\item {Proveniência:(Do rad. de \textunderscore galan\textunderscore , ou do fr. \textunderscore galant\textunderscore )}
\end{itemize}
Gracioso; esbelto.
Donairoso.
Espirituoso.
Distinto.
Bello.
Engraçado.
Homem galante.
\section{Galanteador}
\begin{itemize}
\item {Grp. gram.:m.  e  adj.}
\end{itemize}
O que galanteia.
\section{Galantear}
\begin{itemize}
\item {Grp. gram.:v. t.}
\end{itemize}
\begin{itemize}
\item {Grp. gram.:V. i.}
\end{itemize}
\begin{itemize}
\item {Proveniência:(De \textunderscore galante\textunderscore )}
\end{itemize}
Cortejar.
Tratar com amabilidade (as damas).
Enfeitar.
Dizer galanteios; namorar.
\section{Galanteio}
\begin{itemize}
\item {Grp. gram.:m.}
\end{itemize}
Acto de galantear.
Conversa amorosa.
\section{Galantemente}
\begin{itemize}
\item {Grp. gram.:adv.}
\end{itemize}
De modo galante.
\section{Galantho}
\begin{itemize}
\item {Grp. gram.:m.}
\end{itemize}
\begin{itemize}
\item {Proveniência:(Do gr. \textunderscore gala\textunderscore  + \textunderscore anthos\textunderscore )}
\end{itemize}
Gênero de plantas amaryllídeas.
\section{Galantina}
\begin{itemize}
\item {Grp. gram.:f.}
\end{itemize}
\begin{itemize}
\item {Proveniência:(Do b. lat. \textunderscore galatina\textunderscore )}
\end{itemize}
Iguaria, composta de carnes desossadas e cobertas com geleia.
\section{Galanto}
\begin{itemize}
\item {Grp. gram.:m.}
\end{itemize}
\begin{itemize}
\item {Proveniência:(Do gr. \textunderscore gala\textunderscore  + \textunderscore anthos\textunderscore )}
\end{itemize}
Gênero de plantas amarilídeas.
\section{Galão}
\begin{itemize}
\item {Grp. gram.:m.}
\end{itemize}
\begin{itemize}
\item {Proveniência:(Do germ. \textunderscore gal\textunderscore )}
\end{itemize}
Tira entrançada de prata, oiro, linho, etc., para debruar ou enfeitar.
Tira de linho, com que se fortificam as fendas calafetadas da embarcação.
Corcovo ou salto do cavallo, arqueando êste o dorso.
Tira de prata doirada, tecida com retrós, e que, no boné ou nas mangas da farda, serve para distinguir certas categorias de militares e funccionários.
\section{Galápago}
\begin{itemize}
\item {Grp. gram.:m.}
\end{itemize}
Ulcera na corôa do casco das cavalgaduras.
(Cast. \textunderscore galápago\textunderscore )
\section{Galapo}
\begin{itemize}
\item {Grp. gram.:m.}
\end{itemize}
\begin{itemize}
\item {Grp. gram.:Pl.}
\end{itemize}
\begin{itemize}
\item {Utilização:Prov.}
\end{itemize}
\begin{itemize}
\item {Utilização:trasm.}
\end{itemize}
\begin{itemize}
\item {Utilização:Prov.}
\end{itemize}
\begin{itemize}
\item {Utilização:alent.}
\end{itemize}
Coxim na sella do cavallo.
Ligadura para feridas.
Dedos, na acção de agarrar.
Espécie de dedeiras, com que os ceifeiros resguardam, dos golpes da foice, os dedos médio, anular e mínimo da mão esquerda.
Conjunto das dedeiras, da calleira ou coiro, que reveste a palma da mão, e da correia que prende a calleira ás dedeiras. Cp. \textunderscore calleira\textunderscore , nos \textunderscore Additamentos\textunderscore .
(Cast. \textunderscore galapo\textunderscore )
\section{Galardão}
\begin{itemize}
\item {Grp. gram.:m.}
\end{itemize}
\begin{itemize}
\item {Proveniência:(Do ant. alt. al. \textunderscore wilardon\textunderscore )}
\end{itemize}
Recompensa de serviços importantes; prêmio; glória.
\section{Galardoador}
\begin{itemize}
\item {Grp. gram.:m.  e  adj.}
\end{itemize}
O que galardôa.
\section{Galardoar}
\begin{itemize}
\item {Grp. gram.:v. i.}
\end{itemize}
Dar galardão a.
Premiar; compensar.
Consolar.
\section{Galarim}
\begin{itemize}
\item {Grp. gram.:m.}
\end{itemize}
Cúmulo.
O ponto mais elevado.
Fastígio; opulência.
Valimento.
(Cast. \textunderscore galarin\textunderscore )
\section{Gálatas}
\begin{itemize}
\item {Grp. gram.:m. pl.}
\end{itemize}
\begin{itemize}
\item {Proveniência:(Do lat. \textunderscore galatae\textunderscore )}
\end{itemize}
Habitantes da Galácia.
Antigo povo da Ásia-Menór.
\section{Galateia}
\begin{itemize}
\item {Grp. gram.:f.}
\end{itemize}
\begin{itemize}
\item {Utilização:T. de Mourão}
\end{itemize}
Gandaia, vida airada, tuna.
\section{Galáxia}
\begin{itemize}
\item {fónica:csi}
\end{itemize}
\begin{itemize}
\item {Grp. gram.:f.}
\end{itemize}
\begin{itemize}
\item {Proveniência:(Gr. \textunderscore galaxia\textunderscore )}
\end{itemize}
O mesmo que \textunderscore via-láctea\textunderscore .
Gênero de plantas irídeas.
\section{Galazia}
\begin{itemize}
\item {Grp. gram.:f.}
\end{itemize}
O mesmo que \textunderscore galezia\textunderscore . Cf. Gasp. de S. Bern., \textunderscore Itiner.\textunderscore , 117 e 140.
\section{Galazimo}
\begin{itemize}
\item {Grp. gram.:m.}
\end{itemize}
\begin{itemize}
\item {Proveniência:(Do gr. \textunderscore gala\textunderscore  + \textunderscore zume\textunderscore )}
\end{itemize}
Bebida refrigerante, acídula e gasosa, formada de leite fermentado.
\section{Galazymo}
\begin{itemize}
\item {Grp. gram.:m.}
\end{itemize}
\begin{itemize}
\item {Proveniência:(Do gr. \textunderscore gala\textunderscore  + \textunderscore zume\textunderscore )}
\end{itemize}
Bebida refrigerante, acídula e gasosa, formada de leite fermentado.
\section{Galbano}
\begin{itemize}
\item {Grp. gram.:m.}
\end{itemize}
\begin{itemize}
\item {Proveniência:(Lat. \textunderscore galbanum\textunderscore )}
\end{itemize}
Planta umbellífera, sempre verde.
Substância resinosa, que se extrái da mesma planta.
\section{Galcónia}
\begin{itemize}
\item {Grp. gram.:f.}
\end{itemize}
Planta aquática, de fôlhas delgadas e de flôres encarnadas.
\section{Galdério}
\begin{itemize}
\item {Grp. gram.:m.  e  adj.}
\end{itemize}
\begin{itemize}
\item {Utilização:Pop.}
\end{itemize}
Vadio.
Intrujão.
Gastador.
\section{Galdinas}
\begin{itemize}
\item {Grp. gram.:f. Pl.}
\end{itemize}
\begin{itemize}
\item {Utilização:Gír.}
\end{itemize}
O mesmo que \textunderscore calças\textunderscore .
\section{Galdir}
\begin{itemize}
\item {Grp. gram.:v. t.}
\end{itemize}
\begin{itemize}
\item {Utilização:Prov.}
\end{itemize}
O mesmo que \textunderscore gualdir\textunderscore .
\section{Galdrama}
\begin{itemize}
\item {Grp. gram.:f.}
\end{itemize}
\begin{itemize}
\item {Utilização:Des.}
\end{itemize}
Rameira, rascôa.
\section{Galdrapa}
\begin{itemize}
\item {Grp. gram.:f.}
\end{itemize}
\begin{itemize}
\item {Utilização:Prov.}
\end{itemize}
\begin{itemize}
\item {Utilização:trasm.}
\end{itemize}
Porca muito magra, de barriga pendente em pelhancas.
Mulher magrizela e alta.
(Cp. \textunderscore gualdrapa\textunderscore )
\section{Galdrapinha}
\begin{itemize}
\item {Grp. gram.:f.}
\end{itemize}
\begin{itemize}
\item {Utilização:Gír.}
\end{itemize}
\begin{itemize}
\item {Proveniência:(De \textunderscore galdrapa\textunderscore )}
\end{itemize}
Meretriz muito reles.
\section{Galdripeira}
\begin{itemize}
\item {Grp. gram.:f.}
\end{itemize}
\begin{itemize}
\item {Utilização:Prov.}
\end{itemize}
\begin{itemize}
\item {Utilização:trasm.}
\end{itemize}
Mulher suja e rota.
\section{Galdripo}
\begin{itemize}
\item {Grp. gram.:m.}
\end{itemize}
\begin{itemize}
\item {Utilização:Prov.}
\end{itemize}
Cacho esgaldripado.
\section{Galdrocha}
\begin{itemize}
\item {Grp. gram.:f.}
\end{itemize}
\begin{itemize}
\item {Utilização:Prov.}
\end{itemize}
\begin{itemize}
\item {Utilização:trasm.}
\end{itemize}
O mesmo que \textunderscore pelhancas\textunderscore .
(Cp. \textunderscore galdrapa\textunderscore )
\section{Galdropar}
\begin{itemize}
\item {Grp. gram.:v. i.}
\end{itemize}
\begin{itemize}
\item {Utilização:Gír.}
\end{itemize}
Us. na loc. \textunderscore galdropar da corda\textunderscore , comer da ceia de outrem.
(Cp. \textunderscore galdrope\textunderscore )
\section{Galdrope}
\begin{itemize}
\item {Grp. gram.:m.}
\end{itemize}
\begin{itemize}
\item {Utilização:Náut.}
\end{itemize}
Cabo, com que se puxa a picota da bomba, a bordo, ou se auxilia o govêrno do leme.
\section{Galé}
\begin{itemize}
\item {Grp. gram.:f.}
\end{itemize}
\begin{itemize}
\item {Grp. gram.:M.}
\end{itemize}
\begin{itemize}
\item {Grp. gram.:F. pl.}
\end{itemize}
\begin{itemize}
\item {Utilização:Ext.}
\end{itemize}
\begin{itemize}
\item {Proveniência:(Do gr. \textunderscore gaulis\textunderscore ?)}
\end{itemize}
Antiga embarcação de vela e remos.
Indíviduo condemnado ás galés; grilheta.
Pena dos condemnados a remar em galés.
Trabalhos públicos.
\section{Galé}
\begin{itemize}
\item {Grp. gram.:f.}
\end{itemize}
\begin{itemize}
\item {Utilização:Typ.}
\end{itemize}
\begin{itemize}
\item {Proveniência:(Fr. \textunderscore galée\textunderscore )}
\end{itemize}
Peça quadrangular de madeira ou de ferro, com bordas em três lados contíguos, na qual se assenta a composição typográphica de uma fôlha que se vai imprimir.
\section{Gálea}
\begin{itemize}
\item {Grp. gram.:f.}
\end{itemize}
\begin{itemize}
\item {Proveniência:(Lat. \textunderscore galea\textunderscore )}
\end{itemize}
Capacete de guerreiro; elmo.
\section{Galeaça}
\begin{itemize}
\item {Grp. gram.:f.}
\end{itemize}
\begin{itemize}
\item {Utilização:Ant.}
\end{itemize}
Grande galé de tres mastros. Cf. Pant. de Aveiro, \textunderscore Itiner.\textunderscore , 3. (2.^a ed.).
\section{Galeado}
\begin{itemize}
\item {Grp. gram.:adj.}
\end{itemize}
\begin{itemize}
\item {Proveniência:(Lat. \textunderscore galeatus\textunderscore )}
\end{itemize}
Que tem gálea, coberto com gálea.
\section{Galeanthropia}
\begin{itemize}
\item {Grp. gram.:f.}
\end{itemize}
\begin{itemize}
\item {Proveniência:(Do gr. \textunderscore gale\textunderscore  + \textunderscore anthropos\textunderscore )}
\end{itemize}
Mania, em que o doente se imagina transformado em gato.
\section{Galeantropia}
\begin{itemize}
\item {Grp. gram.:f.}
\end{itemize}
\begin{itemize}
\item {Proveniência:(Do gr. \textunderscore gale\textunderscore  + \textunderscore anthropos\textunderscore )}
\end{itemize}
Mania, em que o doente se imagina transformado em gato.
\section{Galeão}
\begin{itemize}
\item {Grp. gram.:m.}
\end{itemize}
\begin{itemize}
\item {Proveniência:(De \textunderscore galé\textunderscore ^1)}
\end{itemize}
Antigo navio de alto bordo.
Nau de guerra.
Apparelho de pesca de cêrco, que se emprega junto ás costas marítimas.
Embarcação de vela latina, que acompanha aquelle apparelho.
\section{Galeão}
\begin{itemize}
\item {Grp. gram.:m.}
\end{itemize}
\begin{itemize}
\item {Utilização:Typ.}
\end{itemize}
\begin{itemize}
\item {Proveniência:(De \textunderscore galé\textunderscore ^2)}
\end{itemize}
Peça plana e retangular de madeira, com rebordo em um dos lados, na qual o compositor typográphico colloca as linhas que formou no componedor.
\section{Galear}
\begin{itemize}
\item {Grp. gram.:v. i.}
\end{itemize}
Ostentar galas.
Trajar luxuosamente.
\section{Galear}
\begin{itemize}
\item {Grp. gram.:v. t.}
\end{itemize}
\begin{itemize}
\item {Grp. gram.:V. i.}
\end{itemize}
\begin{itemize}
\item {Proveniência:(De \textunderscore galé\textunderscore ^1)}
\end{itemize}
Baloiçar, atirando.
Arremessar.
Baloiçar-se.
\section{Galeato}
\begin{itemize}
\item {Grp. gram.:adj.}
\end{itemize}
\begin{itemize}
\item {Utilização:Fig.}
\end{itemize}
\begin{itemize}
\item {Proveniência:(Lat. \textunderscore galeatus\textunderscore )}
\end{itemize}
Que tem capacete de coiro; galeado.
Defensivo.
\section{Galega}
\begin{itemize}
\item {fónica:lê}
\end{itemize}
\begin{itemize}
\item {Grp. gram.:f.}
\end{itemize}
\begin{itemize}
\item {Proveniência:(Do gr. \textunderscore gala\textunderscore )}
\end{itemize}
Gênero de plantas leguminosas.
\section{Galeiforme}
\begin{itemize}
\item {Grp. gram.:adj.}
\end{itemize}
\begin{itemize}
\item {Proveniência:(De \textunderscore gálea\textunderscore  + \textunderscore fórma\textunderscore )}
\end{itemize}
Que tem fórma de gálea.
\section{Galeira}
\begin{itemize}
\item {Grp. gram.:f.}
\end{itemize}
\begin{itemize}
\item {Utilização:Prov.}
\end{itemize}
\begin{itemize}
\item {Utilização:dur.}
\end{itemize}
Rêgo transversal nas vinhas, para esgôto de águas no inverno.
Quédas de água, na corrente do Doiro, que se formam no estio.
\section{Galela}
\begin{itemize}
\item {Grp. gram.:f.}
\end{itemize}
\begin{itemize}
\item {Utilização:Prov.}
\end{itemize}
\begin{itemize}
\item {Utilização:trasm.}
\end{itemize}
O mesmo que \textunderscore galelo\textunderscore .
\section{Galelo}
\begin{itemize}
\item {Grp. gram.:m.}
\end{itemize}
\begin{itemize}
\item {Utilização:Prov.}
\end{itemize}
\begin{itemize}
\item {Utilização:trasm.}
\end{itemize}
Gomo de laranja.
Cacho de uvas, que fica na vinha, depois de vindimada. (Colhido em V. P. de Aguiar)
\section{Galém}
\begin{itemize}
\item {Grp. gram.:m.}
\end{itemize}
Antigo e pequeno pêso das ilhas de Maldiva.
\section{Galena}
\begin{itemize}
\item {Grp. gram.:f.}
\end{itemize}
\begin{itemize}
\item {Proveniência:(Lat. \textunderscore galena\textunderscore )}
\end{itemize}
Metal, que é o minério de chumbo mais commum, e cujas variedades conservam quási todas sulfureto de prata.
\section{Galengue}
\begin{itemize}
\item {Grp. gram.:m.}
\end{itemize}
Ruminante de Angola.
\section{Galênia}
\begin{itemize}
\item {Grp. gram.:f.}
\end{itemize}
\begin{itemize}
\item {Proveniência:(De \textunderscore Galeno\textunderscore , n. p.)}
\end{itemize}
Gênero de plantas portuláceas.
\section{Galênico}
\begin{itemize}
\item {Grp. gram.:adj.}
\end{itemize}
\begin{itemize}
\item {Grp. gram.:Pl.}
\end{itemize}
Relativo ao systema médico de Galeno.
\textunderscore Remédios galênicos\textunderscore , medicamentos vegetaes, por opposição aos chímicos ou espagíricos.
\section{Galenismo}
\begin{itemize}
\item {Grp. gram.:m.}
\end{itemize}
\begin{itemize}
\item {Proveniência:(De \textunderscore Galeno\textunderscore , n. p.)}
\end{itemize}
Systema médico de Galeno, que consistia principalmente em subordinar os phenómenos da saúde e da doença á acção de quatro humores, o sangue, a bílis, a fleugma e a atrabílis.
\section{Galenista}
\begin{itemize}
\item {Grp. gram.:m.}
\end{itemize}
Partidário do systema de Galeno.
\section{Galenite}
\begin{itemize}
\item {Grp. gram.:f.}
\end{itemize}
\begin{itemize}
\item {Proveniência:(De \textunderscore galena\textunderscore )}
\end{itemize}
Nome, que os geólogos modernos dão á galena.
\section{Galeno}
\begin{itemize}
\item {Grp. gram.:m.}
\end{itemize}
\begin{itemize}
\item {Utilização:Fam.}
\end{itemize}
\begin{itemize}
\item {Proveniência:(De \textunderscore Galeno\textunderscore , n. p.)}
\end{itemize}
Qualquer médico. Cf. Garrett, \textunderscore D. Branca\textunderscore , 44.
\section{Galeola}
\begin{itemize}
\item {Grp. gram.:f.}
\end{itemize}
\begin{itemize}
\item {Utilização:Ant.}
\end{itemize}
\begin{itemize}
\item {Proveniência:(Lat. \textunderscore galeola\textunderscore )}
\end{itemize}
Vaso em fórma de capacete.
\section{Galeolária}
\begin{itemize}
\item {Grp. gram.:f.}
\end{itemize}
\begin{itemize}
\item {Proveniência:(Do lat. \textunderscore galeola\textunderscore )}
\end{itemize}
Zoóphito alaranjado, que se ramifica, formando os seus filamentos uma espécie de renda.
\section{Galeonete}
\begin{itemize}
\item {fónica:nê}
\end{itemize}
\begin{itemize}
\item {Grp. gram.:m.}
\end{itemize}
\begin{itemize}
\item {Proveniência:(De \textunderscore galeão\textunderscore )}
\end{itemize}
Pequena embarcação em fórma de galeão e que acompanha êste na pesca.
\section{Galeota}
\begin{itemize}
\item {Grp. gram.:f.}
\end{itemize}
Pequena galé.
\textunderscore Prego de galeota\textunderscore , o mesmo que \textunderscore prego caibral\textunderscore .
\textunderscore Meia galeota\textunderscore , prego, próprio para pregar solho ou sobrado, e que é o meio termo entre o caibral e o ripal.
\section{Galeote}
\begin{itemize}
\item {Grp. gram.:m.}
\end{itemize}
Galeota.
Remador de galé.
Condemnado ás galés.
(Cast. \textunderscore galeote\textunderscore )
\section{Galeote}
\begin{itemize}
\item {Grp. gram.:m.}
\end{itemize}
Espécie de capa antiga:«\textunderscore ...vestindo um galeote por causa do frio.\textunderscore »R. Lobo, \textunderscore Côrte na Ald.\textunderscore , II, 80.
\section{Galera}
\begin{itemize}
\item {Grp. gram.:f.}
\end{itemize}
Antiga embarcação, comprida e estreita, de vela e remos, com dois ou três mastros.
Designação genêrica das embarcações, que têm três mastros armados á redonda.
Carroça para transportes de bombeiros, em serviço de incêndios.
Carroça de quatro rodas, para transporte de mobília e fardos muito pesados.
Fôrno para fundição.
(B. lat. \textunderscore galera\textunderscore )
\section{Galeria}
\begin{itemize}
\item {Grp. gram.:f.}
\end{itemize}
\begin{itemize}
\item {Utilização:Fig.}
\end{itemize}
Corredor extenso, em que se guardam, dispostos artisticamente, quadros, estátuas, etc.
Espécie de varanda, em theatros ou outros edifícios, que deita para um recinto espaçoso e é destinada ao público.
Pessôas, que se reúnem em galerias desta ordem.
Corredor subterrâneo.
Varanda na parte posterior dos navios.
Mó.
Collecção de quadros, estatuas, etc., organizada artisticamente.
Collecção de estudos biográphicos ou descriptivos.
(B. lat. \textunderscore galeria\textunderscore )
\section{Galeriano}
\begin{itemize}
\item {Grp. gram.:m.  e  adj.}
\end{itemize}
\begin{itemize}
\item {Utilização:Ant.}
\end{itemize}
\begin{itemize}
\item {Proveniência:(De \textunderscore galera\textunderscore )}
\end{itemize}
Remador de galés, como condemnado ou como cativo.
\section{Galerno}
\begin{itemize}
\item {Grp. gram.:adj.}
\end{itemize}
\begin{itemize}
\item {Grp. gram.:M.}
\end{itemize}
\begin{itemize}
\item {Proveniência:(Do bret. \textunderscore gwalarn\textunderscore )}
\end{itemize}
Brando, suave, (falando-se de um vento que sopra de Noroéste).
Vento brando, aprazível.
\section{Galero}
\begin{itemize}
\item {Grp. gram.:m.}
\end{itemize}
\begin{itemize}
\item {Utilização:Ant.}
\end{itemize}
\begin{itemize}
\item {Proveniência:(Lat. \textunderscore galerus\textunderscore )}
\end{itemize}
O mesmo que \textunderscore gálea\textunderscore .
Barrete de pelles.
Chapéu, que, entre os Romanos, só podia sêr usado pelos flâmines de Júpiter.
\section{Galérucas}
\begin{itemize}
\item {Grp. gram.:f. pl.}
\end{itemize}
Insectos herbívoros, da ordem dos coleópteros.
(Relaciona-se com \textunderscore galero\textunderscore ?)
\section{Galezia}
\begin{itemize}
\item {Grp. gram.:f.}
\end{itemize}
\begin{itemize}
\item {Utilização:Fam.}
\end{itemize}
\begin{itemize}
\item {Proveniência:(Do rad. de \textunderscore galé\textunderscore ^1)}
\end{itemize}
Velhacaria.
Trapaça; maroteira.
\section{Galfarro}
\begin{itemize}
\item {Grp. gram.:m.}
\end{itemize}
\begin{itemize}
\item {Utilização:Pop.}
\end{itemize}
\begin{itemize}
\item {Utilização:Fig.}
\end{itemize}
\begin{itemize}
\item {Utilização:Gír.}
\end{itemize}
\begin{itemize}
\item {Utilização:ant.}
\end{itemize}
\begin{itemize}
\item {Utilização:Chul.}
\end{itemize}
Beleguim.
Meirinho.
Official de diligências.
Onzeneiro.
Comilão.
Vádio.
Piolho grande.
(Cast. \textunderscore galfarro\textunderscore )
\section{Galga}
\begin{itemize}
\item {Grp. gram.:f.}
\end{itemize}
\begin{itemize}
\item {Utilização:Pop.}
\end{itemize}
\begin{itemize}
\item {Utilização:Gír.}
\end{itemize}
A fêmea do galgo.
Boato falso, pêta.
Fome.
Animal amphíbio da América, (\textunderscore canis gallica\textunderscore ).
\section{Galga}
\begin{itemize}
\item {Grp. gram.:f.}
\end{itemize}
\begin{itemize}
\item {Utilização:Prov.}
\end{itemize}
\begin{itemize}
\item {Utilização:trasm.}
\end{itemize}
Ancoreta.
Mó, de eixo horizontal, nos lagares de azeite.
Pedra grande a rebolar por uma ladeira abaixo.
Haste de madeira, segura por fóra da extremidade do eixo dos carros alentejanos, para resguardo e para travar.
(Cp. cast. \textunderscore galga\textunderscore )
\section{Galgação}
\begin{itemize}
\item {Grp. gram.:f.}
\end{itemize}
\begin{itemize}
\item {Utilização:Carp.}
\end{itemize}
\begin{itemize}
\item {Proveniência:(De \textunderscore galgar\textunderscore )}
\end{itemize}
Acto de endireitar a superfície de uma tábua ou madeira com plaina ou garlopa.
\section{Galgadeira}
\begin{itemize}
\item {Grp. gram.:f.}
\end{itemize}
\begin{itemize}
\item {Utilização:Prov.}
\end{itemize}
\begin{itemize}
\item {Utilização:dur.}
\end{itemize}
\begin{itemize}
\item {Proveniência:(De \textunderscore galgar\textunderscore )}
\end{itemize}
Instrumento de carpinteiro, com que, nos lados das tábuas, se traçam riscos parallelos á aresta das mesmas tábuas.
O mesmo que \textunderscore graminho\textunderscore ?
\section{Galgão}
\begin{itemize}
\item {Grp. gram.:m.}
\end{itemize}
\begin{itemize}
\item {Utilização:Prov.}
\end{itemize}
\begin{itemize}
\item {Utilização:alent.}
\end{itemize}
\begin{itemize}
\item {Proveniência:(De \textunderscore galgar\textunderscore )}
\end{itemize}
Salto, pulo.
\section{Galgar}
\begin{itemize}
\item {Grp. gram.:v. t.}
\end{itemize}
\begin{itemize}
\item {Utilização:Fig.}
\end{itemize}
\begin{itemize}
\item {Utilização:Prov.}
\end{itemize}
\begin{itemize}
\item {Utilização:beir.}
\end{itemize}
\begin{itemize}
\item {Grp. gram.:V. i.}
\end{itemize}
\begin{itemize}
\item {Utilização:Fig.}
\end{itemize}
\begin{itemize}
\item {Proveniência:(De \textunderscore galgo\textunderscore )}
\end{itemize}
Transpor, alargando as pernas.
Transpor, saltar por cima de: \textunderscore galgar um muro\textunderscore .
Percorrer: \textunderscore galgar léguas de estrada\textunderscore .
Alinhar.
Calcular distâncias a compasso em (trabalhos de latoaria).
Riscar com a galgadeira.
Pular; trepar.
Elevar-se rapidamente.
Recalcitrar, grimpar. Cf. Castilho, \textunderscore Sabichonas\textunderscore , 220.
\section{Galgaz}
\begin{itemize}
\item {Grp. gram.:adj.}
\end{itemize}
Semelhante a galgo.
Esguio; magro.
\section{Galgo}
\begin{itemize}
\item {Grp. gram.:m.}
\end{itemize}
\begin{itemize}
\item {Proveniência:(Do lat. \textunderscore gallicus\textunderscore )}
\end{itemize}
Cão pernalto e esguio, muito empregado em caça de lebres.
\section{Galgueira}
\begin{itemize}
\item {Grp. gram.:f.}
\end{itemize}
\begin{itemize}
\item {Proveniência:(De \textunderscore galgueiro\textunderscore )}
\end{itemize}
Cova artificial, para depósito de águas, ou para plantar bacêllo, árvores, etc.
\section{Galgueiro}
\begin{itemize}
\item {Grp. gram.:adj.}
\end{itemize}
\begin{itemize}
\item {Utilização:Prov.}
\end{itemize}
\begin{itemize}
\item {Utilização:trasm.}
\end{itemize}
\begin{itemize}
\item {Proveniência:(De \textunderscore galgar\textunderscore )}
\end{itemize}
Que corre em declívio, (falando-se de um regato ou de água).
\section{Galguenho}
\begin{itemize}
\item {Grp. gram.:m.}
\end{itemize}
\begin{itemize}
\item {Utilização:Cyn.}
\end{itemize}
\begin{itemize}
\item {Proveniência:(De \textunderscore galgo\textunderscore )}
\end{itemize}
Variedade de podengo de caça, vulgar no Alentejo e na Beira-Baixa.
\section{Galha}
\begin{itemize}
\item {Grp. gram.:f.}
\end{itemize}
Nome, que os pescadores dão á primeira barbatana dorsal dos peixes, a qual se avista ás vezes á flôr da água.
\section{Galha}
\begin{itemize}
\item {Grp. gram.:f.}
\end{itemize}
\begin{itemize}
\item {Proveniência:(Do lat. \textunderscore galla\textunderscore )}
\end{itemize}
Fruto globular de árvore glandífera, não contendo semente.
Excrescência de certos frutos, de fórma variada, e produzida pela picada de certos insectos.
\section{Galhada}
\begin{itemize}
\item {Grp. gram.:f.}
\end{itemize}
\begin{itemize}
\item {Utilização:Bras}
\end{itemize}
\begin{itemize}
\item {Utilização:Prov.}
\end{itemize}
\begin{itemize}
\item {Utilização:trasm.}
\end{itemize}
\begin{itemize}
\item {Utilização:Prov.}
\end{itemize}
\begin{itemize}
\item {Utilização:trasm.}
\end{itemize}
\begin{itemize}
\item {Utilização:Prov.}
\end{itemize}
\begin{itemize}
\item {Utilização:dur.}
\end{itemize}
\begin{itemize}
\item {Proveniência:(De \textunderscore galho\textunderscore )}
\end{itemize}
Cornos de ruminantes.
Ramagem de arvoredo.
Ramificação do cacho de uvas; uns poucos de bagos, sustentados nos respectivos pés.
Rasgão no peito, produzido por um galho ou graveto.
Ponto, em que se bifurcam as pernas das calças.
\section{Galhano}
\begin{itemize}
\item {Grp. gram.:adj.}
\end{itemize}
\begin{itemize}
\item {Utilização:ant.}
\end{itemize}
\begin{itemize}
\item {Utilização:Chul.}
\end{itemize}
Esfarrapado; mal entrajado.
(Cp. \textunderscore galhudo\textunderscore )
\section{Galação}
\begin{itemize}
\item {Grp. gram.:f.}
\end{itemize}
\begin{itemize}
\item {Utilização:Bras}
\end{itemize}
O mesmo que \textunderscore galadura\textunderscore .
\section{Galacrista}
\begin{itemize}
\item {Grp. gram.:f.}
\end{itemize}
Planta ornamental, cuja florescência imita a crista do galo.
(Cp. \textunderscore gallicrista\textunderscore )
\section{Galadura}
\begin{itemize}
\item {Grp. gram.:f.}
\end{itemize}
Acto ou efeito de galar.
Ponto branco que, na gema do ovo, indica a fecundação.
\section{Galão}
\begin{itemize}
\item {Grp. gram.:m.}
\end{itemize}
\begin{itemize}
\item {Proveniência:(Ingl. \textunderscore gallon\textunderscore )}
\end{itemize}
Antiga medida inglesa para líquidos, que se usou também em Portugal, e ainda se usa no fornecimento dos vernizes para os caminhos de ferro.--Cada galão corresponde a quatro litros e meio.
\section{Galar}
\begin{itemize}
\item {Grp. gram.:v. t.}
\end{itemize}
\begin{itemize}
\item {Utilização:Pop.}
\end{itemize}
\begin{itemize}
\item {Grp. gram.:V. p.}
\end{itemize}
\begin{itemize}
\item {Proveniência:(De \textunderscore galo\textunderscore )}
\end{itemize}
Fecundar, (falando-se das galináceas).
Fornicar.
Colocar-se:«\textunderscore galou-se de repente sobre mim\textunderscore ». F. Manuel, \textunderscore Apólogos\textunderscore .
\section{Galaripo}
\begin{itemize}
\item {Grp. gram.:m.}
\end{itemize}
\begin{itemize}
\item {Utilização:Prov.}
\end{itemize}
\begin{itemize}
\item {Utilização:beir.}
\end{itemize}
\begin{itemize}
\item {Utilização:Prov.}
\end{itemize}
\begin{itemize}
\item {Utilização:minh.}
\end{itemize}
Rapaz, que já pretende namorar.
O mesmo que \textunderscore gallo\textunderscore ^1, elevação na testa ou na cabeça, por efeito de pancada.
(Cp. \textunderscore gallo\textunderscore ^1)
\section{Galarispo}
\begin{itemize}
\item {Grp. gram.:m.}
\end{itemize}
\begin{itemize}
\item {Utilização:Prov.}
\end{itemize}
\begin{itemize}
\item {Utilização:trasm.}
\end{itemize}
O mesmo que \textunderscore galaripo\textunderscore .
\section{Galas}
\begin{itemize}
\item {Grp. gram.:m. pl.}
\end{itemize}
Povos fronteiriços da Abissínia.
\section{Galato}
\begin{itemize}
\item {Grp. gram.:m.}
\end{itemize}
\begin{itemize}
\item {Proveniência:(Do lat. \textunderscore galla\textunderscore )}
\end{itemize}
Combinação do ácido gálico com uma base.
\section{Galear}
\begin{itemize}
\item {Grp. gram.:v. i.}
\end{itemize}
\begin{itemize}
\item {Proveniência:(De \textunderscore galo\textunderscore )}
\end{itemize}
Diz-se do toireiro se, de capote ao ombro e de costas voltadas para o toiro, abre os braços, quando o toiro arranca, e os vai movendo com o corpo, da direita para a esquerda.
\section{Galeciano}
\begin{itemize}
\item {Grp. gram.:adj.}
\end{itemize}
\begin{itemize}
\item {Proveniência:(Do lat. \textunderscore Gallaecia\textunderscore , n. p.)}
\end{itemize}
O mesmo que \textunderscore galiziano\textunderscore .
\section{Galega}
\begin{itemize}
\item {fónica:lê}
\end{itemize}
\begin{itemize}
\item {Grp. gram.:adj. f.}
\end{itemize}
\begin{itemize}
\item {Utilização:Ant.}
\end{itemize}
\begin{itemize}
\item {Grp. gram.:F.}
\end{itemize}
Dizia-se da terra, que não é fértil ou que é charneca.
Espécie de ginja vermelha, muito ácida.
Espécie de couve.
Casta de uva branca de Ourém.
Variedade de azeitona.
\section{Galegada}
\begin{itemize}
\item {Grp. gram.:f.}
\end{itemize}
\begin{itemize}
\item {Utilização:Pop.}
\end{itemize}
\begin{itemize}
\item {Utilização:Fam.}
\end{itemize}
Ajuntamento de galegos.
Acção própria de galego.
Brutalidade.
Acção ou dito grosseiro.
\section{Galegaria}
\begin{itemize}
\item {Grp. gram.:f.}
\end{itemize}
Porção de galegos.
Falario de galegos. Cf. Garrett, \textunderscore Alfageme\textunderscore .
\section{Galego}
\begin{itemize}
\item {fónica:lê}
\end{itemize}
\begin{itemize}
\item {Grp. gram.:m.}
\end{itemize}
\begin{itemize}
\item {Utilização:Pop.}
\end{itemize}
\begin{itemize}
\item {Grp. gram.:Adj.}
\end{itemize}
\begin{itemize}
\item {Utilização:Prov.}
\end{itemize}
\begin{itemize}
\item {Utilização:Prov.}
\end{itemize}
\begin{itemize}
\item {Utilização:beir.}
\end{itemize}
\begin{itemize}
\item {Utilização:Ant.}
\end{itemize}
Aquele que é natural da Galiza.
Dialecto da Galiza.
Mariola, moço de fretes.
Casta de uva preta de Colares.
Homem grosseiro, incivil.
Relativo á Galiza.
Ordinário: \textunderscore ginja galega\textunderscore .
Diz-se do vento do norte.
E diz-se também de uma espécie de trigo mole.
Diz-se da mesa em que não há pão.
Dizia-se de uma qualidade de tecido de linho:«\textunderscore Deixo 4 varas de linho galego...\textunderscore »(De um testamento de 1691)
\section{Galeguice}
\begin{itemize}
\item {Grp. gram.:f.}
\end{itemize}
Acto ou modos de galego.
\section{Galeguinho}
\begin{itemize}
\item {Grp. gram.:m.}
\end{itemize}
Variedade de uva de Azeitão.
\section{Galeguismo}
\begin{itemize}
\item {Grp. gram.:m.}
\end{itemize}
\begin{itemize}
\item {Proveniência:(De \textunderscore galego\textunderscore )}
\end{itemize}
Palavra ou locução privativa da Galiza.
\section{Galeio}
\begin{itemize}
\item {Grp. gram.:m.}
\end{itemize}
Acto de galear.
\section{Galeira}
\begin{itemize}
\item {Grp. gram.:f.}
\end{itemize}
\begin{itemize}
\item {Utilização:ant.}
\end{itemize}
\begin{itemize}
\item {Utilização:Fam.}
\end{itemize}
O mesmo que \textunderscore forcado\textunderscore .
\section{Galeirão}
\begin{itemize}
\item {Grp. gram.:m.}
\end{itemize}
\begin{itemize}
\item {Proveniência:(Do cast. \textunderscore gallarón\textunderscore )}
\end{itemize}
Ave pernalta, (\textunderscore fulca atra\textunderscore , Lin.).
O mesmo que \textunderscore abibe\textunderscore .
\section{Galeno}
\begin{itemize}
\item {Grp. gram.:m.}
\end{itemize}
\begin{itemize}
\item {Utilização:Prov.}
\end{itemize}
\begin{itemize}
\item {Proveniência:(De \textunderscore gallo\textunderscore ?)}
\end{itemize}
O mesmo que \textunderscore abibe\textunderscore .
\section{Galês}
\begin{itemize}
\item {Grp. gram.:adj.}
\end{itemize}
\begin{itemize}
\item {Grp. gram.:M.}
\end{itemize}
Relativo ao país de Gales.
Língua antiga do país de Gales.
\section{Galharda}
\begin{itemize}
\item {Grp. gram.:f.}
\end{itemize}
\begin{itemize}
\item {Proveniência:(De \textunderscore galhardo\textunderscore )}
\end{itemize}
Dança e música antigas. Cf. \textunderscore Aulegrafia\textunderscore , 121.
\section{Galhardamente}
\begin{itemize}
\item {Grp. gram.:adv.}
\end{itemize}
De modo galhardo.
Com galhardia.
\section{Galhardear}
\begin{itemize}
\item {Grp. gram.:v. i.}
\end{itemize}
\begin{itemize}
\item {Grp. gram.:V. t.}
\end{itemize}
Mostrar-se galhardo.
Sobresair; brilhar.
Ostentar; pompear.
\section{Galhardete}
\begin{itemize}
\item {fónica:dê}
\end{itemize}
\begin{itemize}
\item {Grp. gram.:m.}
\end{itemize}
\begin{itemize}
\item {Proveniência:(It. \textunderscore gagliardetto\textunderscore )}
\end{itemize}
Bandeirinha farpada, que se põe no alto dos mastros, como adorno ou sinal.
Bandeira, para enfeite de ruas ou edifícios, em occasião de festa.
\section{Galhardia}
\begin{itemize}
\item {Grp. gram.:f.}
\end{itemize}
Qualidade daquelle ou daquillo que é galhardo.
\section{Galhardo}
\begin{itemize}
\item {Grp. gram.:adj.}
\end{itemize}
\begin{itemize}
\item {Grp. gram.:M.}
\end{itemize}
\begin{itemize}
\item {Utilização:Náut.}
\end{itemize}
\begin{itemize}
\item {Utilização:Prov.}
\end{itemize}
\begin{itemize}
\item {Utilização:beir.}
\end{itemize}
\begin{itemize}
\item {Proveniência:(Do fr. \textunderscore gaillard\textunderscore )}
\end{itemize}
Donairoso; elegante.
Folgazão.
Esforçado.
Bizarro, generoso.
Castello de prôa ou de popa.
Antiga arma defensiva.
O diabo.
\section{Galhas}
\begin{itemize}
\item {Grp. gram.:f. pl.}
\end{itemize}
\begin{itemize}
\item {Utilização:Bras. do N}
\end{itemize}
\begin{itemize}
\item {Proveniência:(De \textunderscore galho\textunderscore )}
\end{itemize}
Cornos de ruminantes; galhada.
\section{Galhastro}
\begin{itemize}
\item {Grp. gram.:m.}
\end{itemize}
\begin{itemize}
\item {Utilização:Prov.}
\end{itemize}
\begin{itemize}
\item {Utilização:trasm.}
\end{itemize}
Animal com um só testículo.
\section{Galheira}
\begin{itemize}
\item {Grp. gram.:f.}
\end{itemize}
\begin{itemize}
\item {Proveniência:(De \textunderscore galho\textunderscore )}
\end{itemize}
Processo de poda, usado em alguns pontos da região trasmontana.
\section{Galheiro}
\begin{itemize}
\item {Grp. gram.:m.}
\end{itemize}
\begin{itemize}
\item {Utilização:Prov.}
\end{itemize}
\begin{itemize}
\item {Utilização:Bras}
\end{itemize}
\begin{itemize}
\item {Proveniência:(De \textunderscore galho\textunderscore )}
\end{itemize}
Fogueira de galhos ou de ramos, em a noite de San-João.
O mesmo que \textunderscore loiceiro\textunderscore .
Espécie de veado grande, veado dos campos.
\section{Galheta}
\begin{itemize}
\item {fónica:lhê}
\end{itemize}
\begin{itemize}
\item {Grp. gram.:f.}
\end{itemize}
\begin{itemize}
\item {Utilização:Prov.}
\end{itemize}
Cada um dos dois pequenos vasos de vidro, para azeite e vinagre, no serviço de mesa.
Pequeno vaso, que contém o vinho ou a agua, para o serviço da Missa.
Instrumento de vidro, usado em laboratórios chímicos.
O mesmo que \textunderscore corvo-marinho\textunderscore .
(Cast. \textunderscore galleta\textunderscore )
\section{Galheta}
\begin{itemize}
\item {fónica:lhê}
\end{itemize}
\begin{itemize}
\item {Grp. gram.:f.}
\end{itemize}
\begin{itemize}
\item {Proveniência:(De \textunderscore galho\textunderscore )}
\end{itemize}
Trombeta de guerra, entre os pretos de Lourenço-Marques, feita de chifre de cabrito.
\section{Galheta}
\begin{itemize}
\item {fónica:lhê}
\end{itemize}
\begin{itemize}
\item {Grp. gram.:f.}
\end{itemize}
\begin{itemize}
\item {Utilização:Gír.}
\end{itemize}
\begin{itemize}
\item {Proveniência:(Do fr. \textunderscore galette\textunderscore , bolacha)}
\end{itemize}
Bofetada: \textunderscore olha que apanhas duas galhetas\textunderscore .
\section{Galheteiro}
\begin{itemize}
\item {Grp. gram.:m.}
\end{itemize}
Utensílio de mesa, que contém especialmente as galhetas.
\section{Galheto}
\begin{itemize}
\item {fónica:lhê}
\end{itemize}
\begin{itemize}
\item {Grp. gram.:m.}
\end{itemize}
\begin{itemize}
\item {Utilização:Prov.}
\end{itemize}
O mesmo que \textunderscore galheteiro\textunderscore .
\section{Galhipo}
\begin{itemize}
\item {Grp. gram.:m.}
\end{itemize}
\begin{itemize}
\item {Utilização:T. de Lindoso}
\end{itemize}
\begin{itemize}
\item {Proveniência:(De \textunderscore galho\textunderscore )}
\end{itemize}
Chifre de bode, que contém medulla de sabugo ou trapos chamuscados, e um pedaço de quartzo, para se fazer lume e acender o cigarro.
\section{Galhistro}
\begin{itemize}
\item {Grp. gram.:m.}
\end{itemize}
O mesmo que \textunderscore galhastro\textunderscore .
\section{Galho}
\begin{itemize}
\item {Grp. gram.:m.}
\end{itemize}
\begin{itemize}
\item {Utilização:Prov.}
\end{itemize}
\begin{itemize}
\item {Utilização:dur.}
\end{itemize}
Ramo de árvore.
Parte do ramo, que fica ligada ao tronco, depois de partido o mesmo ramo.
Esgalho.
Cacho, escádea.
Chifre de ruminantes.
Ódio; zanga.
\section{Galhofa}
\begin{itemize}
\item {Grp. gram.:f.}
\end{itemize}
\begin{itemize}
\item {Utilização:T. de Penafiel}
\end{itemize}
Motejo.
Gracejo.
Folia.
Escárneo.
Espécie de bolo de farinha e ovos.
Peixe de Portugal, do gênero múgil.
(Cast. \textunderscore gallofa\textunderscore )
\section{Galhofada}
\begin{itemize}
\item {Grp. gram.:f.}
\end{itemize}
Grande galhofa.
\section{Galhofar}
\begin{itemize}
\item {Grp. gram.:v. i.}
\end{itemize}
Fazer galhofa; divertir-se ruidosamente.
Zombar. Cf. Camillo, \textunderscore Cancion. Al.\textunderscore , 261.
\section{Galhofaria}
\begin{itemize}
\item {Grp. gram.:f.}
\end{itemize}
\begin{itemize}
\item {Proveniência:(De \textunderscore galhofar\textunderscore )}
\end{itemize}
O mesmo que \textunderscore galhofada\textunderscore ; folguedo; festim.
\section{Galhofear}
\begin{itemize}
\item {Grp. gram.:v. i.}
\end{itemize}
O mesmo que \textunderscore galhofar\textunderscore .
\section{Galhofeiro}
\begin{itemize}
\item {Grp. gram.:m.  e  adj.}
\end{itemize}
O que faz galhofa; zombeteiro; brincalhão.
\section{Galhofento}
\begin{itemize}
\item {Grp. gram.:adj.}
\end{itemize}
O mesmo que \textunderscore galhofeiro\textunderscore .
\section{Galhudo}
\begin{itemize}
\item {Grp. gram.:adj.}
\end{itemize}
\begin{itemize}
\item {Utilização:Ant.}
\end{itemize}
\begin{itemize}
\item {Grp. gram.:M.}
\end{itemize}
\begin{itemize}
\item {Proveniência:(De \textunderscore galho\textunderscore )}
\end{itemize}
Que tem galhos.
Que tem chifres grandes.
Desprezível, desajeitado.
Nome de um peixe.
\section{Galhusco}
\begin{itemize}
\item {Grp. gram.:m.}
\end{itemize}
O mesmo que \textunderscore galhastro\textunderscore .
\section{Gália}
\begin{itemize}
\item {Grp. gram.:f.}
\end{itemize}
O mesmo que \textunderscore galião\textunderscore .
\section{Galiambo}
\begin{itemize}
\item {Grp. gram.:m.}
\end{itemize}
\begin{itemize}
\item {Proveniência:(Lat. \textunderscore galliambus\textunderscore )}
\end{itemize}
Verso grego ou latino de seis pés, em que domina o jambo.
Obra, escripta naquela espécie de verso.
\section{Galião}
\begin{itemize}
\item {Grp. gram.:m.}
\end{itemize}
Erva, que se aplicava em secar o leite das mulheres.
\section{Galicanismo}
\begin{itemize}
\item {Grp. gram.:m.}
\end{itemize}
Doutrina dos galicanos.
\section{Galicano}
\begin{itemize}
\item {Grp. gram.:adj.}
\end{itemize}
\begin{itemize}
\item {Grp. gram.:M.}
\end{itemize}
\begin{itemize}
\item {Proveniência:(Lat. \textunderscore gallicanus\textunderscore )}
\end{itemize}
Relativo á Gália; relativo á França.
Relativo á Igreja francesa: \textunderscore liberdades galicanas\textunderscore .
Partidário ou defensor das liberdades galicanas.
\section{Galicanto}
\begin{itemize}
\item {Grp. gram.:m.}
\end{itemize}
\begin{itemize}
\item {Utilização:Ant.}
\end{itemize}
\begin{itemize}
\item {Proveniência:(De \textunderscore galo\textunderscore  + \textunderscore cantar\textunderscore )}
\end{itemize}
O mesmo que \textunderscore galicínio\textunderscore .
\section{Galicar}
\begin{itemize}
\item {Grp. gram.:v. t.}
\end{itemize}
\begin{itemize}
\item {Utilização:Pleb.}
\end{itemize}
Contagiar de gálico.
\section{Galicentro}
\begin{itemize}
\item {Grp. gram.:m.}
\end{itemize}
\begin{itemize}
\item {Proveniência:(Do lat. \textunderscore gallus\textunderscore  + \textunderscore centrum\textunderscore )}
\end{itemize}
Erva, também conhecida por \textunderscore coração de galo\textunderscore .
\section{Galiciano}
\begin{itemize}
\item {Grp. gram.:m.  e  adj.}
\end{itemize}
O mesmo que \textunderscore galiziano\textunderscore .
\section{Galicina}
\begin{itemize}
\item {Grp. gram.:f.}
\end{itemize}
Éter metílico do ácido gálico, empregado como antiséptico.
\section{Galicínio}
\begin{itemize}
\item {Grp. gram.:m.}
\end{itemize}
\begin{itemize}
\item {Proveniência:(Lat. \textunderscore gallicinium\textunderscore )}
\end{itemize}
Canto do galo.
Hora matutina, em que o galo canta.
\section{Galiciparla}
\begin{itemize}
\item {Grp. gram.:m.}
\end{itemize}
\begin{itemize}
\item {Utilização:P. us.}
\end{itemize}
\begin{itemize}
\item {Proveniência:(De \textunderscore gálico\textunderscore ^1 + \textunderscore parlar\textunderscore )}
\end{itemize}
Aquele que fala afrancesadamente.
Amigo de galicismos; galicista. Cf. Filinto III, 253; V, 17; Garrett, \textunderscore Retr. de Vênus\textunderscore , 196.
\section{Galicismo}
\begin{itemize}
\item {Grp. gram.:m.}
\end{itemize}
\begin{itemize}
\item {Proveniência:(Do lat. \textunderscore gallicus\textunderscore )}
\end{itemize}
Palavra ou frase, de formação ou indole afrancesada, e inútil ou opposta ao gênio da língua portuguesa.
Palavra derivada directamente do francês; francesismo.
\section{Galicista}
\begin{itemize}
\item {Grp. gram.:m.}
\end{itemize}
Aquele que usa galicismos; amigo de galicismos. Cf. Camillo, \textunderscore Noites de Insómn.\textunderscore , III, 53.
(Cp. \textunderscore galicismo\textunderscore )
\section{Gálico}
\begin{itemize}
\item {Grp. gram.:adj.}
\end{itemize}
\begin{itemize}
\item {Grp. gram.:M.}
\end{itemize}
\begin{itemize}
\item {Utilização:Pleb.}
\end{itemize}
\begin{itemize}
\item {Proveniência:(Lat. \textunderscore gallicus\textunderscore )}
\end{itemize}
Relativo á Gália, gaulês.
O mesmo que \textunderscore sífilis\textunderscore .
\section{Gálico}
\begin{itemize}
\item {Grp. gram.:adj.}
\end{itemize}
\begin{itemize}
\item {Proveniência:(Do lat. \textunderscore galla\textunderscore )}
\end{itemize}
Diz-se de um ácido, extraído da noz de galha.
\section{Galícola}
\begin{itemize}
\item {Grp. gram.:adj.}
\end{itemize}
\begin{itemize}
\item {Proveniência:(Do lat. \textunderscore galla\textunderscore  + \textunderscore colere\textunderscore )}
\end{itemize}
Que vive ou aparece nas galhas.
Diz-se da filoxera, que se manifesta nas galhas ou empôlas da fôlha da videira.
\section{Galifato}
\begin{itemize}
\item {Grp. gram.:m.}
\end{itemize}
\begin{itemize}
\item {Utilização:Prov.}
\end{itemize}
\begin{itemize}
\item {Utilização:trasm.}
\end{itemize}
O mesmo que \textunderscore garoto\textunderscore . (Colhido em Alijó)
\section{Galigée}
\begin{itemize}
\item {Grp. gram.:f.}
\end{itemize}
\begin{itemize}
\item {Utilização:Ant.}
\end{itemize}
O mesmo que \textunderscore galilé\textunderscore . Cf. \textunderscore Port. Mon. Hist.\textunderscore , \textunderscore Script.\textunderscore , 289.
\section{Galilé}
\begin{itemize}
\item {Grp. gram.:f.}
\end{itemize}
\begin{itemize}
\item {Utilização:Ant.}
\end{itemize}
\begin{itemize}
\item {Utilização:Prov.}
\end{itemize}
\begin{itemize}
\item {Proveniência:(Do b. lat. \textunderscore galilaea\textunderscore )}
\end{itemize}
Cemitério, destinado ao entêrro de pessôas nobres em alguns conventos. Cf. Herculano, \textunderscore Bobo\textunderscore , 295.
A parte alpendrada dos claustros.
Dependência alpendrada dos claustros, onde se celebravam as assembleias dos parochianos.
Agrupamento de garotos.
\section{Galileia}
\begin{itemize}
\item {Grp. gram.:f.}
\end{itemize}
O mesmo que \textunderscore galilé\textunderscore .
\section{Galileias}
\begin{itemize}
\item {Grp. gram.:f. pl.}
\end{itemize}
\begin{itemize}
\item {Utilização:Pop.}
\end{itemize}
O mesmo que \textunderscore Galliza\textunderscore :«\textunderscore nunca lá dessas galileias saiu cabeça tão romba\textunderscore ». Garrett.
\section{Galileu}
\begin{itemize}
\item {Grp. gram.:m.}
\end{itemize}
\begin{itemize}
\item {Grp. gram.:Adj.}
\end{itemize}
\begin{itemize}
\item {Proveniência:(Lat. \textunderscore galilaeus\textunderscore )}
\end{itemize}
Habitante da Galileia.
Relativo á Galileia.
\section{Galimar}
\begin{itemize}
\item {Grp. gram.:v. t.}
\end{itemize}
Cortar pelo galimo.
\section{Galimatias}
\begin{itemize}
\item {Grp. gram.:m.}
\end{itemize}
\begin{itemize}
\item {Proveniência:(Fr. \textunderscore galimatias\textunderscore )}
\end{itemize}
Confusão no falar.
Discurso obscuro; imbróglio.
Aranzel.
\section{Galimatizar}
\begin{itemize}
\item {Grp. gram.:v. i.}
\end{itemize}
Fazer aranzel; discorrer confusamente:«\textunderscore e já Píndaro a flux galimatiza\textunderscore ». Filinto, VIII, 45.
\section{Galimo}
\begin{itemize}
\item {Grp. gram.:m.}
\end{itemize}
Superfície de prancha ou madeiro, que se galiva pelos troços.
\section{Galináceas}
\begin{itemize}
\item {Grp. gram.:f. pl.}
\end{itemize}
\begin{itemize}
\item {Proveniência:(De \textunderscore galináceo\textunderscore )}
\end{itemize}
Ordem de aves, geralmente granívoras, que compreende as galinhas, os perus, as perdizes, etc.
\section{Galináceo}
\begin{itemize}
\item {Grp. gram.:adj.}
\end{itemize}
\begin{itemize}
\item {Proveniência:(Lat. \textunderscore gallinaceus\textunderscore )}
\end{itemize}
Relativo á ordem das galináceas.
\section{Galinário}
\begin{itemize}
\item {Grp. gram.:m.}
\end{itemize}
\begin{itemize}
\item {Proveniência:(Do lat. \textunderscore gallina\textunderscore )}
\end{itemize}
Aquele que fazia a compra das galinhas para a hucharia real.
Menino, que se criava e educava no palácio real.
O mesmo que \textunderscore infanção\textunderscore .
\section{Galindrau}
\begin{itemize}
\item {Grp. gram.:m.}
\end{itemize}
Instrumento, com que os carpinteiros de barcos repuxam as tábuas para o lugar próprio.
(Cp. \textunderscore galindréu\textunderscore )
\section{Galindréu}
\begin{itemize}
\item {Grp. gram.:m.}
\end{itemize}
\begin{itemize}
\item {Utilização:Náut.}
\end{itemize}
Chapa de ferro, que aguenta o mastro contra a bancada de uma embarcação.
\section{Galinha}
\begin{itemize}
\item {Grp. gram.:f.}
\end{itemize}
\begin{itemize}
\item {Utilização:Prov.}
\end{itemize}
\begin{itemize}
\item {Utilização:dur.}
\end{itemize}
\begin{itemize}
\item {Grp. gram.:Pl.}
\end{itemize}
\begin{itemize}
\item {Grp. gram.:Loc.}
\end{itemize}
\begin{itemize}
\item {Utilização:fam.}
\end{itemize}
\begin{itemize}
\item {Proveniência:(Do lat. \textunderscore gallina\textunderscore )}
\end{itemize}
Fêmea do galo.
Má sorte; desdita.
Mau olhado.
Espécie de jôgo popular.
\textunderscore Quando as galinhas tiverem dentes\textunderscore , nunca.
\section{Galinhaça}
\begin{itemize}
\item {Grp. gram.:f.}
\end{itemize}
\begin{itemize}
\item {Utilização:Pop.}
\end{itemize}
Excremento de galinhas.
\section{Galinhaço}
\begin{itemize}
\item {Grp. gram.:m.}
\end{itemize}
\begin{itemize}
\item {Utilização:Pop.}
\end{itemize}
\begin{itemize}
\item {Proveniência:(Do lat. \textunderscore gallinaceus\textunderscore )}
\end{itemize}
O mesmo que \textunderscore galinhaça\textunderscore .
Porção de galinhas, ou as galinhas em geral.
\section{Galinhame}
\begin{itemize}
\item {Grp. gram.:m.}
\end{itemize}
Porção de galinhas.
\section{Galinhó}
\begin{itemize}
\item {Grp. gram.:m.}
\end{itemize}
\begin{itemize}
\item {Utilização:Prov.}
\end{itemize}
\begin{itemize}
\item {Utilização:trasm.}
\end{itemize}
Gomo de laranja; galelo.
\section{Galinsoga}
\begin{itemize}
\item {Grp. gram.:f.}
\end{itemize}
\begin{itemize}
\item {Proveniência:(De \textunderscore Galinsoga\textunderscore , n. p.)}
\end{itemize}
Gênero de plantas compostas.
\section{Gálio}
\begin{itemize}
\item {Grp. gram.:m.}
\end{itemize}
\begin{itemize}
\item {Proveniência:(Gr. \textunderscore galion\textunderscore )}
\end{itemize}
Gênero de plantas rubiáceas, com que se póde coalhar o leite.
\section{Galipó}
\begin{itemize}
\item {Grp. gram.:m.}
\end{itemize}
O mesmo que \textunderscore galipote\textunderscore .
(Cast. \textunderscore galipodio\textunderscore )
\section{Galipódio}
\begin{itemize}
\item {Grp. gram.:m.}
\end{itemize}
O mesmo que \textunderscore galipote\textunderscore .
(Cast. \textunderscore galipodio\textunderscore )
\section{Galipote}
\begin{itemize}
\item {Grp. gram.:m.}
\end{itemize}
Terebenthina impura, sólida, privada do seu óleo essencial.
Incenso branco.
Resina, que fica no tronco do pinheiro, depois de extrahida a therebenthina.
Resina, com que se barra o fundo de algumas embarcações mercantes.
\section{Galisia}
\begin{itemize}
\item {Grp. gram.:f.}
\end{itemize}
\begin{itemize}
\item {Utilização:Bras. do N}
\end{itemize}
Difficuldade; embaraço.
Novidade.
\section{Galivação}
\begin{itemize}
\item {Grp. gram.:f.}
\end{itemize}
Acto de galivar.
\section{Galivar}
\begin{itemize}
\item {Grp. gram.:v. t.}
\end{itemize}
Tornar apropriado, dar o devido feitio a.
Tracejar.
(Provavelmente, do cast. \textunderscore galibo\textunderscore , que se relaciona com o port. \textunderscore calibre\textunderscore )
\section{Galizabra}
\begin{itemize}
\item {Grp. gram.:f.}
\end{itemize}
Embarcação de vela, no Mediterrâneo.
\section{Gallação}
\begin{itemize}
\item {Grp. gram.:f.}
\end{itemize}
\begin{itemize}
\item {Utilização:Bras}
\end{itemize}
O mesmo que \textunderscore galladura\textunderscore .
\section{Gallacrista}
\begin{itemize}
\item {Grp. gram.:f.}
\end{itemize}
Planta ornamental, cuja florescência imita a crista do gallo.
(Cp. \textunderscore gallicrista\textunderscore )
\section{Galladura}
\begin{itemize}
\item {Grp. gram.:f.}
\end{itemize}
Acto ou effeito de gallar.
Ponto branco que, na gemma do ovo, indica a fecundação.
\section{Gallão}
\begin{itemize}
\item {Grp. gram.:m.}
\end{itemize}
\begin{itemize}
\item {Proveniência:(Ingl. \textunderscore gallon\textunderscore )}
\end{itemize}
Antiga medida inglesa para líquidos, que se usou também em Portugal, e ainda se usa no fornecimento dos vernizes para os caminhos de ferro.--Cada gallão corresponde a quatro litros e meio.
\section{Gallar}
\begin{itemize}
\item {Grp. gram.:v. t.}
\end{itemize}
\begin{itemize}
\item {Utilização:Pop.}
\end{itemize}
\begin{itemize}
\item {Grp. gram.:V. p.}
\end{itemize}
\begin{itemize}
\item {Proveniência:(De \textunderscore gallo\textunderscore )}
\end{itemize}
Fecundar, (falando-se das gallináceas).
Fornicar.
Collocar-se:«\textunderscore gallou-se de repente sobre mim\textunderscore ». F. Manuel, \textunderscore Apólogos\textunderscore .
\section{Gallaripo}
\begin{itemize}
\item {Grp. gram.:m.}
\end{itemize}
\begin{itemize}
\item {Utilização:Prov.}
\end{itemize}
\begin{itemize}
\item {Utilização:beir.}
\end{itemize}
\begin{itemize}
\item {Utilização:Prov.}
\end{itemize}
\begin{itemize}
\item {Utilização:minh.}
\end{itemize}
Rapaz, que já pretende namorar.
O mesmo que \textunderscore gallo\textunderscore ^1, elevação na testa ou na cabeça, por effeito de pancada.
(Cp. \textunderscore gallo\textunderscore ^1)
\section{Gallarispo}
\begin{itemize}
\item {Grp. gram.:m.}
\end{itemize}
\begin{itemize}
\item {Utilização:Prov.}
\end{itemize}
\begin{itemize}
\item {Utilização:trasm.}
\end{itemize}
O mesmo que \textunderscore gallaripo\textunderscore .
\section{Gallas}
\begin{itemize}
\item {Grp. gram.:m. pl.}
\end{itemize}
Povos fronteiriços da Abyssínia.
\section{Gallato}
\begin{itemize}
\item {Grp. gram.:m.}
\end{itemize}
\begin{itemize}
\item {Proveniência:(Do lat. \textunderscore galla\textunderscore )}
\end{itemize}
Combinação do ácido gállico com uma base.
\section{Gallear}
\begin{itemize}
\item {Grp. gram.:v. i.}
\end{itemize}
\begin{itemize}
\item {Proveniência:(De \textunderscore gallo\textunderscore )}
\end{itemize}
Diz-se do toireiro se, de capote ao ombro e de costas voltadas para o toiro, abre os braços, quando o toiro arranca, e os vai movendo com o corpo, da direita para a esquerda.
\section{Galleciano}
\begin{itemize}
\item {Grp. gram.:adj.}
\end{itemize}
\begin{itemize}
\item {Proveniência:(Do lat. \textunderscore Gallaecia\textunderscore , n. p.)}
\end{itemize}
O mesmo que \textunderscore galliziano\textunderscore .
\section{Gallega}
\begin{itemize}
\item {fónica:lê}
\end{itemize}
\begin{itemize}
\item {Grp. gram.:adj. f.}
\end{itemize}
\begin{itemize}
\item {Utilização:Ant.}
\end{itemize}
\begin{itemize}
\item {Grp. gram.:F.}
\end{itemize}
Dizia-se da terra, que não é fértil ou que é charneca.
Espécie de ginja vermelha, muito ácida.
Espécie de couve.
Casta de uva branca de Ourém.
Variedade de azeitona.
\section{Gallegada}
\begin{itemize}
\item {Grp. gram.:f.}
\end{itemize}
\begin{itemize}
\item {Utilização:Pop.}
\end{itemize}
\begin{itemize}
\item {Utilização:Fam.}
\end{itemize}
Ajuntamento de gallegos.
Acção própria de gallego.
Brutalidade.
Acção ou dito grosseiro.
\section{Gallegaria}
\begin{itemize}
\item {Grp. gram.:f.}
\end{itemize}
Porção de gallegos.
Falario de gallegos. Cf. Garrett, \textunderscore Alfageme\textunderscore .
\section{Gallego}
\begin{itemize}
\item {fónica:lê}
\end{itemize}
\begin{itemize}
\item {Grp. gram.:m.}
\end{itemize}
\begin{itemize}
\item {Utilização:Pop.}
\end{itemize}
\begin{itemize}
\item {Grp. gram.:Adj.}
\end{itemize}
\begin{itemize}
\item {Utilização:Prov.}
\end{itemize}
\begin{itemize}
\item {Utilização:Prov.}
\end{itemize}
\begin{itemize}
\item {Utilização:beir.}
\end{itemize}
\begin{itemize}
\item {Utilização:Ant.}
\end{itemize}
Aquelle que é natural da Galliza.
Dialecto da Galliza.
Mariola, moço de fretes.
Casta de uva preta de Collares.
Homem grosseiro, incivil.
Relativo á Galliza.
Ordinário: \textunderscore ginja gallega\textunderscore .
Diz-se do vento do norte.
E diz-se também de uma espécie de trigo molle.
Diz-se da mesa em que não há pão.
Dizia-se de uma qualidade de tecido de linho:«\textunderscore Deixo 4 varas de linho galego...\textunderscore »(De um testamento de 1691)
\section{Gallego-de-montemor}
\begin{itemize}
\item {Grp. gram.:m.}
\end{itemize}
Variedade de uva. Cf. \textunderscore Rev. Agron.\textunderscore , I, 18.
\section{Gallego-doirado}
\begin{itemize}
\item {Grp. gram.:m.}
\end{itemize}
Casta de uva extremenha.
\section{Gallego-forcado}
\begin{itemize}
\item {Grp. gram.:m.}
\end{itemize}
Variedade de uva de Azeitão.
\section{Gallego-negrão}
\begin{itemize}
\item {Grp. gram.:m.}
\end{itemize}
Variedade de azeitona negrucha.
\section{Gallego-rapado}
\begin{itemize}
\item {Grp. gram.:adj.}
\end{itemize}
Diz-se de uma espécie de milho molle.
\section{Galleguice}
\begin{itemize}
\item {Grp. gram.:f.}
\end{itemize}
Acto ou modos de gallego.
\section{Galleguinho}
\begin{itemize}
\item {Grp. gram.:m.}
\end{itemize}
Variedade de uva de Azeitão.
\section{Galleguismo}
\begin{itemize}
\item {Grp. gram.:m.}
\end{itemize}
\begin{itemize}
\item {Proveniência:(De \textunderscore gallego\textunderscore )}
\end{itemize}
Palavra ou locução privativa da Galliza.
\section{Galleio}
\begin{itemize}
\item {Grp. gram.:m.}
\end{itemize}
Acto de gallear.
\section{Galleira}
\begin{itemize}
\item {Grp. gram.:f.}
\end{itemize}
\begin{itemize}
\item {Utilização:ant.}
\end{itemize}
\begin{itemize}
\item {Utilização:Fam.}
\end{itemize}
O mesmo que \textunderscore forcado\textunderscore .
\section{Galleirão}
\begin{itemize}
\item {Grp. gram.:m.}
\end{itemize}
\begin{itemize}
\item {Proveniência:(Do cast. \textunderscore gallarón\textunderscore )}
\end{itemize}
Ave pernalta, (\textunderscore fulca atra\textunderscore , Lin.).
O mesmo que \textunderscore abibe\textunderscore .
\section{Galleno}
\begin{itemize}
\item {Grp. gram.:m.}
\end{itemize}
\begin{itemize}
\item {Utilização:Prov.}
\end{itemize}
\begin{itemize}
\item {Proveniência:(De \textunderscore gallo\textunderscore ?)}
\end{itemize}
O mesmo que \textunderscore abibe\textunderscore .
\section{Gallês}
\begin{itemize}
\item {Grp. gram.:adj.}
\end{itemize}
\begin{itemize}
\item {Grp. gram.:M.}
\end{itemize}
Relativo ao país de Galles.
Língua antiga do país de Galles.
\section{Gállia}
\begin{itemize}
\item {Grp. gram.:f.}
\end{itemize}
O mesmo que \textunderscore gallião\textunderscore .
\section{Galliambo}
\begin{itemize}
\item {Grp. gram.:m.}
\end{itemize}
\begin{itemize}
\item {Proveniência:(Lat. \textunderscore galliambus\textunderscore )}
\end{itemize}
Verso grego ou latino de seis pés, em que domina o jambo.
Obra, escripta naquella espécie de verso.
\section{Gallião}
\begin{itemize}
\item {Grp. gram.:m.}
\end{itemize}
Erva, que se applicava em secar o leite das mulheres.
\section{Gallicanismo}
\begin{itemize}
\item {Grp. gram.:m.}
\end{itemize}
Doutrina dos gallicanos.
\section{Gallicano}
\begin{itemize}
\item {Grp. gram.:adj.}
\end{itemize}
\begin{itemize}
\item {Grp. gram.:M.}
\end{itemize}
\begin{itemize}
\item {Proveniência:(Lat. \textunderscore gallicanus\textunderscore )}
\end{itemize}
Relativo á Gállia; relativo á França.
Relativo á Igreja francesa: \textunderscore liberdades gallicanas\textunderscore .
Partidário ou defensor das liberdades gallicanas.
\section{Gallicanto}
\begin{itemize}
\item {Grp. gram.:m.}
\end{itemize}
\begin{itemize}
\item {Utilização:Ant.}
\end{itemize}
\begin{itemize}
\item {Proveniência:(De \textunderscore gallo\textunderscore  + \textunderscore cantar\textunderscore )}
\end{itemize}
O mesmo que \textunderscore gallicínio\textunderscore .
\section{Gallicar}
\begin{itemize}
\item {Grp. gram.:v. t.}
\end{itemize}
\begin{itemize}
\item {Utilização:Pleb.}
\end{itemize}
Contagiar de gállico.
\section{Gallicentro}
\begin{itemize}
\item {Grp. gram.:m.}
\end{itemize}
\begin{itemize}
\item {Proveniência:(Do lat. \textunderscore gallus\textunderscore  + \textunderscore centrum\textunderscore )}
\end{itemize}
Erva, também conhecida por \textunderscore coração de gallo\textunderscore .
\section{Galliciano}
\begin{itemize}
\item {Grp. gram.:m.  e  adj.}
\end{itemize}
O mesmo que \textunderscore galliziano\textunderscore .
\section{Gallicina}
\begin{itemize}
\item {Grp. gram.:f.}
\end{itemize}
Éther methýlico do ácido gállico, empregado como antiséptico.
\section{Gallicínio}
\begin{itemize}
\item {Grp. gram.:m.}
\end{itemize}
\begin{itemize}
\item {Proveniência:(Lat. \textunderscore gallicinium\textunderscore )}
\end{itemize}
Canto do gallo.
Hora matutina, em que o gallo canta.
\section{Galliciparla}
\begin{itemize}
\item {Grp. gram.:m.}
\end{itemize}
\begin{itemize}
\item {Utilização:P. us.}
\end{itemize}
\begin{itemize}
\item {Proveniência:(De \textunderscore gállico\textunderscore ^1 + \textunderscore parlar\textunderscore )}
\end{itemize}
Aquelle que fala afrancesadamente.
Amigo de gallicismos; gallicista. Cf. Filinto III, 253; V, 17; Garrett, \textunderscore Retr. de Vênus\textunderscore , 196.
\section{Gallicismo}
\begin{itemize}
\item {Grp. gram.:m.}
\end{itemize}
\begin{itemize}
\item {Proveniência:(Do lat. \textunderscore gallicus\textunderscore )}
\end{itemize}
Palavra ou phrase, de formação ou indole afrancesada, e inútil ou opposta ao gênio da língua portuguesa.
Palavra derivada directamente do francês; francesismo.
\section{Gallicista}
\begin{itemize}
\item {Grp. gram.:m.}
\end{itemize}
Aquelle que usa gallicismos; amigo de gallicismos. Cf. Camillo, \textunderscore Noites de Insómn.\textunderscore , III, 53.
(Cp. \textunderscore gallicismo\textunderscore )
\section{Gállico}
\begin{itemize}
\item {Grp. gram.:adj.}
\end{itemize}
\begin{itemize}
\item {Grp. gram.:M.}
\end{itemize}
\begin{itemize}
\item {Utilização:Pleb.}
\end{itemize}
\begin{itemize}
\item {Proveniência:(Lat. \textunderscore gallicus\textunderscore )}
\end{itemize}
Relativo á Gállia, gaulês.
O mesmo que \textunderscore sýphilis\textunderscore .
\section{Gállico}
\begin{itemize}
\item {Grp. gram.:adj.}
\end{itemize}
\begin{itemize}
\item {Proveniência:(Do lat. \textunderscore galla\textunderscore )}
\end{itemize}
Diz-se de um ácido, extrahído da noz de galha.
\section{Gallícola}
\begin{itemize}
\item {Grp. gram.:adj.}
\end{itemize}
\begin{itemize}
\item {Proveniência:(Do lat. \textunderscore galla\textunderscore  + \textunderscore colere\textunderscore )}
\end{itemize}
Que vive ou apparece nas galhas.
Diz-se da phylloxera, que se manifesta nas galhas ou empôlas da fôlha da videira.
\section{Gallicrista}
\begin{itemize}
\item {Grp. gram.:f.}
\end{itemize}
O mesmo que \textunderscore gallacrista\textunderscore . Cf. B. Pereira, \textunderscore Prosódia\textunderscore .
\section{Gallináceas}
\begin{itemize}
\item {Grp. gram.:f. pl.}
\end{itemize}
\begin{itemize}
\item {Proveniência:(De \textunderscore gallináceo\textunderscore )}
\end{itemize}
Ordem de aves, geralmente granívoras, que comprehende as gallinhas, os perus, as perdizes, etc.
\section{Gallináceo}
\begin{itemize}
\item {Grp. gram.:adj.}
\end{itemize}
\begin{itemize}
\item {Proveniência:(Lat. \textunderscore gallinaceus\textunderscore )}
\end{itemize}
Relativo á ordem das gallináceas.
\section{Gallinário}
\begin{itemize}
\item {Grp. gram.:m.}
\end{itemize}
\begin{itemize}
\item {Proveniência:(Do lat. \textunderscore gallina\textunderscore )}
\end{itemize}
Aquelle que fazia a compra das gallinhas para a hucharia real.
Menino, que se criava e educava no palácio real.
O mesmo que \textunderscore infanção\textunderscore .
\section{Gallinha}
\begin{itemize}
\item {Grp. gram.:f.}
\end{itemize}
\begin{itemize}
\item {Utilização:Prov.}
\end{itemize}
\begin{itemize}
\item {Utilização:dur.}
\end{itemize}
\begin{itemize}
\item {Grp. gram.:Pl.}
\end{itemize}
\begin{itemize}
\item {Grp. gram.:Loc.}
\end{itemize}
\begin{itemize}
\item {Utilização:fam.}
\end{itemize}
\begin{itemize}
\item {Proveniência:(Do lat. \textunderscore gallina\textunderscore )}
\end{itemize}
Fêmea do gallo.
Má sorte; desdita.
Mau olhado.
Espécie de jôgo popular.
\textunderscore Quando as gallinhas tiverem dentes\textunderscore , nunca.
\section{Gallinhaça}
\begin{itemize}
\item {Grp. gram.:f.}
\end{itemize}
\begin{itemize}
\item {Utilização:Pop.}
\end{itemize}
Excremento de gallinhas.
\section{Gallinha-cega}
\begin{itemize}
\item {Grp. gram.:f.}
\end{itemize}
Espécie de jôgo popular, o mesmo que \textunderscore cabra-cega\textunderscore .
\section{Gallinha-choca}
\begin{itemize}
\item {Grp. gram.:f.}
\end{itemize}
\begin{itemize}
\item {Utilização:Pop.}
\end{itemize}
Pessôa doente e descòrada; pessôa achacadiça.
\section{Gallinhaço}
\begin{itemize}
\item {Grp. gram.:m.}
\end{itemize}
\begin{itemize}
\item {Utilização:Pop.}
\end{itemize}
\begin{itemize}
\item {Proveniência:(Do lat. \textunderscore gallinaceus\textunderscore )}
\end{itemize}
O mesmo que \textunderscore gallinhaça\textunderscore .
Porção de gallinhas, ou as gallinhas em geral.
\section{Gallinha-da-índia}
\begin{itemize}
\item {Grp. gram.:f.}
\end{itemize}
Ave gallinácea, (\textunderscore numida meleagris\textunderscore ).
\section{Gallinha-de-água}
\begin{itemize}
\item {Grp. gram.:f.}
\end{itemize}
\begin{itemize}
\item {Utilização:T. da Bairrada}
\end{itemize}
O mesmo que \textunderscore rabila\textunderscore .
\section{Gallinha-de-angola}
\begin{itemize}
\item {Grp. gram.:f.}
\end{itemize}
\begin{itemize}
\item {Utilização:Bras}
\end{itemize}
Ave gallinácea, também conhecida por \textunderscore guiné\textunderscore .
\section{Gallinha-do-mar}
\begin{itemize}
\item {Grp. gram.:f.}
\end{itemize}
Peixe de Portugal.
\section{Gallinhame}
\begin{itemize}
\item {Grp. gram.:m.}
\end{itemize}
Porção de gallinhas.
\section{Gallinha-sultana}
\begin{itemize}
\item {Grp. gram.:f.}
\end{itemize}
Ave ribeirinha.
\section{Galinheira}
\begin{itemize}
\item {Grp. gram.:f.}
\end{itemize}
Mulher, que vende galinhas.
\section{Galinheiro}
\begin{itemize}
\item {Grp. gram.:m.}
\end{itemize}
\begin{itemize}
\item {Utilização:T. de Lisbôa}
\end{itemize}
\begin{itemize}
\item {Utilização:Pop.}
\end{itemize}
\begin{itemize}
\item {Grp. gram.:Loc.}
\end{itemize}
\begin{itemize}
\item {Utilização:pop.}
\end{itemize}
\begin{itemize}
\item {Proveniência:(Do lat. \textunderscore gallinarius\textunderscore )}
\end{itemize}
Capoeira.
Poleiro.
Vendedor de galinhas.
Lugar nos teatros, por cima dos camarotes, no qual se acumulam os espectadores, sem número fixo; torrinhas.
Cachaço, pescoço.
\textunderscore Ir ao galinheiro\textunderscore , dar pancadas, bater.
\section{Galinhola}
\begin{itemize}
\item {Grp. gram.:f.}
\end{itemize}
\begin{itemize}
\item {Proveniência:(De \textunderscore galinha\textunderscore )}
\end{itemize}
Ave pernalta e longipenne.
\section{Galinhota}
\begin{itemize}
\item {Grp. gram.:f.}
\end{itemize}
\begin{itemize}
\item {Proveniência:(De \textunderscore galinha\textunderscore )}
\end{itemize}
Ave pernalta, (\textunderscore fulica chloropus\textunderscore ).
\section{Galinocultura}
\begin{itemize}
\item {Grp. gram.:f.}
\end{itemize}
\begin{itemize}
\item {Utilização:Bras}
\end{itemize}
\begin{itemize}
\item {Utilização:Neol.}
\end{itemize}
\begin{itemize}
\item {Proveniência:(Do lat. \textunderscore gallina\textunderscore  + \textunderscore cultura\textunderscore )}
\end{itemize}
Criação de galinhas.
\section{Galinsectos}
\begin{itemize}
\item {Grp. gram.:m. pl.}
\end{itemize}
\begin{itemize}
\item {Proveniência:(Fr. \textunderscore gallinsecte\textunderscore )}
\end{itemize}
Família de insectos hemípteros.
\section{Gálio}
\begin{itemize}
\item {Grp. gram.:m.}
\end{itemize}
Antiga língua das Gálias, pertencente ao ramo céltico.
Indivíduo natural das Gálias.
\section{Galiparla}
\begin{itemize}
\item {Grp. gram.:m.}
\end{itemize}
O mesmo que \textunderscore galiciparla\textunderscore .
\section{Galiqueira}
\begin{itemize}
\item {Grp. gram.:f.}
\end{itemize}
\begin{itemize}
\item {Utilização:Pleb.}
\end{itemize}
\begin{itemize}
\item {Proveniência:(De \textunderscore gallicar\textunderscore )}
\end{itemize}
Doença sifilitica.
\section{Galiré}
\begin{itemize}
\item {Grp. gram.:f.}
\end{itemize}
\begin{itemize}
\item {Utilização:Bras. do N}
\end{itemize}
Espécie de galinha muito pequena.
\section{Galismo}
\begin{itemize}
\item {Grp. gram.:m.}
\end{itemize}
\begin{itemize}
\item {Utilização:P. us.}
\end{itemize}
\begin{itemize}
\item {Proveniência:(De \textunderscore Gall\textunderscore , n. p.)}
\end{itemize}
O mesmo que \textunderscore Frenologia\textunderscore .
\section{Galispo}
\begin{itemize}
\item {Grp. gram.:m.}
\end{itemize}
\begin{itemize}
\item {Grp. gram.:Adj.}
\end{itemize}
\begin{itemize}
\item {Utilização:Prov.}
\end{itemize}
\begin{itemize}
\item {Utilização:alent.}
\end{itemize}
Pequeno galo.
O mesmo que \textunderscore abibe\textunderscore .
Que tem um só testículo, (falando-se de burros ou cavalos).
\section{Galista}
\begin{itemize}
\item {Grp. gram.:adj.}
\end{itemize}
\begin{itemize}
\item {Grp. gram.:M.}
\end{itemize}
Relativo a Gall ou á sua doutrina.
Partidário de Gall.
(Cp. \textunderscore galismo\textunderscore )
\section{Galiziano}
\begin{itemize}
\item {Grp. gram.:adj.}
\end{itemize}
\begin{itemize}
\item {Proveniência:(De \textunderscore Galiza\textunderscore , n. p.)}
\end{itemize}
Diz-se do dialecto, da poesia e dos trovadores de Portugal e da Galiza, nos primeiros séculos da nacionalidade portuguesa.
\section{Gallinheira}
\begin{itemize}
\item {Grp. gram.:f.}
\end{itemize}
Mulher, que vende gallinhas.
\section{Gallinheiro}
\begin{itemize}
\item {Grp. gram.:m.}
\end{itemize}
\begin{itemize}
\item {Utilização:T. de Lisbôa}
\end{itemize}
\begin{itemize}
\item {Utilização:Pop.}
\end{itemize}
\begin{itemize}
\item {Grp. gram.:Loc.}
\end{itemize}
\begin{itemize}
\item {Utilização:pop.}
\end{itemize}
\begin{itemize}
\item {Proveniência:(Do lat. \textunderscore gallinarius\textunderscore )}
\end{itemize}
Capoeira.
Poleiro.
Vendedor de gallinhas.
Lugar nos theatros, por cima dos camarotes, no qual se acumulam os espectadores, sem número fixo; torrinhas.
Cachaço, pescoço.
\textunderscore Ir ao gallinheiro\textunderscore , dar pancadas, bater.
\section{Gallinhola}
\begin{itemize}
\item {Grp. gram.:f.}
\end{itemize}
\begin{itemize}
\item {Proveniência:(De \textunderscore gallinha\textunderscore )}
\end{itemize}
Ave pernalta e longipenne.
\section{Gallinhota}
\begin{itemize}
\item {Grp. gram.:f.}
\end{itemize}
\begin{itemize}
\item {Proveniência:(De \textunderscore gallinha\textunderscore )}
\end{itemize}
Ave pernalta, (\textunderscore fulica chloropus\textunderscore ).
\section{Gallinocultura}
\begin{itemize}
\item {Grp. gram.:f.}
\end{itemize}
\begin{itemize}
\item {Utilização:Bras}
\end{itemize}
\begin{itemize}
\item {Utilização:Neol.}
\end{itemize}
\begin{itemize}
\item {Proveniência:(Do lat. \textunderscore gallina\textunderscore  + \textunderscore cultura\textunderscore )}
\end{itemize}
Criação de gallinhas.
\section{Gallinsectos}
\begin{itemize}
\item {Grp. gram.:m. pl.}
\end{itemize}
\begin{itemize}
\item {Proveniência:(Fr. \textunderscore gallinsecte\textunderscore )}
\end{itemize}
Família de insectos hemípteros.
\section{Gállio}
\begin{itemize}
\item {Grp. gram.:m.}
\end{itemize}
Antiga língua das Gállias, pertencente ao ramo céltico.
Indivíduo natural das Gállias.
\section{Galliparla}
\begin{itemize}
\item {Grp. gram.:m.}
\end{itemize}
O mesmo que \textunderscore galliciparla\textunderscore .
\section{Galliqueira}
\begin{itemize}
\item {Grp. gram.:f.}
\end{itemize}
\begin{itemize}
\item {Utilização:Pleb.}
\end{itemize}
\begin{itemize}
\item {Proveniência:(De \textunderscore gallicar\textunderscore )}
\end{itemize}
Doença syphilitica.
\section{Galliré}
\begin{itemize}
\item {Grp. gram.:f.}
\end{itemize}
\begin{itemize}
\item {Utilização:Bras. do N}
\end{itemize}
Espécie de gallinha muito pequena.
\section{Gallismo}
\begin{itemize}
\item {Grp. gram.:m.}
\end{itemize}
\begin{itemize}
\item {Utilização:P. us.}
\end{itemize}
\begin{itemize}
\item {Proveniência:(De \textunderscore Gall\textunderscore , n. p.)}
\end{itemize}
O mesmo que \textunderscore Phrenologia\textunderscore .
\section{Gallispo}
\begin{itemize}
\item {Grp. gram.:m.}
\end{itemize}
\begin{itemize}
\item {Grp. gram.:Adj.}
\end{itemize}
\begin{itemize}
\item {Utilização:Prov.}
\end{itemize}
\begin{itemize}
\item {Utilização:alent.}
\end{itemize}
Pequeno gallo.
O mesmo que \textunderscore abibe\textunderscore .
Que tem um só testículo, (falando-se de burros ou cavallos).
\section{Gallista}
\begin{itemize}
\item {Grp. gram.:adj.}
\end{itemize}
\begin{itemize}
\item {Grp. gram.:M.}
\end{itemize}
Relativo a Gall ou á sua doutrina.
Partidário de Gall.
(Cp. \textunderscore gallismo\textunderscore )
\section{Gallizão}
\begin{itemize}
\item {Grp. gram.:m.}
\end{itemize}
O mesmo que \textunderscore milhão\textunderscore ^2. Cf. \textunderscore Bibl. da G. do Campo\textunderscore , 303.
\section{Galliziano}
\begin{itemize}
\item {Grp. gram.:adj.}
\end{itemize}
\begin{itemize}
\item {Proveniência:(De \textunderscore Galliza\textunderscore , n. p.)}
\end{itemize}
Diz-se do dialecto, da poesia e dos trovadores de Portugal e da Galliza, nos primeiros séculos da nacionalidade portuguesa.
\section{Gallo}
\begin{itemize}
\item {Grp. gram.:m.}
\end{itemize}
\begin{itemize}
\item {Utilização:Pop.}
\end{itemize}
\begin{itemize}
\item {Utilização:Prov.}
\end{itemize}
\begin{itemize}
\item {Utilização:alent.}
\end{itemize}
\begin{itemize}
\item {Proveniência:(Lat. \textunderscore gallus\textunderscore )}
\end{itemize}
Gênero de aves gallináceas, de crista carnuda e asas curtas e largas.
Elevação na testa ou na cabeça, produzida por pancada.
Peixe de Portugal.
Variedade de ameixa.
\section{Gallo}
\begin{itemize}
\item {Grp. gram.:m.  e  adj.}
\end{itemize}
\begin{itemize}
\item {Proveniência:(Lat. \textunderscore gallus\textunderscore )}
\end{itemize}
O mesmo ou melhor que \textunderscore gaulês\textunderscore .
\section{Gallocrista}
\begin{itemize}
\item {Grp. gram.:f.}
\end{itemize}
O mesmo que \textunderscore gallacrista\textunderscore .
\section{Gallo-da-serra}
\begin{itemize}
\item {Grp. gram.:m.}
\end{itemize}
Ave brasileira, amarelada, de pernas robustas, com esporões como o gallo.
\section{Gallo-de-bando}
\begin{itemize}
\item {Grp. gram.:m.}
\end{itemize}
Pássaro brasileiro, nocivo aos frutos.
\section{Gallo-de-campina}
\begin{itemize}
\item {Grp. gram.:m.}
\end{itemize}
\begin{itemize}
\item {Utilização:Bras}
\end{itemize}
Passarinho de cabeça vermelha.
\section{Gallo-dos-rochedos}
\begin{itemize}
\item {Grp. gram.:m.}
\end{itemize}
Formosa ave americana, (\textunderscore rupicola pipra\textunderscore , Lin.), que faz o ninho em cavernas ou em fendas de rochedos.
\section{Gallomania}
\begin{itemize}
\item {Grp. gram.:f.}
\end{itemize}
Qualidade de gallomaníaco.
\section{Gallomaníaco}
\begin{itemize}
\item {Grp. gram.:m.  e  adj.}
\end{itemize}
\begin{itemize}
\item {Proveniência:(De \textunderscore gallo\textunderscore ^2 + \textunderscore mania\textunderscore )}
\end{itemize}
Indivíduo, que admira excessivamente a França ou as coisas de França e procura imitá-las, até no que ellas têm de inacceitável.
\section{Gallómano}
\begin{itemize}
\item {Grp. gram.:m.  e  adj.}
\end{itemize}
O mesmo que \textunderscore gallomaníaco\textunderscore . Cf. Garrett, \textunderscore Retr. de Vênus\textunderscore , 196.
\section{Gallophobia}
\begin{itemize}
\item {Grp. gram.:f.}
\end{itemize}
Qualidade de quem é gallóphobo.
\section{Gallóphobo}
\begin{itemize}
\item {Grp. gram.:m.  e  adj.}
\end{itemize}
\begin{itemize}
\item {Proveniência:(De \textunderscore gallo\textunderscore ^2 + gr. \textunderscore phobos\textunderscore )}
\end{itemize}
O que tem ódio aos Franceses ou á França.
\section{Gallo-românico}
\begin{itemize}
\item {Grp. gram.:m.}
\end{itemize}
Um dos cinco ramos principaes das línguas novi-latinas, que comprehende o francês, o provençal e o catalão.
\section{Gallo-romano}
\begin{itemize}
\item {Grp. gram.:adj.}
\end{itemize}
Relativo a Gállios e Romanos, depois da conquista romana das Gállias.
\section{Galluchada}
\begin{itemize}
\item {Grp. gram.:f.}
\end{itemize}
Porção de galluchos.
\section{Gallucho}
\begin{itemize}
\item {Grp. gram.:m.}
\end{itemize}
\begin{itemize}
\item {Utilização:Fig.}
\end{itemize}
\begin{itemize}
\item {Proveniência:(De \textunderscore gallo\textunderscore ^1)}
\end{itemize}
Recruta; soldado bisonho.
Novato, caloiro.
Sujeito acanhado, inexperiente.
\section{Galo}
\begin{itemize}
\item {Grp. gram.:m.}
\end{itemize}
\begin{itemize}
\item {Utilização:Pop.}
\end{itemize}
\begin{itemize}
\item {Utilização:Prov.}
\end{itemize}
\begin{itemize}
\item {Utilização:alent.}
\end{itemize}
\begin{itemize}
\item {Proveniência:(Lat. \textunderscore gallus\textunderscore )}
\end{itemize}
Gênero de aves galináceas, de crista carnuda e asas curtas e largas.
Elevação na testa ou na cabeça, produzida por pancada.
Peixe de Portugal.
Variedade de ameixa.
\section{Galo}
\begin{itemize}
\item {Grp. gram.:m.  e  adj.}
\end{itemize}
\begin{itemize}
\item {Proveniência:(Lat. \textunderscore gallus\textunderscore )}
\end{itemize}
O mesmo ou melhor que \textunderscore gaulês\textunderscore .
\section{Galocha}
\begin{itemize}
\item {Grp. gram.:f.}
\end{itemize}
\begin{itemize}
\item {Utilização:Prov.}
\end{itemize}
\begin{itemize}
\item {Utilização:trasm.}
\end{itemize}
\begin{itemize}
\item {Proveniência:(Fr. \textunderscore galoche\textunderscore )}
\end{itemize}
Espécie de calçado, com rasto de madeira ou borracha.
Chinela de borracha, que se calça por cima dos sapatos ou botas, como preservativo da humidade.
Rebento do enxêrto.
Peça de metal, no bordo do navio, por onde labora um virador, uma espia, etc.
Primeiro sulco, que se faz, para abrir uma valla.
\section{Galocrista}
\begin{itemize}
\item {Grp. gram.:f.}
\end{itemize}
O mesmo que \textunderscore galacrista\textunderscore .
\section{Galofobia}
\begin{itemize}
\item {Grp. gram.:f.}
\end{itemize}
Qualidade de quem é galófobo.
\section{Galófobo}
\begin{itemize}
\item {Grp. gram.:m.  e  adj.}
\end{itemize}
\begin{itemize}
\item {Proveniência:(De \textunderscore galo\textunderscore ^2 + gr. \textunderscore phobos\textunderscore )}
\end{itemize}
O que tem ódio aos Franceses ou á França.
\section{Galolo}
\begin{itemize}
\item {Grp. gram.:m.}
\end{itemize}
Língua, falada em Timor, nos reinos de léste.
\section{Galomania}
\begin{itemize}
\item {Grp. gram.:f.}
\end{itemize}
Qualidade de galomaníaco.
\section{Galomaníaco}
\begin{itemize}
\item {Grp. gram.:m.  e  adj.}
\end{itemize}
\begin{itemize}
\item {Proveniência:(De \textunderscore galo\textunderscore ^2 + \textunderscore mania\textunderscore )}
\end{itemize}
Indivíduo, que admira excessivamente a França ou as coisas de França e procura imitá-las, até no que elas têm de inaceitável.
\section{Galómano}
\begin{itemize}
\item {Grp. gram.:m.  e  adj.}
\end{itemize}
O mesmo que \textunderscore galomaníaco\textunderscore . Cf. Garrett, \textunderscore Retr. de Vênus\textunderscore , 196.
\section{Galonar}
\begin{itemize}
\item {Grp. gram.:v. t.}
\end{itemize}
(V.agaloar)
\section{Galopada}
\begin{itemize}
\item {Grp. gram.:f.}
\end{itemize}
O mesmo que \textunderscore galope\textunderscore .
\section{Galopado}
\begin{itemize}
\item {Grp. gram.:adj.}
\end{itemize}
\begin{itemize}
\item {Proveniência:(De \textunderscore galopar\textunderscore )}
\end{itemize}
Ensinado a galopar.
\section{Galopador}
\begin{itemize}
\item {Grp. gram.:m.  e  adj.}
\end{itemize}
Indivíduo, ou cavalgadura, que galopa bem.
\section{Galopante}
\begin{itemize}
\item {Grp. gram.:adj.}
\end{itemize}
Que galopa.
\textunderscore Tísica galopante\textunderscore , tísica granulosa, de desenlace rápido.
\section{Galopar}
\begin{itemize}
\item {Grp. gram.:v. i.}
\end{itemize}
\begin{itemize}
\item {Grp. gram.:V. t.}
\end{itemize}
\begin{itemize}
\item {Proveniência:(Do germ. \textunderscore hlaupan\textunderscore )}
\end{itemize}
Andar a galope, depressa.
Andar, baixando e levantando alternadamente a parte deanteira e a traseira, (falando-se de carruagens do caminho de ferro).
Percorrer rapidamente.
\section{Galope}
\begin{itemize}
\item {Grp. gram.:m.}
\end{itemize}
\begin{itemize}
\item {Utilização:Fig.}
\end{itemize}
\begin{itemize}
\item {Utilização:Náut.}
\end{itemize}
\begin{itemize}
\item {Utilização:Bras. do S}
\end{itemize}
\begin{itemize}
\item {Proveniência:(De \textunderscore galopar\textunderscore )}
\end{itemize}
A mais rápida andadura de alguns animaes, especialmente do cavallo.
Espécie de dança a dois tempos.
Corrida veloz.
Parte dos mastros, entre a encapelladura e a borla.
Acto de galopar uma carruagem do caminho de ferro.
Admoestação; censura.
\section{Galopear}
\begin{itemize}
\item {Grp. gram.:v. i.}
\end{itemize}
\begin{itemize}
\item {Utilização:Bras. do N}
\end{itemize}
O mesmo que \textunderscore galopar\textunderscore .
\section{Galopim}
\begin{itemize}
\item {Grp. gram.:m.}
\end{itemize}
Rapaz brejeiro.
Garoto.
Aquelle que angaria votos para eleições.
(Cast. \textunderscore galopin\textunderscore , do rad. de \textunderscore galopar\textunderscore )
\section{Galopinagem}
\begin{itemize}
\item {Grp. gram.:f.}
\end{itemize}
Acto ou effeito de galopinar.
\section{Galopinar}
\begin{itemize}
\item {Grp. gram.:v. i.}
\end{itemize}
Têr vida de galopim.
Angariar votos para eleições.
\section{Galpão}
\begin{itemize}
\item {Grp. gram.:m.}
\end{itemize}
\begin{itemize}
\item {Utilização:Bras. do S}
\end{itemize}
Varanda; alpendre.
(Do azteca)
\section{Galra}
\begin{itemize}
\item {Grp. gram.:f.}
\end{itemize}
\begin{itemize}
\item {Utilização:Gír.}
\end{itemize}
\begin{itemize}
\item {Proveniência:(De \textunderscore galrar\textunderscore )}
\end{itemize}
A voz.
\section{Galracho}
\begin{itemize}
\item {Grp. gram.:m.}
\end{itemize}
\begin{itemize}
\item {Utilização:Prov.}
\end{itemize}
\begin{itemize}
\item {Utilização:beir.}
\end{itemize}
O mesmo que \textunderscore escalracho\textunderscore .
\section{Galradeira}
\begin{itemize}
\item {Grp. gram.:f.}
\end{itemize}
\begin{itemize}
\item {Utilização:ant.}
\end{itemize}
\begin{itemize}
\item {Utilização:Gír.}
\end{itemize}
\begin{itemize}
\item {Proveniência:(De \textunderscore galrar\textunderscore )}
\end{itemize}
A lingua.
\section{Galrão}
\begin{itemize}
\item {Grp. gram.:m.  e  adj.}
\end{itemize}
\begin{itemize}
\item {Proveniência:(De \textunderscore galrar\textunderscore )}
\end{itemize}
Tagarela.
O que fala muito.
\section{Galrar}
\begin{itemize}
\item {Grp. gram.:v. i.}
\end{itemize}
\begin{itemize}
\item {Utilização:T. de Turquel}
\end{itemize}
\begin{itemize}
\item {Proveniência:(Do lat. p. us. \textunderscore garrulare\textunderscore , que deu \textunderscore garlar\textunderscore , donde, por metáth., \textunderscore galrar\textunderscore )}
\end{itemize}
Falar á tôa.
Falar muito, sem necessidade.
Parolar.
Blasonar.
Desenvolver-se rapidamente: \textunderscore com as últimas chuvas, as searas galraram\textunderscore .
\section{Galreador}
\begin{itemize}
\item {Grp. gram.:m.  e  adj.}
\end{itemize}
O que galreia.
\section{Galrear}
\begin{itemize}
\item {Grp. gram.:v. i.}
\end{itemize}
\begin{itemize}
\item {Proveniência:(De \textunderscore galra\textunderscore )}
\end{itemize}
O mesmo que \textunderscore galrar\textunderscore .
Diz-se especialmente das crianças, que emittem vozes, sem articular palavras.
\section{Galreiro}
\begin{itemize}
\item {Grp. gram.:adj.}
\end{itemize}
\begin{itemize}
\item {Utilização:Fam.}
\end{itemize}
\begin{itemize}
\item {Proveniência:(De \textunderscore galrar\textunderscore )}
\end{itemize}
Que fala muito.
\section{Galrejador}
\begin{itemize}
\item {Grp. gram.:m.  e  adj.}
\end{itemize}
O que galreja.
\section{Galrejar}
\begin{itemize}
\item {Grp. gram.:v. i.}
\end{itemize}
O mesmo que \textunderscore galrear\textunderscore .
\section{Galricho}
\begin{itemize}
\item {Grp. gram.:m.}
\end{itemize}
O mesmo que \textunderscore galrito\textunderscore .
\section{Galricho}
\begin{itemize}
\item {Grp. gram.:m.}
\end{itemize}
\begin{itemize}
\item {Utilização:T. de Turquel}
\end{itemize}
Copo muito pequeno.
\section{Galripo}
\begin{itemize}
\item {Grp. gram.:m.}
\end{itemize}
\begin{itemize}
\item {Utilização:Prov.}
\end{itemize}
Saco de pano, para coar as fezes do vinho. Cf. Júl. Moreira, \textunderscore Estudos da Ling. Port.\textunderscore , I, 189.
\section{Galrito}
\begin{itemize}
\item {Grp. gram.:m.}
\end{itemize}
\begin{itemize}
\item {Utilização:Prov.}
\end{itemize}
\begin{itemize}
\item {Utilização:trasm.}
\end{itemize}
Rêde, para pescar peixe miúdo.
Saco, para coar vinho.
(Metáth. do cast. \textunderscore garlito\textunderscore )
\section{Galuchada}
\begin{itemize}
\item {Grp. gram.:f.}
\end{itemize}
Porção de galuchos.
\section{Galucho}
\begin{itemize}
\item {Grp. gram.:m.}
\end{itemize}
\begin{itemize}
\item {Utilização:Fig.}
\end{itemize}
\begin{itemize}
\item {Proveniência:(De \textunderscore galo\textunderscore ^1)}
\end{itemize}
Recruta; soldado bisonho.
Novato, caloiro.
Sujeito acanhado, inexperiente.
\section{Galula}
\begin{itemize}
\item {Grp. gram.:f.}
\end{itemize}
\begin{itemize}
\item {Utilização:Prov.}
\end{itemize}
\begin{itemize}
\item {Utilização:trasm.}
\end{itemize}
Bôa qualidade de alguma coisa para se comer.
\section{Galúmpio}
\begin{itemize}
\item {Grp. gram.:m.}
\end{itemize}
\begin{itemize}
\item {Utilização:Prov.}
\end{itemize}
\begin{itemize}
\item {Utilização:alg.}
\end{itemize}
O mesmo que \textunderscore baloiço\textunderscore .
\section{Galvânico}
\begin{itemize}
\item {Grp. gram.:adj.}
\end{itemize}
Relativo ao galvanismo.
\section{Galvanismo}
\begin{itemize}
\item {Grp. gram.:m.}
\end{itemize}
\begin{itemize}
\item {Proveniência:(De \textunderscore Galvani\textunderscore , n. p.)}
\end{itemize}
Electricidade, produzida por contacto de certos corpos ou por acções chímicas.
Phenómenos eléctricos nos músculos.
\section{Galvanização}
\begin{itemize}
\item {Grp. gram.:f.}
\end{itemize}
Acto ou effeito de galvanizar.
\section{Galvanizante}
\begin{itemize}
\item {Grp. gram.:adj.}
\end{itemize}
\begin{itemize}
\item {Utilização:Fig.}
\end{itemize}
Que galvaniza.
Que dá novo alento ou vida nova.
\section{Galvanizantemente}
\begin{itemize}
\item {Grp. gram.:adv.}
\end{itemize}
De modo galvanizante.
\section{Galvanizar}
\begin{itemize}
\item {Grp. gram.:v. t.}
\end{itemize}
\begin{itemize}
\item {Utilização:Fig.}
\end{itemize}
Electrizar, por meio de pilha galvânica ou voltaica.
Pratear ou doirar por meio da galvanoplástica.
Dar movimento aos músculos, em vida ou pouco depois da morte, por meio da electricidade galvânica.
Reanimar.
Dar vida fictícia a.
(Cp. \textunderscore galvanismo\textunderscore )
\section{Galvanocáustica}
\begin{itemize}
\item {Grp. gram.:f.}
\end{itemize}
\begin{itemize}
\item {Proveniência:(De \textunderscore galvanismo\textunderscore  + gr. \textunderscore kaustikos\textunderscore )}
\end{itemize}
Conjunto das operações cirúrgicas, auxiliadas pelo calor eléctrico.
\section{Galvanocáustico}
\begin{itemize}
\item {Grp. gram.:adj.}
\end{itemize}
Relativo á galvanocáustica.
\section{Galvanografia}
\begin{itemize}
\item {Grp. gram.:f.}
\end{itemize}
\begin{itemize}
\item {Proveniência:(De \textunderscore galvanismo\textunderscore  + gr. \textunderscore graphein\textunderscore )}
\end{itemize}
Processo galvanoplástico de gravura.
\section{Galvanographia}
\begin{itemize}
\item {Grp. gram.:f.}
\end{itemize}
\begin{itemize}
\item {Proveniência:(De \textunderscore galvanismo\textunderscore  + gr. \textunderscore graphein\textunderscore )}
\end{itemize}
Processo galvanoplástico de gravura.
\section{Galvanogravura}
\begin{itemize}
\item {Grp. gram.:f.}
\end{itemize}
Processo de reproduzir objectos, por meio de uma corrente eléctrica, buril e verniz dos gravadores. Cf. F. Lapa, \textunderscore Phýs. e Chím.\textunderscore , II, 80.
\section{Galvanólise}
\begin{itemize}
\item {Grp. gram.:f.}
\end{itemize}
Electrólise cirúrgica.
\section{Galvanólyse}
\begin{itemize}
\item {Grp. gram.:f.}
\end{itemize}
Electrólyse cirúrgica.
\section{Galvanomagnético}
\begin{itemize}
\item {Grp. gram.:adj.}
\end{itemize}
Relativo ao galvanomagnetismo.
\section{Galvanomagnetismo}
\begin{itemize}
\item {Grp. gram.:m.}
\end{itemize}
\begin{itemize}
\item {Proveniência:(De \textunderscore galvanismo\textunderscore  + \textunderscore magnetismo\textunderscore )}
\end{itemize}
Conjunto de phenómenos, em que se produzem effeitos magnéticos, por meio do galvanismo.
\section{Galvanómetro}
\begin{itemize}
\item {Grp. gram.:m.}
\end{itemize}
\begin{itemize}
\item {Proveniência:(De \textunderscore galvanismo\textunderscore  + gr. \textunderscore metron\textunderscore )}
\end{itemize}
Instrumento, para medir a intensidade das correntes galvânicas.
\section{Galvanoplastia}
\begin{itemize}
\item {Grp. gram.:f.}
\end{itemize}
\begin{itemize}
\item {Proveniência:(De \textunderscore galvanismo\textunderscore  + \textunderscore plástica\textunderscore )}
\end{itemize}
Arte de applicar uma camada metállica sôbre qualquer substância, por meio da pilha galvânica.
\section{Galvanoplástica}
\begin{itemize}
\item {Grp. gram.:f.}
\end{itemize}
\begin{itemize}
\item {Proveniência:(De \textunderscore galvanismo\textunderscore  + \textunderscore plástica\textunderscore )}
\end{itemize}
Arte de applicar uma camada metállica sôbre qualquer substância, por meio da pilha galvânica.
\section{Galvanoplástico}
\begin{itemize}
\item {Grp. gram.:adj.}
\end{itemize}
Relativo á galvanoplastia.
\section{Galvanopunctura}
\begin{itemize}
\item {Grp. gram.:f.}
\end{itemize}
\begin{itemize}
\item {Utilização:Cir.}
\end{itemize}
\begin{itemize}
\item {Proveniência:(De \textunderscore galvanismo\textunderscore  + \textunderscore punctura\textunderscore )}
\end{itemize}
Méthodo de tratamento, com que se transmitte aos tecidos a acção chímica das correntes galvânicas, por meio de agulhas, introduzidas nos órgãos ou tumores.
\section{Galvanopuntura}
\begin{itemize}
\item {Grp. gram.:f.}
\end{itemize}
\begin{itemize}
\item {Utilização:Cir.}
\end{itemize}
\begin{itemize}
\item {Proveniência:(De \textunderscore galvanismo\textunderscore  + \textunderscore puntura\textunderscore )}
\end{itemize}
Método de tratamento, com que se transmite aos tecidos a acção química das correntes galvânicas, por meio de agulhas, introduzidas nos órgãos ou tumores.
\section{Galvanoscópio}
\begin{itemize}
\item {Grp. gram.:m.}
\end{itemize}
\begin{itemize}
\item {Proveniência:(De \textunderscore galvanismo\textunderscore  + gr. \textunderscore skopein\textunderscore )}
\end{itemize}
Instrumento, que torna sensíveis á vista os effeitos galvânicos.
\section{Galvanotaxia}
\begin{itemize}
\item {fónica:csi}
\end{itemize}
\begin{itemize}
\item {Grp. gram.:f.}
\end{itemize}
Movimentos, executados pelo protoplasma, sob a influência de corrente eléctrica.
\section{Galvanoterapia}
\begin{itemize}
\item {Grp. gram.:f.}
\end{itemize}
\begin{itemize}
\item {Proveniência:(De \textunderscore galvanismo\textunderscore  + gr. \textunderscore therapeia\textunderscore )}
\end{itemize}
Aplicação do galvanismo á terapêutica.
\section{Galvanotherapia}
\begin{itemize}
\item {Grp. gram.:f.}
\end{itemize}
\begin{itemize}
\item {Proveniência:(De \textunderscore galvanismo\textunderscore  + gr. \textunderscore therapeia\textunderscore )}
\end{itemize}
Applicação do galvanismo á therapêutica.
\section{Galvanotherápico}
\begin{itemize}
\item {Grp. gram.:adj.}
\end{itemize}
Relativo á galvanòtherapia.
\section{Galvanotipia}
\begin{itemize}
\item {Grp. gram.:f.}
\end{itemize}
\begin{itemize}
\item {Proveniência:(De \textunderscore galvanismo\textunderscore  + \textunderscore tipo\textunderscore )}
\end{itemize}
Aplicação galvanòplástica á estereotipia.
\section{Galvanotropismo}
\begin{itemize}
\item {Grp. gram.:m.}
\end{itemize}
\begin{itemize}
\item {Utilização:Bot.}
\end{itemize}
Phenómeno, que se dá nas plantas aquáticas, inclinando-se as raízes de um para o outro lado, sob a acção de uma corrente eléctrica, que atravessa a água.
\section{Galvanotypia}
\begin{itemize}
\item {Grp. gram.:f.}
\end{itemize}
\begin{itemize}
\item {Proveniência:(De \textunderscore galvanismo\textunderscore  + \textunderscore typo\textunderscore )}
\end{itemize}
Applicação galvanòplástica á estereotypia.
\section{Galveta}
\begin{itemize}
\item {fónica:vê}
\end{itemize}
\begin{itemize}
\item {Grp. gram.:f.}
\end{itemize}
Pequena e ligeira embarcação indiana.
Parte de uma armação de atum.
\section{Gama}
\begin{itemize}
\item {Grp. gram.:f.}
\end{itemize}
Fêmea de gamo^1.
\section{Gamacha}
\begin{itemize}
\item {Grp. gram.:f.}
\end{itemize}
\begin{itemize}
\item {Utilização:Ant.}
\end{itemize}
\begin{itemize}
\item {Proveniência:(De \textunderscore gama\textunderscore ?)}
\end{itemize}
Mulher infiel?:«\textunderscore ...tyrannia de uma gamacha proterva...\textunderscore »Soropita, \textunderscore Prosas\textunderscore , 146.
\section{Gamão}
\begin{itemize}
\item {Grp. gram.:m.}
\end{itemize}
\begin{itemize}
\item {Utilização:Bot.}
\end{itemize}
\begin{itemize}
\item {Utilização:Prov.}
\end{itemize}
\begin{itemize}
\item {Utilização:alent.}
\end{itemize}
\begin{itemize}
\item {Proveniência:(Do celt. \textunderscore cammon\textunderscore ?)}
\end{itemize}
Jôgo de asar e cálculo, entre dois parceiros, com quinze tábulas cada um.
Tabuleiro, sôbre que se joga o gamão.
Planta liliácea.
Haste da abrótea.
\section{Gamar}
\begin{itemize}
\item {Grp. gram.:v. t.}
\end{itemize}
\begin{itemize}
\item {Utilização:Gír.}
\end{itemize}
Furtar com subtileza.
(Cp. \textunderscore gramar\textunderscore ^1)
\section{Gamarra}
\begin{itemize}
\item {Grp. gram.:f.}
\end{itemize}
Correia, que se ata, da cilha ao bocal ou cabeção da cavalgadura, para que ella não levante muito a cabeça.
(Cast. \textunderscore gamarra\textunderscore )
\section{Gamba}
\begin{itemize}
\item {Grp. gram.:f.}
\end{itemize}
\begin{itemize}
\item {Utilização:Mús.}
\end{itemize}
\begin{itemize}
\item {Proveniência:(It. \textunderscore gamba\textunderscore )}
\end{itemize}
Registo de órgão, com tubos delgados, cujo timbre especial imita instrumentos de corda.
\section{Gambá}
\begin{itemize}
\item {Grp. gram.:m.}
\end{itemize}
O mesmo que \textunderscore sarigueia\textunderscore .
\section{Gambadonas}
\begin{itemize}
\item {Grp. gram.:f. pl.}
\end{itemize}
\begin{itemize}
\item {Utilização:Náut.}
\end{itemize}
Cordas, em que se envolvem os mastros, para os tornar mais fortes.
\section{Gambadonos}
\begin{itemize}
\item {Grp. gram.:m. pl.}
\end{itemize}
O mesmo que \textunderscore gambadonas\textunderscore .
\section{Gambarra}
\begin{itemize}
\item {Grp. gram.:f.}
\end{itemize}
\begin{itemize}
\item {Utilização:Bras}
\end{itemize}
Embarcação pequena.
\section{Gambérria}
\begin{itemize}
\item {Grp. gram.:f.}
\end{itemize}
\begin{itemize}
\item {Utilização:Pop.}
\end{itemize}
\begin{itemize}
\item {Proveniência:(Do it. \textunderscore gambiera\textunderscore )}
\end{itemize}
Pancada com o pé nas pernas de outro indivíduo, para o fazer caír; cambapé.
Trampolina, tranquibérnia.
Guerreia, motim, desordem.
\section{Gambetas}
\begin{itemize}
\item {fónica:bê}
\end{itemize}
\begin{itemize}
\item {Grp. gram.:f. pl.}
\end{itemize}
\begin{itemize}
\item {Utilização:Prov.}
\end{itemize}
\begin{itemize}
\item {Utilização:minh.}
\end{itemize}
As pernas.
(Cp. \textunderscore gâmbia\textunderscore )
\section{Gâmbia}
\begin{itemize}
\item {Grp. gram.:f.}
\end{itemize}
\begin{itemize}
\item {Utilização:Pop.}
\end{itemize}
\begin{itemize}
\item {Proveniência:(Do b. lat. \textunderscore gamba\textunderscore )}
\end{itemize}
Perna.
\section{Gambiarra}
\begin{itemize}
\item {Grp. gram.:f.}
\end{itemize}
Rampa de luzes na parte superior dos palcos.
\section{Gâmbias}
\begin{itemize}
\item {Grp. gram.:m.  e  f.}
\end{itemize}
\begin{itemize}
\item {Utilização:Prov.}
\end{itemize}
\begin{itemize}
\item {Utilização:trasm.}
\end{itemize}
\begin{itemize}
\item {Proveniência:(De \textunderscore gâmbia\textunderscore )}
\end{itemize}
Pessôa muito alta. (Colhido em Murça)
\section{Gambito}
\begin{itemize}
\item {Grp. gram.:m.}
\end{itemize}
\begin{itemize}
\item {Proveniência:(Do it. \textunderscore gambetto\textunderscore , de \textunderscore gamba\textunderscore )}
\end{itemize}
Ardil, para prostrar o adversário.
Um dos lances do xadrez.
\section{Gambo}
\begin{itemize}
\item {Grp. gram.:m.}
\end{itemize}
Árvore angolense, no Duque-de-Bragança.
\section{Gambôa}
\begin{itemize}
\item {Grp. gram.:f.}
\end{itemize}
\begin{itemize}
\item {Utilização:Prov.}
\end{itemize}
Fruto do gamboeiro.
O mesmo que \textunderscore marmelo\textunderscore .
(Cast. \textunderscore zamboa\textunderscore , talvez do ár.)
\section{Gambôa}
\begin{itemize}
\item {Grp. gram.:f.}
\end{itemize}
\begin{itemize}
\item {Utilização:Bras}
\end{itemize}
Pequeno esteiro, que se enche com o fluxo da maré, e fica em sêco com o refluxo.
\section{Gambôa}
\begin{itemize}
\item {Grp. gram.:f.}
\end{itemize}
\begin{itemize}
\item {Utilização:T. de Moçambique}
\end{itemize}
Estacaria para pesca.
\section{Gambocha}
\begin{itemize}
\item {Grp. gram.:f.}
\end{itemize}
Lotação de lans de differentes qualidades, (na linguagem de cardadores e colchoeiros).
\section{Gamboeiro}
\begin{itemize}
\item {Grp. gram.:m.}
\end{itemize}
\begin{itemize}
\item {Proveniência:(De \textunderscore gambôa\textunderscore )}
\end{itemize}
Variedade de marmeleiro, (\textunderscore cydonia vulgaris britannica\textunderscore ).
\section{Gamboína}
\begin{itemize}
\item {Grp. gram.:f.}
\end{itemize}
\begin{itemize}
\item {Utilização:Pop.}
\end{itemize}
\begin{itemize}
\item {Proveniência:(Do rad. do it. \textunderscore gamba\textunderscore )}
\end{itemize}
Trapaça no jôgo.
\section{Gambonito}
\begin{itemize}
\item {Grp. gram.:m.}
\end{itemize}
\begin{itemize}
\item {Utilização:Prov.}
\end{itemize}
\begin{itemize}
\item {Utilização:beir.}
\end{itemize}
Variedade de planta.
\section{Gambota}
\begin{itemize}
\item {Grp. gram.:f.}
\end{itemize}
Molde ou arco de madeira, para a construcção de uma abóbada; cimbres.
(Refl. de \textunderscore cambota\textunderscore )
\section{Gambozinos}
\begin{itemize}
\item {Grp. gram.:m. pl.}
\end{itemize}
\begin{itemize}
\item {Utilização:Prov.}
\end{itemize}
Peixes ou pássaros imaginários, com que, por brincadeira, se logravam os pacóvios, convidando-os ou mandando-os á pesca ou á caça dêsses peixes ou pássaros.
\textunderscore Andar aos gambozinos\textunderscore , andar á tuna, vadiar; andar desnorteado, á tôa.
\section{Gama}
\begin{itemize}
\item {Grp. gram.:f.}
\end{itemize}
\begin{itemize}
\item {Utilização:Fig.}
\end{itemize}
\begin{itemize}
\item {Grp. gram.:M.}
\end{itemize}
\begin{itemize}
\item {Proveniência:(Gr. \textunderscore gamma\textunderscore , nome da letra \textunderscore g\textunderscore )}
\end{itemize}
Sucessão de sons de uma oitava musical; escala.
Série de ideias, teorias, etc.
Nome de letra, que no alfabeto grego corresponde ao nosso \textunderscore g\textunderscore .
\section{Gamarografia}
\begin{itemize}
\item {Grp. gram.:f.}
\end{itemize}
\begin{itemize}
\item {Proveniência:(Do gr. \textunderscore gammaros\textunderscore , caranguejo, e \textunderscore graphein\textunderscore , descrever)}
\end{itemize}
Parte da Zoologia, que trata dos crustáceos.
\section{Gamarográfico}
\begin{itemize}
\item {Grp. gram.:adj.}
\end{itemize}
Relativo á gamarografia.
\section{Gamarógrafo}
\begin{itemize}
\item {Grp. gram.:m.}
\end{itemize}
Naturalista, que se dedica ao estudo da gamarografia.
\section{Gamarólito}
\begin{itemize}
\item {Grp. gram.:f.}
\end{itemize}
\begin{itemize}
\item {Proveniência:(Do gr. \textunderscore gammaros\textunderscore  + \textunderscore lithos\textunderscore )}
\end{itemize}
Crustáceo fóssil.
\section{Gamarologia}
\begin{itemize}
\item {Grp. gram.:f.}
\end{itemize}
\begin{itemize}
\item {Proveniência:(Do gr. \textunderscore gammaros\textunderscore  + \textunderscore logos\textunderscore )}
\end{itemize}
Tratado científico dos crustáceos.
\section{Gamarológico}
\begin{itemize}
\item {Grp. gram.:adj.}
\end{itemize}
Relativo á gamarologia.
\section{Gamarólogo}
\begin{itemize}
\item {Grp. gram.:m.}
\end{itemize}
Naturalista, que é perito em gamarologia.
\section{Gameiro}
\begin{itemize}
\item {Grp. gram.:adj.}
\end{itemize}
Diz-se de uma variedade de milho amarelo. Cf. \textunderscore Portugal Agrícola\textunderscore , 9.^o anno, 367.
\section{Gamela}
\begin{itemize}
\item {Grp. gram.:f.}
\end{itemize}
\begin{itemize}
\item {Proveniência:(De \textunderscore gama\textunderscore )}
\end{itemize}
Pequena corça.
\section{Gamela}
\begin{itemize}
\item {Grp. gram.:m.}
\end{itemize}
\begin{itemize}
\item {Utilização:Prov.}
\end{itemize}
\begin{itemize}
\item {Utilização:minh.}
\end{itemize}
Indivíduo boçal, lorpa.
\section{Gamela}
\begin{itemize}
\item {Grp. gram.:f.}
\end{itemize}
\begin{itemize}
\item {Proveniência:(Do lat. \textunderscore camella\textunderscore , que me parece vir de \textunderscore camum\textunderscore , espécie de cerveja)}
\end{itemize}
Grande vasilha de madeira, em fórma de tigela.
Escudella.
Erva santomense, de fruto leitoso e medicinal.
\section{Gamelã}
\begin{itemize}
\item {Grp. gram.:m.}
\end{itemize}
Instrumento indiano, semelhante ás marimbas.
\section{Gamelada}
\begin{itemize}
\item {Grp. gram.:f.}
\end{itemize}
Porção de comida, contida numa gamela.
Líquido, contido numa gamela.
\section{Gamelan}
\begin{itemize}
\item {Grp. gram.:m.}
\end{itemize}
Instrumento indiano, semelhante ás marimbas.
\section{Gamelão}
\begin{itemize}
\item {Grp. gram.:m.}
\end{itemize}
\begin{itemize}
\item {Utilização:T. da Bairrada}
\end{itemize}
Gamella grande.
\section{Gamelas}
\begin{itemize}
\item {Grp. gram.:m. pl.}
\end{itemize}
O mesmo que \textunderscore gameleiros\textunderscore .
\section{Gameleira}
\begin{itemize}
\item {Grp. gram.:f.}
\end{itemize}
Árvore resinosa do Brasil, nas regiões do Amazonas.
O mesmo que \textunderscore coajinguva\textunderscore .
Árvore brasileira, (\textunderscore ficus doliaria\textunderscore ).
\section{Gameleiros}
\begin{itemize}
\item {Grp. gram.:m. pl.}
\end{itemize}
\begin{itemize}
\item {Proveniência:(De \textunderscore gamela\textunderscore )}
\end{itemize}
Nome, que os exploradores do Brasil deram a algumas tríbos de tupinambás, por usarem no beiço inferior um largo buraco, tapado com uma rodela de cuia.
\section{Gamélia}
\begin{itemize}
\item {Grp. gram.:f.}
\end{itemize}
\begin{itemize}
\item {Proveniência:(Gr. \textunderscore gamelia\textunderscore )}
\end{itemize}
Gênero de insectos coleópteros tetrâmeros.
\section{Gamélias}
\begin{itemize}
\item {Grp. gram.:f. pl.}
\end{itemize}
\begin{itemize}
\item {Proveniência:(Gr. \textunderscore gamelion\textunderscore )}
\end{itemize}
Festas, que os Gregos faziam a Juno, sob o epítheto de Gamélia.
\section{Gamélio}
\begin{itemize}
\item {Grp. gram.:m.}
\end{itemize}
\begin{itemize}
\item {Proveniência:(Gr. \textunderscore gamelion\textunderscore )}
\end{itemize}
Mês atheniense, em que se celebravam as gamélias, e em que se costumavam celebrar as núpcias.
\section{Gamella}
\begin{itemize}
\item {Grp. gram.:f.}
\end{itemize}
\begin{itemize}
\item {Proveniência:(Do lat. \textunderscore camella\textunderscore , que me parece vir de \textunderscore camum\textunderscore , espécie de cerveja)}
\end{itemize}
Grande vasilha de madeira, em fórma de tigela.
Escudella.
Erva santhomense, de fruto leitoso e medicinal.
\section{Gamellada}
\begin{itemize}
\item {Grp. gram.:f.}
\end{itemize}
Porção de comida, contida numa gamella.
Líquido, contido numa gamella.
\section{Gamellão}
\begin{itemize}
\item {Grp. gram.:m.}
\end{itemize}
\begin{itemize}
\item {Utilização:T. da Bairrada}
\end{itemize}
Gamella grande.
\section{Gamellas}
\begin{itemize}
\item {Grp. gram.:m. pl.}
\end{itemize}
O mesmo que \textunderscore gamelleiros\textunderscore .
\section{Gamelleira}
\begin{itemize}
\item {Grp. gram.:f.}
\end{itemize}
Árvore brasileira, (\textunderscore ficus doliaria\textunderscore ).
\section{Gamelleiros}
\begin{itemize}
\item {Grp. gram.:m. pl.}
\end{itemize}
\begin{itemize}
\item {Proveniência:(De \textunderscore gamella\textunderscore )}
\end{itemize}
Nome, que os exploradores do Brasil deram a algumas tríbos de tupinambás, por usarem no beiço inferior um largo buraco, tapado com uma rodela de cuia.
\section{Gamello}
\begin{itemize}
\item {fónica:mê}
\end{itemize}
\begin{itemize}
\item {Grp. gram.:m.}
\end{itemize}
\begin{itemize}
\item {Proveniência:(Do rad. de \textunderscore gamella\textunderscore )}
\end{itemize}
Vasilha longa, em que se deita água ou comida para o gado.
\section{Gamellório}
\begin{itemize}
\item {Grp. gram.:m.}
\end{itemize}
\begin{itemize}
\item {Utilização:Prov.}
\end{itemize}
\begin{itemize}
\item {Utilização:minh.}
\end{itemize}
\begin{itemize}
\item {Proveniência:(De \textunderscore gamella\textunderscore )}
\end{itemize}
O mesmo que \textunderscore comezaina\textunderscore .
\section{Gamellote}
\begin{itemize}
\item {Grp. gram.:m.}
\end{itemize}
Pequeno gamello.
\section{Gamelo}
\begin{itemize}
\item {fónica:mê}
\end{itemize}
\begin{itemize}
\item {Grp. gram.:m.}
\end{itemize}
\begin{itemize}
\item {Proveniência:(Do rad. de \textunderscore gamella\textunderscore )}
\end{itemize}
Vasilha longa, em que se deita água ou comida para o gado.
\section{Gamelório}
\begin{itemize}
\item {Grp. gram.:m.}
\end{itemize}
\begin{itemize}
\item {Utilização:Prov.}
\end{itemize}
\begin{itemize}
\item {Utilização:minh.}
\end{itemize}
\begin{itemize}
\item {Proveniência:(De \textunderscore gamela\textunderscore )}
\end{itemize}
O mesmo que \textunderscore comezaina\textunderscore .
\section{Gamelote}
\begin{itemize}
\item {Grp. gram.:m.}
\end{itemize}
Pequeno gamelo.
\section{Gamenho}
\begin{itemize}
\item {Grp. gram.:m.  e  adj.}
\end{itemize}
\begin{itemize}
\item {Utilização:Pop.}
\end{itemize}
\begin{itemize}
\item {Proveniência:(Do fr. \textunderscore gamin\textunderscore ?)}
\end{itemize}
Indivíduo garrido, vistoso, peralta, casquilho.
Tunante, vadio. Cf. Camillo, \textunderscore Brasileira\textunderscore , 224.
\section{Gameta}
\begin{itemize}
\item {fónica:mê}
\end{itemize}
\begin{itemize}
\item {Grp. gram.:f.}
\end{itemize}
\begin{itemize}
\item {Utilização:Prov.}
\end{itemize}
\begin{itemize}
\item {Utilização:trasm.}
\end{itemize}
\begin{itemize}
\item {Proveniência:(De \textunderscore gamo\textunderscore ^2?)}
\end{itemize}
O mesmo que \textunderscore lentilha\textunderscore .
\section{Gâmeto}
\begin{itemize}
\item {Grp. gram.:m.}
\end{itemize}
\begin{itemize}
\item {Utilização:Bot.}
\end{itemize}
\begin{itemize}
\item {Proveniência:(Gr. \textunderscore gametes\textunderscore )}
\end{itemize}
Cada uma das duas céllulas, entre as quaes se opéra a fecundação.
\section{Gâmia}
\begin{itemize}
\item {Grp. gram.:adj. f.}
\end{itemize}
\begin{itemize}
\item {Utilização:Prov.}
\end{itemize}
\begin{itemize}
\item {Utilização:trasm.}
\end{itemize}
Diz-se da mulher que se precipita com avidez sôbre alguma coisa que lhe appeteça.
(Cp. \textunderscore gamar\textunderscore )
\section{Gamma}
\begin{itemize}
\item {Grp. gram.:f.}
\end{itemize}
\begin{itemize}
\item {Utilização:Fig.}
\end{itemize}
\begin{itemize}
\item {Grp. gram.:M.}
\end{itemize}
\begin{itemize}
\item {Proveniência:(Gr. \textunderscore gamma\textunderscore , nome da letra \textunderscore g\textunderscore )}
\end{itemize}
Successão de sons de uma oitava musical; escala.
Série de ideias, theorias, etc.
Nome de letra, que no alphabeto grego corresponde ao nosso \textunderscore g\textunderscore .
\section{Gammarographia}
\begin{itemize}
\item {Grp. gram.:f.}
\end{itemize}
\begin{itemize}
\item {Proveniência:(Do gr. \textunderscore gammaros\textunderscore , caranguejo, e \textunderscore graphein\textunderscore , descrever)}
\end{itemize}
Parte da Zoologia, que trata dos crustáceos.
\section{Gammarográphico}
\begin{itemize}
\item {Grp. gram.:adj.}
\end{itemize}
Relativo á gammarographia.
\section{Gammarógrapho}
\begin{itemize}
\item {Grp. gram.:m.}
\end{itemize}
Naturalista, que se dedica ao estudo da gammarographia.
\section{Gammarólitho}
\begin{itemize}
\item {Grp. gram.:f.}
\end{itemize}
\begin{itemize}
\item {Proveniência:(Do gr. \textunderscore gammaros\textunderscore  + \textunderscore lithos\textunderscore )}
\end{itemize}
Crustáceo fóssil.
\section{Gammarologia}
\begin{itemize}
\item {Grp. gram.:f.}
\end{itemize}
\begin{itemize}
\item {Proveniência:(Do gr. \textunderscore gammaros\textunderscore  + \textunderscore logos\textunderscore )}
\end{itemize}
Tratado scientífico dos crustáceos.
\section{Gammarológico}
\begin{itemize}
\item {Grp. gram.:adj.}
\end{itemize}
Relativo á gammarologia.
\section{Gammarólogo}
\begin{itemize}
\item {Grp. gram.:m.}
\end{itemize}
Naturalista, que é perito em gammarologia.
\section{Gamo}
\begin{itemize}
\item {Grp. gram.:m.}
\end{itemize}
\begin{itemize}
\item {Proveniência:(Do lat. \textunderscore dama\textunderscore )}
\end{itemize}
Espécie de veado, que tem achatada a parte superior dos galhos e comprida a cauda.
\section{Gamo}
\begin{itemize}
\item {Grp. gram.:m.}
\end{itemize}
\begin{itemize}
\item {Utilização:Prov.}
\end{itemize}
\begin{itemize}
\item {Utilização:beir.}
\end{itemize}
O mesmo que \textunderscore gomo\textunderscore , divisão natural de certos frutos.
\section{Gamo...}
\begin{itemize}
\item {Grp. gram.:pref.}
\end{itemize}
\begin{itemize}
\item {Proveniência:(Gr. \textunderscore gamos\textunderscore )}
\end{itemize}
(designativo da união ou soldadura de certos órgãos ou partes vegetaes)
\section{Gamofilia}
\begin{itemize}
\item {Grp. gram.:f.}
\end{itemize}
\begin{itemize}
\item {Utilização:Bot.}
\end{itemize}
Carácter do gamofilo.
\section{Gamofilo}
\begin{itemize}
\item {Grp. gram.:adj.}
\end{itemize}
\begin{itemize}
\item {Utilização:Bot.}
\end{itemize}
\begin{itemize}
\item {Proveniência:(Do gr. \textunderscore gamos\textunderscore  + \textunderscore phullon\textunderscore )}
\end{itemize}
Formado pela soldadura de fôlhas.
Que tem fôlhas soldadas umas ás outras.
\section{Gamogastro}
\begin{itemize}
\item {Grp. gram.:adj.}
\end{itemize}
\begin{itemize}
\item {Utilização:Bot.}
\end{itemize}
\begin{itemize}
\item {Proveniência:(Do gr. \textunderscore gammos\textunderscore  + \textunderscore gaster\textunderscore )}
\end{itemize}
Diz-se da flôr, que tem os ovários soldados.
\section{Gamologia}
\begin{itemize}
\item {Grp. gram.:f.}
\end{itemize}
\begin{itemize}
\item {Proveniência:(Do gr. \textunderscore gamos\textunderscore  + \textunderscore logos\textunderscore )}
\end{itemize}
Discurso ou tratado á cêrca do casamento.
\section{Gamomania}
\begin{itemize}
\item {Grp. gram.:f.}
\end{itemize}
\begin{itemize}
\item {Proveniência:(Do gr. \textunderscore gamos\textunderscore  + \textunderscore mania\textunderscore )}
\end{itemize}
Loucura, caracterizada pela monomania do casamento.
\section{Gamomaníaco}
\begin{itemize}
\item {Grp. gram.:m.}
\end{itemize}
Aquelle que soffre gamomania.
\section{Gamopetalia}
\begin{itemize}
\item {Grp. gram.:f.}
\end{itemize}
\begin{itemize}
\item {Utilização:Bot.}
\end{itemize}
Estado da corolla que é gamopétala.
\section{Gamopétalo}
\begin{itemize}
\item {Grp. gram.:adj.}
\end{itemize}
\begin{itemize}
\item {Utilização:Bot.}
\end{itemize}
\begin{itemize}
\item {Proveniência:(Do gr. \textunderscore gamos\textunderscore  + \textunderscore petalon\textunderscore )}
\end{itemize}
Que tem unidas as pétalas.
\section{Gamophyllia}
\begin{itemize}
\item {Grp. gram.:f.}
\end{itemize}
\begin{itemize}
\item {Utilização:Bot.}
\end{itemize}
Carácter do gamophyllo.
\section{Gamophyllo}
\begin{itemize}
\item {Grp. gram.:adj.}
\end{itemize}
\begin{itemize}
\item {Utilização:Bot.}
\end{itemize}
\begin{itemize}
\item {Proveniência:(Do gr. \textunderscore gamos\textunderscore  + \textunderscore phullon\textunderscore )}
\end{itemize}
Formado pela soldadura de fôlhas.
Que tem fôlhas soldadas umas ás outras.
\section{Gamosepalia}
\begin{itemize}
\item {fónica:sé}
\end{itemize}
\begin{itemize}
\item {Grp. gram.:f.}
\end{itemize}
\begin{itemize}
\item {Utilização:Bot.}
\end{itemize}
Carácter ou estado de gamosépalo.
\section{Gamosépalo}
\begin{itemize}
\item {fónica:sé}
\end{itemize}
\begin{itemize}
\item {Grp. gram.:adj.}
\end{itemize}
\begin{itemize}
\item {Utilização:Bot.}
\end{itemize}
\begin{itemize}
\item {Proveniência:(De \textunderscore gamo...\textunderscore  + \textunderscore sépala\textunderscore )}
\end{itemize}
Que tem unidas as sépalas.
\section{Gamossepalia}
\begin{itemize}
\item {Grp. gram.:f.}
\end{itemize}
\begin{itemize}
\item {Utilização:Bot.}
\end{itemize}
Carácter ou estado de gamosépalo.
\section{Gamossépalo}
\begin{itemize}
\item {Grp. gram.:adj.}
\end{itemize}
\begin{itemize}
\item {Utilização:Bot.}
\end{itemize}
\begin{itemize}
\item {Proveniência:(De \textunderscore gamo...\textunderscore  + \textunderscore sépala\textunderscore )}
\end{itemize}
Que tem unidas as sépalas.
\section{Gamostilo}
\begin{itemize}
\item {Grp. gram.:adj.}
\end{itemize}
\begin{itemize}
\item {Utilização:Bot.}
\end{itemize}
\begin{itemize}
\item {Proveniência:(Do gr. \textunderscore gamos\textunderscore  + \textunderscore stule\textunderscore )}
\end{itemize}
Formado pela união de estiletes.
\section{Gamostylo}
\begin{itemize}
\item {Grp. gram.:adj.}
\end{itemize}
\begin{itemize}
\item {Utilização:Bot.}
\end{itemize}
\begin{itemize}
\item {Proveniência:(Do gr. \textunderscore gamos\textunderscore  + \textunderscore stule\textunderscore )}
\end{itemize}
Formado pela união de estiletes.
\section{Gamote}
\begin{itemize}
\item {Grp. gram.:m.}
\end{itemize}
\begin{itemize}
\item {Proveniência:(Do rad. de \textunderscore gamella\textunderscore ^1)}
\end{itemize}
Vasilha de madeira, com que se esgota a água das cavernas, nas pequenas embarcações.
\section{Gamuta}
\begin{itemize}
\item {Grp. gram.:f.}
\end{itemize}
Filamentos, que pendem da base das fôlhas de certas palmeiras e de que se fazem cordas.
(Cp. \textunderscore gamuti\textunderscore )
\section{Gamuti}
\begin{itemize}
\item {Grp. gram.:m.}
\end{itemize}
Arvore indiana, de fibras têxteis.
\section{Gana}
\begin{itemize}
\item {Grp. gram.:f.}
\end{itemize}
\begin{itemize}
\item {Utilização:Prov.}
\end{itemize}
\begin{itemize}
\item {Utilização:minh.}
\end{itemize}
Grande appetite ou vontade.
Desejo de fazer mal.
Fome.
Ramo grande de árvore. (Colhido em Barcellos)
\section{Ganacha}
\begin{itemize}
\item {Grp. gram.:f.}
\end{itemize}
\begin{itemize}
\item {Proveniência:(Do it. \textunderscore ganacia\textunderscore )}
\end{itemize}
Maxilla inferior do cavallo.
Bordo posterior da face dos solípedes, que tem por base o bordo arredondado do osso maxillar inferior.
\section{Ganadeiro}
\begin{itemize}
\item {Grp. gram.:m.}
\end{itemize}
\begin{itemize}
\item {Utilização:Prov.}
\end{itemize}
\begin{itemize}
\item {Utilização:Prov.}
\end{itemize}
\begin{itemize}
\item {Utilização:trasm.}
\end{itemize}
Designação genérica dos guardadores de gado, porqueiros, vaqueiros, eguariços e ovelheiros.
Especialmente o possuidor de gado lanígero.
(Cast. \textunderscore ganadero\textunderscore )
\section{Ganância}
\begin{itemize}
\item {Grp. gram.:f.}
\end{itemize}
\begin{itemize}
\item {Utilização:Ext.}
\end{itemize}
O mesmo que \textunderscore ganho\textunderscore .
Ganho illícito.
Ambição.
(B. lat. \textunderscore ganantia\textunderscore )
\section{Ganancioso}
\begin{itemize}
\item {Grp. gram.:adj.}
\end{itemize}
\begin{itemize}
\item {Proveniência:(De \textunderscore ganância\textunderscore )}
\end{itemize}
Em que há lucro.
Útil.
Relativo a lucro exaggerado.
Que só tem em mira o lucro, lícito ou illícito: \textunderscore homem ganancioso\textunderscore .
\section{Ganapa}
\begin{itemize}
\item {Grp. gram.:f.}
\end{itemize}
\begin{itemize}
\item {Utilização:T. do Fundão}
\end{itemize}
Rapariga desavergonhada e de maus costumes.
(Relaciona-se com o cast. \textunderscore gañapano\textunderscore ?)
\section{Ganapão}
\begin{itemize}
\item {Grp. gram.:m.}
\end{itemize}
\begin{itemize}
\item {Utilização:Prov.}
\end{itemize}
\begin{itemize}
\item {Utilização:trasm.}
\end{itemize}
Pequena rêde, no extremo de uma vara, para apanhar a sardinha que cái á água depois de desemmalhada.
Trabalhador assalariado, ganhapão.
(Cast. \textunderscore ganapano\textunderscore  = port. \textunderscore ganhapão\textunderscore )
\section{Ganapé}
\begin{itemize}
\item {Grp. gram.:m.}
\end{itemize}
\begin{itemize}
\item {Utilização:Ant.}
\end{itemize}
Espécie de manta para cobrir os pés.
Espécie de cobertor.
\section{Ganau}
\begin{itemize}
\item {Grp. gram.:m.}
\end{itemize}
\begin{itemize}
\item {Utilização:Pleb.}
\end{itemize}
Piolho ladro, chato.
\section{Gança}
\begin{itemize}
\item {Grp. gram.:f.}
\end{itemize}
\begin{itemize}
\item {Utilização:Ant.}
\end{itemize}
O mesmo que \textunderscore alimpadura\textunderscore .
O mesmo que \textunderscore ganância\textunderscore .
\section{Gança}
\begin{itemize}
\item {Grp. gram.:f.}
\end{itemize}
\begin{itemize}
\item {Utilização:Ant.}
\end{itemize}
Meretriz.
(Da mesma or. que \textunderscore gança\textunderscore ^1? Ou por \textunderscore gansa\textunderscore , de \textunderscore ganso\textunderscore ?)
\section{Gançar}
\begin{itemize}
\item {Grp. gram.:v. t.}
\end{itemize}
\begin{itemize}
\item {Utilização:Ant.}
\end{itemize}
\begin{itemize}
\item {Utilização:Prov.}
\end{itemize}
\begin{itemize}
\item {Utilização:alent.}
\end{itemize}
\begin{itemize}
\item {Proveniência:(De \textunderscore gança\textunderscore ^1)}
\end{itemize}
O mesmo que \textunderscore ganhar\textunderscore .
Caçar.
Dar.
\section{Gancar}
\begin{itemize}
\item {Grp. gram.:m.}
\end{itemize}
Cultivador de terras bravias na Índia portuguesa.
Cobrador de rendas, na Índia.
(Do conc.)
\section{Gancaria}
\begin{itemize}
\item {Grp. gram.:f.}
\end{itemize}
Assembleia de gancares.
Offício de gancar.
Aldeia ou terras, comprehendidas na jurisdicção de um gancar.
\section{Gancha}
\begin{itemize}
\item {Grp. gram.:f.}
\end{itemize}
\begin{itemize}
\item {Utilização:Prov.}
\end{itemize}
\begin{itemize}
\item {Utilização:trasm.}
\end{itemize}
\begin{itemize}
\item {Utilização:Prov.}
\end{itemize}
\begin{itemize}
\item {Utilização:minh.}
\end{itemize}
\begin{itemize}
\item {Grp. gram.:Adj. f.}
\end{itemize}
\begin{itemize}
\item {Proveniência:(De \textunderscore gancho\textunderscore )}
\end{itemize}
Gadanha para o feno.
Ramo de árvore; pernada.
O mesmo que \textunderscore gâmia\textunderscore .
\section{Ganchar}
\begin{itemize}
\item {Grp. gram.:v. t.}
\end{itemize}
Agarrar com gancho.
\section{Gancharra}
\begin{itemize}
\item {Grp. gram.:f.}
\end{itemize}
\begin{itemize}
\item {Utilização:T. da Bairrada}
\end{itemize}
O mesmo que \textunderscore garrancha\textunderscore .
\section{Gancheado}
\begin{itemize}
\item {Grp. gram.:adj.}
\end{itemize}
Que tem fórma de gancho.
\section{Ganchear}
\begin{itemize}
\item {Grp. gram.:v. i.}
\end{itemize}
\begin{itemize}
\item {Utilização:Pop.}
\end{itemize}
\begin{itemize}
\item {Proveniência:(De \textunderscore gancho\textunderscore )}
\end{itemize}
Realizar um pequeno trabalho extraordinário; fazer biscates.
\section{Gancheta}
\begin{itemize}
\item {fónica:chê}
\end{itemize}
\begin{itemize}
\item {Grp. gram.:f.}
\end{itemize}
\begin{itemize}
\item {Utilização:Prov.}
\end{itemize}
\begin{itemize}
\item {Utilização:minh.}
\end{itemize}
Pequeno gancho no estremo de uma vara, para se pendurarem alguns apparelhos de pesca.
\section{Gancheta}
\begin{itemize}
\item {fónica:chê}
\end{itemize}
\begin{itemize}
\item {Grp. gram.:f.}
\end{itemize}
\begin{itemize}
\item {Utilização:Prov.}
\end{itemize}
\begin{itemize}
\item {Utilização:alent.}
\end{itemize}
Instrumento de estucador, para aperfeiçoar os ornatos.
\section{Ganchinho}
\begin{itemize}
\item {Grp. gram.:m.}
\end{itemize}
\begin{itemize}
\item {Utilização:Pop.}
\end{itemize}
\begin{itemize}
\item {Proveniência:(De \textunderscore gancho\textunderscore )}
\end{itemize}
Trabalho eventual, fóra das horas habituaes de serviço.
\section{Gancho}
\begin{itemize}
\item {Grp. gram.:m.}
\end{itemize}
\begin{itemize}
\item {Utilização:Pop.}
\end{itemize}
\begin{itemize}
\item {Utilização:Prov.}
\end{itemize}
\begin{itemize}
\item {Utilização:trasm.}
\end{itemize}
\begin{itemize}
\item {Utilização:Prov.}
\end{itemize}
\begin{itemize}
\item {Utilização:trasm.}
\end{itemize}
\begin{itemize}
\item {Utilização:Prov.}
\end{itemize}
Peça curva de metal ou de outra substância resistente, para suspender quaesquer pesos.
Anzol.
Arame curvo, ou utensílio semelhante, com que as mulheres seguram o cabello.
Qualquer objecto em fórma de gancho.
Gratificação por serviço extraordinário.
Serviço extraordinário, pequeno serviço, biscate.
Ramo de árvore.
Cada uma das duas extremidades do gume da enxada, quando o gume é cavado no meio.
Ancinho, de dentes de ferro, para carregar e descarregar estrume, estender mato para formar estrumeiras, etc.
(Cast. \textunderscore gancho\textunderscore . Do ár. \textunderscore gondj\textunderscore ?)
\section{Ganchorra}
\begin{itemize}
\item {fónica:chô}
\end{itemize}
\begin{itemize}
\item {Grp. gram.:f.}
\end{itemize}
\begin{itemize}
\item {Utilização:Gír.}
\end{itemize}
Grande gancho para atracar os barcos.
Mão.
\section{Ganchoso}
\begin{itemize}
\item {Grp. gram.:adj.}
\end{itemize}
Curvo como um gancho.
\section{Ganço}
\begin{itemize}
\item {Grp. gram.:m.}
\end{itemize}
\begin{itemize}
\item {Utilização:Ant.}
\end{itemize}
O mesmo que \textunderscore ganho\textunderscore .
(Cp. \textunderscore gança\textunderscore ^1)
\section{Ganda}
\begin{itemize}
\item {Grp. gram.:f.}
\end{itemize}
Nome do rhinoceronte, na Índia portuguesa.
\section{Gandaeiro}
\begin{itemize}
\item {Grp. gram.:m.}
\end{itemize}
\begin{itemize}
\item {Proveniência:(De \textunderscore gandaia\textunderscore )}
\end{itemize}
Aquelle que anda á gandaia.
Trapeiro.
Tunante; vadio.
\section{Gandaia}
\begin{itemize}
\item {Grp. gram.:f.}
\end{itemize}
\begin{itemize}
\item {Utilização:Gír.}
\end{itemize}
Acto de procurar no lixo quaesquer objectos que tinham algum valor.
Profissão de trapeiro.
Vadiagem; mandriice.
(Cast. \textunderscore gandaya\textunderscore )
\section{Gandaiar}
\begin{itemize}
\item {Grp. gram.:v. i.}
\end{itemize}
Andar á gandaia; andar á tuna; vadiar.
\section{Gandaíce}
\begin{itemize}
\item {Grp. gram.:f.}
\end{itemize}
\begin{itemize}
\item {Proveniência:(De \textunderscore gandaia\textunderscore )}
\end{itemize}
Modos ou acção de gandaeiro. Cf. Camillo, \textunderscore Quéda de Um Anjo\textunderscore , 57; \textunderscore Mulher Fatal\textunderscore , 30.
\section{Gandaieiro}
\begin{itemize}
\item {Grp. gram.:m.}
\end{itemize}
O mesmo ou melhor que \textunderscore gandaeiro\textunderscore .
\section{Gandaio}
\begin{itemize}
\item {Grp. gram.:m.}
\end{itemize}
\begin{itemize}
\item {Utilização:Prov.}
\end{itemize}
\begin{itemize}
\item {Utilização:trasm.}
\end{itemize}
Indivíduo alto.
\section{Gândara}
\begin{itemize}
\item {Grp. gram.:f.}
\end{itemize}
\begin{itemize}
\item {Utilização:Prov.}
\end{itemize}
\begin{itemize}
\item {Utilização:trasm.}
\end{itemize}
\begin{itemize}
\item {Proveniência:(Do b. lat. \textunderscore gandera\textunderscore )}
\end{itemize}
Terreno areoso, pouco productivo, ou estéril.
Terreno despovoado, mas coberto de plantas agrestes.
Pedaço de esteva sêca, que o gado vai tombando pelo monte ou que ficou em as boiças, depois de arder o mato.
\section{Gandares}
\begin{itemize}
\item {Grp. gram.:m.}
\end{itemize}
Pano de algodão com listras azues, usado na Índia ou na África.
\section{Gandarês}
\begin{itemize}
\item {Grp. gram.:adj.}
\end{itemize}
\begin{itemize}
\item {Utilização:Prov.}
\end{itemize}
\begin{itemize}
\item {Utilização:trasm.}
\end{itemize}
\begin{itemize}
\item {Grp. gram.:M.}
\end{itemize}
\begin{itemize}
\item {Utilização:T. de Coímbra}
\end{itemize}
Que habita em gândara.
Relativo a gândara.
\textunderscore Sapo gandarês\textunderscore , sapo grande.
Camponês dos arredores de Coímbra.
\section{Gandaru}
\begin{itemize}
\item {Grp. gram.:m.}
\end{itemize}
Árvore americana, de madeira avermelhada e rija.
\section{Gando}
\begin{itemize}
\item {Grp. gram.:m.}
\end{itemize}
\begin{itemize}
\item {Utilização:Gír.}
\end{itemize}
\begin{itemize}
\item {Utilização:Prov.}
\end{itemize}
\begin{itemize}
\item {Utilização:minh.}
\end{itemize}
Piolho.
O mesmo que \textunderscore gado\textunderscore ^1.
\section{Gandra}
\begin{itemize}
\item {Grp. gram.:f.}
\end{itemize}
(V.gândara)
\section{Gandrês}
\begin{itemize}
\item {Grp. gram.:m.  e  adj.}
\end{itemize}
O mesmo que \textunderscore gandarês\textunderscore .
\section{Gandula}
\begin{itemize}
\item {Grp. gram.:m.}
\end{itemize}
\begin{itemize}
\item {Utilização:T. de Gaia}
\end{itemize}
O mesmo que \textunderscore gandulo\textunderscore .
\section{Gandular}
\begin{itemize}
\item {Grp. gram.:v. i.}
\end{itemize}
\begin{itemize}
\item {Utilização:T. de Gaia}
\end{itemize}
Têr vida de gandulo.
Vadiar.
\section{Gandulo}
\begin{itemize}
\item {Grp. gram.:m.}
\end{itemize}
\begin{itemize}
\item {Utilização:Prov.}
\end{itemize}
\begin{itemize}
\item {Utilização:T. de Lanhoso}
\end{itemize}
Garoto.
Vadio.
Tratante.
Aquelle que come do alheio e guarda o que é seu.
(Cast. \textunderscore gandul\textunderscore )
\section{Gandum}
\begin{itemize}
\item {Grp. gram.:m.}
\end{itemize}
\begin{itemize}
\item {Utilização:Ant.}
\end{itemize}
Gandul:«\textunderscore ...agora que sou o gandum da preguiça...\textunderscore »\textunderscore Anat. Joc.\textunderscore , I, 195.
\section{Ganeira}
\begin{itemize}
\item {Grp. gram.:f.}
\end{itemize}
Ramo grande de árvore.
Ganeiro.
Gana.
\section{Ganeiro}
\begin{itemize}
\item {Grp. gram.:m.}
\end{itemize}
\begin{itemize}
\item {Proveniência:(T. as.)}
\end{itemize}
Aquelle que, na marinha asiática, tem a seu cargo os petrechos de guerra, massame, poleame, etc.
\section{Ganez}
\begin{itemize}
\item {Grp. gram.:m.}
\end{itemize}
Divindade familiar na Índia. Cf. Th. Ribeiro, \textunderscore Jornadas\textunderscore , II, 76.
\section{Ganfar}
\begin{itemize}
\item {Grp. gram.:v. t.}
\end{itemize}
\begin{itemize}
\item {Utilização:T. da Bairrada}
\end{itemize}
\begin{itemize}
\item {Utilização:Gír.}
\end{itemize}
Agarrar, catrafilar, deitar as unhas a.
Vender.
\section{Ganga}
\begin{itemize}
\item {Grp. gram.:f.}
\end{itemize}
Ave gallinácea, (\textunderscore pterocles\textunderscore ).
O mesmo que \textunderscore cortiçola\textunderscore .
\section{Ganga}
\begin{itemize}
\item {Grp. gram.:f.}
\end{itemize}
Tecido azul ou amarelo, de fabricação italiana, e muito usado entre nós.
\section{Ganga}
\begin{itemize}
\item {Grp. gram.:m.}
\end{itemize}
Sacerdote gentio no Congo.
\section{Ganga}
\begin{itemize}
\item {Grp. gram.:f.}
\end{itemize}
\begin{itemize}
\item {Utilização:Miner.}
\end{itemize}
A parte não metállica dos veios metallíferos, a qual abrange a massa principal do depósito e contém o mineral.
\section{Gangana}
\begin{itemize}
\item {Grp. gram.:f.}
\end{itemize}
\begin{itemize}
\item {Utilização:Bras}
\end{itemize}
Mulher idosa.--É expressão infantil e carinhosa.
\section{Gangão}
\begin{itemize}
\item {Grp. gram.:m. Loc. adv.}
\end{itemize}
\begin{itemize}
\item {Proveniência:(Do al. \textunderscore gang\textunderscore ?)}
\end{itemize}
\textunderscore De gangão\textunderscore , de escantilhão; de corrida.
\section{Gangão}
\begin{itemize}
\item {Grp. gram.:m.}
\end{itemize}
\begin{itemize}
\item {Utilização:Bras}
\end{itemize}
Espiga de milho atrophiada, com poucos grãos.
\section{Gângaras}
\begin{itemize}
\item {Grp. gram.:f. pl. Loc. adv.}
\end{itemize}
\begin{itemize}
\item {Utilização:Prov.}
\end{itemize}
\begin{itemize}
\item {Utilização:trasm.}
\end{itemize}
\textunderscore De gângaras\textunderscore , indolentemente; de má vontade.
\section{Gangarina}
\begin{itemize}
\item {Grp. gram.:f.}
\end{itemize}
\begin{itemize}
\item {Utilização:Gír.}
\end{itemize}
Igreja.
\section{Gangarreão}
\begin{itemize}
\item {Grp. gram.:m.}
\end{itemize}
\begin{itemize}
\item {Utilização:Bras}
\end{itemize}
Alteração mais ou menos forte, na saúde de alguém.
\section{Gangético}
\begin{itemize}
\item {Grp. gram.:adj.}
\end{itemize}
Relativo ao rio Ganges, ou aos povos que o ladeiam.
\section{Ganglião}
\begin{itemize}
\item {Grp. gram.:m.}
\end{itemize}
(V.gânglio)
\section{Gangliforme}
\begin{itemize}
\item {Grp. gram.:adj.}
\end{itemize}
\begin{itemize}
\item {Proveniência:(De \textunderscore ganglio\textunderscore  + \textunderscore forma\textunderscore )}
\end{itemize}
Que tem fórma de gânglio.
\section{Gânglio}
\begin{itemize}
\item {Grp. gram.:m.}
\end{itemize}
\begin{itemize}
\item {Proveniência:(Gr. \textunderscore ganglion\textunderscore )}
\end{itemize}
Pequeno corpo cinzento e arredondado, que se encontra no trajecto dos nervos.
Pequeno corpo, formado pelo entrelaçamento dos vasos lympháticos.
Pequeno tumor duro, que apparece na passagem dos tendões.
Qualquer órgão, de apparência nodosa.
\section{Ganglioma}
\begin{itemize}
\item {Grp. gram.:m.}
\end{itemize}
\begin{itemize}
\item {Proveniência:(Do rad. de \textunderscore gânglio\textunderscore )}
\end{itemize}
Nome, que se deu ao tumor das glândulas ou dos gânglios lympháticos.
\section{Ganglionar}
\begin{itemize}
\item {Grp. gram.:adj.}
\end{itemize}
\begin{itemize}
\item {Proveniência:(Do gr. \textunderscore ganglion\textunderscore )}
\end{itemize}
Relativo a gânglios ou que é da natureza delles.
\section{Ganglionite}
\begin{itemize}
\item {Grp. gram.:f.}
\end{itemize}
\begin{itemize}
\item {Proveniência:(Do gr. \textunderscore ganglion\textunderscore )}
\end{itemize}
Inflammação dos gânglios.
\section{Gango}
\begin{itemize}
\item {Grp. gram.:m.}
\end{itemize}
\begin{itemize}
\item {Utilização:Prov.}
\end{itemize}
O mesmo que \textunderscore mimo\textunderscore ^1, meiguice.
\section{Gangoncu}
\begin{itemize}
\item {Grp. gram.:m.}
\end{itemize}
Árvore palmácea do Brasil, (\textunderscore attalea speciosa\textunderscore ).
\section{Gangorra}
\begin{itemize}
\item {fónica:gô}
\end{itemize}
\begin{itemize}
\item {Grp. gram.:f.}
\end{itemize}
\begin{itemize}
\item {Utilização:Bras}
\end{itemize}
\begin{itemize}
\item {Utilização:Prov.}
\end{itemize}
\begin{itemize}
\item {Utilização:minh.}
\end{itemize}
Apparelho, para divertimento de rapazes, constituído por uma trave, apoiada pelo meio num espigão, e em cujas extremidades cavalgam.
Armadilha, para apanhar animaes bravios.
Curral, em volta da cozinha.
Cambão especial, nos antigos arados de madeira.
\section{Gangorra}
\begin{itemize}
\item {fónica:gô}
\end{itemize}
\begin{itemize}
\item {Grp. gram.:f.}
\end{itemize}
\begin{itemize}
\item {Utilização:Ant.}
\end{itemize}
Espécie de carapuça.
(Por \textunderscore grangorra\textunderscore , de \textunderscore grande\textunderscore  + \textunderscore gorro\textunderscore ?)
\section{Gangosa}
\begin{itemize}
\item {Grp. gram.:f.}
\end{itemize}
O mesmo que \textunderscore gagosa\textunderscore ^2.
\section{Gangoso}
\begin{itemize}
\item {Grp. gram.:adj.}
\end{itemize}
\begin{itemize}
\item {Utilização:Des.}
\end{itemize}
\begin{itemize}
\item {Proveniência:(T. cast.)}
\end{itemize}
O mesmo que \textunderscore fanhoso\textunderscore .
\section{Gangoso}
\begin{itemize}
\item {Grp. gram.:m.}
\end{itemize}
\begin{itemize}
\item {Utilização:Prov.}
\end{itemize}
\begin{itemize}
\item {Proveniência:(De \textunderscore gango\textunderscore )}
\end{itemize}
O mesmo que \textunderscore mimalho\textunderscore .
\section{Gangrena}
\begin{itemize}
\item {Grp. gram.:f.}
\end{itemize}
\begin{itemize}
\item {Utilização:Fig.}
\end{itemize}
\begin{itemize}
\item {Proveniência:(Lat. \textunderscore gangraena\textunderscore )}
\end{itemize}
Extincção completa da vida orgânica, em qualquer parte molle do corpo, com tendência a propagar-se nas partes vizinhas.
Aquillo que produz destruição.
Desmoralização.
\section{Gangrenar}
\begin{itemize}
\item {Grp. gram.:v. t.}
\end{itemize}
\begin{itemize}
\item {Utilização:Fig.}
\end{itemize}
\begin{itemize}
\item {Grp. gram.:V. i.}
\end{itemize}
Produzir gangrena em.
Perverter: \textunderscore as más leituras gangrenam a mocidade\textunderscore .
Tornar-se gangrenoso.
\section{Gangrenoso}
\begin{itemize}
\item {Grp. gram.:adj.}
\end{itemize}
Que tem gangrena.
Que é da natureza da gangrena.
\section{Ganguela}
\begin{itemize}
\item {Grp. gram.:m.}
\end{itemize}
\begin{itemize}
\item {Grp. gram.:Pl.}
\end{itemize}
Uma das três línguas, faladas no Baroce, em África.
Povo indígena da África central.
\section{Gângula}
\begin{itemize}
\item {Grp. gram.:f.}
\end{itemize}
Ave pernalta da África oriental, (\textunderscore tantalus ibis\textunderscore ).
\section{Ganha}
\begin{itemize}
\item {Grp. gram.:f.}
\end{itemize}
\begin{itemize}
\item {Utilização:Des.}
\end{itemize}
O mesmo que \textunderscore ganho\textunderscore . Cf. João Ribeiro, \textunderscore Selecta Cláss.\textunderscore , 238.
\section{Ganhaço}
\begin{itemize}
\item {Grp. gram.:m.}
\end{itemize}
\begin{itemize}
\item {Utilização:Pop.}
\end{itemize}
O mesmo que \textunderscore ganhuça\textunderscore .
\section{Ganhadeiro}
\begin{itemize}
\item {Grp. gram.:m.  e  adj.}
\end{itemize}
\begin{itemize}
\item {Utilização:Pop.}
\end{itemize}
\begin{itemize}
\item {Proveniência:(De \textunderscore ganhar\textunderscore )}
\end{itemize}
O que tira lucros.
O que trabalha para ganhar.
Ganhão; jornaleiro.
\section{Ganhadia}
\begin{itemize}
\item {Grp. gram.:f.}
\end{itemize}
\begin{itemize}
\item {Utilização:Ant.}
\end{itemize}
\begin{itemize}
\item {Proveniência:(De \textunderscore ganhar\textunderscore )}
\end{itemize}
Ganho; acquisição.
\section{Ganhadiço}
\begin{itemize}
\item {Grp. gram.:adj.}
\end{itemize}
\begin{itemize}
\item {Utilização:T. de Turquel}
\end{itemize}
O mesmo que \textunderscore bastardo\textunderscore , filho illegítimo.
\section{Ganha-dinheiro}
\begin{itemize}
\item {Grp. gram.:m.}
\end{itemize}
Trabalhador.
Aquelle que, sem têr offício, se emprega em qualquer trabalho material.
\section{Ganhador}
\begin{itemize}
\item {Grp. gram.:adj.}
\end{itemize}
\begin{itemize}
\item {Grp. gram.:M.}
\end{itemize}
\begin{itemize}
\item {Utilização:Bras}
\end{itemize}
Que ganha.
Jornaleiro; trabalhador.
Aquelle que ganha.
Aquelle que ganha para o seu senhor.
\section{Ganhamento}
\begin{itemize}
\item {Grp. gram.:m.}
\end{itemize}
\begin{itemize}
\item {Proveniência:(De \textunderscore ganhar\textunderscore )}
\end{itemize}
O mesmo que \textunderscore ganho\textunderscore .
\section{Ganhança}
\begin{itemize}
\item {Grp. gram.:f.}
\end{itemize}
\begin{itemize}
\item {Utilização:Pop.}
\end{itemize}
\begin{itemize}
\item {Proveniência:(De \textunderscore ganhar\textunderscore )}
\end{itemize}
O mesmo que \textunderscore ganho\textunderscore .
\section{Ganhão}
\begin{itemize}
\item {Grp. gram.:m.}
\end{itemize}
\begin{itemize}
\item {Utilização:Prov.}
\end{itemize}
\begin{itemize}
\item {Utilização:beir.}
\end{itemize}
\begin{itemize}
\item {Utilização:Prov.}
\end{itemize}
\begin{itemize}
\item {Utilização:alent.}
\end{itemize}
Aquelle que vive do seu trabalho.
Aquelle que, para viver, lança mão de qualquer trabalho.
Criado de lavoira.
Trabalhador de lavoira, ceifeiro, mondador, etc.
\section{Ganha-pão}
\begin{itemize}
\item {Grp. gram.:m.}
\end{itemize}
Objecto, com cujo auxílio se adquirem os meios de subsistência: \textunderscore a agulha é o ganha-pão da costureira\textunderscore .
Ganhador, ganhão.
\section{Ganha-perde}
\begin{itemize}
\item {Grp. gram.:m.}
\end{itemize}
Jôgo, em que o ganho é para quem primeiro perde.
\section{Ganhar}
\begin{itemize}
\item {Grp. gram.:v. t.}
\end{itemize}
\begin{itemize}
\item {Utilização:Pop.}
\end{itemize}
\begin{itemize}
\item {Grp. gram.:V. i.}
\end{itemize}
Adquirir a posse de.
Lucrar.
Tirar como proveito: \textunderscore ganhar dinheiro\textunderscore .
Alcançar (vantagens): \textunderscore ganhar um prêmio\textunderscore .
Adquirir no jôgo.
Adquirir, como qualidade: \textunderscore êste vinho ganhou um pique\textunderscore .
Receber como consequência.
Dar proveito a.
Criar.
Apoderar-se de.
Attingir: \textunderscore o andarilho ganhou a meta\textunderscore .
Captar.
Tirar ganho ou vantagem.
Aumentar em crédito.
Levar vantagem.
(B. lat. \textunderscore ganeare\textunderscore )
\section{Ganharia}
\begin{itemize}
\item {Grp. gram.:f.}
\end{itemize}
\begin{itemize}
\item {Utilização:Prov.}
\end{itemize}
\begin{itemize}
\item {Utilização:alent.}
\end{itemize}
\begin{itemize}
\item {Proveniência:(De \textunderscore ganhão\textunderscore )}
\end{itemize}
Os ganhões.
Casa, onde se reúnem e dormem os ganhões, á semelhança de soldados em caserna.
\section{Ganhável}
\begin{itemize}
\item {Grp. gram.:adj.}
\end{itemize}
Que se póde ganhar.
\section{Ganha-vida}
\begin{itemize}
\item {Grp. gram.:m.}
\end{itemize}
O mesmo que \textunderscore ganha-pão\textunderscore . Cf. Filinto, XIII, 201.
\section{Ganho}
\begin{itemize}
\item {Grp. gram.:m.}
\end{itemize}
\begin{itemize}
\item {Grp. gram.:Adj.}
\end{itemize}
\begin{itemize}
\item {Proveniência:(De \textunderscore ganhar\textunderscore )}
\end{itemize}
Acto ou effeito de ganhar.
Lucro, vantagem.
Que se ganhou; que se adquiriu: \textunderscore dinheiro ganho\textunderscore .
\section{Ganhó}
\begin{itemize}
\item {Grp. gram.:m.}
\end{itemize}
\begin{itemize}
\item {Utilização:Prov.}
\end{itemize}
\begin{itemize}
\item {Utilização:trasm.}
\end{itemize}
\begin{itemize}
\item {Grp. gram.:F.}
\end{itemize}
O mesmo que \textunderscore galinhó\textunderscore .
Guelra, pescoço.
(Contr. de \textunderscore galinhó\textunderscore )
\section{Ganhoso}
\begin{itemize}
\item {Grp. gram.:adj.}
\end{itemize}
\begin{itemize}
\item {Proveniência:(De \textunderscore ganho\textunderscore )}
\end{itemize}
Que só pensa em ganhos.
Ambicioso; interesseiro.
\section{Ganhóto}
\begin{itemize}
\item {Grp. gram.:m.}
\end{itemize}
\begin{itemize}
\item {Utilização:Prov.}
\end{itemize}
\begin{itemize}
\item {Utilização:trasm.}
\end{itemize}
Seixo redondo e liso, rolado pelas águas.
Inchaço redondo e rijo.
\section{Ganhôto}
\begin{itemize}
\item {Grp. gram.:m.}
\end{itemize}
\begin{itemize}
\item {Utilização:Prov.}
\end{itemize}
\begin{itemize}
\item {Utilização:alg.}
\end{itemize}
Rebento fraco de figueiras, que se poda no inverno, para que se avigorem os lanços fructíferos.
(Por \textunderscore galhôto\textunderscore , de \textunderscore galho\textunderscore ?)
\section{Ganhuça}
\begin{itemize}
\item {Grp. gram.:f.}
\end{itemize}
\begin{itemize}
\item {Utilização:Fam.}
\end{itemize}
\begin{itemize}
\item {Proveniência:(De \textunderscore ganho\textunderscore )}
\end{itemize}
O mesmo que \textunderscore ganho\textunderscore .
\section{Ganhunça}
\begin{itemize}
\item {Grp. gram.:f.}
\end{itemize}
\begin{itemize}
\item {Utilização:Fam.}
\end{itemize}
O mesmo que \textunderscore ganhuça\textunderscore .
\section{Ganicará}
\begin{itemize}
\item {Grp. gram.:m.}
\end{itemize}
\begin{itemize}
\item {Utilização:T. da Índia Portuguesa}
\end{itemize}
Moínho de azeite.
\section{Ganiços}
\begin{itemize}
\item {Grp. gram.:m. pl.}
\end{itemize}
\begin{itemize}
\item {Utilização:ant.}
\end{itemize}
\begin{itemize}
\item {Utilização:Gír.}
\end{itemize}
\begin{itemize}
\item {Proveniência:(Do cast. \textunderscore ganar\textunderscore )}
\end{itemize}
Dados.
\section{Ganideira}
\begin{itemize}
\item {Grp. gram.:f.}
\end{itemize}
Muitos ganidos.
\section{Ganido}
\begin{itemize}
\item {Grp. gram.:m.}
\end{itemize}
\begin{itemize}
\item {Utilização:Fig.}
\end{itemize}
\begin{itemize}
\item {Proveniência:(Lat. \textunderscore gannitus\textunderscore )}
\end{itemize}
Grito doloroso dos cães.
Voz esganiçada.
\section{Ganir}
\begin{itemize}
\item {Grp. gram.:v. i.}
\end{itemize}
\begin{itemize}
\item {Proveniência:(Lat. \textunderscore gannire\textunderscore )}
\end{itemize}
Dar ganidos.
Gemer; gemicar.
Gemer como os cães.
\section{Ganirra}
\begin{itemize}
\item {Grp. gram.:f.}
\end{itemize}
\begin{itemize}
\item {Utilização:Prov.}
\end{itemize}
\begin{itemize}
\item {Utilização:trasm.}
\end{itemize}
Mulher muito reles.
Coisa que não presta para nada.
\section{Ganivete}
\begin{itemize}
\item {Grp. gram.:m.}
\end{itemize}
\begin{itemize}
\item {Utilização:Prov.}
\end{itemize}
\begin{itemize}
\item {Utilização:beir.}
\end{itemize}
O mesmo que \textunderscore canivete\textunderscore ^1.
\section{Ganizar}
\begin{itemize}
\item {Grp. gram.:v. i.}
\end{itemize}
\begin{itemize}
\item {Utilização:Des.}
\end{itemize}
O mesmo que \textunderscore ganir\textunderscore , (falando-se de um cão pequeno).
\section{Ganizes}
\begin{itemize}
\item {Grp. gram.:m. pl.}
\end{itemize}
Peças de osso, que servem num jôgo de rapazes chamado cucarne.
(Da mesma or. que \textunderscore ganiços\textunderscore )
\section{Ganja}
\begin{itemize}
\item {Grp. gram.:f.}
\end{itemize}
Resina de uma espécie de cânhamo, (\textunderscore cannabis indica\textunderscore ).
\section{Ganja}
\begin{itemize}
\item {Grp. gram.:f.}
\end{itemize}
\begin{itemize}
\item {Utilização:Bras}
\end{itemize}
\begin{itemize}
\item {Grp. gram.:Adj.}
\end{itemize}
\begin{itemize}
\item {Utilização:Ant.}
\end{itemize}
Vaidade; presumpção.
O mesmo que \textunderscore ganjento\textunderscore .
Muito confiado; que toma liberdades, que se lhe não dão.
\section{Ganja}
\begin{itemize}
\item {Grp. gram.:f.}
\end{itemize}
Árvore angolense de Caconda.
\section{Ganjento}
\begin{itemize}
\item {Grp. gram.:adj.}
\end{itemize}
\begin{itemize}
\item {Utilização:Bras}
\end{itemize}
\begin{itemize}
\item {Proveniência:(De \textunderscore ganja\textunderscore ^2)}
\end{itemize}
Vaidoso, presumido.
\section{Gannideira}
\begin{itemize}
\item {Grp. gram.:f.}
\end{itemize}
Muitos ganidos.
\section{Gannido}
\begin{itemize}
\item {Grp. gram.:m.}
\end{itemize}
\begin{itemize}
\item {Utilização:Fig.}
\end{itemize}
\begin{itemize}
\item {Proveniência:(Lat. \textunderscore gannitus\textunderscore )}
\end{itemize}
Grito doloroso dos cães.
Voz esganiçada.
\section{Gannir}
\begin{itemize}
\item {Grp. gram.:v. i.}
\end{itemize}
\begin{itemize}
\item {Proveniência:(Lat. \textunderscore gannire\textunderscore )}
\end{itemize}
Dar gannidos.
Gemer; gemicar.
Gemer como os cães.
\section{Ganó}
\begin{itemize}
\item {Grp. gram.:m.}
\end{itemize}
Engenho de açúcar, na Índia portuguesa.
\section{Gano}
\begin{itemize}
\item {Grp. gram.:m.}
\end{itemize}
\begin{itemize}
\item {Utilização:Prov.}
\end{itemize}
\begin{itemize}
\item {Utilização:minh.}
\end{itemize}
Ramo de árvore; gancha; gana.
\section{Ganoga}
\begin{itemize}
\item {Grp. gram.:f.}
\end{itemize}
Nome de um peixe.
(Cp. \textunderscore ganoides\textunderscore )
\section{Ganoídeos}
\begin{itemize}
\item {Grp. gram.:m. pl.}
\end{itemize}
O mesmo que \textunderscore ganoides\textunderscore .
\section{Ganoides}
\begin{itemize}
\item {Grp. gram.:m. pl.}
\end{itemize}
\begin{itemize}
\item {Proveniência:(Do gr. \textunderscore ganos\textunderscore  + \textunderscore eidos\textunderscore )}
\end{itemize}
Ordem de peixes, de escamas brilhantes.
\section{Gansar}
\begin{itemize}
\item {Grp. gram.:v. t.}
\end{itemize}
\begin{itemize}
\item {Utilização:Ant.}
\end{itemize}
Ganhar?:«\textunderscore Emperol gansei perende abonda que um de cem\textunderscore ». G. Vicente, I, 139.
(Provavelmente, êrro gráphico, por \textunderscore gançar\textunderscore . V. \textunderscore gançar\textunderscore )
\section{Ganso}
\begin{itemize}
\item {Grp. gram.:m.}
\end{itemize}
\begin{itemize}
\item {Proveniência:(Do alt. al. \textunderscore gans\textunderscore )}
\end{itemize}
Ave palmípede, da fam. dos lamellirostros.
\section{Ganso}
\begin{itemize}
\item {Grp. gram.:m.}
\end{itemize}
Parte externa e posterior da coxa do boi.
\section{Ganso}
\begin{itemize}
\item {Grp. gram.:m.}
\end{itemize}
\begin{itemize}
\item {Utilização:Gír.}
\end{itemize}
Cruzado novo.
\section{Ganta}
\begin{itemize}
\item {Grp. gram.:f.}
\end{itemize}
Antiga medida de Malaca, correspondente a quási 2 litros.
\section{Ganta}
\begin{itemize}
\item {Grp. gram.:f.}
\end{itemize}
(V.ganda)
\section{Ganzepe}
\begin{itemize}
\item {Grp. gram.:m.}
\end{itemize}
Entalhe em madeira, o qual estreita de baixo para cima.
\section{Ganzi}
\begin{itemize}
\item {Grp. gram.:m.}
\end{itemize}
Grande peixe africano. Cf. Serpa Pinto, I, 299.
\section{Gãocar}
\begin{itemize}
\item {Grp. gram.:m.}
\end{itemize}
O mesmo que \textunderscore gancar\textunderscore .
\section{Gaparuvu}
\begin{itemize}
\item {Grp. gram.:m.}
\end{itemize}
Arvore silvestre do Brasil.
\section{Gapeira}
\begin{itemize}
\item {Grp. gram.:f.}
\end{itemize}
\begin{itemize}
\item {Utilização:Prov.}
\end{itemize}
\begin{itemize}
\item {Utilização:minh.}
\end{itemize}
Doença dos bois.
\section{Gapó}
\begin{itemize}
\item {Grp. gram.:m.}
\end{itemize}
O mesmo que \textunderscore igapó\textunderscore .
\section{Gaponga}
\begin{itemize}
\item {Grp. gram.:f.}
\end{itemize}
\begin{itemize}
\item {Utilização:Bras}
\end{itemize}
Processo, com que os indígenas amazónios pescam o tambaqui, imitando a quéda de frutos na água, para attrahir o frugívoro peixe. Cf. \textunderscore Jornal do Comm.\textunderscore , do Rio, de 12-II-901.
\section{Gapuia}
\begin{itemize}
\item {Grp. gram.:f.}
\end{itemize}
\begin{itemize}
\item {Utilização:Bras. do N}
\end{itemize}
\begin{itemize}
\item {Proveniência:(De \textunderscore gapuiar\textunderscore )}
\end{itemize}
Modo de pescar, atravessando o riacho com estacas cravadas a prumo.
\section{Gapuiar}
\begin{itemize}
\item {Grp. gram.:v. i.}
\end{itemize}
\begin{itemize}
\item {Utilização:Bras. do N}
\end{itemize}
\begin{itemize}
\item {Proveniência:(Do guar. \textunderscore igapiar\textunderscore )}
\end{itemize}
Pescar nos baixios ao acaso.
Apanhar camarões nas pequenas lagôas.
Procurar qualquer coisa ao acaso.
Esgotar uma lagôa, para deixar o peixe em sêco.
\section{Gará}
\begin{itemize}
\item {Grp. gram.:m.}
\end{itemize}
Habitação ou bairro pobre de indígenas, na Índia portuguesa.
\section{Garabanho}
\begin{itemize}
\item {Grp. gram.:m.}
\end{itemize}
\begin{itemize}
\item {Utilização:Prov.}
\end{itemize}
\begin{itemize}
\item {Utilização:trasm.}
\end{itemize}
Balde de lata ou de cortiça, encabado num pau, para tirar dos poços água de rega.
Cabaço.
\section{Garabi}
\begin{itemize}
\item {Grp. gram.:m.}
\end{itemize}
\begin{itemize}
\item {Utilização:Artilh.}
\end{itemize}
Escantilhão de ferro, para verificar o contôrno e perfil da figura exterior da peça. Cf. Leoni, \textunderscore Diccion. de Artilh.\textunderscore , inédito.
(Metáth. de \textunderscore gabari\textunderscore )
\section{Garabu}
\begin{itemize}
\item {Grp. gram.:m.}
\end{itemize}
Planta terebinthácea do Brasil.
\section{Garabulha}
\begin{itemize}
\item {Grp. gram.:f.}
\end{itemize}
\begin{itemize}
\item {Grp. gram.:M.}
\end{itemize}
Confusão.
Garatuja.
Homem intriguista.
(Cp. \textunderscore garabulho\textunderscore )
\section{Garabulhento}
\begin{itemize}
\item {Grp. gram.:adj.}
\end{itemize}
Que tem garabulho.
\section{Garabulho}
\begin{itemize}
\item {Grp. gram.:m.}
\end{itemize}
\begin{itemize}
\item {Proveniência:(Do it. \textunderscore garbuglio\textunderscore )}
\end{itemize}
Aspereza.
Garabulha.
\section{Garafunhas}
\begin{itemize}
\item {Grp. gram.:f. pl.}
\end{itemize}
O mesmo que \textunderscore garafunhos\textunderscore .
\section{Garafunhos}
\begin{itemize}
\item {Grp. gram.:m. pl.}
\end{itemize}
(V.gatafunhos)
\section{Garage}
\begin{itemize}
\item {Grp. gram.:f.}
\end{itemize}
\begin{itemize}
\item {Utilização:Gal}
\end{itemize}
\begin{itemize}
\item {Proveniência:(T. fr.)}
\end{itemize}
Armazém ou casa, para recolher automóveis. Cf. \textunderscore Decreto\textunderscore  de 9-II-911.
\section{Garajão}
\begin{itemize}
\item {Grp. gram.:m.}
\end{itemize}
\begin{itemize}
\item {Utilização:Mad}
\end{itemize}
O mesmo que \textunderscore garajau\textunderscore ^1.
\section{Garajau}
\begin{itemize}
\item {Grp. gram.:m.}
\end{itemize}
\begin{itemize}
\item {Utilização:Açor}
\end{itemize}
Ave palmípede aquática, (\textunderscore sterna fluviatilis\textunderscore ).
Nome, que nalguns pontos de Portugal, se dá á gaivina.
Andorinha-do-mar, (\textunderscore hirundo marina\textunderscore ).
\section{Garajau}
\begin{itemize}
\item {Grp. gram.:m.}
\end{itemize}
\begin{itemize}
\item {Utilização:Bras}
\end{itemize}
Espécie de cesto oblongo e fechado, em que se levam gallinhas e outras aves ao mercado.
Apparelho, para conduzir peixe sêco.
\section{Garamanha}
\begin{itemize}
\item {Grp. gram.:f.}
\end{itemize}
\begin{itemize}
\item {Utilização:Prov.}
\end{itemize}
\begin{itemize}
\item {Utilização:minh.}
\end{itemize}
O mesmo que \textunderscore ancinho\textunderscore .
\section{Garamantes}
\begin{itemize}
\item {Grp. gram.:m. pl.}
\end{itemize}
\begin{itemize}
\item {Proveniência:(Lat. \textunderscore garamantes\textunderscore )}
\end{itemize}
Antigos povos do interior da África.
\section{Garamantite}
\begin{itemize}
\item {Grp. gram.:f.}
\end{itemize}
\begin{itemize}
\item {Proveniência:(Lat. \textunderscore garamantites\textunderscore )}
\end{itemize}
Pedra preciosa, hoje desconhecida.
\section{Garambuio}
\begin{itemize}
\item {Grp. gram.:m.}
\end{itemize}
Espécie de ave mexicana.
\section{Garança}
\begin{itemize}
\item {Grp. gram.:f.}
\end{itemize}
\begin{itemize}
\item {Proveniência:(Fr. \textunderscore garance\textunderscore . Cp. b. lat. \textunderscore varantia\textunderscore , por \textunderscore verantia\textunderscore , do lat. \textunderscore verus\textunderscore )}
\end{itemize}
O mesmo que \textunderscore granza\textunderscore , planta tinctórial.
Côr vermelha, produzida pela granza.
\section{Garançar}
\begin{itemize}
\item {Grp. gram.:v. t.}
\end{itemize}
Tingir com garança.
\section{Garanceira}
\begin{itemize}
\item {Grp. gram.:f.}
\end{itemize}
Campo, em que cresce a garança.
\section{Garancina}
\begin{itemize}
\item {Grp. gram.:f.}
\end{itemize}
Substancia còrante, extrahida de garança.
\section{Garanganja}
\begin{itemize}
\item {Grp. gram.:m.}
\end{itemize}
Uma das línguas da África occidental.
\section{Garanhão}
\begin{itemize}
\item {Grp. gram.:m.}
\end{itemize}
\begin{itemize}
\item {Utilização:Chul.}
\end{itemize}
\begin{itemize}
\item {Proveniência:(Do b. lat. \textunderscore waranis\textunderscore )}
\end{itemize}
Cavallo para padreação.
Homem femeeiro.
\section{Garante}
\begin{itemize}
\item {Grp. gram.:m.  e  f.}
\end{itemize}
\begin{itemize}
\item {Proveniência:(De \textunderscore garantir\textunderscore )}
\end{itemize}
Pessôa que garante, que afiança, que se responsabiliza por alguma coisa.
\section{Garantia}
\begin{itemize}
\item {Grp. gram.:f.}
\end{itemize}
\begin{itemize}
\item {Proveniência:(De \textunderscore garante\textunderscore )}
\end{itemize}
Fiança, abonação.
Responsabilidade.
Segurança.
Aquillo que se garante.
Direito.
\section{Garantidor}
\begin{itemize}
\item {Grp. gram.:adj.}
\end{itemize}
Que garante.
\section{Garantir}
\begin{itemize}
\item {Grp. gram.:v. t.}
\end{itemize}
\begin{itemize}
\item {Proveniência:(Fr. \textunderscore garantir\textunderscore )}
\end{itemize}
Abonar, afiançar; tornar seguro.
Affirmar como certo.
Compensar.
Livrar.
\section{Garanvaz}
\begin{itemize}
\item {Grp. gram.:m.}
\end{itemize}
Antigo tecido de duas côres. Cf. Garrett, \textunderscore Romanceiro\textunderscore , II, 132.
\section{Garão}
\begin{itemize}
\item {Grp. gram.:m.}
\end{itemize}
Espécie de gaivina, (\textunderscore sterna cantiaca\textunderscore , Gm.).
\section{Garapa}
\begin{itemize}
\item {Grp. gram.:f.}
\end{itemize}
\begin{itemize}
\item {Utilização:Bras}
\end{itemize}
\begin{itemize}
\item {Utilização:Bras. da Baía}
\end{itemize}
Bebida refrigerante, que se extrai da cana do açúcar.
Qualquer bebida açucarada e refrigerante.
Árvore brasileira, de madeira própria para construcções, (\textunderscore apuleia praecox\textunderscore , Mart.).
\section{Garapaná}
\begin{itemize}
\item {Grp. gram.:m.}
\end{itemize}
(V.carapaná)
\section{Garapeira}
\begin{itemize}
\item {Grp. gram.:f.}
\end{itemize}
\begin{itemize}
\item {Utilização:Bras. de Pernambuco}
\end{itemize}
Telheiro ou baiuca, á beira dos caminhos, onde o viandante se fornece de garapa, e de milho ou capim, para o animal em que monta.
\section{Garapeiro}
\begin{itemize}
\item {Grp. gram.:m.}
\end{itemize}
Aquelle que vende ou prepara garapa.
\section{Garatuja}
\begin{itemize}
\item {Grp. gram.:f.}
\end{itemize}
Esgar; trejeito.
Tolice.
Rabiscos; gatafunhos.
O mesmo que \textunderscore garatusa\textunderscore .
\section{Garatujar}
\begin{itemize}
\item {Grp. gram.:v. t.}
\end{itemize}
\begin{itemize}
\item {Grp. gram.:V. i.}
\end{itemize}
\begin{itemize}
\item {Proveniência:(Do it. \textunderscore grattugiare\textunderscore )}
\end{itemize}
Rabiscar.
Cobrir com garatujas.
Fazer garatujas.
\section{Garatusa}
\begin{itemize}
\item {Grp. gram.:f.}
\end{itemize}
Lôgro, trapaça.
(Cast. \textunderscore garatusa\textunderscore )
\section{Garaúna}
\begin{itemize}
\item {Grp. gram.:f.}
\end{itemize}
Ave do Brasil.
\section{Garavalha}
\begin{itemize}
\item {Grp. gram.:f.}
\end{itemize}
O mesmo que \textunderscore gravalha\textunderscore .
\section{Garavanço}
\begin{itemize}
\item {Grp. gram.:m.}
\end{itemize}
Pequeno forcado de madeira, com que se limpa o trigo nas eiras.
(Cp. \textunderscore gravanço\textunderscore ^2)
\section{Garavano}
\begin{itemize}
\item {Grp. gram.:m.}
\end{itemize}
\begin{itemize}
\item {Utilização:Prov.}
\end{itemize}
\begin{itemize}
\item {Utilização:trasm.}
\end{itemize}
O mesmo que \textunderscore garabanho\textunderscore .
\section{Garavata}
\begin{itemize}
\item {Grp. gram.:f.}
\end{itemize}
\begin{itemize}
\item {Utilização:Des.}
\end{itemize}
O mesmo que \textunderscore gravata\textunderscore . Cf. B. Pereira, \textunderscore Prosódia\textunderscore , vb. \textunderscore focale\textunderscore .
\section{Garavato}
\begin{itemize}
\item {Grp. gram.:m.}
\end{itemize}
\begin{itemize}
\item {Utilização:T. da Bairrada}
\end{itemize}
Pau, com um gancho numa extremidade, para apanhar fruta; cambo; ladra.
É o mesmo que \textunderscore garavêto\textunderscore .
Gancho no dente do arado ou da charrua, para nelle se prender a mãozinha.
(Cast. \textunderscore garabato\textunderscore )
\section{Garavelho}
\begin{itemize}
\item {fónica:vê}
\end{itemize}
\begin{itemize}
\item {Grp. gram.:m.}
\end{itemize}
\begin{itemize}
\item {Utilização:Prov.}
\end{itemize}
\begin{itemize}
\item {Utilização:beir.}
\end{itemize}
O mesmo que \textunderscore garavêto\textunderscore .
Chamiço, (proveniente da limpeza das árvores).
\section{Garavetar}
\begin{itemize}
\item {Grp. gram.:v. i.}
\end{itemize}
Colhêr gravetos.
Apanhar lenha miúda.
\section{Garavêto}
\begin{itemize}
\item {Grp. gram.:m.}
\end{itemize}
\begin{itemize}
\item {Utilização:Gír.}
\end{itemize}
Cavaco, pedaço de lenha miúda.
Maravalha.
Dedo delgado.
(Cp. \textunderscore garavato\textunderscore )
\section{Garavim}
\begin{itemize}
\item {Grp. gram.:m.}
\end{itemize}
\begin{itemize}
\item {Utilização:Ant.}
\end{itemize}
\begin{itemize}
\item {Proveniência:(Do cast. \textunderscore garbin\textunderscore )}
\end{itemize}
Coifa de seda com lavores de oiro e com renda adeante.
\section{Garavotear}
\begin{itemize}
\item {Grp. gram.:v. i.}
\end{itemize}
\begin{itemize}
\item {Utilização:Ant.}
\end{itemize}
Mostrar azáfama.
Andar de um lado para o outro sem descanso.
(Relaciona-se com \textunderscore garavetar\textunderscore ? ou com \textunderscore gaivota\textunderscore ?)
\section{Garavunha}
\begin{itemize}
\item {Grp. gram.:f.}
\end{itemize}
O mesmo que \textunderscore garatuja\textunderscore .
\section{Garbanceira}
\begin{itemize}
\item {Grp. gram.:f.}
\end{itemize}
\begin{itemize}
\item {Utilização:Prov.}
\end{itemize}
\begin{itemize}
\item {Utilização:trasm.}
\end{itemize}
Espécie de roseira brava, applicada em sebes vivas.
(Cp. cast. \textunderscore garbanzo\textunderscore )
\section{Garbo}
\begin{itemize}
\item {Grp. gram.:m.}
\end{itemize}
\begin{itemize}
\item {Proveniência:(Do ant. alt. al. \textunderscore garawi\textunderscore )}
\end{itemize}
Galhardia, donaire, elegância.
Distincção.
Pundonor; bizarria.
\section{Garbosamente}
\begin{itemize}
\item {Grp. gram.:adv.}
\end{itemize}
De modo garboso; com garbo.
\section{Garbosidade}
\begin{itemize}
\item {Grp. gram.:f.}
\end{itemize}
Qualidade de garboso. Cf. Camillo, \textunderscore Estrêl. Fun.\textunderscore , 137.
\section{Garboso}
\begin{itemize}
\item {Grp. gram.:adj.}
\end{itemize}
Que tem garbo.
\section{Garça}
\begin{itemize}
\item {Grp. gram.:f.}
\end{itemize}
\begin{itemize}
\item {Proveniência:(Do b. lat. \textunderscore gartia\textunderscore )}
\end{itemize}
Ave pernalta aquática, (\textunderscore ardea\textunderscore ).
\section{Garça}
\begin{itemize}
\item {Grp. gram.:f.}
\end{itemize}
Tela muito rala. Cf. Júlio Dinis, \textunderscore Serões\textunderscore , 88.
(Cp. \textunderscore gaza\textunderscore )
\section{Garção}
\begin{itemize}
\item {Grp. gram.:m.}
\end{itemize}
\begin{itemize}
\item {Utilização:Ant.}
\end{itemize}
\begin{itemize}
\item {Proveniência:(Fr. \textunderscore garçon\textunderscore , cast. \textunderscore garzón\textunderscore , do b. lat. \textunderscore garcio\textunderscore )}
\end{itemize}
Rapaz:«\textunderscore era um lindo garção, lindo e audaz.\textunderscore »M. Assis, \textunderscore Brás Cubas\textunderscore .
Homem libertino.
\section{Garção}
\begin{itemize}
\item {Grp. gram.:m.}
\end{itemize}
\begin{itemize}
\item {Proveniência:(De \textunderscore garça\textunderscore )}
\end{itemize}
Espécie de garça grande. Cf. \textunderscore Roteiro de Vasco da Gama\textunderscore .
\section{Garceiro}
\begin{itemize}
\item {Grp. gram.:adj.}
\end{itemize}
Que mata garças.
Caçador de garças.
\section{Garcelha}
\begin{itemize}
\item {fónica:cê}
\end{itemize}
\begin{itemize}
\item {Grp. gram.:f.}
\end{itemize}
\begin{itemize}
\item {Utilização:Ant.}
\end{itemize}
O mesmo que \textunderscore carócha\textunderscore .
\section{Garcenho}
\begin{itemize}
\item {Grp. gram.:m.}
\end{itemize}
\begin{itemize}
\item {Proveniência:(De \textunderscore garça\textunderscore )}
\end{itemize}
Ave pernalta, espécie de garça pequena, (\textunderscore ardeola minuta\textunderscore , Lin.).
\section{Garcez}
\begin{itemize}
\item {Grp. gram.:m.}
\end{itemize}
(V.calcês)
\section{Garcilha}
\begin{itemize}
\item {Grp. gram.:f.}
\end{itemize}
Gênero de plantas tiliáceas da Índia portuguesa.
\section{Garcina}
\begin{itemize}
\item {Grp. gram.:f.}
\end{itemize}
\begin{itemize}
\item {Proveniência:(De \textunderscore garça\textunderscore )}
\end{itemize}
Ave marítima. Cf. \textunderscore Roteiro de D. João de Castro\textunderscore , 298, ed. 1882.
\section{Garciote}
\begin{itemize}
\item {Grp. gram.:m.}
\end{itemize}
Espécie de garça, (\textunderscore ardea bubalcus\textunderscore , Sav.).
\section{Garço}
\begin{itemize}
\item {Grp. gram.:adj.}
\end{itemize}
Esverdeado.
Verde-azulado.
\section{Garçôa}
\begin{itemize}
\item {Grp. gram.:f.}
\end{itemize}
\begin{itemize}
\item {Utilização:Ant.}
\end{itemize}
Rapariga.
(Fem. de \textunderscore garção\textunderscore ^1)
\section{Garçolo}
\begin{itemize}
\item {fónica:çô}
\end{itemize}
\begin{itemize}
\item {Grp. gram.:m.}
\end{itemize}
\begin{itemize}
\item {Utilização:Prov.}
\end{itemize}
O mesmo que \textunderscore garcenho\textunderscore .
\section{Garçota}
\begin{itemize}
\item {Grp. gram.:f.}
\end{itemize}
\begin{itemize}
\item {Grp. gram.:Pl.}
\end{itemize}
\begin{itemize}
\item {Utilização:Ext.}
\end{itemize}
\begin{itemize}
\item {Proveniência:(De \textunderscore garça\textunderscore )}
\end{itemize}
Ave palmípede, (\textunderscore ardea gazetta\textunderscore ).
Pluma de garça. Cf. Corvo, \textunderscore Anno na Côrte\textunderscore , I, 34.
Pennas de garça.
Pennacho.
\section{Garde}
\begin{itemize}
\item {Grp. gram.:m.}
\end{itemize}
Unidade monetária no Haiti, correspondente a 900 reis.
\section{Gardênia}
\begin{itemize}
\item {Grp. gram.:f.}
\end{itemize}
\begin{itemize}
\item {Proveniência:(De \textunderscore Garden\textunderscore , n. p.)}
\end{itemize}
Gênero de plantas rubiáceas, a que pertence o \textunderscore jasmim-do-cabo\textunderscore , (\textunderscore gardenia florida\textunderscore , Lin.).
\section{Gardinfantes}
\begin{itemize}
\item {Grp. gram.:m. pl.}
\end{itemize}
\begin{itemize}
\item {Utilização:Ant.}
\end{itemize}
Objecto de ornato feminíno. Cf. Lobo, \textunderscore Auto do Nascimento\textunderscore .
\section{Gardingato}
\begin{itemize}
\item {Grp. gram.:m.}
\end{itemize}
Qualidade de gardingo. Cf. Herculano, \textunderscore Eurico\textunderscore , 21.
\section{Gardingo}
\begin{itemize}
\item {Grp. gram.:m.}
\end{itemize}
Homem nobre da côrte dos Principes visigodos.
(B. lat. \textunderscore gardingus\textunderscore , talvez de \textunderscore garda\textunderscore , guarda)
\section{Gardunha}
\begin{itemize}
\item {Grp. gram.:f.}
\end{itemize}
\begin{itemize}
\item {Utilização:Ant.}
\end{itemize}
O mesmo que \textunderscore gardunho\textunderscore .
\section{Gardunho}
\begin{itemize}
\item {Grp. gram.:m.}
\end{itemize}
\begin{itemize}
\item {Proveniência:(Do rad. do cast. \textunderscore garduña\textunderscore )}
\end{itemize}
O mesmo que \textunderscore fuínha\textunderscore .
\section{Gare}
\begin{itemize}
\item {Grp. gram.:f.}
\end{itemize}
\begin{itemize}
\item {Utilização:Neol.}
\end{itemize}
\begin{itemize}
\item {Proveniência:(Fr. \textunderscore gare\textunderscore , do alt. al. \textunderscore waron\textunderscore )}
\end{itemize}
Parte das estações de caminhos de ferro, onde embarcam ou desembarcam passageiros e mercadorias.--É preferível \textunderscore embarcadoiro\textunderscore  ou \textunderscore caes\textunderscore .
\section{Garecer}
\begin{itemize}
\item {Grp. gram.:v. t.}
\end{itemize}
\begin{itemize}
\item {Utilização:Ant.}
\end{itemize}
O mesmo que \textunderscore guarecer\textunderscore .
\section{Garela}
\begin{itemize}
\item {Grp. gram.:f.}
\end{itemize}
Perdiz, na época do cio.
\section{Garepe}
\begin{itemize}
\item {Grp. gram.:m.}
\end{itemize}
\begin{itemize}
\item {Utilização:Prov.}
\end{itemize}
\begin{itemize}
\item {Utilização:alg.}
\end{itemize}
Caixão sem tampa, feito de paus grossos, para transportar loiça.
\section{Garete}
\begin{itemize}
\item {fónica:garê}
\end{itemize}
\begin{itemize}
\item {Grp. gram.:m.}
\end{itemize}
\begin{itemize}
\item {Utilização:Açor}
\end{itemize}
Peixe pequeno e vivo, preso no anzol, para servir de isca a peixes maiores.
\section{Garfa}
\begin{itemize}
\item {Grp. gram.:f.}
\end{itemize}
\begin{itemize}
\item {Utilização:Prov.}
\end{itemize}
Pequeno enxame de abelhas, garfo.
\section{Garfada}
\begin{itemize}
\item {Grp. gram.:f.}
\end{itemize}
Porção de comida que um garfo levanta de cada vez.
\section{Garfado}
\begin{itemize}
\item {Grp. gram.:m.}
\end{itemize}
\begin{itemize}
\item {Utilização:Prov.}
\end{itemize}
\begin{itemize}
\item {Proveniência:(De \textunderscore garfar\textunderscore )}
\end{itemize}
Garfada.
Braçado, pequena porção, mancheia.
\section{Garfar}
\begin{itemize}
\item {Grp. gram.:v. t.}
\end{itemize}
\begin{itemize}
\item {Utilização:Agr.}
\end{itemize}
Mexer ou rasgar com garfo.
Enxertar de garfo.
\section{Garfeira}
\begin{itemize}
\item {Grp. gram.:f.}
\end{itemize}
Estojo para garfos.
\section{Garfejar}
\begin{itemize}
\item {Grp. gram.:v. i.}
\end{itemize}
\begin{itemize}
\item {Utilização:Prov.}
\end{itemize}
\begin{itemize}
\item {Utilização:trasm.}
\end{itemize}
\begin{itemize}
\item {Proveniência:(De \textunderscore garfo\textunderscore )}
\end{itemize}
Deitar muitos garfos ou muitos colmos, (falando-se de um grão de semente).
\section{Garfete}
\begin{itemize}
\item {fónica:fê}
\end{itemize}
\begin{itemize}
\item {Grp. gram.:m.}
\end{itemize}
\begin{itemize}
\item {Proveniência:(De \textunderscore garfo\textunderscore )}
\end{itemize}
Instrumento cylíndrico de pau ou de vidro, empregado no fabrico da seda.
\section{Garfilha}
\begin{itemize}
\item {Grp. gram.:f.}
\end{itemize}
Orla de medalha ou de moéda.
\section{Garfo}
\begin{itemize}
\item {Grp. gram.:m.}
\end{itemize}
\begin{itemize}
\item {Utilização:Ant.}
\end{itemize}
Utensílio de mesa com três ou quatro dentes, empregado especialmente em levar do prato á boca os pedaços de comida.
Emprega-se também em trabalhos de cozinha.
Forquilha, para separar da palha o trigo.
Enxêrto, renôvo vegetal.
Forquilha, nas rodas da bicycleta.
Cada um de dois pequenos enxames, que emigram juntos de uma colmeia, onde há excesso de população.
Instrumento de tortura.
(Talvez da mesma or. de \textunderscore gafa\textunderscore ^1)
\section{Garfuana}
\begin{itemize}
\item {Grp. gram.:f.}
\end{itemize}
Planta tinctória do Brasil.
\section{Gargaçalada}
\begin{itemize}
\item {Grp. gram.:f.}
\end{itemize}
Acto de despejar com ruído o líquido de uma vasilha de gargalo. Cf. Garrett, \textunderscore Arco de Sant'Anna\textunderscore , I, 78.
(Por \textunderscore gargalaçada\textunderscore , de \textunderscore gargalaçar\textunderscore )
\section{Gargajola}
\begin{itemize}
\item {Grp. gram.:m.}
\end{itemize}
Rapaz espigado, crescido. Cf. Camillo, \textunderscore Brasileira\textunderscore , 24 e 209.
\section{Gargal}
\begin{itemize}
\item {Grp. gram.:m.}
\end{itemize}
\begin{itemize}
\item {Utilização:Prov.}
\end{itemize}
\begin{itemize}
\item {Utilização:trasm.}
\end{itemize}
O mesmo que \textunderscore argal\textunderscore .
\section{Gargalaçar}
\begin{itemize}
\item {Grp. gram.:v. t.}
\end{itemize}
\begin{itemize}
\item {Proveniência:(Do rad. de \textunderscore gargalo\textunderscore )}
\end{itemize}
Beber, metendo na boca o gargalo da vasilha.
\section{Gargaleira}
\begin{itemize}
\item {Grp. gram.:f.}
\end{itemize}
\begin{itemize}
\item {Proveniência:(De \textunderscore gargalo\textunderscore )}
\end{itemize}
Buraco no bojo de pipas, tonéis, etc.
Batoque.
\section{Gargaleiro}
\begin{itemize}
\item {Grp. gram.:adj.}
\end{itemize}
\begin{itemize}
\item {Proveniência:(De \textunderscore gargalo\textunderscore )}
\end{itemize}
Diz-se de um carro para o transporte das uvas vindimadas.
\section{Gargalejo}
\begin{itemize}
\item {Grp. gram.:m.}
\end{itemize}
\begin{itemize}
\item {Utilização:Pop.}
\end{itemize}
O mesmo que \textunderscore gargarejo\textunderscore .
\section{Gargalhada}
\begin{itemize}
\item {Grp. gram.:f.}
\end{itemize}
\begin{itemize}
\item {Proveniência:(De \textunderscore gargalhar\textunderscore )}
\end{itemize}
Risada prolongada e ruidosa.
Cachinada.
\section{Gargalhadear}
\begin{itemize}
\item {Grp. gram.:v. i.}
\end{itemize}
\begin{itemize}
\item {Proveniência:(De \textunderscore gargalhada\textunderscore )}
\end{itemize}
O mesmo que \textunderscore gargalhar\textunderscore .
\section{Gargalhar}
\begin{itemize}
\item {Grp. gram.:v. i.}
\end{itemize}
\begin{itemize}
\item {Proveniência:(De um rad. commum a \textunderscore gargalo\textunderscore , \textunderscore garganta\textunderscore , etc.)}
\end{itemize}
Soltar gargalhadas.
\section{Gargalheira}
\begin{itemize}
\item {Grp. gram.:f.}
\end{itemize}
\begin{itemize}
\item {Utilização:Fig.}
\end{itemize}
Colleira, com que se prendiam os escravos.
Colleira de cão.
Algemas.
Tyrannia, oppressão.
(Por \textunderscore gargaleira\textunderscore , de \textunderscore gargalo\textunderscore )
\section{Gargalho}
\begin{itemize}
\item {Grp. gram.:m.}
\end{itemize}
Escarro grosso, que se expelle com difficuldade.
(Cast. \textunderscore gargajo\textunderscore )
\section{Gargalicho}
\begin{itemize}
\item {Grp. gram.:m.}
\end{itemize}
\begin{itemize}
\item {Utilização:Prov.}
\end{itemize}
\begin{itemize}
\item {Utilização:trasm.}
\end{itemize}
Bica de pedra, por onde corre, ao ar livre, a água para um tanque ou para uma fonte.
(Por \textunderscore gargulicho\textunderscore , de \textunderscore gárgula\textunderscore )
\section{Gargalo}
\begin{itemize}
\item {Grp. gram.:m.}
\end{itemize}
\begin{itemize}
\item {Utilização:Pop.}
\end{itemize}
\begin{itemize}
\item {Utilização:Chul.}
\end{itemize}
Collo, mais ou menos alongado, de garrafa ou de outra vasilha, com entrada estreita.
Viela.
Entrada estreita.
Pescoço.
(Do mesmo rad. que \textunderscore gargalhar\textunderscore )
\section{Gargaludo}
\begin{itemize}
\item {Grp. gram.:m.  e  adj.}
\end{itemize}
\begin{itemize}
\item {Utilização:Prov.}
\end{itemize}
\begin{itemize}
\item {Utilização:beir.}
\end{itemize}
\begin{itemize}
\item {Proveniência:(De \textunderscore gargalo\textunderscore )}
\end{itemize}
O que tem pescoço alto e desairoso.
\section{Garganeiro}
\begin{itemize}
\item {Grp. gram.:adj.}
\end{itemize}
\begin{itemize}
\item {Utilização:T. de Turquel}
\end{itemize}
Que fala muito e á tôa.
(Cp. \textunderscore garganta\textunderscore )
\section{Garganta}
\begin{itemize}
\item {Grp. gram.:f.}
\end{itemize}
\begin{itemize}
\item {Utilização:Ext.}
\end{itemize}
\begin{itemize}
\item {Utilização:Gír.}
\end{itemize}
O mesmo que \textunderscore larynge\textunderscore .
Parte interior do pescoço, por onde os alimentos passam da boca ao estômago.
Pescoço.
Gomo da cana de açúcar.
Abertura estreita.
Desfiladeiro; passagem estreita entre montanhas.
Moldura reentrante.
Voz: \textunderscore aquella cantora tem bôa garganta\textunderscore .
Parte superior de candeeiro, lanterna ou lampada.
A parte posterior do temão do arado e que se fixa ao dente pela teiró e cunha.
Garrafa.
(Cp. \textunderscore gargalhar\textunderscore )
\section{Gargantão}
\begin{itemize}
\item {Grp. gram.:m.  e  adj.}
\end{itemize}
\begin{itemize}
\item {Proveniência:(De \textunderscore garganta\textunderscore )}
\end{itemize}
O que come muito; voraz.
\section{Garganteado}
\begin{itemize}
\item {Grp. gram.:m.}
\end{itemize}
\begin{itemize}
\item {Proveniência:(De \textunderscore gargantear\textunderscore )}
\end{itemize}
Trinado feito com a voz.
\section{Garganteador}
\begin{itemize}
\item {Grp. gram.:m.  e  adj.}
\end{itemize}
O que garganteia.
\section{Gargantear}
\begin{itemize}
\item {Grp. gram.:v. t.}
\end{itemize}
\begin{itemize}
\item {Utilização:Fam.}
\end{itemize}
\begin{itemize}
\item {Grp. gram.:V. i.}
\end{itemize}
\begin{itemize}
\item {Proveniência:(De \textunderscore garganta\textunderscore )}
\end{itemize}
Pronunciar com voz requebrada.
Cantar.
Fazer trinados com a voz.
Cantar, variando ligeiramente os tons.
\section{Garganteio}
\begin{itemize}
\item {Grp. gram.:m.}
\end{itemize}
Acto ou effeito de gargantear.
O mesmo que \textunderscore garganteado\textunderscore .
\section{Garganteira}
\begin{itemize}
\item {Grp. gram.:f.}
\end{itemize}
\begin{itemize}
\item {Utilização:Prov.}
\end{itemize}
\begin{itemize}
\item {Utilização:trasm.}
\end{itemize}
Brio; incentivo.
\section{Gargantilha}
\begin{itemize}
\item {Grp. gram.:f.}
\end{itemize}
Afogador, para ornato do pescoço.
Collar.
(Cast. \textunderscore gargantilla\textunderscore )
\section{Gargantilho}
\begin{itemize}
\item {Grp. gram.:adj.}
\end{itemize}
\begin{itemize}
\item {Utilização:Bras. do S}
\end{itemize}
Diz-se do cavallo, que tem manchas brancas na garganta.
\section{Gargantoíce}
\begin{itemize}
\item {Grp. gram.:f.}
\end{itemize}
\begin{itemize}
\item {Utilização:Des.}
\end{itemize}
\begin{itemize}
\item {Proveniência:(De \textunderscore gargantão\textunderscore )}
\end{itemize}
Abuso de comer.
Gula:«\textunderscore descaro e gargantoíce faz mendigar\textunderscore ». Herculano, \textunderscore Lendas\textunderscore , I, 95.
\section{Gargantosa}
\begin{itemize}
\item {Grp. gram.:f.}
\end{itemize}
\begin{itemize}
\item {Utilização:Gír.}
\end{itemize}
\begin{itemize}
\item {Proveniência:(De \textunderscore garganta\textunderscore )}
\end{itemize}
Garrafa.
\section{Gargar}
\begin{itemize}
\item {Grp. gram.:v. t.}
\end{itemize}
\begin{itemize}
\item {Utilização:Prov.}
\end{itemize}
\begin{itemize}
\item {Utilização:minh.}
\end{itemize}
Branquear (a roupa), deitando funcho na barrela.
Estonar ou tirar a casca verde de (nozes).
\section{Gargarejamento}
\begin{itemize}
\item {Grp. gram.:m.}
\end{itemize}
Acto ou effeito de gargarejar.
\section{Gargarejar}
\begin{itemize}
\item {Grp. gram.:v. t.}
\end{itemize}
\begin{itemize}
\item {Grp. gram.:V. i.}
\end{itemize}
\begin{itemize}
\item {Utilização:Pop.}
\end{itemize}
\begin{itemize}
\item {Proveniência:(Do b. lat. \textunderscore gargaridiare\textunderscore )}
\end{itemize}
Agitar na bôca com o ar expellido da larynge: \textunderscore gargarejar um líquido medicamentoso\textunderscore .
Agitar qualquer líquido na bôca, por meio do ar que se expelle da garganta.
Namorar, conversando da rua para a janela.
\section{Gargarejo}
\begin{itemize}
\item {Grp. gram.:m.}
\end{itemize}
Acto de gargarejar.
Líquido medicamentoso, para sêr gargarejado.
\section{Gargueiro}
\begin{itemize}
\item {Grp. gram.:m.}
\end{itemize}
\begin{itemize}
\item {Utilização:Pop.}
\end{itemize}
Garganta.
\section{Gárgula}
\begin{itemize}
\item {Grp. gram.:f.}
\end{itemize}
Buraco, por onde escorre a água de uma fonte ou cascata.
Cano estreito, por baixo dos beiraes ou na cimalha das cornijas, para receber as águas dos telhados.
(Cast. \textunderscore gargola\textunderscore , b. lat. \textunderscore gargula\textunderscore )
\section{Garianhinga}
\begin{itemize}
\item {Grp. gram.:f.}
\end{itemize}
Árvore angolense.
\section{Garibalde}
\begin{itemize}
\item {Grp. gram.:m.}
\end{itemize}
Espécie de guindaste, com uma corrente muito longa, que se usa nas alfândegas de Lisbôa e Porto.
\section{Garibáldi}
\begin{itemize}
\item {Grp. gram.:f.}
\end{itemize}
\begin{itemize}
\item {Proveniência:(De \textunderscore Garibaldi\textunderscore , n. p.)}
\end{itemize}
Espécie de camisola encarnada, que se veste exteriormente.
Casaco curto de mulheres.
\section{Garibaldino}
\begin{itemize}
\item {Grp. gram.:m.}
\end{itemize}
Soldado ou partidário de Garibáldi.
\section{Garidela}
\begin{itemize}
\item {Grp. gram.:f.}
\end{itemize}
Planta ranunculácea.
\section{Garidella}
\begin{itemize}
\item {Grp. gram.:f.}
\end{itemize}
Planta ranunculácea.
\section{Garimba}
\begin{itemize}
\item {Grp. gram.:f.}
\end{itemize}
\begin{itemize}
\item {Utilização:Prov.}
\end{itemize}
\begin{itemize}
\item {Utilização:trasm.}
\end{itemize}
O mesmo que \textunderscore garupa\textunderscore .
\section{Garimbar}
\begin{itemize}
\item {Grp. gram.:v. t.}
\end{itemize}
\begin{itemize}
\item {Utilização:Prov.}
\end{itemize}
\begin{itemize}
\item {Utilização:trasm.}
\end{itemize}
Bater em, castigar.
\section{Garimpar}
\begin{itemize}
\item {Grp. gram.:v. i.}
\end{itemize}
\begin{itemize}
\item {Utilização:Bras}
\end{itemize}
\begin{itemize}
\item {Proveniência:(De \textunderscore garimpo\textunderscore )}
\end{itemize}
Exercer o offício de garimpeiro.
\section{Garimpeiro}
\begin{itemize}
\item {Grp. gram.:m.}
\end{itemize}
\begin{itemize}
\item {Utilização:Bras}
\end{itemize}
\begin{itemize}
\item {Proveniência:(De \textunderscore garimpar\textunderscore )}
\end{itemize}
Explorador de diamantes.
\section{Garimpo}
\begin{itemize}
\item {Grp. gram.:m.}
\end{itemize}
\begin{itemize}
\item {Utilização:Bras. de Minas}
\end{itemize}
\begin{itemize}
\item {Utilização:Pop.}
\end{itemize}
\begin{itemize}
\item {Proveniência:(T. bras.)}
\end{itemize}
Lugar, onde se exploram metaes preciosos.
Mineração furtiva.
Garoto, vadio.
\section{Gariteiro}
\begin{itemize}
\item {Grp. gram.:m.}
\end{itemize}
\begin{itemize}
\item {Utilização:Des.}
\end{itemize}
\begin{itemize}
\item {Proveniência:(Do cast. \textunderscore garitero\textunderscore )}
\end{itemize}
Aquelle que tem casa de jôgo.
\section{Garito}
\begin{itemize}
\item {Grp. gram.:m.}
\end{itemize}
\begin{itemize}
\item {Utilização:Des.}
\end{itemize}
\begin{itemize}
\item {Utilização:Prov.}
\end{itemize}
\begin{itemize}
\item {Utilização:dur.}
\end{itemize}
\begin{itemize}
\item {Utilização:Prov.}
\end{itemize}
\begin{itemize}
\item {Utilização:dur.}
\end{itemize}
Casa de jôgo.
Abertura no gargalo da medida de vinho ou do almude, para designar aonde chegam 25 litros.
Córte ou mossa, que soffrem as sirgas, ao roçarem por pedras duríssimas.
(Cast. \textunderscore garito\textunderscore )
\section{Garjau}
\begin{itemize}
\item {Grp. gram.:m.}
\end{itemize}
Pássaro do mar da Índia. Cf. \textunderscore Hist. Trág. Marit.\textunderscore , 175.
\section{Garlindéu}
\begin{itemize}
\item {Grp. gram.:m.}
\end{itemize}
\begin{itemize}
\item {Utilização:Náut.}
\end{itemize}
Peça de ferro, por onde passam os cadernaes das adriças, no tôpo do mastro.
\section{Garlopa}
\begin{itemize}
\item {Grp. gram.:f.}
\end{itemize}
\begin{itemize}
\item {Proveniência:(Do fr. \textunderscore varlope\textunderscore )}
\end{itemize}
Plaina grande.
\section{Garna}
\begin{itemize}
\item {Grp. gram.:f.}
\end{itemize}
\begin{itemize}
\item {Utilização:Bras}
\end{itemize}
O mesmo que \textunderscore chuvisco\textunderscore .--Vem nos diccionários, mas inclino-me a que não existe na linguagem brasileira. Supponho têr havido êrro, proveniente da simples e fácil troca de uma letra: \textunderscore garna\textunderscore  por \textunderscore garua\textunderscore .
(Cp. \textunderscore garua\textunderscore )
\section{Garnacha}
\begin{itemize}
\item {Grp. gram.:f.}
\end{itemize}
\begin{itemize}
\item {Grp. gram.:M.}
\end{itemize}
\begin{itemize}
\item {Proveniência:(Do it. \textunderscore guarnaccia\textunderscore )}
\end{itemize}
Vestimenta talar de sacerdotes e magistrados.
Aquelle que veste garnacha. Cf. Herculano, \textunderscore Cister\textunderscore , I, 174.
\section{Garnacho}
\begin{itemize}
\item {Grp. gram.:m.}
\end{itemize}
\begin{itemize}
\item {Utilização:Pop.}
\end{itemize}
\begin{itemize}
\item {Utilização:Prov.}
\end{itemize}
\begin{itemize}
\item {Utilização:trasm.}
\end{itemize}
\begin{itemize}
\item {Proveniência:(De \textunderscore garnacha\textunderscore )}
\end{itemize}
Gabão.
Espaço angular do peito, descoberto pela camisa, desapertado o botão cimeiro.
\section{Garnar}
\textunderscore v. i. Bras.\textunderscore ?
Chuviscar.
(Cp. \textunderscore garna\textunderscore )
\section{Garnear}
\begin{itemize}
\item {Grp. gram.:v. i.}
\end{itemize}
Alisar com a maceta (sola ou coiro).
(Relaciona-se com \textunderscore garnir\textunderscore ?)
\section{Garnel}
\begin{itemize}
\item {Grp. gram.:m.}
\end{itemize}
(Corr. de \textunderscore granel\textunderscore )
\section{Garnela}
\begin{itemize}
\item {Grp. gram.:f. Loc. adv.}
\end{itemize}
\begin{itemize}
\item {Utilização:Gír.}
\end{itemize}
\begin{itemize}
\item {Proveniência:(De \textunderscore garnel\textunderscore )}
\end{itemize}
\textunderscore Á garnela\textunderscore , á vontade.
\section{Garnimento}
\begin{itemize}
\item {Grp. gram.:m.}
\end{itemize}
\begin{itemize}
\item {Utilização:Des.}
\end{itemize}
\begin{itemize}
\item {Proveniência:(De \textunderscore garnir\textunderscore )}
\end{itemize}
Guarnição, enfeite, adôrno.
\section{Garnir}
\begin{itemize}
\item {Grp. gram.:v. t.}
\end{itemize}
\begin{itemize}
\item {Utilização:Ant.}
\end{itemize}
O mesmo que \textunderscore guarnecer\textunderscore .
(B. lat. \textunderscore garnire\textunderscore )
\section{Garnisé}
\begin{itemize}
\item {Grp. gram.:adj.}
\end{itemize}
\begin{itemize}
\item {Utilização:Bras}
\end{itemize}
Diz-se de uma espécie de gallinha pequena, originária de Guernesey.
\section{Garo}
\begin{itemize}
\item {Grp. gram.:m.}
\end{itemize}
\begin{itemize}
\item {Proveniência:(Lat. \textunderscore garus\textunderscore )}
\end{itemize}
Espécie de lagosta.
\section{Garo}
\begin{itemize}
\item {Grp. gram.:m.}
\end{itemize}
\begin{itemize}
\item {Proveniência:(Lat. \textunderscore garum\textunderscore )}
\end{itemize}
Salmoira, feita dos intestinos do garo^1.
\section{Garôa}
\begin{itemize}
\item {Grp. gram.:f.}
\end{itemize}
\begin{itemize}
\item {Utilização:Bras. do S}
\end{itemize}
O mesmo que \textunderscore chuvisco\textunderscore .
(Do peruano \textunderscore garua\textunderscore )
\section{Garoar}
\begin{itemize}
\item {Grp. gram.:v. i.}
\end{itemize}
\begin{itemize}
\item {Utilização:Bras. do S}
\end{itemize}
\begin{itemize}
\item {Proveniência:(De \textunderscore garôa\textunderscore )}
\end{itemize}
Chuviscar.
\section{Garopa}
\begin{itemize}
\item {fónica:garô}
\end{itemize}
\begin{itemize}
\item {Grp. gram.:f.}
\end{itemize}
O mesmo que \textunderscore garoupa\textunderscore .
\section{Garotada}
\begin{itemize}
\item {Grp. gram.:f.}
\end{itemize}
Conjunto de garotos.
Garotice.
Acção ou palavra própria de garoto.
\section{Garotar}
\begin{itemize}
\item {Grp. gram.:v. i.}
\end{itemize}
Têr vida de garoto.
Fazer garotices.
Gandaiar; andar á tuna.
\section{Garotear}
\begin{itemize}
\item {Grp. gram.:v. t.}
\end{itemize}
\begin{itemize}
\item {Utilização:Bras}
\end{itemize}
O mesmo que \textunderscore garrotear\textunderscore ^1.
\section{Garotete}
\begin{itemize}
\item {fónica:tê}
\end{itemize}
\begin{itemize}
\item {Grp. gram.:m.}
\end{itemize}
Garoto pequeno.
\section{Garotice}
\begin{itemize}
\item {Grp. gram.:f.}
\end{itemize}
Vida de garoto.
Acto ou dito, próprio de garoto.
\section{Garotil}
\begin{itemize}
\item {Grp. gram.:m.}
\end{itemize}
\begin{itemize}
\item {Utilização:Náut.}
\end{itemize}
Parte superior da vela do navio em que há os ilhós onde entram os envergues.
O mesmo que \textunderscore gorotil\textunderscore .
\section{Garoto}
\begin{itemize}
\item {fónica:garô}
\end{itemize}
\begin{itemize}
\item {Grp. gram.:m.}
\end{itemize}
\begin{itemize}
\item {Grp. gram.:Adj.}
\end{itemize}
Rapaz vadio; gaiato.
Rapaz imberbe.
Que brinca ou vadia pelas ruas.
Travesso.
\section{Garoupa}
\begin{itemize}
\item {Grp. gram.:f.}
\end{itemize}
Designação de vários peixes da fam. dos pércidas.
\section{Garoupeira}
\begin{itemize}
\item {Grp. gram.:f.}
\end{itemize}
\begin{itemize}
\item {Utilização:Bras}
\end{itemize}
\begin{itemize}
\item {Proveniência:(De \textunderscore garoupa\textunderscore )}
\end{itemize}
Embarcação, usada na pesca da garoupa, com um mastro ao meio e outro, pequeno, á popa.
\section{Garra}
\begin{itemize}
\item {Grp. gram.:f.}
\end{itemize}
\begin{itemize}
\item {Utilização:Ext.}
\end{itemize}
\begin{itemize}
\item {Utilização:Fig.}
\end{itemize}
\begin{itemize}
\item {Utilização:T. da Bairrada}
\end{itemize}
\begin{itemize}
\item {Grp. gram.:Pl.}
\end{itemize}
\begin{itemize}
\item {Utilização:Bras. do S}
\end{itemize}
\begin{itemize}
\item {Grp. gram.:M.}
\end{itemize}
\begin{itemize}
\item {Utilização:Prov.}
\end{itemize}
\begin{itemize}
\item {Utilização:trasm.}
\end{itemize}
Unha aguçada de algumas feras e aves de rapina.
Unhas, dedos, mãos.
Pêlo comprido em redor das juntas dos pés dos cavallos.
Gavinha.
Tyrannia.
* \textunderscore Bras. do N.\textunderscore  e \textunderscore t. da Bairrada e Alcanena\textunderscore .
Cada uma das extremidades de um coiro, correspondente aos membros deanteiros e traseiros.
Cabedal ou sola ruim.
Arreios velhos e grosseiros.
Cabedal ruim, peludo, que se emprega em tombas, gáspeas, etc.
\section{Garra}
\begin{itemize}
\item {Grp. gram.:f.}
\end{itemize}
Acto de garrar.
\section{Garra}
\begin{itemize}
\item {Grp. gram.:adj. f.}
\end{itemize}
\begin{itemize}
\item {Utilização:Prov.}
\end{itemize}
\begin{itemize}
\item {Utilização:trasm.}
\end{itemize}
\begin{itemize}
\item {Proveniência:(Do cast. \textunderscore guarra\textunderscore , porca)}
\end{itemize}
Diz-se da porca, quando refeita e gorda.
Diz-se da mulher, que não é asseada.
\section{Garrabulho}
\begin{itemize}
\item {Grp. gram.:m.}
\end{itemize}
\begin{itemize}
\item {Utilização:Ant.}
\end{itemize}
O mesmo que \textunderscore garabulha\textunderscore  e \textunderscore garabulho\textunderscore . Cf. Pant. de Aveiro, \textunderscore Itiner.\textunderscore , 47 v.^o, (2.^a ed.).
\section{Garrafa}
\begin{itemize}
\item {Grp. gram.:f.}
\end{itemize}
\begin{itemize}
\item {Proveniência:(Do ár. \textunderscore garraf\textunderscore )}
\end{itemize}
Vaso, especialmente de vidro e com gargalo estreito, destinado a conter líquidos.
Conteúdo de uma garrafa.
\textunderscore Garrafa de Leyde\textunderscore , apparelho condensador de electricidade.
\section{Garrafada}
\begin{itemize}
\item {Grp. gram.:f.}
\end{itemize}
\begin{itemize}
\item {Utilização:Fam.}
\end{itemize}
\begin{itemize}
\item {Utilização:Bras. de Minas}
\end{itemize}
Conteúdo líquido de uma garrafa.
Medicamento líquido, contido numa garrafa.
Remédio de curandeiro.
\section{Garrafal}
\begin{itemize}
\item {Grp. gram.:adj.}
\end{itemize}
Que tem fórma de garrafa.
Graúdo.
Diz-se especialmente da letra manuscrita, quando é grande ou muito legível.
Diz-se de uma casta de ginja e de uma casta de cereja.
\section{Garrafalmente}
\begin{itemize}
\item {Grp. gram.:adv.}
\end{itemize}
\begin{itemize}
\item {Proveniência:(De \textunderscore garrafal\textunderscore )}
\end{itemize}
De modo vistoso ou emphático. Cf. Camillo, \textunderscore Brasileira\textunderscore , 85.
\section{Garrafão}
\begin{itemize}
\item {Grp. gram.:m.}
\end{itemize}
Garrafa grande, ordinariamente empalhada.
\section{Garrafeira}
\begin{itemize}
\item {Grp. gram.:f.}
\end{itemize}
Lugar, onde se guardam garrafas com vinho; frasqueira.
\section{Garraiada}
\begin{itemize}
\item {Grp. gram.:f.}
\end{itemize}
Corrida de garraios.
Ajuntamento de garraios.
\section{Garraio}
\begin{itemize}
\item {Grp. gram.:m.}
\end{itemize}
\begin{itemize}
\item {Utilização:Fam.}
\end{itemize}
Bezerro, que ainda não foi corrido.
Homem novato, inexperiente.
\section{Garrama}
\begin{itemize}
\item {Grp. gram.:f.}
\end{itemize}
\begin{itemize}
\item {Utilização:Ant.}
\end{itemize}
Imposto.
Finta, derrama.
(Do ár.)
\section{Garramar}
\begin{itemize}
\item {Grp. gram.:v. t.}
\end{itemize}
\begin{itemize}
\item {Utilização:Ant.}
\end{itemize}
\begin{itemize}
\item {Proveniência:(De \textunderscore garrama\textunderscore )}
\end{itemize}
Lançar garrama ou impostos sôbre; tributar. Cf. Sousa, \textunderscore Ann. de D. João III\textunderscore , 111.
\section{Garrana}
\begin{itemize}
\item {Grp. gram.:f.}
\end{itemize}
\begin{itemize}
\item {Proveniência:(Do rad. de \textunderscore garrão\textunderscore )}
\end{itemize}
Égua pequena, mas robusta.
\section{Garrancha}
\begin{itemize}
\item {Grp. gram.:f.}
\end{itemize}
\begin{itemize}
\item {Utilização:T. da Bairrada}
\end{itemize}
Pernada de árvore.
Cajado com uma volta na parte superior, á maneira de báculo.
Vara, com gancho ou podôa numa extremidade, para cortar ramos dos pinheiros.
(Cp. \textunderscore garrancho\textunderscore )
\section{Garranchada}
\begin{itemize}
\item {Grp. gram.:f.}
\end{itemize}
Ferida, causada por garrancho.
\section{Garrancho}
\begin{itemize}
\item {Grp. gram.:m.}
\end{itemize}
\begin{itemize}
\item {Utilização:Prov.}
\end{itemize}
\begin{itemize}
\item {Utilização:alent.}
\end{itemize}
\begin{itemize}
\item {Utilização:T. do Fundão}
\end{itemize}
Moléstia no casco das bêstas.
Garaveto.
Arbusto tortuoso.
O mesmo que \textunderscore pernilongo\textunderscore , ave.
Parceiro que, no jôgo do voltarete ou do solo, está parado em quanto os outros jogam.
Espinho, garaveto ou pua, que se introduziu na pata de um animal.
Gadanho.
(Cast. \textunderscore garrancho\textunderscore )
\section{Garranchoso}
\begin{itemize}
\item {Grp. gram.:adj.}
\end{itemize}
Que tem fórma de garrancho.
Torto.
\section{Garrano}
\begin{itemize}
\item {Grp. gram.:m.}
\end{itemize}
\begin{itemize}
\item {Proveniência:(De \textunderscore garrão\textunderscore )}
\end{itemize}
Cavallo pequeno mas robusto.
\section{Garranto}
\begin{itemize}
\item {Grp. gram.:m.}
\end{itemize}
Peixe da ria de Aveiro, semelhante á taínha.
\section{Garrão}
\begin{itemize}
\item {Grp. gram.:m.}
\end{itemize}
\begin{itemize}
\item {Utilização:Bras}
\end{itemize}
\begin{itemize}
\item {Proveniência:(Do rad. de \textunderscore garra\textunderscore )}
\end{itemize}
Nervo da perna do animal cavallar.
\section{Garrar}
\begin{itemize}
\item {Grp. gram.:v. t.}
\end{itemize}
\begin{itemize}
\item {Grp. gram.:V. i.}
\end{itemize}
\begin{itemize}
\item {Proveniência:(Do ár. \textunderscore gara\textunderscore ? Ou relaciona-se com o fr. \textunderscore garer\textunderscore , do ant. al. \textunderscore waron\textunderscore ?)}
\end{itemize}
Desprender (amarras).
Passar além de.
Vogar á mercê das ondas, sêr impellido pelas ondas, (falando-se do navio desancorado).
\section{Garrau}
\begin{itemize}
\item {Grp. gram.:m.}
\end{itemize}
O mesmo que \textunderscore garão\textunderscore .
\section{Garraz}
\begin{itemize}
\item {Grp. gram.:m.}
\end{itemize}
Pano ordinario de algodão.
\section{Garré!}
\begin{itemize}
\item {Grp. gram.:interj.}
\end{itemize}
\begin{itemize}
\item {Utilização:Prov.}
\end{itemize}
\begin{itemize}
\item {Utilização:trasm.}
\end{itemize}
Voz, com que de longe se chamam os porcos.
(Cp. \textunderscore garra\textunderscore ^3)
\section{Garrenta}
\begin{itemize}
\item {Grp. gram.:adj. f.}
\end{itemize}
\begin{itemize}
\item {Utilização:Prov.}
\end{itemize}
\begin{itemize}
\item {Utilização:trasm.}
\end{itemize}
O mesmo que \textunderscore garra\textunderscore ^3.
\section{Garrento}
\begin{itemize}
\item {Grp. gram.:m.}
\end{itemize}
O mesmo que \textunderscore taínha\textunderscore .
(Cp. \textunderscore garranto\textunderscore )
\section{Gárria}
\begin{itemize}
\item {Grp. gram.:f.}
\end{itemize}
Gênero de plantas cannabíneas.
\section{Garriça}
\begin{itemize}
\item {Grp. gram.:f.}
\end{itemize}
\begin{itemize}
\item {Utilização:Prov.}
\end{itemize}
\begin{itemize}
\item {Utilização:trasm.}
\end{itemize}
\begin{itemize}
\item {Proveniência:(De \textunderscore galho\textunderscore  + \textunderscore riçar\textunderscore ?)}
\end{itemize}
Galho de lódão, que, depois de chapotado, engrossou irregularmente, rebentando delle pequenos galhos curtos e emmaranhados.
\section{Garriço}
\begin{itemize}
\item {Grp. gram.:m.}
\end{itemize}
\begin{itemize}
\item {Utilização:Prov.}
\end{itemize}
\begin{itemize}
\item {Utilização:trasm.}
\end{itemize}
Pente de alisar.
(Cp. \textunderscore garriça\textunderscore )
\section{Garrida}
\begin{itemize}
\item {Grp. gram.:f.}
\end{itemize}
\begin{itemize}
\item {Utilização:Prov.}
\end{itemize}
\begin{itemize}
\item {Utilização:alent.}
\end{itemize}
\begin{itemize}
\item {Utilização:T. da Bairrada}
\end{itemize}
\begin{itemize}
\item {Proveniência:(Do lat. \textunderscore garritus\textunderscore ?)}
\end{itemize}
Sineta.
Roda de ferro, que se põe por baixo das grandes pedras para as deslocar.
Colleira, com seis pequenos chocalhos, que se põe aos bois.
Peça, geralmente de ferro, encaixada no cocão, e sôbre a qual gira o eixo do carro; o mesmo que \textunderscore cantadoira\textunderscore .
\section{Garridamente}
\begin{itemize}
\item {Grp. gram.:adv.}
\end{itemize}
De modo garrido.
\section{Garridice}
\begin{itemize}
\item {Grp. gram.:f.}
\end{itemize}
Qualidade daquelle ou daquillo que é garrido.
\section{Garridismo}
\begin{itemize}
\item {Grp. gram.:m.}
\end{itemize}
O mesmo que \textunderscore garridice\textunderscore .
\section{Garrido}
\begin{itemize}
\item {Grp. gram.:adj.}
\end{itemize}
\begin{itemize}
\item {Proveniência:(De \textunderscore garrir\textunderscore )}
\end{itemize}
Elegante.
Vistoso.
Muito enfeitado; casquilho.
Alegre.
\section{Garrir}
\begin{itemize}
\item {Grp. gram.:v. i.}
\end{itemize}
\begin{itemize}
\item {Grp. gram.:V. p.}
\end{itemize}
\begin{itemize}
\item {Proveniência:(Lat. \textunderscore garrire\textunderscore )}
\end{itemize}
Resoar; badalar.
Falar muito; chilrear.
Foliar.
Ostentar galas.
Trajar luxuosamente.
Brilhar.
Trajar com garridice.
\section{Garro}
\begin{itemize}
\item {Grp. gram.:adj.}
\end{itemize}
\begin{itemize}
\item {Utilização:Des.}
\end{itemize}
\begin{itemize}
\item {Grp. gram.:M.}
\end{itemize}
Leproso; que tem sarna.
O mesmo que \textunderscore sarro\textunderscore .
\section{Garro}
\begin{itemize}
\item {Grp. gram.:m.}
\end{itemize}
O mesmo que \textunderscore calambuco\textunderscore . Cf. G. Horta, \textunderscore Collóquios\textunderscore .
\section{Garroba}
\begin{itemize}
\item {fónica:rô}
\end{itemize}
\begin{itemize}
\item {Grp. gram.:f.}
\end{itemize}
\begin{itemize}
\item {Utilização:Prov.}
\end{itemize}
Planta papilionácea, o mesmo que \textunderscore parda\textunderscore . (Colhido em Barca de Alva)
\section{Garrocha}
\begin{itemize}
\item {Grp. gram.:f.}
\end{itemize}
\begin{itemize}
\item {Utilização:Prov.}
\end{itemize}
\begin{itemize}
\item {Utilização:trasm.}
\end{itemize}
\begin{itemize}
\item {Utilização:ant.}
\end{itemize}
\begin{itemize}
\item {Utilização:Gír.}
\end{itemize}
Pau, que tem numa extremidade um ferro farpado, e de que os toireiros se serviam, antes do uso das bandarilhas, nas corridas de toiros.
Croça, capa de palha.
Unha, mão.
(Cast. \textunderscore garrocha\textunderscore )
\section{Garrochador}
\begin{itemize}
\item {Grp. gram.:m.}
\end{itemize}
\begin{itemize}
\item {Utilização:Ant.}
\end{itemize}
\begin{itemize}
\item {Proveniência:(De \textunderscore garrochar\textunderscore )}
\end{itemize}
Aquelle que picava os toiros com garrocha.
\section{Garrochão}
\begin{itemize}
\item {Grp. gram.:m.}
\end{itemize}
Garrocha grande para cavalleiros.
\section{Garrochar}
\begin{itemize}
\item {Grp. gram.:v. t.}
\end{itemize}
Picar com garrocha.
\section{Garrôcho}
\begin{itemize}
\item {Grp. gram.:m.}
\end{itemize}
\begin{itemize}
\item {Utilização:Prov.}
\end{itemize}
\begin{itemize}
\item {Utilização:alg.}
\end{itemize}
O mesmo que \textunderscore garrancho\textunderscore .
\section{Garrotar}
\begin{itemize}
\item {Grp. gram.:v. t.}
\end{itemize}
Estrangular por meio de garrote^1.
\section{Garrote}
\begin{itemize}
\item {Grp. gram.:m.}
\end{itemize}
Pau curto, com que se apertava a corda do enforcado.
Estrangulação, sem que se suspenda a víctima.
(Talvez de \textunderscore garra\textunderscore )
\section{Garrote}
\begin{itemize}
\item {Grp. gram.:m.}
\end{itemize}
\begin{itemize}
\item {Utilização:Bras}
\end{itemize}
\begin{itemize}
\item {Grp. gram.:Adj.}
\end{itemize}
\begin{itemize}
\item {Utilização:Bras. do N}
\end{itemize}
\begin{itemize}
\item {Proveniência:(De \textunderscore garrão\textunderscore ?)}
\end{itemize}
Bezerro de dois a quatro annos de idade.
Diz-se do bezerro, que completou um anno de idade.
\section{Garrote}
\begin{itemize}
\item {Grp. gram.:m.}
\end{itemize}
\begin{itemize}
\item {Utilização:T. de Sabrosa}
\end{itemize}
O mesmo que \textunderscore barrote\textunderscore .
\section{Garrotear}
\begin{itemize}
\item {Grp. gram.:v. t.}
\end{itemize}
O mesmo que \textunderscore garrotar\textunderscore . Cf. Filinto, XVIII, 137.
\section{Garrotear}
\begin{itemize}
\item {Grp. gram.:v. t.}
\end{itemize}
\begin{itemize}
\item {Utilização:Bras. do S}
\end{itemize}
\begin{itemize}
\item {Proveniência:(T. ant. cast.)}
\end{itemize}
Sovar e bater (o coiro), para o amaciar.
\section{Garroteia}
\begin{itemize}
\item {Grp. gram.:f.}
\end{itemize}
\begin{itemize}
\item {Utilização:Ant.}
\end{itemize}
Espécie de tecido.
\section{Garroteia}
\begin{itemize}
\item {Grp. gram.:f.}
\end{itemize}
\begin{itemize}
\item {Utilização:Ant.}
\end{itemize}
O mesmo que \textunderscore jarreteira\textunderscore . Cf. Rui de Pina, \textunderscore Chrón. de Aff. V\textunderscore , XXXI.
\section{Garrotilho}
\begin{itemize}
\item {Grp. gram.:m.}
\end{itemize}
\begin{itemize}
\item {Utilização:Prov.}
\end{itemize}
\begin{itemize}
\item {Utilização:beir.}
\end{itemize}
\begin{itemize}
\item {Proveniência:(De \textunderscore garrote\textunderscore ^1)}
\end{itemize}
Angina aguda, acompanhada de crupe.
Nome de uma doença das vinhas. (Colhido na Guarda)
\section{Garrucha}
\begin{itemize}
\item {Grp. gram.:f.}
\end{itemize}
\begin{itemize}
\item {Grp. gram.:Pl.}
\end{itemize}
\begin{itemize}
\item {Utilização:Náut.}
\end{itemize}
\begin{itemize}
\item {Proveniência:(Do rad. de \textunderscore garra\textunderscore ^1)}
\end{itemize}
Pau curto, com que se armavam as béstas.
Antigo instrumento de tortura.
Argolas de ferro, pregadas no gorotil das velas latinas.
Cabos que se metem nas relingas por entre chicotes.
\section{Garrucha}
\begin{itemize}
\item {Grp. gram.:f.}
\end{itemize}
\begin{itemize}
\item {Utilização:Bras}
\end{itemize}
\begin{itemize}
\item {Utilização:Fig.}
\end{itemize}
Pistola grande.
Bacamarte.
Mulher velha, indígena do Brasil.
\section{Garrucho}
\begin{itemize}
\item {Grp. gram.:m.}
\end{itemize}
O mesmo que \textunderscore garrucha\textunderscore ^1.
\section{Garruço}
\begin{itemize}
\item {Grp. gram.:m.}
\end{itemize}
\begin{itemize}
\item {Utilização:Prov.}
\end{itemize}
O mesmo que \textunderscore carapuço\textunderscore .
(Por \textunderscore gorruço\textunderscore , de \textunderscore gorro\textunderscore ?)
\section{Garruda}
\begin{itemize}
\item {Grp. gram.:adj. f.}
\end{itemize}
\begin{itemize}
\item {Utilização:T. de Alcanena}
\end{itemize}
\begin{itemize}
\item {Proveniência:(De \textunderscore garra\textunderscore ^1)}
\end{itemize}
Diz-se da lan, que tem fio comprido.
\section{Garrular}
\begin{itemize}
\item {Grp. gram.:v. i.}
\end{itemize}
\begin{itemize}
\item {Proveniência:(De \textunderscore gárrulo\textunderscore )}
\end{itemize}
Palrar, tagarelar. Cf. Alencar, \textunderscore Diva\textunderscore .
\section{Garrulice}
\begin{itemize}
\item {Grp. gram.:f.}
\end{itemize}
Qualidade de quem é gárrulo.
\section{Garrulidade}
\begin{itemize}
\item {Grp. gram.:f.}
\end{itemize}
Qualidade de gárrulo. Cf. Castilho, \textunderscore Metam.\textunderscore , 279.
\section{Gárrulo}
\begin{itemize}
\item {Grp. gram.:m.  e  adj.}
\end{itemize}
\begin{itemize}
\item {Utilização:Ext.}
\end{itemize}
\begin{itemize}
\item {Proveniência:(Lat. \textunderscore garrulus\textunderscore )}
\end{itemize}
O que canta muito.
Aquelle que fala muito.
Palrador; tagarela.
\section{Garruncho}
\begin{itemize}
\item {Grp. gram.:m.}
\end{itemize}
\begin{itemize}
\item {Utilização:Náut.}
\end{itemize}
Círculo de ferro, onde passa um cabo de navio.
Círculo de ferro, que se segura nos estais, para pear velas latinas.
Cordão, tirado de um cabo descochado, e que, passando por um sapatilho, se emprega nos punhos da amura e na escota.
(Cp. \textunderscore garrucho\textunderscore )
\section{Gárrya}
\begin{itemize}
\item {Grp. gram.:f.}
\end{itemize}
Gênero de plantas cannabíneas.
\section{Garu}
\begin{itemize}
\item {Grp. gram.:m.}
\end{itemize}
\begin{itemize}
\item {Utilização:Ant.}
\end{itemize}
Matança; carnificina.
\section{Garua}
\begin{itemize}
\item {Grp. gram.:f.}
\end{itemize}
\begin{itemize}
\item {Utilização:Bras}
\end{itemize}
O mesmo que \textunderscore garôa\textunderscore .
\section{Garuar}
\begin{itemize}
\item {Grp. gram.:v. i.}
\end{itemize}
(V.garoar)
\section{Garula}
\begin{itemize}
\item {Grp. gram.:f.}
\end{itemize}
\begin{itemize}
\item {Utilização:Gír.}
\end{itemize}
Perna.
\section{Garulha}
\begin{itemize}
\item {Grp. gram.:f.}
\end{itemize}
\begin{itemize}
\item {Utilização:Prov.}
\end{itemize}
Producção vinícola de um anno.
\section{Garumá}
\begin{itemize}
\item {Grp. gram.:m.}
\end{itemize}
O mesmo que \textunderscore mutamba\textunderscore .
\section{Garunha}
\begin{itemize}
\item {Grp. gram.:adj. f.}
\end{itemize}
\begin{itemize}
\item {Utilização:Prov.}
\end{itemize}
\begin{itemize}
\item {Utilização:trasm.}
\end{itemize}
Diz-se da mulher somítica, avarenta.
\section{Garupa}
\begin{itemize}
\item {Grp. gram.:f.}
\end{itemize}
\begin{itemize}
\item {Utilização:Ext.}
\end{itemize}
\begin{itemize}
\item {Proveniência:(Do b. lat. \textunderscore groppa\textunderscore  de um rad. germ.)}
\end{itemize}
Parte superior da cavalgadura, entre o lombo e a cauda.
Ancas do cavallo.
Alforge ou mala, que se leva na garupa ou atrás da sella.
\section{Garupada}
\begin{itemize}
\item {Grp. gram.:f.}
\end{itemize}
\begin{itemize}
\item {Proveniência:(De \textunderscore garupa\textunderscore )}
\end{itemize}
Salto, dado pela cavalgadura, sem mostrar as ferraduras.
\section{Garupeiro}
\begin{itemize}
\item {Grp. gram.:m.}
\end{itemize}
Homem, que, na Índia portuguesa, traz em exposição pelos povoados as cobras que apanhou e a que arrancou os dentes injectores de veneno.
(Por \textunderscore garoupeiro\textunderscore , de \textunderscore garoupa\textunderscore ?)
\section{Garuva}
\begin{itemize}
\item {Grp. gram.:f.}
\end{itemize}
\begin{itemize}
\item {Utilização:Bras}
\end{itemize}
Árvore silvestre, de madeira amarela.
\section{Gás}
\begin{itemize}
\item {Grp. gram.:m.}
\end{itemize}
\begin{itemize}
\item {Utilização:Prov.}
\end{itemize}
\begin{itemize}
\item {Utilização:Pop.}
\end{itemize}
\begin{itemize}
\item {Grp. gram.:Pl.}
\end{itemize}
Qualquer fluido aeriforme.
Gás de illuminação.
Petróleo de illuminação.
Animação, modos desembaraçados: \textunderscore aquella rapariga tem muito gás\textunderscore .
\textunderscore Gás pobre\textunderscore , aquelle que tem menor poder calorífico, que o gás de illuminação, petróleo, etc.
Vapores do estômago e dos intestinos; ventosidades.
(Cp. cast. \textunderscore gas\textunderscore )
\section{Gasalhado}
\begin{itemize}
\item {Grp. gram.:m.}
\end{itemize}
\begin{itemize}
\item {Utilização:Ant.}
\end{itemize}
\begin{itemize}
\item {Proveniência:(De \textunderscore gasalhar\textunderscore )}
\end{itemize}
Roupas de cama.
Roupas.
Agasalho.
Bom acolhimento, bom trato. Cf. \textunderscore Hist. Trág. Marit.\textunderscore , 64; Fernão Lopes, \textunderscore passim\textunderscore .
\section{Gasalhar}
\begin{itemize}
\item {Grp. gram.:v. t.}
\end{itemize}
\begin{itemize}
\item {Proveniência:(Do ant. alt. al. \textunderscore gasalho\textunderscore )}
\end{itemize}
O mesmo que \textunderscore agasalhar\textunderscore .
\section{Gasalho}
\begin{itemize}
\item {Grp. gram.:m.}
\end{itemize}
O mesmo que \textunderscore agasalho\textunderscore .
\section{Gasalhoso}
\begin{itemize}
\item {Grp. gram.:adj.}
\end{itemize}
Que dá gasalho ou hospitalidade. Cf. \textunderscore Lusíadas\textunderscore , X, 96.
\section{Gascão}
\begin{itemize}
\item {Grp. gram.:m.}
\end{itemize}
\begin{itemize}
\item {Grp. gram.:Adj.}
\end{itemize}
\begin{itemize}
\item {Proveniência:(Fr. \textunderscore gascon\textunderscore )}
\end{itemize}
Dialecto da Gasconha.
Aquelle que é natural da Gasconha.
Relativo á Gasconha.
\section{Gascões}
\begin{itemize}
\item {Grp. gram.:m. pl.}
\end{itemize}
Peças do canhão do freio, em artilharia.
\section{Gaseificação}
\begin{itemize}
\item {Grp. gram.:f.}
\end{itemize}
Acto de gaseificar.
\section{Gaseificar}
\begin{itemize}
\item {Grp. gram.:v. t.}
\end{itemize}
\begin{itemize}
\item {Proveniência:(De \textunderscore gás\textunderscore  + lat. \textunderscore facere\textunderscore )}
\end{itemize}
Reduzir a gás.
\section{Gaseiforme}
\begin{itemize}
\item {Grp. gram.:adj.}
\end{itemize}
\begin{itemize}
\item {Proveniência:(De \textunderscore gás\textunderscore  + \textunderscore fórma\textunderscore )}
\end{itemize}
Que se apresenta em estado gasoso.
\section{Gasganete}
\begin{itemize}
\item {fónica:nê}
\end{itemize}
\begin{itemize}
\item {Grp. gram.:m.}
\end{itemize}
\begin{itemize}
\item {Utilização:Fam.}
\end{itemize}
O mesmo que \textunderscore garganta\textunderscore  ou \textunderscore pescoço\textunderscore .
(Cp. \textunderscore engasgar\textunderscore )
\section{Gasguete}
\begin{itemize}
\item {fónica:guê}
\end{itemize}
\begin{itemize}
\item {Grp. gram.:m.}
\end{itemize}
\begin{itemize}
\item {Utilização:Prov.}
\end{itemize}
\begin{itemize}
\item {Utilização:beir.}
\end{itemize}
O mesmo que \textunderscore gasganete\textunderscore .
\section{Gasguita}
\begin{itemize}
\item {Grp. gram.:adj.}
\end{itemize}
\begin{itemize}
\item {Utilização:Bras. do N}
\end{itemize}
Que fala com difficuldade.
\section{Gasguito}
\begin{itemize}
\item {Grp. gram.:adj.}
\end{itemize}
\begin{itemize}
\item {Utilização:Prov.}
\end{itemize}
\begin{itemize}
\item {Utilização:Bras}
\end{itemize}
Pretensioso, arrebicado. Cf. Camillo, \textunderscore Corja\textunderscore , 13 e 141.
Magro, enfèzado.
\section{Gasificar}
\begin{itemize}
\item {Grp. gram.:v. t.}
\end{itemize}
\begin{itemize}
\item {Proveniência:(De \textunderscore gás\textunderscore  + lat. \textunderscore facere\textunderscore )}
\end{itemize}
Reduzir a gás.
\section{Gasista}
\begin{itemize}
\item {Grp. gram.:m.}
\end{itemize}
\begin{itemize}
\item {Utilização:Bras}
\end{itemize}
\begin{itemize}
\item {Utilização:Neol.}
\end{itemize}
Aquelle que acende os candeeiros do gás, para illuminação pública.
\section{Gasmar}
\begin{itemize}
\item {Grp. gram.:v. t.}
\end{itemize}
\begin{itemize}
\item {Utilização:Ant.}
\end{itemize}
\begin{itemize}
\item {Utilização:Chul.}
\end{itemize}
Abichar, abiscoitar, apanhar:«\textunderscore dois quartos de binho me gasmou.\textunderscore »\textunderscore Anat. Joc.\textunderscore , 434.
\section{Gàsmil}
\begin{itemize}
\item {Grp. gram.:m.}
\end{itemize}
Essência de petróleo, incolor e fugacíssima, que foi, há poucos annos, muito empregada em candeeiros de esponja, para illuminação doméstica.
Gasolene.
\section{Gasnar}
\begin{itemize}
\item {Grp. gram.:v. i.}
\end{itemize}
\begin{itemize}
\item {Utilização:Ant.}
\end{itemize}
O mesmo que \textunderscore grasnar\textunderscore . Cf. \textunderscore Peregrinação\textunderscore , LXXIII.
\section{Gasnate}
\begin{itemize}
\item {Grp. gram.:m.}
\end{itemize}
O mesmo que \textunderscore gasganete\textunderscore .
(Contr. de \textunderscore gasganete\textunderscore )
\section{Gasnete}
\begin{itemize}
\item {fónica:nê}
\end{itemize}
\begin{itemize}
\item {Grp. gram.:m.}
\end{itemize}
O mesmo que \textunderscore gasganete\textunderscore .
(Contr. de \textunderscore gasganete\textunderscore )
\section{Gasogênio}
\begin{itemize}
\item {Grp. gram.:m.}
\end{itemize}
Apparelho, com que se faz a chamada água de Seltz, também conhecido por gasógeno.
(Cp. \textunderscore gasógeno\textunderscore )
\section{Gasógeno}
\begin{itemize}
\item {Grp. gram.:adj.}
\end{itemize}
\begin{itemize}
\item {Grp. gram.:M.}
\end{itemize}
\begin{itemize}
\item {Proveniência:(De \textunderscore gás\textunderscore  + gr. \textunderscore genes\textunderscore )}
\end{itemize}
Que produz gás.
Apparelho, o mesmo que \textunderscore gasogênio\textunderscore .
Mistura de álcool e terebinthina, própria para illuminação.
\section{Gasolene}
\begin{itemize}
\item {Grp. gram.:m.}
\end{itemize}
Um dos líquidos, obtidos pela destillação do petróleo.
O mesmo que \textunderscore gàsmil\textunderscore .
\section{Gasolina}
\begin{itemize}
\item {Grp. gram.:f.}
\end{itemize}
Carbonato de hydrogênio líquido.
\section{Gasólito}
\begin{itemize}
\item {Grp. gram.:adj.}
\end{itemize}
\begin{itemize}
\item {Grp. gram.:M. pl.}
\end{itemize}
\begin{itemize}
\item {Proveniência:(De \textunderscore gás\textunderscore  + gr. \textunderscore lutos\textunderscore )}
\end{itemize}
Que póde converter-se em gás.
Corpos simples, susceptíveis de formar gases permanentes, pela sua combinação com outros corpos simples.
\section{Gasólyto}
\begin{itemize}
\item {Grp. gram.:adj.}
\end{itemize}
\begin{itemize}
\item {Grp. gram.:M. pl.}
\end{itemize}
\begin{itemize}
\item {Proveniência:(De \textunderscore gás\textunderscore  + gr. \textunderscore lutos\textunderscore )}
\end{itemize}
Que póde converter-se em gás.
Corpos simples, susceptíveis de formar gases permanentes, pela sua combinação com outros corpos simples.
\section{Gasómetro}
\begin{itemize}
\item {Grp. gram.:m.}
\end{itemize}
\begin{itemize}
\item {Proveniência:(De \textunderscore gás\textunderscore  + gr. \textunderscore metron\textunderscore )}
\end{itemize}
Apparelho para medir gás.
Reservatório de gás para illuminação.
Fábrica de gás.
\section{Gasosa}
\begin{itemize}
\item {Grp. gram.:f.}
\end{itemize}
\begin{itemize}
\item {Proveniência:(De \textunderscore gasoso\textunderscore )}
\end{itemize}
Limonada gasosa.
\section{Gasoscópio}
\begin{itemize}
\item {Grp. gram.:m.}
\end{itemize}
\begin{itemize}
\item {Proveniência:(De \textunderscore gás\textunderscore  + gr. \textunderscore skopein\textunderscore )}
\end{itemize}
Instrumento, para se conhecer a presença de gases inflammáveis e para se verificar a fermentação alcoólica do vinho.
\section{Gasoso}
\begin{itemize}
\item {Grp. gram.:adj.}
\end{itemize}
\begin{itemize}
\item {Proveniência:(De \textunderscore gás\textunderscore )}
\end{itemize}
Aeriforme.
Que tem a natureza do gás.
Saturado de ácido carbónico.
\section{Gaspa}
\begin{itemize}
\item {Grp. gram.:f.}
\end{itemize}
O mesmo que \textunderscore gáspea\textunderscore .
\section{Gaspacho}
\begin{itemize}
\item {Grp. gram.:m.}
\end{itemize}
\begin{itemize}
\item {Utilização:Prov.}
\end{itemize}
\begin{itemize}
\item {Utilização:alg.}
\end{itemize}
O mesmo que \textunderscore caspacho\textunderscore .
\section{Gáspea}
\begin{itemize}
\item {Grp. gram.:f.}
\end{itemize}
Parte deanteira do calçado, que cobre o pé e é cosida á parte posterior, geralmente como remendo.
\section{Gaspeadeira}
\begin{itemize}
\item {Grp. gram.:f.}
\end{itemize}
Mulher que gaspeia.
\section{Gaspeado}
\begin{itemize}
\item {Grp. gram.:adj.}
\end{itemize}
\begin{itemize}
\item {Utilização:Prov.}
\end{itemize}
\begin{itemize}
\item {Utilização:beir.}
\end{itemize}
\begin{itemize}
\item {Proveniência:(De \textunderscore gaspear\textunderscore )}
\end{itemize}
Diz-se de calças, feitas de pano de várias côres e qualidades, e usadas por vários aldeões. (Colhido na Guarda)
\section{Gaspear}
\begin{itemize}
\item {Grp. gram.:v. t.}
\end{itemize}
Pôr gáspeas em: \textunderscore gaspear botas\textunderscore .
\section{Gaspóia}
\begin{itemize}
\item {Grp. gram.:f.}
\end{itemize}
\begin{itemize}
\item {Utilização:Prov.}
\end{itemize}
\begin{itemize}
\item {Utilização:trasm.}
\end{itemize}
Espécie de água-pé.
\section{Gassaba}
\begin{itemize}
\item {Grp. gram.:f.}
\end{itemize}
(V.igassaba)
\section{Gastador}
\begin{itemize}
\item {Grp. gram.:m.  e  adj.}
\end{itemize}
\begin{itemize}
\item {Utilização:Ext.}
\end{itemize}
\begin{itemize}
\item {Utilização:Des.}
\end{itemize}
O que gasta.
Dissipador; perdulário.
Soldado sapador.
\section{Gastalhão}
\begin{itemize}
\item {Grp. gram.:m.}
\end{itemize}
\begin{itemize}
\item {Utilização:Prov.}
\end{itemize}
\begin{itemize}
\item {Utilização:trasm.}
\end{itemize}
\begin{itemize}
\item {Proveniência:(De \textunderscore gastalho\textunderscore )}
\end{itemize}
Homem alto.
\section{Gastalho}
\begin{itemize}
\item {Grp. gram.:m.}
\end{itemize}
\begin{itemize}
\item {Utilização:Prov.}
\end{itemize}
\begin{itemize}
\item {Utilização:trasm.}
\end{itemize}
Espécie de grampo, com que se apertam aduelas, fôlhas de madeira, etc., nos trabalhos de tanoaria, marcenaria, etc.
Apparelho de tirar água dos poços; burra.
\section{Gastamento}
\begin{itemize}
\item {Grp. gram.:m.}
\end{itemize}
(V.gasto)
\section{Gastão}
\begin{itemize}
\item {Grp. gram.:m.}
\end{itemize}
O mesmo que \textunderscore castão\textunderscore .
\textunderscore Gastão do fuso\textunderscore , o bocadinho do chumbo ou latão, que cobre a pontinha do fuso, facilitando a torcedura do fio.
\section{Gastar}
\begin{itemize}
\item {Grp. gram.:v. t.}
\end{itemize}
\begin{itemize}
\item {Grp. gram.:V. p.}
\end{itemize}
\begin{itemize}
\item {Proveniência:(Do lat. \textunderscore vastare\textunderscore )}
\end{itemize}
Consumir, despender: \textunderscore gastar dinheiro\textunderscore .
Usar.
Desbaratar; dissipar: \textunderscore gastar uma herança\textunderscore .
Deminuir o volume de: \textunderscore gastar a ponteira da bengala com o uso\textunderscore .
Deteriorar: \textunderscore o andar gasta o calçado\textunderscore .
Cansar; esgotar: \textunderscore gastas-me a paciência\textunderscore .
Enfraquecer, extenuar.
Acabar: \textunderscore gastou-se tudo\textunderscore .
Empregar-se: \textunderscore gastou-se muito dinheiro nesta obra\textunderscore .
Exhibir-se.
\section{Gastável}
\begin{itemize}
\item {Grp. gram.:adj.}
\end{itemize}
Que se póde gastar.
Que se gasta muito.
\section{Gáster}
\begin{itemize}
\item {Grp. gram.:m.}
\end{itemize}
\begin{itemize}
\item {Proveniência:(Lat. \textunderscore gaster\textunderscore )}
\end{itemize}
O mesmo que \textunderscore gastro\textunderscore .
\section{Gasterina}
\begin{itemize}
\item {Grp. gram.:f.}
\end{itemize}
\begin{itemize}
\item {Proveniência:(Do lat. \textunderscore gaster\textunderscore )}
\end{itemize}
Sulfato de bismutho solúvel, que se applica contra inflammações intestinaes.
\section{Gasteromicetos}
\begin{itemize}
\item {Grp. gram.:m. pl.}
\end{itemize}
\begin{itemize}
\item {Proveniência:(Do gr. \textunderscore gaster\textunderscore  + \textunderscore mukes\textunderscore )}
\end{itemize}
Gênero de cogumelos.
\section{Gasteromycetos}
\begin{itemize}
\item {Grp. gram.:m. pl.}
\end{itemize}
\begin{itemize}
\item {Proveniência:(Do gr. \textunderscore gaster\textunderscore  + \textunderscore mukes\textunderscore )}
\end{itemize}
Gênero de cogumelos.
\section{Gasterópodes}
\begin{itemize}
\item {Grp. gram.:m. pl.}
\end{itemize}
\begin{itemize}
\item {Proveniência:(Do gr. \textunderscore gaster\textunderscore  + \textunderscore pous\textunderscore )}
\end{itemize}
Classe de molluscos, que comprehende o caracol, a lesma, etc.
\section{Gasteropterígio}
\begin{itemize}
\item {Grp. gram.:adj.}
\end{itemize}
\begin{itemize}
\item {Proveniência:(Do gr. \textunderscore gaster\textunderscore  + \textunderscore pterux\textunderscore )}
\end{itemize}
Diz-se dos peixes, cujas barbatanas ventraes ficam atrás das peitoraes.
\section{Gasteropterýgio}
\begin{itemize}
\item {Grp. gram.:adj.}
\end{itemize}
\begin{itemize}
\item {Proveniência:(Do gr. \textunderscore gaster\textunderscore  + \textunderscore pterux\textunderscore )}
\end{itemize}
Diz-se dos peixes, cujas barbatanas ventraes ficam atrás das peitoraes.
\section{Gasterósteos}
\begin{itemize}
\item {Grp. gram.:m. pl.}
\end{itemize}
\begin{itemize}
\item {Proveniência:(Do gr. \textunderscore gaster\textunderscore  + \textunderscore osteon\textunderscore )}
\end{itemize}
Gênero de pequenos peixes, espinhosos no dorso.
\section{Gasterozoário}
\begin{itemize}
\item {Grp. gram.:m.}
\end{itemize}
\begin{itemize}
\item {Proveniência:(Do gr. \textunderscore gaster\textunderscore  + \textunderscore zoarion\textunderscore )}
\end{itemize}
Animal, em que predomina o systema digestivo.
\section{Gasto}
\begin{itemize}
\item {Grp. gram.:m.}
\end{itemize}
\begin{itemize}
\item {Grp. gram.:Adj.}
\end{itemize}
\begin{itemize}
\item {Utilização:Fig.}
\end{itemize}
Acto ou effeito de gastar.
Aquillo que se gastou.
Que se gastou, que se despendeu: \textunderscore dinheiro gasto\textunderscore .
Cotiado: \textunderscore o teu casaco está muito gasto\textunderscore .
Deteriorado por attrito ou por uso: \textunderscore uma calçada muito gasta\textunderscore .
Combalido; abatido, por excessos ou doenças: \textunderscore um homem gasto\textunderscore .
\section{Gastralgia}
\begin{itemize}
\item {Grp. gram.:f.}
\end{itemize}
\begin{itemize}
\item {Proveniência:(Do gr. \textunderscore gaster\textunderscore  + \textunderscore algos\textunderscore )}
\end{itemize}
Dôr intensa no estômago.
\section{Gastrálgico}
\begin{itemize}
\item {Grp. gram.:adj.}
\end{itemize}
Relativo á gastralgia.
\section{Gastrectasia}
\begin{itemize}
\item {Grp. gram.:f.}
\end{itemize}
\begin{itemize}
\item {Utilização:Med.}
\end{itemize}
\begin{itemize}
\item {Proveniência:(Do gr. \textunderscore gaster\textunderscore  + \textunderscore ektasis\textunderscore )}
\end{itemize}
Dilatação do estômago.
\section{Gastrectomia}
\begin{itemize}
\item {Grp. gram.:f.}
\end{itemize}
\begin{itemize}
\item {Utilização:Cir.}
\end{itemize}
\begin{itemize}
\item {Proveniência:(Do gr. \textunderscore gaster\textunderscore  + \textunderscore ektome\textunderscore )}
\end{itemize}
Operação de cortar qualquer parte do estômago.
\section{Gástrica}
\begin{itemize}
\item {Grp. gram.:f.}
\end{itemize}
\begin{itemize}
\item {Utilização:Pop.}
\end{itemize}
Febre gástrica.
\section{Gastricidade}
\begin{itemize}
\item {Grp. gram.:f.}
\end{itemize}
\begin{itemize}
\item {Utilização:P. us.}
\end{itemize}
\begin{itemize}
\item {Proveniência:(De \textunderscore gástrico\textunderscore )}
\end{itemize}
Estado impuro do estômago; embaraço gástrico.
\section{Gastricismo}
\begin{itemize}
\item {Grp. gram.:m.}
\end{itemize}
\begin{itemize}
\item {Proveniência:(De \textunderscore gástrico\textunderscore )}
\end{itemize}
Systema dos que entendem que a maior parte das doenças procedem das impurezas do estômago.
Impureza do estômago.
\section{Gástrico}
\begin{itemize}
\item {Grp. gram.:adj.}
\end{itemize}
\begin{itemize}
\item {Proveniência:(De \textunderscore gastro\textunderscore )}
\end{itemize}
Relativo ao estômago: \textunderscore incómmodo gástrico\textunderscore .
\section{Gastrídio}
\begin{itemize}
\item {Grp. gram.:m.}
\end{itemize}
\begin{itemize}
\item {Proveniência:(Do gr. \textunderscore gaster\textunderscore )}
\end{itemize}
Gênero de plantas gramíneas.
\section{Gastríloquo}
\begin{itemize}
\item {Grp. gram.:m.  e  adj.}
\end{itemize}
\begin{itemize}
\item {Utilização:Des.}
\end{itemize}
\begin{itemize}
\item {Proveniência:(Do lat. \textunderscore gaster\textunderscore  + \textunderscore loqui\textunderscore )}
\end{itemize}
O mesmo que \textunderscore ventríloquo\textunderscore .
\section{Gastrite}
\begin{itemize}
\item {Grp. gram.:f.}
\end{itemize}
\begin{itemize}
\item {Proveniência:(De \textunderscore gastro\textunderscore )}
\end{itemize}
Inflammação da membrana do estômago.
\section{Gastro}
\begin{itemize}
\item {Grp. gram.:m.}
\end{itemize}
\begin{itemize}
\item {Grp. gram.:M.}
\end{itemize}
\begin{itemize}
\item {Proveniência:(Do lat. \textunderscore gaster\textunderscore )}
\end{itemize}
Elemento, que entra na composição de várias palavras, e significa \textunderscore estômago\textunderscore .
Antigo vaso romano, de grande bojo.
\section{Gastro-adynâmico}
\begin{itemize}
\item {Grp. gram.:adj.}
\end{itemize}
Relativo ao estômago e á adynamia.
\section{Gastrobronchite}
\begin{itemize}
\item {Grp. gram.:f.}
\end{itemize}
Inflammação do estômago e dos brônchios.
\section{Gastrobrosia}
\begin{itemize}
\item {Grp. gram.:f.}
\end{itemize}
\begin{itemize}
\item {Proveniência:(Do gr. \textunderscore gaster\textunderscore  + \textunderscore broein\textunderscore )}
\end{itemize}
Perfuração do estômago.
\section{Gastrocele}
\begin{itemize}
\item {Grp. gram.:m.}
\end{itemize}
\begin{itemize}
\item {Utilização:Med.}
\end{itemize}
\begin{itemize}
\item {Proveniência:(Do gr. \textunderscore gaster\textunderscore  + \textunderscore kele\textunderscore )}
\end{itemize}
Hérnia do estômago.
\section{Gastro-cnêmio}
\begin{itemize}
\item {Grp. gram.:adj.}
\end{itemize}
\begin{itemize}
\item {Proveniência:(Do gr. \textunderscore gaster\textunderscore  + \textunderscore kneme\textunderscore )}
\end{itemize}
Diz-se dos músculos da barriga da perna.
\section{Gastrocolite}
\begin{itemize}
\item {Grp. gram.:f.}
\end{itemize}
Inflammação simultânea do estômago e do cólon.
\section{Gastroconjuntivite}
\begin{itemize}
\item {Grp. gram.:f.}
\end{itemize}
Doença, que ataca a espécie cavallar, quando há calor excessivo, e que consiste na inflammação do estômago e da mucosa ocular.
\section{Gastrodinia}
\begin{itemize}
\item {Grp. gram.:f.}
\end{itemize}
\begin{itemize}
\item {Proveniência:(Do gr. \textunderscore gaster\textunderscore  + \textunderscore odune\textunderscore )}
\end{itemize}
Gênero de neurose da digestão, caracterizada por ansiedade e apêrto no epigastro.
\section{Gastro-duodenal}
\begin{itemize}
\item {Grp. gram.:adj.}
\end{itemize}
Relativo ao estômago e ao duodeno.
\section{Gastro-duodenite}
\begin{itemize}
\item {Grp. gram.:f.}
\end{itemize}
Inflammação do estômago e do duodeno.
\section{Gastrodynia}
\begin{itemize}
\item {Grp. gram.:f.}
\end{itemize}
\begin{itemize}
\item {Proveniência:(Do gr. \textunderscore gaster\textunderscore  + \textunderscore odune\textunderscore )}
\end{itemize}
Gênero de neurose da digestão, caracterizada por ansiedade e apêrto no epigastro.
\section{Gastroencephalite}
\begin{itemize}
\item {Grp. gram.:f.}
\end{itemize}
Inflammação do estômago, complicada de phenómenos nervosos.
\section{Gastro-enteralgia}
\begin{itemize}
\item {Grp. gram.:f.}
\end{itemize}
\begin{itemize}
\item {Utilização:Med.}
\end{itemize}
Gastralgia e enteralgia, concomitantes.
\section{Gastroenterite}
\begin{itemize}
\item {Grp. gram.:f.}
\end{itemize}
Inflammação simultânea do estômago e dos intestinos.
\section{Gastroenterocolite}
\begin{itemize}
\item {Grp. gram.:f.}
\end{itemize}
Inflammação simultânea do estômago e dos intestinos delgado e grosso.
\section{Gastro-epiploico}
\begin{itemize}
\item {Grp. gram.:adj.}
\end{itemize}
Relativo ao estômago e ao epíploon.
\section{Gastroespasmo}
\begin{itemize}
\item {Grp. gram.:m.}
\end{itemize}
Contracção espasmódica do estômago.
\section{Gastro-esplênico}
\begin{itemize}
\item {Grp. gram.:adj.}
\end{itemize}
Relativo ao estômago e ao baço.
\section{Gastrófilo}
\begin{itemize}
\item {Grp. gram.:adj.}
\end{itemize}
Que gosta de comer bem. Cf. Rui Barb., \textunderscore Réplica\textunderscore , 157.
\section{Gastro-hepático}
\begin{itemize}
\item {Grp. gram.:adj.}
\end{itemize}
Relativo ao estômago e ao fígado.
\section{Gastrohepatite}
\begin{itemize}
\item {Grp. gram.:f.}
\end{itemize}
Inflammação do estômago e do fígado.
\section{Gastrohysterotomia}
\begin{itemize}
\item {Grp. gram.:f.}
\end{itemize}
Operação cesariana abdominal, que consiste em abrir o ventre e a madre, para a extracção do féto.
\section{Gastro-intestinal}
\begin{itemize}
\item {Grp. gram.:adj.}
\end{itemize}
Relativo ao estômago e aos intestinos.
\section{Gastrolaryngite}
\begin{itemize}
\item {Grp. gram.:f.}
\end{itemize}
Inflammação do estômago e da larynge.
\section{Gastrólatra}
\begin{itemize}
\item {Grp. gram.:m.}
\end{itemize}
\begin{itemize}
\item {Utilização:P. us.}
\end{itemize}
\begin{itemize}
\item {Proveniência:(Do gr. \textunderscore gaster\textunderscore  + \textunderscore latreuein\textunderscore )}
\end{itemize}
Aquelle que sacrifica tudo ao prazer de comer.
\section{Gastrolatria}
\begin{itemize}
\item {Grp. gram.:f.}
\end{itemize}
Qualidade ou índole de gastrólatra.
\section{Gastrologia}
\begin{itemize}
\item {Grp. gram.:f.}
\end{itemize}
\begin{itemize}
\item {Proveniência:(Do gr. \textunderscore gaster\textunderscore  + \textunderscore logos\textunderscore )}
\end{itemize}
Arte culinária.
Conhecimento profundo dessa arte.
\section{Gastrólogo}
\begin{itemize}
\item {Grp. gram.:m.}
\end{itemize}
Aquelle que conhece bem a gastrologia.
\section{Gastrómelo}
\begin{itemize}
\item {Grp. gram.:m.}
\end{itemize}
\begin{itemize}
\item {Proveniência:(Do gr. \textunderscore gaster\textunderscore  + \textunderscore melos\textunderscore )}
\end{itemize}
Monstro que, além dos membros thorácicos e pélvicos, tem um ou dois membros accessórios inserídos no abdome.
\section{Gastromeningite}
\begin{itemize}
\item {Grp. gram.:f.}
\end{itemize}
Inflammação simultânea do estômago e da meninge.
\section{Gastrometrite}
\begin{itemize}
\item {Grp. gram.:f.}
\end{itemize}
Inflammação simultânea do estômago e da madre.
\section{Gastro-mucosa}
\begin{itemize}
\item {Grp. gram.:adj.}
\end{itemize}
Diz-se da enfermidade em que há gastrite, com superabundancia de secreções mucosas.
\section{Gastronecto}
\begin{itemize}
\item {Grp. gram.:adj.}
\end{itemize}
\begin{itemize}
\item {Grp. gram.:M. pl.}
\end{itemize}
\begin{itemize}
\item {Proveniência:(Do gr. \textunderscore gaster\textunderscore  + \textunderscore nektes\textunderscore )}
\end{itemize}
Diz-se do peixe, que tem as barbatanas abdominaes tão desenvolvidas, que formam um órgão próprio para a natação.
Família de crustáceos decápodes macruros.
\section{Gastronephrite}
\begin{itemize}
\item {Grp. gram.:f.}
\end{itemize}
Doença do estômago, acompanhada de inflammação dos rins.
\section{Gastronomia}
\begin{itemize}
\item {Grp. gram.:f.}
\end{itemize}
\begin{itemize}
\item {Proveniência:(Gr. \textunderscore gastronomia\textunderscore )}
\end{itemize}
Arte de cozinhar, por fórma que se proporcione o maior deleite aos que comem.
\section{Gastronómico}
\begin{itemize}
\item {Grp. gram.:adj.}
\end{itemize}
Relativo á gastronomia.
\section{Gastrónomo}
\begin{itemize}
\item {Grp. gram.:m.}
\end{itemize}
\begin{itemize}
\item {Proveniência:(Do gr. \textunderscore gaster\textunderscore  + \textunderscore nomos\textunderscore )}
\end{itemize}
Aquelle que aprecia as iguarias bem feitas, e procura os maiores prazeres da mesa.
\section{Gastropathia}
\begin{itemize}
\item {Grp. gram.:f.}
\end{itemize}
\begin{itemize}
\item {Utilização:Med.}
\end{itemize}
\begin{itemize}
\item {Proveniência:(Do gr. \textunderscore gaster\textunderscore  + \textunderscore pathos\textunderscore )}
\end{itemize}
Designação genérica das doenças do estômago.
\section{Gastropatia}
\begin{itemize}
\item {Grp. gram.:f.}
\end{itemize}
\begin{itemize}
\item {Utilização:Med.}
\end{itemize}
\begin{itemize}
\item {Proveniência:(Do gr. \textunderscore gaster\textunderscore  + \textunderscore pathos\textunderscore )}
\end{itemize}
Designação genérica das doenças do estômago.
\section{Gastroperitonite}
\begin{itemize}
\item {Grp. gram.:f.}
\end{itemize}
Inflammação do estômago e do peritonéu.
\section{Gastropexia}
\begin{itemize}
\item {fónica:csi}
\end{itemize}
\begin{itemize}
\item {Grp. gram.:f.}
\end{itemize}
\begin{itemize}
\item {Utilização:Med.}
\end{itemize}
\begin{itemize}
\item {Proveniência:(Do gr. \textunderscore gaster\textunderscore  + \textunderscore pexis\textunderscore )}
\end{itemize}
Fixação cirúrgica do estômago.
\section{Gastropharyngite}
\begin{itemize}
\item {Grp. gram.:f.}
\end{itemize}
Inflammação do estômago e da pharynge.
\section{Gastróphilo}
\begin{itemize}
\item {Grp. gram.:adj.}
\end{itemize}
Que gosta de comer bem. Cf. Rui Barb., \textunderscore Réplica\textunderscore , 157.
\section{Gastroplastia}
\begin{itemize}
\item {Grp. gram.:f.}
\end{itemize}
\begin{itemize}
\item {Utilização:Cir.}
\end{itemize}
\begin{itemize}
\item {Proveniência:(Do gr. \textunderscore gaster\textunderscore  + \textunderscore plassein\textunderscore )}
\end{itemize}
Operação, para fazer desapparecer um estreitamento do estômago ou para se fechar o orifício de uma úlcera do mesmo órgão.
\section{Gastroplegia}
\begin{itemize}
\item {Grp. gram.:f.}
\end{itemize}
\begin{itemize}
\item {Utilização:Med.}
\end{itemize}
\begin{itemize}
\item {Proveniência:(Do gr. \textunderscore gaster\textunderscore  + \textunderscore plege\textunderscore )}
\end{itemize}
Paralysia do estômago.
\section{Gastrópodes}
\begin{itemize}
\item {Grp. gram.:m. pl.}
\end{itemize}
\begin{itemize}
\item {Proveniência:(Do gr. \textunderscore gaster\textunderscore , \textunderscore gastros\textunderscore  + \textunderscore pous\textunderscore , \textunderscore podos\textunderscore )}
\end{itemize}
Gêneros de molluscos fósseis. Cp. \textunderscore gasterópodes\textunderscore .
\section{Gastroptose}
\begin{itemize}
\item {Grp. gram.:f.}
\end{itemize}
\begin{itemize}
\item {Utilização:Med.}
\end{itemize}
\begin{itemize}
\item {Proveniência:(Do gr. \textunderscore gaster\textunderscore  + \textunderscore ptosis\textunderscore )}
\end{itemize}
Deslocamento do estômago para baixo.
\section{Gastro-pylórico}
\begin{itemize}
\item {Grp. gram.:adj.}
\end{itemize}
Relativo ao estômago e ao pyloro.
\section{Gastrorrafia}
\begin{itemize}
\item {Grp. gram.:f.}
\end{itemize}
\begin{itemize}
\item {Proveniência:(Do gr. \textunderscore gaster\textunderscore  + \textunderscore rhaphe\textunderscore )}
\end{itemize}
Sutura nas paredes abdominaes.
\section{Gastrorragia}
\begin{itemize}
\item {Grp. gram.:f.}
\end{itemize}
\begin{itemize}
\item {Utilização:Med.}
\end{itemize}
\begin{itemize}
\item {Proveniência:(Do gr. \textunderscore gaster\textunderscore  + \textunderscore rhein\textunderscore )}
\end{itemize}
Hemorragia gástrica.
\section{Gastrorreia}
\begin{itemize}
\item {Grp. gram.:f.}
\end{itemize}
\begin{itemize}
\item {Proveniência:(Do gr. \textunderscore gaster\textunderscore  + \textunderscore rhein\textunderscore )}
\end{itemize}
Catarro estomacal.
\section{Gastrorrhagia}
\begin{itemize}
\item {Grp. gram.:f.}
\end{itemize}
\begin{itemize}
\item {Utilização:Med.}
\end{itemize}
\begin{itemize}
\item {Proveniência:(Do gr. \textunderscore gaster\textunderscore  + \textunderscore rhein\textunderscore )}
\end{itemize}
Hemorrhagia gástrica.
\section{Gastrorrhaphia}
\begin{itemize}
\item {Grp. gram.:f.}
\end{itemize}
\begin{itemize}
\item {Proveniência:(Do gr. \textunderscore gaster\textunderscore  + \textunderscore rhaphe\textunderscore )}
\end{itemize}
Sutura nas paredes abdominaes.
\section{Gastrorrhéa}
\begin{itemize}
\item {Grp. gram.:f.}
\end{itemize}
\begin{itemize}
\item {Proveniência:(Do gr. \textunderscore gaster\textunderscore  + \textunderscore rhein\textunderscore )}
\end{itemize}
Catarro estomacal.
\section{Gastrorrheia}
\begin{itemize}
\item {Grp. gram.:f.}
\end{itemize}
\begin{itemize}
\item {Proveniência:(Do gr. \textunderscore gaster\textunderscore  + \textunderscore rhein\textunderscore )}
\end{itemize}
Catarro estomacal.
\section{Gastrose}
\begin{itemize}
\item {Grp. gram.:f.}
\end{itemize}
\begin{itemize}
\item {Proveniência:(Do gr. \textunderscore gaster\textunderscore )}
\end{itemize}
Designação genérica das doenças do estômago.
\section{Gastrostomia}
\begin{itemize}
\item {Grp. gram.:f.}
\end{itemize}
\begin{itemize}
\item {Utilização:Cir.}
\end{itemize}
\begin{itemize}
\item {Proveniência:(Do gr. \textunderscore gaster\textunderscore  + \textunderscore stoma\textunderscore )}
\end{itemize}
Abertura de uma bôca estomocal, pela qual se introduzam alimentos.
\section{Gastroteca}
\begin{itemize}
\item {Grp. gram.:f.}
\end{itemize}
\begin{itemize}
\item {Proveniência:(Do gr. \textunderscore gaster\textunderscore  + \textunderscore thekè\textunderscore )}
\end{itemize}
Membrana, que reveste o abdome das crisálidas.
\section{Gastrotheca}
\begin{itemize}
\item {Grp. gram.:f.}
\end{itemize}
\begin{itemize}
\item {Proveniência:(Do gr. \textunderscore gaster\textunderscore  + \textunderscore thekè\textunderscore )}
\end{itemize}
Membrana, que reveste o abdome das chrysálidas.
\section{Gastrotomia}
\begin{itemize}
\item {Grp. gram.:f.}
\end{itemize}
Operação cirúrgica, para se extrahir do estômago um corpo extranho.
Puncção do rúmen, praticada nos ruminantes, atacados de tympanite.
(Cp. \textunderscore gastrótomo\textunderscore )
\section{Gastrótomo}
\begin{itemize}
\item {Grp. gram.:m.}
\end{itemize}
\begin{itemize}
\item {Proveniência:(Do gr. \textunderscore gaster\textunderscore  + \textunderscore tome\textunderscore )}
\end{itemize}
Instrumento, com que se abre o abdome dos animaes, atacados de tympanite.
\section{Gastro-vascular}
\begin{itemize}
\item {Grp. gram.:adj.}
\end{itemize}
\begin{itemize}
\item {Utilização:Ant.}
\end{itemize}
Relativo ao tubo digestivo e aos vasos.
\section{Gastrozoário}
\begin{itemize}
\item {Grp. gram.:m.}
\end{itemize}
\begin{itemize}
\item {Proveniência:(Do gr. \textunderscore gaster\textunderscore  + \textunderscore zoon\textunderscore )}
\end{itemize}
Hydra que, no pólypo hydráceo, desempenha as funcções da digestão.
\section{Gástrula}
\begin{itemize}
\item {Grp. gram.:f.}
\end{itemize}
\begin{itemize}
\item {Proveniência:(Do gr. \textunderscore gaster\textunderscore )}
\end{itemize}
Cavidade, resultante da envaginação de uma metade da blástula na outra metade.
Fórmula larvar inicial, commum a todos os seres animados.
\section{Gastrulação}
\begin{itemize}
\item {Grp. gram.:f.}
\end{itemize}
Formação da gástrula.
\section{Gastura}
\begin{itemize}
\item {Grp. gram.:f.}
\end{itemize}
\begin{itemize}
\item {Utilização:Bras}
\end{itemize}
\begin{itemize}
\item {Proveniência:(De \textunderscore gastar\textunderscore )}
\end{itemize}
O mesmo que \textunderscore comichão\textunderscore .
\section{Gata}
\begin{itemize}
\item {Grp. gram.:f.}
\end{itemize}
\begin{itemize}
\item {Utilização:Náut.}
\end{itemize}
\begin{itemize}
\item {Utilização:Náut.}
\end{itemize}
\begin{itemize}
\item {Utilização:Pop.}
\end{itemize}
\begin{itemize}
\item {Utilização:Prov.}
\end{itemize}
\begin{itemize}
\item {Utilização:escol.}
\end{itemize}
\begin{itemize}
\item {Utilização:Prov.}
\end{itemize}
\begin{itemize}
\item {Utilização:trasm.}
\end{itemize}
\begin{itemize}
\item {Grp. gram.:Loc.}
\end{itemize}
\begin{itemize}
\item {Utilização:pop.}
\end{itemize}
Fêmea do gato.
Peixe marítimo, (\textunderscore scyllium catulus\textunderscore ).
Pepino do Egypto.
Uma das gáveas, superior á mezena.
Âncora, cujas unhas formam um só corpo e giram na extremidade da haste, cravando-se no fundo ao mesmo tempo. (Também se chama \textunderscore ferro de tesoira\textunderscore )
Antiga máquina de guerra, espécie de catapulta.
O mesmo que \textunderscore bebedeira\textunderscore .
O mesmo que \textunderscore reprovação\textunderscore . (Colhido em Penafiel)
Resguardo de parede, que, no pombal, se eleva acima do respectivo telhado, para o lado do vento frio, dando o aspecto de uma golla levanda.
\textunderscore Gata borralheira\textunderscore , mulher cuidadosa em serviços caseiros.
\textunderscore Não poder com uma gata pelo rabo\textunderscore , estar muito fraco, abatido.
Não têr importância, não têr recursos.
\textunderscore Andar de gatas\textunderscore , o mesmo que \textunderscore andar de gatinhas\textunderscore .
\section{Gatafunhos}
\begin{itemize}
\item {Grp. gram.:m. pl.}
\end{itemize}
\begin{itemize}
\item {Proveniência:(Do rad. de \textunderscore gato\textunderscore )}
\end{itemize}
O mesmo que [[garatujas|garatuja]].
\section{Gatal}
\begin{itemize}
\item {Grp. gram.:adj.}
\end{itemize}
\begin{itemize}
\item {Utilização:Fam.}
\end{itemize}
Relativo a gato:«\textunderscore nada na esposa encontra de condição gatal\textunderscore ». Filinto, XII, 71.
\section{Gatanhada}
\begin{itemize}
\item {Grp. gram.:f.}
\end{itemize}
\begin{itemize}
\item {Utilização:Prov.}
\end{itemize}
\begin{itemize}
\item {Proveniência:(De \textunderscore gatanhar\textunderscore )}
\end{itemize}
Arranhadura de gato.
\section{Gatanhar}
\begin{itemize}
\item {Grp. gram.:v. t.}
\end{itemize}
\begin{itemize}
\item {Utilização:Prov.}
\end{itemize}
Arranhar, (falando-se do gato).
O mesmo que \textunderscore agatanhar\textunderscore .
\section{Gatanho}
\begin{itemize}
\item {Grp. gram.:m.}
\end{itemize}
\begin{itemize}
\item {Utilização:Prov.}
\end{itemize}
\begin{itemize}
\item {Utilização:minh.}
\end{itemize}
O mesmo que \textunderscore gatanhada\textunderscore .
\section{Gatanho}
\begin{itemize}
\item {Grp. gram.:m.}
\end{itemize}
\begin{itemize}
\item {Utilização:T. da Bairrada}
\end{itemize}
Espécie de tojo, também chamado tojo-gatão.
\section{Gatão}
\begin{itemize}
\item {Grp. gram.:m.}
\end{itemize}
\begin{itemize}
\item {Grp. gram.:Adj.}
\end{itemize}
Gato grande, gatarrão.
Diz-se de uma espécie de tojo:«\textunderscore ...uns carrascaes cercados de tojo gatão\textunderscore ». Corvo, \textunderscore Anno na Côrte\textunderscore , III, 28.
\section{Gatar}
\begin{itemize}
\item {Grp. gram.:v. t.}
\end{itemize}
\begin{itemize}
\item {Utilização:Prov.}
\end{itemize}
\begin{itemize}
\item {Utilização:escol.}
\end{itemize}
\begin{itemize}
\item {Proveniência:(De \textunderscore gata\textunderscore )}
\end{itemize}
Reprovar num exame. (Colhido em Penafiel)
\section{Gataria}
\begin{itemize}
\item {Grp. gram.:f.}
\end{itemize}
Porção de gatos.
\section{Gatária}
\begin{itemize}
\item {Grp. gram.:f.}
\end{itemize}
Planta labiada, espécie de hortelan, também conhecida por \textunderscore erva dos gatos\textunderscore , (\textunderscore nepeta cataria\textunderscore , Lin.).
\section{Gatarrão}
\begin{itemize}
\item {Grp. gram.:m.}
\end{itemize}
Gato grande.
\section{Gatázio}
\begin{itemize}
\item {Grp. gram.:m.}
\end{itemize}
\begin{itemize}
\item {Utilização:Pop.}
\end{itemize}
\begin{itemize}
\item {Proveniência:(Do rad. de \textunderscore gato\textunderscore )}
\end{itemize}
Garra, unha; dedos.
\section{Gateado}
\begin{itemize}
\item {Grp. gram.:adj.}
\end{itemize}
\begin{itemize}
\item {Utilização:Bras. do S}
\end{itemize}
Diz-se do cavallo baio e do amarelo-avermelhado.
\section{Gatear}
\begin{itemize}
\item {Grp. gram.:v. t.}
\end{itemize}
\begin{itemize}
\item {Grp. gram.:V. i.}
\end{itemize}
\begin{itemize}
\item {Utilização:Fam.}
\end{itemize}
\begin{itemize}
\item {Proveniência:(De \textunderscore gato\textunderscore )}
\end{itemize}
Segurar com grampos ou gatos de ferro.
Consertar, segurando com gatos de metal.
Ralhar, contender.
\section{Gateira}
\begin{itemize}
\item {Grp. gram.:f.}
\end{itemize}
\begin{itemize}
\item {Utilização:Prov.}
\end{itemize}
\begin{itemize}
\item {Utilização:trasm.}
\end{itemize}
\begin{itemize}
\item {Utilização:Pop.}
\end{itemize}
\begin{itemize}
\item {Proveniência:(De \textunderscore gato\textunderscore )}
\end{itemize}
Buraco nas portas, para a passagem dos gatos.
Fresta, trapeira, sôbre o telhado, para entrar o ar e a luz.
Postigo no paiol, a bordo.
O mesmo que \textunderscore bueiro\textunderscore .
Bebedeira.
\section{Gateiro}
\begin{itemize}
\item {Grp. gram.:m.  e  adj.}
\end{itemize}
O que gosta de gatos.
Aquelle que deita gatos em loiça.
\section{Gatenho}
\begin{itemize}
\item {Grp. gram.:adj.}
\end{itemize}
\begin{itemize}
\item {Utilização:Ant.}
\end{itemize}
\begin{itemize}
\item {Utilização:Prov.}
\end{itemize}
\begin{itemize}
\item {Utilização:beir.}
\end{itemize}
Inculto, que está de poisio.
Estéril; que produziu pouco: \textunderscore lavoira gatenha\textunderscore .
\section{Gatesco}
\begin{itemize}
\item {fónica:tês}
\end{itemize}
\begin{itemize}
\item {Grp. gram.:adj.}
\end{itemize}
\begin{itemize}
\item {Utilização:Des.}
\end{itemize}
Próprio de gatos; relativo a gatos.
\section{Gatesgo}
\begin{itemize}
\item {fónica:tês}
\end{itemize}
\begin{itemize}
\item {Grp. gram.:adj.}
\end{itemize}
\begin{itemize}
\item {Grp. gram.:Loc. adv.}
\end{itemize}
\begin{itemize}
\item {Grp. gram.:Loc.}
\end{itemize}
\begin{itemize}
\item {Utilização:Loc. da Bairrada}
\end{itemize}
Próprio do gato.
\textunderscore Á gatesga\textunderscore , á maneira de gato. Cf. \textunderscore Anat. Joc.\textunderscore , I, 290.
Sem arte, desajeitadamente.
\section{Gaticida}
\begin{itemize}
\item {Grp. gram.:f.}
\end{itemize}
Aquelle que mata gatos.
(Cp. \textunderscore gaticídio\textunderscore )
\section{Gaticídio}
\begin{itemize}
\item {Grp. gram.:m.}
\end{itemize}
\begin{itemize}
\item {Proveniência:(Do lat. \textunderscore cattus\textunderscore  + \textunderscore caedere\textunderscore )}
\end{itemize}
Morte violenta de gato.
\section{Gatilho}
\begin{itemize}
\item {Grp. gram.:m.}
\end{itemize}
\begin{itemize}
\item {Proveniência:(De \textunderscore gato\textunderscore )}
\end{itemize}
Peça dos fechos de arma de fogo, a qual, tocada com o dedo, faz disparar a arma.
\section{Gatimanhos}
\begin{itemize}
\item {Grp. gram.:m. pl.}
\end{itemize}
\begin{itemize}
\item {Proveniência:(Do lat. \textunderscore catus\textunderscore  + \textunderscore manus\textunderscore ?)}
\end{itemize}
Gesticulação ridícula.
Sinaes, feitos com as mãos.
O mesmo que \textunderscore gatafunhos\textunderscore . Cf. \textunderscore Eufrosina\textunderscore , 181.
\section{Gatimonhos}
\begin{itemize}
\item {Grp. gram.:m. pl.}
\end{itemize}
\begin{itemize}
\item {Utilização:Bras. do N}
\end{itemize}
O mesmo que \textunderscore gatimanhos\textunderscore .
\section{Gatimónias}
\begin{itemize}
\item {Grp. gram.:f. pl.}
\end{itemize}
\begin{itemize}
\item {Utilização:Prov.}
\end{itemize}
\begin{itemize}
\item {Utilização:trasm.}
\end{itemize}
Momices.
Tropelias; barulho de crianças.
O mesmo que \textunderscore gatimanhos\textunderscore .
\section{Gatina}
\begin{itemize}
\item {Grp. gram.:f.}
\end{itemize}
\begin{itemize}
\item {Proveniência:(It. \textunderscore gattina\textunderscore , pequeno gato)}
\end{itemize}
Doença peculiar dos bichos da seda.
\section{Gatinha}
\begin{itemize}
\item {Grp. gram.:f.}
\end{itemize}
\begin{itemize}
\item {Grp. gram.:Pl.}
\end{itemize}
Pequena gata.
\textunderscore Andar de gatinhas\textunderscore , acto de andar, pondo as mãos no chão.
\section{Gatinhar}
\begin{itemize}
\item {Grp. gram.:v. i.}
\end{itemize}
O mesmo que \textunderscore engatinhar\textunderscore . Cf. Eça, \textunderscore P. Basílio\textunderscore , 70.
\section{Gatinho}
\begin{itemize}
\item {Grp. gram.:m.}
\end{itemize}
\begin{itemize}
\item {Utilização:T. de Aveiro}
\end{itemize}
Flôr de salgueiro.
\section{Gato}
\begin{itemize}
\item {Grp. gram.:m.}
\end{itemize}
\begin{itemize}
\item {Utilização:Náut.}
\end{itemize}
\begin{itemize}
\item {Utilização:Prov.}
\end{itemize}
\begin{itemize}
\item {Utilização:Prov.}
\end{itemize}
\begin{itemize}
\item {Utilização:trasm.}
\end{itemize}
\begin{itemize}
\item {Utilização:Prov.}
\end{itemize}
\begin{itemize}
\item {Utilização:alent.}
\end{itemize}
\begin{itemize}
\item {Utilização:Prov.}
\end{itemize}
\begin{itemize}
\item {Utilização:alent.}
\end{itemize}
\begin{itemize}
\item {Utilização:Bras. do N}
\end{itemize}
\begin{itemize}
\item {Utilização:Fig.}
\end{itemize}
\begin{itemize}
\item {Proveniência:(Do lat. \textunderscore cattus\textunderscore )}
\end{itemize}
Animal doméstico, da ordem dos carnívoros digitígrados, (\textunderscore felis catus\textunderscore ).
Grampo.
Pedaço de metal, que prende e liga loiça quebrada ou rachada.
Excesso de carne na parte superior do pescoço dos cavallos.
Utensílio de tanoeiro, para arquear as vasilhas.
Peça de ferro, com que se endireitam as aduelas das pipas.
Gancho de ferro, na extremidade de um cabo, moitão ou cadernal, para suspender alguma coisa.
Pedaço de metal, que liga duas peças de cantaria.
Espécie de fivela, dentro da qual se ergue e se abaixa o braço ou tranqueta da aldrava.
Êrro, engano, lapso. (Colhido em Penafiel)
O mesmo que \textunderscore mentira\textunderscore .
Pequena pelle, preparada á semelhança e com o feitio de odre, para levar vinho, como se fosse borracha.
O mesmo que \textunderscore zápete\textunderscore , o quatro de paus, no jôgo do truque.
Pequena reprehensão.
Restos de fazenda, que os alfaiates e costureiras guardam para si, depois de feita a obra encommendada.
\section{Gato-do-mato}
\begin{itemize}
\item {Grp. gram.:m.}
\end{itemize}
\begin{itemize}
\item {Utilização:Bras}
\end{itemize}
O mesmo que \textunderscore maracajá\textunderscore .
\section{Gato-pingado}
\begin{itemize}
\item {Grp. gram.:m.}
\end{itemize}
\begin{itemize}
\item {Utilização:Pop.}
\end{itemize}
Indivíduo, que acompanha os enterros a pé, com tocha ou archote na mão.
\section{Gatorro}
\begin{itemize}
\item {fónica:tô}
\end{itemize}
\begin{itemize}
\item {Grp. gram.:m.}
\end{itemize}
O mesmo que \textunderscore gatarrão\textunderscore .
\section{Gato-sapato}
\begin{itemize}
\item {Grp. gram.:m.}
\end{itemize}
\begin{itemize}
\item {Utilização:Fam.}
\end{itemize}
Coisa desprezível.
\textunderscore Fazer gato-sapato de alguém\textunderscore , tratar com desprêzo, fazer joguete de alguém; maltratá-lo.
\section{Gatum}
\begin{itemize}
\item {Grp. gram.:adj.}
\end{itemize}
\begin{itemize}
\item {Utilização:P. us.}
\end{itemize}
Relativo a gato.
\section{Gatunagem}
\begin{itemize}
\item {Grp. gram.:f.}
\end{itemize}
Porção de gatunos; os gatunos; vida de gatuno.
\section{Gatunar}
\begin{itemize}
\item {Grp. gram.:v. i.}
\end{itemize}
Gandaiar, levar vida de gatuno; furtar; larapiar por hábito.
\section{Gatunice}
\begin{itemize}
\item {Grp. gram.:f.}
\end{itemize}
Acto próprio de gatuno; furto.
\section{Gatuno}
\begin{itemize}
\item {Grp. gram.:m.  e  adj.}
\end{itemize}
\begin{itemize}
\item {Proveniência:(Do rad. de \textunderscore gato\textunderscore )}
\end{itemize}
Vadio; larápio; ratoneiro.
\section{Gaturamo}
\begin{itemize}
\item {Grp. gram.:m.}
\end{itemize}
Pequena ave do Brasil.
\section{Gaturar}
\begin{itemize}
\item {Grp. gram.:v. t.}
\end{itemize}
\begin{itemize}
\item {Utilização:Bras. de Minas}
\end{itemize}
\begin{itemize}
\item {Proveniência:(De \textunderscore capturar\textunderscore ?)}
\end{itemize}
Prender.
Furtar.
\section{Gaturda}
\begin{itemize}
\item {Grp. gram.:f.}
\end{itemize}
Antiga música popular, para viola.
\section{Gau}
\begin{itemize}
\item {Grp. gram.:m.}
\end{itemize}
\begin{itemize}
\item {Utilização:Gír.}
\end{itemize}
Piolho, o mesmo que \textunderscore ganau\textunderscore .
\section{Gauchaça}
\begin{itemize}
\item {fónica:ga-u}
\end{itemize}
\begin{itemize}
\item {Grp. gram.:m.}
\end{itemize}
\begin{itemize}
\item {Utilização:Bras. do S}
\end{itemize}
Gaúcho perfeito, completo.
\section{Gauchada}
\begin{itemize}
\item {fónica:ga-u}
\end{itemize}
\begin{itemize}
\item {Grp. gram.:f.}
\end{itemize}
\begin{itemize}
\item {Utilização:Bras. do S}
\end{itemize}
Acto próprio de gaúcho.
\section{Gauchar}
\begin{itemize}
\item {fónica:ga-u}
\end{itemize}
\begin{itemize}
\item {Grp. gram.:v. i.}
\end{itemize}
\begin{itemize}
\item {Utilização:Bras. do S}
\end{itemize}
Praticar o gaúcho os seus costumes; viver como gaúcho.
\section{Gaúcho}
\begin{itemize}
\item {Grp. gram.:m.}
\end{itemize}
\begin{itemize}
\item {Utilização:Bras. do S}
\end{itemize}
\begin{itemize}
\item {Utilização:Bras. do S}
\end{itemize}
Habitante dos campos, em geral procedente de portugueses ou espanhoes, dedicado á criação de gado vacum e cavallar, e notável por seu valor e agilidade.
Animal ou qualquer objecto sem dono.
\section{Gauda}
\begin{itemize}
\item {Grp. gram.:f.}
\end{itemize}
\begin{itemize}
\item {Proveniência:(Do al. \textunderscore waude\textunderscore )}
\end{itemize}
Planta tinctorial, espécie de reseda, (\textunderscore reseda luteola\textunderscore , Lin.).
\section{Gaudar}
\begin{itemize}
\item {Grp. gram.:v. i.}
\end{itemize}
\begin{itemize}
\item {Utilização:ant.}
\end{itemize}
\begin{itemize}
\item {Utilização:Pop.}
\end{itemize}
Guardar gado, apascentar gado.
(Corr. de \textunderscore guardar\textunderscore ? Ou der. de \textunderscore gaudo\textunderscore , por \textunderscore gado\textunderscore ?)
\section{Gaudério}
\begin{itemize}
\item {Grp. gram.:m.}
\end{itemize}
\begin{itemize}
\item {Utilização:Chul.}
\end{itemize}
\begin{itemize}
\item {Utilização:Bras. do N}
\end{itemize}
\begin{itemize}
\item {Proveniência:(De \textunderscore gáudio\textunderscore )}
\end{itemize}
Malandro; vadio.
Folgança; pândega.
Ave, de pennas negras e brilhantes.
\section{Gaudinar}
\begin{itemize}
\item {Grp. gram.:v. i.}
\end{itemize}
\begin{itemize}
\item {Utilização:Gír.}
\end{itemize}
\begin{itemize}
\item {Proveniência:(De \textunderscore gáudio\textunderscore )}
\end{itemize}
Andar na pândega; estroinar.
\section{Gáudio}
\begin{itemize}
\item {Grp. gram.:m.}
\end{itemize}
\begin{itemize}
\item {Proveniência:(Lat. \textunderscore gaudium\textunderscore )}
\end{itemize}
Júbilo.
Brinquedo; folgança.
\section{Gaudioso}
\begin{itemize}
\item {Grp. gram.:adj.}
\end{itemize}
Que tem gáudio; revelador de gáudio. Cf. Filinto, IX, 119.
\section{Gaudipério}
\begin{itemize}
\item {Grp. gram.:m.}
\end{itemize}
\begin{itemize}
\item {Utilização:ant.}
\end{itemize}
\begin{itemize}
\item {Utilização:Gír.}
\end{itemize}
Injúria, que se faz a a um indivíduo, tomando relações illícitas com a mulher ou a amante delle.
\section{Gaulês}
\begin{itemize}
\item {Grp. gram.:adj.}
\end{itemize}
\begin{itemize}
\item {Utilização:Gal}
\end{itemize}
\begin{itemize}
\item {Grp. gram.:M.}
\end{itemize}
\begin{itemize}
\item {Proveniência:(Fr. \textunderscore gaulois\textunderscore , de \textunderscore Gaule\textunderscore , n. p.)}
\end{itemize}
Relativo á Gállia.
Habitante da Gállia.
Idioma dos antigos Gauleses, dialecto das línguas célticas. Cf. Latino, \textunderscore Elogios\textunderscore , 71.
\section{Gaulo}
\begin{itemize}
\item {Grp. gram.:m.}
\end{itemize}
\begin{itemize}
\item {Proveniência:(Gr. \textunderscore gaulos\textunderscore , do phenício)}
\end{itemize}
Embarcação phenícia, quási redonda.
Copo ou vaso, em fórma de navio.
\section{Gaura}
\begin{itemize}
\item {Grp. gram.:f.}
\end{itemize}
\begin{itemize}
\item {Proveniência:(Do gr. \textunderscore gauros\textunderscore )}
\end{itemize}
Gênero de plantas onotheráceas.
\section{Gáurio}
\begin{itemize}
\item {Grp. gram.:adj.}
\end{itemize}
\begin{itemize}
\item {Utilização:Philol.}
\end{itemize}
Diz-se dos idiomas derivados de antigos prácritos, e portanto aparentados com o sânscrito, e como êste pertencentes á família árica ou indo-europeia. Cf. G. Viana, \textunderscore Apostilas\textunderscore , vb. \textunderscore mosteiro\textunderscore .
\section{Gavarro}
\begin{itemize}
\item {Grp. gram.:m.}
\end{itemize}
(V.unheiro)
\section{Gávea}
\begin{itemize}
\item {Grp. gram.:f.}
\end{itemize}
\begin{itemize}
\item {Utilização:Náut.}
\end{itemize}
\begin{itemize}
\item {Grp. gram.:Pl.}
\end{itemize}
\begin{itemize}
\item {Proveniência:(Lat. \textunderscore cavea\textunderscore )}
\end{itemize}
Espécie de tabuleiro ou plataforma, a certa altura de um mastro e atravessada por elle.
Vela, immediatamente superior á vela grande.
Conjunto das três velas das galeras.
A gávea e velacho, nos brigues.
\section{Gaveador}
\begin{itemize}
\item {Grp. gram.:m.}
\end{itemize}
Aquelle que gaveia.
\section{Gavear}
\begin{itemize}
\item {Grp. gram.:v. i.}
\end{itemize}
\begin{itemize}
\item {Utilização:Prov.}
\end{itemize}
\begin{itemize}
\item {Utilização:trasm.}
\end{itemize}
Plantar bacêllo. Cf. Deusdado, \textunderscore Escorços Trasm.\textunderscore , 195.
\section{Gavela}
\begin{itemize}
\item {Grp. gram.:f.}
\end{itemize}
\begin{itemize}
\item {Proveniência:(Do cast. \textunderscore gavilla\textunderscore )}
\end{itemize}
Feixe de espigas.
Paveia.
Braçado; arregaçada.
\section{Gaveta}
\begin{itemize}
\item {fónica:vê}
\end{itemize}
\begin{itemize}
\item {Grp. gram.:f.}
\end{itemize}
\begin{itemize}
\item {Utilização:Des.}
\end{itemize}
\begin{itemize}
\item {Utilização:Gír.}
\end{itemize}
\begin{itemize}
\item {Proveniência:(Do lat. \textunderscore gabata\textunderscore ?)}
\end{itemize}
Caixa corrediça, geralmente sem tampa, e que se embebe em papeleira, cômmoda, etc.
Rosca dos instrumento:«\textunderscore mais cheia de gavetas que a trompa de...\textunderscore »Camões, \textunderscore Seleuco\textunderscore .
Prisão.
\section{Gavetão}
\begin{itemize}
\item {Grp. gram.:m.}
\end{itemize}
\begin{itemize}
\item {Proveniência:(De \textunderscore gaveta\textunderscore )}
\end{itemize}
Gaveta grande.
Peça que, nas máquinas de vapor, regula a distribuição dêste.
\section{Gaveto}
\begin{itemize}
\item {fónica:vê}
\end{itemize}
\begin{itemize}
\item {Grp. gram.:m.}
\end{itemize}
Peça de madeira, convexa ou côncava, em diversos trabalhos de carpintaria.
\section{Gave-tope}
\begin{itemize}
\item {Grp. gram.:m.}
\end{itemize}
\begin{itemize}
\item {Utilização:Náut.}
\end{itemize}
\begin{itemize}
\item {Proveniência:(Do ingl. \textunderscore gaf-top\textunderscore )}
\end{itemize}
Vela, que se usa em mastros, que não têm vêrgas.
\section{Gavial}
\begin{itemize}
\item {Grp. gram.:m.}
\end{itemize}
Grande crocodilo de Ganges, (\textunderscore lacerta gangetica\textunderscore ).
\section{Gavião}
\begin{itemize}
\item {Grp. gram.:m.}
\end{itemize}
\begin{itemize}
\item {Utilização:Agr.}
\end{itemize}
\begin{itemize}
\item {Utilização:T. da Bairrada}
\end{itemize}
\begin{itemize}
\item {Utilização:Carp.}
\end{itemize}
\begin{itemize}
\item {Utilização:Prov.}
\end{itemize}
\begin{itemize}
\item {Utilização:trasm.}
\end{itemize}
\begin{itemize}
\item {Grp. gram.:Adj.}
\end{itemize}
\begin{itemize}
\item {Utilização:Bras. do S}
\end{itemize}
\begin{itemize}
\item {Proveniência:(Do cast. \textunderscore gavilán\textunderscore )}
\end{itemize}
Pequena ave de rapina, (\textunderscore falco nisus\textunderscore ).
Gavinha.
Cada um dos dois últimos dentes de cada lado da maxilla superior do cavallo.
Parte da estribeira, conto.
Orelha, pêta ou bico do sacho. Cf. Aguiar, \textunderscore Proc. de Vin.\textunderscore , 11.
Parte curva e cortante da podôa e da tesoira de podar.
Cada uma das extremidades do gume de um formão ou de outros instrumentos.
Espécie de jôgo de rapazes, em que um, imitando o gavião, persegue os outros, que fazem de pombas.
Diz-se do cavallo, que se não deixa apanhar facilmente.
Vivo, finório.
\section{Gaviete}
\begin{itemize}
\item {fónica:ê}
\end{itemize}
\begin{itemize}
\item {Grp. gram.:m.}
\end{itemize}
\begin{itemize}
\item {Utilização:Náut.}
\end{itemize}
Peça de madeira, que se colloca na popa de uma lancha, para receber a amarra e suspender a âncora.
(Cast. \textunderscore gabiete\textunderscore )
\section{Gavina}
\begin{itemize}
\item {Grp. gram.:f.}
\end{itemize}
\begin{itemize}
\item {Utilização:Prov.}
\end{itemize}
\begin{itemize}
\item {Utilização:dur.}
\end{itemize}
Espécie de podôa sem pêta, usada entre os podadores do Doiro.
\section{Gavinha}
\begin{itemize}
\item {Grp. gram.:f.}
\end{itemize}
Appêndice filamentoso, ás vezes em espiral, com que as plantas sarmentosas e trepadeiras se fixam noutras plantas próximas ou em outros objectos circunjacentes.
Elo, abraço.
\section{Gavinhas}
\begin{itemize}
\item {Grp. gram.:f. pl.}
\end{itemize}
Appêndice filamentoso, ás vezes em espiral, com que as plantas sarmentosas e trepadeiras se fixam noutras plantas próximas ou em outros objectos circunjacentes.
Elo, abraço.
\section{Gavinhoso}
\begin{itemize}
\item {Grp. gram.:adj.}
\end{itemize}
Que tem gavinhas.
\section{Gavionar}
\begin{itemize}
\item {Grp. gram.:v. i.}
\end{itemize}
\begin{itemize}
\item {Utilização:Bras. do S}
\end{itemize}
\begin{itemize}
\item {Proveniência:(De \textunderscore gavião\textunderscore )}
\end{itemize}
Não se deixar apanhar (o cavallo).
\section{Gaviroba}
\begin{itemize}
\item {Grp. gram.:f.}
\end{itemize}
(V.gabiroba)
\section{Gavito}
\begin{itemize}
\item {Grp. gram.:m.}
\end{itemize}
O mesmo que \textunderscore garão\textunderscore .
\section{Gavota}
\begin{itemize}
\item {Grp. gram.:f.}
\end{itemize}
\begin{itemize}
\item {Proveniência:(Fr. \textunderscore gavotte\textunderscore )}
\end{itemize}
Antiga dança francesa, que se vulgarizou entre nós.
Música para essa dança.
\section{Gaxeta}
\begin{itemize}
\item {fónica:xê}
\end{itemize}
\begin{itemize}
\item {Grp. gram.:f.}
\end{itemize}
Tranças de fio de carrêta para ferrar amarras.
Cinta para ferrar velas nas vêrgas.
Trança de linho ou palha, e, ás vezes, de borracha, que se colloca apertada entre os bordos da tampa e a bôca das caldeiras de qualquer máquina, para se fecharem hermeticamente.
(Do genovês \textunderscore gasseta\textunderscore )
\section{Gaz}
\textunderscore m.\textunderscore  (e der.)
(V. \textunderscore gás\textunderscore , etc.)
\section{Gaza}
\begin{itemize}
\item {Grp. gram.:f.}
\end{itemize}
\begin{itemize}
\item {Proveniência:(De \textunderscore Gaza\textunderscore , n. p.)}
\end{itemize}
Tecido leve e transparente.
Pequena moéda de cobre na Pérsia.
\section{Gazânia}
\begin{itemize}
\item {Grp. gram.:f.}
\end{itemize}
Gênero de plantas compostas.
(Do persa \textunderscore gaza\textunderscore , riqueza)
\section{Gazão}
\begin{itemize}
\item {Grp. gram.:m.}
\end{itemize}
\begin{itemize}
\item {Utilização:Neol.}
\end{itemize}
\begin{itemize}
\item {Proveniência:(Fr. \textunderscore gazon\textunderscore )}
\end{itemize}
Relva de jardim.
Terreno coberto de relva.
\section{Gazeador}
\begin{itemize}
\item {Grp. gram.:m.  e  adj.}
\end{itemize}
O que gazeia.
\section{Gazeante}
\begin{itemize}
\item {Grp. gram.:adj.}
\end{itemize}
Que gazeia.
\section{Gazear}
\begin{itemize}
\item {Grp. gram.:v. i.}
\end{itemize}
\begin{itemize}
\item {Grp. gram.:V. t.}
\end{itemize}
Faltar ás aulas, para se divertir ou vadiar.
Faltar á (escola):«\textunderscore um pagem que nos deixava gazear a escola...\textunderscore »M. Assis, \textunderscore Brás Cubas\textunderscore .
\section{Gazear}
\begin{itemize}
\item {Grp. gram.:v. i.}
\end{itemize}
\begin{itemize}
\item {Utilização:Prov.}
\end{itemize}
\begin{itemize}
\item {Utilização:minh.}
\end{itemize}
\begin{itemize}
\item {Proveniência:(Do fr. \textunderscore gazouiller\textunderscore ?)}
\end{itemize}
Cantar (a garça, a andorinha, etc.).
Chilrear; gorgear.
Chalrar, (falando-se de crianças).
\section{Gazeio}
\begin{itemize}
\item {Grp. gram.:m.}
\end{itemize}
Acto de gazear^1.
\section{Gazeio}
\begin{itemize}
\item {Grp. gram.:m.}
\end{itemize}
\begin{itemize}
\item {Proveniência:(De \textunderscore gazear\textunderscore ^2)}
\end{itemize}
Canto da garça, da andorinha e de outras aves.
\section{Gazela}
\begin{itemize}
\item {Grp. gram.:f.}
\end{itemize}
\begin{itemize}
\item {Proveniência:(Do ár. \textunderscore gazel\textunderscore )}
\end{itemize}
Animal do gênero dos antílopes, que anda em bandos e habita na Ásia e na África, (\textunderscore antilope dorcade\textunderscore ).
\section{Gazena}
\begin{itemize}
\item {Grp. gram.:f.}
\end{itemize}
Moéda da Índia.
\section{Gázeo}
\begin{itemize}
\item {Grp. gram.:adj.}
\end{itemize}
\begin{itemize}
\item {Grp. gram.:M. pl.}
\end{itemize}
\begin{itemize}
\item {Utilização:Pop.}
\end{itemize}
O mesmo que \textunderscore garço\textunderscore .
Olhos.
\section{Gázeo}
\begin{itemize}
\item {Grp. gram.:m.}
\end{itemize}
O mesmo que \textunderscore gazeio\textunderscore ^1:«\textunderscore na classe pespeguei valentes gázios.\textunderscore »\textunderscore Hyssope\textunderscore , V, 70.
\section{Gazeta}
\begin{itemize}
\item {fónica:zê}
\end{itemize}
\begin{itemize}
\item {Grp. gram.:f.}
\end{itemize}
\begin{itemize}
\item {Utilização:Fam.}
\end{itemize}
\begin{itemize}
\item {Proveniência:(It. \textunderscore gazzetta\textunderscore , do nome de uma moéda veneziana, segundo a opinião mais provável)}
\end{itemize}
Publicação periódica de artigos políticos ou doutrinários, ou de notícias de qualquer espécie.
Acto de gazear^1.
\section{Gazetal}
\begin{itemize}
\item {Grp. gram.:adj.}
\end{itemize}
\begin{itemize}
\item {Utilização:Burl.}
\end{itemize}
Relativo a gazetas. Cf. Macedo, \textunderscore Burros\textunderscore , 254.
\section{Gazetário}
\begin{itemize}
\item {Grp. gram.:adj.}
\end{itemize}
O mesmo que \textunderscore gazetal\textunderscore . Cf. \textunderscore Anat. Joc.\textunderscore , pról.
\section{Gazetear}
\begin{itemize}
\item {Grp. gram.:v. i.}
\end{itemize}
\begin{itemize}
\item {Proveniência:(De \textunderscore gazeta\textunderscore )}
\end{itemize}
O mesmo que \textunderscore gazear\textunderscore ^1.
\section{Gazeteiro}
\begin{itemize}
\item {Grp. gram.:m.}
\end{itemize}
\begin{itemize}
\item {Utilização:Deprec.}
\end{itemize}
\begin{itemize}
\item {Proveniência:(De \textunderscore gazeta\textunderscore )}
\end{itemize}
Jornalista; noticiarista.
Estudante que gazeia.
\section{Gazetilha}
\begin{itemize}
\item {Grp. gram.:f.}
\end{itemize}
\begin{itemize}
\item {Proveniência:(De \textunderscore gazeta\textunderscore )}
\end{itemize}
Secção jocosa ou satírica de algumas fôlhas periódicas, feita geralmente em verso.
\section{Gazetilheiro}
\begin{itemize}
\item {Grp. gram.:m.}
\end{itemize}
O mesmo que \textunderscore gazetilhista\textunderscore .
\section{Gazetilhista}
\begin{itemize}
\item {Grp. gram.:m.}
\end{itemize}
Aquelle que faz gazetilhas.
\section{Gazetismo}
\begin{itemize}
\item {Grp. gram.:m.}
\end{itemize}
\begin{itemize}
\item {Utilização:Neol.}
\end{itemize}
\begin{itemize}
\item {Proveniência:(De \textunderscore gazeta\textunderscore )}
\end{itemize}
Influência ou dominio, exercido pelos periódicos.
\section{Gazetista}
\begin{itemize}
\item {Grp. gram.:m.}
\end{itemize}
\begin{itemize}
\item {Utilização:P. us.}
\end{itemize}
\begin{itemize}
\item {Proveniência:(De \textunderscore gazeta\textunderscore )}
\end{itemize}
Jornalista, periodista.
\section{Gazia}
\begin{itemize}
\item {Grp. gram.:f.}
\end{itemize}
O mesmo que \textunderscore gaziva\textunderscore .
\section{Gazil}
\begin{itemize}
\item {Grp. gram.:adj.}
\end{itemize}
\begin{itemize}
\item {Utilização:Prov.}
\end{itemize}
\begin{itemize}
\item {Utilização:alg.}
\end{itemize}
\begin{itemize}
\item {Utilização:alent.}
\end{itemize}
Elegante; airoso; bem pôsto.
(Alter. morphológica e phonética de \textunderscore grácil\textunderscore ?)
\section{Gázio}
\begin{itemize}
\item {Grp. gram.:m.}
\end{itemize}
\begin{itemize}
\item {Utilização:Pop.}
\end{itemize}
O mesmo que \textunderscore engaço\textunderscore .
(Cp. \textunderscore ázeo\textunderscore )
\section{Gaziva}
\begin{itemize}
\item {Grp. gram.:f.}
\end{itemize}
\begin{itemize}
\item {Proveniência:(Do ár. \textunderscore gazaua\textunderscore )}
\end{itemize}
Expedição de árabes; cruzada.
\section{Gazofilácio}
\begin{itemize}
\item {Grp. gram.:m.}
\end{itemize}
\begin{itemize}
\item {Utilização:Ext.}
\end{itemize}
\begin{itemize}
\item {Proveniência:(Gr. \textunderscore gazophulakion\textunderscore )}
\end{itemize}
Lugar, em que, no templo, se guardavam os vasos e se recolhiam as oferendas.
Tesoiro.
\section{Gazofilar}
\begin{itemize}
\item {Grp. gram.:v. t.}
\end{itemize}
\begin{itemize}
\item {Utilização:Pleb.}
\end{itemize}
Prender, agarrar.
Surripiar.
\section{Gazola}
\begin{itemize}
\item {Grp. gram.:f.}
\end{itemize}
O mesmo que \textunderscore alcaravão\textunderscore .
\section{Gazola}
\begin{itemize}
\item {Grp. gram.:f.}
\end{itemize}
\begin{itemize}
\item {Utilização:Prov.}
\end{itemize}
Círculo, que se traça no chão e dentro do qual os rapazes fazem girar o pião, jogando.
\section{Gazola}
\begin{itemize}
\item {Grp. gram.:f.}
\end{itemize}
\begin{itemize}
\item {Utilização:T. de Turquel}
\end{itemize}
Garganta.
Voz forte e destemperada.
\section{Gazolar}
\begin{itemize}
\item {Grp. gram.:v. t.}
\end{itemize}
\begin{itemize}
\item {Utilização:Prov.}
\end{itemize}
\begin{itemize}
\item {Proveniência:(De \textunderscore gazola\textunderscore ^2)}
\end{itemize}
O mesmo que \textunderscore marricar\textunderscore .
\section{Gazophylácio}
\begin{itemize}
\item {Grp. gram.:m.}
\end{itemize}
\begin{itemize}
\item {Utilização:Ext.}
\end{itemize}
\begin{itemize}
\item {Proveniência:(Gr. \textunderscore gazophulakion\textunderscore )}
\end{itemize}
Lugar, em que, no templo, se guardavam os vasos e se recolhiam as offerendas.
Thesoiro.
\section{Gazopo}
\begin{itemize}
\item {fónica:zô}
\end{itemize}
\begin{itemize}
\item {Grp. gram.:m.}
\end{itemize}
\begin{itemize}
\item {Utilização:Prov.}
\end{itemize}
\begin{itemize}
\item {Utilização:alent.}
\end{itemize}
Cão pequeno.
\section{Gazua}
\begin{itemize}
\item {Grp. gram.:f.}
\end{itemize}
\begin{itemize}
\item {Utilização:Ant.}
\end{itemize}
O mesmo que \textunderscore gaziva\textunderscore .
\section{Gazua}
\begin{itemize}
\item {Grp. gram.:f.}
\end{itemize}
\begin{itemize}
\item {Proveniência:(Do cast. \textunderscore ganzua\textunderscore )}
\end{itemize}
Chave falsa.
Ferro ou instrumento curvo, com que se podem abrir fechaduras.
\section{Gazula}
\begin{itemize}
\item {Grp. gram.:f.}
\end{itemize}
\begin{itemize}
\item {Utilização:Pop.}
\end{itemize}
O mesmo que \textunderscore gaziva\textunderscore .
\section{Gazupar}
\begin{itemize}
\item {Grp. gram.:v. t.}
\end{itemize}
\begin{itemize}
\item {Utilização:Des.}
\end{itemize}
O mesmo que \textunderscore engazupar\textunderscore .
\section{Gê}
\begin{itemize}
\item {Grp. gram.:m.}
\end{itemize}
Designação da letra \textunderscore g\textunderscore .
\section{Gê}
\begin{itemize}
\item {Grp. gram.:m.}
\end{itemize}
Língua sul-americana, que alguns suppõem fusão do abanheenga com o aimará.
\section{Geada}
\begin{itemize}
\item {Grp. gram.:f.}
\end{itemize}
\begin{itemize}
\item {Utilização:Ext.}
\end{itemize}
\begin{itemize}
\item {Proveniência:(Do lat. \textunderscore gelata\textunderscore )}
\end{itemize}
Orvalho congelado, que fórma camada branca sôbre o solo, telhados, plantas, etc.
Frio excessivo.
\section{Geado}
\begin{itemize}
\item {Grp. gram.:m.}
\end{itemize}
\begin{itemize}
\item {Utilização:Prov.}
\end{itemize}
\begin{itemize}
\item {Utilização:alg.}
\end{itemize}
Taínha, ainda nova, (\textunderscore mugil auratus\textunderscore , Risso).
\section{Gear}
\begin{itemize}
\item {Grp. gram.:v. t.}
\end{itemize}
\begin{itemize}
\item {Grp. gram.:V. i.}
\end{itemize}
\begin{itemize}
\item {Utilização:Ext.}
\end{itemize}
\begin{itemize}
\item {Proveniência:(Do lat. \textunderscore gelare\textunderscore )}
\end{itemize}
Reduzir a gêlo, congelar.
Formar-se geada.
Baixar excessivamente a temperatura: \textunderscore hoje, geou muito\textunderscore .
\section{Gêba}
\begin{itemize}
\item {Grp. gram.:f.}
\end{itemize}
\begin{itemize}
\item {Utilização:Gír.}
\end{itemize}
\begin{itemize}
\item {Utilização:Ant.}
\end{itemize}
O mesmo que \textunderscore gibba\textunderscore .
Mãe velha.
Mulher velha e corcunda.
(Cp. \textunderscore gêbo\textunderscore )
\section{Gebada}
\begin{itemize}
\item {Grp. gram.:f.}
\end{itemize}
\begin{itemize}
\item {Utilização:Pop.}
\end{itemize}
Acto de gebar.
\section{Gebadoira}
\begin{itemize}
\item {Grp. gram.:f.}
\end{itemize}
Instrumento, com que se fazem nas aduelas os encaixes para os tampos.
(Por \textunderscore javradoira\textunderscore , de \textunderscore javrar\textunderscore )
\section{Gebadoura}
\begin{itemize}
\item {Grp. gram.:f.}
\end{itemize}
Instrumento, com que se fazem nas aduelas os encaixes para os tampos.
(Por \textunderscore javradoira\textunderscore , de \textunderscore javrar\textunderscore )
\section{Gebar}
\begin{itemize}
\item {Grp. gram.:v. t.}
\end{itemize}
\begin{itemize}
\item {Utilização:Pop.}
\end{itemize}
\begin{itemize}
\item {Proveniência:(De \textunderscore gêbo\textunderscore )}
\end{itemize}
Amachucar com pancadas (o chapéu).
\section{Gebice}
\begin{itemize}
\item {Grp. gram.:f.}
\end{itemize}
Acto ou modos de gêbo.
\section{Gêbo}
\begin{itemize}
\item {Grp. gram.:adj.}
\end{itemize}
\begin{itemize}
\item {Utilização:Pop.}
\end{itemize}
\begin{itemize}
\item {Grp. gram.:M.}
\end{itemize}
\begin{itemize}
\item {Utilização:ant.}
\end{itemize}
\begin{itemize}
\item {Utilização:Gír.}
\end{itemize}
\begin{itemize}
\item {Proveniência:(Do lat. \textunderscore gibbus\textunderscore )}
\end{itemize}
O mesmo que \textunderscore gibboso\textunderscore .
Mal trajado e sujo.
Farroupilha.
Indivíduo mal vestido.
Espécie de boi indiano, com grande corcova sôbre as espáduas.
Indivíduo velho.
\section{Gebrar}
\begin{itemize}
\item {Grp. gram.:v. t.}
\end{itemize}
Fazer gebre em (aduelas).
\section{Gebre}
\begin{itemize}
\item {Grp. gram.:m.}
\end{itemize}
\begin{itemize}
\item {Utilização:Prov.}
\end{itemize}
Friso na extremidade interior das aduelas, onde se encaixam as extremidades dos tampos.
(Cp. \textunderscore javre\textunderscore )
\section{Gebreira}
\begin{itemize}
\item {Grp. gram.:f.}
\end{itemize}
\begin{itemize}
\item {Utilização:Prov.}
\end{itemize}
\begin{itemize}
\item {Utilização:minh.}
\end{itemize}
Pândega.
Estroinice.
Patuscada.
\section{Gedrite}
\begin{itemize}
\item {Grp. gram.:f.}
\end{itemize}
\begin{itemize}
\item {Utilização:Miner.}
\end{itemize}
Silicato hydratado de alumina, ferro e magnésia, que se encontra nos Pyrenéus.
\section{Geeiro}
\begin{itemize}
\item {Grp. gram.:adj.}
\end{itemize}
\begin{itemize}
\item {Utilização:Prov.}
\end{itemize}
\begin{itemize}
\item {Utilização:trasm.}
\end{itemize}
\begin{itemize}
\item {Proveniência:(De \textunderscore gear\textunderscore )}
\end{itemize}
Que traz ou annuncia geada: \textunderscore vento geeiro\textunderscore .
\section{Geena}
\begin{itemize}
\item {Grp. gram.:f.}
\end{itemize}
\begin{itemize}
\item {Proveniência:(Lat. eccles. \textunderscore gehenna\textunderscore )}
\end{itemize}
Lugar de tormento eterno, pelo fogo.
Inferno.
\section{Geez}
\begin{itemize}
\item {Grp. gram.:m.}
\end{itemize}
A língua da Ethiópia; o ethíope.
\section{Gegé}
\begin{itemize}
\item {Grp. gram.:m.}
\end{itemize}
\begin{itemize}
\item {Utilização:Bras. do N}
\end{itemize}
Prisão; calaboiço.
\section{Gegelado}
\begin{itemize}
\item {Grp. gram.:adj.}
\end{itemize}
(V.agegelado)
\section{Gehenna}
\begin{itemize}
\item {Grp. gram.:f.}
\end{itemize}
\begin{itemize}
\item {Proveniência:(Lat. eccles. \textunderscore gehenna\textunderscore )}
\end{itemize}
Lugar de tormento eterno, pelo fogo.
Inferno.
\section{Geigéria}
\begin{itemize}
\item {Grp. gram.:f.}
\end{itemize}
\begin{itemize}
\item {Proveniência:(De \textunderscore Geiger\textunderscore , n. p.)}
\end{itemize}
Gênero de plantas, da fam. das compostas.
\section{Geio}
\begin{itemize}
\item {Grp. gram.:m.}
\end{itemize}
\begin{itemize}
\item {Utilização:Prov.}
\end{itemize}
\begin{itemize}
\item {Utilização:Prov.}
\end{itemize}
\begin{itemize}
\item {Utilização:dur.}
\end{itemize}
Terreno, entre dois muros ou degraus, para plantação de bacêllo.
Cada um dos arretos, que sustentam terras em socalco; botaréu.
\section{Geio}
\begin{itemize}
\item {Grp. gram.:m.}
\end{itemize}
Acto de gear.
Gêlo.
\section{Geira}
\begin{itemize}
\item {Grp. gram.:f.}
\end{itemize}
\begin{itemize}
\item {Utilização:Ant.}
\end{itemize}
\begin{itemize}
\item {Grp. gram.:Loc. adv.}
\end{itemize}
\begin{itemize}
\item {Utilização:trasm}
\end{itemize}
\begin{itemize}
\item {Utilização:Prov.}
\end{itemize}
\begin{itemize}
\item {Utilização:alent.}
\end{itemize}
\begin{itemize}
\item {Proveniência:(Do lat. \textunderscore diaria\textunderscore )}
\end{itemize}
Antiga medida agrária.
Coirela, belga, leira.
Terreno, que uma junta de bois póde lavrar num dia.
Contribuição do serviço de lavoira.
\textunderscore Á geira\textunderscore , á jorna, a dias.
Porção de terreno, em que podem semear-se quatro alqueires de trigo.
\section{Geirão}
\begin{itemize}
\item {Grp. gram.:m.}
\end{itemize}
Emphyteuta, que pagava ao senhorio o tributo das geiras.
\section{Geissoméria}
\begin{itemize}
\item {Grp. gram.:f.}
\end{itemize}
Gênero de plantas acantháceas.
\section{Geito}
\textunderscore m.\textunderscore  (e der.)
(V. \textunderscore jeito\textunderscore , etc.)
\section{Gelada}
\begin{itemize}
\item {Grp. gram.:f.}
\end{itemize}
\begin{itemize}
\item {Proveniência:(De \textunderscore gelado\textunderscore )}
\end{itemize}
O mesmo que \textunderscore geada\textunderscore .
Orvalho.
Verdura, coberta de geada.
O mesmo que \textunderscore erva-do-orvalho\textunderscore .
\section{Geladiça}
\begin{itemize}
\item {Grp. gram.:adj. f.}
\end{itemize}
\begin{itemize}
\item {Utilização:Prov.}
\end{itemize}
\begin{itemize}
\item {Utilização:alent.}
\end{itemize}
\begin{itemize}
\item {Proveniência:(De \textunderscore gelar\textunderscore )}
\end{itemize}
Diz-se da pedra, que absorve facilmente a água e que por isso é rejeitada para construcções nas regiões frias.
\section{Gelado}
\begin{itemize}
\item {Grp. gram.:m.}
\end{itemize}
\begin{itemize}
\item {Proveniência:(De \textunderscore gelar\textunderscore )}
\end{itemize}
Espécie de doce, tornado frio e consistente por meio do gêlo ou neve.
\section{Gelador}
\begin{itemize}
\item {Grp. gram.:adj.}
\end{itemize}
Que gela.
\section{Geladura}
\begin{itemize}
\item {Grp. gram.:f.}
\end{itemize}
\begin{itemize}
\item {Proveniência:(De \textunderscore gelar\textunderscore )}
\end{itemize}
Séca ou queima, produzida nas plantas pelo frio.
\section{Gelantho}
\begin{itemize}
\item {Grp. gram.:m.}
\end{itemize}
Espécie de verniz medicinal.
\section{Gelanto}
\begin{itemize}
\item {Grp. gram.:m.}
\end{itemize}
Espécie de verniz medicinal.
\section{Gelar}
\begin{itemize}
\item {Grp. gram.:v. t.}
\end{itemize}
\begin{itemize}
\item {Utilização:Fig.}
\end{itemize}
\begin{itemize}
\item {Grp. gram.:V. i.}
\end{itemize}
\begin{itemize}
\item {Utilização:Fig.}
\end{itemize}
\begin{itemize}
\item {Proveniência:(Lat. \textunderscore gelare\textunderscore )}
\end{itemize}
Congelar; tornar muito frio; traspassar de frio: \textunderscore o dia de ontem gelou-me\textunderscore .
Causar espanto a.
Paralysar de assombro.
Converter-se em gêlo.
Esfriar muito.
Estar entorpecido.
Desalentar-se.
Ficar assombrado.
\section{Gelásimo}
\begin{itemize}
\item {Grp. gram.:m.}
\end{itemize}
\begin{itemize}
\item {Proveniência:(Gr. \textunderscore gelasimos\textunderscore )}
\end{itemize}
Gênero de crustáceos decápodes.
\section{Gelasina}
\begin{itemize}
\item {Grp. gram.:f.}
\end{itemize}
\begin{itemize}
\item {Utilização:Ant.}
\end{itemize}
\begin{itemize}
\item {Proveniência:(Do lat. \textunderscore gelasinus\textunderscore )}
\end{itemize}
Covinha, que se fórma nas faces de algumas pessôas, quando riem.
\section{Gelatina}
\begin{itemize}
\item {Grp. gram.:f.}
\end{itemize}
\begin{itemize}
\item {Proveniência:(Lat. \textunderscore gelatina\textunderscore )}
\end{itemize}
Substância animal, transparente, que, dissolvida em água quente, toma consistência e fórma a geleia.
\section{Gelatiniforme}
\begin{itemize}
\item {Grp. gram.:adj.}
\end{itemize}
\begin{itemize}
\item {Proveniência:(De \textunderscore gelatina\textunderscore  + \textunderscore forma\textunderscore )}
\end{itemize}
Que tem apparência de gelatina.
\section{Gelatinoso}
\begin{itemize}
\item {Grp. gram.:adj.}
\end{itemize}
\begin{itemize}
\item {Proveniência:(De \textunderscore gelatina\textunderscore )}
\end{itemize}
Que tem a natureza ou o aspecto da geleia.
Pegajoso.
\section{Gelba}
\begin{itemize}
\item {Grp. gram.:f.}
\end{itemize}
Embarcação do Mar-Vermelho. Cf. \textunderscore Peregrinação\textunderscore , V.
(Do ár.?)
\section{Geléa}
\begin{itemize}
\item {Grp. gram.:f.}
\end{itemize}
\begin{itemize}
\item {Proveniência:(Do fr. \textunderscore geléa\textunderscore )}
\end{itemize}
Qualquer extracto mucilaginoso de substâncias animaes ou vegetaes, que, pelo resfriamento, adquire consistência branda e trêmula.
\section{Geleia}
\begin{itemize}
\item {Grp. gram.:f.}
\end{itemize}
\begin{itemize}
\item {Proveniência:(Do fr. \textunderscore geléa\textunderscore )}
\end{itemize}
Qualquer extracto mucilaginoso de substâncias animaes ou vegetaes, que, pelo resfriamento, adquire consistência branda e trêmula.
\section{Geleira}
\begin{itemize}
\item {Grp. gram.:f.}
\end{itemize}
Montão de gêlo.
Cavidade, em que, nas altas montanhas, se fórma gêlo.
Apparelho para fabricar gêlo.
Cavidade subterrânea, em que se conserva gêlo para o estio.
\section{Gelfa}
\begin{itemize}
\item {Grp. gram.:f.}
\end{itemize}
\begin{itemize}
\item {Utilização:T. dos campos de Coímbra}
\end{itemize}
\begin{itemize}
\item {Utilização:Gír.}
\end{itemize}
Pastagem, acto de pastar: \textunderscore os bois andam á gelfa\textunderscore .
Velha.
\section{Gelha}
\begin{itemize}
\item {fónica:gê}
\end{itemize}
\begin{itemize}
\item {Grp. gram.:f.}
\end{itemize}
\begin{itemize}
\item {Utilização:Ext.}
\end{itemize}
Grão de cereaes, que tem a pellícula enrugada e que se não desenvolveu completamente.
Ruga na pellícula de grãos ou frutos.
Ruga na cara, carquilha.
Prega ou dobra casual num tecido.
\section{Gelidez}
\begin{itemize}
\item {Grp. gram.:f.}
\end{itemize}
Qualidade ou estado do que é gélido.
\section{Gélido}
\begin{itemize}
\item {Grp. gram.:adj.}
\end{itemize}
\begin{itemize}
\item {Proveniência:(Lat. \textunderscore gelidus\textunderscore )}
\end{itemize}
Muito frio.
Enregelado.
Paralysado.
Que paralysa.
\section{Gelina}
\begin{itemize}
\item {Grp. gram.:f.}
\end{itemize}
\begin{itemize}
\item {Proveniência:(De \textunderscore geleia\textunderscore )}
\end{itemize}
Princípio, que se extrai de certos tecidos, especialmente dos ossos, e que, pela ebullição, produz a gelatina.
\section{Géliva}
\begin{itemize}
\item {Grp. gram.:f.}
\end{itemize}
\begin{itemize}
\item {Utilização:Gal}
\end{itemize}
\begin{itemize}
\item {Proveniência:(Fr. \textunderscore gélive\textunderscore )}
\end{itemize}
Pedra porosa, que, embebendo-se de água, que se congela, facilmente estala.--Francesismo inútil e mal accentuado, que se me depara num programma official do ensino primário complementar.
(Cp. \textunderscore geladiça\textunderscore )
\section{Gelmende}
\begin{itemize}
\item {Grp. gram.:m.}
\end{itemize}
\begin{itemize}
\item {Utilização:Prov.}
\end{itemize}
\begin{itemize}
\item {Utilização:trasm.}
\end{itemize}
O mesmo que \textunderscore gilmendes\textunderscore .
\section{Gelmo}
\begin{itemize}
\item {Grp. gram.:m.}
\end{itemize}
\begin{itemize}
\item {Utilização:T. de Turquel}
\end{itemize}
Criaturinha indefesa.
\section{Gêlo}
\begin{itemize}
\item {Grp. gram.:m.}
\end{itemize}
\begin{itemize}
\item {Utilização:Ext.}
\end{itemize}
\begin{itemize}
\item {Utilização:Fig.}
\end{itemize}
\begin{itemize}
\item {Proveniência:(Lat. \textunderscore gelu\textunderscore )}
\end{itemize}
Água solidificada pelo frio.
Solidificação de qualquer líquido, produzida pelo frio.
Frio excessivo.
Indifferença; insensibilidade: \textunderscore homem de gêlo\textunderscore !
\section{Gelosa}
\begin{itemize}
\item {Grp. gram.:f.}
\end{itemize}
\begin{itemize}
\item {Utilização:Pharm.}
\end{itemize}
\begin{itemize}
\item {Proveniência:(De \textunderscore gêlo\textunderscore )}
\end{itemize}
Preparação medicamentosa, derivada do agar-agar.
\section{Gelosia}
\begin{itemize}
\item {Grp. gram.:f.}
\end{itemize}
\begin{itemize}
\item {Utilização:Ant.}
\end{itemize}
\begin{itemize}
\item {Proveniência:(De \textunderscore geloso\textunderscore , sob a infl. do fr. \textunderscore jalousie\textunderscore )}
\end{itemize}
Grade de fasquias de madeira, cruzadas intervalladamente, que occupa o vão de uma janela.
Rótula; janela de rótula.
Ciúme; cuidado.
\section{Geloso}
\begin{itemize}
\item {Grp. gram.:adj.}
\end{itemize}
\begin{itemize}
\item {Utilização:Ant.}
\end{itemize}
O mesmo que \textunderscore zeloso\textunderscore .
(Cp. fr. \textunderscore jaloux\textunderscore )
\section{Gelsemina}
\begin{itemize}
\item {Grp. gram.:f.}
\end{itemize}
\begin{itemize}
\item {Proveniência:(Do ár. \textunderscore gelsem\textunderscore )}
\end{itemize}
Gênero de arbustos, cujo typo habita na América do Sul.
\section{Gelva}
\begin{itemize}
\item {Grp. gram.:f.}
\end{itemize}
(V.gelba)
\section{Gema}
\begin{itemize}
\item {Grp. gram.:f.}
\end{itemize}
\begin{itemize}
\item {Proveniência:(Lat. \textunderscore gemma\textunderscore )}
\end{itemize}
Parte de um vegetal, que o póde reproduzir.
Rebento, gomo.
Pez, que se extrai do pinheiro por meio de golpes.
Saliência carnosa de alguns animaes, a qual, depois de separada, constitue novo indivíduo.
Parte amarela e interior do ovo.
Centro, parte essencial.
Qualquer pedra preciosa.
Aquilo que é mais puro, que é genuíno.
\section{Gemação}
\begin{itemize}
\item {Grp. gram.:f.}
\end{itemize}
\begin{itemize}
\item {Proveniência:(De \textunderscore gemar\textunderscore )}
\end{itemize}
Efeito de gemar.
Conjunto ou disposição das gemas de uma planta.
\section{Gemada}
\begin{itemize}
\item {Grp. gram.:f.}
\end{itemize}
\begin{itemize}
\item {Proveniência:(De \textunderscore gemado\textunderscore )}
\end{itemize}
Gema de ovo ou porção de gemas de ovos, batidas com açúcar e um líquido quente.
\section{Gemado}
\begin{itemize}
\item {Grp. gram.:adj.}
\end{itemize}
\begin{itemize}
\item {Proveniência:(De \textunderscore gemar\textunderscore )}
\end{itemize}
Que tem gemas.
Enxertado de gemma.
Que tem côr semelhante á gema de ovo.
\section{Gemante}
\begin{itemize}
\item {Grp. gram.:adj.}
\end{itemize}
\begin{itemize}
\item {Proveniência:(Lat. \textunderscore gemmans\textunderscore )}
\end{itemize}
Que brilha como pedras preciosas.
\section{Gemar}
\begin{itemize}
\item {Grp. gram.:v. t.}
\end{itemize}
\begin{itemize}
\item {Grp. gram.:V. i.}
\end{itemize}
\begin{itemize}
\item {Proveniência:(Lat. \textunderscore gemmare\textunderscore )}
\end{itemize}
Enxertar com gema, borbulha ou rebento.
Preparar com gemas de ovo.
Lançar rebentos.
\section{Gêmea}
\begin{itemize}
\item {Grp. gram.:f.}
\end{itemize}
\begin{itemize}
\item {Utilização:Ant.}
\end{itemize}
\begin{itemize}
\item {Grp. gram.:Pl.}
\end{itemize}
\begin{itemize}
\item {Proveniência:(De \textunderscore gêmeo\textunderscore )}
\end{itemize}
Conjunto de 64 talhos, nas marinhas.
\textunderscore Pôr-se em gêmeas\textunderscore , defrontar-se, tornar-se igual. Cf. Cenáculo, \textunderscore Pastoral\textunderscore , 13.
\section{Gemebundo}
\begin{itemize}
\item {Grp. gram.:adj.}
\end{itemize}
\begin{itemize}
\item {Proveniência:(Lat. \textunderscore gemebundus\textunderscore )}
\end{itemize}
O mesmo que \textunderscore gemente\textunderscore .
\section{Gemedoiro}
\begin{itemize}
\item {Grp. gram.:m.}
\end{itemize}
Successão de gemidos.
Ruído, como de quem geme.
\section{Gemedor}
\begin{itemize}
\item {Grp. gram.:adj.}
\end{itemize}
\begin{itemize}
\item {Grp. gram.:M.}
\end{itemize}
Que geme.
Aquelle que geme.
\section{Gemedouro}
\begin{itemize}
\item {Grp. gram.:m.}
\end{itemize}
Successão de gemidos.
Ruído, como de quem geme.
\section{Gemelgar}
\begin{itemize}
\item {Grp. gram.:v. t.}
\end{itemize}
\begin{itemize}
\item {Utilização:Prov.}
\end{itemize}
\begin{itemize}
\item {Utilização:trasm.}
\end{itemize}
\begin{itemize}
\item {Proveniência:(De \textunderscore gemelgo\textunderscore )}
\end{itemize}
Têr (duas crias).
Têr (duas gemmas) a planta.
\section{Gemelgo}
\begin{itemize}
\item {Grp. gram.:m.  e  adj.}
\end{itemize}
\begin{itemize}
\item {Utilização:Prov.}
\end{itemize}
\begin{itemize}
\item {Utilização:trasm.}
\end{itemize}
\begin{itemize}
\item {Proveniência:(Lat. hyp. \textunderscore gemellicus\textunderscore )}
\end{itemize}
O mesmo que \textunderscore gêmeo\textunderscore .
\section{Gemelhicar}
\begin{itemize}
\item {Grp. gram.:v. i.}
\end{itemize}
O mesmo que \textunderscore gemicar\textunderscore .
\section{Gemellos}
\begin{itemize}
\item {Grp. gram.:adj. pl.}
\end{itemize}
\begin{itemize}
\item {Utilização:Anat.}
\end{itemize}
\begin{itemize}
\item {Proveniência:(Lat. \textunderscore gemellus\textunderscore )}
\end{itemize}
Gêmeos, (falando-se dos músculos).
\section{Gemelos}
\begin{itemize}
\item {Grp. gram.:adj. pl.}
\end{itemize}
\begin{itemize}
\item {Utilização:Anat.}
\end{itemize}
\begin{itemize}
\item {Proveniência:(Lat. \textunderscore gemellus\textunderscore )}
\end{itemize}
Gêmeos, (falando-se dos músculos).
\section{Gemente}
\begin{itemize}
\item {Grp. gram.:adj.}
\end{itemize}
\begin{itemize}
\item {Proveniência:(Lat. \textunderscore gemens\textunderscore )}
\end{itemize}
Que geme.
\section{Gêmeo}
\begin{itemize}
\item {Grp. gram.:m.  e  adj.}
\end{itemize}
\begin{itemize}
\item {Grp. gram.:Pl.}
\end{itemize}
\begin{itemize}
\item {Proveniência:(Do lat. \textunderscore geminus\textunderscore )}
\end{itemize}
O que nasceu do mesmo parto que outrem: \textunderscore irmãos gêmeos\textunderscore .
Idêntico, igual.
Constellação zodiacal.
\section{Gemer}
\begin{itemize}
\item {Grp. gram.:v. i.}
\end{itemize}
\begin{itemize}
\item {Utilização:Fig.}
\end{itemize}
\begin{itemize}
\item {Grp. gram.:V. t.}
\end{itemize}
\begin{itemize}
\item {Utilização:Prov.}
\end{itemize}
\begin{itemize}
\item {Utilização:alg.}
\end{itemize}
\begin{itemize}
\item {Proveniência:(Lat. \textunderscore gemere\textunderscore )}
\end{itemize}
Exprimir dôr moral ou phýsica, por meio de vozes inarticuladas.
Soltar lamentos.
Suspirar dolorosamente.
Padecer: \textunderscore os que gemem na miséria\textunderscore .
Produzir som triste e monótono: \textunderscore o vento geme nos ciprestes\textunderscore .
Ranger: \textunderscore ouviu-se gemer a porta nos gonzos\textunderscore .
Abrir fenda, fender-se:«\textunderscore a abóbada tinha gemido\textunderscore ». Herculano, \textunderscore Lendas\textunderscore , I, 278.
Lastimar; torcer, (falando-se da vida).
Transudar, resumar: \textunderscore as paredes estavam a gemer água\textunderscore .
\section{Gemiás}
\begin{itemize}
\item {Grp. gram.:m. pl.}
\end{itemize}
Indígenas do Amazonas, nas margens do Juruá.
\section{Gemicar}
\begin{itemize}
\item {Grp. gram.:v. i.}
\end{itemize}
\begin{itemize}
\item {Proveniência:(De \textunderscore gemer\textunderscore )}
\end{itemize}
Gemer baixo, mas continuamente.
\section{Gemido}
\begin{itemize}
\item {Grp. gram.:m.}
\end{itemize}
\begin{itemize}
\item {Proveniência:(Lat. \textunderscore gemitus\textunderscore )}
\end{itemize}
Acto de gemer.
\section{Gemidor}
\begin{itemize}
\item {Grp. gram.:adj.}
\end{itemize}
O mesmo que \textunderscore gemedor\textunderscore . Cf. Filinto, II, 90.
\section{Gemífero}
\begin{itemize}
\item {Grp. gram.:adj.}
\end{itemize}
\begin{itemize}
\item {Proveniência:(Lat. \textunderscore gemmífer\textunderscore )}
\end{itemize}
Que produz ou tem pedras preciosas.
Que tem ou produz rebentos.
\section{Geminação}
\begin{itemize}
\item {Grp. gram.:f.}
\end{itemize}
\begin{itemize}
\item {Utilização:Gram.}
\end{itemize}
\begin{itemize}
\item {Proveniência:(Lat. \textunderscore geminatio\textunderscore )}
\end{itemize}
Duplicação de letra consoante.
\section{Geminado}
\begin{itemize}
\item {Grp. gram.:adj.}
\end{itemize}
\begin{itemize}
\item {Utilização:Bot.}
\end{itemize}
\begin{itemize}
\item {Utilização:Constr.}
\end{itemize}
\begin{itemize}
\item {Proveniência:(Lat. \textunderscore geminatus\textunderscore )}
\end{itemize}
O mesmo que \textunderscore duplicado\textunderscore .
Diz-se dos órgãos vegetaes, dispostos dois a dois.
Diz-se das janelas, com duas peças de caixilhos, que se abrem para os lados.
\section{Geminar}
\begin{itemize}
\item {Grp. gram.:v.}
\end{itemize}
\begin{itemize}
\item {Utilização:t. Gram.}
\end{itemize}
\begin{itemize}
\item {Proveniência:(Lat. \textunderscore geminare\textunderscore )}
\end{itemize}
Duplicar (letras consoantes).
\section{Geminável}
\begin{itemize}
\item {Grp. gram.:adj.}
\end{itemize}
\begin{itemize}
\item {Utilização:Gram.}
\end{itemize}
Que se póde geminar.
\section{Gêmino}
\begin{itemize}
\item {Grp. gram.:adj.}
\end{itemize}
\begin{itemize}
\item {Proveniência:(Lat. \textunderscore geminus\textunderscore )}
\end{itemize}
O mesmo que \textunderscore geminado\textunderscore .
\section{Gêmio}
\begin{itemize}
\item {Grp. gram.:m.  e  adj.}
\end{itemize}
\begin{itemize}
\item {Proveniência:(Lat. \textunderscore geminus\textunderscore )}
\end{itemize}
O mesmo ou melhor que \textunderscore gêmeo\textunderscore .
\section{Gemiparidade}
\begin{itemize}
\item {Grp. gram.:f.}
\end{itemize}
\begin{itemize}
\item {Utilização:Agr.}
\end{itemize}
\begin{itemize}
\item {Proveniência:(De \textunderscore gemíparo\textunderscore )}
\end{itemize}
Reprodução por meio de gemas ou rebentos.
\section{Gemíparo}
\begin{itemize}
\item {Grp. gram.:adj.}
\end{itemize}
\begin{itemize}
\item {Proveniência:(Do lat. \textunderscore gemma\textunderscore  + \textunderscore parere\textunderscore )}
\end{itemize}
Que produz rebentos.
\section{Gêmito}
\begin{itemize}
\item {Grp. gram.:m.}
\end{itemize}
\begin{itemize}
\item {Utilização:Poét.}
\end{itemize}
\begin{itemize}
\item {Proveniência:(Lat. \textunderscore gemitus\textunderscore )}
\end{itemize}
O gemer (do vento).
\section{Gemma}
\begin{itemize}
\item {Grp. gram.:f.}
\end{itemize}
\begin{itemize}
\item {Proveniência:(Lat. \textunderscore gemma\textunderscore )}
\end{itemize}
Parte de um vegetal, que o póde reproduzir.
Rebento, gomo.
Pez, que se extrai do pinheiro por meio de golpes.
Saliência carnosa de alguns animaes, a qual, depois de separada, constitue novo indivíduo.
Parte amarela e interior do ovo.
Centro, parte essencial.
Qualquer pedra preciosa.
Aquillo que é mais puro, que é genuíno.
\section{Gemmação}
\begin{itemize}
\item {Grp. gram.:f.}
\end{itemize}
\begin{itemize}
\item {Proveniência:(De \textunderscore gemmar\textunderscore )}
\end{itemize}
Effeito de gemmar.
Conjunto ou disposição das gemmas de uma planta.
\section{Gemmada}
\begin{itemize}
\item {Grp. gram.:f.}
\end{itemize}
\begin{itemize}
\item {Proveniência:(De \textunderscore gemmado\textunderscore )}
\end{itemize}
Gemma de ovo ou porção de gemmas de ovos, batidas com açúcar e um líquido quente.
\section{Gemmado}
\begin{itemize}
\item {Grp. gram.:adj.}
\end{itemize}
\begin{itemize}
\item {Proveniência:(De \textunderscore gemmar\textunderscore )}
\end{itemize}
Que tem gemmas.
Enxertado de gemma.
Que tem côr semelhante á gemma de ovo.
\section{Gemmante}
\begin{itemize}
\item {Grp. gram.:adj.}
\end{itemize}
\begin{itemize}
\item {Proveniência:(Lat. \textunderscore gemmans\textunderscore )}
\end{itemize}
Que brilha como pedras preciosas.
\section{Gemmar}
\begin{itemize}
\item {Grp. gram.:v. t.}
\end{itemize}
\begin{itemize}
\item {Grp. gram.:V. i.}
\end{itemize}
\begin{itemize}
\item {Proveniência:(Lat. \textunderscore gemmare\textunderscore )}
\end{itemize}
Enxertar com gemma, borbulha ou rebento.
Preparar com gemmas de ovo.
Lançar rebentos.
\section{Gemmífero}
\begin{itemize}
\item {Grp. gram.:adj.}
\end{itemize}
\begin{itemize}
\item {Proveniência:(Lat. \textunderscore gemmífer\textunderscore )}
\end{itemize}
Que produz ou tem pedras preciosas.
Que tem ou produz rebentos.
\section{Gemmiparidade}
\begin{itemize}
\item {Grp. gram.:f.}
\end{itemize}
\begin{itemize}
\item {Utilização:Agr.}
\end{itemize}
\begin{itemize}
\item {Proveniência:(De \textunderscore gemmíparo\textunderscore )}
\end{itemize}
Reproducção por meio de gemmas ou rebentos.
\section{Gemmíparo}
\begin{itemize}
\item {Grp. gram.:adj.}
\end{itemize}
\begin{itemize}
\item {Proveniência:(Do lat. \textunderscore gemma\textunderscore  + \textunderscore parere\textunderscore )}
\end{itemize}
Que produz rebentos.
\section{Gêmmula}
\begin{itemize}
\item {Grp. gram.:f.}
\end{itemize}
\begin{itemize}
\item {Proveniência:(Lat. \textunderscore gemmula\textunderscore )}
\end{itemize}
Pequena gemma.
\section{Gemónias}
\begin{itemize}
\item {Grp. gram.:f. pl.}
\end{itemize}
\begin{itemize}
\item {Utilização:Fig.}
\end{itemize}
\begin{itemize}
\item {Proveniência:(Lat. \textunderscore gemoniae\textunderscore )}
\end{itemize}
Lugar, onde se expunham e executavam os criminosos, em Roma.
Extremo ultraje.
Desgraça infamante.
\section{Gêmula}
\begin{itemize}
\item {Grp. gram.:f.}
\end{itemize}
\begin{itemize}
\item {Proveniência:(Lat. \textunderscore gemmula\textunderscore )}
\end{itemize}
Pequena gema.
\section{Gemursa}
\begin{itemize}
\item {Grp. gram.:f.}
\end{itemize}
\begin{itemize}
\item {Utilização:Ant.}
\end{itemize}
\begin{itemize}
\item {Proveniência:(Lat. \textunderscore gemursa\textunderscore )}
\end{itemize}
Inchação ou tumor entre os dedos dos pés.
\section{Genal}
\begin{itemize}
\item {Grp. gram.:adj.}
\end{itemize}
\begin{itemize}
\item {Proveniência:(Do lat. \textunderscore gena\textunderscore )}
\end{itemize}
Relativo ás faces.
\section{Genciana}
\begin{itemize}
\item {Grp. gram.:f.}
\end{itemize}
\begin{itemize}
\item {Proveniência:(Lat. \textunderscore gentiana\textunderscore )}
\end{itemize}
Gênero de plantas, que crescem nas montanhas.
\section{Gencianáceas}
\begin{itemize}
\item {Grp. gram.:f. pl.}
\end{itemize}
O mesmo ou melhor que \textunderscore genciâneas\textunderscore .
\section{Genciâneas}
\begin{itemize}
\item {Grp. gram.:f. pl.}
\end{itemize}
Família de plantas, que têm por tipo a genciana.
\section{Gencianela}
\begin{itemize}
\item {Grp. gram.:f.}
\end{itemize}
Genciana amarela.
\section{Gencianina}
\begin{itemize}
\item {Grp. gram.:f.}
\end{itemize}
\begin{itemize}
\item {Utilização:Chím.}
\end{itemize}
Princípio, descoberto na raíz da genciana e que, com outras substâncias, constitue a base do extracto de genciana, em pharmácia.
\section{Gendarmaria}
\begin{itemize}
\item {Grp. gram.:f.}
\end{itemize}
\begin{itemize}
\item {Utilização:Neol.}
\end{itemize}
\begin{itemize}
\item {Proveniência:(De \textunderscore gendarme\textunderscore )}
\end{itemize}
Corpo de soldados franceses, incumbidos de velar pela segurança e tranquillidade pública. Cf. Garrett, \textunderscore Port. na Balança\textunderscore , 232.
\section{Gendarme}
\begin{itemize}
\item {Grp. gram.:m.}
\end{itemize}
\begin{itemize}
\item {Utilização:Neol.}
\end{itemize}
\begin{itemize}
\item {Proveniência:(Fr. \textunderscore gendarme\textunderscore )}
\end{itemize}
Soldado francês, pertencente á gendarmaria.
\section{Gendiroba}
\begin{itemize}
\item {Grp. gram.:f.}
\end{itemize}
O mesmo que \textunderscore nandiroba\textunderscore .
\section{Genealogia}
\begin{itemize}
\item {Grp. gram.:f.}
\end{itemize}
\begin{itemize}
\item {Utilização:Fam.}
\end{itemize}
\begin{itemize}
\item {Proveniência:(Do gr. \textunderscore genea\textunderscore  + \textunderscore logos\textunderscore )}
\end{itemize}
Série ascendente ou descendente de antepassados.
Linhagem.
Exposição das origens e ramificações de uma família.
Origem, procedência: \textunderscore a genealogia da cólera-morbo\textunderscore .
\section{Genealogicamente}
\begin{itemize}
\item {Grp. gram.:adv.}
\end{itemize}
Por ordem genealógica.
\section{Genealógico}
\begin{itemize}
\item {Grp. gram.:adj.}
\end{itemize}
\begin{itemize}
\item {Grp. gram.:M.}
\end{itemize}
Relativo á genealogia.

O mesmo que \textunderscore genealogista\textunderscore . Cf. Herculano, \textunderscore Opúsc.\textunderscore , III, 18; Latino, \textunderscore Humboldt\textunderscore , 42; Filinto, XVIII, 88.
\section{Genealogista}
\begin{itemize}
\item {Grp. gram.:m.}
\end{itemize}
\begin{itemize}
\item {Proveniência:(De \textunderscore genealogia\textunderscore )}
\end{itemize}
Aquelle que se dedica a estudos genealógicos.
\section{Genearcha}
\begin{itemize}
\item {fónica:ca}
\end{itemize}
\begin{itemize}
\item {Grp. gram.:m.}
\end{itemize}
\begin{itemize}
\item {Proveniência:(Do gr. \textunderscore genea\textunderscore  + \textunderscore arkhe\textunderscore )}
\end{itemize}
Progenitor de uma família, de uma linhagem ou de uma espécie.
\section{Genearca}
\begin{itemize}
\item {Grp. gram.:m.}
\end{itemize}
\begin{itemize}
\item {Proveniência:(Do gr. \textunderscore genea\textunderscore  + \textunderscore arkhe\textunderscore )}
\end{itemize}
Progenitor de uma família, de uma linhagem ou de uma espécie.
\section{Genebra}
\begin{itemize}
\item {Grp. gram.:f.}
\end{itemize}
\begin{itemize}
\item {Proveniência:(Do fr. \textunderscore genièvre\textunderscore )}
\end{itemize}
Bebida alcoólica, formada de aguardente, em que se destillaram ou maceraram bagas de zimbro.
\section{Genebrada}
\begin{itemize}
\item {Grp. gram.:f.}
\end{itemize}
Água açucarada, misturada com genebra e suco ou casca de limão.
\section{Genebrês}
\begin{itemize}
\item {Grp. gram.:adj.}
\end{itemize}
\begin{itemize}
\item {Grp. gram.:M.}
\end{itemize}
Relativo á cidade de Genebra.
Indivíduo que nasceu em Genebra ou habita em Genebra.
\section{Genebro}
\begin{itemize}
\item {Grp. gram.:adj.}
\end{itemize}
Relativo á Genebra. Cf. Filinto, V, 87.
\section{Genela}
\begin{itemize}
\item {Grp. gram.:f.}
\end{itemize}
\begin{itemize}
\item {Utilização:Ant.}
\end{itemize}
\begin{itemize}
\item {Proveniência:(Do lat. \textunderscore gena\textunderscore ?)}
\end{itemize}
O mesmo que \textunderscore coifa\textunderscore .
\section{Genepi}
\begin{itemize}
\item {Grp. gram.:m.}
\end{itemize}
Planta medicinal, tónica e sudorífica, (\textunderscore artemisia glacialis\textunderscore , Lin.), que se encontra nos Alpes, e de que há duas espécies, uma negra e outra branca.
\section{Genequim}
\begin{itemize}
\item {Grp. gram.:m.}
\end{itemize}
Espécie ordinária de algodão fiado.
\section{Gener}
\begin{itemize}
\item {Grp. gram.:v. i.}
\end{itemize}
\begin{itemize}
\item {Utilização:Ant.}
\end{itemize}
Encher-se.
Crescer.
Trasbordar.
(Cp. cast. \textunderscore llenar\textunderscore )
\section{General}
\begin{itemize}
\item {Grp. gram.:m.}
\end{itemize}
\begin{itemize}
\item {Utilização:Fig.}
\end{itemize}
\begin{itemize}
\item {Grp. gram.:Adj.}
\end{itemize}
\begin{itemize}
\item {Proveniência:(Lat. \textunderscore generalis\textunderscore )}
\end{itemize}
Graduação militar, immediatamente superior á de coronel.

Chefe, caudilho.

O mesmo que \textunderscore geral\textunderscore ^1. Cf. Usque, \textunderscore Tribulações\textunderscore , 35.
\section{Generala}
\begin{itemize}
\item {Grp. gram.:f.}
\end{itemize}
\begin{itemize}
\item {Utilização:Fam.}
\end{itemize}
\begin{itemize}
\item {Proveniência:(Do rad. de \textunderscore general\textunderscore )}
\end{itemize}
Certo toque de tambor ou de trombeta, para chamar tropas ás armas ou a postos.
Mulher de general.
\section{Generalado}
\begin{itemize}
\item {Grp. gram.:m.}
\end{itemize}
\begin{itemize}
\item {Proveniência:(Do lat. \textunderscore generalis\textunderscore )}
\end{itemize}
Posto de general.
Dignidade do geral de uma Ordem religiosa.
\section{Generalato}
\begin{itemize}
\item {Grp. gram.:m.}
\end{itemize}
\begin{itemize}
\item {Proveniência:(Do lat. \textunderscore generalis\textunderscore )}
\end{itemize}
Posto de general.
Dignidade do geral de uma Ordem religiosa.
\section{Generalidade}
\begin{itemize}
\item {Grp. gram.:f.}
\end{itemize}
\begin{itemize}
\item {Grp. gram.:Pl.}
\end{itemize}
\begin{itemize}
\item {Proveniência:(Lat. \textunderscore generalitas\textunderscore )}
\end{itemize}
Qualidade daquillo que é geral.
Rudimentos, princípios elementares, ideias fundamentaes.
\section{Generalíssimo}
\begin{itemize}
\item {Grp. gram.:m.}
\end{itemize}
\begin{itemize}
\item {Proveniência:(De \textunderscore general\textunderscore )}
\end{itemize}
Chefe superior de um exército.
Título do Soberano de uma nação, em relação ao exército.
\section{Generalização}
\begin{itemize}
\item {Grp. gram.:f.}
\end{itemize}
Acto ou effeito de generalizar.
Propriedade ou estado daquillo que se tornou geral.
\section{Generalizador}
\begin{itemize}
\item {Grp. gram.:adj.}
\end{itemize}
Que generaliza:«\textunderscore faculdade generalizadora.\textunderscore »Latino, \textunderscore Humboldt\textunderscore , 259.
\section{Generalizar}
\begin{itemize}
\item {Grp. gram.:v. t.}
\end{itemize}
\begin{itemize}
\item {Proveniência:(Do lat. \textunderscore generalis\textunderscore )}
\end{itemize}
Tornar geral, não fazer restricções individuaes: \textunderscore generalizar censuras\textunderscore .
Vulgarizar; diffundir; tornar commum: \textunderscore generalizar principios sãos\textunderscore .
\section{Generante}
\begin{itemize}
\item {Grp. gram.:adj.}
\end{itemize}
\begin{itemize}
\item {Proveniência:(Lat. \textunderscore generans\textunderscore )}
\end{itemize}
Que gera.
\section{Generativo}
\begin{itemize}
\item {Grp. gram.:adj.}
\end{itemize}
\begin{itemize}
\item {Proveniência:(Do lat. \textunderscore generare\textunderscore )}
\end{itemize}
Relativo a geração; que póde gerar.
\section{Generatriz}
\begin{itemize}
\item {Grp. gram.:f.  e  adj.}
\end{itemize}
\begin{itemize}
\item {Proveniência:(Lat. \textunderscore generatrix\textunderscore )}
\end{itemize}
O mesmo que \textunderscore geratriz\textunderscore .
\section{Generear}
\begin{itemize}
\item {Grp. gram.:v. t.}
\end{itemize}
\begin{itemize}
\item {Utilização:Des.}
\end{itemize}
O mesmo que \textunderscore gerar\textunderscore . Cf. G. Viana, \textunderscore Apostilas\textunderscore .
\section{Genéria}
\begin{itemize}
\item {Grp. gram.:f.}
\end{itemize}
Gênero de plantas ornamentaes. Cf. A. Ennes, \textunderscore Lazaristas\textunderscore , 25.
\section{Genericamente}
\begin{itemize}
\item {Grp. gram.:adv.}
\end{itemize}
De modo genérico; em geral.
\section{Genérico}
\begin{itemize}
\item {Grp. gram.:adj.}
\end{itemize}
\begin{itemize}
\item {Proveniência:(De \textunderscore gênero\textunderscore )}
\end{itemize}
Relativo ao gênero.
Geral; tratado na generalidade.
\section{Gênero}
\begin{itemize}
\item {Grp. gram.:m.}
\end{itemize}
\begin{itemize}
\item {Utilização:Vulg.}
\end{itemize}
\begin{itemize}
\item {Utilização:Gram.}
\end{itemize}
\begin{itemize}
\item {Grp. gram.:Pl.}
\end{itemize}
\begin{itemize}
\item {Proveniência:(Do lat. \textunderscore genus\textunderscore )}
\end{itemize}
Carácter commum a diversas espécies.
Aquillo que comprehende muitas espécies.
Conjunto de indivíduos, que têm os mesmos caracteres essenciaes.
Reunião de corpos orgânicos ou inorgânicos, que constituem espécies.
Espécie.
Família, ordem, clásse.
Qualidade.
Maneira, modo.
Carácter da elocução de um autor, ou do estilo usado numa época.
Feição artística.
Assumpto ou natureza, commum a diversas producções artísticas ou literárias.
Subdivisão nas bellas-artes.
Propriedade, que os substantivos têm, de representar os sexos.
Mercadorias; producções agrícolas.
\section{Género}
\begin{itemize}
\item {Grp. gram.:m.}
\end{itemize}
\begin{itemize}
\item {Utilização:Vulg.}
\end{itemize}
\begin{itemize}
\item {Utilização:Gram.}
\end{itemize}
\begin{itemize}
\item {Grp. gram.:Pl.}
\end{itemize}
\begin{itemize}
\item {Proveniência:(Do lat. \textunderscore genus\textunderscore )}
\end{itemize}
Carácter commum a diversas espécies.
Aquillo que comprehende muitas espécies.
Conjunto de indivíduos, que têm os mesmos caracteres essenciaes.
Reunião de corpos orgânicos ou inorgânicos, que constituem espécies.
Espécie.
Família, ordem, clásse.
Qualidade.
Maneira, modo.
Carácter da elocução de um autor, ou do estilo usado numa época.
Feição artística.
Assumpto ou natureza, commum a diversas producções artísticas ou literárias.
Subdivisão nas bellas-artes.
Propriedade, que os substantivos têm, de representar os sexos.
Mercadorias; producções agrícolas.
\section{Generosamente}
\begin{itemize}
\item {Grp. gram.:adv.}
\end{itemize}
De modo generoso^1.
Com generosidade.
\section{Generosidade}
\begin{itemize}
\item {Grp. gram.:f.}
\end{itemize}
\begin{itemize}
\item {Proveniência:(Lat. \textunderscore generositas\textunderscore )}
\end{itemize}
Qualidade daquelle ou daquillo que é generoso.
Acção generosa: \textunderscore praticar generosidades\textunderscore .
\section{Generoso}
\begin{itemize}
\item {Grp. gram.:adj.}
\end{itemize}
\begin{itemize}
\item {Utilização:Fig.}
\end{itemize}
\begin{itemize}
\item {Proveniência:(Lat. \textunderscore generosus\textunderscore )}
\end{itemize}
Nobre por natureza ou por origem: \textunderscore sangue generoso\textunderscore .
Que tem sentimentos nobres.
Franco; magnânimo; benevolente.
Fiel.
Bizarro.
Valente.
Sublime; da melhor qualidade: \textunderscore vinho generoso\textunderscore .
\section{Generoso}
\begin{itemize}
\item {Grp. gram.:m.}
\end{itemize}
\begin{itemize}
\item {Utilização:Bras. do S}
\end{itemize}
Ente fantástico que, segundo a crença popular, entrava invisivelmente nas casas, fazia barulho nos quartos, tocava instrumentos, etc.
\section{Gênese}
\begin{itemize}
\item {Grp. gram.:f.}
\end{itemize}
O mesmo que \textunderscore gênesis\textunderscore , geração e origem.
\section{Génese}
\begin{itemize}
\item {Grp. gram.:f.}
\end{itemize}
O mesmo que \textunderscore gênesis\textunderscore , geração e origem.
\section{Genesíaco}
\begin{itemize}
\item {Grp. gram.:adj.}
\end{itemize}
\begin{itemize}
\item {Proveniência:(De \textunderscore gênesis\textunderscore )}
\end{itemize}
Relativo ao gênesis.
Concernente á geração: \textunderscore faculdades genesíacas\textunderscore .
\section{Genesiário}
\begin{itemize}
\item {Grp. gram.:m.}
\end{itemize}
\begin{itemize}
\item {Utilização:Des.}
\end{itemize}
\begin{itemize}
\item {Proveniência:(De \textunderscore gênesis\textunderscore )}
\end{itemize}
Tronco de uma raça.
Indivíduo, que é a origem de uma raça. Cf. \textunderscore Anat. Joc.\textunderscore , I, (dedicat.).
\section{Genésico}
\begin{itemize}
\item {Grp. gram.:adj.}
\end{itemize}
O mesmo que \textunderscore genesíaco\textunderscore .
\section{Genesim}
\begin{itemize}
\item {Grp. gram.:m.}
\end{itemize}
Aula, em que os rabinos portugueses liam e explicavam os livros do \textunderscore Pentateuco\textunderscore , os primeiros dos quaes é o \textunderscore Gênesis\textunderscore .
(Talvez fórma hebraica rabinica do gr. \textunderscore genesis\textunderscore )
\section{Gênesis}
\begin{itemize}
\item {Grp. gram.:f.}
\end{itemize}
\begin{itemize}
\item {Grp. gram.:M.}
\end{itemize}
\begin{itemize}
\item {Proveniência:(Gr. \textunderscore genesis\textunderscore )}
\end{itemize}
Formação de seres, desde uma origem; geração.
Primeira parte do \textunderscore Antigo Testamento\textunderscore , em que se descreve a criação do mundo e a successão dos primeiros homens.
Systema cosmogónico.
\section{Genethlíaco}
\begin{itemize}
\item {Grp. gram.:adj.}
\end{itemize}
\begin{itemize}
\item {Grp. gram.:M.}
\end{itemize}
\begin{itemize}
\item {Proveniência:(Gr. \textunderscore genethliakos\textunderscore )}
\end{itemize}
Relativo ao nascimento.
Aquelle que prevê o futuro pela observação dos astros.
\section{Genethliologia}
\begin{itemize}
\item {Grp. gram.:f.}
\end{itemize}
\begin{itemize}
\item {Proveniência:(Do gr. \textunderscore genethlion\textunderscore  + \textunderscore logos\textunderscore )}
\end{itemize}
Arte de predizer o futuro pela observação dos astros.
Arte de explicar o horóscopo.
\section{Genethliológico}
\begin{itemize}
\item {Grp. gram.:adj.}
\end{itemize}
Relativo á genethliologia.
\section{Genético}
\begin{itemize}
\item {Grp. gram.:adj.}
\end{itemize}
\begin{itemize}
\item {Proveniência:(Do gr. \textunderscore genete\textunderscore )}
\end{itemize}
O mesmo que \textunderscore genesíaco\textunderscore ; relativo ás funcções da geração. Cf. Latino, \textunderscore Or. da Corôa\textunderscore , XXVIII.
\section{Genetlíaco}
\begin{itemize}
\item {Grp. gram.:adj.}
\end{itemize}
\begin{itemize}
\item {Grp. gram.:M.}
\end{itemize}
\begin{itemize}
\item {Proveniência:(Gr. \textunderscore genethliakos\textunderscore )}
\end{itemize}
Relativo ao nascimento.
Aquele que prevê o futuro pela observação dos astros.
\section{Genetliologia}
\begin{itemize}
\item {Grp. gram.:f.}
\end{itemize}
\begin{itemize}
\item {Proveniência:(Do gr. \textunderscore genethlion\textunderscore  + \textunderscore logos\textunderscore )}
\end{itemize}
Arte de predizer o futuro pela observação dos astros.
Arte de explicar o horóscopo.
\section{Genetliológico}
\begin{itemize}
\item {Grp. gram.:adj.}
\end{itemize}
Relativo á genetliologia.
\section{Genetriz}
\begin{itemize}
\item {Grp. gram.:f.}
\end{itemize}
\begin{itemize}
\item {Proveniência:(Lat. \textunderscore genetrix\textunderscore )}
\end{itemize}
Aquella que gera; a mãe.
\section{Gengiberáceas}
\begin{itemize}
\item {Grp. gram.:f. pl.}
\end{itemize}
Família de plantas, que têm por typo o gengibre.
\section{Gengibirra}
\begin{itemize}
\item {Grp. gram.:f.}
\end{itemize}
Bebida fermentada, espécie de cerveja, usada entre os indígenas do norte do Brasil.
\section{Gengibre}
\begin{itemize}
\item {Grp. gram.:m.  ou  f.}
\end{itemize}
\begin{itemize}
\item {Proveniência:(Do ingl. \textunderscore gingerbeer\textunderscore )}
\end{itemize}
Planta vivaz das regiões tropicaes.
\section{Gengiva}
\begin{itemize}
\item {Grp. gram.:f.}
\end{itemize}
\begin{itemize}
\item {Proveniência:(Do lat. \textunderscore gingiva\textunderscore )}
\end{itemize}
Tecido fibro-muscular, em que estão os alvéolos dentários.
\section{Gengival}
\begin{itemize}
\item {Grp. gram.:adj.}
\end{itemize}
Relativo á gengiva.
\section{Gengivite}
\begin{itemize}
\item {Grp. gram.:f.}
\end{itemize}
Inflammação das gengivas.
\section{Genial}
\begin{itemize}
\item {Grp. gram.:adj.}
\end{itemize}
\begin{itemize}
\item {Utilização:Fig.}
\end{itemize}
\begin{itemize}
\item {Proveniência:(Lat. \textunderscore genialis\textunderscore )}
\end{itemize}
Relativo a gênio, a índole ou inclinação.
Próprio de um grande talento, de um gênio: \textunderscore obra genial\textunderscore .
Prazenteiro.
\section{Genialmente}
\begin{itemize}
\item {Grp. gram.:adv.}
\end{itemize}
De modo genial.
\section{Geniculado}
\begin{itemize}
\item {Grp. gram.:adj.}
\end{itemize}
\begin{itemize}
\item {Utilização:anat.}
\end{itemize}
\begin{itemize}
\item {Utilização:Bot.}
\end{itemize}
\begin{itemize}
\item {Proveniência:(Lat. \textunderscore geniculatus\textunderscore )}
\end{itemize}
Dobrado, em fórma de joêlho.
\section{Gênio}
\begin{itemize}
\item {Grp. gram.:m.}
\end{itemize}
\begin{itemize}
\item {Utilização:Fig.}
\end{itemize}
\begin{itemize}
\item {Utilização:Pop.}
\end{itemize}
\begin{itemize}
\item {Proveniência:(Lat. \textunderscore genius\textunderscore )}
\end{itemize}
Espírito bom ou mau que, segundo os antigos, presidia ao destino de cada homem.
Cada um dos espíritos, que se suppunha dominarem cada um dos elementos da natureza.
Espírito, inspirador de uma arte, de uma virtude, de um vício, etc.
Grande talento innato: \textunderscore homem de gênio\textunderscore .
Aptidão especial.
Índole, carácter, temperamento: \textunderscore a pequena tem mau gênio\textunderscore .
Pessôa, que possue poder intellectual: \textunderscore Camões foi um gênio\textunderscore .
Irascibilidade: \textunderscore sempre tem um gênio esta mulher!\textunderscore 
\section{Génio}
\begin{itemize}
\item {Grp. gram.:m.}
\end{itemize}
\begin{itemize}
\item {Utilização:Fig.}
\end{itemize}
\begin{itemize}
\item {Utilização:Pop.}
\end{itemize}
\begin{itemize}
\item {Proveniência:(Lat. \textunderscore genius\textunderscore )}
\end{itemize}
Espírito bom ou mau que, segundo os antigos, presidia ao destino de cada homem.
Cada um dos espíritos, que se suppunha dominarem cada um dos elementos da natureza.
Espírito, inspirador de uma arte, de uma virtude, de um vício, etc.
Grande talento innato: \textunderscore homem de génio\textunderscore .
Aptidão especial.
Índole, carácter, temperamento: \textunderscore a pequena tem mau génio\textunderscore .
Pessôa, que possue poder intellectual: \textunderscore Camões foi um génio\textunderscore .
Irascibilidade: \textunderscore sempre tem um génio esta mulher!\textunderscore 
\section{Genioglosso}
\begin{itemize}
\item {Grp. gram.:adj.}
\end{itemize}
\begin{itemize}
\item {Proveniência:(Do gr. \textunderscore geneion\textunderscore  + \textunderscore glossa\textunderscore )}
\end{itemize}
Relativo ao queixo e á língua.
\section{Genioplastia}
\begin{itemize}
\item {Grp. gram.:f.}
\end{itemize}
\begin{itemize}
\item {Proveniência:(Do gr. \textunderscore geneion\textunderscore  + \textunderscore plassein\textunderscore )}
\end{itemize}
Restauração do queixo.
\section{Genista}
\begin{itemize}
\item {Grp. gram.:f.}
\end{itemize}
\begin{itemize}
\item {Utilização:P. us.}
\end{itemize}
\begin{itemize}
\item {Proveniência:(Lat. \textunderscore genista\textunderscore )}
\end{itemize}
O mesmo que \textunderscore giesta\textunderscore .
\section{Genísteas}
\begin{itemize}
\item {Grp. gram.:f. pl.}
\end{itemize}
\begin{itemize}
\item {Proveniência:(Do lat. \textunderscore genista\textunderscore )}
\end{itemize}
Subtríbo de plantas leguminosas, a que pertence a giesta.
\section{Genital}
\begin{itemize}
\item {Grp. gram.:adj.}
\end{itemize}
\begin{itemize}
\item {Proveniência:(Lat. \textunderscore genitalis\textunderscore )}
\end{itemize}
Que diz respeito á geração; que serve para a geração: \textunderscore órgãos genitaes\textunderscore .
\section{Genitivo}
\begin{itemize}
\item {Grp. gram.:m.}
\end{itemize}
\begin{itemize}
\item {Utilização:Gram.}
\end{itemize}
\begin{itemize}
\item {Proveniência:(Lat. \textunderscore genitivus\textunderscore )}
\end{itemize}
Caso, em que as palavras declináveis representam geralmente um complemento restrictivo e, ás vezes, circunstancial; ou caso, em que os nomes são empregados como complemento de outros nomes, de alguns verbos e, em grego, de algumas preposições.
\section{Gênito}
\begin{itemize}
\item {Grp. gram.:adj.}
\end{itemize}
\begin{itemize}
\item {Proveniência:(Lat. \textunderscore genitus\textunderscore )}
\end{itemize}
O mesmo que [[gerado|gerar]].
\section{Gênito-crural}
\begin{itemize}
\item {Grp. gram.:adj.}
\end{itemize}
Relativo aos órgãos da geração e ás coxas.
\section{Gênito-espinal}
\begin{itemize}
\item {Grp. gram.:adj.}
\end{itemize}
Relativo aos órgãos genitaes e á espinha dorsal.
\section{Genitor}
\begin{itemize}
\item {Grp. gram.:m.}
\end{itemize}
\begin{itemize}
\item {Proveniência:(Lat. \textunderscore genitor\textunderscore )}
\end{itemize}
Aquelle que gera.
\section{Gênito-urinário}
\begin{itemize}
\item {Grp. gram.:adj.}
\end{itemize}
Relativo aos órgãos da geração e á excreção da urina.
\section{Genitriz}
\begin{itemize}
\item {Grp. gram.:f.}
\end{itemize}
O mesmo que \textunderscore genetriz\textunderscore .
Mãe. Cf. Castilho, \textunderscore Sabichonas\textunderscore , 166.
\section{Genitura}
\begin{itemize}
\item {Grp. gram.:f.}
\end{itemize}
\begin{itemize}
\item {Utilização:Des.}
\end{itemize}
\begin{itemize}
\item {Proveniência:(Lat. \textunderscore genitura\textunderscore )}
\end{itemize}
Geração; raça.
\section{Genoês}
\begin{itemize}
\item {Grp. gram.:m.  e  adj.}
\end{itemize}
O mesmo que \textunderscore genovês\textunderscore . Cf. Azurara, \textunderscore Chrón. de D. Pedro\textunderscore , C. XIX; Gil Vicente, \textunderscore Auto da Fé\textunderscore .
\section{Genoplastia}
\begin{itemize}
\item {Grp. gram.:f.}
\end{itemize}
\begin{itemize}
\item {Proveniência:(Do lat. \textunderscore gena\textunderscore  + gr. \textunderscore plassein\textunderscore )}
\end{itemize}
Operação cirúrgica, com que se repara a perda de um pedaço da face com outro pedaço de outra parte do corpo operado.
\section{Genovévano}
\begin{itemize}
\item {Grp. gram.:m.}
\end{itemize}
\begin{itemize}
\item {Utilização:Des.}
\end{itemize}
Cónego regular de Santa-Genoveva.
\section{Genovês}
\begin{itemize}
\item {Grp. gram.:adj.}
\end{itemize}
\begin{itemize}
\item {Grp. gram.:M.}
\end{itemize}
Relativo a Gênova.
Habitante de Gênova.
\section{Genovisco}
\begin{itemize}
\item {Grp. gram.:m.  e  adj.}
\end{itemize}
\begin{itemize}
\item {Utilização:Ant.}
\end{itemize}
O mesmo que \textunderscore genovês\textunderscore .
\section{Genra}
\begin{itemize}
\item {Grp. gram.:f.}
\end{itemize}
\begin{itemize}
\item {Utilização:Prov.}
\end{itemize}
\begin{itemize}
\item {Utilização:trasm.}
\end{itemize}
\begin{itemize}
\item {Proveniência:(De \textunderscore genro\textunderscore )}
\end{itemize}
O mesmo que \textunderscore nora\textunderscore ^2, (parentesco).
\section{Genro}
\begin{itemize}
\item {Grp. gram.:m.}
\end{itemize}
\begin{itemize}
\item {Proveniência:(Do lat. \textunderscore gener\textunderscore )}
\end{itemize}
Designação do marido, em relação aos pais de sua mulher.
\section{Gentaça}
\begin{itemize}
\item {Grp. gram.:f.}
\end{itemize}
Ralé; gente ordinária; ínfima plebe.
\section{Gentalha}
\begin{itemize}
\item {Grp. gram.:f.}
\end{itemize}
\begin{itemize}
\item {Utilização:Deprec.}
\end{itemize}
Ralé; gente ordinária; ínfima plebe.
\section{Gente}
\begin{itemize}
\item {Grp. gram.:f.}
\end{itemize}
\begin{itemize}
\item {Proveniência:(Lat. \textunderscore gens\textunderscore , \textunderscore gentis\textunderscore )}
\end{itemize}
Quantidade de pessôas: \textunderscore encontrei muita gente\textunderscore .
População: \textunderscore a gente daquella terra\textunderscore .
Habitantes de uma região.
Humanidade: \textunderscore a gente começou em Adão?\textunderscore 
Pessôas, que têm a mesma natureza, a mesma profissão, as mesmas ideias, os mesmos hábitos: \textunderscore a gente das fábricas\textunderscore .
Fôrça armada: \textunderscore o commandante tinha pouca gente\textunderscore .
Família.
Nós, quando falamos:«\textunderscore a gente não os lia, porque não tinhamos vagar\textunderscore ». Camillo, \textunderscore Viuva do Enforc.\textunderscore , II, 12.
\section{Gentiaga}
\begin{itemize}
\item {Grp. gram.:f.}
\end{itemize}
\begin{itemize}
\item {Utilização:Pop.}
\end{itemize}
Grande porção de gente.
Poviléu, ralé.
\section{Gentes!}
\begin{itemize}
\item {Grp. gram.:interj.}
\end{itemize}
\begin{itemize}
\item {Utilização:Bras. de Minas}
\end{itemize}
(Para revelar grande alegria)
\section{Gentil}
\begin{itemize}
\item {Grp. gram.:adj.}
\end{itemize}
\begin{itemize}
\item {Utilização:Fig.}
\end{itemize}
\begin{itemize}
\item {Grp. gram.:M.}
\end{itemize}
\begin{itemize}
\item {Proveniência:(Lat. \textunderscore gentílis\textunderscore )}
\end{itemize}
Que tem nobreza.
Cavalheiroso: \textunderscore procedimento gentil\textunderscore .
Esbelto; elegante: \textunderscore moço gentil\textunderscore .
Puro; aprazível.
Nome de várias espécies de moedinhas de oiro, em tempo de D. Fernando.
\section{Gentileza}
\begin{itemize}
\item {Grp. gram.:f.}
\end{itemize}
Qualidade daquelle ou daquillo que é gentil.
Acção nobre, distinta.
Esfôrço.
Bizarria; delicadeza.
\section{Gentilhomem}
\begin{itemize}
\item {fónica:ló}
\end{itemize}
\begin{itemize}
\item {Grp. gram.:m.}
\end{itemize}
\begin{itemize}
\item {Grp. gram.:Adj.}
\end{itemize}
\begin{itemize}
\item {Proveniência:(De \textunderscore gentil\textunderscore  + \textunderscore homem\textunderscore )}
\end{itemize}
Homem nobre, distinto; fidalgo; cavalheiro.
Elegante, airoso. Cf. \textunderscore Peregrinação\textunderscore , CIII.
\section{Gentil-homem}
\begin{itemize}
\item {fónica:ló}
\end{itemize}
\begin{itemize}
\item {Grp. gram.:m.}
\end{itemize}
\begin{itemize}
\item {Grp. gram.:Adj.}
\end{itemize}
\begin{itemize}
\item {Proveniência:(De \textunderscore gentil\textunderscore  + \textunderscore homem\textunderscore )}
\end{itemize}
Homem nobre, distinto; fidalgo; cavalheiro.
Elegante, airoso. Cf. \textunderscore Peregrinação\textunderscore , CIII.
\section{Gentilicamente}
\begin{itemize}
\item {Grp. gram.:adj.}
\end{itemize}
De modo gentílico; á maneira dos gentios.
\section{Gentilício}
\begin{itemize}
\item {Grp. gram.:adj.}
\end{itemize}
\begin{itemize}
\item {Proveniência:(Lat. \textunderscore gentilicius\textunderscore )}
\end{itemize}
O mesmo que \textunderscore gentílico\textunderscore .
\section{Gentílico}
\begin{itemize}
\item {Grp. gram.:adj.}
\end{itemize}
\begin{itemize}
\item {Utilização:Gram.}
\end{itemize}
\begin{itemize}
\item {Proveniência:(Do lat. \textunderscore gentílis\textunderscore )}
\end{itemize}
Relativo aos gentios.
Próprio de gentios: \textunderscore costumes gentílicos\textunderscore .
Designativo da nação a que alguém pertence: \textunderscore «português»,«castelhano»,«russo», são nomes gentílicos\textunderscore .
\section{Gentilidade}
\begin{itemize}
\item {Grp. gram.:f.}
\end{itemize}
\begin{itemize}
\item {Proveniência:(Lat. \textunderscore gentilitas\textunderscore )}
\end{itemize}
Religião dos gentios.
Os gentios.
Paganismo.
\section{Gentilismo}
\begin{itemize}
\item {Grp. gram.:m.}
\end{itemize}
\begin{itemize}
\item {Proveniência:(Do lat. \textunderscore gentilis\textunderscore )}
\end{itemize}
O mesmo que \textunderscore gentilidade\textunderscore .
Antiguidade pagan.
Paganismo.
Povos gentios.
\section{Gentilizar}
\begin{itemize}
\item {Grp. gram.:v. t.}
\end{itemize}
\begin{itemize}
\item {Grp. gram.:V. i.}
\end{itemize}
\begin{itemize}
\item {Proveniência:(Do lat. \textunderscore gentilis\textunderscore )}
\end{itemize}
Tornar gentio.
Dar feição de gentio a.
Praticar o culto pagão.
\section{Gentilmente}
\begin{itemize}
\item {Grp. gram.:adv.}
\end{itemize}
De modo gentil.
Bizarramente; com pundonor.
\section{Gentinha}
\begin{itemize}
\item {Grp. gram.:f.}
\end{itemize}
\begin{itemize}
\item {Utilização:Deprec.}
\end{itemize}
\begin{itemize}
\item {Proveniência:(De \textunderscore gente\textunderscore )}
\end{itemize}
Gentalha.
Pessôas mexeriqueiras.
\section{Gentio}
\begin{itemize}
\item {Grp. gram.:m.}
\end{itemize}
\begin{itemize}
\item {Utilização:Pop.}
\end{itemize}
\begin{itemize}
\item {Grp. gram.:Adj.}
\end{itemize}
\begin{itemize}
\item {Proveniência:(Do lat. \textunderscore genitivus\textunderscore , de \textunderscore genitus\textunderscore )}
\end{itemize}
Aquelle que segue a religião pagan.
Idólatra.
Grande quantidade de gente.
Que segue o paganismo; que não é civilizado; selvagem.
\section{Genuense}
\begin{itemize}
\item {Grp. gram.:m.  e  adj.}
\end{itemize}
\begin{itemize}
\item {Proveniência:(Lat. \textunderscore genuensis\textunderscore )}
\end{itemize}
O mesmo que \textunderscore genovês\textunderscore .
\section{Genuês}
\begin{itemize}
\item {Grp. gram.:m.  e  adj.}
\end{itemize}
O mesmo ou melhor que \textunderscore genoês\textunderscore .
\section{Genuflectir}
\begin{itemize}
\item {Grp. gram.:v. i.}
\end{itemize}
\begin{itemize}
\item {Grp. gram.:V. t.}
\end{itemize}
\begin{itemize}
\item {Proveniência:(Do lat. \textunderscore genu\textunderscore  + \textunderscore flectere\textunderscore )}
\end{itemize}
Dobrar o joêlho, ajoelhar. Cf. Camillo, \textunderscore Volcões\textunderscore , 65; \textunderscore Perfil\textunderscore , 209.
Dobrar pelo joêlho:«\textunderscore genuflectiu a perna direita.\textunderscore »Camillo, \textunderscore Brasileira\textunderscore .
\section{Genuflexão}
\begin{itemize}
\item {fónica:csão}
\end{itemize}
\begin{itemize}
\item {Grp. gram.:f.}
\end{itemize}
\begin{itemize}
\item {Proveniência:(Do lat. \textunderscore genu\textunderscore  + \textunderscore flexio\textunderscore )}
\end{itemize}
Acto de dobrar o joêlho ou de ajoelhar.
\section{Genuflexo}
\begin{itemize}
\item {fónica:cso}
\end{itemize}
\begin{itemize}
\item {Grp. gram.:adj.}
\end{itemize}
\begin{itemize}
\item {Proveniência:(Do lat. \textunderscore genu\textunderscore  + \textunderscore flexus\textunderscore )}
\end{itemize}
Que dobrou o joêlho, que ajoelhou. Cf. Camillo, \textunderscore Narcót.\textunderscore , I, 140.
\section{Genuflexório}
\begin{itemize}
\item {fónica:csó}
\end{itemize}
\begin{itemize}
\item {Grp. gram.:m.}
\end{itemize}
\begin{itemize}
\item {Proveniência:(De \textunderscore genuflexão\textunderscore )}
\end{itemize}
Estrado com encôsto, em que se ajoelha para orar.
\section{Genuinamente}
\begin{itemize}
\item {fónica:nu-i}
\end{itemize}
\begin{itemize}
\item {Grp. gram.:adv.}
\end{itemize}
De modo genuino.
Lidimamente.
\section{Genuinidade}
\begin{itemize}
\item {fónica:nu-i}
\end{itemize}
\begin{itemize}
\item {Grp. gram.:f.}
\end{itemize}
Qualidade daquillo que é genuíno.
\section{Genuíno}
\begin{itemize}
\item {Grp. gram.:adj.}
\end{itemize}
\begin{itemize}
\item {Proveniência:(Lat. \textunderscore genuinus\textunderscore )}
\end{itemize}
Puro, sem mistura nem alteração; natural: \textunderscore vinho genuíno\textunderscore .
Próprio: \textunderscore o sentido genuíno de uma phrase\textunderscore .
Sincero.
\section{Genuísco}
\begin{itemize}
\item {Grp. gram.:m.  e  adj.}
\end{itemize}
\begin{itemize}
\item {Utilização:Ant.}
\end{itemize}
O mesmo que \textunderscore genuense\textunderscore . Cf. \textunderscore Roteiro de Vasco da Gama\textunderscore .
\section{Geo...}
\begin{itemize}
\item {Grp. gram.:pref.}
\end{itemize}
\begin{itemize}
\item {Proveniência:(Do gr. \textunderscore ge\textunderscore )}
\end{itemize}
(designativo de \textunderscore terra\textunderscore )
\section{Geocêntrico}
\begin{itemize}
\item {Grp. gram.:adj.}
\end{itemize}
\begin{itemize}
\item {Proveniência:(De \textunderscore geo...\textunderscore  + \textunderscore centro\textunderscore )}
\end{itemize}
Em que a Terra se considera centro dos movimentos dos astros: \textunderscore systema geocêntrico\textunderscore .
\section{Geocentrismo}
\begin{itemize}
\item {Grp. gram.:m.}
\end{itemize}
Systema geocêntrico.
\section{Geocentrista}
\begin{itemize}
\item {Grp. gram.:m.}
\end{itemize}
Partidário do systema dos que consideram a Terra como centro do systema planetário. (Cp. \textunderscore geocêntrico\textunderscore )
\section{Geocinético}
\begin{itemize}
\item {Grp. gram.:adj.}
\end{itemize}
\begin{itemize}
\item {Proveniência:(Do gr. \textunderscore ge\textunderscore  + \textunderscore kinetikos\textunderscore )}
\end{itemize}
Diz-se dos phenómenos geológicos, que comprehendem os movimentos que modificam a superfície do globo e os movimentos convulsivos ou terremotos.
\section{Geocorisa}
\begin{itemize}
\item {Grp. gram.:f.}
\end{itemize}
\begin{itemize}
\item {Utilização:Zool.}
\end{itemize}
\begin{itemize}
\item {Proveniência:(Do gr. \textunderscore ge\textunderscore  + \textunderscore koris\textunderscore )}
\end{itemize}
Família de parasitas, a que pertence o percevejo vulgar ou da cama.
\section{Geocrático}
\begin{itemize}
\item {Grp. gram.:adj.}
\end{itemize}
\begin{itemize}
\item {Utilização:Geol.}
\end{itemize}
Diz-se do movimento de abaixamento do nível de uma corrente.
\section{Geodesia}
\begin{itemize}
\item {Grp. gram.:f.}
\end{itemize}
\begin{itemize}
\item {Proveniência:(Gr. \textunderscore geodaisia\textunderscore )}
\end{itemize}
Sciência, que trata da fórma e grandeza da Terra ou de uma parte da sua superfície.
\section{Geodesicamente}
\begin{itemize}
\item {Grp. gram.:adv.}
\end{itemize}
\begin{itemize}
\item {Proveniência:(De \textunderscore geodésico\textunderscore )}
\end{itemize}
Segundo os preceitos da Geodesia.
\section{Geodésico}
\begin{itemize}
\item {Grp. gram.:adj.}
\end{itemize}
Relativo á Geodesia: \textunderscore dirigir trabalhos geodésicos\textunderscore .
\section{Geodesígrafo}
\begin{itemize}
\item {Grp. gram.:m.}
\end{itemize}
\begin{itemize}
\item {Proveniência:(Do gr. \textunderscore geodaisia\textunderscore  + \textunderscore graphein\textunderscore )}
\end{itemize}
Instrumento geodésico, que reúne as propriedades da plancheta e do grafómetro, e cujo uso se não generalizou.
\section{Geodesígrapho}
\begin{itemize}
\item {Grp. gram.:m.}
\end{itemize}
\begin{itemize}
\item {Proveniência:(Do gr. \textunderscore geodaisia\textunderscore  + \textunderscore graphein\textunderscore )}
\end{itemize}
Instrumento geodésico, que reúne as propriedades da plancheta e do graphómetro, e cujo uso se não generalizou.
\section{Geode}
\begin{itemize}
\item {Grp. gram.:m.}
\end{itemize}
\begin{itemize}
\item {Proveniência:(Lat. \textunderscore geodes\textunderscore )}
\end{itemize}
Pedra ôca, que contém crystaes.
\section{Geodético}
\begin{itemize}
\item {Grp. gram.:adj.}
\end{itemize}
Relativo ao geode. Cf. Latino, \textunderscore Humboldt\textunderscore , 127 e 166.
\section{Geodinâmica}
\begin{itemize}
\item {Grp. gram.:f.}
\end{itemize}
\begin{itemize}
\item {Proveniência:(De \textunderscore geo...\textunderscore  + \textunderscore dinâmica\textunderscore )}
\end{itemize}
Parte da Geologia, que trata das acções e fenómenos, que se passam entre as diversas partes componentes da Terra, e das modificações que daí resultam.
\section{Geodynâmica}
\begin{itemize}
\item {Grp. gram.:f.}
\end{itemize}
\begin{itemize}
\item {Proveniência:(De \textunderscore geo...\textunderscore  + \textunderscore dynâmica\textunderscore )}
\end{itemize}
Parte da Geologia, que trata das acções e phenómenos, que se passam entre as diversas partes componentes da Terra, e das modificações que daí resultam.
\section{Geofagia}
\begin{itemize}
\item {Grp. gram.:f.}
\end{itemize}
Hábito de comer terra.
(Cp. \textunderscore geófago\textunderscore )
\section{Geófago}
\begin{itemize}
\item {Grp. gram.:m.  e  adj.}
\end{itemize}
\begin{itemize}
\item {Proveniência:(Do gr. \textunderscore ge\textunderscore  + \textunderscore phagein\textunderscore )}
\end{itemize}
O que come terra.
\section{Geogenia}
\begin{itemize}
\item {Grp. gram.:f.}
\end{itemize}
\begin{itemize}
\item {Proveniência:(Do gr. \textunderscore ge\textunderscore  + \textunderscore genes\textunderscore )}
\end{itemize}
Sciência, que se occupa da origem da Terra.
\section{Geogênico}
\begin{itemize}
\item {Grp. gram.:adj.}
\end{itemize}
Relativo á Geogenia. Cf. Latino, \textunderscore Humboldt\textunderscore , 116.
\section{Geognosia}
\begin{itemize}
\item {Grp. gram.:f.}
\end{itemize}
\begin{itemize}
\item {Proveniência:(Do gr. \textunderscore ge\textunderscore  + \textunderscore gnosis\textunderscore )}
\end{itemize}
Tratado da estructura da parte sólida do globo terrestre.
\section{Geognóstico}
\begin{itemize}
\item {Grp. gram.:adj.}
\end{itemize}
Relativo á geognosia.
\section{Geografar}
\begin{itemize}
\item {Grp. gram.:v. t.}
\end{itemize}
\begin{itemize}
\item {Utilização:Neol.}
\end{itemize}
Descrever geograficamente.--Us. por C. Laet.
\section{Geografia}
\begin{itemize}
\item {Grp. gram.:f.}
\end{itemize}
Ciência, que tem por objecto o conhecimento das diferentes partes da superfície da Terra, e da descripção e recíproca situação dessas partes.
Descripção da Terra.
Tratado geográfico.
(Cp. \textunderscore geógrafo\textunderscore )
\section{Geograficamente}
\begin{itemize}
\item {Grp. gram.:adv.}
\end{itemize}
De modo geográfico.
\section{Geográfico}
\begin{itemize}
\item {Grp. gram.:adj.}
\end{itemize}
\begin{itemize}
\item {Proveniência:(Gr. \textunderscore geographikos\textunderscore )}
\end{itemize}
Relativo á Geografia.
\section{Geographar}
\begin{itemize}
\item {Grp. gram.:v. t.}
\end{itemize}
\begin{itemize}
\item {Utilização:Neol.}
\end{itemize}
Descrever geographicamente.--Us. por C. Laet.
\section{Geographia}
\begin{itemize}
\item {Grp. gram.:f.}
\end{itemize}
Sciência, que tem por objecto o conhecimento das differentes partes da superfície da Terra, e da descripção e recíproca situação dessas partes.
Descripção da Terra.
Tratado geográphico.
(Cp. \textunderscore geógrapho\textunderscore )
\section{Geographicamente}
\begin{itemize}
\item {Grp. gram.:adv.}
\end{itemize}
De modo geográphico.
\section{Geográphico}
\begin{itemize}
\item {Grp. gram.:adj.}
\end{itemize}
\begin{itemize}
\item {Proveniência:(Gr. \textunderscore geographikos\textunderscore )}
\end{itemize}
Relativo á Geographia.
\section{Geohistória}
\begin{itemize}
\item {Grp. gram.:f.}
\end{itemize}
\begin{itemize}
\item {Proveniência:(De \textunderscore geo...\textunderscore  + \textunderscore história\textunderscore )}
\end{itemize}
História da Terra ou da sua evolução, desde a sua origem até ao seu estado actual.
\section{Geohydrographia}
\begin{itemize}
\item {Grp. gram.:f.}
\end{itemize}
Descripção da parte sólida e da parte líquida da Terra.
\section{Geoide}
\begin{itemize}
\item {Grp. gram.:m.}
\end{itemize}
\begin{itemize}
\item {Proveniência:(Do gr. \textunderscore ge\textunderscore  + \textunderscore eidos\textunderscore )}
\end{itemize}
Fórma, limitada pela superfície média dos mares, prolongada através da terra firme.
\section{Geoidrografia}
\begin{itemize}
\item {fónica:o-i}
\end{itemize}
\begin{itemize}
\item {Grp. gram.:f.}
\end{itemize}
Descripção da parte sólida e da parte líquida da Terra.
\section{Geoistória}
\begin{itemize}
\item {fónica:o-is}
\end{itemize}
\begin{itemize}
\item {Grp. gram.:f.}
\end{itemize}
\begin{itemize}
\item {Proveniência:(De \textunderscore geo...\textunderscore  + \textunderscore história\textunderscore )}
\end{itemize}
História da Terra ou da sua evolução, desde a sua origem até ao seu estado actual.
\section{Geôlho}
\begin{itemize}
\item {Grp. gram.:m.}
\end{itemize}
\begin{itemize}
\item {Utilização:Ant.}
\end{itemize}
\begin{itemize}
\item {Proveniência:(Do lat. \textunderscore genuculum\textunderscore )}
\end{itemize}
O mesmo que \textunderscore joêlho\textunderscore .
\section{Geologia}
\begin{itemize}
\item {Grp. gram.:f.}
\end{itemize}
\begin{itemize}
\item {Proveniência:(Do gr. \textunderscore ge\textunderscore  + \textunderscore logos\textunderscore )}
\end{itemize}
Sciência, que tem por objecto a história natural da Terra, o conhecimento da fórma exterior do globo, o estudo dos differentes terrenos, da formação delles e da sua posição actual.
\section{Geológico}
\begin{itemize}
\item {Grp. gram.:adj.}
\end{itemize}
Relativo á Geologia.
\section{Geológo}
\begin{itemize}
\item {Grp. gram.:m.}
\end{itemize}
\begin{itemize}
\item {Proveniência:(Do gr. \textunderscore ge\textunderscore  + \textunderscore logos\textunderscore )}
\end{itemize}
Aquelle que é versado em Geologia ou que escreve á cêrca della.
\section{Geomagnetífero}
\begin{itemize}
\item {Grp. gram.:m.}
\end{itemize}
\begin{itemize}
\item {Proveniência:(De \textunderscore geo...\textunderscore  + \textunderscore magnete\textunderscore  + lat. \textunderscore ferre\textunderscore )}
\end{itemize}
Apparelho, para applicar a electricidade á cultura do tabaco.
\section{Geomancia}
\begin{itemize}
\item {Grp. gram.:f.}
\end{itemize}
\begin{itemize}
\item {Proveniência:(Do gr. \textunderscore ge\textunderscore  + \textunderscore manteia\textunderscore )}
\end{itemize}
Adivinhação por figuras e linhas, resultantes de pontos feitos ao acaso e de círculos traçados sôbre a terra.
\section{Geomante}
\begin{itemize}
\item {Grp. gram.:m.}
\end{itemize}
Aquelle que pratica a geomancia.
\section{Geomântico}
\begin{itemize}
\item {Grp. gram.:adj.}
\end{itemize}
Relativo á geomancia.
\section{Geómetra}
\begin{itemize}
\item {Grp. gram.:m.}
\end{itemize}
\begin{itemize}
\item {Utilização:Ext.}
\end{itemize}
\begin{itemize}
\item {Proveniência:(Lat. \textunderscore geometra\textunderscore )}
\end{itemize}
Aquelle que é versado em Geometria ou escreve a respeito della.
Mathemático.
\section{Geometral}
\begin{itemize}
\item {Grp. gram.:adj.}
\end{itemize}
\begin{itemize}
\item {Grp. gram.:M.}
\end{itemize}
\begin{itemize}
\item {Utilização:Mathem.}
\end{itemize}
\begin{itemize}
\item {Proveniência:(De \textunderscore geómetra\textunderscore )}
\end{itemize}
Que mostra a dimensão, posição e fórma das partes de uma obra.
Em perspectiva, o plano sôbre que está traçada a projecção horizontal.
\section{Geometria}
\begin{itemize}
\item {Grp. gram.:f.}
\end{itemize}
\begin{itemize}
\item {Proveniência:(Lat. \textunderscore geometria\textunderscore )}
\end{itemize}
Sciência, que tem por objecto a medida das linhas, das superfícies e dos volumes.
Tratado geométrico: \textunderscore a Geometria de Euclides\textunderscore .
\section{Geometricamente}
\begin{itemize}
\item {Grp. gram.:adv.}
\end{itemize}
De modo geométrico.
\section{Geométrico}
\begin{itemize}
\item {Grp. gram.:adj.}
\end{itemize}
\begin{itemize}
\item {Proveniência:(Lat. \textunderscore geometricus\textunderscore )}
\end{itemize}
Relativo á Geometria ou conforme ás suas regras.
\section{Geometrografia}
\begin{itemize}
\item {Grp. gram.:f.}
\end{itemize}
\begin{itemize}
\item {Utilização:Mathem.}
\end{itemize}
Arte das construcções geométricas.
\section{Geometrographia}
\begin{itemize}
\item {Grp. gram.:f.}
\end{itemize}
\begin{itemize}
\item {Utilização:Mathem.}
\end{itemize}
Arte das construcções geométricas.
\section{Geomorfografia}
\begin{itemize}
\item {Grp. gram.:f.}
\end{itemize}
\begin{itemize}
\item {Proveniência:(Do gr. \textunderscore ge\textunderscore  + \textunderscore morphe\textunderscore  + \textunderscore graphein\textunderscore )}
\end{itemize}
Descripção da fórma da Terra.
\section{Geomorfologia}
\begin{itemize}
\item {Grp. gram.:f.}
\end{itemize}
\begin{itemize}
\item {Proveniência:(Do gr. \textunderscore ge\textunderscore  + \textunderscore morphe\textunderscore  + \textunderscore logos\textunderscore )}
\end{itemize}
Tratado, á cêrca da fórma da Terra.
\section{Geomorfológico}
\begin{itemize}
\item {Grp. gram.:adj.}
\end{itemize}
Relativo á geomorfologia.
\section{Geomorphographia}
\begin{itemize}
\item {Grp. gram.:f.}
\end{itemize}
\begin{itemize}
\item {Proveniência:(Do gr. \textunderscore ge\textunderscore  + \textunderscore morphe\textunderscore  + \textunderscore graphein\textunderscore )}
\end{itemize}
Descripção da fórma da Terra.
\section{Geomorphologia}
\begin{itemize}
\item {Grp. gram.:f.}
\end{itemize}
\begin{itemize}
\item {Proveniência:(Do gr. \textunderscore ge\textunderscore  + \textunderscore morphe\textunderscore  + \textunderscore logos\textunderscore )}
\end{itemize}
Tratado, á cêrca da fórma da Terra.
\section{Geomorphológico}
\begin{itemize}
\item {Grp. gram.:adj.}
\end{itemize}
Relativo á geomorphologia.
\section{Geonoma}
\begin{itemize}
\item {Grp. gram.:f.}
\end{itemize}
\begin{itemize}
\item {Proveniência:(Gr. \textunderscore geonomos\textunderscore )}
\end{itemize}
Gênero de palmeiras.
\section{Geophagia}
\begin{itemize}
\item {Grp. gram.:f.}
\end{itemize}
Hábito de comer terra.
(Cp. \textunderscore geóphago\textunderscore )
\section{Geóphago}
\begin{itemize}
\item {Grp. gram.:m.  e  adj.}
\end{itemize}
\begin{itemize}
\item {Proveniência:(Do gr. \textunderscore ge\textunderscore  + \textunderscore phagein\textunderscore )}
\end{itemize}
O que come terra.
\section{Geopitecos}
\begin{itemize}
\item {Grp. gram.:m. pl.}
\end{itemize}
\begin{itemize}
\item {Proveniência:(Do gr. \textunderscore ge\textunderscore  + \textunderscore pithekos\textunderscore )}
\end{itemize}
Tríbo de quadrúmanos.
\section{Geopithecos}
\begin{itemize}
\item {Grp. gram.:m. pl.}
\end{itemize}
\begin{itemize}
\item {Proveniência:(Do gr. \textunderscore ge\textunderscore  + \textunderscore pithekos\textunderscore )}
\end{itemize}
Tríbo de quadrúmanos.
\section{Georama}
\begin{itemize}
\item {Grp. gram.:m.}
\end{itemize}
\begin{itemize}
\item {Proveniência:(Do gr. \textunderscore ge\textunderscore  + \textunderscore orama\textunderscore )}
\end{itemize}
Representação, em relêvo, da superfície terrestre.
\section{Georgeofone}
\begin{itemize}
\item {Grp. gram.:m.}
\end{itemize}
Instrumento de metal, semelhante ao saxofone e inventado para o substituir.
\section{Georgeophone}
\begin{itemize}
\item {Grp. gram.:m.}
\end{itemize}
Instrumento de metal, semelhante ao saxofone e inventado para o substituir.
\section{Georgiano}
\begin{itemize}
\item {Grp. gram.:m.}
\end{itemize}
\begin{itemize}
\item {Proveniência:(De \textunderscore Geórgia\textunderscore , n. p.)}
\end{itemize}
Aquelle que nasceu na Geórgia.
Um dos idiomas do Cáucaso.
\section{Georgina}
\begin{itemize}
\item {Grp. gram.:f.}
\end{itemize}
\begin{itemize}
\item {Proveniência:(De \textunderscore George\textunderscore , n. p.)}
\end{itemize}
Nome que, na Europa central, se dá á dhália.
Moéda de prata, na antiga república de Gênova.
\section{Georgíneas}
\begin{itemize}
\item {Grp. gram.:f. pl.}
\end{itemize}
Família de plantas, que têm por typo a georgina.
\section{Geosauro}
\begin{itemize}
\item {fónica:sau}
\end{itemize}
\begin{itemize}
\item {Grp. gram.:m.}
\end{itemize}
\begin{itemize}
\item {Proveniência:(Do gr. \textunderscore ge\textunderscore  + \textunderscore sauros\textunderscore )}
\end{itemize}
Reptil fóssil, semelhante ao crocodilo.
\section{Geoso}
\begin{itemize}
\item {Grp. gram.:adj.}
\end{itemize}
\begin{itemize}
\item {Proveniência:(De \textunderscore gear\textunderscore )}
\end{itemize}
Em que há geada.
\section{Geossauro}
\begin{itemize}
\item {Grp. gram.:m.}
\end{itemize}
\begin{itemize}
\item {Proveniência:(Do gr. \textunderscore ge\textunderscore  + \textunderscore sauros\textunderscore )}
\end{itemize}
Reptil fóssil, semelhante ao crocodilo.
\section{Gefirenses}
\begin{itemize}
\item {Grp. gram.:m. pl.}
\end{itemize}
Sacerdotes de Gefireia, (um dos cognomes de Ceres). Cf. Castilho, \textunderscore Fastos\textunderscore , II, 662.
\section{Geostática}
\begin{itemize}
\item {Grp. gram.:f.}
\end{itemize}
\begin{itemize}
\item {Proveniência:(De \textunderscore geo...\textunderscore  + \textunderscore estática\textunderscore )}
\end{itemize}
Equilibrio do globo terrestre.
\section{Geotaxia}
\begin{itemize}
\item {fónica:csi}
\end{itemize}
\begin{itemize}
\item {Grp. gram.:f.}
\end{itemize}
\begin{itemize}
\item {Utilização:Bot.}
\end{itemize}
\begin{itemize}
\item {Proveniência:(Do gr. \textunderscore ge\textunderscore  + \textunderscore taxis\textunderscore )}
\end{itemize}
Propriedade, attribuida a certos órgãos vegetaes, de serem attrahidos ou repellidos pela gravidade.
\section{Geotectónica}
\begin{itemize}
\item {Grp. gram.:f.}
\end{itemize}
Subdivisão da Geologia, em que se trata das massas rochosas, com relação á sua fórma geral e disposição no globo terrestre.
\section{Geotérmico}
\begin{itemize}
\item {Grp. gram.:adj.}
\end{itemize}
\begin{itemize}
\item {Proveniência:(Do gr. \textunderscore ge\textunderscore  + \textunderscore thermon\textunderscore )}
\end{itemize}
Diz-se do grau de temperatura, correspondente á diferença de profundidade terrestre, correlativa de 1° do termómetro centígrado.
\section{Geothérmico}
\begin{itemize}
\item {Grp. gram.:adj.}
\end{itemize}
\begin{itemize}
\item {Proveniência:(Do gr. \textunderscore ge\textunderscore  + \textunderscore thermon\textunderscore )}
\end{itemize}
Diz-se do grau de temperatura, correspondente á differença de profundidade terrestre, correlativa de 1° do thermómetro centígrado.
\section{Geotropismo}
\begin{itemize}
\item {Grp. gram.:m.}
\end{itemize}
Qualidade, que as plantas têm, de dirigir sempre a radícula para a terra e o caulículo para o céu.
O mesmo que \textunderscore geotaxia\textunderscore .
(Cp. \textunderscore geótropo\textunderscore )
\section{Geótropo}
\begin{itemize}
\item {Grp. gram.:m.}
\end{itemize}
\begin{itemize}
\item {Proveniência:(Do gr. \textunderscore ge\textunderscore  + \textunderscore tropos\textunderscore )}
\end{itemize}
Gênero de insectos coleópteros pentâmeros.
\section{Gephyrenses}
\begin{itemize}
\item {Grp. gram.:m. pl.}
\end{itemize}
Sacerdotes de Gephyreia, (um dos cognomes de Ceres). Cf. Castilho, \textunderscore Fastos\textunderscore , II, 662.
\section{Gepiás}
\begin{itemize}
\item {Grp. gram.:m. pl.}
\end{itemize}
Uma das tríbos indígenas da região do Amazonas.
\section{Gequitibá}
\begin{itemize}
\item {Grp. gram.:m.}
\end{itemize}
Espécie de líchen, (\textunderscore pyxidaria macrocarpa\textunderscore ).
\section{Gera}
\begin{itemize}
\item {Grp. gram.:f.}
\end{itemize}
\begin{itemize}
\item {Utilização:Gír.}
\end{itemize}
Carne.
(Alter. do fr. \textunderscore chaire\textunderscore ?)
\section{Geração}
\begin{itemize}
\item {Grp. gram.:f.}
\end{itemize}
\begin{itemize}
\item {Utilização:Ext.}
\end{itemize}
\begin{itemize}
\item {Proveniência:(Do lat. \textunderscore generatio\textunderscore )}
\end{itemize}
Acto de gerar ou sêr gerado.
Grau de filiação; linhagem.
Genealogia.
Conjunto dos homens da mesma época: \textunderscore a nova geração conta soberbos talentos\textunderscore .
Duração média da vida do homem: \textunderscore aquella crença passou de geração para geração\textunderscore .
Formação, desenvolvimento; derivação.
\section{Gèração}
\begin{itemize}
\item {Grp. gram.:f.}
\end{itemize}
\begin{itemize}
\item {Utilização:Ext.}
\end{itemize}
\begin{itemize}
\item {Proveniência:(Do lat. \textunderscore generatio\textunderscore )}
\end{itemize}
Acto de gerar ou sêr gerado.
Grau de filiação; linhagem.
Genealogia.
Conjunto dos homens da mesma época: \textunderscore a nova gèração conta soberbos talentos\textunderscore .
Duração média da vida do homem: \textunderscore aquella crença passou de gèração para gèração\textunderscore .
Formação, desenvolvimento; derivação.
\section{Gerador}
\begin{itemize}
\item {Grp. gram.:adj.}
\end{itemize}
\begin{itemize}
\item {Grp. gram.:M.}
\end{itemize}
\begin{itemize}
\item {Proveniência:(Do lat. \textunderscore generator\textunderscore )}
\end{itemize}
Que gera.
Aquelle que gera.
Aquelle que cria.
Aquelle que produz.
Parte das máquinas de vapor, em que o vapor se produz.
\section{Geraes}
\begin{itemize}
\item {Grp. gram.:m. pl.}
\end{itemize}
\begin{itemize}
\item {Utilização:Bras}
\end{itemize}
\begin{itemize}
\item {Proveniência:(De \textunderscore geral\textunderscore ^2?)}
\end{itemize}
Diz-se que alguém está nos seus \textunderscore geraes\textunderscore , quando está satisfeito com a posição que occupa ou quando não cabe em si de contente.
\section{Gerais}
\begin{itemize}
\item {Grp. gram.:m. pl.}
\end{itemize}
\begin{itemize}
\item {Utilização:Bras}
\end{itemize}
\begin{itemize}
\item {Proveniência:(De \textunderscore geral\textunderscore ^2?)}
\end{itemize}
Diz-se que alguém está nos seus \textunderscore gerais\textunderscore , quando está satisfeito com a posição que ocupa ou quando não cabe em si de contente.
\section{Geral}
\begin{itemize}
\item {Grp. gram.:adj.}
\end{itemize}
\begin{itemize}
\item {Utilização:Ext.}
\end{itemize}
\begin{itemize}
\item {Grp. gram.:M.}
\end{itemize}
\begin{itemize}
\item {Grp. gram.:Pl.}
\end{itemize}
\begin{itemize}
\item {Proveniência:(Lat. \textunderscore generalis\textunderscore )}
\end{itemize}
Commum á maior parte; genérico: \textunderscore o bem geral\textunderscore .
Universal.
A maior parte.
Chefe de Ordem religiosa.
Acto de fazer as vasas no jôgo, não fazendo nenhuma o outro ou outros parceiros.
Claustro, em que estão as aulas, na Universidade de Coimbra.
\section{Geral}
\begin{itemize}
\item {Grp. gram.:m.}
\end{itemize}
\begin{itemize}
\item {Utilização:Bras. do N}
\end{itemize}
\begin{itemize}
\item {Grp. gram.:Pl.}
\end{itemize}
\begin{itemize}
\item {Utilização:Bras}
\end{itemize}
Terreno, coberto de mato; charneca.
Lugares ermos, longínquos.
\section{Geral}
\begin{itemize}
\item {Grp. gram.:m.}
\end{itemize}
\begin{itemize}
\item {Utilização:Ant.}
\end{itemize}
O mesmo que \textunderscore general\textunderscore . Cf. Pant. de Aveiro, \textunderscore Itiner.\textunderscore , 24 v.^o e 167, (2.^a ed.).
\section{Geralista}
\begin{itemize}
\item {Grp. gram.:m.}
\end{itemize}
\begin{itemize}
\item {Utilização:Bras}
\end{itemize}
Habitante de Minas-Geraes; mineiro.
\section{Geralmente}
\begin{itemize}
\item {Grp. gram.:adv.}
\end{itemize}
De modo geral.
Em geral; commummente; por via de regra.
\section{Geraniáceas}
\begin{itemize}
\item {Grp. gram.:f. pl.}
\end{itemize}
Família de plantas, que têm por typo o gerânio.
\section{Gerânio}
\begin{itemize}
\item {Grp. gram.:m.}
\end{itemize}
\begin{itemize}
\item {Proveniência:(Gr. \textunderscore geranion\textunderscore )}
\end{itemize}
Gênero de plantas, cujo fruto é composto de cinco cápsulas, terminando cada uma em aresta.
\section{Gerar}
\begin{itemize}
\item {Grp. gram.:v. t.}
\end{itemize}
\begin{itemize}
\item {Grp. gram.:V. i.  e  p.}
\end{itemize}
\begin{itemize}
\item {Proveniência:(Do lat. \textunderscore generare\textunderscore )}
\end{itemize}
Dar o sêr a; criar: \textunderscore gerar um filho\textunderscore .
Produzir.
Formar.
Causar: \textunderscore gerar desordens\textunderscore .
Fazer produzir.
Nascer; formar-se.
\section{Gerárdia}
\begin{itemize}
\item {Grp. gram.:f.}
\end{itemize}
\begin{itemize}
\item {Proveniência:(De \textunderscore Gérard\textunderscore , n. p.)}
\end{itemize}
Gênero de plantas americanas.
\section{Gerardíadas}
\begin{itemize}
\item {Grp. gram.:f. pl.}
\end{itemize}
Tribo de plantas, que tem por typo a gerárdia.
\section{Geratacaca}
\begin{itemize}
\item {Grp. gram.:f.}
\end{itemize}
(V.manacá)
\section{Geratriz}
\begin{itemize}
\item {Grp. gram.:adj.}
\end{itemize}
\begin{itemize}
\item {Grp. gram.:F.}
\end{itemize}
\begin{itemize}
\item {Proveniência:(Do lat. \textunderscore generatrix\textunderscore )}
\end{itemize}
Que gera: \textunderscore linha geratriz\textunderscore .
Aquella que gera.
\section{Gerbão}
\begin{itemize}
\item {Grp. gram.:m.}
\end{itemize}
Designação vulgar do \textunderscore urgebão\textunderscore . Cf. S. Costa, \textunderscore Hist. das Pl. Med.\textunderscore 
\section{Gerbéria}
\begin{itemize}
\item {Grp. gram.:f.}
\end{itemize}
\begin{itemize}
\item {Proveniência:(De \textunderscore Gerber\textunderscore , n. p.)}
\end{itemize}
Gênero de plantas do Cabo da Bôa-Esperança.
\section{Gerbo}
\begin{itemize}
\item {Grp. gram.:m.}
\end{itemize}
Mammífero roedor, (\textunderscore dipus gerboa\textunderscore ).
\section{Gereba}
\begin{itemize}
\item {Grp. gram.:f.}
\end{itemize}
\begin{itemize}
\item {Utilização:Bras. do N}
\end{itemize}
\begin{itemize}
\item {Grp. gram.:M.}
\end{itemize}
Ave aquática, negra.
Espécie de urubu, de cabeça encarnada.
Certa figura de fogo de artifício.
Arame, com uma bucha de estopa na extremidade, para limpeza de canalizações.
Indivíduo desajeitado e gingão.
\section{Gerecer-se}
\begin{itemize}
\item {Grp. gram.:v. p.}
\end{itemize}
\begin{itemize}
\item {Utilização:Prov.}
\end{itemize}
\begin{itemize}
\item {Utilização:beir.}
\end{itemize}
\begin{itemize}
\item {Utilização:Ant.}
\end{itemize}
\begin{itemize}
\item {Proveniência:(Do lat. \textunderscore generescere\textunderscore )}
\end{itemize}
Formar-se, apparecer, (especialmente, falando-se de tumores, furúnculos, etc.). Cf. G. Vicente.
\section{Gereiro}
\begin{itemize}
\item {Grp. gram.:m.}
\end{itemize}
\begin{itemize}
\item {Utilização:Gír.}
\end{itemize}
\begin{itemize}
\item {Proveniência:(De \textunderscore gera\textunderscore )}
\end{itemize}
Açougue.
\section{Gerencia}
\begin{itemize}
\item {Grp. gram.:f.}
\end{itemize}
Acto de gerir; funcções de gerente; administração.
Exercício de funcções administrativas.
(Cp. \textunderscore gerente\textunderscore )
\section{Gerente}
\begin{itemize}
\item {Grp. gram.:m. ,  f.  e  adj.}
\end{itemize}
\begin{itemize}
\item {Proveniência:(Lat. \textunderscore gerens\textunderscore )}
\end{itemize}
Pessôa que gere ou administra negócios.
\section{Gereraca}
\begin{itemize}
\item {Grp. gram.:f.}
\end{itemize}
Cobra do Brasil, muito venenosa.
\section{Geresiano}
\begin{itemize}
\item {Grp. gram.:adj.}
\end{itemize}
Relativo ao Gerês.
\section{Gergelim}
\begin{itemize}
\item {Grp. gram.:m.}
\end{itemize}
\begin{itemize}
\item {Proveniência:(Do ár. \textunderscore jonjoli\textunderscore )}
\end{itemize}
Planta bignoniácea, (\textunderscore seramum indicum\textunderscore ).
Semente desta planta.
Bolo, em que entram sementes da mesma planta.
\section{Gergilada}
\begin{itemize}
\item {Grp. gram.:f.}
\end{itemize}
Bolo, em cuja composição entra gergelim, farinha e calda de açúcar.
(Por \textunderscore gergelada\textunderscore , de \textunderscore gergelim\textunderscore )
\section{Geribanda}
\begin{itemize}
\item {Grp. gram.:f.}
\end{itemize}
\begin{itemize}
\item {Utilização:Pop.}
\end{itemize}
O mesmo que \textunderscore sarabanda\textunderscore .
\section{Gericada}
\begin{itemize}
\item {Grp. gram.:f.}
\end{itemize}
Porção de gericos, burricada.
\section{Gerical}
\begin{itemize}
\item {Grp. gram.:adj.}
\end{itemize}
Relativo a gerico.
Próprio de gerico.
\section{Gerico}
\begin{itemize}
\item {Grp. gram.:m.}
\end{itemize}
\begin{itemize}
\item {Utilização:Fam.}
\end{itemize}
\begin{itemize}
\item {Proveniência:(Do lat. \textunderscore gericus\textunderscore , de \textunderscore gerere\textunderscore , trazer, transportar? Cp. B. Pereira, vb. \textunderscore gericus\textunderscore )}
\end{itemize}
O mesmo que \textunderscore jumento\textunderscore .
\section{Gericocim}
\begin{itemize}
\item {Grp. gram.:m.}
\end{itemize}
\begin{itemize}
\item {Utilização:Ant.}
\end{itemize}
\begin{itemize}
\item {Proveniência:(De \textunderscore gerico\textunderscore ?)}
\end{itemize}
Asno?:«\textunderscore Que gericocins, salvanor!\textunderscore »G. Vicente, I, 217.
\section{Gerifalte}
\begin{itemize}
\item {Grp. gram.:m.}
\end{itemize}
Espécie de falcão, robusto e airoso.
(Cp. ár. \textunderscore zorafate\textunderscore , tirado talvez do port. ou do cast.)
\section{Gerigonça}
\begin{itemize}
\item {Grp. gram.:f.}
\end{itemize}
(V.geringonça)
\section{Gerigoto}
\begin{itemize}
\item {fónica:gô}
\end{itemize}
\begin{itemize}
\item {Grp. gram.:adj.}
\end{itemize}
\begin{itemize}
\item {Utilização:Prov.}
\end{itemize}
Ligeiro, lesto.
Fino; activo.
\section{Gerimendro}
\begin{itemize}
\item {Grp. gram.:m.}
\end{itemize}
\begin{itemize}
\item {Utilização:Prov.}
\end{itemize}
\begin{itemize}
\item {Utilização:dur.}
\end{itemize}
Pêssego; o mesmo que \textunderscore gilmendes\textunderscore .
\section{Gerimum}
\begin{itemize}
\item {Grp. gram.:m.}
\end{itemize}
(V.jirimu)
\section{Geringonça}
\begin{itemize}
\item {Grp. gram.:f.}
\end{itemize}
\begin{itemize}
\item {Utilização:Pleb.}
\end{itemize}
Calão.
Gíria.
Coisa mal feita e que facilmente se destrói.
(Cast. \textunderscore jerigonza\textunderscore )
\section{Gerir}
\begin{itemize}
\item {Grp. gram.:v. t.}
\end{itemize}
\begin{itemize}
\item {Proveniência:(Lat. \textunderscore gerere\textunderscore )}
\end{itemize}
Administrar, dirigir: \textunderscore gerir uma fábrica\textunderscore .
Regular.
\section{Geriticaca}
\begin{itemize}
\item {Grp. gram.:f.}
\end{itemize}
Formoso mammífero do Brasil, pouco menor que um rato grande.
\section{Geriza}
\begin{itemize}
\item {Grp. gram.:f.}
\end{itemize}
\begin{itemize}
\item {Utilização:bras}
\end{itemize}
\begin{itemize}
\item {Utilização:Ant.}
\end{itemize}
Raiva, ira; ódio.
\section{Germanada}
\begin{itemize}
\item {Grp. gram.:f.}
\end{itemize}
\begin{itemize}
\item {Proveniência:(De \textunderscore germano\textunderscore )}
\end{itemize}
Conjunto de irmãos; parentela.
\section{Germanadamente}
\begin{itemize}
\item {Grp. gram.:adv.}
\end{itemize}
\begin{itemize}
\item {Utilização:Des.}
\end{itemize}
\begin{itemize}
\item {Proveniência:(De \textunderscore germano\textunderscore )}
\end{itemize}
Irmanmente; com união.
\section{Germanal}
\begin{itemize}
\item {Grp. gram.:adj.}
\end{itemize}
Próprio de irmãos. Cf. Filinto, XIV, 248.
\section{Germanar}
\begin{itemize}
\item {Grp. gram.:v. t.}
\end{itemize}
\begin{itemize}
\item {Proveniência:(De \textunderscore germano\textunderscore )}
\end{itemize}
Tornar semelhante; irmanar.
Reunir.
\section{Germândrea}
\begin{itemize}
\item {Grp. gram.:f.}
\end{itemize}
Gênero de plantas labiadas.
\section{Germanía}
\begin{itemize}
\item {Grp. gram.:f.}
\end{itemize}
\begin{itemize}
\item {Utilização:Des.}
\end{itemize}
O mesmo que \textunderscore gíria\textunderscore , \textunderscore calão\textunderscore . Cf. \textunderscore Eufrosina\textunderscore , 278.
\section{Germanicamente}
\begin{itemize}
\item {Grp. gram.:adv.}
\end{itemize}
\begin{itemize}
\item {Proveniência:(De \textunderscore germânico\textunderscore )}
\end{itemize}
Á maneira dos Germanos ou dos Alemães.
\section{Germânico}
\begin{itemize}
\item {Grp. gram.:adj.}
\end{itemize}
\begin{itemize}
\item {Grp. gram.:M.}
\end{itemize}
\begin{itemize}
\item {Proveniência:(Lat. \textunderscore germanicus\textunderscore )}
\end{itemize}
Relativo á Germânia.
Conjunto das línguas dos povos germânicos.
\section{Germanidade}
\begin{itemize}
\item {Grp. gram.:f.}
\end{itemize}
\begin{itemize}
\item {Utilização:Ant.}
\end{itemize}
\begin{itemize}
\item {Proveniência:(De \textunderscore germano\textunderscore )}
\end{itemize}
O mesmo que \textunderscore irmandade\textunderscore .
\section{Germanismo}
\begin{itemize}
\item {Grp. gram.:m.}
\end{itemize}
\begin{itemize}
\item {Proveniência:(Do lat. \textunderscore Germania\textunderscore , n. p.)}
\end{itemize}
Palavra ou phrase, peculiar á língua aleman.
Imitações de coisas alemans.
Amor excessivo a tudo que procede de Alemanha.
\section{Germanissimamente}
\begin{itemize}
\item {Grp. gram.:adv.}
\end{itemize}
\begin{itemize}
\item {Utilização:Des.}
\end{itemize}
\begin{itemize}
\item {Proveniência:(De \textunderscore germanissimo\textunderscore , sup. de \textunderscore germano\textunderscore )}
\end{itemize}
Á maneira de bons irmãos.
\section{Germanista}
\begin{itemize}
\item {Grp. gram.:m.}
\end{itemize}
Aquelle que estuda as línguas e literaturas germânicas.
(Cp. \textunderscore germanismo\textunderscore )
\section{Germanização}
\begin{itemize}
\item {Grp. gram.:f.}
\end{itemize}
Acto ou effeito de germanizar.
\section{Germanizar}
\begin{itemize}
\item {Grp. gram.:v. t.}
\end{itemize}
\begin{itemize}
\item {Proveniência:(De \textunderscore Germanos\textunderscore )}
\end{itemize}
Dar feição aleman a.
\section{Germanmente}
\begin{itemize}
\item {Grp. gram.:adv.}
\end{itemize}
\begin{itemize}
\item {Utilização:Ant.}
\end{itemize}
O mesmo que \textunderscore irmanmente\textunderscore .
\section{Germano}
\begin{itemize}
\item {Grp. gram.:m.  e  adj.}
\end{itemize}
\begin{itemize}
\item {Utilização:Fig.}
\end{itemize}
\begin{itemize}
\item {Proveniência:(Lat. \textunderscore germanus\textunderscore )}
\end{itemize}
O que procedeu do mesmo pai e da mesma mãe, (falando-se de irmãos).
Verdadeiro, puro.
\section{Germanos}
\begin{itemize}
\item {Grp. gram.:m. pl.}
\end{itemize}
\begin{itemize}
\item {Proveniência:(Lat. \textunderscore germani\textunderscore )}
\end{itemize}
Povos, que habitavam entre o Rheno, o Danúbio, o Vístula e o mar.
\section{Germão}
\begin{itemize}
\item {Grp. gram.:m.}
\end{itemize}
\begin{itemize}
\item {Proveniência:(Fr. \textunderscore germon\textunderscore )}
\end{itemize}
Golfinho.
Gênero de peixes, cuja espécie typo se encontra no mar da Mancha.
\section{Germão}
\begin{itemize}
\item {Grp. gram.:m.  e  adj.}
\end{itemize}
O mesmo que \textunderscore germano\textunderscore . Cf. Filinto, IX, 221.
\section{Germe}
\begin{itemize}
\item {Grp. gram.:m.}
\end{itemize}
\begin{itemize}
\item {Utilização:Ext.}
\end{itemize}
\begin{itemize}
\item {Proveniência:(Lat. \textunderscore germen\textunderscore )}
\end{itemize}
Princípio de um novo sêr; embryão.
Rudimento de qualquer sêr organizado, vegetal ou animal.
Parte da semente, de que se fórma a planta.
Cicatrícula do ovo das aves.
Causa, origem.
Estado rudimentar.
\section{Germen}
\begin{itemize}
\item {Grp. gram.:m.}
\end{itemize}
(V.germe)
\section{Germicida}
\begin{itemize}
\item {Grp. gram.:adj.}
\end{itemize}
\begin{itemize}
\item {Proveniência:(Do lat. \textunderscore germen\textunderscore  + \textunderscore caedere\textunderscore )}
\end{itemize}
Que destrói os germes ou embryões.
\section{Germinação}
\begin{itemize}
\item {Grp. gram.:f.}
\end{itemize}
\begin{itemize}
\item {Utilização:Fig.}
\end{itemize}
\begin{itemize}
\item {Proveniência:(Lat. \textunderscore germinatio\textunderscore )}
\end{itemize}
Desenvolvimento do embryão vegetal, desembaraçando-se das capas da semente.
Desenvolvimento do germe dos bolbos.
Evolução; expansão lenta.
\section{Germinadoiro}
\begin{itemize}
\item {Grp. gram.:m.}
\end{itemize}
\begin{itemize}
\item {Proveniência:(De \textunderscore germinar\textunderscore )}
\end{itemize}
Lugar subterrâneo, em que germina a cevada em montão, para o fabríco da cerveja.
\section{Germinador}
\begin{itemize}
\item {Grp. gram.:adj.}
\end{itemize}
\begin{itemize}
\item {Proveniência:(Lat. \textunderscore germinator\textunderscore )}
\end{itemize}
Que faz germinar.
\section{Germinadouro}
\begin{itemize}
\item {Grp. gram.:m.}
\end{itemize}
\begin{itemize}
\item {Proveniência:(De \textunderscore germinar\textunderscore )}
\end{itemize}
Lugar subterrâneo, em que germina a cevada em montão, para o fabríco da cerveja.
\section{Germinal}
\begin{itemize}
\item {Grp. gram.:adj.}
\end{itemize}
\begin{itemize}
\item {Grp. gram.:M.}
\end{itemize}
\begin{itemize}
\item {Proveniência:(Do lat. \textunderscore germen\textunderscore )}
\end{itemize}
Relativo ao germe.
Sétimo mês do anno, no calendário da primeira república francesa.
\section{Germinante}
\begin{itemize}
\item {Grp. gram.:adj.}
\end{itemize}
\begin{itemize}
\item {Proveniência:(Lat. \textunderscore germinans\textunderscore )}
\end{itemize}
Que germina.
\section{Germinar}
\begin{itemize}
\item {Grp. gram.:v. i.}
\end{itemize}
\begin{itemize}
\item {Utilização:Fig.}
\end{itemize}
\begin{itemize}
\item {Grp. gram.:V. t.}
\end{itemize}
\begin{itemize}
\item {Utilização:P. us.}
\end{itemize}
\begin{itemize}
\item {Proveniência:(Lat. \textunderscore germinare\textunderscore )}
\end{itemize}
Começar a desenvolver-se, (falando-se da semente, bolbos, etc.).
Deitar rebentos; grelar.
Têr princípio.
Desenvolver-se.
Produzir; dar causa a.
\section{Germinativo}
\begin{itemize}
\item {Grp. gram.:adj.}
\end{itemize}
\begin{itemize}
\item {Proveniência:(Lat. \textunderscore germinativus\textunderscore )}
\end{itemize}
O mesmo que \textunderscore germinante\textunderscore .
\section{Germindade}
\begin{itemize}
\item {Grp. gram.:f.}
\end{itemize}
\begin{itemize}
\item {Utilização:Prov.}
\end{itemize}
\begin{itemize}
\item {Utilização:minh.}
\end{itemize}
O mesmo que \textunderscore germanidade\textunderscore .
Conjunto de irmãos.
Parentela. (Colhido em Barcelos)
\section{Germiníparo}
\begin{itemize}
\item {Grp. gram.:adj.}
\end{itemize}
\begin{itemize}
\item {Proveniência:(Do lat. \textunderscore germen\textunderscore  + \textunderscore parere\textunderscore )}
\end{itemize}
Que se reproduz por meio de germes.
\section{Germinista}
\begin{itemize}
\item {Grp. gram.:m.}
\end{itemize}
\begin{itemize}
\item {Grp. gram.:Adj.}
\end{itemize}
\begin{itemize}
\item {Proveniência:(De \textunderscore germen\textunderscore )}
\end{itemize}
Sectário da theoria, segundo a qual as partes mutiladas de certos seres se reproduzem por meio dos germes reparadores.
Relativo ao germe.
\section{Germo}
\begin{itemize}
\item {Grp. gram.:m.}
\end{itemize}
\begin{itemize}
\item {Utilização:Des.}
\end{itemize}
O mesmo que germe:«\textunderscore ...aconteceu que... as sementes de dessabor... até alli germo... lançassem hástias...\textunderscore »Filinto, VIII, 243.
\section{Gerocomia}
\begin{itemize}
\item {Grp. gram.:f.}
\end{itemize}
\begin{itemize}
\item {Proveniência:(Do gr. \textunderscore geras\textunderscore  + \textunderscore komein\textunderscore )}
\end{itemize}
Hygiene dos velhos.
\section{Geroglifo}
\textunderscore m.\textunderscore  (e der.)
(V. \textunderscore jeroglifo\textunderscore , etc.)
\section{Geroglypho}
\textunderscore m.\textunderscore  (e der.)
(V. \textunderscore jeroglypho\textunderscore , etc.)
\section{Geromó}
\begin{itemize}
\item {Grp. gram.:m.}
\end{itemize}
(V.girimu)
\section{Gerontocómio}
\begin{itemize}
\item {Grp. gram.:m.}
\end{itemize}
\begin{itemize}
\item {Proveniência:(Lat. \textunderscore gerontocomium\textunderscore )}
\end{itemize}
Hospício para velhos, no Baixo-Império.
\section{Gerontotrófio}
\begin{itemize}
\item {Grp. gram.:m.}
\end{itemize}
O mesmo que \textunderscore gerontocómio\textunderscore .
\section{Gerontotróphio}
\begin{itemize}
\item {Grp. gram.:m.}
\end{itemize}
O mesmo que \textunderscore gerontocómio\textunderscore .
\section{Geropiga}
\begin{itemize}
\item {Grp. gram.:f.}
\end{itemize}
Vinho, a que se suspendeu a fermentação, por meio de aguardente.
Bebida, feita de mosto, aguardente e açúcar.
(Por \textunderscore xaropiga\textunderscore , de \textunderscore xarope\textunderscore ?)
\section{Gerozemo}
\begin{itemize}
\item {Grp. gram.:m.}
\end{itemize}
Espécie de desembargador nos antigos tribunaes de Nanquim.
\section{Gerres}
\begin{itemize}
\item {Grp. gram.:m. pl.}
\end{itemize}
\begin{itemize}
\item {Proveniência:(Lat. \textunderscore gerres\textunderscore )}
\end{itemize}
Insectos frontirostros, da ordem dos hemípteros.
\section{Gerundial}
\begin{itemize}
\item {Grp. gram.:adj.}
\end{itemize}
\begin{itemize}
\item {Utilização:Gram.}
\end{itemize}
Relativo ao gerúndio.
\section{Gerundífico}
\begin{itemize}
\item {Grp. gram.:adj.}
\end{itemize}
Mal feito, (falando-se de versos, como os faria Frei Gerúndio de Compassas, se se tivesse dado, em vez da prédica á metrificância):«\textunderscore ...sôbre os teus versos gerundíficos...\textunderscore »Filinto, VIII, 243.
\section{Gerúndio}
\begin{itemize}
\item {Grp. gram.:m.}
\end{itemize}
\begin{itemize}
\item {Utilização:Gram.}
\end{itemize}
\begin{itemize}
\item {Proveniência:(Lat. \textunderscore gerundium\textunderscore )}
\end{itemize}
Forma invariável, ligada aos verbos, resultante da mudança do \textunderscore r\textunderscore  final do infinitivo em \textunderscore ndo\textunderscore : \textunderscore louvando\textunderscore , \textunderscore devendo\textunderscore , \textunderscore fugindo...\textunderscore 
\section{Gerundivo}
\begin{itemize}
\item {Grp. gram.:m.}
\end{itemize}
\begin{itemize}
\item {Proveniência:(Lat. \textunderscore gerundivus\textunderscore )}
\end{itemize}
Designação moderna da fórma verbal latina, terminada em \textunderscore ndus\textunderscore : \textunderscore laudandus\textunderscore , \textunderscore delendus\textunderscore , etc.
\section{Gervão}
\begin{itemize}
\item {Grp. gram.:m.}
\end{itemize}
(V.ogervão)
\section{Gervília}
\begin{itemize}
\item {Grp. gram.:f.}
\end{itemize}
Gênero de molluscos fósseis.
\section{Gerzelim}
\begin{itemize}
\item {Grp. gram.:m.}
\end{itemize}
\begin{itemize}
\item {Utilização:Bras}
\end{itemize}
O mesmo que \textunderscore gergelim\textunderscore .
\section{Gerzerli}
\begin{itemize}
\item {Grp. gram.:m.}
\end{itemize}
O mesmo que \textunderscore gergelim\textunderscore . Cf. Garrett, \textunderscore Romanceiro\textunderscore , II, 9.
\section{Gés}
\begin{itemize}
\item {Grp. gram.:m. pl.}
\end{itemize}
Nome de várias tríbos dos tupinambás, no Brasil.
\section{Gesneráceas}
\begin{itemize}
\item {Grp. gram.:f. pl.}
\end{itemize}
(V.gesneriáceas)
\section{Gesnéreas}
\begin{itemize}
\item {Grp. gram.:f. pl.}
\end{itemize}
(V.gesneriáceas)
\section{Gesnéria}
\begin{itemize}
\item {Grp. gram.:f.}
\end{itemize}
\begin{itemize}
\item {Proveniência:(De \textunderscore Gesner\textunderscore , n. p.)}
\end{itemize}
Gênero de plantas das regiões quentes da América.
\section{Gesneriáceas}
\begin{itemize}
\item {Grp. gram.:f. pl.}
\end{itemize}
Família de plantas, que têm por typo a gesnéria.
\section{Geso}
\begin{itemize}
\item {fónica:gê}
\end{itemize}
\begin{itemize}
\item {Grp. gram.:m.}
\end{itemize}
\begin{itemize}
\item {Proveniência:(Lat. \textunderscore gaesum\textunderscore )}
\end{itemize}
Lança pesada e comprida dos Gállios.--Filinto, VI, 189, chama-lhe erradamente lança africana.
\section{Gessada}
\begin{itemize}
\item {Grp. gram.:f.}
\end{itemize}
\begin{itemize}
\item {Proveniência:(De \textunderscore gessar\textunderscore )}
\end{itemize}
Massa, em que os doiradores assentam o oiro, e que é formada de bolo-armênio, hematite e algumas gotas de azeite.
\section{Gessal}
\begin{itemize}
\item {Grp. gram.:m.}
\end{itemize}
O mesmo que \textunderscore gesseira\textunderscore .
\section{Gessar}
\begin{itemize}
\item {Grp. gram.:v. t.}
\end{itemize}
Revestir com gêsso, para pintar ou doirar; estucar.
\section{Gesseira}
\begin{itemize}
\item {Grp. gram.:f.}
\end{itemize}
Terreno, donde se extrai o gêsso.
\section{Gesseiro}
\begin{itemize}
\item {Grp. gram.:m.}
\end{itemize}
Aquelle que trabalha em gêsso.
\section{Gessete}
\begin{itemize}
\item {fónica:sê}
\end{itemize}
\begin{itemize}
\item {Grp. gram.:m.}
\end{itemize}
Pedaço de gêsso, com que se esboçam desenhos ornamentaes.
\section{Gesso}
\begin{itemize}
\item {fónica:gê}
\end{itemize}
\begin{itemize}
\item {Grp. gram.:m.}
\end{itemize}
\begin{itemize}
\item {Utilização:Ext.}
\end{itemize}
\begin{itemize}
\item {Utilização:Gír.}
\end{itemize}
\begin{itemize}
\item {Proveniência:(Do lat. hyp. \textunderscore gipsum\textunderscore )}
\end{itemize}
Sulfato de cal hydratado.
Objecto de arte, moldado em gêsso.
Vinho.
\section{Gessoso}
\begin{itemize}
\item {Grp. gram.:adj.}
\end{itemize}
Diz-se do terreno, em que domina o gêsso.
\section{Gesta}
\begin{itemize}
\item {Grp. gram.:f.}
\end{itemize}
\begin{itemize}
\item {Proveniência:(Lat. \textunderscore gesta\textunderscore )}
\end{itemize}
História.
Acontecimento histórico.
Façanha, feitos guerreiros. Cf. Camillo, \textunderscore Myst. de Lisb.\textunderscore , II, 27.
\section{Gestação}
\begin{itemize}
\item {Grp. gram.:f.}
\end{itemize}
\begin{itemize}
\item {Utilização:Fig.}
\end{itemize}
\begin{itemize}
\item {Proveniência:(Lat. \textunderscore gestatio\textunderscore )}
\end{itemize}
Tempo, que medeia entre a concepção e o nascimento dos mammíferos.
Gravidez.
Elaboração.
\section{Gestante}
\begin{itemize}
\item {Grp. gram.:adj.}
\end{itemize}
\begin{itemize}
\item {Proveniência:(Lat. \textunderscore gestans\textunderscore )}
\end{itemize}
Que contém o embryão.
\section{Gestão}
\begin{itemize}
\item {Grp. gram.:m.}
\end{itemize}
\begin{itemize}
\item {Proveniência:(Lat. \textunderscore gestio\textunderscore )}
\end{itemize}
Acto de gerir.
Gerência.
Administração.
\section{Gestatório}
\begin{itemize}
\item {Grp. gram.:adj.}
\end{itemize}
\begin{itemize}
\item {Proveniência:(Lat. \textunderscore gestatorius\textunderscore )}
\end{itemize}
Relativo á gestação ou gravidez.
Que se póde transportar.
\section{Geiser}
\begin{itemize}
\item {Grp. gram.:m.}
\end{itemize}
\begin{itemize}
\item {Proveniência:(T. island.)}
\end{itemize}
Jacto de água quente, que sái do interior da terra.
\section{Geiserite}
\begin{itemize}
\item {Grp. gram.:f.}
\end{itemize}
\begin{itemize}
\item {Proveniência:(De \textunderscore geiser\textunderscore )}
\end{itemize}
Concreção silicosa, que se fórma junto do orifício dos geiseres da Islândia.
\section{Gesticulação}
\begin{itemize}
\item {Grp. gram.:f.}
\end{itemize}
\begin{itemize}
\item {Proveniência:(Lat. \textunderscore gesticulatio\textunderscore )}
\end{itemize}
Acto ou effeito de gesticular.
\section{Gesticulado}
\begin{itemize}
\item {Grp. gram.:m.}
\end{itemize}
\begin{itemize}
\item {Grp. gram.:Adj.}
\end{itemize}
\begin{itemize}
\item {Proveniência:(De \textunderscore gesticular\textunderscore )}
\end{itemize}
Gesticulação.
Indicado por gestos: \textunderscore uma recusa gesticulada\textunderscore .
\section{Gesticulador}
\begin{itemize}
\item {Grp. gram.:m.  e  adj.}
\end{itemize}
\begin{itemize}
\item {Proveniência:(Lat. \textunderscore gesticulator\textunderscore )}
\end{itemize}
O que gesticula.
\section{Gesticular}
\begin{itemize}
\item {Grp. gram.:v. i.}
\end{itemize}
\begin{itemize}
\item {Grp. gram.:V. t.}
\end{itemize}
\begin{itemize}
\item {Utilização:P. us.}
\end{itemize}
\begin{itemize}
\item {Proveniência:(Lat. \textunderscore gesticulari\textunderscore )}
\end{itemize}
Fazer gestos.
Exprimir-se por mímica.
Exprimir por gestos: \textunderscore gesticular uma ordem\textunderscore .
\section{Gesto}
\begin{itemize}
\item {Grp. gram.:m.}
\end{itemize}
\begin{itemize}
\item {Proveniência:(Lat. \textunderscore gestus\textunderscore )}
\end{itemize}
Movimento do corpo, especialmente dos braços e cabeça, para exprimir ideias.
Sinal.
Mímica.
Apparência; physionomia; modo.
\section{Gestor}
\begin{itemize}
\item {Grp. gram.:m.}
\end{itemize}
\begin{itemize}
\item {Proveniência:(Lat. \textunderscore gestor\textunderscore )}
\end{itemize}
O mesmo que \textunderscore gerente\textunderscore .
\section{Gestrela}
\begin{itemize}
\item {Grp. gram.:f.}
\end{itemize}
\begin{itemize}
\item {Utilização:Prov.}
\end{itemize}
\begin{itemize}
\item {Utilização:alent.}
\end{itemize}
Espécie de planta, (\textunderscore ephedra fragilis\textunderscore , Desf.).
\section{Geta}
\begin{itemize}
\item {Grp. gram.:m.}
\end{itemize}
\begin{itemize}
\item {Proveniência:(De \textunderscore getas\textunderscore )}
\end{itemize}
Palerma; cretino. Cf. Camillo, \textunderscore Serões\textunderscore , VI, 39.
\section{Getape}
\begin{itemize}
\item {Grp. gram.:m.}
\end{itemize}
Planta annual, cujas sementes pulverizadas os indígenas da Guiné applicam contra as úlceras da córnea.
\section{Getas}
\begin{itemize}
\item {Grp. gram.:m. pl.}
\end{itemize}
\begin{itemize}
\item {Proveniência:(Lat. \textunderscore getae\textunderscore )}
\end{itemize}
Antigo povo da Thrácia sôbre o Danúbio, do qual se diz que era quási selvagem.
\section{Gético}
\begin{itemize}
\item {Grp. gram.:adj.}
\end{itemize}
\begin{itemize}
\item {Proveniência:(Lat. \textunderscore geticus\textunderscore )}
\end{itemize}
Relativo aos Getas ou ao país dos Getas:«\textunderscore ...géticas moradas...\textunderscore »Castilho, \textunderscore Fastos\textunderscore , I, 258.
\section{Gétulos}
\begin{itemize}
\item {Grp. gram.:m. pl.}
\end{itemize}
\begin{itemize}
\item {Proveniência:(Lat. \textunderscore getuli\textunderscore )}
\end{itemize}
Povos da Getúlia, na Lýbia interior.
\section{Geyser}
\begin{itemize}
\item {Grp. gram.:m.}
\end{itemize}
\begin{itemize}
\item {Proveniência:(T. island.)}
\end{itemize}
Jacto de água quente, que sái do interior da terra.
\section{Geyserite}
\begin{itemize}
\item {Grp. gram.:f.}
\end{itemize}
\begin{itemize}
\item {Proveniência:(De \textunderscore geyser\textunderscore )}
\end{itemize}
Concreção silicosa, que se fórma junto do orifício dos geyseres da Islândia.
\section{Gezerino}
\begin{itemize}
\item {Proveniência:(Do it. \textunderscore ghiazzerino\textunderscore )}
\end{itemize}
\textunderscore adj.\textunderscore  (e der.)
O mesmo que \textunderscore jazerino\textunderscore , etc.
\section{Gia}
\begin{itemize}
\item {Grp. gram.:f.}
\end{itemize}
\begin{itemize}
\item {Utilização:Bras}
\end{itemize}
Batrácio, semelhante á ran, mas muito maior, e de dorso escuro e comestível.
\section{Giacotim}
\begin{itemize}
\item {Grp. gram.:m.}
\end{itemize}
Espécie de faisão.
\section{Giba}
\begin{itemize}
\item {Grp. gram.:f.}
\end{itemize}
\begin{itemize}
\item {Utilização:Náut.}
\end{itemize}
\begin{itemize}
\item {Utilização:Prov.}
\end{itemize}
\begin{itemize}
\item {Utilização:trasm.}
\end{itemize}
\begin{itemize}
\item {Proveniência:(Lat. \textunderscore gibba\textunderscore )}
\end{itemize}
Corcunda.
Última vela da prôa, semelhante á bujarrona.
Cada uma das duas fases secundárias da lua, entre os quartos e o novilúnio.
\section{Giba}
\begin{itemize}
\item {Grp. gram.:f.}
\end{itemize}
Erva medicinal da ilha de San-Thomé.
\section{Gibaldeira}
\begin{itemize}
\item {Grp. gram.:f.}
\end{itemize}
(V.gilbardeira)
\section{Gibanete}
\begin{itemize}
\item {fónica:nê}
\end{itemize}
\begin{itemize}
\item {Grp. gram.:m.}
\end{itemize}
\begin{itemize}
\item {Utilização:Ant.}
\end{itemize}
\begin{itemize}
\item {Proveniência:(De \textunderscore gibão\textunderscore )}
\end{itemize}
Pequena coiraça de ferro ou de malha de aço.
\section{Gibangue}
\begin{itemize}
\item {Grp. gram.:f.}
\end{itemize}
Espécie de palmeira da Oceânia, (\textunderscore carypha gibanga\textunderscore ).
\section{Gibão}
\begin{itemize}
\item {Grp. gram.:m.}
\end{itemize}
\begin{itemize}
\item {Utilização:Bras}
\end{itemize}
Vestidura antiga, que cobria os homens desde o pescoço á cintura.
Collete.
Espécie de casaco curto que se veste sôbre a camisa.
Veste de coiro, usada pelos vaqueiros.
(Por \textunderscore aljubão\textunderscore , de \textunderscore aljuba\textunderscore ?)
\section{Gibão}
\begin{itemize}
\item {Grp. gram.:m.}
\end{itemize}
\begin{itemize}
\item {Proveniência:(Fr. \textunderscore gibbon\textunderscore )}
\end{itemize}
Macaco antropoide.
\section{Gibarra}
\begin{itemize}
\item {Grp. gram.:adj.}
\end{itemize}
\begin{itemize}
\item {Utilização:Bras. do N}
\end{itemize}
Muito grande.
Que é de estatura muito elevada.
\section{Gibba}
\begin{itemize}
\item {Grp. gram.:f.}
\end{itemize}
\begin{itemize}
\item {Utilização:Náut.}
\end{itemize}
\begin{itemize}
\item {Utilização:Prov.}
\end{itemize}
\begin{itemize}
\item {Utilização:trasm.}
\end{itemize}
\begin{itemize}
\item {Proveniência:(Lat. \textunderscore gibba\textunderscore )}
\end{itemize}
Corcunda.
Última vela da prôa, semelhante á bujarrona.
Cada uma das duas phases secundárias da lua, entre os quartos e o novilúnio.
\section{Gibbão}
\begin{itemize}
\item {Grp. gram.:m.}
\end{itemize}
\begin{itemize}
\item {Proveniência:(Fr. \textunderscore gibbon\textunderscore )}
\end{itemize}
Macaco anthropoide.
\section{Gibbo}
\begin{itemize}
\item {Grp. gram.:m.}
\end{itemize}
O mesmo que \textunderscore gibba\textunderscore .
\section{Gibbosidade}
\begin{itemize}
\item {Grp. gram.:f.}
\end{itemize}
\begin{itemize}
\item {Proveniência:(De \textunderscore gibboso\textunderscore )}
\end{itemize}
Curvatura da columna vertebral, com elevação exterior.
Gibba.
Curvatura convexa.
Proeminência do tecido gorduroso no dorso de alguns animaes.
\section{Gibboso}
\begin{itemize}
\item {Grp. gram.:m.  e  adj.}
\end{itemize}
\begin{itemize}
\item {Proveniência:(Lat. \textunderscore gibbosus\textunderscore )}
\end{itemize}
O que tem gibba.
Corcovado.
Convexo.
\section{Gibiteiro}
\begin{itemize}
\item {Grp. gram.:m.}
\end{itemize}
\begin{itemize}
\item {Utilização:Ant.}
\end{itemize}
Fabricante de gibanetes.
(Por \textunderscore gibateiro\textunderscore , contr. de \textunderscore gibaneteiro\textunderscore , de \textunderscore gibanete\textunderscore )
\section{Gibo}
\begin{itemize}
\item {Grp. gram.:m.}
\end{itemize}
O mesmo que \textunderscore giba\textunderscore ^1.
\section{Gibóia}
\begin{itemize}
\item {Grp. gram.:f.}
\end{itemize}
Grande serpente, a maior do Brasil.
(Do tupi \textunderscore gi\textunderscore  + \textunderscore boia\textunderscore )
\section{Giboiaçu}
\begin{itemize}
\item {Grp. gram.:f.}
\end{itemize}
\begin{itemize}
\item {Utilização:Bras}
\end{itemize}
O mesmo que \textunderscore gibóia\textunderscore .
\section{Gibosidade}
\begin{itemize}
\item {Grp. gram.:f.}
\end{itemize}
\begin{itemize}
\item {Proveniência:(De \textunderscore giboso\textunderscore )}
\end{itemize}
Curvatura da coluna vertebral, com elevação exterior.
Giba.
Curvatura convexa.
Proeminência do tecido gorduroso no dorso de alguns animaes.
\section{Giboso}
\begin{itemize}
\item {Grp. gram.:m.  e  adj.}
\end{itemize}
\begin{itemize}
\item {Proveniência:(Lat. \textunderscore gibbosus\textunderscore )}
\end{itemize}
O que tem giba.
Corcovado.
Convexo.
\section{Gicão}
\begin{itemize}
\item {Grp. gram.:m.}
\end{itemize}
Planta crucífera do Brasil, (\textunderscore serpaea cearensis\textunderscore ).
\section{Gido}
\begin{itemize}
\item {Grp. gram.:adj.}
\end{itemize}
\begin{itemize}
\item {Utilização:Prov.}
\end{itemize}
\begin{itemize}
\item {Utilização:beir.}
\end{itemize}
O mesmo que \textunderscore jeitoso\textunderscore .
\section{Giesta}
\begin{itemize}
\item {Grp. gram.:f.}
\end{itemize}
\begin{itemize}
\item {Proveniência:(Lat. \textunderscore genista\textunderscore , que também se escreveu \textunderscore genesta\textunderscore )}
\end{itemize}
Gênero de plantas leguminosas, a que pertencem vários arbustos de flôres amarelas.
\section{Giestal}
\begin{itemize}
\item {Grp. gram.:m.}
\end{itemize}
Lugar, onde crescem giestas.
\section{Giesteira}
\begin{itemize}
\item {Grp. gram.:f.}
\end{itemize}
O mesmo que \textunderscore giesta\textunderscore .
Árvore açoreana, cuja madeira é empregada por marceneiros em carroças, embarcações, etc.
\section{Giesteiro}
\begin{itemize}
\item {Grp. gram.:m.}
\end{itemize}
O mesmo que \textunderscore giesta\textunderscore .
\section{Giestoso}
\begin{itemize}
\item {Grp. gram.:adj.}
\end{itemize}
Em que há giestas.
\section{Giga}
\begin{itemize}
\item {Grp. gram.:f.}
\end{itemize}
Selha larga e pouco alta.
Canastra, em fórma de selha.
\section{Giganhos}
\begin{itemize}
\item {Grp. gram.:m. pl.}
\end{itemize}
Antigo povo selvagem da Ásia. Cf. \textunderscore Peregrinação\textunderscore , LXXIII.
\section{Giganta}
\begin{itemize}
\item {Grp. gram.:f.}
\end{itemize}
\begin{itemize}
\item {Proveniência:(De \textunderscore gigante\textunderscore )}
\end{itemize}
Mulher, de estatura descommunal.
\section{Gigante}
\begin{itemize}
\item {Grp. gram.:m.}
\end{itemize}
\begin{itemize}
\item {Utilização:Bot.}
\end{itemize}
\begin{itemize}
\item {Grp. gram.:Adj.}
\end{itemize}
\begin{itemize}
\item {Utilização:Fig.}
\end{itemize}
\begin{itemize}
\item {Proveniência:(Do lat. \textunderscore gigas\textunderscore , \textunderscore gigantis\textunderscore )}
\end{itemize}
Homem, de estatura descommunal.
Animal do grande corpulência.
Arcobotante.
Malvaisco; altheia.
Muito alto.
Descommunal.
Admirável; sublime.
\textunderscore Erva gigante\textunderscore , o mesmo que \textunderscore acantho\textunderscore .
\section{Gigantear}
\begin{itemize}
\item {Grp. gram.:v. i.}
\end{itemize}
Tornar-se gigante.
Crescer, engrandecer-se.
\section{Giganteia}
\begin{itemize}
\item {Grp. gram.:f.}
\end{itemize}
O mesmo que \textunderscore tupinambo\textunderscore .
\section{Giganteu}
\begin{itemize}
\item {Grp. gram.:adj.}
\end{itemize}
\begin{itemize}
\item {Grp. gram.:M.}
\end{itemize}
\begin{itemize}
\item {Utilização:Constr.}
\end{itemize}
\begin{itemize}
\item {Proveniência:(Lat. \textunderscore giganteus\textunderscore )}
\end{itemize}
Que tem estatura de gigante.
Que tem altura desmedida.
Grandioso; prodigioso. Cf. Castilho, \textunderscore Fastos\textunderscore , III, 65.
Supporte de alvenaria, para sustentar muralhas.
\section{Gigantez}
\begin{itemize}
\item {Grp. gram.:f.}
\end{itemize}
Qualidade de gigante. Cf. Filinto, I, 268.
\section{Gigantescamente}
\begin{itemize}
\item {Grp. gram.:adv.}
\end{itemize}
De modo gigantesco.
\section{Gigantesco}
\begin{itemize}
\item {fónica:tês}
\end{itemize}
\begin{itemize}
\item {Grp. gram.:adj.}
\end{itemize}
(V.giganteu)
\section{Gigântico}
\begin{itemize}
\item {Grp. gram.:adj.}
\end{itemize}
O mesmo que \textunderscore giganteu\textunderscore :«\textunderscore ...gigânticos rochedos...\textunderscore »Garrett, \textunderscore Flôres sem Fruto\textunderscore , 78.
\section{Gigantífero}
\begin{itemize}
\item {Grp. gram.:adj.}
\end{itemize}
Que produz gigantes. Cf. \textunderscore Viriato Trág.\textunderscore , VIII, 122.
\section{Gigantil}
\begin{itemize}
\item {Grp. gram.:adj.}
\end{itemize}
\begin{itemize}
\item {Proveniência:(De \textunderscore gigante\textunderscore )}
\end{itemize}
Diz-se de uma variedade de milho amarelo.
\section{Gigantismo}
\begin{itemize}
\item {Grp. gram.:m.}
\end{itemize}
\begin{itemize}
\item {Utilização:Bot.}
\end{itemize}
\begin{itemize}
\item {Proveniência:(De \textunderscore gigante\textunderscore )}
\end{itemize}
Desenvolvimento extraordinário e anormal de uma planta.
\section{Gigantófono}
\begin{itemize}
\item {Grp. gram.:adj.}
\end{itemize}
\begin{itemize}
\item {Proveniência:(Do gr. \textunderscore gigas\textunderscore  + \textunderscore phone\textunderscore )}
\end{itemize}
Que sôa fortemente, que troveja. Cf. Filinto, I, 233.
\section{Gigantografia}
\begin{itemize}
\item {Grp. gram.:f.}
\end{itemize}
História de gigantes.
\section{Gigantographia}
\begin{itemize}
\item {Grp. gram.:f.}
\end{itemize}
História de gigantes.
\section{Gigantólitho}
\begin{itemize}
\item {Grp. gram.:f.}
\end{itemize}
\begin{itemize}
\item {Utilização:Miner.}
\end{itemize}
Silicato hydratado de alumina e ferro.
\section{Gigantólito}
\begin{itemize}
\item {Grp. gram.:f.}
\end{itemize}
\begin{itemize}
\item {Utilização:Miner.}
\end{itemize}
Silicato hidratado de alumina e ferro.
\section{Gigantologia}
\begin{itemize}
\item {Grp. gram.:f.}
\end{itemize}
Tratado ou discurso, á cêrca de gigantes.
\section{Gigantomachia}
\begin{itemize}
\item {fónica:qui}
\end{itemize}
\begin{itemize}
\item {Grp. gram.:f.}
\end{itemize}
\begin{itemize}
\item {Proveniência:(Gr. \textunderscore gigantomakhia\textunderscore )}
\end{itemize}
Combate mythológico dos gigantes contra os deuses.
\section{Gigantomaquia}
\begin{itemize}
\item {Grp. gram.:f.}
\end{itemize}
\begin{itemize}
\item {Proveniência:(Gr. \textunderscore gigantomakhia\textunderscore )}
\end{itemize}
Combate mitológico dos gigantes contra os deuses.
\section{Gigantóphono}
\begin{itemize}
\item {Grp. gram.:adj.}
\end{itemize}
\begin{itemize}
\item {Proveniência:(Do gr. \textunderscore gigas\textunderscore  + \textunderscore phone\textunderscore )}
\end{itemize}
Que sôa fortemente, que troveja. Cf. Filinto, I, 233.
\section{Gígia}
\begin{itemize}
\item {Grp. gram.:f.}
\end{itemize}
Robustíssima árvore intertropical, (\textunderscore parinarium capense\textunderscore , Harw.?).
\section{Gigo}
\begin{itemize}
\item {Grp. gram.:m.}
\end{itemize}
\begin{itemize}
\item {Proveniência:(De \textunderscore giga\textunderscore ^1)}
\end{itemize}
O mesmo que \textunderscore cabaz\textunderscore .
Ramo de árvore com frutos.
\section{Gigote}
\begin{itemize}
\item {Grp. gram.:m.}
\end{itemize}
\begin{itemize}
\item {Proveniência:(Fr. \textunderscore gigot\textunderscore )}
\end{itemize}
Guisado, em que entra carne desfiada, manteiga e caldo.
\section{Giguefo}
\begin{itemize}
\item {fónica:guê}
\end{itemize}
\begin{itemize}
\item {Grp. gram.:m.}
\end{itemize}
(V.inguefo)
\section{Gila}
\begin{itemize}
\item {Grp. gram.:f.}
\end{itemize}
Designação resumida da \textunderscore gilacaiota\textunderscore .
\section{Gilacaiota}
\begin{itemize}
\item {Grp. gram.:f.}
\end{itemize}
Pequena abóbora, de que se faz doce e que também é conhecida por \textunderscore melão do Malabar\textunderscore .
\section{Gilbarbeira}
\begin{itemize}
\item {Grp. gram.:f.}
\end{itemize}
\begin{itemize}
\item {Utilização:Prov.}
\end{itemize}
\begin{itemize}
\item {Utilização:minh.}
\end{itemize}
O mesmo que \textunderscore gilbardeira\textunderscore ?
Planta áspera, de fôlhas picantes, que nasce nos vallados e nas silveiras.
\section{Gilbardeira}
\begin{itemize}
\item {Grp. gram.:f.}
\end{itemize}
Espécie de murta brava, de pequenos frutos redondos como a cereja e de fôlhas com sabor picante, (\textunderscore ruscus aculeatus\textunderscore , Lin.).
\section{Gile}
\begin{itemize}
\item {Grp. gram.:m.}
\end{itemize}
O mesmo que \textunderscore bútua\textunderscore .
\section{Gília}
\begin{itemize}
\item {Grp. gram.:f.}
\end{itemize}
Planta annual.
\section{Gillonário}
\begin{itemize}
\item {Grp. gram.:m.}
\end{itemize}
\begin{itemize}
\item {Utilização:Ant.}
\end{itemize}
Donzel; infanção.
(B. lat. \textunderscore gillonarius\textunderscore )
\section{Gilmendes}
\begin{itemize}
\item {Grp. gram.:m.}
\end{itemize}
Variedade de pêssegos, de pelle branca e polpa açucarada.
\section{Giló}
\begin{itemize}
\item {Grp. gram.:f.}
\end{itemize}
Planta solânea, (\textunderscore solanum melongena\textunderscore  ou \textunderscore ovigerum\textunderscore ).
\section{Gilonário}
\begin{itemize}
\item {Grp. gram.:m.}
\end{itemize}
\begin{itemize}
\item {Utilização:Ant.}
\end{itemize}
Donzel; infanção.
(B. lat. \textunderscore gillonarius\textunderscore )
\section{Gilvaz}
\begin{itemize}
\item {Grp. gram.:m.}
\end{itemize}
\begin{itemize}
\item {Proveniência:(De \textunderscore Gil Vaz\textunderscore , n. p.?)}
\end{itemize}
Golpe ou cicatriz no rosto.
\section{Gilvicentesco}
\begin{itemize}
\item {fónica:tês}
\end{itemize}
\begin{itemize}
\item {Grp. gram.:adj.}
\end{itemize}
Relativo a Gil Vicente.
\section{Gim}
\begin{itemize}
\item {Grp. gram.:m.}
\end{itemize}
\begin{itemize}
\item {Proveniência:(Do ingl. \textunderscore gin\textunderscore )}
\end{itemize}
Instrumento, para encurvar as calhas das linhas férreas.
\section{Gimbe}
\begin{itemize}
\item {Grp. gram.:m.}
\end{itemize}
Ave de rapina, da África occidental, (\textunderscore bubo muculosus\textunderscore ).
\section{Gimbipotente}
\begin{itemize}
\item {Grp. gram.:adj.}
\end{itemize}
\begin{itemize}
\item {Utilização:Chul.}
\end{itemize}
\begin{itemize}
\item {Proveniência:(De \textunderscore gimbo\textunderscore ^2 + \textunderscore potente\textunderscore )}
\end{itemize}
Que tem muito dinheiro; que é ricaço. Cf. Macedo, \textunderscore Burros\textunderscore , 136.
\section{Gimbo}
\begin{itemize}
\item {Grp. gram.:m.}
\end{itemize}
Pássaro africano, (\textunderscore merops apiaster\textunderscore ).
\section{Gimbo}
\begin{itemize}
\item {Grp. gram.:m.}
\end{itemize}
\begin{itemize}
\item {Utilização:Ant.}
\end{itemize}
\begin{itemize}
\item {Utilização:Gír.}
\end{itemize}
Dinheiro.
\section{Gimbolinha}
\begin{itemize}
\item {Grp. gram.:f.}
\end{itemize}
\begin{itemize}
\item {Utilização:Gír.}
\end{itemize}
Vinho.
\section{Gimbololo}
\begin{itemize}
\item {Grp. gram.:m.}
\end{itemize}
Espécie de crocodilo, (\textunderscore crocodilus frontatus\textunderscore ).
\section{Gimbrar}
\begin{itemize}
\item {Grp. gram.:v. i.}
\end{itemize}
\begin{itemize}
\item {Utilização:T. de Pare -de-Coira}
\end{itemize}
\begin{itemize}
\item {Utilização:des.}
\end{itemize}
Figurar; tomar ares de importância; impor-se á consideração dos outros.
\section{Gimbulo}
\begin{itemize}
\item {Grp. gram.:m.}
\end{itemize}
Nome, com que os indígenas africanos designam o cão selvagem.
\section{Gimo}
\begin{itemize}
\item {Grp. gram.:m.}
\end{itemize}
\begin{itemize}
\item {Utilização:T. de Turquel}
\end{itemize}
O mesmo que \textunderscore gemido\textunderscore .
\section{Ginari}
\begin{itemize}
\item {Grp. gram.:m.}
\end{itemize}
O mesmo que \textunderscore nili\textunderscore .
\section{Gineta}
\begin{itemize}
\item {fónica:nê}
\end{itemize}
\begin{itemize}
\item {Grp. gram.:f.}
\end{itemize}
\begin{itemize}
\item {Proveniência:(Do ár. \textunderscore jernít\textunderscore )}
\end{itemize}
Mammífero carnívoro.
Gato bravo, cuja pelle é applicada a vestuário, especialmente em França.
\section{Gineta}
\begin{itemize}
\item {fónica:nê}
\end{itemize}
\begin{itemize}
\item {Grp. gram.:f.}
\end{itemize}
\begin{itemize}
\item {Proveniência:(De uma fórma adj. de \textunderscore ginete\textunderscore ^1)}
\end{itemize}
Systema de equitação, com estribo curto.
Pôsto de capitão.
Antiga insígnia de capitães.
\section{Ginetaço}
\begin{itemize}
\item {Grp. gram.:m.}
\end{itemize}
\begin{itemize}
\item {Utilização:Bras}
\end{itemize}
Ginete, que tem bom garbo e andadura.
\section{Ginetário}
\begin{itemize}
\item {Grp. gram.:m.}
\end{itemize}
\begin{itemize}
\item {Utilização:Ant.}
\end{itemize}
Aquelle que montava á gineta.
\section{Ginete}
\begin{itemize}
\item {fónica:nê}
\end{itemize}
\begin{itemize}
\item {Grp. gram.:m.}
\end{itemize}
\begin{itemize}
\item {Utilização:Ant.}
\end{itemize}
\begin{itemize}
\item {Utilização:Bras}
\end{itemize}
\begin{itemize}
\item {Grp. gram.:Adj.}
\end{itemize}
\begin{itemize}
\item {Utilização:Bras. do N}
\end{itemize}
\begin{itemize}
\item {Proveniência:(Do gr. \textunderscore gunetes\textunderscore )}
\end{itemize}
Cavallo de bôa raça, pequeno mas bem proporcionado.
Cavalleiro, armado de lança e adarga.
Cavalleiro.
Peixe de Portugal.
Sella grosseira, usada pelos vaqueiros do Ceará e do Brasil do Sul.
Ágil.
Inquieto.
\section{Ginete}
\begin{itemize}
\item {fónica:nê}
\end{itemize}
\begin{itemize}
\item {Grp. gram.:m.}
\end{itemize}
\begin{itemize}
\item {Utilização:Ant.}
\end{itemize}
(?):«\textunderscore ...foy cortando de ginete.\textunderscore »\textunderscore Hist. Trág. Marít.\textunderscore , 53.
(Por \textunderscore guinete\textunderscore , de \textunderscore guinar\textunderscore ?)
\section{Gineto}
\begin{itemize}
\item {fónica:nê}
\end{itemize}
\begin{itemize}
\item {Grp. gram.:m.}
\end{itemize}
Espécie de animal carnívoro, semelhante á raposa.
(Do ár.)
\section{Ginga}
\begin{itemize}
\item {Grp. gram.:f.}
\end{itemize}
\begin{itemize}
\item {Proveniência:(De \textunderscore gingar\textunderscore )}
\end{itemize}
Espécie de remo, que, apoiado num encaixe sôbre a popa, faz andar a embarcação.
\section{Gingação}
\begin{itemize}
\item {Grp. gram.:f.}
\end{itemize}
Acto de gingar.
\section{Gingado}
\begin{itemize}
\item {Grp. gram.:adj.}
\end{itemize}
\begin{itemize}
\item {Proveniência:(De \textunderscore gingar\textunderscore )}
\end{itemize}
Em que há modos de gingão ou de fadista:«\textunderscore ...pimpões que resvalam com um piparote gingado o feltro para a nuca.\textunderscore »Camillo, \textunderscore Sebenta\textunderscore .
\section{Gingador}
\begin{itemize}
\item {Grp. gram.:m.}
\end{itemize}
\begin{itemize}
\item {Proveniência:(De \textunderscore gingar\textunderscore )}
\end{itemize}
Barqueiro, que trabalha com a ginga.
\section{Ginga-lumbango}
\begin{itemize}
\item {Grp. gram.:m.}
\end{itemize}
Trepadeira africana, leguminosa, cujas raízes são consideradas como aphrodisíacas pelos indígenas.
\section{Gingante}
\begin{itemize}
\item {Grp. gram.:adj.}
\end{itemize}
Que ginga.
\section{Gingão}
\begin{itemize}
\item {Grp. gram.:adj.}
\end{itemize}
\begin{itemize}
\item {Grp. gram.:M.}
\end{itemize}
\begin{itemize}
\item {Utilização:Gír.}
\end{itemize}
\begin{itemize}
\item {Proveniência:(De \textunderscore gingar\textunderscore )}
\end{itemize}
Que ginga.
Próprio de quem ginga.
Brigão; fadista.
Homem coxo.
\section{Gingar}
\begin{itemize}
\item {Grp. gram.:v. i.}
\end{itemize}
\begin{itemize}
\item {Utilização:T. do Fundão}
\end{itemize}
\begin{itemize}
\item {Utilização:T. de Turquel}
\end{itemize}
\begin{itemize}
\item {Proveniência:(Do cast. \textunderscore ginglar\textunderscore )}
\end{itemize}
Inclinar-se, ora para um, ora para outro lado, andando.
Bambolear-se.
Navegar com ginga.
Caçoar, chalacear.
Recusar-se, um pouco desdenhosamente, á satisfação de um pedido.
\section{Gingas}
\begin{itemize}
\item {Grp. gram.:m. pl.}
\end{itemize}
Um dos povos do Congo.
\section{Gingelim}
\begin{itemize}
\item {Grp. gram.:m.}
\end{itemize}
O mesmo que \textunderscore gergelim\textunderscore .
\section{Gingelina}
\begin{itemize}
\item {Grp. gram.:f.}
\end{itemize}
Tecido do lan com fio de seda, vulgarmente conhecido por \textunderscore lan de camelo\textunderscore .
\section{Gingerlina}
\begin{itemize}
\item {Grp. gram.:f.}
\end{itemize}
Tecido do lan com fio de seda, vulgarmente conhecido por \textunderscore lan de camelo\textunderscore .
\section{Gingídio}
\begin{itemize}
\item {Grp. gram.:m.}
\end{itemize}
\begin{itemize}
\item {Proveniência:(Do gr. \textunderscore gingidion\textunderscore )}
\end{itemize}
Planta umbellífera, amargosa.
\section{Gingiva}
\begin{itemize}
\item {Grp. gram.:f.}
\end{itemize}
\begin{itemize}
\item {Proveniência:(Do lat. \textunderscore gingiva\textunderscore )}
\end{itemize}
O mesmo ou melhor que \textunderscore gengiva\textunderscore .
Tecido fibro-muscular, em que estão os alvéolos dentários.
\section{Ginglarão}
\begin{itemize}
\item {Grp. gram.:m.}
\end{itemize}
\begin{itemize}
\item {Utilização:Prov.}
\end{itemize}
\begin{itemize}
\item {Utilização:trasm.}
\end{itemize}
\begin{itemize}
\item {Proveniência:(Do cast. \textunderscore ginglar\textunderscore )}
\end{itemize}
Acto de retoiçar-se numa espécie de trapézio, feito só de corda.
A corda, empregada nesse exercício.
\section{Gínglimo}
\begin{itemize}
\item {Grp. gram.:m.}
\end{itemize}
\begin{itemize}
\item {Proveniência:(Gr. \textunderscore ginglumos\textunderscore , gonzo)}
\end{itemize}
Articulação, que só dá movimento em dois sentidos opostos.
Charneira.
Articulação, em fórma de charneira, como a do cotovelo.
\section{Gínglymo}
\begin{itemize}
\item {Grp. gram.:m.}
\end{itemize}
\begin{itemize}
\item {Proveniência:(Gr. \textunderscore ginglumos\textunderscore , gonzo)}
\end{itemize}
Articulação, que só dá movimento em dois sentidos oppostos.
Charneira.
Articulação, em fórma de charneira, como a do cotovelo.
\section{Gingo}
\begin{itemize}
\item {Grp. gram.:m.}
\end{itemize}
\begin{itemize}
\item {Utilização:Prov.}
\end{itemize}
\begin{itemize}
\item {Utilização:alent.}
\end{itemize}
O mesmo que \textunderscore gingação\textunderscore .
Dança de roda.
\section{Gingôa}
\begin{itemize}
\item {Grp. gram.:f.}
\end{itemize}
Nome, que na África austro-central se dá a qualquer de duas árvores monocotyledóneas, que pouco divergem, e são ambas empregadas no fabrico de esteiras.
\section{Gingrar}
\begin{itemize}
\item {Grp. gram.:v. t.}
\end{itemize}
\begin{itemize}
\item {Utilização:Ant.}
\end{itemize}
\begin{itemize}
\item {Proveniência:(Do cast. \textunderscore ginglar\textunderscore ?)}
\end{itemize}
Mofar; folgar:«\textunderscore gingrai lá com taes cachopas\textunderscore ». G. Vicente, I, 139.
\section{Gingrina}
\begin{itemize}
\item {Grp. gram.:f.}
\end{itemize}
\begin{itemize}
\item {Utilização:Ant.}
\end{itemize}
Espécie de gaita, que imitava a voz do pato.
(Relaciona-se com \textunderscore gingrar\textunderscore ?)
\section{Ginguba}
\begin{itemize}
\item {Grp. gram.:f.}
\end{itemize}
O mesmo que \textunderscore amendoim\textunderscore .
\section{Gingue-ganene}
\begin{itemize}
\item {Grp. gram.:m.}
\end{itemize}
Arbusto africano, de folhas inteiras, glabras, e flôres hermaphroditas de corolla amarela.
\section{Gingueiro}
\begin{itemize}
\item {Grp. gram.:adj.}
\end{itemize}
\begin{itemize}
\item {Utilização:Prov.}
\end{itemize}
\begin{itemize}
\item {Utilização:trasm.}
\end{itemize}
\begin{itemize}
\item {Proveniência:(De \textunderscore gingar\textunderscore )}
\end{itemize}
Elegante, peralta.
\section{Ginhal}
\begin{itemize}
\item {Grp. gram.:f.}
\end{itemize}
\begin{itemize}
\item {Utilização:T. da Bairrada}
\end{itemize}
\begin{itemize}
\item {Proveniência:(De \textunderscore Junho\textunderscore ? Neste caso, deverá escrever-se \textunderscore jinhal\textunderscore )}
\end{itemize}
Variedade de ameixa redonda, acastanhada e temporan.
\section{Ginja}
\begin{itemize}
\item {Grp. gram.:f.}
\end{itemize}
\begin{itemize}
\item {Grp. gram.:M.}
\end{itemize}
\begin{itemize}
\item {Utilização:Fam.}
\end{itemize}
\begin{itemize}
\item {Proveniência:(Do fr. \textunderscore guigne\textunderscore , b. lat. \textunderscore guina\textunderscore , nome de cereja, que também se encontra nas línguas germânicas e eslavas)}
\end{itemize}
Fruto da ginjeira.
Velhote; pessôa magra e avelhentada.
\section{Ginjal}
\begin{itemize}
\item {Grp. gram.:adj.}
\end{itemize}
\begin{itemize}
\item {Proveniência:(De \textunderscore ginja\textunderscore )}
\end{itemize}
Lugar, onde crescem ginjeiras.
\section{Ginjeira}
\begin{itemize}
\item {Grp. gram.:f.}
\end{itemize}
\begin{itemize}
\item {Proveniência:(De \textunderscore ginja\textunderscore )}
\end{itemize}
Variedade de cerejeira, cujo fruto é agri-doce e tem pé mais curto que as cerejas em geral.
Nome de algumas plantas americanas, estranhas ás rosáceas.
\section{Ginjinha}
\begin{itemize}
\item {Grp. gram.:f.}
\end{itemize}
\begin{itemize}
\item {Utilização:Pop.}
\end{itemize}
Bebida, feita de aguardente, com ginja de infusão.
\section{Gino}
\begin{itemize}
\item {Grp. gram.:m.}
\end{itemize}
\begin{itemize}
\item {Utilização:Prov.}
\end{itemize}
\begin{itemize}
\item {Utilização:trasm.}
\end{itemize}
O mesmo que \textunderscore rebento\textunderscore .
\section{Ginsão}
\begin{itemize}
\item {Grp. gram.:m.}
\end{itemize}
Planta araliácea do Brasil, (\textunderscore panax quinquefolium\textunderscore , ou \textunderscore aralia canadensis\textunderscore ).
\section{Ginzeu}
\begin{itemize}
\item {Grp. gram.:m.}
\end{itemize}
Formiga venenosa e preta de Angola.
\section{Gio}
\begin{itemize}
\item {Grp. gram.:m.}
\end{itemize}
\begin{itemize}
\item {Utilização:Náut.}
\end{itemize}
Nome de duas peças curvas de madeira, que formam ângulo, entalhando-se entre si e no contra-cadaste.
\section{Giota}
\begin{itemize}
\item {Grp. gram.:f.}
\end{itemize}
\begin{itemize}
\item {Utilização:Prov.}
\end{itemize}
\begin{itemize}
\item {Utilização:dur.}
\end{itemize}
Excremento humano.
\section{Gique}
\begin{itemize}
\item {Grp. gram.:m.}
\end{itemize}
\begin{itemize}
\item {Utilização:Bras}
\end{itemize}
O mesmo que \textunderscore imbuzeiro\textunderscore .
\section{Giqui}
\begin{itemize}
\item {Grp. gram.:m.}
\end{itemize}
\begin{itemize}
\item {Utilização:Bras}
\end{itemize}
Armadilha de pesca.
\section{Giquirili}
\begin{itemize}
\item {Grp. gram.:m.}
\end{itemize}
\begin{itemize}
\item {Utilização:Bras}
\end{itemize}
Planta leguminosa, (\textunderscore abrus precatorius\textunderscore ).
\section{Giquitaia}
\begin{itemize}
\item {Grp. gram.:f.}
\end{itemize}
\begin{itemize}
\item {Utilização:Bras}
\end{itemize}
Formiga pequena e vermelha.
O mesmo que \textunderscore giquitara\textunderscore .
\section{Giquitara}
\begin{itemize}
\item {Grp. gram.:f.}
\end{itemize}
Espécie de pimenta.
Môsca vermelha e pequena do Pará.
\section{Gira}
\begin{itemize}
\item {Grp. gram.:f.}
\end{itemize}
O mesmo que \textunderscore gíria\textunderscore .
\section{Gira}
\begin{itemize}
\item {Grp. gram.:f.}
\end{itemize}
\begin{itemize}
\item {Utilização:Pop.}
\end{itemize}
\begin{itemize}
\item {Utilização:Prov.}
\end{itemize}
\begin{itemize}
\item {Utilização:minh.}
\end{itemize}
\begin{itemize}
\item {Grp. gram.:M.}
\end{itemize}
\begin{itemize}
\item {Utilização:Prov.}
\end{itemize}
\begin{itemize}
\item {Utilização:dur.}
\end{itemize}
\begin{itemize}
\item {Grp. gram.:Adj.}
\end{itemize}
\begin{itemize}
\item {Utilização:Bras. do N}
\end{itemize}
Acto de girar, de passear; passeio.
Ronda.
Indivíduo inconstante, amalucado.
Um tanto doido; meio maluco.
\section{Giraba}
\begin{itemize}
\item {Grp. gram.:f.}
\end{itemize}
\begin{itemize}
\item {Utilização:Bras}
\end{itemize}
O mesmo que \textunderscore côco\textunderscore ^1.
\section{Giraçal}
\begin{itemize}
\item {Grp. gram.:adj.}
\end{itemize}
\begin{itemize}
\item {Utilização:Ant.}
\end{itemize}
Dizia-se do arroz de primeira qualidade, vindo da Ásia.
\section{Giração}
\begin{itemize}
\item {Grp. gram.:f.}
\end{itemize}
Acto ou effeito de girar.
\section{Girador}
\begin{itemize}
\item {Grp. gram.:adj.}
\end{itemize}
\begin{itemize}
\item {Grp. gram.:M.}
\end{itemize}
Que gira; que faz girar.
Aquelle ou aquillo que gira ou faz girar.
\section{Girafa}
\begin{itemize}
\item {Grp. gram.:f.}
\end{itemize}
\begin{itemize}
\item {Utilização:Pop.}
\end{itemize}
\begin{itemize}
\item {Proveniência:(Do ár. \textunderscore zarafa\textunderscore )}
\end{itemize}
Grande mammífero, da ordem dos ruminantes, notável principalmente pelo comprimento de pescoço.
Constellação no hemisphério boreal.
Mulher alta e de pescoço comprido.
\section{Girafalte}
\begin{itemize}
\item {Grp. gram.:m.}
\end{itemize}
O mesmo que \textunderscore gerifalte\textunderscore .
Cf. J. Sousa, \textunderscore Vestígios da L. Ar.\textunderscore 
\section{Girafalto}
\begin{itemize}
\item {Grp. gram.:m.}
\end{itemize}
(V.gerifalte)
\section{Giraldinha}
\begin{itemize}
\item {Grp. gram.:f.}
\end{itemize}
\begin{itemize}
\item {Utilização:Gír.}
\end{itemize}
Patuscada.
\section{Girândola}
\begin{itemize}
\item {Grp. gram.:f.}
\end{itemize}
\begin{itemize}
\item {Proveniência:(It. \textunderscore girandola\textunderscore )}
\end{itemize}
Travessão ou roda com orifícios, em cada um dos quaes se pendura um foguete, que sobe e estoira ao mesmo tempo que os demais.
Conjunto dos foguetes, agrupados nesse travessão ou roda.
\section{Girante}
\begin{itemize}
\item {Grp. gram.:adj.}
\end{itemize}
Que gira.
\section{Girão}
\begin{itemize}
\item {Grp. gram.:m.}
\end{itemize}
\begin{itemize}
\item {Utilização:Ant.}
\end{itemize}
\begin{itemize}
\item {Utilização:Bras}
\end{itemize}
\begin{itemize}
\item {Utilização:Fig.}
\end{itemize}
\begin{itemize}
\item {Proveniência:(Do germ. \textunderscore ger\textunderscore )}
\end{itemize}
Cercadura ou debrum de vestuário.
Pedaço de pano.
Coirela.
Triângulo equilátero nos escudos heráldicos.
Apparelho de madeira, para secar carne.--Como brasileirismo, parece-me que os diccionaristas confundem \textunderscore girão\textunderscore  com \textunderscore girau\textunderscore .
Regaço, seio.
\section{Girar}
\begin{itemize}
\item {Grp. gram.:v. i.}
\end{itemize}
\begin{itemize}
\item {Grp. gram.:V. t.}
\end{itemize}
\begin{itemize}
\item {Proveniência:(De \textunderscore giro\textunderscore )}
\end{itemize}
Andar em giro.
Descrever giro ou curva.
Vaguear.
Agitar-se.
Correr.
Lidar; negociar.
Circundar.
Percorrer em volta.
Percorrer.
\section{Girasol}
\begin{itemize}
\item {fónica:sol}
\end{itemize}
\begin{itemize}
\item {Grp. gram.:m.}
\end{itemize}
\begin{itemize}
\item {Proveniência:(De \textunderscore girar\textunderscore  + \textunderscore sol\textunderscore )}
\end{itemize}
Planta, da fam. das compostas, cuja flôr se volta para o sol.
Pedra preciosa, que reflecte brilhantemente os raios do sol.
Variedade de arroz, muito estimada na Índia.
\section{Girassol}
\begin{itemize}
\item {Grp. gram.:m.}
\end{itemize}
\begin{itemize}
\item {Proveniência:(De \textunderscore girar\textunderscore  + \textunderscore sol\textunderscore )}
\end{itemize}
Planta, da fam. das compostas, cuja flôr se volta para o sol.
Pedra preciosa, que reflecte brilhantemente os raios do sol.
Variedade de arroz, muito estimada na Índia.
\section{Girata}
\begin{itemize}
\item {Grp. gram.:f.}
\end{itemize}
\begin{itemize}
\item {Utilização:Pop.}
\end{itemize}
\begin{itemize}
\item {Proveniência:(Do rad. de \textunderscore girar\textunderscore )}
\end{itemize}
Passeio, giro.
\section{Giratacachém}
\begin{itemize}
\item {Grp. gram.:m.}
\end{itemize}
(V.girafa)
\section{Giratório}
\begin{itemize}
\item {Grp. gram.:adj.}
\end{itemize}
\begin{itemize}
\item {Proveniência:(De \textunderscore girar\textunderscore )}
\end{itemize}
O mesmo que \textunderscore circulatório\textunderscore .
\section{Girau}
\begin{itemize}
\item {Grp. gram.:m.}
\end{itemize}
\begin{itemize}
\item {Utilização:Bras}
\end{itemize}
Estrado, em que se assentam os passageiros que uma jangada transporta.
Leito de paus sôbre forquilhas cravadas no chão, no qual se põe o derribador de árvores corpulentas.
Palanque, dentro de casa, entre o pavimento e o tecto, para arrumação de objectos vários.
\section{Giravolta}
\begin{itemize}
\item {Grp. gram.:f.}
\end{itemize}
\begin{itemize}
\item {Utilização:Fam.}
\end{itemize}
\begin{itemize}
\item {Proveniência:(De \textunderscore girar\textunderscore  + \textunderscore voltar\textunderscore )}
\end{itemize}
Digressão, passeio; viravolta.
\section{Gíria}
\begin{itemize}
\item {Grp. gram.:f.}
\end{itemize}
\begin{itemize}
\item {Utilização:Pop.}
\end{itemize}
Linguagem especial ou privativa de fadistas, gatunos, etc., para não serem comprehendidos por outrem.
Calão.
Linguagem peculiar aos que exercem uma profissão ou arte.
Esperteza, astúcia.
(Relaciona-se com \textunderscore geringonça\textunderscore )
\section{Girianta}
\begin{itemize}
\item {Grp. gram.:f.}
\end{itemize}
Gíria.
Taberna.
\section{Giribanda}
\begin{itemize}
\item {Grp. gram.:f.}
\end{itemize}
\begin{itemize}
\item {Utilização:Pop.}
\end{itemize}
Gamarra.
Descompostura.
\section{Giribato}
\begin{itemize}
\item {Grp. gram.:m.}
\end{itemize}
\begin{itemize}
\item {Utilização:Gír.}
\end{itemize}
Vinho.
\section{Girifalte}
\begin{itemize}
\item {Grp. gram.:m.}
\end{itemize}
\begin{itemize}
\item {Utilização:Ant.}
\end{itemize}
O mesmo que \textunderscore gerifalte\textunderscore .
\section{Girigote}
\begin{itemize}
\item {Grp. gram.:adj.}
\end{itemize}
\begin{itemize}
\item {Utilização:Pop.}
\end{itemize}
Velhaco; trapaceiro.
(Cp. \textunderscore girigoto\textunderscore )
\section{Girigoto}
\begin{itemize}
\item {fónica:gô}
\end{itemize}
\begin{itemize}
\item {Grp. gram.:m.  e  adj.}
\end{itemize}
\begin{itemize}
\item {Utilização:Gír.}
\end{itemize}
O que fala gíria.
\section{Girimato}
\begin{itemize}
\item {Grp. gram.:m.}
\end{itemize}
Planta verbenácea do Brasil, (\textunderscore vitex gardneriana\textunderscore ).
\section{Girimu}
\begin{itemize}
\item {Grp. gram.:f.}
\end{itemize}
Espécie de abóbora amarelada do Brasil, (\textunderscore cucurbita major rotunda\textunderscore ).
Fruta desta planta.
Nome de várias plantas cucurbitáceas do Brasil.
\section{Girimum}
\begin{itemize}
\item {Grp. gram.:m.}
\end{itemize}
O mesmo que \textunderscore girimu\textunderscore .
\section{Gírio}
\begin{itemize}
\item {Grp. gram.:adj.}
\end{itemize}
\begin{itemize}
\item {Utilização:Pop.}
\end{itemize}
\begin{itemize}
\item {Utilização:Fig.}
\end{itemize}
\begin{itemize}
\item {Proveniência:(Do rad. de \textunderscore gíria\textunderscore )}
\end{itemize}
Que fala gíria.
Que usa de gíria, de astúcia.
\section{Gírio}
\begin{itemize}
\item {Grp. gram.:adj.}
\end{itemize}
\begin{itemize}
\item {Utilização:Prov.}
\end{itemize}
\begin{itemize}
\item {Utilização:beir.}
\end{itemize}
Activo, esperto.
Que é agenciador de meios de vida.
(Relaciona-se com \textunderscore girar\textunderscore ?)
\section{Giripití}
\begin{itemize}
\item {Grp. gram.:m.}
\end{itemize}
O mesmo que \textunderscore geribita\textunderscore .
\section{Girita}
\begin{itemize}
\item {Grp. gram.:f.}
\end{itemize}
\begin{itemize}
\item {Utilização:Bras. do N}
\end{itemize}
Mammífero, provavelmente o mesmo que \textunderscore bombardeiro\textunderscore .
\section{Giritana}
\begin{itemize}
\item {Grp. gram.:m.}
\end{itemize}
Variedade de feijão.
\section{Giriti}
\begin{itemize}
\item {Grp. gram.:m.}
\end{itemize}
Árvore de Angola.
\section{Giro}
\begin{itemize}
\item {Grp. gram.:m.}
\end{itemize}
\begin{itemize}
\item {Utilização:Bras. do N}
\end{itemize}
\begin{itemize}
\item {Utilização:Fam.}
\end{itemize}
\begin{itemize}
\item {Proveniência:(Do gr. \textunderscore guros\textunderscore )}
\end{itemize}
Volta, circuito.
Circunlóquio.
Vez.
Commércio.
Jôgo de quatro parceiros, ao bilhar.
Diz-se o indivíduo, que facilmente consegue negócios lucrativos.
Passeio, pequena excursão: \textunderscore fui dar um giro\textunderscore .
\section{Giroé}
\begin{itemize}
\item {Grp. gram.:m.}
\end{itemize}
Ave africana, (\textunderscore polydauges leucogaster\textunderscore ).
\section{Giroeta}
\begin{itemize}
\item {fónica:ê}
\end{itemize}
\begin{itemize}
\item {Grp. gram.:f.}
\end{itemize}
\begin{itemize}
\item {Utilização:Gal}
\end{itemize}
\begin{itemize}
\item {Proveniência:(Fr. \textunderscore girouette\textunderscore )}
\end{itemize}
Catavento, ventoínha. Cf. Macedo, \textunderscore Burros\textunderscore , 247.
\section{Girofalco}
\begin{itemize}
\item {Grp. gram.:m.}
\end{itemize}
O mesmo que \textunderscore gerifalte\textunderscore .
\section{Girofle}
\begin{itemize}
\item {Grp. gram.:m.}
\end{itemize}
\begin{itemize}
\item {Proveniência:(Fr. \textunderscore girafle\textunderscore )}
\end{itemize}
Cravo da Índia.
\section{Giroma}
\begin{itemize}
\item {Grp. gram.:m.}
\end{itemize}
\begin{itemize}
\item {Utilização:Bot.}
\end{itemize}
\begin{itemize}
\item {Proveniência:(Gr. \textunderscore guroma\textunderscore )}
\end{itemize}
Receptáculo orbicular dos órgãos reproductores de alguns líchens.
\section{Gironda}
\begin{itemize}
\item {Grp. gram.:f.}
\end{itemize}
\begin{itemize}
\item {Utilização:Prov.}
\end{itemize}
\begin{itemize}
\item {Utilização:alent.}
\end{itemize}
\begin{itemize}
\item {Utilização:T. de Serpa}
\end{itemize}
Fêmea do javali, quando completamente desenvolvida ou velha.
Qualquer porca velha.
\section{Gironda}
\begin{itemize}
\item {Grp. gram.:f.}
\end{itemize}
\begin{itemize}
\item {Proveniência:(De \textunderscore Gironda\textunderscore , n. p.)}
\end{itemize}
O partido dos girondinos em França.
\section{Girondino}
\begin{itemize}
\item {Grp. gram.:adj.}
\end{itemize}
\begin{itemize}
\item {Grp. gram.:M.}
\end{itemize}
\begin{itemize}
\item {Grp. gram.:Pl.}
\end{itemize}
\begin{itemize}
\item {Proveniência:(De \textunderscore Gironda\textunderscore , n. p.)}
\end{itemize}
Relativo á Gironda, em França.
Relativo ao partido da Gironda.
Membro da Gironda.
Partido republicano moderado, que se formou em França, em 1791.
\section{Giropo}
\begin{itemize}
\item {fónica:girô}
\end{itemize}
\begin{itemize}
\item {Grp. gram.:m.}
\end{itemize}
\begin{itemize}
\item {Utilização:ant.}
\end{itemize}
\begin{itemize}
\item {Utilização:Gír.}
\end{itemize}
Caldo.
\section{Girote}
\begin{itemize}
\item {Grp. gram.:m.}
\end{itemize}
\begin{itemize}
\item {Utilização:Gír.}
\end{itemize}
\begin{itemize}
\item {Proveniência:(De \textunderscore girar\textunderscore )}
\end{itemize}
Vàdio.
\section{Giroto}
\begin{itemize}
\item {fónica:girô}
\end{itemize}
\begin{itemize}
\item {Grp. gram.:adj.}
\end{itemize}
\begin{itemize}
\item {Utilização:Prov.}
\end{itemize}
\begin{itemize}
\item {Utilização:trasm.}
\end{itemize}
Que vai dar uma volta longe de casa: \textunderscore gallinha girota\textunderscore .
(Cp. \textunderscore girote\textunderscore )
\section{Gisnado}
\begin{itemize}
\item {Grp. gram.:adj.}
\end{itemize}
\begin{itemize}
\item {Utilização:Prov.}
\end{itemize}
\begin{itemize}
\item {Utilização:trasm.}
\end{itemize}
Unido, apertado nas juntas: \textunderscore aduelas gisnadas\textunderscore .
\section{Gitano}
\begin{itemize}
\item {Grp. gram.:m.}
\end{itemize}
O mesmo que \textunderscore cigano\textunderscore ^1.
(Contr. de \textunderscore egitano\textunderscore , de \textunderscore Egíto\textunderscore  = \textunderscore Egypto\textunderscore , n. p.)
\section{Gitirana}
\begin{itemize}
\item {Grp. gram.:f.}
\end{itemize}
Planta convolvulácea, (\textunderscore argyreia alagoana\textunderscore ).
\section{Gito}
\begin{itemize}
\item {Grp. gram.:m.}
\end{itemize}
\begin{itemize}
\item {Proveniência:(Do fr. \textunderscore jet\textunderscore ?)}
\end{itemize}
Cano, que conduz para o molde o metal fundido.
\section{Gitó}
\begin{itemize}
\item {Grp. gram.:m.}
\end{itemize}
O mesmo que \textunderscore utuaba\textunderscore .
\section{Giz}
\begin{itemize}
\item {Grp. gram.:m.}
\end{itemize}
\begin{itemize}
\item {Utilização:Bras. do N}
\end{itemize}
\begin{itemize}
\item {Proveniência:(Do ár. \textunderscore jibs\textunderscore )}
\end{itemize}
Variedade de carbonato de cal, usada especialmente para escrever em ardósia ou em quadro preto de escolas.
Traço rectilíneo a ferro quente, com que se assinala o animal vacum, que vai sendo inventariado.
\section{Gizamento}
\begin{itemize}
\item {Grp. gram.:m.}
\end{itemize}
Acto de gizar.
\section{Gizar}
\begin{itemize}
\item {Grp. gram.:v. t.}
\end{itemize}
\begin{itemize}
\item {Utilização:Fig.}
\end{itemize}
\begin{itemize}
\item {Utilização:Gír.}
\end{itemize}
\begin{itemize}
\item {Grp. gram.:V. i.}
\end{itemize}
Traçar com giz.
Delinear.
Furtar.
\textunderscore Gizar por\textunderscore , dar mostras ou apparência de:«\textunderscore ...que só pela cara ou qualquer outra parecença gizavão pelo judaísmo...\textunderscore »Filinto, I, 346.
\section{Glabela}
\begin{itemize}
\item {Grp. gram.:f.}
\end{itemize}
\begin{itemize}
\item {Utilização:Anat.}
\end{itemize}
\begin{itemize}
\item {Proveniência:(Do lat. \textunderscore glabellus\textunderscore )}
\end{itemize}
Espaço, compreendido entre as sobrancelhas.
\section{Glabella}
\begin{itemize}
\item {Grp. gram.:f.}
\end{itemize}
\begin{itemize}
\item {Utilização:Anat.}
\end{itemize}
\begin{itemize}
\item {Proveniência:(Do lat. \textunderscore glabellus\textunderscore )}
\end{itemize}
Espaço, comprehendido entre as sobrancelhas.
\section{Glabrismo}
\begin{itemize}
\item {Grp. gram.:m.}
\end{itemize}
\begin{itemize}
\item {Utilização:Bot.}
\end{itemize}
\begin{itemize}
\item {Proveniência:(De \textunderscore glabro\textunderscore )}
\end{itemize}
Monstruosidade vegetal, caracterizada pela ausência de pêlos em vegetaes que ordinariamente os têm.
\section{Glabriúsculo}
\begin{itemize}
\item {Grp. gram.:adj.}
\end{itemize}
Quási glabro.
(Dem. de \textunderscore glabro\textunderscore )
\section{Glabro}
\begin{itemize}
\item {Grp. gram.:adj.}
\end{itemize}
\begin{itemize}
\item {Utilização:Bot.}
\end{itemize}
\begin{itemize}
\item {Proveniência:(Do lat. \textunderscore glaber\textunderscore )}
\end{itemize}
Diz-se dos órgãos vegetaes, que não tem pêlos nem glândulas.
Calvo.
\section{Glacial}
\begin{itemize}
\item {Grp. gram.:adj.}
\end{itemize}
\begin{itemize}
\item {Utilização:Fig.}
\end{itemize}
\begin{itemize}
\item {Grp. gram.:F.}
\end{itemize}
\begin{itemize}
\item {Utilização:Bras}
\end{itemize}
\begin{itemize}
\item {Proveniência:(Lat. \textunderscore glacialis\textunderscore )}
\end{itemize}
Gelado.
Relativo ao gêlo.
Muito frio: \textunderscore manhan glacial\textunderscore .
Que não tem animação; que não é expansivo; reservado.
Planta hortense.
\section{Glaciar}
\begin{itemize}
\item {Grp. gram.:m.}
\end{itemize}
\begin{itemize}
\item {Proveniência:(Do lat. \textunderscore glacies\textunderscore )}
\end{itemize}
O mesmo ou melhor que \textunderscore geleira\textunderscore .
Depósito de neve, que extravasa, formando como um rio gelado que desce pelos desfiladeiros, e que se parte em grandes pedaços, quando em contacto com algum lago ou mar, pedaços que, boiando, parecem montanhas de gêlo e se chamam \textunderscore ice-bergues\textunderscore .
\section{Glaciário}
\begin{itemize}
\item {Grp. gram.:adj.}
\end{itemize}
\begin{itemize}
\item {Proveniência:(Do lat. \textunderscore glacies\textunderscore )}
\end{itemize}
Relativo ao gêlo ou ás geleiras.
E diz-se do periodo geológico, em que a temperatura de vastas regiões era muito inferior á de hoje.
\section{Glaciarista}
\begin{itemize}
\item {Grp. gram.:m.}
\end{itemize}
Geólogo, que se occupa especialmente do período glaciário.
\section{Glacinda}
\begin{itemize}
\item {Grp. gram.:f.}
\end{itemize}
Nome de uma flôr. Cf. \textunderscore Bibl. da G. do Campo\textunderscore , 335.
\section{Gladiado}
\begin{itemize}
\item {Grp. gram.:adj.}
\end{itemize}
\begin{itemize}
\item {Utilização:Heráld.}
\end{itemize}
\begin{itemize}
\item {Proveniência:(Lat. \textunderscore gladiatus\textunderscore )}
\end{itemize}
Que é comprimido e tem arestas salientes, como um corpo cortante.
Ensiforme.
\section{Gladiador}
\begin{itemize}
\item {Grp. gram.:m.}
\end{itemize}
\begin{itemize}
\item {Utilização:Ext.}
\end{itemize}
\begin{itemize}
\item {Proveniência:(Lat. \textunderscore gladiator\textunderscore )}
\end{itemize}
Aquelle que combatia nos circos romanos, contra outros homens ou contra feras, para divertimento público.
Duellista.
\section{Gladiar-se}
\begin{itemize}
\item {Grp. gram.:v. p.}
\end{itemize}
O mesmo que \textunderscore digladiar\textunderscore . Cf. Latino, \textunderscore Elogios\textunderscore , 294.
\section{Gladiatório}
\begin{itemize}
\item {Grp. gram.:adj.}
\end{itemize}
\begin{itemize}
\item {Proveniência:(Lat. \textunderscore gladiatorius\textunderscore )}
\end{itemize}
Relativo a gladiador.
\section{Gladiatura}
\begin{itemize}
\item {Grp. gram.:f.}
\end{itemize}
\begin{itemize}
\item {Proveniência:(Lat. \textunderscore gladiatura\textunderscore )}
\end{itemize}
Combate de gladiadores.
Arte de gladiar.
\section{Gladífero}
\begin{itemize}
\item {Grp. gram.:adj.}
\end{itemize}
\begin{itemize}
\item {Utilização:Zool.}
\end{itemize}
\begin{itemize}
\item {Proveniência:(Do lat. \textunderscore gladius\textunderscore  + \textunderscore ferre\textunderscore )}
\end{itemize}
Que tem prolongamento em fórma de espada.
\section{Gládio}
\begin{itemize}
\item {Grp. gram.:m.}
\end{itemize}
\begin{itemize}
\item {Utilização:Fig.}
\end{itemize}
\begin{itemize}
\item {Proveniência:(Lat. \textunderscore gladius\textunderscore )}
\end{itemize}
Espada de dois gumes; espada.
Punhal.
Poder, fôrça.
Combate.
\section{Gladíolo}
\begin{itemize}
\item {Grp. gram.:m.}
\end{itemize}
\begin{itemize}
\item {Proveniência:(Lat. \textunderscore gladiolus\textunderscore )}
\end{itemize}
Gênero de plantas irídeas; o mesmo que \textunderscore espadana\textunderscore .
\section{Glaiadina}
\begin{itemize}
\item {Grp. gram.:f.}
\end{itemize}
Substância glutinosa, que se mistura com o vinho, para o tornar grosso e claro.
(Cp. fr. \textunderscore glayeul\textunderscore  = lat. \textunderscore gladiolus\textunderscore )
\section{Glamonta}
\begin{itemize}
\item {Grp. gram.:f.}
\end{itemize}
\begin{itemize}
\item {Utilização:Prov.}
\end{itemize}
\begin{itemize}
\item {Utilização:trasm.}
\end{itemize}
Varas de árvores desfolhadas.
\section{Glandado}
\begin{itemize}
\item {Grp. gram.:adj.}
\end{itemize}
\begin{itemize}
\item {Utilização:Heráld.}
\end{itemize}
Que termina em glande, (falando-se de peças do escudo heráldico).
\section{Glande}
\begin{itemize}
\item {Grp. gram.:f.}
\end{itemize}
\begin{itemize}
\item {Proveniência:(Lat. \textunderscore glans\textunderscore , \textunderscore glandis\textunderscore )}
\end{itemize}
O mesmo que \textunderscore bolota\textunderscore .
Objecto, de fórma semelhante á da bolota.
\section{Glandífero}
\begin{itemize}
\item {Grp. gram.:adj.}
\end{itemize}
\begin{itemize}
\item {Proveniência:(Lat. \textunderscore glandifer\textunderscore )}
\end{itemize}
Que tem ou produz bolotas.
\section{Glandiforme}
\begin{itemize}
\item {Grp. gram.:adj.}
\end{itemize}
\begin{itemize}
\item {Proveniência:(De \textunderscore glande\textunderscore  + \textunderscore fórma\textunderscore )}
\end{itemize}
Que tem fórma de glande.
\section{Glândula}
\begin{itemize}
\item {Grp. gram.:f.}
\end{itemize}
\begin{itemize}
\item {Proveniência:(Lat. \textunderscore glandula\textunderscore )}
\end{itemize}
Pequena glande, órgão esponjoso ou vascular que segrega um líquido orgânico.
Órgão vegetal, que contém líquido.
\section{Glandulação}
\begin{itemize}
\item {Grp. gram.:f.}
\end{itemize}
Estructura ou disposição das glândulas.
\section{Glandular}
\begin{itemize}
\item {Grp. gram.:adj.}
\end{itemize}
O mesmo que \textunderscore glanduloso\textunderscore .
\section{Glandulífero}
\begin{itemize}
\item {Grp. gram.:adj.}
\end{itemize}
\begin{itemize}
\item {Proveniência:(Do lat. \textunderscore glandula\textunderscore  + \textunderscore ferre\textunderscore )}
\end{itemize}
Que tem glândulas.
\section{Glanduliforme}
\begin{itemize}
\item {Grp. gram.:adj.}
\end{itemize}
\begin{itemize}
\item {Proveniência:(De \textunderscore glandula\textunderscore  + \textunderscore fórma\textunderscore )}
\end{itemize}
Que tem fórma de glândula.
\section{Glandulinas}
\begin{itemize}
\item {Grp. gram.:f. pl.}
\end{itemize}
\begin{itemize}
\item {Utilização:Zool.}
\end{itemize}
Seres, da classes dos protozoários,
\section{Glanduloso}
\begin{itemize}
\item {Grp. gram.:adj.}
\end{itemize}
\begin{itemize}
\item {Proveniência:(Lat. \textunderscore glandulosus\textunderscore )}
\end{itemize}
Que tem fórma ou natureza semelhante á da glândula.
\section{Glão}
\begin{itemize}
\item {Grp. gram.:m.}
\end{itemize}
\begin{itemize}
\item {Utilização:Prov.}
\end{itemize}
\begin{itemize}
\item {Utilização:trasm.}
\end{itemize}
Grêlo de batatas, recolhidas em casa.
\section{Gláucia}
\begin{itemize}
\item {Grp. gram.:f.}
\end{itemize}
\begin{itemize}
\item {Proveniência:(De \textunderscore glauco\textunderscore )}
\end{itemize}
Espécie de papoila.
\section{Gláucico}
\begin{itemize}
\item {Grp. gram.:adj.}
\end{itemize}
\begin{itemize}
\item {Proveniência:(De \textunderscore glauco\textunderscore )}
\end{itemize}
Que tem côr mais ou menos verde.
\section{Glaucina}
\begin{itemize}
\item {Grp. gram.:f.}
\end{itemize}
Alcaloide, extrahido de gláucia.
\section{Gláucio}
\begin{itemize}
\item {Grp. gram.:m.}
\end{itemize}
O mesmo que \textunderscore gláucia\textunderscore .
\section{Glauco}
\begin{itemize}
\item {Grp. gram.:adj.}
\end{itemize}
\begin{itemize}
\item {Proveniência:(Lat. \textunderscore glaucus\textunderscore )}
\end{itemize}
Esverdeado; que tem a côr verde-mar.
\section{Glaucoma}
\begin{itemize}
\item {Grp. gram.:m.}
\end{itemize}
\begin{itemize}
\item {Utilização:Med.}
\end{itemize}
\begin{itemize}
\item {Proveniência:(Gr. \textunderscore glaukoma\textunderscore )}
\end{itemize}
Doença de olhos, em que, geralmente, apparece aumento de tensão intraocular, e o humor vítreo se torna opaco.
\section{Glaucomatoso}
\begin{itemize}
\item {Grp. gram.:adj.}
\end{itemize}
\begin{itemize}
\item {Proveniência:(De \textunderscore glaucoma\textunderscore )}
\end{itemize}
Diz-se do ôlho, cuja consistência é superior á normal.
\section{Gleba}
\begin{itemize}
\item {Grp. gram.:f.}
\end{itemize}
\begin{itemize}
\item {Proveniência:(Lat. \textunderscore gleba\textunderscore )}
\end{itemize}
Torrão.
Solo cultivável.
Terreno.
Terreno, que contém mineral.
Terreno feudal.
\section{Glebário}
\begin{itemize}
\item {Grp. gram.:m.}
\end{itemize}
\begin{itemize}
\item {Utilização:Jur.}
\end{itemize}
\begin{itemize}
\item {Utilização:Des.}
\end{itemize}
Possuidor de gleba.
\section{Glena}
\begin{itemize}
\item {Grp. gram.:f.}
\end{itemize}
\begin{itemize}
\item {Utilização:Anat.}
\end{itemize}
\begin{itemize}
\item {Proveniência:(Do gr. \textunderscore glene\textunderscore )}
\end{itemize}
Cavidade de um osso, em que se articula outro.
\section{Glenodina}
\begin{itemize}
\item {Grp. gram.:f.}
\end{itemize}
\begin{itemize}
\item {Proveniência:(Do gr. \textunderscore glene\textunderscore  + \textunderscore dinos\textunderscore )}
\end{itemize}
Gênero de infusórios.
\section{Glenoidal}
\begin{itemize}
\item {Grp. gram.:adj.}
\end{itemize}
\begin{itemize}
\item {Proveniência:(De \textunderscore glenoide\textunderscore )}
\end{itemize}
Que se articula na glena.
\section{Glenoide}
\begin{itemize}
\item {Grp. gram.:adj.}
\end{itemize}
\begin{itemize}
\item {Proveniência:(Do gr. \textunderscore glene\textunderscore  + \textunderscore eídos\textunderscore )}
\end{itemize}
O mesmo que \textunderscore glenoidal\textunderscore .
\section{Glenoídeo}
\begin{itemize}
\item {Grp. gram.:adj.}
\end{itemize}
O mesmo que \textunderscore glenoidal\textunderscore .
\section{Gleucométrico}
\begin{itemize}
\item {Grp. gram.:adj.}
\end{itemize}
Relativo ao gleucómetro.
\section{Gleucómetro}
\begin{itemize}
\item {Grp. gram.:m.}
\end{itemize}
\begin{itemize}
\item {Proveniência:(Do gr. \textunderscore gleukos\textunderscore  + \textunderscore metron\textunderscore )}
\end{itemize}
Instrumento, com que se mede a quantidade de açúcar contido no mosto.
\section{Glicéria}
\begin{itemize}
\item {Grp. gram.:f.}
\end{itemize}
\begin{itemize}
\item {Proveniência:(De \textunderscore Glicera\textunderscore , n. p.)}
\end{itemize}
Gênero de plantas gramíneas.
\section{Glicéridas}
\begin{itemize}
\item {Grp. gram.:f. pl.}
\end{itemize}
Plantas, que tem por typo a glycéria.
\section{Gliconite}
\begin{itemize}
\item {Grp. gram.:f.}
\end{itemize}
Espécie de asphalto.
\section{Glioma}
\begin{itemize}
\item {Grp. gram.:m.}
\end{itemize}
\begin{itemize}
\item {Proveniência:(Do gr. \textunderscore glia\textunderscore , colla)}
\end{itemize}
Neoplasia do tecido intersticial dos centros nervosos, incluindo a retina.
\section{Globa}
\begin{itemize}
\item {Grp. gram.:f.}
\end{itemize}
\begin{itemize}
\item {Proveniência:(T. ind.)}
\end{itemize}
Gênero de plantas zingiberáceas.
\section{Globba}
\begin{itemize}
\item {Grp. gram.:f.}
\end{itemize}
\begin{itemize}
\item {Proveniência:(T. ind.)}
\end{itemize}
Gênero de plantas zingiberáceas.
\section{Globicéfalos}
\begin{itemize}
\item {Grp. gram.:m. pl}
\end{itemize}
Gênero de delfins do Mediterrâneo.
\section{Globicéphalos}
\begin{itemize}
\item {Grp. gram.:m. pl}
\end{itemize}
Gênero de delfins do Mediterrâneo.
\section{Globiconca}
\begin{itemize}
\item {Grp. gram.:f.}
\end{itemize}
\begin{itemize}
\item {Proveniência:(Do lat. \textunderscore globus\textunderscore  + \textunderscore concha\textunderscore )}
\end{itemize}
Gênero de molluscos gasterópodes.
\section{Globicórneo}
\begin{itemize}
\item {Grp. gram.:m.}
\end{itemize}
\begin{itemize}
\item {Proveniência:(Do lat. \textunderscore globus\textunderscore  + \textunderscore cornu\textunderscore )}
\end{itemize}
Gênero de insectos pentâmeros.
\section{Globífero}
\begin{itemize}
\item {Grp. gram.:adj.}
\end{itemize}
\begin{itemize}
\item {Proveniência:(Do lat. \textunderscore globus\textunderscore  + \textunderscore ferre\textunderscore )}
\end{itemize}
Que produz frutos arredondados.
\section{Globifloro}
\begin{itemize}
\item {Grp. gram.:adj.}
\end{itemize}
\begin{itemize}
\item {Proveniência:(Do lat. \textunderscore globus\textunderscore  + \textunderscore flos\textunderscore )}
\end{itemize}
Que tem flôres globosas.
\section{Globo}
\begin{itemize}
\item {fónica:glô}
\end{itemize}
\begin{itemize}
\item {Grp. gram.:m.}
\end{itemize}
\begin{itemize}
\item {Grp. gram.:Loc. adv.}
\end{itemize}
\begin{itemize}
\item {Proveniência:(Lat. \textunderscore globus\textunderscore )}
\end{itemize}
Corpo esphérico.
Bóla.
A esphera terrestre.
Representação esphérica do systema planetário.
\textunderscore Em globo\textunderscore , por junto; no conjunto, na totalidade.
\section{Globosidade}
\begin{itemize}
\item {Grp. gram.:f.}
\end{itemize}
Qualidade daquillo que é globoso.
\section{Globoso}
\begin{itemize}
\item {Grp. gram.:adj.}
\end{itemize}
\begin{itemize}
\item {Proveniência:(Lat. \textunderscore globosus\textunderscore )}
\end{itemize}
Que tem fórma de globo.
\section{Globular}
\begin{itemize}
\item {Grp. gram.:adj.}
\end{itemize}
O mesmo que \textunderscore globuloso\textunderscore .
\section{Globulária}
\begin{itemize}
\item {Grp. gram.:f.}
\end{itemize}
\begin{itemize}
\item {Proveniência:(De \textunderscore glóbulo\textunderscore )}
\end{itemize}
Planta dicotyledónea, cujas fôlhas se enrolam em fórma de bóla.
\section{Globulárias}
\begin{itemize}
\item {Grp. gram.:f. pl.}
\end{itemize}
\begin{itemize}
\item {Proveniência:(De \textunderscore globulária\textunderscore )}
\end{itemize}
Família de plantas dicotyledóneas, estabelecida por De-Candolle, á custa das primuláceas de Jussieu.
\section{Globulina}
\begin{itemize}
\item {Grp. gram.:f.}
\end{itemize}
\begin{itemize}
\item {Proveniência:(De \textunderscore glóbulo\textunderscore )}
\end{itemize}
Grãos de chlorophylla, que, segundo Turpin, constituem todo o tecido vegetal.
Substância orgânica, que se dissolve na água, coagulando-se logo, a uma temperatura elevada, e que, na urina, é symptoma de nephrite aguda ou catarro da bexiga.
\section{Globulinuria}
\begin{itemize}
\item {Grp. gram.:f.}
\end{itemize}
\begin{itemize}
\item {Proveniência:(De \textunderscore globulina\textunderscore  + gr. \textunderscore ouron\textunderscore )}
\end{itemize}
Variedade de albuminuria, caracterizada pela exclusiva presença de globulina na urina.
\section{Glóbulo}
\begin{itemize}
\item {Grp. gram.:m.}
\end{itemize}
\begin{itemize}
\item {Proveniência:(Lat. \textunderscore globulus\textunderscore )}
\end{itemize}
Pequeno globo.
Corpúsculo arredondado, dos que se encontram em tecidos e líquidos animaes.
\section{Glicera}
\begin{itemize}
\item {Grp. gram.:f.}
\end{itemize}
Animal radiário, que habita as grandes profundidades oceânicas.
\section{Gliceramina}
\begin{itemize}
\item {Grp. gram.:f.}
\end{itemize}
\begin{itemize}
\item {Proveniência:(De \textunderscore glicerina\textunderscore  + \textunderscore amoníaco\textunderscore )}
\end{itemize}
Amoníaco composto, derivado da glicerina.
\section{Glicerato}
\begin{itemize}
\item {Grp. gram.:m.}
\end{itemize}
\begin{itemize}
\item {Proveniência:(De \textunderscore glicerina\textunderscore )}
\end{itemize}
Preparação farmacêutica, que tem por base a glicerina.
\section{Glicéreo}
\begin{itemize}
\item {Grp. gram.:adj.}
\end{itemize}
O mesmo que \textunderscore glicérico\textunderscore .
\section{Glicérico}
\begin{itemize}
\item {Grp. gram.:adj.}
\end{itemize}
\begin{itemize}
\item {Utilização:Chím.}
\end{itemize}
\begin{itemize}
\item {Proveniência:(De \textunderscore glicerina\textunderscore )}
\end{itemize}
Que tem por base a glicerina.
\section{Glicerina}
\begin{itemize}
\item {Grp. gram.:f.}
\end{itemize}
\begin{itemize}
\item {Proveniência:(Do gr. \textunderscore glukeros\textunderscore )}
\end{itemize}
Líquido xaroposo, de sabôr açucarado, e que é a base de todas as gorduras.
\section{Glicerofosfato}
\begin{itemize}
\item {Grp. gram.:m.}
\end{itemize}
Composto químico, medicinal.
\section{Gliceróleo}
\begin{itemize}
\item {Grp. gram.:m.}
\end{itemize}
\begin{itemize}
\item {Proveniência:(De \textunderscore glicéreo\textunderscore  + \textunderscore óleo\textunderscore )}
\end{itemize}
Medicamento, que tem a glicerina como excipiente.
\section{Glícico}
\begin{itemize}
\item {Grp. gram.:adj.}
\end{itemize}
\begin{itemize}
\item {Proveniência:(Do gr. \textunderscore glukus\textunderscore )}
\end{itemize}
Diz-se de um ácido, resultante da acção dos álcalis sôbre a glicose.
\section{Glicina}
\begin{itemize}
\item {Grp. gram.:f.}
\end{itemize}
\begin{itemize}
\item {Proveniência:(Do gr. \textunderscore glukus\textunderscore )}
\end{itemize}
Óxido metálico, descoberto na esmeralda.
Substância cristalina açucarada, que se encontra no líquido contido em a noz do côco.
\section{Glicínia}
\begin{itemize}
\item {Grp. gram.:f.}
\end{itemize}
\begin{itemize}
\item {Proveniência:(Do gr. \textunderscore glukus\textunderscore )}
\end{itemize}
Planta leguminosa, ornamental, (\textunderscore glycinia nigricans\textunderscore ).
\section{Glicocarbóleo}
\begin{itemize}
\item {Grp. gram.:m.}
\end{itemize}
Medicamento, feito de glicerina, ácido fênico e essência de alfazema.
\section{Glicocola}
\begin{itemize}
\item {Grp. gram.:f.}
\end{itemize}
\begin{itemize}
\item {Proveniência:(Do gr. \textunderscore glukus\textunderscore  + \textunderscore kolla\textunderscore )}
\end{itemize}
Corpo cristalizável, branco e de sabor açucarado, resultante da acção do ácido sulfúrico sôbre a gelatina.
\section{Glicogenia}
\begin{itemize}
\item {Grp. gram.:f.}
\end{itemize}
\begin{itemize}
\item {Proveniência:(De \textunderscore glicógeno\textunderscore )}
\end{itemize}
Producção do açúcar no organismo animal.
\section{Glicogênico}
\begin{itemize}
\item {Grp. gram.:adj.}
\end{itemize}
Relativo á glicogenia.
\section{Glicógeno}
\begin{itemize}
\item {Grp. gram.:adj.}
\end{itemize}
\begin{itemize}
\item {Proveniência:(Do gr. \textunderscore glukus\textunderscore  + \textunderscore genea\textunderscore )}
\end{itemize}
Que produz açúcar.
\section{Glicol}
\begin{itemize}
\item {Grp. gram.:m.}
\end{itemize}
\begin{itemize}
\item {Proveniência:(De \textunderscore glicerina\textunderscore  e \textunderscore álcool\textunderscore )}
\end{itemize}
Substância intermediária ao álcool e á glicerina, pelas suas propriedades físicas e químicas.
\section{Glicólise}
\begin{itemize}
\item {Grp. gram.:f.}
\end{itemize}
\begin{itemize}
\item {Proveniência:(Do gr. \textunderscore glukus\textunderscore  + \textunderscore lusis\textunderscore )}
\end{itemize}
Transformação da glicose no seio do organismo animal.
\section{Glicolítico}
\begin{itemize}
\item {Grp. gram.:adj.}
\end{itemize}
Que tem a propriedade de realizar a glicólise.
\section{Glicómetro}
\begin{itemize}
\item {Grp. gram.:m.}
\end{itemize}
\begin{itemize}
\item {Proveniência:(Do gr. \textunderscore glukus\textunderscore  + \textunderscore metron\textunderscore )}
\end{itemize}
O mesmo que \textunderscore gleucómetro\textunderscore .
\section{Glicónico}
\begin{itemize}
\item {Grp. gram.:adj.}
\end{itemize}
\begin{itemize}
\item {Proveniência:(Do gr. \textunderscore Glucon\textunderscore , n. p.)}
\end{itemize}
Diz-se de um verso grego ou latino, composto de um espondeu e dois dáctilos.
\section{Glicofosfatado}
\begin{itemize}
\item {Grp. gram.:adj.}
\end{itemize}
Que tem glicol e fósforo.
\section{Glicofosfato}
\begin{itemize}
\item {Grp. gram.:m.}
\end{itemize}
Substância, composta de glicol e fósforo.
\section{Glicosana}
\begin{itemize}
\item {Grp. gram.:f.}
\end{itemize}
\begin{itemize}
\item {Proveniência:(De \textunderscore glicose\textunderscore )}
\end{itemize}
Um dos productos da acção do calor sôbre a glicose.
\section{Glicose}
\begin{itemize}
\item {Grp. gram.:f.}
\end{itemize}
\begin{itemize}
\item {Proveniência:(Do gr. \textunderscore glukus\textunderscore )}
\end{itemize}
\begin{itemize}
\item {Proveniência:(V. }
\end{itemize}
\begin{itemize}
\item {Proveniência:glycose}
\end{itemize}
\begin{itemize}
\item {Proveniência:)}
\end{itemize}
Açúcar de uvas ou do amido.
\section{Glicosina}
\begin{itemize}
\item {Grp. gram.:f.}
\end{itemize}
\begin{itemize}
\item {Utilização:Chím.}
\end{itemize}
\begin{itemize}
\item {Proveniência:(De \textunderscore glicose\textunderscore )}
\end{itemize}
Base cristalina, branca e volátil, resultante da acção do amoníaco sôbre a glicosana.
\section{Glicosuria}
\begin{itemize}
\item {Grp. gram.:f.}
\end{itemize}
\begin{itemize}
\item {Proveniência:(Do gr. \textunderscore glukus\textunderscore  + \textunderscore ouron\textunderscore )}
\end{itemize}
O mesmo que \textunderscore diabete\textunderscore .
\section{Glicosúrico}
\begin{itemize}
\item {Grp. gram.:adj.}
\end{itemize}
\begin{itemize}
\item {Grp. gram.:M.}
\end{itemize}
Relativo á glicosuria.
Aquele que sofre glicosuria.
\section{Glifo}
\begin{itemize}
\item {Grp. gram.:m.}
\end{itemize}
\begin{itemize}
\item {Proveniência:(Do gr. \textunderscore gluphe\textunderscore )}
\end{itemize}
Cavidade em ornatos arquitectónicos.
\section{Glíptica}
\begin{itemize}
\item {Grp. gram.:f.}
\end{itemize}
\begin{itemize}
\item {Proveniência:(Do gr. \textunderscore gluptos\textunderscore )}
\end{itemize}
Arte de gravar em pedras preciosas.
\section{Gliptodonte}
\begin{itemize}
\item {Grp. gram.:m.}
\end{itemize}
\begin{itemize}
\item {Proveniência:(Do gr. \textunderscore gluptos\textunderscore  + \textunderscore odous\textunderscore , \textunderscore odontos\textunderscore )}
\end{itemize}
Gênero fóssil de grandes mamíferos desdentados.
\section{Gliptognosia}
\begin{itemize}
\item {Grp. gram.:f.}
\end{itemize}
\begin{itemize}
\item {Proveniência:(Do gr. \textunderscore gluptos\textunderscore  + \textunderscore glosis\textunderscore )}
\end{itemize}
Conhecimento de pedras preciosas.
\section{Gliptografia}
\begin{itemize}
\item {Grp. gram.:f.}
\end{itemize}
\begin{itemize}
\item {Proveniência:(Do gr. \textunderscore gluptos\textunderscore  + \textunderscore graphein\textunderscore )}
\end{itemize}
Descripção ou tratado de pedras preciosas gravadas.
\section{Globuloso}
\begin{itemize}
\item {Grp. gram.:adj.}
\end{itemize}
Que tem fórma de glóbulo; reduzido a glóbulos.
\section{Glom}
\begin{itemize}
\item {Grp. gram.:m.}
\end{itemize}
Arvore santhomense, de propriedades purgativas.
\section{Glomerar}
\begin{itemize}
\item {Grp. gram.:v. t.}
\end{itemize}
(V.agglomerar)
\section{Glómeris}
\begin{itemize}
\item {Grp. gram.:m. pl.}
\end{itemize}
\begin{itemize}
\item {Proveniência:(Do lat. \textunderscore glomus\textunderscore )}
\end{itemize}
Família de insectos myriápodes.
\section{Glomérula}
\begin{itemize}
\item {Grp. gram.:f.}
\end{itemize}
\begin{itemize}
\item {Utilização:Bot.}
\end{itemize}
\begin{itemize}
\item {Proveniência:(Do lat. \textunderscore glomus\textunderscore )}
\end{itemize}
Aggregação irregular de flôres ou de frutos.
\section{Glonoína}
\begin{itemize}
\item {Grp. gram.:f.}
\end{itemize}
O mesmo que \textunderscore nitro-glycerina\textunderscore .
\section{Glorear}
\begin{itemize}
\item {Grp. gram.:v. t.}
\end{itemize}
\begin{itemize}
\item {Utilização:Des.}
\end{itemize}
O mesmo que \textunderscore gloriar\textunderscore . Cf. \textunderscore Eufrosina\textunderscore , 352; Filinto, II, 121; III, 278.
\section{Glória}
\begin{itemize}
\item {Grp. gram.:f.}
\end{itemize}
\begin{itemize}
\item {Proveniência:(Lat. \textunderscore gloria\textunderscore )}
\end{itemize}
Nomeada honrosa.
Bôa fama.
Honra.
Esplendor.
Magnificência.
Preito.
Grande mérito.
Bem-aventurança eterna.
Círculo luminoso ou auréola, que se representa em volta de um santo ou de uma personagem illustre.
Certo jôgo de dados.
\section{Gloriabundo}
\begin{itemize}
\item {Grp. gram.:adj.}
\end{itemize}
\begin{itemize}
\item {Utilização:P. us.}
\end{itemize}
\begin{itemize}
\item {Proveniência:(Lat. \textunderscore gloriabundus\textunderscore )}
\end{itemize}
Que se gloría; que faz ostentação.
\section{Gloria-Patri}
\begin{itemize}
\item {fónica:glória-pátri}
\end{itemize}
\begin{itemize}
\item {Grp. gram.:m.}
\end{itemize}
Versículo, que se reza ou canta depois de certas orações ecclesiásticas.
(Primeiras palavras lat. dêsse versículo)
\section{Gloriar}
\begin{itemize}
\item {Grp. gram.:v. t.}
\end{itemize}
\begin{itemize}
\item {Grp. gram.:V. p.}
\end{itemize}
\begin{itemize}
\item {Proveniência:(Lat. \textunderscore gloriari\textunderscore )}
\end{itemize}
Cobrir de glória.
Cobrir-se de glória.
Ufanar-se; envaidar-se:«\textunderscore porem Deos, que se gloria de confundir os sábios...\textunderscore »\textunderscore Luz e Calor\textunderscore , 75.«\textunderscore ...da terra que se gloria de o chamar seu filho\textunderscore ». Camillo, \textunderscore Homem de Brios\textunderscore , 22.
\section{Glorificação}
\begin{itemize}
\item {Grp. gram.:f.}
\end{itemize}
\begin{itemize}
\item {Proveniência:(Lat. \textunderscore glorificatio\textunderscore )}
\end{itemize}
Acto de glorificar.
Ascensão dos justos á bem-aventurança.
\section{Glorificador}
\begin{itemize}
\item {Grp. gram.:m.  e  adj.}
\end{itemize}
O que glorifica.
\section{Glorificante}
\begin{itemize}
\item {Grp. gram.:adj.}
\end{itemize}
\begin{itemize}
\item {Proveniência:(Lat. \textunderscore glorificans\textunderscore )}
\end{itemize}
Que glorifica.
\section{Glorificar}
\begin{itemize}
\item {Grp. gram.:v. t.}
\end{itemize}
\begin{itemize}
\item {Grp. gram.:V. p.}
\end{itemize}
\begin{itemize}
\item {Proveniência:(Lat. \textunderscore glorificare\textunderscore )}
\end{itemize}
Prestar homenagem ou glória a.
Honrar.
Canonizar.
Alcançar glória.
\section{Glorificativo}
\begin{itemize}
\item {Grp. gram.:adj.}
\end{itemize}
Próprio para glorificar.
\section{Gloríola}
\begin{itemize}
\item {Grp. gram.:f.}
\end{itemize}
\begin{itemize}
\item {Proveniência:(Lat. \textunderscore gloriola\textunderscore )}
\end{itemize}
Pequena glória que se tira de pequenas coisas.
Bôa reputação, immerecida. Cf. Herculano, \textunderscore Hist. de Port.\textunderscore , I, 497; \textunderscore Opúsc.\textunderscore , IV, 175; Latino, \textunderscore Humboldt\textunderscore , 61; Camillo, \textunderscore Cancion. Al.\textunderscore , V.
\section{Gloriosamente}
\begin{itemize}
\item {Grp. gram.:adv.}
\end{itemize}
De modo glorioso; com glória.
\section{Glorioso}
\begin{itemize}
\item {Grp. gram.:adj.}
\end{itemize}
\begin{itemize}
\item {Proveniência:(Lat. \textunderscore gloriosus\textunderscore )}
\end{itemize}
Cheio de glória.
Honroso.
Victorioso.
Que dá glória ou honra.
\section{Glosa}
\begin{itemize}
\item {Grp. gram.:f.}
\end{itemize}
\begin{itemize}
\item {Utilização:Pop.}
\end{itemize}
\begin{itemize}
\item {Proveniência:(Do gr. \textunderscore glossa\textunderscore )}
\end{itemize}
Nota explicativa, sôbre as palavras ou sentido de um texto, e escrita de ordinário á margem.
Interpretação; commentário.
Censura.
Composição poética, cada uma de cujas estrophes termina por um dos versos de um mote.
Suppressão.
\section{Glosador}
\begin{itemize}
\item {Grp. gram.:m.}
\end{itemize}
Aquelle que glosa.
\section{Glosar}
\begin{itemize}
\item {Grp. gram.:v. t.}
\end{itemize}
\begin{itemize}
\item {Utilização:Pop.}
\end{itemize}
\begin{itemize}
\item {Proveniência:(De \textunderscore glosa\textunderscore )}
\end{itemize}
Annotar; commentar.
Explicar.
Criticar.
Desenvolver em verso (um mote).
Annullar; supprimir.
\section{Glosinha}
\begin{itemize}
\item {Grp. gram.:f.}
\end{itemize}
\begin{itemize}
\item {Utilização:Prov.}
\end{itemize}
\begin{itemize}
\item {Utilização:alent.}
\end{itemize}
Variedade de azeitona.
\section{Glossa}
\textunderscore f.\textunderscore  (e der.) \textunderscore Ant.\textunderscore 
O mesmo que \textunderscore glosa\textunderscore , etc. Cf. B. Pereira, \textunderscore Prosódia\textunderscore , vb. \textunderscore scholiastes\textunderscore .
\section{Glossálgia}
\begin{itemize}
\item {Grp. gram.:m.}
\end{itemize}
\begin{itemize}
\item {Proveniência:(Do gr. \textunderscore glossa\textunderscore  + \textunderscore algos\textunderscore )}
\end{itemize}
Dôr ou enfermidade na língua.
\section{Glossantraz}
\begin{itemize}
\item {Grp. gram.:m.}
\end{itemize}
\begin{itemize}
\item {Proveniência:(Do gr. \textunderscore glossa\textunderscore  + \textunderscore antrax\textunderscore )}
\end{itemize}
Carbúnculo dos cavallos, que se lhes desenvolve principalmente na língua.
\section{Glossário}
\begin{itemize}
\item {Grp. gram.:m.}
\end{itemize}
\begin{itemize}
\item {Proveniência:(Lat. \textunderscore glossarium\textunderscore )}
\end{itemize}
Livro ou vocabulário, em que se explicam palavras obscuras.
Diccionário téchnico.
\section{Glossiano}
\begin{itemize}
\item {Grp. gram.:adj.}
\end{itemize}
\begin{itemize}
\item {Utilização:Anat.}
\end{itemize}
\begin{itemize}
\item {Proveniência:(Do gr. \textunderscore glossa\textunderscore )}
\end{itemize}
Relativo á língua.
\section{Glóssico}
\begin{itemize}
\item {Grp. gram.:adj.}
\end{itemize}
O mesmo que \textunderscore glossiano\textunderscore .
\section{Glossifónia}
\begin{itemize}
\item {Grp. gram.:f.}
\end{itemize}
\begin{itemize}
\item {Proveniência:(Do gr. \textunderscore glossa\textunderscore  + \textunderscore siphon\textunderscore )}
\end{itemize}
Espécie de pequena sanguesuga, que se enrola como os bichos de conta.
\section{Glossina}
\begin{itemize}
\item {Grp. gram.:f.}
\end{itemize}
O mesmo que \textunderscore tsetsé\textunderscore .
\section{Glossiphónia}
\begin{itemize}
\item {Grp. gram.:f.}
\end{itemize}
\begin{itemize}
\item {Proveniência:(Do gr. \textunderscore glossa\textunderscore  + \textunderscore siphon\textunderscore )}
\end{itemize}
Espécie de pequena sanguesuga, que se enrola como os bichos de conta.
\section{Glossite}
\begin{itemize}
\item {Grp. gram.:f.}
\end{itemize}
\begin{itemize}
\item {Proveniência:(Do gr. \textunderscore glossa\textunderscore )}
\end{itemize}
Inflammação da língua.
\section{Glossocele}
\begin{itemize}
\item {Grp. gram.:m.}
\end{itemize}
\begin{itemize}
\item {Proveniência:(Do gr. \textunderscore glossa\textunderscore  + \textunderscore kele\textunderscore )}
\end{itemize}
Doença da língua, que a obriga a estender-se para fóra da bôca.
\section{Glossócomo}
\begin{itemize}
\item {Grp. gram.:m.}
\end{itemize}
\begin{itemize}
\item {Proveniência:(Gr. \textunderscore glossokomon\textunderscore )}
\end{itemize}
Apparelho antigo, que servia para a reducção das fracturas e luxações da coxa e da perna.
\section{Glossódia}
\begin{itemize}
\item {Grp. gram.:f.}
\end{itemize}
Gênero de orchídeas.
\section{Glossodonte}
\begin{itemize}
\item {Grp. gram.:adj.}
\end{itemize}
\begin{itemize}
\item {Utilização:Zool.}
\end{itemize}
\begin{itemize}
\item {Proveniência:(Do gr. \textunderscore glossa\textunderscore  + \textunderscore odous\textunderscore , \textunderscore odontos\textunderscore )}
\end{itemize}
Diz-se do animal que tem dentes na língua.
\section{Glossografia}
\begin{itemize}
\item {Grp. gram.:f.}
\end{itemize}
\begin{itemize}
\item {Proveniência:(Do gr. \textunderscore glossa\textunderscore  + \textunderscore graphein\textunderscore )}
\end{itemize}
Investigação de palavras antigas ou obscuras.
Descripção anatómica da língua.
\section{Glossográfico}
\begin{itemize}
\item {Grp. gram.:adj.}
\end{itemize}
Relativo á glossografia.
\section{Glossógrafo}
\begin{itemize}
\item {Grp. gram.:m.}
\end{itemize}
Aquele que trabalha em glossografia.
\section{Glossographia}
\begin{itemize}
\item {Grp. gram.:f.}
\end{itemize}
\begin{itemize}
\item {Proveniência:(Do gr. \textunderscore glossa\textunderscore  + \textunderscore graphein\textunderscore )}
\end{itemize}
Investigação de palavras antigas ou obscuras.
Descripção anatómica da língua.
\section{Glossográphico}
\begin{itemize}
\item {Grp. gram.:adj.}
\end{itemize}
Relativo á glossographia.
\section{Glossógrapho}
\begin{itemize}
\item {Grp. gram.:m.}
\end{itemize}
Aquelle que trabalha em glossographia.
\section{Glossohyal}
\begin{itemize}
\item {Grp. gram.:m.}
\end{itemize}
\begin{itemize}
\item {Proveniência:(Do gr. \textunderscore glossa\textunderscore  + \textunderscore ...hyal\textunderscore )}
\end{itemize}
Apóphyse lingual do osso hyoide.
\section{Glossoial}
\begin{itemize}
\item {fónica:so-i}
\end{itemize}
\begin{itemize}
\item {Grp. gram.:m.}
\end{itemize}
\begin{itemize}
\item {Proveniência:(Do gr. \textunderscore glossa\textunderscore  + \textunderscore ...hyal\textunderscore )}
\end{itemize}
Apófise lingual do osso hioide.
\section{Glossoide}
\begin{itemize}
\item {Grp. gram.:adj.}
\end{itemize}
\begin{itemize}
\item {Proveniência:(Do gr. \textunderscore glossa\textunderscore  + \textunderscore eidos\textunderscore )}
\end{itemize}
Semelhante á língua.
\section{Glossologia}
\begin{itemize}
\item {Grp. gram.:f.}
\end{itemize}
\begin{itemize}
\item {Proveniência:(Do gr. \textunderscore glossa\textunderscore  + \textunderscore logos\textunderscore )}
\end{itemize}
O mesmo que \textunderscore glóttica\textunderscore .
Conjunto dos termos ou vocábulos, empregados numa especialidade dos conhecimentos humanos.
\section{Glossológico}
\begin{itemize}
\item {Grp. gram.:adj.}
\end{itemize}
Relativo á glossologia.
\section{Glossologista}
\begin{itemize}
\item {Grp. gram.:m.}
\end{itemize}
Aquelle que se occupa de glossologia. Cf. Latino, \textunderscore Elogios\textunderscore , 54.
\section{Glossólogo}
\begin{itemize}
\item {Grp. gram.:m.}
\end{itemize}
Aquelle que é perito em glossologia.
\section{Glossomancia}
\begin{itemize}
\item {Grp. gram.:f.}
\end{itemize}
\begin{itemize}
\item {Proveniência:(Do gr. \textunderscore glossa\textunderscore  + \textunderscore manteia\textunderscore )}
\end{itemize}
Arte de conhecer o carácter de uma pessôa pela fórma da língua.
\section{Glossomante}
\begin{itemize}
\item {Grp. gram.:m.}
\end{itemize}
Aquelle que pratica a glossomancia.
\section{Glossopetra}
\begin{itemize}
\item {Grp. gram.:f.}
\end{itemize}
\begin{itemize}
\item {Proveniência:(Do gr. \textunderscore glossa\textunderscore  + \textunderscore petra\textunderscore )}
\end{itemize}
Pedra fóssil, que representa uma língua e é o dente de um peixe fóssil.
\section{Glossoteca}
\begin{itemize}
\item {Grp. gram.:f.}
\end{itemize}
\begin{itemize}
\item {Utilização:Zool.}
\end{itemize}
\begin{itemize}
\item {Proveniência:(Do gr. \textunderscore glossa\textunderscore  + \textunderscore theke\textunderscore )}
\end{itemize}
Parte da crisálida, em que se aloja a língua do insecto.
\section{Glossotheca}
\begin{itemize}
\item {Grp. gram.:f.}
\end{itemize}
\begin{itemize}
\item {Utilização:Zool.}
\end{itemize}
\begin{itemize}
\item {Proveniência:(Do gr. \textunderscore glossa\textunderscore  + \textunderscore theke\textunderscore )}
\end{itemize}
Parte da chrysálida, em que se aloja a língua do insecto.
\section{Glossotomia}
\begin{itemize}
\item {Grp. gram.:f.}
\end{itemize}
\begin{itemize}
\item {Proveniência:(Do gr. \textunderscore glossa\textunderscore  + \textunderscore tome\textunderscore )}
\end{itemize}
Dissecção ou amputação da língua.
\section{Glote}
\begin{itemize}
\item {Grp. gram.:f.}
\end{itemize}
\begin{itemize}
\item {Proveniência:(Do gr. \textunderscore glotta\textunderscore )}
\end{itemize}
Abertura da laringe, circunscrita pelas duas cordas vocaes inferiores.
\section{Gloterar}
\begin{itemize}
\item {Grp. gram.:v. i.}
\end{itemize}
Diz-se da cegonha, quando solta a voz. Cf. Castilho, \textunderscore Fastos\textunderscore , III, 324.
\section{Glótica}
\begin{itemize}
\item {Grp. gram.:f.}
\end{itemize}
\begin{itemize}
\item {Proveniência:(De \textunderscore glótico\textunderscore )}
\end{itemize}
Ciência da linguagem.
\section{Glótico}
\begin{itemize}
\item {Grp. gram.:adj.}
\end{itemize}
Relativo á glote.
\section{Glotite}
\begin{itemize}
\item {Grp. gram.:f.}
\end{itemize}
Inflamação da glote.
\section{Glotização}
\begin{itemize}
\item {Grp. gram.:f.}
\end{itemize}
\begin{itemize}
\item {Utilização:Philol.}
\end{itemize}
\begin{itemize}
\item {Proveniência:(De \textunderscore glote\textunderscore )}
\end{itemize}
Pronunciação das consoantes com resonância na faringe.
\section{Glotologia}
\begin{itemize}
\item {Grp. gram.:f.}
\end{itemize}
Estudo científico das línguas.
O mesmo que \textunderscore glótica\textunderscore .
(Cp. \textunderscore glotólogo\textunderscore )
\section{Glotológico}
\begin{itemize}
\item {Grp. gram.:adj.}
\end{itemize}
Relativo á glotologia.
\section{Glotólogo}
\begin{itemize}
\item {Grp. gram.:m.}
\end{itemize}
\begin{itemize}
\item {Proveniência:(Do gr. \textunderscore glotta\textunderscore  + \textunderscore logos\textunderscore )}
\end{itemize}
Aquele que cultiva a glotologia ou que é perito nela.
\section{Glotte}
\begin{itemize}
\item {Grp. gram.:f.}
\end{itemize}
\begin{itemize}
\item {Proveniência:(Do gr. \textunderscore glotta\textunderscore )}
\end{itemize}
Abertura da larynge, circunscrita pelas duas cordas vocaes inferiores.
\section{Glóttica}
\begin{itemize}
\item {Grp. gram.:f.}
\end{itemize}
\begin{itemize}
\item {Proveniência:(De \textunderscore glóttico\textunderscore )}
\end{itemize}
Sciência da linguagem.
\section{Glóttico}
\begin{itemize}
\item {Grp. gram.:adj.}
\end{itemize}
Relativo á glotte.
\section{Glottite}
\begin{itemize}
\item {Grp. gram.:f.}
\end{itemize}
Inflammação da glotte.
\section{Glottização}
\begin{itemize}
\item {Grp. gram.:f.}
\end{itemize}
\begin{itemize}
\item {Utilização:Philol.}
\end{itemize}
\begin{itemize}
\item {Proveniência:(De \textunderscore glotte\textunderscore )}
\end{itemize}
Pronunciação das consoantes com resonância na pharynge.
\section{Glottologia}
\begin{itemize}
\item {Grp. gram.:f.}
\end{itemize}
Estudo scientífico das línguas.
O mesmo que \textunderscore glóttica\textunderscore .
(Cp. \textunderscore glottólogo\textunderscore )
\section{Glottológico}
\begin{itemize}
\item {Grp. gram.:adj.}
\end{itemize}
Relativo á glottologia.
\section{Glottólogo}
\begin{itemize}
\item {Grp. gram.:m.}
\end{itemize}
\begin{itemize}
\item {Proveniência:(Do gr. \textunderscore glotta\textunderscore  + \textunderscore logos\textunderscore )}
\end{itemize}
Aquelle que cultiva a glottologia ou que é perito nella.
\section{Gloxinia}
\begin{itemize}
\item {fónica:csi}
\end{itemize}
\begin{itemize}
\item {Grp. gram.:f.}
\end{itemize}
Gênero de plantas escrofularíneas.
\section{Glu-glu}
\begin{itemize}
\item {Grp. gram.:m.}
\end{itemize}
\begin{itemize}
\item {Proveniência:(T. onom.)}
\end{itemize}
Voz imitativa da do peru.
Voz imitativa do som de um líquido, que sai de gargalo estreito de um vaso.
\section{Gluma}
\begin{itemize}
\item {Grp. gram.:f.}
\end{itemize}
\begin{itemize}
\item {Utilização:Bot.}
\end{itemize}
\begin{itemize}
\item {Proveniência:(Lat. \textunderscore gluma\textunderscore )}
\end{itemize}
Invólucro da flôr das gramíneas, a que serve de cálice e de corolla.
\section{Glumáceas}
\begin{itemize}
\item {Grp. gram.:f. pl.}
\end{itemize}
\begin{itemize}
\item {Proveniência:(De \textunderscore glumáceo\textunderscore )}
\end{itemize}
Classe de plantas, que comprehende gramíneas, cyperáceas e juncos.
\section{Glumáceo}
\begin{itemize}
\item {Grp. gram.:adj.}
\end{itemize}
\begin{itemize}
\item {Utilização:Bot.}
\end{itemize}
\begin{itemize}
\item {Proveniência:(De \textunderscore gluma\textunderscore )}
\end{itemize}
Que não tem periantho propriamente dito, mas brácteas.
\section{Glutão}
\begin{itemize}
\item {Grp. gram.:adj.}
\end{itemize}
\begin{itemize}
\item {Grp. gram.:M.}
\end{itemize}
\begin{itemize}
\item {Grp. gram.:Pl.}
\end{itemize}
\begin{itemize}
\item {Proveniência:(Lat. \textunderscore gluto\textunderscore )}
\end{itemize}
Que come muito e com avidez.
Homem glutão.
Gênero de mammíferos carnívoros.
\section{Gluteína}
\begin{itemize}
\item {Grp. gram.:f.}
\end{itemize}
\begin{itemize}
\item {Proveniência:(De \textunderscore glúten\textunderscore )}
\end{itemize}
Substância amarela, que entra na composição da gordura da salamandra aquática, (\textunderscore triton cristatus\textunderscore ).
\section{Glúten}
\begin{itemize}
\item {Grp. gram.:f.}
\end{itemize}
\begin{itemize}
\item {Proveniência:(Lat. \textunderscore gluten\textunderscore , colla)}
\end{itemize}
Matéria orgânica, viscosa e azotada, que fica da farinha dos cereaes, quando se separa dêstes o amido.
\section{Glúteo}
\begin{itemize}
\item {Grp. gram.:adj.}
\end{itemize}
\begin{itemize}
\item {Utilização:Anat.}
\end{itemize}
\begin{itemize}
\item {Proveniência:(Do gr. \textunderscore glutos\textunderscore )}
\end{itemize}
Relativo ás nádegas: \textunderscore artéria glútea\textunderscore .
\section{Glutina}
\begin{itemize}
\item {Grp. gram.:f.}
\end{itemize}
\begin{itemize}
\item {Proveniência:(Do lat. \textunderscore glutinum\textunderscore )}
\end{itemize}
Albumina vegetal.
\section{Glutinar}
\begin{itemize}
\item {Grp. gram.:v. t.}
\end{itemize}
\begin{itemize}
\item {Proveniência:(Lat. \textunderscore glutinare\textunderscore )}
\end{itemize}
O mesmo que \textunderscore conglutinar\textunderscore .
\section{Glutinativo}
\begin{itemize}
\item {Grp. gram.:adj.}
\end{itemize}
\begin{itemize}
\item {Proveniência:(Lat. \textunderscore glutinativus\textunderscore )}
\end{itemize}
O mesmo que \textunderscore agglutinativo\textunderscore .
\section{Glutinosidade}
\begin{itemize}
\item {Grp. gram.:f.}
\end{itemize}
Qualidade de glutinoso. Cf. \textunderscore Techn. Rur.\textunderscore , I, 212.
\section{Glutinoso}
\begin{itemize}
\item {Grp. gram.:adj.}
\end{itemize}
\begin{itemize}
\item {Proveniência:(Lat. \textunderscore glutinosus\textunderscore )}
\end{itemize}
Que tem glúten; parecido ao glúten.
\section{Glutonaria}
\begin{itemize}
\item {Grp. gram.:f.}
\end{itemize}
\begin{itemize}
\item {Proveniência:(Do lat. \textunderscore gluto\textunderscore )}
\end{itemize}
Qualidade ou vício de glutão.
\section{Glutonia}
\begin{itemize}
\item {Grp. gram.:f.}
\end{itemize}
O mesmo que \textunderscore glutonaria\textunderscore .
\section{Glutónico}
\begin{itemize}
\item {Grp. gram.:adj.}
\end{itemize}
Relativo a glutão.
\section{Glycera}
\begin{itemize}
\item {Grp. gram.:f.}
\end{itemize}
Animal radiário, que habita as grandes profundidades oceânicas.
\section{Glyceramina}
\begin{itemize}
\item {Grp. gram.:f.}
\end{itemize}
\begin{itemize}
\item {Proveniência:(De \textunderscore glycerina\textunderscore  + \textunderscore ammoníaco\textunderscore )}
\end{itemize}
Ammoníaco composto, derivado da glycerina.
\section{Glycerato}
\begin{itemize}
\item {Grp. gram.:m.}
\end{itemize}
\begin{itemize}
\item {Proveniência:(De \textunderscore glycerina\textunderscore )}
\end{itemize}
Preparação pharmacêutica, que tem por base a glycerina.
\section{Glycéreo}
\begin{itemize}
\item {Grp. gram.:adj.}
\end{itemize}
O mesmo que \textunderscore glycérico\textunderscore .
\section{Glycéria}
\begin{itemize}
\item {Grp. gram.:f.}
\end{itemize}
Planta gramínea.
\section{Glycérico}
\begin{itemize}
\item {Grp. gram.:adj.}
\end{itemize}
\begin{itemize}
\item {Utilização:Chím.}
\end{itemize}
\begin{itemize}
\item {Proveniência:(De \textunderscore glycerina\textunderscore )}
\end{itemize}
Que tem por base a glycerina.
\section{Glycerina}
\begin{itemize}
\item {Grp. gram.:f.}
\end{itemize}
\begin{itemize}
\item {Proveniência:(Do gr. \textunderscore glukeros\textunderscore )}
\end{itemize}
Líquido xaroposo, de sabôr açucarado, e que é a base de todas as gorduras.
\section{Glyceróleo}
\begin{itemize}
\item {Grp. gram.:m.}
\end{itemize}
\begin{itemize}
\item {Proveniência:(De \textunderscore glycéreo\textunderscore  + \textunderscore óleo\textunderscore )}
\end{itemize}
Medicamento, que tem a glycerina como excipiente.
\section{Glycerophosphato}
\begin{itemize}
\item {Grp. gram.:m.}
\end{itemize}
Composto chímico, medicinal.
\section{Glýcico}
\begin{itemize}
\item {Grp. gram.:adj.}
\end{itemize}
\begin{itemize}
\item {Proveniência:(Do gr. \textunderscore glukus\textunderscore )}
\end{itemize}
Diz-se de um ácido, resultante da acção dos álcalis sôbre a glycose.
\section{Glycina}
\begin{itemize}
\item {Grp. gram.:f.}
\end{itemize}
\begin{itemize}
\item {Proveniência:(Do gr. \textunderscore glukus\textunderscore )}
\end{itemize}
Óxydo metállico, descoberto na esmeralda.
Substância crystallina açucarada, que se encontra no líquido contido em a noz do côco.
\section{Glycínia}
\begin{itemize}
\item {Grp. gram.:f.}
\end{itemize}
\begin{itemize}
\item {Proveniência:(Do gr. \textunderscore glukus\textunderscore )}
\end{itemize}
Planta leguminosa, ornamental, (\textunderscore glycinia nigricans\textunderscore ).
\section{Glycocarbóleo}
\begin{itemize}
\item {Grp. gram.:m.}
\end{itemize}
Medicamento, feito de glycerina, ácido phênico e essência de alfazema.
\section{Glycocolla}
\begin{itemize}
\item {Grp. gram.:f.}
\end{itemize}
\begin{itemize}
\item {Proveniência:(Do gr. \textunderscore glukus\textunderscore  + \textunderscore kolla\textunderscore )}
\end{itemize}
Corpo crystallizável, branco e de sabor açucarado, resultante da acção do ácido sulfúrico sôbre a gelatina.
\section{Glycogenia}
\begin{itemize}
\item {Grp. gram.:f.}
\end{itemize}
\begin{itemize}
\item {Proveniência:(De \textunderscore glycógeno\textunderscore )}
\end{itemize}
Producção do açúcar no organismo animal.
\section{Glycogênico}
\begin{itemize}
\item {Grp. gram.:adj.}
\end{itemize}
Relativo á glycogenia.
\section{Glycógeno}
\begin{itemize}
\item {Grp. gram.:adj.}
\end{itemize}
\begin{itemize}
\item {Proveniência:(Do gr. \textunderscore glukus\textunderscore  + \textunderscore genea\textunderscore )}
\end{itemize}
Que produz açúcar.
\section{Glycol}
\begin{itemize}
\item {Grp. gram.:m.}
\end{itemize}
\begin{itemize}
\item {Proveniência:(De \textunderscore glycerina\textunderscore  e \textunderscore álcool\textunderscore )}
\end{itemize}
Substância intermediária ao álcool e á glycerina, pelas suas propriedades phýsicas e chímicas.
\section{Glycólyse}
\begin{itemize}
\item {Grp. gram.:f.}
\end{itemize}
\begin{itemize}
\item {Proveniência:(Do gr. \textunderscore glukus\textunderscore  + \textunderscore lusis\textunderscore )}
\end{itemize}
Transformação da glycose no seio do organismo animal.
\section{Glycolýtico}
\begin{itemize}
\item {Grp. gram.:adj.}
\end{itemize}
Que tem a propriedade de realizar a glycólyse.
\section{Glycómetro}
\begin{itemize}
\item {Grp. gram.:m.}
\end{itemize}
\begin{itemize}
\item {Proveniência:(Do gr. \textunderscore glukus\textunderscore  + \textunderscore metron\textunderscore )}
\end{itemize}
O mesmo que \textunderscore gleucómetro\textunderscore .
\section{Glycónico}
\begin{itemize}
\item {Grp. gram.:adj.}
\end{itemize}
\begin{itemize}
\item {Proveniência:(Do gr. \textunderscore Glucon\textunderscore , n. p.)}
\end{itemize}
Diz-se de um verso grego ou latino, composto de um espondeu e dois dáctylos.
\section{Glycophosphatado}
\begin{itemize}
\item {Grp. gram.:adj.}
\end{itemize}
Que tem glycol e phósphoro.
\section{Glycophosphato}
\begin{itemize}
\item {Grp. gram.:m.}
\end{itemize}
Substância, composta de glycol e phósphoro.
\section{Glycosana}
\begin{itemize}
\item {Grp. gram.:f.}
\end{itemize}
\begin{itemize}
\item {Proveniência:(De \textunderscore glycose\textunderscore )}
\end{itemize}
Um dos productos da acção do calor sôbre a glycose.
\section{Glycose}
\begin{itemize}
\item {Grp. gram.:f.}
\end{itemize}
\begin{itemize}
\item {Proveniência:(Do gr. \textunderscore glukus\textunderscore )}
\end{itemize}
Açúcar de uvas ou do amido.
\section{Glycosina}
\begin{itemize}
\item {Grp. gram.:f.}
\end{itemize}
\begin{itemize}
\item {Utilização:Chím.}
\end{itemize}
\begin{itemize}
\item {Proveniência:(De \textunderscore glycose\textunderscore )}
\end{itemize}
Base crystallina, branca e volátil, resultante da acção do ammoníaco sôbre a glycosana.
\section{Glycosuria}
\begin{itemize}
\item {Grp. gram.:f.}
\end{itemize}
\begin{itemize}
\item {Proveniência:(Do gr. \textunderscore glukus\textunderscore  + \textunderscore ouron\textunderscore )}
\end{itemize}
O mesmo que \textunderscore diabete\textunderscore .
\section{Glycosúrico}
\begin{itemize}
\item {Grp. gram.:adj.}
\end{itemize}
\begin{itemize}
\item {Grp. gram.:M.}
\end{itemize}
Relativo á glycosuria.
Aquelle que soffre glycosuria.
\section{Glypho}
\begin{itemize}
\item {Grp. gram.:m.}
\end{itemize}
\begin{itemize}
\item {Proveniência:(Do gr. \textunderscore gluphe\textunderscore )}
\end{itemize}
Cavidade em ornatos architectónicos.
\section{Glýptica}
\begin{itemize}
\item {Grp. gram.:f.}
\end{itemize}
\begin{itemize}
\item {Proveniência:(Do gr. \textunderscore gluptos\textunderscore )}
\end{itemize}
Arte de gravar em pedras preciosas.
\section{Glyptodonte}
\begin{itemize}
\item {Grp. gram.:m.}
\end{itemize}
\begin{itemize}
\item {Proveniência:(Do gr. \textunderscore gluptos\textunderscore  + \textunderscore odous\textunderscore , \textunderscore odontos\textunderscore )}
\end{itemize}
Gênero fóssil de grandes mammíferos desdentados.
\section{Glyptognosia}
\begin{itemize}
\item {Grp. gram.:f.}
\end{itemize}
\begin{itemize}
\item {Proveniência:(Do gr. \textunderscore gluptos\textunderscore  + \textunderscore glosis\textunderscore )}
\end{itemize}
Conhecimento de pedras preciosas.
\section{Glyptographia}
\begin{itemize}
\item {Grp. gram.:f.}
\end{itemize}
\begin{itemize}
\item {Proveniência:(Do gr. \textunderscore gluptos\textunderscore  + \textunderscore graphein\textunderscore )}
\end{itemize}
Descripção ou tratado de pedras preciosas gravadas.
\section{Glyptologia}
\begin{itemize}
\item {Grp. gram.:f.}
\end{itemize}
\begin{itemize}
\item {Proveniência:(Do gr. \textunderscore gluptos\textunderscore  + \textunderscore logos\textunderscore )}
\end{itemize}
Tratado á cêrca de pedras gravadas, antigas.
\section{Glyptospermas}
\begin{itemize}
\item {Grp. gram.:f. pl.}
\end{itemize}
(V.anonáceas)
\section{Glyptotheca}
\begin{itemize}
\item {Grp. gram.:f.}
\end{itemize}
\begin{itemize}
\item {Proveniência:(Do gr. \textunderscore gluptos\textunderscore  + \textunderscore theke\textunderscore )}
\end{itemize}
Collecção ou museu de pedras gravadas.
\section{Gmelínia}
\begin{itemize}
\item {Grp. gram.:f.}
\end{itemize}
\begin{itemize}
\item {Proveniência:(De \textunderscore Gmelin\textunderscore , n. p.)}
\end{itemize}
Gênero de plantas verbenáceas.
\section{Gnafálio}
\begin{itemize}
\item {Grp. gram.:m.}
\end{itemize}
(V.cotonária)
\section{Gnaphálio}
\begin{itemize}
\item {Grp. gram.:m.}
\end{itemize}
(V.cotonária)
\section{Gnatáfanos}
\begin{itemize}
\item {Grp. gram.:m. pl.}
\end{itemize}
\begin{itemize}
\item {Proveniência:(Do gr. \textunderscore gnathos\textunderscore , mandíbula, e \textunderscore aphanos\textunderscore , que não brilha)}
\end{itemize}
Gênero de insectos coleópteros de Java.
\section{Gnatápteros}
\begin{itemize}
\item {Grp. gram.:m. pl.}
\end{itemize}
\begin{itemize}
\item {Proveniência:(Do gr. \textunderscore gnathos\textunderscore , mandíbula + \textunderscore a\textunderscore , priv. + \textunderscore pteron\textunderscore , asa)}
\end{itemize}
Ordem de insectos sem asas.
\section{Gnatháphanos}
\begin{itemize}
\item {Grp. gram.:m. pl.}
\end{itemize}
\begin{itemize}
\item {Proveniência:(Do gr. \textunderscore gnathos\textunderscore , mandíbula, e \textunderscore aphanos\textunderscore , que não brilha)}
\end{itemize}
Gênero de insectos coleópteros de Java.
\section{Gnathápteros}
\begin{itemize}
\item {Grp. gram.:m. pl.}
\end{itemize}
\begin{itemize}
\item {Proveniência:(Do gr. \textunderscore gnathos\textunderscore , mandíbula + \textunderscore a\textunderscore , priv. + \textunderscore pteron\textunderscore , asa)}
\end{itemize}
Ordem de insectos sem asas.
\section{Gnáthides}
\begin{itemize}
\item {Grp. gram.:f. pl.}
\end{itemize}
\begin{itemize}
\item {Proveniência:(Do gr. \textunderscore gnathos\textunderscore )}
\end{itemize}
Os ramos da mandíbula dos insectos.
\section{Gnathodonte}
\begin{itemize}
\item {Grp. gram.:adj.}
\end{itemize}
\begin{itemize}
\item {Utilização:Zool.}
\end{itemize}
\begin{itemize}
\item {Proveniência:(Do gr. \textunderscore gnathos\textunderscore  + \textunderscore odous\textunderscore , \textunderscore odontos\textunderscore )}
\end{itemize}
Que tem os dentes inseridos na espessura das maxillas.
\section{Gnátides}
\begin{itemize}
\item {Grp. gram.:f. pl.}
\end{itemize}
\begin{itemize}
\item {Proveniência:(Do gr. \textunderscore gnathos\textunderscore )}
\end{itemize}
Os ramos da mandíbula dos insectos.
\section{Gnaticídio}
\begin{itemize}
\item {Grp. gram.:m.}
\end{itemize}
\begin{itemize}
\item {Utilização:Des.}
\end{itemize}
\begin{itemize}
\item {Proveniência:(Do lat. \textunderscore gnatus\textunderscore  = \textunderscore natus\textunderscore  + \textunderscore caedere\textunderscore )}
\end{itemize}
O mesmo que \textunderscore filicídio\textunderscore .
\section{Gnatodonte}
\begin{itemize}
\item {Grp. gram.:adj.}
\end{itemize}
\begin{itemize}
\item {Utilização:Zool.}
\end{itemize}
\begin{itemize}
\item {Proveniência:(Do gr. \textunderscore gnathos\textunderscore  + \textunderscore odous\textunderscore , \textunderscore odontos\textunderscore )}
\end{itemize}
Que tem os dentes inseridos na espessura das maxilas.
\section{Gneis}
\begin{itemize}
\item {Grp. gram.:m.}
\end{itemize}
\begin{itemize}
\item {Proveniência:(T. al.)}
\end{itemize}
Rocha, composta de feldspatho e mica, e que só differe do granito na textura, a qual, em vez de sêr massiça, é xistoide.
\section{Gneiss}
\begin{itemize}
\item {Grp. gram.:m.}
\end{itemize}
\begin{itemize}
\item {Proveniência:(T. al.)}
\end{itemize}
Rocha, composta de feldspatho e mica, e que só differe do granito na textura, a qual, em vez de sêr massiça, é xistoide.
\section{Gneissoide}
\begin{itemize}
\item {Grp. gram.:adj.}
\end{itemize}
\begin{itemize}
\item {Utilização:Geol.}
\end{itemize}
Diz-se do granito, em que as partículas de mica tendem a dispor-se como no gneiss.
\section{Gnetáceas}
\begin{itemize}
\item {Grp. gram.:f. pl.}
\end{itemize}
Família de plantas, que têm por typo o gneto.
\section{Gneto}
\begin{itemize}
\item {Grp. gram.:m.}
\end{itemize}
Arvore das Molucas, (\textunderscore gnetum\textunderscore ).
\section{Gnídia}
\begin{itemize}
\item {Grp. gram.:f.}
\end{itemize}
O mesmo que \textunderscore daphne\textunderscore .
\section{Gnoma}
\begin{itemize}
\item {Grp. gram.:f.}
\end{itemize}
\begin{itemize}
\item {Proveniência:(Do gr. \textunderscore gnome\textunderscore )}
\end{itemize}
Adágio; sentença moral.
\section{Gnómico}
\begin{itemize}
\item {Grp. gram.:adj.}
\end{itemize}
Relativo a gnoma.
\section{Gnomo}
\begin{itemize}
\item {Grp. gram.:m.}
\end{itemize}
\begin{itemize}
\item {Proveniência:(Do gr. \textunderscore gnome\textunderscore ?)}
\end{itemize}
Espírito, que, segundo os cabalistas, preside á Terra e a tudo que ella encerra, como as ondinas á água, os silfos ao ar, e as salamandras ao fogo.
\section{Gnomo}
\begin{itemize}
\item {Grp. gram.:m.}
\end{itemize}
O mesmo que \textunderscore gnómon\textunderscore .
\section{Gnomo}
\begin{itemize}
\item {Grp. gram.:m.}
\end{itemize}
\begin{itemize}
\item {Proveniência:(Gr. \textunderscore gnomon\textunderscore )}
\end{itemize}
Ponteiro ou qualquer instrumento, que marque a altura do Sol pela direcção da sombra.
\section{Gnomologia}
\begin{itemize}
\item {Grp. gram.:f.}
\end{itemize}
\begin{itemize}
\item {Proveniência:(De \textunderscore gnomólogo\textunderscore )}
\end{itemize}
Philosophia sentenciosa.
\section{Gnomológico}
\begin{itemize}
\item {Grp. gram.:adj.}
\end{itemize}
Relativo a gnomologia.
\section{Gnomólogo}
\begin{itemize}
\item {Grp. gram.:m.}
\end{itemize}
\begin{itemize}
\item {Proveniência:(Do gr. \textunderscore gnome\textunderscore  + \textunderscore logos\textunderscore )}
\end{itemize}
Aquelle que discorre ou escreve sentenciosamente.
\section{Gnómon}
\begin{itemize}
\item {Grp. gram.:m.}
\end{itemize}
\begin{itemize}
\item {Proveniência:(Gr. \textunderscore gnomon\textunderscore )}
\end{itemize}
Ponteiro ou qualquer instrumento, que marque a altura do Sol pela direcção da sombra.
\section{Gnomónica}
\begin{itemize}
\item {Grp. gram.:f.}
\end{itemize}
\begin{itemize}
\item {Proveniência:(De \textunderscore gnomónico\textunderscore )}
\end{itemize}
Arte de construir gnómons.
\section{Gnomónico}
\begin{itemize}
\item {Grp. gram.:adj.}
\end{itemize}
Relativo aos gnómones.
\section{Gnomonista}
\begin{itemize}
\item {Grp. gram.:m.}
\end{itemize}
Aquelle que se occupa de gnomónica ou que escreve á cêrca della.
\section{Gnose}
\begin{itemize}
\item {Grp. gram.:f.}
\end{itemize}
\begin{itemize}
\item {Proveniência:(Gr. \textunderscore gnosis\textunderscore )}
\end{itemize}
Sciência superior ás crenças vulgares.
Saber, por excellência; gnosticismo.
\section{Gnosímaco}
\begin{itemize}
\item {Grp. gram.:m.}
\end{itemize}
\begin{itemize}
\item {Proveniência:(Do gr. \textunderscore gnosis\textunderscore  + \textunderscore makestai\textunderscore )}
\end{itemize}
Membro de uma seita do século VII, a qual rejeitava todos os conhecimentos religiosos e fazia consistir a religião na prática das bôas obras.
\section{Gnóssio}
\begin{itemize}
\item {Grp. gram.:adj.}
\end{itemize}
\begin{itemize}
\item {Utilização:Ext.}
\end{itemize}
Relativo a Gnosso, antiga capital de Creta.
Relativo a Creta:«\textunderscore poisam na gnóssia praia...\textunderscore »Castilho, \textunderscore Fastos\textunderscore , III, 73.--Castilho ou o seu secretário escreveu \textunderscore gnóxia\textunderscore , certamente por equívoco.
\section{Gnosticismo}
\begin{itemize}
\item {Grp. gram.:m.}
\end{itemize}
\begin{itemize}
\item {Proveniência:(De \textunderscore gnóstico\textunderscore )}
\end{itemize}
Systema theológico e philosóphico, cujos partidários diziam têr um conhecimento sublime da natureza e attributos de Deus.
\section{Gnóstico}
\begin{itemize}
\item {Grp. gram.:m.}
\end{itemize}
\begin{itemize}
\item {Proveniência:(Gr. \textunderscore gnostikos\textunderscore )}
\end{itemize}
Hereje, partidário do gnosticismo.
\section{Gnugnu}
\begin{itemize}
\item {Grp. gram.:m.}
\end{itemize}
Árvore americana, especialmente do México e da Califórnia, semelhante á mancenilheira e cujos frutos são mortíferos.
\section{Gôa}
\begin{itemize}
\item {Grp. gram.:f.}
\end{itemize}
\begin{itemize}
\item {Utilização:Ant.}
\end{itemize}
\begin{itemize}
\item {Proveniência:(De \textunderscore Gôa\textunderscore , n. p.)}
\end{itemize}
Medida de três palmos chamados \textunderscore de gôa\textunderscore , e usada antigamente pelos constructores de naus. Cf. Fern. Oliveira, \textunderscore Livro da Fábr. das Naus\textunderscore .
\textunderscore Palmo de Gôa\textunderscore , medida antiga, igual a um palmo ordinário e mais uma pollegada.
\section{Goacapi}
\begin{itemize}
\item {Grp. gram.:m.}
\end{itemize}
\begin{itemize}
\item {Utilização:Bras}
\end{itemize}
Cada um dos paus, sôbre que se constrói o girau.
\section{Goacari}
\begin{itemize}
\item {Grp. gram.:m.}
\end{itemize}
Peixe fluvial do Brasil.
\section{Goajuru}
\begin{itemize}
\item {Grp. gram.:m.}
\end{itemize}
Arvore sertaneja do Brasil.
\section{Goananá}
\begin{itemize}
\item {Grp. gram.:m.}
\end{itemize}
Ave palmípede do Brasil.
\section{Goanhambigue}
\begin{itemize}
\item {Grp. gram.:m.}
\end{itemize}
Formosa ave do Brasil.
\section{Goano}
\begin{itemize}
\item {Grp. gram.:m.  e  adj.}
\end{itemize}
O mesmo que \textunderscore goense\textunderscore .
\section{Gobelim}
\begin{itemize}
\item {Grp. gram.:m.}
\end{itemize}
\begin{itemize}
\item {Proveniência:(De \textunderscore Gobelins\textunderscore , n. p.)}
\end{itemize}
Tapeçaria rica, fabricada em Paris.
\section{Gobelino}
\begin{itemize}
\item {Grp. gram.:adj.}
\end{itemize}
Diz-se de uma espécie de ponto, em rendaria.
(Cp. \textunderscore gobelim\textunderscore )
\section{Gobião}
\begin{itemize}
\item {Grp. gram.:m.}
\end{itemize}
\begin{itemize}
\item {Proveniência:(Do gr. \textunderscore kobios\textunderscore )}
\end{itemize}
Peixe malacopterýgio, o mesmo que \textunderscore cadoz\textunderscore .
\section{Gobioides}
\begin{itemize}
\item {Grp. gram.:m. pl.}
\end{itemize}
\begin{itemize}
\item {Proveniência:(Do gr. \textunderscore kobios\textunderscore  + \textunderscore eidos\textunderscore )}
\end{itemize}
Gênero de peixes, que têm por typo o gobião.
\section{Gobo}
\begin{itemize}
\item {fónica:gô}
\end{itemize}
\begin{itemize}
\item {Grp. gram.:m.}
\end{itemize}
\begin{itemize}
\item {Proveniência:(It. \textunderscore gobbo\textunderscore . Cp. \textunderscore godo\textunderscore ^2 e \textunderscore gódo\textunderscore . Serão a mesma coisa?)}
\end{itemize}
Calhau, pedra para calcetar.
\section{Gocete}
\begin{itemize}
\item {fónica:cê}
\end{itemize}
\begin{itemize}
\item {Grp. gram.:m.}
\end{itemize}
\begin{itemize}
\item {Utilização:Ant.}
\end{itemize}
Peça da armadura, que se ajustava debaixo dos braços.
\section{Gocha}
\begin{itemize}
\item {fónica:gô}
\end{itemize}
\begin{itemize}
\item {Grp. gram.:f.}
\end{itemize}
\begin{itemize}
\item {Utilização:Prov.}
\end{itemize}
\begin{itemize}
\item {Utilização:alg.}
\end{itemize}
Paveia de mato, que se junta nos terrenos, para se queimar, e adubarem-se as terras com a sua cinza.
\section{Goche}
\begin{itemize}
\item {Grp. gram.:adj.}
\end{itemize}
\begin{itemize}
\item {Proveniência:(Do fr. \textunderscore gauche\textunderscore )}
\end{itemize}
O mesmo que \textunderscore gocho\textunderscore ^1. Cf. Camillo, \textunderscore Corja\textunderscore , 201.
\section{Gocho}
\begin{itemize}
\item {fónica:gô}
\end{itemize}
\begin{itemize}
\item {Grp. gram.:m.  e  adj.}
\end{itemize}
\begin{itemize}
\item {Utilização:Ant.}
\end{itemize}
\begin{itemize}
\item {Utilização:Prov.}
\end{itemize}
\begin{itemize}
\item {Utilização:trasm.}
\end{itemize}
Desajeitado?:«\textunderscore o gocho luta por pés\textunderscore ». Simão Mach., 69 v.^o
Que vê pouco, precisando fechar um tanto as pálpebras, para distinguir os objectos.
(Provavelmente, do fr. \textunderscore gauche\textunderscore )
\section{Gocho}
\begin{itemize}
\item {fónica:gô}
\end{itemize}
\begin{itemize}
\item {Grp. gram.:m.}
\end{itemize}
\begin{itemize}
\item {Utilização:Prov.}
\end{itemize}
\begin{itemize}
\item {Utilização:trasm.}
\end{itemize}
O mesmo que \textunderscore pescoço\textunderscore .
\section{Godalha}
\begin{itemize}
\item {Grp. gram.:f.}
\end{itemize}
\begin{itemize}
\item {Utilização:Prov.}
\end{itemize}
\begin{itemize}
\item {Utilização:trasm.}
\end{itemize}
\begin{itemize}
\item {Utilização:Ext.}
\end{itemize}
\begin{itemize}
\item {Proveniência:(De \textunderscore gódia\textunderscore ?)}
\end{itemize}
Cabra nova e muito inquieta.
Rapariga leviana.
\section{Gòdé}
\begin{itemize}
\item {Grp. gram.:m.}
\end{itemize}
\begin{itemize}
\item {Proveniência:(Fr. \textunderscore godet\textunderscore )}
\end{itemize}
Tigelinha, em que se desfaz a tinta, para o desenho da aguarela.
\section{Gode}
\begin{itemize}
\item {Grp. gram.:m.}
\end{itemize}
\begin{itemize}
\item {Utilização:Prov.}
\end{itemize}
\begin{itemize}
\item {Utilização:minh.}
\end{itemize}
Pedra rolada, o mesmo que \textunderscore gógo\textunderscore  e que \textunderscore gódo\textunderscore ^2.
\section{Godenho}
\begin{itemize}
\item {Grp. gram.:m.}
\end{itemize}
Casta de uva.
\section{Goderim}
\begin{itemize}
\item {Grp. gram.:m.}
\end{itemize}
Colcha estofada da Índia.
\section{Godétia}
\begin{itemize}
\item {Grp. gram.:f.}
\end{itemize}
\begin{itemize}
\item {Proveniência:(De \textunderscore Godet\textunderscore , n. p.)}
\end{itemize}
Gênero de plantas onagrárias, procedentes da Califórnia e do Chile.
\section{Gódia}
\begin{itemize}
\item {Grp. gram.:f.}
\end{itemize}
\begin{itemize}
\item {Utilização:Prov.}
\end{itemize}
\begin{itemize}
\item {Utilização:trasm.}
\end{itemize}
Bulha; desavença; altercação.
\section{Godião}
\begin{itemize}
\item {Grp. gram.:m.}
\end{itemize}
Peixe de Portugal.
\section{Godilhão}
\begin{itemize}
\item {Grp. gram.:m.}
\end{itemize}
Nó, formado de fios empastados.
Grumo, que se fórma na calda ou na farinha.
\section{Godilho}
\begin{itemize}
\item {Grp. gram.:m.}
\end{itemize}
Variedade de uva branca algarvia.
\section{Godo}
\begin{itemize}
\item {Grp. gram.:adj.}
\end{itemize}
\begin{itemize}
\item {Grp. gram.:M. pl.}
\end{itemize}
\begin{itemize}
\item {Proveniência:(Do lat. \textunderscore goti\textunderscore )}
\end{itemize}
Relativo aos Godos ou á Góthia.
Antigos povos germânicos.
\section{Godo}
\begin{itemize}
\item {Grp. gram.:m.}
\end{itemize}
\begin{itemize}
\item {Utilização:Prov.}
\end{itemize}
\begin{itemize}
\item {Utilização:minh.}
\end{itemize}
Pequeno seixo rolado.
(Cp. \textunderscore gôgo\textunderscore )
\section{Godorim}
\begin{itemize}
\item {Grp. gram.:m.}
\end{itemize}
\begin{itemize}
\item {Utilização:Ant.}
\end{itemize}
O mesmo que \textunderscore goderim\textunderscore .
\section{Godrim}
\begin{itemize}
\item {Grp. gram.:m.}
\end{itemize}
\begin{itemize}
\item {Utilização:Ant.}
\end{itemize}
O mesmo que \textunderscore goderim\textunderscore .
\section{Goeirana}
\begin{itemize}
\item {Grp. gram.:f.}
\end{itemize}
\begin{itemize}
\item {Utilização:Bras}
\end{itemize}
Árvore silvestre, cuja madeira se emprega em caixaria.
\section{Goéla}
\begin{itemize}
\item {Grp. gram.:f.}
\end{itemize}
\begin{itemize}
\item {Grp. gram.:Pl.}
\end{itemize}
\begin{itemize}
\item {Utilização:Pop.}
\end{itemize}
\begin{itemize}
\item {Proveniência:(Do cast. \textunderscore goliella\textunderscore )}
\end{itemize}
Garganta.
Entrada dos canaes, que põem em communicação a bôca com o estômago e os pulmões.
Goélas de quem engole grandes bocados.
\section{Goéla-de-leão}
\begin{itemize}
\item {Grp. gram.:f.}
\end{itemize}
\begin{itemize}
\item {Utilização:Bras. do N}
\end{itemize}
O mesmo que \textunderscore mimo-de-vênus\textunderscore .
\section{Goéla-de-pato}
\begin{itemize}
\item {Grp. gram.:f.}
\end{itemize}
Planta euphorbiácea do Brasil.
\section{Goelar}
\begin{itemize}
\item {fónica:go-e}
\end{itemize}
\begin{itemize}
\item {Grp. gram.:v. i.}
\end{itemize}
\begin{itemize}
\item {Proveniência:(De \textunderscore goéla\textunderscore )}
\end{itemize}
Gritar.
Falar muito.
\section{Goense}
\begin{itemize}
\item {Grp. gram.:adj.}
\end{itemize}
\begin{itemize}
\item {Grp. gram.:M.}
\end{itemize}
Relativo a Gôa.
Aquelle que nasceu em Gôa.
\section{Góes}
\begin{itemize}
\item {Grp. gram.:m.}
\end{itemize}
(?):«\textunderscore ...foi metendo a galé tanto de ló, que fez do penão goes\textunderscore ». Couto, \textunderscore Déc.\textunderscore  VII, 8.
\section{Goês}
\begin{itemize}
\item {Grp. gram.:m.  e  adj.}
\end{itemize}
O mesmo que \textunderscore goense\textunderscore .
\section{Goesiano}
\begin{itemize}
\item {Grp. gram.:adj.}
\end{itemize}
Relativo a Damião do Góes ou ás suas obras.
\section{Goétea}
\begin{itemize}
\item {Grp. gram.:f.}
\end{itemize}
Gênero de plantas malváceas.
\section{Goéthea}
\begin{itemize}
\item {Grp. gram.:f.}
\end{itemize}
Gênero de plantas malváceas.
\section{Gofé}
\begin{itemize}
\item {Grp. gram.:m.}
\end{itemize}
Árvore medicinal da Índia e da ilha de San-Thomé.
\section{Gofrador}
\begin{itemize}
\item {Grp. gram.:m.}
\end{itemize}
Instrumento para gofrar.
\section{Gofradura}
\begin{itemize}
\item {Grp. gram.:f.}
\end{itemize}
Acto ou effeito de gofrar.
\section{Gofrante}
\begin{itemize}
\item {Grp. gram.:m.}
\end{itemize}
\begin{itemize}
\item {Proveniência:(De \textunderscore gofrar\textunderscore )}
\end{itemize}
Parte superior do gofrador.
\section{Gofrar}
\begin{itemize}
\item {Grp. gram.:v. t.}
\end{itemize}
Fazer a nervura de (fôlhas ou flôres artificiaes).
(Cast. \textunderscore gofrar\textunderscore )
\section{Gogada}
\begin{itemize}
\item {Grp. gram.:f.}
\end{itemize}
\begin{itemize}
\item {Utilização:Prov.}
\end{itemize}
\begin{itemize}
\item {Utilização:trasm.}
\end{itemize}
Pancada com o seixo que chamam gógo.
\section{Gogo}
\begin{itemize}
\item {Grp. gram.:m.}
\end{itemize}
O mesmo que \textunderscore gosma\textunderscore .
\section{Gõgo}
\begin{itemize}
\item {Grp. gram.:m.}
\end{itemize}
\begin{itemize}
\item {Utilização:Prov.}
\end{itemize}
\begin{itemize}
\item {Utilização:trasm.}
\end{itemize}
Seixo liso, sôbre que os sapateiros batem a sola.
Pedra pequena e redonda, rolada pelas águas.
Pedra oval, encravada no fundo do rodízio e que gira em cima da ran.
(Cp. \textunderscore gobo\textunderscore , e \textunderscore godo\textunderscore ^2)
\section{Gôgo}
\begin{itemize}
\item {Grp. gram.:m.}
\end{itemize}
\begin{itemize}
\item {Utilização:Prov.}
\end{itemize}
\begin{itemize}
\item {Utilização:trasm.}
\end{itemize}
Seixo liso, sôbre que os sapateiros batem a sola.
Pedra pequena e redonda, rolada pelas águas.
Pedra oval, encravada no fundo do rodízio e que gira em cima da ran.
(Cp. \textunderscore gobo\textunderscore , e \textunderscore godo\textunderscore ^2)
\section{Gògó}
\begin{itemize}
\item {Grp. gram.:m.}
\end{itemize}
Árvore da ilha de San-Thomé.
\section{Gògó}
\begin{itemize}
\item {Grp. gram.:m.}
\end{itemize}
\begin{itemize}
\item {Utilização:Bras}
\end{itemize}
Saliência na parte anterior do pescoço, formada pela cartilagem tymoide; pomo de Adão.
\section{Goguento}
\begin{itemize}
\item {Grp. gram.:adj}
\end{itemize}
Atacado de gogo.
Gosmento.
\section{Goiaba}
\begin{itemize}
\item {Grp. gram.:f.}
\end{itemize}
\begin{itemize}
\item {Proveniência:(Do guar. \textunderscore cuiapa\textunderscore )}
\end{itemize}
Fruto de goiabeira; goiabeira.
\section{Goiabada}
\begin{itemize}
\item {Grp. gram.:f.}
\end{itemize}
Doce de goiaba.
\section{Goiabeira}
\begin{itemize}
\item {Grp. gram.:f.}
\end{itemize}
Árvore myrtácea da América e da África, (\textunderscore psidium guayava\textunderscore , Raddi).
\section{Goiabeirana}
\begin{itemize}
\item {Grp. gram.:f.}
\end{itemize}
Espécie de goiabeira.
\section{Goiamum}
\begin{itemize}
\item {Grp. gram.:m.}
\end{itemize}
Crustáceo azulado e de sabor agradável, parecido com o caranguejo.
\section{Goiano}
\begin{itemize}
\item {Grp. gram.:adj.}
\end{itemize}
\begin{itemize}
\item {Grp. gram.:M.}
\end{itemize}
Relativo ao Estado de Goiás, no Brasil.
Habitante de Goiás.
\section{Goiar}
\begin{itemize}
\item {Grp. gram.:v. i.}
\end{itemize}
\begin{itemize}
\item {Utilização:Ant.}
\end{itemize}
Dar ais; gemer.
(Por \textunderscore guaiar\textunderscore , de \textunderscore guai?\textunderscore )
\section{Goiás}
\begin{itemize}
\item {Grp. gram.:m. pl.}
\end{itemize}
O mesmo ou melhor que \textunderscore goiases\textunderscore .
\section{Goiás}
\begin{itemize}
\item {Grp. gram.:m.}
\end{itemize}
\begin{itemize}
\item {Utilização:Bras}
\end{itemize}
Espécie de caranguejo, que tem carne saborosa.
\section{Goiases}
\begin{itemize}
\item {Grp. gram.:m. pl.}
\end{itemize}
Extincta nação de índios do Brasil, em Goiás.
\section{Goidélico}
\begin{itemize}
\item {Grp. gram.:adj.}
\end{itemize}
Diz-se de um dos grupos das línguas célticas.
\section{Goilão}
\begin{itemize}
\item {Grp. gram.:m.}
\end{itemize}
\begin{itemize}
\item {Utilização:T. de Turquel}
\end{itemize}
Homem alto e desajeitado; trangalhadanças.
(Por \textunderscore goelão\textunderscore , de \textunderscore goéla?\textunderscore )
\section{Goitacazes}
\begin{itemize}
\item {Grp. gram.:m. pl.}
\end{itemize}
Antiga nação de índios do Brasil, de que restam algumas famílias dispersas ao sul do Estado do Espírito-Santo.
\section{Goiti}
\begin{itemize}
\item {Grp. gram.:m.}
\end{itemize}
Árvore fructifera dos sertões do Brasil.
\section{Goiva}
\begin{itemize}
\item {Grp. gram.:f.}
\end{itemize}
\begin{itemize}
\item {Utilização:Ant.}
\end{itemize}
\begin{itemize}
\item {Utilização:Prov.}
\end{itemize}
\begin{itemize}
\item {Utilização:extrem.}
\end{itemize}
Espécie do formão, para lavrar meias canas côncavas.
Agulha, com que o artilheiro desimpedia o ouvido da peça.
Leito de corrente, fundo e estreito: \textunderscore alguns mouchões do Tejo são separados por goivas\textunderscore .
(B. lat. \textunderscore gubia\textunderscore , talvez do lat. \textunderscore cavea\textunderscore )
\section{Goivado}
\begin{itemize}
\item {Grp. gram.:m.}
\end{itemize}
\begin{itemize}
\item {Utilização:Náut.}
\end{itemize}
\begin{itemize}
\item {Proveniência:(De \textunderscore goivar\textunderscore )}
\end{itemize}
Cavidade, em fórma de meia cana, para aguentar a alça, numa peça de poleame.
\section{Goivadura}
\begin{itemize}
\item {Grp. gram.:f.}
\end{itemize}
\begin{itemize}
\item {Proveniência:(De \textunderscore goivar\textunderscore )}
\end{itemize}
Entalhe, feito com goiva.
\section{Goivar}
\begin{itemize}
\item {Grp. gram.:v. t.}
\end{itemize}
\begin{itemize}
\item {Utilização:Ext.}
\end{itemize}
Cortar com goiva.
Ferir muito:«\textunderscore sete goivas goivaram a Virgem Maria.\textunderscore »Th. Braga, \textunderscore Povo Português\textunderscore , II, 146.
\section{Goivaria}
\begin{itemize}
\item {Grp. gram.:f.}
\end{itemize}
\begin{itemize}
\item {Utilização:Prov.}
\end{itemize}
Jardim de goivos.
\section{Goiveiro}
\begin{itemize}
\item {Grp. gram.:m.}
\end{itemize}
\begin{itemize}
\item {Proveniência:(De \textunderscore goivo\textunderscore )}
\end{itemize}
Nome de várias plantas crucíferas.
\section{Goivete}
\begin{itemize}
\item {fónica:vê}
\end{itemize}
\begin{itemize}
\item {Grp. gram.:m.}
\end{itemize}
\begin{itemize}
\item {Proveniência:(De \textunderscore goiva\textunderscore )}
\end{itemize}
Espécie de plaina com dois ferros.
\section{Goivir}
\begin{itemize}
\item {Grp. gram.:v. i.}
\end{itemize}
\begin{itemize}
\item {Utilização:Ant.}
\end{itemize}
O mesmo que \textunderscore gozar\textunderscore .
(Cp. lat. \textunderscore gaudere\textunderscore )
\section{Goivo}
\begin{itemize}
\item {Grp. gram.:m.}
\end{itemize}
\begin{itemize}
\item {Utilização:Ant.}
\end{itemize}
\begin{itemize}
\item {Proveniência:(Do lat. \textunderscore gaudium\textunderscore )}
\end{itemize}
Flôr do goiveiro.
Goiveiro.
O mesmo que \textunderscore gáudio\textunderscore .
\section{Goja}
\begin{itemize}
\item {fónica:gô}
\end{itemize}
\begin{itemize}
\item {Grp. gram.:f.}
\end{itemize}
\begin{itemize}
\item {Utilização:Prov.}
\end{itemize}
\begin{itemize}
\item {Utilização:trasm.}
\end{itemize}
O mesmo que \textunderscore cobra\textunderscore . (Colhido em V. P. de Aguiar)
\section{Gójo}
\begin{itemize}
\item {Grp. gram.:m.}
\end{itemize}
\begin{itemize}
\item {Utilização:Prov.}
\end{itemize}
\begin{itemize}
\item {Utilização:trasm.}
\end{itemize}
Qualquer animal pequeno, (gallinha, pomba, coêlho, etc).
\section{Gôjo}
\begin{itemize}
\item {Grp. gram.:m.}
\end{itemize}
\begin{itemize}
\item {Utilização:Prov.}
\end{itemize}
\begin{itemize}
\item {Utilização:trasm.}
\end{itemize}
Qualquer animal pequeno, (gallinha, pomba, coêlho, etc).
\section{Gola}
\begin{itemize}
\item {Grp. gram.:f.}
\end{itemize}
\begin{itemize}
\item {Utilização:Prov.}
\end{itemize}
\begin{itemize}
\item {Utilização:beir.}
\end{itemize}
\begin{itemize}
\item {Utilização:Prov.}
\end{itemize}
\begin{itemize}
\item {Utilização:trasm.}
\end{itemize}
Espécie de redemoínho, que se fórma nos pegos dos rios ou ribeiras.
O mesmo que \textunderscore goéla\textunderscore .
\section{Gola}
\begin{itemize}
\item {Grp. gram.:f.}
\end{itemize}
\begin{itemize}
\item {Utilização:Prov.}
\end{itemize}
\begin{itemize}
\item {Utilização:trasm.}
\end{itemize}
\begin{itemize}
\item {Proveniência:(Do lat. \textunderscore collum\textunderscore )}
\end{itemize}
Parte do vestuário, junto ao pescoço ou em volta dele.
Colarinho.
Linha ou espaço entre os lados de um ângulo saliente, nas fortificações.
Moldura, cuja superfície é, em parte, côncava, e, em parte, convexa.
Goéla, garganta. (Colhido em V. P. de Aguiar.)
\section{Golada}
\begin{itemize}
\item {Grp. gram.:f.}
\end{itemize}
Canal de navegação, no extremo dos bancos de areia de uma barra, pelo qual podem passar pequenas embarcações.
(Cp. \textunderscore goleta\textunderscore ^1)
\section{Golada}
\begin{itemize}
\item {Grp. gram.:f.}
\end{itemize}
\begin{itemize}
\item {Utilização:Pop.}
\end{itemize}
\begin{itemize}
\item {Proveniência:(De \textunderscore gole\textunderscore )}
\end{itemize}
Um pouco de vinho ou de outra bebida alcoólica.
\section{Golango}
\begin{itemize}
\item {Grp. gram.:m.}
\end{itemize}
Espécie de antílope, na África.
\section{Golangômbia}
\begin{itemize}
\item {Grp. gram.:f.}
\end{itemize}
Pássaro dentirostro de Benguela.
\section{Golar}
\begin{itemize}
\item {Grp. gram.:v. t.  e  i.}
\end{itemize}
\begin{itemize}
\item {Utilização:Pop.}
\end{itemize}
O mesmo que \textunderscore gorar\textunderscore .
\section{Golazeira}
\begin{itemize}
\item {Grp. gram.:f.}
\end{itemize}
(V.gorazeira)
\section{Golcori}
\begin{itemize}
\item {Grp. gram.:m.}
\end{itemize}
Jóia, com que as indianas adornam o pescoço. Cf. Th. Ribeiro, \textunderscore Jornadas\textunderscore , II, 104.
\section{Goldbáchia}
\begin{itemize}
\item {fónica:qui}
\end{itemize}
\begin{itemize}
\item {Grp. gram.:f.}
\end{itemize}
\begin{itemize}
\item {Proveniência:(De \textunderscore Goldbach\textunderscore , n. p.)}
\end{itemize}
Gênero de plantas crucíferas.
\section{Goldfússia}
\begin{itemize}
\item {Grp. gram.:f.}
\end{itemize}
Gênero de plantas acantháceas.
\section{Goldbáquia}
\begin{itemize}
\item {Grp. gram.:f.}
\end{itemize}
\begin{itemize}
\item {Proveniência:(De \textunderscore Goldbach\textunderscore , n. p.)}
\end{itemize}
Gênero de plantas crucíferas.
\section{Gole}
\begin{itemize}
\item {Grp. gram.:m.}
\end{itemize}
Porção de líquido, que se engole de uma vez; trago.
(Refl. de \textunderscore engulir\textunderscore )
\section{Golelha}
\begin{itemize}
\item {fónica:lê}
\end{itemize}
\begin{itemize}
\item {Grp. gram.:f.}
\end{itemize}
\begin{itemize}
\item {Utilização:Fam.}
\end{itemize}
O mesmo que \textunderscore esóphago\textunderscore .
(Provavelmente do cast. \textunderscore goliella\textunderscore )
\section{Golelhar}
\begin{itemize}
\item {Grp. gram.:v. i.}
\end{itemize}
\begin{itemize}
\item {Utilização:Fam.}
\end{itemize}
\begin{itemize}
\item {Proveniência:(De \textunderscore golelha\textunderscore )}
\end{itemize}
Tagarelar; dar á língua.
\section{Golelheiro}
\begin{itemize}
\item {Grp. gram.:m.  e  adj.}
\end{itemize}
\begin{itemize}
\item {Proveniência:(De \textunderscore golelha\textunderscore )}
\end{itemize}
Mexeriqueiro; tagarela; palrador.
\section{Goles}
\begin{itemize}
\item {Grp. gram.:m. pl.}
\end{itemize}
\begin{itemize}
\item {Utilização:Heráld.}
\end{itemize}
\begin{itemize}
\item {Proveniência:(Fr. \textunderscore gueules\textunderscore )}
\end{itemize}
A côr vermelha, nos brasões.
\section{Goleta}
\begin{itemize}
\item {fónica:lê}
\end{itemize}
\begin{itemize}
\item {Grp. gram.:f.}
\end{itemize}
\begin{itemize}
\item {Proveniência:(De \textunderscore gola\textunderscore )}
\end{itemize}
Angra.
Pequena barra.
Canal, que dá accesso a um porto.
\section{Goleta}
\begin{itemize}
\item {fónica:lê}
\end{itemize}
\begin{itemize}
\item {Grp. gram.:f.}
\end{itemize}
Pequena escuna espanhola, de gávea á prôa.
(Cast. \textunderscore goleta\textunderscore )
\section{Goleta}
\begin{itemize}
\item {fónica:lê}
\end{itemize}
\begin{itemize}
\item {Grp. gram.:f.}
\end{itemize}
O mesmo que \textunderscore golada\textunderscore ^2.
\section{Golfada}
\begin{itemize}
\item {Grp. gram.:f.}
\end{itemize}
\begin{itemize}
\item {Utilização:Fig.}
\end{itemize}
\begin{itemize}
\item {Proveniência:(De \textunderscore golfar\textunderscore )}
\end{itemize}
Líquido, que sai de um jacto.
Jôrro.
Vómito.
Ímpeto.
\section{Gólfão}
\begin{itemize}
\item {Grp. gram.:m.}
\end{itemize}
\begin{itemize}
\item {Utilização:Des.}
\end{itemize}
O mesmo que \textunderscore golfo\textunderscore . Cf. \textunderscore Peregrinação\textunderscore , LXX.
Planta nympheácea, o mesmo que \textunderscore nenúfar\textunderscore .
Espécie de genciana do Brasil.
\section{Golfar}
\begin{itemize}
\item {Grp. gram.:v. t.}
\end{itemize}
\begin{itemize}
\item {Grp. gram.:V. i.}
\end{itemize}
\begin{itemize}
\item {Proveniência:(De \textunderscore gôlfo\textunderscore )}
\end{itemize}
Expellir em golfadas.
Jorrar.
Vomitar.
Arremessar em grande quantidade.
Expedir.
Correr em goladas.
Expellir impetuosamente um líquido.
Irromper impetuosamente.
\section{Golfejar}
\begin{itemize}
\item {Grp. gram.:v. i.}
\end{itemize}
\begin{itemize}
\item {Proveniência:(De \textunderscore gôlfo\textunderscore )}
\end{itemize}
Golfar repetidas vezes.
\section{Golfim}
\begin{itemize}
\item {Grp. gram.:m.}
\end{itemize}
O mesmo que \textunderscore golfinho\textunderscore .
\section{Golfim-e-baleia}
\begin{itemize}
\item {Grp. gram.:m.}
\end{itemize}
Espécie de jôgo popular.
\section{Golfinho}
\begin{itemize}
\item {Grp. gram.:m.}
\end{itemize}
\begin{itemize}
\item {Utilização:Heráld.}
\end{itemize}
\begin{itemize}
\item {Utilização:Ant.}
\end{itemize}
\begin{itemize}
\item {Utilização:Gír.}
\end{itemize}
\begin{itemize}
\item {Proveniência:(Do lat. \textunderscore delphinus\textunderscore )}
\end{itemize}
Grande peixe marítimo, carnívoro, da fam. dos cetáceos.
Representação dêste animal em armaria.
Cada uma das asas da peça de artilharia.
Corcunda.
\section{Golfo}
\begin{itemize}
\item {fónica:gôl}
\end{itemize}
\begin{itemize}
\item {Grp. gram.:m.}
\end{itemize}
\begin{itemize}
\item {Utilização:Prov.}
\end{itemize}
\begin{itemize}
\item {Utilização:alg.}
\end{itemize}
\begin{itemize}
\item {Proveniência:(Do gr. \textunderscore kolpos\textunderscore )}
\end{itemize}
Parte de um mar, que entra nas terras, e cuja abertura do lado do mar é extraordinariamente muito larga.
Peça de ferro, em que giram as missagras das portinholas do navio.
O mesmo que \textunderscore golfada\textunderscore . Cf. Camillo, \textunderscore Brasileira\textunderscore , 78 e 353.
Abysmo, pégo.
Planta, o mesmo que \textunderscore gólfão\textunderscore , nenúfar ou lírio de água.
\section{Gôlfo-amarelo}
\begin{itemize}
\item {Grp. gram.:m.}
\end{itemize}
\begin{itemize}
\item {Utilização:Bot.}
\end{itemize}
Espécie de gólfão.
\section{Gôlfo-branco}
\begin{itemize}
\item {Grp. gram.:m.}
\end{itemize}
\begin{itemize}
\item {Utilização:Bot.}
\end{itemize}
O mesmo que \textunderscore gólfão\textunderscore .
\section{Golgota}
\begin{itemize}
\item {Grp. gram.:m.}
\end{itemize}
\begin{itemize}
\item {Utilização:Fig.}
\end{itemize}
\begin{itemize}
\item {Proveniência:(De \textunderscore Gólgotha\textunderscore , n. p.)}
\end{itemize}
Suplicio atroz.
Lugar de sofrimento.
\section{Gólgotha}
\begin{itemize}
\item {Grp. gram.:m.}
\end{itemize}
\begin{itemize}
\item {Utilização:Fig.}
\end{itemize}
\begin{itemize}
\item {Proveniência:(De \textunderscore Gólgotha\textunderscore , n. p.)}
\end{itemize}
Supplicio atroz.
Lugar de soffrimento.
\section{Golgueira}
\begin{itemize}
\item {Grp. gram.:f.}
\end{itemize}
\begin{itemize}
\item {Utilização:Prov.}
\end{itemize}
\begin{itemize}
\item {Utilização:trasm.}
\end{itemize}
O mesmo que \textunderscore peito\textunderscore .
\section{Golipão}
\begin{itemize}
\item {Grp. gram.:m.}
\end{itemize}
\begin{itemize}
\item {Utilização:T. da Bairrada}
\end{itemize}
\begin{itemize}
\item {Proveniência:(De \textunderscore engole\textunderscore  + \textunderscore pão\textunderscore ?)}
\end{itemize}
Aquelle que come com soffreguidão.
Lambão.
\section{Golipar}
\begin{itemize}
\item {Grp. gram.:v. i.}
\end{itemize}
\begin{itemize}
\item {Utilização:T. da Bairrada}
\end{itemize}
Comer soffregamente, parecendo que engole sem mastigar.
(Cp. \textunderscore golipão\textunderscore )
\section{Golla}
\begin{itemize}
\item {Grp. gram.:f.}
\end{itemize}
\begin{itemize}
\item {Utilização:Prov.}
\end{itemize}
\begin{itemize}
\item {Utilização:trasm.}
\end{itemize}
\begin{itemize}
\item {Proveniência:(Do lat. \textunderscore collum\textunderscore )}
\end{itemize}
Parte do vestuário, junto ao pescoço ou em volta delle.
Collarinho.
Linha ou espaço entre os lados de um ângulo saliente, nas fortificações.
Moldura, cuja superfície é, em parte, côncava, e, em parte, convexa.
Goéla, garganta. (Colhido em V. P. de Aguiar.)
\section{Golar}
\begin{itemize}
\item {Grp. gram.:v. i.}
\end{itemize}
\begin{itemize}
\item {Utilização:Prov.}
\end{itemize}
\begin{itemize}
\item {Utilização:trasm.}
\end{itemize}
\begin{itemize}
\item {Proveniência:(De \textunderscore golas\textunderscore )}
\end{itemize}
Berrar muito.
\section{Golas}
\begin{itemize}
\item {Grp. gram.:f. pl.}
\end{itemize}
\begin{itemize}
\item {Utilização:Prov.}
\end{itemize}
\begin{itemize}
\item {Utilização:trasm.}
\end{itemize}
Goélas.
(Cp. \textunderscore gola\textunderscore ^1)
\section{Goleira}
\begin{itemize}
\item {Grp. gram.:f.}
\end{itemize}
\begin{itemize}
\item {Utilização:Prov.}
\end{itemize}
\begin{itemize}
\item {Proveniência:(De \textunderscore gola\textunderscore ^1)}
\end{itemize}
O mesmo que \textunderscore coleira\textunderscore ^3.
\section{Golinha}
\begin{itemize}
\item {Grp. gram.:f.}
\end{itemize}
\begin{itemize}
\item {Utilização:Ant.}
\end{itemize}
\begin{itemize}
\item {Proveniência:(De \textunderscore gola\textunderscore ^1)}
\end{itemize}
Argola, pregada num poste, á qual se prende alguém pelo pescoço.
Cabeção com volta engomada, que se usava com a beca.
\section{Gollar}
\begin{itemize}
\item {Grp. gram.:v. i.}
\end{itemize}
\begin{itemize}
\item {Utilização:Prov.}
\end{itemize}
\begin{itemize}
\item {Utilização:trasm.}
\end{itemize}
\begin{itemize}
\item {Proveniência:(De \textunderscore gollas\textunderscore )}
\end{itemize}
Berrar muito.
\section{Gollas}
\begin{itemize}
\item {Grp. gram.:f. pl.}
\end{itemize}
\begin{itemize}
\item {Utilização:Prov.}
\end{itemize}
\begin{itemize}
\item {Utilização:trasm.}
\end{itemize}
Goélas.
(Cp. \textunderscore golla\textunderscore )
\section{Golleira}
\begin{itemize}
\item {Grp. gram.:f.}
\end{itemize}
\begin{itemize}
\item {Utilização:Prov.}
\end{itemize}
\begin{itemize}
\item {Proveniência:(De \textunderscore golla\textunderscore )}
\end{itemize}
O mesmo que \textunderscore colleira\textunderscore .
\section{Gollinha}
\begin{itemize}
\item {Grp. gram.:f.}
\end{itemize}
\begin{itemize}
\item {Utilização:Ant.}
\end{itemize}
\begin{itemize}
\item {Proveniência:(De \textunderscore golla\textunderscore )}
\end{itemize}
Argola, pregada num poste, á qual se prende alguém pelo pescoço.
Cabeção com volta engomada, que se usava com a beca.
\section{Golo}
\begin{itemize}
\item {fónica:gô}
\end{itemize}
\begin{itemize}
\item {Grp. gram.:adj.}
\end{itemize}
\begin{itemize}
\item {Utilização:Pop.}
\end{itemize}
O mesmo que \textunderscore gôro\textunderscore .
\section{Golo}
\begin{itemize}
\item {fónica:gô}
\end{itemize}
\begin{itemize}
\item {Grp. gram.:m.}
\end{itemize}
\begin{itemize}
\item {Utilização:Pop.}
\end{itemize}
O mesmo que \textunderscore gole\textunderscore .
\section{Golococo}
\begin{itemize}
\item {fónica:cô}
\end{itemize}
\begin{itemize}
\item {Grp. gram.:m.}
\end{itemize}
Ave de rapina da África occidental.
\section{Golpada}
\begin{itemize}
\item {Grp. gram.:f.}
\end{itemize}
\begin{itemize}
\item {Utilização:Pop.}
\end{itemize}
Grande golpe.
\section{Golpázio}
\begin{itemize}
\item {Grp. gram.:m.}
\end{itemize}
Grande golpe.
\section{Golpe}
\begin{itemize}
\item {Grp. gram.:m.}
\end{itemize}
\begin{itemize}
\item {Utilização:Gír.}
\end{itemize}
\begin{itemize}
\item {Grp. gram.:Loc. adv.}
\end{itemize}
Pancada, dada por um corpo arremessado ou caído.
Ferimento, feito por instrumento cortante ou contundente.
Córte.
Acção de cortar ou atalhar uma difficuldade ou um perigo.
Lance; crise.
Gole.
Ímpeto; chofre.
Porção de coisas ou pessôas, que irrompem de uma vez.
Peça de ferro ou de outro metal, onde se firma o braço ou tranqueta da aldrava.
Algibeira.
\textunderscore Gatuno de golpe\textunderscore , ladrão hábil, que, aproveitando ajuntamento de gente ou o descuido de alguém, lhe subtrai o relógio, ou alfinete de peito.
\textunderscore De golpe\textunderscore , subitamente. Cf. Camillo, \textunderscore Retr. de Ricard.\textunderscore , 219 e 270.
(B. lat. \textunderscore colpus\textunderscore )
\section{Golpeante}
\begin{itemize}
\item {Grp. gram.:adj.}
\end{itemize}
Que golpeia.
\section{Golpear}
\begin{itemize}
\item {Grp. gram.:v. t.}
\end{itemize}
\begin{itemize}
\item {Utilização:Fig.}
\end{itemize}
Dar golpes em.
Recortar.
Affligir profundamente.
\section{Golpelha}
\begin{itemize}
\item {fónica:pê}
\end{itemize}
\begin{itemize}
\item {Grp. gram.:f.}
\end{itemize}
\begin{itemize}
\item {Utilização:Ant.}
\end{itemize}
\begin{itemize}
\item {Proveniência:(Do lat. \textunderscore vulpecula\textunderscore )}
\end{itemize}
O mesmo que \textunderscore raposa\textunderscore .
\section{Golpelha}
\begin{itemize}
\item {fónica:pê}
\end{itemize}
\begin{itemize}
\item {Grp. gram.:f.}
\end{itemize}
\begin{itemize}
\item {Proveniência:(Do lat. \textunderscore corbicula\textunderscore )}
\end{itemize}
Grande alcofa.
Alforge de esparto.
\section{Golpelheira}
\begin{itemize}
\item {Grp. gram.:f.}
\end{itemize}
\begin{itemize}
\item {Utilização:Prov.}
\end{itemize}
\begin{itemize}
\item {Proveniência:(De \textunderscore golpelha\textunderscore ^1)}
\end{itemize}
Covil de raposas. (Colhido em Azeméis)
\section{Goma}
\begin{itemize}
\item {Grp. gram.:f.}
\end{itemize}
\begin{itemize}
\item {Utilização:Bras}
\end{itemize}
\begin{itemize}
\item {Proveniência:(Do lat. \textunderscore cumma\textunderscore )}
\end{itemize}
Substância vegetal, translúcida e viscosa.
Substâncias, que se empregam na collagem do vinho.
Tumor siphylítico, de origem terciária.
O mesmo que \textunderscore tapioca\textunderscore .
\section{Goma}
\begin{itemize}
\item {Grp. gram.:f.}
\end{itemize}
\begin{itemize}
\item {Utilização:Prov.}
\end{itemize}
\begin{itemize}
\item {Proveniência:(De \textunderscore gomo\textunderscore )}
\end{itemize}
Feixe de sarmentos secos, que ficaram da poda.
\section{Gomacaxaca}
\begin{itemize}
\item {Grp. gram.:f.}
\end{itemize}
Pássaro dentirostro da África.
\section{Gomadeira}
\begin{itemize}
\item {Grp. gram.:f.}
\end{itemize}
\begin{itemize}
\item {Proveniência:(De \textunderscore gomar\textunderscore ^2)}
\end{itemize}
O mesmo que \textunderscore engomadeira\textunderscore .
\section{Gomado}
\begin{itemize}
\item {Grp. gram.:m.}
\end{itemize}
Planta ampelídea, de Cabo-Verde.
\section{Goma-elástica}
\begin{itemize}
\item {Grp. gram.:f.}
\end{itemize}
O mesmo que \textunderscore borracha\textunderscore ^1.
\section{Gomar}
\begin{itemize}
\item {Grp. gram.:v. i.}
\end{itemize}
\begin{itemize}
\item {Proveniência:(De \textunderscore gomo\textunderscore )}
\end{itemize}
Lançar gomos ou rebentos; abrotar.
\section{Gomar}
\begin{itemize}
\item {Grp. gram.:v. t.}
\end{itemize}
(V.engomar)
\section{Gomarra}
\begin{itemize}
\item {Grp. gram.:f.}
\end{itemize}
\begin{itemize}
\item {Utilização:Prov.}
\end{itemize}
\begin{itemize}
\item {Utilização:trasm.}
\end{itemize}
O mesmo que \textunderscore gallinha\textunderscore .
(Cast. \textunderscore gomarra\textunderscore )
\section{Gomas}
\begin{itemize}
\item {Grp. gram.:f. pl.}
\end{itemize}
\begin{itemize}
\item {Utilização:Prov.}
\end{itemize}
\begin{itemize}
\item {Utilização:trasm.}
\end{itemize}
Lombas, entre os valleiros, na plantação das vinhas.
\section{Gomba}
\begin{itemize}
\item {Grp. gram.:f.}
\end{itemize}
Árvore de Cabínda, própria para mobília e outros artefactos.
\section{Gombo}
\begin{itemize}
\item {Grp. gram.:m.}
\end{itemize}
(Nome que, erradamente, os diccionários dão ao gombô)
\section{Gombô}
\begin{itemize}
\item {Grp. gram.:m.}
\end{itemize}
\begin{itemize}
\item {Utilização:Bras}
\end{itemize}
O mesmo que \textunderscore quiabo\textunderscore .
\section{Gomedar}
\begin{itemize}
\item {Grp. gram.:m.}
\end{itemize}
Espécie de punhal do Oriente:«\textunderscore ...os adargueiros com suas espadas e gomedares na cinta.\textunderscore »\textunderscore Chrôn. dos Reis de Bisnaga\textunderscore , 28.
\section{Gomeiro}
\begin{itemize}
\item {Grp. gram.:m.}
\end{itemize}
Fabricante ou vendedor de goma.
\section{Gomeleira}
\begin{itemize}
\item {Grp. gram.:f.}
\end{itemize}
\begin{itemize}
\item {Proveniência:(Do rad. de \textunderscore gomo\textunderscore )}
\end{itemize}
Rebento, que nasce junto ao tronco das árvores e lhes rouba a seiva.
\section{Gomenol}
\begin{itemize}
\item {Grp. gram.:m.}
\end{itemize}
Espécie de óleo medicamentoso, contra a tuberculose, neuralgia, etc.
\section{Gomia}
\begin{itemize}
\item {Grp. gram.:f.}
\end{itemize}
O mesmo que \textunderscore agomia\textunderscore ^1.
\section{Gomiada}
\begin{itemize}
\item {Grp. gram.:f.}
\end{itemize}
Golpe, feito com gomia.
\section{Gomificar}
\begin{itemize}
\item {Grp. gram.:v. i.}
\end{itemize}
\begin{itemize}
\item {Utilização:Des.}
\end{itemize}
Transformar-se em goma.
\section{Gomil}
\begin{itemize}
\item {Grp. gram.:m.}
\end{itemize}
Jarro de boca estreita, para água ou para outro líquido.
\section{Gomiloso}
\begin{itemize}
\item {Grp. gram.:adj.}
\end{itemize}
Semelhante ao gomil.
\section{Gomitar}
\begin{itemize}
\item {Grp. gram.:v. t.  e  i.}
\end{itemize}
\begin{itemize}
\item {Utilização:Pop.}
\end{itemize}
O mesmo que \textunderscore vomitar\textunderscore .
\section{Gómito}
\begin{itemize}
\item {Grp. gram.:m.}
\end{itemize}
\begin{itemize}
\item {Utilização:Pop.}
\end{itemize}
O mesmo que \textunderscore vómito\textunderscore .
\section{Gomma}
\textunderscore f.\textunderscore  (e der.)
O mesmo que \textunderscore goma\textunderscore ^1, etc.
\section{Gomo}
\begin{itemize}
\item {Grp. gram.:m.}
\end{itemize}
Rebento dos vegetaes, que se transforma em ramo ou folha.
Cada uma das divisões naturaes de certos frutos, como a laranja.
\section{Gomo-bile}
\begin{itemize}
\item {Grp. gram.:m.}
\end{itemize}
Árvore de Cabinda, própria para construcções navaes.
\section{Gomor}
\begin{itemize}
\item {Grp. gram.:m.}
\end{itemize}
Medida ou vasilha, em que os Hebreus recolhiam o maná. Cf. Vieira, VI, 245.
\section{Gomosidade}
\begin{itemize}
\item {Grp. gram.:f.}
\end{itemize}
Qualidade daquillo que é gomoso.
\section{Gomoso}
\begin{itemize}
\item {Grp. gram.:adj.}
\end{itemize}
Que produz goma.
Viscoso; consistente como a goma.
\section{Gômphia}
\begin{itemize}
\item {Grp. gram.:f.}
\end{itemize}
\begin{itemize}
\item {Proveniência:(Do gr. \textunderscore gomphos\textunderscore )}
\end{itemize}
Gênero de plantas terebintháceas.
\section{Gomphocarpo}
\begin{itemize}
\item {Grp. gram.:m.}
\end{itemize}
\begin{itemize}
\item {Proveniência:(Do gr. \textunderscore gomphos\textunderscore  + \textunderscore karpos\textunderscore )}
\end{itemize}
Gênero de plantas asclepiádeas, cujos frutos são cobertos de pontas.
\section{Gomphose}
\begin{itemize}
\item {Grp. gram.:f.}
\end{itemize}
\begin{itemize}
\item {Utilização:Anat.}
\end{itemize}
\begin{itemize}
\item {Proveniência:(Gr. \textunderscore gomphosis\textunderscore )}
\end{itemize}
Articulação immóvel, como a dos dentes nos alvéolos.
\section{Gomphrena}
\begin{itemize}
\item {Grp. gram.:f.}
\end{itemize}
Designação antiga de um género de plantas amarantáceas.
\section{Gonçala}
\begin{itemize}
\item {Grp. gram.:f.}
\end{itemize}
Casta de uva.
\section{Gonçalinho}
\begin{itemize}
\item {Grp. gram.:m.}
\end{itemize}
\begin{itemize}
\item {Utilização:Prov.}
\end{itemize}
Espécie de alvéloa.
\section{Gonçalo-alves}
\begin{itemize}
\item {Grp. gram.:m.}
\end{itemize}
O mesmo que \textunderscore gurubu\textunderscore .
\section{Gonçalo-pires}
\begin{itemize}
\item {Grp. gram.:m.}
\end{itemize}
Casta de uva preta na região do Doiro.
\section{Gonda}
\begin{itemize}
\item {Grp. gram.:f.}
\end{itemize}
Nome brasileiro de uma planta europeia, resedácea, (\textunderscore reseda luteola\textunderscore ).
\section{Gondão}
\begin{itemize}
\item {Grp. gram.:m.}
\end{itemize}
Arvore de Timor.
\section{Gonde}
\begin{itemize}
\item {Grp. gram.:m.}
\end{itemize}
Língua dravídica, do grupo decânico.
\section{Gondo}
\begin{itemize}
\item {Grp. gram.:m.}
\end{itemize}
Tartaruga de Catumbella, (\textunderscore gysimopus aegyptiacus\textunderscore ).
\section{Gôndola}
\begin{itemize}
\item {Grp. gram.:f.}
\end{itemize}
\begin{itemize}
\item {Utilização:Ant.}
\end{itemize}
\begin{itemize}
\item {Utilização:Bras}
\end{itemize}
\begin{itemize}
\item {Utilização:Bras}
\end{itemize}
\begin{itemize}
\item {Proveniência:(It. \textunderscore gondola\textunderscore )}
\end{itemize}
Pequena embarcação de remos, com as extremidades um pouco levantadas, e que serve especialmente para navegar em canaes.
Barco.
Carro de praça, espécie de pequeno ómnibus.
Véstia de abas curtas.
\section{Gondolar}
\begin{itemize}
\item {Grp. gram.:v. i.}
\end{itemize}
Andar em gôndola. Cf. Camillo, \textunderscore Narcót.\textunderscore , I, 210.
\section{Gondoleiro}
\begin{itemize}
\item {Grp. gram.:m.}
\end{itemize}
Tripulante de gôndola.
\section{Gondonga}
\begin{itemize}
\item {Grp. gram.:f.}
\end{itemize}
Grande antílope da Zambézia, do tamanho de um boi e de carne excellente.
\section{Gonete}
\begin{itemize}
\item {fónica:nê}
\end{itemize}
\begin{itemize}
\item {Grp. gram.:m.}
\end{itemize}
Pua; trado.
(Por \textunderscore gunete\textunderscore , do lat. \textunderscore cuneus\textunderscore ?)
\section{Gonfalão}
\begin{itemize}
\item {Grp. gram.:m.}
\end{itemize}
\begin{itemize}
\item {Utilização:Ant.}
\end{itemize}
O mesmo que \textunderscore bandeira\textunderscore .
\section{Gônfia}
\begin{itemize}
\item {Grp. gram.:f.}
\end{itemize}
\begin{itemize}
\item {Proveniência:(Do gr. \textunderscore gomphos\textunderscore )}
\end{itemize}
Gênero de plantas terebintáceas.
\section{Gonfocarpo}
\begin{itemize}
\item {Grp. gram.:m.}
\end{itemize}
\begin{itemize}
\item {Proveniência:(Do gr. \textunderscore gomphos\textunderscore  + \textunderscore karpos\textunderscore )}
\end{itemize}
Gênero de plantas asclepiádeas, cujos frutos são cobertos de pontas.
\section{Gonfose}
\begin{itemize}
\item {Grp. gram.:f.}
\end{itemize}
\begin{itemize}
\item {Utilização:Anat.}
\end{itemize}
\begin{itemize}
\item {Proveniência:(Gr. \textunderscore gomphosis\textunderscore )}
\end{itemize}
Articulação immóvel, como a dos dentes nos alvéolos.
\section{Gonfrena}
\begin{itemize}
\item {Grp. gram.:f.}
\end{itemize}
Designação antiga de um género de plantas amarantáceas.
\section{Gonga}
\begin{itemize}
\item {Grp. gram.:f.}
\end{itemize}
Ave africana, (\textunderscore niscetus opilogaster\textunderscore ).--Provavelmente, é o nome que os diccionaristas dão incorrectamente ao \textunderscore gongá\textunderscore .
\section{Gongá}
\begin{itemize}
\item {Grp. gram.:m.}
\end{itemize}
\begin{itemize}
\item {Utilização:Bras do N}
\end{itemize}
\begin{itemize}
\item {Utilização:Bras do Rio}
\end{itemize}
Espécie de sabiá.
Espécie de pequena cesta, com tampa.
(Do quimbundo \textunderscore gonga\textunderscore )
\section{Gongilango}
\begin{itemize}
\item {Grp. gram.:m.}
\end{itemize}
\begin{itemize}
\item {Utilização:Bot.}
\end{itemize}
\begin{itemize}
\item {Proveniência:(Do gr. \textunderscore gongulos\textunderscore  + \textunderscore angos\textunderscore )}
\end{itemize}
Parte das plantas criptógâmicas, que encerra os corpúsculos reproductores.
\section{Gongilar}
\begin{itemize}
\item {Grp. gram.:adj.}
\end{itemize}
\begin{itemize}
\item {Utilização:Bot.}
\end{itemize}
Relativo aos gôngilos.
Gemíparo.
\section{Gôngilo}
\begin{itemize}
\item {Grp. gram.:m.}
\end{itemize}
\begin{itemize}
\item {Utilização:Bot.}
\end{itemize}
\begin{itemize}
\item {Proveniência:(Gr. \textunderscore gongulos\textunderscore , redondo)}
\end{itemize}
Corpúsculo reproductor de algumas plantas.
\section{Gongo}
\begin{itemize}
\item {Grp. gram.:m.}
\end{itemize}
O mesmo que \textunderscore tam-tam\textunderscore .
\section{Gongo}
\begin{itemize}
\item {Grp. gram.:m.}
\end{itemize}
\begin{itemize}
\item {Utilização:Bras. da Baía}
\end{itemize}
Vara, com um ferro curvo na ponta, e de que os barqueiros se servem, para se segurar aos ramos das árvores das margens dos rios.
\section{Gongó}
\begin{itemize}
\item {Grp. gram.:m.}
\end{itemize}
Arvore africana, de cujo fruto se extrai uma bebida que, depois de fermentada, embriaga. Cf. Capello e Ivens, I, 34.
\section{Gongolô}
\begin{itemize}
\item {Grp. gram.:m.}
\end{itemize}
\begin{itemize}
\item {Utilização:Bras. do N}
\end{itemize}
Pequeno myriápode, que se enrola, quando lhe tocam.
O mesmo que \textunderscore bicho-de-conta\textunderscore ?
\section{Gôngora}
\begin{itemize}
\item {Grp. gram.:f.}
\end{itemize}
\begin{itemize}
\item {Proveniência:(De \textunderscore Gôngora\textunderscore , n. p. de um governador do Peru)}
\end{itemize}
Gênero de orchídeas.
\section{Gongorado}
\begin{itemize}
\item {Grp. gram.:adj.}
\end{itemize}
\begin{itemize}
\item {Utilização:Des.}
\end{itemize}
Em que há gongorismo: \textunderscore phrase gongorada\textunderscore .
\section{Gongoricamente}
\begin{itemize}
\item {Grp. gram.:adv.}
\end{itemize}
De modo gongórico.
\section{Gongórico}
\begin{itemize}
\item {Grp. gram.:adj.}
\end{itemize}
\begin{itemize}
\item {Proveniência:(De \textunderscore Gôngora\textunderscore , n. p.)}
\end{itemize}
Relativo ao gongorismo.
\section{Gongorismo}
\begin{itemize}
\item {Grp. gram.:m.}
\end{itemize}
\begin{itemize}
\item {Proveniência:(De \textunderscore Gôngora\textunderscore , n. p.)}
\end{itemize}
Estilo pretensioso, muito abundante de ornatos e trocadilhos, á imitação do poéta castelhano Gôngora.
\section{Gongorista}
\begin{itemize}
\item {Grp. gram.:m.  e  adj.}
\end{itemize}
\begin{itemize}
\item {Proveniência:(De \textunderscore Gôngora\textunderscore , n. p.)}
\end{itemize}
Partidário ou imitador do gongorismo.
\section{Gongorizar}
\begin{itemize}
\item {Grp. gram.:v. t.}
\end{itemize}
\begin{itemize}
\item {Grp. gram.:V. i.}
\end{itemize}
Dar feição gongórica a.
Poetar gongoricamente:«\textunderscore ...dado que haja gongorizado sobejamente\textunderscore ». Camillo, \textunderscore Cav. em Ruínas\textunderscore , 88.
\section{Gongrona}
\begin{itemize}
\item {Grp. gram.:f.}
\end{itemize}
\begin{itemize}
\item {Proveniência:(Gr. \textunderscore gongrone\textunderscore )}
\end{itemize}
Tubérculo fungoso, no tronco das árvores.
Doença, o mesmo que \textunderscore papeira\textunderscore .
\section{Gongylango}
\begin{itemize}
\item {Grp. gram.:m.}
\end{itemize}
\begin{itemize}
\item {Utilização:Bot.}
\end{itemize}
\begin{itemize}
\item {Proveniência:(Do gr. \textunderscore gongulos\textunderscore  + \textunderscore angos\textunderscore )}
\end{itemize}
Parte das plantas cryptógâmicas, que encerra os corpúsculos reproductores.
\section{Gongylar}
\begin{itemize}
\item {Grp. gram.:adj.}
\end{itemize}
\begin{itemize}
\item {Utilização:Bot.}
\end{itemize}
Relativo aos gôngylos.
Gemmíparo.
\section{Gôngylo}
\begin{itemize}
\item {Grp. gram.:m.}
\end{itemize}
\begin{itemize}
\item {Utilização:Bot.}
\end{itemize}
\begin{itemize}
\item {Proveniência:(Gr. \textunderscore gongulos\textunderscore , redondo)}
\end{itemize}
Corpúsculo reproductor de algumas plantas.
\section{Goníaco}
\begin{itemize}
\item {Grp. gram.:adj.}
\end{itemize}
\begin{itemize}
\item {Utilização:Anat.}
\end{itemize}
Relativo ao gónio.
\section{Gonídia}
\begin{itemize}
\item {Grp. gram.:f.}
\end{itemize}
\begin{itemize}
\item {Utilização:Bot.}
\end{itemize}
\begin{itemize}
\item {Proveniência:(Do gr. \textunderscore gonos\textunderscore , producção)}
\end{itemize}
Céllulas verdes, que, nas algas e nos líchens, formam uma camada contínua, em que parece residir todo o poder vegetativo daquellas plantas.
\section{Gonilha}
\begin{itemize}
\item {Grp. gram.:f.}
\end{itemize}
\begin{itemize}
\item {Utilização:Prov.}
\end{itemize}
O mesmo que \textunderscore gravata\textunderscore . Cf. Camillo, \textunderscore Vingança\textunderscore , 95; \textunderscore Myst. de Lisb\textunderscore ., I, 18, 137 e 167.
(Por \textunderscore gonelha\textunderscore , dissimilação de \textunderscore golelha\textunderscore , de \textunderscore gola\textunderscore )
\section{Gonímico}
\begin{itemize}
\item {Grp. gram.:adj.}
\end{itemize}
\begin{itemize}
\item {Proveniência:(Do gr. \textunderscore gonimos\textunderscore , que produz)}
\end{itemize}
Relativo á gonídia.
\section{Gónio}
\begin{itemize}
\item {Grp. gram.:m.}
\end{itemize}
\begin{itemize}
\item {Utilização:Anat.}
\end{itemize}
\begin{itemize}
\item {Proveniência:(Do gr. \textunderscore gonia\textunderscore )}
\end{itemize}
Região angular do maxillar superior.
\section{Goniocarpo}
\begin{itemize}
\item {Grp. gram.:m.}
\end{itemize}
\begin{itemize}
\item {Utilização:Bot.}
\end{itemize}
\begin{itemize}
\item {Proveniência:(Do gr. \textunderscore gonia\textunderscore  + \textunderscore karpos\textunderscore )}
\end{itemize}
Gênero de plantas da China, Japão e Austrália.
\section{Goniógrafo}
\begin{itemize}
\item {Grp. gram.:m.}
\end{itemize}
\begin{itemize}
\item {Proveniência:(Do gr. \textunderscore gonia\textunderscore  + \textunderscore graphein\textunderscore )}
\end{itemize}
Pequeno instrumento, destinado a dar graficamente qualquer ângulo.
\section{Goniógrapho}
\begin{itemize}
\item {Grp. gram.:m.}
\end{itemize}
\begin{itemize}
\item {Proveniência:(Do gr. \textunderscore gonia\textunderscore  + \textunderscore graphein\textunderscore )}
\end{itemize}
Pequeno instrumento, destinado a dar graphicamente qualquer ângulo.
\section{Goniometria}
\begin{itemize}
\item {Grp. gram.:f.}
\end{itemize}
\begin{itemize}
\item {Proveniência:(De \textunderscore goniómetro\textunderscore )}
\end{itemize}
Arte de medir ângulos.
\section{Goniométrico}
\begin{itemize}
\item {Grp. gram.:adj.}
\end{itemize}
Relativo á goniometria.
\section{Goniómetro}
\begin{itemize}
\item {Grp. gram.:m.}
\end{itemize}
\begin{itemize}
\item {Proveniência:(Do gr. \textunderscore gonia\textunderscore  + \textunderscore metron\textunderscore )}
\end{itemize}
Instrumento, para medir ângulos.
\section{Gonióstomos}
\begin{itemize}
\item {Grp. gram.:m. pl.}
\end{itemize}
\begin{itemize}
\item {Proveniência:(Do gr. \textunderscore gonia\textunderscore  + \textunderscore stoma\textunderscore )}
\end{itemize}
Género de molluscos, de bôca angulosa.
\section{Goniteca}
\begin{itemize}
\item {Grp. gram.:f.}
\end{itemize}
\begin{itemize}
\item {Utilização:Zool.}
\end{itemize}
\begin{itemize}
\item {Proveniência:(Do gr. \textunderscore gonu\textunderscore  + \textunderscore theke\textunderscore )}
\end{itemize}
Cavidade da coxa dos insectos, na qual se aloja a base da tibia.
\section{Gonocele}
\begin{itemize}
\item {Grp. gram.:m.}
\end{itemize}
\begin{itemize}
\item {Proveniência:(Do gr. \textunderscore gonu\textunderscore  + \textunderscore kele\textunderscore )}
\end{itemize}
Inchação dos joêlhos.
\section{Gonocele}
\begin{itemize}
\item {Grp. gram.:m.}
\end{itemize}
\begin{itemize}
\item {Proveniência:(Do gr. \textunderscore gonos\textunderscore , esperma, e \textunderscore kele\textunderscore , tumor)}
\end{itemize}
Acumulação de esperma nos vasos seminíferos.
\section{Gonocóccico}
\begin{itemize}
\item {Grp. gram.:adj.}
\end{itemize}
Relativo ao gonococco: \textunderscore infecção gonocóccica\textunderscore .
\section{Gonococco}
\begin{itemize}
\item {Grp. gram.:m.}
\end{itemize}
\begin{itemize}
\item {Proveniência:(Do gr. \textunderscore gonos\textunderscore  + \textunderscore kokkos\textunderscore )}
\end{itemize}
Micróbio da blenorrhagia.
\section{Gonocócico}
\begin{itemize}
\item {Grp. gram.:adj.}
\end{itemize}
Relativo ao gonococo: \textunderscore infecção gonocócica\textunderscore .
\section{Gonococo}
\begin{itemize}
\item {Grp. gram.:m.}
\end{itemize}
\begin{itemize}
\item {Proveniência:(Do gr. \textunderscore gonos\textunderscore  + \textunderscore kokkos\textunderscore )}
\end{itemize}
Micróbio da blenorragia.
\section{Gonóforo}
\begin{itemize}
\item {Grp. gram.:m.}
\end{itemize}
\begin{itemize}
\item {Utilização:Bot.}
\end{itemize}
\begin{itemize}
\item {Proveniência:(Do gr. \textunderscore gonos\textunderscore  + \textunderscore phoros\textunderscore )}
\end{itemize}
Prolongamento do receptáculo, que sustenta só os estames e o pistilo.
\section{Gonono}
\begin{itemize}
\item {Grp. gram.:m.}
\end{itemize}
Árvore de Moçambique, própria para vigas.
\section{Gonóphoro}
\begin{itemize}
\item {Grp. gram.:m.}
\end{itemize}
\begin{itemize}
\item {Utilização:Bot.}
\end{itemize}
\begin{itemize}
\item {Proveniência:(Do gr. \textunderscore gonos\textunderscore  + \textunderscore phoros\textunderscore )}
\end{itemize}
Prolongamento do receptáculo, que sustenta só os estames e o pistillo.
\section{Gonorol}
\begin{itemize}
\item {Grp. gram.:m.}
\end{itemize}
\begin{itemize}
\item {Utilização:Chím.}
\end{itemize}
Producto da condensação dos princípios activos do sândalo.
\section{Gonorreia}
\begin{itemize}
\item {Grp. gram.:f.}
\end{itemize}
\begin{itemize}
\item {Proveniência:(Do gr. \textunderscore gonos\textunderscore  + \textunderscore rhein\textunderscore )}
\end{itemize}
Corrimento mucoso pelo canal da uretra.
\section{Gonorreico}
\begin{itemize}
\item {Grp. gram.:adj.}
\end{itemize}
Relativo á gonorreia.
\section{Gonorrhéa}
\begin{itemize}
\item {Grp. gram.:f.}
\end{itemize}
\begin{itemize}
\item {Proveniência:(Do gr. \textunderscore gonos\textunderscore  + \textunderscore rhein\textunderscore )}
\end{itemize}
Corrimento mucoso pelo canal da uretra.
\section{Gonorrheia}
\begin{itemize}
\item {Grp. gram.:f.}
\end{itemize}
\begin{itemize}
\item {Proveniência:(Do gr. \textunderscore gonos\textunderscore  + \textunderscore rhein\textunderscore )}
\end{itemize}
Corrimento mucoso pelo canal da uretra.
\section{Gonorrheico}
\begin{itemize}
\item {Grp. gram.:adj.}
\end{itemize}
Relativo á gonorrheia.
\section{Gonozoário}
\begin{itemize}
\item {Grp. gram.:m.}
\end{itemize}
Hydra, que, no pólypo-hydráceo, desempenha as funcções da reproducção da espécie.
\section{Gonu}
\begin{itemize}
\item {Grp. gram.:m.}
\end{itemize}
Planta cucurbitácea, (\textunderscore wildebrandia hibiscoides\textunderscore ).
\section{Gonytheca}
\begin{itemize}
\item {Grp. gram.:f.}
\end{itemize}
\begin{itemize}
\item {Utilização:Zool.}
\end{itemize}
\begin{itemize}
\item {Proveniência:(Do gr. \textunderscore gonu\textunderscore  + \textunderscore theke\textunderscore )}
\end{itemize}
Cavidade da coxa dos insectos, na qual se aloja a base da tibia.
\section{Gonzálea}
\begin{itemize}
\item {Grp. gram.:f.}
\end{itemize}
\begin{itemize}
\item {Proveniência:(De \textunderscore Gonzalez\textunderscore , n. p.)}
\end{itemize}
Gênero de plantas da América do Sul.
\section{Gonzo}
\begin{itemize}
\item {Grp. gram.:m.}
\end{itemize}
\begin{itemize}
\item {Proveniência:(Do lat. \textunderscore gomphus\textunderscore ?)}
\end{itemize}
Peça de dois anéis enganchados, pregados em peças distintas, uma fixa e outra movediça.
Bisagra.
Quícios; dobradiça.
\section{Goodênia}
\begin{itemize}
\item {Grp. gram.:f.}
\end{itemize}
\begin{itemize}
\item {Proveniência:(De \textunderscore Goodenough\textunderscore , n. p.)}
\end{itemize}
Gênero de plantas campanuláceas.
\section{Goodeniáceas}
\begin{itemize}
\item {Grp. gram.:f. pl.}
\end{itemize}
\begin{itemize}
\item {Proveniência:(De \textunderscore goodênia\textunderscore )}
\end{itemize}
Família de plantas, formada por Brown á custa das campanuláceas.
\section{Gopiara}
\begin{itemize}
\item {Grp. gram.:f.}
\end{itemize}
\begin{itemize}
\item {Utilização:Bras}
\end{itemize}
Terra, em que se podem lavrar minas de diamantes.
\section{Gorá}
\begin{itemize}
\item {Grp. gram.:m.}
\end{itemize}
Instrumento dos indígenas sul-africanos, formado de uma tripa retesada num arco e que se faz vibrar, soprando-a fortemente por uma penna de avestruz.
\section{Goral}
\begin{itemize}
\item {Grp. gram.:m.}
\end{itemize}
Cabrito montês do Himalaia.
\section{Gorar}
\begin{itemize}
\item {Grp. gram.:v. t.}
\end{itemize}
\begin{itemize}
\item {Grp. gram.:V. i.  e  p.}
\end{itemize}
\begin{itemize}
\item {Proveniência:(De \textunderscore gôro\textunderscore )}
\end{itemize}
Mallograr.
Inutilizar.
Corromper-se na incubação (falando-se do ovo).
Inutilizar-se; frustrar-se: \textunderscore gorar-se um plano\textunderscore .
\section{Gorarema}
\begin{itemize}
\item {Grp. gram.:f.}
\end{itemize}
Árvore silvestre do Brasil.
\section{Goraz}
\begin{itemize}
\item {Grp. gram.:m.}
\end{itemize}
\begin{itemize}
\item {Proveniência:(Do lat. \textunderscore vorax\textunderscore ?)}
\end{itemize}
Peixe esparoide, (\textunderscore pagellus controdontus\textunderscore ).
Ave pernalta, (\textunderscore nycticorax europaeus\textunderscore ).
\section{Gorazeira}
\begin{itemize}
\item {Grp. gram.:f.}
\end{itemize}
\begin{itemize}
\item {Proveniência:(De \textunderscore gorazeiro\textunderscore )}
\end{itemize}
Apparelho de linhas e anzóes para a pesca do goraz e de outros peixes.
\section{Gorazeiro}
\begin{itemize}
\item {Grp. gram.:adj.}
\end{itemize}
Relativo a goraz.
\section{Gorazeiro}
\begin{itemize}
\item {Grp. gram.:adj.}
\end{itemize}
\begin{itemize}
\item {Utilização:P. us.}
\end{itemize}
\begin{itemize}
\item {Proveniência:(De \textunderscore gorar\textunderscore ?)}
\end{itemize}
Em que há muitos reveses.
\section{Gordaço}
\begin{itemize}
\item {Grp. gram.:adj.}
\end{itemize}
Muito gordo.
\section{Gordalhaço}
\begin{itemize}
\item {Grp. gram.:adj.}
\end{itemize}
O mesmo que \textunderscore gordalhudo\textunderscore .
\section{Gordalhudo}
\begin{itemize}
\item {Grp. gram.:adj.}
\end{itemize}
O mesmo que \textunderscore gordanchudo\textunderscore .
\section{Gordalhufo}
\begin{itemize}
\item {Grp. gram.:adj.}
\end{itemize}
O mesmo que \textunderscore gordanchudo\textunderscore . Cf. Eça, \textunderscore P. Amaro\textunderscore , 103, 344 e 389.
\section{Gordan}
\begin{itemize}
\item {Grp. gram.:f.}
\end{itemize}
\begin{itemize}
\item {Utilização:Pop.}
\end{itemize}
\begin{itemize}
\item {Proveniência:(De \textunderscore gordo\textunderscore )}
\end{itemize}
O mesmo que \textunderscore gordura\textunderscore .
\section{Gordanchudo}
\begin{itemize}
\item {Grp. gram.:adj.}
\end{itemize}
\begin{itemize}
\item {Utilização:Fam.}
\end{itemize}
\begin{itemize}
\item {Proveniência:(Do rad. de \textunderscore gordo\textunderscore )}
\end{itemize}
Que é muito gordo.
Barrigudo.
\section{Gordiano}
\begin{itemize}
\item {Grp. gram.:adj.}
\end{itemize}
O mesmo que \textunderscore górdio\textunderscore ^1.
\section{Górdio}
\begin{itemize}
\item {Grp. gram.:adj.}
\end{itemize}
\begin{itemize}
\item {Proveniência:(De \textunderscore Górdio\textunderscore , n. p.)}
\end{itemize}
\textunderscore Nó górdio\textunderscore , grande difficuldade.
\section{Górdio}
\begin{itemize}
\item {Grp. gram.:m.}
\end{itemize}
Animal muito delgado e longo, que se cria nos tanques, (\textunderscore gordium aquaticum\textunderscore )--Porque êste animal dá o aspecto de cabello de mulher, formou-se a superstição de que um cabello de mulher, metido na água ou no leite, se converte em cobra.
\section{Gordo}
\begin{itemize}
\item {fónica:gôr}
\end{itemize}
\begin{itemize}
\item {Grp. gram.:adj.}
\end{itemize}
\begin{itemize}
\item {Utilização:Fig.}
\end{itemize}
\begin{itemize}
\item {Utilização:Fig.}
\end{itemize}
\begin{itemize}
\item {Grp. gram.:M.}
\end{itemize}
\begin{itemize}
\item {Proveniência:(Do lat. \textunderscore gurdus\textunderscore )}
\end{itemize}
Análogo á gordura.
Untuoso.
Que tem gordura ou matéria sebácea: \textunderscore carne gorda\textunderscore .
Cujo tecido adiposo está muito desenvolvido: \textunderscore homem gordo\textunderscore .
Sujo de gordura.
Forte, apto para bôa producção, (falando-se de um terreno)
Alentado.
Importante, considerável.
\textunderscore Letras gordas\textunderscore , instrucção escassa, ignorância.
Qualquer substância gorda.
Homem gordo.
\textunderscore Dia de gordo\textunderscore , dia em que a Igreja não prescreve abstinência de carne.
\section{Gordote}
\begin{itemize}
\item {Grp. gram.:adj.}
\end{itemize}
O mesmo que \textunderscore gorducho\textunderscore .
\section{Gorducho}
\begin{itemize}
\item {Grp. gram.:adj.}
\end{itemize}
Um tanto gordo.
\section{Gordura}
\begin{itemize}
\item {Grp. gram.:f.}
\end{itemize}
\begin{itemize}
\item {Utilização:Fig.}
\end{itemize}
Substância animal untuosa e de pouca consistência, que se derrete facilmente.
Qualidade de quem ou daquillo que é gordo.
Obesidade; nediez.
Apparência oleosa.
\section{Gordurento}
\begin{itemize}
\item {Grp. gram.:adj.}
\end{itemize}
Que tem gordura; besuntado; sujo.
\section{Gorduroso}
\begin{itemize}
\item {Grp. gram.:adj.}
\end{itemize}
\begin{itemize}
\item {Proveniência:(De \textunderscore gordura\textunderscore )}
\end{itemize}
Que tem a consistência da gordura; gordurento.
\section{Goreiro}
\begin{itemize}
\item {Grp. gram.:adj.}
\end{itemize}
\begin{itemize}
\item {Utilização:Prov.}
\end{itemize}
\begin{itemize}
\item {Utilização:dur.}
\end{itemize}
\begin{itemize}
\item {Proveniência:(De \textunderscore gorar\textunderscore )}
\end{itemize}
Que produz pouco e mal.
Diz-se especialmente da videira, que, por longevidade ou por outro motivo, produz poucos cachos e êstes imperfeitos.
\section{Gorga}
\begin{itemize}
\item {Grp. gram.:f.}
\end{itemize}
\begin{itemize}
\item {Utilização:Prov.}
\end{itemize}
\begin{itemize}
\item {Utilização:minh.}
\end{itemize}
Espécie de planta, (\textunderscore sperbula arvensis\textunderscore , Lin.).
\section{Gorgão}
\begin{itemize}
\item {Grp. gram.:m.}
\end{itemize}
\begin{itemize}
\item {Utilização:T. de Lanhoso}
\end{itemize}
\begin{itemize}
\item {Proveniência:(De \textunderscore gorga\textunderscore )}
\end{itemize}
Espécie de planta (\textunderscore sperbula radicans\textunderscore ).
\section{Gorgaz}
\begin{itemize}
\item {Grp. gram.:m.}
\end{itemize}
O mesmo que \textunderscore gorguz\textunderscore .
\section{Gorge}
\begin{itemize}
\item {Grp. gram.:f.}
\end{itemize}
\begin{itemize}
\item {Utilização:Prov.}
\end{itemize}
\begin{itemize}
\item {Utilização:minh.}
\end{itemize}
O mesmo que \textunderscore gorja\textunderscore .
\section{Gorgi}
\begin{itemize}
\item {Grp. gram.:m.}
\end{itemize}
Planta aquática do Brasil.
\section{Gorgolão}
\begin{itemize}
\item {Grp. gram.:m.}
\end{itemize}
O mesmo que \textunderscore gorgolhão\textunderscore .
Pequena golfada; bocado:«\textunderscore a boca a escumar gorgolões de pão de ló.\textunderscore »Camillo, \textunderscore Corja\textunderscore , 14.
\section{Gorgolar}
\begin{itemize}
\item {Grp. gram.:v. i.}
\end{itemize}
Sair em golfada ou gorgolão.
(Cp. \textunderscore gorgolão\textunderscore )
\section{Gorgolejante}
\begin{itemize}
\item {Grp. gram.:adj.}
\end{itemize}
Que gorgoleja.
\section{Gorgolejar}
\begin{itemize}
\item {Grp. gram.:v. i.}
\end{itemize}
\begin{itemize}
\item {Grp. gram.:V. t.}
\end{itemize}
Produzir o ruído especial do gargarejo, bebendo.
Beber, gorgolejando:«\textunderscore gorgolejava vinhos engarrafados.\textunderscore »Camillo, \textunderscore Brasileira\textunderscore , 47.
(Alter. de \textunderscore gargarejar\textunderscore )
\section{Gorgolejo}
\begin{itemize}
\item {Grp. gram.:m.}
\end{itemize}
Acto de gorgolejar.
\section{Gorgoleta}
\begin{itemize}
\item {fónica:lê}
\end{itemize}
\begin{itemize}
\item {Grp. gram.:f.}
\end{itemize}
\begin{itemize}
\item {Utilização:Ant.}
\end{itemize}
\begin{itemize}
\item {Proveniência:(Do rad. de \textunderscore gorgolejar\textunderscore )}
\end{itemize}
Vaso de barro, com um ralo, por onde a água, passando, produz ruído.
O mesmo que \textunderscore gole\textunderscore . Cf. \textunderscore Fénix Renasc.\textunderscore , IV, 491.
\section{Gorgolhão}
\begin{itemize}
\item {Grp. gram.:m.}
\end{itemize}
\begin{itemize}
\item {Proveniência:(De \textunderscore gorgolhar\textunderscore )}
\end{itemize}
Golfada.
Pequeno jacto de água; borbotão.
\section{Gorgolhar}
\begin{itemize}
\item {Grp. gram.:v. i.}
\end{itemize}
\begin{itemize}
\item {Proveniência:(Do rad. de \textunderscore gorgolejar\textunderscore )}
\end{itemize}
Brotar em gorgolhão.
\section{Gorgoli}
\begin{itemize}
\item {Grp. gram.:m.}
\end{itemize}
\begin{itemize}
\item {Proveniência:(T. asiát.)}
\end{itemize}
Vaso com água, em que se immerge o tubo do cachimbo para esfriar o fumo.
\section{Gorgolo}
\begin{itemize}
\item {Grp. gram.:m.}
\end{itemize}
\begin{itemize}
\item {Utilização:Prov.}
\end{itemize}
\begin{itemize}
\item {Utilização:trasm.}
\end{itemize}
Recato, grande cuidado.
\section{Gorgomil}
\begin{itemize}
\item {Grp. gram.:m.}
\end{itemize}
\begin{itemize}
\item {Utilização:Prov.}
\end{itemize}
\begin{itemize}
\item {Utilização:trasm.}
\end{itemize}
O mesmo que \textunderscore gorgomila\textunderscore .
\section{Gorgomila}
\begin{itemize}
\item {Grp. gram.:f.}
\end{itemize}
\begin{itemize}
\item {Utilização:Ant.}
\end{itemize}
O mesmo que \textunderscore gorgomilos\textunderscore . Cf. \textunderscore Mestre Giraldo\textunderscore .
\section{Gorgomileira}
\begin{itemize}
\item {Grp. gram.:f.}
\end{itemize}
\begin{itemize}
\item {Utilização:Ant.}
\end{itemize}
\begin{itemize}
\item {Proveniência:(De \textunderscore gorgomilo\textunderscore )}
\end{itemize}
O mesmo que \textunderscore goéla\textunderscore . Cf. G. Vicente, I, 171.
\section{Gorgomilo}
\begin{itemize}
\item {Grp. gram.:m.}
\end{itemize}
\begin{itemize}
\item {Utilização:Pop.}
\end{itemize}
O mesmo que \textunderscore gorgomilos\textunderscore .
\section{Gorgomilos}
\begin{itemize}
\item {Grp. gram.:m. pl.}
\end{itemize}
Goélas.
Princípio do esóphago.
(Do mesmo rad. que \textunderscore gorgolejar\textunderscore )
\section{Gorgóneo}
\begin{itemize}
\item {Grp. gram.:adj.}
\end{itemize}
\begin{itemize}
\item {Utilização:Poét.}
\end{itemize}
\begin{itemize}
\item {Proveniência:(Lat. \textunderscore gorgoneus\textunderscore )}
\end{itemize}
Relativo a Medusa.
\section{Gorgónia}
\begin{itemize}
\item {Grp. gram.:f.}
\end{itemize}
Espécie de árvore animal, uma das raras variedades de animaes fixos. Cf. Caminhoá, \textunderscore Bot. Ger. e Méd.\textunderscore 
\section{Gorgorão}
\begin{itemize}
\item {Grp. gram.:m.}
\end{itemize}
\begin{itemize}
\item {Proveniência:(Fr. \textunderscore gourgouran\textunderscore )}
\end{itemize}
Tecido encorpado de seda ou lan.
\section{Gorgòtó}
\begin{itemize}
\item {Grp. gram.:m.}
\end{itemize}
\begin{itemize}
\item {Utilização:Prov.}
\end{itemize}
\begin{itemize}
\item {Utilização:trasm.}
\end{itemize}
\textunderscore Ir-se tudo de gorgòtó\textunderscore , gastar-se tudo em patuscadas, em comes e bebes.
(Relaciona-se com \textunderscore gorgueira\textunderscore )
\section{Gorgueira}
\begin{itemize}
\item {Grp. gram.:f.}
\end{itemize}
O mesmo que \textunderscore gorjeira\textunderscore .
\section{Gorgulhento}
\begin{itemize}
\item {Grp. gram.:adj.}
\end{itemize}
\begin{itemize}
\item {Utilização:Bras}
\end{itemize}
Diz-se do terreno, que abunda nos seixos, chamados gorgulhos.
\section{Gorgulho}
\begin{itemize}
\item {Grp. gram.:m.}
\end{itemize}
\begin{itemize}
\item {Utilização:Bras}
\end{itemize}
\begin{itemize}
\item {Proveniência:(Do lat. \textunderscore curculio\textunderscore )}
\end{itemize}
Insecto coleóptero, nocivo especialmente aos celleiros.
Seixinhos de grés, de quartzo e de sílex, ora soltos, ora ligados por uma argilla amarela e vermelha.
\section{Gorguz}
\begin{itemize}
\item {Grp. gram.:m.}
\end{itemize}
\begin{itemize}
\item {Proveniência:(Do ár. \textunderscore guerguit\textunderscore )}
\end{itemize}
Virotão.
Antiga arma de arremêsso, que se despedia com béstas.
\section{Gorilha}
\begin{itemize}
\item {Grp. gram.:m.}
\end{itemize}
\begin{itemize}
\item {Proveniência:(De \textunderscore gorillas\textunderscore , nome que se deu a mulheres cabelludas, que os Carthagineses diziam têr encontrado na costa de África)}
\end{itemize}
Macaco anthropomorpho, (\textunderscore gorilla gina\textunderscore ).
\section{Gorilla}
\begin{itemize}
\item {Grp. gram.:m.}
\end{itemize}
O mesmo que gorilha. Cf. Latino, \textunderscore Or. da Corôa\textunderscore , XLIII.
\section{Gorinhatá}
\begin{itemize}
\item {Grp. gram.:m.}
\end{itemize}
\begin{itemize}
\item {Utilização:Bras}
\end{itemize}
Avezinha canora, (\textunderscore euphonia violacea\textunderscore ).
\section{Gorinos}
\begin{itemize}
\item {Grp. gram.:m. pl.}
\end{itemize}
O mesmo que \textunderscore guarinos\textunderscore .
\section{Gorja}
\begin{itemize}
\item {Grp. gram.:f.}
\end{itemize}
\begin{itemize}
\item {Utilização:Ant.}
\end{itemize}
\begin{itemize}
\item {Proveniência:(Do lat. \textunderscore gurges\textunderscore . Cp. fr. \textunderscore gorge\textunderscore )}
\end{itemize}
Garganta.
A parte mais estreita da quilha.
Cachaço. Cf. Camillo, \textunderscore Volcões\textunderscore , 153.
\section{Gorjal}
\begin{itemize}
\item {Grp. gram.:m.}
\end{itemize}
\begin{itemize}
\item {Utilização:Ant.}
\end{itemize}
\begin{itemize}
\item {Utilização:Prov.}
\end{itemize}
\begin{itemize}
\item {Utilização:beir.}
\end{itemize}
\begin{itemize}
\item {Proveniência:(De \textunderscore gorja\textunderscore )}
\end{itemize}
Parte da armadura, para defesa do pescoço.
Espécie do collar, geralmente de prata, cravejado de pedras finas e usado outrora por senhoras.
O mesmo que \textunderscore bôca\textunderscore : \textunderscore cala essa gorja\textunderscore .
\section{Gorjeado}
\begin{itemize}
\item {Grp. gram.:adj.}
\end{itemize}
Em que há gorjeios ou harmonias:«\textunderscore pavilhão gorjeado de pássaros.\textunderscore »Camillo, \textunderscore Brasileira\textunderscore , 30.
\section{Gorjeador}
\begin{itemize}
\item {Grp. gram.:adj.}
\end{itemize}
Que gorjeia.
\section{Gorjear}
\begin{itemize}
\item {Grp. gram.:v. i.}
\end{itemize}
\begin{itemize}
\item {Grp. gram.:V. t.}
\end{itemize}
\begin{itemize}
\item {Proveniência:(De \textunderscore gorja\textunderscore )}
\end{itemize}
Emittir sons agradáveis da garganta.
Trilar.
Cantar.
Exprimir em gorjeios.
\section{Gorjeio}
\begin{itemize}
\item {Grp. gram.:m.}
\end{itemize}
\begin{itemize}
\item {Utilização:Fig.}
\end{itemize}
Acto ou effeito de gorjear.
O chilrear das crianças.
\section{Gorjeira}
\begin{itemize}
\item {Grp. gram.:f.}
\end{itemize}
\begin{itemize}
\item {Proveniência:(De \textunderscore gorja\textunderscore )}
\end{itemize}
Renda ou pano de adôrno para o pescoço.
Parte da antiga armadura, para defesa do pescoço.
\section{Gorjel}
\begin{itemize}
\item {Grp. gram.:m.}
\end{itemize}
\begin{itemize}
\item {Utilização:Des.}
\end{itemize}
O mesmo que \textunderscore gorjal\textunderscore .
\section{Gorjelim}
\begin{itemize}
\item {Grp. gram.:m.}
\end{itemize}
\begin{itemize}
\item {Utilização:Ant.}
\end{itemize}
\begin{itemize}
\item {Proveniência:(De \textunderscore gorjel\textunderscore )}
\end{itemize}
O mesmo que \textunderscore gorjal\textunderscore .
\section{Gorjeta}
\begin{itemize}
\item {fónica:jê}
\end{itemize}
\begin{itemize}
\item {Grp. gram.:f.}
\end{itemize}
\begin{itemize}
\item {Proveniência:(De \textunderscore gorja\textunderscore )}
\end{itemize}
Bebida, com que se gratifica um pequeno serviço.
Dinheiro, para pagar essa bebida.
Gratificação; espórtula.
Escopro delgado, para lavrar mármore.
\section{Gorjete}
\begin{itemize}
\item {fónica:jê}
\end{itemize}
\begin{itemize}
\item {Grp. gram.:m.}
\end{itemize}
\begin{itemize}
\item {Proveniência:(De \textunderscore gorja\textunderscore )}
\end{itemize}
Peça de vestuário, formada de collar e peitilho, para se sobrepor á camisa de dormir, dispensando outra camisa.
Camisote.
\section{Gorjilo}
\begin{itemize}
\item {Grp. gram.:m.}
\end{itemize}
\begin{itemize}
\item {Utilização:Bot.}
\end{itemize}
\begin{itemize}
\item {Proveniência:(Do rad. de \textunderscore gorja\textunderscore )}
\end{itemize}
Espaço entre os torilos das plantas.
\section{Gorne}
\begin{itemize}
\item {Grp. gram.:m.}
\end{itemize}
\begin{itemize}
\item {Utilização:Náut.}
\end{itemize}
\begin{itemize}
\item {Proveniência:(It. \textunderscore gorna\textunderscore )}
\end{itemize}
Abertura dos moitões, onde se encaixam as rodas, para laborarem os cabos da embarcação.
\section{Gornir}
\begin{itemize}
\item {Grp. gram.:v.}
\end{itemize}
\begin{itemize}
\item {Utilização:t. Náut.}
\end{itemize}
Passar nos gornes (os cabos).
\section{Gôro}
\begin{itemize}
\item {Grp. gram.:adj.}
\end{itemize}
\begin{itemize}
\item {Utilização:Fig.}
\end{itemize}
\begin{itemize}
\item {Proveniência:(Do gr. \textunderscore ourios\textunderscore ?)}
\end{itemize}
Que se gorou, (falando-se do ovo).
Que se frustrou, que se inutilizou.
\section{Gorondozi}
\begin{itemize}
\item {Grp. gram.:m.}
\end{itemize}
\begin{itemize}
\item {Utilização:Bot.}
\end{itemize}
Trepadeira de Moçambique.
\section{Gorotil}
\begin{itemize}
\item {Grp. gram.:m.}
\end{itemize}
\begin{itemize}
\item {Utilização:Náut.}
\end{itemize}
O lado mais alto das velas de uma embarcação.
Acto de envergar as vêrgas.
(Corr. de \textunderscore corutilho\textunderscore , de \textunderscore coruto\textunderscore ?)
\section{Gorototo}
\begin{itemize}
\item {fónica:tô}
\end{itemize}
\begin{itemize}
\item {Grp. gram.:m.}
\end{itemize}
Pássaro dentirostro da África occidental.
\section{Gorovinhas}
\begin{itemize}
\item {Grp. gram.:f. pl.}
\end{itemize}
Pregas no vestido.
\section{Gorpelha}
\begin{itemize}
\item {fónica:pê}
\end{itemize}
\begin{itemize}
\item {Grp. gram.:f.}
\end{itemize}
\begin{itemize}
\item {Utilização:Prov.}
\end{itemize}
\begin{itemize}
\item {Utilização:alg.}
\end{itemize}
O mesmo que \textunderscore golpelha\textunderscore ^2.
\section{Gorra}
\begin{itemize}
\item {fónica:gô}
\end{itemize}
\begin{itemize}
\item {Grp. gram.:f.}
\end{itemize}
\begin{itemize}
\item {Grp. gram.:Loc. adv.}
\end{itemize}
Carapuça.
Espécie de barrete.
De gorra, de camaradagem, de sociedade.
(Cast. \textunderscore gorra\textunderscore )
\section{Gorra}
\begin{itemize}
\item {fónica:gô}
\end{itemize}
\begin{itemize}
\item {Grp. gram.:f.}
\end{itemize}
\begin{itemize}
\item {Utilização:Prov.}
\end{itemize}
\begin{itemize}
\item {Utilização:alent.}
\end{itemize}
\begin{itemize}
\item {Utilização:Prov.}
\end{itemize}
\begin{itemize}
\item {Utilização:alent.}
\end{itemize}
Casca de gorreiro.
Trança de esparto ou piaçaba, a que se prendem os alcatruzes das noras.
(Corr. de \textunderscore corra\textunderscore )
\section{Gorreiro}
\begin{itemize}
\item {Grp. gram.:m.}
\end{itemize}
\begin{itemize}
\item {Utilização:Prov.}
\end{itemize}
\begin{itemize}
\item {Utilização:alent.}
\end{itemize}
\begin{itemize}
\item {Proveniência:(De \textunderscore gorra\textunderscore ^2)}
\end{itemize}
O mesmo que \textunderscore trovisco\textunderscore .
\section{Gorrião}
\begin{itemize}
\item {Grp. gram.:m.}
\end{itemize}
Pássaro conirostro, espécie de pardal.
(Cast. \textunderscore gorrión\textunderscore )
\section{Gorro}
\begin{itemize}
\item {fónica:gô}
\end{itemize}
\begin{itemize}
\item {Grp. gram.:m.}
\end{itemize}
Barrete preto e comprido, com feitio de saca, e usado por estudantes que trajam capa e batina.
Chapéu de senhora, redondo e mais curto que o gorro de estudante.
Carapuça, semelhante ao gorro de estudante.
(Cp. \textunderscore gorra\textunderscore ^1)
\section{Gorujuba}
\begin{itemize}
\item {Grp. gram.:m.}
\end{itemize}
\begin{itemize}
\item {Utilização:Bras}
\end{itemize}
Peixe de água doce.
\section{Gorumixama}
\begin{itemize}
\item {Grp. gram.:f.}
\end{itemize}
\begin{itemize}
\item {Utilização:Bras}
\end{itemize}
O mesmo que \textunderscore jabuticaba\textunderscore .
\section{Gorungugi}
\begin{itemize}
\item {Grp. gram.:m.}
\end{itemize}
\begin{itemize}
\item {Utilização:Bras}
\end{itemize}
Gênero de myriápodes.
\section{Gorvata}
\begin{itemize}
\item {Grp. gram.:f.}
\end{itemize}
\begin{itemize}
\item {Utilização:Prov.}
\end{itemize}
\begin{itemize}
\item {Utilização:alg.}
\end{itemize}
O mesmo que \textunderscore gravata\textunderscore .
(Cp. cast. \textunderscore corbata\textunderscore )
\section{Gosma}
\begin{itemize}
\item {Grp. gram.:f.}
\end{itemize}
\begin{itemize}
\item {Utilização:Pop.}
\end{itemize}
\begin{itemize}
\item {Proveniência:(Do fr. \textunderscore gourme\textunderscore )}
\end{itemize}
Doença da língua das aves, especialmente das gallináceas.
Inflammação nas mucosas das vias respiratórias dos poldros.
Escarro.
\section{Gosmar}
\begin{itemize}
\item {Grp. gram.:v. t.}
\end{itemize}
\begin{itemize}
\item {Grp. gram.:V. i.}
\end{itemize}
\begin{itemize}
\item {Proveniência:(De \textunderscore gosma\textunderscore )}
\end{itemize}
Escarrar.
Proferir, tossindo ou escarrando.
Expellir escarros.
\section{Gosmento}
\begin{itemize}
\item {Grp. gram.:adj.}
\end{itemize}
\begin{itemize}
\item {Utilização:Ext.}
\end{itemize}
Que tem gosma.
Que escarra muito.
Fraco, adoentado.
\section{Gosmoso}
\begin{itemize}
\item {Grp. gram.:adj.}
\end{itemize}
O mesmo que \textunderscore gosmento\textunderscore .
\section{Gôso}
\begin{itemize}
\item {Grp. gram.:m.}
\end{itemize}
Cão pequeno e vulgar.
(Catalão \textunderscore gos\textunderscore , cão)
\section{Gostar}
\begin{itemize}
\item {Grp. gram.:v. i.}
\end{itemize}
\begin{itemize}
\item {Grp. gram.:V. t.}
\end{itemize}
\begin{itemize}
\item {Proveniência:(Lat. \textunderscore gustare\textunderscore )}
\end{itemize}
Achar sabor agradável: \textunderscore gostar de trutas\textunderscore .
Sentir prazer: \textunderscore gósto de passear\textunderscore .
Têr amizade: \textunderscore gósto da minha Antónia\textunderscore .
Têr inclinação.
Usar.
Dar-se bem: \textunderscore gostar da convivência\textunderscore .
Provar; experimentar.
Têr satisfação com.
\section{Gostável}
\begin{itemize}
\item {Grp. gram.:adj.}
\end{itemize}
\begin{itemize}
\item {Proveniência:(De \textunderscore gostar\textunderscore )}
\end{itemize}
Que dá gôsto.
Que agrada.
Aprazível.
\section{Gostilho}
\begin{itemize}
\item {Grp. gram.:m.}
\end{itemize}
Pequeno gôsto; gostinho. Cf. M. Bernárdez, \textunderscore N. Floresta\textunderscore , IV, 407.
\section{Gôsto}
\begin{itemize}
\item {Grp. gram.:m.}
\end{itemize}
\begin{itemize}
\item {Utilização:Fig.}
\end{itemize}
\begin{itemize}
\item {Proveniência:(Lat. \textunderscore gustus\textunderscore )}
\end{itemize}
Sentido, que nos deixa conhecer o sabôr de alguma coisa.
Sabor.
Paladar.
Prazer.
Sympathia.
Critério.
Elegância: \textunderscore trajar com gôsto\textunderscore .
Carácter; maneira.
\section{Gostosamente}
\begin{itemize}
\item {Grp. gram.:adv.}
\end{itemize}
De modo gostoso.
\section{Gostos-da-vida}
\begin{itemize}
\item {Grp. gram.:f.}
\end{itemize}
Designação vulgar de uma espécie de ameixa grande, comprida e amarelada, que é doce ao provar-se e depois azêda, o que deu origem ao seu nome.
\section{Gostoso}
\begin{itemize}
\item {Grp. gram.:adj.}
\end{itemize}
Que tem bom sabor.
Que dá gôsto.
Que revela prazer.
Que dá prazer.
\section{Gota}
\begin{itemize}
\item {fónica:gô}
\end{itemize}
\begin{itemize}
\item {Grp. gram.:f.}
\end{itemize}
\begin{itemize}
\item {Proveniência:(Do lat. \textunderscore gutta\textunderscore )}
\end{itemize}
Pinga.
Pingo.
Pequeníssima porção de liquido.
Lágrima.
Pequenino ornato architectónico.
Inflammação das partes fibrosas e ligamentosas das articulações.
\textunderscore Gota coral\textunderscore , o mesmo que \textunderscore epilepsia\textunderscore .
\textunderscore Gota serena\textunderscore , perda total da vista, sem lesão apparente.
\section{Gotado}
\begin{itemize}
\item {Grp. gram.:adj.}
\end{itemize}
\begin{itemize}
\item {Proveniência:(Do lat. \textunderscore guttatus\textunderscore )}
\end{itemize}
Que tem gotas.
Ornado de gotas.
\section{Gote}
\begin{itemize}
\item {Grp. gram.:m.}
\end{itemize}
\begin{itemize}
\item {Utilização:T. da Afr. or. port}
\end{itemize}
Peça de pau, com que se equilibram as panelas e as cestas.
\section{Gotear}
\textunderscore v. i.\textunderscore  (e der.)
O mesmo que \textunderscore gotejar\textunderscore , etc. Cp. Filinto, XIV, 231.
\section{Goteira}
\begin{itemize}
\item {Grp. gram.:f.}
\end{itemize}
\begin{itemize}
\item {Proveniência:(De \textunderscore gota\textunderscore )}
\end{itemize}
Cano, que recebe dos telhados a água pluvial, levando-a para fóra das paredes.
Telha de beiral, donde escorre a água pluvial para o chão ou para o cano que recebe água de todo o beiral.
Fenda ou buraco no telhado, donde cái água dentro de casa.
\section{Gotejamento}
\begin{itemize}
\item {Grp. gram.:m.}
\end{itemize}
Acto ou effeito de gotejar.
\section{Gotejante}
\begin{itemize}
\item {Grp. gram.:adj.}
\end{itemize}
Que goteja.
\section{Gotejar}
\begin{itemize}
\item {Grp. gram.:v. i.}
\end{itemize}
\begin{itemize}
\item {Grp. gram.:V. t.}
\end{itemize}
Caír em gotas.
Entornar, verter ou deixar caír, gota a gota.
\section{Gótico}
\begin{itemize}
\item {Grp. gram.:adj.}
\end{itemize}
\begin{itemize}
\item {Proveniência:(Lat. \textunderscore goticus\textunderscore )}
\end{itemize}
Relativo a Godos; que provém dos Godos.
\section{Gotingo}
\begin{itemize}
\item {Grp. gram.:m.}
\end{itemize}
Arvore da Índia portuguesa.
\section{Gotismo}
\begin{itemize}
\item {Grp. gram.:m.}
\end{itemize}
Affeição a coisas góticas.
Carácter dos costumes e instituições dos Godos. Cf. Filinto, VI, 178.
(Cp. \textunderscore gótico\textunderscore )
\section{Goto}
\begin{itemize}
\item {fónica:gô}
\end{itemize}
\begin{itemize}
\item {Grp. gram.:m.}
\end{itemize}
\begin{itemize}
\item {Utilização:Pop.}
\end{itemize}
\begin{itemize}
\item {Utilização:Fig.}
\end{itemize}
\begin{itemize}
\item {Proveniência:(Do lat. \textunderscore guttur\textunderscore )}
\end{itemize}
Entrada da larynge.
Glotte.
\textunderscore Dar no goto\textunderscore , produzir suffocação quando se engole.
Causar estranheza.
\section{Gotoso}
\begin{itemize}
\item {Grp. gram.:m.  e  adj.}
\end{itemize}
O que padece gota.
\section{Gotúlio}
\begin{itemize}
\item {Grp. gram.:m.}
\end{itemize}
\begin{itemize}
\item {Utilização:Prov.}
\end{itemize}
\begin{itemize}
\item {Utilização:pop.}
\end{itemize}
\begin{itemize}
\item {Proveniência:(Do rad. de \textunderscore gota\textunderscore )}
\end{itemize}
Uma pinga ou copo de vinho, muito saboreada: \textunderscore bebeu-lhe um gotúlio e fez vispere\textunderscore .
\section{Gotulho}
\begin{itemize}
\item {Grp. gram.:m.}
\end{itemize}
\begin{itemize}
\item {Utilização:Prov.}
\end{itemize}
\begin{itemize}
\item {Utilização:pop.}
\end{itemize}
O mesmo que \textunderscore gotúlio\textunderscore . Cf. Camillo, \textunderscore Onde está a Fel.\textunderscore , 299.
\section{Gotrim}
\begin{itemize}
\item {Grp. gram.:m.}
\end{itemize}
(V.godrim)
\section{Gouve}
\begin{itemize}
\item {Grp. gram.:m.}
\end{itemize}
Ave africana, (\textunderscore lamprocolius acuticaudus\textunderscore , Bocage).
\section{Gouveio}
\begin{itemize}
\item {Grp. gram.:m.  e  adj.}
\end{itemize}
\begin{itemize}
\item {Proveniência:(De \textunderscore Gouveia\textunderscore , n. p.)}
\end{itemize}
O mesmo que \textunderscore verdelho\textunderscore .
\section{Gouveio-branco}
\begin{itemize}
\item {Grp. gram.:m.}
\end{itemize}
Casta de uva, na região do Doiro.
\section{Gouveio-melano}
\begin{itemize}
\item {Grp. gram.:m.}
\end{itemize}
Casta de uva, na região do Doiro.
\section{Gouveio-pardo}
\begin{itemize}
\item {Grp. gram.:m.}
\end{itemize}
Casta de uva, na região do Doiro.
\section{Gouver}
\begin{itemize}
\item {Grp. gram.:v. t.}
\end{itemize}
(Fórma preferida a \textunderscore gouvir\textunderscore  por Castilho, nuns commentários mss. a uma ed. do diccion. de Moraes, baseando-se em que Rui de Pina escreveu \textunderscore gouvessem\textunderscore , e não \textunderscore gouvissem\textunderscore )
\section{Gouvinte}
\begin{itemize}
\item {Grp. gram.:adj.}
\end{itemize}
Que gouve.
\section{Gouvir}
\begin{itemize}
\item {Grp. gram.:v. t.}
\end{itemize}
\begin{itemize}
\item {Utilização:Ant.}
\end{itemize}
\begin{itemize}
\item {Proveniência:(Do lat. \textunderscore gaudere\textunderscore )}
\end{itemize}
O mesmo que \textunderscore gozar\textunderscore :«\textunderscore ...e nos praz que ajão e gouvão de todalas graças...\textunderscore »Doc. do séc. XVI, no \textunderscore Instituto\textunderscore , LIX, 190.
\section{Governação}
\begin{itemize}
\item {Grp. gram.:f.}
\end{itemize}
Acto de governar.
O mesmo que \textunderscore govêrno\textunderscore .
\section{Governadeira}
\begin{itemize}
\item {Grp. gram.:f.  e  adj.}
\end{itemize}
\begin{itemize}
\item {Proveniência:(De \textunderscore governar\textunderscore )}
\end{itemize}
Mulher, que administra bem a sua casa.
\section{Governado}
\begin{itemize}
\item {Grp. gram.:adj.}
\end{itemize}
\begin{itemize}
\item {Utilização:Gír.}
\end{itemize}
\begin{itemize}
\item {Proveniência:(De \textunderscore governar\textunderscore )}
\end{itemize}
Que se sabe governar.
Poupado.
Armado.
\section{Governador}
\begin{itemize}
\item {Grp. gram.:m.  e  adj.}
\end{itemize}
\begin{itemize}
\item {Proveniência:(Do lat. \textunderscore gubernator\textunderscore )}
\end{itemize}
O que governa.
\textunderscore Governador civil\textunderscore , magistrado superior de districto administrativo, em Portugal.
\section{Governadora}
\begin{itemize}
\item {Grp. gram.:f.  e  adj.}
\end{itemize}
\begin{itemize}
\item {Proveniência:(De \textunderscore governador\textunderscore )}
\end{itemize}
Governadeira.
Mulher do governador.
Mulher, que rege um Estado, no impedimento de um Soberano.
\section{Governadura}
\begin{itemize}
\item {Grp. gram.:f.}
\end{itemize}
\begin{itemize}
\item {Utilização:Ant.}
\end{itemize}
\begin{itemize}
\item {Proveniência:(Do lat. \textunderscore gubernatura\textunderscore )}
\end{itemize}
Certo trabalho em ferro, na Índia portuguesa. Cf. \textunderscore Lembranças das Cousas da Índia\textunderscore , nos \textunderscore Subsidios\textunderscore  de Felner.
\section{Governalho}
\begin{itemize}
\item {Grp. gram.:m.}
\end{itemize}
\begin{itemize}
\item {Utilização:Ant.}
\end{itemize}
\begin{itemize}
\item {Proveniência:(Do lat. \textunderscore gubernaculum\textunderscore )}
\end{itemize}
Leme de embarcação.
Govêrno. Cf. \textunderscore Roteiro de Vasco da Gama\textunderscore .
\section{Governamental}
\begin{itemize}
\item {Grp. gram.:adj.}
\end{itemize}
\begin{itemize}
\item {Grp. gram.:M.}
\end{itemize}
\begin{itemize}
\item {Proveniência:(De \textunderscore governamento\textunderscore )}
\end{itemize}
Relativo ao govêrno; ministerial; partidário de um ministério: \textunderscore gazeta governamental\textunderscore .
Aquelle que é partidário de um ministério.
\section{Governamento}
\begin{itemize}
\item {Grp. gram.:m.}
\end{itemize}
\begin{itemize}
\item {Utilização:Des.}
\end{itemize}
\begin{itemize}
\item {Proveniência:(De \textunderscore governar\textunderscore )}
\end{itemize}
O mesmo que \textunderscore govêrno\textunderscore .
\section{Governança}
\begin{itemize}
\item {Grp. gram.:f.}
\end{itemize}
\begin{itemize}
\item {Utilização:Deprec.}
\end{itemize}
O mesmo que \textunderscore govêrno\textunderscore .
\section{Governanta}
\begin{itemize}
\item {Grp. gram.:f.}
\end{itemize}
\begin{itemize}
\item {Proveniência:(De \textunderscore governante\textunderscore )}
\end{itemize}
Mulher, que administra casa alheia.
\section{Governante}
\begin{itemize}
\item {Grp. gram.:m. ,  f.  e  adj.}
\end{itemize}
Pessôa que governa.
\section{Governar}
\begin{itemize}
\item {Grp. gram.:v. t.}
\end{itemize}
\begin{itemize}
\item {Utilização:Marn.}
\end{itemize}
\begin{itemize}
\item {Grp. gram.:V. i.}
\end{itemize}
\begin{itemize}
\item {Grp. gram.:V. p.}
\end{itemize}
\begin{itemize}
\item {Proveniência:(Do lat. \textunderscore gubernare\textunderscore )}
\end{itemize}
Dirigir com o leme (uma embarcação).
Dirigir.
Administrar: \textunderscore governar uma casa\textunderscore .
Reger.
Regular o andamento de: \textunderscore governar um cavallo\textunderscore .
Têr poder ou autoridade sôbre.
Imperar em: \textunderscore governar um Estado\textunderscore .
Encher de água (a marinha).
Encaminhar-se.
Exercer autoridade; imperar.
Cuidar dos seus interêsses: \textunderscore o Silva sabe governar-se\textunderscore .
\section{Governativo}
\begin{itemize}
\item {Grp. gram.:adj.}
\end{itemize}
\begin{itemize}
\item {Proveniência:(De \textunderscore governar\textunderscore )}
\end{itemize}
Relativo ao govêrno.
\section{Governatriz}
\begin{itemize}
\item {Grp. gram.:m.  e  adj.}
\end{itemize}
\begin{itemize}
\item {Proveniência:(Do lat. \textunderscore gubernatrix\textunderscore )}
\end{itemize}
Aquella que governa ou é directora.
Própria para governar.
\section{Governável}
\begin{itemize}
\item {Grp. gram.:adj.}
\end{itemize}
Que se póde governar ou dirigir; dócil: \textunderscore somos um povo muito governável\textunderscore .
\section{Governelo}
\begin{itemize}
\item {fónica:nê}
\end{itemize}
\begin{itemize}
\item {Grp. gram.:m.}
\end{itemize}
\begin{itemize}
\item {Utilização:Ant.}
\end{itemize}
\begin{itemize}
\item {Proveniência:(De \textunderscore govêrno\textunderscore )}
\end{itemize}
Mantimento, sustento.
\section{Governichar}
\begin{itemize}
\item {Grp. gram.:v. i.}
\end{itemize}
\begin{itemize}
\item {Utilização:Deprec.}
\end{itemize}
Fazer mau govêrno, com facções e galopins.
\section{Governicho}
\begin{itemize}
\item {Grp. gram.:m.}
\end{itemize}
\begin{itemize}
\item {Utilização:Fam.}
\end{itemize}
\begin{itemize}
\item {Proveniência:(De \textunderscore govêrno\textunderscore )}
\end{itemize}
Exercício de um pequeno emprêgo.
Administração de um pequeno districto ultramarino.
Sinecura.
\section{Governículo}
\begin{itemize}
\item {Grp. gram.:m.}
\end{itemize}
\begin{itemize}
\item {Utilização:Bras}
\end{itemize}
O mesmo que \textunderscore governicho\textunderscore .
\section{Governismo}
\begin{itemize}
\item {Grp. gram.:m.}
\end{itemize}
\begin{itemize}
\item {Utilização:Neol.}
\end{itemize}
\begin{itemize}
\item {Proveniência:(De \textunderscore govêrno\textunderscore )}
\end{itemize}
Systema de governar ou mandar, de modo autoritário.
Exercício dictatorial do poder.
\section{Governista}
\begin{itemize}
\item {Grp. gram.:m.  e  adj.}
\end{itemize}
O mesmo que \textunderscore governamental\textunderscore .
\section{Governita}
\begin{itemize}
\item {Grp. gram.:f.}
\end{itemize}
\begin{itemize}
\item {Utilização:Des.}
\end{itemize}
\begin{itemize}
\item {Proveniência:(De \textunderscore govêrno\textunderscore )}
\end{itemize}
Fardel, alforje com provisões de mantimentos para viagem.
\section{Govêrno}
\begin{itemize}
\item {Grp. gram.:m.}
\end{itemize}
\begin{itemize}
\item {Proveniência:(De \textunderscore governar\textunderscore )}
\end{itemize}
Leme de navio.
Acto ou effeito de governar.
Administração económica: \textunderscore o govêrno de uma Companhia\textunderscore .
Poder executivo, ou conjunto das pessôas que administram superiormente um Estado, uma província, uma companhia, etc.
Ministério: \textunderscore o Govêrno apresentou-se hoje ás Câmaras\textunderscore .
Systema ou modo por que se rege um Estado: \textunderscore govêrno representativo\textunderscore .
Freio.
Regime.
Comporta entre os reservatórios das salinas.
Depósito geral das águas das salinas.
Território, em que se exerce jurisdicção de um governador.
Decurso do tempo, em que alguém governou; reinado: \textunderscore no govêrno de D. José I\textunderscore .
\section{Goveta}
\begin{itemize}
\item {fónica:vê}
\end{itemize}
\begin{itemize}
\item {Grp. gram.:f.}
\end{itemize}
\begin{itemize}
\item {Utilização:Ant.}
\end{itemize}
O mesmo que \textunderscore gualteira\textunderscore .
\section{Goveta}
\begin{itemize}
\item {fónica:vê}
\end{itemize}
\begin{itemize}
\item {Grp. gram.:f.}
\end{itemize}
O mesmo que \textunderscore govete\textunderscore .
\section{Govete}
\begin{itemize}
\item {fónica:vê}
\end{itemize}
\begin{itemize}
\item {Grp. gram.:m.}
\end{itemize}
Cepo de carpinteiro, com uma peça lateral e móvel, ligada por parafusos, para regular a distância, a que há de fazer-se o rebaixamento da madeira.
(Por \textunderscore goivete\textunderscore , de \textunderscore goiva\textunderscore ?)
\section{Gozar}
\begin{itemize}
\item {Grp. gram.:v. t.}
\end{itemize}
\begin{itemize}
\item {Grp. gram.:V. i.}
\end{itemize}
\begin{itemize}
\item {Grp. gram.:V. p.}
\end{itemize}
Usar ou possuir (coisa útil ou agradável).
Fruir.
Viver agradavelmente.
Têr prazer.
Tirar proveito ou satisfação.
(Cp. cast. \textunderscore gozar\textunderscore )
\section{Gozaria}
\begin{itemize}
\item {Grp. gram.:f.}
\end{itemize}
\begin{itemize}
\item {Utilização:Ant.}
\end{itemize}
\begin{itemize}
\item {Proveniência:(De \textunderscore gôzo\textunderscore )}
\end{itemize}
Mordacidade; má língua.
\section{Gozete}
\begin{itemize}
\item {fónica:zê}
\end{itemize}
\begin{itemize}
\item {Grp. gram.:m.}
\end{itemize}
\begin{itemize}
\item {Utilização:Ant.}
\end{itemize}
\begin{itemize}
\item {Proveniência:(Do cast. \textunderscore gocete\textunderscore )}
\end{itemize}
Peça de coiro, na lança, para resguardo da mão.
\section{Gozil}
\begin{itemize}
\item {Grp. gram.:m.}
\end{itemize}
\begin{itemize}
\item {Utilização:Ant.}
\end{itemize}
O mesmo que \textunderscore aguazil\textunderscore . Cf. \textunderscore Roteiro de Vasco da Gama\textunderscore .
\section{Gôzo}
\begin{itemize}
\item {Grp. gram.:m.}
\end{itemize}
Acto de gozar.
Utilidade.
Satisfação.
(Cp. cast. \textunderscore gozo\textunderscore )
\section{Gozoso}
\begin{itemize}
\item {Grp. gram.:adj.}
\end{itemize}
Em que há gôzo.
Que tem gôzo ou prazer.
Que revela gôzo.
\section{Graado}
\begin{itemize}
\item {Grp. gram.:adj.}
\end{itemize}
\begin{itemize}
\item {Utilização:Ant.}
\end{itemize}
Grato, agradável, sympáthico. Cf. \textunderscore Port. Mon. Hist.\textunderscore , \textunderscore Script.\textunderscore , 284.
\section{Graal}
\begin{itemize}
\item {Grp. gram.:m.}
\end{itemize}
Vaso santo, de que, segundo a crença medieval, Jesus se serviu na ceia com os Apóstolos, e em que José de Arimatheia recolheu o sangue das chagas de Christo, vaso que, suppondo-se perdido, foi descoberto no saque de Cesareia em 1102, e possuído successivamente pelos Genoveses e Franceses.
(Cp. \textunderscore gral\textunderscore )
\section{Grabano}
\begin{itemize}
\item {Grp. gram.:m.}
\end{itemize}
\begin{itemize}
\item {Utilização:Prov.}
\end{itemize}
\begin{itemize}
\item {Utilização:trasm.}
\end{itemize}
Cabaço, ou vaso de fôlha, o mesmo que \textunderscore garabanho\textunderscore .
\section{Grabatério}
\begin{itemize}
\item {Grp. gram.:m.}
\end{itemize}
\begin{itemize}
\item {Proveniência:(Do lat. \textunderscore grabatus\textunderscore )}
\end{itemize}
Cada um dos antigos sectários, que, só no leito mortuário, recebiam o baptismo.
Fabricante de leitos, na Idade-Média.
\section{Grabato}
\begin{itemize}
\item {Grp. gram.:m.}
\end{itemize}
\begin{itemize}
\item {Proveniência:(Lat. \textunderscore grabatus\textunderscore )}
\end{itemize}
Leito pequeno e pobre. Cf. Castilho, \textunderscore Pal. de Um Crente\textunderscore , 166.
\section{Graça}
\begin{itemize}
\item {Grp. gram.:f.}
\end{itemize}
\begin{itemize}
\item {Grp. gram.:Loc. adv.}
\end{itemize}
\begin{itemize}
\item {Grp. gram.:Pl.}
\end{itemize}
\begin{itemize}
\item {Utilização:Prov.}
\end{itemize}
\begin{itemize}
\item {Utilização:dur.}
\end{itemize}
\begin{itemize}
\item {Grp. gram.:Interj.}
\end{itemize}
\begin{itemize}
\item {Proveniência:(Lat. \textunderscore gratia\textunderscore )}
\end{itemize}
Benevolência.
Favor, mercê.
Perdão.
Commutação de pena.
Feição agradável, agrado: \textunderscore rosto cheio de graça\textunderscore .
Apparência attrahente; attractivo.
Elegância em falar e escrever.
Dicção espirituosa.
Gracejo: \textunderscore dirigiu-lhe uma graça\textunderscore .
Dom sobrenatural, como meio de salvação ou santificação.
Bôa acceitação, privança: \textunderscore estar na graça de alguém\textunderscore .
\textunderscore Graça pesada\textunderscore , gracejo offensivo.
\textunderscore De graça\textunderscore , gratuitamente.
Agradecimento: \textunderscore demos graças a Deus\textunderscore .
Benefícios espirituaes concedidos pela Igreja.
Designação de três deusas do Paganismo.
Tremoços curtidos, que a vendedeira dá de graça, depois de encher a medida que se lhe comprou: \textunderscore tome lá as graças\textunderscore .
Bem haja; muito obrigado: \textunderscore graças, meu Deus\textunderscore !
\section{Gracejador}
\begin{itemize}
\item {Grp. gram.:m.  e  adj.}
\end{itemize}
O que graceja.
\section{Gracejar}
\begin{itemize}
\item {Grp. gram.:v. i.}
\end{itemize}
\begin{itemize}
\item {Grp. gram.:V. t.}
\end{itemize}
\begin{itemize}
\item {Proveniência:(De \textunderscore graça\textunderscore )}
\end{itemize}
Dizer gracejos.
Ter ditos espirituosos.
Exprimir, por gracejo.
\section{Gracejo}
\begin{itemize}
\item {Grp. gram.:m.}
\end{itemize}
\begin{itemize}
\item {Proveniência:(De \textunderscore gracejar\textunderscore )}
\end{itemize}
Acto ou expressão zombeteira, mas inoffensiva; graça, pilhéria.
\section{Gracera}
\begin{itemize}
\item {Grp. gram.:f.}
\end{itemize}
\begin{itemize}
\item {Utilização:Ant.}
\end{itemize}
Vestuário de mulher.
Roupa de mulher.
\section{Graceta}
\begin{itemize}
\item {fónica:cê}
\end{itemize}
\begin{itemize}
\item {Grp. gram.:f.}
\end{itemize}
O mesmo que \textunderscore gracejo\textunderscore :«\textunderscore as mais galhofeiras das gracetas com que eram remoqueados.\textunderscore »Camillo, \textunderscore Filha do Reg.\textunderscore , 150.
\section{Grácil}
\begin{itemize}
\item {Grp. gram.:adj.}
\end{itemize}
\begin{itemize}
\item {Proveniência:(Lat. \textunderscore gracilis\textunderscore )}
\end{itemize}
Delgado; delicado: \textunderscore aquella grácil criatura\textunderscore .
Subtil; fino.
\section{Gracilidade}
\begin{itemize}
\item {Grp. gram.:f.}
\end{itemize}
\begin{itemize}
\item {Proveniência:(Lat. \textunderscore gracilitas\textunderscore )}
\end{itemize}
Qualidade daquelle ou daquillo que é grácil.
\section{Gracilifoliado}
\begin{itemize}
\item {Grp. gram.:adj.}
\end{itemize}
\begin{itemize}
\item {Utilização:Bot.}
\end{itemize}
\begin{itemize}
\item {Proveniência:(Do lat. \textunderscore gracilis\textunderscore  + \textunderscore folium\textunderscore )}
\end{itemize}
Que tem fôlhas delgadas.
\section{Gracilípede}
\begin{itemize}
\item {Grp. gram.:adj.}
\end{itemize}
\begin{itemize}
\item {Utilização:Zool.}
\end{itemize}
\begin{itemize}
\item {Proveniência:(Do lat. \textunderscore gracilis\textunderscore  + \textunderscore pes\textunderscore )}
\end{itemize}
Que tem pés delgados.
\section{Gracilirostro}
\begin{itemize}
\item {fónica:rós}
\end{itemize}
\begin{itemize}
\item {Grp. gram.:adj.}
\end{itemize}
\begin{itemize}
\item {Utilização:Zool.}
\end{itemize}
\begin{itemize}
\item {Proveniência:(Do lat. \textunderscore gracilis\textunderscore  + \textunderscore rostrum\textunderscore )}
\end{itemize}
Que tem bico delgado.
\section{Gracilirrostro}
\begin{itemize}
\item {Grp. gram.:adj.}
\end{itemize}
\begin{itemize}
\item {Utilização:Zool.}
\end{itemize}
\begin{itemize}
\item {Proveniência:(Do lat. \textunderscore gracilis\textunderscore  + \textunderscore rostrum\textunderscore )}
\end{itemize}
Que tem bico delgado.
\section{Gracinda}
\begin{itemize}
\item {Grp. gram.:f.}
\end{itemize}
\begin{itemize}
\item {Utilização:Bras}
\end{itemize}
O mesmo que \textunderscore glacinda\textunderscore .
\section{Gracíola}
\begin{itemize}
\item {Grp. gram.:f.}
\end{itemize}
\begin{itemize}
\item {Proveniência:(Lat. \textunderscore gratiola\textunderscore )}
\end{itemize}
Planta escrofularínea.
\section{Gracioladas}
\begin{itemize}
\item {Grp. gram.:f. pl.}
\end{itemize}
\begin{itemize}
\item {Proveniência:(De \textunderscore graciolado\textunderscore )}
\end{itemize}
Tríbo de plantas, que têm por typo a gracíola.
\section{Graciolado}
\begin{itemize}
\item {Grp. gram.:adj.}
\end{itemize}
Relativo ou semelhante á \textunderscore gracíola\textunderscore .
\section{Graciolina}
\begin{itemize}
\item {Grp. gram.:f.}
\end{itemize}
\begin{itemize}
\item {Utilização:Chím.}
\end{itemize}
Princípio amargo, extrahido da gracíola.
\section{Graciosa}
\begin{itemize}
\item {Grp. gram.:f.}
\end{itemize}
\begin{itemize}
\item {Proveniência:(De \textunderscore gracioso\textunderscore )}
\end{itemize}
Espécie de uva.
Gracíola.
\section{Graciosamente}
\begin{itemize}
\item {Grp. gram.:adv.}
\end{itemize}
De modo gracioso; com graça.
Amavelmente.
Por favor.
\section{Graciosidade}
\begin{itemize}
\item {Grp. gram.:f.}
\end{itemize}
Qualidade daquelle ou daquillo que é gracioso.
\section{Gracioso}
\begin{itemize}
\item {Grp. gram.:adj.}
\end{itemize}
\begin{itemize}
\item {Grp. gram.:M.}
\end{itemize}
\begin{itemize}
\item {Proveniência:(Lat. \textunderscore gratiosus\textunderscore )}
\end{itemize}
Em que há graça; que tem graça: \textunderscore versos graciosos\textunderscore .
Gracejador.
Dado ou feito de graça.
Motejador; chocarreiro; bobo.
\section{Gracir}
\begin{itemize}
\item {Grp. gram.:v. t.}
\end{itemize}
\begin{itemize}
\item {Utilização:Ant.}
\end{itemize}
\begin{itemize}
\item {Proveniência:(De \textunderscore graça\textunderscore . Cp. \textunderscore gratir\textunderscore )}
\end{itemize}
O mesmo que \textunderscore agradecer\textunderscore .
\section{Graçola}
\begin{itemize}
\item {Grp. gram.:f.}
\end{itemize}
\begin{itemize}
\item {Utilização:Fam.}
\end{itemize}
\begin{itemize}
\item {Grp. gram.:M.}
\end{itemize}
\begin{itemize}
\item {Proveniência:(De \textunderscore graça\textunderscore )}
\end{itemize}
Gracejo de mau gosto.
Chocarrice.
Aquelle que diz graçolas.
\section{Graçolar}
\begin{itemize}
\item {Grp. gram.:v. i.}
\end{itemize}
Dizer graçolas.
\section{Gradação}
\begin{itemize}
\item {Grp. gram.:f.}
\end{itemize}
\begin{itemize}
\item {Proveniência:(Lat. \textunderscore gradatio\textunderscore )}
\end{itemize}
Aumento ou deminuição gradual.
Passagem ou transição gradual.
Progressão de ideias, ascendente ou descendente.
\section{Gradador}
\begin{itemize}
\item {Grp. gram.:m.}
\end{itemize}
Aquelle que grada.
Grade, instrumento agrícola.
\section{Gradadura}
\begin{itemize}
\item {Grp. gram.:f.}
\end{itemize}
Acto ou effeito de gradar^1.
\section{Gradagem}
\begin{itemize}
\item {Grp. gram.:f.}
\end{itemize}
Acto de gradar^1.
\section{Gradar}
\begin{itemize}
\item {Grp. gram.:v. t.}
\end{itemize}
Aplanar ou esterroar com grade (terreno lavrado).
\section{Gradar}
\begin{itemize}
\item {Grp. gram.:v. i.}
\end{itemize}
Tornar-se grado.
Crescer. (Muito us. em Trás-os-Montes)
\section{Gradaria}
\begin{itemize}
\item {Grp. gram.:f.}
\end{itemize}
Série de grades ou de tabiques, formada de barras parallelas.
\section{Gradário}
\begin{itemize}
\item {Grp. gram.:adj.}
\end{itemize}
\begin{itemize}
\item {Utilização:Des.}
\end{itemize}
\begin{itemize}
\item {Proveniência:(Lat. \textunderscore gradarius\textunderscore )}
\end{itemize}
Que anda a passo, devagar, sem chouto, (falando-se do cavallo).
Que tem estilo sereno, fluente.
\section{Gradativo}
\begin{itemize}
\item {Grp. gram.:adj.}
\end{itemize}
\begin{itemize}
\item {Utilização:Neol.}
\end{itemize}
\begin{itemize}
\item {Proveniência:(Do lat. \textunderscore gradatus\textunderscore )}
\end{itemize}
Que procede por graus ou gradualmente; em que há gradação. Cf. Herculano, \textunderscore Hist. de Port.\textunderscore , III, 382; \textunderscore Opúsc.\textunderscore , III, 244 e IV, 82.
\section{Gradaús}
\begin{itemize}
\item {Grp. gram.:m. pl.}
\end{itemize}
Tríbo de índios, nas margens do Araguaia.
\section{Grade}
\begin{itemize}
\item {Grp. gram.:f.}
\end{itemize}
\begin{itemize}
\item {Utilização:Prov.}
\end{itemize}
\begin{itemize}
\item {Utilização:alent.}
\end{itemize}
\begin{itemize}
\item {Proveniência:(Do lat. \textunderscore crates\textunderscore )}
\end{itemize}
Tabique ou armação, formada de peças de metal ou madeira, com intervallos, e destinada a resguardar ou vedar um lugar.
Locutório de convento ou de cadeia.
Caixilho, em que o pintor assenta uma tela que há de pintar.
Caixa de ripas intervalladas, para transporte de mobília.
Instrumento agricola, formado de travessas parallelas em que há dentes de ferro ou madeira, para aplanar, afofar e esterroar a terra lavrada.
Molde para fazer telha ou tijolo.
Instrumento dentado, para limpar cavalgaduras.
Instrumento de veterinário, para cauterizar feridas de animaes.
Ancinho grande, de cabo comprido e curvo, para tirar o carvão das fórmas.
\section{Gradeamento}
\begin{itemize}
\item {Grp. gram.:m.}
\end{itemize}
Acto ou effeito de gradear.
Grade de ferro, para vedar jardim, pátio, etc.
\section{Gradear}
\begin{itemize}
\item {Grp. gram.:v. t.}
\end{itemize}
Pôr grades em.
Limitar com grades.
Ornar de grades.
Gradar^1.
\section{Gradecer}
\begin{itemize}
\item {Grp. gram.:v. i.}
\end{itemize}
O mesmo que \textunderscore gradar\textunderscore ^2.
\section{Gradeira}
\begin{itemize}
\item {Grp. gram.:f.}
\end{itemize}
\begin{itemize}
\item {Proveniência:(De \textunderscore grade\textunderscore )}
\end{itemize}
Freira, que acompanhava outra ou outras ao locutório.
\section{Gradelhas}
\begin{itemize}
\item {fónica:dê}
\end{itemize}
\begin{itemize}
\item {Grp. gram.:f. pl.}
\end{itemize}
\begin{itemize}
\item {Proveniência:(Do lat. \textunderscore craticula\textunderscore )}
\end{itemize}
Espécie de malha rara, na armadura antiga.
\section{Gradeza}
\begin{itemize}
\item {Grp. gram.:f.}
\end{itemize}
Qualidade daquillo que é \textunderscore grado\textunderscore ^1.
\section{Gradil}
\begin{itemize}
\item {Grp. gram.:m.}
\end{itemize}
\begin{itemize}
\item {Utilização:Bras}
\end{itemize}
\begin{itemize}
\item {Proveniência:(De \textunderscore grade\textunderscore )}
\end{itemize}
Grade, que circunda um jardim, um campo, uma praça, etc. Cf. \textunderscore Jorn. do Recife\textunderscore , de 2-XII-900.
\section{Gradim}
\begin{itemize}
\item {Grp. gram.:f.}
\end{itemize}
\begin{itemize}
\item {Proveniência:(Do fr. \textunderscore gradine\textunderscore )}
\end{itemize}
Instrumento de escultor, para tirar as asperezas, deixadas pelo ponteiro.
\section{Gradinada}
\begin{itemize}
\item {Grp. gram.:f.}
\end{itemize}
Retoque, feito com gradim.
\section{Gradinar}
\begin{itemize}
\item {Grp. gram.:v. t.}
\end{itemize}
\begin{itemize}
\item {Grp. gram.:V. i.}
\end{itemize}
Amaciar ou retocar com gradim.
Trabalhar com gradim.
\section{Gradinata}
\begin{itemize}
\item {Grp. gram.:f.}
\end{itemize}
\begin{itemize}
\item {Utilização:Ant.}
\end{itemize}
\begin{itemize}
\item {Proveniência:(De \textunderscore grade\textunderscore )}
\end{itemize}
Balaustrada de escada ou de varanda.
\section{Grado}
\begin{itemize}
\item {Grp. gram.:adj.}
\end{itemize}
\begin{itemize}
\item {Utilização:Fig.}
\end{itemize}
\begin{itemize}
\item {Proveniência:(Do lat. \textunderscore granatus\textunderscore )}
\end{itemize}
Graúdo.
Bem desenvolvido: \textunderscore milho grado\textunderscore .
Importante, notável.
\section{Grado}
\begin{itemize}
\item {Grp. gram.:m.}
\end{itemize}
\begin{itemize}
\item {Grp. gram.:Adj.}
\end{itemize}
\begin{itemize}
\item {Utilização:Ant.}
\end{itemize}
\begin{itemize}
\item {Proveniência:(Do lat. \textunderscore gratus\textunderscore )}
\end{itemize}
O mesmo que \textunderscore vontade\textunderscore , (us. nas loc. \textunderscore de bom grado\textunderscore , \textunderscore de mau grado\textunderscore , \textunderscore mau grado meu\textunderscore ).
O mesmo que \textunderscore grato\textunderscore .
\section{Grado}
\begin{itemize}
\item {Grp. gram.:m.}
\end{itemize}
\begin{itemize}
\item {Proveniência:(Lat. \textunderscore gradus\textunderscore )}
\end{itemize}
Passo.
Andadura.
\textunderscore Vencer o grado\textunderscore , dizia-se antigamente por ganhar prêmio em corridas ou desportes, e correspondia ao moderno e inútil anglicismo \textunderscore bater o record\textunderscore . Cf. Rui Pina, \textunderscore Chrón. de Aff. V\textunderscore , CXXXI.
Cada uma das cem partes iguaes, em que se divide o quadrante, na divisão centesimal da circunferência.
\section{...grado}
\begin{itemize}
\item {Grp. gram.:suf.}
\end{itemize}
\begin{itemize}
\item {Proveniência:(Do lat. \textunderscore gradi\textunderscore )}
\end{itemize}
(que entra na composição de têrmos designativos do modo de andar de certos animaes)
\section{Graduação}
\begin{itemize}
\item {Grp. gram.:f.}
\end{itemize}
Acto ou effeito de graduar.
Divisão do círculo em graus, minutos e segundos.
Divisão da escala de um instrumento.
Posição social.
Pôsto militar, honorífico ou sem proventos.
\section{Graduadamente}
\begin{itemize}
\item {Grp. gram.:adv.}
\end{itemize}
De modo graduado; com graduação.
\section{Graduador}
\begin{itemize}
\item {Grp. gram.:m.  e  adj.}
\end{itemize}
\begin{itemize}
\item {Proveniência:(De \textunderscore graduar\textunderscore )}
\end{itemize}
O que gradua.
\section{Gradual}
\begin{itemize}
\item {Grp. gram.:adj.}
\end{itemize}
\begin{itemize}
\item {Grp. gram.:M.}
\end{itemize}
\begin{itemize}
\item {Proveniência:(Do lat. \textunderscore gradus\textunderscore )}
\end{itemize}
Que tem ou revela gradação.
Versículos da Missa, entre a \textunderscore Epístola\textunderscore  e o \textunderscore Evangelho\textunderscore .
Livro, que tem o cantochão das rezas ecclesiásticas.
\section{Gradualmente}
\begin{itemize}
\item {Grp. gram.:adv.}
\end{itemize}
De modo gradual.
\section{Graduamento}
\begin{itemize}
\item {Grp. gram.:m.}
\end{itemize}
O mesmo que \textunderscore graduação\textunderscore .
\section{Graduar}
\begin{itemize}
\item {Grp. gram.:v. t.}
\end{itemize}
\begin{itemize}
\item {Grp. gram.:V. p.}
\end{itemize}
\begin{itemize}
\item {Proveniência:(Do lat. \textunderscore gradus\textunderscore )}
\end{itemize}
Dispor em graus.
Ordenar em categorias.
Dirigir de modo gradual.
Cotejar.
Conferir grau universitário a.
Conferir um pôsto militar, meramente honorífico, a: \textunderscore graduar um capitão em major\textunderscore .
Regular.
Proporcionar.
Classificar por certa ordem: \textunderscore graduar os concorrentes a um lugar\textunderscore .
Tomar grau universitário.
\section{Gradura}
\begin{itemize}
\item {Grp. gram.:f.}
\end{itemize}
\begin{itemize}
\item {Utilização:Prov.}
\end{itemize}
\begin{itemize}
\item {Utilização:trasm.}
\end{itemize}
Designação genérica do feijão.
\section{Graeiro}
\begin{itemize}
\item {Grp. gram.:m.}
\end{itemize}
\begin{itemize}
\item {Proveniência:(Do lat. \textunderscore granarius\textunderscore )}
\end{itemize}
Grão de chumbo ou de cereaes.
\section{Graelada}
\begin{itemize}
\item {fónica:gra-e}
\end{itemize}
\begin{itemize}
\item {Grp. gram.:f.}
\end{itemize}
\begin{itemize}
\item {Utilização:Prov.}
\end{itemize}
\begin{itemize}
\item {Utilização:trasm.}
\end{itemize}
\begin{itemize}
\item {Proveniência:(De \textunderscore graélo\textunderscore )}
\end{itemize}
O mesmo que \textunderscore saraivada\textunderscore .
\section{Graélo}
\begin{itemize}
\item {Grp. gram.:m.}
\end{itemize}
\begin{itemize}
\item {Utilização:Prov.}
\end{itemize}
\begin{itemize}
\item {Utilização:trasm.}
\end{itemize}
\begin{itemize}
\item {Proveniência:(De \textunderscore grão\textunderscore )}
\end{itemize}
Granizo, saraiva.
\section{Grafila}
\begin{itemize}
\item {Grp. gram.:f.}
\end{itemize}
\begin{itemize}
\item {Proveniência:(Do rad. do gr. \textunderscore graphein\textunderscore )}
\end{itemize}
Orla de medalha ou de moéda, na qual se abre a inscripção.
\section{Grafilha}
\begin{itemize}
\item {Grp. gram.:f.}
\end{itemize}
\begin{itemize}
\item {Utilização:Ant.}
\end{itemize}
O mesmo que \textunderscore grafila\textunderscore . Cf. \textunderscore Provas da Hist. Gen.\textunderscore , III, 186.
\section{Grafito}
\begin{itemize}
\item {Grp. gram.:m.}
\end{itemize}
\begin{itemize}
\item {Proveniência:(It. \textunderscore grafitto\textunderscore )}
\end{itemize}
Designação do que se acha escrito em paredes das cidades e monumentos da antiguidade.
\section{Grafitto}
\begin{itemize}
\item {Grp. gram.:m.}
\end{itemize}
\begin{itemize}
\item {Proveniência:(It. \textunderscore grafitto\textunderscore )}
\end{itemize}
Designação do que se acha escrito em paredes das cidades e monumentos da antiguidade.
\section{Gragoatá}
\begin{itemize}
\item {Grp. gram.:m.}
\end{itemize}
Planta medicinal do Brasil. Cf. \textunderscore Diário do Congresso\textunderscore , de 11-X-900.
\section{Graiar}
\begin{itemize}
\item {Grp. gram.:v. i.}
\end{itemize}
\begin{itemize}
\item {Utilização:Prov.}
\end{itemize}
\begin{itemize}
\item {Utilização:trasm.}
\end{itemize}
O mesmo que \textunderscore gradar\textunderscore ^2.
\section{Graieiro}
\begin{itemize}
\item {Grp. gram.:m.}
\end{itemize}
(V.graeiro). Cf. Camillo, \textunderscore Brasileira\textunderscore , 128.
\section{Graínha}
\begin{itemize}
\item {Grp. gram.:f.}
\end{itemize}
\begin{itemize}
\item {Proveniência:(De \textunderscore grão\textunderscore )}
\end{itemize}
Semente das uvas.
\section{Graixa}
\begin{itemize}
\item {Grp. gram.:f.}
\end{itemize}
(Corr. de \textunderscore graxa\textunderscore )
\section{Grajau}
\begin{itemize}
\item {Grp. gram.:m.}
\end{itemize}
O mesmo que \textunderscore garajau\textunderscore ^1.
\section{Grajeia}
\begin{itemize}
\item {Grp. gram.:f.}
\end{itemize}
O mesmo ou melhor que \textunderscore grangeia\textunderscore .
(Cp. cast. \textunderscore grajea\textunderscore )
\section{Gral}
\begin{itemize}
\item {Grp. gram.:m.}
\end{itemize}
(V.almofariz)(Contr. de \textunderscore graal\textunderscore  &gt; \textunderscore granal\textunderscore )
\section{Gralha}
\begin{itemize}
\item {Grp. gram.:f.}
\end{itemize}
\begin{itemize}
\item {Utilização:Fig.}
\end{itemize}
\begin{itemize}
\item {Grp. gram.:Pl.}
\end{itemize}
\begin{itemize}
\item {Proveniência:(Do lat. \textunderscore gracula\textunderscore )}
\end{itemize}
Pássaro conirostro, da fam. dos corvos.
Letra ou sinal gráphico, invertido ou collocado fóra do seu lugar, na composição typográphica.
Mulher tagarela.
Espécie de jôgo popular.
\section{Gralhada}
\begin{itemize}
\item {Grp. gram.:f.}
\end{itemize}
\begin{itemize}
\item {Utilização:Fig.}
\end{itemize}
\begin{itemize}
\item {Proveniência:(De \textunderscore gralhar\textunderscore )}
\end{itemize}
Canto simultâneo de muitos pássaros.
Falácia, vozearia confusa.
\section{Gralhador}
\begin{itemize}
\item {Grp. gram.:m.  e  adj.}
\end{itemize}
O que gralha.
\section{Gralhar}
\begin{itemize}
\item {Grp. gram.:v. i.}
\end{itemize}
\begin{itemize}
\item {Utilização:Fig.}
\end{itemize}
\begin{itemize}
\item {Proveniência:(De \textunderscore gralha\textunderscore )}
\end{itemize}
Grasnar, (falando-se da gralha e de algumas outras aves, como o gaio).
Falar confusamente.
Tagarelar.
\section{Gralheada}
\begin{itemize}
\item {Grp. gram.:f.}
\end{itemize}
Acto de gralhear.
O mesmo que \textunderscore gralhada\textunderscore . Cf. Camillo, \textunderscore Caveira\textunderscore , 62.
\section{Gralhear}
\begin{itemize}
\item {Grp. gram.:v. i.}
\end{itemize}
O mesmo que \textunderscore gralhar\textunderscore .
Chilrear alto. Cf. Camillo, \textunderscore Noites de Insómn.\textunderscore , VII, 40.
\section{Gralheira}
\begin{itemize}
\item {Grp. gram.:f.}
\end{itemize}
\begin{itemize}
\item {Utilização:Prov.}
\end{itemize}
Sítio, onde as gralhas formam bando.
\section{Gralho}
\begin{itemize}
\item {Grp. gram.:m.}
\end{itemize}
\begin{itemize}
\item {Proveniência:(Do lat. \textunderscore graculus\textunderscore )}
\end{itemize}
Gralha.
Ave nocturna de rapina.
Corvo marínho.
Nome de um pássaro conirostro, (\textunderscore mainatus\textunderscore ).
\section{Grã}
\begin{itemize}
\item {Grp. gram.:f.}
\end{itemize}
\begin{itemize}
\item {Utilização:Prov.}
\end{itemize}
\begin{itemize}
\item {Utilização:Prov.}
\end{itemize}
\begin{itemize}
\item {Utilização:trasm.}
\end{itemize}
\begin{itemize}
\item {Proveniência:(Do lat. \textunderscore granum\textunderscore )}
\end{itemize}
Galha do uma espécie de carvalho (\textunderscore quercus coccifera\textunderscore ).
Insecto hemíptero, de côr vermelha, empregado em tinturaria e pharmácia, (\textunderscore coccus ilicis\textunderscore , Lin.).
Tecido, tinto com gran.
Côr escarlate.
O mesmo que \textunderscore graínha\textunderscore .
Moléstia do gado suino, que se manifesta por uma excrescência carnosa na bôca.
\section{Gralídeas}
\begin{itemize}
\item {Grp. gram.:f. pl.}
\end{itemize}
\begin{itemize}
\item {Utilização:Zool.}
\end{itemize}
Ordem de aves, que tem por tipo a gralha.
(Palavra mal formada de \textunderscore gralha\textunderscore  + gr. \textunderscore eidos\textunderscore . Que eu saiba, só em catalão o nome daquela ave se escreve \textunderscore gralla\textunderscore )
\section{Grallídeas}
\begin{itemize}
\item {Grp. gram.:f. pl.}
\end{itemize}
\begin{itemize}
\item {Utilização:Zool.}
\end{itemize}
Ordem de aves, que tem por typo a gralha.
(Palavra mal formada de \textunderscore gralha\textunderscore  + gr. \textunderscore eidos\textunderscore . Que eu saiba, só em catalão o nome daquella ave se escreve \textunderscore gralla\textunderscore )
\section{Grama}
\begin{itemize}
\item {Grp. gram.:f.}
\end{itemize}
\begin{itemize}
\item {Proveniência:(Do lat. \textunderscore gramen\textunderscore )}
\end{itemize}
Nome de várias plantas, da fam. das gramíneas.
\section{Grama}
\begin{itemize}
\item {Grp. gram.:f.}
\end{itemize}
O mesmo que \textunderscore gramadeira\textunderscore . (Colhido em Turquel)
\section{Grama}
\begin{itemize}
\item {Grp. gram.:m.}
\end{itemize}
\begin{itemize}
\item {Proveniência:(Gr. \textunderscore gramma\textunderscore )}
\end{itemize}
Pêso de um centímetro cúbico de água destilada.
Unidade das medidas de pêso, no sistema métrico ou decimal.
\section{Gramada}
\begin{itemize}
\item {Grp. gram.:f.}
\end{itemize}
\begin{itemize}
\item {Utilização:T. de Pare -de-Coira}
\end{itemize}
\begin{itemize}
\item {Utilização:des.}
\end{itemize}
Acto de gramar^1 ou de trabalhar com a gramadeira.
\section{Gramadeira}
\begin{itemize}
\item {Grp. gram.:f.}
\end{itemize}
\begin{itemize}
\item {Utilização:Prov.}
\end{itemize}
\begin{itemize}
\item {Proveniência:(De \textunderscore gramar\textunderscore ^1)}
\end{itemize}
Peça de madeira, com que se trilha o linho antes de espadelado.
Gancho, com que nas cavallariças se puxa a palha para a mangedoira.
Mulher, que trabalha com a gramadeira.
\section{Gramado}
\begin{itemize}
\item {Grp. gram.:m.}
\end{itemize}
\begin{itemize}
\item {Utilização:Bras}
\end{itemize}
\begin{itemize}
\item {Proveniência:(De \textunderscore gramar\textunderscore ^2)}
\end{itemize}
Terreno, onde cresce a grama.
\section{Gramafónio}
\begin{itemize}
\item {Grp. gram.:m.}
\end{itemize}
\begin{itemize}
\item {Proveniência:(Do gr. \textunderscore gramma\textunderscore  + \textunderscore phone\textunderscore )}
\end{itemize}
Fonógrafo aperfeiçoado, que reproduz os sons por meio de discos.
\section{Gramalheira}
\begin{itemize}
\item {Grp. gram.:f.}
\end{itemize}
Corrente de ferro, que suspende a caldeira sôbre o lume.
Régua fixa, junto a carris de ferro, na qual engranza uma roda motora dentada da locomotiva de caminhos de ferro, em rampa muito íngreme.
(Cast. \textunderscore gramallera\textunderscore , fr. \textunderscore cremaillière\textunderscore )
\section{Gramanto}
\begin{itemize}
\item {Grp. gram.:m.}
\end{itemize}
\begin{itemize}
\item {Proveniência:(Do gr. \textunderscore gramma\textunderscore  + \textunderscore anthos\textunderscore )}
\end{itemize}
Gênero de plantas crassuláceas.
\section{Gramão}
\begin{itemize}
\item {Grp. gram.:m.}
\end{itemize}
\begin{itemize}
\item {Proveniência:(De \textunderscore grama\textunderscore )}
\end{itemize}
Espécie de grama medicinal, (\textunderscore cynodon dactylon\textunderscore ).
\section{Gramar}
\begin{itemize}
\item {Grp. gram.:v. t.}
\end{itemize}
\begin{itemize}
\item {Utilização:Fam.}
\end{itemize}
\begin{itemize}
\item {Utilização:Pop.}
\end{itemize}
Trilhar com gramadeira (o linho).
Engulir.
Apanhar (uma sova).
\section{Gramar}
\begin{itemize}
\item {Grp. gram.:v. t.}
\end{itemize}
\begin{itemize}
\item {Utilização:Bras}
\end{itemize}
Cobrir ou plantar de grama.
\section{Gramar}
\begin{itemize}
\item {Grp. gram.:v. i.}
\end{itemize}
\begin{itemize}
\item {Utilização:Prov.}
\end{itemize}
\begin{itemize}
\item {Utilização:beir.}
\end{itemize}
O mesmo que \textunderscore clamar\textunderscore .
\section{Gramasso}
\begin{itemize}
\item {Grp. gram.:m.}
\end{itemize}
\begin{itemize}
\item {Utilização:Prov.}
\end{itemize}
O mesmo que \textunderscore argamassa\textunderscore . Cf. Júl. Moreira, \textunderscore Estudos da Líng. Port.\textunderscore , I, 170.
\section{Gramata}
\begin{itemize}
\item {Grp. gram.:f.}
\end{itemize}
(V.barrilheira)
\section{Gramátego}
\begin{itemize}
\item {Grp. gram.:m.}
\end{itemize}
\begin{itemize}
\item {Utilização:Ant.}
\end{itemize}
Grammático.
\section{Gramática}
\begin{itemize}
\item {Grp. gram.:f.}
\end{itemize}
\begin{itemize}
\item {Proveniência:(Lat. \textunderscore grammatica\textunderscore )}
\end{itemize}
Estudo ou tratado dos factos da linguagem falada e escrita, e das leis naturaes que a regulam.
Livro, em que se expõem e se explicam as regras da linguagem.
\section{Gramatical}
\begin{itemize}
\item {Grp. gram.:adj.}
\end{itemize}
\begin{itemize}
\item {Proveniência:(Lat. \textunderscore grammaticalis\textunderscore )}
\end{itemize}
Relativo á gramática: \textunderscore análise gramatical\textunderscore .
Que está conforme á gramática.
\section{Gramaticalismo}
\begin{itemize}
\item {Grp. gram.:m.}
\end{itemize}
\begin{itemize}
\item {Utilização:Neol.}
\end{itemize}
\begin{itemize}
\item {Proveniência:(De \textunderscore gramatical\textunderscore )}
\end{itemize}
Escrúpulo exagerado na construcção gramatical das frases.
\section{Gramaticalmente}
\begin{itemize}
\item {Grp. gram.:adv.}
\end{itemize}
De modo gramatical.
Segundo as regras da gramática.
\section{Gramaticão}
\begin{itemize}
\item {Grp. gram.:m.}
\end{itemize}
\begin{itemize}
\item {Proveniência:(De \textunderscore gramático\textunderscore )}
\end{itemize}
Aquele que suppõe ser bom gramático.
Aquele que sabe só gramática.
\section{Gramaticar}
\begin{itemize}
\item {Grp. gram.:v. i.}
\end{itemize}
\begin{itemize}
\item {Utilização:Fam.}
\end{itemize}
Tratar questões de gramática.
Ensinar gramática.
\section{Gramaticista}
\begin{itemize}
\item {Grp. gram.:m.}
\end{itemize}
Aquele que é versado em gramática.
\section{Gramático}
\begin{itemize}
\item {Grp. gram.:adj.}
\end{itemize}
\begin{itemize}
\item {Grp. gram.:M.}
\end{itemize}
\begin{itemize}
\item {Proveniência:(Lat. \textunderscore grammaticus\textunderscore )}
\end{itemize}
Relativo á gramática.
Aquele que se dedica a estudos gramaticaes ou escreve sôbre gramática.
\section{Gramaticógrafo}
\begin{itemize}
\item {Grp. gram.:m.}
\end{itemize}
Aquele que escreve á cêrca de gramática. Cf. Júl. Ribeiro, \textunderscore Gramat.\textunderscore , (no prefácio).
(Do. gr. \textunderscore grammatike\textunderscore  + \textunderscore graphein\textunderscore )
\section{Gramaticologia}
\begin{itemize}
\item {Grp. gram.:f.}
\end{itemize}
\begin{itemize}
\item {Utilização:Neol.}
\end{itemize}
\begin{itemize}
\item {Proveniência:(Do gr. \textunderscore grammatike\textunderscore  + \textunderscore logos\textunderscore )}
\end{itemize}
Estudo científico da gramática.
\section{Gramaticológico}
\begin{itemize}
\item {Grp. gram.:adj.}
\end{itemize}
Relativo á gramaticologia.
\section{Gramaticólogo}
\begin{itemize}
\item {Grp. gram.:m.}
\end{itemize}
Aquele que se dedica á gramaticologia.
\section{Gramatiquice}
\begin{itemize}
\item {Grp. gram.:f.}
\end{itemize}
\begin{itemize}
\item {Proveniência:(De \textunderscore gramático\textunderscore )}
\end{itemize}
Rigorismo pedantesco em linguagem.
\section{Gramatista}
\begin{itemize}
\item {Grp. gram.:m.}
\end{itemize}
\begin{itemize}
\item {Proveniência:(Gr. \textunderscore grammatistes\textunderscore )}
\end{itemize}
Aquele que, entre os antigos, ensinava as crianças a lêr e a escrever.
\section{Gramatita}
\begin{itemize}
\item {Grp. gram.:f.}
\end{itemize}
\begin{itemize}
\item {Proveniência:(Do gr. \textunderscore gramme\textunderscore , linha)}
\end{itemize}
Variedade de rocha, parecida ao anfíbolo, e cuja côr oscila entre o branco nacarado e o pardo.
O mesmo que \textunderscore gramita\textunderscore ?
\section{Gramatologia}
\begin{itemize}
\item {Grp. gram.:f.}
\end{itemize}
Tratado das letras, alfabeto, silabação, leitura e escrita.
(Do. gr. \textunderscore grammata\textunderscore  + \textunderscore logos\textunderscore )
\section{Gramatológico}
\begin{itemize}
\item {Grp. gram.:adj.}
\end{itemize}
Relativo á gramatologia.
\section{Grambe}
\begin{itemize}
\item {Grp. gram.:m.}
\end{itemize}
Gênero de plantas crucíferas.
\section{Grameiras}
\begin{itemize}
\item {Grp. gram.:f. pl.}
\end{itemize}
Orifícios, que rodeiam os cadinhos nos fornos de fundir bronze.
\section{Gramelho}
\begin{itemize}
\item {fónica:mê}
\end{itemize}
\begin{itemize}
\item {Grp. gram.:m.}
\end{itemize}
\begin{itemize}
\item {Utilização:Prov.}
\end{itemize}
O mesmo que \textunderscore gramilo\textunderscore .
\section{Grâmia}
\begin{itemize}
\item {Grp. gram.:f.}
\end{itemize}
\begin{itemize}
\item {Utilização:Des.}
\end{itemize}
\begin{itemize}
\item {Proveniência:(Lat. \textunderscore gramiae\textunderscore )}
\end{itemize}
O mesmo que \textunderscore remela\textunderscore .
\section{Gramilho}
\begin{itemize}
\item {Grp. gram.:m.}
\end{itemize}
O mesmo que \textunderscore gramilo\textunderscore .
\section{Gramilo}
\begin{itemize}
\item {Grp. gram.:m.}
\end{itemize}
\begin{itemize}
\item {Utilização:Prov.}
\end{itemize}
\begin{itemize}
\item {Utilização:minh.}
\end{itemize}
\begin{itemize}
\item {Utilização:Prov.}
\end{itemize}
\begin{itemize}
\item {Proveniência:(Do cast. \textunderscore gramil\textunderscore ?)}
\end{itemize}
Peça de pau ou de ferro, em fórma de quarto de círculo, e com a qual se fixa a cravelha, do lado interior da porta, para que os gatunos ou os indiscretos não possam, do lado de fóra, levantar o fecho e abrir a porta.
Espécie de rêde, para apanhar pássaros. (Colhido em Foscôa)
\section{Gramíneas}
\begin{itemize}
\item {Grp. gram.:f. pl.}
\end{itemize}
\begin{itemize}
\item {Proveniência:(De \textunderscore gramíneo\textunderscore )}
\end{itemize}
Família de plantas monocotyledóneas, de fôlhas longas e estreitas em geral, á qual pertence o trigo, o arroz, o milho, e muitas outras.
\section{Gramíneo}
\begin{itemize}
\item {Grp. gram.:adj.}
\end{itemize}
\begin{itemize}
\item {Proveniência:(Lat. \textunderscore gramineus\textunderscore )}
\end{itemize}
Que tem a natureza da grama.
\section{Graminhar}
\begin{itemize}
\item {Grp. gram.:v.}
\end{itemize}
\begin{itemize}
\item {Utilização:t. Carp.}
\end{itemize}
Riscar com graminho.
Acertar ou rectificar com graminho.
\section{Graminheira}
\begin{itemize}
\item {Grp. gram.:f.  e  adj.}
\end{itemize}
\begin{itemize}
\item {Utilização:Prov.}
\end{itemize}
\begin{itemize}
\item {Proveniência:(De \textunderscore grama\textunderscore ^1. Cp. \textunderscore gramíneo\textunderscore )}
\end{itemize}
Planta ou raíz de planta, nociva á cultura e aproveitável para o gado.
\section{Graminho}
\begin{itemize}
\item {Grp. gram.:m.}
\end{itemize}
\begin{itemize}
\item {Proveniência:(Do cast. \textunderscore gramil\textunderscore )}
\end{itemize}
Instrumento de carpinteiro e marceneiro, para traçar riscos parallelos á borda das tábuas.
Um dos aprestos, na antiga fabricação das naus. Cf. Fern. Oliveira, \textunderscore Fábr. das Naus\textunderscore , 95.
\section{Graminícola}
\begin{itemize}
\item {Grp. gram.:adj.}
\end{itemize}
\begin{itemize}
\item {Utilização:Zool.}
\end{itemize}
\begin{itemize}
\item {Proveniência:(Do lat. \textunderscore gramen\textunderscore  + \textunderscore colere\textunderscore )}
\end{itemize}
Que vive na palha ou nos campos de cereaes.
\section{Graminifólio}
\begin{itemize}
\item {Grp. gram.:adj.}
\end{itemize}
\begin{itemize}
\item {Utilização:Bot.}
\end{itemize}
\begin{itemize}
\item {Proveniência:(Do lat. \textunderscore gramen\textunderscore  + \textunderscore folium\textunderscore )}
\end{itemize}
Que tem fôlhas semelhantes ás das gramíneas.
\section{Graminiforme}
\begin{itemize}
\item {Grp. gram.:adj.}
\end{itemize}
\begin{itemize}
\item {Proveniência:(Do lat. \textunderscore gramen\textunderscore  + \textunderscore forma\textunderscore )}
\end{itemize}
Semelhante ás gramíneas.
\section{Graminoso}
\begin{itemize}
\item {Grp. gram.:adj.}
\end{itemize}
\begin{itemize}
\item {Proveniência:(Lat. \textunderscore graminosus\textunderscore )}
\end{itemize}
Que tem muita grama.
\section{Gramipolpo}
\begin{itemize}
\item {Grp. gram.:m.}
\end{itemize}
(V.cárabo)
\section{Gramita}
\begin{itemize}
\item {Grp. gram.:f.}
\end{itemize}
\begin{itemize}
\item {Proveniência:(Do gr. \textunderscore gramme\textunderscore , linha)}
\end{itemize}
Nome de várias pedras, cujas côres representam linhas.
\section{Gramma}
\begin{itemize}
\item {Grp. gram.:m.}
\end{itemize}
\begin{itemize}
\item {Proveniência:(Gr. \textunderscore gramma\textunderscore )}
\end{itemize}
Pêso de um centímetro cúbico de água destillada.
Unidade das medidas de pêso, no systema métrico ou decimal.
\section{Grammantho}
\begin{itemize}
\item {Grp. gram.:m.}
\end{itemize}
\begin{itemize}
\item {Proveniência:(Do gr. \textunderscore gramma\textunderscore  + \textunderscore anthos\textunderscore )}
\end{itemize}
Gênero de plantas crassuláceas.
\section{Grammática}
\begin{itemize}
\item {Grp. gram.:f.}
\end{itemize}
\begin{itemize}
\item {Proveniência:(Lat. \textunderscore grammatica\textunderscore )}
\end{itemize}
Estudo ou tratado dos factos da linguagem falada e escrita, e das leis naturaes que a regulam.
Livro, em que se expõem e se explicam as regras da linguagem.
\section{Grammatical}
\begin{itemize}
\item {Grp. gram.:adj.}
\end{itemize}
\begin{itemize}
\item {Proveniência:(Lat. \textunderscore grammaticalis\textunderscore )}
\end{itemize}
Relativo á grammática: \textunderscore anályse grammatical\textunderscore .
Que está conforme á grammática.
\section{Grammaticalismo}
\begin{itemize}
\item {Grp. gram.:m.}
\end{itemize}
\begin{itemize}
\item {Utilização:Neol.}
\end{itemize}
\begin{itemize}
\item {Proveniência:(De \textunderscore grammatical\textunderscore )}
\end{itemize}
Escrúpulo exagerado na construcção grammatical das phrases.
\section{Grammaticalmente}
\begin{itemize}
\item {Grp. gram.:adv.}
\end{itemize}
De modo grammatical.
Segundo as regras da grammática.
\section{Grammaticão}
\begin{itemize}
\item {Grp. gram.:m.}
\end{itemize}
\begin{itemize}
\item {Proveniência:(De \textunderscore grammático\textunderscore )}
\end{itemize}
Aquelle que suppõe ser bom grammático.
Aquelle que sabe só grammática.
\section{Grammaticar}
\begin{itemize}
\item {Grp. gram.:v. i.}
\end{itemize}
\begin{itemize}
\item {Utilização:Fam.}
\end{itemize}
Tratar questões de grammática.
Ensinar grammática.
\section{Grammaticista}
\begin{itemize}
\item {Grp. gram.:m.}
\end{itemize}
Aquelle que é versado em grammática.
\section{Grammático}
\begin{itemize}
\item {Grp. gram.:adj.}
\end{itemize}
\begin{itemize}
\item {Grp. gram.:M.}
\end{itemize}
\begin{itemize}
\item {Proveniência:(Lat. \textunderscore grammaticus\textunderscore )}
\end{itemize}
Relativo á grammática.
Aquelle que se dedica a estudos grammaticaes ou escreve sôbre grammática.
\section{Grammaticógrapho}
\begin{itemize}
\item {Grp. gram.:m.}
\end{itemize}
Aquelle que escreve á cêrca de grammática. Cf. Júl. Ribeiro, \textunderscore Gramat.\textunderscore , (no prefácio).
(Do. gr. \textunderscore grammatike\textunderscore  + \textunderscore graphein\textunderscore )
\section{Grammaticologia}
\begin{itemize}
\item {Grp. gram.:f.}
\end{itemize}
\begin{itemize}
\item {Utilização:Neol.}
\end{itemize}
\begin{itemize}
\item {Proveniência:(Do gr. \textunderscore grammatike\textunderscore  + \textunderscore logos\textunderscore )}
\end{itemize}
Estudo scientífico da grammática.
\section{Grammaticológico}
\begin{itemize}
\item {Grp. gram.:adj.}
\end{itemize}
Relativo á grammaticologia.
\section{Grammaticólogo}
\begin{itemize}
\item {Grp. gram.:m.}
\end{itemize}
Aquelle que se dedica á grammaticologia.
\section{Grammatiquice}
\begin{itemize}
\item {Grp. gram.:f.}
\end{itemize}
\begin{itemize}
\item {Proveniência:(De \textunderscore grammático\textunderscore )}
\end{itemize}
Rigorismo pedantesco em linguagem.
\section{Grammatista}
\begin{itemize}
\item {Grp. gram.:m.}
\end{itemize}
\begin{itemize}
\item {Proveniência:(Gr. \textunderscore grammatistes\textunderscore )}
\end{itemize}
Aquelle que, entre os antigos, ensinava as crianças a lêr e a escrever.
\section{Grammatita}
\begin{itemize}
\item {Grp. gram.:f.}
\end{itemize}
\begin{itemize}
\item {Proveniência:(Do gr. \textunderscore gramme\textunderscore , linha)}
\end{itemize}
Variedade de rocha, parecida ao amphíbolo, e cuja côr oscilla entre o branco nacarado e o pardo.
O mesmo que \textunderscore grammita\textunderscore ?
\section{Grammatologia}
\begin{itemize}
\item {Grp. gram.:f.}
\end{itemize}
Tratado das letras, alphabeto, syllabação, leitura e escrita.
(Do. gr. \textunderscore grammata\textunderscore  + \textunderscore logos\textunderscore )
\section{Grammatológico}
\begin{itemize}
\item {Grp. gram.:adj.}
\end{itemize}
Relativo á grammatologia.
\section{Grammita}
\begin{itemize}
\item {Grp. gram.:f.}
\end{itemize}
\begin{itemize}
\item {Proveniência:(Do gr. \textunderscore gramme\textunderscore , linha)}
\end{itemize}
Nome de várias pedras, cujas côres representam linhas.
\section{Grammómetro}
\begin{itemize}
\item {Grp. gram.:m.}
\end{itemize}
\begin{itemize}
\item {Proveniência:(Do gr. \textunderscore gramme\textunderscore  + \textunderscore metron\textunderscore )}
\end{itemize}
Espécie de divisor mecânico, empregado em desenho.
\section{Grammónemo}
\begin{itemize}
\item {Grp. gram.:m.}
\end{itemize}
\begin{itemize}
\item {Proveniência:(Do gr. \textunderscore gramma\textunderscore  + \textunderscore nema\textunderscore )}
\end{itemize}
Gênero de infusórios, da fam. dos bacillários.
\section{Grammontino}
\begin{itemize}
\item {Grp. gram.:m.}
\end{itemize}
\begin{itemize}
\item {Proveniência:(De \textunderscore Grammont\textunderscore , n. p.)}
\end{itemize}
Frade de uma Ordem, fundada em França no século XI.
\section{Grammophone}
\begin{itemize}
\item {Grp. gram.:m.}
\end{itemize}
\begin{itemize}
\item {Proveniência:(Do gr. \textunderscore gramma\textunderscore  + \textunderscore phone\textunderscore )}
\end{itemize}
Phonógrapho aperfeiçoado, que reproduz os sons por meio de discos.
\section{Gramofone}
\begin{itemize}
\item {Grp. gram.:m.}
\end{itemize}
\begin{itemize}
\item {Proveniência:(Do gr. \textunderscore gramma\textunderscore  + \textunderscore phone\textunderscore )}
\end{itemize}
Fonógrafo aperfeiçoado, que reproduz os sons por meio de discos.
\section{Gramómetro}
\begin{itemize}
\item {Grp. gram.:m.}
\end{itemize}
\begin{itemize}
\item {Proveniência:(Do gr. \textunderscore gramme\textunderscore  + \textunderscore metron\textunderscore )}
\end{itemize}
Espécie de divisor mecânico, empregado em desenho.
\section{Gramondé}
\begin{itemize}
\item {Grp. gram.:m.}
\end{itemize}
Planta melastomácea do Brasil.
\section{Gramónemo}
\begin{itemize}
\item {Grp. gram.:m.}
\end{itemize}
\begin{itemize}
\item {Proveniência:(Do gr. \textunderscore gramma\textunderscore  + \textunderscore nema\textunderscore )}
\end{itemize}
Gênero de infusórios, da fam. dos bacilários.
\section{Gramonilhos}
\begin{itemize}
\item {Grp. gram.:m. pl.}
\end{itemize}
\begin{itemize}
\item {Utilização:Prov.}
\end{itemize}
\begin{itemize}
\item {Utilização:alent.}
\end{itemize}
O mesmo que \textunderscore gramosilhos\textunderscore .
\section{Gramonta}
\begin{itemize}
\item {Grp. gram.:f.}
\end{itemize}
\begin{itemize}
\item {Utilização:Prov.}
\end{itemize}
\begin{itemize}
\item {Utilização:trasm.}
\end{itemize}
O mesmo que \textunderscore glamonta\textunderscore .
\section{Gramontino}
\begin{itemize}
\item {Grp. gram.:m.}
\end{itemize}
\begin{itemize}
\item {Proveniência:(De \textunderscore Gramont\textunderscore , n. p.)}
\end{itemize}
Frade de uma Ordem, fundada em França no século XI.
\section{Gramosilhos}
\begin{itemize}
\item {Grp. gram.:m. pl.}
\end{itemize}
\begin{itemize}
\item {Utilização:Prov.}
\end{itemize}
\begin{itemize}
\item {Utilização:trasm.}
\end{itemize}
O mesmo que \textunderscore gambozinos\textunderscore .
\section{Grampa}
\begin{itemize}
\item {Grp. gram.:f.}
\end{itemize}
Instrumento náutico, para apertar, por meio de roscas ou parafusos.
(Cp. \textunderscore grampo\textunderscore )
\section{Grampo}
\begin{itemize}
\item {Grp. gram.:m.}
\end{itemize}
\begin{itemize}
\item {Proveniência:(Do al. \textunderscore krampe\textunderscore )}
\end{itemize}
Peça de metal, que segura e liga duas pedras numa construcção.
Haste de ferro ou madeira, para segurar peças em que se trabalha.
Peça no cano da espingarda, na qual se segura a mola da baioneta.
Grande escápula de parafuso, que se fixa nos tectos: \textunderscore candeeiro suspenso por um grampo\textunderscore .
\section{Gramponau}
\begin{itemize}
\item {Grp. gram.:adj.}
\end{itemize}
\begin{itemize}
\item {Utilização:Ant.}
\end{itemize}
\begin{itemize}
\item {Proveniência:(Do rad. de \textunderscore grampo\textunderscore . Cp. fr. \textunderscore cramponner\textunderscore )}
\end{itemize}
Defraudador.
Embusteiro; que engrampa.
\section{Gran}
\begin{itemize}
\item {Grp. gram.:adj.}
\end{itemize}
(Abrev. de \textunderscore grande\textunderscore ): \textunderscore um gran senhor\textunderscore .
\section{Gran}
\begin{itemize}
\item {Grp. gram.:f.}
\end{itemize}
\begin{itemize}
\item {Utilização:Prov.}
\end{itemize}
\begin{itemize}
\item {Utilização:Prov.}
\end{itemize}
\begin{itemize}
\item {Utilização:trasm.}
\end{itemize}
\begin{itemize}
\item {Proveniência:(Do lat. \textunderscore granum\textunderscore )}
\end{itemize}
Galha do uma espécie de carvalho (\textunderscore quercus coccifera\textunderscore ).
Insecto hemíptero, de côr vermelha, empregado em tinturaria e pharmácia, (\textunderscore coccus ilicis\textunderscore , Lin.).
Tecido, tinto com gran.
Côr escarlate.
O mesmo que \textunderscore graínha\textunderscore .
Moléstia do gado suino, que se manifesta por uma excrescência carnosa na bôca.
\section{Grana}
\begin{itemize}
\item {Grp. gram.:f.}
\end{itemize}
O mesmo que \textunderscore gran\textunderscore ^2, tecido. Cf. Garrett, \textunderscore Romanceiro\textunderscore , II, 132.
\section{Granacha}
\begin{itemize}
\item {Grp. gram.:f.}
\end{itemize}
\begin{itemize}
\item {Utilização:Ant.}
\end{itemize}
O mesmo que \textunderscore garnacha\textunderscore . Cf. \textunderscore Anat. Joc.\textunderscore , 29.
\section{Granada}
\begin{itemize}
\item {Grp. gram.:f.}
\end{itemize}
\begin{itemize}
\item {Utilização:Bras}
\end{itemize}
\begin{itemize}
\item {Proveniência:(Do lat. \textunderscore granatum\textunderscore )}
\end{itemize}
Projéctil, que tinha a fórma de roman, e que se enchia de pólvora a que se lançava fogo.
Pequena bomba.
Pedra fina, ferruginosa, que tem a fórma de um rhomboide de doze faces, e uma côr arroxeada.
Espécie de tecido de seda.
O mesmo que \textunderscore roman\textunderscore ^1.
\textunderscore Côr de granada\textunderscore , encarnado.
\section{Granadeiro}
\begin{itemize}
\item {Grp. gram.:m.}
\end{itemize}
\begin{itemize}
\item {Utilização:Fig.}
\end{itemize}
\begin{itemize}
\item {Proveniência:(De \textunderscore granada\textunderscore )}
\end{itemize}
Soldado, que lançava granadas.
Soldado, pertencente á companhia que vai na deanteira de cada regimento.
Homem alto, corpulento.
\section{Granadil}
\begin{itemize}
\item {Grp. gram.:m. e adj.}
\end{itemize}
O mesmo que \textunderscore granadino\textunderscore ^2. Cf. \textunderscore Cêrco de Mazagão\textunderscore .
\section{Granadilho}
\begin{itemize}
\item {Grp. gram.:m.}
\end{itemize}
\begin{itemize}
\item {Proveniência:(Do rad. de \textunderscore granada\textunderscore )}
\end{itemize}
Madeira de macacaúba.
\section{Granadina}
\begin{itemize}
\item {Grp. gram.:f.}
\end{itemize}
Mulher, natural ou habitante de Granada.
\section{Granadina}
\begin{itemize}
\item {Grp. gram.:f.}
\end{itemize}
\begin{itemize}
\item {Proveniência:(De \textunderscore granada\textunderscore )}
\end{itemize}
Tecido arrendado de seda geralmente escura.
Tecido de algodão, arrendado e fino.
\section{Granadino}
\begin{itemize}
\item {Grp. gram.:adj.}
\end{itemize}
\begin{itemize}
\item {Proveniência:(De \textunderscore granada\textunderscore )}
\end{itemize}
Que tem côr de roman.
\section{Granadino}
\begin{itemize}
\item {Grp. gram.:adj.}
\end{itemize}
\begin{itemize}
\item {Grp. gram.:M.}
\end{itemize}
Relativo a Granada.
Habitante de Granada.
\section{Granador}
\begin{itemize}
\item {Grp. gram.:m.}
\end{itemize}
Apparelho para granar pólvora.
\section{Granal}
\begin{itemize}
\item {Grp. gram.:adj.}
\end{itemize}
\begin{itemize}
\item {Grp. gram.:M.}
\end{itemize}
\begin{itemize}
\item {Utilização:Prov.}
\end{itemize}
\begin{itemize}
\item {Utilização:alent.}
\end{itemize}
\begin{itemize}
\item {Proveniência:(Do lat. \textunderscore granum\textunderscore )}
\end{itemize}
Relativo ao grão ou grãos.
Seara de grão-de-bico.
\section{Granalha}
\begin{itemize}
\item {Grp. gram.:f.}
\end{itemize}
\begin{itemize}
\item {Proveniência:(Do lat. \textunderscore granum\textunderscore )}
\end{itemize}
O mesmo que \textunderscore granulação\textunderscore .
Pequenos fragmentos, em fórma de grânulos ou de palhetas, a que se reduz o metal fundido, nas operações que precedem a amoedação. Cf. F. de Mendonça, \textunderscore Vocab. Téchn.\textunderscore 
\section{Granar}
\begin{itemize}
\item {Grp. gram.:v. t.}
\end{itemize}
\begin{itemize}
\item {Grp. gram.:V. i.}
\end{itemize}
\begin{itemize}
\item {Utilização:Bras}
\end{itemize}
\begin{itemize}
\item {Proveniência:(Do lat. \textunderscore granum\textunderscore )}
\end{itemize}
Dar fórma de grão a: \textunderscore granar pólvora\textunderscore .
Desenvolver-se em grãos (o milho).
\section{Granatária}
\begin{itemize}
\item {Grp. gram.:adj. f.}
\end{itemize}
\begin{itemize}
\item {Utilização:Bras}
\end{itemize}
Diz-se da balança de precisão, em pharmácia.
(Cp. \textunderscore granal\textunderscore )
\section{Granate}
\begin{itemize}
\item {Grp. gram.:m.}
\end{itemize}
\begin{itemize}
\item {Proveniência:(Do lat. \textunderscore granatum\textunderscore )}
\end{itemize}
Pedra fina, o mesmo que \textunderscore granada\textunderscore .
\section{Granáteas}
\begin{itemize}
\item {Grp. gram.:f. pl.}
\end{itemize}
O mesmo que \textunderscore punicáceas\textunderscore .
\section{Granatina}
\begin{itemize}
\item {Grp. gram.:f.}
\end{itemize}
\begin{itemize}
\item {Proveniência:(Do lat. \textunderscore granatum\textunderscore )}
\end{itemize}
Substância particular, descoberta por Landerer na roman.
\section{Gran-bêsta}
\begin{itemize}
\item {Grp. gram.:f.}
\end{itemize}
O mesmo que \textunderscore alce\textunderscore .
\section{Grança}
\begin{itemize}
\item {Grp. gram.:f.}
\end{itemize}
\begin{itemize}
\item {Utilização:Ant.}
\end{itemize}
\begin{itemize}
\item {Proveniência:(Do rad. do lat. \textunderscore granum\textunderscore )}
\end{itemize}
Alimpadura de cereaes.
O mesmo que \textunderscore garança\textunderscore . Cf. Filinto, XVII, 132.
\section{Grancha}
\begin{itemize}
\item {Grp. gram.:f.}
\end{itemize}
(Fórma ant. de \textunderscore granja\textunderscore )
\section{Gran-cruz}
\begin{itemize}
\item {Grp. gram.:f.}
\end{itemize}
\begin{itemize}
\item {Grp. gram.:M.}
\end{itemize}
Cruz decorativa, pendente de uma fita, e usada pelos primeiros dignitários de algumas Ordens de cavallaria.
Grau, correspondente a esta insignia.
Dignitário que tem a gran-cruz.
\section{Grandalhão}
\begin{itemize}
\item {Grp. gram.:adj.}
\end{itemize}
Muito grande.
\section{Grande}
\begin{itemize}
\item {Grp. gram.:adj.}
\end{itemize}
\begin{itemize}
\item {Grp. gram.:Loc. adv.}
\end{itemize}
\begin{itemize}
\item {Grp. gram.:M.}
\end{itemize}
\begin{itemize}
\item {Proveniência:(Lat. \textunderscore grandis\textunderscore )}
\end{itemize}
Que tem dimensões mais que ordinárias.
Vasto, extenso: \textunderscore grande território\textunderscore .
Profundo.
Comprido: \textunderscore vara grande\textunderscore .
Crescido, desenvolvido.
Duradoiro.
Importante: \textunderscore grande riqueza\textunderscore .
Poderoso: \textunderscore grande monarcha\textunderscore .
Ponderoso, grave: \textunderscore grandes razões\textunderscore .
Desmedido; descommunal.
Heróico.
Copioso.
Intenso: \textunderscore grande nevoeiro\textunderscore .
Bom; magnânimo.
Respeitável: \textunderscore grande sábio\textunderscore .
Magnífico: \textunderscore grande festa\textunderscore .
Numeroso: \textunderscore grande exêrcito\textunderscore .
Immenso.
Título de certos príncipes soberanos.
\textunderscore Á grande\textunderscore , ou \textunderscore de grande\textunderscore , com magnificência; com largueza.
Á maneira dos grandes.
Pessôa nobre, rica, poderosa: \textunderscore têr inveja aos grandes\textunderscore .
Aquelle ou aquillo que é grande.
\section{Grande-alexandre}
\begin{itemize}
\item {Grp. gram.:f.}
\end{itemize}
Pêra, a mesma que \textunderscore barbosa\textunderscore .
\section{Grandear}
\begin{itemize}
\item {Grp. gram.:v. i.}
\end{itemize}
\begin{itemize}
\item {Utilização:Des.}
\end{itemize}
\begin{itemize}
\item {Proveniência:(De \textunderscore grande\textunderscore )}
\end{itemize}
Bojar.
Crescer em volume.
\section{Gran-de-carrasco}
\begin{itemize}
\item {Grp. gram.:m.}
\end{itemize}
O mesmo que \textunderscore gran\textunderscore ^2, insecto.
\section{Grandeira}
\begin{itemize}
\item {Grp. gram.:f.}
\end{itemize}
Maço, para bater palha, nas estrebarias.
\section{Grandemente}
\begin{itemize}
\item {Grp. gram.:adv.}
\end{itemize}
\begin{itemize}
\item {Proveniência:(De \textunderscore grande\textunderscore )}
\end{itemize}
Com grandeza; muito.
\section{Grande-oblíquo}
\begin{itemize}
\item {Grp. gram.:adj.}
\end{itemize}
\begin{itemize}
\item {Utilização:Anat.}
\end{itemize}
Diz-se de um dos seis músculos oculares, a funcção do qual é fazer rolar o ôlho para o lado do canto interno.
\section{Grandessíssimo}
\begin{itemize}
\item {Grp. gram.:adj.}
\end{itemize}
\begin{itemize}
\item {Utilização:Pop.}
\end{itemize}
Muito grande. Cf. Júl. Dinís, \textunderscore Pupillas\textunderscore , 12; Garrett, \textunderscore Viagens\textunderscore , I, 23.
(Corr. de \textunderscore grandíssimo\textunderscore )
\section{Grandevo}
\begin{itemize}
\item {Grp. gram.:adj.}
\end{itemize}
\begin{itemize}
\item {Proveniência:(Lat. \textunderscore grandaevus\textunderscore )}
\end{itemize}
Muito velho, muito idoso. Cf. Fr. Barreto, \textunderscore Eneida\textunderscore , I, 29.
\section{Grandeza}
\begin{itemize}
\item {Grp. gram.:f.}
\end{itemize}
\begin{itemize}
\item {Grp. gram.:Pl.}
\end{itemize}
Qualidade daquelle ou daquillo que é grande.
Em mathemática, tudo que é susceptível de aumento ou deminuição.
Grau de intensidade da luz das estrêllas: \textunderscore estrêllas de primeira grandeza\textunderscore .
Título honorífico de grande do reino.
Bizarria.
Generosidade; ostentação.
Abundância.
Dignidades; bens materiaes: \textunderscore amar grandezas\textunderscore .
\section{Grande-zornal}
\begin{itemize}
\item {Grp. gram.:m.}
\end{itemize}
Ave, o mesmo que \textunderscore tordeira\textunderscore .
\section{Grandiloquência}
\begin{itemize}
\item {Grp. gram.:f.}
\end{itemize}
\begin{itemize}
\item {Proveniência:(De \textunderscore grandíloquo\textunderscore )}
\end{itemize}
Qualidade do estilo muito elevado, grandioso.
\section{Grandíloquo}
\begin{itemize}
\item {Grp. gram.:adj.}
\end{itemize}
\begin{itemize}
\item {Proveniência:(Lat. \textunderscore grandiloquus\textunderscore )}
\end{itemize}
Que tem linguagem nobre, elevada, pomposa.
\section{Grandiosamente}
\begin{itemize}
\item {Grp. gram.:adv.}
\end{itemize}
De modo grandioso.
\section{Grafar}
\begin{itemize}
\item {Grp. gram.:v. t.}
\end{itemize}
\begin{itemize}
\item {Proveniência:(Do gr. \textunderscore graphein\textunderscore )}
\end{itemize}
Dar fórma, por escrito, a (uma palavra).
Ortografar.
\section{Grafia}
\begin{itemize}
\item {Grp. gram.:f.}
\end{itemize}
\begin{itemize}
\item {Proveniência:(Do gr. \textunderscore graphein\textunderscore )}
\end{itemize}
Modo de escrever; ortografia.
\section{Grafiário}
\begin{itemize}
\item {Grp. gram.:adj.}
\end{itemize}
\begin{itemize}
\item {Grp. gram.:M.}
\end{itemize}
\begin{itemize}
\item {Proveniência:(Lat. \textunderscore graphiarius\textunderscore )}
\end{itemize}
Relativo ao ponteiro ou estilete, com que se escrevia.
Estojo, em que se guardava aquele estilete.
\section{Gráfica}
\begin{itemize}
\item {Grp. gram.:f.}
\end{itemize}
\begin{itemize}
\item {Proveniência:(De \textunderscore gráfico\textunderscore )}
\end{itemize}
Arte de grafar os vocábulos. Cf. Latino, \textunderscore Or. da Corôa\textunderscore , CCXXXV.
\section{Graficamente}
\begin{itemize}
\item {Grp. gram.:adv.}
\end{itemize}
De modo gráfico.
\section{Gráfico}
\begin{itemize}
\item {Grp. gram.:adj.}
\end{itemize}
\begin{itemize}
\item {Proveniência:(Gr. \textunderscore graphikos\textunderscore )}
\end{itemize}
Relativo a grafia.
Representado por desenho ou figuras geométricas.
Relativo á arte de reproduzir pela tipografia, gravura, estereotipia e processos correlativos.
\section{Gráfio}
\begin{itemize}
\item {Grp. gram.:m.}
\end{itemize}
\begin{itemize}
\item {Proveniência:(Lat. \textunderscore graphium\textunderscore )}
\end{itemize}
Espécie de ponteiro ou buril, com que os antigos escreviam em tábuas enceradas.
\section{Grafitação}
\begin{itemize}
\item {Grp. gram.:f.}
\end{itemize}
Acto de grafitar.
\section{Grafitar}
\begin{itemize}
\item {Grp. gram.:v. t.}
\end{itemize}
Converter em grafite.
\section{Grafite}
\begin{itemize}
\item {Grp. gram.:f.}
\end{itemize}
\begin{itemize}
\item {Proveniência:(Do gr. \textunderscore graphein\textunderscore )}
\end{itemize}
O mesmo que \textunderscore plumbagina\textunderscore .
\section{Grafítico}
\begin{itemize}
\item {Grp. gram.:adj.}
\end{itemize}
Relativo á grafite.
\section{Grafo...}
\begin{itemize}
\item {Grp. gram.:pref.}
\end{itemize}
\begin{itemize}
\item {Proveniência:(Do gr. \textunderscore graphein\textunderscore )}
\end{itemize}
(designativo de escrita, traço, gravura, etc.)
\section{Grafoestática}
\begin{itemize}
\item {Grp. gram.:f.}
\end{itemize}
(V.grafostática)
\section{Grafofone}
\begin{itemize}
\item {Grp. gram.:m.}
\end{itemize}
\begin{itemize}
\item {Proveniência:(Do gr. \textunderscore graphein\textunderscore  + \textunderscore phone\textunderscore )}
\end{itemize}
Fonógrafo aperfeiçoado, que reproduz os sons por meio de cilindros.
\section{Grafofónio}
\begin{itemize}
\item {Grp. gram.:m.}
\end{itemize}
\begin{itemize}
\item {Proveniência:(Do gr. \textunderscore graphein\textunderscore  + \textunderscore phone\textunderscore )}
\end{itemize}
Fonógrafo aperfeiçoado, que reproduz os sons por meio de cilindros.
\section{Grafognosia}
\begin{itemize}
\item {Grp. gram.:f.}
\end{itemize}
\begin{itemize}
\item {Proveniência:(Do gr. \textunderscore graphein\textunderscore  + \textunderscore gnosis\textunderscore )}
\end{itemize}
Arte de conhecer o autor de uma escrita, pela grafia usada por ele.
\section{Grafologia}
\begin{itemize}
\item {Grp. gram.:f.}
\end{itemize}
\begin{itemize}
\item {Proveniência:(De \textunderscore grafólogo\textunderscore )}
\end{itemize}
Arte ou suposta teoria de quem é grafólogo.
Ciência geral da escrita, considerada materialmente, isto é, na sua fórma, posição, dimensões e noutras circunstâncias normaes e anormaes.
\section{Grafológico}
\begin{itemize}
\item {Grp. gram.:adj.}
\end{itemize}
Relativo á grafologia.
\section{Grafólogo}
\begin{itemize}
\item {Grp. gram.:m.}
\end{itemize}
\begin{itemize}
\item {Proveniência:(Do gr. \textunderscore graphein\textunderscore  + \textunderscore logos\textunderscore )}
\end{itemize}
Aquele que, pelo traçado de uma escrita, procura ou presume conhecer o carácter ou índole de quem escreveu.
\section{Grafómetro}
\begin{itemize}
\item {Grp. gram.:m.}
\end{itemize}
\begin{itemize}
\item {Proveniência:(Do gr. \textunderscore graphein\textunderscore  + \textunderscore metron\textunderscore )}
\end{itemize}
Instrumento, com que se medem ângulos sôbre um terreno.
\section{Grafonomia}
\begin{itemize}
\item {Grp. gram.:f.}
\end{itemize}
\begin{itemize}
\item {Proveniência:(Do gr. \textunderscore graphein\textunderscore  + \textunderscore nomos\textunderscore )}
\end{itemize}
Estudo das fórmas, que um indivíduo pratica na sua grafia.
\section{Grafopsicologia}
\begin{itemize}
\item {fónica:co}
\end{itemize}
\begin{itemize}
\item {Grp. gram.:f.}
\end{itemize}
\begin{itemize}
\item {Proveniência:(Do gr. \textunderscore graphein\textunderscore  + \textunderscore psukhe\textunderscore  + \textunderscore logos\textunderscore )}
\end{itemize}
Estudo psicológico de um indivíduo, pela observação da fórma da sua letra.
\section{Grafostática}
\begin{itemize}
\item {Grp. gram.:f.}
\end{itemize}
\begin{itemize}
\item {Proveniência:(De \textunderscore grafo...\textunderscore  + \textunderscore estática\textunderscore )}
\end{itemize}
Aplicação da Geometria á resolução de problemas da Mecânica (estática).
\section{Grandiosidade}
\begin{itemize}
\item {Grp. gram.:f.}
\end{itemize}
Qualidade daquillo que é grandioso.
\section{Grandioso}
\begin{itemize}
\item {Grp. gram.:adj.}
\end{itemize}
\begin{itemize}
\item {Proveniência:(Do lat. \textunderscore grandis\textunderscore )}
\end{itemize}
Muito grande; elevado.
Nobre; magnificente.
\section{Grandíssimo}
\begin{itemize}
\item {Grp. gram.:adj.}
\end{itemize}
Muito grande.
\section{Grandor}
\begin{itemize}
\item {Grp. gram.:m.}
\end{itemize}
\begin{itemize}
\item {Utilização:Ant.}
\end{itemize}
O mesmo que \textunderscore grandeza\textunderscore .
\section{Grândula}
\begin{itemize}
\item {Grp. gram.:f.}
\end{itemize}
\begin{itemize}
\item {Utilização:Prov.}
\end{itemize}
\begin{itemize}
\item {Utilização:alg.}
\end{itemize}
O mesmo que \textunderscore glândula\textunderscore .
\section{Grandulim}
\begin{itemize}
\item {Grp. gram.:m.}
\end{itemize}
\begin{itemize}
\item {Proveniência:(Do rad. de \textunderscore grande\textunderscore )}
\end{itemize}
Nome, que se deu a uma ave da Arábia, talvez o abestruz.
\section{Gran-duquesa}
\begin{itemize}
\item {Grp. gram.:f.}
\end{itemize}
Soberana de um grão-ducado.
Mulher de um grão-duque.
\section{Grandura}
\begin{itemize}
\item {Grp. gram.:f.}
\end{itemize}
\begin{itemize}
\item {Utilização:Pop.}
\end{itemize}
O mesmo que \textunderscore grandeza\textunderscore .
\section{Grané}
\begin{itemize}
\item {Grp. gram.:m.}
\end{itemize}
\begin{itemize}
\item {Utilização:Gír.}
\end{itemize}
Cavallo.
(Cp. \textunderscore grani\textunderscore )
\section{Graneado}
\begin{itemize}
\item {Grp. gram.:m.}
\end{itemize}
\begin{itemize}
\item {Utilização:Bras}
\end{itemize}
Espécie de coiro da Rússia, para calçado, cobertura de carruagens, etc.
\section{Granel}
\begin{itemize}
\item {Grp. gram.:m.}
\end{itemize}
\begin{itemize}
\item {Grp. gram.:Loc. adv.}
\end{itemize}
Celleiro, tulha.
Trecho de composição typográphica, antes de paginada.
\textunderscore A granel\textunderscore , em montão; á mistura.
Desalinhadamente.
(Cp. lat. \textunderscore granarius\textunderscore )
\section{Grangará}
\begin{itemize}
\item {Grp. gram.:m.}
\end{itemize}
\begin{itemize}
\item {Utilização:Bras}
\end{itemize}
\begin{itemize}
\item {Utilização:chul. de Pernambuco.}
\end{itemize}
Pessôa alta e de pouco valor.
\section{Grangear}
\textunderscore v. t.\textunderscore  (e der.)
(V. \textunderscore granjear\textunderscore , etc.)
\section{Grangeia}
\begin{itemize}
\item {Grp. gram.:f.}
\end{itemize}
\begin{itemize}
\item {Proveniência:(Do fr. \textunderscore dragée\textunderscore )}
\end{itemize}
Confeito miúdo.
Pílula ou grânulo medicamentoso, preparado num tacho com xarope aromático.
\section{Granhar}
\begin{itemize}
\item {Grp. gram.:v. t.}
\end{itemize}
\begin{itemize}
\item {Utilização:Ant.}
\end{itemize}
Abandonar.
Desprezar.
\section{Grani}
\begin{itemize}
\item {Grp. gram.:m.}
\end{itemize}
\begin{itemize}
\item {Utilização:Gír.}
\end{itemize}
Égua.
(Do cigano de Espanha \textunderscore grasni\textunderscore , égua)
\section{Granido}
\begin{itemize}
\item {Grp. gram.:m.}
\end{itemize}
\begin{itemize}
\item {Proveniência:(De \textunderscore granir\textunderscore )}
\end{itemize}
Desenho ou gravura a pontos miúdos.
\section{Granidor}
\begin{itemize}
\item {Grp. gram.:m.}
\end{itemize}
Espécie de caixa, em que se colloca a pedra lithográphica, para granir.
\section{Granífero}
\begin{itemize}
\item {Grp. gram.:adj.}
\end{itemize}
\begin{itemize}
\item {Proveniência:(Lat. \textunderscore granifer\textunderscore )}
\end{itemize}
Que produz grãos.
\section{Graniforme}
\begin{itemize}
\item {Grp. gram.:adj.}
\end{itemize}
\begin{itemize}
\item {Proveniência:(Do lat. \textunderscore granum\textunderscore  + \textunderscore forma\textunderscore )}
\end{itemize}
Que tem fórma de grão.
\section{Granir}
\begin{itemize}
\item {Grp. gram.:v. t.}
\end{itemize}
\begin{itemize}
\item {Proveniência:(Do lat. \textunderscore granum\textunderscore )}
\end{itemize}
Desenhar ou gravar a pontos miúdos.
Limpar (pedra lithográphica).
\section{Granisé}
\begin{itemize}
\item {Grp. gram.:adj.}
\end{itemize}
\begin{itemize}
\item {Utilização:Prov.}
\end{itemize}
\begin{itemize}
\item {Utilização:dur.}
\end{itemize}
Diz-se de uma variedade de gallinha, de raça de Guernesey.
\section{Granita}
\begin{itemize}
\item {Grp. gram.:f.}
\end{itemize}
\begin{itemize}
\item {Proveniência:(Do lat. \textunderscore granum\textunderscore )}
\end{itemize}
Glóbulo de qualquer substância molle.
Excremento de cabras e de outros animaes.
Bagulho ou graínha, semente de uva.
\section{Granitar}
\begin{itemize}
\item {Grp. gram.:v. t.}
\end{itemize}
Dar fórma de granita a.
Reduzir a granitas.
\section{Granítico}
\begin{itemize}
\item {Grp. gram.:adj.}
\end{itemize}
Que tem a natureza do granito.
\section{Granito}
\begin{itemize}
\item {Grp. gram.:m.}
\end{itemize}
\begin{itemize}
\item {Proveniência:(Do lat. \textunderscore granum\textunderscore )}
\end{itemize}
Pequeno grão.
Rocha granular, em crystaes mais ou menos volumosos e aggregados.
Espécie de aguardente, em que entra a essência do anis.
\section{Granito}
\begin{itemize}
\item {Grp. gram.:m.}
\end{itemize}
\begin{itemize}
\item {Utilização:Bras. de Minas}
\end{itemize}
Sol ou calor intenso, depois de muitos dias de chuva.
(Alter. de \textunderscore veranico\textunderscore ?)
\section{Granitoide}
\begin{itemize}
\item {Grp. gram.:adj.}
\end{itemize}
\begin{itemize}
\item {Proveniência:(De \textunderscore granito\textunderscore  + gr. \textunderscore eidos\textunderscore )}
\end{itemize}
Semelhante ao granito.
\section{Granitoso}
\begin{itemize}
\item {Grp. gram.:adj.}
\end{itemize}
O mesmo que \textunderscore granítico\textunderscore .
\section{Granívoro}
\begin{itemize}
\item {Grp. gram.:adj.}
\end{itemize}
\begin{itemize}
\item {Grp. gram.:M.}
\end{itemize}
\begin{itemize}
\item {Proveniência:(Do lat. \textunderscore granum\textunderscore  + \textunderscore vorare\textunderscore )}
\end{itemize}
Que se alimenta de grãos ou sementes.
Animal, que se alimenta de grãos ou sementes.
\section{Granizada}
\begin{itemize}
\item {Grp. gram.:f.}
\end{itemize}
\begin{itemize}
\item {Utilização:Fig.}
\end{itemize}
\begin{itemize}
\item {Proveniência:(De \textunderscore granizo\textunderscore )}
\end{itemize}
Quantidade de granizo.
Aquillo que cái em abundância, á semelhança do granizo.
\section{Granizar}
\begin{itemize}
\item {Grp. gram.:v. t.}
\end{itemize}
\begin{itemize}
\item {Proveniência:(Do lat. \textunderscore granum\textunderscore )}
\end{itemize}
Granitar.
Dar fórma granular a.
\section{Granizar}
\begin{itemize}
\item {Grp. gram.:v. i.}
\end{itemize}
\begin{itemize}
\item {Grp. gram.:V. t.}
\end{itemize}
Caír granizo.
Caír como granizo:«\textunderscore as pedradas granizam nestas carvalheiras\textunderscore ». Camillo, \textunderscore Narcót.\textunderscore , 262.
Atirar como granizo contra:«\textunderscore gastei a pólvora e o chumbo, granizando o lobo\textunderscore ». Camillo, \textunderscore Nov. do Minho\textunderscore , IX, 17.
\section{Granizo}
\begin{itemize}
\item {Grp. gram.:m.}
\end{itemize}
\begin{itemize}
\item {Utilização:Fig.}
\end{itemize}
\begin{itemize}
\item {Proveniência:(Do rad. do lat. \textunderscore granum\textunderscore )}
\end{itemize}
Saraiva; chuva de pedra.
Porção de coisas miúdas, que cáem ou são expellidas.
\section{Granja}
\begin{itemize}
\item {Grp. gram.:f.}
\end{itemize}
\begin{itemize}
\item {Proveniência:(Do lat. \textunderscore granea\textunderscore , talvez por intermédio do fr. \textunderscore grange\textunderscore )}
\end{itemize}
Prédio rústico, que se cultiva.
Casal.
Edifício, em que se recolhem os frutos de uma herdade.
Abegoaria.
\section{Granjaria}
\begin{itemize}
\item {Grp. gram.:f.}
\end{itemize}
Reunião de granjas.
\section{Granjeador}
\begin{itemize}
\item {Grp. gram.:m.  e  adj.}
\end{itemize}
O que granjeia.
\section{Granjear}
\begin{itemize}
\item {Grp. gram.:v. t.}
\end{itemize}
\begin{itemize}
\item {Utilização:Ant.}
\end{itemize}
\begin{itemize}
\item {Proveniência:(De \textunderscore granja\textunderscore )}
\end{itemize}
Amanhar ou cultivar: \textunderscore granjear herdades\textunderscore .
Adquirir; obter com trabalho ou esfôrço: \textunderscore granjear meios de vida\textunderscore .
Esmiuçar.
\section{Granjearia}
\begin{itemize}
\item {Grp. gram.:f.}
\end{itemize}
\begin{itemize}
\item {Utilização:Des.}
\end{itemize}
\begin{itemize}
\item {Proveniência:(De \textunderscore granjear\textunderscore )}
\end{itemize}
Cultura; lavoira.
Granja.
Producto; lucro. Cf. \textunderscore Eufrosina\textunderscore , 45.
\section{Granjeeiro}
\begin{itemize}
\item {Grp. gram.:m.}
\end{itemize}
O mesmo que \textunderscore granjeiro\textunderscore . Cf. Garrett, \textunderscore Viagens\textunderscore .
\section{Granjeio}
\begin{itemize}
\item {Grp. gram.:m.}
\end{itemize}
\begin{itemize}
\item {Utilização:Ext.}
\end{itemize}
\begin{itemize}
\item {Utilização:Fig.}
\end{itemize}
Acto de granjear.
Cultura, lavoira.
Colheita de productos agricolas.
Lucro.
Trabalho para commodidades ou interesses.
\section{Granjeiro}
\begin{itemize}
\item {Grp. gram.:m.}
\end{itemize}
Cultivador de granja.
Agricultor.
Rendeiro.
\section{Granjola}
\begin{itemize}
\item {Grp. gram.:m. ,  f.  e  adj.}
\end{itemize}
\begin{itemize}
\item {Utilização:Pop.}
\end{itemize}
\begin{itemize}
\item {Proveniência:(Do rad. de \textunderscore grande\textunderscore )}
\end{itemize}
Pessôa corpulenta.
\section{Granjola}
\begin{itemize}
\item {Grp. gram.:m.}
\end{itemize}
\begin{itemize}
\item {Utilização:Deprec.}
\end{itemize}
\begin{itemize}
\item {Proveniência:(De \textunderscore Granja\textunderscore , n. p.)}
\end{itemize}
Membro de um partido político, que se reorganizou em reuniões celebradas na Granja, junto ao Porto.
\section{Granjolada}
\begin{itemize}
\item {Grp. gram.:f.}
\end{itemize}
O mesmo que \textunderscore granjolice\textunderscore .
\section{Granjolice}
\begin{itemize}
\item {Grp. gram.:f.}
\end{itemize}
Patifaria, fajardice.
\section{Granoso}
\begin{itemize}
\item {Grp. gram.:adj.}
\end{itemize}
\begin{itemize}
\item {Proveniência:(Lat. \textunderscore granosus\textunderscore )}
\end{itemize}
Que tem grãos.
\section{Granulação}
\begin{itemize}
\item {Grp. gram.:f.}
\end{itemize}
\begin{itemize}
\item {Proveniência:(Lat. \textunderscore granulatio\textunderscore )}
\end{itemize}
Acto ou effeito de granular^2.
Granito.
Porção de glóbulos, na superfície de um órgão ou de uma membrana.
\section{Granulagem}
\begin{itemize}
\item {Grp. gram.:f.}
\end{itemize}
Acto de granular^2.
\section{Granular}
\begin{itemize}
\item {Grp. gram.:adj.}
\end{itemize}
\begin{itemize}
\item {Proveniência:(De \textunderscore grânulo\textunderscore )}
\end{itemize}
Semelhante ao grão, na fórma.
Composto de pequenos grãos.
\section{Granular}
\begin{itemize}
\item {Grp. gram.:v. t.}
\end{itemize}
Dar fórma de grânulo a.
\section{Granuliforme}
\begin{itemize}
\item {Grp. gram.:adj.}
\end{itemize}
\begin{itemize}
\item {Proveniência:(De \textunderscore grânulo\textunderscore  + \textunderscore fórma\textunderscore )}
\end{itemize}
Que tem fórma de grânulo ou de grânulos aggregados.
\section{Granulite}
\begin{itemize}
\item {Grp. gram.:f.}
\end{itemize}
Mineral que, só com o microscópio, se distingue do granito.
(Cp. \textunderscore grânulo\textunderscore )
\section{Granulítico}
\begin{itemize}
\item {Grp. gram.:adj.}
\end{itemize}
Em que há granulite.
\section{Grânulo}
\begin{itemize}
\item {Grp. gram.:m.}
\end{itemize}
\begin{itemize}
\item {Proveniência:(Lat. \textunderscore granulum\textunderscore )}
\end{itemize}
Pequeno grão.
Glóbulo.
Pequena pílula.
Cada uma das pequenas saliências de uma superficie áspera.
\section{Granuloma}
\begin{itemize}
\item {Grp. gram.:m.}
\end{itemize}
\begin{itemize}
\item {Proveniência:(De \textunderscore grânulo\textunderscore )}
\end{itemize}
Tumor, formado de tecido granuloso.
\section{Granulosidade}
\begin{itemize}
\item {Grp. gram.:f.}
\end{itemize}
Qualidade daquillo que é granuloso.
\section{Granuloso}
\begin{itemize}
\item {Grp. gram.:adj.}
\end{itemize}
\begin{itemize}
\item {Proveniência:(De \textunderscore grânulo\textunderscore )}
\end{itemize}
Formado de grânulos.
Que tem a superfície áspera.
Em que há granulações.
\section{Granza}
\begin{itemize}
\item {Grp. gram.:f.}
\end{itemize}
\begin{itemize}
\item {Proveniência:(Do rad. do lat. \textunderscore granum\textunderscore )}
\end{itemize}
Planta rubiácea; ruiva.
\section{Granzal}
\begin{itemize}
\item {Grp. gram.:m.}
\end{itemize}
\begin{itemize}
\item {Proveniência:(Do rad. do lat. \textunderscore granum\textunderscore )}
\end{itemize}
Terreno, onde crescem granzas.
Terreno, semeado de grãos de bico.
\section{Grão}
\begin{itemize}
\item {Grp. gram.:m.}
\end{itemize}
\begin{itemize}
\item {Utilização:Ext.}
\end{itemize}
\begin{itemize}
\item {Utilização:Pop.}
\end{itemize}
\begin{itemize}
\item {Utilização:Gír.}
\end{itemize}
\begin{itemize}
\item {Utilização:Gír.}
\end{itemize}
\begin{itemize}
\item {Proveniência:(Do lat. \textunderscore granum\textunderscore )}
\end{itemize}
Bago de cereaes.
Fruto ou semente de trigo ou de outras plantas.
Pêso antigo, que correspondia proximamente á vigésima parte de um grama.
Glóbulo.
Pequeno corpo arredondado.
Testículo.
Arroz.
Cruzado novo.
\section{Grão}
\begin{itemize}
\item {Grp. gram.:adj.}
\end{itemize}
(Fórma abreviada de \textunderscore grande\textunderscore )
\section{Grão-cruz}
\begin{itemize}
\item {Grp. gram.:m.}
\end{itemize}
O mesmo que \textunderscore gran-cruz\textunderscore .
\section{Grão-de-bico}
\begin{itemize}
\item {Grp. gram.:m.}
\end{itemize}
Planta leguminosa, (\textunderscore ciser arretinum\textunderscore ).
O fruto dessa planta.
\section{Grão-de-gallo}
\begin{itemize}
\item {Grp. gram.:m.}
\end{itemize}
\begin{itemize}
\item {Utilização:Bras}
\end{itemize}
Fruto comestível, (\textunderscore lucuma torta\textunderscore , De-Candolle).
\section{Grão-de-pulha}
\begin{itemize}
\item {Grp. gram.:m}
\end{itemize}
Planta leguminosa da Índia portuguesa, (\textunderscore phaseolus mungo\textunderscore , Lin.).
\section{Grão-ducado}
\begin{itemize}
\item {Grp. gram.:m.}
\end{itemize}
País, governado por um grão-duque: \textunderscore o grão-ducado de Luxemburgo\textunderscore .
\section{Grão-ducal}
\begin{itemize}
\item {Grp. gram.:adj.}
\end{itemize}
Relativo a grão-duque ou a grão-ducado.
\section{Grão-duque}
\begin{itemize}
\item {Grp. gram.:m.}
\end{itemize}
Título de alguns Príncipes soberanos.
Príncipe da família imperial russa.
Ave nocturna, o mesmo que \textunderscore bufo\textunderscore ^2.
\section{Grão-mestrado}
\begin{itemize}
\item {Grp. gram.:m.}
\end{itemize}
Dignidade ou cargo de grão-mestre.
\section{Grão-mestre}
\begin{itemize}
\item {Grp. gram.:m.}
\end{itemize}
O mais alto dignitário de uma Ordem de cavallaria, da maçonaria de uma região, etc.
\section{Grão-tinhoso}
\begin{itemize}
\item {Grp. gram.:m.}
\end{itemize}
\begin{itemize}
\item {Utilização:Pop.}
\end{itemize}
O mesmo que \textunderscore diabo\textunderscore .
\section{Grão-vizir}
\begin{itemize}
\item {Grp. gram.:m.}
\end{itemize}
Primeiro ministro do Império otomano.
\section{Grãozeiro}
\begin{itemize}
\item {Grp. gram.:m.}
\end{itemize}
\begin{itemize}
\item {Utilização:T. da Bairrada}
\end{itemize}
O mesmo que \textunderscore graeiro\textunderscore .
\section{Grapa}
\begin{itemize}
\item {Grp. gram.:f.}
\end{itemize}
Ferida, na deanteira das curvas e na traseira dos braços da bêsta.
\section{Grapecique}
\begin{itemize}
\item {Grp. gram.:m.}
\end{itemize}
\begin{itemize}
\item {Utilização:Bras}
\end{itemize}
Árvore silvestre.
\section{Graphar}
\begin{itemize}
\item {Grp. gram.:v. t.}
\end{itemize}
\begin{itemize}
\item {Proveniência:(Do gr. \textunderscore graphein\textunderscore )}
\end{itemize}
Dar fórma, por escrito, a (uma palavra).
Orthographar.
\section{Graphia}
\begin{itemize}
\item {Grp. gram.:f.}
\end{itemize}
\begin{itemize}
\item {Proveniência:(Do gr. \textunderscore graphein\textunderscore )}
\end{itemize}
Modo de escrever; orthographia.
\section{Graphiário}
\begin{itemize}
\item {Grp. gram.:adj.}
\end{itemize}
\begin{itemize}
\item {Grp. gram.:M.}
\end{itemize}
\begin{itemize}
\item {Proveniência:(Lat. \textunderscore graphiarius\textunderscore )}
\end{itemize}
Relativo ao ponteiro ou estilete, com que se escrevia.
Estojo, em que se guardava aquelle estilete.
\section{Gráphica}
\begin{itemize}
\item {Grp. gram.:f.}
\end{itemize}
\begin{itemize}
\item {Proveniência:(De \textunderscore gráphico\textunderscore )}
\end{itemize}
Arte de graphar os vocábulos. Cf. Latino, \textunderscore Or. da Corôa\textunderscore , CCXXXV.
\section{Graphicamente}
\begin{itemize}
\item {Grp. gram.:adv.}
\end{itemize}
De modo gráphico.
\section{Gráphico}
\begin{itemize}
\item {Grp. gram.:adj.}
\end{itemize}
\begin{itemize}
\item {Proveniência:(Gr. \textunderscore graphikos\textunderscore )}
\end{itemize}
Relativo a graphia.
Representado por desenho ou figuras geométricas.
Relativo á arte de reproduzir pela typographia, gravura, estereotypia e processos correlativos.
\section{Gráphio}
\begin{itemize}
\item {Grp. gram.:m.}
\end{itemize}
\begin{itemize}
\item {Proveniência:(Lat. \textunderscore graphium\textunderscore )}
\end{itemize}
Espécie de ponteiro ou buril, com que os antigos escreviam em tábuas enceradas.
\section{Graphitação}
\begin{itemize}
\item {Grp. gram.:f.}
\end{itemize}
Acto de graphitar.
\section{Graphitar}
\begin{itemize}
\item {Grp. gram.:v. t.}
\end{itemize}
Converter em graphite.
\section{Graphite}
\begin{itemize}
\item {Grp. gram.:f.}
\end{itemize}
\begin{itemize}
\item {Proveniência:(Do gr. \textunderscore graphein\textunderscore )}
\end{itemize}
O mesmo que \textunderscore plumbagina\textunderscore .
\section{Graphítico}
\begin{itemize}
\item {Grp. gram.:adj.}
\end{itemize}
Relativo á graphite.
\section{Grapho...}
\begin{itemize}
\item {Grp. gram.:pref.}
\end{itemize}
\begin{itemize}
\item {Proveniência:(Do gr. \textunderscore graphein\textunderscore )}
\end{itemize}
(designativo de escrita, traço, gravura, etc.)
\section{Graphoestática}
\begin{itemize}
\item {Grp. gram.:f.}
\end{itemize}
(V.graphostática)
\section{Graphognosia}
\begin{itemize}
\item {Grp. gram.:f.}
\end{itemize}
\begin{itemize}
\item {Proveniência:(Do gr. \textunderscore graphein\textunderscore  + \textunderscore gnosis\textunderscore )}
\end{itemize}
Arte de conhecer o autor de uma escrita, pela graphia usada por elle.
\section{Graphologia}
\begin{itemize}
\item {Grp. gram.:f.}
\end{itemize}
\begin{itemize}
\item {Proveniência:(De \textunderscore graphólogo\textunderscore )}
\end{itemize}
Arte ou supposta theoria de quem é graphólogo.
Sciência geral da escrita, considerada materialmente, isto é, na sua fórma, posição, dimensões e noutras circunstâncias normaes e anormaes.
\section{Graphológico}
\begin{itemize}
\item {Grp. gram.:adj.}
\end{itemize}
Relativo á graphologia.
\section{Graphólogo}
\begin{itemize}
\item {Grp. gram.:m.}
\end{itemize}
\begin{itemize}
\item {Proveniência:(Do gr. \textunderscore graphein\textunderscore  + \textunderscore logos\textunderscore )}
\end{itemize}
Aquelle que, pelo traçado de uma escrita, procura ou presume conhecer o carácter ou índole de quem escreveu.
\section{Graphómetro}
\begin{itemize}
\item {Grp. gram.:m.}
\end{itemize}
\begin{itemize}
\item {Proveniência:(Do gr. \textunderscore graphein\textunderscore  + \textunderscore metron\textunderscore )}
\end{itemize}
Instrumento, com que se medem ângulos sôbre um terreno.
\section{Graphonomia}
\begin{itemize}
\item {Grp. gram.:f.}
\end{itemize}
\begin{itemize}
\item {Proveniência:(Do gr. \textunderscore graphein\textunderscore  + \textunderscore nomos\textunderscore )}
\end{itemize}
Estudo das fórmas, que um indivíduo pratica na sua graphia.
\section{Graphophone}
\begin{itemize}
\item {Grp. gram.:m.}
\end{itemize}
\begin{itemize}
\item {Proveniência:(Do gr. \textunderscore graphein\textunderscore  + \textunderscore phone\textunderscore )}
\end{itemize}
Phonógrapho aperfeiçoado, que reproduz os sons por meio de cylindros.
\section{Graphopsychologia}
\begin{itemize}
\item {fónica:co}
\end{itemize}
\begin{itemize}
\item {Grp. gram.:f.}
\end{itemize}
\begin{itemize}
\item {Proveniência:(Do gr. \textunderscore graphein\textunderscore  + \textunderscore psukhe\textunderscore  + \textunderscore logos\textunderscore )}
\end{itemize}
Estudo psychológico de um indivíduo, pela observação da fórma da sua letra.
\section{Graphostática}
\begin{itemize}
\item {Grp. gram.:f.}
\end{itemize}
\begin{itemize}
\item {Proveniência:(De \textunderscore grapho...\textunderscore  + \textunderscore estática\textunderscore )}
\end{itemize}
Applicação da Geometria á resolução de problemas da Mecânica (estática).
\section{Grapiapunha}
\begin{itemize}
\item {Grp. gram.:f.}
\end{itemize}
Arvore leguminosa do Brasil.
\section{Grapso}
\begin{itemize}
\item {Grp. gram.:m.}
\end{itemize}
\begin{itemize}
\item {Proveniência:(Do gr. \textunderscore graphein\textunderscore ?)}
\end{itemize}
Gênero de crustáceos decápodes.
\section{Grapsoide}
\begin{itemize}
\item {Grp. gram.:adj.}
\end{itemize}
\begin{itemize}
\item {Grp. gram.:M. pl.}
\end{itemize}
\begin{itemize}
\item {Proveniência:(De \textunderscore grapso\textunderscore  + gr. \textunderscore eidos\textunderscore )}
\end{itemize}
Semelhante ao grapso.
Tríbo de crustáceos, que têm por typo o grapso.
\section{Graptólithos}
\begin{itemize}
\item {Grp. gram.:m. pl.}
\end{itemize}
\begin{itemize}
\item {Proveniência:(Do gr. \textunderscore graptos\textunderscore  + \textunderscore lithos\textunderscore )}
\end{itemize}
Ordem de celenterados nadadores.
\section{Graptólitos}
\begin{itemize}
\item {Grp. gram.:m. pl.}
\end{itemize}
\begin{itemize}
\item {Proveniência:(Do gr. \textunderscore graptos\textunderscore  + \textunderscore lithos\textunderscore )}
\end{itemize}
Ordem de celenterados nadadores.
\section{Grasnada}
\begin{itemize}
\item {Grp. gram.:f.}
\end{itemize}
\begin{itemize}
\item {Utilização:Fig.}
\end{itemize}
Acto ou effeito de grasnar.
Vozearia; falario.
\section{Grasnadela}
\begin{itemize}
\item {Grp. gram.:f.}
\end{itemize}
(V.grasnada)
\section{Grasnador}
\begin{itemize}
\item {Grp. gram.:adj.}
\end{itemize}
\begin{itemize}
\item {Grp. gram.:M.}
\end{itemize}
Que grasna.
Aquelle que grasna.
\section{Grasnante}
\begin{itemize}
\item {Grp. gram.:adj.}
\end{itemize}
Que grasna.
\section{Grasnar}
\begin{itemize}
\item {Grp. gram.:v. i.}
\end{itemize}
\begin{itemize}
\item {Utilização:Fig.}
\end{itemize}
\begin{itemize}
\item {Grp. gram.:M.}
\end{itemize}
Crocitar.
Soltar a voz, (falando-se do corvo, do pato ou da ran).
Vozear.
Voz do corvo, do pato ou da ran.
(Talvez contr. de \textunderscore grazinar\textunderscore . Cp. cast. \textunderscore graznar\textunderscore )
\section{Grasneiro}
\begin{itemize}
\item {Grp. gram.:adj.}
\end{itemize}
Que grasna. Cf. Garrett, \textunderscore Flôres sem Fruto\textunderscore , 48.
\section{Grasnido}
\begin{itemize}
\item {Grp. gram.:m.}
\end{itemize}
\begin{itemize}
\item {Proveniência:(De \textunderscore grasnar\textunderscore )}
\end{itemize}
O mesmo que \textunderscore grasnada\textunderscore .
\section{Grasno}
\begin{itemize}
\item {Grp. gram.:m.}
\end{itemize}
\begin{itemize}
\item {Proveniência:(De \textunderscore grasnar\textunderscore )}
\end{itemize}
O mesmo que \textunderscore grasnada\textunderscore .
\section{Grassar}
\begin{itemize}
\item {Grp. gram.:v. i.}
\end{itemize}
\begin{itemize}
\item {Proveniência:(Lat. \textunderscore grassari\textunderscore )}
\end{itemize}
Alastrar-se, desenvolver-se progressivamente.
Diffundir-se; propagar-se.
\section{Grassento}
\begin{itemize}
\item {Grp. gram.:adj.}
\end{itemize}
\begin{itemize}
\item {Proveniência:(De \textunderscore grasso\textunderscore )}
\end{itemize}
Crasso.
Que tem a consistência da graxa.
\section{Grasseta}
\begin{itemize}
\item {fónica:sê}
\end{itemize}
\begin{itemize}
\item {Grp. gram.:f.}
\end{itemize}
\begin{itemize}
\item {Proveniência:(Fr. \textunderscore grassete\textunderscore )}
\end{itemize}
Planta utriculariácea, vivaz, das regiões pantanosas.
\section{Grassitar}
\begin{itemize}
\item {Grp. gram.:v. i.}
\end{itemize}
Diz-se do pato, quando solta a voz. Cf. Castilho, \textunderscore Fastos\textunderscore , III, 324.
\section{Grasso}
\begin{itemize}
\item {Grp. gram.:adj.}
\end{itemize}
\begin{itemize}
\item {Utilização:Des.}
\end{itemize}
\begin{itemize}
\item {Proveniência:(Do lat. \textunderscore crassus\textunderscore )}
\end{itemize}
Gorduroso:«\textunderscore ...enxúndia, gordura e outras coisas grassas\textunderscore ». \textunderscore Usque\textunderscore , 36, v.^o.
\section{Gratamente}
\begin{itemize}
\item {Grp. gram.:adv.}
\end{itemize}
De modo grato; com reconhecimento: \textunderscore portou-se gratamente\textunderscore .
Com satisfação.
Agradavelmente: \textunderscore passámos o dia gratamente\textunderscore .
\section{Grateia}
\begin{itemize}
\item {Grp. gram.:f.}
\end{itemize}
\begin{itemize}
\item {Proveniência:(Do fr. \textunderscore gratter\textunderscore ?)}
\end{itemize}
Instrumento, para limpar o fundo dos rios.
O mesmo ou melhor que \textunderscore busca-vidas\textunderscore .
\section{Grateleiro}
\begin{itemize}
\item {Grp. gram.:m.}
\end{itemize}
Gênero de plantas terebintháceas da América.
\section{Gratidão}
\begin{itemize}
\item {Grp. gram.:f.}
\end{itemize}
\begin{itemize}
\item {Proveniência:(Do lat. \textunderscore gratitudo\textunderscore )}
\end{itemize}
Qualidade de quem é grato.
Agradecimento; reconhecimento de um benefício que se recebe.
\section{Gratificação}
\begin{itemize}
\item {Grp. gram.:f.}
\end{itemize}
\begin{itemize}
\item {Proveniência:(Lat. \textunderscore gratificatio\textunderscore )}
\end{itemize}
Acto ou effeito de gratificar.
Aquillo com que se gratifica.
\section{Gratificador}
\begin{itemize}
\item {Grp. gram.:m.  e  adj.}
\end{itemize}
\begin{itemize}
\item {Proveniência:(Lat. \textunderscore gratificator\textunderscore )}
\end{itemize}
O que gratifica.
\section{Gratificar}
\begin{itemize}
\item {Grp. gram.:v. t.}
\end{itemize}
\begin{itemize}
\item {Grp. gram.:V. i.}
\end{itemize}
\begin{itemize}
\item {Proveniência:(Lat. \textunderscore gratificari\textunderscore )}
\end{itemize}
Brindar, em testemunho de reconhecimento.
Remunerar.
Dar gorgeta a.
Pagar o serviço extraordinário de.
Dar graças; mostrar-se reconhecido.
\section{Gratifício}
\begin{itemize}
\item {Grp. gram.:m.}
\end{itemize}
\begin{itemize}
\item {Utilização:Des.}
\end{itemize}
Agradecimento, sinal de agradecimento.
(Cp. \textunderscore gratífico\textunderscore )
\section{Gratífico}
\begin{itemize}
\item {Grp. gram.:adj.}
\end{itemize}
Que manifesta gratidão.
Que exprime agrado.
(B. lat. \textunderscore gratificus\textunderscore . Cp. lat. \textunderscore gratificari\textunderscore )
\section{Gratir}
\begin{itemize}
\item {Grp. gram.:v. t.}
\end{itemize}
\begin{itemize}
\item {Utilização:Ant.}
\end{itemize}
\begin{itemize}
\item {Proveniência:(Do lat. \textunderscore gratis\textunderscore )}
\end{itemize}
O mesmo que \textunderscore agradecer\textunderscore .
\section{Gratis}
\begin{itemize}
\item {fónica:grátis}
\end{itemize}
\begin{itemize}
\item {Grp. gram.:adv.}
\end{itemize}
\begin{itemize}
\item {Proveniência:(T. lat.)}
\end{itemize}
Gratuitamente, de graça.
\section{Grato}
\begin{itemize}
\item {Grp. gram.:adj.}
\end{itemize}
\begin{itemize}
\item {Proveniência:(Lat. \textunderscore gratus\textunderscore )}
\end{itemize}
Agradável; aprazível.
Suave.
Que tem gratidão; agradecido.
\section{Gratuidade}
\begin{itemize}
\item {Grp. gram.:f.}
\end{itemize}
(Contr. de \textunderscore gratuitidade\textunderscore )
\section{Gratuitamente}
\begin{itemize}
\item {Grp. gram.:adv.}
\end{itemize}
De modo gratuito.
De graça; sem interesse.
\section{Gratuitidade}
\begin{itemize}
\item {Grp. gram.:f.}
\end{itemize}
Qualidade daquillo que é gratuito.
\section{Gratuito}
\begin{itemize}
\item {Grp. gram.:adj.}
\end{itemize}
\begin{itemize}
\item {Proveniência:(Lat. \textunderscore gratuitus\textunderscore )}
\end{itemize}
Concedido ou feito de graça ou espontaneamente.
Espontâneo; desinteressado.
Que não tem fundamento: \textunderscore accusação gratuita\textunderscore .
\section{Gratulação}
\begin{itemize}
\item {Grp. gram.:f.}
\end{itemize}
\begin{itemize}
\item {Proveniência:(Lat. \textunderscore gratulatio\textunderscore )}
\end{itemize}
Acto ou effeito de gratular.
\section{Gratular}
\begin{itemize}
\item {Grp. gram.:v. t.}
\end{itemize}
\begin{itemize}
\item {Proveniência:(Lat. \textunderscore gratulari\textunderscore )}
\end{itemize}
Mostrar-se grato a.
Felicitar.
Dar parabens a.
\section{Gratulatório}
\begin{itemize}
\item {Grp. gram.:adj.}
\end{itemize}
\begin{itemize}
\item {Proveniência:(Lat. \textunderscore gratulatorius\textunderscore )}
\end{itemize}
Em que se manifesta gratidão.
Próprio para felicitar.
\section{Grátulo}
\begin{itemize}
\item {Grp. gram.:adj.}
\end{itemize}
\begin{itemize}
\item {Utilização:Ant.}
\end{itemize}
\begin{itemize}
\item {Proveniência:(De \textunderscore gratular\textunderscore )}
\end{itemize}
O mesmo que \textunderscore gratulatório\textunderscore .
\section{Grau}
\begin{itemize}
\item {Grp. gram.:m.}
\end{itemize}
\begin{itemize}
\item {Grp. gram.:Loc. adv.}
\end{itemize}
\begin{itemize}
\item {Proveniência:(Lat. \textunderscore gradus\textunderscore )}
\end{itemize}
Passo.
Ordem, jerarchia, classe.
Medida.
Intensidade.
Posição.
Modo de sêr.
Título acadêmico.
Distância entre mais de uma geração e o tronco commum.
Cada uma das trezentas e sessenta partes, em que se divide um círculo.
Cada uma das divisões da escala de alguns instrumentos.
Expoente, em arithmética.
Somma dos expoentes das incógnitas, no termo em que essa somma é maior, (em álgebra).
\textunderscore Em alto grau\textunderscore , muitissimo; superiormente; enormemente.
\section{Graúdo}
\begin{itemize}
\item {Grp. gram.:adj.}
\end{itemize}
\begin{itemize}
\item {Grp. gram.:M. pl.}
\end{itemize}
\begin{itemize}
\item {Proveniência:(De \textunderscore grão\textunderscore )}
\end{itemize}
Grande; crescido.
Importante.
Classe dos ricos ou poderosos.
\section{Graúlho}
\begin{itemize}
\item {Grp. gram.:m.}
\end{itemize}
\begin{itemize}
\item {Proveniência:(Do rad. de \textunderscore grão\textunderscore )}
\end{itemize}
Semente de uva, bagulho, graínha.
\section{Graúna}
\begin{itemize}
\item {Grp. gram.:f.}
\end{itemize}
(V.braúna)
\section{Graussá}
\begin{itemize}
\item {Grp. gram.:m.}
\end{itemize}
Caranguejo pequeno de Pernambuco.
\section{Gravação}
\begin{itemize}
\item {Grp. gram.:f.}
\end{itemize}
\begin{itemize}
\item {Proveniência:(Lat. \textunderscore gravatio\textunderscore )}
\end{itemize}
Acto ou effeito de gravar^2.
\section{Gravador}
\begin{itemize}
\item {Grp. gram.:adj.}
\end{itemize}
\begin{itemize}
\item {Grp. gram.:M.}
\end{itemize}
\begin{itemize}
\item {Proveniência:(Lat. \textunderscore gravator\textunderscore )}
\end{itemize}
Que grava.
Aquelle que grava.
Artista que faz gravuras.
\section{Gravadura}
\begin{itemize}
\item {Grp. gram.:f.}
\end{itemize}
(V.gravura)
\section{Gravajo}
\begin{itemize}
\item {Grp. gram.:m.}
\end{itemize}
\begin{itemize}
\item {Utilização:Prov.}
\end{itemize}
\begin{itemize}
\item {Utilização:alg.}
\end{itemize}
\begin{itemize}
\item {Proveniência:(De \textunderscore gravar\textunderscore ^2)}
\end{itemize}
Aggravo, offensa.
\section{Gravalha}
\begin{itemize}
\item {Grp. gram.:f.}
\end{itemize}
\begin{itemize}
\item {Utilização:Prov.}
\end{itemize}
\begin{itemize}
\item {Utilização:minh.}
\end{itemize}
Caruma sêca.
\section{Gravalhoiço}
\begin{itemize}
\item {Grp. gram.:m.}
\end{itemize}
\begin{itemize}
\item {Utilização:T. de Turquel}
\end{itemize}
Homem, que se impõe, por sua figura e traje.
\section{Gravame}
\begin{itemize}
\item {Grp. gram.:m.}
\end{itemize}
\begin{itemize}
\item {Proveniência:(Lat. \textunderscore gravamen\textunderscore )}
\end{itemize}
Acto de molestar; vexame.
Encargo.
\section{Gravamento}
\begin{itemize}
\item {Grp. gram.:m.}
\end{itemize}
Acto de gravar^1.
\section{Gravana}
\begin{itemize}
\item {Grp. gram.:f.}
\end{itemize}
\begin{itemize}
\item {Utilização:Náut.}
\end{itemize}
\begin{itemize}
\item {Utilização:marit.}
\end{itemize}
\begin{itemize}
\item {Utilização:Gír.}
\end{itemize}
Vento fresco de Sul e Suéste, que sopra no golfo da Guiné, especialmente nas vizinhanças de San-Thomé.
\textunderscore Safar gravana\textunderscore , desembaraçar-se, trabalhar com ligeireza.
Estação sêca, em San-Thomé.
\section{Gravanada}
\begin{itemize}
\item {Grp. gram.:f.}
\end{itemize}
\begin{itemize}
\item {Utilização:T. da Bairrada}
\end{itemize}
Bátega de chuva grossa e pouco duradoira.
Saravaida, acompanhada de vento.
(Relaciona-se com \textunderscore gravana\textunderscore ?)
\section{Gravancelo}
\begin{itemize}
\item {Grp. gram.:m.}
\end{itemize}
\begin{itemize}
\item {Proveniência:(Do rad. de \textunderscore gravanço\textunderscore ^1)}
\end{itemize}
(V.esparavão)
\section{Gravanço}
\begin{itemize}
\item {Grp. gram.:m.}
\end{itemize}
\begin{itemize}
\item {Proveniência:(Do cast. \textunderscore garbanzo\textunderscore )}
\end{itemize}
O mesmo que \textunderscore grão-de-bico\textunderscore .
\section{Gravanço}
\begin{itemize}
\item {Grp. gram.:m.}
\end{itemize}
\begin{itemize}
\item {Utilização:Prov.}
\end{itemize}
Espécie de gadanho.
\section{Gravanha}
\begin{itemize}
\item {Grp. gram.:f.}
\end{itemize}
O mesmo que \textunderscore gravalha\textunderscore .
\section{Gravanzudo}
\begin{itemize}
\item {Grp. gram.:adj.}
\end{itemize}
\begin{itemize}
\item {Proveniência:(De \textunderscore gravanzo\textunderscore , por \textunderscore gravanço\textunderscore ^1)}
\end{itemize}
Diz-se de um esparavão.
\section{Gravar}
\begin{itemize}
\item {Grp. gram.:v. t.}
\end{itemize}
\begin{itemize}
\item {Proveniência:(Do al. \textunderscore graben\textunderscore , análogo ao gr. \textunderscore graphein\textunderscore )}
\end{itemize}
Esculpir.
Lavrar com buril ou cinzel: \textunderscore gravar uma inscripção\textunderscore .
Estampar.
Fixar: \textunderscore gravar na memória\textunderscore .
Assinalar.
Marcar com sêllo ou ferrete.
\section{Gravar}
\begin{itemize}
\item {Grp. gram.:v. t.}
\end{itemize}
\begin{itemize}
\item {Proveniência:(Lat. \textunderscore gravare\textunderscore )}
\end{itemize}
Molestar.
Onerar.
Vexar.
\section{Gravata}
\begin{itemize}
\item {Grp. gram.:f.}
\end{itemize}
\begin{itemize}
\item {Proveniência:(Fr. \textunderscore cravate\textunderscore )}
\end{itemize}
Ornato, que se usa á volta do pescoço e que consiste geralmente num lenço, fita ou pequena manta, formando laço adeante.
Colleira de coiro, que os militares usavam.
\section{Gravatá}
\begin{itemize}
\item {Grp. gram.:m.}
\end{itemize}
\begin{itemize}
\item {Utilização:Bras}
\end{itemize}
Nome de várias plantas bromeliáceas.
\section{Gravatão}
\begin{itemize}
\item {Grp. gram.:m.}
\end{itemize}
\begin{itemize}
\item {Proveniência:(De \textunderscore gravata\textunderscore )}
\end{itemize}
Pedante.
Homem de prosápia van.
\section{Gravataria}
\begin{itemize}
\item {Grp. gram.:f.}
\end{itemize}
Lugar ou estabelecimento, onde se vendem ou fabricam gravatas.
Grande porção de gravatas.
\section{Gravata-vermelha}
\begin{itemize}
\item {Grp. gram.:m.}
\end{itemize}
Pequena ave americana, (\textunderscore motacilla rubecula\textunderscore , Lin.).
\section{Gravateiro}
\begin{itemize}
\item {Grp. gram.:m.}
\end{itemize}
Fabricante ou vendedor de gravatas.
\section{Gravatil}
\begin{itemize}
\item {Grp. gram.:m.}
\end{itemize}
\begin{itemize}
\item {Utilização:Prov.}
\end{itemize}
Instrumento de carpinteiro, espécie de plaina, com que se faz a fêmea de um entalhe, em fórma de triângulo.
\section{Gravatilho}
\begin{itemize}
\item {Grp. gram.:m.}
\end{itemize}
\begin{itemize}
\item {Proveniência:(De \textunderscore gravato\textunderscore . Cp. \textunderscore graveto\textunderscore )}
\end{itemize}
Gancho da agulha, com que os marinheiros remendam as velas.
\section{Gravatinha}
\begin{itemize}
\item {Grp. gram.:f.}
\end{itemize}
Gravata de mulher; pequena gravata.
\section{Gravativo}
\begin{itemize}
\item {Grp. gram.:adj.}
\end{itemize}
\begin{itemize}
\item {Utilização:Med.}
\end{itemize}
\begin{itemize}
\item {Proveniência:(De \textunderscore gravar\textunderscore ^2)}
\end{itemize}
Diz-se da dôr, acompanhada de uma sensação de pêso.
\section{Gravato}
\begin{itemize}
\item {Grp. gram.:m.}
\end{itemize}
O mesmo que \textunderscore garavato\textunderscore .
\section{Grave}
\begin{itemize}
\item {Grp. gram.:adj.}
\end{itemize}
\begin{itemize}
\item {Grp. gram.:M.}
\end{itemize}
\begin{itemize}
\item {Grp. gram.:M.  e  adj.}
\end{itemize}
\begin{itemize}
\item {Utilização:Prov.}
\end{itemize}
\begin{itemize}
\item {Utilização:alent.}
\end{itemize}
\begin{itemize}
\item {Proveniência:(Lat. \textunderscore gravis\textunderscore )}
\end{itemize}
Que tem certo pêso.
Sujeito á acção da gravidade.
Ponderoso, sério, importante: \textunderscore motivos graves\textunderscore .
Distinto, nobre.
Circunspecto: \textunderscore pessôa grave\textunderscore .
Doloroso; intenso: \textunderscore soffrimento grave\textunderscore .
Que sôa com menor número de vibrações que o som ou voz aguda.
Diz-se das palavras e dos versos que têm accento predominante na última sýllaba; paroxýtono.
Aquillo que tem pêso.
Nota baixa, na música.
Cadência de marcha militar.
Indivíduo bem vestido, entrajado ceremoniosamente.
\section{Grave}
\begin{itemize}
\item {Grp. gram.:m.}
\end{itemize}
Moéda de prata, do valor de 21 reis, em tempo de D. Fernando.
\section{Grave}
\begin{itemize}
\item {Grp. gram.:m.}
\end{itemize}
(?)«\textunderscore ...e os que eram bem armados haviam de têr... luvas e estoque e grave\textunderscore ». Fern. Lopes, \textunderscore Chrón. de D. Fern.\textunderscore , c. LXXXVII.
\section{Gravela}
\begin{itemize}
\item {Grp. gram.:f.}
\end{itemize}
Resíduos secos da uva que foi espremida.
Fezes de vinho.
(Provn. \textunderscore gravel\textunderscore )
\section{Gravela}
\begin{itemize}
\item {Grp. gram.:f.}
\end{itemize}
\begin{itemize}
\item {Utilização:Med.}
\end{itemize}
\begin{itemize}
\item {Proveniência:(Fr. \textunderscore gravelle\textunderscore )}
\end{itemize}
Doença, resultante de pedra ou areias no rim ou na bexiga.
\section{Gravelado}
\begin{itemize}
\item {Grp. gram.:adj.}
\end{itemize}
Relativo a gravela; extrahido de gravela.
\section{Gravelho}
\begin{itemize}
\item {fónica:vê}
\end{itemize}
\begin{itemize}
\item {Grp. gram.:m.}
\end{itemize}
\begin{itemize}
\item {Utilização:Prov.}
\end{itemize}
\begin{itemize}
\item {Utilização:trasm.}
\end{itemize}
Cravelha da porta.
(Alter. de \textunderscore cravelho\textunderscore )
\section{Gravella}
\begin{itemize}
\item {Grp. gram.:f.}
\end{itemize}
\begin{itemize}
\item {Utilização:Med.}
\end{itemize}
\begin{itemize}
\item {Proveniência:(Fr. \textunderscore gravelle\textunderscore )}
\end{itemize}
Doença, resultante de pedra ou areias no rim ou na bexiga.
\section{Gravelloso}
\begin{itemize}
\item {Grp. gram.:m.}
\end{itemize}
Indivíduo que soffre gravella. Cf. \textunderscore Diár. de Noticias\textunderscore , de 28-VI-910.
\section{Graveloso}
\begin{itemize}
\item {Grp. gram.:m.}
\end{itemize}
Indivíduo que sofre gravela. Cf. \textunderscore Diár. de Noticias\textunderscore , de 28-VI-910.
\section{Gravemente}
\begin{itemize}
\item {Grp. gram.:adv.}
\end{itemize}
De modo grave.
Com gravidade; com seriedade.
Com importância; ceremoniosamente.
\section{Graveolência}
\begin{itemize}
\item {Grp. gram.:f.}
\end{itemize}
\begin{itemize}
\item {Proveniência:(Lat. \textunderscore graveolentia\textunderscore )}
\end{itemize}
Mau cheiro.
\section{Graveolente}
\begin{itemize}
\item {Grp. gram.:adj.}
\end{itemize}
\begin{itemize}
\item {Proveniência:(Lat. \textunderscore graveolens\textunderscore )}
\end{itemize}
Que tem cheiro forte.
Que cheira mal.
\section{Graveolento}
\begin{itemize}
\item {Grp. gram.:adj.}
\end{itemize}
\begin{itemize}
\item {Utilização:Des.}
\end{itemize}
O mesmo que \textunderscore graveolente\textunderscore .
\section{Graveta}
\begin{itemize}
\item {fónica:vê}
\end{itemize}
\begin{itemize}
\item {Grp. gram.:f.}
\end{itemize}
\begin{itemize}
\item {Utilização:Prov.}
\end{itemize}
\begin{itemize}
\item {Utilização:minh.}
\end{itemize}
Espécie de fateixa, com quatro unhas de ferro em fórma de espetos, usada pelos pescadores de bacalhau em fundos pedregosos.
Ancinho grande de ferro, com seis dentes.
(Por \textunderscore craveta\textunderscore , de \textunderscore cravar\textunderscore ?)
\section{Gravetar}
\begin{itemize}
\item {Grp. gram.:v. i.}
\end{itemize}
Fazer gravetos.
\section{Graveto}
\begin{itemize}
\item {fónica:vê}
\end{itemize}
\begin{itemize}
\item {Grp. gram.:m.}
\end{itemize}
\begin{itemize}
\item {Utilização:Bras}
\end{itemize}
\begin{itemize}
\item {Utilização:Prov.}
\end{itemize}
O mesmo que \textunderscore garavato\textunderscore .
Árvore silvestre, cuja madeira se emprega em caixaria.
Ancinho, para apanhar sargação.
\section{Graveza}
\begin{itemize}
\item {Grp. gram.:f.}
\end{itemize}
\begin{itemize}
\item {Utilização:Des.}
\end{itemize}
O mesmo que \textunderscore gravidade\textunderscore .
\section{Gravi}
\begin{itemize}
\item {Grp. gram.:m.}
\end{itemize}
O mesmo que \textunderscore gravim\textunderscore .
\section{Gravidação}
\begin{itemize}
\item {Grp. gram.:f.}
\end{itemize}
O mesmo que \textunderscore gravidez\textunderscore . Cf. Filinto, IX, 115.
\section{Gravidade}
\begin{itemize}
\item {Grp. gram.:f.}
\end{itemize}
\begin{itemize}
\item {Grp. gram.:Pl.}
\end{itemize}
\begin{itemize}
\item {Utilização:Prov.}
\end{itemize}
\begin{itemize}
\item {Utilização:alent.}
\end{itemize}
\begin{itemize}
\item {Proveniência:(Lat. \textunderscore gravitas\textunderscore )}
\end{itemize}
Qualidade daquelle ou daquillo que é grave.
Attracção dos corpos para o centro da terra.
Sisudez; seriedade; ponderação.
Intensidade.
Estado ou qualidade daquillo que dá cuidado: \textunderscore gravidade de uma doença\textunderscore .
Circunstancia perigosa.
Aggravação perigosa de uma doença: \textunderscore é doente de gravidade\textunderscore .
Enfeites ou ornamentos de vestuário.
\section{Gravidar}
\begin{itemize}
\item {Grp. gram.:v. t.}
\end{itemize}
Tornar grávida ou prenhe. Cf. Rui Barb., \textunderscore Réplica\textunderscore , 157.
\section{Gravidez}
\begin{itemize}
\item {Grp. gram.:f.}
\end{itemize}
\begin{itemize}
\item {Proveniência:(De \textunderscore grávido\textunderscore )}
\end{itemize}
Qualidade ou estado de grávido.
Estado da mulher e das fêmeas em geral, durante o tempo em que se desenvolve o respectivo embryão.
Gestação; prenhez.
\section{Grávido}
\begin{itemize}
\item {Grp. gram.:adj.}
\end{itemize}
\begin{itemize}
\item {Proveniência:(Lat. \textunderscore gravidus\textunderscore )}
\end{itemize}
Que se acha em estado de gravidez.
Prenhe.
Muito cheio; carregado.
\section{Gravígrado}
\begin{itemize}
\item {Grp. gram.:adj.}
\end{itemize}
\begin{itemize}
\item {Grp. gram.:M. pl.}
\end{itemize}
\begin{itemize}
\item {Proveniência:(Do lat. \textunderscore gravis\textunderscore  + \textunderscore gradus\textunderscore )}
\end{itemize}
Que tem o andar pesado.
Ordem de mammíferos, que têm o andar pesado.
\section{Gravim}
\begin{itemize}
\item {Grp. gram.:m.}
\end{itemize}
(V.garavim)
\section{Gravímetro}
\begin{itemize}
\item {Grp. gram.:m.}
\end{itemize}
\begin{itemize}
\item {Proveniência:(Do lat. \textunderscore gravis\textunderscore  + gr. \textunderscore metron\textunderscore )}
\end{itemize}
Instrumento, para determinar o pêso específico de certos corpos.
\section{Gravíola}
\begin{itemize}
\item {Grp. gram.:f.}
\end{itemize}
Arvore fructífera do Brasil.
Fruto dessa árvore.
\section{Gravisca}
\begin{itemize}
\item {Grp. gram.:f.}
\end{itemize}
\begin{itemize}
\item {Proveniência:(De \textunderscore gravisco\textunderscore )}
\end{itemize}
Mulher arisca, esquiva.
\section{Gravisco}
\begin{itemize}
\item {Grp. gram.:adj.}
\end{itemize}
\begin{itemize}
\item {Utilização:Ant.}
\end{itemize}
\begin{itemize}
\item {Proveniência:(De \textunderscore grave\textunderscore )}
\end{itemize}
Que mostra aspecto grave.
Esquivo, arisco. Cf. G. Vicente.
\section{Gravitação}
\begin{itemize}
\item {Grp. gram.:f.}
\end{itemize}
Fôrça, pela qual as partículas da matéria se attrahem reciprocamente, na razão directa das massas e na inversa do quadrado das distâncias.
Acto de gravitar.
\section{Gravitante}
\begin{itemize}
\item {Grp. gram.:adj.}
\end{itemize}
Que gravita.
\section{Gravitar}
\begin{itemize}
\item {Grp. gram.:v. i.}
\end{itemize}
\begin{itemize}
\item {Utilização:Fig.}
\end{itemize}
\begin{itemize}
\item {Proveniência:(De \textunderscore grave\textunderscore )}
\end{itemize}
Tender para um ponto ou centro, pela fôrça da gravitação.
Andar em volta de um corpo celeste, attrahido por elle, (falando-se dos astros): \textunderscore a Terra gravita em volta do Sol\textunderscore .
Diz-se de coisas ou pessôas, que seguem o destino de outras, em situação secundária.
\section{Gravito}
\begin{itemize}
\item {Grp. gram.:adj.}
\end{itemize}
Diz-se de toiro, que tem pouco marcada a volta natural dos cornos, sendo êstes quási direitos e altos.
\section{Gravoso}
\begin{itemize}
\item {Grp. gram.:adj.}
\end{itemize}
\begin{itemize}
\item {Proveniência:(De \textunderscore grave\textunderscore )}
\end{itemize}
Que vexa, que opprime.
Oneroso.
\section{Gravotear}
\begin{itemize}
\item {Grp. gram.:v.}
\end{itemize}
\begin{itemize}
\item {Utilização:t. Carp.}
\end{itemize}
Riscar com o compasso, por onde se há de serrar.
\section{Gravunha}
\begin{itemize}
\item {Grp. gram.:f.}
\end{itemize}
O mesmo que \textunderscore garatuja\textunderscore .
\section{Gravunhar}
\begin{itemize}
\item {Grp. gram.:v. i.}
\end{itemize}
O mesmo que \textunderscore garatujar\textunderscore :«\textunderscore ...sem papel em que gravunhe.\textunderscore »Filinto, VII, 75.
\section{Gravura}
\begin{itemize}
\item {Grp. gram.:f.}
\end{itemize}
Acto ou effeito de gravar.
Arte de gravar.
Obra esculpida, de pouca grossura.
Trabalho de gravador: \textunderscore gravura em madeira\textunderscore .
Estampa: \textunderscore comprar gravuras\textunderscore .
\section{Graxa}
\begin{itemize}
\item {Grp. gram.:f.}
\end{itemize}
\begin{itemize}
\item {Utilização:Prov.}
\end{itemize}
\begin{itemize}
\item {Utilização:minh.}
\end{itemize}
\begin{itemize}
\item {Utilização:Mad}
\end{itemize}
\begin{itemize}
\item {Utilização:Prov.}
\end{itemize}
\begin{itemize}
\item {Proveniência:(De \textunderscore graxo\textunderscore )}
\end{itemize}
Pó de fuligem ou de outras substâncias, para polir calçado e outros objectos.
Resina odorífera da thuia.
Doença, que, em certos animaes, faz derreter a gordura.
O mesmo que \textunderscore gordura\textunderscore .
Banha de porco.
Óleo, extrahido das tripas da sardinha. (Colhido em Varzim)
\section{Graxear}
\begin{itemize}
\item {Grp. gram.:v. i.}
\end{itemize}
\begin{itemize}
\item {Utilização:Bras}
\end{itemize}
\begin{itemize}
\item {Proveniência:(De \textunderscore graxa\textunderscore )}
\end{itemize}
O mesmo que \textunderscore namorar\textunderscore .
\section{Graxeiro}
\begin{itemize}
\item {Grp. gram.:m.}
\end{itemize}
\begin{itemize}
\item {Utilização:bras}
\end{itemize}
\begin{itemize}
\item {Utilização:Neol.}
\end{itemize}
Aquelle que unta ou lubrifica peças de máquinas. Cf. \textunderscore Diár. Official\textunderscore , suppl. ao n.^o 143 de 16-X-900.
\section{Graxo}
\begin{itemize}
\item {Grp. gram.:adj.}
\end{itemize}
\begin{itemize}
\item {Utilização:P. us.}
\end{itemize}
\begin{itemize}
\item {Proveniência:(Do lat. \textunderscore crassus\textunderscore )}
\end{itemize}
Que tem gordura; oleoso.
\section{Grazina}
\begin{itemize}
\item {Grp. gram.:m., f.  e  adj.}
\end{itemize}
\begin{itemize}
\item {Utilização:Fam.}
\end{itemize}
\begin{itemize}
\item {Grp. gram.:F.}
\end{itemize}
\begin{itemize}
\item {Proveniência:(De \textunderscore grazinar\textunderscore )}
\end{itemize}
Pessôa, que fala muito, que grita, que resmunga.
O mesmo que \textunderscore gaivina\textunderscore .
\section{Grazinada}
\begin{itemize}
\item {Grp. gram.:f.}
\end{itemize}
Acto ou effeito de grazinar.
\section{Grazinador}
\begin{itemize}
\item {Grp. gram.:m.  e  adj.}
\end{itemize}
Indivíduo que grazina.
\section{Grazinar}
\begin{itemize}
\item {Grp. gram.:v. i.}
\end{itemize}
\begin{itemize}
\item {Proveniência:(Do it. \textunderscore cracidare\textunderscore ?)}
\end{itemize}
Falar muito e alto.
Palrar.
Importunar, falando ou lamentando-se.
\section{Graziolo}
\begin{itemize}
\item {Grp. gram.:m.}
\end{itemize}
Casta de uva branca, talvez a mesma que \textunderscore dona-branca\textunderscore .
\section{Gré}
\begin{itemize}
\item {Grp. gram.:m.}
\end{itemize}
\begin{itemize}
\item {Utilização:Bras}
\end{itemize}
Um dos compartimentos do curral-de-peixe.
\section{Grebas}
\begin{itemize}
\item {fónica:grê}
\end{itemize}
\begin{itemize}
\item {Grp. gram.:f. pl.}
\end{itemize}
O mesmo que \textunderscore grevas\textunderscore .
\section{Grebe}
\begin{itemize}
\item {Grp. gram.:m.}
\end{itemize}
(V. \textunderscore colimbo\textunderscore ^1)
\section{Greciano}
\begin{itemize}
\item {Grp. gram.:m.  e  adj.}
\end{itemize}
\begin{itemize}
\item {Utilização:Des.}
\end{itemize}
O mesmo que \textunderscore grego\textunderscore . Cf. Castro, \textunderscore Proph. de Bandarra\textunderscore , 25 v.^o.
\section{Grecisco}
\begin{itemize}
\item {Grp. gram.:adj.}
\end{itemize}
\begin{itemize}
\item {Utilização:Ant.}
\end{itemize}
\begin{itemize}
\item {Proveniência:(De \textunderscore Grécia\textunderscore , n. p. Cp. \textunderscore francisco\textunderscore )}
\end{itemize}
Relativo á Grécia.
Dizia-se de certos panos e fazendas, cuja natureza é hoje diffícil determinar.
\section{Grecismo}
\begin{itemize}
\item {Grp. gram.:m.}
\end{itemize}
\begin{itemize}
\item {Proveniência:(Do lat. \textunderscore graecus\textunderscore )}
\end{itemize}
Locução peculiar á língua grega.
\section{Grecizar}
\begin{itemize}
\item {Grp. gram.:v. t.}
\end{itemize}
\begin{itemize}
\item {Proveniência:(Do lat. \textunderscore graecus\textunderscore )}
\end{itemize}
Dar fórma, feição ou costumes gregos a.
\section{Greco...}
\begin{itemize}
\item {Proveniência:(Do lat. \textunderscore graecus\textunderscore )}
\end{itemize}
Elemento que entra na composição de palavras, com o significado de \textunderscore grego\textunderscore  ou \textunderscore relativo a Gregos\textunderscore .
\section{Greco-italiano}
\begin{itemize}
\item {Grp. gram.:adj.}
\end{itemize}
Relativo a Gregos e Italianos.
\section{Greco-latino}
\begin{itemize}
\item {Grp. gram.:adj.}
\end{itemize}
Relativo ao latim e ao grego, ou a Gregos e Romanos.
\section{Grecomania}
\begin{itemize}
\item {Grp. gram.:f.}
\end{itemize}
Mania de imitar os usos ou a língua dos Gregos.
Paixão pelas coisas da Grécia.
\section{Greco-romano}
\begin{itemize}
\item {Grp. gram.:adj.}
\end{itemize}
Relativo a Gregos e Romanos.
\section{Greco-russo}
\begin{itemize}
\item {Grp. gram.:adj.}
\end{itemize}
Relativo a Gregos e Russos.
\section{Greco-siciliano}
\begin{itemize}
\item {Grp. gram.:adj.}
\end{itemize}
Diz-se do dialecto grego, que se falava na Sicília.
\section{Greco-turco}
\begin{itemize}
\item {Grp. gram.:adj.}
\end{itemize}
Relativo a Gregos e Turcos.
\section{Greda}
\begin{itemize}
\item {fónica:grê}
\end{itemize}
\begin{itemize}
\item {Grp. gram.:f.}
\end{itemize}
\begin{itemize}
\item {Proveniência:(Do lat. \textunderscore creta\textunderscore )}
\end{itemize}
Espécie de barro, macio, pulverulento e amarelado, empregado geralmente em tirar nódoas de madeira.
\section{Gredelém}
\begin{itemize}
\item {Grp. gram.:adj.}
\end{itemize}
\begin{itemize}
\item {Utilização:Des.}
\end{itemize}
\begin{itemize}
\item {Proveniência:(Do fr. \textunderscore gris de lin\textunderscore )}
\end{itemize}
Que tem côr semelhante á da flôr do linho.
\section{Gredoso}
\begin{itemize}
\item {Grp. gram.:adj.}
\end{itemize}
Em que há greda.
Que tem o aspecto da greda.
\section{Grega}
\begin{itemize}
\item {fónica:gré}
\end{itemize}
\begin{itemize}
\item {Grp. gram.:f.}
\end{itemize}
Cercadura architectónica, composta de linhas rectas entrelaçadas.
(Do \textunderscore grego\textunderscore )
\section{Gregal}
\begin{itemize}
\item {Grp. gram.:adj.}
\end{itemize}
\begin{itemize}
\item {Proveniência:(Lat. \textunderscore gregalis\textunderscore )}
\end{itemize}
Relativo a grei.
\section{Gregal}
\begin{itemize}
\item {Grp. gram.:adj.}
\end{itemize}
\begin{itemize}
\item {Proveniência:(Lat. \textunderscore graecalis\textunderscore )}
\end{itemize}
Relativo a Gregos.
Que sopra da Grécia ou do Nordeste, (falando-se de um vento do Mediterrâneo).
\section{Gregalada}
\begin{itemize}
\item {Grp. gram.:f.}
\end{itemize}
Rajada de vento gregal.
\section{Gregarina}
\begin{itemize}
\item {Grp. gram.:f.}
\end{itemize}
\begin{itemize}
\item {Proveniência:(Do lat. \textunderscore gregarius\textunderscore )}
\end{itemize}
Gênero de vermes intestinaes, que abrange duas espécies que vivem, em grande quantidade, no corpo de certos insectos.
\section{Gregário}
\begin{itemize}
\item {Grp. gram.:adj.}
\end{itemize}
\begin{itemize}
\item {Proveniência:(Lat. \textunderscore gregarius\textunderscore )}
\end{itemize}
O mesmo que \textunderscore gregal\textunderscore ^1.
Que faz parte de uma grei.
Que gosta ou tem o hábito de andar em bando: \textunderscore aves gregárias\textunderscore .
\section{Gregários}
\begin{itemize}
\item {Grp. gram.:m. pl.}
\end{itemize}
\begin{itemize}
\item {Proveniência:(Lat. \textunderscore gregarius\textunderscore )}
\end{itemize}
Família de pássaros, que comprehende, além de outros, os que vivem ordinariamente em bando.
\section{Grege}
\begin{itemize}
\item {Grp. gram.:f.}
\end{itemize}
\begin{itemize}
\item {Proveniência:(Lat. \textunderscore grex\textunderscore , \textunderscore gregis\textunderscore )}
\end{itemize}
O mesmo que \textunderscore grei\textunderscore .
\section{Grego}
\begin{itemize}
\item {fónica:grê}
\end{itemize}
\begin{itemize}
\item {Grp. gram.:m.}
\end{itemize}
\begin{itemize}
\item {Grp. gram.:Pl.}
\end{itemize}
\begin{itemize}
\item {Grp. gram.:Adj.}
\end{itemize}
\begin{itemize}
\item {Utilização:pop.}
\end{itemize}
\begin{itemize}
\item {Utilização:Fig.}
\end{itemize}
\begin{itemize}
\item {Utilização:Fam.}
\end{itemize}
\begin{itemize}
\item {Proveniência:(Do lat. \textunderscore graecus\textunderscore )}
\end{itemize}
Individuo, natural da Grécia.
Língua dos Gregos.
Povo hellênico, que deu o seu nome á Grécia.
Habitantes da Grécia.
Relativo á Grécia.
Obscuro; inintelligível: \textunderscore isso, para mim, é grego\textunderscore .
Embaraçado, atrapalhado: \textunderscore vi-me grego para atinar com o caminho\textunderscore .
\section{Gregoge}
\begin{itemize}
\item {Grp. gram.:m.}
\end{itemize}
Supplício antigo, que se usava na costa do Malabar e que consistia em atravessar com ferros as mãos, os pés, o pescoço e o peito do paciente. Cf. \textunderscore Peregrinação\textunderscore , XIX.
\section{Gregoriano}
\begin{itemize}
\item {Grp. gram.:adj.}
\end{itemize}
Relativo ao Papa Gregório I: \textunderscore canto gregoriano\textunderscore .
Relativo ao Papa Gregório XVI: \textunderscore correcção gregoriana\textunderscore .
\section{Gregorina}
\begin{itemize}
\item {Grp. gram.:f.}
\end{itemize}
O mesmo que \textunderscore radiolário\textunderscore .
\section{Gregório}
\begin{itemize}
\item {Grp. gram.:m.}
\end{itemize}
\begin{itemize}
\item {Utilização:Gír.}
\end{itemize}
\begin{itemize}
\item {Grp. gram.:Adj.}
\end{itemize}
\begin{itemize}
\item {Utilização:Burl.}
\end{itemize}
Pênis.
O mesmo que \textunderscore grego\textunderscore . Cf. Filinto, II, 193.
\section{Gregório}
\begin{itemize}
\item {Grp. gram.:m.}
\end{itemize}
\begin{itemize}
\item {Utilização:Fam.}
\end{itemize}
\begin{itemize}
\item {Proveniência:(T. onom.)}
\end{itemize}
\textunderscore Chamar pelo gregório\textunderscore , vomitar com ânsia.
\section{Gregotins}
\begin{itemize}
\item {Grp. gram.:m. pl.}
\end{itemize}
O mesmo que [[garatujas|garatuja]].
(Da soletração das últimas letras do alphabeto grego, \textunderscore i grego\textunderscore  + \textunderscore til\textunderscore )
\section{Greguejar}
\begin{itemize}
\item {Grp. gram.:v. i.}
\end{itemize}
\begin{itemize}
\item {Utilização:Fam.}
\end{itemize}
Falar grego. Cf. Filinto, XI, 194 e I, 49.
\section{Grei}
\begin{itemize}
\item {Grp. gram.:f.}
\end{itemize}
\begin{itemize}
\item {Utilização:Fig.}
\end{itemize}
\begin{itemize}
\item {Utilização:Ant.}
\end{itemize}
\begin{itemize}
\item {Proveniência:(Do lat. \textunderscore grex\textunderscore )}
\end{itemize}
Rebanho de gado miúdo.
Sociedade.
Partido.
Parochianos ou diocesanos.
Povo: \textunderscore servir o rei e a grei\textunderscore .
\section{Greiro}
\begin{itemize}
\item {Grp. gram.:m.}
\end{itemize}
Córte, aberto nos muros das marinhas.
\section{Greiro}
\begin{itemize}
\item {Grp. gram.:m.}
\end{itemize}
\begin{itemize}
\item {Utilização:Prov.}
\end{itemize}
\begin{itemize}
\item {Utilização:minh.}
\end{itemize}
Grão de milho grosso.
(Por \textunderscore graeiro\textunderscore , de \textunderscore grão\textunderscore )
\section{Greja}
\begin{itemize}
\item {Grp. gram.:f.}
\end{itemize}
\begin{itemize}
\item {Utilização:Ant.}
\end{itemize}
O mesmo que \textunderscore igreja\textunderscore . Cf. Frei Fortun., \textunderscore Inéd.\textunderscore , I, 308.
\section{Grejó}
\begin{itemize}
\item {Grp. gram.:f.}
\end{itemize}
\begin{itemize}
\item {Utilização:Prov.}
\end{itemize}
\begin{itemize}
\item {Utilização:trasm.}
\end{itemize}
Igreja pequena.
(Aphér. de \textunderscore igrejó\textunderscore  = \textunderscore igrejola\textunderscore )
\section{Grela}
\begin{itemize}
\item {Grp. gram.:f.}
\end{itemize}
\begin{itemize}
\item {Proveniência:(Fr. \textunderscore grêle\textunderscore )}
\end{itemize}
Instrumento de penteeiro, para amaciar os pentes de alisar.
\section{Grelado}
\begin{itemize}
\item {Grp. gram.:adj.}
\end{itemize}
\begin{itemize}
\item {Proveniência:(De \textunderscore grelar\textunderscore )}
\end{itemize}
Que tem grêlo.
Que lançou espiga.
Que começou a grelar.
\section{Grelar}
\begin{itemize}
\item {Grp. gram.:v. i.}
\end{itemize}
Lançar grelos.
Germinar.
Espigar.
\section{Grelha}
\begin{itemize}
\item {fónica:grê}
\end{itemize}
\begin{itemize}
\item {Grp. gram.:f.}
\end{itemize}
\begin{itemize}
\item {Proveniência:(Do lat. \textunderscore craticula\textunderscore ?)}
\end{itemize}
Espécie de pequena grade de ferro, em que se assam ou torram substâncias comestíveis.
Objecto análogo, sôbre que se acende o carvão nos fogareiros, fornalhas, etc.
Antigo instrumento de supplício.
\section{Grelha}
\begin{itemize}
\item {fónica:grê}
\end{itemize}
\begin{itemize}
\item {Grp. gram.:f.}
\end{itemize}
\begin{itemize}
\item {Utilização:Gír.}
\end{itemize}
Peru.
\section{Grelhar}
\begin{itemize}
\item {Grp. gram.:v. t.}
\end{itemize}
Assar ou torrar na grelha.
\section{Grelheiro}
\begin{itemize}
\item {Grp. gram.:m.}
\end{itemize}
Operário, que tem a seu cargo as grelhas de uma officina.
\section{Grêlo}
\begin{itemize}
\item {Grp. gram.:m.}
\end{itemize}
\begin{itemize}
\item {Utilização:Chul. da Bairrada.}
\end{itemize}
\begin{itemize}
\item {Proveniência:(Do cat. \textunderscore grill\textunderscore )}
\end{itemize}
Gemma, que se desenvolve na semente.
Bolbo.
Rebento.
Haste de algumas plantas, antes de desabrocharem as flôres.
O mesmo que \textunderscore clitóris\textunderscore .
\section{Grelos}
\begin{itemize}
\item {fónica:grê}
\end{itemize}
\begin{itemize}
\item {Grp. gram.:m. pl.}
\end{itemize}
\begin{itemize}
\item {Utilização:T. de Algodres}
\end{itemize}
O mesmo que \textunderscore excremento\textunderscore .
\section{Gremial}
\begin{itemize}
\item {Grp. gram.:adj.}
\end{itemize}
\begin{itemize}
\item {Grp. gram.:M.}
\end{itemize}
\begin{itemize}
\item {Proveniência:(Lat. \textunderscore gremialis\textunderscore )}
\end{itemize}
Relativo a grêmio.
Pano, que se põe nos joêlhos do um Prelado officiante, quando está sentado.
\section{Gremilha}
\begin{itemize}
\item {Grp. gram.:f.}
\end{itemize}
\begin{itemize}
\item {Proveniência:(Fr. \textunderscore gremille\textunderscore )}
\end{itemize}
Peixe do centro e norte da Europa, semelhante á perca.
\section{Grêmio}
\begin{itemize}
\item {Grp. gram.:m.}
\end{itemize}
\begin{itemize}
\item {Proveniência:(Lat. \textunderscore gremium\textunderscore )}
\end{itemize}
Seio, regaço.
Corporação, communidade; assembleia.
\section{Grempa}
\begin{itemize}
\item {Grp. gram.:f.}
\end{itemize}
\begin{itemize}
\item {Utilização:Prov.}
\end{itemize}
Espécie de catavento.
(Corr. de \textunderscore grimpa\textunderscore )
\section{Grenache}
\begin{itemize}
\item {Grp. gram.:m.}
\end{itemize}
Casta de uva de Azeitão. Cf. \textunderscore Rev. Agronómica\textunderscore , I, 8.
\section{Grencho}
\begin{itemize}
\item {Grp. gram.:adj.}
\end{itemize}
O mesmo que \textunderscore crespo\textunderscore , (falando-se de cabello). (Colhido na Guarda)
\section{Grenetina}
\begin{itemize}
\item {Grp. gram.:f.}
\end{itemize}
\begin{itemize}
\item {Proveniência:(De \textunderscore Grenet\textunderscore , n. p.)}
\end{itemize}
Gelatina purificada.
\section{Grenha}
\begin{itemize}
\item {Grp. gram.:f.}
\end{itemize}
\begin{itemize}
\item {Utilização:Ext.}
\end{itemize}
\begin{itemize}
\item {Utilização:Prov.}
\end{itemize}
\begin{itemize}
\item {Utilização:alent.}
\end{itemize}
\begin{itemize}
\item {Proveniência:(Do lat. \textunderscore crinis\textunderscore )}
\end{itemize}
Cabello em desalinho.
Crina do leão.
Bosque denso, emmaranhado.
Variedade de couve.
\section{Grepo}
\begin{itemize}
\item {Grp. gram.:m.}
\end{itemize}
Antigo sacerdote, no Pegu:«\textunderscore ...uma prática que teve com um dos grepos...\textunderscore »\textunderscore Peregrinação\textunderscore , CLXIV.
(Cp. \textunderscore talagrepo\textunderscore )
\section{Grés}
\begin{itemize}
\item {Grp. gram.:m.}
\end{itemize}
\begin{itemize}
\item {Utilização:Gal}
\end{itemize}
\begin{itemize}
\item {Proveniência:(Do fr. \textunderscore grés\textunderscore )}
\end{itemize}
Rocha, formada de grãos de areia fina.
Espécie de argilla, misturada com areia fina, e empregada em ollaria.
Pó daquella pedra, empregado especialmente em polir metaes.--Melhor português é \textunderscore arenito\textunderscore .
\section{Gresca}
\begin{itemize}
\item {fónica:grés}
\end{itemize}
\begin{itemize}
\item {Grp. gram.:f.}
\end{itemize}
\begin{itemize}
\item {Utilização:Prov.}
\end{itemize}
\begin{itemize}
\item {Utilização:trasm.}
\end{itemize}
Barulho grave.
Desordem com pancadaria.
\section{Gresífero}
\begin{itemize}
\item {Grp. gram.:adj.}
\end{itemize}
Diz-se do terreno, em que há grés.
\section{Grèsiforme}
\begin{itemize}
\item {Grp. gram.:adj.}
\end{itemize}
\begin{itemize}
\item {Proveniência:(De \textunderscore grés\textunderscore  + \textunderscore fórma\textunderscore )}
\end{itemize}
Que tem a apparência do grés.
\section{Grêta}
\begin{itemize}
\item {Grp. gram.:f.}
\end{itemize}
\begin{itemize}
\item {Grp. gram.:Pl.}
\end{itemize}
\begin{itemize}
\item {Proveniência:(Do rad. de \textunderscore gretar\textunderscore )}
\end{itemize}
O mesmo que \textunderscore fenda\textunderscore .
Abertura estreita.
Fendas na prega dos joêlhos dos cavallos, difficultando-lhes a articulação.
\section{Gretado}
\begin{itemize}
\item {Grp. gram.:adj.}
\end{itemize}
\begin{itemize}
\item {Utilização:Heráld.}
\end{itemize}
Em que há gretas: \textunderscore a varina tem os pés gretados\textunderscore .
Diz-se das vieiras, do leão, ou de outros emblemas, quando estriados ou listrados.
\section{Gretadura}
\begin{itemize}
\item {Grp. gram.:f.}
\end{itemize}
Acto ou effeito de gretar.
Grêta.
\section{Gretar}
\begin{itemize}
\item {Grp. gram.:v. t.}
\end{itemize}
\begin{itemize}
\item {Grp. gram.:V. i.  e  p.}
\end{itemize}
\begin{itemize}
\item {Utilização:Fig.}
\end{itemize}
\begin{itemize}
\item {Proveniência:(Do lat. \textunderscore crepitare\textunderscore . Cp. \textunderscore crepitar\textunderscore )}
\end{itemize}
Abrir fenda em.
Fendêr-se.
Desmanchar-se.
Descompor-se.
Estalar, fendendo-se.
\section{Greu}
\begin{itemize}
\item {Grp. gram.:m.}
\end{itemize}
\begin{itemize}
\item {Utilização:Ant.}
\end{itemize}
\begin{itemize}
\item {Proveniência:(Do lat. \textunderscore gratus\textunderscore )}
\end{itemize}
Grado, vontade.
\section{Grevado}
\begin{itemize}
\item {Grp. gram.:adj.}
\end{itemize}
Calçado de grevas.
\section{Grevas}
\begin{itemize}
\item {fónica:grê}
\end{itemize}
\begin{itemize}
\item {Grp. gram.:f. pl.}
\end{itemize}
\begin{itemize}
\item {Utilização:Ant.}
\end{itemize}
Parte da armadura, que cobria a perna, do joêlho para baixo.
(Cast. \textunderscore greba\textunderscore )
\section{Greve}
\begin{itemize}
\item {Grp. gram.:f.}
\end{itemize}
\begin{itemize}
\item {Proveniência:(Fr. \textunderscore grève\textunderscore )}
\end{itemize}
Conluio de operários, de estudantes, de funccionários, etc., que recusam trabalhar, ou comparecer aonde os chama o dever, em-quanto lhes não attendam certas reclamações.
\section{Grevília}
\begin{itemize}
\item {Grp. gram.:f.}
\end{itemize}
\begin{itemize}
\item {Proveniência:(De \textunderscore Greville\textunderscore , n. p.)}
\end{itemize}
Gênero de plantas.
\section{Grevíllia}
\begin{itemize}
\item {Grp. gram.:f.}
\end{itemize}
\begin{itemize}
\item {Proveniência:(De \textunderscore Greville\textunderscore , n. p.)}
\end{itemize}
Gênero de plantas.
\section{Grèvista}
\begin{itemize}
\item {Grp. gram.:m.  e  f.}
\end{itemize}
Pessôa, que promove uma greve ou se associa a ella.
\section{Grezisco}
\begin{itemize}
\item {Grp. gram.:m.  e  adj.}
\end{itemize}
\begin{itemize}
\item {Utilização:Ant.}
\end{itemize}
O mesmo que \textunderscore grecisco\textunderscore .
\section{Grias}
\begin{itemize}
\item {Grp. gram.:m.}
\end{itemize}
Árvore gutífera da Jamaica.
\section{Gricha}
\begin{itemize}
\item {Grp. gram.:f.}
\end{itemize}
\begin{itemize}
\item {Utilização:Prov.}
\end{itemize}
\begin{itemize}
\item {Utilização:trasm.}
\end{itemize}
Fenda em rocha ou fraga, principalmente se della sái água.
\section{Gridelim}
\begin{itemize}
\item {Grp. gram.:adj.}
\end{itemize}
O mesmo que \textunderscore gredelém\textunderscore .
\section{Grielo}
\begin{itemize}
\item {Grp. gram.:m.}
\end{itemize}
Planta rosácea do Cabo da Bôa-Esperança.
\section{Grifa}
\begin{itemize}
\item {Grp. gram.:f.}
\end{itemize}
\begin{itemize}
\item {Utilização:Des.}
\end{itemize}
\begin{itemize}
\item {Proveniência:(Fr. \textunderscore griffe\textunderscore )}
\end{itemize}
Garra; unha adunca.
\section{Grifardo}
\begin{itemize}
\item {Grp. gram.:m.}
\end{itemize}
\begin{itemize}
\item {Proveniência:(Do rad. de \textunderscore grifa\textunderscore )}
\end{itemize}
Espécie de águia africana.
\section{Grifenho}
\begin{itemize}
\item {Grp. gram.:adj.}
\end{itemize}
Que tem grifas; rapace. Cf. Camillo, \textunderscore Coisas Leves\textunderscore , 27; Garrett, \textunderscore Retr. de Vênus\textunderscore , 46 e 85.
\section{Griffínia}
\begin{itemize}
\item {Grp. gram.:f.}
\end{itemize}
\begin{itemize}
\item {Proveniência:(De \textunderscore Griffin\textunderscore , n. p.)}
\end{itemize}
Gênero de plantas bulbosas, da fam. das amaryllídeas.
\section{Grífico}
\begin{itemize}
\item {Grp. gram.:adj.}
\end{itemize}
Relativo a grifo.
\section{Grifínia}
\begin{itemize}
\item {Grp. gram.:f.}
\end{itemize}
\begin{itemize}
\item {Proveniência:(De \textunderscore Griffin\textunderscore , n. p.)}
\end{itemize}
Gênero de plantas bulbosas, da fam. das amarilídeas.
\section{Grifo}
\begin{itemize}
\item {Grp. gram.:m.}
\end{itemize}
\begin{itemize}
\item {Utilização:Typ.}
\end{itemize}
\begin{itemize}
\item {Proveniência:(Do gr. \textunderscore griphos\textunderscore )}
\end{itemize}
Enigma; questão embaraçada.
Locução ambígua.
Caracteres itálicos; itálico.
\section{Grijó}
\begin{itemize}
\item {Grp. gram.:m.}
\end{itemize}
\begin{itemize}
\item {Utilização:Ant.}
\end{itemize}
O mesmo que \textunderscore igrejó\textunderscore .
\section{Grilada}
\begin{itemize}
\item {Grp. gram.:f.}
\end{itemize}
\begin{itemize}
\item {Utilização:Deprec.}
\end{itemize}
\begin{itemize}
\item {Proveniência:(De \textunderscore grilo\textunderscore ^3)}
\end{itemize}
A ordem dos frades grilos.
\section{Grileira}
\begin{itemize}
\item {Grp. gram.:f.}
\end{itemize}
\begin{itemize}
\item {Proveniência:(De \textunderscore grilo\textunderscore )}
\end{itemize}
Parte de uma armadilha para apanhar pássaros, á qual se prende um grilo como isca.
\section{Grilha}
\begin{itemize}
\item {Grp. gram.:f.}
\end{itemize}
\begin{itemize}
\item {Utilização:Ant.}
\end{itemize}
Qualquer coisa, que entrava na formação de certos peloiros.
Anel de ferro? Cf. Rui Freire, \textunderscore Commentários\textunderscore , I, 6.
(Relaciona-se com \textunderscore grelha\textunderscore ? Cp. fr. \textunderscore grille\textunderscore )
\section{Grilhagem}
\begin{itemize}
\item {Grp. gram.:f.}
\end{itemize}
\begin{itemize}
\item {Proveniência:(De \textunderscore grilho\textunderscore )}
\end{itemize}
Cadeia de anéis de metal.
\section{Grilhão}
\begin{itemize}
\item {Grp. gram.:m.}
\end{itemize}
\begin{itemize}
\item {Utilização:Fig.}
\end{itemize}
\begin{itemize}
\item {Proveniência:(De \textunderscore grilho\textunderscore )}
\end{itemize}
Cadeia de metal.
Cordão de oiro, com que as mulheres do povo enfeitam o pescoço.
Cadeia de oiro, com que se prende o relógio ao collete.
Laço, prisão: \textunderscore os grilhões do amor\textunderscore .
Algema.
\section{Grilharia}
\begin{itemize}
\item {Grp. gram.:f.}
\end{itemize}
Barulho ou som estridente, como o cantar de muitos grillos. Cf. Camillo, \textunderscore Sc. da Foz\textunderscore , 144.
(Por \textunderscore grillaria\textunderscore , de \textunderscore grillo\textunderscore )
\section{Grilheta}
\begin{itemize}
\item {fónica:lhê}
\end{itemize}
\begin{itemize}
\item {Grp. gram.:f.}
\end{itemize}
\begin{itemize}
\item {Grp. gram.:M.}
\end{itemize}
\begin{itemize}
\item {Proveniência:(De \textunderscore grilha\textunderscore )}
\end{itemize}
Grande anel de ferro, na extremidade de uma corrente do mesmo metal, a que se prendiam os condemnados a trabalhos públicos.
O condemnado a trabalhos públicos.
\section{Grilho}
\begin{itemize}
\item {Grp. gram.:m.}
\end{itemize}
\begin{itemize}
\item {Utilização:Ant.}
\end{itemize}
O mesmo que \textunderscore grilhão\textunderscore .
(Provavelmente, da mesma or. que \textunderscore grilha\textunderscore . Cp. cast. \textunderscore grillo\textunderscore )
\section{Grillada}
\begin{itemize}
\item {Grp. gram.:f.}
\end{itemize}
\begin{itemize}
\item {Utilização:Deprec.}
\end{itemize}
\begin{itemize}
\item {Proveniência:(De \textunderscore grillo\textunderscore ^3)}
\end{itemize}
A ordem dos frades grillos.
\section{Grilleira}
\begin{itemize}
\item {Grp. gram.:f.}
\end{itemize}
\begin{itemize}
\item {Proveniência:(De \textunderscore grillo\textunderscore )}
\end{itemize}
Parte de uma armadilha para apanhar pássaros, á qual se prende um grillo como isca.
\section{Grillo}
\begin{itemize}
\item {Grp. gram.:m.}
\end{itemize}
\begin{itemize}
\item {Proveniência:(Lat. \textunderscore grillus\textunderscore )}
\end{itemize}
Pequeno insecto orthóptoro, que, sendo macho, produz um ruído especial com o attrito das asas.
Espécie de jôgo popular.
\section{Grillo}
\begin{itemize}
\item {Grp. gram.:m.}
\end{itemize}
\begin{itemize}
\item {Utilização:Gír.}
\end{itemize}
O mesmo que \textunderscore relógio\textunderscore .
(Por \textunderscore grilho\textunderscore , propriamente a corrente que segura o relógio)
\section{Grillo}
\begin{itemize}
\item {Grp. gram.:adj.}
\end{itemize}
Dizia-se dos frades de certa Ordem: \textunderscore frades grillos\textunderscore .
\section{Grilo}
\begin{itemize}
\item {Grp. gram.:m.}
\end{itemize}
\begin{itemize}
\item {Proveniência:(Lat. \textunderscore grillus\textunderscore )}
\end{itemize}
Pequeno insecto ortóptoro, que, sendo macho, produz um ruído especial com o atrito das asas.
Espécie de jôgo popular.
\section{Grilo}
\begin{itemize}
\item {Grp. gram.:m.}
\end{itemize}
\begin{itemize}
\item {Utilização:Gír.}
\end{itemize}
O mesmo que \textunderscore relógio\textunderscore .
(Por \textunderscore grilho\textunderscore , propriamente a corrente que segura o relógio)
\section{Grilo}
\begin{itemize}
\item {Grp. gram.:adj.}
\end{itemize}
Dizia-se dos frades de certa Ordem: \textunderscore frades grilos\textunderscore .
\section{Grima}
\begin{itemize}
\item {Grp. gram.:f.}
\end{itemize}
\begin{itemize}
\item {Utilização:Prov.}
\end{itemize}
\begin{itemize}
\item {Utilização:trasm.}
\end{itemize}
\begin{itemize}
\item {Proveniência:(Do ant. alt. al. \textunderscore grim\textunderscore , furioso)}
\end{itemize}
Ódio; raiva.
Grande terror, pavor.
\section{Grimaça}
\begin{itemize}
\item {Grp. gram.:f.}
\end{itemize}
\begin{itemize}
\item {Utilização:inútil}
\end{itemize}
\begin{itemize}
\item {Utilização:Gal}
\end{itemize}
\begin{itemize}
\item {Proveniência:(Fr. \textunderscore grimace\textunderscore )}
\end{itemize}
Trejeito, esgares.
\section{Grimpa}
\begin{itemize}
\item {Grp. gram.:f.}
\end{itemize}
\begin{itemize}
\item {Utilização:Fig.}
\end{itemize}
\begin{itemize}
\item {Utilização:Gír.}
\end{itemize}
\begin{itemize}
\item {Proveniência:(Do holl. \textunderscore grippen\textunderscore ? Cp. fr. \textunderscore grimper\textunderscore )}
\end{itemize}
Lâmina, que, girando em volta de um eixo pela acção do vento, indica a direcção dêste.
Catavento.
O ponto mais elevado de um objecto; coruto.
Voz altaneira, orgulhosa, de respingão: \textunderscore o rapaz já levanta a grimpa\textunderscore .
O mesmo que \textunderscore cabeça\textunderscore .
\section{Grimpado}
\begin{itemize}
\item {Grp. gram.:adj.}
\end{itemize}
\begin{itemize}
\item {Proveniência:(De \textunderscore grimpar\textunderscore )}
\end{itemize}
Que tem grimpa.
Que está na grimpa, no auge.
\section{Grimpar}
\begin{itemize}
\item {Grp. gram.:v. i.}
\end{itemize}
\begin{itemize}
\item {Proveniência:(De \textunderscore grimpa\textunderscore )}
\end{itemize}
Investir, arremessar-se contra alguém.
Respingar.
Responder desabridamente, com insolência.
\section{Grinalda}
\begin{itemize}
\item {Grp. gram.:f.}
\end{itemize}
\begin{itemize}
\item {Utilização:Fig.}
\end{itemize}
Corôa de flôres, de ramos, de pedraria, etc.
Enfeite de flôres entrelaçadas, formando banda.
Ornato de architectónico de fôlhas ou flôres.
Capella.
Moldura da popa de um navio.
Florilégio ou anthologia literária.
(Metáth. de \textunderscore guirtanda\textunderscore )
\section{Grindélia}
\begin{itemize}
\item {Grp. gram.:f.}
\end{itemize}
O mesmo que \textunderscore aurélia\textunderscore , planta.
\section{Grinfar}
\begin{itemize}
\item {Grp. gram.:v. i.}
\end{itemize}
\begin{itemize}
\item {Proveniência:(T. onom.?)}
\end{itemize}
O mesmo que \textunderscore trissar\textunderscore .
\section{Gringal}
\begin{itemize}
\item {Grp. gram.:m.}
\end{itemize}
\begin{itemize}
\item {Proveniência:(Do al. \textunderscore gering\textunderscore )}
\end{itemize}
Espécie de pano alemão.
\section{Gringo}
\begin{itemize}
\item {Grp. gram.:m.}
\end{itemize}
\begin{itemize}
\item {Utilização:Bras}
\end{itemize}
\begin{itemize}
\item {Utilização:Deprec.}
\end{itemize}
Estrangeiro.
\section{Gringolim}
\begin{itemize}
\item {Grp. gram.:m.}
\end{itemize}
\begin{itemize}
\item {Utilização:Gír.}
\end{itemize}
Qualquer bebida espirituosa.
\section{Gripal}
\begin{itemize}
\item {Grp. gram.:adj.}
\end{itemize}
Relativo a gripe.
\section{Gripe}
\begin{itemize}
\item {Grp. gram.:f.}
\end{itemize}
\begin{itemize}
\item {Proveniência:(Fr. \textunderscore grippe\textunderscore )}
\end{itemize}
Catarro epidêmico; inflamação epidêmica de membranas mucosas.
\section{Gríphico}
\begin{itemize}
\item {Grp. gram.:adj.}
\end{itemize}
Relativo a gripho.
\section{Gripho}
\begin{itemize}
\item {Grp. gram.:m.}
\end{itemize}
\begin{itemize}
\item {Utilização:Typ.}
\end{itemize}
\begin{itemize}
\item {Proveniência:(Do gr. \textunderscore griphos\textunderscore )}
\end{itemize}
Enigma; questão embaraçada.
Locução ambígua.
Caracteres itálicos; itálico.
\section{Grippal}
\begin{itemize}
\item {Grp. gram.:adj.}
\end{itemize}
Relativo a grippe.
\section{Grippe}
\begin{itemize}
\item {Grp. gram.:f.}
\end{itemize}
\begin{itemize}
\item {Proveniência:(Fr. \textunderscore grippe\textunderscore )}
\end{itemize}
Catarro epidêmico; inflammação epidêmica de membranas mucosas.
\section{Gríquas}
\begin{itemize}
\item {Grp. gram.:m. pl.}
\end{itemize}
Povo mestiço da África austro-occidental, procedente da mistura de Namáquas com Holandeses.
\section{Gris}
\begin{itemize}
\item {Grp. gram.:adj.}
\end{itemize}
\begin{itemize}
\item {Grp. gram.:F.}
\end{itemize}
\begin{itemize}
\item {Proveniência:(Fr. \textunderscore gris\textunderscore )}
\end{itemize}
Cinzento, tirante a azul.
Pardo.
Pelliça parda, própria para agasalho ou ornato e procedente de um esquilo do norte da Europa.
\section{Gris}
\begin{itemize}
\item {Grp. gram.:m.}
\end{itemize}
O mesmo que \textunderscore griso\textunderscore .
\section{Grisalho}
\begin{itemize}
\item {Grp. gram.:adj.}
\end{itemize}
\begin{itemize}
\item {Proveniência:(Fr. \textunderscore grisaille\textunderscore )}
\end{itemize}
Cinzento; pardo.
Mesclado de preto ou loiro e branco, (falando-se do cabello).
\section{Grisandra}
\begin{itemize}
\item {Grp. gram.:f.}
\end{itemize}
Erva campestre, amarela e de fôlhas recurvadas.
\section{Grisão}
\begin{itemize}
\item {Grp. gram.:adj.}
\end{itemize}
Diz-se de alguns dialectos românicos, que são vernáculos na Suíça.
\section{Gríse}
\begin{itemize}
\item {Grp. gram.:m.}
\end{itemize}
\begin{itemize}
\item {Utilização:Ant.}
\end{itemize}
Espécie de tecido pardacento.
(Cp. \textunderscore gris\textunderscore ^1)
\section{Griséu}
\begin{itemize}
\item {Grp. gram.:adj.}
\end{itemize}
\begin{itemize}
\item {Grp. gram.:M. pl.}
\end{itemize}
\begin{itemize}
\item {Utilização:Prov.}
\end{itemize}
\begin{itemize}
\item {Utilização:alg.}
\end{itemize}
Que tem côr cinzenta, tirante a verde.
Ervilhas maduras.
(B. lat. \textunderscore griseus\textunderscore )
\section{Grisisco}
\begin{itemize}
\item {Grp. gram.:adj.}
\end{itemize}
\begin{itemize}
\item {Proveniência:(De \textunderscore gris\textunderscore )}
\end{itemize}
Pardacento. Cf. Herculano, \textunderscore Bobo\textunderscore , 151.
\section{Griso}
\begin{itemize}
\item {Grp. gram.:m.}
\end{itemize}
\begin{itemize}
\item {Utilização:Gír.}
\end{itemize}
Frio.
\section{Grisó}
\begin{itemize}
\item {Grp. gram.:m.}
\end{itemize}
\begin{itemize}
\item {Utilização:T. de Turquel}
\end{itemize}
\begin{itemize}
\item {Utilização:fam.}
\end{itemize}
Azeite.
\section{Grisões}
\begin{itemize}
\item {Grp. gram.:m. pl.}
\end{itemize}
Povos da Suíça, nos Alpes.
\section{Grisu}
\begin{itemize}
\item {Grp. gram.:m.}
\end{itemize}
\begin{itemize}
\item {Proveniência:(Fr. \textunderscore grisou\textunderscore )}
\end{itemize}
Gás explosivo, que se infiltra ás vezes nas minas de carvão, inflammando-se e produzindo graves sinistros.
\section{Grita}
\begin{itemize}
\item {Grp. gram.:f.}
\end{itemize}
Acto de gritar.
Gritaria; alarido.
\section{Gritada}
\begin{itemize}
\item {Grp. gram.:f.}
\end{itemize}
(V.gritaria)
\section{Gritadeira}
\begin{itemize}
\item {Grp. gram.:f.}
\end{itemize}
\begin{itemize}
\item {Proveniência:(De \textunderscore gritar\textunderscore )}
\end{itemize}
Mulher, que grita muito.
Gritaria.
Planta rubiácea do Brasil.
\section{Gritador}
\begin{itemize}
\item {Grp. gram.:m.  e  adj.}
\end{itemize}
O que grita.
O que fala em voz muito alta.
\section{Gritante}
\begin{itemize}
\item {Grp. gram.:adj.}
\end{itemize}
Que grita.
Que se manifesta em gritos.
\section{Gritar}
\begin{itemize}
\item {Grp. gram.:v. i.}
\end{itemize}
\begin{itemize}
\item {Grp. gram.:V. t.}
\end{itemize}
\begin{itemize}
\item {Proveniência:(Do lat. \textunderscore quiritare\textunderscore , de \textunderscore quirites\textunderscore )}
\end{itemize}
Soltar gritos.
Pedir soccorro, bradando.
Falar muito alto.
Queixar-se.
Ralhar.
Proferir em alta voz: \textunderscore gritar injúrias\textunderscore .
Pedir, bradando: \textunderscore gritar auxílio\textunderscore .
\section{Gritaria}
\begin{itemize}
\item {Grp. gram.:f.}
\end{itemize}
Muitos gritos.
Barulho; successão de gritos.
\section{Griteira}
\begin{itemize}
\item {Grp. gram.:f.}
\end{itemize}
\begin{itemize}
\item {Utilização:Prov.}
\end{itemize}
\begin{itemize}
\item {Utilização:trasm.}
\end{itemize}
O mesmo que \textunderscore gritaria\textunderscore .
\section{Grito}
\begin{itemize}
\item {Grp. gram.:m.}
\end{itemize}
\begin{itemize}
\item {Utilização:Ext.}
\end{itemize}
\begin{itemize}
\item {Proveniência:(De \textunderscore gritar\textunderscore )}
\end{itemize}
Voz, emittida com esfôrço, de maneira que possa ouvir-se ao longe.
Vozes inarticuladas, a que nos obriga a dôr ou a paixão.
Clamor de uma multidão.
Voz de alguns animaes: \textunderscore o grito das arapongas\textunderscore .
Palavras empoladas, proferidas em alta voz por um orador ou por um poéta.
\section{Grivar}
\begin{itemize}
\item {Grp. gram.:v. i.}
\end{itemize}
\begin{itemize}
\item {Grp. gram.:M.}
\end{itemize}
\begin{itemize}
\item {Utilização:Náut.}
\end{itemize}
O mesmo que \textunderscore panejar\textunderscore .
O estremecer das testas das velas, quando uma guinada aproxima muito o navio ao vento.
\section{Grizeta}
\begin{itemize}
\item {Grp. gram.:f.}
\end{itemize}
\begin{itemize}
\item {Utilização:Prov.}
\end{itemize}
\begin{itemize}
\item {Utilização:trasm.}
\end{itemize}
Peça de metal, em que se enfia a torcida das lâmpadas.
Lamparina.
Caixa que, nas lanternas, contém o azeite.
Lanterna; luminária.
\section{Gró}
\begin{itemize}
\item {Grp. gram.:m.}
\end{itemize}
Bebida; o mesmo que \textunderscore grogue\textunderscore ? Cf. Macedo, \textunderscore Burros\textunderscore , 257.
\section{Gróbia}
\begin{itemize}
\item {Grp. gram.:f.}
\end{itemize}
\begin{itemize}
\item {Proveniência:(De \textunderscore Groby\textunderscore , n. p.)}
\end{itemize}
Gênero de orquídeas.
\section{Gróbya}
\begin{itemize}
\item {Grp. gram.:f.}
\end{itemize}
\begin{itemize}
\item {Proveniência:(De \textunderscore Groby\textunderscore , n. p.)}
\end{itemize}
Gênero de orchídeas.
\section{Groçaí}
\begin{itemize}
\item {Grp. gram.:m.}
\end{itemize}
\begin{itemize}
\item {Utilização:Bras}
\end{itemize}
Árvore silvestre e leguminosa, cuja madeira serve para frechas.
\section{Grodote}
\begin{itemize}
\item {Grp. gram.:m.}
\end{itemize}
\begin{itemize}
\item {Utilização:Ant.}
\end{itemize}
Espécie de tecido.
\section{Groenlandês}
\begin{itemize}
\item {fónica:gro-e}
\end{itemize}
\begin{itemize}
\item {Grp. gram.:m.}
\end{itemize}
\begin{itemize}
\item {Grp. gram.:Adj.}
\end{itemize}
Habitante da Groenlândia.
Língua, falada pelos habitantes desta região.
Relativo á Groenlândia.
\section{Grogojó}
\begin{itemize}
\item {Grp. gram.:m.}
\end{itemize}
Planta cucurbitácea do Brasil, (\textunderscore cucurbita ovoides\textunderscore ).
\section{Grifa}
\begin{itemize}
\item {Grp. gram.:f.}
\end{itemize}
Fêmea do grifo^1.
\section{Grifar}
\begin{itemize}
\item {Grp. gram.:v. t.}
\end{itemize}
Escrever em grifo ou em letra grifa.
Passar um traço, por baixo de letras ou palavras manuscritas, que se pretende sejam impressas em tipo itálico.
\section{Grifo}
\begin{itemize}
\item {Grp. gram.:m.}
\end{itemize}
\begin{itemize}
\item {Proveniência:(Do gr. \textunderscore grups\textunderscore )}
\end{itemize}
Animal fabuloso, com cabeça de águia e garras de leão.
Ave de rapina, (\textunderscore vultur fulvus\textunderscore ).
\section{Grifo}
\begin{itemize}
\item {Grp. gram.:adj.}
\end{itemize}
\begin{itemize}
\item {Grp. gram.:M.}
\end{itemize}
\begin{itemize}
\item {Proveniência:(De \textunderscore Gryph\textunderscore , n. p. de um professor alemão)}
\end{itemize}
Diz-se de uma fórma de letra, também conhecida por \textunderscore italica\textunderscore , \textunderscore bastarda\textunderscore  e \textunderscore aldina\textunderscore .
Letra itálica ou bastarda.
\section{Grogolejar}
\begin{itemize}
\item {Grp. gram.:v. i.}
\end{itemize}
O mesmo que \textunderscore gorgolejar\textunderscore . Cf. Camillo, \textunderscore Brasileira\textunderscore , 279.
\section{Grogue}
\begin{itemize}
\item {Grp. gram.:m.}
\end{itemize}
\begin{itemize}
\item {Proveniência:(Do ingl. \textunderscore grog\textunderscore .)}
\end{itemize}
Bebida alcoólica, misturada com água, açúcar e casca de limão.
\section{Groja}
\begin{itemize}
\item {Grp. gram.:f.}
\end{itemize}
\begin{itemize}
\item {Utilização:Prov.}
\end{itemize}
\begin{itemize}
\item {Utilização:alg.}
\end{itemize}
Garganta forte.
Voz forte.
(Metáth. de \textunderscore gorja\textunderscore )
\section{Grojeira}
\begin{itemize}
\item {Grp. gram.:f.}
\end{itemize}
\begin{itemize}
\item {Utilização:Prov.}
\end{itemize}
\begin{itemize}
\item {Utilização:trasm.}
\end{itemize}
\begin{itemize}
\item {Proveniência:(De \textunderscore groja\textunderscore )}
\end{itemize}
O mesmo que \textunderscore collarinho\textunderscore .
\section{Grojer}
\begin{itemize}
\item {Grp. gram.:v. i.}
\end{itemize}
\begin{itemize}
\item {Utilização:Prov.}
\end{itemize}
\begin{itemize}
\item {Utilização:trasm.}
\end{itemize}
\begin{itemize}
\item {Proveniência:(De \textunderscore groja\textunderscore ?)}
\end{itemize}
Chorar; gemer.
Rugir. (Colhido em V. P. de Aguiar)
\section{Grolado}
\begin{itemize}
\item {Grp. gram.:m.}
\end{itemize}
\begin{itemize}
\item {Utilização:Bras. do N}
\end{itemize}
Doce de algumas frutas, a que se não tira a casca: \textunderscore grolado de caju\textunderscore ; \textunderscore grolado de goiaba\textunderscore .
\section{Grolar}
\begin{itemize}
\item {Grp. gram.:v. t.  e  i.}
\end{itemize}
O mesmo que \textunderscore gorar\textunderscore . Cf. Camillo, \textunderscore Cancion. Al\textunderscore , 326; Arn. Gama, \textunderscore Motim\textunderscore , 215.
\section{Grólia}
\begin{itemize}
\item {Grp. gram.:f.}
\end{itemize}
(Metáth. pop. \textunderscore glória\textunderscore ). Cf. Lobo, \textunderscore Auto do Nascimento\textunderscore .
\section{Grôlo}
\begin{itemize}
\item {Grp. gram.:adj.}
\end{itemize}
O mesmo que \textunderscore gôro\textunderscore . Cf. Camillo, \textunderscore Noites de Insómn.\textunderscore , V, 89.
\section{Groma}
\begin{itemize}
\item {Grp. gram.:f.}
\end{itemize}
\begin{itemize}
\item {Utilização:Prov.}
\end{itemize}
\begin{itemize}
\item {Utilização:trasm.}
\end{itemize}
\begin{itemize}
\item {Proveniência:(Do cast. \textunderscore broma\textunderscore )}
\end{itemize}
Folgança ruidosa.
Pândega.
\section{Groma}
\begin{itemize}
\item {Grp. gram.:f.}
\end{itemize}
\begin{itemize}
\item {Proveniência:(Lat. \textunderscore groma\textunderscore )}
\end{itemize}
Vara de sete pés, com que os Romanos mediam os campos.
\section{Gromática}
\begin{itemize}
\item {Grp. gram.:f.}
\end{itemize}
\begin{itemize}
\item {Proveniência:(De \textunderscore gromático\textunderscore )}
\end{itemize}
Arte de agrimensura.
\section{Gromático}
\begin{itemize}
\item {Grp. gram.:adj.}
\end{itemize}
\begin{itemize}
\item {Proveniência:(Lat. \textunderscore gromaticus\textunderscore )}
\end{itemize}
Relativo á agrimensura.
\section{Gromphena}
\begin{itemize}
\item {Grp. gram.:f.}
\end{itemize}
\begin{itemize}
\item {Proveniência:(Lat. \textunderscore gromphena\textunderscore )}
\end{itemize}
Ave da Sardenha, semelhante ao grou.
\section{Gronfena}
\begin{itemize}
\item {Grp. gram.:f.}
\end{itemize}
\begin{itemize}
\item {Proveniência:(Lat. \textunderscore gromphena\textunderscore )}
\end{itemize}
Ave da Sardenha, semelhante ao grou.
\section{Gronho}
\begin{itemize}
\item {Grp. gram.:m.}
\end{itemize}
Variedade de pêra, muito conhecida ao norte do país.
Variedade de maçan.
\section{Gronóvia}
\begin{itemize}
\item {Grp. gram.:f.}
\end{itemize}
Gênero de plantas parietárias.
\section{Gropa}
\begin{itemize}
\item {Grp. gram.:f.}
\end{itemize}
\begin{itemize}
\item {Utilização:Ant.}
\end{itemize}
O mesmo que \textunderscore garupa\textunderscore . Cf. \textunderscore Viriato Trág.\textunderscore , XIX, 25.
\section{Grós}
\begin{itemize}
\item {Grp. gram.:m.}
\end{itemize}
\begin{itemize}
\item {Utilização:Ant.}
\end{itemize}
\textunderscore Vender a grós\textunderscore , vender por grosso, por junto.
(Cp. fr. \textunderscore gros\textunderscore )
\section{Grosa}
\begin{itemize}
\item {Grp. gram.:f.}
\end{itemize}
\begin{itemize}
\item {Proveniência:(Do fr. \textunderscore grosse\textunderscore )}
\end{itemize}
Doze dúzias.
\section{Grosa}
\begin{itemize}
\item {Grp. gram.:f.}
\end{itemize}
Instrumento de aço ou ferro, semelhante á lima, para desbastar madeira ou ferro.
Faca, de fio embotado, para descarnar pelles.
\section{Grosa}
\begin{itemize}
\item {Grp. gram.:f.}
\end{itemize}
\begin{itemize}
\item {Utilização:Ant.}
\end{itemize}
O mesmo que \textunderscore glosa\textunderscore . Cf. G. Vicente, I, 228.
Maledicência, murmuração. Cf. Pant. de Aveiro, \textunderscore Itiner.\textunderscore , 251, (3.^a ed.).
\section{Grosador}
\begin{itemize}
\item {Grp. gram.:m.}
\end{itemize}
\begin{itemize}
\item {Proveniência:(De \textunderscore grosar\textunderscore ^1)}
\end{itemize}
Aquelle que murmura ou diffama. Cf. Pant. de Aveiro, \textunderscore Itiner.\textunderscore , 3, (2.^a ed.).
\section{Grosar}
\begin{itemize}
\item {Grp. gram.:v. t.}
\end{itemize}
\begin{itemize}
\item {Utilização:Ant.}
\end{itemize}
Glosar. Cf. \textunderscore Eufrosina\textunderscore , (passim).
\section{Grosar}
\begin{itemize}
\item {Grp. gram.:v. t.}
\end{itemize}
Desbastar ou alisar com grosa^2.
\section{Groseira}
\begin{itemize}
\item {Grp. gram.:f.}
\end{itemize}
\begin{itemize}
\item {Utilização:Prov.}
\end{itemize}
\begin{itemize}
\item {Utilização:alent.}
\end{itemize}
Corda.
Corda com muitos anzoes para a pesca.
Barco que se emprega na pesca ao anzol.
\section{Groselha}
\begin{itemize}
\item {fónica:grosê}
\end{itemize}
\begin{itemize}
\item {Grp. gram.:f.}
\end{itemize}
\begin{itemize}
\item {Grp. gram.:Adj.}
\end{itemize}
\begin{itemize}
\item {Proveniência:(Fr. \textunderscore groseille\textunderscore )}
\end{itemize}
Fruto da groselheira.
Xarope de groselhas.
Que tem a côr acerejada da groselha.
\section{Groselheira}
\begin{itemize}
\item {Grp. gram.:f.}
\end{itemize}
\begin{itemize}
\item {Proveniência:(De \textunderscore groselha\textunderscore . Cp. \textunderscore grossulária\textunderscore )}
\end{itemize}
Planta grossulária.
\section{Groselheiro}
\begin{itemize}
\item {Grp. gram.:m.}
\end{itemize}
O mesmo que \textunderscore groselheira\textunderscore .
\section{Grosmar}
\begin{itemize}
\item {Grp. gram.:v. t.  e  i.}
\end{itemize}
O mesmo que \textunderscore gosmar\textunderscore . Cf. \textunderscore Eufrosina\textunderscore , 327.
\section{Grossa}
\begin{itemize}
\item {Grp. gram.:f.}
\end{itemize}
\begin{itemize}
\item {Utilização:Ant.}
\end{itemize}
O mesmo que \textunderscore glosa\textunderscore .
\section{Grossagrana}
\begin{itemize}
\item {Grp. gram.:f.}
\end{itemize}
\begin{itemize}
\item {Proveniência:(Do it. \textunderscore grosso\textunderscore  + \textunderscore grana\textunderscore )}
\end{itemize}
Tecido napolitano, semelhante ao tafetá, mas mais encorpado.
\section{Grossamento}
\begin{itemize}
\item {Grp. gram.:m.}
\end{itemize}
\begin{itemize}
\item {Utilização:Ant.}
\end{itemize}
Arte de grossar.
Glosa; entrelinha.
\section{Grossar}
\begin{itemize}
\item {Grp. gram.:v. t.}
\end{itemize}
\begin{itemize}
\item {Utilização:Ant.}
\end{itemize}
O mesmo que \textunderscore glosar\textunderscore .
\section{Grossaria}
\begin{itemize}
\item {Grp. gram.:f.}
\end{itemize}
\begin{itemize}
\item {Proveniência:(De \textunderscore grosso\textunderscore )}
\end{itemize}
Tecido grosso de linho ou algodão.
Falta de delicadeza, incivilidade, expressão grosseira.
\section{Grosseira}
\begin{itemize}
\item {Grp. gram.:f.}
\end{itemize}
\begin{itemize}
\item {Utilização:Bras. de Minas}
\end{itemize}
Qualquer erupção de pelle, (sarna, brotoeja, etc.).
\section{Grosseiramente}
\begin{itemize}
\item {Grp. gram.:adv.}
\end{itemize}
De modo grosseiro.
Incivilmente.
\section{Grosseirão}
\begin{itemize}
\item {Grp. gram.:adj.}
\end{itemize}
\begin{itemize}
\item {Grp. gram.:M.  e  adj.}
\end{itemize}
\begin{itemize}
\item {Utilização:Fig.}
\end{itemize}
\begin{itemize}
\item {Proveniência:(De \textunderscore grosseiro\textunderscore )}
\end{itemize}
Muito grosso.
Ordinário.
Mal educado; incivil.
\section{Grosseirismo}
\begin{itemize}
\item {Grp. gram.:m.}
\end{itemize}
Modos ou hábito de grosseiro.
Qualidade de grosseiro. Cf. Camillo, \textunderscore Estrêll. Funestas\textunderscore , 21.
\section{Grosseiro}
\begin{itemize}
\item {Grp. gram.:adj.}
\end{itemize}
\begin{itemize}
\item {Utilização:Fig.}
\end{itemize}
Que é grosso ou de má qualidade.
Mal feito, rude.
Áspero.
Incivil.
Immoral.
Inculto.
Immundo.
\section{Grossidão}
\begin{itemize}
\item {Grp. gram.:f.}
\end{itemize}
\begin{itemize}
\item {Utilização:Ant.}
\end{itemize}
O mesmo que \textunderscore grossura\textunderscore .
\section{Grossina}
\begin{itemize}
\item {Grp. gram.:f.}
\end{itemize}
\begin{itemize}
\item {Utilização:Prov.}
\end{itemize}
\begin{itemize}
\item {Utilização:alent.}
\end{itemize}
\begin{itemize}
\item {Proveniência:(De \textunderscore grosso\textunderscore )}
\end{itemize}
Crosta esbranquiçada da língua.
O mesmo que \textunderscore saburra\textunderscore .
\section{Grôsso}
\begin{itemize}
\item {Grp. gram.:adj.}
\end{itemize}
\begin{itemize}
\item {Utilização:Ant.}
\end{itemize}
\begin{itemize}
\item {Grp. gram.:M.}
\end{itemize}
\begin{itemize}
\item {Utilização:Gír.}
\end{itemize}
\begin{itemize}
\item {Grp. gram.:Adv.}
\end{itemize}
\begin{itemize}
\item {Proveniência:(Lat. \textunderscore grossus\textunderscore )}
\end{itemize}
Que tem grande circunferência ou volume: \textunderscore árvore grossa\textunderscore .
Sólido.
Compacto.
Grave.
Abundante.
Grosseiro, (na accepção prop. e fig.).
Consistente, espêsso: \textunderscore caldo grôsso\textunderscore .
Fértil, productivo. Cf. Pant. de Aveiro, \textunderscore Itiner.\textunderscore , 53 e 255 v.^o, (2.^a ed.).
A parte mais grossa.
A maior parte.
Antiga moéda portuguesa, equivalente ao real de prata.
Bêbedo.
Muito.
Com fôrça.
Com gravidade; em tom baixo: \textunderscore falar grôsso\textunderscore .
\section{Grôsso-de-nápoles}
\begin{itemize}
\item {Grp. gram.:m.}
\end{itemize}
O mesmo que \textunderscore grossagrana\textunderscore .
\section{Grossor}
\begin{itemize}
\item {Grp. gram.:m.}
\end{itemize}
\begin{itemize}
\item {Utilização:Prov.}
\end{itemize}
\begin{itemize}
\item {Utilização:alg.}
\end{itemize}
\begin{itemize}
\item {Utilização:Ant.}
\end{itemize}
O mesmo que \textunderscore grossura\textunderscore .
\section{Grossulária}
\begin{itemize}
\item {Grp. gram.:f.}
\end{itemize}
\begin{itemize}
\item {Proveniência:(Lat. \textunderscore grossularia\textunderscore )}
\end{itemize}
Antigo nome scientífico da groselheira.
Hoje, secção do gênero groselheira.
\section{Grossulárias}
\begin{itemize}
\item {Grp. gram.:f. pl.}
\end{itemize}
Família de plantas dicotyledóneas, que têm por typo a groselheira.
(Pl. de \textunderscore grossularia\textunderscore )
\section{Grossularina}
\begin{itemize}
\item {Grp. gram.:f.}
\end{itemize}
\begin{itemize}
\item {Proveniência:(De \textunderscore grossulária\textunderscore )}
\end{itemize}
Substância, que se acha nos frutos ácidos, sob a fórma de geleia.
\section{Grossulina}
\begin{itemize}
\item {Grp. gram.:f.}
\end{itemize}
O mesmo que \textunderscore grossularina\textunderscore .
\section{Grossura}
\begin{itemize}
\item {Grp. gram.:f.}
\end{itemize}
\begin{itemize}
\item {Utilização:Gír.}
\end{itemize}
\begin{itemize}
\item {Utilização:Ant.}
\end{itemize}
Qualidade daquelle ou daquillo que é grosso.
Corpulência.
Medida de um sólido, entre a sua superfície anterior e a posterior.
Bebedeira.
Fertilidade. Cf. Pant. de Aveiro, \textunderscore Itiner.\textunderscore , 328, (2.^a ed.).
\section{Grota}
\begin{itemize}
\item {Grp. gram.:f.}
\end{itemize}
\begin{itemize}
\item {Utilização:Bras}
\end{itemize}
Abertura, feita pelas águas na ribanceira ou margem de um rio, e pela qual ellas sáem, alagando os campos marginaes.
Terreno em plano inclinado, na intersecção de duas montanhas.
(Cp. \textunderscore gruta\textunderscore )
\section{Grota}
\begin{itemize}
\item {Grp. gram.:m.}
\end{itemize}
\begin{itemize}
\item {Utilização:Bras. de Goiás}
\end{itemize}
Indivíduo de alta posição social.
\section{Grotão}
\begin{itemize}
\item {Grp. gram.:m.}
\end{itemize}
\begin{itemize}
\item {Utilização:Bras}
\end{itemize}
Grande grota^1. Cf. Art. Guimarães, \textunderscore Fazenda do Paraíso\textunderscore .
\section{Grotesco}
\begin{itemize}
\item {fónica:tês}
\end{itemize}
\textunderscore m.\textunderscore  e \textunderscore adj.\textunderscore  (e der.)
O mesmo que \textunderscore grutesco\textunderscore , etc.
\section{Grou}
\begin{itemize}
\item {Grp. gram.:m.}
\end{itemize}
\begin{itemize}
\item {Proveniência:(Do lat. \textunderscore grus\textunderscore )}
\end{itemize}
Ave pernalta, da fam. dos cultrirostros, (\textunderscore grus cinerea\textunderscore ).
Constellação austral.
\section{Gròzinho}
\begin{itemize}
\item {Grp. gram.:m.}
\end{itemize}
\begin{itemize}
\item {Utilização:Prov.}
\end{itemize}
\begin{itemize}
\item {Proveniência:(De \textunderscore gró?\textunderscore )}
\end{itemize}
Refeição no campo; piqueninque.
\section{Grua}
\begin{itemize}
\item {Grp. gram.:f.}
\end{itemize}
\begin{itemize}
\item {Utilização:Náut.}
\end{itemize}
\begin{itemize}
\item {Proveniência:(Fr. \textunderscore grue\textunderscore )}
\end{itemize}
Roldana do guindaste da prôa.
Maquinismo com calabre, para levantar grandes pesos.
Máquina, para introduzir água nas locomotivas.
\section{Gruaria}
\begin{itemize}
\item {Grp. gram.:f.}
\end{itemize}
\begin{itemize}
\item {Utilização:Ant.}
\end{itemize}
\begin{itemize}
\item {Proveniência:(De \textunderscore gruim\textunderscore )}
\end{itemize}
Herdade, que pagava fôro de gruim.
\section{Gruau}
\begin{itemize}
\item {Grp. gram.:m.}
\end{itemize}
\begin{itemize}
\item {Utilização:Prov.}
\end{itemize}
O mesmo que \textunderscore maçarico\textunderscore .
\section{Grudadoiro}
\begin{itemize}
\item {Grp. gram.:m.}
\end{itemize}
\begin{itemize}
\item {Proveniência:(De \textunderscore grudar\textunderscore )}
\end{itemize}
Série de cavalletes de madeira ou de ferro, sôbre os quaes se estendem as teias para secar, nas fábricas de lanificios, depois de mergulhadas em colla ou grude.
\section{Grudadouro}
\begin{itemize}
\item {Grp. gram.:m.}
\end{itemize}
\begin{itemize}
\item {Proveniência:(De \textunderscore grudar\textunderscore )}
\end{itemize}
Série de cavalletes de madeira ou de ferro, sôbre os quaes se estendem as teias para secar, nas fábricas de lanificios, depois de mergulhadas em colla ou grude.
\section{Grudador}
\begin{itemize}
\item {Grp. gram.:m.  e  adj.}
\end{itemize}
O que gruda.
\section{Grudadura}
\begin{itemize}
\item {Grp. gram.:f.}
\end{itemize}
Acto ou effeito de grudar.
\section{Grudar}
\begin{itemize}
\item {Grp. gram.:v. t.}
\end{itemize}
\begin{itemize}
\item {Utilização:Prov.}
\end{itemize}
\begin{itemize}
\item {Utilização:trasm.}
\end{itemize}
\begin{itemize}
\item {Grp. gram.:V. i.}
\end{itemize}
\begin{itemize}
\item {Utilização:Fig.}
\end{itemize}
\begin{itemize}
\item {Utilização:Gír.}
\end{itemize}
\begin{itemize}
\item {Utilização:Chul.}
\end{itemize}
\begin{itemize}
\item {Grp. gram.:V. p.}
\end{itemize}
\begin{itemize}
\item {Utilização:Bras. do N}
\end{itemize}
Pegar com grude.
Ligar; unir.
Illudir, lograr.
Ligar-se com grude.
Juntar-se.
Ajustar-se; combinar-se.
Convir.
Us. na loc. \textunderscore não gruda\textunderscore , não péga, não me convence, não vale nada.
Unirem-se os corpos de dois indivíduos, para lutar.
\section{Grude}
\begin{itemize}
\item {Grp. gram.:m.  ou  f.}
\end{itemize}
\begin{itemize}
\item {Utilização:Bras. do N}
\end{itemize}
\begin{itemize}
\item {Proveniência:(Do lat. \textunderscore gluten\textunderscore )}
\end{itemize}
Espécie de colla, com que se unem e pegam peças de madeira.
Massa, usada na fabricação do calçado.
Luta corporal de duas pessôas.
Desordem, motim.
\section{Grueiro}
\begin{itemize}
\item {Grp. gram.:adj.}
\end{itemize}
Diz-se do falcão, ensinado para caçar grous.
(Por \textunderscore groueiro\textunderscore , de \textunderscore grou\textunderscore )
\section{Grugulejar}
\begin{itemize}
\item {Grp. gram.:v. i.}
\end{itemize}
Cantar (o peru)
Imitar a voz do peru:«\textunderscore outras vezes, em quanto elle latia de cão e grugulejava de peru ou miava de gato...\textunderscore »Camillo, \textunderscore Volcões\textunderscore , 157. (T. onom.).
\section{Grugunzar}
\begin{itemize}
\item {Grp. gram.:v. i.}
\end{itemize}
\begin{itemize}
\item {Utilização:Bras. do N}
\end{itemize}
Meditar; parafusar.
\section{Grugutuba}
\begin{itemize}
\item {Grp. gram.:m.}
\end{itemize}
Casta de feijão.
\section{Gruieiro}
\begin{itemize}
\item {Grp. gram.:adj.}
\end{itemize}
\begin{itemize}
\item {Utilização:Ant.}
\end{itemize}
O mesmo que \textunderscore grueiro\textunderscore .
\section{Gruim}
\begin{itemize}
\item {Grp. gram.:m.}
\end{itemize}
\begin{itemize}
\item {Utilização:Ant.}
\end{itemize}
Focinho de porco.
Porco.
O mesmo que [[rabeiras|rabeira]] ou varreduras de cereaes, na eira, para os porcos.
(Cp. fr. \textunderscore groin\textunderscore )
\section{Gruir}
\begin{itemize}
\item {Grp. gram.:v. i.}
\end{itemize}
\begin{itemize}
\item {Utilização:Ant.}
\end{itemize}
Correr, fazendo algazarra.
(Relaciona-se com \textunderscore grueiro\textunderscore ? ou com \textunderscore gruim\textunderscore ?)
\section{Grulha}
\begin{itemize}
\item {Grp. gram.:m.  e  f.}
\end{itemize}
\begin{itemize}
\item {Grp. gram.:M.}
\end{itemize}
\begin{itemize}
\item {Utilização:Gír.}
\end{itemize}
\begin{itemize}
\item {Grp. gram.:F.}
\end{itemize}
\begin{itemize}
\item {Utilização:Des.}
\end{itemize}
Pessôa, que fala muito.
Porco.
O mesmo que \textunderscore barulho\textunderscore . Cf. Filinto, IV, 205.
\section{Grulhada}
\begin{itemize}
\item {Grp. gram.:f.}
\end{itemize}
\begin{itemize}
\item {Utilização:Fig.}
\end{itemize}
\begin{itemize}
\item {Proveniência:(De \textunderscore grulhar\textunderscore )}
\end{itemize}
Vozes de grou.
Gritaria.
\section{Grulhar}
\begin{itemize}
\item {Grp. gram.:v. i.}
\end{itemize}
\begin{itemize}
\item {Proveniência:(De \textunderscore grulha\textunderscore )}
\end{itemize}
Falar muito.
Palrar; tagarelar.
\section{Grulhento}
\begin{itemize}
\item {Grp. gram.:adj.}
\end{itemize}
\begin{itemize}
\item {Utilização:Prov.}
\end{itemize}
\begin{itemize}
\item {Utilização:alent.}
\end{itemize}
Que grulha, que é muito falador.
\section{Grumar}
\begin{itemize}
\item {Grp. gram.:v. t.}
\end{itemize}
\begin{itemize}
\item {Grp. gram.:V. i.  e  p.}
\end{itemize}
Dar fórma de grumo a.
Reduzir a grumos.
Tomar a fórma de grumos.
\section{Grumati}
\begin{itemize}
\item {Grp. gram.:m.}
\end{itemize}
Grande árvore medicinal da ilha de San-Thomé.
\section{Grumecência}
\begin{itemize}
\item {Grp. gram.:f.}
\end{itemize}
Estado daquillo que engrumeceu.
Propriedade daquillo que póde engrumecer.
\section{Grumecer}
\begin{itemize}
\item {Grp. gram.:v. t., i.  e  p.}
\end{itemize}
\begin{itemize}
\item {Proveniência:(De \textunderscore grumo\textunderscore )}
\end{itemize}
O mesmo que \textunderscore engrumecer\textunderscore .
\section{Grumetagem}
\begin{itemize}
\item {Grp. gram.:f.}
\end{itemize}
Os grumetes de um navio de guerra.
\section{Grumete}
\begin{itemize}
\item {fónica:méoumê}
\end{itemize}
\begin{itemize}
\item {Grp. gram.:m.}
\end{itemize}
\begin{itemize}
\item {Proveniência:(Do ingl. \textunderscore groon\textunderscore  + \textunderscore mate\textunderscore )}
\end{itemize}
Marinheiro, que tem na armada a graduação inferior.
Habitante de Cacheu.
\section{Grumixá}
\begin{itemize}
\item {Grp. gram.:m.}
\end{itemize}
\begin{itemize}
\item {Utilização:Bras}
\end{itemize}
Espécie de casulo córneo, que se encontra nos rios e pertence a uma larva.
\section{Grumixama}
\begin{itemize}
\item {Grp. gram.:f.}
\end{itemize}
\begin{itemize}
\item {Utilização:Bras}
\end{itemize}
Fruto da gramixameira.
O mesmo que \textunderscore grumixameira\textunderscore .
(Do tupi \textunderscore ibamíxana\textunderscore )
\section{Grumixameira}
\begin{itemize}
\item {Grp. gram.:f.}
\end{itemize}
\begin{itemize}
\item {Proveniência:(De \textunderscore grumixama\textunderscore )}
\end{itemize}
Arvoreta myrtácea da América.
\section{Grumo}
\begin{itemize}
\item {Grp. gram.:m.}
\end{itemize}
\begin{itemize}
\item {Proveniência:(Lat. \textunderscore grumus\textunderscore )}
\end{itemize}
Grânulo.
Pequeno coágulo de albumina, de fibrína ou de caseína.
\section{Grumoso}
\begin{itemize}
\item {Grp. gram.:adj.}
\end{itemize}
\begin{itemize}
\item {Proveniência:(De \textunderscore grumo\textunderscore )}
\end{itemize}
O mesmo que \textunderscore granuloso\textunderscore .
\section{Grumuchama}
\begin{itemize}
\item {Grp. gram.:m.}
\end{itemize}
(V.grumixama)
\section{Grúmulo}
\begin{itemize}
\item {Grp. gram.:m.}
\end{itemize}
\begin{itemize}
\item {Proveniência:(Lat. \textunderscore grumulus\textunderscore )}
\end{itemize}
Pequeno grumo.
\section{Gruna}
\begin{itemize}
\item {Grp. gram.:f.}
\end{itemize}
\begin{itemize}
\item {Utilização:Bras}
\end{itemize}
Lugar, onde trabalham garimpeiros.
\section{Grunha}
\begin{itemize}
\item {Grp. gram.:f.}
\end{itemize}
Variedade de maçan.
(Cp. \textunderscore gronho\textunderscore )
\section{Grunhatá}
\begin{itemize}
\item {Grp. gram.:m.}
\end{itemize}
\begin{itemize}
\item {Utilização:Bras}
\end{itemize}
Pequeno pássaro, amarelo por baixo e escuro por cima, que alguns confundem com o gaturamo. Cf. B. C. Rubim, \textunderscore Vocabulário Bras.\textunderscore 
\section{Grunhideira}
\begin{itemize}
\item {Grp. gram.:f.}
\end{itemize}
\begin{itemize}
\item {Utilização:Gír.}
\end{itemize}
\begin{itemize}
\item {Proveniência:(De \textunderscore grunhir\textunderscore )}
\end{itemize}
A língua.
\section{Grunhideiro}
\begin{itemize}
\item {Grp. gram.:adj.}
\end{itemize}
Que grunhe. Cf. Filinto, V, 68.
\section{Grunhidela}
\begin{itemize}
\item {Grp. gram.:f.}
\end{itemize}
Acto de grunhir.
Acto de resmungar. Cf. Castilho, \textunderscore Doente de Scisma\textunderscore , 63.
\section{Grunhido}
\begin{itemize}
\item {Grp. gram.:m.}
\end{itemize}
\begin{itemize}
\item {Proveniência:(Do lat. \textunderscore grunnitus\textunderscore )}
\end{itemize}
A voz do porco.
\section{Grunhidor}
\begin{itemize}
\item {Grp. gram.:adj.}
\end{itemize}
\begin{itemize}
\item {Grp. gram.:M.}
\end{itemize}
Que grunhe.
Aquelle que grunhe.
\section{Grunhir}
\begin{itemize}
\item {Grp. gram.:v. i.}
\end{itemize}
\begin{itemize}
\item {Proveniência:(Lat. \textunderscore grunnire\textunderscore )}
\end{itemize}
Soltar grunhidos (o porco).
Resmungar.
Soltar vozes que lembram a do porco.
\section{Grupa}
\begin{itemize}
\item {Grp. gram.:f.}
\end{itemize}
\begin{itemize}
\item {Utilização:Ant.}
\end{itemize}
O mesmo que \textunderscore garupa\textunderscore . Cf. \textunderscore Viriato Trág.\textunderscore , XVI, 39 e 70.
\section{Grupamento}
\begin{itemize}
\item {Grp. gram.:m.}
\end{itemize}
Acto ou effeito de grupar.
\section{Grupar}
\begin{itemize}
\item {Grp. gram.:v. t.}
\end{itemize}
O mesmo que \textunderscore agrupar\textunderscore .
\section{Grupo}
\begin{itemize}
\item {Grp. gram.:m.}
\end{itemize}
\begin{itemize}
\item {Proveniência:(It. \textunderscore gruppo\textunderscore )}
\end{itemize}
Reunião de objectos, que se vêem de uma vez ou com um lance de olhos.
Conjunto de coisas, que formam um todo.
Reunião de pessôas.
Pequena associação.
\section{Gruta}
\begin{itemize}
\item {Grp. gram.:f.}
\end{itemize}
\begin{itemize}
\item {Proveniência:(Do lat. \textunderscore crupta\textunderscore )}
\end{itemize}
Caverna natural ou artificial; antro; lapa.
\section{Grutescamente}
\begin{itemize}
\item {fónica:tês}
\end{itemize}
\begin{itemize}
\item {Grp. gram.:adv.}
\end{itemize}
De modo grutesco.
\section{Grutesco}
\begin{itemize}
\item {fónica:tês}
\end{itemize}
\begin{itemize}
\item {Grp. gram.:adj.}
\end{itemize}
\begin{itemize}
\item {Grp. gram.:M. pl.}
\end{itemize}
\begin{itemize}
\item {Proveniência:(De \textunderscore gruta\textunderscore )}
\end{itemize}
Ridículo; caricato.
Pintura ou escultura, que representa grutas.
Espécie de arabescos.
Ornatos artísticos, que reproduzem objectos da natureza.
\section{Gruzínio}
\begin{itemize}
\item {Grp. gram.:m.}
\end{itemize}
Uma das línguas do Cáucaso, o mesmo que \textunderscore georgiano\textunderscore .
\section{Grypha}
\begin{itemize}
\item {Grp. gram.:f.}
\end{itemize}
Fêmea do grypho^1.
\section{Gryphar}
\begin{itemize}
\item {Grp. gram.:v. t.}
\end{itemize}
Escrever em grypho ou em letra grypha.
Passar um traço, por baixo de letras ou palavras manuscritas, que se pretende sejam impressas em typo itálico.
\section{Grýphico}
\begin{itemize}
\item {Grp. gram.:adj.}
\end{itemize}
Relativo ao grypho^1.
\section{Grypho}
\begin{itemize}
\item {Grp. gram.:m.}
\end{itemize}
\begin{itemize}
\item {Proveniência:(Do gr. \textunderscore grups\textunderscore )}
\end{itemize}
Animal fabuloso, com cabeça de águia e garras de leão.
Ave de rapina, (\textunderscore vultur fulvus\textunderscore ).
\section{Grypho}
\begin{itemize}
\item {Grp. gram.:adj.}
\end{itemize}
\begin{itemize}
\item {Grp. gram.:M.}
\end{itemize}
\begin{itemize}
\item {Proveniência:(De \textunderscore Gryph\textunderscore , n. p. de um professor alemão)}
\end{itemize}
Diz-se de uma fórma de letra, também conhecida por \textunderscore italica\textunderscore , \textunderscore bastarda\textunderscore  e \textunderscore aldina\textunderscore .
Letra itálica ou bastarda.
\section{Guaaibe-ambe}
\begin{itemize}
\item {Grp. gram.:m.}
\end{itemize}
Planta brasileira, da fam. das myrtáceas, (\textunderscore psidium aromaticum\textunderscore ).
\section{Guabiju}
\begin{itemize}
\item {Grp. gram.:m.}
\end{itemize}
Fruto do guabijueiro.
O mesmo que \textunderscore guabijueiro\textunderscore .
\section{Guabijueiro}
\begin{itemize}
\item {Grp. gram.:m.}
\end{itemize}
\begin{itemize}
\item {Proveniência:(De \textunderscore guabiju\textunderscore )}
\end{itemize}
Arvoreta myrtácea do Brasil.
\section{Guabira}
\begin{itemize}
\item {Grp. gram.:f.}
\end{itemize}
O mesmo que \textunderscore guabiju\textunderscore .
\section{Guabiraba}
\begin{itemize}
\item {Grp. gram.:f.}
\end{itemize}
O mesmo que \textunderscore guabirabeira\textunderscore ; fruto da guabirabeira.
\section{Guabirabeira}
\begin{itemize}
\item {Grp. gram.:f.}
\end{itemize}
\begin{itemize}
\item {Proveniência:(De \textunderscore guabiraba\textunderscore )}
\end{itemize}
Planta borragínea do Brasil.
\section{Guabiraguaçu}
\begin{itemize}
\item {Grp. gram.:m.}
\end{itemize}
O mesmo que \textunderscore guabijueiro\textunderscore .
\section{Guabiroba}
\begin{itemize}
\item {Grp. gram.:f.}
\end{itemize}
Fruta da guabirobeira; o mesmo que \textunderscore guabirobeira\textunderscore .
\section{Guabirobeira}
\begin{itemize}
\item {Grp. gram.:f.}
\end{itemize}
\begin{itemize}
\item {Utilização:Bras}
\end{itemize}
\begin{itemize}
\item {Proveniência:(De \textunderscore guabiroba\textunderscore )}
\end{itemize}
Nome de diversas espécies de árvores myrtáceas.
\section{Guabirota}
\begin{itemize}
\item {Grp. gram.:f.}
\end{itemize}
\begin{itemize}
\item {Utilização:Bras}
\end{itemize}
Âmago amargo das extremidades de certas palmeiras.
\section{Guabiru}
\begin{itemize}
\item {Grp. gram.:m.}
\end{itemize}
\begin{itemize}
\item {Utilização:Bras. do N}
\end{itemize}
\begin{itemize}
\item {Proveniência:(T. tupi)}
\end{itemize}
Espécie de rato.
\section{Guaburu}
\begin{itemize}
\item {Grp. gram.:m.}
\end{itemize}
\begin{itemize}
\item {Utilização:Bras}
\end{itemize}
Árvore silvestre, que dá boa madeira.
\section{Guacá}
\begin{itemize}
\item {Grp. gram.:m.}
\end{itemize}
\begin{itemize}
\item {Utilização:Bras}
\end{itemize}
\begin{itemize}
\item {Proveniência:(T. tupi)}
\end{itemize}
Nome vulgar de duas espécies de árvores sapotáceas.
\section{Guacarés}
\begin{itemize}
\item {Grp. gram.:m. pl.}
\end{itemize}
Tribo do Alto-Amazonas.
\section{Guaça-tinga}
\begin{itemize}
\item {Grp. gram.:f.}
\end{itemize}
\begin{itemize}
\item {Utilização:Bras}
\end{itemize}
Árvore, cuja madeira se emprega em construcções.
\section{Guache}
\begin{itemize}
\item {Grp. gram.:m.}
\end{itemize}
\begin{itemize}
\item {Utilização:Bras}
\end{itemize}
Ave negra, de cauda amarela, proximamente do tamanho de uma pomba.
\section{Guache}
\begin{itemize}
\item {Grp. gram.:m.}
\end{itemize}
O mesmo que \textunderscore guacho\textunderscore ^1.
\section{Guacho}
\begin{itemize}
\item {Grp. gram.:m.}
\end{itemize}
\begin{itemize}
\item {Proveniência:(Fr. \textunderscore gouache\textunderscore )}
\end{itemize}
Pintura brilhante e avelludada, feita com uma mistura de tintas em pó, água e goma arábica.
\section{Guacho}
\begin{itemize}
\item {Grp. gram.:adj.}
\end{itemize}
\begin{itemize}
\item {Utilização:Bras}
\end{itemize}
Diz-se do cavallinho ou do bezerro criado em casa, ou que é criado pela própria mãe.
(Do quíchua \textunderscore huaccha\textunderscore )
\section{Guacina}
\begin{itemize}
\item {Grp. gram.:f.}
\end{itemize}
Substância amarga, que se extrái do guaco.
\section{Guaco}
\begin{itemize}
\item {Grp. gram.:m.}
\end{itemize}
Nome indígena de uma planta synanthérea, que cresce á beira do rio Madalena, na América, (\textunderscore mikania guaco\textunderscore )
\section{Guaçu}
\begin{itemize}
\item {Grp. gram.:adj.}
\end{itemize}
\begin{itemize}
\item {Utilização:Bras}
\end{itemize}
\begin{itemize}
\item {Proveniência:(T. tupi)}
\end{itemize}
Grande; maior.
\section{Guassu}
\begin{itemize}
\item {Grp. gram.:adj.}
\end{itemize}
\begin{itemize}
\item {Utilização:Bras}
\end{itemize}
\begin{itemize}
\item {Proveniência:(T. tupi)}
\end{itemize}
Grande; maior.
\section{Guacuman}
\begin{itemize}
\item {Grp. gram.:m.}
\end{itemize}
\begin{itemize}
\item {Utilização:Bras}
\end{itemize}
Espécie de palmeira, cuja casca, muito combustível, se emprega como isca.
\section{Guacuris}
\begin{itemize}
\item {Grp. gram.:m.}
\end{itemize}
\begin{itemize}
\item {Utilização:Bras}
\end{itemize}
Espécie de palmeira dos sertões.
\section{Guadamecil}
\begin{itemize}
\item {Grp. gram.:m.}
\end{itemize}
\begin{itemize}
\item {Utilização:Ant.}
\end{itemize}
O mesmo que \textunderscore guadamecim\textunderscore .
\section{Guadamecileiro}
\begin{itemize}
\item {Grp. gram.:m.}
\end{itemize}
\begin{itemize}
\item {Proveniência:(De \textunderscore guadamecil\textunderscore )}
\end{itemize}
Fabricante de guadamecins.
Aquelle que guardava os guadamecins da casa real.
\section{Guadamecim}
\begin{itemize}
\item {Grp. gram.:m.}
\end{itemize}
\begin{itemize}
\item {Proveniência:(Do ár. \textunderscore guadamesi\textunderscore )}
\end{itemize}
Antiga tapeçaria de coiro pintado.
\section{Guadimá}
\begin{itemize}
\item {Grp. gram.:m.}
\end{itemize}
\begin{itemize}
\item {Utilização:Bras}
\end{itemize}
\begin{itemize}
\item {Proveniência:(De \textunderscore gado-do-mato\textunderscore ? Cf. J. Ribeiro, \textunderscore Diccion. Gram.\textunderscore , 69)}
\end{itemize}
Boi bravo, toiro.
\section{Guadramilês}
\begin{itemize}
\item {Grp. gram.:m.}
\end{itemize}
\begin{itemize}
\item {Proveniência:(De \textunderscore Guadramil\textunderscore , \textunderscore n. p.\textunderscore )}
\end{itemize}
Dialecto trasmontano. Cf. G. Vianna, \textunderscore Classificação das Línguas\textunderscore , 10.
\section{Guaguaçu}
\begin{itemize}
\item {Grp. gram.:m.}
\end{itemize}
\begin{itemize}
\item {Utilização:Bras}
\end{itemize}
Árvore silvestre, de que se extrai, por incisão, um óleo apreciado.
\section{Guai!}
\begin{itemize}
\item {Grp. gram.:interj.}
\end{itemize}
\begin{itemize}
\item {Utilização:Ant.}
\end{itemize}
O mesmo que \textunderscore ai!\textunderscore :«\textunderscore guai de quem má fama cobra!\textunderscore »\textunderscore Eufrosina\textunderscore , 111«\textunderscore Guai de mi!\textunderscore »Usque, 16. v.^o
(Cp. lat. \textunderscore vae\textunderscore )
\section{Guaia}
\begin{itemize}
\item {Grp. gram.:f.}
\end{itemize}
\begin{itemize}
\item {Utilização:Ant.}
\end{itemize}
\begin{itemize}
\item {Proveniência:(De \textunderscore guaiar\textunderscore )}
\end{itemize}
Chôro; lamento. Cf. G. Vicente, \textunderscore Pranto de M. Parda\textunderscore .
\section{Guaiaca}
\begin{itemize}
\item {Grp. gram.:f.}
\end{itemize}
\begin{itemize}
\item {Utilização:Bras. do S}
\end{itemize}
Bôlsa de coiro, que se pendura á cinta, para levar dinheiro e outros objectos.
(Do quichua \textunderscore huayaca\textunderscore )
\section{Guaiacão}
\begin{itemize}
\item {Grp. gram.:m.}
\end{itemize}
O mesmo que \textunderscore guaiaco\textunderscore . Cf. B. Pereira, \textunderscore Prosódia\textunderscore , vb. \textunderscore guaiacum\textunderscore .
\section{Guaiaco}
\begin{itemize}
\item {Grp. gram.:m.}
\end{itemize}
\begin{itemize}
\item {Proveniência:(De \textunderscore guaican\textunderscore , palavra indígena da ilha de San-Domingos)}
\end{itemize}
Gênero de árvores das Antilhas, pau-santo.
\section{Guaiacol}
\begin{itemize}
\item {Grp. gram.:m.}
\end{itemize}
Corpo oxygenado, que se obtém pela destillação da resina de guaiaco.
\section{Guaiado}
\begin{itemize}
\item {Grp. gram.:adj.}
\end{itemize}
\begin{itemize}
\item {Utilização:Bras}
\end{itemize}
\begin{itemize}
\item {Utilização:Ant.}
\end{itemize}
\begin{itemize}
\item {Proveniência:(De \textunderscore guaiar\textunderscore )}
\end{itemize}
Plangente; lamentoso.
\section{Guaiambé}
\begin{itemize}
\item {Grp. gram.:m.}
\end{itemize}
\begin{itemize}
\item {Utilização:Bras}
\end{itemize}
Arbusto silvestre, de fôlhas medicinaes.
(Cp. \textunderscore guaimbé\textunderscore )
\section{Guaianás}
\begin{itemize}
\item {Grp. gram.:m. pl.}
\end{itemize}
\begin{itemize}
\item {Utilização:Bras}
\end{itemize}
O mesmo que \textunderscore guaianases\textunderscore .
\section{Guaianases}
\begin{itemize}
\item {Grp. gram.:m. pl.}
\end{itemize}
Nação de Índios do Brasil, que dominaram no território do actual estado de San-Paulo.
\section{Guaiar}
\begin{itemize}
\item {Grp. gram.:v. i.}
\end{itemize}
\begin{itemize}
\item {Utilização:Bras}
\end{itemize}
\begin{itemize}
\item {Utilização:ant.}
\end{itemize}
\begin{itemize}
\item {Utilização:Ant.}
\end{itemize}
\begin{itemize}
\item {Proveniência:(De \textunderscore guai\textunderscore )}
\end{itemize}
Soltar ais ou lamentações.
Queixar-se; lamentar-se. Cf. C. Neto, \textunderscore Saldunes\textunderscore , 112.
Cantar, em estilo de lamentação. Cf. Pant. de Aveiro, \textunderscore Itiner.\textunderscore , 262 v.^o, (2.^a ed.)
\section{Guaiara}
\begin{itemize}
\item {Grp. gram.:f.}
\end{itemize}
\begin{itemize}
\item {Utilização:Bras}
\end{itemize}
Cinturão envernizado, com pregaria, que serve para levar dinheiro, armas, tabaco, etc.
\section{Guaicanans}
\begin{itemize}
\item {Grp. gram.:m. pl.}
\end{itemize}
Uma das tribus aborígenes de San-Paulo, talvez o mesmo que \textunderscore guaianás\textunderscore .
\section{Guaicuru}
\begin{itemize}
\item {Grp. gram.:m.}
\end{itemize}
\begin{itemize}
\item {Grp. gram.:Pl.}
\end{itemize}
Uma das línguas indígenas de Paraguai.
Nação de Índios, que dominaram nas margens do Paraguai.
\section{Guaicuru}
\begin{itemize}
\item {Grp. gram.:m.}
\end{itemize}
\begin{itemize}
\item {Utilização:Bras}
\end{itemize}
Planta medicinal, o mesmo que \textunderscore baicuru\textunderscore .
\section{Guaimbé}
\begin{itemize}
\item {fónica:a-im}
\end{itemize}
\begin{itemize}
\item {Grp. gram.:m.}
\end{itemize}
O mesmo que \textunderscore imbé\textunderscore .
\section{Guaipá}
\begin{itemize}
\item {Grp. gram.:m.}
\end{itemize}
Árvore brasileira, de espinhos venenosos.
\section{Guajará}
\begin{itemize}
\item {Grp. gram.:m.}
\end{itemize}
Nome brasileiro de uma planta combretácea.
\section{Guajaraba}
\begin{itemize}
\item {Grp. gram.:m.}
\end{itemize}
Espécie de palmeira do México.
O mesmo que \textunderscore guajará\textunderscore ?--As duas variantes teriam analogia em \textunderscore piaçá\textunderscore  e \textunderscore piaçaba\textunderscore .
\section{Guajarutas}
\begin{itemize}
\item {Grp. gram.:m. pl.}
\end{itemize}
Tríbo de Índios do Brasil, em Mato-Grôsso.
\section{Guajeru}
\begin{itemize}
\item {Grp. gram.:m.}
\end{itemize}
\begin{itemize}
\item {Utilização:Bras}
\end{itemize}
\begin{itemize}
\item {Proveniência:(T. tupi)}
\end{itemize}
Planta rosácea do litoral da América do Sul, (\textunderscore chrysobalanus icaco\textunderscore ).
\section{Guajojaras}
\begin{itemize}
\item {Grp. gram.:m. pl.}
\end{itemize}
\begin{itemize}
\item {Utilização:Bras}
\end{itemize}
Uma das tríbos aborígenes do Maranhão.
\section{Guajuru}
\begin{itemize}
\item {Grp. gram.:m.}
\end{itemize}
\begin{itemize}
\item {Utilização:Bras}
\end{itemize}
O mesmo que \textunderscore guajeru\textunderscore .
\section{Gualde}
\begin{itemize}
\item {Grp. gram.:adj.}
\end{itemize}
O mesmo que \textunderscore jalne\textunderscore . Cf. R. Lobo, \textunderscore Côrte na Aldeia\textunderscore , II, 53.
\section{Gualdipério}
\begin{itemize}
\item {Grp. gram.:m.}
\end{itemize}
\begin{itemize}
\item {Utilização:Burl.}
\end{itemize}
\begin{itemize}
\item {Proveniência:(Do rad. de \textunderscore gualdir\textunderscore )}
\end{itemize}
Traição de namorado.
\section{Gualdir}
\begin{itemize}
\item {Grp. gram.:v. t.}
\end{itemize}
\begin{itemize}
\item {Utilização:Fam.}
\end{itemize}
Comer; gastar; dissipar.
(Talvez do vasconço)
\section{Gualdo}
\begin{itemize}
\item {Grp. gram.:adj.}
\end{itemize}
O mesmo que \textunderscore jalne\textunderscore .
\section{Gualdra}
\begin{itemize}
\item {Grp. gram.:f.}
\end{itemize}
Espécie de argola, por onde se puxam e abrem gavetas.
(Cp. \textunderscore aldrava\textunderscore )
\section{Gualdrapa}
\begin{itemize}
\item {Grp. gram.:f.}
\end{itemize}
\begin{itemize}
\item {Utilização:Ant.}
\end{itemize}
Xairel, espécie de manta, que se estende debaixo da sella, pendendo aos lados.
Grandes abas de um casacão.
(Cast. \textunderscore gualdrapa\textunderscore )
\section{Gualdripar}
\begin{itemize}
\item {Grp. gram.:v. t.}
\end{itemize}
\begin{itemize}
\item {Utilização:Fam.}
\end{itemize}
\begin{itemize}
\item {Proveniência:(Do rad. de \textunderscore gualdir\textunderscore )}
\end{itemize}
O mesmo que \textunderscore furtar\textunderscore .
\section{Gualdrope}
\begin{itemize}
\item {Grp. gram.:m.}
\end{itemize}
(V.galdrope)
\section{Gualtaria}
\begin{itemize}
\item {Grp. gram.:f.}
\end{itemize}
\begin{itemize}
\item {Utilização:Des.}
\end{itemize}
\begin{itemize}
\item {Proveniência:(Do rad. de \textunderscore gualteira\textunderscore )}
\end{itemize}
Vida ou modos de valentão.
\section{Gualteira}
\begin{itemize}
\item {Grp. gram.:f.}
\end{itemize}
\begin{itemize}
\item {Utilização:Des.}
\end{itemize}
Carapuça de pastor.
\section{Gualtéria}
\begin{itemize}
\item {Grp. gram.:f.}
\end{itemize}
Planta ericácea, (\textunderscore gualtheria procumbens\textunderscore , Lin.).
\section{Gualterina}
\begin{itemize}
\item {Grp. gram.:f.}
\end{itemize}
Substância, extraida da gualtéria.
\section{Gualtespa}
\begin{itemize}
\item {fónica:tês}
\end{itemize}
\begin{itemize}
\item {Grp. gram.:f.}
\end{itemize}
\begin{itemize}
\item {Utilização:Ant.}
\end{itemize}
Espécie de capacete.
(Cp. \textunderscore gualteira\textunderscore )
\section{Gualthéria}
\begin{itemize}
\item {Grp. gram.:f.}
\end{itemize}
Planta ericácea, (\textunderscore gualtheria procumbens\textunderscore , Lin.).
\section{Gualtherina}
\begin{itemize}
\item {Grp. gram.:f.}
\end{itemize}
Substância, extrahida da gualthéria.
\section{Guamajacu}
\begin{itemize}
\item {Grp. gram.:m.}
\end{itemize}
Nome brasileiro de um peixe esclerodermo.
\section{Guambu}
\begin{itemize}
\item {Grp. gram.:m.}
\end{itemize}
\begin{itemize}
\item {Utilização:Bras}
\end{itemize}
O mesmo que \textunderscore picão\textunderscore ^1.
\section{Guambuco}
\begin{itemize}
\item {Grp. gram.:m.}
\end{itemize}
Arvore angolense, de fibras têxteis.
\section{Guamirim}
\begin{itemize}
\item {Grp. gram.:m.}
\end{itemize}
\begin{itemize}
\item {Utilização:Bras}
\end{itemize}
Gênero de árvores silvestres, de que se conhecem três espécies.
\section{Guampa}
\begin{itemize}
\item {Grp. gram.:f.}
\end{itemize}
\begin{itemize}
\item {Utilização:Bras}
\end{itemize}
O mesmo que \textunderscore chifre\textunderscore .
Copo ou vaso, feito de chifre.
Chifre, em que se transporta ou se guarda água.
\section{Guampaço}
\begin{itemize}
\item {Grp. gram.:m.}
\end{itemize}
\begin{itemize}
\item {Utilização:Bras}
\end{itemize}
\begin{itemize}
\item {Proveniência:(De \textunderscore guampa\textunderscore )}
\end{itemize}
Marrada, cornada.
\section{Guanacás}
\begin{itemize}
\item {Grp. gram.:m. pl.}
\end{itemize}
\begin{itemize}
\item {Utilização:Bras}
\end{itemize}
Uma das tríbos aborígenes do Ceará.
\section{Guanaco}
\begin{itemize}
\item {Grp. gram.:m.}
\end{itemize}
\begin{itemize}
\item {Proveniência:(Do peruv. \textunderscore huanaco\textunderscore )}
\end{itemize}
Mammífero camelídeo das florestas.
\section{Guanandi}
\begin{itemize}
\item {Grp. gram.:m.}
\end{itemize}
Gênero de árvores brasileiras.
\section{Guanandirana}
\begin{itemize}
\item {Grp. gram.:f.}
\end{itemize}
\begin{itemize}
\item {Utilização:Bras}
\end{itemize}
Arvore silvestre.
\section{Guanás}
\begin{itemize}
\item {Grp. gram.:m. pl.}
\end{itemize}
Uma das tríbos aborígenes de Mato-Grôsso.
\section{Guanases}
\begin{itemize}
\item {Grp. gram.:m. pl.}
\end{itemize}
\begin{itemize}
\item {Utilização:Bras}
\end{itemize}
Uma das tríbos aborígenes de Mato-Grôsso.
\section{Guanchos}
\begin{itemize}
\item {Grp. gram.:m. pl.}
\end{itemize}
Antigos habitantes de Tenerife.
\section{Guandeiro}
\begin{itemize}
\item {Grp. gram.:m.}
\end{itemize}
\begin{itemize}
\item {Utilização:Bras}
\end{itemize}
\begin{itemize}
\item {Proveniência:(De \textunderscore guando\textunderscore )}
\end{itemize}
Planta leguminosa da América, introduzida talvez da África.
\section{Guando}
\begin{itemize}
\item {Grp. gram.:m.}
\end{itemize}
\begin{itemize}
\item {Utilização:Bras}
\end{itemize}
\begin{itemize}
\item {Proveniência:(T. afr.?)}
\end{itemize}
Fruto do guandeiro.
\section{Guandu}
\begin{itemize}
\item {Grp. gram.:m.}
\end{itemize}
\begin{itemize}
\item {Utilização:Bras}
\end{itemize}
O mesmo que \textunderscore guando\textunderscore .
\section{Guaneira}
\begin{itemize}
\item {Grp. gram.:f.}
\end{itemize}
\begin{itemize}
\item {Utilização:Bras}
\end{itemize}
Depósito de guano.
\section{Guanevanas}
\begin{itemize}
\item {Grp. gram.:m. pl.}
\end{itemize}
\begin{itemize}
\item {Utilização:Bras}
\end{itemize}
Tríbo, que habitou nos sertões do Pará.
\section{Guanguau}
\begin{itemize}
\item {Grp. gram.:m.}
\end{itemize}
\begin{itemize}
\item {Utilização:Ant.}
\end{itemize}
Imposto, que pagavam as casas de jôgo, em alguns pontos da Índia portuguesa.
\section{Guânico}
\begin{itemize}
\item {Grp. gram.:adj.}
\end{itemize}
\begin{itemize}
\item {Proveniência:(De \textunderscore guano\textunderscore )}
\end{itemize}
Diz-se de um ácido, derivado da guanina.
\section{Guanina}
\begin{itemize}
\item {Grp. gram.:f.}
\end{itemize}
Substância azotada, descoberta no guano.
\section{Guaninás}
\begin{itemize}
\item {Grp. gram.:m. pl.}
\end{itemize}
\begin{itemize}
\item {Utilização:Bras}
\end{itemize}
Uma das tríbos aborígenes de Mato-Grôsso.
O mesmo que \textunderscore guanás\textunderscore ?
\section{Guano}
\begin{itemize}
\item {Grp. gram.:m.}
\end{itemize}
\begin{itemize}
\item {Proveniência:(Do peruv. \textunderscore huano\textunderscore )}
\end{itemize}
Acumulação de excrementos de aves aquáticas, que se encontra nas costas do Peru, e que se emprega no adubo das terras.
Adubo para terras, preparado artificialmente com substâncias orgânicas.
\section{Guante}
\begin{itemize}
\item {Grp. gram.:m.}
\end{itemize}
\begin{itemize}
\item {Utilização:Ant.}
\end{itemize}
Luva de ferro da armadura antiga.
(Cast. \textunderscore guante\textunderscore )
\section{Guapamente}
\begin{itemize}
\item {Grp. gram.:adv.}
\end{itemize}
De modo guapo.
\section{Guaparaíba}
\begin{itemize}
\item {Grp. gram.:f.}
\end{itemize}
Espécie de mangue.
\section{Guaparambo}
\begin{itemize}
\item {Grp. gram.:m.}
\end{itemize}
\begin{itemize}
\item {Utilização:Bras}
\end{itemize}
Espécie de mangue bravo.
\section{Guaparonga}
\begin{itemize}
\item {Grp. gram.:f.}
\end{itemize}
Nome brasileiro de uma planta myrtácea, (\textunderscore marliera tomentosa\textunderscore , Camb.).
\section{Guapeba}
\begin{itemize}
\item {Grp. gram.:f.}
\end{itemize}
\begin{itemize}
\item {Utilização:Bras}
\end{itemize}
Fruta da guapebeira; a guapebeira.
\section{Guapebeira}
\begin{itemize}
\item {Grp. gram.:f.}
\end{itemize}
\begin{itemize}
\item {Utilização:Bras}
\end{itemize}
\begin{itemize}
\item {Proveniência:(De \textunderscore guapeba\textunderscore )}
\end{itemize}
Planta cucurbitácea.
\section{Guaperva}
\begin{itemize}
\item {Grp. gram.:f.}
\end{itemize}
Espécie de xarroco.
\section{Guapetão}
\begin{itemize}
\item {Grp. gram.:m.  e  adj.}
\end{itemize}
\begin{itemize}
\item {Utilização:Bras. do S}
\end{itemize}
Mui guapo.
\section{Guapeva}
\begin{itemize}
\item {Grp. gram.:f.}
\end{itemize}
\begin{itemize}
\item {Utilização:Bras}
\end{itemize}
O mesmo que \textunderscore guapeba\textunderscore .
\section{Guapiara}
\begin{itemize}
\item {Grp. gram.:f.}
\end{itemize}
\begin{itemize}
\item {Utilização:Bras}
\end{itemize}
O mesmo que \textunderscore gupiara\textunderscore .
\section{Guapice}
\begin{itemize}
\item {Grp. gram.:f.}
\end{itemize}
Qualidade daquelle que é guapo.
\section{Guapicobaíba}
\begin{itemize}
\item {Grp. gram.:f.}
\end{itemize}
Planta leguminosa do Brasil.
\section{Guapindaias}
\begin{itemize}
\item {Grp. gram.:m. pl.}
\end{itemize}
\begin{itemize}
\item {Utilização:Bras}
\end{itemize}
Uma das tríbos aborígenes de Mato-Grôsso.
\section{Guapironga}
\begin{itemize}
\item {Grp. gram.:f.}
\end{itemize}
(V.guaparonga)
\section{Guapo}
\begin{itemize}
\item {Grp. gram.:adj.}
\end{itemize}
\begin{itemize}
\item {Utilização:Pop.}
\end{itemize}
\begin{itemize}
\item {Proveniência:(Do lat. hyp. \textunderscore vappus\textunderscore )}
\end{itemize}
Corajoso; valente.
Bello; airoso; elegante.
\section{Guaporanga}
\begin{itemize}
\item {Grp. gram.:f.}
\end{itemize}
O mesmo que \textunderscore guaparonga\textunderscore ?
\section{Guapuí}
\begin{itemize}
\item {Grp. gram.:m.}
\end{itemize}
Planta bignoniácea do Brasil.
\section{Guapurunga}
\begin{itemize}
\item {Grp. gram.:f.}
\end{itemize}
\begin{itemize}
\item {Utilização:Bras}
\end{itemize}
O mesmo que \textunderscore guaparonga\textunderscore ?
\section{Guapurungueira}
\begin{itemize}
\item {Grp. gram.:f.}
\end{itemize}
\begin{itemize}
\item {Utilização:Bras}
\end{itemize}
O mesmo que \textunderscore guaparonga\textunderscore ?
\section{Guaputini}
\begin{itemize}
\item {Grp. gram.:m.}
\end{itemize}
Árvore brasileira.
\section{Guaquica}
\begin{itemize}
\item {Grp. gram.:f.}
\end{itemize}
\begin{itemize}
\item {Utilização:Bras}
\end{itemize}
Planta myrtácea da América.
(Talvez do tupi)
\section{Guará}
\begin{itemize}
\item {Grp. gram.:m.}
\end{itemize}
\begin{itemize}
\item {Utilização:Bras}
\end{itemize}
Mammífero, do gênero cão.
(Corr. de \textunderscore aguará\textunderscore , t. dos aborígenes do sul do Brasil)
\section{Guará}
\begin{itemize}
\item {Grp. gram.:m.}
\end{itemize}
\begin{itemize}
\item {Utilização:Bras}
\end{itemize}
Ave pernalta da América, (\textunderscore ibis rubra\textunderscore ).
(Do tupi)
\section{Guarabirola}
\begin{itemize}
\item {Grp. gram.:f.}
\end{itemize}
\begin{itemize}
\item {Utilização:Bras}
\end{itemize}
Gênero de plantas myrtáceas medicinaes.
\section{Guarabu}
\begin{itemize}
\item {Grp. gram.:f.}
\end{itemize}
Árvore leguminosa do Brasil.
\section{Guaraçaí}
\begin{itemize}
\item {Grp. gram.:f.}
\end{itemize}
Árvore leguminosa do Brasil.
\section{Guaracão}
\begin{itemize}
\item {Grp. gram.:m.}
\end{itemize}
\begin{itemize}
\item {Utilização:Bras}
\end{itemize}
Espécie de cão bravío.
(Cp. \textunderscore guará\textunderscore ^1)
\section{Guaracica}
\begin{itemize}
\item {Grp. gram.:f.}
\end{itemize}
\begin{itemize}
\item {Utilização:Bras}
\end{itemize}
Árvore silvestre, de que se fazem ripas.
\section{Guaraitá}
\begin{itemize}
\item {Grp. gram.:m.}
\end{itemize}
\begin{itemize}
\item {Utilização:Bras}
\end{itemize}
Árvore sapotácea dos sertões.
\section{Guaraiúba}
\begin{itemize}
\item {Grp. gram.:f.}
\end{itemize}
Nome de um peixe do Brasil.
\section{Guarajuba}
\begin{itemize}
\item {Grp. gram.:f.}
\end{itemize}
Árvore combretácea do Brasil.
\section{Guarajus}
\begin{itemize}
\item {Grp. gram.:m. pl.}
\end{itemize}
Tríbo de Indios brasileiros, que dominaram perto do Guaporé.
\section{Guaral}
\begin{itemize}
\item {Grp. gram.:m.}
\end{itemize}
\begin{itemize}
\item {Proveniência:(T. ár.)}
\end{itemize}
Espécie de aranha, que se encontra nos desertos da Lýbia, e que, segundo se diz, é comestível para os Árabes.
\section{Guaraná}
\begin{itemize}
\item {Grp. gram.:f.}
\end{itemize}
\begin{itemize}
\item {Proveniência:(Do rad. de \textunderscore guarani\textunderscore )}
\end{itemize}
Planta sapindácea do Brasil.
Resina da mesma planta.
Substância alimentícia, preparada pelos Guaranis do Uruguai e do Pará.
\section{Guarane}
\begin{itemize}
\item {Grp. gram.:m.}
\end{itemize}
O mesmo que \textunderscore garanvaz\textunderscore . Cf. Garrett, \textunderscore Romanceiro\textunderscore , II, 132.
\section{Guaranhem}
\begin{itemize}
\item {Grp. gram.:m.}
\end{itemize}
O mesmo que \textunderscore buranhém\textunderscore .
\section{Guarani}
\begin{itemize}
\item {Grp. gram.:m.}
\end{itemize}
\begin{itemize}
\item {Grp. gram.:Pl.}
\end{itemize}
Língua dos Guaranis.
Indivíduo da raça dos guaranis.
Uma das mais notáveis e numerosas nações indígenas da América do Sul, que dominou principalmente entre o Paraná, o Paraguai e o Uruguai.
\section{Guaranina}
\begin{itemize}
\item {Grp. gram.:f.}
\end{itemize}
Alcaloide, extrahido da guaraná.
\section{Guarapa}
\begin{itemize}
\item {Grp. gram.:f.}
\end{itemize}
Sumo da cana de açúcar.
\section{Guaraparés}
\begin{itemize}
\item {Grp. gram.:m. pl.}
\end{itemize}
Uma das tríbos aborígenes de Mato-Grôsso.
\section{Guarapari}
\begin{itemize}
\item {Grp. gram.:m.}
\end{itemize}
Árvore silvestre, de madeira arroxeada.
\section{Guaraparim}
\begin{itemize}
\item {Grp. gram.:m.}
\end{itemize}
\begin{itemize}
\item {Utilização:Bras}
\end{itemize}
Árvore silvestre, de madeira arroxeada.
\section{Guarapé}
\begin{itemize}
\item {Grp. gram.:m.}
\end{itemize}
Planta saxifragácea do Brasil.
\section{Guarapiapunha}
\begin{itemize}
\item {Grp. gram.:f.}
\end{itemize}
\begin{itemize}
\item {Utilização:Bras}
\end{itemize}
O mesmo que \textunderscore grapiapunha\textunderscore .
\section{Guarapicica}
\begin{itemize}
\item {Grp. gram.:f.}
\end{itemize}
\begin{itemize}
\item {Utilização:Bras}
\end{itemize}
Árvore silvestre, cuja madeira, com veios escuros, é usada em marcenaria.
\section{Guarapiranga}
\begin{itemize}
\item {Grp. gram.:f.}
\end{itemize}
Árvore brasileira.
\section{Guarapoca}
\begin{itemize}
\item {Grp. gram.:f.}
\end{itemize}
\begin{itemize}
\item {Utilização:Bras}
\end{itemize}
Árvore silvestre.
\section{Guaraquim}
\begin{itemize}
\item {Grp. gram.:m.}
\end{itemize}
O mesmo que \textunderscore erva-moira\textunderscore .
\section{Guararema}
\begin{itemize}
\item {Grp. gram.:f.}
\end{itemize}
O mesmo que \textunderscore ibirarema\textunderscore .
\section{Guaratan}
\begin{itemize}
\item {Grp. gram.:f.}
\end{itemize}
Formosa árvore brasileira, alta e de tronco liso.
\section{Guaratimbo}
\begin{itemize}
\item {Grp. gram.:m.}
\end{itemize}
\begin{itemize}
\item {Utilização:Bras}
\end{itemize}
Árvore, que cresce á beira dos rios e tem raíz venenosa.
\section{Guaraúna}
\begin{itemize}
\item {Grp. gram.:f.}
\end{itemize}
\begin{itemize}
\item {Utilização:Bras}
\end{itemize}
Árvore leguminosa, o mesmo que \textunderscore braúna\textunderscore .
\section{Guaraxim}
\begin{itemize}
\item {Grp. gram.:m.}
\end{itemize}
\begin{itemize}
\item {Utilização:Bras}
\end{itemize}
Espécie de pequeno cão bravío.
(Cp. \textunderscore guará\textunderscore ^1)
\section{Guaraz}
\begin{itemize}
\item {Grp. gram.:m.}
\end{itemize}
Pássaro brasileiro, de que se diz nascer branco e tornar-se depois vermelho.
\section{Guarda}
\begin{itemize}
\item {Grp. gram.:f.}
\end{itemize}
\begin{itemize}
\item {Grp. gram.:M.}
\end{itemize}
Acto ou effeito de guardar.
Cuidado, vigilância, a respeito de alguma coisa ou pessôa.
Amparo.
Benevolência.
Parte da arma branca, que resguarda a mão.
Posição especial, no jôgo da esgrima, para aparar o golpe do adversário.
Fôlha branca ou de côr, que resguarda o princípio e o fim de um livro.
Vara, que o podador conserva na videira.
Serviço de guardar ou vigiar, desempenhado por militares: \textunderscore estar de guarda\textunderscore .
Militar ou militares, que desempenham êsse serviço: \textunderscore a guarda do quartel\textunderscore .
Mulher encarregada de guardar alguma coisa.
Peitoril, anteparo: \textunderscore a guarda da varanda\textunderscore .
\textunderscore Guardas do Norte\textunderscore , a Ursa-Maior e a Ursa-Menor. Cf. Pant. de Aveiro, \textunderscore Itiner.\textunderscore , 204, (2.^a ed.).
Homem, encarregado de vigiar ou guardar alguma coisa: \textunderscore os guardas campestres\textunderscore .
\section{Guarda-arnês}
\begin{itemize}
\item {Grp. gram.:m.}
\end{itemize}
Lugar, onde se guardam as guarnições de cavallaria.
\section{Guarda-barreira}
\begin{itemize}
\item {Grp. gram.:m.}
\end{itemize}
Empregado aduaneiro, que fiscaliza a entrada dos gêneros de consumo, ás portas da cidade.
Empregado de linhas férreas, que vigía as barreiras nas passagens de nível.
\section{Guarda-braço}
\begin{itemize}
\item {Grp. gram.:m.}
\end{itemize}
Parte da antiga armadura, correspondente aos braços.
\section{Guarda-cadeiras}
\begin{itemize}
\item {Grp. gram.:m.}
\end{itemize}
Filete horizontal de madeira, que, nas paredes de algumas salas, evita que as costas das cadeiras prejudiquem o papel, o estuque, etc., que reveste as paredes.
\section{Guarda-calhas}
\begin{itemize}
\item {Grp. gram.:m.}
\end{itemize}
\begin{itemize}
\item {Utilização:Bras}
\end{itemize}
Peça metállica, em fórma de gradeamento, á frente das máquinas das estradas de ferro, para arredar dos carris qualquer objecto.
\section{Guarda-cama}
\begin{itemize}
\item {Grp. gram.:m.}
\end{itemize}
\begin{itemize}
\item {Utilização:Prov.}
\end{itemize}
Rodapé ou cortina, geralmente de chita e bordada ou guarnecida de renda, que se suspende em volta dos leitos, para encobrir o espaço que fica debaixo da cama e para ornato desta.
\section{Guarda-cartucho}
\begin{itemize}
\item {Grp. gram.:m.}
\end{itemize}
Caixa cylíndrica de sola, para dois ou três cartuchos de peça de artilharia, e que é posta a tiracollo pela praça nomeada para municiar a bôca de fogo.
\section{Guarda-cascos}
\begin{itemize}
\item {Grp. gram.:m.}
\end{itemize}
Prolongamento do bôrdo exterior da ferradura, no lugar da pinça.
\section{Guarda-chapim}
\begin{itemize}
\item {Grp. gram.:m.}
\end{itemize}
Pequeno muro ou fiada de cantaria, sôbre que se assenta uma grade.
Guarnecimento, geralmente de cantaria, que acompanha lateralmente os degraus de uma escada.
\section{Guarda-chuva}
\begin{itemize}
\item {Grp. gram.:m.}
\end{itemize}
Armação de varetas, móveis, coberta de pano, para resguardar da chuva ou do sol as pessôas; chapéu de chuva.
\section{Guarda-comidas}
\begin{itemize}
\item {Grp. gram.:m.}
\end{itemize}
Espécie de dispensa portátil, feita especialmente de arame, para guardar iguarias ou substâncias alimentícias.
\section{Guardacós}
\begin{itemize}
\item {Grp. gram.:m.}
\end{itemize}
\begin{itemize}
\item {Utilização:Ant.}
\end{itemize}
Casaco, que, cobrindo o corpo, o apertava. Cf. Rebello, \textunderscore Contos e Lendas\textunderscore , 139 e 150.
(B. lat. \textunderscore gardacosium\textunderscore )
\section{Guarda-costas}
\begin{itemize}
\item {Grp. gram.:m.}
\end{itemize}
\begin{itemize}
\item {Utilização:Fig.}
\end{itemize}
Navio, que, percorrendo a costa marítima, procura evitar o contrabando.
Pessôa, que acompanha outra, para a defender de alguma aggressão.
\section{Guardadeira}
\begin{itemize}
\item {Grp. gram.:f.  e  adj.}
\end{itemize}
\begin{itemize}
\item {Proveniência:(De \textunderscore guardar\textunderscore )}
\end{itemize}
Mulher que guarda; que observa certos preceitos.
\section{Guardadeiro}
\begin{itemize}
\item {Grp. gram.:m.}
\end{itemize}
\begin{itemize}
\item {Utilização:Ant.}
\end{itemize}
\begin{itemize}
\item {Proveniência:(De \textunderscore guardar\textunderscore )}
\end{itemize}
Casa ou pôsto da guarda.
\section{Guardador}
\begin{itemize}
\item {Grp. gram.:m.  e  adj.}
\end{itemize}
\begin{itemize}
\item {Utilização:Fig.}
\end{itemize}
O que guarda.
O que observa certos preceitos: \textunderscore guardador do jejum da Quaresma\textunderscore .
Avarento.
\section{Guarda-faceira}
\begin{itemize}
\item {Grp. gram.:m.}
\end{itemize}
Correia lateral da cabeçada dos cavallos. Cf. M. C. Andrade, \textunderscore Arte de Cavall.\textunderscore 
\section{Guarda-fato}
\begin{itemize}
\item {Grp. gram.:m.}
\end{itemize}
Espécie de armário móvel, em que se guarda fato.
\section{Guarda-fechos}
\begin{itemize}
\item {Grp. gram.:m.}
\end{itemize}
Peça de coiro, com que se cobrem os fechos da espingarda, para evitar que se enferrugem.
\section{Guarda-fio}
\begin{itemize}
\item {Grp. gram.:m.}
\end{itemize}
Homem, encarregado de vigiar as linhas telegráphicas.
\section{Guarda-fogo}
\begin{itemize}
\item {Grp. gram.:m.}
\end{itemize}
Peça metállica, que se põe deante da chaminé, para evitar incêndios.
Parede que, entre prédios contíguos, se eleva até ao pau de fileira, para evitar a communicação de incêndios.
\section{Guarda-freio}
\begin{itemize}
\item {Grp. gram.:m.}
\end{itemize}
Empregado de linhas férreas, que vigia os freios das carruagens.
Empregado, que guia os carros eléctricos.
\section{Guarda-infante}
\begin{itemize}
\item {Grp. gram.:m.}
\end{itemize}
\begin{itemize}
\item {Utilização:Des.}
\end{itemize}
Merinaque; crinolina.
Arco de ferro, coberto de fita, para tufar as saias.
Anquinhas.
Donaire. Cf. F. Manuel, \textunderscore Carta de Guia\textunderscore , 101; Corvo, \textunderscore Anno na Côrte\textunderscore , I, 72.
\section{Guarda-jóias}
\begin{itemize}
\item {Grp. gram.:m.}
\end{itemize}
Antigo empregado da casa real, encarregado de guardar as jóias.
Vaso, cofre, qualquer utensílio, em que se guardam jóias e outros adereços.
\section{Guarda-lama}
\begin{itemize}
\item {Grp. gram.:m.}
\end{itemize}
Resguardo, que, aos lados de uma carruagem, evita que a lama salpique as portinholas.
Anteparo de madeira, coiro ou ferro, que, num carro e adeante do cocheiro, impede que os cavallos atirem lama a quem vai sentado na almofada.
Extremidade inferior e massiça da baínha da espada.
Peça de fazenda, forte, que fórra a parte inferior do vestido das mulheres.
Esporim.
\section{Guardalate}
\begin{itemize}
\item {Grp. gram.:m.}
\end{itemize}
\begin{itemize}
\item {Utilização:Ant.}
\end{itemize}
Espécie de tecido grosseiro. Cf. S. Viterbo, \textunderscore Industr. Têxteis\textunderscore , 46.
\section{Guarda-leme}
\begin{itemize}
\item {Grp. gram.:m.}
\end{itemize}
Peça de artilharia, junto ao leme da embarcação.
\section{Guarda-linha}
\begin{itemize}
\item {Grp. gram.:m.  e  f.}
\end{itemize}
Pessôa, que vigia as linhas férreas.
\section{Guarda-livros}
\begin{itemize}
\item {Grp. gram.:m.}
\end{itemize}
Empregado commercial, que regista o movimento do commércio em uma ou mais casas.
\section{Guarda-loiça}
\begin{itemize}
\item {Grp. gram.:m.}
\end{itemize}
Armário, para guardar loiça.
Prateleira; cantoneira.
\section{Guarda-lume}
\begin{itemize}
\item {Grp. gram.:m.}
\end{itemize}
O mesmo que \textunderscore guarda-fogo\textunderscore .
\section{Guarda-maior}
\begin{itemize}
\item {Grp. gram.:m.}
\end{itemize}
\begin{itemize}
\item {Utilização:Des.}
\end{itemize}
(V.guarda-mór)
\section{Guarda-mancebos}
\begin{itemize}
\item {Grp. gram.:m. pl.}
\end{itemize}
\begin{itemize}
\item {Utilização:Náut.}
\end{itemize}
Cabos, que servem de corrimão aos marinheiros, no extremo da prôa.
\section{Guarda-mão}
\begin{itemize}
\item {Grp. gram.:m.}
\end{itemize}
Arco, que resguarda a mão, entre os copos e a maçan da espada.
\section{Guarda-marinha}
\begin{itemize}
\item {Grp. gram.:m.}
\end{itemize}
Posto da armada, immediatamente inferior ao do segundo tenente e superior ao de aspirante.
\section{Guarda-mato}
\begin{itemize}
\item {Grp. gram.:m.}
\end{itemize}
Peça da espingarda, em fórma de arco, para resguardar o gatilho.
Vallado, que limita matagaes ou terras de pastagens.
Pelle, com que os pastores e alguns trabalhadores do campo, resguardam as pernas.
Valla exterior ás salinas, para receber as águas de terrenos adjacentes e impedir que ellas entrem na marinha.
\section{Guarda-menor}
\begin{itemize}
\item {Grp. gram.:m.}
\end{itemize}
Empregado inferior dos tribunaes das Relações.
\section{Guardamento}
\begin{itemize}
\item {Grp. gram.:m.}
\end{itemize}
Acto de guardar; guarda.
\section{Guarda-mór}
\begin{itemize}
\item {Grp. gram.:m.}
\end{itemize}
Antigo official, que commandava vinte archeiros ou alabardeiros da casa real.
Empregado superior de algumas repartições públicas e tribunaes.
\section{Guardamoria}
\begin{itemize}
\item {Grp. gram.:f.}
\end{itemize}
\begin{itemize}
\item {Utilização:Bras}
\end{itemize}
Cargo de guarda-mór.
\section{Guarda-morrão}
\begin{itemize}
\item {Grp. gram.:m.}
\end{itemize}
Vaso cylíndrico de metal, para transporte de morrão, em serviço de batarias.
\section{Guardanapo}
\begin{itemize}
\item {Grp. gram.:m.}
\end{itemize}
\begin{itemize}
\item {Utilização:Ant.}
\end{itemize}
\begin{itemize}
\item {Proveniência:(De \textunderscore guardar\textunderscore  + fr. \textunderscore nappe\textunderscore , se não de \textunderscore guardar\textunderscore  +it. \textunderscore nappo\textunderscore , copo)}
\end{itemize}
Pano branco e quadrado, com que, á mesa, se limpa a bôca ou se resguarda o fato, para evitar nódoas.
Lenço de assoar.
\section{Guarda-nocturno}
\begin{itemize}
\item {Grp. gram.:m.}
\end{itemize}
Indivíduo que, por conta dos habitantes dos arruamentos, guarda de noite as entradas das habitações, rondando e vigiando.
\section{Guarda-patrão}
\begin{itemize}
\item {Grp. gram.:m.}
\end{itemize}
Encôsto, que, nas pequenas embarcações, separa do lugar do homem do leme o resto do barco.
\section{Guarda-pé}
\begin{itemize}
\item {Grp. gram.:m.}
\end{itemize}
\begin{itemize}
\item {Utilização:Ant.}
\end{itemize}
\begin{itemize}
\item {Proveniência:(De \textunderscore guardar\textunderscore  + \textunderscore pé\textunderscore )}
\end{itemize}
Brial, sáia que as mulheres usavam por baixo das roupas abertas. Cf. Camillo, \textunderscore Noites de Insómn.\textunderscore , III, 56.
\section{Guarda-pé-de-primavera}
\begin{itemize}
\item {Grp. gram.:m.}
\end{itemize}
Espécie de saia antiga de tecido fino:«\textunderscore tenho hũ capote de pinhoella verde e hũ guarda pee de primavera com renda preta á roda.\textunderscore »(De um testamento de 1642)
\section{Guarda-peito}
\begin{itemize}
\item {Grp. gram.:m.}
\end{itemize}
\begin{itemize}
\item {Utilização:Bras. dos sertões do N}
\end{itemize}
Pedaço de pelle, que se prende ao pescoço e á cintura, servindo de collete.
\section{Guarda-pisa}
\begin{itemize}
\item {Grp. gram.:m.}
\end{itemize}
Guarda-lama das sáias das mulheres.
\section{Guarda-pó}
\begin{itemize}
\item {Grp. gram.:m.}
\end{itemize}
Fôrro, que reveste o vigamento superior das casas.
Casaco leve e comprido, que se veste sôbre todo o fato, para o resguardar do pó, mormente em viagem.
\section{Guarda-porta}
\begin{itemize}
\item {Grp. gram.:m.}
\end{itemize}
\begin{itemize}
\item {Utilização:Des.}
\end{itemize}
O mesmo que \textunderscore reposteiro\textunderscore ^1:«\textunderscore ...de maneira que a via eu por hũa guarda-porta de esguelha...\textunderscore »\textunderscore Eufrosina\textunderscore , act. I, sc. 1.
\section{Guarda-portão}
\begin{itemize}
\item {Grp. gram.:m.}
\end{itemize}
O mesmo que \textunderscore porteiro\textunderscore .
\section{Guarda-prata}
\begin{itemize}
\item {Grp. gram.:m.  e  f.}
\end{itemize}
Móvel, em que se guardam baixellas ou outra loiça, e constituído por dois corpos, um dos quaes é sobreposto ao outro, mas separado por um espaço, que dá ao corpo inferior o aspecto de aparador.
\section{Guarda-quédas}
\begin{itemize}
\item {Grp. gram.:m.}
\end{itemize}
(V.pára-quédas)
\section{Guardar}
\begin{itemize}
\item {Grp. gram.:v. t.}
\end{itemize}
\begin{itemize}
\item {Grp. gram.:V. p.}
\end{itemize}
\begin{itemize}
\item {Proveniência:(Do ant. alt. al. \textunderscore warten\textunderscore )}
\end{itemize}
Vigiar para defesa ou protecção.
Defender.
Acautelar.
Têr cuidado em conservar seguro ou preso.
Conservar: \textunderscore guardar frutas\textunderscore .
Arrecadar: \textunderscore guardar dinheiro\textunderscore .
Observar: \textunderscore guardar preceitos\textunderscore .
Reservar; preservar.
Occultar, não revelar: \textunderscore guardar segrêdo\textunderscore .
Adiar.
Cumprir.
Têr devoção a.
Precaver-se.
Abster-se; evitar alguma coisa: \textunderscore guardar-se de tentações\textunderscore .
\section{Guarda-raios}
\begin{itemize}
\item {Grp. gram.:m.}
\end{itemize}
(V.pára-raios)
\section{Guarda-rios}
\begin{itemize}
\item {Grp. gram.:m.}
\end{itemize}
O mesmo que \textunderscore pica-peixe\textunderscore .
\section{Guarda-roupa}
\begin{itemize}
\item {Grp. gram.:m.  e  f.}
\end{itemize}
Pessôa, encarregada de guardar roupas e alfaias num theatro, communidade, casa particular, etc.
Guarda-fato.
Depósito de fatos e alfaias num theatro.
Estabelecimento, em que se alugam roupas, especialmente na occasião do Carnaval.
Nome de uma planta, da fam. das compostas, espécie de abrótamo.
\section{Guarda-sellos}
\begin{itemize}
\item {Grp. gram.:m.}
\end{itemize}
\begin{itemize}
\item {Utilização:Ant.}
\end{itemize}
O mesmo que \textunderscore chanceller\textunderscore .
\section{Guarda-sol}
\begin{itemize}
\item {Grp. gram.:m.}
\end{itemize}
O mesmo que \textunderscore guarda-chuva\textunderscore .
\section{Guardasoleiro}
\begin{itemize}
\item {fónica:so}
\end{itemize}
\begin{itemize}
\item {Grp. gram.:m.}
\end{itemize}
Fabricante de guarda-sóes.
\section{Guardassoleiro}
\begin{itemize}
\item {Grp. gram.:m.}
\end{itemize}
Fabricante de guarda-sóes.
\section{Guarda-tufo}
\begin{itemize}
\item {Grp. gram.:m.}
\end{itemize}
\begin{itemize}
\item {Proveniência:(De \textunderscore guarda\textunderscore  + \textunderscore tufo\textunderscore ^3)}
\end{itemize}
Muro, adeante da alcatruzada.
\section{Guarda-vassoiras}
\begin{itemize}
\item {Grp. gram.:m.}
\end{itemize}
Tira de madeira, ao fundo das paredes, em volta de um compartimento, para evitar que o varrer, os pés das cadeiras ou a lavagem do soalho prejudiquem o papel, o estuque, etc., das paredes.
Rodapé.
\section{Guarda-vento}
\begin{itemize}
\item {Grp. gram.:m.}
\end{itemize}
Reposteiro ou anteparo de madeira, collocado dentro das igrejas ou de outros edifícios, junto á porta principal, para os resguardar do vento e das vistas dos transeuntes.
\section{Guarda-vestidos}
\begin{itemize}
\item {Grp. gram.:m.}
\end{itemize}
Espécie de armário móvel, com cabides, para dependurar e guardar fato, especialmente de senhoras.
\section{Guarda-vinho}
\begin{itemize}
\item {Grp. gram.:m.}
\end{itemize}
Muro dos lagares em que se faz o vinho.
\section{Guarda-vista}
\begin{itemize}
\item {Grp. gram.:m.}
\end{itemize}
Peça, que se colloca deante dos olhos, para os proteger contra a intensidade da luz.
Bandeira de candeeiro ou castiçal.
Pantalha.
\section{Guarda-volante}
\begin{itemize}
\item {Grp. gram.:m.}
\end{itemize}
\begin{itemize}
\item {Grp. gram.:F.}
\end{itemize}
Peça, que resguarda o volante dos relógios.
Soldado ou soldados, que fazem guarda, sem estacionar, mas girando em várias direcções.
\section{Guarda-voz}
\begin{itemize}
\item {Grp. gram.:m.}
\end{itemize}
Cúpula de alguns púlpitos, destinada a fazer que a voz do prègador desça e se espalhe convenientemente pelo auditório.
\section{Guardeamento}
\begin{itemize}
\item {Grp. gram.:m.}
\end{itemize}
Acto ou effeito de guardear.
\section{Guardear}
\begin{itemize}
\item {Grp. gram.:v. t.}
\end{itemize}
Pôr guardas ou resguardos ao longo de: \textunderscore guardear uma varanda\textunderscore .
\section{Guárdia}
\begin{itemize}
\item {Grp. gram.:f.}
\end{itemize}
\begin{itemize}
\item {Utilização:Ant.}
\end{itemize}
O mesmo que \textunderscore guarda\textunderscore . Cf. Usque, 27 v.^o
\section{Guardiania}
\begin{itemize}
\item {Grp. gram.:f.}
\end{itemize}
Emprêgo de guardião.
\section{Guardião}
\begin{itemize}
\item {Grp. gram.:m.}
\end{itemize}
\begin{itemize}
\item {Utilização:Pop.}
\end{itemize}
Funccionário superior de alguns conventos.
Guarda-costas.
Pôsto inferior na armada, desempenhado por quem dirige as praças em trabalhos de marinheiro.
Planta cucurbitácea do Brasil.
(Cast. \textunderscore guardian\textunderscore )
\section{Guardinfante}
\begin{itemize}
\item {Grp. gram.:m.}
\end{itemize}
O mesmo que \textunderscore guarda-infante\textunderscore .
\section{Guardins}
\begin{itemize}
\item {Grp. gram.:m. pl.}
\end{itemize}
\begin{itemize}
\item {Utilização:Náut.}
\end{itemize}
Cabos, encapellados no penol da carangueja, para aguentarem, de bombordo a estibordo.
(Cast. \textunderscore guardin\textunderscore )
\section{Guardinvão}
\begin{itemize}
\item {Grp. gram.:m.}
\end{itemize}
Espécie de jôgo popular.
\section{Guardo}
\begin{itemize}
\item {Grp. gram.:m.}
\end{itemize}
\begin{itemize}
\item {Utilização:Des.}
\end{itemize}
O mesmo que \textunderscore guarda\textunderscore ; acto de guardar:«\textunderscore ...e lhe deu a guardo o castello de prôa.\textunderscore »Filinto, \textunderscore D. Man.\textunderscore , II, 47.
\section{Guardonho}
\begin{itemize}
\item {Grp. gram.:adj.}
\end{itemize}
\begin{itemize}
\item {Utilização:ant.}
\end{itemize}
\begin{itemize}
\item {Utilização:Pop.}
\end{itemize}
\begin{itemize}
\item {Proveniência:(De \textunderscore guardar\textunderscore )}
\end{itemize}
Que é muito económico, sovina.
\section{Guardoso}
\begin{itemize}
\item {Grp. gram.:adj.}
\end{itemize}
\begin{itemize}
\item {Utilização:Des.}
\end{itemize}
O mesmo que \textunderscore guardonho\textunderscore .
\section{Guaré}
\begin{itemize}
\item {Grp. gram.:m.}
\end{itemize}
Nome brasileiro de uma planta meliácea.
\section{Guarecedor}
\begin{itemize}
\item {Grp. gram.:adj.}
\end{itemize}
Que guarece.
\section{Guarecer}
\begin{itemize}
\item {Grp. gram.:v. t.}
\end{itemize}
\begin{itemize}
\item {Utilização:Ant.}
\end{itemize}
\begin{itemize}
\item {Grp. gram.:V. i.  e  p.}
\end{itemize}
\begin{itemize}
\item {Proveniência:(De \textunderscore guarir\textunderscore )}
\end{itemize}
Curar.
Salvar, livrar.
Curar-se.
\section{Guarente}
\begin{itemize}
\item {Grp. gram.:m.}
\end{itemize}
Fazenda, que sobra ao encurtarem-se por baixo capas ou capotes.
\section{Guarerova}
\begin{itemize}
\item {Grp. gram.:f.}
\end{itemize}
\begin{itemize}
\item {Utilização:Bras}
\end{itemize}
O mesmo que \textunderscore guariroba\textunderscore .
\section{Guarguaru}
\begin{itemize}
\item {Grp. gram.:m.}
\end{itemize}
Pequeno peixe do Brasil.
\section{Guari}
\begin{itemize}
\item {Grp. gram.:m.}
\end{itemize}
Palmeira americana.
Ave palmípede da África occidental.
\section{Guariba}
\begin{itemize}
\item {Grp. gram.:m.}
\end{itemize}
Macaco da América, (\textunderscore simia seniculus\textunderscore ).
Pequena ave, semelhante ao periquito.
\section{Guaribu}
\begin{itemize}
\item {Grp. gram.:m.}
\end{itemize}
Planta silvestre de Pernambuco, o mesmo que \textunderscore federal\textunderscore .
\section{Guaricanga}
\begin{itemize}
\item {Grp. gram.:f.}
\end{itemize}
\begin{itemize}
\item {Utilização:Bras}
\end{itemize}
Espécie de palmeira baixa.
\section{Guaricema}
\begin{itemize}
\item {Grp. gram.:m.  ou  f.}
\end{itemize}
Variedade de peixe brasileiro.
\section{Guarida}
\begin{itemize}
\item {Grp. gram.:f.}
\end{itemize}
\begin{itemize}
\item {Utilização:Fig.}
\end{itemize}
\begin{itemize}
\item {Proveniência:(Do rad. de \textunderscore guarir\textunderscore )}
\end{itemize}
Covil de feras.
Abrigo; refúgio.
Protecção.
Guarita.
\section{Guarida}
\begin{itemize}
\item {Grp. gram.:f.}
\end{itemize}
\begin{itemize}
\item {Utilização:T. de Melgaço}
\end{itemize}
Rêgo permanente, ou rêgo por onde vai sempre água.
\section{Guarinos}
\begin{itemize}
\item {Grp. gram.:m. pl.}
\end{itemize}
Tríbo de Índios pacíficos de Goiás, no Brasil.
\section{Guarir}
\begin{itemize}
\item {Grp. gram.:v. t.  e  i.}
\end{itemize}
\begin{itemize}
\item {Utilização:Des.}
\end{itemize}
\begin{itemize}
\item {Proveniência:(Do gót. \textunderscore wargen\textunderscore )}
\end{itemize}
O mesmo que \textunderscore guarecer\textunderscore .
\section{Guariroba}
\begin{itemize}
\item {Grp. gram.:f.}
\end{itemize}
\begin{itemize}
\item {Utilização:Bras}
\end{itemize}
Espécie de palmeira.
\section{Guarita}
\begin{itemize}
\item {Grp. gram.:f.}
\end{itemize}
\begin{itemize}
\item {Utilização:Des.}
\end{itemize}
Tôrre, nos ângulos dos antigos baluartes, para abrigo de sentinelas.
Casa portatil de madeira, para abrigo de sentinelas.
Baiúca, taberna. Cf. Tolentino, \textunderscore Bilhar\textunderscore .
(Cp. \textunderscore guarida\textunderscore ^1)
\section{Guariterés}
\begin{itemize}
\item {Grp. gram.:m. pl.}
\end{itemize}
\begin{itemize}
\item {Utilização:Bras}
\end{itemize}
Uma das tríbos aborígenes de Mato-Grôsso.
\section{Guariúba}
\begin{itemize}
\item {Grp. gram.:f.}
\end{itemize}
Árvore do norte do Brasil, que dá bôa madeira para construcções.
\section{Guarnecedor}
\begin{itemize}
\item {Grp. gram.:m.  e  adj.}
\end{itemize}
O que guarnece.
\section{Guarnecer}
\begin{itemize}
\item {Grp. gram.:v. t.}
\end{itemize}
\begin{itemize}
\item {Utilização:Fig.}
\end{itemize}
\begin{itemize}
\item {Proveniência:(De \textunderscore guarnir\textunderscore )}
\end{itemize}
Prover do necessário.
Fortalecer.
Pôr fôrças militares em: \textunderscore guarnecer uma praça\textunderscore .
Caiar (paredes), depois de rebocadas.
Adornar, enfeitar: \textunderscore guarnecer uma sala\textunderscore .
Pôr ornato na fímbria de; enfeitar nas bordas: \textunderscore guarnecer um vestido\textunderscore .
\section{Guarnecimento}
\begin{itemize}
\item {Grp. gram.:m.}
\end{itemize}
Acto ou effeito de guarnecer.
Guarnição.
\section{Guarnição}
\begin{itemize}
\item {Grp. gram.:f.}
\end{itemize}
\begin{itemize}
\item {Proveniência:(De \textunderscore guarnecer\textunderscore )}
\end{itemize}
Aquillo que guarnece.
Tropas, que defendem uma praça.
Equipagem de navio.
Punhos e copos da espada.
Enfeite.
Orla enfeitada.
Enfeite da fímbria ou das outras extremidades de um vestuário.
Girão.
Jaêzes.
Parte da ferradura, que resai do bôrdo do casco.
Peças de metal ou madeira, em que os impressores apertam as páginas.
\section{Guarnicioneiro}
\begin{itemize}
\item {Grp. gram.:m.}
\end{itemize}
Indivíduo, que tinha a seu cargo os arreios dos coches da casa real.
(Cast. \textunderscore guarnicionero\textunderscore )
\section{Guarnimento}
\begin{itemize}
\item {Grp. gram.:m.}
\end{itemize}
\begin{itemize}
\item {Utilização:Ant.}
\end{itemize}
\begin{itemize}
\item {Proveniência:(De \textunderscore guarnir\textunderscore )}
\end{itemize}
O mesmo que \textunderscore guarnição\textunderscore .
\section{Guarnir}
\begin{itemize}
\item {Grp. gram.:v. t.}
\end{itemize}
\begin{itemize}
\item {Utilização:Ant.}
\end{itemize}
O mesmo que \textunderscore guarnecer\textunderscore .
(B. lat. \textunderscore garnire\textunderscore )
\section{Guaroupás}
\begin{itemize}
\item {Grp. gram.:m.}
\end{itemize}
\begin{itemize}
\item {Utilização:Ant.}
\end{itemize}
O mesmo que \textunderscore gurupés\textunderscore . Cf. Gil Vicente, \textunderscore Triunfo do Soberbo\textunderscore .
\section{Guarra}
\begin{itemize}
\item {Grp. gram.:f.}
\end{itemize}
\begin{itemize}
\item {Utilização:Ant.}
\end{itemize}
Alarido de dôr; lamentação. Cf. G. Vicente.
(Relaciona-se com \textunderscore guai\textunderscore ?)
\section{Guarro}
\begin{itemize}
\item {Grp. gram.:m.}
\end{itemize}
\begin{itemize}
\item {Utilização:Des.}
\end{itemize}
Doença dos cavallos, produzida por ferro, osso ou outro objecto, que se lhe crava nas mãos.
\section{Godão}
\begin{itemize}
\item {Grp. gram.:m.}
\end{itemize}
\begin{itemize}
\item {Utilização:T. da Índia port}
\end{itemize}
Armazém, depósito.
(Do indo-inglês \textunderscore godown\textunderscore , do mal. \textunderscore gadong\textunderscore )
\section{Guaiaba}
\textunderscore f.\textunderscore  (e der)
O mesmo ou melhor que \textunderscore goiaba\textunderscore , etc. Cf. Beaurepaire-Rohan, \textunderscore Diccion. de Vocab. Bras\textunderscore .
\section{Guar-te}
(Contr. de \textunderscore guarda-te\textunderscore )
\section{Guarubá}
\begin{itemize}
\item {Grp. gram.:m.}
\end{itemize}
Pássaro brasileiro, nocivo aos frutos.
\section{Guarula}
\begin{itemize}
\item {Grp. gram.:f.}
\end{itemize}
Espécie de periquito, de pescoço encarnado.
\section{Guarulhos}
\begin{itemize}
\item {Grp. gram.:m. pl.}
\end{itemize}
Índios do Brasil, no território do Espírito-Santo, civilizados pelos primeiros missionários que ali entraram.
\section{Guaruman}
\begin{itemize}
\item {Grp. gram.:m.}
\end{itemize}
\begin{itemize}
\item {Utilização:Bras. do N}
\end{itemize}
Espécie de palmeira.
\section{Guarús}
\begin{itemize}
\item {Grp. gram.:m. pl.}
\end{itemize}
O mesmo que \textunderscore guarulhos\textunderscore .
\section{Guasca}
\begin{itemize}
\item {Grp. gram.:f.}
\end{itemize}
\begin{itemize}
\item {Utilização:Bras}
\end{itemize}
\begin{itemize}
\item {Grp. gram.:M.}
\end{itemize}
\begin{itemize}
\item {Utilização:Bras. do S}
\end{itemize}
Correia de coiro cru.
O mesmo que \textunderscore caipira\textunderscore .
(Do quíchua \textunderscore huasa\textunderscore )
\section{Guascaço}
\begin{itemize}
\item {Grp. gram.:m.}
\end{itemize}
Pancada com guasca.
\section{Guasquear}
\begin{itemize}
\item {Grp. gram.:v. t.}
\end{itemize}
\begin{itemize}
\item {Utilização:Bras. do S}
\end{itemize}
Bater com guasca.
\section{Guatambu}
\begin{itemize}
\item {Grp. gram.:m.}
\end{itemize}
Madeira do Brasil.
\section{Guatapuma}
\begin{itemize}
\item {Grp. gram.:f.}
\end{itemize}
Pau brasil das Antilhas.
\section{Guatós}
\begin{itemize}
\item {Grp. gram.:m. pl.}
\end{itemize}
Nome de várias tríbos de Índios do Brasil, ao norte de Cuiabá.
\section{Guau}
\begin{itemize}
\item {Grp. gram.:m.}
\end{itemize}
\begin{itemize}
\item {Utilização:Bras}
\end{itemize}
Espécie de dança, entre os indígenas.
\section{Guaxima}
\begin{itemize}
\item {Grp. gram.:f.}
\end{itemize}
\begin{itemize}
\item {Utilização:Bras}
\end{itemize}
Gênero de arbustos malváceos, de casca têxtil.

O mesmo que \textunderscore malvaísco\textunderscore .
(Tupi \textunderscore aguaixima\textunderscore )
\section{Guaximba-preta}
\begin{itemize}
\item {Grp. gram.:f.}
\end{itemize}
Planta urticácea do Brasil.
\section{Guaxinguba}
\begin{itemize}
\item {Grp. gram.:f.}
\end{itemize}
\begin{itemize}
\item {Utilização:Bras. do N}
\end{itemize}
Árvore urticácea, de cuja casca alguns selvagens fazem tangas e camisas.
\section{Guaxinim}
\begin{itemize}
\item {Grp. gram.:m.}
\end{itemize}
\begin{itemize}
\item {Utilização:Bras}
\end{itemize}
Espécie de raposa, que se sustenta de caranguejos.
\section{Guaxis}
\begin{itemize}
\item {Grp. gram.:m. pl.}
\end{itemize}
\begin{itemize}
\item {Utilização:Bras}
\end{itemize}
Uma das tríbos indígenas de Mato-Grôsso.
\section{Guaxuma}
\begin{itemize}
\item {Grp. gram.:f.}
\end{itemize}
\begin{itemize}
\item {Utilização:Bras}
\end{itemize}
O mesmo que \textunderscore guaxima\textunderscore .
\section{Guaxupé}
\begin{itemize}
\item {Grp. gram.:m.}
\end{itemize}
\begin{itemize}
\item {Utilização:Bras}
\end{itemize}
Espécie de penteado. Cf. Taunay, \textunderscore Innocência\textunderscore , 394.
\section{Guayaba}
\textunderscore f.\textunderscore  (e der)
O mesmo ou melhor que \textunderscore goiaba\textunderscore , etc. Cf. Beaurepaire-Rohan, \textunderscore Diccion. de Vocab. Bras\textunderscore .
\section{Guayaca}
\begin{itemize}
\item {Grp. gram.:f.}
\end{itemize}
\begin{itemize}
\item {Utilização:Bras. do S}
\end{itemize}
Bôlsa de coiro, que se prende á cinta e em que o viandante guarda dinheiro e outros pequenos objectos.
(Do quichua \textunderscore huayaca\textunderscore )
\section{Guazil}
\begin{itemize}
\item {Grp. gram.:m.}
\end{itemize}
Governador, entre os Árabes e os Persas.
(Cp. \textunderscore aguazil\textunderscore )
\section{Guazilado}
\begin{itemize}
\item {Grp. gram.:m.}
\end{itemize}
Cargo de guazil.
\section{Guazuma}
\begin{itemize}
\item {Grp. gram.:f.}
\end{itemize}
Gênero de árvores da América tropical.
\section{Guazupucu}
\begin{itemize}
\item {Grp. gram.:f.}
\end{itemize}
Cabrito montês da América.
\section{Guça}
\begin{itemize}
\item {Grp. gram.:f.}
\end{itemize}
\begin{itemize}
\item {Utilização:Ant.}
\end{itemize}
Pressa; actividade; diligência.
(Cp. \textunderscore aguçar\textunderscore )
\section{Guçuso}
\begin{itemize}
\item {Grp. gram.:m.}
\end{itemize}
Arbusto combretáceo de Angola.
\section{Gudão}
\begin{itemize}
\item {Grp. gram.:m.}
\end{itemize}
\begin{itemize}
\item {Utilização:T. da Índia port}
\end{itemize}
Armazém, depósito.
(Do indo-inglês \textunderscore godown\textunderscore , do mal. \textunderscore gadong\textunderscore )
\section{Gudinha}
\begin{itemize}
\item {Grp. gram.:f.}
\end{itemize}
\begin{itemize}
\item {Utilização:Ant.}
\end{itemize}
\begin{itemize}
\item {Proveniência:(Do ant. fr. \textunderscore gaudine\textunderscore )}
\end{itemize}
Pequena quinta de recreio.
\section{Gudre}
\begin{itemize}
\item {Grp. gram.:f.}
\end{itemize}
\begin{itemize}
\item {Utilização:Des.}
\end{itemize}
O mesmo que \textunderscore grude\textunderscore .
(Metáth.)
\section{Guebro}
\begin{itemize}
\item {Grp. gram.:m.}
\end{itemize}
\begin{itemize}
\item {Proveniência:(Do pers. \textunderscore ghebr\textunderscore , infiel)}
\end{itemize}
Nome, dado pelos Muçulmanos aos sectários de Zoroastro.
O mesmo que \textunderscore parse\textunderscore .
\section{Gueche}
\begin{itemize}
\item {Grp. gram.:m.}
\end{itemize}
Espécie de adobe, na Índia portuguesa.
\section{Guedé}
\begin{itemize}
\item {Grp. gram.:m.}
\end{itemize}
\begin{itemize}
\item {Utilização:Ant.}
\end{itemize}
Espécie de pequena embarcação.
\section{Guedelha}
\begin{itemize}
\item {fónica:dê}
\end{itemize}
\begin{itemize}
\item {Grp. gram.:f.}
\end{itemize}
\begin{itemize}
\item {Grp. gram.:M.}
\end{itemize}
\begin{itemize}
\item {Utilização:Prov.}
\end{itemize}
\begin{itemize}
\item {Proveniência:(Do ant. fr. \textunderscore gade\textunderscore  + \textunderscore lain\textunderscore ?)}
\end{itemize}
\textunderscore f.\textunderscore  (e der.)
O mesmo que \textunderscore gadelha\textunderscore , etc.
Cabello desgrenhado e comprido.
Grenha; melena.
Madeixa de quaesquer fios.
Diabo.
\section{Guegue}
\begin{itemize}
\item {Grp. gram.:m.}
\end{itemize}
Gênero de plantas medicinaes da ilha de San-Thomé, (\textunderscore spondeas lutea\textunderscore , Lin.).
\section{Guegue-falso}
\begin{itemize}
\item {Grp. gram.:m.}
\end{itemize}
Erva medicinal da ilha de San-Thomé.
\section{Guegueses}
\begin{itemize}
\item {Grp. gram.:m. pl.}
\end{itemize}
Nação de Índios brasileiros, que habitaram nas cabeceiras do Parnaíba.
\section{Gueice}
\begin{itemize}
\item {Grp. gram.:m.}
\end{itemize}
\begin{itemize}
\item {Utilização:Des.}
\end{itemize}
\begin{itemize}
\item {Proveniência:(Do ár. \textunderscore geis\textunderscore )}
\end{itemize}
O mesmo que \textunderscore argamassa\textunderscore .
\section{Gueiro}
\begin{itemize}
\item {Grp. gram.:m.}
\end{itemize}
\begin{itemize}
\item {Utilização:Prov.}
\end{itemize}
Uma das peças da asna.
(Relaciona-se com \textunderscore guieiro\textunderscore ?)
\section{Gueiro}
\begin{itemize}
\item {Grp. gram.:m.}
\end{itemize}
\begin{itemize}
\item {Utilização:T. da Áfr. Or. Port}
\end{itemize}
Casa, onde se reúnem, para dormir, rapazes e raparigas.
\section{Gueixa}
\textunderscore fem.\textunderscore  de \textunderscore gueixo\textunderscore .
\section{Gueixo}
\begin{itemize}
\item {Grp. gram.:m.}
\end{itemize}
\begin{itemize}
\item {Utilização:Açor}
\end{itemize}
O mesmo que \textunderscore novilho\textunderscore .
\section{Gueja}
\begin{itemize}
\item {Grp. gram.:f.}
\end{itemize}
\begin{itemize}
\item {Proveniência:(Do ingl. \textunderscore gauge\textunderscore )}
\end{itemize}
Régua, para verificar a largura da via férrea.
\section{Guelra}
\begin{itemize}
\item {Grp. gram.:f.}
\end{itemize}
Apparelho respiratório dos animaes, que vivem ou podem viver na água.
Brânchias.
\section{Guelricho}
\begin{itemize}
\item {Grp. gram.:m.}
\end{itemize}
\begin{itemize}
\item {Utilização:Pesc.}
\end{itemize}
Armadilha de nassas e botirões.
\section{Guelrita}
\begin{itemize}
\item {Grp. gram.:f.}
\end{itemize}
\begin{itemize}
\item {Utilização:Prov.}
\end{itemize}
\begin{itemize}
\item {Utilização:beir.}
\end{itemize}
Cesto grande de vêrga, empregado na pesca de peixes de água doce.
(Cp. \textunderscore guelricho\textunderscore ^1)
\section{Guenso}
\begin{itemize}
\item {Grp. gram.:m.}
\end{itemize}
Árvore de Angola, (\textunderscore combretum dipterum\textunderscore ).
\section{Guenzo}
\begin{itemize}
\item {Grp. gram.:adj.}
\end{itemize}
\begin{itemize}
\item {Utilização:Bras}
\end{itemize}
Adoentado.
Enfraquecido; enfèzado.
Trangalhadanças.
\section{Gueos}
\begin{itemize}
\item {Grp. gram.:m. pl.}
\end{itemize}
Antigo povo da Ásia. Cf. \textunderscore Peregrinação\textunderscore , XLI; Barros, \textunderscore Déc.\textunderscore  III, l. II, c. 5.
\section{Guere}
\begin{itemize}
\item {Grp. gram.:m.}
\end{itemize}
Ave trepadora.
\section{Guereroba}
\begin{itemize}
\item {Grp. gram.:f.}
\end{itemize}
Planta apocýnea do Maranhão.
O mesmo que \textunderscore guariroba\textunderscore ?
\section{Guerirova}
\begin{itemize}
\item {Grp. gram.:f.}
\end{itemize}
O mesmo que \textunderscore guereroba\textunderscore . Cf. Júl. Ribeiro, \textunderscore Carne\textunderscore .
\section{Guerra}
\begin{itemize}
\item {Grp. gram.:f.}
\end{itemize}
\begin{itemize}
\item {Utilização:Ext.}
\end{itemize}
\begin{itemize}
\item {Utilização:Fig.}
\end{itemize}
\begin{itemize}
\item {Proveniência:(Do ant. alt. al. \textunderscore werra\textunderscore )}
\end{itemize}
Luta com armas, entre nações ou entre partidos.
Campanha.
Luta.
Arte militar: \textunderscore escola de guerra\textunderscore .
Negócios militares: \textunderscore Ministério da Guerra\textunderscore .
Opposição: \textunderscore aquelle Deputado faz guerra ao Govêrno\textunderscore .
\section{Guerreador}
\begin{itemize}
\item {Grp. gram.:m.  e  adj.}
\end{itemize}
O que guerreia.
\section{Guerreão}
\begin{itemize}
\item {Grp. gram.:m.  e  adj.}
\end{itemize}
\begin{itemize}
\item {Utilização:Prov.}
\end{itemize}
\begin{itemize}
\item {Utilização:alg.}
\end{itemize}
\begin{itemize}
\item {Proveniência:(De \textunderscore guerrear\textunderscore )}
\end{itemize}
Homem desordeiro.
\section{Guerrear}
\begin{itemize}
\item {Grp. gram.:v. t.}
\end{itemize}
\begin{itemize}
\item {Utilização:Fig.}
\end{itemize}
\begin{itemize}
\item {Grp. gram.:V. i.}
\end{itemize}
Fazer guerra a.
Hostilizar: \textunderscore a França já guerreou Portugal\textunderscore .
Fazer opposição a.
Opprimir; perseguir.
Combater.
Fazer guerra.
\section{Guerreia}
\begin{itemize}
\item {Grp. gram.:f.}
\end{itemize}
\begin{itemize}
\item {Utilização:Prov.}
\end{itemize}
\begin{itemize}
\item {Utilização:beir.}
\end{itemize}
\begin{itemize}
\item {Proveniência:(De \textunderscore guerrear\textunderscore )}
\end{itemize}
Desordem entre rapazes; luta.
\section{Guerreiramente}
\begin{itemize}
\item {Grp. gram.:adv.}
\end{itemize}
De modo guerreiro.
\section{Guerreiro}
\begin{itemize}
\item {Grp. gram.:m.}
\end{itemize}
\begin{itemize}
\item {Grp. gram.:Adj.}
\end{itemize}
\begin{itemize}
\item {Proveniência:(De \textunderscore guerra\textunderscore )}
\end{itemize}
Aquelle que guerreia.
Aquelle que tem ânimo bellicoso.
Aquelle que tem entrado em guerras, portando-se com valentia.
Aquelle que exerce a profissão das armas.
Relativo a guerra; combativo: \textunderscore indole guerreira\textunderscore .
\section{Guerrento}
\begin{itemize}
\item {Grp. gram.:adj.}
\end{itemize}
\begin{itemize}
\item {Utilização:Prov.}
\end{itemize}
\begin{itemize}
\item {Utilização:trasm.}
\end{itemize}
\begin{itemize}
\item {Proveniência:(De \textunderscore guerreia\textunderscore )}
\end{itemize}
Rabugento; enfadonho.
\section{Guerrilha}
\begin{itemize}
\item {Grp. gram.:f.}
\end{itemize}
\begin{itemize}
\item {Grp. gram.:M.}
\end{itemize}
\begin{itemize}
\item {Proveniência:(De \textunderscore guerra\textunderscore )}
\end{itemize}
Pequeno corpo de guerreiros voluntários, que, sem subordinação á disciplina do exército, atacam geralmente o inimigo fóra de campo ou por emboscada.
Bando de ladrões.
Tropa indisciplinada.
Facção política, sem elementos para constituír partído disciplinado.
Guerrilheiro.
\section{Guerrilhagem}
\begin{itemize}
\item {Grp. gram.:f.}
\end{itemize}
\begin{itemize}
\item {Proveniência:(De \textunderscore guerrilhar\textunderscore )}
\end{itemize}
Vida de guerrilheiro.
Os guerrilheiros.
\section{Guerrilhar}
\begin{itemize}
\item {Grp. gram.:v. i.}
\end{itemize}
\begin{itemize}
\item {Proveniência:(De \textunderscore guerrilha\textunderscore )}
\end{itemize}
Ser guerrilheiro.
\section{Guerrilheiro}
\begin{itemize}
\item {Grp. gram.:m.}
\end{itemize}
Indivíduo, pertencente a uma guerrilha.
Chefe de guerrilha.
\section{Guesso}
\begin{itemize}
\item {fónica:guê}
\end{itemize}
\begin{itemize}
\item {Grp. gram.:adj.}
\end{itemize}
\begin{itemize}
\item {Utilização:Gír.}
\end{itemize}
\begin{itemize}
\item {Proveniência:(Do fr. \textunderscore gauche\textunderscore ?)}
\end{itemize}
Desajeitado; que é um trangalhadanças.
\section{Gueta}
\begin{itemize}
\item {fónica:gu-ê}
\end{itemize}
\begin{itemize}
\item {Grp. gram.:f.}
\end{itemize}
\begin{itemize}
\item {Utilização:Prov.}
\end{itemize}
\begin{itemize}
\item {Utilização:trasm.}
\end{itemize}
O mesmo que \textunderscore véstia\textunderscore ^1.
\section{Guetárdeas}
\begin{itemize}
\item {Grp. gram.:f. pl.}
\end{itemize}
\begin{itemize}
\item {Utilização:Bot.}
\end{itemize}
Nome, dado por De-Candolle a uma família de plantas, á custa das rubiáceas.
\section{Guete}
\begin{itemize}
\item {fónica:guê}
\end{itemize}
\begin{itemize}
\item {Grp. gram.:m.}
\end{itemize}
\begin{itemize}
\item {Utilização:Ant.}
\end{itemize}
Documento público, com que o judeu, convertido á fé christan, se desquitava da mulher, se esta por mais de um anno permanecia na fé judaica.
(Relaciona-se com o al. \textunderscore wacht\textunderscore ? Cp. fr. \textunderscore guet\textunderscore )
\section{Guexa}
\begin{itemize}
\item {fónica:guê}
\end{itemize}
\begin{itemize}
\item {Grp. gram.:f.}
\end{itemize}
\begin{itemize}
\item {Utilização:Prov.}
\end{itemize}
O mesmo que \textunderscore quixa\textunderscore .
O mesmo que \textunderscore gueixa\textunderscore .
\section{Guexo}
\begin{itemize}
\item {fónica:guê}
\end{itemize}
\begin{itemize}
\item {Grp. gram.:m.}
\end{itemize}
O mesmo que \textunderscore gueixo\textunderscore .
\section{Guía}
\begin{itemize}
\item {Grp. gram.:f.}
\end{itemize}
\begin{itemize}
\item {Utilização:Prov.}
\end{itemize}
\begin{itemize}
\item {Utilização:Bras}
\end{itemize}
\begin{itemize}
\item {Utilização:Prov.}
\end{itemize}
\begin{itemize}
\item {Grp. gram.:M.}
\end{itemize}
Acto ou effeito de guiar.
Pessôa, que guía.
Documento, com que se recebem mercadorias ou encommendas.
Relação ou documento, que acompanha a correspondência official.
Vara, em que se apoia a vide empada.
Cada uma das pennas maiores das asas das aves.
Titulo de várias publicações didácticas.
Roteiro.
Cada uma das correias compridas, que, seguras pelo cocheiro, estão em communicação com os freios dos cavallos de tiro.
Correia, afivelada na argola do cabeção de um cavallo, nos exercícios de picadeiro.
Parelha de cavallos, que, á frente de outra ou outras, puxa com ellas uma carruagem.
Os cabellos extremos do bigode.
Peça, que dirige o movimento da haste do êmbolo em máquinas de vapor.
Tábua, em que se enfia a cana do graminho.
Cabo náutico, que serve de direcção aos objectos.
Madeiro, que se põe no estaleiro, para servir de direcção ás escoras do navio.
Qualquer rebento ou ramo novo de uma árvore.
Espécie de enguia.
O mesmo que \textunderscore soga\textunderscore , com que se prendem bois.
Homem, que guía, conduz ou dirige.
Animal, que vai á frente de um rebanho, guiando-o ou abrindo-lhe caminho.
Fôlha, caderno, opúsculo ou livro, que contém indicações úteis, como as do horário dos combóios, de outros serviços de viação, do serviço postal, etc.
\section{Guiabelha}
\begin{itemize}
\item {fónica:bê}
\end{itemize}
\begin{itemize}
\item {Grp. gram.:f.}
\end{itemize}
O mesmo que \textunderscore diabelha\textunderscore .
\section{Guiaca}
\begin{itemize}
\item {Grp. gram.:f.}
\end{itemize}
O mesmo que \textunderscore ébano\textunderscore .
(Por \textunderscore guaiacana\textunderscore , de \textunderscore guaiaco\textunderscore ? Cp. \textunderscore guaiaco\textunderscore )
\section{Guiacana}
\begin{itemize}
\item {Grp. gram.:f.}
\end{itemize}
O mesmo que \textunderscore ébano\textunderscore .
(Por \textunderscore guaiacana\textunderscore , de \textunderscore guaiaco\textunderscore ? Cp. \textunderscore guaiaco\textunderscore )
\section{Guiada}
\begin{itemize}
\item {Grp. gram.:f.}
\end{itemize}
\begin{itemize}
\item {Utilização:Bras}
\end{itemize}
O mesmo que \textunderscore aguilhada\textunderscore .
\section{Guiador}
\begin{itemize}
\item {Grp. gram.:adj.}
\end{itemize}
\begin{itemize}
\item {Grp. gram.:M.}
\end{itemize}
Que guía.
Aquelle que guía.
Índice de livros de escrituração.
Velocipedista que, fazendo parte de um equipo, vai na frente e o guía.
\section{Guia-enxêrto}
\begin{itemize}
\item {Grp. gram.:m.}
\end{itemize}
Máquina, para fazer enxertos, inventada recentemente por Castelbon, francês.
\section{Guiagem}
\begin{itemize}
\item {Grp. gram.:f.}
\end{itemize}
\begin{itemize}
\item {Proveniência:(De \textunderscore guiar\textunderscore )}
\end{itemize}
Imposto sôbre transportes.
\section{Guiaiamum}
\begin{itemize}
\item {Grp. gram.:m.}
\end{itemize}
O mesmo que \textunderscore goiamum\textunderscore .
\section{Guiamento}
\begin{itemize}
\item {Grp. gram.:m.}
\end{itemize}
Acto ou effeito de guiar.
\section{Guiamú}
\begin{itemize}
\item {Grp. gram.:m.}
\end{itemize}
\begin{itemize}
\item {Utilização:Bras. do Rio}
\end{itemize}
Malta de capoeiras.
\section{Guianas}
\begin{itemize}
\item {fónica:gùi}
\end{itemize}
\begin{itemize}
\item {Grp. gram.:m. pl.}
\end{itemize}
Indígenas do norte do Brasil, nas margens do Araça.
\section{Guianês}
\begin{itemize}
\item {fónica:gùi}
\end{itemize}
\begin{itemize}
\item {Grp. gram.:adj.}
\end{itemize}
\begin{itemize}
\item {Grp. gram.:M.}
\end{itemize}
Relativo á Guiana.
Habitante de Guiana.
\section{Guiante}
\begin{itemize}
\item {Grp. gram.:adj.}
\end{itemize}
\begin{itemize}
\item {Utilização:Des.}
\end{itemize}
Que guia.
\section{Guião}
\begin{itemize}
\item {Grp. gram.:m.}
\end{itemize}
\begin{itemize}
\item {Proveniência:(De \textunderscore guiar\textunderscore )}
\end{itemize}
Pendão, estandarte, que vai á frente de algumas procissões ou irmandades.
Estandarte, que ia na frente das tropas.
Cavalleiro, que levava êsse estandarte.
Sinal antigo que, no fim de uma linha de música, indicava a primeira nota da linha seguinte.
\section{Guiaquilite}
\begin{itemize}
\item {Grp. gram.:f.}
\end{itemize}
Resina fóssil de Guaiaquil, (América do Sul).
\section{Guiar}
\begin{itemize}
\item {fónica:gùi}
\end{itemize}
\begin{itemize}
\item {Grp. gram.:v. t.}
\end{itemize}
\begin{itemize}
\item {Utilização:Fig.}
\end{itemize}
\begin{itemize}
\item {Utilização:Prov.}
\end{itemize}
\begin{itemize}
\item {Utilização:minh.}
\end{itemize}
\begin{itemize}
\item {Grp. gram.:V. i.}
\end{itemize}
\begin{itemize}
\item {Proveniência:(Do b. lat. \textunderscore guidare\textunderscore )}
\end{itemize}
Servir de guía a.
Conduzir; dirigir.
Aconselhar.
Proteger.
Ensinar; governar (cavallos).
Compor, consertar.
Ir.
Navegar; mostrar direcção.
\section{Guibo}
\begin{itemize}
\item {Grp. gram.:m.}
\end{itemize}
\begin{itemize}
\item {Utilização:Gír.}
\end{itemize}
Artelho.
\section{Guicho}
\begin{itemize}
\item {Grp. gram.:adj.}
\end{itemize}
\begin{itemize}
\item {Utilização:Prov.}
\end{itemize}
\begin{itemize}
\item {Utilização:trasm.}
\end{itemize}
\begin{itemize}
\item {Utilização:minh.}
\end{itemize}
\begin{itemize}
\item {Utilização:Prov.}
\end{itemize}
\begin{itemize}
\item {Utilização:trasm.}
\end{itemize}
Muito vivo, buliçoso, (falando-se de crianças ou de certos animaes, como o rato).
Seguro, direitinho, viçoso, (falando-se de um vegetal, transplantado de há pouco).
(Cp. \textunderscore guisso\textunderscore )
\section{Guicoxo}
\begin{itemize}
\item {Grp. gram.:m.}
\end{itemize}
\begin{itemize}
\item {Utilização:Bras}
\end{itemize}
Peixe, espécie de raia.
\section{Guieira}
\begin{itemize}
\item {Grp. gram.:f.}
\end{itemize}
\begin{itemize}
\item {Utilização:Constr.}
\end{itemize}
\begin{itemize}
\item {Utilização:Pop.}
\end{itemize}
\begin{itemize}
\item {Proveniência:(De \textunderscore guieiro\textunderscore )}
\end{itemize}
O mesmo que \textunderscore rincão\textunderscore  do telhado.
Vento brando mas frio.
\section{Guieiro}
\begin{itemize}
\item {Grp. gram.:adj.}
\end{itemize}
\begin{itemize}
\item {Grp. gram.:M.}
\end{itemize}
\begin{itemize}
\item {Utilização:Prov.}
\end{itemize}
\begin{itemize}
\item {Utilização:minh.}
\end{itemize}
\begin{itemize}
\item {Utilização:Constr.}
\end{itemize}
\begin{itemize}
\item {Proveniência:(De \textunderscore guiar\textunderscore )}
\end{itemize}
Que serve de guia ou que vai na frente.
Aquelle que vai á frente, guiando, (falando-se especialmente de um animal que vai adeante de um rebanho).
O mesmo que \textunderscore aguieiro\textunderscore .
Rêgo, por onde se guia a água.
O mesmo que \textunderscore guieira\textunderscore .
\section{Guiga}
\begin{itemize}
\item {Grp. gram.:f.}
\end{itemize}
\begin{itemize}
\item {Proveniência:(Do ingl. \textunderscore gig\textunderscore )}
\end{itemize}
Barco estreito e comprido, próprio para regatas.
\section{Guigó}
\begin{itemize}
\item {Grp. gram.:m.}
\end{itemize}
Grande árvore medicinal da ilha de San-Thomé.
\section{Guilha}
\begin{itemize}
\item {Grp. gram.:m.}
\end{itemize}
\begin{itemize}
\item {Utilização:Des.}
\end{itemize}
\begin{itemize}
\item {Utilização:Fig.}
\end{itemize}
Colheita abundante de cereaes.
Velhacaria; fraude.
(Cast. \textunderscore guilla\textunderscore )
\section{Guilherme}
\begin{itemize}
\item {Grp. gram.:m.}
\end{itemize}
\begin{itemize}
\item {Utilização:Prov.}
\end{itemize}
\begin{itemize}
\item {Utilização:minh.}
\end{itemize}
\begin{itemize}
\item {Proveniência:(De \textunderscore Guilherme\textunderscore , n. p.)}
\end{itemize}
Instrumento de carpinteiro, com que se fazem os filetes das portas, junturas das tábuas, frisos, etc.
Pequena plaina de carpinteiro, para desbastar as esquinas dos recortes, feitos pelos chamados \textunderscore machos-fêmeas\textunderscore .
Moéda de oiro holandesa, do valor de 3$825 reis proximamente.
\section{Guilho}
\begin{itemize}
\item {Grp. gram.:m.}
\end{itemize}
\begin{itemize}
\item {Utilização:Prov.}
\end{itemize}
\begin{itemize}
\item {Utilização:beir.}
\end{itemize}
Espigão de ferro ou de pedra, que termina inferiormente o eixo do rodízio.
Cunha de ferro, para fender a pedra.
(Cp. ingl. \textunderscore guill\textunderscore )
\section{Guilhote}
\begin{itemize}
\item {Grp. gram.:m.}
\end{itemize}
\begin{itemize}
\item {Utilização:Des.}
\end{itemize}
\begin{itemize}
\item {Utilização:Fig.}
\end{itemize}
\begin{itemize}
\item {Proveniência:(De \textunderscore guilha\textunderscore )}
\end{itemize}
Homem, que faz a colheita de terrenos que não semeou.
Velhaco.
Defraudador. Cf. \textunderscore Eufrosina\textunderscore , (pról.).
\section{Guilhotina}
\begin{itemize}
\item {Grp. gram.:f.}
\end{itemize}
\begin{itemize}
\item {Proveniência:(Fr. \textunderscore guillotine\textunderscore )}
\end{itemize}
Instrumento, com que se decepa a cabeça dos condemnados á pena de morte.
\section{Guilhotinar}
\begin{itemize}
\item {Grp. gram.:v. t.}
\end{itemize}
Decepar com a guilhotina.
\section{Guillochador}
\begin{itemize}
\item {Grp. gram.:m.}
\end{itemize}
\begin{itemize}
\item {Proveniência:(De \textunderscore guilloché\textunderscore )}
\end{itemize}
Official, que faz guillochés. Cf. \textunderscore Inquér. Industr.\textunderscore , p. II, l. II, 244.
\section{Guillochés}
\begin{itemize}
\item {Grp. gram.:m.  e  f. pl.}
\end{itemize}
\begin{itemize}
\item {Proveniência:(Fr. \textunderscore guilloché\textunderscore )}
\end{itemize}
Ornato, composto de linhas e traços que se cruzam.
\section{Guilochador}
\begin{itemize}
\item {Grp. gram.:m.}
\end{itemize}
\begin{itemize}
\item {Proveniência:(De \textunderscore guiloché\textunderscore )}
\end{itemize}
Oficial, que faz guilochés. Cf. \textunderscore Inquér. Industr.\textunderscore , p. II, l. II, 244.
\section{Guim!}
\begin{itemize}
\item {fónica:gu-im}
\end{itemize}
\begin{itemize}
\item {Grp. gram.:interj.}
\end{itemize}
\begin{itemize}
\item {Utilização:Prov.}
\end{itemize}
\begin{itemize}
\item {Utilização:alg.}
\end{itemize}
Voz, com que se chamam os porcos.
\section{Guimbarda}
\begin{itemize}
\item {Grp. gram.:f.}
\end{itemize}
Jôgo de cartas, entre cinco a nove pessôas, no qual há cinco bolos, formados pelas entradas dos parceiros, sendo o primeiro bolo ganho por quem tiver a dama de copas, que chamam a \textunderscore guimbarda\textunderscore . Cf. \textunderscore Manual dos Jogos\textunderscore , 238.
\section{Guimbé}
\begin{itemize}
\item {Grp. gram.:m.}
\end{itemize}
\begin{itemize}
\item {Utilização:Bras. de S. Paulo}
\end{itemize}
Planta, talvez o mesmo que \textunderscore imbé\textunderscore .
\section{Guimbombo}
\begin{itemize}
\item {Grp. gram.:m.}
\end{itemize}
O mesmo que \textunderscore gombo\textunderscore . Cf. S. Costa, \textunderscore Hist. das Pl. Med.\textunderscore 
\section{Guina}
\begin{itemize}
\item {Grp. gram.:f.}
\end{itemize}
O mesmo que \textunderscore gana\textunderscore :«\textunderscore ás vezes davam-lhe guinas de fugir\textunderscore ». Camillo, \textunderscore Viuva do Enforc.\textunderscore , III, 56.
\section{Guina}
\begin{itemize}
\item {Grp. gram.:f.}
\end{itemize}
Árvore rubiácea do Brasil.
\section{Guinada}
\begin{itemize}
\item {Grp. gram.:f.}
\end{itemize}
\begin{itemize}
\item {Utilização:Ext.}
\end{itemize}
\begin{itemize}
\item {Utilização:Pop.}
\end{itemize}
\begin{itemize}
\item {Proveniência:(De \textunderscore guinar\textunderscore )}
\end{itemize}
Desvio, que uma embarcação faz, da sua esteira.
Salto, que o cavallo dá, para se esquivar ao castigo do cavalleiro.
Dôr viva e súbita.
Impressão súbita, ataque:«\textunderscore cascalhar uma guinada de riso\textunderscore ». Camillo, \textunderscore Retr. de Ricard.\textunderscore , c. 11.
\section{Guinalda}
\begin{itemize}
\item {Grp. gram.:f.}
\end{itemize}
\begin{itemize}
\item {Utilização:Prov.}
\end{itemize}
\begin{itemize}
\item {Utilização:trasm.}
\end{itemize}
Tuna.
Vadiagem; vida airada.
\section{Guinaldeiro}
\begin{itemize}
\item {Grp. gram.:adj.}
\end{itemize}
Que gosta de andar de guinalda ou na guinalda.
\section{Guinaldice}
\begin{itemize}
\item {Grp. gram.:f.}
\end{itemize}
Disposição para a guinalda.
Qualidade de guinaldeiro.
\section{Guinar}
\begin{itemize}
\item {Grp. gram.:v. i.}
\end{itemize}
\begin{itemize}
\item {Grp. gram.:V. t.}
\end{itemize}
\begin{itemize}
\item {Grp. gram.:V. p.}
\end{itemize}
\begin{itemize}
\item {Utilização:Ant.}
\end{itemize}
\begin{itemize}
\item {Utilização:Chul.}
\end{itemize}
Mover-se ás guinadas.
Desviar-se (uma embarcação) da sua esteira.
Desviar-se rapidamente:«\textunderscore guinou de uma cadeira para outra\textunderscore ». Camillo, \textunderscore Cav. em Ruínas\textunderscore , 194 e 205.
Voltar rapidamente:«\textunderscore guinando feiamente os olhos de lado a lado\textunderscore ». Camillo, \textunderscore Caveira\textunderscore , 239.
Esgueirar-se, escapulir-se. Cf. \textunderscore Anat. Joc.\textunderscore , I, 443.
(Talvez do ingl. \textunderscore gin\textunderscore , surpresa, cilada)
\section{Guincha}
\begin{itemize}
\item {Grp. gram.:f.}
\end{itemize}
\begin{itemize}
\item {Utilização:Prov.}
\end{itemize}
\begin{itemize}
\item {Utilização:trasm.}
\end{itemize}
Sachola.
\section{Guinchada}
\begin{itemize}
\item {Grp. gram.:f.}
\end{itemize}
\begin{itemize}
\item {Proveniência:(De \textunderscore guinchar\textunderscore )}
\end{itemize}
Série de guinchos; gritaria.
\section{Guinchado}
\begin{itemize}
\item {Grp. gram.:m.}
\end{itemize}
\begin{itemize}
\item {Proveniência:(De \textunderscore guinchar\textunderscore )}
\end{itemize}
Série de guinchos; gritaria.
\section{Guinchante}
\begin{itemize}
\item {Grp. gram.:adj.}
\end{itemize}
\begin{itemize}
\item {Proveniência:(De \textunderscore guinchar\textunderscore )}
\end{itemize}
Que dá guinchos.
\section{Guinchar}
\begin{itemize}
\item {Grp. gram.:v. i.}
\end{itemize}
\begin{itemize}
\item {Utilização:Fam.}
\end{itemize}
Dar guinchos.
\section{Guinchelro}
\begin{itemize}
\item {Grp. gram.:m.}
\end{itemize}
\begin{itemize}
\item {Utilização:Prov.}
\end{itemize}
\begin{itemize}
\item {Utilização:trasm.}
\end{itemize}
Pequeno galho de uma árvore.
(Talvez por \textunderscore guicheiro\textunderscore , de \textunderscore guicho\textunderscore )
\section{Guincho}
\begin{itemize}
\item {Grp. gram.:m.}
\end{itemize}
\begin{itemize}
\item {Utilização:Fam.}
\end{itemize}
\begin{itemize}
\item {Utilização:Prov.}
\end{itemize}
\begin{itemize}
\item {Utilização:minh.}
\end{itemize}
\begin{itemize}
\item {Utilização:Fig.}
\end{itemize}
\begin{itemize}
\item {Proveniência:(T. onom.)}
\end{itemize}
Som agudo e inarticulado do homem e de alguns animaes.
Gaivão.
Apparelho para levantar pesos.
\textunderscore Ninho de guincho\textunderscore , ninho, em que a ave faz provisão de cibo.
Casa farta.
Pechincha.
\section{Guincho-da-taínha}
\begin{itemize}
\item {Grp. gram.:f.}
\end{itemize}
Nome vulgar de uma ave, (\textunderscore circaetus brachydatylus\textunderscore , Gould). Cf. Alb. Giraldes, \textunderscore Philos. Nat.\textunderscore , III, 98.
\section{Guinda}
\begin{itemize}
\item {Grp. gram.:f.}
\end{itemize}
\begin{itemize}
\item {Utilização:Náut.}
\end{itemize}
\begin{itemize}
\item {Utilização:Ant.}
\end{itemize}
\begin{itemize}
\item {Proveniência:(De \textunderscore guindar\textunderscore )}
\end{itemize}
Corda para guindar.
Altura dos mastros, mastaréus, etc.
\section{Guindagem}
\begin{itemize}
\item {Grp. gram.:f.}
\end{itemize}
\begin{itemize}
\item {Utilização:Ant.}
\end{itemize}
Acto de guindar.
Acto de alguém se elevar socialmente ou de melhorar em fortuna:«\textunderscore de todas estas encomendas não tirei comissões nem guindagens\textunderscore ». (De um testamento de 1692)
\section{Guindaleta}
\begin{itemize}
\item {fónica:lê}
\end{itemize}
\begin{itemize}
\item {Grp. gram.:f.}
\end{itemize}
O mesmo que \textunderscore guindalete\textunderscore .
\section{Guindalete}
\begin{itemize}
\item {fónica:lê}
\end{itemize}
\begin{itemize}
\item {Grp. gram.:m.}
\end{itemize}
\begin{itemize}
\item {Proveniência:(Do rad. de \textunderscore guindar\textunderscore )}
\end{itemize}
Cabo do guindaste.
\section{Guinda-maina}
\begin{itemize}
\item {Grp. gram.:f.}
\end{itemize}
\begin{itemize}
\item {Utilização:Ant.}
\end{itemize}
\begin{itemize}
\item {Proveniência:(De \textunderscore guindar\textunderscore  + \textunderscore amainar\textunderscore )}
\end{itemize}
O arrear da bandeira de um navio, em sinal de cortesia ou despedida, para com outro navio.
Acto de abater e elevar logo a bandeira.
\section{Guindamento}
\begin{itemize}
\item {Grp. gram.:m.}
\end{itemize}
Acto ou effeito de guindar.
\section{Guindar}
\begin{itemize}
\item {Grp. gram.:v. t.}
\end{itemize}
\begin{itemize}
\item {Proveniência:(Do ant. alt. al. \textunderscore windan\textunderscore )}
\end{itemize}
Içar, levantar.
Elevar, tornar empolado, pretensioso.
Erguer a uma posição elevada.
\section{Guindareza}
\begin{itemize}
\item {Grp. gram.:f.}
\end{itemize}
(V.guindalete)
\section{Guindaste}
\begin{itemize}
\item {Grp. gram.:m.}
\end{itemize}
\begin{itemize}
\item {Utilização:Prov.}
\end{itemize}
\begin{itemize}
\item {Proveniência:(Do rad. de \textunderscore guindar\textunderscore )}
\end{itemize}
Apparelho, para levantar ou guindar grandes pesos, na descarga de navios.
Apparelho de tirar água dos poços; burra, cegonha.
\section{Guinde}
\begin{itemize}
\item {Grp. gram.:m.}
\end{itemize}
Espécie de jarro asiático.
(Do marata)
\section{Guindola}
\begin{itemize}
\item {Grp. gram.:f.}
\end{itemize}
\begin{itemize}
\item {Proveniência:(Do rad. de \textunderscore guindar\textunderscore )}
\end{itemize}
Apparelhos provisórios de uma embarcação desmastreada; barquilha.
\section{Guiné}
\begin{itemize}
\item {Grp. gram.:f.}
\end{itemize}
\begin{itemize}
\item {Utilização:Bras}
\end{itemize}
\begin{itemize}
\item {Utilização:Prov.}
\end{itemize}
\begin{itemize}
\item {Utilização:alent.}
\end{itemize}
O mesmo que \textunderscore galinha-de-angola\textunderscore .
Lugar, onde sopra muito vento.
\section{Guineense}
\begin{itemize}
\item {Grp. gram.:m.}
\end{itemize}
Grupo de línguas indígenas da Guiné.
\section{Guineia}
\begin{itemize}
\item {Grp. gram.:f.}
\end{itemize}
\begin{itemize}
\item {Utilização:Bras}
\end{itemize}
Variedade de forragem.
\section{Guines}
\begin{itemize}
\item {Grp. gram.:m.}
\end{itemize}
\begin{itemize}
\item {Utilização:Gír.}
\end{itemize}
Cinco reis.
Dinheiro.
(Corr. de \textunderscore guinéu\textunderscore ^1)
\section{Guinéu}
\begin{itemize}
\item {Grp. gram.:m.}
\end{itemize}
\begin{itemize}
\item {Proveniência:(Do ingl. \textunderscore guinea\textunderscore )}
\end{itemize}
Moéda inglesa de oiro, que valia 21 xelins.
\section{Guinéu}
\begin{itemize}
\item {Grp. gram.:m.}
\end{itemize}
Habitante da Guiné.
\section{Guingão}
\begin{itemize}
\item {Grp. gram.:m.}
\end{itemize}
\begin{itemize}
\item {Utilização:Ant.}
\end{itemize}
Tecido fino de algodão.
Bôrra da seda.
\section{Guingau}
\begin{itemize}
\item {Grp. gram.:m.}
\end{itemize}
O mesmo que \textunderscore guingão\textunderscore .
\section{Guingueta}
\begin{itemize}
\item {fónica:guê}
\end{itemize}
\begin{itemize}
\item {Grp. gram.:f.}
\end{itemize}
\begin{itemize}
\item {Utilização:Ant.}
\end{itemize}
\begin{itemize}
\item {Proveniência:(De \textunderscore guingão\textunderscore )}
\end{itemize}
Trança de cabello, coberta com fita preta. Cf. Corvo, \textunderscore Anno na Côrte\textunderscore , c. IV, 59; Filinto, IX, 145.
\section{Guingueto}
\begin{itemize}
\item {fónica:guê}
\end{itemize}
\begin{itemize}
\item {Grp. gram.:m.}
\end{itemize}
\begin{itemize}
\item {Proveniência:(De \textunderscore guingão\textunderscore )}
\end{itemize}
Espécie de camelão ligeiro e listrado, que se fabricava em Amiens.
\section{Guinilha}
\begin{itemize}
\item {Grp. gram.:m.}
\end{itemize}
\begin{itemize}
\item {Utilização:Bras}
\end{itemize}
Cavallo, que anda pouco.
(Relaciona-se com \textunderscore guinar\textunderscore ?)
\section{Guino}
\begin{itemize}
\item {Grp. gram.:m.}
\end{itemize}
\begin{itemize}
\item {Utilização:Prov.}
\end{itemize}
\begin{itemize}
\item {Utilização:trasm.}
\end{itemize}
\begin{itemize}
\item {Utilização:Chul.}
\end{itemize}
Moéda de cinco reis.
Guines.
\section{Guinola}
\begin{itemize}
\item {Grp. gram.:f.}
\end{itemize}
\begin{itemize}
\item {Utilização:Ant.}
\end{itemize}
Mímica e dança burlesca, em que entravam Judeus e Cristãos e que se usavam principalmente em procissões e outras solennidades. Cf. B. Rebello, \textunderscore Ementas\textunderscore , 10.
\section{Guinpaguará}
\begin{itemize}
\item {Grp. gram.:m.}
\end{itemize}
Serpente da América do Sul.
\section{Guipura}
\begin{itemize}
\item {Grp. gram.:f.}
\end{itemize}
\begin{itemize}
\item {Proveniência:(Fr. \textunderscore guipure\textunderscore )}
\end{itemize}
Renda muita fina.
\section{Guipuscoano}
\begin{itemize}
\item {Grp. gram.:m.}
\end{itemize}
\begin{itemize}
\item {Grp. gram.:Pl.}
\end{itemize}
Um dos dialectos do vasconço.
Habitantes de Guipúscoa.
\section{Guirá}
\begin{itemize}
\item {Grp. gram.:f.}
\end{itemize}
Planta loranthácea do Brasil.
\section{Guiraca}
\begin{itemize}
\item {Grp. gram.:f.}
\end{itemize}
Pássaro conirostro da América.
\section{Guirantanga}
\begin{itemize}
\item {Grp. gram.:f.}
\end{itemize}
Espécie de grou do Brasil.
\section{Guiraponga}
\begin{itemize}
\item {Grp. gram.:f.}
\end{itemize}
\begin{itemize}
\item {Utilização:Bras}
\end{itemize}
O mesmo que \textunderscore araponga\textunderscore .
\section{Guiraru}
\begin{itemize}
\item {Grp. gram.:m.}
\end{itemize}
Variedade de melro do Brasil.
\section{Guiratangema}
\begin{itemize}
\item {Grp. gram.:m.}
\end{itemize}
Pássaro conirostro da América.
\section{Guirlanda}
\begin{itemize}
\item {Grp. gram.:f.}
\end{itemize}
\begin{itemize}
\item {Utilização:Náut.}
\end{itemize}
\begin{itemize}
\item {Utilização:Prov.}
\end{itemize}
\begin{itemize}
\item {Utilização:alent.}
\end{itemize}
\begin{itemize}
\item {Proveniência:(It. \textunderscore guirlanda\textunderscore )}
\end{itemize}
Anel de corda nos cabos das vêrgas.
Peças de madeira forte, para encruzar as peças verticaes e interiores da carcaça de um navio.
O mesmo que \textunderscore loiceiro\textunderscore .
\section{Guirnalda}
\begin{itemize}
\item {Grp. gram.:f.}
\end{itemize}
\begin{itemize}
\item {Proveniência:(T. cast.)}
\end{itemize}
O mesmo que \textunderscore guirlanda\textunderscore .
\section{Guisa}
\begin{itemize}
\item {Grp. gram.:f.}
\end{itemize}
\begin{itemize}
\item {Utilização:P. us.}
\end{itemize}
\begin{itemize}
\item {Grp. gram.:Loc. adv.}
\end{itemize}
\begin{itemize}
\item {Utilização:Ant.}
\end{itemize}
\begin{itemize}
\item {Proveniência:(Do ant. alt. al. \textunderscore wisa\textunderscore )}
\end{itemize}
Maneira.
Feição.
\textunderscore Á guisa\textunderscore , completamente.
Como deve sêr.
Dizia-se que um cavalleiro ou peão estava armado á guisa, quando estava perfeitamente armado, sem nada lhe faltar.
E armado \textunderscore á meia guisa\textunderscore , quando estava armado á maneira commum, mas não bem.
\section{Guisa}
\begin{itemize}
\item {Grp. gram.:f.}
\end{itemize}
\begin{itemize}
\item {Utilização:T. de Cabo-Verde}
\end{itemize}
Commemoração de um fallecimento, ao cabo de mês ou anno, reunindo-se vizinhos e amigos em casa dos enlutados, para chorar, cantar e comer.
\section{Guisadamente}
\begin{itemize}
\item {Grp. gram.:adv.}
\end{itemize}
\begin{itemize}
\item {Utilização:Ant.}
\end{itemize}
\begin{itemize}
\item {Proveniência:(De \textunderscore guisa\textunderscore ^1)}
\end{itemize}
Convenientemente; da melhor maneira.
\section{Guisado}
\begin{itemize}
\item {Grp. gram.:m.}
\end{itemize}
\begin{itemize}
\item {Proveniência:(De \textunderscore guisar\textunderscore )}
\end{itemize}
Iguaria com refogado.
\section{Guisamento}
\begin{itemize}
\item {Grp. gram.:m.}
\end{itemize}
\begin{itemize}
\item {Utilização:Des.}
\end{itemize}
\begin{itemize}
\item {Proveniência:(De \textunderscore guisar\textunderscore )}
\end{itemize}
Alfaias de igreja.
Vinho e hóstias, para a Missa.
Aprestos militares.
\section{Guisante}
\begin{itemize}
\item {Grp. gram.:f.}
\end{itemize}
\begin{itemize}
\item {Utilização:Prov.}
\end{itemize}
\begin{itemize}
\item {Utilização:alent.}
\end{itemize}
Variedade de ervilha.
\section{Guisar}
\begin{itemize}
\item {Grp. gram.:v. t.}
\end{itemize}
\begin{itemize}
\item {Utilização:Ant.}
\end{itemize}
\begin{itemize}
\item {Proveniência:(De \textunderscore guisa\textunderscore )}
\end{itemize}
Preparar com refogado.
Preparar.
Ajudar.
Dar ensejo ou aso a.
\section{Guisinho}
\begin{itemize}
\item {Grp. gram.:m.}
\end{itemize}
\begin{itemize}
\item {Utilização:Mad}
\end{itemize}
O mesmo que \textunderscore abibe\textunderscore .
\section{Guisso}
\begin{itemize}
\item {Grp. gram.:m.}
\end{itemize}
\begin{itemize}
\item {Utilização:Prov.}
\end{itemize}
\begin{itemize}
\item {Utilização:minh.}
\end{itemize}
Ponta de ramo, ou cada um dos restos miúdos que ficam da lenha no lugar onde esta esteve.
Pauzinho; graveto.
\section{Guita}
\begin{itemize}
\item {Grp. gram.:f.}
\end{itemize}
\begin{itemize}
\item {Grp. gram.:M.}
\end{itemize}
\begin{itemize}
\item {Utilização:Gír. lisb.}
\end{itemize}
\begin{itemize}
\item {Proveniência:(Do lat. \textunderscore vitta\textunderscore )}
\end{itemize}
O mesmo que \textunderscore barbante\textunderscore .
Soldado da policia ou da guarda municipal.
\section{Guitarra}
\begin{itemize}
\item {Grp. gram.:f.}
\end{itemize}
\begin{itemize}
\item {Proveniência:(It. \textunderscore chitarra\textunderscore  = lat. \textunderscore cithara\textunderscore )}
\end{itemize}
Instrumento de cordas, com um braço dividido em meios tons por filetes de metal.
\section{Guitarrada}
\begin{itemize}
\item {Grp. gram.:f.}
\end{itemize}
\begin{itemize}
\item {Utilização:Pop.}
\end{itemize}
Concêrto de guitarras.
Toque de guitarra.
\section{Guitarrão}
\begin{itemize}
\item {Grp. gram.:m.}
\end{itemize}
\begin{itemize}
\item {Utilização:Des.}
\end{itemize}
O mesmo que \textunderscore violão\textunderscore .
\section{Guitarrear}
\begin{itemize}
\item {Grp. gram.:v. i.}
\end{itemize}
\begin{itemize}
\item {Grp. gram.:V. t.}
\end{itemize}
Tocar guitarra.
Cantar ao som da guitarra. Cf. Camillo, \textunderscore Corja\textunderscore , 32 e 161.
\section{Guitarreiro}
\begin{itemize}
\item {Grp. gram.:m.}
\end{itemize}
Aquelle que faz guitarras.
Guitarrista.
\section{Guitarréu}
\begin{itemize}
\item {Grp. gram.:m.}
\end{itemize}
Espécie de guitarra.
\section{Guitarrilha}
\begin{itemize}
\item {Grp. gram.:f.}
\end{itemize}
Pequena guitarra. Cf. Junqueiro, \textunderscore M. de D. João\textunderscore , 175, 204, 208 e 211.
\section{Guitarrista}
\begin{itemize}
\item {Grp. gram.:m.}
\end{itemize}
Aquelle que toca guitarra ou ensina a tocar guitarra.
\section{Guititiroba}
\begin{itemize}
\item {Grp. gram.:f.}
\end{itemize}
Planta sapotácea do Brasil, (\textunderscore lucena rivicola\textunderscore ).
\section{Guiunga}
\begin{itemize}
\item {Grp. gram.:f.}
\end{itemize}
Árvore angolense de Caconda.
\section{Guixo}
\begin{itemize}
\item {Grp. gram.:adj.}
\end{itemize}
\begin{itemize}
\item {Utilização:Prov.}
\end{itemize}
\begin{itemize}
\item {Utilização:trasm.}
\end{itemize}
\begin{itemize}
\item {Utilização:minh.}
\end{itemize}
\begin{itemize}
\item {Utilização:Prov.}
\end{itemize}
\begin{itemize}
\item {Utilização:trasm.}
\end{itemize}
Muito vivo, buliçoso, (falando-se de crianças ou de certos animaes, como o rato).
Seguro, direitinho, viçoso, (falando-se de um vegetal, transplantado de há pouco).
(Cp. \textunderscore guisso\textunderscore )
\section{Guizalhada}
\begin{itemize}
\item {Grp. gram.:f.}
\end{itemize}
\begin{itemize}
\item {Proveniência:(De \textunderscore guizalhar\textunderscore )}
\end{itemize}
Som continuado de guizos. Cf. Th. Ribeiro, \textunderscore Jornadas\textunderscore , II, 109.
\section{Guizalhar}
\begin{itemize}
\item {Grp. gram.:v. i.}
\end{itemize}
Agitar guizos, fazendo-os soar.
\section{Guizeira}
\begin{itemize}
\item {Grp. gram.:f.}
\end{itemize}
\begin{itemize}
\item {Proveniência:(De \textunderscore guizo\textunderscore )}
\end{itemize}
Correia, a que se prendem os guizos, em volta do pescoço do animal.
\section{Guizo}
\begin{itemize}
\item {Grp. gram.:m.}
\end{itemize}
\begin{itemize}
\item {Grp. gram.:Pl.}
\end{itemize}
\begin{itemize}
\item {Utilização:Prov.}
\end{itemize}
\begin{itemize}
\item {Utilização:dur.}
\end{itemize}
\begin{itemize}
\item {Utilização:Pop.}
\end{itemize}
Pequeno globo de metal, que produz ruído, ao agitar-se com os pequeninos corpos que contém.
Testículos.
\section{Gujarás}
\begin{itemize}
\item {Grp. gram.:m. pl.}
\end{itemize}
Tribo de aborígenes do Pará.
\section{Gula}
\begin{itemize}
\item {Grp. gram.:f.}
\end{itemize}
\begin{itemize}
\item {Utilização:Ant.}
\end{itemize}
\begin{itemize}
\item {Proveniência:(Lat. \textunderscore gula\textunderscore )}
\end{itemize}
O mesmo que \textunderscore glutonaria\textunderscore .
Grande amor a bôas iguarias.
Gulodice.
Moldura, em fórma de S, na cornija ou cimalha.
Espécie de plaina.
Goéla.
\section{Gula-mocha}
\begin{itemize}
\item {Grp. gram.:f.}
\end{itemize}
Cepo de carpintaria, com que se fazem molduras, em fórma da gula sem filete.
\section{Gulandim}
\begin{itemize}
\item {Grp. gram.:f.}
\end{itemize}
Nome de várias árvores gutíferas do Brasil.
\section{Gulapa}
\begin{itemize}
\item {Grp. gram.:f.}
\end{itemize}
\begin{itemize}
\item {Utilização:Prov.}
\end{itemize}
\begin{itemize}
\item {Utilização:minh.}
\end{itemize}
Gulodice.
Affeição a bons pitéus.
(Cp. \textunderscore gula\textunderscore )
\section{Gulapão}
\begin{itemize}
\item {Grp. gram.:adj.}
\end{itemize}
\begin{itemize}
\item {Utilização:Prov.}
\end{itemize}
\begin{itemize}
\item {Utilização:minh.}
\end{itemize}
\begin{itemize}
\item {Proveniência:(De \textunderscore gulapa\textunderscore )}
\end{itemize}
O mesmo que \textunderscore guloso\textunderscore .
\section{Gulazar}
\begin{itemize}
\item {Grp. gram.:v. i.}
\end{itemize}
\begin{itemize}
\item {Utilização:Prov.}
\end{itemize}
\begin{itemize}
\item {Utilização:minh.}
\end{itemize}
\begin{itemize}
\item {Proveniência:(De \textunderscore gula\textunderscore )}
\end{itemize}
Comer gulosamente.
Sêr amigo de bons bocados.
\section{Guleima}
\begin{itemize}
\item {Grp. gram.:m.}
\end{itemize}
\begin{itemize}
\item {Utilização:Burl.}
\end{itemize}
\begin{itemize}
\item {Proveniência:(Do rad. de \textunderscore gula\textunderscore )}
\end{itemize}
O mesmo que \textunderscore comilão\textunderscore .
\section{Guleimar}
\begin{itemize}
\item {Grp. gram.:v. i.}
\end{itemize}
\begin{itemize}
\item {Utilização:Prov.}
\end{itemize}
\begin{itemize}
\item {Utilização:trasm.}
\end{itemize}
\begin{itemize}
\item {Proveniência:(De \textunderscore guleima\textunderscore )}
\end{itemize}
Comer e beber muito.
\section{Gulherite}
\begin{itemize}
\item {Grp. gram.:m.}
\end{itemize}
\begin{itemize}
\item {Utilização:Prov.}
\end{itemize}
\begin{itemize}
\item {Utilização:trasm.}
\end{itemize}
Caspacho.
Qualquer iguaria simples ou feita á pressa.
(Cp. \textunderscore gula\textunderscore )
\section{Gulheriteiro}
\begin{itemize}
\item {Grp. gram.:adj.}
\end{itemize}
Que anda sempre a cuidar de gulherites.
\section{Gulheritice}
\begin{itemize}
\item {Grp. gram.:f.}
\end{itemize}
\begin{itemize}
\item {Utilização:Prov.}
\end{itemize}
\begin{itemize}
\item {Utilização:trasm.}
\end{itemize}
Gulherite.
Disposição para gulheriteiro; gulodice.
\section{Gulodice}
\begin{itemize}
\item {Grp. gram.:f.}
\end{itemize}
O mesmo que \textunderscore gulosice\textunderscore .
Doce ou qualquer iguaria muito appetitosa.
\section{Gulosa}
\begin{itemize}
\item {Grp. gram.:f.}
\end{itemize}
\begin{itemize}
\item {Utilização:Prov.}
\end{itemize}
\begin{itemize}
\item {Utilização:trasm.}
\end{itemize}
\begin{itemize}
\item {Proveniência:(De \textunderscore guloso\textunderscore )}
\end{itemize}
Vara comprida, rachada e aberta no topo, com que se alcançam e colhem na árvore os frutos; ladra.
\section{Gulosar}
\begin{itemize}
\item {Grp. gram.:v. i.}
\end{itemize}
\begin{itemize}
\item {Proveniência:(De \textunderscore guloso\textunderscore )}
\end{itemize}
Comer gulosices.
Debicar na comida, comer pouco de várias coisas.
\section{Guloseima}
\begin{itemize}
\item {Grp. gram.:f.}
\end{itemize}
\begin{itemize}
\item {Proveniência:(De \textunderscore guloso\textunderscore )}
\end{itemize}
Gula, vício de comer muito.
Predilecção das coisas doces ou de iguarias muito escolhidas.
Manjar doce, delicado ou muito saboroso.
\section{Gulosice}
\begin{itemize}
\item {Grp. gram.:f.}
\end{itemize}
\begin{itemize}
\item {Proveniência:(De \textunderscore guloso\textunderscore )}
\end{itemize}
Gula, vício de comer muito.
Predilecção das coisas doces ou de iguarias muito escolhidas.
Manjar doce, delicado ou muito saboroso.
\section{Gulosidade}
\begin{itemize}
\item {Grp. gram.:f.}
\end{itemize}
\begin{itemize}
\item {Utilização:Prov.}
\end{itemize}
\begin{itemize}
\item {Utilização:alg.}
\end{itemize}
O mesmo que \textunderscore gulosice\textunderscore .
\section{Gulosina}
\begin{itemize}
\item {Grp. gram.:f.}
\end{itemize}
(V.gulosice)
\section{Gulosinar}
\begin{itemize}
\item {Grp. gram.:v. i.}
\end{itemize}
O mesmo que \textunderscore gulosar\textunderscore ; lambujar.
\section{Gulosinha}
\begin{itemize}
\item {Grp. gram.:f.}
\end{itemize}
Casta de azeitona.
\section{Guloso}
\begin{itemize}
\item {Grp. gram.:m.  e  adj.}
\end{itemize}
\begin{itemize}
\item {Grp. gram.:M.}
\end{itemize}
\begin{itemize}
\item {Proveniência:(De \textunderscore gula\textunderscore )}
\end{itemize}
O que gosta de gulosices.
O que tem o vício da gula.
Peixe do norte do Brasil.
\section{Gume}
\begin{itemize}
\item {Grp. gram.:m.}
\end{itemize}
\begin{itemize}
\item {Utilização:Fig.}
\end{itemize}
\begin{itemize}
\item {Proveniência:(Do lat. \textunderscore acumen\textunderscore )}
\end{itemize}
O lado afiado de um instrumento cortante.
Perspicácia, agudeza.
\section{Gúmena}
\begin{itemize}
\item {Grp. gram.:f.}
\end{itemize}
Calabre da embarcação.
(Cast. \textunderscore gúmena\textunderscore )
\section{Gumífero}
\begin{itemize}
\item {Grp. gram.:adj.}
\end{itemize}
\begin{itemize}
\item {Proveniência:(Do lat. \textunderscore gummi\textunderscore  + \textunderscore ferre\textunderscore )}
\end{itemize}
Que produz goma.
\section{Gummífero}
\begin{itemize}
\item {Grp. gram.:adj.}
\end{itemize}
\begin{itemize}
\item {Proveniência:(Do lat. \textunderscore gummi\textunderscore  + \textunderscore ferre\textunderscore )}
\end{itemize}
Que produz goma.
\section{Gumoso}
\begin{itemize}
\item {Grp. gram.:adj.}
\end{itemize}
Que tem gume. Cf. F. Lapa, \textunderscore Proc. de Vinif.\textunderscore , 27.
\section{Guna}
\begin{itemize}
\item {Grp. gram.:f.}
\end{itemize}
Planta trepadeira da ilha de San-Thomé.
\section{Gunchelim}
\begin{itemize}
\item {Grp. gram.:m.}
\end{itemize}
Planta hortense de Dio.
\section{Guncho}
\begin{itemize}
\item {Grp. gram.:m.}
\end{itemize}
Nome, que se deu a uma ave que se encontrava ou se encontra na lagôa de Obidos.
\section{Gunda}
\begin{itemize}
\item {Grp. gram.:f.}
\end{itemize}
Árvore africana, applicável a construcções.
\section{Gunda-rupsa}
\begin{itemize}
\item {Grp. gram.:f.}
\end{itemize}
Arbusto medicinal de Moçambique.
\section{Gundeiro}
\begin{itemize}
\item {Grp. gram.:m.}
\end{itemize}
Árvore de Dio, de cuja fruta se extrai excellente goma.
\section{Gúndia}
\begin{itemize}
\item {Grp. gram.:f.}
\end{itemize}
Pequeno barco asiático.
\section{Gundra}
\begin{itemize}
\item {Grp. gram.:f.}
\end{itemize}
(V.gúndia)
\section{Gundu}
\begin{itemize}
\item {Grp. gram.:m.}
\end{itemize}
Erva medicinal da ilha de San-Thomé.
\section{Gune}
\begin{itemize}
\item {Grp. gram.:m.}
\end{itemize}
Substância filamentosa, de que na Índia se fabríca um pano grosseiro.
\section{Gunello}
\begin{itemize}
\item {Grp. gram.:m.}
\end{itemize}
Peixe gobioide do Mediterrâneo, (\textunderscore blennius gunnellus\textunderscore ).
\section{Gunelo}
\begin{itemize}
\item {Grp. gram.:m.}
\end{itemize}
Peixe gobioide do Mediterrâneo, (\textunderscore blennius gunnellus\textunderscore ).
\section{Gúnera}
\begin{itemize}
\item {Grp. gram.:f.}
\end{itemize}
\begin{itemize}
\item {Proveniência:(De \textunderscore Guner\textunderscore , n. p.)}
\end{itemize}
Gênero de plantas de jardim.
\section{Guneráceas}
\begin{itemize}
\item {Grp. gram.:f. pl.}
\end{itemize}
\begin{itemize}
\item {Proveniência:(De \textunderscore gúnera\textunderscore )}
\end{itemize}
Grupo de plantas, da fam. das urticáceas.
\section{Gunfar}
\begin{itemize}
\item {Grp. gram.:v. i.}
\end{itemize}
\begin{itemize}
\item {Utilização:Prov.}
\end{itemize}
\begin{itemize}
\item {Utilização:beir.}
\end{itemize}
O mesmo que \textunderscore choramigar\textunderscore . (Colhido no Fundão)
(Methath. de \textunderscore fungar\textunderscore )
\section{Gunga}
\begin{itemize}
\item {Grp. gram.:f.}
\end{itemize}
Ruminante de Angola.
\section{Gungieiro}
\begin{itemize}
\item {Grp. gram.:m.}
\end{itemize}
Planta venenosa de Dio, (\textunderscore abrus praecatorius\textunderscore ).
\section{Gungongila}
\begin{itemize}
\item {Grp. gram.:m.}
\end{itemize}
Grande ave africana, de plumagem, pelle e carne escuras.
\section{Gunigobó}
\begin{itemize}
\item {Grp. gram.:m.}
\end{itemize}
Árvore santhomense, de propriedades medicinaes.
\section{Gupiara}
\begin{itemize}
\item {Grp. gram.:f.}
\end{itemize}
\begin{itemize}
\item {Utilização:Bras}
\end{itemize}
O mesmo que \textunderscore gopiara\textunderscore .
\section{Gura}
\begin{itemize}
\item {Grp. gram.:f.}
\end{itemize}
Instrumento musical dos Hotentotes.
Barrete veneziano.
Ave da Nova-Guiné.
\section{Guraçaim}
\begin{itemize}
\item {Grp. gram.:m.}
\end{itemize}
\begin{itemize}
\item {Utilização:Bras. do N}
\end{itemize}
Nome de um peixe.
\section{Gurami}
\begin{itemize}
\item {Grp. gram.:m.}
\end{itemize}
Grande peixe, muito saboroso, originário dos mares da China e da Sonda, (\textunderscore osphronemus olfax\textunderscore ), cuja aclimação na Europa se está tentando.
\section{Gurandirana}
\begin{itemize}
\item {Grp. gram.:f.}
\end{itemize}
\begin{itemize}
\item {Utilização:Bras}
\end{itemize}
O mesmo que \textunderscore guanandirana\textunderscore .
\section{Guraputepoca}
\begin{itemize}
\item {Grp. gram.:f.}
\end{itemize}
\begin{itemize}
\item {Utilização:Bras}
\end{itemize}
Espécie de ave.
\section{Guratan}
\begin{itemize}
\item {Grp. gram.:m.}
\end{itemize}
\begin{itemize}
\item {Utilização:Bras}
\end{itemize}
Árvore silvestre.
\section{Guraúra}
\begin{itemize}
\item {Grp. gram.:f.}
\end{itemize}
\begin{itemize}
\item {Utilização:Bras}
\end{itemize}
Árvore silvestre.
\section{Gurejuba}
\begin{itemize}
\item {Grp. gram.:f.}
\end{itemize}
\begin{itemize}
\item {Utilização:Bras}
\end{itemize}
Grande peixe, de cujo bucho se faz excellente colla.
\section{Gurgau}
\begin{itemize}
\item {Grp. gram.:m.}
\end{itemize}
\begin{itemize}
\item {Utilização:Prov.}
\end{itemize}
Seixo rolado do fundo dos rios.
Brita para estradas.
\section{Gurguez}
\begin{itemize}
\item {Grp. gram.:m.}
\end{itemize}
O mesmo que \textunderscore gorguz\textunderscore .
\section{Gurguri}
\begin{itemize}
\item {Grp. gram.:m.}
\end{itemize}
Espécie de narguilhé, usado por Baneanes e Moiros da África oriental.
\section{Guri}
\begin{itemize}
\item {Grp. gram.:m.}
\end{itemize}
\begin{itemize}
\item {Utilização:Bras. do S}
\end{itemize}
\begin{itemize}
\item {Utilização:Bras. do Rio}
\end{itemize}
\begin{itemize}
\item {Proveniência:(T. tupi)}
\end{itemize}
O mesmo que \textunderscore criança\textunderscore .
Bagre pequeno.
\section{Guriba}
\begin{itemize}
\item {Grp. gram.:m. ,  f.  e  adj.}
\end{itemize}
\begin{itemize}
\item {Utilização:Bras. do Rio}
\end{itemize}
Diz-se da ave que tem as pennas arrepiadas.
\section{Guri-guri!}
\begin{itemize}
\item {Grp. gram.:interj.}
\end{itemize}
\begin{itemize}
\item {Utilização:Prov.}
\end{itemize}
\begin{itemize}
\item {Utilização:minh.}
\end{itemize}
(Serve para chamar os porcos)
\section{Gurijuba}
\begin{itemize}
\item {Grp. gram.:m.}
\end{itemize}
O mesmo que \textunderscore gurejuba\textunderscore .
\section{Gurindiba}
\begin{itemize}
\item {Grp. gram.:f.}
\end{itemize}
Planta do Brasil, (\textunderscore traganum scariosus\textunderscore ).
\section{Guriri}
\begin{itemize}
\item {Grp. gram.:m.}
\end{itemize}
\begin{itemize}
\item {Utilização:Bras}
\end{itemize}
\begin{itemize}
\item {Proveniência:(T. tupi)}
\end{itemize}
Espécie de palmeira.
\section{Gurita}
\begin{itemize}
\item {Grp. gram.:f.}
\end{itemize}
\begin{itemize}
\item {Utilização:Bras. do sertão}
\end{itemize}
Égua velha.
\section{Gurita}
\begin{itemize}
\item {Grp. gram.:f.}
\end{itemize}
\begin{itemize}
\item {Utilização:Pop.}
\end{itemize}
\begin{itemize}
\item {Utilização:Prov.}
\end{itemize}
\begin{itemize}
\item {Utilização:alent.}
\end{itemize}
O mesmo que \textunderscore guarita\textunderscore .
Marco da triangulação geodésica.
\section{Guriteiro}
\begin{itemize}
\item {Grp. gram.:m.}
\end{itemize}
\begin{itemize}
\item {Proveniência:(De \textunderscore gurita\textunderscore ^2)}
\end{itemize}
Tabulageiro; aquelle que tem casa de jôgo. Cf. Tolentino, \textunderscore Bilhar\textunderscore .
\section{Gurma}
\begin{itemize}
\item {Grp. gram.:f.}
\end{itemize}
\begin{itemize}
\item {Proveniência:(Fr. \textunderscore gourme\textunderscore )}
\end{itemize}
Doença dos potros, durante a dentição. Cp. \textunderscore gosma\textunderscore .
\section{Gurrião}
\begin{itemize}
\item {Grp. gram.:m.}
\end{itemize}
\begin{itemize}
\item {Utilização:Prov.}
\end{itemize}
O mesmo que \textunderscore pardal\textunderscore . Cf. Ed. Sequeira, \textunderscore Ovos e Ninhos\textunderscore .
\section{Gurubu}
\begin{itemize}
\item {Grp. gram.:m.}
\end{itemize}
\begin{itemize}
\item {Utilização:Bras}
\end{itemize}
Árvore silvestre, terebinthácea, que serve em carpintaria e de que se extrai tinta roxa.
\section{Gimnandro}
\begin{itemize}
\item {Grp. gram.:adj.}
\end{itemize}
\begin{itemize}
\item {Utilização:Bot.}
\end{itemize}
\begin{itemize}
\item {Proveniência:(Do gr. \textunderscore gumnos\textunderscore  + \textunderscore aner\textunderscore )}
\end{itemize}
Que tem os estames nus.
\section{Gimnanto}
\begin{itemize}
\item {Grp. gram.:adj.}
\end{itemize}
\begin{itemize}
\item {Utilização:Bot.}
\end{itemize}
\begin{itemize}
\item {Proveniência:(Do gr. \textunderscore gumnos\textunderscore  + \textunderscore anthos\textunderscore )}
\end{itemize}
Cujas flôres não têm invólucro algum.
\section{Gimnasial}
\begin{itemize}
\item {Grp. gram.:adj.}
\end{itemize}
Relativo a gimnásio.
\section{Gimnasiarca}
\begin{itemize}
\item {Grp. gram.:m.}
\end{itemize}
\begin{itemize}
\item {Proveniência:(Gr. \textunderscore gumnasiarkhes\textunderscore )}
\end{itemize}
Chefe ou director de exercícios gimnásticos, entre os antigos.
\section{Gimnásio}
\begin{itemize}
\item {Grp. gram.:m.}
\end{itemize}
\begin{itemize}
\item {Proveniência:(Gr. \textunderscore gumnasion\textunderscore )}
\end{itemize}
Lugar, em que se pratíca a gimnástica.
Estabelecimento de ensino secundário na Alemanha.
\section{Gimnasta}
\begin{itemize}
\item {Grp. gram.:m.}
\end{itemize}
\begin{itemize}
\item {Proveniência:(Gr. \textunderscore gumnastes\textunderscore )}
\end{itemize}
Aquele que pratíca a gimnástica.
Aquele que é hábil em gimnástica; acrobata.
\section{Gimnaste}
\begin{itemize}
\item {Grp. gram.:m.}
\end{itemize}
O mesmo que \textunderscore gimnasta\textunderscore .
\section{Gimnástica}
\begin{itemize}
\item {Grp. gram.:f.}
\end{itemize}
\begin{itemize}
\item {Utilização:Fig.}
\end{itemize}
\begin{itemize}
\item {Proveniência:(De \textunderscore gimnástico\textunderscore )}
\end{itemize}
Arte ou acto de exercitar o corpo para o fortificar.
Exercício de discorrer.
\section{Gimnástico}
\begin{itemize}
\item {Grp. gram.:adj.}
\end{itemize}
\begin{itemize}
\item {Proveniência:(Gr. \textunderscore gumnastikos\textunderscore )}
\end{itemize}
Relativo a gimnástica.
\section{Gimnetos}
\begin{itemize}
\item {Grp. gram.:m. pl.}
\end{itemize}
\begin{itemize}
\item {Proveniência:(Do gr. \textunderscore gumnos\textunderscore , nu)}
\end{itemize}
Nome, que, em Argos, se dava aos escravos, por andarem mal vestidos ou quási nus.
\section{Gimnetros}
\begin{itemize}
\item {Grp. gram.:m. pl.}
\end{itemize}
\begin{itemize}
\item {Proveniência:(Do gr. \textunderscore gumnos\textunderscore  + \textunderscore etron\textunderscore )}
\end{itemize}
Gênero de peixes acantopterígios.
\section{Gímnico}
\begin{itemize}
\item {Grp. gram.:adj.}
\end{itemize}
O mesmo que \textunderscore gimnástico\textunderscore .
\section{Gimno...}
\begin{itemize}
\item {Proveniência:(Do gr. \textunderscore gumnos\textunderscore )}
\end{itemize}
Elemento, que entra na formação de várias palavras, com a significação de \textunderscore nu\textunderscore .
\section{Gimnoblasto}
\begin{itemize}
\item {Grp. gram.:adj.}
\end{itemize}
\begin{itemize}
\item {Utilização:Bot.}
\end{itemize}
\begin{itemize}
\item {Proveniência:(Do gr. \textunderscore gumnos\textunderscore  + \textunderscore blastos\textunderscore )}
\end{itemize}
Diz-se da planta, que não tem o embrião contido em cavidade particular.
\section{Gimnocarpo}
\begin{itemize}
\item {Grp. gram.:adj.}
\end{itemize}
\begin{itemize}
\item {Utilização:Bot.}
\end{itemize}
\begin{itemize}
\item {Proveniência:(Do gr. \textunderscore gumnos\textunderscore  + \textunderscore karpos\textunderscore )}
\end{itemize}
Diz-se dos frutos descobertos, que não são soldados com algum órgão acessório.
\section{Gimnocaule}
\begin{itemize}
\item {Grp. gram.:adj.}
\end{itemize}
\begin{itemize}
\item {Utilização:Bot.}
\end{itemize}
\begin{itemize}
\item {Proveniência:(Do gr. \textunderscore gumnos\textunderscore  + \textunderscore kaulos\textunderscore )}
\end{itemize}
Que tem a haste nua de fôlhas.
\section{Gimnocéfalo}
\begin{itemize}
\item {Grp. gram.:adj.}
\end{itemize}
\begin{itemize}
\item {Utilização:Zool.}
\end{itemize}
\begin{itemize}
\item {Proveniência:(Do gr. \textunderscore gumnos\textunderscore  + \textunderscore kephale\textunderscore )}
\end{itemize}
Que tem a cabeça nua, sem pêlos ou sem penas.
\section{Gimnoclado}
\begin{itemize}
\item {Grp. gram.:m.}
\end{itemize}
\begin{itemize}
\item {Proveniência:(Do gr. \textunderscore gumnos\textunderscore  + \textunderscore klados\textunderscore )}
\end{itemize}
Gênero de plantas leguminosas.
\section{Gimnodermo}
\begin{itemize}
\item {Grp. gram.:adj.}
\end{itemize}
\begin{itemize}
\item {Utilização:Zool.}
\end{itemize}
\begin{itemize}
\item {Proveniência:(Do gr. \textunderscore gumnos\textunderscore  + \textunderscore derma\textunderscore )}
\end{itemize}
Que tem pele nua.
\section{Gimnodonte}
\begin{itemize}
\item {Grp. gram.:adj.}
\end{itemize}
\begin{itemize}
\item {Utilização:Zool.}
\end{itemize}
\begin{itemize}
\item {Proveniência:(Do gr. \textunderscore gumnos\textunderscore  + \textunderscore odous\textunderscore )}
\end{itemize}
Que tem os dentes á vista.
\section{Gimnofídio}
\begin{itemize}
\item {Grp. gram.:adj.}
\end{itemize}
\begin{itemize}
\item {Proveniência:(Do gr. \textunderscore gumnos\textunderscore  + \textunderscore ophis\textunderscore )}
\end{itemize}
Diz-se das serpentes, que têm a pele nua, lisa e viscosa.
\section{Gimnógino}
\begin{itemize}
\item {Grp. gram.:adj.}
\end{itemize}
\begin{itemize}
\item {Utilização:Bot.}
\end{itemize}
\begin{itemize}
\item {Proveniência:(Do gr. \textunderscore gumnos\textunderscore  + \textunderscore gune\textunderscore )}
\end{itemize}
Que tem o ovário nu.
\section{Gimnogonfos}
\begin{itemize}
\item {Grp. gram.:m. pl.}
\end{itemize}
\begin{itemize}
\item {Proveniência:(Do gr. \textunderscore gumnos\textunderscore  + \textunderscore gomphos\textunderscore )}
\end{itemize}
Animálculos infusórios, cujos dentes se ligam á maxila só pela base.
\section{Gimnograma}
\begin{itemize}
\item {Grp. gram.:m.}
\end{itemize}
\begin{itemize}
\item {Proveniência:(Do gr. \textunderscore gumnos\textunderscore  + \textunderscore gramma\textunderscore )}
\end{itemize}
Gênero de fêtos.
\section{Gimnopedia}
\begin{itemize}
\item {Grp. gram.:f.}
\end{itemize}
\begin{itemize}
\item {Proveniência:(Gr. \textunderscore gumnopaidia\textunderscore )}
\end{itemize}
Antiga dança espartana, executada por homens e crianças nuas, em certa festa anual e ao som de hinos compostos para êsse efeito.
\section{Gimnópode}
\begin{itemize}
\item {Grp. gram.:adj.}
\end{itemize}
\begin{itemize}
\item {Grp. gram.:M. pl.}
\end{itemize}
\begin{itemize}
\item {Proveniência:(Do gr. \textunderscore gumnos\textunderscore  + \textunderscore pous\textunderscore , \textunderscore podos\textunderscore )}
\end{itemize}
Que tem os pés nus.
Família de reptis.
\section{Gimnópomo}
\begin{itemize}
\item {Grp. gram.:adj.}
\end{itemize}
\begin{itemize}
\item {Utilização:Ichthyol.}
\end{itemize}
\begin{itemize}
\item {Proveniência:(Do gr. \textunderscore gumnos\textunderscore  + \textunderscore poma\textunderscore )}
\end{itemize}
Que tem os opérculos nus.
\section{Gimnóptero}
\begin{itemize}
\item {Grp. gram.:adj.}
\end{itemize}
\begin{itemize}
\item {Utilização:Zool.}
\end{itemize}
\begin{itemize}
\item {Grp. gram.:M. pl.}
\end{itemize}
\begin{itemize}
\item {Proveniência:(Do gr. \textunderscore gumnos\textunderscore  + \textunderscore pteron\textunderscore )}
\end{itemize}
Que tem as asas nuas, sem escamas.
Secção da classe dos insectos, que compreende os que têm asas lisas, sem elitros nem escamas farináceas.
\section{Gimnosofista}
\begin{itemize}
\item {Grp. gram.:m.}
\end{itemize}
\begin{itemize}
\item {Proveniência:(Gr. \textunderscore gumnosophistes\textunderscore )}
\end{itemize}
Filósopho indiano, que se abstinha de carnes e se dedicava á contemplação mística.
\section{Gimnospermas}
\begin{itemize}
\item {Grp. gram.:f. pl.}
\end{itemize}
\begin{itemize}
\item {Proveniência:(Do gr. \textunderscore gumnos\textunderscore  + \textunderscore sperma\textunderscore )}
\end{itemize}
Plantas dicotiledóneas, cujas flôres têm quatro grãos nus ao fundo do cálice.
\section{Gimnospermia}
\begin{itemize}
\item {Grp. gram.:f.}
\end{itemize}
Ordem das plantas gimnospermas.
\section{Gimnospérmico}
\begin{itemize}
\item {Grp. gram.:adj.}
\end{itemize}
O mesmo que \textunderscore gimnospermo\textunderscore .
\section{Gimnospermo}
\begin{itemize}
\item {Grp. gram.:adj.}
\end{itemize}
Relativo á gimnospermia.
\section{Gimnospódia}
\begin{itemize}
\item {Grp. gram.:f.}
\end{itemize}
\begin{itemize}
\item {Proveniência:(Do gr. \textunderscore gumnos\textunderscore , nu, + \textunderscore pous\textunderscore , \textunderscore podos\textunderscore , pé)}
\end{itemize}
Dança e música, executada por pessôas descalças?:«\textunderscore ...a gimnospódia, que se canta e dança...\textunderscore »\textunderscore Viriato Trág.\textunderscore , XI, 44.
\section{Gimnosporado}
\begin{itemize}
\item {Grp. gram.:adj.}
\end{itemize}
\begin{itemize}
\item {Utilização:Bot.}
\end{itemize}
\begin{itemize}
\item {Proveniência:(Do gr. \textunderscore gumnos\textunderscore  + \textunderscore spora\textunderscore )}
\end{itemize}
Que tem os poros livres em cavidade comum, pela absorpção da teca.
\section{Gimnósporo}
\begin{itemize}
\item {Grp. gram.:adj.}
\end{itemize}
O mesmo que \textunderscore gimnosporado\textunderscore .
\section{Gimnossofia}
\begin{itemize}
\item {Grp. gram.:f.}
\end{itemize}
Doutrina dos gimnosofistas.
\section{Gimnossomo}
\begin{itemize}
\item {Grp. gram.:adj.}
\end{itemize}
\begin{itemize}
\item {Utilização:Zool.}
\end{itemize}
\begin{itemize}
\item {Proveniência:(Do gr. \textunderscore gumnos\textunderscore  + \textunderscore soma\textunderscore )}
\end{itemize}
Que tem o corpo nu.
\section{Gimnóstomo}
\begin{itemize}
\item {Grp. gram.:adj.}
\end{itemize}
\begin{itemize}
\item {Utilização:Bot.}
\end{itemize}
\begin{itemize}
\item {Proveniência:(Do gr. \textunderscore gumnos\textunderscore  + \textunderscore stoma\textunderscore )}
\end{itemize}
Cuja bôca não tem apendices.
\section{Gimnotetraspermo}
\begin{itemize}
\item {Grp. gram.:adj.}
\end{itemize}
\begin{itemize}
\item {Utilização:Bot.}
\end{itemize}
\begin{itemize}
\item {Proveniência:(Do gr. \textunderscore gumnos\textunderscore  + \textunderscore tetra\textunderscore  + \textunderscore sperma\textunderscore )}
\end{itemize}
Diz-se da planta, cujo ovário é profundamente dividido em quatro partes, imitando quatro grãos nus no fundo do cálice.
\section{Gimnoto}
\begin{itemize}
\item {Grp. gram.:m.}
\end{itemize}
\begin{itemize}
\item {Proveniência:(Do gr. \textunderscore gumnos\textunderscore  + \textunderscore notos\textunderscore )}
\end{itemize}
Gênero de peixes, da fam. das enguias.
\section{Gimnuro}
\begin{itemize}
\item {Grp. gram.:adj.}
\end{itemize}
\begin{itemize}
\item {Utilização:Zool.}
\end{itemize}
\begin{itemize}
\item {Grp. gram.:M. pl.}
\end{itemize}
\begin{itemize}
\item {Proveniência:(Do gr. \textunderscore gumnos\textunderscore  + \textunderscore oura\textunderscore )}
\end{itemize}
Que tem a cauda nua.
Secção da fam. dos macacos, que compreende os sapajus de cauda nua e calosa.
\section{Ginandria}
\begin{itemize}
\item {Grp. gram.:f.}
\end{itemize}
\begin{itemize}
\item {Proveniência:(Do gr. \textunderscore gune\textunderscore  + \textunderscore aner\textunderscore )}
\end{itemize}
Classe de vegetaes, cujos estames estão insertos nos pistilos, (segundo o systema de Lin.).
\section{Ginantropo}
\begin{itemize}
\item {Grp. gram.:m.}
\end{itemize}
\begin{itemize}
\item {Proveniência:(Do gr. \textunderscore gune\textunderscore  + \textunderscore anthropos\textunderscore )}
\end{itemize}
Hermafrodita, que participa mais das qualidades físicas da mulher, do que das do homem.
\section{Ginásio}
\begin{itemize}
\item {Grp. gram.:m.}
\end{itemize}
\begin{itemize}
\item {Proveniência:(Gr. \textunderscore gumnasion\textunderscore )}
\end{itemize}
Lugar, em que se pratíca a ginástica.
Estabelecimento de ensino secundário na Alemanha.
\section{Ginasta}
\begin{itemize}
\item {Grp. gram.:m.}
\end{itemize}
\begin{itemize}
\item {Proveniência:(Gr. \textunderscore gumnastes\textunderscore )}
\end{itemize}
Aquele que pratíca a ginástica.
Aquele que é hábil em ginástica; acrobata.
\section{Ginaste}
\begin{itemize}
\item {Grp. gram.:m.}
\end{itemize}
O mesmo que \textunderscore ginasta\textunderscore .
\section{Ginástica}
\begin{itemize}
\item {Grp. gram.:f.}
\end{itemize}
\begin{itemize}
\item {Utilização:Fig.}
\end{itemize}
\begin{itemize}
\item {Proveniência:(De \textunderscore ginástico\textunderscore )}
\end{itemize}
Arte ou acto de exercitar o corpo para o fortificar.
Exercício de discorrer.
\section{Ginástico}
\begin{itemize}
\item {Grp. gram.:adj.}
\end{itemize}
\begin{itemize}
\item {Proveniência:(Gr. \textunderscore gumnastikos\textunderscore )}
\end{itemize}
Relativo a ginástica.
\section{Gineceu}
\begin{itemize}
\item {Grp. gram.:m.}
\end{itemize}
\begin{itemize}
\item {Utilização:Bot.}
\end{itemize}
\begin{itemize}
\item {Proveniência:(Gr. \textunderscore gunaikeion\textunderscore )}
\end{itemize}
Na antiguidade, aposento de mulheres.
Na Idade-Média, manufactura, onde os senhores obrigavam as vassalas a trabalhar em lan ou seda.
Conjunto dos pistilos ou dos órgãos femininos de uma flôr.
\section{Gineco...}
\begin{itemize}
\item {Proveniência:(Do gr. \textunderscore gune\textunderscore , \textunderscore gunaikos\textunderscore )}
\end{itemize}
Elemento, que entra na formação de várias palavras, com a significação de \textunderscore mulher\textunderscore  ou \textunderscore feminino\textunderscore .
\section{Ginecocracia}
\begin{itemize}
\item {Grp. gram.:f.}
\end{itemize}
\begin{itemize}
\item {Proveniência:(Do gr. \textunderscore gunaikos\textunderscore  + \textunderscore kratein\textunderscore )}
\end{itemize}
Preponderância das mulheres na governação pública.
\section{Ginecocrata}
\begin{itemize}
\item {Grp. gram.:m.}
\end{itemize}
Partidário da ginecocracia.
\section{Ginecocrático}
\begin{itemize}
\item {Grp. gram.:adj.}
\end{itemize}
Relativo á ginecocracia.
\section{Ginecografia}
\begin{itemize}
\item {Grp. gram.:f.}
\end{itemize}
\begin{itemize}
\item {Proveniência:(Do gr. \textunderscore gunaikos\textunderscore  + \textunderscore graphein\textunderscore )}
\end{itemize}
O mesmo que \textunderscore ginecologia\textunderscore .
\section{Ginecográfico}
\begin{itemize}
\item {Grp. gram.:adj.}
\end{itemize}
Relativo á ginecografia.
\section{Ginecologia}
\begin{itemize}
\item {Grp. gram.:f.}
\end{itemize}
\begin{itemize}
\item {Proveniência:(Do gr. \textunderscore gunaikos\textunderscore  + \textunderscore logos\textunderscore )}
\end{itemize}
Tratado á cêrca das mulheres.
Terapêutica das doenças peculiares ás mulheres.
\section{Ginecológico}
\begin{itemize}
\item {Grp. gram.:adj.}
\end{itemize}
Relativo á ginecologia.
\section{Ginecologista}
\begin{itemize}
\item {Grp. gram.:m.}
\end{itemize}
Tratadista de ginecologia.
\section{Ginecólogo}
\begin{itemize}
\item {Grp. gram.:m.}
\end{itemize}
Aquele que é perito em ginecologia.
\section{Ginecomania}
\begin{itemize}
\item {Grp. gram.:f.}
\end{itemize}
\begin{itemize}
\item {Proveniência:(Do gr. \textunderscore gunaikos\textunderscore  + \textunderscore mania\textunderscore )}
\end{itemize}
Paixão excessiva por mulheres.
\section{Ginecomasta}
\begin{itemize}
\item {Grp. gram.:m.}
\end{itemize}
\begin{itemize}
\item {Proveniência:(Do gr. \textunderscore gunaikos\textunderscore  + \textunderscore mastos\textunderscore )}
\end{itemize}
Homem, que tem as mamas tão desenvolvidas como as das mulheres.
\section{Ginecónomos}
\begin{itemize}
\item {Grp. gram.:m. pl.}
\end{itemize}
\begin{itemize}
\item {Proveniência:(Do gr. \textunderscore gunaikos\textunderscore  + \textunderscore nomos\textunderscore )}
\end{itemize}
Magistrados atenienses, que velavam pelo bom comportamento das mulheres, multando as que se distinguiam pelo luxo ou por adornos excessivos.
\section{Ginecossofia}
\begin{itemize}
\item {Grp. gram.:f.}
\end{itemize}
O mesmo que \textunderscore ginecologia\textunderscore .
\section{Ginério}
\begin{itemize}
\item {Grp. gram.:m.}
\end{itemize}
\begin{itemize}
\item {Proveniência:(Do gr. \textunderscore gune\textunderscore  + \textunderscore erion\textunderscore )}
\end{itemize}
Planta arundinácea, da fam. das gramíneas, espécie de penacheiro, procedente da América tropical, e aclimada já na Europa, como planta ornamental.
\section{Gino...}
\begin{itemize}
\item {Grp. gram.:pref.}
\end{itemize}
\begin{itemize}
\item {Proveniência:(Do gr. \textunderscore gun\textunderscore )}
\end{itemize}
(designativo de fêmea ou de pistillo)
\section{Ginobase}
\begin{itemize}
\item {Grp. gram.:m.}
\end{itemize}
\begin{itemize}
\item {Utilização:Bot.}
\end{itemize}
\begin{itemize}
\item {Proveniência:(De \textunderscore gino...\textunderscore  + \textunderscore base\textunderscore )}
\end{itemize}
Base de um estilete único e engrossado, sobreposto aos lóculos de um ovário dividido.
\section{Ginobásico}
\begin{itemize}
\item {Grp. gram.:adj.}
\end{itemize}
Que nasce da base do ovário; que tem ginobase.
\section{Ginofobia}
\begin{itemize}
\item {Grp. gram.:f.}
\end{itemize}
\begin{itemize}
\item {Proveniência:(Do gr. \textunderscore gune\textunderscore  + \textunderscore phobein\textunderscore )}
\end{itemize}
Aversão infundada ás mulheres.
Medo mórbido de mulheres.
\section{Ginófobo}
\begin{itemize}
\item {Grp. gram.:m.}
\end{itemize}
Aquele que tem ginofobia.
\section{Ginoforado}
\begin{itemize}
\item {Grp. gram.:adj.}
\end{itemize}
Que tem ou fórma ginóforo.
\section{Ginóforo}
\begin{itemize}
\item {Grp. gram.:m.}
\end{itemize}
\begin{itemize}
\item {Utilização:Bot.}
\end{itemize}
\begin{itemize}
\item {Proveniência:(Do gr. \textunderscore gune\textunderscore  + \textunderscore phoros\textunderscore )}
\end{itemize}
Sustentáculo, que nasce do receptáculo da flôr e que só contém órgãos femininos.
\section{Ginópode}
\begin{itemize}
\item {Grp. gram.:adj.}
\end{itemize}
\begin{itemize}
\item {Proveniência:(Do gr. \textunderscore gune\textunderscore  + \textunderscore pous\textunderscore )}
\end{itemize}
O mesmo que \textunderscore podógino\textunderscore .
\section{Ginostema}
\begin{itemize}
\item {Grp. gram.:m.}
\end{itemize}
\begin{itemize}
\item {Utilização:Bot.}
\end{itemize}
\begin{itemize}
\item {Proveniência:(Do gr. \textunderscore gune\textunderscore  + \textunderscore stemon\textunderscore )}
\end{itemize}
Parte da flôr das orquídeas, que contém os estames e o estigma.
\section{Gipaeto}
\begin{itemize}
\item {fónica:ê}
\end{itemize}
\begin{itemize}
\item {Grp. gram.:m.}
\end{itemize}
\begin{itemize}
\item {Proveniência:(Do gr. \textunderscore gups\textunderscore , \textunderscore gupos\textunderscore  + \textunderscore aetos\textunderscore )}
\end{itemize}
Gênero de aves, intermediárias ao falcão e ao abutre.
\section{Gipelomorfas}
\begin{itemize}
\item {Grp. gram.:f. pl.}
\end{itemize}
\begin{itemize}
\item {Utilização:Zool.}
\end{itemize}
Ordem de aves, que têm por tipo o noitibó.
\section{Gípseo}
\begin{itemize}
\item {Grp. gram.:adj.}
\end{itemize}
\begin{itemize}
\item {Proveniência:(Do lat. \textunderscore gupsum\textunderscore )}
\end{itemize}
Feito de gêsso.
\section{Gipsífero}
\begin{itemize}
\item {Grp. gram.:adj.}
\end{itemize}
\begin{itemize}
\item {Proveniência:(Do lat. \textunderscore gupsum\textunderscore  + lat. \textunderscore ferre\textunderscore )}
\end{itemize}
Que contém gêsso.
\section{Gipso}
\begin{itemize}
\item {Grp. gram.:m.}
\end{itemize}
\begin{itemize}
\item {Proveniência:(Do lat. \textunderscore gypsum\textunderscore )}
\end{itemize}
Pó branco e sêco, chamado também \textunderscore gêsso de Paris\textunderscore , e que se encontra em fórma de cristaes transparentes.
\section{Gipsófila}
\begin{itemize}
\item {Grp. gram.:f.}
\end{itemize}
\begin{itemize}
\item {Proveniência:(Do gr. \textunderscore gupsos\textunderscore  + \textunderscore philos\textunderscore )}
\end{itemize}
Gênero de plantas cariofiláceas.
\section{Girino}
\begin{itemize}
\item {Grp. gram.:m.}
\end{itemize}
\begin{itemize}
\item {Grp. gram.:Pl.}
\end{itemize}
\begin{itemize}
\item {Proveniência:(Gr. \textunderscore gurinos\textunderscore )}
\end{itemize}
Fórma larvar, pisciforme, dos batrácios anuros.
Insectos carnívoros, coleópteros.
\section{Girocarpo}
\begin{itemize}
\item {Grp. gram.:m.}
\end{itemize}
\begin{itemize}
\item {Proveniência:(Do gr. \textunderscore guros\textunderscore  + \textunderscore karpos\textunderscore )}
\end{itemize}
Gênero de plantas, com flôres policarpas.
\section{Giróforo}
\begin{itemize}
\item {Grp. gram.:m.}
\end{itemize}
\begin{itemize}
\item {Proveniência:(Do gr. \textunderscore guros\textunderscore  + \textunderscore phoros\textunderscore )}
\end{itemize}
Gênero de líchens que têm giromas.
\section{Giromancia}
\begin{itemize}
\item {Grp. gram.:f.}
\end{itemize}
\begin{itemize}
\item {Proveniência:(Do gr. \textunderscore guros\textunderscore  + \textunderscore manteia\textunderscore )}
\end{itemize}
Suposta arte de adivinhar, marchando em roda.
\section{Giromântico}
\begin{itemize}
\item {Grp. gram.:m.}
\end{itemize}
Aquele que pratíca a giromancia.
\section{Giroplano}
\begin{itemize}
\item {Grp. gram.:m.}
\end{itemize}
\begin{itemize}
\item {Proveniência:(De \textunderscore gurus\textunderscore , lat. \textunderscore plano\textunderscore )}
\end{itemize}
Aparelho aerostático, que é a combinação do aeroplano com o helicóptero.
\section{Giroscópio}
\begin{itemize}
\item {Grp. gram.:m.}
\end{itemize}
\begin{itemize}
\item {Proveniência:(Do gr. \textunderscore guros\textunderscore  + \textunderscore skopein\textunderscore )}
\end{itemize}
Instrumento, para demonstrar o desvio de um corpo que gira livremente em volta da terra, em relação a pontos fixos tomados sôbre a superfície do globo.
Instrumento, para demonstrar a rotação da terra.
\section{Girosela}
\begin{itemize}
\item {Grp. gram.:f.}
\end{itemize}
Pequena e formosa planta primulácea, de flôres rosadas, (\textunderscore dodecatheon meadia\textunderscore , Lin.).
\section{Giróvago}
\begin{itemize}
\item {Grp. gram.:m.}
\end{itemize}
\begin{itemize}
\item {Proveniência:(Do lat. \textunderscore gyrus\textunderscore  + \textunderscore vagare\textunderscore )}
\end{itemize}
Cada um dos monges, que, nos primeiros tempos do monachismo, andavam de terra em terra e de cela em cella, não permanecendo nunca no mesmo sítio mais de três ou quatro dias, e vivendo de esmolas.
\section{Gurugumba}
\begin{itemize}
\item {Grp. gram.:f.}
\end{itemize}
\begin{itemize}
\item {Utilização:Bras}
\end{itemize}
Espécie de cacete.
\section{Gurumarim}
\begin{itemize}
\item {Grp. gram.:m.}
\end{itemize}
\begin{itemize}
\item {Utilização:Bras}
\end{itemize}
Árvore silvestre, de que se conhecem duas espécies.
\section{Gurumete}
\begin{itemize}
\item {fónica:mê}
\end{itemize}
\begin{itemize}
\item {Grp. gram.:m.}
\end{itemize}
O mesmo que \textunderscore grumete\textunderscore .
\section{Gurumichama}
\begin{itemize}
\item {Grp. gram.:f.}
\end{itemize}
(V.grumixama)
\section{Gurumichameira}
\begin{itemize}
\item {Grp. gram.:f.}
\end{itemize}
(V.grumixameira)
\section{Gurundi}
\begin{itemize}
\item {Grp. gram.:m.}
\end{itemize}
Gênero de aves brasileiras, nocivas aos frutos.
\section{Gurupema}
\begin{itemize}
\item {Grp. gram.:f.}
\end{itemize}
\begin{itemize}
\item {Utilização:Bras}
\end{itemize}
O mesmo que \textunderscore urupema\textunderscore .
\section{Gurupés}
\begin{itemize}
\item {Grp. gram.:m.}
\end{itemize}
\begin{itemize}
\item {Utilização:Náut.}
\end{itemize}
\begin{itemize}
\item {Proveniência:(Do fr. \textunderscore beauprés\textunderscore ?)}
\end{itemize}
Mastro, na extremidade da prôa do navio.
\section{Gurutuba}
\begin{itemize}
\item {Grp. gram.:m.}
\end{itemize}
Espécie de feijão.
\section{Gusa}
\begin{itemize}
\item {Grp. gram.:f.}
\end{itemize}
\begin{itemize}
\item {Proveniência:(Do al. \textunderscore guss\textunderscore )}
\end{itemize}
Metal fundido, para lastrar navios.
\section{Gusano}
\begin{itemize}
\item {Grp. gram.:m.}
\end{itemize}
Verme, que se produz na madeira e a fura, (\textunderscore teredo navalis\textunderscore ).
Verme, que se cria nas substâncias em decomposição.
Tavão.
(Cast. \textunderscore gusano\textunderscore )
\section{Gusla}
\begin{itemize}
\item {Grp. gram.:f.}
\end{itemize}
Espécie de rabeca, de uma só corda, usada no Oriente, e cujos sons são suavíssimos.
\section{Gustação}
\begin{itemize}
\item {Grp. gram.:f.}
\end{itemize}
\begin{itemize}
\item {Proveniência:(Lat. \textunderscore gustatio\textunderscore )}
\end{itemize}
Acto de provar.
\section{Gustadoiro}
\begin{itemize}
\item {Grp. gram.:m.}
\end{itemize}
Alimento, que se dá ao farroupo, alternadamente com a comida dos montados. (Colhido no Alentejo)
(Cp. \textunderscore gustatório\textunderscore )
\section{Gustadouro}
\begin{itemize}
\item {Grp. gram.:m.}
\end{itemize}
Alimento, que se dá ao farroupo, alternadamente com a comida dos montados. (Colhido no Alentejo)
(Cp. \textunderscore gustatório\textunderscore )
\section{Gustativo}
\begin{itemize}
\item {Grp. gram.:adj.}
\end{itemize}
\begin{itemize}
\item {Proveniência:(Do lat. \textunderscore gustare\textunderscore )}
\end{itemize}
Relativo ao sentido do gôsto.
\section{Gustatório}
\begin{itemize}
\item {Grp. gram.:m.}
\end{itemize}
\begin{itemize}
\item {Utilização:Des.}
\end{itemize}
\begin{itemize}
\item {Proveniência:(Lat. \textunderscore gustatorium\textunderscore )}
\end{itemize}
Primeiro prato de comida, para abrir o appetite.
\section{Gustávia}
\begin{itemize}
\item {Grp. gram.:f.}
\end{itemize}
\begin{itemize}
\item {Proveniência:(De \textunderscore Gustavo\textunderscore , n. p.)}
\end{itemize}
Gênero de copadas árvores do Brasil.
\section{Guta}
\begin{itemize}
\item {Grp. gram.:f.}
\end{itemize}
\begin{itemize}
\item {Proveniência:(Do mal. \textunderscore getah\textunderscore )}
\end{itemize}
Espécie de goma que se extrai da guteira.
\section{Gutapercha}
\begin{itemize}
\item {Grp. gram.:f.}
\end{itemize}
\begin{itemize}
\item {Proveniência:(Do mal. \textunderscore getah-pertejah\textunderscore )}
\end{itemize}
Matéria glutinosa, extrahida de uma planta sapotácea.
\section{Gute}
\begin{itemize}
\item {Grp. gram.:adj.}
\end{itemize}
\begin{itemize}
\item {Utilização:Gír.}
\end{itemize}
\begin{itemize}
\item {Proveniência:(Do al. \textunderscore gut\textunderscore )}
\end{itemize}
Bom.
\section{Guté}
\begin{itemize}
\item {Grp. gram.:m.}
\end{itemize}
Árvore fructífera do Brasil.
\section{Gutedra}
\begin{itemize}
\item {Grp. gram.:f.}
\end{itemize}
\begin{itemize}
\item {Utilização:Ant.}
\end{itemize}
O mesmo que \textunderscore polainas\textunderscore .
\section{Guteira}
\begin{itemize}
\item {Grp. gram.:f.}
\end{itemize}
\begin{itemize}
\item {Proveniência:(De \textunderscore guta\textunderscore )}
\end{itemize}
Árvore gutífera, (\textunderscore garcinia cambogia\textunderscore ).
\section{Gutíferas}
\begin{itemize}
\item {Grp. gram.:f. pl.}
\end{itemize}
\begin{itemize}
\item {Proveniência:(De \textunderscore gutífero\textunderscore )}
\end{itemize}
Ordem de plantas, que têm por typo a guteira.
\section{Gutífero}
\begin{itemize}
\item {Grp. gram.:adj.}
\end{itemize}
\begin{itemize}
\item {Proveniência:(Do lat. \textunderscore gutta\textunderscore  + \textunderscore ferre\textunderscore )}
\end{itemize}
Que deita gotas.
\section{Gutífero}
\begin{itemize}
\item {Grp. gram.:adj.}
\end{itemize}
\begin{itemize}
\item {Proveniência:(De \textunderscore guta\textunderscore  + lat. \textunderscore ferre\textunderscore )}
\end{itemize}
Relativo ou semelhante á guta.
\section{Gutina}
\begin{itemize}
\item {Grp. gram.:f.}
\end{itemize}
Árvore chilena, cuja madeira se emprega em tinturaría e dá côr preta.
\section{Guto}
\begin{itemize}
\item {Grp. gram.:m.}
\end{itemize}
\begin{itemize}
\item {Proveniência:(Lat. \textunderscore guttus\textunderscore )}
\end{itemize}
Vaso, de gargalo estreito, donde o líquido sái, gota a gota.
Vaso dos sacrifícios, entre os Romanos.
\section{Guttífero}
\begin{itemize}
\item {Grp. gram.:adj.}
\end{itemize}
\begin{itemize}
\item {Proveniência:(Do lat. \textunderscore gutta\textunderscore  + \textunderscore ferre\textunderscore )}
\end{itemize}
Que deita gotas.
\section{Gutto}
\begin{itemize}
\item {Grp. gram.:m.}
\end{itemize}
\begin{itemize}
\item {Proveniência:(Lat. \textunderscore guttus\textunderscore )}
\end{itemize}
Vaso, de gargalo estreito, donde o líquido sái, gota a gota.
Vaso dos sacrifícios, entre os Romanos.
\section{Guttural}
\begin{itemize}
\item {Grp. gram.:adj.}
\end{itemize}
\begin{itemize}
\item {Proveniência:(Do lat. \textunderscore guttur\textunderscore )}
\end{itemize}
Relativo á garganta.
Modificado pela garganta, (falando-se do som).
\section{Gutturalização}
\begin{itemize}
\item {Grp. gram.:f.}
\end{itemize}
Acto ou effeito de gutturalizar.
\section{Gutturalizar}
\begin{itemize}
\item {Grp. gram.:v.}
\end{itemize}
\begin{itemize}
\item {Utilização:t. Gram.}
\end{itemize}
\begin{itemize}
\item {Proveniência:(De \textunderscore guttural\textunderscore )}
\end{itemize}
Pronunciar certas letras, de maneira que a parte posterior da língua se arqueie para o palato molle, sem o tocar, e a pharynge se expanda, como succede com o \textunderscore l\textunderscore  final de sýllaba e com as vogaes nasaes do norte de Portugal.
\section{Gutturalmente}
\begin{itemize}
\item {Grp. gram.:adv.}
\end{itemize}
De modo guttural.
Com auxílio da garganta.
\section{Gutturoso}
\begin{itemize}
\item {Grp. gram.:adj.}
\end{itemize}
\begin{itemize}
\item {Grp. gram.:M.}
\end{itemize}
\begin{itemize}
\item {Proveniência:(Do lat. \textunderscore guttur\textunderscore )}
\end{itemize}
Diz-se de certos musgos, que têm apóphyse volumosa.
Diz-se do animal, que tem dilatada a parte anterior do pescoço.
Espécie de antílope, de grande larynge.
\section{Gutural}
\begin{itemize}
\item {Grp. gram.:adj.}
\end{itemize}
\begin{itemize}
\item {Proveniência:(Do lat. \textunderscore guttur\textunderscore )}
\end{itemize}
Relativo á garganta.
Modificado pela garganta, (falando-se do som).
\section{Guturalização}
\begin{itemize}
\item {Grp. gram.:f.}
\end{itemize}
Acto ou efeito de guturalizar.
\section{Guturalizar}
\begin{itemize}
\item {Grp. gram.:v.}
\end{itemize}
\begin{itemize}
\item {Utilização:t. Gram.}
\end{itemize}
\begin{itemize}
\item {Proveniência:(De \textunderscore guttural\textunderscore )}
\end{itemize}
Pronunciar certas letras, de maneira que a parte posterior da língua se arqueie para o palato mole, sem o tocar, e a faringe se expanda, como sucede com o \textunderscore l\textunderscore  final de sílaba e com as vogaes nasaes do norte de Portugal.
\section{Guturalmente}
\begin{itemize}
\item {Grp. gram.:adv.}
\end{itemize}
De modo gutural.
Com auxílio da garganta.
\section{Guturoso}
\begin{itemize}
\item {Grp. gram.:adj.}
\end{itemize}
\begin{itemize}
\item {Grp. gram.:M.}
\end{itemize}
\begin{itemize}
\item {Proveniência:(Do lat. \textunderscore guttur\textunderscore )}
\end{itemize}
Diz-se de certos musgos, que têm apófise volumosa.
Diz-se do animal, que tem dilatada a parte anterior do pescoço.
Espécie de antílope, de grande laringe.
\section{Guzarate}
\begin{itemize}
\item {Grp. gram.:m.}
\end{itemize}
O mesmo ou melhor que \textunderscore guzerate\textunderscore ^1.
\section{Guzerate}
\begin{itemize}
\item {Grp. gram.:m.}
\end{itemize}
\begin{itemize}
\item {Grp. gram.:Pl.}
\end{itemize}
Língua indígena da região do mesmo nome.
Habitantes de Guzerate.
\section{Guzla}
\begin{itemize}
\item {Grp. gram.:f.}
\end{itemize}
(V.gusla)
\section{Guzo}
\begin{itemize}
\item {Grp. gram.:m.}
\end{itemize}
\begin{itemize}
\item {Utilização:Bras. do S}
\end{itemize}
O mesmo que \textunderscore fôrça\textunderscore .
\section{Gymnandro}
\begin{itemize}
\item {Grp. gram.:adj.}
\end{itemize}
\begin{itemize}
\item {Utilização:Bot.}
\end{itemize}
\begin{itemize}
\item {Proveniência:(Do gr. \textunderscore gumnos\textunderscore  + \textunderscore aner\textunderscore )}
\end{itemize}
Que tem os estames nus.
\section{Gymnantho}
\begin{itemize}
\item {Grp. gram.:adj.}
\end{itemize}
\begin{itemize}
\item {Utilização:Bot.}
\end{itemize}
\begin{itemize}
\item {Proveniência:(Do gr. \textunderscore gumnos\textunderscore  + \textunderscore anthos\textunderscore )}
\end{itemize}
Cujas flôres não têm invólucro algum.
\section{Gymnasial}
\begin{itemize}
\item {Grp. gram.:adj.}
\end{itemize}
Relativo a gymnásio.
\section{Gymnasiarcha}
\begin{itemize}
\item {fónica:ca}
\end{itemize}
\begin{itemize}
\item {Grp. gram.:m.}
\end{itemize}
\begin{itemize}
\item {Proveniência:(Gr. \textunderscore gumnasiarkhes\textunderscore )}
\end{itemize}
Chefe ou director de exercícios gymnásticos, entre os antigos.
\section{Gymnásio}
\begin{itemize}
\item {Grp. gram.:m.}
\end{itemize}
\begin{itemize}
\item {Proveniência:(Gr. \textunderscore gumnasion\textunderscore )}
\end{itemize}
Lugar, em que se pratíca a gymnástica.
Estabelecimento de ensino secundário na Alemanha.
\section{Gymnasta}
\begin{itemize}
\item {Grp. gram.:m.}
\end{itemize}
\begin{itemize}
\item {Proveniência:(Gr. \textunderscore gumnastes\textunderscore )}
\end{itemize}
Aquelle que pratíca a gymnástica.
Aquelle que é hábil em gymnástica; acrobata.
\section{Gymnaste}
\begin{itemize}
\item {Grp. gram.:m.}
\end{itemize}
O mesmo que \textunderscore gymnasta\textunderscore .
\section{Gymnástica}
\begin{itemize}
\item {Grp. gram.:f.}
\end{itemize}
\begin{itemize}
\item {Utilização:Fig.}
\end{itemize}
\begin{itemize}
\item {Proveniência:(De \textunderscore gymnástico\textunderscore )}
\end{itemize}
Arte ou acto de exercitar o corpo para o fortificar.
Exercício de discorrer.
\section{Gymnástico}
\begin{itemize}
\item {Grp. gram.:adj.}
\end{itemize}
\begin{itemize}
\item {Proveniência:(Gr. \textunderscore gumnastikos\textunderscore )}
\end{itemize}
Relativo a gymnástica.
\section{Gymnetos}
\begin{itemize}
\item {Grp. gram.:m. pl.}
\end{itemize}
\begin{itemize}
\item {Proveniência:(Do gr. \textunderscore gumnos\textunderscore , nu)}
\end{itemize}
Nome, que, em Argos, se dava aos escravos, por andarem mal vestidos ou quási nus.
\section{Gymnetros}
\begin{itemize}
\item {Grp. gram.:m. pl.}
\end{itemize}
\begin{itemize}
\item {Proveniência:(Do gr. \textunderscore gumnos\textunderscore  + \textunderscore etron\textunderscore )}
\end{itemize}
Gênero de peixes acanthopterýgios.
\section{Gýmnico}
\begin{itemize}
\item {Grp. gram.:adj.}
\end{itemize}
O mesmo que \textunderscore gymnástico\textunderscore .
\section{Gymno...}
\begin{itemize}
\item {Proveniência:(Do gr. \textunderscore gumnos\textunderscore )}
\end{itemize}
Elemento, que entra na formação de várias palavras, com a significação de \textunderscore nu\textunderscore .
\section{Gymnoblasto}
\begin{itemize}
\item {Grp. gram.:adj.}
\end{itemize}
\begin{itemize}
\item {Utilização:Bot.}
\end{itemize}
\begin{itemize}
\item {Proveniência:(Do gr. \textunderscore gumnos\textunderscore  + \textunderscore blastos\textunderscore )}
\end{itemize}
Diz-se da planta, que não tem o embryão contido em cavidade particular.
\section{Gymnocarpo}
\begin{itemize}
\item {Grp. gram.:adj.}
\end{itemize}
\begin{itemize}
\item {Utilização:Bot.}
\end{itemize}
\begin{itemize}
\item {Proveniência:(Do gr. \textunderscore gumnos\textunderscore  + \textunderscore karpos\textunderscore )}
\end{itemize}
Diz-se dos frutos descobertos, que não são soldados com algum órgão accessório.
\section{Gymnocaule}
\begin{itemize}
\item {Grp. gram.:adj.}
\end{itemize}
\begin{itemize}
\item {Utilização:Bot.}
\end{itemize}
\begin{itemize}
\item {Proveniência:(Do gr. \textunderscore gumnos\textunderscore  + \textunderscore kaulos\textunderscore )}
\end{itemize}
Que tem a haste nua de fôlhas.
\section{Gymnocéphalo}
\begin{itemize}
\item {Grp. gram.:adj.}
\end{itemize}
\begin{itemize}
\item {Utilização:Zool.}
\end{itemize}
\begin{itemize}
\item {Proveniência:(Do gr. \textunderscore gumnos\textunderscore  + \textunderscore kephale\textunderscore )}
\end{itemize}
Que tem a cabeça nua, sem pêlos ou sem pennas.
\section{Gymnoclado}
\begin{itemize}
\item {Grp. gram.:m.}
\end{itemize}
\begin{itemize}
\item {Proveniência:(Do gr. \textunderscore gumnos\textunderscore  + \textunderscore klados\textunderscore )}
\end{itemize}
Gênero de plantas leguminosas.
\section{Gymnodermo}
\begin{itemize}
\item {Grp. gram.:adj.}
\end{itemize}
\begin{itemize}
\item {Utilização:Zool.}
\end{itemize}
\begin{itemize}
\item {Proveniência:(Do gr. \textunderscore gumnos\textunderscore  + \textunderscore derma\textunderscore )}
\end{itemize}
Que tem pelle nua.
\section{Gymnodonte}
\begin{itemize}
\item {Grp. gram.:adj.}
\end{itemize}
\begin{itemize}
\item {Utilização:Zool.}
\end{itemize}
\begin{itemize}
\item {Proveniência:(Do gr. \textunderscore gumnos\textunderscore  + \textunderscore odous\textunderscore )}
\end{itemize}
Que tem os dentes á vista.
\section{Gymnogomphos}
\begin{itemize}
\item {Grp. gram.:m. pl.}
\end{itemize}
\begin{itemize}
\item {Proveniência:(Do gr. \textunderscore gumnos\textunderscore  + \textunderscore gomphos\textunderscore )}
\end{itemize}
Animálculos infusórios, cujos dentes se ligam á maxilla só pela base.
\section{Gymnogramma}
\begin{itemize}
\item {Grp. gram.:m.}
\end{itemize}
\begin{itemize}
\item {Proveniência:(Do gr. \textunderscore gumnos\textunderscore  + \textunderscore gramma\textunderscore )}
\end{itemize}
Gênero de fêtos.
\section{Gymnógyno}
\begin{itemize}
\item {Grp. gram.:adj.}
\end{itemize}
\begin{itemize}
\item {Utilização:Bot.}
\end{itemize}
\begin{itemize}
\item {Proveniência:(Do gr. \textunderscore gumnos\textunderscore  + \textunderscore gune\textunderscore )}
\end{itemize}
Que tem o ovário nu.
\section{Gymnopedia}
\begin{itemize}
\item {Grp. gram.:f.}
\end{itemize}
\begin{itemize}
\item {Proveniência:(Gr. \textunderscore gumnopaidia\textunderscore )}
\end{itemize}
Antiga dança espartana, executada por homens e crianças nuas, em certa festa annual e ao som de hymnos compostos para êsse effeito.
\section{Gymnophídio}
\begin{itemize}
\item {Grp. gram.:adj.}
\end{itemize}
\begin{itemize}
\item {Proveniência:(Do gr. \textunderscore gumnos\textunderscore  + \textunderscore ophis\textunderscore )}
\end{itemize}
Diz-se das serpentes, que têm a pelle nua, lisa e viscosa.
\section{Gymnópode}
\begin{itemize}
\item {Grp. gram.:adj.}
\end{itemize}
\begin{itemize}
\item {Grp. gram.:M. pl.}
\end{itemize}
\begin{itemize}
\item {Proveniência:(Do gr. \textunderscore gumnos\textunderscore  + \textunderscore pous\textunderscore , \textunderscore podos\textunderscore )}
\end{itemize}
Que tem os pés nus.
Família de reptis.
\section{Gymnópomo}
\begin{itemize}
\item {Grp. gram.:adj.}
\end{itemize}
\begin{itemize}
\item {Utilização:Ichthyol.}
\end{itemize}
\begin{itemize}
\item {Proveniência:(Do gr. \textunderscore gumnos\textunderscore  + \textunderscore poma\textunderscore )}
\end{itemize}
Que tem os opérculos nus.
\section{Gymnóptero}
\begin{itemize}
\item {Grp. gram.:adj.}
\end{itemize}
\begin{itemize}
\item {Utilização:Zool.}
\end{itemize}
\begin{itemize}
\item {Grp. gram.:M. pl.}
\end{itemize}
\begin{itemize}
\item {Proveniência:(Do gr. \textunderscore gumnos\textunderscore  + \textunderscore pteron\textunderscore )}
\end{itemize}
Que tem as asas nuas, sem escamas.
Secção da classe dos insectos, que comprehende os que têm asas lisas, sem elytros nem escamas farináceas.
\section{Gymnosomo}
\begin{itemize}
\item {fónica:sô}
\end{itemize}
\begin{itemize}
\item {Grp. gram.:adj.}
\end{itemize}
\begin{itemize}
\item {Utilização:Zool.}
\end{itemize}
\begin{itemize}
\item {Proveniência:(Do gr. \textunderscore gumnos\textunderscore  + \textunderscore soma\textunderscore )}
\end{itemize}
Que tem o corpo nu.
\section{Gymnosophia}
\begin{itemize}
\item {Grp. gram.:f.}
\end{itemize}
Doutrina dos gymnosophistas.
\section{Gymnosophista}
\begin{itemize}
\item {Grp. gram.:m.}
\end{itemize}
\begin{itemize}
\item {Proveniência:(Gr. \textunderscore gumnosophistes\textunderscore )}
\end{itemize}
Philósopho indiano, que se abstinha de carnes e se dedicava á contemplação mýstica.
\section{Gymnospermas}
\begin{itemize}
\item {Grp. gram.:f. pl.}
\end{itemize}
\begin{itemize}
\item {Proveniência:(Do gr. \textunderscore gumnos\textunderscore  + \textunderscore sperma\textunderscore )}
\end{itemize}
Plantas dicotyledóneas, cujas flôres têm quatro grãos nus ao fundo do cálice.
\section{Gymnospermia}
\begin{itemize}
\item {Grp. gram.:f.}
\end{itemize}
Ordem das plantas gymnospermas.
\section{Gymnospérmico}
\begin{itemize}
\item {Grp. gram.:adj.}
\end{itemize}
O mesmo que \textunderscore gymnospermo\textunderscore .
\section{Gymnospermo}
\begin{itemize}
\item {Grp. gram.:adj.}
\end{itemize}
Relativo á gymnospermia.
\section{Gymnospódia}
\begin{itemize}
\item {Grp. gram.:f.}
\end{itemize}
\begin{itemize}
\item {Proveniência:(Do gr. \textunderscore gumnos\textunderscore , nu, + \textunderscore pous\textunderscore , \textunderscore podos\textunderscore , pé)}
\end{itemize}
Dança e música, executada por pessôas descalças?:«\textunderscore ...a gymnospódia, que se canta e dança...\textunderscore »\textunderscore Viriato Trág.\textunderscore , XI, 44.
\section{Gymnosporado}
\begin{itemize}
\item {Grp. gram.:adj.}
\end{itemize}
\begin{itemize}
\item {Utilização:Bot.}
\end{itemize}
\begin{itemize}
\item {Proveniência:(Do gr. \textunderscore gumnos\textunderscore  + \textunderscore spora\textunderscore )}
\end{itemize}
Que tem os poros livres em cavidade commum, pela absorpção da theca.
\section{Gymnósporo}
\begin{itemize}
\item {Grp. gram.:adj.}
\end{itemize}
O mesmo que \textunderscore gymnosporado\textunderscore .
\section{Gymnóstomo}
\begin{itemize}
\item {Grp. gram.:adj.}
\end{itemize}
\begin{itemize}
\item {Utilização:Bot.}
\end{itemize}
\begin{itemize}
\item {Proveniência:(Do gr. \textunderscore gumnos\textunderscore  + \textunderscore stoma\textunderscore )}
\end{itemize}
Cuja bôca não tem appendices.
\section{Gymnotetraspermo}
\begin{itemize}
\item {Grp. gram.:adj.}
\end{itemize}
\begin{itemize}
\item {Utilização:Bot.}
\end{itemize}
\begin{itemize}
\item {Proveniência:(Do gr. \textunderscore gumnos\textunderscore  + \textunderscore tetra\textunderscore  + \textunderscore sperma\textunderscore )}
\end{itemize}
Diz-se da planta, cujo ovário é profundamente dividido em quatro partes, imitando quatro grãos nus no fundo do cálice.
\section{Gymnoto}
\begin{itemize}
\item {Grp. gram.:m.}
\end{itemize}
\begin{itemize}
\item {Proveniência:(Do gr. \textunderscore gumnos\textunderscore  + \textunderscore notos\textunderscore )}
\end{itemize}
Gênero de peixes, da fam. das enguias.
\section{Gymnuro}
\begin{itemize}
\item {Grp. gram.:adj.}
\end{itemize}
\begin{itemize}
\item {Utilização:Zool.}
\end{itemize}
\begin{itemize}
\item {Grp. gram.:M. pl.}
\end{itemize}
\begin{itemize}
\item {Proveniência:(Do gr. \textunderscore gumnos\textunderscore  + \textunderscore oura\textunderscore )}
\end{itemize}
Que tem a cauda nua.
Secção da fam. dos macacos, que comprehende os sapajus de cauda nua e callosa.
\section{Gynandria}
\begin{itemize}
\item {Grp. gram.:f.}
\end{itemize}
\begin{itemize}
\item {Proveniência:(Do gr. \textunderscore gune\textunderscore  + \textunderscore aner\textunderscore )}
\end{itemize}
Classe de vegetaes, cujos estames estão insertos nos pistillos, (segundo o systema de Lin.)
\section{Gynanthropo}
\begin{itemize}
\item {Grp. gram.:m.}
\end{itemize}
\begin{itemize}
\item {Proveniência:(Do gr. \textunderscore gune\textunderscore  + \textunderscore anthropos\textunderscore )}
\end{itemize}
Hermaphrodita, que participa mais das qualidades phýsicas da mulher, do que das do homem.
\section{Gyneceu}
\begin{itemize}
\item {Grp. gram.:m.}
\end{itemize}
\begin{itemize}
\item {Utilização:Bot.}
\end{itemize}
\begin{itemize}
\item {Proveniência:(Gr. \textunderscore gunaikeion\textunderscore )}
\end{itemize}
Na antiguidade, aposento de mulheres.
Na Idade-Média, manufactura, onde os senhores obrigavam as vassallas a trabalhar em lan ou seda.
Conjunto dos pistillos ou dos órgãos femininos de uma flôr.
\section{Gyneco...}
\begin{itemize}
\item {Proveniência:(Do gr. \textunderscore gune\textunderscore , \textunderscore gunaikos\textunderscore )}
\end{itemize}
Elemento, que entra na formação de várias palavras, com a significação de \textunderscore mulher\textunderscore  ou \textunderscore feminino\textunderscore .
\section{Gynecocracia}
\begin{itemize}
\item {Grp. gram.:f.}
\end{itemize}
\begin{itemize}
\item {Proveniência:(Do gr. \textunderscore gunaikos\textunderscore  + \textunderscore kratein\textunderscore )}
\end{itemize}
Preponderância das mulheres na governação pública.
\section{Gynecocrata}
\begin{itemize}
\item {Grp. gram.:m.}
\end{itemize}
Partidário da gynecocracia.
\section{Gynecocrático}
\begin{itemize}
\item {Grp. gram.:adj.}
\end{itemize}
Relativo á gynecocracia.
\section{Gynecographia}
\begin{itemize}
\item {Grp. gram.:f.}
\end{itemize}
\begin{itemize}
\item {Proveniência:(Do gr. \textunderscore gunaikos\textunderscore  + \textunderscore graphein\textunderscore )}
\end{itemize}
O mesmo que \textunderscore gynecologia\textunderscore .
\section{Gynecográphico}
\begin{itemize}
\item {Grp. gram.:adj.}
\end{itemize}
Relativo á gynecographia.
\section{Gynecologia}
\begin{itemize}
\item {Grp. gram.:f.}
\end{itemize}
\begin{itemize}
\item {Proveniência:(Do gr. \textunderscore gunaikos\textunderscore  + \textunderscore logos\textunderscore )}
\end{itemize}
Tratado á cêrca das mulheres.
Therapêutica das doenças peculiares ás mulheres.
\section{Gynecológico}
\begin{itemize}
\item {Grp. gram.:adj.}
\end{itemize}
Relativo á gynecologia.
\section{Gynecologista}
\begin{itemize}
\item {Grp. gram.:m.}
\end{itemize}
Tratadista de gynecologia.
\section{Gynecólogo}
\begin{itemize}
\item {Grp. gram.:m.}
\end{itemize}
Aquelle que é perito em gynecologia.
\section{Gynecomania}
\begin{itemize}
\item {Grp. gram.:f.}
\end{itemize}
\begin{itemize}
\item {Proveniência:(Do gr. \textunderscore gunaikos\textunderscore  + \textunderscore mania\textunderscore )}
\end{itemize}
Paixão excessiva por mulheres.
\section{Gynecomasta}
\begin{itemize}
\item {Grp. gram.:m.}
\end{itemize}
\begin{itemize}
\item {Proveniência:(Do gr. \textunderscore gunaikos\textunderscore  + \textunderscore mastos\textunderscore )}
\end{itemize}
Homem, que tem as mamas tão desenvolvidas como as das mulheres.
\section{Gynecónomos}
\begin{itemize}
\item {Grp. gram.:m. pl.}
\end{itemize}
\begin{itemize}
\item {Proveniência:(Do gr. \textunderscore gunaikos\textunderscore  + \textunderscore nomos\textunderscore )}
\end{itemize}
Magistrados athenienses, que velavam pelo bom comportamento das mulheres, multando as que se distinguiam pelo luxo ou por adornos excessivos.
\section{Gynecosophia}
\begin{itemize}
\item {fónica:so}
\end{itemize}
\begin{itemize}
\item {Grp. gram.:f.}
\end{itemize}
O mesmo que \textunderscore gynecologia\textunderscore .
\section{Gynério}
\begin{itemize}
\item {Grp. gram.:m.}
\end{itemize}
\begin{itemize}
\item {Proveniência:(Do gr. \textunderscore gune\textunderscore  + \textunderscore erion\textunderscore )}
\end{itemize}
Planta arundinácea, da fam. das gramíneas, espécie de pennacheiro, procedente da América tropical, e aclimada já na Europa, como planta ornamental.
\section{Gyno...}
\begin{itemize}
\item {Grp. gram.:pref.}
\end{itemize}
\begin{itemize}
\item {Proveniência:(Do gr. \textunderscore gun\textunderscore )}
\end{itemize}
(designativo de fêmea ou de pistillo)
\section{Gynobase}
\begin{itemize}
\item {Grp. gram.:m.}
\end{itemize}
\begin{itemize}
\item {Utilização:Bot.}
\end{itemize}
\begin{itemize}
\item {Proveniência:(De \textunderscore gyno...\textunderscore  + \textunderscore base\textunderscore )}
\end{itemize}
Base de um estilete único e engrossado, sobreposto aos lóculos de um ovário dividido.
\section{Gynobásico}
\begin{itemize}
\item {Grp. gram.:adj.}
\end{itemize}
Que nasce da base do ovário; que tem gynobase.
\section{Gynophobia}
\begin{itemize}
\item {Grp. gram.:f.}
\end{itemize}
\begin{itemize}
\item {Proveniência:(Do gr. \textunderscore gune\textunderscore  + \textunderscore phobein\textunderscore )}
\end{itemize}
Aversão infundada ás mulheres.
Medo mórbido de mulheres.
\section{Gynóphobo}
\begin{itemize}
\item {Grp. gram.:m.}
\end{itemize}
Aquelle que tem gynophobia.
\section{Gynophorado}
\begin{itemize}
\item {Grp. gram.:adj.}
\end{itemize}
Que tem ou fórma gynóphoro.
\section{Gynóphoro}
\begin{itemize}
\item {Grp. gram.:m.}
\end{itemize}
\begin{itemize}
\item {Utilização:Bot.}
\end{itemize}
\begin{itemize}
\item {Proveniência:(Do gr. \textunderscore gune\textunderscore  + \textunderscore phoros\textunderscore )}
\end{itemize}
Sustentáculo, que nasce do receptáculo da flôr e que só contém órgãos femininos.
\section{Gynópode}
\begin{itemize}
\item {Grp. gram.:adj.}
\end{itemize}
\begin{itemize}
\item {Proveniência:(Do gr. \textunderscore gune\textunderscore  + \textunderscore pous\textunderscore )}
\end{itemize}
O mesmo que \textunderscore podógyno\textunderscore .
\section{Gynostema}
\begin{itemize}
\item {Grp. gram.:m.}
\end{itemize}
\begin{itemize}
\item {Utilização:Bot.}
\end{itemize}
\begin{itemize}
\item {Proveniência:(Do gr. \textunderscore gune\textunderscore  + \textunderscore stemon\textunderscore )}
\end{itemize}
Parte da flôr das orchídeas, que contém os estames e o estigma.
\section{Gypaeto}
\begin{itemize}
\item {fónica:ê}
\end{itemize}
\begin{itemize}
\item {Grp. gram.:m.}
\end{itemize}
\begin{itemize}
\item {Proveniência:(Do gr. \textunderscore gups\textunderscore , \textunderscore gupos\textunderscore  + \textunderscore aetos\textunderscore )}
\end{itemize}
Gênero de aves, intermediárias ao falcão e ao abutre.
\section{Gypelomorphas}
\begin{itemize}
\item {Grp. gram.:f. pl.}
\end{itemize}
\begin{itemize}
\item {Utilização:Zool.}
\end{itemize}
Ordem de aves, que têm por typo o noitibó.
\section{Gýpseo}
\begin{itemize}
\item {Grp. gram.:adj.}
\end{itemize}
\begin{itemize}
\item {Proveniência:(Do lat. \textunderscore gupsum\textunderscore )}
\end{itemize}
Feito de gêsso.
\section{Gypsífero}
\begin{itemize}
\item {Grp. gram.:adj.}
\end{itemize}
\begin{itemize}
\item {Proveniência:(Do lat. \textunderscore gupsum\textunderscore  + lat. \textunderscore ferre\textunderscore )}
\end{itemize}
Que contém gêsso.
\section{Gypso}
\begin{itemize}
\item {Grp. gram.:m.}
\end{itemize}
\begin{itemize}
\item {Proveniência:(Do lat. \textunderscore gypsum\textunderscore )}
\end{itemize}
Pó branco e sêco, chamado também \textunderscore gêsso de Paris\textunderscore , e que se encontra em fórma de crystaes transparentes.
\section{Gypsóphila}
\begin{itemize}
\item {Grp. gram.:f.}
\end{itemize}
\begin{itemize}
\item {Proveniência:(Do gr. \textunderscore gupsos\textunderscore  + \textunderscore philos\textunderscore )}
\end{itemize}
Gênero de plantas caryophylláceas.
\section{Gyrino}
\begin{itemize}
\item {Grp. gram.:m.}
\end{itemize}
\begin{itemize}
\item {Grp. gram.:Pl.}
\end{itemize}
\begin{itemize}
\item {Proveniência:(Gr. \textunderscore gurinos\textunderscore )}
\end{itemize}
Fórma larvar, pisciforme, dos batrácios anuros.
Insectos carnívoros, coleópteros.
\section{Gyrocarpo}
\begin{itemize}
\item {Grp. gram.:m.}
\end{itemize}
\begin{itemize}
\item {Proveniência:(Do gr. \textunderscore guros\textunderscore  + \textunderscore karpos\textunderscore )}
\end{itemize}
Gênero de plantas, com flôres polycarpas.
\section{Gyromancia}
\begin{itemize}
\item {Grp. gram.:f.}
\end{itemize}
\begin{itemize}
\item {Proveniência:(Do gr. \textunderscore guros\textunderscore  + \textunderscore manteia\textunderscore )}
\end{itemize}
Supposta arte de adivinhar, marchando em roda.
\section{Gyromântico}
\begin{itemize}
\item {Grp. gram.:m.}
\end{itemize}
Aquelle que pratíca a gyromancia.
\section{Gyróphoro}
\begin{itemize}
\item {Grp. gram.:m.}
\end{itemize}
\begin{itemize}
\item {Proveniência:(Do gr. \textunderscore guros\textunderscore  + \textunderscore phoros\textunderscore )}
\end{itemize}
Gênero de líchens que têm gyromas.
\section{Gyroplano}
\begin{itemize}
\item {Grp. gram.:m.}
\end{itemize}
\begin{itemize}
\item {Proveniência:(De \textunderscore gurus\textunderscore , lat. \textunderscore plano\textunderscore )}
\end{itemize}
Apparelho aerostático, que é a combinação do aeroplano com o helicóptero.
\section{Gyroscópio}
\begin{itemize}
\item {Grp. gram.:m.}
\end{itemize}
\begin{itemize}
\item {Proveniência:(Do gr. \textunderscore guros\textunderscore  + \textunderscore skopein\textunderscore )}
\end{itemize}
Instrumento, para demonstrar o desvio de um corpo que gira livremente em volta da terra, em relação a pontos fixos tomados sôbre a superfície do globo.
Instrumento, para demonstrar a rotação da terra.
\section{Gyrosella}
\begin{itemize}
\item {Grp. gram.:f.}
\end{itemize}
Pequena e formosa planta primulácea, de flôres rosadas, (\textunderscore dodecatheon meadia\textunderscore , Lin.).
\section{Gyróvago}
\begin{itemize}
\item {Grp. gram.:m.}
\end{itemize}
\begin{itemize}
\item {Proveniência:(Do lat. \textunderscore gyrus\textunderscore  + \textunderscore vagare\textunderscore )}
\end{itemize}
Cada um dos monges, que, nos primeiros tempos do monachismo, andavam de terra em terra e de cella em cella, não permanecendo nunca no mesmo sítio mais de três ou quatro dias, e vivendo de esmolas.
\section{Gerosolimitano}
\begin{itemize}
\item {Grp. gram.:adj.}
\end{itemize}
\end{document}