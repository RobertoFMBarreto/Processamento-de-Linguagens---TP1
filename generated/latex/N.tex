
\begin{itemize}
\item {Proveniência: }
\end{itemize}\documentclass{article}
\usepackage[portuguese]{babel}
\title{N}
\begin{document}
Que tem cabeça em fórma de ventosa ou sugadoiro.
\section{Neo-classicismo}
\begin{itemize}
\item {Grp. gram.:m.}
\end{itemize}
Imitação actual dos antigos escritores ou artistas clássicos. Cf. R. Jorge, \textunderscore El Greco\textunderscore , 33.
\section{Neo-clássico}
\begin{itemize}
\item {Grp. gram.:m.}
\end{itemize}
Aquelle que pratica o neo-classicismo.
\section{Nerónio}
\begin{itemize}
\item {Grp. gram.:adj.}
\end{itemize}
\begin{itemize}
\item {Proveniência:(Lat. \textunderscore neronius\textunderscore )}
\end{itemize}
Relativo a Nero; o mesmo que \textunderscore nerónico\textunderscore .
\section{Nosómano}
\begin{itemize}
\item {Grp. gram.:m.  e  adj.}
\end{itemize}
O mesmo que \textunderscore nosomaníaco\textunderscore .
\section{N}
\begin{itemize}
\item {fónica:êne}
\end{itemize}
\begin{itemize}
\item {Grp. gram.:m.}
\end{itemize}
Décima quarta letra do alphabeto português.
Quando maiúsculo e seguido de um ponto, indica o nome de pessôa desconhecida ou de pessôa que se não quere nomear.
Abrev. das palavras latinas \textunderscore nota\textunderscore  ou \textunderscore nota bene\textunderscore .
Abrev. de \textunderscore norte\textunderscore .
\section{Na}
Flexão fem. de \textunderscore no\textunderscore .
(Cp. \textunderscore no\textunderscore ^1 e \textunderscore no\textunderscore ^2)
\section{Nã}
\begin{itemize}
\item {Grp. gram.:adv.}
\end{itemize}
\begin{itemize}
\item {Utilização:Pop.}
\end{itemize}
O mesmo que \textunderscore nada\textunderscore  ou \textunderscore não\textunderscore .
\section{Naalianismo}
\begin{itemize}
\item {Grp. gram.:m.}
\end{itemize}
Doutrina ou heresia dos Naalianos.
\section{Naalianos}
\begin{itemize}
\item {Grp. gram.:m. pl.}
\end{itemize}
Antigos herejes, que sustentavam doutrina quási semelhante á dos Valentinianos.
(Cp. \textunderscore valentinianos\textunderscore )
\section{Naba}
\begin{itemize}
\item {Grp. gram.:adj.}
\end{itemize}
Diz-se de uma espécie de couve, que tem a apparência do nabo.
\section{Nababia}
\begin{itemize}
\item {Grp. gram.:f.}
\end{itemize}
Região governada por um nababo.
\section{Nababo}
\begin{itemize}
\item {Grp. gram.:m.}
\end{itemize}
\begin{itemize}
\item {Utilização:Ext.}
\end{itemize}
Principe ou governador de província na Índia muçulmana.
Indivíduo, que enriqueceu na Índia.
Indivíduo muito rico.
(Do indostano \textunderscore nauab\textunderscore )
\section{Nabada}
\begin{itemize}
\item {Grp. gram.:f.}
\end{itemize}
\begin{itemize}
\item {Proveniência:(De \textunderscore nabo\textunderscore )}
\end{itemize}
Guisado ou doce de cabeças de nabo.
Roda, que formam os quatro braços da fateixa.
\section{Nabal}
\begin{itemize}
\item {Grp. gram.:m.}
\end{itemize}
Terreno onde crescem nabos.
\section{Nabantino}
\begin{itemize}
\item {Grp. gram.:adj.}
\end{itemize}
\begin{itemize}
\item {Proveniência:(Do lat. \textunderscore Nabantia\textunderscore , n. p.)}
\end{itemize}
Relativo a Nabância ou ao Nabão.
Relativo á cidade de Tomar, junto da qual teve assento Nabância.
\section{Nábão}
\begin{itemize}
\item {Grp. gram.:m.}
\end{itemize}
\begin{itemize}
\item {Utilização:Ant.}
\end{itemize}
\begin{itemize}
\item {Proveniência:(Do lat. hyp. \textunderscore navulum\textunderscore  = lat. \textunderscore naulum?\textunderscore  Cp. \textunderscore nábulo\textunderscore )}
\end{itemize}
Imposto, pago por pescadores, para terem a faculdade de pescar em pôrto estranho, e que consistia num peixe por cada embarcação.
\section{Nabam}
\begin{itemize}
\item {Grp. gram.:m.}
\end{itemize}
\begin{itemize}
\item {Utilização:Ant.}
\end{itemize}
\begin{itemize}
\item {Proveniência:(Do lat. hyp. \textunderscore navulum\textunderscore  = lat. \textunderscore naulum?\textunderscore  Cp. \textunderscore nábulo\textunderscore )}
\end{itemize}
Imposto, pago por pescadores, para terem a faculdade de pescar em pôrto estranho, e que consistia num peixe por cada embarcação.
\section{Nabiça}
\begin{itemize}
\item {Grp. gram.:f.}
\end{itemize}
Rama do nabo que ainda não attingiu desenvolvimento completo.
O mesmo que \textunderscore nabo\textunderscore ^1, quando êste ainda não attingiu completo desenvolvimento.
\section{Nabiçal}
\begin{itemize}
\item {Grp. gram.:m.}
\end{itemize}
Terreno, onde nascem nabiças.
\section{Nabinha}
\begin{itemize}
\item {Grp. gram.:f.}
\end{itemize}
\begin{itemize}
\item {Utilização:Prov.}
\end{itemize}
Semente de nabos, couves, repolhos, etc.
\section{Náblio}
\begin{itemize}
\item {Grp. gram.:m.}
\end{itemize}
\begin{itemize}
\item {Proveniência:(Lat. \textunderscore nablium\textunderscore )}
\end{itemize}
Antigo instrumento de doze cordas, com caixa de resonância na parte superior:«\textunderscore ...o som do náblio sonoro\textunderscore ». Herculano, \textunderscore Harpa do Crente\textunderscore .
\section{Nablónio}
\begin{itemize}
\item {Grp. gram.:m.}
\end{itemize}
Gênero de plantas, da fam. das compostas.
\section{Nabo}
\begin{itemize}
\item {Grp. gram.:m.}
\end{itemize}
\begin{itemize}
\item {Grp. gram.:Pl. Loc.}
\end{itemize}
\begin{itemize}
\item {Utilização:Fig.}
\end{itemize}
\begin{itemize}
\item {Utilização:fam.}
\end{itemize}
\begin{itemize}
\item {Proveniência:(Do lat. \textunderscore napus\textunderscore )}
\end{itemize}
Planta crucífera, (\textunderscore brassica napus\textunderscore ).
A raiz dessa planta.

\textunderscore Tirar nabos do púcaro\textunderscore , investigar curiosamente, interrogando. Cf. Camillo, \textunderscore Estrêllas Fun.\textunderscore , 124.
\section{Nabo}
\begin{itemize}
\item {Grp. gram.:m.}
\end{itemize}
\begin{itemize}
\item {Utilização:Ant.}
\end{itemize}
O mesmo que \textunderscore nábulo\textunderscore .
\section{Nábulo}
\begin{itemize}
\item {Grp. gram.:m.}
\end{itemize}
\begin{itemize}
\item {Utilização:Ant.}
\end{itemize}
O mesmo que \textunderscore nábão\textunderscore . Cf. Viterbo, \textunderscore Elucid.\textunderscore 
\section{Naca}
\begin{itemize}
\item {Grp. gram.:f.}
\end{itemize}
\begin{itemize}
\item {Utilização:Bras}
\end{itemize}
O mesmo que \textunderscore nacada\textunderscore .
\section{Nacada}
\begin{itemize}
\item {Grp. gram.:f.}
\end{itemize}
\begin{itemize}
\item {Proveniência:(De \textunderscore naco\textunderscore )}
\end{itemize}
Pedaço; grande fatia; naco.
\section{Nação}
\begin{itemize}
\item {Grp. gram.:f.}
\end{itemize}
\begin{itemize}
\item {Grp. gram.:Pl.}
\end{itemize}
\begin{itemize}
\item {Proveniência:(Do lat. \textunderscore natio\textunderscore )}
\end{itemize}
Conjunto dos habitantes de um território, ligados por interesses communs, e considerados como pertencentes á mesma raça.
Região ou país, que se governa por leis próprias.
O Estado.
Naturalidade, procedência: \textunderscore Elias Costa, de nação judaica\textunderscore .
Grupo de indivíduos, com caracteres communs.
Casta.
Origem, nascimento: \textunderscore aquelle sujeito é ruim de nação\textunderscore . (Ouvido em Turquel)

Em linguagem bíblica, os gentios ou pagãos.
\section{Nácar}
\begin{itemize}
\item {Grp. gram.:m.}
\end{itemize}
\begin{itemize}
\item {Utilização:Ext.}
\end{itemize}
\begin{itemize}
\item {Proveniência:(Do pers. \textunderscore nakar\textunderscore )}
\end{itemize}
Substância branca e brilhante, que reveste interiormente várias conchas, e que tem a propriedade de refranger a luz por uma forma agradável á vista.
Côr de carmim, côr de rosa.
\section{Nacarado}
\begin{itemize}
\item {Grp. gram.:adj.}
\end{itemize}
Que tem o aspecto ou côr do nácar; róseo; acarminado.
\section{Nacarar}
\begin{itemize}
\item {Grp. gram.:v. t.}
\end{itemize}
\begin{itemize}
\item {Utilização:Fig.}
\end{itemize}
Dar aspecto ou côr de nácar a.
Cobrir de nácar.
Tornar rubro ou rosado: \textunderscore o pudor nacarou-lhe o rosto\textunderscore .
\section{Nacarino}
\begin{itemize}
\item {Grp. gram.:adj.}
\end{itemize}
O mesmo que \textunderscore nacarado\textunderscore .
\section{Naceja}
\begin{itemize}
\item {Grp. gram.:f.}
\end{itemize}
\begin{itemize}
\item {Utilização:Prov.}
\end{itemize}
\begin{itemize}
\item {Utilização:beir.}
\end{itemize}
O mesmo que \textunderscore narceja\textunderscore .
\section{Nacela}
\begin{itemize}
\item {Grp. gram.:f.}
\end{itemize}
\begin{itemize}
\item {Proveniência:(Do lat. \textunderscore navicella\textunderscore )}
\end{itemize}
Escócia; moldura côncava, na base de uma columna.
\section{Nachampô}
\begin{itemize}
\item {Grp. gram.:m.}
\end{itemize}
Planta indiana, de flôres amarelas, intensamente aromáticas. Cf. Th. Ribeiro, \textunderscore Jornadas\textunderscore , II, 107.
\section{Nachenim}
\begin{itemize}
\item {Grp. gram.:m.}
\end{itemize}
O mesmo que \textunderscore nachinim\textunderscore .
\section{Nachinim}
\begin{itemize}
\item {Grp. gram.:m.}
\end{itemize}
Planta cerealífera, da Índia, (\textunderscore eleusíne indica\textunderscore , Gaert.). Cf. Dalgado, \textunderscore Flora\textunderscore .
\section{Nacho}
\begin{itemize}
\item {Grp. gram.:adj.}
\end{itemize}
\begin{itemize}
\item {Utilização:Prov.}
\end{itemize}
\begin{itemize}
\item {Utilização:trasm.}
\end{itemize}
Que tem nariz chato.
(Cast. \textunderscore nacho\textunderscore , do lat. \textunderscore nar'e'llus\textunderscore  = \textunderscore nariculus\textunderscore , como \textunderscore sacho\textunderscore , de \textunderscore sarculus\textunderscore )
\section{Nacibo}
\begin{itemize}
\item {Grp. gram.:m.}
\end{itemize}
\begin{itemize}
\item {Utilização:Ant.}
\end{itemize}
\begin{itemize}
\item {Proveniência:(De \textunderscore nacer\textunderscore  = \textunderscore nascer\textunderscore ?)}
\end{itemize}
Destino ou sina, com que alguém nasce.
\section{Nacional}
\begin{itemize}
\item {Grp. gram.:adj.}
\end{itemize}
\begin{itemize}
\item {Grp. gram.:M.}
\end{itemize}
\begin{itemize}
\item {Proveniência:(Do lat. \textunderscore natio\textunderscore )}
\end{itemize}
Relativo a uma nação: \textunderscore brios nacionaes\textunderscore .
Que é de uma nação: \textunderscore trigo nacional\textunderscore .
Individuo, natural de um país, (por contraposição a estrangeiro).
\section{Nacionalidade}
\begin{itemize}
\item {Grp. gram.:f.}
\end{itemize}
\begin{itemize}
\item {Proveniência:(De \textunderscore nacional\textunderscore )}
\end{itemize}
Qualidade de nacional.
Origem nacional de uma coisa ou pessôa.
Naturalidade.
Nação; independência politica: \textunderscore a nossa nacionalidade\textunderscore .
\section{Nacionalista}
\begin{itemize}
\item {Grp. gram.:adj.}
\end{itemize}
\begin{itemize}
\item {Utilização:Neol.}
\end{itemize}
\begin{itemize}
\item {Proveniência:(De \textunderscore nacional\textunderscore )}
\end{itemize}
Relativo á independência e interesses nacionaes; patriótico.
\section{Nacionalismo}
\begin{itemize}
\item {Grp. gram.:m.}
\end{itemize}
\begin{itemize}
\item {Utilização:Neol.}
\end{itemize}
\begin{itemize}
\item {Proveniência:(De \textunderscore nacional\textunderscore )}
\end{itemize}
Patriotismo.
\section{Nacionalização}
\begin{itemize}
\item {Grp. gram.:f.}
\end{itemize}
Acto de nacionalizar.
\section{Nacionalizador}
\begin{itemize}
\item {Grp. gram.:m.}
\end{itemize}
Aquelle que nacionaliza.
\section{Nacionalizar}
\begin{itemize}
\item {Grp. gram.:v. t.}
\end{itemize}
Tornar nacional; dar carácter nacional a.
Naturalizar.
Aclimar.
\section{Nacionalmente}
\begin{itemize}
\item {Grp. gram.:adv.}
\end{itemize}
De modo nacional.
Em nome de um govêrno ou de uma nação.
\section{Naco}
\begin{itemize}
\item {Grp. gram.:m.}
\end{itemize}
Grande pedaço de pão.
Grande pedaço de qualquer coisa.
\section{Nacornim}
\begin{itemize}
\item {Grp. gram.:m.}
\end{itemize}
O mesmo que \textunderscore narcornim\textunderscore .
\section{Nacra}
\begin{itemize}
\item {Grp. gram.:f.}
\end{itemize}
\begin{itemize}
\item {Utilização:Ant.}
\end{itemize}
O mesmo que \textunderscore nácar\textunderscore . Cf. B. Pereira, \textunderscore Prosódia\textunderscore , vb. \textunderscore pínna\textunderscore .
\section{Nacre}
\begin{itemize}
\item {Grp. gram.:m.}
\end{itemize}
Fórma pop. de \textunderscore nácar\textunderscore .
\section{Nacrite}
\begin{itemize}
\item {Grp. gram.:m.}
\end{itemize}
\begin{itemize}
\item {Utilização:Geol.}
\end{itemize}
\begin{itemize}
\item {Proveniência:(Do fr. \textunderscore nacre\textunderscore )}
\end{itemize}
Uma das variedades mais importantes do caulim.
\section{Nada}
\begin{itemize}
\item {Grp. gram.:m.}
\end{itemize}
\begin{itemize}
\item {Grp. gram.:Pron. indef.}
\end{itemize}
\begin{itemize}
\item {Grp. gram.:Adv.}
\end{itemize}
\begin{itemize}
\item {Grp. gram.:Loc. conj.}
\end{itemize}
A não existência.
Effeito do aniquilamento.
Ausência de quantidade.
Bagatela.
Aquillo que não existe.
Coisa nulla; inutilidade.
Nenhuma coisa.
Não.
\textunderscore Nada menos\textunderscore , todavia, contudo. Cf. Filinto, \textunderscore D. Man.\textunderscore , I, 66 e 132.
(Da loc. lat. \textunderscore nulla res nata\textunderscore )
\section{Nadabau}
\begin{itemize}
\item {Grp. gram.:m.}
\end{itemize}
\begin{itemize}
\item {Utilização:T. de Aveiro}
\end{itemize}
Nome de várias espécies de lemnáceas, muito usadas no alimento dos patos.
\section{Nadadeiro}
\begin{itemize}
\item {Grp. gram.:adj.}
\end{itemize}
\begin{itemize}
\item {Utilização:P. us.}
\end{itemize}
O mesmo que \textunderscore nadador\textunderscore . Cf. Dom. Vieira, vb. \textunderscore procellárias\textunderscore .
\section{Nadador}
\begin{itemize}
\item {Grp. gram.:adj.}
\end{itemize}
\begin{itemize}
\item {Grp. gram.:M.}
\end{itemize}
\begin{itemize}
\item {Proveniência:(Do lat. \textunderscore natator\textunderscore )}
\end{itemize}
Que nada.
Que serve para nadar.
Aquelle que nada.
\section{Nadadura}
\begin{itemize}
\item {Grp. gram.:f.}
\end{itemize}
Acto ou effeito de \textunderscore nadar\textunderscore ; natação.
\section{Nadale}
\begin{itemize}
\item {Grp. gram.:m.}
\end{itemize}
\begin{itemize}
\item {Utilização:Ant.}
\end{itemize}
Espécie de anil.
\section{Nadante}
\begin{itemize}
\item {Grp. gram.:adj.}
\end{itemize}
\begin{itemize}
\item {Proveniência:(Do lat. \textunderscore natans\textunderscore )}
\end{itemize}
Que nada; que fluctua á tona da água.
\section{Nadar}
\begin{itemize}
\item {Grp. gram.:v. i.}
\end{itemize}
\begin{itemize}
\item {Utilização:Fig.}
\end{itemize}
\begin{itemize}
\item {Utilização:Gír.}
\end{itemize}
\begin{itemize}
\item {Grp. gram.:V. t.}
\end{itemize}
\begin{itemize}
\item {Utilização:Ant.}
\end{itemize}
\begin{itemize}
\item {Proveniência:(Do lat. \textunderscore natare\textunderscore )}
\end{itemize}
Sustentar-se e mover-se sôbre a água, pelo movimento de certas partes do corpo.
Fluctuar.
Estar immerso num liquido; estar molhado.
Engolfar-se suavemente, com prazer: \textunderscore nadar em delícias\textunderscore .
Justificar-se.
Percorrer a nado: \textunderscore nadar um rio\textunderscore .
\section{Nádega}
\begin{itemize}
\item {Grp. gram.:f.}
\end{itemize}
\begin{itemize}
\item {Grp. gram.:Pl.}
\end{itemize}
\begin{itemize}
\item {Proveniência:(Do lat. hyp. \textunderscore natica\textunderscore , de \textunderscore nates\textunderscore )}
\end{itemize}
Cada uma das partes carnudas e globosas, que formam a parte superior e traseira das coxas.
Parte carnuda, por baixo e atrás da garupa das cavalgaduras.
O assento.
\section{Nadegada}
\begin{itemize}
\item {Grp. gram.:f.}
\end{itemize}
O mesmo que \textunderscore nalgada\textunderscore .
\section{Nadegudo}
\begin{itemize}
\item {Grp. gram.:adj.}
\end{itemize}
Que tem grandes nádegas.
\section{Nadegueiro}
\begin{itemize}
\item {Grp. gram.:adj.}
\end{itemize}
Relativo a nádegas.
Situado nas nádegas.
\section{Nadichinha}
\begin{itemize}
\item {Grp. gram.:m.}
\end{itemize}
\begin{itemize}
\item {Utilização:Prov.}
\end{itemize}
O mesmo que \textunderscore nadinha\textunderscore .
\section{Nadinha}
\begin{itemize}
\item {Grp. gram.:m.}
\end{itemize}
Pequena porção de qualquer coisa.
Quási nada. Cf. Rebello, \textunderscore Contos e Lendas\textunderscore , 82 e 91.
\section{Nadir}
\begin{itemize}
\item {Grp. gram.:m.}
\end{itemize}
Ponto do céu, em que terminaria a vertical que se tirasse do ponto em que estamos e passasse pelo centro da Terra.
(B. lat. \textunderscore nadir\textunderscore , do ár. \textunderscore nazir\textunderscore )
\section{Nadiral}
\begin{itemize}
\item {Grp. gram.:adj.}
\end{itemize}
Relativo ao nadir.
\section{Nadível}
\begin{itemize}
\item {Grp. gram.:adj.}
\end{itemize}
\begin{itemize}
\item {Utilização:Des.}
\end{itemize}
Que se póde passar a nado. Cf. Barros, \textunderscore Déc.\textunderscore  I, 169.
(Do \textunderscore nado\textunderscore ^1)
\section{Nadível}
\begin{itemize}
\item {Grp. gram.:adj.}
\end{itemize}
\begin{itemize}
\item {Utilização:Des.}
\end{itemize}
O mesmo que \textunderscore nativo\textunderscore .
\section{Nadivo}
\begin{itemize}
\item {Grp. gram.:adj.}
\end{itemize}
O mesmo que \textunderscore nativo\textunderscore .
\section{Nado}
\begin{itemize}
\item {Grp. gram.:m.}
\end{itemize}
\begin{itemize}
\item {Grp. gram.:Loc. adv.}
\end{itemize}
Acto de nadar.
Espaço, que se póde percorrer, nadando.
\textunderscore A nado\textunderscore , nadando; á superfície da água: \textunderscore passar um rio a nado\textunderscore .
\section{Nado}
\begin{itemize}
\item {Grp. gram.:adj.}
\end{itemize}
\begin{itemize}
\item {Proveniência:(Lat. \textunderscore natus\textunderscore )}
\end{itemize}
O mesmo que \textunderscore nascido\textunderscore : \textunderscore o Octávio foi nado e criado em Lisbôa\textunderscore .
\section{Nafé}
\begin{itemize}
\item {Grp. gram.:m.}
\end{itemize}
O mesmo que \textunderscore quiabo\textunderscore .
\section{Náfega}
\begin{itemize}
\item {Grp. gram.:f.}
\end{itemize}
Macieira, o mesmo que \textunderscore anáfega\textunderscore .
Fruto dessa árvore. Cf. Pant. de Aveiro, \textunderscore Itiner.\textunderscore , 214, (2.^a ed.).
\section{Náfego}
\begin{itemize}
\item {Grp. gram.:adj.}
\end{itemize}
\begin{itemize}
\item {Grp. gram.:M.}
\end{itemize}
\begin{itemize}
\item {Utilização:Veter.}
\end{itemize}
Que tem quadril ou anca mais pequena que a outra, (falando-se do cavallo).
Fractura do osso eleon do cavallo, a qual lhe torna desiguaes os quadrís.
(Por \textunderscore náfrego\textunderscore , alter. de \textunderscore náufrago\textunderscore , seg. Car. Michaëlis)
\section{Nafo}
\begin{itemize}
\item {Grp. gram.:adj.}
\end{itemize}
\begin{itemize}
\item {Utilização:T. do Ribatejo}
\end{itemize}
Diz-se de um indivíduo, que tem um ombro descaído.
(Cp. \textunderscore náfego\textunderscore )
\section{Náfrico}
\begin{itemize}
\item {Grp. gram.:adj.}
\end{itemize}
\begin{itemize}
\item {Utilização:Prov.}
\end{itemize}
\begin{itemize}
\item {Utilização:trasm.}
\end{itemize}
Derreado de um quadril, (falando-se de cavallo ou jumento).
(Corr. de \textunderscore náfrego\textunderscore , se não fórma anterior)
\section{Naga}
\begin{itemize}
\item {Grp. gram.:f.}
\end{itemize}
O mesmo que \textunderscore naja\textunderscore .
\section{Nagalhé}
\begin{itemize}
\item {Grp. gram.:m.}
\end{itemize}
O mesmo ou melhor que \textunderscore lagalhé\textunderscore .
(Por \textunderscore negalhé\textunderscore , de \textunderscore negalho\textunderscore ?)
\section{Nagalho}
\begin{itemize}
\item {Grp. gram.:m.}
\end{itemize}
\begin{itemize}
\item {Utilização:Prov.}
\end{itemize}
\begin{itemize}
\item {Utilização:trasm.}
\end{itemize}
Lenço de pescoço, gravata.
O mesmo que \textunderscore negalho\textunderscore .
\section{Naga-musadio}
\begin{itemize}
\item {Grp. gram.:m.}
\end{itemize}
Árvore rubiácea da India.
\section{Nagana}
\begin{itemize}
\item {Grp. gram.:f.}
\end{itemize}
Doença sul-africana dos bois, cavallos, cães, etc., produzida pela mordedura do tsétsé, e que por isso também se chama \textunderscore doença da môsca tsétsé\textunderscore .
\section{Nagão}
\begin{itemize}
\item {Grp. gram.:m.}
\end{itemize}
Árvore do Malabar.
\section{Nagar}
\begin{itemize}
\item {Grp. gram.:m.}
\end{itemize}
Espécie de tambor indiano.
\section{Nagera}
\begin{itemize}
\item {Grp. gram.:f.}
\end{itemize}
\begin{itemize}
\item {Utilização:Prov.}
\end{itemize}
O mesmo que \textunderscore galleirão\textunderscore .
\section{Nagi}
\begin{itemize}
\item {Grp. gram.:m.}
\end{itemize}
Árvore japonesa, de fruto semelhante á cereja.
\section{Nagiagite}
\begin{itemize}
\item {Grp. gram.:f.}
\end{itemize}
\begin{itemize}
\item {Utilização:Miner.}
\end{itemize}
Espécie de mineral, que contém oiro, antimónio, enxôfre, chumbo e telúrio.
\section{Nágua}
\begin{itemize}
\item {Grp. gram.:f.}
\end{itemize}
\begin{itemize}
\item {Utilização:Prov.}
\end{itemize}
\begin{itemize}
\item {Utilização:trasm.}
\end{itemize}
O mesmo que \textunderscore anágua\textunderscore .
(Cp. gall. \textunderscore nagua\textunderscore )
\section{Nagual}
\begin{itemize}
\item {Grp. gram.:m.}
\end{itemize}
Aquelle que, segundo as sciências occultas, tem natureza commum á de outrem, ou que sente o que outrem sente.
\section{Nagual}
\begin{itemize}
\item {Grp. gram.:m.}
\end{itemize}
Feiticeiro ou necromante, entre os Índios.
\section{Nagualismo}
\begin{itemize}
\item {Grp. gram.:m.}
\end{itemize}
Solidariedade animal, ou qualidade do que é \textunderscore nagual\textunderscore ^1.
\section{Naguim}
\begin{itemize}
\item {Grp. gram.:m.}
\end{itemize}
Arvoreta indiana.
\section{Nagul}
\begin{itemize}
\item {Grp. gram.:m.}
\end{itemize}
Arvoreta da Índia portuguesa.
\section{Nagyagyte}
\begin{itemize}
\item {Grp. gram.:f.}
\end{itemize}
\begin{itemize}
\item {Utilização:Miner.}
\end{itemize}
Espécie de mineral, que contém oiro, antimónio, enxôfre, chumbo e tellúrio.
\section{Nai}
\begin{itemize}
\item {Grp. gram.:m.}
\end{itemize}
Instrumento de sopro, muito usado entre os Árabes, Persas e Turcos. Cf. \textunderscore Diccion. Mus.\textunderscore 
\section{Naia}
\begin{itemize}
\item {Grp. gram.:f.}
\end{itemize}
O mesmo que \textunderscore náiade\textunderscore . Cf. Filinto, V, 201.
\section{Naia}
\begin{itemize}
\item {Grp. gram.:f.}
\end{itemize}
\begin{itemize}
\item {Utilização:Gír.}
\end{itemize}
O mesmo que \textunderscore mãe\textunderscore .
(Gall. \textunderscore nay\textunderscore )
\section{Naiadáceas}
\begin{itemize}
\item {Grp. gram.:f. pl.}
\end{itemize}
O mesmo ou melhor que \textunderscore naiádeas\textunderscore .
\section{Náiade}
\begin{itemize}
\item {Grp. gram.:f.}
\end{itemize}
\begin{itemize}
\item {Grp. gram.:Pl.}
\end{itemize}
\begin{itemize}
\item {Proveniência:(Lat. \textunderscore naias\textunderscore , \textunderscore naiadis\textunderscore )}
\end{itemize}
Divindade inferior que, segundo o polytheísmo, presidia ás fonte e rios.
Nympha das águas.
Gênero de plantas aquáticas.
Gênero de vermes aquáticos ápodes.
Família de molluscos de água doce.
Espécie de aranhas, que mergulham na água.
\section{Naiádeas}
\begin{itemize}
\item {Grp. gram.:f. pl.}
\end{itemize}
\begin{itemize}
\item {Proveniência:(De \textunderscore naiádeo\textunderscore )}
\end{itemize}
Família de plantas, organizada por Jussieu, á qual os botânicos modernos foram retirando vários gêneros para constituírem outras famílias.
\section{Naiádeo}
\begin{itemize}
\item {Grp. gram.:adj.}
\end{itemize}
Relativo ou semelhante à náiade, planta.
\section{Naibe}
\begin{itemize}
\item {Grp. gram.:m.}
\end{itemize}
Superintendente das leis e da religião, nas Maldivas.
\section{Naibre}
\begin{itemize}
\item {Grp. gram.:m.}
\end{itemize}
\begin{itemize}
\item {Utilização:Ant.}
\end{itemize}
Fiscal de pesos, medidas e mercadorias no Egypto. Cf. F. Alvares, \textunderscore Preste João\textunderscore .
(O mesmo que \textunderscore naibe\textunderscore ?)
\section{Naifa}
\begin{itemize}
\item {Grp. gram.:f.}
\end{itemize}
\begin{itemize}
\item {Utilização:Gír.}
\end{itemize}
\begin{itemize}
\item {Proveniência:(Do ingl. \textunderscore knife\textunderscore )}
\end{itemize}
Navalha; faca.
\section{Naifada}
\begin{itemize}
\item {Grp. gram.:f.}
\end{itemize}
\begin{itemize}
\item {Utilização:Gír.}
\end{itemize}
Golpe de naifa.
\section{Naife}
\begin{itemize}
\item {Grp. gram.:adj.}
\end{itemize}
\begin{itemize}
\item {Utilização:Ant.}
\end{itemize}
(?):«\textunderscore ...muitos diamantes naifes de roca velha.\textunderscore »\textunderscore Peregrinação\textunderscore , XXXIX.
\section{Nafta}
\begin{itemize}
\item {Grp. gram.:f.}
\end{itemize}
\begin{itemize}
\item {Proveniência:(Lat. \textunderscore naphthas\textunderscore )}
\end{itemize}
Betume liquido, incolor, muito inflamável, volátil, de cheiro vivo e penetrante.
\section{Naftagil}
\begin{itemize}
\item {Grp. gram.:m.}
\end{itemize}
Espécie de betume natural.
\section{Naftalânio}
\begin{itemize}
\item {Grp. gram.:m.}
\end{itemize}
Substância verde-escura, aplicada contra as dermatoses.
\section{Naftalasa}
\begin{itemize}
\item {Grp. gram.:m.}
\end{itemize}
\begin{itemize}
\item {Utilização:Chím.}
\end{itemize}
Produto, obtido pela acção do cloro, bromo e iodo sôbre a naftalina.
\section{Naftalina}
\begin{itemize}
\item {Grp. gram.:f.}
\end{itemize}
\begin{itemize}
\item {Utilização:Chím.}
\end{itemize}
\begin{itemize}
\item {Proveniência:(De \textunderscore nafta\textunderscore )}
\end{itemize}
Substância, que se apresenta em fórma do grãos cristalinos ou lâminas romboédricas, friáveis, de brilho um tanto nacarado.
\section{Nafteína}
\begin{itemize}
\item {Grp. gram.:f.}
\end{itemize}
Substância mineral complexa, descoberta em alguns terrenos de França.
\section{Naftol}
\begin{itemize}
\item {Grp. gram.:m.}
\end{itemize}
\begin{itemize}
\item {Proveniência:(De \textunderscore nafta\textunderscore )}
\end{itemize}
Designação dos phenoes monoatómicos e diatómicos, derivados da naftalina.
\section{Naifo}
\begin{itemize}
\item {Grp. gram.:adj.}
\end{itemize}
\begin{itemize}
\item {Utilização:T. de Tôrres-Novas}
\end{itemize}
Diz-se do animal, que tem os pés tortos.
O mesmo que \textunderscore náfego\textunderscore .
\section{Naifrir}
\begin{itemize}
\item {Grp. gram.:v. i.}
\end{itemize}
\begin{itemize}
\item {Utilização:minh}
\end{itemize}
\begin{itemize}
\item {Utilização:Gír.}
\end{itemize}
Nadar.
\section{Naiófitas}
\begin{itemize}
\item {Grp. gram.:f. pl.}
\end{itemize}
\begin{itemize}
\item {Utilização:Bot.}
\end{itemize}
\begin{itemize}
\item {Proveniência:(Do gr. \textunderscore naias\textunderscore  + \textunderscore phuton\textunderscore )}
\end{itemize}
Nome, proposto de Guillon, em substituição de \textunderscore hidrofitas\textunderscore .
\section{Naióphytas}
\begin{itemize}
\item {Grp. gram.:f. pl.}
\end{itemize}
\begin{itemize}
\item {Utilização:Bot.}
\end{itemize}
\begin{itemize}
\item {Proveniência:(Do gr. \textunderscore naias\textunderscore  + \textunderscore phuton\textunderscore )}
\end{itemize}
Nome, proposto de Guillon, em substituição de \textunderscore hydrophytas\textunderscore .
\section{Naipe}
\begin{itemize}
\item {Grp. gram.:m.}
\end{itemize}
\begin{itemize}
\item {Utilização:Fig.}
\end{itemize}
Sinal, que distingue cada um dos quatro grupos de cartas de jogar.
Cada um dêsses grupos.
Condição, qualidade; igualha.
(Cast. \textunderscore naipe\textunderscore )
\section{Naipeiro}
\begin{itemize}
\item {Grp. gram.:adj.}
\end{itemize}
Relativo a naipe. Cf. Dom. Vieira, vb. \textunderscore trunfada\textunderscore .
\section{Naique}
\begin{itemize}
\item {Grp. gram.:m.}
\end{itemize}
\begin{itemize}
\item {Proveniência:(Do indost. \textunderscore naiak\textunderscore )}
\end{itemize}
Empregado inferior, espécie de contínuo, nas Repartições da antiga Índia portuguesa.
\section{Nairângia}
\begin{itemize}
\item {Grp. gram.:f.}
\end{itemize}
Espécie de adivinhação astrológica entre os Árabes, fundada nos diversos phenómenos do Sol e da Lua.
\section{Naire}
\begin{itemize}
\item {Grp. gram.:m.}
\end{itemize}
Militar nobre, entre os Índios do Malabar.
(Do malaialim \textunderscore naiar\textunderscore )
\section{Nais}
\begin{itemize}
\item {Grp. gram.:m.}
\end{itemize}
\begin{itemize}
\item {Utilização:Zool.}
\end{itemize}
Gênero de anélidos.
\section{Naja}
\begin{itemize}
\item {Grp. gram.:f.}
\end{itemize}
Serpente venenosa das regiões tropicaes.
Áspide; cuspideira.
\section{Najá}
\begin{itemize}
\item {Grp. gram.:f.}
\end{itemize}
Espécie de palmeira do Brasil.
\section{Nalagu}
\begin{itemize}
\item {Grp. gram.:m.}
\end{itemize}
Árvore do Malabar.
\section{Nale}
\begin{itemize}
\item {Grp. gram.:m.}
\end{itemize}
Antiga medida das ilhas de Maldiva, correspondente a pouco mais de meio litro.
\section{Nalga}
\begin{itemize}
\item {Grp. gram.:f.}
\end{itemize}
O mesmo que \textunderscore nádega\textunderscore . Cf. Castilho, \textunderscore Fausto\textunderscore , 356.
\section{Nalgada}
\begin{itemize}
\item {Grp. gram.:f.}
\end{itemize}
\begin{itemize}
\item {Utilização:P. us.}
\end{itemize}
\begin{itemize}
\item {Proveniência:(De \textunderscore nalga\textunderscore )}
\end{itemize}
Pancada nas nádegas, ou quéda sôbre as nádegas.
\section{Nam}
\begin{itemize}
\item {Grp. gram.:adv.}
\end{itemize}
O mesmo que \textunderscore não\textunderscore ^1.
\section{Nama}
\begin{itemize}
\item {Grp. gram.:f.}
\end{itemize}
Planta angolense, de fibras têxtis.
\section{Na-maciota}
\begin{itemize}
\item {Grp. gram.:loc. adv.}
\end{itemize}
\begin{itemize}
\item {Utilização:Bras}
\end{itemize}
\begin{itemize}
\item {Proveniência:(De \textunderscore macio\textunderscore )}
\end{itemize}
Com jeito.
Com lábia.
\section{Namáquas}
\begin{itemize}
\item {Grp. gram.:m. pl.}
\end{itemize}
Povo da África austro-occidental, tributário da Inglaterra.
\section{Namarraes}
\begin{itemize}
\item {Grp. gram.:m. pl.}
\end{itemize}
Povo selvagem e aguerrido da costa oriental da África, pertencente á grande raça dos Macuas.
\section{Namarrais}
\begin{itemize}
\item {Grp. gram.:m. pl.}
\end{itemize}
Povo selvagem e aguerrido da costa oriental da África, pertencente á grande raça dos Macuas.
\section{Nambi}
\begin{itemize}
\item {Grp. gram.:m.}
\end{itemize}
\begin{itemize}
\item {Utilização:Bras}
\end{itemize}
\begin{itemize}
\item {Grp. gram.:Adj.}
\end{itemize}
\begin{itemize}
\item {Utilização:Bras}
\end{itemize}
\begin{itemize}
\item {Proveniência:(T. tupi)}
\end{itemize}
Orelha.
Cavallo que tem uma orelha caída.
Que não tem orelhas ou que tem só uma.
\section{Nambicuaras}
\begin{itemize}
\item {Grp. gram.:m. pl.}
\end{itemize}
Tríbo de Índios do Brasil, nas margens do rio do Peixe, tributário do Tapajós.
\section{Nambu}
\begin{itemize}
\item {Grp. gram.:m.}
\end{itemize}
\begin{itemize}
\item {Utilização:Bras}
\end{itemize}
Planta, o mesmo que \textunderscore inhame\textunderscore .
Espécie de perdiz, de bico encarnado e sem rabo.
\section{Namburi}
\begin{itemize}
\item {Grp. gram.:m.}
\end{itemize}
Supremo sacerdote do Malabar.
\section{Namoração}
\begin{itemize}
\item {Grp. gram.:f.}
\end{itemize}
O mesmo que \textunderscore namôro\textunderscore .
\section{Namorada}
\begin{itemize}
\item {Grp. gram.:f.}
\end{itemize}
\begin{itemize}
\item {Utilização:Prov.}
\end{itemize}
\begin{itemize}
\item {Utilização:alent.}
\end{itemize}
\begin{itemize}
\item {Proveniência:(De \textunderscore namorado\textunderscore )}
\end{itemize}
Rapariga ou mulhér, que alguém namora ou galanteia.
Conversada.
O mesmo que \textunderscore carrapiço\textunderscore , por se dizer que um indivíduo tem tantas namoradas, quantos os carrapiços que se lhe pegaram ao fato.
\section{Namoradamente}
\begin{itemize}
\item {Grp. gram.:adv.}
\end{itemize}
Á maneira de namorado.
\section{Namoradeira}
\begin{itemize}
\item {Grp. gram.:f.}
\end{itemize}
Mulher ou rapariga, que gosta de namorar ou de sêr requestada.
Mulher, que tem muitos namorados.
(Fem. de \textunderscore namoradeiro\textunderscore )
\section{Namoradeiro}
\begin{itemize}
\item {Grp. gram.:m. e adj.}
\end{itemize}
O que faz ou acceita galanteios facilmente; galanteador; o que namora muito.
\section{Namoradiço}
\begin{itemize}
\item {Grp. gram.:adj.}
\end{itemize}
(V.namoradeiro)
\section{Namorado}
\begin{itemize}
\item {Grp. gram.:adj.}
\end{itemize}
\begin{itemize}
\item {Grp. gram.:M.}
\end{itemize}
\begin{itemize}
\item {Utilização:Bot.}
\end{itemize}
\begin{itemize}
\item {Utilização:Bras}
\end{itemize}
Galanteado.
Propício ao amor.
Próprio de amantes.
Meigo, amoroso; apaixonado.
Amatório.
Aquelle que é requestado.
Aquelle que requesta.
Amante.
Fruto do verbasco.
Espécie de peixe.
\section{Namorador}
\begin{itemize}
\item {Grp. gram.:m.  e  adj.}
\end{itemize}
O que namora.
Galanteador; namoradeiro.
\section{Namoramento}
\begin{itemize}
\item {Grp. gram.:m.}
\end{itemize}
(V.namôro)
\section{Namorar}
\begin{itemize}
\item {Grp. gram.:v. t.}
\end{itemize}
\begin{itemize}
\item {Utilização:Prov.}
\end{itemize}
\begin{itemize}
\item {Utilização:dur.}
\end{itemize}
\begin{itemize}
\item {Utilização:minh.}
\end{itemize}
\begin{itemize}
\item {Grp. gram.:V. i.}
\end{itemize}
\begin{itemize}
\item {Grp. gram.:V. p.}
\end{itemize}
Requestar, pretender o amor de.
Galantear.
Cativar; attrahir: \textunderscore namora-me aquelle jardim\textunderscore .
Desejar muito.
Empregar esforços para obter.
Desflorar (uma donzella).
Fazer galanteios amorosos.
Procurar inspirar amor.
Deixar-se dominar por uma affeição.
Estar encantado; apaixonar-se.
(Aphér. de \textunderscore enamorar\textunderscore )
\section{Namoricar}
\begin{itemize}
\item {Grp. gram.:v. t.}
\end{itemize}
\begin{itemize}
\item {Proveniência:(De \textunderscore namorico\textunderscore )}
\end{itemize}
Requestar levianamente.
Namorar por pouco tempo.
\section{Namorichar}
\begin{itemize}
\item {Grp. gram.:v. t.}
\end{itemize}
\begin{itemize}
\item {Utilização:Prov.}
\end{itemize}
\begin{itemize}
\item {Utilização:trasm.}
\end{itemize}
O mesmo que \textunderscore namoricar\textunderscore .
\section{Namoricho}
\begin{itemize}
\item {Grp. gram.:m.}
\end{itemize}
\begin{itemize}
\item {Utilização:Prov.}
\end{itemize}
\begin{itemize}
\item {Utilização:trasm.}
\end{itemize}
O mesmo que \textunderscore namorico\textunderscore .
\section{Namorico}
\begin{itemize}
\item {Grp. gram.:m.}
\end{itemize}
Namôro passageiro; galanteio por distracção.
\section{Namoriscar}
\begin{itemize}
\item {Grp. gram.:v. t.}
\end{itemize}
O mesmo que \textunderscore namoricar\textunderscore . Cf. Camillo, \textunderscore Doze Casam.\textunderscore , 39.
\section{Namorisco}
\begin{itemize}
\item {Grp. gram.:m.}
\end{itemize}
O mesmo que \textunderscore namorico\textunderscore .
\section{Namorismo}
\begin{itemize}
\item {Grp. gram.:m.}
\end{itemize}
\begin{itemize}
\item {Utilização:Fam.}
\end{itemize}
Táctica de namorar.
\section{Namorista}
\begin{itemize}
\item {Grp. gram.:m.}
\end{itemize}
O mesmo que \textunderscore namorador\textunderscore .
\section{Namôro}
\begin{itemize}
\item {Grp. gram.:m.}
\end{itemize}
Acto de namorar.
Galanteio.
Pessôa namorada: \textunderscore conversava com o namôro\textunderscore .
\section{Namoxim}
\begin{itemize}
\item {Grp. gram.:m.}
\end{itemize}
Fruição de propriedades, que pertenciam aos Jesuítas, em Gôa.
\section{Namunu}
\begin{itemize}
\item {Grp. gram.:m.}
\end{itemize}
Arvore de Moçambique.
\section{Nana}
\begin{itemize}
\item {Grp. gram.:f.}
\end{itemize}
\begin{itemize}
\item {Utilização:Prov.}
\end{itemize}
\begin{itemize}
\item {Grp. gram.:Interj.}
\end{itemize}
\begin{itemize}
\item {Utilização:pop.}
\end{itemize}
\begin{itemize}
\item {Proveniência:(It. \textunderscore nanna\textunderscore )}
\end{itemize}
Canto para acalentar.
O mesmo que \textunderscore boneca\textunderscore .
\textunderscore Pois nana\textunderscore ! pois não!
\section{Nana}
\begin{itemize}
\item {Grp. gram.:f.}
\end{itemize}
Planta americana, cujo fruto é de sabor parecido ao da pêra.
\section{Naná}
\begin{itemize}
\item {Grp. gram.:m.}
\end{itemize}
\begin{itemize}
\item {Utilização:Bras}
\end{itemize}
\begin{itemize}
\item {Proveniência:(T. tupi)}
\end{itemize}
O mesmo que \textunderscore ananás\textunderscore .
\section{Nanacuru}
\begin{itemize}
\item {Grp. gram.:m.}
\end{itemize}
\begin{itemize}
\item {Utilização:Bras}
\end{itemize}
Espécie de cacto.
\section{Nanai}
\begin{itemize}
\item {Grp. gram.:adv.}
\end{itemize}
\begin{itemize}
\item {Utilização:Gír.}
\end{itemize}
Nada.
(Or. ind.)
\section{Nanal}
\begin{itemize}
\item {Grp. gram.:m.}
\end{itemize}
Espécie de roseira indiana.
\section{Nanan}
\begin{itemize}
\item {Grp. gram.:f.}
\end{itemize}
\begin{itemize}
\item {Utilização:Bras}
\end{itemize}
O mesmo que \textunderscore nhanhan\textunderscore .
\section{Nanante}
\begin{itemize}
\item {Grp. gram.:f.}
\end{itemize}
Gênero de plantas herbáceas, marítimas.
\section{Nanar}
\begin{itemize}
\item {Grp. gram.:v. i.}
\end{itemize}
\begin{itemize}
\item {Utilização:Infant.}
\end{itemize}
\begin{itemize}
\item {Proveniência:(De \textunderscore nana\textunderscore . Cp. \textunderscore ninar\textunderscore )}
\end{itemize}
O mesmo que \textunderscore dormir\textunderscore .
\section{Nanceato}
\begin{itemize}
\item {Grp. gram.:m.}
\end{itemize}
\begin{itemize}
\item {Utilização:Chím.}
\end{itemize}
Designação antiga do lactato.
\section{Nanceico}
\begin{itemize}
\item {Grp. gram.:adj.}
\end{itemize}
Diz-se de um ácido, que se fórma durante a fermentação de matérias vegetaes.
\section{Nancíbea}
\begin{itemize}
\item {Grp. gram.:f.}
\end{itemize}
Planta rubiácea do Brasil.
\section{Nandapôa}
\begin{itemize}
\item {Grp. gram.:f.}
\end{itemize}
Espécie de cegonha do Brasil.
\section{Nandina}
\begin{itemize}
\item {Grp. gram.:f.}
\end{itemize}
\begin{itemize}
\item {Proveniência:(Do jap. \textunderscore nandin\textunderscore )}
\end{itemize}
Gênero de plantas asiáticas, berberídeas.
\section{Nandiroba}
\begin{itemize}
\item {Grp. gram.:f.}
\end{itemize}
Arbusto cucurbitáceo do Brasil, (\textunderscore fevillea nhandiroba\textunderscore ).
\section{Nandiróbeas}
\begin{itemize}
\item {Grp. gram.:f. pl.}
\end{itemize}
Família de plantas, que tem por typo a nandiroba, no systema de Saint-Hilaire.
Segundo De-Candolle, tríbo de cucurbitáceas.
\section{Nandu}
\begin{itemize}
\item {Grp. gram.:m.}
\end{itemize}
Abestruz americano.
\section{Nanga}
\begin{itemize}
\item {Grp. gram.:m.}
\end{itemize}
\begin{itemize}
\item {Utilização:T. da Áfr. Or. Port}
\end{itemize}
Curandeiro.
\section{Nangone}
\begin{itemize}
\item {Grp. gram.:m.}
\end{itemize}
Arbusto africano, de fôlhas alternas, compostas, e flôres miúdas, papilionáceas, roxas.
\section{Nangor}
\begin{itemize}
\item {Grp. gram.:m.}
\end{itemize}
Mammífero ruminante, espécie de antílope.
\section{Nangoro}
\begin{itemize}
\item {Grp. gram.:m.}
\end{itemize}
\begin{itemize}
\item {Utilização:T. da Índia port}
\end{itemize}
Espécie de arado.
\section{Nangra}
\begin{itemize}
\item {Grp. gram.:f.}
\end{itemize}
\begin{itemize}
\item {Utilização:Prov.}
\end{itemize}
\begin{itemize}
\item {Utilização:trasm.}
\end{itemize}
O mesmo que \textunderscore boneca\textunderscore .
\section{Nanguina}
\begin{itemize}
\item {Grp. gram.:f.}
\end{itemize}
\begin{itemize}
\item {Utilização:Bras}
\end{itemize}
Tecido de algodão, que se fabrica na Inglaterra, e se reexporta para África.
\section{Nani}
\begin{itemize}
\item {Grp. gram.:m.}
\end{itemize}
Arvore, notável pela dureza da sua madeira.
\section{Nanica}
\begin{itemize}
\item {Grp. gram.:f.}
\end{itemize}
\begin{itemize}
\item {Utilização:Bras. do Rio}
\end{itemize}
\begin{itemize}
\item {Proveniência:(De \textunderscore nanico\textunderscore )}
\end{itemize}
Gallinha de raça pequena.
\section{Nanico}
\begin{itemize}
\item {Grp. gram.:adj.}
\end{itemize}
\begin{itemize}
\item {Proveniência:(Do lat. \textunderscore nanus\textunderscore )}
\end{itemize}
Achanado; apoucado.
Que tem pequeno corpo.
\section{Nanisco}
\begin{itemize}
\item {Grp. gram.:m.}
\end{itemize}
Gênero de insectos lamellicórneos.
\section{Nanismo}
\begin{itemize}
\item {Grp. gram.:m.}
\end{itemize}
\begin{itemize}
\item {Proveniência:(Do lat. \textunderscore nanus\textunderscore )}
\end{itemize}
Estado ou defeito de anão.
\section{Nanja}
\begin{itemize}
\item {Grp. gram.:adv.}
\end{itemize}
\begin{itemize}
\item {Utilização:Pop.}
\end{itemize}
\begin{itemize}
\item {Proveniência:(De \textunderscore não\textunderscore  + \textunderscore já\textunderscore )}
\end{itemize}
Não; mais não.
Nunca.
\section{Nanna}
\begin{itemize}
\item {Grp. gram.:f.}
\end{itemize}
Planta americana, cujo fruto é de sabor parecido ao da pêra.
\section{Nanó}
\begin{itemize}
\item {Grp. gram.:m.}
\end{itemize}
Árvore indiana, espécie de mareta.
\section{Nanocefalia}
\begin{itemize}
\item {Grp. gram.:f.}
\end{itemize}
\begin{itemize}
\item {Proveniência:(De \textunderscore nanocéfalo\textunderscore )}
\end{itemize}
O mesmo que \textunderscore microcefalia\textunderscore .
\section{Nanocéfalo}
\begin{itemize}
\item {Grp. gram.:m.}
\end{itemize}
\begin{itemize}
\item {Proveniência:(Do gr. \textunderscore nanos\textunderscore  + \textunderscore kephale\textunderscore )}
\end{itemize}
Aquele que tem nanocefalia.
\section{Nanocephalia}
\begin{itemize}
\item {Grp. gram.:f.}
\end{itemize}
\begin{itemize}
\item {Proveniência:(De \textunderscore nanocéphalo\textunderscore )}
\end{itemize}
O mesmo que \textunderscore microcephalia\textunderscore .
\section{Nanocéphalo}
\begin{itemize}
\item {Grp. gram.:m.}
\end{itemize}
\begin{itemize}
\item {Proveniência:(Do gr. \textunderscore nanos\textunderscore  + \textunderscore kephale\textunderscore )}
\end{itemize}
Aquelle que tem nanocephalia.
\section{Nanocormia}
\begin{itemize}
\item {Grp. gram.:f.}
\end{itemize}
\begin{itemize}
\item {Proveniência:(Do gr. \textunderscore nanos\textunderscore  + \textunderscore kormos\textunderscore )}
\end{itemize}
Pequenez anómala do tronco humano.
\section{Nanódea}
\begin{itemize}
\item {Grp. gram.:f.}
\end{itemize}
Gênero de plantas santaláceas, originárias do Brasil.
\section{Nanófia}
\begin{itemize}
\item {Grp. gram.:f.}
\end{itemize}
Gênero de insectos neurópteros.
Gênero de insectos coleópteros tetrâmeros.
\section{Nanomelia}
\begin{itemize}
\item {Grp. gram.:f.}
\end{itemize}
\begin{itemize}
\item {Proveniência:(Do gr. \textunderscore nanos\textunderscore  + \textunderscore melos\textunderscore )}
\end{itemize}
Pequenez anómala dos membros do corpo humano.
\section{Nanos}
\begin{itemize}
\item {Grp. gram.:m. pl.}
\end{itemize}
Serranos de Angola.
\section{Nanquim}
\begin{itemize}
\item {Grp. gram.:m.}
\end{itemize}
\begin{itemize}
\item {Proveniência:(De \textunderscore Nanquim\textunderscore , n. p.)}
\end{itemize}
Tecido de algodão ou ganga amarela, que vinha antigamente da China.
Tinta preta, procedente da China, e que se usa em desenhos e aguarelas.
\section{Nantilda}
\begin{itemize}
\item {Grp. gram.:f.}
\end{itemize}
Gênero de insectos lepidópteros nocturnos.
\section{Não}
\begin{itemize}
\item {Grp. gram.:adv.}
\end{itemize}
\begin{itemize}
\item {Proveniência:(Do lat. \textunderscore non\textunderscore )}
\end{itemize}
Partícula negativa, opposta a \textunderscore sim\textunderscore .
De maneira nenhuma.
Emprega-se também como partícula expletiva, sem significação:«\textunderscore antes que o rico não escape das mãos da morte, se escapam os bens das mãos dos ricos\textunderscore ». Bernárdez, \textunderscore Floresta\textunderscore , I, 125.
\section{Não}
\begin{itemize}
\item {Proveniência:(Do lat. \textunderscore nam\textunderscore )}
\end{itemize}
Partícula interjeccional ou interrogativa, com a significação de \textunderscore porventura\textunderscore , segundo doutrina acceitável de Castilho:«\textunderscore que piedade não mostrou a bella Maria!\textunderscore »Reis Quita.«\textunderscore Quantos italianismos e castelhanismos não introduziu Camões nos seus poemas\textunderscore ». Latino, \textunderscore Elog. Acad\textunderscore . Cf. Figueiredo, \textunderscore Lições Prát.\textunderscore , vol. I.
\section{Não-eu}
\begin{itemize}
\item {Grp. gram.:m.}
\end{itemize}
\begin{itemize}
\item {Utilização:Philos.}
\end{itemize}
Realidade objectiva; o mundo externo; tudo que não é princípio racional.
\section{Não-filha}
\begin{itemize}
\item {Grp. gram.:f.}
\end{itemize}
\begin{itemize}
\item {Utilização:Prov.}
\end{itemize}
\begin{itemize}
\item {Utilização:beir.}
\end{itemize}
O mesmo que \textunderscore enteada\textunderscore .
\section{Não-filho}
\begin{itemize}
\item {Grp. gram.:m.}
\end{itemize}
\begin{itemize}
\item {Utilização:Prov.}
\end{itemize}
\begin{itemize}
\item {Utilização:beir.}
\end{itemize}
O mesmo que \textunderscore enteado\textunderscore .
\section{Não-me-deixes}
\begin{itemize}
\item {Grp. gram.:m.}
\end{itemize}
Planta, da fam. das compostas, (\textunderscore senecio elegans\textunderscore ).
\section{Não-sei-que-diga}
\begin{itemize}
\item {Grp. gram.:m.}
\end{itemize}
\begin{itemize}
\item {Utilização:Bras. do N}
\end{itemize}
O mesmo que \textunderscore diabo\textunderscore : \textunderscore tentou-me não-sei-que-diga\textunderscore .
\section{Não-sêr}
\begin{itemize}
\item {Grp. gram.:m.}
\end{itemize}
A não existência; o nada. Cf. Herculano, \textunderscore Hist. de Port.\textunderscore , I, 260.
\section{Não-te-esqueças}
\begin{itemize}
\item {Grp. gram.:m.}
\end{itemize}
Planta lithospérmea, variedade de myosota, (\textunderscore miosotis Welwitschii\textunderscore , Bss.). Cf. P. Coutinho, \textunderscore Flora de Port.\textunderscore , 497.
\section{Napáceo}
\begin{itemize}
\item {Grp. gram.:adj.}
\end{itemize}
\begin{itemize}
\item {Utilização:Bot.}
\end{itemize}
\begin{itemize}
\item {Proveniência:(Do lat. \textunderscore napus\textunderscore )}
\end{itemize}
Diz-se das raízes, que têm a fórma de cabeça de nabo.
\section{Napéas}
\begin{itemize}
\item {Grp. gram.:f. pl.}
\end{itemize}
\begin{itemize}
\item {Proveniência:(Do gr. \textunderscore napaios\textunderscore )}
\end{itemize}
Nymphas dos bosques; drýades.
\section{Napeias}
\begin{itemize}
\item {Grp. gram.:f. pl.}
\end{itemize}
\begin{itemize}
\item {Proveniência:(Do gr. \textunderscore napaios\textunderscore )}
\end{itemize}
Nymphas dos bosques; drýades.
\section{Napeiro}
\begin{itemize}
\item {Grp. gram.:adj.}
\end{itemize}
Indolente; dorminhoco.
\section{Napelina}
\begin{itemize}
\item {Grp. gram.:f.}
\end{itemize}
Substância narcótica, extrahida do napello.
\section{Napellina}
\begin{itemize}
\item {Grp. gram.:f.}
\end{itemize}
Substância narcótica, extrahida do napello.
\section{Napello}
\begin{itemize}
\item {fónica:pê}
\end{itemize}
\begin{itemize}
\item {Grp. gram.:m.}
\end{itemize}
\begin{itemize}
\item {Proveniência:(Lat. \textunderscore napellus\textunderscore )}
\end{itemize}
Planta ranunculácea, muito venenosa, espécie de acónito, (\textunderscore aconitum napellus\textunderscore ).
\section{Napelo}
\begin{itemize}
\item {fónica:pê}
\end{itemize}
\begin{itemize}
\item {Grp. gram.:m.}
\end{itemize}
\begin{itemize}
\item {Proveniência:(Lat. \textunderscore napellus\textunderscore )}
\end{itemize}
Planta ranunculácea, muito venenosa, espécie de acónito, (\textunderscore aconitum napellus\textunderscore ).
\section{Napeva}
\begin{itemize}
\item {Grp. gram.:adj.}
\end{itemize}
\begin{itemize}
\item {Utilização:Bras}
\end{itemize}
Nanico; que tem pernas curtas, (falando-se do gallo ou da gallinha).
\section{Naphtha}
\begin{itemize}
\item {Grp. gram.:f.}
\end{itemize}
\begin{itemize}
\item {Proveniência:(Lat. \textunderscore naphthas\textunderscore )}
\end{itemize}
Betume liquido, incolor, muito inflamável, volátil, de cheiro vivo e penetrante.
\section{Naphthagil}
\begin{itemize}
\item {Grp. gram.:m.}
\end{itemize}
Espécie de betume natural.
\section{Naphthalânio}
\begin{itemize}
\item {Grp. gram.:m.}
\end{itemize}
Substância verde-escura, applicada contra as dermatoses.
\section{Naphthalasa}
\begin{itemize}
\item {Grp. gram.:m.}
\end{itemize}
\begin{itemize}
\item {Utilização:Chím.}
\end{itemize}
Produto, obtido pela acção do chloro, bromo e iodo sôbre a naphtalina.
\section{Naphthalina}
\begin{itemize}
\item {Grp. gram.:f.}
\end{itemize}
\begin{itemize}
\item {Utilização:Chím.}
\end{itemize}
\begin{itemize}
\item {Proveniência:(De \textunderscore naphta\textunderscore )}
\end{itemize}
Substância, que se apresenta em fórma do grãos crystallinos ou lâminas rhomboédricas, friáveis, de brilho um tanto nacarado.
\section{Naphtheína}
\begin{itemize}
\item {Grp. gram.:f.}
\end{itemize}
Substância mineral complexa, descoberta em alguns terrenos de França.
\section{Naphthol}
\begin{itemize}
\item {Grp. gram.:m.}
\end{itemize}
\begin{itemize}
\item {Proveniência:(De \textunderscore naphta\textunderscore )}
\end{itemize}
Designação dos phenoes monoatómicos e diatómicos, derivados da naphtalina.
\section{Napiforme}
\begin{itemize}
\item {Grp. gram.:adj.}
\end{itemize}
\begin{itemize}
\item {Proveniência:(Do lat. \textunderscore napus\textunderscore  + \textunderscore forma\textunderscore )}
\end{itemize}
Que tem fórma de cabeça de nabo; napáceo.
\section{Napistas}
\begin{itemize}
\item {Grp. gram.:m.}
\end{itemize}
Os Gregos, que são partidários da Rússia.
\section{Napófito}
\begin{itemize}
\item {Grp. gram.:m.}
\end{itemize}
Gênero de plantas chenopodiáceas.
\section{Napoleão}
\begin{itemize}
\item {Grp. gram.:m.}
\end{itemize}
\begin{itemize}
\item {Proveniência:(De \textunderscore Napoleão\textunderscore , n. p.)}
\end{itemize}
Moéda francesa de oiro, equivalente a vinte francos.
Moéda francesa de prata equivalente a 5 francos.
\section{Napoleónea}
\begin{itemize}
\item {Grp. gram.:f.}
\end{itemize}
Gênero de plantas estyráceas, (\textunderscore napoleona africana\textunderscore , Beauv.).
\section{Napoleónico}
\begin{itemize}
\item {Grp. gram.:adj.}
\end{itemize}
\begin{itemize}
\item {Grp. gram.:M.}
\end{itemize}
Relativo a Napoleão ou ao seu systema politico e militar.
Partidário napoleonismo; bonapartista.
\section{Napoleonismo}
\begin{itemize}
\item {Grp. gram.:m.}
\end{itemize}
Partido político, que tem ou tinha por chefe Napoleão ou um príncipe da sua familia.
\section{Napoleonista}
\begin{itemize}
\item {Grp. gram.:m.}
\end{itemize}
Sectário do napoleonismo.
\section{Napoleonite}
\begin{itemize}
\item {Grp. gram.:f.}
\end{itemize}
\begin{itemize}
\item {Utilização:Miner.}
\end{itemize}
Espécie de rocha da Córsega.
\section{Napolês}
\begin{itemize}
\item {Grp. gram.:m. e adj.}
\end{itemize}
\begin{itemize}
\item {Utilização:Des.}
\end{itemize}
O mesmo que \textunderscore napolitano\textunderscore :«\textunderscore ...estoutros pintam napoleses de cabello doce...\textunderscore »\textunderscore Eufrosina\textunderscore , act. I, sc. 2.
\section{Napolitano}
\begin{itemize}
\item {Grp. gram.:adj.}
\end{itemize}
\begin{itemize}
\item {Grp. gram.:M.}
\end{itemize}
\begin{itemize}
\item {Proveniência:(Lat. \textunderscore neapolitanus\textunderscore )}
\end{itemize}
Relativo a Nápoles.
Habitante de Napoles.
\section{Napopé}
\begin{itemize}
\item {Grp. gram.:f.}
\end{itemize}
Ave columbina do Brasil.
\section{Napóphito}
\begin{itemize}
\item {Grp. gram.:m.}
\end{itemize}
Gênero de plantas chenopodiáceas.
\section{Naquela}
Flexão fem. de naquele.
\section{Naquele}
\begin{itemize}
\item {fónica:quê}
\end{itemize}
Expressão contraida, equivalente a \textunderscore em aquele\textunderscore .
(Cp. \textunderscore no\textunderscore ^1)
\section{Naquella}
Flexão fem. de naquelle.
\section{Naquelle}
\begin{itemize}
\item {fónica:quê}
\end{itemize}
Expressão contrahida, equivalente a \textunderscore em aquelle\textunderscore .
(Cp. no^1)
\section{Naquillo}
Expressão contrahida equivalente a \textunderscore em aquillo\textunderscore .
(Cp. \textunderscore no\textunderscore ^1)
\section{Naquilo}
Expressão contraida equivalente a \textunderscore em aquilo\textunderscore .
(Cp. \textunderscore no\textunderscore ^1)
\section{Narancharia}
\begin{itemize}
\item {Grp. gram.:f.}
\end{itemize}
\begin{itemize}
\item {Utilização:Ant.}
\end{itemize}
O mesmo que \textunderscore laranjal\textunderscore .
(Por \textunderscore naranjaria\textunderscore , do cast. \textunderscore naranja\textunderscore , laranja, do ar. \textunderscore naranj\textunderscore .)
\section{Narbonense}
\begin{itemize}
\item {Grp. gram.:m. e adj.}
\end{itemize}
\begin{itemize}
\item {Grp. gram.:M.}
\end{itemize}
\begin{itemize}
\item {Proveniência:(De \textunderscore Narbona\textunderscore , n. p.)}
\end{itemize}
Relativo a Narbona.
Habitante de Narbona.
\section{Narbonês}
\begin{itemize}
\item {Grp. gram.:adj.}
\end{itemize}
\begin{itemize}
\item {Grp. gram.:M.}
\end{itemize}
\begin{itemize}
\item {Proveniência:(De \textunderscore Narbona\textunderscore , n. p.)}
\end{itemize}
Relativo a Narbona.
Habitante de Narbona.
\section{Narcafto}
Casca aromática da árvore do incenso, usada em farmácia.
\section{Narcaphto}
\begin{itemize}
\item {Grp. gram.:m.}
\end{itemize}
Casca aromática da árvore do incenso, usada em pharmácia.
\section{Narcapto}
\begin{itemize}
\item {Grp. gram.:m.}
\end{itemize}
Planta indiana, semelhante á figueira brava.
\section{Narceína}
\begin{itemize}
\item {Grp. gram.:f.}
\end{itemize}
\begin{itemize}
\item {Proveniência:(Do gr. \textunderscore narke\textunderscore )}
\end{itemize}
Substância, descoberta no ópio.
\section{Narceja}
\begin{itemize}
\item {Grp. gram.:f.}
\end{itemize}
Ave pernalta dos campos.
\section{Narcisar-se}
\begin{itemize}
\item {Grp. gram.:v. p.}
\end{itemize}
\begin{itemize}
\item {Proveniência:(De \textunderscore Narciso\textunderscore , n. p. myth)}
\end{itemize}
Rever-se na sua beleza ou nos seus méritos; desvanecer-se; envaidar-se.
\section{Narciso}
\begin{itemize}
\item {Grp. gram.:m.}
\end{itemize}
\begin{itemize}
\item {Proveniência:(Lat. \textunderscore narcissus\textunderscore )}
\end{itemize}
Gênero de plantas, da fam. das amaryllídeas.
\section{Narcísseas}
\begin{itemize}
\item {Grp. gram.:f. pl.}
\end{itemize}
\begin{itemize}
\item {Proveniência:(Do lat. \textunderscore narcissus\textunderscore )}
\end{itemize}
Família de plantas cujos gêneros fôram repartidos por outras fam., as liliáceas, amaryllídeas e bromeliáceas.
\section{Narcissina}
\begin{itemize}
\item {Grp. gram.:f.}
\end{itemize}
\begin{itemize}
\item {Utilização:Chím.}
\end{itemize}
\begin{itemize}
\item {Proveniência:(Do lat. \textunderscore narcissus\textunderscore )}
\end{itemize}
Substância branca, que se encontra nos narcisos.
\section{Narcisso}
\begin{itemize}
\item {Grp. gram.:m.}
\end{itemize}
(V.narciso)
\section{Narcissoide}
\begin{itemize}
\item {Grp. gram.:adj.}
\end{itemize}
\begin{itemize}
\item {Utilização:Bot.}
\end{itemize}
\begin{itemize}
\item {Grp. gram.:F. pl.}
\end{itemize}
\begin{itemize}
\item {Proveniência:(Do gr. \textunderscore narkissos\textunderscore  + \textunderscore eidos\textunderscore )}
\end{itemize}
Semelhante ao narciso.
As differentes espécies de narcisos.
\section{Narcolepsia}
\begin{itemize}
\item {Grp. gram.:f.}
\end{itemize}
\begin{itemize}
\item {Proveniência:(Do gr. \textunderscore narkos\textunderscore  + \textunderscore lepris\textunderscore )}
\end{itemize}
Tendência irresistível para o somno, manifestada como symptoma de algumas moléstias.
\section{Narcornim}
\begin{itemize}
\item {Grp. gram.:m.}
\end{itemize}
\begin{itemize}
\item {Utilização:T. da Índia port}
\end{itemize}
Escrivão ou secretário de corporação provincial.
(Do sânscr.)
\section{Narcose}
\begin{itemize}
\item {Grp. gram.:m.}
\end{itemize}
\begin{itemize}
\item {Proveniência:(Gr. \textunderscore narkosis\textunderscore )}
\end{itemize}
O mesmo que \textunderscore narcotismo\textunderscore .
O mesmo que \textunderscore anestesia\textunderscore .
\section{Narcótico}
\begin{itemize}
\item {Grp. gram.:adj.}
\end{itemize}
\begin{itemize}
\item {Grp. gram.:M.}
\end{itemize}
\begin{itemize}
\item {Utilização:Fig.}
\end{itemize}
\begin{itemize}
\item {Proveniência:(Gr. \textunderscore narkotikos\textunderscore )}
\end{itemize}
Que produz narcotismo.
Que entorpece ou faz adormecer.
Substância, que paralysa as funcções do cérebro ou o entorpece.
Coisa ou pessôa enfadonha, que causa somno.
\section{Narcotina}
\begin{itemize}
\item {Grp. gram.:f.}
\end{itemize}
\begin{itemize}
\item {Proveniência:(De \textunderscore narcótico\textunderscore )}
\end{itemize}
Alcóide, que se encontra com a morphina em o ópio.
\section{Narcotismo}
\begin{itemize}
\item {Grp. gram.:m.}
\end{itemize}
\begin{itemize}
\item {Proveniência:(De \textunderscore narcótico\textunderscore )}
\end{itemize}
Conjunto dos effeitos produzidos por substâncias narcóticas.
\section{Narcotização}
\begin{itemize}
\item {Grp. gram.:f.}
\end{itemize}
Acto ou effeito de narcotizar.
\section{Narcotizar}
\begin{itemize}
\item {Grp. gram.:v. t.}
\end{itemize}
Dar narcótico a.
Adormecer, entorpecer, paralysar.
\section{Nardino}
\begin{itemize}
\item {Grp. gram.:adj.}
\end{itemize}
\begin{itemize}
\item {Proveniência:(Lat. \textunderscore nardinus\textunderscore )}
\end{itemize}
Relativo a nardo.
\section{Nardo}
\begin{itemize}
\item {Grp. gram.:m.}
\end{itemize}
\begin{itemize}
\item {Utilização:Ext.}
\end{itemize}
\begin{itemize}
\item {Proveniência:(Lat. \textunderscore nardus\textunderscore )}
\end{itemize}
Gênero de plantas gramíneas.
Raiz aromática, de que os antigos se serviam como perfume e que se suppõe sêr a raiz do nardo índico ou espicarnado, (\textunderscore andropogon nardus\textunderscore , Lin.).
Perfume, comparável ao do nardo; bálsamo.
\section{Nárdoa}
\begin{itemize}
\item {Grp. gram.:f.}
\end{itemize}
Gênero de reptis ophídios.
\section{Nardósmia}
\begin{itemize}
\item {Grp. gram.:f.}
\end{itemize}
\begin{itemize}
\item {Proveniência:(Do gr. \textunderscore nardos\textunderscore  + \textunderscore osme\textunderscore )}
\end{itemize}
Gênero de plantas herbáceas.
\section{Naregâmia}
\begin{itemize}
\item {Grp. gram.:f.}
\end{itemize}
Gênero de plantas meliáceas.
\section{Narguilé}
\begin{itemize}
\item {Grp. gram.:m.}
\end{itemize}
Cachimbo turco, índio ou persa, composto de um fornilho, um tubo longo e um vaso cheio de água perfumada, que o fumo atravessa antes de chegar á boca.
(Pers. \textunderscore narguileh\textunderscore )
\section{Narguilhé}
\begin{itemize}
\item {Grp. gram.:m.}
\end{itemize}
(V.narguilé)
\section{Narícula}
\begin{itemize}
\item {Grp. gram.:f.}
\end{itemize}
\begin{itemize}
\item {Grp. gram.:Pl.}
\end{itemize}
\begin{itemize}
\item {Proveniência:(Do lat. \textunderscore nares\textunderscore , por intermédio do b. lat. \textunderscore naricus\textunderscore )}
\end{itemize}
Cada uma das duas fossas nasaes; venta.
Ventas, nariz.
\section{Narigada}
\begin{itemize}
\item {Grp. gram.:f.}
\end{itemize}
\begin{itemize}
\item {Proveniência:(Do b. lat. \textunderscore naricus\textunderscore . Cp. lat. \textunderscore naricare\textunderscore )}
\end{itemize}
Pancada com o nariz.
\section{Nariganga}
\begin{itemize}
\item {Grp. gram.:m.}
\end{itemize}
\begin{itemize}
\item {Utilização:Fam.}
\end{itemize}
\begin{itemize}
\item {Grp. gram.:Adj.}
\end{itemize}
\begin{itemize}
\item {Utilização:Fam.}
\end{itemize}
Grande nariz.
Que tem grande nariz.
\section{Narigão}
\begin{itemize}
\item {Grp. gram.:m.}
\end{itemize}
\begin{itemize}
\item {Grp. gram.:Adj.}
\end{itemize}
Grande nariz.
Narigudo.
(Cp. lat. \textunderscore naricare\textunderscore )
\section{Narigudo}
\begin{itemize}
\item {Grp. gram.:adj.}
\end{itemize}
Que tem nariz grande.
(Cp. lat. \textunderscore naricare\textunderscore )
\section{Narigueta}
\begin{itemize}
\item {fónica:guê}
\end{itemize}
\begin{itemize}
\item {Grp. gram.:m.  e  f.}
\end{itemize}
\begin{itemize}
\item {Utilização:Prov.}
\end{itemize}
Nariz grande.
Homem de nariz grande.
O mesmo que \textunderscore nariguete\textunderscore .
\section{Nariguete}
\begin{itemize}
\item {fónica:guê}
\end{itemize}
\begin{itemize}
\item {Grp. gram.:m.}
\end{itemize}
\begin{itemize}
\item {Utilização:Fam.}
\end{itemize}
Aquelle que tem nariz mal feito, torto ou achatado.
(Cp. \textunderscore narícula\textunderscore )
\section{Narina}
\begin{itemize}
\item {Grp. gram.:f.}
\end{itemize}
(V.narícula)\textunderscore Narina\textunderscore  é francesia inútil.
\section{Nariz}
\begin{itemize}
\item {Grp. gram.:m.}
\end{itemize}
\begin{itemize}
\item {Grp. gram.:Pl.}
\end{itemize}
\begin{itemize}
\item {Utilização:Ext.}
\end{itemize}
\begin{itemize}
\item {Grp. gram.:Loc.}
\end{itemize}
\begin{itemize}
\item {Utilização:Fam.}
\end{itemize}
\begin{itemize}
\item {Grp. gram.:Loc.}
\end{itemize}
\begin{itemize}
\item {Utilização:Fam.}
\end{itemize}
\begin{itemize}
\item {Proveniência:(Lat. hyp. \textunderscore narix\textunderscore )}
\end{itemize}
Parte saliente, pyramidal e triangular do rosto, a qual constitue o órgão do olfacto.
Ferrolho, a que está ligado o lacete da fechadura.
Ventas.
Rosto.
\textunderscore Meter o nariz\textunderscore , intrometer-se.
\textunderscore Nariz de cera\textunderscore , preâmbulo emphático, numa lição ou discurso; lugar commum.
\section{Narizite}
\begin{itemize}
\item {Grp. gram.:f.}
\end{itemize}
\begin{itemize}
\item {Utilização:Burl.}
\end{itemize}
\begin{itemize}
\item {Proveniência:(De \textunderscore nariz\textunderscore )}
\end{itemize}
Perversão da pituitária:«\textunderscore entendi que havia perversão na minha pituitária, uma narizite...\textunderscore »Camillo, \textunderscore Noites de Insómn.\textunderscore , IV, 88.
\section{Narlela}
\begin{itemize}
\item {Grp. gram.:f.}
\end{itemize}
\begin{itemize}
\item {Utilização:T. da Índia port}
\end{itemize}
\begin{itemize}
\item {Proveniência:(De \textunderscore narlo\textunderscore )}
\end{itemize}
Azeite ou óleo de côco.
\section{Narlo}
\begin{itemize}
\item {Grp. gram.:m.}
\end{itemize}
Nome, que na Índia portuguesa se dá ao côco maduro.
\section{Narosca}
\begin{itemize}
\item {Grp. gram.:f.}
\end{itemize}
\begin{itemize}
\item {Utilização:Prov.}
\end{itemize}
\begin{itemize}
\item {Utilização:trasm.}
\end{itemize}
O mesmo que \textunderscore marosca\textunderscore .
\section{Narouco}
\begin{itemize}
\item {Grp. gram.:adj.}
\end{itemize}
\begin{itemize}
\item {Utilização:Prov.}
\end{itemize}
\begin{itemize}
\item {Utilização:trasm.}
\end{itemize}
Estúpido, palerma.
\section{Narra}
\begin{itemize}
\item {Grp. gram.:f.}
\end{itemize}
Árvore das Filippinas, aromática, tinctória e resistente ao fogo.
\section{Narração}
\begin{itemize}
\item {Grp. gram.:f.}
\end{itemize}
\begin{itemize}
\item {Proveniência:(Lat. \textunderscore narratio\textunderscore )}
\end{itemize}
Acto ou effeito de narrar.
Exposição verbal ou escrita de uma série de factos.
Narrativa.
Parte do discurso, em que se expõem os factos, que servem de thema ao mesmo discurso.
\section{Narrado}
\begin{itemize}
\item {Grp. gram.:m.}
\end{itemize}
Aquillo que se narrou; o mesmo que \textunderscore narração\textunderscore .
\section{Narrador}
\begin{itemize}
\item {Grp. gram.:adj.}
\end{itemize}
\begin{itemize}
\item {Grp. gram.:M.}
\end{itemize}
\begin{itemize}
\item {Proveniência:(Lat. \textunderscore narrator\textunderscore )}
\end{itemize}
Que narra, que refere.
Aquelle que narra.
Aquelle que diz histórias ou contos.
Aquelle que expõe as particularidades de um acontecimento.
\section{Narrar}
\begin{itemize}
\item {Grp. gram.:v. t.}
\end{itemize}
\begin{itemize}
\item {Proveniência:(Lat. \textunderscore narrare\textunderscore )}
\end{itemize}
Expor minuciosamente; relatar; contar.
\section{Narrativa}
\begin{itemize}
\item {Grp. gram.:f.}
\end{itemize}
\begin{itemize}
\item {Proveniência:(De \textunderscore narrativo\textunderscore )}
\end{itemize}
O mesmo que \textunderscore narração\textunderscore .
Conto, história.
Exposição dos pormenores de um facto ou série de factos.
\section{Narrativamente}
\begin{itemize}
\item {Grp. gram.:adv.}
\end{itemize}
De modo narrativo; por meio de narração.
\section{Narrativo}
\begin{itemize}
\item {Grp. gram.:adj.}
\end{itemize}
\begin{itemize}
\item {Proveniência:(Lat. \textunderscore narrativus\textunderscore )}
\end{itemize}
Relativo a narração.
Que tem carácter de narração.
Que expõe minuciosamente.
\section{Narrável}
\begin{itemize}
\item {Grp. gram.:adj.}
\end{itemize}
\begin{itemize}
\item {Proveniência:(Do lat. \textunderscore narrabilis\textunderscore )}
\end{itemize}
Que se póde narrar.
\section{Narro}
\begin{itemize}
\item {Grp. gram.:m.}
\end{itemize}
\begin{itemize}
\item {Utilização:Gír.}
\end{itemize}
Gato.
Cão.
\section{Nartécia}
\begin{itemize}
\item {Grp. gram.:f.}
\end{itemize}
\begin{itemize}
\item {Proveniência:(Do gr. \textunderscore narthex\textunderscore )}
\end{itemize}
Gênero de plantas liliáceas.
\section{Nártex}
\begin{itemize}
\item {Grp. gram.:m.}
\end{itemize}
\begin{itemize}
\item {Utilização:Ext.}
\end{itemize}
\begin{itemize}
\item {Proveniência:(Gr. \textunderscore narthex\textunderscore )}
\end{itemize}
Espécie de alpendre, á entrada das antigas basílicas, para conter os catecúmenos, energúmenos e penitentes.
Pórtico.
\section{Narthécia}
\begin{itemize}
\item {Grp. gram.:f.}
\end{itemize}
\begin{itemize}
\item {Proveniência:(Do gr. \textunderscore narthex\textunderscore )}
\end{itemize}
Gênero de plantas liliáceas.
\section{Nárthex}
\begin{itemize}
\item {Grp. gram.:m.}
\end{itemize}
\begin{itemize}
\item {Utilização:Ext.}
\end{itemize}
\begin{itemize}
\item {Proveniência:(Gr. \textunderscore narthex\textunderscore )}
\end{itemize}
Espécie de alpendre, á entrada das antigas basílicas, para conter os cathecúmenos, energúmenos e penitentes.
Pórtico.
\section{Naru}
\begin{itemize}
\item {Grp. gram.:m.}
\end{itemize}
\begin{itemize}
\item {Utilização:T. da Índ. Port}
\end{itemize}
O mesmo que \textunderscore filária\textunderscore .
\section{Narval}
\begin{itemize}
\item {Grp. gram.:m.}
\end{itemize}
\begin{itemize}
\item {Proveniência:(Do al. \textunderscore narwall\textunderscore )}
\end{itemize}
Gênero de cetáceos, da fam. dos delphininos.
\section{Narvalina}
\begin{itemize}
\item {Grp. gram.:f.}
\end{itemize}
Gênero de plantas, da fam. das compostas.
\section{Nas}
\textunderscore f. pl.\textunderscore  de \textunderscore no\textunderscore ^1 e \textunderscore no\textunderscore ^2.
\section{Nasal}
\begin{itemize}
\item {Grp. gram.:adj.}
\end{itemize}
\begin{itemize}
\item {Grp. gram.:M.}
\end{itemize}
\begin{itemize}
\item {Utilização:Anat.}
\end{itemize}
\begin{itemize}
\item {Grp. gram.:F. pl.}
\end{itemize}
\begin{itemize}
\item {Proveniência:(Do lat. \textunderscore nasus\textunderscore )}
\end{itemize}
Relativo ao nariz.
Que está no nariz ou faz parte do nariz: \textunderscore as fossas nasaes\textunderscore .
Diz-se do som, da letra ou da sýllaba, cuja emissão ou cuja pronúncia é modificada pelo nariz.
Fanhoso.
O osso do nariz.
Letras, para cuja phonação contribue o ar que passa pelas fossas nasaes.
\section{Nasalação}
\begin{itemize}
\item {Grp. gram.:f.}
\end{itemize}
Acto ou effeito de nasalar; som nasal.
\section{Nasalar}
\begin{itemize}
\item {Grp. gram.:v. t.}
\end{itemize}
Tornar nasal; pronunciar nasalmente.
\section{Nasalmente}
\begin{itemize}
\item {Grp. gram.:adv.}
\end{itemize}
De modo nasal; fanhosamente.
\section{Nasalidade}
\begin{itemize}
\item {Grp. gram.:f.}
\end{itemize}
Qualidade de nasal.
\section{Nasalizar}
\textunderscore v. t.\textunderscore  (e der.)
O mesmo que \textunderscore nasalar\textunderscore , etc.
\section{Nasara}
\begin{itemize}
\item {Grp. gram.:f.}
\end{itemize}
Moéda de prata, quadrada, que se cunhava em Tunes.
\section{Nasardo}
\begin{itemize}
\item {Grp. gram.:m.}
\end{itemize}
Registo de órgão, cujo som dá ideia de quem fala nasalmente.
(Cp. cast. \textunderscore nasardo\textunderscore , do lat. \textunderscore nasus\textunderscore )
\section{Nasávia}
\begin{itemize}
\item {Grp. gram.:f.}
\end{itemize}
Gênero de plantas, da fam. das compostas.
\section{Nasaviáceas}
\begin{itemize}
\item {Grp. gram.:f. pl.}
\end{itemize}
Tríbo de plantas, que tem por typo a nasávia.
\section{Nascediço}
\begin{itemize}
\item {Grp. gram.:adj.}
\end{itemize}
Que vai nascendo: \textunderscore trigo nascediço\textunderscore . Cf. Garrett, \textunderscore Catão\textunderscore , 5; Filinto, XV, 45 e XVII, 34.
\section{Nascedoiro}
\begin{itemize}
\item {Grp. gram.:m.}
\end{itemize}
\begin{itemize}
\item {Proveniência:(Do lat. \textunderscore nasciturus\textunderscore )}
\end{itemize}
Orifício do útero.
Lugar onde se nasce.
\section{Nascedouro}
\begin{itemize}
\item {Grp. gram.:m.}
\end{itemize}
\begin{itemize}
\item {Proveniência:(Do lat. \textunderscore nasciturus\textunderscore )}
\end{itemize}
Orifício do útero.
Lugar onde se nasce.
\section{Nasceiro}
\begin{itemize}
\item {Grp. gram.:m.}
\end{itemize}
\begin{itemize}
\item {Proveniência:(De \textunderscore nascer\textunderscore )}
\end{itemize}
Nascente ou fonte. Cf. Assis, \textunderscore Águas\textunderscore , IV, 182.
\section{Nascença}
\begin{itemize}
\item {Grp. gram.:f.}
\end{itemize}
\begin{itemize}
\item {Utilização:Pop.}
\end{itemize}
Acto de nascer.
Princípio; origem.
Leicenço, tumor.
\section{Nascente}
\begin{itemize}
\item {Grp. gram.:adj.}
\end{itemize}
\begin{itemize}
\item {Utilização:Heráld.}
\end{itemize}
\begin{itemize}
\item {Grp. gram.:M.}
\end{itemize}
\begin{itemize}
\item {Grp. gram.:F.}
\end{itemize}
\begin{itemize}
\item {Proveniência:(Lat. \textunderscore nascens\textunderscore )}
\end{itemize}
Que nasce.
Que começa.
Que vem apontando: \textunderscore o Sol nascente\textunderscore .
Diz-se da figura, de que só se vê a parte superior rompendo de qualquer móvel heráldico.
Lado do horizonte, donde surge o Sol; oriente.
Ponto, onde nasce ou começa uma corrente de água; fonte; manancial. (Neste sentido, também se lhe dá o gênero masc.)
\section{Nascer}
\begin{itemize}
\item {Grp. gram.:v. i.}
\end{itemize}
Saír do ventre materno, vir ao mundo, á luz: \textunderscore meu filho nasceu em 1896\textunderscore .
Saír do ovo.
Brotar da terra: \textunderscore o trigo já nasce\textunderscore .
Germinar.
Principiar.
Humanar-se.
Começar a manifestar-se.
Surgir no horizonte: \textunderscore nascer a Lua\textunderscore .
Originar-se, provir: \textunderscore daqui nascerão discórdias\textunderscore .
Constituír-se, formar-se.
Resaír, formar saliência ou relêvo.
(B. lat. \textunderscore nascere\textunderscore )
\section{Nascida}
\begin{itemize}
\item {Grp. gram.:f.}
\end{itemize}
\begin{itemize}
\item {Utilização:Pop.}
\end{itemize}
\begin{itemize}
\item {Proveniência:(De \textunderscore nascido\textunderscore )}
\end{itemize}
Abscesso, furúnculo.
\section{Nascidiço}
\begin{itemize}
\item {Grp. gram.:adj.}
\end{itemize}
\begin{itemize}
\item {Proveniência:(De \textunderscore nascer\textunderscore )}
\end{itemize}
Nativo, natural.
\section{Nascido}
\begin{itemize}
\item {Grp. gram.:adj.}
\end{itemize}
\begin{itemize}
\item {Grp. gram.:M.}
\end{itemize}
\begin{itemize}
\item {Proveniência:(De \textunderscore nascer\textunderscore )}
\end{itemize}
Que nasceu.
Dado á luz.
O mesmo que \textunderscore nascida\textunderscore .
\section{Nascimento}
\begin{itemize}
\item {Grp. gram.:m.}
\end{itemize}
\begin{itemize}
\item {Utilização:Fig.}
\end{itemize}
Acto ou effeito de nascer.
Nascença.
Raça, estirpe: \textunderscore nascimento illustre\textunderscore .
Origem, causa; princípio.
\section{Náscio}
\begin{itemize}
\item {Grp. gram.:m.}
\end{itemize}
Gênero de insectos coleópteros pentâmeros.
\section{Nascituro}
\begin{itemize}
\item {Grp. gram.:m.  e  adj.}
\end{itemize}
\begin{itemize}
\item {Proveniência:(Lat. \textunderscore nasciturus\textunderscore )}
\end{itemize}
Aquelle que há de nascer, (falando-se dos seres concebidos e ainda não dados á luz).
\section{Nascível}
\begin{itemize}
\item {Grp. gram.:adj.}
\end{itemize}
\begin{itemize}
\item {Proveniência:(Lat. \textunderscore nascibilis\textunderscore )}
\end{itemize}
Que póde nascer.
\section{Násica}
\begin{itemize}
\item {Grp. gram.:f.}
\end{itemize}
Gênero de mammíferos quadrúmanos.
Espécie de oxycéphala.
\section{Nasícola}
\begin{itemize}
\item {Grp. gram.:m.}
\end{itemize}
\begin{itemize}
\item {Proveniência:(Do lat. \textunderscore nasus\textunderscore  + \textunderscore colere\textunderscore )}
\end{itemize}
Verme, que se desenvolve em o nariz.
\section{Nasicórneo}
\begin{itemize}
\item {Grp. gram.:adj.}
\end{itemize}
\begin{itemize}
\item {Utilização:Zool.}
\end{itemize}
\begin{itemize}
\item {Grp. gram.:M.}
\end{itemize}
\begin{itemize}
\item {Grp. gram.:Pl.}
\end{itemize}
\begin{itemize}
\item {Proveniência:(Do lat. \textunderscore nasus\textunderscore  + \textunderscore cornu\textunderscore )}
\end{itemize}
Que tem sôbre o nariz uma saliência córnea.
Nome de um insecto coleóptero.
Família de mammíferos, a que pertence o antílope.
\section{Naso-palatino}
\begin{itemize}
\item {Grp. gram.:adj.}
\end{itemize}
\begin{itemize}
\item {Proveniência:(Do lat. \textunderscore nasus\textunderscore  + \textunderscore palatum\textunderscore )}
\end{itemize}
Relativo ao nariz e ao palato.
\section{Naso-palpebral}
\begin{itemize}
\item {Grp. gram.:adj.}
\end{itemize}
\begin{itemize}
\item {Utilização:Anat.}
\end{itemize}
Diz-se do músculo orbicular das pálpebras.
\section{Naso-transversal}
\begin{itemize}
\item {Grp. gram.:adj.}
\end{itemize}
\begin{itemize}
\item {Utilização:Anat.}
\end{itemize}
Diz-se do músculo que dilata as narículas, levantando a asa interna de cada uma dellas.
\section{Nassa}
\begin{itemize}
\item {Grp. gram.:f.}
\end{itemize}
\begin{itemize}
\item {Utilização:Prov.}
\end{itemize}
\begin{itemize}
\item {Utilização:minh.}
\end{itemize}
\begin{itemize}
\item {Utilização:Prov.}
\end{itemize}
\begin{itemize}
\item {Utilização:Prov.}
\end{itemize}
\begin{itemize}
\item {Utilização:trasm.}
\end{itemize}
\begin{itemize}
\item {Utilização:Prov.}
\end{itemize}
\begin{itemize}
\item {Utilização:minh.}
\end{itemize}
\begin{itemize}
\item {Proveniência:(Lat. \textunderscore nassa\textunderscore )}
\end{itemize}
Artefacto de vimes, de fórma afunilada, para pescar.
Espécie de pequeno cesto, para apanhar pássaros.
Espécie de cesto cónico de vime, que se suspende da bica do lagar, para que, ao caír do mosto na pia ou dorna, não vá também folhelho ou bagaço.
Bebedeira.
Chocadeira, de madeira e arame, para gallinhas.
\section{Nassada}
\begin{itemize}
\item {Grp. gram.:f.}
\end{itemize}
Grande porção de nassas.
Porção de peixe, apanhado pela nassa.
\section{Nássi}
\begin{itemize}
\item {Grp. gram.:m.}
\end{itemize}
Grande peixe africano. Cf. Serpa Pinto, I, 299.
\section{Nasso}
\begin{itemize}
\item {Grp. gram.:m.}
\end{itemize}
\begin{itemize}
\item {Utilização:Prov.}
\end{itemize}
Espécie de nassa.
Armadilha, para apanhar pássaros, também chamada gaiola.
\section{Nastriforme}
\begin{itemize}
\item {Grp. gram.:adj.}
\end{itemize}
\begin{itemize}
\item {Proveniência:(De \textunderscore nastro\textunderscore  + \textunderscore forma\textunderscore )}
\end{itemize}
Que tem fórma de nastro.
\section{Nastro}
\begin{itemize}
\item {Grp. gram.:m.}
\end{itemize}
\begin{itemize}
\item {Proveniência:(It. \textunderscore nastro\textunderscore )}
\end{itemize}
Fita estreita; trena.
\section{Nastúrcio}
\begin{itemize}
\item {Grp. gram.:m.}
\end{itemize}
\begin{itemize}
\item {Proveniência:(Lat. \textunderscore nasturtium\textunderscore )}
\end{itemize}
Gênero de plantas crucíferas, de largas fôlhas.
\section{Nata}
\begin{itemize}
\item {Grp. gram.:f.}
\end{itemize}
\begin{itemize}
\item {Utilização:Bras. do N}
\end{itemize}
\begin{itemize}
\item {Utilização:Fig.}
\end{itemize}
Parte gordurosa do leite, que toma á superfície a fórma de pellícula.
Creme.
A polpa de qualquer fruto.
A melhor parte de qualquer coisa; o escol: \textunderscore a nata dos estudantes\textunderscore .
(Provavelmente do lat. \textunderscore natus\textunderscore )
\section{Natação}
\begin{itemize}
\item {Grp. gram.:f.}
\end{itemize}
\begin{itemize}
\item {Proveniência:(Lat. \textunderscore natatio\textunderscore )}
\end{itemize}
Acto de nadar.
Arte de nadar.
Systema de locomoção, própria dos animaes que vivem na água.
\section{Natadeira}
\begin{itemize}
\item {Grp. gram.:f.}
\end{itemize}
\begin{itemize}
\item {Proveniência:(De \textunderscore nata\textunderscore )}
\end{itemize}
Bacia larga, em que o leite se expõe ao máximo contacto do ar, para se coalhar facilmente.
\section{Natado}
\begin{itemize}
\item {Grp. gram.:adj.}
\end{itemize}
Coberto de nata ou de nateiro.
\section{Natal}
\begin{itemize}
\item {Grp. gram.:adj.}
\end{itemize}
\begin{itemize}
\item {Grp. gram.:M.}
\end{itemize}
\begin{itemize}
\item {Proveniência:(Lat. \textunderscore natalis\textunderscore )}
\end{itemize}
Relativo a nascimento.
Onde se deu o nascimento: \textunderscore terra natal\textunderscore .
O dia do nascimento
Dia do anniversário de um nascimento.
Dia, em que se festeja o nascimento de Christo.
\section{Natalício}
\begin{itemize}
\item {Grp. gram.:adj.}
\end{itemize}
\begin{itemize}
\item {Proveniência:(Lat. \textunderscore natalicius\textunderscore )}
\end{itemize}
Relativo ao dia do natal: \textunderscore anniversário natalício\textunderscore .
\section{Natalidade}
\begin{itemize}
\item {Grp. gram.:f.}
\end{itemize}
\begin{itemize}
\item {Utilização:Neol.}
\end{itemize}
\begin{itemize}
\item {Proveniência:(Do lat. \textunderscore natalitas\textunderscore )}
\end{itemize}
Conjunto de nascimentos.
Cifra da população, relativa a certa região ou certa época.
\section{Natátil}
\begin{itemize}
\item {Grp. gram.:adj.}
\end{itemize}
\begin{itemize}
\item {Proveniência:(Lat. \textunderscore natatilis\textunderscore )}
\end{itemize}
Que sobrenada ou que póde boiar á superfície da água.
\section{Natatório}
\begin{itemize}
\item {Grp. gram.:adj.}
\end{itemize}
\begin{itemize}
\item {Grp. gram.:M.}
\end{itemize}
\begin{itemize}
\item {Proveniência:(Lat. \textunderscore natatorius\textunderscore )}
\end{itemize}
Relativo á natação.
Tanque, próprio para nadar.
Aquário, piscina.
\section{Nateirado}
\begin{itemize}
\item {Grp. gram.:adj.}
\end{itemize}
\begin{itemize}
\item {Proveniência:(De \textunderscore nateiro\textunderscore )}
\end{itemize}
Coberto de nateiro; coberto de nata.
\section{Nateiro}
\begin{itemize}
\item {Grp. gram.:m.}
\end{itemize}
\begin{itemize}
\item {Proveniência:(De \textunderscore nata\textunderscore )}
\end{itemize}
Camada de lodo, formada pela poeira ou detritos orgânicos, misturados com a água pluvial.
Lodo, proveniente de enxurradas ou deixado por uma corrente que se espraia.
\section{Natento}
\begin{itemize}
\item {Grp. gram.:adj.}
\end{itemize}
\begin{itemize}
\item {Proveniência:(De \textunderscore nata\textunderscore )}
\end{itemize}
Nateirado; fértil.
Amanteigado.
\section{Nática}
\begin{itemize}
\item {Grp. gram.:f.}
\end{itemize}
O mesmo que \textunderscore nátice\textunderscore .
\section{Nátice}
\begin{itemize}
\item {Grp. gram.:f.}
\end{itemize}
Gênero de molluscos gasterópodes.
\section{Naticeiro}
\begin{itemize}
\item {Grp. gram.:m.}
\end{itemize}
Pequeno mollusco, parasito das nátices.
\section{Naticoídeos}
\begin{itemize}
\item {Grp. gram.:m. pl.}
\end{itemize}
Família de molluscos gasterópodes, que tem por typo a nática.
\section{Natiforme}
\begin{itemize}
\item {Grp. gram.:adj.}
\end{itemize}
\begin{itemize}
\item {Proveniência:(Do lat. \textunderscore nates\textunderscore  + \textunderscore forma\textunderscore )}
\end{itemize}
Que tem fórma de nádegas: \textunderscore crânio natiforme\textunderscore .
\section{Nati-morto}
\begin{itemize}
\item {Grp. gram.:m.}
\end{itemize}
\begin{itemize}
\item {Utilização:bras}
\end{itemize}
\begin{itemize}
\item {Utilização:Neol.}
\end{itemize}
\begin{itemize}
\item {Proveniência:(Do lat. \textunderscore natus\textunderscore  + \textunderscore mortuus\textunderscore )}
\end{itemize}
Aquelle que nasceu morto.
\section{Natio}
\begin{itemize}
\item {Grp. gram.:m.}
\end{itemize}
\begin{itemize}
\item {Proveniência:(Do lat. \textunderscore nativus\textunderscore )}
\end{itemize}
Terreno, em que crescem plantas sem cultura prévia.
\section{Nativamente}
\begin{itemize}
\item {Grp. gram.:adv.}
\end{itemize}
De modo nativo; de modo natural.
\section{Natividade}
\begin{itemize}
\item {Grp. gram.:f.}
\end{itemize}
\begin{itemize}
\item {Proveniência:(Lat. \textunderscore nativitas\textunderscore )}
\end{itemize}
Nascimento, (especialmente o de Christo ou dos santos).
\section{Nativismo}
\begin{itemize}
\item {Grp. gram.:m.}
\end{itemize}
Qualidade de nativista; aversão a estrangeiros.
\section{Nativista}
\begin{itemize}
\item {Grp. gram.:adj.}
\end{itemize}
\begin{itemize}
\item {Utilização:Bras}
\end{itemize}
\begin{itemize}
\item {Proveniência:(De \textunderscore nativo\textunderscore )}
\end{itemize}
Relativo aos Indígenas.
Favorável aos Indígenas, com a aversão a estrangeiros, especialmente os Portugueses.
\section{Nativitários}
\begin{itemize}
\item {Grp. gram.:m. pl.}
\end{itemize}
\begin{itemize}
\item {Proveniência:(Do lat. \textunderscore nativitas\textunderscore )}
\end{itemize}
Herejes, que sustentavam que o Verbo teve princípio e não é eterno.
\section{Nativo}
\begin{itemize}
\item {Grp. gram.:adj.}
\end{itemize}
\begin{itemize}
\item {Proveniência:(Lat. \textunderscore nativus\textunderscore )}
\end{itemize}
Que nasce, que é natural.
Produzido pela acção da natureza.
Congênito.
Desartificioso, singelo.
Nacional.
Diz-se da água, que nasce numa propriedade, ou que não provém de corrente estranha ou de nascente distante.
\section{Nato}
\begin{itemize}
\item {Grp. gram.:adj.}
\end{itemize}
\begin{itemize}
\item {Proveniência:(Lat. \textunderscore natus\textunderscore )}
\end{itemize}
Nado, nascido.
Congênito; natural.
\section{Natomia}
\begin{itemize}
\item {Grp. gram.:f.}
\end{itemize}
\begin{itemize}
\item {Utilização:Pop.}
\end{itemize}
O mesmo que \textunderscore anatomia\textunderscore . Cf. Sousa, \textunderscore Vida do Arceb.\textunderscore , III, 14.
\section{Náo}
\begin{itemize}
\item {Grp. gram.:f.}
\end{itemize}
\begin{itemize}
\item {Utilização:Ext.}
\end{itemize}
\begin{itemize}
\item {Proveniência:(Do lat. \textunderscore navis\textunderscore )}
\end{itemize}
Grande navio de guerra ou grande navio mercante.
Tripulação.
Qualquer navio.
\section{Natia}
\begin{itemize}
\item {Grp. gram.:m.}
\end{itemize}
Gênero de reptis sáurios.
\section{Natrão}
\begin{itemize}
\item {Grp. gram.:m.}
\end{itemize}
\begin{itemize}
\item {Utilização:Chím.}
\end{itemize}
Carbonato de soda cristalizado, que certas águas, com soda carbonatada em dissolução, deixam depositar, evaporando-se. Cf. Castilho, \textunderscore Fastos\textunderscore , II, 321 e 322.
(Da fórma alatinada \textunderscore natro\textunderscore , \textunderscore natronis\textunderscore )
\section{Nátrio}
\begin{itemize}
\item {Grp. gram.:m.}
\end{itemize}
Designação antiga do sódio.
(Cp. \textunderscore natro\textunderscore )
\section{Natro}
\begin{itemize}
\item {Grp. gram.:m.}
\end{itemize}
\begin{itemize}
\item {Utilização:Chím.}
\end{itemize}
Carbonato de soda crystallizado, que certas águas, com soda carbonatada em dissolução, deixam depositar, evaporando-se. Cf. Castilho, \textunderscore Fastos\textunderscore , II, 321 e 322.
(Da fórma alatinada \textunderscore natro\textunderscore , \textunderscore natronis\textunderscore )
\section{Natrólita}
\begin{itemize}
\item {Grp. gram.:f.}
\end{itemize}
\begin{itemize}
\item {Proveniência:(T. hybr., do ár. \textunderscore natrum\textunderscore  + gr. \textunderscore lithos\textunderscore )}
\end{itemize}
Silicato de alumínio e sódio.
\section{Natrólitha}
\begin{itemize}
\item {Grp. gram.:f.}
\end{itemize}
\begin{itemize}
\item {Proveniência:(T. hybr., do ár. \textunderscore natrum\textunderscore  + gr. \textunderscore lithos\textunderscore )}
\end{itemize}
Silicato de alumínio e sódio.
\section{Natrólitho}
\begin{itemize}
\item {Grp. gram.:m.}
\end{itemize}
O mesmo ou melhor que \textunderscore natrólitha\textunderscore .
\section{Natrólito}
\begin{itemize}
\item {Grp. gram.:m.}
\end{itemize}
O mesmo ou melhor que \textunderscore natrólita\textunderscore .
\section{Natrómetro}
\begin{itemize}
\item {Grp. gram.:m.}
\end{itemize}
\begin{itemize}
\item {Proveniência:(Do ár. \textunderscore natrum\textunderscore  + gr. \textunderscore metron\textunderscore )}
\end{itemize}
Apparelho, para medir a quantidade de soda, contida na soda do commércio.
\section{Natronalúmen}
\begin{itemize}
\item {Grp. gram.:m.}
\end{itemize}
\begin{itemize}
\item {Utilização:Miner.}
\end{itemize}
Alúmen nativo da soda, ou composto dos sulfatos de soda e de alumina.
\section{Natrum}
\begin{itemize}
\item {Grp. gram.:m.}
\end{itemize}
\begin{itemize}
\item {Proveniência:(Do ár. \textunderscore natrun\textunderscore )}
\end{itemize}
O mesmo que \textunderscore natro\textunderscore .
\section{Natura}
\begin{itemize}
\item {Grp. gram.:f.}
\end{itemize}
\begin{itemize}
\item {Utilização:Poét.}
\end{itemize}
\begin{itemize}
\item {Utilização:Ant.}
\end{itemize}
\begin{itemize}
\item {Proveniência:(Lat. \textunderscore natura\textunderscore )}
\end{itemize}
O mesmo que \textunderscore natureza\textunderscore :«\textunderscore Os dons, que dá Pomona, alli natura produze...\textunderscore »\textunderscore Lusíadas\textunderscore , IX, 58.
Direito de sêr herdeiro em alguma igreja, mosteiro ou lugar pio.
Dinheiro ou alimentos, que se recebiam por aquelle direito.
Órgãos genitaes.
\section{Natural}
\begin{itemize}
\item {Grp. gram.:adj.}
\end{itemize}
\begin{itemize}
\item {Utilização:Heráld.}
\end{itemize}
\begin{itemize}
\item {Grp. gram.:M.}
\end{itemize}
\begin{itemize}
\item {Proveniência:(Lat. \textunderscore naturalis\textunderscore )}
\end{itemize}
Relativo á natureza ou conforme a ella: \textunderscore tendências naturaes\textunderscore .
Produzido pela natureza.
Lógico, que segue a ordem regular das coisas: \textunderscore resultado natural\textunderscore .
Espontâneo: \textunderscore expansão natural\textunderscore .
Em que não há artifício.
Ingênito.
Oriundo, originário: \textunderscore natural do Brasil\textunderscore .
Provável.
Apropriado.
Próprio.
Humano: \textunderscore é natural o amor materno\textunderscore .
Diz-se da figura, que representa objecto da natureza, como homem, astro, árvore, etc.
Indígena; aquelle que pertence pelo nascimento a uma localidade ou região: \textunderscore os naturaes do Pôrto\textunderscore .
Descendente e herdeiro de padroeiro de igreja.
Carácter; índole: \textunderscore é de mau natural\textunderscore .
Realidade.
Aquillo que é simples ou conforme á natureza.
\section{Naturaleza}
\begin{itemize}
\item {Grp. gram.:f.}
\end{itemize}
\begin{itemize}
\item {Utilização:Ant.}
\end{itemize}
O mesmo que \textunderscore natureza\textunderscore . Cf. Usque, f. 10. Pant. de Aveiro, \textunderscore Itiner.\textunderscore , 17, v^o (2^a ed.).
(Cp. cast. \textunderscore naturaleza\textunderscore )
\section{Naturalidade}
\begin{itemize}
\item {Grp. gram.:f.}
\end{itemize}
\begin{itemize}
\item {Proveniência:(Lat. \textunderscore naturalitas\textunderscore )}
\end{itemize}
Qualidade do que é natural.
Espontaneidade; singeleza: \textunderscore naturalidade de estilo\textunderscore .
Nascimento: \textunderscore a terra da minha naturalidade\textunderscore .
Naturalização.
\section{Naturalismo}
\begin{itemize}
\item {Grp. gram.:m.}
\end{itemize}
\begin{itemize}
\item {Proveniência:(De \textunderscore natural\textunderscore )}
\end{itemize}
Estado do que é produzido pela natureza.
Doutrina philosóphica e religiosa dos que attribuem tudo à natureza, como primeiro princípio.
\section{Naturalista}
\begin{itemize}
\item {Grp. gram.:m.  e  f.}
\end{itemize}
\begin{itemize}
\item {Grp. gram.:Adj.}
\end{itemize}
\begin{itemize}
\item {Proveniência:(De \textunderscore natural\textunderscore )}
\end{itemize}
Pessôa, que se occupa especialmente do estudo das producções da natureza.
Sectário do naturalismo.
Relativo ao naturalismo.
\section{Naturalístico}
\begin{itemize}
\item {Grp. gram.:adj.}
\end{itemize}
\begin{itemize}
\item {Utilização:Neol.}
\end{itemize}
\begin{itemize}
\item {Proveniência:(De \textunderscore naturalista\textunderscore )}
\end{itemize}
Relativo aos naturalistas ou aos seus estudos.
\section{Naturalização}
\begin{itemize}
\item {Grp. gram.:f.}
\end{itemize}
Acto de naturalizar.
Acto de adquirir, sendo estrangeiro, direitos garantidos aos nacionais.
Acto de aclimar.
Introducção, numa lingua, de palavras ou locuções de outra língua.
\section{Naturalizado}
\begin{itemize}
\item {Grp. gram.:adj.}
\end{itemize}
Que, sendo estrangeiro, se naturalizou ou adquiriu direitos de nacional.
\section{Naturalizar}
\begin{itemize}
\item {Grp. gram.:v. t.}
\end{itemize}
\begin{itemize}
\item {Proveniência:(De \textunderscore natural\textunderscore )}
\end{itemize}
Conceder (a um estrangeiro) os direitos dos cidadãos de um país; nacionalizar.
Aclimar.
Familiarizar.
Adoptar como nacional ou vernáculo: \textunderscore naturalizar um vocábulo\textunderscore .
\section{Naturalmente}
\begin{itemize}
\item {Grp. gram.:adv.}
\end{itemize}
De modo natural.
Só pelas fôrças da natureza, isto é, sem intervenção do sobrenatural.
Com probabilidade.
\section{Naturança}
\begin{itemize}
\item {Grp. gram.:f.}
\end{itemize}
\begin{itemize}
\item {Utilização:Ant.}
\end{itemize}
Participação de herdeiro nos bens de uma igreja ou mosteiro.
(Cp. \textunderscore natura\textunderscore )
\section{Natureza}
\begin{itemize}
\item {Grp. gram.:f.}
\end{itemize}
\begin{itemize}
\item {Utilização:Pop.}
\end{itemize}
Conjunto de todos os seres que constituem o universo.
Conjunto das leis que presidem á existência das coisas e á successão dos seres.
Fôrça activa, que estabeleceu e conserva a ordem natural de quanto existe.
Ordem natural do universo.
Aquillo que constitue um sêr em geral, criado ou incriado.
Essência ou condição própria de um sêr ou de uma coisa.
Conjunto das propriedades de um sêr organizado.
Constituição de um corpo.
Temperamento de cada indivíduo; carácter, índole: \textunderscore o ódio é contrário á minha natureza\textunderscore .
Condição do homem, considerada anteriormente á civilização.
Objecto real de uma pintura ou esculptura.
Intestinos; funcções digestivas: \textunderscore a fruta faz bem á natureza\textunderscore .
Partes pudendas do homem e da mulher.
(Contr. de \textunderscore naturaleza\textunderscore )
\section{Naturismo}
\begin{itemize}
\item {Grp. gram.:m.}
\end{itemize}
(V.naturalismo)
\section{Naturista}
\begin{itemize}
\item {Grp. gram.:m. ,  f.  e  adj.}
\end{itemize}
\begin{itemize}
\item {Proveniência:(De \textunderscore natura\textunderscore )}
\end{itemize}
Sectário do naturalismo.
\section{Natya}
\begin{itemize}
\item {Grp. gram.:m.}
\end{itemize}
Gênero de reptis sáurios.
\section{Nau}
\begin{itemize}
\item {Grp. gram.:f.}
\end{itemize}
\begin{itemize}
\item {Utilização:Ext.}
\end{itemize}
\begin{itemize}
\item {Proveniência:(Do lat. \textunderscore navis\textunderscore )}
\end{itemize}
Grande navio de guerra ou grande navio mercante.
Tripulação.
Qualquer navio.
\section{Nauas}
\begin{itemize}
\item {Grp. gram.:m. pl.}
\end{itemize}
Indígenas do Brasil, que habitaram no Alto Juruá.
\section{Naüatle}
\begin{itemize}
\item {Grp. gram.:m.}
\end{itemize}
Língua ou dialecto do México.
\section{Náuclea}
\begin{itemize}
\item {Grp. gram.:f.}
\end{itemize}
Gênero de plantas rubiáceas, trepadeiras, originárias da China, Índia, etc.
\section{Náucora}
\begin{itemize}
\item {Grp. gram.:f.}
\end{itemize}
\begin{itemize}
\item {Proveniência:(Do gr. \textunderscore naus\textunderscore  + \textunderscore koris\textunderscore )}
\end{itemize}
Gênero de insectos, hemípteros.
\section{Naucóridas}
\begin{itemize}
\item {Grp. gram.:f. pl.}
\end{itemize}
O mesmo que [[naucorídeos|naucorídeo]].
\section{Naucorídeo}
\begin{itemize}
\item {Grp. gram.:adj.}
\end{itemize}
\begin{itemize}
\item {Grp. gram.:M. pl.}
\end{itemize}
Relativo ou semelhante á náucora.
Grupo de insectos que tem por typo a náucora.
\section{Naudina}
\begin{itemize}
\item {Grp. gram.:f.}
\end{itemize}
Planta berberídea.
\section{Naufragante}
\begin{itemize}
\item {Grp. gram.:adj.}
\end{itemize}
\begin{itemize}
\item {Grp. gram.:M.  e  f.}
\end{itemize}
\begin{itemize}
\item {Proveniência:(Lat. \textunderscore naufragans\textunderscore )}
\end{itemize}
Que naufraga.
Pessôa, que naufraga.
\section{Naufragar}
\begin{itemize}
\item {Grp. gram.:v. i.}
\end{itemize}
\begin{itemize}
\item {Utilização:Fig.}
\end{itemize}
\begin{itemize}
\item {Grp. gram.:V. t.}
\end{itemize}
\begin{itemize}
\item {Proveniência:(Lat. \textunderscore naufragare\textunderscore )}
\end{itemize}
Soçobrar ou desfazer-se no mar, (falando-se do navio).
Soffrer naufrágio, (falando-se dos mareantes).
Extinguir-se.
Causar naufrágio a. Cf. Rebello, \textunderscore Mocidade\textunderscore , I, 220.
\section{Naufragável}
\begin{itemize}
\item {Grp. gram.:adj.}
\end{itemize}
Que póde naufragar.
\section{Naufrágio}
\begin{itemize}
\item {Grp. gram.:m.}
\end{itemize}
\begin{itemize}
\item {Utilização:Fig.}
\end{itemize}
\begin{itemize}
\item {Utilização:Prov.}
\end{itemize}
\begin{itemize}
\item {Utilização:trasm.}
\end{itemize}
\begin{itemize}
\item {Proveniência:(Lat. \textunderscore naufragium\textunderscore )}
\end{itemize}
Perda de um navio, que se afunda no mar ou que se despedaça nas costas, baixios, etc.
Acto de se afundar um navio.
Desgraça, grande prejuízo.
Ruína; decadência.
Desastre em linha férrea. (Colhido na Régua)
\section{Náufrago}
\begin{itemize}
\item {Grp. gram.:adj.}
\end{itemize}
\begin{itemize}
\item {Grp. gram.:M.}
\end{itemize}
\begin{itemize}
\item {Utilização:Fig.}
\end{itemize}
\begin{itemize}
\item {Proveniência:(Lat. \textunderscore naufragus\textunderscore )}
\end{itemize}
Que naufragou.
Resultante de naufrágio.
Naufragoso.
Indivíduo, que ia em navio que naufragou.
Indivíduo infeliz, decadente, arruínado.
\section{Naufragoso}
\begin{itemize}
\item {Grp. gram.:adj.}
\end{itemize}
\begin{itemize}
\item {Utilização:Fig.}
\end{itemize}
\begin{itemize}
\item {Proveniência:(De \textunderscore náufrago\textunderscore )}
\end{itemize}
Que póde causar naufrágios.
Perigoso.
\section{Naumachia}
\begin{itemize}
\item {fónica:qui}
\end{itemize}
\begin{itemize}
\item {Grp. gram.:f.}
\end{itemize}
\begin{itemize}
\item {Proveniência:(Lat. \textunderscore naumachia\textunderscore )}
\end{itemize}
Combate naval, simulado.
Lugar, onde se dava esse simulacro.
\section{Naumachiário}
\begin{itemize}
\item {fónica:qui}
\end{itemize}
\begin{itemize}
\item {Grp. gram.:adj.}
\end{itemize}
\begin{itemize}
\item {Grp. gram.:M.}
\end{itemize}
\begin{itemize}
\item {Proveniência:(Lat. \textunderscore naumachiarius\textunderscore )}
\end{itemize}
Relativo a naumachia.
Aquelle que entra em combate naval.
\section{Naumáchico}
\begin{itemize}
\item {fónica:qui}
\end{itemize}
\begin{itemize}
\item {Grp. gram.:adj.}
\end{itemize}
Relativo a naumachia.
\section{Náumacho}
\begin{itemize}
\item {fónica:co}
\end{itemize}
\begin{itemize}
\item {Grp. gram.:m.}
\end{itemize}
\begin{itemize}
\item {Proveniência:(Lat. \textunderscore naumachus\textunderscore )}
\end{itemize}
Soldado naumachiário.
\section{Náumaco}
\begin{itemize}
\item {Grp. gram.:m.}
\end{itemize}
\begin{itemize}
\item {Proveniência:(Lat. \textunderscore naumachus\textunderscore )}
\end{itemize}
Soldado naumaquiário.
\section{Naumaquia}
\begin{itemize}
\item {Grp. gram.:f.}
\end{itemize}
\begin{itemize}
\item {Proveniência:(Lat. \textunderscore naumachia\textunderscore )}
\end{itemize}
Combate naval, simulado.
Lugar, onde se dava esse simulacro.
\section{Naumaquiário}
\begin{itemize}
\item {Grp. gram.:adj.}
\end{itemize}
\begin{itemize}
\item {Grp. gram.:M.}
\end{itemize}
\begin{itemize}
\item {Proveniência:(Lat. \textunderscore naumachiarius\textunderscore )}
\end{itemize}
Relativo a naumaquia.
Aquele que entra em combate naval.
\section{Naumáquico}
\begin{itemize}
\item {Grp. gram.:adj.}
\end{itemize}
Relativo a naumaquia.
\section{Naumbúrgia}
\begin{itemize}
\item {Grp. gram.:f.}
\end{itemize}
Gênero de plantas primuláceas.
\section{Naupacto}
\begin{itemize}
\item {Grp. gram.:m.}
\end{itemize}
Gênero de insectos coleópteros tetrâmeros.
\section{Naupathia}
\begin{itemize}
\item {Grp. gram.:f.}
\end{itemize}
\begin{itemize}
\item {Proveniência:(Do gr. \textunderscore naus\textunderscore  + \textunderscore pathos\textunderscore )}
\end{itemize}
Enjôo do mar.
\section{Naupatia}
\begin{itemize}
\item {Grp. gram.:f.}
\end{itemize}
\begin{itemize}
\item {Proveniência:(Do gr. \textunderscore naus\textunderscore  + \textunderscore pathos\textunderscore )}
\end{itemize}
Enjôo do mar.
\section{Náupeta}
\begin{itemize}
\item {Grp. gram.:f.}
\end{itemize}
\begin{itemize}
\item {Proveniência:(Do lat. \textunderscore navis\textunderscore  + \textunderscore petere\textunderscore )}
\end{itemize}
Gênero de insectos orthópteros.
\section{Nauscopia}
\begin{itemize}
\item {Grp. gram.:f.}
\end{itemize}
Arte de empregar o nauscópio.
(Cf. \textunderscore nauscópio\textunderscore )
\section{Nauscópico}
\begin{itemize}
\item {Grp. gram.:adj.}
\end{itemize}
Relativo á nauscopia.
\section{Nauscópio}
\begin{itemize}
\item {Grp. gram.:m.}
\end{itemize}
\begin{itemize}
\item {Proveniência:(Do gr. \textunderscore naus\textunderscore  + \textunderscore skopein\textunderscore )}
\end{itemize}
Instrumento, para vêr da terra navios a grande distância ou vice-versa.
\section{Náusea}
\begin{itemize}
\item {Grp. gram.:f.}
\end{itemize}
\begin{itemize}
\item {Utilização:Ext.}
\end{itemize}
\begin{itemize}
\item {Utilização:Fig.}
\end{itemize}
\begin{itemize}
\item {Proveniência:(Lat. \textunderscore nausea\textunderscore )}
\end{itemize}
Enjôo ou ânsia, produzida pelo balanço da embarcação, no mar.
Ansiedade, acompanhada de vómito; disposição para vomitar.
Repugnância.
Nojo; sentimento de repulsão.
\section{Nauseabundo}
\begin{itemize}
\item {Grp. gram.:adj.}
\end{itemize}
\begin{itemize}
\item {Utilização:Fig.}
\end{itemize}
\begin{itemize}
\item {Proveniência:(Lat. \textunderscore nauseabundus\textunderscore )}
\end{itemize}
Que causa náuseas.
Nojento, repugnante.
\section{Nauseado}
\begin{itemize}
\item {Grp. gram.:adj.}
\end{itemize}
\begin{itemize}
\item {Utilização:Fig.}
\end{itemize}
\begin{itemize}
\item {Grp. gram.:M.}
\end{itemize}
\begin{itemize}
\item {Utilização:Mad}
\end{itemize}
Que tem enjôo; que está indisposto ou ansiado, como quem vai vomitar.
Que sente repulsão ou nojo.
O mesmo que \textunderscore enjôo\textunderscore .
\section{Nauseante}
\begin{itemize}
\item {Grp. gram.:adj.}
\end{itemize}
\begin{itemize}
\item {Proveniência:(Lat. \textunderscore nauseans\textunderscore )}
\end{itemize}
O mesmo que \textunderscore nauseabundo\textunderscore .
\section{Nausear}
\begin{itemize}
\item {Grp. gram.:v. t.}
\end{itemize}
\begin{itemize}
\item {Grp. gram.:V. i.}
\end{itemize}
\begin{itemize}
\item {Proveniência:(Lat. \textunderscore nauseare\textunderscore )}
\end{itemize}
Causar náuseas a.
Enjoar; repugnar a.
Têr náuseas; enjoar.
\section{Nauseativo}
\begin{itemize}
\item {Grp. gram.:adj.}
\end{itemize}
O mesmo que \textunderscore nauseabundo\textunderscore .
\section{Nauseento}
\begin{itemize}
\item {Grp. gram.:adj.}
\end{itemize}
Que se nauseia com facilidade.
\section{Nauseosamente}
\begin{itemize}
\item {Grp. gram.:adv.}
\end{itemize}
De modo nauseoso.
Com repulsão.
\section{Nauseoso}
\begin{itemize}
\item {Grp. gram.:adj.}
\end{itemize}
\begin{itemize}
\item {Proveniência:(Lat. \textunderscore nauseosus\textunderscore )}
\end{itemize}
O mesmo que \textunderscore nauseabundo\textunderscore .
\section{Nauta}
\begin{itemize}
\item {Grp. gram.:m.}
\end{itemize}
\begin{itemize}
\item {Proveniência:(Lat. \textunderscore nauta\textunderscore )}
\end{itemize}
Aquelle que navega; navegante.
Marinheiro.
\section{Nautaques}
\begin{itemize}
\item {Grp. gram.:m. pl.}
\end{itemize}
Nome depreciativo que, no Oriente, se dava aos Baloches. Cf. Barros, \textunderscore Déc.\textunderscore  III, l. VII. c. 2.
\section{Nautarel}
\begin{itemize}
\item {Grp. gram.:m.}
\end{itemize}
Antigo empregado superior de alfândega, na China. Cf. \textunderscore Peregrinação\textunderscore , XLIL.
\section{Náutica}
\begin{itemize}
\item {Grp. gram.:f.}
\end{itemize}
\begin{itemize}
\item {Proveniência:(De \textunderscore náutico\textunderscore )}
\end{itemize}
Arte de navegar.
\section{Nautilídeo}
\begin{itemize}
\item {Grp. gram.:adj.}
\end{itemize}
\begin{itemize}
\item {Grp. gram.:M. pl.}
\end{itemize}
\begin{itemize}
\item {Proveniência:(Do gr. \textunderscore nautilos\textunderscore  + \textunderscore eidos\textunderscore )}
\end{itemize}
O mesmo que \textunderscore nautilóide\textunderscore .
Família de cephalópodes fósseis.
\section{Náutico}
\begin{itemize}
\item {Grp. gram.:adj.}
\end{itemize}
\begin{itemize}
\item {Grp. gram.:M.}
\end{itemize}
\begin{itemize}
\item {Proveniência:(Lat. \textunderscore nauticus\textunderscore )}
\end{itemize}
Relativo á navegação: \textunderscore empresas náuticas\textunderscore .
Aquelle que é versado em náutica.
\section{Nautilinos}
\begin{itemize}
\item {Grp. gram.:m. pl.}
\end{itemize}
\begin{itemize}
\item {Proveniência:(De \textunderscore náutilo\textunderscore )}
\end{itemize}
Molluscos, que formam uma secção do gênero náutilo.
\section{Nautilita}
\begin{itemize}
\item {Grp. gram.:f.}
\end{itemize}
\begin{itemize}
\item {Utilização:Des.}
\end{itemize}
Náutilo fóssil.
\section{Náutilo}
\begin{itemize}
\item {Grp. gram.:m.}
\end{itemize}
\begin{itemize}
\item {Proveniência:(Gr. \textunderscore nautilos\textunderscore )}
\end{itemize}
Mollusco cephalópode cuja concha é dividida em muitos compartimentos.
Navio submarino.
\section{Nautilóide}
\begin{itemize}
\item {Grp. gram.:adj.}
\end{itemize}
\begin{itemize}
\item {Proveniência:(Do gr. \textunderscore nautilos\textunderscore  + \textunderscore eidos\textunderscore )}
\end{itemize}
Relativo ou semelhante ao náutilo (mollusco).
\section{Nautódico}
\begin{itemize}
\item {Grp. gram.:m.}
\end{itemize}
\begin{itemize}
\item {Proveniência:(Do gr. \textunderscore nautes\textunderscore  + \textunderscore dikes\textunderscore )}
\end{itemize}
Magistrado subalterno, na Grécia antiga.
\section{Nautografia}
\begin{itemize}
\item {Grp. gram.:f.}
\end{itemize}
\begin{itemize}
\item {Utilização:Neol.}
\end{itemize}
Descripção do aparelho dos navios e das respectivas manobras.
(Cp. \textunderscore nautógrafo\textunderscore )
\section{Nautógrafo}
\begin{itemize}
\item {Grp. gram.:m.}
\end{itemize}
\begin{itemize}
\item {Proveniência:(Do gr. \textunderscore nautes\textunderscore  + \textunderscore graphein\textunderscore )}
\end{itemize}
Aquele que se ocupa da nautografia.
\section{Nautographia}
\begin{itemize}
\item {Grp. gram.:f.}
\end{itemize}
\begin{itemize}
\item {Utilização:Neol.}
\end{itemize}
Descripção do apparelho dos navios e das respectivas manobras.
(Cp. \textunderscore nautógrapho\textunderscore )
\section{Nautógrapho}
\begin{itemize}
\item {Grp. gram.:m.}
\end{itemize}
\begin{itemize}
\item {Proveniência:(Do gr. \textunderscore nautes\textunderscore  + \textunderscore graphein\textunderscore )}
\end{itemize}
Aquelle que se occupa da nautographia.
\section{Nava}
\begin{itemize}
\item {Grp. gram.:f.}
\end{itemize}
\begin{itemize}
\item {Utilização:Des.}
\end{itemize}
Planície, planura.
(Do romanço \textunderscore naba\textunderscore , planície)
\section{Navagem}
\begin{itemize}
\item {Grp. gram.:f.}
\end{itemize}
\begin{itemize}
\item {Utilização:Ant.}
\end{itemize}
\begin{itemize}
\item {Proveniência:(Do lat. \textunderscore navis\textunderscore )}
\end{itemize}
O mesmo que \textunderscore navegagem\textunderscore .
\section{Naval}
\begin{itemize}
\item {Grp. gram.:adj.}
\end{itemize}
\begin{itemize}
\item {Proveniência:(Lat. \textunderscore navalis\textunderscore )}
\end{itemize}
Relativo a navios ou navegação: \textunderscore progressos navaes\textunderscore .
Náutico.
\section{Navalha}
\begin{itemize}
\item {Grp. gram.:f.}
\end{itemize}
\begin{itemize}
\item {Utilização:Fig.}
\end{itemize}
\begin{itemize}
\item {Utilização:Pesc.}
\end{itemize}
\begin{itemize}
\item {Proveniência:(Do lat. \textunderscore novacula\textunderscore )}
\end{itemize}
Instrumento, formado de uma lâmina cortante, e de um cabo que protege o fio da mesma lâmina, quando o instrumento se fecha.
Mollusco semelhante ao cabo de uma navalha, o mesmo que \textunderscore lingueirão\textunderscore .
Pessôa, que tem má lingua.
Frio intenso.
Vara com um caranguejo, para apanhar o polvo.
\section{Navalhada}
\begin{itemize}
\item {Grp. gram.:f.}
\end{itemize}
Golpe de navalha.
\section{Navalhão}
\begin{itemize}
\item {Grp. gram.:m.}
\end{itemize}
\begin{itemize}
\item {Utilização:Prov.}
\end{itemize}
\begin{itemize}
\item {Utilização:trasm.}
\end{itemize}
Navalha grande.
Cada uma das lâminas de aço, ligadas á cabeça da broca, em artilharia.
Pedaço húmido de terreno entre as searas, que se não cultiva, para dar erva.
\section{Navalhar}
\begin{itemize}
\item {Grp. gram.:v. t.}
\end{itemize}
\begin{itemize}
\item {Utilização:Fig.}
\end{itemize}
\begin{itemize}
\item {Proveniência:(De \textunderscore navalha\textunderscore )}
\end{itemize}
Dar navalhadas em; golpear.
Magoar muito, torturar:«\textunderscore e não era só isto navalhar-lhe o coração.\textunderscore »Camillo, \textunderscore Myst. de Fafe\textunderscore .
\section{Navalheira}
\begin{itemize}
\item {Grp. gram.:f.}
\end{itemize}
\begin{itemize}
\item {Proveniência:(De \textunderscore navalha\textunderscore )}
\end{itemize}
Crustáceo pernilongo.
\section{Navalhista}
\begin{itemize}
\item {Grp. gram.:m.}
\end{itemize}
\begin{itemize}
\item {Proveniência:(De \textunderscore navalha\textunderscore )}
\end{itemize}
Aquelle que dá navalhadas; faquista. Cf. Camillo, \textunderscore Brasileira\textunderscore , 191.
\section{Návão}
\begin{itemize}
\item {Grp. gram.:m.}
\end{itemize}
\begin{itemize}
\item {Utilização:Ant.}
\end{itemize}
O mesmo que \textunderscore nábão\textunderscore . Cf. Cortesão, \textunderscore Subs.\textunderscore 
\section{Navarca}
\begin{itemize}
\item {Grp. gram.:m.}
\end{itemize}
(V.navarco)
\section{Navarcha}
\begin{itemize}
\item {fónica:ca}
\end{itemize}
\begin{itemize}
\item {Grp. gram.:m.}
\end{itemize}
(V.navarcho)
\section{Navarchia}
\begin{itemize}
\item {fónica:qui}
\end{itemize}
\begin{itemize}
\item {Grp. gram.:f.}
\end{itemize}
Titulo ou dignidade de navarcha.
\section{Navárchico}
\begin{itemize}
\item {fónica:qui}
\end{itemize}
\begin{itemize}
\item {Grp. gram.:adj.}
\end{itemize}
Relativo a navarcho ou á navarchia.
\section{Navarcho}
\begin{itemize}
\item {fónica:co}
\end{itemize}
\begin{itemize}
\item {Grp. gram.:m.}
\end{itemize}
\begin{itemize}
\item {Utilização:Poét.}
\end{itemize}
\begin{itemize}
\item {Proveniência:(Lat. \textunderscore navarchus\textunderscore )}
\end{itemize}
Capitão do navio.
\section{Navarco}
\begin{itemize}
\item {Grp. gram.:m.}
\end{itemize}
\begin{itemize}
\item {Utilização:Poét.}
\end{itemize}
\begin{itemize}
\item {Proveniência:(Lat. \textunderscore navarchus\textunderscore )}
\end{itemize}
Capitão do navio.
\section{Navarquia}
\begin{itemize}
\item {Grp. gram.:f.}
\end{itemize}
Titulo ou dignidade de navarca.
\section{Navárquico}
\begin{itemize}
\item {Grp. gram.:adj.}
\end{itemize}
Relativo a navarco ou á navarquia.
\section{Navarra}
\begin{itemize}
\item {Grp. gram.:f.}
\end{itemize}
\begin{itemize}
\item {Proveniência:(De \textunderscore navarro\textunderscore )}
\end{itemize}
Sorte de toireiro, executada com o capote, que se tira ligeiramente por baixo do focinho do animal, dando o toireiro uma volta sôbre os calcanhares e ficando em posição de repetir a sorte.
\section{Navarrês}
\begin{itemize}
\item {Grp. gram.:m.  e  adj.}
\end{itemize}
O mesmo que \textunderscore navarro\textunderscore .
\section{Navarrina}
\begin{itemize}
\item {Grp. gram.:adj. f.}
\end{itemize}
\begin{itemize}
\item {Proveniência:(De \textunderscore Navarra\textunderscore , n. p.)}
\end{itemize}
Diz-se de uma antiga raça de cavallos de Navarra, do Bearn e do Rossilhão.
\section{Navarro}
\begin{itemize}
\item {Grp. gram.:adj.}
\end{itemize}
\begin{itemize}
\item {Grp. gram.:M.}
\end{itemize}
Relativo a Navarra.
Aquelle que é natural de Navarra.
Dialecto de Navarra.
\section{Nave}
\begin{itemize}
\item {Grp. gram.:f.}
\end{itemize}
\begin{itemize}
\item {Utilização:Fig.}
\end{itemize}
\begin{itemize}
\item {Utilização:Ant.}
\end{itemize}
\begin{itemize}
\item {Proveniência:(Lat. \textunderscore navis\textunderscore )}
\end{itemize}
Parte interior da igreja, desde a entrada ao santuário.
Corpo da igreja.
Espaço longitudinal, entre fileiras de columnas, que sustentam a abóbada de uma igreja.
Templo.
O mesmo que \textunderscore nau\textunderscore .
\section{Navegabilidade}
\begin{itemize}
\item {Grp. gram.:f.}
\end{itemize}
Qualidade de navegável.
\section{Navegação}
\begin{itemize}
\item {Grp. gram.:f.}
\end{itemize}
\begin{itemize}
\item {Proveniência:(Lat. \textunderscore navigatio\textunderscore )}
\end{itemize}
Acto de navegar.
Náutica.
Commércio maritimo.
Grande viagem por mar.
\section{Navegado}
\begin{itemize}
\item {Grp. gram.:adj.}
\end{itemize}
\begin{itemize}
\item {Proveniência:(De \textunderscore navegar\textunderscore )}
\end{itemize}
Cruzado por navios; percorrido por navegantes:«\textunderscore mares, nunca dantes navegados\textunderscore », \textunderscore Lusíadas\textunderscore , I, 1.
\section{Navegador}
\begin{itemize}
\item {Grp. gram.:adj.}
\end{itemize}
\begin{itemize}
\item {Grp. gram.:M.}
\end{itemize}
\begin{itemize}
\item {Proveniência:(Do lat. \textunderscore navigator\textunderscore )}
\end{itemize}
Que navega.
Habituado a navegar.
Que sabe navegar.
Aquelle que navega; aquelle que faz longa ou notável navegação.
Mareante.
\section{Navegagem}
\begin{itemize}
\item {Grp. gram.:f.}
\end{itemize}
\begin{itemize}
\item {Utilização:Ant.}
\end{itemize}
\begin{itemize}
\item {Proveniência:(De \textunderscore navegar\textunderscore )}
\end{itemize}
Frete de embarcação.
Preço da passagem em barco.
\section{Navegante}
\begin{itemize}
\item {Grp. gram.:adj.}
\end{itemize}
\begin{itemize}
\item {Grp. gram.:M.}
\end{itemize}
\begin{itemize}
\item {Proveniência:(Lat. \textunderscore navigans\textunderscore )}
\end{itemize}
Que navega.
Aquelle que navega; navegador.
Crustáceo, o mesmo que \textunderscore labugante\textunderscore .
\section{Navegante}
\begin{itemize}
\item {Grp. gram.:m.}
\end{itemize}
Crustáceo decápode, marítimo, um pouco mais pequeno que a lagosta e munido de duas fortes torqueses nos braços (\textunderscore homarus vulgaris\textunderscore ).
(Cp. cast. \textunderscore lobogante\textunderscore )
\section{Navegar}
\begin{itemize}
\item {Grp. gram.:v. t.}
\end{itemize}
\begin{itemize}
\item {Utilização:P. us.}
\end{itemize}
\begin{itemize}
\item {Grp. gram.:V. i.}
\end{itemize}
\begin{itemize}
\item {Proveniência:(Lat. \textunderscore navigare\textunderscore )}
\end{itemize}
Percorrer em navio (o mar).
Transportar em navio.
Transportar-se em navio.
Viajar pelo mar.
Seguir viagem (um navio).
\section{Navegável}
\begin{itemize}
\item {Grp. gram.:adj.}
\end{itemize}
\begin{itemize}
\item {Proveniência:(Lat. \textunderscore navigabilis\textunderscore )}
\end{itemize}
Que se póde percorrer em navio ou barco.
\section{Navêgo}
\begin{itemize}
\item {Grp. gram.:m.}
\end{itemize}
\begin{itemize}
\item {Utilização:Des.}
\end{itemize}
Arte de navegar.
Navegação:«\textunderscore ...sem indústria para o navêgo...\textunderscore »Filinto, \textunderscore D. Man.\textunderscore , I, 26.
\section{Naveta}
\begin{itemize}
\item {fónica:vê}
\end{itemize}
\begin{itemize}
\item {Grp. gram.:f.}
\end{itemize}
\begin{itemize}
\item {Utilização:Ant.}
\end{itemize}
\begin{itemize}
\item {Proveniência:(De \textunderscore nave\textunderscore )}
\end{itemize}
Pequeno vaso, do feitio de um barco, no qual, por occasião de certas festas de igreja, se guarda e se serve o incenso para os thuríbulos.
Espécie de pequena lançadeira, com que se faz certa renda.
Lançadeira de máquina de costura.
Pequena nau, pequena embarcação. Cf. Gaspar Correia, \textunderscore Lendas\textunderscore , I, 301.
\section{Návia}
\begin{itemize}
\item {Grp. gram.:f.}
\end{itemize}
\begin{itemize}
\item {Proveniência:(Lat. \textunderscore navia\textunderscore )}
\end{itemize}
Vaso ou tronco cavado em fórma de navio, usado pelos antigos nas vindimas.
\section{Naviamento}
\begin{itemize}
\item {Grp. gram.:m.}
\end{itemize}
\begin{itemize}
\item {Utilização:Ant.}
\end{itemize}
\begin{itemize}
\item {Proveniência:(De \textunderscore navio\textunderscore )}
\end{itemize}
O mesmo que \textunderscore navegação\textunderscore . Cf. Frei Fortun., \textunderscore Inéd.\textunderscore , 310.
\section{Naviarra}
\begin{itemize}
\item {Grp. gram.:f.}
\end{itemize}
\begin{itemize}
\item {Utilização:Ant.}
\end{itemize}
\begin{itemize}
\item {Proveniência:(De \textunderscore navio\textunderscore )}
\end{itemize}
Talvez o mesmo que \textunderscore barcaça\textunderscore . Cf. G. Vicente, I, 223.
\section{Navicela}
\begin{itemize}
\item {Grp. gram.:f.}
\end{itemize}
\begin{itemize}
\item {Proveniência:(Lat. \textunderscore navicella\textunderscore )}
\end{itemize}
O mesmo que naveta:«\textunderscore ...lançou da navicela o incenso no turíbulo\textunderscore ». Camillo, \textunderscore Cav. em Ruínas\textunderscore , 99.
\section{Navicella}
\begin{itemize}
\item {Grp. gram.:f.}
\end{itemize}
\begin{itemize}
\item {Proveniência:(Lat. \textunderscore navicella\textunderscore )}
\end{itemize}
O mesmo que naveta:«\textunderscore ...lançou da navicella o incenso no thuríbulo\textunderscore ». Camillo, \textunderscore Cav. em Ruínas\textunderscore , 99.
\section{Navícula}
\begin{itemize}
\item {Grp. gram.:f.}
\end{itemize}
\begin{itemize}
\item {Proveniência:(Lat. \textunderscore navícula\textunderscore )}
\end{itemize}
Peça ou órgão, com a fórma de navio.
\section{Navicular}
\begin{itemize}
\item {Grp. gram.:adj.}
\end{itemize}
\begin{itemize}
\item {Proveniência:(Lat. \textunderscore navicularis\textunderscore )}
\end{itemize}
Relativo a navícula.
\section{Naviculário}
\begin{itemize}
\item {Grp. gram.:m.}
\end{itemize}
\begin{itemize}
\item {Proveniência:(Lat. \textunderscore navicularius\textunderscore )}
\end{itemize}
Proprietário ou armador de navios, na antiguidade.
\section{Naviforme}
\begin{itemize}
\item {Grp. gram.:adj.}
\end{itemize}
\begin{itemize}
\item {Proveniência:(De \textunderscore nave\textunderscore ^1 + \textunderscore fórma\textunderscore )}
\end{itemize}
Que tem fórma de navio.
\section{Navífrago}
\begin{itemize}
\item {Grp. gram.:adj.}
\end{itemize}
\begin{itemize}
\item {Utilização:Poét.}
\end{itemize}
\begin{itemize}
\item {Proveniência:(Lat. \textunderscore navifragus\textunderscore )}
\end{itemize}
Que despedaça navios; naufragoso.
\section{Navigabilidade}
\begin{itemize}
\item {Grp. gram.:f.}
\end{itemize}
\begin{itemize}
\item {Proveniência:(Do lat. \textunderscore navigabilis\textunderscore )}
\end{itemize}
Qualidade do que é navegável.
\section{Navígero}
\begin{itemize}
\item {Grp. gram.:adj.}
\end{itemize}
\begin{itemize}
\item {Utilização:Poét.}
\end{itemize}
\begin{itemize}
\item {Proveniência:(Lat. \textunderscore naviger\textunderscore )}
\end{itemize}
O mesmo que \textunderscore navegável\textunderscore .
\section{Navim}
\begin{itemize}
\item {Grp. gram.:m.}
\end{itemize}
\begin{itemize}
\item {Utilização:T. da Índia port}
\end{itemize}
\begin{itemize}
\item {Proveniência:(Do conc. \textunderscore nãv[~i]\textunderscore , em nome de)}
\end{itemize}
Título de compra, feito por notário.
\section{Navio}
\begin{itemize}
\item {Grp. gram.:m.}
\end{itemize}
\begin{itemize}
\item {Proveniência:(Lat. \textunderscore navigium\textunderscore )}
\end{itemize}
Grande embarcação.
Qualquer embarcação.
\section{Navio-ó-versa}
\begin{itemize}
\item {Grp. gram.:loc. adv.}
\end{itemize}
\begin{itemize}
\item {Utilização:Prov.}
\end{itemize}
\begin{itemize}
\item {Utilização:beir.}
\end{itemize}
O mesmo que \textunderscore vice-versa\textunderscore .
\section{Navisfério}
\begin{itemize}
\item {Grp. gram.:m.}
\end{itemize}
\begin{itemize}
\item {Utilização:Bras}
\end{itemize}
\begin{itemize}
\item {Proveniência:(Do lat. \textunderscore navis\textunderscore  + \textunderscore sphera\textunderscore )}
\end{itemize}
Instrumento náutico. Cf. \textunderscore Tarifa da Alfândega\textunderscore , no Brasil.
\section{Navisphério}
\begin{itemize}
\item {Grp. gram.:m.}
\end{itemize}
\begin{itemize}
\item {Utilização:Bras}
\end{itemize}
\begin{itemize}
\item {Proveniência:(Do lat. \textunderscore navis\textunderscore  + \textunderscore sphera\textunderscore )}
\end{itemize}
Instrumento náutico. Cf. \textunderscore Tarifa da Alfândega\textunderscore , no Brasil.
\section{Naxa}
\begin{itemize}
\item {Grp. gram.:f.}
\end{itemize}
Árvore angolense.
\section{Naxanim}
\begin{itemize}
\item {Grp. gram.:m.}
\end{itemize}
O mesmo que \textunderscore luco\textunderscore ^2.
\section{Náxaro}
\begin{itemize}
\item {Grp. gram.:m.  e  adj.}
\end{itemize}
\begin{itemize}
\item {Utilização:Prov.}
\end{itemize}
\begin{itemize}
\item {Utilização:beir.}
\end{itemize}
Indivíduo de nariz achatado ou mal feito.
\section{Náxia}
\begin{itemize}
\item {fónica:csi}
\end{itemize}
\begin{itemize}
\item {Grp. gram.:f.}
\end{itemize}
Gênero de crustáceos decápodes.
\section{Náxio}
\begin{itemize}
\item {fónica:csi}
\end{itemize}
\begin{itemize}
\item {Grp. gram.:m.}
\end{itemize}
\begin{itemize}
\item {Proveniência:(Lat. \textunderscore naxium\textunderscore )}
\end{itemize}
Pedra, com que se polia o mármore e com que os lapidários facetavam e poliam pedras preciosas.
\section{Naza}
\begin{itemize}
\item {Grp. gram.:f.}
\end{itemize}
O mesmo que \textunderscore naxa\textunderscore .
\section{Nazareno}
\begin{itemize}
\item {Grp. gram.:adj.}
\end{itemize}
\begin{itemize}
\item {Grp. gram.:M.}
\end{itemize}
\begin{itemize}
\item {Utilização:Restrict.}
\end{itemize}
\begin{itemize}
\item {Utilização:Ext.}
\end{itemize}
\begin{itemize}
\item {Proveniência:(Lat. \textunderscore nazarenus\textunderscore )}
\end{itemize}
Relativo a Nazaré.
Habitante de Nazaré.
Christo.
Cada um dos primeiros christãos.
Designação, que os Portugueses tiveram na Ásia.
\section{Nazáreo}
\begin{itemize}
\item {Grp. gram.:adj.}
\end{itemize}
O mesmo que \textunderscore nazareno\textunderscore . Cf. Filinto, XIV, 104.
\section{Nazarita}
\begin{itemize}
\item {Grp. gram.:m.  e  f.}
\end{itemize}
\begin{itemize}
\item {Proveniência:(Do hebr. \textunderscore natzar\textunderscore )}
\end{itemize}
Pessôa que, entre os Judeus, se consagrava a Deus, por voto próprio ou de seus pais.
\section{Nazaritismo}
\begin{itemize}
\item {Grp. gram.:m.}
\end{itemize}
Doutrina dos Nazaritas.
\section{Názi}
\begin{itemize}
\item {Grp. gram.:m.}
\end{itemize}
Árvore de Moçambique, de fibras têxteis.
\section{Nazianzeno}
\begin{itemize}
\item {Grp. gram.:adj.}
\end{itemize}
Relativo á cidade de Nazianzo.
Natural de Nazianzo.
\section{Názir}
\begin{itemize}
\item {Grp. gram.:m.}
\end{itemize}
Superintendente das mesquitas, entre os Orientaes.
Tribunal supremo, na Pérsia.
\section{Nazireu}
\begin{itemize}
\item {Grp. gram.:m.}
\end{itemize}
\begin{itemize}
\item {Proveniência:(Do hebr. \textunderscore nazir\textunderscore , consagrado)}
\end{itemize}
Hebreu, que fazia voto de não cortar o cabello nem beber vinho.
\section{Nazoreus}
\begin{itemize}
\item {Grp. gram.:m. pl.}
\end{itemize}
Antiga seita christan.
\section{N. B.}
Abrev. da expressão latina \textunderscore nota bene\textunderscore , (repara bem).
\section{N. E.}
\begin{itemize}
\item {Grp. gram.:m.}
\end{itemize}
Abrev. de \textunderscore Nordéste\textunderscore .
\section{Nearca}
\begin{itemize}
\item {Grp. gram.:m.}
\end{itemize}
O mesmo que \textunderscore nearco\textunderscore .
\section{Nearcha}
\begin{itemize}
\item {fónica:ca}
\end{itemize}
\begin{itemize}
\item {Grp. gram.:m.}
\end{itemize}
O mesmo que \textunderscore nearcho\textunderscore .
\section{Nearcho}
\begin{itemize}
\item {fónica:co}
\end{itemize}
\begin{itemize}
\item {Grp. gram.:m.}
\end{itemize}
\begin{itemize}
\item {Proveniência:(Lat. \textunderscore nearchus\textunderscore )}
\end{itemize}
Espécie de almirante nas antigas armadas gregas.
\section{Nearco}
\begin{itemize}
\item {Grp. gram.:m.}
\end{itemize}
\begin{itemize}
\item {Proveniência:(Lat. \textunderscore nearchus\textunderscore )}
\end{itemize}
Espécie de almirante nas antigas armadas gregas.
\section{Nebália}
\begin{itemize}
\item {Grp. gram.:f.}
\end{itemize}
Espécie de caranguejo do Mar-Branco.
\section{Nebel}
\begin{itemize}
\item {Grp. gram.:m.}
\end{itemize}
Medida de capacidade, entre os Hebreus.
\section{Nebel}
\begin{itemize}
\item {Grp. gram.:m.}
\end{itemize}
O mesmo que \textunderscore náblio\textunderscore :«\textunderscore ...modulando o nebel, via-se o vate...\textunderscore »Herculano, \textunderscore Harpa do Crente\textunderscore .
(Cp. cast. \textunderscore nebel\textunderscore )
\section{Neblina}
\begin{itemize}
\item {Grp. gram.:f.}
\end{itemize}
\begin{itemize}
\item {Utilização:Ext.}
\end{itemize}
Grande nevoeiro.
Névoa densa e rasteira.
Sombra, trevas.
(Cast. \textunderscore neblina\textunderscore )
\section{Neblinar}
\begin{itemize}
\item {Grp. gram.:v. i.}
\end{itemize}
\begin{itemize}
\item {Utilização:açor}
\end{itemize}
\begin{itemize}
\item {Utilização:Bras}
\end{itemize}
\begin{itemize}
\item {Proveniência:(De \textunderscore neblina\textunderscore )}
\end{itemize}
Chuviscar quási imperceptivelmente.
\section{Nebri}
\begin{itemize}
\item {Grp. gram.:m.  e  adj.}
\end{itemize}
Falcão, adestrado para a caça.
(Cast. \textunderscore nebli\textunderscore )
\section{Nébrida}
\begin{itemize}
\item {Grp. gram.:f.}
\end{itemize}
Gênero de peixes acanthòpterýgios.
\section{Nébride}
\begin{itemize}
\item {Grp. gram.:f.}
\end{itemize}
\begin{itemize}
\item {Proveniência:(Do lat. \textunderscore nebris\textunderscore )}
\end{itemize}
Pelle de gamo, com que se vestiam bacchantes.
\section{Nébrido}
\begin{itemize}
\item {Grp. gram.:adj.}
\end{itemize}
Relativo a nébride:«\textunderscore nebridos despojos.\textunderscore »Filinto, II, 140.--Melhor seria \textunderscore nebrídeo\textunderscore .
\section{Nebrina}
\begin{itemize}
\item {Grp. gram.:f.}
\end{itemize}
O mesmo ou melhor que \textunderscore neblina\textunderscore . Cf. Camillo, \textunderscore Brasileira\textunderscore , 313.
\section{Nebrinoso}
\begin{itemize}
\item {Grp. gram.:adj.}
\end{itemize}
Em que há nebrina: \textunderscore dia nebrinoso\textunderscore . Cf. Arn. Gama, \textunderscore Segr. do Abb.\textunderscore , 9.
\section{Nebrite}
\begin{itemize}
\item {Grp. gram.:f.}
\end{itemize}
\begin{itemize}
\item {Proveniência:(Lat. \textunderscore nebritis\textunderscore )}
\end{itemize}
Pedra preciosa, conhecida dos antigos e consagrada a Baccho.
\section{Nébula}
\begin{itemize}
\item {Grp. gram.:f.}
\end{itemize}
\begin{itemize}
\item {Proveniência:(Lat. \textunderscore nebula\textunderscore )}
\end{itemize}
O mesmo que \textunderscore névoa\textunderscore ; nevoeiro. Cf. Carneiro Ribeiro, \textunderscore Gramm. Philos.\textunderscore , 54.
\section{Nebulento}
\begin{itemize}
\item {Grp. gram.:adj.}
\end{itemize}
\begin{itemize}
\item {Proveniência:(Do lat. \textunderscore nebula\textunderscore )}
\end{itemize}
O mesmo que \textunderscore nevoento\textunderscore .
\section{Nebulosa}
\begin{itemize}
\item {Grp. gram.:f.}
\end{itemize}
\begin{itemize}
\item {Proveniência:(De \textunderscore nebuloso\textunderscore )}
\end{itemize}
Mancha esbranquiçada, que se observa no firmamento estrellado, e que é o reflexo do agrupamento de estrellas mui distantes.
\section{Nebulosidade}
\begin{itemize}
\item {Grp. gram.:f.}
\end{itemize}
\begin{itemize}
\item {Proveniência:(Lat. \textunderscore nebulositas\textunderscore )}
\end{itemize}
Qualidade ou estado do que é nebuloso.
\section{Nebuloso}
\begin{itemize}
\item {Grp. gram.:adj.}
\end{itemize}
\begin{itemize}
\item {Proveniência:(Lat. \textunderscore nebulosus\textunderscore )}
\end{itemize}
Coberto de nuvens ou de névoa.
Obscuro; sombrio.
Pouco perceptível; incomprehensível, mysterioso: \textunderscore estilo nebuloso\textunderscore .
Torvo, ameaçador.
Triste.
\section{Necaneia}
\begin{itemize}
\item {Grp. gram.:f.}
\end{itemize}
Tecido indiano, listrado de azul e branco.
\section{Nécara}
\begin{itemize}
\item {Grp. gram.:f.}
\end{itemize}
\begin{itemize}
\item {Utilização:Prov.}
\end{itemize}
\begin{itemize}
\item {Utilização:trasm.}
\end{itemize}
\begin{itemize}
\item {Grp. gram.:Pl.}
\end{itemize}
Cada um dos cinco seixos ou pedrinhas roladas, com que se faz um jôgo entre raparigas.
Designação dêsse jôgo.
\section{Necear}
\begin{itemize}
\item {Grp. gram.:v. i.}
\end{itemize}
Dizer ou praticar necedades; dizer sandices.
(Cast. \textunderscore necear\textunderscore )
\section{Necedade}
\begin{itemize}
\item {Grp. gram.:f.}
\end{itemize}
Ignorância crassa; estupidez.
Tolice; disparate.
(Cast. \textunderscore necedad\textunderscore )
\section{Necedade}
\begin{itemize}
\item {Grp. gram.:f.}
\end{itemize}
\begin{itemize}
\item {Utilização:pop.}
\end{itemize}
\begin{itemize}
\item {Utilização:Ant.}
\end{itemize}
(Contr. de \textunderscore necessidade\textunderscore )
\section{Necessária}
\begin{itemize}
\item {Grp. gram.:f.}
\end{itemize}
\begin{itemize}
\item {Utilização:Fam.}
\end{itemize}
\begin{itemize}
\item {Proveniência:(De \textunderscore necessário\textunderscore )}
\end{itemize}
Latrina, privada.
\section{Necessáriamente}
\begin{itemize}
\item {Grp. gram.:adv.}
\end{itemize}
De modo necessário; infallivelmente; inevitavelmente.
\section{Necessário}
\begin{itemize}
\item {Grp. gram.:adj.}
\end{itemize}
\begin{itemize}
\item {Grp. gram.:M.}
\end{itemize}
\begin{itemize}
\item {Proveniência:(Lat. \textunderscore necessarius\textunderscore )}
\end{itemize}
Que tem de sêr.
Fatal.
Indispensável.
Inevitavel.
Preciso; util.
Aquillo que é indispensavel.
\section{Necessidade}
\begin{itemize}
\item {Grp. gram.:f.}
\end{itemize}
\begin{itemize}
\item {Utilização:Pop.}
\end{itemize}
\begin{itemize}
\item {Proveniência:(Lat. \textunderscore necessitas\textunderscore )}
\end{itemize}
Aquillo que é absolutamente necessário.
Fatalidade.
Aquillo que se não póde evitar.
Qualidade de necessário.
Aquillo que obriga ou constrange.
Pobreza; míngua: \textunderscore padecer necessidades\textunderscore .
Acto de urinar; dejecção: \textunderscore fazer uma necessidade\textunderscore .
\section{Necessitado}
\begin{itemize}
\item {Grp. gram.:m.}
\end{itemize}
Indivíduo pobre, indigente.
\section{Necessitante}
\begin{itemize}
\item {Grp. gram.:adj.}
\end{itemize}
Que necessita.
\section{Necessitar}
\begin{itemize}
\item {Grp. gram.:v. t.}
\end{itemize}
\begin{itemize}
\item {Grp. gram.:V. i.}
\end{itemize}
\begin{itemize}
\item {Proveniência:(Do lat. \textunderscore necessitas\textunderscore )}
\end{itemize}
Têr necessidade de.
Constranger, forçar.
Reduzir á indigência.
Exigir, tornar necessário.
Têr necessidade ou conveniência: \textunderscore o rapaz necessita de estudar\textunderscore .
Sentir privações ou necessidades.
\section{Necessitário}
\begin{itemize}
\item {Grp. gram.:m.}
\end{itemize}
\begin{itemize}
\item {Proveniência:(Do lat. \textunderscore necessitas\textunderscore )}
\end{itemize}
Sectário do fatalismo.
\section{Necessitoso}
\begin{itemize}
\item {Grp. gram.:adj.}
\end{itemize}
O mesmo que \textunderscore necessitante\textunderscore .
\section{Necestade}
\begin{itemize}
\item {Grp. gram.:f.}
\end{itemize}
\begin{itemize}
\item {Utilização:Port. de Ceilão}
\end{itemize}
O mesmo que \textunderscore necessidade\textunderscore .
\section{Nécio}
\begin{itemize}
\item {Grp. gram.:adj.}
\end{itemize}
O mesmo que \textunderscore néscio\textunderscore .
(Cast. \textunderscore nécio\textunderscore )
\section{Necissiano}
\begin{itemize}
\item {Grp. gram.:m.  e  adj.}
\end{itemize}
Nome, que se dava aos que figuravam nas necíssias.
\section{Necíssias}
\begin{itemize}
\item {Grp. gram.:f. pl.}
\end{itemize}
Festas solennes, em honra dos mortos, entre os antigos Gregos.
\section{Necodá}
\begin{itemize}
\item {Grp. gram.:m.}
\end{itemize}
\begin{itemize}
\item {Utilização:Ant.}
\end{itemize}
Chefe militar na Índia.
\section{Necra}
\begin{itemize}
\item {Grp. gram.:f.}
\end{itemize}
\begin{itemize}
\item {Utilização:Prov.}
\end{itemize}
\begin{itemize}
\item {Utilização:trasm.}
\end{itemize}
O mesmo que \textunderscore boneca\textunderscore .
(Cp. \textunderscore bonecro\textunderscore )
\section{Necróbia}
\begin{itemize}
\item {Grp. gram.:f.}
\end{itemize}
\begin{itemize}
\item {Proveniência:(Do gr. \textunderscore nekros\textunderscore  + \textunderscore bios\textunderscore )}
\end{itemize}
Gênero de insectos coleópteros pentâmeros.
\section{Necrobiose}
\begin{itemize}
\item {Grp. gram.:f.}
\end{itemize}
\begin{itemize}
\item {Utilização:Med.}
\end{itemize}
\begin{itemize}
\item {Proveniência:(Do gr. \textunderscore nekros\textunderscore  + \textunderscore bios\textunderscore )}
\end{itemize}
Morte dos elementos anatómicos num organismo vivo, em consequência do estado senil ou mórbido dêsses elementos.
\section{Necrobiótico}
\begin{itemize}
\item {Grp. gram.:adj.}
\end{itemize}
Relativo á necrobiose.
\section{Necrodulia}
\begin{itemize}
\item {Grp. gram.:f.}
\end{itemize}
\begin{itemize}
\item {Proveniência:(Do gr. \textunderscore nekros\textunderscore  + \textunderscore douleia\textunderscore )}
\end{itemize}
Culto dos mortos.
Culto, que os Chineses tributam ás almas dos seus antepassados.
\section{Necrofagia}
\begin{itemize}
\item {Grp. gram.:f.}
\end{itemize}
Alimentação cárnea.
(Cp. \textunderscore necrófago\textunderscore )
\section{Necrófago}
\begin{itemize}
\item {Grp. gram.:adj.}
\end{itemize}
\begin{itemize}
\item {Proveniência:(Do gr. \textunderscore nekros\textunderscore  + \textunderscore phagein\textunderscore )}
\end{itemize}
Que se alimenta de animaes mortos ou de substâncias em decomposição.
Aquele que se alimenta de qualquer carne ou que não é vegetalista.
\section{Necrofilia}
\begin{itemize}
\item {Grp. gram.:f.}
\end{itemize}
\begin{itemize}
\item {Utilização:Med.}
\end{itemize}
\begin{itemize}
\item {Proveniência:(Do gr. \textunderscore nekros\textunderscore  + \textunderscore philos\textunderscore )}
\end{itemize}
Mania ou aberração moral, que arrasta para a sensual e asquerosa profanação de cadáveres. Cf. Sousa Martins, \textunderscore Nosographia\textunderscore .
\section{Necrófilo}
\begin{itemize}
\item {Grp. gram.:m.}
\end{itemize}
\begin{itemize}
\item {Utilização:Med.}
\end{itemize}
Gênero de insectos clavicórneos.
Aquele que tem necrofilia.
\section{Necrofobia}
\begin{itemize}
\item {Grp. gram.:f.}
\end{itemize}
Qualidade de quem é necrófobo.
\section{Necrofóbico}
\begin{itemize}
\item {Grp. gram.:adj.}
\end{itemize}
Relativo á necrofobia.
\section{Necrófobo}
\begin{itemize}
\item {Grp. gram.:adj.}
\end{itemize}
\begin{itemize}
\item {Proveniência:(Do gr. \textunderscore nekros\textunderscore  + \textunderscore phobos\textunderscore )}
\end{itemize}
Que tem muito medo da morte.
\section{Necróforo}
\begin{itemize}
\item {Grp. gram.:m.}
\end{itemize}
\begin{itemize}
\item {Proveniência:(Do gr. \textunderscore nekros\textunderscore  + \textunderscore phoros\textunderscore )}
\end{itemize}
Gênero de grandes insectos coleópteros.
\section{Necrografia}
\begin{itemize}
\item {Grp. gram.:f.}
\end{itemize}
\begin{itemize}
\item {Proveniência:(Do gr. \textunderscore nekros\textunderscore  + \textunderscore graphein\textunderscore )}
\end{itemize}
Descripção dos corpos inanimados.
\section{Necrográfico}
\begin{itemize}
\item {Grp. gram.:adj.}
\end{itemize}
Relativo á necrografia.
\section{Necrografismo}
\begin{itemize}
\item {Grp. gram.:m.}
\end{itemize}
\begin{itemize}
\item {Proveniência:(De \textunderscore necrografia\textunderscore )}
\end{itemize}
Preferência, que certos médicos dão ao estudo dos cadáveres e esqueletos, sôbre o estudo dos seres vivos.
\section{Necrógrafo}
\begin{itemize}
\item {Grp. gram.:m.}
\end{itemize}
\begin{itemize}
\item {Proveniência:(Do gr. \textunderscore nekros\textunderscore  + \textunderscore graphein\textunderscore )}
\end{itemize}
Aquele que descreve cadáveres.
\section{Necrographia}
\begin{itemize}
\item {Grp. gram.:f.}
\end{itemize}
\begin{itemize}
\item {Proveniência:(Do gr. \textunderscore nekros\textunderscore  + \textunderscore graphein\textunderscore )}
\end{itemize}
Descripção dos corpos inanimados.
\section{Necrográphico}
\begin{itemize}
\item {Grp. gram.:adj.}
\end{itemize}
Relativo á necrographia.
\section{Necrographismo}
\begin{itemize}
\item {Grp. gram.:m.}
\end{itemize}
\begin{itemize}
\item {Proveniência:(De \textunderscore necrographia\textunderscore )}
\end{itemize}
Preferência, que certos médicos dão ao estudo dos cadáveres e esqueletos, sôbre o estudo dos seres vivos.
\section{Necrógrapho}
\begin{itemize}
\item {Grp. gram.:m.}
\end{itemize}
\begin{itemize}
\item {Proveniência:(Do gr. \textunderscore nekros\textunderscore  + \textunderscore graphein\textunderscore )}
\end{itemize}
Aquelle que descreve cadáveres.
\section{Necrólatra}
\begin{itemize}
\item {Grp. gram.:m.}
\end{itemize}
Aquelle que adora os mortos.
(Cp. \textunderscore necrolatria\textunderscore )
\section{Necrolatria}
\begin{itemize}
\item {Grp. gram.:f.}
\end{itemize}
\begin{itemize}
\item {Proveniência:(Do gr. \textunderscore nekros\textunderscore  + \textunderscore latreia\textunderscore )}
\end{itemize}
Culto dos mortos.
\section{Necrolátrico}
\begin{itemize}
\item {Grp. gram.:adj.}
\end{itemize}
Relativo á necrolatria.
\section{Necrologia}
\begin{itemize}
\item {Grp. gram.:f.}
\end{itemize}
Relação de óbitos.
Collecção de notícias, relativas a pessôas fallecidas.
(Cp. \textunderscore necrólogo\textunderscore )
\section{Necrológico}
\begin{itemize}
\item {Grp. gram.:adj.}
\end{itemize}
Relativo a necrologia ou necrológio; obituário.
\section{Necrológio}
\begin{itemize}
\item {Grp. gram.:m.}
\end{itemize}
O mesmo que \textunderscore necrologia\textunderscore .
Artigo de jornal, em que se celebram as virtudes ou méritos de pessôa fallecida.
(Cp. \textunderscore necrologia\textunderscore )
\section{Necrologista}
\begin{itemize}
\item {Grp. gram.:m.}
\end{itemize}
Escrevedor de necrológios.
\section{Necrólogo}
\begin{itemize}
\item {Grp. gram.:m.}
\end{itemize}
\begin{itemize}
\item {Proveniência:(Do gr. \textunderscore nekros\textunderscore  + \textunderscore logos\textunderscore )}
\end{itemize}
Aquelle que faz notícias necrológicas.
\section{Necromancia}
\begin{itemize}
\item {Grp. gram.:f.}
\end{itemize}
\begin{itemize}
\item {Proveniência:(Lat. \textunderscore necromantia\textunderscore )}
\end{itemize}
Supposta arte de adivinhar o futuro, por meio da evocação dos mortos.
Esconjuro.
\section{Necromante}
\begin{itemize}
\item {Grp. gram.:m. ,  f.  e  adj.}
\end{itemize}
Pessôa, que pratíca a necromancia.
(Cp. \textunderscore necromancia\textunderscore )
\section{Necromântico}
\begin{itemize}
\item {Grp. gram.:adj.}
\end{itemize}
\begin{itemize}
\item {Proveniência:(De \textunderscore necromante\textunderscore )}
\end{itemize}
Relativo á necromancia.
\section{Necronita}
\begin{itemize}
\item {Grp. gram.:f.}
\end{itemize}
\begin{itemize}
\item {Proveniência:(Do gr. \textunderscore nekros\textunderscore , \textunderscore nekron\textunderscore )}
\end{itemize}
Substância crystallina, que rasga o vidro.
\section{Necropathia}
\begin{itemize}
\item {Grp. gram.:f.}
\end{itemize}
\begin{itemize}
\item {Utilização:Med.}
\end{itemize}
\begin{itemize}
\item {Proveniência:(Do gr. \textunderscore nekros\textunderscore  + \textunderscore pathos\textunderscore )}
\end{itemize}
Disposição geral para necroses.
\section{Necropatia}
\begin{itemize}
\item {Grp. gram.:f.}
\end{itemize}
\begin{itemize}
\item {Utilização:Med.}
\end{itemize}
\begin{itemize}
\item {Proveniência:(Do gr. \textunderscore nekros\textunderscore  + \textunderscore pathos\textunderscore )}
\end{itemize}
Disposição geral para necroses.
\section{Necrophagia}
\begin{itemize}
\item {Grp. gram.:f.}
\end{itemize}
Alimentação cárnea.
(Cp. \textunderscore necróphago\textunderscore )
\section{Necróphago}
\begin{itemize}
\item {Grp. gram.:adj.}
\end{itemize}
\begin{itemize}
\item {Proveniência:(Do gr. \textunderscore nekros\textunderscore  + \textunderscore phagein\textunderscore )}
\end{itemize}
Que se alimenta de animaes mortos ou de substâncias em decomposição.
Aquelle que se alimenta de qualquer carne ou que não é vegetalista.
\section{Necrophilia}
\begin{itemize}
\item {Grp. gram.:f.}
\end{itemize}
\begin{itemize}
\item {Utilização:Med.}
\end{itemize}
\begin{itemize}
\item {Proveniência:(Do gr. \textunderscore nekros\textunderscore  + \textunderscore philos\textunderscore )}
\end{itemize}
Mania ou aberração moral, que arrasta para a sensual e asquerosa profanação de cadáveres. Cf. Sousa Martins, \textunderscore Nosographia\textunderscore .
\section{Necróphilo}
\begin{itemize}
\item {Grp. gram.:m.}
\end{itemize}
\begin{itemize}
\item {Utilização:Med.}
\end{itemize}
Gênero de insectos clavicórneos.
Aquelle que tem necrophilia.
\section{Necrophobia}
\begin{itemize}
\item {Grp. gram.:f.}
\end{itemize}
Qualidade de quem é necróphobo.
\section{Necrophóbico}
\begin{itemize}
\item {Grp. gram.:adj.}
\end{itemize}
Relativo á necrophobia.
\section{Necróphobo}
\begin{itemize}
\item {Grp. gram.:adj.}
\end{itemize}
\begin{itemize}
\item {Proveniência:(Do gr. \textunderscore nekros\textunderscore  + \textunderscore phobos\textunderscore )}
\end{itemize}
Que tem muito medo da morte.
\section{Necróphoro}
\begin{itemize}
\item {Grp. gram.:m.}
\end{itemize}
\begin{itemize}
\item {Proveniência:(Do gr. \textunderscore nekros\textunderscore  + \textunderscore phoros\textunderscore )}
\end{itemize}
Gênero de grandes insectos coleópteros.
\section{Necrópole}
\begin{itemize}
\item {Grp. gram.:f.}
\end{itemize}
\begin{itemize}
\item {Utilização:Fig.}
\end{itemize}
\begin{itemize}
\item {Proveniência:(Do gr. \textunderscore nekros\textunderscore  + \textunderscore polis\textunderscore )}
\end{itemize}
Lugar, onde se enterram os finados; cemitério.
Povoação, cujos habitantes revelam pouca actividade.
\section{Necropsia}
\begin{itemize}
\item {Grp. gram.:f.}
\end{itemize}
\begin{itemize}
\item {Proveniência:(Do gr. \textunderscore nekros\textunderscore  + \textunderscore opsis\textunderscore )}
\end{itemize}
O mesmo que \textunderscore autópsia\textunderscore .
\section{Necrópsico}
\begin{itemize}
\item {Grp. gram.:adj.}
\end{itemize}
Relativo á necropsia.
\section{Necrosado}
\begin{itemize}
\item {Grp. gram.:adj.}
\end{itemize}
Que tem necrose.
\section{Necróscia}
\begin{itemize}
\item {Grp. gram.:f.}
\end{itemize}
\begin{itemize}
\item {Proveniência:(Do gr. \textunderscore nekros\textunderscore  + \textunderscore skia\textunderscore )}
\end{itemize}
Gênero de insectos.
\section{Necroscopia}
\begin{itemize}
\item {Grp. gram.:f.}
\end{itemize}
\begin{itemize}
\item {Proveniência:(Do gr. \textunderscore nekros\textunderscore  + \textunderscore skopein\textunderscore )}
\end{itemize}
Exame ou dissecção de cadáveres.
\section{Necroscópico}
\begin{itemize}
\item {Grp. gram.:adj.}
\end{itemize}
Relativo á necroscopia.
\section{Necrose}
\begin{itemize}
\item {Grp. gram.:f.}
\end{itemize}
\begin{itemize}
\item {Proveniência:(Lat. \textunderscore necrosis\textunderscore )}
\end{itemize}
Estado de um osso ou da parte de um osso, privada de vida.
Doença dos vegetaes, caracterizada por manchas negras, sob as quaes tecidos se decompõem.
\section{Necrotério}
\begin{itemize}
\item {Grp. gram.:m.}
\end{itemize}
\begin{itemize}
\item {Proveniência:(Do gr. \textunderscore nekros\textunderscore  + \textunderscore terein\textunderscore )}
\end{itemize}
Lugar, onde se expõem os cadáveres, que vão sêr autopsiados, ou cuja identidade é necessário reconhecer antes de sepultados.
\section{Necrótomo}
\begin{itemize}
\item {Grp. gram.:m.}
\end{itemize}
\begin{itemize}
\item {Proveniência:(Do gr. \textunderscore nekros\textunderscore  + \textunderscore tome\textunderscore )}
\end{itemize}
Apparelho, formado por uma caixa de lata, em determinadas condições, para desinfecção dos exemplares de história natural, atacados de larvas de insectos.
\section{Necróvora}
\begin{itemize}
\item {Grp. gram.:f.}
\end{itemize}
\begin{itemize}
\item {Proveniência:(Do gr. \textunderscore nekros\textunderscore  + lat. \textunderscore vorare\textunderscore )}
\end{itemize}
Gênero de insectos clavicórneos.
\section{Nectândrias}
\begin{itemize}
\item {Grp. gram.:f. pl.}
\end{itemize}
Tríbo de plantas lauráceas.
\section{Néctar}
\begin{itemize}
\item {Grp. gram.:m.}
\end{itemize}
\begin{itemize}
\item {Utilização:Ext.}
\end{itemize}
\begin{itemize}
\item {Utilização:Fig.}
\end{itemize}
\begin{itemize}
\item {Proveniência:(Lat. \textunderscore nectar\textunderscore )}
\end{itemize}
Bebida dos deuses, segundo a fábula.
Qualquer bebida saborosa.
Delícia; refrigério.
Suco doce de várias plantas.
\section{Nectáreo}
\begin{itemize}
\item {Grp. gram.:adj.}
\end{itemize}
\begin{itemize}
\item {Proveniência:(Lat. \textunderscore nectareus\textunderscore )}
\end{itemize}
Relativo ou semelhante ao néctar.
\section{Nectarífero}
\begin{itemize}
\item {Grp. gram.:adj.}
\end{itemize}
\begin{itemize}
\item {Proveniência:(Do lat. \textunderscore nectar\textunderscore  + \textunderscore ferre\textunderscore )}
\end{itemize}
Que produz néctar.
\section{Nectário}
\begin{itemize}
\item {Grp. gram.:m.}
\end{itemize}
\begin{itemize}
\item {Proveniência:(De \textunderscore néctar\textunderscore )}
\end{itemize}
Parte das flôres, que segrega o suco que as abelhas haurem, para o fabríco do mel.
\section{Nectarizar}
\begin{itemize}
\item {Grp. gram.:v. t.}
\end{itemize}
\begin{itemize}
\item {Utilização:Fig.}
\end{itemize}
\begin{itemize}
\item {Proveniência:(De \textunderscore néctar\textunderscore )}
\end{itemize}
Adoçar; deliciar:«\textunderscore ...lhe houvesse nectarizado os lábios\textunderscore ». Camillo, \textunderscore Caveira\textunderscore , 143.
\section{Nectaroscordo}
\begin{itemize}
\item {Grp. gram.:m.}
\end{itemize}
Gênero de plantas liliáceas.
\section{Nectarostigma}
\begin{itemize}
\item {Grp. gram.:m.}
\end{itemize}
\begin{itemize}
\item {Utilização:Bot.}
\end{itemize}
\begin{itemize}
\item {Proveniência:(De \textunderscore nectário\textunderscore  + \textunderscore estigma\textunderscore )}
\end{itemize}
Mancha, na base das pétalas de certas flôres.
\section{Nectaroteca}
\begin{itemize}
\item {Grp. gram.:m.}
\end{itemize}
\begin{itemize}
\item {Utilização:Bot.}
\end{itemize}
\begin{itemize}
\item {Proveniência:(Do gr. \textunderscore nektar\textunderscore  + \textunderscore theke\textunderscore )}
\end{itemize}
A parte da flôr, que envolve o nectário.
\section{Nectarotheca}
\begin{itemize}
\item {Grp. gram.:m.}
\end{itemize}
\begin{itemize}
\item {Utilização:Bot.}
\end{itemize}
\begin{itemize}
\item {Proveniência:(Do gr. \textunderscore nektar\textunderscore  + \textunderscore theke\textunderscore )}
\end{itemize}
A parte da flôr, que envolve o nectário.
\section{Necídales}
\begin{itemize}
\item {Grp. gram.:m. pl.}
\end{itemize}
(V. [[necidalídeos|necidalídeo]])
\section{Necidalídeo}
\begin{itemize}
\item {Grp. gram.:adj.}
\end{itemize}
\begin{itemize}
\item {Grp. gram.:Pl.}
\end{itemize}
Relativo ou semelhante ao necídalo.
Tríbo de insectos coleópteros, da família dos longicórneos, que tem por tipo o necídalo.
\section{Necídalo}
\begin{itemize}
\item {Grp. gram.:m.}
\end{itemize}
\begin{itemize}
\item {Proveniência:(Gr. \textunderscore nekudalus\textunderscore )}
\end{itemize}
Nome científico do bicho da seda, quando se transforma em borboleta.
Gênero de insectos coleópteros heterómeros.
\section{Neciomancia}
\begin{itemize}
\item {Grp. gram.:f.}
\end{itemize}
\begin{itemize}
\item {Proveniência:(Lat. \textunderscore necyomantia\textunderscore )}
\end{itemize}
O mesmo que \textunderscore necromancia\textunderscore .
\section{Néctico}
\begin{itemize}
\item {Grp. gram.:adj.}
\end{itemize}
\begin{itemize}
\item {Utilização:Miner.}
\end{itemize}
\begin{itemize}
\item {Proveniência:(Gr. \textunderscore nektikos\textunderscore )}
\end{itemize}
Que tem a propriedade de sobrenadar ou de fluctuar na água.
\section{Nectópode}
\begin{itemize}
\item {Grp. gram.:adj.}
\end{itemize}
\begin{itemize}
\item {Utilização:Zool.}
\end{itemize}
\begin{itemize}
\item {Grp. gram.:M. pl.}
\end{itemize}
\begin{itemize}
\item {Proveniência:(Do gr. \textunderscore nektos\textunderscore  + \textunderscore pous\textunderscore , \textunderscore podos\textunderscore )}
\end{itemize}
Que tem os pés achatados e membranosos, próprios para nadar.
Família de molluscos, que têm só uma barbatana no abdome.
Família de insectos coleópteros pentâmeros.
Sub-classe de mammíferos, que comprehende amphíbios e cetáceos.
Ordem de pássaros, que têm pés membranosos.
\section{Necýdales}
\begin{itemize}
\item {Grp. gram.:m. pl.}
\end{itemize}
(V. [[necydalídeos|necydalídeo]])
\section{Necydalídeo}
\begin{itemize}
\item {Grp. gram.:adj.}
\end{itemize}
\begin{itemize}
\item {Grp. gram.:Pl.}
\end{itemize}
Relativo ou semelhante ao necýdalo.
Tríbo de insectos coleópteros, da família dos longicórneos, que tem por typo o necýdalo.
\section{Necýdalo}
\begin{itemize}
\item {Grp. gram.:m.}
\end{itemize}
\begin{itemize}
\item {Proveniência:(Gr. \textunderscore nekudalus\textunderscore )}
\end{itemize}
Nome scientífico do bicho da seda, quando se transforma em borboleta.
Gênero de insectos coleópteros heterómeros.
\section{Necyomancia}
\begin{itemize}
\item {Grp. gram.:f.}
\end{itemize}
\begin{itemize}
\item {Proveniência:(Lat. \textunderscore necyomantia\textunderscore )}
\end{itemize}
O mesmo que \textunderscore necromancia\textunderscore .
\section{Nediez}
\begin{itemize}
\item {Grp. gram.:f.}
\end{itemize}
Qualidade do que é nédio.
Aspecto lustroso, proveniente da gordura.
\section{Nédio}
\begin{itemize}
\item {Grp. gram.:adj.}
\end{itemize}
\begin{itemize}
\item {Proveniência:(Do lat. \textunderscore nitidus\textunderscore )}
\end{itemize}
Luzidio.
Nítido.
Que tem a pelle lustrosa, por effeito de gordura.
\section{Needâmia}
\begin{itemize}
\item {Grp. gram.:f.}
\end{itemize}
Gênero de plantas epacrídeas.
\section{Needhâmia}
\begin{itemize}
\item {Grp. gram.:f.}
\end{itemize}
Gênero de plantas epacrídeas.
\section{Neerlandês}
\begin{itemize}
\item {Grp. gram.:adj.}
\end{itemize}
\begin{itemize}
\item {Grp. gram.:M.}
\end{itemize}
Relativo á Neerlândia ou aos Países-Baixos.
Habitante da Neerlândia.
Língua, o mesmo que \textunderscore holandês\textunderscore , ou grupo de línguas, que abrange o holandês e o flamengo.
\section{Neésia}
\begin{itemize}
\item {Grp. gram.:f.}
\end{itemize}
Gênero de plantas arbustivas, da fam. das esterculiáceas.
\section{Nefa}
\begin{itemize}
\item {Grp. gram.:f.}
\end{itemize}
\begin{itemize}
\item {Utilização:Prov.}
\end{itemize}
\begin{itemize}
\item {Utilização:trasm.}
\end{itemize}
O mesmo que \textunderscore boneca\textunderscore .
\section{Nefandamente}
\begin{itemize}
\item {Grp. gram.:adv.}
\end{itemize}
De modo nefando; torpemente; sacrilegamente.
\section{Nefando}
\begin{itemize}
\item {Grp. gram.:adj.}
\end{itemize}
\begin{itemize}
\item {Proveniência:(Lat. \textunderscore nefandus\textunderscore )}
\end{itemize}
Que é indigno de se nomear; execrável.
Torpe; sacrilego; perverso.
Contrário á natureza.
\section{Nefariamente}
\begin{itemize}
\item {Grp. gram.:adv.}
\end{itemize}
De modo nefário; nefandamente.
\section{Nefário}
\begin{itemize}
\item {Grp. gram.:adj.}
\end{itemize}
\begin{itemize}
\item {Proveniência:(Lat. \textunderscore nefarius\textunderscore )}
\end{itemize}
O mesmo que \textunderscore nefando\textunderscore .
Extremamente malvado.
\section{Nefas}
\begin{itemize}
\item {Grp. gram.:m.}
\end{itemize}
\begin{itemize}
\item {Proveniência:(Lat. \textunderscore nefas\textunderscore )}
\end{itemize}
Aquillo que não é lícito.
Aquillo que é illegítimo; illegitimidade.--É expressão latina, us. principalmente na loc. adv. \textunderscore por fas e por nefas\textunderscore , por meios lícitos e illícitos.
\section{Nefasto}
\begin{itemize}
\item {Grp. gram.:adj.}
\end{itemize}
\begin{itemize}
\item {Proveniência:(Lat. \textunderscore nefastus\textunderscore )}
\end{itemize}
Que é de mau agoiro; que causa desgraça; funesto; triste; trágico.
\section{Nega}
\begin{itemize}
\item {Grp. gram.:m.}
\end{itemize}
Espécie de cerejeira do Canadá.
\section{Nega}
\begin{itemize}
\item {Grp. gram.:conj.}
\end{itemize}
\begin{itemize}
\item {Utilização:Ant.}
\end{itemize}
O mesmo que \textunderscore nego\textunderscore . Cp. G. Vicente, I, 261.
\section{Nega}
\begin{itemize}
\item {Grp. gram.:f.}
\end{itemize}
\begin{itemize}
\item {Utilização:Constr.}
\end{itemize}
\begin{itemize}
\item {Proveniência:(De \textunderscore negar\textunderscore )}
\end{itemize}
Negação; falta de vocação.
Incidente, no jôgo do bilhar, que redunda em proveito do parceiro contrário.
Ponto, em que uma estaca se não póde enterrar mais: \textunderscore bater uma estaca até á nega\textunderscore .
\section{Negabelha}
\begin{itemize}
\item {Grp. gram.:f.}
\end{itemize}
Planta crucífera.
\section{Negaça}
\begin{itemize}
\item {Grp. gram.:f.}
\end{itemize}
Engôdo, isca; chamariz.
Provocação.
Engano, lôgro.
\section{Negação}
\begin{itemize}
\item {Grp. gram.:f.}
\end{itemize}
\begin{itemize}
\item {Proveniência:(Lat. \textunderscore negatio\textunderscore )}
\end{itemize}
Acto ou effeito de negar.
Inaptidão, falta de vocação.
Falta, carência.
\section{Negaceador}
\begin{itemize}
\item {Grp. gram.:m.  e  adj.}
\end{itemize}
O que negaceia.
\section{Negacear}
\begin{itemize}
\item {Grp. gram.:v. t.}
\end{itemize}
\begin{itemize}
\item {Grp. gram.:V. i.}
\end{itemize}
Attrahir por meio de negaça; provocar; enganar.
Fazer negaças.
\section{Negaceiro}
\begin{itemize}
\item {Grp. gram.:m.  e  adj.}
\end{itemize}
O mesmo que \textunderscore negaceador\textunderscore .
\section{Negácia}
\begin{itemize}
\item {Grp. gram.:f.}
\end{itemize}
\begin{itemize}
\item {Utilização:Prov.}
\end{itemize}
O mesmo que \textunderscore negaça\textunderscore . Cf. Jazente, II, 320.
\section{Negado}
\begin{itemize}
\item {Grp. gram.:adj.}
\end{itemize}
Recusado; defeso.
\section{Negador}
\begin{itemize}
\item {Grp. gram.:m.  e  adj.}
\end{itemize}
\begin{itemize}
\item {Proveniência:(Lat. \textunderscore negator\textunderscore )}
\end{itemize}
O que nega.
\section{Negalho}
\begin{itemize}
\item {Grp. gram.:m.}
\end{itemize}
\begin{itemize}
\item {Utilização:Fig.}
\end{itemize}
Pequeno novelo ou pequena porção de linhas, para coser.
Cadexo.
Cordel, com que se liga alguma coisa; atilho.
Pequena porção ou pequena coisa.
Indivíduo de pequena estatura.
(Por \textunderscore ligalho\textunderscore , de \textunderscore ligar\textunderscore )
\section{Negamento}
\begin{itemize}
\item {Grp. gram.:m.}
\end{itemize}
O mesmo que \textunderscore negação\textunderscore .
\section{Negar}
\begin{itemize}
\item {Grp. gram.:v. i.}
\end{itemize}
\begin{itemize}
\item {Grp. gram.:V. i.}
\end{itemize}
\begin{itemize}
\item {Proveniência:(Lat. \textunderscore negare\textunderscore )}
\end{itemize}
Dizer que (uma coisa) não é verdadeira ou que não existe.
Affirmar que não, relativamente a (alguma coisa).
Rejeitar, recusar: \textunderscore negar responsabilidades\textunderscore .
Não permittir; prohibir: \textunderscore negar licença\textunderscore .
Não reconhecer.
Repudiar.
Desmentir.
Dizer que não.
\section{Negativa}
\begin{itemize}
\item {Grp. gram.:f.}
\end{itemize}
\begin{itemize}
\item {Proveniência:(De \textunderscore negativo\textunderscore )}
\end{itemize}
Negação.
Preposição, com que se nega alguma coisa.
Partícula, que exprime negação.
\section{Negativamente}
\begin{itemize}
\item {Grp. gram.:adv.}
\end{itemize}
De modo negativo.
\section{Negatividade}
\begin{itemize}
\item {Grp. gram.:f.}
\end{itemize}
\begin{itemize}
\item {Utilização:Phýs.}
\end{itemize}
\begin{itemize}
\item {Proveniência:(De \textunderscore negativo\textunderscore )}
\end{itemize}
Estado de um corpo, que revela a electricidade negativa.
\section{Negativista}
\begin{itemize}
\item {Grp. gram.:adj.}
\end{itemize}
Relativo á philosophia negativa. Cf. Th. Braga, \textunderscore Mod. Ideias na Lit. Port.\textunderscore , 232.
\section{Negativo}
\begin{itemize}
\item {Grp. gram.:adj.}
\end{itemize}
\begin{itemize}
\item {Utilização:Chím.}
\end{itemize}
\begin{itemize}
\item {Utilização:Phýs.}
\end{itemize}
\begin{itemize}
\item {Utilização:Mathem.}
\end{itemize}
\begin{itemize}
\item {Proveniência:(Lat. \textunderscore negativus\textunderscore )}
\end{itemize}
Que exprime ou contém negação.
Contraproducente.
Que exprime ausência ou falta.
Nullo.
Prohibitivo.
Diz-se da substância, que representa o papel de ácido.
Diz-se da electricidade desenvolvida em corpos resinosos.
Diz-se de uma quantidade menor que zero.
\section{Negatório}
\begin{itemize}
\item {Grp. gram.:adj.}
\end{itemize}
\begin{itemize}
\item {Proveniência:(Lat. \textunderscore negatorius\textunderscore )}
\end{itemize}
Que nega.
\section{Negável}
\begin{itemize}
\item {Grp. gram.:adj.}
\end{itemize}
Que se póde negar.
\section{Negligência}
\begin{itemize}
\item {Grp. gram.:f.}
\end{itemize}
\begin{itemize}
\item {Proveniência:(Lat. \textunderscore negligentia\textunderscore )}
\end{itemize}
Incúria.
Falta de diligência.
Desleixo; preguiça.
Desattenção; menosprêzo.
\section{Negligenciar}
\begin{itemize}
\item {Grp. gram.:v. t.}
\end{itemize}
Tratar com negligência; descurar; desattender.
\section{Negligente}
\begin{itemize}
\item {Grp. gram.:adj.}
\end{itemize}
\begin{itemize}
\item {Proveniência:(Lat. \textunderscore negligens\textunderscore )}
\end{itemize}
Que tem negligência.
Em que se mostra negligência.
Froixo, lânguido.
\section{Negligentemente}
\begin{itemize}
\item {Grp. gram.:adv.}
\end{itemize}
De modo negligente.
\section{Nego}
\begin{itemize}
\item {Grp. gram.:conj.}
\end{itemize}
\begin{itemize}
\item {Utilização:Ant.}
\end{itemize}
O mesmo que \textunderscore senão\textunderscore .
Algumas vezes, é voz expletiva. Cf. \textunderscore D. Nunes de Lião\textunderscore , G. Vicente, etc.
\section{Negociação}
\begin{itemize}
\item {Grp. gram.:f.}
\end{itemize}
\begin{itemize}
\item {Proveniência:(Lat. \textunderscore negotiatio\textunderscore )}
\end{itemize}
Acto ou effeito de negociar; negócio.
\section{Negociado}
\begin{itemize}
\item {Grp. gram.:adj.}
\end{itemize}
\begin{itemize}
\item {Utilização:Des.}
\end{itemize}
\begin{itemize}
\item {Proveniência:(De \textunderscore negociar\textunderscore )}
\end{itemize}
Activo; trabalhador:«\textunderscore melhor é sêr preguiçoso que homem negociado.\textunderscore »G. Vicente.
\section{Negociador}
\begin{itemize}
\item {Grp. gram.:m.  e  adj.}
\end{itemize}
\begin{itemize}
\item {Proveniência:(Lat. \textunderscore negotiator\textunderscore )}
\end{itemize}
Que negocía.
\section{Negociamento}
\begin{itemize}
\item {Grp. gram.:m.}
\end{itemize}
\begin{itemize}
\item {Proveniência:(De \textunderscore negociar\textunderscore )}
\end{itemize}
Negociação.
Acto de empregar; applicação.
\section{Negociante}
\begin{itemize}
\item {Grp. gram.:m.  e  f.}
\end{itemize}
\begin{itemize}
\item {Proveniência:(Lat. \textunderscore negotians\textunderscore )}
\end{itemize}
Pessôa, que exerce o commércio.
Pessôa, que trata de negócios.
\section{Negociar}
\begin{itemize}
\item {Grp. gram.:v. i.}
\end{itemize}
\begin{itemize}
\item {Grp. gram.:V. t.}
\end{itemize}
\begin{itemize}
\item {Proveniência:(Lat. \textunderscore negotiari\textunderscore )}
\end{itemize}
Fazer negócio; commerciar; traficar.
Agenciar.
Preparar convênio ou tratado, de govêrno para govêrno.
Fazer transacção commercial a respeito de.
Agenciar.
Promover.
Pactuar.
Permutar.
Preparar, apparelhar.
Contratar, ajustar: \textunderscore negociar a compra de um prédio\textunderscore .
\section{Negociarrão}
\begin{itemize}
\item {Grp. gram.:m.}
\end{itemize}
\begin{itemize}
\item {Utilização:Fam.}
\end{itemize}
Negócio de grande interesse.
\section{Negociata}
\begin{itemize}
\item {Grp. gram.:f.}
\end{itemize}
Negócio, em que há lôgro ou trapaça.
\section{Negociável}
\begin{itemize}
\item {Grp. gram.:adj.}
\end{itemize}
Que se póde negociar, que póde sêr objecto de permutação ou de transacção commercial.
\section{Negócio}
\begin{itemize}
\item {Grp. gram.:m.}
\end{itemize}
\begin{itemize}
\item {Proveniência:(Lat. \textunderscore negotium\textunderscore )}
\end{itemize}
O mesmo que commércio.
Tráfico.
Transacções mercantis.
Negociação.
Empresa.
Questão pendente.
Qualquer assumpto.
Ajuste.
\section{Negociosamente}
\begin{itemize}
\item {Grp. gram.:adv.}
\end{itemize}
De modo negocioso; cuidadosamente; com afan.
\section{Negocioso}
\begin{itemize}
\item {Grp. gram.:adj.}
\end{itemize}
\begin{itemize}
\item {Proveniência:(Lat. \textunderscore negotiosus\textunderscore )}
\end{itemize}
Que tem muitos negócios.
Activo, cuidadoso.
\section{Negra}
\begin{itemize}
\item {fónica:nê}
\end{itemize}
\begin{itemize}
\item {Grp. gram.:f.}
\end{itemize}
\begin{itemize}
\item {Utilização:Ext.}
\end{itemize}
\begin{itemize}
\item {Utilização:Pesc.}
\end{itemize}
\begin{itemize}
\item {Proveniência:(De \textunderscore negro\textunderscore )}
\end{itemize}
Mulher negra.
Escrava.
Mulher, que trabalha muito.
Nódoa negra na pelle.
No jôgo, a partida que desempata as anteriores.
Cardume de sardinha.
Negrinha, ave.
\section{Negraço}
\begin{itemize}
\item {Grp. gram.:m.}
\end{itemize}
Homem muito negro; homem negro.
\section{Negrada}
\begin{itemize}
\item {Grp. gram.:f.}
\end{itemize}
Porção de negros; negralhada. Cf. Júl. Ribeiro, \textunderscore Carne\textunderscore .
\section{Negra-de-pote}
\begin{itemize}
\item {Grp. gram.:f.}
\end{itemize}
\begin{itemize}
\item {Utilização:Ant.}
\end{itemize}
Negra, que transportava água, do chafariz para os domicílios.
\section{Negraínha}
\begin{itemize}
\item {Grp. gram.:f.}
\end{itemize}
Casta de uva preta ordinária.
Casta de azeitona, o mesmo que \textunderscore negrôa\textunderscore .
\section{Negral}
\begin{itemize}
\item {Grp. gram.:adj.}
\end{itemize}
\begin{itemize}
\item {Grp. gram.:Pl.}
\end{itemize}
Negro ou quási negro.
Diz-se das bexigas, (varíola), em que as pústulas ennegrecem e ás vezes gangrenam.
\section{Negralhada}
\begin{itemize}
\item {Grp. gram.:f.}
\end{itemize}
\begin{itemize}
\item {Utilização:Fam.}
\end{itemize}
O mesmo que \textunderscore negraria\textunderscore .
\section{Negralhão}
\begin{itemize}
\item {Grp. gram.:m.}
\end{itemize}
\begin{itemize}
\item {Utilização:Pop.}
\end{itemize}
Negro corpulento.
\section{Negra-moira}
\begin{itemize}
\item {Grp. gram.:f.  e  adj.}
\end{itemize}
Variedade de uva tinta.
\section{Negra-molle}
\begin{itemize}
\item {Grp. gram.:f.  e  adj.}
\end{itemize}
Variedade de uva tinta.
\section{Negrão}
\begin{itemize}
\item {Grp. gram.:m.}
\end{itemize}
\begin{itemize}
\item {Utilização:Prov.}
\end{itemize}
\begin{itemize}
\item {Utilização:alent.}
\end{itemize}
\begin{itemize}
\item {Proveniência:(De \textunderscore negro\textunderscore )}
\end{itemize}
Variedade de uva tinta do Doiro.
Peixe de Portugal, o mesmo que \textunderscore negrete\textunderscore .
Variedade de azeitona.
\section{Negrão-francês}
\begin{itemize}
\item {Grp. gram.:m.}
\end{itemize}
O mesmo que \textunderscore tinturão\textunderscore .
\section{Negrão-miúdo}
\begin{itemize}
\item {Grp. gram.:m.}
\end{itemize}
Casta de azeitona, o mesmo que \textunderscore negrucha\textunderscore .
\section{Negraria}
\begin{itemize}
\item {Grp. gram.:f.}
\end{itemize}
Multidão de negros.
\section{Negregado}
\begin{itemize}
\item {Grp. gram.:adj.}
\end{itemize}
\begin{itemize}
\item {Proveniência:(Do lat. \textunderscore nigricatus\textunderscore )}
\end{itemize}
Desgraçado.
Trabalhoso.
\section{Negregoso}
\begin{itemize}
\item {Grp. gram.:adj.}
\end{itemize}
Muito negro.
\section{Negregura}
\begin{itemize}
\item {Grp. gram.:f.}
\end{itemize}
(V.negrura)
\section{Negreirinha}
\begin{itemize}
\item {Grp. gram.:f.}
\end{itemize}
\begin{itemize}
\item {Proveniência:(De \textunderscore negro\textunderscore )}
\end{itemize}
Espécie de ameixa pequena, comprida, preta, maculada de azul.
\section{Negreiro}
\begin{itemize}
\item {Grp. gram.:m.  e  adj.}
\end{itemize}
Traficante de negros.
Navio que faz tráfico de escravos. Cf. Camillo, \textunderscore Myst. de Lisb.\textunderscore , II, 17.
\section{Negrejamento}
\begin{itemize}
\item {Grp. gram.:m.}
\end{itemize}
Acto de negrejar. Cf. Eça. \textunderscore P. Amaro\textunderscore , 534.
\section{Negrejante}
\begin{itemize}
\item {Grp. gram.:adj.}
\end{itemize}
\begin{itemize}
\item {Proveniência:(Do lat. \textunderscore nigricans\textunderscore )}
\end{itemize}
Que negreja.
\section{Negrejar}
\begin{itemize}
\item {Grp. gram.:v. i.}
\end{itemize}
\begin{itemize}
\item {Utilização:Fig.}
\end{itemize}
\begin{itemize}
\item {Proveniência:(Do lat. \textunderscore nigricare\textunderscore )}
\end{itemize}
Sêr negro.
Tornar-se negro.
Mostrar-se negro.
Causar sombra.
Estar de luto.
Mostrar-se triste.
Revelar-se desagradavelmente.
Infundir tristeza.
\section{Negrela}
\begin{itemize}
\item {Grp. gram.:f.}
\end{itemize}
\begin{itemize}
\item {Utilização:T. de Turquel}
\end{itemize}
\begin{itemize}
\item {Proveniência:(De \textunderscore negro\textunderscore )}
\end{itemize}
Ave palmípede, (\textunderscore fulígula cristata\textunderscore , Lin.).
O mesmo que \textunderscore ecchymose\textunderscore .
\section{Negresia}
\begin{itemize}
\item {Grp. gram.:f.}
\end{itemize}
\begin{itemize}
\item {Utilização:T. de Turquel}
\end{itemize}
Reunião de muitas coisas negras: \textunderscore o faval tem tanto piolho, que é uma negresia\textunderscore .
\section{Negrete}
\begin{itemize}
\item {fónica:grê}
\end{itemize}
\begin{itemize}
\item {Grp. gram.:m.}
\end{itemize}
\begin{itemize}
\item {Proveniência:(De \textunderscore negro\textunderscore )}
\end{itemize}
Pequeno peixe fluvial do gênero mugem.
\section{Negridão}
\begin{itemize}
\item {Grp. gram.:f.}
\end{itemize}
O mesmo que \textunderscore negrura\textunderscore .
\section{Negrigente}
\begin{itemize}
\item {Grp. gram.:adj.}
\end{itemize}
\begin{itemize}
\item {Utilização:Ant.}
\end{itemize}
O mesmo que \textunderscore negligente\textunderscore .
\section{Negrilha}
\begin{itemize}
\item {Grp. gram.:f.}
\end{itemize}
\begin{itemize}
\item {Utilização:Prov.}
\end{itemize}
\begin{itemize}
\item {Utilização:alent.}
\end{itemize}
Variedade de couve.
\section{Negrilho}
\begin{itemize}
\item {Grp. gram.:m.}
\end{itemize}
\begin{itemize}
\item {Utilização:Prov.}
\end{itemize}
\begin{itemize}
\item {Utilização:trasm.}
\end{itemize}
\begin{itemize}
\item {Grp. gram.:Pl.}
\end{itemize}
\begin{itemize}
\item {Proveniência:(De \textunderscore negro\textunderscore )}
\end{itemize}
Indivíduo negro, de pouca idade.
Variedade de antigo tecido negro de lan.
O mesmo que \textunderscore ulmeiro\textunderscore .
Vidrilhos pretos.
\section{Negrinha}
\begin{itemize}
\item {Grp. gram.:f.}
\end{itemize}
\begin{itemize}
\item {Proveniência:(De \textunderscore negrinho\textunderscore )}
\end{itemize}
Planta herbácea, que nasce nos trigaes.
Ave palmípede, (\textunderscore oidemia nigra\textunderscore , ou \textunderscore fuligula cristata\textunderscore , Lin.).
Ave, o mesmo que \textunderscore chasco\textunderscore ^2.
Vara, que servia de insígnia ao mordomo-mór do paço.
\section{Negrinho}
\begin{itemize}
\item {Grp. gram.:m.  e  adj.}
\end{itemize}
\begin{itemize}
\item {Proveniência:(De \textunderscore negro\textunderscore )}
\end{itemize}
Variedade de chouriço, também conhecido por \textunderscore chouriço moiro\textunderscore .
Variedade de uva tinta.
Negrela.
\section{Negritos}
\begin{itemize}
\item {Grp. gram.:m. pl.}
\end{itemize}
Indigenas da Nova Guiné e dos archipélagos próximos.
\section{Negro}
\begin{itemize}
\item {Grp. gram.:adj.}
\end{itemize}
\begin{itemize}
\item {Grp. gram.:M.}
\end{itemize}
\begin{itemize}
\item {Proveniência:(Do lat. \textunderscore niger\textunderscore )}
\end{itemize}
Que é da côr mais privada de luz ou opposta á branca.
Escuro, preto.
Sombrio.
Escurecido pelo tempo ou pelo sol.
Vestido de preto.
Lúgubre; triste.
Funesto; maldito: \textunderscore negra sorte\textunderscore .
Execrável.
Indivíduo de raça negra, preto.
Escravo.
Sombras, trevas.
Negrinha, ave.
\textunderscore Negro dos bosques\textunderscore , insecto lepidóptero.
\section{Negrôa}
\begin{itemize}
\item {Grp. gram.:adj. f.}
\end{itemize}
Diz-se de uma casta de azeitona, também conhecida pelos nomes de \textunderscore negral\textunderscore , \textunderscore cereal\textunderscore , \textunderscore madural\textunderscore , \textunderscore mean\textunderscore , \textunderscore miúda\textunderscore , \textunderscore mollar\textunderscore , \textunderscore mollarinha\textunderscore , \textunderscore negrão\textunderscore , \textunderscore toural\textunderscore , etc.
\section{Negrófilo}
\begin{itemize}
\item {Grp. gram.:adj.}
\end{itemize}
\begin{itemize}
\item {Grp. gram.:M.}
\end{itemize}
\begin{itemize}
\item {Proveniência:(De \textunderscore negro\textunderscore  + gr. \textunderscore philos\textunderscore )}
\end{itemize}
Que gósta dos Negros.
Partidário da abolição da escravatura.
\section{Negróide}
\begin{itemize}
\item {Grp. gram.:adj.}
\end{itemize}
\begin{itemize}
\item {Grp. gram.:M.}
\end{itemize}
\begin{itemize}
\item {Proveniência:(De \textunderscore negro\textunderscore  + gr. \textunderscore eidos\textunderscore )}
\end{itemize}
Semelhante aos Negros.
Indivíduo, semelhante aos da raça negra.
\section{Negrola}
\begin{itemize}
\item {Grp. gram.:f.}
\end{itemize}
O mesmo que \textunderscore negrinha\textunderscore , ave.
\section{Negro-melro}
\begin{itemize}
\item {Grp. gram.:m.}
\end{itemize}
\begin{itemize}
\item {Utilização:Prov.}
\end{itemize}
O mesmo que \textunderscore melro\textunderscore .
\section{Negro-mina}
\begin{itemize}
\item {Grp. gram.:m.}
\end{itemize}
\begin{itemize}
\item {Utilização:Bras}
\end{itemize}
Árvore silvestre.
\section{Negróphilo}
\begin{itemize}
\item {Grp. gram.:adj.}
\end{itemize}
\begin{itemize}
\item {Grp. gram.:M.}
\end{itemize}
\begin{itemize}
\item {Proveniência:(De \textunderscore negro\textunderscore  + gr. \textunderscore philos\textunderscore )}
\end{itemize}
Que gósta dos Negros.
Partidário da abolição da escravatura.
\section{Negror}
\begin{itemize}
\item {Grp. gram.:m.}
\end{itemize}
\begin{itemize}
\item {Proveniência:(Do lat. \textunderscore nigror\textunderscore )}
\end{itemize}
Negrura; escuridão densa.
\section{Negrotins}
\begin{itemize}
\item {Grp. gram.:m. pl.}
\end{itemize}
\begin{itemize}
\item {Utilização:Des.}
\end{itemize}
O mesmo que \textunderscore gregotins\textunderscore .
\section{Negrucha}
\begin{itemize}
\item {Grp. gram.:adj. f.}
\end{itemize}
O mesmo que \textunderscore negrôa\textunderscore .
\section{Negrucho}
\begin{itemize}
\item {Grp. gram.:adj.}
\end{itemize}
\begin{itemize}
\item {Utilização:Prov.}
\end{itemize}
\begin{itemize}
\item {Utilização:trasm.}
\end{itemize}
Um tanto negro.
\section{Negrume}
\begin{itemize}
\item {Grp. gram.:m.}
\end{itemize}
O mesmo que \textunderscore negrura\textunderscore .
Cerração de nuvens; trevas.
Tristeza.
\section{Negrura}
\begin{itemize}
\item {Grp. gram.:f.}
\end{itemize}
\begin{itemize}
\item {Utilização:Fig.}
\end{itemize}
Qualidade do que é negro.
Escuridão.
Perversidade.
Rudeza.
Crime.
\section{Negruza}
\begin{itemize}
\item {Grp. gram.:f.}
\end{itemize}
Variedade de azeitona, o mesmo que \textunderscore negrôa\textunderscore ; negrucha.
\section{Negum}
\begin{itemize}
\item {Grp. gram.:pron.}
\end{itemize}
\begin{itemize}
\item {Utilização:Ant.}
\end{itemize}
\begin{itemize}
\item {Proveniência:(Do lat. \textunderscore nec\textunderscore  + \textunderscore unus\textunderscore )}
\end{itemize}
O mesmo que \textunderscore nenhum\textunderscore .
\section{Negundo}
\begin{itemize}
\item {Grp. gram.:m.}
\end{itemize}
Gênero de plantas aceríneas.
\section{Negus}
\begin{itemize}
\item {Grp. gram.:m.}
\end{itemize}
Título do soberano da Abyssínia.
\section{Neiquibar}
\begin{itemize}
\item {Grp. gram.:m.}
\end{itemize}
Chefe de aldeia, na antiga Índia portuguesa.
\section{Neixente}
\begin{itemize}
\item {Grp. gram.:m.}
\end{itemize}
Rês?:«\textunderscore ...quando o pastor haja cerrado... no aprisco os seus neixentes...\textunderscore »Castilho, \textunderscore Fastos\textunderscore , II, 101.
\section{Neja}
\begin{itemize}
\item {Grp. gram.:adj.}
\end{itemize}
\begin{itemize}
\item {Utilização:Prov.}
\end{itemize}
O mesmo que \textunderscore nanja\textunderscore .
\section{Nela}
Expressão contraida, equivalente a \textunderscore em ela\textunderscore .
(Cp. \textunderscore no\textunderscore ^1)
\section{Neldo}
\begin{itemize}
\item {Grp. gram.:m.}
\end{itemize}
\begin{itemize}
\item {Proveniência:(Do cast. \textunderscore aneldo\textunderscore )}
\end{itemize}
Variedade de maçan.
\section{Nele}
\begin{itemize}
\item {Grp. gram.:m.}
\end{itemize}
\begin{itemize}
\item {Utilização:Ant.}
\end{itemize}
Arroz com casca, na Índia portuguesa.
\section{Nele}
\begin{itemize}
\item {Grp. gram.:m.}
\end{itemize}
Antiga moéda francesa.
\section{Nele}
\begin{itemize}
\item {fónica:nê}
\end{itemize}
Expressão contraida equivalente a \textunderscore em êle\textunderscore .
(Cp. \textunderscore no\textunderscore ^1)
\section{Nelênsia}
\begin{itemize}
\item {Grp. gram.:f.}
\end{itemize}
Gênero de plantas acantháceas.
\section{Nélia}
\begin{itemize}
\item {Grp. gram.:f.}
\end{itemize}
Gênero de plantas rosáceas.
\section{Nelita}
\begin{itemize}
\item {Grp. gram.:f.}
\end{itemize}
Gênero de plantas leguminosas.
\section{Nella}
Expressão contrahida, equivalente a \textunderscore em ella\textunderscore .
(Cp. \textunderscore no\textunderscore ^1)
\section{Nelle}
Expressão contrahida equivalente a \textunderscore em êle\textunderscore .
(Cp. \textunderscore no\textunderscore ^1)
\section{Néllia}
\begin{itemize}
\item {Grp. gram.:f.}
\end{itemize}
Gênero de plantas rosáceas.
\section{Nello}
\begin{itemize}
\item {Utilização:Ant.}
\end{itemize}
\begin{itemize}
\item {Proveniência:(Do lat. \textunderscore in\textunderscore  + \textunderscore illo\textunderscore )}
\end{itemize}
\textunderscore pron.\textunderscore  ou expressão contrahida, equivalente a \textunderscore em\textunderscore  + \textunderscore ello\textunderscore .
Nisso; a tal respeito.
\section{Nelo}
\begin{itemize}
\item {fónica:nê}
\end{itemize}
\begin{itemize}
\item {Utilização:Ant.}
\end{itemize}
\begin{itemize}
\item {Proveniência:(Do lat. \textunderscore in\textunderscore  + \textunderscore illo\textunderscore )}
\end{itemize}
\textunderscore pron.\textunderscore  ou expressão contrahida, equivalente a \textunderscore em\textunderscore  + \textunderscore elo\textunderscore .
Nisso; a tal respeito.
\section{Nelsónia}
\begin{itemize}
\item {Grp. gram.:f.}
\end{itemize}
\begin{itemize}
\item {Proveniência:(De \textunderscore Nelson\textunderscore , n. p.)}
\end{itemize}
Gênero de plantas acantháceas.
\section{Nelúmbio}
\begin{itemize}
\item {Grp. gram.:f.}
\end{itemize}
O mesmo que \textunderscore nelumbo\textunderscore .
\section{Nelumbo}
\begin{itemize}
\item {Grp. gram.:m.}
\end{itemize}
\begin{itemize}
\item {Proveniência:(Do lat. bot. \textunderscore nelumbium\textunderscore )}
\end{itemize}
Gênero de plantas nympheáceas da América e da Ásia.
\section{Nelumbóneas}
\begin{itemize}
\item {Grp. gram.:f. pl.}
\end{itemize}
Família de plantas mònòcotyledóneas, que tem por typo o \textunderscore nelumbo\textunderscore .
\section{Nelúmula}
\begin{itemize}
\item {Grp. gram.:f.}
\end{itemize}
\begin{itemize}
\item {Utilização:Bot.}
\end{itemize}
Espécie de jasmim do Malabar.
\section{Nem}
\begin{itemize}
\item {Grp. gram.:adv.}
\end{itemize}
\begin{itemize}
\item {Grp. gram.:Conj.}
\end{itemize}
\begin{itemize}
\item {Utilização:Des.}
\end{itemize}
\begin{itemize}
\item {Proveniência:(Do lat. \textunderscore nec\textunderscore )}
\end{itemize}
Não; até não.
E não.
E sem.
Ou.
Não, se bem que.--Algumas vezes, equivale a uma negativa, precedida da conj. \textunderscore e\textunderscore . E outras vezes exprime negação, modificada variamente.
\section{Nefálio}
\begin{itemize}
\item {Grp. gram.:m.}
\end{itemize}
Gênero de insectos longicórneos.
\section{Nefelemancia}
\begin{itemize}
\item {Grp. gram.:f.}
\end{itemize}
\begin{itemize}
\item {Proveniência:(Do gr. \textunderscore nephele\textunderscore  + \textunderscore manteia\textunderscore )}
\end{itemize}
Arte de adivinhar, pela inspecção das nuvens.
\section{Nemálitho}
\begin{itemize}
\item {Grp. gram.:m.}
\end{itemize}
\begin{itemize}
\item {Utilização:Miner.}
\end{itemize}
\begin{itemize}
\item {Proveniência:(Do gr. \textunderscore nema\textunderscore  + \textunderscore lithos\textunderscore )}
\end{itemize}
Óxydo hydratado de magnésio.
\section{Nemálito}
\begin{itemize}
\item {Grp. gram.:m.}
\end{itemize}
\begin{itemize}
\item {Utilização:Miner.}
\end{itemize}
\begin{itemize}
\item {Proveniência:(Do gr. \textunderscore nema\textunderscore  + \textunderscore lithos\textunderscore )}
\end{itemize}
Óxido hidratado de magnésio.
\section{Nemapalpo}
\begin{itemize}
\item {Grp. gram.:m.}
\end{itemize}
Gênero de insectos dípteros.
\section{Nematanto}
\begin{itemize}
\item {Grp. gram.:m.}
\end{itemize}
\begin{itemize}
\item {Proveniência:(Do gr. \textunderscore nema\textunderscore  + \textunderscore anthos\textunderscore )}
\end{itemize}
Gênero de plantas gesneriáceas.
\section{Nemathanto}
\begin{itemize}
\item {Grp. gram.:m.}
\end{itemize}
\begin{itemize}
\item {Proveniência:(Do gr. \textunderscore nema\textunderscore  + \textunderscore anthos\textunderscore )}
\end{itemize}
Gênero de plantas gesneriáceas.
\section{Nematélia}
\begin{itemize}
\item {Grp. gram.:f.}
\end{itemize}
Gênero de cogumelos.
\section{Nematelmintho}
\begin{itemize}
\item {Grp. gram.:m.}
\end{itemize}
\begin{itemize}
\item {Proveniência:(De \textunderscore nêmato\textunderscore  + \textunderscore helmintho\textunderscore )}
\end{itemize}
Classe de vermes parasitos, cylíndricos, e de sexos separados.
\section{Nematelminto}
\begin{itemize}
\item {Grp. gram.:m.}
\end{itemize}
\begin{itemize}
\item {Proveniência:(De \textunderscore nêmato\textunderscore  + \textunderscore helmintho\textunderscore )}
\end{itemize}
Classe de vermes parasitos, cilíndricos, e de sexos separados.
\section{Nêmato}
\begin{itemize}
\item {Grp. gram.:m.}
\end{itemize}
\begin{itemize}
\item {Proveniência:(Do gr. \textunderscore nema\textunderscore )}
\end{itemize}
Gênero de insectos hymenópteros.
\section{Nematócero}
\begin{itemize}
\item {Grp. gram.:adj.}
\end{itemize}
\begin{itemize}
\item {Utilização:Zool.}
\end{itemize}
\begin{itemize}
\item {Proveniência:(Do gr. \textunderscore nema\textunderscore  + \textunderscore keras\textunderscore )}
\end{itemize}
Que tem antennas filiformes.
\section{Nematociste}
\begin{itemize}
\item {Grp. gram.:m.}
\end{itemize}
\begin{itemize}
\item {Utilização:Zool.}
\end{itemize}
\begin{itemize}
\item {Proveniência:(Do gr. \textunderscore nema\textunderscore  + \textunderscore kustis\textunderscore )}
\end{itemize}
Cada um dos órgãos urticantes dos celenterados.
\section{Nematocisto}
\begin{itemize}
\item {Grp. gram.:m.}
\end{itemize}
\begin{itemize}
\item {Utilização:Zool.}
\end{itemize}
\begin{itemize}
\item {Proveniência:(Do gr. \textunderscore nema\textunderscore  + \textunderscore kustis\textunderscore )}
\end{itemize}
Cada um dos órgãos urticantes dos celenterados.
\section{Nematocysto}
\begin{itemize}
\item {Grp. gram.:m.}
\end{itemize}
\begin{itemize}
\item {Utilização:Zool.}
\end{itemize}
\begin{itemize}
\item {Proveniência:(Do gr. \textunderscore nema\textunderscore  + \textunderscore kustis\textunderscore )}
\end{itemize}
Cada um dos órgãos urticantes dos celenterados.
\section{Nematófora}
\begin{itemize}
\item {Grp. gram.:f.}
\end{itemize}
\begin{itemize}
\item {Proveniência:(Do gr. \textunderscore nema\textunderscore  + \textunderscore phoros\textunderscore )}
\end{itemize}
Gênero de insectos coleópteros pentâmeros.
\section{Nematóide}
\begin{itemize}
\item {Grp. gram.:adj.}
\end{itemize}
\begin{itemize}
\item {Grp. gram.:M. pl.}
\end{itemize}
\begin{itemize}
\item {Proveniência:(Do gr. \textunderscore nema\textunderscore  + \textunderscore eidos\textunderscore )}
\end{itemize}
Que tem a fórma de um fio.
Ordem de helminthos cylíndricos e filiformes.
\section{Nematómice}
\begin{itemize}
\item {Grp. gram.:m.}
\end{itemize}
Tríbo de cogumelos, de fórma filamentosa.
\section{Nematómyce}
\begin{itemize}
\item {Grp. gram.:m.}
\end{itemize}
Tríbo de cogumelos, de fórma filamentosa.
\section{Nematóphora}
\begin{itemize}
\item {Grp. gram.:f.}
\end{itemize}
\begin{itemize}
\item {Proveniência:(Do gr. \textunderscore nema\textunderscore  + \textunderscore phoros\textunderscore )}
\end{itemize}
Gênero de insectos coleópteros pentâmeros.
\section{Nemátope}
\begin{itemize}
\item {Grp. gram.:m.}
\end{itemize}
\begin{itemize}
\item {Proveniência:(Do gr. \textunderscore nema\textunderscore  + \textunderscore pous\textunderscore )}
\end{itemize}
Gênero de insectos hemípteros.
\section{Nemátopo}
\begin{itemize}
\item {Grp. gram.:m.}
\end{itemize}
\begin{itemize}
\item {Proveniência:(Do gr. \textunderscore nema\textunderscore  + \textunderscore pous\textunderscore )}
\end{itemize}
Gênero de insectos hemípteros.
\section{Nemazoários}
\begin{itemize}
\item {Grp. gram.:m. pl.}
\end{itemize}
\begin{itemize}
\item {Proveniência:(Do gr. \textunderscore nema\textunderscore  + \textunderscore zoon\textunderscore )}
\end{itemize}
Classe de seres ambíguos que, segundo alguns naturalistas, é comprehendida entre os infusórios, e, segundo outros, entre as algas.
\section{Nemba}
\begin{itemize}
\item {Grp. gram.:f.}
\end{itemize}
Árvore do Congo, própria para mastros de embarcações.
\section{Nembo}
\begin{itemize}
\item {Grp. gram.:m.}
\end{itemize}
Massiço entre vãos, em obra de pedreiro.
Casta de uva. Cf. \textunderscore Tech. Rur.\textunderscore , 565.
\section{Nembrar}
\textunderscore v. t.\textunderscore  (e der.) \textunderscore Ant.\textunderscore 
O mesmo que \textunderscore lembrar\textunderscore , etc.
\section{Nembro}
\begin{itemize}
\item {Grp. gram.:m.}
\end{itemize}
\begin{itemize}
\item {Utilização:Prov.}
\end{itemize}
\begin{itemize}
\item {Utilização:minh.}
\end{itemize}
\begin{itemize}
\item {Utilização:Ant.}
\end{itemize}
O mesmo que \textunderscore membro\textunderscore . Cf. Mestre Geraldo, \textunderscore Enferm. das Aves\textunderscore .
\section{Nemedra}
\begin{itemize}
\item {Grp. gram.:f.}
\end{itemize}
Gênero de plantas meliáceas.
\section{Nemeóbio}
\begin{itemize}
\item {Grp. gram.:m.}
\end{itemize}
Gênero de insectos lepidópteros.
\section{Nemeófila}
\begin{itemize}
\item {Grp. gram.:f.}
\end{itemize}
Gênero de insectos lepidópteros.
\section{Nemeóphila}
\begin{itemize}
\item {Grp. gram.:f.}
\end{itemize}
Gênero de insectos lepidópteros.
\section{Nemerte}
\begin{itemize}
\item {Grp. gram.:m.}
\end{itemize}
Gênero de helminthos marinhos.
\section{Nemésia}
\begin{itemize}
\item {Grp. gram.:f.}
\end{itemize}
Gênero de plantas escrofularíneas.
\section{Nemésico}
\begin{itemize}
\item {Grp. gram.:adj.}
\end{itemize}
\begin{itemize}
\item {Proveniência:(De \textunderscore Nêmesis\textunderscore , n. p. myth. da deusa da vingança)}
\end{itemize}
Que é instrumento de vingança:«\textunderscore ...com que lhe festejaram o látego nemésico\textunderscore ». Camillo, \textunderscore Cancion. Al.\textunderscore , 362.
\section{Nêmesis}
\begin{itemize}
\item {Grp. gram.:m.}
\end{itemize}
Gênero de crustáceos.
\section{Nemestrina}
\begin{itemize}
\item {Grp. gram.:f.}
\end{itemize}
Gênero de insectos dípteros.
\section{Nemeu}
\begin{itemize}
\item {Grp. gram.:adj.}
\end{itemize}
\begin{itemize}
\item {Utilização:Poét.}
\end{itemize}
\begin{itemize}
\item {Grp. gram.:Pl.}
\end{itemize}
\begin{itemize}
\item {Proveniência:(Lat. \textunderscore nemeaeus\textunderscore )}
\end{itemize}
Diz-se do leão, afogado por Hércules no bosque de Nêmea. Cf. Filinto, IX, 207.
Diz-se dos jogos públicos, instituidos por Hércules, perto de Nêmea.
\section{Nemicelo}
\begin{itemize}
\item {Grp. gram.:m.}
\end{itemize}
Gênero de insectos xylóphagos.
\section{Nemigalha}
\begin{itemize}
\item {Grp. gram.:pron.}
\end{itemize}
\begin{itemize}
\item {Utilização:Ant.}
\end{itemize}
\begin{itemize}
\item {Proveniência:(De \textunderscore nem\textunderscore  + \textunderscore migalha\textunderscore )}
\end{itemize}
Nada, nenhuma coisa.
\section{Nemóbia}
\begin{itemize}
\item {Grp. gram.:f.}
\end{itemize}
Gênero de insectos orthópteros.
\section{Nemoblasto}
\begin{itemize}
\item {Grp. gram.:m.}
\end{itemize}
\begin{itemize}
\item {Utilização:Bot.}
\end{itemize}
\begin{itemize}
\item {Proveniência:(Do gr. \textunderscore nema\textunderscore  + \textunderscore blastos\textunderscore )}
\end{itemize}
Embryão filiforme.
\section{Nemocéfalo}
\begin{itemize}
\item {Grp. gram.:m.}
\end{itemize}
\begin{itemize}
\item {Proveniência:(Do gr. \textunderscore nema\textunderscore  + \textunderscore kephale\textunderscore )}
\end{itemize}
Gênero de insectos coleópteros pentâmeros.
\section{Nemocéphalo}
\begin{itemize}
\item {Grp. gram.:m.}
\end{itemize}
\begin{itemize}
\item {Proveniência:(Do gr. \textunderscore nema\textunderscore  + \textunderscore kephale\textunderscore )}
\end{itemize}
Gênero de insectos coleópteros pentâmeros.
\section{Nemóceros}
\begin{itemize}
\item {Grp. gram.:m. pl.}
\end{itemize}
\begin{itemize}
\item {Proveniência:(Do gr. \textunderscore nema\textunderscore  + \textunderscore keras\textunderscore )}
\end{itemize}
Família de insectos dípteros, que tem antennas filiformes.
\section{Nemofante}
\begin{itemize}
\item {Grp. gram.:m.}
\end{itemize}
Gênero de plantas ilicíneas.
\section{Nemófilo}
\begin{itemize}
\item {Grp. gram.:m.}
\end{itemize}
\begin{itemize}
\item {Proveniência:(Do gr. \textunderscore nemos\textunderscore  + \textunderscore philos\textunderscore )}
\end{itemize}
Gênero de plantas americanas, ornamentaes, da fam. das hydrófiláceas.
\section{Nemófora}
\begin{itemize}
\item {Grp. gram.:f.}
\end{itemize}
\begin{itemize}
\item {Proveniência:(Do gr. \textunderscore nema\textunderscore  + \textunderscore phoros\textunderscore )}
\end{itemize}
Gênero de insectos lepidópteros.
\section{Nemólitho}
\begin{itemize}
\item {Grp. gram.:m.}
\end{itemize}
\begin{itemize}
\item {Proveniência:(Do gr. \textunderscore nemos\textunderscore  + \textunderscore lithos\textunderscore )}
\end{itemize}
Rocha arborizada.
\section{Nemólito}
\begin{itemize}
\item {Grp. gram.:m.}
\end{itemize}
\begin{itemize}
\item {Proveniência:(Do gr. \textunderscore nemos\textunderscore  + \textunderscore lithos\textunderscore )}
\end{itemize}
Rocha arborizada.
\section{Nemopantho}
\begin{itemize}
\item {Grp. gram.:m.}
\end{itemize}
Gênero de plantas rhamnáceas.
\section{Nemopanto}
\begin{itemize}
\item {Grp. gram.:m.}
\end{itemize}
Gênero de plantas ramnáceas.
\section{Nemophante}
\begin{itemize}
\item {Grp. gram.:m.}
\end{itemize}
Gênero de plantas ilicíneas.
\section{Nemóphilo}
\begin{itemize}
\item {Grp. gram.:m.}
\end{itemize}
\begin{itemize}
\item {Proveniência:(Do gr. \textunderscore nemos\textunderscore  + \textunderscore philos\textunderscore )}
\end{itemize}
Gênero de plantas americanas, ornamentaes, da fam. das hydróphylláceas.
\section{Nemóphora}
\begin{itemize}
\item {Grp. gram.:f.}
\end{itemize}
\begin{itemize}
\item {Proveniência:(Do gr. \textunderscore nema\textunderscore  + \textunderscore phoros\textunderscore )}
\end{itemize}
Gênero de insectos lepidópteros.
\section{Nemópode}
\begin{itemize}
\item {Grp. gram.:m.}
\end{itemize}
Gênero de insectos dípteros.
\section{Nemoral}
\begin{itemize}
\item {Grp. gram.:adj.}
\end{itemize}
\begin{itemize}
\item {Proveniência:(Lat. \textunderscore nemoralis\textunderscore )}
\end{itemize}
Relativo a bosques; que existe nos bosques.
\section{Nemoreira}
\begin{itemize}
\item {Grp. gram.:f.}
\end{itemize}
\begin{itemize}
\item {Utilização:T. das Caldas da Raínha}
\end{itemize}
\begin{itemize}
\item {Proveniência:(Do lat. hyp. \textunderscore nemoraria\textunderscore )}
\end{itemize}
Parte arborizada de uma estrada.
Arvoredo basto.
\section{Nemoroso}
\begin{itemize}
\item {Grp. gram.:adj.}
\end{itemize}
\begin{itemize}
\item {Proveniência:(Lat. \textunderscore nemorosus\textunderscore )}
\end{itemize}
Sombreado de árvores; coberto de arvoredo.
Próprio do arvoredo.
\section{Nemosomo}
\begin{itemize}
\item {fónica:sô}
\end{itemize}
\begin{itemize}
\item {Grp. gram.:m.}
\end{itemize}
Gênero de insectos coleópteros tetrâmeros.
\section{Nemossomo}
\begin{itemize}
\item {Grp. gram.:m.}
\end{itemize}
Gênero de insectos coleópteros tetrâmeros.
\section{Nemótelo}
\begin{itemize}
\item {Grp. gram.:m.}
\end{itemize}
\begin{itemize}
\item {Proveniência:(Do gr. \textunderscore nema\textunderscore  + \textunderscore telos\textunderscore )}
\end{itemize}
Gênero de insectos dípteros.
\section{Nemplé}
\begin{itemize}
\item {Grp. gram.:m.}
\end{itemize}
Arbusto medicinal da Guiné.
\section{Nemu}
\begin{itemize}
\item {Grp. gram.:adv.}
\end{itemize}
\begin{itemize}
\item {Utilização:Ant.}
\end{itemize}
No mesmo instante; immediatamente.
\section{Nena}
\begin{itemize}
\item {Grp. gram.:f.}
\end{itemize}
\begin{itemize}
\item {Utilização:T. de Miranda}
\end{itemize}
Grande bebedeira.
\section{Nena}
\begin{itemize}
\item {Grp. gram.:f.}
\end{itemize}
\begin{itemize}
\item {Utilização:Fam.}
\end{itemize}
O mesmo que \textunderscore boneca\textunderscore .
\section{Nêncio}
\begin{itemize}
\item {Grp. gram.:m.}
\end{itemize}
\begin{itemize}
\item {Utilização:Prov.}
\end{itemize}
\begin{itemize}
\item {Utilização:alent.}
\end{itemize}
O mesmo que \textunderscore néscio\textunderscore .
\section{Nendi}
\begin{itemize}
\item {Grp. gram.:m.}
\end{itemize}
Ave columbina da África.
\section{Nèné}
\begin{itemize}
\item {Grp. gram.:m.}
\end{itemize}
\begin{itemize}
\item {Utilização:Fam.}
\end{itemize}
Criança.
Criança recem-nascida ou de poucos meses.
\section{Nêngara}
\begin{itemize}
\item {Grp. gram.:f.}
\end{itemize}
\begin{itemize}
\item {Utilização:Prov.}
\end{itemize}
\begin{itemize}
\item {Utilização:trasm.}
\end{itemize}
O mesmo que \textunderscore nengra\textunderscore .
\section{Nêngaro}
\begin{itemize}
\item {Grp. gram.:f.}
\end{itemize}
\begin{itemize}
\item {Utilização:Prov.}
\end{itemize}
\begin{itemize}
\item {Utilização:trasm.}
\end{itemize}
O mesmo que \textunderscore nengro\textunderscore .
\section{Nengra}
\begin{itemize}
\item {Grp. gram.:f.}
\end{itemize}
\begin{itemize}
\item {Utilização:Prov.}
\end{itemize}
\begin{itemize}
\item {Proveniência:(De \textunderscore nengro\textunderscore )}
\end{itemize}
Menina recém-nascida.
Boneca.
\section{Nengro}
\begin{itemize}
\item {Grp. gram.:m.}
\end{itemize}
\begin{itemize}
\item {Utilização:Prov.}
\end{itemize}
Menino recém-nascido.
\section{Nenho}
\begin{itemize}
\item {Grp. gram.:adj.}
\end{itemize}
(V.inhenho)
\section{Nenhum}
\begin{itemize}
\item {Grp. gram.:adj.}
\end{itemize}
\begin{itemize}
\item {Proveniência:(De \textunderscore nem\textunderscore  + \textunderscore um\textunderscore , interposta uma consoante palatal)}
\end{itemize}
Nem um.
Nullo.
\section{Nenhumamente}
\begin{itemize}
\item {Grp. gram.:adv.}
\end{itemize}
De modo nenhum. Cf. Castilho, \textunderscore Fastos\textunderscore , I, 292.
\section{Nenhures}
\begin{itemize}
\item {Grp. gram.:adv.}
\end{itemize}
\begin{itemize}
\item {Proveniência:(De \textunderscore nenhum\textunderscore )}
\end{itemize}
Em nenhuma parte.
\section{Nênia}
\begin{itemize}
\item {Grp. gram.:f.}
\end{itemize}
\begin{itemize}
\item {Proveniência:(Lat. \textunderscore nenia\textunderscore )}
\end{itemize}
Canto fúnebre.
Canção triste.
\section{Nênio}
\begin{itemize}
\item {Grp. gram.:m.}
\end{itemize}
Fachada nos theatros antigos.
\section{Neno}
\begin{itemize}
\item {Grp. gram.:m.}
\end{itemize}
\begin{itemize}
\item {Utilização:T. de Lamego}
\end{itemize}
O mesmo que \textunderscore boneco\textunderscore .
\section{Nente}
\begin{itemize}
\item {Grp. gram.:adv.}
\end{itemize}
Nada; não. Cf. Camillo, \textunderscore Corja\textunderscore , 154.
\section{Nentes}
\begin{itemize}
\item {Grp. gram.:adv.}
\end{itemize}
\begin{itemize}
\item {Utilização:Gír.}
\end{itemize}
Nada; não. Cf. Camillo, \textunderscore Corja\textunderscore , 154.
\section{Nenúfar}
\begin{itemize}
\item {Grp. gram.:m.}
\end{itemize}
\begin{itemize}
\item {Proveniência:(Do ár. \textunderscore neinufar\textunderscore )}
\end{itemize}
Gênero de plantas aquáticas, que serve de typo ás nympheáceas.
\section{Néo...}
\begin{itemize}
\item {Grp. gram.:pref.}
\end{itemize}
\begin{itemize}
\item {Proveniência:(Do gr. \textunderscore neos\textunderscore )}
\end{itemize}
(designativo de \textunderscore novo\textunderscore )
\section{Néo-Bhramanismo}
\begin{itemize}
\item {Grp. gram.:m.}
\end{itemize}
O mesmo que \textunderscore induísmo\textunderscore .
\section{Néo-Budismo}
\begin{itemize}
\item {Grp. gram.:m.}
\end{itemize}
Observância actual das theorias budistas.
Applicação dessas theorias á literatura.
\section{Néo-budista}
\begin{itemize}
\item {Grp. gram.:adj.}
\end{itemize}
\begin{itemize}
\item {Grp. gram.:M.}
\end{itemize}
Relativo ao néo-budismo.
Sectário do néo-budismo.
\section{Néo-Catholicismo}
\begin{itemize}
\item {Grp. gram.:m.}
\end{itemize}
Doutrina, que procura aproximar o catholicismo das ideias modernas de progresso e liberdade.
\section{Néo-cathólico}
\begin{itemize}
\item {Grp. gram.:adj.}
\end{itemize}
\begin{itemize}
\item {Grp. gram.:M.}
\end{itemize}
Relativo ao néo-catholicismo.
Sectário do néo-catholicismo.
\section{Néo-céltico}
\begin{itemize}
\item {Grp. gram.:adj.}
\end{itemize}
Diz-se das línguas vivas, derivadas das célticas.
\section{Neocomiano}
\begin{itemize}
\item {Grp. gram.:adj.}
\end{itemize}
\begin{itemize}
\item {Utilização:Zool.}
\end{itemize}
Diz-se do uma espécie de terreno cretáceo.
\section{Neocór}
\begin{itemize}
\item {Grp. gram.:m.}
\end{itemize}
\begin{itemize}
\item {Utilização:Mús.}
\end{itemize}
Nome, com que dantes se designava a trompa de pistões.
\section{Neócoro}
\begin{itemize}
\item {Grp. gram.:m.}
\end{itemize}
\begin{itemize}
\item {Proveniência:(Lat. \textunderscore neocorus\textunderscore )}
\end{itemize}
Aquelle que, entre os pagãos, velava pela limpeza e bôa ordem dos templos.
\section{Néo-cýclico}
\begin{itemize}
\item {Grp. gram.:adj.}
\end{itemize}
Que succedeu no começo de certo período chronológico.
\section{Neófita}
\begin{itemize}
\item {Grp. gram.:f.}
\end{itemize}
\begin{itemize}
\item {Proveniência:(Lat. \textunderscore neophyta\textunderscore )}
\end{itemize}
Primitivamente, a pagan que, pouco antes, se convertera ao Cristianismo.
Pessôa do sexo feminino, que acaba de sêr baptizada.
Noviça.
\section{Neófito}
\begin{itemize}
\item {Grp. gram.:m.}
\end{itemize}
\begin{itemize}
\item {Proveniência:(Lat. \textunderscore neophytus\textunderscore )}
\end{itemize}
Antigamente, pagão, que tinha abraçado, havia pouco, o Cristianismo.
O que recebe ou acabou de receber o baptismo.
Noviço; principiante.
Indivíduo, admitido recentemente numa corporação.
\section{Néo-formação}
\begin{itemize}
\item {Grp. gram.:f.}
\end{itemize}
Formação incipiente de tecidos orgânicos.
Formação anormal ou pathológica de tecidos orgânicos.
\section{Neógala}
\begin{itemize}
\item {Grp. gram.:m.}
\end{itemize}
\begin{itemize}
\item {Utilização:Med.}
\end{itemize}
\begin{itemize}
\item {Proveniência:(Do gr. \textunderscore neos\textunderscore  + \textunderscore gala\textunderscore )}
\end{itemize}
O primeiro leite, segregado depois do colostro.
\section{Neógamo}
\begin{itemize}
\item {Grp. gram.:m.}
\end{itemize}
\begin{itemize}
\item {Proveniência:(Do gr. \textunderscore neo\textunderscore  + \textunderscore gamos\textunderscore )}
\end{itemize}
Indivíduo recém-casado.
\section{Neogêneo}
\begin{itemize}
\item {Grp. gram.:adj.}
\end{itemize}
\begin{itemize}
\item {Utilização:Geol.}
\end{itemize}
\begin{itemize}
\item {Proveniência:(Do gr. \textunderscore neos\textunderscore  + \textunderscore genos\textunderscore )}
\end{itemize}
Diz-se de uma espécie de terreno da série terciária.
\section{Néo-gongorismo}
\begin{itemize}
\item {Grp. gram.:m.}
\end{itemize}
Linguagem, estilo ou gosto literário dos néo-gongoristas.
\section{Néo-gongorista}
\begin{itemize}
\item {Grp. gram.:m.}
\end{itemize}
Escritor moderno, que imita o gongorismo.
\section{Neo-gótico}
\begin{itemize}
\item {Grp. gram.:adj.}
\end{itemize}
Diz-se das decorações ou de quaesquer trabalhos artísticos, modernos, que imitam o gôsto ou a feição gótica. Cf. Herculano, \textunderscore Quest., Públ.\textunderscore , II, 25.
\section{Neografia}
\begin{itemize}
\item {Grp. gram.:f.}
\end{itemize}
\begin{itemize}
\item {Proveniência:(De \textunderscore neógrafo\textunderscore )}
\end{itemize}
Ortografia nova, quer seguindo estrictamente a prosódia, quer observando todo o rigor etimológico.
\section{Neografismo}
\begin{itemize}
\item {Grp. gram.:m.}
\end{itemize}
Conjunto das regras ou princípios dos neógrafos.
\section{Neógrafo}
\begin{itemize}
\item {Grp. gram.:m.  e  adj.}
\end{itemize}
\begin{itemize}
\item {Proveniência:(Do gr. \textunderscore neos\textunderscore  + \textunderscore graphein\textunderscore )}
\end{itemize}
Aquele que admite ou pratíca ortografia nova.
\section{Neographia}
\begin{itemize}
\item {Grp. gram.:f.}
\end{itemize}
\begin{itemize}
\item {Proveniência:(De \textunderscore neógrapho\textunderscore )}
\end{itemize}
Orthographia nova, quer seguindo estrictamente a prosódia, quer observando todo o rigor etymológico.
\section{Neographismo}
\begin{itemize}
\item {Grp. gram.:m.}
\end{itemize}
Conjunto das regras ou princípios dos neógraphos.
\section{Neógrapho}
\begin{itemize}
\item {Grp. gram.:m.  e  adj.}
\end{itemize}
\begin{itemize}
\item {Proveniência:(Do gr. \textunderscore neos\textunderscore  + \textunderscore graphein\textunderscore )}
\end{itemize}
Aquelle que admitte ou pratíca orthographia nova.
\section{Néo-grego}
\begin{itemize}
\item {Grp. gram.:adj.}
\end{itemize}
Grego moderno.
\section{Néo-latino}
\begin{itemize}
\item {Grp. gram.:adj.}
\end{itemize}
Diz-se das línguas modernas, que derivam do latim.
Diz-se das nações, cuja língua ou civilização procede da latina.
\section{Néo-linguista}
\begin{itemize}
\item {Grp. gram.:m.}
\end{itemize}
\begin{itemize}
\item {Utilização:Neol.}
\end{itemize}
Aquelle que é perito em línguas modernas.
\section{Neolíthica}
\begin{itemize}
\item {Grp. gram.:f.}
\end{itemize}
\begin{itemize}
\item {Utilização:Archeol.}
\end{itemize}
\begin{itemize}
\item {Proveniência:(De \textunderscore neolíthico\textunderscore )}
\end{itemize}
Segundo período da idade de pedra.
Idade da pedra polida.
\section{Neolíthico}
\begin{itemize}
\item {Grp. gram.:adj.}
\end{itemize}
\begin{itemize}
\item {Proveniência:(Do gr. \textunderscore neos\textunderscore  + \textunderscore lithos\textunderscore )}
\end{itemize}
Relativo a pedra polida.
\section{Neolítica}
\begin{itemize}
\item {Grp. gram.:f.}
\end{itemize}
\begin{itemize}
\item {Utilização:Archeol.}
\end{itemize}
\begin{itemize}
\item {Proveniência:(De \textunderscore neolítico\textunderscore )}
\end{itemize}
Segundo período da idade de pedra.
Idade da pedra polida.
\section{Neolítico}
\begin{itemize}
\item {Grp. gram.:adj.}
\end{itemize}
\begin{itemize}
\item {Proveniência:(Do gr. \textunderscore neos\textunderscore  + \textunderscore lithos\textunderscore )}
\end{itemize}
Relativo a pedra polida.
\section{Neologia}
\begin{itemize}
\item {Grp. gram.:f.}
\end{itemize}
\begin{itemize}
\item {Proveniência:(De \textunderscore neólogo\textunderscore )}
\end{itemize}
Emprêgo de palavras novas ou de novas accepções.
Admissão de doutrinas muito recentes.
\section{Neologicamente}
\begin{itemize}
\item {Grp. gram.:adv.}
\end{itemize}
De modo neológico.
\section{Neológico}
\begin{itemize}
\item {Grp. gram.:adj.}
\end{itemize}
Relativo á neologia.
\section{Neologismo}
\begin{itemize}
\item {Grp. gram.:m.}
\end{itemize}
\begin{itemize}
\item {Proveniência:(De \textunderscore neologia\textunderscore )}
\end{itemize}
O mesmo que \textunderscore neologia\textunderscore .
Palavra ou phrase nova.
Doutrina nova.
\section{Neologista}
\begin{itemize}
\item {Grp. gram.:m. ,  f.  e  adj.}
\end{itemize}
\begin{itemize}
\item {Proveniência:(De \textunderscore neologia\textunderscore )}
\end{itemize}
Pessôa, que emprega neologismos.
\section{Neólogo}
\begin{itemize}
\item {Grp. gram.:m.  e  adj.}
\end{itemize}
\begin{itemize}
\item {Proveniência:(Do gr. \textunderscore neos\textunderscore  + \textunderscore logos\textunderscore )}
\end{itemize}
O mesmo que \textunderscore neologista\textunderscore .
\section{Néo-lutheranismo}
\begin{itemize}
\item {Grp. gram.:m.}
\end{itemize}
Doutrina politicò-religiosa, preconizada na Alemanha em 1848.
\section{Néo-lutherano}
\begin{itemize}
\item {Grp. gram.:m.}
\end{itemize}
Sectário do néo-lutheranismo.
\section{Néo-membrana}
\begin{itemize}
\item {Grp. gram.:f.}
\end{itemize}
\begin{itemize}
\item {Utilização:Anat.}
\end{itemize}
Membrana vascular, de formação recente, e cujos elementos fundamentaes são fibras semelhantes ás das membranas normaes do organismo.
\section{Neomênia}
\begin{itemize}
\item {Grp. gram.:f.}
\end{itemize}
\begin{itemize}
\item {Proveniência:(Lat. \textunderscore neomenia\textunderscore )}
\end{itemize}
Designação antiga da lua-nova.
Festa, que os antigos celebravam em cada novilúnio.
\section{Neo-mysticismo}
\begin{itemize}
\item {Grp. gram.:m.}
\end{itemize}
Adopção actual do mysticismo em literatura.
\section{Néo-mýstico}
\begin{itemize}
\item {Grp. gram.:adj.}
\end{itemize}
\begin{itemize}
\item {Grp. gram.:M.}
\end{itemize}
Relativo ao néo-mysticismo.
Sectário do néo-mysticismo.
\section{Néon}
\begin{itemize}
\item {Grp. gram.:m.}
\end{itemize}
\begin{itemize}
\item {Proveniência:(Do gr. \textunderscore neos\textunderscore )}
\end{itemize}
Um dos elementos da atmosphera, recentemente descobertos.
\section{Neonómio}
\begin{itemize}
\item {Grp. gram.:m.}
\end{itemize}
\begin{itemize}
\item {Proveniência:(Do gr. \textunderscore neos\textunderscore  + \textunderscore nomos\textunderscore )}
\end{itemize}
Membro de uma seita christan, que rejeitava o \textunderscore Antigo Testamento\textunderscore , acceitando só o \textunderscore Evangelho\textunderscore .
\section{Néo-phobia}
\begin{itemize}
\item {Grp. gram.:f.}
\end{itemize}
Aversão a invenções, a descobrimentos, a progressos, a tudo que é novo.
\section{Néo-phonema}
\begin{itemize}
\item {Grp. gram.:m.}
\end{itemize}
Phonema que, na língua vernácula, é novo em relação á língua mãe, como o \textunderscore lh\textunderscore  e o \textunderscore nh\textunderscore  nas línguas românicas. Cf. Júl. Ribeiro, \textunderscore Diccion. Gram.\textunderscore 
\section{Neóphyta}
\begin{itemize}
\item {Grp. gram.:f.}
\end{itemize}
\begin{itemize}
\item {Proveniência:(Lat. \textunderscore neophyta\textunderscore )}
\end{itemize}
Primitivamente, a pagan que, pouco antes, se convertera ao Christianismo.
Pessôa do sexo feminino, que acaba de sêr baptizada.
Noviça.
\section{Neóphyto}
\begin{itemize}
\item {Grp. gram.:m.}
\end{itemize}
\begin{itemize}
\item {Proveniência:(Lat. \textunderscore neophytus\textunderscore )}
\end{itemize}
Antigamente, pagão, que tinha abraçado, havia pouco, o Christianismo.
O que recebe ou acabou de receber o baptismo.
Noviço; principiante.
Indivíduo, admittido recentemente numa corporação.
\section{Neoplasia}
\begin{itemize}
\item {Grp. gram.:f.}
\end{itemize}
O mesmo que \textunderscore neoplastia\textunderscore .
Néoformação pathológica.
\section{Neoplásico}
\begin{itemize}
\item {Grp. gram.:adj.}
\end{itemize}
Relativo á neoplasia.
\section{Neoplasma}
\begin{itemize}
\item {Grp. gram.:m.}
\end{itemize}
\begin{itemize}
\item {Proveniência:(Do gr. \textunderscore neos\textunderscore  + \textunderscore plasma\textunderscore )}
\end{itemize}
Tecido orgânico, de formação recente.
\section{Neoplasmo}
\begin{itemize}
\item {Grp. gram.:m.}
\end{itemize}
O mesmo que \textunderscore neoplasma\textunderscore .
\section{Neoplastia}
\begin{itemize}
\item {Grp. gram.:f.}
\end{itemize}
\begin{itemize}
\item {Proveniência:(Do gr. \textunderscore neos\textunderscore  + \textunderscore plassein\textunderscore )}
\end{itemize}
Renovação dos tecidos orgânicos, por adherência ou autoplastia.
\section{Neoplástico}
\begin{itemize}
\item {Grp. gram.:adj.}
\end{itemize}
Relativo á neoplastia.
\section{Neo-platónico}
\begin{itemize}
\item {Grp. gram.:adj.}
\end{itemize}
Relativo ao neo-platonismo.
\section{Néo-platonismo}
\begin{itemize}
\item {Grp. gram.:m.}
\end{itemize}
Doutrina dos que misturavam o antigo platonismo com a theologia e demonologia.
\section{Néo-quinhentismo}
\begin{itemize}
\item {Grp. gram.:m.}
\end{itemize}
Estilo, gôsto ou escola dos néo-quinhentistas.
\section{Néo-quinhentista}
\begin{itemize}
\item {Grp. gram.:m.}
\end{itemize}
Escritor moderno, que imita a linguagem ou o estilo dos quinhentistas.
\section{Neorama}
\begin{itemize}
\item {Grp. gram.:m.}
\end{itemize}
\begin{itemize}
\item {Proveniência:(Do gr. \textunderscore neos\textunderscore  + \textunderscore orama\textunderscore )}
\end{itemize}
Espécie de panorama, representando o interior de um edifício.
\section{Néo-romano}
\begin{itemize}
\item {Grp. gram.:adj.}
\end{itemize}
Relativo aos Romanos dos últimos tempos. Cf. Latino, \textunderscore Elogios\textunderscore , 51.
\section{Neossina}
\begin{itemize}
\item {Grp. gram.:f.}
\end{itemize}
\begin{itemize}
\item {Proveniência:(Do gr. \textunderscore neossia\textunderscore )}
\end{itemize}
Substância orgânica, que se encontra em o ninho da andorinha da China, (\textunderscore hirundo esculenta\textunderscore , Lin.)
\section{Neozoico}
\begin{itemize}
\item {Grp. gram.:adj.}
\end{itemize}
\begin{itemize}
\item {Utilização:Geol.}
\end{itemize}
\begin{itemize}
\item {Proveniência:(Do gr. \textunderscore neos\textunderscore  + \textunderscore zoon\textunderscore )}
\end{itemize}
Relativo aos seres, que mais recentemente viveram sôbre a terra.
\section{Nepa}
\begin{itemize}
\item {Grp. gram.:f.}
\end{itemize}
\begin{itemize}
\item {Proveniência:(Lat. \textunderscore nepa\textunderscore )}
\end{itemize}
Gênero de insectos hemípteros.
\section{Nepalês}
\begin{itemize}
\item {Grp. gram.:m.}
\end{itemize}
\begin{itemize}
\item {Proveniência:(De \textunderscore Nepal\textunderscore , n. p.)}
\end{itemize}
Língua, falada no reino de Nepal.
\section{Nepali}
\begin{itemize}
\item {Grp. gram.:m.}
\end{itemize}
\begin{itemize}
\item {Proveniência:(De \textunderscore Nepal\textunderscore , n. p.)}
\end{itemize}
Língua, falada no reino de Nepal.
\section{Nepenta}
\begin{itemize}
\item {Grp. gram.:f.}
\end{itemize}
\begin{itemize}
\item {Proveniência:(Gr. \textunderscore nepenthes\textunderscore )}
\end{itemize}
Gênero de plantas asiáticas.
\section{Nepentáceas}
\begin{itemize}
\item {Grp. gram.:f. pl.}
\end{itemize}
\begin{itemize}
\item {Proveniência:(De \textunderscore nepenta\textunderscore )}
\end{itemize}
Ordem de plantas dicotiledóneas.
\section{Nepentha}
\begin{itemize}
\item {Grp. gram.:f.}
\end{itemize}
\begin{itemize}
\item {Proveniência:(Gr. \textunderscore nepenthes\textunderscore )}
\end{itemize}
Gênero de plantas asiáticas.
\section{Nepentháceas}
\begin{itemize}
\item {Grp. gram.:f. pl.}
\end{itemize}
\begin{itemize}
\item {Proveniência:(De \textunderscore nepentha\textunderscore )}
\end{itemize}
Ordem de plantas dycotyledóneas.
\section{Neperiano}
\begin{itemize}
\item {Grp. gram.:adj.}
\end{itemize}
Que foi inventado por Neper, (falando-se de logaríthmos).
\section{Nephálio}
\begin{itemize}
\item {Grp. gram.:m.}
\end{itemize}
Gênero de insectos longicórneos.
\section{Nephelemancia}
\begin{itemize}
\item {Grp. gram.:f.}
\end{itemize}
\begin{itemize}
\item {Proveniência:(Do gr. \textunderscore nephele\textunderscore  + \textunderscore manteia\textunderscore )}
\end{itemize}
Arte de adivinhar, pela inspecção das nuvens.
\section{Nefelibata}
\begin{itemize}
\item {Grp. gram.:m.  e  adj.}
\end{itemize}
\begin{itemize}
\item {Utilização:Fig.}
\end{itemize}
\begin{itemize}
\item {Utilização:deprec.}
\end{itemize}
\begin{itemize}
\item {Utilização:Ext.}
\end{itemize}
O que anda ou vive nas nuvens.
Próprio de quem vive nas nuvens.
Literato excêntrico, que desconhece ou despreza os processos conhecidos e o bom senso literário.
Indivíduo que, dominado por um supposto ideal, não attende aos factos da vida positiva nem ás lições da experiência.
(Do gr.\textunderscore  nephele\textunderscore  + \textunderscore bates\textunderscore )
\section{Nefelíbata}
\begin{itemize}
\item {Grp. gram.:m.  e  adj.}
\end{itemize}
\begin{itemize}
\item {Utilização:Fig.}
\end{itemize}
\begin{itemize}
\item {Utilização:deprec.}
\end{itemize}
\begin{itemize}
\item {Utilização:Ext.}
\end{itemize}
O que anda ou vive nas nuvens.
Próprio de quem vive nas nuvens.
Literato excêntrico, que desconhece ou despreza os processos conhecidos e o bom senso literário.
Indivíduo que, dominado por um supposto ideal, não attende aos factos da vida positiva nem ás lições da experiência.
(Do gr.\textunderscore  nephele\textunderscore  + \textunderscore bates\textunderscore )
\section{Nefelibatice}
\begin{itemize}
\item {Grp. gram.:f.}
\end{itemize}
\begin{itemize}
\item {Utilização:Deprec.}
\end{itemize}
Opinião, acto ou dito, próprio de nefelibata.
\section{Nefelibático}
\begin{itemize}
\item {Grp. gram.:adj.}
\end{itemize}
Relativo a nefelibata.
\section{Nefelibatismo}
\begin{itemize}
\item {Grp. gram.:m.}
\end{itemize}
Qualidade ou sistema de nefelibata.
\section{Nefelina}
\begin{itemize}
\item {Grp. gram.:f.}
\end{itemize}
\begin{itemize}
\item {Utilização:Miner.}
\end{itemize}
\begin{itemize}
\item {Proveniência:(Do gr. \textunderscore nephele\textunderscore )}
\end{itemize}
Substância, composta de sílica, alumina e soda, e fusível no maçarico.
\section{Nefelínico}
\begin{itemize}
\item {Grp. gram.:adj.}
\end{itemize}
\begin{itemize}
\item {Utilização:Miner.}
\end{itemize}
Que contém nefelina.
\section{Nefelinito}
\begin{itemize}
\item {Grp. gram.:m.}
\end{itemize}
\begin{itemize}
\item {Utilização:Miner.}
\end{itemize}
Rocha basáltica, em que predomina a nefelina.
\section{Nefélio}
\begin{itemize}
\item {Grp. gram.:m.}
\end{itemize}
\begin{itemize}
\item {Proveniência:(Lat. \textunderscore nephelion\textunderscore )}
\end{itemize}
Pequena mancha, na camada exterior da córnea, e que deixa passar a luz como através de uma nuvem.
\section{Nefelite}
\begin{itemize}
\item {Grp. gram.:f.}
\end{itemize}
\begin{itemize}
\item {Proveniência:(Do gr. \textunderscore nephele\textunderscore , nuvem)}
\end{itemize}
Espécie de mineral, vizinha do feldspato, e que se encontra com abundância nas lavas do Vesúvio.
\section{Nefelitito}
\begin{itemize}
\item {Grp. gram.:m.}
\end{itemize}
\begin{itemize}
\item {Utilização:Geol.}
\end{itemize}
Espécie de tephrito, em que predomina a nefelite.
\section{Nefelóide}
\begin{itemize}
\item {Grp. gram.:adj.}
\end{itemize}
\begin{itemize}
\item {Proveniência:(Do gr. \textunderscore nephele\textunderscore  + \textunderscore eidos\textunderscore )}
\end{itemize}
Que tem o aspecto de nuvem.
\section{Néfodo}
\begin{itemize}
\item {Grp. gram.:m.}
\end{itemize}
Gênero de insectos coleópteros heterómeros.
\section{Nefralgia}
\begin{itemize}
\item {Grp. gram.:f.}
\end{itemize}
\begin{itemize}
\item {Utilização:Med.}
\end{itemize}
\begin{itemize}
\item {Proveniência:(Do gr. \textunderscore nephros\textunderscore  + \textunderscore algos\textunderscore )}
\end{itemize}
Dôr de rins, acompanhada de tremuras, calefrios, abundância de urina e, ás vezes, vómitos.
\section{Nefrálgico}
\begin{itemize}
\item {Grp. gram.:adj.}
\end{itemize}
Relativo a nefralgia.
\section{Nefrectomia}
\begin{itemize}
\item {Grp. gram.:f.}
\end{itemize}
\begin{itemize}
\item {Utilização:Cir.}
\end{itemize}
\begin{itemize}
\item {Proveniência:(Do gr. \textunderscore nephros\textunderscore  + \textunderscore ektome\textunderscore )}
\end{itemize}
Extirpação total dos rins.
\section{Nefrelita}
\begin{itemize}
\item {Grp. gram.:f.}
\end{itemize}
\begin{itemize}
\item {Utilização:Miner.}
\end{itemize}
Espécie de serpentina.
\section{Nefrelita}
\begin{itemize}
\item {Grp. gram.:f.}
\end{itemize}
\begin{itemize}
\item {Proveniência:(Do gr. \textunderscore nephros\textunderscore )}
\end{itemize}
Nome, que se dava á gordura que rodeia os rins.
\section{Nefrelmíntico}
\begin{itemize}
\item {Grp. gram.:adj.}
\end{itemize}
\begin{itemize}
\item {Utilização:Med.}
\end{itemize}
\begin{itemize}
\item {Proveniência:(Do gr. \textunderscore nephros\textunderscore  + \textunderscore helminthos\textunderscore )}
\end{itemize}
Que tem vermes nos rins.
\section{Nefrenfraxe}
\begin{itemize}
\item {Grp. gram.:f.}
\end{itemize}
\begin{itemize}
\item {Utilização:Med.}
\end{itemize}
Inchação ou obstrucção dos rins.
\section{Nefrina}
\begin{itemize}
\item {Grp. gram.:f.}
\end{itemize}
\begin{itemize}
\item {Utilização:Med.}
\end{itemize}
\begin{itemize}
\item {Proveniência:(Do gr. \textunderscore nephros\textunderscore )}
\end{itemize}
O mesmo que \textunderscore ureia\textunderscore .
\section{Nefroflegmasia}
\begin{itemize}
\item {Grp. gram.:f.}
\end{itemize}
\begin{itemize}
\item {Utilização:Med.}
\end{itemize}
Inflamação dos rins.
\section{Nefroflegmático}
\begin{itemize}
\item {Grp. gram.:adj.}
\end{itemize}
Procedente das mucosidades que se soltam dos rins.
\section{Nefroge}
\begin{itemize}
\item {Grp. gram.:m.}
\end{itemize}
Gênero de arbustos trepadores, monospermos.
\section{Nefrografia}
\begin{itemize}
\item {Grp. gram.:f.}
\end{itemize}
\begin{itemize}
\item {Proveniência:(Do gr. \textunderscore nephros\textunderscore  + \textunderscore graphein\textunderscore )}
\end{itemize}
Descripção dos rins.
\section{Nefrográfico}
\begin{itemize}
\item {Grp. gram.:adj.}
\end{itemize}
Relativo á nefrografia.
\section{Nefrógrafo}
\begin{itemize}
\item {Grp. gram.:m.}
\end{itemize}
\begin{itemize}
\item {Proveniência:(Do gr. \textunderscore nephros\textunderscore  + \textunderscore graphein\textunderscore )}
\end{itemize}
Aquele que escreve á cêrca da estructura, funcções e doenças dos rins.
\section{Nefróide}
\begin{itemize}
\item {Grp. gram.:adj.}
\end{itemize}
\begin{itemize}
\item {Proveniência:(Do gr. \textunderscore nephros\textunderscore  + \textunderscore eidos\textunderscore )}
\end{itemize}
Que tem fórma de rim.
\section{Nefrolitíase}
\begin{itemize}
\item {Grp. gram.:f.}
\end{itemize}
\begin{itemize}
\item {Proveniência:(De \textunderscore nefrólitho\textunderscore )}
\end{itemize}
Doença, causada pelos cálculos que se formam nos rins.
\section{Nefrolítico}
\begin{itemize}
\item {Grp. gram.:adj.}
\end{itemize}
Relativo ao nefrólito.
Dependente do nefrólito.
\section{Nefrólito}
\begin{itemize}
\item {Grp. gram.:m.}
\end{itemize}
\begin{itemize}
\item {Utilização:Med.}
\end{itemize}
\begin{itemize}
\item {Proveniência:(Do gr. \textunderscore nephros\textunderscore  + \textunderscore lithos\textunderscore )}
\end{itemize}
Pedra ou cálculo, que se fórma nos rins.
\section{Nefrolitotomia}
\begin{itemize}
\item {Grp. gram.:f.}
\end{itemize}
\begin{itemize}
\item {Proveniência:(Do gr. \textunderscore nephros\textunderscore  + \textunderscore lithos\textunderscore  + \textunderscore tome\textunderscore )}
\end{itemize}
Operação cirúrgica, com que se abre o rim, para extrahir alguma pedra ou cálculo.
\section{Nefrolitotómico}
\begin{itemize}
\item {Grp. gram.:adj.}
\end{itemize}
Relativo á nefrolithotomia.
\section{Nefrologia}
\begin{itemize}
\item {Grp. gram.:f.}
\end{itemize}
\begin{itemize}
\item {Proveniência:(Do gr. \textunderscore nephros\textunderscore  + \textunderscore logos\textunderscore )}
\end{itemize}
Tratado á cêrca dos rins.
\section{Nefrológico}
\begin{itemize}
\item {Grp. gram.:adj.}
\end{itemize}
Relativo á nefrologia.
\section{Nefrologista}
\begin{itemize}
\item {Grp. gram.:m.}
\end{itemize}
\begin{itemize}
\item {Proveniência:(De \textunderscore nefrologia\textunderscore )}
\end{itemize}
Aquele que estuda especialmente as doenças dos rins.
\section{Nefrólogo}
\begin{itemize}
\item {Grp. gram.:m.}
\end{itemize}
\begin{itemize}
\item {Proveniência:(Do gr. \textunderscore nephros\textunderscore  + \textunderscore logos\textunderscore )}
\end{itemize}
Aquele que escreveu sôbre nefrologia.
\section{Néfrope}
\begin{itemize}
\item {Grp. gram.:m.}
\end{itemize}
\begin{itemize}
\item {Proveniência:(Do gr. \textunderscore nephros\textunderscore  + \textunderscore ops\textunderscore )}
\end{itemize}
Gênero de crustáceos decápodes.
\section{Nefropexia}
\begin{itemize}
\item {fónica:csi}
\end{itemize}
\begin{itemize}
\item {Grp. gram.:f.}
\end{itemize}
\begin{itemize}
\item {Utilização:Med.}
\end{itemize}
\begin{itemize}
\item {Proveniência:(Do gr. \textunderscore nephros\textunderscore  + \textunderscore pexis\textunderscore )}
\end{itemize}
Fixação de um rim móvel.
\section{Nefropiose}
\begin{itemize}
\item {Grp. gram.:f.}
\end{itemize}
\begin{itemize}
\item {Utilização:Med.}
\end{itemize}
\begin{itemize}
\item {Proveniência:(Do gr. \textunderscore nephros\textunderscore  + \textunderscore puon\textunderscore )}
\end{itemize}
Supuração do rim.
\section{Nefroplegia}
\begin{itemize}
\item {Grp. gram.:f.}
\end{itemize}
\begin{itemize}
\item {Utilização:Med.}
\end{itemize}
\begin{itemize}
\item {Proveniência:(Do gr. \textunderscore nephros\textunderscore  + \textunderscore plessein\textunderscore )}
\end{itemize}
Paralisia dos rins.
\section{Nefroplégico}
\begin{itemize}
\item {Grp. gram.:adj.}
\end{itemize}
Relativo á nefroplegia.
\section{Nefropletora}
\begin{itemize}
\item {Grp. gram.:f.}
\end{itemize}
\begin{itemize}
\item {Utilização:Med.}
\end{itemize}
Pletora dos rins.
\section{Nefropletórico}
\begin{itemize}
\item {Grp. gram.:adj.}
\end{itemize}
Que procede da nefropletora.
\section{Nefroptose}
\begin{itemize}
\item {Grp. gram.:f.}
\end{itemize}
\begin{itemize}
\item {Utilização:Med.}
\end{itemize}
\begin{itemize}
\item {Proveniência:(Do gr. \textunderscore nephros\textunderscore  + \textunderscore ptosis\textunderscore )}
\end{itemize}
Prolapso ou deslocamento e mobilidade anormal do rim.
\section{Nefrorragia}
\begin{itemize}
\item {Grp. gram.:f.}
\end{itemize}
\begin{itemize}
\item {Utilização:Med.}
\end{itemize}
\begin{itemize}
\item {Proveniência:(Do gr. \textunderscore nephros\textunderscore  + \textunderscore rhagein\textunderscore )}
\end{itemize}
Derramamento de sangue proveniente dos rins.
\section{Nefrostomia}
\begin{itemize}
\item {Grp. gram.:f.}
\end{itemize}
\begin{itemize}
\item {Utilização:Med.}
\end{itemize}
\begin{itemize}
\item {Proveniência:(Do gr. \textunderscore nephros\textunderscore  + \textunderscore stoma\textunderscore )}
\end{itemize}
Estabelecimento de uma fístula renal.
\section{Nefrótoma}
\begin{itemize}
\item {Grp. gram.:f.}
\end{itemize}
\begin{itemize}
\item {Proveniência:(Do gr. \textunderscore nephros\textunderscore  + \textunderscore tome\textunderscore )}
\end{itemize}
Gênero de insectos dípteros.
\section{Nefrotomia}
\begin{itemize}
\item {Grp. gram.:f.}
\end{itemize}
\begin{itemize}
\item {Proveniência:(Do gr. \textunderscore nephros\textunderscore  + \textunderscore tome\textunderscore )}
\end{itemize}
Operação, que consiste em extrair cálculos urinários ou uma porção de pus, fazendo uma incisão nos rins, pela região lombar.
\section{Nefrotómico}
\begin{itemize}
\item {Grp. gram.:adj.}
\end{itemize}
Relativo á nefrotomia.
\section{Neftalita}
\begin{itemize}
\item {Grp. gram.:m.}
\end{itemize}
Israelita da tríbo de Nephtali.
\section{Néftea}
\begin{itemize}
\item {Grp. gram.:f.}
\end{itemize}
Gênero de pólipos.
\section{Nephelibata}
\begin{itemize}
\item {Grp. gram.:m.  e  adj.}
\end{itemize}
\begin{itemize}
\item {Utilização:Fig.}
\end{itemize}
\begin{itemize}
\item {Utilização:deprec.}
\end{itemize}
\begin{itemize}
\item {Utilização:Ext.}
\end{itemize}
O que anda ou vive nas nuvens.
Próprio de quem vive nas nuvens.
Literato excêntrico, que desconhece ou despreza os processos conhecidos e o bom senso literário.
Indivíduo que, dominado por um supposto ideal, não attende aos factos da vida positiva nem ás lições da experiência.
(Do gr.\textunderscore  nephele\textunderscore  + \textunderscore bates\textunderscore )
\section{Nephelibatice}
\begin{itemize}
\item {Grp. gram.:f.}
\end{itemize}
\begin{itemize}
\item {Utilização:Deprec.}
\end{itemize}
Opinião, acto ou dito, próprio de nephelibata.
\section{Nephelibático}
\begin{itemize}
\item {Grp. gram.:adj.}
\end{itemize}
Relativo a nephelibata.
\section{Nephelibatismo}
\begin{itemize}
\item {Grp. gram.:m.}
\end{itemize}
Qualidade ou systema de nephelibata.
\section{Nephelina}
\begin{itemize}
\item {Grp. gram.:f.}
\end{itemize}
\begin{itemize}
\item {Utilização:Miner.}
\end{itemize}
\begin{itemize}
\item {Proveniência:(Do gr. \textunderscore nephele\textunderscore )}
\end{itemize}
Substância, composta de sílica, alumina e soda, e fusível no maçarico.
\section{Nephelínico}
\begin{itemize}
\item {Grp. gram.:adj.}
\end{itemize}
\begin{itemize}
\item {Utilização:Miner.}
\end{itemize}
Que contém nephelina.
\section{Nephelinito}
\begin{itemize}
\item {Grp. gram.:m.}
\end{itemize}
\begin{itemize}
\item {Utilização:Miner.}
\end{itemize}
Rocha basáltica, em que predomina a nephelina.
\section{Nephélio}
\begin{itemize}
\item {Grp. gram.:m.}
\end{itemize}
\begin{itemize}
\item {Proveniência:(Lat. \textunderscore nephelion\textunderscore )}
\end{itemize}
Pequena mancha, na camada exterior da córnea, e que deixa passar a luz como através de uma nuvem.
\section{Nephelite}
\begin{itemize}
\item {Grp. gram.:f.}
\end{itemize}
\begin{itemize}
\item {Proveniência:(Do gr. \textunderscore nephele\textunderscore , nuvem)}
\end{itemize}
Espécie de mineral, vizinha do feldspatho, e que se encontra com abundância nas lavas do Vesúvio.
\section{Nephelitito}
\begin{itemize}
\item {Grp. gram.:m.}
\end{itemize}
\begin{itemize}
\item {Utilização:Geol.}
\end{itemize}
Espécie de tephrito, em que predomina a nephelite.
\section{Nephelóide}
\begin{itemize}
\item {Grp. gram.:adj.}
\end{itemize}
\begin{itemize}
\item {Proveniência:(Do gr. \textunderscore nephele\textunderscore  + \textunderscore eidos\textunderscore )}
\end{itemize}
Que tem o aspecto de nuvem.
\section{Néphodo}
\begin{itemize}
\item {Grp. gram.:m.}
\end{itemize}
Gênero de insectos coleópteros heterómeros.
\section{Nephralgia}
\begin{itemize}
\item {Grp. gram.:f.}
\end{itemize}
\begin{itemize}
\item {Utilização:Med.}
\end{itemize}
\begin{itemize}
\item {Proveniência:(Do gr. \textunderscore nephros\textunderscore  + \textunderscore algos\textunderscore )}
\end{itemize}
Dôr de rins, acompanhada de tremuras, calefrios, abundância de urina e, ás vezes, vómitos.
\section{Nephrálgico}
\begin{itemize}
\item {Grp. gram.:adj.}
\end{itemize}
Relativo a nephralgia.
\section{Nephrectomia}
\begin{itemize}
\item {Grp. gram.:f.}
\end{itemize}
\begin{itemize}
\item {Utilização:Cir.}
\end{itemize}
\begin{itemize}
\item {Proveniência:(Do gr. \textunderscore nephros\textunderscore  + \textunderscore ektome\textunderscore )}
\end{itemize}
Extirpação total dos rins.
\section{Nephrelita}
\begin{itemize}
\item {Grp. gram.:f.}
\end{itemize}
\begin{itemize}
\item {Utilização:Miner.}
\end{itemize}
Espécie de serpentina.
\section{Nephrelita}
\begin{itemize}
\item {Grp. gram.:f.}
\end{itemize}
\begin{itemize}
\item {Proveniência:(Do gr. \textunderscore nephros\textunderscore )}
\end{itemize}
Nome, que se dava á gordura que rodeia os rins.
\section{Nephrelmíntico}
\begin{itemize}
\item {Grp. gram.:adj.}
\end{itemize}
\begin{itemize}
\item {Utilização:Med.}
\end{itemize}
\begin{itemize}
\item {Proveniência:(Do gr. \textunderscore nephros\textunderscore  + \textunderscore helminthos\textunderscore )}
\end{itemize}
Que tem vermes nos rins.
\section{Nephremphraxe}
\begin{itemize}
\item {Grp. gram.:f.}
\end{itemize}
\begin{itemize}
\item {Utilização:Med.}
\end{itemize}
Inchação ou obstrucção dos rins.
\section{Nephrina}
\begin{itemize}
\item {Grp. gram.:f.}
\end{itemize}
\begin{itemize}
\item {Utilização:Med.}
\end{itemize}
\begin{itemize}
\item {Proveniência:(Do gr. \textunderscore nephros\textunderscore )}
\end{itemize}
O mesmo que \textunderscore ureia\textunderscore .
\section{Nephroge}
\begin{itemize}
\item {Grp. gram.:m.}
\end{itemize}
Gênero de arbustos trepadores, monospermos.
\section{Nephrographia}
\begin{itemize}
\item {Grp. gram.:f.}
\end{itemize}
\begin{itemize}
\item {Proveniência:(Do gr. \textunderscore nephros\textunderscore  + \textunderscore graphein\textunderscore )}
\end{itemize}
Descripção dos rins.
\section{Nephrográphico}
\begin{itemize}
\item {Grp. gram.:adj.}
\end{itemize}
Relativo á nephrographia.
\section{Nephrógrapho}
\begin{itemize}
\item {Grp. gram.:m.}
\end{itemize}
\begin{itemize}
\item {Proveniência:(Do gr. \textunderscore nephros\textunderscore  + \textunderscore graphein\textunderscore )}
\end{itemize}
Aquelle que escreve á cêrca da estructura, funcções e doenças dos rins.
\section{Nephróide}
\begin{itemize}
\item {Grp. gram.:adj.}
\end{itemize}
\begin{itemize}
\item {Proveniência:(Do gr. \textunderscore nephros\textunderscore  + \textunderscore eidos\textunderscore )}
\end{itemize}
Que tem fórma de rim.
\section{Nephrolithíase}
\begin{itemize}
\item {Grp. gram.:f.}
\end{itemize}
\begin{itemize}
\item {Proveniência:(De \textunderscore nephrólitho\textunderscore )}
\end{itemize}
Doença, causada pelos cálculos que se formam nos rins.
\section{Nephrolíthico}
\begin{itemize}
\item {Grp. gram.:adj.}
\end{itemize}
Relativo ao nephrólitho.
Dependente do nephrólitho.
\section{Nephrólitho}
\begin{itemize}
\item {Grp. gram.:m.}
\end{itemize}
\begin{itemize}
\item {Utilização:Med.}
\end{itemize}
\begin{itemize}
\item {Proveniência:(Do gr. \textunderscore nephros\textunderscore  + \textunderscore lithos\textunderscore )}
\end{itemize}
Pedra ou cálculo, que se fórma nos rins.
\section{Nephrolithotomia}
\begin{itemize}
\item {Grp. gram.:f.}
\end{itemize}
\begin{itemize}
\item {Proveniência:(Do gr. \textunderscore nephros\textunderscore  + \textunderscore lithos\textunderscore  + \textunderscore tome\textunderscore )}
\end{itemize}
Operação cirúrgica, com que se abre o rim, para extrahir alguma pedra ou cálculo.
\section{Nephrolithotómico}
\begin{itemize}
\item {Grp. gram.:adj.}
\end{itemize}
Relativo á nephrolithotomia.
\section{Nephrologia}
\begin{itemize}
\item {Grp. gram.:f.}
\end{itemize}
\begin{itemize}
\item {Proveniência:(Do gr. \textunderscore nephros\textunderscore  + \textunderscore logos\textunderscore )}
\end{itemize}
Tratado á cêrca dos rins.
\section{Nephrológico}
\begin{itemize}
\item {Grp. gram.:adj.}
\end{itemize}
Relativo á nephrologia.
\section{Nephrologista}
\begin{itemize}
\item {Grp. gram.:m.}
\end{itemize}
\begin{itemize}
\item {Proveniência:(De \textunderscore nephrologia\textunderscore )}
\end{itemize}
Aquelle que estuda especialmente as doenças dos rins.
\section{Nephrólogo}
\begin{itemize}
\item {Grp. gram.:m.}
\end{itemize}
\begin{itemize}
\item {Proveniência:(Do gr. \textunderscore nephros\textunderscore  + \textunderscore logos\textunderscore )}
\end{itemize}
Aquelle que escreveu sôbre nephrologia.
\section{Néphrope}
\begin{itemize}
\item {Grp. gram.:m.}
\end{itemize}
\begin{itemize}
\item {Proveniência:(Do gr. \textunderscore nephros\textunderscore  + \textunderscore ops\textunderscore )}
\end{itemize}
Gênero de crustáceos decápodes.
\section{Nephropexia}
\begin{itemize}
\item {fónica:csi}
\end{itemize}
\begin{itemize}
\item {Grp. gram.:f.}
\end{itemize}
\begin{itemize}
\item {Utilização:Med.}
\end{itemize}
\begin{itemize}
\item {Proveniência:(Do gr. \textunderscore nephros\textunderscore  + \textunderscore pexis\textunderscore )}
\end{itemize}
Fixação de um rim móvel.
\section{Nephrophlegmasia}
\begin{itemize}
\item {Grp. gram.:f.}
\end{itemize}
\begin{itemize}
\item {Utilização:Med.}
\end{itemize}
Inflammação dos rins.
\section{Nephrophlegmático}
\begin{itemize}
\item {Grp. gram.:adj.}
\end{itemize}
Procedente das mucosidades que se soltam dos rins.
\section{Nephroplegia}
\begin{itemize}
\item {Grp. gram.:f.}
\end{itemize}
\begin{itemize}
\item {Utilização:Med.}
\end{itemize}
\begin{itemize}
\item {Proveniência:(Do gr. \textunderscore nephros\textunderscore  + \textunderscore plessein\textunderscore )}
\end{itemize}
Paralysia dos rins.
\section{Nephroplégico}
\begin{itemize}
\item {Grp. gram.:adj.}
\end{itemize}
Relativo á nephroplegia.
\section{Nephroplethora}
\begin{itemize}
\item {Grp. gram.:f.}
\end{itemize}
\begin{itemize}
\item {Utilização:Med.}
\end{itemize}
Plethora dos rins.
\section{Nephroplethórico}
\begin{itemize}
\item {Grp. gram.:adj.}
\end{itemize}
Que procede da nephroplethora.
\section{Nephroptose}
\begin{itemize}
\item {Grp. gram.:f.}
\end{itemize}
\begin{itemize}
\item {Utilização:Med.}
\end{itemize}
\begin{itemize}
\item {Proveniência:(Do gr. \textunderscore nephros\textunderscore  + \textunderscore ptosis\textunderscore )}
\end{itemize}
Prolapso ou deslocamento e mobilidade anormal do rim.
\section{Nephropyose}
\begin{itemize}
\item {Grp. gram.:f.}
\end{itemize}
\begin{itemize}
\item {Utilização:Med.}
\end{itemize}
\begin{itemize}
\item {Proveniência:(Do gr. \textunderscore nephros\textunderscore  + \textunderscore puon\textunderscore )}
\end{itemize}
Suppuração do rim.
\section{Nephrorrhagia}
\begin{itemize}
\item {Grp. gram.:f.}
\end{itemize}
\begin{itemize}
\item {Utilização:Med.}
\end{itemize}
\begin{itemize}
\item {Proveniência:(Do gr. \textunderscore nephros\textunderscore  + \textunderscore rhagein\textunderscore )}
\end{itemize}
Derramamento de sangue proveniente dos rins.
\section{Nephrostomia}
\begin{itemize}
\item {Grp. gram.:f.}
\end{itemize}
\begin{itemize}
\item {Utilização:Med.}
\end{itemize}
\begin{itemize}
\item {Proveniência:(Do gr. \textunderscore nephros\textunderscore  + \textunderscore stoma\textunderscore )}
\end{itemize}
Estabelecimento de uma fístula renal.
\section{Nephrótoma}
\begin{itemize}
\item {Grp. gram.:f.}
\end{itemize}
\begin{itemize}
\item {Proveniência:(Do gr. \textunderscore nephros\textunderscore  + \textunderscore tome\textunderscore )}
\end{itemize}
Gênero de insectos dípteros.
\section{Nephrotomia}
\begin{itemize}
\item {Grp. gram.:f.}
\end{itemize}
\begin{itemize}
\item {Proveniência:(Do gr. \textunderscore nephros\textunderscore  + \textunderscore tome\textunderscore )}
\end{itemize}
Operação, que consiste em extrahir cálculos urinários ou uma porção de pus, fazendo uma incisão nos rins, pela região lombar.
\section{Nephrotómico}
\begin{itemize}
\item {Grp. gram.:adj.}
\end{itemize}
Relativo á nephrotomia.
\section{Nephtalita}
\begin{itemize}
\item {Grp. gram.:m.}
\end{itemize}
Israelita da tríbo de Nephtali.
\section{Néphtea}
\begin{itemize}
\item {Grp. gram.:f.}
\end{itemize}
Gênero de pólypos.
\section{Nepote}
\begin{itemize}
\item {Grp. gram.:m.}
\end{itemize}
\begin{itemize}
\item {Utilização:Ext.}
\end{itemize}
\begin{itemize}
\item {Proveniência:(Do lat. \textunderscore nepos\textunderscore , \textunderscore nepotis\textunderscore )}
\end{itemize}
Sobrinho do Papa.
Favorito, valido.
\section{Nepotismo}
\begin{itemize}
\item {Grp. gram.:m.}
\end{itemize}
\begin{itemize}
\item {Utilização:Ext.}
\end{itemize}
\begin{itemize}
\item {Proveniência:(De \textunderscore nepote\textunderscore )}
\end{itemize}
Autoridade excessiva, que os sobrinhos e outros parentes dos Papas exerceram na administração ecclesiástica.
Patronato, protecção excessiva, dada a parentes por alguém altamente collocado.
\section{Neptunaes}
\begin{itemize}
\item {Grp. gram.:f. pl.}
\end{itemize}
Festas, em honra de Neptuno, entre os antigos Romanos.
\section{Neptuniano}
\begin{itemize}
\item {Grp. gram.:adj.}
\end{itemize}
\begin{itemize}
\item {Utilização:Geol.}
\end{itemize}
\begin{itemize}
\item {Proveniência:(De \textunderscore Neptuno\textunderscore , n. p. myth.)}
\end{itemize}
Relativo ao Oceano.
Diz-se especialmente de terrenos, que devem a sua origem á água.
\section{Neptunino}
\begin{itemize}
\item {Grp. gram.:adj.}
\end{itemize}
\begin{itemize}
\item {Proveniência:(Lat. \textunderscore neptunius\textunderscore )}
\end{itemize}
O mesmo que \textunderscore neptuniano\textunderscore .
\section{Neptúnio}
\begin{itemize}
\item {Grp. gram.:adj.}
\end{itemize}
\begin{itemize}
\item {Proveniência:(Lat. \textunderscore neptunius\textunderscore )}
\end{itemize}
O mesmo que \textunderscore neptuniano\textunderscore .
\section{Neptunismo}
\begin{itemize}
\item {Grp. gram.:m.}
\end{itemize}
\begin{itemize}
\item {Proveniência:(De \textunderscore Neptuno\textunderscore , n. p.)}
\end{itemize}
Theoria, que attribue á água exclusivamente a formação das rochas que constituem a crusta do globo.
\section{Neptunista}
\begin{itemize}
\item {Grp. gram.:m. ,  f.  e  adj.}
\end{itemize}
\begin{itemize}
\item {Proveniência:(De \textunderscore Neptuno\textunderscore , n. p.)}
\end{itemize}
Pessôa partidária do neptunismo.
\section{Neptuno}
\begin{itemize}
\item {Grp. gram.:m.}
\end{itemize}
\begin{itemize}
\item {Utilização:Poét.}
\end{itemize}
Divindade, que, segundo a Mythologia romana, presidia ao mar.
Nome de um planeta, descoberto em 1846 por Galle.
O mar.
\section{Nequícia}
\begin{itemize}
\item {Grp. gram.:f.}
\end{itemize}
\begin{itemize}
\item {Proveniência:(Lat. \textunderscore nequitia\textunderscore )}
\end{itemize}
Maldade, perversidade. Cf. \textunderscore Lusíadas\textunderscore , VIII, 65.
\section{Nereida}
\begin{itemize}
\item {Grp. gram.:f.}
\end{itemize}
\begin{itemize}
\item {Proveniência:(Lat. \textunderscore nereis\textunderscore , \textunderscore nereidis\textunderscore )}
\end{itemize}
Cada uma das nymphas, que presidiam ao mar.
Gênero de anelídeos.
\section{Nereide}
\begin{itemize}
\item {Grp. gram.:f.}
\end{itemize}
O mesmo ou melhor que \textunderscore nereida\textunderscore .
\section{Nereilepas}
\begin{itemize}
\item {Grp. gram.:m.}
\end{itemize}
Animálculo parasita do eremita-bernardo.
(Palavra, mal formada, do gr. \textunderscore nereis\textunderscore  + \textunderscore lepas\textunderscore )
\section{Nereocrístia}
\begin{itemize}
\item {Grp. gram.:f.}
\end{itemize}
Planta submarina, filiforme e flexível.
\section{Nereocrýstia}
\begin{itemize}
\item {Grp. gram.:f.}
\end{itemize}
Planta submarina, filiforme e flexível.
\section{Néria}
\begin{itemize}
\item {Grp. gram.:f.}
\end{itemize}
Gênero de insectos lepidópteros.
\section{Neriáceas}
\begin{itemize}
\item {Grp. gram.:f. pl.}
\end{itemize}
\begin{itemize}
\item {Proveniência:(De \textunderscore neriáceo\textunderscore )}
\end{itemize}
Família de plantas, formada á custa das apocýneas e que tem por typo o loendro.
\section{Neriáceo}
\begin{itemize}
\item {Grp. gram.:adj.}
\end{itemize}
\begin{itemize}
\item {Proveniência:(De \textunderscore nério\textunderscore )}
\end{itemize}
Relativo ou semelhante ao loendro.
\section{Nerinda}
\begin{itemize}
\item {Grp. gram.:f.}
\end{itemize}
Espécie de tecido branco de algodão, fabricado na Índia.
\section{Nério}
\begin{itemize}
\item {Grp. gram.:m.}
\end{itemize}
\begin{itemize}
\item {Proveniência:(Lat. \textunderscore nerium\textunderscore )}
\end{itemize}
Designação scientífica do loendro.
\section{Nerita}
\begin{itemize}
\item {Grp. gram.:f.}
\end{itemize}
\begin{itemize}
\item {Proveniência:(Do gr. \textunderscore neros\textunderscore )}
\end{itemize}
Gênero de molluscos gasterópodes.
\section{Neritáceos}
\begin{itemize}
\item {Grp. gram.:m. pl.}
\end{itemize}
\begin{itemize}
\item {Proveniência:(De \textunderscore nerita\textunderscore )}
\end{itemize}
Família de molluscos gasterópodes.
\section{Nerócila}
\begin{itemize}
\item {Grp. gram.:f.}
\end{itemize}
Gênero de crustáceos isópodes.
\section{Néroli}
\begin{itemize}
\item {Grp. gram.:m.}
\end{itemize}
\begin{itemize}
\item {Proveniência:(De \textunderscore Neroli\textunderscore , n. p.)}
\end{itemize}
Óleo, extrahido da flôr de laranjeira.
\section{Nerónico}
\begin{itemize}
\item {Grp. gram.:adj.}
\end{itemize}
\begin{itemize}
\item {Proveniência:(Do lat. \textunderscore Nero\textunderscore , \textunderscore Neronis\textunderscore , n. p.)}
\end{itemize}
Relativo a Nero; próprio de Nero:«\textunderscore perseguição nerónica\textunderscore ». Deusdado, \textunderscore Escorços\textunderscore , 95.
\section{Nertera}
\begin{itemize}
\item {Grp. gram.:f.}
\end{itemize}
Gênero de plantas rubiáceas.
\section{Nerto}
\begin{itemize}
\item {Grp. gram.:m.}
\end{itemize}
Gênero de insectos coleópteros tetrâmeros.
\section{Nervação}
\begin{itemize}
\item {Grp. gram.:f.}
\end{itemize}
\begin{itemize}
\item {Utilização:Bot.}
\end{itemize}
\begin{itemize}
\item {Proveniência:(De \textunderscore nervo\textunderscore )}
\end{itemize}
Distribuição das nervuras nos vegetaes.
\section{Nervado}
\begin{itemize}
\item {Grp. gram.:adj.}
\end{itemize}
\begin{itemize}
\item {Proveniência:(De \textunderscore nervo\textunderscore )}
\end{itemize}
Que tem nervuras.
Feito de tiras de coiro.
\section{Nerval}
\begin{itemize}
\item {Grp. gram.:adj.}
\end{itemize}
\begin{itemize}
\item {Proveniência:(Lat. \textunderscore nervalis\textunderscore )}
\end{itemize}
Relativo a nervo.
Próprio dos nervos.
Nervoso.
\section{Nérveo}
\begin{itemize}
\item {Grp. gram.:adj.}
\end{itemize}
\begin{itemize}
\item {Proveniência:(De \textunderscore nervo\textunderscore )}
\end{itemize}
O mesmo que \textunderscore nerval\textunderscore ; nervino.
\section{Nervino}
\begin{itemize}
\item {Grp. gram.:adj.}
\end{itemize}
\begin{itemize}
\item {Grp. gram.:M.}
\end{itemize}
\begin{itemize}
\item {Proveniência:(Lat. \textunderscore nervinus\textunderscore )}
\end{itemize}
Relativo a nervos.
Que influe sôbre os nervos.
Medicamento, que tem acção sôbre os nervos.
\section{Nervo}
\begin{itemize}
\item {fónica:nêr}
\end{itemize}
\begin{itemize}
\item {Grp. gram.:m.}
\end{itemize}
\begin{itemize}
\item {Utilização:Pop.}
\end{itemize}
\begin{itemize}
\item {Utilização:Fig.}
\end{itemize}
\begin{itemize}
\item {Proveniência:(Lat. \textunderscore nervus\textunderscore )}
\end{itemize}
Cada um dos pequenos filamentos, que põem em communicação o cérebro e a medulla espinal com a circunferência do corpo, e que transmittem as sensações ao centro e as vontades á circunferência.
Órgão de sensação e movimento nos animaes.
Tendão.
Energia; robustez.
Cada um dos veios das fôlhas dos vegetaes.
Nome de vários ornatos ou molduras, em Architectura.
\section{Nervosado}
\begin{itemize}
\item {Grp. gram.:m.  e  adj.}
\end{itemize}
\begin{itemize}
\item {Utilização:Neol.}
\end{itemize}
Indivíduo, que soffre dos nervos.
\section{Nervosamente}
\begin{itemize}
\item {Grp. gram.:adv.}
\end{itemize}
De modo nervoso.
Por influência dos nervos.
\section{Nervosidade}
\begin{itemize}
\item {Grp. gram.:f.}
\end{itemize}
\begin{itemize}
\item {Proveniência:(Lat. \textunderscore nervositas\textunderscore )}
\end{itemize}
Qualidade do que é nervoso.
Conjunto ou fôrça de nervos; nervosismo.
\section{Nervosismo}
\begin{itemize}
\item {Grp. gram.:m.}
\end{itemize}
\begin{itemize}
\item {Proveniência:(De \textunderscore nervoso\textunderscore )}
\end{itemize}
Systema, em que todos os phenómenos mórbidos são attribuidos ás aberrações da fôrça nervosa.
Doença, caracterizada por perturbações do systema nervoso.
\section{Nervoso}
\begin{itemize}
\item {Grp. gram.:adj.}
\end{itemize}
\begin{itemize}
\item {Grp. gram.:M.}
\end{itemize}
\begin{itemize}
\item {Proveniência:(Lat. \textunderscore nervosus\textunderscore )}
\end{itemize}
Relativo a nervos.
Que tem nervos.
Que procede dos nervos ou tem a sua séde nos nervos.
Enérgico; vigoroso.
Excitado, exaltado.
Que padece dos nervos.
Que tem nervuras salientes.
Doença dos nervos.
Neuropathia; hysterismo.
\section{Nervudo}
\begin{itemize}
\item {Grp. gram.:adj.}
\end{itemize}
\begin{itemize}
\item {Utilização:Fig.}
\end{itemize}
Que tem nervos fortes.
Musculoso, forte.
\section{Nérvulo}
\begin{itemize}
\item {Grp. gram.:m.}
\end{itemize}
\begin{itemize}
\item {Utilização:Bot.}
\end{itemize}
\begin{itemize}
\item {Proveniência:(Lat. \textunderscore nervulus\textunderscore )}
\end{itemize}
Fascículo de fibras nas placentas, constituíndo o tecido vascular.
\section{Nervura}
\begin{itemize}
\item {Grp. gram.:f.}
\end{itemize}
\begin{itemize}
\item {Proveniência:(De \textunderscore nervo\textunderscore )}
\end{itemize}
Fibra saliente na superfície das fôlhas e das pétalas.
Linha ou moldura saliente, que separa os panos de uma abóbada.
Moldura redonda, sôbre o contôrno das mísulas.
\section{Nesciamente}
\begin{itemize}
\item {Grp. gram.:adv.}
\end{itemize}
De modo néscio; estupidamente.
\section{Nescidade}
\begin{itemize}
\item {Grp. gram.:f.}
\end{itemize}
\begin{itemize}
\item {Proveniência:(De \textunderscore néscio\textunderscore )}
\end{itemize}
Fórma erudita de \textunderscore necedade\textunderscore .
\section{Nescídia}
\begin{itemize}
\item {Grp. gram.:f.}
\end{itemize}
Gênero de plantas rubiáceas.
\section{Néscio}
\begin{itemize}
\item {Grp. gram.:adj.}
\end{itemize}
\begin{itemize}
\item {Grp. gram.:M.}
\end{itemize}
\begin{itemize}
\item {Proveniência:(Lat. \textunderscore nescius\textunderscore )}
\end{itemize}
Que não sabe; ignorante.
Inepto.
Estúpido.
Indivíduo inepto.
\section{Nese-nese}
\begin{itemize}
\item {Grp. gram.:m.}
\end{itemize}
\begin{itemize}
\item {Utilização:Prov.}
\end{itemize}
\begin{itemize}
\item {Utilização:trasm.}
\end{itemize}
Bocadinho insignificante; um quási nada.
(Relaciona-se com \textunderscore és-não-és\textunderscore ?)
\section{Nesga}
\begin{itemize}
\item {Grp. gram.:f.}
\end{itemize}
\begin{itemize}
\item {Utilização:Fig.}
\end{itemize}
Peça triângular de pano, cosida entre os quartos de uma peça de vestuário.
Pequeno espaço de terreno, entre terrenos extensos.
Retalho; pequena porção de uma coisa: \textunderscore uma nesga de sol\textunderscore .
\section{Néslia}
\begin{itemize}
\item {Grp. gram.:f.}
\end{itemize}
Gênero de plantas crucíferas.
\section{Nêspara}
\begin{itemize}
\item {Grp. gram.:f.}
\end{itemize}
\begin{itemize}
\item {Utilização:Ant.}
\end{itemize}
O mesmo que \textunderscore nêspera\textunderscore . Cf. \textunderscore Eufrosina\textunderscore , 169.
\section{Nespéra}
\begin{itemize}
\item {Grp. gram.:f.}
\end{itemize}
Árvore esterculiácea de San-Thomé.
\section{Nêspera}
\begin{itemize}
\item {Grp. gram.:f.}
\end{itemize}
\begin{itemize}
\item {Proveniência:(Do lat. \textunderscore mespilum\textunderscore )}
\end{itemize}
Fruto de nespereira, um tanto ácido, e com vários caroços.
\section{Nespereira}
\begin{itemize}
\item {Grp. gram.:f.}
\end{itemize}
\begin{itemize}
\item {Proveniência:(De \textunderscore nêspera\textunderscore )}
\end{itemize}
Árvore fructífera, da fam. das rosáceas, (\textunderscore nespilus germanica\textunderscore , Lin.).
\section{Nessa}
Expressão contrahida, equivalente a \textunderscore em essa\textunderscore .
(Cp. \textunderscore no\textunderscore ^1)
\section{Nessa}
\begin{itemize}
\item {Grp. gram.:f.}
\end{itemize}
Mês, que os antigos árabes intercalavam, de três em três annos, para que o anno lunar correspondesse ao solar.
\section{Nesse}
\begin{itemize}
\item {fónica:nê}
\end{itemize}
Expressão contrahida, equivalente a \textunderscore em esse\textunderscore .
(Cp. \textunderscore no\textunderscore ^1)
\section{Nessora}
\begin{itemize}
\item {Grp. gram.:adv.}
\end{itemize}
\begin{itemize}
\item {Utilização:Ant.}
\end{itemize}
\begin{itemize}
\item {Proveniência:(De \textunderscore nessa\textunderscore  + \textunderscore hora\textunderscore )}
\end{itemize}
O mesmo que \textunderscore logo\textunderscore . Cf. G. Vicente.
\section{Nesta}
Expressão contrahida, equivalente a \textunderscore em esta\textunderscore .
(Cp. \textunderscore no\textunderscore ^1)
\section{Neste}
\begin{itemize}
\item {fónica:nês}
\end{itemize}
Expressão contrahida, equivalente a \textunderscore em este\textunderscore .
(Cp. \textunderscore no\textunderscore ^1)
\section{Néstis}
\begin{itemize}
\item {Grp. gram.:m.}
\end{itemize}
Gênero de peixes acanthopterýgios.
\section{Nestlera}
\begin{itemize}
\item {Grp. gram.:f.}
\end{itemize}
\begin{itemize}
\item {Proveniência:(De \textunderscore Nestler\textunderscore , n. p.)}
\end{itemize}
Gênero de plantas, da fam. das compostas.
\section{Nestor}
\begin{itemize}
\item {Grp. gram.:m.}
\end{itemize}
\begin{itemize}
\item {Utilização:Ext.}
\end{itemize}
\begin{itemize}
\item {Utilização:Bras}
\end{itemize}
\begin{itemize}
\item {Proveniência:(De \textunderscore Nestor\textunderscore , n. p.)}
\end{itemize}
Nome de um velho e eloquente herói grego, que assistiu á guerra de Tróia.
Homem velho e prudente.
Insecto, de asas azues.
\section{Nestorianismo}
\begin{itemize}
\item {Grp. gram.:m.}
\end{itemize}
\begin{itemize}
\item {Proveniência:(De \textunderscore Nestório\textunderscore , n. p.)}
\end{itemize}
Seita religiosa dos que, no século V, sustentavam a separação entre as naturezas divina e humana de Christo.
\section{Nestoriano}
\begin{itemize}
\item {Grp. gram.:adj.}
\end{itemize}
\begin{itemize}
\item {Grp. gram.:M.}
\end{itemize}
Relativo ao Nestorianismo.
Sectário do nestorianismo.
\section{Neta}
\begin{itemize}
\item {Grp. gram.:f.}
\end{itemize}
\begin{itemize}
\item {Proveniência:(Do b. lat. \textunderscore nepta\textunderscore )}
\end{itemize}
Indivíduo do sexo feminino, em relação aos pais de seus pais.
\section{Neta}
\begin{itemize}
\item {Grp. gram.:f.}
\end{itemize}
\begin{itemize}
\item {Utilização:Bras}
\end{itemize}
\begin{itemize}
\item {Proveniência:(Do ingl. \textunderscore net\textunderscore )}
\end{itemize}
Nome, que os pescadores da Nazaré dão aos apparelhos medianos de arrastar para terra.
A escuma mais fina, que deita o melado, quando ferve, nos engenhos de açúcar.
\section{Nêta}
\begin{itemize}
\item {Grp. gram.:f.}
\end{itemize}
\begin{itemize}
\item {Utilização:Bras}
\end{itemize}
\begin{itemize}
\item {Proveniência:(Do ingl. \textunderscore net\textunderscore )}
\end{itemize}
Nome, que os pescadores da Nazaré dão aos apparelhos medianos de arrastar para terra.
A escuma mais fina, que deita o melado, quando ferve, nos engenhos de açúcar.
\section{Nêta}
\begin{itemize}
\item {Grp. gram.:pron.}
\end{itemize}
\begin{itemize}
\item {Utilização:Açor}
\end{itemize}
Nada. (Colhido em San-Jorge)
\section{Netamente}
\begin{itemize}
\item {Grp. gram.:adv.}
\end{itemize}
\begin{itemize}
\item {Utilização:Des.}
\end{itemize}
Limpamente; de modo \textunderscore neto\textunderscore ^2.
\section{Nêtas}
\begin{itemize}
\item {Grp. gram.:adv.}
\end{itemize}
\begin{itemize}
\item {Utilização:Açor}
\end{itemize}
De maneira nenhuma.
(Cp. \textunderscore nêta\textunderscore )
\section{Netinha}
\begin{itemize}
\item {Grp. gram.:f.}
\end{itemize}
Nome que os pescadores da Nazaré dão a um apparelho de arrastar, melhor que o chamado \textunderscore neta\textunderscore .
\section{Netinho}
\begin{itemize}
\item {Grp. gram.:m.}
\end{itemize}
\begin{itemize}
\item {Utilização:Açor}
\end{itemize}
Bolo de milho.
\section{Neto}
\begin{itemize}
\item {Grp. gram.:m.}
\end{itemize}
\begin{itemize}
\item {Utilização:Gír.}
\end{itemize}
\begin{itemize}
\item {Utilização:Prov.}
\end{itemize}
\begin{itemize}
\item {Utilização:minh.}
\end{itemize}
\begin{itemize}
\item {Utilização:T. de Villa Real}
\end{itemize}
\begin{itemize}
\item {Grp. gram.:Pl.}
\end{itemize}
\begin{itemize}
\item {Proveniência:(Do b. lat. \textunderscore neptus\textunderscore )}
\end{itemize}
Designação do indivíduo do sexo masculino, em relação aos pais de seus pais. Filho do filho ou da filha, em relação ao avô ou á avó.
Cavalleiro que, nas toiradas, anda transmittindo ordens.
Copo. Cf. P. Ivo, \textunderscore Contos\textunderscore , 8.
Grêlo de couve.
Cada um dos grelos, que formam a segunda rodada na couve.
Descendentes; vindoiros.
Posteridade.
\section{Neto}
\begin{itemize}
\item {Grp. gram.:adj.}
\end{itemize}
\begin{itemize}
\item {Utilização:Des.}
\end{itemize}
\begin{itemize}
\item {Proveniência:(Fr. \textunderscore net\textunderscore )}
\end{itemize}
Nítido; limpo.
Liso Cf. Bernárdez, \textunderscore Luz e Calor\textunderscore , 244; G. Vicente, I, 349.
\section{Neuma}
\begin{itemize}
\item {Grp. gram.:f.}
\end{itemize}
\begin{itemize}
\item {Utilização:Mús.}
\end{itemize}
\begin{itemize}
\item {Utilização:ant.}
\end{itemize}
\begin{itemize}
\item {Grp. gram.:M. pl.}
\end{itemize}
\begin{itemize}
\item {Proveniência:(Gr. \textunderscore pneuma\textunderscore )}
\end{itemize}
Curta melodia, que, se vocaliza sem palavras ou sôbre a última sýllaba da última palavra.
Gesto de assentimento ou recusa.
Série de notas, entoadas como uma só sýllaba; grupo de duas ou mais notas.
Sinaes, com que na Idade-Média se escrevia o cantochão.
\section{Neumado}
\begin{itemize}
\item {Grp. gram.:adj.}
\end{itemize}
\begin{itemize}
\item {Grp. gram.:M.}
\end{itemize}
\begin{itemize}
\item {Proveniência:(De \textunderscore neuma\textunderscore )}
\end{itemize}
Cuja notação é neumática, (fallando-se de missaes).
Antiphonário manuscrito.
\section{Neumático}
\begin{itemize}
\item {Grp. gram.:adv.}
\end{itemize}
Relativo á neuma.
\section{Neumógrafo}
\begin{itemize}
\item {Grp. gram.:m.}
\end{itemize}
\begin{itemize}
\item {Proveniência:(Do gr. \textunderscore pneuma\textunderscore  + \textunderscore graphein\textunderscore )}
\end{itemize}
Aparelho clínico, para marcar as inspirações humanas.
\section{Neumógrapho}
\begin{itemize}
\item {Grp. gram.:m.}
\end{itemize}
\begin{itemize}
\item {Proveniência:(Do gr. \textunderscore pneuma\textunderscore  + \textunderscore graphein\textunderscore )}
\end{itemize}
Apparelho clínico, para marcar as inspirações humanas.
\section{Neuracantho}
\begin{itemize}
\item {Grp. gram.:m.}
\end{itemize}
Gênero de plantas acantháceas.
\section{Neuracanto}
\begin{itemize}
\item {Grp. gram.:m.}
\end{itemize}
Gênero de plantas acantáceas.
\section{Neurachne}
\begin{itemize}
\item {Grp. gram.:f.}
\end{itemize}
\begin{itemize}
\item {Proveniência:(Do gr. \textunderscore neuron\textunderscore  + \textunderscore akhne\textunderscore )}
\end{itemize}
Gênero de plantas gramíneas.
\section{Neuracne}
\begin{itemize}
\item {Grp. gram.:f.}
\end{itemize}
\begin{itemize}
\item {Proveniência:(Do gr. \textunderscore neuron\textunderscore  + \textunderscore akhne\textunderscore )}
\end{itemize}
Gênero de plantas gramíneas.
\section{Neurada}
\begin{itemize}
\item {Grp. gram.:f.}
\end{itemize}
\begin{itemize}
\item {Proveniência:(Do gr. \textunderscore neuron\textunderscore  + \textunderscore aden\textunderscore )}
\end{itemize}
Gênero de plantas rosáceas.
\section{Neurádeas}
\begin{itemize}
\item {Grp. gram.:f. pl.}
\end{itemize}
\begin{itemize}
\item {Proveniência:(De \textunderscore neurada\textunderscore )}
\end{itemize}
Tríbo de plantas, estabelecida por De-Candolle, na fam. das rosáceas.
\section{Neuragmia}
\begin{itemize}
\item {Grp. gram.:f.}
\end{itemize}
\begin{itemize}
\item {Utilização:Med.}
\end{itemize}
\begin{itemize}
\item {Proveniência:(Do gr. \textunderscore neuron\textunderscore  + \textunderscore agmos\textunderscore )}
\end{itemize}
Ruptura ou secção de um cordão nervoso.
\section{Neurágmico}
\begin{itemize}
\item {Grp. gram.:adj.}
\end{itemize}
Relativo á neuragmia.
\section{Neural}
\begin{itemize}
\item {Grp. gram.:adj.}
\end{itemize}
\begin{itemize}
\item {Proveniência:(Do gr. \textunderscore neuron\textunderscore )}
\end{itemize}
Relativo a nervos.
Próprio dos nervos; nerval.
\section{Neuralgia}
\begin{itemize}
\item {Grp. gram.:f.}
\end{itemize}
\begin{itemize}
\item {Proveniência:(Do gr. \textunderscore neuron\textunderscore  + \textunderscore algos\textunderscore )}
\end{itemize}
Doença, caracterizada por dôr viva no trajecto de um nervo e das suas ramificações, sem alteração apparente da parte dolorida.
\section{Neurálgico}
\begin{itemize}
\item {Grp. gram.:adj.}
\end{itemize}
Relativo ou semelhante á neuralgia.
\section{Neurarterial}
\begin{itemize}
\item {Grp. gram.:adj.}
\end{itemize}
\begin{itemize}
\item {Proveniência:(De \textunderscore neuro...\textunderscore  + \textunderscore arterial\textunderscore )}
\end{itemize}
Relativo aos nervos e ás artérias.
\section{Neurastenia}
\begin{itemize}
\item {Grp. gram.:f.}
\end{itemize}
\begin{itemize}
\item {Proveniência:(Do gr. \textunderscore neuron\textunderscore  + \textunderscore a\textunderscore  priv. + \textunderscore stenos\textunderscore )}
\end{itemize}
Fraqueza ou debilidade nervosa.
Esgotamento nervoso. Cf. Sousa Martins, \textunderscore Nosographia\textunderscore .
\section{Neurastênico}
\begin{itemize}
\item {Grp. gram.:adj.}
\end{itemize}
\begin{itemize}
\item {Grp. gram.:M.  e  adj.}
\end{itemize}
Relativo á neurastenia.
Aquele que padece neurastenía.
\section{Neurasthenia}
\begin{itemize}
\item {Grp. gram.:f.}
\end{itemize}
\begin{itemize}
\item {Proveniência:(Do gr. \textunderscore neuron\textunderscore  + \textunderscore a\textunderscore  priv. + \textunderscore stenos\textunderscore )}
\end{itemize}
Fraqueza ou debilidade nervosa.
Esgotamento nervoso. Cf. Sousa Martins, \textunderscore Nosographia\textunderscore .
\section{Neurasthênico}
\begin{itemize}
\item {Grp. gram.:adj.}
\end{itemize}
\begin{itemize}
\item {Grp. gram.:M.  e  adj.}
\end{itemize}
Relativo á neurasthenia.
Aquelle que padece neurasthenía.
\section{Neuraxe}
\begin{itemize}
\item {fónica:cse}
\end{itemize}
\begin{itemize}
\item {Grp. gram.:f.}
\end{itemize}
\begin{itemize}
\item {Utilização:Anat.}
\end{itemize}
\begin{itemize}
\item {Proveniência:(Do gr. \textunderscore neuron\textunderscore  + \textunderscore axis\textunderscore )}
\end{itemize}
Conjunto dos centros nervosos, (encéphalo e medulla).
\section{Neurectomia}
\begin{itemize}
\item {Grp. gram.:f.}
\end{itemize}
\begin{itemize}
\item {Proveniência:(Do gr. \textunderscore neuron\textunderscore  + \textunderscore ex\textunderscore  + \textunderscore tome\textunderscore )}
\end{itemize}
Extracção de uma parte de um nervo.
\section{Nêuria}
\begin{itemize}
\item {Grp. gram.:f.}
\end{itemize}
Gênero de insectos lepidópteros nocturnos.
\section{Nêurico}
\begin{itemize}
\item {Grp. gram.:adj.}
\end{itemize}
\begin{itemize}
\item {Proveniência:(Do gr. \textunderscore neuron\textunderscore )}
\end{itemize}
Relativo a nervos ou ao systema nervoso. Cf. Sousa Martins, \textunderscore Nosographia\textunderscore .
\section{Neurilema}
\begin{itemize}
\item {Grp. gram.:m.}
\end{itemize}
\begin{itemize}
\item {Proveniência:(Do gr. \textunderscore neuron\textunderscore  + \textunderscore eilema\textunderscore )}
\end{itemize}
Tecido laminar e pouco resistente, que envolve os nervos.
\section{Neurilemite}
\begin{itemize}
\item {Grp. gram.:f.}
\end{itemize}
Inflammação do neurilema.
\section{Neurilidade}
\begin{itemize}
\item {Grp. gram.:f.}
\end{itemize}
\begin{itemize}
\item {Proveniência:(Do gr. \textunderscore neuron\textunderscore )}
\end{itemize}
Propriedade, inherente ás fibras nervosas, de transmittir as impressões e a vontade:«\textunderscore tal é a neurilidade... de um nerasthênico bem definido...\textunderscore »Sousa Martins, \textunderscore Nosographia\textunderscore .
\section{Neurina}
\begin{itemize}
\item {Grp. gram.:f.}
\end{itemize}
\begin{itemize}
\item {Utilização:Chím.}
\end{itemize}
\begin{itemize}
\item {Proveniência:(Do gr. \textunderscore neuron\textunderscore )}
\end{itemize}
Base, resultante do desdobramento do protagão, sob a acção da água de baryta concentrada.
\section{Neurisma}
\begin{itemize}
\item {Grp. gram.:m.}
\end{itemize}
(V.aneurisma)
\section{Neurina}
\begin{itemize}
\item {Grp. gram.:f.}
\end{itemize}
\begin{itemize}
\item {Utilização:Chím.}
\end{itemize}
\begin{itemize}
\item {Proveniência:(Do gr. \textunderscore neuron\textunderscore )}
\end{itemize}
Base ammoniacal, tirada do cérebro por Liebreich.
\section{Neurino}
\begin{itemize}
\item {Grp. gram.:adj.}
\end{itemize}
O mesmo que \textunderscore nervino\textunderscore .
\section{Neurite}
\begin{itemize}
\item {Grp. gram.:f.}
\end{itemize}
\begin{itemize}
\item {Proveniência:(Do gr. \textunderscore neuron\textunderscore )}
\end{itemize}
Inflammação de um nervo.
\section{Neurítica}
\begin{itemize}
\item {Grp. gram.:f.}
\end{itemize}
\begin{itemize}
\item {Utilização:Neol.}
\end{itemize}
\begin{itemize}
\item {Proveniência:(Do gr. \textunderscore neuron\textunderscore )}
\end{itemize}
Therapêutica dos nervos.
\section{Neurítico}
\begin{itemize}
\item {Grp. gram.:adj.}
\end{itemize}
O mesmo que \textunderscore nervino\textunderscore .
Próprio para curar doenças nervosas.
\section{Neuro...}
\begin{itemize}
\item {Grp. gram.:pref.}
\end{itemize}
\begin{itemize}
\item {Proveniência:(Gr. \textunderscore neuron\textunderscore )}
\end{itemize}
(designativo de nervo ou relativo a nervos)
\section{Neurobalística}
\begin{itemize}
\item {Grp. gram.:f.}
\end{itemize}
\begin{itemize}
\item {Proveniência:(De \textunderscore neuro...\textunderscore  + \textunderscore balistica\textunderscore )}
\end{itemize}
Designação genérica das antigas máquinas de guerra, em que a fôrça se transmittia por cordas, á falta da pólvora, que ainda se não tinha inventado.
\section{Neuróbata}
\begin{itemize}
\item {Grp. gram.:m.}
\end{itemize}
\begin{itemize}
\item {Utilização:Des.}
\end{itemize}
\begin{itemize}
\item {Proveniência:(Lat. \textunderscore neurobata\textunderscore )}
\end{itemize}
O mesmo que \textunderscore acrobata\textunderscore .
\section{Neuroblasto}
\begin{itemize}
\item {Grp. gram.:m.}
\end{itemize}
\begin{itemize}
\item {Utilização:Anat.}
\end{itemize}
\begin{itemize}
\item {Proveniência:(Do gr. \textunderscore neuron\textunderscore  + \textunderscore blastos\textunderscore )}
\end{itemize}
Céllula nervosa embryonária.
\section{Neurocarpo}
\begin{itemize}
\item {Grp. gram.:m.}
\end{itemize}
\begin{itemize}
\item {Proveniência:(Do gr. \textunderscore neuron\textunderscore  + \textunderscore karpos\textunderscore )}
\end{itemize}
Gênero de plantas leguminosas.
\section{Neurodermia}
\begin{itemize}
\item {Grp. gram.:f.}
\end{itemize}
\begin{itemize}
\item {Utilização:Med.}
\end{itemize}
\begin{itemize}
\item {Proveniência:(Do gr. \textunderscore neuron\textunderscore  + \textunderscore derma\textunderscore )}
\end{itemize}
Neurose cutânea, com prurido intenso.
\section{Neurogamia}
\begin{itemize}
\item {Grp. gram.:f.}
\end{itemize}
\begin{itemize}
\item {Proveniência:(Do gr. \textunderscore neuron\textunderscore  + \textunderscore gamos\textunderscore )}
\end{itemize}
O mesmo que \textunderscore magnetismo\textunderscore  animal.
\section{Neurogenia}
\begin{itemize}
\item {Grp. gram.:f.}
\end{itemize}
\begin{itemize}
\item {Proveniência:(Do gr. \textunderscore neuron\textunderscore  + \textunderscore genea\textunderscore )}
\end{itemize}
Parte da Anatomia, que trata da formação dos nervos.
\section{Neurografia}
\begin{itemize}
\item {Grp. gram.:f.}
\end{itemize}
\begin{itemize}
\item {Proveniência:(De \textunderscore neurógrafo\textunderscore )}
\end{itemize}
Descripção dos nervos.
\section{Neurógrafo}
\begin{itemize}
\item {Grp. gram.:m.}
\end{itemize}
\begin{itemize}
\item {Proveniência:(Do gr. \textunderscore neuron\textunderscore  + \textunderscore graphein\textunderscore )}
\end{itemize}
Aquele que se ocupa de neurografia.
\section{Neurographia}
\begin{itemize}
\item {Grp. gram.:f.}
\end{itemize}
\begin{itemize}
\item {Proveniência:(De \textunderscore neurógrapho\textunderscore )}
\end{itemize}
Descripção dos nervos.
\section{Neurógrapho}
\begin{itemize}
\item {Grp. gram.:m.}
\end{itemize}
\begin{itemize}
\item {Proveniência:(Do gr. \textunderscore neuron\textunderscore  + \textunderscore graphein\textunderscore )}
\end{itemize}
Aquelle que se occupa de neurographia.
\section{Neurólitho}
\begin{itemize}
\item {Grp. gram.:m.}
\end{itemize}
\begin{itemize}
\item {Utilização:Miner.}
\end{itemize}
\begin{itemize}
\item {Proveniência:(Do gr. \textunderscore neuron\textunderscore  + \textunderscore lithos\textunderscore )}
\end{itemize}
Variedade de opala.
\section{Neurólito}
\begin{itemize}
\item {Grp. gram.:m.}
\end{itemize}
\begin{itemize}
\item {Utilização:Miner.}
\end{itemize}
\begin{itemize}
\item {Proveniência:(Do gr. \textunderscore neuron\textunderscore  + \textunderscore lithos\textunderscore )}
\end{itemize}
Variedade de opala.
\section{Neurologia}
\begin{itemize}
\item {Grp. gram.:f.}
\end{itemize}
\begin{itemize}
\item {Proveniência:(Do gr. \textunderscore neuron\textunderscore  + \textunderscore logos\textunderscore )}
\end{itemize}
Parte da Anatomia, que se occupa do systema nervoso.
\section{Neurológico}
\begin{itemize}
\item {Grp. gram.:adj.}
\end{itemize}
Relativo á neurologia.
\section{Neurologista}
\begin{itemize}
\item {Grp. gram.:m.}
\end{itemize}
Aquelle que se occupa de neurologia.
\section{Neurólogo}
\begin{itemize}
\item {Grp. gram.:m.}
\end{itemize}
O mesmo que \textunderscore neurologista\textunderscore .
\section{Neuroma}
\begin{itemize}
\item {Grp. gram.:m.}
\end{itemize}
\begin{itemize}
\item {Proveniência:(Do gr. \textunderscore neuron\textunderscore )}
\end{itemize}
Tumor sub-cutâneo e muito doloroso, que se desenvolve na espessura do tecido dos nervos, entre os filetes que o constituem.
\section{Neuromério}
\begin{itemize}
\item {Grp. gram.:m.}
\end{itemize}
\begin{itemize}
\item {Proveniência:(Do gr. \textunderscore neuron\textunderscore  + \textunderscore meros\textunderscore )}
\end{itemize}
Parte nervosa do metâmero.
\section{Neurómero}
\begin{itemize}
\item {Grp. gram.:m.}
\end{itemize}
\begin{itemize}
\item {Proveniência:(Do gr. \textunderscore neuron\textunderscore  + \textunderscore meros\textunderscore )}
\end{itemize}
Parte nervosa do metâmero.
\section{Neuromo}
\begin{itemize}
\item {Grp. gram.:m.}
\end{itemize}
Gênero de insectos neurópteros.
\section{Neuróna}
\begin{itemize}
\item {Grp. gram.:m.}
\end{itemize}
O mesmo que \textunderscore neuróne\textunderscore .
\section{Neuróne}
\begin{itemize}
\item {Grp. gram.:m.}
\end{itemize}
\begin{itemize}
\item {Proveniência:(Do gr. \textunderscore neuron\textunderscore )}
\end{itemize}
Céllula nervosa, com duas espécies de prolongamentos, que se não anastomosam, mas que se encadeiam, como que formando articulações.
\section{Neurónico}
\begin{itemize}
\item {Grp. gram.:adj.}
\end{itemize}
Relativo aos neurónes.
\section{Neurónio}
\begin{itemize}
\item {Grp. gram.:m.}
\end{itemize}
O mesmo ou melhor que \textunderscore neuróne\textunderscore .
\section{Neuroparalisia}
\begin{itemize}
\item {Grp. gram.:f.}
\end{itemize}
Paralisia dos nervos.
\section{Neuroparalítico}
\begin{itemize}
\item {Grp. gram.:adj.}
\end{itemize}
Relativo á neuroparalisia.
Que tem caracteres desta doença.
\section{Neuroparalysia}
\begin{itemize}
\item {Grp. gram.:f.}
\end{itemize}
Paralysia dos nervos.
\section{Neuroparalýtico}
\begin{itemize}
\item {Grp. gram.:adj.}
\end{itemize}
Relativo á neuroparalysia.
Que tem caracteres desta doença.
\section{Neuropata}
\begin{itemize}
\item {Grp. gram.:m.  e  adj.}
\end{itemize}
\begin{itemize}
\item {Proveniência:(Do gr. \textunderscore neuron\textunderscore  + \textunderscore pathos\textunderscore )}
\end{itemize}
Aquele que padece neuropatia.
\section{Neuropatha}
\begin{itemize}
\item {Grp. gram.:m.  e  adj.}
\end{itemize}
\begin{itemize}
\item {Proveniência:(Do gr. \textunderscore neuron\textunderscore  + \textunderscore pathos\textunderscore )}
\end{itemize}
Aquelle que padece neuropathia.
\section{Neuropathia}
\begin{itemize}
\item {Grp. gram.:f.}
\end{itemize}
Doença, caracterizada por perturbação das funcções orgânicas, sem lesão apparente e que se suppõe têr a sua séde no systema nervoso.
(Cp. \textunderscore neuropatha\textunderscore )
\section{Neuropáthico}
\begin{itemize}
\item {Grp. gram.:adj.}
\end{itemize}
Relativo á neuropathia.
\section{Neuropathologia}
\begin{itemize}
\item {Grp. gram.:f.}
\end{itemize}
\begin{itemize}
\item {Proveniência:(Do gr. \textunderscore neuron\textunderscore  + \textunderscore pathos\textunderscore  + \textunderscore logos\textunderscore )}
\end{itemize}
Tratado das doenças nervosas.
\section{Neuropathológico}
\begin{itemize}
\item {Grp. gram.:adj.}
\end{itemize}
Relativo á neuropathologia.
\section{Neuropatia}
\begin{itemize}
\item {Grp. gram.:f.}
\end{itemize}
Doença, caracterizada por perturbação das funcções orgânicas, sem lesão aparente e que se supõe têr a sua séde no sistema nervoso.
(Cp. \textunderscore neuropata\textunderscore )
\section{Neuropático}
\begin{itemize}
\item {Grp. gram.:adj.}
\end{itemize}
Relativo á neuropatia.
\section{Neuropatologia}
\begin{itemize}
\item {Grp. gram.:f.}
\end{itemize}
\begin{itemize}
\item {Proveniência:(Do gr. \textunderscore neuron\textunderscore  + \textunderscore pathos\textunderscore  + \textunderscore logos\textunderscore )}
\end{itemize}
Tratado das doenças nervosas.
\section{Neuropatológico}
\begin{itemize}
\item {Grp. gram.:adj.}
\end{itemize}
Relativo á neuropatologia.
\section{Neuropira}
\begin{itemize}
\item {Grp. gram.:f.}
\end{itemize}
\begin{itemize}
\item {Utilização:Med.}
\end{itemize}
\begin{itemize}
\item {Proveniência:(Do gr. \textunderscore neuron\textunderscore  + \textunderscore pur\textunderscore )}
\end{itemize}
Febre nervosa.
\section{Neuropírico}
\begin{itemize}
\item {Grp. gram.:adj.}
\end{itemize}
Relativo á neuropira.
\section{Neuropterologia}
\begin{itemize}
\item {Grp. gram.:f.}
\end{itemize}
\begin{itemize}
\item {Proveniência:(Do gr. \textunderscore neuron\textunderscore  + \textunderscore pteron\textunderscore  + \textunderscore logos\textunderscore )}
\end{itemize}
Descripção dos insectos neurópteros.
\section{Neuropterológico}
\begin{itemize}
\item {Grp. gram.:adj.}
\end{itemize}
Relativo á neuropterologia.
\section{Neurópteros}
\begin{itemize}
\item {Grp. gram.:m. pl.}
\end{itemize}
\begin{itemize}
\item {Proveniência:(Do gr. \textunderscore neuron\textunderscore  + \textunderscore pteron\textunderscore )}
\end{itemize}
Ordem de insectos, que comprehende aquelles que têm nas asas nervuras, dispostas de maneira que formam uma rede de malhas, mais ou menos regulares.
\section{Neuropyra}
\begin{itemize}
\item {Grp. gram.:f.}
\end{itemize}
\begin{itemize}
\item {Utilização:Med.}
\end{itemize}
\begin{itemize}
\item {Proveniência:(Do gr. \textunderscore neuron\textunderscore  + \textunderscore pur\textunderscore )}
\end{itemize}
Febre nervosa.
\section{Neuropýrico}
\begin{itemize}
\item {Grp. gram.:adj.}
\end{itemize}
Relativo á neuropyra.
\section{Neurosclerose}
\begin{itemize}
\item {Grp. gram.:f.}
\end{itemize}
\begin{itemize}
\item {Utilização:Med.}
\end{itemize}
\begin{itemize}
\item {Proveniência:(Do gr. \textunderscore neuron\textunderscore  + \textunderscore skleros\textunderscore )}
\end{itemize}
Esclerose do tecido nervoso.
\section{Neurose}
\begin{itemize}
\item {Grp. gram.:f.}
\end{itemize}
\begin{itemize}
\item {Proveniência:(Do gr. \textunderscore neuron\textunderscore )}
\end{itemize}
Qualquer doença nervosa.
Neuropathia.
\section{Neurostenia}
\begin{itemize}
\item {Grp. gram.:f.}
\end{itemize}
\begin{itemize}
\item {Proveniência:(Do gr. \textunderscore neuron\textunderscore  + \textunderscore sthenos\textunderscore )}
\end{itemize}
Irritação dos nervos.
\section{Neurostênico}
\begin{itemize}
\item {Grp. gram.:adj.}
\end{itemize}
\begin{itemize}
\item {Grp. gram.:M.}
\end{itemize}
Relativo á neurostenia.
Aquele que padece neurostenia.
\section{Neurosthenia}
\begin{itemize}
\item {Grp. gram.:f.}
\end{itemize}
\begin{itemize}
\item {Proveniência:(Do gr. \textunderscore neuron\textunderscore  + \textunderscore sthenos\textunderscore )}
\end{itemize}
Irritação dos nervos.
\section{Neurosthênico}
\begin{itemize}
\item {Grp. gram.:adj.}
\end{itemize}
\begin{itemize}
\item {Grp. gram.:M.}
\end{itemize}
Relativo á neurosthenia.
Aquelle que padece neurosthenia.
\section{Neurótico}
\begin{itemize}
\item {Grp. gram.:adj.}
\end{itemize}
\begin{itemize}
\item {Grp. gram.:M.}
\end{itemize}
Relativo ao systema nervoso.
Aquelle que soffre neurose.
\section{Neurotomia}
\begin{itemize}
\item {Grp. gram.:f.}
\end{itemize}
Dissecção dos nervos.
(Cp. \textunderscore neurótomo\textunderscore )
\section{Neurotômio}
\begin{itemize}
\item {Grp. gram.:m.}
\end{itemize}
\begin{itemize}
\item {Proveniência:(Do gr. \textunderscore neuron\textunderscore  + \textunderscore tome\textunderscore )}
\end{itemize}
Escalpêlo, com que se pratica a neurotomia.
\section{Neurótomo}
\begin{itemize}
\item {Grp. gram.:m.}
\end{itemize}
\begin{itemize}
\item {Proveniência:(Do gr. \textunderscore neuron\textunderscore  + \textunderscore tome\textunderscore )}
\end{itemize}
Escalpêlo, com que se pratica a neurotomia.
\section{Neutral}
\begin{itemize}
\item {Grp. gram.:adj.}
\end{itemize}
\begin{itemize}
\item {Proveniência:(Lat. \textunderscore neutralis\textunderscore )}
\end{itemize}
O mesmo que \textunderscore neutro\textunderscore .
Imparcial.
Que não intervém a favor nem contra, num pleito ou contenda.
Indifferente.
\section{Neutralidade}
\begin{itemize}
\item {Grp. gram.:f.}
\end{itemize}
\begin{itemize}
\item {Utilização:Chím.}
\end{itemize}
Qualidade ou estado de neutral.
Situação de uma potência neutral, entre duas ou mais nações belligerantes.
Qualidade de certos corpos, que não são ácidos nem alcalinos.
Extincção recíproca das propriedades características do ácido e da base que constituem um sal neutro.
Neutralização das acções phýsicas de certos corpos.
\section{Neutralização}
\begin{itemize}
\item {Grp. gram.:f.}
\end{itemize}
Acto ou effeito de neutralizar.
\section{Neutralizante}
\begin{itemize}
\item {Grp. gram.:adj.}
\end{itemize}
Que neutraliza. Cf. \textunderscore Techn. Rur.\textunderscore , 75.
\section{Neutralizar}
\begin{itemize}
\item {Grp. gram.:v. t.}
\end{itemize}
Tornar neutral.
Tornar inertes as propriedades de.
Annular: \textunderscore neutralizar a influência de alguém\textunderscore .
\section{Neutralmente}
\begin{itemize}
\item {Grp. gram.:adv.}
\end{itemize}
De modo neutral.
Sem tomar partido por uma ou outra de duas ou mais entidades que contendem.
\section{Neutro}
\begin{itemize}
\item {Grp. gram.:adj.}
\end{itemize}
\begin{itemize}
\item {Utilização:Gram.}
\end{itemize}
\begin{itemize}
\item {Utilização:Chím.}
\end{itemize}
\begin{itemize}
\item {Utilização:Bot.}
\end{itemize}
\begin{itemize}
\item {Proveniência:(Lat. \textunderscore neuter\textunderscore )}
\end{itemize}
Que é intransitivo ou que não é activo nem passivo, (falando-se de verbos).
Que não é masculino nem feminino, (falando-se de nomes).
Que não é acido nem alcalino.
Diz-se do animal que não tem sexo ou que não é capaz de se reproduzir.
Que não tem órgãos sexuaes.
Diz-se de quem não toma partido por uma de duas ou mais entidades que contendem.
Indifferente; inactivo.
Que pertence a país neutral.
\section{Nevada}
\begin{itemize}
\item {Grp. gram.:f.}
\end{itemize}
\begin{itemize}
\item {Proveniência:(De \textunderscore nevado\textunderscore )}
\end{itemize}
Acto de caír neve.
Porção de neve, que cái de uma vez.
\section{Nevado}
\begin{itemize}
\item {Grp. gram.:adj.}
\end{itemize}
\begin{itemize}
\item {Grp. gram.:M.}
\end{itemize}
Branco como a neve.
Branqueado.
Frio.
Que tem algumas pequenas manchas brancas, (falando-se do toiro).
Espécie de pó branco, formado por grânulos arredondados, em que o calor solar transforma os elementos crystallinos, de que a neve se compõe.
\section{Nevão}
\begin{itemize}
\item {Grp. gram.:m.}
\end{itemize}
Grande porção de neve, caindo.
\section{Nevar}
\begin{itemize}
\item {Grp. gram.:v. t.}
\end{itemize}
\begin{itemize}
\item {Grp. gram.:V. i.}
\end{itemize}
\begin{itemize}
\item {Utilização:Fig.}
\end{itemize}
\begin{itemize}
\item {Grp. gram.:V. p.}
\end{itemize}
\begin{itemize}
\item {Proveniência:(Do lat. \textunderscore nivare\textunderscore )}
\end{itemize}
Cobrir de neve.
Esfriar por meio de gêlo ou neve.
Caír neve.
Branquejar.
Tornar-se branco, encanecer:«\textunderscore os cabellos a nevarem-se-lhe...\textunderscore »Camillo.
\section{Nevasca}
\begin{itemize}
\item {Grp. gram.:f.}
\end{itemize}
Nevada, acompanhada de tempestade.
(Cp. cast. \textunderscore nevasca\textunderscore )
\section{Nevasqueira}
\begin{itemize}
\item {Grp. gram.:f.}
\end{itemize}
\begin{itemize}
\item {Utilização:Prov.}
\end{itemize}
\begin{itemize}
\item {Utilização:trasm.}
\end{itemize}
O mesmo que \textunderscore murugem\textunderscore .
\section{Neve}
\begin{itemize}
\item {Grp. gram.:f.}
\end{itemize}
\begin{itemize}
\item {Utilização:Fig.}
\end{itemize}
\begin{itemize}
\item {Proveniência:(Lat. \textunderscore nix\textunderscore , \textunderscore nivis\textunderscore )}
\end{itemize}
Água congelada, que cái da atmosphera em frocos finos e muito brancos.
Alvura.
Cans.
Frio intenso.
Iguaria gelada, em que entra açúcar com leite ou com o suco de certas frutas; sorvete.
\section{Nêveda}
\begin{itemize}
\item {Grp. gram.:f.}
\end{itemize}
\begin{itemize}
\item {Proveniência:(Do lat. \textunderscore nepeta\textunderscore )}
\end{itemize}
Nome de várias plantas, especialmente da \textunderscore satureja calamintha\textunderscore , Lin., também conhecida por \textunderscore erva das azeitonas\textunderscore .
\section{Neveira}
\begin{itemize}
\item {Grp. gram.:f.}
\end{itemize}
Lugar, em que se fabríca gêlo ou se guarda neve.
Geleira.
Sorveteira.
\section{Neveiro}
\begin{itemize}
\item {Grp. gram.:m.}
\end{itemize}
Vendedor de neve ou de sorvetes.
\section{Neviscar}
\begin{itemize}
\item {Grp. gram.:v. i.}
\end{itemize}
Caír neve em pequena quantidade.
\section{Nevo}
\begin{itemize}
\item {Grp. gram.:m.}
\end{itemize}
\begin{itemize}
\item {Proveniência:(Lat. \textunderscore naevus\textunderscore )}
\end{itemize}
Mancha, que algumas crianças trazem na pelle, quando nascem.
\section{Névoa}
\begin{itemize}
\item {Grp. gram.:f.}
\end{itemize}
\begin{itemize}
\item {Utilização:Fig.}
\end{itemize}
\begin{itemize}
\item {Utilização:fig.}
\end{itemize}
\begin{itemize}
\item {Utilização:Ant.}
\end{itemize}
\begin{itemize}
\item {Proveniência:(Do lat. \textunderscore nebula\textunderscore )}
\end{itemize}
Vapor aquoso, que obscurece um pouco a atmosphera.
Obscuridade.
Mancha que, formando-se na córnea, turva a vista.
Aquillo que embaraça a vista.
O que difficulta a comprehensão.
Substância, que se condensa na urina.
Embuste, fantasmagoria.
\section{Nevôa}
\begin{itemize}
\item {Grp. gram.:f.}
\end{itemize}
\begin{itemize}
\item {Utilização:Açor}
\end{itemize}
\begin{itemize}
\item {Proveniência:(De \textunderscore nevoar-se\textunderscore )}
\end{itemize}
O mesmo que \textunderscore névoa\textunderscore .
\section{Nevoaça}
\begin{itemize}
\item {Grp. gram.:f.}
\end{itemize}
O mesmo que \textunderscore nevoeiro\textunderscore .
\section{Nevoar-se}
\begin{itemize}
\item {Grp. gram.:v. p.}
\end{itemize}
Cobrir-se de névoa; obscurecer-se; ennevoar-se.
\section{Nevoeira}
\begin{itemize}
\item {Grp. gram.:f.}
\end{itemize}
Variedade de uva tinta, também conhecida por \textunderscore farinhota\textunderscore .
\section{Nevoeirada}
\begin{itemize}
\item {Grp. gram.:f.}
\end{itemize}
\begin{itemize}
\item {Utilização:Prov.}
\end{itemize}
Nevoeiro denso e prolongado.
\section{Nevoeiro}
\begin{itemize}
\item {Grp. gram.:m.}
\end{itemize}
\begin{itemize}
\item {Utilização:Fig.}
\end{itemize}
\begin{itemize}
\item {Proveniência:(De \textunderscore névoa\textunderscore )}
\end{itemize}
Névoa espêssa.
Obscuridade.
Nevoeira.
\section{Nevoentar-se}
\begin{itemize}
\item {Grp. gram.:v. p.}
\end{itemize}
\begin{itemize}
\item {Utilização:Fig.}
\end{itemize}
\begin{itemize}
\item {Proveniência:(De \textunderscore nevoento\textunderscore )}
\end{itemize}
O mesmo que \textunderscore nevoar-se\textunderscore .
Tornar-se obscuro, inintelligível.
\section{Nevoento}
\begin{itemize}
\item {Grp. gram.:adj.}
\end{itemize}
\begin{itemize}
\item {Utilização:Fig.}
\end{itemize}
\begin{itemize}
\item {Proveniência:(De \textunderscore névoa\textunderscore )}
\end{itemize}
O mesmo que [[ennevoado|ennevoar]].
Obscuro; pouco comprehensível.
\section{Nevoso}
\begin{itemize}
\item {Grp. gram.:adj.}
\end{itemize}
\begin{itemize}
\item {Proveniência:(Do lat. \textunderscore nivosus\textunderscore )}
\end{itemize}
Nevado; nevoento.
\section{Nevresia}
\begin{itemize}
\item {Grp. gram.:f.}
\end{itemize}
\begin{itemize}
\item {Utilização:Prov.}
\end{itemize}
\begin{itemize}
\item {Utilização:trasm.}
\end{itemize}
Grande porção; grande número.
\section{Nevri}
\begin{itemize}
\item {Grp. gram.:m.}
\end{itemize}
\begin{itemize}
\item {Utilização:Ant.}
\end{itemize}
O mesmo que \textunderscore nebri\textunderscore . Cf. Fernão Lopes.
\section{Nevro...}
Prefixo inexacto, designativo de \textunderscore nervo\textunderscore  ou de relativo a \textunderscore nervos\textunderscore .
A fórma rigorosa é \textunderscore neuro...\textunderscore  Portanto \textunderscore neuralgia\textunderscore , \textunderscore neurite\textunderscore , \textunderscore neuroma\textunderscore , \textunderscore neurópteros\textunderscore , \textunderscore neurose\textunderscore , \textunderscore neurotomia\textunderscore , etc.
\section{Newtonianismo}
\begin{itemize}
\item {Grp. gram.:m.}
\end{itemize}
\begin{itemize}
\item {Proveniência:(De \textunderscore newtoniano\textunderscore )}
\end{itemize}
Philosophia natural de Newton, ou systema de Newton, relativamente ás causas dos movimentos dos corpos celestes.
\section{Newtoniano}
\begin{itemize}
\item {Grp. gram.:adj.}
\end{itemize}
\begin{itemize}
\item {Proveniência:(De \textunderscore Newton\textunderscore , n. p.)}
\end{itemize}
Relativo ao newtonianismo; que segue êste systema.
\section{Newtonismo}
\begin{itemize}
\item {Grp. gram.:m.}
\end{itemize}
O mesmo que \textunderscore newtonianismo\textunderscore .
\section{Nexo}
\begin{itemize}
\item {fónica:cso}
\end{itemize}
\begin{itemize}
\item {Grp. gram.:m.}
\end{itemize}
\begin{itemize}
\item {Proveniência:(Lat. \textunderscore nexus\textunderscore )}
\end{itemize}
União, ligação, vínculo.
Connexão.
Aquelle que, entre os Romanos, servia como escravo o seu credor, até que saldasse a dívida respectiva.
\section{Nháfete}
\begin{itemize}
\item {Grp. gram.:m.}
\end{itemize}
\begin{itemize}
\item {Utilização:Ant.}
\end{itemize}
Designação injuriosa de christão novo.
(Corr. de \textunderscore neóphyto\textunderscore )
\section{Nhambi}
\begin{itemize}
\item {Grp. gram.:m.}
\end{itemize}
\begin{itemize}
\item {Utilização:Bras}
\end{itemize}
Planta, da fam. das compostas, (\textunderscore anthenus\textunderscore ).
\section{Nhambu}
\begin{itemize}
\item {Grp. gram.:m.}
\end{itemize}
O mesmo que \textunderscore jambu\textunderscore .
Ave brasileira, espécie de perdiz, geralmente vermelha, o mesmo que \textunderscore nambu\textunderscore .
\section{Nhandipapo}
\begin{itemize}
\item {Grp. gram.:m.}
\end{itemize}
\begin{itemize}
\item {Utilização:Bras}
\end{itemize}
Árvore silvestre.
\section{Nhandiroba}
\begin{itemize}
\item {Grp. gram.:f.}
\end{itemize}
O mesmo que \textunderscore nandiroba\textunderscore .
\section{Nhandirova}
\begin{itemize}
\item {Grp. gram.:f.}
\end{itemize}
\begin{itemize}
\item {Utilização:Bras}
\end{itemize}
O mesmo que \textunderscore nandiroba\textunderscore .
\section{Nhandu}
\begin{itemize}
\item {Grp. gram.:m.}
\end{itemize}
\begin{itemize}
\item {Utilização:Bras}
\end{itemize}
O mesmo que \textunderscore ema\textunderscore ^1.
(Do tupi)
\section{Nhangue}
\begin{itemize}
\item {Grp. gram.:m.}
\end{itemize}
Ave pernalta da África occidental.
\section{Nhanhá}
\begin{itemize}
\item {Grp. gram.:f.}
\end{itemize}
O mesmo que \textunderscore nhanhan\textunderscore .
\section{Nhanhan}
\begin{itemize}
\item {Grp. gram.:f.}
\end{itemize}
\begin{itemize}
\item {Utilização:Bras}
\end{itemize}
Tratamento carinhoso, dado a meninas.
\section{Nhanica}
\begin{itemize}
\item {Grp. gram.:f.}
\end{itemize}
Árvore myrtácea do Brasil.
\section{Nhanzinha}
\begin{itemize}
\item {Grp. gram.:f.}
\end{itemize}
\begin{itemize}
\item {Utilização:Bras}
\end{itemize}
Fórma contrahida de \textunderscore nhanhanzinha\textunderscore , dem. de \textunderscore nhanhan\textunderscore .
\section{Nhapim}
\begin{itemize}
\item {Grp. gram.:m.}
\end{itemize}
Ave canora do Brasil, também conhecida por \textunderscore soldado\textunderscore  e \textunderscore encontro\textunderscore .
\section{Nhato}
\begin{itemize}
\item {Grp. gram.:adj.}
\end{itemize}
\begin{itemize}
\item {Utilização:Bras. de S. Paulo}
\end{itemize}
Que tem proeminente a maxilla inferior.
(Cp. \textunderscore prógnatho\textunderscore )
\section{Nhele}
\begin{itemize}
\item {Grp. gram.:m.}
\end{itemize}
Pequeno peixe africano. Cf. Serpa Pinto, I, 298.
\section{Nhengaíbas}
\begin{itemize}
\item {Grp. gram.:m. pl.}
\end{itemize}
Índios do Brasil, que habitaram na ilha de Marajó.
\section{Nhô}
\begin{itemize}
\item {Grp. gram.:m.}
\end{itemize}
O mesmo que \textunderscore nhôr\textunderscore .
\section{Nhonha}
\begin{itemize}
\item {Grp. gram.:adj. f.}
\end{itemize}
\textunderscore Língua nhonha\textunderscore , o dialecto crioulo-português de Macau.
(Deturpação de \textunderscore senhora\textunderscore )
\section{Nhonhô}
\begin{itemize}
\item {Grp. gram.:m.}
\end{itemize}
\begin{itemize}
\item {Utilização:Bras. do S}
\end{itemize}
Tratamento familiar, que se dá aos meninos.
\section{Nhôr}
\begin{itemize}
\item {Grp. gram.:m.}
\end{itemize}
\begin{itemize}
\item {Utilização:Bras}
\end{itemize}
Abrev. de \textunderscore senhor\textunderscore .
\section{Nhôra}
\begin{itemize}
\item {Grp. gram.:f.}
\end{itemize}
\begin{itemize}
\item {Utilização:Bras}
\end{itemize}
Abrev. de \textunderscore senhora\textunderscore .
\section{Nhozinho}
\begin{itemize}
\item {Grp. gram.:m.}
\end{itemize}
\begin{itemize}
\item {Utilização:Bras}
\end{itemize}
Abrev. pop. de \textunderscore senhorzinho\textunderscore , dem. de \textunderscore senhor\textunderscore .
\section{Nhu}
\begin{itemize}
\item {Grp. gram.:m.}
\end{itemize}
Espécie de antílope africano.
\section{Nhum}
\begin{itemize}
\item {Grp. gram.:pron.}
\end{itemize}
\begin{itemize}
\item {Utilização:Ant.}
\end{itemize}
O mesmo que \textunderscore nenhum\textunderscore . Cf. Castanheda, \textunderscore Descobr. da Índ.\textunderscore , pról.
\section{Nhumbo}
\begin{itemize}
\item {Grp. gram.:m.}
\end{itemize}
Corpulento animal da Zambézia, semelhante ao búfalo.
\section{Nhumue}
\begin{itemize}
\item {Grp. gram.:m.}
\end{itemize}
Arbusto de Moçambique.
\section{Nhundi}
\begin{itemize}
\item {Grp. gram.:m.}
\end{itemize}
Arbusto moçambicano.
\section{Nhurro}
\begin{itemize}
\item {Grp. gram.:m.}
\end{itemize}
\begin{itemize}
\item {Utilização:Gír. do Pôrto.}
\end{itemize}
Pataco.
\section{Ni}
\begin{itemize}
\item {Grp. gram.:conj.}
\end{itemize}
\begin{itemize}
\item {Utilização:Ant.}
\end{itemize}
O mesmo que \textunderscore nem\textunderscore .
\section{Niagem}
\begin{itemize}
\item {Grp. gram.:f.}
\end{itemize}
\begin{itemize}
\item {Utilização:Ant.}
\end{itemize}
Lençaria grossa de linho cru.
(Corr. de \textunderscore linhagem\textunderscore )
\section{Nial}
\begin{itemize}
\item {Grp. gram.:m.}
\end{itemize}
\begin{itemize}
\item {Utilização:Prov.}
\end{itemize}
\begin{itemize}
\item {Utilização:trasm.}
\end{itemize}
\begin{itemize}
\item {Proveniência:(Do lat. \textunderscore nidalis\textunderscore )}
\end{itemize}
O mesmo que \textunderscore ninho\textunderscore . (Colhido em Miranda)
\section{Nianeca}
\begin{itemize}
\item {Grp. gram.:m.}
\end{itemize}
Língua angolense; lunhaneca.
\section{Nibora}
\begin{itemize}
\item {Grp. gram.:f.}
\end{itemize}
Gênero de plantas aquáticas, da fam. das acantháceas.
\section{Nibu}
\begin{itemize}
\item {Grp. gram.:m.}
\end{itemize}
Moéda japonesa de prata.
\section{Nica}
\begin{itemize}
\item {Grp. gram.:f.}
\end{itemize}
\begin{itemize}
\item {Utilização:Fam.}
\end{itemize}
\begin{itemize}
\item {Utilização:Des.}
\end{itemize}
\begin{itemize}
\item {Grp. gram.:Pl.}
\end{itemize}
\begin{itemize}
\item {Grp. gram.:Loc.}
\end{itemize}
\begin{itemize}
\item {Utilização:fam.}
\end{itemize}
\begin{itemize}
\item {Proveniência:(Do lat. \textunderscore nichil\textunderscore , por \textunderscore nihil\textunderscore ?)}
\end{itemize}
Bagatela, insignificância.
Impertinência; puerilidade.
Trapaça.

\textunderscore Fazer nicas\textunderscore , fazer figas; escarnecer. Cf. Filinto, XI, 123.
\section{Nica}
\begin{itemize}
\item {Grp. gram.:f.}
\end{itemize}
\begin{itemize}
\item {Utilização:Prov.}
\end{itemize}
\begin{itemize}
\item {Utilização:beir.}
\end{itemize}
Escalavradura ou brecha, feita num pião com ferroadas de outro pião.
\section{Nicada}
\begin{itemize}
\item {Grp. gram.:f.}
\end{itemize}
Acto ou effeito de nicar.
\section{Nicanço}
\begin{itemize}
\item {Grp. gram.:m.}
\end{itemize}
\begin{itemize}
\item {Utilização:Prov.}
\end{itemize}
\begin{itemize}
\item {Utilização:alent.}
\end{itemize}
O mesmo que \textunderscore alicranço\textunderscore .
\section{Nicandra}
\begin{itemize}
\item {Grp. gram.:f.}
\end{itemize}
O mesmo que \textunderscore nicandro\textunderscore .
\section{Nicandro}
\begin{itemize}
\item {Grp. gram.:m.}
\end{itemize}
\begin{itemize}
\item {Proveniência:(De \textunderscore Nicandro\textunderscore , n. p.)}
\end{itemize}
Gênero de plantas solâneas, originárias do Peru.
\section{Nicané}
\begin{itemize}
\item {Grp. gram.:m.}
\end{itemize}
Tecido florentino de algodão, que se exporta para a África.
\section{Nicar}
\begin{itemize}
\item {Grp. gram.:v. i.}
\end{itemize}
\begin{itemize}
\item {Utilização:Pop.}
\end{itemize}
\begin{itemize}
\item {Grp. gram.:V. i.}
\end{itemize}
\begin{itemize}
\item {Utilização:Gír.}
\end{itemize}
Picar com o bico, (falando-se de aves).
Escalavrar ou rachar com o bico de um pião (outro pião).
Têr cóito.
\section{Niceno}
\begin{itemize}
\item {Grp. gram.:adj.}
\end{itemize}
Relativo á cidade de Niceia.
\section{Nicerotiano}
\begin{itemize}
\item {Grp. gram.:m.}
\end{itemize}
\begin{itemize}
\item {Proveniência:(De \textunderscore Nicerote\textunderscore , n. p.)}
\end{itemize}
Unguento oloroso, usado antigamente em pharmácia.
\section{Nicha}
\begin{itemize}
\item {Grp. gram.:f.}
\end{itemize}
\begin{itemize}
\item {Utilização:Prov.}
\end{itemize}
\begin{itemize}
\item {Utilização:trasm.}
\end{itemize}
\begin{itemize}
\item {Proveniência:(De \textunderscore nicho\textunderscore )}
\end{itemize}
Buraco no chão, para o jôgo da choca.
\section{Nichilar}
\begin{itemize}
\item {Grp. gram.:v. t.}
\end{itemize}
\begin{itemize}
\item {Utilização:Ant.}
\end{itemize}
\begin{itemize}
\item {Proveniência:(Do b. lat. \textunderscore nichilare\textunderscore , por \textunderscore nihilari\textunderscore )}
\end{itemize}
O mesmo que \textunderscore aniquilar\textunderscore . Cf. \textunderscore Aulegrafia\textunderscore , 3O.
\section{Nicho}
\begin{itemize}
\item {Grp. gram.:m.}
\end{itemize}
\begin{itemize}
\item {Utilização:Fam.}
\end{itemize}
\begin{itemize}
\item {Utilização:Fig.}
\end{itemize}
\begin{itemize}
\item {Proveniência:(Do it. \textunderscore nicchia\textunderscore ?)}
\end{itemize}
Cavidade em parede, para collocação de uma estátua, imagem, vaso, etc.
Compartimento de estante ou armário; vão.
Emprêgo rendoso e de pouco trabalho.
Emprêgo.
Pequena habitação.
Retiro, insulamento.
\section{Níckel}
\textunderscore m.\textunderscore  (e der.)
(V. \textunderscore níquel\textunderscore , etc.)
(Sueco \textunderscore nickel\textunderscore )
\section{Nicles}
\begin{itemize}
\item {Grp. gram.:adv.}
\end{itemize}
\begin{itemize}
\item {Utilização:Gír.}
\end{itemize}
\begin{itemize}
\item {Proveniência:(Do lat. \textunderscore nichil\textunderscore )}
\end{itemize}
Coisa nenhuma; nada.
\section{Nico}
\begin{itemize}
\item {Grp. gram.:m.}
\end{itemize}
\begin{itemize}
\item {Utilização:Ant.}
\end{itemize}
O mesmo que \textunderscore nica\textunderscore ^1.
O mesmo que \textunderscore macaco\textunderscore .
\section{Nico}
\begin{itemize}
\item {Grp. gram.:adj.}
\end{itemize}
O mesmo que \textunderscore niquento\textunderscore :«\textunderscore terás as viagens nicas...\textunderscore »Castilho.
\section{Nicociana}
\begin{itemize}
\item {Grp. gram.:f.}
\end{itemize}
\begin{itemize}
\item {Utilização:Bot.}
\end{itemize}
\begin{itemize}
\item {Proveniência:(De \textunderscore Nicot\textunderscore , n. p.)}
\end{itemize}
Designação scientífica do tabaco.
\section{Nicociâneas}
\begin{itemize}
\item {Grp. gram.:f. pl.}
\end{itemize}
\begin{itemize}
\item {Proveniência:(De \textunderscore nicociâneo\textunderscore )}
\end{itemize}
Tríbo de plantas solâneas, que tem por typo a nicociana.
\section{Nicociâneo}
\begin{itemize}
\item {Grp. gram.:adj.}
\end{itemize}
Relativo ou semelhante á nicociana.
\section{Nicocianina}
\begin{itemize}
\item {Grp. gram.:f.}
\end{itemize}
\begin{itemize}
\item {Proveniência:(De \textunderscore nicociana\textunderscore )}
\end{itemize}
Substância, extrahida das fôlhas verdes do tabaco.
\section{Nicola}
\begin{itemize}
\item {Grp. gram.:f.}
\end{itemize}
\begin{itemize}
\item {Utilização:Gír.}
\end{itemize}
Acto de nicar, \textunderscore v. i.\textunderscore 
\section{Nicolaítas}
\begin{itemize}
\item {Grp. gram.:m. pl.}
\end{itemize}
Seita judaica do tempo dos Apóstolos. Cf. Castilho, \textunderscore Fastos\textunderscore , I, 531.
\section{Nicolato}
\begin{itemize}
\item {Grp. gram.:m.}
\end{itemize}
\begin{itemize}
\item {Proveniência:(De \textunderscore níquel\textunderscore )}
\end{itemize}
Combinação do óxydo nicólico com uma base salificável.
\section{Nicólico}
\begin{itemize}
\item {Grp. gram.:adj.}
\end{itemize}
\begin{itemize}
\item {Proveniência:(De \textunderscore níquel\textunderscore )}
\end{itemize}
Diz-se de um dos óxydos do níquel e dos sáes que derivam dêsse óxydo.
\section{Nicolsónia}
\begin{itemize}
\item {Grp. gram.:f.}
\end{itemize}
\begin{itemize}
\item {Proveniência:(De \textunderscore Nicolson\textunderscore , n. p.)}
\end{itemize}
Gênero de plantas leguminosas da América tropical.
\section{Nicotice}
\begin{itemize}
\item {Grp. gram.:f.}
\end{itemize}
\begin{itemize}
\item {Utilização:T. da Bairrada}
\end{itemize}
Nica; futilidade; bagatela.
\section{Nicótico}
\begin{itemize}
\item {Grp. gram.:adj.}
\end{itemize}
\begin{itemize}
\item {Proveniência:(De \textunderscore Nicot.\textunderscore , n. p.)}
\end{itemize}
Relativo ao tabaco.
\section{Nicotina}
\begin{itemize}
\item {Grp. gram.:f.}
\end{itemize}
\begin{itemize}
\item {Proveniência:(De \textunderscore nicotino\textunderscore )}
\end{itemize}
Alcalóide orgânico, que existe no tabaco.
\section{Nicotino}
\begin{itemize}
\item {Grp. gram.:adj.}
\end{itemize}
\begin{itemize}
\item {Proveniência:(De \textunderscore Nicot.\textunderscore , n. p.)}
\end{itemize}
Próprio do tabaco; soporífero.
\section{Nicotizado}
\begin{itemize}
\item {Grp. gram.:adj.}
\end{itemize}
Impregnado dos vapores ou fumo do tabaco.
Entoxicado pela nicotina.
(Cp. \textunderscore nicótico\textunderscore )
\section{Nicromancia}
\begin{itemize}
\item {Grp. gram.:f.}
\end{itemize}
(Corr. de \textunderscore necromancia\textunderscore )
\section{Nictação}
\begin{itemize}
\item {Grp. gram.:f.}
\end{itemize}
\begin{itemize}
\item {Proveniência:(Lat. \textunderscore nictatio\textunderscore )}
\end{itemize}
Acto de abaixar ou levantar as pálpebras, sob impressão de luz intensa; pestanejo.
\section{Nictitante}
\begin{itemize}
\item {Grp. gram.:adj.}
\end{itemize}
\begin{itemize}
\item {Utilização:Hist. nat.}
\end{itemize}
\begin{itemize}
\item {Proveniência:(De um hyp. \textunderscore nictitar\textunderscore , freq. do lat. \textunderscore nictare\textunderscore )}
\end{itemize}
Que tem ou exerce nictação, como a membrana que constitue a terceira pálpebra lateral das aves.
\section{Nicto}
\begin{itemize}
\item {Grp. gram.:m.}
\end{itemize}
Nome que, nalguns pontos da África, se dá á doença do somno. Cf. Capello e Ivens, I, 125.
(Cp. \textunderscore nictação\textunderscore )
\section{Nidificação}
\begin{itemize}
\item {Grp. gram.:f.}
\end{itemize}
Acto de \textunderscore nidificar\textunderscore .
\section{Nidificar}
\begin{itemize}
\item {Grp. gram.:v. i.}
\end{itemize}
\begin{itemize}
\item {Proveniência:(Lat. \textunderscore nidificare\textunderscore )}
\end{itemize}
Formar ninho. Cf. Quevedo, \textunderscore Aff. Africano\textunderscore , 91.
\section{Nidor}
\begin{itemize}
\item {Grp. gram.:m.}
\end{itemize}
\begin{itemize}
\item {Utilização:Des.}
\end{itemize}
\begin{itemize}
\item {Proveniência:(Lat. \textunderscore nidor\textunderscore )}
\end{itemize}
Cheiro, que sobe e sái do estômago em que há indigestão.
\section{Nidorela}
\begin{itemize}
\item {Grp. gram.:f.}
\end{itemize}
Gênero de plantas, da fam. das compostas.
\section{Nidorella}
\begin{itemize}
\item {Grp. gram.:f.}
\end{itemize}
Gênero de plantas, da fam. das compostas.
\section{Nidoroso}
\begin{itemize}
\item {Grp. gram.:adj.}
\end{itemize}
\begin{itemize}
\item {Proveniência:(Lat. \textunderscore nidorosus\textunderscore )}
\end{itemize}
Que tem cheiro; que cheira desagradavelmente.
Que tem bafio.
\section{Nidulária}
\begin{itemize}
\item {Grp. gram.:f.}
\end{itemize}
Gênero de molluscos gasteromycetos.
\section{Niebúria}
\begin{itemize}
\item {Grp. gram.:f.}
\end{itemize}
Gênero de plantas caparídeas.
\section{Nieiro}
\begin{itemize}
\item {Grp. gram.:m.}
\end{itemize}
\begin{itemize}
\item {Utilização:Prov.}
\end{itemize}
\begin{itemize}
\item {Utilização:trasm.}
\end{itemize}
\begin{itemize}
\item {Proveniência:(Do lat. hyp. \textunderscore nidarius\textunderscore , de \textunderscore nidus\textunderscore , ninho)}
\end{itemize}
Lugar, onde as gallinhas costumam pôr ovos.
\section{Niello}
\begin{itemize}
\item {Grp. gram.:m.}
\end{itemize}
\begin{itemize}
\item {Proveniência:(Fr. \textunderscore nielle\textunderscore )}
\end{itemize}
Esmalte preto, o mesmo que nigella.
\section{Nielo}
\begin{itemize}
\item {Grp. gram.:m.}
\end{itemize}
\begin{itemize}
\item {Proveniência:(Fr. \textunderscore nielle\textunderscore )}
\end{itemize}
Esmalte preto, o mesmo que nigella.
\section{Nigalho}
\begin{itemize}
\item {Grp. gram.:m.}
\end{itemize}
\begin{itemize}
\item {Utilização:Prov.}
\end{itemize}
\begin{itemize}
\item {Utilização:minh.}
\end{itemize}
Bocadinho.
O mesmo que \textunderscore negalho\textunderscore .
\section{Nigela}
\begin{itemize}
\item {Grp. gram.:f.}
\end{itemize}
\begin{itemize}
\item {Proveniência:(Lat. \textunderscore nigella\textunderscore )}
\end{itemize}
Gênero de plantas ranunculáceas.
Ornato de esmalte preto, em obras de ourivezaria.
\section{Nigelar}
\begin{itemize}
\item {Grp. gram.:v. t.}
\end{itemize}
\begin{itemize}
\item {Proveniência:(De \textunderscore nigela\textunderscore )}
\end{itemize}
Ornar com esmalte preto.
\section{Nigelina}
\begin{itemize}
\item {Grp. gram.:f.}
\end{itemize}
\begin{itemize}
\item {Utilização:Chím.}
\end{itemize}
Substância amarga, extraida da nigela.
\section{Nigella}
\begin{itemize}
\item {Grp. gram.:f.}
\end{itemize}
\begin{itemize}
\item {Proveniência:(Lat. \textunderscore nigella\textunderscore )}
\end{itemize}
Gênero de plantas ranunculáceas.
Ornato de esmalte preto, em obras de ourivezaria.
\section{Nigellar}
\begin{itemize}
\item {Grp. gram.:v. t.}
\end{itemize}
\begin{itemize}
\item {Proveniência:(De \textunderscore nigella\textunderscore )}
\end{itemize}
Ornar com esmalte preto.
\section{Nigellina}
\begin{itemize}
\item {Grp. gram.:f.}
\end{itemize}
\begin{itemize}
\item {Utilização:Chím.}
\end{itemize}
Substância amarga, extrahida da nigella.
\section{Nigérrimo}
\begin{itemize}
\item {Grp. gram.:adj.}
\end{itemize}
\begin{itemize}
\item {Proveniência:(Lat. \textunderscore nigerrimus\textunderscore )}
\end{itemize}
Muito negro.
\section{Nígoa}
\begin{itemize}
\item {Grp. gram.:f.}
\end{itemize}
(V.nígua)
\section{Nigricóneo}
\begin{itemize}
\item {Grp. gram.:adj.}
\end{itemize}
\begin{itemize}
\item {Utilização:Zool.}
\end{itemize}
\begin{itemize}
\item {Proveniência:(Do lat. \textunderscore niger\textunderscore  + \textunderscore cornu\textunderscore )}
\end{itemize}
Que tem antennas negras.
\section{Nigrilho}
\begin{itemize}
\item {Grp. gram.:m.}
\end{itemize}
\begin{itemize}
\item {Utilização:Bot.}
\end{itemize}
O mesmo que \textunderscore negrilho\textunderscore , planta. Cf. P. Coutinho, \textunderscore Flora de Port\textunderscore , 170.
\section{Nigrina}
\begin{itemize}
\item {Grp. gram.:f.}
\end{itemize}
\begin{itemize}
\item {Utilização:Miner.}
\end{itemize}
Planta da China.
Variedade de titanato do ferro amorpho.
\section{Nigrípede}
\begin{itemize}
\item {Grp. gram.:adj.}
\end{itemize}
\begin{itemize}
\item {Utilização:Zool.}
\end{itemize}
\begin{itemize}
\item {Grp. gram.:M.}
\end{itemize}
\begin{itemize}
\item {Proveniência:(Do lat. \textunderscore niger\textunderscore  + \textunderscore pes\textunderscore )}
\end{itemize}
Que tem pés negros ou escuros.
Variedade de mammifero felino, do tamanho do gato.
\section{Nigripene}
\begin{itemize}
\item {Grp. gram.:adj.}
\end{itemize}
\begin{itemize}
\item {Utilização:Zool.}
\end{itemize}
\begin{itemize}
\item {Proveniência:(Do lat. \textunderscore niger\textunderscore  + \textunderscore penna\textunderscore )}
\end{itemize}
Que tem asas ou elitros negros.
\section{Nigripenné}
\begin{itemize}
\item {Grp. gram.:adj.}
\end{itemize}
\begin{itemize}
\item {Utilização:Zool.}
\end{itemize}
\begin{itemize}
\item {Proveniência:(Do lat. \textunderscore niger\textunderscore  + \textunderscore penna\textunderscore )}
\end{itemize}
Que tem asas ou elythros negros.
\section{Nigrirostro}
\begin{itemize}
\item {fónica:ros}
\end{itemize}
\begin{itemize}
\item {Grp. gram.:adj.}
\end{itemize}
\begin{itemize}
\item {Utilização:Zool.}
\end{itemize}
\begin{itemize}
\item {Proveniência:(Do lat. \textunderscore niger\textunderscore  + \textunderscore rostrum\textunderscore )}
\end{itemize}
Que tem bico ou tromba escura.
\section{Nigrirrostro}
\begin{itemize}
\item {Grp. gram.:adj.}
\end{itemize}
\begin{itemize}
\item {Utilização:Zool.}
\end{itemize}
\begin{itemize}
\item {Proveniência:(Do lat. \textunderscore niger\textunderscore  + \textunderscore rostrum\textunderscore )}
\end{itemize}
Que tem bico ou tromba escura.
\section{Nigritela}
\begin{itemize}
\item {Grp. gram.:f.}
\end{itemize}
Gênero de orquídeas.
\section{Nigritella}
\begin{itemize}
\item {Grp. gram.:f.}
\end{itemize}
Gênero de orchídeas.
\section{Nigromancia}
\textunderscore f.\textunderscore  (e der.)
O mesmo que \textunderscore necromancia\textunderscore , etc.
\section{Nígua}
\begin{itemize}
\item {Grp. gram.:f.}
\end{itemize}
Insecto díptero, americano, (\textunderscore pulex penetrans\textunderscore ), que se introduz nos pés das pessôas, depositando alli ovos, de que saem immediatamemte novos indivíduos, que causam grandes dores e ás vezes a morte.
\section{Nihilismo}
\begin{itemize}
\item {Grp. gram.:m.}
\end{itemize}
\begin{itemize}
\item {Proveniência:(Do lat. \textunderscore nihil\textunderscore )}
\end{itemize}
Aniquilamento, reducção a nada; descrença absoluta.
Seita russa, que tem por objecto a destruição da ordem social estabelecida.
\section{Nihilista}
\begin{itemize}
\item {Grp. gram.:m. ,  f.  e  adj.}
\end{itemize}
\begin{itemize}
\item {Proveniência:(Do lat. \textunderscore nihil\textunderscore )}
\end{itemize}
Pessôa, que segue o nihilismo.
\section{Niilismo}
\begin{itemize}
\item {Grp. gram.:m.}
\end{itemize}
\begin{itemize}
\item {Proveniência:(Do lat. \textunderscore nihil\textunderscore )}
\end{itemize}
Aniquilamento, reducção a nada; descrença absoluta.
Seita russa, que tem por objecto a destruição da ordem social estabelecida.
\section{Niilista}
\begin{itemize}
\item {Grp. gram.:m. ,  f.  e  adj.}
\end{itemize}
\begin{itemize}
\item {Proveniência:(Do lat. \textunderscore nihil\textunderscore )}
\end{itemize}
Pessôa, que segue o niilismo.
\section{Nilas}
\begin{itemize}
\item {Grp. gram.:m.}
\end{itemize}
Tecido de cascas de árvores e seda, fabricado na Índia.
\section{Nilgó}
\begin{itemize}
\item {Grp. gram.:m.}
\end{itemize}
Grande antílope da Índia, (\textunderscore antilope picta\textunderscore ).
\section{Nilíaco}
\begin{itemize}
\item {Grp. gram.:adj.}
\end{itemize}
O mesmo que \textunderscore nílico\textunderscore .
\section{Nílico}
\begin{itemize}
\item {Grp. gram.:adj.}
\end{itemize}
Relativo ao rio Nilo.
\section{Nílio}
\begin{itemize}
\item {Grp. gram.:m.}
\end{itemize}
\begin{itemize}
\item {Proveniência:(Do b. lat. \textunderscore nilius\textunderscore )}
\end{itemize}
Gênero de insectos coleópteros heterómeros.
\section{Nillas}
\begin{itemize}
\item {Grp. gram.:m.}
\end{itemize}
Tecido de cascas de árvores e seda, fabricado na Índia.
\section{Nilométrico}
\begin{itemize}
\item {Grp. gram.:adj.}
\end{itemize}
Relativo ao nilómetro.
\section{Nilómetro}
\begin{itemize}
\item {Grp. gram.:m.}
\end{itemize}
\begin{itemize}
\item {Proveniência:(De \textunderscore Nilo\textunderscore , n. p. + gr. \textunderscore metron\textunderscore )}
\end{itemize}
Columna graduada, para medir a altura das cheias periódicas do Nilo.
\section{Nilótico}
\begin{itemize}
\item {Grp. gram.:adj.}
\end{itemize}
Relativo ao Nilo ou aos povos das margens do Nilo.
Nílico; nilíaco. Cf. \textunderscore Lusíadas\textunderscore , IV, 62.
\section{Nilsónia}
\begin{itemize}
\item {Grp. gram.:f.}
\end{itemize}
\begin{itemize}
\item {Proveniência:(De \textunderscore Nilson\textunderscore , n. p.)}
\end{itemize}
Gênero de plantas fósseis.
\section{Nim}
\begin{itemize}
\item {Grp. gram.:m.}
\end{itemize}
Pano antigo, que se fabricava no Languedoc.
\section{Nimbar}
\begin{itemize}
\item {Grp. gram.:v. t.}
\end{itemize}
\begin{itemize}
\item {Utilização:Neol.}
\end{itemize}
Cercar de nimbo ou de auréola.
\section{Nimbífero}
\begin{itemize}
\item {Grp. gram.:adj.}
\end{itemize}
\begin{itemize}
\item {Utilização:Poét.}
\end{itemize}
\begin{itemize}
\item {Proveniência:(Lat. \textunderscore nimbifer\textunderscore )}
\end{itemize}
Que traz chuva.
\section{Nimbo}
\begin{itemize}
\item {Grp. gram.:m.}
\end{itemize}
\begin{itemize}
\item {Proveniência:(Lat. \textunderscore nimbus\textunderscore )}
\end{itemize}
Chuva ligeira.
Nuvem, que se desfaz em chuva.
Círculo luminoso, que envolve a cabeça das imagens dos santos.
Auréola; resplendor.
\section{Nimbo}
\begin{itemize}
\item {Grp. gram.:m.}
\end{itemize}
\begin{itemize}
\item {Utilização:Bot.}
\end{itemize}
O mesmo que \textunderscore amargoseira\textunderscore .
\section{Nimboso}
\begin{itemize}
\item {Grp. gram.:adj.}
\end{itemize}
Coberto de nimbo; chuvoso.
\section{Nimiamente}
\begin{itemize}
\item {Grp. gram.:adv.}
\end{itemize}
De modo nímio; excessivamente; com exaggêro.
\section{Nimiedade}
\begin{itemize}
\item {Grp. gram.:f.}
\end{itemize}
Qualidade do que é nímio; demasia.
\section{Nímio}
\begin{itemize}
\item {Grp. gram.:adj.}
\end{itemize}
\begin{itemize}
\item {Proveniência:(Lat. \textunderscore nimius\textunderscore )}
\end{itemize}
Excessivo, demasiado.
\section{Nina}
\begin{itemize}
\item {Grp. gram.:f.}
\end{itemize}
O mesmo que \textunderscore arruela\textunderscore .
\section{Nina}
\begin{itemize}
\item {Grp. gram.:f.}
\end{itemize}
(V. \textunderscore nana\textunderscore ^1)
\section{Ninainar}
\begin{itemize}
\item {Grp. gram.:v. i.}
\end{itemize}
\begin{itemize}
\item {Utilização:Prov.}
\end{itemize}
\begin{itemize}
\item {Utilização:trasm.}
\end{itemize}
O mesmo que \textunderscore morangar\textunderscore .
\section{Ninar}
\begin{itemize}
\item {Grp. gram.:v. t.}
\end{itemize}
\begin{itemize}
\item {Utilização:Infant.}
\end{itemize}
\begin{itemize}
\item {Grp. gram.:V. i.}
\end{itemize}
\begin{itemize}
\item {Grp. gram.:Loc.}
\end{itemize}
\begin{itemize}
\item {Utilização:fam.}
\end{itemize}
\begin{itemize}
\item {Proveniência:(De \textunderscore nina\textunderscore ^2)}
\end{itemize}
Acalentar.
Dormir.
\textunderscore Estar ninando\textunderscore , estar na tinta, não fazer caso.
\section{Ningamecha}
\begin{itemize}
\item {Grp. gram.:m.}
\end{itemize}
Funccionário superior no Monomotapá, de funcções semelhantes ás do grão-vizir.
\section{Ningi}
\begin{itemize}
\item {Grp. gram.:m.}
\end{itemize}
Certa raíz, de que na África os Pretos fazem uma espécie de cerveja.
\section{Ningresningres}
\begin{itemize}
\item {Grp. gram.:m.}
\end{itemize}
\begin{itemize}
\item {Proveniência:(Do rad. de \textunderscore ninguém\textunderscore )}
\end{itemize}
Indivíduo acanhado, inhenho.
João-ninguém.
\section{Ningrimanço}
\begin{itemize}
\item {Grp. gram.:m.}
\end{itemize}
Instrumento, com que se lavram as marinhas.
\section{Ninguém}
\begin{itemize}
\item {Grp. gram.:pron.}
\end{itemize}
\begin{itemize}
\item {Proveniência:(Do lat. \textunderscore nec\textunderscore  + \textunderscore quem\textunderscore )}
\end{itemize}
Nenhuma pessôa.
\section{Ningum}
\begin{itemize}
\item {Grp. gram.:pron.}
\end{itemize}
\begin{itemize}
\item {Utilização:Prov.}
\end{itemize}
\begin{itemize}
\item {Utilização:minh.}
\end{itemize}
\begin{itemize}
\item {Proveniência:(Do lat. \textunderscore nec\textunderscore  + \textunderscore unus\textunderscore )}
\end{itemize}
O mesmo que \textunderscore nenhum\textunderscore .
\section{Nifão}
\begin{itemize}
\item {Grp. gram.:m.}
\end{itemize}
Gênero de peixes acantopterígios.
\section{Ninhada}
\begin{itemize}
\item {Grp. gram.:f.}
\end{itemize}
\begin{itemize}
\item {Utilização:Fam.}
\end{itemize}
\begin{itemize}
\item {Utilização:Fig.}
\end{itemize}
\begin{itemize}
\item {Proveniência:(De \textunderscore ninho\textunderscore )}
\end{itemize}
Ovos ou avezinhas, contidas num ninho.
Filhos de um só parto da fêmea de um animal.
Viveiro.
Valhacoito.
Porção de filhos pequenos, filharada.
Reunião de pessôas mal intencionadas.
\section{Ninhar}
\begin{itemize}
\item {Grp. gram.:v. i.}
\end{itemize}
Fazer ninho. Cf. Dom. Vieira, vb. \textunderscore encasar-se\textunderscore .
\section{Ninharia}
\begin{itemize}
\item {Grp. gram.:f.}
\end{itemize}
\begin{itemize}
\item {Proveniência:(Do cast. \textunderscore niñeria\textunderscore , criancice)}
\end{itemize}
Insignificância; nica.
\section{Ninhego}
\begin{itemize}
\item {fónica:nhê}
\end{itemize}
\begin{itemize}
\item {Grp. gram.:adj.}
\end{itemize}
Que foi tirado do ninho.
\section{Ninheiro}
\begin{itemize}
\item {Grp. gram.:m.}
\end{itemize}
\begin{itemize}
\item {Utilização:Prov.}
\end{itemize}
\begin{itemize}
\item {Utilização:trasm.}
\end{itemize}
\begin{itemize}
\item {Grp. gram.:M.}
\end{itemize}
\begin{itemize}
\item {Utilização:Prov.}
\end{itemize}
\begin{itemize}
\item {Utilização:trasm.}
\end{itemize}
\begin{itemize}
\item {Proveniência:(De \textunderscore ninho\textunderscore )}
\end{itemize}
Lugar, onde as gallinhas põem habitualmente ovos.
Monte, grande porção.
\section{Ninheiro}
\begin{itemize}
\item {Grp. gram.:m.  e  adj.}
\end{itemize}
\begin{itemize}
\item {Utilização:Prov.}
\end{itemize}
\begin{itemize}
\item {Utilização:minh.}
\end{itemize}
Homem que se occupa muito de ninharias.
(Cp. \textunderscore ninharia\textunderscore )
\section{Ninheria}
\begin{itemize}
\item {Grp. gram.:f.}
\end{itemize}
\begin{itemize}
\item {Utilização:Des.}
\end{itemize}
O mesmo que \textunderscore ninharia\textunderscore . Cf. B. Pereira, \textunderscore Prosódia\textunderscore , vb. \textunderscore nugae\textunderscore .
\section{Ninho}
\begin{itemize}
\item {Grp. gram.:m.}
\end{itemize}
\begin{itemize}
\item {Proveniência:(Do lat. \textunderscore nidus\textunderscore ?)}
\end{itemize}
Pequena habitação das aves, por ellas construída para a postura dos ovos e procriação dos filhos.
Lugar, em que se recolhem ou dormem os animais.
Toca.
Abrigo, retiro, esconderijo.
Amparo, confôrto.
Refúgio.
Covil.
Pátria.
Casa paterna.
\section{Nini}
\begin{itemize}
\item {Grp. gram.:m.  e  f.}
\end{itemize}
\begin{itemize}
\item {Utilização:Infant.}
\end{itemize}
Menino ou menina.
\section{Ninivita}
\begin{itemize}
\item {Grp. gram.:adj.}
\end{itemize}
\begin{itemize}
\item {Grp. gram.:M.}
\end{itemize}
Relativo a Nínive.
Habitante de Nínive.
Systema de escrita cuneiforme, usada em Nínive.
\section{Nino}
\begin{itemize}
\item {Grp. gram.:m.}
\end{itemize}
\begin{itemize}
\item {Utilização:ant.}
\end{itemize}
\begin{itemize}
\item {Utilização:Pop.}
\end{itemize}
O mesmo que \textunderscore menino\textunderscore :«\textunderscore ...mas era nino chiquito, para mais não entendia.\textunderscore »Alv. Azevedo, \textunderscore Cancion. da Mad.\textunderscore , 120.
(Cp. cast. \textunderscore niño\textunderscore )
\section{Ninocha}
\begin{itemize}
\item {Grp. gram.:m.  e  f.}
\end{itemize}
\begin{itemize}
\item {Utilização:Prov.}
\end{itemize}
Pessôa acanhada.
Pessôa sonsa.
(Cp. \textunderscore nino\textunderscore )
\section{Nínsia}
\begin{itemize}
\item {Grp. gram.:m.}
\end{itemize}
Sacerdote japonês, a que servem de guarda numerosos nobres.
\section{Niobato}
\begin{itemize}
\item {Grp. gram.:m.}
\end{itemize}
Sal oxygenado de nióbio.
\section{Nióbico}
\begin{itemize}
\item {Grp. gram.:adj.}
\end{itemize}
Relativo ao nióbio.
Feito dêste metal.
\section{Nióbio}
\begin{itemize}
\item {Grp. gram.:m.}
\end{itemize}
Novo metal, descoberto por H. Rose, em 1844.
\section{Nipa}
\begin{itemize}
\item {Grp. gram.:f.}
\end{itemize}
Gênero de árvores da Ásia meridional, de cujo fruto se extrái bebida agradável.
(Do mal.)
\section{Nipáceas}
\begin{itemize}
\item {Grp. gram.:f. pl.}
\end{itemize}
\begin{itemize}
\item {Proveniência:(De \textunderscore nipáceo\textunderscore )}
\end{itemize}
Família de plantas, que tem por typo a nipa.
\section{Nipáceo}
\begin{itemize}
\item {Grp. gram.:adj.}
\end{itemize}
Relativo ou semelhante á nipa.
\section{Níparo}
\begin{itemize}
\item {Grp. gram.:m.}
\end{itemize}
\begin{itemize}
\item {Utilização:Prov.}
\end{itemize}
\begin{itemize}
\item {Utilização:alent.}
\end{itemize}
O mesmo que \textunderscore testículo\textunderscore .
\section{Nipeira}
\begin{itemize}
\item {Grp. gram.:f.}
\end{itemize}
O mesmo que \textunderscore nipa\textunderscore , árvore.
\section{Niphão}
\begin{itemize}
\item {Grp. gram.:m.}
\end{itemize}
Gênero de peixes acanthopterýgios.
\section{Nipónico}
\begin{itemize}
\item {Grp. gram.:adj.}
\end{itemize}
\begin{itemize}
\item {Utilização:Neol.}
\end{itemize}
Relativo a Nippon, designação indígena do Japão. Cf. Venceslau de Moraes, \textunderscore Dai-Nippon\textunderscore .
\section{Nippónico}
\begin{itemize}
\item {Grp. gram.:adj.}
\end{itemize}
\begin{itemize}
\item {Utilização:Neol.}
\end{itemize}
Relativo a Nippon, designação indígena do Japão. Cf. Venceslau de Moraes, \textunderscore Dai-Nippon\textunderscore .
\section{Nique}
\begin{itemize}
\item {Grp. gram.:m.}
\end{itemize}
\begin{itemize}
\item {Utilização:Açor}
\end{itemize}
Furo, feito num pião com a ferreta de outro.
(Cp. \textunderscore nicar\textunderscore )
\section{Níquel}
\begin{itemize}
\item {Grp. gram.:m.}
\end{itemize}
Metal de côr intermédia á da prata e á do estanho.
(Sueco \textunderscore nickel\textunderscore )
\section{Niquelagem}
\begin{itemize}
\item {Grp. gram.:f.}
\end{itemize}
Acto de niquelar.
\section{Niquelar}
\begin{itemize}
\item {Grp. gram.:v. t.}
\end{itemize}
Misturar ou cobrir de níquel.
Dar apparência de níquel a.
\section{Niquelífero}
\begin{itemize}
\item {Grp. gram.:adj.}
\end{itemize}
\begin{itemize}
\item {Proveniência:(De \textunderscore níquel\textunderscore  + lat. \textunderscore ferre\textunderscore )}
\end{itemize}
Que contém níquel.
\section{Niquelina}
\begin{itemize}
\item {Grp. gram.:f.}
\end{itemize}
Principal minério do níquel.
Mineral, em cuja composição entra, além do níquel, o arsênio, o antimónio, o cobalto, etc.
\section{Niquelite}
\begin{itemize}
\item {Grp. gram.:f.}
\end{itemize}
Arseniato de níquel natural.
Niquelina.
\section{Niquento}
\begin{itemize}
\item {Grp. gram.:adj.}
\end{itemize}
\begin{itemize}
\item {Proveniência:(De \textunderscore nica\textunderscore )}
\end{itemize}
Que se occupa em ninharias; impertinente.
\section{Niquice}
\begin{itemize}
\item {Grp. gram.:f.}
\end{itemize}
\begin{itemize}
\item {Proveniência:(De \textunderscore nica\textunderscore )}
\end{itemize}
Qualidade de niquento.
Ninharia, bagatela.
\section{Niquim}
\begin{itemize}
\item {Grp. gram.:m.}
\end{itemize}
\begin{itemize}
\item {Utilização:Bras}
\end{itemize}
Peixe acanthopterýgio, (\textunderscore thalassophryna maculata\textunderscore ), de espinhos venenosos no dorso.
\section{Nisca}
\begin{itemize}
\item {Grp. gram.:f.}
\end{itemize}
\begin{itemize}
\item {Utilização:Prov.}
\end{itemize}
Pequena ave de arribação.
\section{Níscaro}
\begin{itemize}
\item {Grp. gram.:m.}
\end{itemize}
\begin{itemize}
\item {Utilização:Prov.}
\end{itemize}
\begin{itemize}
\item {Utilização:trasm.}
\end{itemize}
O mesmo que \textunderscore míscaro\textunderscore .
\section{Niscato}
\begin{itemize}
\item {Grp. gram.:m.}
\end{itemize}
\begin{itemize}
\item {Utilização:T. da Bairrada}
\end{itemize}
O mesmo que \textunderscore biscato\textunderscore .
\section{Nisco}
\begin{itemize}
\item {Grp. gram.:m.}
\end{itemize}
\begin{itemize}
\item {Utilização:Prov.}
\end{itemize}
\begin{itemize}
\item {Utilização:minh.}
\end{itemize}
O mesmo que \textunderscore míscaro\textunderscore .
\section{Níspero}
\begin{itemize}
\item {Grp. gram.:m.}
\end{itemize}
\begin{itemize}
\item {Utilização:Prov.}
\end{itemize}
\begin{itemize}
\item {Utilização:dur.}
\end{itemize}
Carne que, depois de cozida, conserva o aspecto de um músculo contrahido e apresenta camadas entremeadas de pedaços de gelatina. (Colhido no Pôrto)
\section{Nísquinho}
\begin{itemize}
\item {Grp. gram.:m.}
\end{itemize}
\begin{itemize}
\item {Utilização:Prov.}
\end{itemize}
O mesmo que \textunderscore nisquito\textunderscore .
\section{Nisquito}
\begin{itemize}
\item {Grp. gram.:m.}
\end{itemize}
\begin{itemize}
\item {Utilização:Prov.}
\end{itemize}
O mesmo que \textunderscore niscato\textunderscore .
\section{Nisso}
Expressão contrahida, equivalente a \textunderscore em isso\textunderscore .
(Cp. \textunderscore no\textunderscore ^1)
\section{Nissólia}
\begin{itemize}
\item {Grp. gram.:f.}
\end{itemize}
Gênero de plantas papilionáceas.
\section{Nisto}
Expressão contrahida, equivalente a \textunderscore em isto\textunderscore .
(Cp. \textunderscore no\textunderscore ^1)
\section{Nitela}
\begin{itemize}
\item {Grp. gram.:f.}
\end{itemize}
Gênero de insectos hymenópteros.
\section{Nitente}
\begin{itemize}
\item {Grp. gram.:adj.}
\end{itemize}
\begin{itemize}
\item {Proveniência:(Lat. \textunderscore nitens\textunderscore )}
\end{itemize}
Que se esforça; resistente.
\section{Nitente}
\begin{itemize}
\item {Grp. gram.:adj.}
\end{itemize}
\begin{itemize}
\item {Proveniência:(Lat. \textunderscore nitens\textunderscore )}
\end{itemize}
Que resplandece; nítido.
\section{Nitescência}
\begin{itemize}
\item {Grp. gram.:f.}
\end{itemize}
\begin{itemize}
\item {Utilização:Neol.}
\end{itemize}
\begin{itemize}
\item {Proveniência:(Do lat. \textunderscore nitescens\textunderscore )}
\end{itemize}
Brilho, esplendor.
\section{Nitidamente}
\begin{itemize}
\item {Grp. gram.:adv.}
\end{itemize}
De modo nítido; claramente; brilhantemente.
\section{Nitidez}
\begin{itemize}
\item {Grp. gram.:f.}
\end{itemize}
\begin{itemize}
\item {Utilização:Fig.}
\end{itemize}
Qualidade de nítido.
Brilho; fulgor.
Candidez.
\section{Nitideza}
\begin{itemize}
\item {Grp. gram.:f.}
\end{itemize}
O mesmo que \textunderscore nitidez\textunderscore .
\section{Nitidifloro}
\begin{itemize}
\item {Grp. gram.:adj.}
\end{itemize}
\begin{itemize}
\item {Utilização:Bot.}
\end{itemize}
\begin{itemize}
\item {Proveniência:(Do lat. \textunderscore nitidus\textunderscore  + \textunderscore flos\textunderscore )}
\end{itemize}
Que tem flôres brilhantes.
\section{Nítido}
\begin{itemize}
\item {Grp. gram.:adj.}
\end{itemize}
\begin{itemize}
\item {Proveniência:(Lat. \textunderscore nitidus\textunderscore )}
\end{itemize}
Que brilha; brilhante; fulgente.
Límpido.
Polido.
Limpo.
\section{Nitídula}
\begin{itemize}
\item {Grp. gram.:f.}
\end{itemize}
Gênero de insectos coleópteros pentâmeros.
(Dem. do lat. \textunderscore nitida\textunderscore , de \textunderscore nitidus\textunderscore )
\section{Nitidular}
\begin{itemize}
\item {Grp. gram.:adj.}
\end{itemize}
\begin{itemize}
\item {Grp. gram.:M. pl.}
\end{itemize}
Relativo ou semelhante á \textunderscore nitídula\textunderscore .
Tríbo de insectos, que tem por typo a nitídula.
\section{Nitidúlios}
\begin{itemize}
\item {Grp. gram.:m. pl.}
\end{itemize}
\begin{itemize}
\item {Proveniência:(De \textunderscore nitídula\textunderscore )}
\end{itemize}
Família de insectos coleópteros, que contém cêrca de 400 espécies de todos os pontos do globo.
\section{Nitídulos}
\begin{itemize}
\item {Grp. gram.:m. pl.}
\end{itemize}
\begin{itemize}
\item {Proveniência:(De \textunderscore nitídula\textunderscore )}
\end{itemize}
Família de insectos coleópteros, que contém cêrca de 400 espécies de todos os pontos do globo.
\section{Nítoes}
\begin{itemize}
\item {Grp. gram.:m.}
\end{itemize}
Demónio ou génio que, nos negócios graves, é consultado como oráculo pelos habitantes das Molucas.
\section{Nitrado}
\begin{itemize}
\item {Grp. gram.:adj.}
\end{itemize}
Que contém nitro.
\section{Nitragina}
\begin{itemize}
\item {Grp. gram.:f.}
\end{itemize}
Substância, composta de fermentos vegetaes, destinada ao adubo das plantas leguminosas.
\section{Nitral}
\begin{itemize}
\item {Grp. gram.:m.}
\end{itemize}
O mesmo que \textunderscore nitreira\textunderscore .
\section{Nitranilato}
\begin{itemize}
\item {Grp. gram.:m.}
\end{itemize}
\begin{itemize}
\item {Utilização:Chím.}
\end{itemize}
Sal, formado pela combinação do ácido nitranílico com uma base.
(Cp. \textunderscore nitranílico\textunderscore )
\section{Nitranílico}
\begin{itemize}
\item {Grp. gram.:adj.}
\end{itemize}
Diz-se de um ácido, produzido pela acção do ácido nítrico sôbre o índigo.
(Do \textunderscore nitro\textunderscore  + \textunderscore anil\textunderscore )
\section{Nitrária}
\begin{itemize}
\item {Grp. gram.:f.}
\end{itemize}
Gênero de plantas.
\section{Nitrariáceas}
\begin{itemize}
\item {Grp. gram.:f. pl.}
\end{itemize}
Pequena família de plantas, que tem por typo a nitrária.
\section{Nitratado}
\begin{itemize}
\item {Grp. gram.:adj.}
\end{itemize}
\begin{itemize}
\item {Utilização:Chím.}
\end{itemize}
Convertido em nitrato.
\section{Nitratina}
\begin{itemize}
\item {Grp. gram.:f.}
\end{itemize}
Azotato de nitro, também conhecido por \textunderscore salitre do Chile\textunderscore .
\section{Nitratite}
\begin{itemize}
\item {Grp. gram.:f.}
\end{itemize}
\begin{itemize}
\item {Proveniência:(De \textunderscore nitro\textunderscore )}
\end{itemize}
Classe dos étheres nítricos explosivos, a que pertence a nitro-glycerina.
\section{Nitrato}
\begin{itemize}
\item {Grp. gram.:m.}
\end{itemize}
\begin{itemize}
\item {Proveniência:(Lat. \textunderscore nitratus\textunderscore )}
\end{itemize}
Sal, formado pela combinação do ácido nítrico com uma base.
\section{Nitreira}
\begin{itemize}
\item {Grp. gram.:f.}
\end{itemize}
\begin{itemize}
\item {Proveniência:(De \textunderscore nitro\textunderscore )}
\end{itemize}
Lugar, onde se fórma o nitro.
Cisterna, destinada a receber os liquidos que escorrem dos estábulos ou das montureiras.
\section{Nítrico}
\begin{itemize}
\item {Grp. gram.:adj.}
\end{itemize}
\begin{itemize}
\item {Proveniência:(De \textunderscore nitro\textunderscore )}
\end{itemize}
Diz-se do ácido, formado de dois equivalentes de azoto e cinco de oxigênio, e a que hoje se chama \textunderscore azótico\textunderscore .
\section{Nitrido}
\begin{itemize}
\item {Grp. gram.:m.}
\end{itemize}
Acto de nitrir.
Rincho.
\section{Nitridor}
\begin{itemize}
\item {Grp. gram.:adj.}
\end{itemize}
\begin{itemize}
\item {Grp. gram.:M.}
\end{itemize}
Que nitre ou que rincha.
Animal que rincha.
\section{Nitrificação}
\begin{itemize}
\item {Grp. gram.:f.}
\end{itemize}
Acto ou effeito de \textunderscore nitrificar\textunderscore .
\section{Nitrificar}
\begin{itemize}
\item {Grp. gram.:v. t.}
\end{itemize}
\begin{itemize}
\item {Proveniência:(Do lat. \textunderscore nitrum\textunderscore  + \textunderscore facere\textunderscore )}
\end{itemize}
Transformar em nitrato.
Cobrir de nitro.
\section{Nitrínico}
\begin{itemize}
\item {Grp. gram.:adj.}
\end{itemize}
Diz-se do ácido, produzido pela acção da potassa ou da soda sôbre o éther oxálico.
\section{Nitrir}
\begin{itemize}
\item {Grp. gram.:v. i.}
\end{itemize}
\begin{itemize}
\item {Grp. gram.:M.}
\end{itemize}
\begin{itemize}
\item {Proveniência:(It. \textunderscore nitrire\textunderscore )}
\end{itemize}
Rinchar.
Acto de rinchar.
\section{Nitrito}
\begin{itemize}
\item {Grp. gram.:m.}
\end{itemize}
\begin{itemize}
\item {Utilização:Chím.}
\end{itemize}
Sal, produzido pela combinação do ácido nitroso com uma base.
O mesmo que \textunderscore azotito\textunderscore .
\section{Nitro}
\begin{itemize}
\item {Grp. gram.:m.}
\end{itemize}
\begin{itemize}
\item {Proveniência:(Lat. \textunderscore nitrum\textunderscore )}
\end{itemize}
Designação vulgar do nitrato de potassa e do azotato de potassa; salitre.
\section{Nitro-amido}
\begin{itemize}
\item {Grp. gram.:m.}
\end{itemize}
Explosivo, o mesmo que \textunderscore xyloidina\textunderscore .
\section{Nitrobarito}
\begin{itemize}
\item {Grp. gram.:m.}
\end{itemize}
\begin{itemize}
\item {Utilização:Miner.}
\end{itemize}
Nitrato de bário.
\section{Nitrobaryto}
\begin{itemize}
\item {Grp. gram.:m.}
\end{itemize}
\begin{itemize}
\item {Utilização:Miner.}
\end{itemize}
Nitrato de báryo.
\section{Nitro-benzina}
\begin{itemize}
\item {Grp. gram.:f.}
\end{itemize}
Combinação de ácido nítrico e benzina.
\section{Nitro-carboneto}
\begin{itemize}
\item {Grp. gram.:m.}
\end{itemize}
Substância explosiva e derivada dos carbonatos.
\section{Nitro-cellulose}
\begin{itemize}
\item {Grp. gram.:f.}
\end{itemize}
Explosivo, o mesmo que \textunderscore pyroxylina\textunderscore .
\section{Nitrofórmio}
\begin{itemize}
\item {Grp. gram.:m.}
\end{itemize}
\begin{itemize}
\item {Utilização:Chím.}
\end{itemize}
Corpo incolor, solidificável sob temperatura baixa, e crystallizável em cubos, que se solvem na água, tornando-a amarela.
(Cp. \textunderscore chlorofórmio\textunderscore , \textunderscore iodofórmio\textunderscore , etc.)
\section{Nitrogênio}
\begin{itemize}
\item {Grp. gram.:m.}
\end{itemize}
\begin{itemize}
\item {Proveniência:(Do gr. \textunderscore nitron\textunderscore  + \textunderscore genos\textunderscore )}
\end{itemize}
Nome, que se propôs para designar o azoto.
\section{Nitro-glycerina}
\begin{itemize}
\item {Grp. gram.:f.}
\end{itemize}
Substância liquida, cuja combustão produz explosão muito violenta.
\section{Nitro-glycol}
\begin{itemize}
\item {Grp. gram.:m.}
\end{itemize}
Éther nítrico dos glycoes.
\section{Nitro-hydrocellulose}
\begin{itemize}
\item {Grp. gram.:f.}
\end{itemize}
Substância explosiva.
\section{Nitro-hydrochlórico}
\begin{itemize}
\item {Grp. gram.:adj.}
\end{itemize}
\begin{itemize}
\item {Utilização:Chím.}
\end{itemize}
Diz-se de um ácido, que é a mistura do ácido nítrico com o ácido chlorhýdrico, e que dissolve o oiro e a platina.
\section{Nitromagnesito}
\begin{itemize}
\item {Grp. gram.:m.}
\end{itemize}
\begin{itemize}
\item {Utilização:Miner.}
\end{itemize}
Azotato hydratado de magnésio.
\section{Nitro-mannita}
\begin{itemize}
\item {Grp. gram.:f.}
\end{itemize}
Substância explosiva da mannita.
\section{Nitro-mannite}
\begin{itemize}
\item {Grp. gram.:f.}
\end{itemize}
Substância explosiva da mannita.
\section{Nitro-mannitana}
\begin{itemize}
\item {Grp. gram.:f.}
\end{itemize}
Explosivo extrahido da mannitana.
\section{Nitro-methane}
\begin{itemize}
\item {Grp. gram.:m.}
\end{itemize}
O mesmo que nitro-carboneto.
\section{Nitrómetro}
\begin{itemize}
\item {Grp. gram.:m.}
\end{itemize}
\begin{itemize}
\item {Proveniência:(Do gr. \textunderscore nitron\textunderscore  + \textunderscore metron\textunderscore )}
\end{itemize}
Instrumento, para se experimentar o salitre do commércio.
\section{Nitromuriático}
\begin{itemize}
\item {Grp. gram.:adj.}
\end{itemize}
\begin{itemize}
\item {Utilização:Chím.}
\end{itemize}
O mesmo que \textunderscore nitro-hydrochlórico\textunderscore .
\section{Nitromuriato}
\begin{itemize}
\item {Grp. gram.:m.}
\end{itemize}
\begin{itemize}
\item {Utilização:Chím.}
\end{itemize}
Sal, formado pela combinação do ácido nitromuriático com uma base.
\section{Nitronita}
\begin{itemize}
\item {Grp. gram.:f.}
\end{itemize}
\begin{itemize}
\item {Proveniência:(Do gr. \textunderscore nitron\textunderscore )}
\end{itemize}
Mineral, que serve para a preparação do ácido nítrico, e constitue um nitrato de sódio.
\section{Nitro-saccharato}
\begin{itemize}
\item {Grp. gram.:m.}
\end{itemize}
\begin{itemize}
\item {Utilização:Chím.}
\end{itemize}
Sal, formado pela acção do ácido nitro-sacchárico sôbre uma base salificável.
\section{Nitro-sacchárico}
\begin{itemize}
\item {Grp. gram.:adj.}
\end{itemize}
Diz-se de um ácido, formado pela acção do ácido nitroso sôbre o açúcar de gelatina.
\section{Nitro-saccharose}
\begin{itemize}
\item {Grp. gram.:f.}
\end{itemize}
Substância explosiva, obtida pela acção do ácido nítrico sôbre o açúcar.
\section{Nitrose}
\begin{itemize}
\item {Grp. gram.:f.}
\end{itemize}
\begin{itemize}
\item {Proveniência:(De \textunderscore nitro\textunderscore )}
\end{itemize}
Composição explosiva de aldehydos mixtos.
\section{Nitrosidade}
\begin{itemize}
\item {Grp. gram.:f.}
\end{itemize}
Qualidade de nitroso ou daquillo que contém nitro.
\section{Nitroso}
\begin{itemize}
\item {Grp. gram.:adj.}
\end{itemize}
\begin{itemize}
\item {Proveniência:(Lat. \textunderscore nitrosus\textunderscore )}
\end{itemize}
O mesmo que \textunderscore nitrado\textunderscore ; salitroso.
\section{Nivator}
\begin{itemize}
\item {Grp. gram.:m.}
\end{itemize}
Espécie de faisão da Índia. Cp. \textunderscore Peregrinação\textunderscore , LXXXIII.
\section{Niveal}
\begin{itemize}
\item {Grp. gram.:adj.}
\end{itemize}
\begin{itemize}
\item {Proveniência:(De \textunderscore níveo\textunderscore )}
\end{itemize}
Que vive em a neve.
Relativo ao inverno.
\section{Nível}
\begin{itemize}
\item {Grp. gram.:m.}
\end{itemize}
\begin{itemize}
\item {Utilização:Fig.}
\end{itemize}
\begin{itemize}
\item {Proveniência:(Do fr. \textunderscore niveau\textunderscore )}
\end{itemize}
Instrumento, para verificar se um plano está horizontal.
Horizontalidade.
Igualha, igualdade.
Altura; situação.
\textunderscore Curva de nível\textunderscore , secção de terreno por um plano horizontal.
(O povo do norte diz \textunderscore nivél\textunderscore  e \textunderscore livél\textunderscore ; e libél é a pronúncia e fórma lidimanente portuguesa, do lat. \textunderscore libellum\textunderscore .)
\section{Nivél}
\begin{itemize}
\item {Grp. gram.:m.}
\end{itemize}
\begin{itemize}
\item {Utilização:Fig.}
\end{itemize}
Instrumento, para verificar se um plano está horizontal.
Horizontalidade.
Igualha, igualdade.
Altura; situação.
* \textunderscore Curva de nível\textunderscore , secção de terreno por um plano horizontal.
(O povo do norte diz \textunderscore nivél\textunderscore  e \textunderscore livél\textunderscore ; e libél é a pronúncia e fórma lidimanente portuguesa, do lat. \textunderscore libellum\textunderscore . Do fr. \textunderscore niveau\textunderscore )
\section{Nivelador}
\begin{itemize}
\item {Grp. gram.:m.  e  adj.}
\end{itemize}
O que nivela.
\section{Nivelamento}
\begin{itemize}
\item {Grp. gram.:m.}
\end{itemize}
Acto ou effeito de nivelar.
\section{Nivelar}
\begin{itemize}
\item {Grp. gram.:v. t.}
\end{itemize}
\begin{itemize}
\item {Utilização:Fig.}
\end{itemize}
Tornar horizontal.
Collocar na mesma plana.
Igualar.
Acalmar.
Medir com o nivel.
Destruír; arrasar.
\section{Nivênia}
\begin{itemize}
\item {Grp. gram.:f.}
\end{itemize}
Gênero de plantas proteáceas.
\section{Níveo}
\begin{itemize}
\item {Grp. gram.:adj.}
\end{itemize}
\begin{itemize}
\item {Proveniência:(Lat. \textunderscore niveus\textunderscore )}
\end{itemize}
Relativo á neve; muito branco: \textunderscore o níveo collo\textunderscore .
\section{Nivéola}
\begin{itemize}
\item {Grp. gram.:f.}
\end{itemize}
Gênero de plantas amaryllídeas, (\textunderscore nivea viola\textunderscore ).
\section{Nivoso}
\begin{itemize}
\item {Grp. gram.:m.}
\end{itemize}
\begin{itemize}
\item {Grp. gram.:Adj.}
\end{itemize}
\begin{itemize}
\item {Utilização:Poét.}
\end{itemize}
\begin{itemize}
\item {Proveniência:(Lat. \textunderscore nivosus\textunderscore )}
\end{itemize}
Quarto mês do calendário da primeira República francesa.
Coberto de neve; em que há neve; nevoso.
\section{Nixos}
\begin{itemize}
\item {Grp. gram.:m.}
\end{itemize}
Nome de uma constellação.
\section{Niza}
\begin{itemize}
\item {Grp. gram.:f.}
\end{itemize}
\begin{itemize}
\item {Utilização:Prov.}
\end{itemize}
Espécie de casaco curto.
Jaquetão ordinário, geralmente de saragoça. Cf. Camillo, \textunderscore Brasileira\textunderscore , 62.
(Talvez do turc. \textunderscore nízan\textunderscore , soldado de 1.^a linha, por allusão ao casaco curto, usado por tal milícia)
\section{Nízaro}
\begin{itemize}
\item {Grp. gram.:m.}
\end{itemize}
\begin{itemize}
\item {Utilização:Prov.}
\end{itemize}
\begin{itemize}
\item {Utilização:minh.}
\end{itemize}
O mesmo que \textunderscore niza\textunderscore .
(Cp. \textunderscore nízera\textunderscore )
\section{Nízera}
\begin{itemize}
\item {Grp. gram.:f.}
\end{itemize}
Espécie de casaca curta, de botões amarelos, usada no século passado pelos lavradores de Refóios do Lima.
(Cp. \textunderscore niza\textunderscore )
\section{N. N. E.}
\begin{itemize}
\item {Grp. gram.:m.}
\end{itemize}
Abrev. de \textunderscore Nornordeste\textunderscore .
\section{N. N. O.}
\begin{itemize}
\item {Grp. gram.:m.}
\end{itemize}
Abrev. de \textunderscore Nornoroéste\textunderscore .
\section{No}
\begin{itemize}
\item {fónica:nu}
\end{itemize}
Expressão contrahida, equivalente a \textunderscore em o\textunderscore .
O \textunderscore n\textunderscore  da ant. prep. \textunderscore en\textunderscore  feria naturalmente o artigo \textunderscore o\textunderscore , antepondo-se-lhe e dando a pronúncia \textunderscore ê-nu\textunderscore . Pela quéda da vogal nasalada, ficou simplesmente \textunderscore no\textunderscore .
\section{No}
\begin{itemize}
\item {fónica:nu}
\end{itemize}
\begin{itemize}
\item {Grp. gram.:pron.}
\end{itemize}
\begin{itemize}
\item {Proveniência:(Do pron. \textunderscore lo\textunderscore , que, precedido de sýllaba nasal, se assimilou parcialmente: \textunderscore amam-lo\textunderscore  = \textunderscore amam-no\textunderscore )}
\end{itemize}
O mesmo que \textunderscore o\textunderscore ^2, depois de sýllaba nasalada: \textunderscore fazem-no\textunderscore , \textunderscore dizem-no\textunderscore , \textunderscore bem no vi\textunderscore , \textunderscore não no quero\textunderscore .
\section{No}
\begin{itemize}
\item {fónica:nó?}
\end{itemize}
\begin{itemize}
\item {Grp. gram.:adv.}
\end{itemize}
\begin{itemize}
\item {Utilização:Ant.}
\end{itemize}
O mesmo que \textunderscore não\textunderscore ^1. Cf. Sim. Machado, fol. 15; G. Vicente, \textunderscore Inês Pereira\textunderscore ; etc.
\section{Nó}
\begin{itemize}
\item {Grp. gram.:m.}
\end{itemize}
\begin{itemize}
\item {Utilização:Prov.}
\end{itemize}
\begin{itemize}
\item {Proveniência:(Do lat. \textunderscore nodus\textunderscore )}
\end{itemize}
Laço, feito de corda ou de coisa semelhante, cujas extremidades passam uma pela outra, apertando-se.
Parte mais espêssa e dura na madeira, no mármore, etc.
Milha marítima, ou unidade da distância percorrida por um navio.
Articulação das phalanges dos dedos.
Concreção, nas articulações dos dedos dos gotosos.
Orifício, por onde passa cada um dos fios da urdidura.
O ponto grave, o mais diffícil de um negócio.
Ligação, enlace: \textunderscore os namorados deram o nó\textunderscore .
Pontos de inserção das fôlhas das plantas gramíneas.
Saliência anterior da garganta.
Embaraço.
Enrêdo, intriga.
\textunderscore Nó cego\textunderscore , nó apertado por duas voltas.
\textunderscore Nó de Adão\textunderscore , designação vulgar da saliência do corpo hyóide, a qual comprehende a região onde está aquelle osso e o ângulo anterior e saliente, formado pelas cartilagens larýngeas; (allusão á lenda de que o pomo do peccado ficou a meio da garganta de Adão)
\section{N. O.}
\begin{itemize}
\item {Grp. gram.:m.}
\end{itemize}
Abrev. de \textunderscore Noroéste\textunderscore .
\section{Nôa}
\begin{itemize}
\item {Grp. gram.:f.}
\end{itemize}
\begin{itemize}
\item {Proveniência:(Do lat. \textunderscore nonus\textunderscore , contr. de \textunderscore novenus\textunderscore )}
\end{itemize}
Hora do offício divino, entre as sextas e as vésperas.
\section{Nóbile}
\begin{itemize}
\item {Grp. gram.:adj.}
\end{itemize}
\begin{itemize}
\item {Utilização:Ant.}
\end{itemize}
\begin{itemize}
\item {Proveniência:(Lat. \textunderscore nobilis\textunderscore )}
\end{itemize}
O mesmo que \textunderscore nóbre\textunderscore .
\section{Nobiliarchia}
\begin{itemize}
\item {fónica:qui}
\end{itemize}
\begin{itemize}
\item {Grp. gram.:f.}
\end{itemize}
\begin{itemize}
\item {Proveniência:(Do lat. \textunderscore nobilis\textunderscore  + gr. \textunderscore arkhe\textunderscore )}
\end{itemize}
Tratado das origens e tradições de famílias nobres.
\section{Nobiliárchico}
\begin{itemize}
\item {fónica:qui}
\end{itemize}
\begin{itemize}
\item {Grp. gram.:adj.}
\end{itemize}
Relativo á nobiliarchia.
\section{Nobiliarchista}
\begin{itemize}
\item {fónica:quis}
\end{itemize}
\begin{itemize}
\item {Grp. gram.:m.}
\end{itemize}
O mesmo que \textunderscore nobiliarista\textunderscore .
\section{Nobiliário}
\begin{itemize}
\item {Grp. gram.:adj.}
\end{itemize}
\begin{itemize}
\item {Grp. gram.:M.}
\end{itemize}
\begin{itemize}
\item {Proveniência:(Do lat. \textunderscore nobilis\textunderscore )}
\end{itemize}
Relativo a nobreza.
Registo de famílias nobres; nobiliarchia.
\section{Nobiliarista}
\begin{itemize}
\item {Grp. gram.:m.  e  f.}
\end{itemize}
Pessôa, versada em nobiliários.
\section{Nobiliarquia}
\begin{itemize}
\item {Grp. gram.:f.}
\end{itemize}
\begin{itemize}
\item {Proveniência:(Do lat. \textunderscore nobilis\textunderscore  + gr. \textunderscore arkhe\textunderscore )}
\end{itemize}
Tratado das origens e tradições de famílias nobres.
\section{Nobiliárquico}
\begin{itemize}
\item {Grp. gram.:adj.}
\end{itemize}
Relativo á nobiliarquia.
\section{Nobiliarquista}
\begin{itemize}
\item {Grp. gram.:m.}
\end{itemize}
O mesmo que \textunderscore nobiliarista\textunderscore .
\section{Nobilíssimo}
\begin{itemize}
\item {Grp. gram.:adj.}
\end{itemize}
\begin{itemize}
\item {Proveniência:(Lat. \textunderscore nobilissimus\textunderscore )}
\end{itemize}
Muito nobre.
\section{Nobilitação}
\begin{itemize}
\item {Grp. gram.:f.}
\end{itemize}
Acto ou effeito de nobilitar.
\section{Nobilitante}
\begin{itemize}
\item {Grp. gram.:adj.}
\end{itemize}
\begin{itemize}
\item {Proveniência:(Lat. \textunderscore nobilitans\textunderscore )}
\end{itemize}
Que nobilita.
\section{Nobilitar}
\begin{itemize}
\item {Grp. gram.:v. t.}
\end{itemize}
\begin{itemize}
\item {Proveniência:(Lat. \textunderscore nobilitare\textunderscore )}
\end{itemize}
Tornar nobre.
Dar fóros ou títulos de nobreza a.
Engrandecer; celebrar.
\section{Nóbio}
\begin{itemize}
\item {Grp. gram.:m.}
\end{itemize}
\begin{itemize}
\item {Utilização:Prov.}
\end{itemize}
\begin{itemize}
\item {Utilização:minh.}
\end{itemize}
O mesmo que \textunderscore noivo\textunderscore .
(Cp. cast. \textunderscore novio\textunderscore )
\section{Nóbre}
\begin{itemize}
\item {Grp. gram.:adj.}
\end{itemize}
\begin{itemize}
\item {Grp. gram.:M.}
\end{itemize}
\begin{itemize}
\item {Proveniência:(Do lat. \textunderscore nobilis\textunderscore )}
\end{itemize}
Muito conhecido.
Illustre; notável; célebre.
Que procede de estirpe illustre.
Majestoso.
Generoso; bizarro: \textunderscore nobre procedimento\textunderscore .
Valente.
Elevado, sublime.
Relativo á classe dos que pertencem a raças illustres.
Que recebeu títulos nobiliárchicos.
Indivíduo nobre, por nascimento ou por graça do chefe do Estado.
\section{Nobre}
\begin{itemize}
\item {Grp. gram.:adj.}
\end{itemize}
Diz-se do toiro, o mesmo que \textunderscore boiante\textunderscore .
\section{Nobrecer}
\textunderscore v. t.\textunderscore  e \textunderscore p.\textunderscore  (e der.)
O mesmo que \textunderscore ennobrecer\textunderscore , etc.
\section{Nobremente}
\begin{itemize}
\item {Grp. gram.:adv.}
\end{itemize}
De modo nobre.
Á maneira dos nobres.
Generosamente; com bizarria.
\section{Nobreza}
\begin{itemize}
\item {Grp. gram.:f.}
\end{itemize}
Qualidade de nobre; fidalguia.
Excellência.
Classe dos nobres.
Generosidade; magnanimidade.
Variedade de tecido de seda. Cf. \textunderscore Inquér. Industr.\textunderscore , p. II, l. III, 75.
\section{Noca}
\begin{itemize}
\item {Grp. gram.:f.}
\end{itemize}
\begin{itemize}
\item {Utilização:T. de Caminha}
\end{itemize}
Cada um dos nós dos dedos.
\section{Noção}
\begin{itemize}
\item {Grp. gram.:f.}
\end{itemize}
\begin{itemize}
\item {Proveniência:(Lat. \textunderscore notio\textunderscore )}
\end{itemize}
Conhecimento.
Informação.
Ideia.
Conhecimento elementar; exposição summária: \textunderscore noções de Mathemática\textunderscore .
\section{Nocente}
\begin{itemize}
\item {Grp. gram.:adj.}
\end{itemize}
\begin{itemize}
\item {Proveniência:(Lat. \textunderscore nocens\textunderscore )}
\end{itemize}
Prejudicial, nocivo.
\section{Nocha}
\begin{itemize}
\item {Grp. gram.:f.}
\end{itemize}
Grande árvore rosácea de Angola, (\textunderscore parinarium mobola\textunderscore , Oliver.).
\section{Nochatra}
\begin{itemize}
\item {Grp. gram.:m.}
\end{itemize}
Sal ammoníaco.
\section{Nocilho}
\begin{itemize}
\item {Grp. gram.:f.}
\end{itemize}
\begin{itemize}
\item {Proveniência:(De \textunderscore noz\textunderscore , por infl. do lat. \textunderscore nux\textunderscore , \textunderscore nucis\textunderscore )}
\end{itemize}
Planta verbenácea da Índia portuguesa, (\textunderscore clerodendron inerme\textunderscore , Gaert.).
\section{Nocional}
\begin{itemize}
\item {Grp. gram.:adj.}
\end{itemize}
\begin{itemize}
\item {Proveniência:(Do lat. \textunderscore notio\textunderscore )}
\end{itemize}
Relativo a noção; que tem o carácter de noção.
\section{Nocivamente}
\begin{itemize}
\item {Grp. gram.:adv.}
\end{itemize}
De modo nocivo; com damno; prejudicialmente.
\section{Nocividade}
\begin{itemize}
\item {Grp. gram.:f.}
\end{itemize}
Qualidade de nocivo, de prejudicial.
\section{Nocivo}
\begin{itemize}
\item {Grp. gram.:adj.}
\end{itemize}
\begin{itemize}
\item {Proveniência:(Lat. \textunderscore nocivus\textunderscore )}
\end{itemize}
Que prejudica, que causa damno.
\section{Noctambulação}
\begin{itemize}
\item {Grp. gram.:f.}
\end{itemize}
\begin{itemize}
\item {Proveniência:(Do lat. \textunderscore nox\textunderscore  + \textunderscore ambulare\textunderscore )}
\end{itemize}
Acção de andar de noite; acto de somnâmbulo.
\section{Noctambulismo}
\begin{itemize}
\item {Grp. gram.:m.}
\end{itemize}
Qualidade ou estado de noctâmbulo.
\section{Noctâmbulo}
\begin{itemize}
\item {Grp. gram.:adj.}
\end{itemize}
\begin{itemize}
\item {Grp. gram.:M.}
\end{itemize}
\begin{itemize}
\item {Proveniência:(Do lat. \textunderscore nox\textunderscore , \textunderscore notis\textunderscore  + \textunderscore ambulare\textunderscore )}
\end{itemize}
Que anda de noite; noctívago.
Somnâmbulo.
\section{Nocte}
\begin{itemize}
\item {Grp. gram.:f.}
\end{itemize}
\begin{itemize}
\item {Utilização:Ant.}
\end{itemize}
O mesmo que \textunderscore noite\textunderscore :«\textunderscore e logo naquella nocte da morte do Duque...\textunderscore »R. de Pina, \textunderscore D. João II\textunderscore , c. XVIII.
\section{Nocticolor}
\begin{itemize}
\item {Grp. gram.:adj.}
\end{itemize}
\begin{itemize}
\item {Proveniência:(Lat. \textunderscore nocticolor\textunderscore )}
\end{itemize}
Que é da côr da noite; escuro.
\section{Noctífero}
\begin{itemize}
\item {Grp. gram.:adj.}
\end{itemize}
\begin{itemize}
\item {Utilização:Poét.}
\end{itemize}
\begin{itemize}
\item {Proveniência:(Do lat. \textunderscore nox\textunderscore  + \textunderscore ferre\textunderscore )}
\end{itemize}
O mesmo que \textunderscore noctígeno\textunderscore .
\section{Noctifloro}
\begin{itemize}
\item {Grp. gram.:adj.}
\end{itemize}
\begin{itemize}
\item {Proveniência:(Do lat. \textunderscore nox\textunderscore  + \textunderscore flos\textunderscore )}
\end{itemize}
Diz-se das plantas, cujas flôres se abrem ao anoitecer e se fecham de manhan.
\section{Noctífobo}
\begin{itemize}
\item {Grp. gram.:adj.}
\end{itemize}
\begin{itemize}
\item {Proveniência:(T. hýbr., do lat. \textunderscore nox\textunderscore , \textunderscore noctis\textunderscore  + \textunderscore phobein\textunderscore )}
\end{itemize}
Que tem horror á noite, ás trevas.
\section{Noctífugo}
\begin{itemize}
\item {Grp. gram.:adj.}
\end{itemize}
O mesmo que \textunderscore noctíphobo\textunderscore .
\section{Noctígeno}
\begin{itemize}
\item {Grp. gram.:adj.}
\end{itemize}
\begin{itemize}
\item {Proveniência:(Do lat. \textunderscore nox\textunderscore , \textunderscore noctis\textunderscore  + gr. \textunderscore genea\textunderscore )}
\end{itemize}
Que produz sombras, que espalha trevas.
\section{Noctígero}
\begin{itemize}
\item {Grp. gram.:adj.}
\end{itemize}
\begin{itemize}
\item {Proveniência:(Do lat. \textunderscore nox\textunderscore , \textunderscore noctis\textunderscore  + \textunderscore gerere\textunderscore )}
\end{itemize}
O mesmo ou melhor que \textunderscore noctígeno\textunderscore . Cf. Castilho, \textunderscore Fastos\textunderscore , III, 129.
\section{Noctilião}
\begin{itemize}
\item {Grp. gram.:m.}
\end{itemize}
Gênero de mammíferos chirópteros.
Espécie de morcego.
\section{Noctílio}
\begin{itemize}
\item {Grp. gram.:m.}
\end{itemize}
Gênero de mammíferos chirópteros.
Espécie de morcego.
\section{Noctiluca}
\begin{itemize}
\item {Grp. gram.:f.}
\end{itemize}
\begin{itemize}
\item {Utilização:Poét.}
\end{itemize}
\begin{itemize}
\item {Proveniência:(Do lat. \textunderscore nox\textunderscore , \textunderscore noctis\textunderscore  + \textunderscore lucere\textunderscore )}
\end{itemize}
A lua.
O mesmo que \textunderscore noctiluco\textunderscore .
\section{Noctilúcio}
\begin{itemize}
\item {Grp. gram.:adj.}
\end{itemize}
\begin{itemize}
\item {Proveniência:(Do lat. \textunderscore nox\textunderscore  + \textunderscore lucere\textunderscore )}
\end{itemize}
Diz-se dos corpos, que luzem de noite.
\section{Noctiluco}
\begin{itemize}
\item {Grp. gram.:m.}
\end{itemize}
Protozoário phosphorescente, dos que formam a ardentia marítima.
(Cp. \textunderscore noctiluca\textunderscore )
\section{Noctiluz}
\begin{itemize}
\item {Grp. gram.:m.}
\end{itemize}
\begin{itemize}
\item {Utilização:P. us.}
\end{itemize}
\begin{itemize}
\item {Proveniência:(Do lat. \textunderscore nox\textunderscore  + \textunderscore lux\textunderscore )}
\end{itemize}
O mesmo que \textunderscore pyrilampo\textunderscore .
\section{Noctíphobo}
\begin{itemize}
\item {Grp. gram.:adj.}
\end{itemize}
\begin{itemize}
\item {Proveniência:(T. hýbr., do lat. \textunderscore nox\textunderscore , \textunderscore noctis\textunderscore  + \textunderscore phobein\textunderscore )}
\end{itemize}
Que tem horror á noite, ás trevas.
\section{Noctívago}
\begin{itemize}
\item {Grp. gram.:adj.}
\end{itemize}
\begin{itemize}
\item {Utilização:Poét.}
\end{itemize}
\begin{itemize}
\item {Proveniência:(Lat. \textunderscore noctivagus\textunderscore )}
\end{itemize}
Que anda ou vagueia de noite; nocturno.
\section{Noctívolo}
\begin{itemize}
\item {Grp. gram.:adj.}
\end{itemize}
\begin{itemize}
\item {Proveniência:(Do lat. \textunderscore nox\textunderscore  + \textunderscore volare\textunderscore )}
\end{itemize}
Que vôa de noite.
\section{Nóctua}
\begin{itemize}
\item {Grp. gram.:f.}
\end{itemize}
\begin{itemize}
\item {Proveniência:(Lat. \textunderscore noctua\textunderscore )}
\end{itemize}
Nome, dado por alguns naturalistas a um gênero de aves nocturnas, que tem por typo a coruja.
Gênero de insectos lepidópteros nocturnos.
\section{Noctuelitos}
\begin{itemize}
\item {Grp. gram.:m. pl.}
\end{itemize}
Tríbo de insectos lepidópteros nocturnos.
\section{Nocturlábio}
\begin{itemize}
\item {Grp. gram.:m.}
\end{itemize}
Espécie de relógio antigo, que, pela posição da estrêla do norte, marcava de noite as horas.
(Palavra mal formada, do lat. \textunderscore nox\textunderscore , \textunderscore noctis\textunderscore , e sob a influência de astrolábio)
\section{Nocturnal}
\begin{itemize}
\item {Grp. gram.:adj.}
\end{itemize}
\begin{itemize}
\item {Proveniência:(Lat. \textunderscore nocturnalis\textunderscore )}
\end{itemize}
O mesmo que \textunderscore nocturno\textunderscore .
\section{Nocturno}
\begin{itemize}
\item {Grp. gram.:adj.}
\end{itemize}
\begin{itemize}
\item {Grp. gram.:M.}
\end{itemize}
\begin{itemize}
\item {Grp. gram.:Pl.}
\end{itemize}
\begin{itemize}
\item {Proveniência:(Lat. \textunderscore nocturnus\textunderscore )}
\end{itemize}
Relativo á noite.
Que se realiza de noite: \textunderscore trabalho nocturno\textunderscore .
Que anda ou vagueia de noite.
Uma das partes do officio divino.
Designação de várias composições musicaes, de carácter tranquillo, simples e suave.
Uma das secções das aves de rapina.
Secção de insectos lepidópteros.
\section{Nóda}
\begin{itemize}
\item {Grp. gram.:f.}
\end{itemize}
\begin{itemize}
\item {Utilização:P. us.}
\end{itemize}
\begin{itemize}
\item {Proveniência:(Do lat. \textunderscore nota\textunderscore )}
\end{itemize}
O mesmo que \textunderscore nódoa\textunderscore . Cf. Usque, 50.
\section{Nodal}
\begin{itemize}
\item {Grp. gram.:adj.}
\end{itemize}
\begin{itemize}
\item {Utilização:Phýs.}
\end{itemize}
\begin{itemize}
\item {Proveniência:(Do lat. \textunderscore nodus\textunderscore )}
\end{itemize}
Relativo a nó ou nós.
\textunderscore Linha nodal\textunderscore , linha formada sôbre uma superfície, dividindo esta em duas partes que vibram em sentido opposto.
\section{Nodar}
\begin{itemize}
\item {Grp. gram.:v. t.}
\end{itemize}
\begin{itemize}
\item {Proveniência:(Do lat. \textunderscore nodus\textunderscore )}
\end{itemize}
Segurar com nó. Cf. Macedo, \textunderscore Burros\textunderscore , 242.
\section{Nodicórneo}
\begin{itemize}
\item {Grp. gram.:adj.}
\end{itemize}
\begin{itemize}
\item {Utilização:Zool.}
\end{itemize}
\begin{itemize}
\item {Proveniência:(Do lat. \textunderscore nodus\textunderscore  + \textunderscore cornu\textunderscore )}
\end{itemize}
Que tem antennas nodosas.
\section{Nodifloro}
\begin{itemize}
\item {Grp. gram.:adj.}
\end{itemize}
\begin{itemize}
\item {Proveniência:(Do lat. \textunderscore nodus\textunderscore  + \textunderscore flos\textunderscore , \textunderscore florís\textunderscore )}
\end{itemize}
Diz-se das plantas, cujas flôres nascem dos nós.
\section{Nodo}
\begin{itemize}
\item {Grp. gram.:m.}
\end{itemize}
\begin{itemize}
\item {Utilização:Ext.}
\end{itemize}
\begin{itemize}
\item {Proveniência:(Lat. \textunderscore nodus\textunderscore )}
\end{itemize}
Ponto de intersecção da eclíptica com a órbita de um planeta.
Tumor duro, que se fórma em volta das articulações dos ossos.
Parte proeminente de certos ossos.
\section{Nódoa}
\begin{itemize}
\item {Grp. gram.:f.}
\end{itemize}
\begin{itemize}
\item {Utilização:Fig.}
\end{itemize}
\begin{itemize}
\item {Proveniência:(Do lat. \textunderscore notula\textunderscore )}
\end{itemize}
Vestígio de um corpo ou substância suja.
Mancha.
Mancha na pelle, em resultado de contusão.
Mácula.
Deslustre.
Ignomínia; afronta.
\section{Nodoante}
\begin{itemize}
\item {Grp. gram.:adj.}
\end{itemize}
Que nodôa.
\section{Nodoar}
\begin{itemize}
\item {Grp. gram.:v. t.}
\end{itemize}
O mesmo que \textunderscore ennodoar\textunderscore .
\section{Nódio}
\begin{itemize}
\item {Grp. gram.:m.}
\end{itemize}
\begin{itemize}
\item {Proveniência:(De \textunderscore Nodim\textunderscore , n. p. do descobridor)}
\end{itemize}
Metal novo, semelhante ao aluminio, mas mais leve e brilhante.
\section{Nodosidade}
\begin{itemize}
\item {Grp. gram.:f.}
\end{itemize}
Qualidade ou estado do que é nodoso.
\section{Nodoso}
\begin{itemize}
\item {Grp. gram.:adj.}
\end{itemize}
\begin{itemize}
\item {Proveniência:(Lat. \textunderscore nodosus\textunderscore )}
\end{itemize}
Que tem nós ou saliências: \textunderscore tronco nodoso\textunderscore .
Proeminente.
\section{Nodulária}
\begin{itemize}
\item {Grp. gram.:f.}
\end{itemize}
\begin{itemize}
\item {Proveniência:(De \textunderscore nódulo\textunderscore )}
\end{itemize}
Gênero de plantas conferveáceas.
\section{Nódulo}
\begin{itemize}
\item {Grp. gram.:m.}
\end{itemize}
\begin{itemize}
\item {Proveniência:(Lat. \textunderscore nodulus\textunderscore )}
\end{itemize}
Pequeno nó.
\section{Noduloso}
\begin{itemize}
\item {Grp. gram.:adj.}
\end{itemize}
Que tem nódulos.
\section{Noel}
\begin{itemize}
\item {Grp. gram.:m.}
\end{itemize}
\begin{itemize}
\item {Utilização:Artilh.}
\end{itemize}
Peça cylindrica e ôca de madeira, que se introduz no meio do petardo, quando êste se carrega.
\section{Noela-táli}
\begin{itemize}
\item {Grp. gram.:m.}
\end{itemize}
Árvore do Malabar, de fôlhas semelhantes ás da laranjeira.
\section{Noemiano}
\begin{itemize}
\item {fónica:no-e}
\end{itemize}
\begin{itemize}
\item {Grp. gram.:m.}
\end{itemize}
Membro de uma seita mahometana, que attribue a Deus o poder de fazer mal.
\section{Noetarca}
\begin{itemize}
\item {fónica:no-e}
\end{itemize}
\begin{itemize}
\item {Grp. gram.:m.}
\end{itemize}
\begin{itemize}
\item {Proveniência:(Do gr. \textunderscore noetos\textunderscore  + \textunderscore arkhe\textunderscore )}
\end{itemize}
Designação do princípio primário, segundo os philósophos eclécticos de Alexandria.
\section{Noetarcha}
\begin{itemize}
\item {fónica:no-e,câ}
\end{itemize}
\begin{itemize}
\item {Grp. gram.:m.}
\end{itemize}
\begin{itemize}
\item {Proveniência:(Do gr. \textunderscore noetos\textunderscore  + \textunderscore arkhe\textunderscore )}
\end{itemize}
Designação do princípio primário, segundo os philósophos eclécticos de Alexandria.
\section{Noete}
\begin{itemize}
\item {fónica:no-ê}
\end{itemize}
\begin{itemize}
\item {Grp. gram.:m.}
\end{itemize}
\begin{itemize}
\item {Proveniência:(Fr. \textunderscore nouet.\textunderscore )}
\end{itemize}
Peça metállica e redonda, em que convergem as varetas do chapéu de chuva.
\section{Nogada}
\begin{itemize}
\item {Grp. gram.:f.}
\end{itemize}
\begin{itemize}
\item {Proveniência:(Do lat. \textunderscore nux\textunderscore )}
\end{itemize}
Flôr de nogueira.
Doce de nozes.
Môlho, em que entra o miolo de nozes.
\section{Nogádo}
\begin{itemize}
\item {Grp. gram.:m.}
\end{itemize}
\begin{itemize}
\item {Proveniência:(Do lat. \textunderscore nux\textunderscore )}
\end{itemize}
Doce de nozes ou de amêndoas ou de pinhões, etc, misturados com mel; pinhoada.
\section{Nógado}
\begin{itemize}
\item {Grp. gram.:m.}
\end{itemize}
\begin{itemize}
\item {Proveniência:(Do lat. \textunderscore nux\textunderscore )}
\end{itemize}
Doce de nozes ou de amêndoas ou de pinhões, etc, misturados com mel; pinhoada.
\section{Nogaico}
\begin{itemize}
\item {Grp. gram.:m.}
\end{itemize}
Uma das línguas do grupo tártaro.
\section{Nogal}
\begin{itemize}
\item {Grp. gram.:m.}
\end{itemize}
\begin{itemize}
\item {Proveniência:(Do lat. \textunderscore nucalis\textunderscore )}
\end{itemize}
O mesmo que \textunderscore nogueiral\textunderscore .
\section{Nogão}
\begin{itemize}
\item {Grp. gram.:m.}
\end{itemize}
\begin{itemize}
\item {Utilização:Prov.}
\end{itemize}
\begin{itemize}
\item {Utilização:minh.}
\end{itemize}
Variedade de noz graúda.
(Cp. \textunderscore nogueira\textunderscore )
\section{Nogueira}
\begin{itemize}
\item {Grp. gram.:f.}
\end{itemize}
\begin{itemize}
\item {Proveniência:(Do b. lat. \textunderscore nocaria\textunderscore , por \textunderscore nucaria\textunderscore , do lat. \textunderscore nux\textunderscore )}
\end{itemize}
Gênero de árvores juglândeas, (\textunderscore juglans\textunderscore ).
Madeira dessas árvores: \textunderscore mesa de nogueira\textunderscore .
\section{Nogueirado}
\begin{itemize}
\item {Grp. gram.:adj.}
\end{itemize}
Semelhante á côr da madeira de nogueira.
\section{Nogueiral}
\begin{itemize}
\item {Grp. gram.:m.}
\end{itemize}
Terreno, onde crescem nogueiras.
\section{Nohemiano}
\begin{itemize}
\item {Grp. gram.:m.}
\end{itemize}
Membro de uma seita mahometana, que attribue a Deus o poder de fazer mal.
\section{Noira}
\begin{itemize}
\item {Grp. gram.:f.}
\end{itemize}
Espécie de papagaio, o mesmo que \textunderscore loiro\textunderscore .
(Mal. \textunderscore nori\textunderscore )
\section{Noitada}
\begin{itemize}
\item {Grp. gram.:f.}
\end{itemize}
Espaço de uma noite.
Noite desvelada; insómnia.
Divertimento durante uma noite.
Trabalho durante a noite.
\section{Noite}
\begin{itemize}
\item {Grp. gram.:f.}
\end{itemize}
\begin{itemize}
\item {Utilização:Fig.}
\end{itemize}
\begin{itemize}
\item {Proveniência:(Lat. \textunderscore nox\textunderscore , \textunderscore noctis\textunderscore )}
\end{itemize}
Espaço de tempo, que vai desde o crepúsculo da tarde até o crepúsculo da manhan.
Escuridão.
Noitada.
Trevas do espírito.
Ignorância.
\textunderscore Noite velha\textunderscore , alta noite.
\section{Noitecer}
\begin{itemize}
\item {Grp. gram.:v. i.}
\end{itemize}
O mesmo que \textunderscore anoitecer\textunderscore .
\section{Noitibó}
\begin{itemize}
\item {Grp. gram.:m.}
\end{itemize}
\begin{itemize}
\item {Utilização:Fig.}
\end{itemize}
\begin{itemize}
\item {Proveniência:(Do b. lat. hyp. \textunderscore noctívolus?\textunderscore )}
\end{itemize}
Pássaro fissirostro.
Pessôa pouco sociável ou que só apparece de noite.
\section{Noitinha}
\begin{itemize}
\item {Grp. gram.:f.}
\end{itemize}
\begin{itemize}
\item {Proveniência:(De \textunderscore noite\textunderscore )}
\end{itemize}
Crepúsculo da tarde.
O anoitecer.
\section{Noiva}
\begin{itemize}
\item {Grp. gram.:f.}
\end{itemize}
\begin{itemize}
\item {Proveniência:(Do b. lat. \textunderscore novia\textunderscore )}
\end{itemize}
Mulher, que está para casar; mulher recém-casada.
\section{Noivado}
\begin{itemize}
\item {Grp. gram.:m.}
\end{itemize}
\begin{itemize}
\item {Proveniência:(De \textunderscore noivo\textunderscore )}
\end{itemize}
Dia do casamento.
Festa do casamento; boda; matrimónio.
\section{Noivar}
\begin{itemize}
\item {Grp. gram.:v. i.}
\end{itemize}
\begin{itemize}
\item {Utilização:Poét.}
\end{itemize}
\begin{itemize}
\item {Proveniência:(De \textunderscore noivo\textunderscore )}
\end{itemize}
Celebrar noivado.
Cortejar ou galantear a pessôa que se vai desposar.
Diz-se das aves que preparam a sua reproducção.
\section{Noivo}
\begin{itemize}
\item {Grp. gram.:m.}
\end{itemize}
\begin{itemize}
\item {Grp. gram.:Pl.}
\end{itemize}
\begin{itemize}
\item {Proveniência:(Do lat. hyp. \textunderscore novius\textunderscore , de \textunderscore novus\textunderscore )}
\end{itemize}
Indivíduo, que está para casar.
O recém-casado.
O homem e a mulher, que vão unir-se pelo casamento ou que têm ajustado o seu casamento.
Espôso e respectiva espôsa, recentemente casados.
\section{Nojado}
\begin{itemize}
\item {Grp. gram.:adj.}
\end{itemize}
O mesmo que \textunderscore anojado\textunderscore .
\section{Nojentamente}
\begin{itemize}
\item {Grp. gram.:adv.}
\end{itemize}
De modo nojento.
\section{Nojento}
\begin{itemize}
\item {Grp. gram.:adj.}
\end{itemize}
Que causa nojo; repugnante.
Que se enjôa facilmente.
\section{Nojo}
\begin{itemize}
\item {fónica:nô}
\end{itemize}
\begin{itemize}
\item {Grp. gram.:m.}
\end{itemize}
\begin{itemize}
\item {Utilização:Ant.}
\end{itemize}
\begin{itemize}
\item {Utilização:Ant.}
\end{itemize}
Náusea; repulsão, enjôo.
Pesar.
Luto.
Aquillo que inspira asco ou repugnância.
Damno.
Descontentamento, desgôsto.
(Aphérese de \textunderscore enojo\textunderscore )
\section{Nojosamente}
\begin{itemize}
\item {Grp. gram.:adv.}
\end{itemize}
De modo nojoso; nojentamente.
\section{Nojoso}
\begin{itemize}
\item {Grp. gram.:adj.}
\end{itemize}
\begin{itemize}
\item {Proveniência:(De \textunderscore nojo\textunderscore )}
\end{itemize}
O mesmo que \textunderscore nojento\textunderscore .
Desgostoso.
Que está de luto.
\section{Nola}
\begin{itemize}
\item {Grp. gram.:f.}
\end{itemize}
\begin{itemize}
\item {Utilização:Bot.}
\end{itemize}
\begin{itemize}
\item {Proveniência:(Do lat. \textunderscore nola\textunderscore )}
\end{itemize}
O mesmo que \textunderscore norça\textunderscore ^2. Cf. D. G. Dalgado, \textunderscore Flora\textunderscore , 85.
\section{Nola}
\begin{itemize}
\item {Grp. gram.:f.}
\end{itemize}
Gênero de insectos lepidópteros nocturnos.
(Relaciona-se com \textunderscore nola\textunderscore ^1?)
\section{Nolana}
\begin{itemize}
\item {Grp. gram.:f.}
\end{itemize}
\begin{itemize}
\item {Proveniência:(De \textunderscore nola\textunderscore ^1)}
\end{itemize}
Gênero de plantas da América do Sul.
\section{Nolanáceas}
\begin{itemize}
\item {Grp. gram.:f. pl.}
\end{itemize}
Família de plantas, que tem por typo a nolana.
\section{Nolétia}
\begin{itemize}
\item {Grp. gram.:f.}
\end{itemize}
Gênero de plantas, da fam. das compostas.
\section{Nolição}
\begin{itemize}
\item {Grp. gram.:f.}
\end{itemize}
\begin{itemize}
\item {Proveniência:(Do lat. \textunderscore nolo\textunderscore ; cp. \textunderscore nolentia\textunderscore )}
\end{itemize}
Acto ou effeito de não querer.
\section{Nolina}
\begin{itemize}
\item {Grp. gram.:f.}
\end{itemize}
Gênero de plantas melantháceas.
\section{Nom}
\begin{itemize}
\item {Grp. gram.:adv.}
\end{itemize}
\begin{itemize}
\item {Utilização:Ant.}
\end{itemize}
O mesmo que \textunderscore não\textunderscore ^1.
\section{Noma}
\begin{itemize}
\item {Grp. gram.:m.}
\end{itemize}
\begin{itemize}
\item {Utilização:Med.}
\end{itemize}
\begin{itemize}
\item {Proveniência:(Gr. \textunderscore nome\textunderscore )}
\end{itemize}
Gangrena da bôca.
\section{Nómada}
\begin{itemize}
\item {Grp. gram.:m.}
\end{itemize}
(V.nómade)
\section{Nómade}
\begin{itemize}
\item {Grp. gram.:adj.}
\end{itemize}
\begin{itemize}
\item {Utilização:Ext.}
\end{itemize}
\begin{itemize}
\item {Grp. gram.:M. pl.}
\end{itemize}
\begin{itemize}
\item {Proveniência:(Lat. \textunderscore nomas\textunderscore , \textunderscore nomadis\textunderscore )}
\end{itemize}
Que não tem habitação fixa, (falando-se de tribos ou raças).
Vagabundo.
Povos que, não pertencendo a determinado país, andam vagueando, sem fixar residência.
Gênero de insectos hymnópteros.
\section{Nomadismo}
\begin{itemize}
\item {Grp. gram.:m.}
\end{itemize}
Modo de vida dos Nómades.
\section{Nomáfila}
\begin{itemize}
\item {Grp. gram.:f.}
\end{itemize}
\begin{itemize}
\item {Proveniência:(Do gr. \textunderscore nomas\textunderscore  + \textunderscore philos\textunderscore )}
\end{itemize}
Gênero de plantas acantáceas.
\section{Nomancia}
\begin{itemize}
\item {Grp. gram.:f.}
\end{itemize}
Supposta arte de adivinhar por meio das letras de um nome próprio.
(Certamente, por \textunderscore onomancia\textunderscore , do gr. \textunderscore onoma\textunderscore  + \textunderscore manteia\textunderscore )
\section{Nomântico}
\begin{itemize}
\item {Grp. gram.:adj.}
\end{itemize}
Relativo á nomancia.
\section{Nomáphila}
\begin{itemize}
\item {Grp. gram.:f.}
\end{itemize}
\begin{itemize}
\item {Proveniência:(Do gr. \textunderscore nomas\textunderscore  + \textunderscore philos\textunderscore )}
\end{itemize}
Gênero de plantas acantháceas.
\section{Nomarca}
\begin{itemize}
\item {Grp. gram.:m.}
\end{itemize}
\begin{itemize}
\item {Proveniência:(Do gr. \textunderscore nomos\textunderscore  + \textunderscore arkhe\textunderscore )}
\end{itemize}
Governador de um nomo, no antigo Egipto.
\section{Nomarcado}
\begin{itemize}
\item {Grp. gram.:m.}
\end{itemize}
Govêrno ou funções de nomarca.
\section{Nomarcha}
\begin{itemize}
\item {fónica:ca}
\end{itemize}
\begin{itemize}
\item {Grp. gram.:m.}
\end{itemize}
\begin{itemize}
\item {Proveniência:(Do gr. \textunderscore nomos\textunderscore  + \textunderscore arkhe\textunderscore )}
\end{itemize}
Governador de um nomo, no antigo Egypto.
\section{Nomarchado}
\begin{itemize}
\item {fónica:ca}
\end{itemize}
\begin{itemize}
\item {Grp. gram.:m.}
\end{itemize}
Govêrno ou funcções de nomarcha.
\section{Nome}
\begin{itemize}
\item {Grp. gram.:m.}
\end{itemize}
\begin{itemize}
\item {Utilização:Deprec.}
\end{itemize}
\begin{itemize}
\item {Grp. gram.:Loc. adv.}
\end{itemize}
\begin{itemize}
\item {Grp. gram.:Loc. prep.}
\end{itemize}
\begin{itemize}
\item {Utilização:Fig.}
\end{itemize}
\begin{itemize}
\item {Proveniência:(Lat. \textunderscore nomen\textunderscore )}
\end{itemize}
Palavra, que designa pessôa, animal ou coisa, e que é geralmente um substântivo ou um adjectivo qualificativo.
Qualificação.
Reputação; nomeada: \textunderscore escritor de bom nome\textunderscore .
Título.
Raça.
Appellido; alcunha.
Palavra ou qualificação injuriosa: \textunderscore chamou-lhe nomes\textunderscore .
\textunderscore Em nome\textunderscore , nominalmente, sem realidade.
\textunderscore Em nome de\textunderscore , com autorização de, em lugar de, em attenção a, por motivo de, a pretexto de.
\textunderscore Dar nome\textunderscore , nomear, appellidar.
Tornar illustre ou afamado: \textunderscore as«Pupillas»deram nome ao autor\textunderscore .
\textunderscore Nome próprio\textunderscore , nome de baptismo.
Nome, que se applica privativamente a nações, povoações, montes, rios, mares, etc.
\section{Nomeação}
\begin{itemize}
\item {Grp. gram.:f.}
\end{itemize}
\begin{itemize}
\item {Proveniência:(Do lat. \textunderscore nominatio\textunderscore )}
\end{itemize}
Acto ou effeito de nomear.
Direito de nomear.
Despacho de alguém, para sêr provido num emprêgo.
Provisão.
\section{Nomeada}
\begin{itemize}
\item {Grp. gram.:f.}
\end{itemize}
\begin{itemize}
\item {Proveniência:(De \textunderscore nomear\textunderscore )}
\end{itemize}
Reputação.
Fama.
Bôa fama.
Antiga e pequena moéda portuguesa.
\section{Nomeadamente}
\begin{itemize}
\item {Grp. gram.:adv.}
\end{itemize}
\begin{itemize}
\item {Proveniência:(De \textunderscore nomeado\textunderscore )}
\end{itemize}
Especialmente; principalmente.
Especificadamente.
\section{Nomeado}
\begin{itemize}
\item {Grp. gram.:adj.}
\end{itemize}
\begin{itemize}
\item {Proveniência:(De \textunderscore nomear\textunderscore )}
\end{itemize}
Despachado para exercer cargo ou emprêgo.
\section{Nomeador}
\begin{itemize}
\item {Grp. gram.:m.  e  adj.}
\end{itemize}
\begin{itemize}
\item {Proveniência:(Do lat. \textunderscore nominator\textunderscore )}
\end{itemize}
O que nomeia.
\section{Nomeadura}
\begin{itemize}
\item {Grp. gram.:f.}
\end{itemize}
(V.nomeação)
\section{Nomeante}
\begin{itemize}
\item {Grp. gram.:m.  e  adj.}
\end{itemize}
\begin{itemize}
\item {Proveniência:(Do lat. \textunderscore nominans\textunderscore )}
\end{itemize}
O mesmo que \textunderscore nomeador\textunderscore .
\section{Nomear}
\begin{itemize}
\item {Grp. gram.:v. t.}
\end{itemize}
\begin{itemize}
\item {Proveniência:(Do lat. \textunderscore nominare\textunderscore )}
\end{itemize}
Designar pelo nome, pronunciar o nome de.
Appellidar.
Despachar para emprêgo ou instituir.
Empregar, despachar.
Qualificar.
Organizar, instituir.
Dar como emprêgo, conferir o cargo de:«\textunderscore ...o administrador nomeou a regedoria no de Prazins\textunderscore ». Camillo, \textunderscore Brasileira\textunderscore , 272.
\section{Nomega}
\begin{itemize}
\item {Grp. gram.:f.}
\end{itemize}
\begin{itemize}
\item {Utilização:Bras}
\end{itemize}
Terra fria, em que não dá sol. Cf. \textunderscore Notícia\textunderscore , do Rio, de 8-I-903.
\section{Nomenclador}
\begin{itemize}
\item {Grp. gram.:adj.}
\end{itemize}
\begin{itemize}
\item {Grp. gram.:M.}
\end{itemize}
\begin{itemize}
\item {Utilização:Ant.}
\end{itemize}
\begin{itemize}
\item {Proveniência:(Do lat. \textunderscore nomenclator\textunderscore )}
\end{itemize}
Que nomeia ou classifica.
Aquelle que se dedica á nomenclatura ou classificação das sciências.
Criado, que annuncia as visitas.
\section{Nomenclatura}
\begin{itemize}
\item {Grp. gram.:f.}
\end{itemize}
\begin{itemize}
\item {Proveniência:(Lat. \textunderscore nomenclatura\textunderscore )}
\end{itemize}
Conjunto dos vocábulos de um diccionário.
Conjunto de termos peculiares a uma arte ou sciência.
Méthodo para classificar os objectos de uma sciência ou arte.
Lista.
\section{Nómico}
\begin{itemize}
\item {Grp. gram.:m.}
\end{itemize}
\begin{itemize}
\item {Proveniência:(Gr. \textunderscore nomikos\textunderscore )}
\end{itemize}
Funccionário ecclesiástico do rito grego, encarregado de fazer observar as rubricas e normas da liturgia.
\section{Nómina}
\begin{itemize}
\item {Grp. gram.:f.}
\end{itemize}
\begin{itemize}
\item {Utilização:Ant.}
\end{itemize}
Oração, escrita e guardada numa bolsinha, para livrar de certos males.
Bolsa, em que se guarda essa oração.
Prego doirado em arreios de animaes de carga.
Lista ou relação de nomes.
(\textunderscore Pl.\textunderscore  do lat. \textunderscore nomen\textunderscore )
\section{Nominação}
\begin{itemize}
\item {Grp. gram.:f.}
\end{itemize}
\begin{itemize}
\item {Proveniência:(Lat. \textunderscore nominatio\textunderscore )}
\end{itemize}
Figura de Rhetórica, com que se dá nome a uma coisa que o não tem.
\section{Nominal}
\begin{itemize}
\item {Grp. gram.:adj.}
\end{itemize}
\begin{itemize}
\item {Proveniência:(Lat. \textunderscore nominalis\textunderscore )}
\end{itemize}
Relativo a nome.
Que só existe em nome; que não é real: \textunderscore 100$000 reis nominaes\textunderscore .
\section{Nominalismo}
\begin{itemize}
\item {Grp. gram.:m.}
\end{itemize}
\begin{itemize}
\item {Proveniência:(De \textunderscore nominal\textunderscore )}
\end{itemize}
Systema philosóphico dos que entendiam que as espécies, os gêneros e as entidades não eram seres reaes, mas sim abstractos.
\section{Nominalista}
\begin{itemize}
\item {Grp. gram.:m.  e  f.}
\end{itemize}
\begin{itemize}
\item {Grp. gram.:Adj.}
\end{itemize}
\begin{itemize}
\item {Proveniência:(De \textunderscore nominal\textunderscore )}
\end{itemize}
Pessôa, que segue o nominalismo.
Relativo ao nominalismo.
\section{Nominalmente}
\begin{itemize}
\item {Grp. gram.:adv.}
\end{itemize}
De modo nominal.
Nomeadamente.
Abstractamente; no nome; sem realidade.
\section{Nominativo}
\begin{itemize}
\item {Grp. gram.:adj.}
\end{itemize}
\begin{itemize}
\item {Grp. gram.:M.}
\end{itemize}
\begin{itemize}
\item {Utilização:Gram.}
\end{itemize}
\begin{itemize}
\item {Proveniência:(Lat. \textunderscore nominativus\textunderscore )}
\end{itemize}
Que tem nome ou que denomina.
Nas línguas que têm casos, caso recto ou primeiro caso dos nomes declináveis, e que na oração serve de sujeito e de attributo ou predicativo.
\section{Nómio}
\begin{itemize}
\item {Grp. gram.:m.}
\end{itemize}
Gênero de insectos coleópteros.
\section{Nomisma}
\begin{itemize}
\item {Grp. gram.:f.}
\end{itemize}
\begin{itemize}
\item {Proveniência:(Lat. \textunderscore numisma\textunderscore  e \textunderscore nomisma\textunderscore )}
\end{itemize}
\textunderscore m.\textunderscore  (e der.)
O mesmo ou melhor que \textunderscore numisma\textunderscore , etc.
Moéda cunhada.
\section{Nomísmia}
\begin{itemize}
\item {Grp. gram.:f.}
\end{itemize}
Gênero de plantas leguminosas.
\section{Nomo}
\begin{itemize}
\item {Grp. gram.:m.}
\end{itemize}
\begin{itemize}
\item {Proveniência:(Do gr. \textunderscore nomos\textunderscore )}
\end{itemize}
Divisão territorial do antigo Egýpto, espécie de districto ou província.
\section{Nomo...}
\begin{itemize}
\item {Grp. gram.:pref.}
\end{itemize}
\begin{itemize}
\item {Proveniência:(Do gr. \textunderscore nomos\textunderscore )}
\end{itemize}
(designativo de \textunderscore regra\textunderscore  ou \textunderscore preceito\textunderscore )
\section{Nomocânon}
\begin{itemize}
\item {Grp. gram.:m.}
\end{itemize}
\begin{itemize}
\item {Proveniência:(Do gr. \textunderscore nomos\textunderscore  + \textunderscore kanon\textunderscore )}
\end{itemize}
Collecção de cânones, ou de leis imperiaes que lhes dizem respeito.
\section{Nomografia}
\begin{itemize}
\item {Grp. gram.:f.}
\end{itemize}
\begin{itemize}
\item {Proveniência:(Do gr. \textunderscore nomos\textunderscore  + \textunderscore graphein\textunderscore )}
\end{itemize}
Ciência das leis e da sua interpretação.
\section{Nomographia}
\begin{itemize}
\item {Grp. gram.:f.}
\end{itemize}
\begin{itemize}
\item {Proveniência:(Do gr. \textunderscore nomos\textunderscore  + \textunderscore graphein\textunderscore )}
\end{itemize}
Sciência das leis e da sua interpretação.
\section{Nomologia}
\begin{itemize}
\item {Grp. gram.:f.}
\end{itemize}
\begin{itemize}
\item {Proveniência:(Do gr. \textunderscore nomos\textunderscore  + \textunderscore logos\textunderscore )}
\end{itemize}
Estudo das leis, que presidem aos phenómenos naturaes.
\section{Nomóteta}
\begin{itemize}
\item {Grp. gram.:m.}
\end{itemize}
\begin{itemize}
\item {Proveniência:(Lat. \textunderscore nomothetes\textunderscore )}
\end{itemize}
Cada um dos membros da grande comissão legislativa que, formada de indivíduos que tinham sido juízes, era encarregada, entre os Atenienses, da revisão das leis existentes.
\section{Nomotético}
\begin{itemize}
\item {Grp. gram.:adj.}
\end{itemize}
\begin{itemize}
\item {Utilização:Des.}
\end{itemize}
\begin{itemize}
\item {Proveniência:(De \textunderscore nomóteta\textunderscore )}
\end{itemize}
Relativo á legislação ou ao processo de fazer leis.
\section{Nomótheta}
\begin{itemize}
\item {Grp. gram.:m.}
\end{itemize}
\begin{itemize}
\item {Proveniência:(Lat. \textunderscore nomothetes\textunderscore )}
\end{itemize}
Cada um dos membros da grande commissão legislativa que, formada de indivíduos que tinham sido juízes, era encarregada, entre os Athenienses, da revisão das leis existentes.
\section{Nomothético}
\begin{itemize}
\item {Grp. gram.:adj.}
\end{itemize}
\begin{itemize}
\item {Utilização:Des.}
\end{itemize}
\begin{itemize}
\item {Proveniência:(De \textunderscore nomótheta\textunderscore )}
\end{itemize}
Relativo á legislação ou ao processo de fazer leis.
\section{Nomoxim}
\begin{itemize}
\item {Grp. gram.:m.}
\end{itemize}
Official ou servidor dos pagodes ou das communidades, na Índia Portuguesa.
\section{Nona}
\begin{itemize}
\item {Grp. gram.:f.}
\end{itemize}
\begin{itemize}
\item {Utilização:Ant.}
\end{itemize}
\begin{itemize}
\item {Proveniência:(Lat. eccles. \textunderscore nonna\textunderscore )}
\end{itemize}
O mesmo que \textunderscore monja\textunderscore  ou \textunderscore freira\textunderscore .
\section{Nona}
\begin{itemize}
\item {Grp. gram.:f.}
\end{itemize}
O mesmo que \textunderscore anona\textunderscore ^1.
\section{Nona}
\begin{itemize}
\item {Grp. gram.:f.}
\end{itemize}
\begin{itemize}
\item {Utilização:Mús.}
\end{itemize}
\begin{itemize}
\item {Grp. gram.:Pl.}
\end{itemize}
\begin{itemize}
\item {Proveniência:(Do lat. \textunderscore nonus\textunderscore )}
\end{itemize}
Estrophe de nove versos, usada especialmente em cancioneiros antigos.
Uma das horas, em que os Romanos dividiam o dia e que correspondia ás três horas da tarde.
Intervallo de nove graus.
A segunda das duas partes, em que os Romanos dividiam o mês.
\section{Nonada}
\begin{itemize}
\item {Grp. gram.:m.}
\end{itemize}
Insignificância, bagatela.
(Cast. \textunderscore nonada\textunderscore )
\section{Nonagenário}
\begin{itemize}
\item {Grp. gram.:m.  e  adj.}
\end{itemize}
\begin{itemize}
\item {Proveniência:(Lat. \textunderscore nonagenarius\textunderscore )}
\end{itemize}
O homem que tem noventa annos.
\section{Nonagésima}
\begin{itemize}
\item {Grp. gram.:f.}
\end{itemize}
\begin{itemize}
\item {Proveniência:(De \textunderscore nonagésimo\textunderscore )}
\end{itemize}
Cada uma das noventa partes, em que um todo se póde dividir.
\section{Nonagésimo}
\begin{itemize}
\item {Grp. gram.:adj.}
\end{itemize}
\begin{itemize}
\item {Grp. gram.:M.}
\end{itemize}
\begin{itemize}
\item {Proveniência:(Lat. \textunderscore nonagesimus\textunderscore )}
\end{itemize}
Que occupa o último lugar numa série de noventa.
Nonagésima.
\section{Nonágono}
\begin{itemize}
\item {Grp. gram.:m.}
\end{itemize}
(V.enneágono)
\section{Nonágria}
\begin{itemize}
\item {Grp. gram.:f.}
\end{itemize}
Gênero de insectos lepidópteros nocturnos.
\section{Nonal}
\begin{itemize}
\item {Grp. gram.:m.}
\end{itemize}
Espécie de tecido antigo.
\section{Nonavo}
\begin{itemize}
\item {Grp. gram.:m.}
\end{itemize}
\begin{itemize}
\item {Utilização:Chím.}
\end{itemize}
Um dos carbonetos do grupo formênico.
\section{Nonas}
\begin{itemize}
\item {Grp. gram.:f. pl.}
\end{itemize}
\begin{itemize}
\item {Proveniência:(Lat. \textunderscore nonae\textunderscore , de \textunderscore nonus\textunderscore )}
\end{itemize}
O nono dia antes dos idos, no antigo calendário romano.
\section{Nonatélia}
\begin{itemize}
\item {Grp. gram.:f.}
\end{itemize}
Gênero de plantas rubiáceas.
\section{Nonato}
\begin{itemize}
\item {Grp. gram.:adj.}
\end{itemize}
\begin{itemize}
\item {Proveniência:(Do lat. \textunderscore non\textunderscore  + \textunderscore natus\textunderscore )}
\end{itemize}
Diz-se do indivíduo, que não saiu naturalmente do ventre materno mas, sim, por meio da operação cesariana.
\section{Nonca}
\begin{itemize}
\item {Grp. gram.:adv.}
\end{itemize}
\begin{itemize}
\item {Utilização:Ant.}
\end{itemize}
Nunca.
\section{Nonde}
\begin{itemize}
\item {Grp. gram.:m.}
\end{itemize}
Grande árvore de Moçambique, cuja madeira se emprega em almadias.
\section{Nondo}
\begin{itemize}
\item {Grp. gram.:m.}
\end{itemize}
Quadrúpede de Çofala.
\section{Nones}
\begin{itemize}
\item {Grp. gram.:m.  e  adj.}
\end{itemize}
\begin{itemize}
\item {Utilização:Ant.}
\end{itemize}
\begin{itemize}
\item {Proveniência:(De \textunderscore nono\textunderscore , relativo a \textunderscore nove\textunderscore , que é um dos números ímpares?)}
\end{itemize}
O mesmo ou melhor que \textunderscore nunes\textunderscore .
\section{Nonga}
\begin{itemize}
\item {Grp. gram.:f.}
\end{itemize}
Pequeno cacete, usado pelos Negros da África oriental.
\section{Nongentésimo}
\begin{itemize}
\item {Grp. gram.:adj.}
\end{itemize}
O mesmo que \textunderscore noningentésimo\textunderscore .
\section{Noningentésimo}
\begin{itemize}
\item {Grp. gram.:adj.}
\end{itemize}
\begin{itemize}
\item {Proveniência:(Lat. \textunderscore noningentesimus\textunderscore )}
\end{itemize}
Que numa série de 900 occupa o último lugar.
\section{Noninha}
\begin{itemize}
\item {Grp. gram.:m.  e  f.}
\end{itemize}
\begin{itemize}
\item {Utilização:Prov.}
\end{itemize}
\begin{itemize}
\item {Utilização:trasm.}
\end{itemize}
Pessôa muito indolente, sem préstimo.
\section{Nónio}
\begin{itemize}
\item {Grp. gram.:m.}
\end{itemize}
\begin{itemize}
\item {Proveniência:(De \textunderscore Nonius\textunderscore  ou \textunderscore Nonnius\textunderscore , n. lat. de Pedro Nunes, mathemático português)}
\end{itemize}
Instrumento de Mathemática, para medir as fracções de uma divisão, numa escala graduada.
Escala dêsse instrumento.
\section{Nonnato}
\begin{itemize}
\item {Grp. gram.:adj.}
\end{itemize}
\begin{itemize}
\item {Proveniência:(Do lat. \textunderscore non\textunderscore  + \textunderscore natus\textunderscore )}
\end{itemize}
Diz-se do indivíduo, que não saiu naturalmente do ventre materno mas, sim, por meio da operação cesariana.
\section{Nono}
\begin{itemize}
\item {Grp. gram.:m.  e  adj.}
\end{itemize}
\begin{itemize}
\item {Proveniência:(Lat. \textunderscore nonus\textunderscore )}
\end{itemize}
Último objecto ou número de uma série de nove.
\section{Nono}
\begin{itemize}
\item {Grp. gram.:m.}
\end{itemize}
\begin{itemize}
\item {Utilização:Ant.}
\end{itemize}
\begin{itemize}
\item {Proveniência:(Lat. eccles. \textunderscore nonnus\textunderscore )}
\end{itemize}
O mesmo que \textunderscore frade\textunderscore  ou \textunderscore monge\textunderscore .
\section{Nonô}
\begin{itemize}
\item {Grp. gram.:m.}
\end{itemize}
O mesmo que \textunderscore nhonhô\textunderscore .
\section{Nonobstância}
\begin{itemize}
\item {Grp. gram.:f.}
\end{itemize}
Dá-se êste nome á terceira parte das provisões da côrte pontifícia, por começar sempre pelas palavras latinas \textunderscore non obstantibus\textunderscore .
\section{Nonopétalo}
\begin{itemize}
\item {Grp. gram.:adj.}
\end{itemize}
\begin{itemize}
\item {Utilização:Bot.}
\end{itemize}
\begin{itemize}
\item {Proveniência:(De \textunderscore nono\textunderscore  + \textunderscore pétala\textunderscore )}
\end{itemize}
Que tem nove pétalas.
\section{Nonquenha}
\begin{itemize}
\item {Grp. gram.:f.}
\end{itemize}
Árvore de Angola.
\section{Nónuplo}
\begin{itemize}
\item {Grp. gram.:adj.}
\end{itemize}
\begin{itemize}
\item {Proveniência:(Do lat. \textunderscore nonus\textunderscore  + suff. \textunderscore plex\textunderscore , de \textunderscore duplex\textunderscore )}
\end{itemize}
Diz-se de uma quantidade, equivalente nove vezes a outra.
\section{Nonusse}
\begin{itemize}
\item {Grp. gram.:m.}
\end{itemize}
\begin{itemize}
\item {Proveniência:(Lat. \textunderscore nonussis\textunderscore )}
\end{itemize}
Antiga moéda romana, do valor de nove asses. Cf. Castilho, \textunderscore Fastos\textunderscore , I, 354.
\section{Noologia}
\begin{itemize}
\item {Grp. gram.:f.}
\end{itemize}
\begin{itemize}
\item {Proveniência:(Do gr. \textunderscore noos\textunderscore  + \textunderscore logos\textunderscore )}
\end{itemize}
(Synónymo desusado de \textunderscore psychologia\textunderscore )
\section{Noológico}
\begin{itemize}
\item {Grp. gram.:adj.}
\end{itemize}
Relativo á noologia.
\section{Noostênico}
\begin{itemize}
\item {Grp. gram.:adj.}
\end{itemize}
\begin{itemize}
\item {Utilização:Med.}
\end{itemize}
\begin{itemize}
\item {Proveniência:(Do gr. \textunderscore noos\textunderscore  + \textunderscore sthenos\textunderscore )}
\end{itemize}
Que activa acidentalmente a inteligência, como o café.
\section{Noosthênico}
\begin{itemize}
\item {Grp. gram.:adj.}
\end{itemize}
\begin{itemize}
\item {Utilização:Med.}
\end{itemize}
\begin{itemize}
\item {Proveniência:(Do gr. \textunderscore noos\textunderscore  + \textunderscore sthenos\textunderscore )}
\end{itemize}
Que activa accidentalmente a intelligência, como o café.
\section{Nopa}
\begin{itemize}
\item {Grp. gram.:f.}
\end{itemize}
Planta anonácea da ilha de San-Thomé.
\section{Nopal}
\begin{itemize}
\item {Grp. gram.:m.}
\end{itemize}
\begin{itemize}
\item {Proveniência:(Fr. \textunderscore nopal\textunderscore )}
\end{itemize}
Nome dos cactos, que se empregam na criação da cochinilha, (\textunderscore cactus opuntia\textunderscore , Lin.).
\section{Nopaláceas}
\begin{itemize}
\item {Grp. gram.:f. pl.}
\end{itemize}
Família de cactos, a que pertence o nopal.
\section{Nora}
\begin{itemize}
\item {Grp. gram.:f.}
\end{itemize}
\begin{itemize}
\item {Utilização:Prov.}
\end{itemize}
\begin{itemize}
\item {Utilização:alg.}
\end{itemize}
\begin{itemize}
\item {Proveniência:(Do ár. \textunderscore naoure\textunderscore )}
\end{itemize}
Apparelho, para extrahir água de poços ou cisternas, e cuja parte principal é uma roda que faz girar uma corda, a que estão presos os alcatruzes.
Poço, donde se extrai água por meio do engenho.
\section{Nora}
\begin{itemize}
\item {Grp. gram.:f.}
\end{itemize}
\begin{itemize}
\item {Proveniência:(Do lat. \textunderscore nurus\textunderscore )}
\end{itemize}
Mulher casada ou viuva, em relação aos pais de seu marido.
\section{Norça}
\begin{itemize}
\item {Grp. gram.:f.}
\end{itemize}
\begin{itemize}
\item {Utilização:Prov.}
\end{itemize}
\begin{itemize}
\item {Utilização:alent.}
\end{itemize}
Pequena estaca de oliveira, em plantio.
\section{Norça}
\begin{itemize}
\item {Grp. gram.:f.}
\end{itemize}
O mesmo que \textunderscore norça-branca\textunderscore .
\section{Norça-branca}
\begin{itemize}
\item {Grp. gram.:f.}
\end{itemize}
Planta cucurbitácea, (\textunderscore bryonia dioica\textunderscore , Jacq.). Cf. P. Coutinho, \textunderscore Flora de Port.\textunderscore , 597.
\section{Norça-preta}
\begin{itemize}
\item {Grp. gram.:f.}
\end{itemize}
Planta solânea, também conhecida por \textunderscore uva-de-cão\textunderscore . Cf. P. Coutinho, \textunderscore Flora de Port.\textunderscore , 143.
\section{Nordestada}
\begin{itemize}
\item {Grp. gram.:f.}
\end{itemize}
O mesmo que \textunderscore nordestia\textunderscore .
\section{Nordeste}
\begin{itemize}
\item {Grp. gram.:m.}
\end{itemize}
\begin{itemize}
\item {Grp. gram.:Adj.}
\end{itemize}
\begin{itemize}
\item {Proveniência:(Do fr. \textunderscore nord\textunderscore  + \textunderscore est\textunderscore )}
\end{itemize}
Ponto, situado entre o Norte e o Léste, a igual distância de cada um dêstes.
Vento que sopra do lado dêsse ponto.
Relativo ao Nordeste.
\section{Nordesteado}
\begin{itemize}
\item {Grp. gram.:adj.}
\end{itemize}
\begin{itemize}
\item {Proveniência:(De \textunderscore nordestear\textunderscore )}
\end{itemize}
Que vai na direcção do Nordeste.
\section{Nordestear}
\begin{itemize}
\item {Grp. gram.:v. i.}
\end{itemize}
Navegar para o lado do Nordeste.
Inclinar-se do Norte para Léste, (falando-se da agulha magnética).
\section{Nordestia}
\begin{itemize}
\item {Grp. gram.:f.}
\end{itemize}
Vento frio do nordeste.
\section{Nore}
\begin{itemize}
\item {Grp. gram.:m.}
\end{itemize}
\begin{itemize}
\item {Utilização:Ant.}
\end{itemize}
O mesmo que \textunderscore loiro\textunderscore ^1, papagaio:«\textunderscore ...papagaios a que chamam nores\textunderscore ». Barros, \textunderscore Déc.\textunderscore  VI, c. VIII e 10.
(Do malaio \textunderscore nori\textunderscore )
\section{Nórdico}
\begin{itemize}
\item {Grp. gram.:m.}
\end{itemize}
\begin{itemize}
\item {Proveniência:(Do al. \textunderscore nord\textunderscore , norte)}
\end{itemize}
O mesmo que \textunderscore norreno\textunderscore . Cf. Latino, \textunderscore Elogíos\textunderscore , I, 73.
\section{Nória}
\begin{itemize}
\item {Grp. gram.:f.}
\end{itemize}
\begin{itemize}
\item {Utilização:Prov.}
\end{itemize}
\begin{itemize}
\item {Utilização:trasm.}
\end{itemize}
Engenho, para tirar água dos poços, o mesmo que \textunderscore nora\textunderscore ^1.
(Colhido em Valpaços)
\section{Nórico}
\begin{itemize}
\item {Grp. gram.:m.}
\end{itemize}
\begin{itemize}
\item {Proveniência:(Lat. \textunderscore noricus\textunderscore )}
\end{itemize}
Habitante da Nórica.
Antigo dialecto da Nórica.
\section{Norina}
\begin{itemize}
\item {Grp. gram.:f.}
\end{itemize}
\begin{itemize}
\item {Utilização:Chím.}
\end{itemize}
Oxydo de nório.
\section{Nório}
\begin{itemize}
\item {Grp. gram.:m.}
\end{itemize}
Metal pouco conhecido, extrahido dos óxydos misturados com os zircónios de algumas regiões.
\section{Norite}
\begin{itemize}
\item {Grp. gram.:f.}
\end{itemize}
Variedade de granito.
\section{Norma}
\begin{itemize}
\item {Grp. gram.:f.}
\end{itemize}
\begin{itemize}
\item {Proveniência:(Lat. \textunderscore norma\textunderscore )}
\end{itemize}
Regra, preceito, lei.
Modêlo.
\section{Normal}
\begin{itemize}
\item {Grp. gram.:adj.}
\end{itemize}
\begin{itemize}
\item {Grp. gram.:F.}
\end{itemize}
\begin{itemize}
\item {Proveniência:(Lat. \textunderscore normalis\textunderscore )}
\end{itemize}
Conforme á norma; exemplar.
\textunderscore Escola normal\textunderscore , aquella, cujos alumnos se preparam para o professorado.
Linha recta, que passa pelo ponto de tangência e é perpendicular á tangente de uma curva ou ao plano tangente de uma superfície.
\section{Normalidade}
\begin{itemize}
\item {Grp. gram.:f.}
\end{itemize}
Qualidade de normal.
\section{Normalista}
\begin{itemize}
\item {Grp. gram.:adj.}
\end{itemize}
\begin{itemize}
\item {Grp. gram.:M.}
\end{itemize}
Que tem o curso da escola normal.
Professor, que cursou a escola normal.
\section{Normalizar}
\begin{itemize}
\item {Grp. gram.:v. t.}
\end{itemize}
Tornar normal.
Regularizar.
\section{Normalmente}
\begin{itemize}
\item {Grp. gram.:adv.}
\end{itemize}
De modo normal; segundo as normas; segundo o uso.
\section{Normando}
\begin{itemize}
\item {Grp. gram.:adj.}
\end{itemize}
\begin{itemize}
\item {Grp. gram.:M.}
\end{itemize}
\begin{itemize}
\item {Proveniência:(Do angl. sax. \textunderscore north\textunderscore  + got. \textunderscore man\textunderscore )}
\end{itemize}
Relativo á Normândia.
Diz-se de uma espécie de caracteres typográphicos encorpados.
Habitante da Normândia.
Língua dos Normandos.
Antiga moéda normanda.
Typo ou letra normanda.
\section{Normativo}
\begin{itemize}
\item {Grp. gram.:adj.}
\end{itemize}
Que tem a qualidade ou fôrça de norma.
\section{Nornordeste}
\begin{itemize}
\item {Grp. gram.:m.}
\end{itemize}
\begin{itemize}
\item {Proveniência:(De \textunderscore Norte\textunderscore  + \textunderscore Nordeste\textunderscore )}
\end{itemize}
Ponto, entre o Norte e o Nordeste, a igual distância dêstes.
Vento, que sopra o lado dêsse ponto.
\section{Nornoroéste}
\begin{itemize}
\item {Grp. gram.:m.}
\end{itemize}
\begin{itemize}
\item {Proveniência:(De \textunderscore Norte\textunderscore  + \textunderscore Noroéste\textunderscore )}
\end{itemize}
Ponto entre o Norte e o Noroéste, a igual distância dêstes.
Vento, que sopra do lado dêsse ponto.
\section{Noro}
\begin{itemize}
\item {fónica:nô}
\end{itemize}
\begin{itemize}
\item {Grp. gram.:m.}
\end{itemize}
\begin{itemize}
\item {Utilização:Prov.}
\end{itemize}
\begin{itemize}
\item {Utilização:minh.}
\end{itemize}
\begin{itemize}
\item {Proveniência:(De \textunderscore nora\textunderscore ^2)}
\end{itemize}
O mesmo que \textunderscore genro\textunderscore . (Colhido em Barcelos)
\section{Noroéste}
\begin{itemize}
\item {Grp. gram.:m.}
\end{itemize}
\begin{itemize}
\item {Grp. gram.:Adj.}
\end{itemize}
\begin{itemize}
\item {Proveniência:(De \textunderscore Norte\textunderscore  + \textunderscore Oeste\textunderscore )}
\end{itemize}
Ponto entre o Norte e o Oéste, a igual distância dêstes.
Vento, que sopra do lado dêsse ponto.
Relativo a Noroéste.
\section{Noroesteado}
\begin{itemize}
\item {Grp. gram.:adj.}
\end{itemize}
Que vai na direcção de Noroéste.
\section{Noroestear}
\begin{itemize}
\item {Grp. gram.:v. i.}
\end{itemize}
\begin{itemize}
\item {Proveniência:(De \textunderscore Noroéste\textunderscore )}
\end{itemize}
Navegar para o lado do Noroéste.
Inclinar-se do Norte para Oéste (falando-se da agulha magnética).
\section{Norogagés}
\begin{itemize}
\item {Grp. gram.:m. pl.}
\end{itemize}
Índios do Brasil, nas margens do Tocantins.
\section{Noroguagés}
\begin{itemize}
\item {Grp. gram.:m. pl.}
\end{itemize}
Índios do Brasil, nas margens do Tocantins.
\section{Nórope}
\begin{itemize}
\item {Grp. gram.:m.}
\end{itemize}
Gênero de reptis.
\section{Norreno}
\begin{itemize}
\item {Grp. gram.:m.}
\end{itemize}
\begin{itemize}
\item {Proveniência:(Do escand. \textunderscore norrana\textunderscore )}
\end{itemize}
Grupo antigo das línguas islandesa, norueguesa, sueca e dinamarquesa.
\section{Norso}
\begin{itemize}
\item {Grp. gram.:m.}
\end{itemize}
Idioma do archipélago de Feroé e de outras ilhas.
\section{Nortada}
\begin{itemize}
\item {Grp. gram.:f.}
\end{itemize}
Vento áspero e frio, que sopra do Norte.
\section{Norte}
\begin{itemize}
\item {Grp. gram.:m.}
\end{itemize}
\begin{itemize}
\item {Utilização:Restrict.}
\end{itemize}
\begin{itemize}
\item {Grp. gram.:Adj.}
\end{itemize}
\begin{itemize}
\item {Proveniência:(Do al. \textunderscore nord\textunderscore )}
\end{itemize}
Um dos pontos cardeaes, que nos fica á esquerda, quando nos voltamos para o Nascente.
Parte do mundo ou do horizonte, correspondente á estrêlla polar.
Pólo da terra, que fica do lado da estrêlla polar.
Vento frio, que sopra dessa banda.
Regiões, que ficam para o lado do Norte.
Parte setentrional de uma região.
Estrêlla polar.
Rumo; guia; direcção.
Relativo ao Norte.
\section{Nortear}
\begin{itemize}
\item {Grp. gram.:v. t.}
\end{itemize}
Encaminhar para o Norte.
Orientar.
Dirigir.
Regular.
\section{Norteiro}
\begin{itemize}
\item {Grp. gram.:m.  e  adj.}
\end{itemize}
O mesmo que \textunderscore nortista\textunderscore .
\section{Nortênia}
\begin{itemize}
\item {Grp. gram.:f.}
\end{itemize}
Gênero de plantas escrofularíneas.
\section{Nortia}
\begin{itemize}
\item {Grp. gram.:f.}
\end{itemize}
O mesmo que \textunderscore nortada\textunderscore . Cf. \textunderscore Filinto\textunderscore , XX, 40 e 215.
\section{Nortista}
\begin{itemize}
\item {Grp. gram.:m.  e  f.}
\end{itemize}
\begin{itemize}
\item {Utilização:Bras}
\end{itemize}
Pessôa, natural dos Estados brasileiros do Norte.
\section{Noruega}
\begin{itemize}
\item {Grp. gram.:f.}
\end{itemize}
\begin{itemize}
\item {Utilização:Bras. do Rio}
\end{itemize}
\begin{itemize}
\item {Proveniência:(De \textunderscore Noruega\textunderscore , n. p.)}
\end{itemize}
Encosta de uma montanha, para o lado do Sul.
Sítio sombrio.
\section{Noruega}
\begin{itemize}
\item {Grp. gram.:m.}
\end{itemize}
\begin{itemize}
\item {Utilização:Ant.}
\end{itemize}
Espécie de açor, para caça de altanaria.
Sujeito arteiro, finório.
\section{Norueguense}
\begin{itemize}
\item {Grp. gram.:m.  e  adj.}
\end{itemize}
O mesmo que \textunderscore norueguês\textunderscore .
\section{Norueguês}
\begin{itemize}
\item {Grp. gram.:adj.}
\end{itemize}
\begin{itemize}
\item {Grp. gram.:M.}
\end{itemize}
Relativo á Noruega.
Habitante da Noruega.
Dialecto vizinho do norso.
\section{Nós}
\begin{itemize}
\item {Grp. gram.:pron.}
\end{itemize}
\begin{itemize}
\item {Proveniência:(Lat. \textunderscore nos\textunderscore )}
\end{itemize}
Indíca pessôas e emprega-se como sujeito de verbos e como regime de preposições:«\textunderscore Nós veremos o que depende de nós\textunderscore ».
\section{Nos}
\begin{itemize}
\item {fónica:nus}
\end{itemize}
\begin{itemize}
\item {Grp. gram.:pron.}
\end{itemize}
Flexão proclítica e enclítica de \textunderscore nós\textunderscore : \textunderscore esperavamos que nos fizesse bem e fez-nos mal\textunderscore .
(Cp. \textunderscore nós\textunderscore )
\section{Nos}
\begin{itemize}
\item {fónica:nus}
\end{itemize}
Expressão contrahida, equivalente a \textunderscore em os\textunderscore .
(Cp. \textunderscore no\textunderscore ^1)
\section{Nos}
\begin{itemize}
\item {fónica:nus}
\end{itemize}
\begin{itemize}
\item {Grp. gram.:pron. pl.}
\end{itemize}
O mesmo que \textunderscore os\textunderscore , depois de sýllaba nasalada.
(Cp. \textunderscore no\textunderscore ^2)
\section{Nosairitas}
\begin{itemize}
\item {Grp. gram.:m. pl.}
\end{itemize}
Seita da Turquia asiática, cujas práticas religiosas são uma mistura de Mahometismo, Judaísmo, Paganismo e Christianismo.
\section{Noscado}
\begin{itemize}
\item {Grp. gram.:adj.}
\end{itemize}
(V.moscado)
\section{Noscar}
\begin{itemize}
\item {Grp. gram.:v. t.}
\end{itemize}
\begin{itemize}
\item {Utilização:Gír.}
\end{itemize}
Partir, quebrar.
\section{Nosco}
\begin{itemize}
\item {fónica:nôs}
\end{itemize}
\begin{itemize}
\item {Utilização:Ant.}
\end{itemize}
\begin{itemize}
\item {Proveniência:(Lat. \textunderscore nobiscum\textunderscore )}
\end{itemize}
Fórma do pron. \textunderscore nós\textunderscore , precedida geralmente da prep. \textunderscore com\textunderscore : \textunderscore Deus seja comnosco\textunderscore .
O mesmo que \textunderscore comnosco\textunderscore . Cf. Viterbo, \textunderscore Elucidário\textunderscore .
\section{Noso...}
\begin{itemize}
\item {Grp. gram.:pref.}
\end{itemize}
\begin{itemize}
\item {Proveniência:(Do gr. \textunderscore nosos\textunderscore )}
\end{itemize}
(designativo de \textunderscore doença\textunderscore )
\section{Nosocomial}
\begin{itemize}
\item {Grp. gram.:adj.}
\end{itemize}
\begin{itemize}
\item {Proveniência:(Do gr. \textunderscore nosokomeion\textunderscore )}
\end{itemize}
Relativo a hospital.
\section{Nosocómico}
\begin{itemize}
\item {Grp. gram.:adj.}
\end{itemize}
\begin{itemize}
\item {Proveniência:(Do gr. \textunderscore nosokomeion\textunderscore )}
\end{itemize}
Relativo a hospital.
\section{Nosocómio}
\begin{itemize}
\item {Grp. gram.:m.}
\end{itemize}
\begin{itemize}
\item {Proveniência:(Lat. \textunderscore nosocomium\textunderscore )}
\end{itemize}
O mesmo que \textunderscore hospital\textunderscore .
\section{Nosocrático}
\begin{itemize}
\item {Grp. gram.:adj.}
\end{itemize}
\begin{itemize}
\item {Proveniência:(Do gr. \textunderscore nosos\textunderscore  + \textunderscore kratein\textunderscore )}
\end{itemize}
Específico, (falando-se de um medicamento).
\section{Nosodendro}
\begin{itemize}
\item {Grp. gram.:m.}
\end{itemize}
\begin{itemize}
\item {Proveniência:(Do gr. \textunderscore nosos\textunderscore  + \textunderscore dendron\textunderscore )}
\end{itemize}
Gênero de insectos clavicórneos.
\section{Nosoderma}
\begin{itemize}
\item {Grp. gram.:f.}
\end{itemize}
\begin{itemize}
\item {Proveniência:(Do gr. \textunderscore nosos\textunderscore  + \textunderscore derma\textunderscore )}
\end{itemize}
Gênero de insectos coleópteros heterómeros.
\section{Nosofobia}
\begin{itemize}
\item {Grp. gram.:f.}
\end{itemize}
\begin{itemize}
\item {Proveniência:(Do gr. \textunderscore nosos\textunderscore  + \textunderscore phobein\textunderscore )}
\end{itemize}
Medo de adoecer, que leva o indivíduo a tratar-se de doenças que não tem.
\section{Nosófobo}
\begin{itemize}
\item {Grp. gram.:m.}
\end{itemize}
Aquele que sofre nosofobia.
\section{Nosóforo}
\begin{itemize}
\item {Grp. gram.:m.}
\end{itemize}
\begin{itemize}
\item {Proveniência:(Do gr. \textunderscore nosos\textunderscore  + \textunderscore phoros\textunderscore )}
\end{itemize}
Aparelho de ferro, em que se fórma o leito dos feridos e de outros doentes, para evitar que se magôem.
\section{Nosogenia}
\begin{itemize}
\item {Grp. gram.:f.}
\end{itemize}
\begin{itemize}
\item {Proveniência:(Do gr. \textunderscore nosos\textunderscore  + \textunderscore genea\textunderscore )}
\end{itemize}
Desenvolvimento das doenças.
Theoria dêsse desenvolvimento.
\section{Nosogênico}
\begin{itemize}
\item {Grp. gram.:adj.}
\end{itemize}
Relativo á nosogenia.
\section{Nosografia}
\begin{itemize}
\item {Grp. gram.:f.}
\end{itemize}
\begin{itemize}
\item {Proveniência:(Do gr. \textunderscore nosos\textunderscore  + \textunderscore graphein\textunderscore )}
\end{itemize}
Distribuição metódica das doenças, segundo as suas classes, ordens, gêneros e espécies.
\section{Nosográfico}
\begin{itemize}
\item {Grp. gram.:adj.}
\end{itemize}
Relativo á nosografia.
\section{Nosographia}
\begin{itemize}
\item {Grp. gram.:f.}
\end{itemize}
\begin{itemize}
\item {Proveniência:(Do gr. \textunderscore nosos\textunderscore  + \textunderscore graphein\textunderscore )}
\end{itemize}
Distribuição methódica das doenças, segundo as suas classes, ordens, gêneros e espécies.
\section{Nosográphico}
\begin{itemize}
\item {Grp. gram.:adj.}
\end{itemize}
Relativo á nosographia.
\section{Nosologia}
\begin{itemize}
\item {Grp. gram.:f.}
\end{itemize}
\begin{itemize}
\item {Proveniência:(Do gr. \textunderscore nosos\textunderscore  + \textunderscore logos\textunderscore )}
\end{itemize}
Parte da Medicina, que descreve, define e estuda as doenças em todas as suas circunstâncias.
\section{Nosológico}
\begin{itemize}
\item {Grp. gram.:adj.}
\end{itemize}
Relativo á nosologia.
\section{Nosologista}
\begin{itemize}
\item {Grp. gram.:m.}
\end{itemize}
Aquelle que se occupa de nosologia.
\section{Nosólogo}
\begin{itemize}
\item {Grp. gram.:m.}
\end{itemize}
Aquelle que é perito em nosologia.
\section{Nosomania}
\begin{itemize}
\item {Grp. gram.:f.}
\end{itemize}
\begin{itemize}
\item {Proveniência:(De \textunderscore noso...\textunderscore  + \textunderscore mania\textunderscore )}
\end{itemize}
Espécie de monomania, que faz crêr ao indivíduo que soffre tal ou tal doença, não a tendo realmente.
\section{Nosomaníaco}
\begin{itemize}
\item {Grp. gram.:m.  e  adj.}
\end{itemize}
Aquelle que soffre de nosomania.
\section{Nosomântica}
\begin{itemize}
\item {Grp. gram.:f.}
\end{itemize}
\begin{itemize}
\item {Proveniência:(Do gr. \textunderscore nosos\textunderscore  + \textunderscore manteia\textunderscore )}
\end{itemize}
Supposta arte de curar por meio de encantamentos.
\section{Nosophobia}
\begin{itemize}
\item {Grp. gram.:f.}
\end{itemize}
\begin{itemize}
\item {Proveniência:(Do gr. \textunderscore nosos\textunderscore  + \textunderscore phobein\textunderscore )}
\end{itemize}
Medo de adoecer, que leva o indivíduo a tratar-se de doenças que não tem.
\section{Nosóphobo}
\begin{itemize}
\item {Grp. gram.:m.}
\end{itemize}
Aquelle que soffre nosophobia.
\section{Nosóphoro}
\begin{itemize}
\item {Grp. gram.:m.}
\end{itemize}
\begin{itemize}
\item {Proveniência:(Do gr. \textunderscore nosos\textunderscore  + \textunderscore phoros\textunderscore )}
\end{itemize}
Apparelho de ferro, em que se fórma o leito dos feridos e de outros doentes, para evitar que se magôem.
\section{Nossa}
\begin{itemize}
\item {Grp. gram.:pron.}
\end{itemize}
Flex. fem. de \textunderscore nosso\textunderscore .
\section{Nosso}
\begin{itemize}
\item {Grp. gram.:pron.}
\end{itemize}
\begin{itemize}
\item {Grp. gram.:M. pl.}
\end{itemize}
É possessivo, indicando que alguma coisa é própria de nós; que nos pertence; que nós estimamos muito.
Os nossos parentes, amigos, companheiros, etc.
(Talvez do lat. \textunderscore noster\textunderscore , mas não é fácil explicar como o ablativo latino \textunderscore nostro\textunderscore  produziu o port. \textunderscore nosso\textunderscore )
\section{Nosso-pai}
\begin{itemize}
\item {Grp. gram.:m.}
\end{itemize}
\begin{itemize}
\item {Utilização:Prov.}
\end{itemize}
O mesmo que \textunderscore viático\textunderscore : \textunderscore vou acompanhar nosso-pai\textunderscore .
\section{Nostalgia}
\begin{itemize}
\item {Grp. gram.:f.}
\end{itemize}
\begin{itemize}
\item {Proveniência:(Do gr. \textunderscore nostos\textunderscore  + \textunderscore algos\textunderscore )}
\end{itemize}
Abatimento ou tristeza profunda, resultante das saudades da pátria.
\section{Nostálgico}
\begin{itemize}
\item {Grp. gram.:adj.}
\end{itemize}
\begin{itemize}
\item {Grp. gram.:M.}
\end{itemize}
Relativo á nostalgia.
Que soffre nostalgia.
Indivíduo, que padece nostalgia.
\section{Nostocíneas}
\begin{itemize}
\item {Grp. gram.:f. pl.}
\end{itemize}
Grupo de algas.
\section{Nostomania}
\begin{itemize}
\item {Grp. gram.:f.}
\end{itemize}
\begin{itemize}
\item {Utilização:Med.}
\end{itemize}
\begin{itemize}
\item {Proveniência:(Do gr. \textunderscore nostos\textunderscore  + \textunderscore mania\textunderscore )}
\end{itemize}
O mesmo que \textunderscore nostalgia\textunderscore .
Espécie de alienação mental, produzida pela nostalgia.
\section{Nostro}
\begin{itemize}
\item {Grp. gram.:pron.}
\end{itemize}
\begin{itemize}
\item {Utilização:Ant.}
\end{itemize}
O mesmo que \textunderscore nosso\textunderscore :«\textunderscore ...nostro Senhor...\textunderscore »Conde de Barcelos, \textunderscore Cancion.\textunderscore 
\section{Nota}
\begin{itemize}
\item {Grp. gram.:f.}
\end{itemize}
\begin{itemize}
\item {Proveniência:(Lat. \textunderscore nota\textunderscore )}
\end{itemize}
Acto ou effeito de notar.
Sinal para marcar ou fazer lembrar.
Commentário ou reflexão, que se junta a um escrito.
Apontamento.
Attenção.
Exposição summária.
Reputação: \textunderscore mulhér de má nota\textunderscore .
Apreciação.
Reparo.
Registo das escrituras dos tabelliães.
Offensa.
Êrro.
Sinal, que na música representa um som e a sua duração.
Som.
Timbre.
Voz.
Papel, que representa moéda e é emittido por um Banco.
Communicação escrita e official entre Ministros de países differentes.
\section{Notabilidade}
\begin{itemize}
\item {Grp. gram.:f.}
\end{itemize}
\begin{itemize}
\item {Proveniência:(Do lat. \textunderscore notabilis\textunderscore )}
\end{itemize}
Qualidade do que é notável.
Pessôa notável.
\section{Notabilissimamente}
\begin{itemize}
\item {Grp. gram.:adv.}
\end{itemize}
De modo notabilíssimo.
\section{Notabilíssimo}
\begin{itemize}
\item {Grp. gram.:adj.}
\end{itemize}
\begin{itemize}
\item {Proveniência:(Do lat. \textunderscore notabilis\textunderscore )}
\end{itemize}
Muito notável.
\section{Notabilizar}
\begin{itemize}
\item {Grp. gram.:v. t.}
\end{itemize}
Tornar notável, afamado.
\section{Notação}
\begin{itemize}
\item {Grp. gram.:f.}
\end{itemize}
\begin{itemize}
\item {Utilização:Gram.}
\end{itemize}
\begin{itemize}
\item {Utilização:Mús.}
\end{itemize}
\begin{itemize}
\item {Proveniência:(Lat. \textunderscore notatio\textunderscore )}
\end{itemize}
Acto ou effeito de notar.
Sinal, que modifica os sons das letras, como o accento, o til, a cedilha.
Conjunto de sinaes convencionaes.
Conjunto dos sinaes representativos dos sons e das combinações dêstes.
\section{Notado}
\begin{itemize}
\item {Grp. gram.:adj.}
\end{itemize}
Que dá na vista; notável.
De que se tomou nota.
\section{Notador}
\begin{itemize}
\item {Grp. gram.:m.  e  adj.}
\end{itemize}
O que nota.
\section{Notalgia}
\begin{itemize}
\item {Grp. gram.:f.}
\end{itemize}
\begin{itemize}
\item {Proveniência:(Do gr. \textunderscore notos\textunderscore  + \textunderscore algos\textunderscore )}
\end{itemize}
Dôr na região dorsal, sem phenómenos inflammatórios.
\section{Notar}
\begin{itemize}
\item {Grp. gram.:v. t.}
\end{itemize}
\begin{itemize}
\item {Proveniência:(Lat. \textunderscore notare\textunderscore )}
\end{itemize}
Pôr nota em.
Marcar.
Ditar.
Reparar em; observar: \textunderscore notar êrros\textunderscore .
Explicar.
Estranhar.
Reflectir: \textunderscore mas note bem o que vou dizer\textunderscore .
Accusar.
Representar por caracteres.
Inscrever nas notas do tabellião.
\section{Notariado}
\begin{itemize}
\item {Grp. gram.:adj.}
\end{itemize}
\begin{itemize}
\item {Proveniência:(De \textunderscore notário\textunderscore )}
\end{itemize}
Offício de notário ou de tabellião.
\section{Notarial}
\begin{itemize}
\item {Grp. gram.:adj.}
\end{itemize}
Relativo a notário.
\section{Notário}
\begin{itemize}
\item {Grp. gram.:m.}
\end{itemize}
\begin{itemize}
\item {Proveniência:(Lat. \textunderscore notarius\textunderscore )}
\end{itemize}
Escrivão público; tabellião.
\section{Notável}
\begin{itemize}
\item {Grp. gram.:adj.}
\end{itemize}
\begin{itemize}
\item {Proveniência:(Do lat. \textunderscore notabilis\textunderscore )}
\end{itemize}
Digno de nota: \textunderscore tem descuidos notáveis\textunderscore .
Importante.
Que merece aprêço: \textunderscore obra notável\textunderscore .
Louvável.
Extraordinário.
Insigne: \textunderscore foi orador notável\textunderscore .
Que tem bôa posição social.
\section{Notavelmente}
\begin{itemize}
\item {Grp. gram.:adv.}
\end{itemize}
De modo notável.
\section{Notélea}
\begin{itemize}
\item {Grp. gram.:f.}
\end{itemize}
Gênero de plantas oleáceas.
\section{Notencefalia}
\begin{itemize}
\item {Grp. gram.:f.}
\end{itemize}
Qualidade ou estado de notencéfalo.
\section{Notencefálico}
\begin{itemize}
\item {Grp. gram.:adj.}
\end{itemize}
Relativo á notencefalia.
\section{Notencéfalo}
\begin{itemize}
\item {Grp. gram.:m.  e  adj.}
\end{itemize}
\begin{itemize}
\item {Proveniência:(Do gr. \textunderscore notos\textunderscore  + \textunderscore enkephalos\textunderscore )}
\end{itemize}
Diz-se do monstro, cujo cérebro, formando hérnia, se apoia sôbre as vértebras dorsaes, abertas posteriormente.
\section{Notencephalia}
\begin{itemize}
\item {Grp. gram.:f.}
\end{itemize}
Qualidade ou estado de notencéphalo.
\section{Notencephálico}
\begin{itemize}
\item {Grp. gram.:adj.}
\end{itemize}
Relativo á notencephalia.
\section{Notencéphalo}
\begin{itemize}
\item {Grp. gram.:m.  e  adj.}
\end{itemize}
\begin{itemize}
\item {Proveniência:(Do gr. \textunderscore notos\textunderscore  + \textunderscore enkephalos\textunderscore )}
\end{itemize}
Diz-se do monstro, cujo cérebro, formando hérnia, se apoia sôbre as vértebras dorsaes, abertas posteriormente.
\section{Noterófila}
\begin{itemize}
\item {Grp. gram.:f.}
\end{itemize}
Gênero de plantas melastomáceas.
\section{Noteróphila}
\begin{itemize}
\item {Grp. gram.:f.}
\end{itemize}
Gênero de plantas melastomáceas.
\section{Notho}
\begin{itemize}
\item {Grp. gram.:adj.}
\end{itemize}
\begin{itemize}
\item {Utilização:Ant.}
\end{itemize}
\begin{itemize}
\item {Proveniência:(Gr. \textunderscore nothos\textunderscore )}
\end{itemize}
Que não é legítimo; bastardo.
\section{Nothoscordo}
\begin{itemize}
\item {Grp. gram.:m.}
\end{itemize}
\begin{itemize}
\item {Proveniência:(Do gr. \textunderscore nothos\textunderscore  + \textunderscore scordon\textunderscore )}
\end{itemize}
Gênero de plantas liliáceas.
\section{Notícia}
\begin{itemize}
\item {Grp. gram.:f.}
\end{itemize}
\begin{itemize}
\item {Proveniência:(Lat. \textunderscore notícia\textunderscore )}
\end{itemize}
Informação, conhecimento.
Apontamento, nota.
Nota histórica.
Exposição summária de um acontecimento.
Biografia.
Noção.
Nota; lembrança.
Novidade: \textunderscore jornal de notícias\textunderscore .
Annúncio.
\section{Noticiador}
\begin{itemize}
\item {Grp. gram.:m.  e  adj.}
\end{itemize}
O que noticía.
\section{Noticiar}
\begin{itemize}
\item {Grp. gram.:v. t.}
\end{itemize}
Dar notícia de; annunciar; communicar.
Dizer como novidade.
\section{Noticiário}
\begin{itemize}
\item {Grp. gram.:m.}
\end{itemize}
Conjunto de notícias.
Secção dos periódicos, destinada a publicação de notícias diversas.
\section{Noticiarista}
\begin{itemize}
\item {Grp. gram.:m.}
\end{itemize}
\begin{itemize}
\item {Proveniência:(De \textunderscore noticiar\textunderscore )}
\end{itemize}
Aquelle que dá notícias.
Aquelle que escreve notícias nos periódicos.
\section{Noticieiro}
\begin{itemize}
\item {Grp. gram.:m.}
\end{itemize}
\begin{itemize}
\item {Utilização:Deprec.}
\end{itemize}
O mesmo que \textunderscore noticiarista\textunderscore .
\section{Noticioso}
\begin{itemize}
\item {Grp. gram.:adj.}
\end{itemize}
Que contém muitas notícias.
Que dá ou publica notícias: \textunderscore periódico noticioso\textunderscore .
\section{Notificação}
\begin{itemize}
\item {Grp. gram.:f.}
\end{itemize}
Acto ou effeito de notificar.
\section{Notificador}
\begin{itemize}
\item {Grp. gram.:adj.}
\end{itemize}
O mesmo que \textunderscore notificante\textunderscore .
\section{Notificante}
\begin{itemize}
\item {Grp. gram.:adj.}
\end{itemize}
Que notifica.
\section{Notificar}
\begin{itemize}
\item {Grp. gram.:v. t.}
\end{itemize}
\begin{itemize}
\item {Proveniência:(Lat. \textunderscore notificare\textunderscore )}
\end{itemize}
Dar conhecimento de.
Communicar com certas formalidades.
Avisar; intimar.
\section{Notificativo}
\begin{itemize}
\item {Grp. gram.:adj.}
\end{itemize}
Que serve para notificar.
\section{Notificatório}
\begin{itemize}
\item {Grp. gram.:adj.}
\end{itemize}
Que notifica; notificativo.
\section{Notília}
\begin{itemize}
\item {Grp. gram.:f.}
\end{itemize}
Gênero de orquídeas.
\section{Notióbia}
\begin{itemize}
\item {Grp. gram.:f.}
\end{itemize}
Gênero de insectos coleópteros pentâmeros.
\section{Notíodes}
\begin{itemize}
\item {Grp. gram.:m. pl.}
\end{itemize}
Gênero de insectos coleópteros tetrâmeros.
\section{Notiófila}
\begin{itemize}
\item {Grp. gram.:f.}
\end{itemize}
Gênero de insectos dípteros.
\section{Notiófilo}
\begin{itemize}
\item {Grp. gram.:m.}
\end{itemize}
Gênero de insectos coleópteros pentâmeros.
\section{Notiónomo}
\begin{itemize}
\item {Grp. gram.:m.}
\end{itemize}
Gênero de insectos coleópteros tetrâmeros.
\section{Notióphila}
\begin{itemize}
\item {Grp. gram.:f.}
\end{itemize}
Gênero de insectos dípteros.
\section{Notióphilo}
\begin{itemize}
\item {Grp. gram.:m.}
\end{itemize}
Gênero de insectos coleópteros pentâmeros.
\section{Noto}
\begin{itemize}
\item {Grp. gram.:m.}
\end{itemize}
\begin{itemize}
\item {Utilização:Poét.}
\end{itemize}
\begin{itemize}
\item {Proveniência:(Lat. \textunderscore notus\textunderscore )}
\end{itemize}
Vento do Sul.
\section{Noto}
\begin{itemize}
\item {Grp. gram.:adj.}
\end{itemize}
\begin{itemize}
\item {Utilização:Poét.}
\end{itemize}
\begin{itemize}
\item {Proveniência:(Lat. \textunderscore notus\textunderscore )}
\end{itemize}
Manifesto, patente; sabido. Cf. F. Barreto, \textunderscore Eneida\textunderscore , I, 87.
\section{Noto}
\begin{itemize}
\item {Grp. gram.:adj.}
\end{itemize}
\begin{itemize}
\item {Utilização:Ant.}
\end{itemize}
\begin{itemize}
\item {Proveniência:(Gr. \textunderscore nothos\textunderscore )}
\end{itemize}
Que não é legítimo; bastardo.
\section{Notobáride}
\begin{itemize}
\item {Grp. gram.:f.}
\end{itemize}
Gênero de plantas, da fam. das compostas.
\section{Notobrânchios}
\begin{itemize}
\item {fónica:qui}
\end{itemize}
\begin{itemize}
\item {Grp. gram.:m. pl.}
\end{itemize}
Ordem de molluscos gasterópodes, formada pelos que tem brânchias no dorso.
\section{Notobrânquios}
\begin{itemize}
\item {Grp. gram.:m. pl.}
\end{itemize}
Ordem de moluscos gasterópodes, formada pelos que tem brânquias no dorso.
\section{Notócera}
\begin{itemize}
\item {Grp. gram.:f.}
\end{itemize}
\begin{itemize}
\item {Proveniência:(Do gr. \textunderscore notos\textunderscore  + \textunderscore keras\textunderscore )}
\end{itemize}
Gênero de plantas crucíferas.
Gênero de insectos hemipteros.
\section{Notochlena}
\begin{itemize}
\item {Grp. gram.:f.}
\end{itemize}
Gênero de fêtos polypodiáceos.
\section{Notocórdio}
\begin{itemize}
\item {Grp. gram.:m.}
\end{itemize}
\begin{itemize}
\item {Utilização:Anat.}
\end{itemize}
\begin{itemize}
\item {Proveniência:(Do gr. \textunderscore notos\textunderscore  + \textunderscore khorde\textunderscore )}
\end{itemize}
Corda, de substância molle, cercada pelas vértebras, e precursora da formação do esqueleto.
\section{Notodonte}
\begin{itemize}
\item {Grp. gram.:m.}
\end{itemize}
\begin{itemize}
\item {Proveniência:(Do gr. \textunderscore notos\textunderscore  + \textunderscore odous\textunderscore , \textunderscore odontos\textunderscore )}
\end{itemize}
Gênero de insectos lepidópteros nocturnos.
\section{Notogástrio}
\begin{itemize}
\item {Grp. gram.:m.}
\end{itemize}
\begin{itemize}
\item {Utilização:Zool.}
\end{itemize}
\begin{itemize}
\item {Proveniência:(Do gr. \textunderscore notos\textunderscore  + \textunderscore gaster\textunderscore )}
\end{itemize}
Porção dorsal do abdome, nos animaes articulados.
\section{Notogimno}
\begin{itemize}
\item {Grp. gram.:m.}
\end{itemize}
Gênero de helminthos.
\section{Notogymno}
\begin{itemize}
\item {Grp. gram.:m.}
\end{itemize}
Gênero de helminthos.
\section{Notomelia}
\begin{itemize}
\item {Grp. gram.:f.}
\end{itemize}
Qualidade ou estado de notómelo.
\section{Notomélico}
\begin{itemize}
\item {Grp. gram.:adj.}
\end{itemize}
Relativo á notomelia.
\section{Notómelo}
\begin{itemize}
\item {Grp. gram.:m.  e  adj.}
\end{itemize}
\begin{itemize}
\item {Proveniência:(Do gr. \textunderscore notos\textunderscore  + \textunderscore melos\textunderscore )}
\end{itemize}
Monstro, que tem nas costas \textunderscore um\textunderscore  ou mais membros accessórios.
\section{Notomia}
\begin{itemize}
\item {Grp. gram.:f.}
\end{itemize}
\begin{itemize}
\item {Utilização:Pop.}
\end{itemize}
\begin{itemize}
\item {Utilização:Ant.}
\end{itemize}
O mesmo que \textunderscore anatomia\textunderscore . Cf. \textunderscore Eufrosina\textunderscore , 15.
\section{Notonecta}
\begin{itemize}
\item {Grp. gram.:f.}
\end{itemize}
\begin{itemize}
\item {Proveniência:(Do gr. \textunderscore notos\textunderscore  + \textunderscore nektos\textunderscore )}
\end{itemize}
Gênero de insectos hemípteros, que vive nas águas estagnadas e nada sempre de costas.
\section{Notonecto}
\begin{itemize}
\item {Grp. gram.:m.}
\end{itemize}
\begin{itemize}
\item {Proveniência:(Do gr. \textunderscore notos\textunderscore  + \textunderscore nektos\textunderscore )}
\end{itemize}
Gênero de insectos hemípteros, que vive nas águas estagnadas e nada sempre de costas.
\section{Notónia}
\begin{itemize}
\item {Grp. gram.:f.}
\end{itemize}
Gênero de plantas, da fam. das compostas.
\section{Notópodes}
\begin{itemize}
\item {Grp. gram.:m. pl.}
\end{itemize}
\begin{itemize}
\item {Utilização:Zool.}
\end{itemize}
\begin{itemize}
\item {Proveniência:(Do gr. \textunderscore notos\textunderscore  + \textunderscore pous\textunderscore , \textunderscore podos\textunderscore )}
\end{itemize}
Tríbo de crustáceos decápodes, no systema de Latreille.
\section{Notóptero}
\begin{itemize}
\item {Grp. gram.:m.}
\end{itemize}
Gênero de peixes malacopterýgios.
\section{Notoriamente}
\begin{itemize}
\item {Grp. gram.:adv.}
\end{itemize}
De modo notório; com publicidade.
\section{Notoriedade}
\begin{itemize}
\item {Grp. gram.:f.}
\end{itemize}
Qualidade ou estado do que é notório; publicidade.
\section{Notório}
\begin{itemize}
\item {Grp. gram.:adj.}
\end{itemize}
\begin{itemize}
\item {Proveniência:(Lat. \textunderscore notorius\textunderscore )}
\end{itemize}
Público; sabido de todos; patente, manifesto.
\section{Notoscordo}
\begin{itemize}
\item {Grp. gram.:m.}
\end{itemize}
\begin{itemize}
\item {Proveniência:(Do gr. \textunderscore nothos\textunderscore  + \textunderscore scordon\textunderscore )}
\end{itemize}
Gênero de plantas liliáceas.
\section{Notosômalo}
\begin{itemize}
\item {fónica:sô}
\end{itemize}
\begin{itemize}
\item {Grp. gram.:m.}
\end{itemize}
Gênero de insectos coleópteros tetrâmeros.
\section{Notossômalo}
\begin{itemize}
\item {Grp. gram.:m.}
\end{itemize}
Gênero de insectos coleópteros tetrâmeros.
\section{Notósteno}
\begin{itemize}
\item {Grp. gram.:m.}
\end{itemize}
Gênero de insectos coleópteros tetrâmeros.
\section{Notoxo}
\begin{itemize}
\item {fónica:cso}
\end{itemize}
\begin{itemize}
\item {Grp. gram.:m.}
\end{itemize}
\begin{itemize}
\item {Proveniência:(Do gr. \textunderscore notos\textunderscore  + \textunderscore oxus\textunderscore )}
\end{itemize}
Gênero de insectos coleópteros heterómeros.
\section{Nótula}
\begin{itemize}
\item {Grp. gram.:f.}
\end{itemize}
\begin{itemize}
\item {Proveniência:(Lat. \textunderscore notula\textunderscore )}
\end{itemize}
Pequena nota; pequeno commentário.
\section{Notýlia}
\begin{itemize}
\item {Grp. gram.:f.}
\end{itemize}
Gênero de orchídeas.
\section{Nouca}
\begin{itemize}
\item {Grp. gram.:f.}
\end{itemize}
\begin{itemize}
\item {Utilização:Prov.}
\end{itemize}
\begin{itemize}
\item {Utilização:trasm.}
\end{itemize}
O mesmo que \textunderscore nuca\textunderscore .
\section{Noulétia}
\begin{itemize}
\item {Grp. gram.:f.}
\end{itemize}
Gênero de plantas bignoniáceas.
\section{Noute}
\begin{itemize}
\item {Grp. gram.:f.}
\end{itemize}
\begin{itemize}
\item {Utilização:Fig.}
\end{itemize}
\begin{itemize}
\item {Proveniência:(Lat. \textunderscore nox\textunderscore , \textunderscore noctis\textunderscore )}
\end{itemize}
Espaço de tempo, que vai desde o crepúsculo da tarde até o crepúsculo da manhan.
Escuridão.
Noitada.
Trevas do espírito.
Ignorância.
\textunderscore Noute velha\textunderscore , alta noute.
\section{Noutro}
Expressão contrahida, equivalente a \textunderscore em outro\textunderscore .--Muitos escrevem \textunderscore n'outro\textunderscore , incorrectamente.
\section{Noutrora}
\begin{itemize}
\item {Grp. gram.:adv.}
\end{itemize}
O mesmo que \textunderscore outrora\textunderscore .
(Contr. de \textunderscore noutra\textunderscore , fem. de \textunderscore noutro\textunderscore  + \textunderscore hora\textunderscore )
\section{Nova}
\begin{itemize}
\item {Grp. gram.:f.}
\end{itemize}
\begin{itemize}
\item {Proveniência:(De \textunderscore novo\textunderscore )}
\end{itemize}
Notícia imprevista; notícia; novidade.
\section{Novação}
\begin{itemize}
\item {Grp. gram.:f.}
\end{itemize}
\begin{itemize}
\item {Proveniência:(Lat. \textunderscore novatio\textunderscore )}
\end{itemize}
O mesmo que \textunderscore innovação\textunderscore .
Renovação de um contrato ou obrigação.
\section{Novador}
\begin{itemize}
\item {Grp. gram.:m.  e  adj.}
\end{itemize}
\begin{itemize}
\item {Proveniência:(Do lat. \textunderscore novator\textunderscore )}
\end{itemize}
O mesmo que \textunderscore innovador\textunderscore .
\section{Noval}
\begin{itemize}
\item {Grp. gram.:m.}
\end{itemize}
O mesmo que \textunderscore arroteia\textunderscore .
\section{Novamente}
\begin{itemize}
\item {Grp. gram.:adv.}
\end{itemize}
\begin{itemize}
\item {Utilização:Ant.}
\end{itemize}
\begin{itemize}
\item {Proveniência:(De \textunderscore novo\textunderscore )}
\end{itemize}
De novo; outra vez; repetidamente.
Pouco antes. Cf. \textunderscore Rev. Lus.\textunderscore , XVI, 9.
\section{Novaspyrina}
\begin{itemize}
\item {Grp. gram.:m.}
\end{itemize}
Medicamento contra a gripe.
\section{Novato}
\begin{itemize}
\item {Grp. gram.:m.}
\end{itemize}
\begin{itemize}
\item {Grp. gram.:Adj.}
\end{itemize}
\begin{itemize}
\item {Proveniência:(Lat. \textunderscore novatus\textunderscore )}
\end{itemize}
Estudante novel; caloiro.
Principiante.
Apprendiz.
Indivíduo ingênuo.
Alumno do primeiro anno do qualquer Faculdade da Universidade
Ingênuo, inexperiente.
\section{Nove}
\begin{itemize}
\item {Grp. gram.:adj.}
\end{itemize}
\begin{itemize}
\item {Grp. gram.:M.}
\end{itemize}
\begin{itemize}
\item {Proveniência:(Lat. \textunderscore novem\textunderscore )}
\end{itemize}
Diz-se de um número, formado de oito e mais um, ou immediato a oito; nono.
O algarismo representativo dêsse número.
Carta de jogar, que tem nove pontos.
Aquillo que numa série de nove occupa o último logar.
\section{Novẽa}
\begin{itemize}
\item {Grp. gram.:f.}
\end{itemize}
\begin{itemize}
\item {Utilização:Ant.}
\end{itemize}
O mesmo que \textunderscore novena\textunderscore .
\section{Noveado}
\begin{itemize}
\item {Grp. gram.:adj.}
\end{itemize}
\begin{itemize}
\item {Utilização:Ant.}
\end{itemize}
\begin{itemize}
\item {Proveniência:(De \textunderscore nóveas\textunderscore )}
\end{itemize}
Que satisfaz nóveas.
\section{Nóveas}
\begin{itemize}
\item {Grp. gram.:f. pl.}
\end{itemize}
O mesmo que \textunderscore anóveas\textunderscore .
\section{Novecentos}
\begin{itemize}
\item {Grp. gram.:adj.}
\end{itemize}
\begin{itemize}
\item {Proveniência:(De \textunderscore nove\textunderscore  + \textunderscore cento\textunderscore )}
\end{itemize}
Nove vezes cem.
\section{Novedio}
\begin{itemize}
\item {Grp. gram.:m.}
\end{itemize}
Renôvo, rebento, vergôntea.
\section{Novel}
\begin{itemize}
\item {Grp. gram.:adj.}
\end{itemize}
\begin{itemize}
\item {Proveniência:(Lat. \textunderscore novellus\textunderscore )}
\end{itemize}
O mesmo que novo.
Inexperiente; principiante; bisonho.
\section{Novela}
\begin{itemize}
\item {Grp. gram.:m.}
\end{itemize}
\begin{itemize}
\item {Grp. gram.:Pl.}
\end{itemize}
\begin{itemize}
\item {Proveniência:(Lat. \textunderscore novella\textunderscore )}
\end{itemize}
Romance curto, narração de aventuras interessantes ou recreativas.
Conto.
Enrêdo, intriga.
O mesmo que \textunderscore novelleiro\textunderscore .
Constituições imperiaes de Theodósio e seus successores; constituições do imperador Justiniano
\section{Noveleiro}
\begin{itemize}
\item {Grp. gram.:m.}
\end{itemize}
\begin{itemize}
\item {Utilização:Ant.}
\end{itemize}
\begin{itemize}
\item {Utilização:Bot.}
\end{itemize}
\begin{itemize}
\item {Proveniência:(Do lat. \textunderscore novellus\textunderscore )}
\end{itemize}
Rebento de árvore.
Vergôntea, que nasce ao pé do tronco da árvore; novedio.
Planta caprifoliácea , (\textunderscore viburnum\textunderscore , Lin.), também conhecida pelo nome de \textunderscore rosa-de-gueldres\textunderscore . Cf. P. Coutinho, \textunderscore Flora de Port.\textunderscore , 587.
\section{Noveleiro}
\begin{itemize}
\item {Grp. gram.:m.}
\end{itemize}
\begin{itemize}
\item {Grp. gram.:Adj.}
\end{itemize}
\begin{itemize}
\item {Proveniência:(De \textunderscore novela\textunderscore )}
\end{itemize}
O mesmo que \textunderscore novelista\textunderscore .
Que gosta de dar notícias; intriguista; trapaceiro.
\section{Novelesco}
\begin{itemize}
\item {fónica:lês}
\end{itemize}
\begin{itemize}
\item {Grp. gram.:adj.}
\end{itemize}
Próprio de novela; semelhante a novela. Cf. Latino, \textunderscore Camões\textunderscore , 77, 89, 151 e 242.
\section{Novelista}
\begin{itemize}
\item {Grp. gram.:m.  e  f.}
\end{itemize}
\begin{itemize}
\item {Grp. gram.:Adj.}
\end{itemize}
\begin{itemize}
\item {Proveniência:(De \textunderscore novela\textunderscore )}
\end{itemize}
Pessôa, que faz novelas.
Que gosta de dar notícias.
Que intriga, que enreda.
\section{Novelleiro}
\begin{itemize}
\item {Grp. gram.:m.}
\end{itemize}
\begin{itemize}
\item {Grp. gram.:Adj.}
\end{itemize}
\begin{itemize}
\item {Proveniência:(De \textunderscore novella\textunderscore )}
\end{itemize}
O mesmo que \textunderscore novellista\textunderscore .
Que gosta de dar notícias; intriguista; trapaceiro.
\section{Novellesco}
\begin{itemize}
\item {fónica:lês}
\end{itemize}
\begin{itemize}
\item {Grp. gram.:adj.}
\end{itemize}
Próprio de novella; semelhante a novella. Cf. Latino, \textunderscore Camões\textunderscore , 77, 89, 151 e 242.
\section{Novellista}
\begin{itemize}
\item {Grp. gram.:m.  e  f.}
\end{itemize}
\begin{itemize}
\item {Grp. gram.:Adj.}
\end{itemize}
\begin{itemize}
\item {Proveniência:(De \textunderscore novella\textunderscore )}
\end{itemize}
Pessôa, que faz novellas.
Que gosta de dar notícias.
Que intriga, que enreda.
\section{Novello}
\begin{itemize}
\item {fónica:vê}
\end{itemize}
\begin{itemize}
\item {Grp. gram.:m.}
\end{itemize}
\begin{itemize}
\item {Utilização:Fig.}
\end{itemize}
\begin{itemize}
\item {Grp. gram.:M. Pl.}
\end{itemize}
\begin{itemize}
\item {Proveniência:(Do lat. \textunderscore globellus\textunderscore )}
\end{itemize}
Bóla, formada de fio dobado.
Enrêdo.
Froco.
Planta lonicérea, (\textunderscore viburnum opulus\textunderscore , Lin.).
\textunderscore Novellos da China\textunderscore , o mesmo que \textunderscore hortênsia\textunderscore .
\section{Novembro}
\begin{itemize}
\item {Grp. gram.:m.}
\end{itemize}
\begin{itemize}
\item {Proveniência:(Lat. \textunderscore november\textunderscore )}
\end{itemize}
Décimo primeiro mês do anno, segundo o nosso calendário.
\section{Novemdial}
\begin{itemize}
\item {Grp. gram.:adj.}
\end{itemize}
\begin{itemize}
\item {Proveniência:(Lat. \textunderscore novemdialis\textunderscore )}
\end{itemize}
Dizia-se da festa que, nove dias depois da morte de alguém, era celebrada pela família do finado e que consistia em sacrifícios de expiação.
\section{Novemfoliado}
\begin{itemize}
\item {Grp. gram.:adj.}
\end{itemize}
\begin{itemize}
\item {Utilização:Bot.}
\end{itemize}
\begin{itemize}
\item {Proveniência:(Do lat. \textunderscore novem\textunderscore  + \textunderscore folium\textunderscore )}
\end{itemize}
Que tem nove folíolos.
\section{Novemlobado}
\begin{itemize}
\item {Grp. gram.:adj.}
\end{itemize}
\begin{itemize}
\item {Utilização:Bot.}
\end{itemize}
Que tem nove lóbulos.
\section{Novemnervado}
\begin{itemize}
\item {Grp. gram.:adj.}
\end{itemize}
\begin{itemize}
\item {Utilização:Bot.}
\end{itemize}
Que tem nove nervuras.
\section{Novemvirado}
\begin{itemize}
\item {Grp. gram.:m.}
\end{itemize}
Cargo de novêmviro.
\section{Novemvirato}
\begin{itemize}
\item {Grp. gram.:m.}
\end{itemize}
(V.novemvirado)
\section{Novêmviro}
\begin{itemize}
\item {Grp. gram.:m.}
\end{itemize}
\begin{itemize}
\item {Proveniência:(Do lat. \textunderscore novem\textunderscore  + \textunderscore vir\textunderscore )}
\end{itemize}
Cada um dos nove archontes em Athenas.
Cada um dos nove magistrados romanos, que velavam pela saúde pública.
\section{Novena}
\begin{itemize}
\item {Grp. gram.:f.}
\end{itemize}
\begin{itemize}
\item {Grp. gram.:Pl.}
\end{itemize}
Espaço de nove dias, em que se fazem certas ceremónias religiosas.
Ceremónias de cada um dêsses dias.
Grupo de nove coisas ou pessôas.
O mesmo que \textunderscore nóveas\textunderscore .
(Fem. de \textunderscore noveno\textunderscore )
\section{Novenal}
\begin{itemize}
\item {Grp. gram.:adj.}
\end{itemize}
Relativo a novena.
\section{Novenário}
\begin{itemize}
\item {Grp. gram.:m.}
\end{itemize}
\begin{itemize}
\item {Proveniência:(Lat. \textunderscore novenarius\textunderscore )}
\end{itemize}
Livro de novenas.
\section{Novença}
\begin{itemize}
\item {Grp. gram.:f.}
\end{itemize}
\begin{itemize}
\item {Utilização:Prov.}
\end{itemize}
\begin{itemize}
\item {Proveniência:(De \textunderscore novo\textunderscore )}
\end{itemize}
Renôvo, novedio.
\section{Novendial}
\begin{itemize}
\item {Grp. gram.:adj.}
\end{itemize}
\begin{itemize}
\item {Proveniência:(Lat. \textunderscore novemdialis\textunderscore )}
\end{itemize}
Dizia-se da festa que, nove dias depois da morte de alguém, era celebrada pela família do finado e que consistia em sacrifícios de expiação.
\section{Novenervado}
\begin{itemize}
\item {Grp. gram.:adj.}
\end{itemize}
\begin{itemize}
\item {Utilização:Bot.}
\end{itemize}
Que tem nove nervuras.
\section{Novenfoliado}
\begin{itemize}
\item {Grp. gram.:adj.}
\end{itemize}
\begin{itemize}
\item {Utilização:Bot.}
\end{itemize}
\begin{itemize}
\item {Proveniência:(Do lat. \textunderscore novem\textunderscore  + \textunderscore folium\textunderscore )}
\end{itemize}
Que tem nove folíolos.
\section{Novenlobado}
\begin{itemize}
\item {Grp. gram.:adj.}
\end{itemize}
\begin{itemize}
\item {Utilização:Bot.}
\end{itemize}
Que tem nove lóbulos.
\section{Novênnio}
\begin{itemize}
\item {Grp. gram.:m.}
\end{itemize}
Espaço de nove annos. Cf. Filinto, I, 147.
(Cp. lat. \textunderscore novennis\textunderscore )
\section{Noveno}
\begin{itemize}
\item {Grp. gram.:adj.}
\end{itemize}
\begin{itemize}
\item {Proveniência:(Lat. \textunderscore novenus\textunderscore )}
\end{itemize}
Diz-se do nono dia de uma doença.
O mesmo que \textunderscore nono\textunderscore ^1:«\textunderscore el-rei Carlos, noveno de França...\textunderscore »\textunderscore Jornada de África\textunderscore , c. III.
\section{Noventa}
\begin{itemize}
\item {Grp. gram.:adj.}
\end{itemize}
\begin{itemize}
\item {Proveniência:(Lat. \textunderscore nonaginta\textunderscore )}
\end{itemize}
Nove vezes déz.
\section{Novenvirado}
\begin{itemize}
\item {Grp. gram.:m.}
\end{itemize}
Cargo de novêmviro.
\section{Novenvirato}
\begin{itemize}
\item {Grp. gram.:m.}
\end{itemize}
(V.novemvirado)
\section{Novênviro}
\begin{itemize}
\item {Grp. gram.:m.}
\end{itemize}
\begin{itemize}
\item {Proveniência:(Do lat. \textunderscore novem\textunderscore  + \textunderscore vir\textunderscore )}
\end{itemize}
Cada um dos nove archontes em Athenas.
Cada um dos nove magistrados romanos, que velavam pela saúde pública.
\section{Novercal}
\begin{itemize}
\item {Grp. gram.:adj.}
\end{itemize}
\begin{itemize}
\item {Proveniência:(Lat. \textunderscore novercalis\textunderscore )}
\end{itemize}
Relativo a madrasta. Cf. Filinto, III, 103 e 231.
\section{Nóvi...}
\begin{itemize}
\item {Grp. gram.:pref.}
\end{itemize}
\begin{itemize}
\item {Proveniência:(Do lat. \textunderscore novus\textunderscore )}
\end{itemize}
(designativo de \textunderscore novo\textunderscore  e preferivel a \textunderscore néo...\textunderscore , mormente em palavras românicas ou romanizadas)
\section{Noviça}
\begin{itemize}
\item {Grp. gram.:f.}
\end{itemize}
\begin{itemize}
\item {Proveniência:(De \textunderscore noviço\textunderscore )}
\end{itemize}
Mulher, que se prepara para professar numa Ordem religiosa.
\section{Noviciado}
\begin{itemize}
\item {Grp. gram.:m.}
\end{itemize}
\begin{itemize}
\item {Utilização:Fig.}
\end{itemize}
\begin{itemize}
\item {Proveniência:(Do lat. \textunderscore novicius\textunderscore )}
\end{itemize}
Preparação ou exercícios, a que se sujeitam as pessôas que vão professar numa Ordem religiosa.
Tempo que duram esses exercícios.
Convento ou parte do convento, destinada aos noviços.
Noviciaria.
Apprendizado; tempo que elle dura.
\section{Noviciar}
\begin{itemize}
\item {Grp. gram.:v. i.}
\end{itemize}
\begin{itemize}
\item {Utilização:Ext.}
\end{itemize}
Praticar o noviciado.
Fazer os primeiros exercícios.
Iniciar-se; estrear-se:«\textunderscore foi alli que o estudante noviciou no amor\textunderscore ». Camillo, \textunderscore Volcões\textunderscore , 23.
(Cp. \textunderscore noviciado\textunderscore )
\section{Noviciaria}
\begin{itemize}
\item {Grp. gram.:f.}
\end{itemize}
\begin{itemize}
\item {Proveniência:(Do lat. \textunderscore novicius\textunderscore )}
\end{itemize}
Parte do convento, em que residem os noviços.
\section{Noviciário}
\begin{itemize}
\item {Grp. gram.:adj.}
\end{itemize}
Relativo a noviço.
\section{Noviço}
\begin{itemize}
\item {Grp. gram.:m.}
\end{itemize}
\begin{itemize}
\item {Utilização:Fig.}
\end{itemize}
\begin{itemize}
\item {Grp. gram.:Adj.}
\end{itemize}
\begin{itemize}
\item {Proveniência:(Lat. \textunderscore novicius\textunderscore )}
\end{itemize}
Homem, que se prepara para professar numa Ordem religiosa.
Principiante, apprendiz.
Inexperiente, novato.
\section{Novidade}
\begin{itemize}
\item {Grp. gram.:f.}
\end{itemize}
\begin{itemize}
\item {Utilização:Fam.}
\end{itemize}
\begin{itemize}
\item {Proveniência:(Lat. \textunderscore novitas\textunderscore )}
\end{itemize}
Qualidade do que é novo.
Notícia.
Informação.
Notícia má.
Motim, agitação.
Novos frutos do anno; colheita: \textunderscore a chuva tem feito mal á novidade\textunderscore .
\section{Novidadeiro}
\begin{itemize}
\item {Grp. gram.:m.}
\end{itemize}
Amigo de novidades; mexeriqueiro.
\section{Novidão}
\begin{itemize}
\item {Grp. gram.:f.}
\end{itemize}
\begin{itemize}
\item {Utilização:T. de Turquel}
\end{itemize}
Verduras da mocidade.
\section{Novídeo}
\begin{itemize}
\item {Grp. gram.:m.}
\end{itemize}
O mesmo que \textunderscore novedio\textunderscore .
\section{Nóvi-latino}
\begin{itemize}
\item {Grp. gram.:adj.}
\end{itemize}
O mesmo ou melhor que \textunderscore neó-latino\textunderscore .
\section{Novilha}
\begin{itemize}
\item {Grp. gram.:f.}
\end{itemize}
\begin{itemize}
\item {Utilização:Bras. do N}
\end{itemize}
\begin{itemize}
\item {Proveniência:(De \textunderscore novilho\textunderscore )}
\end{itemize}
Vaca de pouca idade.
Qualquer rês fêmea, que ainda não deu cria: \textunderscore novilha de cabra\textunderscore ; \textunderscore novilha de ovelha\textunderscore ; \textunderscore novilha de vaca\textunderscore .
Vaca, que completou três annos de idade.
\section{Novilhada}
\begin{itemize}
\item {Grp. gram.:f.}
\end{itemize}
Manada de novilhos.
Espectáculo de corrida de novilhos.
\section{Novilheiro}
\begin{itemize}
\item {Grp. gram.:m.}
\end{itemize}
Toireiro de novilhos. Cp. \textunderscore Século\textunderscore , de 15-IX-903.
\section{Novilho}
\begin{itemize}
\item {Grp. gram.:m.}
\end{itemize}
\begin{itemize}
\item {Utilização:Bras. do N}
\end{itemize}
Boi de pouca idade.
Boi, que completou três annos de idade.
(Cp. cast. \textunderscore novillo\textunderscore )
\section{Novilhota}
\begin{itemize}
\item {Grp. gram.:f.}
\end{itemize}
\begin{itemize}
\item {Utilização:Bras. do N}
\end{itemize}
\begin{itemize}
\item {Proveniência:(De \textunderscore novilha\textunderscore )}
\end{itemize}
Vaca, que completou dois annos de idade.
\section{Novilhote}
\begin{itemize}
\item {Grp. gram.:m.}
\end{itemize}
\begin{itemize}
\item {Utilização:Bras. do N}
\end{itemize}
\begin{itemize}
\item {Proveniência:(De \textunderscore novilho\textunderscore )}
\end{itemize}
Boi, que completou dois annos de idade.
\section{Novilunar}
\begin{itemize}
\item {Grp. gram.:adj.}
\end{itemize}
Relativo ao novilúnio.
\section{Novilúnio}
\begin{itemize}
\item {Grp. gram.:m.}
\end{itemize}
\begin{itemize}
\item {Proveniência:(Do lat. \textunderscore novus\textunderscore  + \textunderscore luna\textunderscore )}
\end{itemize}
Lua nova.
O tempo da Lua nova.
\section{Novimestre}
\begin{itemize}
\item {Grp. gram.:m.}
\end{itemize}
\begin{itemize}
\item {Utilização:Des.}
\end{itemize}
Espaço de nove meses.
(Palavra formada por analogia de \textunderscore trimestre\textunderscore  e \textunderscore semestre\textunderscore , de \textunderscore nove\textunderscore  + \textunderscore mês\textunderscore )
\section{Novissimamente}
\begin{itemize}
\item {Grp. gram.:adv.}
\end{itemize}
De modo novíssimo.
Ultimamente.
\section{Novíssimo}
\begin{itemize}
\item {Grp. gram.:adj.}
\end{itemize}
\begin{itemize}
\item {Grp. gram.:M. pl.}
\end{itemize}
\begin{itemize}
\item {Proveniência:(Lat. \textunderscore novissimus\textunderscore )}
\end{itemize}
Muito novo.
Último.
Últimos destinos do homem, segundo o Catholicismo.
\section{Novo}
\begin{itemize}
\item {Grp. gram.:adj.}
\end{itemize}
\begin{itemize}
\item {Grp. gram.:M.}
\end{itemize}
\begin{itemize}
\item {Grp. gram.:Pl.}
\end{itemize}
\begin{itemize}
\item {Utilização:Agr.}
\end{itemize}
\begin{itemize}
\item {Utilização:Ant.}
\end{itemize}
\begin{itemize}
\item {Utilização:Restrict.}
\end{itemize}
\begin{itemize}
\item {Proveniência:(Lat. \textunderscore novus\textunderscore )}
\end{itemize}
Que existe há pouco tempo.
Que ainda não serviu, que tem pouco uso: \textunderscore chapéu novo\textunderscore .
Que começa.
Que tem pouca idade: \textunderscore homem novo\textunderscore .
Outro: \textunderscore temos novo incidente\textunderscore .
Que accresce.
Estranho.
Visto pela primeira vez.
Não estreado.
Inexperiente.
Colheita próxima.
Anno novo.
Gente nova.
Renovos, novidades agrícolas.
Os literatos, que começam a manifestar-se.
\section{Novyorkino}
\begin{itemize}
\item {Grp. gram.:adj.}
\end{itemize}
\begin{itemize}
\item {Grp. gram.:M.}
\end{itemize}
Relativo a Nova-York.
Habitante de Nova-York.
\section{Noxa}
\begin{itemize}
\item {Grp. gram.:f.}
\end{itemize}
Árvore angolense.
\section{Nóxio}
\begin{itemize}
\item {fónica:csi}
\end{itemize}
\begin{itemize}
\item {Grp. gram.:adj.}
\end{itemize}
\begin{itemize}
\item {Proveniência:(Lat. \textunderscore noxius\textunderscore )}
\end{itemize}
O mesmo que \textunderscore nocivo\textunderscore .
\section{Noxiúra}
\begin{itemize}
\item {Grp. gram.:f.}
\end{itemize}
Árvore da Guiné, de fibras têxteis.
\section{Noz}
\begin{itemize}
\item {Grp. gram.:m.}
\end{itemize}
\begin{itemize}
\item {Utilização:Gír.}
\end{itemize}
\begin{itemize}
\item {Utilização:Ant.}
\end{itemize}
\begin{itemize}
\item {Proveniência:(Lat. \textunderscore nux\textunderscore )}
\end{itemize}
Fruto da nogueira.
A cabeça.
Peça de marfim ou de corno, em que assentava a corna da bésta, para depois se disparar a seta.
\section{Nozedo}
\begin{itemize}
\item {fónica:zê}
\end{itemize}
\begin{itemize}
\item {Grp. gram.:m.}
\end{itemize}
Casta de uva.
\section{Nozeira}
\begin{itemize}
\item {Grp. gram.:f.}
\end{itemize}
\begin{itemize}
\item {Utilização:Prov.}
\end{itemize}
\begin{itemize}
\item {Utilização:trasm.}
\end{itemize}
\begin{itemize}
\item {Proveniência:(De \textunderscore noz\textunderscore . Cp. gall. \textunderscore noceira\textunderscore )}
\end{itemize}
O mesmo que \textunderscore nogueira\textunderscore .
\section{Nozelha}
\begin{itemize}
\item {fónica:zê}
\end{itemize}
\begin{itemize}
\item {Grp. gram.:f.}
\end{itemize}
\begin{itemize}
\item {Utilização:Prov.}
\end{itemize}
\begin{itemize}
\item {Utilização:dur.}
\end{itemize}
\begin{itemize}
\item {Proveniência:(De \textunderscore noz\textunderscore )}
\end{itemize}
Bolbo de certas plantas.
\section{Noz-vómica}
\begin{itemize}
\item {Grp. gram.:f.}
\end{itemize}
Fruto venenoso de uma árvore indiana, do qual se extrai a estrychnina.
\section{Nozilhão}
\begin{itemize}
\item {Grp. gram.:m.}
\end{itemize}
\begin{itemize}
\item {Utilização:Pop.}
\end{itemize}
\begin{itemize}
\item {Proveniência:(De \textunderscore nó\textunderscore , dizem os diccionários; mas não repugna que derive de \textunderscore noz\textunderscore )}
\end{itemize}
Tumor, inchação.
\section{N. S.}
\begin{itemize}
\item {Grp. gram.:m.}
\end{itemize}
Abrev. de \textunderscore Nosso Senhor\textunderscore .
\section{Nu}
\begin{itemize}
\item {Grp. gram.:adj.}
\end{itemize}
\begin{itemize}
\item {Grp. gram.:M.}
\end{itemize}
\begin{itemize}
\item {Utilização:Pint.}
\end{itemize}
\begin{itemize}
\item {Proveniência:(Do lat. \textunderscore nudus\textunderscore )}
\end{itemize}
Que não está vestido.
Despido: \textunderscore rapaz nu\textunderscore .
Desfolhado: \textunderscore árvores nuas\textunderscore .
Descalço (falando-se de pés).
Descoberto: \textunderscore cabeça nua\textunderscore .
Escalvado, sem vegetação: \textunderscore oiteiros nus\textunderscore .
Tôsco.
Desguarnecido: \textunderscore uma casa nua\textunderscore .
Desaffectado, simples, sincero: \textunderscore verdade nua e crua\textunderscore .
Que não está na bainha, (falando-se da espada).
O mesmo que \textunderscore nudez\textunderscore : \textunderscore a pintura do nu\textunderscore .
Fem. \textunderscore nua\textunderscore .
\section{Nu}
\begin{itemize}
\item {Grp. gram.:m.}
\end{itemize}
Nome da letra que no alphabeto grego corresponde a \textunderscore n\textunderscore .
\section{Nuamente}
\begin{itemize}
\item {Grp. gram.:adv.}
\end{itemize}
De modo nu.
Em estado de nudez.
Sem atavios, sem enfeites; singelamente, com simplicidade.
\section{Nuaruaque}
\begin{itemize}
\item {Grp. gram.:m.}
\end{itemize}
Um dos idiomas da América do Sul.
\section{Nuba}
\begin{itemize}
\item {Grp. gram.:m.  e  adj.}
\end{itemize}
O mesmo que \textunderscore nubiano\textunderscore .
\section{Nubécula}
\begin{itemize}
\item {Grp. gram.:f.}
\end{itemize}
\begin{itemize}
\item {Proveniência:(Lat. \textunderscore nubecula\textunderscore )}
\end{itemize}
O mesmo que \textunderscore nephélio\textunderscore .
\section{Nubente}
\begin{itemize}
\item {Grp. gram.:m. ,  f.  e  adj.}
\end{itemize}
\begin{itemize}
\item {Proveniência:(Lat. \textunderscore nubens\textunderscore )}
\end{itemize}
Pessôa, que vai casar, que é noivo ou noiva.
\section{Nubiano}
\begin{itemize}
\item {Grp. gram.:adj.}
\end{itemize}
\begin{itemize}
\item {Grp. gram.:M.}
\end{itemize}
Relativo á Núbia.
Habitante da Núbia.
\section{Nubícogo}
\begin{itemize}
\item {Grp. gram.:adj.}
\end{itemize}
\begin{itemize}
\item {Utilização:Poét.}
\end{itemize}
\begin{itemize}
\item {Proveniência:(Do lat. \textunderscore nubes\textunderscore  + \textunderscore cogere\textunderscore )}
\end{itemize}
Que ajunta nuvens.
\section{Nubífero}
\begin{itemize}
\item {Grp. gram.:adj.}
\end{itemize}
\begin{itemize}
\item {Utilização:Poét.}
\end{itemize}
\begin{itemize}
\item {Proveniência:(Lat. \textunderscore nubifer\textunderscore )}
\end{itemize}
Que produz ou traz nuvens.
\section{Nubífugo}
\begin{itemize}
\item {Grp. gram.:adj.}
\end{itemize}
\begin{itemize}
\item {Proveniência:(Do lat. \textunderscore nubes\textunderscore  + \textunderscore fugere\textunderscore )}
\end{itemize}
Que espalha ou desfaz nuvens.
\section{Nubígena}
\begin{itemize}
\item {Grp. gram.:adj.}
\end{itemize}
O mesmo que \textunderscore nubígeno\textunderscore . Cf. F. Barreto, \textunderscore Eneida\textunderscore , VIII, 69.
\section{Nubígeno}
\begin{itemize}
\item {Grp. gram.:adj.}
\end{itemize}
\begin{itemize}
\item {Proveniência:(Lat. \textunderscore nubigenus\textunderscore )}
\end{itemize}
Que provém das nuvens.
\section{Núbil}
\begin{itemize}
\item {Grp. gram.:adj.}
\end{itemize}
\begin{itemize}
\item {Proveniência:(Lat. \textunderscore nubilis\textunderscore )}
\end{itemize}
Que está em idade de casar.
Casadoiro.
\section{Nubilar}
\begin{itemize}
\item {Grp. gram.:m.}
\end{itemize}
\begin{itemize}
\item {Proveniência:(Lat. \textunderscore nubilare\textunderscore )}
\end{itemize}
Lugar, onde se recolhe o trigo, quando se receia chuva.
\section{Nubilário}
\begin{itemize}
\item {Grp. gram.:m.}
\end{itemize}
O mesmo que \textunderscore nubilar\textunderscore .
\section{Nubilidade}
\begin{itemize}
\item {Grp. gram.:f.}
\end{itemize}
Qualidade de núbil; qualidade de casadoiro.
\section{Núbilo}
\begin{itemize}
\item {Grp. gram.:adj.}
\end{itemize}
\begin{itemize}
\item {Proveniência:(Lat. \textunderscore nubilus\textunderscore )}
\end{itemize}
Ennevoado, sombrio. Cf. Filinto, II, 202.
\section{Nubiloso}
\begin{itemize}
\item {Grp. gram.:adj.}
\end{itemize}
\begin{itemize}
\item {Proveniência:(Lat. \textunderscore nubilosus\textunderscore )}
\end{itemize}
O mesmo que \textunderscore nebuloso\textunderscore .
\section{Núbio}
\begin{itemize}
\item {Grp. gram.:m.  e  adj.}
\end{itemize}
O mesmo que \textunderscore nubiano\textunderscore .
\section{Nubívago}
\begin{itemize}
\item {Grp. gram.:adj.}
\end{itemize}
\begin{itemize}
\item {Utilização:Fig.}
\end{itemize}
\begin{itemize}
\item {Proveniência:(Lat. \textunderscore nubivagus\textunderscore )}
\end{itemize}
Que anda pelas nuvens.
Nephelibata.
Sublime.
\section{Nublar}
\begin{itemize}
\item {Grp. gram.:v. t.}
\end{itemize}
\begin{itemize}
\item {Utilização:Fig.}
\end{itemize}
\begin{itemize}
\item {Proveniência:(Do lat. \textunderscore nubilare\textunderscore )}
\end{itemize}
Cobrir de nuvens.
Tornar escuro; toldar; entristecer.
\section{Nubloso}
\begin{itemize}
\item {Grp. gram.:adj.}
\end{itemize}
O mesmo que \textunderscore nubiloso\textunderscore .
\section{Nubrina}
\begin{itemize}
\item {Grp. gram.:f.}
\end{itemize}
\begin{itemize}
\item {Utilização:Prov.}
\end{itemize}
\begin{itemize}
\item {Utilização:trasm.}
\end{itemize}
O mesmo que \textunderscore neblina\textunderscore .
(Cp. \textunderscore nuvre\textunderscore )
\section{Nuca}
\begin{itemize}
\item {Grp. gram.:f.}
\end{itemize}
Parte superior do cachaço, correspondente á vértebra cervical.
(Cp. it. \textunderscore nuca\textunderscore )
\section{Nucal}
\begin{itemize}
\item {Grp. gram.:adj.}
\end{itemize}
Relativo a nuca.
\section{Nução}
\begin{itemize}
\item {Grp. gram.:f.}
\end{itemize}
\begin{itemize}
\item {Proveniência:(Do lat. \textunderscore nutus\textunderscore )}
\end{itemize}
Assentimento.
Vontade, talante, arbítrio.
\section{Núcego}
\begin{itemize}
\item {Grp. gram.:adj.}
\end{itemize}
\begin{itemize}
\item {Utilização:Prov.}
\end{itemize}
\begin{itemize}
\item {Utilização:beir.}
\end{itemize}
Completamente despido, nu.
(Se não procede do lat. hypoth. \textunderscore nudicus\textunderscore  = \textunderscore nudulus\textunderscore , de \textunderscore nudus\textunderscore , é t. composto, cujo primeiro elemento é \textunderscore nu\textunderscore ; o segundo será problemático)
\section{Nucela}
\begin{itemize}
\item {Grp. gram.:f.}
\end{itemize}
O mesmo que \textunderscore núcula\textunderscore .
\section{Nucella}
\begin{itemize}
\item {Grp. gram.:f.}
\end{itemize}
O mesmo que \textunderscore núcula\textunderscore .
\section{Nuciforme}
\begin{itemize}
\item {Grp. gram.:adj.}
\end{itemize}
\begin{itemize}
\item {Proveniência:(Do lat. \textunderscore nux\textunderscore  + \textunderscore forma\textunderscore )}
\end{itemize}
Semelhante á noz.
\section{Núcio}
\begin{itemize}
\item {Grp. gram.:m.}
\end{itemize}
\begin{itemize}
\item {Utilização:Ant.}
\end{itemize}
Lutuosa que, por morte de um colono, os cavalleiros villãos, herdeiros, pagavam ao senhor do solo. Cf. Herculano, \textunderscore Hist. de Port.\textunderscore , IV, 295.
\section{Nucir}
\begin{itemize}
\item {Grp. gram.:v. i.}
\end{itemize}
\begin{itemize}
\item {Utilização:Obsol.}
\end{itemize}
Fazer mal; prejudicar.
(Por \textunderscore nocir\textunderscore , do lat. \textunderscore nocere\textunderscore )
\section{Nucívoro}
\begin{itemize}
\item {Grp. gram.:adj.}
\end{itemize}
\begin{itemize}
\item {Proveniência:(Do lat. \textunderscore nux\textunderscore  + \textunderscore vorare\textunderscore )}
\end{itemize}
Que se alimenta de nozes.
\section{Nucleal}
\begin{itemize}
\item {Grp. gram.:adj.}
\end{itemize}
Relativo a núcleo.
\section{Nuclear}
\begin{itemize}
\item {Grp. gram.:adj.}
\end{itemize}
Relativo a núcleo.
\section{Nucleário}
\begin{itemize}
\item {Grp. gram.:adj.}
\end{itemize}
\begin{itemize}
\item {Utilização:Bot.}
\end{itemize}
\begin{itemize}
\item {Proveniência:(De \textunderscore núcleo\textunderscore )}
\end{itemize}
Relativo ao miolo da noz ou de qualquer fruto.
\section{Nucleína}
\begin{itemize}
\item {Grp. gram.:f.}
\end{itemize}
\begin{itemize}
\item {Proveniência:(De \textunderscore núcleo\textunderscore )}
\end{itemize}
Substância albuminóide, que entra na composição do núcleo cellular.
Chromatina.
\section{Nucleínico}
\begin{itemize}
\item {Grp. gram.:adj.}
\end{itemize}
\begin{itemize}
\item {Proveniência:(De \textunderscore nucleína\textunderscore )}
\end{itemize}
Diz-se de um ácido medicamentoso cuja fórmula é C^{40}H^{54}N^{14}O^{27}P^4.
\section{Núcleo}
\begin{itemize}
\item {Grp. gram.:m.}
\end{itemize}
\begin{itemize}
\item {Utilização:Fig.}
\end{itemize}
\begin{itemize}
\item {Proveniência:(Lat. \textunderscore nucleus\textunderscore )}
\end{itemize}
Miolo da noz e de outros frutos.
A parte interior de uma céllula.
A parte central e mais densa da cabeça de um cometa.
Centro; ponto central.
Ponto essencial.
Séde principal.
A nata, a flôr, o melhor de qualquer coisa.
\section{Nucleobrânchios}
\begin{itemize}
\item {fónica:qui}
\end{itemize}
\begin{itemize}
\item {Grp. gram.:m. pl.}
\end{itemize}
Ordem de molluscos.
\section{Nucleobrânquios}
\begin{itemize}
\item {Grp. gram.:m. pl.}
\end{itemize}
Ordem de moluscos.
\section{Nucléolo}
\begin{itemize}
\item {Grp. gram.:m.}
\end{itemize}
\begin{itemize}
\item {Proveniência:(Lat. \textunderscore nucleolus\textunderscore )}
\end{itemize}
Manchazinha em um núcleo de céllula.
\section{Núcula}
\begin{itemize}
\item {Grp. gram.:f.}
\end{itemize}
\begin{itemize}
\item {Proveniência:(Lat. \textunderscore nucula\textunderscore )}
\end{itemize}
Pequena noz, ou semente do nuculano.
Gênero de molluscos acéphalos.
\section{Nuculâneo}
\begin{itemize}
\item {Grp. gram.:adj.}
\end{itemize}
\begin{itemize}
\item {Proveniência:(De \textunderscore nuculano\textunderscore )}
\end{itemize}
Diz-se do fruto, que tem muitas sementes distintas, como a nêspera.
\section{Nuculano}
\begin{itemize}
\item {Grp. gram.:m.}
\end{itemize}
Fruto, que tem núculas em lóculos ligados ou livres.
\section{Nucular}
\begin{itemize}
\item {Grp. gram.:adj.}
\end{itemize}
\begin{itemize}
\item {Utilização:Bot.}
\end{itemize}
\begin{itemize}
\item {Proveniência:(De \textunderscore núcula\textunderscore )}
\end{itemize}
Relativo a noz; que encerra uma noz.
\section{Nuculoso}
\begin{itemize}
\item {Grp. gram.:adj.}
\end{itemize}
\begin{itemize}
\item {Utilização:Bot.}
\end{itemize}
\begin{itemize}
\item {Proveniência:(De \textunderscore núcula\textunderscore )}
\end{itemize}
Que contém pequenas nozes.
\section{Nudá}
\begin{itemize}
\item {Grp. gram.:m.}
\end{itemize}
Árvore indiana, muito applicada em construcções, (\textunderscore calophylum-inophilum\textunderscore ).
\section{Nudação}
\begin{itemize}
\item {Grp. gram.:f.}
\end{itemize}
\begin{itemize}
\item {Proveniência:(Lat. \textunderscore nudatio\textunderscore )}
\end{itemize}
O mesmo que \textunderscore nudez\textunderscore .
Acto ou effeito de desnudar.
\section{Nudamente}
\begin{itemize}
\item {Grp. gram.:adv.}
\end{itemize}
\begin{itemize}
\item {Proveniência:(Do lat. \textunderscore nudus\textunderscore )}
\end{itemize}
O mesmo que \textunderscore nuamente\textunderscore .
\section{Nudária}
\begin{itemize}
\item {Grp. gram.:f.}
\end{itemize}
Gênero de insectos lepidópteros nocturnos.
\section{Nudez}
\begin{itemize}
\item {Grp. gram.:f.}
\end{itemize}
\begin{itemize}
\item {Proveniência:(Do lat. \textunderscore nudus\textunderscore )}
\end{itemize}
Estado daquelle ou daquillo que se acha nu.
Falta ou ausência de vestuário.
Estado das plantas ou ramos, a que caíram as fôlhas.
Ausência de ornatos.
Privação.
Simplicidade.
\section{Nudeza}
\begin{itemize}
\item {Grp. gram.:f.}
\end{itemize}
O mesmo que \textunderscore nudez\textunderscore .
\section{Nudibrânchio}
\begin{itemize}
\item {fónica:qui}
\end{itemize}
\begin{itemize}
\item {Grp. gram.:adj.}
\end{itemize}
\begin{itemize}
\item {Proveniência:(Do lat. \textunderscore nudus\textunderscore  + \textunderscore branchiae\textunderscore )}
\end{itemize}
Diz-se dos animaes, que têm descobertas as brânchias.
\section{Nudibrânquio}
\begin{itemize}
\item {Grp. gram.:adj.}
\end{itemize}
\begin{itemize}
\item {Proveniência:(Do lat. \textunderscore nudus\textunderscore  + \textunderscore branchiae\textunderscore )}
\end{itemize}
Diz-se dos animaes, que têm descobertas as brânquias.
\section{Nudicaule}
\begin{itemize}
\item {Grp. gram.:adj}
\end{itemize}
\begin{itemize}
\item {Utilização:Bot.}
\end{itemize}
\begin{itemize}
\item {Proveniência:(Do lat. \textunderscore nudus\textunderscore  + \textunderscore caulis\textunderscore )}
\end{itemize}
Que não tem fôlhas no caule.
\section{Nudíparo}
\begin{itemize}
\item {Grp. gram.:adj.}
\end{itemize}
\begin{itemize}
\item {Utilização:Zool.}
\end{itemize}
\begin{itemize}
\item {Proveniência:(Do lat. \textunderscore nudus\textunderscore  + \textunderscore parere\textunderscore )}
\end{itemize}
Diz-se dos animaes ovíparos, cujos ovos se abrem no seio da mãe, onde por algum tempo fica o embryão.
\section{Nudípede}
\begin{itemize}
\item {Grp. gram.:adj.}
\end{itemize}
\begin{itemize}
\item {Proveniência:(Do lat. \textunderscore nudus\textunderscore  + \textunderscore pes\textunderscore , \textunderscore pedis\textunderscore )}
\end{itemize}
Que tem os pés nus.
\section{Nudisexo}
\begin{itemize}
\item {fónica:se}
\end{itemize}
\begin{itemize}
\item {Grp. gram.:adj.}
\end{itemize}
\begin{itemize}
\item {Utilização:Bot.}
\end{itemize}
\begin{itemize}
\item {Proveniência:(Do lat. \textunderscore nudus\textunderscore  + \textunderscore sexus\textunderscore )}
\end{itemize}
Diz-se das flôres, cujos órgãos sexuaes estão descobertos.
\section{Nudissexo}
\begin{itemize}
\item {fónica:cso}
\end{itemize}
\begin{itemize}
\item {Grp. gram.:adj.}
\end{itemize}
\begin{itemize}
\item {Utilização:Bot.}
\end{itemize}
\begin{itemize}
\item {Proveniência:(Do lat. \textunderscore nudus\textunderscore  + \textunderscore sexus\textunderscore )}
\end{itemize}
Diz-se das flôres, cujos órgãos sexuaes estão descobertos.
\section{Nuditarso}
\begin{itemize}
\item {Grp. gram.:adj.}
\end{itemize}
\begin{itemize}
\item {Proveniência:(Do lat. \textunderscore nudus\textunderscore  + gr. \textunderscore tarsos\textunderscore )}
\end{itemize}
Que tem os tarsos nus.
\section{Nudiúsculo}
\begin{itemize}
\item {Grp. gram.:adj.}
\end{itemize}
\begin{itemize}
\item {Utilização:Bot.}
\end{itemize}
\begin{itemize}
\item {Proveniência:(Do lat. \textunderscore nudus\textunderscore )}
\end{itemize}
Quási nu.
\section{Nuelo}
\begin{itemize}
\item {Grp. gram.:adj.}
\end{itemize}
\begin{itemize}
\item {Proveniência:(De \textunderscore nu\textunderscore )}
\end{itemize}
Recém-nascido; implume.
\section{Nueza}
\begin{itemize}
\item {Grp. gram.:f.}
\end{itemize}
\begin{itemize}
\item {Proveniência:(De \textunderscore nu\textunderscore )}
\end{itemize}
O mesmo que nudez. Cf. Filinto, X, 115.
\section{Nictagíneas}
\begin{itemize}
\item {Grp. gram.:f. pl.}
\end{itemize}
\begin{itemize}
\item {Proveniência:(De \textunderscore nictagíneo\textunderscore )}
\end{itemize}
Família de plantas, que tem por tipo as bôas-noites.
\section{Nictagíneo}
\begin{itemize}
\item {Grp. gram.:adj.}
\end{itemize}
\begin{itemize}
\item {Proveniência:(De \textunderscore nictago\textunderscore )}
\end{itemize}
Relativo ou semelhante ás bôas-noites.
\section{Nictago}
\begin{itemize}
\item {Grp. gram.:m.}
\end{itemize}
\begin{itemize}
\item {Proveniência:(Do gr. \textunderscore nux\textunderscore , \textunderscore nuktos\textunderscore )}
\end{itemize}
Nome científico da planta bôas-noites.
\section{Nictalemo}
\begin{itemize}
\item {Grp. gram.:m.}
\end{itemize}
Gênero de insectos lepidópteros.
\section{Níctalo}
\begin{itemize}
\item {Grp. gram.:m.}
\end{itemize}
\begin{itemize}
\item {Proveniência:(Gr. \textunderscore nuktalos\textunderscore )}
\end{itemize}
Sub-gênero de mamíferos, que abrange quatro espécies.
\section{Nictálope}
\begin{itemize}
\item {Grp. gram.:m.  e  f.}
\end{itemize}
\begin{itemize}
\item {Proveniência:(Do gr. \textunderscore nuktalos\textunderscore  + \textunderscore ops\textunderscore )}
\end{itemize}
Pessôa que não vê durante o dia e que só distingue objectos quando escurece ou anoitece.
\section{Nictalopia}
\begin{itemize}
\item {Grp. gram.:f.}
\end{itemize}
Estado do que é nictalope.
\section{Nictalópico}
\begin{itemize}
\item {Grp. gram.:adj.}
\end{itemize}
Relativo á nictalopia.
\section{Nictanto}
\begin{itemize}
\item {Grp. gram.:m.}
\end{itemize}
\begin{itemize}
\item {Utilização:Bot.}
\end{itemize}
\begin{itemize}
\item {Proveniência:(Do gr. \textunderscore nuktos\textunderscore  + \textunderscore anthos\textunderscore )}
\end{itemize}
Arbusto trepador, jasmináceo, chamado também \textunderscore jasmim da Arábia\textunderscore .
\section{Nictélias}
\begin{itemize}
\item {Grp. gram.:f. pl.}
\end{itemize}
\begin{itemize}
\item {Proveniência:(Do lat. \textunderscore nictélius\textunderscore )}
\end{itemize}
Festas de Baccho, que se celebravam de noite, á luz de archotes.
\section{Nictêmero}
\begin{itemize}
\item {Grp. gram.:m.}
\end{itemize}
\begin{itemize}
\item {Proveniência:(Do gr. \textunderscore nux\textunderscore  + \textunderscore hemera\textunderscore )}
\end{itemize}
Espaço de tempo, que abrange dia e noite em 24 horas.
\section{Nicterino}
\begin{itemize}
\item {Grp. gram.:m.}
\end{itemize}
Gênero de insectos coleópteros heterâmeros.
\section{Nictíceo}
\begin{itemize}
\item {Grp. gram.:m.}
\end{itemize}
Gênero de mamíferos chirópteros.
\section{Nictícora}
\begin{itemize}
\item {Grp. gram.:f.}
\end{itemize}
\begin{itemize}
\item {Proveniência:(Do gr. \textunderscore nuktikorax\textunderscore )}
\end{itemize}
Ave, o mesmo que \textunderscore mocho\textunderscore .
\section{Nictímero}
\begin{itemize}
\item {Grp. gram.:m.}
\end{itemize}
O mesmo que \textunderscore nictêmero\textunderscore .
\section{Nictiríbia}
\begin{itemize}
\item {Grp. gram.:f.}
\end{itemize}
\begin{itemize}
\item {Proveniência:(Do gr. \textunderscore nukteros\textunderscore  + \textunderscore bios\textunderscore )}
\end{itemize}
Gênero de insectos dípteros.
\section{Nomismal}
\begin{itemize}
\item {Grp. gram.:adj.}
\end{itemize}
Relativo ou semelhante a nomisma.
\section{Nomismática}
\begin{itemize}
\item {Grp. gram.:f.}
\end{itemize}
\begin{itemize}
\item {Proveniência:(De \textunderscore nomismático\textunderscore )}
\end{itemize}
Ciência, que trata das moédas e medalhas.
\section{Nomismático}
\begin{itemize}
\item {Grp. gram.:adj.}
\end{itemize}
\begin{itemize}
\item {Grp. gram.:M.}
\end{itemize}
\begin{itemize}
\item {Proveniência:(Gr. \textunderscore nomismatikos\textunderscore )}
\end{itemize}
Relativo á Nomismática.
Relativo a medalhas ou moédas.
Aquele que se dedica ao estudo da Nomismática.
\section{Nomismatografia}
\begin{itemize}
\item {Grp. gram.:f.}
\end{itemize}
\begin{itemize}
\item {Proveniência:(De \textunderscore nomismatógrafo\textunderscore )}
\end{itemize}
Tratado nomismático.
Descrição e história de moédas e medalhas.
\section{Nomismatográfico}
\begin{itemize}
\item {Grp. gram.:adj.}
\end{itemize}
Relativo á nomismatografia.
\section{Nomismatógrafo}
\begin{itemize}
\item {Grp. gram.:m.}
\end{itemize}
\begin{itemize}
\item {Proveniência:(Do gr. \textunderscore nomisma\textunderscore  + \textunderscore graphein\textunderscore )}
\end{itemize}
Aquele que é versado em nomismatografia.
\section{Núfar}
\begin{itemize}
\item {Grp. gram.:m.}
\end{itemize}
O mesmo que \textunderscore nenúfar\textunderscore .
Gôlfão amarelo.
\section{Nuga}
\begin{itemize}
\item {Grp. gram.:f.}
\end{itemize}
\begin{itemize}
\item {Proveniência:(Lat. \textunderscore nugae\textunderscore )}
\end{itemize}
Insignificância, bagatela, ninharia.
\section{Nugação}
\begin{itemize}
\item {Grp. gram.:f.}
\end{itemize}
\begin{itemize}
\item {Proveniência:(De \textunderscore nuga\textunderscore )}
\end{itemize}
Argumento vão.
Sophisma ridículo.
\section{Nugacidade}
\begin{itemize}
\item {Grp. gram.:f.}
\end{itemize}
\begin{itemize}
\item {Proveniência:(Lat. \textunderscore nugacitas\textunderscore )}
\end{itemize}
Nuga; frivolidade; futilidade.
Gracejo.
\section{Nugativo}
\begin{itemize}
\item {Grp. gram.:adj.}
\end{itemize}
Em que há nuga; frívolo; ridículo.
\section{Nugatório}
\begin{itemize}
\item {Grp. gram.:adj.}
\end{itemize}
\begin{itemize}
\item {Proveniência:(Lat. \textunderscore nugatorius\textunderscore )}
\end{itemize}
O mesmo que \textunderscore nugativo\textunderscore .
\section{Nuidade}
\begin{itemize}
\item {fónica:nu-i}
\end{itemize}
\begin{itemize}
\item {Grp. gram.:f.}
\end{itemize}
\begin{itemize}
\item {Utilização:Ant.}
\end{itemize}
O mesmo que \textunderscore nudez\textunderscore .
\section{Nulamente}
\begin{itemize}
\item {Grp. gram.:adv.}
\end{itemize}
De modo nulo; sem efeito, sem resultado.
De modo nenhum.
\section{Nulidade}
\begin{itemize}
\item {Grp. gram.:f.}
\end{itemize}
Qualidade do que não é válido.
Falta de condição ou condições necessárias para que tenha valor legal.
Falta de aptidão ou de talento.
Pessôa, que não tem mérito nenhum.
Insignificância, ninharia.
\section{Nulificar}
\begin{itemize}
\item {Grp. gram.:v. t.}
\end{itemize}
\begin{itemize}
\item {Proveniência:(Do lat. \textunderscore nullus\textunderscore  + \textunderscore facere\textunderscore )}
\end{itemize}
(mais us. no Brasil que em Portugal)
O mesmo que \textunderscore anular\textunderscore ^1.
\section{Nulinerve}
\begin{itemize}
\item {Grp. gram.:adj.}
\end{itemize}
\begin{itemize}
\item {Utilização:Bot.}
\end{itemize}
\begin{itemize}
\item {Proveniência:(Do lat. \textunderscore nullus\textunderscore  + \textunderscore nervus\textunderscore )}
\end{itemize}
Diz-se das fôlhas, que não têm nervuras.
\section{Nulípara}
\begin{itemize}
\item {Grp. gram.:adj. f.}
\end{itemize}
\begin{itemize}
\item {Proveniência:(Do lat. \textunderscore nullus\textunderscore  + \textunderscore parere\textunderscore )}
\end{itemize}
Diz-se da fêmea, que nunca pariu.
\section{Nullamente}
\begin{itemize}
\item {Grp. gram.:adv.}
\end{itemize}
De modo nulo; sem effeito, sem resultado.
De modo nenhum.
\section{Nullidade}
\begin{itemize}
\item {Grp. gram.:f.}
\end{itemize}
Qualidade do que não é válido.
Falta de condição ou condições necessárias para que tenha valor legal.
Falta de aptidão ou de talento.
Pessôa, que não tem mérito nenhum.
Insignificância, ninharia.
\section{Nullificar}
\begin{itemize}
\item {Grp. gram.:v. t.}
\end{itemize}
\begin{itemize}
\item {Proveniência:(Do lat. \textunderscore nullus\textunderscore  + \textunderscore facere\textunderscore )}
\end{itemize}
(mais us. no Brasil que em Portugal)
O mesmo que \textunderscore annullar\textunderscore .
\section{Nullinerve}
\begin{itemize}
\item {Grp. gram.:adj.}
\end{itemize}
\begin{itemize}
\item {Utilização:Bot.}
\end{itemize}
\begin{itemize}
\item {Proveniência:(Do lat. \textunderscore nullus\textunderscore  + \textunderscore nervus\textunderscore )}
\end{itemize}
Diz-se das fôlhas, que não têm nervuras.
\section{Nullípara}
\begin{itemize}
\item {Grp. gram.:adj. f.}
\end{itemize}
\begin{itemize}
\item {Proveniência:(Do lat. \textunderscore nullus\textunderscore  + \textunderscore parere\textunderscore )}
\end{itemize}
Diz-se da fêmea, que nunca pariu.
\section{Nullo}
\begin{itemize}
\item {Grp. gram.:adj.}
\end{itemize}
\begin{itemize}
\item {Proveniência:(Lat. \textunderscore nullus\textunderscore )}
\end{itemize}
Nenhum.
Que não é válido; que não tem effeito ou valor: \textunderscore contrato nullo\textunderscore .
Frívolo.
Vão.
Inerte.
\section{Nulo}
\begin{itemize}
\item {Grp. gram.:adj.}
\end{itemize}
\begin{itemize}
\item {Proveniência:(Lat. \textunderscore nullus\textunderscore )}
\end{itemize}
Nenhum.
Que não é válido; que não tem efeito ou valor: \textunderscore contrato nulo\textunderscore .
Frívolo.
Vão.
Inerte.
\section{Num}
Expressão contrahida, equivalente a \textunderscore em um\textunderscore .
Incorrectamente, escreve-se \textunderscore n'um\textunderscore .
(Cp. \textunderscore no\textunderscore ^1)
\section{Num}
\begin{itemize}
\item {Grp. gram.:adv.}
\end{itemize}
\begin{itemize}
\item {Utilização:Pop.}
\end{itemize}
O mesmo que \textunderscore não\textunderscore ^1.
\section{Numa}
Expressão contrahida, equivalente a \textunderscore em uma\textunderscore .
(Cp. \textunderscore num\textunderscore ^1)
\section{Numantino}
\begin{itemize}
\item {Grp. gram.:adj.}
\end{itemize}
\begin{itemize}
\item {Grp. gram.:M.}
\end{itemize}
Relativo a Numância.
Habitante de Numância.
\section{Numária}
\begin{itemize}
\item {Grp. gram.:f.}
\end{itemize}
\begin{itemize}
\item {Proveniência:(De \textunderscore numário\textunderscore )}
\end{itemize}
O mesmo que \textunderscore numismática\textunderscore .
\section{Numário}
\begin{itemize}
\item {Grp. gram.:adj.}
\end{itemize}
\begin{itemize}
\item {Proveniência:(Lat. \textunderscore numarius\textunderscore )}
\end{itemize}
Relativo á numária ou a moédas e medalhas.
\section{Numbela}
\begin{itemize}
\item {Grp. gram.:f.}
\end{itemize}
Espécie de tordo, (\textunderscore crateropus melanops\textunderscore ).
\section{Numbella}
\begin{itemize}
\item {Grp. gram.:f.}
\end{itemize}
Espécie de tordo, (\textunderscore crateropus melanops\textunderscore ).
\section{Nume}
\begin{itemize}
\item {Grp. gram.:m.}
\end{itemize}
\begin{itemize}
\item {Proveniência:(Lat. \textunderscore numen\textunderscore )}
\end{itemize}
Divindade mythológica.
Divindade.
Cada um dos deuses do Paganismo.
Gênio; inspiração.
\section{Numeração}
\begin{itemize}
\item {Grp. gram.:f.}
\end{itemize}
\begin{itemize}
\item {Proveniência:(Lat. \textunderscore numeratio\textunderscore )}
\end{itemize}
Acto ou effeito de numerar.
Arte de lêr e escrever os números.
\section{Numerado}
\begin{itemize}
\item {Grp. gram.:adj.}
\end{itemize}
Indicado por números.
Pôsto em ordem numérica.
\section{Numerador}
\begin{itemize}
\item {Grp. gram.:m.}
\end{itemize}
\begin{itemize}
\item {Proveniência:(Lat. \textunderscore numerator\textunderscore )}
\end{itemize}
O que numera.
O número, que que indica as partes da unidade contidas numa fracção.
Instrumento, para numerar livros, papéis, etc.
\section{Numeral}
\begin{itemize}
\item {Grp. gram.:adj.}
\end{itemize}
\begin{itemize}
\item {Proveniência:(Lat. \textunderscore numeralis\textunderscore )}
\end{itemize}
Relativo a número; que indíca número: \textunderscore vinte é adjectivo numeral\textunderscore .
\section{Numeralmente}
\begin{itemize}
\item {Grp. gram.:adj.}
\end{itemize}
De modo numeral; por meio de números.
\section{Numerar}
\begin{itemize}
\item {Grp. gram.:v. t.}
\end{itemize}
\begin{itemize}
\item {Proveniência:(Lat. \textunderscore numerarius\textunderscore )}
\end{itemize}
Dispor por ordem numérica.
Indicar por meio de números.
Expor methodicamente.
Contar.
Apreciar.
\section{Numerário}
\begin{itemize}
\item {Grp. gram.:adj.}
\end{itemize}
\begin{itemize}
\item {Grp. gram.:M.}
\end{itemize}
\begin{itemize}
\item {Proveniência:(Lat. \textunderscore numerarius\textunderscore )}
\end{itemize}
Relativo a dinheiro.
Moéda cunhada; dinheiro effectivo.
\section{Numerativo}
\begin{itemize}
\item {Grp. gram.:adj.}
\end{itemize}
O mesmo que \textunderscore numeral\textunderscore .
\section{Numerável}
\begin{itemize}
\item {Grp. gram.:adj.}
\end{itemize}
\begin{itemize}
\item {Proveniência:(Lat. \textunderscore numerabilis\textunderscore )}
\end{itemize}
Que se póde numerar.
\section{Numéria}
\begin{itemize}
\item {Grp. gram.:f.}
\end{itemize}
Gênero de insectos lepidópteros nocturnos.
\section{Numericamente}
\begin{itemize}
\item {Grp. gram.:adv.}
\end{itemize}
De modo numérico.
Em números; por meio de números.
\section{Numérico}
\begin{itemize}
\item {Grp. gram.:adj.}
\end{itemize}
Relativo a números; que indica número; numeral.
\section{Número}
\begin{itemize}
\item {Grp. gram.:m.}
\end{itemize}
\begin{itemize}
\item {Utilização:Gram.}
\end{itemize}
\begin{itemize}
\item {Proveniência:(Lat. \textunderscore numerus\textunderscore )}
\end{itemize}
A unidade.
Uma collecção de unidades.
As partes da unidade.
Expressão de quantidade.
Série.
Conta certa.
Algarismo, que numa série indica um lugar de ordem.
Porção; abundância, quantidade.
Cada uma das fôlhas ou cadernos de uma publicação, geralmente periódica e distribuída em mais ou menos exemplares com o mesmo texto.
Fórma, que indica que um nome ou um verbo diz respeito a um ou mais objectos.
Harmonia, resultante de certa disposição das palavras, na prosa ou no verso.
Cadência; regularidade.
\section{Numerosamente}
\begin{itemize}
\item {Grp. gram.:adv.}
\end{itemize}
De modo numeroso.
Em grande número.
\section{Numerosidade}
\begin{itemize}
\item {Grp. gram.:f.}
\end{itemize}
Qualidade de numeroso; grande número.
\section{Numeroso}
\begin{itemize}
\item {Grp. gram.:adj.}
\end{itemize}
\begin{itemize}
\item {Proveniência:(Lat. \textunderscore numerosus\textunderscore )}
\end{itemize}
Que é em grande número.
Abundante.
Harmonioso, suave: \textunderscore linguagem numerosa\textunderscore .
\section{Númida}
\begin{itemize}
\item {Grp. gram.:adj.}
\end{itemize}
\begin{itemize}
\item {Grp. gram.:M.}
\end{itemize}
\begin{itemize}
\item {Proveniência:(Lat. \textunderscore numidae\textunderscore )}
\end{itemize}
Relativo á Numídia.
Habitante da Numídia.
\section{Numídico}
\begin{itemize}
\item {Grp. gram.:adj.}
\end{itemize}
Relativo á Numídia ou aos númidas.
\section{Numiforme}
\begin{itemize}
\item {Grp. gram.:adj.}
\end{itemize}
\begin{itemize}
\item {Proveniência:(Do lat. \textunderscore numus\textunderscore  + \textunderscore forma\textunderscore )}
\end{itemize}
O mesmo que \textunderscore numismal\textunderscore .
\section{Numisma}
\begin{itemize}
\item {Grp. gram.:f.}
\end{itemize}
\begin{itemize}
\item {Proveniência:(Lat. \textunderscore numisma\textunderscore  e \textunderscore nomisma\textunderscore )}
\end{itemize}
Moéda cunhada.
\section{Numismal}
\begin{itemize}
\item {Grp. gram.:adj.}
\end{itemize}
Relativo ou semelhante a numisma.
\section{Numismata}
\begin{itemize}
\item {Grp. gram.:m.}
\end{itemize}
(V. \textunderscore numismático\textunderscore , m.)
\section{Numismática}
\begin{itemize}
\item {Grp. gram.:f.}
\end{itemize}
\begin{itemize}
\item {Proveniência:(De \textunderscore numismático\textunderscore )}
\end{itemize}
Sciência, que trata das moédas e medalhas.
\section{Numismático}
\begin{itemize}
\item {Grp. gram.:adj.}
\end{itemize}
\begin{itemize}
\item {Grp. gram.:M.}
\end{itemize}
\begin{itemize}
\item {Proveniência:(Gr. \textunderscore nomismatikos\textunderscore )}
\end{itemize}
Relativo á Numismática.
Relativo a medalhas ou moédas.
Aquelle que se dedica ao estudo da Numismática.
\section{Numismatographia}
\begin{itemize}
\item {Grp. gram.:f.}
\end{itemize}
\begin{itemize}
\item {Proveniência:(De \textunderscore numismatógrapho\textunderscore )}
\end{itemize}
Tratado numismático.
Descripção e história de moédas e medalhas.
\section{Numismatográphico}
\begin{itemize}
\item {Grp. gram.:adj.}
\end{itemize}
Relativo á numismatographia.
\section{Numismatógrapho}
\begin{itemize}
\item {Grp. gram.:m.}
\end{itemize}
\begin{itemize}
\item {Proveniência:(Do gr. \textunderscore nomisma\textunderscore  + \textunderscore graphein\textunderscore )}
\end{itemize}
Aquelle que é versado em numismatographia.
\section{Nummo}
\begin{itemize}
\item {Grp. gram.:m.}
\end{itemize}
\begin{itemize}
\item {Proveniência:(Lat. \textunderscore nummus\textunderscore )}
\end{itemize}
Moéda; dinheiro.
\section{Numo}
\begin{itemize}
\item {Grp. gram.:m.}
\end{itemize}
\begin{itemize}
\item {Proveniência:(Lat. \textunderscore nummus\textunderscore )}
\end{itemize}
Moéda; dinheiro.
\section{Numular}
\begin{itemize}
\item {Grp. gram.:adj.}
\end{itemize}
O mesmo que \textunderscore numismal\textunderscore .
(Cp. \textunderscore numulário\textunderscore )
\section{Numulária}
\begin{itemize}
\item {Grp. gram.:f.}
\end{itemize}
\begin{itemize}
\item {Proveniência:(Do lat. \textunderscore numularius\textunderscore )}
\end{itemize}
O mesmo que \textunderscore lysimáchia\textunderscore .
O mesmo que \textunderscore numária\textunderscore .
\section{Numulário}
\begin{itemize}
\item {Grp. gram.:m.}
\end{itemize}
\begin{itemize}
\item {Utilização:P. us.}
\end{itemize}
\begin{itemize}
\item {Proveniência:(Lat. \textunderscore numularius\textunderscore )}
\end{itemize}
Banqueiro.
Argentário; capitalista. Cf. Latino, \textunderscore Vasco da Gama\textunderscore , II, 232.
\section{Nunca}
\begin{itemize}
\item {Grp. gram.:adv.}
\end{itemize}
\begin{itemize}
\item {Proveniência:(Lat. \textunderscore nunquam\textunderscore )}
\end{itemize}
Em nenhum tempo; jamais.
Não.
\section{Nuncas}
\begin{itemize}
\item {Grp. gram.:adv.}
\end{itemize}
\begin{itemize}
\item {Utilização:Ant.}
\end{itemize}
O mesmo que \textunderscore nunca\textunderscore . Cf. Viterbo, \textunderscore Elucidário\textunderscore .
\section{Núncia}
\begin{itemize}
\item {Grp. gram.:f.}
\end{itemize}
\begin{itemize}
\item {Proveniência:(De \textunderscore núncio\textunderscore )}
\end{itemize}
Annunciadora; precursora.
\section{Nunciativo}
\begin{itemize}
\item {Grp. gram.:adj.}
\end{itemize}
\begin{itemize}
\item {Proveniência:(Do lat. \textunderscore nunciatus\textunderscore )}
\end{itemize}
Que contém notícia ou participação de alguma coisa.
\section{Nunciatura}
\begin{itemize}
\item {Grp. gram.:f.}
\end{itemize}
\begin{itemize}
\item {Proveniência:(De \textunderscore núncio\textunderscore )}
\end{itemize}
Dignidade de Núncio apostólico.
Residência do Núncio.
Tribunal ecclesiástico, sujeito ao Núncio.
\section{Núncio}
\begin{itemize}
\item {Grp. gram.:m.}
\end{itemize}
\begin{itemize}
\item {Proveniência:(Lat. \textunderscore nuncius\textunderscore )}
\end{itemize}
Annunciador.
Mensageiro.
Embaixador do Papa.
Prenúncio.
\section{Núncio}
\begin{itemize}
\item {Grp. gram.:m.}
\end{itemize}
\begin{itemize}
\item {Utilização:Ant.}
\end{itemize}
Lutuosa que, por morte de um colono, os cavalleiros villãos, herdeiros, pagavam ao senhor do solo. Cf. Herculano, \textunderscore Hist. de Port.\textunderscore , IV, 295.
\section{Nuncupação}
\begin{itemize}
\item {Grp. gram.:f.}
\end{itemize}
\begin{itemize}
\item {Utilização:Jur.}
\end{itemize}
\begin{itemize}
\item {Proveniência:(Lat. \textunderscore nuncupatio\textunderscore )}
\end{itemize}
Designação verbal de herdeiros.
\section{Nuncupativamente}
\begin{itemize}
\item {Grp. gram.:adv.}
\end{itemize}
De modo nuncupativo.
Oralmente, de viva voz.
\section{Nuncupativo}
\begin{itemize}
\item {Grp. gram.:adj.}
\end{itemize}
\begin{itemize}
\item {Proveniência:(Do lat. \textunderscore nuncupatus\textunderscore )}
\end{itemize}
O mesmo que \textunderscore oral\textunderscore .
Feito de viva voz, (falando-se de testamentos ou de disposições de última vontade).
Instituido ou nomeado oralmente, (falando-se de herdeiros).
Simplesmente nominal, não real.
\section{Nuncupato}
\begin{itemize}
\item {Grp. gram.:adj.}
\end{itemize}
\begin{itemize}
\item {Utilização:Des.}
\end{itemize}
\begin{itemize}
\item {Proveniência:(Lat. \textunderscore nuncupatus\textunderscore )}
\end{itemize}
Nomeado.
Designado; indicado.
\section{Nuncupatório}
\begin{itemize}
\item {Grp. gram.:adj.}
\end{itemize}
\begin{itemize}
\item {Proveniência:(Do lat. \textunderscore nuncupator\textunderscore )}
\end{itemize}
Que encerra dedicatória.
\section{Nundinal}
\begin{itemize}
\item {Grp. gram.:adj.}
\end{itemize}
O mesmo que \textunderscore nundinário\textunderscore . Cf. Castilho, \textunderscore Fastos\textunderscore , I, 9 e 243.
\section{Nundinário}
\begin{itemize}
\item {Grp. gram.:adj.}
\end{itemize}
\begin{itemize}
\item {Proveniência:(Lat. \textunderscore nundinarius\textunderscore )}
\end{itemize}
Relativo ás núndinas.
Diz-se especialmente do lugar ou praça, onde se faziam as núndinas.
\section{Núndinas}
\begin{itemize}
\item {Grp. gram.:f. pl.}
\end{itemize}
\begin{itemize}
\item {Proveniência:(Lat. \textunderscore nundinae\textunderscore )}
\end{itemize}
Feira ou mercado que, entre os Romanos, se fazia de nove em nove dias.
\section{Nundo}
\begin{itemize}
\item {Grp. gram.:m.}
\end{itemize}
Árvore angolense de Caconda.
\section{Nunes}
\begin{itemize}
\item {Grp. gram.:m.  e  adj.}
\end{itemize}
\begin{itemize}
\item {Utilização:Pop.}
\end{itemize}
Diz-se do número ímpar.
(Corr. de \textunderscore nones\textunderscore . Cp. \textunderscore nones\textunderscore )
\section{Nupcial}
\begin{itemize}
\item {Grp. gram.:adj.}
\end{itemize}
\begin{itemize}
\item {Proveniência:(Lat. \textunderscore nuptialis\textunderscore )}
\end{itemize}
Relativo a núpcias: \textunderscore registo nupcial\textunderscore .
\section{Nupcialidade}
\begin{itemize}
\item {Grp. gram.:f.}
\end{itemize}
\begin{itemize}
\item {Utilização:Neol.}
\end{itemize}
Estatística de casamentos.
Conjunto dos casamentos, realizados numa época ou numa localidade.
\section{Núpcias}
\begin{itemize}
\item {Grp. gram.:f. pl.}
\end{itemize}
\begin{itemize}
\item {Proveniência:(Lat. \textunderscore nuptiae\textunderscore )}
\end{itemize}
Casamento.
Boda; esponsaes.
\section{Nuper-fallecido}
\begin{itemize}
\item {Grp. gram.:adj.}
\end{itemize}
\begin{itemize}
\item {Proveniência:(Do lat. \textunderscore nuper\textunderscore  + port. \textunderscore fallecido\textunderscore )}
\end{itemize}
O mesmo que \textunderscore recém-fallecido\textunderscore . Cf. Filinto, IX, 380, (ed. París).
\section{Nupérrimo}
\begin{itemize}
\item {Grp. gram.:adj.}
\end{itemize}
\begin{itemize}
\item {Utilização:P. us.}
\end{itemize}
\begin{itemize}
\item {Proveniência:(Do lat. \textunderscore nuperus\textunderscore )}
\end{itemize}
Muito recente; succedido há muito pouco tempo.
\section{Nutação}
\begin{itemize}
\item {Grp. gram.:f.}
\end{itemize}
\begin{itemize}
\item {Proveniência:(Lat. \textunderscore nutatio\textunderscore )}
\end{itemize}
Oscilação do eixo terrestre.
Propriedade, que tem certas flôres, de seguir o movimento apparente do Sol.
Tontura de cabeça.
O mesmo que \textunderscore nuto\textunderscore  ou meneio de cabeça.
\section{Nutante}
\begin{itemize}
\item {Grp. gram.:adj.}
\end{itemize}
\begin{itemize}
\item {Proveniência:(Lat. \textunderscore nutans\textunderscore )}
\end{itemize}
Que nuta; vacillante.
\section{Nutar}
\begin{itemize}
\item {Grp. gram.:v. i.}
\end{itemize}
\begin{itemize}
\item {Proveniência:(Lat. \textunderscore nutare\textunderscore )}
\end{itemize}
Oscillar; vacillar.
\section{Nuticana}
\begin{itemize}
\item {Grp. gram.:f.}
\end{itemize}
Planta da serra de Sintra.
\section{Nuto}
\begin{itemize}
\item {Grp. gram.:m.}
\end{itemize}
\begin{itemize}
\item {Utilização:Fig.}
\end{itemize}
\begin{itemize}
\item {Proveniência:(Lat. \textunderscore nutus\textunderscore )}
\end{itemize}
Acto de menear a cabeça, em sinal de approvação ou consentimento.
Desejo.
Arbitrio.
Mandato.
\section{Nutrição}
\begin{itemize}
\item {Grp. gram.:f.}
\end{itemize}
\begin{itemize}
\item {Proveniência:(Lat. \textunderscore nutritio\textunderscore )}
\end{itemize}
Acto ou effeito de nutrir.
Gordura.
Assimilação dos alimentos.
Acto de reforçar a energia dos medicamentos, por meio de certos ingredientes.
\section{Nutrice}
\begin{itemize}
\item {Grp. gram.:f.}
\end{itemize}
O mesmo que \textunderscore nutriz\textunderscore . Cf. F. Barreto, \textunderscore Eneida\textunderscore , I, 64.
\section{Nutrício}
\begin{itemize}
\item {Grp. gram.:adj.}
\end{itemize}
\begin{itemize}
\item {Proveniência:(Lat. \textunderscore nutricius\textunderscore )}
\end{itemize}
O mesmo que \textunderscore nutritivo\textunderscore .
\section{Nutrido}
\begin{itemize}
\item {Grp. gram.:adj.}
\end{itemize}
\begin{itemize}
\item {Proveniência:(De \textunderscore nutrir\textunderscore )}
\end{itemize}
Aleitado.
Gordo; robusto.
\section{Nutridor}
\begin{itemize}
\item {Grp. gram.:m.  e  adj.}
\end{itemize}
O que nutre.
\section{Nutriente}
\begin{itemize}
\item {Grp. gram.:adj.}
\end{itemize}
\begin{itemize}
\item {Proveniência:(Lat. \textunderscore nutriens\textunderscore )}
\end{itemize}
O mesmo que \textunderscore nutritivo\textunderscore .
\section{Nutrificar}
\begin{itemize}
\item {Grp. gram.:v. t.}
\end{itemize}
\begin{itemize}
\item {Utilização:Neol.}
\end{itemize}
\begin{itemize}
\item {Proveniência:(Lat. \textunderscore nutrificare\textunderscore )}
\end{itemize}
Nutrir, alimentar.
\section{Nutrimental}
\begin{itemize}
\item {Grp. gram.:adj.}
\end{itemize}
\begin{itemize}
\item {Proveniência:(Lat. \textunderscore nutrimentalis\textunderscore )}
\end{itemize}
Próprio para nutrir.
\section{Nutrimento}
\begin{itemize}
\item {Grp. gram.:m.}
\end{itemize}
\begin{itemize}
\item {Proveniência:(Lat. \textunderscore nutrimentum\textunderscore )}
\end{itemize}
O mesmo que \textunderscore nutrição\textunderscore ; sustento.
\section{Nutrir}
\begin{itemize}
\item {Grp. gram.:v. t.}
\end{itemize}
\begin{itemize}
\item {Proveniência:(Lat. \textunderscore nutrire\textunderscore )}
\end{itemize}
Sustentar, alimentar.
Educar.
Desenvolver.
Engordar.
Produzir alimento para.
Manter intacto.
Avigorar; proteger.
Manter intimamente: \textunderscore nutrir esperanças\textunderscore .
\section{Nutritivo}
\begin{itemize}
\item {Grp. gram.:adj.}
\end{itemize}
Que serve para nutrir.
Que nutre; nutriente.
\section{Nutriz}
\begin{itemize}
\item {Grp. gram.:f.}
\end{itemize}
\begin{itemize}
\item {Utilização:Poét.}
\end{itemize}
\begin{itemize}
\item {Proveniência:(Lat. \textunderscore nutrix\textunderscore )}
\end{itemize}
Ama de leite.
A mulher que amamenta.
\section{Nutrízio}
\begin{itemize}
\item {Grp. gram.:m.}
\end{itemize}
\begin{itemize}
\item {Utilização:Prov.}
\end{itemize}
\begin{itemize}
\item {Utilização:alent.}
\end{itemize}
\begin{itemize}
\item {Utilização:Ant.}
\end{itemize}
\begin{itemize}
\item {Proveniência:(Do lat. \textunderscore nutricius\textunderscore )}
\end{itemize}
O mesmo que \textunderscore aio\textunderscore .
\section{Nuve}
\begin{itemize}
\item {Grp. gram.:f.}
\end{itemize}
\begin{itemize}
\item {Utilização:ant.}
\end{itemize}
\begin{itemize}
\item {Utilização:Pop.}
\end{itemize}
O mesmo que \textunderscore nuvem\textunderscore . Cf. \textunderscore Eufrosina\textunderscore , 351; Usque, 8.
\section{Nuvem}
\begin{itemize}
\item {Grp. gram.:f.}
\end{itemize}
\begin{itemize}
\item {Utilização:Fig.}
\end{itemize}
\begin{itemize}
\item {Utilização:Gír.}
\end{itemize}
\begin{itemize}
\item {Proveniência:(Do lat. \textunderscore nubes\textunderscore )}
\end{itemize}
Acervo de vapores que, suspenso na atmosphera, turva o azul do céu.
Obscuridade; sombra.
Turvação da vista.
Porção de fumo ou de pó, que se eleva na atmosphera.
Tristeza.
Aquillo que impede a comprehensão.
Grande porção de coisas reunidas, em movimento.
Capote.
\section{Nuvens-castellas}
\begin{itemize}
\item {Grp. gram.:f. pl.}
\end{itemize}
\begin{itemize}
\item {Utilização:Açor}
\end{itemize}
Nuvens acumuladas, cúmulos.
\section{Núveo}
\begin{itemize}
\item {Grp. gram.:adj.}
\end{itemize}
\begin{itemize}
\item {Utilização:Prov.}
\end{itemize}
\begin{itemize}
\item {Utilização:trasm.}
\end{itemize}
\begin{itemize}
\item {Proveniência:(De \textunderscore nuve\textunderscore )}
\end{itemize}
O mesmo que [[anuveado|anuvear]]. (Colhido em Lagoaça)
\section{Nuvioso}
\begin{itemize}
\item {Grp. gram.:adj.}
\end{itemize}
\begin{itemize}
\item {Proveniência:(De \textunderscore nuvem\textunderscore )}
\end{itemize}
O mesmo que \textunderscore nubiloso\textunderscore .
\section{Nuvre}
\begin{itemize}
\item {Grp. gram.:f.}
\end{itemize}
\begin{itemize}
\item {Utilização:Prov.}
\end{itemize}
\begin{itemize}
\item {Utilização:trasm.}
\end{itemize}
O mesmo que \textunderscore nuvem\textunderscore .
(Cp. cast. \textunderscore nuble\textunderscore )
\section{Nuvrejão}
\begin{itemize}
\item {Grp. gram.:m.}
\end{itemize}
\begin{itemize}
\item {Utilização:Prov.}
\end{itemize}
\begin{itemize}
\item {Utilização:trasm.}
\end{itemize}
Grande nuvem (de mosquitos, gafanhotos, etc.).
(Cp. \textunderscore nuvre\textunderscore )
\section{Núxia}
\begin{itemize}
\item {fónica:csi}
\end{itemize}
\begin{itemize}
\item {Grp. gram.:f.}
\end{itemize}
Gênero de plantas escrofularíneas.
\section{Nuzungulo}
\begin{itemize}
\item {Grp. gram.:m.}
\end{itemize}
Arbusto africano, (\textunderscore corolliflôres\textunderscore , De-Cand.), de fôlhas simples e oppostas, flôres axíllares, brancas, miúdas.
\section{Nyctagíneas}
\begin{itemize}
\item {Grp. gram.:f. pl.}
\end{itemize}
\begin{itemize}
\item {Proveniência:(De \textunderscore nyctagíneo\textunderscore )}
\end{itemize}
Família de plantas, que tem por typo as bôas-noites.
\section{Nyctagíneo}
\begin{itemize}
\item {Grp. gram.:adj.}
\end{itemize}
\begin{itemize}
\item {Proveniência:(De \textunderscore nyctago\textunderscore )}
\end{itemize}
Relativo ou semelhante ás bôas-noites.
\section{Nyctago}
\begin{itemize}
\item {Grp. gram.:m.}
\end{itemize}
\begin{itemize}
\item {Proveniência:(Do gr. \textunderscore nux\textunderscore , \textunderscore nuktos\textunderscore )}
\end{itemize}
Nome scientífico da planta bôas-noites.
\section{Nyctalemo}
\begin{itemize}
\item {Grp. gram.:m.}
\end{itemize}
Gênero de insectos lepidópteros.
\section{Nýctalo}
\begin{itemize}
\item {Grp. gram.:m.}
\end{itemize}
\begin{itemize}
\item {Proveniência:(Gr. \textunderscore nuktalos\textunderscore )}
\end{itemize}
Sub-gênero de mammíferos, que abrange quatro espécies.
\section{Nyctálope}
\begin{itemize}
\item {Grp. gram.:m.  e  f.}
\end{itemize}
\begin{itemize}
\item {Proveniência:(Do gr. \textunderscore nuktalos\textunderscore  + \textunderscore ops\textunderscore )}
\end{itemize}
Pessôa que não vê durante o dia e que só distingue objectos quando escurece ou anoitece.
\section{Nyctalopia}
\begin{itemize}
\item {Grp. gram.:f.}
\end{itemize}
Estado do que é nyctalope.
\section{Nyctalópico}
\begin{itemize}
\item {Grp. gram.:adj.}
\end{itemize}
Relativo á nyctalopia.
\section{Nyctantho}
\begin{itemize}
\item {Grp. gram.:m.}
\end{itemize}
\begin{itemize}
\item {Utilização:Bot.}
\end{itemize}
\begin{itemize}
\item {Proveniência:(Do gr. \textunderscore nuktos\textunderscore  + \textunderscore anthos\textunderscore )}
\end{itemize}
Arbusto trepador, jasmináceo, chamado também \textunderscore jasmim da Arábia\textunderscore .
\section{Nyctélias}
\begin{itemize}
\item {Grp. gram.:f. pl.}
\end{itemize}
\begin{itemize}
\item {Proveniência:(Do lat. \textunderscore nictélius\textunderscore )}
\end{itemize}
Festas de Baccho, que se celebravam de noite, á luz de archotes.
\section{Nyctêmero}
\begin{itemize}
\item {Grp. gram.:m.}
\end{itemize}
\begin{itemize}
\item {Proveniência:(Do gr. \textunderscore nux\textunderscore  + \textunderscore hemera\textunderscore )}
\end{itemize}
Espaço de tempo, que abrange dia e noite em 24 horas.
\section{Nycterínea}
\begin{itemize}
\item {Grp. gram.:f.}
\end{itemize}
\begin{itemize}
\item {Proveniência:(Do gr. \textunderscore nukteros\textunderscore )}
\end{itemize}
Gênero de plantas aromáticas, escrofularíneas, cujas flôres se abrem de noite.
\section{Nycterino}
\begin{itemize}
\item {Grp. gram.:m.}
\end{itemize}
Gênero de insectos coleópteros heterâmeros.
\section{Nyctíceo}
\begin{itemize}
\item {Grp. gram.:m.}
\end{itemize}
Gênero de mamíferos chirópteros.
\section{Nyctícora}
\begin{itemize}
\item {Grp. gram.:f.}
\end{itemize}
\begin{itemize}
\item {Proveniência:(Do gr. \textunderscore nuktikorax\textunderscore )}
\end{itemize}
Ave, o mesmo que \textunderscore mocho\textunderscore .
\section{Nyctímero}
\begin{itemize}
\item {Grp. gram.:m.}
\end{itemize}
O mesmo que \textunderscore nictêmero\textunderscore .
\section{Nyctiríbia}
\begin{itemize}
\item {Grp. gram.:f.}
\end{itemize}
\begin{itemize}
\item {Proveniência:(Do gr. \textunderscore nukteros\textunderscore  + \textunderscore bios\textunderscore )}
\end{itemize}
Gênero de insectos dípteros.
\section{Nictímono}
\begin{itemize}
\item {Grp. gram.:m.}
\end{itemize}
Gênero de mamíferos chirópteros.
\section{Nictóbata}
\begin{itemize}
\item {Grp. gram.:m.}
\end{itemize}
\begin{itemize}
\item {Proveniência:(Do gr. \textunderscore nuktos\textunderscore  + \textunderscore batein\textunderscore )}
\end{itemize}
O mesmo que \textunderscore sonâmbulo\textunderscore .
\section{Nictobatismo}
\begin{itemize}
\item {Grp. gram.:m.}
\end{itemize}
\begin{itemize}
\item {Proveniência:(De \textunderscore nictóbata\textunderscore )}
\end{itemize}
O mesmo que \textunderscore sonambulismo\textunderscore .
\section{Nictoclepto}
\begin{itemize}
\item {Grp. gram.:m.}
\end{itemize}
Gênero de mamíferos roedores.
\section{Nictofobia}
\begin{itemize}
\item {Grp. gram.:f.}
\end{itemize}
\begin{itemize}
\item {Utilização:Med.}
\end{itemize}
\begin{itemize}
\item {Proveniência:(Do gr. \textunderscore nux\textunderscore , \textunderscore nuktos\textunderscore  + \textunderscore phobein\textunderscore )}
\end{itemize}
Mêdo mórbido da noite.
\section{Nictografia}
\begin{itemize}
\item {Grp. gram.:f.}
\end{itemize}
\begin{itemize}
\item {Proveniência:(De \textunderscore nictógrafo\textunderscore )}
\end{itemize}
Arte de escrever ás escuras ou sem fazer uso da vista.
\section{Nictográfico}
\begin{itemize}
\item {Grp. gram.:adj.}
\end{itemize}
Relativo á nictografia.
\section{Nictógrafo}
\begin{itemize}
\item {Grp. gram.:m.}
\end{itemize}
\begin{itemize}
\item {Proveniência:(Do gr. \textunderscore nux\textunderscore , \textunderscore nuktos\textunderscore  + \textunderscore graphein\textunderscore )}
\end{itemize}
Instrumento, para escrever de noite, sem luz ou sem se verem os traços que se fazem.
\section{Nictozoilo}
\begin{itemize}
\item {Grp. gram.:m.}
\end{itemize}
Gênero de insectos coleópteros heterómeros.
\section{Ninfa}
\begin{itemize}
\item {Grp. gram.:f.}
\end{itemize}
\begin{itemize}
\item {Utilização:Fig.}
\end{itemize}
\begin{itemize}
\item {Utilização:Anat.}
\end{itemize}
\begin{itemize}
\item {Proveniência:(Lat. \textunderscore nympha\textunderscore )}
\end{itemize}
Divindade dos rios, dos bosques e dos montes, segundo a Mitologia grega e latina.
Mulher formosa e jovem.
Crisálida.
Cada um dos prolongamentos membranosos que constituem os pequenos lábios da vulva.
\section{Ninfagogo}
\begin{itemize}
\item {Grp. gram.:m.}
\end{itemize}
\begin{itemize}
\item {Proveniência:(Do gr. \textunderscore numphe\textunderscore  + \textunderscore agein\textunderscore )}
\end{itemize}
Mancebo que, entre os antigos Gregos, conduzia a desposada da casa paterna, para a casa do espôso.
\section{Ninfálios}
\begin{itemize}
\item {Grp. gram.:m. pl.}
\end{itemize}
Gênero de lepidópteros.
\section{Ninféa}
\begin{itemize}
\item {Grp. gram.:f.}
\end{itemize}
\begin{itemize}
\item {Proveniência:(Lat. \textunderscore nymphaea\textunderscore )}
\end{itemize}
Nome científico do nenúfar.
\section{Ninfeáceas}
\begin{itemize}
\item {Grp. gram.:f. pl.}
\end{itemize}
Família de plantas aquáticas, que têm por tipo o nenúfar ou ninfeia.
\section{Ninfeia}
\begin{itemize}
\item {Grp. gram.:f.}
\end{itemize}
\begin{itemize}
\item {Proveniência:(Lat. \textunderscore nymphaea\textunderscore )}
\end{itemize}
Nome científico do nenúfar.
\section{Ninfeu}
\begin{itemize}
\item {Grp. gram.:adj.}
\end{itemize}
Relativo ás ninfas ou próprio delas.
Formado por águas doces, (falando-se de terrenos ou rochas).
\section{Ninfídio}
\begin{itemize}
\item {Grp. gram.:m.}
\end{itemize}
Gênero de insectos lepidópteros diurnos.
\section{Ninfóide}
\begin{itemize}
\item {Grp. gram.:adj.}
\end{itemize}
\begin{itemize}
\item {Proveniência:(Do gr. \textunderscore numphe\textunderscore  + \textunderscore eidos\textunderscore )}
\end{itemize}
Que tem fórma de ninfa.
\section{Ninfolepsia}
\begin{itemize}
\item {Grp. gram.:f.}
\end{itemize}
\begin{itemize}
\item {Utilização:Med.}
\end{itemize}
Misantropia dos que desejam especialmente a solidão dos bosques.
Espécie de delírio que, segundo os antigos, se apoderava do homem que tivesse visto uma ninfa.
\section{Ninfómana}
\begin{itemize}
\item {Grp. gram.:adj. f.}
\end{itemize}
Diz-se da mulher, que tem ninfomania.
\section{Ninfomania}
\begin{itemize}
\item {Grp. gram.:f.}
\end{itemize}
\begin{itemize}
\item {Proveniência:(De \textunderscore nympha\textunderscore  + \textunderscore mania\textunderscore )}
\end{itemize}
Tendência para os apetites sensuaes, nas fêmeas dos mamíferos.
Furor uterino.
\section{Ninfomaníaco}
\begin{itemize}
\item {Grp. gram.:adj.}
\end{itemize}
Relativo á ninfomania.
\section{Ninfose}
\begin{itemize}
\item {Grp. gram.:f.}
\end{itemize}
\begin{itemize}
\item {Utilização:Zool.}
\end{itemize}
\begin{itemize}
\item {Proveniência:(De \textunderscore ninfa\textunderscore )}
\end{itemize}
Transformação da lagarta em ninfa ou crisálida.
\section{Ninfotomia}
\begin{itemize}
\item {Grp. gram.:f.}
\end{itemize}
\begin{itemize}
\item {Proveniência:(Do gr. \textunderscore numphe\textunderscore  + \textunderscore tome\textunderscore )}
\end{itemize}
Excisão cirúrgica das ninfas da vulva ou de parte delas.
\section{Ninfotómico}
\begin{itemize}
\item {Grp. gram.:adj.}
\end{itemize}
Relativo á ninfotomia.
\section{Nínfula}
\begin{itemize}
\item {Grp. gram.:f.}
\end{itemize}
Gênero de insectos lepidópteros nocturnos.
\section{Nissa}
\begin{itemize}
\item {Grp. gram.:f.}
\end{itemize}
Gênero de plantas, que serve de tipo ás nissáceas.
\section{Nissáceas}
\begin{itemize}
\item {Grp. gram.:f. pl.}
\end{itemize}
\begin{itemize}
\item {Proveniência:(De \textunderscore nissa\textunderscore )}
\end{itemize}
Família de plantas dicotiledóneas, extraida das santaláceas.
\section{Nistâgmico}
\begin{itemize}
\item {Grp. gram.:adj.}
\end{itemize}
Relativo ao nistagmo.
\section{Nistagmo}
\begin{itemize}
\item {Grp. gram.:m.}
\end{itemize}
\begin{itemize}
\item {Proveniência:(Gr. \textunderscore nustagmos\textunderscore )}
\end{itemize}
Oscilação do globo do ôlho, em tôrno ao seu eixo horizontal ou vertical.
\section{Nyctímono}
\begin{itemize}
\item {Grp. gram.:m.}
\end{itemize}
Gênero de mamíferos chirópteros.
\section{Nyctóbata}
\begin{itemize}
\item {Grp. gram.:m.}
\end{itemize}
\begin{itemize}
\item {Proveniência:(Do gr. \textunderscore nuktos\textunderscore  + \textunderscore batein\textunderscore )}
\end{itemize}
O mesmo que \textunderscore somnâmbulo\textunderscore .
\section{Nyctoclepto}
\begin{itemize}
\item {Grp. gram.:m.}
\end{itemize}
Gênero de mamíferos roedores.
\section{Nyctographia}
\begin{itemize}
\item {Grp. gram.:f.}
\end{itemize}
\begin{itemize}
\item {Proveniência:(De \textunderscore nyctógrafo\textunderscore )}
\end{itemize}
Arte de escrever ás escuras ou sem fazer uso da vista.
\section{Nyctográphico}
\begin{itemize}
\item {Grp. gram.:adj.}
\end{itemize}
Relativo á nyctographia.
\section{Nyctógrapho}
\begin{itemize}
\item {Grp. gram.:m.}
\end{itemize}
\begin{itemize}
\item {Proveniência:(Do gr. \textunderscore nux\textunderscore , \textunderscore nuktos\textunderscore  + \textunderscore graphein\textunderscore )}
\end{itemize}
Instrumento, para escrever de noite, sem luz ou sem se verem os traços que se fazem.
\section{Nyctophobia}
\begin{itemize}
\item {Grp. gram.:f.}
\end{itemize}
\begin{itemize}
\item {Utilização:Med.}
\end{itemize}
\begin{itemize}
\item {Proveniência:(Do gr. \textunderscore nux\textunderscore , \textunderscore nuktos\textunderscore  + \textunderscore phobein\textunderscore )}
\end{itemize}
Mêdo mórbido da noite.
\section{Nyctozoilo}
\begin{itemize}
\item {Grp. gram.:m.}
\end{itemize}
Gênero de insectos coleópteros heterómeros.
\section{Nympha}
\begin{itemize}
\item {Grp. gram.:f.}
\end{itemize}
\begin{itemize}
\item {Utilização:Fig.}
\end{itemize}
\begin{itemize}
\item {Utilização:Anat.}
\end{itemize}
\begin{itemize}
\item {Proveniência:(Lat. \textunderscore nympha\textunderscore )}
\end{itemize}
Divindade dos rios, dos bosques e dos montes, segundo a Mythologia grega e latina.
Mulher formosa e jovem.
Chrysálida.
Cada um dos prolongamentos membranosos que constituem os pequenos lábios da vulva.
\section{Nymphagogo}
\begin{itemize}
\item {Grp. gram.:m.}
\end{itemize}
\begin{itemize}
\item {Proveniência:(Do gr. \textunderscore numphe\textunderscore  + \textunderscore agein\textunderscore )}
\end{itemize}
Mancebo que, entre os antigos Gregos, conduzia a desposada da casa paterna, para a casa do espôso.
\section{Nymphálios}
\begin{itemize}
\item {Grp. gram.:m. pl.}
\end{itemize}
Gênero de lepidópteros.
\section{Nymphéa}
\begin{itemize}
\item {Grp. gram.:f.}
\end{itemize}
\begin{itemize}
\item {Proveniência:(Lat. \textunderscore nymphaea\textunderscore )}
\end{itemize}
Nome scientífico do nenúfar.
\section{Nympheáceas}
\begin{itemize}
\item {Grp. gram.:f. pl.}
\end{itemize}
Família de plantas aquáticas, que têm por typo o nenúfar ou nympheia.
\section{Nympheia}
\begin{itemize}
\item {Grp. gram.:f.}
\end{itemize}
\begin{itemize}
\item {Proveniência:(Lat. \textunderscore nymphaea\textunderscore )}
\end{itemize}
Nome scientífico do nenúfar.
\section{Nympheu}
\begin{itemize}
\item {Grp. gram.:adj.}
\end{itemize}
Relativo ás nymphas ou próprio dellas.
Formado por águas doces, (falando-se de terrenos ou rochas).
\section{Nymphídio}
\begin{itemize}
\item {Grp. gram.:m.}
\end{itemize}
Gênero de insectos lepidópteros diurnos.
\section{Nymphóide}
\begin{itemize}
\item {Grp. gram.:adj.}
\end{itemize}
\begin{itemize}
\item {Proveniência:(Do gr. \textunderscore numphe\textunderscore  + \textunderscore eidos\textunderscore )}
\end{itemize}
Que tem fórma de nympha.
\section{Nympholepsia}
\begin{itemize}
\item {Grp. gram.:f.}
\end{itemize}
\begin{itemize}
\item {Utilização:Med.}
\end{itemize}
Misanthropia dos que desejam especialmente a solidão dos bosques.
Espécie de delírio que, segundo os antigos, se apoderava do homem que tivesse visto uma nympha.
\section{Nymphómana}
\begin{itemize}
\item {Grp. gram.:adj. f.}
\end{itemize}
Diz-se da mulher, que tem nymphomania.
\section{Nymphomania}
\begin{itemize}
\item {Grp. gram.:f.}
\end{itemize}
\begin{itemize}
\item {Proveniência:(De \textunderscore nympha\textunderscore  + \textunderscore mania\textunderscore )}
\end{itemize}
Tendência para os appetites sensuaes, nas fêmeas dos mammíferos.
Furor uterino.
\section{Nymphomaníaco}
\begin{itemize}
\item {Grp. gram.:adj.}
\end{itemize}
Relativo á nymphomania.
\section{Nymphose}
\begin{itemize}
\item {Grp. gram.:f.}
\end{itemize}
\begin{itemize}
\item {Utilização:Zool.}
\end{itemize}
\begin{itemize}
\item {Proveniência:(De \textunderscore nympha\textunderscore )}
\end{itemize}
Transformação da lagarta em nympha ou chrysállida.
\section{Nymphotomia}
\begin{itemize}
\item {Grp. gram.:f.}
\end{itemize}
\begin{itemize}
\item {Proveniência:(Do gr. \textunderscore numphe\textunderscore  + \textunderscore tome\textunderscore )}
\end{itemize}
Excisão cirúrgica das nymphas da vulva ou de parte dellas.
\section{Nymphotómico}
\begin{itemize}
\item {Grp. gram.:adj.}
\end{itemize}
Relativo á nymphotomia.
\section{Nýmphula}
\begin{itemize}
\item {Grp. gram.:f.}
\end{itemize}
Gênero de insectos lepidópteros nocturnos.
\section{Nyssa}
\begin{itemize}
\item {Grp. gram.:f.}
\end{itemize}
Gênero de plantas, que serve de typo ás nyssáceas.
\section{Nyssáceas}
\begin{itemize}
\item {Grp. gram.:f. pl.}
\end{itemize}
\begin{itemize}
\item {Proveniência:(De \textunderscore nyssa\textunderscore )}
\end{itemize}
Família de plantas dicotyledóneas, extrahida das santaláceas.
\section{Nystágmico}
\begin{itemize}
\item {Grp. gram.:adj.}
\end{itemize}
Relativo ao nystagmo.
\section{Nystagmo}
\begin{itemize}
\item {Grp. gram.:m.}
\end{itemize}
\begin{itemize}
\item {Proveniência:(Gr. \textunderscore nustagmos\textunderscore )}
\end{itemize}
\end{document}