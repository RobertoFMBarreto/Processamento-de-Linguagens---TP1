
\begin{itemize}
\item {Proveniência: }
\end{itemize}\documentclass{article}
\usepackage[portuguese]{babel}
\title{O}
\begin{document}
Oscillação do globo do ôlho, em tôrno ao seu eixo horizontal ou vertical.
\section{Olvidamento}
\begin{itemize}
\item {Grp. gram.:m.}
\end{itemize}
\begin{itemize}
\item {Utilização:Des.}
\end{itemize}
\begin{itemize}
\item {Proveniência:(De \textunderscore olvidar\textunderscore )}
\end{itemize}
O mesmo que \textunderscore olvido\textunderscore .
\section{Omphalótribo}
\begin{itemize}
\item {Grp. gram.:m.}
\end{itemize}
\begin{itemize}
\item {Utilização:Med.}
\end{itemize}
\begin{itemize}
\item {Proveniência:(Do gr. \textunderscore omphalos\textunderscore  + \textunderscore tribein\textunderscore )}
\end{itemize}
Apparelho com que se esmaga o cordão umbilical.
\section{Omphalotripsia}
\begin{itemize}
\item {Grp. gram.:f.}
\end{itemize}
\begin{itemize}
\item {Utilização:Med.}
\end{itemize}
Esmagamento do cordão umbilical.
(Cp. \textunderscore omphalótribo\textunderscore )
\section{Onfalótribo}
\begin{itemize}
\item {Grp. gram.:m.}
\end{itemize}
\begin{itemize}
\item {Utilização:Med.}
\end{itemize}
\begin{itemize}
\item {Proveniência:(Do gr. \textunderscore omphalos\textunderscore  + \textunderscore tribein\textunderscore )}
\end{itemize}
Aparelho com que se esmaga o cordão umbilical.
\section{Onfalotripsia}
\begin{itemize}
\item {Grp. gram.:f.}
\end{itemize}
\begin{itemize}
\item {Utilização:Med.}
\end{itemize}
Esmagamento do cordão umbilical.
(Cp. \textunderscore onfalótribo\textunderscore )
\section{O}
\begin{itemize}
\item {fónica:ó}
\end{itemize}
\begin{itemize}
\item {Grp. gram.:m.}
\end{itemize}
\begin{itemize}
\item {Utilização:Ext.}
\end{itemize}
Décima quinta letra do alphabeto português.
Círculo, anel, qualquer objecto de feitio análogo ao daquella letra.
Abrev. de \textunderscore Oéste\textunderscore .
Figura numérica, que se chama \textunderscore cifra\textunderscore  ou \textunderscore zero\textunderscore .
Em música, designa o orifício dos instrumentos de sôpro, ou a corda que, em dado momento, não funcciona.
Como letra numeral, significou 11.
Na música também era sinal para se abrir inteiramente a bôca.
\section{O}
\begin{itemize}
\item {fónica:u}
\end{itemize}
\begin{itemize}
\item {Grp. gram.:Art. def. ,  m. ,  sing.}
\end{itemize}
(Ant. \textunderscore lo\textunderscore , do lat. \textunderscore illum\textunderscore )
\section{O}
\begin{itemize}
\item {fónica:u}
\end{itemize}
\begin{itemize}
\item {Grp. gram.:pron. demonstr. ,  m. ,  sing.}
\end{itemize}
(que se emprega em lugar de um substantivo ou de uma phrase considerada substantivamente)
\section{Ó}
\begin{itemize}
\item {Grp. gram.:interj.}
\end{itemize}
O mesmo que \textunderscore oh!\textunderscore .
\section{Ó}
\begin{itemize}
\item {Grp. gram.:interj.}
\end{itemize}
(para chamar ou invocar)
\section{Ó}
(contr. pop. e ant. da prep. \textunderscore a\textunderscore  e art. \textunderscore o\textunderscore ): \textunderscore foi ó Pôrto\textunderscore . Cf. Gil Vicente, Prestes, etc.
\section{Oacaju}
\begin{itemize}
\item {Grp. gram.:m.}
\end{itemize}
O mesmo que \textunderscore acaju\textunderscore .
\section{Oanaçu}
\begin{itemize}
\item {Grp. gram.:m.}
\end{itemize}
Espécie de palmeira do Brasil.
\section{Oanandi}
\begin{itemize}
\item {Grp. gram.:m.}
\end{itemize}
O mesmo que \textunderscore oanani\textunderscore .
\section{Oanani}
\begin{itemize}
\item {Grp. gram.:m.}
\end{itemize}
Planta clusiácea do Brasil.
\section{Oanassu}
\begin{itemize}
\item {Grp. gram.:m.}
\end{itemize}
Espécie de palmeira do Brasil.
\section{Oariana}
\begin{itemize}
\item {Grp. gram.:f.}
\end{itemize}
Ave do Brasil.
\section{Oaristo}
\begin{itemize}
\item {Grp. gram.:m.}
\end{itemize}
\begin{itemize}
\item {Proveniência:(Gr. \textunderscore oaristus\textunderscore )}
\end{itemize}
Diálogo entre marido e mulher.
Entretenimento intimo.
Collóquio terno.
\section{Oasiano}
\begin{itemize}
\item {Grp. gram.:adj.}
\end{itemize}
\begin{itemize}
\item {Grp. gram.:M.}
\end{itemize}
Relativo a oásis.
Habitante de um oásis.
\section{Oásis}
\begin{itemize}
\item {Grp. gram.:m.}
\end{itemize}
\begin{itemize}
\item {Utilização:Fig.}
\end{itemize}
\begin{itemize}
\item {Proveniência:(Lat. \textunderscore oasis\textunderscore )}
\end{itemize}
Terreno, coberto de vegetação, nos grandes desertos da Ásia e da África.
Lugar aprazível, entre outros que o não são.
Objecto formoso ou agradável, em meio de outros que o não são.
Prazer entre desgostos.
\section{Ob}
\begin{itemize}
\item {Grp. gram.:conj.}
\end{itemize}
\begin{itemize}
\item {Utilização:Ant.}
\end{itemize}
O mesmo que \textunderscore ou\textunderscore ^1.
\section{Ob...}
\begin{itemize}
\item {Grp. gram.:pref.}
\end{itemize}
\begin{itemize}
\item {Proveniência:(Lat. \textunderscore ob\textunderscore )}
\end{itemize}
(designativo geralmente de \textunderscore opposição\textunderscore  ou de \textunderscore inversão\textunderscore )
\section{Oba}
\begin{itemize}
\item {Grp. gram.:f.}
\end{itemize}
\begin{itemize}
\item {Proveniência:(Gr. \textunderscore oba\textunderscore )}
\end{itemize}
Cada uma das seis divisões de cada antiga tríbo atheniense.
\section{Oba}
\begin{itemize}
\item {Grp. gram.:f.}
\end{itemize}
\begin{itemize}
\item {Utilização:Ant.}
\end{itemize}
O mesmo que \textunderscore opa\textunderscore .
Sobrepeliz.
\section{Oba}
\begin{itemize}
\item {Grp. gram.:f.}
\end{itemize}
\begin{itemize}
\item {Proveniência:(Lat. \textunderscore obba\textunderscore )}
\end{itemize}
Grande vaso de barro, de fundo largo, usado nos banquetes romanos, para a mistura de vinhos.
\section{Obá}
\begin{itemize}
\item {Grp. gram.:m.}
\end{itemize}
\begin{itemize}
\item {Utilização:T. santomense}
\end{itemize}
Árvore, o mesmo que \textunderscore mucamba-camba\textunderscore .
\section{Oban}
\begin{itemize}
\item {Grp. gram.:m.}
\end{itemize}
Pequena barra de oiro, que serve de moéda entre os Japoneses.
\section{Obba}
\begin{itemize}
\item {Grp. gram.:f.}
\end{itemize}
\begin{itemize}
\item {Proveniência:(Lat. \textunderscore obba\textunderscore )}
\end{itemize}
Grande vaso de barro, de fundo largo, usado nos banquetes romanos, para a mistura de vinhos.
\section{Obcecação}
\begin{itemize}
\item {Grp. gram.:f.}
\end{itemize}
\begin{itemize}
\item {Utilização:Fig.}
\end{itemize}
\begin{itemize}
\item {Proveniência:(Lat. \textunderscore obcaecatio\textunderscore )}
\end{itemize}
Acto ou effeito de obcecar.
Teimosia; pertinácia.
Insistência num êrro.
\section{Obcecado}
\begin{itemize}
\item {Grp. gram.:adj.}
\end{itemize}
Que tem a intelligência obscurecida.
Contumaz no êrro.
\section{Obcecador}
\begin{itemize}
\item {Grp. gram.:adj.}
\end{itemize}
O mesmo que \textunderscore obcecante\textunderscore .
\section{Obcecante}
\begin{itemize}
\item {Grp. gram.:adj.}
\end{itemize}
Que obceca.
\section{Obcecar}
\begin{itemize}
\item {Grp. gram.:v. t.}
\end{itemize}
\begin{itemize}
\item {Utilização:Fig.}
\end{itemize}
\begin{itemize}
\item {Proveniência:(Lat. \textunderscore obcaecare\textunderscore )}
\end{itemize}
Tornar cego.
Offuscar.
Obscurecer o espírito de.
Desvairar; induzir em êrro.
Tornar contumaz no êrro.
Tornar inintelligivel.
\section{Obcláveo}
\begin{itemize}
\item {Grp. gram.:adj.}
\end{itemize}
\begin{itemize}
\item {Utilização:Bot.}
\end{itemize}
\begin{itemize}
\item {Proveniência:(De \textunderscore Ob...\textunderscore  + \textunderscore clava\textunderscore )}
\end{itemize}
Que tem a fórma de maçan invertida.
\section{Obcomprimido}
\begin{itemize}
\item {Grp. gram.:adj.}
\end{itemize}
\begin{itemize}
\item {Utilização:Bot.}
\end{itemize}
\begin{itemize}
\item {Proveniência:(De \textunderscore Ob...\textunderscore  + \textunderscore comprimido\textunderscore )}
\end{itemize}
Diz-se do ovário ou das sementes das synanthéreas, quando o seu maior diâmetro vai da direita para a esquerda.
\section{Obcónico}
\begin{itemize}
\item {Grp. gram.:adj.}
\end{itemize}
\begin{itemize}
\item {Proveniência:(De \textunderscore Ob...\textunderscore  + \textunderscore cónico\textunderscore )}
\end{itemize}
Que tem a fórma de um cóne invertido.
\section{Obcordado}
\begin{itemize}
\item {Grp. gram.:adj.}
\end{itemize}
\begin{itemize}
\item {Utilização:Bot.}
\end{itemize}
\begin{itemize}
\item {Proveniência:(Do lat. \textunderscore ob\textunderscore  + \textunderscore cor\textunderscore , \textunderscore cordis\textunderscore )}
\end{itemize}
Que tem a fórma de um coração invertido.
\section{Obcordiforme}
\begin{itemize}
\item {Grp. gram.:adj.}
\end{itemize}
\begin{itemize}
\item {Proveniência:(De \textunderscore Ob...\textunderscore  + \textunderscore cordiforme\textunderscore )}
\end{itemize}
Que tem a fórma de um coração invertido.
\section{Obcurrente}
\begin{itemize}
\item {Grp. gram.:adj.}
\end{itemize}
\begin{itemize}
\item {Utilização:Bot.}
\end{itemize}
\begin{itemize}
\item {Proveniência:(Do lat. \textunderscore ob\textunderscore  + \textunderscore currens\textunderscore , \textunderscore currentis\textunderscore )}
\end{itemize}
Diz-se dos septos, que dividem os frutos em cavidades ou compartimentos.
\section{Obdentado}
\begin{itemize}
\item {Grp. gram.:adj.}
\end{itemize}
\begin{itemize}
\item {Utilização:Bot.}
\end{itemize}
\begin{itemize}
\item {Proveniência:(De \textunderscore Ob...\textunderscore  + \textunderscore dentado\textunderscore )}
\end{itemize}
Que tem o bôrbo dentado em pequenos ângulos salientes.
\section{Obducto}
\begin{itemize}
\item {Grp. gram.:adj.}
\end{itemize}
\begin{itemize}
\item {Utilização:Poét.}
\end{itemize}
\begin{itemize}
\item {Proveniência:(Lat. \textunderscore obductus\textunderscore )}
\end{itemize}
Tapado; occulto.
\section{Obduração}
\begin{itemize}
\item {Grp. gram.:f.}
\end{itemize}
\begin{itemize}
\item {Proveniência:(Lat. \textunderscore obduratio\textunderscore )}
\end{itemize}
Acto ou effeito de obdurar.
Obstinação; obcecação.
\section{Obdurado}
\begin{itemize}
\item {Grp. gram.:adj.}
\end{itemize}
\begin{itemize}
\item {Proveniência:(De \textunderscore obdurar\textunderscore )}
\end{itemize}
Obcecado; pertinaz.
\section{Obdurar}
\begin{itemize}
\item {Grp. gram.:v. t.}
\end{itemize}
\begin{itemize}
\item {Utilização:Fig.}
\end{itemize}
\begin{itemize}
\item {Proveniência:(Lat. \textunderscore obdurare\textunderscore )}
\end{itemize}
Tornar duro.
Empedernir.
Tornar obstinado, pertinaz.
Obcecar.
\section{Obedecer}
\begin{itemize}
\item {Grp. gram.:v. i.}
\end{itemize}
\begin{itemize}
\item {Grp. gram.:V. t.}
\end{itemize}
\begin{itemize}
\item {Utilização:Des.}
\end{itemize}
\begin{itemize}
\item {Proveniência:(Do lat. \textunderscore obedire\textunderscore )}
\end{itemize}
Attender, sujeitar-se á vontade de outrem.
Estar dependente de um poder ou autoridade.
Submeter-se.
Ceder; vergar, abater-se.
Cumprir.
Ceder a; estar dependente de.
\section{Obedença}
\begin{itemize}
\item {Grp. gram.:f.}
\end{itemize}
\begin{itemize}
\item {Utilização:Ant.}
\end{itemize}
O mesmo que \textunderscore obediência\textunderscore .
\section{Obediência}
\begin{itemize}
\item {Grp. gram.:f.}
\end{itemize}
\begin{itemize}
\item {Utilização:Ant.}
\end{itemize}
\begin{itemize}
\item {Utilização:Ant.}
\end{itemize}
\begin{itemize}
\item {Proveniência:(Lat. \textunderscore obedientia\textunderscore )}
\end{itemize}
Acto de obedecer.
Homenagem.
Docilidade.
Dependência.
Igreja ou propriedade, dependente de Ordem religiosa.
Sacristia.
Hospício.
\section{Obediencial}
\begin{itemize}
\item {Grp. gram.:adj.}
\end{itemize}
\begin{itemize}
\item {Grp. gram.:M.}
\end{itemize}
Relativo a obediência.
Frade, que tinha licença para passar a outro convento ou para outro fim.
O mesmo que \textunderscore ovençal\textunderscore .
\section{Obediente}
\begin{itemize}
\item {Grp. gram.:adj.}
\end{itemize}
\begin{itemize}
\item {Proveniência:(Lat. \textunderscore obediens\textunderscore )}
\end{itemize}
Que obedece.
Submisso; humilde; dócil.
\section{Obedientemente}
\begin{itemize}
\item {Grp. gram.:adv.}
\end{itemize}
De modo obediente.
Submissamente; humildemente.
\section{Obedinte}
\begin{itemize}
\item {Grp. gram.:adj.}
\end{itemize}
\begin{itemize}
\item {Utilização:Prov.}
\end{itemize}
\begin{itemize}
\item {Utilização:beir.}
\end{itemize}
O mesmo que \textunderscore obediente\textunderscore .
\section{Obélia}
\begin{itemize}
\item {Grp. gram.:f.}
\end{itemize}
\begin{itemize}
\item {Proveniência:(Gr. \textunderscore obelias\textunderscore )}
\end{itemize}
Gênero de polypeiros calcários.
Espécie de bolo ou torta, que se offerecia a Baccho, nos sacrificios pagãos.
Gênero de acalephos medusários.
\section{Obélio}
\begin{itemize}
\item {Grp. gram.:m.}
\end{itemize}
\begin{itemize}
\item {Utilização:Anat.}
\end{itemize}
\begin{itemize}
\item {Proveniência:(Do lat. \textunderscore obelus\textunderscore )}
\end{itemize}
Ponto, em que a sutura sagital se torna momentaneamente simples.
\section{Obélion}
\begin{itemize}
\item {Grp. gram.:m.}
\end{itemize}
\begin{itemize}
\item {Utilização:Anat.}
\end{itemize}
\begin{itemize}
\item {Proveniência:(Do lat. \textunderscore obelus\textunderscore )}
\end{itemize}
Ponto, em que a sutura sagital se torna momentaneamente simples.
\section{Obeliscal}
\begin{itemize}
\item {Grp. gram.:adj.}
\end{itemize}
Relativo a obelisco, ou que tem fórma de obelisco.
\section{Obeliscária}
\begin{itemize}
\item {Grp. gram.:f.}
\end{itemize}
\begin{itemize}
\item {Proveniência:(De \textunderscore obelisco\textunderscore )}
\end{itemize}
Gênero de plantas, da fam. das compostas.
\section{Obelisco}
\begin{itemize}
\item {Grp. gram.:m.}
\end{itemize}
\begin{itemize}
\item {Proveniência:(Lat. \textunderscore obeliscus\textunderscore )}
\end{itemize}
Monumento quadrangular em fórma de agulha, ordinariamente monólitho sôbre um pedestal.
Objecto alto e alongado; óbelo.
\section{Obeliscolícnio}
\begin{itemize}
\item {Grp. gram.:m.}
\end{itemize}
\begin{itemize}
\item {Proveniência:(Gr. \textunderscore obelisko-luknion\textunderscore )}
\end{itemize}
Antigo instrumento de guerra, formado por uma espécie de lança e uma lanterna.
\section{Obeliscolýchnio}
\begin{itemize}
\item {Grp. gram.:m.}
\end{itemize}
\begin{itemize}
\item {Proveniência:(Gr. \textunderscore obelisko-luknion\textunderscore )}
\end{itemize}
Antigo instrumento de guerra, formado por uma espécie de lança e uma lanterna.
\section{Óbelo}
\begin{itemize}
\item {Grp. gram.:m.}
\end{itemize}
\begin{itemize}
\item {Proveniência:(Lat. \textunderscore obelus\textunderscore )}
\end{itemize}
Marca longitudinal, que se punha nas passagens erradas ou adulteradas de um escrito, para se emendarem na reproducção.
\section{Oberar}
\begin{itemize}
\item {Grp. gram.:v. t.}
\end{itemize}
Onerar com dívidas.
Impor obrigação ou encargo a.
(Cp. lat. \textunderscore obaeratus\textunderscore )
\section{Obérea}
\begin{itemize}
\item {Grp. gram.:f.}
\end{itemize}
Gênero de insectos longicórneos.
\section{Oberónia}
\begin{itemize}
\item {Grp. gram.:f.}
\end{itemize}
\begin{itemize}
\item {Proveniência:(De \textunderscore Oberon\textunderscore , n. p.)}
\end{itemize}
Gênero de orchídeas.
\section{Obesidade}
\begin{itemize}
\item {Grp. gram.:f.}
\end{itemize}
\begin{itemize}
\item {Proveniência:(Lat. \textunderscore obesitas\textunderscore )}
\end{itemize}
Qualidade de obeso.
Gordura excessiva de um indivíduo, com proemenência do ventre.
\section{Obeso}
\begin{itemize}
\item {Grp. gram.:adj.}
\end{itemize}
\begin{itemize}
\item {Proveniência:(Lat. \textunderscore obesus\textunderscore )}
\end{itemize}
Diz-se do indivíduo excessivamente gordo e com o ventre proeminente.
Cujos tecidos molles se desenvolveram e se avolumaram extraordinariamente.
Proeminente por hypertrophia do tecido adiposo, (falando-se do ventre).
\section{Obfirmadamente}
\begin{itemize}
\item {Grp. gram.:adv.}
\end{itemize}
De modo obfirmado.
Pertinazmente; obstinadamente.
\section{Obfirmado}
\begin{itemize}
\item {Grp. gram.:adj.}
\end{itemize}
\begin{itemize}
\item {Proveniência:(Lat. \textunderscore obfirmatus\textunderscore )}
\end{itemize}
Muito firme.
Pertinaz; contumaz.
\section{Obfirmar}
\begin{itemize}
\item {Grp. gram.:v. i.}
\end{itemize}
\begin{itemize}
\item {Utilização:Des.}
\end{itemize}
\begin{itemize}
\item {Proveniência:(Lat. \textunderscore obfirmare\textunderscore )}
\end{itemize}
Estar muito firme.
Têr pertinácia, obstinar-se.
\section{Obi}
\begin{itemize}
\item {Grp. gram.:m.}
\end{itemize}
Nome africano da noz de cola.
\section{Óbice}
\begin{itemize}
\item {Grp. gram.:m.}
\end{itemize}
\begin{itemize}
\item {Proveniência:(Lat. \textunderscore obex\textunderscore , \textunderscore obicis\textunderscore )}
\end{itemize}
Impedimento; obstáculo.
\section{Obipom}
\begin{itemize}
\item {Grp. gram.:m.}
\end{itemize}
Uma das línguas dos indígenas do Paraguai.
\section{Óbito}
\begin{itemize}
\item {Grp. gram.:m.}
\end{itemize}
\begin{itemize}
\item {Proveniência:(Lat. \textunderscore obitus\textunderscore )}
\end{itemize}
Fallecimento de pessôa; morte de alguém; passamento.
\section{Obituário}
\begin{itemize}
\item {Grp. gram.:adj.}
\end{itemize}
\begin{itemize}
\item {Grp. gram.:M.}
\end{itemize}
\begin{itemize}
\item {Utilização:Ant.}
\end{itemize}
Relativo a óbito.
Relação de óbitos; mortalidade.
Beneficiado, provido por morte de outrem.
\section{Objecção}
\begin{itemize}
\item {Grp. gram.:f.}
\end{itemize}
\begin{itemize}
\item {Grp. gram.:Pl.}
\end{itemize}
\begin{itemize}
\item {Utilização:Ant.}
\end{itemize}
\begin{itemize}
\item {Proveniência:(Lat. \textunderscore objectio\textunderscore )}
\end{itemize}
Acto ou effeito de objectar.
Óbice.
Dúvida.
Adjacência ou dependências (de uma herdade).
\section{Objectante}
\begin{itemize}
\item {Grp. gram.:adj.}
\end{itemize}
Que objecta.
\section{Objectar}
\begin{itemize}
\item {Grp. gram.:v. t.}
\end{itemize}
\begin{itemize}
\item {Proveniência:(Lat. \textunderscore objectare\textunderscore )}
\end{itemize}
Oppor-se a.
Allegar em sentido contrário.
Expor como dúvida ou como argumento, em opposição ao que outrem allegou ou procurou provar.
\section{Objectiva}
\begin{itemize}
\item {Grp. gram.:f.}
\end{itemize}
\begin{itemize}
\item {Proveniência:(De \textunderscore objectivo\textunderscore )}
\end{itemize}
Vidro ou lente, que está voltada para o objecto que se quer examinar.
Linha, tendente para um ponto, a que se quer chegar, em operações militares.
\section{Objectivação}
\begin{itemize}
\item {Grp. gram.:f.}
\end{itemize}
Acto de objectivar.
\section{Objectivamente}
\begin{itemize}
\item {Grp. gram.:adv.}
\end{itemize}
De modo objectivo.
Relativamente a objecto.
\section{Objectivar}
\begin{itemize}
\item {Grp. gram.:v. t.}
\end{itemize}
Tornar objectivo; considerar real ou existente fóra do espírito.
\section{Objectividade}
\begin{itemize}
\item {Grp. gram.:f.}
\end{itemize}
Qualidade do que é objectivo.
Existência real do que se concebeu no espírito.
Bella manifestação artística, estranha ou independente do carácter ou índole do respectivo artista.
\section{Objectivo}
\begin{itemize}
\item {Grp. gram.:adj.}
\end{itemize}
\begin{itemize}
\item {Utilização:Gram.}
\end{itemize}
\begin{itemize}
\item {Grp. gram.:M.}
\end{itemize}
\begin{itemize}
\item {Proveniência:(De \textunderscore objecto\textunderscore )}
\end{itemize}
Relativo a objectos.
Que diz respeito a objectos exteriores ao espírito.
Que está voltado para o objecto que se quer examinar, (falando-se de um vidro ou lente).
Diz-se da linha de operações militares, tendente para o ponto aonde se quer chegar.
Diz-se de um complemento que se junta immediatamente a um verbo transitivo, ou em que recái directamente a acção do verbo.
Alvo ou fim, a que queremos chegar.
Objecto de uma acção, de uma ideia ou de um sentimento.
\section{Objecto}
\begin{itemize}
\item {Grp. gram.:m.}
\end{itemize}
\begin{itemize}
\item {Proveniência:(Lat. \textunderscore objectum\textunderscore )}
\end{itemize}
Tudo aquillo que se apresenta aos nossos sentidos ou á nossa alma.
Tudo que a nossa vista póde observar.
Tudo que é exterior ao espírito.
Coisa material, sensível.
Assumpto.
Causa.
Intento, alvo.
\section{Objeito}
\begin{itemize}
\item {Grp. gram.:m.}
\end{itemize}
(Fórma des. de \textunderscore objecto\textunderscore , empregada por Camões)
\section{Objurgação}
\begin{itemize}
\item {Grp. gram.:f.}
\end{itemize}
\begin{itemize}
\item {Proveniência:(Lat. \textunderscore objurgatio\textunderscore )}
\end{itemize}
Acto de objurgar; reprehensão violenta; accusação.
\section{Objurgado}
\begin{itemize}
\item {Grp. gram.:adj.}
\end{itemize}
Reprehendido severamente.
Invectivado, apostrophado.
\section{Objurgatória}
\begin{itemize}
\item {Grp. gram.:f.}
\end{itemize}
\begin{itemize}
\item {Proveniência:(De \textunderscore objurgatório\textunderscore )}
\end{itemize}
O mesmo que \textunderscore objurgação\textunderscore .
\section{Objurgatório}
\begin{itemize}
\item {Grp. gram.:adj.}
\end{itemize}
\begin{itemize}
\item {Proveniência:(Lat. \textunderscore objurgatorius\textunderscore )}
\end{itemize}
Relativo a objurgação; que envolve objurgação.
\section{Objurgar}
\begin{itemize}
\item {Grp. gram.:v. t.}
\end{itemize}
\begin{itemize}
\item {Proveniência:(Lat. \textunderscore objurgare\textunderscore )}
\end{itemize}
Invectivar, censurar asperamente.
\section{Oblação}
\begin{itemize}
\item {Grp. gram.:f.}
\end{itemize}
\begin{itemize}
\item {Proveniência:(Lat. \textunderscore oblatio\textunderscore )}
\end{itemize}
Offerta.
Objecto, que se offerece á divindade ou aos santos; oblata.
Qualquer offerecimento.
Oferecimento do pão e do vinho a Deus, na Missa.
Missa, que se offerece a Deus.
\section{Obladagem}
\begin{itemize}
\item {Grp. gram.:f.}
\end{itemize}
\begin{itemize}
\item {Utilização:Ant.}
\end{itemize}
O mesmo que \textunderscore oblata\textunderscore .
\section{Oblata}
\begin{itemize}
\item {Grp. gram.:f.}
\end{itemize}
\begin{itemize}
\item {Grp. gram.:Pl.}
\end{itemize}
\begin{itemize}
\item {Proveniência:(De \textunderscore oblato\textunderscore )}
\end{itemize}
Tudo que se offerece a Deus ou aos santos, na igreja.
O pão e o vinho, que se offerecem a Deus, na Missa.
Qualquer offerta piedosa ou respeitosa.
Freiras de certa Ordem religiosa.
\section{Oblatar}
\begin{itemize}
\item {Grp. gram.:v. t.}
\end{itemize}
\begin{itemize}
\item {Utilização:Ant.}
\end{itemize}
\begin{itemize}
\item {Proveniência:(De \textunderscore oblata\textunderscore )}
\end{itemize}
Offerecer á divindade ou aos santos.
Offerecer.
\section{Oblativamente}
\begin{itemize}
\item {Grp. gram.:adv.}
\end{itemize}
Como oblata, como offerenda; de modo oblativo.
\section{Oblativo}
\begin{itemize}
\item {Grp. gram.:adj.}
\end{itemize}
Em que há oblata; que significa oblata. Cf. C. Neto, \textunderscore Saldunes\textunderscore .
\section{Oblato}
\begin{itemize}
\item {Grp. gram.:m.}
\end{itemize}
\begin{itemize}
\item {Utilização:Ant.}
\end{itemize}
\begin{itemize}
\item {Proveniência:(Lat. \textunderscore oblatus\textunderscore )}
\end{itemize}
Indivíduo, dado pelos pais a um convento, para serviço de Deus.
Leigo, que se offerecia para serviço de uma Ordem religiosa.
\section{Obligar}
\textunderscore v. t.\textunderscore  (e der.)
(Fórma ant. \textunderscore obrigar\textunderscore , etc.)
\section{Obligatório}
\begin{itemize}
\item {Grp. gram.:adj.}
\end{itemize}
\begin{itemize}
\item {Utilização:Des.}
\end{itemize}
O mesmo que \textunderscore obrigatório\textunderscore . Cf. \textunderscore Luz e Calor\textunderscore , 489.
\section{Obligulado}
\begin{itemize}
\item {Grp. gram.:adj.}
\end{itemize}
\begin{itemize}
\item {Utilização:Bot.}
\end{itemize}
\begin{itemize}
\item {Proveniência:(De \textunderscore ob...\textunderscore  + \textunderscore ligulado\textunderscore )}
\end{itemize}
Que se divide internamente em duas linguetas, (falando-se da corolla das flôres).
Diz-se da flôr, que tem corollas com essa divisão.
\section{Obliguliflóreo}
\begin{itemize}
\item {Grp. gram.:adj.}
\end{itemize}
\begin{itemize}
\item {Utilização:Bot.}
\end{itemize}
\begin{itemize}
\item {Proveniência:(De \textunderscore ob...\textunderscore  + \textunderscore ligula\textunderscore  + \textunderscore flóreo\textunderscore )}
\end{itemize}
Que tem flôres de corolla obligulada.
\section{Obliguliforme}
\begin{itemize}
\item {Grp. gram.:adj.}
\end{itemize}
\begin{itemize}
\item {Proveniência:(De \textunderscore ob...\textunderscore  + \textunderscore ligula\textunderscore  + \textunderscore fórma\textunderscore )}
\end{itemize}
Que tem fórma de corolla obligulada.
\section{Oblíqua}
\begin{itemize}
\item {Grp. gram.:f.}
\end{itemize}
\begin{itemize}
\item {Proveniência:(De \textunderscore oblíquo\textunderscore )}
\end{itemize}
Recta, que fórma com outra, ou com uma superfície, ângulo agudo ou obtuso.
\section{Obliquado}
\begin{itemize}
\item {Grp. gram.:adj.}
\end{itemize}
\begin{itemize}
\item {Proveniência:(De \textunderscore obliquar\textunderscore )}
\end{itemize}
Feito ou realizado obliquamente. Cf. C. Lobo, \textunderscore Sát. de Juven.\textunderscore  I, 147.
\section{Obliquamente}
\begin{itemize}
\item {Grp. gram.:adv.}
\end{itemize}
\begin{itemize}
\item {Utilização:Fig.}
\end{itemize}
De modo obliquo.
Indirectamente; por meios indirectos.
\section{Obliquângulo}
\begin{itemize}
\item {Grp. gram.:adj.}
\end{itemize}
\begin{itemize}
\item {Proveniência:(De \textunderscore obliquo\textunderscore  + \textunderscore ângulo\textunderscore )}
\end{itemize}
Que não tem ângulo algum recto, (falando-se de uma figura geométrica).
\section{Obliquar}
\begin{itemize}
\item {Grp. gram.:v. i.}
\end{itemize}
\begin{itemize}
\item {Proveniência:(Lat. \textunderscore obliquare\textunderscore )}
\end{itemize}
Caminhar obliquamente.
Andar de través.
Proceder maliciosamente, com dissimulação.
\section{Obliquidade}
\begin{itemize}
\item {fónica:qu-i}
\end{itemize}
\begin{itemize}
\item {Grp. gram.:f.}
\end{itemize}
Qualidade do que é obliquo.
Posição ou inclinação oblíqua.
\section{Oblíquo}
\begin{itemize}
\item {Grp. gram.:adj.}
\end{itemize}
\begin{itemize}
\item {Utilização:Náut.}
\end{itemize}
\begin{itemize}
\item {Utilização:Anat.}
\end{itemize}
\begin{itemize}
\item {Utilização:Geom.}
\end{itemize}
\begin{itemize}
\item {Utilização:Bot.}
\end{itemize}
\begin{itemize}
\item {Utilização:Fig.}
\end{itemize}
\begin{itemize}
\item {Utilização:Gram.}
\end{itemize}
\begin{itemize}
\item {Proveniência:(Lat. \textunderscore obliquus\textunderscore )}
\end{itemize}
Que não é perpendicular ou direito.
Que se afasta da linha recta.
Inclinado.
Que vai de través.
Em Astronomia, diz-se da esphera em que um dos pólos está acima do horizonte e o outro abaixo, de maneira que o equador e os seus parallelos são oblíquos ao horizonte.
Na arte militar, diz-se da ordem de batalha, em que só uma das alas se expõe ao inimigo, emquanto a outra, retrahindo-se, procura rodear e envolver as fôrças contrárias.
Diz-se do passo ou marcha em linha diagonal, tirada do ponto da partida para o da chegada.
Diz-se da marcha de um navio e que segue rumo intermediário aos pontos cardeaes.
Diz-se de vários músculos, cuja acção se não exerce parallelamente aos planos que dividem perpendicularmente o corpo.
Diz-se do sólido, cujo eixo não é perpendicular á base.
Diz-se da parte de um vegetal, que se desvia do plano do horizonte ou do eixo da planta.
Indirecto.
Feito com malícia.
Dissimulado.
Tortuoso.
Diz-se dos casos da declinação, exceptuado o nominativo, que se chama caso \textunderscore directo\textunderscore .
\section{Obliteração}
\begin{itemize}
\item {Grp. gram.:f.}
\end{itemize}
\begin{itemize}
\item {Utilização:Anat.}
\end{itemize}
\begin{itemize}
\item {Proveniência:(Lat. \textunderscore obliteratio\textunderscore )}
\end{itemize}
Acto ou effeito de obliterar.
Estado de um canal, obstruído por terem adherido as suas paredes ou pela presença de um corpo sólido.
\section{Obliterado}
\begin{itemize}
\item {Grp. gram.:adj.}
\end{itemize}
\begin{itemize}
\item {Utilização:Anat.}
\end{itemize}
Expungido; desvanecido; extinto.
Em que ha obliteração.
\section{Obliterar}
\begin{itemize}
\item {Grp. gram.:v. t.}
\end{itemize}
\begin{itemize}
\item {Proveniência:(Lat. \textunderscore obliterare\textunderscore )}
\end{itemize}
Expungir.
Apagar.
Fazer esquecer.
Obstruír.
Fechar o canal ou cavidade de.
\section{Oblívio}
\begin{itemize}
\item {Grp. gram.:m.}
\end{itemize}
\begin{itemize}
\item {Proveniência:(Lat. \textunderscore oblivium\textunderscore )}
\end{itemize}
O mesmo que \textunderscore olvido\textunderscore  ou \textunderscore esquecimento\textunderscore :«\textunderscore as sciências estavam em tal oblívio, que fizera corar de pejo a uma nação...\textunderscore »Latino, \textunderscore Hist. Pol. e Mil.\textunderscore , I, 299.
\section{Oblongado}
\begin{itemize}
\item {Grp. gram.:adj.}
\end{itemize}
\begin{itemize}
\item {Utilização:Anat.}
\end{itemize}
\begin{itemize}
\item {Proveniência:(De \textunderscore oblongo\textunderscore )}
\end{itemize}
Que tem fórma oblongada.
\textunderscore Medulla oblongada\textunderscore , o bolbo rachidiano.
\section{Oblongifólio}
\begin{itemize}
\item {Grp. gram.:adj.}
\end{itemize}
\begin{itemize}
\item {Utilização:Bot.}
\end{itemize}
\begin{itemize}
\item {Proveniência:(Do lat. \textunderscore oblongus\textunderscore  + \textunderscore folium\textunderscore )}
\end{itemize}
Que tem fôlhas oblongas.
\section{Oblongo}
\begin{itemize}
\item {Grp. gram.:adj.}
\end{itemize}
\begin{itemize}
\item {Proveniência:(Lat. \textunderscore oblongus\textunderscore )}
\end{itemize}
Alongado.
Que tem mais comprimento que largura.
Oval; ellíptico.
\section{Obmissa}
\begin{itemize}
\item {Grp. gram.:adj. f.}
\end{itemize}
\begin{itemize}
\item {Utilização:Mús.}
\end{itemize}
\begin{itemize}
\item {Utilização:ant.}
\end{itemize}
\begin{itemize}
\item {Proveniência:(Lat. \textunderscore obmissos\textunderscore )}
\end{itemize}
\textunderscore Figura obmissa\textunderscore , o mesmo que \textunderscore pausa\textunderscore .
\section{Obnoxiação}
\begin{itemize}
\item {fónica:csi}
\end{itemize}
\begin{itemize}
\item {Grp. gram.:f.}
\end{itemize}
\begin{itemize}
\item {Utilização:Ant.}
\end{itemize}
\begin{itemize}
\item {Proveniência:(De \textunderscore obnóxio\textunderscore )}
\end{itemize}
Cedência da propriedade de sua pessôa ou bens a outrem.
\section{Obnóxio}
\begin{itemize}
\item {fónica:csi}
\end{itemize}
\begin{itemize}
\item {Grp. gram.:adj.}
\end{itemize}
\begin{itemize}
\item {Proveniência:(Lat. \textunderscore obnoxius\textunderscore )}
\end{itemize}
Que se sujeita á punição.
Servil.
Desprezível.
Funesto.
Nefasto; nefando.
\section{Obnubilação}
\begin{itemize}
\item {Grp. gram.:f.}
\end{itemize}
\begin{itemize}
\item {Utilização:Med.}
\end{itemize}
\begin{itemize}
\item {Proveniência:(Do lat. \textunderscore ob\textunderscore  + \textunderscore nubilus\textunderscore )}
\end{itemize}
Deslumbramento ou trevas, phenómeno que se sente nos pródromos de certas doenças ou em consequência de outras.
\section{Oboaz}
\begin{itemize}
\item {Grp. gram.:m.}
\end{itemize}
\begin{itemize}
\item {Utilização:Ant.}
\end{itemize}
O mesmo que \textunderscore bujamé\textunderscore .
O mesmo que \textunderscore oboé\textunderscore .
\section{Oboé}
\begin{itemize}
\item {Grp. gram.:m.}
\end{itemize}
Instrumento musical de sopro, feito de madeira, com palheta dupla.
Registo de órgão.
Registo de harmónios.
\section{Oboísta}
\begin{itemize}
\item {Grp. gram.:m.  e  f.}
\end{itemize}
\begin{itemize}
\item {Proveniência:(Do fr. \textunderscore haut-bois\textunderscore )}
\end{itemize}
Pessôa, que toca oboé.
\section{Óbolo}
\begin{itemize}
\item {Grp. gram.:m.}
\end{itemize}
\begin{itemize}
\item {Utilização:Fig.}
\end{itemize}
\begin{itemize}
\item {Proveniência:(Lat. \textunderscore obolus\textunderscore )}
\end{itemize}
Pequena moéda grega.
Pequeno donativo; esmola.
\section{Obongos}
\begin{itemize}
\item {Grp. gram.:m. pl.}
\end{itemize}
Indígenas das margens do Zaire, de pequena estatura e cabello arruivado.
\section{Oboval}
\begin{itemize}
\item {Grp. gram.:adj.}
\end{itemize}
O mesmo que \textunderscore obóveo\textunderscore .
\section{Obovalado}
\begin{itemize}
\item {Grp. gram.:adj.}
\end{itemize}
O mesmo que \textunderscore obóveo\textunderscore .
\section{Obóveo}
\begin{itemize}
\item {Grp. gram.:adj.}
\end{itemize}
\begin{itemize}
\item {Proveniência:(De \textunderscore ob...\textunderscore  + \textunderscore óveo\textunderscore )}
\end{itemize}
Que tem fórma de um ovo invertido.
\section{Obovóide}
\begin{itemize}
\item {Grp. gram.:adj.}
\end{itemize}
O mesmo que \textunderscore obóveo\textunderscore .
\section{Obra}
\begin{itemize}
\item {Grp. gram.:f.}
\end{itemize}
\begin{itemize}
\item {Utilização:Pop.}
\end{itemize}
\begin{itemize}
\item {Grp. gram.:Loc. prep.}
\end{itemize}
\begin{itemize}
\item {Utilização:Náut.}
\end{itemize}
\begin{itemize}
\item {Utilização:Ant.}
\end{itemize}
\begin{itemize}
\item {Grp. gram.:Loc.}
\end{itemize}
\begin{itemize}
\item {Utilização:fig.}
\end{itemize}
\begin{itemize}
\item {Proveniência:(Do lat. \textunderscore opera\textunderscore )}
\end{itemize}
Effeito do trabalho.
Resultado de uma acção.
Acção.
Trabalho.
Producção literária, sciêntifica ou artística.
Construcção.
Trapaça; tramoia.
\textunderscore Obra acabada\textunderscore , obra perfeita.
\textunderscore Obra de\textunderscore , proximamente a, pouco mais ou menos que:«\textunderscore parece que o granito lhe entrou dentro\textunderscore  (da cabeça) \textunderscore obra de meia pollegada.\textunderscore »Camillo, \textunderscore O Bem e o Mal\textunderscore , 161.
\textunderscore Obras mortas\textunderscore , a parte do navio, que comprehende os castellos da popa, da primeira coberta para cima.
\textunderscore Pau para toda a obra\textunderscore , pessôa ou coisa, que serve para tudo, ou que a tudo se applica.
\section{Obração}
\begin{itemize}
\item {Grp. gram.:f.}
\end{itemize}
\begin{itemize}
\item {Utilização:Ant.}
\end{itemize}
O mesmo que \textunderscore oblação\textunderscore .
\section{Obraçom}
\begin{itemize}
\item {Grp. gram.:f.}
\end{itemize}
\begin{itemize}
\item {Utilização:Ant.}
\end{itemize}
O mesmo que \textunderscore oblação\textunderscore .
\section{Obra-córnea}
\begin{itemize}
\item {Grp. gram.:f.}
\end{itemize}
\begin{itemize}
\item {Utilização:T. de fortificação}
\end{itemize}
Frente, abalaustrada com flancos.
\section{Obrada}
\begin{itemize}
\item {Grp. gram.:f.}
\end{itemize}
(Corr. de \textunderscore oblata\textunderscore )
\section{Obradação}
\begin{itemize}
\item {Grp. gram.:f.}
\end{itemize}
\begin{itemize}
\item {Utilização:Ant.}
\end{itemize}
O mesmo que \textunderscore oblação\textunderscore .
\section{Obradar}
\begin{itemize}
\item {Grp. gram.:v. t.}
\end{itemize}
(V.oblatar)
\section{Obradeira}
\begin{itemize}
\item {Grp. gram.:f.}
\end{itemize}
\begin{itemize}
\item {Utilização:Des.}
\end{itemize}
\begin{itemize}
\item {Proveniência:(De \textunderscore obrada\textunderscore )}
\end{itemize}
Ferro, para fazer hóstias.
Mulher, que apresentava na igreja as offertas deixadas por um testador.
\section{Obrador}
\begin{itemize}
\item {Grp. gram.:m.  e  adj.}
\end{itemize}
\begin{itemize}
\item {Grp. gram.:M.}
\end{itemize}
O que obra.
Obreiro.
\section{Obradório}
\begin{itemize}
\item {Grp. gram.:m.}
\end{itemize}
\begin{itemize}
\item {Utilização:Prov.}
\end{itemize}
\begin{itemize}
\item {Utilização:minh.}
\end{itemize}
\begin{itemize}
\item {Proveniência:(De \textunderscore obrada\textunderscore )}
\end{itemize}
Offerta, que a família de um defunto, no domingo seguinte ao entêrro, manda num cesto ao párocho, e que consta de uma brôa, um bacalhau e uma garrafa de vinho.
\section{Obragem}
\begin{itemize}
\item {Grp. gram.:f.}
\end{itemize}
\begin{itemize}
\item {Proveniência:(De \textunderscore obra\textunderscore )}
\end{itemize}
Obra, execução, lavor de artista.
Acto de construír.
\section{Obrante}
\begin{itemize}
\item {Grp. gram.:adj.}
\end{itemize}
\begin{itemize}
\item {Proveniência:(Lat. \textunderscore operans\textunderscore )}
\end{itemize}
Que obra.
Que é causa ou origem.
\section{Obrar}
\begin{itemize}
\item {Grp. gram.:v. t.}
\end{itemize}
\begin{itemize}
\item {Grp. gram.:V. i.}
\end{itemize}
\begin{itemize}
\item {Proveniência:(Do lat. \textunderscore operare\textunderscore )}
\end{itemize}
Converter em obra.
Realizar; executar.
Fabricar.
Fazer lavor em.
Praticar um acto.
Realizar um trabalho.
Trabalhar.
Têr effeito um medicamento.
Têr resultado.
Fazer dejecção.
\section{Obreeiro}
\begin{itemize}
\item {Grp. gram.:m.}
\end{itemize}
Aquelle que faz obreias. Cf. \textunderscore Eufrosina\textunderscore , 72.
\section{Obregão}
\begin{itemize}
\item {Grp. gram.:m.}
\end{itemize}
\begin{itemize}
\item {Utilização:Ant.}
\end{itemize}
\begin{itemize}
\item {Proveniência:(De \textunderscore obra\textunderscore )}
\end{itemize}
Trabalhador, operário.
\section{Obreia}
\begin{itemize}
\item {Grp. gram.:f.}
\end{itemize}
\begin{itemize}
\item {Utilização:Ant.}
\end{itemize}
Folha de massa, de que se faz a hóstia para o offício divino e as partículas para a communhão.
Pequena fôlha de massa, de vários feitios e cores, para fechar cartas, pegar papéis, etc.
Oblata da Missa.
(Cp. \textunderscore obrada\textunderscore )
\section{Obreira}
\begin{itemize}
\item {Grp. gram.:f.}
\end{itemize}
\begin{itemize}
\item {Proveniência:(De \textunderscore obreiro\textunderscore )}
\end{itemize}
Operária.
Cada uma das abelhas, que formara uma colmeia, presididas pela abelha-mestra.
\section{Obreiro}
\begin{itemize}
\item {Grp. gram.:m.}
\end{itemize}
\begin{itemize}
\item {Grp. gram.:Adj.}
\end{itemize}
\begin{itemize}
\item {Utilização:Gal}
\end{itemize}
\begin{itemize}
\item {Proveniência:(De \textunderscore obrar\textunderscore , se não do fr. \textunderscore ouvrier\textunderscore )}
\end{itemize}
Aquelle que trabalha.
Trabalhador.
Operário.
Aquelle que obra.
Cultivador.
Aquelle que coopera no desenvolvimento de uma empresa ou de uma ideia.
Que trabalha, (falando-se das abelhas presididas pela abelha-mestra).
\section{Obrejar}
\begin{itemize}
\item {Grp. gram.:v. i.}
\end{itemize}
\begin{itemize}
\item {Utilização:Prov.}
\end{itemize}
\begin{itemize}
\item {Utilização:minh.}
\end{itemize}
Tremer, tiritar, (com frio).
(Relaciona-se com \textunderscore varejar\textunderscore ?)
\section{Obrepção}
\begin{itemize}
\item {fónica:re}
\end{itemize}
\begin{itemize}
\item {Grp. gram.:f.}
\end{itemize}
\begin{itemize}
\item {Proveniência:(Lat. \textunderscore obreptio\textunderscore )}
\end{itemize}
Acto de obter qualquer coisa ardilosamente ou por surpresa.
Manha; cavillação, ardil.
\section{Obrepticiamente}
\begin{itemize}
\item {fónica:re}
\end{itemize}
\begin{itemize}
\item {Grp. gram.:adv.}
\end{itemize}
De modo obreptício; fraudulentamente; ardilosamente.
\section{Obreptício}
\begin{itemize}
\item {fónica:re}
\end{itemize}
\begin{itemize}
\item {Grp. gram.:adj.}
\end{itemize}
\begin{itemize}
\item {Proveniência:(Lat. \textunderscore obrepticius\textunderscore )}
\end{itemize}
Obtido por obrepção.
Ardiloso; fraudulento.
\section{Obriga}
\begin{itemize}
\item {Grp. gram.:f.}
\end{itemize}
\begin{itemize}
\item {Proveniência:(De \textunderscore obrigar\textunderscore )}
\end{itemize}
O mesmo que \textunderscore obrigação\textunderscore .
Imposto, que se pagava por exportar peixe.
\section{Obrigação}
\begin{itemize}
\item {Grp. gram.:f.}
\end{itemize}
\begin{itemize}
\item {Utilização:Pop.}
\end{itemize}
\begin{itemize}
\item {Proveniência:(Do lat. \textunderscore obligatio\textunderscore )}
\end{itemize}
Acto de obrigar.
A necessidade moral de praticar ou não praticar um acto.
Favor, obséquio: \textunderscore devo-lhe muitas obrigações\textunderscore .
Dever.
Preceito.
Sujeição.
Dívida.
Título de dívida ou de um contrato.
Escritura.
Cláusula numa convenção.
Offício: \textunderscore foi trabalhar na sua obrigação\textunderscore .
Documento ou título, cujo senhor tem direito a determinado interesse nos lucros de uma associação ou empresa, sem confusão com os direitos dos accionistas: \textunderscore comprou obrigações\textunderscore .
Intimidade.
Pessôas de família.
A família: \textunderscore então como passou? e a sua obrigação como vai\textunderscore ?
\section{Obrigacionário}
\begin{itemize}
\item {Grp. gram.:m.}
\end{itemize}
\begin{itemize}
\item {Utilização:P. us.}
\end{itemize}
O mesmo que \textunderscore obrigacionista\textunderscore .
\section{Obrigacionista}
\begin{itemize}
\item {Grp. gram.:m.  e  f.}
\end{itemize}
\begin{itemize}
\item {Proveniência:(De \textunderscore obrigação\textunderscore )}
\end{itemize}
Pessôa, que tem títulos, chamados \textunderscore obrigações\textunderscore .
Obrigatário.
\section{Obrigado}
\begin{itemize}
\item {Grp. gram.:adj.}
\end{itemize}
Agradecido.
Imposto.
Exigido: \textunderscore serviço obrigado\textunderscore .
Indispensável.
Necessário.
Forçado.
Que tem de se fazer ou de se apresentar.
Cativado por obséquios, amabilidades ou serviços.
Obsequiado; grato, agradecido: \textunderscore muito obrigado, minha senhora\textunderscore .
\section{Obrigador}
\begin{itemize}
\item {Grp. gram.:m.  e  adj.}
\end{itemize}
O que obriga.
Que pelo seu trato, obséquios ou serviços, se impõe ao agradecimento de outrem.
\section{Obrigamento}
\begin{itemize}
\item {Grp. gram.:m.}
\end{itemize}
O mesmo que \textunderscore obrigação\textunderscore .
\section{Obrigante}
\begin{itemize}
\item {Grp. gram.:adj.}
\end{itemize}
Que obriga.
Que cativa com finezas ou obséquios.
\section{Obrigar}
\begin{itemize}
\item {Grp. gram.:v. t.}
\end{itemize}
\begin{itemize}
\item {Grp. gram.:V. i.}
\end{itemize}
\begin{itemize}
\item {Proveniência:(Do lat. \textunderscore obligare\textunderscore )}
\end{itemize}
Mandar, preceituar.
Sujeitar; constranger.
Estimular, mover: \textunderscore obrigar o cavallo a andar\textunderscore .
Ligar ou attrahir por meio de finezas ou obséquios.
Tornar grato; cativar.
Fazer curvar, alterar a posição de.
Impor o cumprimento de cláusulas ou de deveres.
Exigir formalidades ou cumprimento de certos deveres.
\section{Obrigatário}
\begin{itemize}
\item {Grp. gram.:m.}
\end{itemize}
\begin{itemize}
\item {Proveniência:(De \textunderscore obrigar\textunderscore . Cp. fr. \textunderscore obligataire\textunderscore )}
\end{itemize}
Portador ou possuidor de títulos do obrigação, emittidos pelo Govêrno ou por uma Companhia.
\section{Obrigativo}
\begin{itemize}
\item {Grp. gram.:adj.}
\end{itemize}
\begin{itemize}
\item {Proveniência:(Do lat. \textunderscore obligatus\textunderscore )}
\end{itemize}
O mesmo que \textunderscore obrigatório\textunderscore . Cf. Castilho, \textunderscore Fastos\textunderscore , III, 488.
\section{Obrigatoriamente}
\begin{itemize}
\item {Grp. gram.:adv.}
\end{itemize}
De modo obrigatório.
Forçosamente; necessariamente.
\section{Obrigatoriedade}
\begin{itemize}
\item {Grp. gram.:f.}
\end{itemize}
Qualidade de obrigatório: \textunderscore a obrigatoriedade do primeiro ensino\textunderscore .
\section{Obrigatório}
\begin{itemize}
\item {Grp. gram.:adj.}
\end{itemize}
\begin{itemize}
\item {Proveniência:(Do lat. \textunderscore obligatorius\textunderscore )}
\end{itemize}
Que envolve obrigação.
Que tem o poder de obrigar.
Forçoso.
Imposto por lei.
\section{Obringente}
\begin{itemize}
\item {fónica:rin}
\end{itemize}
\begin{itemize}
\item {Grp. gram.:adj.}
\end{itemize}
\begin{itemize}
\item {Utilização:Bot.}
\end{itemize}
\begin{itemize}
\item {Proveniência:(Lat. \textunderscore ob\textunderscore  + \textunderscore ringens\textunderscore )}
\end{itemize}
Que tem bôca revirada, (falando se especialmente da corolla).
\section{Obrio}
\begin{itemize}
\item {Grp. gram.:m.}
\end{itemize}
Gênero de insectos coleópteros longicórneos.
\section{Obrogação}
\begin{itemize}
\item {fónica:ro}
\end{itemize}
\begin{itemize}
\item {Grp. gram.:f.}
\end{itemize}
Acto ou effeito de obrogar.
\section{Obrogar}
\begin{itemize}
\item {fónica:ro}
\end{itemize}
\begin{itemize}
\item {Grp. gram.:v. i.}
\end{itemize}
\begin{itemize}
\item {Utilização:Jur.}
\end{itemize}
\begin{itemize}
\item {Proveniência:(Lat. \textunderscore obrogare\textunderscore )}
\end{itemize}
Derogar uma lei; contrapôr-se uma lei a outra. Cf. V. Ferrer, \textunderscore Dir. Nat.\textunderscore , 43.
\section{Obscenamente}
\begin{itemize}
\item {Grp. gram.:adv.}
\end{itemize}
De modo obsceno.
Com sensualidade; libidinosamente.
Indecentemente.
\section{Obscenidade}
\begin{itemize}
\item {Grp. gram.:f.}
\end{itemize}
\begin{itemize}
\item {Proveniência:(Lat. \textunderscore obscenitas\textunderscore )}
\end{itemize}
Qualidade do que é obsceno.
Acto, dito ou coisa obscena.
Sensualidade.
\section{Obsceno}
\begin{itemize}
\item {Grp. gram.:adj.}
\end{itemize}
\begin{itemize}
\item {Proveniência:(Lat. \textunderscore obscenus\textunderscore )}
\end{itemize}
Torpe; opposto ao pudor: \textunderscore palavras obscenas\textunderscore .
Impuro; sensual.
Que diz ou escreve torpezas ou obscenidades.
\section{Obscuração}
\begin{itemize}
\item {Grp. gram.:f.}
\end{itemize}
\begin{itemize}
\item {Proveniência:(Lat. \textunderscore obscuratio\textunderscore )}
\end{itemize}
Obscurecimento atmosphérico.
\section{Obscuramente}
\begin{itemize}
\item {Grp. gram.:adv.}
\end{itemize}
\begin{itemize}
\item {Utilização:Fig.}
\end{itemize}
De modo obscuro; sem luz.
De modo incógnito; humildemente: \textunderscore viver obscuramente\textunderscore .
\section{Obscurante}
\begin{itemize}
\item {Grp. gram.:adj.}
\end{itemize}
\begin{itemize}
\item {Utilização:Fig.}
\end{itemize}
\begin{itemize}
\item {Proveniência:(Lat. \textunderscore obscurans\textunderscore )}
\end{itemize}
Que obscurece.
Sectário do obscurantismo.
\section{Obscurantismo}
\begin{itemize}
\item {Grp. gram.:m.}
\end{itemize}
\begin{itemize}
\item {Utilização:Fig.}
\end{itemize}
\begin{itemize}
\item {Proveniência:(De \textunderscore obscurante\textunderscore )}
\end{itemize}
Estado do que se acha na escuridão.
Estado de completa ignorância.
Opposição systemática a todo o progresso ou movimento intellectual.
\section{Obscurantista}
\begin{itemize}
\item {Grp. gram.:m. ,  f.  e  adj.}
\end{itemize}
\begin{itemize}
\item {Proveniência:(De \textunderscore obscurante\textunderscore )}
\end{itemize}
Pessôa, que segue as ideias do obscurantismo; obscurante.
\section{Obscurantizar}
\begin{itemize}
\item {Grp. gram.:v. t.}
\end{itemize}
Tornar obscurante.
\section{Obscurecer}
\begin{itemize}
\item {Grp. gram.:v. t.}
\end{itemize}
\begin{itemize}
\item {Utilização:Fig.}
\end{itemize}
\begin{itemize}
\item {Grp. gram.:V. i.}
\end{itemize}
Tornar obscuro.
Enfraquecer; turvar: \textunderscore obscurecer as ideias\textunderscore .
Esconder: \textunderscore as nuvens obscureceram o Sol\textunderscore .
Tornar pouco visível ou pouco intelligivel.
Tornar triste.
Confundir.
Avantajar-se a; supplantar.
Deslustrar.
Tornar-se obscuro.
\section{Obscurecido}
\begin{itemize}
\item {Grp. gram.:adj.}
\end{itemize}
\begin{itemize}
\item {Utilização:Fig.}
\end{itemize}
Em que há pouca luz ou nenhuma.
Deslumbrado.
Esquecido; despercebido.
\section{Obscurecimento}
\begin{itemize}
\item {Grp. gram.:m.}
\end{itemize}
Acto ou effeito de obscurecer.
Escuridão; escassez ou ausência de luz.
\section{Obscureza}
\begin{itemize}
\item {Grp. gram.:f.}
\end{itemize}
O mesmo que \textunderscore obscuridade\textunderscore .
\section{Obscuridade}
\begin{itemize}
\item {Grp. gram.:f.}
\end{itemize}
\begin{itemize}
\item {Utilização:Fig.}
\end{itemize}
\begin{itemize}
\item {Proveniência:(Lat. \textunderscore obscuritas\textunderscore )}
\end{itemize}
Estado do que é obscuro; obscurecimento.
Falta de clareza nas palavras ou nas ideias.
Baixa condição, humildade de nascimento.
\section{Obscuro}
\begin{itemize}
\item {Grp. gram.:adj.}
\end{itemize}
\begin{itemize}
\item {Utilização:Fig.}
\end{itemize}
\begin{itemize}
\item {Proveniência:(Lat. \textunderscore obscurus\textunderscore )}
\end{itemize}
Que não tem luz; muito escuro, tenebroso.
Sombrio.
Pouco claro.
Opaco.
Que se percebe com difficuldade: \textunderscore palavras de sentido obscuro\textunderscore .
Que vive retrahido ou na obscuridade.
Secreto.
Afastado, occulto.
Indistinto; mal definido.
\section{Obsecração}
\begin{itemize}
\item {Grp. gram.:f.}
\end{itemize}
\begin{itemize}
\item {Proveniência:(Lat. \textunderscore obsecratio\textunderscore )}
\end{itemize}
Acto de obsecrar.
Súpplica fervorosa e humilde.
Palavras, com que se obsecra.
\section{Obsecrar}
\begin{itemize}
\item {Grp. gram.:v. t.}
\end{itemize}
\begin{itemize}
\item {Utilização:Des.}
\end{itemize}
\begin{itemize}
\item {Proveniência:(Lat. \textunderscore obsecrare\textunderscore )}
\end{itemize}
Pedir humildemente; rogar com instância.
Supplicar.
\section{Obsequente}
\begin{itemize}
\item {fónica:ze-cu-en}
\end{itemize}
\begin{itemize}
\item {Grp. gram.:adj.}
\end{itemize}
\begin{itemize}
\item {Proveniência:(Lat. \textunderscore obsequens\textunderscore )}
\end{itemize}
Obediente.
Favorável.
Obsequiador.
Que se conforma com a opinião de outrem; condescendente.
\section{Obséquia}
\begin{itemize}
\item {fónica:zé}
\end{itemize}
\begin{itemize}
\item {Grp. gram.:f.}
\end{itemize}
(V.obséquias)
\section{Obsequiador}
\begin{itemize}
\item {fónica:ze}
\end{itemize}
\begin{itemize}
\item {Grp. gram.:m.  e  adj.}
\end{itemize}
O que obsequia.
\section{Obsequiar}
\begin{itemize}
\item {fónica:ze}
\end{itemize}
\begin{itemize}
\item {Grp. gram.:v. t.}
\end{itemize}
\begin{itemize}
\item {Proveniência:(Lat. \textunderscore obsequi\textunderscore )}
\end{itemize}
Fazer obséquios a.
Favorecer.
Presentear.
Tratar agradavelmente; cativar.
\section{Obséquias}
\begin{itemize}
\item {fónica:zé}
\end{itemize}
\begin{itemize}
\item {Grp. gram.:f. pl.}
\end{itemize}
\begin{itemize}
\item {Proveniência:(Lat. \textunderscore obsequiae\textunderscore )}
\end{itemize}
O mesmo que exéquias.
\section{Obséquio}
\begin{itemize}
\item {fónica:zé}
\end{itemize}
\begin{itemize}
\item {Grp. gram.:m.}
\end{itemize}
\begin{itemize}
\item {Proveniência:(Lat. \textunderscore obsequium\textunderscore )}
\end{itemize}
Acção, com que se é agradável ou prestadio a alguém.
Benevolência; favor.
\section{Obsequiosamente}
\begin{itemize}
\item {fónica:ze}
\end{itemize}
\begin{itemize}
\item {Grp. gram.:adv.}
\end{itemize}
De modo obsequioso.
\section{Obsequiosidade}
\begin{itemize}
\item {fónica:ze}
\end{itemize}
\begin{itemize}
\item {Grp. gram.:f.}
\end{itemize}
Qualidade do que é obsequioso.
Benevolência; trato affável.
\section{Obsequioso}
\begin{itemize}
\item {fónica:ze}
\end{itemize}
\begin{itemize}
\item {Grp. gram.:adj.}
\end{itemize}
\begin{itemize}
\item {Proveniência:(Lat. \textunderscore obsequiosus\textunderscore )}
\end{itemize}
Que faz obséquios.
Que envolve obséquio ou tem a natureza delle.
Amável; benévolo.
\section{Observação}
\begin{itemize}
\item {Grp. gram.:f.}
\end{itemize}
\begin{itemize}
\item {Proveniência:(Lat. \textunderscore observatio\textunderscore )}
\end{itemize}
Acto ou effeito de observar.
Execução de preceito ou regra; cumprimento.
Exame minucioso.
Anályse.
Ponderação ou reflexão elucidativa.
Advertência benévola; conselho amigável: \textunderscore fazer observações a alguém\textunderscore .
Pequena censura.
Indagação, feita por um corpo de tropas, da situação e condições das tropas inimigas.
Qualidade ou acto de quem observa claramente factos psychológicos ou factos do mundo material.
Acto de vigiar ou de espiar: \textunderscore o criado manteve-se em observação\textunderscore .
\section{Observacional}
\begin{itemize}
\item {Grp. gram.:adj.}
\end{itemize}
\begin{itemize}
\item {Utilização:Neol.}
\end{itemize}
Relativo a observação.
\section{Observadamente}
\begin{itemize}
\item {Grp. gram.:adv.}
\end{itemize}
\begin{itemize}
\item {Proveniência:(De \textunderscore observado\textunderscore )}
\end{itemize}
Por meio de observação.
Minuciosamente.
\section{Observado}
\begin{itemize}
\item {Grp. gram.:adj.}
\end{itemize}
\begin{itemize}
\item {Proveniência:(De \textunderscore observar\textunderscore )}
\end{itemize}
Executado, cumprido.
Analysado; ponderado.
\section{Observador}
\begin{itemize}
\item {Grp. gram.:adj.}
\end{itemize}
\begin{itemize}
\item {Grp. gram.:M.}
\end{itemize}
\begin{itemize}
\item {Proveniência:(Lat. \textunderscore observator\textunderscore )}
\end{itemize}
Que observa.
Aquelle que observa.
Indivíduo, considerado em relação ao ponto que occupa na superfície do globo e aos objectos e phenómenos que o rodeiam.
Espectador.
O encarregado de observar e registar phenómenos de ordem scientífica.
\section{Observância}
\begin{itemize}
\item {Grp. gram.:f.}
\end{itemize}
\begin{itemize}
\item {Proveniência:(Lat. \textunderscore observantia\textunderscore )}
\end{itemize}
O mesmo que \textunderscore observação\textunderscore .
Execução fiel, cumprimento.
Uso.
Disciplina; penitência.
\section{Observante}
\begin{itemize}
\item {Grp. gram.:m.  e  adj.}
\end{itemize}
\begin{itemize}
\item {Proveniência:(Lat. \textunderscore observans\textunderscore )}
\end{itemize}
O que observa.
Frade de uma Ordem religiosa, da observância de San-Francisco.
\section{Observantino}
\begin{itemize}
\item {Grp. gram.:adj.}
\end{itemize}
\begin{itemize}
\item {Grp. gram.:M.}
\end{itemize}
Relativo aos observantes franciscanos.
Frade observante.
\section{Observar}
\begin{itemize}
\item {Grp. gram.:v. t.}
\end{itemize}
\begin{itemize}
\item {Proveniência:(Lat. \textunderscore observare\textunderscore )}
\end{itemize}
Olhar attentamente para: \textunderscore observar os astros\textunderscore .
Examinar minuciosamente: \textunderscore observar um ferimento\textunderscore .
Espreitar; espiar: \textunderscore observar os passos de alguém\textunderscore .
Estudar.
Cumprir rigorosamente: \textunderscore observar um preceito\textunderscore .
Guardar.
Obedecer a.
Tomar por modêlo.
Fazer notar.
Censurar levemente: \textunderscore observou-lhe que é preciso estudar\textunderscore .
Ponderar; replicar.
\section{Observatório}
\begin{itemize}
\item {Grp. gram.:m.}
\end{itemize}
\begin{itemize}
\item {Proveniência:(De \textunderscore observar\textunderscore )}
\end{itemize}
O mesmo que \textunderscore observação\textunderscore .
Lugar, donde se observa.
Mirante.
Edificio para observações astronómicas e meteorológicas.
\section{Observável}
\begin{itemize}
\item {Grp. gram.:adj.}
\end{itemize}
\begin{itemize}
\item {Proveniência:(Lat. \textunderscore observabilis\textunderscore )}
\end{itemize}
Que póde ou merece sêr observado.
\section{Obsessão}
\begin{itemize}
\item {Grp. gram.:f.}
\end{itemize}
\begin{itemize}
\item {Utilização:Fig.}
\end{itemize}
\begin{itemize}
\item {Proveniência:(Lat. \textunderscore obsessio\textunderscore )}
\end{itemize}
Impertinência excessiva.
Acto ou effeito de importunar ou vexar.
Perseguição diabólica.
Vexame, attribuido á influência do demónio.
Preoccupação constante, ideia fixa.
\section{Obsesso}
\begin{itemize}
\item {Grp. gram.:adj.}
\end{itemize}
\begin{itemize}
\item {Grp. gram.:M.}
\end{itemize}
\begin{itemize}
\item {Proveniência:(Lat. \textunderscore obsessus\textunderscore )}
\end{itemize}
Importunado, vexado.
Indivíduo, que se suppõe atormentado pela influência do demónio.
\section{Obsessor}
\begin{itemize}
\item {Grp. gram.:m.  e  adj.}
\end{itemize}
\begin{itemize}
\item {Proveniência:(Lat. \textunderscore obsessor\textunderscore )}
\end{itemize}
O que causa obsessão; o que importuna.
\section{Obsia}
\begin{itemize}
\item {Grp. gram.:f.}
\end{itemize}
\begin{itemize}
\item {Utilização:Ant.}
\end{itemize}
O mesmo que \textunderscore adussia\textunderscore .
\section{Obsidente}
\begin{itemize}
\item {Grp. gram.:m.  e  adj.}
\end{itemize}
\begin{itemize}
\item {Proveniência:(Lat. \textunderscore obsidens\textunderscore )}
\end{itemize}
O mesmo que \textunderscore obsessor\textunderscore .
O que cérca ou sitia.
\section{Obsidiana}
\begin{itemize}
\item {Grp. gram.:f.}
\end{itemize}
\begin{itemize}
\item {Utilização:Miner.}
\end{itemize}
\begin{itemize}
\item {Proveniência:(Do lat. \textunderscore obsidianus\textunderscore )}
\end{itemize}
Pedra escura ou verde-negra, com apparência de vidro, vulgar em alguns terrenos vulcânicos, especialmente no México e no Peru, onde dantes se usava como faca e como espelho.
\section{Obsidiante}
\begin{itemize}
\item {Grp. gram.:adj.}
\end{itemize}
Que obsidia.
\section{Obsidiar}
\begin{itemize}
\item {Grp. gram.:v. t.}
\end{itemize}
\begin{itemize}
\item {Utilização:Fig.}
\end{itemize}
\begin{itemize}
\item {Proveniência:(Lat. \textunderscore obsidiari\textunderscore )}
\end{itemize}
Fazer cêrco a.
Estar á volta de.
Espiar.
Observar os actos ou a vida de:«\textunderscore ...demónios súcubos e incubos que a obsidiavam.\textunderscore »Camillo, \textunderscore Cav. em Ruínas\textunderscore , 81.
\section{Obsidional}
\begin{itemize}
\item {Grp. gram.:adj.}
\end{itemize}
\begin{itemize}
\item {Proveniência:(Lat. \textunderscore obsidionalis\textunderscore )}
\end{itemize}
Relativo a assédio ou cêrco.
Relativo á arte de cercar ou defender uma praça.
\section{Obsignador}
\begin{itemize}
\item {Grp. gram.:m.}
\end{itemize}
\begin{itemize}
\item {Proveniência:(Lat. \textunderscore obsignator\textunderscore )}
\end{itemize}
Dava-se este nome, entre os Romanos, á testemunha que era chamada para assinar um testamento e pôr-lhe o seu sêllo.
\section{Obsoletar}
\begin{itemize}
\item {Grp. gram.:v. t.}
\end{itemize}
Tornar obsoleto.
\section{Obsoleto}
\begin{itemize}
\item {Grp. gram.:adj.}
\end{itemize}
\begin{itemize}
\item {Proveniência:(Lat. \textunderscore obsoletus\textunderscore )}
\end{itemize}
Que caiu em desuso; antiquado: \textunderscore vocábulos obsoletos\textunderscore .
\section{Obstáculo}
\begin{itemize}
\item {Grp. gram.:m.}
\end{itemize}
\begin{itemize}
\item {Proveniência:(Lat. \textunderscore obstaculum\textunderscore )}
\end{itemize}
Tudo que obsta a alguma coisa.
Impedimento.
Difficuldade.
Barreira: \textunderscore saltar obstáculos\textunderscore .
\section{Obstante}
\begin{itemize}
\item {Grp. gram.:adj.}
\end{itemize}
\begin{itemize}
\item {Grp. gram.:Loc. conj.}
\end{itemize}
\begin{itemize}
\item {Grp. gram.:Loc. prep.}
\end{itemize}
\begin{itemize}
\item {Proveniência:(Lat. \textunderscore obstans\textunderscore )}
\end{itemize}
Que obsta.
\textunderscore Não obstante\textunderscore , contudo, apesar disso.
Apesar de: \textunderscore não obstante os meus bons desejos...\textunderscore 
\section{Obstar}
\begin{itemize}
\item {Grp. gram.:v. i.}
\end{itemize}
\begin{itemize}
\item {Proveniência:(Lat. \textunderscore obstare\textunderscore )}
\end{itemize}
Oppor-se; causar impedimento ou embaraço.
\section{Obstétrica}
\begin{itemize}
\item {Grp. gram.:f.}
\end{itemize}
O mesmo que \textunderscore obstetrícia\textunderscore .
\section{Obstetrical}
\begin{itemize}
\item {Grp. gram.:adj.}
\end{itemize}
O mesmo que \textunderscore obstétrico\textunderscore .
\section{Obstetricano}
\begin{itemize}
\item {Grp. gram.:m.}
\end{itemize}
Gênero de reptís bratrácios, cujas espécies são indígenas da França.
\section{Obstetrícia}
\begin{itemize}
\item {Grp. gram.:f.}
\end{itemize}
\begin{itemize}
\item {Proveniência:(De \textunderscore obstetrício\textunderscore )}
\end{itemize}
Arte, que se occupa dos partos.
\section{Obstetrício}
\begin{itemize}
\item {Grp. gram.:adj.}
\end{itemize}
\begin{itemize}
\item {Proveniência:(Lat. \textunderscore obstetricius\textunderscore )}
\end{itemize}
Relativo aos partos.
\section{Obstétrico}
\begin{itemize}
\item {Grp. gram.:adj.}
\end{itemize}
O mesmo que \textunderscore obstetrício\textunderscore .
\section{Obstetriz}
\begin{itemize}
\item {Grp. gram.:f.}
\end{itemize}
\begin{itemize}
\item {Proveniência:(Lat. \textunderscore obstetrix\textunderscore )}
\end{itemize}
O mesmo que \textunderscore parteira\textunderscore .
\section{Obsticidade}
\begin{itemize}
\item {Grp. gram.:f.}
\end{itemize}
\begin{itemize}
\item {Utilização:Med.}
\end{itemize}
Inclinação da cabeça para um dos ombros, por effeito de rheumatismo ou de outra lesão dos músculos.--Não respondo pela exactidão do termo; mas, embora deformado, talvez se relacione com o lat. \textunderscore stupare\textunderscore , voltar a cabeça para trás.
\section{Obstinação}
\begin{itemize}
\item {Grp. gram.:f.}
\end{itemize}
\begin{itemize}
\item {Proveniência:(Lat. \textunderscore obstinatio\textunderscore )}
\end{itemize}
Acção de quem se obstina.
Firmeza.
Pertinacia.
Reluctância.
\section{Obstinadamente}
\begin{itemize}
\item {Grp. gram.:adv.}
\end{itemize}
De modo obstinado; com obstinação; inflexivelmente.
\section{Obstinado}
\begin{itemize}
\item {Grp. gram.:adj.}
\end{itemize}
Firme; teimoso; pertinaz.
Feito com pertinácia, com insistência.
Inflexível.
\section{Obstinar}
\begin{itemize}
\item {Grp. gram.:v. t.}
\end{itemize}
\begin{itemize}
\item {Proveniência:(Lat. \textunderscore obstinare\textunderscore )}
\end{itemize}
Tornar firme, constante, pertinaz; tornar teimoso.
\section{Obstipação}
\begin{itemize}
\item {Grp. gram.:f.}
\end{itemize}
\begin{itemize}
\item {Utilização:Med.}
\end{itemize}
\begin{itemize}
\item {Proveniência:(De \textunderscore obstipar\textunderscore )}
\end{itemize}
Estado mórbido, conhecido também por prisão de ventre, quando esse estado é habitual.--A obstipação é absoluta, quando há occlusão intestinal. Cf. Verg. Machado, \textunderscore Applic. Med. e Cirúrg. da Electr.\textunderscore , 211 e 212.
\section{Obstipante}
\begin{itemize}
\item {Grp. gram.:adj.}
\end{itemize}
Que obstipa.
\section{Obstipar}
\begin{itemize}
\item {Grp. gram.:v.}
\end{itemize}
\begin{itemize}
\item {Utilização:t. Med.}
\end{itemize}
\begin{itemize}
\item {Proveniência:(Do lat. \textunderscore ob\textunderscore  + \textunderscore stipare\textunderscore )}
\end{itemize}
Produzir obstipação em.
\section{Obstricto}
\begin{itemize}
\item {Grp. gram.:adj.}
\end{itemize}
\begin{itemize}
\item {Proveniência:(Lat. \textunderscore obstrictus\textunderscore )}
\end{itemize}
Constrangido; obrigado.
\section{Obstringir}
\begin{itemize}
\item {Grp. gram.:v. t.}
\end{itemize}
\begin{itemize}
\item {Proveniência:(Lat. \textunderscore obstringere\textunderscore )}
\end{itemize}
Ligar com fôrça, apertar muito.
Estancar.
\section{Obstrução}
\begin{itemize}
\item {Grp. gram.:f.}
\end{itemize}
\begin{itemize}
\item {Proveniência:(Lat. \textunderscore obstructio\textunderscore )}
\end{itemize}
Acto ou effeito de obstruir.
Embaraço nos vasos ou canaes de um corpo animado.
Obturação.
\section{Obstrucção}
\begin{itemize}
\item {Grp. gram.:f.}
\end{itemize}
\begin{itemize}
\item {Proveniência:(Lat. \textunderscore obstructio\textunderscore )}
\end{itemize}
Acto ou effeito de obstruir.
Embaraço nos vasos ou canaes de um corpo animado.
Obturação.
\section{Obstruccionismo}
\begin{itemize}
\item {Grp. gram.:m.}
\end{itemize}
\begin{itemize}
\item {Proveniência:(De \textunderscore obstrucção\textunderscore )}
\end{itemize}
Embaraço, que se oppõe propositadamente ao proseguimento de uma discussão ou de um negócio.
\section{Obstrucionismo}
\begin{itemize}
\item {Grp. gram.:m.}
\end{itemize}
\begin{itemize}
\item {Proveniência:(De \textunderscore obstrução\textunderscore )}
\end{itemize}
Embaraço, que se opõe propositadamente ao proseguimento de uma discussão ou de um negócio.
\section{Obstructivo}
\begin{itemize}
\item {Grp. gram.:adj.}
\end{itemize}
\begin{itemize}
\item {Proveniência:(Do lat. \textunderscore obstructus\textunderscore )}
\end{itemize}
Que obstrue ou que serve para obstruir.
\section{Obstructor}
\begin{itemize}
\item {Grp. gram.:m.  e  adj.}
\end{itemize}
\begin{itemize}
\item {Proveniência:(Do lat. \textunderscore obstructus\textunderscore )}
\end{itemize}
O que obstrue.
\section{Obstruído}
\begin{itemize}
\item {Grp. gram.:m.}
\end{itemize}
Aquelle que soffre obstrucção.
\section{Obstruir}
\begin{itemize}
\item {Grp. gram.:v. t.}
\end{itemize}
\begin{itemize}
\item {Proveniência:(Lat. \textunderscore obstruere\textunderscore )}
\end{itemize}
Fechar, entupir.
Impedir, embaraçar; causar embaraço a.
Impedir a circulação de.
\section{Obstrutivo}
\begin{itemize}
\item {Grp. gram.:adj.}
\end{itemize}
\begin{itemize}
\item {Proveniência:(Do lat. \textunderscore obstructus\textunderscore )}
\end{itemize}
Que obstrue ou que serve para obstruir.
\section{Obstrutor}
\begin{itemize}
\item {Grp. gram.:m.  e  adj.}
\end{itemize}
\begin{itemize}
\item {Proveniência:(Do lat. \textunderscore obstructus\textunderscore )}
\end{itemize}
O que obstrue.
\section{Obstupefacção}
\begin{itemize}
\item {Grp. gram.:f.}
\end{itemize}
\begin{itemize}
\item {Proveniência:(Do lat. \textunderscore obstupefactus\textunderscore )}
\end{itemize}
Estado de quem se acha obstupefacto.
\section{Obstupefacto}
\begin{itemize}
\item {Grp. gram.:adj.}
\end{itemize}
\begin{itemize}
\item {Proveniência:(Lat. \textunderscore obstupefactus\textunderscore )}
\end{itemize}
O mesmo que \textunderscore estupefacto\textunderscore .
\section{Obstúpido}
\begin{itemize}
\item {Grp. gram.:adj.}
\end{itemize}
\begin{itemize}
\item {Proveniência:(Lat. \textunderscore obstupidus\textunderscore )}
\end{itemize}
Pasmado, attónito.
\section{Obsutural}
\begin{itemize}
\item {Grp. gram.:adj.}
\end{itemize}
\begin{itemize}
\item {Utilização:Bot.}
\end{itemize}
\begin{itemize}
\item {Proveniência:(De \textunderscore ob...\textunderscore  + \textunderscore sutura\textunderscore )}
\end{itemize}
Applicado, mas não soldado, ás suturas das válvulas, nos vegetaes.
\section{Obtéctea}
\begin{itemize}
\item {Grp. gram.:f.}
\end{itemize}
\begin{itemize}
\item {Utilização:zool.}
\end{itemize}
\begin{itemize}
\item {Proveniência:(Do lat. \textunderscore ob\textunderscore  + \textunderscore tectus\textunderscore )}
\end{itemize}
Chrysállida, que deixa reconhecer por fóra as partes do insecto.
\section{Obtemperação}
\begin{itemize}
\item {Grp. gram.:f.}
\end{itemize}
\begin{itemize}
\item {Proveniência:(Lat. \textunderscore obtemperatio\textunderscore )}
\end{itemize}
Acto ou effeito de obtemperar.
\section{Obtemperar}
\begin{itemize}
\item {Grp. gram.:v. t.}
\end{itemize}
\begin{itemize}
\item {Grp. gram.:V. i.}
\end{itemize}
\begin{itemize}
\item {Proveniência:(Lat. \textunderscore obtemperare\textunderscore )}
\end{itemize}
Dizer modestamente em resposta; ponderar.
Obedecer, sujeitar-se.
Responder humildemente.
\section{Obtenção}
\begin{itemize}
\item {Grp. gram.:f.}
\end{itemize}
\begin{itemize}
\item {Proveniência:(Lat. \textunderscore obtentio\textunderscore )}
\end{itemize}
Acto ou effeito de obter.
\section{Obtentor}
\begin{itemize}
\item {Grp. gram.:m.  e  adj.}
\end{itemize}
\begin{itemize}
\item {Proveniência:(Do lat. \textunderscore obtentus\textunderscore )}
\end{itemize}
O que obtém.
\section{Obter}
\begin{itemize}
\item {Grp. gram.:v. t.}
\end{itemize}
\begin{itemize}
\item {Proveniência:(Lat. \textunderscore obtinere\textunderscore )}
\end{itemize}
Alcançar ou adquirir (o que se pede ou se deseja).
Achar.
Conseguir.
Impetrar.
\section{Obtestar}
\begin{itemize}
\item {Grp. gram.:v. t.}
\end{itemize}
\begin{itemize}
\item {Proveniência:(Lat. \textunderscore obtestari\textunderscore )}
\end{itemize}
Tomar por testemunha.
Protestar.
Supplicar; instar.
\section{Obtundente}
\begin{itemize}
\item {Grp. gram.:adj.}
\end{itemize}
\begin{itemize}
\item {Proveniência:(Lat. \textunderscore obtundens\textunderscore )}
\end{itemize}
Que obtunde.
\section{Obtundir}
\begin{itemize}
\item {Grp. gram.:v. t.}
\end{itemize}
\begin{itemize}
\item {Utilização:Ant.}
\end{itemize}
\begin{itemize}
\item {Proveniência:(Lat. \textunderscore obtundere\textunderscore )}
\end{itemize}
Contundir.
Tornar obtuso.
Abrandar (a acrimónia dos humores).
\section{Obturação}
\begin{itemize}
\item {Grp. gram.:f.}
\end{itemize}
\begin{itemize}
\item {Proveniência:(Lat. \textunderscore obturatio\textunderscore )}
\end{itemize}
Acto ou effeito de obturar.
\section{Obturador}
\begin{itemize}
\item {Grp. gram.:adj.}
\end{itemize}
\begin{itemize}
\item {Grp. gram.:M.}
\end{itemize}
\begin{itemize}
\item {Proveniência:(Lat. \textunderscore obturator\textunderscore )}
\end{itemize}
Que obtura; que serve para obturar.
Aquillo que serve para obturar.
Substância, que acompanha o póllen das orchídeas.
\section{Obturante}
\begin{itemize}
\item {Grp. gram.:m. ,  f.  e  adj.}
\end{itemize}
\begin{itemize}
\item {Proveniência:(Lat. \textunderscore obturans\textunderscore )}
\end{itemize}
Aquillo que obtura.
O que impede a excreção do suor.
\section{Obturar}
\begin{itemize}
\item {Grp. gram.:v. t.}
\end{itemize}
\begin{itemize}
\item {Proveniência:(Lat. \textunderscore obturare\textunderscore )}
\end{itemize}
Tapar.
Fechar, ajustando-se a.
Entupir.
Obstruír.
Interceptar a communicação ou o escoamento de.
Impedir a passagem da luz por.
\section{Obturbinado}
\begin{itemize}
\item {Grp. gram.:adj.}
\end{itemize}
\begin{itemize}
\item {Utilização:Bot.}
\end{itemize}
\begin{itemize}
\item {Proveniência:(De \textunderscore ob...\textunderscore  + \textunderscore turbina\textunderscore )}
\end{itemize}
Diz-se do invólucro de certos frutos, e diz-se de outros órgãos vegetaes, quando têm fórma de pião invertido.
\section{Obtusado}
\begin{itemize}
\item {Grp. gram.:adj.}
\end{itemize}
\begin{itemize}
\item {Utilização:Bot.}
\end{itemize}
\begin{itemize}
\item {Proveniência:(De \textunderscore obtuso\textunderscore )}
\end{itemize}
Diz-se da fôlha, que tem a extremidade arredondada.
\section{Obtusamente}
\begin{itemize}
\item {Grp. gram.:adv.}
\end{itemize}
De modo obtuso.
Rudemente; estupidamente.
\section{Obtusangulado}
\begin{itemize}
\item {Grp. gram.:adj.}
\end{itemize}
\begin{itemize}
\item {Proveniência:(De \textunderscore obtusângulo\textunderscore )}
\end{itemize}
Que tem ângulos obtusos.
\section{Obtusângulo}
\begin{itemize}
\item {Grp. gram.:adj.}
\end{itemize}
\begin{itemize}
\item {Proveniência:(De \textunderscore obtuso\textunderscore  + \textunderscore ângulo\textunderscore )}
\end{itemize}
Que tem ângulo obtuso.
\section{Obtusão}
\begin{itemize}
\item {Grp. gram.:f.}
\end{itemize}
\begin{itemize}
\item {Proveniência:(Lat. \textunderscore obtusio\textunderscore )}
\end{itemize}
Estado do que é obtuso; ausência de sensibilidade.
\section{Obtusidade}
\begin{itemize}
\item {Grp. gram.:f.}
\end{itemize}
Qualidade de obtuso.
\section{Obtusífido}
\begin{itemize}
\item {Grp. gram.:adj.}
\end{itemize}
\begin{itemize}
\item {Utilização:Bot.}
\end{itemize}
\begin{itemize}
\item {Proveniência:(Do lat. \textunderscore obtusus\textunderscore  + \textunderscore findere\textunderscore )}
\end{itemize}
Dividido em segmentos obtusos.
\section{Obtusifloro}
\begin{itemize}
\item {Grp. gram.:adj.}
\end{itemize}
\begin{itemize}
\item {Utilização:Bot.}
\end{itemize}
Que tem pétalas obtusas.
\section{Obtusifoliado}
\begin{itemize}
\item {Grp. gram.:adj.}
\end{itemize}
\begin{itemize}
\item {Proveniência:(Do lat. \textunderscore obtusus\textunderscore  + \textunderscore folium\textunderscore )}
\end{itemize}
Que tem fôlhas obtusadas.
\section{Obtusifólio}
\begin{itemize}
\item {Grp. gram.:adj.}
\end{itemize}
\begin{itemize}
\item {Utilização:Bot.}
\end{itemize}
\begin{itemize}
\item {Proveniência:(Do lat. \textunderscore obtusus\textunderscore  + \textunderscore folium\textunderscore )}
\end{itemize}
Que tem fôlhas obtusadas.
\section{Obtusilobulado}
\begin{itemize}
\item {Grp. gram.:adj.}
\end{itemize}
\begin{itemize}
\item {Utilização:Bot.}
\end{itemize}
\begin{itemize}
\item {Proveniência:(De \textunderscore obtuso\textunderscore  + \textunderscore lóbulo\textunderscore )}
\end{itemize}
Que tem as fôlhas partidas em lóbulos arredondados e obtusos.
\section{Obtusirostro}
\begin{itemize}
\item {fónica:rós}
\end{itemize}
\begin{itemize}
\item {Grp. gram.:adj.}
\end{itemize}
\begin{itemize}
\item {Utilização:Zool.}
\end{itemize}
\begin{itemize}
\item {Proveniência:(Do lat. \textunderscore obtusus\textunderscore  + \textunderscore rostrum\textunderscore )}
\end{itemize}
Diz-se das aves, que têm a cabeça obtusa e achatada para deante, ou que tém bico obtuso.
\section{Obtusirrostro}
\begin{itemize}
\item {Grp. gram.:adj.}
\end{itemize}
\begin{itemize}
\item {Utilização:Zool.}
\end{itemize}
\begin{itemize}
\item {Proveniência:(Do lat. \textunderscore obtusus\textunderscore  + \textunderscore rostrum\textunderscore )}
\end{itemize}
Diz-se das aves, que têm a cabeça obtusa e achatada para deante, ou que tém bico obtuso.
\section{Obtuso}
\begin{itemize}
\item {Grp. gram.:adj.}
\end{itemize}
\begin{itemize}
\item {Utilização:Geom.}
\end{itemize}
\begin{itemize}
\item {Utilização:Fig.}
\end{itemize}
\begin{itemize}
\item {Proveniência:(Lat. \textunderscore obtusus\textunderscore )}
\end{itemize}
Não agudo; rombo.
Díz-se do ângulo, que é mais aberto que o ângulo recto.
Que tem a extremidade arredondada.
Tôsco.
Estúpido.
\section{Obumbração}
\begin{itemize}
\item {Grp. gram.:f.}
\end{itemize}
\begin{itemize}
\item {Proveniência:(Lat. \textunderscore obumbratio\textunderscore )}
\end{itemize}
Acto ou effeito de obumbrar.
\section{Obumbrar}
\begin{itemize}
\item {Grp. gram.:v. t.}
\end{itemize}
\begin{itemize}
\item {Utilização:Fig.}
\end{itemize}
\begin{itemize}
\item {Proveniência:(Lat. \textunderscore obumbrare\textunderscore )}
\end{itemize}
Cobrir de sombras; anuvear; toldar.
Tornar escuro. Cf. \textunderscore Lusíadas\textunderscore , V, 37.
Disfarçar.
Obsecar.
\section{Obuz}
\begin{itemize}
\item {Grp. gram.:m.}
\end{itemize}
Pequena peça de artilharia, semelhante a um morteiro comprido.
(Cast. \textunderscore obuz\textunderscore , do fr. \textunderscore obus\textunderscore )
\section{Obuzeiro}
\begin{itemize}
\item {Grp. gram.:adj.}
\end{itemize}
\begin{itemize}
\item {Proveniência:(De \textunderscore obuz\textunderscore )}
\end{itemize}
Diz-se dos canhões, que podem atirar projécteis ocos.
Diz-se do navio armado com obuzes.
\section{Obvenção}
\begin{itemize}
\item {Grp. gram.:f.}
\end{itemize}
\begin{itemize}
\item {Proveniência:(Lat. \textunderscore obventio\textunderscore )}
\end{itemize}
Provento ou receita eventual.
Antigo imposto ecclesiástico.
\section{Obverso}
\begin{itemize}
\item {Grp. gram.:m.}
\end{itemize}
\begin{itemize}
\item {Proveniência:(Lat. \textunderscore obversus\textunderscore )}
\end{itemize}
O mesmo que \textunderscore anverso\textunderscore .
\section{Obviar}
\begin{itemize}
\item {Grp. gram.:v. t.}
\end{itemize}
\begin{itemize}
\item {Grp. gram.:V. i.}
\end{itemize}
\begin{itemize}
\item {Proveniência:(Lat. \textunderscore obviare\textunderscore )}
\end{itemize}
Remediar.
Atalhar.
Objectar.
Ir ao encontro de.
Obstar; tomar prevenção.
\section{Obviável}
\begin{itemize}
\item {Grp. gram.:adj.}
\end{itemize}
Que se póde obviar.
\section{Óbvio}
\begin{itemize}
\item {Grp. gram.:adj.}
\end{itemize}
\begin{itemize}
\item {Proveniência:(Lat. \textunderscore obvius\textunderscore )}
\end{itemize}
Que occorre, que está deante.
Patente.
Claro; evidente.
\section{Obvir}
\begin{itemize}
\item {Grp. gram.:v. i.}
\end{itemize}
\begin{itemize}
\item {Utilização:Jur.}
\end{itemize}
\begin{itemize}
\item {Proveniência:(Do lat. \textunderscore obvenire\textunderscore )}
\end{itemize}
Caber ao Estado por herança ou por outra fórma.
\section{Obvolvido}
\begin{itemize}
\item {Grp. gram.:adj.}
\end{itemize}
\begin{itemize}
\item {Utilização:Bot.}
\end{itemize}
\begin{itemize}
\item {Proveniência:(De \textunderscore ob...\textunderscore  + \textunderscore volvido\textunderscore )}
\end{itemize}
Que se enrola sôbre ou em tôrno de outro, (falando-se de órgãos vegetaes).
\section{Oc}
\begin{itemize}
\item {Proveniência:(Lat. \textunderscore hoc\textunderscore )}
\end{itemize}
Partícula que, no dialecto românico falado ao sul do Loire, significava \textunderscore sim\textunderscore .
\textunderscore Língua de oc\textunderscore , língua românica, que se falava entre o Loire e os Pyrenéus.
\section{Oca}
\begin{itemize}
\item {Grp. gram.:f.}
\end{itemize}
Espécie de jôgo, também chamado jôgo da glória.
(Cast. \textunderscore oca\textunderscore )
\section{Oca}
\begin{itemize}
\item {Grp. gram.:f.}
\end{itemize}
Planta oxalídea do Brasil.
\section{Oca}
\begin{itemize}
\item {Grp. gram.:f.}
\end{itemize}
\begin{itemize}
\item {Utilização:Pop.}
\end{itemize}
O mesmo que \textunderscore ocra\textunderscore .
\section{Oca}
\begin{itemize}
\item {Grp. gram.:f.}
\end{itemize}
\begin{itemize}
\item {Utilização:Bras}
\end{itemize}
Casa de indígenas.
\section{Oça}
\begin{itemize}
\item {Grp. gram.:f.}
\end{itemize}
O mesmo que \textunderscore osa\textunderscore .
\section{Ocá}
\begin{itemize}
\item {Grp. gram.:m.}
\end{itemize}
Nome santhomense da mafumeira.
\section{Ocá}
\begin{itemize}
\item {Grp. gram.:m.}
\end{itemize}
Medicamento chinês, espécie de gelatina, resultante da fervura de pelles de burro em água do rio Lei.
\section{Ocaniguinecorni}
\begin{itemize}
\item {Grp. gram.:m.}
\end{itemize}
Pássaro da África occidental, (\textunderscore drymoica angolensis\textunderscore ).
\section{Ocar}
\begin{itemize}
\item {Grp. gram.:v. t.}
\end{itemize}
Tornar oco.
\section{Ocarina}
\begin{itemize}
\item {Grp. gram.:f.}
\end{itemize}
\begin{itemize}
\item {Proveniência:(De \textunderscore ocar\textunderscore )}
\end{itemize}
Instrumento músico, feito de barro e que dá sons semelhantes aos da frauta.
\section{Ocarinista}
\begin{itemize}
\item {Grp. gram.:m.  e  f.}
\end{itemize}
\begin{itemize}
\item {Proveniência:(T. dialectal it.?)}
\end{itemize}
Pessôa, que toca ocarina.
Fabricante ou vendedor de ocarinas.
\section{Ocasião}
\begin{itemize}
\item {Grp. gram.:f.}
\end{itemize}
\begin{itemize}
\item {Proveniência:(Lat. \textunderscore occasio\textunderscore )}
\end{itemize}
Oportunidade casual.
Tempo próprio para se fazer alguma coisa.
Faculdade, motivo, lugar: \textunderscore não houve ocasião de nos vermos\textunderscore .
Conjunto de circunstâncias favoraveis para um acto ou fim.
Vagar; tempo disponivel.
\section{Ocasionado}
\begin{itemize}
\item {Grp. gram.:adj.}
\end{itemize}
\begin{itemize}
\item {Proveniência:(De \textunderscore ocasionar\textunderscore )}
\end{itemize}
Causado; determinado.
\section{Ocasionador}
\begin{itemize}
\item {Grp. gram.:m.  e  adj.}
\end{itemize}
O que ocasiona.
\section{Ocasional}
\begin{itemize}
\item {Grp. gram.:adj.}
\end{itemize}
\begin{itemize}
\item {Proveniência:(Do lat. \textunderscore occasio\textunderscore )}
\end{itemize}
Casual, fortuito; ocasionador.
\section{Ocasionalidade}
\begin{itemize}
\item {Grp. gram.:f.}
\end{itemize}
Qualidade do que é ocasional.
\section{Ocasionalismo}
\begin{itemize}
\item {Grp. gram.:m.}
\end{itemize}
\begin{itemize}
\item {Proveniência:(De \textunderscore ocasional\textunderscore )}
\end{itemize}
Sistema imaginado pelos sectários de Descartes, para explicar as relações da alma com o corpo.
\section{Ocasionalista}
\begin{itemize}
\item {Grp. gram.:m.  e  f.}
\end{itemize}
\begin{itemize}
\item {Proveniência:(De \textunderscore ocasional\textunderscore )}
\end{itemize}
Pessôa, que é partidária do ocasionalismo.
\section{Ocasionalmente}
\begin{itemize}
\item {Grp. gram.:adv.}
\end{itemize}
De modo ocasional.
Casualmente; fortuitamente; em virtude das circumstâncias.
\section{Ocasionar}
\begin{itemize}
\item {Grp. gram.:v. t.}
\end{itemize}
\begin{itemize}
\item {Proveniência:(Do lat. \textunderscore occasio\textunderscore )}
\end{itemize}
Dar ocasião a.
Motivar; proporcionar.
\section{Ocaso}
\begin{itemize}
\item {Grp. gram.:m.}
\end{itemize}
\begin{itemize}
\item {Utilização:Fig.}
\end{itemize}
\begin{itemize}
\item {Proveniência:(Lat. \textunderscore occasus\textunderscore )}
\end{itemize}
O desapparecer do Sol ou de qualquer astro no horizonte.
Hora do sol-pôsto.
Ocidente.
Ruína; extinção; fim.
\section{Occasião}
\begin{itemize}
\item {Grp. gram.:f.}
\end{itemize}
\begin{itemize}
\item {Proveniência:(Lat. \textunderscore occasio\textunderscore )}
\end{itemize}
Opportunidade casual.
Tempo próprio para se fazer alguma coisa.
Faculdade, motivo lugar: \textunderscore não houve occasião de nos vermos\textunderscore .
Conjunto de circunstâncias favoraveis para um acto ou fim.
Vagar; tempo disponivel.
\section{Occasionado}
\begin{itemize}
\item {Grp. gram.:adj.}
\end{itemize}
\begin{itemize}
\item {Proveniência:(De \textunderscore occasionar\textunderscore )}
\end{itemize}
Causado; determinado.
\section{Occasionador}
\begin{itemize}
\item {Grp. gram.:m.  e  adj.}
\end{itemize}
O que occasiona.
\section{Occasional}
\begin{itemize}
\item {Grp. gram.:adj.}
\end{itemize}
\begin{itemize}
\item {Proveniência:(Do lat. \textunderscore occasio\textunderscore )}
\end{itemize}
Casual, fortuito; occasionador.
\section{Occasionalidade}
\begin{itemize}
\item {Grp. gram.:f.}
\end{itemize}
Qualidade do que é occasional.
\section{Occasionalismo}
\begin{itemize}
\item {Grp. gram.:m.}
\end{itemize}
\begin{itemize}
\item {Proveniência:(De \textunderscore occasional\textunderscore )}
\end{itemize}
Systema imaginado pelos sectários de Descartes, para explicar as relações da alma com o corpo.
\section{Occasionalista}
\begin{itemize}
\item {Grp. gram.:m.  e  f.}
\end{itemize}
\begin{itemize}
\item {Proveniência:(De \textunderscore occasional\textunderscore )}
\end{itemize}
Pessôa, que é partidária do occasionalismo.
\section{Occasionalmente}
\begin{itemize}
\item {Grp. gram.:adv.}
\end{itemize}
De modo occasional.
Casualmente; fortuitamente; em virtude das circumstâncias.
\section{Occasionar}
\begin{itemize}
\item {Grp. gram.:v. t.}
\end{itemize}
\begin{itemize}
\item {Proveniência:(Do lat. \textunderscore occasio\textunderscore )}
\end{itemize}
Dar occasião a.
Motivar; proporcionar.
\section{Occaso}
\begin{itemize}
\item {Grp. gram.:m.}
\end{itemize}
\begin{itemize}
\item {Utilização:Fig.}
\end{itemize}
\begin{itemize}
\item {Proveniência:(Lat. \textunderscore occasus\textunderscore )}
\end{itemize}
O desapparecer do Sol ou de qualquer astro no horizonte.
Hora do sol-pôsto.
Occidente.
Ruína; extincção; fim.
\section{Occidental}
\begin{itemize}
\item {Grp. gram.:adj.}
\end{itemize}
\begin{itemize}
\item {Grp. gram.:M. pl.}
\end{itemize}
\begin{itemize}
\item {Proveniência:(Lat. \textunderscore occidentalis\textunderscore )}
\end{itemize}
Relativo ao Occidente.
Situado ao lado do Occidente:«\textunderscore a occidental praia lusitana.\textunderscore »\textunderscore Lusíadas\textunderscore , I, 2.
Que habita as regiões do Occidente: \textunderscore os povos occidentaes\textunderscore .
Que desapparece no horizonte depois do Sol, (falando-se de um astro).
Povos, que habitam o Occidente do antigo continente.
\section{Occidentalidade}
\begin{itemize}
\item {Grp. gram.:f.}
\end{itemize}
Qualidade daquillo que é occidental. Cf. Th. Braga, \textunderscore Hist. da Liter.\textunderscore , (passim).
\section{Occidentalismo}
\begin{itemize}
\item {Grp. gram.:m.}
\end{itemize}
\begin{itemize}
\item {Utilização:Neol.}
\end{itemize}
Conjunto dos conhecimentos, relativos ao Occidente da Europa.
\section{Occidentalista}
\begin{itemize}
\item {Grp. gram.:m.}
\end{itemize}
\begin{itemize}
\item {Utilização:Neol.}
\end{itemize}
Aquelle que se dedica especialmente ao estudo das línguas, literaturas e civilização do Occidente da Europa.
\section{Occidentalização}
\begin{itemize}
\item {Grp. gram.:f.}
\end{itemize}
Acto ou effeito de \textunderscore occidentalizar\textunderscore .
\section{Occidentalizar}
\begin{itemize}
\item {Grp. gram.:v. t.}
\end{itemize}
\begin{itemize}
\item {Utilização:Neol.}
\end{itemize}
Dar o carácter, feição ou usos do Occidente da Europa a: \textunderscore êste reformador queria occidentalizar o Japão\textunderscore .
\section{Occidente}
\begin{itemize}
\item {Grp. gram.:m.}
\end{itemize}
\begin{itemize}
\item {Proveniência:(Lat. \textunderscore occidens\textunderscore )}
\end{itemize}
Lado do horizonte, em que o Sol se põe.
Poente.
Parte do globo terrestre, correspondente a esse lado do horizonte.
Povos ou regiões, que demoram nessa parte do globo.
\section{Occídio}
\begin{itemize}
\item {Grp. gram.:m.}
\end{itemize}
\begin{itemize}
\item {Utilização:Poét.}
\end{itemize}
\begin{itemize}
\item {Proveniência:(Lat. \textunderscore occidium\textunderscore )}
\end{itemize}
O mesmo que \textunderscore assassínio\textunderscore . Cf. Pacheco, \textunderscore Promptuário\textunderscore .
\section{Occíduo}
\begin{itemize}
\item {Grp. gram.:adj.}
\end{itemize}
\begin{itemize}
\item {Utilização:Poét.}
\end{itemize}
\begin{itemize}
\item {Proveniência:(Lat. \textunderscore occiduus\textunderscore )}
\end{itemize}
O mesmo que \textunderscore occidental\textunderscore .
\section{Occipicial}
\begin{itemize}
\item {fónica:csi}
\end{itemize}
\begin{itemize}
\item {Grp. gram.:adj.}
\end{itemize}
\begin{itemize}
\item {Proveniência:(De \textunderscore occipício\textunderscore )}
\end{itemize}
O mesmo que \textunderscore occipital\textunderscore .
\section{Occipício}
\begin{itemize}
\item {fónica:csi}
\end{itemize}
\begin{itemize}
\item {Grp. gram.:m.}
\end{itemize}
\begin{itemize}
\item {Proveniência:(Lat. \textunderscore occipitium\textunderscore )}
\end{itemize}
Parte infero-posterior da cabeça.
\section{Occipital}
\begin{itemize}
\item {fónica:csi}
\end{itemize}
\begin{itemize}
\item {Grp. gram.:adj.}
\end{itemize}
\begin{itemize}
\item {Grp. gram.:M.}
\end{itemize}
\begin{itemize}
\item {Proveniência:(Lat. \textunderscore occipitalis\textunderscore )}
\end{itemize}
Relativo ao occipício.
Que constitue a parede infero-posterior do crânio, (falando-se de um osso).
O mesmo que \textunderscore occipício\textunderscore .
\section{Occípito-atloidiano}
\begin{itemize}
\item {Grp. gram.:adj.}
\end{itemize}
\begin{itemize}
\item {Utilização:Anat.}
\end{itemize}
Relativo ao osso occipital e á vértebra atlas.
\section{Occípito-auricular}
\begin{itemize}
\item {Grp. gram.:adj.}
\end{itemize}
Diz-se de um músculo dos mammiferos, que vai desde a cabeça até o pavilhão da orelha, em todos os animaes de orelhas compridas.
\section{Occípito-axoídeo}
\begin{itemize}
\item {Grp. gram.:adj.}
\end{itemize}
\begin{itemize}
\item {Utilização:Anat.}
\end{itemize}
Relativo ao osso occipital e á vértebra áxis.
\section{Occípito-frontal}
\begin{itemize}
\item {Grp. gram.:adj.}
\end{itemize}
Relativo ao occipício e á testa.
\section{Occípito-meníngeo}
\begin{itemize}
\item {Grp. gram.:adj.}
\end{itemize}
\begin{itemize}
\item {Utilização:Anat.}
\end{itemize}
Relativo ao osso occipital e á dura-máter.
\section{Occípito-parietal}
\begin{itemize}
\item {Grp. gram.:adj.}
\end{itemize}
Relativo aos ossos occipital e parietal.
\section{Occípito-pétreo}
\begin{itemize}
\item {Grp. gram.:adj.}
\end{itemize}
\begin{itemize}
\item {Utilização:Anat.}
\end{itemize}
Formado pelo osso occipital e pela apóphyse pétrea temporal.
\section{Occípito-vertebral}
\begin{itemize}
\item {Grp. gram.:adj.}
\end{itemize}
Relativo ao occipício e ás vértebras.
\section{Occipúcio}
\begin{itemize}
\item {Grp. gram.:m.}
\end{itemize}
O mesmo que \textunderscore occíput\textunderscore .
\section{Occíput}
\begin{itemize}
\item {fónica:ocsípud'}
\end{itemize}
\begin{itemize}
\item {Grp. gram.:m.}
\end{itemize}
\begin{itemize}
\item {Proveniência:(T. lat.)}
\end{itemize}
O mesmo que \textunderscore occipício\textunderscore .
\section{Occisão}
\begin{itemize}
\item {Grp. gram.:f.}
\end{itemize}
\begin{itemize}
\item {Utilização:Des.}
\end{itemize}
\begin{itemize}
\item {Proveniência:(Lat. \textunderscore accisio\textunderscore )}
\end{itemize}
Assassínio; acto de matar.
\section{Occisivo}
\begin{itemize}
\item {Grp. gram.:adj.}
\end{itemize}
\begin{itemize}
\item {Utilização:Des.}
\end{itemize}
\begin{itemize}
\item {Proveniência:(Do lat. \textunderscore occisus\textunderscore )}
\end{itemize}
Que mata.
\section{Occlusão}
\begin{itemize}
\item {Grp. gram.:f.}
\end{itemize}
\begin{itemize}
\item {Proveniência:(Do lat. \textunderscore occlusus\textunderscore )}
\end{itemize}
Acto de fechar.
Doença, em que se suspendem as evacuações fecaes.
Cerramento momentaneo de uma abertura natural.
\section{Occluso}
\begin{itemize}
\item {Grp. gram.:adj.}
\end{itemize}
\begin{itemize}
\item {Proveniência:(Lat. \textunderscore occlusus\textunderscore )}
\end{itemize}
Fechado; em que há occlusão.
\section{Occoembo}
\begin{itemize}
\item {Grp. gram.:m.}
\end{itemize}
Planta herbácea do Brasil.
\section{Occorrência}
\begin{itemize}
\item {Grp. gram.:f.}
\end{itemize}
\begin{itemize}
\item {Proveniência:(De \textunderscore occorrente\textunderscore )}
\end{itemize}
Acto de occorrer; successo; acaso; encontro.
\section{Occorrente}
\begin{itemize}
\item {Grp. gram.:adj.}
\end{itemize}
\begin{itemize}
\item {Proveniência:(Lat. \textunderscore occurrens\textunderscore )}
\end{itemize}
Que occorre; convergente.
\section{Occorrer}
\begin{itemize}
\item {Grp. gram.:v. i.}
\end{itemize}
\begin{itemize}
\item {Proveniência:(Lat. \textunderscore occurrere\textunderscore )}
\end{itemize}
Ir ou vir ao encontro.
Sobrevir; apparecer.
Vir á ideia, lembrar.
Acontecer.
Coincidir.
Obviar.
\section{Occultação}
\begin{itemize}
\item {Grp. gram.:f.}
\end{itemize}
\begin{itemize}
\item {Proveniência:(Lat. \textunderscore occultatio\textunderscore )}
\end{itemize}
Acto ou effeito de occultar.
\section{Occultador}
\begin{itemize}
\item {Grp. gram.:m.  e  adj.}
\end{itemize}
O que occulta.
\section{Occultamente}
\begin{itemize}
\item {Grp. gram.:adv.}
\end{itemize}
De modo occulto; ás escondidas; furtivamente.
\section{Occultante}
\begin{itemize}
\item {Grp. gram.:adj.}
\end{itemize}
\begin{itemize}
\item {Proveniência:(Lat. \textunderscore occultans\textunderscore )}
\end{itemize}
Que occulta.
\section{Occultar}
\begin{itemize}
\item {Grp. gram.:v. t.}
\end{itemize}
\begin{itemize}
\item {Proveniência:(Lat. \textunderscore occultare\textunderscore )}
\end{itemize}
Não deixar vêr.
Esconder.
Sonegar.
Dissimular.
\section{Occultas}
\begin{itemize}
\item {Grp. gram.:f. pl. Loc. adv.}
\end{itemize}
\textunderscore Ás occultas\textunderscore , ou \textunderscore a occultas\textunderscore , o mesmo que \textunderscore occultamente\textunderscore .
\section{Occultismo}
\begin{itemize}
\item {Grp. gram.:m.}
\end{itemize}
\begin{itemize}
\item {Proveniência:(De \textunderscore occulto\textunderscore )}
\end{itemize}
Conjunto das artes ou sciências occultas, como a magia, o espiritismo, etc.
\section{Occultista}
\begin{itemize}
\item {Grp. gram.:m.}
\end{itemize}
Aquelle que se dedica ao occultismo.
\section{Occulto}
\begin{itemize}
\item {Grp. gram.:adj.}
\end{itemize}
\begin{itemize}
\item {Proveniência:(Lat. \textunderscore occultus\textunderscore )}
\end{itemize}
Que só é conhecido pelos seus effeitos, e não por si próprio.
Escondido.
Desconhecido.
Mysterioso; sobrenatural.
Não percorrido nem explorado.
\section{Occupação}
\begin{itemize}
\item {Grp. gram.:f.}
\end{itemize}
\begin{itemize}
\item {Proveniência:(Lat. \textunderscore occupatio\textunderscore )}
\end{itemize}
Acto ou effeito de occupar.
Posse: \textunderscore occupação de um território\textunderscore .
Emprêgo; offício: \textunderscore a sua occupação é funileiro\textunderscore .
\section{Occupadamente}
\begin{itemize}
\item {Grp. gram.:adv.}
\end{itemize}
\begin{itemize}
\item {Proveniência:(De \textunderscore occupado\textunderscore )}
\end{itemize}
Afanosamente; com trabalho.
\section{Occupado}
\begin{itemize}
\item {Grp. gram.:adj.}
\end{itemize}
Que tem alguma coisa que fazer ou em que pensar.
Diz-se da mulher em estado de gravidez.
\section{Occupador}
\begin{itemize}
\item {Grp. gram.:adj.}
\end{itemize}
\begin{itemize}
\item {Proveniência:(Lat. \textunderscore occupator\textunderscore )}
\end{itemize}
Que occupa ou que occupou.
\section{Occupante}
\begin{itemize}
\item {Grp. gram.:adj.}
\end{itemize}
O mesmo que \textunderscore occupador\textunderscore .
\section{Occupar}
\begin{itemize}
\item {Grp. gram.:v. t.}
\end{itemize}
\begin{itemize}
\item {Grp. gram.:V. i.}
\end{itemize}
\begin{itemize}
\item {Grp. gram.:V. p.}
\end{itemize}
\begin{itemize}
\item {Proveniência:(Lat. \textunderscore occupare\textunderscore )}
\end{itemize}
Estar na posse de.
Tomar posse de: \textunderscore occupar um território\textunderscore .
Habitar: \textunderscore occupar uma casa\textunderscore .
Conquistar.
Tomar, encher: \textunderscore occupar espaço\textunderscore .
Sêr objecto de; fixar, attrahir: \textunderscore occupar a attenção do auditório\textunderscore .
Dar trabalho ou cuidado a: \textunderscore occupa-me muito o futuro dos filhos\textunderscore .
Tornar-se grávida, (a mulher).
Empregar-se.
Applicar a attenção.
Gastar o tempo em alguma coisa: \textunderscore occupa-se em dizer mal\textunderscore .
\section{Occursar}
\begin{itemize}
\item {Grp. gram.:v. i.}
\end{itemize}
\begin{itemize}
\item {Utilização:Des.}
\end{itemize}
\begin{itemize}
\item {Proveniência:(Lat. \textunderscore occursare\textunderscore )}
\end{itemize}
Occorrer; apresentar-se deante.
\section{Oceâneo}
\begin{itemize}
\item {Grp. gram.:adj.}
\end{itemize}
O mesmo que \textunderscore oceânico\textunderscore . Cf. Filinto, XIII, 74; XIV, 16.
\section{Oceânea}
\begin{itemize}
\item {Grp. gram.:f.}
\end{itemize}
Gênero de acalephos medusários, cujas espécies são microscópicas.
\section{Oceanicidade}
\begin{itemize}
\item {Grp. gram.:f.}
\end{itemize}
Qualidade de oceânico.
\section{Oceânico}
\begin{itemize}
\item {Grp. gram.:adj.}
\end{itemize}
Relativo ao Oceano ou á Oceânia.
Que vive no Oceano: \textunderscore a fauna oceânica\textunderscore .
\section{Oceânides}
\begin{itemize}
\item {Grp. gram.:f. pl.}
\end{itemize}
\begin{itemize}
\item {Proveniência:(Do lat. \textunderscore Oceanus\textunderscore )}
\end{itemize}
Nymphas do mar, filhas do Oceano, segundo a Mythologia.
\section{Oceano}
\begin{itemize}
\item {Grp. gram.:m.}
\end{itemize}
\begin{itemize}
\item {Utilização:Fig.}
\end{itemize}
\begin{itemize}
\item {Grp. gram.:Adj.}
\end{itemize}
\begin{itemize}
\item {Proveniência:(Lat. \textunderscore Oceanus\textunderscore )}
\end{itemize}
A extensão de água salgada que cérca a Terra.
Mar.
Cada uma das grandes divisões da parte liquida da superfície da Terra.
Grande extensão de água.
Immensidade: \textunderscore um Oceano de amarguras\textunderscore .
O mesmo que \textunderscore oceânico\textunderscore :«\textunderscore ...o mar oceano.\textunderscore »R. Lobo, \textunderscore Côrte na Ald.\textunderscore , I, 7.
\section{Oceanografia}
\begin{itemize}
\item {Grp. gram.:f.}
\end{itemize}
Descripção do Oceano e dos seus productos anímaes ou vegetaes.
(Cp. \textunderscore oceanógrafo\textunderscore )
\section{Oceonográfico}
\begin{itemize}
\item {Grp. gram.:adj.}
\end{itemize}
Relativo á oceanografia.
\section{Oceanografista}
\begin{itemize}
\item {Grp. gram.:m.}
\end{itemize}
Aquele que é perito em oceanografia.
\section{Oceanógrafo}
\begin{itemize}
\item {Grp. gram.:m.}
\end{itemize}
\begin{itemize}
\item {Proveniência:(Do gr. \textunderscore okeanos\textunderscore  + \textunderscore graphein\textunderscore )}
\end{itemize}
Aquele que se ocupa de oceanografia.
\section{Oceanographia}
\begin{itemize}
\item {Grp. gram.:f.}
\end{itemize}
Descripção do Oceano e dos seus productos anímaes ou vegetaes.
(Cp. \textunderscore oceanógrapho\textunderscore )
\section{Oceonográphico}
\begin{itemize}
\item {Grp. gram.:adj.}
\end{itemize}
Relativo á oceanographia.
\section{Oceanographista}
\begin{itemize}
\item {Grp. gram.:m.}
\end{itemize}
Aquelle que é perito em oceanographia.
\section{Oceanógrapho}
\begin{itemize}
\item {Grp. gram.:m.}
\end{itemize}
\begin{itemize}
\item {Proveniência:(Do gr. \textunderscore okeanos\textunderscore  + \textunderscore graphein\textunderscore )}
\end{itemize}
Aquelle que se occupa de oceanographia.
\section{Oceanologia}
\begin{itemize}
\item {Grp. gram.:f.}
\end{itemize}
Tratado das producções vegetaes e animaes do Oceano.
\section{Oceanológico}
\begin{itemize}
\item {Grp. gram.:adj.}
\end{itemize}
Relativo á oceanologia.
\section{Oceanologista}
\begin{itemize}
\item {Grp. gram.:m.}
\end{itemize}
O mesmo que \textunderscore oceanólogo\textunderscore .
\section{Oceanólogo}
\begin{itemize}
\item {Grp. gram.:m.}
\end{itemize}
\begin{itemize}
\item {Proveniência:(Do gr. \textunderscore okeanos\textunderscore  + \textunderscore logos\textunderscore )}
\end{itemize}
Aquelle que é perito em oceanologia.
\section{Ocelado}
\begin{itemize}
\item {Grp. gram.:adj.}
\end{itemize}
\begin{itemize}
\item {Proveniência:(Lat. \textunderscore ocellatus\textunderscore )}
\end{itemize}
Que tem ocelos; mosqueado.
\section{Ocelária}
\begin{itemize}
\item {Grp. gram.:f.}
\end{itemize}
\begin{itemize}
\item {Proveniência:(De \textunderscore ocelo\textunderscore )}
\end{itemize}
Gênero de polipeiros milepóreos, cujo tipo foi encontrado fóssil em alguns terrenos cretáceos.
\section{Océleo}
\begin{itemize}
\item {Grp. gram.:adj.}
\end{itemize}
\begin{itemize}
\item {Utilização:Zool.}
\end{itemize}
\begin{itemize}
\item {Proveniência:(De \textunderscore ocelo\textunderscore )}
\end{itemize}
Marcado com manchas, que dão o aspecto do pupila ocular.
\section{Ocellado}
\begin{itemize}
\item {Grp. gram.:adj.}
\end{itemize}
\begin{itemize}
\item {Proveniência:(Lat. \textunderscore ocellatus\textunderscore )}
\end{itemize}
Que tem ocellos; mosqueado.
\section{Ocellária}
\begin{itemize}
\item {Grp. gram.:f.}
\end{itemize}
\begin{itemize}
\item {Proveniência:(De \textunderscore ocello\textunderscore )}
\end{itemize}
Gênero de polypeiros millepóreos, cujo typo foi encontrado fóssil em alguns terrenos cretáceos.
\section{Ocelífero}
\begin{itemize}
\item {Grp. gram.:adj.}
\end{itemize}
\begin{itemize}
\item {Utilização:Bot.}
\end{itemize}
\begin{itemize}
\item {Proveniência:(Do lat. \textunderscore ocellus\textunderscore  + \textunderscore ferre\textunderscore )}
\end{itemize}
Que tem manchas em fórma de olhos.
\section{Océlleo}
\begin{itemize}
\item {Grp. gram.:adj.}
\end{itemize}
\begin{itemize}
\item {Utilização:Zool.}
\end{itemize}
\begin{itemize}
\item {Proveniência:(De \textunderscore ocello\textunderscore )}
\end{itemize}
Marcado com manchas, que dão o aspecto do pupilla ocular.
\section{Ocellífero}
\begin{itemize}
\item {Grp. gram.:adj.}
\end{itemize}
\begin{itemize}
\item {Utilização:Bot.}
\end{itemize}
\begin{itemize}
\item {Proveniência:(Do lat. \textunderscore ocellus\textunderscore  + \textunderscore ferre\textunderscore )}
\end{itemize}
Que tem manchas em fórma de olhos.
\section{Ocello}
\begin{itemize}
\item {Grp. gram.:m.}
\end{itemize}
\begin{itemize}
\item {Proveniência:(Lat. \textunderscore ocellus\textunderscore )}
\end{itemize}
Olhinho.
Cada um dos pontos arredondados e variegados, que matizam certos órgãos, como pennas, pelles, etc.
\section{Ocellote}
\begin{itemize}
\item {Grp. gram.:m.}
\end{itemize}
\begin{itemize}
\item {Proveniência:(De \textunderscore ocello\textunderscore )}
\end{itemize}
Mammífero carnívoro, de raça felina, em a América central.
\section{Ocelo}
\begin{itemize}
\item {Grp. gram.:m.}
\end{itemize}
\begin{itemize}
\item {Proveniência:(Lat. \textunderscore ocellus\textunderscore )}
\end{itemize}
Olhinho.
Cada um dos pontos arredondados e variegados, que matizam certos órgãos, como pennas, peles, etc.
\section{Ocelote}
\begin{itemize}
\item {Grp. gram.:m.}
\end{itemize}
\begin{itemize}
\item {Proveniência:(De \textunderscore ocelo\textunderscore )}
\end{itemize}
Mamífero carnívoro, de raça felina, em a América central.
\section{Ochas}
\begin{itemize}
\item {Grp. gram.:f. pl.}
\end{itemize}
\begin{itemize}
\item {Utilização:Des.}
\end{itemize}
Litígio; contenda; ralhos. Cf. \textunderscore Eufrozina\textunderscore , 209.
\section{Ochava}
\begin{itemize}
\item {Grp. gram.:f.}
\end{itemize}
\begin{itemize}
\item {Utilização:Ant.}
\end{itemize}
\begin{itemize}
\item {Proveniência:(Do cast. \textunderscore ocho\textunderscore )}
\end{itemize}
O mesmo que \textunderscore oitava\textunderscore . Cf. Herculano, \textunderscore Hist. de Port.\textunderscore , IV, 425, 426 e 433.
\section{Ochavilha}
\begin{itemize}
\item {Grp. gram.:f.}
\end{itemize}
\begin{itemize}
\item {Utilização:Ant.}
\end{itemize}
O mesmo que \textunderscore oitava\textunderscore .
\section{Oche!}
\begin{itemize}
\item {fónica:ô}
\end{itemize}
\begin{itemize}
\item {Grp. gram.:interj.}
\end{itemize}
\begin{itemize}
\item {Utilização:Prov.}
\end{itemize}
\begin{itemize}
\item {Utilização:trasm.}
\end{itemize}
Expressão usada para afagar os bois.
\section{Ochlocracia}
\begin{itemize}
\item {Grp. gram.:f.}
\end{itemize}
\begin{itemize}
\item {Proveniência:(Do gr. \textunderscore okhlos\textunderscore  + \textunderscore kratein\textunderscore )}
\end{itemize}
Govêrno, em que preponderam as classes inferiores ou a plebe.
\section{Ochlocrático}
\begin{itemize}
\item {Grp. gram.:adj.}
\end{itemize}
Relativo á ochlocracia.
\section{Ochna}
\begin{itemize}
\item {Grp. gram.:f.}
\end{itemize}
\begin{itemize}
\item {Proveniência:(Do gr. \textunderscore okne\textunderscore )}
\end{itemize}
Gênero de árvores e arbustos da Ásia e da África.
\section{Ochnáceas}
\begin{itemize}
\item {Grp. gram.:f. pl.}
\end{itemize}
Família de plantas, que tem por typo a ochna.
(Do \textunderscore ochnáceo\textunderscore )
\section{Ochnáceo}
\begin{itemize}
\item {Grp. gram.:adj.}
\end{itemize}
Relativo ou semelhante á ochna.
\section{Ochneas}
\begin{itemize}
\item {Grp. gram.:f.}
\end{itemize}
\begin{itemize}
\item {Proveniência:(De \textunderscore ochna\textunderscore )}
\end{itemize}
Tríbo de plantas ochnáceas.
\section{Ochra}
\begin{itemize}
\item {Grp. gram.:f.}
\end{itemize}
(V.ocra)
\section{Ochráceo}
\begin{itemize}
\item {Grp. gram.:adj.}
\end{itemize}
\begin{itemize}
\item {Proveniência:(Do gr. \textunderscore okhros\textunderscore )}
\end{itemize}
Que tem a côr do ocre.
\section{Ochrádeno}
\begin{itemize}
\item {Grp. gram.:m.}
\end{itemize}
\begin{itemize}
\item {Proveniência:(Do gr. \textunderscore okhros\textunderscore  + \textunderscore aden\textunderscore )}
\end{itemize}
Arbusto resedáceo do Egýpto.
\section{Ochrálea}
\begin{itemize}
\item {Grp. gram.:f.}
\end{itemize}
Gênero de insectos coleópteros.
\section{Ochrantháceas}
\begin{itemize}
\item {Grp. gram.:f. pl.}
\end{itemize}
\begin{itemize}
\item {Proveniência:(De \textunderscore ochrantháceo\textunderscore )}
\end{itemize}
Família de plantas, parecidas ás hypericíneas, das quaes se differençam por terem cinco estames e fólhas estipuladas e denteadas.
\section{Ochrantháceo}
\begin{itemize}
\item {Grp. gram.:adj.}
\end{itemize}
Relativo ou semelhante ao ochrantho.
\section{Ochrantho}
\begin{itemize}
\item {Grp. gram.:m.}
\end{itemize}
\begin{itemize}
\item {Proveniência:(Do gr. \textunderscore okhros\textunderscore  + \textunderscore anthos\textunderscore )}
\end{itemize}
Gênero de plantas, que serve de typo ás ochrantháceas.
\section{Óchrea}
\begin{itemize}
\item {Grp. gram.:f.}
\end{itemize}
\begin{itemize}
\item {Utilização:Bot.}
\end{itemize}
\begin{itemize}
\item {Proveniência:(Do gr. \textunderscore okhros\textunderscore , que contém)}
\end{itemize}
Baínha membranosa, situada na base dos pedúnculos de algumas cyperáceas.
Baínha membranosa e incompleta, que se acha na base das fôlhas de algumas polygóneas.
\section{Ochríase}
\begin{itemize}
\item {Grp. gram.:f.}
\end{itemize}
\begin{itemize}
\item {Proveniência:(Gr. \textunderscore okhriasis\textunderscore )}
\end{itemize}
Doença dos vegetaes, que os torna amarelos.
\section{Ochricórneo}
\begin{itemize}
\item {Grp. gram.:adj.}
\end{itemize}
\begin{itemize}
\item {Utilização:Zool.}
\end{itemize}
\begin{itemize}
\item {Proveniência:(Do gr. \textunderscore okhra\textunderscore  + lat. \textunderscore cornu\textunderscore )}
\end{itemize}
Que tem antennas pardacentas.
\section{Ochrocarpo}
\begin{itemize}
\item {Grp. gram.:m.}
\end{itemize}
\begin{itemize}
\item {Proveniência:(Do gr. \textunderscore okhros\textunderscore  + \textunderscore karpos\textunderscore )}
\end{itemize}
Gênero de plantas gutíferas, cujos frutos contém suco amarelo e abundante.
\section{Ochroíta}
\begin{itemize}
\item {Grp. gram.:f.}
\end{itemize}
\begin{itemize}
\item {Proveniência:(Do gr. \textunderscore okhros\textunderscore )}
\end{itemize}
Espécie de óxydo de ferro.
\section{Ochroíto}
\begin{itemize}
\item {Grp. gram.:m.}
\end{itemize}
O mesmo ou melhor que \textunderscore ochroíta\textunderscore .
\section{Ochrólito}
\begin{itemize}
\item {Grp. gram.:m.}
\end{itemize}
\begin{itemize}
\item {Utilização:Miner.}
\end{itemize}
\begin{itemize}
\item {Proveniência:(Do gr. \textunderscore okhros\textunderscore  + \textunderscore lithos\textunderscore )}
\end{itemize}
Chloroantimoniato de chumbo.
\section{Ochroma}
\begin{itemize}
\item {Grp. gram.:f.}
\end{itemize}
Gênero de plantas esterculiáceas.
\section{Ochrómya}
\begin{itemize}
\item {Grp. gram.:f.}
\end{itemize}
Gênero de insectos dípteros.
\section{Ochrópode}
\begin{itemize}
\item {Grp. gram.:adj.}
\end{itemize}
\begin{itemize}
\item {Utilização:Zool.}
\end{itemize}
\begin{itemize}
\item {Proveniência:(Do gr. \textunderscore okhros\textunderscore  + \textunderscore pous\textunderscore , \textunderscore podos\textunderscore )}
\end{itemize}
Que tem pés amarelos.
\section{Ochróptero}
\begin{itemize}
\item {Grp. gram.:adj.}
\end{itemize}
\begin{itemize}
\item {Utilização:Zool.}
\end{itemize}
\begin{itemize}
\item {Proveniência:(Do gr. \textunderscore okhros\textunderscore  + \textunderscore pteron\textunderscore )}
\end{itemize}
Que tem asas amarelas.
\section{Ochrópyra}
\begin{itemize}
\item {Grp. gram.:f.}
\end{itemize}
\begin{itemize}
\item {Utilização:Med.}
\end{itemize}
\begin{itemize}
\item {Proveniência:(Do gr. \textunderscore okhros\textunderscore  + \textunderscore pur\textunderscore )}
\end{itemize}
A febre amarela.
\section{Ochrosia}
\begin{itemize}
\item {Grp. gram.:f.}
\end{itemize}
\begin{itemize}
\item {Proveniência:(Do gr. \textunderscore okhros\textunderscore )}
\end{itemize}
Doença das plantas, que as torna amarelas.
\section{Ochrósia}
\begin{itemize}
\item {Grp. gram.:f.}
\end{itemize}
\begin{itemize}
\item {Proveniência:(Do gr. \textunderscore okhros\textunderscore )}
\end{itemize}
Gênero de plantas apocýneas, originárias da Nova-Caledónia.
\section{Ochtera}
\begin{itemize}
\item {Grp. gram.:f.}
\end{itemize}
Gênero de insectos dípteros.
\section{Ochthébio}
\begin{itemize}
\item {Grp. gram.:m.}
\end{itemize}
Gênero de insectos coleópteros heterómeros.
\section{Ochthênomo}
\begin{itemize}
\item {Grp. gram.:m.}
\end{itemize}
Gênero de insectos coleópteros heterómeros.
\section{Ochthíphila}
\begin{itemize}
\item {Grp. gram.:f.}
\end{itemize}
Gênero de insectos dípteros.
\section{Ochthódio}
\begin{itemize}
\item {Grp. gram.:m.}
\end{itemize}
Gênero de plantas crucíferas.
\section{Ochyzópode}
\begin{itemize}
\item {fónica:qui}
\end{itemize}
\begin{itemize}
\item {Grp. gram.:m.}
\end{itemize}
Gênero de crustáceos decápodes.
\section{Ócide}
\begin{itemize}
\item {Grp. gram.:f.}
\end{itemize}
Gênero de insectos, da fam. dos escaravelhos.
\section{Ocidental}
\begin{itemize}
\item {Grp. gram.:adj.}
\end{itemize}
\begin{itemize}
\item {Grp. gram.:M. pl.}
\end{itemize}
\begin{itemize}
\item {Proveniência:(Lat. \textunderscore occidentalis\textunderscore )}
\end{itemize}
Relativo ao Ocidente.
Situado ao lado do Ocidente:«\textunderscore a ocidental praia lusitana.\textunderscore »\textunderscore Lusíadas\textunderscore , I, 2.
Que habita as regiões do Ocidente: \textunderscore os povos ocidentaes\textunderscore .
Que desapparece no horizonte depois do Sol, (falando-se de um astro).
Povos, que habitam o Ocidente do antigo continente.
\section{Ocidentalidade}
\begin{itemize}
\item {Grp. gram.:f.}
\end{itemize}
Qualidade daquilo que é ocidental. Cf. Th. Braga, \textunderscore Hist. da Liter.\textunderscore , (passim).
\section{Ocidentalismo}
\begin{itemize}
\item {Grp. gram.:m.}
\end{itemize}
\begin{itemize}
\item {Utilização:Neol.}
\end{itemize}
Conjunto dos conhecimentos, relativos ao Ocidente da Europa.
\section{Ocidentalista}
\begin{itemize}
\item {Grp. gram.:m.}
\end{itemize}
\begin{itemize}
\item {Utilização:Neol.}
\end{itemize}
Aquele que se dedica especialmente ao estudo das línguas, literaturas e civilização do Ocidente da Europa.
\section{Ocidentalização}
\begin{itemize}
\item {Grp. gram.:f.}
\end{itemize}
Acto ou efeito de \textunderscore ocidentalizar\textunderscore .
\section{Ocidentalizar}
\begin{itemize}
\item {Grp. gram.:v. t.}
\end{itemize}
\begin{itemize}
\item {Utilização:Neol.}
\end{itemize}
Dar o carácter, feição ou usos do Ocidente da Europa a: \textunderscore êste reformador queria ocidentalizar o Japão\textunderscore .
\section{Ocidente}
\begin{itemize}
\item {Grp. gram.:m.}
\end{itemize}
\begin{itemize}
\item {Proveniência:(Lat. \textunderscore occidens\textunderscore )}
\end{itemize}
Lado do horizonte, em que o Sol se põe.
Poente.
Parte do globo terrestre, correspondente a esse lado do horizonte.
Povos ou regiões, que demoram nessa parte do globo.
\section{Ocídio}
\begin{itemize}
\item {Grp. gram.:m.}
\end{itemize}
\begin{itemize}
\item {Utilização:Poét.}
\end{itemize}
\begin{itemize}
\item {Proveniência:(Lat. \textunderscore occidium\textunderscore )}
\end{itemize}
O mesmo que \textunderscore assassínio\textunderscore . Cf. Pacheco, \textunderscore Promptuário\textunderscore .
\section{Ocíduo}
\begin{itemize}
\item {Grp. gram.:adj.}
\end{itemize}
\begin{itemize}
\item {Utilização:Poét.}
\end{itemize}
\begin{itemize}
\item {Proveniência:(Lat. \textunderscore occiduus\textunderscore )}
\end{itemize}
O mesmo que \textunderscore ocidental\textunderscore .
\section{Ócimo}
\begin{itemize}
\item {Grp. gram.:m.}
\end{itemize}
\begin{itemize}
\item {Proveniência:(Lat. \textunderscore ocimum\textunderscore )}
\end{itemize}
O mesmo que \textunderscore mangericão\textunderscore .
\section{Ocimoídeas}
\begin{itemize}
\item {Grp. gram.:f. pl.}
\end{itemize}
\begin{itemize}
\item {Proveniência:(Do gr. \textunderscore okimon\textunderscore  + \textunderscore eidos\textunderscore )}
\end{itemize}
Tríbo de plantas labiadas, cujo principal gênero é o mangericão.
\section{Ócio}
\begin{itemize}
\item {Grp. gram.:m.}
\end{itemize}
\begin{itemize}
\item {Utilização:Fig.}
\end{itemize}
\begin{itemize}
\item {Proveniência:(Lat. \textunderscore otium\textunderscore )}
\end{itemize}
Cessação de trabalho.
Vagar, lazer descanso; estado de quem não faz nada.
Preguiça.
Trabalho suave e agradável.
\section{Ociosamente}
\begin{itemize}
\item {Grp. gram.:adv.}
\end{itemize}
De modo ocioso.
\section{Ociosidade}
\begin{itemize}
\item {Grp. gram.:f.}
\end{itemize}
\begin{itemize}
\item {Proveniência:(Lat. \textunderscore otiositas\textunderscore )}
\end{itemize}
Qualidade ou estado de ocioso.
\section{Ocioso}
\begin{itemize}
\item {Grp. gram.:adj.}
\end{itemize}
\begin{itemize}
\item {Grp. gram.:M.}
\end{itemize}
\begin{itemize}
\item {Proveniência:(Lat. \textunderscore otiosus\textunderscore )}
\end{itemize}
Que não trabalha, que não faz nada.
Que não tem profissão ou modo de vida.
Preguiçoso.
Inútil; estéril.
Aquelle que é ocioso.
\section{Ocipúcio}
\begin{itemize}
\item {Grp. gram.:m.}
\end{itemize}
O mesmo que \textunderscore occíput\textunderscore .
\section{Ocisão}
\begin{itemize}
\item {Grp. gram.:f.}
\end{itemize}
\begin{itemize}
\item {Utilização:Des.}
\end{itemize}
\begin{itemize}
\item {Proveniência:(Lat. \textunderscore accisio\textunderscore )}
\end{itemize}
Assassínio; acto de matar.
\section{Ocisivo}
\begin{itemize}
\item {Grp. gram.:adj.}
\end{itemize}
\begin{itemize}
\item {Utilização:Des.}
\end{itemize}
\begin{itemize}
\item {Proveniência:(Do lat. \textunderscore occisus\textunderscore )}
\end{itemize}
Que mata.
\section{Oclocracia}
\begin{itemize}
\item {Grp. gram.:f.}
\end{itemize}
\begin{itemize}
\item {Proveniência:(Do gr. \textunderscore okhlos\textunderscore  + \textunderscore kratein\textunderscore )}
\end{itemize}
Govêrno, em que preponderam as classes inferiores ou a plebe.
\section{Oclocrático}
\begin{itemize}
\item {Grp. gram.:adj.}
\end{itemize}
Relativo á oclocracia.
\section{Oclusão}
\begin{itemize}
\item {Grp. gram.:f.}
\end{itemize}
\begin{itemize}
\item {Proveniência:(Do lat. \textunderscore occlusus\textunderscore )}
\end{itemize}
Acto de fechar.
Doença, em que se suspendem as evacuações fecaes.
Cerramento momentaneo de uma abertura natural.
\section{Ocluso}
\begin{itemize}
\item {Grp. gram.:adj.}
\end{itemize}
\begin{itemize}
\item {Proveniência:(Lat. \textunderscore occlusus\textunderscore )}
\end{itemize}
Fechado; em que há oclusão.
\section{Ocna}
\begin{itemize}
\item {Grp. gram.:f.}
\end{itemize}
\begin{itemize}
\item {Proveniência:(Do gr. \textunderscore okne\textunderscore )}
\end{itemize}
Gênero de árvores e arbustos da Ásia e da África.
\section{Ocnáceas}
\begin{itemize}
\item {Grp. gram.:f. pl.}
\end{itemize}
Família de plantas, que tem por tipo a ocna.
(Do \textunderscore ocnáceo\textunderscore )
\section{Ocnáceo}
\begin{itemize}
\item {Grp. gram.:adj.}
\end{itemize}
Relativo ou semelhante á ocna.
\section{Ócneas}
\begin{itemize}
\item {Grp. gram.:f.}
\end{itemize}
\begin{itemize}
\item {Proveniência:(De \textunderscore ocna\textunderscore )}
\end{itemize}
Tríbo de plantas ocnáceas.
\section{Oco}
\begin{itemize}
\item {fónica:ô}
\end{itemize}
\begin{itemize}
\item {Grp. gram.:adj.}
\end{itemize}
\begin{itemize}
\item {Utilização:Fig.}
\end{itemize}
Que não tem miolo.
Vão.
Escavacado.
Insignificante; fútil: \textunderscore palavras ôcas\textunderscore .
Que não tem juízo.
\section{Ocoembo}
\begin{itemize}
\item {Grp. gram.:m.}
\end{itemize}
Planta herbácea do Brasil.
\section{Ocorrência}
\begin{itemize}
\item {Grp. gram.:f.}
\end{itemize}
\begin{itemize}
\item {Proveniência:(De \textunderscore ocorrente\textunderscore )}
\end{itemize}
Acto de ocorrer; successo; acaso; encontro.
\section{Ocorrente}
\begin{itemize}
\item {Grp. gram.:adj.}
\end{itemize}
\begin{itemize}
\item {Proveniência:(Lat. \textunderscore occurrens\textunderscore )}
\end{itemize}
Que ocorre; convergente.
\section{Ocorrer}
\begin{itemize}
\item {Grp. gram.:v. i.}
\end{itemize}
\begin{itemize}
\item {Proveniência:(Lat. \textunderscore occurrere\textunderscore )}
\end{itemize}
Ir ou vir ao encontro.
Sobrevir; apparecer.
Vir á ideia, lembrar.
Acontecer.
Coincidir.
Obviar.
\section{Ocotéa}
\begin{itemize}
\item {Grp. gram.:f.}
\end{itemize}
Planta laurácea do Brasil.
\section{Ocoteia}
\begin{itemize}
\item {Grp. gram.:f.}
\end{itemize}
Planta laurácea do Brasil.
\section{Ocozoal}
\begin{itemize}
\item {Grp. gram.:m.}
\end{itemize}
Serpente mexicana, de ventre claro, tirante a vermelho.
\section{Ocozol}
\begin{itemize}
\item {Grp. gram.:m.}
\end{itemize}
Planta, do gênero liquidâmbar.
\section{Ocráceo}
\begin{itemize}
\item {Grp. gram.:adj.}
\end{itemize}
\begin{itemize}
\item {Proveniência:(Do gr. \textunderscore okhros\textunderscore )}
\end{itemize}
Que tem a côr do ocre.
\section{Ocrádeno}
\begin{itemize}
\item {Grp. gram.:m.}
\end{itemize}
\begin{itemize}
\item {Proveniência:(Do gr. \textunderscore okhros\textunderscore  + \textunderscore aden\textunderscore )}
\end{itemize}
Arbusto resedáceo do Egípto.
\section{Ocrálea}
\begin{itemize}
\item {Grp. gram.:f.}
\end{itemize}
Gênero de insectos coleópteros.
\section{Ocrantáceas}
\begin{itemize}
\item {Grp. gram.:f. pl.}
\end{itemize}
\begin{itemize}
\item {Proveniência:(De \textunderscore ocrantáceo\textunderscore )}
\end{itemize}
Família de plantas, parecidas ás hipericíneas, das quaes se diferençam por terem cinco estames e fólhas estipuladas e denteadas.
\section{Ocrantáceo}
\begin{itemize}
\item {Grp. gram.:adj.}
\end{itemize}
Relativo ou semelhante ao ocranto.
\section{Ocranto}
\begin{itemize}
\item {Grp. gram.:m.}
\end{itemize}
\begin{itemize}
\item {Proveniência:(Do gr. \textunderscore okhros\textunderscore  + \textunderscore anthos\textunderscore )}
\end{itemize}
Gênero de plantas, que serve de tipo ás ocrantáceas.
\section{Ócrea}
\begin{itemize}
\item {Grp. gram.:f.}
\end{itemize}
\begin{itemize}
\item {Utilização:Bot.}
\end{itemize}
\begin{itemize}
\item {Proveniência:(Do gr. \textunderscore okhros\textunderscore , que contém)}
\end{itemize}
Baínha membranosa, situada na base dos pedúnculos de algumas ciperáceas.
Baínha membranosa e incompleta, que se acha na base das fôlhas de algumas poligóneas.
\section{Ocríase}
\begin{itemize}
\item {Grp. gram.:f.}
\end{itemize}
\begin{itemize}
\item {Proveniência:(Gr. \textunderscore okhriasis\textunderscore )}
\end{itemize}
Doença dos vegetaes, que os torna amarelos.
\section{Ocricórneo}
\begin{itemize}
\item {Grp. gram.:adj.}
\end{itemize}
\begin{itemize}
\item {Utilização:Zool.}
\end{itemize}
\begin{itemize}
\item {Proveniência:(Do gr. \textunderscore okhra\textunderscore  + lat. \textunderscore cornu\textunderscore )}
\end{itemize}
Que tem antenas pardacentas.
\section{Ocrocarpo}
\begin{itemize}
\item {Grp. gram.:m.}
\end{itemize}
\begin{itemize}
\item {Proveniência:(Do gr. \textunderscore okhros\textunderscore  + \textunderscore karpos\textunderscore )}
\end{itemize}
Gênero de plantas gutíferas, cujos frutos contém suco amarelo e abundante.
\section{Ocrocéfalo}
\begin{itemize}
\item {Grp. gram.:adj.}
\end{itemize}
\begin{itemize}
\item {Utilização:Zool.}
\end{itemize}
\begin{itemize}
\item {Proveniência:(Do gr. \textunderscore okhros\textunderscore  + \textunderscore kephale\textunderscore )}
\end{itemize}
Que tem cabeça amarela.
\section{Ocrocéphalo}
\begin{itemize}
\item {Grp. gram.:adj.}
\end{itemize}
\begin{itemize}
\item {Utilização:Zool.}
\end{itemize}
\begin{itemize}
\item {Proveniência:(Do gr. \textunderscore okhros\textunderscore  + \textunderscore kephale\textunderscore )}
\end{itemize}
Que tem cabeça amarela.
\section{Ocroíta}
\begin{itemize}
\item {Grp. gram.:f.}
\end{itemize}
\begin{itemize}
\item {Proveniência:(Do gr. \textunderscore okhros\textunderscore )}
\end{itemize}
Espécie de óxido de ferro.
\section{Ocroíto}
\begin{itemize}
\item {Grp. gram.:m.}
\end{itemize}
O mesmo ou melhor que \textunderscore ocroíta\textunderscore .
\section{Ocrólito}
\begin{itemize}
\item {Grp. gram.:m.}
\end{itemize}
\begin{itemize}
\item {Utilização:Miner.}
\end{itemize}
\begin{itemize}
\item {Proveniência:(Do gr. \textunderscore okhros\textunderscore  + \textunderscore lithos\textunderscore )}
\end{itemize}
Cloroantimoniato de chumbo.
\section{Ocroma}
\begin{itemize}
\item {Grp. gram.:f.}
\end{itemize}
Gênero de plantas esterculiáceas.
\section{Ocrómia}
\begin{itemize}
\item {Grp. gram.:f.}
\end{itemize}
Gênero de insectos dípteros.
\section{Ocrópode}
\begin{itemize}
\item {Grp. gram.:adj.}
\end{itemize}
\begin{itemize}
\item {Utilização:Zool.}
\end{itemize}
\begin{itemize}
\item {Proveniência:(Do gr. \textunderscore okhros\textunderscore  + \textunderscore pous\textunderscore , \textunderscore podos\textunderscore )}
\end{itemize}
Que tem pés amarelos.
\section{Ocróptero}
\begin{itemize}
\item {Grp. gram.:adj.}
\end{itemize}
\begin{itemize}
\item {Utilização:Zool.}
\end{itemize}
\begin{itemize}
\item {Proveniência:(Do gr. \textunderscore okhros\textunderscore  + \textunderscore pteron\textunderscore )}
\end{itemize}
Que tem asas amarelas.
\section{Ocrópira}
\begin{itemize}
\item {Grp. gram.:f.}
\end{itemize}
\begin{itemize}
\item {Utilização:Med.}
\end{itemize}
\begin{itemize}
\item {Proveniência:(Do gr. \textunderscore okhros\textunderscore  + \textunderscore pur\textunderscore )}
\end{itemize}
A febre amarela.
\section{Ocrosia}
\begin{itemize}
\item {Grp. gram.:f.}
\end{itemize}
\begin{itemize}
\item {Proveniência:(Do gr. \textunderscore okhros\textunderscore )}
\end{itemize}
Doença das plantas, que as torna amarelas.
\section{Ocrósia}
\begin{itemize}
\item {Grp. gram.:f.}
\end{itemize}
\begin{itemize}
\item {Proveniência:(Do gr. \textunderscore okhros\textunderscore )}
\end{itemize}
Gênero de plantas apocýneas, originárias da Nova-Caledónia.
\section{Octera}
\begin{itemize}
\item {Grp. gram.:f.}
\end{itemize}
Gênero de insectos dípteros.
\section{Octébio}
\begin{itemize}
\item {Grp. gram.:m.}
\end{itemize}
Gênero de insectos coleópteros heterómeros.
\section{Octênomo}
\begin{itemize}
\item {Grp. gram.:m.}
\end{itemize}
Gênero de insectos coleópteros heterómeros.
\section{Octífila}
\begin{itemize}
\item {Grp. gram.:f.}
\end{itemize}
Gênero de insectos dípteros.
\section{Octódio}
\begin{itemize}
\item {Grp. gram.:m.}
\end{itemize}
Gênero de plantas crucíferas.
\section{Ocra}
\begin{itemize}
\item {Grp. gram.:f.}
\end{itemize}
O mesmo ou melhor que \textunderscore ocre\textunderscore .
\section{Ocre}
\begin{itemize}
\item {Grp. gram.:m.}
\end{itemize}
\begin{itemize}
\item {Proveniência:(Lat. \textunderscore ocra\textunderscore )}
\end{itemize}
Argilla colorida por um óxydo de ferro.
\section{Ocrea}
\begin{itemize}
\item {Grp. gram.:f.}
\end{itemize}
\begin{itemize}
\item {Proveniência:(Lat. \textunderscore ocrea\textunderscore )}
\end{itemize}
Vagem na base do pecíolo de algumas plantas de fôlhas alternas.
\section{Ocreoso}
\begin{itemize}
\item {Grp. gram.:adj.}
\end{itemize}
Relativo á ocra; que é da natureza ou côr della.
\section{Octa...}
O mesmo que \textunderscore octo...\textunderscore 
\section{Octã}
\begin{itemize}
\item {Grp. gram.:f.  e  adj.}
\end{itemize}
\begin{itemize}
\item {Proveniência:(Do lat. \textunderscore octo\textunderscore )}
\end{itemize}
Que se repete de oito em oito dias, (falando-se da febre).
\section{Octacórdio}
\begin{itemize}
\item {Grp. gram.:adj.}
\end{itemize}
\begin{itemize}
\item {Utilização:Mús.}
\end{itemize}
O mesmo que \textunderscore octacordo\textunderscore .
\section{Octacordo}
\begin{itemize}
\item {Grp. gram.:adj.}
\end{itemize}
\begin{itemize}
\item {Grp. gram.:M.}
\end{itemize}
\begin{itemize}
\item {Utilização:Mús.}
\end{itemize}
\begin{itemize}
\item {Proveniência:(Lat. \textunderscore octachordos\textunderscore )}
\end{itemize}
Que tem oito cordas.
Série ou escala de oito cordas ou notas, formada por dois tetracordos dijuntos; escala diatónica.
\section{Octaédrico}
\begin{itemize}
\item {Grp. gram.:adj.}
\end{itemize}
Relativo a octaédro.
O mesmo que \textunderscore octaedriforme\textunderscore .
\section{Octaedriforme}
\begin{itemize}
\item {Grp. gram.:adj.}
\end{itemize}
\begin{itemize}
\item {Proveniência:(De \textunderscore octaédro\textunderscore  + \textunderscore fórma\textunderscore )}
\end{itemize}
Que tem fórma de octaédro.
\section{Octaedrito}
\begin{itemize}
\item {Grp. gram.:m.}
\end{itemize}
\begin{itemize}
\item {Utilização:Miner.}
\end{itemize}
\begin{itemize}
\item {Proveniência:(De \textunderscore octaédro\textunderscore )}
\end{itemize}
Óxido de titânio, cuja fórmula é Ti O^2.
\section{Octaédro}
\begin{itemize}
\item {Grp. gram.:m.}
\end{itemize}
\begin{itemize}
\item {Utilização:Geom.}
\end{itemize}
\begin{itemize}
\item {Proveniência:(Lat. \textunderscore octaedros\textunderscore )}
\end{itemize}
Sólido de oito faces.
\section{Octaetéride}
\begin{itemize}
\item {Grp. gram.:f.}
\end{itemize}
\begin{itemize}
\item {Proveniência:(Lat. \textunderscore octaeteris\textunderscore )}
\end{itemize}
Período de oito annos.
\section{Octaminas}
\begin{itemize}
\item {Grp. gram.:f. pl.}
\end{itemize}
\begin{itemize}
\item {Utilização:Chím.}
\end{itemize}
\begin{itemize}
\item {Proveniência:(De \textunderscore octo...\textunderscore  + \textunderscore aminas\textunderscore )}
\end{itemize}
Aminas, formadas por oito moléculas de ammoníaco.
\section{Octan}
\begin{itemize}
\item {Grp. gram.:f.  e  adj.}
\end{itemize}
\begin{itemize}
\item {Proveniência:(Do lat. \textunderscore octo\textunderscore )}
\end{itemize}
Que se repete de oito em oito dias, (falando-se da febre).
\section{Octana}
\begin{itemize}
\item {Grp. gram.:f.  e  adj.}
\end{itemize}
O mesmo que \textunderscore octan\textunderscore .
\section{Octandria}
\begin{itemize}
\item {Grp. gram.:f.}
\end{itemize}
Qualidade de octandro.
Classe dos vegetaes que têm oito estames.
\section{Octândrico}
\begin{itemize}
\item {Grp. gram.:adj.}
\end{itemize}
O mesmo que \textunderscore octandro\textunderscore .
\section{Octandro}
\begin{itemize}
\item {Grp. gram.:adj.}
\end{itemize}
\begin{itemize}
\item {Utilização:Bot.}
\end{itemize}
\begin{itemize}
\item {Proveniência:(Do gr. \textunderscore oktos\textunderscore  + \textunderscore aner\textunderscore , \textunderscore andros\textunderscore )}
\end{itemize}
Que tem oito estames, livres entre si.
\section{Octangular}
\begin{itemize}
\item {Grp. gram.:adj.}
\end{itemize}
O mesmo que \textunderscore octogonal\textunderscore .
\section{Octano}
\begin{itemize}
\item {Grp. gram.:m.}
\end{itemize}
\begin{itemize}
\item {Utilização:Chím.}
\end{itemize}
Um dos carbonetos do grupo formênico.
\section{Octante}
\begin{itemize}
\item {Grp. gram.:adj.}
\end{itemize}
O mesmo que \textunderscore oitante\textunderscore .
\section{Octantéreo}
\begin{itemize}
\item {Grp. gram.:adj.}
\end{itemize}
O mesmo que \textunderscore octantero\textunderscore .
\section{Octantero}
\begin{itemize}
\item {Grp. gram.:adj.}
\end{itemize}
\begin{itemize}
\item {Proveniência:(De \textunderscore octo...\textunderscore  + \textunderscore antera\textunderscore )}
\end{itemize}
Que tem oito anteras.
\section{Octanthéreo}
\begin{itemize}
\item {Grp. gram.:adj.}
\end{itemize}
O mesmo que \textunderscore octanthero\textunderscore .
\section{Octanthero}
\begin{itemize}
\item {Grp. gram.:adj.}
\end{itemize}
\begin{itemize}
\item {Proveniência:(De \textunderscore octo...\textunderscore  + \textunderscore anthera\textunderscore )}
\end{itemize}
Que tem oito antheras.
\section{Octarillo}
\begin{itemize}
\item {Grp. gram.:m.}
\end{itemize}
Gênero de plantas santaláceas.
\section{Octarilo}
\begin{itemize}
\item {Grp. gram.:m.}
\end{itemize}
Gênero de plantas santaláceas.
\section{Octateucho}
\begin{itemize}
\item {fónica:co}
\end{itemize}
\begin{itemize}
\item {Grp. gram.:m.}
\end{itemize}
\begin{itemize}
\item {Proveniência:(Do gr. \textunderscore okto\textunderscore  + \textunderscore teukhos\textunderscore )}
\end{itemize}
Os oito primeiros livros do \textunderscore Antigo Testamento\textunderscore .
\section{Octateuco}
\begin{itemize}
\item {Grp. gram.:m.}
\end{itemize}
\begin{itemize}
\item {Proveniência:(Do gr. \textunderscore okto\textunderscore  + \textunderscore teukhos\textunderscore )}
\end{itemize}
Os oito primeiros livros do \textunderscore Antigo Testamento\textunderscore .
\section{Octávia}
\begin{itemize}
\item {Grp. gram.:f.}
\end{itemize}
Gênero de plantas rubiáceas.
\section{Octaviano}
\begin{itemize}
\item {Grp. gram.:adj.}
\end{itemize}
Relativo ao tempo do imperador Octávio: \textunderscore a paz octaviana\textunderscore .
\section{Octavo}
\begin{itemize}
\item {Grp. gram.:adj.}
\end{itemize}
\begin{itemize}
\item {Utilização:Ant.}
\end{itemize}
O mesmo que \textunderscore oitavo\textunderscore :«\textunderscore ...do papa Innocêncio octavo\textunderscore ». R. de Pina, \textunderscore João II\textunderscore , c. XX.
\section{Octil}
\begin{itemize}
\item {Grp. gram.:adj.}
\end{itemize}
\begin{itemize}
\item {Proveniência:(Do lat. \textunderscore octo\textunderscore )}
\end{itemize}
Diz-se da posição de dois planetas, que guardam entre si a distância da oitava parte do Zodíaco.
\section{Octilião}
\begin{itemize}
\item {Grp. gram.:m.}
\end{itemize}
\begin{itemize}
\item {Utilização:Arith.}
\end{itemize}
\begin{itemize}
\item {Proveniência:(Do lat. \textunderscore octo\textunderscore . Cp. \textunderscore bilião\textunderscore )}
\end{itemize}
Septilião, multiplicado por mil.
\section{Octillião}
\begin{itemize}
\item {Grp. gram.:m.}
\end{itemize}
\begin{itemize}
\item {Utilização:Arith.}
\end{itemize}
\begin{itemize}
\item {Proveniência:(Do lat. \textunderscore octo\textunderscore . Cp. \textunderscore billião\textunderscore )}
\end{itemize}
Septillião, multiplicado por mil.
\section{Octilo}
\begin{itemize}
\item {Grp. gram.:m.}
\end{itemize}
\begin{itemize}
\item {Utilização:Chím.}
\end{itemize}
\begin{itemize}
\item {Proveniência:(Do gr. \textunderscore okto\textunderscore  + \textunderscore hule\textunderscore )}
\end{itemize}
Radical alcoólico, que contém oito átomos de carbóne.
\section{Octingentésimo}
\begin{itemize}
\item {Grp. gram.:adj.}
\end{itemize}
\begin{itemize}
\item {Proveniência:(Lat. \textunderscore octingentesimus\textunderscore )}
\end{itemize}
Que, numa série de 800, occupa o último lugar.
\section{Octípede}
\begin{itemize}
\item {Grp. gram.:adj.}
\end{itemize}
\begin{itemize}
\item {Proveniência:(Do lat. \textunderscore octo\textunderscore  + \textunderscore pes\textunderscore , \textunderscore pedis\textunderscore )}
\end{itemize}
Que tem oito pés.
\section{Octo...}
\begin{itemize}
\item {Grp. gram.:pref.}
\end{itemize}
\begin{itemize}
\item {Proveniência:(Lat. \textunderscore octo\textunderscore )}
\end{itemize}
(designativo de \textunderscore oito\textunderscore )
\section{Octobléfaro}
\begin{itemize}
\item {Grp. gram.:m.}
\end{itemize}
Gênero de musgos.
\section{Octoblépharo}
\begin{itemize}
\item {Grp. gram.:m.}
\end{itemize}
Gênero de musgos.
\section{Octobóthrio}
\begin{itemize}
\item {Grp. gram.:m.}
\end{itemize}
Gênero de helminthos, parasitos dos peixes.
\section{Octobótrio}
\begin{itemize}
\item {Grp. gram.:m.}
\end{itemize}
Gênero de helmintos, parasitos dos peixes.
\section{Octocórneo}
\begin{itemize}
\item {Grp. gram.:adj.}
\end{itemize}
\begin{itemize}
\item {Utilização:Zool.}
\end{itemize}
\begin{itemize}
\item {Proveniência:(Do lat. \textunderscore octo\textunderscore  + \textunderscore cornu\textunderscore )}
\end{itemize}
Que tem oito cornos.
\section{Octodáctilo}
\begin{itemize}
\item {Grp. gram.:adj.}
\end{itemize}
\begin{itemize}
\item {Utilização:Zool.}
\end{itemize}
\begin{itemize}
\item {Proveniência:(Do gr. \textunderscore okto\textunderscore  + \textunderscore daktulos\textunderscore )}
\end{itemize}
Que tem oito dedos.
\section{Octodáctylo}
\begin{itemize}
\item {Grp. gram.:adj.}
\end{itemize}
\begin{itemize}
\item {Utilização:Zool.}
\end{itemize}
\begin{itemize}
\item {Proveniência:(Do gr. \textunderscore okto\textunderscore  + \textunderscore daktulos\textunderscore )}
\end{itemize}
Que tem oito dedos.
\section{Octodecimal}
\begin{itemize}
\item {Grp. gram.:adj.}
\end{itemize}
\begin{itemize}
\item {Utilização:Miner.}
\end{itemize}
Diz-se do crystal, que apresenta dezóito faces.
\section{Octodícera}
\begin{itemize}
\item {Grp. gram.:f.}
\end{itemize}
Gênero de musgos.
\section{Octodonte}
\begin{itemize}
\item {Grp. gram.:m.}
\end{itemize}
\begin{itemize}
\item {Proveniência:(Do gr. \textunderscore okto\textunderscore  + \textunderscore odous\textunderscore , \textunderscore odontos\textunderscore )}
\end{itemize}
Gênero de plantas rubiáceas.
\section{Octoduodecimal}
\begin{itemize}
\item {Grp. gram.:adj.}
\end{itemize}
\begin{itemize}
\item {Utilização:Miner.}
\end{itemize}
Diz-se do crystal, cuja superfície tem vinte faces, das quaes oito, imaginando-se prolongadas, produziriam o octaédro, e os outros um dodecaédro.
\section{Octófido}
\begin{itemize}
\item {Grp. gram.:adj.}
\end{itemize}
\begin{itemize}
\item {Utilização:Bot.}
\end{itemize}
\begin{itemize}
\item {Proveniência:(Do lat. \textunderscore octo\textunderscore  + \textunderscore findere\textunderscore )}
\end{itemize}
Que apresenta oito recortes, cuja profundidade é, pelo menos, igual a metade do comprimento total do respectivo órgão.
\section{Octofilo}
\begin{itemize}
\item {Grp. gram.:adj.}
\end{itemize}
\begin{itemize}
\item {Utilização:Bot.}
\end{itemize}
\begin{itemize}
\item {Proveniência:(Do gr. \textunderscore okto\textunderscore  + \textunderscore phullon\textunderscore )}
\end{itemize}
Que tem oito fôlhas ou folíolos.
\section{Octóforo}
\begin{itemize}
\item {Grp. gram.:m.}
\end{itemize}
\begin{itemize}
\item {Proveniência:(Lat. \textunderscore octophoron\textunderscore )}
\end{itemize}
Espécie de liteira que, entre os antigos Gregos e Romanos, era transportada por oito escravos.
\section{Octogamia}
\begin{itemize}
\item {Grp. gram.:f.}
\end{itemize}
Estado ou qualidade de octógamo.
\section{Octógamo}
\begin{itemize}
\item {Grp. gram.:adj.}
\end{itemize}
\begin{itemize}
\item {Proveniência:(Do gr. \textunderscore okto\textunderscore  + \textunderscore gamos\textunderscore )}
\end{itemize}
Que casou oito vezes, ou que tem oito mulheres ao mesmo tempo.
\section{Octogenário}
\begin{itemize}
\item {Grp. gram.:m.  e  adj.}
\end{itemize}
\begin{itemize}
\item {Proveniência:(Lat. \textunderscore octogenarius\textunderscore )}
\end{itemize}
Aquelle que tem oitenta annos.
\section{Octogentésimo}
\begin{itemize}
\item {Grp. gram.:adj.}
\end{itemize}
O mesmo que \textunderscore octingentésimo\textunderscore .
\section{Octogésimo}
\begin{itemize}
\item {Grp. gram.:adj.}
\end{itemize}
\begin{itemize}
\item {Proveniência:(Lat. \textunderscore octogesimus\textunderscore )}
\end{itemize}
Que, numa série de oitenta, occupa o último lugar.
\section{Octoginia}
\begin{itemize}
\item {Grp. gram.:f.}
\end{itemize}
Qualidade de octógino.
Ordem de plantas que, no sistema de Linneu, compreende as que têm oito pistilos.
\section{Octogínico}
\begin{itemize}
\item {Grp. gram.:adj.}
\end{itemize}
Relativo á octoginia.
\section{Octógino}
\begin{itemize}
\item {Grp. gram.:adj.}
\end{itemize}
\begin{itemize}
\item {Utilização:Bot.}
\end{itemize}
\begin{itemize}
\item {Proveniência:(Do gr. \textunderscore okto\textunderscore  + \textunderscore gune\textunderscore )}
\end{itemize}
Que tem oito pistilos.
\section{Octoglosso}
\begin{itemize}
\item {Grp. gram.:m.}
\end{itemize}
\begin{itemize}
\item {Proveniência:(Do gr. \textunderscore okto\textunderscore  + \textunderscore glossa\textunderscore )}
\end{itemize}
Gênero de insectos, caracterizados principalmente por um lábio inferior com oito lóbulos alongados.
\section{Octogonal}
\begin{itemize}
\item {Grp. gram.:adj.}
\end{itemize}
Que tem oito ângulos, cuja base é um octógono.
\section{Octógono}
\begin{itemize}
\item {Grp. gram.:adj.}
\end{itemize}
\begin{itemize}
\item {Grp. gram.:M.}
\end{itemize}
\begin{itemize}
\item {Proveniência:(Do gr. \textunderscore okto\textunderscore  + \textunderscore gonos\textunderscore )}
\end{itemize}
Octogonal.
Polýgono de oito ângulos.
Construcção em fórma de octógono.
\section{Octogynia}
\begin{itemize}
\item {Grp. gram.:f.}
\end{itemize}
Qualidade de octógyno.
Ordem de plantas que, no systema de Linneu, comprehende as que têm oito pistillos.
\section{Octogýnico}
\begin{itemize}
\item {Grp. gram.:adj.}
\end{itemize}
Relativo á octogynia.
\section{Octógyno}
\begin{itemize}
\item {Grp. gram.:adj.}
\end{itemize}
\begin{itemize}
\item {Utilização:Bot.}
\end{itemize}
\begin{itemize}
\item {Proveniência:(Do gr. \textunderscore okto\textunderscore  + \textunderscore gune\textunderscore )}
\end{itemize}
Que tem oito pistillos.
\section{Octolépido}
\begin{itemize}
\item {Grp. gram.:adj.}
\end{itemize}
\begin{itemize}
\item {Utilização:Bot.}
\end{itemize}
\begin{itemize}
\item {Proveniência:(Do gr. \textunderscore okto\textunderscore  + \textunderscore lepsis\textunderscore )}
\end{itemize}
Que tem oito escamas.
\section{Octolobulado}
\begin{itemize}
\item {Grp. gram.:adj.}
\end{itemize}
\begin{itemize}
\item {Utilização:Bot.}
\end{itemize}
\begin{itemize}
\item {Proveniência:(De \textunderscore octo...\textunderscore  + \textunderscore lóbulo\textunderscore )}
\end{itemize}
Que tem oito lóbulos.
\section{Octoméria}
\begin{itemize}
\item {Grp. gram.:f.}
\end{itemize}
\begin{itemize}
\item {Proveniência:(Do gr. \textunderscore okto\textunderscore  + \textunderscore meros\textunderscore )}
\end{itemize}
Gênero de orchídeas.
\section{Octonado}
\begin{itemize}
\item {Grp. gram.:adj.}
\end{itemize}
\begin{itemize}
\item {Proveniência:(Do lat. \textunderscore octo\textunderscore )}
\end{itemize}
Disposto em grupos de oito.
\section{Octonário}
\begin{itemize}
\item {Grp. gram.:m.  e  adj.}
\end{itemize}
\begin{itemize}
\item {Proveniência:(Lat. \textunderscore octonarius\textunderscore )}
\end{itemize}
Diz-se do verso, que tem oito pés.
\section{Octopétalo}
\begin{itemize}
\item {Grp. gram.:adj.}
\end{itemize}
\begin{itemize}
\item {Utilização:Bot.}
\end{itemize}
\begin{itemize}
\item {Proveniência:(De \textunderscore octo...\textunderscore  + \textunderscore pétala\textunderscore )}
\end{itemize}
Que tem oito pétalas.
\section{Octóphoro}
\begin{itemize}
\item {Grp. gram.:m.}
\end{itemize}
\begin{itemize}
\item {Proveniência:(Lat. \textunderscore octophoron\textunderscore )}
\end{itemize}
Espécie de liteira que, entre os antigos Gregos e Romanos, era transportada por oito escravos.
\section{Octophyllo}
\begin{itemize}
\item {Grp. gram.:adj.}
\end{itemize}
\begin{itemize}
\item {Utilização:Bot.}
\end{itemize}
\begin{itemize}
\item {Proveniência:(Do gr. \textunderscore okto\textunderscore  + \textunderscore phullon\textunderscore )}
\end{itemize}
Que tem oito fôlhas ou folíolos.
\section{Octópode}
\begin{itemize}
\item {Grp. gram.:adj.}
\end{itemize}
\begin{itemize}
\item {Utilização:Zool.}
\end{itemize}
\begin{itemize}
\item {Grp. gram.:M. pl.}
\end{itemize}
\begin{itemize}
\item {Proveniência:(Do gr. \textunderscore okto\textunderscore  + \textunderscore pous, podos\textunderscore )}
\end{itemize}
Que tem oito pés ou tentáculos.
Molluscos cephalópodes de oito pés.
\section{Octopódio}
\begin{itemize}
\item {Grp. gram.:m}
\end{itemize}
Antigo estandarte pontifício, dividido em oito flâmmulas, no qual se representava algum santo.
(Cp. \textunderscore octópode\textunderscore )
\section{Octoreme}
\begin{itemize}
\item {fónica:re}
\end{itemize}
\begin{itemize}
\item {Grp. gram.:m.}
\end{itemize}
\begin{itemize}
\item {Proveniência:(De \textunderscore octo...\textunderscore  + \textunderscore remo\textunderscore )}
\end{itemize}
Navio antigo, com oito ordens de remos ou oito remadores de cada banda.
\section{Octorreme}
\begin{itemize}
\item {Grp. gram.:m.}
\end{itemize}
\begin{itemize}
\item {Proveniência:(De \textunderscore octo...\textunderscore  + \textunderscore remo\textunderscore )}
\end{itemize}
Navio antigo, com oito ordens de remos ou oito remadores de cada banda.
\section{Octosépalo}
\begin{itemize}
\item {fónica:sé}
\end{itemize}
\begin{itemize}
\item {Grp. gram.:adj.}
\end{itemize}
\begin{itemize}
\item {Utilização:Bot.}
\end{itemize}
\begin{itemize}
\item {Proveniência:(De \textunderscore octo...\textunderscore  + \textunderscore sépala\textunderscore )}
\end{itemize}
Que tem oito sépalas.
\section{Octosesdecimal}
\begin{itemize}
\item {fónica:ses}
\end{itemize}
\begin{itemize}
\item {Grp. gram.:adj.}
\end{itemize}
\begin{itemize}
\item {Utilização:Miner.}
\end{itemize}
Diz-se do crystal, que tem a fórma de um prisma de oito lados, terminando por dois ângulos de oito faces.
\section{Octosesvigesimal}
\begin{itemize}
\item {fónica:ses}
\end{itemize}
\begin{itemize}
\item {Grp. gram.:adj.}
\end{itemize}
\begin{itemize}
\item {Utilização:Miner.}
\end{itemize}
Diz-se do crystal que tem trinta e quatro faces.
\section{Octossesdecimal}
\begin{itemize}
\item {Grp. gram.:adj.}
\end{itemize}
\begin{itemize}
\item {Utilização:Miner.}
\end{itemize}
Diz-se do cristal, que tem a fórma de um prisma de oito lados, terminando por dois ângulos de oito faces.
\section{Octossépalo}
\begin{itemize}
\item {Grp. gram.:adj.}
\end{itemize}
\begin{itemize}
\item {Utilização:Bot.}
\end{itemize}
\begin{itemize}
\item {Proveniência:(De \textunderscore octo...\textunderscore  + \textunderscore sépala\textunderscore )}
\end{itemize}
Que tem oito sépalas.
\section{Octossesvigesimal}
\begin{itemize}
\item {Grp. gram.:adj.}
\end{itemize}
\begin{itemize}
\item {Utilização:Miner.}
\end{itemize}
Diz-se do cristal que tem trinta e quatro faces.
\section{Octossilábico}
\begin{itemize}
\item {Grp. gram.:adj.}
\end{itemize}
O mesmo que \textunderscore octossílabo\textunderscore .
\section{Octossílabo}
\begin{itemize}
\item {Grp. gram.:adj.}
\end{itemize}
\begin{itemize}
\item {Proveniência:(De \textunderscore octo...\textunderscore  + \textunderscore sílaba\textunderscore )}
\end{itemize}
Que tem oito sílabas.
\section{Octostêmone}
\begin{itemize}
\item {Grp. gram.:adj.}
\end{itemize}
\begin{itemize}
\item {Utilização:Bot.}
\end{itemize}
\begin{itemize}
\item {Proveniência:(Do gr. \textunderscore okto\textunderscore  + \textunderscore stemon\textunderscore )}
\end{itemize}
Que tem oito estames.
\section{Octostilo}
\begin{itemize}
\item {Grp. gram.:m.}
\end{itemize}
\begin{itemize}
\item {Proveniência:(Do gr. \textunderscore okto\textunderscore  + \textunderscore stulos\textunderscore )}
\end{itemize}
Fachada com oito colunas.
\section{Octostylo}
\begin{itemize}
\item {Grp. gram.:m.}
\end{itemize}
\begin{itemize}
\item {Proveniência:(Do gr. \textunderscore okto\textunderscore  + \textunderscore stulos\textunderscore )}
\end{itemize}
Fachada com oito columnas.
\section{Octosyllábico}
\begin{itemize}
\item {fónica:si}
\end{itemize}
\begin{itemize}
\item {Grp. gram.:adj.}
\end{itemize}
O mesmo que \textunderscore octosýllabo\textunderscore .
\section{Octosýllabo}
\begin{itemize}
\item {fónica:si}
\end{itemize}
\begin{itemize}
\item {Grp. gram.:adj.}
\end{itemize}
\begin{itemize}
\item {Proveniência:(De \textunderscore octo...\textunderscore  + \textunderscore sýllaba\textunderscore )}
\end{itemize}
Que tem oito sýllabas.
\section{Octótoma}
\begin{itemize}
\item {Grp. gram.:f.}
\end{itemize}
Gênero de insectos coleópteros.
\section{Octotrigesimal}
\begin{itemize}
\item {Grp. gram.:adj.}
\end{itemize}
\begin{itemize}
\item {Utilização:Miner.}
\end{itemize}
Diz-se do crystal, que tem trinta e oito faces.
\section{Octovalve}
\begin{itemize}
\item {Grp. gram.:adj.}
\end{itemize}
\begin{itemize}
\item {Proveniência:(De \textunderscore octo...\textunderscore  + \textunderscore valva\textunderscore )}
\end{itemize}
Que tem oito valvas.
\section{Octovalvo}
\begin{itemize}
\item {Grp. gram.:adj.}
\end{itemize}
O mesmo que \textunderscore octovalve\textunderscore .
\section{Octovigesimal}
\begin{itemize}
\item {Grp. gram.:adj.}
\end{itemize}
\begin{itemize}
\item {Utilização:Miner.}
\end{itemize}
Diz-se do crystal, que tem vinte e oito faces.
\section{Octóviro}
\begin{itemize}
\item {Grp. gram.:m.}
\end{itemize}
\begin{itemize}
\item {Proveniência:(Lat. \textunderscore octovir\textunderscore )}
\end{itemize}
Indivíduo, que faz parte de uma corporação de oito membros.
\section{Octual}
\begin{itemize}
\item {Grp. gram.:m.}
\end{itemize}
Medida polaca, para líquidos.
\section{Octuplicar}
\begin{itemize}
\item {Grp. gram.:v. t.}
\end{itemize}
\begin{itemize}
\item {Utilização:P. us.}
\end{itemize}
\begin{itemize}
\item {Proveniência:(De \textunderscore óctuplo\textunderscore )}
\end{itemize}
Multiplicar por oito.
\section{Óctuplo}
\begin{itemize}
\item {Grp. gram.:adj.}
\end{itemize}
\begin{itemize}
\item {Grp. gram.:M.}
\end{itemize}
\begin{itemize}
\item {Proveniência:(Lat. \textunderscore octuplus\textunderscore )}
\end{itemize}
Multiplicado por oito.
Repetido oito vezes.
Quantidade, oito vezes superior a outra.
\section{Octusse}
\begin{itemize}
\item {Grp. gram.:m.}
\end{itemize}
\begin{itemize}
\item {Proveniência:(Lat. \textunderscore octussis\textunderscore )}
\end{itemize}
Somma de oito asses, entre os Romanos. Cf. Castilho, \textunderscore Fastos\textunderscore , I, 354.
\section{Octýlio}
\begin{itemize}
\item {Grp. gram.:m.}
\end{itemize}
O mesmo ou melhor que \textunderscore octylo\textunderscore .
\section{Octylo}
\begin{itemize}
\item {Grp. gram.:m.}
\end{itemize}
\begin{itemize}
\item {Utilização:Chím.}
\end{itemize}
\begin{itemize}
\item {Proveniência:(Do gr. \textunderscore okto\textunderscore  + \textunderscore hule\textunderscore )}
\end{itemize}
Radical alcoólico, que contém oito átomos de carbóne.
\section{Oculação}
\begin{itemize}
\item {Grp. gram.:f.}
\end{itemize}
\begin{itemize}
\item {Proveniência:(Do lat. \textunderscore oculus\textunderscore )}
\end{itemize}
Acção do enxertar numa árvore um ôlho de outra.
\section{Oculado}
\begin{itemize}
\item {Grp. gram.:adj.}
\end{itemize}
\begin{itemize}
\item {Proveniência:(De \textunderscore oculo\textunderscore )}
\end{itemize}
Que tem olhos.
Mosqueado, ocellado.
\section{Ocular}
\begin{itemize}
\item {Grp. gram.:adj.}
\end{itemize}
\begin{itemize}
\item {Grp. gram.:M.  e  f.}
\end{itemize}
\begin{itemize}
\item {Proveniência:(Lat. \textunderscore ocularis\textunderscore )}
\end{itemize}
Relativo ao ôlho ou á vista: \textunderscore moléstia ocular\textunderscore .
Que presenciou, que viu: \textunderscore testemunha ocular\textunderscore .
Lente ou vidro de um óculo.
\section{Ocularmente}
\begin{itemize}
\item {Grp. gram.:adv.}
\end{itemize}
De modo ocular; por meio da vista.
\section{Ocúleos}
\begin{itemize}
\item {Grp. gram.:m. pl.}
\end{itemize}
\begin{itemize}
\item {Proveniência:(Lat. \textunderscore oculeus\textunderscore )}
\end{itemize}
Tríbo de insectos hemípteros.
\section{Oculífero}
\begin{itemize}
\item {Grp. gram.:adj.}
\end{itemize}
\begin{itemize}
\item {Utilização:Zool.}
\end{itemize}
\begin{itemize}
\item {Proveniência:(Do lat. \textunderscore oculus\textunderscore  + \textunderscore ferre\textunderscore )}
\end{itemize}
Que tem ou apresenta um ôlho.
\section{Oculiforme}
\begin{itemize}
\item {Grp. gram.:adj.}
\end{itemize}
\begin{itemize}
\item {Proveniência:(Do lat. \textunderscore oculus\textunderscore  + \textunderscore fórma\textunderscore )}
\end{itemize}
Que tem fórma de ôlho.
\section{Oculina}
\begin{itemize}
\item {Grp. gram.:f.}
\end{itemize}
\begin{itemize}
\item {Proveniência:(Do lat. \textunderscore oculus\textunderscore )}
\end{itemize}
Gênero de polypeiros fósseis.
\section{Oculinomancia}
\begin{itemize}
\item {Grp. gram.:f.}
\end{itemize}
Antigo processo de adivinhação, com que se procurava conhecer os ladrões, por meio de certas operações que lhes faziam nos olhos.
\section{Oculinomântico}
\begin{itemize}
\item {Grp. gram.:adj.}
\end{itemize}
\begin{itemize}
\item {Grp. gram.:M.}
\end{itemize}
Relativo á oculinomancia.
Aquelle que praticava a oculinomancia.
\section{Oculista}
\begin{itemize}
\item {Grp. gram.:m.}
\end{itemize}
\begin{itemize}
\item {Grp. gram.:Adj.}
\end{itemize}
\begin{itemize}
\item {Proveniência:(Do lat. \textunderscore oculus\textunderscore )}
\end{itemize}
Fabricante ou vendedor de óculos.
Médico, que trata especialmente das doenças de olhos.
Que trata das doenças de olhos.
\section{Oculística}
\begin{itemize}
\item {Grp. gram.:f.}
\end{itemize}
\begin{itemize}
\item {Proveniência:(De \textunderscore oculista\textunderscore )}
\end{itemize}
Parte da Medicina, que trata das doenças dos olhos.
\section{Óculo}
\begin{itemize}
\item {Grp. gram.:m.}
\end{itemize}
\begin{itemize}
\item {Utilização:Prov.}
\end{itemize}
\begin{itemize}
\item {Grp. gram.:M. pl.}
\end{itemize}
\begin{itemize}
\item {Proveniência:(Lat. \textunderscore oculus\textunderscore )}
\end{itemize}
Instrumento, armado de lentes, para auxiliar a vista.
Orifício circular, nas paredes de alguns edifícios, para entrar nestes a luz e o ar.
Abertura nas portinholas dos navios, pela qual se enfiam os canos das peças.
Binóculo.
Abertura na terra, por onde se dá luz a uma mina.
Luneta, de cujas extremidades saem hastes, que se adaptam á parte posterior da orelha.
\section{Oculo-musculoso}
\begin{itemize}
\item {Grp. gram.:adj.}
\end{itemize}
\begin{itemize}
\item {Utilização:Anat.}
\end{itemize}
Relativo aos músculos oculares.
\section{Oculoso}
\begin{itemize}
\item {Grp. gram.:adj.}
\end{itemize}
O mesmo que \textunderscore oculado\textunderscore .
\section{Ocultação}
\begin{itemize}
\item {Grp. gram.:f.}
\end{itemize}
\begin{itemize}
\item {Proveniência:(Lat. \textunderscore occultatio\textunderscore )}
\end{itemize}
Acto ou efeito de ocultar.
\section{Ocultador}
\begin{itemize}
\item {Grp. gram.:m.  e  adj.}
\end{itemize}
O que oculta.
\section{Ocultamente}
\begin{itemize}
\item {Grp. gram.:adv.}
\end{itemize}
De modo oculto; ás escondidas: furtivamente.
\section{Ocultante}
\begin{itemize}
\item {Grp. gram.:adj.}
\end{itemize}
\begin{itemize}
\item {Proveniência:(Lat. \textunderscore occultans\textunderscore )}
\end{itemize}
Que oculta.
\section{Ocultar}
\begin{itemize}
\item {Grp. gram.:v. t.}
\end{itemize}
\begin{itemize}
\item {Proveniência:(Lat. \textunderscore occultare\textunderscore )}
\end{itemize}
Não deixar vêr.
Esconder.
Sonegar.
Dissimular.
\section{Ocultas}
\begin{itemize}
\item {Grp. gram.:f. pl. Loc. adv.}
\end{itemize}
\textunderscore Ás ocultas\textunderscore , ou \textunderscore a ocultas\textunderscore , o mesmo que \textunderscore ocultamente\textunderscore .
\section{Ocultismo}
\begin{itemize}
\item {Grp. gram.:m.}
\end{itemize}
\begin{itemize}
\item {Proveniência:(De \textunderscore oculto\textunderscore )}
\end{itemize}
Conjunto das artes ou ciências ocultas, como a magia, o espiritismo, etc.
\section{Ocultista}
\begin{itemize}
\item {Grp. gram.:m.}
\end{itemize}
Aquele que se dedica ao ocultismo.
\section{Oculto}
\begin{itemize}
\item {Grp. gram.:adj.}
\end{itemize}
\begin{itemize}
\item {Proveniência:(Lat. \textunderscore occultus\textunderscore )}
\end{itemize}
Que só é conhecido pelos seus efeitos, e não por si próprio.
Escondido.
Desconhecido.
Misterioso; sobrenatural.
Não percorrido nem explorado.
\section{Ocumba}
\begin{itemize}
\item {Grp. gram.:f.}
\end{itemize}
\begin{itemize}
\item {Utilização:Bras}
\end{itemize}
Árvore silvestre.
\section{Ocupação}
\begin{itemize}
\item {Grp. gram.:f.}
\end{itemize}
\begin{itemize}
\item {Proveniência:(Lat. \textunderscore occupatio\textunderscore )}
\end{itemize}
Acto ou efeito de ocupar.
Posse: \textunderscore ocupação de um território\textunderscore .
Emprêgo; ofício: \textunderscore a sua ocupação é funileiro\textunderscore .
\section{Ocupadamente}
\begin{itemize}
\item {Grp. gram.:adv.}
\end{itemize}
\begin{itemize}
\item {Proveniência:(De \textunderscore ocupado\textunderscore )}
\end{itemize}
Afanosamente; com trabalho.
\section{Ocupado}
\begin{itemize}
\item {Grp. gram.:adj.}
\end{itemize}
Que tem alguma coisa que fazer ou em que pensar.
Diz-se da mulher em estado de gravidez.
\section{Ocupador}
\begin{itemize}
\item {Grp. gram.:adj.}
\end{itemize}
\begin{itemize}
\item {Proveniência:(Lat. \textunderscore occupator\textunderscore )}
\end{itemize}
Que ocupa ou que ocupou.
\section{Ocupante}
\begin{itemize}
\item {Grp. gram.:adj.}
\end{itemize}
O mesmo que \textunderscore ocupador\textunderscore .
\section{Ocupar}
\begin{itemize}
\item {Grp. gram.:v. t.}
\end{itemize}
\begin{itemize}
\item {Grp. gram.:V. i.}
\end{itemize}
\begin{itemize}
\item {Grp. gram.:V. p.}
\end{itemize}
\begin{itemize}
\item {Proveniência:(Lat. \textunderscore occupare\textunderscore )}
\end{itemize}
Estar na posse de.
Tomar posse de: \textunderscore ocupar um território\textunderscore .
Habitar: \textunderscore ocupar uma casa\textunderscore .
Conquistar.
Tomar, encher: \textunderscore ocupar espaço\textunderscore .
Sêr objecto de; fixar, atrair: \textunderscore ocupar a atenção do auditório\textunderscore .
Dar trabalho ou cuidado a: \textunderscore ocupa-me muito o futuro dos filhos\textunderscore .
Tornar-se grávida, (a mulher).
Empregar-se.
Aplicar a atenção.
Gastar o tempo em alguma coisa: \textunderscore ocupa-se em dizer mal\textunderscore .
\section{Ocursar}
\begin{itemize}
\item {Grp. gram.:v. i.}
\end{itemize}
\begin{itemize}
\item {Utilização:Des.}
\end{itemize}
\begin{itemize}
\item {Proveniência:(Lat. \textunderscore occursare\textunderscore )}
\end{itemize}
Ocorrer; apresentar-se deante.
\section{Ócyde}
\begin{itemize}
\item {Grp. gram.:f.}
\end{itemize}
Gênero de insectos, da fam. dos escaravelhos.
\section{Ócypo}
\begin{itemize}
\item {Grp. gram.:m.}
\end{itemize}
\begin{itemize}
\item {Proveniência:(Do gr. \textunderscore okus\textunderscore  + \textunderscore pous\textunderscore , \textunderscore podos\textunderscore )}
\end{itemize}
Gênero de insectos coleópteros pentâmeros.
\section{Ocýpoda}
\begin{itemize}
\item {Grp. gram.:f.}
\end{itemize}
\begin{itemize}
\item {Proveniência:(Do gr. \textunderscore okus\textunderscore  + \textunderscore pous\textunderscore , \textunderscore podos\textunderscore )}
\end{itemize}
Gênero de crustáceos decápodes.
\section{Ocyporitos}
\begin{itemize}
\item {Grp. gram.:m. pl.}
\end{itemize}
Grupo de insectos coleópteros pentâmeros.
\section{Ocýptamo}
\begin{itemize}
\item {Grp. gram.:m.}
\end{itemize}
Gênero de insectos dípteros.
\section{Ocýtera}
\begin{itemize}
\item {Grp. gram.:f.}
\end{itemize}
Gênero de insectos dípteros.
\section{Ocýthoe}
\begin{itemize}
\item {Grp. gram.:f.}
\end{itemize}
Gênero de molluscos cephalópodes.
\section{Ocitócico}
\begin{itemize}
\item {Grp. gram.:adj.}
\end{itemize}
\begin{itemize}
\item {Utilização:Med.}
\end{itemize}
\begin{itemize}
\item {Proveniência:(Do gr. \textunderscore okus\textunderscore , rápido, e \textunderscore tokos\textunderscore , parto)}
\end{itemize}
Diz-se do medicamento que acelera o parto.
\section{Ocytócico}
\begin{itemize}
\item {Grp. gram.:adj.}
\end{itemize}
\begin{itemize}
\item {Utilização:Med.}
\end{itemize}
\begin{itemize}
\item {Proveniência:(Do gr. \textunderscore okus\textunderscore , rápido, e \textunderscore tokos\textunderscore , parto)}
\end{itemize}
Diz-se do medicamento que acelera o parto.
\section{Odacanta}
\begin{itemize}
\item {Grp. gram.:f.}
\end{itemize}
Gênero de insectos coleópteros pentâmeros.
\section{Odalisca}
\begin{itemize}
\item {Grp. gram.:f.}
\end{itemize}
Escrava do harém, ao serviço das mulheres do Sultão.
Impropriamente, cada uma das mulheres, que o Sultão tem no harém.
(Do turco \textunderscore odalik\textunderscore )
\section{Odátria}
\begin{itemize}
\item {Grp. gram.:f.}
\end{itemize}
Reptil sáurio, espécie de lagarto.
\section{Ódax}
\begin{itemize}
\item {Grp. gram.:m.}
\end{itemize}
Gênero de peixes acanthopterýgios.
\section{Odaxismo}
\begin{itemize}
\item {fónica:csis}
\end{itemize}
\begin{itemize}
\item {Grp. gram.:m.}
\end{itemize}
\begin{itemize}
\item {Proveniência:(Gr. \textunderscore odaxismos\textunderscore )}
\end{itemize}
Prurido das gengivas, que precede o nascimento dos dentes.
\section{Ode}
\begin{itemize}
\item {Grp. gram.:f.}
\end{itemize}
\begin{itemize}
\item {Proveniência:(Lat. \textunderscore ode\textunderscore )}
\end{itemize}
Composição poética, dividida em estróphes symétricas.
Primitivamente, composição poética para sêr cantada.
\section{Odeão}
\begin{itemize}
\item {Grp. gram.:m.}
\end{itemize}
\begin{itemize}
\item {Proveniência:(Lat. \textunderscore odeon\textunderscore )}
\end{itemize}
Edifício, destinado entre os Gregos ao ensaio da música, que se havia de cantar nos theatros.
\section{Odézia}
\begin{itemize}
\item {Grp. gram.:f.}
\end{itemize}
Gênero de insectos lepidópteros nocturnos.
\section{Odiá}
\begin{itemize}
\item {Grp. gram.:f.}
\end{itemize}
Dádiva ou presente, em alguns povos antigos da Indo-China. Cf. \textunderscore Peregrinação\textunderscore , XIX.--O mesmo autor, no cap. LXIV, acentua \textunderscore ódia\textunderscore .
\section{Odiar}
\begin{itemize}
\item {Grp. gram.:v. t.}
\end{itemize}
Têr ódio a.
Detestar; sentir repugnância por.
\section{Odiável}
\begin{itemize}
\item {Grp. gram.:adj.}
\end{itemize}
Que se deve odiar. Cf. Camillo, \textunderscore Cavar em Ruínas\textunderscore , 195.
\section{Odiência}
\begin{itemize}
\item {Grp. gram.:f.}
\end{itemize}
\begin{itemize}
\item {Utilização:Ant.}
\end{itemize}
O mesmo que \textunderscore audiência\textunderscore . Cf. \textunderscore Leal Conselheiro\textunderscore .
\section{Odiento}
\begin{itemize}
\item {Grp. gram.:adj.}
\end{itemize}
Que tem ou conserva ódio: \textunderscore homem odiento\textunderscore .
Que revela ódio: \textunderscore actos odientos\textunderscore .
\section{Odina}
\begin{itemize}
\item {Grp. gram.:f.}
\end{itemize}
Gênero de plantas terebintháceas.
\section{Odínico}
\begin{itemize}
\item {Grp. gram.:adj.}
\end{itemize}
Relativo á região escandinava de Odim.
\section{Ódio}
\begin{itemize}
\item {Grp. gram.:m.}
\end{itemize}
\begin{itemize}
\item {Proveniência:(Lat. \textunderscore odium\textunderscore )}
\end{itemize}
Rancor, ira profunda.
Sentimento de repulsão.
Horror; antipathia.
\section{Odiosamente}
\begin{itemize}
\item {Grp. gram.:adv.}
\end{itemize}
De modo odioso.
\section{Odiosidade}
\begin{itemize}
\item {Grp. gram.:f.}
\end{itemize}
\begin{itemize}
\item {Utilização:Des.}
\end{itemize}
Qualidade daquillo que é odioso.
\section{Odioso}
\begin{itemize}
\item {Grp. gram.:adj.}
\end{itemize}
\begin{itemize}
\item {Grp. gram.:M.}
\end{itemize}
\begin{itemize}
\item {Proveniência:(Lat. \textunderscore odiosus\textunderscore )}
\end{itemize}
Que merece ódio; que inspira ódio.
Repellente; detestável.
Condemnável.
Aquillo que é odioso.
\section{Odissaico}
\begin{itemize}
\item {Grp. gram.:adj.}
\end{itemize}
Relativo ao ciclo histórico das odisseias. Cf. Th. Braga, \textunderscore Mod. Ideias\textunderscore , 232.
\section{Odisséa}
\begin{itemize}
\item {Grp. gram.:f.}
\end{itemize}
\begin{itemize}
\item {Utilização:Fig.}
\end{itemize}
\begin{itemize}
\item {Proveniência:(Lat. \textunderscore Odysseia\textunderscore )}
\end{itemize}
Viagem, cheia de aventuras extraordinárias.
Qualquer narração de aventuras extraordinárias.
\section{Odito}
\begin{itemize}
\item {Grp. gram.:m.}
\end{itemize}
\begin{itemize}
\item {Utilização:Miner.}
\end{itemize}
Variedade de mica, de côr pardo-amarelada.
\section{Odográfico}
\begin{itemize}
\item {Grp. gram.:adj.}
\end{itemize}
\begin{itemize}
\item {Proveniência:(Do gr. \textunderscore odos\textunderscore  + \textunderscore graphein\textunderscore )}
\end{itemize}
Que indica ou marca os caminhos.
\section{Odográphico}
\begin{itemize}
\item {Grp. gram.:adj.}
\end{itemize}
\begin{itemize}
\item {Proveniência:(Do gr. \textunderscore odos\textunderscore  + \textunderscore graphein\textunderscore )}
\end{itemize}
Que indica ou marca os caminhos.
\section{Odol}
\begin{itemize}
\item {Grp. gram.:m.}
\end{itemize}
Solução alcoólica de menthol, salol, etc., para tratamento dos dentes e da bôca.
\section{Odometria}
\begin{itemize}
\item {Grp. gram.:f.}
\end{itemize}
Applicação do odómetro.
Arte de fabricar odómetros.
\section{Odométrico}
\begin{itemize}
\item {Grp. gram.:adj.}
\end{itemize}
Relativo á odometria.
\section{Odómetro}
\begin{itemize}
\item {Grp. gram.:m.}
\end{itemize}
\begin{itemize}
\item {Proveniência:(Do gr. \textunderscore odos\textunderscore  + \textunderscore metron\textunderscore )}
\end{itemize}
Instrumento, usado a bordo, para indicar a distância percorrida.
\section{Odonéstide}
\begin{itemize}
\item {Grp. gram.:f.}
\end{itemize}
Gênero de insectos lepidópteros nocturnos.
\section{Odontagogo}
\begin{itemize}
\item {Grp. gram.:m.}
\end{itemize}
\begin{itemize}
\item {Utilização:Med.}
\end{itemize}
\begin{itemize}
\item {Proveniência:(Do gr. \textunderscore odous\textunderscore  + \textunderscore agein\textunderscore )}
\end{itemize}
Instrumento, para extrahir dentes.
\section{Odontagra}
\begin{itemize}
\item {Grp. gram.:f.}
\end{itemize}
\begin{itemize}
\item {Utilização:Med.}
\end{itemize}
\begin{itemize}
\item {Proveniência:(Do gr. \textunderscore odous\textunderscore  + \textunderscore agra\textunderscore )}
\end{itemize}
Dôr rheumática nos dentes.
Dôr de dentes, acompanhada geralmente de inchação da face.
\section{Odontalgia}
\begin{itemize}
\item {Grp. gram.:f.}
\end{itemize}
\begin{itemize}
\item {Proveniência:(Do gr. \textunderscore odous\textunderscore , \textunderscore odontos\textunderscore  + \textunderscore algos\textunderscore )}
\end{itemize}
Dôr nos dentes.
\section{Odontálgico}
\begin{itemize}
\item {Grp. gram.:adj.}
\end{itemize}
Relativo á odontalgia.
Applicável contra a odontalgia.
\section{Odontandra}
\begin{itemize}
\item {Grp. gram.:f.}
\end{itemize}
\begin{itemize}
\item {Proveniência:(Do gr. \textunderscore odous\textunderscore  + \textunderscore aner\textunderscore , \textunderscore andros\textunderscore )}
\end{itemize}
Gênero de plantas meliáceas.
\section{Odontecnia}
\begin{itemize}
\item {Grp. gram.:f.}
\end{itemize}
\begin{itemize}
\item {Proveniência:(Do gr. \textunderscore odous\textunderscore  + \textunderscore tekhne\textunderscore )}
\end{itemize}
Arte de conservar os dentes.
\section{Odontechnia}
\begin{itemize}
\item {Grp. gram.:f.}
\end{itemize}
\begin{itemize}
\item {Proveniência:(Do gr. \textunderscore odous\textunderscore  + \textunderscore tekhne\textunderscore )}
\end{itemize}
Arte de conservar os dentes.
\section{Odonterrana}
\begin{itemize}
\item {Grp. gram.:f.}
\end{itemize}
Planta crucífera.
\section{Odonterrhana}
\begin{itemize}
\item {Grp. gram.:f.}
\end{itemize}
Planta crucífera.
\section{Odôntia}
\begin{itemize}
\item {Grp. gram.:f.}
\end{itemize}
Gênero de insectos lepidópteros nocturnos.
\section{Odontíase}
\begin{itemize}
\item {Grp. gram.:f.}
\end{itemize}
\begin{itemize}
\item {Utilização:Med.}
\end{itemize}
\begin{itemize}
\item {Proveniência:(Gr. \textunderscore odontiasis\textunderscore )}
\end{itemize}
Conjunto de phenómenos, produzidos pelo desenvolvimento dos germes dentários.
\section{Odontina}
\begin{itemize}
\item {Grp. gram.:f.}
\end{itemize}
\begin{itemize}
\item {Proveniência:(Do gr. \textunderscore odous\textunderscore , \textunderscore odontos\textunderscore )}
\end{itemize}
Opiato para limpeza de dentes.
Qualquer medicamento dentífrico.
\section{Odontite}
\begin{itemize}
\item {Grp. gram.:f.}
\end{itemize}
\begin{itemize}
\item {Proveniência:(Do gr. \textunderscore odous\textunderscore , \textunderscore odontos\textunderscore )}
\end{itemize}
Inflammação da polpa dental.
\section{Odontito}
\begin{itemize}
\item {Grp. gram.:m.}
\end{itemize}
Gênero de plantas escrofularíneas.
\section{Odontóbio}
\begin{itemize}
\item {Grp. gram.:m.}
\end{itemize}
\begin{itemize}
\item {Proveniência:(Do gr. \textunderscore odous\textunderscore  + \textunderscore bios\textunderscore )}
\end{itemize}
Gênero de helminthos, pouco conhecidos, cuja única espécie foi encontrada nas barbas da baleia.
\section{Odontocarfa}
\begin{itemize}
\item {Grp. gram.:f.}
\end{itemize}
Gênero de plantas, da fam. das compostas.
\section{Odontócaro}
\begin{itemize}
\item {Grp. gram.:m.}
\end{itemize}
Gênero de insectos coleópteros pentâmeros.
\section{Odontocarpha}
\begin{itemize}
\item {Grp. gram.:f.}
\end{itemize}
Gênero de plantas, da fam. das compostas.
\section{Odontócero}
\begin{itemize}
\item {Grp. gram.:m.}
\end{itemize}
\begin{itemize}
\item {Proveniência:(Do gr. \textunderscore odous\textunderscore  + \textunderscore keras\textunderscore )}
\end{itemize}
Gênero de insectos longicórneos.
\section{Odontocorino}
\begin{itemize}
\item {Grp. gram.:m.}
\end{itemize}
Gênero de insectos coleópteros tetrâmeros.
\section{Odontocoryno}
\begin{itemize}
\item {Grp. gram.:m.}
\end{itemize}
Gênero de insectos coleópteros tetrâmeros.
\section{Odontodermos}
\begin{itemize}
\item {Grp. gram.:m. pl.}
\end{itemize}
\begin{itemize}
\item {Proveniência:(Do gr. \textunderscore odous\textunderscore  + \textunderscore derma\textunderscore )}
\end{itemize}
Classe de cogumelos, cuja cabeça é denteada superiormente.
\section{Odontódero}
\begin{itemize}
\item {Grp. gram.:m.}
\end{itemize}
Gênero de insectos da Nova-Holanda.
\section{Odontogenia}
\begin{itemize}
\item {Grp. gram.:f.}
\end{itemize}
\begin{itemize}
\item {Proveniência:(Do gr. \textunderscore odous\textunderscore  + \textunderscore genea\textunderscore )}
\end{itemize}
Desenvolvimento ou formação dos dentes.
Parte da Physiologia, que trata da maneira por que se desenvolvem os dentes.
\section{Odontoglosso}
\begin{itemize}
\item {Grp. gram.:m.}
\end{itemize}
\begin{itemize}
\item {Proveniência:(Do gr. \textunderscore odous\textunderscore  + \textunderscore glossa\textunderscore )}
\end{itemize}
Gênero de orchídeas.
\section{Odontognatho}
\begin{itemize}
\item {Grp. gram.:m.}
\end{itemize}
\begin{itemize}
\item {Proveniência:(Do gr. \textunderscore odous\textunderscore  + \textunderscore gnathes\textunderscore )}
\end{itemize}
Gênero de peixes malacopterýgios.
\section{Odontognato}
\begin{itemize}
\item {Grp. gram.:m.}
\end{itemize}
\begin{itemize}
\item {Proveniência:(Do gr. \textunderscore odous\textunderscore  + \textunderscore gnathes\textunderscore )}
\end{itemize}
Gênero de peixes malacopterígios.
\section{Odontografia}
\begin{itemize}
\item {Grp. gram.:f.}
\end{itemize}
\begin{itemize}
\item {Proveniência:(Do gr. \textunderscore odous\textunderscore , \textunderscore odontos\textunderscore  + \textunderscore graphein\textunderscore )}
\end{itemize}
Tratado á cêrca dos dentes.
\section{Odontográfico}
\begin{itemize}
\item {Grp. gram.:adj.}
\end{itemize}
Relativo á odontografia.
\section{Odontographia}
\begin{itemize}
\item {Grp. gram.:f.}
\end{itemize}
\begin{itemize}
\item {Proveniência:(Do gr. \textunderscore odous\textunderscore , \textunderscore odontos\textunderscore  + \textunderscore graphein\textunderscore )}
\end{itemize}
Tratado á cêrca dos dentes.
\section{Odontográphico}
\begin{itemize}
\item {Grp. gram.:adj.}
\end{itemize}
Relativo á odontographia.
\section{Odontóide}
\begin{itemize}
\item {Grp. gram.:adj.}
\end{itemize}
\begin{itemize}
\item {Proveniência:(Do gr. \textunderscore odous\textunderscore , \textunderscore odontos\textunderscore  + \textunderscore eidos\textunderscore )}
\end{itemize}
Que tem fórma de dente.
\section{Odontoídeo}
\begin{itemize}
\item {Grp. gram.:adj.}
\end{itemize}
O mesmo que \textunderscore odontóide\textunderscore .
\section{Odontólita}
\begin{itemize}
\item {Grp. gram.:f.}
\end{itemize}
\begin{itemize}
\item {Proveniência:(Do gr. \textunderscore odous\textunderscore , \textunderscore odontos\textunderscore  + \textunderscore lithos\textunderscore )}
\end{itemize}
Sarro ou pedra dos dentes.
\section{Odontolitíase}
\begin{itemize}
\item {Grp. gram.:f.}
\end{itemize}
\begin{itemize}
\item {Proveniência:(Do gr. \textunderscore odous\textunderscore  + \textunderscore lithos\textunderscore )}
\end{itemize}
Formação do tártaro ou pedra dos dentes.
\section{Odontólitha}
\begin{itemize}
\item {Grp. gram.:f.}
\end{itemize}
\begin{itemize}
\item {Proveniência:(Do gr. \textunderscore odous\textunderscore , \textunderscore odontos\textunderscore  + \textunderscore lithos\textunderscore )}
\end{itemize}
Sarro ou pedra dos dentes.
\section{Odontolithíase}
\begin{itemize}
\item {Grp. gram.:f.}
\end{itemize}
\begin{itemize}
\item {Proveniência:(Do gr. \textunderscore odous\textunderscore  + \textunderscore lithos\textunderscore )}
\end{itemize}
Formação do tártaro ou pedra dos dentes.
\section{Odontólofo}
\begin{itemize}
\item {Grp. gram.:m.}
\end{itemize}
Gênero de plantas, da fam. das compostas.
\section{Odontologia}
\begin{itemize}
\item {Grp. gram.:f.}
\end{itemize}
\begin{itemize}
\item {Proveniência:(Do gr. \textunderscore odous\textunderscore , \textunderscore odontos\textunderscore  + \textunderscore logos\textunderscore )}
\end{itemize}
O mesmo que \textunderscore odontographia\textunderscore .
Tratado das doenças e hygiene dos dentes.
\section{Odontológico}
\begin{itemize}
\item {Grp. gram.:adj.}
\end{itemize}
Relativo á odontologia.
\section{Odontologista}
\begin{itemize}
\item {Grp. gram.:m.  e  f.}
\end{itemize}
Pessôa, que se occupa de odontologia.
\section{Odontoloma}
\begin{itemize}
\item {Grp. gram.:f.}
\end{itemize}
Gênero de plantas, da fam. das compostas.
\section{Odontólopho}
\begin{itemize}
\item {Grp. gram.:m.}
\end{itemize}
Gênero de plantas, da fam. das compostas.
\section{Odontoma}
\begin{itemize}
\item {Grp. gram.:m.}
\end{itemize}
\begin{itemize}
\item {Proveniência:(Do gr. \textunderscore odous\textunderscore , \textunderscore odontos\textunderscore )}
\end{itemize}
Tumor, coberto de uma camada de esmalte e produzido sôbre o lado de um dente.
\section{Odontómaco}
\begin{itemize}
\item {Grp. gram.:m.}
\end{itemize}
\begin{itemize}
\item {Proveniência:(Do gr. \textunderscore odous\textunderscore  + \textunderscore makhein\textunderscore )}
\end{itemize}
Gênero de insectos coleópteros.
Gênero de insectos hymenópteros.
\section{Odontómia}
\begin{itemize}
\item {Grp. gram.:f.}
\end{itemize}
Gênero de insectos dípteros.
\section{Odôntomo}
\begin{itemize}
\item {Grp. gram.:m.}
\end{itemize}
Gênero de insectos coleópteros pentâmeros.
\section{Odontómya}
\begin{itemize}
\item {Grp. gram.:f.}
\end{itemize}
Gênero de insectos dípteros.
\section{Odontónice}
\begin{itemize}
\item {Grp. gram.:m.}
\end{itemize}
Gênero de insectos coleópteros pentâmeros.
\section{Odontónyce}
\begin{itemize}
\item {Grp. gram.:m.}
\end{itemize}
Gênero de insectos coleópteros pentâmeros.
\section{Odontópetra}
\begin{itemize}
\item {Grp. gram.:f.}
\end{itemize}
Dente fóssil de peixes de vários gêneros.
\section{Odôntopo}
\begin{itemize}
\item {Grp. gram.:m.}
\end{itemize}
\begin{itemize}
\item {Proveniência:(Do gr. \textunderscore odous\textunderscore  + \textunderscore pous\textunderscore )}
\end{itemize}
Gênero de insectos coleópteros heterómeros.
\section{Odontoptéride}
\begin{itemize}
\item {Grp. gram.:f.}
\end{itemize}
\begin{itemize}
\item {Proveniência:(Do gr. \textunderscore odous\textunderscore  + \textunderscore pteron\textunderscore )}
\end{itemize}
Gênero de fêtos fósseis.
\section{Odontoramphos}
\begin{itemize}
\item {fónica:ram}
\end{itemize}
\begin{itemize}
\item {Grp. gram.:m. pl.}
\end{itemize}
\begin{itemize}
\item {Proveniência:(Do gr. \textunderscore odous\textunderscore  + \textunderscore ramphos\textunderscore )}
\end{itemize}
Família de aves, de mandíbulas denteadas na borda.
\section{Odontorranfos}
\begin{itemize}
\item {Grp. gram.:m. pl.}
\end{itemize}
\begin{itemize}
\item {Proveniência:(Do gr. \textunderscore odous\textunderscore  + \textunderscore ramphos\textunderscore )}
\end{itemize}
Família de aves, de mandíbulas denteadas na borda.
\section{Odontorina}
\begin{itemize}
\item {Grp. gram.:f.}
\end{itemize}
Gênero de insectos coleópteros pentâmeros.
\section{Odontorragia}
\begin{itemize}
\item {Grp. gram.:f.}
\end{itemize}
\begin{itemize}
\item {Proveniência:(Do gr. \textunderscore odous\textunderscore , \textunderscore odontos\textunderscore  + \textunderscore rhegnumi\textunderscore )}
\end{itemize}
Hemorrhagia no alvéolo de um dente.
\section{Odontorrhagia}
\begin{itemize}
\item {Grp. gram.:f.}
\end{itemize}
\begin{itemize}
\item {Proveniência:(Do gr. \textunderscore odous\textunderscore , \textunderscore odontos\textunderscore  + \textunderscore rhegnumi\textunderscore )}
\end{itemize}
Hemorrhagia no alvéolo de um dente.
\section{Odontoscélide}
\begin{itemize}
\item {Grp. gram.:f.}
\end{itemize}
Gênero de insectos coleópteros da América.
\section{Odontose}
\begin{itemize}
\item {Grp. gram.:f.}
\end{itemize}
\begin{itemize}
\item {Utilização:Med.}
\end{itemize}
\begin{itemize}
\item {Proveniência:(Do gr. \textunderscore odous\textunderscore , \textunderscore odontos\textunderscore )}
\end{itemize}
O mesmo que \textunderscore dentição\textunderscore .
\section{Odontóstomo}
\begin{itemize}
\item {Grp. gram.:adj.}
\end{itemize}
\begin{itemize}
\item {Utilização:Zool.}
\end{itemize}
\begin{itemize}
\item {Proveniência:(Do gr. \textunderscore odous\textunderscore  + \textunderscore stoma\textunderscore )}
\end{itemize}
Diz-se dos molluscos, que têm a bôca denteada.
\section{Odontota}
\begin{itemize}
\item {Grp. gram.:f.}
\end{itemize}
Gênero de insectos coleópteros da América.
\section{Odontotechnia}
\begin{itemize}
\item {Grp. gram.:f.}
\end{itemize}
\begin{itemize}
\item {Proveniência:(Do gr. \textunderscore odous\textunderscore , \textunderscore odontos\textunderscore  + \textunderscore tekhne\textunderscore )}
\end{itemize}
Arte de dentista.
\section{Odontotéchnico}
\begin{itemize}
\item {Grp. gram.:adj.}
\end{itemize}
Relativo á odontotechnia.
\section{Odontotecnia}
\begin{itemize}
\item {Grp. gram.:f.}
\end{itemize}
\begin{itemize}
\item {Proveniência:(Do gr. \textunderscore odous\textunderscore , \textunderscore odontos\textunderscore  + \textunderscore tekhne\textunderscore )}
\end{itemize}
Arte de dentista.
\section{Odontotécnico}
\begin{itemize}
\item {Grp. gram.:adj.}
\end{itemize}
Relativo á odontotecnia.
\section{Odontótricho}
\begin{itemize}
\item {fónica:co}
\end{itemize}
\begin{itemize}
\item {Grp. gram.:m.}
\end{itemize}
\begin{itemize}
\item {Proveniência:(Do gr. \textunderscore odous\textunderscore , \textunderscore trikhx\textunderscore )}
\end{itemize}
Gênero de plantas, da fam. das compostas.
\section{Odontótrico}
\begin{itemize}
\item {Grp. gram.:m.}
\end{itemize}
\begin{itemize}
\item {Proveniência:(Do gr. \textunderscore odous\textunderscore , \textunderscore trikhx\textunderscore )}
\end{itemize}
Gênero de plantas, da fam. das compostas.
\section{Odontoxesta}
\begin{itemize}
\item {Grp. gram.:m.}
\end{itemize}
O mesmo ou melhor que \textunderscore odontoxesto\textunderscore .
\section{Odontoxesto}
\begin{itemize}
\item {Grp. gram.:m.}
\end{itemize}
\begin{itemize}
\item {Utilização:Cir.}
\end{itemize}
\begin{itemize}
\item {Proveniência:(Gr. \textunderscore odontoxestes\textunderscore )}
\end{itemize}
Instrumento, para tirar a cárie dos dentes.
\section{Odôntria}
\begin{itemize}
\item {Grp. gram.:f.}
\end{itemize}
Gênero de insectos lamellicórneos.
\section{Odor}
\begin{itemize}
\item {Grp. gram.:m.}
\end{itemize}
\begin{itemize}
\item {Proveniência:(Lat. \textunderscore odor\textunderscore )}
\end{itemize}
Cheiro; perfume; aroma.
\section{Odorante}
\begin{itemize}
\item {Grp. gram.:adj.}
\end{itemize}
\begin{itemize}
\item {Proveniência:(Lat. \textunderscore odorans\textunderscore )}
\end{itemize}
O mesmo que \textunderscore odorífero\textunderscore .
\section{Odorar}
\begin{itemize}
\item {Grp. gram.:v. i.}
\end{itemize}
\begin{itemize}
\item {Utilização:Ant.}
\end{itemize}
\begin{itemize}
\item {Proveniência:(Lat. \textunderscore odorare\textunderscore )}
\end{itemize}
Exhalar odor ou cheiro.
Têr aroma. Cf. Frei Fortun., \textunderscore Inéditos\textunderscore , 311.
\section{Odórico}
\begin{itemize}
\item {Grp. gram.:adj.}
\end{itemize}
\begin{itemize}
\item {Utilização:Chím.}
\end{itemize}
Diz-se dos saes, que têm a odorina por base.
\section{Odorífero}
\begin{itemize}
\item {Grp. gram.:adj.}
\end{itemize}
\begin{itemize}
\item {Proveniência:(Lat. \textunderscore odorifer\textunderscore )}
\end{itemize}
Que exhala cheiro.
Que exhala bom cheiro; aromático.
\section{Odorífico}
\begin{itemize}
\item {Grp. gram.:adj.}
\end{itemize}
\begin{itemize}
\item {Proveniência:(Do lat. \textunderscore odor\textunderscore  + \textunderscore facere\textunderscore )}
\end{itemize}
O mesmo que \textunderscore odorífero\textunderscore .
\section{Odorifumante}
\begin{itemize}
\item {Grp. gram.:adj.}
\end{itemize}
\begin{itemize}
\item {Utilização:Poét.}
\end{itemize}
\begin{itemize}
\item {Proveniência:(De \textunderscore odor\textunderscore  + \textunderscore fumo\textunderscore )}
\end{itemize}
Que exhala fumo cheiroso.
\section{Odorina}
\begin{itemize}
\item {Grp. gram.:f.}
\end{itemize}
\begin{itemize}
\item {Utilização:Chím.}
\end{itemize}
\begin{itemize}
\item {Proveniência:(De \textunderscore odor\textunderscore )}
\end{itemize}
Base salificável, extrahida do óleo animal de Dippel.
\section{Odoroso}
\begin{itemize}
\item {Grp. gram.:adj.}
\end{itemize}
O mesmo que \textunderscore odorífero\textunderscore .
\section{Odoroscopia}
\begin{itemize}
\item {Grp. gram.:f.}
\end{itemize}
\begin{itemize}
\item {Proveniência:(Do lat. \textunderscore odor\textunderscore  + gr. \textunderscore skopein\textunderscore )}
\end{itemize}
Processo para avaliar as emanações odoríferas.
\section{Odoroscópico}
\begin{itemize}
\item {Grp. gram.:adj.}
\end{itemize}
Relativo á odoroscopia.
\section{Odoróscopo}
\begin{itemize}
\item {Grp. gram.:adj.}
\end{itemize}
Que serve para conhecer ou apreciar os cheiros.
\section{Odrada}
\begin{itemize}
\item {Grp. gram.:f.}
\end{itemize}
\begin{itemize}
\item {Utilização:Prov.}
\end{itemize}
\begin{itemize}
\item {Utilização:trasm.}
\end{itemize}
\begin{itemize}
\item {Proveniência:(De \textunderscore odre\textunderscore )}
\end{itemize}
Pancada com o corpo no chão.
\section{Odraria}
\begin{itemize}
\item {Grp. gram.:f.}
\end{itemize}
\begin{itemize}
\item {Proveniência:(De \textunderscore odre\textunderscore )}
\end{itemize}
Loja ou officina de odreiro.
\section{Odre}
\begin{itemize}
\item {fónica:ô}
\end{itemize}
\begin{itemize}
\item {Grp. gram.:m.}
\end{itemize}
\begin{itemize}
\item {Utilização:Pesc.}
\end{itemize}
\begin{itemize}
\item {Utilização:Fig.}
\end{itemize}
\begin{itemize}
\item {Utilização:Deprec.}
\end{itemize}
\begin{itemize}
\item {Proveniência:(Do lat. \textunderscore uter\textunderscore )}
\end{itemize}
Vasilha de coiro ou de pelles, geralmente da pelle de animaes caprídeos, e destinada a transportar líquidos.
Fluctuador de cordas, nos apparelhos de arrastar para terra.
Pessôa gorda.
Pessôa, que se embriaga.
Indivíduo, que tem certos defeitos em grande quantidade:«\textunderscore um odre de vaidade\textunderscore ».
\section{Odreiro}
\begin{itemize}
\item {Grp. gram.:m.}
\end{itemize}
Fabricante ou vendedor de odres.
\section{Odulço}
\begin{itemize}
\item {Grp. gram.:m.}
\end{itemize}
Árvore da Índia portuguesa.
\section{Odyssaico}
\begin{itemize}
\item {Grp. gram.:adj.}
\end{itemize}
Relativo ao cyclo histórico das odysseias. Cf. Th. Braga, \textunderscore Mod. Ideias\textunderscore , 232.
\section{Odysseia}
\begin{itemize}
\item {Grp. gram.:f.}
\end{itemize}
\begin{itemize}
\item {Utilização:Fig.}
\end{itemize}
\begin{itemize}
\item {Proveniência:(Lat. \textunderscore Odysseia\textunderscore )}
\end{itemize}
Viagem, cheia de aventuras extraordinárias.
Qualquer narração de aventuras extraordinárias.
\section{Oé!}
\begin{itemize}
\item {Grp. gram.:interj.}
\end{itemize}
\begin{itemize}
\item {Utilização:Bras}
\end{itemize}
Oh!
\section{Oêna}
\begin{itemize}
\item {Grp. gram.:f.}
\end{itemize}
Espécie de pombo bravo.
\section{Oéste}
\begin{itemize}
\item {Grp. gram.:m.}
\end{itemize}
\begin{itemize}
\item {Grp. gram.:Adj.}
\end{itemize}
\begin{itemize}
\item {Proveniência:(Do germ. \textunderscore west\textunderscore )}
\end{itemize}
Lado do horizonte, onde o Sol desapparece.
Poente; Occidente.
Vento que sopra dêsse lado.
Um dos pontos cardeaes, que nos fica á esquerda, quando nos voltamos para o Norte.
Regiões, que ficam do lado do Poente.
Relativo ao Poente.
Que sopra do lado do Poente:«\textunderscore em dois dias de ventos oestes...\textunderscore »\textunderscore Peregrinação\textunderscore , XL.
\section{Oéta}
\begin{itemize}
\item {Grp. gram.:f.}
\end{itemize}
\begin{itemize}
\item {Utilização:Ant.}
\end{itemize}
Vestuário de homem, espécie de meia túnica.
\section{Ofegada}
\begin{itemize}
\item {Grp. gram.:f.}
\end{itemize}
\begin{itemize}
\item {Utilização:Prov.}
\end{itemize}
\begin{itemize}
\item {Utilização:minh.}
\end{itemize}
Acto de ofegar.
Ofêgo. (Colhido em Barcelos)
\section{Ofegante}
\begin{itemize}
\item {Grp. gram.:adj.}
\end{itemize}
\begin{itemize}
\item {Proveniência:(Do lat. \textunderscore offocans\textunderscore )}
\end{itemize}
O mesmo que \textunderscore ofegoso\textunderscore .
\section{Ofegar}
\begin{itemize}
\item {Grp. gram.:v. i.}
\end{itemize}
\begin{itemize}
\item {Proveniência:(Lat. \textunderscore offocare\textunderscore )}
\end{itemize}
Respirar com dificuldade, ou com ruído produzido pelo cansaço.
\section{Ofêgo}
\begin{itemize}
\item {Grp. gram.:m.}
\end{itemize}
\begin{itemize}
\item {Proveniência:(De \textunderscore ofegar\textunderscore )}
\end{itemize}
Respiração ruidosa ou difícil; cansaço.
\section{Ofegoso}
\begin{itemize}
\item {Grp. gram.:adj.}
\end{itemize}
\begin{itemize}
\item {Utilização:Fig.}
\end{itemize}
\begin{itemize}
\item {Proveniência:(De \textunderscore ofegar\textunderscore )}
\end{itemize}
Que está ofegando.
Ansioso, ávido.
\section{Ofeguento}
\begin{itemize}
\item {Grp. gram.:adj.}
\end{itemize}
O mesmo que \textunderscore ofegoso\textunderscore .
\section{Ofendedor}
\begin{itemize}
\item {Grp. gram.:m.  e  adj.}
\end{itemize}
O que ofende.
\section{Ofender}
\begin{itemize}
\item {Grp. gram.:v. t.}
\end{itemize}
\begin{itemize}
\item {Proveniência:(Lat. \textunderscore offendere\textunderscore )}
\end{itemize}
Fazer mal a; lesar.
Injuriar; melindrar.
Desgostar; escandalizar.
\section{Ofendículo}
\begin{itemize}
\item {Grp. gram.:adj.}
\end{itemize}
\begin{itemize}
\item {Proveniência:(Lat. \textunderscore offendiculum\textunderscore )}
\end{itemize}
Estôrvo, impecilho.
Objecto, que faz tropeçar.
\section{Ofendido}
\begin{itemize}
\item {Grp. gram.:adj.}
\end{itemize}
\begin{itemize}
\item {Grp. gram.:M.}
\end{itemize}
\begin{itemize}
\item {Proveniência:(De \textunderscore ofender\textunderscore )}
\end{itemize}
Injuriado.
Lesado; prejudicado.
Indivíduo, a quem ofenderam: \textunderscore ouvir as queixas do ofendido\textunderscore .
\section{Ofensa}
\begin{itemize}
\item {Grp. gram.:f.}
\end{itemize}
\begin{itemize}
\item {Proveniência:(Lat. \textunderscore offensa\textunderscore )}
\end{itemize}
Acto ou efeito de ofender.
Acto de fazer mal a alguém, por actos ou palavras.
Desacato, que se faz á divindade, peccando.
Menosprêzo ou postergação de quaesquer preceitos ou regras.
Mágoa ou resentimento moral da pessôa ofendida: \textunderscore tenho muitas ofensas\textunderscore .
\section{Ofensão}
\begin{itemize}
\item {Grp. gram.:f.}
\end{itemize}
\begin{itemize}
\item {Proveniência:(Lat. \textunderscore offensio\textunderscore )}
\end{itemize}
O mesmo que \textunderscore ofensa\textunderscore .
Ataque, combate; peleja.
\section{Ofensiva}
\begin{itemize}
\item {Grp. gram.:f.}
\end{itemize}
\begin{itemize}
\item {Proveniência:(De \textunderscore ofensivo\textunderscore )}
\end{itemize}
Acto ou situação de quem ataca; iniciativa de atacar: \textunderscore um deles tomou a ofensiva\textunderscore .
\section{Ofensivamente}
\begin{itemize}
\item {Grp. gram.:adv.}
\end{itemize}
De modo ofensivo.
Por meio de ataque; aggressivamente.
\section{Ofensivo}
\begin{itemize}
\item {Grp. gram.:adj.}
\end{itemize}
\begin{itemize}
\item {Proveniência:(De \textunderscore ofenso\textunderscore )}
\end{itemize}
Que ofende; próprio para ofender ou atacar; lesivo; prejudicial.
\section{Ofenso}
\begin{itemize}
\item {Grp. gram.:adj.}
\end{itemize}
\begin{itemize}
\item {Proveniência:(Lat. \textunderscore offensus\textunderscore )}
\end{itemize}
Que recebeu ofensa; que foi ofendido.
\section{Ofensor}
\begin{itemize}
\item {Grp. gram.:m.  e  adj.}
\end{itemize}
\begin{itemize}
\item {Proveniência:(Lat. \textunderscore offensor\textunderscore )}
\end{itemize}
O mesmo que \textunderscore ofendedor\textunderscore .
\section{Oferecedor}
\begin{itemize}
\item {Grp. gram.:m.  e  adj.}
\end{itemize}
O que oferece.
\section{Oferecer}
\begin{itemize}
\item {Grp. gram.:v. t.}
\end{itemize}
\begin{itemize}
\item {Proveniência:(Lat. \textunderscore offerre\textunderscore )}
\end{itemize}
Levar á presença de.
Mostrar, apresentar.
Exibir ou propor, para que seja aceito: \textunderscore oferecer serviços\textunderscore .
Propor.
Expor.
Ministrar; facultar.
Imolar; dedicar: \textunderscore os Pagãos ofereciam as víctimas aos deuses\textunderscore .
Fazer oferta de: \textunderscore oferecer um relógio\textunderscore .
\section{Oferecimento}
\begin{itemize}
\item {Grp. gram.:m.}
\end{itemize}
Acto ou efeito de oferecer.
Manifestação verbal do desejo de sêr útil ou agradável a alguém.
\section{Oferenda}
\begin{itemize}
\item {Grp. gram.:f.}
\end{itemize}
\begin{itemize}
\item {Proveniência:(Do lat. \textunderscore offerendus\textunderscore )}
\end{itemize}
Objecto, que se oferece; oblata.
\section{Oferendar}
\begin{itemize}
\item {Grp. gram.:v. t.}
\end{itemize}
\begin{itemize}
\item {Proveniência:(De \textunderscore oferenda\textunderscore )}
\end{itemize}
O mesmo que \textunderscore oblatar\textunderscore .
\section{Oferendo}
\begin{itemize}
\item {Grp. gram.:m.}
\end{itemize}
\begin{itemize}
\item {Utilização:bras}
\end{itemize}
\begin{itemize}
\item {Utilização:Neol.}
\end{itemize}
\begin{itemize}
\item {Proveniência:(Lat. \textunderscore offerendus\textunderscore )}
\end{itemize}
Homem metediço, intrometido.
\section{Oferente}
\begin{itemize}
\item {Grp. gram.:m.  e  adj.}
\end{itemize}
\begin{itemize}
\item {Proveniência:(Lat. \textunderscore offerens\textunderscore )}
\end{itemize}
O que oferece; oferecedor.
\section{Oferta}
\begin{itemize}
\item {Grp. gram.:f.}
\end{itemize}
Aquilo que se oferece.
Oferecimento; oferenda; dádiva.
Retribuição de certos actos litúrgicos.
(B. lat. \textunderscore offerta\textunderscore )
\section{Ofertamento}
\begin{itemize}
\item {Grp. gram.:m.}
\end{itemize}
Acto ou efeito de ofertar.
\section{Ofertante}
\begin{itemize}
\item {Grp. gram.:m.  e  adj.}
\end{itemize}
Aquele que oferta.
\section{Ofertar}
\begin{itemize}
\item {Grp. gram.:v. t.}
\end{itemize}
Dar como oferta; oferecer.
\section{Oferteira}
\begin{itemize}
\item {Grp. gram.:f.}
\end{itemize}
\begin{itemize}
\item {Proveniência:(De \textunderscore oferta\textunderscore )}
\end{itemize}
Mulher, que conduz fogaças ou ofertas á igreja.
\section{Ofertório}
\begin{itemize}
\item {Grp. gram.:m.}
\end{itemize}
\begin{itemize}
\item {Proveniência:(Lat. \textunderscore offertorium\textunderscore )}
\end{itemize}
Oração ou parte da Missa, em que se oferece a Deus a hóstia e o vinho.
Acto de angariar ofertas para festas de igreja.
\section{Ofeso}
\begin{itemize}
\item {Grp. gram.:adj.}
\end{itemize}
\begin{itemize}
\item {Utilização:Ant.}
\end{itemize}
\begin{itemize}
\item {Proveniência:(Do lat. \textunderscore offensus\textunderscore )}
\end{itemize}
O mesmo que \textunderscore ofendido\textunderscore . Cf. Castilho, \textunderscore Fastos\textunderscore , 11, 65, 67 e 111.
\section{Offegada}
\begin{itemize}
\item {Grp. gram.:f.}
\end{itemize}
\begin{itemize}
\item {Utilização:Prov.}
\end{itemize}
\begin{itemize}
\item {Utilização:minh.}
\end{itemize}
Acto de offegar.
Offêgo.
(Colhido em Barcelos)
\section{Offegante}
\begin{itemize}
\item {Grp. gram.:adj.}
\end{itemize}
\begin{itemize}
\item {Proveniência:(Do lat. \textunderscore offocans\textunderscore )}
\end{itemize}
O mesmo que \textunderscore offegoso\textunderscore .
\section{Offegar}
\begin{itemize}
\item {Grp. gram.:v. i.}
\end{itemize}
\begin{itemize}
\item {Proveniência:(Lat. \textunderscore offocare\textunderscore )}
\end{itemize}
Respirar com difficuldade, ou com ruído produzido pelo cansaço.
\section{Offêgo}
\begin{itemize}
\item {Grp. gram.:m.}
\end{itemize}
\begin{itemize}
\item {Proveniência:(De \textunderscore offegar\textunderscore )}
\end{itemize}
Respiração ruidosa ou diffícil; cansaço.
\section{Offegoso}
\begin{itemize}
\item {Grp. gram.:adj.}
\end{itemize}
\begin{itemize}
\item {Utilização:Fig.}
\end{itemize}
\begin{itemize}
\item {Proveniência:(De \textunderscore offegar\textunderscore )}
\end{itemize}
Que está offegando.
Ansioso, ávido.
\section{Offeguento}
\begin{itemize}
\item {Grp. gram.:adj.}
\end{itemize}
O mesmo que \textunderscore offegoso\textunderscore .
\section{Offenbachesco}
\begin{itemize}
\item {fónica:quê}
\end{itemize}
\begin{itemize}
\item {Grp. gram.:adj.}
\end{itemize}
Semelhante ás músicas de Offenbach.
Extravagante, falando-se música. Cf. Camillo, \textunderscore Cancion. Alegre\textunderscore , 160.
\section{Offenbachiano}
\begin{itemize}
\item {fónica:qui}
\end{itemize}
\begin{itemize}
\item {Grp. gram.:adj.}
\end{itemize}
O mesmo que \textunderscore offenbachesco\textunderscore .
\section{Offenbachismo}
\begin{itemize}
\item {fónica:quis}
\end{itemize}
\begin{itemize}
\item {Grp. gram.:m.}
\end{itemize}
Gênero ou systema de música, comparável á de Offenbach.
\section{Offendedor}
\begin{itemize}
\item {Grp. gram.:m.  e  adj.}
\end{itemize}
O que offende.
\section{Offender}
\begin{itemize}
\item {Grp. gram.:v. t.}
\end{itemize}
\begin{itemize}
\item {Proveniência:(Lat. \textunderscore offendere\textunderscore )}
\end{itemize}
Fazer mal a; lesar.
Injuriar; melindrar.
Desgostar; escandalizar.
\section{Offendículo}
\begin{itemize}
\item {Grp. gram.:adj.}
\end{itemize}
\begin{itemize}
\item {Proveniência:(Lat. \textunderscore offendiculum\textunderscore )}
\end{itemize}
Estôrvo, impecilho.
Objecto, que faz tropeçar.
\section{Offendido}
\begin{itemize}
\item {Grp. gram.:adj.}
\end{itemize}
\begin{itemize}
\item {Grp. gram.:M.}
\end{itemize}
\begin{itemize}
\item {Proveniência:(De \textunderscore offender\textunderscore )}
\end{itemize}
Injuriado.
Lesado; prejudicado.
Indivíduo, a quem offenderam: \textunderscore ouvir as queixas do offendido\textunderscore .
\section{Offensa}
\begin{itemize}
\item {Grp. gram.:f.}
\end{itemize}
\begin{itemize}
\item {Proveniência:(Lat. \textunderscore offensa\textunderscore )}
\end{itemize}
Acto ou effeito de offender.
Acto de fazer mal a alguém, por actos ou palavras.
Desacato, que se faz á divindade, peccando.
Menosprêzo ou postergação de quaesquer preceitos ou regras.
Mágoa ou resentimento moral da pessôa offendida: \textunderscore tenho muitas offensas\textunderscore .
\section{Offensão}
\begin{itemize}
\item {Grp. gram.:f.}
\end{itemize}
\begin{itemize}
\item {Proveniência:(Lat. \textunderscore offensio\textunderscore )}
\end{itemize}
O mesmo que \textunderscore offensa\textunderscore .
Ataque, combate; peleja.
\section{Offensiva}
\begin{itemize}
\item {Grp. gram.:f.}
\end{itemize}
\begin{itemize}
\item {Proveniência:(De \textunderscore offensivo\textunderscore )}
\end{itemize}
Acto ou situação de quem ataca; iniciativa de atacar: \textunderscore um delles tomou a offensiva\textunderscore .
\section{Offensivamente}
\begin{itemize}
\item {Grp. gram.:adv.}
\end{itemize}
De modo offensivo.
Por meio de ataque; aggressivamente.
\section{Offensivo}
\begin{itemize}
\item {Grp. gram.:adj.}
\end{itemize}
\begin{itemize}
\item {Proveniência:(De \textunderscore offenso\textunderscore )}
\end{itemize}
Que offende; próprio para offender ou atacar; lesivo; prejudicial.
\section{Offenso}
\begin{itemize}
\item {Grp. gram.:adj.}
\end{itemize}
\begin{itemize}
\item {Proveniência:(Lat. \textunderscore offensus\textunderscore )}
\end{itemize}
Que recebeu offensa; que foi offendido.
\section{Offensor}
\begin{itemize}
\item {Grp. gram.:m.  e  adj.}
\end{itemize}
\begin{itemize}
\item {Proveniência:(Lat. \textunderscore offensor\textunderscore )}
\end{itemize}
O mesmo que \textunderscore offendedor\textunderscore .
\section{Offerecedor}
\begin{itemize}
\item {Grp. gram.:m.  e  adj.}
\end{itemize}
O que offerece.
\section{Offerecer}
\begin{itemize}
\item {Grp. gram.:v. t.}
\end{itemize}
\begin{itemize}
\item {Proveniência:(Lat. \textunderscore offerre\textunderscore )}
\end{itemize}
Levar á presença de.
Mostrar, apresentar.
Exhibir ou propor, para que seja acceito: \textunderscore offerecer serviços\textunderscore .
Propor.
Expor.
Ministrar; facultar.
Immolar; dedicar: \textunderscore os Pagãos offereciam as víctimas aos deuses\textunderscore .
Fazer offerta de: \textunderscore offerecer um relógio\textunderscore .
\section{Offerecimento}
\begin{itemize}
\item {Grp. gram.:m.}
\end{itemize}
Acto ou effeito de offerecer.
Manifestação verbal do desejo de sêr útil ou agradável a alguém.
\section{Offerenda}
\begin{itemize}
\item {Grp. gram.:f.}
\end{itemize}
\begin{itemize}
\item {Proveniência:(Do lat. \textunderscore offerendus\textunderscore )}
\end{itemize}
Objecto, que se offerece; oblata.
\section{Offerendar}
\begin{itemize}
\item {Grp. gram.:v. t.}
\end{itemize}
\begin{itemize}
\item {Proveniência:(De \textunderscore offerenda\textunderscore )}
\end{itemize}
O mesmo que \textunderscore oblatar\textunderscore .
\section{Offerendo}
\begin{itemize}
\item {Grp. gram.:m.}
\end{itemize}
\begin{itemize}
\item {Utilização:bras}
\end{itemize}
\begin{itemize}
\item {Utilização:Neol.}
\end{itemize}
\begin{itemize}
\item {Proveniência:(Lat. \textunderscore offerendus\textunderscore )}
\end{itemize}
Homem metediço, intrometido.
\section{Offerente}
\begin{itemize}
\item {Grp. gram.:m.  e  adj.}
\end{itemize}
\begin{itemize}
\item {Proveniência:(Lat. \textunderscore offerens\textunderscore )}
\end{itemize}
O que offerece; offerecedor.
\section{Offerta}
\begin{itemize}
\item {Grp. gram.:f.}
\end{itemize}
Aquillo que se offerece.
Offerecimento; offerenda; dádiva.
Retribuição de certos actos litúrgicos.
(B. lat. \textunderscore offerta\textunderscore )
\section{Offertamento}
\begin{itemize}
\item {Grp. gram.:m.}
\end{itemize}
Acto ou effeito de offertar.
\section{Offertante}
\begin{itemize}
\item {Grp. gram.:m.  e  adj.}
\end{itemize}
Aquelle que offerta.
\section{Offertar}
\begin{itemize}
\item {Grp. gram.:v. t.}
\end{itemize}
Dar como offerta; offerecer.
\section{Offerteira}
\begin{itemize}
\item {Grp. gram.:f.}
\end{itemize}
\begin{itemize}
\item {Proveniência:(De \textunderscore offerta\textunderscore )}
\end{itemize}
Mulher, que conduz fogaças ou offertas á igreja.
\section{Offertório}
\begin{itemize}
\item {Grp. gram.:m.}
\end{itemize}
\begin{itemize}
\item {Proveniência:(Lat. \textunderscore offertorium\textunderscore )}
\end{itemize}
Oração ou parte da Missa, em que se offerece a Deus a hóstia e o vinho.
Acto de angariar offertas para festas de igreja.
\section{Offêso}
\begin{itemize}
\item {Grp. gram.:adj.}
\end{itemize}
\begin{itemize}
\item {Utilização:Ant.}
\end{itemize}
\begin{itemize}
\item {Proveniência:(Do lat. \textunderscore offensus\textunderscore )}
\end{itemize}
O mesmo que \textunderscore offendido\textunderscore . Cf. Castilho, \textunderscore Fastos\textunderscore , 11, 65, 67 e 111.
\section{Officiador}
\begin{itemize}
\item {Grp. gram.:m.  e  adj.}
\end{itemize}
O que officia.
\section{Official}
\begin{itemize}
\item {Grp. gram.:adj.}
\end{itemize}
\begin{itemize}
\item {Grp. gram.:M.}
\end{itemize}
\begin{itemize}
\item {Proveniência:(Lat. \textunderscore officialis\textunderscore )}
\end{itemize}
Proposto por autoridade reconhecida ou emanada della: \textunderscore instrucções officiaes\textunderscore .
Relativo a autoridade ou ás pessôas que a constituem.
Relativo ao Govêrno: \textunderscore a fôlha official\textunderscore .
Burocrático: \textunderscore formalidades officiaes\textunderscore .
Indivíduo, que vive do seu offício.
Aquelle que, exercendo um offício, tem categoria inferior á de mestre: \textunderscore official de serralheiro\textunderscore .
Militar, de qualquer graduação superior á de sargento.
Funccionário que, nas repartições públicas, tem graduação superior á dos amanuenses e inferior á dos chefes.
Marinheiro militar, de graduação superior á de guarda-marinha.
Empregado inferior, judicial ou administrativo, a quem cumpre fazer citações e intimações e executar outras diligências.
\textunderscore Officiaes de justiça\textunderscore , os escrivães e contadores dos tribunaes, e ainda os chamados officiaes de diligências.
Dignitário de certas ordens honoríficas.
\textunderscore Official-mór da casa real\textunderscore , um dos empregados superiores do paço.
\textunderscore Official menor da casa real\textunderscore , um dos empregados menores do paço.
Peixe dos Açores.
\section{Officiala}
\begin{itemize}
\item {Grp. gram.:f.}
\end{itemize}
\begin{itemize}
\item {Utilização:Prov.}
\end{itemize}
\begin{itemize}
\item {Utilização:dur.}
\end{itemize}
\begin{itemize}
\item {Proveniência:(De \textunderscore official\textunderscore )}
\end{itemize}
Costureira de modista. Cf. Camillo, \textunderscore Quéda\textunderscore , 172.
\section{Officialato}
\begin{itemize}
\item {Grp. gram.:m.}
\end{itemize}
Cargo ou dignidade de official.
\section{Official-da-sala}
\begin{itemize}
\item {Grp. gram.:m.}
\end{itemize}
\begin{itemize}
\item {Utilização:Bras}
\end{itemize}
Arbusto medicinal (\textunderscore asclépias umbellata\textunderscore ).
\section{Officialidade}
\begin{itemize}
\item {Grp. gram.:f.}
\end{itemize}
Conjunto de officiaes do exército ou de um regimento.
\section{Officialmente}
\begin{itemize}
\item {Grp. gram.:adv.}
\end{itemize}
De modo official; em nome da autoridade ou do Govêrno.
\section{Officiante}
\begin{itemize}
\item {Grp. gram.:m.  e  adj.}
\end{itemize}
\begin{itemize}
\item {Grp. gram.:F.}
\end{itemize}
\begin{itemize}
\item {Proveniência:(De \textunderscore officiar\textunderscore )}
\end{itemize}
O que officia ou preside ao offício divino.
Freira, que está de semana no côro.
\section{Officiar}
\begin{itemize}
\item {Grp. gram.:v. i.}
\end{itemize}
Celebrar o offício divino.
Dirigir um offício ou communicação official a alguém.
\section{Officioso}
\begin{itemize}
\item {Grp. gram.:m.}
\end{itemize}
\begin{itemize}
\item {Utilização:Des.}
\end{itemize}
Livro litúrgico, que contém os offícios com indicações do canto.
\section{Officina}
\begin{itemize}
\item {Grp. gram.:f.}
\end{itemize}
\begin{itemize}
\item {Utilização:Fig.}
\end{itemize}
\begin{itemize}
\item {Proveniência:(Lat. \textunderscore officina\textunderscore )}
\end{itemize}
Lugar, onde se exerce um offício: \textunderscore officina de caldeireiro\textunderscore .
Laboratório.
Lugar, onde se guardam os utensílios de uma indústria ou arte.
Dependência das igrejas e de outros edifícios, destinada a refeitório, dispensa, cozinha, etc.
Lugar, em que há grandes transformações.
\section{Officinal}
\begin{itemize}
\item {Grp. gram.:adj.}
\end{itemize}
\begin{itemize}
\item {Proveniência:(De \textunderscore officina\textunderscore )}
\end{itemize}
Relativo a preparações pharmacêuticas.
Que se applica em pharmácia.
\section{Offício}
\begin{itemize}
\item {Grp. gram.:m.}
\end{itemize}
\begin{itemize}
\item {Grp. gram.:Pl.}
\end{itemize}
\begin{itemize}
\item {Proveniência:(Lat. \textunderscore officium\textunderscore )}
\end{itemize}
Dever; obrigação natural.
Incumbência.
Destino especial.
Cargo pessoal; profissão: \textunderscore offício de chapeleiro\textunderscore .
Occupação.
Alcofa para ferramentas de sapateiro.
Conjunto de ceremónias de uma festa religiosa: \textunderscore assistir aos offícios divinos\textunderscore .
Communicação escrita, de origem official, em fórma de carta e em matéria de serviço público: \textunderscore escrever offícios, como um amanuense\textunderscore .
\textunderscore Offício divino\textunderscore , a Missa.
Diligência, intervenção: \textunderscore agradeço-lhe os seus bons offícios\textunderscore .
Jôgo popular.
\section{Officiosamente}
\begin{itemize}
\item {Grp. gram.:adv.}
\end{itemize}
De modo officioso.
Extra-officialmente; sem obrigação; por favor.
\section{Officiosidade}
\begin{itemize}
\item {Grp. gram.:f.}
\end{itemize}
\begin{itemize}
\item {Proveniência:(Lat. \textunderscore officiositas\textunderscore )}
\end{itemize}
Qualidade do que é officioso.
\section{Officioso}
\begin{itemize}
\item {Grp. gram.:adj.}
\end{itemize}
\begin{itemize}
\item {Utilização:Jur.}
\end{itemize}
\begin{itemize}
\item {Proveniência:(Lat. \textunderscore officiosus\textunderscore )}
\end{itemize}
Serviçal.
Em que há vontade de sêr agradável ou prestável.
Desinteressado.
Que não tem o carácter de official ou dependente da autoridade.
Gratuito.
Inoffensivo.
Advogado \textunderscore officioso\textunderscore , aquelle que não tem procuração do réu para que o defenda, e a quem o presidente do tribunal incumbe a defeza.
\section{Offreção}
\begin{itemize}
\item {Grp. gram.:f.}
\end{itemize}
\begin{itemize}
\item {Utilização:Ant.}
\end{itemize}
Offerenda.
Luvas ou peita, com que se obtinha dos officiaes da Corôa a concessão de casaes ermos. Cf. Herculano, \textunderscore Hist. de Port.\textunderscore , III, 357.
(Por \textunderscore offerção\textunderscore , do b. lat. \textunderscore offertio\textunderscore , do lat. \textunderscore offere\textunderscore )
\section{Offrenda}
\begin{itemize}
\item {Grp. gram.:f.}
\end{itemize}
O mesmo que \textunderscore offerenda\textunderscore .
\section{Offuscação}
\begin{itemize}
\item {Grp. gram.:f.}
\end{itemize}
\begin{itemize}
\item {Proveniência:(Lat. \textunderscore offuscatio\textunderscore )}
\end{itemize}
Acto ou effeito de offuscar.
Obscurecimento.
\section{Offuscado}
\begin{itemize}
\item {Grp. gram.:adj.}
\end{itemize}
\begin{itemize}
\item {Proveniência:(De \textunderscore offuscar\textunderscore )}
\end{itemize}
Escurecido.
Supplantado.
\section{Offuscamento}
\begin{itemize}
\item {Grp. gram.:m.}
\end{itemize}
O mesmo que \textunderscore offuscação\textunderscore .
\section{Offuscar}
\begin{itemize}
\item {Grp. gram.:v. t.}
\end{itemize}
\begin{itemize}
\item {Utilização:Fig.}
\end{itemize}
\begin{itemize}
\item {Proveniência:(Lat. \textunderscore offuscare\textunderscore )}
\end{itemize}
Tornar escuro; encobrir; deslumbrar; enturvar.
Obcecar; turvar a intelligência de.
Tornar menos intenso.
Tirar o prestígio a: \textunderscore êste orador offuscaria Cícero\textunderscore .
\section{Oficiador}
\begin{itemize}
\item {Grp. gram.:m.  e  adj.}
\end{itemize}
O que oficia.
\section{Oficial}
\begin{itemize}
\item {Grp. gram.:adj.}
\end{itemize}
\begin{itemize}
\item {Grp. gram.:M.}
\end{itemize}
\begin{itemize}
\item {Proveniência:(Lat. \textunderscore officialis\textunderscore )}
\end{itemize}
Proposto por autoridade reconhecida ou emanada dela: \textunderscore instruções oficiaes\textunderscore .
Relativo a autoridade ou ás pessôas que a constituem.
Relativo ao Govêrno: \textunderscore a fôlha oficial\textunderscore .
Burocrático: \textunderscore formalidades oficiaes\textunderscore .
Indivíduo, que vive do seu ofício.
Aquele que, exercendo um ofício, tem categoria inferior á de mestre: \textunderscore oficial de serralheiro\textunderscore .
Militar, de qualquer graduação superior á de sargento.
Funcionário que, nas repartições públicas, tem graduação superior á dos amanuenses e inferior á dos chefes.
Marinheiro militar, de graduação superior á de guarda-marinha.
Empregado inferior, judicial ou administrativo, a quem cumpre fazer citações e intimações e executar outras diligências.
\textunderscore Oficiaes de justiça\textunderscore , os escrivães e contadores dos tribunaes, e ainda os chamados oficiaes de diligências.
Dignitário de certas ordens honoríficas.
\textunderscore Oficial-mór da casa real\textunderscore , um dos empregados superiores do paço.
\textunderscore Oficial menor da casa real\textunderscore , um dos empregados menores do paço.
Peixe dos Açores.
\section{Oficiala}
\begin{itemize}
\item {Grp. gram.:f.}
\end{itemize}
\begin{itemize}
\item {Utilização:Prov.}
\end{itemize}
\begin{itemize}
\item {Utilização:dur.}
\end{itemize}
\begin{itemize}
\item {Proveniência:(De \textunderscore oficial\textunderscore )}
\end{itemize}
Costureira de modista. Cf. Camillo, \textunderscore Quéda\textunderscore , 172.
\section{Oficialato}
\begin{itemize}
\item {Grp. gram.:m.}
\end{itemize}
Cargo ou dignidade de oficial.
\section{Oficialidade}
\begin{itemize}
\item {Grp. gram.:f.}
\end{itemize}
Conjunto de oficiaes do exército ou de um regimento.
\section{Oficialmente}
\begin{itemize}
\item {Grp. gram.:adv.}
\end{itemize}
De modo oficial; em nome da autoridade ou do Govêrno.
\section{Oficiante}
\begin{itemize}
\item {Grp. gram.:m.  e  adj.}
\end{itemize}
\begin{itemize}
\item {Grp. gram.:F.}
\end{itemize}
\begin{itemize}
\item {Proveniência:(De \textunderscore oficiar\textunderscore )}
\end{itemize}
O que oficia ou preside ao ofício divino.
Freira, que está de semana no côro.
\section{Oficiar}
\begin{itemize}
\item {Grp. gram.:v. i.}
\end{itemize}
Celebrar o ofício divino.
Dirigir um ofício ou comunicação oficial a alguém.
\section{Oficioso}
\begin{itemize}
\item {Grp. gram.:m.}
\end{itemize}
\begin{itemize}
\item {Utilização:Des.}
\end{itemize}
Livro litúrgico, que contém os ofícios com indicações do canto.
\section{Oficina}
\begin{itemize}
\item {Grp. gram.:f.}
\end{itemize}
\begin{itemize}
\item {Utilização:Fig.}
\end{itemize}
\begin{itemize}
\item {Proveniência:(Lat. \textunderscore officina\textunderscore )}
\end{itemize}
Lugar, onde se exerce um ofício: \textunderscore oficina de caldeireiro\textunderscore .
Laboratório.
Lugar, onde se guardam os utensílios de uma indústria ou arte.
Dependência das igrejas e de outros edifícios, destinada a refeitório, dispensa, cozinha, etc.
Lugar, em que há grandes transformações.
\section{Oficinal}
\begin{itemize}
\item {Grp. gram.:adj.}
\end{itemize}
\begin{itemize}
\item {Proveniência:(De \textunderscore oficina\textunderscore )}
\end{itemize}
Relativo a preparações farmacêuticas.
Que se aplica em farmácia.
\section{Ofício}
\begin{itemize}
\item {Grp. gram.:m.}
\end{itemize}
\begin{itemize}
\item {Grp. gram.:Pl.}
\end{itemize}
\begin{itemize}
\item {Proveniência:(Lat. \textunderscore officium\textunderscore )}
\end{itemize}
Dever; obrigação natural.
Incumbência.
Destino especial.
Cargo pessoal; profissão: \textunderscore ofício de chapeleiro\textunderscore .
Occupação.
Alcofa para ferramentas de sapateiro.
Conjunto de ceremónias de uma festa religiosa: \textunderscore assistir aos ofícios divinos\textunderscore .
Communicação escrita, de origem oficial, em fórma de carta e em matéria de serviço público: \textunderscore escrever ofícios, como um amanuense\textunderscore .
\textunderscore Ofício divino\textunderscore , a Missa.
Diligência, intervenção: \textunderscore agradeço-lhe os seus bons ofícios\textunderscore .
Jôgo popular.
\section{Oficiosamente}
\begin{itemize}
\item {Grp. gram.:adv.}
\end{itemize}
De modo oficioso.
Extra-oficialmente; sem obrigação; por favor.
\section{Oficiosidade}
\begin{itemize}
\item {Grp. gram.:f.}
\end{itemize}
\begin{itemize}
\item {Proveniência:(Lat. \textunderscore officiositas\textunderscore )}
\end{itemize}
Qualidade do que é oficioso.
\section{Oficioso}
\begin{itemize}
\item {Grp. gram.:adj.}
\end{itemize}
\begin{itemize}
\item {Utilização:Jur.}
\end{itemize}
\begin{itemize}
\item {Proveniência:(Lat. \textunderscore officiosus\textunderscore )}
\end{itemize}
Serviçal.
Em que há vontade de sêr agradável ou prestável.
Desinteressado.
Que não tem o carácter de oficial ou dependente da autoridade.
Gratuito.
Inoffensivo.
Advogado \textunderscore oficioso\textunderscore , aquele que não tem procuração do réu para que o defenda, e a quem o presidente do tribunal incumbe a defeza.
\section{Ofiranganga}
\begin{itemize}
\item {Grp. gram.:f.}
\end{itemize}
Árvore angolense de Caconda.
\section{Ofó}
\begin{itemize}
\item {Grp. gram.:m.}
\end{itemize}
Tubérculo venenoso, semelhante ao cuini, na ilha de San-Thomé.
\section{Ofreção}
\begin{itemize}
\item {Grp. gram.:f.}
\end{itemize}
\begin{itemize}
\item {Utilização:Ant.}
\end{itemize}
Oferenda.
Luvas ou peita, com que se obtinha dos oficiaes da Corôa a concessão de casaes ermos. Cf. Herculano, \textunderscore Hist. de Port.\textunderscore , III, 357.
(Por \textunderscore oferção\textunderscore , do b. lat. \textunderscore offertio\textunderscore , do lat. \textunderscore offere\textunderscore )
\section{Ofrenda}
\begin{itemize}
\item {Grp. gram.:f.}
\end{itemize}
O mesmo que \textunderscore oferenda\textunderscore .
\section{Ofuscação}
\begin{itemize}
\item {Grp. gram.:f.}
\end{itemize}
\begin{itemize}
\item {Proveniência:(Lat. \textunderscore offuscatio\textunderscore )}
\end{itemize}
Acto ou efeito de ofuscar.
Obscurecimento.
\section{Ofuscado}
\begin{itemize}
\item {Grp. gram.:adj.}
\end{itemize}
\begin{itemize}
\item {Proveniência:(De \textunderscore ofuscar\textunderscore )}
\end{itemize}
Escurecido.
Suplantado.
\section{Ofuscamento}
\begin{itemize}
\item {Grp. gram.:m.}
\end{itemize}
O mesmo que \textunderscore ofuscação\textunderscore .
\section{Ofuscar}
\begin{itemize}
\item {Grp. gram.:v. t.}
\end{itemize}
\begin{itemize}
\item {Utilização:Fig.}
\end{itemize}
\begin{itemize}
\item {Proveniência:(Lat. \textunderscore offuscare\textunderscore )}
\end{itemize}
Tornar escuro; encobrir; deslumbrar; enturvar.
Obcecar; turvar a inteligência de.
Tornar menos intenso.
Tirar o prestígio a: \textunderscore êste orador ofuscaria Cícero\textunderscore .
\section{Ofreçom}
\begin{itemize}
\item {Grp. gram.:f.}
\end{itemize}
\begin{itemize}
\item {Utilização:Ant.}
\end{itemize}
O mesmo que \textunderscore offerta\textunderscore .
\section{Ogano}
\begin{itemize}
\item {Grp. gram.:adv.}
\end{itemize}
\begin{itemize}
\item {Utilização:Ant.}
\end{itemize}
Neste anno. Cf. \textunderscore Eufrosina\textunderscore , 277.
(Da loc. lat. \textunderscore hoc anno\textunderscore )
\section{Ogcode}
\begin{itemize}
\item {Grp. gram.:m.}
\end{itemize}
Gênero de insectos dípteros.
\section{Ogcódera}
\begin{itemize}
\item {Grp. gram.:f.}
\end{itemize}
O mesmo que \textunderscore ogcódero\textunderscore .
\section{Ogcódero}
\begin{itemize}
\item {Grp. gram.:m.}
\end{itemize}
Gênero de insectos coleópteros da América.
\section{Ógea}
\begin{itemize}
\item {Grp. gram.:f.}
\end{itemize}
Ave de rapina.
\section{Ogervão}
\begin{itemize}
\item {Grp. gram.:m.}
\end{itemize}
(V.urgebão)
\section{Ogiera}
\begin{itemize}
\item {Grp. gram.:f.}
\end{itemize}
Gênero de plantas americanas, da fam. das compostas.
\section{Ogiva}
\begin{itemize}
\item {Grp. gram.:f.}
\end{itemize}
\begin{itemize}
\item {Utilização:Archit.}
\end{itemize}
\begin{itemize}
\item {Proveniência:(Fr. \textunderscore ogive\textunderscore )}
\end{itemize}
Nerouras ou arestas salientes, que se cortam diagonalmente, formando um ângulo, cujos lados terminam geralmente sôbre as linhas dos centros.
Arco diagonal de uma abóbada gótica.
\section{Ogival}
\begin{itemize}
\item {Grp. gram.:f.}
\end{itemize}
Relativo a ogiva ou que tem fórma de ogiva: \textunderscore janela ogival\textunderscore .
\section{Oglifa}
\begin{itemize}
\item {Grp. gram.:f.}
\end{itemize}
Gênero de plantas, da fam. das compostas.
\section{Ogo}
\begin{itemize}
\item {Grp. gram.:m.}
\end{itemize}
\begin{itemize}
\item {Utilização:Prov.}
\end{itemize}
\begin{itemize}
\item {Utilização:dur.}
\end{itemize}
Cada uma das cordas que, nos barcos grandes do Doiro, partem do alto do mastro e vão prender-se, uma do cada lado, á extremidade do traste.
\section{Ogra}
\begin{itemize}
\item {Grp. gram.:f.}
\end{itemize}
Supposta fêmea do ogro:«\textunderscore ser sogra de uma harpia e de uma ogra...\textunderscore »Castilho, \textunderscore Sabichonas\textunderscore , 188.
\section{Ogro}
\begin{itemize}
\item {Grp. gram.:m.}
\end{itemize}
\begin{itemize}
\item {Proveniência:(Fr. \textunderscore ogre\textunderscore )}
\end{itemize}
Designação afrancesada do um monstro imaginário que come crianças, e que em português se diz \textunderscore papão\textunderscore .
\section{Ogidromito}
\begin{itemize}
\item {Grp. gram.:m.}
\end{itemize}
\begin{itemize}
\item {Proveniência:(Do gr. \textunderscore ogugios\textunderscore  + \textunderscore dromo\textunderscore )}
\end{itemize}
Gênero de crustáceos, cuja espécie típica foi encontrada em estado fóssil nos terrenos jurássicos.
\section{Ogígio}
\begin{itemize}
\item {Grp. gram.:m.}
\end{itemize}
\begin{itemize}
\item {Proveniência:(Gr. \textunderscore ogugios\textunderscore )}
\end{itemize}
Gênero de crustáceos, caracterizados por terem corpo elíptico, cabeça grande e prolongada, abdome muito desenvolvido e tórax do oito a déz anéis.
\section{Ogydromito}
\begin{itemize}
\item {Grp. gram.:m.}
\end{itemize}
\begin{itemize}
\item {Proveniência:(Do gr. \textunderscore ogugios\textunderscore  + \textunderscore dromo\textunderscore )}
\end{itemize}
Gênero de crustáceos, cuja espécie týpica foi encontrada em estado fóssil nos terrenos jurássicos.
\section{Ogýgio}
\begin{itemize}
\item {Grp. gram.:m.}
\end{itemize}
\begin{itemize}
\item {Proveniência:(Gr. \textunderscore ogugios\textunderscore )}
\end{itemize}
Gênero de crustáceos, caracterizados por terem corpo ellíptico, cabeça grande e prolongada, abdome muito desenvolvido e thórax do oito a déz anéis.
\section{Oh!}
\begin{itemize}
\item {Grp. gram.:interj.}
\end{itemize}
(designativo de \textunderscore espanto\textunderscore , \textunderscore alegria\textunderscore , \textunderscore dôr\textunderscore , \textunderscore repugnância\textunderscore , etc.)
(Por \textunderscore ó\textunderscore ! do lat. \textunderscore o\textunderscore )
\section{Óhmetro}
\begin{itemize}
\item {Grp. gram.:m.}
\end{itemize}
O mesmo que \textunderscore ohmiómetro\textunderscore .
\section{Ohmio}
\begin{itemize}
\item {Grp. gram.:m.}
\end{itemize}
\begin{itemize}
\item {Utilização:Phýs.}
\end{itemize}
\begin{itemize}
\item {Proveniência:(De \textunderscore Ohm\textunderscore , n. p.)}
\end{itemize}
Unidade prática de resistência eléctrica.
\section{Ohmiómetro}
\begin{itemize}
\item {Grp. gram.:m.}
\end{itemize}
\begin{itemize}
\item {Utilização:Phýs.}
\end{itemize}
Apparelho, com que se avalia a intensidade do óhmio.
\section{Oiacás}
\begin{itemize}
\item {Grp. gram.:m. pl.}
\end{itemize}
Indígenas do norte do Brasil.
\section{Oiça}
\begin{itemize}
\item {Grp. gram.:f.}
\end{itemize}
\begin{itemize}
\item {Utilização:Fam.}
\end{itemize}
\begin{itemize}
\item {Proveniência:(De \textunderscore oiço\textunderscore , pres. do indic. de \textunderscore ouvir\textunderscore )}
\end{itemize}
Ouvido; o sentido da audição:«\textunderscore esgaravata as orelhas, para aguçar as ouças\textunderscore ». Filinto, IX, 94.
\section{Oiça}
\begin{itemize}
\item {Grp. gram.:f.}
\end{itemize}
Chavelha ou peça de pau, que segura na canga o tamoeiro.
\section{Oídio}
\begin{itemize}
\item {Grp. gram.:m.}
\end{itemize}
Gênero de cogumelos parasitos, uma espécie do qual produz uma doença das uvas, conhecida pelo nome do mesmo parasito e pelo nome de \textunderscore poeiro\textunderscore .
(Dem. do gr. \textunderscore oon\textunderscore )
\section{Oigalé!}
\begin{itemize}
\item {Grp. gram.:interj.}
\end{itemize}
\begin{itemize}
\item {Utilização:Bras. do S}
\end{itemize}
(Designa \textunderscore admiração\textunderscore )
\section{Oigar}
\begin{itemize}
\item {Grp. gram.:v. t.}
\end{itemize}
\begin{itemize}
\item {Utilização:Prov.}
\end{itemize}
\begin{itemize}
\item {Utilização:trasm.}
\end{itemize}
Dispor em feixes (a lenha).
Vencer, deitar por terra, lutando arca por arca.
(Por \textunderscore iguar\textunderscore , contr. de \textunderscore igualar\textunderscore )
\section{Oíl}
\begin{itemize}
\item {Proveniência:(Do lat. \textunderscore hoc\textunderscore  + \textunderscore illud\textunderscore )}
\end{itemize}
Partícula que, no antigo dialecto do norte da França, significava \textunderscore sim\textunderscore .
\textunderscore Língua de oíl\textunderscore , dialecto românico, falando ao norte da França.
\section{Oiospermo}
\begin{itemize}
\item {Grp. gram.:m.}
\end{itemize}
Gênero de plantas, da fam. das compostas.
\section{Oira}
\begin{itemize}
\item {Grp. gram.:f.}
\end{itemize}
\begin{itemize}
\item {Proveniência:(Do lat. \textunderscore aura\textunderscore )}
\end{itemize}
Perturbação da cabeça, produzida por fraqueza ou debilidade.
\section{Oira}
\begin{itemize}
\item {Grp. gram.:f.}
\end{itemize}
\begin{itemize}
\item {Utilização:Bras}
\end{itemize}
Larva de um grande carapaná, que penetra na pelle dos homens e dos animaes.
\section{Oirada}
\begin{itemize}
\item {Grp. gram.:adj. f.}
\end{itemize}
\begin{itemize}
\item {Utilização:Prov.}
\end{itemize}
\begin{itemize}
\item {Utilização:beir.}
\end{itemize}
Diz-se da rapariga enfeitada com objectos de oiro.
\section{Oirado}
\begin{itemize}
\item {Grp. gram.:adj.}
\end{itemize}
\begin{itemize}
\item {Proveniência:(De \textunderscore oirar\textunderscore )}
\end{itemize}
Que tem oira.
\section{Oirama}
\begin{itemize}
\item {Grp. gram.:f.}
\end{itemize}
\begin{itemize}
\item {Utilização:Bras. de Minas}
\end{itemize}
\begin{itemize}
\item {Utilização:Ext.}
\end{itemize}
Dinheiro em oiro.
Dinheiro.
\section{Oirana}
\begin{itemize}
\item {Grp. gram.:f.}
\end{itemize}
\begin{itemize}
\item {Utilização:Bras}
\end{itemize}
Arbusto, semelhante ao salgueiro.
\section{Oirar}
\begin{itemize}
\item {Grp. gram.:v. i.}
\end{itemize}
\begin{itemize}
\item {Proveniência:(De \textunderscore oira\textunderscore )}
\end{itemize}
Têr tonturas de cabeça.
\section{Oirar}
\begin{itemize}
\item {Grp. gram.:v. t.}
\end{itemize}
\begin{itemize}
\item {Proveniência:(De \textunderscore oiro\textunderscore )}
\end{itemize}
Dotar ou prendar com oiro (a noiva).
\section{Oirejante}
\begin{itemize}
\item {Grp. gram.:adj.}
\end{itemize}
Que oireja.
\section{Oirejar}
\begin{itemize}
\item {Grp. gram.:v. t.}
\end{itemize}
\begin{itemize}
\item {Utilização:Neol.}
\end{itemize}
Brilhar como oiro; brilhar (qualquer objecto de oiro ou doirado).
\section{Oirejar}
\begin{itemize}
\item {Grp. gram.:v. t.}
\end{itemize}
O mesmo que \textunderscore oirar\textunderscore ^2. Cf. D. Bernárdez, \textunderscore Lima\textunderscore , 101.
\section{Oiriçar}
\begin{itemize}
\item {Grp. gram.:v. t.}
\end{itemize}
Tornar semelhante aos pêlos do oiriço.
Eriçar; tornar áspero: \textunderscore o gato oiriça o pêlo\textunderscore .
\section{Oiriceira}
\begin{itemize}
\item {Grp. gram.:f.}
\end{itemize}
Depósito de oiriços com castanhas, para que estas se conservem frescas e sans.
\section{Oiriceiro}
\begin{itemize}
\item {Grp. gram.:m.}
\end{itemize}
O mesmo que \textunderscore oiriceira\textunderscore .
\section{Oirichuvo}
\begin{itemize}
\item {Grp. gram.:adj.}
\end{itemize}
\begin{itemize}
\item {Utilização:Poét.}
\end{itemize}
\begin{itemize}
\item {Proveniência:(De \textunderscore oiro\textunderscore  + \textunderscore chuva\textunderscore )}
\end{itemize}
Que espalha em chuva de oiro.
\section{Oiriço}
\begin{itemize}
\item {Grp. gram.:m.}
\end{itemize}
\begin{itemize}
\item {Proveniência:(Do lat. \textunderscore ericius\textunderscore )}
\end{itemize}
Invólucro espinhoso de alguns frutos.
\textunderscore Oiriço cacheiro\textunderscore , animal revestido de espinhos, que serve de typo aos erinacídeos.
\textunderscore Oiriço do mar\textunderscore , animal echinoderme.
\section{Oirincu}
\begin{itemize}
\item {Grp. gram.:m.}
\end{itemize}
\begin{itemize}
\item {Utilização:Ant.}
\end{itemize}
\begin{itemize}
\item {Proveniência:(De \textunderscore oiro\textunderscore  + \textunderscore em\textunderscore  + \textunderscore cu\textunderscore )}
\end{itemize}
O mesmo que \textunderscore pyrilampo\textunderscore .
\section{Oiro}
\begin{itemize}
\item {Grp. gram.:m.}
\end{itemize}
\begin{itemize}
\item {Utilização:Fig.}
\end{itemize}
\begin{itemize}
\item {Grp. gram.:Pl.}
\end{itemize}
\begin{itemize}
\item {Proveniência:(Do lat. \textunderscore aurum\textunderscore )}
\end{itemize}
Metal de brilho amarelo, que se cunham as moédas de maior valor e fabrícam certas jóias.
Dinheiro.
Jóias: \textunderscore a varina traz muito oiro\textunderscore .
Côr amarela e brilhante: \textunderscore o oiro dos seus cabellos\textunderscore .
Preciosidade.
Qualidade ou objecto de grande valor.
Grande valor, grande merecimento: \textunderscore o silêncio é de oiro\textunderscore .
Um dos quatro naipes das cartas de jogar.
\section{Oiro-fio}
\begin{itemize}
\item {Grp. gram.:loc. adv.}
\end{itemize}
Em igual proporção; parallelamente; exactamente:«\textunderscore quando as conchas da balança se equilibram oiro-fio...\textunderscore »Camillo, \textunderscore Viuva do Enforc.\textunderscore , II, 9,«\textunderscore ...o bem e o mal estão oiro-fio na condição humana\textunderscore ». Idem, \textunderscore Retr. de Ricard.\textunderscore , 94.
\section{Oiropele}
\begin{itemize}
\item {Grp. gram.:m.}
\end{itemize}
O mesmo que \textunderscore ouropel\textunderscore . Cf. Castilho, \textunderscore Fausto\textunderscore , 39.
\section{Oiropelle}
\begin{itemize}
\item {Grp. gram.:m.}
\end{itemize}
O mesmo que \textunderscore ouropel\textunderscore . Cf. Castilho, \textunderscore Fausto\textunderscore , 39.
\section{Oiro-pigmento}
\begin{itemize}
\item {Grp. gram.:m.}
\end{itemize}
Mineral fusível e venenoso, composto de arsênico e enxôfre.
\section{Oiro-pimenta}
\begin{itemize}
\item {Grp. gram.:m.}
\end{itemize}
O mesmo que \textunderscore oiro-pigmento\textunderscore . Cf. M. Pinto. \textunderscore Veter.\textunderscore , I, 405.
\section{Oiro-vale}
\begin{itemize}
\item {Grp. gram.:m.}
\end{itemize}
Planta da serra de Sintra.
\section{Oirudo}
\begin{itemize}
\item {Grp. gram.:adj.}
\end{itemize}
\begin{itemize}
\item {Utilização:Bras. de Minas}
\end{itemize}
\begin{itemize}
\item {Proveniência:(De \textunderscore oiro\textunderscore )}
\end{itemize}
O mesmo que \textunderscore rico\textunderscore .
\section{Oitante}
\begin{itemize}
\item {Grp. gram.:m.}
\end{itemize}
\begin{itemize}
\item {Proveniência:(Do lat. \textunderscore octans\textunderscore )}
\end{itemize}
Distâncía de 45°, entre o Sol e outro astro.
Arco de 45°.
Instrumento náutico para medir alturas e distâncias.
\section{Oitão}
\begin{itemize}
\item {Grp. gram.:m.}
\end{itemize}
\begin{itemize}
\item {Utilização:Prov.}
\end{itemize}
\begin{itemize}
\item {Utilização:minh.}
\end{itemize}
\begin{itemize}
\item {Proveniência:(Do lat. hyp. \textunderscore altanus\textunderscore ?)}
\end{itemize}
Parte lateral de um edifício.
Pedaço de muro alto.
\section{Oitava}
\begin{itemize}
\item {Grp. gram.:f.}
\end{itemize}
\begin{itemize}
\item {Utilização:Bras}
\end{itemize}
\begin{itemize}
\item {Utilização:Mús.}
\end{itemize}
\begin{itemize}
\item {Proveniência:(De \textunderscore oitavo\textunderscore )}
\end{itemize}
Cada uma das oito partes iguaes, em que se póde dividir alguma coisa.
Antigo pêso de pharmácia, correspondente á oitava parte de uma onça.
Antigo imposto dos que pagavam ao Estado ou a uma corporação a oitava parto de certos rendimentos.
Espaço do oito dias, consagrado a alguma festa religiosa.
O último dia dêsse espaço de tempo.
Intervallo, entre duas notas musicaes do mesmo nome e differente tom.
Estrophe de oito versos.
Unidade monetária, correspondente a 1.200 reis, no Estado de Mato-Grosso.
\textunderscore Oitava real\textunderscore , registo de órgão, com tubos de estanho.
\textunderscore Oitava tapada\textunderscore , registo de órgão, de tubos iguaes aos da oitava real, mas tapados, resoando por isso em oitava abaixo, isto é, em unisono com o diapasão.
\section{Oitavado}
\begin{itemize}
\item {Grp. gram.:adj.}
\end{itemize}
\begin{itemize}
\item {Grp. gram.:M.}
\end{itemize}
Que tem oito faces contíguas, formando ângulo contíguo.
Dança popular, no século XVIII.
O mesmo que \textunderscore oitavina\textunderscore .
\section{Oitavar}
\begin{itemize}
\item {Grp. gram.:v. t.}
\end{itemize}
\begin{itemize}
\item {Proveniência:(De \textunderscore oitavo\textunderscore )}
\end{itemize}
Tornar oitavado; dividir em oito partes; dividir em oitavas musicaes.
\section{Oitavário}
\begin{itemize}
\item {Grp. gram.:m.}
\end{itemize}
\begin{itemize}
\item {Proveniência:(De \textunderscore oitavo\textunderscore )}
\end{itemize}
Festa religiosa de oito dias; oitava.
Livro, que contem as orações relativas ao oitavário.
\section{Oitaveiro}
\begin{itemize}
\item {Grp. gram.:m.  e  adj.}
\end{itemize}
O que pagava o imposto chamado oitava ou oitavo.
\section{Oitaviante}
\begin{itemize}
\item {Grp. gram.:adj.}
\end{itemize}
\begin{itemize}
\item {Utilização:Mús.}
\end{itemize}
\begin{itemize}
\item {Proveniência:(De \textunderscore oitaviar\textunderscore , por \textunderscore oitavar\textunderscore )}
\end{itemize}
\textunderscore Frauta oitaviante\textunderscore , registro de órgãos modernos, com o orifício a meia altura, e que produz o som harmonioso da oitava superior.
\section{Oitavina}
\begin{itemize}
\item {Grp. gram.:f.}
\end{itemize}
Pequena viola, bandurra, cavaquinho.
\section{Oitavino}
\begin{itemize}
\item {Grp. gram.:m.}
\end{itemize}
Pequena frauta que, com metade da dimensão da frauta usual, produz a mesma escala em oitava superior.
\section{Oitavo}
\begin{itemize}
\item {Grp. gram.:adj.}
\end{itemize}
\begin{itemize}
\item {Grp. gram.:M.}
\end{itemize}
\begin{itemize}
\item {Proveniência:(Lat. \textunderscore octavus\textunderscore )}
\end{itemize}
Que numa série de oito occupa o último lugar.
A oitava parte.
Antigo imposto.
\section{Oiteiral}
\begin{itemize}
\item {Grp. gram.:adj.}
\end{itemize}
Relativo a oiteiro. Cf. Filinto, I, 161.
\section{Oiteirete}
\begin{itemize}
\item {fónica:teirê}
\end{itemize}
\begin{itemize}
\item {Grp. gram.:m.}
\end{itemize}
Pequeno oiteiro. Cf. Filinto, XV, 303.
\section{Oiteirista}
\begin{itemize}
\item {Grp. gram.:m.}
\end{itemize}
Aquelle que trovava nos oiteiros conventuaes.
\section{Oiteiro}
\begin{itemize}
\item {Grp. gram.:m.}
\end{itemize}
Pequeno monte.
Collina.
Festa no pátio dos conventos, em que os poétas glosavam os motes propostos pelas freiras.
(Por \textunderscore alteiro\textunderscore , de \textunderscore alto\textunderscore )
\section{Oitenta}
\begin{itemize}
\item {Grp. gram.:adj.}
\end{itemize}
\begin{itemize}
\item {Grp. gram.:M.}
\end{itemize}
\begin{itemize}
\item {Proveniência:(Do lat. \textunderscore octoginta\textunderscore )}
\end{itemize}
Déz vezes oito.
Representação dêsse número em algarismos ou conta romana.
O que occupa o último lugar numa série de oitenta.
\section{Oitentão}
\begin{itemize}
\item {Grp. gram.:m.  e  adj.}
\end{itemize}
\begin{itemize}
\item {Utilização:Pop.}
\end{itemize}
O mesmo que \textunderscore octogenário\textunderscore .
\section{Oiti}
\begin{itemize}
\item {Grp. gram.:m.}
\end{itemize}
Nome de várias plantas do Brasil.
\section{Oiticica}
\begin{itemize}
\item {Grp. gram.:f.}
\end{itemize}
\begin{itemize}
\item {Utilização:Bras}
\end{itemize}
Árvore silvestre.
\section{Oitituruba}
\begin{itemize}
\item {Grp. gram.:f.}
\end{itemize}
(V.tuturubá)
\section{Oito}
\begin{itemize}
\item {Grp. gram.:adj.}
\end{itemize}
\begin{itemize}
\item {Grp. gram.:M.}
\end{itemize}
\begin{itemize}
\item {Utilização:Prolóq.}
\end{itemize}
\begin{itemize}
\item {Grp. gram.:Loc.}
\end{itemize}
\begin{itemize}
\item {Utilização:bras. do N}
\end{itemize}
\begin{itemize}
\item {Proveniência:(Lat. \textunderscore octo\textunderscore )}
\end{itemize}
Diz-se do número cardinal, formado de sete e mais um.
Oitavo.
O algarismo representativo do número oito.
Carta de jogar, que tem oito pontos.
Aquillo que numa série de oito occupa o último lugar.
\textunderscore Ou oito ou oitenta\textunderscore , ou tudo ou nada.
\textunderscore Tomar um oito\textunderscore , embriagar-se.
\section{Oitocentos}
\begin{itemize}
\item {Grp. gram.:adj.}
\end{itemize}
\begin{itemize}
\item {Proveniência:(De \textunderscore oito\textunderscore  + \textunderscore cento\textunderscore )}
\end{itemize}
Oito vezes cem.
\section{Oitubro}
\begin{itemize}
\item {Grp. gram.:m.}
\end{itemize}
(V.outubro)
\section{Oja}
\begin{itemize}
\item {Grp. gram.:f.}
\end{itemize}
\begin{itemize}
\item {Utilização:Ant.}
\end{itemize}
Nome de uma ave, talvez o mesmo que \textunderscore ujo\textunderscore . Cf. G. Vicente.
\section{Ójea}
\begin{itemize}
\item {Grp. gram.:f.}
\end{itemize}
O mesmo que \textunderscore oja\textunderscore .
\section{Ojeriza}
\begin{itemize}
\item {Grp. gram.:f.}
\end{itemize}
\begin{itemize}
\item {Utilização:Des.}
\end{itemize}
O mesmo que \textunderscore antipathia\textunderscore .
(Do cast.)
\section{Ola}
\begin{itemize}
\item {Grp. gram.:f.}
\end{itemize}
\begin{itemize}
\item {Utilização:Prov.}
\end{itemize}
\begin{itemize}
\item {Utilização:trasm.}
\end{itemize}
Remoínho na água.
(Cast. \textunderscore ola\textunderscore )
\section{Ola}
\textunderscore f. Ant.\textunderscore  (ainda us. na Índia port.)
Fôlha de palma:«\textunderscore ...pondo-lhe a ola na cabeça, foi aclamado rey.\textunderscore »\textunderscore Conquista do Pegu\textunderscore , XIII.
Fôlha de palmeira, preparada para nella se escrever.
Carta ou qualquer documento, escrito numa só fôlha.
Cartel:«\textunderscore ...num momento de enviar a ola do desafio.\textunderscore »B. Pato, \textunderscore Port. na Índia\textunderscore , 32.
\section{Olá}
\begin{itemize}
\item {Grp. gram.:interj.}
\end{itemize}
\begin{itemize}
\item {Proveniência:(De \textunderscore ó\textunderscore ^2 + \textunderscore lá\textunderscore )}
\end{itemize}
(para chamar, ou para exprimir admiração)
\section{Olace}
\begin{itemize}
\item {Grp. gram.:m.}
\end{itemize}
O mesmo ou melhor que \textunderscore oláceo\textunderscore .
\section{Oláceo}
\begin{itemize}
\item {Grp. gram.:m.}
\end{itemize}
\begin{itemize}
\item {Proveniência:(Do lat. \textunderscore olax\textunderscore )}
\end{itemize}
Gênero de plantas glabas.
\section{Olacíneas}
\begin{itemize}
\item {Grp. gram.:f. pl.}
\end{itemize}
\begin{itemize}
\item {Proveniência:(De \textunderscore olacíneo\textunderscore )}
\end{itemize}
Família de plantas, que tem por typo o oláceo.
\section{Olacíneo}
\begin{itemize}
\item {Grp. gram.:adj.}
\end{itemize}
Relativo ou semelhante ao oláceo.
\section{Olaeira}
\begin{itemize}
\item {Grp. gram.:f.}
\end{itemize}
Árvore leguminosa, (\textunderscore cercis siliquastrum\textunderscore ).
\section{Olaia}
\begin{itemize}
\item {Grp. gram.:f.}
\end{itemize}
Árvore leguminosa, (\textunderscore cercis siliquastrum\textunderscore ).
\section{Olampi}
\begin{itemize}
\item {Grp. gram.:m.}
\end{itemize}
Espécie de resina americana, impropriamente chamada \textunderscore goma\textunderscore  em pharmácia.
\section{Olânico}
\begin{itemize}
\item {Grp. gram.:adj.}
\end{itemize}
\begin{itemize}
\item {Utilização:Chím.}
\end{itemize}
Diz-se dos saes, que têm por base a olanina.
\section{Olanina}
\begin{itemize}
\item {Grp. gram.:f.}
\end{itemize}
\begin{itemize}
\item {Utilização:Chím.}
\end{itemize}
Substância, descoberta no óleo mineral.
\section{Olaré!}
\begin{itemize}
\item {Grp. gram.:interj.}
\end{itemize}
(designativa de affirmação e satisfação) Cf. Eça, \textunderscore P. Amaro\textunderscore , 215.
\section{Ólcades}
\begin{itemize}
\item {Grp. gram.:m. pl.}
\end{itemize}
\begin{itemize}
\item {Proveniência:(Lat. \textunderscore Olcades\textunderscore )}
\end{itemize}
Antigo povo da Hispânia, destruído por Anníbal.
\section{Oldembúrgia}
\begin{itemize}
\item {Grp. gram.:f.}
\end{itemize}
Gênero de arbustos do Cabo da Bôa-Esperança.
\section{Oldemburguês}
\begin{itemize}
\item {Grp. gram.:adj.}
\end{itemize}
Relativo ao principado de Oldemburgo.
\section{Oldenlândia}
\begin{itemize}
\item {Grp. gram.:f.}
\end{itemize}
Gênero de plantas rubiáceas.
\section{Olé!}
\begin{itemize}
\item {Grp. gram.:interj.}
\end{itemize}
(designativa de \textunderscore affirmação\textunderscore , e o mesmo que \textunderscore olá\textunderscore )
Também serve para chamar.
\section{Olea}
\begin{itemize}
\item {Grp. gram.:f.}
\end{itemize}
Grande navio de carga, usado antigamente.
\section{Oleáceo}
\begin{itemize}
\item {Proveniência:(Lat. \textunderscore oleaceus\textunderscore )}
\end{itemize}
\textunderscore adj.\textunderscore  (e der.)
O mesmo que \textunderscore oleagíneo\textunderscore , etc.
\section{Oleado}
\begin{itemize}
\item {Grp. gram.:m.}
\end{itemize}
\begin{itemize}
\item {Proveniência:(De \textunderscore olear\textunderscore )}
\end{itemize}
Pano, tornado impermeável por meio de verniz ou de outra substância análoga.
\section{Oleagíneas}
\begin{itemize}
\item {Grp. gram.:f. pl.}
\end{itemize}
\begin{itemize}
\item {Proveniência:(De \textunderscore oleagíneo\textunderscore )}
\end{itemize}
Família de plantas, que tem por typo a oliveira.
\section{Oleagíneo}
\begin{itemize}
\item {Grp. gram.:adj.}
\end{itemize}
\begin{itemize}
\item {Proveniência:(Lat. \textunderscore oleagineus\textunderscore )}
\end{itemize}
Relativo ou semelhante á oliveira.
\section{Oleaginoso}
\begin{itemize}
\item {Grp. gram.:adj.}
\end{itemize}
\begin{itemize}
\item {Proveniência:(De \textunderscore oleagíneo\textunderscore )}
\end{itemize}
Que contém óleo ou que é da natureza do óleo.
\section{Oleanário}
\begin{itemize}
\item {Grp. gram.:adj.}
\end{itemize}
\begin{itemize}
\item {Utilização:Chím.}
\end{itemize}
Que exhala cheiro de azeite.
\section{Oleandro}
\begin{itemize}
\item {Grp. gram.:m.}
\end{itemize}
O mesmo que \textunderscore loendro\textunderscore .
\section{Olear}
\begin{itemize}
\item {Grp. gram.:v. t.}
\end{itemize}
Cobrir de óleo.
Impregnar de uma substância oleosa.
\section{Olearia}
\begin{itemize}
\item {Grp. gram.:f.}
\end{itemize}
Fábrica de óleos.
\section{Oleastro}
\begin{itemize}
\item {Grp. gram.:m.}
\end{itemize}
\begin{itemize}
\item {Proveniência:(Lat. \textunderscore oleaster\textunderscore )}
\end{itemize}
O mesmo que \textunderscore zambujeiro\textunderscore .
\section{Oleato}
\begin{itemize}
\item {Grp. gram.:m.}
\end{itemize}
\begin{itemize}
\item {Utilização:Chím.}
\end{itemize}
\begin{itemize}
\item {Proveniência:(De \textunderscore óleo\textunderscore )}
\end{itemize}
Sal, formado pela combinação do ácido oleico com uma base.
\section{Olecraniano}
\begin{itemize}
\item {Grp. gram.:adj.}
\end{itemize}
Relativo ao olécrano.
\section{Olecrânio}
\begin{itemize}
\item {Grp. gram.:m.}
\end{itemize}
O mesmo que \textunderscore olécrano\textunderscore .
\section{Olécrano}
\begin{itemize}
\item {Grp. gram.:m.}
\end{itemize}
\begin{itemize}
\item {Proveniência:(Gr. \textunderscore olekranon\textunderscore )}
\end{itemize}
Saliência arredondada da extremidade umeral do cúbito.
\section{Olefina}
\begin{itemize}
\item {Grp. gram.:f.}
\end{itemize}
\begin{itemize}
\item {Utilização:Chím.}
\end{itemize}
Hydrocarboneto não saturado, em que o número de átomos de hydrogênio é duplo dos do carbone.
\section{Oleico}
\begin{itemize}
\item {Grp. gram.:adj.}
\end{itemize}
\begin{itemize}
\item {Proveniência:(De \textunderscore óleo\textunderscore )}
\end{itemize}
Diz-se de um ácido, produzido pela saponificação do azeite e de outros óleos.
\section{Oleícola}
\begin{itemize}
\item {Grp. gram.:adj.}
\end{itemize}
\begin{itemize}
\item {Proveniência:(Do lat. \textunderscore oleum\textunderscore  + \textunderscore colere\textunderscore )}
\end{itemize}
Relativo á cultura das oliveiras e ao commércio do azeite.
\section{Oleicultor}
\begin{itemize}
\item {fónica:le-i}
\end{itemize}
\begin{itemize}
\item {Grp. gram.:m.}
\end{itemize}
Aquelle que se occupa de oleicultura.
\section{Oleicultura}
\begin{itemize}
\item {fónica:le-i}
\end{itemize}
\begin{itemize}
\item {Grp. gram.:f.}
\end{itemize}
\begin{itemize}
\item {Proveniência:(Do lat. \textunderscore oleum\textunderscore  + \textunderscore cultura\textunderscore )}
\end{itemize}
Indústria do fabríco, tratamento e conservação do azeite.
\section{Oleídeo}
\begin{itemize}
\item {Grp. gram.:adj.}
\end{itemize}
\begin{itemize}
\item {Grp. gram.:M. pl.}
\end{itemize}
\begin{itemize}
\item {Proveniência:(De \textunderscore óleo\textunderscore  + gr. \textunderscore eidos\textunderscore )}
\end{itemize}
Relativo ou semelhante ao azeite.
Família dos corpos oleosos.
\section{Oleífero}
\begin{itemize}
\item {Grp. gram.:adj.}
\end{itemize}
Que produz óleo.
\section{Oleificante}
\begin{itemize}
\item {fónica:le-i}
\end{itemize}
\begin{itemize}
\item {Grp. gram.:adj.}
\end{itemize}
\begin{itemize}
\item {Proveniência:(Do lat. \textunderscore oleum\textunderscore  + \textunderscore facere\textunderscore )}
\end{itemize}
Que produz óleo.
\section{Oleifoliado}
\begin{itemize}
\item {fónica:le-i}
\end{itemize}
\begin{itemize}
\item {Grp. gram.:adj.}
\end{itemize}
\begin{itemize}
\item {Proveniência:(Do lat. \textunderscore oleum\textunderscore  + \textunderscore folium\textunderscore )}
\end{itemize}
Que tem fôlhas, semelhantes ás da oliveira.
\section{Oleígeno}
\begin{itemize}
\item {Grp. gram.:adj.}
\end{itemize}
\begin{itemize}
\item {Proveniência:(Do lat. \textunderscore oleum\textunderscore  + \textunderscore genere\textunderscore )}
\end{itemize}
Que produz líquido, semelhante ao óleo.
\section{Oleíla}
\begin{itemize}
\item {Grp. gram.:f.}
\end{itemize}
\begin{itemize}
\item {Proveniência:(Do lat. \textunderscore oleum\textunderscore )}
\end{itemize}
Nome chímico do azeite.
\section{Oleína}
\begin{itemize}
\item {Grp. gram.:f.}
\end{itemize}
\begin{itemize}
\item {Proveniência:(Do lat. \textunderscore oleum\textunderscore )}
\end{itemize}
Substância orgânica e gorda, que faz parte de todos os óleos vegetaes e da maior parte dos óleos gordos.
\section{Oleíneas}
\begin{itemize}
\item {Grp. gram.:f. pl.}
\end{itemize}
(V.oleagíneas)
\section{Olembro-negro}
\begin{itemize}
\item {Grp. gram.:m.}
\end{itemize}
Planta da serra de Sintra.
\section{Olenário}
\begin{itemize}
\item {Grp. gram.:adj.}
\end{itemize}
\begin{itemize}
\item {Utilização:Bot.}
\end{itemize}
\begin{itemize}
\item {Proveniência:(Do lat. bot. \textunderscore olenaris\textunderscore )}
\end{itemize}
Que cheira a óleo.
\section{Olência}
\begin{itemize}
\item {Grp. gram.:f.}
\end{itemize}
Qualidade de olente.
\section{Oleneira}
\begin{itemize}
\item {Grp. gram.:f.}
\end{itemize}
Gênero de crustáceos isópodes.
\section{Oleno}
\begin{itemize}
\item {Grp. gram.:m.}
\end{itemize}
Gênero de insectos coleópteros.
\section{Olente}
\begin{itemize}
\item {Grp. gram.:adj.}
\end{itemize}
\begin{itemize}
\item {Proveniência:(Lat. \textunderscore olens\textunderscore )}
\end{itemize}
Cheiroso, aromático.
\section{Óleo}
\begin{itemize}
\item {Grp. gram.:m.}
\end{itemize}
\begin{itemize}
\item {Utilização:Bras}
\end{itemize}
\begin{itemize}
\item {Proveniência:(Lat. \textunderscore oleum\textunderscore )}
\end{itemize}
Líquido gorduroso, que se extrai do fruto da oliveira.
Líquido semelhante, extraindo de outras substâncias vegetaes.
Gênero de árvores silvestres, leguminosas, de bôa madeira para vários usos.
\section{Óleo-barrão}
\begin{itemize}
\item {Grp. gram.:m.}
\end{itemize}
Grande árvore santhomense, própria para construcções.
\section{Óleo-cabureiba}
\begin{itemize}
\item {Grp. gram.:m.}
\end{itemize}
\begin{itemize}
\item {Utilização:Bras}
\end{itemize}
Árvore leguminosa, do gênero \textunderscore óleo\textunderscore .
\section{Óleo-comumbá}
\begin{itemize}
\item {Grp. gram.:m.}
\end{itemize}
\begin{itemize}
\item {Utilização:Bras}
\end{itemize}
Árvore leguminosa, espécie do gênero \textunderscore óleo\textunderscore .
\section{Óleo-de-setembro}
\begin{itemize}
\item {Grp. gram.:m.}
\end{itemize}
\begin{itemize}
\item {Utilização:Ant.}
\end{itemize}
\begin{itemize}
\item {Utilização:Gír.}
\end{itemize}
O mesmo que \textunderscore vinho\textunderscore .
\section{Óleo-de-zambujo}
\begin{itemize}
\item {Grp. gram.:m.}
\end{itemize}
\begin{itemize}
\item {Utilização:Ant.}
\end{itemize}
\begin{itemize}
\item {Utilização:Gír.}
\end{itemize}
O mesmo que \textunderscore pancadaria\textunderscore .
\section{Oleografia}
\begin{itemize}
\item {Grp. gram.:f.}
\end{itemize}
\begin{itemize}
\item {Proveniência:(De \textunderscore óleo\textunderscore  + gr. \textunderscore graphein\textunderscore )}
\end{itemize}
Processo moderno, com que se transmittem para uma tela nova os quadros pintados a óleo noutra tela.
Quadro, feito por êste processo.
\section{Oleográfico}
\begin{itemize}
\item {Grp. gram.:adj.}
\end{itemize}
Relativo á oleografia.
\section{Oleographia}
\begin{itemize}
\item {Grp. gram.:f.}
\end{itemize}
\begin{itemize}
\item {Proveniência:(De \textunderscore óleo\textunderscore  + gr. \textunderscore graphein\textunderscore )}
\end{itemize}
Processo moderno, com que se transmittem para uma tela nova os quadros pintados a óleo noutra tela.
Quadro, feito por êste processo.
\section{Oleográphico}
\begin{itemize}
\item {Grp. gram.:adj.}
\end{itemize}
Relativo á oleographia.
\section{Oleogravura}
\begin{itemize}
\item {Grp. gram.:f.}
\end{itemize}
\begin{itemize}
\item {Proveniência:(De \textunderscore óleo\textunderscore  + \textunderscore gravura\textunderscore )}
\end{itemize}
Processo de reproduzir pela gravura um quadro pintado a óleo.
\section{Oleol}
\begin{itemize}
\item {Grp. gram.:m.}
\end{itemize}
\begin{itemize}
\item {Utilização:Pharm.}
\end{itemize}
Óleo fixo natural.
Producto chímico, fabricado com azeite e destinado a evitar a azedia do vinho. Cf. \textunderscore Archivo Rur.\textunderscore , VII, n.^o 19.
\section{Oleolado}
\begin{itemize}
\item {Grp. gram.:m.}
\end{itemize}
Óleo medicinal, preparado por infusão ou decocção.
O mesmo que \textunderscore oleolato\textunderscore .
\section{Oleolatado}
\begin{itemize}
\item {Grp. gram.:m.}
\end{itemize}
\begin{itemize}
\item {Proveniência:(De \textunderscore oleolato\textunderscore )}
\end{itemize}
Medicamento, composto de óleos essenciaes.
\section{Oleolato}
\begin{itemize}
\item {Grp. gram.:m.}
\end{itemize}
Óleo essencial.
Óleo medicinal, preparado por infusão ou decocção.
\section{Oleólico}
\begin{itemize}
\item {Grp. gram.:adj.}
\end{itemize}
Diz-se do medicamento, cujo excipiente é o óleo ou o azeite.
\section{Oleolito}
\begin{itemize}
\item {Grp. gram.:m.}
\end{itemize}
Medicamento, que tem o óleo por excipiente.
\section{Oleomel}
\begin{itemize}
\item {Grp. gram.:m.}
\end{itemize}
Óleo doce, que se dizia estillar de uma árvore de Palmira.
\section{Oleómetro}
\begin{itemize}
\item {Grp. gram.:m.}
\end{itemize}
\begin{itemize}
\item {Proveniência:(Do lat. \textunderscore oleum\textunderscore  + gr. \textunderscore metron\textunderscore )}
\end{itemize}
Areómetro, com que se avalia a densidade dos óleos.
\section{Oleona}
\begin{itemize}
\item {Grp. gram.:f.}
\end{itemize}
Substância liquida, que se obtém pela destillação de uma mistura de cal e ácido oleico.
\section{Óleo-pardo}
\begin{itemize}
\item {Grp. gram.:m.}
\end{itemize}
\begin{itemize}
\item {Utilização:Bras}
\end{itemize}
Árvore leguminosa, espécie do gênero \textunderscore óleo\textunderscore .
\section{Oleoricinato}
\begin{itemize}
\item {fónica:ri}
\end{itemize}
\begin{itemize}
\item {Grp. gram.:m.}
\end{itemize}
\begin{itemize}
\item {Utilização:Chím.}
\end{itemize}
Sal, resultante da combinação do ácido oleoricínico com uma base.
\section{Oleoricínico}
\begin{itemize}
\item {fónica:ri}
\end{itemize}
\begin{itemize}
\item {Grp. gram.:adj.}
\end{itemize}
Diz-se do ácido, produzido pela saponificação do óleo de rícino.
\section{Oleorricinato}
\begin{itemize}
\item {Grp. gram.:m.}
\end{itemize}
\begin{itemize}
\item {Utilização:Chím.}
\end{itemize}
Sal, resultante da combinação do ácido oleorricínico com uma base.
\section{Oleorricínico}
\begin{itemize}
\item {Grp. gram.:adj.}
\end{itemize}
Diz-se do ácido, produzido pela saponificação do óleo de rícino.
\section{Oleosidade}
\begin{itemize}
\item {Grp. gram.:f.}
\end{itemize}
Qualidade do que é oleoso.
\section{Oleoso}
\begin{itemize}
\item {Grp. gram.:adj.}
\end{itemize}
\begin{itemize}
\item {Proveniência:(Lat. \textunderscore oleosus\textunderscore )}
\end{itemize}
Que tem óleo; gorduroso.
\section{Óleo-sulfúrico}
\begin{itemize}
\item {Grp. gram.:adj.}
\end{itemize}
Diz-se de um ácido, resultante da combinação do ácido oleico com o ácido sulfúrico.
\section{Óleo-vermelho}
\begin{itemize}
\item {Grp. gram.:m.}
\end{itemize}
\begin{itemize}
\item {Utilização:Bras}
\end{itemize}
Árvore leguminosa, espécie do gênero \textunderscore óleo\textunderscore .
\section{Oleráceo}
\begin{itemize}
\item {Grp. gram.:adj.}
\end{itemize}
\begin{itemize}
\item {Proveniência:(Do lat. \textunderscore olus\textunderscore , \textunderscore olerís\textunderscore )}
\end{itemize}
Relativo a legumes.
Relativo aos vegetaes, que se empregam como alimento.
\section{Oleroso}
\begin{itemize}
\item {Grp. gram.:adj.}
\end{itemize}
\begin{itemize}
\item {Proveniência:(Do lat. \textunderscore olera\textunderscore )}
\end{itemize}
Em que há plantas leguminosas; em que há hortaliça. Cf. Castilho, \textunderscore Fastos\textunderscore , I, 23.
\section{Oleula}
\begin{itemize}
\item {Grp. gram.:f.}
\end{itemize}
Óleo essencial de uma planta, em pharmácia.
\section{Oleulado}
\begin{itemize}
\item {Grp. gram.:adj.}
\end{itemize}
Diz-se dos medicamentos, formados de óleos voláteis.
\section{Olêulico}
\begin{itemize}
\item {Grp. gram.:adj.}
\end{itemize}
Diz-se dos medicamentos, que têm por base um óleo volátil.
\section{Oléus}
\begin{itemize}
\item {Grp. gram.:m. pl.}
\end{itemize}
\begin{itemize}
\item {Utilização:Bras}
\end{itemize}
Tríbo de aborígenes de Mato-Grosso.
\section{Olfacção}
\begin{itemize}
\item {Grp. gram.:f.}
\end{itemize}
\begin{itemize}
\item {Proveniência:(Lat. \textunderscore olfactio\textunderscore )}
\end{itemize}
Exercício do olfacto; acto de cheirar.
\section{Olfáctico}
\begin{itemize}
\item {Grp. gram.:adj.}
\end{itemize}
\begin{itemize}
\item {Utilização:Neol.}
\end{itemize}
Relativo ao olfacto:«\textunderscore ...a impressão olfáctica das flores\textunderscore ». Ortigão, \textunderscore Hollanda\textunderscore , 34.
\section{Olfactivo}
\begin{itemize}
\item {Grp. gram.:adj.}
\end{itemize}
Próprio para o olfacto; relativo ao olfacto.
\section{Olfacto}
\begin{itemize}
\item {Grp. gram.:m.}
\end{itemize}
\begin{itemize}
\item {Proveniência:(Lat. \textunderscore olfactus\textunderscore )}
\end{itemize}
Sentido, com que se percebe o cheiro.
Cheiro; faro.
\section{Olfactório}
\begin{itemize}
\item {Grp. gram.:adj.}
\end{itemize}
Relativo ao olfacto.
\section{Olférsia}
\begin{itemize}
\item {Grp. gram.:f.}
\end{itemize}
Gênero de insectos dípteros.
\section{Olfortum}
\begin{itemize}
\item {Grp. gram.:adj.}
\end{itemize}
\begin{itemize}
\item {Utilização:Ant.}
\end{itemize}
Que tem cheiro desagradável; que provoca náuseas.
(Talvez se relacione com \textunderscore fartum\textunderscore )
\section{Olga}
\begin{itemize}
\item {Grp. gram.:f.}
\end{itemize}
\begin{itemize}
\item {Utilização:Prov.}
\end{itemize}
\begin{itemize}
\item {Utilização:trasm.}
\end{itemize}
Belga, coirela.
Planície entre oiteiros.
\section{Ôlha}
\begin{itemize}
\item {Grp. gram.:f.}
\end{itemize}
\begin{itemize}
\item {Proveniência:(Do lat. \textunderscore olla\textunderscore )}
\end{itemize}
Comida, preparada com legumes e carnes substanciosas.
Gordura de caldo.
Caldo gordo.
Panela, para se fazer ôlha.
\section{Olha-a-água}
\begin{itemize}
\item {Grp. gram.:m.}
\end{itemize}
Arbusto africano, de fôlhas alternas, glabras, e flôres axillares em grupos de duas ou três.
\section{Olhada}
\begin{itemize}
\item {Grp. gram.:f.}
\end{itemize}
Lance de olhos; acto de olhar.
\section{Olhadela}
\begin{itemize}
\item {Grp. gram.:f.}
\end{itemize}
\begin{itemize}
\item {Utilização:Pop.}
\end{itemize}
Lance de olhos; acto de olhar.
\section{Olhado}
\begin{itemize}
\item {Grp. gram.:m.}
\end{itemize}
\begin{itemize}
\item {Proveniência:(De \textunderscore olhar\textunderscore )}
\end{itemize}
Feitiço ou quebranto que, segundo a superstição popular, é produzido pelo olhar de alguém.
\section{Olhador}
\begin{itemize}
\item {Grp. gram.:m.  e  adj.}
\end{itemize}
O que olha.
\section{Olhadura}
\begin{itemize}
\item {Grp. gram.:f.}
\end{itemize}
O mesmo que \textunderscore olhadela\textunderscore .
\section{Olhal}
\begin{itemize}
\item {Grp. gram.:m.}
\end{itemize}
\begin{itemize}
\item {Grp. gram.:Pl.}
\end{itemize}
\begin{itemize}
\item {Utilização:Veter.}
\end{itemize}
\begin{itemize}
\item {Proveniência:(Do lat. \textunderscore ocularis\textunderscore )}
\end{itemize}
Vão, entre os pilares de pontes ou arcadas: \textunderscore o barco atravessou o olhal da ponte\textunderscore .
Buraco, a que se adapta a espoleta, que communica fogo ás peças de artilharia.
Depressão sôbre as arcadas dos olhos do cavallo.
\section{Olhalva}
\begin{itemize}
\item {Grp. gram.:f.}
\end{itemize}
\begin{itemize}
\item {Utilização:Prov.}
\end{itemize}
Terreno, que se lavra duas vezes no anno e que duas vezes produz.
\section{Ola}
\begin{itemize}
\item {Grp. gram.:f.}
\end{itemize}
\begin{itemize}
\item {Utilização:Ant.}
\end{itemize}
\begin{itemize}
\item {Proveniência:(Lat. \textunderscore olla\textunderscore )}
\end{itemize}
O mesmo que \textunderscore olaria\textunderscore . Cf. \textunderscore Tombo do Estado da Índia\textunderscore , 198, 201 e 202.
O mesmo que \textunderscore panela\textunderscore .
\section{Olhalvo}
\begin{itemize}
\item {Grp. gram.:adj.}
\end{itemize}
\begin{itemize}
\item {Grp. gram.:M.}
\end{itemize}
\begin{itemize}
\item {Proveniência:(De \textunderscore ôlho\textunderscore  + \textunderscore alvo\textunderscore )}
\end{itemize}
Que tem os olhos cercados de malhas brancas ou que põe os olhos em alvo, erguendo a cabeça, (falando-se do cavallo).
Peixe de Portugal.
O mesmo que \textunderscore olhalva\textunderscore . (Colhido em Leiria)
\section{Olhandilhas}
\begin{itemize}
\item {Grp. gram.:m.}
\end{itemize}
\begin{itemize}
\item {Utilização:Bras}
\end{itemize}
O mesmo que \textunderscore farricoco\textunderscore . Cf. Pacheco, Promptuário.
(Por \textunderscore hollandilhas\textunderscore , allusão ao pano dêsse nome, usado talvez por farricocos ou em armações fúnebres)
\section{Olhante}
\begin{itemize}
\item {Grp. gram.:m.  e  adj.}
\end{itemize}
\begin{itemize}
\item {Utilização:Des.}
\end{itemize}
\begin{itemize}
\item {Proveniência:(De \textunderscore olhar\textunderscore )}
\end{itemize}
O mesmo que \textunderscore olhador\textunderscore .
Aquelle que repara em tudo ou a quem nada escapa. Cf. \textunderscore Diccion. de Nomes, Vozes\textunderscore , etc.
\section{Olhapim}
\begin{itemize}
\item {Grp. gram.:m.}
\end{itemize}
\begin{itemize}
\item {Utilização:Prov.}
\end{itemize}
\begin{itemize}
\item {Utilização:trasm.}
\end{itemize}
O mesmo que \textunderscore larápio\textunderscore .
\section{Olhar}
\begin{itemize}
\item {Grp. gram.:v. t.}
\end{itemize}
\begin{itemize}
\item {Grp. gram.:V. i.}
\end{itemize}
\begin{itemize}
\item {Grp. gram.:M.}
\end{itemize}
\begin{itemize}
\item {Proveniência:(De \textunderscore ôlho\textunderscore )}
\end{itemize}
Fitar os olhos em: \textunderscore olhar o céu\textunderscore .
Encarar.
Contemplar.
Estar de frente de.
Estar voltado para.
Tomar conta de.
Observar: \textunderscore olhar os acontecimentos\textunderscore .
Ponderar.
Investigar.
Julgar: \textunderscore olhar mal acções de outrem\textunderscore .
Voltar os olhos, applicar o sentido da vista.
Attender.
Importar-se: \textunderscore quem olha pelos órfãos\textunderscore ?
Estar voltado ou fronteiro: \textunderscore uma janela, que olha para o mar\textunderscore .
\textunderscore Olhar para a sombra\textunderscore , diz-se dos rapazes ou raparigas, que começam a sentir vaidade e a pretender namorar.
Acto de olhar; aspecto dos olhos.
\section{Olharada}
\begin{itemize}
\item {Grp. gram.:f.}
\end{itemize}
\begin{itemize}
\item {Utilização:T. de Turquel}
\end{itemize}
Olhadela furtiva.
\section{Olharapo}
\begin{itemize}
\item {Grp. gram.:m.}
\end{itemize}
\begin{itemize}
\item {Utilização:Prov.}
\end{itemize}
Lobishomem; fantasma.
Papão. (Colhido em Penaguião e Anadia)
\section{Olheirado}
\begin{itemize}
\item {Grp. gram.:adj.}
\end{itemize}
Que tem olheiras: \textunderscore a pequena acordou muito olheirada\textunderscore .
\section{Olheirão}
\begin{itemize}
\item {Grp. gram.:m.}
\end{itemize}
Ôlho grande.
Grande nascente de água.
\section{Olheiras}
\begin{itemize}
\item {Grp. gram.:f. pl.}
\end{itemize}
\begin{itemize}
\item {Proveniência:(De \textunderscore ôlho\textunderscore )}
\end{itemize}
Manchas escuras ou azuladas, em volta dos olhos, indicando geralmente soffrimento phýsico ou moral.
\section{Olheiro}
\begin{itemize}
\item {Grp. gram.:m.}
\end{itemize}
\begin{itemize}
\item {Utilização:T. de Aveiro}
\end{itemize}
\begin{itemize}
\item {Proveniência:(De \textunderscore ôlho\textunderscore )}
\end{itemize}
Aquelle que olha por alguma coisa, o que vigia certos trabalhos.
Informador.
Ponto, donde rebenta a água no solo; nascente de água.
Cachão.
Pequeno lago, mal distinto sob uma fina camada de areia, espalhada pelo vento, entre os médãos da costa.
\section{Olhento}
\begin{itemize}
\item {Grp. gram.:adj.}
\end{itemize}
\begin{itemize}
\item {Proveniência:(De \textunderscore ôlho\textunderscore )}
\end{itemize}
Que tem olhos, poros ou buracos.
\section{Olhetado}
\begin{itemize}
\item {Grp. gram.:m.}
\end{itemize}
\begin{itemize}
\item {Proveniência:(De \textunderscore olhete\textunderscore )}
\end{itemize}
Vara curta da videira, cujos olhos, por serem poucos, deverão rebentar com mais fôrça.
\section{Olhete}
\begin{itemize}
\item {fónica:lhê}
\end{itemize}
\begin{itemize}
\item {Grp. gram.:m.}
\end{itemize}
\begin{itemize}
\item {Utilização:Bras}
\end{itemize}
Pequeno ôlho.
Pequena cavidade em fórma de ôlho, nas articulações dos braços e das pernas.
Peixe marítimo.
\section{Olhiagudo}
\begin{itemize}
\item {Grp. gram.:adj.}
\end{itemize}
Que tem olhar penetrante. Cf. Filinto, IX, 270.
\section{Olhibranco}
\begin{itemize}
\item {Grp. gram.:adj.}
\end{itemize}
O mesmo que \textunderscore olhalvo\textunderscore .
\section{Olhica}
\begin{itemize}
\item {Grp. gram.:m.}
\end{itemize}
\begin{itemize}
\item {Utilização:Prov.}
\end{itemize}
\begin{itemize}
\item {Utilização:alent.}
\end{itemize}
\begin{itemize}
\item {Proveniência:(De \textunderscore ôlho\textunderscore )}
\end{itemize}
Aquelle que espreita.
\section{Olhinegro}
\begin{itemize}
\item {Grp. gram.:adj.}
\end{itemize}
Que tem olhos negros.
\section{Olhipreto}
\begin{itemize}
\item {Grp. gram.:adj.}
\end{itemize}
O mesmo que \textunderscore olhinegro\textunderscore . Cf. Filinto, XIX, 226; XIV, 266.
\section{Olhiridente}
\begin{itemize}
\item {fónica:ri}
\end{itemize}
\begin{itemize}
\item {Grp. gram.:adj.}
\end{itemize}
Que tem olhar alegre. Cf. Filinto, XIII, 220.
\section{Olhirridente}
\begin{itemize}
\item {Grp. gram.:adj.}
\end{itemize}
Que tem olhar alegre. Cf. Filinto, XIII, 220.
\section{Olhitoiro}
\begin{itemize}
\item {Grp. gram.:adj.}
\end{itemize}
Que tem olhar de boi. Cf. Filinto, V, 89.
\section{Olhizaino}
\begin{itemize}
\item {Grp. gram.:m.  e  adj.}
\end{itemize}
\begin{itemize}
\item {Utilização:Pop.}
\end{itemize}
\begin{itemize}
\item {Proveniência:(De \textunderscore ôlho\textunderscore  + \textunderscore zaino\textunderscore )}
\end{itemize}
Zanaga.
\section{Olhizarco}
\begin{itemize}
\item {Grp. gram.:adj.}
\end{itemize}
\begin{itemize}
\item {Proveniência:(De \textunderscore ôlho\textunderscore  + \textunderscore zarco\textunderscore )}
\end{itemize}
Que tem olhos azues claros.
Diz-se do cavallo que tem cada ôlho de sua côr.
\section{Ôlho}
\begin{itemize}
\item {Grp. gram.:m.}
\end{itemize}
\begin{itemize}
\item {Utilização:T. de Alcanena}
\end{itemize}
\begin{itemize}
\item {Utilização:Chul.}
\end{itemize}
\begin{itemize}
\item {Utilização:Gír.}
\end{itemize}
\begin{itemize}
\item {Grp. gram.:Loc. adv.}
\end{itemize}
\begin{itemize}
\item {Utilização:Náut.}
\end{itemize}
\begin{itemize}
\item {Grp. gram.:Loc. adv.}
\end{itemize}
\begin{itemize}
\item {Grp. gram.:Loc.}
\end{itemize}
\begin{itemize}
\item {Utilização:fam.}
\end{itemize}
\begin{itemize}
\item {Grp. gram.:Pl.}
\end{itemize}
\begin{itemize}
\item {Utilização:Fam.}
\end{itemize}
\begin{itemize}
\item {Proveniência:(Do lat. \textunderscore oculus\textunderscore )}
\end{itemize}
Órgão da vista.
Vista.
Percepção.
Claridade.
Aquillo que illumina ou esclarece.
Orifício circular ou oval.
Aro de qualquer ferramenta, como enxada, enxó, martelo, etc., por onde se enfia o cabo.
Batoque.
Orifício, por onde se extrai o vinho dos tonéis, pipas, etc.
Nuvem carregada e negra.
Olhal.
Ocello.
Objecto semelhante ao ôlho humano.
Poro.
Botão ou rebento das plantas.
Cada um dos pontos, em que as batatas e outros tubérculos grelam.
Porção de qualquer casca, que serviu num tanque de curtimenta.
Orifício do ânus.
Tostão.
\textunderscore Ôlho de água\textunderscore , ponto donde surge ou rebenta uma nascente de água.
Sorvedoiro marítimo, causado pelo redemoínho da água.
\textunderscore Ôlho nu\textunderscore , vista desarmada ou exercida sem auxílio de qualquer instrumento óptico.
\textunderscore Ôlho da Providência\textunderscore , a providência divina, o cuidado de Deus pelas suas criaturas.
\textunderscore Ôlho vivo\textunderscore , finura, intelligência, percepção fácil.
\textunderscore A ôlho\textunderscore , o mesmo que a \textunderscore olhos vistos\textunderscore :«\textunderscore ...a fazenda dos Pimentas ia perecendo a ôlho.\textunderscore »Camillo, \textunderscore Retr. de Ricard.\textunderscore , 11.
\textunderscore Estar a ôlho\textunderscore , diz-se da âncora, quando o anete apparece á superfície da água.
\textunderscore Dar de ôlho\textunderscore , piscar os olhos, para communicar particularmente qualquer ideia.
\textunderscore Têr debaixo de ôlho\textunderscore , não desviar a attenção e o cuidado de.
\textunderscore Trazer de ôlho\textunderscore , vigiar, estar acautelado a respeito de. Cf. Camillo, \textunderscore Enjeitada\textunderscore , 55.
\textunderscore A ôlho\textunderscore , visivelmente; a olhos vistos:«\textunderscore prosperar a ôlho.\textunderscore »Camillo, \textunderscore Quéda\textunderscore , 10. (2.^a ed., 1887).
\textunderscore Pregar ôlho\textunderscore , dormir.
\textunderscore Têr lume no ôlho\textunderscore , sêr esperto.
Luneta; óculos.
\textunderscore A ôlhos vistos\textunderscore , claramente, evidentemente.
\textunderscore A ôlhos visto\textunderscore , observado perfeitamente. Cf. Figueiredo, \textunderscore Liç. Prát.\textunderscore 
\textunderscore Saltar aos ôlhos\textunderscore , sêr evidente, incontestável.
\section{Ôlho-branco}
\begin{itemize}
\item {Grp. gram.:m.}
\end{itemize}
Peixe plagióstomo, pardo-acinzado por cima, e branco por baixo.
\section{Ôlho-cia}
\begin{itemize}
\item {Grp. gram.:m.}
\end{itemize}
Casta de uva de Tôrres Vedras.
\section{Ôlho-de-bode}
\begin{itemize}
\item {Grp. gram.:m.}
\end{itemize}
\begin{itemize}
\item {Utilização:T. de Setúbal}
\end{itemize}
Veio de água na praia. Cf. \textunderscore Museu Techn.\textunderscore , 83.
\section{Ôlho-de-boi}
\begin{itemize}
\item {Grp. gram.:m.}
\end{itemize}
\begin{itemize}
\item {Utilização:Gír.}
\end{itemize}
Peixe de Portugal e do Brasil.
Cruzado novo.
Planta, o mesmo que \textunderscore buphtalmo\textunderscore .
Abertura circular ou ellíptica, nos tectos ou paredes, para dar luz ao interior do edifício; clarabóia.
Variedade de maçan, o mesmo que \textunderscore baionesa\textunderscore .
\section{Ôlho-de-chede}
\begin{itemize}
\item {Grp. gram.:m.}
\end{itemize}
Casta de uva, na região do Doiro.
\section{Ôlho-de-gallo}
\begin{itemize}
\item {Grp. gram.:m.}
\end{itemize}
Casta de uva.
\section{Ôlho-de-gato}
\begin{itemize}
\item {Grp. gram.:m.}
\end{itemize}
O mesmo que \textunderscore bonduque\textunderscore .
\section{Ôlho-de-lebre}
\begin{itemize}
\item {Grp. gram.:m.}
\end{itemize}
Casta de uva branca, extremenha.
\section{Ôlho-de-mocho}
\begin{itemize}
\item {Grp. gram.:m.}
\end{itemize}
Planta forraginosa, (\textunderscore tolpis barbata\textunderscore , Gart.).
\section{Ôlho-de-mosquito}
\begin{itemize}
\item {Grp. gram.:m.}
\end{itemize}
\begin{itemize}
\item {Utilização:Bras. de Minas}
\end{itemize}
Diamante de pouco pêso e de pouco valor.
\section{Ôlho-de-pargo}
\begin{itemize}
\item {Grp. gram.:m.}
\end{itemize}
Casta de uva de Azeitão.
\section{Ôlho-de-perdiz}
\begin{itemize}
\item {Grp. gram.:m.}
\end{itemize}
Pequeno callo redondo.
\section{Ôlho-de-santa-luzia}
\begin{itemize}
\item {Grp. gram.:m.}
\end{itemize}
O mesmo que \textunderscore trapoeraba\textunderscore .
\section{Ôlho-de-sapo}
\begin{itemize}
\item {Grp. gram.:m.}
\end{itemize}
\begin{itemize}
\item {Grp. gram.:Adj.}
\end{itemize}
\begin{itemize}
\item {Utilização:T. de Turquel}
\end{itemize}
Casta de uva.
O mesmo que \textunderscore granuloso\textunderscore , (falando-se de um terreno).
\section{Ôlho-de-vidro}
\begin{itemize}
\item {Grp. gram.:m.}
\end{itemize}
\begin{itemize}
\item {Utilização:Bras. do N}
\end{itemize}
Espécie de abelha, que fórma o seu ninho debaixo da terra.
\section{Ôlho-meirinho}
\begin{itemize}
\item {Grp. gram.:m.}
\end{itemize}
\begin{itemize}
\item {Utilização:Prov.}
\end{itemize}
Remoínho de água, no rio.
Nascente de água em meio de um campo, no inverno.
(Por \textunderscore ôlho-marinho\textunderscore )
\section{Olhora}
\begin{itemize}
\item {Grp. gram.:adv.}
\end{itemize}
\begin{itemize}
\item {Utilização:T. de Ílhavo}
\end{itemize}
Talvez por \textunderscore olhe ora\textunderscore , olhe agora.
\section{Ôlho-rapado}
\begin{itemize}
\item {Grp. gram.:m.}
\end{itemize}
Variedade de pêra, de fórma esquisita, por têr o ôlho completamente rapado ou raso.
\section{Ôlho-roxo}
\begin{itemize}
\item {Grp. gram.:m.}
\end{itemize}
\begin{itemize}
\item {Utilização:Bras}
\end{itemize}
Espécie de mandioca, de raíz comprida.
\section{Olhos-do-diabo}
\begin{itemize}
\item {Grp. gram.:m. pl.}
\end{itemize}
\begin{itemize}
\item {Utilização:Bras}
\end{itemize}
O mesmo que \textunderscore adónis-da-itália\textunderscore .
\section{Olhudo}
\begin{itemize}
\item {Grp. gram.:adj.}
\end{itemize}
\begin{itemize}
\item {Grp. gram.:M.}
\end{itemize}
Que tem olhos grandes.
Peixe, da fam. dos pércidas.
\section{Olíbano}
\begin{itemize}
\item {Grp. gram.:m.}
\end{itemize}
\begin{itemize}
\item {Proveniência:(Do lat. \textunderscore oleum\textunderscore  + \textunderscore Libanus\textunderscore , n. p.)}
\end{itemize}
Goma resina, usada outrora como vulnerária.
Espécie de incenso.
\section{Olifante}
\begin{itemize}
\item {Grp. gram.:m.}
\end{itemize}
\begin{itemize}
\item {Utilização:Mús.}
\end{itemize}
\begin{itemize}
\item {Utilização:Ant.}
\end{itemize}
\begin{itemize}
\item {Proveniência:(Fr. \textunderscore olifant\textunderscore , deformação de \textunderscore elephant\textunderscore )}
\end{itemize}
Corneta, usada na Idade-Média e feita de um dente de elephante.
\section{Oligacto}
\begin{itemize}
\item {Grp. gram.:m.}
\end{itemize}
Gênero de plantas, da fam. das compostas.
\section{Oligandra}
\begin{itemize}
\item {Grp. gram.:f.}
\end{itemize}
\begin{itemize}
\item {Proveniência:(Do gr. \textunderscore oligos\textunderscore  + \textunderscore aner\textunderscore )}
\end{itemize}
Gênero de plantas brasileiras, da fam. das compostas.
\section{Oligantera}
\begin{itemize}
\item {Grp. gram.:f.}
\end{itemize}
\begin{itemize}
\item {Proveniência:(Do gr. \textunderscore oligos\textunderscore  + \textunderscore antheros\textunderscore )}
\end{itemize}
Gênero de plantas herbáceas, chenopodiáces.
\section{Oliganthera}
\begin{itemize}
\item {Grp. gram.:f.}
\end{itemize}
\begin{itemize}
\item {Proveniência:(Do gr. \textunderscore oligos\textunderscore  + \textunderscore antheros\textunderscore )}
\end{itemize}
Gênero de plantas herbáceas, chenopodiáces.
\section{Oligarca}
\begin{itemize}
\item {Grp. gram.:f.}
\end{itemize}
\begin{itemize}
\item {Proveniência:(Do gr. \textunderscore oligos\textunderscore  + \textunderscore arkhe\textunderscore )}
\end{itemize}
Sectário da oligarquia.
Aquele que faz parte de uma oligarquia.
\section{Oligarcha}
\begin{itemize}
\item {fónica:ca}
\end{itemize}
\begin{itemize}
\item {Grp. gram.:f.}
\end{itemize}
\begin{itemize}
\item {Proveniência:(Do gr. \textunderscore oligos\textunderscore  + \textunderscore arkhe\textunderscore )}
\end{itemize}
Sectário da oligarchia.
Aquelle que faz parte de uma oligarchia.
\section{Oligarchia}
\begin{itemize}
\item {fónica:qui}
\end{itemize}
\begin{itemize}
\item {Grp. gram.:f.}
\end{itemize}
\begin{itemize}
\item {Utilização:Fig.}
\end{itemize}
\begin{itemize}
\item {Proveniência:(De \textunderscore oligarcha\textunderscore )}
\end{itemize}
Fórma de govêrno, em que o poder está na mão de poucas pessôas ou de poucas famílias.
Preponderância de um pequeno número de pessôas nos negócios públicos.
\section{Oligarchicamente}
\begin{itemize}
\item {fónica:qui}
\end{itemize}
\begin{itemize}
\item {Grp. gram.:adv.}
\end{itemize}
De modo oligárchico; á maneira de oligarchia.
\section{Oligárchico}
\begin{itemize}
\item {fónica:qui}
\end{itemize}
\begin{itemize}
\item {Grp. gram.:adj.}
\end{itemize}
Relativo á oligarchia.
Que tem o carácter de oligarchia.
\section{Oligarquia}
\begin{itemize}
\item {Grp. gram.:f.}
\end{itemize}
\begin{itemize}
\item {Utilização:Fig.}
\end{itemize}
\begin{itemize}
\item {Proveniência:(De \textunderscore oligarca\textunderscore )}
\end{itemize}
Fórma de govêrno, em que o poder está na mão de poucas pessôas ou de poucas famílias.
Preponderância de um pequeno número de pessôas nos negócios públicos.
\section{Oligarquicamente}
\begin{itemize}
\item {Grp. gram.:adv.}
\end{itemize}
De modo oligárquico; á maneira de oligarquia.
\section{Oligárquico}
\begin{itemize}
\item {Grp. gram.:adj.}
\end{itemize}
Relativo á oligarquia.
Que tem o carácter de oligarquia.
\section{Oligarrena}
\begin{itemize}
\item {Grp. gram.:f.}
\end{itemize}
Gênero de plantas epacrídeas.
\section{Oligarrhena}
\begin{itemize}
\item {Grp. gram.:f.}
\end{itemize}
Gênero de plantas epacrídeas
\section{Oligístico}
\begin{itemize}
\item {Grp. gram.:adj.}
\end{itemize}
O mesmo que \textunderscore oligisto\textunderscore .
\section{Oligisto}
\begin{itemize}
\item {Grp. gram.:adj.}
\end{itemize}
\begin{itemize}
\item {Utilização:Miner.}
\end{itemize}
\begin{itemize}
\item {Grp. gram.:M.}
\end{itemize}
\begin{itemize}
\item {Proveniência:(Gr. \textunderscore olígistos\textunderscore )}
\end{itemize}
Diz-se do ferro pouco rico em substância metállica.
Mineral pouco rico em metal.
\section{Oligoblenia}
\begin{itemize}
\item {Grp. gram.:f.}
\end{itemize}
\begin{itemize}
\item {Utilização:Med.}
\end{itemize}
\begin{itemize}
\item {Proveniência:(Do gr. \textunderscore oligos\textunderscore  + \textunderscore blenna\textunderscore )}
\end{itemize}
Falta de secreção mucosa.
\section{Oligoblennia}
\begin{itemize}
\item {Grp. gram.:f.}
\end{itemize}
\begin{itemize}
\item {Utilização:Med.}
\end{itemize}
\begin{itemize}
\item {Proveniência:(Do gr. \textunderscore oligos\textunderscore  + \textunderscore blenna\textunderscore )}
\end{itemize}
Falta de secreção mucosa.
\section{Oligocarpo}
\begin{itemize}
\item {Grp. gram.:m.}
\end{itemize}
\begin{itemize}
\item {Proveniência:(Do gr. \textunderscore oligos\textunderscore  + \textunderscore karpos\textunderscore )}
\end{itemize}
Gênero de plantas, da fam. das compostas.
\section{Oligocênico}
\begin{itemize}
\item {Grp. gram.:adj.}
\end{itemize}
O mesmo que \textunderscore oligoceno\textunderscore .
\section{Oligoceno}
\begin{itemize}
\item {Grp. gram.:adj.}
\end{itemize}
\begin{itemize}
\item {Utilização:Geol.}
\end{itemize}
\begin{itemize}
\item {Proveniência:(Do gr. \textunderscore oligos\textunderscore  + \textunderscore kainos\textunderscore )}
\end{itemize}
Diz-se do terreno que, segundo Beyrich, constitue uma secção entre o eoceno e o mioceno.
\section{Oligócero}
\begin{itemize}
\item {Grp. gram.:m.}
\end{itemize}
\begin{itemize}
\item {Proveniência:(Do gr. \textunderscore oligos\textunderscore  + \textunderscore keras\textunderscore )}
\end{itemize}
Gênero de insectos coleópteros.
\section{Oligocholia}
\begin{itemize}
\item {fónica:co}
\end{itemize}
\begin{itemize}
\item {Grp. gram.:f.}
\end{itemize}
\begin{itemize}
\item {Utilização:Med.}
\end{itemize}
\begin{itemize}
\item {Proveniência:(Do gr. \textunderscore oligos\textunderscore  + \textunderscore khole\textunderscore )}
\end{itemize}
Secreção pouco abundante de bílis.
\section{Oligóchrono}
\begin{itemize}
\item {Grp. gram.:adj.}
\end{itemize}
\begin{itemize}
\item {Proveniência:(Do gr. \textunderscore oligos\textunderscore  + \textunderscore khronos\textunderscore )}
\end{itemize}
Que vive ou subsiste por pouco tempo.
\section{Oligochronómetro}
\begin{itemize}
\item {Grp. gram.:m.}
\end{itemize}
\begin{itemize}
\item {Proveniência:(De \textunderscore oligos\textunderscore  gr. + \textunderscore chronómetro\textunderscore )}
\end{itemize}
Instrumento, para medir pequenas fracções de tempo.
\section{Oligochylia}
\begin{itemize}
\item {fónica:qui}
\end{itemize}
\begin{itemize}
\item {Grp. gram.:f.}
\end{itemize}
\begin{itemize}
\item {Utilização:Med.}
\end{itemize}
\begin{itemize}
\item {Proveniência:(Do gr. \textunderscore oligos\textunderscore  + \textunderscore khulos\textunderscore )}
\end{itemize}
Falta de suco nutritivo.
\section{Oligochylo}
\begin{itemize}
\item {fónica:qui}
\end{itemize}
\begin{itemize}
\item {Grp. gram.:adj.}
\end{itemize}
\begin{itemize}
\item {Proveniência:(Do gr. \textunderscore oligos\textunderscore  + \textunderscore khulos\textunderscore )}
\end{itemize}
Diz-se das substâncias alimentares pouco nutritivas.
\section{Oligocitemia}
\begin{itemize}
\item {Grp. gram.:f.}
\end{itemize}
\begin{itemize}
\item {Utilização:Med.}
\end{itemize}
\begin{itemize}
\item {Proveniência:(Do gr. \textunderscore oligos\textunderscore  + \textunderscore kutos\textunderscore  + \textunderscore haima\textunderscore )}
\end{itemize}
Deminuição do número dos glóbulos do sangue.
\section{Oligoclásio}
\begin{itemize}
\item {Grp. gram.:m.}
\end{itemize}
\begin{itemize}
\item {Utilização:Miner.}
\end{itemize}
\begin{itemize}
\item {Proveniência:(Do gr. \textunderscore oligos\textunderscore  + \textunderscore klasis\textunderscore )}
\end{itemize}
Mineral, composto de sílica, alumina, peróxido de ferro, soda, potassa, cal e magnésia.
\section{Oligocolia}
\begin{itemize}
\item {Grp. gram.:f.}
\end{itemize}
\begin{itemize}
\item {Utilização:Med.}
\end{itemize}
\begin{itemize}
\item {Proveniência:(Do gr. \textunderscore oligos\textunderscore  + \textunderscore khole\textunderscore )}
\end{itemize}
Secreção pouco abundante de bílis.
\section{Oligócrono}
\begin{itemize}
\item {Grp. gram.:adj.}
\end{itemize}
\begin{itemize}
\item {Proveniência:(Do gr. \textunderscore oligos\textunderscore  + \textunderscore khronos\textunderscore )}
\end{itemize}
Que vive ou subsiste por pouco tempo.
\section{Oligocronómetro}
\begin{itemize}
\item {Grp. gram.:m.}
\end{itemize}
\begin{itemize}
\item {Proveniência:(De \textunderscore oligos\textunderscore  gr. + \textunderscore cronómetro\textunderscore )}
\end{itemize}
Instrumento, para medir pequenas fracções de tempo.
\section{Oligocracia}
\begin{itemize}
\item {Utilização:Deprec.}
\end{itemize}
\begin{itemize}
\item {Proveniência:(Do gr. \textunderscore oligos\textunderscore  + \textunderscore krateía\textunderscore )}
\end{itemize}
Aristocracia pouco numerosa.
\section{Oligocrático}
\begin{itemize}
\item {Grp. gram.:adj.}
\end{itemize}
Relativo á oligocracia.
\section{Oligocythemia}
\begin{itemize}
\item {Grp. gram.:f.}
\end{itemize}
\begin{itemize}
\item {Utilização:Med.}
\end{itemize}
\begin{itemize}
\item {Proveniência:(Do gr. \textunderscore oligos\textunderscore  + \textunderscore kutos\textunderscore  + \textunderscore haima\textunderscore )}
\end{itemize}
Deminuição do número dos glóbulos do sangue.
\section{Oligodacria}
\begin{itemize}
\item {Grp. gram.:f.}
\end{itemize}
\begin{itemize}
\item {Utilização:Med.}
\end{itemize}
\begin{itemize}
\item {Proveniência:(Do gr. \textunderscore oligos\textunderscore  + \textunderscore dakrus\textunderscore )}
\end{itemize}
Secreção pouco abundante de lágrimas.
\section{Oligódora}
\begin{itemize}
\item {Grp. gram.:f.}
\end{itemize}
\begin{itemize}
\item {Proveniência:(Do gr. \textunderscore oligos\textunderscore  + \textunderscore doron\textunderscore )}
\end{itemize}
Gênero de plantas, da fam. das compostas.
\section{Oligoemía}
\begin{itemize}
\item {fónica:go-e}
\end{itemize}
\begin{itemize}
\item {Grp. gram.:f.}
\end{itemize}
\begin{itemize}
\item {Proveniência:(Do gr. \textunderscore oligos\textunderscore  + \textunderscore haima\textunderscore )}
\end{itemize}
O mesmo que \textunderscore anemia\textunderscore .
\section{Oligofarmácia}
\begin{itemize}
\item {Grp. gram.:f.}
\end{itemize}
\begin{itemize}
\item {Proveniência:(Do gr. \textunderscore oligos\textunderscore  + \textunderscore pharmakeía\textunderscore )}
\end{itemize}
Pequena farmácia, farmácia que se compõe de um limitado número de medicamentos.
\section{Oligofármaco}
\begin{itemize}
\item {Grp. gram.:adj.}
\end{itemize}
\begin{itemize}
\item {Proveniência:(Do gr. \textunderscore oligos\textunderscore  + \textunderscore pharmakon\textunderscore )}
\end{itemize}
Que segue o método farmaceutico muito simplificado.
\section{Oligofilo}
\begin{itemize}
\item {Grp. gram.:adj.}
\end{itemize}
\begin{itemize}
\item {Utilização:Bot.}
\end{itemize}
\begin{itemize}
\item {Proveniência:(Do gr. \textunderscore oligos\textunderscore  + \textunderscore phullon\textunderscore )}
\end{itemize}
Que tem poucas fôlhas.
\section{Oligohemía}
\begin{itemize}
\item {Grp. gram.:f.}
\end{itemize}
\begin{itemize}
\item {Proveniência:(Do gr. \textunderscore oligos\textunderscore  + \textunderscore haima\textunderscore )}
\end{itemize}
O mesmo que \textunderscore anemia\textunderscore .
\section{Oligohydria}
\begin{itemize}
\item {Grp. gram.:f.}
\end{itemize}
\begin{itemize}
\item {Utilização:Med.}
\end{itemize}
\begin{itemize}
\item {Proveniência:(Do gr. \textunderscore oligos\textunderscore  + \textunderscore hudor\textunderscore )}
\end{itemize}
Raridade ou falta de suor.
\section{Oligoidria}
\begin{itemize}
\item {fónica:go-i}
\end{itemize}
\begin{itemize}
\item {Grp. gram.:f.}
\end{itemize}
\begin{itemize}
\item {Utilização:Med.}
\end{itemize}
\begin{itemize}
\item {Proveniência:(Do gr. \textunderscore oligos\textunderscore  + \textunderscore hudor\textunderscore )}
\end{itemize}
Raridade ou falta de suor.
\section{Oligomania}
\begin{itemize}
\item {Grp. gram.:f.}
\end{itemize}
\begin{itemize}
\item {Utilização:Med.}
\end{itemize}
\begin{itemize}
\item {Proveniência:(Do gr. \textunderscore oligos\textunderscore  + \textunderscore mania\textunderscore )}
\end{itemize}
Mania restringida a um certo número de ideias.
\section{Oligonéride}
\begin{itemize}
\item {Grp. gram.:f.}
\end{itemize}
Gênero de plantas resedáceas.
\section{Oligopharmácia}
\begin{itemize}
\item {Grp. gram.:f.}
\end{itemize}
\begin{itemize}
\item {Proveniência:(Do gr. \textunderscore oligos\textunderscore  + \textunderscore pharmakeía\textunderscore )}
\end{itemize}
Pequena pharmácia, pharmácia que se compõe de um limitado número de medicamentos.
\section{Oligophármaco}
\begin{itemize}
\item {Grp. gram.:adj.}
\end{itemize}
\begin{itemize}
\item {Proveniência:(Do gr. \textunderscore oligos\textunderscore  + \textunderscore pharmakon\textunderscore )}
\end{itemize}
Que segue o méthodo pharmaceutico muito simplificado.
\section{Oligophyllo}
\begin{itemize}
\item {Grp. gram.:adj.}
\end{itemize}
\begin{itemize}
\item {Utilização:Bot.}
\end{itemize}
\begin{itemize}
\item {Proveniência:(Do gr. \textunderscore oligos\textunderscore  + \textunderscore phullon\textunderscore )}
\end{itemize}
Que tem poucas fôlhas.
\section{Oligopionia}
\begin{itemize}
\item {Grp. gram.:f.}
\end{itemize}
\begin{itemize}
\item {Utilização:Med.}
\end{itemize}
Falta de gordura; magreza.
\section{Oligópode}
\begin{itemize}
\item {Grp. gram.:m.}
\end{itemize}
\begin{itemize}
\item {Proveniência:(Do gr. \textunderscore oligos\textunderscore  + \textunderscore pous\textunderscore , \textunderscore podos\textunderscore )}
\end{itemize}
Gênero de peixes acanthopterýgios.
\section{Oligoposia}
\begin{itemize}
\item {Grp. gram.:f.}
\end{itemize}
\begin{itemize}
\item {Utilização:Med.}
\end{itemize}
\begin{itemize}
\item {Proveniência:(Gr. \textunderscore oligoposia\textunderscore )}
\end{itemize}
Deminuição da sêde.
Deminuição na quantidade de bebidas.
\section{Oligopsiquia}
\begin{itemize}
\item {Grp. gram.:f.}
\end{itemize}
\begin{itemize}
\item {Utilização:Med.}
\end{itemize}
\begin{itemize}
\item {Proveniência:(Do gr. \textunderscore oligos\textunderscore  + \textunderscore psukhe\textunderscore )}
\end{itemize}
Imbecilidade; inteligência escassa.
\section{Oligopsychia}
\begin{itemize}
\item {fónica:qui}
\end{itemize}
\begin{itemize}
\item {Grp. gram.:f.}
\end{itemize}
\begin{itemize}
\item {Utilização:Med.}
\end{itemize}
\begin{itemize}
\item {Proveniência:(Do gr. \textunderscore oligos\textunderscore  + \textunderscore psukhe\textunderscore )}
\end{itemize}
Imbecilidade; intelligência escassa.
\section{Oligoquilia}
\begin{itemize}
\item {Grp. gram.:f.}
\end{itemize}
\begin{itemize}
\item {Utilização:Med.}
\end{itemize}
\begin{itemize}
\item {Proveniência:(Do gr. \textunderscore oligos\textunderscore  + \textunderscore khulos\textunderscore )}
\end{itemize}
Falta de suco nutritivo.
\section{Oligoquilo}
\begin{itemize}
\item {Grp. gram.:adj.}
\end{itemize}
\begin{itemize}
\item {Proveniência:(Do gr. \textunderscore oligos\textunderscore  + \textunderscore khulos\textunderscore )}
\end{itemize}
Diz-se das substâncias alimentares pouco nutritivas.
\section{Olígoro}
\begin{itemize}
\item {Grp. gram.:m.}
\end{itemize}
Gênero de insectos coleópteros heterómeros.
\section{Oligosialia}
\begin{itemize}
\item {fónica:si}
\end{itemize}
\begin{itemize}
\item {Grp. gram.:f.}
\end{itemize}
\begin{itemize}
\item {Utilização:Med.}
\end{itemize}
\begin{itemize}
\item {Proveniência:(Do gr. \textunderscore oligos\textunderscore  + \textunderscore sialon\textunderscore )}
\end{itemize}
Escassa secreção de saliva.
\section{Oligospermia}
\begin{itemize}
\item {Grp. gram.:f.}
\end{itemize}
\begin{itemize}
\item {Utilização:Med.}
\end{itemize}
\begin{itemize}
\item {Proveniência:(Do gr. \textunderscore oligos\textunderscore  + \textunderscore sperma\textunderscore )}
\end{itemize}
Escassa secreção de esperma.
\section{Oligospermo}
\begin{itemize}
\item {Grp. gram.:adj.}
\end{itemize}
\begin{itemize}
\item {Utilização:Bot.}
\end{itemize}
\begin{itemize}
\item {Proveniência:(Do gr. \textunderscore oligos\textunderscore  + \textunderscore sperma\textunderscore )}
\end{itemize}
Que tem poucas sementes.
\section{Oligossialia}
\begin{itemize}
\item {Grp. gram.:f.}
\end{itemize}
\begin{itemize}
\item {Utilização:Med.}
\end{itemize}
\begin{itemize}
\item {Proveniência:(Do gr. \textunderscore oligos\textunderscore  + \textunderscore sialon\textunderscore )}
\end{itemize}
Escassa secreção de saliva.
\section{Olígota}
\begin{itemize}
\item {Grp. gram.:f.}
\end{itemize}
Gênero de insectos coleópteros tetrâmeros.
\section{Oligotrichia}
\begin{itemize}
\item {fónica:qui}
\end{itemize}
\begin{itemize}
\item {Grp. gram.:f.}
\end{itemize}
\begin{itemize}
\item {Proveniência:(Do gr. \textunderscore oligos\textunderscore  + \textunderscore trikhos\textunderscore )}
\end{itemize}
Falta de cabello.
\section{Oligotriquia}
\begin{itemize}
\item {Grp. gram.:f.}
\end{itemize}
\begin{itemize}
\item {Proveniência:(Do gr. \textunderscore oligos\textunderscore  + \textunderscore trikhos\textunderscore )}
\end{itemize}
Falta de cabelo.
\section{Oligotrofia}
\begin{itemize}
\item {Grp. gram.:f.}
\end{itemize}
\begin{itemize}
\item {Utilização:Med.}
\end{itemize}
\begin{itemize}
\item {Proveniência:(Do gr. \textunderscore oligos\textunderscore  + \textunderscore trephein\textunderscore )}
\end{itemize}
Deminuição da nutrição das partes do corpo.
\section{Oligotrophia}
\begin{itemize}
\item {Grp. gram.:f.}
\end{itemize}
\begin{itemize}
\item {Utilização:Med.}
\end{itemize}
\begin{itemize}
\item {Proveniência:(Do gr. \textunderscore oligos\textunderscore  + \textunderscore trephein\textunderscore )}
\end{itemize}
Deminuição da nutrição das partes do corpo.
\section{Oliguresia}
\begin{itemize}
\item {Grp. gram.:f.}
\end{itemize}
\begin{itemize}
\item {Utilização:Med.}
\end{itemize}
\begin{itemize}
\item {Proveniência:(Do gr. \textunderscore oligos\textunderscore  + \textunderscore ouron\textunderscore )}
\end{itemize}
Secreção pouco abundante de urina.
\section{Oliguria}
\begin{itemize}
\item {Grp. gram.:f.}
\end{itemize}
\begin{itemize}
\item {Utilização:Med.}
\end{itemize}
\begin{itemize}
\item {Proveniência:(Do gr. \textunderscore oligos\textunderscore  + \textunderscore ouron\textunderscore )}
\end{itemize}
Deminuição da quantidade de urina excretada.
\section{Oligúrico}
\begin{itemize}
\item {Grp. gram.:adj.}
\end{itemize}
\begin{itemize}
\item {Grp. gram.:M.}
\end{itemize}
Relativo á oliguria.
Aquelle que padece oliguria.
\section{Olina}
\begin{itemize}
\item {Grp. gram.:f.}
\end{itemize}
Gênero de insectos dípteros.
\section{Olineas}
\begin{itemize}
\item {Grp. gram.:f. pl.}
\end{itemize}
Família de plantas, intermediárias ás melastomáceas e myrtáceas.
\section{Olintolite}
\begin{itemize}
\item {Grp. gram.:f.}
\end{itemize}
Espécie de grossulária.
\section{Ólio}
\begin{itemize}
\item {Grp. gram.:m.}
\end{itemize}
Gênero de grandes aranhas africanas.
\section{Olisiponense}
\begin{itemize}
\item {Grp. gram.:adj.}
\end{itemize}
\begin{itemize}
\item {Proveniência:(Do lat. \textunderscore Olisipo\textunderscore , \textunderscore Olisiponis\textunderscore , n. p.)}
\end{itemize}
Relativo a Lisbôa.
\section{Olistena}
\begin{itemize}
\item {Grp. gram.:f.}
\end{itemize}
Gênero de insectos coleópteros heterómeros.
\section{Olistero}
\begin{itemize}
\item {Grp. gram.:m.}
\end{itemize}
Gênero de insectos coleópteros pentâmeros.
\section{Olisthena}
\begin{itemize}
\item {Grp. gram.:f.}
\end{itemize}
Gênero de insectos coleópteros heterómeros.
\section{Olisthero}
\begin{itemize}
\item {Grp. gram.:m.}
\end{itemize}
Gênero de insectos coleópteros pentâmeros.
\section{Olísthopo}
\begin{itemize}
\item {Grp. gram.:m.}
\end{itemize}
Gênero de insectos coleópteros pentâmeros.
\section{Olístopo}
\begin{itemize}
\item {Grp. gram.:m.}
\end{itemize}
Gênero de insectos coleópteros pentâmeros.
\section{Oliva}
\begin{itemize}
\item {Grp. gram.:f.}
\end{itemize}
\begin{itemize}
\item {Utilização:Poét.}
\end{itemize}
\begin{itemize}
\item {Grp. gram.:Pl.}
\end{itemize}
\begin{itemize}
\item {Proveniência:(Lat. \textunderscore oliva\textunderscore )}
\end{itemize}
O mesmo que \textunderscore azeitona\textunderscore ^1.
Oliveira.
Ornatos architectónicos em fórma de azeitonas.
Parótidas do cavallo.
\section{Oliva}
\begin{itemize}
\item {Grp. gram.:f.}
\end{itemize}
Gênero de molluscos gasterópodes.
\section{Oliváceo}
\begin{itemize}
\item {Grp. gram.:adj.}
\end{itemize}
\begin{itemize}
\item {Proveniência:(De \textunderscore oliva\textunderscore )}
\end{itemize}
Que é da côr da azeitona.
\section{Olival}
\begin{itemize}
\item {Grp. gram.:m.}
\end{itemize}
\begin{itemize}
\item {Proveniência:(De \textunderscore oliva\textunderscore )}
\end{itemize}
Terreno, onde crescem oliveiras.
\section{Olaria}
\begin{itemize}
\item {Grp. gram.:f.}
\end{itemize}
\begin{itemize}
\item {Proveniência:(Do lat. \textunderscore olla\textunderscore )}
\end{itemize}
Fábrica de loiça de barro.
\section{Oleiro}
\begin{itemize}
\item {Grp. gram.:m.}
\end{itemize}
\begin{itemize}
\item {Proveniência:(De \textunderscore ola\textunderscore )}
\end{itemize}
Aquele que trabalha em olaria.
\section{Olimpíada}
\begin{itemize}
\item {Grp. gram.:f.}
\end{itemize}
(V. \textunderscore olimpíade\textunderscore , que é a fórma exacta)
\section{Olimpíade}
\begin{itemize}
\item {Proveniência:(Lat. \textunderscore olympias\textunderscore , \textunderscore olympiadis\textunderscore )}
\end{itemize}
Período de quatro anos, decorrido entre duas celebrações consecutivas dos jogos olímpicos.
\section{Olímpico}
\begin{itemize}
\item {Grp. gram.:adj.}
\end{itemize}
\begin{itemize}
\item {Utilização:Fig.}
\end{itemize}
\begin{itemize}
\item {Proveniência:(Lat. \textunderscore olympicus\textunderscore )}
\end{itemize}
Relativo ao Olimpo.
Relativo á cidade de Olímpia, que deu o nome aos jogos olímpicos.
Divino; sublime; majestoso.
\section{Olímpio}
\begin{itemize}
\item {Grp. gram.:adj.}
\end{itemize}
\begin{itemize}
\item {Proveniência:(Lat. \textunderscore olympius\textunderscore )}
\end{itemize}
O mesmo que \textunderscore olímpico\textunderscore . Cf. Filinto, XVI, 122.
\section{Olimpo}
\begin{itemize}
\item {Grp. gram.:m.}
\end{itemize}
\begin{itemize}
\item {Utilização:Poét.}
\end{itemize}
\begin{itemize}
\item {Proveniência:(Lat. \textunderscore olympus\textunderscore )}
\end{itemize}
Habitação das divindades do Paganismo.
Céu.
Deuses e deusas do Olimpo.
\section{Olira}
\begin{itemize}
\item {Grp. gram.:f.}
\end{itemize}
\begin{itemize}
\item {Proveniência:(Lat. \textunderscore olyra\textunderscore )}
\end{itemize}
Gênero de plantas gramíneas, que serve de tipo ás olíreas.
\section{Olíreas}
\begin{itemize}
\item {Grp. gram.:f. pl.}
\end{itemize}
Tríbo de plantas gramíneas da América tropical.
\section{Oliva-porphýria}
\begin{itemize}
\item {Grp. gram.:f.}
\end{itemize}
Mollusco gasterópode, do gênero oliva.
\section{Olivar}
\begin{itemize}
\item {Grp. gram.:adj.}
\end{itemize}
\begin{itemize}
\item {Proveniência:(De \textunderscore oliva\textunderscore )}
\end{itemize}
Que tem fórma de azeitona.
\section{Olivário}
\begin{itemize}
\item {Grp. gram.:adj.}
\end{itemize}
O mesmo que \textunderscore olivar\textunderscore .
\section{Olivedo}
\begin{itemize}
\item {fónica:vê}
\end{itemize}
\begin{itemize}
\item {Grp. gram.:m.}
\end{itemize}
\begin{itemize}
\item {Proveniência:(Do lat. \textunderscore olivetum\textunderscore )}
\end{itemize}
O mesmo que \textunderscore olival\textunderscore .
\section{Oliveira}
\begin{itemize}
\item {Grp. gram.:f.}
\end{itemize}
\begin{itemize}
\item {Proveniência:(De \textunderscore oliva\textunderscore )}
\end{itemize}
Gênero de árvores, que serve de typo ás oleáceas.
\section{Oliveiral}
\begin{itemize}
\item {Grp. gram.:m.}
\end{itemize}
\begin{itemize}
\item {Proveniência:(De \textunderscore oliveira\textunderscore )}
\end{itemize}
O mesmo que \textunderscore olival\textunderscore .
\section{Olivel}
\textunderscore m.\textunderscore  (e der.)
Fórma preferível a \textunderscore nível\textunderscore , etc. Cf. \textunderscore Hist. Insulana\textunderscore , II, 34; Resende, \textunderscore Miscell.\textunderscore ; Herculano, etc.
\section{Olíveo}
\begin{itemize}
\item {Grp. gram.:adj.}
\end{itemize}
\begin{itemize}
\item {Proveniência:(De \textunderscore oliva\textunderscore )}
\end{itemize}
Relativo á oliveira.
\section{Olivéria}
\begin{itemize}
\item {Grp. gram.:f.}
\end{itemize}
\begin{itemize}
\item {Proveniência:(De \textunderscore Oliver\textunderscore , n. p.)}
\end{itemize}
Gênero de plantas umbellíferas.
\section{Oliveta}
\begin{itemize}
\item {fónica:vê}
\end{itemize}
\begin{itemize}
\item {Grp. gram.:f.}
\end{itemize}
\begin{itemize}
\item {Utilização:Ant.}
\end{itemize}
Espécie de laqueca.
\section{Olivicultor}
\begin{itemize}
\item {Grp. gram.:m.}
\end{itemize}
Aquelle que se occupa de olivicultura.
\section{Olivicultura}
\begin{itemize}
\item {Grp. gram.:f.}
\end{itemize}
\begin{itemize}
\item {Proveniência:(Do lat. \textunderscore oliva\textunderscore  + \textunderscore cultura\textunderscore )}
\end{itemize}
Cultura de olivaes.
O mesmo que \textunderscore oleicultura\textunderscore .
\section{Olivífero}
\begin{itemize}
\item {Grp. gram.:adj.}
\end{itemize}
\begin{itemize}
\item {Proveniência:(Do lat. \textunderscore oliva\textunderscore  + \textunderscore ferre\textunderscore )}
\end{itemize}
Que produz oliveiras.
\section{Olivila}
\begin{itemize}
\item {Grp. gram.:f.}
\end{itemize}
\begin{itemize}
\item {Utilização:Chím.}
\end{itemize}
Princípio immediato dos vegetaes, descoberto no suco concreto que a oliveira destilla.
\section{Olivina}
\begin{itemize}
\item {Grp. gram.:f.}
\end{itemize}
\begin{itemize}
\item {Utilização:Miner.}
\end{itemize}
\begin{itemize}
\item {Proveniência:(Do lat. \textunderscore olivum\textunderscore )}
\end{itemize}
Variedade de peridoto, de côr azeitonada.
\section{Oliviássico}
\begin{itemize}
\item {Grp. gram.:adj.}
\end{itemize}
Relativo á olivina.
\section{Olla}
\begin{itemize}
\item {Grp. gram.:f.}
\end{itemize}
\begin{itemize}
\item {Utilização:Ant.}
\end{itemize}
\begin{itemize}
\item {Proveniência:(Lat. \textunderscore olla\textunderscore )}
\end{itemize}
O mesmo que \textunderscore ollaria\textunderscore . Cf. \textunderscore Tombo do Estado da Índia\textunderscore , 198, 201 e 202.
O mesmo que \textunderscore panela\textunderscore .
\section{Ollaria}
\begin{itemize}
\item {Grp. gram.:f.}
\end{itemize}
\begin{itemize}
\item {Proveniência:(Do lat. \textunderscore olla\textunderscore )}
\end{itemize}
Fábrica de loiça de barro.
\section{Olleiro}
\begin{itemize}
\item {Grp. gram.:m.}
\end{itemize}
\begin{itemize}
\item {Proveniência:(De \textunderscore olla\textunderscore )}
\end{itemize}
Aquelle que trabalha em ollaria.
\section{Olmafi}
\begin{itemize}
\item {Grp. gram.:m.}
\end{itemize}
\begin{itemize}
\item {Utilização:Ant.}
\end{itemize}
O mesmo que \textunderscore marfim\textunderscore .
\section{Olmedal}
\begin{itemize}
\item {Grp. gram.:m.}
\end{itemize}
\begin{itemize}
\item {Proveniência:(De \textunderscore olmedo\textunderscore )}
\end{itemize}
Terreno, onde crescem olmos.
\section{Olmédia}
\begin{itemize}
\item {Grp. gram.:f.}
\end{itemize}
Gênero de plantas artocárpeas.
\section{Olmedo}
\begin{itemize}
\item {fónica:mê}
\end{itemize}
\begin{itemize}
\item {Grp. gram.:m.}
\end{itemize}
\begin{itemize}
\item {Proveniência:(De \textunderscore olmo\textunderscore )}
\end{itemize}
O mesmo que \textunderscore olmedal\textunderscore .
\section{Olmeira}
\begin{itemize}
\item {Grp. gram.:f.}
\end{itemize}
O mesmo que \textunderscore ulmária\textunderscore .
\section{Olmeiro}
\begin{itemize}
\item {Grp. gram.:m.}
\end{itemize}
O mesmo que \textunderscore olmo\textunderscore .
\section{Olmo}
\begin{itemize}
\item {fónica:ôl}
\end{itemize}
\begin{itemize}
\item {Grp. gram.:m.}
\end{itemize}
\begin{itemize}
\item {Proveniência:(Do lat. \textunderscore ulmus\textunderscore )}
\end{itemize}
Gênero de grandes árvores, que comprehende várias espécies, e cuja madeira é muito applicada em construcções.
\section{Olococo}
\begin{itemize}
\item {Grp. gram.:m.}
\end{itemize}
Grande ave africana, (\textunderscore elotharsus\textunderscore ).
\section{Ológrafo}
\begin{itemize}
\item {Grp. gram.:m.  e  adj.}
\end{itemize}
\begin{itemize}
\item {Utilização:Jur.}
\end{itemize}
\begin{itemize}
\item {Proveniência:(Do gr. \textunderscore holos\textunderscore  + \textunderscore graphein\textunderscore )}
\end{itemize}
Dizia-se do testamento, que era todo escrito pela mão do testador. Cf. F. Borges, \textunderscore Diccion. Jur.\textunderscore 
\section{Ológrapho}
\begin{itemize}
\item {Grp. gram.:m.  e  adj.}
\end{itemize}
\begin{itemize}
\item {Utilização:Jur.}
\end{itemize}
\begin{itemize}
\item {Proveniência:(Do gr. \textunderscore holos\textunderscore  + \textunderscore graphein\textunderscore )}
\end{itemize}
Dizia-se do testamento, que era todo escrito pela mão do testador. Cf. F. Borges, \textunderscore Diccion. Jur.\textunderscore 
\section{Olopetalar}
\begin{itemize}
\item {Grp. gram.:adj.}
\end{itemize}
O mesmo que \textunderscore olopetalário\textunderscore .
\section{Olopetalário}
\begin{itemize}
\item {Grp. gram.:adj.}
\end{itemize}
\begin{itemize}
\item {Utilização:Bot.}
\end{itemize}
\begin{itemize}
\item {Proveniência:(Do gr. \textunderscore holos\textunderscore  + \textunderscore petalon\textunderscore )}
\end{itemize}
Diz-se das flôres, cujos estames, no todo ou em parte, estão, com o pistillo, transformados em pétalas. Cf. De-Candolle.
\section{Olofro}
\begin{itemize}
\item {Grp. gram.:m.}
\end{itemize}
Gênero de insectos coleópteros pentâmeros.
\section{Olophro}
\begin{itemize}
\item {Grp. gram.:m.}
\end{itemize}
Gênero de insectos coleópteros pentâmeros.
\section{Olor}
\begin{itemize}
\item {Grp. gram.:m.}
\end{itemize}
\begin{itemize}
\item {Utilização:Poét.}
\end{itemize}
\begin{itemize}
\item {Proveniência:(Lat. \textunderscore olor\textunderscore )}
\end{itemize}
Cheiro agradável; aroma.
\section{Oloroso}
\begin{itemize}
\item {Grp. gram.:adj.}
\end{itemize}
\begin{itemize}
\item {Utilização:Poét.}
\end{itemize}
Que tem olor.
\section{Olunhaneca}
\begin{itemize}
\item {Grp. gram.:m.}
\end{itemize}
Língua indígena do interior de Mossâmedes.
\section{Olvidar}
\begin{itemize}
\item {Grp. gram.:v. t.}
\end{itemize}
\begin{itemize}
\item {Proveniência:(De \textunderscore olvido\textunderscore )}
\end{itemize}
Esquecer-se de.
Desapprender.
\section{Olvidável}
\begin{itemize}
\item {Grp. gram.:adj.}
\end{itemize}
Que se deve ou se póde olvidar.
\section{Olvido}
\begin{itemize}
\item {Grp. gram.:m.}
\end{itemize}
\begin{itemize}
\item {Utilização:Poét.}
\end{itemize}
\begin{itemize}
\item {Proveniência:(Do lat. \textunderscore oblitus\textunderscore )}
\end{itemize}
Acto ou effeito de olvidar.
Descanso, repoiso.
\section{Olympíada}
\begin{itemize}
\item {Grp. gram.:f.}
\end{itemize}
(V. \textunderscore olympíade\textunderscore , que é a fórma exacta)
\section{Olympíade}
\begin{itemize}
\item {Proveniência:(Lat. \textunderscore olympias\textunderscore , \textunderscore olympiadis\textunderscore )}
\end{itemize}
Período de quatro annos, decorrido entre duas celebrações consecutivas dos jogos olýmpicos.
\section{Olýmpico}
\begin{itemize}
\item {Grp. gram.:adj.}
\end{itemize}
\begin{itemize}
\item {Utilização:Fig.}
\end{itemize}
\begin{itemize}
\item {Proveniência:(Lat. \textunderscore olympicus\textunderscore )}
\end{itemize}
Relativo ao Olympo.
Relativo á cidade de Olýmpia, que deu o nome aos jogos olýmpicos.
Divino; sublime; majestoso.
\section{Olýmpio}
\begin{itemize}
\item {Grp. gram.:adj.}
\end{itemize}
\begin{itemize}
\item {Proveniência:(Lat. \textunderscore olympius\textunderscore )}
\end{itemize}
O mesmo que \textunderscore olýmpico\textunderscore . Cf. Filinto, XVI, 122.
\section{Olympo}
\begin{itemize}
\item {Grp. gram.:m.}
\end{itemize}
\begin{itemize}
\item {Utilização:Poét.}
\end{itemize}
\begin{itemize}
\item {Proveniência:(Lat. \textunderscore olympus\textunderscore )}
\end{itemize}
Habitação das divindades do Paganismo.
Céu.
Deuses e deusas do Olympo.
\section{Olyra}
\begin{itemize}
\item {Grp. gram.:f.}
\end{itemize}
\begin{itemize}
\item {Proveniência:(Lat. \textunderscore olyra\textunderscore )}
\end{itemize}
Gênero de plantas gramíneas, que serve de typo ás olýreas.
\section{Olýreas}
\begin{itemize}
\item {Grp. gram.:f. pl.}
\end{itemize}
Tríbo de plantas gramíneas da América tropical.
\section{Omacefalia}
\begin{itemize}
\item {Grp. gram.:f.}
\end{itemize}
Qualidade de omacéfalo.
\section{Omacefaliano}
\begin{itemize}
\item {Grp. gram.:adj.}
\end{itemize}
Que tem omacefalia.
\section{Omacefálico}
\begin{itemize}
\item {Grp. gram.:adj.}
\end{itemize}
Relativo á omacefalia.
\section{Omacéfalo}
\begin{itemize}
\item {Grp. gram.:m.}
\end{itemize}
\begin{itemize}
\item {Proveniência:(Do gr. \textunderscore omos\textunderscore  + \textunderscore kephale\textunderscore )}
\end{itemize}
Monstro, que tem a cabeça mal conformada e carece de braços.
\section{Omacephalia}
\begin{itemize}
\item {Grp. gram.:f.}
\end{itemize}
Qualidade de omacéphalo.
\section{Omacephaliano}
\begin{itemize}
\item {Grp. gram.:adj.}
\end{itemize}
Que tem omacephalia.
\section{Omacephálico}
\begin{itemize}
\item {Grp. gram.:adj.}
\end{itemize}
Relativo á omacephalia.
\section{Omacéphalo}
\begin{itemize}
\item {Grp. gram.:m.}
\end{itemize}
\begin{itemize}
\item {Proveniência:(Do gr. \textunderscore omos\textunderscore  + \textunderscore kephale\textunderscore )}
\end{itemize}
Monstro, que tem a cabeça mal conformada e carece de braços.
\section{Omádio}
\begin{itemize}
\item {Grp. gram.:m.}
\end{itemize}
Gênero de insectos coleópteros tetrâmeros.
\section{Omado}
\begin{itemize}
\item {Grp. gram.:m.}
\end{itemize}
\begin{itemize}
\item {Utilização:Prov.}
\end{itemize}
\begin{itemize}
\item {Utilização:minh.}
\end{itemize}
O mesmo que \textunderscore malhal\textunderscore  das pipas ou tonéis.
\section{Omagem}
\begin{itemize}
\item {Grp. gram.:f.}
\end{itemize}
\begin{itemize}
\item {Utilização:ant.}
\end{itemize}
\begin{itemize}
\item {Utilização:Pop.}
\end{itemize}
O mesmo que \textunderscore imagem\textunderscore . Cf. G. Vicente, I.
\section{Omagra}
\begin{itemize}
\item {Grp. gram.:f.}
\end{itemize}
\begin{itemize}
\item {Utilização:Med.}
\end{itemize}
\begin{itemize}
\item {Proveniência:(Do gr. \textunderscore omos\textunderscore  + \textunderscore agra\textunderscore )}
\end{itemize}
Doença de gota, que ataca as espáduas.
\section{Omágua}
\begin{itemize}
\item {Grp. gram.:f.}
\end{itemize}
Idioma dos Botocudos.
\section{Omáguas}
\begin{itemize}
\item {Grp. gram.:m. pl.}
\end{itemize}
Índios das margens do Amazonas.
\section{Omalanto}
\begin{itemize}
\item {Grp. gram.:m.}
\end{itemize}
\begin{itemize}
\item {Proveniência:(Do gr. \textunderscore omalos\textunderscore  + \textunderscore anthos\textunderscore )}
\end{itemize}
Gênero de plantas euforbiáceas.
\section{Omalantho}
\begin{itemize}
\item {Grp. gram.:m.}
\end{itemize}
\begin{itemize}
\item {Proveniência:(Do gr. \textunderscore omalos\textunderscore  + \textunderscore anthos\textunderscore )}
\end{itemize}
Gênero de plantas euphorbiáceas.
\section{Omalgia}
\begin{itemize}
\item {Grp. gram.:f.}
\end{itemize}
\begin{itemize}
\item {Proveniência:(Do gr. \textunderscore omos\textunderscore  + \textunderscore algos\textunderscore )}
\end{itemize}
Dôr no ombro.
\section{Omálio}
\begin{itemize}
\item {Grp. gram.:m.}
\end{itemize}
Gênero de insectos coleópteros pentâmeros.
\section{Omaliso}
\begin{itemize}
\item {Grp. gram.:m.}
\end{itemize}
Gênero de insectos coleópteros pentâmeros.
\section{Omalófia}
\begin{itemize}
\item {Grp. gram.:f.}
\end{itemize}
Gênero de insectos coleópteros pentâmeros.
\section{Omalogastro}
\begin{itemize}
\item {Grp. gram.:m.}
\end{itemize}
\begin{itemize}
\item {Proveniência:(Do gr. \textunderscore omalos\textunderscore  + \textunderscore gaster\textunderscore )}
\end{itemize}
Gênero de insectos dípteros.
\section{Omalóphia}
\begin{itemize}
\item {Grp. gram.:f.}
\end{itemize}
Gênero de insectos coleópteros pentâmeros.
\section{Omalópode}
\begin{itemize}
\item {Grp. gram.:adj.}
\end{itemize}
\begin{itemize}
\item {Grp. gram.:M. pl.}
\end{itemize}
\begin{itemize}
\item {Proveniência:(Do gr. \textunderscore omalos\textunderscore  + \textunderscore pous\textunderscore  + \textunderscore podos\textunderscore )}
\end{itemize}
Que tem pés chatos.
Família de coleópteros.
\section{Omalópteros}
\begin{itemize}
\item {Grp. gram.:m. pl.}
\end{itemize}
\begin{itemize}
\item {Proveniência:(Do gr. \textunderscore omalos\textunderscore  + \textunderscore pteron\textunderscore )}
\end{itemize}
Ordem de insectos dípteros.
\section{Omaloramphos}
\begin{itemize}
\item {fónica:ram}
\end{itemize}
\begin{itemize}
\item {Grp. gram.:m. pl.}
\end{itemize}
Família de aves, que comprehende os pássaros que têm bico largo, e plano na base.
\section{Omalorranfos}
\begin{itemize}
\item {Grp. gram.:m. pl.}
\end{itemize}
Família de aves, que comprehende os pássaros que têm bico largo, e plano na base.
\section{Omaso}
\begin{itemize}
\item {Grp. gram.:m. pl.}
\end{itemize}
\begin{itemize}
\item {Proveniência:(Lat. \textunderscore omasum\textunderscore )}
\end{itemize}
Terceiro estômago dos ruminantes; folhoso.
\section{Omasthrocacia}
\begin{itemize}
\item {Grp. gram.:f.}
\end{itemize}
\begin{itemize}
\item {Utilização:Med.}
\end{itemize}
Cárie da articulação da espádua.
\section{Omastrefo}
\begin{itemize}
\item {Grp. gram.:m.}
\end{itemize}
Gênero de moluscos cefalópodes.
\section{Omastrocacia}
\begin{itemize}
\item {Grp. gram.:f.}
\end{itemize}
\begin{itemize}
\item {Utilização:Med.}
\end{itemize}
Cárie da articulação da espádua.
\section{Omatódio}
\begin{itemize}
\item {Grp. gram.:m.}
\end{itemize}
\begin{itemize}
\item {Proveniência:(Do gr. \textunderscore omma\textunderscore  + \textunderscore eidos\textunderscore )}
\end{itemize}
Gênero de orquídeas do Cabo da Bôa-Esperança.
\section{Omatolampo}
\begin{itemize}
\item {Grp. gram.:m.}
\end{itemize}
Gênero de insectos coleópteros tetrâmeros.
\section{Omatopleia}
\begin{itemize}
\item {Grp. gram.:f.}
\end{itemize}
Gênero de helmintos, descoberto no Mar-Vermelho.
\section{Ombiacho}
\begin{itemize}
\item {Grp. gram.:m.}
\end{itemize}
Espécie de adivinho, entre os Malgaches.
\section{Ombo}
\begin{itemize}
\item {Grp. gram.:m.}
\end{itemize}
Árvore indiana.
O mesmo que \textunderscore ombu\textunderscore ?
\section{Ombreira}
\begin{itemize}
\item {Grp. gram.:f.}
\end{itemize}
\begin{itemize}
\item {Utilização:Fig.}
\end{itemize}
\begin{itemize}
\item {Proveniência:(De \textunderscore ombro\textunderscore )}
\end{itemize}
Parte de vestuário, correspondente aos ombros.
Cada uma das duas partes lateraes e fixas, que sustentam a vêrga ou os enchaméis superiores da porta.
Entrada.
\section{Ombro}
\begin{itemize}
\item {Grp. gram.:m.}
\end{itemize}
\begin{itemize}
\item {Utilização:Fig.}
\end{itemize}
\begin{itemize}
\item {Proveniência:(Do lat. \textunderscore umerus\textunderscore )}
\end{itemize}
A parte mais alta do braço humano.
Espádua.
Fôrça.
Esfôrço: \textunderscore meter ombros a uma empresa\textunderscore .
\section{Ombu}
\begin{itemize}
\item {Grp. gram.:m.}
\end{itemize}
Árvore colossal da América do Sul.
\section{Omecio}
\begin{itemize}
\item {Grp. gram.:m.}
\end{itemize}
\begin{itemize}
\item {Utilização:Ant.}
\end{itemize}
O mesmo que \textunderscore homicídio\textunderscore .
\section{Ómega}
\begin{itemize}
\item {Grp. gram.:m.}
\end{itemize}
\begin{itemize}
\item {Utilização:Fig.}
\end{itemize}
\begin{itemize}
\item {Proveniência:(Gr. \textunderscore omega\textunderscore )}
\end{itemize}
Última letra do alphabeto grego.
Fim, termo.
\section{Omeleta}
\begin{itemize}
\item {fónica:lê}
\end{itemize}
\begin{itemize}
\item {Grp. gram.:f.}
\end{itemize}
\begin{itemize}
\item {Proveniência:(Fr. \textunderscore omelette\textunderscore )}
\end{itemize}
Ovos batidos conjuntamente e fritos, em fórma de troixa ou posta. Cf. Filinto, II, 4.
\section{Omezio}
\begin{itemize}
\item {Grp. gram.:m.}
\end{itemize}
\begin{itemize}
\item {Utilização:Ant.}
\end{itemize}
O mesmo que \textunderscore homicídio\textunderscore .
\section{Omezio}
\begin{itemize}
\item {Grp. gram.:m.}
\end{itemize}
\begin{itemize}
\item {Utilização:Ant.}
\end{itemize}
Coito, terra coitada. Cf. Fern. Lopes, \textunderscore Chrón. de D. Fern.\textunderscore , II, 86.
\section{Omíadas}
\begin{itemize}
\item {Grp. gram.:f. pl.}
\end{itemize}
O mesmo ou melhor que \textunderscore omíades\textunderscore .
\section{Omíades}
\begin{itemize}
\item {Grp. gram.:m. pl.}
\end{itemize}
Dinastia árabe, procedente de Omias, tio de Mahomet.
\section{Omiádico}
\begin{itemize}
\item {Grp. gram.:adj.}
\end{itemize}
Relativo aos Omíadas. Cf. Herculano, \textunderscore Opúsc.\textunderscore , III, 209.
\section{Ómicro}
\begin{itemize}
\item {Grp. gram.:m.}
\end{itemize}
\begin{itemize}
\item {Proveniência:(Gr. \textunderscore omikron\textunderscore )}
\end{itemize}
Nome da letra, que no alphabeto grego corresponde a \textunderscore o\textunderscore .
\section{Ominar}
\begin{itemize}
\item {Grp. gram.:v. t.}
\end{itemize}
\begin{itemize}
\item {Proveniência:(Lat. \textunderscore ominari\textunderscore )}
\end{itemize}
Agoirar.
\section{Ominoso}
\begin{itemize}
\item {Grp. gram.:adj.}
\end{itemize}
\begin{itemize}
\item {Proveniência:(Lat. \textunderscore ominosus\textunderscore )}
\end{itemize}
Agoirento; nefasto.
Detestável.
\section{Omóide}
\begin{itemize}
\item {Grp. gram.:m.  e  adj.}
\end{itemize}
\begin{itemize}
\item {Proveniência:(Do gr. \textunderscore omos\textunderscore  + \textunderscore eidos\textunderscore )}
\end{itemize}
Diz-se de um dos ossos palatinos das aves.
\section{Omiri}
\begin{itemize}
\item {Grp. gram.:m.}
\end{itemize}
\begin{itemize}
\item {Utilização:Bras}
\end{itemize}
Árvore, que produz estoraque.
\section{Omissão}
\begin{itemize}
\item {Grp. gram.:f.}
\end{itemize}
\begin{itemize}
\item {Proveniência:(Lat. \textunderscore omissio\textunderscore )}
\end{itemize}
Acto ou effeito de omittir; aquillo que se omittiu; falta; lacuna.
\section{Omisso}
\begin{itemize}
\item {Grp. gram.:adj.}
\end{itemize}
\begin{itemize}
\item {Proveniência:(Lat. \textunderscore omissus\textunderscore )}
\end{itemize}
Que revela falta ou esquecimento.
Que não previu certas hypótheses, (falando-se de leis, regulamentos ou preceitos escritos).
Descuidado; negligente.
\section{Omissor}
\begin{itemize}
\item {Grp. gram.:adj.}
\end{itemize}
O mesmo que \textunderscore omissório\textunderscore .
\section{Omissório}
\begin{itemize}
\item {Grp. gram.:adj.}
\end{itemize}
\begin{itemize}
\item {Proveniência:(Do lat. \textunderscore omissus\textunderscore )}
\end{itemize}
Que determina omissão.
\section{Omitido}
\begin{itemize}
\item {Grp. gram.:adj.}
\end{itemize}
\begin{itemize}
\item {Proveniência:(De \textunderscore omitir\textunderscore )}
\end{itemize}
Postergado; esquecido.
\section{Omitir}
\begin{itemize}
\item {Grp. gram.:v. t.}
\end{itemize}
\begin{itemize}
\item {Proveniência:(Lat. \textunderscore omittere\textunderscore )}
\end{itemize}
Preterir.
Deixar de fazer ou dizer.
Olvidar.
\section{Omittido}
\begin{itemize}
\item {Grp. gram.:adj.}
\end{itemize}
\begin{itemize}
\item {Proveniência:(De \textunderscore omittir\textunderscore )}
\end{itemize}
Postergado; esquecido.
\section{Omittir}
\begin{itemize}
\item {Grp. gram.:v. t.}
\end{itemize}
\begin{itemize}
\item {Proveniência:(Lat. \textunderscore omittere\textunderscore )}
\end{itemize}
Preterir.
Deixar de fazer ou dizer.
Olvidar.
\section{Omiziám}
\begin{itemize}
\item {Grp. gram.:m.}
\end{itemize}
\begin{itemize}
\item {Utilização:Ant.}
\end{itemize}
O mesmo que \textunderscore homicida\textunderscore .
\section{Omiziero}
\begin{itemize}
\item {Grp. gram.:m.}
\end{itemize}
\begin{itemize}
\item {Utilização:Ant.}
\end{itemize}
O mesmo que \textunderscore homicida\textunderscore .
\section{Omízio}
\begin{itemize}
\item {Grp. gram.:m.}
\end{itemize}
\begin{itemize}
\item {Utilização:Ant.}
\end{itemize}
Homicídio.
Malefício ou crime, que merecia pena de morte.
Ódio, inímizade.
\section{Ommastrepho}
\begin{itemize}
\item {Grp. gram.:m.}
\end{itemize}
Gênero de molluscos cephalópodes.
\section{Ommatódio}
\begin{itemize}
\item {Grp. gram.:m.}
\end{itemize}
\begin{itemize}
\item {Proveniência:(Do gr. \textunderscore omma\textunderscore  + \textunderscore eidos\textunderscore )}
\end{itemize}
Gênero de orchídeas do Cabo da Bôa-Esperança.
\section{Ommatolampo}
\begin{itemize}
\item {Grp. gram.:m.}
\end{itemize}
Gênero de insectos coleópteros tetrâmeros.
\section{Ommatopleia}
\begin{itemize}
\item {Grp. gram.:f.}
\end{itemize}
Gênero de helminthos, descoberto no Mar-Vermelho.
\section{Ommíadas}
\begin{itemize}
\item {Grp. gram.:f. pl.}
\end{itemize}
O mesmo ou melhor que \textunderscore ommíades\textunderscore .
\section{Ommíades}
\begin{itemize}
\item {Grp. gram.:m. pl.}
\end{itemize}
Dynastia árabe, procedente de Ommias, tio de Mahomet.
\section{Ommiádico}
\begin{itemize}
\item {Grp. gram.:adj.}
\end{itemize}
Relativo aos Ommíadas. Cf. Herculano, \textunderscore Opúsc.\textunderscore , III, 209.
\section{Omni...}
\begin{itemize}
\item {Grp. gram.:pref.}
\end{itemize}
\begin{itemize}
\item {Proveniência:(Lat. \textunderscore omnis\textunderscore )}
\end{itemize}
(designativo de \textunderscore tudo\textunderscore  ou \textunderscore todos\textunderscore )
\section{Ómnia}
\begin{itemize}
\item {Grp. gram.:f.}
\end{itemize}
\begin{itemize}
\item {Utilização:Prov.}
\end{itemize}
\begin{itemize}
\item {Utilização:Ant.}
\end{itemize}
\begin{itemize}
\item {Proveniência:(Lat. \textunderscore omnia\textunderscore )}
\end{itemize}
Pomar, horta, de plantação variada.
Tudo.
\section{Omnibebedor}
\begin{itemize}
\item {Grp. gram.:m.}
\end{itemize}
\begin{itemize}
\item {Utilização:Burl.}
\end{itemize}
Aquelle que aprecia todas as bebidas. Cf. A. Mendes, \textunderscore Plágios\textunderscore , XIII.
\section{Ómnibus}
\begin{itemize}
\item {Grp. gram.:m.}
\end{itemize}
\begin{itemize}
\item {Proveniência:(Lat. \textunderscore omnibus\textunderscore )}
\end{itemize}
Grande carruagem de aluguel.
Carro para transporte de pessôas em viagem; diligência.
\section{Omnicolor}
\begin{itemize}
\item {Grp. gram.:adj.}
\end{itemize}
\begin{itemize}
\item {Proveniência:(Lat. \textunderscore omnicolor\textunderscore )}
\end{itemize}
Que tem todas as côres.
\section{Omnicriador}
\begin{itemize}
\item {Grp. gram.:m.  e  adj.}
\end{itemize}
Aquelle que tudo criou. Cf. Filinto, I, 131.
\section{Omniforme}
\begin{itemize}
\item {Grp. gram.:adj.}
\end{itemize}
\begin{itemize}
\item {Proveniência:(Lat. \textunderscore omniformis\textunderscore )}
\end{itemize}
Que tem todas as fórmas conhecidas.
Susceptível de tomar todas as fórmas.
\section{Omnigênero}
\begin{itemize}
\item {Grp. gram.:adj.}
\end{itemize}
\begin{itemize}
\item {Proveniência:(Do lat. \textunderscore omnis\textunderscore  + \textunderscore genus\textunderscore )}
\end{itemize}
Relativo a todos os gêneros.
\section{Omnilingue}
\begin{itemize}
\item {Grp. gram.:adj.}
\end{itemize}
\begin{itemize}
\item {Proveniência:(Do lat. \textunderscore omnis\textunderscore  + \textunderscore língua\textunderscore )}
\end{itemize}
Que conhece todas as línguas; que é polyglotta. Cf. Macedo, \textunderscore Burros\textunderscore , 12.
\section{Omnimodamente}
\begin{itemize}
\item {Grp. gram.:adv.}
\end{itemize}
\begin{itemize}
\item {Proveniência:(De \textunderscore omnímodo\textunderscore )}
\end{itemize}
De todos os modos possíveis.
\section{Omnímodo}
\begin{itemize}
\item {Grp. gram.:adj.}
\end{itemize}
\begin{itemize}
\item {Proveniência:(Lat. \textunderscore omnimodus\textunderscore )}
\end{itemize}
Que é de todos os modos ou gêneros.
Illimitado.
Que não tem restricções.
\section{Omnipalrante}
\begin{itemize}
\item {Grp. gram.:adj.}
\end{itemize}
\begin{itemize}
\item {Utilização:Burl.}
\end{itemize}
Que fala de tudo.
\section{Omniparente}
\begin{itemize}
\item {Grp. gram.:adj.}
\end{itemize}
\begin{itemize}
\item {Utilização:Poét.}
\end{itemize}
\begin{itemize}
\item {Proveniência:(Lat. \textunderscore omniparens\textunderscore )}
\end{itemize}
Que produz tudo.
Que tudo cria.
\section{Omnipatente}
\begin{itemize}
\item {Grp. gram.:adj.}
\end{itemize}
\begin{itemize}
\item {Proveniência:(Lat. \textunderscore omnipatens\textunderscore )}
\end{itemize}
Patente para todos; público. Cf. \textunderscore Ulisseia\textunderscore , X, 1.
\section{Omnipessoal}
\begin{itemize}
\item {Grp. gram.:adj.}
\end{itemize}
\begin{itemize}
\item {Utilização:Gram.}
\end{itemize}
Que tem todas as pessôas, (falando-se de verbos); que não é unipessoal.
\section{Omnipotência}
\begin{itemize}
\item {Grp. gram.:f.}
\end{itemize}
\begin{itemize}
\item {Proveniência:(Lat. \textunderscore omnipotentia\textunderscore )}
\end{itemize}
Qualidade de omnipotente.
\section{Omnipotente}
\begin{itemize}
\item {Grp. gram.:adj.}
\end{itemize}
\begin{itemize}
\item {Grp. gram.:M.}
\end{itemize}
\begin{itemize}
\item {Proveniência:(Lat. \textunderscore omnipotens\textunderscore )}
\end{itemize}
Que póde tudo.
Todo poderoso.
Deus.
\section{Omnipotentemente}
\begin{itemize}
\item {Grp. gram.:adv.}
\end{itemize}
De modo omnipotente.
\section{Omnipresença}
\begin{itemize}
\item {Grp. gram.:f.}
\end{itemize}
\begin{itemize}
\item {Proveniência:(De \textunderscore omni...\textunderscore  + \textunderscore presença\textunderscore )}
\end{itemize}
Qualidade do que é omnipresente.
\section{Omnipresente}
\begin{itemize}
\item {Grp. gram.:adj.}
\end{itemize}
\begin{itemize}
\item {Proveniência:(De \textunderscore omni...\textunderscore  + \textunderscore presente\textunderscore )}
\end{itemize}
Que está em toda a parte.
Que tem o carácter da ubiquidade.
\section{Omnisciência}
\begin{itemize}
\item {Grp. gram.:f.}
\end{itemize}
Qualidade de omnisciente.
(Do \textunderscore omni...\textunderscore  + \textunderscore sciência\textunderscore )
\section{Omnisciente}
\begin{itemize}
\item {Grp. gram.:adj.}
\end{itemize}
\begin{itemize}
\item {Proveniência:(De \textunderscore omni...\textunderscore  + \textunderscore sciente\textunderscore )}
\end{itemize}
Que sabe tudo.
\section{Omníscio}
\begin{itemize}
\item {Grp. gram.:adj.}
\end{itemize}
O mesmo que \textunderscore omnisciente\textunderscore .
\section{Omnitónico}
\begin{itemize}
\item {Grp. gram.:adj.}
\end{itemize}
\begin{itemize}
\item {Utilização:Mús.}
\end{itemize}
\begin{itemize}
\item {Proveniência:(Do lat. \textunderscore omnis\textunderscore  + \textunderscore tonus\textunderscore )}
\end{itemize}
Diz-se dos instrumentos, que se podem tocar em todos os tons, como succede á trompa, ao clarim e á corneta, desde que receberam o aperfeiçoamento dos cylindros.
\section{Omnividente}
\begin{itemize}
\item {Grp. gram.:adj.}
\end{itemize}
\begin{itemize}
\item {Proveniência:(Do lat. \textunderscore omnis\textunderscore  + \textunderscore videns\textunderscore )}
\end{itemize}
Que tudo vê, que tudo conhece. Cf. Latino, \textunderscore Humboldt\textunderscore , 415.
\section{Omnívomo}
\begin{itemize}
\item {Grp. gram.:adj.}
\end{itemize}
\begin{itemize}
\item {Proveniência:(Do lat. \textunderscore omnis\textunderscore  + \textunderscore vomere\textunderscore )}
\end{itemize}
Que vomita quanto toma.
\section{Omnívoro}
\begin{itemize}
\item {Grp. gram.:adj.}
\end{itemize}
\begin{itemize}
\item {Proveniência:(Do lat. \textunderscore omnis\textunderscore  + \textunderscore vorare\textunderscore )}
\end{itemize}
Que come de tudo.
Que se alimenta de substâncias animaes ou vegetaes.
\section{Omo}
\begin{itemize}
\item {Grp. gram.:m.}
\end{itemize}
\begin{itemize}
\item {Utilização:Zool.}
\end{itemize}
\begin{itemize}
\item {Proveniência:(Do gr. \textunderscore omos\textunderscore )}
\end{itemize}
O mesmo que \textunderscore ombro\textunderscore  ou \textunderscore espádua\textunderscore .
\section{Omo...}
\begin{itemize}
\item {Grp. gram.:pref.}
\end{itemize}
\begin{itemize}
\item {Proveniência:(Do gr. \textunderscore omos\textunderscore )}
\end{itemize}
Designativo de \textunderscore ombro\textunderscore .
\section{Omoalgia}
\begin{itemize}
\item {Grp. gram.:f.}
\end{itemize}
O mesmo que \textunderscore omalgia\textunderscore .
\section{Omoclavicular}
\begin{itemize}
\item {Grp. gram.:adj.}
\end{itemize}
\begin{itemize}
\item {Proveniência:(De \textunderscore omo...\textunderscore  + \textunderscore clavicular\textunderscore )}
\end{itemize}
Relativo á omoplata e á clavícula.
\section{Omocótila}
\begin{itemize}
\item {Grp. gram.:f.}
\end{itemize}
\begin{itemize}
\item {Utilização:Anat.}
\end{itemize}
\begin{itemize}
\item {Proveniência:(Do gr. \textunderscore omos\textunderscore  + \textunderscore kotule\textunderscore )}
\end{itemize}
Cavidade da omoplata, que recebe a cabeça do úmero.
\section{Omocótyla}
\begin{itemize}
\item {Grp. gram.:f.}
\end{itemize}
\begin{itemize}
\item {Utilização:Anat.}
\end{itemize}
\begin{itemize}
\item {Proveniência:(Do gr. \textunderscore omos\textunderscore  + \textunderscore kotule\textunderscore )}
\end{itemize}
Cavidade da omoplata, que recebe a cabeça do úmero.
\section{Omóforo}
\begin{itemize}
\item {Grp. gram.:m.}
\end{itemize}
\begin{itemize}
\item {Proveniência:(Do gr. \textunderscore omos\textunderscore  + \textunderscore phoros\textunderscore )}
\end{itemize}
Gênero de insectos coleópteros tetrâmeros.
\section{Omoídeo}
\begin{itemize}
\item {Grp. gram.:m.  e  adj.}
\end{itemize}
O mesmo que \textunderscore omóide\textunderscore .
\section{Omóphoro}
\begin{itemize}
\item {Grp. gram.:m.}
\end{itemize}
\begin{itemize}
\item {Proveniência:(Do gr. \textunderscore omos\textunderscore  + \textunderscore phoros\textunderscore )}
\end{itemize}
Gênero de insectos coleópteros tetrâmeros.
\section{Omoplata}
\begin{itemize}
\item {Grp. gram.:f.}
\end{itemize}
\begin{itemize}
\item {Proveniência:(Do gr. \textunderscore omos\textunderscore  + \textunderscore plate\textunderscore )}
\end{itemize}
Parte posterior do ombro, formada por um osso largo e triangular.
\section{Omphacino}
\begin{itemize}
\item {Grp. gram.:adj.}
\end{itemize}
\begin{itemize}
\item {Proveniência:(Lat. \textunderscore omphacinus\textunderscore )}
\end{itemize}
Diz-se do azeite, fabricado de azeitonas verdes.
\section{Omphácio}
\begin{itemize}
\item {Grp. gram.:m.}
\end{itemize}
\begin{itemize}
\item {Proveniência:(Lat. \textunderscore omphacium\textunderscore )}
\end{itemize}
Pedra preciosa, transparente, verde-escura.
\section{Omphaleia}
\begin{itemize}
\item {Grp. gram.:f.}
\end{itemize}
\begin{itemize}
\item {Proveniência:(Do gr. \textunderscore omphalos\textunderscore )}
\end{itemize}
Gênero de plantas euphorbiáceas da Guiana e das Antilhas.
\section{Omphália}
\begin{itemize}
\item {Grp. gram.:f.}
\end{itemize}
\begin{itemize}
\item {Proveniência:(Do gr. \textunderscore omphalos\textunderscore )}
\end{itemize}
Gênero de molluscos cephalópodes.
\section{Omphalite}
\begin{itemize}
\item {Grp. gram.:f.}
\end{itemize}
\begin{itemize}
\item {Utilização:Med.}
\end{itemize}
\begin{itemize}
\item {Proveniência:(Do gr. \textunderscore omphalos\textunderscore )}
\end{itemize}
Inflammação no umbigo.
\section{Omphalocarpo}
\begin{itemize}
\item {Grp. gram.:m.}
\end{itemize}
\begin{itemize}
\item {Proveniência:(Do gr. \textunderscore omphalos\textunderscore  + \textunderscore karpos\textunderscore )}
\end{itemize}
Gênero de árvores sapotáceas da África tropical.
\section{Omphalocele}
\begin{itemize}
\item {Grp. gram.:f.}
\end{itemize}
\begin{itemize}
\item {Proveniência:(Do gr. \textunderscore omphalos\textunderscore  + \textunderscore kele\textunderscore )}
\end{itemize}
Tumor no umbigo.
\section{Omphalódio}
\begin{itemize}
\item {Grp. gram.:m.}
\end{itemize}
\begin{itemize}
\item {Utilização:Bot.}
\end{itemize}
\begin{itemize}
\item {Proveniência:(Do gr. \textunderscore omphalos\textunderscore  + \textunderscore eidos\textunderscore )}
\end{itemize}
Protuberância no ponto médio do umbigo do grão, onde terminam os vasos nutritivos.
\section{Omphalomancia}
\begin{itemize}
\item {Grp. gram.:f.}
\end{itemize}
\begin{itemize}
\item {Proveniência:(Do gr. \textunderscore omphalos\textunderscore  + \textunderscore manteia\textunderscore )}
\end{itemize}
Supposta arte de adivinhar quantos filhos terá uma mulher, examinando-se o número de nós que apresenta o cordão umbilical do primeiro filho.
\section{Omphalomântico}
\begin{itemize}
\item {Grp. gram.:adj.}
\end{itemize}
Relativo á omphalomancia.
\section{Omphalo-mesentérico}
\begin{itemize}
\item {Grp. gram.:adj.}
\end{itemize}
\begin{itemize}
\item {Proveniência:(De \textunderscore omphalos\textunderscore  gr. + \textunderscore mesentérico\textunderscore )}
\end{itemize}
Relativo ao umbigo e ao mesentério.
\section{Omphaloncia}
\begin{itemize}
\item {Grp. gram.:f.}
\end{itemize}
\begin{itemize}
\item {Utilização:Med.}
\end{itemize}
O mesmo que \textunderscore omphalophyma\textunderscore .
\section{Omphalophyma}
\begin{itemize}
\item {Grp. gram.:m.}
\end{itemize}
\begin{itemize}
\item {Utilização:Med.}
\end{itemize}
\begin{itemize}
\item {Proveniência:(Do gr. \textunderscore omphalos\textunderscore  + \textunderscore phuma\textunderscore )}
\end{itemize}
Tumor no umbigo.
\section{Omphalopsychos}
\begin{itemize}
\item {fónica:cos}
\end{itemize}
\begin{itemize}
\item {Grp. gram.:m. pl.}
\end{itemize}
\begin{itemize}
\item {Proveniência:(Do gr. \textunderscore omphalos\textunderscore  + \textunderscore psukhe\textunderscore )}
\end{itemize}
Nome de alguns sectários do quietismo.
Illuminados que, contemplando e fixando o umbigo, julgavam communicar com a divindade e vêr o que elles chamavam luz do Thabor.
\section{Omphalóptico}
\begin{itemize}
\item {Grp. gram.:adj.}
\end{itemize}
\begin{itemize}
\item {Utilização:Phýs.}
\end{itemize}
\begin{itemize}
\item {Proveniência:(Do gr. \textunderscore omphalos\textunderscore  + \textunderscore ops\textunderscore )}
\end{itemize}
Diz-se do crystal óptico, que é convexo do ambas as faces.
\section{Omphalorrhagia}
\begin{itemize}
\item {Grp. gram.:f.}
\end{itemize}
\begin{itemize}
\item {Utilização:Med.}
\end{itemize}
\begin{itemize}
\item {Proveniência:(Do gr. \textunderscore omphalos\textunderscore  + \textunderscore rhegnumi\textunderscore )}
\end{itemize}
Hemorrhagia umbilical, especialmente nos recém-nascidos.
\section{Omphalorrhágico}
\begin{itemize}
\item {Grp. gram.:adj.}
\end{itemize}
Relativo a omphalorrhagia.
\section{Omphalosito}
\begin{itemize}
\item {fónica:si}
\end{itemize}
\begin{itemize}
\item {Grp. gram.:m.}
\end{itemize}
\begin{itemize}
\item {Utilização:Terat.}
\end{itemize}
\begin{itemize}
\item {Proveniência:(Do gr. \textunderscore omphalos\textunderscore  + \textunderscore sitos\textunderscore )}
\end{itemize}
Monstro, a que faltam muitos órgãos e que têm uma vida incompleta, a qual cessa, logo que se rompa o cordão umbilical.
\section{Omphalotomia}
\begin{itemize}
\item {Grp. gram.:f.}
\end{itemize}
\begin{itemize}
\item {Proveniência:(Do gr. \textunderscore omphalos\textunderscore  + \textunderscore tome\textunderscore )}
\end{itemize}
Córte do cordão umbilical.
\section{Omphalotómico}
\begin{itemize}
\item {Grp. gram.:adj.}
\end{itemize}
Relativo á omphalotomia.
\section{Omphra}
\begin{itemize}
\item {Grp. gram.:f.}
\end{itemize}
Gênero de insectos coleópteros pentâmeros.
\section{Omucháti}
\begin{itemize}
\item {Grp. gram.:m.}
\end{itemize}
Árvore do sul de Angola, (\textunderscore bauhinia reticulata\textunderscore ).
\section{Omunhândi}
\begin{itemize}
\item {Grp. gram.:m.}
\end{itemize}
Árvore angolense, bôa para construcções.
\section{Ona}
\begin{itemize}
\item {Grp. gram.:f.}
\end{itemize}
\begin{itemize}
\item {Utilização:Ant.}
\end{itemize}
(V.alna)
\section{Onagra}
\begin{itemize}
\item {Grp. gram.:f.}
\end{itemize}
\begin{itemize}
\item {Proveniência:(Gr. \textunderscore onagra\textunderscore )}
\end{itemize}
Planta herbácea, de rebentos e raízes alimentares.
\section{Onagrariáceas}
\begin{itemize}
\item {Grp. gram.:f. pl.}
\end{itemize}
O mesmo ou melhor que \textunderscore onagrárias\textunderscore .
\section{Onagrariáceo}
\begin{itemize}
\item {Grp. gram.:adj.}
\end{itemize}
O mesmo que \textunderscore onagrário\textunderscore .
\section{Onagrárias}
\begin{itemize}
\item {Grp. gram.:f. pl.}
\end{itemize}
\begin{itemize}
\item {Proveniência:(De \textunderscore onagrário\textunderscore )}
\end{itemize}
Família de plantas, que tem por typo a onagra.
\section{Onagrário}
\begin{itemize}
\item {Grp. gram.:adj.}
\end{itemize}
Relativo ou semelhante á onagra.
\section{Onagre}
\begin{itemize}
\item {Grp. gram.:m.}
\end{itemize}
Antiga máquina de guerra, o mesmo que \textunderscore onagro\textunderscore .
\section{Onágreas}
\begin{itemize}
\item {Grp. gram.:f. pl.}
\end{itemize}
\begin{itemize}
\item {Utilização:Bot.}
\end{itemize}
\begin{itemize}
\item {Proveniência:(De \textunderscore onagra\textunderscore )}
\end{itemize}
Tríbo de onagrariáceas.
\section{Onagro}
\begin{itemize}
\item {Grp. gram.:m.}
\end{itemize}
\begin{itemize}
\item {Proveniência:(Lat. \textunderscore onager\textunderscore )}
\end{itemize}
Burro selvagem.
Burro.
Antiga máchina de guerra.
\section{Onanismo}
\begin{itemize}
\item {Grp. gram.:m.}
\end{itemize}
\begin{itemize}
\item {Proveniência:(De \textunderscore Onan\textunderscore , n. p. bíblico)}
\end{itemize}
O mesmo que \textunderscore masturbação\textunderscore .
\section{Onanista}
\begin{itemize}
\item {Grp. gram.:m.}
\end{itemize}
Aquelle que se entrega ao onanismo.
\section{Onanizar-se}
\begin{itemize}
\item {Grp. gram.:v. p.}
\end{itemize}
Masturbar-se.
(Cp. \textunderscore onanismo\textunderscore )
\section{Onça}
\begin{itemize}
\item {Grp. gram.:f.}
\end{itemize}
\begin{itemize}
\item {Utilização:Bras. do N}
\end{itemize}
\begin{itemize}
\item {Utilização:Pop.}
\end{itemize}
\begin{itemize}
\item {Grp. gram.:Loc.}
\end{itemize}
\begin{itemize}
\item {Utilização:fam.}
\end{itemize}
\begin{itemize}
\item {Proveniência:(Do lat. \textunderscore uncia\textunderscore )}
\end{itemize}
Pêso antigo, equivalente á décima sexta parte do arrátel.
Pêso de oito drachmas.
Moéda havanesa de oiro, do valor de 17 piastras.
Moéda espanhola, correspondente a 14.672 reis.
Espécie de jôgo, semelhante ao chamado das damas.
Pacotilha de tabaco, de pêso variável.
\textunderscore Andar á onça\textunderscore , não têr dinheiro. Cf. Camillo, \textunderscore Hist. e Sentim.\textunderscore , 164.
\section{Onça}
\begin{itemize}
\item {Grp. gram.:f.}
\end{itemize}
\begin{itemize}
\item {Grp. gram.:M.  e  adj.}
\end{itemize}
\begin{itemize}
\item {Utilização:Bras}
\end{itemize}
Mammífero felino, (\textunderscore felis uncia\textunderscore ).
Valente; invencível.
Pessôa muito feia.
\section{Onção}
\begin{itemize}
\item {Grp. gram.:m.}
\end{itemize}
Onça grande?:«\textunderscore elephantes, onçãos, crocodilos.\textunderscore »Filinto, I, 292.
\section{Onceiro}
\begin{itemize}
\item {Grp. gram.:m.}
\end{itemize}
\begin{itemize}
\item {Utilização:Bras. de Minas}
\end{itemize}
Cão, caçador de onças.
\section{Onchídia}
\begin{itemize}
\item {fónica:qui}
\end{itemize}
\begin{itemize}
\item {Grp. gram.:f.}
\end{itemize}
\begin{itemize}
\item {Utilização:Bot.}
\end{itemize}
Parte interna do botão dos vegetaes ou o seu núcleo vital interno.
\section{Onchídio}
\begin{itemize}
\item {fónica:qui}
\end{itemize}
\begin{itemize}
\item {Grp. gram.:m.}
\end{itemize}
Gênero de molluscos gasterópodes.
\section{Oncial}
\begin{itemize}
\item {Grp. gram.:adj.}
\end{itemize}
O mesmo ou melhor que \textunderscore uncial\textunderscore . Cf. Herculano, \textunderscore Hist. de Port.\textunderscore , II, 489; Arn. Gama, \textunderscore Última Dona\textunderscore , 51 e 56.
\section{Onco}
\begin{itemize}
\item {Grp. gram.:m.}
\end{itemize}
\begin{itemize}
\item {Utilização:Ant.}
\end{itemize}
\begin{itemize}
\item {Proveniência:(Do lat. \textunderscore uncus\textunderscore )}
\end{itemize}
Lugar esconso.
Angra, cercada de montes.
\section{Oncoba}
\begin{itemize}
\item {Grp. gram.:f.}
\end{itemize}
Gênero de plantas bixáceas.
\section{Oncocéfalo}
\begin{itemize}
\item {Grp. gram.:m.}
\end{itemize}
\begin{itemize}
\item {Proveniência:(Do gr. \textunderscore onkos\textunderscore  + \textunderscore kephale\textunderscore )}
\end{itemize}
Gênero de insectos hemípteros.
\section{Oncocéphalo}
\begin{itemize}
\item {Grp. gram.:m.}
\end{itemize}
\begin{itemize}
\item {Proveniência:(Do gr. \textunderscore onkos\textunderscore  + \textunderscore kephale\textunderscore )}
\end{itemize}
Gênero de insectos hemípteros.
\section{Oncóforo}
\begin{itemize}
\item {Grp. gram.:m.}
\end{itemize}
\begin{itemize}
\item {Proveniência:(Do gr. \textunderscore onkos\textunderscore  + \textunderscore phoros\textunderscore )}
\end{itemize}
Gênero de musgos.
\section{Oncologia}
\begin{itemize}
\item {Grp. gram.:f.}
\end{itemize}
\begin{itemize}
\item {Utilização:Med.}
\end{itemize}
\begin{itemize}
\item {Proveniência:(Do gr. \textunderscore onkos\textunderscore  + \textunderscore logos\textunderscore )}
\end{itemize}
Estudo dos tumores.
\section{Oncóphoro}
\begin{itemize}
\item {Grp. gram.:m.}
\end{itemize}
\begin{itemize}
\item {Proveniência:(Do gr. \textunderscore onkos\textunderscore  + \textunderscore phoros\textunderscore )}
\end{itemize}
Gênero de musgos.
\section{Oncosperma}
\begin{itemize}
\item {Grp. gram.:f.}
\end{itemize}
\begin{itemize}
\item {Proveniência:(Do gr. \textunderscore onkos\textunderscore  + \textunderscore sperma\textunderscore )}
\end{itemize}
Gênero de palmeiras indianas.
\section{Oncósporo}
\begin{itemize}
\item {Grp. gram.:m.}
\end{itemize}
\begin{itemize}
\item {Proveniência:(Do gr. \textunderscore onkos\textunderscore  + \textunderscore poros\textunderscore )}
\end{itemize}
Gênero de plantas pittospóreas.
\section{Oncóstemo}
\begin{itemize}
\item {Grp. gram.:m.}
\end{itemize}
Gênero de plantas myrsíneas.
\section{Oncotomia}
\begin{itemize}
\item {Grp. gram.:f.}
\end{itemize}
\begin{itemize}
\item {Proveniência:(Do gr. \textunderscore onkos\textunderscore  + \textunderscore tome\textunderscore )}
\end{itemize}
Incisão de um tumor.
\section{Oncotómico}
\begin{itemize}
\item {Grp. gram.:f.}
\end{itemize}
Relativo á oncotomia.
\section{Onda}
\begin{itemize}
\item {Grp. gram.:f.}
\end{itemize}
\begin{itemize}
\item {Utilização:Poét.}
\end{itemize}
\begin{itemize}
\item {Utilização:Fig.}
\end{itemize}
\begin{itemize}
\item {Proveniência:(Do lat. \textunderscore unda\textunderscore )}
\end{itemize}
Água que se agita.
Água do mar.
Mar.
Água.
Grande agitação, tropel.
Derramamento de grande porção de liquido.
Grande agglomeração de pessôas em movimento.
Ondulação; objecto que ondula: \textunderscore as ondas do cabello\textunderscore .
Cada um dos accessos mais violentos da hydrophobia. Cf. Macedo Pinto, \textunderscore Comp. de Veter.\textunderscore , I, 306.
\section{Ondaca}
\begin{itemize}
\item {Grp. gram.:f.}
\end{itemize}
\begin{itemize}
\item {Utilização:T. de Mossáme}
\end{itemize}
\begin{itemize}
\item {Utilização:des.}
\end{itemize}
Questão judicial.
Processo.
Julgamento.
\section{Ondar}
\begin{itemize}
\item {Proveniência:(De \textunderscore onda\textunderscore )}
\end{itemize}
\textunderscore v. i.\textunderscore  (e der.)
O mesmo que \textunderscore ondear\textunderscore .
\section{Ondatra}
\begin{itemize}
\item {Grp. gram.:m.}
\end{itemize}
Mammífero roedor, na América meridional.
\section{Onde}
\begin{itemize}
\item {Grp. gram.:adv.}
\end{itemize}
\begin{itemize}
\item {Utilização:Prov.}
\end{itemize}
\begin{itemize}
\item {Utilização:minh.}
\end{itemize}
\begin{itemize}
\item {Proveniência:(Do lat. \textunderscore unde\textunderscore )}
\end{itemize}
No qual lugar.
Em que.
Quando aliás: \textunderscore minha mulher, depois daquelle olhado, ficou uma preguiçosa, onde ella era tão trabalhadeira\textunderscore .
O mesmo que \textunderscore aonde\textunderscore :«\textunderscore irei onde os outros forem.\textunderscore »\textunderscore Eufrosina\textunderscore , 152.
\section{Ondeado}
\begin{itemize}
\item {Grp. gram.:adj.}
\end{itemize}
\begin{itemize}
\item {Proveniência:(De \textunderscore ondear\textunderscore )}
\end{itemize}
Que tem ondas.
Disposto em curvas, á maneira de ondas: \textunderscore cabello ondeado\textunderscore .
\section{Ondeamento}
\begin{itemize}
\item {Grp. gram.:m.}
\end{itemize}
Acto ou effeito de ondear.
\section{Ondeante}
\begin{itemize}
\item {Grp. gram.:adj.}
\end{itemize}
Que ondeia; que ondula; o mesmo que \textunderscore ondeado\textunderscore .
\section{Ondear}
\begin{itemize}
\item {Grp. gram.:v. i.}
\end{itemize}
\begin{itemize}
\item {Grp. gram.:V. t.}
\end{itemize}
\begin{itemize}
\item {Proveniência:(De \textunderscore onda\textunderscore )}
\end{itemize}
Fazer ondulações ou ondas.
Estender-se, tornando-se alternadamente côncavo e convexo.
Serpear.
Tornar onduloso.
Dar movimento como o das ondas a.
Tornar sinuoso.
\section{Ondeio}
\begin{itemize}
\item {Grp. gram.:m.}
\end{itemize}
O mesmo que \textunderscore ondeamento\textunderscore . Cf. Filinto, VI, 186.
\section{Ondeirada}
\begin{itemize}
\item {Grp. gram.:f.}
\end{itemize}
\begin{itemize}
\item {Utilização:Prov.}
\end{itemize}
\begin{itemize}
\item {Utilização:minh.}
\end{itemize}
\begin{itemize}
\item {Proveniência:(De \textunderscore onda\textunderscore . Cp. fr. \textunderscore ondée\textunderscore )}
\end{itemize}
Alternativa de sol e chuviscos; chuva passageira.
\section{Onde-quer-que}
\begin{itemize}
\item {Grp. gram.:loc. conj.}
\end{itemize}
Em qualquer lugar que.
No lugar em que por acaso.
\section{Ondim}
\begin{itemize}
\item {Grp. gram.:adj. m.}
\end{itemize}
\begin{itemize}
\item {Proveniência:(De \textunderscore onda\textunderscore )}
\end{itemize}
Gênio do amor, que vive nas águas, segundo a imaginação dos poétas. Cf. Castilho, \textunderscore Primavera\textunderscore , 233.
\section{Ondina}
\begin{itemize}
\item {Grp. gram.:f.}
\end{itemize}
O mesmo que \textunderscore ondim\textunderscore . Cf. Junqueiro, \textunderscore D. João\textunderscore , 200.
\section{Ondiongulo}
\begin{itemize}
\item {Grp. gram.:m.}
\end{itemize}
Ave trepadora da África, (\textunderscore tocus nasatus\textunderscore ).
\section{Ondó}
\begin{itemize}
\item {Grp. gram.:m.}
\end{itemize}
\begin{itemize}
\item {Utilização:T. da Índia port}
\end{itemize}
Poço ou reservatório de água.
\section{Ondulação}
\begin{itemize}
\item {Grp. gram.:f.}
\end{itemize}
\begin{itemize}
\item {Proveniência:(De \textunderscore ondular\textunderscore )}
\end{itemize}
Formação de ondas pouco agitadas.
Movimento, semelhante ao das ondas.
Apparência dêste movimento.
Conjunto de saliências e depressões alternadas.
Cada uma das pregas, formadas por um tecido que o vento entumece e agita.
Movimento oscillatório, que se transmitte a um líquido ou flúido.
\section{Ondulado}
\begin{itemize}
\item {Grp. gram.:adj.}
\end{itemize}
\begin{itemize}
\item {Proveniência:(De \textunderscore ondular\textunderscore )}
\end{itemize}
Que apresenta ondulações: \textunderscore portas onduladas\textunderscore  (de ferro).
Franzido, pregueado.
\section{Ondulante}
\begin{itemize}
\item {Grp. gram.:adj.}
\end{itemize}
\begin{itemize}
\item {Proveniência:(De \textunderscore ondular\textunderscore )}
\end{itemize}
O mesmo que \textunderscore ondeante\textunderscore .
\section{Ondular}
\begin{itemize}
\item {Grp. gram.:v. i.}
\end{itemize}
\begin{itemize}
\item {Grp. gram.:V. t.}
\end{itemize}
\begin{itemize}
\item {Proveniência:(Do lat. \textunderscore undula\textunderscore )}
\end{itemize}
Formar pequenas ondas.
O mesmo que \textunderscore ondear\textunderscore .
Dar o movimento da ondulação a.
O mesmo que \textunderscore ondear\textunderscore .
\section{Ondulatório}
\begin{itemize}
\item {Grp. gram.:adj.}
\end{itemize}
\begin{itemize}
\item {Proveniência:(De \textunderscore ondular\textunderscore )}
\end{itemize}
Que tem ondulação.
Ondeante. Cf. B. Moreno, \textunderscore Com. do Campo\textunderscore , II, 63 e 101.
\section{Ondulina}
\begin{itemize}
\item {Grp. gram.:f.}
\end{itemize}
Substância líquida, em que entra o ammónio e que se emprega em lavagens do corpo e das roupas.
\section{Ondulosamente}
\begin{itemize}
\item {Grp. gram.:adv.}
\end{itemize}
De modo onduloso; com ondulação.
\section{Onduloso}
\begin{itemize}
\item {Grp. gram.:adj.}
\end{itemize}
\begin{itemize}
\item {Proveniência:(De \textunderscore ondular\textunderscore )}
\end{itemize}
Que fórma ondulações.
\section{Onemania}
\begin{itemize}
\item {Grp. gram.:f.}
\end{itemize}
\begin{itemize}
\item {Utilização:Med.}
\end{itemize}
\begin{itemize}
\item {Proveniência:(Do gr. \textunderscore one\textunderscore  + \textunderscore mania\textunderscore )}
\end{itemize}
Obsessão impulsiva para fazer compras.
\section{Onerado}
\begin{itemize}
\item {Grp. gram.:adj.}
\end{itemize}
\begin{itemize}
\item {Proveniência:(De \textunderscore onerar\textunderscore )}
\end{itemize}
Sujeito a um ónus; sobrecarregado.
\section{Onerar}
\begin{itemize}
\item {Grp. gram.:v. t.}
\end{itemize}
\begin{itemize}
\item {Proveniência:(Lat. \textunderscore onerare\textunderscore )}
\end{itemize}
Impor ónus a.
Sujeitar a ónus.
Sobrecarregar; vexar.
Aggravar com tributos.
\section{Onerário}
\begin{itemize}
\item {Grp. gram.:adj.}
\end{itemize}
\begin{itemize}
\item {Proveniência:(Lat. \textunderscore onerarius\textunderscore )}
\end{itemize}
Próprio para transportar carga.
Que póde supportar pesos.
\section{Onerosamente}
\begin{itemize}
\item {Grp. gram.:adv.}
\end{itemize}
De modo oneroso; com encargos.
\section{Onerosidade}
\begin{itemize}
\item {Grp. gram.:f.}
\end{itemize}
\begin{itemize}
\item {Proveniência:(Lat. \textunderscore onerositas\textunderscore )}
\end{itemize}
Qualidade do que é oneroso.
\section{Oneroso}
\begin{itemize}
\item {Grp. gram.:adj.}
\end{itemize}
\begin{itemize}
\item {Proveniência:(Lat. \textunderscore onerosus\textunderscore )}
\end{itemize}
Que envolve ónus; pesado.
De que resulta encargo.
Vexatório.
\section{Onésia}
\begin{itemize}
\item {Grp. gram.:f.}
\end{itemize}
Gênero de insectos dípteros.
\section{Onesita}
\begin{itemize}
\item {Grp. gram.:f.}
\end{itemize}
O mesmo que \textunderscore onesito\textunderscore .
\section{Onesito}
\begin{itemize}
\item {Grp. gram.:m.}
\end{itemize}
\begin{itemize}
\item {Utilização:Miner.}
\end{itemize}
Variedade de limonita ou hydrato de ferro.
\section{Oneta}
\begin{itemize}
\item {Grp. gram.:m.}
\end{itemize}
O mesmo que \textunderscore oquim\textunderscore .
\section{Onfacino}
\begin{itemize}
\item {Grp. gram.:adj.}
\end{itemize}
\begin{itemize}
\item {Proveniência:(Lat. \textunderscore omphacinus\textunderscore )}
\end{itemize}
Diz-se do azeite, fabricado de azeitonas verdes.
\section{Onfácio}
\begin{itemize}
\item {Grp. gram.:m.}
\end{itemize}
\begin{itemize}
\item {Proveniência:(Lat. \textunderscore omphacium\textunderscore )}
\end{itemize}
Pedra preciosa, transparente, verde-escura.
\section{Onfaleia}
\begin{itemize}
\item {Grp. gram.:f.}
\end{itemize}
\begin{itemize}
\item {Proveniência:(Do gr. \textunderscore omphalos\textunderscore )}
\end{itemize}
Gênero de plantas euforbiáceas da Guiana e das Antilhas.
\section{Onfália}
\begin{itemize}
\item {Grp. gram.:f.}
\end{itemize}
\begin{itemize}
\item {Proveniência:(Do gr. \textunderscore omphalos\textunderscore )}
\end{itemize}
Gênero de moluscos cefalópodes.
\section{Onfalite}
\begin{itemize}
\item {Grp. gram.:f.}
\end{itemize}
\begin{itemize}
\item {Utilização:Med.}
\end{itemize}
\begin{itemize}
\item {Proveniência:(Do gr. \textunderscore omphalos\textunderscore )}
\end{itemize}
Inflamação no umbigo.
\section{Onfalocarpo}
\begin{itemize}
\item {Grp. gram.:m.}
\end{itemize}
\begin{itemize}
\item {Proveniência:(Do gr. \textunderscore omphalos\textunderscore  + \textunderscore karpos\textunderscore )}
\end{itemize}
Gênero de árvores sapotáceas da África tropical.
\section{Onfalocele}
\begin{itemize}
\item {Grp. gram.:f.}
\end{itemize}
\begin{itemize}
\item {Proveniência:(Do gr. \textunderscore omphalos\textunderscore  + \textunderscore kele\textunderscore )}
\end{itemize}
Tumor no umbigo.
\section{Onfalódio}
\begin{itemize}
\item {Grp. gram.:m.}
\end{itemize}
\begin{itemize}
\item {Utilização:Bot.}
\end{itemize}
\begin{itemize}
\item {Proveniência:(Do gr. \textunderscore omphalos\textunderscore  + \textunderscore eidos\textunderscore )}
\end{itemize}
Protuberância no ponto médio do umbigo do grão, onde terminam os vasos nutritivos.
\section{Onfalomancia}
\begin{itemize}
\item {Grp. gram.:f.}
\end{itemize}
\begin{itemize}
\item {Proveniência:(Do gr. \textunderscore omphalos\textunderscore  + \textunderscore manteia\textunderscore )}
\end{itemize}
Suposta arte de adivinhar quantos filhos terá uma mulher, examinando-se o número de nós que apresenta o cordão umbilical do primeiro filho.
\section{Onfalomântico}
\begin{itemize}
\item {Grp. gram.:adj.}
\end{itemize}
Relativo á omfalomancia.
\section{Onfaloncia}
\begin{itemize}
\item {Grp. gram.:f.}
\end{itemize}
\begin{itemize}
\item {Utilização:Med.}
\end{itemize}
O mesmo que \textunderscore onfalofima\textunderscore .
\section{Onfalofima}
\begin{itemize}
\item {Grp. gram.:m.}
\end{itemize}
\begin{itemize}
\item {Utilização:Med.}
\end{itemize}
\begin{itemize}
\item {Proveniência:(Do gr. \textunderscore omphalos\textunderscore  + \textunderscore phuma\textunderscore )}
\end{itemize}
Tumor no umbigo.
\section{Onfalopsicos}
\begin{itemize}
\item {Grp. gram.:m. pl.}
\end{itemize}
\begin{itemize}
\item {Proveniência:(Do gr. \textunderscore omphalos\textunderscore  + \textunderscore psukhe\textunderscore )}
\end{itemize}
Nome de alguns sectários do quietismo.
Iluminados que, contemplando e fixando o umbigo, julgavam communicar com a divindade e vêr o que eles chamavam luz do Thabor.
\section{Onfalóptico}
\begin{itemize}
\item {Grp. gram.:adj.}
\end{itemize}
\begin{itemize}
\item {Utilização:Phýs.}
\end{itemize}
\begin{itemize}
\item {Proveniência:(Do gr. \textunderscore omphalos\textunderscore  + \textunderscore ops\textunderscore )}
\end{itemize}
Diz-se do cristal óptico, que é convexo do ambas as faces.
\section{Onfalorragia}
\begin{itemize}
\item {Grp. gram.:f.}
\end{itemize}
\begin{itemize}
\item {Utilização:Med.}
\end{itemize}
\begin{itemize}
\item {Proveniência:(Do gr. \textunderscore omphalos\textunderscore  + \textunderscore rhegnumi\textunderscore )}
\end{itemize}
Hemorragia umbilical, especialmente nos recém-nascidos.
\section{Onfalorrágico}
\begin{itemize}
\item {Grp. gram.:adj.}
\end{itemize}
Relativo a onfalorragia.
\section{Onfalossito}
\begin{itemize}
\item {Grp. gram.:m.}
\end{itemize}
\begin{itemize}
\item {Utilização:Terat.}
\end{itemize}
\begin{itemize}
\item {Proveniência:(Do gr. \textunderscore omphalos\textunderscore  + \textunderscore sitos\textunderscore )}
\end{itemize}
Monstro, a que faltam muitos órgãos e que têm uma vida incompleta, a qual cessa, logo que se rompa o cordão umbilical.
\section{Onfalotomia}
\begin{itemize}
\item {Grp. gram.:f.}
\end{itemize}
\begin{itemize}
\item {Proveniência:(Do gr. \textunderscore omphalos\textunderscore  + \textunderscore tome\textunderscore )}
\end{itemize}
Córte do cordão umbilical.
\section{Onfalotómico}
\begin{itemize}
\item {Grp. gram.:adj.}
\end{itemize}
Relativo á onfalotomia.
\section{Onfra}
\begin{itemize}
\item {Grp. gram.:f.}
\end{itemize}
Gênero de insectos coleópteros pentâmeros.
\section{Onglete}
\begin{itemize}
\item {fónica:glê}
\end{itemize}
\begin{itemize}
\item {Grp. gram.:m.}
\end{itemize}
\begin{itemize}
\item {Proveniência:(Fr. \textunderscore onglet\textunderscore )}
\end{itemize}
Espécie de buril de gravadores e serralheiros.
\section{Ongolo}
\begin{itemize}
\item {Grp. gram.:m.}
\end{itemize}
Pássaro dentirostro da África occidental, (\textunderscore oriolus larvatus\textunderscore ).
\section{Onguari}
\begin{itemize}
\item {Grp. gram.:m.}
\end{itemize}
Ave gallinácea da África occidental, (\textunderscore pternistes sclaterii\textunderscore ).
\section{Onhenga}
\begin{itemize}
\item {Grp. gram.:f.}
\end{itemize}
\begin{itemize}
\item {Utilização:T. do Sul de Angola}
\end{itemize}
O mesmo que \textunderscore painço\textunderscore .
\section{Oniogoso}
\begin{itemize}
\item {Grp. gram.:m.}
\end{itemize}
Peixe dos mares do Japão, (\textunderscore pelor japonicum\textunderscore ).
\section{Onírico}
\begin{itemize}
\item {Grp. gram.:adj.}
\end{itemize}
\begin{itemize}
\item {Proveniência:(Do gr. \textunderscore oneiros\textunderscore )}
\end{itemize}
Relativo a sonhos.
\section{Onirocricia}
\begin{itemize}
\item {Grp. gram.:f.}
\end{itemize}
\begin{itemize}
\item {Proveniência:(Do gr. \textunderscore oneiros\textunderscore  + \textunderscore krinein\textunderscore )}
\end{itemize}
Arte de interpretar os sonhos.
\section{Onirocrita}
\begin{itemize}
\item {Grp. gram.:m.}
\end{itemize}
\begin{itemize}
\item {Proveniência:(Do gr. \textunderscore oneiros\textunderscore  + \textunderscore krites\textunderscore )}
\end{itemize}
Aquelle que explica sonhos ou os interpreta.
\section{Onirocrítica}
\begin{itemize}
\item {Grp. gram.:f.}
\end{itemize}
O mesmo que \textunderscore onirocricia\textunderscore .
\section{Onirodinia}
\begin{itemize}
\item {Grp. gram.:f.}
\end{itemize}
\begin{itemize}
\item {Utilização:Med.}
\end{itemize}
\begin{itemize}
\item {Proveniência:(Do gr. \textunderscore oneiros\textunderscore  + \textunderscore odune\textunderscore )}
\end{itemize}
Dôr que se sente em sonhos.
Também se deu êste nome ao sonambulismo.
\section{Onirodynia}
\begin{itemize}
\item {Grp. gram.:f.}
\end{itemize}
\begin{itemize}
\item {Utilização:Med.}
\end{itemize}
\begin{itemize}
\item {Proveniência:(Do gr. \textunderscore oneiros\textunderscore  + \textunderscore odune\textunderscore )}
\end{itemize}
Dôr que se sente em sonhos.
Também se deu êste nome ao sonambulismo.
\section{Onirogmo}
\begin{itemize}
\item {Grp. gram.:m.}
\end{itemize}
\begin{itemize}
\item {Utilização:Med.}
\end{itemize}
\begin{itemize}
\item {Proveniência:(Gr. \textunderscore oneirogmos\textunderscore )}
\end{itemize}
Pollução nocturna, em resultado de sonho lascivo.
\section{Oniromancia}
\begin{itemize}
\item {Grp. gram.:f.}
\end{itemize}
\begin{itemize}
\item {Proveniência:(Do gr. \textunderscore oneiros\textunderscore  + \textunderscore manteia\textunderscore )}
\end{itemize}
O mesmo que \textunderscore onirocricia\textunderscore .
\section{Onirópolo}
\begin{itemize}
\item {Grp. gram.:m.}
\end{itemize}
\begin{itemize}
\item {Proveniência:(Gr. \textunderscore oneiropolos\textunderscore )}
\end{itemize}
Aquelle que adivinha ou interpreta os sonhos de outrem.
\section{Oniroscopia}
\begin{itemize}
\item {Grp. gram.:f.}
\end{itemize}
\begin{itemize}
\item {Proveniência:(Do gr. \textunderscore oneiros\textunderscore  + \textunderscore skopein\textunderscore )}
\end{itemize}
O mesmo que \textunderscore onirocricia\textunderscore .
\section{Oníscia}
\begin{itemize}
\item {Grp. gram.:f.}
\end{itemize}
Gênero de molluscos gasterópodes.
(Cp. \textunderscore onisco\textunderscore )
\section{Onisco}
\begin{itemize}
\item {Grp. gram.:m.}
\end{itemize}
\begin{itemize}
\item {Proveniência:(Lat. \textunderscore oniscus\textunderscore )}
\end{itemize}
Antiga designação do bicho-de-conta.
O mesmo que \textunderscore ónix\textunderscore .
\section{Oníscoda}
\begin{itemize}
\item {Grp. gram.:f.}
\end{itemize}
\begin{itemize}
\item {Proveniência:(Do gr. \textunderscore oniskos\textunderscore  + \textunderscore eidos\textunderscore ?)}
\end{itemize}
Gênero de crustáceos decápodes.
\section{Oniscografia}
\begin{itemize}
\item {Grp. gram.:f.}
\end{itemize}
\begin{itemize}
\item {Proveniência:(Do gr. \textunderscore oniscos\textunderscore  + \textunderscore graphein\textunderscore )}
\end{itemize}
Tratado zoológico á cêrca do bicho-de-conta.
\section{Oniscográfico}
\begin{itemize}
\item {Grp. gram.:adj.}
\end{itemize}
Relativo á oniscografia.
\section{Oniscographia}
\begin{itemize}
\item {Grp. gram.:f.}
\end{itemize}
\begin{itemize}
\item {Proveniência:(Do gr. \textunderscore oniscos\textunderscore  + \textunderscore graphein\textunderscore )}
\end{itemize}
Tratado zoológico á cêrca do bicho-de-conta.
\section{Oniscográphico}
\begin{itemize}
\item {Grp. gram.:adj.}
\end{itemize}
Relativo á oniscographia.
\section{Oniticello}
\begin{itemize}
\item {Grp. gram.:m.}
\end{itemize}
Gênero de insectos coleópteros pentâmeros.
\section{Oniticelo}
\begin{itemize}
\item {Grp. gram.:m.}
\end{itemize}
Gênero de insectos coleópteros pentâmeros.
\section{Onjudo}
\begin{itemize}
\item {Grp. gram.:adj.}
\end{itemize}
\begin{itemize}
\item {Utilização:Ant.}
\end{itemize}
O mesmo que \textunderscore ungido\textunderscore .
Baptizado; christão.
(Por \textunderscore unjudo\textunderscore , part. ant. de \textunderscore ungir\textunderscore )
\section{Onobríqueas}
\begin{itemize}
\item {Grp. gram.:f. pl.}
\end{itemize}
Tríbo de plantas leguminosas, que tem por tipo o onobríquis.
\section{Onobríquis}
\begin{itemize}
\item {Grp. gram.:m.}
\end{itemize}
\begin{itemize}
\item {Proveniência:(Gr. \textunderscore onobrokhis\textunderscore )}
\end{itemize}
O mesmo que \textunderscore sanfeno\textunderscore .
\section{Onobroma}
\begin{itemize}
\item {Grp. gram.:f.}
\end{itemize}
Gênero de plantas orientaes, da fam. das compostas.
\section{Onobrýcheas}
\begin{itemize}
\item {fónica:qui}
\end{itemize}
\begin{itemize}
\item {Grp. gram.:f. pl.}
\end{itemize}
Tríbo de plantas leguminosas, que tem por typo o onobrýchis.
\section{Onobrýchis}
\begin{itemize}
\item {fónica:qui}
\end{itemize}
\begin{itemize}
\item {Grp. gram.:m.}
\end{itemize}
\begin{itemize}
\item {Proveniência:(Gr. \textunderscore onobrokhis\textunderscore )}
\end{itemize}
O mesmo que \textunderscore sanfeno\textunderscore .
\section{Onocéfalo}
\begin{itemize}
\item {Grp. gram.:m.}
\end{itemize}
\begin{itemize}
\item {Proveniência:(Do gr. \textunderscore onos\textunderscore  + \textunderscore kephale\textunderscore )}
\end{itemize}
Gênero de insectos longicórneos.
\section{Onocentauro}
\begin{itemize}
\item {Grp. gram.:m.}
\end{itemize}
\begin{itemize}
\item {Utilização:Poét.}
\end{itemize}
\begin{itemize}
\item {Proveniência:(Do gr. \textunderscore onos\textunderscore  + \textunderscore kentauros\textunderscore )}
\end{itemize}
Monstro fabuloso, metade homem e metade burro.
\section{Onocéphalo}
\begin{itemize}
\item {Grp. gram.:m.}
\end{itemize}
\begin{itemize}
\item {Proveniência:(Do gr. \textunderscore onos\textunderscore  + \textunderscore kephale\textunderscore )}
\end{itemize}
Gênero de insectos longicórneos.
\section{Onóclea}
\begin{itemize}
\item {Grp. gram.:f.}
\end{itemize}
Gênero de fêtos polypodiáceos da América.
\section{Onócola}
\begin{itemize}
\item {Grp. gram.:f.}
\end{itemize}
\begin{itemize}
\item {Proveniência:(Do gr. \textunderscore onos\textunderscore  + \textunderscore kolon\textunderscore )}
\end{itemize}
Monstro fabuloso, com pés de burro.
\section{Onocrótalo}
\begin{itemize}
\item {Grp. gram.:m.}
\end{itemize}
\begin{itemize}
\item {Proveniência:(Gr. \textunderscore onokrotalos\textunderscore )}
\end{itemize}
O mesmo que \textunderscore pelicano\textunderscore .
\section{Onofrita}
\begin{itemize}
\item {Grp. gram.:f.}
\end{itemize}
O mesmo que \textunderscore onofrito\textunderscore .
\section{Onofrito}
\begin{itemize}
\item {Grp. gram.:m.}
\end{itemize}
\begin{itemize}
\item {Proveniência:(De \textunderscore Onofre\textunderscore , n. p.)}
\end{itemize}
Mineral, com a apparência de cobre pardacento.
\section{Onolatria}
\begin{itemize}
\item {Grp. gram.:f.}
\end{itemize}
\begin{itemize}
\item {Proveniência:(Do gr. \textunderscore onos\textunderscore  + \textunderscore latreia\textunderscore )}
\end{itemize}
Confiança que, entre os antigos, se depositava nas virtudes medicinaes das differentes partes do jumento.
\section{Onomancia}
\textunderscore f.\textunderscore  (e der.)
(V. \textunderscore onomatomancia\textunderscore , etc.) Cf. \textunderscore Hyssope\textunderscore , 118 e 137.
\section{Onomania}
\begin{itemize}
\item {Grp. gram.:f.}
\end{itemize}
O mesmo que \textunderscore onemania\textunderscore .
\section{Onomástica}
\begin{itemize}
\item {Grp. gram.:f.}
\end{itemize}
Lista, relação ou catálogo de nomes.
(Fem. de \textunderscore onomástico\textunderscore )
\section{Onicofagia}
\begin{itemize}
\item {Grp. gram.:f.}
\end{itemize}
\begin{itemize}
\item {Proveniência:(Do gr. \textunderscore onux\textunderscore  + \textunderscore phagein\textunderscore )}
\end{itemize}
Hábito ou mania de roer as unhas.
\section{Onicófago}
\begin{itemize}
\item {Grp. gram.:m.}
\end{itemize}
Aquele que tem onicofagia.
\section{Onicofima}
\begin{itemize}
\item {Grp. gram.:m.}
\end{itemize}
\begin{itemize}
\item {Utilização:Med.}
\end{itemize}
\begin{itemize}
\item {Proveniência:(Do gr. \textunderscore onux\textunderscore  + \textunderscore phuma\textunderscore )}
\end{itemize}
Tumefacção das unhas.
\section{Onicogripose}
\begin{itemize}
\item {Grp. gram.:f.}
\end{itemize}
\begin{itemize}
\item {Utilização:Med.}
\end{itemize}
\begin{itemize}
\item {Proveniência:(Do gr. \textunderscore onux\textunderscore  + \textunderscore gruposis\textunderscore )}
\end{itemize}
Encurvadura das unhas.
\section{Onicomancia}
\begin{itemize}
\item {Grp. gram.:f.}
\end{itemize}
\begin{itemize}
\item {Proveniência:(Do gr. \textunderscore onux\textunderscore  + \textunderscore manteia\textunderscore )}
\end{itemize}
Supposta adivinhação, pela observação das unhas.
\section{Onicomântico}
\begin{itemize}
\item {Grp. gram.:adj.}
\end{itemize}
Relativo á \textunderscore onicomancia\textunderscore .
\section{Onicopatia}
\begin{itemize}
\item {Grp. gram.:f.}
\end{itemize}
\begin{itemize}
\item {Utilização:Med.}
\end{itemize}
\begin{itemize}
\item {Proveniência:(Do gr. \textunderscore onux\textunderscore  + \textunderscore pathos\textunderscore )}
\end{itemize}
Moléstia nas unhas.
\section{Onicoptose}
\begin{itemize}
\item {Grp. gram.:f.}
\end{itemize}
\begin{itemize}
\item {Utilização:Med.}
\end{itemize}
\begin{itemize}
\item {Proveniência:(Do gr. \textunderscore onux\textunderscore  + \textunderscore ptosis\textunderscore )}
\end{itemize}
Quéda das unhas.
\section{Onígena}
\begin{itemize}
\item {Grp. gram.:f.}
\end{itemize}
\begin{itemize}
\item {Proveniência:(Do gr. \textunderscore onux\textunderscore  + \textunderscore genea\textunderscore )}
\end{itemize}
Gênero de cogumelos gasteromicetos.
\section{Onipterígia}
\begin{itemize}
\item {Grp. gram.:f.}
\end{itemize}
\begin{itemize}
\item {Proveniência:(Do gr. \textunderscore onux\textunderscore  + \textunderscore pteron\textunderscore )}
\end{itemize}
Gênero de insectos coleópteros pentâmeros.
\section{Oniquia}
\begin{itemize}
\item {Grp. gram.:f.}
\end{itemize}
\begin{itemize}
\item {Proveniência:(Do gr. \textunderscore onux\textunderscore , \textunderscore onukhos\textunderscore )}
\end{itemize}
Inflamação das unhas.
\section{Oníquio}
\begin{itemize}
\item {Grp. gram.:m.}
\end{itemize}
\begin{itemize}
\item {Proveniência:(Do gr. \textunderscore onux\textunderscore , unha)}
\end{itemize}
Gênero de fêtos polipodiáceos.
\section{Oniquito}
\begin{itemize}
\item {Grp. gram.:adj.}
\end{itemize}
Diz-se de uma espécie de alabastro que contém ónix.
\section{Ónix}
\begin{itemize}
\item {Grp. gram.:m.}
\end{itemize}
\begin{itemize}
\item {Proveniência:(Lat. \textunderscore onix\textunderscore )}
\end{itemize}
Ágata muito fina, que apresenta camadas paralelas de diferentes côres.
\section{Onixe}
\begin{itemize}
\item {fónica:cse}
\end{itemize}
\begin{itemize}
\item {Grp. gram.:m.}
\end{itemize}
\begin{itemize}
\item {Proveniência:(Do gr. \textunderscore onux\textunderscore )}
\end{itemize}
Unha encravada.
\section{Oníxis}
\begin{itemize}
\item {Grp. gram.:m.}
\end{itemize}
\begin{itemize}
\item {Proveniência:(Do gr. \textunderscore onux\textunderscore )}
\end{itemize}
Unha encravada.
\section{Onomástico}
\begin{itemize}
\item {Grp. gram.:adj.}
\end{itemize}
\begin{itemize}
\item {Grp. gram.:M.}
\end{itemize}
\begin{itemize}
\item {Proveniência:(Gr. \textunderscore onomastikos\textunderscore )}
\end{itemize}
Relativo aos nomes próprios.
Onomástica.
\section{Onomático}
\begin{itemize}
\item {Grp. gram.:adj.}
\end{itemize}
\begin{itemize}
\item {Proveniência:(Do gr. \textunderscore onoma\textunderscore )}
\end{itemize}
Relativo a nome.
\section{Onomatologia}
\begin{itemize}
\item {Grp. gram.:f.}
\end{itemize}
\begin{itemize}
\item {Proveniência:(Do gr. \textunderscore onoma\textunderscore  + \textunderscore logos\textunderscore )}
\end{itemize}
Tratado de nomes ou da sua classificação.
\section{Onomatológico}
\begin{itemize}
\item {Grp. gram.:adj.}
\end{itemize}
Relativo á onomatologia.
\section{Onomatólogo}
\begin{itemize}
\item {Grp. gram.:m.}
\end{itemize}
Aquelle que é perito em onomatologia.
\section{Onomatomancia}
\begin{itemize}
\item {Grp. gram.:f.}
\end{itemize}
\begin{itemize}
\item {Proveniência:(Do gr. \textunderscore onoma\textunderscore  + \textunderscore manteia\textunderscore )}
\end{itemize}
Adivinhação, baseada em o nome de pessôas, em o número das letras dêsse nome, etc.
\section{Onomatomania}
\begin{itemize}
\item {Grp. gram.:f.}
\end{itemize}
\begin{itemize}
\item {Utilização:Med.}
\end{itemize}
\begin{itemize}
\item {Proveniência:(Do gr. \textunderscore onoma\textunderscore  + \textunderscore mania\textunderscore )}
\end{itemize}
Insistência mórbida em procurar um vocábulo, ou em evitar uma expressão, cuja pronúncia afflige.
\section{Onomatomaníaco}
\begin{itemize}
\item {Grp. gram.:m.  e  adj.}
\end{itemize}
Aquelle que soffre onomatomania. Cf. Sousa Martins, \textunderscore Nosographia\textunderscore .
\section{Onomatómano}
\begin{itemize}
\item {Grp. gram.:m.  e  adj.}
\end{itemize}
Aquelle que soffre onomatomania. Cf. Sousa Martins, \textunderscore Nosographia\textunderscore .
\section{Onomatomântico}
\begin{itemize}
\item {Grp. gram.:adj.}
\end{itemize}
Relativo á onomatomancia.
\section{Onomatopaico}
\begin{itemize}
\item {Grp. gram.:adj.}
\end{itemize}
Relativo á onomatopeia.
Que tem o carácter da onomatopeia: \textunderscore vocábulo onomatopaico\textunderscore .
\section{Onomatopéa}
\begin{itemize}
\item {Grp. gram.:f.}
\end{itemize}
\begin{itemize}
\item {Proveniência:(Gr. \textunderscore onomatopeia\textunderscore )}
\end{itemize}
Formação de uma palavra, cujo som imita o que significa.
Palavra, com essa formação.
\section{Onomatopeia}
\begin{itemize}
\item {Grp. gram.:f.}
\end{itemize}
\begin{itemize}
\item {Proveniência:(Gr. \textunderscore onomatopeia\textunderscore )}
\end{itemize}
Formação de uma palavra, cujo som imita o que significa.
Palavra, com essa formação.
\section{Onomatópico}
\begin{itemize}
\item {Grp. gram.:adj.}
\end{itemize}
O mesmo que \textunderscore onomatopaico\textunderscore .
\section{Onomatópose}
\begin{itemize}
\item {Grp. gram.:f.}
\end{itemize}
Nome disfarçado; pseudónymo.
\section{Onónide}
\begin{itemize}
\item {Grp. gram.:m.}
\end{itemize}
\begin{itemize}
\item {Proveniência:(Gr. \textunderscore onopordos\textunderscore )}
\end{itemize}
Gênero de plantas herbáceas, da fam. das compostas.
\section{Onopórdio}
\begin{itemize}
\item {Grp. gram.:m.}
\end{itemize}
\begin{itemize}
\item {Proveniência:(Gr. \textunderscore ononis\textunderscore )}
\end{itemize}
Gênero de plantas herbáceas, da fam. das compostas.
\section{Onóscelo}
\begin{itemize}
\item {Grp. gram.:m.}
\end{itemize}
Monstro com pés de jumento.
\section{Onoséride}
\begin{itemize}
\item {Grp. gram.:f.}
\end{itemize}
Gênero de plantas, typo das onoserídeas.
\section{Onoserídeas}
\begin{itemize}
\item {Grp. gram.:f. pl.}
\end{itemize}
Tríbo de plantas, composta de ervas vivazes, originárias da Nova-Granada.
\section{Onosma}
\begin{itemize}
\item {Grp. gram.:f.}
\end{itemize}
\begin{itemize}
\item {Proveniência:(Do gr. \textunderscore onos\textunderscore , burro, e \textunderscore osme\textunderscore , cheiro)}
\end{itemize}
Gênero de plantas borragíneas.
\section{Onosmódio}
\begin{itemize}
\item {Grp. gram.:m.}
\end{itemize}
Gênero de plantas asperifóliáceas da América boreal.
\section{Onotauro}
\begin{itemize}
\item {Grp. gram.:m.}
\end{itemize}
\begin{itemize}
\item {Proveniência:(Do gr. \textunderscore onus\textunderscore  + lat. \textunderscore taurus\textunderscore )}
\end{itemize}
Quadrúpede, filho do toiro e jumenta, ou do jumento e vaca, ou de cavallo e vaca ou de toiro e égua.
\section{Onotera}
\begin{itemize}
\item {Grp. gram.:f.}
\end{itemize}
\begin{itemize}
\item {Utilização:Bot.}
\end{itemize}
\begin{itemize}
\item {Proveniência:(Lat. \textunderscore onothera\textunderscore )}
\end{itemize}
O mesmo que \textunderscore onagra\textunderscore .
\section{Onoteráceas}
\begin{itemize}
\item {Grp. gram.:f. pl.}
\end{itemize}
\begin{itemize}
\item {Proveniência:(De \textunderscore onoteráceo\textunderscore )}
\end{itemize}
O mesmo que \textunderscore onagrárias\textunderscore .
\section{Onoteráceo}
\begin{itemize}
\item {Grp. gram.:adj.}
\end{itemize}
\begin{itemize}
\item {Proveniência:(De \textunderscore onotera\textunderscore )}
\end{itemize}
O mesmo que \textunderscore onagrário\textunderscore .
\section{Onothera}
\begin{itemize}
\item {Grp. gram.:f.}
\end{itemize}
\begin{itemize}
\item {Utilização:Bot.}
\end{itemize}
\begin{itemize}
\item {Proveniência:(Lat. \textunderscore onothera\textunderscore )}
\end{itemize}
O mesmo que \textunderscore onagra\textunderscore .
\section{Onotheráceas}
\begin{itemize}
\item {Grp. gram.:f. pl.}
\end{itemize}
\begin{itemize}
\item {Proveniência:(De \textunderscore onotheráceo\textunderscore )}
\end{itemize}
O mesmo que \textunderscore onagrárias\textunderscore .
\section{Onotheráceo}
\begin{itemize}
\item {Grp. gram.:adj.}
\end{itemize}
\begin{itemize}
\item {Proveniência:(De \textunderscore onothera\textunderscore )}
\end{itemize}
O mesmo que \textunderscore onagrário\textunderscore .
\section{Onquídia}
\begin{itemize}
\item {Grp. gram.:f.}
\end{itemize}
\begin{itemize}
\item {Utilização:Bot.}
\end{itemize}
Parte interna do botão dos vegetaes ou o seu núcleo vital interno.
\section{Onquídio}
\begin{itemize}
\item {Grp. gram.:m.}
\end{itemize}
Gênero de moluscos gasterópodes.
\section{Ontem}
\begin{itemize}
\item {Grp. gram.:adv.}
\end{itemize}
\begin{itemize}
\item {Utilização:Ext.}
\end{itemize}
\begin{itemize}
\item {Proveniência:(Do lat. \textunderscore ante\textunderscore  ou, melhor, de \textunderscore á\textunderscore  + \textunderscore noute\textunderscore  &gt; \textunderscore anoute\textunderscore  &gt; \textunderscore anonte\textunderscore  &gt; \textunderscore aonte\textunderscore  &gt; \textunderscore onte\textunderscore  &gt; \textunderscore ontem\textunderscore )}
\end{itemize}
No dia immediatamente anterior ao actual ou ao de hoje.
No tempo ou época que precedeu immediatamente a época ou o tempo actual.
No tempo que passou, mas cuja recordação está presente.--A orthogr. \textunderscore hontem\textunderscore , embora usada, não tem justificação scientífica.
\section{Ontófago}
\begin{itemize}
\item {Grp. gram.:m.}
\end{itemize}
\begin{itemize}
\item {Proveniência:(Do gr. \textunderscore on\textunderscore , \textunderscore ontos\textunderscore  + \textunderscore phagein\textunderscore )}
\end{itemize}
Gênero de insectos coleópteros pentâmeros.
\section{Ontogênese}
\begin{itemize}
\item {Grp. gram.:f.}
\end{itemize}
\begin{itemize}
\item {Proveniência:(Do gr. \textunderscore ontos\textunderscore  + \textunderscore genesis\textunderscore )}
\end{itemize}
Producção de seres orgânicos. Cf. Latino, \textunderscore Or. da Corôa\textunderscore , CLXIV.
\section{Ontogenético}
\begin{itemize}
\item {Grp. gram.:adj.}
\end{itemize}
Relativo á ontogênese.
\section{Ontogenia}
\begin{itemize}
\item {Grp. gram.:f.}
\end{itemize}
O mesmo que \textunderscore ontogonia\textunderscore .
Desenvolvimento do indivíduo, evolução peculiar aos seres de cada espécie.
\section{Ontogênico}
\begin{itemize}
\item {Grp. gram.:adj.}
\end{itemize}
Relativo a \textunderscore ontogenia\textunderscore .
\section{Ontogonia}
\begin{itemize}
\item {Grp. gram.:f.}
\end{itemize}
\begin{itemize}
\item {Proveniência:(Do gr. \textunderscore on\textunderscore , \textunderscore ontos\textunderscore  + \textunderscore gonos\textunderscore )}
\end{itemize}
História da producção dos seres organizados sôbre a terra.
\section{Ontogónico}
\begin{itemize}
\item {Grp. gram.:adj.}
\end{itemize}
Relativo á ontogonia.
\section{Ontologia}
\begin{itemize}
\item {Grp. gram.:f.}
\end{itemize}
\begin{itemize}
\item {Utilização:Med.}
\end{itemize}
\begin{itemize}
\item {Proveniência:(Do gr. \textunderscore on\textunderscore , \textunderscore ontos\textunderscore  + \textunderscore logos\textunderscore )}
\end{itemize}
Theoria ou sciência do sêr.
Metaphýsica.
Doutrina que, ao contrário da doutrina physiológica, não liga os phenómenos pathológicos aos phenómenos regulares da vida.
\section{Ontologicamente}
\begin{itemize}
\item {Grp. gram.:adv.}
\end{itemize}
De modo ontológico; segundo a ontologia.
\section{Ontológico}
\begin{itemize}
\item {Grp. gram.:adj.}
\end{itemize}
Relativo á ontologia.
\section{Ontologista}
\begin{itemize}
\item {Grp. gram.:m.  e  f.}
\end{itemize}
Pessôa, que se occupa de ontologia.
\section{Ontóphago}
\begin{itemize}
\item {Grp. gram.:m.}
\end{itemize}
\begin{itemize}
\item {Proveniência:(Do gr. \textunderscore on\textunderscore , \textunderscore ontos\textunderscore  + \textunderscore phagein\textunderscore )}
\end{itemize}
Gênero de insectos coleópteros pentâmeros.
\section{Ónus}
\begin{itemize}
\item {Grp. gram.:m.}
\end{itemize}
\begin{itemize}
\item {Utilização:Fig.}
\end{itemize}
\begin{itemize}
\item {Proveniência:(Lat. \textunderscore onus\textunderscore )}
\end{itemize}
Pêso.
Aquillo que pesa, carga.
Encargo, obrigação.
Imposto gravoso.
\section{Onusto}
\begin{itemize}
\item {Grp. gram.:adj.}
\end{itemize}
\begin{itemize}
\item {Proveniência:(Lat. \textunderscore onustus\textunderscore )}
\end{itemize}
Carregado; sobrecarregado; repleto.
\section{Onychia}
\begin{itemize}
\item {fónica:qui}
\end{itemize}
\begin{itemize}
\item {Grp. gram.:f.}
\end{itemize}
\begin{itemize}
\item {Proveniência:(Do gr. \textunderscore onux\textunderscore , \textunderscore onukhos\textunderscore )}
\end{itemize}
Inflammação das unhas.
\section{Onýchio}
\begin{itemize}
\item {fónica:qui}
\end{itemize}
\begin{itemize}
\item {Grp. gram.:m.}
\end{itemize}
\begin{itemize}
\item {Proveniência:(Do gr. \textunderscore onux\textunderscore , unha)}
\end{itemize}
Gênero de fetos polypodiáceos.
\section{Onychito}
\begin{itemize}
\item {fónica:qui}
\end{itemize}
\begin{itemize}
\item {Grp. gram.:adj.}
\end{itemize}
Diz-se de uma espécie do alabastro que contém ónix.
\section{Onychogripose}
\begin{itemize}
\item {fónica:co}
\end{itemize}
\begin{itemize}
\item {Grp. gram.:f.}
\end{itemize}
\begin{itemize}
\item {Utilização:Med.}
\end{itemize}
\begin{itemize}
\item {Proveniência:(Do gr. \textunderscore onux\textunderscore  + \textunderscore gruposis\textunderscore )}
\end{itemize}
Encurvadura das unhas.
\section{Onychomancia}
\begin{itemize}
\item {fónica:co}
\end{itemize}
\begin{itemize}
\item {Grp. gram.:f.}
\end{itemize}
\begin{itemize}
\item {Proveniência:(Do gr. \textunderscore onux\textunderscore  + \textunderscore manteia\textunderscore )}
\end{itemize}
Supposta adivinhação, pela observação das unhas.
\section{Onychomântico}
\begin{itemize}
\item {fónica:co}
\end{itemize}
\begin{itemize}
\item {Grp. gram.:adj.}
\end{itemize}
Relativo á \textunderscore onychomancia\textunderscore .
\section{Onychopathia}
\begin{itemize}
\item {fónica:co}
\end{itemize}
\begin{itemize}
\item {Grp. gram.:f.}
\end{itemize}
\begin{itemize}
\item {Utilização:Med.}
\end{itemize}
\begin{itemize}
\item {Proveniência:(Do gr. \textunderscore onux\textunderscore  + \textunderscore pathos\textunderscore )}
\end{itemize}
Moléstia nas unhas.
\section{Onychophagia}
\begin{itemize}
\item {fónica:co}
\end{itemize}
\begin{itemize}
\item {Grp. gram.:f.}
\end{itemize}
\begin{itemize}
\item {Proveniência:(Do gr. \textunderscore onux\textunderscore  + \textunderscore phagein\textunderscore )}
\end{itemize}
Hábito ou mania de roer as unhas.
\section{Onychóphago}
\begin{itemize}
\item {fónica:có}
\end{itemize}
\begin{itemize}
\item {Grp. gram.:m.}
\end{itemize}
Aquelle que tem onychophagia.
\section{Onychophyma}
\begin{itemize}
\item {fónica:có}
\end{itemize}
\begin{itemize}
\item {Grp. gram.:m.}
\end{itemize}
\begin{itemize}
\item {Utilização:Med.}
\end{itemize}
\begin{itemize}
\item {Proveniência:(Do gr. \textunderscore onux\textunderscore  + \textunderscore phuma\textunderscore )}
\end{itemize}
Tumefacção das unhas.
\section{Onychoptose}
\begin{itemize}
\item {fónica:co}
\end{itemize}
\begin{itemize}
\item {Grp. gram.:f.}
\end{itemize}
\begin{itemize}
\item {Utilização:Med.}
\end{itemize}
\begin{itemize}
\item {Proveniência:(Do gr. \textunderscore onux\textunderscore  + \textunderscore ptosis\textunderscore )}
\end{itemize}
Quéda das unhas.
\section{Onýgena}
\begin{itemize}
\item {Grp. gram.:f.}
\end{itemize}
\begin{itemize}
\item {Proveniência:(Do gr. \textunderscore onux\textunderscore  + \textunderscore genea\textunderscore )}
\end{itemize}
Gênero de cogumelos gasteromycetos.
\section{Onypterýgia}
\begin{itemize}
\item {Grp. gram.:f.}
\end{itemize}
\begin{itemize}
\item {Proveniência:(Do gr. \textunderscore onux\textunderscore  + \textunderscore pteron\textunderscore )}
\end{itemize}
Gênero de insectos coleópteros pentâmeros.
\section{Ónyx}
\begin{itemize}
\item {Grp. gram.:m.}
\end{itemize}
\begin{itemize}
\item {Proveniência:(Lat. \textunderscore onix\textunderscore )}
\end{itemize}
Ágata muito fina, que apresenta camadas parallelas de differentes côres.
\section{Onýxis}
\begin{itemize}
\item {Grp. gram.:m.}
\end{itemize}
\begin{itemize}
\item {Proveniência:(Do gr. \textunderscore onux\textunderscore )}
\end{itemize}
Unha encravada.
\section{Onze}
\begin{itemize}
\item {Grp. gram.:adj.}
\end{itemize}
\begin{itemize}
\item {Grp. gram.:M.}
\end{itemize}
\begin{itemize}
\item {Proveniência:(Do lat. \textunderscore undecim\textunderscore )}
\end{itemize}
Diz-se do número cardinal, formado de déz e mais um.
Décimo primeiro: \textunderscore Luís Onze\textunderscore .
O que numa série de onze occupa o último lugar.
\section{Onze-letras}
\begin{itemize}
\item {Grp. gram.:f.}
\end{itemize}
\begin{itemize}
\item {Utilização:Pop.}
\end{itemize}
Mulher, que alcovita.
(Referência ás 11 letras de \textunderscore alcoviteira\textunderscore . Cp. \textunderscore déz-e-um\textunderscore )
\section{Onzena}
\begin{itemize}
\item {Grp. gram.:f.}
\end{itemize}
\begin{itemize}
\item {Utilização:Fig.}
\end{itemize}
\begin{itemize}
\item {Utilização:Des.}
\end{itemize}
\begin{itemize}
\item {Proveniência:(De \textunderscore onze\textunderscore )}
\end{itemize}
Juro de onze por cento.
Juro excessivo.
Porção de onze objectos.
\section{Onzenar}
\begin{itemize}
\item {Grp. gram.:v. i.}
\end{itemize}
\begin{itemize}
\item {Proveniência:(De \textunderscore onzena\textunderscore )}
\end{itemize}
Praticar a usura.
Levar grandes juros de quantia emprestada.
Mexericar, intrigar.
\section{Onzenário}
\begin{itemize}
\item {Grp. gram.:adj.}
\end{itemize}
\begin{itemize}
\item {Grp. gram.:M.}
\end{itemize}
Relativo á onzena.
Que é usurário.
Usurário.
\section{Onzenear}
\begin{itemize}
\item {Grp. gram.:v. i.}
\end{itemize}
O mesmo que \textunderscore onzenar\textunderscore .
\section{Onzeneiro}
\begin{itemize}
\item {Grp. gram.:m.  e  adj.}
\end{itemize}
\begin{itemize}
\item {Proveniência:(De \textunderscore onzena\textunderscore )}
\end{itemize}
O mesmo que \textunderscore onzenário\textunderscore .
Mexeriqueiro; intrigante.
\section{Onzenice}
\begin{itemize}
\item {Grp. gram.:f.}
\end{itemize}
\begin{itemize}
\item {Proveniência:(De \textunderscore onzena\textunderscore )}
\end{itemize}
Hábito de mexericar; bisbilhotice.
\section{Onzeno}
\begin{itemize}
\item {Grp. gram.:adj.}
\end{itemize}
\begin{itemize}
\item {Proveniência:(De \textunderscore onze\textunderscore )}
\end{itemize}
O mesmo que \textunderscore undécimo\textunderscore :«\textunderscore ...dos reys de Portugal o onzeno.\textunderscore »Pina, \textunderscore D. Duarte\textunderscore .
\section{Ó-ó}
\begin{itemize}
\item {Grp. gram.:m.}
\end{itemize}
\begin{itemize}
\item {Utilização:Infant.}
\end{itemize}
Berço.
Acto de dormir: \textunderscore fazer ó-ó\textunderscore .
\section{Oocíano}
\begin{itemize}
\item {Grp. gram.:m.}
\end{itemize}
Gênero de insectos coleópteros.
\section{Ooclinínia}
\begin{itemize}
\item {Grp. gram.:m.}
\end{itemize}
Gênero de plantas americanas, da fam. das compostas.
\section{Oocýano}
\begin{itemize}
\item {Grp. gram.:m.}
\end{itemize}
Gênero de insectos coleópteros.
\section{Oode}
\begin{itemize}
\item {Grp. gram.:m.}
\end{itemize}
\begin{itemize}
\item {Proveniência:(Do gr. \textunderscore oon\textunderscore  + \textunderscore eidos\textunderscore )}
\end{itemize}
Gênero de insectos coleópteros, pentâmeros.
\section{Oodo}
\begin{itemize}
\item {Grp. gram.:m.}
\end{itemize}
\begin{itemize}
\item {Proveniência:(Do gr. \textunderscore oon\textunderscore  + \textunderscore eidos\textunderscore )}
\end{itemize}
Gênero de insectos coleópteros, pentâmeros.
\section{Ooforalgia}
\begin{itemize}
\item {Grp. gram.:f.}
\end{itemize}
\begin{itemize}
\item {Utilização:Med.}
\end{itemize}
\begin{itemize}
\item {Proveniência:(Do gr. \textunderscore oon\textunderscore  + \textunderscore phoros\textunderscore  + \textunderscore algos\textunderscore )}
\end{itemize}
Dôr no ovário.
\section{Ooforectomia}
\begin{itemize}
\item {Grp. gram.:f.}
\end{itemize}
\begin{itemize}
\item {Utilização:Med.}
\end{itemize}
\begin{itemize}
\item {Proveniência:(Do gr. \textunderscore oon\textunderscore  + \textunderscore phoros\textunderscore  + \textunderscore ektome\textunderscore )}
\end{itemize}
Ablação do ovário.
\section{Ooforídea}
\begin{itemize}
\item {Grp. gram.:f.}
\end{itemize}
\begin{itemize}
\item {Utilização:Bot.}
\end{itemize}
\begin{itemize}
\item {Proveniência:(Do gr. \textunderscore oon\textunderscore  + \textunderscore phoros\textunderscore )}
\end{itemize}
Cápsula, que, nos licopódios, contém corpos globulosos.
\section{Ooforite}
\begin{itemize}
\item {Grp. gram.:f.}
\end{itemize}
\begin{itemize}
\item {Utilização:Med.}
\end{itemize}
Inflamação dos ovários da mulher.
(Do \textunderscore oóforo\textunderscore )
\section{Oóforo}
\begin{itemize}
\item {Grp. gram.:m.}
\end{itemize}
\begin{itemize}
\item {Proveniência:(Do gr. \textunderscore oon\textunderscore  + \textunderscore phoros\textunderscore )}
\end{itemize}
Uma das designações do \textunderscore ovário\textunderscore .
\section{Ooforomania}
\begin{itemize}
\item {Grp. gram.:f.}
\end{itemize}
\begin{itemize}
\item {Utilização:Med.}
\end{itemize}
\begin{itemize}
\item {Proveniência:(Do gr. \textunderscore oon\textunderscore  + \textunderscore phoros\textunderscore  + \textunderscore mania\textunderscore )}
\end{itemize}
Perturbações nervosas, relacionadas com lesões no ovário.
\section{Oogastro}
\begin{itemize}
\item {Grp. gram.:m.}
\end{itemize}
\begin{itemize}
\item {Proveniência:(Do gr. \textunderscore oon\textunderscore  + \textunderscore gaster\textunderscore )}
\end{itemize}
Gênero de insectos coleópteros.
\section{Oogónio}
\begin{itemize}
\item {Grp. gram.:m.}
\end{itemize}
\begin{itemize}
\item {Utilização:Bot.}
\end{itemize}
\begin{itemize}
\item {Proveniência:(Do gr. \textunderscore oon\textunderscore , ovo + \textunderscore gon\textunderscore , gerar)}
\end{itemize}
Orgão feminino das tallóphytas.
\section{Oolíthico}
\begin{itemize}
\item {Grp. gram.:adj.}
\end{itemize}
Relativo ao oólitho.
\section{Oólitho}
\begin{itemize}
\item {Grp. gram.:m.}
\end{itemize}
\begin{itemize}
\item {Proveniência:(Do gr. \textunderscore oon\textunderscore  + \textunderscore lithos\textunderscore )}
\end{itemize}
Variedade de calcário, composto de pequeninos grãos, semelhantes a ovos de peixe.
\section{Oolítico}
\begin{itemize}
\item {Grp. gram.:adj.}
\end{itemize}
Relativo ao oólito.
\section{Óolito}
\begin{itemize}
\item {Grp. gram.:m.}
\end{itemize}
\begin{itemize}
\item {Proveniência:(Do gr. \textunderscore oon\textunderscore  + \textunderscore lithos\textunderscore )}
\end{itemize}
Variedade de calcário, composto de pequeninos grãos, semelhantes a ovos de peixe.
\section{Oologia}
\begin{itemize}
\item {Grp. gram.:f.}
\end{itemize}
\begin{itemize}
\item {Proveniência:(Do gr. \textunderscore oon\textunderscore  + \textunderscore logos\textunderscore )}
\end{itemize}
Descripção do ovo, sob o ponto de vista da geração.
\section{Oológico}
\begin{itemize}
\item {Grp. gram.:adj.}
\end{itemize}
Relativo á oologia.
\section{Oomancia}
\begin{itemize}
\item {Grp. gram.:f.}
\end{itemize}
\begin{itemize}
\item {Proveniência:(Do gr. \textunderscore oon\textunderscore  + \textunderscore manteia\textunderscore )}
\end{itemize}
Arte de adivinhar, que se praticava por meio de ovos.
\section{Oomântico}
\begin{itemize}
\item {Grp. gram.:adj.}
\end{itemize}
Relativo á oomancia.
\section{Oómetra}
\begin{itemize}
\item {Grp. gram.:f.}
\end{itemize}
\begin{itemize}
\item {Utilização:Bot.}
\end{itemize}
\begin{itemize}
\item {Proveniência:(De \textunderscore oon\textunderscore  + \textunderscore metra\textunderscore )}
\end{itemize}
Ovário das plantas phanerogâmicas.
\section{Oonim}
\begin{itemize}
\item {Grp. gram.:m.}
\end{itemize}
\begin{itemize}
\item {Proveniência:(Do gr. \textunderscore oon\textunderscore )}
\end{itemize}
Nome, que se deu á alteração, soffrida pela albumina, da clara do ovo, quando separada e exposta por um mês a uma temperatura inferior a zero.
\section{Oonina}
\begin{itemize}
\item {Grp. gram.:f.}
\end{itemize}
Membrana reticulada, que contém a albumina da clara do ovo em suas céllulas.
(Cp. \textunderscore oonim\textunderscore )
\section{Oophoralgia}
\begin{itemize}
\item {Grp. gram.:f.}
\end{itemize}
\begin{itemize}
\item {Utilização:Med.}
\end{itemize}
\begin{itemize}
\item {Proveniência:(Do gr. \textunderscore oon\textunderscore  + \textunderscore phoros\textunderscore  + \textunderscore algos\textunderscore )}
\end{itemize}
Dôr no ovário.
\section{Oophorectomia}
\begin{itemize}
\item {Grp. gram.:f.}
\end{itemize}
\begin{itemize}
\item {Utilização:Med.}
\end{itemize}
\begin{itemize}
\item {Proveniência:(Do gr. \textunderscore oon\textunderscore  + \textunderscore phoros\textunderscore  + \textunderscore ektome\textunderscore )}
\end{itemize}
Ablação do ovário.
\section{Oophorídea}
\begin{itemize}
\item {Grp. gram.:f.}
\end{itemize}
\begin{itemize}
\item {Utilização:Bot.}
\end{itemize}
\begin{itemize}
\item {Proveniência:(Do gr. \textunderscore oon\textunderscore  + \textunderscore phoros\textunderscore )}
\end{itemize}
Cápsula, que, nos lycopódios, contém corpos globulosos.
\section{Oophorite}
\begin{itemize}
\item {Grp. gram.:f.}
\end{itemize}
\begin{itemize}
\item {Utilização:Med.}
\end{itemize}
Inflammação dos ovários da mulher.
(Do \textunderscore oóphoro\textunderscore )
\section{Oóphoro}
\begin{itemize}
\item {Grp. gram.:m.}
\end{itemize}
\begin{itemize}
\item {Proveniência:(Do gr. \textunderscore oon\textunderscore  + \textunderscore phoros\textunderscore )}
\end{itemize}
Uma das designações do \textunderscore ovário\textunderscore .
\section{Oophoromania}
\begin{itemize}
\item {Grp. gram.:f.}
\end{itemize}
\begin{itemize}
\item {Utilização:Med.}
\end{itemize}
\begin{itemize}
\item {Proveniência:(Do gr. \textunderscore oon\textunderscore  + \textunderscore phoros\textunderscore  + \textunderscore mania\textunderscore )}
\end{itemize}
Perturbações nervosas, relacionadas com lesões no ovário.
\section{Oóptero}
\begin{itemize}
\item {Grp. gram.:m.}
\end{itemize}
\begin{itemize}
\item {Proveniência:(Do gr. \textunderscore oon\textunderscore  + \textunderscore pteron\textunderscore )}
\end{itemize}
Gênero de insectos coleópteros pentâmeros.
\section{Ooscopia}
\begin{itemize}
\item {Grp. gram.:f.}
\end{itemize}
\begin{itemize}
\item {Proveniência:(Do gr. \textunderscore oon\textunderscore  + \textunderscore okopein\textunderscore )}
\end{itemize}
O mesmo que \textunderscore oomancia\textunderscore .
\section{Ooscópico}
\begin{itemize}
\item {Grp. gram.:adj.}
\end{itemize}
Relativo á ooscopia.
\section{Oosfera}
\begin{itemize}
\item {Grp. gram.:f.}
\end{itemize}
\begin{itemize}
\item {Utilização:Bot.}
\end{itemize}
\begin{itemize}
\item {Proveniência:(Do gr. \textunderscore oon\textunderscore  + \textunderscore sphaira\textunderscore )}
\end{itemize}
Célula feminina que, depois de fecundada, se transforma em ovo.
\section{Oosomo}
\begin{itemize}
\item {fónica:sô}
\end{itemize}
\begin{itemize}
\item {Grp. gram.:m.}
\end{itemize}
\begin{itemize}
\item {Proveniência:(Do gr. \textunderscore oon\textunderscore  + \textunderscore soma\textunderscore )}
\end{itemize}
Gênero de insectos coleópteros tetrâmeros.
\section{Oosphera}
\begin{itemize}
\item {Grp. gram.:f.}
\end{itemize}
\begin{itemize}
\item {Utilização:Bot.}
\end{itemize}
\begin{itemize}
\item {Proveniência:(Do gr. \textunderscore oon\textunderscore  + \textunderscore sphaira\textunderscore )}
\end{itemize}
Céllula feminina que, depois de fecundada, se transforma em ovo.
\section{Oossomo}
\begin{itemize}
\item {Grp. gram.:m.}
\end{itemize}
\begin{itemize}
\item {Proveniência:(Do gr. \textunderscore oon\textunderscore  + \textunderscore soma\textunderscore )}
\end{itemize}
Gênero de insectos coleópteros tetrâmeros.
\section{Ooteca}
\begin{itemize}
\item {Grp. gram.:f.}
\end{itemize}
\begin{itemize}
\item {Utilização:Bot.}
\end{itemize}
\begin{itemize}
\item {Proveniência:(Do gr. \textunderscore oon\textunderscore  + \textunderscore theke\textunderscore )}
\end{itemize}
Ovário dos fêtos.
\section{Ootheca}
\begin{itemize}
\item {Grp. gram.:f.}
\end{itemize}
\begin{itemize}
\item {Utilização:Bot.}
\end{itemize}
\begin{itemize}
\item {Proveniência:(Do gr. \textunderscore oon\textunderscore  + \textunderscore theke\textunderscore )}
\end{itemize}
Ovário dos fêtos.
\section{Oótoma}
\begin{itemize}
\item {Grp. gram.:f.}
\end{itemize}
\begin{itemize}
\item {Proveniência:(Do gr. \textunderscore oon\textunderscore  + \textunderscore tome\textunderscore )}
\end{itemize}
Gênero de insectos coleópteros pentâmeros.
\section{Opa}
\begin{itemize}
\item {Grp. gram.:f.}
\end{itemize}
\begin{itemize}
\item {Proveniência:(Do gr. \textunderscore ope\textunderscore ?)}
\end{itemize}
Espécie do capa sem mangas, tendo, no lugar destas, buracos por onde se enfiam os braços, e usada em actos solennes pelos irmãos de confrarias religiosas.
\section{Opacamente}
\begin{itemize}
\item {Grp. gram.:adv.}
\end{itemize}
De modo opaco; com opacidade.
\section{Opacidade}
\begin{itemize}
\item {Grp. gram.:f.}
\end{itemize}
\begin{itemize}
\item {Proveniência:(Lat. \textunderscore opacitas\textunderscore )}
\end{itemize}
Qualidade do que é opaco; sombra densa.
Lugar sombrio.
\section{Opaco}
\begin{itemize}
\item {Grp. gram.:adj.}
\end{itemize}
\begin{itemize}
\item {Proveniência:(Lat. \textunderscore opacus\textunderscore )}
\end{itemize}
Coberto de sombra.
Obscuro.
Turvo.
Que não deixa passar a luz.
\section{Opado}
\begin{itemize}
\item {Grp. gram.:adj.}
\end{itemize}
Grosso; entumecido.
Balofo.
Inchado.
(Contr. de \textunderscore oppilado\textunderscore )
\section{Opala}
\begin{itemize}
\item {Grp. gram.:f.}
\end{itemize}
\begin{itemize}
\item {Proveniência:(Fr. \textunderscore opale\textunderscore )}
\end{itemize}
Pedra quartzosa, de côr leitosa e azulada, mas que, exposta á incidência da luz, apresenta côres vivas e variadas.
\section{Opalanda}
\begin{itemize}
\item {Grp. gram.:f.}
\end{itemize}
\begin{itemize}
\item {Proveniência:(Do fr. ant. \textunderscore houpelande\textunderscore )}
\end{itemize}
Vestuário talar; opa grande. Cf. \textunderscore Peregrinação\textunderscore , LXXXII.
\section{Opalescência}
\begin{itemize}
\item {Grp. gram.:f.}
\end{itemize}
Qualidade do que é opalescente.
\section{Opalescente}
\begin{itemize}
\item {Grp. gram.:adj.}
\end{itemize}
O mesmo que \textunderscore opalino\textunderscore .
\section{Opálias}
\begin{itemize}
\item {Grp. gram.:f. pl.}
\end{itemize}
\begin{itemize}
\item {Proveniência:(Lat. \textunderscore opalia\textunderscore )}
\end{itemize}
Festas romanas, em honra da deusa Ops.
\section{Opalífero}
\begin{itemize}
\item {Grp. gram.:adj.}
\end{itemize}
\begin{itemize}
\item {Proveniência:(Do lat. \textunderscore opalus\textunderscore  + \textunderscore ferre\textunderscore )}
\end{itemize}
Diz-se do mineral, que póde adquirir brilho meio resinoso.
\section{Opalina}
\begin{itemize}
\item {Grp. gram.:f.}
\end{itemize}
\begin{itemize}
\item {Proveniência:(De \textunderscore opalino\textunderscore )}
\end{itemize}
Gênero de infusórios que se encontra no ventre das rans.
\section{Opalino}
\begin{itemize}
\item {Grp. gram.:adj.}
\end{itemize}
Que tem a côr ou os reflexos da opala.
\section{Opalizado}
\begin{itemize}
\item {Grp. gram.:adj.}
\end{itemize}
Que tem fórma de opala; convertido em opala.
\section{Opalizar}
\begin{itemize}
\item {Grp. gram.:v. t.}
\end{itemize}
Dar côr de opala a. Cf. Camillo, \textunderscore Narcót.\textunderscore , I, 163.
\section{Oparlanda}
\begin{itemize}
\item {Grp. gram.:f.}
\end{itemize}
\begin{itemize}
\item {Utilização:Ant.}
\end{itemize}
O mesmo que \textunderscore opalanda\textunderscore .
\section{Opática}
\begin{itemize}
\item {Grp. gram.:f.}
\end{itemize}
Planta da serra de Sintra.--T. grave ou esdrúxulo? Cf. Juromenha, \textunderscore Cintra Pinturesca\textunderscore .
Provavelmente, fórma pop. de \textunderscore hepática\textunderscore .
\section{Opatrídeo}
\begin{itemize}
\item {Grp. gram.:adj.}
\end{itemize}
\begin{itemize}
\item {Grp. gram.:M. pl.}
\end{itemize}
\begin{itemize}
\item {Proveniência:(De \textunderscore ópatro\textunderscore  + gr. \textunderscore eidos\textunderscore )}
\end{itemize}
Relativo ou semelhante ao ópatro.
Tríbo de insectos, que têm por typo o ópatro.
\section{Ópatro}
\begin{itemize}
\item {Grp. gram.:m.}
\end{itemize}
Insecto, que vive nos detritos das fôlhas, madeira, estrume, etc. e ataca os viveiros das videiras.
\section{Opção}
\begin{itemize}
\item {Grp. gram.:f.}
\end{itemize}
\begin{itemize}
\item {Proveniência:(Lat. \textunderscore optio\textunderscore )}
\end{itemize}
Acto ou faculdade do optar; preferência; escolha.
\section{Ópera}
\begin{itemize}
\item {Grp. gram.:f.}
\end{itemize}
\begin{itemize}
\item {Utilização:Restrict.}
\end{itemize}
\begin{itemize}
\item {Proveniência:(It. \textunderscore opera\textunderscore )}
\end{itemize}
Poema dramático, posto em música.
Grande poema lýrico, composto de versos recitativos, canto e dança, sem diálogo falado.
Theatro, em que se representam esses poemas.
\textunderscore Ópera cómica\textunderscore , espécie de ópera em que se alterna a música e o diálogo falado de feição cómica.
\section{Operação}
\begin{itemize}
\item {Grp. gram.:f.}
\end{itemize}
\begin{itemize}
\item {Proveniência:(Lat. \textunderscore operatio\textunderscore )}
\end{itemize}
Acto ou effeito de operar.
Complexo de meios, combinados para a consecução de um fim.
Qualquer transacção commercial.
Cálculo ou série de cálculos para se obter um resultado ou resolver um problema.
Tudo que o cirurgião, no desempenho de seus mesteres e por meio de instrumento ou sem elle, pratíca num corpo vivo.
Movimento de ataque ou defesa de um exército.
Tudo que o chímico ou o pharmacêutico faz para analysar um corpo, determinar combinações ou preparar medicamentos.
\section{Operado}
\begin{itemize}
\item {Grp. gram.:m.}
\end{itemize}
\begin{itemize}
\item {Proveniência:(De \textunderscore operar\textunderscore )}
\end{itemize}
Aquelle que soffreu operação cirúrgica.
\section{Operador}
\begin{itemize}
\item {Grp. gram.:adj.}
\end{itemize}
\begin{itemize}
\item {Proveniência:(Do lat. \textunderscore operator\textunderscore )}
\end{itemize}
Que opéra.
Aquelle que faz operações cirúrgicas ou chímicas.
Órgão, que nas máquinas opéra o trabalho.
\section{Operagem}
\begin{itemize}
\item {Grp. gram.:f.}
\end{itemize}
Trabalho de operários. Cf. Rui Barb., \textunderscore Réplica\textunderscore , 158.
\section{Operante}
\begin{itemize}
\item {Grp. gram.:adj.}
\end{itemize}
\begin{itemize}
\item {Proveniência:(Lat. \textunderscore operans\textunderscore )}
\end{itemize}
Que opéra; próprio para operar.
\section{Operar}
\begin{itemize}
\item {Grp. gram.:v. t.}
\end{itemize}
\begin{itemize}
\item {Grp. gram.:V. i.}
\end{itemize}
\begin{itemize}
\item {Proveniência:(Lat. \textunderscore operari\textunderscore )}
\end{itemize}
Produzir (um effeito, alguma coisa).
Obrar; executar.
Sujeitar a uma operação cirúrgica: \textunderscore operar um abscesso\textunderscore .
Procurar o resultado mathemático ou chímico de.
Têr effeito.
Produzir dejecções.
Fazer uma operação.
\section{Ofato}
\begin{itemize}
\item {Grp. gram.:m.}
\end{itemize}
Variedade de mármore.
\section{Ofélia}
\begin{itemize}
\item {Grp. gram.:f.}
\end{itemize}
Gênero de plantas gencianáceas.
\section{Ofia}
\begin{itemize}
\item {Grp. gram.:f.}
\end{itemize}
Gênero de aves da América do Sul.
\section{Ofíase}
\begin{itemize}
\item {Grp. gram.:f.}
\end{itemize}
\begin{itemize}
\item {Proveniência:(Gr. \textunderscore ophiasis\textunderscore )}
\end{itemize}
Espécie de alopecia, em que os cabelos caem por partes.
\section{Oficálcia}
\begin{itemize}
\item {Grp. gram.:f.}
\end{itemize}
\begin{itemize}
\item {Utilização:Miner.}
\end{itemize}
\begin{itemize}
\item {Proveniência:(De \textunderscore ophi...\textunderscore  + \textunderscore cálcio\textunderscore )}
\end{itemize}
Rocha calcária avermelhada, misturada com silicato de magnésia.
\section{Oficéfalo}
\begin{itemize}
\item {Grp. gram.:m.}
\end{itemize}
\begin{itemize}
\item {Proveniência:(Do gr. \textunderscore ophis\textunderscore  + \textunderscore kephale\textunderscore )}
\end{itemize}
Gênero de peixes acantoptòrígios.
\section{Oficlide}
\begin{itemize}
\item {Grp. gram.:m.}
\end{itemize}
\begin{itemize}
\item {Utilização:Mús.}
\end{itemize}
\begin{itemize}
\item {Proveniência:(Do gr. \textunderscore ophis\textunderscore  + \textunderscore kleis\textunderscore )}
\end{itemize}
Instrumento grave, de metal, com chaves, chamado, por corrupção, \textunderscore figle\textunderscore .
\section{Ófico}
\begin{itemize}
\item {Grp. gram.:adj.}
\end{itemize}
\begin{itemize}
\item {Utilização:Des.}
\end{itemize}
\begin{itemize}
\item {Proveniência:(Do gr. \textunderscore ophis\textunderscore )}
\end{itemize}
Dizia-se do medicamento, contra o veneno das serpentes.
\section{Ofidiastro}
\begin{itemize}
\item {Grp. gram.:m.}
\end{itemize}
Gênero de equinodermes.
\section{Ofídico}
\begin{itemize}
\item {Grp. gram.:adj.}
\end{itemize}
\begin{itemize}
\item {Proveniência:(Do gr. \textunderscore ophis\textunderscore )}
\end{itemize}
Relativo a serpente; próprio de serpente.
\section{Ofídeo}
\begin{itemize}
\item {Grp. gram.:adj.}
\end{itemize}
\begin{itemize}
\item {Grp. gram.:M. pl.}
\end{itemize}
\begin{itemize}
\item {Proveniência:(Do gr. \textunderscore ophis\textunderscore  + \textunderscore eidos\textunderscore )}
\end{itemize}
Semelhante a uma serpente.
Ordem de reptis, de epiderme escamosa.
Gênero de peixes ápodes.
\section{Ofidismo}
\begin{itemize}
\item {Grp. gram.:m.}
\end{itemize}
\begin{itemize}
\item {Proveniência:(Do gr. \textunderscore ophis\textunderscore )}
\end{itemize}
Estudo do veneno das serpentes.
Efeitos dêsse veneno.
\section{Ofidómona}
\begin{itemize}
\item {Grp. gram.:f.}
\end{itemize}
Gênero de infusórios.
\section{Ofidossaurios}
\begin{itemize}
\item {Grp. gram.:m. pl.}
\end{itemize}
Ordem de reptis, que compreende os ofídios e os sáurios.
\section{Ofiocéfalo}
\begin{itemize}
\item {Grp. gram.:m.}
\end{itemize}
\begin{itemize}
\item {Proveniência:(Do gr. \textunderscore ophis\textunderscore  + \textunderscore kephale\textunderscore )}
\end{itemize}
Espécie de helmintos.
\section{Ofiócoma}
\begin{itemize}
\item {Grp. gram.:f.}
\end{itemize}
\begin{itemize}
\item {Proveniência:(Do gr. \textunderscore ophis\textunderscore  + \textunderscore kome\textunderscore )}
\end{itemize}
Gênero de equínodermes, estabelecido por Agassiz.
\section{Ofioderma}
\begin{itemize}
\item {Grp. gram.:f.}
\end{itemize}
\begin{itemize}
\item {Proveniência:(Do gr. \textunderscore ophis\textunderscore  + \textunderscore derma\textunderscore )}
\end{itemize}
Gênero de fêtos da Oceânia.
\section{Ofiodonte}
\begin{itemize}
\item {Grp. gram.:m.}
\end{itemize}
\begin{itemize}
\item {Proveniência:(Do gr. \textunderscore ophis\textunderscore  + \textunderscore odous\textunderscore )}
\end{itemize}
Dente fóssil de serpente.
\section{Ofiofagia}
\begin{itemize}
\item {Grp. gram.:f.}
\end{itemize}
Qualidade ou hábito de ofiófago.
\section{Ofiofágico}
\begin{itemize}
\item {Grp. gram.:adj.}
\end{itemize}
Relativo á ofiofagia.
\section{Ofiófago}
\begin{itemize}
\item {Grp. gram.:m.  e  adj.}
\end{itemize}
\begin{itemize}
\item {Proveniência:(Gr. \textunderscore ophiophagos\textunderscore )}
\end{itemize}
O que se sustenta de serpentes.
\section{Ofioftalmo}
\begin{itemize}
\item {Grp. gram.:m.}
\end{itemize}
\begin{itemize}
\item {Proveniência:(Do gr. \textunderscore ophis\textunderscore  + \textunderscore ophthalmos\textunderscore )}
\end{itemize}
Gênero de reptis.
\section{Ofioglossáceos}
\begin{itemize}
\item {Grp. gram.:m. pl.}
\end{itemize}
Tríbo de fêtos, que tem por tipo o ofioglosso.
\section{Ofíoglossita}
\begin{itemize}
\item {Grp. gram.:f.}
\end{itemize}
\begin{itemize}
\item {Proveniência:(Do gr. \textunderscore ophis\textunderscore  + \textunderscore glossa\textunderscore )}
\end{itemize}
Língua de serpente, fossilizada.
\section{Ofioglosso}
\begin{itemize}
\item {Grp. gram.:m.}
\end{itemize}
\begin{itemize}
\item {Proveniência:(Do gr. \textunderscore ophis\textunderscore  + \textunderscore glossa\textunderscore )}
\end{itemize}
Gênero de fêtos.
\section{Ofiografia}
\begin{itemize}
\item {Grp. gram.:f.}
\end{itemize}
\begin{itemize}
\item {Proveniência:(De \textunderscore ofiógrafo\textunderscore )}
\end{itemize}
Descripção das serpentes.
\section{Ofiográfico}
\begin{itemize}
\item {Grp. gram.:adj.}
\end{itemize}
Relativo á ofiografia.
\section{Ofiógrafo}
\begin{itemize}
\item {Grp. gram.:m.}
\end{itemize}
\begin{itemize}
\item {Proveniência:(Do gr. \textunderscore ophis\textunderscore  + \textunderscore graphein\textunderscore )}
\end{itemize}
Aquele que se aplica á ofiografia.
\section{Ofióide}
\begin{itemize}
\item {Grp. gram.:adj.}
\end{itemize}
O mesmo que \textunderscore ofioídeo\textunderscore .
\section{Ofioídeo}
\begin{itemize}
\item {Grp. gram.:adj.}
\end{itemize}
\begin{itemize}
\item {Grp. gram.:M. pl.}
\end{itemize}
\begin{itemize}
\item {Proveniência:(Do gr. \textunderscore ophis\textunderscore  + \textunderscore eidos\textunderscore )}
\end{itemize}
Que tem semelhança com a serpente.
Família de peixes que, pela sua fórma, se assemelham ás serpentes.
\section{Ofiólatra}
\begin{itemize}
\item {Grp. gram.:m.}
\end{itemize}
Adorador de serpentes.
(Cp. \textunderscore ofiolatria\textunderscore )
\section{Ofiolatria}
\begin{itemize}
\item {Grp. gram.:f.}
\end{itemize}
\begin{itemize}
\item {Proveniência:(Do gr. \textunderscore ophis\textunderscore  + \textunderscore latreia\textunderscore )}
\end{itemize}
Adoração das serpentes.
\section{Ofiólita}
\begin{itemize}
\item {Grp. gram.:f.}
\end{itemize}
\begin{itemize}
\item {Utilização:Miner.}
\end{itemize}
\begin{itemize}
\item {Proveniência:(Do gr. \textunderscore ophis\textunderscore  + \textunderscore lithos\textunderscore )}
\end{itemize}
Variedade de rocha composta, cuja base é o talco ou a serpentina.
\section{Ofiolítico}
\begin{itemize}
\item {Grp. gram.:adj.}
\end{itemize}
Diz-se do terreno em que há ofiólita.
\section{Ofiólito}
\begin{itemize}
\item {Grp. gram.:m.}
\end{itemize}
O mesmo ou melhor que \textunderscore ofiólita\textunderscore .
\section{Ofiologia}
\begin{itemize}
\item {Grp. gram.:f.}
\end{itemize}
\begin{itemize}
\item {Proveniência:(Do gr. \textunderscore ophis\textunderscore  + \textunderscore logos\textunderscore )}
\end{itemize}
Tratado á cêrca das serpentes.
\section{Ofiológico}
\begin{itemize}
\item {Grp. gram.:adj.}
\end{itemize}
Relativo á ofiologia.
\section{Ofiologista}
\begin{itemize}
\item {Grp. gram.:m.}
\end{itemize}
Aquele que se ocupa de ofiologia.
\section{Ofiomancia}
\begin{itemize}
\item {Grp. gram.:f.}
\end{itemize}
\begin{itemize}
\item {Proveniência:(Do gr. \textunderscore ophis\textunderscore  + \textunderscore manteia\textunderscore )}
\end{itemize}
Suposta arte de adivinhar, pela observação de serpentes.
\section{Ofiomântico}
\begin{itemize}
\item {Grp. gram.:adj.}
\end{itemize}
Relativo á ofiomancia.
\section{Ofiomórfico}
\begin{itemize}
\item {Grp. gram.:adj.}
\end{itemize}
O mesmo que \textunderscore ofiomorfita\textunderscore .
\section{Ofiomorfita}
\begin{itemize}
\item {Grp. gram.:f.}
\end{itemize}
\begin{itemize}
\item {Proveniência:(De \textunderscore ofiomorfo\textunderscore )}
\end{itemize}
Fóssil, que tem a aparência de serpente.
\section{Ofiomorfo}
\begin{itemize}
\item {Grp. gram.:adj.}
\end{itemize}
\begin{itemize}
\item {Grp. gram.:M. pl.}
\end{itemize}
\begin{itemize}
\item {Proveniência:(Do gr. \textunderscore ophis\textunderscore  + \textunderscore morphe\textunderscore )}
\end{itemize}
Que tem fórma de serpente.
Gênero de insectos coleópteros.
\section{Ofiónice}
\begin{itemize}
\item {Grp. gram.:m.}
\end{itemize}
Gênero de equinodermes.
\section{Ofiope}
\begin{itemize}
\item {Grp. gram.:m.}
\end{itemize}
\begin{itemize}
\item {Proveniência:(Do gr. \textunderscore ophis\textunderscore  + \textunderscore ops\textunderscore )}
\end{itemize}
Gênero de reptis sáurios.
\section{Ofiópogo}
\begin{itemize}
\item {Grp. gram.:m.}
\end{itemize}
Gênero de plantas esmiláceas.
\section{Ofiorriza}
\begin{itemize}
\item {Grp. gram.:f.}
\end{itemize}
\begin{itemize}
\item {Proveniência:(Do gr. \textunderscore ophis\textunderscore  + \textunderscore rhiza\textunderscore )}
\end{itemize}
Gênero de plantas herbáceas indianas, da fam. das rubiáceas.
\section{Ofioscórodo}
\begin{itemize}
\item {Grp. gram.:m.}
\end{itemize}
Gênero de plantas liliáceas.
\section{Ofiossema}
\begin{itemize}
\item {Grp. gram.:f.}
\end{itemize}
\begin{itemize}
\item {Proveniência:(Do gr. \textunderscore ophis\textunderscore  + \textunderscore sema\textunderscore )}
\end{itemize}
Gênero de vermes intestinaes.
\section{Ofiospérmeas}
\begin{itemize}
\item {Grp. gram.:f. pl.}
\end{itemize}
\begin{itemize}
\item {Proveniência:(De \textunderscore ofiospermo\textunderscore )}
\end{itemize}
Família de plantas, cujos gêneros têm sido distribuídos por outras famílias.
\section{Ofiospermo}
\begin{itemize}
\item {Grp. gram.:m.}
\end{itemize}
\begin{itemize}
\item {Proveniência:(Do gr. \textunderscore ophis\textunderscore  + \textunderscore sperma\textunderscore )}
\end{itemize}
Gênero de plantas sapotáceas.
\section{Ofióstoma}
\begin{itemize}
\item {Grp. gram.:f.}
\end{itemize}
\begin{itemize}
\item {Proveniência:(Do gr. \textunderscore ophis\textunderscore  + \textunderscore stoma\textunderscore )}
\end{itemize}
Gênero de vermes intestinaes.
\section{Ofiosuro}
\begin{itemize}
\item {Grp. gram.:m.}
\end{itemize}
\begin{itemize}
\item {Proveniência:(Do gr. \textunderscore ophis\textunderscore  + \textunderscore oura\textunderscore )}
\end{itemize}
Gênero de peixes ápodes.
\section{Ofióxilo}
\begin{itemize}
\item {fónica:csi}
\end{itemize}
\begin{itemize}
\item {Grp. gram.:m.}
\end{itemize}
\begin{itemize}
\item {Proveniência:(Do gr. \textunderscore ophis\textunderscore  + \textunderscore xule\textunderscore )}
\end{itemize}
Gênero de plantas apocíneas.
\section{Ofira}
\begin{itemize}
\item {Grp. gram.:f.}
\end{itemize}
Gênero de insectos dípteros.
\section{Ofissáureos}
\begin{itemize}
\item {Grp. gram.:m. pl.}
\end{itemize}
Família de reptis, que tem por tipo o ofissauro.
\section{Ofissauro}
\begin{itemize}
\item {Grp. gram.:m.}
\end{itemize}
\begin{itemize}
\item {Proveniência:(Do gr. \textunderscore ophis\textunderscore  + \textunderscore sauros\textunderscore )}
\end{itemize}
Gênero de reptis americanos, semelhantes á serpente.
\section{Ofisuro}
\begin{itemize}
\item {Grp. gram.:m.}
\end{itemize}
O mesmo que \textunderscore ofiosuro\textunderscore .
\section{Ofita}
\begin{itemize}
\item {Grp. gram.:f.}
\end{itemize}
\begin{itemize}
\item {Utilização:Miner.}
\end{itemize}
\begin{itemize}
\item {Proveniência:(Lat. \textunderscore ophites\textunderscore )}
\end{itemize}
Designação, dada por geólogos a rochas de diferente composição, especialmente as porfiróides esverdeadas, com cristaes de feldspato.
\section{Ofitas}
\begin{itemize}
\item {Grp. gram.:m. pl.}
\end{itemize}
\begin{itemize}
\item {Proveniência:(Do gr. \textunderscore ophis\textunderscore )}
\end{itemize}
Herejes, que sustentavam que Cristo tomara a figura de serpente para tentar Eva.
\section{Ofite}
\begin{itemize}
\item {Grp. gram.:f.}
\end{itemize}
\begin{itemize}
\item {Proveniência:(Lat. \textunderscore ophites\textunderscore )}
\end{itemize}
Gênero de insectos coleópteros pentâmeros.
O mesmo que \textunderscore ofita\textunderscore .
\section{Ofítico}
\begin{itemize}
\item {Grp. gram.:adj.}
\end{itemize}
Relativo á ofita.
\section{Ofitina}
\begin{itemize}
\item {Grp. gram.:f.}
\end{itemize}
\begin{itemize}
\item {Utilização:Miner.}
\end{itemize}
\begin{itemize}
\item {Proveniência:(De \textunderscore ofita\textunderscore )}
\end{itemize}
Base da ofita, ou pórfiro verde.
\section{Ofito}
\begin{itemize}
\item {Grp. gram.:m.}
\end{itemize}
O mesmo que \textunderscore ofita\textunderscore .
Gênero de serpentes.
\section{Ofitoso}
\begin{itemize}
\item {Grp. gram.:adj.}
\end{itemize}
\begin{itemize}
\item {Utilização:Miner.}
\end{itemize}
\begin{itemize}
\item {Proveniência:(De \textunderscore ofita\textunderscore ])}
\end{itemize}
Que está reunido por um cimento de serpentina.
\section{Ofiúco}
\begin{itemize}
\item {Grp. gram.:m.}
\end{itemize}
\begin{itemize}
\item {Proveniência:(Lat. \textunderscore ophiuchus\textunderscore )}
\end{itemize}
Constelação boreal, chamada também \textunderscore Serpentária\textunderscore  ou \textunderscore Esculápio\textunderscore .
\section{Ofiuríneas}
\begin{itemize}
\item {Grp. gram.:f. pl.}
\end{itemize}
Tríbo de plantas gramíneas, que tem por tipo o ofiúro.
\section{Ofiúro}
\begin{itemize}
\item {Grp. gram.:m.}
\end{itemize}
\begin{itemize}
\item {Proveniência:(Do gr. \textunderscore ophis\textunderscore  + \textunderscore oura\textunderscore )}
\end{itemize}
Gênero de plantas gramíneas, procedentes do Malabar e da Nova-Holanda.
O mesmo que \textunderscore estrela-do-mar\textunderscore .
\section{Ofiúsa}
\begin{itemize}
\item {Grp. gram.:f.}
\end{itemize}
\begin{itemize}
\item {Proveniência:(Lat. \textunderscore ophiusa\textunderscore )}
\end{itemize}
Gênero de insectos lepidópteros nocturnos.
\section{Ofónio}
\begin{itemize}
\item {Grp. gram.:m.}
\end{itemize}
Gênero de insectos coleópteros pentâmeros.
\section{Ófride}
\begin{itemize}
\item {Grp. gram.:f.}
\end{itemize}
O mesmo que \textunderscore ófris\textunderscore .
\section{Ofrídeas}
\begin{itemize}
\item {Grp. gram.:f. pl.}
\end{itemize}
\begin{itemize}
\item {Proveniência:(De \textunderscore ófride\textunderscore )}
\end{itemize}
Tríbo de plantas, da fam. das orquídeas, e cujo tipo é o ófris.
\section{Ofrídios}
\begin{itemize}
\item {Grp. gram.:m. pl.}
\end{itemize}
Gênero de infusórios.
(Cp. \textunderscore ófride\textunderscore )
\section{Ófrio}
\begin{itemize}
\item {Grp. gram.:m.}
\end{itemize}
\begin{itemize}
\item {Proveniência:(Do gr. \textunderscore ophrus\textunderscore )}
\end{itemize}
Ponto craniométrico, que fica entre os dois sobrolhos.
\section{Ofrioglena}
\begin{itemize}
\item {Grp. gram.:f.}
\end{itemize}
Gênero de infusórios, que vivem na água doce.
\section{Ófrion}
\begin{itemize}
\item {Grp. gram.:m.}
\end{itemize}
\begin{itemize}
\item {Proveniência:(Do gr. \textunderscore ophrus\textunderscore )}
\end{itemize}
Ponto craniométrico, que fica entre os dois sobrolhos.
\section{Ofríope}
\begin{itemize}
\item {Grp. gram.:m.}
\end{itemize}
\begin{itemize}
\item {Proveniência:(Do gr. \textunderscore ophrus\textunderscore  + \textunderscore ops\textunderscore )}
\end{itemize}
Gênero de insectos longicórneos.
\section{Ófris}
\begin{itemize}
\item {Grp. gram.:m.}
\end{itemize}
\begin{itemize}
\item {Proveniência:(Do gr. \textunderscore ophrus\textunderscore )}
\end{itemize}
Grupo de plantas, que crescem principalmente no litoral do Mediterrâneo, e cujas fôlhas, esquisitamente coloridas, têm a fórma de mosca.
\section{Oftalgia}
\begin{itemize}
\item {Grp. gram.:f.}
\end{itemize}
\begin{itemize}
\item {Utilização:Med.}
\end{itemize}
Dôr nos olhos, sem inflamação.
(Por \textunderscore ophthalmalgia\textunderscore , do gr. \textunderscore ophthalmos\textunderscore  + \textunderscore algos\textunderscore )
\section{Oftalmalgia}
\begin{itemize}
\item {Grp. gram.:f.}
\end{itemize}
\begin{itemize}
\item {Utilização:Med.}
\end{itemize}
\begin{itemize}
\item {Proveniência:(Do gr. \textunderscore ophthalmos\textunderscore  + \textunderscore algos\textunderscore )}
\end{itemize}
Neuralgia nos olhos.
\section{Oftalmálgico}
\begin{itemize}
\item {Grp. gram.:adj.}
\end{itemize}
Relativo á oftalmalgia.
\section{Oftalmia}
\begin{itemize}
\item {Grp. gram.:f.}
\end{itemize}
\begin{itemize}
\item {Utilização:Med.}
\end{itemize}
\begin{itemize}
\item {Proveniência:(Lat. \textunderscore ophthalmia\textunderscore )}
\end{itemize}
Inflamação do ôlho.
\section{Oftalmiatra}
\begin{itemize}
\item {Grp. gram.:m.}
\end{itemize}
\begin{itemize}
\item {Proveniência:(Do gr. \textunderscore ophthalmos\textunderscore  + \textunderscore iatros\textunderscore )}
\end{itemize}
Médico, que trata especialmente de doenças de olhos.
\section{Oftalmiatria}
\begin{itemize}
\item {Grp. gram.:f.}
\end{itemize}
O mesmo que \textunderscore oftalmologia\textunderscore .
(Cp. \textunderscore ophthalmiatra\textunderscore )
\section{Oftálmico}
\begin{itemize}
\item {Grp. gram.:adj.}
\end{itemize}
\begin{itemize}
\item {Grp. gram.:M.}
\end{itemize}
\begin{itemize}
\item {Proveniência:(Lat. \textunderscore ophthalmicus\textunderscore )}
\end{itemize}
Relativo á oftalmia.
Relativo ao ôlho.
Aplicável contra a oftalmia.
Medicamento contra ela.
\section{Oftálmio}
\begin{itemize}
\item {Grp. gram.:m.}
\end{itemize}
\begin{itemize}
\item {Proveniência:(Do gr. \textunderscore ophthalmos\textunderscore )}
\end{itemize}
Pedra fabulosa que, segundo se dizia, tornava invisível quem a trouxesse.
\section{Oftalmita}
\begin{itemize}
\item {Grp. gram.:f.}
\end{itemize}
\begin{itemize}
\item {Proveniência:(Do gr. \textunderscore ophthalmos\textunderscore )}
\end{itemize}
Nome, que os Gregos davam a certas pedras, (ágatas), formadas de círculos concentricos, imitando olhos.
\section{Oftalmite}
\begin{itemize}
\item {Grp. gram.:f.}
\end{itemize}
\begin{itemize}
\item {Utilização:Med.}
\end{itemize}
\begin{itemize}
\item {Proveniência:(Do gr. \textunderscore ophtalmos\textunderscore )}
\end{itemize}
Fleimão no ôlho.
\section{Oftalmobiótica}
\begin{itemize}
\item {Grp. gram.:f.}
\end{itemize}
\begin{itemize}
\item {Utilização:Med.}
\end{itemize}
\begin{itemize}
\item {Proveniência:(Do gr. \textunderscore ophthalmos\textunderscore  + \textunderscore bios\textunderscore )}
\end{itemize}
Higiene, relativa aos olhos.
\section{Oftalmocele}
\begin{itemize}
\item {Grp. gram.:m.  e  f.}
\end{itemize}
\begin{itemize}
\item {Utilização:Med.}
\end{itemize}
O mesmo que \textunderscore exoftalmo\textunderscore .
\section{Oftalmocopia}
\begin{itemize}
\item {Grp. gram.:f.}
\end{itemize}
\begin{itemize}
\item {Utilização:Med.}
\end{itemize}
\begin{itemize}
\item {Proveniência:(Do gr. \textunderscore ophthalmos\textunderscore  + \textunderscore kopos\textunderscore )}
\end{itemize}
Enfraquecimento da vista.
\section{Oftalmodinia}
\begin{itemize}
\item {Grp. gram.:f.}
\end{itemize}
\begin{itemize}
\item {Utilização:Med.}
\end{itemize}
\begin{itemize}
\item {Proveniência:(Do gr. \textunderscore ophthalmos\textunderscore  + \textunderscore odune\textunderscore )}
\end{itemize}
Dôr reumática nos olhos.
\section{Oftalmografia}
\begin{itemize}
\item {Grp. gram.:f.}
\end{itemize}
\begin{itemize}
\item {Proveniência:(Do gr. \textunderscore ophthalmos\textunderscore  + \textunderscore graphein\textunderscore )}
\end{itemize}
Descripção do ôlho.
\section{Oftalmográfico}
\begin{itemize}
\item {Grp. gram.:adj.}
\end{itemize}
Relativo á oftalmografia.
\section{Oftalmógrafo}
\begin{itemize}
\item {Grp. gram.:m.}
\end{itemize}
Aquele que se ocupa de oftalmografia.
\section{Oftalmólito}
\begin{itemize}
\item {Grp. gram.:m.}
\end{itemize}
\begin{itemize}
\item {Proveniência:(Do gr. \textunderscore ophthalmos\textunderscore  + \textunderscore lithos\textunderscore )}
\end{itemize}
Concreção ocular.
\section{Oftalmologia}
\begin{itemize}
\item {Grp. gram.:f.}
\end{itemize}
\begin{itemize}
\item {Proveniência:(Do gr. \textunderscore ophthalmos\textunderscore  + \textunderscore logos\textunderscore )}
\end{itemize}
Tratamento medicinal do ôlho.
Parte da Medicina, que tem por objecto o estudo dos olhos e das doenças deles.
\section{Oftalmológico}
\begin{itemize}
\item {Grp. gram.:adj.}
\end{itemize}
Relativo á oftalmologia.
\section{Oftalmologista}
\begin{itemize}
\item {Grp. gram.:m.}
\end{itemize}
Médico, que se ocupa de doenças de olhos.
(Cp. \textunderscore oftalmologia\textunderscore )
\section{Operariado}
\begin{itemize}
\item {Grp. gram.:m.}
\end{itemize}
A classe dos operários.
\section{Operário}
\begin{itemize}
\item {Grp. gram.:m.}
\end{itemize}
\begin{itemize}
\item {Utilização:Fig.}
\end{itemize}
\begin{itemize}
\item {Grp. gram.:Adj.}
\end{itemize}
\begin{itemize}
\item {Proveniência:(Lat. \textunderscore operarius\textunderscore )}
\end{itemize}
Artífice.
Jornaleiro.
Trabalhador de fábrica.
Aquelle que coopera na realização de uma ideia ou no bem-estar da sociedade.
Relativo ao trabalho ou aos operários: \textunderscore uma cooperativa operária\textunderscore .
\section{Operativo}
\begin{itemize}
\item {Grp. gram.:adj.}
\end{itemize}
\begin{itemize}
\item {Proveniência:(De \textunderscore operar\textunderscore )}
\end{itemize}
Relativo a obras; que produz effeito.
\section{Operatório}
\begin{itemize}
\item {Grp. gram.:adj.}
\end{itemize}
\begin{itemize}
\item {Proveniência:(Lat. \textunderscore operatorius\textunderscore )}
\end{itemize}
Relativo a operações: \textunderscore instrumentos operatórios\textunderscore .
\section{Operável}
\begin{itemize}
\item {Grp. gram.:adj.}
\end{itemize}
Que se póde operar.
\section{Operculado}
\begin{itemize}
\item {Grp. gram.:adj.}
\end{itemize}
\begin{itemize}
\item {Proveniência:(Lat. \textunderscore operculatus\textunderscore )}
\end{itemize}
Que tem opérculos, ou é fechado por um opérculo.
\section{Opercular}
\begin{itemize}
\item {Grp. gram.:adj.}
\end{itemize}
Que serve de opérculo.
\section{Operculária}
\begin{itemize}
\item {Grp. gram.:f.}
\end{itemize}
\begin{itemize}
\item {Proveniência:(De \textunderscore opercular\textunderscore )}
\end{itemize}
Gênero de plantas rubiáceas.
\section{Operculariáceas}
\begin{itemize}
\item {Grp. gram.:f. pl.}
\end{itemize}
Família de plantas, que tem por typo a operculária.
\section{Operculífero}
\begin{itemize}
\item {Grp. gram.:adj.}
\end{itemize}
\begin{itemize}
\item {Proveniência:(Do lat. \textunderscore operculum\textunderscore  + \textunderscore ferre\textunderscore )}
\end{itemize}
Que tem opérculo.
\section{Operculiforme}
\begin{itemize}
\item {Grp. gram.:adj.}
\end{itemize}
\begin{itemize}
\item {Proveniência:(Do lat. \textunderscore operculum\textunderscore  + \textunderscore forma\textunderscore )}
\end{itemize}
Que tem fórma de opérculo.
\section{Operculina}
\begin{itemize}
\item {Grp. gram.:f.}
\end{itemize}
\begin{itemize}
\item {Proveniência:(De \textunderscore opérculo\textunderscore )}
\end{itemize}
Gênero de animaes rhizópodes.
\section{Operculita}
\begin{itemize}
\item {Grp. gram.:f.}
\end{itemize}
Opérculo fóssil.
\section{Opérculo}
\begin{itemize}
\item {Grp. gram.:m.}
\end{itemize}
\begin{itemize}
\item {Proveniência:(Lat. \textunderscore operculum\textunderscore )}
\end{itemize}
Órgão vegetal, que encobre um orifício natural.
Apparelho ósseo, que protege as guelras de certos peixes.
Substância córnea ou calcária, que tapa, mais ou menos, a abertura das conchas univalves.
Tampa de thuribulo.
\section{Opereta}
\begin{itemize}
\item {fónica:perê}
\end{itemize}
\begin{itemize}
\item {Grp. gram.:f.}
\end{itemize}
\begin{itemize}
\item {Proveniência:(It. \textunderscore operetta\textunderscore )}
\end{itemize}
Pequena ópera, poema lýrico de texto simples e feição popular.
\section{Operista}
\begin{itemize}
\item {Grp. gram.:m.}
\end{itemize}
\begin{itemize}
\item {Utilização:bras}
\end{itemize}
\begin{itemize}
\item {Utilização:Neol.}
\end{itemize}
Compositor de óperas.
\section{Operlanda}
\begin{itemize}
\item {Grp. gram.:f.}
\end{itemize}
\begin{itemize}
\item {Utilização:Ant.}
\end{itemize}
O mesmo que \textunderscore opalanda\textunderscore .
\section{Operosidade}
\begin{itemize}
\item {Grp. gram.:f.}
\end{itemize}
Qualidade de operoso; actividade.
\section{Operoso}
\begin{itemize}
\item {Grp. gram.:adj.}
\end{itemize}
\begin{itemize}
\item {Proveniência:(Lat. \textunderscore operosus\textunderscore )}
\end{itemize}
Que opéra.
Laborioso; productivo.
\section{Operto}
\begin{itemize}
\item {Grp. gram.:m.}
\end{itemize}
\begin{itemize}
\item {Proveniência:(Lat. \textunderscore opertus\textunderscore )}
\end{itemize}
Lugar secreto, onde se faziam os sacrifícios à deusa Cybele.
Âmphora, que se collocava á entrada da sala, onde se reuniam os adoradores daquella deusa.
\section{Opetíola}
\begin{itemize}
\item {Grp. gram.:f.}
\end{itemize}
Gênero de plantas aráceas.
\section{Ophato}
\begin{itemize}
\item {Grp. gram.:m.}
\end{itemize}
Variedade de mármore.
\section{Ophélia}
\begin{itemize}
\item {Grp. gram.:f.}
\end{itemize}
Gênero de plantas gencianáceas.
\section{Ophi...}
\begin{itemize}
\item {Proveniência:(Do gr. \textunderscore ophis\textunderscore )}
\end{itemize}
Elemento, que entra na composição de várias palavras, com o significado de \textunderscore serpente\textunderscore .
\section{Ophia}
\begin{itemize}
\item {Grp. gram.:f.}
\end{itemize}
Gênero de aves da América do Sul.
\section{Ophíase}
\begin{itemize}
\item {Grp. gram.:f.}
\end{itemize}
\begin{itemize}
\item {Proveniência:(Gr. \textunderscore ophiasis\textunderscore )}
\end{itemize}
Espécie de alopecia, em que os cabellos caem por partes.
\section{Ophicálcia}
\begin{itemize}
\item {Grp. gram.:f.}
\end{itemize}
\begin{itemize}
\item {Utilização:Miner.}
\end{itemize}
\begin{itemize}
\item {Proveniência:(De \textunderscore ophi...\textunderscore  + \textunderscore cálcio\textunderscore )}
\end{itemize}
Rocha calcária avermelhada, misturada com silicato de magnésia.
\section{Ophicéfalo}
\begin{itemize}
\item {Grp. gram.:m.}
\end{itemize}
\begin{itemize}
\item {Proveniência:(Do gr. \textunderscore ophis\textunderscore  + \textunderscore kephale\textunderscore )}
\end{itemize}
Gênero de peixes acanthoptòrýgios.
\section{Ophiclide}
\begin{itemize}
\item {Grp. gram.:m.}
\end{itemize}
\begin{itemize}
\item {Utilização:Mús.}
\end{itemize}
\begin{itemize}
\item {Proveniência:(Do gr. \textunderscore ophis\textunderscore  + \textunderscore kleis\textunderscore )}
\end{itemize}
Instrumento grave, de metal, com chaves, chamado, por corrupção, \textunderscore figle\textunderscore .
\section{Óphico}
\begin{itemize}
\item {Grp. gram.:adj.}
\end{itemize}
\begin{itemize}
\item {Utilização:Des.}
\end{itemize}
\begin{itemize}
\item {Proveniência:(Do gr. \textunderscore ophis\textunderscore )}
\end{itemize}
Dizia-se do medicamento, contra o veneno das serpentes.
\section{Ophidiastro}
\begin{itemize}
\item {Grp. gram.:m.}
\end{itemize}
Gênero de echinodermes.
\section{Ophídico}
\begin{itemize}
\item {Grp. gram.:adj.}
\end{itemize}
\begin{itemize}
\item {Proveniência:(Do gr. \textunderscore ophis\textunderscore )}
\end{itemize}
Relativo a serpente; próprio de serpente.
\section{Ophídio}
\begin{itemize}
\item {Grp. gram.:adj.}
\end{itemize}
\begin{itemize}
\item {Grp. gram.:M. pl.}
\end{itemize}
\begin{itemize}
\item {Proveniência:(Do gr. \textunderscore ophis\textunderscore  + \textunderscore eidos\textunderscore )}
\end{itemize}
Semelhante a uma serpente.
Ordem de reptis, de epiderme escamosa.
Gênero de peixes ápodes.
\section{Ophidismo}
\begin{itemize}
\item {Grp. gram.:m.}
\end{itemize}
\begin{itemize}
\item {Proveniência:(Do gr. \textunderscore ophis\textunderscore )}
\end{itemize}
Estudo do veneno das serpentes.
Effeitos dêsse veneno.
\section{Ophidómona}
\begin{itemize}
\item {Grp. gram.:f.}
\end{itemize}
Gênero de infusórios.
\section{Ophidosáurios}
\begin{itemize}
\item {fónica:sau}
\end{itemize}
\begin{itemize}
\item {Grp. gram.:m. pl.}
\end{itemize}
Ordem de reptis, que comprehende os ophídios e os sáurios.
\section{Ophio...}
\begin{itemize}
\item {Proveniência:(Do gr. \textunderscore ophis\textunderscore )}
\end{itemize}
Elemento, que entra na composição de várias palavras, com o significado de \textunderscore serpente\textunderscore .
\section{Ophiocéphalo}
\begin{itemize}
\item {Grp. gram.:m.}
\end{itemize}
\begin{itemize}
\item {Proveniência:(Do gr. \textunderscore ophis\textunderscore  + \textunderscore kephale\textunderscore )}
\end{itemize}
Espécie de helminthos.
\section{Ophiócoma}
\begin{itemize}
\item {Grp. gram.:f.}
\end{itemize}
\begin{itemize}
\item {Proveniência:(Do gr. \textunderscore ophis\textunderscore  + \textunderscore kome\textunderscore )}
\end{itemize}
Gênero de echinodermes, estabelecido por Agassiz.
\section{Ophioderma}
\begin{itemize}
\item {Grp. gram.:f.}
\end{itemize}
\begin{itemize}
\item {Proveniência:(Do gr. \textunderscore ophis\textunderscore  + \textunderscore derma\textunderscore )}
\end{itemize}
Gênero de fêtos da Oceânia.
\section{Ophiodonte}
\begin{itemize}
\item {Grp. gram.:m.}
\end{itemize}
\begin{itemize}
\item {Proveniência:(Do gr. \textunderscore ophis\textunderscore  + \textunderscore odous\textunderscore )}
\end{itemize}
Dente fóssil de serpente.
\section{Ophioglossáceos}
\begin{itemize}
\item {Grp. gram.:m. pl.}
\end{itemize}
Tríbo de fêtos, que tem por typo o ophioglosso.
\section{Ophioglossita}
\begin{itemize}
\item {Grp. gram.:f.}
\end{itemize}
\begin{itemize}
\item {Proveniência:(Do gr. \textunderscore ophis\textunderscore  + \textunderscore glossa\textunderscore )}
\end{itemize}
Língua de serpente, fossilizada.
\section{Ophioglosso}
\begin{itemize}
\item {Grp. gram.:m.}
\end{itemize}
\begin{itemize}
\item {Proveniência:(Do gr. \textunderscore ophis\textunderscore  + \textunderscore glossa\textunderscore )}
\end{itemize}
Gênero de fêtos.
\section{Ophiographia}
\begin{itemize}
\item {Grp. gram.:f.}
\end{itemize}
\begin{itemize}
\item {Proveniência:(De \textunderscore ophiógrapho\textunderscore )}
\end{itemize}
Descripção das serpentes.
\section{Ophiográphico}
\begin{itemize}
\item {Grp. gram.:adj.}
\end{itemize}
Relativo á ophiographia.
\section{Ophiógrapho}
\begin{itemize}
\item {Grp. gram.:m.}
\end{itemize}
\begin{itemize}
\item {Proveniência:(Do gr. \textunderscore ophis\textunderscore  + \textunderscore graphein\textunderscore )}
\end{itemize}
Aquelle que se applica á ophiographia.
\section{Ophióide}
\begin{itemize}
\item {Grp. gram.:adj.}
\end{itemize}
O mesmo que \textunderscore ophioídeo\textunderscore .
\section{Ophioídeo}
\begin{itemize}
\item {Grp. gram.:adj.}
\end{itemize}
\begin{itemize}
\item {Grp. gram.:M. pl.}
\end{itemize}
\begin{itemize}
\item {Proveniência:(Do gr. \textunderscore ophis\textunderscore  + \textunderscore eidos\textunderscore )}
\end{itemize}
Que tem semelhança com a serpente.
Família de peixes que, pela sua fórma, se assemelham ás serpentes.
\section{Ophiólatra}
\begin{itemize}
\item {Grp. gram.:m.}
\end{itemize}
Adorador de serpentes.
(Cp. \textunderscore ophiolatria\textunderscore )
\section{Ophiolatria}
\begin{itemize}
\item {Grp. gram.:f.}
\end{itemize}
\begin{itemize}
\item {Proveniência:(Do gr. \textunderscore ophis\textunderscore  + \textunderscore latreia\textunderscore )}
\end{itemize}
Adoração das serpentes.
\section{Ophiólitha}
\begin{itemize}
\item {Grp. gram.:f.}
\end{itemize}
\begin{itemize}
\item {Utilização:Miner.}
\end{itemize}
\begin{itemize}
\item {Proveniência:(Do gr. \textunderscore ophis\textunderscore  + \textunderscore lithos\textunderscore )}
\end{itemize}
Variedade de rocha composta, cuja base é o talco ou a serpentina.
\section{Ophiolíthico}
\begin{itemize}
\item {Grp. gram.:adj.}
\end{itemize}
Diz-se do terreno em que há ophiólitha.
\section{Ophiólitho}
\begin{itemize}
\item {Grp. gram.:m.}
\end{itemize}
O mesmo ou melhor que \textunderscore ophiólitha\textunderscore .
\section{Ophiologia}
\begin{itemize}
\item {Grp. gram.:f.}
\end{itemize}
\begin{itemize}
\item {Proveniência:(Do gr. \textunderscore ophis\textunderscore  + \textunderscore logos\textunderscore )}
\end{itemize}
Tratado á cêrca das serpentes.
\section{Ophiológico}
\begin{itemize}
\item {Grp. gram.:adj.}
\end{itemize}
Relativo á ophiologia.
\section{Ophiologista}
\begin{itemize}
\item {Grp. gram.:m.}
\end{itemize}
Aquelle que se occupa de ophiologia.
\section{Ophiomancia}
\begin{itemize}
\item {Grp. gram.:f.}
\end{itemize}
\begin{itemize}
\item {Proveniência:(Do gr. \textunderscore ophis\textunderscore  + \textunderscore manteia\textunderscore )}
\end{itemize}
Supposta arte de adivinhar, pela observação de serpentes.
\section{Ophiomântico}
\begin{itemize}
\item {Grp. gram.:adj.}
\end{itemize}
Relativo á ophiomancia.
\section{Ophiomórphico}
\begin{itemize}
\item {Grp. gram.:adj.}
\end{itemize}
O mesmo que \textunderscore ophiomorphita\textunderscore .
\section{Ophiomorphita}
\begin{itemize}
\item {Grp. gram.:f.}
\end{itemize}
\begin{itemize}
\item {Proveniência:(De \textunderscore ophiomorpho\textunderscore )}
\end{itemize}
Fóssil, que tem a apparência de serpente.
\section{Ophiomorpho}
\begin{itemize}
\item {Grp. gram.:adj.}
\end{itemize}
\begin{itemize}
\item {Grp. gram.:M. pl.}
\end{itemize}
\begin{itemize}
\item {Proveniência:(Do gr. \textunderscore ophis\textunderscore  + \textunderscore morphe\textunderscore )}
\end{itemize}
Que tem fórma de serpente.
Gênero de insectos coleópteros.
\section{Ophiónyce}
\begin{itemize}
\item {Grp. gram.:m.}
\end{itemize}
Gênero de echinodermes.
\section{Ophiope}
\begin{itemize}
\item {Grp. gram.:m.}
\end{itemize}
\begin{itemize}
\item {Proveniência:(Do gr. \textunderscore ophis\textunderscore  + \textunderscore ops\textunderscore )}
\end{itemize}
Gênero de reptis sáurios.
\section{Ophiophagia}
\begin{itemize}
\item {Grp. gram.:f.}
\end{itemize}
Qualidade ou hábito de ophióphago.
\section{Ophiophágico}
\begin{itemize}
\item {Grp. gram.:adj.}
\end{itemize}
Relativo á ophiophagia.
\section{Ophióphago}
\begin{itemize}
\item {Grp. gram.:m.  e  adj.}
\end{itemize}
\begin{itemize}
\item {Proveniência:(Gr. \textunderscore ophiophagos\textunderscore )}
\end{itemize}
O que se sustenta de serpentes.
\section{Ophiophthalmo}
\begin{itemize}
\item {Grp. gram.:m.}
\end{itemize}
\begin{itemize}
\item {Proveniência:(Do gr. \textunderscore ophis\textunderscore  + \textunderscore ophthalmos\textunderscore )}
\end{itemize}
Gênero de reptis.
\section{Ophiópogo}
\begin{itemize}
\item {Grp. gram.:m.}
\end{itemize}
Gênero de plantas esmiláceas.
\section{Ophiorrhiza}
\begin{itemize}
\item {Grp. gram.:f.}
\end{itemize}
\begin{itemize}
\item {Proveniência:(Do gr. \textunderscore ophis\textunderscore  + \textunderscore rhiza\textunderscore )}
\end{itemize}
Gênero de plantas herbáceas indianas, da fam. das rubiáceas.
\section{Ophioscórodo}
\begin{itemize}
\item {Grp. gram.:m.}
\end{itemize}
Gênero de plantas liliáceas.
\section{Ophiosema}
\begin{itemize}
\item {fónica:sê}
\end{itemize}
\begin{itemize}
\item {Grp. gram.:f.}
\end{itemize}
\begin{itemize}
\item {Proveniência:(Do gr. \textunderscore ophis\textunderscore  + \textunderscore sema\textunderscore )}
\end{itemize}
Gênero de vermes intestinaes.
\section{Ophiospérmeas}
\begin{itemize}
\item {Grp. gram.:f. pl.}
\end{itemize}
\begin{itemize}
\item {Proveniência:(De \textunderscore ophiospermo\textunderscore )}
\end{itemize}
Família de plantas, cujos gêneros têm sido distribuídos por outras famílias.
\section{Ophiospermo}
\begin{itemize}
\item {Grp. gram.:m.}
\end{itemize}
\begin{itemize}
\item {Proveniência:(Do gr. \textunderscore ophis\textunderscore  + \textunderscore sperma\textunderscore )}
\end{itemize}
Gênero de plantas sapotáceas.
\section{Ophióstoma}
\begin{itemize}
\item {Grp. gram.:f.}
\end{itemize}
\begin{itemize}
\item {Proveniência:(Do gr. \textunderscore ophis\textunderscore  + \textunderscore stoma\textunderscore )}
\end{itemize}
Gênero de vermes intestinaes.
\section{Ophiosuro}
\begin{itemize}
\item {Grp. gram.:m.}
\end{itemize}
\begin{itemize}
\item {Proveniência:(Do gr. \textunderscore ophis\textunderscore  + \textunderscore oura\textunderscore )}
\end{itemize}
Gênero de peixes ápodes.
\section{Ophióxylo}
\begin{itemize}
\item {Grp. gram.:m.}
\end{itemize}
\begin{itemize}
\item {Proveniência:(Do gr. \textunderscore ophis\textunderscore  + \textunderscore xule\textunderscore )}
\end{itemize}
Gênero de plantas apocýneas.
\section{Ophisáureos}
\begin{itemize}
\item {fónica:sau}
\end{itemize}
\begin{itemize}
\item {Grp. gram.:m. pl.}
\end{itemize}
Família de reptis, que tem por typo o ophisauro.
\section{Ophisauro}
\begin{itemize}
\item {fónica:sau}
\end{itemize}
\begin{itemize}
\item {Grp. gram.:m.}
\end{itemize}
\begin{itemize}
\item {Proveniência:(Do gr. \textunderscore ophis\textunderscore  + \textunderscore sauros\textunderscore )}
\end{itemize}
Gênero de reptis americanos, semelhantes á serpente.
\section{Ophisuro}
\begin{itemize}
\item {Grp. gram.:m.}
\end{itemize}
O mesmo que \textunderscore ophiosuro\textunderscore .
\section{Ophita}
\begin{itemize}
\item {Grp. gram.:f.}
\end{itemize}
\begin{itemize}
\item {Utilização:Miner.}
\end{itemize}
\begin{itemize}
\item {Proveniência:(Lat. \textunderscore ophites\textunderscore )}
\end{itemize}
Designação, dada por geólogos a rochas de differente composição, especialmente as porphyróides esverdeadas, com crystaes de feldspato.
\section{Ophitas}
\begin{itemize}
\item {Grp. gram.:m. pl.}
\end{itemize}
\begin{itemize}
\item {Proveniência:(Do gr. \textunderscore ophis\textunderscore )}
\end{itemize}
Herejes, que sustentavam que Christo tomara a figura de serpente para tentar Eva.
\section{Ophite}
\begin{itemize}
\item {Grp. gram.:f.}
\end{itemize}
\begin{itemize}
\item {Proveniência:(Lat. \textunderscore ophites\textunderscore )}
\end{itemize}
Gênero de insectos coleópteros pentâmeros.
O mesmo que \textunderscore ophita\textunderscore .
\section{Ophítico}
\begin{itemize}
\item {Grp. gram.:adj.}
\end{itemize}
Relativo á ophita.
\section{Ophitina}
\begin{itemize}
\item {Grp. gram.:f.}
\end{itemize}
\begin{itemize}
\item {Utilização:Miner.}
\end{itemize}
\begin{itemize}
\item {Proveniência:(De \textunderscore ophita\textunderscore )}
\end{itemize}
Base da ophita, ou pórphyro verde.
\section{Ophito}
\begin{itemize}
\item {Grp. gram.:m.}
\end{itemize}
O mesmo que \textunderscore ophita\textunderscore .
Gênero de serpentes.
\section{Ophitoso}
\begin{itemize}
\item {Grp. gram.:adj.}
\end{itemize}
\begin{itemize}
\item {Utilização:Miner.}
\end{itemize}
\begin{itemize}
\item {Proveniência:(De \textunderscore ophita\textunderscore )}
\end{itemize}
Que está reunido por um cimento de serpentina.
\section{Ophiúcho}
\begin{itemize}
\item {fónica:co}
\end{itemize}
\begin{itemize}
\item {Grp. gram.:m.}
\end{itemize}
\begin{itemize}
\item {Proveniência:(Lat. \textunderscore ophiuchus\textunderscore )}
\end{itemize}
Constellação boreal, chamada também \textunderscore Serpentária\textunderscore  ou \textunderscore Esculápio\textunderscore .
\section{Ophiuríneas}
\begin{itemize}
\item {Grp. gram.:f. pl.}
\end{itemize}
Tríbo de plantas gramíneas, que tem por typo o ophiúro.
\section{Ophiúro}
\begin{itemize}
\item {Grp. gram.:m.}
\end{itemize}
\begin{itemize}
\item {Proveniência:(Do gr. \textunderscore ophis\textunderscore  + \textunderscore oura\textunderscore )}
\end{itemize}
Gênero de plantas gramíneas, procedentes do Malabar e da Nova-Hollanda.
O mesmo que \textunderscore estrella-do-mar\textunderscore .
\section{Ophiúsa}
\begin{itemize}
\item {Grp. gram.:f.}
\end{itemize}
\begin{itemize}
\item {Proveniência:(Lat. \textunderscore ophiusa\textunderscore )}
\end{itemize}
Gênero de insectos lepidópteros nocturnos.
\section{Ophónio}
\begin{itemize}
\item {Grp. gram.:m.}
\end{itemize}
Gênero de insectos coleópteros pentâmeros.
\section{Óphryde}
\begin{itemize}
\item {Grp. gram.:f.}
\end{itemize}
O mesmo que \textunderscore óphrys\textunderscore .
\section{Ophrýdeas}
\begin{itemize}
\item {Grp. gram.:f. pl.}
\end{itemize}
\begin{itemize}
\item {Proveniência:(De \textunderscore óphryde\textunderscore )}
\end{itemize}
Tríbo de plantas, da fam. das orchídeas, e cujo typo é o óphrys.
\section{Ophrýdios}
\begin{itemize}
\item {Grp. gram.:m. pl.}
\end{itemize}
Gênero de infusórios.
(Cp. \textunderscore óphryde\textunderscore )
\section{Ophryoglena}
\begin{itemize}
\item {Grp. gram.:f.}
\end{itemize}
Gênero de infusórios, que vivem na água doce.
\section{Óphryon}
\begin{itemize}
\item {Grp. gram.:m.}
\end{itemize}
\begin{itemize}
\item {Proveniência:(Do gr. \textunderscore ophrus\textunderscore )}
\end{itemize}
Ponto craniométrico, que fica entre os dois sobrolhos.
\section{Ophrýope}
\begin{itemize}
\item {Grp. gram.:m.}
\end{itemize}
\begin{itemize}
\item {Proveniência:(Do gr. \textunderscore ophrus\textunderscore  + \textunderscore ops\textunderscore )}
\end{itemize}
Gênero de insectos longicórneos.
\section{Óphrys}
\begin{itemize}
\item {Grp. gram.:m.}
\end{itemize}
\begin{itemize}
\item {Proveniência:(Do gr. \textunderscore ophrus\textunderscore )}
\end{itemize}
Grupo de plantas, que crescem principalmente no litoral do Mediterrâneo, e cujas fôlhas, esquisitamente coloridas, têm a fórma de mosca.
\section{Ophthalgia}
\begin{itemize}
\item {Grp. gram.:f.}
\end{itemize}
\begin{itemize}
\item {Utilização:Med.}
\end{itemize}
Dôr nos olhos, sem inflammação.
(Por \textunderscore ophthalmalgia\textunderscore , do gr. \textunderscore ophthalmos\textunderscore  + \textunderscore algos\textunderscore )
\section{Ophthalmalgia}
\begin{itemize}
\item {Grp. gram.:f.}
\end{itemize}
\begin{itemize}
\item {Utilização:Med.}
\end{itemize}
\begin{itemize}
\item {Proveniência:(Do gr. \textunderscore ophthalmos\textunderscore  + \textunderscore algos\textunderscore )}
\end{itemize}
Neuralgia nos olhos.
\section{Ophthalmálgico}
\begin{itemize}
\item {Grp. gram.:adj.}
\end{itemize}
Relativo á ophthalmalgia.
\section{Ophthalmia}
\begin{itemize}
\item {Grp. gram.:f.}
\end{itemize}
\begin{itemize}
\item {Utilização:Med.}
\end{itemize}
\begin{itemize}
\item {Proveniência:(Lat. \textunderscore ophthalmia\textunderscore )}
\end{itemize}
Inflammação do ôlho.
\section{Ophthalmiatra}
\begin{itemize}
\item {Grp. gram.:m.}
\end{itemize}
\begin{itemize}
\item {Proveniência:(Do gr. \textunderscore ophthalmos\textunderscore  + \textunderscore iatros\textunderscore )}
\end{itemize}
Médico, que trata especialmente de doenças de olhos.
\section{Ophthalmiatria}
\begin{itemize}
\item {Grp. gram.:f.}
\end{itemize}
O mesmo que \textunderscore ophthalmologia\textunderscore .
(Cp. \textunderscore ophthalmiatra\textunderscore )
\section{Ophthálmico}
\begin{itemize}
\item {Grp. gram.:adj.}
\end{itemize}
\begin{itemize}
\item {Grp. gram.:M.}
\end{itemize}
\begin{itemize}
\item {Proveniência:(Lat. \textunderscore ophthalmicus\textunderscore )}
\end{itemize}
Relativo á ophthalmia.
Relativo ao ôlho.
Applicável contra a ophthalmia.
Medicamento contra ella.
\section{Ophthálmio}
\begin{itemize}
\item {Grp. gram.:m.}
\end{itemize}
\begin{itemize}
\item {Proveniência:(Do gr. \textunderscore ophthalmos\textunderscore )}
\end{itemize}
Pedra fabulosa que, segundo se dizia, tornava invisível quem a trouxesse.
\section{Ophthalmita}
\begin{itemize}
\item {Grp. gram.:f.}
\end{itemize}
\begin{itemize}
\item {Proveniência:(Do gr. \textunderscore ophthalmos\textunderscore )}
\end{itemize}
Nome, que os Gregos davam a certas pedras, (ágatas), formadas de círculos concentricos, imitando olhos.
\section{Ophtalmite}
\begin{itemize}
\item {Grp. gram.:f.}
\end{itemize}
\begin{itemize}
\item {Utilização:Med.}
\end{itemize}
\begin{itemize}
\item {Proveniência:(Do gr. \textunderscore ophtalmos\textunderscore )}
\end{itemize}
Fleimão no ôlho.
\section{Ophthalmobiótica}
\begin{itemize}
\item {Grp. gram.:f.}
\end{itemize}
\begin{itemize}
\item {Utilização:Med.}
\end{itemize}
\begin{itemize}
\item {Proveniência:(Do gr. \textunderscore ophthalmos\textunderscore  + \textunderscore bios\textunderscore )}
\end{itemize}
Hygiene, relativa aos olhos.
\section{Ophthalmocele}
\begin{itemize}
\item {Grp. gram.:m.  e  f.}
\end{itemize}
\begin{itemize}
\item {Utilização:Med.}
\end{itemize}
O mesmo que \textunderscore exophthalmo\textunderscore .
\section{Ophthalmocopia}
\begin{itemize}
\item {Grp. gram.:f.}
\end{itemize}
\begin{itemize}
\item {Utilização:Med.}
\end{itemize}
\begin{itemize}
\item {Proveniência:(Do gr. \textunderscore ophthalmos\textunderscore  + \textunderscore kopos\textunderscore )}
\end{itemize}
Enfraquecimento da vista.
\section{Ophthalmodynia}
\begin{itemize}
\item {Grp. gram.:f.}
\end{itemize}
\begin{itemize}
\item {Utilização:Med.}
\end{itemize}
\begin{itemize}
\item {Proveniência:(Do gr. \textunderscore ophthalmos\textunderscore  + \textunderscore odune\textunderscore )}
\end{itemize}
Dôr rheumática nos olhos.
\section{Ophthalmographia}
\begin{itemize}
\item {Grp. gram.:f.}
\end{itemize}
\begin{itemize}
\item {Proveniência:(Do gr. \textunderscore ophthalmos\textunderscore  + \textunderscore graphein\textunderscore )}
\end{itemize}
Descripção do ôlho.
\section{Ophthalmográphico}
\begin{itemize}
\item {Grp. gram.:adj.}
\end{itemize}
Relativo á ophthalmographia.
\section{Ophthalmógrapho}
\begin{itemize}
\item {Grp. gram.:m.}
\end{itemize}
Aquelle que se occupa de ophthalmographia.
\section{Ophthalmólitho}
\begin{itemize}
\item {Grp. gram.:m.}
\end{itemize}
\begin{itemize}
\item {Proveniência:(Do gr. \textunderscore ophthalmos\textunderscore  + \textunderscore lithos\textunderscore )}
\end{itemize}
Concreção ocular.
\section{Ophthalmologia}
\begin{itemize}
\item {Grp. gram.:f.}
\end{itemize}
\begin{itemize}
\item {Proveniência:(Do gr. \textunderscore ophthalmos\textunderscore  + \textunderscore logos\textunderscore )}
\end{itemize}
Tratamento medicinal do ôlho.
Parte da Medicina, que tem por objecto o estudo dos olhos e das doenças delles.
\section{Ophthalmológico}
\begin{itemize}
\item {Grp. gram.:adj.}
\end{itemize}
Relativo á ophthalmologia.
\section{Ophthalmologista}
\begin{itemize}
\item {Grp. gram.:m.}
\end{itemize}
Médico, que se occupa de doenças de olhos.
(Cp. \textunderscore ophthalmologia\textunderscore )
\section{Oftalmólogo}
\begin{itemize}
\item {Grp. gram.:m.}
\end{itemize}
(V.oftalmologista)
\section{Oftalmomalacia}
\begin{itemize}
\item {Grp. gram.:f.}
\end{itemize}
\begin{itemize}
\item {Utilização:Med.}
\end{itemize}
\begin{itemize}
\item {Proveniência:(Do gr. \textunderscore ophthalmos\textunderscore  + \textunderscore malakos\textunderscore )}
\end{itemize}
Amolecimento mórbido do ôlho.
\section{Oftalmometria}
\begin{itemize}
\item {Grp. gram.:f.}
\end{itemize}
Conhecimento e uso do oftalmómetro.
\section{Oftalmómetro}
\begin{itemize}
\item {Grp. gram.:m.}
\end{itemize}
\begin{itemize}
\item {Proveniência:(Do gr. \textunderscore ophthalmos\textunderscore  + \textunderscore metron\textunderscore )}
\end{itemize}
Instrumento para medir as curvaturas da superfície refringente do ôlho.
\section{Oftalmoplastia}
\begin{itemize}
\item {Grp. gram.:f.}
\end{itemize}
\begin{itemize}
\item {Utilização:Cir.}
\end{itemize}
\begin{itemize}
\item {Proveniência:(Do gr. \textunderscore ophtalmos\textunderscore  + \textunderscore plassein\textunderscore )}
\end{itemize}
Prótese ocular.
\section{Oftalmoplegia}
\begin{itemize}
\item {Grp. gram.:f.}
\end{itemize}
\begin{itemize}
\item {Utilização:Med.}
\end{itemize}
\begin{itemize}
\item {Proveniência:(Do gr. \textunderscore ophthalmos\textunderscore  + \textunderscore plessein\textunderscore )}
\end{itemize}
Paralisia dos músculos do ôlho.
\section{Oftalmoplégico}
\begin{itemize}
\item {Grp. gram.:adj.}
\end{itemize}
Relativo á oftalmoplegia.
\section{Oftalmoptose}
\begin{itemize}
\item {Grp. gram.:f.}
\end{itemize}
\begin{itemize}
\item {Utilização:Med.}
\end{itemize}
\begin{itemize}
\item {Proveniência:(Do gr. \textunderscore ophthalmos\textunderscore  + \textunderscore ptosis\textunderscore )}
\end{itemize}
Saída do ôlho para fóra da órbita.
\section{Oftalmorragia}
\begin{itemize}
\item {Grp. gram.:f.}
\end{itemize}
\begin{itemize}
\item {Utilização:Med.}
\end{itemize}
\begin{itemize}
\item {Proveniência:(Do gr. \textunderscore ophthalmos\textunderscore  + \textunderscore rhagein\textunderscore )}
\end{itemize}
Hemorragia na conjuntiva ocular.
\section{Oftalmoscopia}
\begin{itemize}
\item {Grp. gram.:f.}
\end{itemize}
Arte de empregar o oftalmoscópio.
(Cp. \textunderscore oftalmoscópio\textunderscore )
\section{Oftalmoscópio}
\begin{itemize}
\item {Grp. gram.:m.}
\end{itemize}
\begin{itemize}
\item {Proveniência:(Do gr. \textunderscore ophthalmos\textunderscore  + \textunderscore skopein\textunderscore )}
\end{itemize}
Instrumento, para examinar a parte interior do ôlho.
\section{Oftalmóstato}
\begin{itemize}
\item {Grp. gram.:m.}
\end{itemize}
\begin{itemize}
\item {Proveniência:(Do gr. \textunderscore ophthalmos\textunderscore  + \textunderscore statos\textunderscore )}
\end{itemize}
Instrumento, para conservar abertas as pálpebras, em certas operações sobre o ôlho.
O mesmo que \textunderscore blefaróstato\textunderscore .
\section{Oftalmoteca}
\begin{itemize}
\item {Grp. gram.:f.}
\end{itemize}
\begin{itemize}
\item {Utilização:Zool.}
\end{itemize}
\begin{itemize}
\item {Proveniência:(Do gr. \textunderscore ophthalmos\textunderscore  + \textunderscore theke\textunderscore )}
\end{itemize}
Parte do corpo da crisálida, que cobre os olhos do insecto.
\section{Oftalmoterapêutica}
\begin{itemize}
\item {Grp. gram.:f.}
\end{itemize}
\begin{itemize}
\item {Proveniência:(Do gr. \textunderscore ophthalmos\textunderscore  + \textunderscore therapeia\textunderscore )}
\end{itemize}
Terapêutica das doenças de olhos.
\section{Oftalmoterapía}
\begin{itemize}
\item {Grp. gram.:f.}
\end{itemize}
\begin{itemize}
\item {Proveniência:(Do gr. \textunderscore ophthalmos\textunderscore  + \textunderscore therapeia\textunderscore )}
\end{itemize}
Terapêutica das doenças de olhos.
\section{Oftalmoterápico}
\begin{itemize}
\item {Grp. gram.:adj.}
\end{itemize}
Relativo á oftalmoterapía.
\section{Oftalmotomia}
\begin{itemize}
\item {Grp. gram.:f.}
\end{itemize}
\begin{itemize}
\item {Utilização:Cir.}
\end{itemize}
\begin{itemize}
\item {Proveniência:(Do gr. \textunderscore ophthalmos\textunderscore  + \textunderscore tome\textunderscore )}
\end{itemize}
Extirpação do ôlho.
Parte da Anatomia, que tem por objecto a dissecção do ôlho.
\section{Oftalmotómico}
\begin{itemize}
\item {Grp. gram.:adj.}
\end{itemize}
Relativo á ofthalmotomia.
\section{Oftalmoxise}
\begin{itemize}
\item {fónica:csi}
\end{itemize}
\begin{itemize}
\item {Grp. gram.:f.}
\end{itemize}
\begin{itemize}
\item {Utilização:Med.}
\end{itemize}
\begin{itemize}
\item {Utilização:Ant.}
\end{itemize}
\begin{itemize}
\item {Proveniência:(Do gr. \textunderscore ophthalmos\textunderscore  + \textunderscore xusis\textunderscore )}
\end{itemize}
Escarificação, que se praticava na conjuntiva do ôlho, em casos de oftalmia.
\section{Oftalmoxistro}
\begin{itemize}
\item {fónica:csis}
\end{itemize}
\begin{itemize}
\item {Grp. gram.:m.}
\end{itemize}
\begin{itemize}
\item {Utilização:Med.}
\end{itemize}
\begin{itemize}
\item {Proveniência:(Do gr. \textunderscore ophthalmos\textunderscore  + \textunderscore xustron\textunderscore )}
\end{itemize}
Instrumento, espécie de pincel, para limpar a superficie do ôlho.
\section{Ophthalmólogo}
\begin{itemize}
\item {Grp. gram.:m.}
\end{itemize}
(V.ophthalmologista)
\section{Ophthalmomalacia}
\begin{itemize}
\item {Grp. gram.:f.}
\end{itemize}
\begin{itemize}
\item {Utilização:Med.}
\end{itemize}
\begin{itemize}
\item {Proveniência:(Do gr. \textunderscore ophthalmos\textunderscore  + \textunderscore malakos\textunderscore )}
\end{itemize}
Amollecimento mórbido do ôlho.
\section{Ophthalmometria}
\begin{itemize}
\item {Grp. gram.:f.}
\end{itemize}
Conhecimento e uso do ophthalmómetro.
\section{Ophthalmómetro}
\begin{itemize}
\item {Grp. gram.:m.}
\end{itemize}
\begin{itemize}
\item {Proveniência:(Do gr. \textunderscore ophthalmos\textunderscore  + \textunderscore metron\textunderscore )}
\end{itemize}
Instrumento para medir as curvaturas da superfície refringente do ôlho.
\section{Ophthalmoloplastia}
\begin{itemize}
\item {Grp. gram.:f.}
\end{itemize}
\begin{itemize}
\item {Utilização:Cir.}
\end{itemize}
\begin{itemize}
\item {Proveniência:(Do gr. \textunderscore ophtalmos\textunderscore  + \textunderscore plassein\textunderscore )}
\end{itemize}
Próthese ocular.
\section{Ophtalmoplegia}
\begin{itemize}
\item {Grp. gram.:f.}
\end{itemize}
\begin{itemize}
\item {Utilização:Med.}
\end{itemize}
\begin{itemize}
\item {Proveniência:(Do gr. \textunderscore ophthalmos\textunderscore  + \textunderscore plessein\textunderscore )}
\end{itemize}
Paralysia dos músculos do ôlho.
\section{Ophthalmoplégico}
\begin{itemize}
\item {Grp. gram.:adj.}
\end{itemize}
Relativo á ophthalmoplegia.
\section{Ophthalmoptose}
\begin{itemize}
\item {Grp. gram.:f.}
\end{itemize}
\begin{itemize}
\item {Utilização:Med.}
\end{itemize}
\begin{itemize}
\item {Proveniência:(Do gr. \textunderscore ophthalmos\textunderscore  + \textunderscore ptosis\textunderscore )}
\end{itemize}
Saída do ôlho para fóra da órbita.
\section{Ophthalmorrhagia}
\begin{itemize}
\item {Grp. gram.:f.}
\end{itemize}
\begin{itemize}
\item {Utilização:Med.}
\end{itemize}
\begin{itemize}
\item {Proveniência:(Do gr. \textunderscore ophthalmos\textunderscore  + \textunderscore rhagein\textunderscore )}
\end{itemize}
Hemorrhagia na conjuntiva ocular.
\section{Ophthalmoscopia}
\begin{itemize}
\item {Grp. gram.:f.}
\end{itemize}
Arte de empregar o ophthalmoscópio.
(Cp. \textunderscore ophthalmoscópio\textunderscore )
\section{Ophthalmoscópio}
\begin{itemize}
\item {Grp. gram.:m.}
\end{itemize}
\begin{itemize}
\item {Proveniência:(Do gr. \textunderscore ophthalmos\textunderscore  + \textunderscore skopein\textunderscore )}
\end{itemize}
Instrumento, para examinar a parte interior do ôlho.
\section{Ophthalmóstato}
\begin{itemize}
\item {Grp. gram.:m.}
\end{itemize}
\begin{itemize}
\item {Proveniência:(Do gr. \textunderscore ophthalmos\textunderscore  + \textunderscore statos\textunderscore )}
\end{itemize}
Instrumento, para conservar abertas as pálpebras, em certas operações sobre o ôlho.
O mesmo que \textunderscore blepharóstato\textunderscore .
\section{Ophthalmotheca}
\begin{itemize}
\item {Grp. gram.:f.}
\end{itemize}
\begin{itemize}
\item {Utilização:Zool.}
\end{itemize}
\begin{itemize}
\item {Proveniência:(Do gr. \textunderscore ophthalmos\textunderscore  + \textunderscore theke\textunderscore )}
\end{itemize}
Parte do corpo da chrysállida, que cobre os olhos do insecto.
\section{Ophthalmotherapêutica}
\begin{itemize}
\item {Grp. gram.:f.}
\end{itemize}
\begin{itemize}
\item {Proveniência:(Do gr. \textunderscore ophthalmos\textunderscore  + \textunderscore therapeia\textunderscore )}
\end{itemize}
Therapêutica das doenças de olhos.
\section{Ophthalmotherapía}
\begin{itemize}
\item {Grp. gram.:f.}
\end{itemize}
\begin{itemize}
\item {Proveniência:(Do gr. \textunderscore ophthalmos\textunderscore  + \textunderscore therapeia\textunderscore )}
\end{itemize}
Therapêutica das doenças de olhos.
\section{Ophthalmotherápico}
\begin{itemize}
\item {Grp. gram.:adj.}
\end{itemize}
Relativo á ophthalmotherapía.
\section{Ophthalmotomia}
\begin{itemize}
\item {Grp. gram.:f.}
\end{itemize}
\begin{itemize}
\item {Utilização:Cir.}
\end{itemize}
\begin{itemize}
\item {Proveniência:(Do gr. \textunderscore ophthalmos\textunderscore  + \textunderscore tome\textunderscore )}
\end{itemize}
Extirpação do ôlho.
Parte da Anatomia, que tem por objecto a dissecção do ôlho.
\section{Ophthalmotómico}
\begin{itemize}
\item {Grp. gram.:adj.}
\end{itemize}
Relativo á ophthalmotomia.
\section{Ophthalmoxyse}
\begin{itemize}
\item {fónica:csi}
\end{itemize}
\begin{itemize}
\item {Grp. gram.:f.}
\end{itemize}
\begin{itemize}
\item {Utilização:Med.}
\end{itemize}
\begin{itemize}
\item {Utilização:Ant.}
\end{itemize}
\begin{itemize}
\item {Proveniência:(Do gr. \textunderscore ophthalmos\textunderscore  + \textunderscore xusis\textunderscore )}
\end{itemize}
Escarificação, que se praticava na conjuntiva do ôlho, em casos de ophthalmia.
\section{Ophthalmoxystro}
\begin{itemize}
\item {fónica:csis}
\end{itemize}
\begin{itemize}
\item {Grp. gram.:m.}
\end{itemize}
\begin{itemize}
\item {Utilização:Med.}
\end{itemize}
\begin{itemize}
\item {Proveniência:(Do gr. \textunderscore ophthalmos\textunderscore  + \textunderscore xustron\textunderscore )}
\end{itemize}
Instrumento, espécie de pincel, para limpar a superficie do ôlho.
\section{Ophyra}
\begin{itemize}
\item {Grp. gram.:f.}
\end{itemize}
Gênero de insectos dípteros.
\section{Opiáceo}
\begin{itemize}
\item {Grp. gram.:adj.}
\end{itemize}
\begin{itemize}
\item {Proveniência:(De \textunderscore opiar\textunderscore )}
\end{itemize}
Misturado com ópio; que contém ópio.
\section{Opiado}
\begin{itemize}
\item {Grp. gram.:adj.}
\end{itemize}
\begin{itemize}
\item {Proveniência:(De \textunderscore opiar\textunderscore )}
\end{itemize}
Misturado com ópio; que contém ópio.
\section{Ópia}
\begin{itemize}
\item {Grp. gram.:f.}
\end{itemize}
\begin{itemize}
\item {Proveniência:(Lat. \textunderscore oppia\textunderscore )}
\end{itemize}
Nome de uma lei romana contra o luxo e excessivas despesas das mulheres.
\section{Opiar}
\begin{itemize}
\item {Grp. gram.:v. t.}
\end{itemize}
Misturar com ópio; deitar ópio em.
\section{Opiato}
\begin{itemize}
\item {Grp. gram.:m.}
\end{itemize}
\begin{itemize}
\item {Utilização:Pharm.}
\end{itemize}
Electuário, em que entra o ópio.
\section{Opidano}
\begin{itemize}
\item {Grp. gram.:m.}
\end{itemize}
\begin{itemize}
\item {Utilização:Ant.}
\end{itemize}
\begin{itemize}
\item {Proveniência:(Lat. \textunderscore oppidanus\textunderscore )}
\end{itemize}
Vizinho ou morador de uma localidade; conterrâneo.
\section{Ôpide}
\begin{itemize}
\item {Grp. gram.:m.}
\end{itemize}
Gênero de molluscos fósseis.
\section{Ópido}
\begin{itemize}
\item {Grp. gram.:m.}
\end{itemize}
\begin{itemize}
\item {Utilização:Poét.}
\end{itemize}
\begin{itemize}
\item {Proveniência:(Lat. \textunderscore oppidum\textunderscore )}
\end{itemize}
Cidade forte; praça fortificada. Cf. Filinto, \textunderscore D. Man.\textunderscore , I, 205.
\section{Opífero}
\begin{itemize}
\item {Grp. gram.:adj.}
\end{itemize}
\begin{itemize}
\item {Utilização:Poét.}
\end{itemize}
\begin{itemize}
\item {Proveniência:(Lat. \textunderscore opifer\textunderscore )}
\end{itemize}
Que dá auxílio; que socorre; auxiliar.
\section{Opífice}
\begin{itemize}
\item {Grp. gram.:m.}
\end{itemize}
\begin{itemize}
\item {Utilização:Des.}
\end{itemize}
\begin{itemize}
\item {Proveniência:(Lat. \textunderscore opifex\textunderscore )}
\end{itemize}
O mesmo que \textunderscore artífice\textunderscore . Cf. Filinto, \textunderscore D. Man.\textunderscore , III, 213.
\section{Opilação}
\begin{itemize}
\item {Grp. gram.:f.}
\end{itemize}
\begin{itemize}
\item {Utilização:Med.}
\end{itemize}
\begin{itemize}
\item {Proveniência:(Lat. \textunderscore oppilatio\textunderscore )}
\end{itemize}
Acto ou efeito de opilar; obstrucção.
\section{Opilante}
\begin{itemize}
\item {Grp. gram.:adj.}
\end{itemize}
O mesmo que \textunderscore opilativo\textunderscore .
\section{Opilar}
\begin{itemize}
\item {Grp. gram.:v. t.}
\end{itemize}
\begin{itemize}
\item {Proveniência:(Lat. \textunderscore oppilare\textunderscore )}
\end{itemize}
Obstruír; causar oclusão a.
\section{Opilativo}
\begin{itemize}
\item {Grp. gram.:adj.}
\end{itemize}
\begin{itemize}
\item {Proveniência:(De \textunderscore opilar\textunderscore )}
\end{itemize}
Que causa opilação.
Que tende a obstruír-se.
\section{Opilencia}
\begin{itemize}
\item {Grp. gram.:f.}
\end{itemize}
(Fórma antiga de \textunderscore epilepsia\textunderscore )
\section{Opília}
\begin{itemize}
\item {Grp. gram.:f.}
\end{itemize}
Gênero de plantas olacíneas.
\section{Ópilo}
\begin{itemize}
\item {Grp. gram.:m.}
\end{itemize}
Gênero de insectos coleópteros tetrâmeros.
\section{Opimo}
\begin{itemize}
\item {Grp. gram.:adj.}
\end{itemize}
\begin{itemize}
\item {Grp. gram.:Pl.}
\end{itemize}
\begin{itemize}
\item {Utilização:Fig.}
\end{itemize}
\begin{itemize}
\item {Proveniência:(Lat. \textunderscore opimus\textunderscore )}
\end{itemize}
Fecundo; excellente; abundante.
Diz-se dos despojos, colhidos pelo general romano que, por sua própria mão, matava o general do exército inimigo.
Diz-se de grandes e bellas vantagens, de excellentes acquisições, etc.: \textunderscore frutos opimos de uma empresa\textunderscore .
\section{Opinante}
\begin{itemize}
\item {Grp. gram.:m.  e  adj.}
\end{itemize}
\begin{itemize}
\item {Proveniência:(Lat. \textunderscore opinans\textunderscore )}
\end{itemize}
O que opina.
\section{Opinar}
\begin{itemize}
\item {Grp. gram.:v. t.  e  i.}
\end{itemize}
\begin{itemize}
\item {Proveniência:(Lat. \textunderscore opinari\textunderscore )}
\end{itemize}
Formar juízo, julgar.
Têr opinião de.
Dizer, expondo o que julga.
\section{Opinático}
\begin{itemize}
\item {Grp. gram.:adj.}
\end{itemize}
O mesmo que \textunderscore opiniático\textunderscore . Cf. \textunderscore Peregrinação\textunderscore , CLII, VII.
\section{Opinativo}
\begin{itemize}
\item {Grp. gram.:adj.}
\end{itemize}
\begin{itemize}
\item {Proveniência:(De \textunderscore opinar\textunderscore )}
\end{itemize}
Que se baseia na opinião particular.
Sujeito á divergência das opiniões individuaes.
Discutível; duvidoso, incerto.
\section{Opinável}
\begin{itemize}
\item {Grp. gram.:adj.}
\end{itemize}
\begin{itemize}
\item {Proveniência:(Lat. \textunderscore opinabilis\textunderscore )}
\end{itemize}
Que se póde opinar.
Sujeito a diversas opiniões.
Baseado em conjecturas.
\section{Opinião}
\begin{itemize}
\item {Grp. gram.:f.}
\end{itemize}
\begin{itemize}
\item {Utilização:Ant.}
\end{itemize}
\begin{itemize}
\item {Proveniência:(Lat. \textunderscore opinio\textunderscore )}
\end{itemize}
Juízo ou sentimento, que se manifesta em assumpto sujeito a deliberação.
Parecer, voto.
Crença.
Fama: \textunderscore a opinião é-lhe desfavorável\textunderscore .
Emprehendimento.
\section{Opiniaticidade}
\begin{itemize}
\item {Grp. gram.:f.}
\end{itemize}
Qualidade de opiniático. Cf. S. Romero, \textunderscore M. de Assis\textunderscore , 98.
\section{Opiniático}
\begin{itemize}
\item {Grp. gram.:adj.}
\end{itemize}
\begin{itemize}
\item {Proveniência:(De \textunderscore opinião\textunderscore )}
\end{itemize}
Teimoso, aferrado a sua opinião ou á sua vontade.
Contumaz.
Orgulhoso.
\section{Opinioso}
\begin{itemize}
\item {Grp. gram.:adj.}
\end{itemize}
\begin{itemize}
\item {Proveniência:(Lat. \textunderscore opiniosus\textunderscore )}
\end{itemize}
O mesmo que \textunderscore opiniático\textunderscore .
\section{Ópio}
\begin{itemize}
\item {Grp. gram.:m.}
\end{itemize}
\begin{itemize}
\item {Proveniência:(Lat. \textunderscore opium\textunderscore )}
\end{itemize}
Suco, extrahido das cápsulas de diversas espécies de papoila.
\section{Opiofagia}
\begin{itemize}
\item {Grp. gram.:f.}
\end{itemize}
Qualidade de opiófago.
\section{Opiófago}
\begin{itemize}
\item {Grp. gram.:m.}
\end{itemize}
\begin{itemize}
\item {Proveniência:(Do gr. \textunderscore opion\textunderscore  + \textunderscore phagein\textunderscore )}
\end{itemize}
Aquele que come ópio.
\section{Opiologia}
\begin{itemize}
\item {Grp. gram.:f.}
\end{itemize}
\begin{itemize}
\item {Proveniência:(Do gr. \textunderscore opion\textunderscore  + \textunderscore logos\textunderscore )}
\end{itemize}
Tratado á cêrca do ópio.
\section{Opiológico}
\begin{itemize}
\item {Grp. gram.:adj.}
\end{itemize}
Relativo á opiologia.
\section{Opiophagia}
\begin{itemize}
\item {Grp. gram.:f.}
\end{itemize}
Qualidade de opióphago.
\section{Opióphago}
\begin{itemize}
\item {Grp. gram.:m.}
\end{itemize}
\begin{itemize}
\item {Proveniência:(Do gr. \textunderscore opion\textunderscore  + \textunderscore phagein\textunderscore )}
\end{itemize}
Aquelle que come ópio.
\section{Opiparamente}
\begin{itemize}
\item {Grp. gram.:adv.}
\end{itemize}
De modo opíparo; lautamente.
\section{Opíparo}
\begin{itemize}
\item {Grp. gram.:adj.}
\end{itemize}
\begin{itemize}
\item {Proveniência:(Lat. \textunderscore opiparus\textunderscore )}
\end{itemize}
Sumptuoso; magnificente; lauto.
\section{Opísthion}
\begin{itemize}
\item {Grp. gram.:m.}
\end{itemize}
\begin{itemize}
\item {Proveniência:(Gr. \textunderscore opisthion\textunderscore )}
\end{itemize}
Ponto médio do bôrdo posterior do buraco occipital.
\section{Opisthobranchios}
\begin{itemize}
\item {fónica:qui}
\end{itemize}
\begin{itemize}
\item {Grp. gram.:m. pl.}
\end{itemize}
\begin{itemize}
\item {Proveniência:(Do gr. \textunderscore opisthen\textunderscore  + \textunderscore brankhia\textunderscore )}
\end{itemize}
Divisão de molluscos gasterópodes.
\section{Opisthocéphalo}
\begin{itemize}
\item {Grp. gram.:m.}
\end{itemize}
\begin{itemize}
\item {Utilização:Anat.}
\end{itemize}
O mesmo que \textunderscore occipício\textunderscore .
\section{Opisthocyphose}
\begin{itemize}
\item {Grp. gram.:f.}
\end{itemize}
\begin{itemize}
\item {Utilização:Med.}
\end{itemize}
\begin{itemize}
\item {Proveniência:(Do gr. \textunderscore opisthen\textunderscore  + \textunderscore cuphosis\textunderscore )}
\end{itemize}
Curvatura da espinha dorsal para trás.
\section{Opisthódomo}
\begin{itemize}
\item {Grp. gram.:m.}
\end{itemize}
\begin{itemize}
\item {Proveniência:(Do gr. \textunderscore opisthen\textunderscore  + \textunderscore domos\textunderscore )}
\end{itemize}
Pórtico ou vestíbulo de um templo, na parte posterior.
\section{Opisthogástrico}
\begin{itemize}
\item {Grp. gram.:adj.}
\end{itemize}
\begin{itemize}
\item {Utilização:Anat.}
\end{itemize}
\begin{itemize}
\item {Proveniência:(Do gr. \textunderscore opisten\textunderscore  + \textunderscore gaster\textunderscore )}
\end{itemize}
Situado atrás do estômago.
\section{Opisthognatho}
\begin{itemize}
\item {Grp. gram.:m.}
\end{itemize}
Gênero de peixes acanthopterýgios.
\section{Opisthographia}
\begin{itemize}
\item {Grp. gram.:f.}
\end{itemize}
Qualidade ou estado de opisthógrapho.
\section{Opisthógrapho}
\begin{itemize}
\item {Grp. gram.:adj.}
\end{itemize}
\begin{itemize}
\item {Grp. gram.:M.}
\end{itemize}
\begin{itemize}
\item {Proveniência:(Lat. \textunderscore opisthographus\textunderscore )}
\end{itemize}
Que está escrito por detrás.
Fôlha ou documento, escrito de ambos os lados.
\section{Opisthotónico}
\begin{itemize}
\item {Grp. gram.:adj.}
\end{itemize}
Relativo ao opisthótono.
\section{Opisthótono}
\begin{itemize}
\item {Grp. gram.:m.}
\end{itemize}
\begin{itemize}
\item {Utilização:Med.}
\end{itemize}
\begin{itemize}
\item {Proveniência:(Lat. \textunderscore opisthotonos\textunderscore )}
\end{itemize}
Tétano, que obriga o doente a curvar-se para trás.
\section{Opístio}
\begin{itemize}
\item {Grp. gram.:m.}
\end{itemize}
\begin{itemize}
\item {Proveniência:(Gr. \textunderscore opisthion\textunderscore )}
\end{itemize}
Ponto médio do bôrdo posterior do buraco occipital.
\section{Opístion}
\begin{itemize}
\item {Grp. gram.:m.}
\end{itemize}
\begin{itemize}
\item {Proveniência:(Gr. \textunderscore opisthion\textunderscore )}
\end{itemize}
Ponto médio do bôrdo posterior do buraco occipital.
\section{Opistobranquios}
\begin{itemize}
\item {Grp. gram.:m. pl.}
\end{itemize}
\begin{itemize}
\item {Proveniência:(Do gr. \textunderscore opisthen\textunderscore  + \textunderscore brankhia\textunderscore )}
\end{itemize}
Divisão de moluscos gasterópodes.
\section{Opistocéfalo}
\begin{itemize}
\item {Grp. gram.:m.}
\end{itemize}
\begin{itemize}
\item {Utilização:Anat.}
\end{itemize}
O mesmo que \textunderscore occipício\textunderscore .
\section{Opistocifose}
\begin{itemize}
\item {Grp. gram.:f.}
\end{itemize}
\begin{itemize}
\item {Utilização:Med.}
\end{itemize}
\begin{itemize}
\item {Proveniência:(Do gr. \textunderscore opisthen\textunderscore  + \textunderscore cuphosis\textunderscore )}
\end{itemize}
Curvatura da espinha dorsal para trás.
\section{Opistódomo}
\begin{itemize}
\item {Grp. gram.:m.}
\end{itemize}
\begin{itemize}
\item {Proveniência:(Do gr. \textunderscore opisthen\textunderscore  + \textunderscore domos\textunderscore )}
\end{itemize}
Pórtico ou vestíbulo de um templo, na parte posterior.
\section{Opistogástrico}
\begin{itemize}
\item {Grp. gram.:adj.}
\end{itemize}
\begin{itemize}
\item {Utilização:Anat.}
\end{itemize}
\begin{itemize}
\item {Proveniência:(Do gr. \textunderscore opisten\textunderscore  + \textunderscore gaster\textunderscore )}
\end{itemize}
Situado atrás do estômago.
\section{Opistognato}
\begin{itemize}
\item {Grp. gram.:m.}
\end{itemize}
Gênero de peixes acantopterígios.
\section{Opistografia}
\begin{itemize}
\item {Grp. gram.:f.}
\end{itemize}
Qualidade ou estado de opistógrafo.
\section{Opistógrafo}
\begin{itemize}
\item {Grp. gram.:adj.}
\end{itemize}
\begin{itemize}
\item {Grp. gram.:M.}
\end{itemize}
\begin{itemize}
\item {Proveniência:(Lat. \textunderscore opisthographus\textunderscore )}
\end{itemize}
Que está escrito por detrás.
Fôlha ou documento, escrito de ambos os lados.
\section{Opistotónico}
\begin{itemize}
\item {Grp. gram.:adj.}
\end{itemize}
Relativo ao opistótono.
\section{Opistótono}
\begin{itemize}
\item {Grp. gram.:m.}
\end{itemize}
\begin{itemize}
\item {Utilização:Med.}
\end{itemize}
\begin{itemize}
\item {Proveniência:(Lat. \textunderscore opisthotonos\textunderscore )}
\end{itemize}
Tétano, que obriga o doente a curvar-se para trás.
\section{Opitimo}
\begin{itemize}
\item {Grp. gram.:m.}
\end{itemize}
Planta da serra de Sintra.
\section{Opízia}
\begin{itemize}
\item {Grp. gram.:f.}
\end{itemize}
Gênero de plantas gramíneas.
\section{Oplário}
\begin{itemize}
\item {Grp. gram.:m.}
\end{itemize}
\begin{itemize}
\item {Utilização:Bot.}
\end{itemize}
Pedúnculo oco, em fórma de funil, que sustenta a fructificação de certos líchens.
\section{Oplismeno}
\begin{itemize}
\item {Grp. gram.:m.}
\end{itemize}
Gênero de plantas gramíneas.
\section{Óplo}
\begin{itemize}
\item {Grp. gram.:m.}
\end{itemize}
\begin{itemize}
\item {Proveniência:(Gr. \textunderscore oplon\textunderscore )}
\end{itemize}
Escudo oval, usado pela antiga infantaria grega.
\section{Oplocnemo}
\begin{itemize}
\item {Grp. gram.:m.}
\end{itemize}
Gênero de insectos coleópteros tetrâmeros.
\section{Oplóforo}
\begin{itemize}
\item {Grp. gram.:m.}
\end{itemize}
\begin{itemize}
\item {Proveniência:(Do gr. \textunderscore oplon\textunderscore  + \textunderscore phoros\textunderscore )}
\end{itemize}
Gênero de crustáceos decápodes.
\section{Óplon}
\begin{itemize}
\item {Grp. gram.:m.}
\end{itemize}
\begin{itemize}
\item {Proveniência:(Gr. \textunderscore oplon\textunderscore )}
\end{itemize}
Escudo oval, usado pela antiga infantaria grega.
\section{Oplóphoro}
\begin{itemize}
\item {Grp. gram.:m.}
\end{itemize}
\begin{itemize}
\item {Proveniência:(Do gr. \textunderscore oplon\textunderscore  + \textunderscore phoros\textunderscore )}
\end{itemize}
Gênero de crustáceos decápodes.
\section{Oplotério}
\begin{itemize}
\item {Grp. gram.:m.}
\end{itemize}
Gênero de paquidermes fósseis.
\section{Oplothério}
\begin{itemize}
\item {Grp. gram.:m.}
\end{itemize}
Gênero de pachydermes fósseis.
\section{Opobalsameira}
\begin{itemize}
\item {Grp. gram.:f.}
\end{itemize}
\begin{itemize}
\item {Proveniência:(De \textunderscore opobálsamo\textunderscore )}
\end{itemize}
Árvore burserácea, (\textunderscore balsamodendron gileadense\textunderscore ).
\section{Opobálsamo}
\begin{itemize}
\item {Grp. gram.:m.}
\end{itemize}
\begin{itemize}
\item {Proveniência:(Do gr. \textunderscore opos\textunderscore  + \textunderscore balsamon\textunderscore )}
\end{itemize}
Bálsamo, extrahido da opobalsameira.
\section{Opocárpaso}
\begin{itemize}
\item {Grp. gram.:m.}
\end{itemize}
\begin{itemize}
\item {Proveniência:(Gr. \textunderscore opocarpason\textunderscore )}
\end{itemize}
Espécie de resina ou goma, com que na Abyssínia se dá consistência aos estofos.
\section{Opocárpatho}
\begin{itemize}
\item {Grp. gram.:f.}
\end{itemize}
\begin{itemize}
\item {Proveniência:(Lat. \textunderscore opocarpathon\textunderscore )}
\end{itemize}
Planta venenosa, conhecida dos antigos.
Veneno, extrahido dessa planta.
\section{Opocárpato}
\begin{itemize}
\item {Grp. gram.:f.}
\end{itemize}
\begin{itemize}
\item {Proveniência:(Lat. \textunderscore opocarpathon\textunderscore )}
\end{itemize}
Planta venenosa, conhecida dos antigos.
Veneno, extraido dessa planta.
\section{Opocefalia}
\begin{itemize}
\item {Grp. gram.:f.}
\end{itemize}
Conformação de opocéfalo.
\section{Opocéfalo}
\begin{itemize}
\item {Grp. gram.:m.}
\end{itemize}
\begin{itemize}
\item {Proveniência:(Do gr. \textunderscore ops\textunderscore  + \textunderscore kephale\textunderscore )}
\end{itemize}
Monstro sem bôca, de maxilas atrofiadas, e orelhas reunidas sôbre a cabeça.
\section{Opocephalia}
\begin{itemize}
\item {Grp. gram.:f.}
\end{itemize}
Conformação de opocéphalo.
\section{Opocéphalo}
\begin{itemize}
\item {Grp. gram.:m.}
\end{itemize}
\begin{itemize}
\item {Proveniência:(Do gr. \textunderscore ops\textunderscore  + \textunderscore kephale\textunderscore )}
\end{itemize}
Monstro sem bôca, de maxillas atrophiadas, e orelhas reunidas sôbre a cabeça.
\section{Opodeldoque}
\begin{itemize}
\item {Grp. gram.:m.}
\end{itemize}
\begin{itemize}
\item {Proveniência:(Do fr. \textunderscore opodeldoc\textunderscore )}
\end{itemize}
Nome de um bálsamo pharmacêutico, applicado contra dôres rheumáticas.
\section{Opódimo}
\begin{itemize}
\item {Grp. gram.:m.}
\end{itemize}
\begin{itemize}
\item {Proveniência:(Do gr. \textunderscore ops\textunderscore , \textunderscore opos\textunderscore  + \textunderscore didumos\textunderscore )}
\end{itemize}
Monstro, cuja cabeça se divide em duas partes distintas, da região ocular para cima.
\section{Opódymo}
\begin{itemize}
\item {Grp. gram.:m.}
\end{itemize}
\begin{itemize}
\item {Proveniência:(Do gr. \textunderscore ops\textunderscore , \textunderscore opos\textunderscore  + \textunderscore didumos\textunderscore )}
\end{itemize}
Monstro, cuja cabeça se divide em duas partes distintas, da região ocular para cima.
\section{Opoente}
\begin{itemize}
\item {Grp. gram.:m.}
\end{itemize}
\begin{itemize}
\item {Grp. gram.:M.}
\end{itemize}
O mesmo que \textunderscore oponente\textunderscore .
Aquele que se opõe; opositor. Cf. Sousa, \textunderscore Vida do Arceb.\textunderscore , I, 64.
\section{Opol}
\begin{itemize}
\item {Grp. gram.:m.}
\end{itemize}
\begin{itemize}
\item {Utilização:Ant.}
\end{itemize}
\begin{itemize}
\item {Proveniência:(Do gr. \textunderscore opos\textunderscore )}
\end{itemize}
Suco das plantas em geral.
\section{Oponente}
\begin{itemize}
\item {Grp. gram.:adj.}
\end{itemize}
\begin{itemize}
\item {Proveniência:(Lat. \textunderscore opponens\textunderscore )}
\end{itemize}
Que se opõe; oposto.
\section{Opopónax}
\begin{itemize}
\item {fónica:nacse}
\end{itemize}
\begin{itemize}
\item {Grp. gram.:m.}
\end{itemize}
Goma, extrahida de uma planta umbellífera.
(Directamente, do fr.)
\section{Opor}
\begin{itemize}
\item {Grp. gram.:v. t.}
\end{itemize}
\begin{itemize}
\item {Proveniência:(Do lat. \textunderscore opponere\textunderscore )}
\end{itemize}
Colocar contra ou defronte de.
Pôr obstáculo a.
Colocar, formando contraste.
Proceder contrariamente a.
Objectar.
Cotejar.
Dispor para a luta.
\section{Opórica}
\begin{itemize}
\item {Grp. gram.:f.}
\end{itemize}
\begin{itemize}
\item {Proveniência:(Lat. \textunderscore oporice\textunderscore )}
\end{itemize}
Antigo medicamento, composto de certos frutos.
\section{Oportunamente}
\begin{itemize}
\item {Grp. gram.:adv.}
\end{itemize}
De modo oportuno.
A tempo; na ocasião própria.
Convenientemente.
\section{Oportunidade}
\begin{itemize}
\item {Grp. gram.:f.}
\end{itemize}
\begin{itemize}
\item {Proveniência:(Lat. \textunderscore opportunitas\textunderscore )}
\end{itemize}
Qualidade do que é oportuno.
Ocasião favorável.
Conveniência.
\section{Oportunismo}
\begin{itemize}
\item {Grp. gram.:m.}
\end{itemize}
\begin{itemize}
\item {Proveniência:(De \textunderscore oportuno\textunderscore )}
\end{itemize}
Sistema político, que transige com as circunstâncias e se acomoda a elas.
\section{Oportunista}
\begin{itemize}
\item {Grp. gram.:m. ,  f.  e  adj.}
\end{itemize}
\begin{itemize}
\item {Proveniência:(De \textunderscore oportuno\textunderscore )}
\end{itemize}
Pessôa, partidária do oportunismo.
\section{Oportuno}
\begin{itemize}
\item {Grp. gram.:adj.}
\end{itemize}
\begin{itemize}
\item {Proveniência:(Lat. \textunderscore opportunus\textunderscore )}
\end{itemize}
Que vem a tempo, ou segundo o que se deseja.
Apropriado, comodo.
Favorável.
Feito a propósito.
\section{Oposição}
\begin{itemize}
\item {Grp. gram.:f.}
\end{itemize}
\begin{itemize}
\item {Proveniência:(Lat. \textunderscore oppositio\textunderscore )}
\end{itemize}
Acto ou efeito de opor.
Qualidade do que é oposto.
Impedimento.
Figura de Retórica, com que se reúnem ideias que parecem contrárias.
Parcialidade ou facção política, que se opõe ao Govêrno.
Discordância de proposições.
Contraste de fórmas, em belas-artes.
Hostilidade.
Situação de dois astros que, em relação á Terra, occupam pontos diametralmente opostos.
\section{Oposicionista}
\begin{itemize}
\item {Grp. gram.:m. ,  f.  e  adj.}
\end{itemize}
\begin{itemize}
\item {Proveniência:(Do lat. \textunderscore oppositio\textunderscore )}
\end{itemize}
Pessôa, que faz oposição.
\section{Oposina}
\begin{itemize}
\item {Grp. gram.:f.}
\end{itemize}
\begin{itemize}
\item {Utilização:Chím.}
\end{itemize}
\begin{itemize}
\item {Proveniência:(Do gr. \textunderscore opos\textunderscore , suco)}
\end{itemize}
Substância albuminóide, que existe na carne muscular.
\section{Opositiflor}
\begin{itemize}
\item {Grp. gram.:adj.}
\end{itemize}
\begin{itemize}
\item {Utilização:Bot.}
\end{itemize}
\begin{itemize}
\item {Proveniência:(Lat. \textunderscore oppositus\textunderscore )}
\end{itemize}
Que tem as flôres em pedunculos opostos.
\section{Opositifólio}
\begin{itemize}
\item {Grp. gram.:adj.}
\end{itemize}
\begin{itemize}
\item {Utilização:Bot.}
\end{itemize}
\begin{itemize}
\item {Proveniência:(Do lat. \textunderscore oppositus\textunderscore  + \textunderscore folium\textunderscore )}
\end{itemize}
Que tem fôlhas opostas.
Que nasce em frente das fôlhas.
\section{Opositivo}
\begin{itemize}
\item {Grp. gram.:adj.}
\end{itemize}
\begin{itemize}
\item {Proveniência:(Do lat. \textunderscore oppositus\textunderscore )}
\end{itemize}
Oposto.
Colocado em frente de outro, (falando-se dos órgãos de uma planta).
\section{Opósito}
\begin{itemize}
\item {Grp. gram.:m.}
\end{itemize}
\begin{itemize}
\item {Grp. gram.:M.}
\end{itemize}
O mesmo que \textunderscore oposto\textunderscore .
O mesmo que \textunderscore inimigo\textunderscore . Cf. \textunderscore Usque\textunderscore , 26 e 40 v.^o; Pant. de Aveiro, \textunderscore Itiner.\textunderscore , 172, (2.^a ed.)
\section{Opositor}
\begin{itemize}
\item {Grp. gram.:adj.}
\end{itemize}
\begin{itemize}
\item {Grp. gram.:M.}
\end{itemize}
\begin{itemize}
\item {Proveniência:(Do lat. \textunderscore oppositus\textunderscore )}
\end{itemize}
Que se opõe.
Indivíduo que concorre a um emprêgo.
Candidato.
\section{Opostamente}
\begin{itemize}
\item {Grp. gram.:adv.}
\end{itemize}
De modo oposto.
Contrariamente; em sentido oposto ou inverso.
\section{Oposto}
\begin{itemize}
\item {Grp. gram.:adj.}
\end{itemize}
\begin{itemize}
\item {Utilização:Bot.}
\end{itemize}
\begin{itemize}
\item {Utilização:Geom.}
\end{itemize}
\begin{itemize}
\item {Grp. gram.:M.}
\end{itemize}
\begin{itemize}
\item {Proveniência:(Lat. \textunderscore oppositus\textunderscore )}
\end{itemize}
Fronteiro.
Que causa obstáculo.
Contrário.
Diferente.
Que faz ou causa oposição.
Diz-se das fôlhas que, no mesmo eixo e no mesmo plano horizontal, ficam reciprocamente fronteiras.
Diz-se das figuras que tem ângulos opostos.
E diz-se dos ângulos sólidos ou planos, que são formados em partes contrárias por superfícies ou linhas que se cortam num ponto.
Aquilo que se opõe ou é contrário.
\section{Opoterapía}
\begin{itemize}
\item {Grp. gram.:f.}
\end{itemize}
\begin{itemize}
\item {Utilização:Med.}
\end{itemize}
\begin{itemize}
\item {Proveniência:(Do gr. \textunderscore opos\textunderscore  + \textunderscore therapeia\textunderscore )}
\end{itemize}
Método terapêutico, que consiste na injecção subcutânea de sucos ou extractos orgânicos.
\section{Opoterápico}
\begin{itemize}
\item {Grp. gram.:adj.}
\end{itemize}
Relativo á opoterapía.
\section{Opotherapía}
\begin{itemize}
\item {Grp. gram.:f.}
\end{itemize}
\begin{itemize}
\item {Utilização:Med.}
\end{itemize}
\begin{itemize}
\item {Proveniência:(Do gr. \textunderscore opos\textunderscore  + \textunderscore therapeia\textunderscore )}
\end{itemize}
Méthodo therapêutico, que consiste na injecção subcutânea de sucos ou extractos orgânicos.
\section{Opotherápico}
\begin{itemize}
\item {Grp. gram.:adj.}
\end{itemize}
Relativo á opotherapía.
\section{Óppia}
\begin{itemize}
\item {Grp. gram.:f.}
\end{itemize}
\begin{itemize}
\item {Proveniência:(Lat. \textunderscore oppia\textunderscore )}
\end{itemize}
Nome de uma lei romana contra o luxo e excessivas despesas das mulheres.
\section{Oppidano}
\begin{itemize}
\item {Grp. gram.:m.}
\end{itemize}
\begin{itemize}
\item {Utilização:Ant.}
\end{itemize}
\begin{itemize}
\item {Proveniência:(Lat. \textunderscore oppidanus\textunderscore )}
\end{itemize}
Vizinho ou morador de uma localidade; conterrâneo.
\section{Óppido}
\begin{itemize}
\item {Grp. gram.:m.}
\end{itemize}
\begin{itemize}
\item {Utilização:Poét.}
\end{itemize}
\begin{itemize}
\item {Proveniência:(Lat. \textunderscore oppidum\textunderscore )}
\end{itemize}
Cidade forte; praça fortificada. Cf. Filinto, \textunderscore D. Man.\textunderscore , I, 205.
\section{Oppilação}
\begin{itemize}
\item {Grp. gram.:f.}
\end{itemize}
\begin{itemize}
\item {Utilização:Med.}
\end{itemize}
\begin{itemize}
\item {Proveniência:(Lat. \textunderscore oppilatio\textunderscore )}
\end{itemize}
Acto ou effeito de oppilar; obstrucção.
\section{Oppilante}
\begin{itemize}
\item {Grp. gram.:adj.}
\end{itemize}
O mesmo que \textunderscore oppilativo\textunderscore .
\section{Oppilar}
\begin{itemize}
\item {Grp. gram.:v. t.}
\end{itemize}
\begin{itemize}
\item {Proveniência:(Lat. \textunderscore oppilare\textunderscore )}
\end{itemize}
Obstruír; causar occlusão a.
\section{Oppilativo}
\begin{itemize}
\item {Grp. gram.:adj.}
\end{itemize}
\begin{itemize}
\item {Proveniência:(De \textunderscore oppilar\textunderscore )}
\end{itemize}
Que causa oppilação.
Que tende a obstruír-se.
\section{Oppoente}
\begin{itemize}
\item {Grp. gram.:m.}
\end{itemize}
\begin{itemize}
\item {Grp. gram.:M.}
\end{itemize}
O mesmo que \textunderscore opponente\textunderscore .
Aquelle que se oppõe; oppositor. Cf. Sousa, \textunderscore Vida do Arceb.\textunderscore , I, 64.
\section{Opponente}
\begin{itemize}
\item {Grp. gram.:adj.}
\end{itemize}
\begin{itemize}
\item {Proveniência:(Lat. \textunderscore opponens\textunderscore )}
\end{itemize}
Que se oppõe; opposto.
\section{Oppor}
\begin{itemize}
\item {Grp. gram.:v. t.}
\end{itemize}
\begin{itemize}
\item {Proveniência:(Do lat. \textunderscore opponere\textunderscore )}
\end{itemize}
Collocar contra ou defronte de.
Pôr obstáculo a.
Collocar, formando contraste.
Proceder contrariamente a.
Objectar.
Cotejar.
Dispor para a luta.
\section{Opportunamente}
\begin{itemize}
\item {Grp. gram.:adv.}
\end{itemize}
De modo opportuno.
A tempo; na occasião própria.
Convenientemente.
\section{Opportunidade}
\begin{itemize}
\item {Grp. gram.:f.}
\end{itemize}
\begin{itemize}
\item {Proveniência:(Lat. \textunderscore opportunitas\textunderscore )}
\end{itemize}
Qualidade do que é opportuno.
Occasião favorável.
Conveniência.
\section{Opportunismo}
\begin{itemize}
\item {Grp. gram.:m.}
\end{itemize}
\begin{itemize}
\item {Proveniência:(De \textunderscore opportuno\textunderscore )}
\end{itemize}
Systema político, que transige com as circunstâncias e se acommoda a ellas.
\section{Opportunista}
\begin{itemize}
\item {Grp. gram.:m. ,  f.  e  adj.}
\end{itemize}
\begin{itemize}
\item {Proveniência:(De \textunderscore opportuno\textunderscore )}
\end{itemize}
Pessôa, partidária do opportunismo.
\section{Opportuno}
\begin{itemize}
\item {Grp. gram.:adj.}
\end{itemize}
\begin{itemize}
\item {Proveniência:(Lat. \textunderscore opportunus\textunderscore )}
\end{itemize}
Que vem a tempo, ou segundo o que se deseja.
Apropriado, commodo.
Favorável.
Feito a propósito.
\section{Opposição}
\begin{itemize}
\item {Grp. gram.:f.}
\end{itemize}
\begin{itemize}
\item {Proveniência:(Lat. \textunderscore oppositio\textunderscore )}
\end{itemize}
Acto ou effeito de oppor.
Qualidade do que é opposto.
Impedimento.
Figura de Rhetórica, com que se reúnem ideias que parecem contrárias.
Parcialidade ou facção política, que se oppõe ao Govêrno.
Discordância de proposições.
Contraste de fórmas, em bellas-artes.
Hostilidade.
Situação de dois astros que, em relação á Terra, occupam pontos diametralmente oppostos.
\section{Opposicionista}
\begin{itemize}
\item {Grp. gram.:m. ,  f.  e  adj.}
\end{itemize}
\begin{itemize}
\item {Proveniência:(Do lat. \textunderscore oppositio\textunderscore )}
\end{itemize}
Pessôa, que faz opposição.
\section{Oppositiflor}
\begin{itemize}
\item {Grp. gram.:adj.}
\end{itemize}
\begin{itemize}
\item {Utilização:Bot.}
\end{itemize}
\begin{itemize}
\item {Proveniência:(Lat. \textunderscore oppositus\textunderscore )}
\end{itemize}
Que tem as flôres em pedunculos oppostos.
\section{Oppositifloro}
\begin{itemize}
\item {Grp. gram.:adj.}
\end{itemize}
\begin{itemize}
\item {Utilização:Bot.}
\end{itemize}
\begin{itemize}
\item {Proveniência:(Lat. \textunderscore oppositus\textunderscore )}
\end{itemize}
Que tem as flôres em pedunculos oppostos.
\section{Oppositifólio}
\begin{itemize}
\item {Grp. gram.:adj.}
\end{itemize}
\begin{itemize}
\item {Utilização:Bot.}
\end{itemize}
\begin{itemize}
\item {Proveniência:(Do lat. \textunderscore oppositus\textunderscore  + \textunderscore folium\textunderscore )}
\end{itemize}
Que tem fôlhas oppostas.
Que nasce em frente das fôlhas.
\section{Oppositivo}
\begin{itemize}
\item {Grp. gram.:adj.}
\end{itemize}
\begin{itemize}
\item {Proveniência:(Do lat. \textunderscore oppositus\textunderscore )}
\end{itemize}
Opposto.
Collocado em frente de outro, (falando-se dos órgãos de uma planta).
\section{Oppósito}
\begin{itemize}
\item {Grp. gram.:m.}
\end{itemize}
\begin{itemize}
\item {Grp. gram.:M.}
\end{itemize}
O mesmo que \textunderscore opposto\textunderscore .
O mesmo que \textunderscore inimigo\textunderscore . Cf. \textunderscore Usque\textunderscore , 26 e 40 v.^o; Pant. de Aveiro, \textunderscore Itiner.\textunderscore , 172, (2.^a ed.)
\section{Oppositor}
\begin{itemize}
\item {Grp. gram.:adj.}
\end{itemize}
\begin{itemize}
\item {Grp. gram.:M.}
\end{itemize}
\begin{itemize}
\item {Proveniência:(Do lat. \textunderscore oppositus\textunderscore )}
\end{itemize}
Que se oppõe.
Indivíduo que concorre a um emprêgo.
Candidato.
\section{Oppostamente}
\begin{itemize}
\item {Grp. gram.:adv.}
\end{itemize}
De modo opposto.
Contrariamente; em sentido opposto ou inverso.
\section{Opposto}
\begin{itemize}
\item {Grp. gram.:adj.}
\end{itemize}
\begin{itemize}
\item {Utilização:Bot.}
\end{itemize}
\begin{itemize}
\item {Utilização:Geom.}
\end{itemize}
\begin{itemize}
\item {Grp. gram.:M.}
\end{itemize}
\begin{itemize}
\item {Proveniência:(Lat. \textunderscore oppositus\textunderscore )}
\end{itemize}
Fronteiro.
Que causa obstáculo.
Contrário.
Differente.
Que faz ou causa opposição.
Diz-se das fôlhas que, no mesmo eixo e no mesmo plano horizontal, ficam reciprocamente fronteiras.
Diz-se das figuras que tem ângulos oppostos.
E diz-se dos ângulos sólidos ou planos, que são formados em partes contrárias por superfícies ou linhas que se cortam num ponto.
Aquillo que se oppõe ou é contrário.
\section{Oppressão}
\begin{itemize}
\item {Grp. gram.:f.}
\end{itemize}
\begin{itemize}
\item {Proveniência:(Lat. \textunderscore oppressio\textunderscore )}
\end{itemize}
Acto ou effeito de opprimir.
Difficuldade de respiração.
Incommodo de quem respira mal.
Abatimento de fôrças.
Vexame; tyrannia.
\section{Oppressivo}
\begin{itemize}
\item {Grp. gram.:adj.}
\end{itemize}
\begin{itemize}
\item {Proveniência:(De \textunderscore oppresso\textunderscore )}
\end{itemize}
Que opprime.
Que serve para opprimir.
\section{Oppresso}
\begin{itemize}
\item {Grp. gram.:adj.}
\end{itemize}
\begin{itemize}
\item {Proveniência:(Lat. \textunderscore oppressus\textunderscore )}
\end{itemize}
Opprimido.
\section{Oppressor}
\begin{itemize}
\item {Grp. gram.:m.  e  adj.}
\end{itemize}
\begin{itemize}
\item {Proveniência:(Lat. \textunderscore oppressor\textunderscore )}
\end{itemize}
O que opprime.
\section{Opressão}
\begin{itemize}
\item {Grp. gram.:f.}
\end{itemize}
\begin{itemize}
\item {Proveniência:(Lat. \textunderscore oppressio\textunderscore )}
\end{itemize}
Acto ou efeito de oprimir.
Dificuldade de respiração.
Incomodo de quem respira mal.
Abatimento de fôrças.
Vexame; tirania.
\section{Opressivo}
\begin{itemize}
\item {Grp. gram.:adj.}
\end{itemize}
\begin{itemize}
\item {Proveniência:(De \textunderscore opresso\textunderscore )}
\end{itemize}
Que oprime.
Que serve para oprimir.
\section{Opresso}
\begin{itemize}
\item {Grp. gram.:adj.}
\end{itemize}
\begin{itemize}
\item {Proveniência:(Lat. \textunderscore oppressus\textunderscore )}
\end{itemize}
Oprimido.
\section{Opressor}
\begin{itemize}
\item {Grp. gram.:m.  e  adj.}
\end{itemize}
\begin{itemize}
\item {Proveniência:(Lat. \textunderscore oppressor\textunderscore )}
\end{itemize}
O que oprime.
\section{Oppressório}
\begin{itemize}
\item {Grp. gram.:adj.}
\end{itemize}
O mesmo que \textunderscore oppressivo\textunderscore .
\section{Opprimente}
\begin{itemize}
\item {Grp. gram.:adj.}
\end{itemize}
Que opprime; oppressivo.
\section{Opprimido}
\begin{itemize}
\item {Grp. gram.:adj.}
\end{itemize}
\begin{itemize}
\item {Grp. gram.:M.}
\end{itemize}
\begin{itemize}
\item {Proveniência:(De \textunderscore opprimir\textunderscore )}
\end{itemize}
Vexado perseguido.
Indivíduo opprimido.
\section{Opprimir}
\begin{itemize}
\item {Grp. gram.:v. t.}
\end{itemize}
\begin{itemize}
\item {Proveniência:(Lat. \textunderscore opprimere\textunderscore )}
\end{itemize}
Carregar muito.
Vexar.
Tyrannizar.
Violentar.
Aniquilar.
Perseguir.
Affligir.
\section{Oppróbio}
\begin{itemize}
\item {Grp. gram.:m.}
\end{itemize}
\begin{itemize}
\item {Proveniência:(Lat. \textunderscore opprobium\textunderscore )}
\end{itemize}
A maior deshonra.
Ignomínia.
Infâmia.
Affronta infamante.
Extrema abjecção.
\section{Opprobrioso}
\begin{itemize}
\item {Grp. gram.:adj.}
\end{itemize}
\begin{itemize}
\item {Proveniência:(Lat. \textunderscore opprobriosus\textunderscore )}
\end{itemize}
Que causa oppróbio.
Em que há oppróbrio.
Extremamente vergonhoso; infamante.
\section{Oppugnação}
\begin{itemize}
\item {Grp. gram.:f.}
\end{itemize}
\begin{itemize}
\item {Proveniência:(Lat. \textunderscore oppugnatio\textunderscore )}
\end{itemize}
Acto ou effeito de oppugnar.
Assalto; ataque.
\section{Oppugnador}
\begin{itemize}
\item {Grp. gram.:m.  e  adj.}
\end{itemize}
\begin{itemize}
\item {Proveniência:(Lat. \textunderscore oppugnator\textunderscore )}
\end{itemize}
O que oppugna.
\section{Oppugnar}
\begin{itemize}
\item {Grp. gram.:v. t.}
\end{itemize}
\begin{itemize}
\item {Utilização:Fig.}
\end{itemize}
\begin{itemize}
\item {Proveniência:(Lat. \textunderscore oppugnare\textunderscore )}
\end{itemize}
Pugnar contra; atacar.
Refutar, rejeitar.
\section{Opressório}
\begin{itemize}
\item {Grp. gram.:adj.}
\end{itemize}
O mesmo que \textunderscore opressivo\textunderscore .
\section{Oprimente}
\begin{itemize}
\item {Grp. gram.:adj.}
\end{itemize}
Que oprime; opressivo.
\section{Oprimido}
\begin{itemize}
\item {Grp. gram.:adj.}
\end{itemize}
\begin{itemize}
\item {Grp. gram.:M.}
\end{itemize}
\begin{itemize}
\item {Proveniência:(De \textunderscore oprimir\textunderscore )}
\end{itemize}
Vexado perseguido.
Indivíduo oprimido.
\section{Oprimir}
\begin{itemize}
\item {Grp. gram.:v. t.}
\end{itemize}
\begin{itemize}
\item {Proveniência:(Lat. \textunderscore opprimere\textunderscore )}
\end{itemize}
Carregar muito.
Vexar.
Tiranizar.
Violentar.
Aniquilar.
Perseguir.
Afligir.
\section{Opróbio}
\begin{itemize}
\item {Grp. gram.:m.}
\end{itemize}
\begin{itemize}
\item {Proveniência:(Lat. \textunderscore opprobium\textunderscore )}
\end{itemize}
A maior deshonra.
Ignomínia.
Infâmia.
Afronta infamante.
Extrema abjecção.
\section{Oprobrioso}
\begin{itemize}
\item {Grp. gram.:adj.}
\end{itemize}
\begin{itemize}
\item {Proveniência:(Lat. \textunderscore opprobriosus\textunderscore )}
\end{itemize}
Que causa opróbio.
Em que há opróbrio.
Extremamente vergonhoso; infamante.
\section{Opsígono}
\begin{itemize}
\item {Grp. gram.:adj.}
\end{itemize}
\begin{itemize}
\item {Proveniência:(Gr. \textunderscore opsigonos\textunderscore )}
\end{itemize}
Diz-se dos dentes, nascidos depois dos molares.
\section{Opsimathia}
\begin{itemize}
\item {Grp. gram.:f.}
\end{itemize}
\begin{itemize}
\item {Proveniência:(Do gr. \textunderscore ops\textunderscore  + \textunderscore mathein\textunderscore )}
\end{itemize}
Vontade tardia de saber ou de apprender.
\section{Opsimatia}
\begin{itemize}
\item {Grp. gram.:f.}
\end{itemize}
\begin{itemize}
\item {Proveniência:(Do gr. \textunderscore ops\textunderscore  + \textunderscore mathein\textunderscore )}
\end{itemize}
Vontade tardia de saber ou de aprender.
\section{Opsiometria}
\begin{itemize}
\item {Grp. gram.:f.}
\end{itemize}
Applicação do opsiómetro.
\section{Opsiométrico}
\begin{itemize}
\item {Grp. gram.:adj.}
\end{itemize}
Relativo á opsiometria.
\section{Opsiómetro}
\begin{itemize}
\item {Grp. gram.:m.}
\end{itemize}
\begin{itemize}
\item {Proveniência:(Do gr. \textunderscore opsis\textunderscore  + \textunderscore metron\textunderscore )}
\end{itemize}
Instrumento, para determinar os limites da vista distinta.
\section{Opsofagia}
\begin{itemize}
\item {Grp. gram.:f.}
\end{itemize}
Qualidade de opsófago.
\section{Opsófago}
\begin{itemize}
\item {Grp. gram.:m.  e  adj.}
\end{itemize}
\begin{itemize}
\item {Proveniência:(Do gr. \textunderscore opsos\textunderscore  + \textunderscore phagein\textunderscore )}
\end{itemize}
Aquele que é gastrónomo, que é amigo de bôas iguarias.
\section{Opsologia}
\begin{itemize}
\item {Grp. gram.:f.}
\end{itemize}
\begin{itemize}
\item {Proveniência:(Do gr. \textunderscore opsos\textunderscore  + \textunderscore logos\textunderscore )}
\end{itemize}
Tratado de arte culinária.
\section{Opsomania}
\begin{itemize}
\item {Grp. gram.:f.}
\end{itemize}
\begin{itemize}
\item {Utilização:Med.}
\end{itemize}
\begin{itemize}
\item {Proveniência:(Do gr. \textunderscore opsos\textunderscore  + \textunderscore mania\textunderscore )}
\end{itemize}
Gôsto exclusivo por uma espécie de alimento.
\section{Opsomaníaco}
\begin{itemize}
\item {Grp. gram.:adj.}
\end{itemize}
\begin{itemize}
\item {Grp. gram.:M.}
\end{itemize}
Relativo á opsomania.
O mesmo que \textunderscore opsómano\textunderscore .
\section{Opsómano}
\begin{itemize}
\item {Grp. gram.:m.}
\end{itemize}
Aquelle que tem opsomania.
\section{Opsophagia}
\begin{itemize}
\item {Grp. gram.:f.}
\end{itemize}
Qualidade de opsóphago.
\section{Opsóphago}
\begin{itemize}
\item {Grp. gram.:m.  e  adj.}
\end{itemize}
\begin{itemize}
\item {Proveniência:(Do gr. \textunderscore opsos\textunderscore  + \textunderscore phagein\textunderscore )}
\end{itemize}
Aquelle que é gastrónomo, que é amigo de bôas iguarias.
\section{Optação}
\begin{itemize}
\item {Grp. gram.:f.}
\end{itemize}
\begin{itemize}
\item {Proveniência:(Lat. \textunderscore optatio\textunderscore )}
\end{itemize}
Acto ou faculdade de optar, opção.
Expressão exclamativa de um voto ou desejo.
\section{Optar}
\begin{itemize}
\item {Grp. gram.:v. i.}
\end{itemize}
\begin{itemize}
\item {Grp. gram.:V. t.}
\end{itemize}
\begin{itemize}
\item {Proveniência:(Lat. \textunderscore optare\textunderscore )}
\end{itemize}
Determinar-se pela escolha de alguma coisa, entre outras; dar preferência.
Escolher, preferir. Cf. Castilho, \textunderscore Misanthropo\textunderscore , 168.
\section{Optativamente}
\begin{itemize}
\item {Grp. gram.:adv.}
\end{itemize}
De modo optativo; com preferência.
\section{Optativo}
\begin{itemize}
\item {Grp. gram.:adj.}
\end{itemize}
\begin{itemize}
\item {Utilização:Gram.}
\end{itemize}
\begin{itemize}
\item {Proveniência:(Lat. \textunderscore optativus\textunderscore )}
\end{itemize}
Que envolve ou exprime desejo.
Diz-se do modo verbal, cujas fórmas se expressam pelo subjunctivo commum em proposições independentes, para exprimir num facto positivo ou negativo: \textunderscore Praza a Deus que sejas feliz! Não abandones o teu plano\textunderscore .
\section{Óptica}
\begin{itemize}
\item {Grp. gram.:f.}
\end{itemize}
\begin{itemize}
\item {Utilização:Fig.}
\end{itemize}
\begin{itemize}
\item {Proveniência:(Lat. \textunderscore optica\textunderscore )}
\end{itemize}
Tratado da luz e das leis que presidem á visão.
Perspectiva.
\section{Opticamente}
\begin{itemize}
\item {Grp. gram.:adv.}
\end{itemize}
De modo óptico; com os carateres ópticos.
\section{Opticidade}
\begin{itemize}
\item {Grp. gram.:f.}
\end{itemize}
Qualidade de óptico.
\section{Opticista}
\begin{itemize}
\item {Grp. gram.:m.}
\end{itemize}
Aquelle que se occupa da óptica ou que nella é versado.
\section{Opugnação}
\begin{itemize}
\item {Grp. gram.:f.}
\end{itemize}
\begin{itemize}
\item {Proveniência:(Lat. \textunderscore oppugnatio\textunderscore )}
\end{itemize}
Acto ou efeito de opugnar.
Assalto; ataque.
\section{Opugnador}
\begin{itemize}
\item {Grp. gram.:m.  e  adj.}
\end{itemize}
\begin{itemize}
\item {Proveniência:(Lat. \textunderscore oppugnator\textunderscore )}
\end{itemize}
O que opugna.
\section{Opugnar}
\begin{itemize}
\item {Grp. gram.:v. t.}
\end{itemize}
\begin{itemize}
\item {Utilização:Fig.}
\end{itemize}
\begin{itemize}
\item {Proveniência:(Lat. \textunderscore oppugnare\textunderscore )}
\end{itemize}
Pugnar contra; atacar.
Refutar, rejeitar.
\section{Óptico}
\begin{itemize}
\item {Grp. gram.:adj.}
\end{itemize}
\begin{itemize}
\item {Utilização:Anat.}
\end{itemize}
\begin{itemize}
\item {Utilização:Mathem.}
\end{itemize}
\begin{itemize}
\item {Utilização:Phýs.}
\end{itemize}
\begin{itemize}
\item {Grp. gram.:M.}
\end{itemize}
\begin{itemize}
\item {Proveniência:(Gr. \textunderscore optikos\textunderscore )}
\end{itemize}
Relativo á visão.
Concernente á óptica.
Diz-se do nervo encephálico, que se reparte em dois, cada um dos quaes se dirige ao globo ocular e o atravessa, formando a retina.
Diz-se do ângulo, que tem o vértice no ôlho do observador e cujos lados passam pelas extremidades de qualquer das dimensões do objecto observado.
Diz-se do eixo, que passa pelo centro da pupilla e do ôlho.
Diz-se do poder ou boas condições, que um apparelho tem, para por elle se observar distintamente um objecto e as suas particularidades.
Aquelle que é versado em óptica.
Fabricante de instrumentos de óptica.
\section{Opticografia}
\begin{itemize}
\item {Grp. gram.:f.}
\end{itemize}
Estudo ou trabalho metódico, á cêrca da óptica.
(Cp. \textunderscore opticógrafo\textunderscore )
\section{Ópticográfico}
\begin{itemize}
\item {Grp. gram.:adj.}
\end{itemize}
Relativo á \textunderscore opticografia\textunderscore .
\section{Opticógrafo}
\begin{itemize}
\item {Grp. gram.:m.}
\end{itemize}
\begin{itemize}
\item {Proveniência:(Do gr. \textunderscore optikos\textunderscore  + \textunderscore graphein\textunderscore )}
\end{itemize}
Aquele que é versado em opticografia.
\section{Opticographia}
\begin{itemize}
\item {Grp. gram.:f.}
\end{itemize}
Estudo ou trabalho methódico, á cêrca da óptica.
(Cp. \textunderscore opticógrapho\textunderscore )
\section{Ópticográphico}
\begin{itemize}
\item {Grp. gram.:adj.}
\end{itemize}
Relativo á \textunderscore opticographia\textunderscore .
\section{Opticógrapho}
\begin{itemize}
\item {Grp. gram.:m.}
\end{itemize}
\begin{itemize}
\item {Proveniência:(Do gr. \textunderscore optikos\textunderscore  + \textunderscore graphein\textunderscore )}
\end{itemize}
Aquelle que é versado em opticographia.
\section{Opticometria}
\begin{itemize}
\item {Grp. gram.:f.}
\end{itemize}
Applicação do opticómetro.
\section{Opticométrico}
\begin{itemize}
\item {Grp. gram.:adj.}
\end{itemize}
Relativo á opticometria.
\section{Opticómetro}
\begin{itemize}
\item {Grp. gram.:m.}
\end{itemize}
\begin{itemize}
\item {Proveniência:(Do gr. \textunderscore optikos\textunderscore  + \textunderscore metron\textunderscore )}
\end{itemize}
Instrumento, com que se mede o grau de intensão da vista de cada indivíduo, para que bem se escolham os vidros para os óculos.
\section{Optimacia}
\begin{itemize}
\item {Grp. gram.:f.}
\end{itemize}
Conjunto ou reunião de optimates.
Os optimates.
A aristocracia.
(Cp. \textunderscore optimates\textunderscore )
\section{Optimamente}
\begin{itemize}
\item {Grp. gram.:adv.}
\end{itemize}
De modo óptimo; excellentemente.
\section{Optimates}
\begin{itemize}
\item {Grp. gram.:m. pl.}
\end{itemize}
\begin{itemize}
\item {Utilização:Fig.}
\end{itemize}
\begin{itemize}
\item {Proveniência:(Lat. \textunderscore optimates\textunderscore )}
\end{itemize}
Magnates; grandes de uma nação.
\section{Optimismo}
\begin{itemize}
\item {Grp. gram.:m.}
\end{itemize}
\begin{itemize}
\item {Utilização:Ext.}
\end{itemize}
\begin{itemize}
\item {Proveniência:(De \textunderscore óptimo\textunderscore )}
\end{itemize}
Systema philosóphico dos que sustentam sêr o mundo o melhor dos mundos possíveis, ou que Deus fez as coisas, segundo a perfeição das suas ideias.
Philosophia ou systema dos que têm fé no progresso moral e material da humanidade, na melhoria das condições actuaes, na evolução social para o bem e para óptimo.
Tendência para achar tudo bem, mormente em política.
\section{Optimista}
\begin{itemize}
\item {Grp. gram.:adj.}
\end{itemize}
\begin{itemize}
\item {Grp. gram.:M.  e  f.}
\end{itemize}
\begin{itemize}
\item {Proveniência:(De \textunderscore óptimo\textunderscore )}
\end{itemize}
Relativo ao optimismo.
Partidário do optimismo.
Pessôa partidária do optimismo.
Pessôa, que tudo acha bom ou que, segundo a expressão familiar, vê tudo côr de rosa.
\section{Óptimo}
\begin{itemize}
\item {Grp. gram.:adj.}
\end{itemize}
\begin{itemize}
\item {Proveniência:(Lat. \textunderscore optimus\textunderscore )}
\end{itemize}
Muito bom; excellente; magnífico.
\section{Optómetro}
\begin{itemize}
\item {Grp. gram.:m.}
\end{itemize}
\begin{itemize}
\item {Proveniência:(Do gr. \textunderscore optesthai\textunderscore  + \textunderscore metron\textunderscore )}
\end{itemize}
Instrumento, para determinar os limites da visão distinta e o grau de astygmatismo dos olhos.
\section{Opulência}
\begin{itemize}
\item {Grp. gram.:f.}
\end{itemize}
\begin{itemize}
\item {Utilização:Fig.}
\end{itemize}
\begin{itemize}
\item {Proveniência:(Lat. \textunderscore opulentia\textunderscore )}
\end{itemize}
Qualidade de opulento.
Abundância de riqueza.
Riquezas.
Abundância.
Magnificência.
Classe das pessôas opulentas.
Corpulência, grande desenvolvimento de fórmas.
\section{Opulentamente}
\begin{itemize}
\item {Grp. gram.:adv.}
\end{itemize}
De modo opulento; com opulência.
\section{Opulentar}
\begin{itemize}
\item {Grp. gram.:v. t.}
\end{itemize}
\begin{itemize}
\item {Proveniência:(Lat. \textunderscore opulentare\textunderscore )}
\end{itemize}
Tornar opulento; engrandecêr.
\section{Opulento}
\begin{itemize}
\item {Grp. gram.:adj.}
\end{itemize}
\begin{itemize}
\item {Utilização:Fig.}
\end{itemize}
\begin{itemize}
\item {Proveniência:(Lat. \textunderscore opulentus\textunderscore )}
\end{itemize}
Que tem opulência.
Copioso.
Magnífico; bello.
Composto.
Grande; muito desenvolvido: \textunderscore cabelleira opulenta\textunderscore .
\section{Opumbulume}
\begin{itemize}
\item {Grp. gram.:m.}
\end{itemize}
Fruto da África central.
\section{Opúncia}
\begin{itemize}
\item {Grp. gram.:f.}
\end{itemize}
Planta cactácea, (\textunderscore cactus opuntia\textunderscore ).
\section{Opunciáceas}
\begin{itemize}
\item {Grp. gram.:f. pl.}
\end{itemize}
Ordem de plantas, a que pertence a opúncia.
\section{Opúsculo}
\begin{itemize}
\item {Grp. gram.:m.}
\end{itemize}
\begin{itemize}
\item {Proveniência:(Lat. \textunderscore opusculum\textunderscore )}
\end{itemize}
Pequena obra, sôbre artes, sciências, etc.
Folheto.
\section{Oqueá}
\begin{itemize}
\item {Grp. gram.:f.}
\end{itemize}
Antiga moéda da Índia portuguesa e da Abyssínia.--Antigamente, também se escreveu \textunderscore oquia\textunderscore . Cf. Castanhoso, \textunderscore Christóvão da Gama\textunderscore .
\section{Oquim}
\begin{itemize}
\item {Grp. gram.:m.}
\end{itemize}
\begin{itemize}
\item {Utilização:T. de Angola}
\end{itemize}
Mammífero comestível, que vive debaixo da terra e se alimenta das raízes das árvores.
\section{Oquizópode}
\begin{itemize}
\item {Grp. gram.:m.}
\end{itemize}
Gênero de crustáceos decápodes.
\section{Ora}
\begin{itemize}
\item {Grp. gram.:conj.}
\end{itemize}
\begin{itemize}
\item {Grp. gram.:Adv. conj.}
\end{itemize}
\begin{itemize}
\item {Grp. gram.:Adv.}
\end{itemize}
\begin{itemize}
\item {Grp. gram.:Interj.}
\end{itemize}
\begin{itemize}
\item {Proveniência:(Lat. \textunderscore hora\textunderscore )}
\end{itemize}
(que se repete no princípio de várias phrases, ligando-as)
Umas vezes..., outras vezes; não só..., mas também.
Mas, além disso.
Agora, presentemente.
(designativa de \textunderscore dúvida\textunderscore  ou \textunderscore menosprêzo\textunderscore )
\section{Ora}
\begin{itemize}
\item {Grp. gram.:m.}
\end{itemize}
Medida grega de comprimento.
\section{Orabalão}
\begin{itemize}
\item {Grp. gram.:m.}
\end{itemize}
O mesmo que \textunderscore orobalão\textunderscore . Cf. \textunderscore Peregrinação\textunderscore , CC, v.^o 1.
\section{Oração}
\begin{itemize}
\item {Grp. gram.:f.}
\end{itemize}
\begin{itemize}
\item {Proveniência:(Do lat. \textunderscore oratio\textunderscore )}
\end{itemize}
Locução, ou expressão verbal de um pensamento ou de um juízo.
Proposição.
Discurso, para sêr pronunciado em público.
Sermão.
Prece ou súpplica, dirigida a Deus ou aos santos.
\section{Oracional}
\begin{itemize}
\item {Grp. gram.:adj.}
\end{itemize}
\begin{itemize}
\item {Utilização:Gram.}
\end{itemize}
\begin{itemize}
\item {Proveniência:(Do lat. \textunderscore oratio\textunderscore )}
\end{itemize}
Relativo a oração, ou a proposição.
\section{Oraçoeiro}
\begin{itemize}
\item {Grp. gram.:m.}
\end{itemize}
\begin{itemize}
\item {Utilização:Ant.}
\end{itemize}
Livro de orações.
\section{Oracular}
\begin{itemize}
\item {Grp. gram.:adj.}
\end{itemize}
Relativo a oráculo, ou que é próprio delle.
\section{Oracular}
\begin{itemize}
\item {Grp. gram.:v. i.}
\end{itemize}
Falar como oráculo. Cf. \textunderscore Fenix Renasc.\textunderscore , 68.
\section{Oracularmente}
\begin{itemize}
\item {Grp. gram.:adv.}
\end{itemize}
Á maneira de oráculo:«\textunderscore saltam para a imprensa a discretear oracularmente.\textunderscore »Castilho, \textunderscore Montalverne\textunderscore .
\section{Oraculino}
\begin{itemize}
\item {Grp. gram.:adj.}
\end{itemize}
Próprio de oráculo.
\section{Oraculizado}
\begin{itemize}
\item {Grp. gram.:adj.}
\end{itemize}
\begin{itemize}
\item {Utilização:P. us.}
\end{itemize}
Emittido ou pronunciado por oráculo, (falando-se de som ou voz). Cf. Filinto, VIII, 44.
\section{Oraculizante}
\begin{itemize}
\item {Grp. gram.:adj.}
\end{itemize}
Que oraculiza.
\section{Oraculizar}
\begin{itemize}
\item {Grp. gram.:v. i.}
\end{itemize}
O mesmo que \textunderscore oracular\textunderscore ^2.
\section{Oráculo}
\begin{itemize}
\item {Grp. gram.:m.}
\end{itemize}
\begin{itemize}
\item {Utilização:Fig.}
\end{itemize}
\begin{itemize}
\item {Utilização:Des.}
\end{itemize}
\begin{itemize}
\item {Proveniência:(Lat. \textunderscore oraculum\textunderscore )}
\end{itemize}
Resposta dos deuses a quem os consultava.
Divindade, que respondia aos que a consultavam.
Sentença ou decisão infallível.
Pessôa, cujas palavras inspiram confiança absoluta.
Palavras de grande autoridade.
Oratório; capella, pequena igreja.
\section{Orada}
\begin{itemize}
\item {Grp. gram.:f.}
\end{itemize}
\begin{itemize}
\item {Utilização:Pop.}
\end{itemize}
\begin{itemize}
\item {Proveniência:(De \textunderscore orar\textunderscore )}
\end{itemize}
Lugar, em que se ora ou reza.
Ermida, capella fóra do povoado.
\section{Orador}
\begin{itemize}
\item {Grp. gram.:m.}
\end{itemize}
\begin{itemize}
\item {Proveniência:(Lat. \textunderscore orator\textunderscore )}
\end{itemize}
Aquelle que sabe fazer discursos.
Aquelle que é eloquente.
Aquelle que está discursando.
\section{Orago}
\begin{itemize}
\item {Grp. gram.:m.}
\end{itemize}
\begin{itemize}
\item {Proveniência:(Do lat. \textunderscore oraculum\textunderscore )}
\end{itemize}
Santo, a que é dedicado um templo ou capella.
Invocação.
Oráculo.
\section{Orágono}
\begin{itemize}
\item {Grp. gram.:m.}
\end{itemize}
(Fórma ant. de \textunderscore oráculo\textunderscore . Cf. \textunderscore Eufrosina\textunderscore , 8)
\section{Oral}
\begin{itemize}
\item {Grp. gram.:adj.}
\end{itemize}
\begin{itemize}
\item {Grp. gram.:M.}
\end{itemize}
\begin{itemize}
\item {Utilização:Ant.}
\end{itemize}
\begin{itemize}
\item {Proveniência:(Lat. \textunderscore oralis\textunderscore )}
\end{itemize}
Relativo á boca.
Que é articulado ou pronunciado, (em opposição a \textunderscore escrito\textunderscore ): \textunderscore linguagem oral\textunderscore .
Véu fino, com que as senhoras mais honestas ou recatadas velavam o rosto.
\section{...orama}
\begin{itemize}
\item {Grp. gram.:suf. m.}
\end{itemize}
\begin{itemize}
\item {Proveniência:(Do gr. \textunderscore orama\textunderscore )}
\end{itemize}
(Designativo de \textunderscore espectáculo\textunderscore  ou \textunderscore vista\textunderscore )
\section{Orangistas}
\begin{itemize}
\item {Grp. gram.:m. pl.}
\end{itemize}
Os naturaes de Orange, na África do Sul.
\section{Orangotango}
\begin{itemize}
\item {Grp. gram.:m.}
\end{itemize}
Grande macaco anthropomorpho.
(Do malaio \textunderscore óran\textunderscore  + \textunderscore utan\textunderscore , homem dos bosques)
\section{Orar}
\begin{itemize}
\item {Grp. gram.:v. i.}
\end{itemize}
\begin{itemize}
\item {Grp. gram.:V. t.}
\end{itemize}
\begin{itemize}
\item {Proveniência:(Lat. \textunderscore orare\textunderscore )}
\end{itemize}
Pronunciar um discurso.
Falar em público, declamar.
Dirigir súpplicas a Deus ou aos santos.
Fazer oração, rezar.
Pedir, supplicar.
\section{Orária}
\begin{itemize}
\item {Grp. gram.:f.  e  adj.}
\end{itemize}
\begin{itemize}
\item {Proveniência:(Do lat. \textunderscore orarius\textunderscore )}
\end{itemize}
Dizia-se, entre os antigos, da embarcação que só navega junto da costa.
\section{Orário}
\begin{itemize}
\item {Grp. gram.:m.}
\end{itemize}
\begin{itemize}
\item {Proveniência:(Lat. \textunderscore orarium\textunderscore )}
\end{itemize}
Espécie de lenço, usado pelos Romanos para limpar a boca e o suor do rosto.
\section{Ora-sus!}
\begin{itemize}
\item {Grp. gram.:interj.}
\end{itemize}
O mesmo que \textunderscore sus!\textunderscore 
\section{Orate}
\begin{itemize}
\item {Grp. gram.:m.}
\end{itemize}
O mesmo que \textunderscore louco\textunderscore ^1; indivíduo pouco sensato; idiota.
(Cast. \textunderscore orate\textunderscore , do gr. \textunderscore orates\textunderscore , visionáro)
\section{Oratória}
\begin{itemize}
\item {Grp. gram.:f.}
\end{itemize}
\begin{itemize}
\item {Proveniência:(De \textunderscore oratório\textunderscore )}
\end{itemize}
Arte de discorrer ou falar em público.
Peça dramática, baseada na vida de um santo, ou em factos da História Sagrada.
\section{Oratoriamente}
\begin{itemize}
\item {Grp. gram.:adv.}
\end{itemize}
De modo oratório; com modos de orador; á maneira de discurso.
\section{Oratoriano}
\begin{itemize}
\item {Grp. gram.:m.  e  adj.}
\end{itemize}
Membro da congregação do Oratório.
\section{Oratório}
\begin{itemize}
\item {Grp. gram.:adj.}
\end{itemize}
\begin{itemize}
\item {Grp. gram.:M.}
\end{itemize}
\begin{itemize}
\item {Proveniência:(Lat. \textunderscore oratorius\textunderscore )}
\end{itemize}
Relativo á oratória ou ao orador.
Nicho ou armário, que contém imagens de santos.
Peça dramática, oratória.
Nome de uma antiga congregação religiosa.
Casa, onde habitavam os membros dessa congregação.
Lugar, onde os condemnados á morte faziam oração antes do supplício.
\section{Orbe}
\begin{itemize}
\item {Grp. gram.:m.}
\end{itemize}
\begin{itemize}
\item {Proveniência:(Lat. \textunderscore orbis\textunderscore )}
\end{itemize}
Esphera, globo.
Redondeza.
Mundo.
Qualquer corpo celeste.
\section{Orbícola}
\begin{itemize}
\item {Grp. gram.:adj.}
\end{itemize}
\begin{itemize}
\item {Proveniência:(Do lat. \textunderscore orbis\textunderscore  + \textunderscore colere\textunderscore )}
\end{itemize}
Que viaja por toda parte.
Cosmopolita; orbívago.
\section{Orbícula}
\begin{itemize}
\item {Grp. gram.:f.}
\end{itemize}
\begin{itemize}
\item {Proveniência:(Do lat. \textunderscore orbiculus\textunderscore )}
\end{itemize}
Mollusco acéphalo.
\section{Orbicular}
\begin{itemize}
\item {Grp. gram.:adj.}
\end{itemize}
\begin{itemize}
\item {Grp. gram.:M.}
\end{itemize}
\begin{itemize}
\item {Proveniência:(Lat. \textunderscore orbicularis\textunderscore )}
\end{itemize}
Que tem forma de orbe; globular; circular.
Que contorna alguma coisa.
Músculo orbicular.
\section{Orbicularmente}
\begin{itemize}
\item {Grp. gram.:adv.}
\end{itemize}
De modo orbicular.
\section{Orbículo}
\begin{itemize}
\item {Grp. gram.:m.}
\end{itemize}
\begin{itemize}
\item {Proveniência:(Lat. \textunderscore orbiculos\textunderscore )}
\end{itemize}
Receptáculo orbicular, ou espécie de bôlsa, que cérca os órgãos da fructificação de algumas plantas.
\section{Orbilha}
\begin{itemize}
\item {Grp. gram.:f.}
\end{itemize}
\begin{itemize}
\item {Utilização:Bot.}
\end{itemize}
\begin{itemize}
\item {Proveniência:(De \textunderscore orbe\textunderscore )}
\end{itemize}
Espécie de cúpula orbicular dos líchens.
\section{Órbita}
\begin{itemize}
\item {Grp. gram.:f.}
\end{itemize}
\begin{itemize}
\item {Utilização:Fig.}
\end{itemize}
\begin{itemize}
\item {Proveniência:(Lat. \textunderscore orbita\textunderscore )}
\end{itemize}
Caminho, que um corpo celeste parece percorrer.
Esphera de acção.
Cavidade óssea, em que está o globo do ôlho.
Contôrno do ôlho das aves.
\section{Orbitário}
\begin{itemize}
\item {Grp. gram.:adj.}
\end{itemize}
Relativo á órbita do ôlho.
\section{Orbitelo}
\begin{itemize}
\item {Grp. gram.:adj.}
\end{itemize}
\begin{itemize}
\item {Proveniência:(De \textunderscore órbita\textunderscore )}
\end{itemize}
Diz-se de vários insectos, que formam teias, compostas de círculos concêntricos.
\section{Orbivácuo}
\begin{itemize}
\item {Grp. gram.:adj.}
\end{itemize}
\begin{itemize}
\item {Utilização:Des.}
\end{itemize}
O mesmo que \textunderscore orbívago\textunderscore .
\section{Orbívago}
\begin{itemize}
\item {Grp. gram.:adj.}
\end{itemize}
\begin{itemize}
\item {Utilização:Poét.}
\end{itemize}
\begin{itemize}
\item {Proveniência:(Do lat. \textunderscore orbis\textunderscore  + \textunderscore vagagre\textunderscore )}
\end{itemize}
Que vagueia pelo orbe; orbícola.
\section{Orbulita}
\begin{itemize}
\item {Grp. gram.:f.}
\end{itemize}
Gênero de conchas fósseis.
\section{Orca}
\begin{itemize}
\item {Grp. gram.:f.}
\end{itemize}
\begin{itemize}
\item {Proveniência:(Lat. \textunderscore orca\textunderscore )}
\end{itemize}
Mammífero cetáceo.
Vaso de barro, do feitio da âmphora, mas mais pequeno.
\section{Orca}
\begin{itemize}
\item {Grp. gram.:f.}
\end{itemize}
\begin{itemize}
\item {Utilização:Prov.}
\end{itemize}
O mesmo que \textunderscore anta\textunderscore ^2.
\section{Orça}
\begin{itemize}
\item {Grp. gram.:f.}
\end{itemize}
\begin{itemize}
\item {Grp. gram.:Loc. adv.}
\end{itemize}
\begin{itemize}
\item {Proveniência:(De \textunderscore orçar\textunderscore )}
\end{itemize}
O mesmo que \textunderscore bolina\textunderscore .
Acto de orçar ou calcular.
\textunderscore Á orça\textunderscore , calculado a ôlho, por junto, aproximadamente.
\section{Orça}
\begin{itemize}
\item {Grp. gram.:f.}
\end{itemize}
\begin{itemize}
\item {Utilização:Prov.}
\end{itemize}
\begin{itemize}
\item {Utilização:trasm.}
\end{itemize}
(V.horsa)
Mulher alta e desajeitada.
\section{Orçador}
\begin{itemize}
\item {Grp. gram.:m.  e  adj.}
\end{itemize}
O que orça.
\section{Orçamental}
\begin{itemize}
\item {Grp. gram.:adj.}
\end{itemize}
Relativo a orçamento.
\section{Orçamentário}
\begin{itemize}
\item {Grp. gram.:adj.}
\end{itemize}
O mesmo que \textunderscore orçamental\textunderscore .
\section{Orçamento}
\begin{itemize}
\item {Grp. gram.:m.}
\end{itemize}
\begin{itemize}
\item {Proveniência:(De \textunderscore orçar\textunderscore )}
\end{itemize}
Acto ou effeito de orçar.
Cálculo de receita e despesa.
Cálculo ou apreciação do que é preciso para se realizar qualquer obra ou empresa.
\section{Orçamentologia}
\begin{itemize}
\item {Grp. gram.:f.}
\end{itemize}
\begin{itemize}
\item {Proveniência:(De \textunderscore orçamento\textunderscore  + gr. \textunderscore logos\textunderscore )}
\end{itemize}
Arte de organizar orçamentos.
\section{Orçamentologista}
\begin{itemize}
\item {Grp. gram.:m.}
\end{itemize}
Aquelle que se occupa de orçamentologia.
\section{Orçamentólogo}
\begin{itemize}
\item {Grp. gram.:m.}
\end{itemize}
Aquelle que é perito em orçamentologia.
\section{Orçaneta}
\begin{itemize}
\item {Grp. gram.:f.}
\end{itemize}
Planta borragínea, (\textunderscore anchusa tinctoria\textunderscore ).
\section{Orçar}
\begin{itemize}
\item {Grp. gram.:v. t.}
\end{itemize}
\begin{itemize}
\item {Grp. gram.:V. i.}
\end{itemize}
\begin{itemize}
\item {Utilização:Náut.}
\end{itemize}
\begin{itemize}
\item {Utilização:Fig.}
\end{itemize}
\begin{itemize}
\item {Proveniência:(It. \textunderscore orzare\textunderscore )}
\end{itemize}
Calcular, computar.
Designar previamente o que se tem a despender com.
Designar aproximadamente (uma despesa futura).
Ir á orça, ou á bolina, tomar a direcção do vento.
Aproximar-se.
Estar quási a tocar um ponto ou alguma coisa.
\section{Orçaz}
\begin{itemize}
\item {Grp. gram.:m.}
\end{itemize}
Parte inferior de uma rêde de pesca.
\section{Orcela}
\begin{itemize}
\item {Grp. gram.:f.}
\end{itemize}
Musgo tinctorial das Canárias e de Cabo-Verde, empregado pelos antigos em purpurear os tecidos.
(Metáth. de \textunderscore rocella\textunderscore , de \textunderscore roca\textunderscore ^2)
\section{Orcela}
\begin{itemize}
\item {Grp. gram.:f.}
\end{itemize}
\begin{itemize}
\item {Utilização:T. da Bairrada}
\end{itemize}
\begin{itemize}
\item {Grp. gram.:f.}
\end{itemize}
\begin{itemize}
\item {Utilização:Prov.}
\end{itemize}
Cada uma das peças de madeira, que se erguem sôbre um dos lados do lagar de vinho e entre as quaes há uma travessa que serve de eixo á vara do lagar.
Cada uma das peças parallelas que sustentam uma haste de ferro ou madeira, que serve de eixo á vara do lugar no seu movimento ascendente ou descendente.
(Relaciona-se com \textunderscore orçar\textunderscore ?)
\section{Orchata}
\begin{itemize}
\item {Grp. gram.:f.}
\end{itemize}
Emulsão refrigerante, feita de pevides descascadas de cucurbitáceas, pisadas e preparadas com açúcar.
Bebida, preparada por uma decocção de cevada com amêndoas doces, pisadas.
(Cast. \textunderscore orchata\textunderscore )
\section{Orchestra}
\begin{itemize}
\item {fónica:qués}
\end{itemize}
\begin{itemize}
\item {Grp. gram.:f.}
\end{itemize}
\begin{itemize}
\item {Utilização:Poét.}
\end{itemize}
\begin{itemize}
\item {Proveniência:(Lat. \textunderscore orchestra\textunderscore )}
\end{itemize}
Lugar, occupado pelos músicos instrumentistas, num theatro ou numa festa qualquer.
Conjunto de músicos instrumentistas, que executam as peças de música ou acompanham o canto.
Parte instrumental de uma partitura.
Conjunto de sons harmoniosos.
\section{Orchestração}
\begin{itemize}
\item {fónica:qués}
\end{itemize}
\begin{itemize}
\item {Grp. gram.:f.}
\end{itemize}
Acto ou arte de orchestrar.
\section{Orchestral}
\begin{itemize}
\item {fónica:ques}
\end{itemize}
\begin{itemize}
\item {Grp. gram.:adj.}
\end{itemize}
Relativo a orchestra.
\section{Orchestrar}
\begin{itemize}
\item {fónica:ques}
\end{itemize}
\begin{itemize}
\item {Grp. gram.:v. t.}
\end{itemize}
Dispor ou organizar (peça musical), para sêr executada por orchestra.
\section{Orchestrino}
\begin{itemize}
\item {fónica:ques}
\end{itemize}
\begin{itemize}
\item {Grp. gram.:m.}
\end{itemize}
\begin{itemize}
\item {Proveniência:(De \textunderscore orchestra\textunderscore )}
\end{itemize}
Piano que imitava a rabeca, a viola e o violoncello.
\section{Orchialgia}
\begin{itemize}
\item {fónica:qui}
\end{itemize}
\begin{itemize}
\item {Grp. gram.:f.}
\end{itemize}
\begin{itemize}
\item {Utilização:Med.}
\end{itemize}
\begin{itemize}
\item {Proveniência:(Do gr. \textunderscore orkhis\textunderscore  + \textunderscore algos\textunderscore )}
\end{itemize}
Neuralgia do testículo.
\section{Órchide}
\begin{itemize}
\item {fónica:qui}
\end{itemize}
\begin{itemize}
\item {Grp. gram.:f.}
\end{itemize}
\begin{itemize}
\item {Proveniência:(Do gr. \textunderscore orkhis\textunderscore )}
\end{itemize}
Gênero de plantas, que serve de typo ás orchídeas.
\section{Orchidáceas}
\begin{itemize}
\item {fónica:qui}
\end{itemize}
\begin{itemize}
\item {Grp. gram.:f. pl.}
\end{itemize}
Família de plantas, o mesmo ou melhor que \textunderscore orchídeas\textunderscore .
\section{Orchideáceo}
\begin{itemize}
\item {fónica:qui}
\end{itemize}
\begin{itemize}
\item {Grp. gram.:adj.}
\end{itemize}
\begin{itemize}
\item {Utilização:Bot.}
\end{itemize}
\begin{itemize}
\item {Proveniência:(De \textunderscore órchide\textunderscore )}
\end{itemize}
Diz-se das raizes formadas de dois tubérculos collados como as das orchídeas.
\section{Orchídeas}
\begin{itemize}
\item {fónica:qui}
\end{itemize}
\begin{itemize}
\item {Grp. gram.:f. pl.}
\end{itemize}
\begin{itemize}
\item {Proveniência:(Do gr. \textunderscore orkhis\textunderscore  + \textunderscore eidos\textunderscore )}
\end{itemize}
Família de plantas monocotyledóneas e tuberculosas.
\section{Orchidifloro}
\begin{itemize}
\item {fónica:qui}
\end{itemize}
\begin{itemize}
\item {Grp. gram.:adj.}
\end{itemize}
\begin{itemize}
\item {Utilização:Bot.}
\end{itemize}
Que dá flôres semelhantes ás das orchídeas.
\section{Orchidóphilo}
\begin{itemize}
\item {fónica:qui}
\end{itemize}
\begin{itemize}
\item {Grp. gram.:m.  e  adj.}
\end{itemize}
Amador ou colleccionador de orchídeas.
\section{Orchiocele}
\begin{itemize}
\item {fónica:qui}
\end{itemize}
\begin{itemize}
\item {Grp. gram.:m.}
\end{itemize}
\begin{itemize}
\item {Proveniência:(Do gr. \textunderscore orkhis\textunderscore  + \textunderscore kele\textunderscore )}
\end{itemize}
Tumor no testículo.
\section{Orchiotomia}
\begin{itemize}
\item {fónica:qui}
\end{itemize}
\begin{itemize}
\item {Grp. gram.:f.}
\end{itemize}
Extracção cirúrgica de um ou dos dois testículos.
(Cp. \textunderscore orchiótomo\textunderscore )
\section{Orchiotómico}
\begin{itemize}
\item {fónica:qui}
\end{itemize}
\begin{itemize}
\item {Grp. gram.:adj.}
\end{itemize}
Relativo á orchiotomia.
\section{Orchiótomo}
\begin{itemize}
\item {fónica:qui}
\end{itemize}
\begin{itemize}
\item {Grp. gram.:m.}
\end{itemize}
\begin{itemize}
\item {Proveniência:(Do gr. \textunderscore orkhis\textunderscore  + \textunderscore tome\textunderscore )}
\end{itemize}
Instrumento, com que se pratica a orchiotomia.
\section{Orchípeda}
\begin{itemize}
\item {fónica:qui}
\end{itemize}
\begin{itemize}
\item {Grp. gram.:f.}
\end{itemize}
\begin{itemize}
\item {Proveniência:(Do lat. \textunderscore orchis\textunderscore  + \textunderscore pes\textunderscore , \textunderscore pedis\textunderscore )}
\end{itemize}
Gênero de plantas apocýneas.
\section{Orchita}
\begin{itemize}
\item {fónica:qui}
\end{itemize}
\begin{itemize}
\item {Grp. gram.:f.}
\end{itemize}
\begin{itemize}
\item {Proveniência:(Lat. \textunderscore orchita\textunderscore )}
\end{itemize}
Variedade de azeitona, conhecida dos antigos.
\section{Orchite}
\begin{itemize}
\item {fónica:qui}
\end{itemize}
\begin{itemize}
\item {Grp. gram.:f.}
\end{itemize}
\begin{itemize}
\item {Proveniência:(Do gr. \textunderscore orkhis\textunderscore )}
\end{itemize}
Inflammação de um ou dos dois testículos.
\section{Orchítico}
\begin{itemize}
\item {fónica:qui}
\end{itemize}
\begin{itemize}
\item {Grp. gram.:adj.}
\end{itemize}
Relativo á orchite.
Applicável contra a orchite.
\section{Orchotomia}
\begin{itemize}
\item {fónica:co}
\end{itemize}
\begin{itemize}
\item {Grp. gram.:f.}
\end{itemize}
(V.orchiotomia)
\section{Orcina}
\begin{itemize}
\item {Grp. gram.:f.}
\end{itemize}
Substância còrante de uma espécie de líchen, (\textunderscore variolaria orcina\textunderscore ).
\section{Orcino}
\begin{itemize}
\item {Grp. gram.:adj.}
\end{itemize}
\begin{itemize}
\item {Proveniência:(Lat. \textunderscore orcinus\textunderscore )}
\end{itemize}
Dizia-se do escravo, a quem a alforria era concedida em testamento.
\section{Orco}
\begin{itemize}
\item {Grp. gram.:m.}
\end{itemize}
\begin{itemize}
\item {Utilização:Poét.}
\end{itemize}
\begin{itemize}
\item {Proveniência:(Lat. \textunderscore orcus\textunderscore )}
\end{itemize}
Região dos mortos; inferno.
\section{Orcotomia}
\begin{itemize}
\item {Grp. gram.:f.}
\end{itemize}
(V.orquiotomia)
\section{Ordálio}
\begin{itemize}
\item {Grp. gram.:m.}
\end{itemize}
\begin{itemize}
\item {Proveniência:(Do anglo-sax. \textunderscore ordal\textunderscore )}
\end{itemize}
Qualquer prova jurídica, usada na Idade-Média, sob o nome de \textunderscore juizo de Deus\textunderscore .
\section{Ordeirismo}
\begin{itemize}
\item {Grp. gram.:m.}
\end{itemize}
\begin{itemize}
\item {Proveniência:(De \textunderscore ordeiro\textunderscore )}
\end{itemize}
Systema dos amigos da ordem.
\section{Ordeiro}
\begin{itemize}
\item {Grp. gram.:m.  e  adj.}
\end{itemize}
\begin{itemize}
\item {Proveniência:(De \textunderscore ordem\textunderscore )}
\end{itemize}
Indivíduo amigo da ordem.
Conservador; pacífico.
\section{Ordem}
\begin{itemize}
\item {Grp. gram.:f.}
\end{itemize}
\begin{itemize}
\item {Grp. gram.:Loc. adv.}
\end{itemize}
\begin{itemize}
\item {Proveniência:(Do lat. \textunderscore ordo\textunderscore )}
\end{itemize}
Disposição methódica: \textunderscore pôr em ordem os seus livros\textunderscore .
Marcha ou funccionamento regular.
Regularidade.
Apropriada combinação de meios.
Ajuntamento.
Classe, série.
Qualidade de quem é methódico e discreto nos seus actos: \textunderscore proceder com ordem\textunderscore .
Bôa administração.
Praxe, uso, lei.
Mandado: \textunderscore receber uma ordem\textunderscore .
Disciplina.
Maneira.
Modo de sêr.
Sociedade religiosa: \textunderscore a ordem dos Carmelitas\textunderscore .
Confraria.
Espécie de título honorífico para recompensa de méritos.
Insignia correspondente a esse título.
Sacramento, que dá a faculdade de exercer certas funcções ecclesiásticas; (neste sentido é mais us. no pl.): \textunderscore receber ordens sacras\textunderscore .
Cada um dos coros, formados pelos anjos, segundo a Theologia.
Subdivisão immediata de uma classe de seres animaes ou vegetaes.
Cada um dos systemas clássicos de Architectura: \textunderscore a ordem dórica\textunderscore .
Occasião, em que um commandante distribue ordens aos seus subordinados.
Publicação, feita pelo commando de um corpo militar, ou pelo chefe de certos serviços, e contendo instrucções várias, sôbre serviço e movimento do pessoal.
Publicação official de leis, regulamentos, etc., relativos ao exército ou á armada.
Regulamento militar.
\textunderscore Por ordem\textunderscore , successivamente, ordenadamente.
\textunderscore Dizer ordem a\textunderscore , referir-se, dizer respeito.
\section{Ordenação}
\begin{itemize}
\item {Grp. gram.:f.}
\end{itemize}
\begin{itemize}
\item {Proveniência:(Do lat. \textunderscore ordinatio\textunderscore )}
\end{itemize}
Acto ou effeito de ordenar.
Vontade superior.
Ordem.
Collação de ordens ecclesiásticas.
\section{Ordenada}
\begin{itemize}
\item {Grp. gram.:f.}
\end{itemize}
\begin{itemize}
\item {Utilização:Mathem.}
\end{itemize}
Distância, entre um ponto e uma recta ou um plano, medida parallelamente a uma dada direcção.
(Fem. de \textunderscore ordenado\textunderscore )
\section{Ordenadamente}
\begin{itemize}
\item {Grp. gram.:adv.}
\end{itemize}
De modo ordenado.
Por ordem; successivamente.
\section{Ordenado}
\begin{itemize}
\item {Grp. gram.:m.}
\end{itemize}
\begin{itemize}
\item {Proveniência:(De \textunderscore ordenar\textunderscore )}
\end{itemize}
Retribuição de um empregado.
Aquillo que se paga periodicamente por um serviço effectivo, geralmente público ou official.
\section{Ordenador}
\begin{itemize}
\item {Grp. gram.:m.  e  adj.}
\end{itemize}
\begin{itemize}
\item {Proveniência:(Do Lat. \textunderscore ordinator\textunderscore )}
\end{itemize}
O que ordena.
\section{Ordenamento}
\begin{itemize}
\item {Grp. gram.:m.}
\end{itemize}
O mesmo que \textunderscore ordenação\textunderscore .
\section{Ordenança}
\begin{itemize}
\item {Grp. gram.:f.}
\end{itemize}
\begin{itemize}
\item {Utilização:Des.}
\end{itemize}
\begin{itemize}
\item {Utilização:Ant.}
\end{itemize}
\begin{itemize}
\item {Proveniência:(De \textunderscore ordenar\textunderscore )}
\end{itemize}
Regulamento de manobras militares.
Soldado, que está ás ordens de uma repartição ou autoridade.
Ordem.
Tropa; exército.
\section{Ordenar}
\begin{itemize}
\item {Grp. gram.:v. t.}
\end{itemize}
\begin{itemize}
\item {Grp. gram.:V. i.}
\end{itemize}
\begin{itemize}
\item {Utilização:Des.}
\end{itemize}
\begin{itemize}
\item {Proveniência:(Do lat. \textunderscore ordinare\textunderscore )}
\end{itemize}
Pôr por ordem, regular, dispôr.
Mandar, determinar.
Conferir o Sacramento da Ordem a.
Dar ordens.
\section{Ordenável}
\begin{itemize}
\item {Grp. gram.:adj.}
\end{itemize}
\begin{itemize}
\item {Proveniência:(Do lat. \textunderscore ordinabilis\textunderscore )}
\end{itemize}
Que se póde ordenar.
\section{Ordenha}
\begin{itemize}
\item {Grp. gram.:f.}
\end{itemize}
Acto de ordenhar. Cf. Rev. \textunderscore Tradição\textunderscore , XI, 130.
\section{Ordenhador}
\begin{itemize}
\item {Grp. gram.:m.  e  adj.}
\end{itemize}
O que ordenha.
\section{Ordenhar}
\begin{itemize}
\item {Grp. gram.:v. t.}
\end{itemize}
\begin{itemize}
\item {Utilização:T. de Caminha}
\end{itemize}
O mesmo que \textunderscore mungir\textunderscore .
Amanhar (peixe).
\section{Ordinal}
\begin{itemize}
\item {Grp. gram.:adj.}
\end{itemize}
\begin{itemize}
\item {Proveniência:(Lat. \textunderscore ordinalis\textunderscore )}
\end{itemize}
Relativo á ordem dos números: \textunderscore «décimo»é adjectivo numeral ordinal\textunderscore .
\section{Ordinando}
\begin{itemize}
\item {Grp. gram.:m.  e  adj.}
\end{itemize}
\begin{itemize}
\item {Proveniência:(Lat. \textunderscore ordinandus\textunderscore )}
\end{itemize}
Aquelle que está preparado ou se prepara para receber Ordens sacras.
\section{Ordinante}
\begin{itemize}
\item {Grp. gram.:m.  e  adj.}
\end{itemize}
\begin{itemize}
\item {Proveniência:(Lat. \textunderscore ordinans\textunderscore )}
\end{itemize}
O que confere ordens ecclesiásticas.
\section{Ordinariamente}
\begin{itemize}
\item {Grp. gram.:adv.}
\end{itemize}
De modo ordinário; commummente; com frequencia.
\section{Ordinário}
\begin{itemize}
\item {Grp. gram.:adj.}
\end{itemize}
\begin{itemize}
\item {Grp. gram.:M.}
\end{itemize}
\begin{itemize}
\item {Grp. gram.:Loc. adv.}
\end{itemize}
\begin{itemize}
\item {Proveniência:(Lat. \textunderscore ordinaríus\textunderscore )}
\end{itemize}
Conforme á ordem usual.
Usual.
Commum, vulgar, trivial, geral: \textunderscore os trâmites ordinários de um processo\textunderscore .
Regular.
Frequente.
Familiar.
Medíocre; inferior: \textunderscore pano ordinário\textunderscore .
Que tem pouco valor.
Prosaico.
Próprio das classes mais baixas da sociedade: \textunderscore modos ordinários\textunderscore .
Aquillo que acontece ou se faz habitualmente.
Superior ecclesiástico.
Regulamentação escrita do modo de recitar os offícios divinos.
\textunderscore De ordinário\textunderscore , ou \textunderscore pelo ordinário\textunderscore , ordinariamente, geralmente, em regra. Cf. Camillo, \textunderscore O Bem e o Mal\textunderscore , 52.
\section{Ordinatório}
\begin{itemize}
\item {Grp. gram.:adj.}
\end{itemize}
\begin{itemize}
\item {Utilização:bras}
\end{itemize}
\begin{itemize}
\item {Utilização:Jur.}
\end{itemize}
Diz-se das regras judiciaes, que respeitam á instrucção do processo, e ao procedimento das partes e dos juizes. Cf. \textunderscore Jorn.-do-Com.\textunderscore , do Rio, de 17-VI-902.
\section{Ordinhar}
\textunderscore v. t.\textunderscore  (e der.)
Fórma ant. de \textunderscore ordenar\textunderscore , etc.
\section{Ordir}
\textunderscore v. t.\textunderscore  (e der.)
(V. \textunderscore urdir\textunderscore , etc.)
\section{Ordo}
\begin{itemize}
\item {Grp. gram.:m.}
\end{itemize}
\begin{itemize}
\item {Utilização:Ant.}
\end{itemize}
\begin{itemize}
\item {Proveniência:(Do lat. \textunderscore ordeum\textunderscore )}
\end{itemize}
O mesmo que \textunderscore cevada\textunderscore .
\section{Ordume}
\begin{itemize}
\item {Grp. gram.:m.}
\end{itemize}
O mesmo ou melhor que \textunderscore urdume\textunderscore . Cf. Castilho, \textunderscore Fastos\textunderscore , II, 93; \textunderscore Metam.\textunderscore , 188.
\section{Oréada}
\begin{itemize}
\item {Grp. gram.:f.}
\end{itemize}
\begin{itemize}
\item {Proveniência:(Lat. \textunderscore oreas\textunderscore , \textunderscore orealis\textunderscore )}
\end{itemize}
Cada uma das nymphas, que presidiam aos bosques e montes.
\section{Oréade}
\begin{itemize}
\item {Grp. gram.:f.}
\end{itemize}
\begin{itemize}
\item {Utilização:Poét.}
\end{itemize}
\begin{itemize}
\item {Proveniência:(Lat. \textunderscore oreas\textunderscore , \textunderscore orealis\textunderscore )}
\end{itemize}
Cada uma das nymphas, que presidiam aos bosques e montes.
\section{Orear}
\begin{itemize}
\item {Grp. gram.:v. t.}
\end{itemize}
\begin{itemize}
\item {Utilização:Bras. do S}
\end{itemize}
\begin{itemize}
\item {Proveniência:(T. cast.)}
\end{itemize}
Arejar.
Expor ao ar (roupa húmida) para secar.
\section{Orégão}
\begin{itemize}
\item {Grp. gram.:m.}
\end{itemize}
\begin{itemize}
\item {Proveniência:(Do lat. \textunderscore origanum\textunderscore )}
\end{itemize}
Planta labiada.
\section{Oregógeno}
\begin{itemize}
\item {Grp. gram.:adj.}
\end{itemize}
\begin{itemize}
\item {Utilização:Med.}
\end{itemize}
\begin{itemize}
\item {Proveniência:(Do gr. \textunderscore orego\textunderscore  + \textunderscore genos\textunderscore )}
\end{itemize}
Diz-se da funcção de um órgão, que activa o appetite.
\section{Orelana}
\begin{itemize}
\item {Grp. gram.:f.}
\end{itemize}
O mesmo que \textunderscore urucu\textunderscore .
\section{Orelha}
\begin{itemize}
\item {fónica:orê}
\end{itemize}
\begin{itemize}
\item {Grp. gram.:f.}
\end{itemize}
\begin{itemize}
\item {Utilização:Bot.}
\end{itemize}
\begin{itemize}
\item {Utilização:Carp.}
\end{itemize}
\begin{itemize}
\item {Utilização:Bot.}
\end{itemize}
\begin{itemize}
\item {Utilização:Prov.}
\end{itemize}
\begin{itemize}
\item {Utilização:alent.}
\end{itemize}
\begin{itemize}
\item {Utilização:Bras. do S}
\end{itemize}
\begin{itemize}
\item {Grp. gram.:Pl.}
\end{itemize}
\begin{itemize}
\item {Proveniência:(Do b. lat. \textunderscore oricula\textunderscore )}
\end{itemize}
Apparelho ou órgão da audição.
Ouvido:«\textunderscore e vindo ás orelhas do nosso rei...\textunderscore »F. Manuel, \textunderscore Apólogos\textunderscore .
Concha do ouvido.
Hélice do capitel corínthio.
Appêndice, na base das fôlhas de algumas plantas.
Córte ou chanfradura, na extremidade de vigas, escoras, etc, para as ligar a outra peça.
Palavra, que faz parte do nome de várias plantas.
Peça parallelogrâmica de madeira, que assenta sôbre as aivecas do arado.
\textunderscore Pisar na orelha\textunderscore , ficar de pé (o cavalleiro), adeante do cavallo, quando êste cai.
A parte superior e bi-partida do martelo e do sacho.
As aivecas do arado.
\section{Orelhada}
\begin{itemize}
\item {Grp. gram.:f.}
\end{itemize}
Puxão de orelhas; orelhão.
\section{Orelha-de-cabra}
\begin{itemize}
\item {Grp. gram.:f.}
\end{itemize}
Planta plantagínea, (\textunderscore plantago lagopus\textunderscore , Lin.).
\section{Orelha-de-cão}
\begin{itemize}
\item {Grp. gram.:f.}
\end{itemize}
Árvore, africana, de fôlhas compostas, e flôres em fórma de orelha.
\section{Orelha-de-gato}
\begin{itemize}
\item {Grp. gram.:f.}
\end{itemize}
\begin{itemize}
\item {Utilização:Bras}
\end{itemize}
Planta hypericácea, de fôlhas vulnerárias.
\section{Orelha-de-lebre}
\begin{itemize}
\item {Grp. gram.:f.}
\end{itemize}
\begin{itemize}
\item {Utilização:Prov.}
\end{itemize}
\begin{itemize}
\item {Utilização:alg.}
\end{itemize}
Espécie de milho amarelo, cuja espiga deita fôlhas parecidas a orelhas de lebre.
Candelária dos jardins.
\section{Orelha-de-monge}
\begin{itemize}
\item {Grp. gram.:f.}
\end{itemize}
O mesmo que \textunderscore coucelo\textunderscore .
\section{Orelha-de-mula}
\begin{itemize}
\item {Grp. gram.:f.}
\end{itemize}
\begin{itemize}
\item {Utilização:Prov.}
\end{itemize}
\begin{itemize}
\item {Utilização:alg.}
\end{itemize}
\begin{itemize}
\item {Utilização:Náut.}
\end{itemize}
Espécie de milho amarelo, cuja espiga deita umas fôlhas semelhantes a orelha de mula.
Pequena vela triangular, que alguns navios usam por cima do sôbre-joanetinho. Cf. M. C. Campos, \textunderscore Vocab. Marujo\textunderscore .
\section{Orelha-de-onça}
\begin{itemize}
\item {Grp. gram.:f.}
\end{itemize}
\begin{itemize}
\item {Utilização:Bras}
\end{itemize}
Planta menispermácea, medicinal.
\section{Orelha-de-pau}
\begin{itemize}
\item {Grp. gram.:f.}
\end{itemize}
\begin{itemize}
\item {Utilização:Bras}
\end{itemize}
Pequeno cogumelo.
\section{Orelha-de-rato}
\begin{itemize}
\item {Grp. gram.:f.}
\end{itemize}
\begin{itemize}
\item {Utilização:Bot.}
\end{itemize}
O mesmo que \textunderscore myosota\textunderscore .
\section{Orelha-de-toupeira}
\begin{itemize}
\item {Grp. gram.:f.}
\end{itemize}
Espécie de lírio.
\section{Orelha-de-urso}
\begin{itemize}
\item {Grp. gram.:f.}
\end{itemize}
Planta primulácea, (\textunderscore primula auricula\textunderscore , Lin.)-
\section{Orelhado}
\begin{itemize}
\item {Grp. gram.:adj.}
\end{itemize}
\begin{itemize}
\item {Grp. gram.:M.}
\end{itemize}
\begin{itemize}
\item {Utilização:Ant.}
\end{itemize}
Que tem orelhas ou orelhetes.
Appêndice de barrete, para cobrir as orelhas.
\section{Orelhano}
\begin{itemize}
\item {Grp. gram.:adj.}
\end{itemize}
\begin{itemize}
\item {Utilização:Bras. do S}
\end{itemize}
Diz-se do gado vaccum, que não é marcado nas orelhas.
\section{Orelhão}
\begin{itemize}
\item {Grp. gram.:m.}
\end{itemize}
Puxão de orelhas.
Inflammação em tôrno das parótidas.
Parte do tear, nas fábricas de seda.
Peixe dos Açores.
\section{Orelha-redonda}
\begin{itemize}
\item {Grp. gram.:m.}
\end{itemize}
\begin{itemize}
\item {Utilização:Bras}
\end{itemize}
\begin{itemize}
\item {Utilização:Bras. do N}
\end{itemize}
Boi orelhano.
Animal, que não foi domesticado e que não tem sinal do seu dono.
\section{Orelhas-de-abbade}
\begin{itemize}
\item {Grp. gram.:f. pl.}
\end{itemize}
\begin{itemize}
\item {Utilização:Prov.}
\end{itemize}
\begin{itemize}
\item {Utilização:minh.}
\end{itemize}
Fritura, que se dá de presente, em dia de Anno-Bom. Cf. Castilho, \textunderscore Fastos\textunderscore , I, 349.
\section{Orelheira}
\begin{itemize}
\item {Grp. gram.:f.}
\end{itemize}
\begin{itemize}
\item {Utilização:Prov.}
\end{itemize}
\begin{itemize}
\item {Utilização:trasm.}
\end{itemize}
\begin{itemize}
\item {Utilização:Ant.}
\end{itemize}
Orelhas de um animal, especialmente de porco.
Cada um de dois paus que, no arado, convergem na relha e servem de aivecas para alargar o rêgo e voltar a leiva.
Cabeçal, travesseiro. Cf. \textunderscore Tombo do Estado da Índia\textunderscore , 52.
\section{Orelhete}
\begin{itemize}
\item {fónica:lhê}
\end{itemize}
\begin{itemize}
\item {Grp. gram.:m.}
\end{itemize}
\begin{itemize}
\item {Utilização:Bot.}
\end{itemize}
Appendículo, em fórma de orelha, na base das fôlhas de algumas plantas, chamado também \textunderscore orelha\textunderscore .
\section{Orelhíssimo}
\begin{itemize}
\item {Grp. gram.:adj.}
\end{itemize}
Que tem grandes orelhas.
Burrical; asnático.--Nunca vi nenhures o t., a não sêr em Filinto, V, 20.«\textunderscore ...em frente do orelhissimo francelho\textunderscore ».
(Por \textunderscore orelhudíssimo\textunderscore )
\section{Orelhudo}
\begin{itemize}
\item {Grp. gram.:adj.}
\end{itemize}
\begin{itemize}
\item {Utilização:Pop.}
\end{itemize}
\begin{itemize}
\item {Utilização:Fig.}
\end{itemize}
\begin{itemize}
\item {Utilização:Bras}
\end{itemize}
\begin{itemize}
\item {Grp. gram.:M.}
\end{itemize}
\begin{itemize}
\item {Utilização:Pop.}
\end{itemize}
\begin{itemize}
\item {Proveniência:(De \textunderscore orelha\textunderscore )}
\end{itemize}
Que tem orelhas grandes.
Estúpido; teimoso.
O mesmo que \textunderscore orelhano\textunderscore .
Burro.
\section{Orelina}
\begin{itemize}
\item {Grp. gram.:f.}
\end{itemize}
O mesmo que \textunderscore bixina\textunderscore .
\section{Orélia}
\begin{itemize}
\item {Grp. gram.:f.}
\end{itemize}
Planta apocýnea do Brasil.
\section{Orellana}
\begin{itemize}
\item {Grp. gram.:f.}
\end{itemize}
O mesmo que \textunderscore urucu\textunderscore .
\section{Orellina}
\begin{itemize}
\item {Grp. gram.:f.}
\end{itemize}
O mesmo que \textunderscore bixina\textunderscore .
\section{Oremanaus}
\begin{itemize}
\item {Grp. gram.:m. pl.}
\end{itemize}
Antiga nação de Índios da Guiana brasileira, dos quaes procedem os Manaus.
\section{Oreóbolo}
\begin{itemize}
\item {Grp. gram.:m.}
\end{itemize}
Gênero de plantas cyperáceas.
\section{Oreocálide}
\begin{itemize}
\item {Grp. gram.:f.}
\end{itemize}
Gênero de plantas proteáceas.
\section{Oreocállide}
\begin{itemize}
\item {Grp. gram.:f.}
\end{itemize}
Gênero de plantas proteáceas.
\section{Oreodafne}
\begin{itemize}
\item {Grp. gram.:f.}
\end{itemize}
\begin{itemize}
\item {Proveniência:(Do gr. \textunderscore oros\textunderscore , \textunderscore oreos\textunderscore  + \textunderscore daphne\textunderscore )}
\end{itemize}
Gênero de plantas lauríneas.
\section{Oreodaphne}
\begin{itemize}
\item {Grp. gram.:f.}
\end{itemize}
\begin{itemize}
\item {Proveniência:(Do gr. \textunderscore oros\textunderscore , \textunderscore oreos\textunderscore  + \textunderscore daphne\textunderscore )}
\end{itemize}
Gênero de plantas lauríneas.
\section{Oreodoxa}
\begin{itemize}
\item {Grp. gram.:f.}
\end{itemize}
Gênero de palmeiras.
\section{Oreófila}
\begin{itemize}
\item {Grp. gram.:f.}
\end{itemize}
\begin{itemize}
\item {Proveniência:(Do gr. \textunderscore oros\textunderscore , \textunderscore oreos\textunderscore  + \textunderscore philos\textunderscore )}
\end{itemize}
Gênero de plantas, da fam. das compostas.
\section{Oreóforo}
\begin{itemize}
\item {Grp. gram.:m.}
\end{itemize}
\begin{itemize}
\item {Proveniência:(Do gr. \textunderscore oros\textunderscore , \textunderscore oreos\textunderscore  + \textunderscore phoros\textunderscore )}
\end{itemize}
Gênero de moluscos decápodes.
\section{Oreognosia}
\begin{itemize}
\item {Grp. gram.:f.}
\end{itemize}
\begin{itemize}
\item {Proveniência:(Do gr. \textunderscore oros\textunderscore , \textunderscore oreos\textunderscore  + \textunderscore gnosis\textunderscore )}
\end{itemize}
Conhecimento das montanhas e da sua estructura.
\section{Oreognóstico}
\begin{itemize}
\item {Grp. gram.:adj.}
\end{itemize}
Relativo á oreognosia.
\section{Oreógrafo}
\textunderscore m.\textunderscore  (e der.)
O mesmo que \textunderscore orógrafo\textunderscore , etc.
\section{Oreógrapho}
\textunderscore m.\textunderscore  (e der.)
O mesmo que \textunderscore orógrapho\textunderscore , etc.
\section{Oreóphila}
\begin{itemize}
\item {Grp. gram.:f.}
\end{itemize}
\begin{itemize}
\item {Proveniência:(Do gr. \textunderscore oros\textunderscore , \textunderscore oreos\textunderscore  + \textunderscore philos\textunderscore )}
\end{itemize}
Gênero de plantas, da fam. das compostas.
\section{Oreóphoro}
\begin{itemize}
\item {Grp. gram.:m.}
\end{itemize}
\begin{itemize}
\item {Proveniência:(Do gr. \textunderscore oros\textunderscore , \textunderscore oreos\textunderscore  + \textunderscore phoros\textunderscore )}
\end{itemize}
Gênero de molluscos decápodes.
\section{Oressa}
\begin{itemize}
\item {Grp. gram.:f.}
\end{itemize}
\begin{itemize}
\item {Utilização:Prov.}
\end{itemize}
Aragem, viração.
(Por \textunderscore auressa\textunderscore , de \textunderscore aura\textunderscore )
\section{Orexina}
\begin{itemize}
\item {fónica:csi}
\end{itemize}
\begin{itemize}
\item {Grp. gram.:f.}
\end{itemize}
O mesmo que \textunderscore phenylhydroquinazolina\textunderscore .
\section{Órfam}
\begin{itemize}
\item {Grp. gram.:m.  e  adj.}
\end{itemize}
(V.órfão)
\section{Órfan}
\begin{itemize}
\item {Grp. gram.:f.}
\end{itemize}
(Flex. fem. de \textunderscore órfão\textunderscore )
\section{Orfanar}
\begin{itemize}
\item {Grp. gram.:v. t.}
\end{itemize}
\begin{itemize}
\item {Utilização:Fig.}
\end{itemize}
\begin{itemize}
\item {Proveniência:(Do lat. \textunderscore orphanus\textunderscore )}
\end{itemize}
Tornar órfão.
Privar:«\textunderscore ...communidade orfanada da sua maior glória\textunderscore ». Castilho, \textunderscore Montalverne\textunderscore .
\section{Orfanato}
\begin{itemize}
\item {Grp. gram.:m.}
\end{itemize}
\begin{itemize}
\item {Proveniência:(Do lat. \textunderscore orphanus\textunderscore )}
\end{itemize}
Estabelecimento pio, onde se recolhem, se sustentam e se educam órfãos.
\section{Orfandade}
\begin{itemize}
\item {Grp. gram.:f.}
\end{itemize}
\begin{itemize}
\item {Utilização:Fig.}
\end{itemize}
Estado de quem é órfão.
Os órfãos.
Privação, desamparo.
\section{Órfano}
\begin{itemize}
\item {Grp. gram.:m.}
\end{itemize}
\begin{itemize}
\item {Utilização:T. de Ceilão}
\end{itemize}
O mesmo que \textunderscore órfão\textunderscore .
\section{Orfanologia}
\begin{itemize}
\item {Grp. gram.:f.}
\end{itemize}
\begin{itemize}
\item {Proveniência:(Do gr. \textunderscore orphanos\textunderscore  + \textunderscore logos\textunderscore )}
\end{itemize}
Parte da sciência jurídica, que trata dos órfãos.
O conjunto de leis orfanológicas.
Repartição, onde se tratam negócios orfanológicos.
\section{Orfanológico}
\begin{itemize}
\item {Grp. gram.:adj.}
\end{itemize}
Relativo á orfanologia ou aos órfãos.
\section{Orfanotrófio}
\begin{itemize}
\item {Grp. gram.:m.}
\end{itemize}
\begin{itemize}
\item {Proveniência:(Do gr. \textunderscore orphanos\textunderscore  + \textunderscore trophe\textunderscore )}
\end{itemize}
Asilo de órfãos, entre os antigos Gregos.
\section{Orfanotróphio}
\begin{itemize}
\item {Grp. gram.:m.}
\end{itemize}
\begin{itemize}
\item {Proveniência:(Do gr. \textunderscore orphanos\textunderscore  + \textunderscore trophe\textunderscore )}
\end{itemize}
Asylo de órfãos, entre os antigos Gregos.
\section{Órfão}
\begin{itemize}
\item {Grp. gram.:adj.}
\end{itemize}
\begin{itemize}
\item {Utilização:Fig.}
\end{itemize}
\begin{itemize}
\item {Grp. gram.:M.}
\end{itemize}
\begin{itemize}
\item {Proveniência:(Do lat. \textunderscore orphanus\textunderscore )}
\end{itemize}
Que não tem pais.
Que não tem pai ou mãe.
Privado.
Vazio.
Que perdeu quem lhe era querido ou o protegia.
Aquelle que ficou órfão.
\section{Orga}
\begin{itemize}
\item {Grp. gram.:f.}
\end{itemize}
Planta caryophyllácea, (\textunderscore spergula arvensis\textunderscore , Lin).
\section{Orgada}
\begin{itemize}
\item {Grp. gram.:f.}
\end{itemize}
\begin{itemize}
\item {Utilização:Prov.}
\end{itemize}
\begin{itemize}
\item {Utilização:alg.}
\end{itemize}
\begin{itemize}
\item {Proveniência:(De \textunderscore órgão\textunderscore ?)}
\end{itemize}
Esqueleto; ossos de cadáver.
\section{Orgadura}
\begin{itemize}
\item {Grp. gram.:f.}
\end{itemize}
\begin{itemize}
\item {Utilização:Prov.}
\end{itemize}
\begin{itemize}
\item {Utilização:alg.}
\end{itemize}
O mesmo que \textunderscore orgada\textunderscore .
\section{Órgam}
(V.órgão)
\section{Organeiro}
\begin{itemize}
\item {Grp. gram.:m.}
\end{itemize}
\begin{itemize}
\item {Utilização:Ant.}
\end{itemize}
\begin{itemize}
\item {Proveniência:(Do lat. \textunderscore organarius\textunderscore )}
\end{itemize}
Fabricante de órgãos.
O encarregado da limpeza e tratamento dos órgãos de uma igreja.
Organista, tangedor de órgão.
\section{Organicamente}
\begin{itemize}
\item {Grp. gram.:adv.}
\end{itemize}
De modo orgânico.
\section{Organicismo}
\begin{itemize}
\item {Grp. gram.:m.}
\end{itemize}
\begin{itemize}
\item {Proveniência:(De \textunderscore orgânico\textunderscore )}
\end{itemize}
Theoria dos que attribuem qualquer doença à lesão material de um órgão.
\section{Organicista}
\begin{itemize}
\item {Grp. gram.:m.  e  f.}
\end{itemize}
\begin{itemize}
\item {Proveniência:(De \textunderscore orgânico\textunderscore )}
\end{itemize}
Pessôa partidária do organicismo.
\section{Orgânico}
\begin{itemize}
\item {Grp. gram.:adj.}
\end{itemize}
\begin{itemize}
\item {Proveniência:(Lat. \textunderscore organicus\textunderscore )}
\end{itemize}
Relativo aos órgãos ou aos seres organizados.
Inherente ao organismo.
Conforme a uma lei geral, (falando-se da formação das palavras).
Fundamental, que serve de base a uma instituição: \textunderscore o decreto orgânico do Ministério da Justiça\textunderscore .
\section{Organismo}
\begin{itemize}
\item {Grp. gram.:m.}
\end{itemize}
\begin{itemize}
\item {Proveniência:(Gr. \textunderscore organismos\textunderscore )}
\end{itemize}
Disposição dos órgãos, nos seres vivos.
Constituição orgânica, temperamento.
Corpo organizado, que existe ou póde existir independentemente.
Conjunto de partes, que concorrem para determinado fim: \textunderscore o organismo de um relógio\textunderscore .
Conjunto das funcções que os órgãos executam.
\section{Organista}
\begin{itemize}
\item {Grp. gram.:m.  e  f.}
\end{itemize}
\begin{itemize}
\item {Proveniência:(Do lat. \textunderscore organus\textunderscore )}
\end{itemize}
Pessôa, que toca órgão.
\section{Organito}
\begin{itemize}
\item {Grp. gram.:m.}
\end{itemize}
\begin{itemize}
\item {Utilização:Anat.}
\end{itemize}
\begin{itemize}
\item {Proveniência:(Do lat. \textunderscore organum\textunderscore )}
\end{itemize}
Corpo organizado, de fórma regular, mas incapaz de se reproduzir, tal como os glóbulos do sangue, os grânulos do amido, os espermatozoides, etc.
\section{Organização}
\begin{itemize}
\item {Grp. gram.:f.}
\end{itemize}
Acto ou effeito de organizar.
Estructura.
Disposição de alguma coisa para certo fim.
Constituição phýsica do corpo humano: \textunderscore organização robusta\textunderscore .
Constituição; instituição.
\section{Organizado}
\begin{itemize}
\item {Grp. gram.:adj.}
\end{itemize}
Que tem órgãos.
\section{Organizador}
\begin{itemize}
\item {Grp. gram.:m.  e  adj.}
\end{itemize}
O que organiza.
\section{Organizar}
\begin{itemize}
\item {Grp. gram.:v. t.}
\end{itemize}
\begin{itemize}
\item {Proveniência:(De \textunderscore órgão\textunderscore )}
\end{itemize}
Dispor.
Instituir: \textunderscore organizar uma associação\textunderscore .
Constituír em organismo.
Formar; apropriar.
\section{Organizável}
\begin{itemize}
\item {Grp. gram.:adj.}
\end{itemize}
Que se póde organizar.
\section{Organogenesia}
\begin{itemize}
\item {Grp. gram.:f.}
\end{itemize}
\begin{itemize}
\item {Proveniência:(Do gr. \textunderscore organon\textunderscore  + \textunderscore genesis\textunderscore )}
\end{itemize}
Descripção da maneira como se desenvolvem os órgãos, depois do estado embryonário.
\section{Organogenésico}
\begin{itemize}
\item {Grp. gram.:adj.}
\end{itemize}
Relativo á organogenesia.
\section{Organogenia}
\begin{itemize}
\item {Grp. gram.:f.}
\end{itemize}
O mesmo que \textunderscore organogenesia\textunderscore .
\section{Organogênico}
\begin{itemize}
\item {Grp. gram.:adj.}
\end{itemize}
Relativo á organogenia.
\section{Organografia}
\begin{itemize}
\item {Grp. gram.:f.}
\end{itemize}
\begin{itemize}
\item {Utilização:Gram.}
\end{itemize}
\begin{itemize}
\item {Proveniência:(Do gr. \textunderscore organon\textunderscore  + \textunderscore graphein\textunderscore )}
\end{itemize}
Descripção dos órgãos de um sêr organizado.
O mesmo que \textunderscore flexionismo\textunderscore .
\section{Organográfico}
\begin{itemize}
\item {Grp. gram.:adj.}
\end{itemize}
Relativo á organografia.
\section{Organographia}
\begin{itemize}
\item {Grp. gram.:f.}
\end{itemize}
\begin{itemize}
\item {Utilização:Gram.}
\end{itemize}
\begin{itemize}
\item {Proveniência:(Do gr. \textunderscore organon\textunderscore  + \textunderscore graphein\textunderscore )}
\end{itemize}
Descripção dos órgãos de um sêr organizado.
O mesmo que \textunderscore flexionismo\textunderscore .
\section{Organográphico}
\begin{itemize}
\item {Grp. gram.:adj.}
\end{itemize}
Relativo á organographia.
\section{Organóide}
\begin{itemize}
\item {Grp. gram.:adj.}
\end{itemize}
\begin{itemize}
\item {Proveniência:(Do gr. \textunderscore organon\textunderscore  + \textunderscore eidos\textunderscore )}
\end{itemize}
Que tem apparência de órgão ou de corpo organizado.
\section{Organoléptico}
\begin{itemize}
\item {Grp. gram.:adj.}
\end{itemize}
\begin{itemize}
\item {Proveniência:(Do gr. \textunderscore organon\textunderscore  + \textunderscore leptos\textunderscore )}
\end{itemize}
Diz-se das propriedades, com que os corpos impressionam os sentidos, e das acções que os corpos podem exercer nos órgãos interiores de um corpo vivo.
\section{Organonímia}
\begin{itemize}
\item {Grp. gram.:f.}
\end{itemize}
\begin{itemize}
\item {Proveniência:(Do gr. \textunderscore organon\textunderscore  + \textunderscore onuma\textunderscore )}
\end{itemize}
Arte de coordenar os nomes dados aos órgãos dos seres vivos.
\section{Organonímico}
\begin{itemize}
\item {Grp. gram.:adj.}
\end{itemize}
Relativo á organonímia.
\section{Organonýmia}
\begin{itemize}
\item {Grp. gram.:f.}
\end{itemize}
\begin{itemize}
\item {Proveniência:(Do gr. \textunderscore organon\textunderscore  + \textunderscore onuma\textunderscore )}
\end{itemize}
Arte de coordenar os nomes dados aos órgãos dos seres vivos.
\section{Organonýmico}
\begin{itemize}
\item {Grp. gram.:adj.}
\end{itemize}
Relativo á organonýmia.
\section{Organopathia}
\begin{itemize}
\item {Grp. gram.:f.}
\end{itemize}
\begin{itemize}
\item {Proveniência:(Do gr. \textunderscore organon\textunderscore  + \textunderscore pathos\textunderscore )}
\end{itemize}
Doença dos órgãos em geral.
\section{Organopatia}
\begin{itemize}
\item {Grp. gram.:f.}
\end{itemize}
\begin{itemize}
\item {Proveniência:(Do gr. \textunderscore organon\textunderscore  + \textunderscore pathos\textunderscore )}
\end{itemize}
Doença dos órgãos em geral.
\section{Organoplastia}
\begin{itemize}
\item {Grp. gram.:f.}
\end{itemize}
\begin{itemize}
\item {Proveniência:(Do gr. \textunderscore organon\textunderscore  + \textunderscore plassein\textunderscore )}
\end{itemize}
Arte de modificar artificialmente a fórma dos seres vivos.
\section{Organoplástico}
\begin{itemize}
\item {Grp. gram.:adj.}
\end{itemize}
Relativo á organoplastia.
\section{Organoscopia}
\begin{itemize}
\item {Grp. gram.:f.}
\end{itemize}
\begin{itemize}
\item {Proveniência:(Do gr. \textunderscore organon\textunderscore  + \textunderscore skopein\textunderscore )}
\end{itemize}
Observação ou estudo dos órgãos de um indivíduo, para dessa observação se tirarem conclusões relativas á índole, ás inclinações, paixões, etc.
\section{Organoscópico}
\begin{itemize}
\item {Grp. gram.:adj.}
\end{itemize}
Relativo á organoscopia.
\section{Organotáctico}
\begin{itemize}
\item {Grp. gram.:adj.}
\end{itemize}
Relativo á organotaxia.
\section{Organotaxia}
\begin{itemize}
\item {fónica:csi}
\end{itemize}
\begin{itemize}
\item {Grp. gram.:f.}
\end{itemize}
\begin{itemize}
\item {Proveniência:(Do gr. \textunderscore organon\textunderscore  + \textunderscore taxis\textunderscore )}
\end{itemize}
Agrupamento dos seres vivos, segundo as relações íntimas da sua organização.
\section{Organotáxico}
\begin{itemize}
\item {fónica:csi}
\end{itemize}
\begin{itemize}
\item {Grp. gram.:adj.}
\end{itemize}
Relativo á organotaxia.
\section{Organsim}
\begin{itemize}
\item {Grp. gram.:m.}
\end{itemize}
\begin{itemize}
\item {Proveniência:(De \textunderscore órgão\textunderscore )}
\end{itemize}
O primeiro fio de seda, que se deita no tear, para formar urdidura.
\section{Organsinar}
\begin{itemize}
\item {Grp. gram.:v. t.}
\end{itemize}
\begin{itemize}
\item {Proveniência:(De \textunderscore organsim\textunderscore )}
\end{itemize}
Tecer em rodas apropriadas (fios de seda bruta), para formar o organsim.
\section{Organsino}
\begin{itemize}
\item {Grp. gram.:m.}
\end{itemize}
O mesmo que \textunderscore organsim\textunderscore . Cf. \textunderscore Inquér. Industr.\textunderscore , 3.^a p., 237.
\section{Órgão}
\begin{itemize}
\item {Grp. gram.:m.}
\end{itemize}
\begin{itemize}
\item {Utilização:Prov.}
\end{itemize}
\begin{itemize}
\item {Utilização:Des.}
\end{itemize}
\begin{itemize}
\item {Utilização:Fig.}
\end{itemize}
\begin{itemize}
\item {Utilização:Ant.}
\end{itemize}
\begin{itemize}
\item {Proveniência:(Do lat. \textunderscore organum\textunderscore )}
\end{itemize}
Cada uma das partes de um maquinismo, que exercem funcção especial.
Instrumento musical de vento e teclado.
Parte de um organismo com funcção especial.
Peça de madeira, por onde passam os fios, que entretecem a esteira.
Cada um de dois rolos de madeira, no tear, num dos quaes se enrola o fio da teia e no outro o pano que se vai tecendo.
Instrumento músico:«\textunderscore nos salgueiros pendurei os órgãos em que cantava\textunderscore ». Camões, paráphrase ao salmo 136.
Meio de acção.
Meio; intermediário.
Periódico ou outro meio, com que se manifesta a opinião do um partido, seita, etc.
Metal de voz.
\section{Orgasmo}
\begin{itemize}
\item {Grp. gram.:m.}
\end{itemize}
\begin{itemize}
\item {Proveniência:(Gr. \textunderscore orgasmos\textunderscore )}
\end{itemize}
Aumento de acção vital do um órgão, muitas vezes com turgescência.
\section{Orgástico}
\begin{itemize}
\item {Grp. gram.:adj.}
\end{itemize}
Relativo ao orgasmo.
\section{Orge}
\begin{itemize}
\item {Grp. gram.:f.}
\end{itemize}
\begin{itemize}
\item {Utilização:Ant.}
\end{itemize}
\begin{itemize}
\item {Proveniência:(Fr. \textunderscore orge\textunderscore )}
\end{itemize}
(V.cevada)
\section{Orgevão}
\begin{itemize}
\item {Grp. gram.:m.}
\end{itemize}
O mesmo que \textunderscore urgebão\textunderscore .
\section{Orgía}
\begin{itemize}
\item {Grp. gram.:f.}
\end{itemize}
\begin{itemize}
\item {Utilização:Fig.}
\end{itemize}
\begin{itemize}
\item {Proveniência:(Lat. \textunderscore orgia\textunderscore )}
\end{itemize}
(\textunderscore órgia\textunderscore , segundo Castilho, de acôrdo com o latim)
Festim ou banquete, licencioso.
Bacchanal.
Desordem, tumulto.
\section{Orgíaco}
\begin{itemize}
\item {Grp. gram.:adj.}
\end{itemize}
\begin{itemize}
\item {Proveniência:(Gr. \textunderscore orgiakos\textunderscore )}
\end{itemize}
Relativo á orgia ou que tem o carácter della.
\section{Orgiasta}
\begin{itemize}
\item {Grp. gram.:m.  e  f.}
\end{itemize}
\begin{itemize}
\item {Utilização:Des.}
\end{itemize}
\begin{itemize}
\item {Proveniência:(Gr. \textunderscore orgiastes\textunderscore )}
\end{itemize}
Pessôa, que tomava parte em orgias.
\section{Orgiástico}
\begin{itemize}
\item {Grp. gram.:adj.}
\end{itemize}
\begin{itemize}
\item {Proveniência:(De \textunderscore orgiasta\textunderscore )}
\end{itemize}
Relativo a orgias. Cf. Camillo, \textunderscore Narcót.\textunderscore , II, 248.
\section{Órgio}
\begin{itemize}
\item {Grp. gram.:adj.}
\end{itemize}
O mesmo que \textunderscore orgíaco\textunderscore . Cf. Castilho, \textunderscore Metam.\textunderscore , 151.
\section{Orgulhar}
\begin{itemize}
\item {Grp. gram.:v. t.}
\end{itemize}
\begin{itemize}
\item {Grp. gram.:V. p.}
\end{itemize}
Causar orgulho a.
Produzir ufania em.
Ensoberbecer.
Sentir orgulho; ensoberbecer-se; ufanar-se.
\section{Orgulhecer}
\begin{itemize}
\item {Grp. gram.:v. t.  e  p.}
\end{itemize}
O mesmo que \textunderscore orgulhar\textunderscore .
\section{Orgulho}
\begin{itemize}
\item {Grp. gram.:m.}
\end{itemize}
\begin{itemize}
\item {Proveniência:(Do ant. al. \textunderscore orguol\textunderscore )}
\end{itemize}
Sentimento ou estado da alma, onde se fórma o conceito elevado que alguém faz de si próprio.
Soberba.
Pondunor, sentimento de dignidade pessoal.
Legitima ufania.
\section{Orgulhosamente}
\begin{itemize}
\item {Grp. gram.:adv.}
\end{itemize}
De modo orgulhoso.
Com arrogância; com ufania.
\section{Orgulhoso}
\begin{itemize}
\item {Grp. gram.:m.  e  adj.}
\end{itemize}
Aquelle que tem orgulho.
Aquelle que manifesta orgulho.
\section{Oriá}
\begin{itemize}
\item {Grp. gram.:m.}
\end{itemize}
Uma das línguas índicas vernáculas, faladas em Orixá.
\section{Oríbata}
\begin{itemize}
\item {Grp. gram.:m.}
\end{itemize}
\begin{itemize}
\item {Proveniência:(Do gr. \textunderscore oros\textunderscore  + \textunderscore bates\textunderscore )}
\end{itemize}
Nome que se dava a certos volatins gregos.
\section{Oricalcito}
\begin{itemize}
\item {Grp. gram.:m.}
\end{itemize}
\begin{itemize}
\item {Proveniência:(De \textunderscore oricalco\textunderscore )}
\end{itemize}
Hidrocarbonato de cobre e de zinco.
\section{Oricalco}
\begin{itemize}
\item {Grp. gram.:m.}
\end{itemize}
\begin{itemize}
\item {Proveniência:(Lat. \textunderscore orichalcum\textunderscore )}
\end{itemize}
Nome, com que os Gregos denominaram, umas vezes, o cobre puro, outras o latão, (cobre e zinco), e outras o bronze, (cobre e estanho).
\section{Orichalcito}
\begin{itemize}
\item {fónica:cal}
\end{itemize}
\begin{itemize}
\item {Grp. gram.:m.}
\end{itemize}
\begin{itemize}
\item {Proveniência:(De \textunderscore orichalco\textunderscore )}
\end{itemize}
Hydrocarbonato de cobre e de zinco.
\section{Orichalco}
\begin{itemize}
\item {fónica:cal}
\end{itemize}
\begin{itemize}
\item {Grp. gram.:m.}
\end{itemize}
\begin{itemize}
\item {Proveniência:(Lat. \textunderscore orichalcum\textunderscore )}
\end{itemize}
Nome, com que os Gregos denominaram, umas vezes, o cobre puro, outras o latão, (cobre e zinco), e outras o bronze, (cobre e estanho).
\section{Oricuriá}
\begin{itemize}
\item {Grp. gram.:m.}
\end{itemize}
Ave nocturna, de canto lúgubre, nas margens do Tocantins.
\section{Orida}
\begin{itemize}
\item {Grp. gram.:f.}
\end{itemize}
Planta leguminosa de Dio.
\section{Orientação}
\begin{itemize}
\item {Grp. gram.:f.}
\end{itemize}
\begin{itemize}
\item {Utilização:Fig.}
\end{itemize}
\begin{itemize}
\item {Utilização:Náut.}
\end{itemize}
Acto ou arte de se orientar.
Direcção, impulso.
Disposição, que se dá ás velas e vêrgas, para receberem o impulso do vento.
\section{Orientador}
\begin{itemize}
\item {Grp. gram.:adj.}
\end{itemize}
\begin{itemize}
\item {Grp. gram.:M.}
\end{itemize}
Que orienta.
Guia, director.
Instrumento, para determinar o Oriente, em relação a qualquer ponto.
\section{Oriental}
\begin{itemize}
\item {Grp. gram.:adj.}
\end{itemize}
\begin{itemize}
\item {Grp. gram.:M.  e  f.}
\end{itemize}
\begin{itemize}
\item {Grp. gram.:M. pl.}
\end{itemize}
\begin{itemize}
\item {Proveniência:(Lat. \textunderscore orientalis\textunderscore )}
\end{itemize}
Relativo ao Oriente: \textunderscore contos orientaes\textunderscore .
Que está no Oriente ou que vem do Oriente.
Próprio do Oriente.
Pessôa, que é ou foi do Oriente ou da Ásia.
Os povos da Ásia.
\section{Orientalidade}
\begin{itemize}
\item {Grp. gram.:f.}
\end{itemize}
Qualidade do que é oriental.
\section{Orientalismo}
\begin{itemize}
\item {Grp. gram.:m.}
\end{itemize}
\begin{itemize}
\item {Proveniência:(De \textunderscore oriental\textunderscore )}
\end{itemize}
Conjunto dos conhecimentos relativos ao Oriente e dos costumes dos povos orientaes.
Sciência de orientalista.
\section{Orientalista}
\begin{itemize}
\item {Grp. gram.:m.  e  f.}
\end{itemize}
\begin{itemize}
\item {Proveniência:(De \textunderscore oriental\textunderscore )}
\end{itemize}
Pessôa, versada no conhecimento das línguas e literaturas do Oriente.
\section{Orientalizar}
\begin{itemize}
\item {Grp. gram.:v. t.}
\end{itemize}
\begin{itemize}
\item {Grp. gram.:V. p.}
\end{itemize}
Dar feição ou usos de oriental a.
Adquirir os hábitos ou os costumes do Oriente. Cf. B. Pato, \textunderscore Port. na Índia\textunderscore , 216; Ol. Martins, \textunderscore Hellenismo\textunderscore , p. XXX.
\section{Orientar}
\begin{itemize}
\item {Grp. gram.:v. t.}
\end{itemize}
\begin{itemize}
\item {Proveniência:(De \textunderscore oriente\textunderscore )}
\end{itemize}
Determinar, em relação ao Oriente, a posição de. Adaptar á direcção dos pontos cardeaes.
Encaminhar, guiar.
Dirigir o rumo de.
\section{Oriente}
\begin{itemize}
\item {Grp. gram.:m.}
\end{itemize}
\begin{itemize}
\item {Utilização:Fig.}
\end{itemize}
\begin{itemize}
\item {Proveniência:(Lat. \textunderscore oriens\textunderscore )}
\end{itemize}
Lado do horizonte, em que o Sol apparece quando nasce.
Nascente, levante.
Povos da Ásia, especialmente os que estão próximo do mar do Levante.
Lustro natural das pérolas e aljófares.
Lado direito de uma carta geográphica.
\textunderscore Grande oriente\textunderscore , loja maçónica, a que estão subordinadas as outras do mesmo país.
Comêço.
\section{Orifício}
\begin{itemize}
\item {Grp. gram.:m.}
\end{itemize}
\begin{itemize}
\item {Proveniência:(Lat. \textunderscore orificium\textunderscore )}
\end{itemize}
Entrada estreita.
Saída, á maneira de bôca.
Abertura.
Buraquinho.
\section{Oriflama}
\begin{itemize}
\item {Grp. gram.:f.}
\end{itemize}
O mesmo que \textunderscore auriflama\textunderscore .
\section{Oriflamma}
\begin{itemize}
\item {Grp. gram.:f.}
\end{itemize}
O mesmo que \textunderscore auriflamma\textunderscore .
\section{Oriforme}
\begin{itemize}
\item {Grp. gram.:adj.}
\end{itemize}
\begin{itemize}
\item {Proveniência:(Do lat. \textunderscore os\textunderscore , \textunderscore oris\textunderscore  + \textunderscore forma\textunderscore )}
\end{itemize}
Que tem fórma de bôca.
\section{Origem}
\begin{itemize}
\item {Grp. gram.:f.}
\end{itemize}
\begin{itemize}
\item {Proveniência:(Lat. \textunderscore origo\textunderscore )}
\end{itemize}
Lugar, ponto ou facto, donde alguma coisa provém.
Princípio, nascimento, causa.
Proveniência.
Tronco de gerações.
Ponto de partida.
Base; pretexto.--Foi voc. masculino. Cf. Pant. de Aveiro, \textunderscore Itiner.\textunderscore , 281, (2.^a ed.).
\section{Originador}
\begin{itemize}
\item {Grp. gram.:m.  e  adj.}
\end{itemize}
O que origina.
\section{Original}
\begin{itemize}
\item {Grp. gram.:adj.}
\end{itemize}
\begin{itemize}
\item {Grp. gram.:M.}
\end{itemize}
\begin{itemize}
\item {Utilização:Fam.}
\end{itemize}
\begin{itemize}
\item {Proveniência:(Lat. \textunderscore originalis\textunderscore )}
\end{itemize}
Relativo a origem.
Primitivo.
Que parece inventado ou imaginado, sem modêlo.
Que tem cunho ou carácter próprio.
Que procede sem imitar ninguém.
Que não tem semelhante.
Extraordinário.
Extravagante.
Esquisito.
Manuscrito primitivo de um texto, de uma acta, etc.
Obra de arte, que serve de typo e não é imitação, mas póde sêr imitada ou copiada.
Aquillo que provém da origem, que tem carácter próprio.
Texto, em que uma obra foi escrita, (em opposição a \textunderscore imitação\textunderscore  ou \textunderscore traducção\textunderscore ).
Qualquer manuscrito ou extracto impresso, destinado á composição typográphica de um jornal, livro, etc.
Pessôa excêntrica.
\section{Originalidade}
\begin{itemize}
\item {Grp. gram.:f.}
\end{itemize}
Qualidade do que é original.
\section{Originalmente}
\begin{itemize}
\item {Grp. gram.:adv.}
\end{itemize}
De modo original.
Relativamente á origem.
\section{Originar}
\begin{itemize}
\item {Grp. gram.:v. t.}
\end{itemize}
\begin{itemize}
\item {Proveniência:(Do lat. \textunderscore origo\textunderscore )}
\end{itemize}
Dar origem a.
Determinar, causar.
Predispor.
\section{Originariamente}
\begin{itemize}
\item {Grp. gram.:adv.}
\end{itemize}
De modo originário.
Na origem, no princípio.
\section{Originário}
\begin{itemize}
\item {Grp. gram.:adj.}
\end{itemize}
\begin{itemize}
\item {Proveniência:(Lat. \textunderscore originarius\textunderscore )}
\end{itemize}
Que tem a sua origem em alguém, em alguma coisa, em alguma localidade ou região.
Conforme á sua origem.
Conservado desde a sua origem: \textunderscore carácter originário\textunderscore .
\section{Originistas}
\begin{itemize}
\item {Grp. gram.:m. pl.}
\end{itemize}
Herejes, discípulos de Orígínes, os quaes sustentavam que a Bem-aventurança não é eterna.
\section{Origma}
\begin{itemize}
\item {Grp. gram.:m.}
\end{itemize}
Abysmo, em que se precipitavam os criminosos, entre os Athenienses.
\section{Origone}
\begin{itemize}
\item {Grp. gram.:m.}
\end{itemize}
\begin{itemize}
\item {Utilização:Bras. do S}
\end{itemize}
Talhadas de pêssego, sêcas ao sol, com as quaes se faz um doce de calda.
(Cp. cast. \textunderscore orejón\textunderscore )
\section{Orilha}
\begin{itemize}
\item {Grp. gram.:f.}
\end{itemize}
Filete, em volta de uma obra de ourivezaria.
Borda, orla.
(Cast. \textunderscore orilla\textunderscore )
\section{Ório}
\begin{itemize}
\item {Grp. gram.:m.}
\end{itemize}
\begin{itemize}
\item {Utilização:Ant.}
\end{itemize}
O mesmo que \textunderscore ordo\textunderscore .
\section{Orió}
\begin{itemize}
\item {Grp. gram.:m.}
\end{itemize}
Planta indiana, (\textunderscore panicum italicum\textunderscore ).
(Do conc.)
\section{Órion}
\begin{itemize}
\item {Grp. gram.:m.}
\end{itemize}
\begin{itemize}
\item {Proveniência:(De \textunderscore Orion\textunderscore , n. p. myth.)}
\end{itemize}
Constellação do hemisphério austral.
\section{Orite}
\begin{itemize}
\item {Grp. gram.:f.}
\end{itemize}
Gênero de plantas proteáceas.
\section{Orítia}
\begin{itemize}
\item {Grp. gram.:f.}
\end{itemize}
\begin{itemize}
\item {Proveniência:(De \textunderscore Oríthya\textunderscore , n. p. myth.)}
\end{itemize}
Gênero de plantas liliáceas.
\section{Oritina}
\begin{itemize}
\item {Grp. gram.:f.}
\end{itemize}
Gênero de plantas proteáceas.
\section{Orítya}
\begin{itemize}
\item {Grp. gram.:f.}
\end{itemize}
\begin{itemize}
\item {Proveniência:(De \textunderscore Oríthya\textunderscore , n. p. myth.)}
\end{itemize}
Gênero de plantas liliáceas.
\section{Oriundo}
\begin{itemize}
\item {Grp. gram.:adj.}
\end{itemize}
\begin{itemize}
\item {Proveniência:(Lat. \textunderscore oriundus\textunderscore )}
\end{itemize}
Originário; procedente, proveniente; natural.
\section{Oriz}
\begin{itemize}
\item {Grp. gram.:m.}
\end{itemize}
\begin{itemize}
\item {Utilização:Ant.}
\end{itemize}
O mesmo que \textunderscore ourives\textunderscore .
(Cp. \textunderscore orizes\textunderscore )
\section{Orizes}
\begin{itemize}
\item {Grp. gram.:m. pl.}
\end{itemize}
\begin{itemize}
\item {Utilização:Ant.}
\end{itemize}
Os que trabalhavam em oiro ou prata; os ourives.
(Contr. de \textunderscore ourivezes\textunderscore , pl. ant. de \textunderscore ourives\textunderscore )
\section{Orizes procazes}
\begin{itemize}
\item {Grp. gram.:m. pl.}
\end{itemize}
Tribos tapuias, que viviam nas serras da Baía, (Brasil).
\section{Orjo}
\begin{itemize}
\item {Grp. gram.:m.}
\end{itemize}
\begin{itemize}
\item {Utilização:Ant.}
\end{itemize}
O mesmo que \textunderscore orge\textunderscore  ou \textunderscore cevada\textunderscore .
\section{Orla}
\begin{itemize}
\item {Grp. gram.:f.}
\end{itemize}
\begin{itemize}
\item {Proveniência:(Do lat. \textunderscore hypoth\textunderscore . \textunderscore orula\textunderscore , dem. de \textunderscore ora\textunderscore )}
\end{itemize}
Filete, sob o ornato oval de um capitel.
Bôrdo.
Borda.
Tira.
Margem.
Rebordo de uma cratera.
Cairel.
Guarnição.
Bainha.
Cercadura.
\section{Orladeira}
\begin{itemize}
\item {Grp. gram.:f.}
\end{itemize}
\begin{itemize}
\item {Proveniência:(De \textunderscore orlar\textunderscore )}
\end{itemize}
Operária que debrua. Cf. \textunderscore Inquér. Indust.\textunderscore , 2.^a p., l. III, 182.
\section{Orladura}
\begin{itemize}
\item {Grp. gram.:f.}
\end{itemize}
Acto ou effeito de orlar.
Orla.
\section{Orlar}
\begin{itemize}
\item {Grp. gram.:v. t.}
\end{itemize}
Pôr orla em.
Ornar ou guarnecer de orla.
Debruar, embainhar.
Ornar em volta; rodear.
\section{Orfaico}
\begin{itemize}
\item {Grp. gram.:adj.}
\end{itemize}
O mesmo que \textunderscore orfeico\textunderscore .
\section{Orfásio}
\begin{itemize}
\item {Grp. gram.:m.}
\end{itemize}
Antigo instrumento de cordas, parecido ao alaúde e que foi usado no século XVII.
\section{Orfeão}
\begin{itemize}
\item {Grp. gram.:m.}
\end{itemize}
\begin{itemize}
\item {Proveniência:(Fr. \textunderscore orphéon\textunderscore , de \textunderscore Orpheu\textunderscore , n. p.)}
\end{itemize}
Escola de canto.
Sociedade, cujos membros se dedicam á prática da música vocal e do canto coral, sem acompanhamento.
\section{Orfeico}
\begin{itemize}
\item {Grp. gram.:adj.}
\end{itemize}
\begin{itemize}
\item {Proveniência:(Lat. \textunderscore orpheicus\textunderscore )}
\end{itemize}
Relativo á música.
\section{Orfeónico}
\begin{itemize}
\item {Grp. gram.:adj.}
\end{itemize}
Relativo ao orfeão.
\section{Orfeonista}
\begin{itemize}
\item {Grp. gram.:m.}
\end{itemize}
Membro de um orfeão.
\section{Orfeonístico}
\begin{itemize}
\item {Grp. gram.:adj.}
\end{itemize}
\begin{itemize}
\item {Utilização:Neol.}
\end{itemize}
Relativo a orfeonista.
\section{Orfeoteleste}
\begin{itemize}
\item {Grp. gram.:m.}
\end{itemize}
Aquele que interpretava os mistérios introduzidos na Grécia por Orfeu.
\section{Órfia}
\begin{itemize}
\item {Grp. gram.:f.}
\end{itemize}
Gênero de peixes malacopterígios abdominaes.
\section{Órfio}
\begin{itemize}
\item {Grp. gram.:m.}
\end{itemize}
Gênero de plantas gencianáceas.
\section{Orfneia}
\begin{itemize}
\item {Grp. gram.:f.}
\end{itemize}
Gênero de crustáceos decápodes.
\section{Orlean}
\begin{itemize}
\item {Grp. gram.:f.}
\end{itemize}
\begin{itemize}
\item {Proveniência:(De \textunderscore Orleans\textunderscore , n. p.)}
\end{itemize}
Tecido leve, de lan e algodão.
\section{Orleanês}
\begin{itemize}
\item {Grp. gram.:adj.}
\end{itemize}
\begin{itemize}
\item {Grp. gram.:M.}
\end{itemize}
Relativo a Orleans.
Habitante de Orleans.
\section{Orleanismo}
\begin{itemize}
\item {Grp. gram.:m.}
\end{itemize}
Partido ou opiniões políticas dos Orleanistas.
\section{Orleanista}
\begin{itemize}
\item {Grp. gram.:m.}
\end{itemize}
Partidário da família Orleans, que reinou em França.
\section{Orlo}
\begin{itemize}
\item {Grp. gram.:m.}
\end{itemize}
Antigo instrumento de sôpro, com meia volta. Cf. \textunderscore Peregrinação\textunderscore , LXIX.
(Cast. \textunderscore orlo\textunderscore )
\section{Ormocarpo}
\begin{itemize}
\item {Grp. gram.:m.}
\end{itemize}
Gênero de plantas leguminosas, papilionáceas.
\section{Ormorrhiza}
\begin{itemize}
\item {Grp. gram.:f.}
\end{itemize}
Gênero de plantas umbellíferas.
\section{Ormorriza}
\begin{itemize}
\item {Grp. gram.:f.}
\end{itemize}
Gênero de plantas umbelíferas.
\section{Ormósia}
\begin{itemize}
\item {Grp. gram.:f.}
\end{itemize}
Gênero de plantas leguminosas.
\section{Ormosolência}
\begin{itemize}
\item {Grp. gram.:f.}
\end{itemize}
Gênero de plantas umbellíferas.
\section{Ornado}
\begin{itemize}
\item {Grp. gram.:adj.}
\end{itemize}
\begin{itemize}
\item {Proveniência:(De \textunderscore ornar\textunderscore )}
\end{itemize}
Diz-se do mato, rijo, duro, de terra areenta.
\section{Ornador}
\begin{itemize}
\item {Grp. gram.:m.  e  adj.}
\end{itemize}
\begin{itemize}
\item {Proveniência:(Do lat. \textunderscore ornator\textunderscore )}
\end{itemize}
O que orna.
\section{Ornamentação}
\begin{itemize}
\item {Grp. gram.:f.}
\end{itemize}
Acto ou effeito de ornamentar.
\section{Ornamentador}
\begin{itemize}
\item {Grp. gram.:m.  e  adj.}
\end{itemize}
O mesmo que \textunderscore ornamentista\textunderscore .
\section{Ornamental}
\begin{itemize}
\item {Grp. gram.:adj.}
\end{itemize}
Relativo a ornamentos; próprio para adôrno, ou para ornamentar.
\section{Ornamentar}
\begin{itemize}
\item {Grp. gram.:v. t.}
\end{itemize}
\begin{itemize}
\item {Proveniência:(De \textunderscore ornamento\textunderscore )}
\end{itemize}
Ornar; decorar, pôr adornos em.
Enfeitar; abrilhantar.
\section{Ornamentista}
\begin{itemize}
\item {Grp. gram.:m.  e  f.}
\end{itemize}
\begin{itemize}
\item {Proveniência:(De \textunderscore ornamento\textunderscore )}
\end{itemize}
Pessôa, que ornamenta, que faz ornatos, em construcções.
\section{Ornamento}
\begin{itemize}
\item {Grp. gram.:m.}
\end{itemize}
\begin{itemize}
\item {Grp. gram.:Pl.}
\end{itemize}
\begin{itemize}
\item {Proveniência:(Lat. \textunderscore ornamentum\textunderscore )}
\end{itemize}
Acto ou effeito de ornar; ornato.
Aquillo que orna.
Pessôa illustre, que pelos seus méritos ou trabalhos honra ou nobilita uma classe, uma corporação, uma instituição.
Paramentos.
\section{Ornar}
\begin{itemize}
\item {Grp. gram.:v. t.}
\end{itemize}
\begin{itemize}
\item {Utilização:Fig.}
\end{itemize}
\begin{itemize}
\item {Proveniência:(Lat. \textunderscore ornare\textunderscore )}
\end{itemize}
Tornar formoso, insigne, distinto.
Enfeitar.
Engrandecer.
Embellezar.
Glorificar.
\section{Ornato}
\begin{itemize}
\item {Grp. gram.:m.}
\end{itemize}
\begin{itemize}
\item {Proveniência:(Lat. \textunderscore ornatus\textunderscore )}
\end{itemize}
O mesmo que \textunderscore ornamento\textunderscore .
Effeito de ornar.
Aquillo que orna.
Cópia artística de qualquer assumpto da natureza morta.
Composição pictural ou architectural, proveniente da fantasia.
Aquillo que dá luz, fôrça e graça a um discurso.
Requebro no canto.
\section{Ornear}
\begin{itemize}
\item {Grp. gram.:v. i.}
\end{itemize}
O mesmo que \textunderscore ornejar\textunderscore .
\section{Orneio}
\begin{itemize}
\item {Grp. gram.:m.}
\end{itemize}
O mesmo que \textunderscore ornejo\textunderscore .
\section{Ornejador}
\begin{itemize}
\item {Grp. gram.:m.  e  adj.}
\end{itemize}
Animal que orneja.
\section{Ornejar}
\begin{itemize}
\item {Grp. gram.:v. i.}
\end{itemize}
O mesmo que \textunderscore zurrar\textunderscore ^1.
\section{Ornejo}
\begin{itemize}
\item {Grp. gram.:m.}
\end{itemize}
\begin{itemize}
\item {Proveniência:(De \textunderscore ornejar\textunderscore )}
\end{itemize}
O mesmo que \textunderscore zurro\textunderscore .
\section{Ornis}
\begin{itemize}
\item {Grp. gram.:m.}
\end{itemize}
Espécie de musselina, que vem da Índia.
\section{Ornístomo}
\begin{itemize}
\item {Grp. gram.:m.}
\end{itemize}
\begin{itemize}
\item {Proveniência:(Do gr. \textunderscore ornis\textunderscore  + \textunderscore stoma\textunderscore )}
\end{itemize}
Gênero de insectos coleópteros, longicôrneos.
\section{Ornithídia}
\begin{itemize}
\item {Grp. gram.:f.}
\end{itemize}
Gênero de orchídeas.
\section{Ornithito}
\begin{itemize}
\item {Grp. gram.:m.}
\end{itemize}
\begin{itemize}
\item {Utilização:Miner.}
\end{itemize}
\begin{itemize}
\item {Proveniência:(Do gr. \textunderscore ornis\textunderscore , \textunderscore ornithos\textunderscore )}
\end{itemize}
Phosphato hydratado de cálcio.
\section{Ornithóbia}
\begin{itemize}
\item {Grp. gram.:f.}
\end{itemize}
\begin{itemize}
\item {Proveniência:(Do gr. \textunderscore ornis\textunderscore , \textunderscore ornithos\textunderscore  + \textunderscore bios\textunderscore )}
\end{itemize}
Gênero de insectos dípteros.
\section{Ornithóbio}
\begin{itemize}
\item {Grp. gram.:m.}
\end{itemize}
\begin{itemize}
\item {Proveniência:(Do gr. \textunderscore ornis\textunderscore , \textunderscore ornithos\textunderscore  + \textunderscore bios\textunderscore )}
\end{itemize}
Animálculo parasito dos cysnes.
\section{Ornithocéphalo}
\begin{itemize}
\item {Grp. gram.:m.}
\end{itemize}
\begin{itemize}
\item {Proveniência:(Do gr. \textunderscore ornithos\textunderscore  + \textunderscore kephale\textunderscore )}
\end{itemize}
Gênero de orchídeas.
\section{Ornithodélphyo}
\begin{itemize}
\item {Grp. gram.:adj.}
\end{itemize}
O mesmo ou melhor que \textunderscore ornithodelpho\textunderscore .
\section{Ornithodelpho}
\begin{itemize}
\item {Grp. gram.:m.  e  adj.}
\end{itemize}
\begin{itemize}
\item {Proveniência:(Do gr. \textunderscore ornis\textunderscore , \textunderscore ornithos\textunderscore  + \textunderscore delphus\textunderscore )}
\end{itemize}
O mesmo que \textunderscore monotremo\textunderscore .
\section{Ornithógalo}
\begin{itemize}
\item {Grp. gram.:m.}
\end{itemize}
\begin{itemize}
\item {Proveniência:(Gr. \textunderscore ornithogalon\textunderscore )}
\end{itemize}
Gênero de plantas liliáceas.
\section{Ornithoglosso}
\begin{itemize}
\item {Grp. gram.:m.}
\end{itemize}
\begin{itemize}
\item {Proveniência:(Do gr. \textunderscore ornis\textunderscore , \textunderscore ornithos\textunderscore  + \textunderscore glossa\textunderscore )}
\end{itemize}
Gênero de plantas colchicáceas.
\section{Ornithoídeo}
\begin{itemize}
\item {Grp. gram.:adj.}
\end{itemize}
\begin{itemize}
\item {Grp. gram.:M. pl.}
\end{itemize}
\begin{itemize}
\item {Proveniência:(Do gr. \textunderscore ornis\textunderscore , \textunderscore ornithos\textunderscore  + \textunderscore eidos\textunderscore )}
\end{itemize}
Que tem semelhança com uma ave.
Família de reptis, que têm semelhança com aves.
\section{Ornithólitho}
\begin{itemize}
\item {Grp. gram.:m.}
\end{itemize}
\begin{itemize}
\item {Proveniência:(Do gr. \textunderscore ornis\textunderscore , \textunderscore ornithos\textunderscore  + \textunderscore lithos\textunderscore )}
\end{itemize}
Ave fóssil.
Ossada fóssil de uma ave.
\section{Ornithologia}
\begin{itemize}
\item {Grp. gram.:f.}
\end{itemize}
Tratado á cêrca das aves.
(Cp. \textunderscore ornithólogo\textunderscore )
\section{Ornithológico}
\begin{itemize}
\item {Grp. gram.:adj.}
\end{itemize}
Relativo á ornithologia.
\section{Ornithologista}
\begin{itemize}
\item {Grp. gram.:m.  e  f.}
\end{itemize}
Pessôa, que se occupa de ornithologia.
\section{Ornithólogo}
\begin{itemize}
\item {Grp. gram.:m.}
\end{itemize}
\begin{itemize}
\item {Proveniência:(Do gr. \textunderscore ornis\textunderscore , \textunderscore ornithos\textunderscore  + \textunderscore logos\textunderscore )}
\end{itemize}
Aquelle que é versado em ornithologia.
\section{Ornithomancia}
\begin{itemize}
\item {Grp. gram.:f.}
\end{itemize}
\begin{itemize}
\item {Proveniência:(Do gr. \textunderscore ornis\textunderscore , \textunderscore ornithos\textunderscore , ave, e \textunderscore manteia\textunderscore , adivinhação)}
\end{itemize}
Adivinhação, por meio do canto ou do vôo das aves.
\section{Ornithomania}
\begin{itemize}
\item {Grp. gram.:f.}
\end{itemize}
\begin{itemize}
\item {Proveniência:(Do gr. \textunderscore ornis\textunderscore , \textunderscore ornithos\textunderscore  + \textunderscore mania\textunderscore )}
\end{itemize}
Affeição exaggerada ou mórbida ás aves.
\section{Ornithomântico}
\begin{itemize}
\item {Grp. gram.:adj.}
\end{itemize}
Relativo á ornithomancia.
\section{Ornithómya}
\begin{itemize}
\item {Grp. gram.:f.}
\end{itemize}
Gênero de insectos dípteros.
\section{Ornithomyzo}
\begin{itemize}
\item {Grp. gram.:m.  e  adj.}
\end{itemize}
\begin{itemize}
\item {Proveniência:(Do gr. \textunderscore ornis\textunderscore , \textunderscore ornithos\textunderscore  + \textunderscore muzein\textunderscore )}
\end{itemize}
Diz-se de um insecto, que suga o sangue das aves.
\section{Ornithóphilo}
\begin{itemize}
\item {Grp. gram.:m.  e  adj.}
\end{itemize}
\begin{itemize}
\item {Proveniência:(Do gr. \textunderscore ornis\textunderscore , \textunderscore ornithos\textunderscore  + \textunderscore philos\textunderscore )}
\end{itemize}
Indivíduo que se dedica, por prazer, á ornithologia.
\section{Ornithophonia}
\begin{itemize}
\item {Grp. gram.:f.}
\end{itemize}
\begin{itemize}
\item {Proveniência:(Do gr. \textunderscore ornis\textunderscore , \textunderscore ornithos\textunderscore  + \textunderscore phone\textunderscore )}
\end{itemize}
Imitação do canto das aves.
\section{Ornithophónio}
\begin{itemize}
\item {Grp. gram.:m.}
\end{itemize}
\begin{itemize}
\item {Proveniência:(Do gr. \textunderscore ornis\textunderscore , \textunderscore ornithos\textunderscore  + \textunderscore phone\textunderscore )}
\end{itemize}
Instrumento, que imita o canto das aves.
\section{Orníthopo}
\begin{itemize}
\item {Grp. gram.:m.}
\end{itemize}
\begin{itemize}
\item {Proveniência:(Do gr. \textunderscore ornis\textunderscore , \textunderscore ornithos\textunderscore  + \textunderscore pous\textunderscore )}
\end{itemize}
Gênero de plantas leguminosas.
\section{Ornithóptero}
\begin{itemize}
\item {Grp. gram.:m.}
\end{itemize}
\begin{itemize}
\item {Proveniência:(Do gr. \textunderscore ornis\textunderscore , \textunderscore ornithos\textunderscore  + \textunderscore pteron\textunderscore )}
\end{itemize}
Gênero de insectos lepidópteros diurnos.
\section{Ornithorrhinco}
\begin{itemize}
\item {Grp. gram.:m.}
\end{itemize}
\begin{itemize}
\item {Proveniência:(Do gr. \textunderscore ornis\textunderscore , \textunderscore ornithos\textunderscore  + \textunderscore rhunkos\textunderscore )}
\end{itemize}
Mammífero de bico córneo, da Austrália.
\section{Ornithoscopia}
\begin{itemize}
\item {Grp. gram.:f.}
\end{itemize}
\begin{itemize}
\item {Proveniência:(Do gr. \textunderscore ornis\textunderscore , \textunderscore ornithos\textunderscore  + \textunderscore skopein\textunderscore )}
\end{itemize}
Observação das aves, afim de se preverem successos futuros.
Ornithomancia.
\section{Ornithoscópico}
\begin{itemize}
\item {Grp. gram.:adj.}
\end{itemize}
Relativo á ornithoscopia.
\section{Ornithóscopo}
\begin{itemize}
\item {Grp. gram.:m.}
\end{itemize}
Aquelle que pratíca a ornithoscopia.
\section{Ornithotomia}
\begin{itemize}
\item {Grp. gram.:f.}
\end{itemize}
\begin{itemize}
\item {Proveniência:(Do gr. \textunderscore ornis\textunderscore , \textunderscore ornithos\textunderscore  + \textunderscore tome\textunderscore )}
\end{itemize}
Dissecção das aves.
\section{Ornithotrophia}
\begin{itemize}
\item {Grp. gram.:f.}
\end{itemize}
\begin{itemize}
\item {Proveniência:(Do gr. \textunderscore ornis\textunderscore , \textunderscore ornithos\textunderscore  + \textunderscore trophe\textunderscore )}
\end{itemize}
Criação de aves.
\section{Ornitídia}
\begin{itemize}
\item {Grp. gram.:f.}
\end{itemize}
Gênero de orquídeas.
\section{Ornitito}
\begin{itemize}
\item {Grp. gram.:m.}
\end{itemize}
\begin{itemize}
\item {Utilização:Miner.}
\end{itemize}
\begin{itemize}
\item {Proveniência:(Do gr. \textunderscore ornis\textunderscore , \textunderscore ornithos\textunderscore )}
\end{itemize}
Fosfato hidratado de cálcio.
\section{Ornitóbia}
\begin{itemize}
\item {Grp. gram.:f.}
\end{itemize}
\begin{itemize}
\item {Proveniência:(Do gr. \textunderscore ornis\textunderscore , \textunderscore ornithos\textunderscore  + \textunderscore bios\textunderscore )}
\end{itemize}
Gênero de insectos dípteros.
\section{Ornitóbio}
\begin{itemize}
\item {Grp. gram.:m.}
\end{itemize}
\begin{itemize}
\item {Proveniência:(Do gr. \textunderscore ornis\textunderscore , \textunderscore ornithos\textunderscore  + \textunderscore bios\textunderscore )}
\end{itemize}
Animálculo parasito dos cisnes.
\section{Ornitocéfalo}
\begin{itemize}
\item {Grp. gram.:m.}
\end{itemize}
\begin{itemize}
\item {Proveniência:(Do gr. \textunderscore ornithos\textunderscore  + \textunderscore kephale\textunderscore )}
\end{itemize}
Gênero de orquídeas.
\section{Ornitodélfio}
\begin{itemize}
\item {Grp. gram.:adj.}
\end{itemize}
O mesmo ou melhor que \textunderscore ornitodelfo\textunderscore .
\section{Ornitodelfo}
\begin{itemize}
\item {Grp. gram.:m.  e  adj.}
\end{itemize}
\begin{itemize}
\item {Proveniência:(Do gr. \textunderscore ornis\textunderscore , \textunderscore ornithos\textunderscore  + \textunderscore delphus\textunderscore )}
\end{itemize}
O mesmo que \textunderscore monotremo\textunderscore .
\section{Ornitógalo}
\begin{itemize}
\item {Grp. gram.:m.}
\end{itemize}
\begin{itemize}
\item {Proveniência:(Gr. \textunderscore ornithogalon\textunderscore )}
\end{itemize}
Gênero de plantas liliáceas.
\section{Ornitoglosso}
\begin{itemize}
\item {Grp. gram.:m.}
\end{itemize}
\begin{itemize}
\item {Proveniência:(Do gr. \textunderscore ornis\textunderscore , \textunderscore ornithos\textunderscore  + \textunderscore glossa\textunderscore )}
\end{itemize}
Gênero de plantas colquicáceas.
\section{Ornitoídeo}
\begin{itemize}
\item {Grp. gram.:adj.}
\end{itemize}
\begin{itemize}
\item {Grp. gram.:M. pl.}
\end{itemize}
\begin{itemize}
\item {Proveniência:(Do gr. \textunderscore ornis\textunderscore , \textunderscore ornithos\textunderscore  + \textunderscore eidos\textunderscore )}
\end{itemize}
Que tem semelhança com uma ave.
Família de reptis, que têm semelhança com aves.
\section{Ornitólito}
\begin{itemize}
\item {Grp. gram.:m.}
\end{itemize}
\begin{itemize}
\item {Proveniência:(Do gr. \textunderscore ornis\textunderscore , \textunderscore ornithos\textunderscore  + \textunderscore lithos\textunderscore )}
\end{itemize}
Ave fóssil.
Ossada fóssil de uma ave.
\section{Ornitologia}
\begin{itemize}
\item {Grp. gram.:f.}
\end{itemize}
Tratado á cêrca das aves.
(Cp. \textunderscore ornitólogo\textunderscore )
\section{Ornitológico}
\begin{itemize}
\item {Grp. gram.:adj.}
\end{itemize}
Relativo á ornitologia.
\section{Ornitologista}
\begin{itemize}
\item {Grp. gram.:m.  e  f.}
\end{itemize}
Pessôa, que se ocupa de ornitologia.
\section{Ornitólogo}
\begin{itemize}
\item {Grp. gram.:m.}
\end{itemize}
\begin{itemize}
\item {Proveniência:(Do gr. \textunderscore ornis\textunderscore , \textunderscore ornithos\textunderscore  + \textunderscore logos\textunderscore )}
\end{itemize}
Aquele que é versado em ornitologia.
\section{Ornitomancia}
\begin{itemize}
\item {Grp. gram.:f.}
\end{itemize}
\begin{itemize}
\item {Proveniência:(Do gr. \textunderscore ornis\textunderscore , \textunderscore ornithos\textunderscore , ave, e \textunderscore manteia\textunderscore , adivinhação)}
\end{itemize}
Adivinhação, por meio do canto ou do vôo das aves.
\section{Ornitomania}
\begin{itemize}
\item {Grp. gram.:f.}
\end{itemize}
\begin{itemize}
\item {Proveniência:(Do gr. \textunderscore ornis\textunderscore , \textunderscore ornithos\textunderscore  + \textunderscore mania\textunderscore )}
\end{itemize}
Afeição exagerada ou mórbida ás aves.
\section{Ornitomântico}
\begin{itemize}
\item {Grp. gram.:adj.}
\end{itemize}
Relativo á ornitomancia.
\section{Ornitómia}
\begin{itemize}
\item {Grp. gram.:f.}
\end{itemize}
Gênero de insectos dípteros.
\section{Ornitomizo}
\begin{itemize}
\item {Grp. gram.:m.  e  adj.}
\end{itemize}
\begin{itemize}
\item {Proveniência:(Do gr. \textunderscore ornis\textunderscore , \textunderscore ornithos\textunderscore  + \textunderscore muzein\textunderscore )}
\end{itemize}
Diz-se de um insecto, que suga o sangue das aves.
\section{Ornitófilo}
\begin{itemize}
\item {Grp. gram.:m.  e  adj.}
\end{itemize}
\begin{itemize}
\item {Proveniência:(Do gr. \textunderscore ornis\textunderscore , \textunderscore ornithos\textunderscore  + \textunderscore philos\textunderscore )}
\end{itemize}
Indivíduo que se dedica, por prazer, á ornitologia.
\section{Ornitofonia}
\begin{itemize}
\item {Grp. gram.:f.}
\end{itemize}
\begin{itemize}
\item {Proveniência:(Do gr. \textunderscore ornis\textunderscore , \textunderscore ornithos\textunderscore  + \textunderscore phone\textunderscore )}
\end{itemize}
Imitação do canto das aves.
\section{Ornitofónio}
\begin{itemize}
\item {Grp. gram.:m.}
\end{itemize}
\begin{itemize}
\item {Proveniência:(Do gr. \textunderscore ornis\textunderscore , \textunderscore ornithos\textunderscore  + \textunderscore phone\textunderscore )}
\end{itemize}
Instrumento, que imita o canto das aves.
\section{Ornítopo}
\begin{itemize}
\item {Grp. gram.:m.}
\end{itemize}
\begin{itemize}
\item {Proveniência:(Do gr. \textunderscore ornis\textunderscore , \textunderscore ornithos\textunderscore  + \textunderscore pous\textunderscore )}
\end{itemize}
Gênero de plantas leguminosas.
\section{Ornitóptero}
\begin{itemize}
\item {Grp. gram.:m.}
\end{itemize}
\begin{itemize}
\item {Proveniência:(Do gr. \textunderscore ornis\textunderscore , \textunderscore ornithos\textunderscore  + \textunderscore pteron\textunderscore )}
\end{itemize}
Gênero de insectos lepidópteros diurnos.
\section{Ornitorrinco}
\begin{itemize}
\item {Grp. gram.:m.}
\end{itemize}
\begin{itemize}
\item {Proveniência:(Do gr. \textunderscore ornis\textunderscore , \textunderscore ornithos\textunderscore  + \textunderscore rhunkos\textunderscore )}
\end{itemize}
Mamífero de bico córneo, da Austrália.
\section{Ornitoscopia}
\begin{itemize}
\item {Grp. gram.:f.}
\end{itemize}
\begin{itemize}
\item {Proveniência:(Do gr. \textunderscore ornis\textunderscore , \textunderscore ornithos\textunderscore  + \textunderscore skopein\textunderscore )}
\end{itemize}
Observação das aves, afim de se preverem successos futuros.
Ornitomancia.
\section{Ornitoscópico}
\begin{itemize}
\item {Grp. gram.:adj.}
\end{itemize}
Relativo á ornitoscopia.
\section{Ornitóscopo}
\begin{itemize}
\item {Grp. gram.:m.}
\end{itemize}
Aquele que pratíca a ornitoscopia.
\section{Ornitotomia}
\begin{itemize}
\item {Grp. gram.:f.}
\end{itemize}
\begin{itemize}
\item {Proveniência:(Do gr. \textunderscore ornis\textunderscore , \textunderscore ornithos\textunderscore  + \textunderscore tome\textunderscore )}
\end{itemize}
Dissecção das aves.
\section{Ornitotrofia}
\begin{itemize}
\item {Grp. gram.:f.}
\end{itemize}
\begin{itemize}
\item {Proveniência:(Do gr. \textunderscore ornis\textunderscore , \textunderscore ornithos\textunderscore  + \textunderscore trophe\textunderscore )}
\end{itemize}
Criação de aves.
\section{Ôro}
\begin{itemize}
\item {Grp. gram.:m.}
\end{itemize}
\begin{itemize}
\item {Utilização:Prov.}
\end{itemize}
\begin{itemize}
\item {Utilização:minh.}
\end{itemize}
A parte pôdre da madeira. (Colhido em Barcelos)
\section{Óro}
\begin{itemize}
\item {Grp. gram.:adj.}
\end{itemize}
\begin{itemize}
\item {Utilização:Prov.}
\end{itemize}
\begin{itemize}
\item {Utilização:minh.}
\end{itemize}
\textunderscore Óro anno\textunderscore , o anno passado.
\section{Orobalão}
\begin{itemize}
\item {Grp. gram.:m.}
\end{itemize}
Homem nobre de Malaca.
\section{Orobanca}
\begin{itemize}
\item {Grp. gram.:f.}
\end{itemize}
\begin{itemize}
\item {Proveniência:(Lat. \textunderscore orobanche\textunderscore )}
\end{itemize}
Planta parasita, de haste carnuda.
\section{Orobancha}
\begin{itemize}
\item {fónica:ca}
\end{itemize}
\begin{itemize}
\item {Grp. gram.:f.}
\end{itemize}
\begin{itemize}
\item {Proveniência:(Lat. \textunderscore orobanche\textunderscore )}
\end{itemize}
Planta parasita, de haste carnuda.
\section{Orobâncheas}
\begin{itemize}
\item {fónica:que}
\end{itemize}
\begin{itemize}
\item {Grp. gram.:f. pl.}
\end{itemize}
\begin{itemize}
\item {Proveniência:(De \textunderscore orobâncheo\textunderscore )}
\end{itemize}
Família de plantas, que tem por typo a orobancha.
\section{Orobâncheo}
\begin{itemize}
\item {fónica:que}
\end{itemize}
\begin{itemize}
\item {Grp. gram.:adj.}
\end{itemize}
Relativo ou semelhante á orobancha.
\section{Orobânqueas}
\begin{itemize}
\item {Grp. gram.:f. pl.}
\end{itemize}
\begin{itemize}
\item {Proveniência:(De \textunderscore orobânqueo\textunderscore )}
\end{itemize}
Família de plantas, que tem por tipo a orobanca.
\section{Orobânqueo}
\begin{itemize}
\item {Grp. gram.:adj.}
\end{itemize}
Relativo ou semelhante á orobanca.
\section{Orobita}
\begin{itemize}
\item {Grp. gram.:f.}
\end{itemize}
\begin{itemize}
\item {Proveniência:(Do gr. \textunderscore orobos\textunderscore )}
\end{itemize}
Certa concreção calcária espheroidal.
\section{Órobo}
\begin{itemize}
\item {Grp. gram.:m.}
\end{itemize}
\begin{itemize}
\item {Proveniência:(Lat. \textunderscore orobus\textunderscore )}
\end{itemize}
Gênero de plantas leguminosas papilionáceas, (\textunderscore vicia ervilia\textunderscore , Lin.), também conhecido por \textunderscore gero\textunderscore  e \textunderscore ervilha-de-pombo\textunderscore .
\section{Orobo-das-boticas}
\begin{itemize}
\item {Grp. gram.:m.}
\end{itemize}
O mesmo que \textunderscore ervilha-de-pomba\textunderscore .
\section{Oroça}
\begin{itemize}
\item {Grp. gram.:f.}
\end{itemize}
\begin{itemize}
\item {Utilização:Ant.}
\end{itemize}
Benefício ecclesiástico, em que eram providos certos indivíduos, revertendo as rendas para quem os apresentava.
(Cp. \textunderscore coroça\textunderscore )
\section{Oródino}
\begin{itemize}
\item {Grp. gram.:m.}
\end{itemize}
Gênero de insectos coleópteros tetrâmeros.
\section{Orofeia}
\begin{itemize}
\item {Grp. gram.:f.}
\end{itemize}
Gênero de plantas anonáceas.
\section{Orogenia}
\begin{itemize}
\item {Grp. gram.:f.}
\end{itemize}
\begin{itemize}
\item {Proveniência:(Do gr. \textunderscore oros\textunderscore  + \textunderscore genea\textunderscore )}
\end{itemize}
Formação das montanhas.
\section{Orogênico}
\begin{itemize}
\item {Grp. gram.:adj.}
\end{itemize}
\begin{itemize}
\item {Utilização:Geol.}
\end{itemize}
\begin{itemize}
\item {Proveniência:(De \textunderscore orogenia\textunderscore )}
\end{itemize}
Diz-se dos movimentos, que produzem o relêvo dos montes e modificam as desigualdades da superfície do solo.
\section{Orognosia}
\begin{itemize}
\item {Grp. gram.:f.}
\end{itemize}
\begin{itemize}
\item {Proveniência:(Do gr. \textunderscore oros\textunderscore  + \textunderscore gnosis\textunderscore )}
\end{itemize}
Descripção ou sciência da formação das montanhas.
\section{Orognóstico}
\begin{itemize}
\item {Grp. gram.:adj.}
\end{itemize}
Relativo á orognósia.
\section{Orografia}
\begin{itemize}
\item {Grp. gram.:f.}
\end{itemize}
Tratado ou descripção das montanhas.
(Cp. \textunderscore orógrapho\textunderscore )
\section{Orográfico}
\begin{itemize}
\item {Grp. gram.:adj.}
\end{itemize}
Relativo á orografia.
\section{Orógrafo}
\begin{itemize}
\item {Grp. gram.:m.}
\end{itemize}
\begin{itemize}
\item {Proveniência:(Do gr. \textunderscore oros\textunderscore  + \textunderscore graphein\textunderscore )}
\end{itemize}
Tratadista de orografia.
\section{Orographia}
\begin{itemize}
\item {Grp. gram.:f.}
\end{itemize}
Tratado ou descripção das montanhas.
(Cp. \textunderscore orógrapho\textunderscore )
\section{Orográphico}
\begin{itemize}
\item {Grp. gram.:adj.}
\end{itemize}
Relativo á orographia.
\section{Orógrapho}
\begin{itemize}
\item {Grp. gram.:m.}
\end{itemize}
\begin{itemize}
\item {Proveniência:(Do gr. \textunderscore oros\textunderscore  + \textunderscore graphein\textunderscore )}
\end{itemize}
Tratadista de orographia.
\section{Orohydrographia}
\begin{itemize}
\item {Grp. gram.:f.}
\end{itemize}
\begin{itemize}
\item {Proveniência:(De \textunderscore oros\textunderscore  gr. + \textunderscore hydrographia\textunderscore )}
\end{itemize}
Descripção de montanhas e correntes de água. Cf. Pacheco, \textunderscore Promptuário\textunderscore .
\section{Orohydrográphico}
\begin{itemize}
\item {Grp. gram.:adj.}
\end{itemize}
Relativo á orohydrographia.
\section{Oroidrografia}
\begin{itemize}
\item {fónica:oro-i}
\end{itemize}
\begin{itemize}
\item {Grp. gram.:f.}
\end{itemize}
\begin{itemize}
\item {Proveniência:(De \textunderscore oros\textunderscore  gr. + \textunderscore hydrographia\textunderscore )}
\end{itemize}
Descripção de montanhas e correntes de água. Cf. Pacheco, \textunderscore Promptuário\textunderscore .
\section{Oroidrográfico}
\begin{itemize}
\item {fónica:oro-i}
\end{itemize}
\begin{itemize}
\item {Grp. gram.:adj.}
\end{itemize}
Relativo á orohidrografia.
\section{Orologia}
\begin{itemize}
\item {Grp. gram.:f.}
\end{itemize}
\begin{itemize}
\item {Proveniência:(Do gr. \textunderscore oros\textunderscore  + \textunderscore logos\textunderscore )}
\end{itemize}
O mesmo que \textunderscore orognosia\textunderscore .
\section{Orológico}
\begin{itemize}
\item {Grp. gram.:adj.}
\end{itemize}
Relativo á orologia.
\section{Oronciáceas}
\begin{itemize}
\item {Grp. gram.:f. pl.}
\end{itemize}
Tribo de plantas, a que serve de typo o orôncio.
\section{Orôncio}
\begin{itemize}
\item {Grp. gram.:m.}
\end{itemize}
Gênero de plantas aráceas.
\section{Oroneta}
\begin{itemize}
\item {fónica:nê}
\end{itemize}
\begin{itemize}
\item {Grp. gram.:f.}
\end{itemize}
Rede, com que os levantinos pescam o peixe voador.
\section{Oropécio}
\begin{itemize}
\item {Grp. gram.:m.}
\end{itemize}
Gênero de plantas gramíneas.
\section{Oropheia}
\begin{itemize}
\item {Grp. gram.:f.}
\end{itemize}
Gênero de plantas anonáceas.
\section{Orosfera}
\begin{itemize}
\item {Grp. gram.:f.}
\end{itemize}
\begin{itemize}
\item {Proveniência:(Do gr. \textunderscore oros\textunderscore  + \textunderscore sphaira\textunderscore )}
\end{itemize}
A parte sólida da superfície do glôbo terrestre.
\section{Orosférico}
\begin{itemize}
\item {Grp. gram.:adj.}
\end{itemize}
Relativo á orosfera.
\section{Orosphera}
\begin{itemize}
\item {Grp. gram.:f.}
\end{itemize}
\begin{itemize}
\item {Proveniência:(Do gr. \textunderscore oros\textunderscore  + \textunderscore sphaira\textunderscore )}
\end{itemize}
A parte sólida da superfície do glôbo terrestre.
\section{Orosphérico}
\begin{itemize}
\item {Grp. gram.:adj.}
\end{itemize}
Relativo á orosphera.
\section{Orphaico}
\begin{itemize}
\item {Grp. gram.:adj.}
\end{itemize}
O mesmo que \textunderscore orpheico\textunderscore .
\section{Orphásio}
\begin{itemize}
\item {Grp. gram.:m.}
\end{itemize}
Antigo instrumento de cordas, parecido ao alaúde e que foi usado no século XVII.
\section{Orpheão}
\begin{itemize}
\item {Grp. gram.:m.}
\end{itemize}
\begin{itemize}
\item {Proveniência:(Fr. \textunderscore orphéon\textunderscore , de \textunderscore Orpheu\textunderscore , n. p.)}
\end{itemize}
Escola de canto.
Sociedade, cujos membros se dedicam á prática da música vocal e do canto coral, sem acompanhamento.
\section{Orpheico}
\begin{itemize}
\item {Grp. gram.:adj.}
\end{itemize}
\begin{itemize}
\item {Proveniência:(Lat. \textunderscore orpheicus\textunderscore )}
\end{itemize}
Relativo á música.
\section{Orpheónico}
\begin{itemize}
\item {Grp. gram.:adj.}
\end{itemize}
Relativo ao orpheão.
\section{Orpheonista}
\begin{itemize}
\item {Grp. gram.:m.}
\end{itemize}
Membro de um orpheão.
\section{Orpheonístico}
\begin{itemize}
\item {Grp. gram.:adj.}
\end{itemize}
\begin{itemize}
\item {Utilização:Neol.}
\end{itemize}
Relativo a orpheonista.
\section{Orpheoteleste}
\begin{itemize}
\item {Grp. gram.:m.}
\end{itemize}
Aquelle que interpretava os mystérios introduzidos na Grécia por Orpheu.
\section{Órphia}
\begin{itemize}
\item {Grp. gram.:f.}
\end{itemize}
Gênero de peixes malacopterýgios abdominaes.
\section{Ôrphio}
\begin{itemize}
\item {Grp. gram.:m.}
\end{itemize}
Gênero de plantas gencianáceas.
\section{Orphneia}
\begin{itemize}
\item {Grp. gram.:f.}
\end{itemize}
Gênero de crustáceos decápodes.
\section{Orquestra}
\begin{itemize}
\item {Grp. gram.:f.}
\end{itemize}
\begin{itemize}
\item {Utilização:Poét.}
\end{itemize}
\begin{itemize}
\item {Proveniência:(Lat. \textunderscore orchestra\textunderscore )}
\end{itemize}
Lugar, ocupado pelos músicos instrumentistas, num teatro ou numa festa qualquer.
Conjunto de músicos instrumentistas, que executam as peças de música ou acompanham o canto.
Parte instrumental de uma partitura.
Conjunto de sons harmoniosos.
\section{Orquestração}
\begin{itemize}
\item {Grp. gram.:f.}
\end{itemize}
Acto ou arte de orquestrar.
\section{Orquestral}
\begin{itemize}
\item {Grp. gram.:adj.}
\end{itemize}
Relativo a orquestra.
\section{Orquestrar}
\begin{itemize}
\item {Grp. gram.:v. t.}
\end{itemize}
Dispor ou organizar (peça musical), para sêr executada por orquestra.
\section{Orquestrino}
\begin{itemize}
\item {Grp. gram.:m.}
\end{itemize}
\begin{itemize}
\item {Proveniência:(De \textunderscore orquestra\textunderscore )}
\end{itemize}
Piano que imitava a rabeca, a viola e o violoncelo.
\section{Orquialgia}
\begin{itemize}
\item {Grp. gram.:f.}
\end{itemize}
\begin{itemize}
\item {Utilização:Med.}
\end{itemize}
\begin{itemize}
\item {Proveniência:(Do gr. \textunderscore orkhis\textunderscore  + \textunderscore algos\textunderscore )}
\end{itemize}
Neuralgia do testículo.
\section{Órquide}
\begin{itemize}
\item {Grp. gram.:f.}
\end{itemize}
\begin{itemize}
\item {Proveniência:(Do gr. \textunderscore orkhis\textunderscore )}
\end{itemize}
Gênero de plantas, que serve de tipo ás orquídeas.
\section{Orquidáceas}
\begin{itemize}
\item {Grp. gram.:f. pl.}
\end{itemize}
Família de plantas, o mesmo ou melhor que \textunderscore orquídeas\textunderscore .
\section{Orquideáceo}
\begin{itemize}
\item {Grp. gram.:adj.}
\end{itemize}
\begin{itemize}
\item {Utilização:Bot.}
\end{itemize}
\begin{itemize}
\item {Proveniência:(De \textunderscore órquide\textunderscore )}
\end{itemize}
Diz-se das raizes formadas de dois tubérculos colados como as das orquídeas.
\section{Orquídeas}
\begin{itemize}
\item {Grp. gram.:f. pl.}
\end{itemize}
\begin{itemize}
\item {Proveniência:(Do gr. \textunderscore orkhis\textunderscore  + \textunderscore eidos\textunderscore )}
\end{itemize}
Família de plantas monocotiledóneas e tuberculosas.
\section{Orquidifloro}
\begin{itemize}
\item {Grp. gram.:adj.}
\end{itemize}
\begin{itemize}
\item {Utilização:Bot.}
\end{itemize}
Que dá flôres semelhantes ás das orquídeas.
\section{Orquidófilo}
\begin{itemize}
\item {Grp. gram.:m.  e  adj.}
\end{itemize}
Amador ou coleccionador de orquídeas.
\section{Orquiocele}
\begin{itemize}
\item {Grp. gram.:m.}
\end{itemize}
\begin{itemize}
\item {Proveniência:(Do gr. \textunderscore orkhis\textunderscore  + \textunderscore kele\textunderscore )}
\end{itemize}
Tumor no testículo.
\section{Orquiotomia}
\begin{itemize}
\item {Grp. gram.:f.}
\end{itemize}
Extracção cirúrgica de um ou dos dois testículos.
(Cp. \textunderscore orchiótomo\textunderscore )
\section{Orquiotómico}
\begin{itemize}
\item {Grp. gram.:adj.}
\end{itemize}
Relativo á orquiotomia.
\section{Orquiótomo}
\begin{itemize}
\item {Grp. gram.:m.}
\end{itemize}
\begin{itemize}
\item {Proveniência:(Do gr. \textunderscore orkhis\textunderscore  + \textunderscore tome\textunderscore )}
\end{itemize}
Instrumento, com que se pratica a orquiotomia.
\section{Orquípeda}
\begin{itemize}
\item {Grp. gram.:f.}
\end{itemize}
\begin{itemize}
\item {Proveniência:(Do lat. \textunderscore orchis\textunderscore  + \textunderscore pes\textunderscore , \textunderscore pedis\textunderscore )}
\end{itemize}
Gênero de plantas apocíneas.
\section{Orquita}
\begin{itemize}
\item {Grp. gram.:f.}
\end{itemize}
\begin{itemize}
\item {Proveniência:(Lat. \textunderscore orchita\textunderscore )}
\end{itemize}
Variedade de azeitona, conhecida dos antigos.
\section{Orquite}
\begin{itemize}
\item {Grp. gram.:f.}
\end{itemize}
\begin{itemize}
\item {Proveniência:(Do gr. \textunderscore orkhis\textunderscore )}
\end{itemize}
Inflamação de um ou dos dois testículos.
\section{Orquítico}
\begin{itemize}
\item {Grp. gram.:adj.}
\end{itemize}
Relativo á orquite.
Aplicável contra a orquite.
\section{Orraca}
\begin{itemize}
\item {Grp. gram.:f.}
\end{itemize}
\begin{itemize}
\item {Utilização:Des.}
\end{itemize}
Bebida alcoólica da Ásia; aguardente de coco com vinho de palma. Cf. \textunderscore Tombo do Estado da Índia\textunderscore , (\textunderscore passim\textunderscore )
\section{Orreiro}
\begin{itemize}
\item {Grp. gram.:m.}
\end{itemize}
\begin{itemize}
\item {Utilização:Prov.}
\end{itemize}
\begin{itemize}
\item {Utilização:alg.}
\end{itemize}
Trave, que entra numa cavidade das lages que forram o poço de certos moínhos. Cf. \textunderscore Portugalia\textunderscore , I, 388.
\section{Orreta}
\begin{itemize}
\item {fónica:rê}
\end{itemize}
\begin{itemize}
\item {Grp. gram.:f.}
\end{itemize}
\begin{itemize}
\item {Utilização:Prov.}
\end{itemize}
\begin{itemize}
\item {Utilização:ant.}
\end{itemize}
\begin{itemize}
\item {Utilização:Prov.}
\end{itemize}
\begin{itemize}
\item {Utilização:trasm.}
\end{itemize}
Valle profundo, que dá estreito espaço para plantações.
Atalho, através dos campos.
(Refl. do lat. \textunderscore ora\textunderscore ?)
\section{Orseta}
\begin{itemize}
\item {fónica:sê}
\end{itemize}
\begin{itemize}
\item {Grp. gram.:f.}
\end{itemize}
Espécie de tecido ordinário, fabricado na Hollanda.
\section{Orsínia}
\begin{itemize}
\item {Grp. gram.:f.}
\end{itemize}
\begin{itemize}
\item {Proveniência:(De \textunderscore Orsiní\textunderscore , n. p.)}
\end{itemize}
Planta brasileira, da fam. das compostas.
\section{Ortagorismo}
\begin{itemize}
\item {Grp. gram.:m.}
\end{itemize}
Designação científica do peixe-lua.
\section{Ortantera}
\begin{itemize}
\item {Grp. gram.:f.}
\end{itemize}
\begin{itemize}
\item {Proveniência:(Do gr. \textunderscore orthos\textunderscore  + \textunderscore antheros\textunderscore )}
\end{itemize}
Gênero de plantas asclepiadacas.
\section{Ortaptodáctilos}
\begin{itemize}
\item {Grp. gram.:m. pl.}
\end{itemize}
\begin{itemize}
\item {Proveniência:(Do gr. \textunderscore orthos\textunderscore  + \textunderscore aptein\textunderscore  + \textunderscore daktulos\textunderscore )}
\end{itemize}
Família de aves de rapina, que compreende as que têm garras mais robustas.
\section{Ortégia}
\begin{itemize}
\item {Grp. gram.:f.}
\end{itemize}
Gênero de plantas caryophylláceas.
\section{Orthagorismo}
\begin{itemize}
\item {Grp. gram.:m.}
\end{itemize}
Designação scientífica do peixe-lua.
\section{Orthanthera}
\begin{itemize}
\item {Grp. gram.:f.}
\end{itemize}
\begin{itemize}
\item {Proveniência:(Do gr. \textunderscore orthos\textunderscore  + \textunderscore antheros\textunderscore )}
\end{itemize}
Gênero de plantas asclepiadacas.
\section{Orthaptodáctylos}
\begin{itemize}
\item {Grp. gram.:m. pl.}
\end{itemize}
\begin{itemize}
\item {Proveniência:(Do gr. \textunderscore orthos\textunderscore  + \textunderscore aptein\textunderscore  + \textunderscore daktulos\textunderscore )}
\end{itemize}
Família de aves de rapina, que comprehende as que têm garras mais robustas.
\section{Orthito}
\begin{itemize}
\item {Grp. gram.:m.}
\end{itemize}
\begin{itemize}
\item {Utilização:Miner.}
\end{itemize}
\begin{itemize}
\item {Proveniência:(Do gr. \textunderscore orthos\textunderscore )}
\end{itemize}
Silicato hydratado de alumínio, cálcio, ferro e cério, com lanthânio e didýmio. Cf. R. Galvão, \textunderscore Vocab.\textunderscore 
\section{Ortho...}
\begin{itemize}
\item {Grp. gram.:pref.}
\end{itemize}
\begin{itemize}
\item {Proveniência:(Do gr. \textunderscore orthos\textunderscore )}
\end{itemize}
(designativo de \textunderscore direito\textunderscore , \textunderscore recto\textunderscore , \textunderscore exacto\textunderscore )
\section{Orthobásico}
\begin{itemize}
\item {Grp. gram.:adj.}
\end{itemize}
\begin{itemize}
\item {Utilização:Miner.}
\end{itemize}
\begin{itemize}
\item {Proveniência:(De \textunderscore ortho...\textunderscore  + \textunderscore base\textunderscore )}
\end{itemize}
Diz-se das substâncias, cujos crystaes têm coordenadas orthogonaes.
\section{Orthócera}
\begin{itemize}
\item {Grp. gram.:f.}
\end{itemize}
\begin{itemize}
\item {Proveniência:(Do gr. \textunderscore orthos\textunderscore  + \textunderscore keras\textunderscore )}
\end{itemize}
Gênero de plantas gramíneas.
\section{Orthóclada}
\begin{itemize}
\item {Grp. gram.:f.}
\end{itemize}
Gênero de plantas gramíneas.
\section{Orthoclásio}
\begin{itemize}
\item {Grp. gram.:f.}
\end{itemize}
\begin{itemize}
\item {Utilização:Geol.}
\end{itemize}
\begin{itemize}
\item {Proveniência:(Do gr. \textunderscore orthos\textunderscore  + \textunderscore klasis\textunderscore )}
\end{itemize}
Mineral do grupo dos feldspathos.
\section{Orthocolimbos}
\begin{itemize}
\item {Grp. gram.:m. pl.}
\end{itemize}
Família de aves aquáticas, que comprehende as que se conservam muito tempo debaixo da água.
\section{Orthocólon}
\begin{itemize}
\item {Grp. gram.:m.}
\end{itemize}
\begin{itemize}
\item {Utilização:Med.}
\end{itemize}
Rigidez de uma articulação, que não permitte moverem-se as peças articuladas.
\section{Orthodáctylo}
\begin{itemize}
\item {Grp. gram.:adj.}
\end{itemize}
\begin{itemize}
\item {Utilização:Zool.}
\end{itemize}
\begin{itemize}
\item {Proveniência:(Do gr. \textunderscore orthos\textunderscore  + \textunderscore dakulos\textunderscore )}
\end{itemize}
Que tem os dedos direitos.
\section{Orthódano}
\begin{itemize}
\item {Grp. gram.:m.}
\end{itemize}
Gênero de plantas leguminosas.
\section{Orthodiagonal}
\begin{itemize}
\item {Grp. gram.:f.  e  adj.}
\end{itemize}
\begin{itemize}
\item {Utilização:Miner.}
\end{itemize}
\begin{itemize}
\item {Proveniência:(De \textunderscore ortho...\textunderscore  + \textunderscore diagonal\textunderscore )}
\end{itemize}
Diz-se de um dos eixos dos crystaes do systema monoclínico. Cf. G. Guimarães, \textunderscore Geologia\textunderscore , 51.
\section{Orthodoma}
\begin{itemize}
\item {Grp. gram.:m.}
\end{itemize}
\begin{itemize}
\item {Utilização:Geol.}
\end{itemize}
\begin{itemize}
\item {Proveniência:(Do gr. \textunderscore orthos\textunderscore  + \textunderscore doma\textunderscore )}
\end{itemize}
Uma das fórmas holoédricas dos mineraes, que constitue prisma transversal.
\section{Orthodonte}
\begin{itemize}
\item {Grp. gram.:adj.}
\end{itemize}
\begin{itemize}
\item {Grp. gram.:M. pl.}
\end{itemize}
\begin{itemize}
\item {Proveniência:(Do gr. \textunderscore orthos\textunderscore  + \textunderscore odous\textunderscore )}
\end{itemize}
Que tem os dentes direitos.
Gênero de musgos, que vivem nos troncos das árvores da África austral.
\section{Orthódoro}
\begin{itemize}
\item {Grp. gram.:m.}
\end{itemize}
\begin{itemize}
\item {Proveniência:(Gr. \textunderscore orthodoron\textunderscore )}
\end{itemize}
Antiga medida linear entre os Gregos, equivalente a onze dedos.
\section{Orictero}
\begin{itemize}
\item {Grp. gram.:m.}
\end{itemize}
\begin{itemize}
\item {Proveniência:(Do gr. \textunderscore oruktes\textunderscore )}
\end{itemize}
Gênero de mamíferos roedores, que minam a terra como as toupeiras.
\section{Oricterope}
\begin{itemize}
\item {Grp. gram.:m.}
\end{itemize}
\begin{itemize}
\item {Proveniência:(Do gr. \textunderscore orukter\textunderscore  + \textunderscore ops\textunderscore )}
\end{itemize}
Quadrúpede sul-africano, que devora as formigas e é semelhante ao tamanduá.
\section{Oricto...}
\begin{itemize}
\item {Grp. gram.:pref.}
\end{itemize}
\begin{itemize}
\item {Proveniência:(Do gr. \textunderscore oruktos\textunderscore )}
\end{itemize}
(designativo de fóssil ou de mineral)
\section{Orictogeologia}
\begin{itemize}
\item {Grp. gram.:f.}
\end{itemize}
\begin{itemize}
\item {Proveniência:(De \textunderscore oricto...\textunderscore  + \textunderscore geologia\textunderscore )}
\end{itemize}
Parte da História Natural, que trata da disposição dos mineraes na terra.
\section{Orictognosia}
\begin{itemize}
\item {Grp. gram.:f.}
\end{itemize}
\begin{itemize}
\item {Proveniência:(Do gr. \textunderscore oruktos\textunderscore  + \textunderscore gnosis\textunderscore )}
\end{itemize}
Parte da História Natural, que ensina a conhecer e a distinguir os metaes.
\section{Orictognosta}
\begin{itemize}
\item {Grp. gram.:m.}
\end{itemize}
Aquele que trata da orictognosia.
\section{Orictografia}
\begin{itemize}
\item {Grp. gram.:f.}
\end{itemize}
\begin{itemize}
\item {Proveniência:(De \textunderscore orictógrafo\textunderscore )}
\end{itemize}
Descripção dos fósseis.
\section{Orictográfico}
\begin{itemize}
\item {Grp. gram.:adj.}
\end{itemize}
Relativo á orictografia.
\section{Orictógrafo}
\begin{itemize}
\item {Grp. gram.:m.}
\end{itemize}
\begin{itemize}
\item {Proveniência:(Do gr. \textunderscore oruktos\textunderscore  + \textunderscore graphein\textunderscore )}
\end{itemize}
Aquele que se occupa de orictografia.
\section{Orictologia}
\begin{itemize}
\item {Grp. gram.:f.}
\end{itemize}
História dos fósseis; tratado á cêrca dos fósseis.
(Cp. \textunderscore orictólogo\textunderscore )
\section{Orictológico}
\begin{itemize}
\item {Grp. gram.:adj.}
\end{itemize}
Relativo á orictologia.
\section{Orictologista}
\begin{itemize}
\item {Grp. gram.:m.  e  f.}
\end{itemize}
Pessôa, que se dedica á orictologia.
\section{Orictólogo}
\begin{itemize}
\item {Grp. gram.:m.}
\end{itemize}
\begin{itemize}
\item {Proveniência:(Do gr. \textunderscore oruktos\textunderscore  + \textunderscore logos\textunderscore )}
\end{itemize}
Aquele que é versado em orictologia.
\section{Orictotecnia}
\begin{itemize}
\item {Grp. gram.:f.}
\end{itemize}
\begin{itemize}
\item {Proveniência:(Do gr. \textunderscore oruktos\textunderscore  + \textunderscore tekhne\textunderscore )}
\end{itemize}
Estudo dos meios, com que se podem procurar as substâncias mineraes, necessárias aos usos da vida.
\section{Orizário}
\begin{itemize}
\item {Grp. gram.:m.}
\end{itemize}
\begin{itemize}
\item {Proveniência:(Do lat. \textunderscore oryza\textunderscore , arroz)}
\end{itemize}
Gênero de polipeiros fósseis.
\section{Orízeas}
\begin{itemize}
\item {Grp. gram.:f. pl.}
\end{itemize}
\begin{itemize}
\item {Proveniência:(Do lat. \textunderscore oryza\textunderscore )}
\end{itemize}
Tríbo de plantas gramíneas, que têm por tipo o arroz.
\section{Orizícola}
\begin{itemize}
\item {Grp. gram.:adj.}
\end{itemize}
\begin{itemize}
\item {Proveniência:(Do lat. \textunderscore oryza\textunderscore  + \textunderscore colere\textunderscore )}
\end{itemize}
Relativo á cultura do arroz.
\section{Orizicultura}
\begin{itemize}
\item {Grp. gram.:f.}
\end{itemize}
\begin{itemize}
\item {Proveniência:(Do lat. \textunderscore oryza\textunderscore  + \textunderscore cultura\textunderscore )}
\end{itemize}
Cultura do arroz.
\section{Oriziforme}
\begin{itemize}
\item {Grp. gram.:adj.}
\end{itemize}
\begin{itemize}
\item {Proveniência:(Do lat. \textunderscore oryza\textunderscore  + \textunderscore forma\textunderscore )}
\end{itemize}
Que tem fórma de arroz.
\section{Orizívoro}
\begin{itemize}
\item {Grp. gram.:adj.}
\end{itemize}
\begin{itemize}
\item {Proveniência:(Do lat. \textunderscore oryza\textunderscore  + \textunderscore vorare\textunderscore )}
\end{itemize}
Diz-se dos animaes, que se alimentam de arroz.
\section{Orthodoxamente}
\begin{itemize}
\item {fónica:csa}
\end{itemize}
\begin{itemize}
\item {Grp. gram.:adv.}
\end{itemize}
De modo orthodoxo.
Com orthodoxia; segundo a doutrina que se considera orthodoxa.
\section{Orthodoxia}
\begin{itemize}
\item {fónica:csi}
\end{itemize}
\begin{itemize}
\item {Grp. gram.:f.}
\end{itemize}
Qualidade do que é orthodoxo.
Doutrina religiosa, considerada como verdadeira.
Conformidade com essa doutrina.
Conformidade com a doutrina da Igreja.
\section{Orthodoxo}
\begin{itemize}
\item {fónica:cso}
\end{itemize}
\begin{itemize}
\item {Grp. gram.:adj.}
\end{itemize}
\begin{itemize}
\item {Grp. gram.:M.}
\end{itemize}
\begin{itemize}
\item {Utilização:Ext.}
\end{itemize}
\begin{itemize}
\item {Proveniência:(Lat. \textunderscore orthodoxus\textunderscore )}
\end{itemize}
Relativo á orthodoxia.
Que está de acôrdo com certos systemas ou com ideias geralmente recebidas.
Indivíduo que segue a doutrina religiosa, considerada como verdadeira.
Aquelle que segue qualquer doutrina estabelecida.
\section{Orthodoxographia}
\begin{itemize}
\item {fónica:cso}
\end{itemize}
\begin{itemize}
\item {Grp. gram.:f.}
\end{itemize}
\begin{itemize}
\item {Proveniência:(Do gr. \textunderscore orthos\textunderscore  + \textunderscore doxa\textunderscore  + \textunderscore graphein\textunderscore )}
\end{itemize}
Tratado á cêrca dos dogmas cathólicos.
\section{Orthodoxográphico}
\begin{itemize}
\item {fónica:cso}
\end{itemize}
\begin{itemize}
\item {Grp. gram.:adj.}
\end{itemize}
Relativo á orthodoxographia.
\section{Orthodoxógrapho}
\begin{itemize}
\item {fónica:csó}
\end{itemize}
\begin{itemize}
\item {Grp. gram.:M.}
\end{itemize}
Aquelle que escreve sôbre a orthodoxographia.
\section{Orthodromia}
\begin{itemize}
\item {Grp. gram.:f.}
\end{itemize}
\begin{itemize}
\item {Utilização:Náut.}
\end{itemize}
\begin{itemize}
\item {Utilização:Topogr.}
\end{itemize}
\begin{itemize}
\item {Proveniência:(Do gr. \textunderscore orthos\textunderscore  + \textunderscore dromos\textunderscore )}
\end{itemize}
Linha mais curta entre os dois pontos extremos da rota de um navio.
A distância mais curta; o traçado recto.
\section{Orthodrómico}
\begin{itemize}
\item {Grp. gram.:adj.}
\end{itemize}
Relativo á orthodromia.
\section{Orthoédrico}
\begin{itemize}
\item {Grp. gram.:adj.}
\end{itemize}
\begin{itemize}
\item {Utilização:Miner.}
\end{itemize}
\begin{itemize}
\item {Proveniência:(Do gr. \textunderscore orthos\textunderscore  + \textunderscore edra\textunderscore )}
\end{itemize}
Diz-se dos crystaes, cujos planos coordenados são perpendiculares entre si.
\section{Orthoepia}
\begin{itemize}
\item {Grp. gram.:f.}
\end{itemize}
\begin{itemize}
\item {Proveniência:(Do gr. \textunderscore orthoepeia\textunderscore )}
\end{itemize}
Pronúncia correcta.
Parte da Grammática, que ensina a bôa pronúncia.--Usa-se a prosódia \textunderscore ortoépia\textunderscore , mas não é rigorosa.
\section{Orthoépico}
\begin{itemize}
\item {Grp. gram.:adj.}
\end{itemize}
Relativo á orthoépia.
\section{Orthofórmio}
\begin{itemize}
\item {Grp. gram.:m.}
\end{itemize}
Pó branco, insipido e incolor, de propriedades anesthésicas, descoberto recentemente, (1898), em Alemanha.
\section{Orthognathismo}
\begin{itemize}
\item {Grp. gram.:m.}
\end{itemize}
\begin{itemize}
\item {Utilização:Zool.}
\end{itemize}
\begin{itemize}
\item {Proveniência:(De \textunderscore orthognatho\textunderscore )}
\end{itemize}
Qualidade de têr os queixos aprumados com a parte superior do rosto.
\section{Orthógnatho}
\begin{itemize}
\item {Grp. gram.:adj.}
\end{itemize}
\begin{itemize}
\item {Proveniência:(Do gr. \textunderscore orthos\textunderscore  + \textunderscore gnathos\textunderscore )}
\end{itemize}
Diz-se da raça humana, cujo bôrdo alveolar e cujos dentes do maxillar superior são um pouco oblíquos para deante.
\section{Orthogonal}
\begin{itemize}
\item {Grp. gram.:adj.}
\end{itemize}
\begin{itemize}
\item {Utilização:Geom.}
\end{itemize}
Diz-se da projecção em que cada linha, que projecta um ponto da figura, é perpendicular ao plano de projecção.
(Cp. \textunderscore orthógono\textunderscore )
\section{Orthogonalmente}
\begin{itemize}
\item {Grp. gram.:adv.}
\end{itemize}
De modo orthogonal.
Perpendicularmente.
\section{Orthógono}
\begin{itemize}
\item {Grp. gram.:adj.}
\end{itemize}
\begin{itemize}
\item {Utilização:Geom.}
\end{itemize}
\begin{itemize}
\item {Proveniência:(Lat. \textunderscore orthogonus\textunderscore )}
\end{itemize}
Diz-se da linha que fórma com outra ângulo recto.
Perpendicular.--A pronúncia exacta seria ortogóno, mas não se usa.
\section{Orthographar}
\begin{itemize}
\item {Grp. gram.:v. t.}
\end{itemize}
Escrever, segundo as regras orthográphicas.
Empregar orthographia particular ou especial a respeito de: \textunderscore orthographar sonicamente um vocábulo\textunderscore .
Applicar a orthographia a.
(Cp. \textunderscore orthógrapho\textunderscore )
\section{Orthographia}
\begin{itemize}
\item {Grp. gram.:f.}
\end{itemize}
\begin{itemize}
\item {Utilização:Geom.}
\end{itemize}
\begin{itemize}
\item {Proveniência:(Lat. \textunderscore orthographia\textunderscore )}
\end{itemize}
Acto e modo de escrever correctamente as palavras de uma língua.
Qualquer maneira de escrever palavras.
Representação de um edificio.
Perfil de uma construcção.
Projecção orthogonal.
\section{Orthographicamente}
\begin{itemize}
\item {Grp. gram.:adv.}
\end{itemize}
De modo orthográphico.
Relativamente á orthographia.
\section{Orthográphico}
\begin{itemize}
\item {Grp. gram.:adj.}
\end{itemize}
Relativo á orthographia, considerada grammatical e geometricamente.
\section{Orthographista}
\begin{itemize}
\item {Grp. gram.:m. ,  f.  e  adj.}
\end{itemize}
Pessôa, que trata de orthographia.
\section{Orthógrapho}
\begin{itemize}
\item {Grp. gram.:m.}
\end{itemize}
\begin{itemize}
\item {Proveniência:(Lat. \textunderscore orthographus\textunderscore )}
\end{itemize}
Aquelle que é versado nas leis orthográphicas.
\section{Ortholexia}
\begin{itemize}
\item {fónica:csi}
\end{itemize}
\begin{itemize}
\item {Grp. gram.:f.}
\end{itemize}
\begin{itemize}
\item {Proveniência:(Do gr. \textunderscore orthos\textunderscore  + \textunderscore lexis\textunderscore )}
\end{itemize}
Fórma correcta de se exprimir; bôa dicção.
\section{Orthólitho}
\begin{itemize}
\item {Grp. gram.:m.}
\end{itemize}
\begin{itemize}
\item {Utilização:Miner.}
\end{itemize}
Rocha microgranítica, composta principalmente de ortósio, mica e hornblenda.
\section{Orthologia}
\begin{itemize}
\item {Grp. gram.:f.}
\end{itemize}
\begin{itemize}
\item {Utilização:Bras}
\end{itemize}
\begin{itemize}
\item {Proveniência:(Do gr. \textunderscore orthos\textunderscore  + \textunderscore logos\textunderscore )}
\end{itemize}
Arte de falar correctamente.
O mesmo que \textunderscore orthoepia\textunderscore .
O mesmo que \textunderscore occultismo\textunderscore . Cf. \textunderscore Notícia\textunderscore , do Rio, de 30-VIII-900.
\section{Orthológico}
\begin{itemize}
\item {Grp. gram.:adj.}
\end{itemize}
Relativo á orthologia.
\section{Orthómega}
\begin{itemize}
\item {Grp. gram.:m.}
\end{itemize}
\begin{itemize}
\item {Proveniência:(Do gr. \textunderscore orthos\textunderscore  + \textunderscore megas\textunderscore )}
\end{itemize}
Gênero de insectos coleópteros longicórneos.
\section{Órthomo}
\begin{itemize}
\item {Grp. gram.:m.}
\end{itemize}
Gênero de insectos coleópteros pentâmeros.
\section{Orthomorphia}
\begin{itemize}
\item {Grp. gram.:f.}
\end{itemize}
\begin{itemize}
\item {Proveniência:(Do gr. \textunderscore orthos\textunderscore  + \textunderscore morphe\textunderscore )}
\end{itemize}
Arte de restituír a uma parte do corpo humano a sua fórma natural.
\section{Orthomórphico}
\begin{itemize}
\item {Grp. gram.:adj.}
\end{itemize}
Relativo a orthomorphia.
\section{Orthomorphismo}
\begin{itemize}
\item {Grp. gram.:m.}
\end{itemize}
Conformação regular.
(Cp. \textunderscore orthomorphia\textunderscore )
\section{Orthopedia}
\begin{itemize}
\item {Grp. gram.:f.}
\end{itemize}
\begin{itemize}
\item {Proveniência:(Do gr. \textunderscore orthos\textunderscore  + \textunderscore pais\textunderscore , \textunderscore paidos\textunderscore )}
\end{itemize}
Arte de prevenir ou corrigir as deformidades do corpo.
\section{Orthopédico}
\begin{itemize}
\item {Grp. gram.:adj.}
\end{itemize}
Relativo á orthopedia.
\section{Orthopedista}
\begin{itemize}
\item {Grp. gram.:m. ,  f.  e  adj.}
\end{itemize}
\begin{itemize}
\item {Proveniência:(De \textunderscore orthopedia\textunderscore )}
\end{itemize}
Pessôa, que exerce a orthopedia, ou dirige estabelecimento orthopédico.
\section{Orthóphidos}
\begin{itemize}
\item {Grp. gram.:m. pl.}
\end{itemize}
\begin{itemize}
\item {Proveniência:(Do gr. \textunderscore orthos\textunderscore  + \textunderscore ophis\textunderscore )}
\end{itemize}
Família de reptis ophídios, que comprehende as serpentes própriamente ditas, isto é, aquellas, cujo corpo é guarnecido de placas córneas.
\section{Orthophonia}
\begin{itemize}
\item {Grp. gram.:f.}
\end{itemize}
\begin{itemize}
\item {Proveniência:(Do gr. \textunderscore orthos\textunderscore  + \textunderscore phone\textunderscore )}
\end{itemize}
Pronunciação (normal), por opposição á gaguez e outros vícios.
\section{Orthophónico}
\begin{itemize}
\item {Grp. gram.:adj.}
\end{itemize}
Relativo á orthophonia.
\section{Orthophrenia}
\begin{itemize}
\item {Grp. gram.:f.}
\end{itemize}
\begin{itemize}
\item {Proveniência:(Do gr. \textunderscore orthos\textunderscore  + \textunderscore phren\textunderscore )}
\end{itemize}
Arte de corrigir as tendências moraes ou intellectuaes.
\section{Orthophrênico}
\begin{itemize}
\item {Grp. gram.:adj.}
\end{itemize}
Relativo á ortophrenia.
\section{Orthophyrite}
\begin{itemize}
\item {Grp. gram.:f.}
\end{itemize}
\begin{itemize}
\item {Utilização:Miner.}
\end{itemize}
\begin{itemize}
\item {Proveniência:(De \textunderscore orthóphyro\textunderscore )}
\end{itemize}
Espécie de orthóphyro, de côr cinzenta escura e de grão fino e uniforme.
\section{Orthóphyro}
\begin{itemize}
\item {Grp. gram.:m.}
\end{itemize}
\begin{itemize}
\item {Proveniência:(De \textunderscore ortho...\textunderscore  + \textunderscore pórphyro\textunderscore )}
\end{itemize}
Pórphyro syenítico, ou pórphyro propriamente dito.
\section{Orthopinacóide}
\begin{itemize}
\item {Grp. gram.:f.}
\end{itemize}
\begin{itemize}
\item {Utilização:Geol.}
\end{itemize}
Diz-se da fórma dos mineraes, limitada por dois planos paralellos entre si e equidistantes do plano que passa pelo eixo principal e pela ortho-diagonal.
\section{Orthopneia}
\begin{itemize}
\item {Grp. gram.:f.}
\end{itemize}
\begin{itemize}
\item {Proveniência:(Gr. \textunderscore orthopnoia\textunderscore )}
\end{itemize}
Difficuldade de respirar, que obriga a estar em pé; asma.
\section{Orthopnoico}
\begin{itemize}
\item {Grp. gram.:adj.}
\end{itemize}
\begin{itemize}
\item {Proveniência:(Gr. \textunderscore orthopnoikos\textunderscore )}
\end{itemize}
Relativo á orthopneia.
Que soffre orthopneia; asmático.
\section{Orthóptero}
\begin{itemize}
\item {Grp. gram.:adj.}
\end{itemize}
\begin{itemize}
\item {Grp. gram.:M.}
\end{itemize}
\begin{itemize}
\item {Grp. gram.:Pl.}
\end{itemize}
\begin{itemize}
\item {Proveniência:(Do gr. \textunderscore orthos\textunderscore  + \textunderscore pteron\textunderscore )}
\end{itemize}
Diz-se dos insectos, cujas asas têm nervuras longitudinaes.
Apparelho aerostático, com que se procura imitar o vôo das aves. Cf. \textunderscore Jorn.-do-Comm.\textunderscore , do Rio, de 18-VI-902.
Ordem de insectos orthópteros.
\section{Orthorhômbico}
\begin{itemize}
\item {Grp. gram.:adj.}
\end{itemize}
\begin{itemize}
\item {Proveniência:(De \textunderscore ortho...\textunderscore  + \textunderscore rhômbico\textunderscore )}
\end{itemize}
Diz-se de um prisma, que tem base rhomboidal.
\section{Orthosa}
\begin{itemize}
\item {Grp. gram.:f.}
\end{itemize}
\begin{itemize}
\item {Proveniência:(Do gr. \textunderscore orthos\textunderscore )}
\end{itemize}
Espécie de feldspato.
\section{Orthóscele}
\begin{itemize}
\item {Grp. gram.:m.}
\end{itemize}
Apparelho para endireitar pernas tortas.
\section{Orthoscópica}
\begin{itemize}
\item {Grp. gram.:adj. f.}
\end{itemize}
\begin{itemize}
\item {Utilização:Phot.}
\end{itemize}
\begin{itemize}
\item {Proveniência:(De \textunderscore orthoscópio\textunderscore )}
\end{itemize}
Diz-se de uma espécie de objectiva.
\section{Orthoscópio}
\begin{itemize}
\item {Grp. gram.:m.}
\end{itemize}
\begin{itemize}
\item {Utilização:Med.}
\end{itemize}
\begin{itemize}
\item {Proveniência:(Do gr. \textunderscore orthos\textunderscore  + \textunderscore skopein\textunderscore )}
\end{itemize}
Apparelho, com que se examina o ôlho, através de uma camada líquida.
\section{Orthose}
O mesmo que \textunderscore orthosa\textunderscore .
\section{Orthósia}
\begin{itemize}
\item {Grp. gram.:f.}
\end{itemize}
\begin{itemize}
\item {Proveniência:(Do gr. \textunderscore orthos\textunderscore )}
\end{itemize}
Gênero de insectos nocturnos.
\section{Orthósio}
\begin{itemize}
\item {Grp. gram.:m.}
\end{itemize}
O mesmo ou melhor que \textunderscore orthose\textunderscore .
\section{Orthosoma}
\begin{itemize}
\item {Grp. gram.:m.}
\end{itemize}
\begin{itemize}
\item {Proveniência:(Do gr. \textunderscore orthos\textunderscore  + \textunderscore soma\textunderscore )}
\end{itemize}
Gênero de helminthos.
\section{Orthospermo}
\begin{itemize}
\item {Grp. gram.:adj.}
\end{itemize}
\begin{itemize}
\item {Utilização:Bot.}
\end{itemize}
\begin{itemize}
\item {Proveniência:(Do gr. \textunderscore orthos\textunderscore  + \textunderscore sperma\textunderscore )}
\end{itemize}
Que tem direito o embryão na semente.
\section{Orthostasia}
\begin{itemize}
\item {Grp. gram.:f.}
\end{itemize}
\begin{itemize}
\item {Proveniência:(Do gr. \textunderscore orthos\textunderscore  + \textunderscore stasis\textunderscore )}
\end{itemize}
Posição vertical.
\section{Orthostático}
\begin{itemize}
\item {Grp. gram.:adj.}
\end{itemize}
\begin{itemize}
\item {Utilização:Med.}
\end{itemize}
Relativo á orthostasia.
Diz-se da albuminuria, que só apparece, quando o indivíduo está em pé.
\section{Orthóstomo}
\begin{itemize}
\item {Grp. gram.:m.}
\end{itemize}
\begin{itemize}
\item {Proveniência:(Do gr. \textunderscore orthos\textunderscore  + \textunderscore stoma\textunderscore )}
\end{itemize}
Gênero de plantas gencianáceas.
\section{Orthostylo}
\begin{itemize}
\item {Grp. gram.:m.}
\end{itemize}
\begin{itemize}
\item {Proveniência:(Do gr. \textunderscore orthos\textunderscore  + \textunderscore stulos\textunderscore )}
\end{itemize}
Renque de columnas, que não formam pórtico.
\section{Orthotheca}
\begin{itemize}
\item {Grp. gram.:f.}
\end{itemize}
\begin{itemize}
\item {Proveniência:(Do gr. \textunderscore orthos\textunderscore  + \textunderscore theke\textunderscore )}
\end{itemize}
Gênero de musgos.
\section{Orthótomo}
\begin{itemize}
\item {Grp. gram.:m.}
\end{itemize}
\begin{itemize}
\item {Proveniência:(Do gr. \textunderscore orthos\textunderscore  + \textunderscore tome\textunderscore )}
\end{itemize}
Gênero de pássaros dentirostros.
\section{Orthotropia}
\begin{itemize}
\item {Grp. gram.:f.}
\end{itemize}
Estado ou qualidade de orthótropo.
\section{Orthótropo}
\begin{itemize}
\item {Grp. gram.:adj.}
\end{itemize}
\begin{itemize}
\item {Utilização:Bot.}
\end{itemize}
\begin{itemize}
\item {Proveniência:(Do gr. \textunderscore orthos\textunderscore  + \textunderscore repein\textunderscore )}
\end{itemize}
Diz-se do óvolo vegetal recto ou em que o micrópulo está diametralmente opposto ao hilo.
\section{Ortiga}
\begin{itemize}
\item {Grp. gram.:f.}
\end{itemize}
\begin{itemize}
\item {Utilização:Ant.}
\end{itemize}
Tiro de peloiro de pedra.
Canhão de 19 palmos, que atirava peloiros de pedra. Cf. Castanheira, l. II, 236; \textunderscore Tombo do Estado da Índia\textunderscore , 11.
\section{Ortiga}
\begin{itemize}
\item {Grp. gram.:f.}
\end{itemize}
\begin{itemize}
\item {Proveniência:(Do lat. \textunderscore urtica\textunderscore )}
\end{itemize}
\textunderscore f.\textunderscore  (e der.)
O mesmo ou melhor que \textunderscore urtiga\textunderscore , etc.
Gênero de plantas bravas, em que se distingue a urtiga commum, cuja haste e cujas fôlhas produzem na pelle um ardor especial.
Peixe dos Açores.
\section{Ortigão}
\begin{itemize}
\item {Grp. gram.:m.}
\end{itemize}
Espécie de ortiga, (\textunderscore urtica dioica\textunderscore , Lin.).
\section{Ortigueira}
\begin{itemize}
\item {Grp. gram.:f.}
\end{itemize}
O mesmo que \textunderscore urtigal\textunderscore .
Ortiga grande.
\section{Ortito}
\begin{itemize}
\item {Grp. gram.:m.}
\end{itemize}
\begin{itemize}
\item {Utilização:Miner.}
\end{itemize}
\begin{itemize}
\item {Proveniência:(Do gr. \textunderscore orthos\textunderscore )}
\end{itemize}
Silicato hidratado de alumínio, cálcio, ferro e cério, com lantânio e didímio. Cf. R. Galvão, \textunderscore Vocab.\textunderscore 
\section{Ortivo}
\begin{itemize}
\item {Grp. gram.:adj.}
\end{itemize}
\begin{itemize}
\item {Proveniência:(Lat. \textunderscore ortivus\textunderscore )}
\end{itemize}
Que nasce; oriental.
\section{Orto...}
\begin{itemize}
\item {Grp. gram.:pref.}
\end{itemize}
\begin{itemize}
\item {Proveniência:(Do gr. \textunderscore orthos\textunderscore )}
\end{itemize}
(designativo de \textunderscore direito\textunderscore , \textunderscore recto\textunderscore , \textunderscore exacto\textunderscore )
\section{Órto}
\begin{itemize}
\item {Grp. gram.:m.}
\end{itemize}
\begin{itemize}
\item {Utilização:Poét.}
\end{itemize}
\begin{itemize}
\item {Proveniência:(Lat. \textunderscore ortus\textunderscore )}
\end{itemize}
Nascimento de um astro.
Nascimento, origem.
\section{Ortobásico}
\begin{itemize}
\item {Grp. gram.:adj.}
\end{itemize}
\begin{itemize}
\item {Utilização:Miner.}
\end{itemize}
\begin{itemize}
\item {Proveniência:(De \textunderscore ortho...\textunderscore  + \textunderscore base\textunderscore )}
\end{itemize}
Diz-se das substâncias, cujos cristaes têm coordenadas ortogonaes.
\section{Ortócera}
\begin{itemize}
\item {Grp. gram.:f.}
\end{itemize}
\begin{itemize}
\item {Proveniência:(Do gr. \textunderscore orthos\textunderscore  + \textunderscore keras\textunderscore )}
\end{itemize}
Gênero de plantas gramíneas.
\section{Ortóclada}
\begin{itemize}
\item {Grp. gram.:f.}
\end{itemize}
Gênero de plantas gramíneas.
\section{Ortoclásio}
\begin{itemize}
\item {Grp. gram.:f.}
\end{itemize}
\begin{itemize}
\item {Utilização:Geol.}
\end{itemize}
\begin{itemize}
\item {Proveniência:(Do gr. \textunderscore orthos\textunderscore  + \textunderscore klasis\textunderscore )}
\end{itemize}
Mineral do grupo dos feldspatos.
\section{Ortocolimbos}
\begin{itemize}
\item {Grp. gram.:m. pl.}
\end{itemize}
Família de aves aquáticas, que compreende as que se conservam muito tempo debaixo da água.
\section{Ortocólon}
\begin{itemize}
\item {Grp. gram.:m.}
\end{itemize}
\begin{itemize}
\item {Utilização:Med.}
\end{itemize}
Rigidez de uma articulação, que não permite moverem-se as peças articuladas.
\section{Ortodáctilo}
\begin{itemize}
\item {Grp. gram.:adj.}
\end{itemize}
\begin{itemize}
\item {Utilização:Zool.}
\end{itemize}
\begin{itemize}
\item {Proveniência:(Do gr. \textunderscore orthos\textunderscore  + \textunderscore dakulos\textunderscore )}
\end{itemize}
Que tem os dedos direitos.
\section{Ortódano}
\begin{itemize}
\item {Grp. gram.:m.}
\end{itemize}
Gênero de plantas leguminosas.
\section{Ortodiagonal}
\begin{itemize}
\item {Grp. gram.:f.  e  adj.}
\end{itemize}
\begin{itemize}
\item {Utilização:Miner.}
\end{itemize}
\begin{itemize}
\item {Proveniência:(De \textunderscore ortho...\textunderscore  + \textunderscore diagonal\textunderscore )}
\end{itemize}
Diz-se de um dos eixos dos cristaes do sistema monoclínico. Cf. G. Guimarães, \textunderscore Geologia\textunderscore , 51.
\section{Ortodoma}
\begin{itemize}
\item {Grp. gram.:m.}
\end{itemize}
\begin{itemize}
\item {Utilização:Geol.}
\end{itemize}
\begin{itemize}
\item {Proveniência:(Do gr. \textunderscore orthos\textunderscore  + \textunderscore doma\textunderscore )}
\end{itemize}
Uma das fórmas holoédricas dos mineraes, que constitue prisma transversal.
\section{Ortodonte}
\begin{itemize}
\item {Grp. gram.:adj.}
\end{itemize}
\begin{itemize}
\item {Grp. gram.:M. pl.}
\end{itemize}
\begin{itemize}
\item {Proveniência:(Do gr. \textunderscore orthos\textunderscore  + \textunderscore odous\textunderscore )}
\end{itemize}
Que tem os dentes direitos.
Gênero de musgos, que vivem nos troncos das árvores da África austral.
\section{Ortódoro}
\begin{itemize}
\item {Grp. gram.:m.}
\end{itemize}
\begin{itemize}
\item {Proveniência:(Gr. \textunderscore orthodoron\textunderscore )}
\end{itemize}
Antiga medida linear entre os Gregos, equivalente a onze dedos.
\section{Ortodoxamente}
\begin{itemize}
\item {fónica:csa}
\end{itemize}
\begin{itemize}
\item {Grp. gram.:adv.}
\end{itemize}
De modo ortodoxo.
Com ortodoxia; segundo a doutrina que se considera ortodoxa.
\section{Ortodoxia}
\begin{itemize}
\item {fónica:csi}
\end{itemize}
\begin{itemize}
\item {Grp. gram.:f.}
\end{itemize}
Qualidade do que é ortodoxo.
Doutrina religiosa, considerada como verdadeira.
Conformidade com essa doutrina.
Conformidade com a doutrina da Igreja.
\section{Ortodoxo}
\begin{itemize}
\item {fónica:cso}
\end{itemize}
\begin{itemize}
\item {Grp. gram.:adj.}
\end{itemize}
\begin{itemize}
\item {Grp. gram.:M.}
\end{itemize}
\begin{itemize}
\item {Utilização:Ext.}
\end{itemize}
\begin{itemize}
\item {Proveniência:(Lat. \textunderscore orthodoxus\textunderscore )}
\end{itemize}
Relativo á ortodoxia.
Que está de acôrdo com certos sistemas ou com ideias geralmente recebidas.
Indivíduo que segue a doutrina religiosa, considerada como verdadeira.
Aquele que segue qualquer doutrina estabelecida.
\section{Ortodoxografia}
\begin{itemize}
\item {fónica:cso}
\end{itemize}
\begin{itemize}
\item {Grp. gram.:f.}
\end{itemize}
\begin{itemize}
\item {Proveniência:(Do gr. \textunderscore orthos\textunderscore  + \textunderscore doxa\textunderscore  + \textunderscore graphein\textunderscore )}
\end{itemize}
Tratado á cêrca dos dogmas católicos.
\section{Ortodoxográfico}
\begin{itemize}
\item {fónica:cso}
\end{itemize}
\begin{itemize}
\item {Grp. gram.:adj.}
\end{itemize}
Relativo á ortodoxografia.
\section{Ortodoxógrafo}
\begin{itemize}
\item {fónica:csó}
\end{itemize}
\begin{itemize}
\item {Grp. gram.:M.}
\end{itemize}
Aquele que escreve sôbre a ortodoxogrfia.
\section{Ortodromia}
\begin{itemize}
\item {Grp. gram.:f.}
\end{itemize}
\begin{itemize}
\item {Utilização:Náut.}
\end{itemize}
\begin{itemize}
\item {Utilização:Topogr.}
\end{itemize}
\begin{itemize}
\item {Proveniência:(Do gr. \textunderscore orthos\textunderscore  + \textunderscore dromos\textunderscore )}
\end{itemize}
Linha mais curta entre os dois pontos extremos da rota de um navio.
A distância mais curta; o traçado recto.
\section{Ortodrómico}
\begin{itemize}
\item {Grp. gram.:adj.}
\end{itemize}
Relativo á ortodromia.
\section{Ortoédrico}
\begin{itemize}
\item {Grp. gram.:adj.}
\end{itemize}
\begin{itemize}
\item {Utilização:Miner.}
\end{itemize}
\begin{itemize}
\item {Proveniência:(Do gr. \textunderscore orthos\textunderscore  + \textunderscore edra\textunderscore )}
\end{itemize}
Diz-se dos cristaes, cujos planos coordenados são perpendiculares entre si.
\section{Ortoepia}
\begin{itemize}
\item {Grp. gram.:f.}
\end{itemize}
\begin{itemize}
\item {Proveniência:(Do gr. \textunderscore orthoepeia\textunderscore )}
\end{itemize}
Pronúncia correcta.
Parte da Gramática, que ensina a bôa pronúncia.--Usa-se a prosódia \textunderscore ortoépia\textunderscore , mas não é rigorosa.
\section{Ortoépico}
\begin{itemize}
\item {Grp. gram.:adj.}
\end{itemize}
Relativo á ortoépia.
\section{Ortófidos}
\begin{itemize}
\item {Grp. gram.:m. pl.}
\end{itemize}
\begin{itemize}
\item {Proveniência:(Do gr. \textunderscore orthos\textunderscore  + \textunderscore ophis\textunderscore )}
\end{itemize}
Família de reptis ofídios, que compreende as serpentes própriamente ditas, isto é, aquelas, cujo corpo é guarnecido de placas córneas.
\section{Ortofirite}
\begin{itemize}
\item {Grp. gram.:f.}
\end{itemize}
\begin{itemize}
\item {Utilização:Miner.}
\end{itemize}
\begin{itemize}
\item {Proveniência:(De \textunderscore ortófiro\textunderscore )}
\end{itemize}
Espécie de ortófiro, de côr cinzenta escura e de grão fino e uniforme.
\section{Ortófiro}
\begin{itemize}
\item {Grp. gram.:m.}
\end{itemize}
\begin{itemize}
\item {Proveniência:(De \textunderscore orto...\textunderscore  + \textunderscore pórfiro\textunderscore )}
\end{itemize}
Pórfiro sienítico, ou pórfiro propriamente dito.
\section{Ortofonia}
\begin{itemize}
\item {Grp. gram.:f.}
\end{itemize}
\begin{itemize}
\item {Proveniência:(Do gr. \textunderscore orthos\textunderscore  + \textunderscore phone\textunderscore )}
\end{itemize}
Pronunciação (normal), por opposição á gaguez e outros vícios.
\section{Ortofónico}
\begin{itemize}
\item {Grp. gram.:adj.}
\end{itemize}
Relativo á ortofonia.
\section{Ortofórmio}
\begin{itemize}
\item {Grp. gram.:m.}
\end{itemize}
Pó branco, insipido e incolor, de propriedades anestésicas, descoberto recentemente, (1898), em Alemanha.
\section{Ortofrenia}
\begin{itemize}
\item {Grp. gram.:f.}
\end{itemize}
\begin{itemize}
\item {Proveniência:(Do gr. \textunderscore orthos\textunderscore  + \textunderscore phren\textunderscore )}
\end{itemize}
Arte de corrigir as tendências moraes ou intelectuaes.
\section{Ortofrênico}
\begin{itemize}
\item {Grp. gram.:adj.}
\end{itemize}
Relativo á ortofrenia.
\section{Ortognatismo}
\begin{itemize}
\item {Grp. gram.:m.}
\end{itemize}
\begin{itemize}
\item {Utilização:Zool.}
\end{itemize}
\begin{itemize}
\item {Proveniência:(De \textunderscore ortognato\textunderscore )}
\end{itemize}
Qualidade de têr os queixos aprumados com a parte superior do rosto.
\section{Ortógnato}
\begin{itemize}
\item {Grp. gram.:adj.}
\end{itemize}
\begin{itemize}
\item {Proveniência:(Do gr. \textunderscore orthos\textunderscore  + \textunderscore gnathos\textunderscore )}
\end{itemize}
Diz-se da raça humana, cujo bôrdo alveolar e cujos dentes do maxilar superior são um pouco oblíquos para deante.
\section{Ortogonal}
\begin{itemize}
\item {Grp. gram.:adj.}
\end{itemize}
\begin{itemize}
\item {Utilização:Geom.}
\end{itemize}
Diz-se da projecção em que cada linha, que projecta um ponto da figura, é perpendicular ao plano de projecção.
(Cp. \textunderscore ortógono\textunderscore )
\section{Ortogonalmente}
\begin{itemize}
\item {Grp. gram.:adv.}
\end{itemize}
De modo ortogonal.
Perpendicularmente.
\section{Ortógono}
\begin{itemize}
\item {Grp. gram.:adj.}
\end{itemize}
\begin{itemize}
\item {Utilização:Geom.}
\end{itemize}
\begin{itemize}
\item {Proveniência:(Lat. \textunderscore orthogonus\textunderscore )}
\end{itemize}
Diz-se da linha que fórma com outra ângulo recto.
Perpendicular.--A pronúncia exacta seria \textunderscore ortogóno\textunderscore , mas não se usa.
\section{Ortografar}
\begin{itemize}
\item {Grp. gram.:v. t.}
\end{itemize}
Escrever, segundo as regras ortográficas.
Empregar ortografia particular ou especial a respeito de: \textunderscore ortografar sonicamente um vocábulo\textunderscore .
Aplicar a ortografia a.
(Cp. \textunderscore ortógrafo\textunderscore )
\section{Ortografia}
\begin{itemize}
\item {Grp. gram.:f.}
\end{itemize}
\begin{itemize}
\item {Utilização:Geom.}
\end{itemize}
\begin{itemize}
\item {Proveniência:(Lat. \textunderscore orthographia\textunderscore )}
\end{itemize}
Acto e modo de escrever correctamente as palavras de uma língua.
Qualquer maneira de escrever palavras.
Representação de um edificio.
Perfil de uma construcção.
Projecção ortogonal.
\section{Ortograficamente}
\begin{itemize}
\item {Grp. gram.:adv.}
\end{itemize}
De modo ortográfico.
Relativamente á ortografia.
\section{Ortográfico}
\begin{itemize}
\item {Grp. gram.:adj.}
\end{itemize}
Relativo á ortografia, considerada gramatical e geometricamente.
\section{Ortografista}
\begin{itemize}
\item {Grp. gram.:m. ,  f.  e  adj.}
\end{itemize}
Pessôa, que trata de ortografia.
\section{Ortógrafo}
\begin{itemize}
\item {Grp. gram.:m.}
\end{itemize}
\begin{itemize}
\item {Proveniência:(Lat. \textunderscore orthographus\textunderscore )}
\end{itemize}
Aquele que é versado nas leis ortográficas.
\section{Ortolexia}
\begin{itemize}
\item {fónica:csi}
\end{itemize}
\begin{itemize}
\item {Grp. gram.:f.}
\end{itemize}
\begin{itemize}
\item {Proveniência:(Do gr. \textunderscore orthos\textunderscore  + \textunderscore lexis\textunderscore )}
\end{itemize}
Fórma correcta de se exprimir; bôa dicção.
\section{Ortólito}
\begin{itemize}
\item {Grp. gram.:m.}
\end{itemize}
\begin{itemize}
\item {Utilização:Miner.}
\end{itemize}
Rocha microgranítica, composta principalmente de ortósio, mica e hornblenda.
\section{Ortologia}
\begin{itemize}
\item {Grp. gram.:f.}
\end{itemize}
\begin{itemize}
\item {Utilização:Bras}
\end{itemize}
\begin{itemize}
\item {Proveniência:(Do gr. \textunderscore orthos\textunderscore  + \textunderscore logos\textunderscore )}
\end{itemize}
Arte de falar correctamente.
O mesmo que \textunderscore ortoepia\textunderscore .
O mesmo que \textunderscore ocultismo\textunderscore . Cf. \textunderscore Notícia\textunderscore , do Rio, de 30-VIII-900.
\section{Ortológico}
\begin{itemize}
\item {Grp. gram.:adj.}
\end{itemize}
Relativo á ortologia.
\section{Ortómega}
\begin{itemize}
\item {Grp. gram.:m.}
\end{itemize}
\begin{itemize}
\item {Proveniência:(Do gr. \textunderscore orthos\textunderscore  + \textunderscore megas\textunderscore )}
\end{itemize}
Gênero de insectos coleópteros longicórneos.
\section{Órtomo}
\begin{itemize}
\item {Grp. gram.:m.}
\end{itemize}
Gênero de insectos coleópteros pentâmeros.
\section{Ortomorfia}
\begin{itemize}
\item {Grp. gram.:f.}
\end{itemize}
\begin{itemize}
\item {Proveniência:(Do gr. \textunderscore orthos\textunderscore  + \textunderscore morphe\textunderscore )}
\end{itemize}
Arte de restituír a uma parte do corpo humano a sua fórma natural.
\section{Ortomórfico}
\begin{itemize}
\item {Grp. gram.:adj.}
\end{itemize}
Relativo a ortomorfia.
\section{Ortomorfismo}
\begin{itemize}
\item {Grp. gram.:m.}
\end{itemize}
Conformação regular.
(Cp. \textunderscore ortomorfia\textunderscore )
\section{Ortopedia}
\begin{itemize}
\item {Grp. gram.:f.}
\end{itemize}
\begin{itemize}
\item {Proveniência:(Do gr. \textunderscore orthos\textunderscore  + \textunderscore pais\textunderscore , \textunderscore paidos\textunderscore )}
\end{itemize}
Arte de prevenir ou corrigir as deformidades do corpo.
\section{Ortopédico}
\begin{itemize}
\item {Grp. gram.:adj.}
\end{itemize}
Relativo á ortopedia.
\section{Ortopedista}
\begin{itemize}
\item {Grp. gram.:m. ,  f.  e  adj.}
\end{itemize}
\begin{itemize}
\item {Proveniência:(De \textunderscore ortopedia\textunderscore )}
\end{itemize}
Pessôa, que exerce a ortopedia, ou dirige estabelecimento ortopédico.
\section{Ortopinacóide}
\begin{itemize}
\item {Grp. gram.:f.}
\end{itemize}
\begin{itemize}
\item {Utilização:Geol.}
\end{itemize}
Diz-se da fórma dos mineraes, limitada por dois planos paralelos entre si e equidistantes do plano que passa pelo eixo principal e pela orto-diagonal.
\section{Ortopneia}
\begin{itemize}
\item {Grp. gram.:f.}
\end{itemize}
\begin{itemize}
\item {Proveniência:(Gr. \textunderscore orthopnoia\textunderscore )}
\end{itemize}
Dificuldade de respirar, que obriga a estar em pé; asma.
\section{Ortopnoico}
\begin{itemize}
\item {Grp. gram.:adj.}
\end{itemize}
\begin{itemize}
\item {Proveniência:(Gr. \textunderscore orthopnoikos\textunderscore )}
\end{itemize}
Relativo á ortopneia.
Que sofre ortopneia; asmático.
\section{Ortóptero}
\begin{itemize}
\item {Grp. gram.:adj.}
\end{itemize}
\begin{itemize}
\item {Grp. gram.:M.}
\end{itemize}
\begin{itemize}
\item {Grp. gram.:Pl.}
\end{itemize}
\begin{itemize}
\item {Proveniência:(Do gr. \textunderscore orthos\textunderscore  + \textunderscore pteron\textunderscore )}
\end{itemize}
Diz-se dos insectos, cujas asas têm nervuras longitudinaes.
Aparelho aerostático, com que se procura imitar o vôo das aves. Cf. \textunderscore Jorn.-do-Comm.\textunderscore , do Rio, de 18-VI-902.
Ordem de insectos ortópteros.
\section{Ortorrômbico}
\begin{itemize}
\item {Grp. gram.:adj.}
\end{itemize}
\begin{itemize}
\item {Proveniência:(De \textunderscore orto...\textunderscore  + \textunderscore rômbico\textunderscore )}
\end{itemize}
Diz-se de um prisma, que tem base romboidal.
\section{Ortosa}
\begin{itemize}
\item {Grp. gram.:f.}
\end{itemize}
\begin{itemize}
\item {Proveniência:(Do gr. \textunderscore orthos\textunderscore )}
\end{itemize}
Espécie de feldspato.
\section{Ortóscele}
\begin{itemize}
\item {Grp. gram.:m.}
\end{itemize}
Aparelho para endireitar pernas tortas.
\section{Ortoscópica}
\begin{itemize}
\item {Grp. gram.:adj. f.}
\end{itemize}
\begin{itemize}
\item {Utilização:Phot.}
\end{itemize}
\begin{itemize}
\item {Proveniência:(De \textunderscore ortoscópio\textunderscore )}
\end{itemize}
Diz-se de uma espécie de objectiva.
\section{Ortoscópio}
\begin{itemize}
\item {Grp. gram.:m.}
\end{itemize}
\begin{itemize}
\item {Utilização:Med.}
\end{itemize}
\begin{itemize}
\item {Proveniência:(Do gr. \textunderscore orthos\textunderscore  + \textunderscore skopein\textunderscore )}
\end{itemize}
Aparelho, com que se examina o ôlho, através de uma camada líquida.
\section{Ortose}
O mesmo que \textunderscore ortosa\textunderscore .
\section{Ortósia}
\begin{itemize}
\item {Grp. gram.:f.}
\end{itemize}
\begin{itemize}
\item {Proveniência:(Do gr. \textunderscore orthos\textunderscore )}
\end{itemize}
Gênero de insectos nocturnos.
\section{Ortósio}
\begin{itemize}
\item {Grp. gram.:m.}
\end{itemize}
O mesmo ou melhor que \textunderscore ortose\textunderscore .
\section{Ortosoma}
\begin{itemize}
\item {Grp. gram.:m.}
\end{itemize}
\begin{itemize}
\item {Proveniência:(Do gr. \textunderscore orthos\textunderscore  + \textunderscore soma\textunderscore )}
\end{itemize}
Gênero de helmintos.
\section{Ortospermo}
\begin{itemize}
\item {Grp. gram.:adj.}
\end{itemize}
\begin{itemize}
\item {Utilização:Bot.}
\end{itemize}
\begin{itemize}
\item {Proveniência:(Do gr. \textunderscore orthos\textunderscore  + \textunderscore sperma\textunderscore )}
\end{itemize}
Que tem direito o embrião na semente.
\section{Ortostasia}
\begin{itemize}
\item {Grp. gram.:f.}
\end{itemize}
\begin{itemize}
\item {Proveniência:(Do gr. \textunderscore orthos\textunderscore  + \textunderscore stasis\textunderscore )}
\end{itemize}
Posição vertical.
\section{Ortostático}
\begin{itemize}
\item {Grp. gram.:adj.}
\end{itemize}
\begin{itemize}
\item {Utilização:Med.}
\end{itemize}
Relativo á ortostasia.
Diz-se da albuminuria, que só apparece, quando o indivíduo está em pé.
\section{Ortóstomo}
\begin{itemize}
\item {Grp. gram.:m.}
\end{itemize}
\begin{itemize}
\item {Proveniência:(Do gr. \textunderscore orthos\textunderscore  + \textunderscore stoma\textunderscore )}
\end{itemize}
Gênero de plantas gencianáceas.
\section{Ortostilo}
\begin{itemize}
\item {Grp. gram.:m.}
\end{itemize}
\begin{itemize}
\item {Proveniência:(Do gr. \textunderscore orthos\textunderscore  + \textunderscore stulos\textunderscore )}
\end{itemize}
Renque de colunas, que não formam pórtico.
\section{Ortoteca}
\begin{itemize}
\item {Grp. gram.:f.}
\end{itemize}
\begin{itemize}
\item {Proveniência:(Do gr. \textunderscore orthos\textunderscore  + \textunderscore theke\textunderscore )}
\end{itemize}
Gênero de musgos.
\section{Ortótomo}
\begin{itemize}
\item {Grp. gram.:m.}
\end{itemize}
\begin{itemize}
\item {Proveniência:(Do gr. \textunderscore orthos\textunderscore  + \textunderscore tome\textunderscore )}
\end{itemize}
Gênero de pássaros dentirostros.
\section{Ortotropia}
\begin{itemize}
\item {Grp. gram.:f.}
\end{itemize}
Estado ou qualidade de ortótropo.
\section{Ortótropo}
\begin{itemize}
\item {Grp. gram.:adj.}
\end{itemize}
\begin{itemize}
\item {Utilização:Bot.}
\end{itemize}
\begin{itemize}
\item {Proveniência:(Do gr. \textunderscore orthos\textunderscore  + \textunderscore repein\textunderscore )}
\end{itemize}
Diz-se do óvolo vegetal recto ou em que o micrópulo está diametralmente oposto ao hilo.
\section{Oruçu}
\begin{itemize}
\item {Grp. gram.:m.}
\end{itemize}
Grande abelha dos sertões do Brasil.
\section{Orucurana}
\begin{itemize}
\item {Grp. gram.:f.}
\end{itemize}
\begin{itemize}
\item {Utilização:Bras}
\end{itemize}
Árvore silvestre, cuja madeira é empregada em carpintaria.
\section{Orumanaus}
\begin{itemize}
\item {Grp. gram.:m. pl.}
\end{itemize}
Antiga nação de Índios da Guiana brasileira, dos quaes procedem os Manaus.
\section{Orumbeba}
\begin{itemize}
\item {Grp. gram.:f.}
\end{itemize}
\begin{itemize}
\item {Utilização:Bras}
\end{itemize}
O mesmo que \textunderscore cardo-palmatória\textunderscore .
\section{Orvaêza}
\begin{itemize}
\item {Grp. gram.:f.}
\end{itemize}
\begin{itemize}
\item {Utilização:Bras}
\end{itemize}
Árvore silvestre, de madeira applicavel a obras de carpintaria.
\section{Orvalhada}
\begin{itemize}
\item {Grp. gram.:f.}
\end{itemize}
\begin{itemize}
\item {Proveniência:(De \textunderscore orvalho\textunderscore )}
\end{itemize}
Orvalho matinal; geada.
\section{Orvalhar}
\begin{itemize}
\item {Grp. gram.:v. t.}
\end{itemize}
\begin{itemize}
\item {Utilização:Fig.}
\end{itemize}
\begin{itemize}
\item {Grp. gram.:V. i.}
\end{itemize}
\begin{itemize}
\item {Utilização:Fig.}
\end{itemize}
Humedecer com orvalho.
Borrifar com qualquer líquido.
Afagar; dar alegria a.
Cair orvalho.
Chuviscar.
\section{Orvalheira}
\begin{itemize}
\item {Grp. gram.:f.}
\end{itemize}
\begin{itemize}
\item {Utilização:T. de Setúbal}
\end{itemize}
Chuva ligeira, nas madrugadas de verão. Cf. Rev. \textunderscore Tradição\textunderscore , V, 11.
\section{Orvalhinha}
\begin{itemize}
\item {Grp. gram.:f.}
\end{itemize}
\begin{itemize}
\item {Proveniência:(De \textunderscore orvalho\textunderscore )}
\end{itemize}
Planta droserácea, (\textunderscore drosera rotundifolia\textunderscore ).
O mesmo que \textunderscore rosella\textunderscore .
\section{Orvalho}
\begin{itemize}
\item {Grp. gram.:m.}
\end{itemize}
\begin{itemize}
\item {Utilização:Ext.}
\end{itemize}
\begin{itemize}
\item {Utilização:Poét.}
\end{itemize}
\begin{itemize}
\item {Utilização:T. da Bairrada}
\end{itemize}
\begin{itemize}
\item {Proveniência:(Do lat. \textunderscore roralis\textunderscore , de \textunderscore ros\textunderscore , \textunderscore roris\textunderscore ?)}
\end{itemize}
Camada de humidade, que, sob a fórma de pequenas gotas, se deposita, durante a noite, sôbre os corpos expostos ao ar livre, quando o céu está limpo.
Gotas, semelhantes ao orvalho.
Aquillo que alegra ou consola, bálsamo.
Grangeia ou pequenos confeitos, com que se adornam certos doces ou iguarias.
\section{Orvalho-do-sol}
\begin{itemize}
\item {Grp. gram.:m.}
\end{itemize}
\begin{itemize}
\item {Utilização:T. de Odemira}
\end{itemize}
Planta, (\textunderscore drosophyllum lusitanicum\textunderscore , Lk.), mais conhecida por \textunderscore pinheiro-baboso\textunderscore .
\section{Orvalhoso}
\begin{itemize}
\item {Grp. gram.:adj.}
\end{itemize}
Em que há orvalho; que deita orvalho.
\section{Orveto}
\begin{itemize}
\item {fónica:vê}
\end{itemize}
\begin{itemize}
\item {Grp. gram.:m.}
\end{itemize}
Gênero de reptís ophídios, que fórma a transição das serpentes para os lagartos, e cuja espécie mais vulgar na Europa é conhecida por \textunderscore serpente-de-vidro\textunderscore .
\section{Oryctero}
\begin{itemize}
\item {Grp. gram.:m.}
\end{itemize}
\begin{itemize}
\item {Proveniência:(Do gr. \textunderscore oruktes\textunderscore )}
\end{itemize}
Gênero de mammíferos roedores, que minam a terra como as toupeiras.
\section{Orycterope}
\begin{itemize}
\item {Grp. gram.:m.}
\end{itemize}
\begin{itemize}
\item {Proveniência:(Do gr. \textunderscore orukter\textunderscore  + \textunderscore ops\textunderscore )}
\end{itemize}
Quadrúpede sul-africano, que devora as formigas e é semelhante ao tamanduá.
\section{Orycto...}
\begin{itemize}
\item {Grp. gram.:pref.}
\end{itemize}
\begin{itemize}
\item {Proveniência:(Do gr. \textunderscore oruktos\textunderscore )}
\end{itemize}
Designativo de fóssil ou de mineral.
\section{Oryctogeologia}
\begin{itemize}
\item {Grp. gram.:f.}
\end{itemize}
\begin{itemize}
\item {Proveniência:(De \textunderscore orycto...\textunderscore  + \textunderscore geologia\textunderscore )}
\end{itemize}
Parte da História Natural, que trata da disposição dos mineraes na terra.
\section{Oryctognosia}
\begin{itemize}
\item {Grp. gram.:f.}
\end{itemize}
\begin{itemize}
\item {Proveniência:(Do gr. \textunderscore oruktos\textunderscore  + \textunderscore gnosis\textunderscore )}
\end{itemize}
Parte da História Natural, que ensina a conhecer e a distinguir os metaes.
\section{Oryctognosta}
\begin{itemize}
\item {Grp. gram.:m.}
\end{itemize}
Aquelle que trata da oryctognosia.
\section{Oryctographia}
\begin{itemize}
\item {Grp. gram.:f.}
\end{itemize}
\begin{itemize}
\item {Proveniência:(De \textunderscore oryctógrapho\textunderscore )}
\end{itemize}
Descripção dos fósseis.
\section{Oryctográphico}
\begin{itemize}
\item {Grp. gram.:adj.}
\end{itemize}
Relativo á oryctographia.
\section{Oryctógrapho}
\begin{itemize}
\item {Grp. gram.:m.}
\end{itemize}
\begin{itemize}
\item {Proveniência:(Do gr. \textunderscore oruktos\textunderscore  + \textunderscore graphein\textunderscore )}
\end{itemize}
Aquelle que se occupa de oryctographia.
\section{Oryctologia}
\begin{itemize}
\item {Grp. gram.:f.}
\end{itemize}
História dos fósseis; tratado á cêrca dos fósseis.
(Cp. \textunderscore oryctólogo\textunderscore )
\section{Oryctológico}
\begin{itemize}
\item {Grp. gram.:adj.}
\end{itemize}
Relativo á oryctologia.
\section{Oryctologista}
\begin{itemize}
\item {Grp. gram.:m.  e  f.}
\end{itemize}
Pessôa, que se dedica á oryctologia.
\section{Oryctólogo}
\begin{itemize}
\item {Grp. gram.:m.}
\end{itemize}
\begin{itemize}
\item {Proveniência:(Do gr. \textunderscore oruktos\textunderscore  + \textunderscore logos\textunderscore )}
\end{itemize}
Aquelle que é versado em oryctologia.
\section{Oryctotechnia}
\begin{itemize}
\item {Grp. gram.:f.}
\end{itemize}
\begin{itemize}
\item {Proveniência:(Do gr. \textunderscore oruktos\textunderscore  + \textunderscore tekhne\textunderscore )}
\end{itemize}
Estudo dos meios, com que se podem procurar as substâncias mineraes, necessárias aos usos da vida.
\section{Óryx}
\begin{itemize}
\item {Grp. gram.:m.}
\end{itemize}
Espécie de antílope sul-africano.
\section{Oryzário}
\begin{itemize}
\item {Grp. gram.:m.}
\end{itemize}
\begin{itemize}
\item {Proveniência:(Do lat. \textunderscore oryza\textunderscore , arroz)}
\end{itemize}
Gênero de polypeiros fósseis.
\section{Orýzeas}
\begin{itemize}
\item {Grp. gram.:f. pl.}
\end{itemize}
\begin{itemize}
\item {Proveniência:(Do lat. \textunderscore oryza\textunderscore )}
\end{itemize}
Tríbo de plantas gramíneas, que têm por typo o arroz.
\section{Oryzícola}
\begin{itemize}
\item {Grp. gram.:adj.}
\end{itemize}
\begin{itemize}
\item {Proveniência:(Do lat. \textunderscore oryza\textunderscore  + \textunderscore colere\textunderscore )}
\end{itemize}
Relativo á cultura do arroz.
\section{Oryzicultura}
\begin{itemize}
\item {Grp. gram.:f.}
\end{itemize}
\begin{itemize}
\item {Proveniência:(Do lat. \textunderscore oryza\textunderscore  + \textunderscore cultura\textunderscore )}
\end{itemize}
Cultura do arroz.
\section{Oryziforme}
\begin{itemize}
\item {Grp. gram.:adj.}
\end{itemize}
\begin{itemize}
\item {Proveniência:(Do lat. \textunderscore oryza\textunderscore  + \textunderscore forma\textunderscore )}
\end{itemize}
Que tem fórma de arroz.
\section{Oryzívoro}
\begin{itemize}
\item {Grp. gram.:adj.}
\end{itemize}
\begin{itemize}
\item {Proveniência:(Do lat. \textunderscore oryza\textunderscore  + \textunderscore vorare\textunderscore )}
\end{itemize}
Diz-se dos animaes, que se alimentam de arroz.
\section{Orizófago}
\begin{itemize}
\item {Grp. gram.:adj.}
\end{itemize}
\begin{itemize}
\item {Proveniência:(Do gr. \textunderscore oruza\textunderscore  + \textunderscore phagein\textunderscore )}
\end{itemize}
Que se alimenta de arroz, (falando-se de homens).
\section{Orizoídeo}
\begin{itemize}
\item {Grp. gram.:adj.}
\end{itemize}
\begin{itemize}
\item {Utilização:Bot.}
\end{itemize}
\begin{itemize}
\item {Proveniência:(Do gr. \textunderscore oruza\textunderscore  + \textunderscore eidos\textunderscore )}
\end{itemize}
Que tem aparência do arroz.
\section{Orizópside}
\begin{itemize}
\item {Grp. gram.:f.}
\end{itemize}
\begin{itemize}
\item {Proveniência:(Do gr. \textunderscore oruza\textunderscore  + \textunderscore ops\textunderscore )}
\end{itemize}
Gênero de plantas monocotiledóneas.
\section{Oryzoídeo}
\begin{itemize}
\item {Grp. gram.:adj.}
\end{itemize}
\begin{itemize}
\item {Utilização:Bot.}
\end{itemize}
\begin{itemize}
\item {Proveniência:(Do gr. \textunderscore oruza\textunderscore  + \textunderscore eidos\textunderscore )}
\end{itemize}
Que tem apparência do arroz.
\section{Oryzóphago}
\begin{itemize}
\item {Grp. gram.:adj.}
\end{itemize}
\begin{itemize}
\item {Proveniência:(Do gr. \textunderscore oruza\textunderscore  + \textunderscore phagein\textunderscore )}
\end{itemize}
Que se alimenta de arroz, (falando-se de homens).
\section{Oryzópside}
\begin{itemize}
\item {Grp. gram.:f.}
\end{itemize}
\begin{itemize}
\item {Proveniência:(Do gr. \textunderscore oruza\textunderscore  + \textunderscore ops\textunderscore )}
\end{itemize}
Gênero de plantas monocotyledóneas.
\section{Orzaga}
\begin{itemize}
\item {Grp. gram.:f.}
\end{itemize}
\begin{itemize}
\item {Utilização:Bot.}
\end{itemize}
Espécie de armole, cujas fôlhas são muito appetecidas pelo gado lanígero.
\section{Orzuna}
\begin{itemize}
\item {Grp. gram.:f.}
\end{itemize}
Árvore da Índia portuguesa.
\section{Os}
\begin{itemize}
\item {fónica:us}
\end{itemize}
\begin{itemize}
\item {Grp. gram.:art. def.  e  pron. demonstr.}
\end{itemize}
(pl. de \textunderscore o\textunderscore ^2 e \textunderscore o\textunderscore ^3)
\section{Osa}
\begin{itemize}
\item {Grp. gram.:f.}
\end{itemize}
\begin{itemize}
\item {Utilização:Ant.}
\end{itemize}
\begin{itemize}
\item {Proveniência:(Lat. \textunderscore osa\textunderscore )}
\end{itemize}
Donativo, que o marido fazia á mulher, no dia immediato ao das núpcias, e que consistia ordinariamente numa espécie de calçado.
Espécie de calçado.
Aquillo que o emphyteuta dava ao senhor das terras, para obter dêste licença de casar.
\section{Osada}
\begin{itemize}
\item {Grp. gram.:f.}
\end{itemize}
\begin{itemize}
\item {Utilização:Ant.}
\end{itemize}
Ousadia? Cp. G. Vicente, \textunderscore Inês Pereira\textunderscore .
\section{Oscheíte}
\begin{itemize}
\item {fónica:que}
\end{itemize}
\begin{itemize}
\item {Grp. gram.:f.}
\end{itemize}
\begin{itemize}
\item {Utilização:Med.}
\end{itemize}
\begin{itemize}
\item {Proveniência:(Do gr. \textunderscore oskheon\textunderscore )}
\end{itemize}
Inflammação do escroto.
\section{Oscheocalasia}
\begin{itemize}
\item {fónica:que}
\end{itemize}
\begin{itemize}
\item {Grp. gram.:f.}
\end{itemize}
\begin{itemize}
\item {Utilização:Med.}
\end{itemize}
\begin{itemize}
\item {Proveniência:(Do gr. \textunderscore oskheon\textunderscore  + \textunderscore kalasis\textunderscore )}
\end{itemize}
Tumor, resultante da hypertrophia do tecido cellular do escroto.
\section{Oscheocele}
\begin{itemize}
\item {fónica:que}
\end{itemize}
\begin{itemize}
\item {Grp. gram.:m.}
\end{itemize}
\begin{itemize}
\item {Utilização:Med.}
\end{itemize}
\begin{itemize}
\item {Proveniência:(Do gr. \textunderscore oskheon\textunderscore  + \textunderscore kele\textunderscore )}
\end{itemize}
Hérnia do escroto.
\section{Oscheólito}
\begin{itemize}
\item {fónica:que}
\end{itemize}
\begin{itemize}
\item {Grp. gram.:m.}
\end{itemize}
\begin{itemize}
\item {Utilização:Med.}
\end{itemize}
\begin{itemize}
\item {Proveniência:(Do gr. \textunderscore oscheon\textunderscore  + \textunderscore lithos\textunderscore )}
\end{itemize}
Concreção calcária no escroto.
\section{Oscheôma}
\begin{itemize}
\item {fónica:que}
\end{itemize}
\begin{itemize}
\item {Grp. gram.:m.}
\end{itemize}
\begin{itemize}
\item {Utilização:Med.}
\end{itemize}
\begin{itemize}
\item {Proveniência:(Do gr. \textunderscore osklion\textunderscore )}
\end{itemize}
Tumor no escroto.
\section{Oscheoplastia}
\begin{itemize}
\item {fónica:que}
\end{itemize}
\begin{itemize}
\item {Grp. gram.:f.}
\end{itemize}
\begin{itemize}
\item {Utilização:Cir.}
\end{itemize}
\begin{itemize}
\item {Proveniência:(Do gr. \textunderscore oskheon\textunderscore  + \textunderscore plassein\textunderscore )}
\end{itemize}
Reparação do escroto, pelos processos autoplásticos.
\section{Oschophórias}
\begin{itemize}
\item {fónica:có}
\end{itemize}
\begin{itemize}
\item {Grp. gram.:f. pl.}
\end{itemize}
\begin{itemize}
\item {Proveniência:(Gr. \textunderscore oskhophoria\textunderscore )}
\end{itemize}
Antigas festas gregas, em que se levavam ramos, carregados de uvas.
\section{Oscilação}
\begin{itemize}
\item {Grp. gram.:f.}
\end{itemize}
\begin{itemize}
\item {Utilização:Fig.}
\end{itemize}
\begin{itemize}
\item {Proveniência:(Lat. \textunderscore oscillatio\textunderscore )}
\end{itemize}
Acto ou efeito de oscilar.
Movimento de um pêndulo que, indo e vindo alternadamente em dois sentidos oppostos, se balança á direita e á esquerda de um ponto central.
Movimento de vaivém.
Hesitação; perplexidade.
\section{Oscilante}
\begin{itemize}
\item {Grp. gram.:adj.}
\end{itemize}
\begin{itemize}
\item {Proveniência:(Lat. \textunderscore oscillans\textunderscore )}
\end{itemize}
Que oscila.
\section{Oscilar}
\begin{itemize}
\item {Grp. gram.:v. t.}
\end{itemize}
\begin{itemize}
\item {Utilização:Fig.}
\end{itemize}
\begin{itemize}
\item {Proveniência:(Lat. \textunderscore oscillare\textunderscore )}
\end{itemize}
Mover-se alternadamente em sentidos opostos.
Balançar-se; vacilar.
Hesitar, duvidar.
\section{Oscilatório}
\begin{itemize}
\item {Grp. gram.:adj.}
\end{itemize}
\begin{itemize}
\item {Proveniência:(Do lat. \textunderscore oscillatus\textunderscore )}
\end{itemize}
Oscilante; que é da natureza da oscilação: \textunderscore movimento oscilatório\textunderscore .
\section{Oscilho}
\begin{itemize}
\item {Grp. gram.:m.}
\end{itemize}
\begin{itemize}
\item {Proveniência:(Do lat. \textunderscore oscillum\textunderscore )}
\end{itemize}
Pequena figura humana, que os antigos suspendiam da estátua de Saturno e que depois lhes servia de amuleto contra quaesquer malefícios.
\section{Oscillação}
\begin{itemize}
\item {Grp. gram.:f.}
\end{itemize}
\begin{itemize}
\item {Utilização:Fig.}
\end{itemize}
\begin{itemize}
\item {Proveniência:(Lat. \textunderscore oscillatio\textunderscore )}
\end{itemize}
Acto ou effeito de oscillar.
Movimento de um pêndulo que, indo e vindo alternadamente em dois sentidos oppostos, se balança á direita e á esquerda de um ponto central.
Movimento de vaivém.
Hesitação; perplexidade.
\section{Oscillante}
\begin{itemize}
\item {Grp. gram.:adj.}
\end{itemize}
\begin{itemize}
\item {Proveniência:(Lat. \textunderscore oscillans\textunderscore )}
\end{itemize}
Que oscilla.
\section{Oscillar}
\begin{itemize}
\item {Grp. gram.:v. t.}
\end{itemize}
\begin{itemize}
\item {Utilização:Fig.}
\end{itemize}
\begin{itemize}
\item {Proveniência:(Lat. \textunderscore oscillare\textunderscore )}
\end{itemize}
Mover-se alternadamente em sentidos oppostos.
Balançar-se; vacillar.
Hesitar, duvidar.
\section{Oscillatório}
\begin{itemize}
\item {Grp. gram.:adj.}
\end{itemize}
\begin{itemize}
\item {Proveniência:(Do lat. \textunderscore oscillatus\textunderscore )}
\end{itemize}
Oscillante; que é da natureza da oscillação: \textunderscore movimento oscillatório\textunderscore .
\section{Óscina}
\begin{itemize}
\item {Grp. gram.:f.}
\end{itemize}
\begin{itemize}
\item {Proveniência:(Do lat. \textunderscore oscen\textunderscore , \textunderscore oscinis\textunderscore )}
\end{itemize}
Gênero de insectos dípteros.
\section{Óscine}
\begin{itemize}
\item {Grp. gram.:m.}
\end{itemize}
\begin{itemize}
\item {Proveniência:(Do lat. \textunderscore oscen\textunderscore , \textunderscore oscinis\textunderscore )}
\end{itemize}
Ave, por cujo canto os áugures regulavam as suas predicções.
\section{Oscitação}
\begin{itemize}
\item {Grp. gram.:f.}
\end{itemize}
\begin{itemize}
\item {Utilização:Med.}
\end{itemize}
\begin{itemize}
\item {Proveniência:(Lat. \textunderscore oscitatio\textunderscore )}
\end{itemize}
Acto de oscitar; bocejo.
\section{Oscitante}
\begin{itemize}
\item {Grp. gram.:adj.}
\end{itemize}
\begin{itemize}
\item {Proveniência:(Lat. \textunderscore oscitans\textunderscore )}
\end{itemize}
Que oscita.
\section{Oscitar}
\begin{itemize}
\item {Grp. gram.:v. i.}
\end{itemize}
\begin{itemize}
\item {Proveniência:(Lat. \textunderscore oscitari\textunderscore )}
\end{itemize}
O mesmo que \textunderscore bocejar\textunderscore .
\section{Osco}
\begin{itemize}
\item {Grp. gram.:adj.}
\end{itemize}
\begin{itemize}
\item {Grp. gram.:M.}
\end{itemize}
\begin{itemize}
\item {Grp. gram.:Pl.}
\end{itemize}
\begin{itemize}
\item {Proveniência:(Lat. \textunderscore oscus\textunderscore )}
\end{itemize}
Relativo aos Oscos.
Antigo idioma, que se falava na Campânia e tinha grandes relações com o latim.
Povos antiquíssimos da Campânia, em Itália.
\section{Oscofórias}
\begin{itemize}
\item {Grp. gram.:f. pl.}
\end{itemize}
\begin{itemize}
\item {Proveniência:(Gr. \textunderscore oskhophoria\textunderscore )}
\end{itemize}
Antigas festas gregas, em que se levavam ramos, carregados de uvas.
\section{Osculação}
\begin{itemize}
\item {Grp. gram.:f.}
\end{itemize}
\begin{itemize}
\item {Utilização:Geom.}
\end{itemize}
\begin{itemize}
\item {Proveniência:(Lat. \textunderscore osculatio\textunderscore )}
\end{itemize}
Acto ou effeito de oscular.
Contacto de duas curvas.
Cruzamento de dois ramos da mesma curva.
\section{Osculador}
\begin{itemize}
\item {Grp. gram.:adj.}
\end{itemize}
\begin{itemize}
\item {Proveniência:(De \textunderscore oscular\textunderscore )}
\end{itemize}
Que oscula.
Que tem contacto, (falando-se de algumas linhas, em Geometria).
\section{Oscular}
\begin{itemize}
\item {Grp. gram.:v. t.}
\end{itemize}
\begin{itemize}
\item {Proveniência:(Lat. \textunderscore osculari\textunderscore )}
\end{itemize}
Dar ósculo em; beijar.
\section{Osculatório}
\begin{itemize}
\item {Grp. gram.:adj.}
\end{itemize}
\begin{itemize}
\item {Proveniência:(Do lat. \textunderscore osculatus\textunderscore )}
\end{itemize}
Relativo á ósculo.
\section{Osculatriz}
\begin{itemize}
\item {Grp. gram.:f.}
\end{itemize}
Linha osculadora.
(Fem. de \textunderscore osculador\textunderscore )
\section{Ósculo}
\begin{itemize}
\item {Grp. gram.:m.}
\end{itemize}
\begin{itemize}
\item {Proveniência:(Lat. \textunderscore osculum\textunderscore )}
\end{itemize}
O mesmo que \textunderscore beijo\textunderscore .
Pequena abertura, na face externa dos grãos pollínicos.
\section{Osfalgia}
\begin{itemize}
\item {Grp. gram.:f.}
\end{itemize}
\begin{itemize}
\item {Utilização:Med.}
\end{itemize}
\begin{itemize}
\item {Proveniência:(Do gr. \textunderscore osphus\textunderscore  + \textunderscore algos\textunderscore )}
\end{itemize}
Dôr do lombo.
\section{Osfálgico}
\begin{itemize}
\item {Grp. gram.:adj.}
\end{itemize}
Relativo á osfalgia.
\section{Osfialgia}
\begin{itemize}
\item {Grp. gram.:f.}
\end{itemize}
O mesmo ou melhor que \textunderscore asfalgia\textunderscore .
\section{Osfite}
\begin{itemize}
\item {Grp. gram.:f.}
\end{itemize}
\begin{itemize}
\item {Utilização:Med.}
\end{itemize}
Inflamação no lombo.
\section{Osfresia}
\begin{itemize}
\item {Grp. gram.:f.}
\end{itemize}
\begin{itemize}
\item {Proveniência:(Do gr. \textunderscore osphresis\textunderscore )}
\end{itemize}
Grande sensibilidade de olfacto.
Faculdade de sentir facilmente os cheiros.
\section{Osfrésico}
\begin{itemize}
\item {Grp. gram.:adj.}
\end{itemize}
Relativo á osfresia.
\section{Osfresiologia}
\begin{itemize}
\item {Grp. gram.:f.}
\end{itemize}
\begin{itemize}
\item {Proveniência:(Do gr. \textunderscore esphresis\textunderscore  + \textunderscore logos\textunderscore )}
\end{itemize}
Tratado dos cheiros ou do olfato.
\section{Osfresiólogo}
\begin{itemize}
\item {Grp. gram.:m.}
\end{itemize}
Aquele que escreve á cêrca de osfresiologia.
\section{Osfrómeno}
\begin{itemize}
\item {Grp. gram.:m.}
\end{itemize}
Gênero de peixes acantopterígios.
\section{Osga}
\begin{itemize}
\item {Grp. gram.:f.}
\end{itemize}
\begin{itemize}
\item {Proveniência:(Do ár. \textunderscore usga\textunderscore )}
\end{itemize}
Reptil sáurio, (\textunderscore gecko\textunderscore ).
\section{Osga}
\begin{itemize}
\item {Grp. gram.:f.}
\end{itemize}
\begin{itemize}
\item {Utilização:Pop.}
\end{itemize}
Aversão, ódio.
\section{Osia}
\begin{itemize}
\item {Grp. gram.:f.}
\end{itemize}
\begin{itemize}
\item {Utilização:Ant.}
\end{itemize}
Capella-mór.
(Cp. \textunderscore adussia\textunderscore )
\section{Osma}
\begin{itemize}
\item {Grp. gram.:f.}
\end{itemize}
Chusma? malta?:«\textunderscore ...beberete para os da osma\textunderscore ». \textunderscore Aulegrafia\textunderscore , 21.
\section{Osmandi}
\begin{itemize}
\item {Grp. gram.:m.}
\end{itemize}
Língua official da Turquia; o turco.
\section{Osmanil}
\begin{itemize}
\item {Grp. gram.:m.}
\end{itemize}
(V.osmandi)
\section{Osmanli}
\begin{itemize}
\item {Grp. gram.:m.}
\end{itemize}
\begin{itemize}
\item {Grp. gram.:Pl.}
\end{itemize}
\begin{itemize}
\item {Utilização:Ext.}
\end{itemize}
Membro da actual dynastia turca.
Os Turcos.
\section{Osmânico}
\begin{itemize}
\item {Grp. gram.:m.}
\end{itemize}
O mesmo ou melhor que \textunderscore osmandi\textunderscore .
\section{Osmar}
\begin{itemize}
\item {Grp. gram.:v. t.}
\end{itemize}
\begin{itemize}
\item {Utilização:Ant.}
\end{itemize}
O mesmo que \textunderscore esmar\textunderscore .
\section{Osmazoma}
\begin{itemize}
\item {Grp. gram.:f.}
\end{itemize}
\begin{itemize}
\item {Utilização:Chím.}
\end{itemize}
\begin{itemize}
\item {Proveniência:(Do gr. \textunderscore osme\textunderscore  + \textunderscore zomos\textunderscore )}
\end{itemize}
Mescla de substâncias, que existe na carne e nalguns cogumelos, e que se encontra no caldo, na proporção de uma parte por sete de gelatina. Cf. Camillo, \textunderscore Narcót.\textunderscore , I, 265.
\section{Osmazómeo}
\begin{itemize}
\item {Grp. gram.:adj.}
\end{itemize}
Que contém osmazoma.
\section{Ósmia}
\begin{itemize}
\item {Grp. gram.:f.}
\end{itemize}
\begin{itemize}
\item {Proveniência:(Do gr. \textunderscore osme\textunderscore )}
\end{itemize}
Gênero de insectos mellíferos.
\section{Osmiato}
\begin{itemize}
\item {Grp. gram.:m.}
\end{itemize}
\begin{itemize}
\item {Utilização:Chím.}
\end{itemize}
\begin{itemize}
\item {Proveniência:(De \textunderscore ósmio\textunderscore )}
\end{itemize}
Combinação do acido ósmico com uma base.
\section{Ósmico}
\begin{itemize}
\item {Grp. gram.:adj.}
\end{itemize}
\begin{itemize}
\item {Utilização:Chím.}
\end{itemize}
Relativo aos saes e a um dos óxydos do ósmio.
\section{Osmidrose}
\begin{itemize}
\item {Grp. gram.:f.}
\end{itemize}
\begin{itemize}
\item {Utilização:Med.}
\end{itemize}
\begin{itemize}
\item {Proveniência:(Do gr. \textunderscore osme\textunderscore  + \textunderscore hudor\textunderscore )}
\end{itemize}
Secreção abundante de suor, com cheiro desagradável.
\section{Osmimétrico}
\begin{itemize}
\item {Grp. gram.:adj.}
\end{itemize}
\begin{itemize}
\item {Proveniência:(Do gr. \textunderscore osme\textunderscore  + \textunderscore metron\textunderscore )}
\end{itemize}
Que serve para medir ou apreciar os cheiros.
\section{Ósmio}
\begin{itemize}
\item {Grp. gram.:m.}
\end{itemize}
\begin{itemize}
\item {Proveniência:(Do gr. \textunderscore osme\textunderscore )}
\end{itemize}
Metal ou metallóide, que se encontra nos minérios de platina.
\section{Osmioso}
\begin{itemize}
\item {Grp. gram.:adj.}
\end{itemize}
Diz-se de um dos óxydos do ósmio.
\section{Osmoderma}
\begin{itemize}
\item {Grp. gram.:f.}
\end{itemize}
\begin{itemize}
\item {Proveniência:(Do gr. \textunderscore osme\textunderscore  + \textunderscore derma\textunderscore )}
\end{itemize}
Gênero de insectos coleópteros pentâmeros.
\section{Osmologia}
\begin{itemize}
\item {Grp. gram.:f.}
\end{itemize}
\begin{itemize}
\item {Proveniência:(Do gr. \textunderscore osme\textunderscore  + \textunderscore logos\textunderscore )}
\end{itemize}
Tratado á cêrca dos aromas.
\section{Osmológico}
\begin{itemize}
\item {Grp. gram.:adj.}
\end{itemize}
Relativo á osmologia.
\section{Osmonda}
\begin{itemize}
\item {Grp. gram.:f.}
\end{itemize}
Gênero de plantas, da fam. dos fêtos, (\textunderscore osmunda\textunderscore ).
\section{Osmondáceas}
\begin{itemize}
\item {Grp. gram.:f. pl.}
\end{itemize}
\begin{itemize}
\item {Proveniência:(De \textunderscore osmondáceo\textunderscore )}
\end{itemize}
Tríbo de plantas, da fam. dos fêtos.
\section{Osmondáceo}
\begin{itemize}
\item {Grp. gram.:adj.}
\end{itemize}
Relativo ou semelhante á osmonda.
\section{Osmose}
\begin{itemize}
\item {Grp. gram.:f.}
\end{itemize}
\begin{itemize}
\item {Utilização:Phýs.}
\end{itemize}
\begin{itemize}
\item {Proveniência:(Do gr. \textunderscore osmos\textunderscore )}
\end{itemize}
Phenomeno, que se produz quando dois líquidos estão separados por occlusão mais ou menos porosa, e que consiste em realizar-se então uma mistura dos dois líquidos.
\section{Osmótico}
\begin{itemize}
\item {Grp. gram.:adj.}
\end{itemize}
Relativo á osmose.
\section{Osmozomo}
\begin{itemize}
\item {Grp. gram.:m.}
\end{itemize}
O mesmo ou melhor que \textunderscore osmazoma\textunderscore . Cf. R. Galvão, \textunderscore Vocab.\textunderscore 
\section{Osmunda}
\begin{itemize}
\item {Grp. gram.:f.}
\end{itemize}
O mesmo que \textunderscore osmonda\textunderscore .
\section{Osória}
\begin{itemize}
\item {Grp. gram.:f.}
\end{itemize}
\begin{itemize}
\item {Proveniência:(Do lat. \textunderscore osor\textunderscore )}
\end{itemize}
Gênero de insectos coleópteros pentâmeros.
\section{Ospedágio}
\begin{itemize}
\item {Grp. gram.:m.}
\end{itemize}
\begin{itemize}
\item {Utilização:Ant.}
\end{itemize}
O mesmo que \textunderscore hospedagem\textunderscore . Cf. Frei Fortun., \textunderscore Inéd.\textunderscore , 311.
\section{Osphalgia}
\begin{itemize}
\item {Grp. gram.:f.}
\end{itemize}
\begin{itemize}
\item {Utilização:Med.}
\end{itemize}
\begin{itemize}
\item {Proveniência:(Do gr. \textunderscore osphus\textunderscore  + \textunderscore algos\textunderscore )}
\end{itemize}
Dôr do lombo.
\section{Osphálgico}
\begin{itemize}
\item {Grp. gram.:adj.}
\end{itemize}
Relativo á osphalgia.
\section{Osphite}
\begin{itemize}
\item {Grp. gram.:f.}
\end{itemize}
\begin{itemize}
\item {Utilização:Med.}
\end{itemize}
Inflammação no lombo.
\section{Osphresia}
\begin{itemize}
\item {Grp. gram.:f.}
\end{itemize}
\begin{itemize}
\item {Proveniência:(Do gr. \textunderscore osphresis\textunderscore )}
\end{itemize}
Grande sensibilidade de olfacto.
Faculdade de sentir facilmente os cheiros.
\section{Osphrésico}
\begin{itemize}
\item {Grp. gram.:adj.}
\end{itemize}
Relativo á osphresia.
\section{Osphresiologia}
\begin{itemize}
\item {Grp. gram.:f.}
\end{itemize}
\begin{itemize}
\item {Proveniência:(Do gr. \textunderscore esphresis\textunderscore  + \textunderscore logos\textunderscore )}
\end{itemize}
Tratado dos cheiros ou do olfato.
\section{Osphresiólogo}
\begin{itemize}
\item {Grp. gram.:m.}
\end{itemize}
Aquelle que escreve á cêrca de osphresiologia.
\section{Osphrómeno}
\begin{itemize}
\item {Grp. gram.:m.}
\end{itemize}
Gênero de peixes acanthopterýgios.
\section{Osphyalgia}
\begin{itemize}
\item {Grp. gram.:f.}
\end{itemize}
O mesmo ou melhor que \textunderscore asphalgia\textunderscore .
\section{Ospitação}
\begin{itemize}
\item {Grp. gram.:f.}
\end{itemize}
\begin{itemize}
\item {Utilização:Ant.}
\end{itemize}
Obrigação de dar poisada ou aposentadoria.
(Cp. lat. \textunderscore hospitari\textunderscore , hospedar)
\section{Osqueíte}
\begin{itemize}
\item {Grp. gram.:f.}
\end{itemize}
\begin{itemize}
\item {Utilização:Med.}
\end{itemize}
\begin{itemize}
\item {Proveniência:(Do gr. \textunderscore oskheon\textunderscore )}
\end{itemize}
Inflamação do escroto.
\section{Osqueocalasia}
\begin{itemize}
\item {Grp. gram.:f.}
\end{itemize}
\begin{itemize}
\item {Utilização:Med.}
\end{itemize}
\begin{itemize}
\item {Proveniência:(Do gr. \textunderscore oskheon\textunderscore  + \textunderscore kalasis\textunderscore )}
\end{itemize}
Tumor, resultante da hipertrophia do tecido celular do escroto.
\section{Osqueocele}
\begin{itemize}
\item {Grp. gram.:m.}
\end{itemize}
\begin{itemize}
\item {Utilização:Med.}
\end{itemize}
\begin{itemize}
\item {Proveniência:(Do gr. \textunderscore oskheon\textunderscore  + \textunderscore kele\textunderscore )}
\end{itemize}
Hérnia do escroto.
\section{Osqueólito}
\begin{itemize}
\item {Grp. gram.:m.}
\end{itemize}
\begin{itemize}
\item {Utilização:Med.}
\end{itemize}
\begin{itemize}
\item {Proveniência:(Do gr. \textunderscore oscheon\textunderscore  + \textunderscore lithos\textunderscore )}
\end{itemize}
Concreção calcária no escroto.
\section{Osqueôma}
\begin{itemize}
\item {Grp. gram.:m.}
\end{itemize}
\begin{itemize}
\item {Utilização:Med.}
\end{itemize}
\begin{itemize}
\item {Proveniência:(Do gr. \textunderscore osklion\textunderscore )}
\end{itemize}
Tumor no escroto.
\section{Osqueoplastia}
\begin{itemize}
\item {Grp. gram.:f.}
\end{itemize}
\begin{itemize}
\item {Utilização:Cir.}
\end{itemize}
\begin{itemize}
\item {Proveniência:(Do gr. \textunderscore oskheon\textunderscore  + \textunderscore plassein\textunderscore )}
\end{itemize}
Reparação do escroto, pelos processos autoplásticos.
\section{Ossa}
\begin{itemize}
\item {Grp. gram.:f.}
\end{itemize}
O mesmo que \textunderscore osa\textunderscore .
\section{Ossada}
\begin{itemize}
\item {Grp. gram.:f.}
\end{itemize}
\begin{itemize}
\item {Utilização:Fig.}
\end{itemize}
Grande quantidade de ossos.
Esqueleto.
Destroços, restos.
Paredes e armação de um edifício.
Ruínas.
Alicerces.
\section{Ossamenta}
\begin{itemize}
\item {Grp. gram.:f.}
\end{itemize}
\begin{itemize}
\item {Proveniência:(De \textunderscore osso\textunderscore )}
\end{itemize}
O mesmo que \textunderscore esqueleto\textunderscore .
\section{Ossamento}
\begin{itemize}
\item {Grp. gram.:m.}
\end{itemize}
(V.ossamenta)
\section{Ossaria}
\begin{itemize}
\item {Grp. gram.:f.}
\end{itemize}
Montão de ossos.
Lugar, onde se guardam ossos; ossuário.
\section{Ossário}
\begin{itemize}
\item {Grp. gram.:m.}
\end{itemize}
\begin{itemize}
\item {Proveniência:(Lat. \textunderscore ossarius\textunderscore )}
\end{itemize}
O mesmo que \textunderscore ossaria\textunderscore .
\section{Ossatura}
\begin{itemize}
\item {Grp. gram.:f.}
\end{itemize}
\begin{itemize}
\item {Proveniência:(De \textunderscore osso\textunderscore )}
\end{itemize}
Ossos de animal.
Esqueleto; ossada.
\section{Ósseo}
\begin{itemize}
\item {Grp. gram.:adj.}
\end{itemize}
\begin{itemize}
\item {Proveniência:(Lat. \textunderscore osseus\textunderscore )}
\end{itemize}
Relativo a osso.
Que tem a natureza de osso.
Que tem ossos.
\section{Osservar}
\textunderscore v. t.\textunderscore  (e der.)
(Fórma pop. de \textunderscore observar\textunderscore , etc.)
\section{Ossetas}
\begin{itemize}
\item {Grp. gram.:m. pl.}
\end{itemize}
\begin{itemize}
\item {Proveniência:(De \textunderscore Osset\textunderscore , n. p.)}
\end{itemize}
Povos, que na Bética ficaram representando os Alanos.
\section{Osseto}
\begin{itemize}
\item {Grp. gram.:m.}
\end{itemize}
Dialecto do grupo irânico.
\section{Ossia}
\begin{itemize}
\item {Grp. gram.:f.}
\end{itemize}
\begin{itemize}
\item {Utilização:Ant.}
\end{itemize}
O mesmo que \textunderscore adussia\textunderscore .
\section{Ossiandrianismo}
\begin{itemize}
\item {Grp. gram.:m.}
\end{itemize}
Seita ou doutrina dos Ossiandrianos.
\section{Ossiandrianos}
\begin{itemize}
\item {Grp. gram.:m. pl.}
\end{itemize}
Sectários protestantes de Ossiandro, discípulo de Luthero.
\section{Ossiânico}
\begin{itemize}
\item {Grp. gram.:adj.}
\end{itemize}
Relativo ás poesias attribuídas a Ossian.
Que tem o carácter dessas poesias.
\section{Ossianismo}
\begin{itemize}
\item {Grp. gram.:m.}
\end{itemize}
Imitação literária de Ossian.
Gênero literário de Ossian.
Admiração excessiva do gênero literário ou das poesias de Ossian.
\section{Ossianista}
\begin{itemize}
\item {Grp. gram.:m.}
\end{itemize}
Imitador de Ossian.
Admirador fanático das poesias de Ossian.
\section{Ossicos}
\begin{itemize}
\item {Grp. gram.:m. pl.}
\end{itemize}
\begin{itemize}
\item {Proveniência:(De \textunderscore osso\textunderscore )}
\end{itemize}
O vómer das bêstas.
\section{Ossiculado}
\begin{itemize}
\item {Grp. gram.:adj.}
\end{itemize}
Que tem ossículos ou ossos; ósseo.
\section{Ossicular}
\begin{itemize}
\item {Grp. gram.:adj.}
\end{itemize}
\begin{itemize}
\item {Proveniência:(Lat. \textunderscore ossicularis\textunderscore )}
\end{itemize}
Relativo ou semelhante a ossículo.
\section{Ossículo}
\begin{itemize}
\item {Grp. gram.:m.}
\end{itemize}
\begin{itemize}
\item {Grp. gram.:Pl.}
\end{itemize}
\begin{itemize}
\item {Utilização:Bot.}
\end{itemize}
\begin{itemize}
\item {Proveniência:(Lat. \textunderscore ossiculum\textunderscore )}
\end{itemize}
Osso pequeno.
Os quatro pequenos ossos do ouvido.
Caroço dos frutos, quando pequeno e não divisível em duas válvulas.
\section{Ossífero}
\begin{itemize}
\item {Grp. gram.:adj.}
\end{itemize}
\begin{itemize}
\item {Proveniência:(Do lat. \textunderscore os\textunderscore , \textunderscore ossis\textunderscore  + \textunderscore ferre\textunderscore )}
\end{itemize}
Que tem ossos.
\section{Ossificação}
\begin{itemize}
\item {Grp. gram.:f.}
\end{itemize}
Acto ou effeito de ossificar.
Formação de ossos.
\section{Ossificar}
\begin{itemize}
\item {Grp. gram.:v. t.}
\end{itemize}
\begin{itemize}
\item {Proveniência:(Do lat. \textunderscore os\textunderscore , \textunderscore ossis\textunderscore  + \textunderscore facere\textunderscore )}
\end{itemize}
Converter em osso.
Mudar em tecido ósseo (outro tecido).
Tornar duro.
\section{Ossífico}
\begin{itemize}
\item {Grp. gram.:adj.}
\end{itemize}
\begin{itemize}
\item {Proveniência:(De \textunderscore ossificar\textunderscore )}
\end{itemize}
Que concorre para a ossificação; ossificado.
\section{Ossifluente}
\begin{itemize}
\item {Grp. gram.:adj.}
\end{itemize}
\begin{itemize}
\item {Utilização:Med.}
\end{itemize}
\begin{itemize}
\item {Proveniência:(De \textunderscore osso\textunderscore  + \textunderscore fluente\textunderscore )}
\end{itemize}
Diz-se do abscesso, formado numa articulação, á custa da decomposição dos ossos.
\section{Ossiforme}
\begin{itemize}
\item {Grp. gram.:adj.}
\end{itemize}
\begin{itemize}
\item {Proveniência:(Do lat. \textunderscore os\textunderscore , \textunderscore ossis\textunderscore  + \textunderscore forma\textunderscore )}
\end{itemize}
Que tem fórma de osso.
\section{Ossífraga}
\begin{itemize}
\item {Grp. gram.:f.}
\end{itemize}
\begin{itemize}
\item {Proveniência:(De \textunderscore ossífrago\textunderscore )}
\end{itemize}
Nome scientífico do xofrango.
\section{Ossífrago}
\begin{itemize}
\item {Grp. gram.:adj.}
\end{itemize}
\begin{itemize}
\item {Proveniência:(Do lat. \textunderscore os\textunderscore , \textunderscore ossis\textunderscore  + \textunderscore frangere\textunderscore )}
\end{itemize}
Que amollece ou parte os ossos, ou determina a fractura delles.
\section{Ossinho}
\begin{itemize}
\item {Grp. gram.:m.}
\end{itemize}
\begin{itemize}
\item {Utilização:Veter.}
\end{itemize}
O mesmo que \textunderscore ossículo\textunderscore .
Exostose, na canela, ou no pé das cavalgaduras. Cf. Macedo Pinto, \textunderscore Comp. de Veter.\textunderscore , I, 316.
\section{Ossívoro}
\begin{itemize}
\item {Grp. gram.:adj.}
\end{itemize}
\begin{itemize}
\item {Proveniência:(Do lat. \textunderscore os\textunderscore , \textunderscore ossis\textunderscore  + \textunderscore vorare\textunderscore )}
\end{itemize}
Que come ossos; que carcome os ossos.
\section{Osso}
\begin{itemize}
\item {fónica:ô}
\end{itemize}
\begin{itemize}
\item {Grp. gram.:m.}
\end{itemize}
\begin{itemize}
\item {Utilização:Fig.}
\end{itemize}
\begin{itemize}
\item {Grp. gram.:Loc.}
\end{itemize}
\begin{itemize}
\item {Utilização:fam.}
\end{itemize}
\begin{itemize}
\item {Grp. gram.:Loc.}
\end{itemize}
\begin{itemize}
\item {Utilização:fam.}
\end{itemize}
\begin{itemize}
\item {Grp. gram.:Pl.}
\end{itemize}
\begin{itemize}
\item {Utilização:Fig.}
\end{itemize}
\begin{itemize}
\item {Proveniência:(Lat. \textunderscore os\textunderscore , \textunderscore ossis\textunderscore )}
\end{itemize}
Parte dura e sólida, que fórma o arcaboiço do corpo dos animaes vertebrados.
Fragmento ou parte dêsse arcaboiço.
Difficuldade.
A parte diffícil de um emprehendimento.
\textunderscore Ossos do offício\textunderscore , encargos ou difficuldades inherentes a certas vantagens, ou que destas resultam naturalmente.
\textunderscore Carne sem osso\textunderscore , lucro ou vantagem perfeita, sem a menor contrariedade.
Restos mortaes.
A vida.
\section{Osso}
\begin{itemize}
\item {fónica:ô}
\end{itemize}
\begin{itemize}
\item {Grp. gram.:m.}
\end{itemize}
\begin{itemize}
\item {Utilização:Ant.}
\end{itemize}
O mesmo que \textunderscore urso\textunderscore .
(Cp. cast. \textunderscore oso\textunderscore )
\section{Ossobô}
\begin{itemize}
\item {Grp. gram.:m.}
\end{itemize}
Ave santhomense, cujo canto costuma annunciar chuva.
\section{Ossonobense}
\begin{itemize}
\item {Grp. gram.:adj.}
\end{itemize}
Que é de Ossónoba. Cf. Herculano, \textunderscore Hist. de Port.\textunderscore , III, 27.
\section{Ossoró}
\begin{itemize}
\item {Grp. gram.:m.}
\end{itemize}
\begin{itemize}
\item {Utilização:T. da Índia port}
\end{itemize}
Sala de recepção.
\section{Ossuário}
\begin{itemize}
\item {Grp. gram.:m.}
\end{itemize}
\begin{itemize}
\item {Proveniência:(Lat. \textunderscore ossuarius\textunderscore )}
\end{itemize}
O mesmo ou melhor que \textunderscore ossário\textunderscore .
Depósito de ossos humanos.
Sepultura commum de muitos cadáveres.
\section{Ossudo}
\begin{itemize}
\item {Grp. gram.:adj.}
\end{itemize}
Que tem grandes ossos, ou ossos muito salientes.
\section{Ossuoso}
\begin{itemize}
\item {Grp. gram.:adj.}
\end{itemize}
O mesmo que \textunderscore ósseo\textunderscore .
\section{Ostaga}
\begin{itemize}
\item {Grp. gram.:f.}
\end{itemize}
\begin{itemize}
\item {Utilização:Náut.}
\end{itemize}
Cabo grosso, que vem por cima da pêga e sustenta as vêrgas em seus moitões.
(Cast. \textunderscore ostaga\textunderscore )
\section{Ostagadura}
\begin{itemize}
\item {Grp. gram.:f.}
\end{itemize}
\begin{itemize}
\item {Utilização:Náut.}
\end{itemize}
Lugar da vêrga, onde se firmam as ostagas.
\section{Ostagra}
\begin{itemize}
\item {Grp. gram.:f.}
\end{itemize}
\begin{itemize}
\item {Proveniência:(Do gr. \textunderscore osteon\textunderscore  + \textunderscore agra\textunderscore )}
\end{itemize}
Instrumento cirúrgico, para elevar, deprimir ou fazer mover os ossos.
\section{Ostai}
\begin{itemize}
\item {Grp. gram.:m.}
\end{itemize}
\begin{itemize}
\item {Utilização:Des.}
\end{itemize}
O mesmo que \textunderscore estai\textunderscore .
\section{Ostar}
\begin{itemize}
\item {Grp. gram.:v. t.}
\end{itemize}
\begin{itemize}
\item {Utilização:Ant.}
\end{itemize}
O mesmo que \textunderscore obstar\textunderscore . Cf. \textunderscore Aulegrafia\textunderscore , 158.
\section{Ostaxa}
\begin{itemize}
\item {Grp. gram.:f.}
\end{itemize}
(V.ostaga)
\section{Oste}
\begin{itemize}
\item {Grp. gram.:m.}
\end{itemize}
\begin{itemize}
\item {Utilização:Ant.}
\end{itemize}
O mesmo que \textunderscore hóspede\textunderscore .
(Contr. de \textunderscore hóspite\textunderscore )
\section{Ostealgia}
\begin{itemize}
\item {Grp. gram.:f.}
\end{itemize}
\begin{itemize}
\item {Proveniência:(Do gr. \textunderscore osteon\textunderscore  + \textunderscore algos\textunderscore )}
\end{itemize}
Dôr nos ossos.
\section{Osteálgico}
\begin{itemize}
\item {Grp. gram.:adj.}
\end{itemize}
Relativo á ostealgia.
\section{Osteda}
\begin{itemize}
\item {fónica:tê}
\end{itemize}
\begin{itemize}
\item {Grp. gram.:f.}
\end{itemize}
\begin{itemize}
\item {Utilização:Ant.}
\end{itemize}
\begin{itemize}
\item {Proveniência:(De \textunderscore Ostende\textunderscore , n. p.?)}
\end{itemize}
Espécie de tecido.
\section{Ostedilha}
\begin{itemize}
\item {Grp. gram.:f.}
\end{itemize}
\begin{itemize}
\item {Utilização:Ant.}
\end{itemize}
Osteda fina.
\section{Osteide}
\begin{itemize}
\item {Grp. gram.:m.}
\end{itemize}
\begin{itemize}
\item {Proveniência:(Do gr. \textunderscore osteon\textunderscore  + \textunderscore eidos\textunderscore )}
\end{itemize}
O mesmo que \textunderscore dente\textunderscore .
Formação óssea, accidental, geralmente mórbida.
Concreção pedregosa, com a apparência de osso.
\section{Osteína}
\begin{itemize}
\item {Grp. gram.:f.}
\end{itemize}
\begin{itemize}
\item {Proveniência:(Do gr. \textunderscore osteon\textunderscore )}
\end{itemize}
Substância orgânica, própria do tecido ósseo.
\section{Osteíte}
\begin{itemize}
\item {Grp. gram.:f.}
\end{itemize}
\begin{itemize}
\item {Proveniência:(Do gr. \textunderscore osteon\textunderscore )}
\end{itemize}
Inflammação do tecido ósseo.
\section{Ostende}
\begin{itemize}
\item {Grp. gram.:m.}
\end{itemize}
\begin{itemize}
\item {Proveniência:(De \textunderscore Ostende\textunderscore , n. p.)}
\end{itemize}
Antiga moéda de Flandres. Cf. F. Manuel, \textunderscore Apólogos\textunderscore .
\section{Osirícera}
\begin{itemize}
\item {Grp. gram.:f.}
\end{itemize}
\begin{itemize}
\item {Proveniência:(Do gr. \textunderscore osuris\textunderscore  + \textunderscore keras\textunderscore )}
\end{itemize}
Gênero de orquídeas.
\section{Osíride}
\begin{itemize}
\item {Grp. gram.:f.}
\end{itemize}
\begin{itemize}
\item {Proveniência:(Do gr. \textunderscore osuris\textunderscore )}
\end{itemize}
Gênero de plantas, o mesmo que \textunderscore valverde\textunderscore .
\section{Osirídeas}
\begin{itemize}
\item {Grp. gram.:f. pl.}
\end{itemize}
\begin{itemize}
\item {Proveniência:(De \textunderscore osíride\textunderscore )}
\end{itemize}
Família de plantas, criada por Brown, mas que, segundo Jussieu, deve sêr compreendida nas santaláceas.
\section{Ostender}
\begin{itemize}
\item {Grp. gram.:v. t.}
\end{itemize}
\begin{itemize}
\item {Utilização:Ant.}
\end{itemize}
\begin{itemize}
\item {Proveniência:(Lat. \textunderscore ostendere\textunderscore )}
\end{itemize}
Mostrar, ostentar.
\section{Ostensão}
\begin{itemize}
\item {Grp. gram.:f.}
\end{itemize}
\begin{itemize}
\item {Proveniência:(Lat. \textunderscore ostensio\textunderscore )}
\end{itemize}
O mesmo que \textunderscore ostentação\textunderscore .
Acto ou effeito de mostrar.
\section{Ostensível}
\begin{itemize}
\item {Grp. gram.:adj.}
\end{itemize}
O mesmo que \textunderscore ostensivo\textunderscore .
\section{Ostensivelmente}
\begin{itemize}
\item {Grp. gram.:adv.}
\end{itemize}
De modo ostensível.
\section{Ostensivo}
\begin{itemize}
\item {Grp. gram.:adj.}
\end{itemize}
\begin{itemize}
\item {Proveniência:(Lat. \textunderscore ostensivus\textunderscore )}
\end{itemize}
Que se póde mostrar.
Próprio para se mostrar.
Que se patenteia; apparente: \textunderscore motivos ostensivos\textunderscore .
\section{Ostensor}
\begin{itemize}
\item {Grp. gram.:m.  e  adj.}
\end{itemize}
\begin{itemize}
\item {Proveniência:(Lat. \textunderscore ostensor\textunderscore )}
\end{itemize}
O que mostra.
\section{Ostensório}
\begin{itemize}
\item {Grp. gram.:adj.}
\end{itemize}
(V.ostensivo)
\section{Ostentação}
\begin{itemize}
\item {Grp. gram.:f.}
\end{itemize}
\begin{itemize}
\item {Proveniência:(Lat. \textunderscore ostentatio\textunderscore )}
\end{itemize}
Acto ou effeito de ostentar.
Vanglória.
Alarde de riquezas, actos ou qualidades próprias.
Exhibição vaidosa.
Apparato; pompa; luxo; magnificência.
\section{Ostentador}
\begin{itemize}
\item {Grp. gram.:adj.}
\end{itemize}
\begin{itemize}
\item {Proveniência:(Lat. \textunderscore ostentator\textunderscore )}
\end{itemize}
Aquelle que procede ou falla com ostentação.
\section{Ostentar}
\begin{itemize}
\item {Grp. gram.:v. t.}
\end{itemize}
\begin{itemize}
\item {Proveniência:(Lat. \textunderscore ostentare\textunderscore )}
\end{itemize}
Exhibir apparatosamente; pompear; alardear; garrir.
Revelar gloriosamente.
\section{Ostentativo}
\begin{itemize}
\item {Grp. gram.:adj.}
\end{itemize}
\begin{itemize}
\item {Proveniência:(Lat. \textunderscore ostentativus\textunderscore )}
\end{itemize}
Que ostenta; ostensivo.
\section{Ostentosamente}
\begin{itemize}
\item {Grp. gram.:adv.}
\end{itemize}
De modo ostentoso.
Com ostentação; pomposamente.
\section{Ostentoso}
\begin{itemize}
\item {Grp. gram.:adj.}
\end{itemize}
\begin{itemize}
\item {Proveniência:(De \textunderscore ostentar\textunderscore )}
\end{itemize}
Feito com ostentação; magnifico; bizarro.
Esplêndido.
Monumental.
Brilhante.
\section{Ósteo...}
\begin{itemize}
\item {Grp. gram.:pref.}
\end{itemize}
\begin{itemize}
\item {Proveniência:(Do gr. \textunderscore osteon\textunderscore )}
\end{itemize}
(designativo de \textunderscore osso\textunderscore )
\section{Osteocele}
\begin{itemize}
\item {Grp. gram.:m.}
\end{itemize}
\begin{itemize}
\item {Utilização:Med.}
\end{itemize}
\begin{itemize}
\item {Proveniência:(Do gr. \textunderscore osteon\textunderscore  + \textunderscore kele\textunderscore )}
\end{itemize}
Tumor, produzido pela ossificação de um saco herniário.
\section{Osteoclasia}
\begin{itemize}
\item {Grp. gram.:f.}
\end{itemize}
\begin{itemize}
\item {Utilização:Cir.}
\end{itemize}
\begin{itemize}
\item {Proveniência:(Do gr. \textunderscore osteon\textunderscore  + \textunderscore klasis\textunderscore )}
\end{itemize}
Acto de quebrar um osso com fins therapêuticos.
\section{Osteoclastia}
\begin{itemize}
\item {Grp. gram.:f.}
\end{itemize}
O mesmo que \textunderscore osteoclasia\textunderscore .
\section{Osteocola}
\begin{itemize}
\item {Grp. gram.:f.}
\end{itemize}
\begin{itemize}
\item {Proveniência:(Gr. \textunderscore osteokolla\textunderscore )}
\end{itemize}
Carbonato de cal, que se deposita nos corpos estranhos, mergulhados na água impregnada dêste sal, e ao qual se atribuia dantes a propriedade de unir ossos fracturados.
\section{Osteocolla}
\begin{itemize}
\item {Grp. gram.:f.}
\end{itemize}
\begin{itemize}
\item {Proveniência:(Gr. \textunderscore osteokolla\textunderscore )}
\end{itemize}
Carbonato de cal, que se deposita nos corpos estranhos, mergulhados na água impregnada dêste sal, e ao qual se attribuia dantes a propriedade de unir ossos fracturados.
\section{Osteócopo}
\begin{itemize}
\item {Grp. gram.:adj.}
\end{itemize}
\begin{itemize}
\item {Utilização:Med.}
\end{itemize}
\begin{itemize}
\item {Proveniência:(Gr. \textunderscore osteokopos\textunderscore )}
\end{itemize}
Diz-se de certas dores nos ossos, geralmente procedentes da sýphilis.
\section{Osteodermo}
\begin{itemize}
\item {Grp. gram.:adj.}
\end{itemize}
\begin{itemize}
\item {Grp. gram.:M. Pl.}
\end{itemize}
\begin{itemize}
\item {Proveniência:(Do gr. \textunderscore osteon\textunderscore  + \textunderscore derma\textunderscore )}
\end{itemize}
Que tem a pelle muito endurecida.
Família de peixes osteodermos.
\section{Osteodesmo}
\begin{itemize}
\item {Grp. gram.:m.}
\end{itemize}
Gênero de molluscos.
\section{Osteodinia}
\begin{itemize}
\item {Grp. gram.:f.}
\end{itemize}
\begin{itemize}
\item {Utilização:Med.}
\end{itemize}
\begin{itemize}
\item {Proveniência:(Do gr. \textunderscore osteon\textunderscore  + \textunderscore odune\textunderscore )}
\end{itemize}
Dôr nos ossos.
\section{Osteodínico}
\begin{itemize}
\item {Grp. gram.:adj.}
\end{itemize}
Relativo á osteodinia.
\section{Osteodynia}
\begin{itemize}
\item {Grp. gram.:f.}
\end{itemize}
\begin{itemize}
\item {Utilização:Med.}
\end{itemize}
\begin{itemize}
\item {Proveniência:(Do gr. \textunderscore osteon\textunderscore  + \textunderscore odune\textunderscore )}
\end{itemize}
Dôr nos ossos.
\section{Osteodýnico}
\begin{itemize}
\item {Grp. gram.:adj.}
\end{itemize}
Relativo á osteodynia.
\section{Osteófago}
\begin{itemize}
\item {Grp. gram.:adj.}
\end{itemize}
\begin{itemize}
\item {Proveniência:(Do gr. \textunderscore osteon\textunderscore  + \textunderscore phagein\textunderscore )}
\end{itemize}
Que come ossos.
\section{Osteofimia}
\begin{itemize}
\item {Grp. gram.:f.}
\end{itemize}
\begin{itemize}
\item {Utilização:Med.}
\end{itemize}
\begin{itemize}
\item {Proveniência:(Do gr. \textunderscore osteon\textunderscore  + \textunderscore phuma\textunderscore )}
\end{itemize}
Tuberculose nos ossos.
\section{Osteófito}
\begin{itemize}
\item {Grp. gram.:f.}
\end{itemize}
\begin{itemize}
\item {Utilização:Med.}
\end{itemize}
\begin{itemize}
\item {Proveniência:(Do gr. \textunderscore osteon\textunderscore  + \textunderscore phuton\textunderscore )}
\end{itemize}
Produção óssea, que resulta algumas vezes das lâminas profundas do periósteo, junto da parte cariada de um ôsso.
\section{Osteogênese}
\begin{itemize}
\item {Grp. gram.:f.}
\end{itemize}
\begin{itemize}
\item {Proveniência:(Do gr. \textunderscore osteon\textunderscore  + \textunderscore genesis\textunderscore )}
\end{itemize}
Formação dos ossos.
\section{Osteogenético}
\begin{itemize}
\item {Grp. gram.:adj.}
\end{itemize}
Relativo a osteogênese.
\section{Osteogenia}
\begin{itemize}
\item {Grp. gram.:f.}
\end{itemize}
\begin{itemize}
\item {Proveniência:(Do gr. \textunderscore osteon\textunderscore  + \textunderscore genea\textunderscore )}
\end{itemize}
Estudo da geração e desenvolvimento da substância dos ossos, do seu tecido e do seu systema.
\section{Osteogênico}
\begin{itemize}
\item {Grp. gram.:adj.}
\end{itemize}
Relativo á osteogenia.
\section{Osteografia}
\begin{itemize}
\item {Grp. gram.:f.}
\end{itemize}
\begin{itemize}
\item {Proveniência:(De \textunderscore osteógrafo\textunderscore )}
\end{itemize}
Descripção ou tratado dos ossos.
\section{Osteográfico}
\begin{itemize}
\item {Grp. gram.:adj.}
\end{itemize}
Relativo á osteografia.
\section{Osteógrafo}
\begin{itemize}
\item {Grp. gram.:m.}
\end{itemize}
\begin{itemize}
\item {Proveniência:(Do gr. \textunderscore osteon\textunderscore  + \textunderscore graphein\textunderscore )}
\end{itemize}
Aquele que se ocupa de osteografia.
\section{Osteographia}
\begin{itemize}
\item {Grp. gram.:f.}
\end{itemize}
\begin{itemize}
\item {Proveniência:(De \textunderscore osteógrapho\textunderscore )}
\end{itemize}
Descripção ou tratado dos ossos.
\section{Osteográphico}
\begin{itemize}
\item {Grp. gram.:adj.}
\end{itemize}
Relativo á osteographia.
\section{Osteógrapho}
\begin{itemize}
\item {Grp. gram.:m.}
\end{itemize}
\begin{itemize}
\item {Proveniência:(Do gr. \textunderscore osteon\textunderscore  + \textunderscore graphein\textunderscore )}
\end{itemize}
Aquelle que se occupa de osteographia.
\section{Osteólise}
\begin{itemize}
\item {Grp. gram.:f.}
\end{itemize}
\begin{itemize}
\item {Utilização:Med.}
\end{itemize}
\begin{itemize}
\item {Proveniência:(Do gr. \textunderscore osteon\textunderscore  + \textunderscore lusis\textunderscore )}
\end{itemize}
Alteração particular do tecido ósseo, a qual produz a destruição dêsse tecido, sem deixar resíduo.
\section{Osteólitho}
\begin{itemize}
\item {Grp. gram.:m.}
\end{itemize}
\begin{itemize}
\item {Proveniência:(Do gr. \textunderscore osteon\textunderscore  + \textunderscore lithos\textunderscore )}
\end{itemize}
Osso fóssil, osso petrificado.
\section{Osteólito}
\begin{itemize}
\item {Grp. gram.:m.}
\end{itemize}
\begin{itemize}
\item {Proveniência:(Do gr. \textunderscore osteon\textunderscore  + \textunderscore lithos\textunderscore )}
\end{itemize}
Osso fóssil, osso petrificado.
\section{Osteologia}
\begin{itemize}
\item {Grp. gram.:f.}
\end{itemize}
\begin{itemize}
\item {Proveniência:(De \textunderscore osteólogo\textunderscore )}
\end{itemize}
Parte da Anatomia, que trata dos ossos.
\section{Osteológico}
\begin{itemize}
\item {Grp. gram.:adj.}
\end{itemize}
Relativo á osteologia.
\section{Osteólogo}
\begin{itemize}
\item {Grp. gram.:m.}
\end{itemize}
\begin{itemize}
\item {Proveniência:(Do gr. \textunderscore osteon\textunderscore  + \textunderscore logos\textunderscore )}
\end{itemize}
Aquelle que é versado em osteologia.
\section{Osteólyse}
\begin{itemize}
\item {Grp. gram.:f.}
\end{itemize}
\begin{itemize}
\item {Utilização:Med.}
\end{itemize}
\begin{itemize}
\item {Proveniência:(Do gr. \textunderscore osteon\textunderscore  + \textunderscore lusis\textunderscore )}
\end{itemize}
Alteração particular do tecido ósseo, a qual produz a destruição dêsse tecido, sem deixar resíduo.
\section{Osteoma}
\begin{itemize}
\item {Grp. gram.:m.}
\end{itemize}
\begin{itemize}
\item {Proveniência:(Do gr. \textunderscore osteon\textunderscore )}
\end{itemize}
Tumor, composto de tecido ósseo.
\section{Osteomalácia}
\begin{itemize}
\item {Grp. gram.:f.}
\end{itemize}
\begin{itemize}
\item {Proveniência:(Do gr. \textunderscore osteon\textunderscore  + \textunderscore malakos\textunderscore )}
\end{itemize}
Amollecimento dos ossos.
\section{Osteomérico}
\begin{itemize}
\item {Grp. gram.:adj.}
\end{itemize}
Relativo ao osteómero.
\section{Osteomério}
\begin{itemize}
\item {Grp. gram.:m.}
\end{itemize}
O mesmo que \textunderscore osteómero\textunderscore .
\section{Osteómero}
\begin{itemize}
\item {Grp. gram.:m.}
\end{itemize}
\begin{itemize}
\item {Proveniência:(Do gr. \textunderscore osteon\textunderscore  + \textunderscore meros\textunderscore )}
\end{itemize}
Parte óssea do metâmero.
\section{Osteometria}
\begin{itemize}
\item {Grp. gram.:f.}
\end{itemize}
\begin{itemize}
\item {Proveniência:(Do gr. \textunderscore osteon\textunderscore  + \textunderscore metron\textunderscore )}
\end{itemize}
Medição dos ossos, nos estudos anthropológicos.
\section{Osteométrico}
\begin{itemize}
\item {Grp. gram.:adj.}
\end{itemize}
Relativo á osteometria.
\section{Osteomielite}
\begin{itemize}
\item {Grp. gram.:f.}
\end{itemize}
\begin{itemize}
\item {Proveniência:(Do gr. \textunderscore osteon\textunderscore  + \textunderscore muelos\textunderscore )}
\end{itemize}
Inflamação da medula dos ossos.
\section{Osteomyelite}
\begin{itemize}
\item {Grp. gram.:f.}
\end{itemize}
\begin{itemize}
\item {Proveniência:(Do gr. \textunderscore osteon\textunderscore  + \textunderscore muelos\textunderscore )}
\end{itemize}
Inflammação da medulla dos ossos.
\section{Osteonecrose}
\begin{itemize}
\item {Grp. gram.:f.}
\end{itemize}
\begin{itemize}
\item {Utilização:Med.}
\end{itemize}
\begin{itemize}
\item {Proveniência:(Do gr. \textunderscore osteon\textunderscore  + \textunderscore nekrosis\textunderscore )}
\end{itemize}
Mortificação ou gangrena dos ossos.
\section{Osteopathia}
\begin{itemize}
\item {Grp. gram.:f.}
\end{itemize}
\begin{itemize}
\item {Proveniência:(Do gr. \textunderscore osteon\textunderscore  + \textunderscore pathos\textunderscore )}
\end{itemize}
Doença dos ossos, em geral.
\section{Osteopáthico}
\begin{itemize}
\item {Grp. gram.:adj.}
\end{itemize}
Relativo á osteopathia.
\section{Osteopatia}
\begin{itemize}
\item {Grp. gram.:f.}
\end{itemize}
\begin{itemize}
\item {Proveniência:(Do gr. \textunderscore osteon\textunderscore  + \textunderscore pathos\textunderscore )}
\end{itemize}
Doença dos ossos, em geral.
\section{Osteopático}
\begin{itemize}
\item {Grp. gram.:adj.}
\end{itemize}
Relativo á osteopatia.
\section{Osteóphago}
\begin{itemize}
\item {Grp. gram.:adj.}
\end{itemize}
\begin{itemize}
\item {Proveniência:(Do gr. \textunderscore osteon\textunderscore  + \textunderscore phagein\textunderscore )}
\end{itemize}
Que come ossos.
\section{Osteophymia}
\begin{itemize}
\item {Grp. gram.:f.}
\end{itemize}
\begin{itemize}
\item {Utilização:Med.}
\end{itemize}
\begin{itemize}
\item {Proveniência:(Do gr. \textunderscore osteon\textunderscore  + \textunderscore phuma\textunderscore )}
\end{itemize}
Tuberculose nos ossos.
\section{Osteóphyto}
\begin{itemize}
\item {Grp. gram.:f.}
\end{itemize}
\begin{itemize}
\item {Utilização:Med.}
\end{itemize}
\begin{itemize}
\item {Proveniência:(Do gr. \textunderscore osteon\textunderscore  + \textunderscore phuton\textunderscore )}
\end{itemize}
Producção óssea, que resulta algumas vezes das lâminas profundas do periósteo, junto da parte cariada de um ôsso.
\section{Osteoplastia}
\begin{itemize}
\item {Grp. gram.:f.}
\end{itemize}
\begin{itemize}
\item {Proveniência:(Do gr. \textunderscore osteon\textunderscore  + \textunderscore plassein\textunderscore )}
\end{itemize}
Reparação de uma parte destruída dos ossos.
\section{Osteoplástico}
\begin{itemize}
\item {Grp. gram.:adj.}
\end{itemize}
Relativo á osteoplastia.
\section{Osteoplasto}
\begin{itemize}
\item {Grp. gram.:m.}
\end{itemize}
\begin{itemize}
\item {Utilização:Anat.}
\end{itemize}
\begin{itemize}
\item {Proveniência:(Do gr. \textunderscore osteon\textunderscore  + \textunderscore plasseín\textunderscore )}
\end{itemize}
Pequena cavidade nos ossos, da qual saem canalículos ramificados.
\section{Osteoporose}
\begin{itemize}
\item {Grp. gram.:f.}
\end{itemize}
\begin{itemize}
\item {Utilização:Med.}
\end{itemize}
\begin{itemize}
\item {Proveniência:(Do gr. \textunderscore osteon\textunderscore  + \textunderscore poros\textunderscore )}
\end{itemize}
Aumento anormal da porosidade dos ossos.
\section{Osteopsathyrose}
\begin{itemize}
\item {Grp. gram.:f.}
\end{itemize}
\begin{itemize}
\item {Utilização:Med.}
\end{itemize}
\begin{itemize}
\item {Proveniência:(Do gr. \textunderscore osteon\textunderscore  + \textunderscore psathuros\textunderscore , friável)}
\end{itemize}
Fríabilidade dos ossos.
\section{Osteosarcoma}
\begin{itemize}
\item {fónica:sar}
\end{itemize}
\begin{itemize}
\item {Grp. gram.:m.}
\end{itemize}
\begin{itemize}
\item {Proveniência:(De \textunderscore osteo...\textunderscore  + \textunderscore sarcoma\textunderscore )}
\end{itemize}
Tumor, que se desenvolve nos ossos.
\section{Osteosclerose}
\begin{itemize}
\item {Grp. gram.:f.}
\end{itemize}
\begin{itemize}
\item {Utilização:Med.}
\end{itemize}
\begin{itemize}
\item {Proveniência:(Do gr. \textunderscore osteon\textunderscore  + \textunderscore sklerosis\textunderscore )}
\end{itemize}
O mesmo que \textunderscore eburnação\textunderscore .
\section{Osteóse}
\begin{itemize}
\item {Grp. gram.:f.}
\end{itemize}
\begin{itemize}
\item {Proveniência:(Do gr. \textunderscore osteon\textunderscore )}
\end{itemize}
O mesmo que \textunderscore ossificação\textunderscore .
\section{Osteospermo}
\begin{itemize}
\item {Grp. gram.:m.}
\end{itemize}
\begin{itemize}
\item {Proveniência:(Do gr. \textunderscore osteon\textunderscore  + \textunderscore sperma\textunderscore )}
\end{itemize}
Gênero de plantas, da fam. das compostas.
\section{Osteossarcoma}
\begin{itemize}
\item {Grp. gram.:m.}
\end{itemize}
\begin{itemize}
\item {Proveniência:(De \textunderscore osteo...\textunderscore  + \textunderscore sarcoma\textunderscore )}
\end{itemize}
Tumor, que se desenvolve nos ossos.
\section{Osteosteátoma}
\begin{itemize}
\item {Grp. gram.:m.}
\end{itemize}
\begin{itemize}
\item {Utilização:Med.}
\end{itemize}
\begin{itemize}
\item {Proveniência:(Do gr. \textunderscore osteon\textunderscore  + \textunderscore steatoma\textunderscore )}
\end{itemize}
Degeneração do tecido ósseo, quando substituído por uma substância amarela, análoga ao tecido gorduroso.
\section{Osteóstomo}
\begin{itemize}
\item {Grp. gram.:adj.}
\end{itemize}
\begin{itemize}
\item {Grp. gram.:M. pl.}
\end{itemize}
\begin{itemize}
\item {Proveniência:(Do gr. \textunderscore osteon\textunderscore  + \textunderscore stoma\textunderscore )}
\end{itemize}
Cuja bôca ou maxilla é óssea.
Família de peixes osteótomos.
\section{Osteótilo}
\begin{itemize}
\item {Grp. gram.:m.}
\end{itemize}
\begin{itemize}
\item {Utilização:Med.}
\end{itemize}
\begin{itemize}
\item {Proveniência:(Do gr. \textunderscore osteon\textunderscore  + \textunderscore tulos\textunderscore )}
\end{itemize}
O mesmo que \textunderscore exostose\textunderscore .
\section{Osteotilose}
\begin{itemize}
\item {Grp. gram.:f.}
\end{itemize}
\begin{itemize}
\item {Utilização:Med.}
\end{itemize}
\begin{itemize}
\item {Proveniência:(De \textunderscore osteótilo\textunderscore )}
\end{itemize}
Formação de calo.
\section{Osteotomia}
\begin{itemize}
\item {Grp. gram.:f.}
\end{itemize}
\begin{itemize}
\item {Proveniência:(Do gr. \textunderscore osteon\textunderscore  + \textunderscore tome\textunderscore )}
\end{itemize}
Tratado da dissecção dos ossos.
\section{Osteotómico}
\begin{itemize}
\item {Grp. gram.:adj.}
\end{itemize}
Relativo á osteotomia.
\section{Osteotomista}
\begin{itemize}
\item {Grp. gram.:m.}
\end{itemize}
\begin{itemize}
\item {Utilização:Cir.}
\end{itemize}
Nome, que se deu a grandes pinças, cujas extremidades superiores são formadas por um anel oval e cortante, para partir os ossos do féto na madre.
(Cp. \textunderscore osteótomo\textunderscore )
\section{Osteótomo}
\begin{itemize}
\item {Grp. gram.:m.}
\end{itemize}
\begin{itemize}
\item {Utilização:Cir.}
\end{itemize}
\begin{itemize}
\item {Proveniência:(Do gr. \textunderscore osteon\textunderscore  + \textunderscore tome\textunderscore )}
\end{itemize}
Serra, em fórma de cadeia, para cortar ossos.
\section{Osteótylo}
\begin{itemize}
\item {Grp. gram.:m.}
\end{itemize}
\begin{itemize}
\item {Utilização:Med.}
\end{itemize}
\begin{itemize}
\item {Proveniência:(Do gr. \textunderscore osteon\textunderscore  + \textunderscore tulos\textunderscore )}
\end{itemize}
O mesmo que \textunderscore exostose\textunderscore .
\section{Osteotylose}
\begin{itemize}
\item {Grp. gram.:f.}
\end{itemize}
\begin{itemize}
\item {Utilização:Med.}
\end{itemize}
\begin{itemize}
\item {Proveniência:(De \textunderscore osteótylo\textunderscore )}
\end{itemize}
Formação de callo.
\section{Osteozoário}
\begin{itemize}
\item {Grp. gram.:m.  e  adj.}
\end{itemize}
\begin{itemize}
\item {Proveniência:(Do gr. \textunderscore osteon\textunderscore  + \textunderscore zoarion\textunderscore )}
\end{itemize}
Animal vertebrado.
\section{Osterício}
\begin{itemize}
\item {Grp. gram.:m.}
\end{itemize}
Gênero de plantas umbellíferas.
\section{Ostiaco}
\begin{itemize}
\item {Grp. gram.:m.}
\end{itemize}
Língua uralo-altaíca, vernácula na Rússia.
\section{Ostiaria}
\begin{itemize}
\item {Grp. gram.:f.}
\end{itemize}
\begin{itemize}
\item {Utilização:Ant.}
\end{itemize}
\begin{itemize}
\item {Proveniência:(De \textunderscore oste\textunderscore )}
\end{itemize}
O mesmo que \textunderscore hospedaria\textunderscore .
\section{Ostiaria}
\begin{itemize}
\item {Grp. gram.:f.}
\end{itemize}
\begin{itemize}
\item {Utilização:Ant.}
\end{itemize}
Caixa, para guarda ou depósito de hóstias.
(Por \textunderscore hostiaria\textunderscore , de \textunderscore hóstia\textunderscore )
\section{Ostiário}
\begin{itemize}
\item {Grp. gram.:m.}
\end{itemize}
\begin{itemize}
\item {Proveniência:(Lat. \textunderscore ostiarius\textunderscore )}
\end{itemize}
Designação antiga daquelle que abre e fecha as portas dos templos e guarda as alfaias do culto.
\section{Ostiolado}
\begin{itemize}
\item {Grp. gram.:adj.}
\end{itemize}
Que tem ostíolos.
\section{Ostíolo}
\begin{itemize}
\item {Grp. gram.:m.}
\end{itemize}
\begin{itemize}
\item {Proveniência:(Lat. \textunderscore ostiolum\textunderscore , de \textunderscore ostium\textunderscore )}
\end{itemize}
Pequena abertura ou orifício.
\section{Ostra}
\begin{itemize}
\item {fónica:ôs}
\end{itemize}
\begin{itemize}
\item {Grp. gram.:f.}
\end{itemize}
\begin{itemize}
\item {Proveniência:(Do lat. \textunderscore ostrea\textunderscore )}
\end{itemize}
Gênero de molluscos acéphalos, de que há uma espécie comestível.
\section{Ostráceas}
\begin{itemize}
\item {Grp. gram.:f. pl.}
\end{itemize}
\begin{itemize}
\item {Proveniência:(De \textunderscore ostráceo\textunderscore )}
\end{itemize}
Família de molluscos, que tem por typo a ostra.
\section{Ostraceiro}
\begin{itemize}
\item {Grp. gram.:m.}
\end{itemize}
Gênero de aves pernaltas, (\textunderscore haematopus ostralegus\textunderscore , Lin.).
\section{Ostráceo}
\begin{itemize}
\item {Grp. gram.:adj.}
\end{itemize}
\begin{itemize}
\item {Grp. gram.:M. pl.}
\end{itemize}
Relativo ou semelhante á ostra.
Família de molluscos, o mesmo que \textunderscore ostráceas\textunderscore .
\section{Ostracião-espinhoso}
\begin{itemize}
\item {Grp. gram.:m.}
\end{itemize}
O mesmo que \textunderscore guamajacu\textunderscore .
\section{Ostracino}
\begin{itemize}
\item {Grp. gram.:adj.}
\end{itemize}
\begin{itemize}
\item {Proveniência:(Do gr. \textunderscore ostrakon\textunderscore )}
\end{itemize}
Que está ou vive sôbre as conchas das ostras.
\section{Ostracismo}
\begin{itemize}
\item {Grp. gram.:m.}
\end{itemize}
\begin{itemize}
\item {Utilização:Fig.}
\end{itemize}
\begin{itemize}
\item {Proveniência:(Lat. \textunderscore ostracismus\textunderscore )}
\end{itemize}
Destêrro por déz annos, a que se condemnavam em Athenas os cidadãos, que por seu mérito ou poderio se tornavam suspeitos.
Exclusão, especialmente da governação pública.
\section{Ostracita}
\begin{itemize}
\item {Grp. gram.:f.}
\end{itemize}
\begin{itemize}
\item {Proveniência:(Lat. \textunderscore ostracites\textunderscore )}
\end{itemize}
Ostra fóssil.
\section{Ostracoderma}
\begin{itemize}
\item {Grp. gram.:f.}
\end{itemize}
\begin{itemize}
\item {Proveniência:(Lat. \textunderscore ostracodermus\textunderscore )}
\end{itemize}
Gênero de cogumelos gasteromycetos.
\section{Ostrácodos}
\begin{itemize}
\item {Grp. gram.:m. pl.}
\end{itemize}
\begin{itemize}
\item {Proveniência:(Do gr. \textunderscore ostrakon\textunderscore  + \textunderscore eidos\textunderscore )}
\end{itemize}
Gênero de molluscos microscópicos.
\section{Ostracologia}
\begin{itemize}
\item {Grp. gram.:f.}
\end{itemize}
\begin{itemize}
\item {Proveniência:(Do gr. \textunderscore ostrakon\textunderscore  + \textunderscore logos\textunderscore )}
\end{itemize}
Parte da História Natural, que expõe a história das conchas.
\section{Ostracológico}
\begin{itemize}
\item {Grp. gram.:adj.}
\end{itemize}
Relativo á ostracologia.
\section{Ostracomorfita}
\begin{itemize}
\item {Grp. gram.:f.}
\end{itemize}
\begin{itemize}
\item {Utilização:Miner.}
\end{itemize}
\begin{itemize}
\item {Proveniência:(Do gr. \textunderscore ostrakon\textunderscore  + \textunderscore morphe\textunderscore )}
\end{itemize}
Ostra ou qualquer concha bivalve em estado fóssil.
\section{Ostracomorphita}
\begin{itemize}
\item {Grp. gram.:f.}
\end{itemize}
\begin{itemize}
\item {Utilização:Miner.}
\end{itemize}
\begin{itemize}
\item {Proveniência:(Do gr. \textunderscore ostrakon\textunderscore  + \textunderscore morphe\textunderscore )}
\end{itemize}
Ostra ou qualquer concha bivalve em estado fóssil.
\section{Ostraria}
\begin{itemize}
\item {Grp. gram.:f.}
\end{itemize}
Grande porção de ostras.
\section{Ostreal}
\begin{itemize}
\item {Grp. gram.:m.}
\end{itemize}
\begin{itemize}
\item {Utilização:Ant.}
\end{itemize}
\begin{itemize}
\item {Proveniência:(Do lat. \textunderscore ostrea\textunderscore )}
\end{itemize}
O mesmo que \textunderscore ostreira\textunderscore .
\section{Ostreário}
\begin{itemize}
\item {Grp. gram.:adj.}
\end{itemize}
\begin{itemize}
\item {Utilização:Zool.}
\end{itemize}
\begin{itemize}
\item {Proveniência:(Lat. \textunderscore ostrearius\textunderscore )}
\end{itemize}
Que vive na concha das ostras.
\section{Ostreicultor}
\begin{itemize}
\item {Grp. gram.:m.}
\end{itemize}
Aquelle que pratíca a ostreicultura.
\section{Ostreicultura}
\begin{itemize}
\item {Grp. gram.:f.}
\end{itemize}
\begin{itemize}
\item {Proveniência:(Do lat. \textunderscore ostrea\textunderscore  + \textunderscore cultura\textunderscore )}
\end{itemize}
Processo para multiplicar e criar ostras.
\section{Ostreiforme}
\begin{itemize}
\item {Grp. gram.:adj.}
\end{itemize}
\begin{itemize}
\item {Proveniência:(Do lat. \textunderscore ostrea\textunderscore  + \textunderscore forma\textunderscore )}
\end{itemize}
Que tem fórma de ostra.
\section{Ostreína}
\begin{itemize}
\item {Grp. gram.:f.}
\end{itemize}
\begin{itemize}
\item {Proveniência:(Do lat. \textunderscore ostrea\textunderscore )}
\end{itemize}
Substância, própria da ostra.
\section{Ostreira}
\begin{itemize}
\item {Grp. gram.:f.}
\end{itemize}
Lugar, onde se criam ostras.
Vendedeira de ostras.
(Fem. de \textunderscore ostreiro\textunderscore )
\section{Ostreiro}
\begin{itemize}
\item {Grp. gram.:adj.}
\end{itemize}
\begin{itemize}
\item {Grp. gram.:M.}
\end{itemize}
Próprio para a pesca das ostras.
Vendedor do ostras.
\section{Ostreíta}
\begin{itemize}
\item {Grp. gram.:f.}
\end{itemize}
O mesmo que \textunderscore ostracita\textunderscore .
\section{Óstria}
\begin{itemize}
\item {Grp. gram.:f.}
\end{itemize}
\begin{itemize}
\item {Proveniência:(Lat. \textunderscore ostrya\textunderscore )}
\end{itemize}
Gênero de plantas cupulíferas.
\section{Ostrífero}
\begin{itemize}
\item {Grp. gram.:adj.}
\end{itemize}
\begin{itemize}
\item {Proveniência:(Do lat. \textunderscore ostrea\textunderscore  + \textunderscore ferre\textunderscore )}
\end{itemize}
Que produz ostras.
\section{Ostrino}
\begin{itemize}
\item {Grp. gram.:adj.}
\end{itemize}
\begin{itemize}
\item {Proveniência:(Lat. \textunderscore ostrinus\textunderscore )}
\end{itemize}
Da côr ou natureza da púrpura.
\section{Ostro}
\begin{itemize}
\item {Grp. gram.:m.}
\end{itemize}
\begin{itemize}
\item {Proveniência:(Lat. \textunderscore ostrum\textunderscore )}
\end{itemize}
O mesmo que \textunderscore púrpura\textunderscore .
\section{Ostrogodo}
\begin{itemize}
\item {Grp. gram.:m.}
\end{itemize}
\begin{itemize}
\item {Grp. gram.:Adj.}
\end{itemize}
\begin{itemize}
\item {Proveniência:(Do germ. \textunderscore ost\textunderscore  + \textunderscore got\textunderscore )}
\end{itemize}
Habitante da Gótia oriental.
Relativo aos Ostrogodos.
\section{Óstrya}
\begin{itemize}
\item {Grp. gram.:f.}
\end{itemize}
\begin{itemize}
\item {Proveniência:(Lat. \textunderscore ostrya\textunderscore )}
\end{itemize}
Gênero de plantas cupulíferas.
\section{Osyrícera}
\begin{itemize}
\item {Grp. gram.:f.}
\end{itemize}
\begin{itemize}
\item {Proveniência:(Do gr. \textunderscore osuris\textunderscore  + \textunderscore keras\textunderscore )}
\end{itemize}
Gênero de orchídeas.
\section{Osýride}
\begin{itemize}
\item {Grp. gram.:f.}
\end{itemize}
\begin{itemize}
\item {Proveniência:(Do gr. \textunderscore osuris\textunderscore )}
\end{itemize}
Gênero de plantas, o mesmo que \textunderscore valverde\textunderscore .
\section{Osyrídeas}
\begin{itemize}
\item {Grp. gram.:f. pl.}
\end{itemize}
\begin{itemize}
\item {Proveniência:(De \textunderscore osýride\textunderscore )}
\end{itemize}
Família de plantas, criada por Brown, mas que, segundo Jussieu, deve sêr comprehendida nas santaláceas.
\section{Ota!}
\begin{itemize}
\item {Grp. gram.:interj.}
\end{itemize}
\begin{itemize}
\item {Utilização:Bras. do S}
\end{itemize}
(designativa de \textunderscore admiração\textunderscore )
\section{Otacústica}
\begin{itemize}
\item {Grp. gram.:f.}
\end{itemize}
\begin{itemize}
\item {Utilização:Med.}
\end{itemize}
\begin{itemize}
\item {Proveniência:(De \textunderscore otacústico\textunderscore )}
\end{itemize}
Sciência, que se occupa do sentido da audição.
\section{Otacústico}
\begin{itemize}
\item {Grp. gram.:adj.}
\end{itemize}
\begin{itemize}
\item {Utilização:Phýs.}
\end{itemize}
\begin{itemize}
\item {Proveniência:(Do gr. \textunderscore ous\textunderscore , \textunderscore otos\textunderscore  + \textunderscore akouein\textunderscore )}
\end{itemize}
Próprio para aperfeiçoar o sentido do ouvido.
\section{Otági}
\begin{itemize}
\item {Grp. gram.:m.}
\end{itemize}
Planta trepadeira da ilha de San-Thomé.
\section{Otalgia}
\begin{itemize}
\item {Grp. gram.:f.}
\end{itemize}
\begin{itemize}
\item {Proveniência:(Do gr. \textunderscore ous\textunderscore , \textunderscore otus\textunderscore  + \textunderscore algos\textunderscore )}
\end{itemize}
Dôr nervosa no ouvido.
\section{Otálgico}
\begin{itemize}
\item {Grp. gram.:adj.}
\end{itemize}
Relativo á otalgia.
Applicável contra a otalgia.
\section{Otantera}
\begin{itemize}
\item {Grp. gram.:f.}
\end{itemize}
\begin{itemize}
\item {Proveniência:(Do gr. \textunderscore ous\textunderscore , \textunderscore otos\textunderscore  + \textunderscore antheros\textunderscore )}
\end{itemize}
Gênero de plantas melastomáceas.
\section{Otanthera}
\begin{itemize}
\item {Grp. gram.:f.}
\end{itemize}
\begin{itemize}
\item {Proveniência:(Do gr. \textunderscore ous\textunderscore , \textunderscore otos\textunderscore  + \textunderscore antheros\textunderscore )}
\end{itemize}
Gênero de plantas melastomáceas.
\section{Otão}
\begin{itemize}
\item {Grp. gram.:m.}
\end{itemize}
\begin{itemize}
\item {Proveniência:(De \textunderscore Othão\textunderscore , n. p.)}
\end{itemize}
Moéda grega, do valor de 3.222 reis proximamente.
\section{Otária}
\begin{itemize}
\item {Grp. gram.:f.}
\end{itemize}
\begin{itemize}
\item {Proveniência:(Do gr. \textunderscore ous\textunderscore , \textunderscore otos\textunderscore )}
\end{itemize}
Gênero de plantas asclepiadáceas.
Espécie de phoca, de orelhas bem visíveis.
\section{Otário}
\begin{itemize}
\item {Grp. gram.:m.}
\end{itemize}
Gênero de crustáceos.
(Cp. \textunderscore otária\textunderscore )
\section{Oteátea}
\begin{itemize}
\item {Grp. gram.:f.}
\end{itemize}
Pássaro dentirostro da África occidental.
\section{Othão}
\begin{itemize}
\item {Grp. gram.:m.}
\end{itemize}
\begin{itemize}
\item {Proveniência:(De \textunderscore Othão\textunderscore , n. p.)}
\end{itemize}
Moéda grega, do valor de 3.222 reis proximamente.
\section{Othona}
\begin{itemize}
\item {Grp. gram.:f.}
\end{itemize}
\begin{itemize}
\item {Proveniência:(De \textunderscore Othon\textunderscore , n. p.)}
\end{itemize}
Gênero de plantas, da fam. das compostas.
\section{Otiatria}
\begin{itemize}
\item {Grp. gram.:f.}
\end{itemize}
O mesmo ou melhor que \textunderscore otoiatría\textunderscore . Cf. R. Galvão, \textunderscore Vocab.\textunderscore 
\section{Ótico}
\begin{itemize}
\item {Grp. gram.:adj.}
\end{itemize}
\begin{itemize}
\item {Proveniência:(Lat. \textunderscore oticus\textunderscore )}
\end{itemize}
Diz-se do medicamento, que se emprega contra doenças de ouvidos.
\section{Otim}
\begin{itemize}
\item {Grp. gram.:m.}
\end{itemize}
Aquelle que no Malabar se emprega em cavar a terra, abrir poços, construir muros, etc.
\section{Otita}
\begin{itemize}
\item {Grp. gram.:f.}
\end{itemize}
\begin{itemize}
\item {Utilização:Miner.}
\end{itemize}
\begin{itemize}
\item {Proveniência:(Do gr. \textunderscore ous\textunderscore , \textunderscore otos\textunderscore )}
\end{itemize}
Ferro argilloso, reniforme.
\section{Otite}
\begin{itemize}
\item {Grp. gram.:f.}
\end{itemize}
\begin{itemize}
\item {Proveniência:(Do gr. \textunderscore ous\textunderscore , \textunderscore otos\textunderscore )}
\end{itemize}
Inflammação da membrana mucosa do ouvido.
\section{Oto...}
\begin{itemize}
\item {Grp. gram.:pref.}
\end{itemize}
\begin{itemize}
\item {Proveniência:(Do gr. \textunderscore ous\textunderscore , \textunderscore otos\textunderscore )}
\end{itemize}
(designativo de \textunderscore orelha\textunderscore )
\section{Otocefalia}
\begin{itemize}
\item {Grp. gram.:f.}
\end{itemize}
Qualidade ou estado de otocéfalo.
\section{Otocéfalo}
\begin{itemize}
\item {Grp. gram.:m.  e  adj.}
\end{itemize}
\begin{itemize}
\item {Proveniência:(Do gr. \textunderscore ous\textunderscore , \textunderscore otos\textunderscore  + \textunderscore kephale\textunderscore )}
\end{itemize}
Monstro, que tem as orelhas confundidas ou aproximadas por baixo da cabeça e atrofiado o aparelho nasal.
\section{Otocephalia}
\begin{itemize}
\item {Grp. gram.:f.}
\end{itemize}
Qualidade ou estado de otocéphalo.
\section{Otocéphalo}
\begin{itemize}
\item {Grp. gram.:m.  e  adj.}
\end{itemize}
\begin{itemize}
\item {Proveniência:(Do gr. \textunderscore ous\textunderscore , \textunderscore otos\textunderscore  + \textunderscore kephale\textunderscore )}
\end{itemize}
Monstro, que tem as orelhas confundidas ou aproximadas por baixo da cabeça e atrophiado o apparelho nasal.
\section{Otografia}
\begin{itemize}
\item {Grp. gram.:f.}
\end{itemize}
\begin{itemize}
\item {Proveniência:(Do gr. \textunderscore ous\textunderscore , \textunderscore otos\textunderscore  + \textunderscore graphein\textunderscore )}
\end{itemize}
Descripção científica do ouvido.
\section{Otográfico}
\begin{itemize}
\item {Grp. gram.:adj.}
\end{itemize}
Relativo á otografia.
\section{Otographia}
\begin{itemize}
\item {Grp. gram.:f.}
\end{itemize}
\begin{itemize}
\item {Proveniência:(Do gr. \textunderscore ous\textunderscore , \textunderscore otos\textunderscore  + \textunderscore graphein\textunderscore )}
\end{itemize}
Descripção scientífica do ouvido.
\section{Otográphico}
\begin{itemize}
\item {Grp. gram.:adj.}
\end{itemize}
Relativo á otographia.
\section{Otoiatra}
\begin{itemize}
\item {fónica:to-i}
\end{itemize}
\begin{itemize}
\item {Grp. gram.:m.}
\end{itemize}
Médico, que exerce a otoiatria.
\section{Otoiatria}
\begin{itemize}
\item {fónica:to-i}
\end{itemize}
\begin{itemize}
\item {Grp. gram.:f.}
\end{itemize}
\begin{itemize}
\item {Proveniência:(Do gr. \textunderscore ous\textunderscore , \textunderscore otos\textunderscore  + \textunderscore iatros\textunderscore )}
\end{itemize}
Tratamento das doenças de ouvidos.
\section{Otólitho}
\begin{itemize}
\item {Grp. gram.:m.}
\end{itemize}
\begin{itemize}
\item {Proveniência:(Do gr. \textunderscore ous\textunderscore , \textunderscore otos\textunderscore  + \textunderscore lithos\textunderscore )}
\end{itemize}
Concreção pedregosa, que se encontra no ouvido de alguns peixes.
\section{Otólito}
\begin{itemize}
\item {Grp. gram.:m.}
\end{itemize}
\begin{itemize}
\item {Proveniência:(Do gr. \textunderscore ous\textunderscore , \textunderscore otos\textunderscore  + \textunderscore lithos\textunderscore )}
\end{itemize}
Concreção pedregosa, que se encontra no ouvido de alguns peixes.
\section{Otologia}
\begin{itemize}
\item {Grp. gram.:f.}
\end{itemize}
\begin{itemize}
\item {Proveniência:(Do gr. \textunderscore ous\textunderscore , \textunderscore otos\textunderscore  + \textunderscore logos\textunderscore )}
\end{itemize}
Tratado á cêrca do ouvido.
\section{Otológico}
\begin{itemize}
\item {Grp. gram.:adj.}
\end{itemize}
Relativo á otologia.
\section{Otomana}
\begin{itemize}
\item {Grp. gram.:f.}
\end{itemize}
\begin{itemize}
\item {Proveniência:(De \textunderscore otomano\textunderscore )}
\end{itemize}
Espécie de sofá, mais largo que os vulgares.
Espécie de tecido para vestidos de senhora.
\section{Ótoa}
\begin{itemize}
\item {Grp. gram.:f.}
\end{itemize}
\begin{itemize}
\item {Proveniência:(De \textunderscore Ott\textunderscore , n. p.)}
\end{itemize}
Gênero de plantas umbelíferas.
\section{Otóforo}
\begin{itemize}
\item {Grp. gram.:m.}
\end{itemize}
\begin{itemize}
\item {Proveniência:(Do gr. \textunderscore ous\textunderscore , \textunderscore otos\textunderscore  + \textunderscore phoros\textunderscore )}
\end{itemize}
Gênero de insectos coleópteros pentâmeros.
\section{Otomano}
\begin{itemize}
\item {Grp. gram.:adj.}
\end{itemize}
\begin{itemize}
\item {Grp. gram.:M.}
\end{itemize}
Relativo á Turquia ou aos seus sultões: \textunderscore império otomano\textunderscore .
Habitante da Turquia, turco.
(Do turco \textunderscore otsmanyy\textunderscore )
\section{Otomão}
\begin{itemize}
\item {Grp. gram.:m.  e  adj.}
\end{itemize}
O mesmo que \textunderscore otomano\textunderscore .
\section{Otombeira}
\begin{itemize}
\item {Grp. gram.:f.}
\end{itemize}
Árvore da Índia portuguesa.
\section{Otona}
\begin{itemize}
\item {Grp. gram.:f.}
\end{itemize}
\begin{itemize}
\item {Proveniência:(De \textunderscore Othon\textunderscore , n. p.)}
\end{itemize}
Gênero de plantas, da fam. das compostas.
\section{Otóphoro}
\begin{itemize}
\item {Grp. gram.:m.}
\end{itemize}
\begin{itemize}
\item {Proveniência:(Do gr. \textunderscore ous\textunderscore , \textunderscore otos\textunderscore  + \textunderscore phoros\textunderscore )}
\end{itemize}
Gênero de insectos coleópteros pentâmeros.
\section{Otoplastia}
\begin{itemize}
\item {Grp. gram.:f.}
\end{itemize}
\begin{itemize}
\item {Proveniência:(Do gr. \textunderscore ous\textunderscore , \textunderscore otos\textunderscore  + \textunderscore plassein\textunderscore )}
\end{itemize}
Restauração cirúrgica da parte externa destruída do ouvido.
\section{Otoplástico}
\begin{itemize}
\item {Grp. gram.:adj.}
\end{itemize}
Relativo á otoplastia.
\section{Otorcular}
\begin{itemize}
\item {Grp. gram.:adj.}
\end{itemize}
\begin{itemize}
\item {Utilização:Anat.}
\end{itemize}
\begin{itemize}
\item {Proveniência:(Do lat. \textunderscore ob\textunderscore  + \textunderscore torcular\textunderscore )}
\end{itemize}
Diz-se de alguns seios cerebraes, que desembocam noutros seios, em vez de se lançarem no torcular.
\section{Otorreia}
\begin{itemize}
\item {Grp. gram.:f.}
\end{itemize}
\begin{itemize}
\item {Proveniência:(Do gr. \textunderscore ous\textunderscore , \textunderscore otos\textunderscore  + \textunderscore rhagein\textunderscore )}
\end{itemize}
Fluxo seroso do ouvido.
\section{Otorrheia}
\begin{itemize}
\item {Grp. gram.:f.}
\end{itemize}
\begin{itemize}
\item {Proveniência:(Do gr. \textunderscore ous\textunderscore , \textunderscore otos\textunderscore  + \textunderscore rhagein\textunderscore )}
\end{itemize}
Fluxo seroso do ouvido.
\section{Otoscópio}
\begin{itemize}
\item {Grp. gram.:m.}
\end{itemize}
\begin{itemize}
\item {Proveniência:(Do gr. \textunderscore ous\textunderscore , \textunderscore otos\textunderscore  + \textunderscore skopein\textunderscore )}
\end{itemize}
Instrumento, para examinar o canal auditivo.
\section{Otostégia}
\begin{itemize}
\item {Grp. gram.:f.}
\end{itemize}
Gênero de plantas labiadas.
\section{Ototerapia}
\begin{itemize}
\item {Grp. gram.:f.}
\end{itemize}
\begin{itemize}
\item {Utilização:Neol.}
\end{itemize}
\begin{itemize}
\item {Proveniência:(Do gr. \textunderscore ous\textunderscore , \textunderscore otos\textunderscore  + \textunderscore therapeia\textunderscore )}
\end{itemize}
Terapêutica dos ouvidos.
Tratamento das doenças dos ouvidos.
\section{Ototerápico}
\begin{itemize}
\item {Grp. gram.:adj.}
\end{itemize}
Relativo á ototerapia.
\section{Ototherapia}
\begin{itemize}
\item {Grp. gram.:f.}
\end{itemize}
\begin{itemize}
\item {Utilização:Neol.}
\end{itemize}
\begin{itemize}
\item {Proveniência:(Do gr. \textunderscore ous\textunderscore , \textunderscore otos\textunderscore  + \textunderscore therapeia\textunderscore )}
\end{itemize}
Therapêutica dos ouvidos.
Tratamento das doenças dos ouvidos.
\section{Ototherápico}
\begin{itemize}
\item {Grp. gram.:adj.}
\end{itemize}
Relativo á ototherapia.
\section{Ototó}
\begin{itemize}
\item {Grp. gram.:m.}
\end{itemize}
Árvore de San-Thomé.
\section{Ototomia}
\begin{itemize}
\item {Grp. gram.:f.}
\end{itemize}
\begin{itemize}
\item {Proveniência:(Do gr. \textunderscore ous\textunderscore , \textunderscore otos\textunderscore  + \textunderscore tome\textunderscore )}
\end{itemize}
Dissecção cirúrgica do ouvido.
\section{Ototómico}
\begin{itemize}
\item {Grp. gram.:adj.}
\end{itemize}
Relativo á ototomia.
\section{Óttoa}
\begin{itemize}
\item {Grp. gram.:f.}
\end{itemize}
\begin{itemize}
\item {Proveniência:(De \textunderscore Ott\textunderscore , n. p.)}
\end{itemize}
Gênero de plantas umbellíferas.
\section{Ou}
\begin{itemize}
\item {Grp. gram.:conj.}
\end{itemize}
\begin{itemize}
\item {Proveniência:(Do lat. \textunderscore aut\textunderscore )}
\end{itemize}
(designativa de \textunderscore alternativa\textunderscore )
\section{Ou}
\begin{itemize}
\item {Grp. gram.:adv.}
\end{itemize}
\begin{itemize}
\item {Utilização:Ant.}
\end{itemize}
O mesmo que \textunderscore onde\textunderscore .
\section{Ou}
Fórma antiga, equivalente a \textunderscore ao\textunderscore , (prep. e art.).
\section{Ouça}
\begin{itemize}
\item {Grp. gram.:f.}
\end{itemize}
\begin{itemize}
\item {Utilização:Fam.}
\end{itemize}
\begin{itemize}
\item {Proveniência:(De \textunderscore oiço\textunderscore , pres. do indic. de \textunderscore ouvir\textunderscore )}
\end{itemize}
Ouvido; o sentido da audição:«\textunderscore esgaravata as orelhas, para aguçar as ouças\textunderscore ». Filinto, IX, 94.
\section{Ouçandé}
\begin{itemize}
\item {Grp. gram.:f.}
\end{itemize}
Planta hortense da Índia portuguesa.
\section{Oução}
\begin{itemize}
\item {Grp. gram.:m.}
\end{itemize}
\begin{itemize}
\item {Utilização:Ant.}
\end{itemize}
Pequenino ácaro, (\textunderscore acarus siro\textunderscore ), que se encontra nos queijos, na farinha, etc.
Lêndeas.
\section{Ouche!}
\begin{itemize}
\item {Grp. gram.:interj.}
\end{itemize}
\begin{itemize}
\item {Utilização:Prov.}
\end{itemize}
\begin{itemize}
\item {Utilização:minh.}
\end{itemize}
O mesmo que \textunderscore eixe!\textunderscore 
\section{Oúco}
\begin{itemize}
\item {Grp. gram.:m.}
\end{itemize}
Árvore leguminosa da África central, de flôres muito odoríferas.
\section{Ougadoiro}
\begin{itemize}
\item {Grp. gram.:m.}
\end{itemize}
\begin{itemize}
\item {Utilização:Prov.}
\end{itemize}
\begin{itemize}
\item {Utilização:minh.}
\end{itemize}
\begin{itemize}
\item {Proveniência:(De \textunderscore ougar\textunderscore , por \textunderscore aguar\textunderscore , de \textunderscore água\textunderscore )}
\end{itemize}
Acto de enriar o linho.
\section{Ougar}
\begin{itemize}
\item {Grp. gram.:v. i.}
\end{itemize}
\begin{itemize}
\item {Utilização:Prov.}
\end{itemize}
O mesmo que \textunderscore aguar\textunderscore .
\section{Ouquia}
\begin{itemize}
\item {Grp. gram.:f.}
\end{itemize}
Antiga moéda de oiro, asiática. Cf. \textunderscore Ethiópia Or.\textunderscore , I, 343.
(Cp. \textunderscore oqueá\textunderscore )
\section{Oura}
\begin{itemize}
\item {Grp. gram.:f.}
\end{itemize}
\begin{itemize}
\item {Proveniência:(Do lat. \textunderscore aura\textunderscore )}
\end{itemize}
Perturbação da cabeça, produzida por fraqueza ou debilidade.
\section{Ourada}
\begin{itemize}
\item {Grp. gram.:adj. f.}
\end{itemize}
\begin{itemize}
\item {Utilização:Prov.}
\end{itemize}
\begin{itemize}
\item {Utilização:beir.}
\end{itemize}
Diz-se da rapariga enfeitada com objectos de ouro.
\section{Ourado}
\begin{itemize}
\item {Grp. gram.:adj.}
\end{itemize}
\begin{itemize}
\item {Proveniência:(De \textunderscore ourar\textunderscore )}
\end{itemize}
Que tem oura.
\section{Ourama}
\begin{itemize}
\item {Grp. gram.:f.}
\end{itemize}
\begin{itemize}
\item {Utilização:Bras. de Minas}
\end{itemize}
\begin{itemize}
\item {Utilização:Ext.}
\end{itemize}
Dinheiro em ouro.
Dinheiro.
\section{Ourar}
\begin{itemize}
\item {Grp. gram.:v. i.}
\end{itemize}
\begin{itemize}
\item {Proveniência:(De \textunderscore oura\textunderscore )}
\end{itemize}
Têr tonturas de cabeça.
\section{Ourar}
\begin{itemize}
\item {Grp. gram.:v. t.}
\end{itemize}
\begin{itemize}
\item {Proveniência:(De \textunderscore ouro\textunderscore )}
\end{itemize}
Dotar ou prendar com ouro (a noiva).
\section{Oureça}
\begin{itemize}
\item {Grp. gram.:f.}
\end{itemize}
\begin{itemize}
\item {Utilização:Prov.}
\end{itemize}
\begin{itemize}
\item {Utilização:trasm.}
\end{itemize}
O mesmo que \textunderscore aragem\textunderscore .
(por \textunderscore aureça\textunderscore , de \textunderscore aura\textunderscore )
\section{Ourega}
\begin{itemize}
\item {Grp. gram.:f.}
\end{itemize}
Peixe marítimo, variedade de raia, (\textunderscore raja lintea\textunderscore ).
\section{Ourégão}
\begin{itemize}
\item {Grp. gram.:m.}
\end{itemize}
Outra fórma de \textunderscore orégão\textunderscore . Cf. P. Coutinho, \textunderscore Flora\textunderscore , 515.
\section{Ourego}
\begin{itemize}
\item {Grp. gram.:m.}
\end{itemize}
\begin{itemize}
\item {Utilização:Prov.}
\end{itemize}
\begin{itemize}
\item {Utilização:beir.}
\end{itemize}
O mesmo que \textunderscore ourégão\textunderscore .
\section{Ourejante}
\begin{itemize}
\item {Grp. gram.:adj.}
\end{itemize}
Que oureja.
\section{Ourejar}
\begin{itemize}
\item {Grp. gram.:v. t.}
\end{itemize}
\begin{itemize}
\item {Utilização:Neol.}
\end{itemize}
Brilhar como ouro; brilhar (qualquer objecto de ouro ou dourado).
\section{Ourejar}
\begin{itemize}
\item {Grp. gram.:v. t.}
\end{itemize}
O mesmo que \textunderscore ourar\textunderscore ^2. Cf. D. Bernárdez, \textunderscore Lima\textunderscore , 101.
\section{Ourela}
\begin{itemize}
\item {Grp. gram.:f.}
\end{itemize}
Orla; margem; cercadura.
(Por \textunderscore orela\textunderscore , do lat. \textunderscore ora\textunderscore )
\section{Ourelo}
\begin{itemize}
\item {fónica:ourê}
\end{itemize}
\begin{itemize}
\item {Grp. gram.:m.}
\end{itemize}
Fita de pano grosso.
Tira.
(Cp. \textunderscore ourela\textunderscore )
\section{Ourém}
\begin{itemize}
\item {Grp. gram.:m.}
\end{itemize}
\begin{itemize}
\item {Utilização:Ant.}
\end{itemize}
\begin{itemize}
\item {Proveniência:(De \textunderscore ouro\textunderscore  + \textunderscore em\textunderscore  + \textunderscore cu\textunderscore )}
\end{itemize}
\begin{itemize}
\item {Proveniência:Casta de uva.
}
\end{itemize}
\textunderscore m.\textunderscore * Ourincu,
O mesmo que \textunderscore pirilampo\textunderscore .
\section{Ouriçar}
\begin{itemize}
\item {Grp. gram.:v. t.}
\end{itemize}
Tornar semelhante aos pêlos do ouriço.
Eriçar; tornar áspero: \textunderscore o gato ouriça o pêlo\textunderscore .
\section{Ouriceira}
\begin{itemize}
\item {Grp. gram.:f.}
\end{itemize}
Depósito de ouriços com castanhas, para que estas se conservem frescas e sans.
\section{Ouriceiro}
\begin{itemize}
\item {Grp. gram.:m.}
\end{itemize}
O mesmo que \textunderscore ouriceira\textunderscore .
\section{Ourichuvo}
\begin{itemize}
\item {Grp. gram.:adj.}
\end{itemize}
\begin{itemize}
\item {Utilização:Poét.}
\end{itemize}
\begin{itemize}
\item {Proveniência:(De \textunderscore ouro\textunderscore  + \textunderscore chuva\textunderscore )}
\end{itemize}
Que espalha em chuva de ouro.
\section{Ouriço}
\begin{itemize}
\item {Grp. gram.:m.}
\end{itemize}
\begin{itemize}
\item {Proveniência:(Do lat. \textunderscore ericius\textunderscore )}
\end{itemize}
Invólucro espinhoso de alguns frutos.
\textunderscore Ouriço cacheiro\textunderscore , animal revestido de espinhos, que serve de tipo aos erinacídeos.
\textunderscore Ouriço do mar\textunderscore , animal equinoderme.
\section{Ourina}
\textunderscore f.\textunderscore  (e der.)
O mesmo que \textunderscore urina\textunderscore , etc.:«\textunderscore ...o mal era a retenção de ourinas.\textunderscore »Sousa, \textunderscore Vida do Arceb.\textunderscore , 184.
\section{Ourincu}
\begin{itemize}
\item {Grp. gram.:m.}
\end{itemize}
\begin{itemize}
\item {Utilização:Ant.}
\end{itemize}
\begin{itemize}
\item {Proveniência:(De \textunderscore ouro\textunderscore  + \textunderscore em\textunderscore  + \textunderscore cu\textunderscore )}
\end{itemize}
O mesmo que \textunderscore pirilampo\textunderscore .
\section{Ourinque}
\begin{itemize}
\item {Grp. gram.:m.}
\end{itemize}
O mesmo que \textunderscore arinque\textunderscore .
\section{Ouripel}
\begin{itemize}
\item {Grp. gram.:m.}
\end{itemize}
O mesmo que \textunderscore ouropel\textunderscore :«\textunderscore ...se os despes dêsse ouripel...\textunderscore »Filinto, XIX, 274.
\section{Ourives}
\begin{itemize}
\item {Grp. gram.:m.}
\end{itemize}
\begin{itemize}
\item {Proveniência:(Do lat. \textunderscore aurifex\textunderscore )}
\end{itemize}
Fabricante ou vendedor de objectos de oiro.
\section{Ourívez}
\begin{itemize}
\item {Grp. gram.:m.}
\end{itemize}
\begin{itemize}
\item {Proveniência:(Do lat. \textunderscore aurifex\textunderscore )}
\end{itemize}
Fabricante ou vendedor de objectos de oiro.
\section{Ourivezaria}
\begin{itemize}
\item {Grp. gram.:f.}
\end{itemize}
\begin{itemize}
\item {Proveniência:(De \textunderscore ourívez\textunderscore  = \textunderscore ourives\textunderscore )}
\end{itemize}
Loja ou estabelecimento de ourives.
Arte de ourives.
\section{Ourivezeiro}
\begin{itemize}
\item {Grp. gram.:m.}
\end{itemize}
\begin{itemize}
\item {Utilização:Ant.}
\end{itemize}
O mesmo que \textunderscore ourives\textunderscore .
\section{Ouro}
\begin{itemize}
\item {Grp. gram.:m.}
\end{itemize}
\begin{itemize}
\item {Utilização:Fig.}
\end{itemize}
\begin{itemize}
\item {Grp. gram.:Pl.}
\end{itemize}
\begin{itemize}
\item {Proveniência:(Do lat. \textunderscore aurum\textunderscore )}
\end{itemize}
Metal de brilho amarelo, que se cunham as moédas de maior valor e fabrícam certas jóias.
Dinheiro.
Jóias: \textunderscore a varina traz muito ouro\textunderscore .
Côr amarela e brilhante: \textunderscore o ouro dos seus cabellos\textunderscore .
Preciosidade.
Qualidade ou objecto de grande valor.
Grande valor, grande merecimento: \textunderscore o silêncio é de ouro\textunderscore .
Um dos quatro naipes das cartas de jogar.
\section{Ourobalão}
\begin{itemize}
\item {Grp. gram.:m.}
\end{itemize}
\begin{itemize}
\item {Utilização:Ant.}
\end{itemize}
O mesmo que \textunderscore orobalão\textunderscore . Cf. \textunderscore Peregrinação\textunderscore , XIII.
\section{Ourolo}
\begin{itemize}
\item {Grp. gram.:m.}
\end{itemize}
\begin{itemize}
\item {Utilização:Ant.}
\end{itemize}
Aro, circuito demarcado, dentro do qual os moradores eram obrigados a certo tributo, ou delle isentos.
(Refl. do lat. \textunderscore ora\textunderscore )
\section{Ouropel}
\begin{itemize}
\item {Grp. gram.:m.}
\end{itemize}
\begin{itemize}
\item {Utilização:Fig.}
\end{itemize}
\begin{itemize}
\item {Proveniência:(Do b. lat. \textunderscore auripellum\textunderscore )}
\end{itemize}
Lâmina fina de latão, imitando oiro.
Oiro falso.
Falso brilho.
Estilo pomposo, encobrindo ausência de ideias ou ideias falsas.
\section{Ourreta}
\begin{itemize}
\item {fónica:rê}
\end{itemize}
\begin{itemize}
\item {Grp. gram.:f.}
\end{itemize}
(V.orreta)
\section{Ourudo}
\begin{itemize}
\item {Grp. gram.:adj.}
\end{itemize}
\begin{itemize}
\item {Utilização:Bras. de Minas}
\end{itemize}
\begin{itemize}
\item {Proveniência:(De \textunderscore ouro\textunderscore )}
\end{itemize}
O mesmo que \textunderscore rico\textunderscore .
\section{Ousadamente}
\begin{itemize}
\item {Grp. gram.:adv.}
\end{itemize}
De modo ousado.
\section{Ousadia}
\begin{itemize}
\item {Grp. gram.:f.}
\end{itemize}
Qualidade daquelle ou daquillo que é ousado.
Atrevimento.
Coragem; galhardia.
\section{Ousado}
\begin{itemize}
\item {Grp. gram.:adj.}
\end{itemize}
\begin{itemize}
\item {Proveniência:(De \textunderscore ousar\textunderscore )}
\end{itemize}
Audaz; corajoso; bizarro.
\section{Ousám}
\begin{itemize}
\item {Grp. gram.:m.}
\end{itemize}
\begin{itemize}
\item {Utilização:Ant.}
\end{itemize}
O mesmo que \textunderscore ousadia\textunderscore .
\section{Ousamento}
\begin{itemize}
\item {Grp. gram.:m.}
\end{itemize}
O mesmo que \textunderscore ousadia\textunderscore .
\section{Ousança}
\begin{itemize}
\item {Grp. gram.:f.}
\end{itemize}
\begin{itemize}
\item {Utilização:Ant.}
\end{itemize}
O mesmo que \textunderscore ousadia\textunderscore .
\section{Ousar}
\begin{itemize}
\item {Grp. gram.:v. t.}
\end{itemize}
\begin{itemize}
\item {Proveniência:(Do lat. \textunderscore ausus\textunderscore )}
\end{itemize}
Atrever-se a; tentar atrevidamente.
Têr coragem para; emprehender.
\section{Ousia}
\begin{itemize}
\item {Grp. gram.:f.}
\end{itemize}
O mesmo que \textunderscore osia\textunderscore .
\section{Ousio}
\begin{itemize}
\item {Grp. gram.:m.}
\end{itemize}
\begin{itemize}
\item {Proveniência:(Do lat. \textunderscore ausus\textunderscore )}
\end{itemize}
O mesmo que \textunderscore ousadia\textunderscore :«\textunderscore ...e tivera o ousio de dizer á mãe que não queria o Rato...\textunderscore »Camillo, \textunderscore Volcoens\textunderscore , 20.
\section{Outão}
\begin{itemize}
\item {Grp. gram.:m.}
\end{itemize}
\begin{itemize}
\item {Utilização:Prov.}
\end{itemize}
\begin{itemize}
\item {Utilização:minh.}
\end{itemize}
\begin{itemize}
\item {Proveniência:(Do lat. hyp. \textunderscore altanus\textunderscore ?)}
\end{itemize}
Parte lateral de um edifício.
Pedaço de muro alto.
\section{Outar}
\begin{itemize}
\item {Grp. gram.:v. t.}
\end{itemize}
\begin{itemize}
\item {Proveniência:(Do lat. \textunderscore optare\textunderscore ?)}
\end{itemize}
Joeirar.
\section{Outiva}
\begin{itemize}
\item {Grp. gram.:f.}
\end{itemize}
Audição; ouvido.
(Contr. de \textunderscore auditiva\textunderscore , fem. de \textunderscore auditivo\textunderscore )
\section{Outo}
\begin{itemize}
\item {Grp. gram.:adj.}
\end{itemize}
\begin{itemize}
\item {Grp. gram.:M.}
\end{itemize}
\begin{itemize}
\item {Utilização:Prolóq.}
\end{itemize}
\begin{itemize}
\item {Grp. gram.:Loc.}
\end{itemize}
\begin{itemize}
\item {Utilização:bras. do N}
\end{itemize}
\begin{itemize}
\item {Proveniência:(Lat. \textunderscore octo\textunderscore )}
\end{itemize}
\textunderscore m.\textunderscore  e \textunderscore adj.\textunderscore  (e der.)
(V. \textunderscore oito\textunderscore , etc.)
Diz-se do número cardinal, formado de sete e mais um.
Outavo.
O algarismo representativo do número outo.
Carta de jogar, que tem outo pontos.
Aquillo que numa série de outo occupa o último lugar.
\textunderscore Ou outo ou outenta\textunderscore , ou tudo ou nada.
\textunderscore Tomar um outo\textunderscore , embriagar-se.
\section{Outo}
\begin{itemize}
\item {Grp. gram.:m.}
\end{itemize}
\begin{itemize}
\item {Proveniência:(De \textunderscore outar\textunderscore )}
\end{itemize}
Palhas ou arestas, que ficam na joeira, depois que se joeiraram cereaes.
\section{Outonada}
\begin{itemize}
\item {Grp. gram.:f.}
\end{itemize}
Colheita, que se faz no Outono.
Outono.
\section{Outonado}
\begin{itemize}
\item {Grp. gram.:adj.}
\end{itemize}
\begin{itemize}
\item {Proveniência:(De \textunderscore outonar\textunderscore )}
\end{itemize}
O mesmo que \textunderscore outoniço\textunderscore .
Mal medrado.
\section{Outonal}
\begin{itemize}
\item {Grp. gram.:adj.}
\end{itemize}
\begin{itemize}
\item {Proveniência:(Do lat. \textunderscore autumnalis\textunderscore )}
\end{itemize}
Relativo ao Outono ou que é próprio delle.
\section{Outonar}
\begin{itemize}
\item {Grp. gram.:v. t.}
\end{itemize}
Cavar e regar com as primeiras águas do Outono.
Alqueivar.
\section{Outonear}
\begin{itemize}
\item {Grp. gram.:v. i.}
\end{itemize}
\begin{itemize}
\item {Utilização:bras}
\end{itemize}
\begin{itemize}
\item {Utilização:Neol.}
\end{itemize}
Passar o Outono: \textunderscore outonear nas praias\textunderscore .
\section{Outoniço}
\begin{itemize}
\item {Grp. gram.:adj.}
\end{itemize}
\begin{itemize}
\item {Grp. gram.:M.}
\end{itemize}
\begin{itemize}
\item {Utilização:Agr.}
\end{itemize}
O mesmo que \textunderscore outonal\textunderscore .
Abôrto parcial dos bagos das uvas.
\section{Outono}
\begin{itemize}
\item {Grp. gram.:m.}
\end{itemize}
\begin{itemize}
\item {Utilização:Fig.}
\end{itemize}
\begin{itemize}
\item {Grp. gram.:Pl.}
\end{itemize}
\begin{itemize}
\item {Utilização:Prov.}
\end{itemize}
\begin{itemize}
\item {Utilização:dur.}
\end{itemize}
\begin{itemize}
\item {Proveniência:(Do lat. \textunderscore autumnus\textunderscore )}
\end{itemize}
Terceira estação do anno, entre o Estio e o Inverno.
Colheita.
Decadência; idade, que precede a velhice.
Cereaes, que se semeiam no Outono.
\section{Outorga}
\begin{itemize}
\item {Grp. gram.:f.}
\end{itemize}
Acto ou effeito de outorgar: \textunderscore a outorga da Carta Constitucional\textunderscore .
\section{Outorgadamente}
\begin{itemize}
\item {Grp. gram.:adv.}
\end{itemize}
De modo outorgado.
Com permissão, com autorização.
\section{Outorgador}
\begin{itemize}
\item {Grp. gram.:m.  e  adj.}
\end{itemize}
O que outorga.
\section{Outorgamento}
\begin{itemize}
\item {Grp. gram.:m.}
\end{itemize}
O mesmo que \textunderscore outorga\textunderscore .
\section{Outorgante}
\begin{itemize}
\item {Grp. gram.:m. ,  f.  e  adj.}
\end{itemize}
Pessôa, que outorga.
Cada uma das partes que intervêm numa escritura pública.
\section{Outorgar}
\begin{itemize}
\item {Grp. gram.:v. t.}
\end{itemize}
\begin{itemize}
\item {Grp. gram.:V. i.}
\end{itemize}
\begin{itemize}
\item {Proveniência:(Do lat. hypoth. \textunderscore auctoricare\textunderscore )}
\end{itemize}
Annuír a, approvar.
Conceder, dar.
Declarar em escritura pública.
Intervir, como parte interessada, numa escritura pública. Cf. Rui Barb., \textunderscore Réplica\textunderscore , 160.
\section{Outrega}
\begin{itemize}
\item {Grp. gram.:f.}
\end{itemize}
\begin{itemize}
\item {Utilização:Ant.}
\end{itemize}
Rixa, desordem súbita, sem premeditação.
Desvairamento.
Paixão, que offusca o juízo.
\section{Outrem}
\begin{itemize}
\item {Grp. gram.:pron. indef.}
\end{itemize}
\begin{itemize}
\item {Proveniência:(Do lat. \textunderscore alter\textunderscore )}
\end{itemize}
Outra pessôa, outras pessoas.
\section{Outri}
\begin{itemize}
\item {Grp. gram.:pron. indef.}
\end{itemize}
\begin{itemize}
\item {Utilização:Ant.}
\end{itemize}
\begin{itemize}
\item {Proveniência:(Do fr. \textunderscore autrui\textunderscore )}
\end{itemize}
O mesmo que \textunderscore outrem\textunderscore .
\section{Outro}
\begin{itemize}
\item {Grp. gram.:adj.}
\end{itemize}
\begin{itemize}
\item {Grp. gram.:Pl.}
\end{itemize}
\begin{itemize}
\item {Proveniência:(Do lat. \textunderscore alter\textunderscore )}
\end{itemize}
Differente.
Distinto de uma coisa ou pessôa.
Seguinte, restante.
Que não está presente.
Semelhante.
Melhor.
Outrem.
A outra gente.
\section{Outro-diaço}
\begin{itemize}
\item {Grp. gram.:loc. adv.}
\end{itemize}
\begin{itemize}
\item {Utilização:T. da Bairrada}
\end{itemize}
\begin{itemize}
\item {Utilização:fam.}
\end{itemize}
Aqui há dias; aqui há tempos.
\section{Outrora}
\begin{itemize}
\item {Grp. gram.:adv.}
\end{itemize}
\begin{itemize}
\item {Proveniência:(De \textunderscore outro\textunderscore  + \textunderscore hora\textunderscore )}
\end{itemize}
Em outro tempo, já passado.
Antigamente.
\section{Outrosim}
\begin{itemize}
\item {fónica:sim}
\end{itemize}
\begin{itemize}
\item {Grp. gram.:adv.}
\end{itemize}
\begin{itemize}
\item {Proveniência:(De \textunderscore outro\textunderscore  + \textunderscore sim\textunderscore )}
\end{itemize}
Igualmente, também; bem assim.
\section{Outrossim}
\begin{itemize}
\item {Grp. gram.:adv.}
\end{itemize}
\begin{itemize}
\item {Proveniência:(De \textunderscore outro\textunderscore  + \textunderscore sim\textunderscore )}
\end{itemize}
Igualmente, também; bem assim.
\section{Outro-tanto}
\begin{itemize}
\item {Grp. gram.:adj.}
\end{itemize}
\begin{itemize}
\item {Utilização:Pop.}
\end{itemize}
Igual, parecido. Cf. Júl. Dinis, \textunderscore Morgadinha\textunderscore , 369.
\section{Outubro}
\begin{itemize}
\item {Grp. gram.:m.}
\end{itemize}
\begin{itemize}
\item {Proveniência:(Do lat. \textunderscore october\textunderscore )}
\end{itemize}
Décimo mês do anno, segundo o nosso calendário.
\section{Ouvença}
\begin{itemize}
\item {Grp. gram.:m.}
\end{itemize}
O mesmo que \textunderscore ovença\textunderscore .
\section{Ouvida}
\begin{itemize}
\item {Grp. gram.:f.}
\end{itemize}
Acto ou effeito de ouvir.
Outiva.
\section{Ouvido}
\begin{itemize}
\item {Grp. gram.:m.}
\end{itemize}
\begin{itemize}
\item {Utilização:Mús.}
\end{itemize}
\begin{itemize}
\item {Utilização:Mús.}
\end{itemize}
\begin{itemize}
\item {Grp. gram.:Loc. adv.}
\end{itemize}
\begin{itemize}
\item {Utilização:Pop.}
\end{itemize}
\begin{itemize}
\item {Grp. gram.:Loc.}
\end{itemize}
\begin{itemize}
\item {Utilização:fam.}
\end{itemize}
Aquelle dos cinco sentidos, que tem por órgão a orelha.
Orelha.
Orifício, por onde se communica fogo a algumas armas ou peças de artilharia.
Facilidade de fixar na memória peças musicaes: \textunderscore êste aprendiz de música tem bom ouvido\textunderscore .
Abertura, no tampo dos instrumentos de corda, por onde se transmitem os sons á caixa de resonância.
Orifício, coberto com chave, nos instrumentos de palheta, para facilitar a emissão de certas notas agudas.
\textunderscore De ouvido\textunderscore , só porque ouviu; sem conhecimentos theóricos; sem fundamento.
\textunderscore Bicho do ouvido\textunderscore , a parte mais interior do ouvido, em fórma de caracol.
\textunderscore Matar o bicho do ouvido\textunderscore , incommodar com palavras, insistentes e maçadoras.
\section{Ouvidor}
\begin{itemize}
\item {Grp. gram.:m.}
\end{itemize}
O que ouve.
Magistrado adjunto a certas repartições públicas.
\section{Ouvidoria}
\begin{itemize}
\item {Grp. gram.:f.}
\end{itemize}
Cargo de ouvidor.
\section{Ouviela}
\begin{itemize}
\item {Grp. gram.:f.}
\end{itemize}
\begin{itemize}
\item {Utilização:Prov.}
\end{itemize}
\begin{itemize}
\item {Proveniência:(Do lat. \textunderscore alveolus\textunderscore ?)}
\end{itemize}
Rêgo ou abertura no solo, para escoamento de águas.
Pantano.
\section{Ouvinte}
\begin{itemize}
\item {Grp. gram.:m. ,  f.  e  adj.}
\end{itemize}
\begin{itemize}
\item {Proveniência:(De \textunderscore ouvir\textunderscore )}
\end{itemize}
Pessôa, que ouve ou que assiste a um discurso, prelecção, etc.
Estudante, que frequenta uma aula, sem estar matriculado.
\section{Ouvir}
\begin{itemize}
\item {Grp. gram.:v. t.}
\end{itemize}
\begin{itemize}
\item {Proveniência:(Do lat. \textunderscore audire\textunderscore )}
\end{itemize}
Perceber pelo sentido do ouvido.
Entender o som ou a palavra de; escutar.
Attender.
Deferir.
Inquirir.
\section{Ouvisto}
\textunderscore part. pop.\textunderscore  de \textunderscore ouvir\textunderscore : \textunderscore tenho ouvisto muita história\textunderscore . Cf. Castilho, \textunderscore Méd. á Fôrça\textunderscore , 25.
\section{Ova}
\begin{itemize}
\item {Grp. gram.:f.}
\end{itemize}
\begin{itemize}
\item {Grp. gram.:Pl.}
\end{itemize}
\begin{itemize}
\item {Proveniência:(Lat. \textunderscore ova\textunderscore )}
\end{itemize}
Ovário dos peixes.
Tumores molles, resultantes da dilatação de certas membranas entre a pelle e os ossos ou cartilagens das bêstas.
\section{Ovação}
\begin{itemize}
\item {Grp. gram.:f.}
\end{itemize}
\begin{itemize}
\item {Utilização:Fig.}
\end{itemize}
\begin{itemize}
\item {Proveniência:(Lat. \textunderscore ovatio\textunderscore )}
\end{itemize}
Acclamação pública; honras enthusiásticas, feitas a alguém.
\section{Ovação}
\begin{itemize}
\item {Grp. gram.:f.}
\end{itemize}
\begin{itemize}
\item {Proveniência:(De \textunderscore ovar\textunderscore )}
\end{itemize}
Conjunto dos ovos dos peixes.
\section{Ovada}
\begin{itemize}
\item {Grp. gram.:f.}
\end{itemize}
Porção de ovos.
\section{Ovado}
\begin{itemize}
\item {Grp. gram.:adj.}
\end{itemize}
\begin{itemize}
\item {Utilização:Bras. do S}
\end{itemize}
\begin{itemize}
\item {Grp. gram.:M.}
\end{itemize}
\begin{itemize}
\item {Utilização:Heráld.}
\end{itemize}
\begin{itemize}
\item {Proveniência:(De \textunderscore ovar\textunderscore )}
\end{itemize}
O mesmo que \textunderscore oval\textunderscore .
Diz-se do cavallo, que tem doença na parte posterior da junta da quartella.
Moldura principal do capitel dórico.
Escudo de fórma oval, pertencente a ecclesiásticos.
\section{Oval}
\begin{itemize}
\item {Grp. gram.:adj.}
\end{itemize}
\begin{itemize}
\item {Grp. gram.:F.}
\end{itemize}
\begin{itemize}
\item {Proveniência:(Lat. \textunderscore ovalis\textunderscore )}
\end{itemize}
Que tem fórma de ovo.
Curva, que tem a fórma da secção longitudinal de um ovo.
\section{Ovalar}
\begin{itemize}
\item {Grp. gram.:v. t.}
\end{itemize}
Tornar oval.
\section{Óvalo}
\begin{itemize}
\item {Grp. gram.:m.}
\end{itemize}
\begin{itemize}
\item {Proveniência:(De \textunderscore ovo\textunderscore )}
\end{itemize}
Ornato oval nos capitéis das Ordens jónica e compósita.
\section{Ovampos}
\begin{itemize}
\item {Grp. gram.:m. pl.}
\end{itemize}
Povos da Ovâmpia, na África.
\section{Óvano}
\begin{itemize}
\item {Grp. gram.:m.}
\end{itemize}
O mesmo que \textunderscore óvalo\textunderscore .
\section{Ovante}
\begin{itemize}
\item {Grp. gram.:adj.}
\end{itemize}
\begin{itemize}
\item {Proveniência:(Lat. \textunderscore ovans\textunderscore )}
\end{itemize}
Triumphante; jubiloso.
\section{Ovar}
\begin{itemize}
\item {Grp. gram.:v. i.}
\end{itemize}
Pôr ovos.
Criar ovos ou ovas.
\section{Ovariano}
\begin{itemize}
\item {Grp. gram.:adj.}
\end{itemize}
Relativo a ovário.
\section{Ovárico}
\begin{itemize}
\item {Grp. gram.:adj.}
\end{itemize}
O mesmo que \textunderscore ovariano\textunderscore ; que fórma o ovário.
\section{Ovariectomia}
\begin{itemize}
\item {Grp. gram.:f.}
\end{itemize}
\begin{itemize}
\item {Utilização:Med.}
\end{itemize}
\begin{itemize}
\item {Proveniência:(Do lat. \textunderscore ovarium\textunderscore  + gr. \textunderscore ektome\textunderscore )}
\end{itemize}
O mesmo ou melhor que \textunderscore ovariotomia\textunderscore .
\section{Ovarino}
\begin{itemize}
\item {Grp. gram.:adj.}
\end{itemize}
\begin{itemize}
\item {Grp. gram.:M.}
\end{itemize}
Relativo a Ovar.
Habitante ou indivíduo natural de Ovar.
\section{Ovário}
\begin{itemize}
\item {Grp. gram.:m.}
\end{itemize}
\begin{itemize}
\item {Utilização:Bot.}
\end{itemize}
\begin{itemize}
\item {Proveniência:(Lat. \textunderscore ovarius\textunderscore )}
\end{itemize}
Órgão dos animaes ovíparos, destinado á producção dos ovos.
Cada um dos dois corpos lateraes do útero, que na fêmea dos mammíferos contém os ovos destinados á fecundação.
Parte do pistillo, que encerra as sementes.
\section{Ovariocele}
\begin{itemize}
\item {Grp. gram.:m.}
\end{itemize}
\begin{itemize}
\item {Proveniência:(De \textunderscore ovário\textunderscore  + gr. \textunderscore kele\textunderscore )}
\end{itemize}
Hérnia no ovário.
\section{Ovarioterapia}
\begin{itemize}
\item {Grp. gram.:f.}
\end{itemize}
\begin{itemize}
\item {Proveniência:(De \textunderscore ovário\textunderscore  + \textunderscore terapia\textunderscore )}
\end{itemize}
Emprêgo terapêutico das preparações ovarianas.
\section{Ovariotherapia}
\begin{itemize}
\item {Grp. gram.:f.}
\end{itemize}
\begin{itemize}
\item {Proveniência:(De \textunderscore ovário\textunderscore  + \textunderscore therapia\textunderscore )}
\end{itemize}
Emprêgo therapêutico das preparações ovarianas.
\section{Ovariotomia}
\begin{itemize}
\item {Grp. gram.:f.}
\end{itemize}
\begin{itemize}
\item {Proveniência:(De \textunderscore ovário\textunderscore  + gr. \textunderscore tome\textunderscore )}
\end{itemize}
Extracção cirúrgica do ovário ou de cistos de ovário.
\section{Ovarismo}
\begin{itemize}
\item {Grp. gram.:m.}
\end{itemize}
\begin{itemize}
\item {Proveniência:(De \textunderscore ovário\textunderscore )}
\end{itemize}
Hypóthese physiólogica, que faz originar de um ovo todos os animaes e até todos os corpos organizados.
\section{Ovarista}
\begin{itemize}
\item {Grp. gram.:m. ,  f.  e  adj.}
\end{itemize}
\begin{itemize}
\item {Proveniência:(De \textunderscore ovário\textunderscore )}
\end{itemize}
Pessôa, partidária do ovarismo.
\section{Ovarite}
\begin{itemize}
\item {Grp. gram.:f.}
\end{itemize}
\begin{itemize}
\item {Utilização:Med.}
\end{itemize}
Inflammação do ovário.
\section{Ove}
\begin{itemize}
\item {Grp. gram.:f.}
\end{itemize}
\begin{itemize}
\item {Utilização:Des.}
\end{itemize}
\begin{itemize}
\item {Proveniência:(Lat. \textunderscore ovis\textunderscore )}
\end{itemize}
O mesmo que \textunderscore ovelha\textunderscore .
\section{Oveiro}
\begin{itemize}
\item {Grp. gram.:adj.}
\end{itemize}
\begin{itemize}
\item {Utilização:Bras. do S}
\end{itemize}
\begin{itemize}
\item {Grp. gram.:M.}
\end{itemize}
\begin{itemize}
\item {Utilização:Pop.}
\end{itemize}
\begin{itemize}
\item {Proveniência:(De \textunderscore ovo\textunderscore )}
\end{itemize}
Diz-se do cavallo ou boi, que tem malhas vermelhas ou pretas sôbre o corpo branco, ou viceversa.
Ânus do falcão.
Ovário das aves.
Vasilha, em que vão os ovos para a mesa, já preparados.
Peixe de Portugal.
\section{Ovelha}
\begin{itemize}
\item {fónica:vê}
\end{itemize}
\begin{itemize}
\item {Grp. gram.:f.}
\end{itemize}
\begin{itemize}
\item {Utilização:Fig.}
\end{itemize}
\begin{itemize}
\item {Proveniência:(Do lat. \textunderscore ovecula\textunderscore )}
\end{itemize}
Fêmea do carneiro.
O christão, relativamente ao seu pastor espiritual.
\section{Ovelhada}
\begin{itemize}
\item {Grp. gram.:f.}
\end{itemize}
Rebanho de ovelhas.
\section{Ovelheiro}
\begin{itemize}
\item {Grp. gram.:m.}
\end{itemize}
Pastor de ovelhas.
\section{Ovelhum}
\begin{itemize}
\item {Grp. gram.:adj.}
\end{itemize}
Relativo a ovelhas, ou a ovelhas, carneiros e cordeiros: \textunderscore gado ovelhum\textunderscore .
\section{Ovelhuno}
\begin{itemize}
\item {Grp. gram.:adj.}
\end{itemize}
O mesmo que \textunderscore ovelhum\textunderscore . Cf. Filinto, XIV, 238; XV, 302.
\section{Óvem}
\begin{itemize}
\item {Grp. gram.:m.}
\end{itemize}
\begin{itemize}
\item {Utilização:Náut.}
\end{itemize}
\begin{itemize}
\item {Proveniência:(Do ant. fr. \textunderscore hauban\textunderscore )}
\end{itemize}
Grosso calabre de navio.
Cada uma das pernadas da enxárcia.
\section{Ovença}
\begin{itemize}
\item {Grp. gram.:f.}
\end{itemize}
\begin{itemize}
\item {Utilização:Ant.}
\end{itemize}
\begin{itemize}
\item {Proveniência:(Do lat. \textunderscore officium\textunderscore ? De \textunderscore houve\textunderscore , de \textunderscore haver\textunderscore ?)}
\end{itemize}
Serviço de mesa e comedorias, entre os cónegos regrantes.
Officina.
\section{Ovencadura}
\begin{itemize}
\item {Grp. gram.:f.}
\end{itemize}
\begin{itemize}
\item {Utilização:Náut.}
\end{itemize}
Conjunto de óvens.
A enxárcia real.
\section{Ovençal}
\begin{itemize}
\item {Grp. gram.:m.}
\end{itemize}
\begin{itemize}
\item {Utilização:Ant.}
\end{itemize}
\begin{itemize}
\item {Proveniência:(De \textunderscore ovença\textunderscore )}
\end{itemize}
Cobrador de rendimentos nacionaes.
Dispenseiro, vedor de ucharia.
\section{Óveo}
\begin{itemize}
\item {Grp. gram.:adj.}
\end{itemize}
\begin{itemize}
\item {Proveniência:(Do lat. \textunderscore ovum\textunderscore )}
\end{itemize}
O mesmo que \textunderscore oval\textunderscore .
Que contém ovos.
\section{Ovetense}
\begin{itemize}
\item {Grp. gram.:adj.}
\end{itemize}
\begin{itemize}
\item {Proveniência:(Do lat. \textunderscore Ovetum\textunderscore , n. p.)}
\end{itemize}
Relativo ao Oviedo. Cf. Herculano, \textunderscore Hist. de Port.\textunderscore , III, 185 e 264; \textunderscore Opúsc.\textunderscore , III 269 e 276.
\section{Oviado}
\begin{itemize}
\item {Grp. gram.:adj.}
\end{itemize}
\begin{itemize}
\item {Utilização:Ant.}
\end{itemize}
\begin{itemize}
\item {Proveniência:(Do lat. \textunderscore ovare\textunderscore )}
\end{itemize}
Triumphante; soberbo.
\section{Oviário}
\begin{itemize}
\item {Grp. gram.:m.}
\end{itemize}
\begin{itemize}
\item {Proveniência:(Lat. \textunderscore oviarium\textunderscore )}
\end{itemize}
O mesmo que \textunderscore ovil\textunderscore .
Rebanho de ovelhas.
\section{Ovículo}
\begin{itemize}
\item {Grp. gram.:m.}
\end{itemize}
\begin{itemize}
\item {Proveniência:(De \textunderscore ovo\textunderscore )}
\end{itemize}
Pequeno ornato oval.
\section{Ovídeos}
\begin{itemize}
\item {Grp. gram.:m. pl.}
\end{itemize}
\begin{itemize}
\item {Proveniência:(Do lat. \textunderscore ovis\textunderscore  + gr. \textunderscore eidos\textunderscore )}
\end{itemize}
Ordem de mammíferos, que comprehende a ovelha, o carneiro e o cordeiro.
\section{Ovidiano}
\begin{itemize}
\item {Grp. gram.:adj.}
\end{itemize}
Relativo a Ovídio, ás suas obras ou ao seu estilo.
\section{Oviducto}
\begin{itemize}
\item {Grp. gram.:m.}
\end{itemize}
\begin{itemize}
\item {Utilização:Anat.}
\end{itemize}
\begin{itemize}
\item {Proveniência:(Do lat. \textunderscore ovum\textunderscore  + \textunderscore ductus\textunderscore )}
\end{itemize}
Canal, que nas aves dá passagem ao ovo, desde que êste deixa o ovário.
Trompa de Fallópio.
\section{Oviela}
\begin{itemize}
\item {Grp. gram.:f.}
\end{itemize}
O mesmo que \textunderscore ouviela\textunderscore .
\section{Ovificação}
\begin{itemize}
\item {Grp. gram.:f.}
\end{itemize}
\begin{itemize}
\item {Proveniência:(Do lat. \textunderscore ovum\textunderscore  + \textunderscore facere\textunderscore )}
\end{itemize}
Formação natural do ovo.
\section{Oviforme}
\begin{itemize}
\item {Grp. gram.:adj.}
\end{itemize}
\begin{itemize}
\item {Proveniência:(Do lat. \textunderscore ovum\textunderscore  + \textunderscore forma\textunderscore )}
\end{itemize}
O mesmo que \textunderscore oval\textunderscore .
\section{Ovil}
\begin{itemize}
\item {Grp. gram.:m.}
\end{itemize}
\begin{itemize}
\item {Proveniência:(Lat. \textunderscore ovile\textunderscore )}
\end{itemize}
Curral de ovelhas; redil; aprisco.
\section{Ovimbundos}
\begin{itemize}
\item {Grp. gram.:m. pl.}
\end{itemize}
Tríbo de Benguela.
\section{Ovinha}
\begin{itemize}
\item {Grp. gram.:f.}
\end{itemize}
\begin{itemize}
\item {Utilização:Prov.}
\end{itemize}
\begin{itemize}
\item {Utilização:trasm.}
\end{itemize}
Ovo do ninho de pássaros.
\section{Ovino}
\begin{itemize}
\item {Grp. gram.:adj.}
\end{itemize}
\begin{itemize}
\item {Utilização:Poét.}
\end{itemize}
\begin{itemize}
\item {Proveniência:(Lat. \textunderscore ovinus\textunderscore )}
\end{itemize}
O mesmo que \textunderscore ovelhum\textunderscore .
\section{Oviparidade}
\begin{itemize}
\item {Grp. gram.:f.}
\end{itemize}
Qualidade de ovíparo.
\section{Oviparismo}
\begin{itemize}
\item {Grp. gram.:m.}
\end{itemize}
O mesmo que \textunderscore oviparidade\textunderscore .
\section{Ovíparo}
\begin{itemize}
\item {Grp. gram.:adj.}
\end{itemize}
\begin{itemize}
\item {Grp. gram.:M.}
\end{itemize}
\begin{itemize}
\item {Proveniência:(Lat. \textunderscore oviparus\textunderscore )}
\end{itemize}
Que põe ovos.
Que se reproduz por meio de ovos.
Animal ovíparo.
\section{Ovisaco}
\begin{itemize}
\item {fónica:sa}
\end{itemize}
\begin{itemize}
\item {Grp. gram.:m.}
\end{itemize}
\begin{itemize}
\item {Utilização:Anat.}
\end{itemize}
\begin{itemize}
\item {Proveniência:(De \textunderscore ovo\textunderscore  + \textunderscore saco\textunderscore )}
\end{itemize}
Vesícula de Graaf.
\section{Oviscapto}
\begin{itemize}
\item {Grp. gram.:m.}
\end{itemize}
\begin{itemize}
\item {Utilização:Zool.}
\end{itemize}
\begin{itemize}
\item {Proveniência:(Do lat. \textunderscore ovum\textunderscore  + gr. \textunderscore skaptein\textunderscore )}
\end{itemize}
Prolongação do abdome das fêmeas de alguns insectos, com a qual depositam os ovos em certas cavidades.
\section{Ovismo}
\begin{itemize}
\item {Grp. gram.:m.}
\end{itemize}
\begin{itemize}
\item {Utilização:Hist. Nat.}
\end{itemize}
\begin{itemize}
\item {Proveniência:(De \textunderscore ovo\textunderscore )}
\end{itemize}
Systema dos que, como Malpighi e outros, entendem que as partes essenciaes de um indivíduo existem na fêmea antes da fecundação, sendo em tal caso a fecundação uma circunstância exterior e accessória.
\section{Ovissaco}
\begin{itemize}
\item {Grp. gram.:m.}
\end{itemize}
\begin{itemize}
\item {Utilização:Anat.}
\end{itemize}
\begin{itemize}
\item {Proveniência:(De \textunderscore ovo\textunderscore  + \textunderscore saco\textunderscore )}
\end{itemize}
Vesícula de Graaf.
\section{Ovista}
\begin{itemize}
\item {Grp. gram.:m.}
\end{itemize}
Partidário do ovismo.
\section{Ovívoro}
\begin{itemize}
\item {Grp. gram.:adj.}
\end{itemize}
\begin{itemize}
\item {Proveniência:(Do lat. \textunderscore ovum\textunderscore  + \textunderscore vorare\textunderscore )}
\end{itemize}
Que se alimenta de ovos.
\section{Ovo}
\begin{itemize}
\item {Grp. gram.:m.}
\end{itemize}
\begin{itemize}
\item {Utilização:Restrict.}
\end{itemize}
\begin{itemize}
\item {Utilização:Fig.}
\end{itemize}
\begin{itemize}
\item {Proveniência:(Lat. \textunderscore ovum\textunderscore )}
\end{itemize}
Corpo, que se fórma no ovário, e que encerra o germe animal e certos líquidos destinados a sustentar esse germe por algum tempo.
Ovo de ave, especialmente de gallinha.
Fórma rudimentar; princípio: \textunderscore expor um assumpto desde o ovo\textunderscore .
\section{Ovo-de-avestruz}
\begin{itemize}
\item {Grp. gram.:m.}
\end{itemize}
Variedade de pêra ordinária.
\section{Ovogenia}
\begin{itemize}
\item {Grp. gram.:f.}
\end{itemize}
\begin{itemize}
\item {Proveniência:(Do gr. \textunderscore oon\textunderscore  + \textunderscore genea\textunderscore )}
\end{itemize}
História physiológica do ovo.
Desenvolvimento do ovo.
\section{Ovogênico}
\begin{itemize}
\item {Grp. gram.:adj.}
\end{itemize}
Relativo á ovogenia.
\section{Ovóide}
\begin{itemize}
\item {Grp. gram.:adj.}
\end{itemize}
\begin{itemize}
\item {Proveniência:(Do gr. \textunderscore oon\textunderscore  + \textunderscore eidos\textunderscore )}
\end{itemize}
O mesmo que \textunderscore oval\textunderscore .
\section{Ovologia}
\begin{itemize}
\item {Grp. gram.:f.}
\end{itemize}
\begin{itemize}
\item {Proveniência:(Do gr. \textunderscore oon\textunderscore  + \textunderscore logos\textunderscore )}
\end{itemize}
Tratado á cêrca dos ovos.
\section{Ovológico}
\begin{itemize}
\item {Grp. gram.:adj.}
\end{itemize}
Relativo á ovologia.
\section{Ovos-molles}
\begin{itemize}
\item {Grp. gram.:m. pl.}
\end{itemize}
\begin{itemize}
\item {Grp. gram.:F.}
\end{itemize}
Iguaria doce, em que entram gemmas de ovos e açúcar.
Variedade de pêra, também conhecida por \textunderscore providência\textunderscore  e \textunderscore camurça\textunderscore .
\section{Ovos-reaes}
\begin{itemize}
\item {Grp. gram.:m. pl.}
\end{itemize}
Antiga espécie de doce de ovos e açúcar.
\section{Ovovivíparo}
\begin{itemize}
\item {Grp. gram.:adj.}
\end{itemize}
\begin{itemize}
\item {Proveniência:(De \textunderscore ovo\textunderscore  + \textunderscore vivíparo\textunderscore )}
\end{itemize}
Diz-se do animal, cujo ovo se parte na madre, para dar saída ao filho.
\section{Ovulação}
\begin{itemize}
\item {Grp. gram.:f.}
\end{itemize}
Saída do óvulo.
\section{Ovulado}
\begin{itemize}
\item {Grp. gram.:adj.}
\end{itemize}
Que tem óvulos.
\section{Ovular}
\begin{itemize}
\item {Grp. gram.:adj.}
\end{itemize}
\begin{itemize}
\item {Proveniência:(De \textunderscore óvulo\textunderscore )}
\end{itemize}
Semelhante a um ovo de gallinha; oval.
\section{Ovuliforme}
\begin{itemize}
\item {Grp. gram.:adj.}
\end{itemize}
\begin{itemize}
\item {Proveniência:(De \textunderscore óvulo\textunderscore  + \textunderscore fórma\textunderscore )}
\end{itemize}
Que tem fórma de óvulo.
\section{Óvulo}
\begin{itemize}
\item {Grp. gram.:m.}
\end{itemize}
Pequeno ovo.
Producto do ovário.
Corpúsculo oval, ligado á placenta das plantas, e que se converte em semente.
(Dem. de \textunderscore ovo\textunderscore )
\section{Owenismo}
\begin{itemize}
\item {Grp. gram.:m.}
\end{itemize}
Systema de associação e cooperação, inventado pelo philósopho inglês Roberto Owen.
\section{Owenista}
\begin{itemize}
\item {Grp. gram.:m.}
\end{itemize}
Partidário do owenismo.
\section{Oxácido}
\begin{itemize}
\item {fónica:csá}
\end{itemize}
\begin{itemize}
\item {Grp. gram.:m.}
\end{itemize}
\begin{itemize}
\item {Proveniência:(Do gr. \textunderscore oxus\textunderscore  + lat. \textunderscore acidus\textunderscore )}
\end{itemize}
Acido, resultante da combinação do oxygênio com um corpo simples.
\section{Oxalá!}
\begin{itemize}
\item {Grp. gram.:interj.}
\end{itemize}
\begin{itemize}
\item {Proveniência:(Do ár. \textunderscore in-xa-'llah\textunderscore , queira Deus)}
\end{itemize}
(designativa de \textunderscore desejo\textunderscore )
\section{Oxalatado}
\begin{itemize}
\item {fónica:csá}
\end{itemize}
\begin{itemize}
\item {Grp. gram.:adj.}
\end{itemize}
\begin{itemize}
\item {Proveniência:(De \textunderscore oxalato\textunderscore )}
\end{itemize}
Diz-se do mineral, convertido em sal pela combinação do ácido oxálico.
\section{Oxalato}
\begin{itemize}
\item {fónica:csa}
\end{itemize}
\begin{itemize}
\item {Grp. gram.:m.}
\end{itemize}
\begin{itemize}
\item {Proveniência:(Do gr. \textunderscore oxalis\textunderscore )}
\end{itemize}
Combinação do ácido oxálico com uma base.
\section{Oxalhýdrico}
\begin{itemize}
\item {fónica:csa}
\end{itemize}
\begin{itemize}
\item {Grp. gram.:adj.}
\end{itemize}
Diz-se de um ácido, produzido pela acção do ácido nítrico sôbre differentes substâncias.
\section{Oxálico}
\begin{itemize}
\item {fónica:csá}
\end{itemize}
\begin{itemize}
\item {Grp. gram.:adj.}
\end{itemize}
\begin{itemize}
\item {Proveniência:(Do gr. \textunderscore oxalis\textunderscore )}
\end{itemize}
Diz-se de um ácido, que se encontra na oxálida ou na azêda.
\section{Oxálida}
\begin{itemize}
\item {fónica:csá}
\end{itemize}
\begin{itemize}
\item {Grp. gram.:f.}
\end{itemize}
\begin{itemize}
\item {Utilização:Bot.}
\end{itemize}
\begin{itemize}
\item {Proveniência:(Lat. \textunderscore oxalis\textunderscore )}
\end{itemize}
Planta, o mesmo que \textunderscore azêda\textunderscore , (\textunderscore oxalis acetosela\textunderscore ).
\section{Oxalidáceas}
\begin{itemize}
\item {fónica:csa}
\end{itemize}
\begin{itemize}
\item {Grp. gram.:f. pl.}
\end{itemize}
O mesmo ou melhor que \textunderscore oxalídeas\textunderscore .
\section{Oxálide}
\begin{itemize}
\item {fónica:csá}
\end{itemize}
\begin{itemize}
\item {Grp. gram.:f.}
\end{itemize}
O mesmo ou melhor que \textunderscore oxálida\textunderscore .
\section{Oxalídeas}
\begin{itemize}
\item {fónica:csa}
\end{itemize}
\begin{itemize}
\item {Grp. gram.:f. pl.}
\end{itemize}
Família de plantas, que tem por typo a oxálide.
\section{Oxalídrico}
\begin{itemize}
\item {fónica:csa}
\end{itemize}
\begin{itemize}
\item {Grp. gram.:adj.}
\end{itemize}
Diz-se de um ácido, produzido pela acção do ácido nítrico sôbre differentes substâncias.
\section{Oxalme}
\begin{itemize}
\item {fónica:csal}
\end{itemize}
\begin{itemize}
\item {Grp. gram.:f.}
\end{itemize}
\begin{itemize}
\item {Proveniência:(Lat. \textunderscore oxalme\textunderscore )}
\end{itemize}
Salmoira azêda.
\section{Oxalovinato}
\begin{itemize}
\item {fónica:csa}
\end{itemize}
\begin{itemize}
\item {Grp. gram.:f.}
\end{itemize}
\begin{itemize}
\item {Utilização:Chím.}
\end{itemize}
Sal, resultante da combinação do ácido oxalovínico com uma base.
\section{Oxalovínico}
\begin{itemize}
\item {fónica:csa}
\end{itemize}
\begin{itemize}
\item {Grp. gram.:adj.}
\end{itemize}
Diz-se de um ácido oxálico e hydrogênio carbonado.
\section{Oxaluria}
\begin{itemize}
\item {fónica:csa}
\end{itemize}
\begin{itemize}
\item {Grp. gram.:f.}
\end{itemize}
\begin{itemize}
\item {Proveniência:(De \textunderscore oxalato\textunderscore  + gr. \textunderscore ouron\textunderscore )}
\end{itemize}
Depósito de oxalato de cal nas urinas.
Condições, que determinam esse depósito.
\section{Oxalúrico}
\begin{itemize}
\item {fónica:csa}
\end{itemize}
\begin{itemize}
\item {Grp. gram.:adj.}
\end{itemize}
\begin{itemize}
\item {Grp. gram.:M.}
\end{itemize}
Relativo á oxaluria.
Aquelle que padece oxaluria.
\section{Oxamala!}
\begin{itemize}
\item {Grp. gram.:interj.}
\end{itemize}
\begin{itemize}
\item {Utilização:Ant.}
\end{itemize}
(Servia para designar \textunderscore compaixão\textunderscore ) Cf. G. Vicente; Resende, \textunderscore Cancioneiro\textunderscore , etc.
\section{Oxamelitana}
\begin{itemize}
\item {fónica:csa}
\end{itemize}
\begin{itemize}
\item {Grp. gram.:f.}
\end{itemize}
\begin{itemize}
\item {Utilização:Chím.}
\end{itemize}
Oxalato de methylena e de ammoníaco.
\section{Oxametana}
\begin{itemize}
\item {fónica:csa}
\end{itemize}
\begin{itemize}
\item {Grp. gram.:f.}
\end{itemize}
\begin{itemize}
\item {Utilização:Chím.}
\end{itemize}
Oxalato anhýdrico de ammoníaco e de bicarboneto de hydrogênio.
\section{Oxâmido}
\begin{itemize}
\item {fónica:csá}
\end{itemize}
\begin{itemize}
\item {Grp. gram.:m.}
\end{itemize}
\begin{itemize}
\item {Utilização:Chím.}
\end{itemize}
Producto da destillação do oxalato de ammoníaco.
\section{Oxarita}
\begin{itemize}
\item {fónica:csa}
\end{itemize}
\begin{itemize}
\item {Grp. gram.:f.}
\end{itemize}
\begin{itemize}
\item {Utilização:Miner.}
\end{itemize}
Substância, composta de silicato de cal, ferro e potassa em estado de hydrato, pardacenta, crystallizada em pequenos octaédros e pouco pesada.
\section{Oxeol}
\begin{itemize}
\item {fónica:ocse}
\end{itemize}
\begin{itemize}
\item {Grp. gram.:m.}
\end{itemize}
Vinagre, considerado como excipiente, em pharmácia.
\section{Oxeolato}
\begin{itemize}
\item {fónica:cse}
\end{itemize}
\begin{itemize}
\item {Grp. gram.:m.}
\end{itemize}
\begin{itemize}
\item {Utilização:Pharm.}
\end{itemize}
Gênero de medicamentos, resultantes da acção dissolvente do vinagre sôbre uma ou mais substâncias medicamentosas.
\section{Oxeóleo}
\begin{itemize}
\item {fónica:cse}
\end{itemize}
\begin{itemize}
\item {Grp. gram.:m.}
\end{itemize}
\begin{itemize}
\item {Utilização:Pharm.}
\end{itemize}
O mesmo que \textunderscore oxeol\textunderscore .
\section{Oxeu}
\begin{itemize}
\item {Grp. gram.:m.}
\end{itemize}
\begin{itemize}
\item {Utilização:Des.}
\end{itemize}
Acto de espantar caça de altanaria, para lhe atirar.
(Cast. \textunderscore ojeu\textunderscore )
\section{Oxhydrylo}
\begin{itemize}
\item {Grp. gram.:m.}
\end{itemize}
\begin{itemize}
\item {Proveniência:(Do gr. \textunderscore oxus\textunderscore  + \textunderscore hudor\textunderscore  + \textunderscore ule\textunderscore )}
\end{itemize}
Radical chímico, formado por um átomo de oxygênio e outro de hydrogênio.
\section{Oxiantero}
\begin{itemize}
\item {fónica:csi}
\end{itemize}
\begin{itemize}
\item {Grp. gram.:f.}
\end{itemize}
\begin{itemize}
\item {Proveniência:(Do gr. \textunderscore oxus\textunderscore  + \textunderscore antheros\textunderscore )}
\end{itemize}
Gênero de orquídeas.
\section{Oxianto}
\begin{itemize}
\item {fónica:csi}
\end{itemize}
\begin{itemize}
\item {Grp. gram.:m.}
\end{itemize}
\begin{itemize}
\item {Proveniência:(Do gr. \textunderscore oxus\textunderscore  + \textunderscore anthos\textunderscore )}
\end{itemize}
Gênero de plantas rubiáceas.
\section{Oxíbafo}
\begin{itemize}
\item {fónica:csi}
\end{itemize}
\begin{itemize}
\item {Grp. gram.:m.}
\end{itemize}
\begin{itemize}
\item {Proveniência:(Lat. \textunderscore oxybaphus\textunderscore )}
\end{itemize}
Vasilha para vinagre, entre os antigos Gregos.
Medida de capacidade entre os Romanos, equivalente a 15 dracmas.
Gênero de plantas nictagíneas.
\section{Oxíbase}
\begin{itemize}
\item {fónica:csi}
\end{itemize}
\begin{itemize}
\item {Grp. gram.:m.}
\end{itemize}
\begin{itemize}
\item {Utilização:Chím.}
\end{itemize}
\begin{itemize}
\item {Proveniência:(Do gr. \textunderscore oxus\textunderscore  + \textunderscore basis\textunderscore )}
\end{itemize}
Óxido, que fórma a base das combinações em que entra.
\section{Oxibásico}
\begin{itemize}
\item {fónica:csi}
\end{itemize}
\begin{itemize}
\item {Grp. gram.:adj.}
\end{itemize}
\begin{itemize}
\item {Utilização:Chím.}
\end{itemize}
\begin{itemize}
\item {Proveniência:(De \textunderscore oxíbase\textunderscore )}
\end{itemize}
Diz-se do sal, que tem por base um óxido.
\section{Oxibrácteo}
\begin{itemize}
\item {fónica:csi}
\end{itemize}
\begin{itemize}
\item {Grp. gram.:adj.}
\end{itemize}
\begin{itemize}
\item {Utilização:Bot.}
\end{itemize}
\begin{itemize}
\item {Proveniência:(Do gr. \textunderscore oxus\textunderscore  + lat. \textunderscore bractea\textunderscore )}
\end{itemize}
Que tem brácteas agudas.
\section{Oxibrometo}
\begin{itemize}
\item {fónica:csi}
\end{itemize}
\begin{itemize}
\item {Grp. gram.:m.}
\end{itemize}
\begin{itemize}
\item {Utilização:Chím.}
\end{itemize}
Combinação de um brometo com um óxido.
\section{Oxibutírico}
\begin{itemize}
\item {fónica:csi}
\end{itemize}
\begin{itemize}
\item {Grp. gram.:adj.}
\end{itemize}
Diz-se de um ácido, que se encontra nalgumas urinas diabéticas e que reduz os saes de cobre.
\section{Oxicárbico}
\begin{itemize}
\item {fónica:csi}
\end{itemize}
\begin{itemize}
\item {Grp. gram.:adj.}
\end{itemize}
\begin{itemize}
\item {Utilização:Chím.}
\end{itemize}
\begin{itemize}
\item {Proveniência:(Do gr. \textunderscore oxus\textunderscore  + lat. \textunderscore carbo\textunderscore )}
\end{itemize}
Que contém oxigênio e carvão.
\section{Oxicedro}
\begin{itemize}
\item {fónica:csi}
\end{itemize}
\begin{itemize}
\item {Grp. gram.:m.}
\end{itemize}
\begin{itemize}
\item {Proveniência:(Do gr. \textunderscore oxus\textunderscore  + \textunderscore kedros\textunderscore )}
\end{itemize}
Árvore conífera do sul da Europa, espécie de zimbro, cuja madeira produz, pela destilação, um óleo conhecido em Veterinária por \textunderscore óleo de zimbro\textunderscore . Cf. \textunderscore Bibl. da G. do Campo\textunderscore , 274.
\section{Oxicéfala}
\begin{itemize}
\item {fónica:csi}
\end{itemize}
\begin{itemize}
\item {Grp. gram.:f.}
\end{itemize}
Grande serpente das regiões tropicaes.
(Cp. \textunderscore oxicéfalo\textunderscore )
\section{Oxicefalia}
\begin{itemize}
\item {fónica:csi}
\end{itemize}
\begin{itemize}
\item {Grp. gram.:f.}
\end{itemize}
Estado ou qualidade de oxicéfalo.
\section{Oxicéfalo}
\begin{itemize}
\item {fónica:csi}
\end{itemize}
\begin{itemize}
\item {Grp. gram.:m.}
\end{itemize}
\begin{itemize}
\item {Grp. gram.:Adj.}
\end{itemize}
\begin{itemize}
\item {Proveniência:(Do gr. \textunderscore oxus\textunderscore  + \textunderscore kephale\textunderscore )}
\end{itemize}
Gênero de crustáceos anfípodes.
Gênero de insectos coleópteros.
O mesmo que \textunderscore acrocéfalo\textunderscore .
\section{Oxícera}
\begin{itemize}
\item {fónica:csi}
\end{itemize}
\begin{itemize}
\item {Grp. gram.:f.}
\end{itemize}
\begin{itemize}
\item {Proveniência:(Do gr. \textunderscore oxus\textunderscore  + \textunderscore keras\textunderscore )}
\end{itemize}
Gênero de insectos dípteros.
\section{Oxiclorato}
\begin{itemize}
\item {fónica:csi}
\end{itemize}
\begin{itemize}
\item {Grp. gram.:m.}
\end{itemize}
\begin{itemize}
\item {Utilização:Chím.}
\end{itemize}
Sal, resultante da combinação do ácido oxiclórico com uma base.
\section{Oxiclórico}
\begin{itemize}
\item {fónica:csi}
\end{itemize}
\begin{itemize}
\item {Grp. gram.:adj.}
\end{itemize}
Diz-se do ácido, que constitue o maior grau de oxigenação do cloro.
\section{Oxicoco}
\begin{itemize}
\item {fónica:csi}
\end{itemize}
\begin{itemize}
\item {Grp. gram.:m.}
\end{itemize}
Gênero de plantas ericáceas.
\section{Oxicrato}
\begin{itemize}
\item {fónica:csi}
\end{itemize}
\begin{itemize}
\item {Grp. gram.:m.}
\end{itemize}
\begin{itemize}
\item {Proveniência:(Gr. \textunderscore oxukraton\textunderscore )}
\end{itemize}
Mistura de vinagre e água em certa proporção.
\section{Oxidabilidade}
\begin{itemize}
\item {fónica:csi}
\end{itemize}
\begin{itemize}
\item {Grp. gram.:f.}
\end{itemize}
Qualidade do que é oxidável.
\section{Oxidação}
\begin{itemize}
\item {fónica:csi}
\end{itemize}
\begin{itemize}
\item {Grp. gram.:f.}
\end{itemize}
Acto de oxidar.
Oxigenação.
\section{Oxidante}
\begin{itemize}
\item {fónica:csi}
\end{itemize}
\begin{itemize}
\item {Grp. gram.:adj.}
\end{itemize}
Que tem a propriedade de oxidar.
\section{Oxidar}
\begin{itemize}
\item {fónica:csi}
\end{itemize}
\begin{itemize}
\item {Grp. gram.:v.}
\end{itemize}
\begin{itemize}
\item {Utilização:t. Chím.}
\end{itemize}
\begin{itemize}
\item {Grp. gram.:V. p.}
\end{itemize}
\begin{itemize}
\item {Utilização:Fig.}
\end{itemize}
\begin{itemize}
\item {Proveniência:(De \textunderscore óxido\textunderscore )}
\end{itemize}
Converter em óxido.
Combinar com o oxigênio.
Oxigenar-se.
Enferrujar-se.
\section{Oxídase}
\begin{itemize}
\item {fónica:csi}
\end{itemize}
\begin{itemize}
\item {Grp. gram.:f.}
\end{itemize}
\begin{itemize}
\item {Proveniência:(De \textunderscore óxido\textunderscore )}
\end{itemize}
Fermento oxidante.
\section{Oxidável}
\begin{itemize}
\item {fónica:csi}
\end{itemize}
\begin{itemize}
\item {Grp. gram.:adj.}
\end{itemize}
Que se pode oxidar.
\section{Oxídio}
\begin{itemize}
\item {fónica:csi}
\end{itemize}
\begin{itemize}
\item {Grp. gram.:m.}
\end{itemize}
\begin{itemize}
\item {Proveniência:(Do gr. \textunderscore oxus\textunderscore )}
\end{itemize}
Gênero de plantas leguminosas.
\section{Óxido}
\begin{itemize}
\item {fónica:csi}
\end{itemize}
\begin{itemize}
\item {Grp. gram.:m.}
\end{itemize}
\begin{itemize}
\item {Utilização:Chím.}
\end{itemize}
\begin{itemize}
\item {Proveniência:(Do gr. \textunderscore oxus\textunderscore )}
\end{itemize}
Composto neutro de oxigênio e de um metalóide ou metal.
\section{Oxidonte}
\begin{itemize}
\item {fónica:csi}
\end{itemize}
\begin{itemize}
\item {Grp. gram.:m.}
\end{itemize}
\begin{itemize}
\item {Proveniência:(Do gr. \textunderscore oxus\textunderscore , agudo, e \textunderscore odous\textunderscore , dente)}
\end{itemize}
Gênero de plantas herbáceas da Nova-Holanda.
\section{Oxidrilo}
\begin{itemize}
\item {fónica:csi}
\end{itemize}
\begin{itemize}
\item {Grp. gram.:m.}
\end{itemize}
\begin{itemize}
\item {Proveniência:(Do gr. \textunderscore oxus\textunderscore  + \textunderscore hudor\textunderscore  + \textunderscore ule\textunderscore )}
\end{itemize}
Radical químico, formado por um átomo de oxigênio e outro de hidrogênio.
\section{Oxidulado}
\begin{itemize}
\item {fónica:csi}
\end{itemize}
\begin{itemize}
\item {Grp. gram.:adj.}
\end{itemize}
Que passou ao estado de oxídulo.
\section{Oxídulo}
\begin{itemize}
\item {fónica:csi}
\end{itemize}
\begin{itemize}
\item {Grp. gram.:m.}
\end{itemize}
\begin{itemize}
\item {Utilização:Chím.}
\end{itemize}
\begin{itemize}
\item {Proveniência:(De \textunderscore óxido\textunderscore )}
\end{itemize}
Primeiro grau inferior da oxidação de um corpo.
\section{Oxietérico}
\begin{itemize}
\item {fónica:csi}
\end{itemize}
\begin{itemize}
\item {Grp. gram.:adj.}
\end{itemize}
Diz-se das luzes, produzidas por oxigênio e éter. Cf. \textunderscore Jorn.-do-Comm.\textunderscore , do Rio, de 27-V-902.
\section{Oxígala}
\begin{itemize}
\item {fónica:csi}
\end{itemize}
\begin{itemize}
\item {Grp. gram.:f.}
\end{itemize}
\begin{itemize}
\item {Proveniência:(Do gr. \textunderscore oxus\textunderscore  + \textunderscore gala\textunderscore )}
\end{itemize}
Leite azêdo.
\section{Oxigenabilidade}
\begin{itemize}
\item {fónica:csi}
\end{itemize}
\begin{itemize}
\item {Grp. gram.:f.}
\end{itemize}
\begin{itemize}
\item {Utilização:Chím.}
\end{itemize}
Qualidade de oxigenável.
\section{Oxigenação}
\begin{itemize}
\item {fónica:csi}
\end{itemize}
\begin{itemize}
\item {Grp. gram.:f.}
\end{itemize}
Acto ou efeito de oxigenar.
\section{Oxigenar}
\begin{itemize}
\item {fónica:csi}
\end{itemize}
\begin{itemize}
\item {Grp. gram.:v. t.}
\end{itemize}
Combinar com o oxigênio; oxidar.
\section{Oxigenável}
\begin{itemize}
\item {fónica:csi}
\end{itemize}
\begin{itemize}
\item {Grp. gram.:adj.}
\end{itemize}
Que se póde oxigenar.
\section{Oxigenífero}
\begin{itemize}
\item {fónica:csi}
\end{itemize}
\begin{itemize}
\item {Grp. gram.:adj.}
\end{itemize}
\begin{itemize}
\item {Proveniência:(De \textunderscore oxigênio\textunderscore  + lat. \textunderscore ferre\textunderscore )}
\end{itemize}
Que tem oxigênio: \textunderscore os glóbulos do sangue são oxigeníferos\textunderscore .
\section{Oxigênio}
\begin{itemize}
\item {fónica:csi}
\end{itemize}
\begin{itemize}
\item {Grp. gram.:m.}
\end{itemize}
\begin{itemize}
\item {Proveniência:(Do gr. \textunderscore oxus\textunderscore  + \textunderscore genos\textunderscore )}
\end{itemize}
Gás simples, que faz parte da atmosfera e que sustenta a respiração e a combustão.
\section{Oxigeusia}
\begin{itemize}
\item {fónica:csi}
\end{itemize}
\begin{itemize}
\item {Grp. gram.:f.}
\end{itemize}
Excessivo desenvolvimento do sentido do gôsto.
\section{Oxígona}
\begin{itemize}
\item {fónica:csi}
\end{itemize}
\begin{itemize}
\item {Grp. gram.:f.}
\end{itemize}
Gênero de insectos coleópteros.
(Cp. \textunderscore oxígono\textunderscore )
\section{Oxigónia}
\begin{itemize}
\item {fónica:csi}
\end{itemize}
\begin{itemize}
\item {Grp. gram.:f.}
\end{itemize}
Gênero de insectos coleópteros.
(Cp. \textunderscore oxígono\textunderscore )
\section{Oxígono}
\begin{itemize}
\item {fónica:csi}
\end{itemize}
\begin{itemize}
\item {Grp. gram.:adj.}
\end{itemize}
\begin{itemize}
\item {Utilização:Geom.}
\end{itemize}
\begin{itemize}
\item {Utilização:Zool.}
\end{itemize}
\begin{itemize}
\item {Proveniência:(Do gr. \textunderscore oxus\textunderscore , agudo, e \textunderscore gonos\textunderscore , ângulo)}
\end{itemize}
O mesmo que \textunderscore acutângulo\textunderscore .
Anguloso, (falando-se de conchas).
\section{Oxigráfide}
\begin{itemize}
\item {fónica:csi}
\end{itemize}
\begin{itemize}
\item {Grp. gram.:f.}
\end{itemize}
Gênero de plantas ranunculáceas.
\section{Oxiídrico}
\begin{itemize}
\item {fónica:csi}
\end{itemize}
\begin{itemize}
\item {Grp. gram.:adj.}
\end{itemize}
Diz-se de um sistema de iluminação, em que entra oxigênio. Cf. \textunderscore Jorn.-do-Comm.\textunderscore , do Rio, de 27-V-902.
\section{Oxílito}
\begin{itemize}
\item {fónica:csi}
\end{itemize}
\begin{itemize}
\item {Grp. gram.:m.}
\end{itemize}
Pastilha de oxigênio, inventada em 1902 pelo Dr. George Jaubert. Cf. \textunderscore Jorn.-do-Comm.\textunderscore , do Rio, de 27-V-902.
\section{Oxilóbio}
\begin{itemize}
\item {fónica:csi}
\end{itemize}
\begin{itemize}
\item {Grp. gram.:m.}
\end{itemize}
Gênero de plantas leguminosas.
\section{Oxímaco}
\begin{itemize}
\item {fónica:csi}
\end{itemize}
\begin{itemize}
\item {Grp. gram.:m.}
\end{itemize}
Ave de rapina, de bico curvo e negro.
\section{Oximalva}
\begin{itemize}
\item {fónica:csi}
\end{itemize}
\begin{itemize}
\item {Grp. gram.:f.}
\end{itemize}
\begin{itemize}
\item {Utilização:Bot.}
\end{itemize}
Espécie de azêda ou labaça da Guiné.
\section{Oximanganato}
\begin{itemize}
\item {fónica:csi}
\end{itemize}
\begin{itemize}
\item {Grp. gram.:m.}
\end{itemize}
\begin{itemize}
\item {Utilização:Chím.}
\end{itemize}
Sal, resultante da combinação do ácido oximangânico com uma base.
\section{Oximangânico}
\begin{itemize}
\item {fónica:csi}
\end{itemize}
\begin{itemize}
\item {Grp. gram.:adj.}
\end{itemize}
Diz-se de um ácido, resultante da combinação do manganés com o oxygenio.
\section{Oximel}
\begin{itemize}
\item {fónica:csi}
\end{itemize}
\begin{itemize}
\item {Grp. gram.:m.}
\end{itemize}
\begin{itemize}
\item {Proveniência:(Do gr. \textunderscore oxus\textunderscore  + \textunderscore meli\textunderscore )}
\end{itemize}
Mistura de água, vinagre e mel.
\section{Oxímero}
\begin{itemize}
\item {fónica:csi}
\end{itemize}
\begin{itemize}
\item {Grp. gram.:m.}
\end{itemize}
\begin{itemize}
\item {Proveniência:(Do gr. \textunderscore oxus\textunderscore  + \textunderscore meros\textunderscore )}
\end{itemize}
Gênero de insectos coleópteros, longicórneos.
\section{Oximetria}
\begin{itemize}
\item {fónica:csi}
\end{itemize}
\begin{itemize}
\item {Grp. gram.:f.}
\end{itemize}
\begin{itemize}
\item {Proveniência:(Do gr. \textunderscore oxus\textunderscore , ácido, e \textunderscore metron\textunderscore , medida)}
\end{itemize}
Processo, para avaliar a quantidade de ácido livre ou de sal ácido, contido numa substância qualquer.
\section{Oxina}
\begin{itemize}
\item {fónica:csi}
\end{itemize}
\begin{itemize}
\item {Grp. gram.:f.}
\end{itemize}
\begin{itemize}
\item {Proveniência:(Do gr. \textunderscore oxus\textunderscore  + \textunderscore oinos\textunderscore )}
\end{itemize}
Vinho azêdo, mas ainda não convertido completamente em vinagre.
\section{Oxiope}
\begin{itemize}
\item {fónica:csi}
\end{itemize}
\begin{itemize}
\item {Grp. gram.:m.}
\end{itemize}
\begin{itemize}
\item {Proveniência:(Do gr. \textunderscore oxus\textunderscore  + \textunderscore ops\textunderscore )}
\end{itemize}
Gênero de insectos coleópteros tetrâmeros.
\section{Oxiopia}
\begin{itemize}
\item {fónica:csi}
\end{itemize}
\begin{itemize}
\item {Grp. gram.:f.}
\end{itemize}
\begin{itemize}
\item {Utilização:Med.}
\end{itemize}
\begin{itemize}
\item {Proveniência:(Do gr. \textunderscore oxus\textunderscore  + \textunderscore ops\textunderscore )}
\end{itemize}
Vista penetrante.
Faculdade de vêr os objectos a grande distância.
\section{Oxiosmia}
\begin{itemize}
\item {Grp. gram.:f.}
\end{itemize}
\begin{itemize}
\item {Proveniência:(Do gr. \textunderscore oxus\textunderscore  + \textunderscore osme\textunderscore )}
\end{itemize}
Sensibilidade extrema do olfacto. Cf. R. Galvão, \textunderscore Vocab.\textunderscore 
\section{Oxipétalo}
\begin{itemize}
\item {fónica:csi}
\end{itemize}
\begin{itemize}
\item {Grp. gram.:m.}
\end{itemize}
\begin{itemize}
\item {Proveniência:(Do gr. \textunderscore oxus\textunderscore  + \textunderscore petalon\textunderscore )}
\end{itemize}
Gênero de plantas asclepíadáceas.
\section{Oxóleo}
\begin{itemize}
\item {fónica:csó}
\end{itemize}
\begin{itemize}
\item {Grp. gram.:m.}
\end{itemize}
\begin{itemize}
\item {Proveniência:(De \textunderscore oxus\textunderscore  gr. + \textunderscore óleo\textunderscore )}
\end{itemize}
Operação pharmacêutica, em que o vinagre é o excipiente.
\section{Oxura}
\begin{itemize}
\item {fónica:csu}
\end{itemize}
\begin{itemize}
\item {Grp. gram.:f.}
\end{itemize}
\begin{itemize}
\item {Proveniência:(Do gr. \textunderscore oxus\textunderscore  + \textunderscore oura\textunderscore )}
\end{itemize}
Gênero de insectos coleópteros heterómeros.
Cp. \textunderscore oxyúro\textunderscore .
\section{Oxyanthero}
\begin{itemize}
\item {fónica:csi}
\end{itemize}
\begin{itemize}
\item {Grp. gram.:f.}
\end{itemize}
\begin{itemize}
\item {Proveniência:(Do gr. \textunderscore oxus\textunderscore  + \textunderscore antheros\textunderscore )}
\end{itemize}
Gênero de orchídeas.
\section{Oxyantho}
\begin{itemize}
\item {fónica:csi}
\end{itemize}
\begin{itemize}
\item {Grp. gram.:m.}
\end{itemize}
\begin{itemize}
\item {Proveniência:(Do gr. \textunderscore oxus\textunderscore  + \textunderscore anthos\textunderscore )}
\end{itemize}
Gênero de plantas rubiáceas.
\section{Oxýbapho}
\begin{itemize}
\item {Grp. gram.:m.}
\end{itemize}
\begin{itemize}
\item {Proveniência:(Lat. \textunderscore oxybaphus\textunderscore )}
\end{itemize}
Vasilha para vinagre, entre os antigos Gregos.
Medida de capacidade entre os Romanos, equivalente a 15 drachmas.
Gênero de plantas nictagíneas.
\section{Oxýbase}
\begin{itemize}
\item {fónica:csi}
\end{itemize}
\begin{itemize}
\item {Grp. gram.:m.}
\end{itemize}
\begin{itemize}
\item {Utilização:Chím.}
\end{itemize}
\begin{itemize}
\item {Proveniência:(Do gr. \textunderscore oxus\textunderscore  + \textunderscore basis\textunderscore )}
\end{itemize}
Óxydo, que fórma a base das combinações em que entra.
\section{Oxybásico}
\begin{itemize}
\item {fónica:csi}
\end{itemize}
\begin{itemize}
\item {Grp. gram.:adj.}
\end{itemize}
\begin{itemize}
\item {Utilização:Chím.}
\end{itemize}
\begin{itemize}
\item {Proveniência:(De \textunderscore oxýbase\textunderscore )}
\end{itemize}
Diz-se do sal, que tem por base um óxydo.
\section{Oxybrácteo}
\begin{itemize}
\item {fónica:csi}
\end{itemize}
\begin{itemize}
\item {Grp. gram.:adj.}
\end{itemize}
\begin{itemize}
\item {Utilização:Bot.}
\end{itemize}
\begin{itemize}
\item {Proveniência:(Do gr. \textunderscore oxus\textunderscore  + lat. \textunderscore bractea\textunderscore )}
\end{itemize}
Que tem brácteas agudas.
\section{Oxybrometo}
\begin{itemize}
\item {fónica:csi}
\end{itemize}
\begin{itemize}
\item {Grp. gram.:m.}
\end{itemize}
\begin{itemize}
\item {Utilização:Chím.}
\end{itemize}
Combinação de um brometo com um óxydo.
\section{Oxybutýrico}
\begin{itemize}
\item {fónica:csi}
\end{itemize}
\begin{itemize}
\item {Grp. gram.:adj.}
\end{itemize}
Diz-se de um ácido, que se encontra nalgumas urinas diabéticas e que reduz os saes de cobre.
\section{Oxycárbico}
\begin{itemize}
\item {fónica:csi}
\end{itemize}
\begin{itemize}
\item {Grp. gram.:adj.}
\end{itemize}
\begin{itemize}
\item {Utilização:Chím.}
\end{itemize}
\begin{itemize}
\item {Proveniência:(Do gr. \textunderscore oxus\textunderscore  + lat. \textunderscore carbo\textunderscore )}
\end{itemize}
Que contém oxygênio e carvão.
\section{Oxycedro}
\begin{itemize}
\item {fónica:csi}
\end{itemize}
\begin{itemize}
\item {Grp. gram.:m.}
\end{itemize}
\begin{itemize}
\item {Proveniência:(Do gr. \textunderscore oxus\textunderscore  + \textunderscore kedros\textunderscore )}
\end{itemize}
Árvore conífera do sul da Europa, espécie de zimbro, cuja madeira produz, pela destillação, um óleo conhecido em Veterinária por \textunderscore óleo de zimbro\textunderscore . Cf. \textunderscore Bibl. da G. do Campo\textunderscore , 274.
\section{Oxycéphala}
\begin{itemize}
\item {fónica:csi}
\end{itemize}
\begin{itemize}
\item {Grp. gram.:f.}
\end{itemize}
Grande serpente das regiões tropicaes.
(Cp. \textunderscore oxycéphalo\textunderscore )
\section{Oxycephalia}
\begin{itemize}
\item {fónica:csi}
\end{itemize}
\begin{itemize}
\item {Grp. gram.:f.}
\end{itemize}
Estado ou qualidade de oxycéphalo.
\section{Oxycéphalo}
\begin{itemize}
\item {fónica:csi}
\end{itemize}
\begin{itemize}
\item {Grp. gram.:m.}
\end{itemize}
\begin{itemize}
\item {Grp. gram.:Adj.}
\end{itemize}
\begin{itemize}
\item {Proveniência:(Do gr. \textunderscore oxus\textunderscore  + \textunderscore kephale\textunderscore )}
\end{itemize}
Gênero de crustáceos amphípodes.
Gênero de insectos coleópteros.
O mesmo que \textunderscore acrocéphalo\textunderscore .
\section{Oxýcera}
\begin{itemize}
\item {fónica:csi}
\end{itemize}
\begin{itemize}
\item {Grp. gram.:f.}
\end{itemize}
\begin{itemize}
\item {Proveniência:(Do gr. \textunderscore oxus\textunderscore  + \textunderscore keras\textunderscore )}
\end{itemize}
Gênero de insectos dípteros.
\section{Oxychlorato}
\begin{itemize}
\item {fónica:csi}
\end{itemize}
\begin{itemize}
\item {Grp. gram.:m.}
\end{itemize}
\begin{itemize}
\item {Utilização:Chím.}
\end{itemize}
Sal, resultante da combinação do ácido oxychlórico com uma base.
\section{Oxychlórico}
\begin{itemize}
\item {fónica:csi}
\end{itemize}
\begin{itemize}
\item {Grp. gram.:adj.}
\end{itemize}
Diz-se do ácido, que constitue o maior grau de oxygenação do chloro.
\section{Oxychloro-carbónico}
\begin{itemize}
\item {Grp. gram.:adj.}
\end{itemize}
Diz-se do ácido, produzido pelo chloro e o carbone com o oxygênio.
\section{Oxycrato}
\begin{itemize}
\item {fónica:csi}
\end{itemize}
\begin{itemize}
\item {Grp. gram.:m.}
\end{itemize}
\begin{itemize}
\item {Proveniência:(Gr. \textunderscore oxukraton\textunderscore )}
\end{itemize}
Mistura de vinagre e água em certa proporção.
\section{Oxydabilidade}
\begin{itemize}
\item {fónica:csi}
\end{itemize}
\begin{itemize}
\item {Grp. gram.:f.}
\end{itemize}
Qualidade do que é oxydável.
\section{Oxydação}
\begin{itemize}
\item {fónica:csi}
\end{itemize}
\begin{itemize}
\item {Grp. gram.:f.}
\end{itemize}
Acto de oxydar.
Oxygenação.
\section{Oxydante}
\begin{itemize}
\item {fónica:csi}
\end{itemize}
\begin{itemize}
\item {Grp. gram.:adj.}
\end{itemize}
Que tem a propriedade de oxydar.
\section{Oxydar}
\begin{itemize}
\item {fónica:csi}
\end{itemize}
\begin{itemize}
\item {Grp. gram.:v.}
\end{itemize}
\begin{itemize}
\item {Utilização:t. Chím.}
\end{itemize}
\begin{itemize}
\item {Grp. gram.:V. p.}
\end{itemize}
\begin{itemize}
\item {Utilização:Fig.}
\end{itemize}
\begin{itemize}
\item {Proveniência:(De \textunderscore óxydo\textunderscore )}
\end{itemize}
Converter em óxydo.
Combinar com o oxygênio.
Oxygenar-se.
Enferrujar-se.
\section{Oxýdase}
\begin{itemize}
\item {fónica:csi}
\end{itemize}
\begin{itemize}
\item {Grp. gram.:f.}
\end{itemize}
\begin{itemize}
\item {Proveniência:(De \textunderscore óxydo\textunderscore )}
\end{itemize}
Fermento oxydante.
\section{Oxydável}
\begin{itemize}
\item {fónica:csi}
\end{itemize}
\begin{itemize}
\item {Grp. gram.:adj.}
\end{itemize}
Que se pode oxydar.
\section{Oxýdio}
\begin{itemize}
\item {fónica:csi}
\end{itemize}
\begin{itemize}
\item {Grp. gram.:m.}
\end{itemize}
\begin{itemize}
\item {Proveniência:(Do gr. \textunderscore oxus\textunderscore )}
\end{itemize}
Gênero de plantas leguminosas.
\section{Óxydo}
\begin{itemize}
\item {fónica:csi}
\end{itemize}
\begin{itemize}
\item {Grp. gram.:m.}
\end{itemize}
\begin{itemize}
\item {Utilização:Chím.}
\end{itemize}
\begin{itemize}
\item {Proveniência:(Do gr. \textunderscore oxus\textunderscore )}
\end{itemize}
Composto neutro de oxygênio e de um metallóide ou metal.
\section{Oxydonte}
\begin{itemize}
\item {fónica:csi}
\end{itemize}
\begin{itemize}
\item {Grp. gram.:m.}
\end{itemize}
\begin{itemize}
\item {Proveniência:(Do gr. \textunderscore oxus\textunderscore , agudo, e \textunderscore odous\textunderscore , dente)}
\end{itemize}
Gênero de plantas herbáceas da Nova-Holanda.
\section{Oxydulado}
\begin{itemize}
\item {fónica:csi}
\end{itemize}
\begin{itemize}
\item {Grp. gram.:adj.}
\end{itemize}
Que passou ao estado de oxýdulo.
\section{Oxýdulo}
\begin{itemize}
\item {fónica:csi}
\end{itemize}
\begin{itemize}
\item {Grp. gram.:m.}
\end{itemize}
\begin{itemize}
\item {Utilização:Chím.}
\end{itemize}
\begin{itemize}
\item {Proveniência:(De \textunderscore óxydo\textunderscore )}
\end{itemize}
Primeiro grau inferior da oxydação de um corpo.
\section{Oxyethérico}
\begin{itemize}
\item {fónica:csi}
\end{itemize}
\begin{itemize}
\item {Grp. gram.:adj.}
\end{itemize}
Diz-se das luzes, produzidas por oxygênio e éther. Cf. \textunderscore Jorn.-do-Comm.\textunderscore , do Rio, de 27-V-902.
\section{Oxýgala}
\begin{itemize}
\item {fónica:csi}
\end{itemize}
\begin{itemize}
\item {Grp. gram.:f.}
\end{itemize}
\begin{itemize}
\item {Proveniência:(Do gr. \textunderscore oxus\textunderscore  + \textunderscore gala\textunderscore )}
\end{itemize}
Leite azêdo.
\section{Oxygenabilidade}
\begin{itemize}
\item {fónica:csi}
\end{itemize}
\begin{itemize}
\item {Grp. gram.:f.}
\end{itemize}
\begin{itemize}
\item {Utilização:Chím.}
\end{itemize}
Qualidade de oxygenável.
\section{Oxygenação}
\begin{itemize}
\item {fónica:csi}
\end{itemize}
\begin{itemize}
\item {Grp. gram.:f.}
\end{itemize}
Acto ou effeito de oxygenar.
\section{Oxygenar}
\begin{itemize}
\item {fónica:csi}
\end{itemize}
\begin{itemize}
\item {Grp. gram.:v. t.}
\end{itemize}
Combinar com o oxygênio; oxydar.
\section{Oxygenável}
\begin{itemize}
\item {fónica:csi}
\end{itemize}
\begin{itemize}
\item {Grp. gram.:adj.}
\end{itemize}
Que se póde oxygenar.
\section{Oxygenífero}
\begin{itemize}
\item {fónica:csi}
\end{itemize}
\begin{itemize}
\item {Grp. gram.:adj.}
\end{itemize}
\begin{itemize}
\item {Proveniência:(De \textunderscore oxygênio\textunderscore  + lat. \textunderscore ferre\textunderscore )}
\end{itemize}
Que tem oxygênio: \textunderscore os glóbulos do sangue são oxygeníferos\textunderscore .
\section{Oxygênio}
\begin{itemize}
\item {fónica:csi}
\end{itemize}
\begin{itemize}
\item {Grp. gram.:m.}
\end{itemize}
\begin{itemize}
\item {Proveniência:(Do gr. \textunderscore oxus\textunderscore  + \textunderscore genos\textunderscore )}
\end{itemize}
Gás simples, que faz parte da atmosphera e que sustenta a respiração e a combustão.
\section{Oxygeusia}
\begin{itemize}
\item {fónica:csi}
\end{itemize}
\begin{itemize}
\item {Grp. gram.:f.}
\end{itemize}
Excessivo desenvolvimento do sentido do gôsto.
\section{Oxýgona}
\begin{itemize}
\item {fónica:csi}
\end{itemize}
\begin{itemize}
\item {Grp. gram.:f.}
\end{itemize}
Gênero de insectos coleópteros.
(Cp. \textunderscore oxýgono\textunderscore )
\section{Oxygónia}
\begin{itemize}
\item {fónica:csi}
\end{itemize}
\begin{itemize}
\item {Grp. gram.:f.}
\end{itemize}
Gênero de insectos coleópteros.
(Cp. \textunderscore oxýgono\textunderscore )
\section{Oxýgono}
\begin{itemize}
\item {fónica:csi}
\end{itemize}
\begin{itemize}
\item {Grp. gram.:adj.}
\end{itemize}
\begin{itemize}
\item {Utilização:Geom.}
\end{itemize}
\begin{itemize}
\item {Utilização:Zool.}
\end{itemize}
\begin{itemize}
\item {Proveniência:(Do gr. \textunderscore oxus\textunderscore , agudo, e \textunderscore gonos\textunderscore , ângulo)}
\end{itemize}
O mesmo que \textunderscore acutângulo\textunderscore .
Anguloso, (falando-se de conchas).
\section{Oxygráphide}
\begin{itemize}
\item {fónica:csi}
\end{itemize}
\begin{itemize}
\item {Grp. gram.:f.}
\end{itemize}
Gênero de plantas ranunculáceas.
\section{Oxyhýdrico}
\begin{itemize}
\item {fónica:csi}
\end{itemize}
\begin{itemize}
\item {Grp. gram.:adj.}
\end{itemize}
Diz-se de um systema de illuminação, em que entra oxygênio. Cf. \textunderscore Jorn.-do-Comm.\textunderscore , do Rio, de 27-V-902.
\section{Oxýlitho}
\begin{itemize}
\item {fónica:csi}
\end{itemize}
\begin{itemize}
\item {Grp. gram.:m.}
\end{itemize}
Pastilha de oxygênio, inventada em 1902 pelo Dr. George Jaubert. Cf. \textunderscore Jorn.-do-Comm.\textunderscore , do Rio, de 27-V-902.
\section{Oxylóbio}
\begin{itemize}
\item {fónica:csi}
\end{itemize}
\begin{itemize}
\item {Grp. gram.:m.}
\end{itemize}
Gênero de plantas leguminosas.
\section{Oxýmaco}
\begin{itemize}
\item {fónica:csi}
\end{itemize}
\begin{itemize}
\item {Grp. gram.:m.}
\end{itemize}
Ave de rapina, de bico curvo e negro.
\section{Oxymalva}
\begin{itemize}
\item {fónica:csi}
\end{itemize}
\begin{itemize}
\item {Grp. gram.:f.}
\end{itemize}
\begin{itemize}
\item {Utilização:Bot.}
\end{itemize}
Espécie de azêda ou labaça da Guiné.
\section{Oxymanganato}
\begin{itemize}
\item {fónica:csi}
\end{itemize}
\begin{itemize}
\item {Grp. gram.:m.}
\end{itemize}
\begin{itemize}
\item {Utilização:Chím.}
\end{itemize}
Sal, resultante da combinação do ácido oxymangânico com uma base.
\section{Oxymangânico}
\begin{itemize}
\item {fónica:csi}
\end{itemize}
\begin{itemize}
\item {Grp. gram.:adj.}
\end{itemize}
Diz-se de um ácido, resultante da combinação do manganés com o oxygenio.
\section{Oxymel}
\begin{itemize}
\item {fónica:csi}
\end{itemize}
\begin{itemize}
\item {Grp. gram.:m.}
\end{itemize}
\begin{itemize}
\item {Proveniência:(Do gr. \textunderscore oxus\textunderscore  + \textunderscore meli\textunderscore )}
\end{itemize}
Mistura de água, vinagre e mel.
\section{Oxýmero}
\begin{itemize}
\item {fónica:csi}
\end{itemize}
\begin{itemize}
\item {Grp. gram.:m.}
\end{itemize}
\begin{itemize}
\item {Proveniência:(Do gr. \textunderscore oxus\textunderscore  + \textunderscore meros\textunderscore )}
\end{itemize}
Gênero de insectos coleópteros, longicórneos.
\section{Oxymetria}
\begin{itemize}
\item {fónica:csi}
\end{itemize}
\begin{itemize}
\item {Grp. gram.:f.}
\end{itemize}
\begin{itemize}
\item {Proveniência:(Do gr. \textunderscore oxus\textunderscore , ácido, e \textunderscore metron\textunderscore , medida)}
\end{itemize}
Processo, para avaliar a quantidade de ácido livre ou de sal ácido, contido numa substância qualquer.
\section{Oxyna}
\begin{itemize}
\item {fónica:csi}
\end{itemize}
\begin{itemize}
\item {Grp. gram.:f.}
\end{itemize}
\begin{itemize}
\item {Proveniência:(Do gr. \textunderscore oxus\textunderscore  + \textunderscore oinos\textunderscore )}
\end{itemize}
Vinho azêdo, mas ainda não convertido completamente em vinagre.
\section{Oxyope}
\begin{itemize}
\item {fónica:csi}
\end{itemize}
\begin{itemize}
\item {Grp. gram.:m.}
\end{itemize}
\begin{itemize}
\item {Proveniência:(Do gr. \textunderscore oxus\textunderscore  + \textunderscore ops\textunderscore )}
\end{itemize}
Gênero de insectos coleópteros tetrâmeros.
\section{Oxyopia}
\begin{itemize}
\item {fónica:csi}
\end{itemize}
\begin{itemize}
\item {Grp. gram.:f.}
\end{itemize}
\begin{itemize}
\item {Utilização:Med.}
\end{itemize}
\begin{itemize}
\item {Proveniência:(Do gr. \textunderscore oxus\textunderscore  + \textunderscore ops\textunderscore )}
\end{itemize}
Vista penetrante.
Faculdade de vêr os objectos a grande distância.
\section{Oxyosmia}
\begin{itemize}
\item {Grp. gram.:f.}
\end{itemize}
\begin{itemize}
\item {Proveniência:(Do gr. \textunderscore oxus\textunderscore  + \textunderscore osme\textunderscore )}
\end{itemize}
Sensibilidade extrema do olfacto. Cf. R. Galvão, \textunderscore Vocab.\textunderscore 
\section{Oxypétalo}
\begin{itemize}
\item {fónica:csi}
\end{itemize}
\begin{itemize}
\item {Grp. gram.:m.}
\end{itemize}
\begin{itemize}
\item {Proveniência:(Do gr. \textunderscore oxus\textunderscore  + \textunderscore petalon\textunderscore )}
\end{itemize}
Gênero de plantas asclepíadáceas.
\section{Oite}
\begin{itemize}
\item {Grp. gram.:adv.}
\end{itemize}
\begin{itemize}
\item {Utilização:Ant.}
\end{itemize}
O mesmo que \textunderscore ontem\textunderscore .
\section{Oxifonia}
\begin{itemize}
\item {fónica:csi}
\end{itemize}
\begin{itemize}
\item {Grp. gram.:f.}
\end{itemize}
\begin{itemize}
\item {Proveniência:(Do gr. \textunderscore oxus\textunderscore  + \textunderscore phone\textunderscore )}
\end{itemize}
Voz aguda.
\section{Oxípilo}
\begin{itemize}
\item {fónica:csi}
\end{itemize}
\begin{itemize}
\item {Grp. gram.:m.}
\end{itemize}
\begin{itemize}
\item {Proveniência:(Do gr. \textunderscore oxus\textunderscore  + lat. \textunderscore pilus\textunderscore )}
\end{itemize}
Gênero de insectos ortópteros.
\section{Oxíptero}
\begin{itemize}
\item {fónica:csi}
\end{itemize}
\begin{itemize}
\item {Grp. gram.:m.}
\end{itemize}
\begin{itemize}
\item {Proveniência:(Do gr. \textunderscore oxus\textunderscore  + \textunderscore pteron\textunderscore )}
\end{itemize}
Gênero de mamíferos, a que pertence o golfinho.
\section{Oxirrânfide}
\begin{itemize}
\item {fónica:csi}
\end{itemize}
\begin{itemize}
\item {Grp. gram.:f.}
\end{itemize}
Gênero de plantas leguminosas.
\section{Oxirregmia}
\begin{itemize}
\item {fónica:csi}
\end{itemize}
\begin{itemize}
\item {Grp. gram.:f.}
\end{itemize}
\begin{itemize}
\item {Utilização:Med.}
\end{itemize}
\begin{itemize}
\item {Proveniência:(Gr. \textunderscore oxuregnia\textunderscore )}
\end{itemize}
Eructação azêda do estômago.
\section{Oxíria}
\begin{itemize}
\item {fónica:csi}
\end{itemize}
\begin{itemize}
\item {Grp. gram.:f.}
\end{itemize}
Gênero de plantas poligóneas.
\section{Oxirrinco}
\begin{itemize}
\item {fónica:csi}
\end{itemize}
\begin{itemize}
\item {Grp. gram.:m.}
\end{itemize}
\begin{itemize}
\item {Grp. gram.:Pl.}
\end{itemize}
Gênero de aves anisodáctilas.
Gênero de insectos coleópteros tetrâmeros.
Família de crustáceos decápodes.
\section{Oxirrodino}
\begin{itemize}
\item {fónica:csi}
\end{itemize}
\begin{itemize}
\item {Grp. gram.:m.}
\end{itemize}
\begin{itemize}
\item {Proveniência:(Do gr. \textunderscore oxus\textunderscore  + \textunderscore rhodon\textunderscore )}
\end{itemize}
Vinagre rosado, que se usava antigamente em farmácia.
\section{Oxíspora}
\begin{itemize}
\item {fónica:csi}
\end{itemize}
\begin{itemize}
\item {Grp. gram.:f.}
\end{itemize}
\begin{itemize}
\item {Proveniência:(Do gr. \textunderscore oxus\textunderscore  + \textunderscore poros\textunderscore )}
\end{itemize}
Gênero de plantas melastomáceas.
\section{Oxissácaro}
\begin{itemize}
\item {fónica:csi}
\end{itemize}
\begin{itemize}
\item {Grp. gram.:m.}
\end{itemize}
\begin{itemize}
\item {Utilização:Pharm.}
\end{itemize}
\begin{itemize}
\item {Proveniência:(Do gr. \textunderscore oxus\textunderscore  + \textunderscore sakkharon\textunderscore )}
\end{itemize}
Mistura de açúcar e vinagre.
\section{Oxissal}
\begin{itemize}
\item {fónica:csi}
\end{itemize}
\begin{itemize}
\item {Grp. gram.:m.}
\end{itemize}
\begin{itemize}
\item {Utilização:Chím.}
\end{itemize}
Sal em cuja base e em cujo ácido entra o oxigênio.
\section{Oxistelma}
\begin{itemize}
\item {fónica:csis}
\end{itemize}
\begin{itemize}
\item {Grp. gram.:f.}
\end{itemize}
Gênero de plantas asclepiadaceas do Oriente.
\section{Oxistermo}
\begin{itemize}
\item {fónica:csis}
\end{itemize}
\begin{itemize}
\item {Grp. gram.:m.}
\end{itemize}
Designação de três gêneros de insectos, pertencentes a famílias diversas.
\section{Oxistilo}
\begin{itemize}
\item {fónica:csis}
\end{itemize}
\begin{itemize}
\item {Grp. gram.:adj.}
\end{itemize}
\begin{itemize}
\item {Proveniência:(Do gr. \textunderscore oxus\textunderscore  + \textunderscore stulos\textunderscore )}
\end{itemize}
Diz-se das conchas, cuja columela é aguda.
\section{Oxistofilo}
\begin{itemize}
\item {fónica:csis}
\end{itemize}
\begin{itemize}
\item {Grp. gram.:m.}
\end{itemize}
Gênero de orquídeas.
\section{Oxístomo}
\begin{itemize}
\item {fónica:csis}
\end{itemize}
\begin{itemize}
\item {Grp. gram.:m.}
\end{itemize}
\begin{itemize}
\item {Grp. gram.:Pl.}
\end{itemize}
\begin{itemize}
\item {Proveniência:(Do gr. \textunderscore oxus\textunderscore  + \textunderscore stoma\textunderscore )}
\end{itemize}
Gênero de insectos coleópteros pentâmeros.
Família de crustáceos decápodes.
\section{Oxitártaro}
\begin{itemize}
\item {fónica:csi}
\end{itemize}
\begin{itemize}
\item {Grp. gram.:m.}
\end{itemize}
\begin{itemize}
\item {Utilização:Chím.}
\end{itemize}
Acetato de potassa.
\section{Oxitónico}
\begin{itemize}
\item {fónica:csi}
\end{itemize}
\begin{itemize}
\item {Grp. gram.:adj.}
\end{itemize}
O mesmo que \textunderscore oxítono\textunderscore .
\section{Oxítono}
\begin{itemize}
\item {fónica:csi}
\end{itemize}
\begin{itemize}
\item {Grp. gram.:adj.}
\end{itemize}
\begin{itemize}
\item {Utilização:Gram.}
\end{itemize}
\begin{itemize}
\item {Grp. gram.:M.}
\end{itemize}
\begin{itemize}
\item {Proveniência:(Do gr. \textunderscore oxus\textunderscore  + \textunderscore tonos\textunderscore )}
\end{itemize}
Diz-se dos vocábulos, cuja sílaba tónica é a última, e que, mais vulgarmente, se chamam agudos.
Vocabulo oxítono.
\section{Oxiúra}
\begin{itemize}
\item {fónica:csi}
\end{itemize}
\begin{itemize}
\item {Grp. gram.:f.}
\end{itemize}
(V.oxiúro)
\section{Oxiúride}
\begin{itemize}
\item {fónica:csi}
\end{itemize}
\begin{itemize}
\item {Grp. gram.:f.}
\end{itemize}
O mesmo que \textunderscore oxiúro\textunderscore .
\section{Oxiúro}
\begin{itemize}
\item {fónica:csi}
\end{itemize}
\begin{itemize}
\item {Grp. gram.:m.}
\end{itemize}
\begin{itemize}
\item {Proveniência:(Do gr. \textunderscore oxus\textunderscore , agudo, e \textunderscore oura\textunderscore , cauda)}
\end{itemize}
Helminto, que vive na parte inferior do intestino recto.
\section{Oxiurose}
\begin{itemize}
\item {fónica:csi-u}
\end{itemize}
\begin{itemize}
\item {Grp. gram.:f.}
\end{itemize}
Doença, produzida por oxiúros.
\section{Oxyphonia}
\begin{itemize}
\item {fónica:csi}
\end{itemize}
\begin{itemize}
\item {Grp. gram.:f.}
\end{itemize}
\begin{itemize}
\item {Proveniência:(Do gr. \textunderscore oxus\textunderscore  + \textunderscore phone\textunderscore )}
\end{itemize}
Voz aguda.
\section{Oxýpilo}
\begin{itemize}
\item {fónica:csi}
\end{itemize}
\begin{itemize}
\item {Grp. gram.:m.}
\end{itemize}
\begin{itemize}
\item {Proveniência:(Do gr. \textunderscore oxus\textunderscore  + lat. \textunderscore pilus\textunderscore )}
\end{itemize}
Gênero de insectos orthópteros.
\section{Oxýptero}
\begin{itemize}
\item {fónica:csi}
\end{itemize}
\begin{itemize}
\item {Grp. gram.:m.}
\end{itemize}
\begin{itemize}
\item {Proveniência:(Do gr. \textunderscore oxus\textunderscore  + \textunderscore pteron\textunderscore )}
\end{itemize}
Gênero de mammíferos, a que pertence o golfinho.
\section{Oxyrâmphide}
\begin{itemize}
\item {fónica:csi,ran}
\end{itemize}
\begin{itemize}
\item {Grp. gram.:f.}
\end{itemize}
Gênero de plantas leguminosas.
\section{Oxyregmia}
\begin{itemize}
\item {fónica:re,csi}
\end{itemize}
\begin{itemize}
\item {Grp. gram.:f.}
\end{itemize}
\begin{itemize}
\item {Utilização:Med.}
\end{itemize}
\begin{itemize}
\item {Proveniência:(Gr. \textunderscore oxuregnia\textunderscore )}
\end{itemize}
Eructação azêda do estômago.
\section{Oxýria}
\begin{itemize}
\item {fónica:csi}
\end{itemize}
\begin{itemize}
\item {Grp. gram.:f.}
\end{itemize}
Gênero de plantas polygóneas.
\section{Oxyrrhincho}
\begin{itemize}
\item {fónica:csi,co}
\end{itemize}
\begin{itemize}
\item {Grp. gram.:f.}
\end{itemize}
Gênero de plantas polygóneas.
\section{Oxyrrhodino}
\begin{itemize}
\item {fónica:csi}
\end{itemize}
\begin{itemize}
\item {Grp. gram.:m.}
\end{itemize}
\begin{itemize}
\item {Proveniência:(Do gr. \textunderscore oxus\textunderscore  + \textunderscore rhodon\textunderscore )}
\end{itemize}
Vinagre rosado, que se usava antigamente em pharmácia.
\section{Oxysáccharo}
\begin{itemize}
\item {fónica:csi,saca}
\end{itemize}
\begin{itemize}
\item {Grp. gram.:m.}
\end{itemize}
\begin{itemize}
\item {Utilização:Pharm.}
\end{itemize}
\begin{itemize}
\item {Proveniência:(Do gr. \textunderscore oxus\textunderscore  + \textunderscore sakkharon\textunderscore )}
\end{itemize}
Mistura de açúcar e vinagre.
\section{Oxysal}
\begin{itemize}
\item {fónica:csi,sal}
\end{itemize}
\begin{itemize}
\item {Grp. gram.:m.}
\end{itemize}
\begin{itemize}
\item {Utilização:Chím.}
\end{itemize}
Sal em cuja base e em cujo ácido entra o oxygênio.
\section{Oxýspora}
\begin{itemize}
\item {fónica:csi}
\end{itemize}
\begin{itemize}
\item {Grp. gram.:f.}
\end{itemize}
\begin{itemize}
\item {Proveniência:(Do gr. \textunderscore oxus\textunderscore  + \textunderscore poros\textunderscore )}
\end{itemize}
Gênero de plantas melastomáceas.
\section{Oxystelma}
\begin{itemize}
\item {fónica:csis}
\end{itemize}
\begin{itemize}
\item {Grp. gram.:f.}
\end{itemize}
Gênero de plantas asclepiadaceas do Oriente.
\section{Oxystermo}
\begin{itemize}
\item {fónica:csis}
\end{itemize}
\begin{itemize}
\item {Grp. gram.:m.}
\end{itemize}
Designação de três gêneros de insectos, pertencentes a famílias diversas.
\section{Oxýstomo}
\begin{itemize}
\item {fónica:csis}
\end{itemize}
\begin{itemize}
\item {Grp. gram.:m.}
\end{itemize}
\begin{itemize}
\item {Grp. gram.:Pl.}
\end{itemize}
\begin{itemize}
\item {Proveniência:(Do gr. \textunderscore oxus\textunderscore  + \textunderscore stoma\textunderscore )}
\end{itemize}
Gênero de insectos coleópteros pentâmeros.
Família de crustáceos decápodes.
\section{Oxystophyllo}
\begin{itemize}
\item {fónica:csis}
\end{itemize}
\begin{itemize}
\item {Grp. gram.:m.}
\end{itemize}
Gênero de orchídeas.
\section{Oxystylo}
\begin{itemize}
\item {fónica:csis}
\end{itemize}
\begin{itemize}
\item {Grp. gram.:adj.}
\end{itemize}
\begin{itemize}
\item {Proveniência:(Do gr. \textunderscore oxus\textunderscore  + \textunderscore stulos\textunderscore )}
\end{itemize}
Diz-se das conchas, cuja columella é aguda.
\section{Oxytártaro}
\begin{itemize}
\item {fónica:csi}
\end{itemize}
\begin{itemize}
\item {Grp. gram.:m.}
\end{itemize}
\begin{itemize}
\item {Utilização:Chím.}
\end{itemize}
Acetato de potassa.
\section{Oxytónico}
\begin{itemize}
\item {fónica:csi}
\end{itemize}
\begin{itemize}
\item {Grp. gram.:adj.}
\end{itemize}
O mesmo que \textunderscore oxýtono\textunderscore .
\section{Oxýtono}
\begin{itemize}
\item {fónica:csi}
\end{itemize}
\begin{itemize}
\item {Grp. gram.:adj.}
\end{itemize}
\begin{itemize}
\item {Utilização:Gram.}
\end{itemize}
\begin{itemize}
\item {Grp. gram.:M.}
\end{itemize}
\begin{itemize}
\item {Proveniência:(Do gr. \textunderscore oxus\textunderscore  + \textunderscore tonos\textunderscore )}
\end{itemize}
Diz-se dos vocábulos, cuja sýllaba tónica é a última, e que, mais vulgarmente, se chamam agudos.
Vocabulo oxýtono.
\section{Oxyúra}
\begin{itemize}
\item {fónica:csi}
\end{itemize}
\begin{itemize}
\item {Grp. gram.:f.}
\end{itemize}
(V.oxyúro)
\section{Oxyúride}
\begin{itemize}
\item {fónica:csi}
\end{itemize}
O mesmo que \textunderscore oxyúro\textunderscore .
\section{Oxyúro}
\begin{itemize}
\item {fónica:csi}
\end{itemize}
\begin{itemize}
\item {Grp. gram.:m.}
\end{itemize}
\begin{itemize}
\item {Proveniência:(Do gr. \textunderscore oxus\textunderscore , agudo, e \textunderscore oura\textunderscore , cauda)}
\end{itemize}
Helmintho, que vive na parte inferior do intestino recto.
\section{Oxyurose}
\begin{itemize}
\item {fónica:csi-u}
\end{itemize}
\begin{itemize}
\item {Grp. gram.:f.}
\end{itemize}
Doença, produzida por oxyúros.
\section{Oyte}
\begin{itemize}
\item {Grp. gram.:adv.}
\end{itemize}
\begin{itemize}
\item {Utilização:Ant.}
\end{itemize}
O mesmo que \textunderscore ontem\textunderscore .
\section{Ozagre}
\begin{itemize}
\item {Grp. gram.:m.}
\end{itemize}
O mesmo que \textunderscore uzagre\textunderscore .
\section{Ozena}
\begin{itemize}
\item {Grp. gram.:f.}
\end{itemize}
\begin{itemize}
\item {Utilização:Zool.}
\end{itemize}
\begin{itemize}
\item {Proveniência:(Do gr. \textunderscore ozaina\textunderscore , cheiro)}
\end{itemize}
Ulceração das membranas mucosas das fossas nasaes, do véu palatino ou do seio maxillar, vertendo pus fétido, que impregna o ar de cheiro repugnante.
Gênero de insectos coleópteros pentâmeros.
\section{Ozénico}
\begin{itemize}
\item {Grp. gram.:adj.}
\end{itemize}
Relativo á ozena, úlcera.
\section{Ózio}
\begin{itemize}
\item {Grp. gram.:m.}
\end{itemize}
Gênero de crustáceos decápodes.
\section{Ozo}
\begin{itemize}
\item {Grp. gram.:m.}
\end{itemize}
\begin{itemize}
\item {Utilização:Ant.}
\end{itemize}
O mesmo que \textunderscore arsênico\textunderscore .
\section{Ozocerite}
\begin{itemize}
\item {Grp. gram.:f.}
\end{itemize}
\begin{itemize}
\item {Proveniência:(Do gr. \textunderscore ozo\textunderscore , cheirar, e \textunderscore keros\textunderscore , cera)}
\end{itemize}
Espécie de resina ou cera fóssil, também chamada pez mineral.
\section{Ozodécero}
\begin{itemize}
\item {Grp. gram.:m.}
\end{itemize}
Gênero de insectos coleópteros.
\section{Ozódera}
\begin{itemize}
\item {Grp. gram.:f.}
\end{itemize}
Gênero de insectos coleópteros.
\section{Ozómena}
\begin{itemize}
\item {Grp. gram.:f.}
\end{itemize}
Gênero de insectos coleópteros.
\section{Ozonador}
\begin{itemize}
\item {Grp. gram.:m.}
\end{itemize}
\begin{itemize}
\item {Proveniência:(De \textunderscore ozone\textunderscore )}
\end{itemize}
Apparelho, para inhalações electrizadas. Cf. \textunderscore País\textunderscore , do Rio, de 13-I-901. Cp. \textunderscore ozonizador\textunderscore .
\section{Ozone}
\begin{itemize}
\item {Grp. gram.:m.}
\end{itemize}
\begin{itemize}
\item {Proveniência:(Do gr. \textunderscore ozein\textunderscore )}
\end{itemize}
Cheiro, que se desenvolve sob a influência das descargas eléctricas, e que é devido ao estado particular, que as descargas produzem no oxygênio.
\section{Ozónio}
\begin{itemize}
\item {Grp. gram.:m.}
\end{itemize}
\begin{itemize}
\item {Proveniência:(Do gr. \textunderscore ozein\textunderscore )}
\end{itemize}
Gênero de cogumelos.
O mesmo ou melhor que \textunderscore ozone\textunderscore .
\section{Ozonização}
\begin{itemize}
\item {Grp. gram.:f.}
\end{itemize}
Acto de ozonizar.
\section{Ozonizado}
\begin{itemize}
\item {Grp. gram.:adj.}
\end{itemize}
\begin{itemize}
\item {Proveniência:(De \textunderscore ozonizar\textunderscore )}
\end{itemize}
Que contém ozónio.
\section{Ozonizador}
\begin{itemize}
\item {Grp. gram.:m.}
\end{itemize}
\begin{itemize}
\item {Proveniência:(De \textunderscore ozonizar\textunderscore )}
\end{itemize}
Apparelho, com que se produz ozónio.
\section{Ozonizar}
\begin{itemize}
\item {Grp. gram.:v. t.}
\end{itemize}
Combinar com o ozónio.
\section{Ozonometria}
\begin{itemize}
\item {Grp. gram.:f.}
\end{itemize}
Applicação do ozonómetro.
\section{Ozonométrico}
\begin{itemize}
\item {Grp. gram.:adj.}
\end{itemize}
Relativo á ozonometria.
\section{Ozonómetro}
\begin{itemize}
\item {Grp. gram.:m.}
\end{itemize}
\begin{itemize}
\item {Proveniência:(Do gr. \textunderscore ozein\textunderscore  + \textunderscore metron\textunderscore )}
\end{itemize}
Apparelho, para determinar a quantidade de ozónio contida num gás.
\section{Ozonoscópico}
\begin{itemize}
\item {Grp. gram.:adj.}
\end{itemize}
\begin{itemize}
\item {Proveniência:(Do gr. \textunderscore ozein\textunderscore  + \textunderscore skopein\textunderscore )}
\end{itemize}
Que serve para verificar a presença do ozónio. Cf. R. Galvão, \textunderscore Vocab.\textunderscore 
\section{Ozórias}
\begin{itemize}
\item {Grp. gram.:f. pl.}
\end{itemize}
Antigo jôgo de cartas.
\section{Ozotamno}
\begin{itemize}
\item {Grp. gram.:m.}
\end{itemize}
Gênero de plantas, da fam. das compostas.
\section{Ozoterita}
\begin{itemize}
\item {Grp. gram.:f.}
\end{itemize}
Substância bituminosa, que se encontra na terra, e de que se extrai a parafina.
\section{Ozothamno}
\begin{itemize}
\item {Grp. gram.:m.}
\end{itemize}
Gênero de plantas, da fam. das compostas.
\section{Ozotherita}
\begin{itemize}
\item {Grp. gram.:f.}
\end{itemize}
\end{document}