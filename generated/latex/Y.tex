
\begin{itemize}
\item {Proveniência: }
\end{itemize}\documentclass{article}
\usepackage[portuguese]{babel}
\title{Y}
\begin{document}
Divisão da classe das aves, que comprehende as gallináceas e as columbinas.
\section{Y}
\begin{itemize}
\item {fónica:úpsilon,úpsilo}
\end{itemize}
\begin{itemize}
\item {Grp. gram.:m.}
\end{itemize}
Vigésima quarta letra do alphabeto português, só tolerável hoje em algumas palavras que procedem do grego ou de certas línguas estrangeiras.
Uma das incógnitas, (em problemas de Mathemática).
Como letra numeral, valeu 150 e, com um til por cima, 150:000.
É vulgar o uso do \textunderscore y\textunderscore  em palavras brasileiras de origem tupi e em muitos nomes geográphicos e ethnográphicos da América. Explicam o phenómeno com o facto de que em taes casos, o \textunderscore i\textunderscore  não representaria o som que primitivamente se convencionou representar por \textunderscore y\textunderscore .
E assim também, na antiga orthogr. port., era vulgar o uso do \textunderscore y\textunderscore  entre vogaes ou no fim de palavras, adeante de outra vogal: \textunderscore meyo\textunderscore , \textunderscore pay\textunderscore , \textunderscore rey\textunderscore , etc.; e explicava-se êsse uso com o facto de que tal letra não representava o valor de \textunderscore i\textunderscore , mas de dois \textunderscore ii\textunderscore . Ora dês que o \textunderscore y\textunderscore , no actual estado da língua, só póde soar como \textunderscore i\textunderscore , nada justifica o emprêgo delle em palavras que não procedam do grego; e ainda nestas, o seu emprêgo seria vantajosamente eliminado, seguindo-se bons exemplos da Italia e da Espanha.
(O \textunderscore y\textunderscore  é falsamente denominado \textunderscore i grego\textunderscore : não existe no alphabeto grego; é convenção latina, para representação do \textunderscore úpsilon\textunderscore , representação incoherente, porque os Latinos umas vezes representaram o \textunderscore úpsilon\textunderscore , por \textunderscore y\textunderscore , outras por \textunderscore u\textunderscore , e outras por \textunderscore i\textunderscore )
\section{Yorkino}
\begin{itemize}
\item {Grp. gram.:adj.}
\end{itemize}
\begin{itemize}
\item {Grp. gram.:M.}
\end{itemize}
Relativo a York.
Aquelle que é natural de York.
\section{York-Madeira}
\begin{itemize}
\item {Grp. gram.:f.}
\end{itemize}
Espécie de videira hýbrida.
\section{Ýpsilon}
\begin{itemize}
\item {Grp. gram.:m.}
\end{itemize}
O mesmo que \textunderscore úpsilon\textunderscore .
Designação da letra \textunderscore Y\textunderscore .
\section{Ytterbite}
\begin{itemize}
\item {Grp. gram.:f.}
\end{itemize}
\begin{itemize}
\item {Proveniência:(De \textunderscore Ytterby\textunderscore , n. p.)}
\end{itemize}
Designação primitiva da \textunderscore gadolinite\textunderscore .
\section{Ýttria}
\begin{itemize}
\item {Grp. gram.:f.}
\end{itemize}
Óxydo de ýttrio.
\section{Ýttrico}
\begin{itemize}
\item {Grp. gram.:adj.}
\end{itemize}
Relativo ao ýttrio.
\section{Ýttrio}
\begin{itemize}
\item {Grp. gram.:m.}
\end{itemize}
\begin{itemize}
\item {Proveniência:(Do rad. de \textunderscore ytterbite\textunderscore )}
\end{itemize}
Metal terroso.
\section{Yttrotantalito}
\begin{itemize}
\item {Grp. gram.:m.}
\end{itemize}
\begin{itemize}
\item {Proveniência:(De \textunderscore ýttrio\textunderscore  + \textunderscore tantalito\textunderscore )}
\end{itemize}
\end{document}