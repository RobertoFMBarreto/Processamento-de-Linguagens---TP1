
\begin{itemize}
\item {Proveniência: }
\end{itemize}\documentclass{article}
\usepackage[portuguese]{babel}
\title{B}
\begin{document}
Diz-so do pão, que não fermentou.
Pão, que não fermentou.
\section{Bacharelesco}
\begin{itemize}
\item {fónica:lês}
\end{itemize}
\begin{itemize}
\item {Grp. gram.:adj.}
\end{itemize}
\begin{itemize}
\item {Utilização:Deprec.}
\end{itemize}
Relativo a bacharel; próprio de bacharel.
\section{Bacterioterapia}
\begin{itemize}
\item {Grp. gram.:f.}
\end{itemize}
\begin{itemize}
\item {Utilização:Med.}
\end{itemize}
\begin{itemize}
\item {Proveniência:(De \textunderscore bactéria\textunderscore  + \textunderscore terapia\textunderscore )}
\end{itemize}
Emprêgo terapêutico de certas culturas microbianas.
\section{Bacteriotherapia}
\begin{itemize}
\item {Grp. gram.:f.}
\end{itemize}
\begin{itemize}
\item {Utilização:Med.}
\end{itemize}
\begin{itemize}
\item {Proveniência:(De \textunderscore bactéria\textunderscore  + \textunderscore therapia\textunderscore )}
\end{itemize}
Emprêgo therapêutico de certas culturas microbianas.
\section{Badaladal}
\begin{itemize}
\item {Grp. gram.:m.}
\end{itemize}
Muitas badaladas:«\textunderscore aquelles badaladaes fúnebres...\textunderscore »Camillo, \textunderscore Volcões\textunderscore , 109.
\section{Badém}
\begin{itemize}
\item {Grp. gram.:m.}
\end{itemize}
Capa, o mesmo que \textunderscore bedém\textunderscore . Cf. B. Pereira, \textunderscore Prosódia\textunderscore , vb. \textunderscore imatium\textunderscore .
\section{Bafum}
\begin{itemize}
\item {Grp. gram.:m.}
\end{itemize}
\begin{itemize}
\item {Utilização:Prov.}
\end{itemize}
Mau cheiro; cheiro desagradável.
(Cp. \textunderscore bafio\textunderscore )
\section{Baga-de-loiro}
\begin{itemize}
\item {Grp. gram.:f.}
\end{itemize}
\begin{itemize}
\item {Utilização:Prov.}
\end{itemize}
\begin{itemize}
\item {Utilização:beir.}
\end{itemize}
Casta de uva preta, o mesmo que \textunderscore baga\textunderscore .
\section{Baixante}
\begin{itemize}
\item {Grp. gram.:f.}
\end{itemize}
\begin{itemize}
\item {Utilização:T. de Barcelos}
\end{itemize}
\begin{itemize}
\item {Proveniência:(De \textunderscore baixar\textunderscore )}
\end{itemize}
A parede inclinada da chaminé.
\section{Baixate}
\begin{itemize}
\item {Grp. gram.:m.}
\end{itemize}
Ferramenta de tanoeiro.
\section{Balanoposthite}
\begin{itemize}
\item {Grp. gram.:f.}
\end{itemize}
\begin{itemize}
\item {Utilização:Med.}
\end{itemize}
\begin{itemize}
\item {Proveniência:(Do gr. \textunderscore balanos\textunderscore  + \textunderscore posthe\textunderscore )}
\end{itemize}
Inflammação da superfície da glândula e da mucosa do prepúcio.
\section{Balanopostite}
\begin{itemize}
\item {Grp. gram.:f.}
\end{itemize}
\begin{itemize}
\item {Utilização:Med.}
\end{itemize}
\begin{itemize}
\item {Proveniência:(Do gr. \textunderscore balanos\textunderscore  + \textunderscore posthe\textunderscore )}
\end{itemize}
Inflamação da superfície da glândula e da mucosa do prepúcio.
\section{Balaústro}
\begin{itemize}
\item {Grp. gram.:m.}
\end{itemize}
O mesmo que \textunderscore balaústre\textunderscore .
\section{Baldroégas}
\begin{itemize}
\item {Grp. gram.:m.}
\end{itemize}
\begin{itemize}
\item {Utilização:Prov.}
\end{itemize}
O mesmo que \textunderscore beldroegas\textunderscore . (Colhido em Barcelos)
\section{Balhava}
\begin{itemize}
\item {Utilização:Prov.}
\end{itemize}
Coisa insignificante, reles. (Colhido em Benavente)
\section{Balona}
\begin{itemize}
\item {Grp. gram.:f.}
\end{itemize}
\begin{itemize}
\item {Utilização:T. de Barcelos}
\end{itemize}
\begin{itemize}
\item {Proveniência:(De \textunderscore balão\textunderscore )}
\end{itemize}
Mentira; boato falso.
\section{Balsamificar}
\begin{itemize}
\item {Grp. gram.:v. t.}
\end{itemize}
O mesmo que \textunderscore balsamizar\textunderscore . Cf. Camillo, \textunderscore Amor de Salv.\textunderscore , 30.
\section{Bamboão}
\begin{itemize}
\item {Grp. gram.:m.}
\end{itemize}
\begin{itemize}
\item {Utilização:Prov.}
\end{itemize}
\begin{itemize}
\item {Utilização:dur.}
\end{itemize}
Espécie de trapézio, em que se baloiçam crianças. (Colhido em Matosinhos)
\section{Banatite}
\begin{itemize}
\item {Grp. gram.:f.}
\end{itemize}
Rocha granitóide, de composição variável.
\section{Banatito}
\begin{itemize}
\item {Grp. gram.:m.}
\end{itemize}
\begin{itemize}
\item {Utilização:Miner.}
\end{itemize}
Rocha granitóide, de composição variável.
\section{Banja}
\begin{itemize}
\item {Grp. gram.:f.}
\end{itemize}
\begin{itemize}
\item {Utilização:Bras. do N}
\end{itemize}
Trapaça ao jôgo.
\section{Banjista}
\begin{itemize}
\item {Grp. gram.:m.}
\end{itemize}
\begin{itemize}
\item {Utilização:Bras. do N}
\end{itemize}
Aquelle que faz banja.
\section{Baoneta}
\begin{itemize}
\item {fónica:nê}
\end{itemize}
\begin{itemize}
\item {Grp. gram.:f.}
\end{itemize}
\begin{itemize}
\item {Utilização:Ant.}
\end{itemize}
O mesmo que \textunderscore baioneta\textunderscore .
\section{Barbas-de-velho}
\begin{itemize}
\item {Grp. gram.:f. pl.}
\end{itemize}
\begin{itemize}
\item {Utilização:Bot.}
\end{itemize}
Planta ranunculácea, (\textunderscore nigella damascena\textunderscore , Lin.). Cf. P. Coutinho, \textunderscore Flora\textunderscore , 237.
\section{Barbos}
\begin{itemize}
\item {Grp. gram.:m. pl.}
\end{itemize}
\begin{itemize}
\item {Utilização:T. de Barcelos}
\end{itemize}
\begin{itemize}
\item {Utilização:Fig.}
\end{itemize}
Excrescência carnosa na boca dos bois, a qual lhes tira a vontade de comer.
O mesmo que \textunderscore fastio\textunderscore .
\section{Barotropismo}
\begin{itemize}
\item {Grp. gram.:m.}
\end{itemize}
\begin{itemize}
\item {Utilização:Physiol.}
\end{itemize}
\begin{itemize}
\item {Proveniência:(Do gr. \textunderscore baros\textunderscore  + \textunderscore trepein\textunderscore )}
\end{itemize}
Propriedade, que tem o protoplasma, de reagir aos contactos e ás vibrações.
\section{Baritite}
\begin{itemize}
\item {Grp. gram.:f.}
\end{itemize}
O mesmo que \textunderscore barytina\textunderscore .
\section{Barrego}
\begin{itemize}
\item {fónica:ré}
\end{itemize}
\begin{itemize}
\item {Grp. gram.:m.}
\end{itemize}
\begin{itemize}
\item {Utilização:T. de Barcelos}
\end{itemize}
Acto de barregar.
\section{Barrete-de-padre}
\begin{itemize}
\item {Grp. gram.:m.}
\end{itemize}
\begin{itemize}
\item {Utilização:Bot.}
\end{itemize}
Planta cucurbitácea, também conhecida por \textunderscore abóbora-de-corôa\textunderscore . P. Coutinho, \textunderscore Flora\textunderscore , 599.
Casta de uva da Madeira.
Cp. \textunderscore barrete-de-clérigo\textunderscore .
\section{Barrileta}
\begin{itemize}
\item {fónica:lê}
\end{itemize}
\begin{itemize}
\item {Grp. gram.:f.}
\end{itemize}
O mesmo que \textunderscore barrilete\textunderscore . Cf. Pant. de Aveiro, \textunderscore Itiner.\textunderscore , 255.
\section{Barrilha-espinhosa}
\begin{itemize}
\item {Grp. gram.:f.}
\end{itemize}
\begin{itemize}
\item {Utilização:Bot.}
\end{itemize}
O mesmo que \textunderscore barrilheira\textunderscore . Cf. P. Coutinho, \textunderscore Flora\textunderscore , 191.
\section{Barroco}
\begin{itemize}
\item {fónica:rô}
\end{itemize}
\begin{itemize}
\item {Grp. gram.:adj.}
\end{itemize}
O mesmo que \textunderscore baroco\textunderscore ^1.
\section{Bartholinite}
\begin{itemize}
\item {Grp. gram.:f.}
\end{itemize}
\begin{itemize}
\item {Utilização:Med.}
\end{itemize}
\begin{itemize}
\item {Proveniência:(De \textunderscore Bartholin\textunderscore , n. p.)}
\end{itemize}
Inflammação da chamada glande de Bartholin, geralmente de origem blennorrágica.
\section{Bartolinite}
\begin{itemize}
\item {Grp. gram.:f.}
\end{itemize}
\begin{itemize}
\item {Utilização:Med.}
\end{itemize}
\begin{itemize}
\item {Proveniência:(De \textunderscore Bartholin\textunderscore , n. p.)}
\end{itemize}
Inflamação da chamada glande de Bartholin, geralmente de origem blenorrágica.
\section{Barytite}
\begin{itemize}
\item {Grp. gram.:f.}
\end{itemize}
O mesmo que \textunderscore barytina\textunderscore .
\section{Basanite}
\begin{itemize}
\item {Grp. gram.:f.}
\end{itemize}
Variedade de jaspe negro, empregado no exame de objectos de oiro, e também chamado \textunderscore pedra-de-toque\textunderscore .
\section{Basanito}
\begin{itemize}
\item {Grp. gram.:m.}
\end{itemize}
Variedade de jaspe negro, empregado no exame de objectos de oiro, e também chamado \textunderscore pedra-de-toque\textunderscore .
\section{Basiótribo}
\begin{itemize}
\item {Grp. gram.:m.}
\end{itemize}
\begin{itemize}
\item {Utilização:Med.}
\end{itemize}
\begin{itemize}
\item {Proveniência:(Do gr. \textunderscore basis\textunderscore  + \textunderscore tribein\textunderscore )}
\end{itemize}
Instrumento, para esmagar a base do crânio do feto.
\section{Basiotripsia}
\begin{itemize}
\item {Grp. gram.:f.}
\end{itemize}
\begin{itemize}
\item {Utilização:Med.}
\end{itemize}
Applicação do basiótribo.
\section{Batalou}
\begin{itemize}
\item {Grp. gram.:m.}
\end{itemize}
Planta, o mesmo que \textunderscore patalou\textunderscore .
\section{Batedoiro}
\begin{itemize}
\item {Grp. gram.:m.}
\end{itemize}
\begin{itemize}
\item {Utilização:Bras. do N}
\end{itemize}
Lugar, onde se reúnem as vacas, acossadas pelas mutucas.
\section{Batedouro}
\begin{itemize}
\item {Grp. gram.:m.}
\end{itemize}
\begin{itemize}
\item {Utilização:Bras. do N}
\end{itemize}
Lugar, onde se reúnem as vacas, acossadas pelas mutucas.
\section{Batoiro}
\begin{itemize}
\item {Grp. gram.:m.}
\end{itemize}
\begin{itemize}
\item {Utilização:Prov.}
\end{itemize}
O mesmo que \textunderscore bitoiro\textunderscore .
\section{Beberar}
\begin{itemize}
\item {Grp. gram.:v. t.}
\end{itemize}
O mesmo que \textunderscore abeberar\textunderscore .
\section{Beiradejar}
\begin{itemize}
\item {Grp. gram.:v. t.}
\end{itemize}
\begin{itemize}
\item {Utilização:Bras. do N}
\end{itemize}
\begin{itemize}
\item {Proveniência:(De \textunderscore beira\textunderscore )}
\end{itemize}
Andar a pé pelos arredores de; contornar.
\section{Belena}
\begin{itemize}
\item {Grp. gram.:f.}
\end{itemize}
\begin{itemize}
\item {Utilização:Prov.}
\end{itemize}
\begin{itemize}
\item {Utilização:minh.}
\end{itemize}
\begin{itemize}
\item {Proveniência:(De \textunderscore beleno\textunderscore )}
\end{itemize}
Mulhér mexeriqueira, intrigante.
\section{Bela-face}
\begin{itemize}
\item {Grp. gram.:m.}
\end{itemize}
Diz-se o cavallo façalvo.
\section{Bella-rosa}
\begin{itemize}
\item {Grp. gram.:f.}
\end{itemize}
Gênero de plantas de jardim, que floresce em Abril.
\section{Bemzinho}
\begin{itemize}
\item {Grp. gram.:m.}
\end{itemize}
Tratamento, dado familiarmente a pessôas muito queridas:«\textunderscore meu bemzinho, certo, certo...\textunderscore »(De uma canção pop.)
\section{Benado}
\begin{itemize}
\item {Grp. gram.:m.}
\end{itemize}
\begin{itemize}
\item {Utilização:T. de Barcelos}
\end{itemize}
\begin{itemize}
\item {Proveniência:(De \textunderscore bem\textunderscore ?)}
\end{itemize}
Salário.
Emprêgo rendoso.
\section{Bençom}
\begin{itemize}
\item {Grp. gram.:f.}
\end{itemize}
\begin{itemize}
\item {Utilização:Ant.}
\end{itemize}
O mesmo que \textunderscore bênção\textunderscore .
\section{Benthamismo}
\begin{itemize}
\item {Grp. gram.:m.}
\end{itemize}
Conjunto das doutrinas philosóphicas de Bentham.
\section{Bento}
\begin{itemize}
\item {Grp. gram.:m.}
\end{itemize}
Móvel antigo:«\textunderscore um bento com cinco gavetas\textunderscore ». (De um testamento de 1691)
\section{Bestagem}
\begin{itemize}
\item {Grp. gram.:f.}
\end{itemize}
\begin{itemize}
\item {Utilização:Bras. do N}
\end{itemize}
Acto de bestar; tolice; inépcia.
\section{Bicancra}
\begin{itemize}
\item {Grp. gram.:f.}
\end{itemize}
\begin{itemize}
\item {Utilização:Prov.}
\end{itemize}
O mesmo que \textunderscore bebedeira\textunderscore .
\section{Bico-de-grou-sanguíneo}
\begin{itemize}
\item {Grp. gram.:m.}
\end{itemize}
\begin{itemize}
\item {Utilização:Bot.}
\end{itemize}
Planta geraniácea, (\textunderscore geranium sanguineum\textunderscore , Lin.) Cf. P. Coutinho, \textunderscore Flora\textunderscore , 372.
\section{Bico-de-pato}
\begin{itemize}
\item {Grp. gram.:m.}
\end{itemize}
\begin{itemize}
\item {Utilização:Bras. do N}
\end{itemize}
Peixe de água doce.
\section{Bico-de-pomba-maiór}
\begin{itemize}
\item {Grp. gram.:m.}
\end{itemize}
\begin{itemize}
\item {Utilização:Bot.}
\end{itemize}
Planta geraniácea, (\textunderscore geranium columbium\textunderscore , Lin.). Cf. P. Coutinho, \textunderscore Flora\textunderscore , 272.
\section{Bico-de-pomba-menór}
\begin{itemize}
\item {Grp. gram.:m.}
\end{itemize}
\begin{itemize}
\item {Utilização:Bot.}
\end{itemize}
Planta geraniácea, (\textunderscore geranium molle\textunderscore , Lin.). Cf. P. Coutinho, \textunderscore Flora\textunderscore , 553.
\section{Bila}
\begin{itemize}
\item {Grp. gram.:f.}
\end{itemize}
\begin{itemize}
\item {Utilização:Ant.}
\end{itemize}
O mesmo que \textunderscore bílis\textunderscore . Cf. P. Rego, \textunderscore Caval. de Brida\textunderscore , 215, etc.
\section{Bilros}
\begin{itemize}
\item {Grp. gram.:m. pl.}
\end{itemize}
Espécie de renda antiga:«\textunderscore ...um luto de bilros da Índia.\textunderscore »(De um testamento de 1691)
\section{Biltra}
\begin{itemize}
\item {Grp. gram.:f.}
\end{itemize}
\begin{itemize}
\item {Utilização:Fam.}
\end{itemize}
\begin{itemize}
\item {Proveniência:(De \textunderscore biltre\textunderscore )}
\end{itemize}
Mulhér patifa, vil. Cf. Castilho, \textunderscore Doente de Scisma\textunderscore , 6.
\section{Biotite}
\begin{itemize}
\item {Grp. gram.:f.}
\end{itemize}
Espécie de mica escura.
\section{Biotítico}
\begin{itemize}
\item {Grp. gram.:adj.}
\end{itemize}
Relativo a biotite.
Que contém biotite.
\section{Biotito}
\begin{itemize}
\item {Grp. gram.:m.}
\end{itemize}
O mesmo ou melhór que \textunderscore biotite\textunderscore .
\section{Biplano}
\begin{itemize}
\item {Grp. gram.:m.}
\end{itemize}
Aeroplano, com dois planos ou asas.
\section{Biquara}
\begin{itemize}
\item {Grp. gram.:f.}
\end{itemize}
\begin{itemize}
\item {Utilização:Bras}
\end{itemize}
Nome de um peixe.
\section{Biriva}
\begin{itemize}
\item {Grp. gram.:adj.}
\end{itemize}
\begin{itemize}
\item {Utilização:Bras}
\end{itemize}
\begin{itemize}
\item {Grp. gram.:M.}
\end{itemize}
\begin{itemize}
\item {Utilização:Bras}
\end{itemize}
Relativo á cidade de San-Paulo.
Habitante de San-Paulo.
\section{Bismuthina}
\begin{itemize}
\item {Grp. gram.:f.}
\end{itemize}
Sulfureto de bismutho mineral.
\section{Bismutina}
\begin{itemize}
\item {Grp. gram.:f.}
\end{itemize}
Sulfureto de bismuto mineral.
\section{Bissulfito}
\begin{itemize}
\item {Grp. gram.:m.}
\end{itemize}
\begin{itemize}
\item {Utilização:Chím.}
\end{itemize}
Sal, resultante da acção do ácido sulfuroso sôbre uma base.
\section{Bissulfureto}
\begin{itemize}
\item {fónica:furê}
\end{itemize}
\begin{itemize}
\item {Grp. gram.:m.}
\end{itemize}
\begin{itemize}
\item {Utilização:Chím.}
\end{itemize}
Qualquer corpo binário, não oxigenado, formado de dois átomos de enxôfre e de um átomo de outro corpo simples.
\section{Bisulfito}
\begin{itemize}
\item {fónica:sul}
\end{itemize}
\begin{itemize}
\item {Grp. gram.:m.}
\end{itemize}
\begin{itemize}
\item {Utilização:Chím.}
\end{itemize}
Sal, resultante da acção do ácido sulfuroso sôbre uma base.
\section{Bisulfureto}
\begin{itemize}
\item {fónica:sul}
\end{itemize}
\begin{itemize}
\item {Grp. gram.:m.}
\end{itemize}
\begin{itemize}
\item {Utilização:Chím.}
\end{itemize}
Qualquer corpo binário, não oxygenado, formado de dois átomos de enxôfre e de um átomo de outro corpo simples.
\section{Blastomicetos}
\begin{itemize}
\item {Grp. gram.:m. pl.}
\end{itemize}
\begin{itemize}
\item {Proveniência:(Do gr. \textunderscore blastos\textunderscore  + \textunderscore mukes\textunderscore )}
\end{itemize}
Grupo de cogumelos, que se reproduzem por gomos.
\section{Blastomycetos}
\begin{itemize}
\item {Grp. gram.:m. pl.}
\end{itemize}
\begin{itemize}
\item {Proveniência:(Do gr. \textunderscore blastos\textunderscore  + \textunderscore mukes\textunderscore )}
\end{itemize}
Grupo de cogumelos, que se reproduzem por gomos.
\section{Blefaraptose}
\begin{itemize}
\item {Grp. gram.:f.}
\end{itemize}
\begin{itemize}
\item {Utilização:Med.}
\end{itemize}
\begin{itemize}
\item {Proveniência:(Do gr. \textunderscore blepharon\textunderscore  + \textunderscore ptosis\textunderscore )}
\end{itemize}
Quéda completa, ou incompleta, da pálpebra superior.
\section{Blefarorrafia}
\begin{itemize}
\item {Grp. gram.:f.}
\end{itemize}
\begin{itemize}
\item {Utilização:Med.}
\end{itemize}
\begin{itemize}
\item {Proveniência:(Do gr. \textunderscore blepharon\textunderscore  + \textunderscore raphe\textunderscore )}
\end{itemize}
Sutura das pálpebras, ou operação, que tem por fim deminuir a abertura palpebral.
\section{Bleima}
\begin{itemize}
\item {Grp. gram.:f.}
\end{itemize}
\begin{itemize}
\item {Utilização:Veter.}
\end{itemize}
Contusão ou pisadura, nos talões das patas dos equídeos. Cf. Macedo Pinto, \textunderscore Comp. de Veter.\textunderscore , I, 353.
\section{Blepharoptose}
\begin{itemize}
\item {Grp. gram.:f.}
\end{itemize}
\begin{itemize}
\item {Utilização:Med.}
\end{itemize}
\begin{itemize}
\item {Proveniência:(Do gr. \textunderscore blepharon\textunderscore  + \textunderscore ptosis\textunderscore )}
\end{itemize}
Quéda completa, ou incompleta, da pálpebra superior.
\section{Blepharorraphia}
\begin{itemize}
\item {Grp. gram.:f.}
\end{itemize}
\begin{itemize}
\item {Utilização:Med.}
\end{itemize}
\begin{itemize}
\item {Proveniência:(Do gr. \textunderscore blepharon\textunderscore  + \textunderscore raphe\textunderscore )}
\end{itemize}
Sutura das pálpebras, ou operação, que tem por fim deminuir a abertura palpebral.
\section{Blepharotique}
\begin{itemize}
\item {Grp. gram.:m.}
\end{itemize}
\begin{itemize}
\item {Utilização:Med.}
\end{itemize}
\begin{itemize}
\item {Proveniência:(De \textunderscore blepharon\textunderscore  gr. + \textunderscore tique\textunderscore )}
\end{itemize}
Tique convulsivo das pálpebras.
\section{Boatar}
\begin{itemize}
\item {Grp. gram.:v. i.}
\end{itemize}
\begin{itemize}
\item {Utilização:Fam.}
\end{itemize}
Espalhar boatos.
\section{Bobar}
\begin{itemize}
\item {Grp. gram.:v. i.}
\end{itemize}
\begin{itemize}
\item {Utilização:Bras. do N}
\end{itemize}
O mesmo que \textunderscore bobear\textunderscore .
\section{Bôcas-de-lobo}
\begin{itemize}
\item {Grp. gram.:f. pl.}
\end{itemize}
\begin{itemize}
\item {Utilização:Bot.}
\end{itemize}
Planta, o mesmo que \textunderscore erva-bezerra\textunderscore .
\section{Boceto}
\begin{itemize}
\item {fónica:cê}
\end{itemize}
\begin{itemize}
\item {Grp. gram.:adj.}
\end{itemize}
O mesmo que \textunderscore boquilavado\textunderscore .
\section{Boiote}
\begin{itemize}
\item {Grp. gram.:m.}
\end{itemize}
\begin{itemize}
\item {Utilização:Bras. do N}
\end{itemize}
Boi pequeno.
\section{Boipeva}
\begin{itemize}
\item {Grp. gram.:f.}
\end{itemize}
\begin{itemize}
\item {Utilização:Bras}
\end{itemize}
Nome de uma cobra venenosa, (\textunderscore lachesis itapetininga\textunderscore ).
Nome de outra cobra, não venenosa, (\textunderscore xenedon merrenii\textunderscore ).
(Do tupi \textunderscore boi\textunderscore , cobra, e \textunderscore peva\textunderscore , que se arrasta)
\section{Bona-fide}
\begin{itemize}
\item {Grp. gram.:f.}
\end{itemize}
Penna de certa ave indiana, usada em adornos femininos? Cf. \textunderscore Decreto\textunderscore  de 21-XI-910.
(Loc. lat., correspondente a \textunderscore de bôa fé\textunderscore )
\section{Bonete}
\begin{itemize}
\item {Grp. gram.:m.}
\end{itemize}
\begin{itemize}
\item {Utilização:Veter.}
\end{itemize}
\begin{itemize}
\item {Proveniência:(Do fr. \textunderscore bonnet\textunderscore )}
\end{itemize}
Segunda cavidade do estômago dos ruminantes, também conhecida por \textunderscore barrete\textunderscore . Cf. Macedo Pinto, \textunderscore Comp. de Veter.\textunderscore , I, 166.
\section{Bonnete}
\begin{itemize}
\item {Grp. gram.:m.}
\end{itemize}
\begin{itemize}
\item {Utilização:Veter.}
\end{itemize}
\begin{itemize}
\item {Proveniência:(Do fr. \textunderscore bonnet\textunderscore )}
\end{itemize}
Segunda cavidade do estômago dos ruminantes, também conhecida por \textunderscore barrete\textunderscore . Cf. Macedo Pinto, \textunderscore Comp. de Veter.\textunderscore , I, 166.
\section{Boquilavado}
\begin{itemize}
\item {Grp. gram.:adj.}
\end{itemize}
Diz-se do cavallo, que tem esbranquiçadas as ventas e a parte inferior da cabeça.
\section{Boracite}
\begin{itemize}
\item {Grp. gram.:f.}
\end{itemize}
Mineral, composto de borato de magnésio e chloreto de magnésio.
\section{Boracito}
\begin{itemize}
\item {Grp. gram.:m.}
\end{itemize}
\begin{itemize}
\item {Utilização:Miner.}
\end{itemize}
Mineral, composto de borato de magnésio e chloreto de magnésio.
\section{Borcado}
\begin{itemize}
\item {Grp. gram.:m.}
\end{itemize}
\begin{itemize}
\item {Utilização:Ant.}
\end{itemize}
O mesmo que \textunderscore brocado\textunderscore . Cf. B. Pereira, \textunderscore Prosodia\textunderscore , vb. \textunderscore staurachium\textunderscore .
\section{Bordalês}
\begin{itemize}
\item {Grp. gram.:adj.}
\end{itemize}
O mesmo que \textunderscore bordelês\textunderscore .
\textunderscore Calda bordalesa\textunderscore , solução de sulfato de cobre em água, com que se tratam videiras e outras plantas.
\section{Bordo-de-monpilhér}
\begin{itemize}
\item {Grp. gram.:m.}
\end{itemize}
\begin{itemize}
\item {Utilização:Bot.}
\end{itemize}
Variedade de bordo, (\textunderscore acer monspessulanus\textunderscore , Lin.), que se encontra em Trás-os-Montes.
\section{Bórra}
\begin{itemize}
\item {Grp. gram.:f.}
\end{itemize}
\begin{itemize}
\item {Utilização:Chul.}
\end{itemize}
\begin{itemize}
\item {Proveniência:(De \textunderscore borrar\textunderscore )}
\end{itemize}
Soltura, diarreia.
\section{Borrachífero}
\begin{itemize}
\item {Grp. gram.:adj.}
\end{itemize}
\begin{itemize}
\item {Utilização:Neol.}
\end{itemize}
\begin{itemize}
\item {Proveniência:(De \textunderscore borracha\textunderscore  + lat. \textunderscore ferre\textunderscore )}
\end{itemize}
Diz-se da planta que produz borracha. Cf. \textunderscore Diário-de-Not.\textunderscore , de 28-II-912.
\section{Borrazeira}
\begin{itemize}
\item {Grp. gram.:f.}
\end{itemize}
\begin{itemize}
\item {Utilização:Bot.}
\end{itemize}
Espécie de salgueiro, (\textunderscore salix cinerea\textunderscore , Lin.). Cf. P. Coutinho, \textunderscore Flora\textunderscore , 160.
\section{Borrêga}
\begin{itemize}
\item {Grp. gram.:f.}
\end{itemize}
\begin{itemize}
\item {Utilização:T. da Guarda}
\end{itemize}
O mesmo que \textunderscore bojega\textunderscore .
\section{Bosó}
\begin{itemize}
\item {Grp. gram.:m.}
\end{itemize}
\begin{itemize}
\item {Utilização:T. de Pernambuco}
\end{itemize}
Espécie de jôgo de dados.
\section{Bosquilha}
\begin{itemize}
\item {Grp. gram.:f.}
\end{itemize}
\begin{itemize}
\item {Utilização:Prov.}
\end{itemize}
\begin{itemize}
\item {Utilização:trasm.}
\end{itemize}
Espécie de atordoamento, produzido pelo calor no gado bovino. Cf. Macedo Pinto, \textunderscore Comp. de Veter.\textunderscore , I, 472.
\section{Botulínico}
\begin{itemize}
\item {Grp. gram.:adj.}
\end{itemize}
Relativo ao butulismo:«\textunderscore a epidemia botulínica de Berlim.\textunderscore »\textunderscore Diário-de-Not.\textunderscore , de Lisbôa, de 25-IX-912.
\section{Brachycephalizar}
\begin{itemize}
\item {fónica:qui}
\end{itemize}
\begin{itemize}
\item {Grp. gram.:v. t.}
\end{itemize}
\begin{itemize}
\item {Utilização:Neol.}
\end{itemize}
Tornar brachycéphala (uma raça), por cruzamento. Cf. \textunderscore Instituto\textunderscore , XLIV, 539.
\section{Brachylogia}
\begin{itemize}
\item {fónica:qui}
\end{itemize}
\begin{itemize}
\item {Grp. gram.:f.}
\end{itemize}
O mesmo ou talvez melhór que \textunderscore brachyologia\textunderscore .
\section{Bracicurto}
\begin{itemize}
\item {Grp. gram.:adj.}
\end{itemize}
\begin{itemize}
\item {Utilização:Veter.}
\end{itemize}
\begin{itemize}
\item {Proveniência:(De \textunderscore braço\textunderscore  + \textunderscore curto\textunderscore )}
\end{itemize}
Diz-se da cavalgadura, que tem joêlhos curvos ou arqueados. Cf. M. Pinto, \textunderscore Comp. de Veter.\textunderscore , I. 416.
\section{Bradifasia}
\begin{itemize}
\item {Grp. gram.:f.}
\end{itemize}
\begin{itemize}
\item {Utilização:Med.}
\end{itemize}
\begin{itemize}
\item {Proveniência:(Do gr. \textunderscore bradus\textunderscore  + \textunderscore phasis\textunderscore )}
\end{itemize}
Lentidão na pronúncia das palavras.
\section{Bradipneia}
\begin{itemize}
\item {Grp. gram.:f.}
\end{itemize}
\begin{itemize}
\item {Utilização:Med.}
\end{itemize}
\begin{itemize}
\item {Proveniência:(Do gr. \textunderscore bradus\textunderscore  + \textunderscore pnein\textunderscore )}
\end{itemize}
Respiração lenta.
\section{Bradyphasia}
\begin{itemize}
\item {Grp. gram.:f.}
\end{itemize}
\begin{itemize}
\item {Utilização:Med.}
\end{itemize}
\begin{itemize}
\item {Proveniência:(Do gr. \textunderscore bradus\textunderscore  + \textunderscore phasis\textunderscore )}
\end{itemize}
Lentidão na pronúncia das palavras.
\section{Bradypnéa}
\begin{itemize}
\item {Grp. gram.:f.}
\end{itemize}
\begin{itemize}
\item {Utilização:Med.}
\end{itemize}
\begin{itemize}
\item {Proveniência:(Do gr. \textunderscore bradus\textunderscore  + \textunderscore pnein\textunderscore )}
\end{itemize}
Respiração lenta.
\section{Bragancês}
\begin{itemize}
\item {Grp. gram.:m.  e  adj.}
\end{itemize}
O mesmo que \textunderscore braganção\textunderscore .
\section{Brancarão}
\begin{itemize}
\item {Grp. gram.:adj.}
\end{itemize}
\begin{itemize}
\item {Utilização:Bras. do N}
\end{itemize}
Quási mulato, na côr.
\section{Brancaria}
\begin{itemize}
\item {Grp. gram.:f.}
\end{itemize}
O mesmo que \textunderscore branquearia\textunderscore .
\section{Branqueiro}
\begin{itemize}
\item {Grp. gram.:m.}
\end{itemize}
O mesmo que \textunderscore branqueira\textunderscore .
\section{Braquicefalizar}
\begin{itemize}
\item {Grp. gram.:v. t.}
\end{itemize}
\begin{itemize}
\item {Utilização:Neol.}
\end{itemize}
Tornar braquicéfala (uma raça), por cruzamento. Cf. \textunderscore Instituto\textunderscore , XLIV, 539.
\section{Braquilogia}
\begin{itemize}
\item {Grp. gram.:f.}
\end{itemize}
O mesmo ou talvez melhór que \textunderscore brachyologia\textunderscore .
\section{Brechiforme}
\begin{itemize}
\item {Grp. gram.:adj.}
\end{itemize}
Que tem os caracteres estratigráphicos da brecha^2.
\section{Brigadier}
\begin{itemize}
\item {Grp. gram.:m.}
\end{itemize}
\begin{itemize}
\item {Utilização:Ant.}
\end{itemize}
\begin{itemize}
\item {Proveniência:(Fr. \textunderscore brigadier\textunderscore )}
\end{itemize}
O mesmo que \textunderscore brigadeiro\textunderscore . Cf. P. Carvalho, \textunderscore Corogr. Port.\textunderscore , III, 436.--Foi gallicismo corrente, nos séc. XVII e XVIII.
\section{Brochão}
\begin{itemize}
\item {Grp. gram.:m.}
\end{itemize}
\begin{itemize}
\item {Utilização:Prov.}
\end{itemize}
\begin{itemize}
\item {Proveniência:(De \textunderscore brocha\textunderscore )}
\end{itemize}
Prego de chapa, para as rodas dos carros.
\section{Brolhar}
\begin{itemize}
\item {Grp. gram.:v. i.}
\end{itemize}
O mesmo que \textunderscore abrolhar\textunderscore , deitar gomos. Cf. B. Pereira, \textunderscore Prosodia\textunderscore , vb. \textunderscore gemmo\textunderscore .
\section{Bronchectasia}
\begin{itemize}
\item {fónica:que}
\end{itemize}
\begin{itemize}
\item {Grp. gram.:f.}
\end{itemize}
\begin{itemize}
\item {Utilização:Med.}
\end{itemize}
\begin{itemize}
\item {Proveniência:(Do gr. \textunderscore bronkhos\textunderscore  + \textunderscore ektasis\textunderscore )}
\end{itemize}
Dilatação dos brônchios.
\section{Bronchorragia}
\begin{itemize}
\item {fónica:co}
\end{itemize}
\begin{itemize}
\item {Grp. gram.:f.}
\end{itemize}
\begin{itemize}
\item {Utilização:Med.}
\end{itemize}
\begin{itemize}
\item {Proveniência:(Do gr. \textunderscore bronkhos\textunderscore  + \textunderscore regnumi\textunderscore )}
\end{itemize}
Hemorragia dos brônchios.
\section{Broncorragia}
\begin{itemize}
\item {Grp. gram.:f.}
\end{itemize}
\begin{itemize}
\item {Utilização:Med.}
\end{itemize}
\begin{itemize}
\item {Proveniência:(Do gr. \textunderscore bronkhos\textunderscore  + \textunderscore regnumi\textunderscore )}
\end{itemize}
Hemorragia dos brônquios.
\section{Bronquectasia}
\begin{itemize}
\item {Grp. gram.:f.}
\end{itemize}
\begin{itemize}
\item {Utilização:Med.}
\end{itemize}
\begin{itemize}
\item {Proveniência:(Do gr. \textunderscore bronkhos\textunderscore  + \textunderscore ektasis\textunderscore )}
\end{itemize}
Dilatação dos brônquios.
\section{Bronzite}
\begin{itemize}
\item {Grp. gram.:f.}
\end{itemize}
Espécie de pyroxênio.
\section{Bronzito}
\begin{itemize}
\item {Grp. gram.:m.}
\end{itemize}
Espécie de pyroxênio.
\section{Brosladura}
\begin{itemize}
\item {Grp. gram.:f.}
\end{itemize}
Acto ou effeito de broslar; bordado. Cf. B. Pereira, \textunderscore Prosodia\textunderscore , vb. \textunderscore praeclavium\textunderscore .
\section{Bufara}
\begin{itemize}
\item {Grp. gram.:f.}
\end{itemize}
\begin{itemize}
\item {Utilização:Ant.}
\end{itemize}
Objectos de bufarinheiro.
Quinquilharias:«\textunderscore ...trinta bocetas de bufara.\textunderscore »(De um testamento de 1692)
(Der. regressiva de \textunderscore bufarinha\textunderscore ?)
\section{Bundaça}
\begin{itemize}
\item {Grp. gram.:f.}
\end{itemize}
\begin{itemize}
\item {Proveniência:(De \textunderscore bunda\textunderscore )}
\end{itemize}
Nádegas grandes.
\section{Bundana}
\begin{itemize}
\item {Grp. gram.:f.}
\end{itemize}
\begin{itemize}
\item {Proveniência:(De \textunderscore bunda\textunderscore )}
\end{itemize}
Nádegas grandes.
\section{Bundão}
\begin{itemize}
\item {Grp. gram.:m.}
\end{itemize}
\begin{itemize}
\item {Utilização:Bras. de San-Paulo}
\end{itemize}
\begin{itemize}
\item {Proveniência:(De \textunderscore bunda\textunderscore )}
\end{itemize}
Nádegas grandes.
\section{Bupréstides}
\begin{itemize}
\item {Grp. gram.:m. pl.}
\end{itemize}
Gênero de insectos, que tem por typo o bupreste. Cf. P. Moraes, \textunderscore Zool. Elem.\textunderscore 
\section{Burgel}
\begin{itemize}
\item {Grp. gram.:m.}
\end{itemize}
\begin{itemize}
\item {Utilização:Ant.}
\end{itemize}
O mesmo que \textunderscore burguês\textunderscore , habitante de burgo. Cf. \textunderscore Port. Ant. e Mod.\textunderscore , vb. \textunderscore burgo\textunderscore .
\section{Burocracial}
\begin{itemize}
\item {Grp. gram.:adj.}
\end{itemize}
\begin{itemize}
\item {Utilização:Neol.}
\end{itemize}
Relativo a burocracia:«\textunderscore impertinência burocracial.\textunderscore »R. Jorge, \textunderscore Peste bubónica\textunderscore .
\section{Burreco}
\begin{itemize}
\item {Grp. gram.:m.}
\end{itemize}
\begin{itemize}
\item {Utilização:Prov.}
\end{itemize}
Burro fraco, ordinário.
\section{Busano}
\begin{itemize}
\item {Grp. gram.:m.}
\end{itemize}
Pedra de construcção, semelhante ao mármore, dos terrenos cretáceos médios.
\section{Butirómetro}
\begin{itemize}
\item {Grp. gram.:m.}
\end{itemize}
\begin{itemize}
\item {Proveniência:(Do gr. \textunderscore buturon\textunderscore  + \textunderscore metron\textunderscore )}
\end{itemize}
Instrumento, para reconhecer a proporção de manteiga, que há no leite.
\section{Butyrómetro}
\begin{itemize}
\item {Grp. gram.:m.}
\end{itemize}
\begin{itemize}
\item {Proveniência:(Do gr. \textunderscore buturon\textunderscore  + \textunderscore metron\textunderscore )}
\end{itemize}
Instrumento, para reconhecer a proporção de manteiga, que há no leite.
\section{Buxeiro}
\begin{itemize}
\item {Grp. gram.:m.}
\end{itemize}
Planta, o mesmo que \textunderscore buxo\textunderscore .
\section{Buxo-da-rocha}
\begin{itemize}
\item {Grp. gram.:m.}
\end{itemize}
Arvoreta da Ilha da Madeira, (\textunderscore catha dryandrí\textunderscore , Lowe).
\section{B}
\begin{itemize}
\item {fónica:bê}
\end{itemize}
\begin{itemize}
\item {Grp. gram.:m.}
\end{itemize}
\begin{itemize}
\item {Utilização:Mús.}
\end{itemize}
\begin{itemize}
\item {Grp. gram.:Adj.}
\end{itemize}
Letra, que representa som labial, e que occupa o segundo lugar no alphabeto português.
Abrev. de \textunderscore bom\textunderscore : \textunderscore no concurso de escrivão, obteve a nota de B\textunderscore .
O segundo grau da escala, na antiga notação alphabética.
Segundo, (falando-se de um número ou de um objecto, que faz parte de uma série): \textunderscore o livro B das Conservatórias do registo predial\textunderscore .
Que é de segunda classe, (falando-se de carruagens do caminho de ferro).
\section{Baalita}
\begin{itemize}
\item {Grp. gram.:m.}
\end{itemize}
\begin{itemize}
\item {Proveniência:(De \textunderscore Baal\textunderscore , n. p.)}
\end{itemize}
Sectário de Baal, divindade principal de alguns povos antigos.
\section{Baanda}
\begin{itemize}
\item {Grp. gram.:f.}
\end{itemize}
O mesmo que \textunderscore banda\textunderscore ^3.
\section{Baango-lango}
\begin{itemize}
\item {Grp. gram.:m.}
\end{itemize}
Árvore do Congo.
\section{Baarás}
\begin{itemize}
\item {Grp. gram.:m.}
\end{itemize}
Erva phosphorescente do Líbano, de fôlhas dentadas e haste branca.
\section{Baba}
\begin{itemize}
\item {Grp. gram.:f.}
\end{itemize}
\begin{itemize}
\item {Grp. gram.:M.}
\end{itemize}
\begin{itemize}
\item {Utilização:Des.}
\end{itemize}
\begin{itemize}
\item {Utilização:Fam.}
\end{itemize}
\begin{itemize}
\item {Proveniência:(Do ar. \textunderscore baba\textunderscore )}
\end{itemize}
Humor, que escorre da bôca.
Espuma, que sái da bôca de certos animaes.
Nome de algumas plantas.
Homem baboso, babeca.
\section{Baba}
\begin{itemize}
\item {Grp. gram.:m.}
\end{itemize}
Pequeno tambor, de fórma cónica, usado em Timor.
\section{Babá}
\begin{itemize}
\item {Grp. gram.:f.}
\end{itemize}
Espécie de pudim.
\section{Babaça}
\begin{itemize}
\item {Grp. gram.:m.  e  f.}
\end{itemize}
\begin{itemize}
\item {Utilização:Bras}
\end{itemize}
Irmão gêmeo ou irman gêmea.
(Do quimb.)
\section{Babacuara}
\begin{itemize}
\item {Grp. gram.:m.  e  f.}
\end{itemize}
\begin{itemize}
\item {Utilização:Bras}
\end{itemize}
\begin{itemize}
\item {Utilização:Bras. do Pará}
\end{itemize}
\begin{itemize}
\item {Grp. gram.:Adj.}
\end{itemize}
\begin{itemize}
\item {Utilização:Bras. do Ceará}
\end{itemize}
O mesmo que \textunderscore caipira\textunderscore .
Pessôa tola, apalermada.
Grande.
Poderoso.
(Do tupi)
\section{Baba-de-boi}
\begin{itemize}
\item {Grp. gram.:f.}
\end{itemize}
\begin{itemize}
\item {Utilização:Bras}
\end{itemize}
Planta, de que se formam sebes.
\section{Babadeira}
\begin{itemize}
\item {Grp. gram.:f.}
\end{itemize}
O mesmo que \textunderscore babadoiro\textunderscore .
\section{Baba-de-moça}
\begin{itemize}
\item {Grp. gram.:f.}
\end{itemize}
\begin{itemize}
\item {Utilização:Bras}
\end{itemize}
Espécie de doce líquido, feito de côco.
\section{Babadinho}
\begin{itemize}
\item {Grp. gram.:adj.}
\end{itemize}
\begin{itemize}
\item {Utilização:Fam.}
\end{itemize}
\begin{itemize}
\item {Proveniência:(De \textunderscore babado\textunderscore ^2)}
\end{itemize}
Que deseja muito alguma coisa.
Lamecha.
Extremoso.
\section{Babado}
\begin{itemize}
\item {Grp. gram.:m.}
\end{itemize}
\begin{itemize}
\item {Utilização:Bras}
\end{itemize}
Folho em pregas, para guarnição de saias, toalhas, etc.
\section{Babado}
\begin{itemize}
\item {Grp. gram.:adj.}
\end{itemize}
\begin{itemize}
\item {Utilização:Fig.}
\end{itemize}
Que deita baba.
Apaixonado.
\section{Babadoiro}
\begin{itemize}
\item {Grp. gram.:m.}
\end{itemize}
\begin{itemize}
\item {Proveniência:(De \textunderscore babar\textunderscore )}
\end{itemize}
Resguardo de pano ou de borracha, no peito das crianças, para que a baba ou a comida lhes não enxovalhe o vestuário.
\section{Babador}
\begin{itemize}
\item {Grp. gram.:m.}
\end{itemize}
\begin{itemize}
\item {Utilização:Bras}
\end{itemize}
O mesmo que \textunderscore babadoiro\textunderscore .
\section{Babadouro}
\begin{itemize}
\item {Grp. gram.:m.}
\end{itemize}
\begin{itemize}
\item {Proveniência:(De \textunderscore babar\textunderscore )}
\end{itemize}
Resguardo de pano ou de borracha, no peito das crianças, para que a baba ou a comida lhes não enxovalhe o vestuário.
\section{Babália}
\begin{itemize}
\item {Grp. gram.:f.}
\end{itemize}
Árvore indiana, (\textunderscore acacia arabica\textunderscore ), o mesmo que \textunderscore babul\textunderscore .
\section{Babanca}
\begin{itemize}
\item {Grp. gram.:m.}
\end{itemize}
\begin{itemize}
\item {Utilização:Prov.}
\end{itemize}
Palerma.
Pacóvio, lorpa.
\section{Babancas}
\begin{itemize}
\item {Grp. gram.:m.}
\end{itemize}
\begin{itemize}
\item {Utilização:Prov.}
\end{itemize}
Palerma.
Pacóvio, lorpa.
\section{Babão}
\begin{itemize}
\item {Grp. gram.:m.  e  adj.}
\end{itemize}
\begin{itemize}
\item {Proveniência:(De \textunderscore babar\textunderscore )}
\end{itemize}
O que se baba.
Idiota.
Bajoujo.
\section{Babar}
\begin{itemize}
\item {Grp. gram.:v. t.}
\end{itemize}
\begin{itemize}
\item {Grp. gram.:V. p.}
\end{itemize}
\begin{itemize}
\item {Utilização:Fam.}
\end{itemize}
Molhar com baba.
Deitar baba.
Falar com difficuldade, balbuciar.
Estar apaixonado; gostar muito.
\section{Babaré}
\begin{itemize}
\item {Grp. gram.:m.}
\end{itemize}
Alarme, rebate, aviso de que há ladrões na vizinhança.
Barulho de grande chusma de pretos, bem ou mal intencionados, (na África portuguesa).
(Do conc.)
\section{Babás}
\begin{itemize}
\item {Grp. gram.:m.}
\end{itemize}
\begin{itemize}
\item {Utilização:ant.}
\end{itemize}
\begin{itemize}
\item {Utilização:Fam.}
\end{itemize}
O mesmo que \textunderscore babeca\textunderscore .
\section{Babatar}
\begin{itemize}
\item {Grp. gram.:v. i.}
\end{itemize}
\begin{itemize}
\item {Utilização:Bras}
\end{itemize}
\begin{itemize}
\item {Proveniência:(Do quimb. \textunderscore cu-babata\textunderscore )}
\end{itemize}
Apalpar: tactear, como os cegos.
\section{Babau!}
\begin{itemize}
\item {Grp. gram.:interj.}
\end{itemize}
\begin{itemize}
\item {Utilização:pop.}
\end{itemize}
Foi-se!
Não tem remédio!
(Talvez t. quimbundo)
\section{Babeca}
\begin{itemize}
\item {Grp. gram.:m.}
\end{itemize}
\begin{itemize}
\item {Utilização:ant.}
\end{itemize}
\begin{itemize}
\item {Utilização:Fam.}
\end{itemize}
Homem néscio, basbaque.
(Cp. \textunderscore babar\textunderscore )
\section{Babeira}
\begin{itemize}
\item {Grp. gram.:f.}
\end{itemize}
\begin{itemize}
\item {Proveniência:(De \textunderscore babar\textunderscore )}
\end{itemize}
Antiga peça de armadura, que cobria a cara, do nariz para baixo.
Pequena abertura, por onde a água passa das canejas para os compartimentos crystallizadores, nalgumas salinas.
\section{Babeiro}
\begin{itemize}
\item {Grp. gram.:m.}
\end{itemize}
O mesmo que \textunderscore babadoiro\textunderscore .
\section{Babel}
\begin{itemize}
\item {Grp. gram.:f.}
\end{itemize}
\begin{itemize}
\item {Proveniência:(De \textunderscore Babel\textunderscore , n. p.)}
\end{itemize}
Confusão de línguas.
Algazarra.
Balbúrdia, confusão.
\section{Babélico}
\begin{itemize}
\item {Grp. gram.:adj.}
\end{itemize}
Relativo a babel.
Confuso, desordenado.
\section{Babete}
\begin{itemize}
\item {Grp. gram.:m.}
\end{itemize}
(V.babadoiro)
\section{Babi}
\begin{itemize}
\item {Grp. gram.:m.}
\end{itemize}
Sectário do babismo.
\section{Babiana}
\begin{itemize}
\item {Grp. gram.:f.}
\end{itemize}
Gênero de plantas irídeas.
\section{Babilarde}
\begin{itemize}
\item {Grp. gram.:m.}
\end{itemize}
Espécie de tutinegra.
\section{Babilónia}
\begin{itemize}
\item {Grp. gram.:f.}
\end{itemize}
\begin{itemize}
\item {Proveniência:(De \textunderscore Babylonia\textunderscore , n. p.)}
\end{itemize}
Grande confusão, babel.
\section{Babiloniado}
\begin{itemize}
\item {Grp. gram.:adj.}
\end{itemize}
Que se tornou babilónio. Cf. Vieira, XI, 200.
\section{Babilónico}
\begin{itemize}
\item {Grp. gram.:adj.}
\end{itemize}
Relativo a Babilónia.
\section{Babilónio}
\begin{itemize}
\item {Grp. gram.:adj.}
\end{itemize}
\begin{itemize}
\item {Utilização:Bras. do N}
\end{itemize}
\begin{itemize}
\item {Grp. gram.:M.}
\end{itemize}
Babilónico.
Muito grande.
Formidável.
Habitante da Babilónia.
\section{Babilonizar}
\begin{itemize}
\item {Grp. gram.:v. t.}
\end{itemize}
\begin{itemize}
\item {Utilização:Des.}
\end{itemize}
Tornar babilónico; corromper.
\section{Babirruça}
\begin{itemize}
\item {Grp. gram.:f.}
\end{itemize}
Quadrúpede indiano, semelhante ao porco, (\textunderscore sus babirusa\textunderscore , Lin.).
(Mal. \textunderscore babiruça\textunderscore , de \textunderscore babí\textunderscore , porco, e \textunderscore ruça\textunderscore , veado)
\section{Babirrussa}
\begin{itemize}
\item {Grp. gram.:f.}
\end{itemize}
(V.babirruça)
\section{Babismo}
\begin{itemize}
\item {Grp. gram.:m.}
\end{itemize}
\begin{itemize}
\item {Proveniência:(De \textunderscore Bab\textunderscore , n. p.)}
\end{itemize}
Seita religiosa, formada há pouco em a Pérsia, tendo por base a magia fundada nos algarismos.
\section{Bablaque}
\begin{itemize}
\item {Grp. gram.:m.}
\end{itemize}
Nome commercial da casca da acácia arábica ou babul.
\section{Bable}
\begin{itemize}
\item {Grp. gram.:m.}
\end{itemize}
Dialecto das Astúrias; o asturiano.
\section{Baboca}
\begin{itemize}
\item {Grp. gram.:f.}
\end{itemize}
\begin{itemize}
\item {Utilização:Bras. do N}
\end{itemize}
Esconderijo.
Sítio impenetrável.(V.biboca)
\section{Babordo}
\begin{itemize}
\item {fónica:bôr}
\end{itemize}
\begin{itemize}
\item {Grp. gram.:m.}
\end{itemize}
\begin{itemize}
\item {Utilização:Ant.}
\end{itemize}
\begin{itemize}
\item {Proveniência:(Do al. \textunderscore backboord\textunderscore )}
\end{itemize}
O mesmo que \textunderscore bombordo\textunderscore .
\section{Babosa}
\begin{itemize}
\item {Grp. gram.:f.}
\end{itemize}
\begin{itemize}
\item {Grp. gram.:Pl.}
\end{itemize}
\begin{itemize}
\item {Proveniência:(De \textunderscore baboso\textunderscore )}
\end{itemize}
Designação vulgar do aloés.
Variedade de ameixa alentejana.
Casta de uva, na região do Doiro, no Alentejo e Algarve.
Gênero de peixes acanthopterýgios.
\section{Baboseira}
\begin{itemize}
\item {Grp. gram.:f.}
\end{itemize}
\begin{itemize}
\item {Proveniência:(De \textunderscore baboso\textunderscore )}
\end{itemize}
Disparate; tolice.
Palavras de baboso.
\section{Babosice}
\begin{itemize}
\item {Grp. gram.:f.}
\end{itemize}
O mesmo que \textunderscore baboseira\textunderscore .
\section{Baboso}
\begin{itemize}
\item {Grp. gram.:adj.}
\end{itemize}
Que se baba.
Parvo; apaixonado.
\section{Babovismo}
\begin{itemize}
\item {Grp. gram.:m.}
\end{itemize}
\begin{itemize}
\item {Proveniência:(De \textunderscore Babeuf\textunderscore , n. p.)}
\end{itemize}
Doutrina social de Babeuf, segundo o qual todos os homens são iguaes absolutamente, sem outra differença que não seja a do sexo e a da idade.
\section{Babovista}
\begin{itemize}
\item {Grp. gram.:m.}
\end{itemize}
Sectário do babovismo.
\section{Babucha}
\begin{itemize}
\item {Grp. gram.:f.}
\end{itemize}
\begin{itemize}
\item {Proveniência:(Fr. \textunderscore babouche\textunderscore )}
\end{itemize}
Pantufo; chinela.
\section{Babuche}
\begin{itemize}
\item {Grp. gram.:f.}
\end{itemize}
O mesmo ou melhor que \textunderscore babucha\textunderscore .
\section{Babugem}
\begin{itemize}
\item {Grp. gram.:f.}
\end{itemize}
\begin{itemize}
\item {Proveniência:(De \textunderscore babar\textunderscore )}
\end{itemize}
Baba.
Espuma, formada pela água que se agita.
Resíduos de comida.
Quaesquer resíduos.
Bagatela.
\section{Babuíno}
\begin{itemize}
\item {Grp. gram.:m.}
\end{itemize}
\begin{itemize}
\item {Proveniência:(Do borg. \textunderscore babouin\textunderscore , menino de berço)}
\end{itemize}
Espécie de macaco, cynocéphalo da Guiné.
\section{Babujado}
\begin{itemize}
\item {Grp. gram.:adj.}
\end{itemize}
Sujo de baba.
\section{Babujar}
\begin{itemize}
\item {Grp. gram.:v. t.}
\end{itemize}
\begin{itemize}
\item {Proveniência:(De \textunderscore babugem\textunderscore )}
\end{itemize}
Sujar com baba ou babugem.
Lisonjear servilmente.
\section{Babul}
\begin{itemize}
\item {Grp. gram.:m.}
\end{itemize}
Pequena árvore, (\textunderscore acacia arabica\textunderscore ), cuja madeira é empregada, como excellente combustível, nas máquinas de vapor, e cujas fôlhas são bom alimento para o gado lanígero, nas regiões intertropicaes.
(Do conc.)
\section{Babunha}
\begin{itemize}
\item {Grp. gram.:f.}
\end{itemize}
Espécie de palmeira do Brasil.
\section{Babuzar}
\begin{itemize}
\item {Grp. gram.:m.}
\end{itemize}
Águia marítima, que se aninha nos rochedos e vive exclusivamente de peixes.
(Talvez do b. lat. \textunderscore balbuzare\textunderscore )
\section{Babylónia}
\begin{itemize}
\item {Grp. gram.:f.}
\end{itemize}
\begin{itemize}
\item {Proveniência:(De \textunderscore Babylonia\textunderscore , n. p.)}
\end{itemize}
Grande confusão, babel.
\section{Babyloniado}
\begin{itemize}
\item {Grp. gram.:adj.}
\end{itemize}
Que se tornou babylónio. Cf. Vieira, XI, 200.
\section{Babylónico}
\begin{itemize}
\item {Grp. gram.:adj.}
\end{itemize}
Relativo a Babylónia.
\section{Babylónio}
\begin{itemize}
\item {Grp. gram.:adj.}
\end{itemize}
\begin{itemize}
\item {Utilização:Bras. do N}
\end{itemize}
\begin{itemize}
\item {Grp. gram.:M.}
\end{itemize}
Babylónico.
Muito grande.
Formidável.
Habitante da Babylónia.
\section{Babylonizar}
\begin{itemize}
\item {Grp. gram.:v. t.}
\end{itemize}
\begin{itemize}
\item {Utilização:Des.}
\end{itemize}
Tornar babylónico; corromper.
\section{Bacaba}
\begin{itemize}
\item {Grp. gram.:m.}
\end{itemize}
Fruto da bacabeira.
Bebida, extrahida dêsse fruto.
\section{Bacabada}
\begin{itemize}
\item {Grp. gram.:f.}
\end{itemize}
\begin{itemize}
\item {Utilização:Bras}
\end{itemize}
\begin{itemize}
\item {Proveniência:(De \textunderscore bacaba\textunderscore )}
\end{itemize}
Iguaria, feita com o fruto da bacabeira.
\section{Bacabai}
\begin{itemize}
\item {Grp. gram.:m.}
\end{itemize}
O mesmo que \textunderscore bacabeira\textunderscore .
\section{Bacabal}
\begin{itemize}
\item {Grp. gram.:m.}
\end{itemize}
Lugar, onde crescem bacabeiras; mato de bacabeiras.
\section{Bacabeira}
\begin{itemize}
\item {Grp. gram.:f.}
\end{itemize}
Espécie de palmeira do Brasil, (\textunderscore oenocarpus bacaba\textunderscore ).
\section{Bacahíris}
\begin{itemize}
\item {Grp. gram.:m. pl.}
\end{itemize}
Índios do Brasil, que dominavam em Mato-Grosso.
\section{Bacaíris}
\begin{itemize}
\item {Grp. gram.:m. pl.}
\end{itemize}
Índios do Brasil, que dominavam em Mato-Grosso.
\section{Bacalar}
\begin{itemize}
\item {Grp. gram.:m.}
\end{itemize}
O mesmo que \textunderscore bacalária\textunderscore .
\section{Bacalares}
\begin{itemize}
\item {Grp. gram.:m. pl.}
\end{itemize}
\begin{itemize}
\item {Utilização:Ant.}
\end{itemize}
Peças de madeira, que se pregam na coberta da popa dos navios.
\section{Bacalária}
\begin{itemize}
\item {Grp. gram.:f.}
\end{itemize}
\begin{itemize}
\item {Utilização:Ant.}
\end{itemize}
Prédio rústico, que continha dez ou doze casaes, cada um dos quaes era servido com uma junta de gado vacum.
(Por \textunderscore vaccalária\textunderscore , de \textunderscore vacca\textunderscore ?)
\section{Bacalário}
\begin{itemize}
\item {Grp. gram.:m.}
\end{itemize}
\begin{itemize}
\item {Utilização:Ant.}
\end{itemize}
Vassallo de condição inferior, que cultivava ou possuía uma bacalária.
(B. lat. \textunderscore baccalarius\textunderscore )
\section{Bacalaureato}
\begin{itemize}
\item {Grp. gram.:m.}
\end{itemize}
\begin{itemize}
\item {Utilização:Ant.}
\end{itemize}
O mesmo que \textunderscore bacharelado\textunderscore .
(B. lat. \textunderscore bacchalaureatus\textunderscore )
\section{Bacalhau}
\begin{itemize}
\item {Grp. gram.:m.}
\end{itemize}
\begin{itemize}
\item {Utilização:Bras}
\end{itemize}
\begin{itemize}
\item {Grp. gram.:Pl.}
\end{itemize}
\begin{itemize}
\item {Utilização:Prov.}
\end{itemize}
\begin{itemize}
\item {Utilização:beir.}
\end{itemize}
\begin{itemize}
\item {Proveniência:(Do b. lat. \textunderscore baccalaureus\textunderscore )}
\end{itemize}
Peixe, da fam. dos gádidas.
Azorrague.
Grandes collarinhos pendentes, que se usavam com certos uniformes.
Collarinhos altos.
Jôgo de rapazes, espécie de bilharda.
Mulher ordinária, tagarela e de costumes pouco exemplares.
\section{Bacalhoada}
\begin{itemize}
\item {Grp. gram.:f.}
\end{itemize}
Pancada com bacalhau.
Grande porção de bacalhau.
Guisado de bacalhau.
\section{Bacalhoeiro}
\begin{itemize}
\item {Grp. gram.:m.}
\end{itemize}
\begin{itemize}
\item {Grp. gram.:Adj.}
\end{itemize}
\begin{itemize}
\item {Utilização:Fam.}
\end{itemize}
\begin{itemize}
\item {Utilização:Prov.}
\end{itemize}
\begin{itemize}
\item {Utilização:beir.}
\end{itemize}
Aquelle que vende bacalhau.
Navio, empregado na pesca do bacalhau.
Que gosta muito de bacalhau.
Grosseiro.
Falador, bisbilhoteiro.
\section{Bacalio}
\begin{itemize}
\item {Grp. gram.:m.}
\end{itemize}
O mesmo que \textunderscore feudo\textunderscore :«\textunderscore Portugal, dado em bacalio a um soldado estrangeiro\textunderscore ». Garrett, \textunderscore Port. na Bal.\textunderscore , 57.
\section{Bacama}
\begin{itemize}
\item {Grp. gram.:f.}
\end{itemize}
\begin{itemize}
\item {Utilização:T. da África port}
\end{itemize}
O mesmo que \textunderscore espôsa\textunderscore .
\section{Bacamartada}
\begin{itemize}
\item {Grp. gram.:f.}
\end{itemize}
Tiro de bacamarte.
\section{Bacamarte}
\begin{itemize}
\item {Grp. gram.:m.}
\end{itemize}
\begin{itemize}
\item {Utilização:Pop.}
\end{itemize}
\begin{itemize}
\item {Utilização:Bras}
\end{itemize}
Arma de fogo, de cano curto e largo.
Livro velho e pesado.
Planta brasileira.
Homem gordo e inútil.
(Cp. \textunderscore bracamarte\textunderscore )
\section{Bacanal}
\begin{itemize}
\item {Grp. gram.:f.}
\end{itemize}
\begin{itemize}
\item {Proveniência:(Lat. \textunderscore bacchanal\textunderscore )}
\end{itemize}
Festa em honra de Baco.
Banquete crapuloso.
Orgia.
Libertinagem.
\section{Bacanálias}
\begin{itemize}
\item {Grp. gram.:f. pl.}
\end{itemize}
\begin{itemize}
\item {Proveniência:(Lat. \textunderscore bacchanalia\textunderscore )}
\end{itemize}
O mesmo que [[bacanaes|bacanal]].
\section{Bacante}
\begin{itemize}
\item {Grp. gram.:f.}
\end{itemize}
\begin{itemize}
\item {Utilização:Fig.}
\end{itemize}
\begin{itemize}
\item {Utilização:Bot.}
\end{itemize}
\begin{itemize}
\item {Utilização:Zool.}
\end{itemize}
\begin{itemize}
\item {Proveniência:(De \textunderscore Baccho\textunderscore , n. p.)}
\end{itemize}
Sacerdotisa de Baco.
Mulher dissoluta.
Planta herbácea, da fam. das compostas.
Espécie de borboleta.
\section{Bacântico}
\begin{itemize}
\item {Grp. gram.:adj.}
\end{itemize}
Relativo ás bacantes; próprio de bacantes, orgíaco.
\section{Bacar}
\begin{itemize}
\item {Grp. gram.:m.}
\end{itemize}
Armazem de panos, na antiga Índia portuguesa.
(Talvez êrro de cópia, de escrita ou de composição, por \textunderscore baçar\textunderscore  ou \textunderscore bazar\textunderscore )
\section{Bacará}
\begin{itemize}
\item {Grp. gram.:m.}
\end{itemize}
\begin{itemize}
\item {Proveniência:(Fr. \textunderscore baccara\textunderscore )}
\end{itemize}
Espécie de jôgo de azar.
\section{Bacará}
\begin{itemize}
\item {Grp. gram.:m.}
\end{itemize}
Crystal em obra, da fábrica da cidade de \textunderscore Baccarat\textunderscore .
\section{Bacarahi}
\begin{itemize}
\item {Grp. gram.:m.}
\end{itemize}
\begin{itemize}
\item {Utilização:Bras. do S}
\end{itemize}
\begin{itemize}
\item {Proveniência:(Do port. \textunderscore vaca\textunderscore  + guar. \textunderscore tai\textunderscore )}
\end{itemize}
Féto de vaca, que se aproveita como alimento appetitoso, quando se mata a rês em estado de prenhez.
\section{Bacaraí}
\begin{itemize}
\item {Grp. gram.:m.}
\end{itemize}
\begin{itemize}
\item {Utilização:Bras. do S}
\end{itemize}
\begin{itemize}
\item {Proveniência:(Do port. \textunderscore vaca\textunderscore  + guar. \textunderscore tai\textunderscore )}
\end{itemize}
Féto de vaca, que se aproveita como alimento appetitoso, quando se mata a rês em estado de prenhez.
\section{Bacárida}
\begin{itemize}
\item {Grp. gram.:f.}
\end{itemize}
O mesmo que \textunderscore bácaro\textunderscore ^1.
\section{Bacarija}
\begin{itemize}
\item {Grp. gram.:f.}
\end{itemize}
Nardo silvestre.
\section{Bácaris}
\begin{itemize}
\item {Grp. gram.:m.}
\end{itemize}
O mesmo que \textunderscore bácaro\textunderscore ^1.
\section{Bácaro}
\begin{itemize}
\item {Grp. gram.:m.}
\end{itemize}
\begin{itemize}
\item {Proveniência:(Lat. \textunderscore baccar\textunderscore  = gr. \textunderscore bakharis\textunderscore )}
\end{itemize}
Planta, com que, segundo Plinio, se fabricavam corôas.
Talvez uma espécie de nardo, conhecida por \textunderscore luvas-de-santa-maria\textunderscore . Cf. Ficalho, \textunderscore Flora dos Lusíadas\textunderscore ; Latino, \textunderscore Camões\textunderscore , 49; \textunderscore Lusíadas\textunderscore , III, 97.
\section{Bácaro}
\begin{itemize}
\item {Grp. gram.:m.}
\end{itemize}
\begin{itemize}
\item {Utilização:Des.}
\end{itemize}
O mesmo ou melhor que \textunderscore bácoro\textunderscore . Cf. \textunderscore Eufrosina\textunderscore , 55.
\section{Bacatela}
\begin{itemize}
\item {Grp. gram.:f.}
\end{itemize}
\begin{itemize}
\item {Utilização:Bras}
\end{itemize}
\begin{itemize}
\item {Utilização:Pop.}
\end{itemize}
O mesmo que \textunderscore bagatela\textunderscore .
\section{Bacaxi}
\begin{itemize}
\item {Grp. gram.:m.}
\end{itemize}
\begin{itemize}
\item {Utilização:Ant.}
\end{itemize}
O mesmo que \textunderscore bocaxi\textunderscore .
\section{Baccalar}
\begin{itemize}
\item {Grp. gram.:m.}
\end{itemize}
O mesmo que \textunderscore baccalária\textunderscore .
\section{Baccalares}
\begin{itemize}
\item {Grp. gram.:m. pl.}
\end{itemize}
\begin{itemize}
\item {Utilização:Ant.}
\end{itemize}
Peças de madeira, que se pregam na coberta da popa dos navios.
\section{Baccalária}
\begin{itemize}
\item {Grp. gram.:f.}
\end{itemize}
\begin{itemize}
\item {Utilização:Ant.}
\end{itemize}
Prédio rústico, que continha dez ou doze casaes, cada um dos quaes era servido com uma junta de gado vacum.
(Por \textunderscore vaccalária\textunderscore , de \textunderscore vacca\textunderscore ?)
\section{Baccalário}
\begin{itemize}
\item {Grp. gram.:m.}
\end{itemize}
\begin{itemize}
\item {Utilização:Ant.}
\end{itemize}
Vassallo de condição inferior, que cultivava ou possuía uma baccalária.
(B. lat. \textunderscore baccalarius\textunderscore )
\section{Baccalaureato}
\begin{itemize}
\item {Grp. gram.:m.}
\end{itemize}
\begin{itemize}
\item {Utilização:Ant.}
\end{itemize}
O mesmo que \textunderscore bacharelado\textunderscore .
(B. lat. \textunderscore bacchalaureatus\textunderscore )
\section{Baccalio}
\begin{itemize}
\item {Grp. gram.:m.}
\end{itemize}
O mesmo que \textunderscore feudo\textunderscore :«\textunderscore Portugal, dado em baccalio a um soldado estrangeiro\textunderscore ». Garrett, \textunderscore Port. na Bal.\textunderscore , 57.
\section{Baccará}
\begin{itemize}
\item {Grp. gram.:m.}
\end{itemize}
\begin{itemize}
\item {Proveniência:(Fr. \textunderscore baccara\textunderscore )}
\end{itemize}
Espécie de jôgo de azar.
\section{Baccará}
\begin{itemize}
\item {Grp. gram.:m.}
\end{itemize}
Crystal em obra, da fábrica da cidade de \textunderscore Baccarat\textunderscore .
\section{Baccárida}
\begin{itemize}
\item {Grp. gram.:f.}
\end{itemize}
O mesmo que \textunderscore báccharo\textunderscore .
\section{Bacchanal}
\begin{itemize}
\item {fónica:ca}
\end{itemize}
\begin{itemize}
\item {Grp. gram.:f.}
\end{itemize}
\begin{itemize}
\item {Proveniência:(Lat. \textunderscore bacchanal\textunderscore )}
\end{itemize}
Festa em honra de Baccho.
Banquete crapuloso.
Orgia.
Libertinagem.
\section{Bacchanálias}
\begin{itemize}
\item {fónica:ca}
\end{itemize}
\begin{itemize}
\item {Grp. gram.:f. pl.}
\end{itemize}
\begin{itemize}
\item {Proveniência:(Lat. \textunderscore bacchanalia\textunderscore )}
\end{itemize}
O mesmo que [[bacchanaes|bacchanal]].
\section{Bacchante}
\begin{itemize}
\item {fónica:can}
\end{itemize}
\begin{itemize}
\item {Grp. gram.:f.}
\end{itemize}
\begin{itemize}
\item {Utilização:Fig.}
\end{itemize}
\begin{itemize}
\item {Utilização:Bot.}
\end{itemize}
\begin{itemize}
\item {Utilização:Zool.}
\end{itemize}
\begin{itemize}
\item {Proveniência:(De \textunderscore Baccho\textunderscore , n. p.)}
\end{itemize}
Sacerdotisa de Baccho.
Mulher dissoluta.
Planta herbácea, da fam. das compostas.
Espécie de borboleta.
\section{Bacchântico}
\begin{itemize}
\item {Grp. gram.:adj.}
\end{itemize}
Relativo ás bacchantes; próprio de bacchantes, orgíaco.
\section{Báccharis}
\begin{itemize}
\item {fónica:ca}
\end{itemize}
\begin{itemize}
\item {Grp. gram.:m.}
\end{itemize}
O mesmo que \textunderscore báccharo\textunderscore .
\section{Báccharo}
\begin{itemize}
\item {fónica:ca}
\end{itemize}
\begin{itemize}
\item {Grp. gram.:m.}
\end{itemize}
\begin{itemize}
\item {Proveniência:(Lat. \textunderscore baccar\textunderscore  = gr. \textunderscore bakharis\textunderscore )}
\end{itemize}
Planta, com que, segundo Plinio, se fabricavam corôas.
Talvez uma espécie de nardo, conhecida por \textunderscore luvas-de-santa-maria\textunderscore . Cf. Ficalho, \textunderscore Flora dos Lusíadas\textunderscore ; Latino, \textunderscore Camões\textunderscore , 49; \textunderscore Lusíadas\textunderscore , III, 97.
\section{Bácchico}
\begin{itemize}
\item {fónica:qui}
\end{itemize}
\begin{itemize}
\item {Grp. gram.:adj.}
\end{itemize}
\begin{itemize}
\item {Proveniência:(Lat. \textunderscore bacchicus\textunderscore )}
\end{itemize}
Relativo a Baccho, ou ao vinho.
Em que há orgia: \textunderscore festas bácchicas\textunderscore .
\section{Bacchio}
\begin{itemize}
\item {fónica:qui}
\end{itemize}
\begin{itemize}
\item {Grp. gram.:m.}
\end{itemize}
\begin{itemize}
\item {Proveniência:(Gr. \textunderscore bakkheius\textunderscore )}
\end{itemize}
Pé de verso grego ou latino, com uma sýllaba breve e duas longas.
\section{Bacchista}
\begin{itemize}
\item {fónica:quis}
\end{itemize}
\begin{itemize}
\item {Grp. gram.:m. ,  f.  e  adj.}
\end{itemize}
\begin{itemize}
\item {Proveniência:(De \textunderscore Baccho\textunderscore , n. p.)}
\end{itemize}
Pessôa dada á embriaguez.
Que gósta de orgias.
\section{Bacciano}
\begin{itemize}
\item {Grp. gram.:adj.}
\end{itemize}
\begin{itemize}
\item {Proveniência:(Do lat. \textunderscore bacca\textunderscore )}
\end{itemize}
Semelhante á baga.
\section{Baccífero}
\begin{itemize}
\item {Grp. gram.:adj.}
\end{itemize}
\begin{itemize}
\item {Proveniência:(Lat. \textunderscore baccifer\textunderscore )}
\end{itemize}
Que tem ou produz baga.
\section{Bacciforme}
\begin{itemize}
\item {Grp. gram.:adj.}
\end{itemize}
\begin{itemize}
\item {Proveniência:(Do lat. \textunderscore bacca\textunderscore  + \textunderscore forma\textunderscore )}
\end{itemize}
Que tem fórma de baga.
\section{Baccívoro}
\begin{itemize}
\item {Grp. gram.:adj.}
\end{itemize}
\begin{itemize}
\item {Proveniência:(Do lat. \textunderscore bacca\textunderscore  + \textunderscore vorare\textunderscore )}
\end{itemize}
Que se alimenta de baga.
\section{Baceira}
\begin{itemize}
\item {Grp. gram.:f.}
\end{itemize}
\begin{itemize}
\item {Proveniência:(De \textunderscore baço\textunderscore )}
\end{itemize}
Febre carbunculosa dos animaes, mortífera e contagiosa, determinada por um bacillo, (\textunderscore bacillus anthracis\textunderscore ), que vem misturado nas forragens produzidas por terrenos onde se enterraram animaes carbunculosos.
\section{Baceiro}
\begin{itemize}
\item {Grp. gram.:adj.}
\end{itemize}
Relativo ao \textunderscore baço\textunderscore .
\section{Bacela}
\begin{itemize}
\item {Grp. gram.:f.}
\end{itemize}
Planta chenopodiácea da Índia portuguesa, (\textunderscore bacella rubra\textunderscore , Lin.) Cf. Delgado, \textunderscore Flora\textunderscore .
\section{Bacelada}
\begin{itemize}
\item {Grp. gram.:f.}
\end{itemize}
Plantação de bacêlos.
\section{Bacelar}
\begin{itemize}
\item {Grp. gram.:v. t.}
\end{itemize}
Plantar de bacêlos.
\section{Baceleira}
\begin{itemize}
\item {Grp. gram.:f.}
\end{itemize}
O mesmo que \textunderscore bacêlo\textunderscore .
\section{Baceleiro}
\begin{itemize}
\item {Grp. gram.:m.}
\end{itemize}
Aquelle que planta bacêlos.
O mesmo que \textunderscore bacêlo\textunderscore ; vide, destinada a sêr plantada. Cf. \textunderscore Bibl. da Gente do Campo\textunderscore , 322.
\section{Bacelia}
\begin{itemize}
\item {Grp. gram.:f.}
\end{itemize}
(V.bacelada)
\section{Bacellada}
\begin{itemize}
\item {Grp. gram.:f.}
\end{itemize}
Plantação de bacêllos.
\section{Bacellar}
\begin{itemize}
\item {Grp. gram.:v. t.}
\end{itemize}
Plantar de bacêllos.
\section{Bacelleira}
\begin{itemize}
\item {Grp. gram.:f.}
\end{itemize}
O mesmo que \textunderscore bacêllo\textunderscore .
\section{Bacelleiro}
\begin{itemize}
\item {Grp. gram.:m.}
\end{itemize}
Aquelle que planta bacêllos.
O mesmo que \textunderscore bacêllo\textunderscore ; vide, destinada a sêr plantada. Cf. \textunderscore Bibl. da Gente do Campo\textunderscore , 322.
\section{Bacellia}
\begin{itemize}
\item {Grp. gram.:f.}
\end{itemize}
(V.bacellada)
\section{Bacêllo}
\begin{itemize}
\item {Grp. gram.:m.}
\end{itemize}
\begin{itemize}
\item {Proveniência:(Lat. \textunderscore bacillum\textunderscore )}
\end{itemize}
Vara de videira, com que se reproduz a vinha, por meio de plantio.
Vinha nova.
\section{Bacêlo}
\begin{itemize}
\item {Grp. gram.:m.}
\end{itemize}
\begin{itemize}
\item {Proveniência:(Lat. \textunderscore bacillum\textunderscore )}
\end{itemize}
Vara de videira, com que se reproduz a vinha, por meio de plantio.
Vinha nova.
\section{Bachá}
\begin{itemize}
\item {Grp. gram.:m.}
\end{itemize}
(V.paçhá)
\section{Bachaleria}
\begin{itemize}
\item {Grp. gram.:f.}
\end{itemize}
\begin{itemize}
\item {Utilização:Des.}
\end{itemize}
O mesmo que \textunderscore bacharelice\textunderscore . Cf. Arn. Gama, \textunderscore Última Dona\textunderscore , 58.
\section{Bacharel}
\begin{itemize}
\item {Grp. gram.:m.}
\end{itemize}
\begin{itemize}
\item {Utilização:Fam.}
\end{itemize}
\begin{itemize}
\item {Utilização:Ant.}
\end{itemize}
\begin{itemize}
\item {Utilização:Ant.}
\end{itemize}
\begin{itemize}
\item {Proveniência:(Fr. \textunderscore bachelier\textunderscore , do b. lat. \textunderscore baccalaureus\textunderscore )}
\end{itemize}
Aquelle que fazia exame, ficando approvado, das disciplinas do 4.^o anno de qualquer faculdade da Universidade.
Aquelle que concluiu formatura numa faculdade da Universidade.
Indivíduo falador, tagarela.
Aquelle que, sendo admittido numa Ordem militar, não tinha riqueza sufficiente ou não tinha ainda idade para têr pendão e caldeira.
Beneficiado de cathedral.
\section{Bacharela}
\begin{itemize}
\item {Grp. gram.:f.}
\end{itemize}
\begin{itemize}
\item {Utilização:Pop.}
\end{itemize}
Mulher palradora, sabichona. Cf. Castilho, \textunderscore Fausto\textunderscore , 266.
\section{Bacharelada}
\begin{itemize}
\item {Grp. gram.:f.}
\end{itemize}
\begin{itemize}
\item {Proveniência:(De \textunderscore bacharel\textunderscore )}
\end{itemize}
Palavriado pretensioso.
\section{Bacharelado}
\begin{itemize}
\item {Grp. gram.:m.}
\end{itemize}
O grau de bacharel.
\section{Bacharelando}
\begin{itemize}
\item {Grp. gram.:m.}
\end{itemize}
Aquelle que estuda na Universidade, preparando-se para receber o grau de bacharel.
\section{Bacharelar}
\begin{itemize}
\item {Grp. gram.:v. i.}
\end{itemize}
\begin{itemize}
\item {Grp. gram.:V. p.}
\end{itemize}
Falar muito, pretensiosamente.
Tomar o grau de bacharel; tornar-se bacharel.
\section{Bachareleiro}
\begin{itemize}
\item {Grp. gram.:adj.}
\end{itemize}
\begin{itemize}
\item {Utilização:Des.}
\end{itemize}
Próprio de bacharel.
Palavroso. Cf. Filinto, VIII, 225.
\section{Bacharelice}
\begin{itemize}
\item {Grp. gram.:f.}
\end{itemize}
Bacharelada.
Costume de falar á tôa.
\section{Bacharelismo}
\begin{itemize}
\item {Grp. gram.:m.}
\end{itemize}
O mesmo que \textunderscore bacharelice\textunderscore .
\section{Bacharelo}
\begin{itemize}
\item {Grp. gram.:adj.}
\end{itemize}
\begin{itemize}
\item {Utilização:Des.}
\end{itemize}
O mesmo que \textunderscore bachareleiro\textunderscore :«\textunderscore iras bacharelas\textunderscore ». Filinto, X, 135.«\textunderscore Com língua bacharela\textunderscore ». \textunderscore Idem\textunderscore , VII, 85.
\section{Bachicar}
\begin{itemize}
\item {Grp. gram.:v. t.}
\end{itemize}
\begin{itemize}
\item {Utilização:Prov.}
\end{itemize}
\begin{itemize}
\item {Utilização:trasm.}
\end{itemize}
\begin{itemize}
\item {Utilização:Prov.}
\end{itemize}
\begin{itemize}
\item {Utilização:beir.}
\end{itemize}
Chapinhar; bater na água com os pés ou com as mãos, principalmente com as mãos.
Borrifar.
\section{Bacia}
\begin{itemize}
\item {Grp. gram.:f.}
\end{itemize}
\begin{itemize}
\item {Utilização:Anat.}
\end{itemize}
\begin{itemize}
\item {Utilização:Constr.}
\end{itemize}
Vaso redondo e largo, que serve principalmente para lavagem de rosto e mãos ou de pés.
Prato de balança.
Bacio.
Prato, com um recorte semi-circular na borda, e em que se deita a água com que se ensabôa a cara, antes de se fazer a barba.
Bandeja.
Peça de metal, côncava, em que trabalha o puxador das campaínhas, á entrada de algumas habitações.
Conjunto de vertentes, que ladeiam o rio.
Depressão de terreno, cercada de montanhas.
Canal, de paredes ósseas, que termina o tronco do corpo humano.
Pedra, em que assentam as grades de sacada e o peitoril do púlpito.
(Cp. \textunderscore bacio\textunderscore )
\section{Baciada}
\begin{itemize}
\item {Grp. gram.:f.}
\end{itemize}
Porção de líquido, contida numa bacia.
\section{Baciado}
\begin{itemize}
\item {Grp. gram.:adj.}
\end{itemize}
Que tem côr baça:«\textunderscore canarim baciado\textunderscore ». Bocage.
\section{Bacial}
\begin{itemize}
\item {Grp. gram.:adj.}
\end{itemize}
Relativo a \textunderscore bacio\textunderscore :«\textunderscore a tampa bacial pôs na cabeça\textunderscore ». Macedo, \textunderscore Burros\textunderscore , 28.
\section{Bacieta}
\begin{itemize}
\item {fónica:ê}
\end{itemize}
(V.bacineta)
\section{Bacila}
\begin{itemize}
\item {Grp. gram.:f.}
\end{itemize}
\begin{itemize}
\item {Proveniência:(De \textunderscore bacillo\textunderscore )}
\end{itemize}
Planta umbellífera, também conhecida por \textunderscore funcho marítimo\textunderscore .
\section{Bacilar}
\begin{itemize}
\item {Grp. gram.:adj.}
\end{itemize}
Relativo a bacilo.
Delgado e comprido como uma varinha.
\section{Bacilária}
\begin{itemize}
\item {Grp. gram.:f.}
\end{itemize}
\begin{itemize}
\item {Proveniência:(De \textunderscore bacilla\textunderscore )}
\end{itemize}
Gênero de algas, de frústulos baciliformes.
\section{Bacilário}
\begin{itemize}
\item {Grp. gram.:adj.}
\end{itemize}
\begin{itemize}
\item {Grp. gram.:M. pl.}
\end{itemize}
\begin{itemize}
\item {Proveniência:(De \textunderscore bacillo\textunderscore )}
\end{itemize}
O mesmo que \textunderscore bacilar\textunderscore .
Família de infusórios polygástricos.
\section{Bacilento}
\begin{itemize}
\item {Grp. gram.:adj.}
\end{itemize}
\begin{itemize}
\item {Utilização:Prov.}
\end{itemize}
\begin{itemize}
\item {Utilização:minh.}
\end{itemize}
O mesmo que \textunderscore macilento\textunderscore .
(Cp. \textunderscore baço\textunderscore )
\section{Baciliforme}
\begin{itemize}
\item {Grp. gram.:adj.}
\end{itemize}
\begin{itemize}
\item {Proveniência:(Do lat. \textunderscore bacillus\textunderscore  + \textunderscore forma\textunderscore )}
\end{itemize}
Delgado e comprido como uma varinha.
\section{Bacilla}
\begin{itemize}
\item {Grp. gram.:f.}
\end{itemize}
\begin{itemize}
\item {Proveniência:(De \textunderscore bacillo\textunderscore )}
\end{itemize}
Planta umbellífera, também conhecida por \textunderscore funcho marítimo\textunderscore .
\section{Bacillar}
\begin{itemize}
\item {Grp. gram.:adj.}
\end{itemize}
Relativo a bacillo.
Delgado e comprido como uma varinha.
\section{Bacillária}
\begin{itemize}
\item {Grp. gram.:f.}
\end{itemize}
\begin{itemize}
\item {Proveniência:(De \textunderscore bacilla\textunderscore )}
\end{itemize}
Gênero de algas, de frústulos bacilliformes.
\section{Bacillário}
\begin{itemize}
\item {Grp. gram.:adj.}
\end{itemize}
\begin{itemize}
\item {Grp. gram.:M. pl.}
\end{itemize}
\begin{itemize}
\item {Proveniência:(De \textunderscore bacillo\textunderscore )}
\end{itemize}
O mesmo que \textunderscore bacillar\textunderscore .
Família de infusórios polygástricos.
\section{Bacilliforme}
\begin{itemize}
\item {Grp. gram.:adj.}
\end{itemize}
\begin{itemize}
\item {Proveniência:(Do lat. \textunderscore bacillus\textunderscore  + \textunderscore forma\textunderscore )}
\end{itemize}
Delgado e comprido como uma varinha.
\section{Bacillo}
\begin{itemize}
\item {Grp. gram.:m.}
\end{itemize}
\begin{itemize}
\item {Proveniência:(Lat. \textunderscore bacillus\textunderscore )}
\end{itemize}
Vibrião.
Bactéria alongada, em fórma de bastonete: \textunderscore há bacillos que não são pathogênicos\textunderscore .
\section{Bacillose}
\begin{itemize}
\item {Grp. gram.:f.}
\end{itemize}
Qualquer doença, determinada por bacillos.
\section{Bacilo}
\begin{itemize}
\item {Grp. gram.:m.}
\end{itemize}
\begin{itemize}
\item {Proveniência:(Lat. \textunderscore bacillus\textunderscore )}
\end{itemize}
Vibrião.
Bactéria alongada, em fórma de bastonete: \textunderscore há bacilos que não são pathogênicos\textunderscore .
\section{Bacilose}
\begin{itemize}
\item {Grp. gram.:f.}
\end{itemize}
Qualquer doença, determinada por bacilos.
\section{Bacineta}
\begin{itemize}
\item {fónica:nê}
\end{itemize}
\begin{itemize}
\item {Grp. gram.:f.}
\end{itemize}
\begin{itemize}
\item {Proveniência:(Fr. \textunderscore bassinette\textunderscore )}
\end{itemize}
Pequena bacia.
\section{Bacinete}
\begin{itemize}
\item {fónica:nê}
\end{itemize}
\begin{itemize}
\item {Grp. gram.:m.}
\end{itemize}
\begin{itemize}
\item {Utilização:Anat.}
\end{itemize}
\begin{itemize}
\item {Proveniência:(Fr. \textunderscore bassinet\textunderscore )}
\end{itemize}
Peça da armadura, por baixo do capacete.
Reservatório renal, onde começam os ureteres.
\section{Bacinica}
\begin{itemize}
\item {Grp. gram.:f.}
\end{itemize}
\begin{itemize}
\item {Utilização:Ant.}
\end{itemize}
O mesmo que \textunderscore bacineta\textunderscore .
\section{Bacinico}
\begin{itemize}
\item {Grp. gram.:m.}
\end{itemize}
Bacio pequeno.
\section{Baciniqueiro}
\begin{itemize}
\item {Grp. gram.:m.  e  adj.}
\end{itemize}
O que limpa ou despeja bacios ou bacinicos.
\section{Bacio}
\begin{itemize}
\item {Grp. gram.:m.}
\end{itemize}
\begin{itemize}
\item {Utilização:Ant.}
\end{itemize}
Bacia de cama; bispote.
Prato grande e achatado, em fórma de bandeja.
(Seg. Diez, é voc. hispânico, pre-romano)
\section{Bacirrabo}
\begin{itemize}
\item {Grp. gram.:m.}
\end{itemize}
\begin{itemize}
\item {Utilização:Ant.}
\end{itemize}
\begin{itemize}
\item {Proveniência:(De \textunderscore baixo\textunderscore  + \textunderscore rabo\textunderscore )}
\end{itemize}
Caudatário; aquelle que levanta a fímbria das vestes pontificaes.
\section{Baço}
\begin{itemize}
\item {Grp. gram.:m.}
\end{itemize}
\begin{itemize}
\item {Utilização:Prov.}
\end{itemize}
\begin{itemize}
\item {Utilização:minh.}
\end{itemize}
\begin{itemize}
\item {Grp. gram.:Adj.}
\end{itemize}
Órgão glandular, situado no hypocôndrio esquerdo, por baixo das falsas costellas.
Mancha na cara.
Moreno; trigueiro-pállido.
Embaciado: \textunderscore olhos baços\textunderscore .
Descòrado.
\section{Baco-baco}
\begin{itemize}
\item {Grp. gram.:m.}
\end{itemize}
\begin{itemize}
\item {Utilização:Bras. do N}
\end{itemize}
\begin{itemize}
\item {Proveniência:(T. onom.)}
\end{itemize}
Tropel cadenciado de cavalgaduras em marcha.
\section{Bacoco}
\begin{itemize}
\item {fónica:cô}
\end{itemize}
\begin{itemize}
\item {Grp. gram.:m.  e  adj.}
\end{itemize}
\begin{itemize}
\item {Utilização:Pop.}
\end{itemize}
Indivíduo pouco atilado; ingênuo; pacóvio.
\section{Baconiano}
\begin{itemize}
\item {Grp. gram.:adj.}
\end{itemize}
Relativo ao philósopho inglês Bacon.
\section{Bácora}
\begin{itemize}
\item {Grp. gram.:f.}
\end{itemize}
\begin{itemize}
\item {Utilização:Prov.}
\end{itemize}
\begin{itemize}
\item {Utilização:alent.}
\end{itemize}
Fêmea do bácoro.
Mulher mexeriqueira, intriguista.
\section{Bacorada}
\begin{itemize}
\item {Grp. gram.:f.}
\end{itemize}
\begin{itemize}
\item {Utilização:Prov.}
\end{itemize}
\begin{itemize}
\item {Utilização:alent.}
\end{itemize}
Rebanho de bacoros.
Asneirola.
Linguagem licenciosa.
\section{Bacorejar}
\begin{itemize}
\item {Grp. gram.:v. i.}
\end{itemize}
\begin{itemize}
\item {Proveniência:(De \textunderscore bácoro\textunderscore ?)}
\end{itemize}
Parecer, palpitar; advir á ideia: \textunderscore bacoreja-me que não tens juízo\textunderscore .
\section{Bacorejar}
\begin{itemize}
\item {Grp. gram.:v. i.}
\end{itemize}
\begin{itemize}
\item {Utilização:Prov.}
\end{itemize}
\begin{itemize}
\item {Utilização:alent.}
\end{itemize}
\begin{itemize}
\item {Proveniência:(De \textunderscore bácora\textunderscore )}
\end{itemize}
Fazer mexericos, enredos.
\section{Bacorejo}
\begin{itemize}
\item {Grp. gram.:m.}
\end{itemize}
\begin{itemize}
\item {Proveniência:(De \textunderscore bacorejar\textunderscore ^1)}
\end{itemize}
Presentimento, palpite.
\section{Bacorinhar}
\begin{itemize}
\item {Grp. gram.:v. i.}
\end{itemize}
(V. \textunderscore bacorejar\textunderscore ^1)
\section{Bacorinho}
\begin{itemize}
\item {Grp. gram.:m.}
\end{itemize}
\begin{itemize}
\item {Utilização:Prov.}
\end{itemize}
Casta de figo pequeno e temporão.
\section{Bácoro}
\begin{itemize}
\item {Grp. gram.:m.}
\end{itemize}
Pequeno porco; leitão.
(B. lat. \textunderscore bacharus\textunderscore )
\section{Bacorote}
\begin{itemize}
\item {Grp. gram.:m.}
\end{itemize}
Bácoro crescido:«\textunderscore um bacorote orgulhoso\textunderscore ». Sá de Miranda.
\section{Bacro}
\begin{itemize}
\item {Grp. gram.:m.}
\end{itemize}
\begin{itemize}
\item {Utilização:Pop.}
\end{itemize}
O mesmo que \textunderscore bácoro\textunderscore .
\section{Bactéria}
\begin{itemize}
\item {Grp. gram.:f.}
\end{itemize}
\begin{itemize}
\item {Grp. gram.:F. pl.}
\end{itemize}
\begin{itemize}
\item {Proveniência:(Gr. \textunderscore bakteria\textunderscore )}
\end{itemize}
Insecto orthóptero das regiões intertropicaes.
Animálculos microscópicos, que produzem a decomposição de substâncias vegetaes e animaes.
Micróbios; bacillos; vibriões.
\section{Bacteriáceas}
\begin{itemize}
\item {Grp. gram.:f. pl.}
\end{itemize}
Algas microscópicas, que aparecem nos pulmões, no linho enriado, etc.
\section{Bactericida}
\begin{itemize}
\item {Grp. gram.:adj.}
\end{itemize}
\begin{itemize}
\item {Proveniência:(Do gr. \textunderscore bakteria\textunderscore  + lat. \textunderscore caedere\textunderscore )}
\end{itemize}
Que destrói as bactérias.
\section{Bactérico}
\begin{itemize}
\item {Grp. gram.:adj.}
\end{itemize}
Relativo a bactérias.
\section{Bacterídia}
\begin{itemize}
\item {Grp. gram.:f.}
\end{itemize}
\begin{itemize}
\item {Proveniência:(Do gr. \textunderscore bakteria\textunderscore  + \textunderscore eidos\textunderscore )}
\end{itemize}
Gênero de bactérias, de corpo filiforme, mais ou menos distinctamente articulado e sempre immóvel.
\section{Bacteriemia}
\begin{itemize}
\item {Grp. gram.:f.}
\end{itemize}
Existência de bactérias no sangue.
\section{Bactério}
\begin{itemize}
\item {Grp. gram.:m.}
\end{itemize}
(V.bactéria)
\section{Bacteriologia}
\begin{itemize}
\item {Grp. gram.:f.}
\end{itemize}
Sciência, que investiga e expõe a fórma, a natureza e os effeitos das bactérias.
(Cp. \textunderscore bacteriólogo\textunderscore )
\section{Bacteriologista}
\begin{itemize}
\item {Grp. gram.:m.}
\end{itemize}
Aquelle que se applica ao estudo da \textunderscore bacteriologia\textunderscore .
\section{Bacteriólogo}
\begin{itemize}
\item {Grp. gram.:m.}
\end{itemize}
\begin{itemize}
\item {Proveniência:(Do gr. \textunderscore bakteria\textunderscore  + \textunderscore logos\textunderscore )}
\end{itemize}
Aquelle que é versado em bacteriologia.
\section{Bacterioscopia}
\begin{itemize}
\item {Grp. gram.:f.}
\end{itemize}
\begin{itemize}
\item {Proveniência:(Do gr. \textunderscore bakteria\textunderscore  + \textunderscore skopein\textunderscore )}
\end{itemize}
Observação scientífica das bactérias.
\section{Bacterioscópico}
\begin{itemize}
\item {Grp. gram.:adj.}
\end{itemize}
Relativo á bacterioscopia.
\section{Bacteriuria}
\begin{itemize}
\item {Grp. gram.:f.}
\end{itemize}
Expulsão de bactérias pela urina, sem lesão manifesta das vias urinárias.
\section{Bactriano}
\begin{itemize}
\item {Grp. gram.:adj.}
\end{itemize}
\begin{itemize}
\item {Grp. gram.:M.}
\end{itemize}
Relativo á \textunderscore Bactriana\textunderscore .
Habitante da Bactriana.
\section{Báctrico}
\begin{itemize}
\item {Grp. gram.:m.  e  adj.}
\end{itemize}
O mesmo que \textunderscore bactriano\textunderscore .
\section{Bactrídeas}
\begin{itemize}
\item {Grp. gram.:f. pl.}
\end{itemize}
Tríbo de plantas, estabelecida por Broguiart.
\section{Báctrio}
\begin{itemize}
\item {Grp. gram.:m.}
\end{itemize}
\begin{itemize}
\item {Grp. gram.:M.  e  adj.}
\end{itemize}
O mesmo que \textunderscore zenda\textunderscore , língua.
O mesmo que \textunderscore bactriano\textunderscore .
\section{Bacu}
\begin{itemize}
\item {Grp. gram.:m.}
\end{itemize}
\begin{itemize}
\item {Utilização:Bras. do N}
\end{itemize}
\begin{itemize}
\item {Utilização:Fig.}
\end{itemize}
Nome de um peixe vulgar.
Pessôa baixa e gorda.
\section{Bacuara}
\begin{itemize}
\item {Grp. gram.:adj.}
\end{itemize}
\begin{itemize}
\item {Utilização:Bras}
\end{itemize}
Esperto, diligente.
\section{Bacuçu}
\begin{itemize}
\item {Grp. gram.:m.}
\end{itemize}
\begin{itemize}
\item {Utilização:Bras. da Baía}
\end{itemize}
Espécie de canôa.
\section{Baculífero}
\begin{itemize}
\item {Grp. gram.:adj.}
\end{itemize}
\begin{itemize}
\item {Proveniência:(Do lat. \textunderscore baculus\textunderscore  + \textunderscore ferre\textunderscore )}
\end{itemize}
Diz-se de uma planta, cuja haste póde servir de bastão ou bengala.
\section{Báculo}
\begin{itemize}
\item {Grp. gram.:m.}
\end{itemize}
\begin{itemize}
\item {Utilização:Ant.}
\end{itemize}
\begin{itemize}
\item {Proveniência:(Lat. \textunderscore baculum\textunderscore )}
\end{itemize}
Bordão alto, cajado.
Bastão episcopal, que tem curva a extremidade superior.
O mesmo que \textunderscore bacêlo\textunderscore .
\section{Bacumixá}
\begin{itemize}
\item {Grp. gram.:m.}
\end{itemize}
Árvore silvestre do Brasil.
\section{Bacurau}
\begin{itemize}
\item {Grp. gram.:m.}
\end{itemize}
\begin{itemize}
\item {Utilização:Bras}
\end{itemize}
\begin{itemize}
\item {Proveniência:(T. onom.)}
\end{itemize}
Ave nocturna.
\section{Bacuri}
\begin{itemize}
\item {Grp. gram.:m.}
\end{itemize}
\begin{itemize}
\item {Grp. gram.:Pl.}
\end{itemize}
Árvore guttífera do Brasil; fruto dessa árvore.
Selvagens do Brasil, entre as nascentes do rio Arinos.
\section{Bacuripari}
\begin{itemize}
\item {Grp. gram.:m.}
\end{itemize}
\begin{itemize}
\item {Utilização:Bras}
\end{itemize}
Árvore fructífera, da fam. das gutíferas.
\section{Bacurubu}
\begin{itemize}
\item {Grp. gram.:m.}
\end{itemize}
Árvore leguminosa do Brasil.
\section{Bacusso}
\begin{itemize}
\item {Grp. gram.:m.}
\end{itemize}
\begin{itemize}
\item {Utilização:Ant.}
\end{itemize}
Espécie de metal.
\section{Bada}
\begin{itemize}
\item {Grp. gram.:f.}
\end{itemize}
O mesmo que \textunderscore abada\textunderscore ^2. Cf. \textunderscore Ethióp. Or.\textunderscore , l. I, c. 1.
\section{Badal}
\begin{itemize}
\item {Grp. gram.:m.}
\end{itemize}
Antigo instrumento cirúrgico, que sustinha o queixo e fazia baixar a língua, para se observar a garganta do doente. Cf. Bluteau.
\section{Badal}
\begin{itemize}
\item {Grp. gram.:m.}
\end{itemize}
\begin{itemize}
\item {Utilização:Pop.}
\end{itemize}
\begin{itemize}
\item {Grp. gram.:M.  e  f.}
\end{itemize}
\begin{itemize}
\item {Utilização:Prov.}
\end{itemize}
\begin{itemize}
\item {Utilização:beir.}
\end{itemize}
O mesmo que \textunderscore badalo\textunderscore .
Pessôa leviana ou amiga da vadiagem.
\section{Badala}
\begin{itemize}
\item {Grp. gram.:f.}
\end{itemize}
\begin{itemize}
\item {Utilização:Prov.}
\end{itemize}
\begin{itemize}
\item {Utilização:alent.}
\end{itemize}
Mulher leviana, sem juízo.
(Cp. \textunderscore badalo\textunderscore )
\section{Badalada}
\begin{itemize}
\item {Grp. gram.:f.}
\end{itemize}
Pancada de badalo.
\section{Badalão}
\begin{itemize}
\item {Grp. gram.:m.}
\end{itemize}
Homem falador e desassisado. Cf. Cortesão, \textunderscore Subs.\textunderscore 
\section{Badalar}
\begin{itemize}
\item {Grp. gram.:v. t.}
\end{itemize}
\begin{itemize}
\item {Grp. gram.:V. i.}
\end{itemize}
\begin{itemize}
\item {Utilização:Fig.}
\end{itemize}
\begin{itemize}
\item {Utilização:Prov.}
\end{itemize}
\begin{itemize}
\item {Utilização:alent.}
\end{itemize}
\begin{itemize}
\item {Proveniência:(De \textunderscore badalo\textunderscore )}
\end{itemize}
Revelar indiscretamente.
Dar badaladas.
Falar muito.
Doidejar, andar á tôa, de um lado para outro.
\section{Badaleira}
\begin{itemize}
\item {Grp. gram.:f.}
\end{itemize}
\begin{itemize}
\item {Utilização:Pop.}
\end{itemize}
\begin{itemize}
\item {Proveniência:(De \textunderscore badalo\textunderscore  e \textunderscore badalar\textunderscore )}
\end{itemize}
Argola, que sustenta o badalo.
Mulher, que fala muito.
\section{Badaleiro}
\begin{itemize}
\item {Grp. gram.:m.}
\end{itemize}
\begin{itemize}
\item {Utilização:Pop.}
\end{itemize}
\begin{itemize}
\item {Proveniência:(De \textunderscore badalar\textunderscore )}
\end{itemize}
Homem, que fala muito, que é indiscreto.
\section{Badalejar}
\begin{itemize}
\item {Grp. gram.:v. i.}
\end{itemize}
\begin{itemize}
\item {Proveniência:(De \textunderscore badalo\textunderscore )}
\end{itemize}
Badalar.
Bater os dentes, com frio ou mêdo.
\section{Badalhó}
\begin{itemize}
\item {Grp. gram.:m.}
\end{itemize}
Casta de figueira.
\section{Badalhoca}
\begin{itemize}
\item {Grp. gram.:f.}
\end{itemize}
\begin{itemize}
\item {Utilização:Prov.}
\end{itemize}
\begin{itemize}
\item {Utilização:beir.}
\end{itemize}
\begin{itemize}
\item {Grp. gram.:Pl.}
\end{itemize}
\begin{itemize}
\item {Utilização:Prov.}
\end{itemize}
\begin{itemize}
\item {Utilização:trasm.}
\end{itemize}
\begin{itemize}
\item {Utilização:beir.}
\end{itemize}
\begin{itemize}
\item {Proveniência:(De \textunderscore badalo\textunderscore )}
\end{itemize}
Mulher suja e repugnante.
Bólas de excremento e terra, pendentes, como badalos, entre as pernas das ovelhas e carneiros.
\section{Badalim}
\begin{itemize}
\item {Grp. gram.:m.}
\end{itemize}
\begin{itemize}
\item {Utilização:Prov.}
\end{itemize}
\begin{itemize}
\item {Utilização:alent.}
\end{itemize}
\begin{itemize}
\item {Proveniência:(De \textunderscore badalo\textunderscore , por analogia com o pênis)}
\end{itemize}
Secreção sebácea, que se cria entre o prepúcio e a glande de alguns indivíduos.
\section{Badalo}
\begin{itemize}
\item {Grp. gram.:m.}
\end{itemize}
\begin{itemize}
\item {Utilização:Fam.}
\end{itemize}
\begin{itemize}
\item {Utilização:Prov.}
\end{itemize}
\begin{itemize}
\item {Utilização:alent.}
\end{itemize}
\begin{itemize}
\item {Proveniência:(Do lat. hyp. \textunderscore battalium\textunderscore )}
\end{itemize}
Peça de metal, que, suspensa por argola no interior de sino, sineta ou campaínha, os faz soar, agitando-se e batendo nelles.
A língua: \textunderscore dar ao badalo\textunderscore .
Homem leviano, sem juízo.
\section{Badame}
\begin{itemize}
\item {Grp. gram.:m.}
\end{itemize}
Instrumento de carpintaria, espécie de formão.
(Cp. cast. \textunderscore badano\textunderscore )
\section{Badameco}
\begin{itemize}
\item {Grp. gram.:m.}
\end{itemize}
\begin{itemize}
\item {Utilização:Ant.}
\end{itemize}
\begin{itemize}
\item {Grp. gram.:Pl.}
\end{itemize}
\begin{itemize}
\item {Utilização:Prov.}
\end{itemize}
\begin{itemize}
\item {Utilização:trasm.}
\end{itemize}
\begin{itemize}
\item {Proveniência:(Do lat. \textunderscore vade\textunderscore  + \textunderscore mecum\textunderscore )}
\end{itemize}
Pasta com papéis ou livros, que os estudantes levam para a escola.
Rapazola.
Homem sem importância.
Testículos.
\section{Badamo}
\begin{itemize}
\item {Grp. gram.:m.}
\end{itemize}
\begin{itemize}
\item {Utilização:Prov.}
\end{itemize}
\begin{itemize}
\item {Utilização:alent.}
\end{itemize}
O mesmo que \textunderscore badame\textunderscore .
\section{Badana}
\begin{itemize}
\item {Grp. gram.:f.}
\end{itemize}
\begin{itemize}
\item {Utilização:Prov.}
\end{itemize}
\begin{itemize}
\item {Utilização:trasm.}
\end{itemize}
\begin{itemize}
\item {Utilização:Bras. do S}
\end{itemize}
\begin{itemize}
\item {Grp. gram.:Pl.}
\end{itemize}
\begin{itemize}
\item {Utilização:Ant.}
\end{itemize}
\begin{itemize}
\item {Proveniência:(Do ár. \textunderscore bitana\textunderscore )}
\end{itemize}
Ovelha velha e magra.
Carne de ovelha velha.
A pelle que pende, em gume, do pescoço do boi, ou qualquer coisa pendente e semelhante a essa pelle.
Pelle macia, lavrada, que se põe em cima do coxonilho.
Annexos de vestuário, estreitos e pendentes.
\section{Badana}
\begin{itemize}
\item {Grp. gram.:m.}
\end{itemize}
\begin{itemize}
\item {Utilização:Fam.}
\end{itemize}
Pacóvio.
Homem insignificante, homúnculo.
\section{Badanal}
\begin{itemize}
\item {Grp. gram.:m.}
\end{itemize}
\begin{itemize}
\item {Utilização:Pop.}
\end{itemize}
Confusão; balbúrdia.
Azáfama; lufa-lufa.
(Talvez corr. de \textunderscore badalar\textunderscore )
\section{Badanar}
\begin{itemize}
\item {Grp. gram.:v. i.}
\end{itemize}
Mover-se froixamente, como pelhancas pendentes:«\textunderscore os beiços badanando ondeiam\textunderscore ». Macedo, \textunderscore Burros\textunderscore , 29.
\section{Badanau}
\begin{itemize}
\item {Grp. gram.:m.}
\end{itemize}
O mesmo que \textunderscore badanal\textunderscore .
\section{Badano}
\begin{itemize}
\item {Grp. gram.:m.}
\end{itemize}
O mesmo que \textunderscore badana\textunderscore ^2.
\section{Badante}
\begin{itemize}
\item {Grp. gram.:adj.}
\end{itemize}
\begin{itemize}
\item {Utilização:Prov.}
\end{itemize}
\begin{itemize}
\item {Utilização:beir.}
\end{itemize}
Inclinado para dentro, (falando-se de obras de pedreiro)
\section{Badarrinha}
\begin{itemize}
\item {Grp. gram.:f.}
\end{itemize}
\begin{itemize}
\item {Utilização:Ant.}
\end{itemize}
(?)«\textunderscore Onças de raiva mortal nas badarrinhas\textunderscore ». G. Vicente, I, 258.
\section{Badega}
\begin{itemize}
\item {Grp. gram.:adj.}
\end{itemize}
\begin{itemize}
\item {Utilização:Bras}
\end{itemize}
\begin{itemize}
\item {Utilização:chul.}
\end{itemize}
Muito grande: \textunderscore laranjas badegas\textunderscore ; \textunderscore pratos badegas\textunderscore .
\section{Badejete}
\begin{itemize}
\item {Grp. gram.:m.}
\end{itemize}
\begin{itemize}
\item {Utilização:Bras}
\end{itemize}
Peixe, pequeno badejo de água salgada.
\section{Badejo}
\begin{itemize}
\item {Grp. gram.:m.}
\end{itemize}
Peixe, semelhante ao bacalhau, mas mais pequeno, que cái nos apparelhos da pesca da pescada.
(Por \textunderscore abbadejo\textunderscore , de \textunderscore abbade\textunderscore )
\section{Badiana}
\begin{itemize}
\item {Grp. gram.:f.}
\end{itemize}
Árvore magnoliácea e fructifera da Ásia.
\section{Badigó}
\begin{itemize}
\item {Grp. gram.:m.}
\end{itemize}
\begin{itemize}
\item {Utilização:Prov.}
\end{itemize}
\begin{itemize}
\item {Utilização:trasm.}
\end{itemize}
Sujeito gordo ou pançudo.
\section{Badil}
\begin{itemize}
\item {Grp. gram.:m.}
\end{itemize}
\begin{itemize}
\item {Utilização:Prov.}
\end{itemize}
\begin{itemize}
\item {Utilização:trasm.}
\end{itemize}
Pá, com que na cozinha se remove o lume ou a cinza.
\section{Badoém}
\begin{itemize}
\item {Grp. gram.:m.}
\end{itemize}
\begin{itemize}
\item {Utilização:Prov.}
\end{itemize}
\begin{itemize}
\item {Utilização:alent.}
\end{itemize}
O mesmo que \textunderscore badana\textunderscore ^2.
\section{Badona}
\begin{itemize}
\item {Grp. gram.:f.}
\end{itemize}
\begin{itemize}
\item {Utilização:Gír.}
\end{itemize}
Cavallo.
\section{Badulaque}
\begin{itemize}
\item {Grp. gram.:m.}
\end{itemize}
(V.bazulaque)
\section{Bádur}
\begin{itemize}
\item {Grp. gram.:m.}
\end{itemize}
Chefe indígena de algum distrito, dependente do Estado da Índia Portuguesa.
\section{Bafagem}
\begin{itemize}
\item {Grp. gram.:f.}
\end{itemize}
\begin{itemize}
\item {Proveniência:(De \textunderscore bafo\textunderscore )}
\end{itemize}
Aragem; brisa.
Acção de \textunderscore bafejar\textunderscore .
\section{Bafareira}
\begin{itemize}
\item {Grp. gram.:f.}
\end{itemize}
Parte de certos alambiques:«\textunderscore enche-se a caldeira externa de água, que fique sempre acima do tecto da bafareira.\textunderscore »F. Lapa, \textunderscore Techn. Rur.\textunderscore , I, 494.
(Por \textunderscore abafareira\textunderscore , de \textunderscore abafar\textunderscore ?)
\section{Bafari}
\begin{itemize}
\item {Grp. gram.:m.}
\end{itemize}
\begin{itemize}
\item {Proveniência:(Do ár. \textunderscore bahi\textunderscore )}
\end{itemize}
Espécie de falcão.
\section{Bafejador}
\begin{itemize}
\item {Grp. gram.:m.  e  adj.}
\end{itemize}
O que bafeja.
\section{Bafejar}
\begin{itemize}
\item {Grp. gram.:v. t.}
\end{itemize}
\begin{itemize}
\item {Grp. gram.:V. i.}
\end{itemize}
Soprar brandamente.
Favorecer: \textunderscore bafejou-o a sorte\textunderscore .
Inspirar.
Exhalar bafo.
\section{Bafejo}
\begin{itemize}
\item {Grp. gram.:m.}
\end{itemize}
O mesmo que \textunderscore bafagem\textunderscore .
Acção de \textunderscore bafejar\textunderscore .
\section{Bafio}
\begin{itemize}
\item {Grp. gram.:m.}
\end{itemize}
\begin{itemize}
\item {Proveniência:(De \textunderscore bafo\textunderscore )}
\end{itemize}
Cheiro peculiar aos objectos húmidos ou privados da renovação do ar.
Môfo.
Exhalação mephítica dêsses objectos.
\section{Baforar}
\begin{itemize}
\item {Grp. gram.:v. i.}
\end{itemize}
Expellir o bafo.
Arrotar.
Vangloriar-se. Cf. Filinto, XIX, 277.
\section{Baderna}
\begin{itemize}
\item {Grp. gram.:f.}
\end{itemize}
\begin{itemize}
\item {Utilização:Bras}
\end{itemize}
Súcia; matulagem.
Pândega.
(Talvez de \textunderscore Baderna\textunderscore , n. p. de uma dançarina)
\section{Baderna}
\begin{itemize}
\item {Grp. gram.:f.}
\end{itemize}
\begin{itemize}
\item {Utilização:Náut.}
\end{itemize}
\begin{itemize}
\item {Proveniência:(Fr. \textunderscore baderne\textunderscore )}
\end{itemize}
Arrebém, para fixar os colhedores, quando se apertam as enxárcias.
\section{Badernar}
\begin{itemize}
\item {Grp. gram.:v. i.}
\end{itemize}
\begin{itemize}
\item {Utilização:Bras}
\end{itemize}
\begin{itemize}
\item {Proveniência:(De \textunderscore baderna\textunderscore ^1)}
\end{itemize}
Pandegar.
Madraçar.
\section{Badola}
\begin{itemize}
\item {Grp. gram.:m.  e  f.}
\end{itemize}
\begin{itemize}
\item {Utilização:T. de Lanhoso}
\end{itemize}
Pessôa simplória e bondosa.
\section{Badoxo}
\begin{itemize}
\item {fónica:dô}
\end{itemize}
\begin{itemize}
\item {Grp. gram.:m.}
\end{itemize}
\begin{itemize}
\item {Utilização:Prov.}
\end{itemize}
\begin{itemize}
\item {Utilização:dur.}
\end{itemize}
Núcleo de novelo.
(Cp. \textunderscore bagoxo\textunderscore )
\section{Baé}
\begin{itemize}
\item {Grp. gram.:f.}
\end{itemize}
\begin{itemize}
\item {Grp. gram.:Adj.}
\end{itemize}
\begin{itemize}
\item {Utilização:Bras. do N}
\end{itemize}
Mulher christan de canarim.
Diz-se de pessôa baixa e grossa.
\section{Baéco}
\begin{itemize}
\item {Grp. gram.:adj.}
\end{itemize}
\begin{itemize}
\item {Utilização:Bras}
\end{itemize}
O mesmo que \textunderscore baé\textunderscore .
\section{Baêta}
\begin{itemize}
\item {Grp. gram.:f.}
\end{itemize}
\begin{itemize}
\item {Grp. gram.:M.}
\end{itemize}
\begin{itemize}
\item {Utilização:Bras}
\end{itemize}
\begin{itemize}
\item {Proveniência:(Do lat. \textunderscore Baetica\textunderscore , n. p.)}
\end{itemize}
Pano felpudo de lan.
Habitante de Minas-Geraes, onde os camponeses usam geralmente jaqueta de baêta.
\section{Baetal}
\begin{itemize}
\item {fónica:ba-e}
\end{itemize}
\begin{itemize}
\item {Grp. gram.:adj.}
\end{itemize}
Próprio de \textunderscore baêta\textunderscore .
\section{Baetão}
\begin{itemize}
\item {fónica:ba-e}
\end{itemize}
\begin{itemize}
\item {Grp. gram.:m.}
\end{itemize}
\begin{itemize}
\item {Utilização:Bras}
\end{itemize}
Baêta grossa.
Cobertor de lan.
\section{Baetilha}
\begin{itemize}
\item {fónica:ba-e}
\end{itemize}
\begin{itemize}
\item {Grp. gram.:f.}
\end{itemize}
Baêta fina.
Tecido de algodão felpudo.
\section{Bafo}
\begin{itemize}
\item {Grp. gram.:m.}
\end{itemize}
\begin{itemize}
\item {Proveniência:(T. onom.)}
\end{itemize}
Hálito.
Ar, exhalado dos pulmões.
Bafagem.
Sopro brando e morno.
Favor.
Abrigo.
Inspiração.
Banho de estufa, entre os pretos de Lourenço-Marques.
\section{Baforada}
\begin{itemize}
\item {Grp. gram.:f.}
\end{itemize}
\begin{itemize}
\item {Proveniência:(De \textunderscore vapor\textunderscore , sob infl. de \textunderscore bafo\textunderscore )}
\end{itemize}
Hálito desagradável.
Bafo forte: \textunderscore baforadas de fumo\textunderscore .
Espalhafato.
\section{Baciano}
\begin{itemize}
\item {Grp. gram.:adj.}
\end{itemize}
\begin{itemize}
\item {Proveniência:(Do lat. \textunderscore bacca\textunderscore )}
\end{itemize}
Semelhante á baga.
\section{Bacífero}
\begin{itemize}
\item {Grp. gram.:adj.}
\end{itemize}
\begin{itemize}
\item {Proveniência:(Lat. \textunderscore baccifer\textunderscore )}
\end{itemize}
Que tem ou produz baga.
\section{Baciforme}
\begin{itemize}
\item {Grp. gram.:adj.}
\end{itemize}
\begin{itemize}
\item {Proveniência:(Do lat. \textunderscore bacca\textunderscore  + \textunderscore forma\textunderscore )}
\end{itemize}
Que tem fórma de baga.
\section{Bacívoro}
\begin{itemize}
\item {Grp. gram.:adj.}
\end{itemize}
\begin{itemize}
\item {Proveniência:(Do lat. \textunderscore bacca\textunderscore  + \textunderscore vorare\textunderscore )}
\end{itemize}
Que se alimenta de baga.
\section{Baé}
\begin{itemize}
\item {Grp. gram.:m.}
\end{itemize}
\begin{itemize}
\item {Utilização:Bras}
\end{itemize}
Fazenda de algodão, fabricada na Inglaterra, e que se reexporta para a costa da África.
\section{Baforda}
\begin{itemize}
\item {Grp. gram.:f.}
\end{itemize}
\begin{itemize}
\item {Utilização:Ant.}
\end{itemize}
Espécie de lança.
Injúria.
(Cp. \textunderscore bafordar\textunderscore )
\section{Bafordar}
\begin{itemize}
\item {Grp. gram.:v. i.}
\end{itemize}
\begin{itemize}
\item {Utilização:Ant.}
\end{itemize}
Jogar a baforda.
Brincar com armas, fingindo combate.
(Cp. b. lat. \textunderscore bagordare\textunderscore )
\section{Bafordo}
\begin{itemize}
\item {Grp. gram.:m.}
\end{itemize}
\begin{itemize}
\item {Utilização:Prov.}
\end{itemize}
\begin{itemize}
\item {Utilização:beir.}
\end{itemize}
\begin{itemize}
\item {Utilização:T. da Bairrada}
\end{itemize}
Azeitona atrophiada, que attinge apenas o tamanho de um confeito.
Bagos miúdos de uva.
(Cast. ant. \textunderscore bofordo\textunderscore )
\section{Baforeira}
\begin{itemize}
\item {Grp. gram.:f.}
\end{itemize}
\begin{itemize}
\item {Proveniência:(Lat. hyp. \textunderscore biferaria\textunderscore .)}
\end{itemize}
Espécie de figueira, abebereira.
Nome vulgar da planta euphorbiácea e lactagoga \textunderscore ricinus communis\textunderscore .
\section{Baforeira}
\begin{itemize}
\item {Grp. gram.:f.}
\end{itemize}
\begin{itemize}
\item {Utilização:Prov.}
\end{itemize}
\begin{itemize}
\item {Utilização:minh.}
\end{itemize}
Jactância, bazófia.
(Cp. \textunderscore baforar\textunderscore )
\section{Bafugem}
\begin{itemize}
\item {Grp. gram.:f.}
\end{itemize}
\begin{itemize}
\item {Utilização:Des.}
\end{itemize}
O mesmo que \textunderscore bafagem\textunderscore . Cf. \textunderscore Peregrinação\textunderscore , LIII.
\section{Bafúrdio}
\begin{itemize}
\item {Grp. gram.:m.}
\end{itemize}
\begin{itemize}
\item {Utilização:Ant.}
\end{itemize}
Cavalhadas, justas, com pouco apparato.
\section{Baga}
\begin{itemize}
\item {Grp. gram.:f.}
\end{itemize}
\begin{itemize}
\item {Utilização:Bras}
\end{itemize}
\begin{itemize}
\item {Proveniência:(Lat. \textunderscore baca\textunderscore )}
\end{itemize}
Pequeno fruto redondo e carnudo.
Gota.
Camarinha (de suor).
Casta de uva preta da Bairrada.
Semente do mamoeiro.
\section{Bagabaga}
\begin{itemize}
\item {Grp. gram.:m.}
\end{itemize}
Nome, que na Guiné se dá ao salalé.
\section{Bagaça}
\begin{itemize}
\item {Grp. gram.:f.}
\end{itemize}
\begin{itemize}
\item {Utilização:Prov.}
\end{itemize}
\begin{itemize}
\item {Utilização:alent.}
\end{itemize}
Mulher de má nota.
\section{Bagaceira}
\begin{itemize}
\item {Grp. gram.:f.}
\end{itemize}
\begin{itemize}
\item {Utilização:Gír.}
\end{itemize}
\begin{itemize}
\item {Utilização:Bras}
\end{itemize}
\begin{itemize}
\item {Utilização:Bras}
\end{itemize}
\begin{itemize}
\item {Utilização:Bras}
\end{itemize}
Lugar, onde se junta o bagaço.
Aguardente.
Conjunto de coisas reles ou inúteis, coisas sem valor.
Monte de lenha, em pilha, dísposta de maneira que a miúda se não confunda com a graúda.
Palavreado oco, chocho, como bagaço.
\section{Bagaceiro}
\begin{itemize}
\item {Grp. gram.:adj.}
\end{itemize}
\begin{itemize}
\item {Grp. gram.:M.}
\end{itemize}
\begin{itemize}
\item {Utilização:Bras}
\end{itemize}
\begin{itemize}
\item {Utilização:T. de Lanhoso}
\end{itemize}
Que come bem o bagaço: \textunderscore o meu porco é bagaceiro\textunderscore .
Aquelle que remove o bagaço do lugar onde se juntou.
Lugar, onde se junta o bagaço do açúcar.
Indivíduo indolente, mandrião.
\section{Bagacina}
\begin{itemize}
\item {Grp. gram.:f.}
\end{itemize}
\begin{itemize}
\item {Utilização:Açor}
\end{itemize}
O mesmo que \textunderscore pedra-pomes\textunderscore .
\section{Bagaço}
\begin{itemize}
\item {Grp. gram.:m.}
\end{itemize}
\begin{itemize}
\item {Utilização:Pop.}
\end{itemize}
\begin{itemize}
\item {Proveniência:(De \textunderscore baga\textunderscore )}
\end{itemize}
Resíduo dos frutos, que foram espremidos para se lhes extrahir o suco.
Abundancia, riqueza: \textunderscore aquelle tem bagaço\textunderscore .
\section{Bagada}
\begin{itemize}
\item {Grp. gram.:f.}
\end{itemize}
\begin{itemize}
\item {Proveniência:(De \textunderscore baga\textunderscore )}
\end{itemize}
Lágrima grossa.
Grande quantidade de bagas.
\section{Bagageira}
\begin{itemize}
\item {Grp. gram.:f.}
\end{itemize}
Subsídio, que o Govêrno dá aos militares e engenheiros de graduação superior, empregados em commissão, a título da despesa para o transporte das bagagens.
Designação, proposta para exprimir o que, em caminhos de ferro, se diz \textunderscore furgão\textunderscore , ou, á francesa, \textunderscore fourgon\textunderscore .
\section{Bagageiro}
\begin{itemize}
\item {Grp. gram.:m.}
\end{itemize}
\begin{itemize}
\item {Utilização:Prov.}
\end{itemize}
\begin{itemize}
\item {Utilização:alent.}
\end{itemize}
\begin{itemize}
\item {Grp. gram.:Adj.}
\end{itemize}
\begin{itemize}
\item {Utilização:Bras}
\end{itemize}
Conductor de bagagens.
Aquelle que leva comida aos trabalhadores.
Diz-se do cavallo, que é o último a chegar á meta, nas corridas.
O mesmo que \textunderscore punga\textunderscore .
\section{Bagagem}
\begin{itemize}
\item {Grp. gram.:f.}
\end{itemize}
\begin{itemize}
\item {Proveniência:(Fr. \textunderscore bagage\textunderscore )}
\end{itemize}
Objectos, que os viajantes levam para seu uso, em pacotes, malas, caixas ou bahus.
Fardeta.
Equipagem.
Conjunto de obras de um escritor: \textunderscore escritor de grande bagagem\textunderscore .
Empecilho.
\section{Bagalhão}
\begin{itemize}
\item {Grp. gram.:m.}
\end{itemize}
Grande bago.
O mesmo que \textunderscore chamadeira\textunderscore .
\section{Bagalho}
\begin{itemize}
\item {Grp. gram.:m.}
\end{itemize}
\begin{itemize}
\item {Utilização:Prov.}
\end{itemize}
\begin{itemize}
\item {Utilização:trasm.}
\end{itemize}
Os bagos da roman.
\section{Bagalhoça}
\begin{itemize}
\item {Grp. gram.:f.}
\end{itemize}
\begin{itemize}
\item {Utilização:Pop.}
\end{itemize}
\begin{itemize}
\item {Proveniência:(De \textunderscore bago\textunderscore )}
\end{itemize}
Dinheiro.
\section{Bagançal}
\begin{itemize}
\item {Grp. gram.:m.}
\end{itemize}
\begin{itemize}
\item {Utilização:T. ind.}
\end{itemize}
Armazém de fazendas.
\section{Bagançarins}
\begin{itemize}
\item {Grp. gram.:m. pl.}
\end{itemize}
Casta de gente de Guzarate, a qual se distinguia dos Baneanes e outras castas, por comer carne e peixe. Cf. Barros, \textunderscore Déc.\textunderscore  IV, l. V, c. 1.
\section{Baganda}
\begin{itemize}
\item {Grp. gram.:m.  e  f.}
\end{itemize}
Pessôa ordinária e sediciosa ou desordeira. Cf. Herculano, \textunderscore Quest. Públ.\textunderscore , II, 132.
\section{Baganha}
\begin{itemize}
\item {Grp. gram.:f.}
\end{itemize}
\begin{itemize}
\item {Utilização:T. da Bairrada}
\end{itemize}
\begin{itemize}
\item {Proveniência:(De \textunderscore bago\textunderscore )}
\end{itemize}
Casulo, que envolve a semente do linho.
Pellícula, que contém qualquer semente.
O mesmo que \textunderscore brulho\textunderscore .
\section{Baganhão}
\begin{itemize}
\item {Grp. gram.:m.}
\end{itemize}
O mesmo que \textunderscore bagalhão\textunderscore . Cf. \textunderscore Bibl. da G. do Campo\textunderscore , 299.
\section{Baganho}
\begin{itemize}
\item {Grp. gram.:m.}
\end{itemize}
\begin{itemize}
\item {Utilização:Prov.}
\end{itemize}
\begin{itemize}
\item {Utilização:beir.}
\end{itemize}
Bagaço de azeitona.
\section{Bagarote}
\begin{itemize}
\item {Grp. gram.:m.}
\end{itemize}
\begin{itemize}
\item {Utilização:Bras. do N}
\end{itemize}
O mesmo que \textunderscore dinheiro\textunderscore .--É mais us. no pl.
(Cp. \textunderscore bagalhoça\textunderscore )
\section{Bagata}
\begin{itemize}
\item {Grp. gram.:f.}
\end{itemize}
\begin{itemize}
\item {Utilização:Gír.}
\end{itemize}
Bruxaria.
\section{Bagatela}
\begin{itemize}
\item {Grp. gram.:f.}
\end{itemize}
Coisa sem valor, inútil.
Frivolidade; ninharia.
(Cast. \textunderscore bagatela\textunderscore )
\section{Bagateleiro}
\begin{itemize}
\item {Grp. gram.:m.}
\end{itemize}
Aquelle que se occupa com bagatelas.
\section{Bagaxa}
\begin{itemize}
\item {Grp. gram.:f.}
\end{itemize}
\begin{itemize}
\item {Utilização:Ant.}
\end{itemize}
Prostituta.
\section{Bagem}
\begin{itemize}
\item {Grp. gram.:f.}
\end{itemize}
(V.vagem)
\section{Bageri}
\begin{itemize}
\item {Grp. gram.:m.}
\end{itemize}
Cereal de Dio, (\textunderscore peniciallia spicata\textunderscore ).
\section{Bago}
\begin{itemize}
\item {Grp. gram.:m.}
\end{itemize}
\begin{itemize}
\item {Utilização:Pop.}
\end{itemize}
\begin{itemize}
\item {Utilização:Ant.}
\end{itemize}
\begin{itemize}
\item {Proveniência:(Lat. \textunderscore baculum\textunderscore )}
\end{itemize}
Qualquer pequeno fruto redondo e carnudo, semelhante ao da uva.
Grão de qualquer coisa, parecido àquelle fruto.
Dinheiro.
O mesmo que \textunderscore báculo\textunderscore :«\textunderscore o abbade, que o bago regedor meteu em meio da contenda\textunderscore ». Garrett, \textunderscore D. Branca\textunderscore , 23.
\section{Bagoado}
\begin{itemize}
\item {Grp. gram.:adj.}
\end{itemize}
Que tem fórma de bago.
\section{Bagoeira}
\begin{itemize}
\item {Grp. gram.:f.}
\end{itemize}
\begin{itemize}
\item {Utilização:Prov.}
\end{itemize}
\begin{itemize}
\item {Utilização:dur.}
\end{itemize}
\begin{itemize}
\item {Proveniência:(De \textunderscore bago\textunderscore )}
\end{itemize}
Lugar, onde há espalhados muitos bagos de uva, por terem caído de parreira ou por outro motivo.
\section{Bagoeirada}
\begin{itemize}
\item {Grp. gram.:f.}
\end{itemize}
\begin{itemize}
\item {Utilização:T. da Bairrada}
\end{itemize}
Grande porção de bagos.
\section{Bago-grosso}
\begin{itemize}
\item {Grp. gram.:m.}
\end{itemize}
Casta de uva alentejana.
\section{Bagoxo}
\begin{itemize}
\item {fónica:gô}
\end{itemize}
\begin{itemize}
\item {Grp. gram.:m.}
\end{itemize}
\begin{itemize}
\item {Utilização:Prov.}
\end{itemize}
Trapo ou papel, que fórma o centro do novelo.
\section{Bagre}
\begin{itemize}
\item {Grp. gram.:m.}
\end{itemize}
Designação de dois gêneros de peixes.
Planta gomosa do Brasil.
\section{Bágua}
\begin{itemize}
\item {Grp. gram.:f.}
\end{itemize}
\begin{itemize}
\item {Utilização:Prov.}
\end{itemize}
\begin{itemize}
\item {Proveniência:(Do lat. \textunderscore bacula\textunderscore )}
\end{itemize}
Lágrima.
\section{Baguá}
\begin{itemize}
\item {Grp. gram.:m.}
\end{itemize}
\begin{itemize}
\item {Utilização:Bras}
\end{itemize}
O mesmo que \textunderscore bagual\textunderscore .
\section{Baguaçá}
\begin{itemize}
\item {Grp. gram.:m.}
\end{itemize}
Árvore sertaneja do Brasil.
\section{Bagual}
\begin{itemize}
\item {Grp. gram.:m.}
\end{itemize}
\begin{itemize}
\item {Utilização:Bras. do S}
\end{itemize}
\begin{itemize}
\item {Proveniência:(T. das Antilhas)}
\end{itemize}
Cavallo branco.
\section{Bagualada}
\begin{itemize}
\item {Grp. gram.:f.}
\end{itemize}
Manada de baguaes.
\section{Baguari}
\begin{itemize}
\item {Grp. gram.:m.}
\end{itemize}
\begin{itemize}
\item {Grp. gram.:Adj.}
\end{itemize}
\begin{itemize}
\item {Utilização:Bras. do S}
\end{itemize}
Espécie de cegonha.
Corpulento.
Vagaroso.
\section{Baé}
\begin{itemize}
\item {Grp. gram.:m.}
\end{itemize}
\begin{itemize}
\item {Utilização:Bras}
\end{itemize}
Fazenda de algodão, fabricada na Inglaterra, e que se reexporta para a costa da África.
\section{Baguear}
\begin{itemize}
\item {Grp. gram.:v. t.}
\end{itemize}
\begin{itemize}
\item {Utilização:Prov.}
\end{itemize}
\begin{itemize}
\item {Utilização:dur.}
\end{itemize}
\begin{itemize}
\item {Proveniência:(De \textunderscore baga\textunderscore )}
\end{itemize}
Tornar mais escuro (o vinho), por meio de líquido que se extrai da baga do sabugueiro.
\section{Baguim}
\begin{itemize}
\item {Grp. gram.:m.}
\end{itemize}
Nome de duas variedades de pêra.
\section{Baguines}
\begin{itemize}
\item {Grp. gram.:m.}
\end{itemize}
\begin{itemize}
\item {Utilização:Gír.}
\end{itemize}
Dinheiro.
(Cp. \textunderscore bago\textunderscore )
\section{Bagulhado}
\begin{itemize}
\item {Grp. gram.:adj.}
\end{itemize}
(V.bagulhoso)
\section{Bagulhento}
\begin{itemize}
\item {Grp. gram.:adj.}
\end{itemize}
O mesmo que \textunderscore bagulhoso\textunderscore .
\section{Bagulho}
\begin{itemize}
\item {Grp. gram.:m.}
\end{itemize}
\begin{itemize}
\item {Utilização:Prov.}
\end{itemize}
\begin{itemize}
\item {Utilização:dur.}
\end{itemize}
\begin{itemize}
\item {Utilização:Gír.}
\end{itemize}
\begin{itemize}
\item {Utilização:Prov.}
\end{itemize}
\begin{itemize}
\item {Utilização:trasm.}
\end{itemize}
\begin{itemize}
\item {Proveniência:(De \textunderscore bago\textunderscore )}
\end{itemize}
Graínha, granita; semente da uva.
Acervo dos bagos pisados.
Dinheiro.
Os bagos da roman.
\section{Bagulhoso}
\begin{itemize}
\item {Grp. gram.:adj.}
\end{itemize}
Que tem bagulho.
\section{Bahé}
\begin{itemize}
\item {Grp. gram.:m.}
\end{itemize}
\begin{itemize}
\item {Utilização:Bras}
\end{itemize}
Fazenda de algodão, fabricada na Inglaterra, e que se reexporta para a costa da África.
\section{Bahia}
\begin{itemize}
\item {Grp. gram.:f.}
\end{itemize}
\begin{itemize}
\item {Utilização:Bras}
\end{itemize}
Pequeno golfo, de bôca estreita.
Lagôa, com communicação para um rio.
Canal, para escoamento de pântanos.
(B. lat. \textunderscore baia\textunderscore )
\section{Bahianada}
\begin{itemize}
\item {Grp. gram.:f.}
\end{itemize}
\begin{itemize}
\item {Utilização:Bras}
\end{itemize}
Fanfarronada de bahiano.
Impostura.
\section{Bahiano}
\begin{itemize}
\item {Grp. gram.:m.}
\end{itemize}
\begin{itemize}
\item {Utilização:Bras. do S}
\end{itemize}
\begin{itemize}
\item {Grp. gram.:Adj.}
\end{itemize}
\begin{itemize}
\item {Utilização:Bras. do S}
\end{itemize}
\begin{itemize}
\item {Utilização:Bras. do N}
\end{itemize}
Aquelle que é natural da Bahia.
O mesmo que \textunderscore baião\textunderscore .
Mau cavalleiro.
Relativo á Bahia.
O mesmo que \textunderscore nortista\textunderscore .
Habitante do campo ou das roças.
\section{Bahu}
\begin{itemize}
\item {Grp. gram.:m.}
\end{itemize}
Caixa de madeira, ordinariamente revestida de coiro cru, e com tampa convexa.
(Alto al. médio \textunderscore behut\textunderscore )
\section{Bahul}
\begin{itemize}
\item {Grp. gram.:m.}
\end{itemize}
Fórma antiga de \textunderscore bahu\textunderscore .
(B. lat. \textunderscore bahulum\textunderscore )
\section{Bahuleiro}
\begin{itemize}
\item {Grp. gram.:m.}
\end{itemize}
\begin{itemize}
\item {Proveniência:(De \textunderscore bahul\textunderscore )}
\end{itemize}
Aquelle que fabríca ou vende bahus.
\section{Baia}
\begin{itemize}
\item {Grp. gram.:f.}
\end{itemize}
Trave ou tábua, que separa as cavalgaduras nas cavallariças.
\section{Baía}
\begin{itemize}
\item {Grp. gram.:f.}
\end{itemize}
\begin{itemize}
\item {Utilização:Bras}
\end{itemize}
Pequeno golfo, de bôca estreita.
Lagôa, com communicação para um rio.
Canal, para escoamento de pântanos.
(B. lat. \textunderscore baia\textunderscore )
\section{Baiacu}
\begin{itemize}
\item {Grp. gram.:m.}
\end{itemize}
\begin{itemize}
\item {Proveniência:(T. tupi)}
\end{itemize}
Homem gordo e baixo.
Nome de um peixe do Brasil.
\section{Baiana}
\begin{itemize}
\item {fónica:ba-i}
\end{itemize}
\begin{itemize}
\item {Grp. gram.:f.}
\end{itemize}
\begin{itemize}
\item {Utilização:Bras. do N}
\end{itemize}
Capa de coiro, que se usa sôbre a sella e em que se conduz roupa e outros objectos.
\section{Baianada}
\begin{itemize}
\item {fónica:ba-i}
\end{itemize}
\begin{itemize}
\item {Grp. gram.:f.}
\end{itemize}
\begin{itemize}
\item {Utilização:Bras}
\end{itemize}
Fanfarronada de baiano.
Impostura.
\section{Baianca}
\begin{itemize}
\item {Grp. gram.:f.}
\end{itemize}
\begin{itemize}
\item {Utilização:Ant.}
\end{itemize}
Barranco.
Quebrada entre vallados.
Caminho estreito entre o baluarte e o fôsso, nas antigas fortalezas.
\section{Baianço}
\begin{itemize}
\item {fónica:ba-i}
\end{itemize}
\begin{itemize}
\item {Grp. gram.:m.}
\end{itemize}
\begin{itemize}
\item {Utilização:Bras}
\end{itemize}
\begin{itemize}
\item {Proveniência:(De \textunderscore Bahia\textunderscore , n. p.)}
\end{itemize}
O mesmo que \textunderscore bahiano\textunderscore .
\section{Baiano}
\begin{itemize}
\item {fónica:ba-i}
\end{itemize}
\begin{itemize}
\item {Grp. gram.:m.}
\end{itemize}
\begin{itemize}
\item {Utilização:Bras. do S}
\end{itemize}
\begin{itemize}
\item {Grp. gram.:Adj.}
\end{itemize}
\begin{itemize}
\item {Utilização:Bras. do S}
\end{itemize}
\begin{itemize}
\item {Utilização:Bras. do N}
\end{itemize}
Aquelle que é natural da Bahia.
O mesmo que \textunderscore baião\textunderscore .
Mau cavalleiro.
Relativo á Bahia.
O mesmo que \textunderscore nortista\textunderscore .
Habitante do campo ou das roças.
\section{Baião}
\begin{itemize}
\item {Grp. gram.:m.}
\end{itemize}
Dança e canto popular, ao som de instrumentos.
(Cp. \textunderscore baiar\textunderscore )
\section{Baiar}
\begin{itemize}
\item {Grp. gram.:v. i.}
\end{itemize}
\begin{itemize}
\item {Utilização:Bras}
\end{itemize}
Dançar.
(Contr. de \textunderscore bailar\textunderscore ?)
\section{Baiardos}
\begin{itemize}
\item {Grp. gram.:m. pl.}
\end{itemize}
\begin{itemize}
\item {Utilização:Náut.}
\end{itemize}
Pedaços de antennas, com que se defendem as mesas da enxárcia ou do traquete, quando se concertam as embarcações.
\section{Baiás}
\begin{itemize}
\item {Grp. gram.:m. pl.}
\end{itemize}
\begin{itemize}
\item {Utilização:Bras}
\end{itemize}
Nação de aborígenes, que viveram em Mato-Grosso.
\section{Baiburuás}
\begin{itemize}
\item {Grp. gram.:m. pl.}
\end{itemize}
Indigenas brasileiros das margens do Juruá.
\section{Baila}
\begin{itemize}
\item {Grp. gram.:f.}
\end{itemize}
(V. \textunderscore balha\textunderscore ^1)
\section{Baila}
\begin{itemize}
\item {Grp. gram.:f.}
\end{itemize}
\begin{itemize}
\item {Proveniência:(De \textunderscore bailar\textunderscore )}
\end{itemize}
O mesmo que \textunderscore bailado\textunderscore .
Espécie de peixe, também conhecido por \textunderscore bailadeira\textunderscore .
\section{Bailada}
\begin{itemize}
\item {Grp. gram.:f.}
\end{itemize}
\begin{itemize}
\item {Utilização:Des.}
\end{itemize}
Acto de \textunderscore bailar\textunderscore .
Bailarico.
\section{Bailadeira}
\begin{itemize}
\item {Grp. gram.:f.}
\end{itemize}
\begin{itemize}
\item {Proveniência:(De \textunderscore bailar\textunderscore )}
\end{itemize}
Mulher, que baila.
Designação de um peixe, (\textunderscore labrax punctatus\textunderscore , Cuv.).
\section{Bailado}
\begin{itemize}
\item {Grp. gram.:m.}
\end{itemize}
\begin{itemize}
\item {Proveniência:(De \textunderscore bailar\textunderscore )}
\end{itemize}
Dança, que se executa no fim ou no intervallo das óperas.
Qualquer dança; baile.
\section{Bailador}
\begin{itemize}
\item {Grp. gram.:m.}
\end{itemize}
Aquelle que baila.
\section{Bailão}
\begin{itemize}
\item {Grp. gram.:m.}
\end{itemize}
Aquelle que baila muito.
Fadista.
\section{Bailar}
\begin{itemize}
\item {Grp. gram.:v. t.  e  i.}
\end{itemize}
Executar, dançando.
Dançar.
Saltar.
Oscillar; tremer.
Afadigar-se.
(B. lat. \textunderscore ballare\textunderscore )
\section{Bailareta}
\begin{itemize}
\item {fónica:larê}
\end{itemize}
\begin{itemize}
\item {Grp. gram.:f.}
\end{itemize}
\begin{itemize}
\item {Utilização:Prov.}
\end{itemize}
\begin{itemize}
\item {Utilização:trasm.}
\end{itemize}
O fruto da esteva, (por allusão a que os rapazes a fazem \textunderscore bailar\textunderscore , a modo de piasca).
\section{Bailarico}
\begin{itemize}
\item {Grp. gram.:m.}
\end{itemize}
Baile popular, ao som de música.
\section{Bailarim}
\begin{itemize}
\item {Grp. gram.:m.}
\end{itemize}
\begin{itemize}
\item {Utilização:Des.}
\end{itemize}
O mesmo que \textunderscore bailarino\textunderscore .
\section{Bailarina}
\begin{itemize}
\item {Grp. gram.:f.}
\end{itemize}
\begin{itemize}
\item {Proveniência:(De \textunderscore bailar\textunderscore )}
\end{itemize}
Mulher, que dança por profissão.
\section{Bailarino}
\begin{itemize}
\item {Grp. gram.:m.}
\end{itemize}
\begin{itemize}
\item {Utilização:Fam.}
\end{itemize}
\begin{itemize}
\item {Proveniência:(De \textunderscore bailar\textunderscore )}
\end{itemize}
Aquelle que dança por profissão.
Aquelle que anda em bicos de pés, meneando muito o corpo.
\section{Bailariqueiro}
\begin{itemize}
\item {Grp. gram.:m.}
\end{itemize}
Frequentador de bailaricos.
\section{Baile}
\begin{itemize}
\item {Grp. gram.:m.}
\end{itemize}
\begin{itemize}
\item {Proveniência:(De \textunderscore bailar\textunderscore )}
\end{itemize}
Reunião de pessôas que dançam.
Dança festiva.
\section{Bailéo}
\begin{itemize}
\item {Grp. gram.:m.}
\end{itemize}
\begin{itemize}
\item {Utilização:Náut.}
\end{itemize}
\begin{itemize}
\item {Proveniência:(De \textunderscore bailar\textunderscore ?)}
\end{itemize}
Andaime, estrado, em que trabalham os constructores, reparadores ou caiadores de edifícios altos.
Palanque.
Estrado alto, donde se combatia nos navios antigos.
Prateleira nas casernas.
Espaço do navio, entre a coberta e o porão.
\section{Bailete}
\begin{itemize}
\item {fónica:lê}
\end{itemize}
\begin{itemize}
\item {Grp. gram.:m.}
\end{itemize}
\begin{itemize}
\item {Proveniência:(Fr. \textunderscore ballet\textunderscore )}
\end{itemize}
Dança mímica; pantomima.
\section{Bailéu}
\begin{itemize}
\item {Grp. gram.:m.}
\end{itemize}
\begin{itemize}
\item {Utilização:Náut.}
\end{itemize}
\begin{itemize}
\item {Proveniência:(De \textunderscore bailar\textunderscore ?)}
\end{itemize}
Andaime, estrado, em que trabalham os constructores, reparadores ou caiadores de edifícios altos.
Palanque.
Estrado alto, donde se combatia nos navios antigos.
Prateleira nas casernas.
Espaço do navio, entre a coberta e o porão.
\section{Bailha}
\begin{itemize}
\item {Grp. gram.:f.}
\end{itemize}
O mesmo que \textunderscore baila\textunderscore ^1. Cf. Filinto, XIX, 233.
\section{Bailhão}
\begin{itemize}
\item {Grp. gram.:m.}
\end{itemize}
O mesmo que \textunderscore bailão\textunderscore .
\section{Bailharote}
\begin{itemize}
\item {Grp. gram.:m.}
\end{itemize}
\begin{itemize}
\item {Utilização:Gír.}
\end{itemize}
Feijão.
(Por \textunderscore bailarote\textunderscore , de \textunderscore bailar\textunderscore )
\section{Bailheiro}
\begin{itemize}
\item {Grp. gram.:adj.}
\end{itemize}
\begin{itemize}
\item {Utilização:Ant.}
\end{itemize}
\begin{itemize}
\item {Proveniência:(De \textunderscore bailha\textunderscore )}
\end{itemize}
Ligeiro. Cf. Fern. Lopes, \textunderscore Chrón. de D. João I\textunderscore , p. II, c. 135.
\section{Bailia}
\begin{itemize}
\item {Grp. gram.:f.}
\end{itemize}
Commenda de bailio.
\section{Bailiado}
\begin{itemize}
\item {Grp. gram.:m.}
\end{itemize}
Dignidade, território, jurisdição, de bailio.
\section{Bailio}
\begin{itemize}
\item {Grp. gram.:m.}
\end{itemize}
\begin{itemize}
\item {Proveniência:(Fr. \textunderscore bailli\textunderscore )}
\end{itemize}
Commendador, nas antigas Ordens militares.
Antigo magistrado provincial.
\section{Bailique}
\begin{itemize}
\item {Grp. gram.:m.}
\end{itemize}
\begin{itemize}
\item {Utilização:Gír.}
\end{itemize}
Quarto na prisão.
Tarimba.
(Relaciona-se com \textunderscore bailéu\textunderscore ?)
\section{Bailo}
\begin{itemize}
\item {Grp. gram.:m.}
\end{itemize}
\begin{itemize}
\item {Utilização:Ant.}
\end{itemize}
O mesmo que \textunderscore baile\textunderscore .
\section{Bailomania}
\begin{itemize}
\item {Grp. gram.:f.}
\end{itemize}
\begin{itemize}
\item {Proveniência:(De \textunderscore baile\textunderscore  + \textunderscore mania\textunderscore )}
\end{itemize}
Paixão por bailes.
\section{Bailomaníaco}
\begin{itemize}
\item {Grp. gram.:adj.}
\end{itemize}
Que tem \textunderscore bailomania\textunderscore .
\section{Bailum}
\begin{itemize}
\item {Grp. gram.:m.}
\end{itemize}
\begin{itemize}
\item {Utilização:Ant.}
\end{itemize}
Primeiro recinto em que se entrava, passada a porta, nas praças fortes da Idade-Média.
\section{Bailundos}
\begin{itemize}
\item {Grp. gram.:m. pl.}
\end{itemize}
Povos de raça cafreal, em Angola.
\section{Baínha}
\begin{itemize}
\item {Grp. gram.:f.}
\end{itemize}
\begin{itemize}
\item {Grp. gram.:Pl.}
\end{itemize}
\begin{itemize}
\item {Utilização:Prov.}
\end{itemize}
\begin{itemize}
\item {Utilização:alent.}
\end{itemize}
\begin{itemize}
\item {Proveniência:(Do lat. \textunderscore vagína\textunderscore )}
\end{itemize}
Estojo, em que se mete a fôlha de uma arma branca ou objecto semelhante.
Dobra, que se cose na extremidade de um pano, para que êste se não desfie.
Doce de grão de bico, envolvido numa capa de massa, em fórma de vagem.
\section{Baínha-de-espada}
\begin{itemize}
\item {Grp. gram.:f.}
\end{itemize}
\begin{itemize}
\item {Utilização:Bras}
\end{itemize}
Árvore silvestre, da fam. das artocárpeas.
\section{Bainhar}
\begin{itemize}
\item {fónica:ba-i}
\end{itemize}
\textunderscore v. t.\textunderscore  (e der.)
O mesmo que \textunderscore embainhar\textunderscore , etc.
\section{Bainheira}
\begin{itemize}
\item {fónica:ba-i}
\end{itemize}
\begin{itemize}
\item {Grp. gram.:f.}
\end{itemize}
\begin{itemize}
\item {Utilização:Ant.}
\end{itemize}
Talvez costureira, que se occupava especialmente em bainhas de vestuário. Cf. Soropita, \textunderscore Prosas\textunderscore , 69.
\section{Bainheiro}
\begin{itemize}
\item {fónica:ba-i}
\end{itemize}
\begin{itemize}
\item {Grp. gram.:m.}
\end{itemize}
Aquelle que faz baínhas de espadas.
Árvore anacardeácea da Índia, (\textunderscore odina Wodier\textunderscore , Roxb.).
\section{Baio}
\begin{itemize}
\item {Grp. gram.:adj.}
\end{itemize}
\begin{itemize}
\item {Grp. gram.:M.}
\end{itemize}
\begin{itemize}
\item {Proveniência:(Lat. \textunderscore badius\textunderscore )}
\end{itemize}
Que tem côr de oiro desmaiado.
Amarelo torrado.
Amulatado.
Cavallo baio. Cf. \textunderscore Viriato Trág.\textunderscore , XI, 69.
\section{Baio}
\begin{itemize}
\item {Grp. gram.:m.}
\end{itemize}
\begin{itemize}
\item {Utilização:Prov.}
\end{itemize}
\begin{itemize}
\item {Utilização:trasm.}
\end{itemize}
Bucho dos animaes; pança.
\section{Baioco}
\begin{itemize}
\item {fónica:ô}
\end{itemize}
\begin{itemize}
\item {Grp. gram.:m.}
\end{itemize}
\begin{itemize}
\item {Utilização:Chul.}
\end{itemize}
Moéda de dez reis. Cf. Camillo, \textunderscore Caveira\textunderscore .
\section{Baiona}
\begin{itemize}
\item {Grp. gram.:f.}
\end{itemize}
\begin{itemize}
\item {Utilização:Prov.}
\end{itemize}
\begin{itemize}
\item {Utilização:trasm.}
\end{itemize}
Urtiga brava.
\section{Baionês}
\begin{itemize}
\item {Grp. gram.:adj.}
\end{itemize}
\begin{itemize}
\item {Utilização:Ant.}
\end{itemize}
Trigueiro:«\textunderscore da côr da maçan chamada baionesa\textunderscore ». Prestes, \textunderscore Autos\textunderscore , 105.
Cf. \textunderscore Ethiópia Or.\textunderscore , c. 32, e Castanheda, \textunderscore Descobr. da Índia\textunderscore , liv. VI.
\section{Baionesa}
\begin{itemize}
\item {Grp. gram.:f.  e  adj.}
\end{itemize}
\begin{itemize}
\item {Proveniência:(De \textunderscore Baiona\textunderscore , n. p.)}
\end{itemize}
Diz-se de uma espécie de maçan grande, doce, e parda junto do pé.--É conhecida especialmente em Trás-os-Montes.
\section{Baioneta}
\begin{itemize}
\item {fónica:nê}
\end{itemize}
\begin{itemize}
\item {Grp. gram.:f.}
\end{itemize}
\begin{itemize}
\item {Proveniência:(Fr. \textunderscore baïonnette\textunderscore )}
\end{itemize}
Arma ponteaguda, que se adapta á extremidade do cano da espingarda.
\section{Baionetada}
\begin{itemize}
\item {Grp. gram.:f.}
\end{itemize}
Golpe de baioneta.
\section{Baioninho}
\begin{itemize}
\item {Grp. gram.:m.}
\end{itemize}
\begin{itemize}
\item {Utilização:Ant.}
\end{itemize}
Cavallinho baio.
\section{Bairrada}
\begin{itemize}
\item {Grp. gram.:m.}
\end{itemize}
Vinho fabricado na Bairrada.
\section{Bairrismo}
\begin{itemize}
\item {Grp. gram.:m.}
\end{itemize}
Qualidade de bairrista.
\section{Bairrista}
\begin{itemize}
\item {Grp. gram.:m. ,  f.  e  adj.}
\end{itemize}
Pessôa, que habita ou frequenta um bairro.
Defensor dos interesses do seu bairro ou da sua terra.
\section{Bairro}
\begin{itemize}
\item {Grp. gram.:m.}
\end{itemize}
Cada uma das partes principaes de uma cidade.
Cada uma das áreas administrativas, em que se divide Lisbôa e Pôrto.
Parte de uma povoação.
(B. lat. \textunderscore barrium\textunderscore )
\section{Baita}
\begin{itemize}
\item {Grp. gram.:adj.}
\end{itemize}
\begin{itemize}
\item {Utilização:Bras. do N}
\end{itemize}
Muito grande.
Famoso.
Destemido.
\section{Baitaca}
\begin{itemize}
\item {Grp. gram.:m.}
\end{itemize}
\begin{itemize}
\item {Utilização:Bras}
\end{itemize}
Espécie de papagaio.
\section{Baiúca}
\begin{itemize}
\item {Grp. gram.:f.}
\end{itemize}
Pequena taberna.
Pequena casa.
Bodega.
\section{Baiuqueiro}
\begin{itemize}
\item {Grp. gram.:adj.}
\end{itemize}
\begin{itemize}
\item {Grp. gram.:M.}
\end{itemize}
Relativo a baiúca.
Frequentador de baiúcas.
Taberneiro.
\section{Baixa}
\begin{itemize}
\item {Grp. gram.:f.}
\end{itemize}
\begin{itemize}
\item {Utilização:Ant.}
\end{itemize}
\begin{itemize}
\item {Proveniência:(De \textunderscore baixar\textunderscore )}
\end{itemize}
Abaixamento.
Depressão de terreno.
Parte, pouco funda, de mar ou rio.
Lugar baixo.
Deminuição de valor: \textunderscore os fundos turcos soffreram grande baixa\textunderscore .
Decadência.
Abatimento.
Despedida de serviço: \textunderscore o soldado teve baixa\textunderscore .
Eliminação judicial da nota de culpa.
Espécie de dança ou antes maneira de dançar, erguendo um pouco os pés.
\section{Baixada}
\begin{itemize}
\item {Grp. gram.:f.}
\end{itemize}
\begin{itemize}
\item {Utilização:Bras}
\end{itemize}
\begin{itemize}
\item {Proveniência:(De \textunderscore baixar\textunderscore )}
\end{itemize}
Depressão de terreno, junto de uma lomba.
Planície entre montanhas.
\section{Baixamar}
\begin{itemize}
\item {Grp. gram.:f.}
\end{itemize}
\begin{itemize}
\item {Proveniência:(De \textunderscore baixo\textunderscore  + \textunderscore mar\textunderscore )}
\end{itemize}
Maré baixa.
\section{Baixamente}
\begin{itemize}
\item {Grp. gram.:adv.}
\end{itemize}
De modo baixo, vil, rasteiro, humilde.
\section{Baixão}
\begin{itemize}
\item {Grp. gram.:m.}
\end{itemize}
\begin{itemize}
\item {Utilização:Ant.}
\end{itemize}
\begin{itemize}
\item {Proveniência:(De \textunderscore baixo\textunderscore )}
\end{itemize}
Instrumento musical, de som baixo.--Êste instrumento podia sêr de sopro ou de cordas, mas dava-se especialmente aquelle nome a um grande instrumento, do gênero das charamelas, e com palheta dupla.
Registo nos órgãos antigos, de tubos com bocal de prata.
Aquelle que toca ou canta em tom baixo, numa orchestra ou concêrto. Cf. \textunderscore Anat. Joc.\textunderscore , I, 290.
\section{Baixãozinho}
\begin{itemize}
\item {Grp. gram.:m.}
\end{itemize}
O mesmo que \textunderscore baixete\textunderscore , instrumento.
\section{Baixar}
\begin{itemize}
\item {Grp. gram.:v. t.}
\end{itemize}
\begin{itemize}
\item {Grp. gram.:V. i.}
\end{itemize}
\begin{itemize}
\item {Proveniência:(De \textunderscore baixo\textunderscore )}
\end{itemize}
Pôr em baixo.
Fazer descer.
Apear.
Arrear.
Inclinar para baixo: \textunderscore baixar a cabeça\textunderscore .
Dar tom mais baixo a.
Abater.
Deminuir de altura.
Descer.
Desacreditar-se; depreciar-se; perder prestígio.
Sêr expedido, (falando-se das ordens emanadas do Govêrno ou autoridades superiores).
\section{Baixeiro}
\begin{itemize}
\item {Grp. gram.:adj.}
\end{itemize}
\begin{itemize}
\item {Utilização:Bras}
\end{itemize}
\begin{itemize}
\item {Grp. gram.:M.}
\end{itemize}
\begin{itemize}
\item {Proveniência:(De \textunderscore baixo\textunderscore )}
\end{itemize}
Que se põe por baixo, (falando-se da peça que fica por baixo do sellim ou dos arreios).
Manta, que se põe por baixo da sella.
\section{Baixel}
\begin{itemize}
\item {Grp. gram.:m.}
\end{itemize}
\begin{itemize}
\item {Proveniência:(Do lat. \textunderscore vascellum\textunderscore )}
\end{itemize}
Embarcação.
Pequeno navio.
\section{Baixel}
\begin{itemize}
\item {Grp. gram.:adj.}
\end{itemize}
\begin{itemize}
\item {Proveniência:(De \textunderscore baixo\textunderscore )}
\end{itemize}
O mesmo que \textunderscore bisco\textunderscore .
\section{Baixela}
\begin{itemize}
\item {Grp. gram.:f.}
\end{itemize}
\begin{itemize}
\item {Proveniência:(Do lat. \textunderscore vascella\textunderscore )}
\end{itemize}
Conjunto dos utensilios para serviço de mesa.
\section{Baixella}
\begin{itemize}
\item {Grp. gram.:f.}
\end{itemize}
\begin{itemize}
\item {Proveniência:(Do lat. \textunderscore vascella\textunderscore )}
\end{itemize}
Conjunto dos utensilios para serviço de mesa.
\section{Baixete}
\begin{itemize}
\item {fónica:xê}
\end{itemize}
\begin{itemize}
\item {Grp. gram.:m.}
\end{itemize}
\begin{itemize}
\item {Proveniência:(De \textunderscore baixo\textunderscore )}
\end{itemize}
Pequeno banco, chanfrado, em que assentam as pipas; malhal.
Antigo instrumento, que era um baixão pequeno e constituía o tenor dos instrumentos de sopro.
\section{Baixeza}
\begin{itemize}
\item {Grp. gram.:f.}
\end{itemize}
Qualidade do que é baixo ou do que está em baixo.
Abatimento.
Inferioridade.
Vileza; indignidade: \textunderscore praticar baixezas\textunderscore .
\section{Baixia}
\begin{itemize}
\item {Grp. gram.:f.}
\end{itemize}
(V.baixio)
\section{Baixinho}
\begin{itemize}
\item {Grp. gram.:adv.}
\end{itemize}
\begin{itemize}
\item {Proveniência:(De \textunderscore baixo\textunderscore )}
\end{itemize}
Em voz muito baixa; em segredo.
\section{Baixio}
\begin{itemize}
\item {Grp. gram.:m.}
\end{itemize}
\begin{itemize}
\item {Proveniência:(De \textunderscore baixo\textunderscore )}
\end{itemize}
Banco de areia, sôbre que tem pouca altura a água do mar.
\section{Baixista}
\begin{itemize}
\item {Grp. gram.:m.  e  adj.}
\end{itemize}
\begin{itemize}
\item {Utilização:Bras}
\end{itemize}
Bolsista, que joga na baixa do câmbio.
\section{Baixo}
\begin{itemize}
\item {Grp. gram.:adj.}
\end{itemize}
\begin{itemize}
\item {Grp. gram.:M.}
\end{itemize}
\begin{itemize}
\item {Grp. gram.:Adv.}
\end{itemize}
\begin{itemize}
\item {Proveniência:(Lat. hyp. \textunderscore bassius\textunderscore , do lat. \textunderscore Bassus\textunderscore , n. p.)}
\end{itemize}
Que tem pouca altura: \textunderscore homem baixo\textunderscore .
Pouco fundo: \textunderscore rio baixo\textunderscore .
Que anda pouco acima de nós: \textunderscore nuvens baixas\textunderscore .
Inferior.
Que desceu da sua elevação normal: \textunderscore preço baixo\textunderscore .
Inclinado.
Decadente.
Desprezível; reles: \textunderscore acção baixa\textunderscore .
Pequeno.
Ordinário.
Barato: \textunderscore o vinho está baixo\textunderscore .
Que mal se ouve: \textunderscore em voz baixa\textunderscore .
Parte inferior.
Baixio.
Homem, que tem voz própria para os sons graves.
O som grave, na música.
Corda grossa de alguns instrumentos.
Em voz baixa: \textunderscore falavam baixo\textunderscore .
Em lugar baixo.
\section{Baixo-bretão}
\begin{itemize}
\item {Grp. gram.:m.}
\end{itemize}
\begin{itemize}
\item {Grp. gram.:Adj.}
\end{itemize}
Habitante da Baixa-Bretanha.
Língua desta região.
Relativo á Baixa-Bretanha.
\section{Baixo-império}
\begin{itemize}
\item {Grp. gram.:m.}
\end{itemize}
\begin{itemize}
\item {Utilização:Ext.}
\end{itemize}
Império grego de Constantinopla.
Sociedade corrompida.
Desmoralização.
\section{Baixo-navarro}
\begin{itemize}
\item {Grp. gram.:m.}
\end{itemize}
Um dos dialectos do vasconço, em França.
\section{Baixo-relêvo}
\begin{itemize}
\item {Grp. gram.:m.}
\end{itemize}
Obra de esculptura, sôbre um fundo a que ficam adherentes as figuras.
\section{Baixote}
\begin{itemize}
\item {Grp. gram.:adj.}
\end{itemize}
Um tanto baixo.
\section{Baixura}
\begin{itemize}
\item {Grp. gram.:f.}
\end{itemize}
\begin{itemize}
\item {Utilização:Prov.}
\end{itemize}
\begin{itemize}
\item {Utilização:alg.}
\end{itemize}
(V.baixeza)
Depressão de terreno.
\section{Bajanco}
\begin{itemize}
\item {Grp. gram.:m.}
\end{itemize}
\begin{itemize}
\item {Utilização:Ant.}
\end{itemize}
Charlatão, que cura com ervas.
(Cp. \textunderscore bajar\textunderscore )
\section{Bajar}
\begin{itemize}
\item {Grp. gram.:v. i.}
\end{itemize}
\begin{itemize}
\item {Utilização:Pop.}
\end{itemize}
\begin{itemize}
\item {Proveniência:(De \textunderscore bagem\textunderscore )}
\end{itemize}
Lançar vagens.
\section{Bajoceta}
\begin{itemize}
\item {fónica:cê}
\end{itemize}
\begin{itemize}
\item {Grp. gram.:f.}
\end{itemize}
\begin{itemize}
\item {Utilização:Ant.}
\end{itemize}
Alforge? fardel? Cf. Sousa, \textunderscore Ann. de D. João III\textunderscore .
\section{Bajogar}
\begin{itemize}
\item {Grp. gram.:v. t.}
\end{itemize}
\begin{itemize}
\item {Utilização:Bras. do Maranhão}
\end{itemize}
Deitar fóra.
Arremessar.
\section{Bajoujar}
\begin{itemize}
\item {Grp. gram.:v. t.}
\end{itemize}
\begin{itemize}
\item {Utilização:Fam.}
\end{itemize}
Acariciar.
Lisonjear, adular.
Amimar.
(Relaciona-se com \textunderscore bajular\textunderscore . Cf. Viana, \textunderscore Apostilas\textunderscore )
\section{Bajoujice}
\begin{itemize}
\item {Grp. gram.:f.}
\end{itemize}
Qualidade de quem é bajoujo.
Acção de \textunderscore bajoujar\textunderscore .
\section{Bajoujo}
\begin{itemize}
\item {Grp. gram.:m.  e  adj.}
\end{itemize}
\begin{itemize}
\item {Proveniência:(De \textunderscore bajoujar\textunderscore )}
\end{itemize}
O que lisonjeia ridiculamente.
Baboso.
Parvo.
\section{Baju}
\begin{itemize}
\item {Grp. gram.:m.}
\end{itemize}
\begin{itemize}
\item {Utilização:T. de Miranda}
\end{itemize}
\begin{itemize}
\item {Proveniência:(T. mal.)}
\end{itemize}
Nome de certo vestuário curto, usado pelas mulheres dos chefes indígenas, em Timor e na Índia.
Casaquinho curto de mulher, semelhante á \textunderscore roupinha\textunderscore  da Beira-Alta.
\section{Bajude}
\begin{itemize}
\item {Grp. gram.:f.}
\end{itemize}
\begin{itemize}
\item {Utilização:T. da Guiné port}
\end{itemize}
Virgem, donzella.
\section{Bajulação}
\begin{itemize}
\item {Grp. gram.:f.}
\end{itemize}
Acção de \textunderscore bajular\textunderscore .
\section{Bajulador}
\begin{itemize}
\item {Grp. gram.:m.  e  adj.}
\end{itemize}
O que bajula.
\section{Bajular}
\begin{itemize}
\item {Grp. gram.:v. t.}
\end{itemize}
\begin{itemize}
\item {Proveniência:(Lat. \textunderscore bajulare\textunderscore )}
\end{itemize}
Lisonjear servilmente; adular.
\section{Bajulia}
\begin{itemize}
\item {Grp. gram.:f.}
\end{itemize}
\begin{itemize}
\item {Utilização:Ant.}
\end{itemize}
O mesmo que \textunderscore bailia\textunderscore .
\section{Bajulice}
\begin{itemize}
\item {Grp. gram.:f.}
\end{itemize}
(V.bajulação)
\section{Bájulo}
\begin{itemize}
\item {Grp. gram.:m.}
\end{itemize}
\begin{itemize}
\item {Utilização:Ant.}
\end{itemize}
\begin{itemize}
\item {Proveniência:(Lat. \textunderscore bajulus\textunderscore )}
\end{itemize}
Carregador, carrejão:«\textunderscore bájulos se chamam aquelles homens que levam aos ombros gravíssimos pesos\textunderscore ». Vieira, VI, 55.
\section{Bajunça}
\begin{itemize}
\item {Grp. gram.:f.}
\end{itemize}
Planta aquática, (\textunderscore carus acutiformis\textunderscore ).
\section{Bala}
\begin{itemize}
\item {Grp. gram.:f.}
\end{itemize}
\begin{itemize}
\item {Utilização:Bras. do S}
\end{itemize}
\begin{itemize}
\item {Utilização:Prov.}
\end{itemize}
\begin{itemize}
\item {Grp. gram.:Pl.}
\end{itemize}
\begin{itemize}
\item {Proveniência:(Do gr. \textunderscore ballein\textunderscore )}
\end{itemize}
Esphera de metal ou de pedra, destinada a ser projéctil de armas de fogo.
Pacote.
Bola.
Rebuçado.
Dinheiro.
Moéda de oiro.
Almofadas, com que antigamente se dava tinta nas fôrmas typográphicas.
\section{Bala}
\begin{itemize}
\item {Grp. gram.:f.}
\end{itemize}
\begin{itemize}
\item {Utilização:T. de Angola}
\end{itemize}
Tubérculo sêco de mandioca, para se fazer o infúndi.
\section{Balache}
\begin{itemize}
\item {Grp. gram.:m.}
\end{itemize}
\begin{itemize}
\item {Grp. gram.:Adj.}
\end{itemize}
\begin{itemize}
\item {Proveniência:(Do ár. \textunderscore balakhsh\textunderscore )}
\end{itemize}
Espécie de espinela vermelha ou alaranjada.
Diz-se da pedra preciosa, conhecida hoje por \textunderscore espinela\textunderscore .
\section{Balaço}
\begin{itemize}
\item {Grp. gram.:m.}
\end{itemize}
Bala grande.
Tiro de bala.
\section{Balado}
\begin{itemize}
\item {Grp. gram.:m.}
\end{itemize}
(V.balido)
\section{Balador}
\begin{itemize}
\item {Grp. gram.:m.}
\end{itemize}
Anacardo.
Fruto desta planta.
\section{Balador}
\begin{itemize}
\item {Grp. gram.:adj.}
\end{itemize}
Que bala: \textunderscore rebanho balador\textunderscore .
\section{Balafo}
\begin{itemize}
\item {Grp. gram.:m.}
\end{itemize}
Instrumento musical, entre os indígenas da Guiné.
\section{Balagate}
\begin{itemize}
\item {Grp. gram.:m.}
\end{itemize}
\begin{itemize}
\item {Proveniência:(De \textunderscore Balaghat\textunderscore , n. p.)}
\end{itemize}
Espécie de pano grosso da Índia.
\section{Balaia}
\begin{itemize}
\item {Grp. gram.:f.}
\end{itemize}
\begin{itemize}
\item {Utilização:T. do Fundão}
\end{itemize}
Cêsto baixo, com tampa e sem asa.
(Cp. \textunderscore balaio\textunderscore )
\section{Balaiada}
\begin{itemize}
\item {Grp. gram.:f.}
\end{itemize}
\begin{itemize}
\item {Utilização:Bras}
\end{itemize}
Revolta dos Balaios no Maranhão, em 1839 a 1840.
\section{Balaieiro}
\begin{itemize}
\item {Grp. gram.:m.}
\end{itemize}
\begin{itemize}
\item {Utilização:Bras}
\end{itemize}
\begin{itemize}
\item {Proveniência:(De \textunderscore balaio\textunderscore )}
\end{itemize}
Um dos dez indivíduos, que habitualmente tripulam a baleeira.
\section{Balaio}
\begin{itemize}
\item {Grp. gram.:m.}
\end{itemize}
\begin{itemize}
\item {Utilização:Bras}
\end{itemize}
\begin{itemize}
\item {Utilização:Bras. do S}
\end{itemize}
Cêsto de palha, em fórma de alguidar.
Farnel.
Dança de pretos, espécie de fandango.
Partidário de Ferreira Balaio, chefe da balaiada.
\section{Balalaica}
\begin{itemize}
\item {Grp. gram.:f.}
\end{itemize}
Espécie de mandolim, de três cordas e de fórma triangular, us. entre os camponeses da Rússia.
\section{Balalina}
\begin{itemize}
\item {Grp. gram.:f.}
\end{itemize}
Insecto coleóptero, cuja fêmea fura a casca, ainda tenra, das avellans, e ali deposita o ovo, donde sái a larva.
\section{Balamalete}
\begin{itemize}
\item {fónica:lê}
\end{itemize}
\begin{itemize}
\item {Grp. gram.:m.}
\end{itemize}
Ave da África occidental.
\section{Balambamba}
\begin{itemize}
\item {Grp. gram.:f.}
\end{itemize}
Ave da África occidental.
\section{Balame}
\begin{itemize}
\item {Grp. gram.:m.}
\end{itemize}
\begin{itemize}
\item {Proveniência:(De \textunderscore bala\textunderscore )}
\end{itemize}
Grande porção de balas.
\section{Balança}
\begin{itemize}
\item {Grp. gram.:f.}
\end{itemize}
\begin{itemize}
\item {Proveniência:(Do lat. hyp. \textunderscore balancia\textunderscore )}
\end{itemize}
Instrumento, que determina o pêso dos corpos, em relação a certa unidade, e cuja parte essencial é uma alavanca, travessão ou cutello.
Equilibrio:«\textunderscore Portugal na balança da Europa\textunderscore ». Garrett.
Relação; confronto: \textunderscore a balança do commércio\textunderscore .
Constellação do Zodíaco.
Sýmbolo (da justiça)
\section{Balançar}
\begin{itemize}
\item {Grp. gram.:v. t.}
\end{itemize}
\begin{itemize}
\item {Grp. gram.:V. i.}
\end{itemize}
\begin{itemize}
\item {Utilização:Fig.}
\end{itemize}
\begin{itemize}
\item {Proveniência:(De \textunderscore balança\textunderscore )}
\end{itemize}
Fazer oscillar.
Equilibrar.
Comparar.
Compensar.
Dar balanço a.
Pesar.
Oscillar.
Hesitar: \textunderscore o meu coração balança na escolha\textunderscore .
\section{Balancé}
\begin{itemize}
\item {Grp. gram.:m.}
\end{itemize}
\begin{itemize}
\item {Proveniência:(Fr. \textunderscore balancé\textunderscore )}
\end{itemize}
Passo de dança, em que o corpo balança de um pé para outro, em tempos iguaes.
Baloiço.
Bailarico.
Apparelho, para cunhagem de moéda.
Máquina, para reproduzir documentos em livros chamados copiadores, para imprimir bilhetes de visita, etc.
\section{Balanceador}
\begin{itemize}
\item {Grp. gram.:m.}
\end{itemize}
O mesmo que \textunderscore balancista\textunderscore .
\section{Balanceadura}
\begin{itemize}
\item {Grp. gram.:f.}
\end{itemize}
O mesmo que \textunderscore balanceamento\textunderscore .
\section{Balanceamento}
\begin{itemize}
\item {Grp. gram.:m.}
\end{itemize}
Acção de \textunderscore balancear\textunderscore .
\section{Balancear}
\begin{itemize}
\item {Grp. gram.:v. t.  e  i.}
\end{itemize}
O mesmo que \textunderscore balançar\textunderscore .
\section{Balanceiro}
\begin{itemize}
\item {Grp. gram.:m.}
\end{itemize}
\begin{itemize}
\item {Proveniência:(De \textunderscore balança\textunderscore )}
\end{itemize}
Peça mecânica, que, em certas máquinas, transmitte movimento a outra peça.
\section{Balancete}
\begin{itemize}
\item {fónica:cê}
\end{itemize}
\begin{itemize}
\item {Grp. gram.:m.}
\end{itemize}
\begin{itemize}
\item {Proveniência:(De \textunderscore balanço\textunderscore )}
\end{itemize}
Balanço, verificação parcial de uma escrituração commercial.
Resumo de um balanço geral.
\section{Balancim}
\begin{itemize}
\item {Grp. gram.:m.}
\end{itemize}
\begin{itemize}
\item {Grp. gram.:Pl.}
\end{itemize}
\begin{itemize}
\item {Proveniência:(De \textunderscore balançar\textunderscore )}
\end{itemize}
O mesmo que \textunderscore balanceiro\textunderscore .
Peça de vehículo, a cujas pontas se prendem os tirantes e que se liga ao carro pela parte média.
Amantilhos.
\section{Balancina}
\begin{itemize}
\item {Grp. gram.:f.}
\end{itemize}
Embarcação antiga, ponteaguda e apparelhada á latina, para pesca e navegação costeira.
\section{Balancista}
\begin{itemize}
\item {Grp. gram.:m.}
\end{itemize}
\begin{itemize}
\item {Utilização:Bras}
\end{itemize}
Empregado da aferição das balanças.
\section{Balanco}
\begin{itemize}
\item {Grp. gram.:m.}
\end{itemize}
Erva nociva, que cresce por entre as searas.
\section{Balanço}
\begin{itemize}
\item {Grp. gram.:m.}
\end{itemize}
\begin{itemize}
\item {Proveniência:(De \textunderscore balançar\textunderscore )}
\end{itemize}
Movimento oscillatório.
Agitação: \textunderscore balanço das ondas\textunderscore .
Solavanco: \textunderscore os balanços da carruagem\textunderscore .
Alteração.
Verificação, resumo, de contas commerciaes.
Verificação da receita e despesa.
Exame escrupuloso: \textunderscore dar balanço aos seus actos\textunderscore .
\section{Balancozenho}
\begin{itemize}
\item {Grp. gram.:adj.}
\end{itemize}
\begin{itemize}
\item {Utilização:Prov.}
\end{itemize}
\begin{itemize}
\item {Utilização:alg.}
\end{itemize}
\begin{itemize}
\item {Proveniência:(De \textunderscore balanço\textunderscore )}
\end{itemize}
Fraquinho, débil.
\section{Balandira}
\begin{itemize}
\item {Grp. gram.:f.}
\end{itemize}
Espécie de ave, (\textunderscore chemalopex aegyptiacus\textunderscore ).
\section{Balandra}
\begin{itemize}
\item {Grp. gram.:f.}
\end{itemize}
\begin{itemize}
\item {Proveniência:(Do b. lat. \textunderscore palandaria\textunderscore )}
\end{itemize}
Embarcação de um só mastro e com coberta.
\section{Balandrão}
\begin{itemize}
\item {Grp. gram.:m.}
\end{itemize}
O mesmo que \textunderscore balandrau\textunderscore .
\section{Balandráo}
\begin{itemize}
\item {Grp. gram.:m.}
\end{itemize}
Opa, de que usam certas irmandades, em solennidades religiosas.
Capote largo e comprido.
Antigo vestuário com capuz e mangas largas.
\section{Balandrau}
\begin{itemize}
\item {Grp. gram.:m.}
\end{itemize}
Opa, de que usam certas irmandades, em solennidades religiosas.
Capote largo e comprido.
Antigo vestuário com capuz e mangas largas.
\section{Balandronada}
\begin{itemize}
\item {Grp. gram.:f.}
\end{itemize}
\begin{itemize}
\item {Utilização:Bras}
\end{itemize}
Fanfarronada.
\section{Balanífero}
\begin{itemize}
\item {Grp. gram.:adj.}
\end{itemize}
\begin{itemize}
\item {Proveniência:(Do lat. \textunderscore balanus\textunderscore  + \textunderscore ferre\textunderscore )}
\end{itemize}
O mesmo que \textunderscore glandífero\textunderscore .
\section{Balanina}
\begin{itemize}
\item {Grp. gram.:f.}
\end{itemize}
\begin{itemize}
\item {Proveniência:(Do lat. \textunderscore balanus\textunderscore )}
\end{itemize}
Insecto coleóptero tetrâmero.
\section{Balanita}
\begin{itemize}
\item {Grp. gram.:f.}
\end{itemize}
\begin{itemize}
\item {Proveniência:(Do lat. \textunderscore balanus\textunderscore )}
\end{itemize}
Gênero de mollúscos acéphalos.
Árvore do Egypto.
Pedra preciosa, com a apparência de um topázio escuro.
\section{Balanite}
\begin{itemize}
\item {Grp. gram.:f.}
\end{itemize}
\begin{itemize}
\item {Utilização:Med.}
\end{itemize}
O mesmo que \textunderscore balanita\textunderscore .
Inflammação da membrana mucosa, que reveste a glande e a face interna do prepúcio.
\section{Balanofóreas}
\begin{itemize}
\item {Grp. gram.:f. pl.}
\end{itemize}
\begin{itemize}
\item {Proveniência:(Do gr. \textunderscore balanos\textunderscore  + \textunderscore phoros\textunderscore )}
\end{itemize}
Família de plantas monocotyledóneas.
\section{Balanoide}
\begin{itemize}
\item {Grp. gram.:adj.}
\end{itemize}
\begin{itemize}
\item {Proveniência:(Do gr. \textunderscore balanos\textunderscore  + \textunderscore eidos\textunderscore )}
\end{itemize}
Semelhante á bolota.
\section{Balanophóreas}
\begin{itemize}
\item {Grp. gram.:f. pl.}
\end{itemize}
\begin{itemize}
\item {Proveniência:(Do gr. \textunderscore balanos\textunderscore  + \textunderscore phoros\textunderscore )}
\end{itemize}
Família de plantas monocotyledóneas.
\section{Balanquim}
\begin{itemize}
\item {Grp. gram.:m.}
\end{itemize}
\begin{itemize}
\item {Proveniência:(De \textunderscore Balanquina\textunderscore , n. p. de uma povoação de Oviedo?)}
\end{itemize}
Antigo e rico vestuário de seda e oiro.
\section{Balanquinho}
\begin{itemize}
\item {Grp. gram.:m.}
\end{itemize}
\begin{itemize}
\item {Utilização:Prov.}
\end{itemize}
\begin{itemize}
\item {Utilização:alent.}
\end{itemize}
Planta, (\textunderscore avena barbata\textunderscore , Brot.)
\section{Balantas}
\begin{itemize}
\item {Grp. gram.:m. pl.}
\end{itemize}
Gentios da Guiné portuguesa.
\section{Balante}
\begin{itemize}
\item {Grp. gram.:adj.}
\end{itemize}
\begin{itemize}
\item {Proveniência:(De \textunderscore balar\textunderscore )}
\end{itemize}
Que bala.
\section{Balão}
\begin{itemize}
\item {Grp. gram.:m.}
\end{itemize}
\begin{itemize}
\item {Utilização:Ant.}
\end{itemize}
\begin{itemize}
\item {Utilização:Bras}
\end{itemize}
\begin{itemize}
\item {Proveniência:(De \textunderscore bala\textunderscore )}
\end{itemize}
Aeróstato.
Globo de vidro, para serviço de laboratório.
Merinaque, saia enfunada, com grande roda.
Globo.
Boato falso, balela.
Espécie de embarcação asiática.
Montão cónico de camadas de madeira, que, entremeadas de terra, têm no vértice um buraco, por onde se lança o fogo, para se fabricar carvão.
\section{Balar}
\begin{itemize}
\item {Grp. gram.:v. t.}
\end{itemize}
\begin{itemize}
\item {Proveniência:(Lat. \textunderscore balare\textunderscore )}
\end{itemize}
Dar balidos.
\section{Balas-reaes}
\begin{itemize}
\item {Grp. gram.:f. pl.}
\end{itemize}
Espécie de doce, que se fabricava no convento de San-Bento, de Évora.
\section{Balata}
\begin{itemize}
\item {Grp. gram.:f.}
\end{itemize}
O mesmo que \textunderscore ballada\textunderscore .
\section{Balata}
\begin{itemize}
\item {Grp. gram.:f.}
\end{itemize}
\begin{itemize}
\item {Utilização:Bras}
\end{itemize}
Seiva de massarandubeira, preparada em fórma de gutapercha e com qualidades similares ás desta.
\section{Balatina}
\begin{itemize}
\item {Grp. gram.:f.}
\end{itemize}
Producto natural de uma árvore americana, usado em Medicina, com applicações análogas ás do \textunderscore collódio\textunderscore .
\section{Balaúste}
\begin{itemize}
\item {Grp. gram.:m.}
\end{itemize}
O mesmo que \textunderscore balaústia\textunderscore .
\section{Balaústia}
\begin{itemize}
\item {Grp. gram.:f.}
\end{itemize}
\begin{itemize}
\item {Proveniência:(Gr. \textunderscore balaustion\textunderscore )}
\end{itemize}
Flôr de romeira silvestre.
Fruto carnudo e coroado, como o da romeira.
\section{Balaustino}
\begin{itemize}
\item {fónica:la-us}
\end{itemize}
\begin{itemize}
\item {Grp. gram.:adj.}
\end{itemize}
\begin{itemize}
\item {Proveniência:(De \textunderscore balaúste\textunderscore )}
\end{itemize}
Que tem côr semelhante á da roman.
\section{Balaústio}
\begin{itemize}
\item {Grp. gram.:m.}
\end{itemize}
(V.balaústia)
\section{Balaústo}
\begin{itemize}
\item {Grp. gram.:m.}
\end{itemize}
O mesmo que \textunderscore balaústia\textunderscore .
\section{Balaustrada}
\begin{itemize}
\item {fónica:la-us}
\end{itemize}
\begin{itemize}
\item {Grp. gram.:f.}
\end{itemize}
Série de balaústres.
\section{Balaustrado}
\begin{itemize}
\item {fónica:la-us}
\end{itemize}
\begin{itemize}
\item {Grp. gram.:adj.}
\end{itemize}
Guarnecido de balaústres.
\section{Balaústre}
\begin{itemize}
\item {Grp. gram.:m.}
\end{itemize}
\begin{itemize}
\item {Proveniência:(It. \textunderscore balaustro\textunderscore )}
\end{itemize}
Columnelo ou pequeno pilar, que sustenta com outros, intervallados, uma travessa ou corrimão.
\section{Balázio}
\begin{itemize}
\item {Grp. gram.:m.}
\end{itemize}
O mesmo que \textunderscore balaço\textunderscore .
\section{Balbo}
\begin{itemize}
\item {Grp. gram.:adj.}
\end{itemize}
\begin{itemize}
\item {Proveniência:(Lat. \textunderscore balbus\textunderscore )}
\end{itemize}
Gago.
\section{Balborda}
\begin{itemize}
\item {Grp. gram.:f.}
\end{itemize}
O mesmo que \textunderscore balbúrdia\textunderscore . Cf. Filinto, X, 134.
\section{Balborinho}
\begin{itemize}
\item {Grp. gram.:m.}
\end{itemize}
\begin{itemize}
\item {Utilização:Pop.}
\end{itemize}
O mesmo que \textunderscore borborinho\textunderscore .
\section{Balbuciação}
\begin{itemize}
\item {Grp. gram.:f.}
\end{itemize}
Acto de \textunderscore balbuciar\textunderscore .
\section{Balbuciadela}
\begin{itemize}
\item {Grp. gram.:f.}
\end{itemize}
\begin{itemize}
\item {Proveniência:(De \textunderscore balbuciar\textunderscore )}
\end{itemize}
Balbuciação, que excita o riso.
\section{Balbuciante}
\begin{itemize}
\item {Grp. gram.:adj.}
\end{itemize}
Que balbucia.
\section{Balbuciantemente}
\begin{itemize}
\item {Grp. gram.:adv.}
\end{itemize}
Como quem balbucia.
Confusamente.
\section{Balbuciar}
\begin{itemize}
\item {Grp. gram.:v. t.}
\end{itemize}
\begin{itemize}
\item {Grp. gram.:V. i.}
\end{itemize}
\begin{itemize}
\item {Proveniência:(Do lat. \textunderscore balbutire\textunderscore )}
\end{itemize}
Articular (palavras) imperfeitamente, como as crianças: \textunderscore balbuciar uma desculpa\textunderscore .
Gaguejar; exprimir-se confusamente ou sem conhecimento do assumpto.
\section{Balbúcie}
\begin{itemize}
\item {Grp. gram.:f.}
\end{itemize}
\begin{itemize}
\item {Proveniência:(Lat. \textunderscore balbuties\textunderscore )}
\end{itemize}
Effeito de quem balbúcia.
\section{Balbuciência}
\begin{itemize}
\item {Grp. gram.:f.}
\end{itemize}
O mesmo que \textunderscore balbúcie\textunderscore .
(Cp. \textunderscore balbuciente\textunderscore )
\section{Balbuciente}
\begin{itemize}
\item {Grp. gram.:adj.}
\end{itemize}
\begin{itemize}
\item {Proveniência:(Lat. \textunderscore balbutiens\textunderscore )}
\end{itemize}
O mesmo que \textunderscore balbuciante\textunderscore .
\section{Balbucio}
\begin{itemize}
\item {Grp. gram.:m.}
\end{itemize}
\begin{itemize}
\item {Utilização:Fig.}
\end{itemize}
Acto de \textunderscore balbuciar\textunderscore .
Ensaio.
\section{Balbúrdia}
\begin{itemize}
\item {Grp. gram.:f.}
\end{itemize}
\begin{itemize}
\item {Proveniência:(T. onom.)}
\end{itemize}
Algazarra; vozearia.
Confusão; desordem.
\section{Balburdiar}
\begin{itemize}
\item {Grp. gram.:v. t.}
\end{itemize}
Causar balbúrdia em.
Tornar confuso. Cf. Camillo, \textunderscore Narcót.\textunderscore , II, 137, onde aliás se escreveu incorrectamente \textunderscore balburdear\textunderscore .
\section{Balada}
\begin{itemize}
\item {Grp. gram.:f.}
\end{itemize}
Antigo canto, acompanhado de música.
Poesia narrativa, que reproduz tradições ou lendas.
(B. lat. \textunderscore ballata\textunderscore )
\section{Balária}
\begin{itemize}
\item {Grp. gram.:f.}
\end{itemize}
(V. \textunderscore candelária\textunderscore , planta)
\section{Balastragem}
\begin{itemize}
\item {Grp. gram.:f.}
\end{itemize}
Acção de assentar o balastro.
\section{Balastrar}
\begin{itemize}
\item {Grp. gram.:v. t.}
\end{itemize}
Cobrir de balastro.
\section{Balastreira}
\begin{itemize}
\item {Grp. gram.:f.}
\end{itemize}
\begin{itemize}
\item {Grp. gram.:f.}
\end{itemize}
\begin{itemize}
\item {Utilização:T. de Barcelos}
\end{itemize}
\begin{itemize}
\item {Proveniência:(De \textunderscore ballastro\textunderscore )}
\end{itemize}
Combóio, que conduz balastro ou outros materiaes, para reparação de vias férreas.
Máquina, que transporta balastro ou terra, de um lugar para outro, em trabalhos de terraplanagem.
\section{Balastro}
\begin{itemize}
\item {Grp. gram.:m.}
\end{itemize}
\begin{itemize}
\item {Utilização:T. de Barcelos}
\end{itemize}
\begin{itemize}
\item {Proveniência:(Ingl. \textunderscore ballast\textunderscore )}
\end{itemize}
Mistura de areia e terra, com que se cobrem as travessas, em que assentam os carris das vias férreas.
Lastro, na construcção de estradas.
\section{Balça}
\begin{itemize}
\item {Grp. gram.:f.}
\end{itemize}
\begin{itemize}
\item {Utilização:Bras}
\end{itemize}
\begin{itemize}
\item {Proveniência:(Lat. \textunderscore baltea\textunderscore , pl. de \textunderscore balteum\textunderscore )}
\end{itemize}
Matagal; terreno inculto, onde crescem arbustos espinhosos.
Tapume de ramos ou silvas.
Engaço ou folhelho das uvas, que fermenta com o mosto na dorna ou lagar.
Dorna, em que se deitam, para êsse fim, as uvas.
Capa de palha ou vime, para envolver objectos de loiça, vidro, etc.
Espécie de funil de madeira, com que se baldeia o vinho.
Jangada.
Antigo estandarte dos Templários.
Espécie de plataforma fluctuante, feita de madeira, e de fórma que possa servir para descarga de navio e, em caso de naufrágio, para salvamento da gente de bordo.
\section{Balcão}
\begin{itemize}
\item {Grp. gram.:m.}
\end{itemize}
\begin{itemize}
\item {Utilização:Prov.}
\end{itemize}
\begin{itemize}
\item {Proveniência:(Do ant. al. \textunderscore balcho\textunderscore )}
\end{itemize}
Varanda larga.
Mostrador ou mesa oblonga, em que, nos estabelecimentos commerciaes, o vendedor expõe e ajusta as mercadorias, pedidas pelo comprador.
Tabuleiro grande, em que se séca o açúcar, nos respectivos engenhos.
Espécie de plataforma, que resai do lado inferior dos camarotes de um theatro, e em que tomam assento espectadores, como em plateia supplementar.
Escada exterior da habitação.
Patamar, no cimo dessa escada.
\section{Balção}
\begin{itemize}
\item {Grp. gram.:m.}
\end{itemize}
\begin{itemize}
\item {Proveniência:(De \textunderscore balça\textunderscore )}
\end{itemize}
Funil de madeira, com que se deita o vinho em pipas e tonéis.
\section{Balcarriada}
\begin{itemize}
\item {Grp. gram.:f.}
\end{itemize}
\begin{itemize}
\item {Utilização:Ant.}
\end{itemize}
Talvez o mesmo que \textunderscore balcorreada\textunderscore :«\textunderscore Nunca tal balcarriada nesta ribeira não vi\textunderscore ». G. Vicente, I, 255.
\section{Balcedo}
\begin{itemize}
\item {fónica:cê}
\end{itemize}
\begin{itemize}
\item {Grp. gram.:m.}
\end{itemize}
\begin{itemize}
\item {Proveniência:(De \textunderscore balça\textunderscore )}
\end{itemize}
Arvoredo espêsso.
\section{Balceira}
\begin{itemize}
\item {Grp. gram.:f.}
\end{itemize}
Matagal, o mesmo que \textunderscore balça\textunderscore .
\section{Balceiro}
\begin{itemize}
\item {Grp. gram.:m.}
\end{itemize}
\begin{itemize}
\item {Grp. gram.:Adj.}
\end{itemize}
\begin{itemize}
\item {Proveniência:(De \textunderscore balça\textunderscore )}
\end{itemize}
Dorna grande, em que se lançam e pisam uvas.
Balceira.
Aquelle que dirige uma jangada.
Relativo a balça ou matagal; silvestre.
Diz-se do cão, que, caçando, entra bem nos bosques.
\section{Balcorreada}
\begin{itemize}
\item {Grp. gram.:f.}
\end{itemize}
\begin{itemize}
\item {Utilização:Ant.}
\end{itemize}
Fatuidade; presumpção balofa.
\section{Balda}
\begin{itemize}
\item {Grp. gram.:f.}
\end{itemize}
\begin{itemize}
\item {Utilização:Gír.}
\end{itemize}
\begin{itemize}
\item {Proveniência:(De \textunderscore baldo\textunderscore )}
\end{itemize}
Defeito.
Mania.
Carta de jogar, que é inútil ou não serve ao naipe do parceiro.
Algibeira de mulher.
\section{Baldada}
\begin{itemize}
\item {Grp. gram.:f.}
\end{itemize}
\begin{itemize}
\item {Proveniência:(De \textunderscore balde\textunderscore )}
\end{itemize}
Porção de líquido, contida num balde.
\section{Baldadamente}
\begin{itemize}
\item {Grp. gram.:adv.}
\end{itemize}
De modo \textunderscore baldado\textunderscore ; inutilmente.
\section{Baldado}
\begin{itemize}
\item {Grp. gram.:adj.}
\end{itemize}
\begin{itemize}
\item {Proveniência:(De \textunderscore baldar\textunderscore )}
\end{itemize}
Frustrado: \textunderscore esfórços baldados\textunderscore .
Inútil.
\section{Baldante}
\begin{itemize}
\item {Grp. gram.:m.}
\end{itemize}
\begin{itemize}
\item {Utilização:Prov.}
\end{itemize}
\begin{itemize}
\item {Utilização:alent.}
\end{itemize}
\begin{itemize}
\item {Proveniência:(De \textunderscore balda\textunderscore ? Por \textunderscore valdante\textunderscore , de \textunderscore valdo\textunderscore ?)}
\end{itemize}
Vadio.
Homem de maus costumes.
\section{Baldão}
\begin{itemize}
\item {Grp. gram.:m.}
\end{itemize}
\begin{itemize}
\item {Grp. gram.:Loc. adv.}
\end{itemize}
Contrariedade; desventura.
Trabalho baldado.
Impropério; offensa.
Onda grande e larga.
\textunderscore De baldão\textunderscore , de roldão.
\section{Baldão}
\begin{itemize}
\item {Grp. gram.:m.}
\end{itemize}
\begin{itemize}
\item {Utilização:Prov.}
\end{itemize}
\begin{itemize}
\item {Utilização:trasm.}
\end{itemize}
O mesmo que \textunderscore canamão\textunderscore .
\section{Baldão}
\begin{itemize}
\item {Utilização:Prov.}
\end{itemize}
\begin{itemize}
\item {Utilização:beir.}
\end{itemize}
Vadio, mandrião.
(Por \textunderscore valdão\textunderscore , de \textunderscore valdo\textunderscore )
\section{Baldaquim}
\begin{itemize}
\item {Grp. gram.:m.}
\end{itemize}
O mesmo que \textunderscore baldaquino\textunderscore .
\section{Baldaquinado}
\begin{itemize}
\item {Grp. gram.:adj.}
\end{itemize}
Feito á semelhança de baldaquino.
\section{Baldaquino}
\begin{itemize}
\item {Grp. gram.:m.}
\end{itemize}
Espécie de dossel, sustentado por columnas.
Pállio.
Obra architectónica, semelhante a uma corôa sustentada por columnas.
(B. lat. \textunderscore baldakinus\textunderscore , de \textunderscore Baldaco\textunderscore , alter. de \textunderscore Bagdad\textunderscore , n. p.)
\section{Baldar}
\begin{itemize}
\item {Grp. gram.:v. t.}
\end{itemize}
\begin{itemize}
\item {Grp. gram.:V. p.}
\end{itemize}
\begin{itemize}
\item {Proveniência:(De \textunderscore baldo\textunderscore )}
\end{itemize}
Frustrar.
Empregar sem bom resultado: \textunderscore baldar súpplicas\textunderscore .
Tornar inútil.
Tornar-se baldo.
Livrar-se de cartas inúteis ao jôgo.
\section{Balde}
\begin{itemize}
\item {Grp. gram.:m.}
\end{itemize}
\begin{itemize}
\item {Grp. gram.:Loc. adv.}
\end{itemize}
\begin{itemize}
\item {Proveniência:(Do rad. de \textunderscore baldo\textunderscore ?)}
\end{itemize}
Vaso grande de madeira ou de fôlha, de fórma quási cylíndrica, para vários usos domésticos e agrícolas.
Pequena pá, comprida e estreita, de que se servem os marnotos, para cortar a lama na baldeação.
\textunderscore De balde\textunderscore , o mesmo que \textunderscore debalde\textunderscore .
\section{Baldeação}
\begin{itemize}
\item {Grp. gram.:f.}
\end{itemize}
Acto de \textunderscore baldear\textunderscore .
Faixa de terreno em volta das salinas, donde se tira terra para a construcção ou reparos das mesmas salinas.
\section{Baldeadeira}
\begin{itemize}
\item {Grp. gram.:f.}
\end{itemize}
\begin{itemize}
\item {Utilização:Prov.}
\end{itemize}
\begin{itemize}
\item {Utilização:alg.}
\end{itemize}
\begin{itemize}
\item {Proveniência:(De \textunderscore baldear\textunderscore )}
\end{itemize}
O mesmo que \textunderscore colhér\textunderscore .
\section{Baldear}
\begin{itemize}
\item {Grp. gram.:v. t.}
\end{itemize}
\begin{itemize}
\item {Utilização:Bras. do N}
\end{itemize}
\begin{itemize}
\item {Proveniência:(De \textunderscore balde\textunderscore ?)}
\end{itemize}
Passar líquidos, de um vaso para outro.
Passar (mercadorias), de um para outro navio.
Passar (bagagens ou passageiros), de um para outro combóio, de um para outro vehículo.
Baloiçar.
Atirar.
Lavar com baldadas.
Transferir.
O mesmo que \textunderscore vomitar\textunderscore .
\section{Baldeiro}
\begin{itemize}
\item {Grp. gram.:adj.}
\end{itemize}
(V. \textunderscore valdeiro\textunderscore , que é fórma preferível)
\section{Baldio}
\begin{itemize}
\item {Grp. gram.:m.}
\end{itemize}
\begin{itemize}
\item {Grp. gram.:Adj.}
\end{itemize}
\begin{itemize}
\item {Proveniência:(Do ár. \textunderscore baladi\textunderscore )}
\end{itemize}
Terreno inculto, maninho.
Baldado; inútil.
Sem cultura.
Estéril.
\section{Baldista}
\begin{itemize}
\item {Grp. gram.:m.}
\end{itemize}
\begin{itemize}
\item {Proveniência:(De \textunderscore balda\textunderscore )}
\end{itemize}
Parceiro que, em certos jogos de vasa, puxa pela balda ou pelo naipe de que só tem uma carta, para, na volta, cortar a carta, puxada por outro parceiro.
\section{Baldo}
\begin{itemize}
\item {Grp. gram.:adj.}
\end{itemize}
\begin{itemize}
\item {Proveniência:(Do ár. \textunderscore batala\textunderscore , segundo Diez)}
\end{itemize}
Baldado.
Carecido.
Que, ao jôgo, não tem cartas de certo naipe.
\section{Baldo}
\begin{itemize}
\item {Grp. gram.:m.}
\end{itemize}
\begin{itemize}
\item {Utilização:T. de Trancoso}
\end{itemize}
O mesmo que \textunderscore balde\textunderscore .
\section{Baldoairo}
\begin{itemize}
\item {Grp. gram.:m.}
\end{itemize}
\begin{itemize}
\item {Utilização:Ant.}
\end{itemize}
Livro de ladainhas e orações.
\section{Baldoar}
\begin{itemize}
\item {Grp. gram.:v. t.}
\end{itemize}
\begin{itemize}
\item {Grp. gram.:V. i.}
\end{itemize}
\begin{itemize}
\item {Utilização:Prov.}
\end{itemize}
\begin{itemize}
\item {Proveniência:(De \textunderscore baldão\textunderscore ^1)}
\end{itemize}
Maltratar com baldões.
Injuriar.
Vociferar; falar, gritando.
\section{Baldoeira}
\begin{itemize}
\item {Grp. gram.:f.}
\end{itemize}
Nome, que nos Olivaes se dá á camarate.
\section{Baldoeiro}
\begin{itemize}
\item {Grp. gram.:m.}
\end{itemize}
\begin{itemize}
\item {Utilização:Prov.}
\end{itemize}
\begin{itemize}
\item {Utilização:minh.}
\end{itemize}
Abertura que, de espaço a espaço, os pedreiros deixam na parede, que constróem, para se segurar cada uma das travessas, sôbre que se formam as pranchas, em que elles andam trabalhando.
O mesmo que \textunderscore bueiro\textunderscore .
\section{Baldosa}
\begin{itemize}
\item {Grp. gram.:f.}
\end{itemize}
\begin{itemize}
\item {Utilização:Prov.}
\end{itemize}
\begin{itemize}
\item {Utilização:alent.}
\end{itemize}
\begin{itemize}
\item {Proveniência:(T. cast.)}
\end{itemize}
Tijolo grande e quadrado.
\section{Baldosinha}
\begin{itemize}
\item {Grp. gram.:f.}
\end{itemize}
\begin{itemize}
\item {Utilização:T. de Serpa}
\end{itemize}
Espécie de tijolo para ladrilhos, mais pequeno que a \textunderscore baldosa\textunderscore .
\section{Baldoso}
\begin{itemize}
\item {Grp. gram.:adj.}
\end{itemize}
\begin{itemize}
\item {Proveniência:(De \textunderscore balde\textunderscore )}
\end{itemize}
Que procede de balde.
Que se esforça inutilmente. Cf. Filinto, VI, 311.
\section{Baldrame}
\begin{itemize}
\item {Grp. gram.:m.}
\end{itemize}
\begin{itemize}
\item {Utilização:Bras}
\end{itemize}
Alicerce de alvenaria.
Base de parede ou muralha.
\section{Baldrejado}
\begin{itemize}
\item {Grp. gram.:adj.}
\end{itemize}
\begin{itemize}
\item {Utilização:Ant.}
\end{itemize}
Sujo, enxovalhado:«\textunderscore olhai cá dona Civil, baldrejada como breviário de igreja\textunderscore ». \textunderscore Eufrosina\textunderscore , 279.
\section{Baldréu}
\begin{itemize}
\item {Grp. gram.:m.}
\end{itemize}
Pellica para luvas.
(Cp. \textunderscore boldrié\textunderscore )
\section{Baldroca}
\begin{itemize}
\item {Grp. gram.:f.}
\end{itemize}
\begin{itemize}
\item {Utilização:Pop.}
\end{itemize}
Fraude; trapaça.
\section{Baldrocar}
\begin{itemize}
\item {Grp. gram.:v. t.}
\end{itemize}
Fazer baldrocas a.
\section{Bale}
\begin{itemize}
\item {Grp. gram.:m.}
\end{itemize}
\begin{itemize}
\item {Utilização:Ant.}
\end{itemize}
\begin{itemize}
\item {Proveniência:(Do ár. \textunderscore vali\textunderscore )}
\end{itemize}
O mesmo que \textunderscore catual\textunderscore . Cf. \textunderscore Roteiro de Vasco da Gama\textunderscore .
\section{Baleação}
\begin{itemize}
\item {Grp. gram.:f.}
\end{itemize}
\begin{itemize}
\item {Utilização:Ant.}
\end{itemize}
\begin{itemize}
\item {Proveniência:(De \textunderscore baleia\textunderscore )}
\end{itemize}
Pesca de baleias.
Azeite, que se extrái das baleias.
\section{Baleal}
\begin{itemize}
\item {Grp. gram.:m.}
\end{itemize}
Ponto marítimo, em que abundam baleias.
Lugar costeiro, de que se aproximam muitas baleias ou em que vivem ou viveram pescadores de baleias.
\section{Balear}
\begin{itemize}
\item {Grp. gram.:v. t.}
\end{itemize}
\begin{itemize}
\item {Utilização:Prov.}
\end{itemize}
\begin{itemize}
\item {Utilização:trasm.}
\end{itemize}
\begin{itemize}
\item {Utilização:Bras}
\end{itemize}
\begin{itemize}
\item {Grp. gram.:Adj.}
\end{itemize}
\begin{itemize}
\item {Proveniência:(De \textunderscore bala\textunderscore )}
\end{itemize}
Limpar com o baleio (o pão nas eiras).
Ferir com bala.
Próprio para dar impulso ou para se arremessar.
\section{Baleárico}
\begin{itemize}
\item {Grp. gram.:adj.}
\end{itemize}
Relativo ás ílhas Baleares.
\section{Baleato}
\begin{itemize}
\item {Grp. gram.:m.}
\end{itemize}
O mesmo que \textunderscore baleote\textunderscore .
\section{Baleeira}
\begin{itemize}
\item {Grp. gram.:f.}
\end{itemize}
Barca, para a pesca de baleias.
\section{Baleeiro}
\begin{itemize}
\item {Grp. gram.:m.}
\end{itemize}
\begin{itemize}
\item {Grp. gram.:Adj.}
\end{itemize}
Pescador de baleias.
Baleeira.
Relativo a baleias.
\section{Balegões}
\begin{itemize}
\item {Grp. gram.:m. pl.}
\end{itemize}
\begin{itemize}
\item {Utilização:Ant.}
\end{itemize}
Borzeguins.
\section{Baleia}
\begin{itemize}
\item {Grp. gram.:f.}
\end{itemize}
\begin{itemize}
\item {Proveniência:(Lat. \textunderscore balaena\textunderscore )}
\end{itemize}
Corpolento mammífero, da ordem dos cetáceos.
Constellação austral.
\section{Baleio}
\begin{itemize}
\item {Grp. gram.:m.}
\end{itemize}
\begin{itemize}
\item {Utilização:Prov.}
\end{itemize}
\begin{itemize}
\item {Utilização:trasm.}
\end{itemize}
\begin{itemize}
\item {Proveniência:(De \textunderscore balear\textunderscore )}
\end{itemize}
Planta herbácea.
Escovalho, com que se varre o grão na eira, e que se faz daquella planta.
\section{Baleira}
\begin{itemize}
\item {Grp. gram.:f.}
\end{itemize}
\begin{itemize}
\item {Utilização:Prov.}
\end{itemize}
\begin{itemize}
\item {Utilização:alent.}
\end{itemize}
Molde, para fundir balas.
\section{Baleiro}
\begin{itemize}
\item {Grp. gram.:m.}
\end{itemize}
\begin{itemize}
\item {Utilização:T. do Rio-de-Janeiro}
\end{itemize}
\begin{itemize}
\item {Proveniência:(De \textunderscore bala\textunderscore )}
\end{itemize}
Vendedor de rebuçados.
\section{Balela}
\begin{itemize}
\item {Grp. gram.:f.}
\end{itemize}
Boato falso; notícia infundada.
\section{Balema}
\begin{itemize}
\item {Grp. gram.:f.}
\end{itemize}
\begin{itemize}
\item {Utilização:Náut.}
\end{itemize}
Cabo, que prende as ostagas ás vêrgas.
\section{Balenação}
\begin{itemize}
\item {Grp. gram.:f.}
\end{itemize}
\begin{itemize}
\item {Utilização:Ant.}
\end{itemize}
\begin{itemize}
\item {Proveniência:(Do lat. \textunderscore balaena\textunderscore )}
\end{itemize}
Pesca de baleias.
\section{Baleote}
\begin{itemize}
\item {Grp. gram.:m.}
\end{itemize}
Baleia pequena.
O filho da baleia.
\section{Balesta}
\begin{itemize}
\item {Grp. gram.:f.}
\end{itemize}
O mesmo que \textunderscore balestra\textunderscore . Cf. \textunderscore Viriato Trág.\textunderscore , X, 85.
\section{Balestilha}
\begin{itemize}
\item {Grp. gram.:f.}
\end{itemize}
\begin{itemize}
\item {Proveniência:(Do lat. \textunderscore ballista\textunderscore )}
\end{itemize}
Instrumento de alveitaria, para sangrar.
Instrumento náutico, o mesmo que \textunderscore radiómetro\textunderscore .
\section{Balestra}
\begin{itemize}
\item {Grp. gram.:f.}
\end{itemize}
\begin{itemize}
\item {Proveniência:(Do lat. \textunderscore ballista\textunderscore )}
\end{itemize}
O mesmo que \textunderscore bésta\textunderscore .
\section{Balestreiro}
\begin{itemize}
\item {Grp. gram.:m.}
\end{itemize}
\begin{itemize}
\item {Proveniência:(De \textunderscore balestra\textunderscore )}
\end{itemize}
Pequeno vão, feito na bacia de uma sacada, no grosso de uma cornija elevada das tôrres medievaes, para se lançarem por êlle béstas, quaesquer projécteis, ou matérias inflammadas, sôbre os sitiantes.
\section{Balestrilha}
\begin{itemize}
\item {Grp. gram.:f.}
\end{itemize}
Instrumento náutico, o mesmo que \textunderscore balestilha\textunderscore .
\section{Balga}
\begin{itemize}
\item {Grp. gram.:f.}
\end{itemize}
\begin{itemize}
\item {Utilização:Prov.}
\end{itemize}
\begin{itemize}
\item {Utilização:trasm.}
\end{itemize}
Palha, que não soffreu trilho e só foi malhada, servindo para colmados.
\section{Balha}
\begin{itemize}
\item {Grp. gram.:f.}
\end{itemize}
Teia de torneio; estacada, vallo:«\textunderscore partiu-se o terreiro com hũa tea ou balha muito bem pintada.\textunderscore »\textunderscore Relação das Festas na Canonização de S. Ignácio\textunderscore , f. 170.
\textunderscore Chamar á balha\textunderscore , provocar.
\textunderscore Vir á balha\textunderscore , vir a propósito; fazer-se lembrado oportunamente.
(Cast. \textunderscore valla\textunderscore )
\section{Balha}
\begin{itemize}
\item {Grp. gram.:f.}
\end{itemize}
\begin{itemize}
\item {Utilização:Ant.}
\end{itemize}
O mesmo que \textunderscore baila\textunderscore ^2.
\section{Balhada}
\begin{itemize}
\item {Grp. gram.:f.}
\end{itemize}
\begin{itemize}
\item {Utilização:Prov.}
\end{itemize}
\begin{itemize}
\item {Utilização:trasm.}
\end{itemize}
Gordura pendente, no pescoço ou na barriga.
(Cp. \textunderscore balhau\textunderscore )
\section{Balhadeira}
\begin{itemize}
\item {Grp. gram.:f.}
\end{itemize}
Nome de um peixe.
\section{Balhadeiro}
\begin{itemize}
\item {Grp. gram.:adj.}
\end{itemize}
\begin{itemize}
\item {Utilização:Des.}
\end{itemize}
\begin{itemize}
\item {Proveniência:(De \textunderscore balhar\textunderscore )}
\end{itemize}
Que balha. Cf. Filinto, IX, 145.
\section{Balhana}
\begin{itemize}
\item {Grp. gram.:f.}
\end{itemize}
\begin{itemize}
\item {Utilização:Prov.}
\end{itemize}
\begin{itemize}
\item {Utilização:alent.}
\end{itemize}
\begin{itemize}
\item {Proveniência:(De \textunderscore balha\textunderscore ^1?)}
\end{itemize}
Porção de mobília.
Conjunto de trastes ou utensílios.
\section{Balhão}
\begin{itemize}
\item {Grp. gram.:m.}
\end{itemize}
(V.bailão)
\section{Balhar}
\begin{itemize}
\item {Grp. gram.:v. t.  e  i.}
\end{itemize}
\begin{itemize}
\item {Utilização:pop.}
\end{itemize}
\begin{itemize}
\item {Utilização:Ant.}
\end{itemize}
\begin{itemize}
\item {Proveniência:(Do lat. \textunderscore ballare\textunderscore )}
\end{itemize}
O mesmo que \textunderscore bailar\textunderscore .
\section{Balharico}
\begin{itemize}
\item {Grp. gram.:m.}
\end{itemize}
\begin{itemize}
\item {Utilização:Pop.}
\end{itemize}
\begin{itemize}
\item {Proveniência:(De \textunderscore balhar\textunderscore )}
\end{itemize}
O mesmo que \textunderscore bailarico\textunderscore .
\section{Balharim}
\begin{itemize}
\item {Grp. gram.:m.}
\end{itemize}
\begin{itemize}
\item {Utilização:T. de Serpa}
\end{itemize}
Tijolo fino, com que se ladrilhavam as salas das casas ricas.
\section{Balharota}
\begin{itemize}
\item {Grp. gram.:f.}
\end{itemize}
\begin{itemize}
\item {Utilização:Prov.}
\end{itemize}
Espécie de dança antiga:«\textunderscore toda arrebicada por balharotas\textunderscore ». Herculano, \textunderscore Pár. de Aldeia\textunderscore , c. III.
Bugalho redondo, a que as crianças adaptam um pé, para o fazer girar como um pião.
(Colhido em Turquel)
\section{Balharote}
\begin{itemize}
\item {Grp. gram.:m.}
\end{itemize}
O mesmo que \textunderscore bailharote\textunderscore .
\section{Balhastros}
\begin{itemize}
\item {Grp. gram.:m. pl.}
\end{itemize}
\begin{itemize}
\item {Utilização:Prov.}
\end{itemize}
\begin{itemize}
\item {Utilização:alg.}
\end{itemize}
\begin{itemize}
\item {Utilização:Chul.}
\end{itemize}
Mobília ou utensílios domésticos.
(Cp. \textunderscore balhana\textunderscore )
\section{Balhau}
\begin{itemize}
\item {Grp. gram.:m.}
\end{itemize}
\begin{itemize}
\item {Utilização:Prov.}
\end{itemize}
\begin{itemize}
\item {Utilização:trasm.}
\end{itemize}
\begin{itemize}
\item {Utilização:Prov.}
\end{itemize}
\begin{itemize}
\item {Utilização:minh.}
\end{itemize}
\begin{itemize}
\item {Proveniência:(De \textunderscore balhar\textunderscore ?)}
\end{itemize}
Mulher gorda e desajeitada, mal feita.
Rapaz ou rapariga, que brinca, saltando com muita desenvoltura.
\section{Balhesta}
\begin{itemize}
\item {Grp. gram.:f.}
\end{itemize}
\begin{itemize}
\item {Utilização:Ant.}
\end{itemize}
\begin{itemize}
\item {Proveniência:(Lat. \textunderscore ballista\textunderscore , por intermédio do cast.)}
\end{itemize}
O mesmo que \textunderscore balestra\textunderscore .
\section{Balhesteira}
\begin{itemize}
\item {Grp. gram.:f.}
\end{itemize}
\begin{itemize}
\item {Utilização:Ant.}
\end{itemize}
\begin{itemize}
\item {Proveniência:(De \textunderscore balhesta\textunderscore )}
\end{itemize}
O mesmo que \textunderscore balestreiro\textunderscore .
\section{Balhestreira}
\begin{itemize}
\item {Grp. gram.:f.}
\end{itemize}
O mesmo que \textunderscore balestreiro\textunderscore .
\section{Balhestro}
\begin{itemize}
\item {Grp. gram.:m.}
\end{itemize}
\begin{itemize}
\item {Grp. gram.:Pl.}
\end{itemize}
\begin{itemize}
\item {Utilização:Prov.}
\end{itemize}
Empecilho; tropêço:«\textunderscore quanto balhestro a mente lhe atulhava\textunderscore ». Filinto, X, 127.
O mesmo que \textunderscore balhastros\textunderscore :«\textunderscore os balhestros do moleiro\textunderscore ». Castilho, \textunderscore Mil e Um Myst.\textunderscore , 106.
\section{Balho}
\begin{itemize}
\item {Grp. gram.:m.}
\end{itemize}
\begin{itemize}
\item {Utilização:pop.}
\end{itemize}
\begin{itemize}
\item {Utilização:Ant.}
\end{itemize}
O mesmo que \textunderscore baile\textunderscore .
\section{Balho}
\begin{itemize}
\item {Grp. gram.:m.}
\end{itemize}
\begin{itemize}
\item {Utilização:Prov.}
\end{itemize}
\begin{itemize}
\item {Utilização:trasm.}
\end{itemize}
O mesmo que \textunderscore baio\textunderscore ^2.
\section{Bali}
\begin{itemize}
\item {Grp. gram.:m.}
\end{itemize}
O mesmo que \textunderscore páli\textunderscore .
\section{Baliado}
\begin{itemize}
\item {Grp. gram.:m.}
\end{itemize}
(V.bailiado)
\section{Baliana}
\begin{itemize}
\item {Grp. gram.:f.}
\end{itemize}
Planta da Guiné, de fôlhas medicinaes.
\section{Balido}
\begin{itemize}
\item {Grp. gram.:m.}
\end{itemize}
\begin{itemize}
\item {Proveniência:(Lat. hyp. \textunderscore balitus\textunderscore , de \textunderscore balire\textunderscore , por \textunderscore balare\textunderscore )}
\end{itemize}
Grito próprio da ovêlha.
\section{Balio}
\begin{itemize}
\item {Grp. gram.:m.}
\end{itemize}
(V.bailio)
\section{Balista}
\begin{itemize}
\item {Grp. gram.:f.}
\end{itemize}
\begin{itemize}
\item {Proveniência:(Lat. \textunderscore ballista\textunderscore )}
\end{itemize}
Máquina de guerra, com que se arremessavam frechas, pedras, etc.
Bésta.
Gênero de peixes.
\section{Balistário}
\begin{itemize}
\item {Grp. gram.:m.}
\end{itemize}
\begin{itemize}
\item {Proveniência:(Lat. \textunderscore ballistarius\textunderscore )}
\end{itemize}
Soldado, empregado nas balistas.
\section{Balística}
\begin{itemize}
\item {Grp. gram.:f.}
\end{itemize}
\begin{itemize}
\item {Proveniência:(De \textunderscore balista\textunderscore )}
\end{itemize}
Sciência, que trata dos projécteis.
\section{Balístico}
\begin{itemize}
\item {Grp. gram.:adj.}
\end{itemize}
Relativo á \textunderscore balística\textunderscore .
\section{Balístico}
\begin{itemize}
\item {Grp. gram.:adj.}
\end{itemize}
\begin{itemize}
\item {Utilização:Prov.}
\end{itemize}
\begin{itemize}
\item {Utilização:alent.}
\end{itemize}
\begin{itemize}
\item {Utilização:pop.}
\end{itemize}
Excellente, magnífico.
\section{Baliza}
\begin{itemize}
\item {Grp. gram.:f.}
\end{itemize}
Marco; estaca ou objecto que marca um limite.
Limite.
Sinal.
Bóia, que indíca o lugar de um baixio.
Madeiro do arcaboiço do navio.
(Cast. \textunderscore valiza\textunderscore )
\section{Balizador}
\begin{itemize}
\item {Grp. gram.:m.}
\end{itemize}
Aquelle que balíza.
\section{Balizagem}
\begin{itemize}
\item {Grp. gram.:f.}
\end{itemize}
Acto de \textunderscore balizar\textunderscore .
\section{Balizamento}
\begin{itemize}
\item {Grp. gram.:m.}
\end{itemize}
O mesmo que \textunderscore balizagem\textunderscore .
\section{Balizar}
\begin{itemize}
\item {Grp. gram.:v. t.}
\end{itemize}
Marcar com balizas.
Limitar.
\section{Ballada}
\begin{itemize}
\item {Grp. gram.:f.}
\end{itemize}
Antigo canto, acompanhado de música.
Poesia narrativa, que reproduz tradições ou lendas.
(B. lat. \textunderscore ballata\textunderscore )
\section{Ballária}
\begin{itemize}
\item {Grp. gram.:f.}
\end{itemize}
(V. \textunderscore candelária\textunderscore , planta)
\section{Ballastragem}
\begin{itemize}
\item {Grp. gram.:f.}
\end{itemize}
Acção de assentar o ballastro.
\section{Ballastrar}
\begin{itemize}
\item {Grp. gram.:v. t.}
\end{itemize}
Cobrir de ballastro.
\section{Ballastreira}
\begin{itemize}
\item {Grp. gram.:f.}
\end{itemize}
\begin{itemize}
\item {Proveniência:(De \textunderscore ballastro\textunderscore )}
\end{itemize}
Combóio, que conduz ballastro ou outros materiaes, para reparação de vias férreas.
\section{Ballastro}
\begin{itemize}
\item {Grp. gram.:m.}
\end{itemize}
\begin{itemize}
\item {Proveniência:(Ingl. \textunderscore ballast\textunderscore )}
\end{itemize}
Mistura de areia e terra, com que se cobrem as travessas, em que assentam os carris das vias férreas.
\section{Ballista}
\begin{itemize}
\item {Grp. gram.:f.}
\end{itemize}
\begin{itemize}
\item {Proveniência:(Lat. \textunderscore ballista\textunderscore )}
\end{itemize}
Máquina de guerra, com que se arremessavam frechas, pedras, etc.
Bésta.
Gênero de peixes.
\section{Ballota}
\begin{itemize}
\item {Grp. gram.:f.}
\end{itemize}
\begin{itemize}
\item {Proveniência:(Gr. \textunderscore ballote\textunderscore )}
\end{itemize}
Planta labiada, vulgarmente conhecida por \textunderscore marroio\textunderscore .
\section{Ballóteas}
\begin{itemize}
\item {Grp. gram.:f. pl.}
\end{itemize}
Subtribo de plantas, que têm por typo a \textunderscore ballota\textunderscore .
\section{Ballotina}
\begin{itemize}
\item {Grp. gram.:f.}
\end{itemize}
Princípio amargo, que é próprio da ballota.
\section{Balmaz}
\begin{itemize}
\item {Grp. gram.:m.}
\end{itemize}
O mesmo que \textunderscore balmázio\textunderscore .
\section{Balmázio}
\begin{itemize}
\item {Grp. gram.:m.}
\end{itemize}
Pequeno prego, de cabeça redonda, usado em várias indústrias.
\section{Balneação}
\begin{itemize}
\item {Grp. gram.:f.}
\end{itemize}
Acto de \textunderscore balnear\textunderscore ^2.
\section{Balnear}
\begin{itemize}
\item {Grp. gram.:adj.}
\end{itemize}
\begin{itemize}
\item {Proveniência:(Lat. \textunderscore balnearis\textunderscore )}
\end{itemize}
Relativo a banhos: \textunderscore a época balnear\textunderscore .
Que tem estabelecimentos de banhos.
Em que se tomam banhos: \textunderscore localidade balnear\textunderscore .
\section{Balnear}
\begin{itemize}
\item {Grp. gram.:v. t.}
\end{itemize}
\begin{itemize}
\item {Proveniência:(Do lat. \textunderscore balneum\textunderscore )}
\end{itemize}
Dar banho a.
\section{Balneário}
\begin{itemize}
\item {Grp. gram.:m.}
\end{itemize}
Estação balnear.
\section{Bálneas}
\begin{itemize}
\item {Grp. gram.:f. pl.}
\end{itemize}
\begin{itemize}
\item {Proveniência:(Lat. \textunderscore balnea\textunderscore , pl. de \textunderscore balneum\textunderscore )}
\end{itemize}
Banhos públicos, entre os Romanos.
\section{Balneatório}
\begin{itemize}
\item {Grp. gram.:adj.}
\end{itemize}
\begin{itemize}
\item {Proveniência:(Lat. \textunderscore balneatorius\textunderscore )}
\end{itemize}
Relativo a banhos.
\section{Balneável}
\begin{itemize}
\item {Grp. gram.:adj.}
\end{itemize}
\begin{itemize}
\item {Proveniência:(De \textunderscore balnear\textunderscore ^2)}
\end{itemize}
Próprio para banhos.
\section{Balneoterapia}
\begin{itemize}
\item {Grp. gram.:f.}
\end{itemize}
\begin{itemize}
\item {Proveniência:(De \textunderscore balnear\textunderscore ^1 + \textunderscore therapia\textunderscore )}
\end{itemize}
Tratamento de doenças por meio de banhos.
\section{Balneoterápico}
\begin{itemize}
\item {Grp. gram.:adj.}
\end{itemize}
Relativo á \textunderscore balneoterapia\textunderscore .
\section{Balneotherapia}
\begin{itemize}
\item {Grp. gram.:f.}
\end{itemize}
\begin{itemize}
\item {Proveniência:(De \textunderscore balnear\textunderscore ^1 + \textunderscore therapia\textunderscore )}
\end{itemize}
Tratamento de doenças por meio de banhos.
\section{Balneotherápico}
\begin{itemize}
\item {Grp. gram.:adj.}
\end{itemize}
Relativo á \textunderscore balneotherapia\textunderscore .
\section{Balo}
\begin{itemize}
\item {Grp. gram.:m.}
\end{itemize}
\begin{itemize}
\item {Utilização:Des.}
\end{itemize}
\begin{itemize}
\item {Proveniência:(De \textunderscore balar\textunderscore )}
\end{itemize}
O mesmo que \textunderscore balido\textunderscore .
\section{Baló}
\begin{itemize}
\item {Grp. gram.:m.}
\end{itemize}
Árvore da Índia Portuguesa.
\section{Baloba}
\begin{itemize}
\item {Grp. gram.:f.}
\end{itemize}
\begin{itemize}
\item {Utilização:T. da Guiné}
\end{itemize}
Cabana, que serve de templo ao deus principal dos indígenas.
\section{Baloches}
\begin{itemize}
\item {Grp. gram.:m. pl.}
\end{itemize}
Habitantes do país, que deveria chamar-se Balochistão, mas que chamam \textunderscore Beluchistão\textunderscore . Cf. Barros, \textunderscore Déc.\textunderscore  III, l. VII, c. 2.
\section{Balofice}
\begin{itemize}
\item {Grp. gram.:f.}
\end{itemize}
Acto ou qualidade de balofo.
\section{Balofo}
\begin{itemize}
\item {fónica:lô}
\end{itemize}
\begin{itemize}
\item {Grp. gram.:adj.}
\end{itemize}
Fofo; volumoso, sem consistência: \textunderscore pão balofo\textunderscore .
Vão.
Cuja apparência excede a realidade: \textunderscore importáncia balofa\textunderscore .
Impostor.
Adiposo.
\section{Baloiçador}
\begin{itemize}
\item {Grp. gram.:m.}
\end{itemize}
Aquelle que baloiça.
\section{Baloiçamento}
\begin{itemize}
\item {Grp. gram.:f.}
\end{itemize}
Acção de \textunderscore baloiçar\textunderscore .
\section{Baloiçar}
\begin{itemize}
\item {Grp. gram.:v. t.}
\end{itemize}
\begin{itemize}
\item {Proveniência:(De \textunderscore baloiço\textunderscore )}
\end{itemize}
O mesmo que \textunderscore balançar\textunderscore .
Sacudir.
\section{Baloiço}
\begin{itemize}
\item {Grp. gram.:m.}
\end{itemize}
Balanço.
Acto de agitar, de sacudir.
Faixa, rede, corda, ou tábua suspensa, em que se baloiçam as crianças; retoiça.
\section{Baloiçoso}
\begin{itemize}
\item {Grp. gram.:adj.}
\end{itemize}
Que se baloiça; que bamboleia. Cf. Castilho, \textunderscore Fastos\textunderscore , II, 135.
\section{Baloiso}
\begin{itemize}
\item {Grp. gram.:m.}
\end{itemize}
\begin{itemize}
\item {Utilização:Prov.}
\end{itemize}
\begin{itemize}
\item {Utilização:alg.}
\end{itemize}
\begin{itemize}
\item {Proveniência:(De \textunderscore bala\textunderscore )}
\end{itemize}
Pedra muito grande.
\section{Balona}
\begin{itemize}
\item {Grp. gram.:f.}
\end{itemize}
\begin{itemize}
\item {Utilização:Ant.}
\end{itemize}
\begin{itemize}
\item {Proveniência:(Do fr. \textunderscore walonne\textunderscore )}
\end{itemize}
Collarinho pendente sôbre os ombros, como o usam hoje algumas crianças.
\section{Balona}
\begin{itemize}
\item {Grp. gram.:f.}
\end{itemize}
Espécie de bomba que, arremessada a grande altura por meio de um morteiro, explode, largando lágrimas e outros fogos de côr.
\section{Balordo}
\begin{itemize}
\item {fónica:lôr}
\end{itemize}
\begin{itemize}
\item {Grp. gram.:adj.}
\end{itemize}
\begin{itemize}
\item {Utilização:P. us.}
\end{itemize}
\begin{itemize}
\item {Grp. gram.:M.}
\end{itemize}
\begin{itemize}
\item {Utilização:Ant.}
\end{itemize}
Sujo, embodegado.
Homem bronco, estúpido.
(Cp. it. \textunderscore balordo\textunderscore )
\section{Balota}
\begin{itemize}
\item {Grp. gram.:f.}
\end{itemize}
\begin{itemize}
\item {Proveniência:(Gr. \textunderscore ballote\textunderscore )}
\end{itemize}
Planta labiada, vulgarmente conhecida por \textunderscore marroio\textunderscore .
\section{Balote}
\begin{itemize}
\item {Grp. gram.:m.}
\end{itemize}
\begin{itemize}
\item {Utilização:Prov.}
\end{itemize}
\begin{itemize}
\item {Utilização:alent.}
\end{itemize}
\begin{itemize}
\item {Utilização:pop.}
\end{itemize}
\begin{itemize}
\item {Proveniência:(De \textunderscore bala\textunderscore )}
\end{itemize}
Pequena bala.
Fardo de algodão.
Copo de vinho.
Bolinha de massa explosiva que, embrulhada em papel e atirada sôbre uma superfície dura, estala com fôrça; estalo.
\section{Balóteas}
\begin{itemize}
\item {Grp. gram.:f. pl.}
\end{itemize}
Subtribo de plantas, que têm por typo a \textunderscore balota\textunderscore .
\section{Baloteiro}
\begin{itemize}
\item {Grp. gram.:m.}
\end{itemize}
\begin{itemize}
\item {Utilização:Prov.}
\end{itemize}
\begin{itemize}
\item {Utilização:alent.}
\end{itemize}
\begin{itemize}
\item {Utilização:pop.}
\end{itemize}
\begin{itemize}
\item {Proveniência:(De \textunderscore balote\textunderscore )}
\end{itemize}
Aquelle que bebe muitos copos de vinho.
\section{Balótiga}
\begin{itemize}
\item {Grp. gram.:f.}
\end{itemize}
\begin{itemize}
\item {Utilização:Prov.}
\end{itemize}
\begin{itemize}
\item {Utilização:trasm.}
\end{itemize}
O mesmo que \textunderscore abrótea\textunderscore .
\section{Balotina}
\begin{itemize}
\item {Grp. gram.:f.}
\end{itemize}
Princípio amargo, que é próprio da balota.
\section{Balouçador}
\begin{itemize}
\item {Grp. gram.:m.}
\end{itemize}
Aquelle que balouça.
\section{Balouçamento}
\begin{itemize}
\item {Grp. gram.:f.}
\end{itemize}
Acção de \textunderscore balouçar\textunderscore .
\section{Balouçar}
\begin{itemize}
\item {Grp. gram.:v. t.}
\end{itemize}
\begin{itemize}
\item {Proveniência:(De \textunderscore baloiço\textunderscore )}
\end{itemize}
O mesmo que \textunderscore balançar\textunderscore .
Sacudir.
\section{Balouço}
\begin{itemize}
\item {Grp. gram.:m.}
\end{itemize}
Balanço.
Acto de agitar, de sacudir.
Faixa, rede, corda, ou tábua suspensa, em que se balouçam as crianças; retoiça.
\section{Balravento}
\begin{itemize}
\item {Grp. gram.:m.}
\end{itemize}
O mesmo que \textunderscore barlavento\textunderscore .
\section{Balrôa}
\begin{itemize}
\item {Grp. gram.:f.}
\end{itemize}
Espécie de arpéu, com que se abordam as embarcações.
\section{Balroar}
\textunderscore v. t.\textunderscore  (e der.)
O mesmo que \textunderscore abalroar\textunderscore , etc. Cf. Camillo, \textunderscore Livro Negro\textunderscore , 297.
\section{Balsa}
\textunderscore f.\textunderscore  (e der.)
(V. \textunderscore balça\textunderscore ^1, etc.)
\section{Balsa}
\begin{itemize}
\item {Grp. gram.:f.}
\end{itemize}
\begin{itemize}
\item {Utilização:Açor}
\end{itemize}
Salgadeira.
\section{Balsamadina}
\begin{itemize}
\item {Grp. gram.:f.}
\end{itemize}
\begin{itemize}
\item {Proveniência:(De \textunderscore bálsamo\textunderscore )}
\end{itemize}
Glândula vegetal, que segrega óleo cheiroso.
\section{Balsamar}
\begin{itemize}
\item {Grp. gram.:v. i.}
\end{itemize}
Destillar bálsamo.
Exhalar aroma.
\section{Balsamária}
\begin{itemize}
\item {Grp. gram.:f.}
\end{itemize}
\begin{itemize}
\item {Proveniência:(De \textunderscore bálsamo\textunderscore )}
\end{itemize}
Árvore indiana.
\section{Balsameia}
\begin{itemize}
\item {Grp. gram.:f.}
\end{itemize}
Suco, que se extrái da planta \textunderscore bálsamo\textunderscore .
\section{Balsameiro}
\begin{itemize}
\item {Grp. gram.:m.}
\end{itemize}
Árvore do bálsamo.
\section{Balsâmeo}
\begin{itemize}
\item {Grp. gram.:adj.}
\end{itemize}
Feito de bálsamo.
\section{Balsamica}
\begin{itemize}
\item {Grp. gram.:f.}
\end{itemize}
O mesmo que \textunderscore balsamina\textunderscore . Cf. \textunderscore Bibl. da G. do Campo\textunderscore , 338.
\section{Balsâmico}
\begin{itemize}
\item {Grp. gram.:adj.}
\end{itemize}
Que tem propriedades do bálsamo.
Perfumado; aromático.
\section{Balsamina}
\begin{itemize}
\item {Grp. gram.:f.}
\end{itemize}
\begin{itemize}
\item {Proveniência:(Do lat. \textunderscore balsaminus\textunderscore )}
\end{itemize}
Planta cucurbitácea.
Planta tropeólea.
\section{Balsamináceas}
\begin{itemize}
\item {Grp. gram.:f. pl.}
\end{itemize}
O mesmo ou melhor que \textunderscore balsamíneas\textunderscore .
\section{Balsamíneas}
\begin{itemize}
\item {Grp. gram.:f. pl.}
\end{itemize}
\begin{itemize}
\item {Proveniência:(De \textunderscore balsamina\textunderscore )}
\end{itemize}
Plantas dicotyledóneas, de corolla polypétala e estames hypogíneos.
\section{Balsamita}
\begin{itemize}
\item {Grp. gram.:f.}
\end{itemize}
\begin{itemize}
\item {Proveniência:(De \textunderscore bálsamo\textunderscore )}
\end{itemize}
Planta, conhecida também por \textunderscore hortelan romana\textunderscore .
\section{Balsamizar}
\begin{itemize}
\item {Grp. gram.:v. t.}
\end{itemize}
\begin{itemize}
\item {Utilização:Neol.}
\end{itemize}
\begin{itemize}
\item {Utilização:Fig.}
\end{itemize}
\begin{itemize}
\item {Proveniência:(De \textunderscore bálsamo\textunderscore )}
\end{itemize}
Tornar balsâmico, aromatizar.
Amenizar.
Alliviar: \textunderscore balsamizar dores\textunderscore .
\section{Bálsamo}
\begin{itemize}
\item {Grp. gram.:m.}
\end{itemize}
\begin{itemize}
\item {Utilização:Ext.}
\end{itemize}
\begin{itemize}
\item {Utilização:Fig.}
\end{itemize}
\begin{itemize}
\item {Utilização:Bot.}
\end{itemize}
\begin{itemize}
\item {Utilização:Gír.}
\end{itemize}
\begin{itemize}
\item {Proveniência:(Lat. \textunderscore balsamum\textunderscore )}
\end{itemize}
Substância aromática de alguns vegetaes.
Perfume.
Confôrto, consolação.
Designação de várias plantas.
Vinho.
\section{Balsamoide}
\begin{itemize}
\item {Grp. gram.:m.}
\end{itemize}
\begin{itemize}
\item {Proveniência:(Do gr. \textunderscore balsamon\textunderscore  + \textunderscore eidos\textunderscore )}
\end{itemize}
Qualquer aroma, cujo typo está no bálsamo.
\section{Balsana}
\begin{itemize}
\item {Grp. gram.:f.}
\end{itemize}
Fita, com que se debruava por baixo o habito dos frades.
\section{Balsão}
\begin{itemize}
\item {Grp. gram.:m.}
\end{itemize}
Estandarte antigo.
Bandeira.
Insígnia.
\section{Balsar}
\begin{itemize}
\item {Grp. gram.:v. t.}
\end{itemize}
\begin{itemize}
\item {Utilização:Gír.}
\end{itemize}
Ladrar.
\section{Balsar}
\begin{itemize}
\item {Grp. gram.:v. i.}
\end{itemize}
(Corr. pop. de \textunderscore valsar\textunderscore )
\section{Balselho}
\begin{itemize}
\item {fónica:sê}
\end{itemize}
\begin{itemize}
\item {Grp. gram.:m.}
\end{itemize}
Pequeno balso.
\section{Balso}
\begin{itemize}
\item {Grp. gram.:m.}
\end{itemize}
\begin{itemize}
\item {Utilização:Náut.}
\end{itemize}
Nó, que se arma num cabo, para içar objectos ou um homem que vai trabalhar no costado do navio, nos mastros, etc.
\section{Balso}
\begin{itemize}
\item {Grp. gram.:m.}
\end{itemize}
\begin{itemize}
\item {Utilização:Prov.}
\end{itemize}
\begin{itemize}
\item {Utilização:alg.}
\end{itemize}
\begin{itemize}
\item {Proveniência:(De \textunderscore balsar\textunderscore ^2)}
\end{itemize}
Baile; bailarico.
\section{Baltar}
\begin{itemize}
\item {Grp. gram.:adj.}
\end{itemize}
Diz-se de uma espécie de videira brava e estéril.
\section{Bálteo}
\begin{itemize}
\item {Grp. gram.:m.}
\end{itemize}
\begin{itemize}
\item {Proveniência:(Lat. \textunderscore balteus\textunderscore )}
\end{itemize}
Cinto, faixa, com que certas autoridades ecclesiásticas apertam os hábitos.
\section{Baluarte}
\begin{itemize}
\item {Grp. gram.:m.}
\end{itemize}
Bastião; fortaleza.
Construcção alta, sustentada por muralhas.
Lugar seguro.
Sustentáculo: \textunderscore o patriotismo é baluarte da independência\textunderscore .
\section{Balufera}
\begin{itemize}
\item {Grp. gram.:m.}
\end{itemize}
Instrumento músico africano, espécie de marimba.
\section{Baluga}
\begin{itemize}
\item {Grp. gram.:f.}
\end{itemize}
\begin{itemize}
\item {Utilização:T. de Alcobaça}
\end{itemize}
O mesmo que \textunderscore vagem\textunderscore .
\section{Balugas}
\begin{itemize}
\item {Grp. gram.:f. pl.}
\end{itemize}
\begin{itemize}
\item {Utilização:Ant.}
\end{itemize}
Borzeguins.
\section{Baluma}
\begin{itemize}
\item {Grp. gram.:f.}
\end{itemize}
\begin{itemize}
\item {Utilização:Náut.}
\end{itemize}
Cordel, que passa por uma baínha das velas latinas.
Parte inferior da vela.
(Cast. ant. \textunderscore baluma\textunderscore )
\section{Balurdo}
\begin{itemize}
\item {Grp. gram.:m.}
\end{itemize}
Grande parafuso, que entra na vara do lagar e suspende pelo bancal o cylindro de pedra, que obriga a vara a espremer os resíduos das uvas ou das azeitonas.
\section{Bamba}
\begin{itemize}
\item {Grp. gram.:f.}
\end{itemize}
Nome de um pássaro e de uma ave palmípede da África occidental.
\section{Bambá}
\begin{itemize}
\item {Grp. gram.:m.}
\end{itemize}
\begin{itemize}
\item {Utilização:Bras}
\end{itemize}
\begin{itemize}
\item {Proveniência:(Do quimb. \textunderscore mbamba\textunderscore , jôgo)}
\end{itemize}
Dança de negros.
Jôgo de cartas.
Bambúrrio, ao jôgo do bilhar.
Desordem, confusão.
\section{Bambá}
\begin{itemize}
\item {Grp. gram.:m.}
\end{itemize}
\begin{itemize}
\item {Utilização:Bras. do N}
\end{itemize}
Sedimento de uma variedade de azeite.
\section{Bambaleadura}
\begin{itemize}
\item {Grp. gram.:f.}
\end{itemize}
Acção de \textunderscore bambalear\textunderscore .
\section{Bambaleante}
\begin{itemize}
\item {Grp. gram.:adj.}
\end{itemize}
Que bambaleia.
\section{Bambalear}
\begin{itemize}
\item {Grp. gram.:v. i.}
\end{itemize}
O mesmo que \textunderscore bambolear\textunderscore .
\section{Bambalhão}
\begin{itemize}
\item {Grp. gram.:adj.}
\end{itemize}
\begin{itemize}
\item {Utilização:Fig.}
\end{itemize}
\begin{itemize}
\item {Proveniência:(Lat. \textunderscore bambalio\textunderscore )}
\end{itemize}
Muito bambo.
Indolente, mollangueirão.
\section{Bamba-lh'as-asas}
\begin{itemize}
\item {Grp. gram.:m.}
\end{itemize}
\begin{itemize}
\item {Utilização:Prov.}
\end{itemize}
\begin{itemize}
\item {Utilização:beir.}
\end{itemize}
O mesmo que \textunderscore trangalhadanças\textunderscore .
\section{Bambalhona}
\begin{itemize}
\item {Grp. gram.:f.}
\end{itemize}
\begin{itemize}
\item {Utilização:Pop.}
\end{itemize}
\begin{itemize}
\item {Grp. gram.:Loc. adv.}
\end{itemize}
\begin{itemize}
\item {Proveniência:(De \textunderscore bambalhão\textunderscore )}
\end{itemize}
Mulher desajeitada e mal vestida.
\textunderscore A bambalhona\textunderscore , negligentemente, sem cuidado: \textunderscore andar vestido á bambalhona\textunderscore .
\section{Bambão}
\begin{itemize}
\item {Grp. gram.:m.}
\end{itemize}
\begin{itemize}
\item {Utilização:Bras}
\end{itemize}
Nome vulgar do pedúnculo interno da jaca.
Corda bamba.
Redoiça.
\section{Bambaquerê}
\begin{itemize}
\item {Grp. gram.:m.}
\end{itemize}
\begin{itemize}
\item {Utilização:Bras. do S}
\end{itemize}
Bailarico, espécie de fandango.
Funcção, que termina em desordem.
\section{Bambar}
\begin{itemize}
\item {Grp. gram.:v. t.}
\end{itemize}
\begin{itemize}
\item {Utilização:Des.}
\end{itemize}
O mesmo que \textunderscore bambear\textunderscore .
\section{Bambaré}
\begin{itemize}
\item {Grp. gram.:m.}
\end{itemize}
O mesmo que \textunderscore babaré\textunderscore .
\section{Bambas}
\begin{itemize}
\item {Grp. gram.:m. pl.}
\end{itemize}
Pôvo africano, uma das divisões dos Fiotes.
\section{Bambê}
\begin{itemize}
\item {Grp. gram.:m.}
\end{itemize}
\begin{itemize}
\item {Utilização:Bras}
\end{itemize}
\begin{itemize}
\item {Proveniência:(Do quimb. \textunderscore mbambi\textunderscore )}
\end{itemize}
Renque de mato, que serve de linha divisória entre duas roças.
\section{Bambear}
\begin{itemize}
\item {Grp. gram.:v. t.}
\end{itemize}
Tornar bambo.
\section{Bambeza}
\begin{itemize}
\item {Grp. gram.:f.}
\end{itemize}
Qualidade de bambo.
\section{Bambinar}
\begin{itemize}
\item {Grp. gram.:v. i.}
\end{itemize}
\begin{itemize}
\item {Utilização:bras}
\end{itemize}
\begin{itemize}
\item {Utilização:Neol.}
\end{itemize}
Agitar-se com a aragem ou com o movimento de andar.
(Cp. \textunderscore bambinela\textunderscore )
\section{Bambinela}
\begin{itemize}
\item {Grp. gram.:f.}
\end{itemize}
Cortina, com que se adornam as janelas interiormente.
(Cp. it. \textunderscore bandinella\textunderscore )
\section{Bambo}
\begin{itemize}
\item {Grp. gram.:adj.}
\end{itemize}
\begin{itemize}
\item {Proveniência:(Do rad. do gr. \textunderscore bambalos\textunderscore )}
\end{itemize}
Froixo; lasso.
Diz-se da corda, que não está retesada pelas extremidades.
\section{Bamboante}
\begin{itemize}
\item {Grp. gram.:adj.}
\end{itemize}
Que bambôa.
\section{Bamboar}
\begin{itemize}
\item {Grp. gram.:v. i.}
\end{itemize}
O mesmo que \textunderscore bambolear\textunderscore .
\section{Bambocha}
\begin{itemize}
\item {Grp. gram.:f.}
\end{itemize}
\begin{itemize}
\item {Utilização:Fam.}
\end{itemize}
\begin{itemize}
\item {Grp. gram.:M.}
\end{itemize}
\begin{itemize}
\item {Utilização:Fam.}
\end{itemize}
\begin{itemize}
\item {Proveniência:(It. \textunderscore bamboccio\textunderscore )}
\end{itemize}
Gênero de pintura, também conhecido por \textunderscore bambochata\textunderscore .
Bródio, festança.
Reunião ridícula.
Folguedo lúbrico.
Aquelle que gosta de bambochatas.
\section{Bambochata}
\begin{itemize}
\item {Grp. gram.:f.}
\end{itemize}
\begin{itemize}
\item {Proveniência:(It. \textunderscore bambocciata\textunderscore )}
\end{itemize}
Pintura, que representa banquetes populares ou scenas burlescas.
Patuscada.
Extravagância.
Orgia.
\section{Bamboleamento}
\begin{itemize}
\item {Grp. gram.:m.}
\end{itemize}
Acto de \textunderscore bambolear\textunderscore .
\section{Bamboleante}
\begin{itemize}
\item {Grp. gram.:adj.}
\end{itemize}
Que bamboleia.
\section{Bambolear}
\begin{itemize}
\item {Grp. gram.:v. i.  e  p.}
\end{itemize}
\begin{itemize}
\item {Proveniência:(De \textunderscore bambo\textunderscore )}
\end{itemize}
Menear-se, balançando o corpo.
Saracotear-se.
Gingar.
\section{Bamboleatriz}
\begin{itemize}
\item {Grp. gram.:adj. f.}
\end{itemize}
Diz-se de uma coisa que bamboleia ou se agita sob a acção do tempo: \textunderscore arvoretas bamboleatrizes\textunderscore .
\section{Bamboleio}
\begin{itemize}
\item {Grp. gram.:m.}
\end{itemize}
O mesmo que \textunderscore bamboleamento\textunderscore .
\section{Bambolim}
\begin{itemize}
\item {Grp. gram.:m.}
\end{itemize}
\begin{itemize}
\item {Proveniência:(De \textunderscore bambo\textunderscore )}
\end{itemize}
Sanefa, sobreposta aos cortinados das portas ou janelas.
\section{Bambolina}
\begin{itemize}
\item {Grp. gram.:f.}
\end{itemize}
Parte do scenário, que liga superiormente os bastidores e finge o tecto, o céu, folhagem, etc.
(Cp. \textunderscore bambolim\textunderscore )
\section{Bamboré}
\begin{itemize}
\item {Grp. gram.:m.}
\end{itemize}
Planta solânea do Brasil.
\section{Bambu}
\begin{itemize}
\item {Grp. gram.:m.}
\end{itemize}
\begin{itemize}
\item {Proveniência:(T. canarim)}
\end{itemize}
Árvore gramínea da Índia.
Bastão, feito da haste dessa árvore.
\section{Bambuada}
\begin{itemize}
\item {Grp. gram.:f.}
\end{itemize}
Pancada com o bambu.
\section{Bambual}
\begin{itemize}
\item {Grp. gram.:m.}
\end{itemize}
Bosque de bambus.
\section{Bambucada}
\begin{itemize}
\item {Grp. gram.:f.}
\end{itemize}
(V.bambuada)
\section{Bambueira}
\begin{itemize}
\item {Grp. gram.:f.}
\end{itemize}
Cada um dos rebentos, que nascem da mesma raíz de bambu.
\section{Bambuelas}
\begin{itemize}
\item {Grp. gram.:m. pl.}
\end{itemize}
Povo sertanejo de Angola.
\section{Bambula}
\begin{itemize}
\item {Grp. gram.:f.}
\end{itemize}
Espécie de guitarra, feita de bambu, e usada na América do Sul.
\section{Bambum}
\begin{itemize}
\item {Grp. gram.:m.}
\end{itemize}
\begin{itemize}
\item {Proveniência:(T. creoilo de Cabo-Verde)}
\end{itemize}
Posição, que as mães, em Cabo-Verde, dão aos filhos, collocando-os ás costas, ligados por uma manta, de fórma que ellas fiquem com os braços livres para o trabalho.
\section{Bambur}
\begin{itemize}
\item {Grp. gram.:m.}
\end{itemize}
Espécie de abelha americana.
\section{Bamburral}
\begin{itemize}
\item {Grp. gram.:m.}
\end{itemize}
\begin{itemize}
\item {Utilização:Bras. do N}
\end{itemize}
Lugar alagadiço, que tem pastagens.
Pequeno arbusto aromático.
Terreno, onde cresce êsse arbusto.
\section{Bamburrice}
\begin{itemize}
\item {Grp. gram.:f.}
\end{itemize}
Hábito de fazer bambúrrios.
Effeito analogo ao bambúrrio.
\section{Bambúrrio}
\begin{itemize}
\item {Grp. gram.:m.}
\end{itemize}
\begin{itemize}
\item {Utilização:Fam.}
\end{itemize}
Fortuna inesperada.
Modo casual, com que se ganha no jôgo do bilhar e em outros.
(B. lat. \textunderscore baburrus\textunderscore , inepto)
\section{Bamburrista}
\begin{itemize}
\item {Grp. gram.:m.}
\end{itemize}
\begin{itemize}
\item {Utilização:Fam.}
\end{itemize}
Aquelle que faz bambúrrios.
Aquelle que em tudo é favorecido da fortuna.
\section{Bamburro}
\begin{itemize}
\item {Grp. gram.:m.}
\end{itemize}
\begin{itemize}
\item {Utilização:Bras}
\end{itemize}
O mesmo ou melhor que \textunderscore bambúrrio\textunderscore .
\section{Bambus}
\begin{itemize}
\item {Grp. gram.:m.}
\end{itemize}
O mesmo que \textunderscore bambu\textunderscore .
\section{Bambusa}
\begin{itemize}
\item {Grp. gram.:f.}
\end{itemize}
\begin{itemize}
\item {Proveniência:(De \textunderscore bambus\textunderscore )}
\end{itemize}
Gênero de plantas arborescentes, da fam. das gramíneas.
\section{Bambusáceas}
\begin{itemize}
\item {Grp. gram.:f. pl.}
\end{itemize}
Tríbo de plantas gramíneas, que tem por typo a bambusa.
\section{Bambusina}
\begin{itemize}
\item {Grp. gram.:f.}
\end{itemize}
\begin{itemize}
\item {Proveniência:(De \textunderscore bambu\textunderscore )}
\end{itemize}
Gênero de algas, que se cría em água doce e em pantanos.
\section{Banabóia}
\begin{itemize}
\item {Grp. gram.:m.  e  f.}
\end{itemize}
\begin{itemize}
\item {Grp. gram.:M.}
\end{itemize}
\begin{itemize}
\item {Utilização:Prov.}
\end{itemize}
\begin{itemize}
\item {Utilização:trasm.}
\end{itemize}
O mesmo que \textunderscore banazola\textunderscore .
Homem vadio.
\section{Banal}
\begin{itemize}
\item {Grp. gram.:adj.}
\end{itemize}
Que pertencia a senhores feudaes, e de que os vassallos se serviam, pagando um foro.
Relativo a êste foro.
Vulgar, trivial.--Nesta última accepção, é gallicismo, segundo alguns grammáticos.
(Do germ.)
\section{Banalidade}
\begin{itemize}
\item {Grp. gram.:f.}
\end{itemize}
Uso de coisas, pertencentes ao senhor feudal.
Trivialidade.
Bagatela.
Qualidade do que é \textunderscore banal\textunderscore .
\section{Banalizar}
\begin{itemize}
\item {Grp. gram.:v. t}
\end{itemize}
\begin{itemize}
\item {Utilização:Neol.}
\end{itemize}
Tornar banal.
\section{Banalmente}
\begin{itemize}
\item {Grp. gram.:adv.}
\end{itemize}
De modo \textunderscore banal\textunderscore ; segundo o costume geral. Cf. Camillo, \textunderscore Brasileira\textunderscore , 321.
\section{Baná-muela}
\begin{itemize}
\item {Grp. gram.:f.}
\end{itemize}
Arvoreta santhomense, de fruto semelhante á banana, mas não comestível.
\section{Banana}
\begin{itemize}
\item {Grp. gram.:f.}
\end{itemize}
\begin{itemize}
\item {Grp. gram.:M.}
\end{itemize}
\begin{itemize}
\item {Proveniência:(T. afr. da Guiné)}
\end{itemize}
Fruto da bananeira.
Pessôa sem energia; palerma.
\section{Bananada}
\begin{itemize}
\item {Grp. gram.:f.}
\end{itemize}
\begin{itemize}
\item {Utilização:Bras}
\end{itemize}
Doce da polpa de banana, engrossado até o ponto de marmelada.
\section{Bananal}
\begin{itemize}
\item {Grp. gram.:f.}
\end{itemize}
\begin{itemize}
\item {Proveniência:(De \textunderscore banana\textunderscore )}
\end{itemize}
Lugar, onde crescem bananeiras.
\section{Banana-pão}
\begin{itemize}
\item {Grp. gram.:f.}
\end{itemize}
Espécie de banana, muito apreciada em San-Thomé.
\section{Bananeira}
\begin{itemize}
\item {Grp. gram.:f.}
\end{itemize}
\begin{itemize}
\item {Proveniência:(De \textunderscore banana\textunderscore )}
\end{itemize}
Planta herbácea, originária das regiões quentes.
\section{Bananeiral}
\begin{itemize}
\item {Grp. gram.:m.}
\end{itemize}
\begin{itemize}
\item {Proveniência:(De \textunderscore bananeira\textunderscore )}
\end{itemize}
O mesmo que \textunderscore bananal\textunderscore .
\section{Bananeirinha-do-brejo}
\begin{itemize}
\item {Grp. gram.:f.}
\end{itemize}
Planta.
Talvez o mesmo que \textunderscore bananeirinha-do-mato\textunderscore .
\section{Bananeirinha-do-charco}
\begin{itemize}
\item {Grp. gram.:f.}
\end{itemize}
\begin{itemize}
\item {Utilização:Bras. de S. Paulo}
\end{itemize}
Planta.
Talvez o mesmo que \textunderscore bananeirinha-do-mato\textunderscore .
\section{Bananeirinha-do-mato}
\begin{itemize}
\item {Grp. gram.:f.}
\end{itemize}
Planta cannácea do Brasil, (\textunderscore canna pedunculata\textunderscore ).
\section{Bananista}
\begin{itemize}
\item {Grp. gram.:m.  e  adj.}
\end{itemize}
Aquelle que se occupa especialmente de bananas ou cultiva bananeiras.
\section{Bananívoro}
\begin{itemize}
\item {Grp. gram.:adj.}
\end{itemize}
\begin{itemize}
\item {Proveniência:(De \textunderscore banana\textunderscore  + lat. \textunderscore vorare\textunderscore )}
\end{itemize}
Que se alimenta de bananas.
\section{Banano}
\begin{itemize}
\item {Grp. gram.:m.}
\end{itemize}
\begin{itemize}
\item {Utilização:Chul.}
\end{itemize}
\begin{itemize}
\item {Utilização:Bras}
\end{itemize}
\begin{itemize}
\item {Utilização:chul.}
\end{itemize}
\begin{itemize}
\item {Proveniência:(T. originário da Madeira)}
\end{itemize}
Bordão, pau grosso ou comprido.
Objecto semelhante a bordão.
Paulada; pancada.
Bofetada.
Quéda.
O mesmo que \textunderscore manguito\textunderscore ^2.
\section{Bananose}
\begin{itemize}
\item {Grp. gram.:f.}
\end{itemize}
\begin{itemize}
\item {Utilização:Bras}
\end{itemize}
Farinha de banana.
\section{Bananzola}
\begin{itemize}
\item {Grp. gram.:m.  e  f.}
\end{itemize}
O mesmo que \textunderscore banazola\textunderscore .
\section{Banara}
\begin{itemize}
\item {Grp. gram.:f.}
\end{itemize}
Gênero de plantas dicotyledóneas.
\section{Banaro}
\begin{itemize}
\item {Grp. gram.:m.}
\end{itemize}
O mesmo que \textunderscore banara\textunderscore .
\section{Banaza}
\begin{itemize}
\item {Grp. gram.:m.}
\end{itemize}
Quadrúpede com três cornos, que Fernão Mendes Pinto affirma têr visto na Ásia.
\section{Banazola}
\begin{itemize}
\item {Grp. gram.:m.  e  f.}
\end{itemize}
Pessôa sem energia, imbecil.
Banana.
(Por \textunderscore bananazola\textunderscore , de \textunderscore banana\textunderscore ?)
\section{Banca}
\begin{itemize}
\item {Grp. gram.:f.}
\end{itemize}
\begin{itemize}
\item {Utilização:Prov.}
\end{itemize}
\begin{itemize}
\item {Utilização:minh.}
\end{itemize}
Mesa ordinária.
Carteira, secretária.
Escritório, profissão de advogado.
Jôgo de azar.
Quantia, que o banqueiro põe na mesa, quando começa o jôgo.
O mesmo que \textunderscore tripeça\textunderscore .
(Cp. \textunderscore banco\textunderscore )
\section{Bancá}
\begin{itemize}
\item {Grp. gram.:m.}
\end{itemize}
Planta venenosa da ilha de San-Thomé, semelhante ao trovisco.
\section{Bancada}
\begin{itemize}
\item {Grp. gram.:f.}
\end{itemize}
\begin{itemize}
\item {Proveniência:(De \textunderscore banca\textunderscore  e \textunderscore banco\textunderscore )}
\end{itemize}
Conjunto de bancos.
Reunião das pessôas, que occupam uma porção de bancos.
Levantamento das paradas, feito pelo banqueiro quando ganha, no jôgo de azar.
\section{Bancal}
\begin{itemize}
\item {Grp. gram.:m.}
\end{itemize}
\begin{itemize}
\item {Proveniência:(De \textunderscore banco\textunderscore )}
\end{itemize}
Pano de cobrir bancos.
Pano de mesa, que se põe por baixo da toalha.
Peça de ferro, chumbada na parte superior da pedra cylíndrica ou pêso dos lagares.
\section{Bancão}
\begin{itemize}
\item {Grp. gram.:m.}
\end{itemize}
Embarcação chinesa.
\section{Bancar}
\begin{itemize}
\item {Grp. gram.:v. t.}
\end{itemize}
\begin{itemize}
\item {Utilização:T. de Moncorvo}
\end{itemize}
Empar (videiras).
\section{Bancaria}
\begin{itemize}
\item {Grp. gram.:f.}
\end{itemize}
\begin{itemize}
\item {Proveniência:(De \textunderscore banco\textunderscore )}
\end{itemize}
Negociação de bullas pontifícias, por meio de banqueiros de Roma.
Grande porção de bancos.
\section{Bancário}
\begin{itemize}
\item {Grp. gram.:adj.}
\end{itemize}
Relativo a banco commercial: \textunderscore administração bancária\textunderscore .
\section{Banca-rota}
\begin{itemize}
\item {Grp. gram.:f.}
\end{itemize}
O mesmo que \textunderscore bancarrota\textunderscore .
\section{Bancarrota}
\begin{itemize}
\item {fónica:rô}
\end{itemize}
\begin{itemize}
\item {Grp. gram.:f.}
\end{itemize}
\begin{itemize}
\item {Proveniência:(It. \textunderscore banca\textunderscore  + \textunderscore rotta\textunderscore )}
\end{itemize}
Fallência commercial, quebra.
Cessação de pagamentos, por parte de um negociante, de uma empresa ou do Estado.
Quebra fraudulenta.
\section{Bancarroteiro}
\begin{itemize}
\item {Grp. gram.:m.  e  adj.}
\end{itemize}
O que faz bancarrota.
\section{Bancarrotismo}
\begin{itemize}
\item {Grp. gram.:m.}
\end{itemize}
Bancarrota, considerada como systema.
Série de bancarrotas.
\section{Banco}
\begin{itemize}
\item {Grp. gram.:m.}
\end{itemize}
Assento, geralmente tôsco, de ferro, madeira ou pedra, e de fórmas variadas.
Escabello.
Mesa estreita e oblonga, sôbre que trabalham certos artífices: \textunderscore o banco do carpinteiro\textunderscore .
Balcão de commércio.
Baixio: \textunderscore o navio despedaçou-se num banco\textunderscore .
Camada de pedra, numa pedreira.
Séde.
Tábua, em que se assentam os remadores.
Compartimento hospitalar, onde se recebem os consulentes externos: \textunderscore o ferido foi curar-se ao banco do hospital\textunderscore .
Estabelecimento de crédito, para transacções de fundos públicos ou particulares: \textunderscore Banco de Portugal\textunderscore ; \textunderscore Banco Ultramarino\textunderscore .
Edifício, onde se realizam essas transacções: \textunderscore estava á porta do banco\textunderscore .
(B. lat. \textunderscore bancum\textunderscore )
\section{Bancôa-carrapichana}
\begin{itemize}
\item {Grp. gram.:f.}
\end{itemize}
\begin{itemize}
\item {Proveniência:(De \textunderscore banco\textunderscore  + \textunderscore Carrapichana\textunderscore , n. p.)}
\end{itemize}
Bancal de lan, com listas variegadas.
\section{Bancocracia}
\begin{itemize}
\item {Grp. gram.:f.}
\end{itemize}
\begin{itemize}
\item {Utilização:Neol.}
\end{itemize}
\begin{itemize}
\item {Proveniência:(De \textunderscore banco\textunderscore  + gr. \textunderscore kratos\textunderscore )}
\end{itemize}
Influência e tyrannia dos banqueiros.
\section{Bancócrata}
\begin{itemize}
\item {Grp. gram.:m.}
\end{itemize}
Partidário da \textunderscore bancocracia\textunderscore .
\section{Bancocrático}
\begin{itemize}
\item {Grp. gram.:adj.}
\end{itemize}
Relativo á \textunderscore bancocracia\textunderscore .
\section{Banco-de-pinchar}
\begin{itemize}
\item {Grp. gram.:m.}
\end{itemize}
\begin{itemize}
\item {Utilização:Ant.}
\end{itemize}
\begin{itemize}
\item {Utilização:Heráld.}
\end{itemize}
Máquina de guerra, para bater muralha com aríete ou vaivem.
Travessão com três pernas, que se applica ás armas reaes, no chefe do escudo, para se differençarem príncipes e infantes.
\section{Banco-roto}
\begin{itemize}
\item {Grp. gram.:m.}
\end{itemize}
\begin{itemize}
\item {Utilização:Des.}
\end{itemize}
O mesmo que \textunderscore bancarrota\textunderscore :«\textunderscore qualquer que se faz amigo do mundo, faz banco-roto com Deus\textunderscore ». Heit. Pinto, \textunderscore Diál. da Lembrança da Morte\textunderscore , c. II.
\section{Bancúmbis}
\begin{itemize}
\item {Grp. gram.:m. pl.}
\end{itemize}
Tríbo cafre.
\section{Banda}
\begin{itemize}
\item {Grp. gram.:f.}
\end{itemize}
\begin{itemize}
\item {Utilização:Bras}
\end{itemize}
\begin{itemize}
\item {Utilização:Heráld.}
\end{itemize}
Parte lateral de um objecto; lado: \textunderscore nesta banda do Tejo\textunderscore .
Lista larga na borda de um vestuário, e de côr geralmente differente da dêste.
Cinta dos officiaes do exército.
Fita larga, faixa.
Tiros, disparados de uma banda do navio.
Corporação de músicos militares: \textunderscore a banda da Guarda Republicana\textunderscore .
Traseira.
Faixa, que atravessa o escudo, inclinando-se da esquerda para a direita e cruzando-se com a contra-banda.
(B. lat. \textunderscore banda\textunderscore , do gót.)
\section{Banda}
\begin{itemize}
\item {Grp. gram.:f.}
\end{itemize}
Reunião; grupo.
Companhia; bando.
(Cp. \textunderscore bando\textunderscore ^1)
\section{Banda}
\begin{itemize}
\item {Grp. gram.:f.}
\end{itemize}
Árvore de Cabinda, cujos ramos se empregam na construcção de cubatas.
\section{Banda}
\begin{itemize}
\item {Grp. gram.:f.}
\end{itemize}
Tecido indígena da Guiné portuguesa.
\section{Bandada}
\begin{itemize}
\item {Grp. gram.:f.}
\end{itemize}
Grande bando.
\section{Bandado}
\begin{itemize}
\item {Grp. gram.:m.}
\end{itemize}
\begin{itemize}
\item {Utilização:Heráld.}
\end{itemize}
Campo coberto de bandas de metal e de côr. Cf. L. Ribeiro, \textunderscore Trat. de Armaria\textunderscore .
\section{Bandagem}
\begin{itemize}
\item {Grp. gram.:f.}
\end{itemize}
\begin{itemize}
\item {Utilização:Neol.}
\end{itemize}
\begin{itemize}
\item {Proveniência:(De \textunderscore banda\textunderscore ^1)}
\end{itemize}
Acto de bandar.
Aquillo que se faz com uma ou mais bandas ou faixas.
Banda ou bandas, destinadas a certo uso.
Faixa, atadura.
Chumaços, fios, compressas.
\section{Bandalheira}
\begin{itemize}
\item {Grp. gram.:f.}
\end{itemize}
O mesmo que \textunderscore bandalhice\textunderscore .
\section{Bandalhice}
\begin{itemize}
\item {Grp. gram.:f.}
\end{itemize}
Acção ou modos de bandalho.
\section{Bandalho}
\begin{itemize}
\item {Grp. gram.:m.}
\end{itemize}
\begin{itemize}
\item {Utilização:Ant.}
\end{itemize}
\begin{itemize}
\item {Utilização:Prov.}
\end{itemize}
\begin{itemize}
\item {Utilização:minh.}
\end{itemize}
\begin{itemize}
\item {Proveniência:(De \textunderscore bando\textunderscore ?)}
\end{itemize}
Trapo, farrapo.
Homem coberto de trapos, desprezível.
Pessôa desavergonhada, sem pundonor.
Homem pretensioso e ridículo.
Pescada que, estando emmalhada por alguns dias, se torna molle e de pouco preço.
\section{Bandana}
\begin{itemize}
\item {Grp. gram.:m.}
\end{itemize}
Lenço vermelho, com desenhos brancos, primitivamente de fabricação indiana.
\section{Bandar}
\begin{itemize}
\item {Grp. gram.:v. t.}
\end{itemize}
\begin{itemize}
\item {Proveniência:(De \textunderscore banda\textunderscore ^1)}
\end{itemize}
Pôr banda em; guarnecer de bandas: \textunderscore bandar um casaco\textunderscore .
\section{Bandara}
\begin{itemize}
\item {Grp. gram.:m.}
\end{itemize}
\begin{itemize}
\item {Utilização:Ant.}
\end{itemize}
Regedor, governador, (em Malaca).
\section{Bandarilha}
\begin{itemize}
\item {Grp. gram.:f.}
\end{itemize}
Farpa, enfeitada com bandeiras ou fitas, e destinada a cravar-se no cachaço dos toiros, quando se correm.
(Por \textunderscore bandeirilha\textunderscore , cast. \textunderscore banderilla\textunderscore )
\section{Bandarilhar}
\begin{itemize}
\item {Grp. gram.:v. i.}
\end{itemize}
Pôr bandarilhas em; farpear.
\section{Bandarilheiro}
\begin{itemize}
\item {Grp. gram.:m.}
\end{itemize}
Aquelle que bandarilha toiros; toireiro; capinha.
\section{Bandarim}
\begin{itemize}
\item {Grp. gram.:m.}
\end{itemize}
\begin{itemize}
\item {Proveniência:(T. as.)}
\end{itemize}
Indíviduo que, na Índia, se emprega em extrahir a sura da palmeira.
O mesmo que \textunderscore sudra\textunderscore .
\section{Bandarra}
\begin{itemize}
\item {Grp. gram.:m.}
\end{itemize}
\begin{itemize}
\item {Grp. gram.:M.  e  f.}
\end{itemize}
\begin{itemize}
\item {Grp. gram.:F.}
\end{itemize}
Mandrião, vadio, fadista.
Pessôa, que frequenta ajuntamentos festivos.
Reunião festiva; multidão.
Moradora de alcoice, meretriz. Cf. Filinto, V, 131.
\section{Bandarrear}
\begin{itemize}
\item {Grp. gram.:v. i.}
\end{itemize}
\begin{itemize}
\item {Utilização:Des.}
\end{itemize}
\begin{itemize}
\item {Proveniência:(De \textunderscore bandarra\textunderscore )}
\end{itemize}
Vadiar; andar á tuna, á gandaia. Cf. Camillo, \textunderscore Caveira\textunderscore , 235.
\section{Bandárrico}
\begin{itemize}
\item {Grp. gram.:adj.}
\end{itemize}
Próprio do Bandarra, o célebre adivinho popular. Cf. Cortesão, \textunderscore Subs.\textunderscore 
\section{Bandarrinha}
\begin{itemize}
\item {Grp. gram.:f.}
\end{itemize}
\begin{itemize}
\item {Utilização:Prov.}
\end{itemize}
\begin{itemize}
\item {Utilização:trasm.}
\end{itemize}
Entranhas?
\textunderscore Tremer-lhe a bandarrinha\textunderscore , estar muito assustado.
\section{Bandarrista}
\begin{itemize}
\item {Grp. gram.:m.  e  f.}
\end{itemize}
Pessôa, que crê nas prophecias do Bandarra.
\section{Bandear}
\begin{itemize}
\item {Grp. gram.:v. t.}
\end{itemize}
Juntar em bando; colligar: \textunderscore bandear-se com jogadores\textunderscore .
\section{Bandear}
\begin{itemize}
\item {Grp. gram.:v. t.}
\end{itemize}
\begin{itemize}
\item {Proveniência:(De \textunderscore banda\textunderscore )}
\end{itemize}
Inclinar para a banda.
Considerar por todos os lados. Cf. Filinto, XII, 205; XVI, 199; XVIII, 243; XXII, 67.
\section{Bandeira}
\begin{itemize}
\item {Grp. gram.:f.}
\end{itemize}
\begin{itemize}
\item {Utilização:Bras}
\end{itemize}
\begin{itemize}
\item {Utilização:Bras. do N}
\end{itemize}
Pano, de uma ou mais côres, que, preso no alto de uma haste, póde desenrolar-se, fluctuando, e serve de distintivo de uma nação ou corporação; balsão, estandarte, pavilhão.
Quebra-luz, guarda-vista, pantalha, nos candeeiros.
Parte superior das portas ou janelas, que é fixa, ou que se póde abrir, separadamente das mesmas janelas ou portas.
Panícula do milho.
Catavento de metal, no alto das tôrres.
Expedição armada, mais ou menos numerosa, destinada a explorar os sertões, ou a castigar os selvagens, que prejudicam os estabelecimentos civilizados.
Espécie de tamanduá, cuja cauda tem o aspecto de uma bandeirola.
(B. lat. \textunderscore banderia\textunderscore , do germ.)
\section{Bandeirante}
\begin{itemize}
\item {Grp. gram.:m.}
\end{itemize}
\begin{itemize}
\item {Proveniência:(De \textunderscore bandeira\textunderscore )}
\end{itemize}
Indivíduo que, no Brasil, faz parte dos bandos, destinados a explorar os sertões, atacar selvagens, etc.
\section{Bandeirar}
\begin{itemize}
\item {Grp. gram.:v. i.}
\end{itemize}
\begin{itemize}
\item {Utilização:Bras}
\end{itemize}
Sêr bandeirante.
\section{Bandeireiro}
\begin{itemize}
\item {Grp. gram.:m.}
\end{itemize}
Fabricante ou vendedor de bandeiras e galhardetes.
\section{Bandeirinha}
\begin{itemize}
\item {Grp. gram.:m.  e  f.}
\end{itemize}
Pessôa muito volúvel em política.
(Dem. de \textunderscore bandeira\textunderscore )
\section{Bandeirista}
\begin{itemize}
\item {Grp. gram.:m.}
\end{itemize}
O mesmo que \textunderscore bandeirante\textunderscore .
\section{Bandeiro}
\begin{itemize}
\item {Grp. gram.:adj.}
\end{itemize}
Que pertence a um bando.
Parcial.
Volúvel.
\section{Bandeirola}
\begin{itemize}
\item {Grp. gram.:f.}
\end{itemize}
Pequena bandeira.
Panícula do milho.
\section{Bandeja}
\begin{itemize}
\item {Grp. gram.:f.}
\end{itemize}
\begin{itemize}
\item {Proveniência:(T. cast.)}
\end{itemize}
Tabuleiro de bordas baixas, que póde sêr feito de uma de várias substâncias, e destinado, ordinariamente, ao serviço de chá, café, vinhos, doces, etc.
Grande abano de palha, para limpar o trigo, aventando-o.
Escudela grande, em que comem os marinheiros.
\section{Bandejão}
\begin{itemize}
\item {Grp. gram.:m.}
\end{itemize}
Grande bandeja. Us. por Castilho.
\section{Bandejar}
\begin{itemize}
\item {Grp. gram.:v. t.}
\end{itemize}
Limpar (o trigo) com bandeja.
\section{Bandejete}
\begin{itemize}
\item {Grp. gram.:m.}
\end{itemize}
Pequena bandeja.
\section{Bandel}
\begin{itemize}
\item {Grp. gram.:m.}
\end{itemize}
\begin{itemize}
\item {Utilização:Ant.}
\end{itemize}
Bairro, destinado a habitação de estrangeiros.
(Do guzarate \textunderscore bandar\textunderscore )
\section{Bandeta}
\begin{itemize}
\item {fónica:dê}
\end{itemize}
\begin{itemize}
\item {Grp. gram.:f.}
\end{itemize}
\begin{itemize}
\item {Proveniência:(De \textunderscore banda\textunderscore )}
\end{itemize}
Chapa estreita de metal.
\section{Bandidismo}
\begin{itemize}
\item {Grp. gram.:m.}
\end{itemize}
Acto ou vida de bandido.
\section{Bandido}
\begin{itemize}
\item {Grp. gram.:m.}
\end{itemize}
\begin{itemize}
\item {Proveniência:(It. \textunderscore bandito\textunderscore )}
\end{itemize}
Salteador; homem, que, fugido á acção da justiça, vive do roubo.
\section{Bandilibós}
\begin{itemize}
\item {Grp. gram.:m.}
\end{itemize}
\begin{itemize}
\item {Utilização:Prov.}
\end{itemize}
\begin{itemize}
\item {Utilização:alent.}
\end{itemize}
Jôgo infantil. Cf. \textunderscore Rev. Lus.\textunderscore , XV, 104.
\section{Bandim}
\begin{itemize}
\item {Grp. gram.:m.}
\end{itemize}
\begin{itemize}
\item {Proveniência:(T. as.)}
\end{itemize}
Porção de território, que cabe a cada cultivador, em certas localidades da Índia.
\section{Banditismo}
\begin{itemize}
\item {Grp. gram.:m.}
\end{itemize}
\begin{itemize}
\item {Proveniência:(Do it. \textunderscore bandito\textunderscore )}
\end{itemize}
O mesmo que \textunderscore bandidismo\textunderscore .
\section{Bando}
\begin{itemize}
\item {Grp. gram.:m.}
\end{itemize}
Ajuntamento de pessôas ou animaes.
Rancho.
Facção.
Companhia de malfeitores.
(B. lat. \textunderscore bandum\textunderscore )
\section{Bando}
\begin{itemize}
\item {Grp. gram.:m.}
\end{itemize}
Proclamação, annúncio público.
Grupo de indivíduos que, percorrendo as ruas, apregôam espectáculo ou outro acontecimento.
(Cp. \textunderscore banho\textunderscore ^2)
\section{Bandó}
\begin{itemize}
\item {Grp. gram.:m.}
\end{itemize}
\begin{itemize}
\item {Proveniência:(Fr. \textunderscore bandeau\textunderscore )}
\end{itemize}
Cada uma das duas partes do cabello que, separadas desde a testa á nuca, se enrolam ou assentam nas fontes da cabeça.
\section{Bandoeiro}
\begin{itemize}
\item {Grp. gram.:adj.}
\end{itemize}
\begin{itemize}
\item {Utilização:Des.}
\end{itemize}
Que pertence a um bando ou facção.
Faccioso, parcial.
\section{Bandoga}
\begin{itemize}
\item {fónica:dô}
\end{itemize}
\begin{itemize}
\item {Grp. gram.:f.}
\end{itemize}
\begin{itemize}
\item {Utilização:Prov.}
\end{itemize}
\begin{itemize}
\item {Utilização:beir.}
\end{itemize}
O mesmo que \textunderscore bandulho\textunderscore .
\section{Bandola}
\begin{itemize}
\item {Grp. gram.:f.}
\end{itemize}
\begin{itemize}
\item {Proveniência:(De \textunderscore banda\textunderscore )}
\end{itemize}
Cinto, em que se suspende o polvorinho.
\section{Bandolear}
\begin{itemize}
\item {Grp. gram.:v. i.}
\end{itemize}
Têr vida de bandoleiro.
Andar á tuna. Cf. Filinto, X, 13.
\section{Bandoleira}
\begin{itemize}
\item {Grp. gram.:f.}
\end{itemize}
Correia a tíracollo, em que os soldados traziam a arma.
(Cast. \textunderscore bandolera\textunderscore )
\section{Bandoleirismo}
\begin{itemize}
\item {Grp. gram.:m.}
\end{itemize}
Vida de bandoleiro.
Bandidismo.
\section{Bandoleiro}
\begin{itemize}
\item {Grp. gram.:m.}
\end{itemize}
\begin{itemize}
\item {Utilização:Pop.}
\end{itemize}
\begin{itemize}
\item {Utilização:Bras}
\end{itemize}
Bandido.
Mentiroso.
Cão, que a todos acompanha indistintamente.
(Cast. \textunderscore bandolero\textunderscore )
\section{Bandoleta}
\begin{itemize}
\item {fónica:lê}
\end{itemize}
\begin{itemize}
\item {Grp. gram.:f.}
\end{itemize}
Espécie de bandolim.
\section{Bandolim}
\begin{itemize}
\item {Grp. gram.:m.}
\end{itemize}
Espécie de viola, com quatro cordas, que se toca com ponteiro, palheta ou com a unha.
(Alter. de \textunderscore mandolim\textunderscore )
\section{Bandolina}
\begin{itemize}
\item {Grp. gram.:f.}
\end{itemize}
Espécie de pomada, feita ordinariamente com mucillagem de pevide de marmello, para fixar ou lustrar o cabello.
(Cp. \textunderscore bandó\textunderscore )
\section{Bandolinista}
\begin{itemize}
\item {Grp. gram.:m.}
\end{itemize}
Tocador de bandolim.
\section{Bandónion}
\begin{itemize}
\item {Grp. gram.:m.}
\end{itemize}
Instrumento músico, recentemente inventado, espécie de acordeão quadrado, com o teclado e mecanismo semelhantes aos da concertina.
\section{Bandoria}
\begin{itemize}
\item {Grp. gram.:f.}
\end{itemize}
\begin{itemize}
\item {Utilização:Ant.}
\end{itemize}
\begin{itemize}
\item {Proveniência:(De \textunderscore bando\textunderscore )}
\end{itemize}
Bando, facção.
Revolta.
Devastação, feita por quadrilha de malfeitores.
Barulho, tropelia.
\section{Bandulho}
\begin{itemize}
\item {Grp. gram.:m.}
\end{itemize}
\begin{itemize}
\item {Utilização:Fam.}
\end{itemize}
\begin{itemize}
\item {Utilização:Ant.}
\end{itemize}
Barriga.
Intestinos.
Cunha de madeira, com que se apertavam as fôrmas typográphicas.
(Cast. \textunderscore bandujo\textunderscore , talvez do lat. hyp. \textunderscore panduculus\textunderscore , dem. do lat. \textunderscore pandus\textunderscore , curvo)
\section{Bandurra}
\begin{itemize}
\item {Grp. gram.:f.}
\end{itemize}
\begin{itemize}
\item {Utilização:Pop.}
\end{itemize}
\begin{itemize}
\item {Proveniência:(Lat. \textunderscore pandura\textunderscore )}
\end{itemize}
Espécie de guitarra.
Viola.
\section{Bandurrear}
\begin{itemize}
\item {Grp. gram.:v. i.}
\end{itemize}
Tocar bandurra.
Foliar, vadiar.
\section{Bandurrilha}
\begin{itemize}
\item {Grp. gram.:f.}
\end{itemize}
\begin{itemize}
\item {Grp. gram.:M.}
\end{itemize}
Pequena bandurra.
Tocador de bandurra.
Vadio.
\section{Bandurrista}
\begin{itemize}
\item {Grp. gram.:m.}
\end{itemize}
Tocador de bandurra.
\section{Baneanes}
\begin{itemize}
\item {Grp. gram.:m. pl.}
\end{itemize}
(V.banianos)
\section{Bangalada}
\begin{itemize}
\item {Grp. gram.:f.}
\end{itemize}
Grande insecto africano, de corpo deprimido, que canta nas árvores, e é comestível para os indígenas.
\section{Bângalas}
\begin{itemize}
\item {Grp. gram.:m. pl.}
\end{itemize}
Indígenas da Lunda, na África occidental.
\section{Bangalé}
\begin{itemize}
\item {Grp. gram.:m.}
\end{itemize}
\begin{itemize}
\item {Utilização:Prov.}
\end{itemize}
Festança campestre.
Comezaina em baiúcas campestres.
(Cp. \textunderscore bangulê\textunderscore )
\section{Bangaló}
\begin{itemize}
\item {Grp. gram.:m.}
\end{itemize}
Habitação campestre, na Índia.
(Marata \textunderscore bangala\textunderscore )
\section{Bangaloango}
\begin{itemize}
\item {Grp. gram.:m.}
\end{itemize}
Árvore africana, de flôres vermelhas.
\section{Bango}
\begin{itemize}
\item {Grp. gram.:m.}
\end{itemize}
Planta indiana, espécie de cânhamo, que fornece o principal elemento da poção narcótica chamada haxixe.
(Do persa \textunderscore beng\textunderscore )
\section{Bangue}
\begin{itemize}
\item {Grp. gram.:m.}
\end{itemize}
O mesmo que \textunderscore bango\textunderscore .
\section{Banguê}
\begin{itemize}
\item {Grp. gram.:m.}
\end{itemize}
\begin{itemize}
\item {Utilização:Bras}
\end{itemize}
Espécie de liteira, com tecto e cortinados de coiro.
Ladrilho, por onde escorrem as espumas que transbordam, nos engenhos de açúcar.
Padiola grosseira, para conducção de terra.
Padiola para conducção de cadáveres.
Anoque.
\section{Banguela}
\begin{itemize}
\item {Grp. gram.:f.}
\end{itemize}
\begin{itemize}
\item {Utilização:Bras}
\end{itemize}
Pessôa desdentada.
O mesmo que \textunderscore benguela\textunderscore .
Sujeito, que fala incorrectamente ou que pronuncía mal as palavras, como se lhe faltassem os dentes.
\section{Banguelê}
\begin{itemize}
\item {Grp. gram.:m.}
\end{itemize}
\begin{itemize}
\item {Utilização:Bras}
\end{itemize}
Desordem, briga.
\section{Bângula}
\begin{itemize}
\item {Grp. gram.:f.}
\end{itemize}
Barco de pesca, no Brasil.
\section{Bangúla}
\begin{itemize}
\item {Grp. gram.:f.}
\end{itemize}
O mesmo que \textunderscore mangula\textunderscore .
\section{Bangular}
\begin{itemize}
\item {Grp. gram.:v. i.}
\end{itemize}
\begin{itemize}
\item {Utilização:Bras}
\end{itemize}
Andar errante.
\section{Bangulê}
\begin{itemize}
\item {Grp. gram.:m.}
\end{itemize}
\begin{itemize}
\item {Utilização:Bras}
\end{itemize}
Dança de negros, ao som de cantigas obscenas.
\section{Banha}
\begin{itemize}
\item {Grp. gram.:f.}
\end{itemize}
Gordura de animaes.
Pomada para o cabello.
\section{Banhado}
\begin{itemize}
\item {Grp. gram.:m.}
\end{itemize}
\begin{itemize}
\item {Utilização:Bras}
\end{itemize}
\begin{itemize}
\item {Proveniência:(De \textunderscore banhar\textunderscore )}
\end{itemize}
Charco, encoberto por ervagem.
\section{Banhar}
\begin{itemize}
\item {Grp. gram.:v. t.}
\end{itemize}
\begin{itemize}
\item {Grp. gram.:V. i.}
\end{itemize}
\begin{itemize}
\item {Utilização:Prov.}
\end{itemize}
\begin{itemize}
\item {Utilização:alent.}
\end{itemize}
\begin{itemize}
\item {Proveniência:(De \textunderscore banho\textunderscore )}
\end{itemize}
Mergulhar em líquido: \textunderscore banhar um cão\textunderscore .
Inundar.
Cobrir de líquido.
Correr (um rio) junto de: \textunderscore o Tejo banha Santarém\textunderscore .
Humedecer: \textunderscore banhar a cabeça com uma esponja embebida em água\textunderscore .
Rodear.
Impregnar.
Colorir.
Deleitar.
Nadar.
\section{Banheira}
\begin{itemize}
\item {Grp. gram.:f.}
\end{itemize}
Tina, em que se toma banho.
Mulher, que exerce funcções de banheiro.
\section{Banheiro}
\begin{itemize}
\item {Grp. gram.:m.}
\end{itemize}
Aquelle que prepara os banhos ou ajuda alguém a tomá-los.
Dono ou administrador de estabelecimento balnear.
O mesmo que \textunderscore banheira\textunderscore , tina.
\section{Banhista}
\begin{itemize}
\item {Grp. gram.:m.  e  f.}
\end{itemize}
Pessôa, que vai tomar banhos numa praia ou em caldas.
\section{Banho}
\begin{itemize}
\item {Grp. gram.:m.}
\end{itemize}
\begin{itemize}
\item {Proveniência:(Lat. \textunderscore balneum\textunderscore )}
\end{itemize}
Acção de banhar: \textunderscore tomar um banho\textunderscore .
Líquido, em que se toma banho: \textunderscore José, prepara-me o banho\textunderscore .
Local, em que se tomam banhos: \textunderscore encontrei-o no banho\textunderscore .
Vaso, que contém uma substância em que se mergulha.
Vasílha, com matéria que há de evaporar-se ou destillar-se.
Ordem de cavallaria em Inglaterra.
\section{Banho}
\begin{itemize}
\item {Grp. gram.:m.}
\end{itemize}
Proclama de casamento.
(B. lat. \textunderscore bannum\textunderscore )
\section{Banho}
\begin{itemize}
\item {Grp. gram.:m.}
\end{itemize}
\begin{itemize}
\item {Utilização:Gal}
\end{itemize}
Galés, grilheta. Cf. B. Pato, \textunderscore Cant. e Sát.\textunderscore , 196.
(Cp. fr. \textunderscore bagne\textunderscore )
\section{Banho-maria}
\begin{itemize}
\item {Grp. gram.:m.}
\end{itemize}
Vasilha com água, collocada sobre o lume e contendo outra, com uma substância que se quere cozer, derreter ou evaporar.
\section{Banianes}
\begin{itemize}
\item {Grp. gram.:m. pl.}
\end{itemize}
O mesmo que \textunderscore banianos\textunderscore .
\section{Banianos}
\begin{itemize}
\item {Grp. gram.:m. pl.}
\end{itemize}
Seita indiana.
Negociantes indianos, que trafícam principalmente na África oriental.
\section{Banibas}
\begin{itemize}
\item {Grp. gram.:m. pl.}
\end{itemize}
Índios da América, que dominavam na Guiana brasileira.
\section{Banimento}
\begin{itemize}
\item {Grp. gram.:m.}
\end{itemize}
Acção de \textunderscore banir\textunderscore .
\section{Banir}
\begin{itemize}
\item {Grp. gram.:v. t.}
\end{itemize}
Desterrar.
Evitar.
Lançar fóra de um país.
Excluir; tirar; supprimir.
(B. lat. \textunderscore bannire\textunderscore )
\section{Banível}
\begin{itemize}
\item {Grp. gram.:adj.}
\end{itemize}
\begin{itemize}
\item {Proveniência:(De \textunderscore banir\textunderscore )}
\end{itemize}
Que merece sêr banido.
\section{Banjo}
\begin{itemize}
\item {Grp. gram.:m.}
\end{itemize}
Tôsco instrumento músico, que é usado pelos indígenas da América Espanhola e consta de um pandeiro, adicionado de um braço comprido com duas ou três cordas, por fórma que o mesmo tocador dedilha as cordas e toca no pandeiro. Cf. \textunderscore Diccion. Mus.\textunderscore 
\section{Banjoísta}
\begin{itemize}
\item {Grp. gram.:m.}
\end{itemize}
Tocador de banjo.
\section{Banqueiro}
\begin{itemize}
\item {Grp. gram.:m.}
\end{itemize}
\begin{itemize}
\item {Utilização:Bras}
\end{itemize}
\begin{itemize}
\item {Utilização:Bras}
\end{itemize}
\begin{itemize}
\item {Proveniência:(De \textunderscore banco\textunderscore  e \textunderscore banca\textunderscore )}
\end{itemize}
Aquelle que faz operações bancárias.
Director de um banco.
Proprietário de estabelecimento bancário.
Homem rico.
O encarregado, em Roma, de expedir bullas pontifícias.
Aquelle que no jôgo tira as cartas e tem dinheiro sôbre a banca para pagar aos parceiros.
Aquelle que tem banca, para jôgo de roleta.
Aquelle que de noite está encarregado da casa das caldeiras, nos engenhos de açúcar.
Banco de cortador, nos açougues.
\section{Banqueta}
\begin{itemize}
\item {fónica:quê}
\end{itemize}
\begin{itemize}
\item {Grp. gram.:f.}
\end{itemize}
\begin{itemize}
\item {Proveniência:(De \textunderscore banca\textunderscore )}
\end{itemize}
Pequena banca.
Degrau interior na muralha, atrás do parapeito, e aonde sobem os atiradores, para melhor descobrir os sitiantes.
Degrau sôbre o altar, para a collocação dos castiçaes.
Fileira dêsses castiçaes.
Espaço entre a aresta inferior do ballastro e a superior do terrapleno ou plataforma da linha férrea.
\section{Banquete}
\begin{itemize}
\item {fónica:quê}
\end{itemize}
\begin{itemize}
\item {Grp. gram.:m.}
\end{itemize}
\begin{itemize}
\item {Proveniência:(De \textunderscore banca\textunderscore )}
\end{itemize}
Refeição apparatosa, festiva.
\section{Banqueteador}
\begin{itemize}
\item {Grp. gram.:m.}
\end{itemize}
Aquelle que banqueteia.
\section{Banquetear}
\begin{itemize}
\item {Grp. gram.:v. t.}
\end{itemize}
\begin{itemize}
\item {Grp. gram.:V. p.}
\end{itemize}
Dar banquete em honra de.
Tomar parte num banquete.
Gastar muito em comidas, em refeições pomposas.
\section{Banquisa}
\begin{itemize}
\item {Grp. gram.:f.}
\end{itemize}
\begin{itemize}
\item {Utilização:Gal}
\end{itemize}
\begin{itemize}
\item {Proveniência:(Fr. \textunderscore banquise\textunderscore )}
\end{itemize}
Banco de gêlo, que impede os mareantes de tocar na barra.
\section{Banti}
\begin{itemize}
\item {Grp. gram.:m.}
\end{itemize}
Planta cerealífera de Dio.
\section{Bantim}
\begin{itemize}
\item {Grp. gram.:m.}
\end{itemize}
\begin{itemize}
\item {Proveniência:(T. as.)}
\end{itemize}
Pequena embarcação índiana.
\section{Bantineiro}
\begin{itemize}
\item {Grp. gram.:m.}
\end{itemize}
Tripulante de um bantim.
\section{Banto}
\begin{itemize}
\item {Grp. gram.:m.  e  adj.}
\end{itemize}
O mesmo que \textunderscore bântu\textunderscore .
\section{Bântu}
\begin{itemize}
\item {Grp. gram.:m.  e  adj.}
\end{itemize}
\begin{itemize}
\item {Grp. gram.:M. pl.}
\end{itemize}
Diz-se do grupo de idiomas africanos, em que a flexão se faz por prefixos.
Raça africana, que se reparte em numerosas tríbos, desde a costa occidental á oriental.
\section{Banza}
\begin{itemize}
\item {Grp. gram.:f.}
\end{itemize}
Residência de régulo, em África.
\section{Banza}
\begin{itemize}
\item {Grp. gram.:f.}
\end{itemize}
\begin{itemize}
\item {Utilização:Chul.}
\end{itemize}
O mesmo que \textunderscore viola\textunderscore ^1.
Guitarra.
\section{Banzar}
\begin{itemize}
\item {Grp. gram.:v. t.}
\end{itemize}
\begin{itemize}
\item {Grp. gram.:V. i.}
\end{itemize}
Espantar; tornar pasmado; surprehender.
Ficar espantado ou pensativo, sôbre coisa que não tem fácil explicação.
(Do quimbundo \textunderscore cubanza\textunderscore )
\section{Banzé}
\begin{itemize}
\item {Grp. gram.:m.}
\end{itemize}
\begin{itemize}
\item {Utilização:Gír.}
\end{itemize}
Desordem.
Festa ruidosa.
(Relaciona-se com \textunderscore banza\textunderscore ^2?)
\section{Banzear}
\begin{itemize}
\item {Grp. gram.:v. i.}
\end{itemize}
Estar banzeiro.
\section{Banzeiro}
\begin{itemize}
\item {Grp. gram.:adj.}
\end{itemize}
\begin{itemize}
\item {Utilização:Bras. do N}
\end{itemize}
\begin{itemize}
\item {Grp. gram.:M.}
\end{itemize}
\begin{itemize}
\item {Utilização:Bras. do N}
\end{itemize}
Diz-se do mar, que se agita brandamente.
E diz-se do jôgo, que se prolonga, sem grande differença nos resultados para os jogadores.
Um tanto bêbedo.
Cambaleante.
Vento forte.
\section{Banzo}
\begin{itemize}
\item {Grp. gram.:m.}
\end{itemize}
\begin{itemize}
\item {Grp. gram.:Adj.}
\end{itemize}
\begin{itemize}
\item {Utilização:Bras}
\end{itemize}
Nostalgia dos negros africanos.
Triste, abatido.
\section{Banzo}
\begin{itemize}
\item {Grp. gram.:m.}
\end{itemize}
Ave africana, (\textunderscore treron calva\textunderscore , Temm.).
\section{Banzos}
\begin{itemize}
\item {Grp. gram.:m. pl.}
\end{itemize}
\begin{itemize}
\item {Proveniência:(T. cast.)}
\end{itemize}
As peças parallelas da escada de mão, nas quaes se embebem os degraus.
Peças lateraes dos bastidores de bordar.
Testeiras das serras braçaes.
Braços do escano, do esquife, do andor.
Partes extremas de uma viga de ferro, formando-lhe testeiras.
\section{Baonesa}
\begin{itemize}
\item {fónica:nê}
\end{itemize}
\begin{itemize}
\item {Grp. gram.:f.  e  adj.}
\end{itemize}
O mesmo que \textunderscore baionesa\textunderscore .
\section{Bapeira}
\begin{itemize}
\item {Grp. gram.:f.}
\end{itemize}
Árvore dos sertões brasileiros.
\section{Baptísia}
\begin{itemize}
\item {Grp. gram.:f.}
\end{itemize}
\begin{itemize}
\item {Proveniência:(Do gr. \textunderscore baptein\textunderscore , tingir)}
\end{itemize}
Gênero de plantas leguminosas, medicinaes.
\section{Baptismal}
\begin{itemize}
\item {Grp. gram.:adj.}
\end{itemize}
Relativo a \textunderscore baptismo\textunderscore .
\section{Baptismo}
\begin{itemize}
\item {Grp. gram.:m.}
\end{itemize}
\begin{itemize}
\item {Utilização:Pop.}
\end{itemize}
\begin{itemize}
\item {Proveniência:(Gr. \textunderscore baptisma\textunderscore )}
\end{itemize}
Ablução; immersão.
Primeiro Sacramento da Igreja christan, que consiste na ablução externa do corpo, para que se realize a purificação da alma.
Acto de administrar aquelle Sacramento.
Festa, para o celebrar.
Iniciação; admissão solenne a uma religião.
Consagração.
Acto de dar nome a uma pessôa ou coisa.
Falsificação do vinho ou do leite, misturando-lhes água.
\section{Baptista}
\begin{itemize}
\item {Grp. gram.:m.}
\end{itemize}
\begin{itemize}
\item {Proveniência:(Gr. \textunderscore baptistes\textunderscore )}
\end{itemize}
Aquelle que baptiza.
Nome, dado por excellência ao apóstolo San-João, que baptizou Christo.
\section{Baptistas}
\begin{itemize}
\item {Grp. gram.:m. pl.}
\end{itemize}
\begin{itemize}
\item {Proveniência:(De \textunderscore baptista\textunderscore )}
\end{itemize}
Seita christan, em que o baptismo só se ministra aos adultos.
\section{Baptistério}
\begin{itemize}
\item {Grp. gram.:m.}
\end{itemize}
\begin{itemize}
\item {Proveniência:(Gr. \textunderscore baptisterion\textunderscore )}
\end{itemize}
Lugar, onde está a pia baptismal.
\section{Baptistino}
\begin{itemize}
\item {Grp. gram.:adj.}
\end{itemize}
Relativo a San-João Baptista: \textunderscore as festas baptistinas de Braga\textunderscore .
\section{Baptizado}
\begin{itemize}
\item {Grp. gram.:m.}
\end{itemize}
Baptismo.
Festa, com que se celebra o baptismo.
Cortejo das pessôas, que vão assistir ao baptismo.
\section{Baptizamento}
\begin{itemize}
\item {Grp. gram.:m.}
\end{itemize}
\begin{itemize}
\item {Utilização:Des.}
\end{itemize}
(V.baptismo)
\section{Baptizante}
\begin{itemize}
\item {Grp. gram.:adj.}
\end{itemize}
Que baptiza.
\section{Baptizar}
\begin{itemize}
\item {Grp. gram.:v. t.}
\end{itemize}
\begin{itemize}
\item {Proveniência:(Lat. \textunderscore baptizare\textunderscore )}
\end{itemize}
Administrar o baptismo a.
Dar nome, alcunha ou epitheto a.
Adulterar (certos líquidos), deitando-lhes água.
\section{Baptizo}
\begin{itemize}
\item {Grp. gram.:m.}
\end{itemize}
\begin{itemize}
\item {Utilização:Pop.}
\end{itemize}
O mesmo que \textunderscore baptizado\textunderscore .
\section{Baquara}
\begin{itemize}
\item {Grp. gram.:m. ,  f.  e  adj.}
\end{itemize}
\begin{itemize}
\item {Utilização:Bras}
\end{itemize}
Pessôa esperta, diligente.
\section{Baque}
\begin{itemize}
\item {Grp. gram.:m.}
\end{itemize}
\begin{itemize}
\item {Proveniência:(T. onom.)}
\end{itemize}
Ruído de um corpo que cái.
Quéda.
Desastre súbito.
Receio íntimo, desconfiança, supposição.
\section{Baque}
\begin{itemize}
\item {Grp. gram.:m.}
\end{itemize}
\begin{itemize}
\item {Utilização:Bras}
\end{itemize}
Instante, momento.
\section{Baquear}
\begin{itemize}
\item {Grp. gram.:v. i.}
\end{itemize}
\begin{itemize}
\item {Grp. gram.:V. p.}
\end{itemize}
Fazer baque^1.
Cair com estrondo.
Cair de repente.
Arruinar-se.
Lançar-se por terra, prostrar-se. Cf. Castilho, \textunderscore Fastos\textunderscore , II, 81.
\textunderscore Peregrinação\textunderscore , c. C e CI.
\section{Baqueche}
\begin{itemize}
\item {Grp. gram.:f.}
\end{itemize}
Planta cucurbitácea de Cabo-Verde.
\section{Baquesim}
\begin{itemize}
\item {Grp. gram.:m.}
\end{itemize}
\begin{itemize}
\item {Utilização:Gír.}
\end{itemize}
Bôlsa.
\section{Baqueta}
\begin{itemize}
\item {fónica:qué}
\end{itemize}
\begin{itemize}
\item {Grp. gram.:f.}
\end{itemize}
\begin{itemize}
\item {Proveniência:(It. \textunderscore baccheta\textunderscore )}
\end{itemize}
Pequena vara, com que se toca tambor.
Vareta de guarda-sol.
\section{Baqueta}
\begin{itemize}
\item {fónica:quê}
\end{itemize}
\begin{itemize}
\item {Grp. gram.:f.}
\end{itemize}
\begin{itemize}
\item {Utilização:T. de Miranda}
\end{itemize}
Coiro de bezerro ou de vitella, para calçado.
(Por \textunderscore vaqueta\textunderscore , de \textunderscore vaca\textunderscore )
\section{Baquetear}
\begin{itemize}
\item {Grp. gram.:v. i.}
\end{itemize}
Tocar tambor com baquetas.
\section{Báquico}
\begin{itemize}
\item {Grp. gram.:adj.}
\end{itemize}
\begin{itemize}
\item {Proveniência:(Lat. \textunderscore bacchicus\textunderscore )}
\end{itemize}
Relativo a Baco, ou ao vinho.
Em que há orgia: \textunderscore festas báquicas\textunderscore .
\section{Baquio}
\begin{itemize}
\item {Grp. gram.:m.}
\end{itemize}
\begin{itemize}
\item {Proveniência:(Gr. \textunderscore bakkheius\textunderscore )}
\end{itemize}
Pé de verso grego ou latino, com uma sýllaba breve e duas longas.
\section{Baquista}
\begin{itemize}
\item {Grp. gram.:m. ,  f.  e  adj.}
\end{itemize}
\begin{itemize}
\item {Proveniência:(De \textunderscore Baccho\textunderscore , n. p.)}
\end{itemize}
Pessôa dada á embriaguez.
Que gósta de orgias.
\section{Baquité}
\begin{itemize}
\item {Grp. gram.:m.}
\end{itemize}
\begin{itemize}
\item {Utilização:Bras}
\end{itemize}
Samburá, que as índias trazem ás costas.
\section{Barabatanas}
\begin{itemize}
\item {Grp. gram.:m. pl.}
\end{itemize}
Índios selvagens das margens do Apaporis, no Brasil.
\section{Barabu}
\begin{itemize}
\item {Grp. gram.:m.}
\end{itemize}
Árvore dos sertões brasileiros.
\section{Baraça}
\begin{itemize}
\item {Grp. gram.:f.}
\end{itemize}
Correia, com que se liga o linho á roca.
Cordão, cordel.
O mesmo que \textunderscore baraço\textunderscore .
\section{Baracejo}
\begin{itemize}
\item {Grp. gram.:m.}
\end{itemize}
\begin{itemize}
\item {Proveniência:(De \textunderscore baraço\textunderscore )}
\end{itemize}
Espécie de esparto.
\section{Baracha}
\begin{itemize}
\item {Grp. gram.:f.}
\end{itemize}
Travessão de lama, que divide os compartimentos das marinhas.--Alguns diccion. dão-lhe erradamente outra significação.
(Cp. \textunderscore maracha\textunderscore )
\section{Barachar}
\begin{itemize}
\item {Grp. gram.:v. t.}
\end{itemize}
Guarnecer com barachas; separar por barachas.
\section{Baraço}
\begin{itemize}
\item {Grp. gram.:m.}
\end{itemize}
\begin{itemize}
\item {Proveniência:(Do ár. \textunderscore maras\textunderscore )}
\end{itemize}
Corda delgada, cordel.
Laço para estrangular: \textunderscore senhor de baraço e cutello\textunderscore .
\section{Barafula}
\begin{itemize}
\item {Grp. gram.:f.}
\end{itemize}
\begin{itemize}
\item {Utilização:Des.}
\end{itemize}
O mesmo que \textunderscore farófia\textunderscore ? Us. por D. Francisco Manuel.
\section{Barafunda}
\begin{itemize}
\item {Grp. gram.:f.}
\end{itemize}
Ajuntamento desordenado; algazarra, barulho, confusão.
Obra de agulha, com crivos, que imita renda.
(Cast. \textunderscore barahunda\textunderscore )
\section{Barafundo}
\begin{itemize}
\item {Grp. gram.:adj.}
\end{itemize}
\begin{itemize}
\item {Utilização:Des.}
\end{itemize}
Em que há barafunda. Cf. Filinto, VI, 9.
\section{Barafustar}
\begin{itemize}
\item {Grp. gram.:v. i.}
\end{itemize}
Agitar-se desordenadamente.
Relutar.
Argumentar sem ordem.
(Cp. cast. \textunderscore barajustar\textunderscore )
\section{Baraia}
\begin{itemize}
\item {Grp. gram.:f.}
\end{itemize}
\begin{itemize}
\item {Utilização:Bras}
\end{itemize}
Espécie de loiro.
\section{Barajuba}
\begin{itemize}
\item {Grp. gram.:f.}
\end{itemize}
\begin{itemize}
\item {Utilização:Bras}
\end{itemize}
Árvore das regiões do Amazonas, própria para construcções.
\section{Baralha}
\begin{itemize}
\item {Grp. gram.:f.}
\end{itemize}
\begin{itemize}
\item {Proveniência:(De \textunderscore baralhar\textunderscore )}
\end{itemize}
Baralho.
Conjunto de cartas, que sobejam, depois de distribuidas aquellas com que se começa o jôgo.
Barulho, desordem.
Intrigas, mexericos.
\section{Baralhadamente}
\begin{itemize}
\item {Grp. gram.:adv.}
\end{itemize}
\begin{itemize}
\item {Proveniência:(De \textunderscore baralhar\textunderscore )}
\end{itemize}
Confusamente.
\section{Baralhador}
\begin{itemize}
\item {Grp. gram.:m.}
\end{itemize}
Aquelle que baralha.
\section{Baralhar}
\begin{itemize}
\item {Grp. gram.:v. t.}
\end{itemize}
\begin{itemize}
\item {Proveniência:(Do b. lat. \textunderscore baraliare\textunderscore )}
\end{itemize}
Misturar; confundir; desordenar.
\section{Baralhau}
\begin{itemize}
\item {Grp. gram.:m.}
\end{itemize}
\begin{itemize}
\item {Utilização:Prov.}
\end{itemize}
\begin{itemize}
\item {Utilização:Fam.}
\end{itemize}
Indivíduo atabalhoado.
(Colhido em Turquel)
\section{Baralho}
\begin{itemize}
\item {Grp. gram.:m.}
\end{itemize}
\begin{itemize}
\item {Proveniência:(De \textunderscore baralha\textunderscore )}
\end{itemize}
Collecção de cartas, necessárias para um jôgo.
\section{Barambaz}
\begin{itemize}
\item {Grp. gram.:m.}
\end{itemize}
\begin{itemize}
\item {Utilização:Des.}
\end{itemize}
Objecto pendente, como sanefa, etc.
Certa guarnição de vestidos.
\section{Baranda}
\begin{itemize}
\item {Grp. gram.:f.}
\end{itemize}
(Fórma pop. de \textunderscore varanda\textunderscore )
\section{Barangandan}
\begin{itemize}
\item {Grp. gram.:m.}
\end{itemize}
\begin{itemize}
\item {Utilização:Bras}
\end{itemize}
Aderêço de prata, usado pelas crioilas na cintura, em dias de festa.
\section{Baranha}
\begin{itemize}
\item {Grp. gram.:f.}
\end{itemize}
(Corr. de \textunderscore maranha\textunderscore )
\section{Baranho}
\begin{itemize}
\item {Grp. gram.:m.}
\end{itemize}
\begin{itemize}
\item {Utilização:Prov.}
\end{itemize}
\begin{itemize}
\item {Utilização:trasm.}
\end{itemize}
Cordão, formado pela erva, que se ceifa á gadanha nos lameiros.
(Cp. \textunderscore baranha\textunderscore )
\section{Barão}
\begin{itemize}
\item {Grp. gram.:m.}
\end{itemize}
\begin{itemize}
\item {Utilização:Ant.}
\end{itemize}
\begin{itemize}
\item {Proveniência:(Lat. \textunderscore baro\textunderscore )}
\end{itemize}
Título de nobreza, immediatamente inferior ao de visconde.
Homem illustre:«\textunderscore as armas e os barões assinalados\textunderscore ». \textunderscore Lusíadas\textunderscore , I, 1.
Senhor feudal.
\section{Baraqueta}
\begin{itemize}
\item {fónica:quê}
\end{itemize}
\begin{itemize}
\item {Grp. gram.:m.}
\end{itemize}
\begin{itemize}
\item {Utilização:T. das Caldas da Raínha}
\end{itemize}
Homem, que gosta muito de vinho, mas que se não embriaga.
\section{Bararua}
\begin{itemize}
\item {Grp. gram.:f.}
\end{itemize}
Peixe do Brasil.
\section{Barata}
\begin{itemize}
\item {Grp. gram.:f.}
\end{itemize}
\begin{itemize}
\item {Proveniência:(Lat. \textunderscore blatta\textunderscore )}
\end{itemize}
Insecto ortóptero, nocturno e caseiro.
\section{Barata}
\begin{itemize}
\item {Grp. gram.:f.}
\end{itemize}
\begin{itemize}
\item {Proveniência:(Fr. \textunderscore baratte\textunderscore )}
\end{itemize}
Batedeira.
Balde ou engenho, em que se bate o leite para fazer manteiga.
\section{Barata}
\begin{itemize}
\item {Grp. gram.:f.}
\end{itemize}
\begin{itemize}
\item {Utilização:Ant.}
\end{itemize}
\begin{itemize}
\item {Proveniência:(De \textunderscore baratar\textunderscore ^1)}
\end{itemize}
Título de dívida.
Permutação.
Contrato.
\section{Baratamente}
\begin{itemize}
\item {Grp. gram.:adv.}
\end{itemize}
Com barateza.
\section{Baratar}
\begin{itemize}
\item {Grp. gram.:v. t.}
\end{itemize}
(V.baratear)
\section{Baratar}
\begin{itemize}
\item {Grp. gram.:v. t.}
\end{itemize}
\begin{itemize}
\item {Utilização:Ant.}
\end{itemize}
Destruir.
(Cp. \textunderscore barata\textunderscore ^2)
\section{Barataria}
\begin{itemize}
\item {Grp. gram.:f.}
\end{itemize}
\begin{itemize}
\item {Proveniência:(De \textunderscore baratar\textunderscore ^1)}
\end{itemize}
Acto de dar, com a mira na retribuição.
Permutação.
Troca fraudulenta de mercadorias, a bordo.
\section{Barateamento}
\begin{itemize}
\item {Grp. gram.:m.}
\end{itemize}
Acto de \textunderscore baratear\textunderscore .
\section{Baratear}
\begin{itemize}
\item {Grp. gram.:v. t.}
\end{itemize}
Tornar barato; vender por preço baixo.
Regatear sôbre o preço de.
\section{Barateio}
\begin{itemize}
\item {Grp. gram.:m.}
\end{itemize}
O mesmo que \textunderscore barateamento\textunderscore .
\section{Barateira}
\begin{itemize}
\item {Grp. gram.:f.}
\end{itemize}
Armadilha, para apanhar baratas.
\section{Barateiro}
\begin{itemize}
\item {Grp. gram.:m.  e  adj.}
\end{itemize}
O que vende barato.
Que quere comprar barato.
Que cobra os baratos á mesa do jôgo.
\section{Barateza}
\begin{itemize}
\item {Grp. gram.:f.}
\end{itemize}
Modicidade de preço.
Qualidade, do que se vende barato.
\section{Bárathro}
\begin{itemize}
\item {Grp. gram.:m.}
\end{itemize}
\begin{itemize}
\item {Proveniência:(Lat. \textunderscore barathrum\textunderscore )}
\end{itemize}
Precipício; abismo.
Inferno.
\section{Barato}
\begin{itemize}
\item {Grp. gram.:m.}
\end{itemize}
\begin{itemize}
\item {Grp. gram.:Adj.}
\end{itemize}
\begin{itemize}
\item {Utilização:Fig.}
\end{itemize}
\begin{itemize}
\item {Grp. gram.:Adv.}
\end{itemize}
Percentagem, deduzida dos ganhos do jôgo, e paga ao dono da tavolagem.
Favor, benefício, concessão.
Facilidade, bom grado: \textunderscore demos de barato que assim seja\textunderscore .
\textunderscore Fazer bom barato de\textunderscore , vender por baixo preço. Cf. \textunderscore Peregrinação\textunderscore ,
XXXV.
Que se vende por baixo preço, que custa pouco dinheiro: \textunderscore um chapéu barato\textunderscore .
Fácil de conseguir ou realizar.
Com barateza, por pouco preço: \textunderscore aquella fruta compra-se barato\textunderscore .
\section{Báratro}
\begin{itemize}
\item {Grp. gram.:m.}
\end{itemize}
\begin{itemize}
\item {Proveniência:(Lat. \textunderscore barathrum\textunderscore )}
\end{itemize}
Precipício; abismo.
Inferno.
\section{Baraúna}
\begin{itemize}
\item {Grp. gram.:f.}
\end{itemize}
\begin{itemize}
\item {Utilização:Bras}
\end{itemize}
Leguminosa, cesalpínea, própria para construcções.
\section{Barba}
\begin{itemize}
\item {Grp. gram.:f.}
\end{itemize}
\begin{itemize}
\item {Grp. gram.:Pl.}
\end{itemize}
\begin{itemize}
\item {Proveniência:(Lat. \textunderscore barba\textunderscore )}
\end{itemize}
Cabellos da parte inferior e das lateraes da cara do homem.
Pêlos no focinho ou no bico de alguns animaes.
O mesmo que \textunderscore queixo\textunderscore ^1.
Pragana de espiga.
A parte inferior do beque de uma embarcação.
Primeira parte das locuções, com que se designam algumas plantas.
Os cabellos do rosto.
Pêlos compridos no focinho de alguns animaes.
Lâminas córneas e flexíveis, extrahidas da bôca da baleia.
Raiz, formada por fascículos de fíbras delgadas.
Conjunto de arestas em certas plantas.
Filamentos lateraes de uma penna.
Feixe de fios ou pêlos.
Arestas filiformes de uma superfície ou das bordas de um objecto mal cortado ou mal aparado.
\section{Barba-a-barba}
\begin{itemize}
\item {Grp. gram.:loc. adv.}
\end{itemize}
Na presença; de cara a cara.
\section{Barba-azul}
\begin{itemize}
\item {Grp. gram.:m.}
\end{itemize}
Pássaro brasileiro das regiões do Amazonas.
\section{Barbacã}
\begin{itemize}
\item {Grp. gram.:f.}
\end{itemize}
Muro, que se construía por fóra das muralhas, e mais baixo que ellas.
(Cast. \textunderscore barbacana\textunderscore )
\section{Barbacan}
\begin{itemize}
\item {Grp. gram.:f.}
\end{itemize}
Muro, que se construía por fóra das muralhas, e mais baixo que ellas.
(Cast. \textunderscore barbacana\textunderscore )
\section{Barbaçana}
\begin{itemize}
\item {Grp. gram.:m.}
\end{itemize}
O mesmo que \textunderscore barbaças\textunderscore .
\section{Barbacão}
\begin{itemize}
\item {Grp. gram.:m.}
\end{itemize}
\begin{itemize}
\item {Utilização:Prov.}
\end{itemize}
\begin{itemize}
\item {Utilização:trasm.}
\end{itemize}
Pedaço de terra sáfara, numa chan, distante do povoado.
\section{Barbaças}
\begin{itemize}
\item {Grp. gram.:m.}
\end{itemize}
\begin{itemize}
\item {Utilização:Fam.}
\end{itemize}
\begin{itemize}
\item {Utilização:Cyn.}
\end{itemize}
Aquelle que tem grandes barbas.
Ancião respeitável.
Variedade de podengo, muito apreciado para a caça de lebres e coêlhos.
\section{Barbacuá}
\begin{itemize}
\item {Grp. gram.:m.}
\end{itemize}
\begin{itemize}
\item {Proveniência:(Do guar. \textunderscore mbara\textunderscore , pau + \textunderscore mbacua\textunderscore , coisa assada)}
\end{itemize}
Antiga designação brasileira do girau ou armação, que hoje se chama \textunderscore carijo\textunderscore .
\section{Barbaçudo}
\begin{itemize}
\item {Grp. gram.:adj.}
\end{itemize}
\begin{itemize}
\item {Proveniência:(De \textunderscore barbaças\textunderscore )}
\end{itemize}
Que tem muita barba.
\section{Barbada}
\begin{itemize}
\item {Grp. gram.:f.}
\end{itemize}
\begin{itemize}
\item {Utilização:Prov.}
\end{itemize}
\begin{itemize}
\item {Utilização:minh.}
\end{itemize}
\begin{itemize}
\item {Proveniência:(De \textunderscore barba\textunderscore )}
\end{itemize}
Beiço inferior do cavallo.
Videira, o mesmo que \textunderscore barbado\textunderscore .
\section{Barbadão}
\begin{itemize}
\item {Grp. gram.:m.}
\end{itemize}
\begin{itemize}
\item {Utilização:ant.}
\end{itemize}
\begin{itemize}
\item {Utilização:Fam.}
\end{itemize}
\begin{itemize}
\item {Proveniência:(De \textunderscore barbado\textunderscore )}
\end{itemize}
Homem de grandes barbas, mas de pouco ânimo ou de pouco préstimo.
\section{Barba-de-bode}
\begin{itemize}
\item {Grp. gram.:f.}
\end{itemize}
O mesmo que \textunderscore cercefi\textunderscore .
\section{Barba-de-cabra}
\begin{itemize}
\item {Grp. gram.:f.}
\end{itemize}
O mesmo que \textunderscore barba-de-bode\textunderscore . Cf. \textunderscore Desengano da Med.\textunderscore , 268.
\section{Barba-de-san-pedro}
\begin{itemize}
\item {Grp. gram.:f.}
\end{itemize}
\begin{itemize}
\item {Utilização:Bras}
\end{itemize}
O mesmo que \textunderscore bôlsa-do-pastor\textunderscore .
\section{Barba-de-velho}
\begin{itemize}
\item {Grp. gram.:f.}
\end{itemize}
\begin{itemize}
\item {Utilização:Bras}
\end{itemize}
Planta parasita, (\textunderscore nigella arvensis\textunderscore , Lin.), com cujos filamentos se enchem almofadas, colchões, etc.
\section{Barbadinho}
\begin{itemize}
\item {Grp. gram.:m.}
\end{itemize}
\begin{itemize}
\item {Proveniência:(De \textunderscore barbado\textunderscore )}
\end{itemize}
Frade franciscano, dos que usavam barba comprida.
\section{Barbado}
\begin{itemize}
\item {Grp. gram.:adj.}
\end{itemize}
\begin{itemize}
\item {Grp. gram.:M.}
\end{itemize}
\begin{itemize}
\item {Proveniência:(Lat. \textunderscore barbatus\textunderscore )}
\end{itemize}
Que tem barba.
Videira tenra, com raiz, para plantações.
\section{Barba-jóvis}
\begin{itemize}
\item {Grp. gram.:f.}
\end{itemize}
Designação antiga de uma planta medicinal. Cf. \textunderscore Regim. contra a Pestenença\textunderscore , (séc. XVI).
\section{Barbal}
\begin{itemize}
\item {Grp. gram.:m.}
\end{itemize}
\begin{itemize}
\item {Proveniência:(De \textunderscore barbo\textunderscore )}
\end{itemize}
Espécie de rede, empregada no Doiro, para a pesca do barbo e de outros peixes.
\section{Barbalho}
\begin{itemize}
\item {Grp. gram.:m.}
\end{itemize}
\begin{itemize}
\item {Proveniência:(De \textunderscore barba\textunderscore )}
\end{itemize}
Raiz filamentosa das plantas.
\section{Barbalhoste}
\begin{itemize}
\item {Grp. gram.:adj.}
\end{itemize}
\begin{itemize}
\item {Utilização:Ant.}
\end{itemize}
\begin{itemize}
\item {Proveniência:(De \textunderscore barbalho\textunderscore ?)}
\end{itemize}
Que não tem préstimo.
\section{Barbante}
\begin{itemize}
\item {Grp. gram.:m.}
\end{itemize}
Cordel, guita.
(Cast. \textunderscore bramante\textunderscore )
\section{Barbaquá}
\begin{itemize}
\item {Grp. gram.:m.}
\end{itemize}
\begin{itemize}
\item {Utilização:Bras}
\end{itemize}
\begin{itemize}
\item {Utilização:des.}
\end{itemize}
Caniçado, em que se fazia a sapeça, para preparação do mate.
\section{Barbaquim}
\begin{itemize}
\item {Grp. gram.:m.}
\end{itemize}
\begin{itemize}
\item {Utilização:Prov.}
\end{itemize}
\begin{itemize}
\item {Utilização:beir.}
\end{itemize}
O mesmo que \textunderscore berbequim\textunderscore .
Arco de pua.
\section{Barbar}
\begin{itemize}
\item {Grp. gram.:v. i.}
\end{itemize}
\begin{itemize}
\item {Utilização:T. de apicultura}
\end{itemize}
Começar a têr barba.
O mesmo que \textunderscore abarbar\textunderscore .
\section{Bárbara}
\begin{itemize}
\item {Grp. gram.:f.}
\end{itemize}
\begin{itemize}
\item {Proveniência:(De \textunderscore bárbaro\textunderscore )}
\end{itemize}
Espécie de syllogismo, na lógica dos escolásticos.
\section{Barbaramente}
\begin{itemize}
\item {Grp. gram.:adv.}
\end{itemize}
De modo \textunderscore bárbaro\textunderscore .
Cruelmente.
\section{Barbaresco}
\begin{itemize}
\item {Grp. gram.:adj.}
\end{itemize}
(V.berberesco)
\section{Barbarez}
\begin{itemize}
\item {Grp. gram.:f.}
\end{itemize}
\begin{itemize}
\item {Utilização:Des.}
\end{itemize}
Qualidade de bárbaro; barbaridade. Cf. Filinto, VI, 219.
\section{Barbaria}
\begin{itemize}
\item {Proveniência:(De \textunderscore bárbaro\textunderscore )}
\end{itemize}
Acção própria de bárbaros.
Falta de civilização.
Crueldade.
Multidão de bárbaros.
\section{Barbaria}
\begin{itemize}
\item {Grp. gram.:f.}
\end{itemize}
\begin{itemize}
\item {Proveniência:(De \textunderscore barba\textunderscore )}
\end{itemize}
(Fórma preferível a \textunderscore barbearia\textunderscore )
\section{Barbárico}
\begin{itemize}
\item {Grp. gram.:adj.}
\end{itemize}
\begin{itemize}
\item {Proveniência:(Lat. \textunderscore barbaricus\textunderscore )}
\end{itemize}
Relativo a bárbaros.
\section{Barbaridade}
\begin{itemize}
\item {Grp. gram.:f.}
\end{itemize}
Acção própria de bárbaros; crueldade.
\section{Barbárie}
\begin{itemize}
\item {Grp. gram.:f.}
\end{itemize}
\begin{itemize}
\item {Proveniência:(Lat. \textunderscore barbaries\textunderscore )}
\end{itemize}
(V. \textunderscore barbaria\textunderscore ^1)
\section{Barbários}
\begin{itemize}
\item {Grp. gram.:m. pl.}
\end{itemize}
\begin{itemize}
\item {Proveniência:(De \textunderscore Barbário\textunderscore , n. p. ant. do Cabo Espichel)}
\end{itemize}
Povos, que habitaram o litoral, entre o Sado e o Tejo.
\section{Barbiloquia}
\begin{itemize}
\item {Grp. gram.:f.}
\end{itemize}
\begin{itemize}
\item {Proveniência:(Do lat. \textunderscore barbarus\textunderscore  + \textunderscore loqui\textunderscore )}
\end{itemize}
Emprêgo de linguagem bárbara.
\section{Barbarisco}
\begin{itemize}
\item {Grp. gram.:adj.}
\end{itemize}
\begin{itemize}
\item {Grp. gram.:M.}
\end{itemize}
O mesmo que \textunderscore barbaresco\textunderscore .
Espécie de tecido antigo.
\section{Barbarismo}
\begin{itemize}
\item {Grp. gram.:m.}
\end{itemize}
\begin{itemize}
\item {Proveniência:(Lat. \textunderscore barbarismus\textunderscore )}
\end{itemize}
Uso de palavras estrangeiras, como se fôssem nacionaes; estrangeirismo.
Êrro, contra a verdadeira significação das palavras.
Errada composição e derivação de vocábulos.
Incorrecção de pronúncia ou de escrita.
Êrro contra a syntaxe; solecismo.
Crueldade.
Condição dos povos rudes.
\section{Barbarisonante}
\begin{itemize}
\item {fónica:so}
\end{itemize}
\begin{itemize}
\item {Grp. gram.:adj.}
\end{itemize}
\begin{itemize}
\item {Proveniência:(De \textunderscore bárbaro\textunderscore  + \textunderscore sonante\textunderscore )}
\end{itemize}
Que se assemelha á pronunciação bárbara.
Que sôa a barbarismo.
\section{Barbarissonante}
\begin{itemize}
\item {Grp. gram.:adj.}
\end{itemize}
\begin{itemize}
\item {Proveniência:(De \textunderscore bárbaro\textunderscore  + \textunderscore sonante\textunderscore )}
\end{itemize}
Que se assemelha á pronunciação bárbara.
Que sôa a barbarismo.
\section{Barbarizar}
\begin{itemize}
\item {Grp. gram.:v. t.}
\end{itemize}
\begin{itemize}
\item {Grp. gram.:V. i.}
\end{itemize}
Tornar bárbaro.
Commeter barbarismos.
\section{Barbarizo}
\begin{itemize}
\item {Grp. gram.:m.}
\end{itemize}
\begin{itemize}
\item {Utilização:Ant.}
\end{itemize}
O mesmo que \textunderscore borborinho\textunderscore .
\section{Bárbaro}
\begin{itemize}
\item {Grp. gram.:adj.}
\end{itemize}
\begin{itemize}
\item {Proveniência:(Lat. \textunderscore barbarus\textunderscore )}
\end{itemize}
Que não tem civilização.
Rude, selvagem.
Cruel.
Incorrecto: \textunderscore expressão bárbara\textunderscore .
\section{Bárbaros}
\begin{itemize}
\item {Grp. gram.:m. pl.}
\end{itemize}
\begin{itemize}
\item {Proveniência:(Lat. \textunderscore barbari\textunderscore )}
\end{itemize}
Povos do Norte, que invadiram o Império Romano do Occidente.
Os estrangeiros, em relação aos Gregos e Romanos.
\section{Barbarrão}
\begin{itemize}
\item {Grp. gram.:m.}
\end{itemize}
(V.barbaças)
\section{Barbasco}
\begin{itemize}
\item {Grp. gram.:m.}
\end{itemize}
(V.verbasco)
\section{Barbas-de-capuchinho}
\begin{itemize}
\item {Grp. gram.:f. pl.}
\end{itemize}
O mesmo que \textunderscore chicória\textunderscore ^1.
\section{Barbata}
\begin{itemize}
\item {Grp. gram.:f.}
\end{itemize}
\begin{itemize}
\item {Utilização:Des.}
\end{itemize}
O mesmo que \textunderscore bravata\textunderscore . Cf. Vieira, V, 447; VIII, 205.
\section{Barbata}
\begin{itemize}
\item {Grp. gram.:f.}
\end{itemize}
\begin{itemize}
\item {Proveniência:(De \textunderscore barba\textunderscore )}
\end{itemize}
Assento do freio, na parte da boca do cavallo, em que não há dentes.
\section{Barbatana}
\begin{itemize}
\item {Grp. gram.:f.}
\end{itemize}
\begin{itemize}
\item {Proveniência:(De \textunderscore barba\textunderscore )}
\end{itemize}
Cada um dos órgãos exteriores, que servem para os peixes se moverem.
\section{Barbatão}
\begin{itemize}
\item {Grp. gram.:m.}
\end{itemize}
\begin{itemize}
\item {Utilização:Bras}
\end{itemize}
\begin{itemize}
\item {Utilização:Bras. do N}
\end{itemize}
Gado bovino que, criando-se nos matos, se tornou bravio.
Bezerro, já crescido.
(Por \textunderscore brabatão\textunderscore , de \textunderscore brabo\textunderscore  = \textunderscore bravo\textunderscore )
\section{Barbate}
\begin{itemize}
\item {Grp. gram.:m.}
\end{itemize}
\begin{itemize}
\item {Utilização:Constr.}
\end{itemize}
Córte ou bôca, em que se ajusta o frechal, na extremidade inferior dos guieiros do madeiramento.
\section{Barbatear}
\begin{itemize}
\item {Grp. gram.:v. t.}
\end{itemize}
\begin{itemize}
\item {Utilização:Des.}
\end{itemize}
\begin{itemize}
\item {Proveniência:(De \textunderscore barbata\textunderscore ^1)}
\end{itemize}
Dizer com arrogância, com prosápia:«\textunderscore ...tinha roncado e barbateado Pedro que... só elle havia de ser constante\textunderscore ». Vieira, II, 333.
\section{Barbatimão}
\begin{itemize}
\item {Grp. gram.:m.}
\end{itemize}
Árvore leguminosa do Brasil, (\textunderscore acacia virginalis\textunderscore ).
\section{Barbato}
\begin{itemize}
\item {Grp. gram.:m.}
\end{itemize}
\begin{itemize}
\item {Proveniência:(Lat. \textunderscore barbatus\textunderscore )}
\end{itemize}
Leigo, que, em certos institutos religiosos, usava barba comprida.
\section{Barbatolas}
\begin{itemize}
\item {Grp. gram.:m.}
\end{itemize}
O mesmo que \textunderscore barbaças\textunderscore . Cf. J. Dinis, \textunderscore Morgadinha\textunderscore , 233.
\section{Barbeação}
\begin{itemize}
\item {Grp. gram.:f.}
\end{itemize}
Acção de \textunderscore barbear\textunderscore .
\section{Barbeadura}
\begin{itemize}
\item {Grp. gram.:f.}
\end{itemize}
O mesmo que \textunderscore barbeação\textunderscore .
\section{Barbear}
\begin{itemize}
\item {Grp. gram.:v. t.}
\end{itemize}
\begin{itemize}
\item {Utilização:Ant.}
\end{itemize}
Rapar a barba a.
Amarrar.
\section{Barbearia}
\begin{itemize}
\item {Grp. gram.:f.}
\end{itemize}
\begin{itemize}
\item {Utilização:T. de Lisbôa}
\end{itemize}
Casa, em que os frades faziam a barba, dentro dos conventos.
Profissão de barbeiro.
Loja de barbeiro.
\section{Barbechar}
\begin{itemize}
\item {Grp. gram.:v. t.}
\end{itemize}
Preparar com o barbecho (uma terra).
\section{Barbecho}
\begin{itemize}
\item {Grp. gram.:m.}
\end{itemize}
O mesmo que \textunderscore barbeito\textunderscore .
(Cast. \textunderscore barbecho\textunderscore )
\section{Barbeirão}
\begin{itemize}
\item {Grp. gram.:m.}
\end{itemize}
\begin{itemize}
\item {Utilização:ant.}
\end{itemize}
\begin{itemize}
\item {Utilização:Fam.}
\end{itemize}
Barba grande.
\section{Barbeiro}
\begin{itemize}
\item {Grp. gram.:m.}
\end{itemize}
\begin{itemize}
\item {Utilização:Fig.}
\end{itemize}
\begin{itemize}
\item {Grp. gram.:Loc.}
\end{itemize}
\begin{itemize}
\item {Utilização:ant.}
\end{itemize}
\begin{itemize}
\item {Utilização:Prov.}
\end{itemize}
Aquelle que tem o offício de rapar ou aparar a barba.
Vento frio, que passa pela cara.
Espécie de jôgo popular.
Peixe marítimo e ordinário, no Brasil.
\textunderscore Barbeiro das espadas\textunderscore , official que se empregava em açacalar e guarnecer espadas e outras armas brancas.
O mesmo que \textunderscore curandeiro\textunderscore .
\section{Barbeirola}
\begin{itemize}
\item {Grp. gram.:m.}
\end{itemize}
\begin{itemize}
\item {Utilização:Deprec.}
\end{itemize}
Barbeiro reles.
\section{Barbeito}
\begin{itemize}
\item {Grp. gram.:m.}
\end{itemize}
\begin{itemize}
\item {Utilização:Prov.}
\end{itemize}
\begin{itemize}
\item {Utilização:minh.}
\end{itemize}
\begin{itemize}
\item {Utilização:Ant.}
\end{itemize}
\begin{itemize}
\item {Proveniência:(Do lat. \textunderscore vervactum\textunderscore )}
\end{itemize}
Primeira lavra de um terreno, para o deixar de alqueive.
Terreno, que produz apenas pastagens fracas.
Valle.
Cômoro, que divide uma propriedade de outra e a resguarda.
\section{Barbela}
\begin{itemize}
\item {Grp. gram.:f.}
\end{itemize}
\begin{itemize}
\item {Grp. gram.:Adj.}
\end{itemize}
\begin{itemize}
\item {Proveniência:(De \textunderscore barba\textunderscore )}
\end{itemize}
Pelle pendente do pescoço do boi.
Saliência adiposa, por baixo do queixo.
Cadeia de ferro, que guarnece inferiormente a barbada do cavallo.
Extremidade farpada da agulha de meia ou de croché.
Diz-se de uma variedade de trigo.
\section{Barbelido}
\begin{itemize}
\item {Grp. gram.:m.}
\end{itemize}
\begin{itemize}
\item {Utilização:Prov.}
\end{itemize}
\begin{itemize}
\item {Utilização:minh.}
\end{itemize}
Agitação á superfície do mar, produzida pela sardinha em cardume.
\section{Barbelões}
\begin{itemize}
\item {Grp. gram.:m. pl.}
\end{itemize}
\begin{itemize}
\item {Proveniência:(Do fr. \textunderscore barbillon\textunderscore )}
\end{itemize}
Dobras da membrana mucosa, debaixo da língua do cavallo ou boi.
Cp. \textunderscore barbilhão\textunderscore .
\section{Barbélula}
\begin{itemize}
\item {Grp. gram.:f.}
\end{itemize}
\begin{itemize}
\item {Utilização:Bot.}
\end{itemize}
\begin{itemize}
\item {Proveniência:(De \textunderscore barbela\textunderscore )}
\end{itemize}
Appêndice do pappilho das synanthéreas, quando é curto, cónico e pontuado.
\section{Barbelulado}
\begin{itemize}
\item {Grp. gram.:adj.}
\end{itemize}
Que tem barbélulas.
\section{Barbeta}
\begin{itemize}
\item {fónica:bê}
\end{itemize}
\begin{itemize}
\item {Grp. gram.:f.}
\end{itemize}
O mesmo que \textunderscore barbete\textunderscore .
\section{Barbete}
\begin{itemize}
\item {fónica:bê}
\end{itemize}
\begin{itemize}
\item {Grp. gram.:m.}
\end{itemize}
\begin{itemize}
\item {Proveniência:(Fr. \textunderscore barbette\textunderscore )}
\end{itemize}
Plataforma, donde a artilharia dispara por cima do parapeito.
\section{Barbialçado}
\begin{itemize}
\item {Grp. gram.:adj.}
\end{itemize}
\begin{itemize}
\item {Utilização:Des.}
\end{itemize}
\begin{itemize}
\item {Proveniência:(De \textunderscore barba\textunderscore  + \textunderscore alçado\textunderscore )}
\end{itemize}
De fronte erguida; de barba alta.
\section{Barbião}
\begin{itemize}
\item {Grp. gram.:m.}
\end{itemize}
\begin{itemize}
\item {Utilização:Prov.}
\end{itemize}
\begin{itemize}
\item {Utilização:trasm.}
\end{itemize}
Cada um dos madeiros, anterior e posterior, que limitam o tabuleiro do carro de bois.
\section{Barbiargênteo}
\begin{itemize}
\item {Grp. gram.:adj.}
\end{itemize}
\begin{itemize}
\item {Proveniência:(De \textunderscore barba\textunderscore  + \textunderscore argênteo\textunderscore )}
\end{itemize}
Que tem barba branca.
\section{Barbicacho}
\begin{itemize}
\item {Grp. gram.:m.}
\end{itemize}
\begin{itemize}
\item {Utilização:Bras. do S}
\end{itemize}
\begin{itemize}
\item {Proveniência:(T. cast.)}
\end{itemize}
Cabresto; cabeçada de corda.
Obstáculo, embaraço: \textunderscore é negócio que tem barbicacho\textunderscore .
Cordão entrançado, cujas pontas, cosidas ao chapéu, o seguram, passando por baixo da barba.
\section{Barbicas}
\begin{itemize}
\item {Grp. gram.:m.}
\end{itemize}
\begin{itemize}
\item {Utilização:Fam.}
\end{itemize}
Homem de pouca barba e de fraca figura.
\section{Barbicha}
\begin{itemize}
\item {Grp. gram.:f.}
\end{itemize}
Pequena barba.
\section{Barbichas}
\begin{itemize}
\item {Grp. gram.:m.}
\end{itemize}
\begin{itemize}
\item {Grp. gram.:F. pl.}
\end{itemize}
O mesmo que \textunderscore barbicas\textunderscore .
Barbas raras e pouco cuidadas.
\section{Barbífero}
\begin{itemize}
\item {Grp. gram.:adj.}
\end{itemize}
\begin{itemize}
\item {Proveniência:(Do lat. \textunderscore barba\textunderscore  + \textunderscore ferre\textunderscore )}
\end{itemize}
Que tem barba.
\section{Barbiforme}
\begin{itemize}
\item {Grp. gram.:adj.}
\end{itemize}
\begin{itemize}
\item {Proveniência:(Do lat. \textunderscore barba\textunderscore  + \textunderscore forma\textunderscore )}
\end{itemize}
Que tem fórma de barba.
\section{Barbilhão}
\begin{itemize}
\item {Grp. gram.:m.}
\end{itemize}
\begin{itemize}
\item {Proveniência:(Fr. \textunderscore barbillon\textunderscore )}
\end{itemize}
Filamento, ao canto da bôca de alguns peixes.
Saliência carnosa, por baixo do bico de algumas aves.
Excrescência, na pelle interior da boca dos bovídeos.
\section{Barbilho}
\begin{itemize}
\item {Grp. gram.:m.}
\end{itemize}
\begin{itemize}
\item {Utilização:Prov.}
\end{itemize}
\begin{itemize}
\item {Utilização:alent.}
\end{itemize}
\begin{itemize}
\item {Utilização:T. da Guarda}
\end{itemize}
\begin{itemize}
\item {Proveniência:(De \textunderscore barba\textunderscore )}
\end{itemize}
Espécie de saco de esparto, com que se envolve o focinho de alguns animaes, para não mamarem ou não damnificarem plantações ou searas.
Cordão de anafaia.
Cadilho.
Embaraço, obstáculo.
Pedaço de pau, que se mete transversalmente na bôca dos chibos e se prende com dois cordéis atrás das orelhas, para impedir a sucção e portanto a mama.
Correia, que liga os canzis, por baixo do pescoço dos bois.
\section{Barbiloiro}
\begin{itemize}
\item {Grp. gram.:adj.}
\end{itemize}
Que tem barba loira.
\section{Barbilongo}
\begin{itemize}
\item {Grp. gram.:adj.}
\end{itemize}
\begin{itemize}
\item {Proveniência:(De \textunderscore barba\textunderscore  + \textunderscore longo\textunderscore )}
\end{itemize}
Que tem barbas compridas.
\section{Barbilouro}
\begin{itemize}
\item {Grp. gram.:adj.}
\end{itemize}
Que tem barba loira.
\section{Barbinegro}
\begin{itemize}
\item {fónica:nê}
\end{itemize}
\begin{itemize}
\item {Grp. gram.:adj.}
\end{itemize}
Que tem barba negra.
\section{Barbinos}
\begin{itemize}
\item {Grp. gram.:m.}
\end{itemize}
Planta parasita do Brasil.
\section{Barbipoente}
\begin{itemize}
\item {Grp. gram.:adj.}
\end{itemize}
O mesmo que \textunderscore barbiponente\textunderscore .
\section{Barbiponente}
\begin{itemize}
\item {Grp. gram.:adj.}
\end{itemize}
Diz-se de um indivíduo, a quem a barba começa a aparecer:«\textunderscore mancebo barbiponente\textunderscore ». \textunderscore Aulegrafia\textunderscore , 89. Cf. Castilho, \textunderscore Avarento\textunderscore , 147.
\section{Barbirostro}
\begin{itemize}
\item {fónica:rós}
\end{itemize}
\begin{itemize}
\item {Grp. gram.:adj.}
\end{itemize}
\begin{itemize}
\item {Utilização:Zool.}
\end{itemize}
\begin{itemize}
\item {Proveniência:(Do lat. \textunderscore barba\textunderscore  + \textunderscore rostrum\textunderscore )}
\end{itemize}
Que tem pêlos no bico.
\section{Barbirrostro}
\begin{itemize}
\item {Grp. gram.:adj.}
\end{itemize}
\begin{itemize}
\item {Utilização:Zool.}
\end{itemize}
\begin{itemize}
\item {Proveniência:(Do lat. \textunderscore barba\textunderscore  + \textunderscore rostrum\textunderscore )}
\end{itemize}
Que tem pêlos no bico.
\section{Barbirruivo}
\begin{itemize}
\item {Grp. gram.:adj.}
\end{itemize}
\begin{itemize}
\item {Utilização:Zool.}
\end{itemize}
\begin{itemize}
\item {Proveniência:(De \textunderscore barba\textunderscore  + \textunderscore ruivo\textunderscore )}
\end{itemize}
Que tem pennas ruivas.
\section{Barbiruivo}
\begin{itemize}
\item {fónica:rui}
\end{itemize}
\begin{itemize}
\item {Grp. gram.:adj.}
\end{itemize}
\begin{itemize}
\item {Utilização:Zool.}
\end{itemize}
\begin{itemize}
\item {Proveniência:(De \textunderscore barba\textunderscore  + \textunderscore ruivo\textunderscore )}
\end{itemize}
Que tem pennas ruivas.
\section{Barbiteso}
\begin{itemize}
\item {fónica:tê}
\end{itemize}
\begin{itemize}
\item {Grp. gram.:adj.}
\end{itemize}
Que tem barba tesa.
Pertinaz.
Corajoso.
\section{Bárbito}
\begin{itemize}
\item {Grp. gram.:m.}
\end{itemize}
\begin{itemize}
\item {Proveniência:(Lat. \textunderscore barbitus\textunderscore )}
\end{itemize}
Lyra de nove cordas, entre os Gregos.
\section{Barbo}
\begin{itemize}
\item {Grp. gram.:m.}
\end{itemize}
\begin{itemize}
\item {Proveniência:(Lat. \textunderscore barbus\textunderscore )}
\end{itemize}
Peixe de água doce.
\section{Barboar}
\begin{itemize}
\item {Grp. gram.:v. i.}
\end{itemize}
\begin{itemize}
\item {Utilização:ant.}
\end{itemize}
\begin{itemize}
\item {Utilização:Fam.}
\end{itemize}
Têr barbas.
\section{Barbosa}
\begin{itemize}
\item {Grp. gram.:f.}
\end{itemize}
Variedade de pêra, também conhecida por \textunderscore grande-alexandre\textunderscore .
\section{Barbosinho}
\begin{itemize}
\item {Grp. gram.:m.}
\end{itemize}
\begin{itemize}
\item {Proveniência:(De \textunderscore barba\textunderscore )}
\end{itemize}
Tumor na língua das aves de rapina.
Excrecência mórbida na boca dos cavallos.
Barbilhão dos peixes.
\section{Barbote}
\begin{itemize}
\item {Grp. gram.:m.}
\end{itemize}
\begin{itemize}
\item {Proveniência:(De \textunderscore barba\textunderscore )}
\end{itemize}
Peça que, nas antigas armaduras, encobria a barba do guerreiro.
Nó, resultante da emenda dos fios do tear.
\section{Barbotina}
\begin{itemize}
\item {Grp. gram.:f.}
\end{itemize}
\begin{itemize}
\item {Proveniência:(Fr. \textunderscore barbotine\textunderscore )}
\end{itemize}
Flôres, não desabrochadas, de várias espécies de artemísia.
Semente de absintho.
\section{Barboto}
\begin{itemize}
\item {fónica:bô}
\end{itemize}
\begin{itemize}
\item {Grp. gram.:m.}
\end{itemize}
Espécie de barbo.
\section{Barbuda}
\begin{itemize}
\item {Grp. gram.:f.}
\end{itemize}
Antiga moéda portuguesa de prata.
Espécie de capacete antigo.
(B. lat. \textunderscore barbuta\textunderscore )
\section{Barbudo}
\begin{itemize}
\item {Grp. gram.:m.}
\end{itemize}
\begin{itemize}
\item {Grp. gram.:Adj.}
\end{itemize}
\begin{itemize}
\item {Proveniência:(De \textunderscore barba\textunderscore )}
\end{itemize}
Ave trepadora das regiões quentes.
Que tem muita barba.
\section{Bárbula}
\begin{itemize}
\item {Grp. gram.:f.}
\end{itemize}
\begin{itemize}
\item {Utilização:Bot.}
\end{itemize}
\begin{itemize}
\item {Proveniência:(De \textunderscore barba\textunderscore )}
\end{itemize}
Corpo vegetal, formado pelas celhas do perístoma, soldadas entre si.
\section{Barbusano}
\begin{itemize}
\item {Grp. gram.:m.}
\end{itemize}
O mesmo que \textunderscore pau-ferro\textunderscore .
\section{Barca}
\begin{itemize}
\item {Grp. gram.:f.}
\end{itemize}
\begin{itemize}
\item {Utilização:Restrict.}
\end{itemize}
\begin{itemize}
\item {Utilização:Náut.}
\end{itemize}
Embarcação larga e pouco funda.
Canção de barqueiros.
Navio de três mastros, immediatamente inferior á galera.
Instrumento, para medir a velocidade do navio; barquilha.
Nome de uma constellação. Cf. B. Pereira, \textunderscore Prosódia\textunderscore , vb. \textunderscore maschlazar\textunderscore .
(B. lat. \textunderscore barca\textunderscore )
\section{Barça}
\begin{itemize}
\item {Grp. gram.:f.}
\end{itemize}
\begin{itemize}
\item {Utilização:Prov.}
\end{itemize}
\begin{itemize}
\item {Utilização:alg.}
\end{itemize}
Capa de vimes para vidros ou loiça; balça.
Cêsto de palma, de fórma cylíndrica, e em que os trabalhadores levam as refeições para o lugar onde trabalham.
(Corr. de \textunderscore balça\textunderscore )
\section{Barcaça}
\begin{itemize}
\item {Grp. gram.:f.}
\end{itemize}
Grande barca.
Embarcação, destinada a serviços auxiliares de navegação.
\section{Barcada}
\begin{itemize}
\item {Grp. gram.:f.}
\end{itemize}
Carga de barca ou de barco.
\section{Barca-da-gacha}
\begin{itemize}
\item {Grp. gram.:f.}
\end{itemize}
Uma das embarcações, que se usam na pesca do atum.
\section{Barca-das-portas}
\begin{itemize}
\item {Grp. gram.:f.}
\end{itemize}
Uma das embarcações, usadas na pesca do atum.
\section{Barca-da-testa}
\begin{itemize}
\item {Grp. gram.:f.}
\end{itemize}
Uma das embarcações, usadas na pesca do atum.
\section{Barcádiga}
\begin{itemize}
\item {Grp. gram.:f.}
\end{itemize}
\begin{itemize}
\item {Utilização:Ant.}
\end{itemize}
O mesmo que \textunderscore barcagem\textunderscore .
\section{Barcagem}
\begin{itemize}
\item {Grp. gram.:f.}
\end{itemize}
\begin{itemize}
\item {Proveniência:(De \textunderscore barca\textunderscore )}
\end{itemize}
Barcada.
Contrato, pelo qual alguém se obriga a transportes por água.
\section{Barcalão}
\begin{itemize}
\item {Grp. gram.:m.}
\end{itemize}
Homem nobre na China. Cf. \textunderscore Peregrinação\textunderscore , CV.
\section{Barcarola}
\begin{itemize}
\item {Grp. gram.:f.}
\end{itemize}
\begin{itemize}
\item {Proveniência:(It. \textunderscore barcaruola\textunderscore )}
\end{itemize}
Canção de gondoleiros venezianos.
Peça musical, semelhante á dos gondoleiros.
Composição poética, acommodada ao estilo das barcarolas.
\section{Barca-volante}
\begin{itemize}
\item {Grp. gram.:f.}
\end{itemize}
Apparelho de pesca, formado de redes e que funcciona á semelhança de um galeão.
\section{Barcego}
\begin{itemize}
\item {Grp. gram.:m.}
\end{itemize}
\begin{itemize}
\item {Utilização:T. de Miranda}
\end{itemize}
O mesmo que \textunderscore barcéu\textunderscore .
\section{Barceiro}
\begin{itemize}
\item {Grp. gram.:m.}
\end{itemize}
Aquelle que faz barças.
\section{Barcelada}
\begin{itemize}
\item {Grp. gram.:f.}
\end{itemize}
\begin{itemize}
\item {Proveniência:(De \textunderscore Barcellos\textunderscore , n. p.?)}
\end{itemize}
Fio muito fino, com que se liga a pata do anzol, em alguns apparelhos de pesca.
\section{Barcellada}
\begin{itemize}
\item {Grp. gram.:f.}
\end{itemize}
\begin{itemize}
\item {Proveniência:(De \textunderscore Barcellos\textunderscore , n. p.?)}
\end{itemize}
Fio muito fino, com que se liga a pata do anzol, em alguns apparelhos de pesca.
\section{Barcellos}
\begin{itemize}
\item {Grp. gram.:f.}
\end{itemize}
\begin{itemize}
\item {Proveniência:(De \textunderscore Barcellos\textunderscore , n. p.)}
\end{itemize}
Espécie de videira portuguesa.
O fruto della.
\section{Barcelonês}
\begin{itemize}
\item {Grp. gram.:adj.}
\end{itemize}
\begin{itemize}
\item {Grp. gram.:M.}
\end{itemize}
Relativo a Barcelona.
Habitante de Barcelona.
\section{Barcelos}
\begin{itemize}
\item {Grp. gram.:f.}
\end{itemize}
\begin{itemize}
\item {Proveniência:(De \textunderscore Barcellos\textunderscore , n. p.)}
\end{itemize}
Espécie de videira portuguesa.
O fruto della.
\section{Barcéu}
\begin{itemize}
\item {Grp. gram.:m.}
\end{itemize}
\begin{itemize}
\item {Utilização:Prov.}
\end{itemize}
\begin{itemize}
\item {Utilização:trasm.}
\end{itemize}
Erva rija e filiforme, de que se fazem esteiras e capachos.
(Cast. \textunderscore barceo\textunderscore )
\section{Barcha}
\begin{itemize}
\item {Grp. gram.:f.}
\end{itemize}
\begin{itemize}
\item {Utilização:Ant.}
\end{itemize}
Navio grande, procedente das regiões do Norte.
(Cp. \textunderscore barca\textunderscore )
\section{Barchote}
\begin{itemize}
\item {Grp. gram.:m.}
\end{itemize}
\begin{itemize}
\item {Utilização:Ant.}
\end{itemize}
O mesmo que \textunderscore barcote\textunderscore .
\section{Barco}
\begin{itemize}
\item {Grp. gram.:m.}
\end{itemize}
Designação genérica de qualquer embarcação.
Embarcação pequena sem coberta.
(Cp. \textunderscore barca\textunderscore )
\section{Barco-da-sacada}
\begin{itemize}
\item {Grp. gram.:m.}
\end{itemize}
Embarcação da costa de Peniche, de convés corrido, com quatro escotilhas e dois mastros. Cf. Ortigão, \textunderscore Culto da Arte\textunderscore .
\section{Barcote}
\begin{itemize}
\item {Grp. gram.:m.}
\end{itemize}
\begin{itemize}
\item {Utilização:Des.}
\end{itemize}
Pequeno barco. Cf. Nunes do Leão, \textunderscore Livro da Fábr. das Naus\textunderscore .
\section{Barda}
\begin{itemize}
\item {Grp. gram.:f.}
\end{itemize}
\begin{itemize}
\item {Utilização:Ant.}
\end{itemize}
\begin{itemize}
\item {Proveniência:(Do nórd. ant. \textunderscore bardi\textunderscore , escudo, defesa)}
\end{itemize}
Tapume de ramos ou silvas entrelaçadas.
Tapume de madeira num curral.
Pranchão, com que se escora um muro que ameaça ruína.
Camada.
Grande quantidade: \textunderscore tem prédios em barda\textunderscore .
Armadura de ferro, para o peito do cavallo.
\section{Bardal}
\begin{itemize}
\item {Grp. gram.:m.}
\end{itemize}
\begin{itemize}
\item {Utilização:Prov.}
\end{itemize}
\begin{itemize}
\item {Utilização:alent.}
\end{itemize}
O mesmo que \textunderscore bradal\textunderscore .
\section{Bardana}
\begin{itemize}
\item {Grp. gram.:f.}
\end{itemize}
Planta synanthérea, (\textunderscore lappa major\textunderscore ).
Planta medicinal, da mesma fam., também chamada \textunderscore pegamassa\textunderscore  ou \textunderscore erva-dos-pegamassos\textunderscore , (\textunderscore xanthium strumarium\textunderscore ).
\section{Bardanal}
\begin{itemize}
\item {Grp. gram.:m.}
\end{itemize}
\begin{itemize}
\item {Utilização:T. de Alcanena}
\end{itemize}
\begin{itemize}
\item {Proveniência:(De \textunderscore bardana\textunderscore )}
\end{itemize}
Terreno fraco ou mal cultivado.
\section{Bardar}
\begin{itemize}
\item {Grp. gram.:v. t.}
\end{itemize}
Cobrir com barda, cercar com bardas.
\section{Bardesano}
\begin{itemize}
\item {Grp. gram.:m.}
\end{itemize}
Aquelle que é natural de Bardês.
\section{Bardia}
\begin{itemize}
\item {Grp. gram.:f.}
\end{itemize}
\begin{itemize}
\item {Utilização:T. de Miranda}
\end{itemize}
\begin{itemize}
\item {Proveniência:(De \textunderscore barda\textunderscore )}
\end{itemize}
Rima de lenha, á porta da habitação.
\section{Bárdico}
\begin{itemize}
\item {Grp. gram.:adj.}
\end{itemize}
\begin{itemize}
\item {Proveniência:(De \textunderscore bardo\textunderscore ^2)}
\end{itemize}
Relativo á poesia ou ao tempo dos bardos.
\section{Bardilho}
\begin{itemize}
\item {Grp. gram.:adj.}
\end{itemize}
\begin{itemize}
\item {Proveniência:(De \textunderscore barda\textunderscore ?)}
\end{itemize}
Diz-se do mármore cinzento do Alentejo.
\section{Bardino}
\begin{itemize}
\item {Grp. gram.:m.}
\end{itemize}
\begin{itemize}
\item {Utilização:Pop.}
\end{itemize}
\begin{itemize}
\item {Utilização:Prov.}
\end{itemize}
\begin{itemize}
\item {Utilização:trasm.}
\end{itemize}
\begin{itemize}
\item {Grp. gram.:Adj.}
\end{itemize}
Estroina; valdevinos; vadio.
Homem velhaco, cruel, vingativo.
Guarda de barda ou bardo.
Ratoneiro campestre.
Que assalta a barda.
\section{Bardito}
\begin{itemize}
\item {Grp. gram.:m.}
\end{itemize}
\begin{itemize}
\item {Proveniência:(Lat. \textunderscore barditus\textunderscore )}
\end{itemize}
Antigo canto de guerra, na Germânia.
\section{Bardo}
\begin{itemize}
\item {Grp. gram.:m.}
\end{itemize}
\begin{itemize}
\item {Utilização:Prov.}
\end{itemize}
\begin{itemize}
\item {Utilização:trasm.}
\end{itemize}
\begin{itemize}
\item {Utilização:dur.}
\end{itemize}
\begin{itemize}
\item {Utilização:Prov.}
\end{itemize}
\begin{itemize}
\item {Utilização:trasm.}
\end{itemize}
\begin{itemize}
\item {Utilização:Prov.}
\end{itemize}
\begin{itemize}
\item {Utilização:minh.}
\end{itemize}
O mesmo que \textunderscore barda\textunderscore .
Terreno cultivado.
Estaca, para empar videiras.
Renque de videiras, ligadas por varas, canas ou arame.
\section{Bardo}
\begin{itemize}
\item {Grp. gram.:m.}
\end{itemize}
\begin{itemize}
\item {Proveniência:(Lat. \textunderscore bardus\textunderscore )}
\end{itemize}
Cantor, que exaltava o valor dos guerreiros, entre os Celtas e entre os Gállios.
Poéta.
\section{Bardo}
\begin{itemize}
\item {Grp. gram.:m.}
\end{itemize}
\begin{itemize}
\item {Utilização:Prov.}
\end{itemize}
Dimensões, tamanho: \textunderscore a pedra que se despenhou, tinha o bardo de um boi\textunderscore .
(Colhido em Turquel)
\section{Bare}
\begin{itemize}
\item {Grp. gram.:m.}
\end{itemize}
Vestígio de antiga exploração aurífera, na Zambezia.
\section{Barege}
\begin{itemize}
\item {Grp. gram.:f.}
\end{itemize}
\begin{itemize}
\item {Proveniência:(De \textunderscore Bareges\textunderscore , n. p.)}
\end{itemize}
Tecido de lan, que recebeu o nome de um valle pyrenaico, em cujas povoações começou a fabricar-se.
\section{Baregina}
\begin{itemize}
\item {Grp. gram.:f.}
\end{itemize}
\begin{itemize}
\item {Proveniência:(De \textunderscore Bareges\textunderscore , n. p.)}
\end{itemize}
Substância orgânica, semelhante ao muco animal, e encontrada nas águas mineraes de Bareges.
\section{Barela}
\begin{itemize}
\item {Grp. gram.:f.}
\end{itemize}
Espécie de galera antiga.
\section{Barés}
\begin{itemize}
\item {Grp. gram.:m. pl.}
\end{itemize}
Cabilda de índios do Pará.
\section{Bareta}
\begin{itemize}
\item {fónica:barê}
\end{itemize}
\begin{itemize}
\item {Grp. gram.:f.}
\end{itemize}
\begin{itemize}
\item {Utilização:Des.}
\end{itemize}
O mesmo que \textunderscore barrete\textunderscore .
\section{Barga}
\begin{itemize}
\item {Grp. gram.:f.}
\end{itemize}
\begin{itemize}
\item {Proveniência:(T. cast.)}
\end{itemize}
Palhoça, cabana.
\section{Barga}
\begin{itemize}
\item {Grp. gram.:f.}
\end{itemize}
Espécie de rede de emmalhar.
\section{Bargado}
\begin{itemize}
\item {Grp. gram.:adj.}
\end{itemize}
\begin{itemize}
\item {Utilização:Bras. do Ceará}
\end{itemize}
\begin{itemize}
\item {Utilização:Bras. da Baía}
\end{itemize}
Esperto, finório, matreiro, (falando-se de gado).
Que compra e não paga.
\section{Bargado}
\begin{itemize}
\item {Grp. gram.:adj.}
\end{itemize}
\begin{itemize}
\item {Utilização:Bras. do N}
\end{itemize}
(V.bragado)
\section{Barganha}
\begin{itemize}
\item {Grp. gram.:f.}
\end{itemize}
\begin{itemize}
\item {Utilização:Fam.}
\end{itemize}
\begin{itemize}
\item {Proveniência:(De \textunderscore barganhar\textunderscore )}
\end{itemize}
Troca.
Trapaça; transacção cavilosa.
\section{Barganhar}
\begin{itemize}
\item {Grp. gram.:v. t.}
\end{itemize}
\begin{itemize}
\item {Proveniência:(Do b. lat. \textunderscore barcaniare\textunderscore )}
\end{itemize}
Trocar.
Vender.
\section{Bargani}
\begin{itemize}
\item {Grp. gram.:m.}
\end{itemize}
Antiga moéda de Gôa.
\section{Barganim}
\begin{itemize}
\item {Grp. gram.:m.}
\end{itemize}
Antiga moéda de Gôa.
\section{Bargantaria}
\begin{itemize}
\item {Grp. gram.:f.}
\end{itemize}
Vida de bargante.
\section{Bargante}
\begin{itemize}
\item {Grp. gram.:m.}
\end{itemize}
\begin{itemize}
\item {Proveniência:(Do b. lat. \textunderscore brigantes\textunderscore )}
\end{itemize}
Homem de maus costumes, libertino.
\section{Bargantear}
\begin{itemize}
\item {Grp. gram.:v. i.}
\end{itemize}
Levar vida de bargante.
\section{Bargela}
\begin{itemize}
\item {Grp. gram.:f.}
\end{itemize}
Peixe de Portugal.
\section{Bargueiro}
\begin{itemize}
\item {Grp. gram.:m.}
\end{itemize}
\begin{itemize}
\item {Proveniência:(De \textunderscore barga\textunderscore ^2)}
\end{itemize}
Aquelle que fazia redes chamadas bargas.
\section{Barilha}
\begin{itemize}
\item {Grp. gram.:f.}
\end{itemize}
(V.barrilha)
\section{Barimbé}
\begin{itemize}
\item {Grp. gram.:m.}
\end{itemize}
\begin{itemize}
\item {Utilização:Bras}
\end{itemize}
Arbusto, de cujo suco se fabríca uma bebida excitante.
\section{Barinel}
\begin{itemize}
\item {Grp. gram.:m.}
\end{itemize}
Antiga embarcação de vela, armando ás vezes remos.
(Corr. de \textunderscore varinel\textunderscore )
\section{Bariolagem}
\begin{itemize}
\item {Grp. gram.:f.}
\end{itemize}
\begin{itemize}
\item {Utilização:Mús.}
\end{itemize}
\begin{itemize}
\item {Proveniência:(Fr. \textunderscore bariolage\textunderscore )}
\end{itemize}
Maneira especial de executar certas peças na rabeca, com emprêgo de cordas soltas. Cf. \textunderscore Diccion. Mus.\textunderscore 
\section{Baris}
\begin{itemize}
\item {Grp. gram.:m. pl.}
\end{itemize}
Indígenas brasileiros das margens do Madeira.
Povo da África Oriental.
\section{Barjoleta}
\begin{itemize}
\item {fónica:lê}
\end{itemize}
\begin{itemize}
\item {Grp. gram.:f.}
\end{itemize}
\begin{itemize}
\item {Proveniência:(T. cast.)}
\end{itemize}
Mochila de coiro ou bolsa de linhagem.
\section{Barlaque}
\begin{itemize}
\item {Grp. gram.:m.}
\end{itemize}
\begin{itemize}
\item {Utilização:T. de Timor}
\end{itemize}
Compra de mulher.
\section{Barlaquear-se}
\begin{itemize}
\item {Grp. gram.:v. p.}
\end{itemize}
\begin{itemize}
\item {Utilização:T. de Timor}
\end{itemize}
\begin{itemize}
\item {Proveniência:(De \textunderscore barlaque\textunderscore )}
\end{itemize}
Adquirir mulher por compra.
\section{Barlaventeador}
\begin{itemize}
\item {Grp. gram.:adj.}
\end{itemize}
Que barlaventeia.
\section{Barlaventear}
\begin{itemize}
\item {Grp. gram.:v. i.}
\end{itemize}
\begin{itemize}
\item {Grp. gram.:V. p.}
\end{itemize}
Dirigir o navio contra a parte donde sopra o vento.
Pôr-se a barlavento.
\section{Barlaventejar}
\begin{itemize}
\item {Grp. gram.:v. i.}
\end{itemize}
\begin{itemize}
\item {Proveniência:(De \textunderscore barlavento\textunderscore )}
\end{itemize}
Deixar ir o navio á mercê do vento.
\section{Barlavento}
\begin{itemize}
\item {Grp. gram.:m.}
\end{itemize}
Bôrdo do navio, que fica para o lado donde sopra o vento.
No Algarve, parte da costa, comprehendida entre o Cabo de Santa-Maria e Sagres, porque ali os ventos dominantes são de Oéste.
\section{Barléria}
\begin{itemize}
\item {Grp. gram.:f.}
\end{itemize}
\begin{itemize}
\item {Proveniência:(De \textunderscore Barlerius\textunderscore , n. p. lat. do bot. fr. \textunderscore Barrelier\textunderscore )}
\end{itemize}
Gênero de plantas acantháceas.
\section{Barnabitas}
\begin{itemize}
\item {Grp. gram.:m. pl.}
\end{itemize}
\begin{itemize}
\item {Proveniência:(De \textunderscore Barnabé\textunderscore , n. p.)}
\end{itemize}
Congregação religiosa, fundada em França, no século XVI.
\section{Barnacle}
\begin{itemize}
\item {Grp. gram.:m.}
\end{itemize}
Ave aquática de arribação.
Ganso do mar.
\section{Barnagaes}
\begin{itemize}
\item {Grp. gram.:m.}
\end{itemize}
Antigo governador ou fronteiro de província litoral, na Abyssínia. Cf. Filinto, \textunderscore D. Man.\textunderscore  III, 272 e 273.
\section{Barnárdia}
\begin{itemize}
\item {Grp. gram.:f.}
\end{itemize}
\begin{itemize}
\item {Proveniência:(De \textunderscore Barnard\textunderscore , n. p.)}
\end{itemize}
Gênero de plantas liliáceas.
\section{Barnegal}
\begin{itemize}
\item {Grp. gram.:m.}
\end{itemize}
Antigo vaso para líquidos.
\section{Baroado}
\begin{itemize}
\item {Grp. gram.:m.}
\end{itemize}
\begin{itemize}
\item {Utilização:Ant.}
\end{itemize}
O mesmo que \textunderscore baronato\textunderscore .
\section{Baroce}
\begin{itemize}
\item {Grp. gram.:m.}
\end{itemize}
\begin{itemize}
\item {Proveniência:(De \textunderscore Barotze\textunderscore , ou antes \textunderscore Baroce\textunderscore , n. p.)}
\end{itemize}
Língua africana, pertencente á família das línguas cafreaes.
\section{Baroco}
\begin{itemize}
\item {fónica:barô}
\end{itemize}
\begin{itemize}
\item {Grp. gram.:adj.}
\end{itemize}
\begin{itemize}
\item {Proveniência:(It. \textunderscore barocco\textunderscore )}
\end{itemize}
Diz-se do trabalho artístico, que é irregular e extravagante.
\section{Baroco}
\begin{itemize}
\item {fónica:barô}
\end{itemize}
\begin{itemize}
\item {Grp. gram.:m.}
\end{itemize}
Termo mnemónico, que na Escolástica indicava um syllogismo.
(Palavra, cada uma de cujas letras tem um sentido convencional)
\section{Barógrafo}
\begin{itemize}
\item {Grp. gram.:m.}
\end{itemize}
O mesmo que \textunderscore barometrógrafo\textunderscore .
\section{Barógrapho}
\begin{itemize}
\item {Grp. gram.:m.}
\end{itemize}
O mesmo que \textunderscore barometrógrapho\textunderscore .
\section{Baroíl}
\begin{itemize}
\item {Grp. gram.:adj.}
\end{itemize}
\begin{itemize}
\item {Utilização:Ant.}
\end{itemize}
Próprio de \textunderscore barão\textunderscore , de homem illustre. Cf. Barros, \textunderscore Déc.\textunderscore  III, 85.
\section{Barol}
\begin{itemize}
\item {Grp. gram.:m.}
\end{itemize}
\begin{itemize}
\item {Utilização:Prov.}
\end{itemize}
O mesmo que \textunderscore bolór\textunderscore .
(Metáth. de \textunderscore balor\textunderscore , por \textunderscore bolór\textunderscore )
\section{Barologia}
\begin{itemize}
\item {Grp. gram.:f.}
\end{itemize}
\begin{itemize}
\item {Proveniência:(Do gr. \textunderscore baros\textunderscore  + \textunderscore logos\textunderscore )}
\end{itemize}
Parte das sciências phýsicas, que trata da gravidade.
\section{Barológico}
\begin{itemize}
\item {Grp. gram.:adj.}
\end{itemize}
Relativo á \textunderscore barologia\textunderscore .
\section{Baromacrómetro}
\begin{itemize}
\item {Grp. gram.:m.}
\end{itemize}
\begin{itemize}
\item {Proveniência:(Do gr. \textunderscore baros\textunderscore  + \textunderscore makros\textunderscore  + \textunderscore metron\textunderscore )}
\end{itemize}
Instrumento, para pesar e medir crianças recém-nascidas.
\section{Barometricamente}
\begin{itemize}
\item {Grp. gram.:adv.}
\end{itemize}
\begin{itemize}
\item {Proveniência:(De \textunderscore barométrico\textunderscore )}
\end{itemize}
Por meio de barómetro.
\section{Barométrico}
\begin{itemize}
\item {Grp. gram.:adj.}
\end{itemize}
Relativo ao \textunderscore barómetro\textunderscore .
\section{Barómetro}
\begin{itemize}
\item {Grp. gram.:m.}
\end{itemize}
\begin{itemize}
\item {Utilização:Fig.}
\end{itemize}
\begin{itemize}
\item {Proveniência:(Do gr. \textunderscore baros\textunderscore  + \textunderscore metron\textunderscore )}
\end{itemize}
Instrumento, com que se mede a pressão da atmosphera.
Aquillo que revela a marcha de certos negócios públicos ou particulares.
\section{Barometrografia}
\begin{itemize}
\item {Grp. gram.:f.}
\end{itemize}
Descripção dos barómetros.
(Cp. \textunderscore barometrógrapho\textunderscore )
\section{Barometrógrafo}
\begin{itemize}
\item {Grp. gram.:m.}
\end{itemize}
\begin{itemize}
\item {Proveniência:(Do gr. \textunderscore baros\textunderscore  + \textunderscore metron\textunderscore  + \textunderscore graphein\textunderscore )}
\end{itemize}
Apparelho, que regista, automática e continuamente, as variações da pressão atmosphérica.
\section{Barometrographia}
\begin{itemize}
\item {Grp. gram.:f.}
\end{itemize}
Descripção dos barómetros.
(Cp. \textunderscore barometrógrapho\textunderscore )
\section{Barometrógrapho}
\begin{itemize}
\item {Grp. gram.:m.}
\end{itemize}
\begin{itemize}
\item {Proveniência:(Do gr. \textunderscore baros\textunderscore  + \textunderscore metron\textunderscore  + \textunderscore graphein\textunderscore )}
\end{itemize}
Apparelho, que regista, automática e continuamente, as variações da pressão atmosphérica.
\section{Baronato}
\begin{itemize}
\item {Grp. gram.:m.}
\end{itemize}
Dignidade de barão.
\section{Baronesa}
\begin{itemize}
\item {Grp. gram.:f.}
\end{itemize}
Mulher, que tem a dignidade de barão ou que casou com barão.
(B. lat. \textunderscore baronissa\textunderscore )
\section{Baroneso}
\begin{itemize}
\item {Grp. gram.:m.}
\end{itemize}
\begin{itemize}
\item {Utilização:Chul.}
\end{itemize}
Marido, sem título, de mulher que é baronesa.
\section{Baronete}
\begin{itemize}
\item {fónica:nê}
\end{itemize}
\begin{itemize}
\item {Grp. gram.:m.}
\end{itemize}
\begin{itemize}
\item {Proveniência:(Ingl. \textunderscore baronet\textunderscore )}
\end{itemize}
Cavalleiro de certa ordem inglesa.
\section{Baronia}
\begin{itemize}
\item {Grp. gram.:f.}
\end{itemize}
\begin{itemize}
\item {Utilização:Ant.}
\end{itemize}
Baronato.
Terra, senhorio, que conferia ao possuidor o título de barão.
Grande feudo, dependente da Corôa francesa.
(B. lat. \textunderscore baronia\textunderscore )
\section{Baronial}
\begin{itemize}
\item {Grp. gram.:adj.}
\end{itemize}
Relativo a baronia ou a barões. Cf. Latino, \textunderscore Elog. Acad.\textunderscore , 154.
\section{Barosânemo}
\begin{itemize}
\item {Grp. gram.:m.}
\end{itemize}
\begin{itemize}
\item {Proveniência:(Do gr. \textunderscore baros\textunderscore  + \textunderscore anemos\textunderscore )}
\end{itemize}
Instrumento, para conhecer a fôrça do vento.
\section{Baroscópio}
\begin{itemize}
\item {Grp. gram.:m.}
\end{itemize}
\begin{itemize}
\item {Proveniência:(Do gr. \textunderscore baros\textunderscore  + \textunderscore skopein\textunderscore )}
\end{itemize}
Instrumento, que mostra a pressão do ar, e demonstra o princípio de Archimedes, applicado aos fluidos elásticos.
\section{Barquear}
\begin{itemize}
\item {Grp. gram.:v. i.}
\end{itemize}
(V.barquejar)
\section{Barqueira}
\begin{itemize}
\item {Grp. gram.:f.}
\end{itemize}
Apparelho de pesca, feito de uma ou duas varas, as quaes têm nos extremos linhas com muitos anzóis.
(Liga-se a \textunderscore barqueira\textunderscore ^3?)
\section{Barqueira}
\begin{itemize}
\item {Grp. gram.:f.}
\end{itemize}
\begin{itemize}
\item {Proveniência:(De \textunderscore Barqueiros\textunderscore , n. p.)}
\end{itemize}
Variedade de maçan.
\section{Barqueira}
\begin{itemize}
\item {Grp. gram.:f.}
\end{itemize}
\begin{itemize}
\item {Proveniência:(De \textunderscore barco\textunderscore )}
\end{itemize}
Mulher, que barqueja.
\section{Barqueiro}
\begin{itemize}
\item {Grp. gram.:m.}
\end{itemize}
Homem, que exerce a profissão de barquejar.
(B. lat. \textunderscore barcarius\textunderscore )
\section{Barqueiros}
\begin{itemize}
\item {Grp. gram.:f.}
\end{itemize}
\begin{itemize}
\item {Proveniência:(De \textunderscore Barqueiros\textunderscore , n. p.)}
\end{itemize}
Variedade de maçan, o mesmo que \textunderscore barqueira\textunderscore ^2.
\section{Barquejar}
\begin{itemize}
\item {Grp. gram.:v. i.}
\end{itemize}
Dirigir barco.
Passear de barco.
\section{Barqueta}
\begin{itemize}
\item {fónica:quê}
\end{itemize}
\begin{itemize}
\item {Grp. gram.:f.}
\end{itemize}
Pequena barca.
\section{Barquete}
\begin{itemize}
\item {fónica:quê}
\end{itemize}
\begin{itemize}
\item {Grp. gram.:m.}
\end{itemize}
Pequeno barco. Cf. Filinto, \textunderscore D. Man.\textunderscore 
\section{Barquilha}
\begin{itemize}
\item {Grp. gram.:f.}
\end{itemize}
\begin{itemize}
\item {Proveniência:(De \textunderscore barco\textunderscore )}
\end{itemize}
Instrumento, com que se avalia a velocidade dos navios.
\section{Barquilheiro}
\begin{itemize}
\item {Grp. gram.:m.}
\end{itemize}
Vendedor de barquilhos.
\section{Barquilho}
\begin{itemize}
\item {Grp. gram.:m.}
\end{itemize}
Espécie de pastel, ôco, e de fórma cylíndrica.
(Cast. \textunderscore barquillo\textunderscore )
\section{Barquinha}
\begin{itemize}
\item {Grp. gram.:f.}
\end{itemize}
\begin{itemize}
\item {Proveniência:(De \textunderscore barca\textunderscore )}
\end{itemize}
Barquilha.
Espécie de pequeno barco, pendente do aeróstato, e onde vai o aeronauta.
\section{Barquinho}
\begin{itemize}
\item {Grp. gram.:m.}
\end{itemize}
\begin{itemize}
\item {Utilização:Prov.}
\end{itemize}
\begin{itemize}
\item {Utilização:alent.}
\end{itemize}
Barco pequeno.
O mesmo que \textunderscore barquino\textunderscore .
Espécie de jôgo popular.
\section{Barquino}
\begin{itemize}
\item {Grp. gram.:m.}
\end{itemize}
\begin{itemize}
\item {Utilização:Prov.}
\end{itemize}
\begin{itemize}
\item {Utilização:alent.}
\end{itemize}
Pelle de chibo, preparada para conter e trasportar água potável.
(Talvez de \textunderscore barco\textunderscore )
\section{Barra}
\begin{itemize}
\item {Grp. gram.:f.}
\end{itemize}
\begin{itemize}
\item {Utilização:Náut.}
\end{itemize}
\begin{itemize}
\item {Utilização:Heráld.}
\end{itemize}
\begin{itemize}
\item {Utilização:Prov.}
\end{itemize}
\begin{itemize}
\item {Utilização:minh.}
\end{itemize}
\begin{itemize}
\item {Utilização:Bras. do N}
\end{itemize}
\begin{itemize}
\item {Utilização:Gír.}
\end{itemize}
\begin{itemize}
\item {Utilização:Náut.}
\end{itemize}
\begin{itemize}
\item {Grp. gram.:M.}
\end{itemize}
\begin{itemize}
\item {Utilização:Fam.}
\end{itemize}
Peça grossa de metal, antes de applicado a qualquer obra: \textunderscore oiro em barra\textunderscore .
Peça de ferro, com que se joga, ganhando aquelle que a atira mais longe.
Jôgo, em que se emprega esta barra.
Peça que, atravessada no mastaréu, o sustenta de pé.
Cana (do leme)
Designação de várias peças de ferro ou metal, applicadas em várias artes e offícios.
Barreira; extremo.
Entrada estreita de um pôrto: \textunderscore a barra de Lisbôa\textunderscore .
Carreira de tábulas, no jôgo do xadrez.
Arco de ferro, na mesa em que se joga o truque.
Fôrro interior das saias, junto á fímbria.
Fita, banda, que guarnece horizontalmente a parte exterior das saias.
Armação de um leito de ferro ou madeira.
Listão, que atravessa o escudo, no brasão.
Instrumento, sôbre que se tosa a baêta.
Espécie de andaime, sôbre as córtes do gado, onde se armazena palha, etc.
As côres avermelhadas do Poente, ao cair da tarde.
Garrafa de vinho.
\textunderscore Barra do cabrestante\textunderscore , cada uma das alavancas de madeira, a que se applica a fôrça braçal, para mover os cabrestantes, que não trabalham a vapor, nem por outro meio mecânico.
Homem robusto.
Aquelle que leva uma empresa a bom êxito.
Pimpão.
\section{Barraca}
\begin{itemize}
\item {Grp. gram.:f.}
\end{itemize}
\begin{itemize}
\item {Utilização:Fam.}
\end{itemize}
Pequena casa de madeira, ou de madeira com palha ou pano ou ramos, etc.
Casa humilde; tenda.
Grande guarda-chuva.
(Cp. b. lat. \textunderscore baraca\textunderscore , de \textunderscore bara\textunderscore )
\section{Barracão}
\begin{itemize}
\item {Grp. gram.:m.}
\end{itemize}
\begin{itemize}
\item {Utilização:Açor}
\end{itemize}
\begin{itemize}
\item {Utilização:Náut.}
\end{itemize}
Alpendre; telheiro, para abrigo provisório.
Grande barraca.
Mercado de peixe.
Toldo de lona, que se arma a bordo ou em tempo de chuva.
\section{Barracar}
\begin{itemize}
\item {Grp. gram.:v. t.}
\end{itemize}
O mesmo que \textunderscore abarracar\textunderscore .
\section{Barracento}
\begin{itemize}
\item {Grp. gram.:adj.}
\end{itemize}
(V.barrento)
\section{Barrachel}
\begin{itemize}
\item {Grp. gram.:m.}
\end{itemize}
\begin{itemize}
\item {Proveniência:(T. cast.)}
\end{itemize}
Antigo official militar, encarregado de apanhar os desertores.
\section{Barraco}
\begin{itemize}
\item {Grp. gram.:m.}
\end{itemize}
\begin{itemize}
\item {Utilização:Prov.}
\end{itemize}
\begin{itemize}
\item {Utilização:trasm.}
\end{itemize}
\begin{itemize}
\item {Proveniência:(De \textunderscore barraca\textunderscore )}
\end{itemize}
Córte para os bois, no campo, só para servir de dia.
\section{Barracório}
\begin{itemize}
\item {Grp. gram.:m.}
\end{itemize}
\begin{itemize}
\item {Utilização:Fam.}
\end{itemize}
Pequeno barracão, ordinário. Cf. Castilho, \textunderscore Fausto\textunderscore , 3.
\section{Barrada}
\begin{itemize}
\item {Grp. gram.:f.}
\end{itemize}
\begin{itemize}
\item {Utilização:Prov.}
\end{itemize}
\begin{itemize}
\item {Utilização:alent.}
\end{itemize}
\begin{itemize}
\item {Proveniência:(De \textunderscore barro\textunderscore )}
\end{itemize}
Terra de semear, nas encostas, fóra das vargens.
\section{Barrado}
\begin{itemize}
\item {Grp. gram.:adj.}
\end{itemize}
\begin{itemize}
\item {Grp. gram.:M.}
\end{itemize}
\begin{itemize}
\item {Utilização:Heráld.}
\end{itemize}
Que tem barra: \textunderscore um vestido barrado\textunderscore .
Campo coberto de barras de metal e de côr.
\section{Barrado}
\begin{itemize}
\item {Grp. gram.:adj.}
\end{itemize}
\begin{itemize}
\item {Grp. gram.:Loc.}
\end{itemize}
\begin{itemize}
\item {Utilização:fam.}
\end{itemize}
Coberto de barro.
\textunderscore Ficar barrado\textunderscore , sair-se mal de qualquer empresa ou propósito.
\section{Barradela}
\begin{itemize}
\item {Grp. gram.:f.}
\end{itemize}
Acto ou effeito de \textunderscore barrar\textunderscore ^2.
\section{Barradura}
\begin{itemize}
\item {Grp. gram.:f.}
\end{itemize}
Acto ou effeito de \textunderscore barrar\textunderscore .
\section{Barragan}
\begin{itemize}
\item {Grp. gram.:f.}
\end{itemize}
\begin{itemize}
\item {Utilização:Ant.}
\end{itemize}
Pelle de camaleão.
(B. lat. \textunderscore barracana\textunderscore )
\section{Barragem}
\begin{itemize}
\item {Grp. gram.:f.}
\end{itemize}
\begin{itemize}
\item {Proveniência:(De \textunderscore barrar\textunderscore ^1)}
\end{itemize}
Tapume, feito de troncos e ramos entrelaçados, dentro da água dos rios, para impedir a passagem do peixe, obrigando-o a convergir para determinado ponto.
\section{Barral}
\begin{itemize}
\item {Grp. gram.:m.}
\end{itemize}
(V.barreiro)
\section{Barramaque}
\begin{itemize}
\item {Grp. gram.:m.}
\end{itemize}
Espécie de tela antiga e preciosa.
\section{Barranca}
\begin{itemize}
\item {Grp. gram.:f.}
\end{itemize}
\begin{itemize}
\item {Utilização:Bras}
\end{itemize}
\begin{itemize}
\item {Utilização:Prov.}
\end{itemize}
\begin{itemize}
\item {Utilização:trasm.}
\end{itemize}
O mesmo que \textunderscore barranco\textunderscore .
Montículo de palha trilhada, que o vento vai juntando nas eiras, quando se limpam os cereaes, aventando-os.
\section{Barranceira}
\begin{itemize}
\item {Grp. gram.:f.}
\end{itemize}
(Alter. de \textunderscore ribanceira\textunderscore )
\section{Barranco}
\begin{itemize}
\item {Grp. gram.:m.}
\end{itemize}
\begin{itemize}
\item {Utilização:T. de Miranda}
\end{itemize}
Lugar, cavado por enxurradas ou por outra cáusa.
Escavação natural; precipício.
Obstáculo.
O mesmo que \textunderscore sepultura\textunderscore .
(B. lat. \textunderscore barrancus\textunderscore )
\section{Barrancoso}
\begin{itemize}
\item {Grp. gram.:adj.}
\end{itemize}
Que tem barrancos.
Perigoso.
\section{Barranhão}
\begin{itemize}
\item {Grp. gram.:m.}
\end{itemize}
(V.barrenhão)
\section{Barrão}
\begin{itemize}
\item {Grp. gram.:m.}
\end{itemize}
(V.varrão)
\section{Barraqueiro}
\begin{itemize}
\item {Grp. gram.:m.}
\end{itemize}
Aquelle que possue barraca, ou que vende em barraca quinquilharias ou comestíveis, etc.
\section{Barraquim}
\begin{itemize}
\item {Grp. gram.:m.}
\end{itemize}
Pequena barraca.
\section{Barrar}
\begin{itemize}
\item {Grp. gram.:v. t.}
\end{itemize}
Converter em barra; guarnecer com barra; atravessar com barra.
\section{Barrar}
\begin{itemize}
\item {Grp. gram.:v. t.}
\end{itemize}
Tapar com barro; cobrir, revestir, de barro.
Cobrir com qualquer substância molle, de modo semelhante ao com que se barram paredes.
\section{Barrário}
\begin{itemize}
\item {Grp. gram.:m.}
\end{itemize}
\begin{itemize}
\item {Utilização:Ant.}
\end{itemize}
Dizia-se de quem morava dentro da villa ou cidade.
(Cp. \textunderscore bairro\textunderscore , cast. \textunderscore barrio\textunderscore )
\section{Barrasco}
\begin{itemize}
\item {Grp. gram.:m.}
\end{itemize}
(V.varrasco)
\section{Barraza}
\begin{itemize}
\item {Grp. gram.:f.}
\end{itemize}
\begin{itemize}
\item {Utilização:Ant.}
\end{itemize}
Armadilha, para apanhar animaes ferozes.
(Talvez corr. de \textunderscore baraça\textunderscore )
\section{Barreada}
\begin{itemize}
\item {Grp. gram.:f.}
\end{itemize}
\begin{itemize}
\item {Utilização:Prov.}
\end{itemize}
\begin{itemize}
\item {Utilização:trasm.}
\end{itemize}
Faixa de terreno, quási ao cimo de uma encosta suave.
\section{Barrear}
\textunderscore v. t.\textunderscore  (e der.) \textunderscore Bras.\textunderscore 
O mesmo que \textunderscore barrar\textunderscore ^2, etc.
\section{Barregã}
\begin{itemize}
\item {Grp. gram.:f.}
\end{itemize}
Concubina.
(Fem. de \textunderscore barregão\textunderscore )
\section{Barregan}
\begin{itemize}
\item {Grp. gram.:f.}
\end{itemize}
Concubina.
(Fem. de \textunderscore barregão\textunderscore )
\section{Barregana}
\begin{itemize}
\item {Grp. gram.:f.}
\end{itemize}
Tecido forte de lan.
(Cp. \textunderscore barragan\textunderscore )
\section{Barregão}
\begin{itemize}
\item {Grp. gram.:m.}
\end{itemize}
Homem amancebado.
(Da mesma or. que \textunderscore barragan\textunderscore ?)
\section{Barregar}
\begin{itemize}
\item {Grp. gram.:v. i.}
\end{itemize}
(V.berregar)
\section{Barregueiro}
\begin{itemize}
\item {Grp. gram.:m.}
\end{itemize}
(V.barregão)
\section{Barreguice}
\begin{itemize}
\item {Grp. gram.:f.}
\end{itemize}
\begin{itemize}
\item {Proveniência:(De \textunderscore barregão\textunderscore )}
\end{itemize}
Concubinato.
\section{Barreio}
\begin{itemize}
\item {Grp. gram.:m.}
\end{itemize}
\begin{itemize}
\item {Utilização:Bras}
\end{itemize}
\begin{itemize}
\item {Proveniência:(De \textunderscore barro\textunderscore )}
\end{itemize}
Pastagem nos barreiros salgados.
\section{Barreira}
\begin{itemize}
\item {Grp. gram.:f.}
\end{itemize}
\begin{itemize}
\item {Proveniência:(De \textunderscore barro\textunderscore )}
\end{itemize}
Terreno argilloso.
Lugar, donde se extrái barro.
\section{Barreira}
\begin{itemize}
\item {Grp. gram.:f.}
\end{itemize}
Trincheira; estacada.
Alvo.
Obstáculo.
Limite; portas, entrada de um povoado.
Lugar, á entrada de uma povoação, onde se cobram direitos, pela introducção de mercadorias, ou gêneros de consumo, na povoação.
(B. lat. \textunderscore barreria\textunderscore )
\section{Barreira-dos-boticários}
\begin{itemize}
\item {Grp. gram.:f.}
\end{itemize}
\begin{itemize}
\item {Utilização:Ant.}
\end{itemize}
Válvula, entre o intestino delgado e grosso.
\section{Barreiro}
\begin{itemize}
\item {Grp. gram.:m.}
\end{itemize}
O mesmo que \textunderscore barreira\textunderscore ^1.
(B. lat. \textunderscore barrarius\textunderscore )
\section{Barrejamento}
\begin{itemize}
\item {Grp. gram.:m.}
\end{itemize}
\begin{itemize}
\item {Utilização:Ant.}
\end{itemize}
\begin{itemize}
\item {Proveniência:(De \textunderscore barrejar\textunderscore )}
\end{itemize}
Invasão; assalto.
\section{Barrejar}
\begin{itemize}
\item {Grp. gram.:v. t.}
\end{itemize}
\begin{itemize}
\item {Utilização:Ant.}
\end{itemize}
\begin{itemize}
\item {Proveniência:(De \textunderscore barra\textunderscore )}
\end{itemize}
Invadir; assaltar.
\section{Barrela}
\begin{itemize}
\item {Grp. gram.:f.}
\end{itemize}
\begin{itemize}
\item {Utilização:Fam.}
\end{itemize}
\begin{itemize}
\item {Utilização:Prov.}
\end{itemize}
\begin{itemize}
\item {Utilização:minh.}
\end{itemize}
Lixívia, dissolução alcalina, em que se immerge a roupa suja, para ficar limpa.
O tirar das manchas, que caíram na reputação de alguém.
Esparrela.
Mulher suja, enxovalhada; barrelona.
(Cp. \textunderscore barrilha\textunderscore )
\section{Barrelão}
\begin{itemize}
\item {Grp. gram.:m.}
\end{itemize}
\begin{itemize}
\item {Utilização:Prov.}
\end{itemize}
\begin{itemize}
\item {Utilização:minh.}
\end{itemize}
\begin{itemize}
\item {Proveniência:(De \textunderscore barrela\textunderscore )}
\end{itemize}
Homem sujo, bodegão.
\section{Barreleira}
\begin{itemize}
\item {Grp. gram.:f.}
\end{itemize}
\begin{itemize}
\item {Utilização:Prov.}
\end{itemize}
\begin{itemize}
\item {Utilização:trasm.}
\end{itemize}
\begin{itemize}
\item {Utilização:Fig.}
\end{itemize}
Mulher, que faz barrelas.
Mulher suja, repugnante.
Um dos apparelhos, nas fábricas de fiação.
\section{Barreleiro}
\begin{itemize}
\item {Grp. gram.:m.}
\end{itemize}
\begin{itemize}
\item {Utilização:T. da Nazaré}
\end{itemize}
\begin{itemize}
\item {Proveniência:(De \textunderscore barrela\textunderscore )}
\end{itemize}
Cinza, com que se fez lixívia.
Pano, que se estende por cima da roupa, e pelo qual se côa a lixívia sôbre ella.
Tripeça de madeira, com tabuleiro circular, em que se lava roupa.
\section{Barrelona}
\begin{itemize}
\item {Grp. gram.:f.}
\end{itemize}
\begin{itemize}
\item {Utilização:Prov.}
\end{itemize}
\begin{itemize}
\item {Utilização:minh.}
\end{itemize}
\begin{itemize}
\item {Proveniência:(De \textunderscore barrela\textunderscore )}
\end{itemize}
Mulher porca, suja.
\section{Barifonia}
\begin{itemize}
\item {Grp. gram.:f.}
\end{itemize}
\begin{itemize}
\item {Utilização:Med.}
\end{itemize}
\begin{itemize}
\item {Proveniência:(Do gr. \textunderscore barus\textunderscore  + \textunderscore phone\textunderscore )}
\end{itemize}
Rouquidão.
\section{Barimetria}
\begin{itemize}
\item {Grp. gram.:f.}
\end{itemize}
\begin{itemize}
\item {Proveniência:(Do gr. \textunderscore barus\textunderscore  + \textunderscore metron\textunderscore )}
\end{itemize}
Medição da gravidade, em Phýsica.
\section{Bário}
\begin{itemize}
\item {Grp. gram.:m.}
\end{itemize}
\begin{itemize}
\item {Proveniência:(Do gr. \textunderscore barus\textunderscore )}
\end{itemize}
Corpo mineral esbranquiçado.
\section{Barita}
\begin{itemize}
\item {Grp. gram.:f.}
\end{itemize}
Óxydo de bário.
\section{Baritina}
\begin{itemize}
\item {Grp. gram.:f.}
\end{itemize}
Sulfato de barita natural.
\section{Barito}
\begin{itemize}
\item {Grp. gram.:m.}
\end{itemize}
O mesmo que \textunderscore barita\textunderscore .
\section{Barítono}
\begin{itemize}
\item {Grp. gram.:m.}
\end{itemize}
\begin{itemize}
\item {Utilização:Gram.}
\end{itemize}
\begin{itemize}
\item {Grp. gram.:Adj.}
\end{itemize}
\begin{itemize}
\item {Utilização:Gram.}
\end{itemize}
\begin{itemize}
\item {Proveniência:(Gr. \textunderscore barutonos\textunderscore )}
\end{itemize}
Cantor, cujo tom de voz é intermédio ao grave e ao agudo.
Palavra, que tem accentuação longa ou tónica, na penúltima sýllaba.
Que tem accentuação longa ou tónica na penúltima sýllaba.
\section{Barrena}
\begin{itemize}
\item {Grp. gram.:f.}
\end{itemize}
\begin{itemize}
\item {Utilização:Prov.}
\end{itemize}
\begin{itemize}
\item {Utilização:alent.}
\end{itemize}
Broca de cavouqueiro.
(Cast. \textunderscore barrena\textunderscore )
\section{Barreneiro}
\begin{itemize}
\item {Grp. gram.:m.}
\end{itemize}
Aquelle que trabalha com barrena.
(Cast. \textunderscore barrenero\textunderscore )
\section{Barrenha}
\begin{itemize}
\item {Grp. gram.:f.}
\end{itemize}
\begin{itemize}
\item {Utilização:Prov.}
\end{itemize}
\begin{itemize}
\item {Utilização:alg.}
\end{itemize}
\begin{itemize}
\item {Utilização:Ant.}
\end{itemize}
\begin{itemize}
\item {Utilização:Prov.}
\end{itemize}
\begin{itemize}
\item {Utilização:trasm.}
\end{itemize}
\begin{itemize}
\item {Utilização:Prov.}
\end{itemize}
\begin{itemize}
\item {Utilização:beir.}
\end{itemize}
\begin{itemize}
\item {Proveniência:(Do cast. \textunderscore barreño\textunderscore )}
\end{itemize}
Vaso de barro para líquidos, espécie de bilha.
Grande tigela sopeira.
Espécie de alguidar.
\section{Barrenhão}
\begin{itemize}
\item {Grp. gram.:m.}
\end{itemize}
\begin{itemize}
\item {Utilização:Prov.}
\end{itemize}
\begin{itemize}
\item {Utilização:alent.}
\end{itemize}
\begin{itemize}
\item {Proveniência:(De \textunderscore barrenha\textunderscore )}
\end{itemize}
Pequeno alguidar.
Bacio.
Recipiente de madeira, onde se faz a travía para os porcos.
\section{Barreno}
\begin{itemize}
\item {Grp. gram.:m.}
\end{itemize}
\begin{itemize}
\item {Utilização:Prov.}
\end{itemize}
\begin{itemize}
\item {Utilização:minh.}
\end{itemize}
Tiro de mina ou de pedreira.
Estampido de morteiro.
(Cast. \textunderscore barreno\textunderscore )
\section{Barrenta}
\begin{itemize}
\item {Grp. gram.:adj. f.}
\end{itemize}
\begin{itemize}
\item {Proveniência:(De \textunderscore barrento\textunderscore )}
\end{itemize}
Diz-se de uma variedade de azeitona.
Diz-se da sardinha, salgada em barricas.
\section{Barrento}
\begin{itemize}
\item {Grp. gram.:adj.}
\end{itemize}
Barroso; que tem muito barro.
Feito de barro.
Que tem a natureza do barro.
\section{Barrer}
\begin{itemize}
\item {Grp. gram.:v. t.}
\end{itemize}
\begin{itemize}
\item {Utilização:Prov.}
\end{itemize}
\begin{itemize}
\item {Utilização:Ant.}
\end{itemize}
O mesmo que \textunderscore varrer\textunderscore :«\textunderscore barria a rua\textunderscore ». \textunderscore Anat. Joc.\textunderscore , I, 277.
\section{Barreta}
\begin{itemize}
\item {fónica:rê}
\end{itemize}
\begin{itemize}
\item {Grp. gram.:f.}
\end{itemize}
Barra pequena.
\section{Barreta}
\begin{itemize}
\item {fónica:rê}
\end{itemize}
\begin{itemize}
\item {Grp. gram.:f.}
\end{itemize}
\begin{itemize}
\item {Utilização:Ant.}
\end{itemize}
O mesmo que \textunderscore barrete\textunderscore .
\section{Barretada}
\begin{itemize}
\item {Grp. gram.:f.}
\end{itemize}
\begin{itemize}
\item {Proveniência:(De \textunderscore barrete\textunderscore )}
\end{itemize}
Acto de saudar alguém, tirando o barrete ou o chapéu da cabeça.
\section{Barrete}
\begin{itemize}
\item {fónica:rê}
\end{itemize}
\begin{itemize}
\item {Grp. gram.:m.}
\end{itemize}
\begin{itemize}
\item {Proveniência:(Do b. lat. \textunderscore birretum\textunderscore )}
\end{itemize}
Cobertura molle, ordinariamente de pano, que se ajusta á cabeça.
Cobertura quadrangular, para a cabeça de clérigo.
Planta celastrínea.
Segunda cavidade do estômago dos ruminantes.
\section{Barrete-de-clérigo}
\begin{itemize}
\item {Grp. gram.:m.}
\end{itemize}
\begin{itemize}
\item {Utilização:Constr.}
\end{itemize}
\begin{itemize}
\item {Utilização:Bot.}
\end{itemize}
\begin{itemize}
\item {Utilização:Prov.}
\end{itemize}
\begin{itemize}
\item {Utilização:alent.}
\end{itemize}
Casta de uva, cujos bagos são segmentados e coloridos diversamente.
Espécie de abóbada, resultante do entrecruzamento de duas abóbadas cylíndricas e iguaes.
Planta celastrínea, (\textunderscore evonymus europaeus\textunderscore , Lin.).
Variedade de morango.
\section{Barreteiro}
\begin{itemize}
\item {Grp. gram.:m.}
\end{itemize}
Aquelle que faz barretes.
\section{Barretina}
\begin{itemize}
\item {Grp. gram.:f.}
\end{itemize}
\begin{itemize}
\item {Utilização:Gír.}
\end{itemize}
\begin{itemize}
\item {Proveniência:(T. cast.)}
\end{itemize}
Antigo chapéu de senhora.
Cobertura alta, que os militares usam na cabeça.
O mesmo que \textunderscore bebedeira\textunderscore .
\section{Barreto}
\begin{itemize}
\item {fónica:rê}
\end{itemize}
\begin{itemize}
\item {Grp. gram.:m.}
\end{itemize}
\begin{itemize}
\item {Utilização:Ant.}
\end{itemize}
O mesmo que \textunderscore barrete\textunderscore .
\section{Barrica}
\begin{itemize}
\item {Grp. gram.:f.}
\end{itemize}
Pequena vasilha, em fórma de pipa, para objectos de mercearia ou drogaria.
\section{Barricada}
\begin{itemize}
\item {Grp. gram.:f.}
\end{itemize}
\begin{itemize}
\item {Proveniência:(Fr. \textunderscore barricade\textunderscore )}
\end{itemize}
Fortificação provisória, trincheira, feita com barricas, carros, estacas, etc.
\section{Barricar}
\begin{itemize}
\item {Grp. gram.:v. t.}
\end{itemize}
\begin{itemize}
\item {Proveniência:(De \textunderscore barrica\textunderscore )}
\end{itemize}
Defender com barricada.
\section{Barrieira}
\begin{itemize}
\item {Grp. gram.:f.}
\end{itemize}
\begin{itemize}
\item {Utilização:Ant.}
\end{itemize}
Espécie de diadema, guarnecido de pedrarias.
(Talvez do fr. \textunderscore barrière\textunderscore )
\section{Barriera}
\begin{itemize}
\item {Grp. gram.:f.}
\end{itemize}
\begin{itemize}
\item {Utilização:Ant.}
\end{itemize}
O mesmo que \textunderscore barrieira\textunderscore .
\section{Barriga}
\begin{itemize}
\item {Grp. gram.:f.}
\end{itemize}
\begin{itemize}
\item {Proveniência:(Do ant. al. \textunderscore baldrich\textunderscore ?)}
\end{itemize}
Cavidade abdominal; ventre; pança.
Bojo.
Saliência.
Parte carnuda e posterior (da perna).
\section{Barrigada}
\begin{itemize}
\item {Grp. gram.:f.}
\end{itemize}
\begin{itemize}
\item {Proveniência:(De \textunderscore barriga\textunderscore )}
\end{itemize}
Pançada; effeito de encher muito a barriga, comendo.
Acto de fartar-se; fartadela.
\section{Barriga-de-freira}
\begin{itemize}
\item {Grp. gram.:f.}
\end{itemize}
O mesmo que \textunderscore encharcada\textunderscore .
\section{Barrigadinho}
\begin{itemize}
\item {Grp. gram.:m.}
\end{itemize}
\begin{itemize}
\item {Utilização:Bras}
\end{itemize}
Nome de um peixe pequeno.
\section{Barrigal}
\begin{itemize}
\item {Grp. gram.:adj.}
\end{itemize}
Relativo á barriga. Cf. Arn. Gama, \textunderscore Motim\textunderscore , 79.
\section{Barrigana}
\begin{itemize}
\item {Grp. gram.:m.  e  adj.}
\end{itemize}
\begin{itemize}
\item {Utilização:Prov.}
\end{itemize}
\begin{itemize}
\item {Grp. gram.:M.}
\end{itemize}
O mesmo que \textunderscore barrigudo\textunderscore .
Aquelle que é barrigudo.
\section{Barriganha}
\begin{itemize}
\item {Grp. gram.:m.  e  adj.}
\end{itemize}
\begin{itemize}
\item {Utilização:Prov.}
\end{itemize}
\begin{itemize}
\item {Grp. gram.:M.}
\end{itemize}
O mesmo que \textunderscore barrigudo\textunderscore .
Aquelle que é barrigudo.
\section{Barrigão}
\begin{itemize}
\item {Grp. gram.:m.}
\end{itemize}
\begin{itemize}
\item {Utilização:Fam.}
\end{itemize}
Grande barriga.
\section{Barriguda}
\begin{itemize}
\item {Grp. gram.:f.}
\end{itemize}
\begin{itemize}
\item {Grp. gram.:Adj. f.}
\end{itemize}
\begin{itemize}
\item {Utilização:Bras. do N}
\end{itemize}
\begin{itemize}
\item {Proveniência:(De \textunderscore barrigudo\textunderscore )}
\end{itemize}
Árvore brasileira, também chamada \textunderscore árvore-da-lan\textunderscore .
Prenhe, grávida.
\section{Barrigudo}
\begin{itemize}
\item {Grp. gram.:adj.}
\end{itemize}
\begin{itemize}
\item {Grp. gram.:M.}
\end{itemize}
\begin{itemize}
\item {Utilização:Bras}
\end{itemize}
\begin{itemize}
\item {Proveniência:(De \textunderscore barriga\textunderscore )}
\end{itemize}
Que tem grande barriga.
Espécie de macaco.
\section{Barrigueira}
\begin{itemize}
\item {Grp. gram.:f.}
\end{itemize}
\begin{itemize}
\item {Utilização:Bras}
\end{itemize}
\begin{itemize}
\item {Utilização:Prov.}
\end{itemize}
\begin{itemize}
\item {Utilização:alent.}
\end{itemize}
\begin{itemize}
\item {Proveniência:(De \textunderscore barriga\textunderscore )}
\end{itemize}
Peça dos arreios, que passa pela barriga da bêsta.
Trança ou corda que, passando por baixo da barriga das bêstas, prende a extremidade na parte superior dos canzis.
\section{Barrigueiro}
\begin{itemize}
\item {Grp. gram.:m.}
\end{itemize}
O mesmo que \textunderscore barrigueira\textunderscore .
\section{Barriguinha}
\begin{itemize}
\item {Grp. gram.:f.}
\end{itemize}
\begin{itemize}
\item {Proveniência:(De \textunderscore barriga\textunderscore )}
\end{itemize}
Peixe de Portugal.
\section{Barril}
\begin{itemize}
\item {Grp. gram.:m.}
\end{itemize}
\begin{itemize}
\item {Utilização:Prov.}
\end{itemize}
\begin{itemize}
\item {Utilização:alg.}
\end{itemize}
Pequena barrica, que serve só para líquidos.
Pequeno vaso, feito de aduelas.
Bilha de barro, de grande bojo, gargalo estreito e duas asas.
Nome de uma armação da pesca do atum.
\section{Barrilada}
\begin{itemize}
\item {Grp. gram.:f.}
\end{itemize}
\begin{itemize}
\item {Utilização:Pleb.}
\end{itemize}
\begin{itemize}
\item {Proveniência:(De \textunderscore barril\textunderscore )}
\end{itemize}
Porção de líquido, contido num barril.
Travessura; motim.
\section{Barrileira}
\begin{itemize}
\item {Grp. gram.:f.}
\end{itemize}
\begin{itemize}
\item {Proveniência:(De \textunderscore barril\textunderscore )}
\end{itemize}
Vasilha, em que se faz a decoada, com que se lavam as fôrmas typográphicas.
Mesa, em que se junta o sôro da coalhada, e donde cai para um balde.
Francelho.
\section{Barrilete}
\begin{itemize}
\item {fónica:lê}
\end{itemize}
\begin{itemize}
\item {Grp. gram.:m.}
\end{itemize}
\begin{itemize}
\item {Utilização:Prov.}
\end{itemize}
\begin{itemize}
\item {Utilização:alent.}
\end{itemize}
\begin{itemize}
\item {Proveniência:(De \textunderscore barril\textunderscore )}
\end{itemize}
Instrumento de ferro, com que os carpinteiros, marceneiros e entalhadores prendem ao banco a madeira que lavram.
Pequeno barril.
Pequena peça de clarinete, em fórma de barril.
\section{Barrilha}
\begin{itemize}
\item {Grp. gram.:f.}
\end{itemize}
Cinza da barrilheira.
(Cast. \textunderscore barrilla\textunderscore )
\section{Barrilheira}
\begin{itemize}
\item {Grp. gram.:f.}
\end{itemize}
\begin{itemize}
\item {Proveniência:(De \textunderscore barrilha\textunderscore )}
\end{itemize}
Planta herbácea, que contém muita soda, e de cuja cinza se faz barrela.
\section{Barrir}
\begin{itemize}
\item {Grp. gram.:v. i.}
\end{itemize}
\begin{itemize}
\item {Proveniência:(Lat. \textunderscore barrire\textunderscore )}
\end{itemize}
Diz-se da voz do elephante e de outros animaes.
\section{Barrisco}
\begin{itemize}
\item {Grp. gram.:m. Loc. adv.}
\end{itemize}
\begin{itemize}
\item {Utilização:Ant.}
\end{itemize}
\begin{itemize}
\item {Utilização:Prov.}
\end{itemize}
\begin{itemize}
\item {Utilização:trasm.}
\end{itemize}
\textunderscore A barrisco\textunderscore , unanimemente.
\textunderscore De borrisco\textunderscore , feito a fio, levando tudo adeante.
(Por \textunderscore varrisco\textunderscore , de \textunderscore varrer\textunderscore )
\section{Barrisco}
\begin{itemize}
\item {Grp. gram.:m.}
\end{itemize}
\begin{itemize}
\item {Utilização:T. da Bairrada}
\end{itemize}
\begin{itemize}
\item {Proveniência:(De \textunderscore barro\textunderscore )}
\end{itemize}
Terreno barrento.
\section{Barrista}
\begin{itemize}
\item {Grp. gram.:m.}
\end{itemize}
Aquelle que trabalha ou modela em barro.
\section{Barrista}
\begin{itemize}
\item {Grp. gram.:m.}
\end{itemize}
Acrobata, que trabalha em barras fixas.
\section{Barrito}
\begin{itemize}
\item {Grp. gram.:m.}
\end{itemize}
\begin{itemize}
\item {Proveniência:(Lat. \textunderscore barritus\textunderscore )}
\end{itemize}
A voz do elephante e de outros animaes.
\section{Barro}
\begin{itemize}
\item {Grp. gram.:m.}
\end{itemize}
\begin{itemize}
\item {Utilização:Pop.}
\end{itemize}
\begin{itemize}
\item {Utilização:Ant.}
\end{itemize}
Argilla; terra própria para fabríco de loiça.
Coisa insignificante.
Pequena habitação campestre.
(Talvez da mesma or. que \textunderscore barra\textunderscore )
\section{Barrôa}
\begin{itemize}
\item {Grp. gram.:f.}
\end{itemize}
\begin{itemize}
\item {Utilização:Prov.}
\end{itemize}
\begin{itemize}
\item {Utilização:alent.}
\end{itemize}
Mulher, que vem do norte, trabalhar com os macobios ou passadores.
(Corr. de \textunderscore beirôa\textunderscore , fem. de \textunderscore beirão\textunderscore ?)
\section{Barroca}
\begin{itemize}
\item {Grp. gram.:f.}
\end{itemize}
\begin{itemize}
\item {Utilização:Prov.}
\end{itemize}
\begin{itemize}
\item {Utilização:beir.}
\end{itemize}
\begin{itemize}
\item {Proveniência:(De \textunderscore barro\textunderscore )}
\end{itemize}
O mesmo que \textunderscore barranco\textunderscore .
Barreiro.
Escavação natural.
Passagem funda entre penedos ou barrocos.
\section{Barrocal}
\begin{itemize}
\item {Grp. gram.:f.}
\end{itemize}
Lugar, onde há barrocas.
\section{Barrocal}
\begin{itemize}
\item {Grp. gram.:m.}
\end{itemize}
\begin{itemize}
\item {Utilização:Prov.}
\end{itemize}
\begin{itemize}
\item {Utilização:beir.}
\end{itemize}
\begin{itemize}
\item {Proveniência:(De \textunderscore barroco\textunderscore )}
\end{itemize}
Lugar, onde há muitos penedos insulados ou barrocos.
\section{Barrocão}
\begin{itemize}
\item {Grp. gram.:m.}
\end{itemize}
Grande barroca.
\section{Barroco}
\begin{itemize}
\item {fónica:rô}
\end{itemize}
\begin{itemize}
\item {Grp. gram.:m.}
\end{itemize}
O mesmo que \textunderscore barroca\textunderscore .
Penedo insulado e de fórma irregular.
(Cp. \textunderscore barroca\textunderscore )
\section{Barroco}
\begin{itemize}
\item {fónica:rô}
\end{itemize}
\begin{itemize}
\item {Grp. gram.:m.}
\end{itemize}
Pérola, de superfície irregular.
\section{Barrões}
\begin{itemize}
\item {Grp. gram.:m. pl.}
\end{itemize}
\begin{itemize}
\item {Utilização:T. de Santarém}
\end{itemize}
Adventícios, que periodicamente ali vão procurar trabalho.
(Cp. \textunderscore barrôa\textunderscore )
\section{Barromaque}
\begin{itemize}
\item {Grp. gram.:m.}
\end{itemize}
\begin{itemize}
\item {Utilização:Ant.}
\end{itemize}
Provavelmente, o mesmo que \textunderscore maromaque\textunderscore .
\section{Barroneira}
\begin{itemize}
\item {Grp. gram.:adj. f.}
\end{itemize}
\begin{itemize}
\item {Utilização:Prov.}
\end{itemize}
\begin{itemize}
\item {Utilização:minh.}
\end{itemize}
Diz-se da porca, que procura o porco.
(Por \textunderscore varroneira\textunderscore , de \textunderscore varrão\textunderscore )
\section{Barroqueira}
\begin{itemize}
\item {Grp. gram.:f.  e  adj.}
\end{itemize}
\begin{itemize}
\item {Utilização:Prov.}
\end{itemize}
\begin{itemize}
\item {Utilização:alent.}
\end{itemize}
\begin{itemize}
\item {Proveniência:(De \textunderscore barroco\textunderscore ^1)}
\end{itemize}
Mó, para farinha ordinária, em opposição a alveira.
\section{Barroqueiral}
\begin{itemize}
\item {Grp. gram.:m.}
\end{itemize}
Lugar, onde há muitos barroqueiros.
\section{Barroqueiro}
\begin{itemize}
\item {Grp. gram.:m.}
\end{itemize}
\begin{itemize}
\item {Utilização:Prov.}
\end{itemize}
\begin{itemize}
\item {Utilização:alent.}
\end{itemize}
\begin{itemize}
\item {Proveniência:(De \textunderscore barroco\textunderscore ^1)}
\end{itemize}
Barroco, pedra tôsca.
\section{Barrosão}
\begin{itemize}
\item {Grp. gram.:adj.}
\end{itemize}
O mesmo que \textunderscore barrosinho\textunderscore :«\textunderscore energia barrosan\textunderscore ». Camillo, \textunderscore Corja\textunderscore , 172.
\section{Barrosinho}
\begin{itemize}
\item {Grp. gram.:adj.}
\end{itemize}
\begin{itemize}
\item {Proveniência:(De \textunderscore Barroso\textunderscore , n. p.)}
\end{itemize}
Relativo a Barroso; procedente da região de Barroso, ou que se cria nella.
\section{Barroso}
\begin{itemize}
\item {Grp. gram.:adj.}
\end{itemize}
\begin{itemize}
\item {Utilização:Bras}
\end{itemize}
\begin{itemize}
\item {Grp. gram.:M.}
\end{itemize}
\begin{itemize}
\item {Proveniência:(De \textunderscore barro\textunderscore )}
\end{itemize}
Cheio ou coberto de barro.
Diz-se do boi ou vacca branca.
Peixe plagióstomo, de focinho alongado e chato, e pelle coberta de um invólucro granuloso.
\section{Barrotado}
\begin{itemize}
\item {Grp. gram.:m.}
\end{itemize}
\begin{itemize}
\item {Utilização:Constr.}
\end{itemize}
Systema de barrotes, dispostos de maneira, que supportem vigas ou se lhe preguem fasquias.
\section{Barrotar}
\begin{itemize}
\item {Grp. gram.:v. t.}
\end{itemize}
Segurar com barrotes.
\section{Barrote}
\begin{itemize}
\item {Grp. gram.:m.}
\end{itemize}
\begin{itemize}
\item {Proveniência:(De \textunderscore barra\textunderscore )}
\end{itemize}
Trave grossa e curta, que sustém solhos, tábuas, etc.
\section{Barrotear}
\begin{itemize}
\item {Grp. gram.:v. t.}
\end{itemize}
(V.barrotar)
\section{Barroteiro}
\begin{itemize}
\item {Grp. gram.:m.}
\end{itemize}
\begin{itemize}
\item {Utilização:T. de Aveiro}
\end{itemize}
Cada um dos dois remadores que, no castellinho do barco de pesca, governam os cabos contra a rebentação. Cf. rev. \textunderscore Tradição\textunderscore , IV, 151.
\section{Barrotim}
\begin{itemize}
\item {Grp. gram.:m.}
\end{itemize}
Barrote pequeno.
\section{Barruço}
\begin{itemize}
\item {Grp. gram.:m.}
\end{itemize}
\begin{itemize}
\item {Utilização:Prov.}
\end{itemize}
\begin{itemize}
\item {Utilização:beir.}
\end{itemize}
O mesmo que \textunderscore barrete\textunderscore .
\section{Barrufar}
\begin{itemize}
\item {Grp. gram.:v. t.}
\end{itemize}
\begin{itemize}
\item {Utilização:Prov.}
\end{itemize}
\begin{itemize}
\item {Utilização:Ant.}
\end{itemize}
O mesmo que \textunderscore borrifar\textunderscore .
\section{Barrufo}
\begin{itemize}
\item {Grp. gram.:m.}
\end{itemize}
\begin{itemize}
\item {Utilização:Prov.}
\end{itemize}
\begin{itemize}
\item {Utilização:Ant.}
\end{itemize}
O mesmo que \textunderscore borrifo\textunderscore .
\section{Barruma}
\begin{itemize}
\item {Grp. gram.:f.}
\end{itemize}
(Fórma pop. de \textunderscore verruma\textunderscore )
\section{Barrunchão}
\begin{itemize}
\item {Grp. gram.:m.}
\end{itemize}
\begin{itemize}
\item {Utilização:Prov.}
\end{itemize}
\begin{itemize}
\item {Utilização:beir.}
\end{itemize}
\begin{itemize}
\item {Proveniência:(De \textunderscore barro\textunderscore )}
\end{itemize}
Grande alguidar.
\section{Barrunta}
\begin{itemize}
\item {Grp. gram.:m.}
\end{itemize}
\begin{itemize}
\item {Utilização:Prov.}
\end{itemize}
\begin{itemize}
\item {Utilização:trasm.}
\end{itemize}
\begin{itemize}
\item {Proveniência:(T. cast.)}
\end{itemize}
Labrego; bodegão.
\section{Barruntar}
\begin{itemize}
\item {Grp. gram.:v. t.}
\end{itemize}
\begin{itemize}
\item {Utilização:Pop.}
\end{itemize}
Conjecturar; suspeitar.
Têr noticia de. Cf. \textunderscore Eufrosina\textunderscore , 104.
(Cast. \textunderscore barruntar\textunderscore )
\section{Barrunto}
\begin{itemize}
\item {Grp. gram.:m.}
\end{itemize}
Acção de \textunderscore barruntar\textunderscore . Cf. Castilho, \textunderscore Fausto\textunderscore , 172.
(Cast. \textunderscore barrunto\textunderscore )
\section{Bartavela}
\begin{itemize}
\item {Grp. gram.:f.}
\end{itemize}
\begin{itemize}
\item {Proveniência:(Fr. \textunderscore bartavelle\textunderscore )}
\end{itemize}
Espécie de perdiz avermelhada.
\section{Bartavella}
\begin{itemize}
\item {Grp. gram.:f.}
\end{itemize}
\begin{itemize}
\item {Proveniência:(Fr. \textunderscore bartavelle\textunderscore )}
\end{itemize}
Espécie de perdiz avermelhada.
\section{Bartholomeu}
\begin{itemize}
\item {Grp. gram.:m.}
\end{itemize}
\begin{itemize}
\item {Utilização:Prov.}
\end{itemize}
Ave, o mesmo que \textunderscore papa-figos\textunderscore .
\section{Barthónia}
\begin{itemize}
\item {Grp. gram.:f.}
\end{itemize}
Planta annual, de flôr doirada.
\section{Bartolomeu}
\begin{itemize}
\item {Grp. gram.:m.}
\end{itemize}
\begin{itemize}
\item {Utilização:Prov.}
\end{itemize}
Ave, o mesmo que \textunderscore papa-figos\textunderscore .
\section{Bartónia}
\begin{itemize}
\item {Grp. gram.:f.}
\end{itemize}
Planta annual, de flôr doirada.
\section{Bartrâmia}
\begin{itemize}
\item {Grp. gram.:f.}
\end{itemize}
\begin{itemize}
\item {Proveniência:(De \textunderscore Bartram\textunderscore , n. p.)}
\end{itemize}
Espécie de musgo vivaz.
\section{Baru}
\begin{itemize}
\item {Grp. gram.:m.}
\end{itemize}
Árvore leguminosa do Brasil.
\section{Baruísta}
\begin{itemize}
\item {Grp. gram.:m.}
\end{itemize}
Habitante do Barué, em África.
\section{Barulhar}
\begin{itemize}
\item {Grp. gram.:v. t.}
\end{itemize}
Amotinar.
Misturar.
Pôr em barulho.
\section{Barulheira}
\begin{itemize}
\item {Grp. gram.:f.}
\end{itemize}
Grande barulho.
Confusão.
\section{Barulheiro}
\begin{itemize}
\item {Grp. gram.:adj.}
\end{itemize}
O mesmo que \textunderscore barulhento\textunderscore .
\section{Barulhento}
\begin{itemize}
\item {Grp. gram.:adj.}
\end{itemize}
Que faz barulho; desordeiro.
\section{Barulho}
\begin{itemize}
\item {Grp. gram.:m.}
\end{itemize}
Estrondo.
Desordem; motim.
Multidão de coisas em desordem.
Alarde.
(Cp. cast. \textunderscore barbulla\textunderscore )
\section{Barulhoso}
\begin{itemize}
\item {Grp. gram.:adj.}
\end{itemize}
O mesmo que \textunderscore barulhento\textunderscore . Cf. Arn. Gama, \textunderscore Motim\textunderscore , 318.
\section{Barururus}
\begin{itemize}
\item {Grp. gram.:m. pl.}
\end{itemize}
Selvagens brasileiros, nas margens do Barururu, affluente do Amazonas.
\section{Barymetria}
\begin{itemize}
\item {Grp. gram.:f.}
\end{itemize}
\begin{itemize}
\item {Proveniência:(Do gr. \textunderscore barus\textunderscore  + \textunderscore metron\textunderscore )}
\end{itemize}
Medição da gravidade, em Phýsica.
\section{Báryo}
\begin{itemize}
\item {Grp. gram.:m.}
\end{itemize}
\begin{itemize}
\item {Proveniência:(Do gr. \textunderscore barus\textunderscore )}
\end{itemize}
Corpo mineral esbranquiçado.
\section{Baryphonia}
\begin{itemize}
\item {Grp. gram.:f.}
\end{itemize}
\begin{itemize}
\item {Utilização:Med.}
\end{itemize}
\begin{itemize}
\item {Proveniência:(Do gr. \textunderscore barus\textunderscore  + \textunderscore phone\textunderscore )}
\end{itemize}
Rouquidão.
\section{Baryta}
\begin{itemize}
\item {Grp. gram.:f.}
\end{itemize}
Óxydo de báryo.
\section{Barytina}
\begin{itemize}
\item {Grp. gram.:f.}
\end{itemize}
Sulfato de baryta natural.
\section{Baryto}
\begin{itemize}
\item {Grp. gram.:m.}
\end{itemize}
O mesmo que \textunderscore baryta\textunderscore .
\section{Barýtono}
\begin{itemize}
\item {Grp. gram.:m.}
\end{itemize}
\begin{itemize}
\item {Utilização:Gram.}
\end{itemize}
\begin{itemize}
\item {Grp. gram.:Adj.}
\end{itemize}
\begin{itemize}
\item {Utilização:Gram.}
\end{itemize}
\begin{itemize}
\item {Proveniência:(Gr. \textunderscore barutonos\textunderscore )}
\end{itemize}
Cantor, cujo tom de voz é intermédio ao grave e ao agudo.
Palavra, que tem accentuação longa ou tónica, na penúltima sýllaba.
Que tem accentuação longa ou tónica na penúltima sýllaba.
\section{Barzabum}
\begin{itemize}
\item {Grp. gram.:m.}
\end{itemize}
\begin{itemize}
\item {Utilização:Prov.}
\end{itemize}
\begin{itemize}
\item {Utilização:beir.}
\end{itemize}
(Corr. de \textunderscore belzebu\textunderscore )
\section{Barzoneiro}
\begin{itemize}
\item {Grp. gram.:m.  e  adj.}
\end{itemize}
\begin{itemize}
\item {Utilização:P. us.}
\end{itemize}
\begin{itemize}
\item {Proveniência:(Do cast. \textunderscore barzonear\textunderscore )}
\end{itemize}
Ocioso; vàdio.
\section{Basa}
\begin{itemize}
\item {Grp. gram.:f.}
\end{itemize}
\begin{itemize}
\item {Utilização:Des.}
\end{itemize}
O mesmo que \textunderscore base\textunderscore . Cf. Castilho, \textunderscore Metam.\textunderscore , 304.
\section{Basalarte}
\begin{itemize}
\item {Grp. gram.:m.}
\end{itemize}
(?)«\textunderscore ...um bulhão bem garnido, á guisa de basalarte.\textunderscore »Fernão Lopes, \textunderscore Chrón. de D. Fern.\textunderscore , c. CII.
\section{Basalisco}
\begin{itemize}
\item {Grp. gram.:m.}
\end{itemize}
Peça de artilharia, o mesmo que \textunderscore basilisco\textunderscore . Cf. \textunderscore Livro das Monções\textunderscore , n.^o 13.
\section{Basáltico}
\begin{itemize}
\item {Grp. gram.:adj.}
\end{itemize}
Formado de basalto.
\section{Basaltiforme}
\begin{itemize}
\item {Grp. gram.:adj.}
\end{itemize}
Semelhante ao \textunderscore basalto\textunderscore .
\section{Basalto}
\begin{itemize}
\item {Grp. gram.:m.}
\end{itemize}
\begin{itemize}
\item {Proveniência:(Lat. \textunderscore basaltes\textunderscore )}
\end{itemize}
Rocha, de origem ígnea, muito dura e ordinariamente escura.
\section{Basarisco}
\begin{itemize}
\item {Grp. gram.:m.}
\end{itemize}
O mesmo que \textunderscore basalisco\textunderscore .
\section{Basbana}
\begin{itemize}
\item {Grp. gram.:m.  e  adj.}
\end{itemize}
\begin{itemize}
\item {Utilização:Prov.}
\end{itemize}
\begin{itemize}
\item {Utilização:alg.}
\end{itemize}
Estólido, parvo.
(Cp. \textunderscore basbaque\textunderscore )
\section{Basbaque}
\begin{itemize}
\item {Grp. gram.:m.}
\end{itemize}
\begin{itemize}
\item {Utilização:Pop.}
\end{itemize}
Indivíduo, que se espanta de tudo.
Pateta; parvo.
\section{Basbaquice}
\begin{itemize}
\item {Grp. gram.:f.}
\end{itemize}
Modos ou acção de basbaque.
\section{Basbaquismo}
\begin{itemize}
\item {Grp. gram.:m.}
\end{itemize}
\begin{itemize}
\item {Utilização:P. us.}
\end{itemize}
O mesmo que \textunderscore basbaquice\textunderscore .
\section{Bascamar}
\begin{itemize}
\item {Grp. gram.:v. t.}
\end{itemize}
Atormentar (alguém), arrancando-lhe cabello, barbas e sobrancelhas. Cf. B. Pato, \textunderscore Port. na Índia\textunderscore , 152.
\section{Basco}
\begin{itemize}
\item {Grp. gram.:m.}
\end{itemize}
O mesmo que \textunderscore vasconço\textunderscore .
\section{Basculhadeira}
\begin{itemize}
\item {Grp. gram.:f.}
\end{itemize}
Mulher, que basculha.
\section{Basculhadela}
\begin{itemize}
\item {Grp. gram.:f.}
\end{itemize}
Acção de \textunderscore basculhar\textunderscore .
\section{Basculhador}
\begin{itemize}
\item {Grp. gram.:m.}
\end{itemize}
Aquelle que basculha.
\section{Basculhar}
\begin{itemize}
\item {Grp. gram.:v. t.}
\end{itemize}
Varrer com basculho.
Pesquisar; esquadrinhar.
\section{Basculho}
\begin{itemize}
\item {Grp. gram.:m.}
\end{itemize}
\begin{itemize}
\item {Utilização:Fig.}
\end{itemize}
\begin{itemize}
\item {Utilização:Prov.}
\end{itemize}
\begin{itemize}
\item {Utilização:trasm.}
\end{itemize}
Vassoira, de cabo comprido, para limpar tectos ou objectos altos.
Pessôa enxovalhada: \textunderscore Aquella mulher é um vasculho\textunderscore .
Indivíduo, que se occupa em trabalhos muito ordinários.
Rapaz bochechudo, gorducho.
\section{Básculo}
\begin{itemize}
\item {Grp. gram.:m.}
\end{itemize}
\begin{itemize}
\item {Proveniência:(Do fr. \textunderscore bascule\textunderscore , sendo a acentuação do t. port. devida á analogia com outros voc., terminados em \textunderscore ulo\textunderscore , átono)}
\end{itemize}
Espécie de ponte levadiça.
\section{Base}
\begin{itemize}
\item {Grp. gram.:f.}
\end{itemize}
\begin{itemize}
\item {Utilização:Mús.}
\end{itemize}
\begin{itemize}
\item {Utilização:Chím.}
\end{itemize}
\begin{itemize}
\item {Utilização:Topogr.}
\end{itemize}
\begin{itemize}
\item {Utilização:Bot.}
\end{itemize}
\begin{itemize}
\item {Proveniência:(Lat. \textunderscore basis\textunderscore )}
\end{itemize}
Aquillo que supporta o pêso de um objecto.
Parte inferior: \textunderscore base de uma columna\textunderscore .
Princípio, fundamento: \textunderscore a base de um systema\textunderscore .
Plano opposto ao vértice.
Pedestal.
A parte de uma construcção, que se firma immediatamente no solo, resaindo do corpo que sustenta.
Círculo, que termina um cylindro, sendo perpendicular ao eixo dêste.
Número invariável, com que se define um systema de numeração.
Nota tónica.
Elemento electropositivo de um corpo composto.
Linha recta, a que se referem todas as outras, no levantamento de uma planta topográphica.
A parte de um órgão vegetal mais próxima do seu ponto de inserção.
Origem ou ponto de inserção dos órgãos de uma planta ou de um corpo animal.
Ingrediente principal, que entra numa mistura chímica.
\section{Baseamento}
\begin{itemize}
\item {Grp. gram.:m.}
\end{itemize}
\begin{itemize}
\item {Proveniência:(De \textunderscore basear\textunderscore )}
\end{itemize}
Corpo grande e massiço, em que assenta um edifício, geralmente mais largo que alto.
\section{Basear}
\begin{itemize}
\item {Grp. gram.:v. t.}
\end{itemize}
\begin{itemize}
\item {Proveniência:(De \textunderscore base\textunderscore )}
\end{itemize}
Estabelecer as bases de; fundamentar, firmar: \textunderscore basear uma argumentação\textunderscore .
\section{Baselga}
\begin{itemize}
\item {Grp. gram.:adj.}
\end{itemize}
\begin{itemize}
\item {Utilização:Pop.}
\end{itemize}
Barrigudo. Cf. Castilho, \textunderscore Fausto\textunderscore , 374.
(Fórma. evolutiva de \textunderscore basílica\textunderscore )
\section{Basicidade}
\begin{itemize}
\item {Grp. gram.:f.}
\end{itemize}
\begin{itemize}
\item {Utilização:Chím.}
\end{itemize}
\begin{itemize}
\item {Proveniência:(De \textunderscore básico\textunderscore )}
\end{itemize}
Propriedade, que um corpo tem, de servir de base numa mistura.
\section{Básico}
\begin{itemize}
\item {Grp. gram.:adj.}
\end{itemize}
Que serve de base; principal, essencial: \textunderscore princípios básicos\textunderscore .
\section{Basidiomicetos}
\begin{itemize}
\item {Grp. gram.:m. pl.}
\end{itemize}
\begin{itemize}
\item {Proveniência:(Do gr. \textunderscore basidios\textunderscore  + \textunderscore mukes\textunderscore  + \textunderscore etos\textunderscore )}
\end{itemize}
Ordem de fungos, a que pertence o cogumelo commum.
\section{Basidiomycetos}
\begin{itemize}
\item {Grp. gram.:m. pl.}
\end{itemize}
\begin{itemize}
\item {Proveniência:(Do gr. \textunderscore basidios\textunderscore  + \textunderscore mukes\textunderscore  + \textunderscore etos\textunderscore )}
\end{itemize}
Ordem de fungos, a que pertence o cogumelo commum.
\section{Basificação}
\begin{itemize}
\item {Grp. gram.:f.}
\end{itemize}
Acção de \textunderscore basificar-se\textunderscore .
\section{Basificar-se}
\begin{itemize}
\item {Grp. gram.:v. p.}
\end{itemize}
\begin{itemize}
\item {Proveniência:(Do lat. \textunderscore basis\textunderscore  + \textunderscore facere\textunderscore )}
\end{itemize}
Converter-se em base.
Diz-se, em Chímica, de um corpo que passa para o estado de base.
\section{Basifixo}
\begin{itemize}
\item {fónica:cso}
\end{itemize}
\begin{itemize}
\item {Grp. gram.:adj.}
\end{itemize}
\begin{itemize}
\item {Proveniência:(De \textunderscore base\textunderscore  + \textunderscore fixo\textunderscore )}
\end{itemize}
Fixo pela base ou na base de alguma coisa.
\section{Basilar}
\begin{itemize}
\item {Grp. gram.:adj.}
\end{itemize}
\begin{itemize}
\item {Proveniência:(Fr. \textunderscore basilaire\textunderscore )}
\end{itemize}
Básico.
Que nasce da base.
\section{Basilarmente}
\begin{itemize}
\item {Grp. gram.:adv.}
\end{itemize}
\begin{itemize}
\item {Utilização:Neol.}
\end{itemize}
De modo basilar; essencialmente. Cf. Sousa Martins, \textunderscore Nosogr.\textunderscore 
\section{Basílica}
\begin{itemize}
\item {Grp. gram.:f.}
\end{itemize}
\begin{itemize}
\item {Utilização:Ant.}
\end{itemize}
\begin{itemize}
\item {Proveniência:(Lat. \textunderscore basilica\textunderscore )}
\end{itemize}
Igreja principal.
Relicário.
Espécie de barraca pyramidal, coberta de damasco, e levada nas procissões da Sé patriarchal.
Palácio.
Edifício dos tribunaes.
\section{Basilicão}
\begin{itemize}
\item {Grp. gram.:m.}
\end{itemize}
\begin{itemize}
\item {Proveniência:(Gr. \textunderscore basilikon\textunderscore )}
\end{itemize}
Unguento de pez, resina, cera e azeite.
\section{Basilicário}
\begin{itemize}
\item {Grp. gram.:m.}
\end{itemize}
\begin{itemize}
\item {Utilização:Ant.}
\end{itemize}
\begin{itemize}
\item {Proveniência:(De \textunderscore basílica\textunderscore )}
\end{itemize}
Clérigo, que assistia ao Papa, ao Bispo, ou ao sacerdote, quando officiavam.
\section{Basílico}
\begin{itemize}
\item {Grp. gram.:adj.}
\end{itemize}
\begin{itemize}
\item {Utilização:Anat.}
\end{itemize}
\begin{itemize}
\item {Grp. gram.:M.}
\end{itemize}
\begin{itemize}
\item {Proveniência:(Gr. \textunderscore basilikos\textunderscore )}
\end{itemize}
Diz-se de uma veia, que sobe na parte interna do braço.
Planta annual, labiada.
\section{Basilisco}
\begin{itemize}
\item {Grp. gram.:m.}
\end{itemize}
\begin{itemize}
\item {Proveniência:(Lat. \textunderscore basiliscus\textunderscore )}
\end{itemize}
Lagarto fabuloso, a que se attribuia o poder de matar com a vista.
Reptil americano.
Antiga peça de artilharia.
\section{Basim}
\begin{itemize}
\item {Grp. gram.:f.}
\end{itemize}
\begin{itemize}
\item {Proveniência:(Do b. gr. \textunderscore bombaxion\textunderscore , de \textunderscore bombux\textunderscore ?)}
\end{itemize}
Tecido de algodão de Bengala.
\section{Basinérveo}
\begin{itemize}
\item {Grp. gram.:adj.}
\end{itemize}
\begin{itemize}
\item {Utilização:Bot.}
\end{itemize}
\begin{itemize}
\item {Proveniência:(De \textunderscore base\textunderscore  + \textunderscore nérveo\textunderscore )}
\end{itemize}
Diz-se das fôlhas, cujas nervuras partem da base.
\section{Basione}
\begin{itemize}
\item {Grp. gram.:m.}
\end{itemize}
\begin{itemize}
\item {Proveniência:(Do gr. \textunderscore basis\textunderscore )}
\end{itemize}
Ponto craniométrico, na linha média da base do crânio sôbre o bordo anterior do buraco occipital.
\section{Basiótico}
\begin{itemize}
\item {Grp. gram.:m.}
\end{itemize}
\begin{itemize}
\item {Utilização:Anat.}
\end{itemize}
\begin{itemize}
\item {Proveniência:(De \textunderscore base\textunderscore )}
\end{itemize}
Osso autónomo, que constitue a parte mais deanteira da base do occipital.
\section{Basiopinacoide}
\begin{itemize}
\item {Grp. gram.:adj.}
\end{itemize}
\begin{itemize}
\item {Utilização:Miner.}
\end{itemize}
\begin{itemize}
\item {Proveniência:(Do gr. \textunderscore basis\textunderscore  + \textunderscore pinax\textunderscore  + \textunderscore eidos\textunderscore )}
\end{itemize}
Diz-se da fórma do crystal, limitada por dois planos parallelos entre si e a dois eixos cristallográphicos.
\section{Básis}
\begin{itemize}
\item {Grp. gram.:m.}
\end{itemize}
\begin{itemize}
\item {Utilização:Ant.}
\end{itemize}
O mesmo que \textunderscore base\textunderscore ; assento, lugar:«\textunderscore o zodiaco, onde os doze animais tem seu básis\textunderscore ». \textunderscore Eufrosina\textunderscore , 24.
\section{Basofobia}
\begin{itemize}
\item {Grp. gram.:f.}
\end{itemize}
\begin{itemize}
\item {Utilização:Med.}
\end{itemize}
Abasia fóbica.
\section{Basophobia}
\begin{itemize}
\item {Grp. gram.:f.}
\end{itemize}
\begin{itemize}
\item {Utilização:Med.}
\end{itemize}
Abasia phóbica.
\section{Bassáride}
\begin{itemize}
\item {Grp. gram.:f.}
\end{itemize}
\begin{itemize}
\item {Proveniência:(Lat. \textunderscore bassaris\textunderscore )}
\end{itemize}
Sacerdotiza de Baccho, bacchante.
\section{Bassorina}
\begin{itemize}
\item {Grp. gram.:f.}
\end{itemize}
\begin{itemize}
\item {Proveniência:(De \textunderscore Bassorá\textunderscore , n. p.)}
\end{itemize}
Princípio immediato da goma de Bassorá.
\section{Bassutos}
\begin{itemize}
\item {Grp. gram.:m. pl.}
\end{itemize}
Povos da África austro-central.
\section{Basta}
\begin{itemize}
\item {Grp. gram.:f.}
\end{itemize}
\begin{itemize}
\item {Utilização:Prov.}
\end{itemize}
\begin{itemize}
\item {Utilização:minh.}
\end{itemize}
\begin{itemize}
\item {Grp. gram.:Interj.}
\end{itemize}
Cordel, com que se atravessam os colchões ou almofadas, para segurar o enchimento.
Pequena peça de pano ou lan, que remata êsses cordéis na face do colchão.
Préga, que se faz na roupa, especialmente nas saias, para as tornar mais curtas. Cf. O. Pratt, \textunderscore Ling. Minh.\textunderscore 
Não mais!(V.bastar)
\section{Basta}
\begin{itemize}
\item {Grp. gram.:f.}
\end{itemize}
\begin{itemize}
\item {Utilização:Gír.}
\end{itemize}
O mesmo que \textunderscore bata\textunderscore ^2.
\section{Bastamente}
\begin{itemize}
\item {Grp. gram.:adv.}
\end{itemize}
\begin{itemize}
\item {Proveniência:(De \textunderscore basto\textunderscore )}
\end{itemize}
Em chusma; compactamente.
\section{Bastante}
\begin{itemize}
\item {Grp. gram.:adj.}
\end{itemize}
\begin{itemize}
\item {Utilização:Ant.}
\end{itemize}
\begin{itemize}
\item {Grp. gram.:Adv.}
\end{itemize}
\begin{itemize}
\item {Proveniência:(De \textunderscore bastar\textunderscore )}
\end{itemize}
Que basta, que é sufficiente: \textunderscore recursos bastantes\textunderscore .
Possante, robusto.
Sufficientemente.
Muito: \textunderscore soffreu bastante\textunderscore .
\section{Bastantemente}
\begin{itemize}
\item {Grp. gram.:adv.}
\end{itemize}
\begin{itemize}
\item {Proveniência:(De \textunderscore bastante\textunderscore )}
\end{itemize}
De modo sufficiente.
Muito.
\section{Bastantissimamente}
\begin{itemize}
\item {Grp. gram.:adv.}
\end{itemize}
De modo \textunderscore bastantíssimo\textunderscore . Cf. \textunderscore Ethiopia Or.\textunderscore , II, 311.
\section{Bastantíssimo}
\begin{itemize}
\item {Grp. gram.:adj.}
\end{itemize}
Mais que bastante; elevado a grande número ou quantidade. Cf. Rui Barbosa, \textunderscore Répl.\textunderscore , 157.
\section{Bastão}
\begin{itemize}
\item {Grp. gram.:m.}
\end{itemize}
\begin{itemize}
\item {Grp. gram.:Pl.}
\end{itemize}
\begin{itemize}
\item {Utilização:Heráld.}
\end{itemize}
\begin{itemize}
\item {Grp. gram.:Adj.}
\end{itemize}
\begin{itemize}
\item {Proveniência:(De um rad. commum a \textunderscore basto\textunderscore , ao provn. \textunderscore bastir\textunderscore , e que se encontra no gr. \textunderscore bastazein\textunderscore  e \textunderscore bastax\textunderscore )}
\end{itemize}
Pau, que se póde trazer na mão, como apoio, como arma ou como insígnia.
Bordão.
Grande bengala.
Vinho encorpado e muito tinto.
Palas estreitas, que cobrem o campo do escudo ou parte delle.
Muito basto, denso. Cf. \textunderscore Techn. Rur.\textunderscore , I, 15.
\section{Bastar}
\begin{itemize}
\item {Grp. gram.:v. i.}
\end{itemize}
Sêr sufficiente: \textunderscore basta vê-la, para a adorarmos\textunderscore .
Satisfazer, sêr adequado.
(Do mesmo rad. de \textunderscore basto\textunderscore  e \textunderscore bastão\textunderscore )
\section{Bastarda}
\begin{itemize}
\item {Grp. gram.:f.}
\end{itemize}
Parece têr sido um dos systemas de equitação, talvez o da estardiota, que era opposto ao da gineta:«\textunderscore entender da bastarda e da gineta\textunderscore ». R. Lobo, \textunderscore Côrte na Aldeia\textunderscore , I, 11.
\section{Bastardão}
\begin{itemize}
\item {Grp. gram.:m.}
\end{itemize}
Espécie de lima, de serrilha ou picado entre grosso e fino.
Casta de uva.
\section{Bastardear}
\begin{itemize}
\item {Grp. gram.:v. t.}
\end{itemize}
(V.abastardar)
\section{Bastardeira}
\begin{itemize}
\item {Grp. gram.:f.}
\end{itemize}
Casta de uva preta, semelhante ao bastardo.
\section{Bastardeiro}
\begin{itemize}
\item {Grp. gram.:adj.}
\end{itemize}
Diz-se do vinho, fabricado de baldoeira.
\section{Bastardia}
\begin{itemize}
\item {Grp. gram.:f.}
\end{itemize}
Qualidade de quem é bastardo.
Ramo bastardo de uma família.
Degeneração.
\section{Bastardia}
\begin{itemize}
\item {Grp. gram.:f.}
\end{itemize}
Planta malvácea, semelhante ao abutilão.
\section{Bastardinha}
\begin{itemize}
\item {Grp. gram.:f.}
\end{itemize}
Espécie de lima, de serrilha menos grossa que a do \textunderscore bastardão\textunderscore .
\section{Bastardinho}
\begin{itemize}
\item {Grp. gram.:m.}
\end{itemize}
\begin{itemize}
\item {Proveniência:(De \textunderscore bastardo\textunderscore )}
\end{itemize}
Espécie de calligraphia, semelhante ao bastardo, mas mais miúda.
Espécie de uva, o mesmo que \textunderscore bastardo\textunderscore .
\section{Bastardo}
\begin{itemize}
\item {Grp. gram.:adj.}
\end{itemize}
\begin{itemize}
\item {Grp. gram.:M.}
\end{itemize}
\begin{itemize}
\item {Proveniência:(Fr. \textunderscore bâtard\textunderscore )}
\end{itemize}
Que nasceu fóra de matrimónio.
Modificado, degenerado.
Filho illegítimo.
Espécie de uva preta, de bagos pequenos e muito juntos.
Espécie de calligraphia, inclinada e cheia.
Antiga moéda de 10 soldos.
Antiga vela triangular de pequena embarcação.
Cabo náutico, que entra nos furos das lebres.
Antiga moéda de estanho, mandada cunhar em Malaca por Affonso de Albuquerque. Cf. Barros, \textunderscore Déc.\textunderscore  II, l. VI, c. 6.
\section{Bastardo-branco}
\begin{itemize}
\item {Grp. gram.:m.}
\end{itemize}
Variedade de uva de Azeitão.
\section{Bastardo-roxo}
\begin{itemize}
\item {Grp. gram.:m.}
\end{itemize}
Variedade de uva de Azeitão.
\section{Baste}
\begin{itemize}
\item {Grp. gram.:m.}
\end{itemize}
\begin{itemize}
\item {Utilização:Ant.}
\end{itemize}
\begin{itemize}
\item {Proveniência:(Fr. \textunderscore bât\textunderscore )}
\end{itemize}
Sella das cavalgaduras que transportam peças, cofres e reparos da artilharia de campanha.
\section{Bastear}
\begin{itemize}
\item {Grp. gram.:v. i.}
\end{itemize}
Pôr bastas em; acolchoar.
\section{Bastecer}
\textunderscore v. t.\textunderscore  (e der.)
(V. \textunderscore abastecer\textunderscore , etc.)
\section{Basteirar}
\begin{itemize}
\item {Grp. gram.:v. t.}
\end{itemize}
\begin{itemize}
\item {Utilização:Bras. do S}
\end{itemize}
Produzir basteiras (o lombilho).
\section{Basteiras}
\begin{itemize}
\item {Grp. gram.:f. pl.}
\end{itemize}
\begin{itemize}
\item {Utilização:Bras. do S}
\end{itemize}
\begin{itemize}
\item {Proveniência:(De \textunderscore basto\textunderscore ^3)}
\end{itemize}
Manchas de pêlos brancos, no lombo do cavallo, onde assentam os bastos ou lombilhos.
\section{Basteirear}
\begin{itemize}
\item {Grp. gram.:v. i.}
\end{itemize}
O mesmo que \textunderscore basteirar\textunderscore .
\section{Basteiro}
\begin{itemize}
\item {Grp. gram.:adj.}
\end{itemize}
\begin{itemize}
\item {Proveniência:(De \textunderscore basto\textunderscore ^2)}
\end{itemize}
Diz-se, na costa de Aveiro, do mar, quando, picado pelo vento do Norte, encarreira muitas ondas á praía.
\section{Bastiães}
\begin{itemize}
\item {Grp. gram.:m. pl.}
\end{itemize}
\begin{itemize}
\item {Utilização:Ant.}
\end{itemize}
\begin{itemize}
\item {Proveniência:(De \textunderscore basta\textunderscore ^1? ou alter. de \textunderscore bestiães\textunderscore , do lat. \textunderscore bestia\textunderscore , porque êsses lavores representaram primitivamente anímaes?)}
\end{itemize}
Lavores altos; trabalhos em relêvo.
\section{Bastião}
\begin{itemize}
\item {Grp. gram.:m.}
\end{itemize}
Muro, que serve de anteparo ao ângulo saliente de uma fortaleza.
(Cast. \textunderscore bastión\textunderscore )
\section{Bastião}
\begin{itemize}
\item {Grp. gram.:m.}
\end{itemize}
Moéda de prata.(V.xerafim)
\section{Bastibarbo}
\begin{itemize}
\item {Grp. gram.:adj.}
\end{itemize}
Que tem barba basta. Cf. Filinto, VIII, 27.
\section{Bastida}
\begin{itemize}
\item {Grp. gram.:f.}
\end{itemize}
\begin{itemize}
\item {Proveniência:(De \textunderscore bastir\textunderscore )}
\end{itemize}
Trincheira de paus; ripado.
Bastidão.
Antiga máquina de guerra, muito alta, sôbre rodas.
\section{Bastidão}
\begin{itemize}
\item {Grp. gram.:f.}
\end{itemize}
Qualidade do que é basto.
Multidão; espessura: \textunderscore a bastidão do arvoredo\textunderscore .
\section{Bastidor}
\begin{itemize}
\item {Grp. gram.:m.}
\end{itemize}
\begin{itemize}
\item {Grp. gram.:Pl.}
\end{itemize}
\begin{itemize}
\item {Utilização:Fig.}
\end{itemize}
\begin{itemize}
\item {Proveniência:(De \textunderscore bastir\textunderscore )}
\end{itemize}
Espécie de caixilho, em que os bordadores pregam o estôfo em que se executa o bordado.
Cada uma das decorações lateraes de um palco.
Intervallos, que separam estas decorações.
Coisas íntimas ou particulares (da política, das finanças, etc.).
\section{Bastilha}
\begin{itemize}
\item {Grp. gram.:f.}
\end{itemize}
\begin{itemize}
\item {Utilização:Ant.}
\end{itemize}
\begin{itemize}
\item {Proveniência:(Fr. \textunderscore bastille\textunderscore )}
\end{itemize}
Fortaleza.
\section{Bastilhão}
\begin{itemize}
\item {Grp. gram.:m.}
\end{itemize}
\begin{itemize}
\item {Utilização:Ant.}
\end{itemize}
\begin{itemize}
\item {Proveniência:(De \textunderscore bastilha\textunderscore )}
\end{itemize}
Torreão, o mesmo que \textunderscore cubelo\textunderscore .
\section{Bastimento}
\begin{itemize}
\item {Grp. gram.:m.}
\end{itemize}
(V.abastecimento). Cf. Vieira, VI, 109.
\section{Bastinha}
\begin{itemize}
\item {Grp. gram.:adj. f.}
\end{itemize}
\begin{itemize}
\item {Utilização:Prov.}
\end{itemize}
\begin{itemize}
\item {Utilização:alent.}
\end{itemize}
Diz-se da mulher, pouco asseada ou porca. Cf. \textunderscore Rev. Lus.\textunderscore , XV, 104.
\section{Bastio}
\begin{itemize}
\item {Grp. gram.:m.}
\end{itemize}
\begin{itemize}
\item {Utilização:Prov.}
\end{itemize}
\begin{itemize}
\item {Utilização:alent.}
\end{itemize}
\begin{itemize}
\item {Proveniência:(De \textunderscore basto\textunderscore )}
\end{itemize}
Moita espêssa; agglomeração de árvores ou plantas.
\section{Bastiões}
\begin{itemize}
\item {Grp. gram.:m. pl.}
\end{itemize}
(V.bastiães)
\section{Bastir}
\begin{itemize}
\item {Grp. gram.:v. t.}
\end{itemize}
\begin{itemize}
\item {Utilização:Ant.}
\end{itemize}
Armar (o pano de um guarda-chuva).
Edificar.
Tornar forte.
Supportar.
Formar com pêlo (o chapéu).
(B. lat. \textunderscore bastire\textunderscore )
\section{Bastissagem}
\begin{itemize}
\item {Grp. gram.:f.}
\end{itemize}
\begin{itemize}
\item {Proveniência:(De \textunderscore bastir\textunderscore )}
\end{itemize}
Acto de preparar o pêlo, para se formar o chapéu.
\section{Basto}
\begin{itemize}
\item {Grp. gram.:m.}
\end{itemize}
Ás de paus, no jôgo do voltarete.
(Cast. \textunderscore basto\textunderscore )
\section{Basto}
\begin{itemize}
\item {Grp. gram.:adj.}
\end{itemize}
\begin{itemize}
\item {Proveniência:(Do lat. \textunderscore vastus\textunderscore ?)}
\end{itemize}
Numeroso.
Compacto; espêsso: \textunderscore cabelleira basta\textunderscore .
\section{Basto}
\begin{itemize}
\item {Grp. gram.:m.}
\end{itemize}
\begin{itemize}
\item {Utilização:Bras}
\end{itemize}
Espécie de lombilho.
(Cp. \textunderscore baste\textunderscore )
\section{Bastonada}
\begin{itemize}
\item {Grp. gram.:f.}
\end{itemize}
Pancada com bastão.
\section{Bastonário}
\begin{itemize}
\item {Grp. gram.:m.}
\end{itemize}
\begin{itemize}
\item {Utilização:Ant.}
\end{itemize}
Bedel.
(B. lat. \textunderscore bastonarius\textunderscore )
\section{Bastonete}
\begin{itemize}
\item {Grp. gram.:m.}
\end{itemize}
Pequeno bastão, varinha.
Bacillo alongado, mycelliforme, articulado.
\section{Bastos}
\begin{itemize}
\item {Grp. gram.:m. pl.}
\end{itemize}
\begin{itemize}
\item {Utilização:Gír.}
\end{itemize}
Rede, que faz parte do saco, nos apparelhos da pesca da sardinha.
Dedos.
(Pl. de \textunderscore basto\textunderscore ^2)
\section{Bastura}
\begin{itemize}
\item {Grp. gram.:f.}
\end{itemize}
(V.bastidão)
\section{Basutos}
\begin{itemize}
\item {Grp. gram.:m. pl.}
\end{itemize}
O mesmo que \textunderscore bassutos\textunderscore .
\section{Bata}
\begin{itemize}
\item {Grp. gram.:f.}
\end{itemize}
\begin{itemize}
\item {Utilização:Ant.}
\end{itemize}
\begin{itemize}
\item {Utilização:Bras}
\end{itemize}
Vestido inteiriço de mulher.
Chambre para homem.
Partes acolchoadas e parallelas do lombilho.
\section{Bata}
\begin{itemize}
\item {Grp. gram.:f.}
\end{itemize}
\begin{itemize}
\item {Utilização:Gír.}
\end{itemize}
Mão.
\section{Bata}
\begin{itemize}
\item {Grp. gram.:f.}
\end{itemize}
O mesmo que \textunderscore bate\textunderscore ^1.
\section{Bataforma}
\begin{itemize}
\item {Grp. gram.:f.}
\end{itemize}
\begin{itemize}
\item {Utilização:T. de Pinhel}
\end{itemize}
Parede de campo ou vinha.
(Relaciona-se com \textunderscore plataforma\textunderscore ?)
\section{Batagem}
\begin{itemize}
\item {Grp. gram.:f.}
\end{itemize}
\begin{itemize}
\item {Proveniência:(De \textunderscore bater\textunderscore )}
\end{itemize}
Acção de bater os casulos da seda, para enredar os fios destramados.
\section{Batalha}
\begin{itemize}
\item {Grp. gram.:f.}
\end{itemize}
Combate entre exércitos.
Discussão; matéria de discussão.
Principal argumento.
Grande esforço, luta: \textunderscore as batalhas da vida\textunderscore .
Fileira de tropa, disposta para combate.
Espécie de jôgo de cartas.
Árvore silvestre do Brasil.
(B. lat. \textunderscore batallia\textunderscore )
\section{Batalhação}
\begin{itemize}
\item {Grp. gram.:f.}
\end{itemize}
\begin{itemize}
\item {Utilização:Fam.}
\end{itemize}
\begin{itemize}
\item {Proveniência:(De \textunderscore batalhar\textunderscore )}
\end{itemize}
Persistência de esforços; porfia.
\section{Batalhador}
\begin{itemize}
\item {Grp. gram.:m.}
\end{itemize}
\begin{itemize}
\item {Grp. gram.:Adj.}
\end{itemize}
Aquelle que batalha.
Defensor corajoso.
Que batalha.
\section{Batalhante}
\begin{itemize}
\item {Grp. gram.:adj.}
\end{itemize}
\begin{itemize}
\item {Utilização:Heráld.}
\end{itemize}
Que batalha.
Diz-se do leão rompente, quando no campo do escudo se figura arremetendo contra outro leão.
\section{Batalhão}
\begin{itemize}
\item {Grp. gram.:m.}
\end{itemize}
\begin{itemize}
\item {Utilização:Fam.}
\end{itemize}
\begin{itemize}
\item {Utilização:Prov.}
\end{itemize}
\begin{itemize}
\item {Proveniência:(De \textunderscore batalha\textunderscore )}
\end{itemize}
Corpo de infantaria, que faz parte de um regimento, e que se subdivide em companhias.
Grande quantidade de pessôas.
Série de leiras parallelas. (Colhido em Turquel).
\section{Batalhar}
\begin{itemize}
\item {Grp. gram.:v. i.}
\end{itemize}
Dar batalha; combater; pelejar.
Discutir porfiadamente.
Esforçar-se.
\section{Batalho}
\begin{itemize}
\item {Grp. gram.:m.}
\end{itemize}
\begin{itemize}
\item {Utilização:Prov.}
\end{itemize}
O mesmo que \textunderscore bilharda\textunderscore .
\section{Batangas}
\begin{itemize}
\item {Grp. gram.:m.}
\end{itemize}
Variedade de cafezeiro.
\section{Batão}
\begin{itemize}
\item {Grp. gram.:m.}
\end{itemize}
\begin{itemize}
\item {Proveniência:(De \textunderscore bater\textunderscore ?)}
\end{itemize}
Passo de dança antiga, em que se furtava com o pé o lugar do outro.
\section{Batão}
\begin{itemize}
\item {Grp. gram.:m.}
\end{itemize}
\begin{itemize}
\item {Utilização:Ant.}
\end{itemize}
Ágio da moéda, no Pegu e em Malaca.
\section{Batarda}
\begin{itemize}
\item {Grp. gram.:f.}
\end{itemize}
\begin{itemize}
\item {Utilização:Prov.}
\end{itemize}
O mesmo que \textunderscore abetarda\textunderscore .
\section{Bataque}
\begin{itemize}
\item {Grp. gram.:m.}
\end{itemize}
Língua de alguns archipélagos austraes da Oceânia.
\section{Bataréu}
\begin{itemize}
\item {Grp. gram.:m.}
\end{itemize}
\begin{itemize}
\item {Utilização:Prov.}
\end{itemize}
\begin{itemize}
\item {Utilização:beir.}
\end{itemize}
O mesmo que \textunderscore botaréu\textunderscore .
\section{Bataréu}
\begin{itemize}
\item {Grp. gram.:m.}
\end{itemize}
\begin{itemize}
\item {Utilização:Prov.}
\end{itemize}
\begin{itemize}
\item {Utilização:alg.}
\end{itemize}
Designação depreciativa, hoje desusada, de cada um dos antigos batalhões da Guarda Nacional.
\section{Bataria}
\begin{itemize}
\item {Grp. gram.:f.}
\end{itemize}
(Fórma antiga de \textunderscore bateria\textunderscore  e mais portuguesa do que esta)
\section{Batás}
\begin{itemize}
\item {Grp. gram.:m. pl.}
\end{itemize}
Povo antigo da Malásia. Cf. \textunderscore Peregrinação\textunderscore , XXV.
\section{Batata}
\begin{itemize}
\item {Grp. gram.:f.}
\end{itemize}
\begin{itemize}
\item {Utilização:Fam.}
\end{itemize}
\begin{itemize}
\item {Grp. gram.:Pl.}
\end{itemize}
\begin{itemize}
\item {Utilização:Prov.}
\end{itemize}
\begin{itemize}
\item {Utilização:beir.}
\end{itemize}
Planta solânea, de raízes tuberculosas e comestíveis.
Cada tubérculo desta planta.
Tubérculo de outras plantas.
Nariz muito grosso.
Petas, mentirolas.
(Cp. cast. \textunderscore patata\textunderscore )
\section{Batatá}
\begin{itemize}
\item {Grp. gram.:m.}
\end{itemize}
\begin{itemize}
\item {Utilização:Bras}
\end{itemize}
Fruta de uma árvore sapotácea.
(Do \textunderscore tupi\textunderscore )
\section{Batata-atum}
\begin{itemize}
\item {Grp. gram.:f.}
\end{itemize}
\begin{itemize}
\item {Utilização:T. de Setúbal}
\end{itemize}
O mesmo que \textunderscore fun-fun-gá-gá\textunderscore .
\section{Batatada}
\begin{itemize}
\item {Grp. gram.:f.}
\end{itemize}
Grande porção de batatas.
Doce, feito de batatas.
\section{Batata-de-veado}
\begin{itemize}
\item {Grp. gram.:f.}
\end{itemize}
\begin{itemize}
\item {Utilização:Bras}
\end{itemize}
Espécie de mandioca.
\section{Batatal}
\begin{itemize}
\item {Grp. gram.:f.}
\end{itemize}
Terreno, em que crescem batatas.
\section{Batatão}
\begin{itemize}
\item {Grp. gram.:m.}
\end{itemize}
\begin{itemize}
\item {Utilização:Bras}
\end{itemize}
O mesmo que \textunderscore boitatá\textunderscore .
\section{Batateira}
\begin{itemize}
\item {Grp. gram.:f.}
\end{itemize}
\begin{itemize}
\item {Proveniência:(De \textunderscore batata\textunderscore )}
\end{itemize}
O mesmo que \textunderscore batata\textunderscore , (planta).
\section{Batateiral}
\begin{itemize}
\item {Grp. gram.:f.}
\end{itemize}
O mesmo que \textunderscore batatal\textunderscore .
\section{Batateiro}
\begin{itemize}
\item {Grp. gram.:m.}
\end{itemize}
\begin{itemize}
\item {Grp. gram.:Adj.}
\end{itemize}
\begin{itemize}
\item {Utilização:Pop.}
\end{itemize}
\begin{itemize}
\item {Utilização:Bras}
\end{itemize}
\begin{itemize}
\item {Utilização:Prov.}
\end{itemize}
\begin{itemize}
\item {Utilização:beir.}
\end{itemize}
O mesmo que \textunderscore batateira\textunderscore .
Que gosta muito de batatas.
Que fala incorrectamente ou pronuncía mal.
Que é useiro em dizer petas.
\section{Batatífago}
\begin{itemize}
\item {Grp. gram.:m.}
\end{itemize}
Que se alimenta de batatas. Cf. Filinto, V, 111 e 103.
(Voc. extravagante, mal formado de \textunderscore batata\textunderscore  e gr. \textunderscore phagein\textunderscore )
\section{Batatinha}
\begin{itemize}
\item {Grp. gram.:f.}
\end{itemize}
\begin{itemize}
\item {Proveniência:(De \textunderscore batata\textunderscore )}
\end{itemize}
Planta medicinal do Brasil.
Espécie de tubérculo ou galha, produzida por certos insectos nas raízes de algumas plantas.
\section{Batatíphago}
\begin{itemize}
\item {Grp. gram.:m.}
\end{itemize}
Que se alimenta de batatas. Cf. Filinto, V, 111 e 103.
(Voc. extravagante, mal formado de \textunderscore batata\textunderscore  e gr. \textunderscore phagein\textunderscore )
\section{Batatudo}
\begin{itemize}
\item {Grp. gram.:adj.}
\end{itemize}
\begin{itemize}
\item {Utilização:Fam.}
\end{itemize}
Grosso como a batata: \textunderscore nariz batatudo\textunderscore .
\section{Batauá}
\begin{itemize}
\item {Grp. gram.:m.}
\end{itemize}
O mesmo que \textunderscore patauá\textunderscore .
\section{Batávia}
\begin{itemize}
\item {Grp. gram.:f.}
\end{itemize}
\begin{itemize}
\item {Utilização:Ant.}
\end{itemize}
\begin{itemize}
\item {Proveniência:(De \textunderscore Batávia\textunderscore , n. ant. da Hollanda)}
\end{itemize}
Pano fino de linho, o mesmo que \textunderscore hollanda\textunderscore .
Espécie de tabaco.
\section{Batávico}
\begin{itemize}
\item {Grp. gram.:adj.}
\end{itemize}
Relativo á Batávia, ou á Hollanda.
\section{Batávio}
\begin{itemize}
\item {Grp. gram.:adj.}
\end{itemize}
O mesmo que \textunderscore batávico\textunderscore ; hollandês. Cf. \textunderscore Caramuru\textunderscore , XI, 38.
\section{Bate}
\begin{itemize}
\item {Grp. gram.:m.}
\end{itemize}
\begin{itemize}
\item {Utilização:Ant.}
\end{itemize}
\begin{itemize}
\item {Proveniência:(Do conc. \textunderscore bat\textunderscore )}
\end{itemize}
Arroz em casca.
Quantia malaia, equivalente a 40:000 cruzados. Cf. \textunderscore Peregrinação\textunderscore , XV.
\section{Bate}
\begin{itemize}
\item {Grp. gram.:m.}
\end{itemize}
\begin{itemize}
\item {Utilização:Prov.}
\end{itemize}
\begin{itemize}
\item {Utilização:minh.}
\end{itemize}
\begin{itemize}
\item {Proveniência:(De \textunderscore bater\textunderscore )}
\end{itemize}
O mesmo que \textunderscore pão-de-ló\textunderscore .
Rosca de pão-de-ló.
\section{Bataria}
\begin{itemize}
\item {Grp. gram.:f.}
\end{itemize}
\begin{itemize}
\item {Utilização:Phýs.}
\end{itemize}
\begin{itemize}
\item {Utilização:Ant.}
\end{itemize}
\begin{itemize}
\item {Utilização:Marcen.}
\end{itemize}
\begin{itemize}
\item {Utilização:Artilh.}
\end{itemize}
\begin{itemize}
\item {Utilização:Bras. do N}
\end{itemize}
\begin{itemize}
\item {Proveniência:(De \textunderscore bater\textunderscore )}
\end{itemize}
Luta, assalto.
Fileira de peças de artilharia.
O disparar das peças de artilharia.
Lugar abrigado, donde disparam as peças de artilharia.
Conjunto de apparelhos condensadores de electricidade, em communicação uns com os outros.
Apparelho, formado de chapas alternadas, de cobre e zinco.
Utensílios culinários de metal.
Conjunto de esperas, alinhadas no banco de carpinteiro, para encostar ou apertar a peça de madeira em que se trabalha.
Acção de bater.
Série de encaixes no tabuão do banco, onde se mete a espera, de encontro á qual se fixa a madeira, em que se trabalha.
Primeira unidade administrativa ou agrupamento de soldados, sob o commando de um capitão, e equivalente á companhia na infantaria, e ao esquadrão na cavallaria.
Processo, com que os seringueiros extráem o látex, por meio de uma ordem dupla de tigelinhas em cada seringueira.
\section{Batea}
\begin{itemize}
\item {Grp. gram.:f.}
\end{itemize}
(Domingos Vieira, lê \textunderscore bátea\textunderscore , Moraes lê \textunderscore batéa\textunderscore )(V.bateia)
\section{Bateada}
\begin{itemize}
\item {Grp. gram.:f.}
\end{itemize}
\begin{itemize}
\item {Proveniência:(De \textunderscore bateia\textunderscore )}
\end{itemize}
Porção de minério, contido numa bateia.
\section{Batear}
\begin{itemize}
\item {Grp. gram.:v. t.}
\end{itemize}
Lavar em bateia.
\section{Bate-barba}
\begin{itemize}
\item {Grp. gram.:f.}
\end{itemize}
\begin{itemize}
\item {Utilização:Prov.}
\end{itemize}
\begin{itemize}
\item {Utilização:alg.}
\end{itemize}
O mesmo que \textunderscore batibarba\textunderscore .
Discussão acalorada.
\section{Bate-bôca}
\begin{itemize}
\item {Grp. gram.:f.}
\end{itemize}
\begin{itemize}
\item {Utilização:Bras. do N}
\end{itemize}
Discussão violenta; altercação.
\section{Bate-chapéu}
\begin{itemize}
\item {Grp. gram.:m.}
\end{itemize}
Pequena abelha do Brasil.
\section{Bate-chinela}
\begin{itemize}
\item {Grp. gram.:m.}
\end{itemize}
\begin{itemize}
\item {Utilização:Bras. do N}
\end{itemize}
Bailarico; dança de gente ordinária.
\section{Bate-collas}
\begin{itemize}
\item {Grp. gram.:m.}
\end{itemize}
Apparelho, para encorporar nos líquidos fermentados a colla ou goma que os deve clarificar.
\section{Batecu}
\begin{itemize}
\item {fónica:bá}
\end{itemize}
\begin{itemize}
\item {Grp. gram.:m.}
\end{itemize}
\begin{itemize}
\item {Proveniência:(De \textunderscore bater\textunderscore  + \textunderscore cu\textunderscore )}
\end{itemize}
Pancada com as nádegas, caindo.
Pancada com a mão nas nádegas: \textunderscore deu-lhe dois batecus\textunderscore .
\section{Batecum}
\begin{itemize}
\item {Grp. gram.:m.}
\end{itemize}
\begin{itemize}
\item {Utilização:Bras}
\end{itemize}
Barulho de sapateados e palmas.
Barulho de pancadas com pés, martelo, etc.
Pulsação forte do coração ou de artérias.
\section{Batedeira}
\begin{itemize}
\item {Grp. gram.:f.}
\end{itemize}
\begin{itemize}
\item {Proveniência:(De \textunderscore bater\textunderscore )}
\end{itemize}
Balde de madeira, em que se bate o leite para fazer manteiga; barata^2.
Apparelho, para agglomerar os glóbulos da nata.
\section{Batedela}
\begin{itemize}
\item {Grp. gram.:f.}
\end{itemize}
Acção de \textunderscore bater\textunderscore .
\section{Batedoiro}
\begin{itemize}
\item {Grp. gram.:m.}
\end{itemize}
\begin{itemize}
\item {Proveniência:(De \textunderscore bater\textunderscore )}
\end{itemize}
Pedra, em que as lavadeiras batem a roupa, lavando-a.
Lugar, em que se batem ou se sacodem quaesquer objectos.
\section{Batedor}
\begin{itemize}
\item {Grp. gram.:m.}
\end{itemize}
\begin{itemize}
\item {Utilização:Bras}
\end{itemize}
\begin{itemize}
\item {Utilização:Ant.}
\end{itemize}
\begin{itemize}
\item {Utilização:Fig.}
\end{itemize}
\begin{itemize}
\item {Utilização:Espir.}
\end{itemize}
\begin{itemize}
\item {Utilização:Bras. do N}
\end{itemize}
\begin{itemize}
\item {Proveniência:(De \textunderscore bater\textunderscore )}
\end{itemize}
Aquelle ou aquillo que bate.
Cunhador (de moéda).
Cada um dos soldados, que vão adeante de um corpo de tropas, explorando terreno.
Aquelle que levanta caça, para que esta vá têr aonde a esperam.
Cada um dos soldados ou criados, fardados e montados, que precedem a carruagem de pessôas reaes.
Instrumento, em que se lava o grão de fécula.
Instrumento, para debulhar milho, espécie de mangual.
Aldrava grande, para bater nas portas.
O mesmo que \textunderscore precursor\textunderscore .
Diz-se do espirito, que se revela por pancadas ou ruidos de vária espécie.
Lugar, onde se reúne o gado, acossado pelas môscas.
\section{Batedouro}
\begin{itemize}
\item {Grp. gram.:m.}
\end{itemize}
\begin{itemize}
\item {Proveniência:(De \textunderscore bater\textunderscore )}
\end{itemize}
Pedra, em que as lavadeiras batem a roupa, lavando-a.
Lugar, em que se batem ou se sacodem quaesquer objectos.
\section{Batedura}
\begin{itemize}
\item {Grp. gram.:f.}
\end{itemize}
Acção de \textunderscore bater\textunderscore .
\section{Bate-estacas}
\begin{itemize}
\item {Grp. gram.:m.}
\end{itemize}
Apparelho, para cravar estacas.
\section{Bate-fôlha}
\begin{itemize}
\item {Grp. gram.:m.}
\end{itemize}
\begin{itemize}
\item {Proveniência:(De \textunderscore bater\textunderscore  + \textunderscore fôlha\textunderscore )}
\end{itemize}
Aquelle que reduz a fôlhas muito delgadas um metal malleável, para doiradura e trabalhos análogos.
Latoeiro; funileiro.
\section{Bátega}
\begin{itemize}
\item {Grp. gram.:f.}
\end{itemize}
\begin{itemize}
\item {Grp. gram.:Pl.}
\end{itemize}
\begin{itemize}
\item {Utilização:Mús.}
\end{itemize}
\begin{itemize}
\item {Proveniência:(Do ár. \textunderscore batia\textunderscore )}
\end{itemize}
Antiga bacia de metal.
Porção de líquido, que essa bacia comportava.
Pancada (de água); chuva grossa.
O mesmo que [[pratos|prato]], (no port. do séc. XVI e XVII).
\section{Bateia}
\begin{itemize}
\item {Grp. gram.:f.}
\end{itemize}
Vaso, em que se lavam as areias auríferas.
(Or. incerta. Se fôsse \textunderscore bátea\textunderscore , como alguns pretendem, viria do ár. \textunderscore bátia\textunderscore , e seria contr. de \textunderscore bátega\textunderscore )
\section{Bateira}
\begin{itemize}
\item {Grp. gram.:f.}
\end{itemize}
Pequena embarcação sem quilha.
(Do mesmo rad. que \textunderscore batel\textunderscore )
\section{Batel}
\begin{itemize}
\item {Grp. gram.:m.}
\end{itemize}
\begin{itemize}
\item {Proveniência:(Lat. \textunderscore batellum\textunderscore )}
\end{itemize}
Barco pequeno, canôa.
\section{Batela}
\begin{itemize}
\item {Grp. gram.:f.}
\end{itemize}
\begin{itemize}
\item {Proveniência:(De \textunderscore batel\textunderscore )}
\end{itemize}
Barco chato e pequeno, usado no norte do Minho.
\section{Batelada}
\begin{itemize}
\item {Grp. gram.:f.}
\end{itemize}
Carga de um batel.
Grande quantidade (de objectos): \textunderscore uma batelada de livros\textunderscore .
\section{Batelão}
\begin{itemize}
\item {Grp. gram.:m.}
\end{itemize}
\begin{itemize}
\item {Utilização:Bras}
\end{itemize}
\begin{itemize}
\item {Proveniência:(De \textunderscore batel\textunderscore )}
\end{itemize}
Grande barca, para transporte de objectos muito pesados: \textunderscore um batelão de draga\textunderscore .
Canôa curta, de grande bôca e pontal.
\section{Bateleiro}
\begin{itemize}
\item {Grp. gram.:m.}
\end{itemize}
Aquelle que dirige batel.
\section{Batelo}
\begin{itemize}
\item {Grp. gram.:m.}
\end{itemize}
\begin{itemize}
\item {Utilização:T. do Ribatejo}
\end{itemize}
Apparelho, para tirar água dos poços.
\section{Batente}
\begin{itemize}
\item {Grp. gram.:m.}
\end{itemize}
\begin{itemize}
\item {Utilização:Carp.}
\end{itemize}
\begin{itemize}
\item {Utilização:Carp.}
\end{itemize}
\begin{itemize}
\item {Proveniência:(De \textunderscore bater\textunderscore )}
\end{itemize}
Ombreira, em que bate a porta, ao fechar-se.
Meia porta, em que bate a outra meia, ao fechar-se.
Aldrava.
Lugar, onde a maré bate e se quebra.
Tabuado de pinho, de duas pollegadas de grossura.
Régua ou fasquia, com que se guarnece a extremidade interior de uma meia porta, para se tapar a linha de juncção com a outra meia porta.
\section{Bate-orelha}
\begin{itemize}
\item {Grp. gram.:m.}
\end{itemize}
\begin{itemize}
\item {Utilização:Fam.}
\end{itemize}
Burro.
Homem estúpido.
\section{Batepandé}
\begin{itemize}
\item {Grp. gram.:m.}
\end{itemize}
\begin{itemize}
\item {Utilização:Bras}
\end{itemize}
Jôgo de cabra-cega.
\section{Bate-prego}
\begin{itemize}
\item {Grp. gram.:m.}
\end{itemize}
\begin{itemize}
\item {Utilização:Bras}
\end{itemize}
Marteladas, que servem de sinal, para que os operários suspendam o trabalho.
\section{Bater}
\begin{itemize}
\item {Grp. gram.:v. t.}
\end{itemize}
\begin{itemize}
\item {Grp. gram.:V. i.}
\end{itemize}
\begin{itemize}
\item {Proveniência:(Do b. lat. \textunderscore battere\textunderscore )}
\end{itemize}
Dar pancada ou pancadas em.
Abater, deminuir, o volume de.
Cunhar (moéda).
Vencer, derrotar: \textunderscore bater o inimigo\textunderscore .
Agitar (as asas)
Percorrer: \textunderscore bater o mato\textunderscore .
Dar com (o pé, as palmas das mãos, etc.): \textunderscore bater palmas\textunderscore .
Dar pancada ou pancadas: \textunderscore bater na sogra\textunderscore .
Dirigir-se, fixar a mira: \textunderscore o cocheiro bateu para Sintra\textunderscore .
Mover-se (em retirada)
Ir de encontro: \textunderscore bater contra a parede\textunderscore .
Ir com pressa.
\section{Bateria}
\begin{itemize}
\item {Grp. gram.:f.}
\end{itemize}
\begin{itemize}
\item {Utilização:Phýs.}
\end{itemize}
\begin{itemize}
\item {Utilização:Ant.}
\end{itemize}
\begin{itemize}
\item {Utilização:Marcen.}
\end{itemize}
\begin{itemize}
\item {Utilização:Artilh.}
\end{itemize}
\begin{itemize}
\item {Utilização:Bras. do N}
\end{itemize}
\begin{itemize}
\item {Proveniência:(De \textunderscore bater\textunderscore )}
\end{itemize}
Luta, assalto.
Fileira de peças de artilharia.
O disparar das peças de artilharia.
Lugar abrigado, donde disparam as peças de artilharia.
Conjunto de apparelhos condensadores de electricidade, em communicação uns com os outros.
Apparelho, formado de chapas alternadas, de cobre e zinco.
Utensílios culinários de metal.
Conjunto de esperas, alinhadas no banco de carpinteiro, para encostar ou apertar a peça de madeira em que se trabalha.
Acção de bater.
Série de encaixes no tabuão do banco, onde se mete a espera, de encontro á qual se fixa a madeira, em que se trabalha.
Primeira unidade administrativa ou agrupamento de soldados, sob o commando de um capitão, e equivalente á companhia na infantaria, e ao esquadrão na cavallaria.
Processo, com que os seringueiros extráem o látex, por meio de uma ordem dupla de tigelinhas em cada seringueira.
\section{Baterola}
\begin{itemize}
\item {Grp. gram.:f.}
\end{itemize}
\begin{itemize}
\item {Utilização:Ant.}
\end{itemize}
O mesmo que \textunderscore remo\textunderscore ?:«\textunderscore se forem grandes esses tiros, nem o escalmo nem a baterola os soffrerão\textunderscore ». Fern. de Oliv., \textunderscore Arte da guerra\textunderscore , 47, v.^o.
\section{Bate-sornas}
\begin{itemize}
\item {Grp. gram.:m.}
\end{itemize}
\begin{itemize}
\item {Utilização:Gír. de Lisbôa.}
\end{itemize}
Ladrão nocturno.
Gatuno, que explora as algibeiras dos ingênuos que adormecem nos bancos das praças públicas.
\section{Bathographia}
\begin{itemize}
\item {Grp. gram.:f.}
\end{itemize}
\begin{itemize}
\item {Utilização:Geol.}
\end{itemize}
\begin{itemize}
\item {Proveniência:(Do gr. \textunderscore bathos\textunderscore  + \textunderscore graphein\textunderscore )}
\end{itemize}
Estudo das depressões da superfície sólida do globo.
\section{Bathómetro}
\begin{itemize}
\item {Grp. gram.:m.}
\end{itemize}
(V.bathýmetro)
\section{Bathymetria}
\begin{itemize}
\item {Grp. gram.:f.}
\end{itemize}
Medida das profundezas do mar.
(Cp. \textunderscore bathýmetro\textunderscore )
\section{Bathymétrico}
\begin{itemize}
\item {Grp. gram.:adj.}
\end{itemize}
Relativo á \textunderscore bathymetria\textunderscore .
\section{Bathýmetro}
\begin{itemize}
\item {Grp. gram.:m.}
\end{itemize}
\begin{itemize}
\item {Proveniência:(Do gr. \textunderscore bathus\textunderscore  + \textunderscore metron\textunderscore )}
\end{itemize}
Instrumento, para indicar a profundidade do mar, sem emprêgo da linha de sonda.
\section{Batibanda}
\begin{itemize}
\item {Grp. gram.:f.}
\end{itemize}
(V.platibanda)
\section{Batibarba}
\begin{itemize}
\item {Grp. gram.:f.}
\end{itemize}
\begin{itemize}
\item {Proveniência:(De \textunderscore bater\textunderscore  + \textunderscore barba\textunderscore )}
\end{itemize}
Pancada com a mão por baixo da barba.
Reprehensão rude.
\section{Batibarbo}
\begin{itemize}
\item {Grp. gram.:m.}
\end{itemize}
\begin{itemize}
\item {Utilização:Prov.}
\end{itemize}
\begin{itemize}
\item {Utilização:trasm.}
\end{itemize}
O mesmo que \textunderscore batibarba\textunderscore .
\section{Batição}
\begin{itemize}
\item {Grp. gram.:f.}
\end{itemize}
\begin{itemize}
\item {Utilização:Bras. do N}
\end{itemize}
\begin{itemize}
\item {Proveniência:(De \textunderscore bater\textunderscore )}
\end{itemize}
Maneira de pescar tartarugas nos lagos, batendo a água com varas, para que os amphíbios procurem a beira, onde os frecheiros as esperam.
\section{Batida}
\begin{itemize}
\item {Grp. gram.:f.}
\end{itemize}
\begin{itemize}
\item {Grp. gram.:Loc. adv.}
\end{itemize}
Acção de \textunderscore bater\textunderscore .
Censura.
Montaria: \textunderscore uma batida ás lebres\textunderscore .
\textunderscore De batida\textunderscore , á pressa, precipitadamente.
\section{Batidela}
\begin{itemize}
\item {Grp. gram.:f.}
\end{itemize}
\begin{itemize}
\item {Utilização:Fam.}
\end{itemize}
Acto de \textunderscore bater\textunderscore  ou sacudir: \textunderscore batidela dos tapêtes\textunderscore .
\section{Batido}
\begin{itemize}
\item {Grp. gram.:adj.}
\end{itemize}
\begin{itemize}
\item {Proveniência:(De \textunderscore bater\textunderscore )}
\end{itemize}
Vulgar, trivial: \textunderscore o necrológio é gênero muito batido\textunderscore .
Degastado, cotiado: \textunderscore um fato muito batido\textunderscore .
\section{Batilhar}
\begin{itemize}
\item {Grp. gram.:v. i.}
\end{itemize}
Bater água mansamente. Cf. Eça, in \textunderscore Rev. Occid.\textunderscore , I, 74.
\section{Batim}
\begin{itemize}
\item {Grp. gram.:m.}
\end{itemize}
\begin{itemize}
\item {Utilização:Ant.}
\end{itemize}
Espécie de barco. Cf. \textunderscore Ethiopia Or.\textunderscore , II, 208.
\section{Batimento}
\begin{itemize}
\item {Grp. gram.:m.}
\end{itemize}
\begin{itemize}
\item {Utilização:Des.}
\end{itemize}
\begin{itemize}
\item {Proveniência:(De \textunderscore bater\textunderscore )}
\end{itemize}
O mesmo que \textunderscore embate\textunderscore .
\section{Batimetria}
\begin{itemize}
\item {Grp. gram.:f.}
\end{itemize}
Medida das profundezas do mar.
(Cp. \textunderscore bathýmetro\textunderscore )
\section{Batimétrico}
\begin{itemize}
\item {Grp. gram.:adj.}
\end{itemize}
Relativo á \textunderscore batimetria\textunderscore .
\section{Batímetro}
\begin{itemize}
\item {Grp. gram.:m.}
\end{itemize}
\begin{itemize}
\item {Proveniência:(Do gr. \textunderscore bathus\textunderscore  + \textunderscore metron\textunderscore )}
\end{itemize}
Instrumento, para indicar a profundidade do mar, sem emprêgo da linha de sonda.
\section{Batina}
\begin{itemize}
\item {Grp. gram.:f.}
\end{itemize}
Vestuário talar dos abbades.
Vestido talar do padre, dos estudantes da Universidade e de outras escolas.
Abbatina.
(Por \textunderscore abbatina\textunderscore , do lat. \textunderscore abbas\textunderscore , \textunderscore abbatis\textunderscore )
\section{Batinga}
\begin{itemize}
\item {Grp. gram.:f.}
\end{itemize}
Árvore myrtácea do Brasil.
\section{Batinguacá}
\begin{itemize}
\item {Grp. gram.:f.}
\end{itemize}
Árvore do Brasil.
\section{Batiputá}
\begin{itemize}
\item {Grp. gram.:m.}
\end{itemize}
Arbusto brasileiro, de semente oleosa e medicinal.
\section{Batisella}
\begin{itemize}
\item {fónica:sé}
\end{itemize}
\begin{itemize}
\item {Grp. gram.:m.}
\end{itemize}
\begin{itemize}
\item {Utilização:Pop.}
\end{itemize}
\begin{itemize}
\item {Proveniência:(De \textunderscore bater\textunderscore  + \textunderscore sella\textunderscore )}
\end{itemize}
Mau cavalleiro.
\section{Batissela}
\begin{itemize}
\item {Grp. gram.:m.}
\end{itemize}
\begin{itemize}
\item {Utilização:Pop.}
\end{itemize}
\begin{itemize}
\item {Proveniência:(De \textunderscore bater\textunderscore  + \textunderscore sella\textunderscore )}
\end{itemize}
Mau cavalleiro.
\section{Bato}
\begin{itemize}
\item {Grp. gram.:m.}
\end{itemize}
\begin{itemize}
\item {Proveniência:(De \textunderscore bater\textunderscore )}
\end{itemize}
Jôgo infantil, com pequenas pedras.
\section{Batoca}
\begin{itemize}
\item {Grp. gram.:f.}
\end{itemize}
(V.soquete)
\section{Batocaduras}
\begin{itemize}
\item {Grp. gram.:f. pl.}
\end{itemize}
\begin{itemize}
\item {Utilização:Ant.}
\end{itemize}
Chapas e cavilhas, que seguram as mesas das enxárcias contra o costado do navio.
(Cp. \textunderscore batoque\textunderscore )
\section{Batocar}
\begin{itemize}
\item {Grp. gram.:v. t.}
\end{itemize}
Fechar com batoque.
\section{Batoco}
\begin{itemize}
\item {fónica:tô}
\end{itemize}
\begin{itemize}
\item {Grp. gram.:m.}
\end{itemize}
\begin{itemize}
\item {Utilização:Prov.}
\end{itemize}
Ave, espécie de pica-pau ou trepadeira.
\section{Batoco}
\begin{itemize}
\item {fónica:tô}
\end{itemize}
\begin{itemize}
\item {Grp. gram.:m.}
\end{itemize}
\begin{itemize}
\item {Utilização:Prov.}
\end{itemize}
\begin{itemize}
\item {Utilização:trasm.}
\end{itemize}
Barranco; barrocal.
\section{Batografia}
\begin{itemize}
\item {Grp. gram.:f.}
\end{itemize}
\begin{itemize}
\item {Utilização:Geol.}
\end{itemize}
\begin{itemize}
\item {Proveniência:(Do gr. \textunderscore bathos\textunderscore  + \textunderscore graphein\textunderscore )}
\end{itemize}
Estudo das depressões da superfície sólida do globo.
\section{Batologia}
\begin{itemize}
\item {Grp. gram.:f.}
\end{itemize}
\begin{itemize}
\item {Proveniência:(Gr. \textunderscore battologia\textunderscore )}
\end{itemize}
Repetição inútil de um pensamento pelas mesmas palavras.
\section{Batologicamente}
\begin{itemize}
\item {Grp. gram.:adv.}
\end{itemize}
\begin{itemize}
\item {Proveniência:(De \textunderscore battológico\textunderscore )}
\end{itemize}
Com batologia.
\section{Batológico}
\begin{itemize}
\item {Grp. gram.:adj.}
\end{itemize}
Relativo á \textunderscore batologia\textunderscore .
\section{Batómetro}
\begin{itemize}
\item {Grp. gram.:m.}
\end{itemize}
(V.batímetro)
\section{Batoque}
\begin{itemize}
\item {Grp. gram.:m.}
\end{itemize}
\begin{itemize}
\item {Utilização:Fam.}
\end{itemize}
\begin{itemize}
\item {Utilização:Prov.}
\end{itemize}
\begin{itemize}
\item {Utilização:trasm.}
\end{itemize}
\begin{itemize}
\item {Utilização:Prov.}
\end{itemize}
\begin{itemize}
\item {Utilização:alent.}
\end{itemize}
\begin{itemize}
\item {Proveniência:(De \textunderscore bater\textunderscore ?)}
\end{itemize}
Rôlha grossa, com que se tapa o orifício que há na parte superior do bojo da pipa ou do tonel.
O mesmo orifício.
Homem atarracado.
Pequeno pau, aguçado de ambos os lados, e que serve para um jôgo de rapazes.
O mesmo que \textunderscore solavanco\textunderscore .
\section{Batoqueira}
\begin{itemize}
\item {Grp. gram.:f.}
\end{itemize}
\begin{itemize}
\item {Utilização:ant.}
\end{itemize}
\begin{itemize}
\item {Utilização:Fam.}
\end{itemize}
\begin{itemize}
\item {Proveniência:(De \textunderscore batoque\textunderscore )}
\end{itemize}
Orifício, correspondente ao batoque (rôlha).
Casa esconsa e pouco decente.
\section{Batoqueiro}
\begin{itemize}
\item {Grp. gram.:m.}
\end{itemize}
\begin{itemize}
\item {Utilização:Prov.}
\end{itemize}
\begin{itemize}
\item {Utilização:dur.}
\end{itemize}
\begin{itemize}
\item {Proveniência:(De \textunderscore batoque\textunderscore )}
\end{itemize}
Tanoeiro, que acompanhava as pipas, para qualquer concêrto eventual.
\section{Batoré}
\begin{itemize}
\item {Grp. gram.:adj.}
\end{itemize}
\begin{itemize}
\item {Utilização:Bras. do N}
\end{itemize}
O mesmo que \textunderscore baé\textunderscore ^2.
\section{Batorelha}
\begin{itemize}
\item {fónica:torê}
\end{itemize}
\begin{itemize}
\item {Grp. gram.:m.}
\end{itemize}
O mesmo que \textunderscore bate-orelha\textunderscore .
\section{Batoréu}
\begin{itemize}
\item {Grp. gram.:m.}
\end{itemize}
O mesmo que \textunderscore botaréu\textunderscore .
\section{Batota}
\begin{itemize}
\item {Grp. gram.:f.}
\end{itemize}
Trapaça ao jôgo.
Jôgo de azar.
Casa de jôgo.
Lôgro, burla.
(Cp. cast. \textunderscore malute\textunderscore , candonga)
\section{Batota}
\begin{itemize}
\item {Grp. gram.:f.}
\end{itemize}
Peixe marítimo do Brasil.
\section{Batotar}
\begin{itemize}
\item {Grp. gram.:v. i.}
\end{itemize}
O mesmo que \textunderscore batotear\textunderscore .
\section{Batotear}
\begin{itemize}
\item {Grp. gram.:v. i.}
\end{itemize}
Fazer batota.
Jogar batota.
\section{Batoteiro}
\begin{itemize}
\item {Grp. gram.:m.}
\end{itemize}
Aquelle que faz batota.
Aquelle que frequenta muito os jogos de azar.
\section{Batourar}
\begin{itemize}
\item {Grp. gram.:v. i.}
\end{itemize}
\begin{itemize}
\item {Utilização:Prov.}
\end{itemize}
\begin{itemize}
\item {Utilização:minh.}
\end{itemize}
Dar pancadas repetidas; martelar.
\section{Batráchios}
\begin{itemize}
\item {fónica:qui}
\end{itemize}
\begin{itemize}
\item {Grp. gram.:m. pl.}
\end{itemize}
O mesmo que \textunderscore batrácios\textunderscore .
\section{Batrachoide}
\begin{itemize}
\item {fónica:coi}
\end{itemize}
\begin{itemize}
\item {Grp. gram.:adj.}
\end{itemize}
\begin{itemize}
\item {Grp. gram.:Pl.}
\end{itemize}
\begin{itemize}
\item {Proveniência:(Do gr. \textunderscore batrakhos\textunderscore  + \textunderscore eidos\textunderscore )}
\end{itemize}
Relativo á ran.
Gênero de peixes, parecidos aos embryões das rans.
\section{Batrachóphago}
\begin{itemize}
\item {fónica:có}
\end{itemize}
\begin{itemize}
\item {Grp. gram.:m. e adj.}
\end{itemize}
\begin{itemize}
\item {Proveniência:(Do gr. \textunderscore batrakhos\textunderscore  + \textunderscore phagein\textunderscore )}
\end{itemize}
O que come rans.
\section{Batrachospérmeas}
\begin{itemize}
\item {fónica:cos}
\end{itemize}
\begin{itemize}
\item {Grp. gram.:f. pl.}
\end{itemize}
O mesmo que \textunderscore hydróphytas\textunderscore .
\section{Batrácios}
\begin{itemize}
\item {Grp. gram.:m. pl.}
\end{itemize}
\begin{itemize}
\item {Proveniência:(Do gr. \textunderscore batrakhos\textunderscore )}
\end{itemize}
Animaes vertebrados, da classe dos reptis, e de organização análoga á da ran.
\section{Batracófago}
\begin{itemize}
\item {Grp. gram.:m. e adj.}
\end{itemize}
\begin{itemize}
\item {Proveniência:(Do gr. \textunderscore batrakhos\textunderscore  + \textunderscore phagein\textunderscore )}
\end{itemize}
O que come rans.
\section{Batracoide}
\begin{itemize}
\item {Grp. gram.:adj.}
\end{itemize}
\begin{itemize}
\item {Grp. gram.:Pl.}
\end{itemize}
\begin{itemize}
\item {Proveniência:(Do gr. \textunderscore batrakhos\textunderscore  + \textunderscore eidos\textunderscore )}
\end{itemize}
Relativo á ran.
Gênero de peixes, parecidos aos embryões das rans.
\section{Batracospérmeas}
\begin{itemize}
\item {Grp. gram.:f. pl.}
\end{itemize}
O mesmo que \textunderscore hydróphytas\textunderscore .
\section{Batráquios}
\begin{itemize}
\item {Grp. gram.:m. pl.}
\end{itemize}
O mesmo que \textunderscore batrácios\textunderscore .
\section{Battologia}
\begin{itemize}
\item {Grp. gram.:f.}
\end{itemize}
\begin{itemize}
\item {Proveniência:(Gr. \textunderscore battologia\textunderscore )}
\end{itemize}
Repetição inútil de um pensamento pelas mesmas palavras.
\section{Battologicamente}
\begin{itemize}
\item {Grp. gram.:adv.}
\end{itemize}
\begin{itemize}
\item {Proveniência:(De \textunderscore battológico\textunderscore )}
\end{itemize}
Com battologia.
\section{Battológico}
\begin{itemize}
\item {Grp. gram.:adj.}
\end{itemize}
Relativo á \textunderscore battologia\textunderscore .
\section{Batucar}
\begin{itemize}
\item {Grp. gram.:v. i.}
\end{itemize}
Dançar o batuque.
Martelar, dar pancadas repetidas.
\section{Batudo}
\begin{itemize}
\item {Grp. gram.:adj.}
\end{itemize}
\begin{itemize}
\item {Utilização:Ant.}
\end{itemize}
\begin{itemize}
\item {Grp. gram.:Loc. adv.}
\end{itemize}
Batido.
\textunderscore A malho batudo\textunderscore , ao toque da campa.
\section{Batueira}
\begin{itemize}
\item {Grp. gram.:f.}
\end{itemize}
O mesmo que \textunderscore batuera\textunderscore .
\section{Batuera}
\begin{itemize}
\item {Grp. gram.:f.}
\end{itemize}
\begin{itemize}
\item {Utilização:Bras}
\end{itemize}
Maçaroca de milho, depois de esbagoada.
(Do tupi \textunderscore abatiuera\textunderscore )
\section{Batuja}
\begin{itemize}
\item {Grp. gram.:f.}
\end{itemize}
\begin{itemize}
\item {Utilização:Prov.}
\end{itemize}
\begin{itemize}
\item {Utilização:minh.}
\end{itemize}
O mesmo que \textunderscore botija\textunderscore .
\section{Batuque}
\begin{itemize}
\item {Grp. gram.:m.}
\end{itemize}
\begin{itemize}
\item {Utilização:Bras. do N}
\end{itemize}
\begin{itemize}
\item {Proveniência:(Do rad. de \textunderscore bater\textunderscore ?)}
\end{itemize}
Dança especial, entre os negros de Angola.
Baile do povo.
Acto de batucar, de martelar, de fazer bulha.
\section{Batuta}
\begin{itemize}
\item {Grp. gram.:f.}
\end{itemize}
\begin{itemize}
\item {Proveniência:(It. \textunderscore battuta\textunderscore )}
\end{itemize}
Bastão curto ou pequena vara, com que os regentes de orchestra marcam o andamento da música e indicam a entrada dos diversos instrumentos.
\section{Baú}
\begin{itemize}
\item {Grp. gram.:m.}
\end{itemize}
Caixa de madeira, ordinariamente revestida de coiro cru, e com tampa convexa.
(Alto al. médio \textunderscore behut\textunderscore )
\section{Bauaris}
\begin{itemize}
\item {Grp. gram.:m. pl.}
\end{itemize}
Indígenas do Brasil, nas margens do Juruá.
\section{Baudelaireano}
\begin{itemize}
\item {fónica:bó-de-lé}
\end{itemize}
\begin{itemize}
\item {Grp. gram.:adj.}
\end{itemize}
Relativo ao poéta Baudelaire.
\section{Bauhínia}
\begin{itemize}
\item {Grp. gram.:f.}
\end{itemize}
\begin{itemize}
\item {Proveniência:(De \textunderscore Bauhin\textunderscore , n. p.)}
\end{itemize}
Gênero de plantas leguminosas.
\section{Baul}
\begin{itemize}
\item {Grp. gram.:m.}
\end{itemize}
Fórma antiga de \textunderscore baú\textunderscore .
(B. lat. \textunderscore bahulum\textunderscore )
\section{Bauleiro}
\begin{itemize}
\item {fónica:ba-u}
\end{itemize}
\begin{itemize}
\item {Grp. gram.:m.}
\end{itemize}
\begin{itemize}
\item {Proveniência:(De \textunderscore bahul\textunderscore )}
\end{itemize}
Aquelle que fabríca ou vende baús.
\section{Baunilha}
\begin{itemize}
\item {Grp. gram.:f.}
\end{itemize}
\begin{itemize}
\item {Proveniência:(Do cast. \textunderscore vainilla\textunderscore )}
\end{itemize}
Planta trepadeira, ornamental e vulgar.
Orchídea, sarmentosa, originária da América.
Fruto dessa orchídea.
Licor, feito da essencia dêsse fruto.
\section{Baunilha-dos-jardins}
\begin{itemize}
\item {Grp. gram.:f.}
\end{itemize}
\begin{itemize}
\item {Utilização:Bras}
\end{itemize}
O mesmo que \textunderscore heliotrópio\textunderscore .
\section{Baunilhão}
\begin{itemize}
\item {Grp. gram.:m.}
\end{itemize}
\begin{itemize}
\item {Proveniência:(De \textunderscore baunilha\textunderscore )}
\end{itemize}
Planta, semelhante á baunilha, mas mais escura e menos aromática.
\section{Bautismo}
\begin{itemize}
\item {Grp. gram.:m.}
\end{itemize}
\begin{itemize}
\item {Utilização:Ant.}
\end{itemize}
O mesmo que \textunderscore baptismo\textunderscore .
\section{Bautizar}
\begin{itemize}
\item {Grp. gram.:v. t.}
\end{itemize}
\begin{itemize}
\item {Utilização:Prov.}
\end{itemize}
\begin{itemize}
\item {Utilização:Ant.}
\end{itemize}
O mesmo que \textunderscore baptizar\textunderscore .
\section{Bávaro}
\begin{itemize}
\item {Grp. gram.:m.}
\end{itemize}
\begin{itemize}
\item {Grp. gram.:Adj.}
\end{itemize}
Habitante da Baviera.
Relativo á Baviera.
\section{Bavina}
\begin{itemize}
\item {Grp. gram.:f.}
\end{itemize}
Espécie de vestal, consagrada desde a infância ao serviço das divindades gentílicas, na Índia portuguesa.
\section{Baxá}
\begin{itemize}
\item {Grp. gram.:m.}
\end{itemize}
O mesmo que \textunderscore paxá\textunderscore . Cf. Pant. de Aveiro, \textunderscore Itiner.\textunderscore , 9 v.^o, (2.^a ed.).
\section{Baxe! baxe!}
\begin{itemize}
\item {Grp. gram.:interj.}
\end{itemize}
\begin{itemize}
\item {Utilização:Prov.}
\end{itemize}
\begin{itemize}
\item {Utilização:trasm.}
\end{itemize}
Us. para chamar cachorrinhos, quando começam a entender.
\section{Baxete}
\begin{itemize}
\item {fónica:xê}
\end{itemize}
\begin{itemize}
\item {Grp. gram.:m.}
\end{itemize}
(V.baixete)
\section{Baxete}
\begin{itemize}
\item {fónica:xê}
\end{itemize}
\begin{itemize}
\item {Grp. gram.:m.}
\end{itemize}
\begin{itemize}
\item {Utilização:Bras. do N}
\end{itemize}
Rapadura pequena, (quadradinho de açúcar mascavo).
\section{Baxim}
\begin{itemize}
\item {Grp. gram.:m.}
\end{itemize}
\begin{itemize}
\item {Utilização:Ant.}
\end{itemize}
Espécie de embarcação.
(Cp. \textunderscore baixel\textunderscore ^1)
\section{Baxiúba}
\begin{itemize}
\item {Grp. gram.:f.}
\end{itemize}
Espécie de palmeira do Brasil.
\section{Baxo}
\textunderscore adj.\textunderscore  (e der.)
O mesmo que \textunderscore baixo\textunderscore , etc.
\section{Baxtera}
\begin{itemize}
\item {Grp. gram.:f.}
\end{itemize}
\begin{itemize}
\item {Proveniência:(De \textunderscore Baxter\textunderscore , n. p.)}
\end{itemize}
Planta brasileira.
\section{Bazar}
\begin{itemize}
\item {Grp. gram.:m.}
\end{itemize}
\begin{itemize}
\item {Proveniência:(T. ár., de or. persa)}
\end{itemize}
Mercado oriental.
Estabelecimento, em que se expõem e se vendem objectos antigos e raros.
Pavilhão, barraca provisória, em que há fazendas e objectos variádos, que se adjudicam por sorteio.
Grande centro de commércio; empório.
\section{Bazareiro}
\begin{itemize}
\item {Grp. gram.:m.}
\end{itemize}
Mercador de bazar.
\section{Bazaruco}
\begin{itemize}
\item {Grp. gram.:m.}
\end{itemize}
\begin{itemize}
\item {Utilização:Gír.}
\end{itemize}
Antiga moéda da Índia portuguesa.
Pataco.
\section{Bazarugo}
\begin{itemize}
\item {Grp. gram.:m.}
\end{itemize}
\begin{itemize}
\item {Utilização:T. da Bairrada}
\end{itemize}
Indivíduo muito gordo.
Bazulaque.
(Cp. \textunderscore bazaruco\textunderscore )
\section{Bazé}
\begin{itemize}
\item {Grp. gram.:m.}
\end{itemize}
\begin{itemize}
\item {Utilização:Bras. do N}
\end{itemize}
Tabaco ruim.
\section{Bazófia}
\begin{itemize}
\item {Grp. gram.:f.}
\end{itemize}
\begin{itemize}
\item {Grp. gram.:M.}
\end{itemize}
\begin{itemize}
\item {Utilização:Prov.}
\end{itemize}
Vaidade, prosápia, fanfarronice.
Guisado, feito com restos de comida.
Espécie de doce, o mesmo que \textunderscore farófia\textunderscore .
Aquelle que tem bazófia; fanfarrão.
(Cast. \textunderscore bazofia\textunderscore )
\section{Bazofiar}
\begin{itemize}
\item {Grp. gram.:v. i.}
\end{itemize}
Têr bazófia.
\section{Bazófio}
\begin{itemize}
\item {Grp. gram.:m.}
\end{itemize}
\begin{itemize}
\item {Utilização:Pop.}
\end{itemize}
\begin{itemize}
\item {Grp. gram.:Adj.}
\end{itemize}
Aquelle que tem bazófia.
Que tem bazófia.
Em que há bazófia:«\textunderscore palavrões bazófios\textunderscore ». Filinto, V, 14.
\section{Bazulaque}
\begin{itemize}
\item {Grp. gram.:m.}
\end{itemize}
\begin{itemize}
\item {Utilização:Bras}
\end{itemize}
Chanfana, guisado de figados e bofes.
Miudezas.
Cosmético.
Homem gordo e baixo.
Doce de côco ralado e mel.
\section{Bdélio}
\begin{itemize}
\item {Grp. gram.:m.}
\end{itemize}
\begin{itemize}
\item {Proveniência:(Lat. \textunderscore bdellium\textunderscore )}
\end{itemize}
Goma-resina do Oriente.
\section{Bdéllio}
\begin{itemize}
\item {Grp. gram.:m.}
\end{itemize}
\begin{itemize}
\item {Proveniência:(Lat. \textunderscore bdellium\textunderscore )}
\end{itemize}
Goma-resina do Oriente.
\section{Bdellómetro}
\begin{itemize}
\item {Grp. gram.:m.}
\end{itemize}
\begin{itemize}
\item {Proveniência:(Do gr. \textunderscore bdella\textunderscore  + \textunderscore metron\textunderscore )}
\end{itemize}
Instrumento, destinado a substituir as sanguesugas, permittindo o calcular-se o sangue extrahido e o poder-se accelerar ou retardar a emissão do sangue.
\section{Bdelómetro}
\begin{itemize}
\item {Grp. gram.:m.}
\end{itemize}
\begin{itemize}
\item {Proveniência:(Do gr. \textunderscore bdella\textunderscore  + \textunderscore metron\textunderscore )}
\end{itemize}
Instrumento, destinado a substituir as sanguesugas, permittindo o calcular-se o sangue extrahido e o poder-se accelerar ou retardar a emissão do sangue.
\section{Bé}
\begin{itemize}
\item {Utilização:t. onom.}
\end{itemize}
Designação da voz das ovelhas.
\section{Beata}
\begin{itemize}
\item {Grp. gram.:f.}
\end{itemize}
\begin{itemize}
\item {Utilização:Pop.}
\end{itemize}
Ponta de cigarro.
Moéda de 5 reis.
\section{Beata}
\begin{itemize}
\item {Grp. gram.:f.}
\end{itemize}
\begin{itemize}
\item {Utilização:Prov.}
\end{itemize}
\begin{itemize}
\item {Utilização:alent.}
\end{itemize}
Lebre.
\section{Beata}
\begin{itemize}
\item {Grp. gram.:f.}
\end{itemize}
\begin{itemize}
\item {Proveniência:(De \textunderscore beato\textunderscore )}
\end{itemize}
Mulher excessivamente devota, de exaggerados escrúpulos religiosos.
\section{Beatamente}
\begin{itemize}
\item {Grp. gram.:adv.}
\end{itemize}
De modo \textunderscore beato\textunderscore .
Á maneira dos beatos.
\section{Beatão}
\begin{itemize}
\item {Grp. gram.:m.}
\end{itemize}
Grande beato, com hypocrisia.
\section{Beataria}
\begin{itemize}
\item {Grp. gram.:f.}
\end{itemize}
Beatice.
Multidão de beatos ou beatas.
\section{Beatás}
\begin{itemize}
\item {Grp. gram.:m. pl.}
\end{itemize}
Designação genérica dos fêtos, na ilha de San-Thomé.
\section{Beateiro}
\begin{itemize}
\item {Grp. gram.:m.  e  adj.}
\end{itemize}
O que tem a convivência de beatos ou beatas.
\section{Beateiro}
\begin{itemize}
\item {Grp. gram.:m.}
\end{itemize}
\begin{itemize}
\item {Utilização:T. de Lisbôa}
\end{itemize}
\begin{itemize}
\item {Proveniência:(De \textunderscore beata\textunderscore ^1)}
\end{itemize}
Aquelle que percorre as ruas e as entradas dos cafés, á procura de pontas de cigarro e de charuto.
\section{Beatério}
\begin{itemize}
\item {Grp. gram.:m.}
\end{itemize}
Beatice.
Práticas de pessôas beatas.
Systema, partido, dessas pessôas.
Os beatos e as beatas.
\section{Beatice}
\begin{itemize}
\item {Grp. gram.:f.}
\end{itemize}
\begin{itemize}
\item {Proveniência:(De \textunderscore beato\textunderscore )}
\end{itemize}
Devoção fingida; hypocrisia religiosa.
\section{Beatificação}
\begin{itemize}
\item {Grp. gram.:f.}
\end{itemize}
Acto de \textunderscore beatificar\textunderscore .
\section{Beatificador}
\begin{itemize}
\item {Grp. gram.:m.}
\end{itemize}
Aquelle que beatifica.
\section{Beatificamente}
\begin{itemize}
\item {Grp. gram.:adv.}
\end{itemize}
De modo \textunderscore beatifíco\textunderscore .
\section{Beatificante}
\begin{itemize}
\item {Grp. gram.:adj.}
\end{itemize}
Que beatifica.
\section{Beatificar}
\begin{itemize}
\item {Grp. gram.:v. t.}
\end{itemize}
\begin{itemize}
\item {Proveniência:(Lat. \textunderscore beatificare\textunderscore )}
\end{itemize}
Tornar ou declarar bem-aventurado.
Tornar feliz.
Dar o renome de santo ou justo a.
Louvar muito.
\section{Beatífico}
\begin{itemize}
\item {Grp. gram.:adj.}
\end{itemize}
\begin{itemize}
\item {Proveniência:(Lat. \textunderscore beatificus\textunderscore )}
\end{itemize}
Que torna bem-aventurado.
Relativo a êxtases, transportes, arrôbos mýsticos.
Que dá a suprema felicidade.
\section{Beatilha}
\begin{itemize}
\item {Grp. gram.:f.}
\end{itemize}
\begin{itemize}
\item {Proveniência:(De \textunderscore beata\textunderscore ^3? O mesmo voc. que \textunderscore baetilha\textunderscore ?)}
\end{itemize}
Touca branca de freiras.
Pano de linho ou algodão, de que se faziam essas toucas.
\section{Beatismo}
\begin{itemize}
\item {Grp. gram.:m.}
\end{itemize}
O mesmo que \textunderscore beatice\textunderscore .
\section{Beatíssimo}
\begin{itemize}
\item {Grp. gram.:adj.}
\end{itemize}
\begin{itemize}
\item {Proveniência:(De \textunderscore beato\textunderscore )}
\end{itemize}
Tratamento honorífico dos Papas.
\section{Beatitude}
\begin{itemize}
\item {Grp. gram.:f.}
\end{itemize}
\begin{itemize}
\item {Proveniência:(Lat. \textunderscore beatitudo\textunderscore )}
\end{itemize}
Felicidade suprema.
Bem-aventurança celeste.
Felicidade de quem se absorve em contemplações mýsticas.
Bem-estar, felicidade tranquilla.
Tratamento papal.
\section{Beato}
\begin{itemize}
\item {Grp. gram.:m.}
\end{itemize}
\begin{itemize}
\item {Grp. gram.:Adj.}
\end{itemize}
\begin{itemize}
\item {Proveniência:(Lat. \textunderscore beatus\textunderscore )}
\end{itemize}
Homem de grande devoção religiosa, real ou apparente.
Aquelle que foi beatificado pela Igreja.
Beatificado.
Feliz.
Exaggeradamente devoto, com sinceridade ou fingimento.
\section{Beato}
\begin{itemize}
\item {Grp. gram.:m.}
\end{itemize}
\begin{itemize}
\item {Utilização:Bras. do N}
\end{itemize}
Fio, que se destrama dos tecidos.
\section{Beatôrro}
\begin{itemize}
\item {Grp. gram.:m.}
\end{itemize}
\begin{itemize}
\item {Utilização:Pop.}
\end{itemize}
\begin{itemize}
\item {Proveniência:(De \textunderscore beato\textunderscore )}
\end{itemize}
Santarrão.
Hypócrita.
Beatão. Cf. Castilho, \textunderscore Tartufo\textunderscore , 15.
\section{Beba}
\begin{itemize}
\item {fónica:bê}
\end{itemize}
\begin{itemize}
\item {Grp. gram.:f.}
\end{itemize}
Casta de uva branca algarvia.
\section{Bêbado}
\begin{itemize}
\item {Grp. gram.:m.  e  adj.}
\end{itemize}
O mesmo que \textunderscore bêbedo\textunderscore .
\section{Bebarro}
\begin{itemize}
\item {Grp. gram.:m. pl.}
\end{itemize}
(V.bêbedo)
\section{Bebdomancia}
\begin{itemize}
\item {Grp. gram.:f.}
\end{itemize}
Supposta arte de adivinhar, por meio de dois pauzinhos, encantados para darem resposta. Cf. Castilho, \textunderscore Fastos\textunderscore , III, 317.
\section{Bebé}
\begin{itemize}
\item {fónica:bé-bé}
\end{itemize}
\begin{itemize}
\item {Grp. gram.:m.  e  f.}
\end{itemize}
\begin{itemize}
\item {Proveniência:(Do ingl. \textunderscore baby\textunderscore )}
\end{itemize}
Criança, considerada como uma boneca que se enfeita.
\section{Bêbeda}
\begin{itemize}
\item {Grp. gram.:f.}
\end{itemize}
\begin{itemize}
\item {Utilização:Prov.}
\end{itemize}
Bebedeira.
(Colhido em Turquel)
\section{Bebedeira}
\begin{itemize}
\item {Grp. gram.:f.}
\end{itemize}
\begin{itemize}
\item {Proveniência:(De \textunderscore bêbedo\textunderscore )}
\end{itemize}
Estado de quem se embriagou; embriaguez, borracheira.
Incómmodo, resultante da ingestão de bebidas alcoólicas ou da aspiração de substâncias narcóticas.
\section{Bebedice}
\begin{itemize}
\item {Grp. gram.:f.}
\end{itemize}
\begin{itemize}
\item {Proveniência:(De \textunderscore bêbedo\textunderscore )}
\end{itemize}
Vício de beber immoderadamente.
Bebedeira.
\section{Bêbedo}
\begin{itemize}
\item {Grp. gram.:m.}
\end{itemize}
\begin{itemize}
\item {Grp. gram.:Adj.}
\end{itemize}
\begin{itemize}
\item {Proveniência:(Do lat. \textunderscore bibitus\textunderscore )}
\end{itemize}
Homem, dado ao vício da embriaguez.
Patife.
Homem desavergonhado.
Que está perturbado por ingestão de bebida alcoólica ou por têr aspirado substância narcótica.
Peixe da Póvoa de Varzim.
\section{Bebedoiro}
\begin{itemize}
\item {Grp. gram.:m.}
\end{itemize}
\begin{itemize}
\item {Proveniência:(De \textunderscore beber\textunderscore )}
\end{itemize}
Lugar, vaso, pia, tanque, em que os animaes bebem água.
\section{Bebedolas}
\begin{itemize}
\item {Grp. gram.:m.}
\end{itemize}
\begin{itemize}
\item {Proveniência:(De \textunderscore bêbedo\textunderscore )}
\end{itemize}
Homem que se embriaga habitualmente.
\section{Bebedor}
\begin{itemize}
\item {Grp. gram.:m.  e  adj.}
\end{itemize}
\begin{itemize}
\item {Proveniência:(Lat. \textunderscore bibitor\textunderscore )}
\end{itemize}
O que bebe muito; bêbedo.
\section{Bebedouro}
\begin{itemize}
\item {Grp. gram.:m.}
\end{itemize}
\begin{itemize}
\item {Proveniência:(De \textunderscore beber\textunderscore )}
\end{itemize}
Lugar, vaso, pia, tanque, em que os animaes bebem água.
\section{Bebeerina}
\begin{itemize}
\item {Grp. gram.:f.}
\end{itemize}
Substância medicinal, extrahida do bebeeru.
\section{Bebeeru}
\begin{itemize}
\item {Grp. gram.:m.}
\end{itemize}
\begin{itemize}
\item {Utilização:Bras}
\end{itemize}
Árvore laurínea, (\textunderscore nectandra rodiei\textunderscore , Rob.).
\section{Bebena}
\begin{itemize}
\item {Grp. gram.:f.}
\end{itemize}
\begin{itemize}
\item {Utilização:Prov.}
\end{itemize}
\begin{itemize}
\item {Utilização:minh.}
\end{itemize}
O mesmo que \textunderscore meretriz\textunderscore .
\section{Beber}
\begin{itemize}
\item {Grp. gram.:v. t.}
\end{itemize}
\begin{itemize}
\item {Grp. gram.:V. i.}
\end{itemize}
\begin{itemize}
\item {Utilização:Prov.}
\end{itemize}
\begin{itemize}
\item {Utilização:Constr.}
\end{itemize}
\begin{itemize}
\item {Proveniência:(Lat. \textunderscore bibere\textunderscore )}
\end{itemize}
Ingerir (um líquido): \textunderscore beber água\textunderscore .
Gastar com bebidas: \textunderscore bebeu quanto tinha\textunderscore .
Admittir em si, no seu espírito.
Reter na memória.
Supportar.
Ingerir um líquido.
Resair da superfície circunjacente: \textunderscore aquella tábua bebe para fóra\textunderscore . (Colhido no Fundão).
\textunderscore Cavallo que bebe em branco\textunderscore , cavallo, que é bocalvo, ou que tem malha branca, da boca ao nariz.
\section{Bêbera}
\begin{itemize}
\item {Grp. gram.:f.}
\end{itemize}
\begin{itemize}
\item {Proveniência:(Do lat. \textunderscore bifera\textunderscore )}
\end{itemize}
Figo temporão, grande, preto e alongado.
\section{Beberagem}
\begin{itemize}
\item {Grp. gram.:f.}
\end{itemize}
Cozimento medicinal.
Bebida desagradável.
Água de sêmeas para animaes.
(Cp. cast. \textunderscore brebage\textunderscore )
\section{Bebereira}
\begin{itemize}
\item {Grp. gram.:f.}
\end{itemize}
Figueira, que dá bêberas.
\section{Bebêres}
\begin{itemize}
\item {Grp. gram.:f. pl.}
\end{itemize}
(V.bebes)
\section{Beberete}
\begin{itemize}
\item {fónica:berê}
\end{itemize}
\begin{itemize}
\item {Grp. gram.:m.}
\end{itemize}
\begin{itemize}
\item {Proveniência:(De \textunderscore beber\textunderscore )}
\end{itemize}
Simples refeição, que consta principalmente de licores e vinhos.
\section{Bebericador}
\begin{itemize}
\item {Grp. gram.:m.}
\end{itemize}
Aquelle que beberica.
\section{Bebericar}
\begin{itemize}
\item {Grp. gram.:v. t.  e  i.}
\end{itemize}
Beber pouco, mas muitas vezes.
\section{Beberina}
\begin{itemize}
\item {Grp. gram.:f.}
\end{itemize}
Alcaloide, extrahido da casca do beberu. Cf. \textunderscore Pharmacopeia Port.\textunderscore 
\section{Beberragem}
\begin{itemize}
\item {Grp. gram.:f.}
\end{itemize}
(V.beberagem)
\section{Beberrão}
\begin{itemize}
\item {Grp. gram.:m.  e  adj.}
\end{itemize}
\begin{itemize}
\item {Proveniência:(De \textunderscore beber\textunderscore )}
\end{itemize}
O que bebe muito; borrachão.
\section{Beberraz}
\begin{itemize}
\item {Grp. gram.:m.  e  adj.}
\end{itemize}
(V.beberrão)
\section{Beberrica}
\begin{itemize}
\item {Grp. gram.:m.}
\end{itemize}
\begin{itemize}
\item {Utilização:Prov.}
\end{itemize}
\begin{itemize}
\item {Utilização:beir.}
\end{itemize}
Bêbedo, borracho.
\section{Beberricar}
\textunderscore v. t.\textunderscore  e \textunderscore i.\textunderscore  (e der.)
O mesmo que \textunderscore bebericar\textunderscore , etc.
\section{Beberricas}
\begin{itemize}
\item {Grp. gram.:m.}
\end{itemize}
\begin{itemize}
\item {Utilização:Des.}
\end{itemize}
O mesmo que \textunderscore beberrico\textunderscore .
\section{Beberrico}
\begin{itemize}
\item {Grp. gram.:m.}
\end{itemize}
(V.bebericador)
\section{Beberrona}
\begin{itemize}
\item {Grp. gram.:f.}
\end{itemize}
\begin{itemize}
\item {Proveniência:(De \textunderscore beberrão\textunderscore )}
\end{itemize}
Mulher, que se embriaga, que bebe muito.
\section{Beberronia}
\begin{itemize}
\item {Grp. gram.:f.}
\end{itemize}
Qualidade de beberrão.
Ajuntamento de beberrões.
\section{Beberrote}
\begin{itemize}
\item {Grp. gram.:m.}
\end{itemize}
\begin{itemize}
\item {Utilização:Pop.}
\end{itemize}
O mesmo que \textunderscore beberrão\textunderscore .
\section{Beberu}
\begin{itemize}
\item {Grp. gram.:m.}
\end{itemize}
Planta medicinal.
O mesmo que \textunderscore bebeeru\textunderscore ?
\section{Bebes}
\begin{itemize}
\item {Grp. gram.:m.}
\end{itemize}
\begin{itemize}
\item {Utilização:Pop.}
\end{itemize}
Aquillo que se bebe; bebidas: \textunderscore gasta tudo em comes e bebes\textunderscore .
\section{Bebida}
\begin{itemize}
\item {Grp. gram.:f.}
\end{itemize}
\begin{itemize}
\item {Utilização:Bras}
\end{itemize}
\begin{itemize}
\item {Proveniência:(De \textunderscore beber\textunderscore )}
\end{itemize}
Aquillo que se bebe.
Líquido, preparado com álcool.
Vinho ou outro líquido alcoólico, próprio para se beber.
Hábito de beber muito: \textunderscore a bebida leva-lhe tudo\textunderscore .
O mesmo que \textunderscore bebedoiro\textunderscore .
\section{Bebível}
\begin{itemize}
\item {Grp. gram.:adj.}
\end{itemize}
Que se póde beber; potável.
\section{Bebo}
\begin{itemize}
\item {fónica:bê}
\end{itemize}
\begin{itemize}
\item {Grp. gram.:m.}
\end{itemize}
Espécie de peixe, também chamado \textunderscore bêbedo\textunderscore .
\section{Bébra}
\begin{itemize}
\item {Grp. gram.:f.}
\end{itemize}
\begin{itemize}
\item {Proveniência:(Lat. \textunderscore bebra\textunderscore )}
\end{itemize}
Lança curta, usada pelos Bárbaros, no tempo dos Romanos.
\section{Bêbra}
\begin{itemize}
\item {Grp. gram.:f.}
\end{itemize}
\begin{itemize}
\item {Utilização:Pop.}
\end{itemize}
O mesmo que \textunderscore bêbera\textunderscore .
\section{Beca}
\begin{itemize}
\item {Grp. gram.:f.}
\end{itemize}
Veste talar e preta de funccionários judiciaes e dos alumnos de alguns seminários.
Profissão de quem usa beca.
\section{Becabunga}
\begin{itemize}
\item {Grp. gram.:f.}
\end{itemize}
Espécie de verónica, da fam. das escrofularíneas.
\section{Becedização}
\begin{itemize}
\item {Grp. gram.:f.}
\end{itemize}
\begin{itemize}
\item {Utilização:Mús.}
\end{itemize}
Antigo systema de solfejar, em que se applicava ao solfejo a nomenclatura alphabética dos sons.
(Das letras \textunderscore B\textunderscore , \textunderscore C\textunderscore , \textunderscore D\textunderscore .)
\section{Bechamel}
\begin{itemize}
\item {Grp. gram.:m.}
\end{itemize}
\begin{itemize}
\item {Proveniência:(De \textunderscore Bechamel\textunderscore , n. p.)}
\end{itemize}
Môlho, feito de gorduras e legumes.
\section{Béchico}
\begin{itemize}
\item {fónica:qui}
\end{itemize}
\begin{itemize}
\item {Grp. gram.:m.  e  adj.}
\end{itemize}
\begin{itemize}
\item {Proveniência:(Gr. \textunderscore bekhikos\textunderscore )}
\end{itemize}
Aquillo que é bom contra a tosse.
\section{Bechuanas}
\begin{itemize}
\item {Grp. gram.:m. pl.}
\end{itemize}
Povos da África austro-central.
\section{Bechucarias}
\begin{itemize}
\item {Grp. gram.:f. pl.}
\end{itemize}
\begin{itemize}
\item {Utilização:Ant.}
\end{itemize}
O mesmo que \textunderscore quinquilharias\textunderscore .
\section{Bêco}
\begin{itemize}
\item {Grp. gram.:m.}
\end{itemize}
\begin{itemize}
\item {Proveniência:(Do lat. \textunderscore viculus\textunderscore ?)}
\end{itemize}
Rua estreita e curta, ás vezes sem saída.
\section{Bécua}
\begin{itemize}
\item {Grp. gram.:f.}
\end{itemize}
\begin{itemize}
\item {Utilização:Prov.}
\end{itemize}
\begin{itemize}
\item {Utilização:extrem.}
\end{itemize}
O mesmo que \textunderscore abibe\textunderscore .
\section{Becuínha}
\begin{itemize}
\item {Grp. gram.:f.}
\end{itemize}
\begin{itemize}
\item {Utilização:Prov.}
\end{itemize}
\begin{itemize}
\item {Utilização:extrem.}
\end{itemize}
\begin{itemize}
\item {Proveniência:(De \textunderscore bécua\textunderscore )}
\end{itemize}
O mesmo que \textunderscore abibe\textunderscore ; bécua pequena.
\section{Bedalço}
\begin{itemize}
\item {Grp. gram.:m.}
\end{itemize}
Árvore da Índia portuguesa.
\section{Bedalha}
\begin{itemize}
\item {Grp. gram.:f.}
\end{itemize}
\begin{itemize}
\item {Utilização:Prov.}
\end{itemize}
\begin{itemize}
\item {Utilização:trasm.}
\end{itemize}
Presente de núpcias, dado á noiva pelo noivo ou pelas amigas della.
(Relaciona-se com \textunderscore viadalhas\textunderscore ?)
\section{Bedame}
(V.badame)
\section{Bedegar}
\begin{itemize}
\item {Grp. gram.:m.}
\end{itemize}
O mesmo que \textunderscore bedeguar\textunderscore .
\section{Bedeguar}
\begin{itemize}
\item {Grp. gram.:m.}
\end{itemize}
Espécie de galha olorosa, produzida nos ramos de certas roseiras pela picada de diversos insectos.
\section{Bedel}
\begin{itemize}
\item {Grp. gram.:m.}
\end{itemize}
\begin{itemize}
\item {Proveniência:(Do b. lat. \textunderscore bedellus\textunderscore )}
\end{itemize}
Empregado, que na Universidade faz a chamada e aponta as faltas dos estudantes e lentes.
\section{Bedelhar}
\begin{itemize}
\item {Grp. gram.:v. i.}
\end{itemize}
Meter o bedelho, intrometer-se.
Conversar familiarmente, cavaquear.
\section{Bedelho}
\begin{itemize}
\item {fónica:dê}
\end{itemize}
\begin{itemize}
\item {Grp. gram.:m.}
\end{itemize}
Tranqueta ou ferrolho, que se levanta por meio da aldrava.
Criançola, rapazelho.
Trunfo pequeno.
\textunderscore Meter o bedelho\textunderscore , intrometer-se importunamente (em conversas ou assumptos estranhos).
\section{Bedém}
\begin{itemize}
\item {Grp. gram.:m.}
\end{itemize}
Túnica moirisca, curta e sem mangas.
Capa palhiça ou de coiro ou de esparto, contra a chuva.
(Ár. \textunderscore beden\textunderscore )
\section{Bédon}
\begin{itemize}
\item {Grp. gram.:m.}
\end{itemize}
Espécie de tamboril medieval, de fuste muito longo.
\section{Bedrelhos}
\begin{itemize}
\item {fónica:dre}
\end{itemize}
\begin{itemize}
\item {Grp. gram.:m. pl.}
\end{itemize}
\begin{itemize}
\item {Utilização:T. de Chaves}
\end{itemize}
O jôgo das nécaras.
\section{Bedro}
\begin{itemize}
\item {fónica:bê}
\end{itemize}
\begin{itemize}
\item {Grp. gram.:m.}
\end{itemize}
\begin{itemize}
\item {Utilização:Prov.}
\end{itemize}
O mesmo que \textunderscore bredo\textunderscore .
\section{Beduí}
\begin{itemize}
\item {Grp. gram.:m.}
\end{itemize}
O mesmo que \textunderscore beduím\textunderscore .
\section{Beduím}
\begin{itemize}
\item {Grp. gram.:m.}
\end{itemize}
\begin{itemize}
\item {Utilização:Fig.}
\end{itemize}
\begin{itemize}
\item {Proveniência:(Do ár. \textunderscore badain\textunderscore )}
\end{itemize}
Arabe, que vive no deserto.
Homem selvagem, brutal.
\section{Beduíno}
\begin{itemize}
\item {Grp. gram.:m.}
\end{itemize}
(Fórma moderna e afrancesada, em vez de \textunderscore beduím\textunderscore )
\section{Bedum}
\begin{itemize}
\item {Grp. gram.:m.}
\end{itemize}
(Corr. de \textunderscore bodum\textunderscore )
\section{Beetria}
\begin{itemize}
\item {Grp. gram.:f.}
\end{itemize}
\begin{itemize}
\item {Utilização:Ant.}
\end{itemize}
Povoação, que em Portugal tinha, entre outros direitos, o de eleger os seus administradores.
(Alter. de \textunderscore benefactoria\textunderscore , do lat. \textunderscore benefacere\textunderscore )
\section{Befa}
\begin{itemize}
\item {fónica:bê}
\end{itemize}
\begin{itemize}
\item {Grp. gram.:f.}
\end{itemize}
\begin{itemize}
\item {Utilização:Des.}
\end{itemize}
\begin{itemize}
\item {Proveniência:(It. \textunderscore beffa\textunderscore )}
\end{itemize}
Zombaria.
Burla.
\section{Begarim}
\begin{itemize}
\item {Grp. gram.:m.}
\end{itemize}
\begin{itemize}
\item {Utilização:T. de Gôa}
\end{itemize}
Trabalhador rural.
\section{Begónia}
\begin{itemize}
\item {Grp. gram.:f.}
\end{itemize}
\begin{itemize}
\item {Proveniência:(De \textunderscore Begon\textunderscore , n. p.)}
\end{itemize}
Gênero de plantas ornamentaes.
\section{Begoniáceas}
\begin{itemize}
\item {Grp. gram.:f. pl.}
\end{itemize}
Família de plantas ornamentaes, que têm por typo a \textunderscore begónia\textunderscore .
\section{Begue}
\begin{itemize}
\item {Grp. gram.:m.}
\end{itemize}
\begin{itemize}
\item {Utilização:Ant.}
\end{itemize}
\begin{itemize}
\item {Proveniência:(Do turc. \textunderscore beg\textunderscore )}
\end{itemize}
O mesmo que \textunderscore bei\textunderscore .
\section{Begueiro}
\begin{itemize}
\item {fónica:bé}
\end{itemize}
\begin{itemize}
\item {Grp. gram.:m.  e  adj.}
\end{itemize}
\begin{itemize}
\item {Utilização:Prov.}
\end{itemize}
\begin{itemize}
\item {Utilização:minh.}
\end{itemize}
\begin{itemize}
\item {Grp. gram.:M.}
\end{itemize}
\begin{itemize}
\item {Utilização:Prov.}
\end{itemize}
\begin{itemize}
\item {Utilização:minh.}
\end{itemize}
Diz-se do jumento, quando pequeno.
Bêsta de carga, especialmente o mulo.
\section{Beguina}
\begin{itemize}
\item {Grp. gram.:f.}
\end{itemize}
Mulher religiosa, pertencente á seita dos beguinos.
\section{Beguinaria}
\begin{itemize}
\item {Grp. gram.:f.}
\end{itemize}
\begin{itemize}
\item {Proveniência:(De \textunderscore beguino\textunderscore )}
\end{itemize}
Clausura, em que viviam beguinos ou beguinas.
\section{Beguino}
\begin{itemize}
\item {Grp. gram.:m.}
\end{itemize}
\begin{itemize}
\item {Utilização:Ant.}
\end{itemize}
\begin{itemize}
\item {Grp. gram.:Adj.}
\end{itemize}
\begin{itemize}
\item {Utilização:Ant.}
\end{itemize}
\begin{itemize}
\item {Grp. gram.:Loc. adv.}
\end{itemize}
\begin{itemize}
\item {Proveniência:(Fr. \textunderscore beguin\textunderscore )}
\end{itemize}
Membro de uma seita do século XIII.
Homem penitente e pobre.
Frade mendicante.
Beato, hypócrita.
\textunderscore Á beguina\textunderscore , á maneira das beguinas ou dos beguinos:«\textunderscore vestido á beguina\textunderscore ». Rebello, \textunderscore Contos e Lendas\textunderscore , 59.
\section{Behetria}
\begin{itemize}
\item {Grp. gram.:f.}
\end{itemize}
\begin{itemize}
\item {Utilização:Ant.}
\end{itemize}
Povoação, que em Portugal tinha, entre outros direitos, o de eleger os seus administradores.
(Alter. de \textunderscore benefactoria\textunderscore , do lat. \textunderscore benefacere\textunderscore )
\section{Bei}
\begin{itemize}
\item {Grp. gram.:m.}
\end{itemize}
Governador de algumas províncias muçulmanas.
(Do turco \textunderscore beg\textunderscore )
\section{Beiça}
\begin{itemize}
\item {Grp. gram.:f.}
\end{itemize}
\begin{itemize}
\item {Utilização:Chul.}
\end{itemize}
O mesmo que \textunderscore beiço\textunderscore ; beiço grande e caído.
Beiço inferior.
\section{Beiçada}
\begin{itemize}
\item {Grp. gram.:f.}
\end{itemize}
\begin{itemize}
\item {Utilização:Chul.}
\end{itemize}
\begin{itemize}
\item {Proveniência:(De \textunderscore beiça\textunderscore )}
\end{itemize}
Beiços grossos e pendentes.
\section{Beiçana}
\begin{itemize}
\item {Grp. gram.:f.}
\end{itemize}
\begin{itemize}
\item {Grp. gram.:M.}
\end{itemize}
O mesmo que \textunderscore beiçada\textunderscore .
Aquelle que tem beiçada.
\section{Beiçarrão}
\begin{itemize}
\item {Grp. gram.:m.}
\end{itemize}
\begin{itemize}
\item {Utilização:Fam.}
\end{itemize}
Grande beiço.
\section{Beicinho}
\begin{itemize}
\item {Grp. gram.:m.}
\end{itemize}
Beiço pequeno.
\textunderscore Fazer beicinho\textunderscore , mostrar-se agastado, dispor-se para chorar, amuar-se, (falando de crianças).
\section{Beiço}
\begin{itemize}
\item {Grp. gram.:m.}
\end{itemize}
Cada uma das duas partes avermelhadas, que constituem o contorno externo da bôca.
Lábio.
Bôrdo.
\textunderscore Fazer beiço\textunderscore , amuar-se.
\section{Beiçó}
\begin{itemize}
\item {Grp. gram.:f.}
\end{itemize}
\begin{itemize}
\item {Utilização:Prov.}
\end{itemize}
O mesmo que \textunderscore moéla\textunderscore ^1.
(Cp. \textunderscore moiçó\textunderscore )
\section{Beiçoca}
\begin{itemize}
\item {Grp. gram.:f.}
\end{itemize}
\begin{itemize}
\item {Utilização:Pop.}
\end{itemize}
Beiço grosso.
\section{Beiçola}
\begin{itemize}
\item {Grp. gram.:m.  e  f.}
\end{itemize}
Beiço grande.
Pessôa beiçuda.
\section{Beiçudo}
\begin{itemize}
\item {Grp. gram.:adj.}
\end{itemize}
Que tem beiços grossos.
\section{Beijado}
\begin{itemize}
\item {Grp. gram.:adj.}
\end{itemize}
Aproximado, unido:«\textunderscore a sala tinha beijados com as paredes muitos assentos\textunderscore ». Filinto, \textunderscore D. Man.\textunderscore , I, 97.
\section{Beijador}
\begin{itemize}
\item {Grp. gram.:m.  e  adj.}
\end{itemize}
O que beija.
\section{Beijaflor}
\begin{itemize}
\item {Grp. gram.:m.}
\end{itemize}
\begin{itemize}
\item {Proveniência:(De \textunderscore beijar\textunderscore  + \textunderscore flôr\textunderscore )}
\end{itemize}
Formosa ave brasileira, que absorve o néctar das flôres.
\section{Beijamão}
\begin{itemize}
\item {Grp. gram.:m.}
\end{itemize}
Acção de beijar a mão.
\section{Beijapé}
\begin{itemize}
\item {Grp. gram.:m.}
\end{itemize}
Acção de beijar o pé.
\section{Beijar}
\begin{itemize}
\item {Grp. gram.:v. t.}
\end{itemize}
\begin{itemize}
\item {Proveniência:(Lat. \textunderscore basiare\textunderscore )}
\end{itemize}
Dar beijo em; oscular.
Tocar de leve.
Inclinar-se até tocar em.
\section{Beijinho}
\begin{itemize}
\item {Grp. gram.:m.}
\end{itemize}
\begin{itemize}
\item {Utilização:Prov.}
\end{itemize}
\begin{itemize}
\item {Utilização:trasm.}
\end{itemize}
Beijo leve, pouco sensível ou pouco demorado.
A nata, a flôr, o que há de melhor entre indivíduos ou coisas: \textunderscore é o beijinho da rapaziada\textunderscore .
Espécie de bolo pequenino.
Farinha fina de trigo, separada da sêmea.
Pequeno búzio, que não excede um centímetro de comprimento.
\section{Beijo}
\begin{itemize}
\item {Grp. gram.:m.}
\end{itemize}
\begin{itemize}
\item {Proveniência:(Do lat. \textunderscore basium\textunderscore )}
\end{itemize}
Acção de chegar os labios fechados a alguém ou a alguma coisa, abrindo-os depois com um pequeno ruído; ósculo.
\section{Beijoca}
\begin{itemize}
\item {Grp. gram.:f.}
\end{itemize}
\begin{itemize}
\item {Utilização:Pop.}
\end{itemize}
Beijo, que se ouve a distância.
\section{Beijocador}
\begin{itemize}
\item {Grp. gram.:adj.}
\end{itemize}
\begin{itemize}
\item {Grp. gram.:M.}
\end{itemize}
\begin{itemize}
\item {Utilização:Ant.}
\end{itemize}
Que beijoca.
Sinal postiço ao canto da boca.
\section{Beijocar}
\begin{itemize}
\item {Grp. gram.:v. t.}
\end{itemize}
Dar beijocas em.
Dar beijos amiúde em.
\section{Beijoim}
\begin{itemize}
\item {Grp. gram.:m.}
\end{itemize}
\begin{itemize}
\item {Proveniência:(Do ár. \textunderscore luban jaui\textunderscore , supprimida a 1.^a sýllaba)}
\end{itemize}
Resina amarelada e aromática, cuja substância entra na composição de cosméticos, pastilhas, etc.--Fórma preferível é \textunderscore benjoim\textunderscore , que soffreu a infl. de \textunderscore beijo\textunderscore , passando para \textunderscore beijoim\textunderscore .
\section{Beijoínico}
\begin{itemize}
\item {Grp. gram.:adj.}
\end{itemize}
Diz-se de um ácido, que se extrai do beijoim.
\section{Beijoqueiro}
\begin{itemize}
\item {Grp. gram.:adj.}
\end{itemize}
\begin{itemize}
\item {Utilização:Fam.}
\end{itemize}
Que gosta de beijocar.
Menineiro; caricioso.
\section{Beijoquinho}
\begin{itemize}
\item {Grp. gram.:m.}
\end{itemize}
\begin{itemize}
\item {Utilização:Prov.}
\end{itemize}
\begin{itemize}
\item {Utilização:beir.}
\end{itemize}
\begin{itemize}
\item {Proveniência:(De \textunderscore beijoca\textunderscore )}
\end{itemize}
O mesmo que \textunderscore menino\textunderscore , especialmente o que é bonito ou meigo.
\section{Beijos-de-freira}
\begin{itemize}
\item {Grp. gram.:m. pl.}
\end{itemize}
\begin{itemize}
\item {Utilização:Bot.}
\end{itemize}
O mesmo que [[candelária-dos-jardins|candelária]].
\section{Beiju}
\begin{itemize}
\item {Grp. gram.:m.}
\end{itemize}
\begin{itemize}
\item {Utilização:Bras}
\end{itemize}
Espécie de filhó, feita de tapioca e também da massa da mandioca.
(Do tupi)
\section{Beijupirá}
\begin{itemize}
\item {Grp. gram.:m.}
\end{itemize}
\begin{itemize}
\item {Proveniência:(T. tupi)}
\end{itemize}
Peixe do Brasil, muito estimado.
\section{Beiju-xica}
\begin{itemize}
\item {Grp. gram.:m.}
\end{itemize}
\begin{itemize}
\item {Utilização:Bras. do Pará}
\end{itemize}
Espécie de filhó da massa da mandioca.
(Cp. \textunderscore beiju\textunderscore )
\section{Beilhique}
\begin{itemize}
\item {Grp. gram.:m.}
\end{itemize}
Território governado por um bei.
\section{Beilhó}
\begin{itemize}
\item {Grp. gram.:m.}
\end{itemize}
Bolo frito, de farinha e abóbora.
\section{Beilhós}
\begin{itemize}
\item {Grp. gram.:m.  e  f.}
\end{itemize}
O mesmo que \textunderscore beilhó\textunderscore .
\section{Beira}
\begin{itemize}
\item {Grp. gram.:f.}
\end{itemize}
\begin{itemize}
\item {Utilização:Prov.}
\end{itemize}
\begin{itemize}
\item {Utilização:beir.}
\end{itemize}
Proximidade: \textunderscore chegou á beira do grupo\textunderscore .
Borda; margem: \textunderscore á beira do Tejo\textunderscore .
Água pluvial, que por uma ruptura do telhado cái dentro de casa.
\section{Beiracampo}
\begin{itemize}
\item {Grp. gram.:f.}
\end{itemize}
\begin{itemize}
\item {Utilização:Bras}
\end{itemize}
\begin{itemize}
\item {Proveniência:(De \textunderscore beira\textunderscore  + \textunderscore campo\textunderscore )}
\end{itemize}
Terreno, comprehendido entre o limite de um campo e o ponto em que, a começar daquelle, se perfazem 600 braças.
\section{Beirada}
\begin{itemize}
\item {Grp. gram.:f.}
\end{itemize}
\begin{itemize}
\item {Utilização:Bras. do N}
\end{itemize}
O mesmo que \textunderscore beiral\textunderscore .
Beira, margem. Cf. Herculano, \textunderscore Quest. Púb.\textunderscore , II, 27.
Cercanias, arredores.
\section{Beirado}
\begin{itemize}
\item {Grp. gram.:m.}
\end{itemize}
O mesmo que \textunderscore beiral\textunderscore .
\section{Beiral}
\begin{itemize}
\item {Grp. gram.:m.}
\end{itemize}
Beira do telhado.
Fileira de telhas, que formam a parte mais baixa do telhado.
Água, que cái de uma telha do beirado.
A beira ou borda de qualquer coisa:«\textunderscore sente-se neste beiral da eira\textunderscore ». Camillo, \textunderscore Cavar em Ruínas\textunderscore , 58.
\section{Beiramar}
\begin{itemize}
\item {Grp. gram.:f.}
\end{itemize}
\begin{itemize}
\item {Proveniência:(De \textunderscore beira\textunderscore  + \textunderscore mar\textunderscore )}
\end{itemize}
Beira do mar; praia.
\section{Beirame}
\begin{itemize}
\item {Grp. gram.:m.}
\end{itemize}
Pano fino de algodão, que vinha da Índia.
\section{Beiraminho}
\begin{itemize}
\item {Grp. gram.:m.}
\end{itemize}
\begin{itemize}
\item {Proveniência:(De \textunderscore beirame\textunderscore )}
\end{itemize}
Pano fino da Índia.
\section{Beirão}
\begin{itemize}
\item {Grp. gram.:m.  e  adj.}
\end{itemize}
O mesmo que \textunderscore beirense\textunderscore .
\section{Beirar}
\begin{itemize}
\item {Grp. gram.:v. t.}
\end{itemize}
\begin{itemize}
\item {Utilização:P. us.}
\end{itemize}
\begin{itemize}
\item {Utilização:Bras}
\end{itemize}
Orlar; debruar.
Caminhar á beira ou pela margem de.
Abeirar-se de.
\section{Beirense}
\begin{itemize}
\item {Grp. gram.:adj.}
\end{itemize}
\begin{itemize}
\item {Grp. gram.:M.}
\end{itemize}
\begin{itemize}
\item {Proveniência:(De \textunderscore Beira\textunderscore , n. p.)}
\end{itemize}
Relativo á província da Beira-Alta ou Baixa.
Indivíduo natural da Beira.
\section{Beirinha}
\begin{itemize}
\item {Grp. gram.:f.}
\end{itemize}
O mesmo que \textunderscore alvéloa\textunderscore .
\section{Beirinha}
\begin{itemize}
\item {Grp. gram.:f.}
\end{itemize}
\begin{itemize}
\item {Proveniência:(De \textunderscore beira\textunderscore )}
\end{itemize}
Estado de muito próximo: \textunderscore sentou-se á beirinha delle\textunderscore .
\section{Beiro}
\begin{itemize}
\item {Grp. gram.:m.}
\end{itemize}
Barco de Timor, feito do tronco cavado de uma árvore.
\section{Beirôa}
\begin{itemize}
\item {Grp. gram.:f.}
\end{itemize}
\begin{itemize}
\item {Utilização:Prov.}
\end{itemize}
\begin{itemize}
\item {Utilização:alent.}
\end{itemize}
\begin{itemize}
\item {Proveniência:(De \textunderscore beirão\textunderscore )}
\end{itemize}
Mulher da província da Beira-Alta ou da Beira-Baixa.
Espécie de chocalho.
\section{Beisa}
\begin{itemize}
\item {Grp. gram.:f.}
\end{itemize}
Animal africano. Cf. Capello e Ivens, I, 61.
\section{Beisar}
\begin{itemize}
\item {Grp. gram.:v. t.}
\end{itemize}
\begin{itemize}
\item {Utilização:Ant.}
\end{itemize}
O mesmo que \textunderscore beijar\textunderscore . Cf. Moraes.
\section{Beja}
\begin{itemize}
\item {Grp. gram.:f.}
\end{itemize}
\begin{itemize}
\item {Utilização:Ant.}
\end{itemize}
Coberta de navio.
\section{Bejaldro}
\begin{itemize}
\item {Grp. gram.:m.}
\end{itemize}
\begin{itemize}
\item {Utilização:Prov.}
\end{itemize}
\begin{itemize}
\item {Utilização:trasm.}
\end{itemize}
O mesmo que \textunderscore casaco\textunderscore .
(Colhido em Sabrosa)
\section{Bejense}
\begin{itemize}
\item {Grp. gram.:adj.}
\end{itemize}
\begin{itemize}
\item {Grp. gram.:M.}
\end{itemize}
Relativo á cidade de Beja.
Habitante de Beja.
\section{Bejo}
\begin{itemize}
\item {Grp. gram.:m.}
\end{itemize}
(V.beijo)
\section{Bejoéga}
\begin{itemize}
\item {Grp. gram.:f.}
\end{itemize}
O mesmo que \textunderscore bejoga\textunderscore .
\section{Bejoga}
\begin{itemize}
\item {Grp. gram.:f.}
\end{itemize}
\begin{itemize}
\item {Utilização:Prov.}
\end{itemize}
\begin{itemize}
\item {Utilização:trasm.}
\end{itemize}
\begin{itemize}
\item {Proveniência:(Do lat. \textunderscore vesucula\textunderscore )}
\end{itemize}
Empôla, bolha nos pés, cheia de lympha.
\section{Bejú}
\begin{itemize}
\item {Grp. gram.:m.}
\end{itemize}
O mesmo que \textunderscore beiju\textunderscore .
\section{Bejula}
\begin{itemize}
\item {Grp. gram.:f.}
\end{itemize}
\begin{itemize}
\item {Utilização:T. da África or. port}
\end{itemize}
Bebida fermentada, feita de farinha de milho ou de outro cereal.--Também se me deparou a fórma \textunderscore bejala\textunderscore . Qual das duas é a exacta?
\section{Bel}
\begin{itemize}
\item {Grp. gram.:adj.}
\end{itemize}
Bello, (raramente usado, como em \textunderscore a bel prazer\textunderscore , á vontade).
(Contr. de \textunderscore bello\textunderscore )
\section{Bela}
\begin{itemize}
\item {Grp. gram.:f.}
\end{itemize}
Mulher bela: \textunderscore galantear as belas\textunderscore .
(Fem. de \textunderscore bello\textunderscore )
\section{Belacíssimo}
\begin{itemize}
\item {Grp. gram.:adj.}
\end{itemize}
\begin{itemize}
\item {Proveniência:(Do lat. \textunderscore bellax\textunderscore )}
\end{itemize}
Belicoso; aguerrido.
\section{Beladona}
\begin{itemize}
\item {Grp. gram.:f.}
\end{itemize}
\begin{itemize}
\item {Proveniência:(It. \textunderscore belladonna\textunderscore )}
\end{itemize}
Planta herbácea, venenosa e medicinal, da fam. das soláneas.
\section{Beladónio}
\begin{itemize}
\item {Grp. gram.:m.}
\end{itemize}
Extracto de beladona. Cf. \textunderscore Pharmacopeia Port.\textunderscore 
\section{Belambo}
\begin{itemize}
\item {Grp. gram.:m.}
\end{itemize}
Árvore angolense, no Duque-de-Bragança.
\section{Belamente}
\begin{itemize}
\item {Grp. gram.:adv.}
\end{itemize}
De modo \textunderscore belo\textunderscore .
Excelentemente.
\section{Belancia}
\begin{itemize}
\item {Grp. gram.:f.}
\end{itemize}
\begin{itemize}
\item {Utilização:Prov.}
\end{itemize}
O mesmo que \textunderscore melancia\textunderscore .
\section{Belanta}
\begin{itemize}
\item {Grp. gram.:f.}
\end{itemize}
Rede volante da pescada, no Doiro.
(Corr. de \textunderscore volante\textunderscore )
\section{Belão}
\begin{itemize}
\item {Grp. gram.:m.}
\end{itemize}
\begin{itemize}
\item {Utilização:Prov.}
\end{itemize}
\begin{itemize}
\item {Utilização:trasm.}
\end{itemize}
Lombada entre dois sulcos, não accessível á água da rega.
\section{Belarte}
\begin{itemize}
\item {Grp. gram.:m.}
\end{itemize}
\begin{itemize}
\item {Utilização:Des.}
\end{itemize}
Espécie de tecido de lan.
(Cast. \textunderscore velarte\textunderscore )
\section{Belasiano}
\begin{itemize}
\item {Grp. gram.:adj.}
\end{itemize}
\begin{itemize}
\item {Utilização:Geol.}
\end{itemize}
Diz-se de uma das espécies de terreno cretáceo.
\section{Belatrice}
\begin{itemize}
\item {Grp. gram.:f.}
\end{itemize}
O mesmo que \textunderscore belatriz\textunderscore .
\section{Belatriz}
\begin{itemize}
\item {Grp. gram.:f.}
\end{itemize}
\begin{itemize}
\item {Proveniência:(Lat. \textunderscore bellatrix\textunderscore )}
\end{itemize}
Mulher guerreira.
Grande estrêlla da constellação Oríon.
\section{Belborinho}
\begin{itemize}
\item {Grp. gram.:m.}
\end{itemize}
\begin{itemize}
\item {Utilização:Pop.}
\end{itemize}
O mesmo que \textunderscore borborinho\textunderscore . Cf. \textunderscore Viriato Trág.\textunderscore , XVI, 69.
\section{Belbotreira}
\begin{itemize}
\item {Grp. gram.:f.}
\end{itemize}
\begin{itemize}
\item {Utilização:Prov.}
\end{itemize}
\begin{itemize}
\item {Utilização:trasm.}
\end{itemize}
Mulher mentirosa, mexeriqueira.
\section{Bélbute}
\begin{itemize}
\item {Grp. gram.:m.}
\end{itemize}
\begin{itemize}
\item {Proveniência:(Do ingl. \textunderscore velvet\textunderscore )}
\end{itemize}
Tecido de algodão avelludado.
\section{Belbutina}
\begin{itemize}
\item {Grp. gram.:f.}
\end{itemize}
Bélbute fino.
\section{Belchior}
\begin{itemize}
\item {Grp. gram.:m.}
\end{itemize}
\begin{itemize}
\item {Utilização:Bras}
\end{itemize}
\begin{itemize}
\item {Proveniência:(De \textunderscore Belchior\textunderscore , n. p.)}
\end{itemize}
Mercador de objectos velhos e usados; ferro-velho.
Alfarrabista.
\section{Beldade}
\begin{itemize}
\item {Grp. gram.:f.}
\end{itemize}
\begin{itemize}
\item {Proveniência:(Lat. \textunderscore bellitas\textunderscore )}
\end{itemize}
Belleza.
Mulher bella.
\section{Beldar}
\begin{itemize}
\item {Grp. gram.:v. i.}
\end{itemize}
\begin{itemize}
\item {Utilização:Prov.}
\end{itemize}
\begin{itemize}
\item {Utilização:trasm.}
\end{itemize}
Dar á taramela, tagarelar; falar sem tom nem som.
(Talvez corr. de \textunderscore badalar\textunderscore )
\section{Beldro}
\begin{itemize}
\item {fónica:bêl}
\end{itemize}
\begin{itemize}
\item {Grp. gram.:m.}
\end{itemize}
\begin{itemize}
\item {Utilização:Prov.}
\end{itemize}
\begin{itemize}
\item {Utilização:trasm.}
\end{itemize}
Planta hortense, de producção espontânea.
O mesmo que \textunderscore bredo\textunderscore .
\section{Beldroca}
\begin{itemize}
\item {Grp. gram.:f.}
\end{itemize}
\begin{itemize}
\item {Utilização:Prov.}
\end{itemize}
O mesmo que \textunderscore baldroca\textunderscore .
\section{Beldroéga}
\begin{itemize}
\item {Grp. gram.:f.}
\end{itemize}
Planta hortense, da fam. das portuláceas, (\textunderscore portulaca oleracea\textunderscore , Lin.).
\section{Beldroegas}
\begin{itemize}
\item {Grp. gram.:m.}
\end{itemize}
\begin{itemize}
\item {Utilização:Pop.}
\end{itemize}
Pessôa boçal.
\section{Beledis}
\begin{itemize}
\item {Grp. gram.:m. pl.}
\end{itemize}
Colónia muçulmana, que se estabeleceu na península hispânica e era formada de egýpcios e árabes. Cf. Herculano, \textunderscore Hist. de Port.\textunderscore , III, 201.
\section{Beleguim}
\begin{itemize}
\item {Grp. gram.:m.}
\end{itemize}
Antigo empregado judicial, que citava, prendia, etc.
Hoje, designação depreciativa dos officiaes de diligencias, agentes policiaes, etc.
\section{Belemnite}
\begin{itemize}
\item {Grp. gram.:f.}
\end{itemize}
Gênero de mollúscos cephalópodes, fósseis.
\section{Belemzada}
\begin{itemize}
\item {Grp. gram.:f.}
\end{itemize}
Revolta, feita em Belém, em 1836.
\section{Belencufa}
\begin{itemize}
\item {Grp. gram.:f.}
\end{itemize}
Árvore da Guiné.
\section{Belendengues}
\begin{itemize}
\item {Grp. gram.:m. pl.}
\end{itemize}
\begin{itemize}
\item {Utilização:Bras. do S}
\end{itemize}
Cavallaria de veteranos para defesa da fronteira.
\section{Beleno}
\begin{itemize}
\item {Grp. gram.:adj.}
\end{itemize}
\begin{itemize}
\item {Utilização:Prov.}
\end{itemize}
\begin{itemize}
\item {Utilização:minh.}
\end{itemize}
Pouco ajuizado.
\section{Beletrista}
\begin{itemize}
\item {Grp. gram.:m.  e  f.}
\end{itemize}
\begin{itemize}
\item {Utilização:Neol.}
\end{itemize}
\begin{itemize}
\item {Proveniência:(Al. \textunderscore belletrist\textunderscore , do fr. \textunderscore belles-letres\textunderscore )}
\end{itemize}
Pessôa, que cultiva a beletrística.
\section{Beletrística}
\begin{itemize}
\item {Grp. gram.:f.}
\end{itemize}
\begin{itemize}
\item {Proveniência:(Al. \textunderscore belletristika\textunderscore , de \textunderscore belletrist\textunderscore )}
\end{itemize}
Designação das obras que constituem a chamada \textunderscore literatura amena\textunderscore , (romances, poesia, etc.).
\section{Beletrístico}
\begin{itemize}
\item {Grp. gram.:adj.}
\end{itemize}
Relativo á \textunderscore beletrística\textunderscore . Cf. C. Michaëlis, \textunderscore Estatinga\textunderscore , 12.
\section{Belevália}
\begin{itemize}
\item {Grp. gram.:f.}
\end{itemize}
\begin{itemize}
\item {Proveniência:(De \textunderscore Belleval\textunderscore , n. p. de um bot. fr.)}
\end{itemize}
Gênero de plantas liliáceas.
\section{Beleza}
\begin{itemize}
\item {fónica:lê}
\end{itemize}
\begin{itemize}
\item {Grp. gram.:f.}
\end{itemize}
\begin{itemize}
\item {Utilização:Gír.}
\end{itemize}
Qualidade do que é bello, agradável ou que desperta admiração: \textunderscore a beleza do Sol\textunderscore .
Mulher bela: \textunderscore a Maria é uma beleza\textunderscore .
Coisa bela ou muito agradável: \textunderscore êste manjar está uma beleza\textunderscore .
Melena, cabeleira: \textunderscore o gajo cortou as belezas\textunderscore .
\section{Bélfa}
\begin{itemize}
\item {Grp. gram.:f.}
\end{itemize}
\begin{itemize}
\item {Utilização:Prov.}
\end{itemize}
\begin{itemize}
\item {Utilização:trasm.}
\end{itemize}
\begin{itemize}
\item {Proveniência:(Do lat. \textunderscore bellua\textunderscore )}
\end{itemize}
O mesmo que \textunderscore molhelha\textunderscore ^1.
Melga.
\section{Bêlfa}
\begin{itemize}
\item {Grp. gram.:f.}
\end{itemize}
\begin{itemize}
\item {Utilização:Prov.}
\end{itemize}
\begin{itemize}
\item {Utilização:minh.}
\end{itemize}
\begin{itemize}
\item {Proveniência:(De \textunderscore bêlfo\textunderscore ?)}
\end{itemize}
Bazófia, prosápia.
\section{Belfaças}
\begin{itemize}
\item {Grp. gram.:f. pl.}
\end{itemize}
Bêlfas grandes.
\section{Belfarinheiro}
\begin{itemize}
\item {Grp. gram.:m.}
\end{itemize}
(Corr. trasm. de \textunderscore bufarinheiro\textunderscore )
\section{Bêlfas}
\begin{itemize}
\item {Grp. gram.:f. pl.}
\end{itemize}
\begin{itemize}
\item {Proveniência:(De \textunderscore bêlfo\textunderscore )}
\end{itemize}
Bochechas.
\section{Belfécio}
\begin{itemize}
\item {Grp. gram.:m.}
\end{itemize}
\begin{itemize}
\item {Utilização:Pop.}
\end{itemize}
Homem de grandes nádegas.
Indivíduo ridículo, mulherengo ou covarde.
\section{Bêlfo}
\begin{itemize}
\item {Grp. gram.:adj.}
\end{itemize}
\begin{itemize}
\item {Utilização:Pop.}
\end{itemize}
\begin{itemize}
\item {Utilização:Prov.}
\end{itemize}
\begin{itemize}
\item {Utilização:trasm.}
\end{itemize}
\begin{itemize}
\item {Grp. gram.:M.}
\end{itemize}
\begin{itemize}
\item {Utilização:Gír.}
\end{itemize}
Que tem beiços grossos.
Cujo beiço inferior é mais grosso que o superior.
Que fala mal, confusamente, como quem tem a bôca cheia.
Que tem os dentes rombos, e mal póde comer a erva, (falando-se de certos animaes).
Cão.
\section{Belfurinheiro}
\begin{itemize}
\item {Grp. gram.:m.}
\end{itemize}
O mesmo que \textunderscore bufarinheiro\textunderscore . Cf. Camillo, \textunderscore Quéda\textunderscore , 129.
\section{Belga}
\begin{itemize}
\item {Grp. gram.:f.}
\end{itemize}
\begin{itemize}
\item {Utilização:Prov.}
\end{itemize}
\begin{itemize}
\item {Grp. gram.:Pl.}
\end{itemize}
\begin{itemize}
\item {Utilização:Prov.}
\end{itemize}
\begin{itemize}
\item {Utilização:alg.}
\end{itemize}
\begin{itemize}
\item {Utilização:Prov.}
\end{itemize}
\begin{itemize}
\item {Utilização:alent.}
\end{itemize}
Pequeno campo cultivado; coirela; geira; secção de geira.
Cada uma das secções de um prédio rústico, separadas por batoréus, arretos, regos parallelos ou vallados.
Reunião de moreias.
Cada um dos regos parallelos, com que se divide o terreno, antes de lavrado, para que a semente se espalhe com a possível igualdade.
(Cp. cast. \textunderscore vega\textunderscore )
\section{Belga}
\begin{itemize}
\item {Grp. gram.:adj.}
\end{itemize}
\begin{itemize}
\item {Grp. gram.:M.}
\end{itemize}
\begin{itemize}
\item {Proveniência:(Lat. \textunderscore belgae\textunderscore )}
\end{itemize}
Relativo á Bélgica.
Habitante da Bélgica.
\section{Belgata}
\begin{itemize}
\item {Grp. gram.:f.}
\end{itemize}
Planta anti-febril da África Occidental, (\textunderscore andropogonnardus\textunderscore , Lin.).
Variedade de aguardente, muito apreciada nas nossas possessões da África Oriental.
\section{Belhão}
\begin{itemize}
\item {Grp. gram.:m.}
\end{itemize}
O mesmo ou melhor que \textunderscore bilhão\textunderscore ^2:«\textunderscore a prata se lhe tem convertido em cobre, e a fama e a opulência de tanto milhão, em belhão\textunderscore ». Vieira.
\section{Belharaco}
\begin{itemize}
\item {Grp. gram.:m.}
\end{itemize}
\begin{itemize}
\item {Utilização:T. de Ilhavo}
\end{itemize}
Beilhó com abóbora.
\section{Belho}
\begin{itemize}
\item {fónica:bê}
\end{itemize}
\begin{itemize}
\item {Grp. gram.:m.}
\end{itemize}
\begin{itemize}
\item {Utilização:Pop.}
\end{itemize}
Tranqueta, lingueta de fechadura.
Bedelho.
(Contr. de \textunderscore bedelho\textunderscore )
\section{Belhó}
\begin{itemize}
\item {Grp. gram.:m.}
\end{itemize}
(V.beilhó)
\section{Beliche}
\begin{itemize}
\item {Grp. gram.:m.}
\end{itemize}
Compartimento de camarote, ou de câmara de navio, onde se collocam as camas dos passageiros.
(Do mal.)
\section{Bélico}
\begin{itemize}
\item {Grp. gram.:adj.}
\end{itemize}
\begin{itemize}
\item {Proveniência:(Lat. \textunderscore bellicus\textunderscore )}
\end{itemize}
Relativo á guerra, próprio da guerra: \textunderscore apparato bélico\textunderscore .
\section{Belicosidade}
\begin{itemize}
\item {Grp. gram.:f.}
\end{itemize}
Qualidade de belicoso.
\section{Belicoso}
\begin{itemize}
\item {Grp. gram.:adj.}
\end{itemize}
\begin{itemize}
\item {Utilização:Prov.}
\end{itemize}
\begin{itemize}
\item {Utilização:trasm.}
\end{itemize}
\begin{itemize}
\item {Proveniência:(Lat. \textunderscore bellicosus\textunderscore )}
\end{itemize}
Que tem ânimo aguerrido.
Habituado á guerra.
Que incita á guerra.
Disposto para a guerra.
Rabugento, (falando-se de crianças).
\section{Belida}
\begin{itemize}
\item {Grp. gram.:f.}
\end{itemize}
Névoa, mancha esbranquiçada, na córnea do ôlho.
\section{Belieiro}
\begin{itemize}
\item {Grp. gram.:m.}
\end{itemize}
\begin{itemize}
\item {Utilização:Bras}
\end{itemize}
Espécie de bomba.
\section{Beligerância}
\begin{itemize}
\item {Grp. gram.:f.}
\end{itemize}
\begin{itemize}
\item {Utilização:Neol.}
\end{itemize}
Qualidade do que é \textunderscore beligerante\textunderscore .
\section{Beligerante}
\begin{itemize}
\item {Grp. gram.:adj.}
\end{itemize}
\begin{itemize}
\item {Proveniência:(Lat. \textunderscore belligerans\textunderscore )}
\end{itemize}
Que faz guerra; que está em guerra, em luta: \textunderscore os dois países beligerantes\textunderscore .
\section{Belígero}
\begin{itemize}
\item {Grp. gram.:adj.}
\end{itemize}
\begin{itemize}
\item {Utilização:Poét.}
\end{itemize}
\begin{itemize}
\item {Proveniência:(Lat. \textunderscore belliger\textunderscore )}
\end{itemize}
Belicoso.
Que serve na guerra.
\section{Belindre}
\begin{itemize}
\item {Grp. gram.:m.}
\end{itemize}
\begin{itemize}
\item {Utilização:T. de Lisbôa}
\end{itemize}
Pedrinha ou vidro redondo, com que se joga o alguergue.
Jôgo de rapazes, em que procuram ganhar tentos ou valores, acertando com a pedra numa pequena cavidade, aberta na terra, e impellindo com um dedo da mão direita o próprio belindre contra o do parceiro.
\section{Belino}
\begin{itemize}
\item {Grp. gram.:m.}
\end{itemize}
\begin{itemize}
\item {Utilização:Bras}
\end{itemize}
Espécie de videira.
\section{Belipotente}
\begin{itemize}
\item {Grp. gram.:adj.}
\end{itemize}
\begin{itemize}
\item {Proveniência:(Lat. \textunderscore bellipotens\textunderscore )}
\end{itemize}
Poderoso na guerra: \textunderscore a belipotente Inglaterra\textunderscore .
\section{Beliquete}
\begin{itemize}
\item {fónica:quê}
\end{itemize}
\begin{itemize}
\item {Grp. gram.:m.}
\end{itemize}
\begin{itemize}
\item {Utilização:Bras}
\end{itemize}
Pequeno compartimento, sem ar e sem luz, próprio para arrumação de trastes velhos.
Cafua.
(Cp. \textunderscore beliche\textunderscore )
\section{Belisária}
\begin{itemize}
\item {Grp. gram.:f.}
\end{itemize}
\begin{itemize}
\item {Proveniência:(De \textunderscore Belisario\textunderscore , n. p.)}
\end{itemize}
Pequena moéda, que o jogador feliz dá ao que perdeu tudo, para que este continue a jogar.
\section{Belisário}
\begin{itemize}
\item {Grp. gram.:adj.}
\end{itemize}
\begin{itemize}
\item {Proveniência:(De \textunderscore Belisário\textunderscore , n. p.)}
\end{itemize}
Pobre, desventurado. Cf. Filinto, XIII, 212.
\section{Belisca}
\begin{itemize}
\item {Grp. gram.:f.}
\end{itemize}
\begin{itemize}
\item {Utilização:Agr.}
\end{itemize}
\begin{itemize}
\item {Proveniência:(De \textunderscore beliscar\textunderscore )}
\end{itemize}
Acto de cortar com a unha o sarmento, antes da floração, para concentrar a seiva nos olhos que hão de constituir no anno seguinte os gomos fructíferos.
\section{Beliscadura}
\begin{itemize}
\item {Grp. gram.:f.}
\end{itemize}
Acção de \textunderscore beliscar\textunderscore .
\section{Beliscão}
\begin{itemize}
\item {Grp. gram.:m.}
\end{itemize}
O mesmo que \textunderscore beliscadura\textunderscore .
\section{Beliscar}
\begin{itemize}
\item {Grp. gram.:v. t.}
\end{itemize}
\begin{itemize}
\item {Proveniência:(Do lat. \textunderscore vellicare\textunderscore , de \textunderscore vellere\textunderscore ?)}
\end{itemize}
Apertar (a pelle) com as unhas dos dedos pollegar e indicador.
Ferir de leve; tocar levemente.
Estimular.
\section{Belisco}
\begin{itemize}
\item {Grp. gram.:m.}
\end{itemize}
(V.beliscadura)
\section{Belíssono}
\begin{itemize}
\item {Grp. gram.:adj}
\end{itemize}
\begin{itemize}
\item {Proveniência:(Lat. \textunderscore bellisonus\textunderscore )}
\end{itemize}
Que tem som guerreiro: \textunderscore a trombeta belisona\textunderscore .
\section{Belistreca}
\begin{itemize}
\item {Grp. gram.:f.}
\end{itemize}
\begin{itemize}
\item {Utilização:Prov.}
\end{itemize}
\begin{itemize}
\item {Utilização:beir.}
\end{itemize}
Rapariga buliçosa e activa.
\section{Beliz}
\begin{itemize}
\item {Grp. gram.:m.}
\end{itemize}
\begin{itemize}
\item {Utilização:P. us.}
\end{itemize}
\begin{itemize}
\item {Grp. gram.:Adj.}
\end{itemize}
\begin{itemize}
\item {Proveniência:(Do ár. \textunderscore iblis\textunderscore )}
\end{itemize}
Pessôa ladina, endiabrada.
Favorito.
Esperto; endiabrado.
\section{Bella}
\begin{itemize}
\item {Grp. gram.:f.}
\end{itemize}
Mulher bella: \textunderscore galantear as bellas\textunderscore .
(Fem. de \textunderscore bello\textunderscore )
\section{Bellacíssimo}
\begin{itemize}
\item {Grp. gram.:adj.}
\end{itemize}
\begin{itemize}
\item {Proveniência:(Do lat. \textunderscore bellax\textunderscore )}
\end{itemize}
Bellicoso; aguerrido.
\section{Bella-de-felgueiras}
\begin{itemize}
\item {Grp. gram.:f.}
\end{itemize}
Variedade de pêra, muito apreciada.
\section{Belladona}
\begin{itemize}
\item {Grp. gram.:f.}
\end{itemize}
\begin{itemize}
\item {Proveniência:(It. \textunderscore belladonna\textunderscore )}
\end{itemize}
Planta herbácea, venenosa e medicinal, da fam. das soláneas.
\section{Belladónio}
\begin{itemize}
\item {Grp. gram.:m.}
\end{itemize}
Extracto de belladona. Cf. \textunderscore Pharmacopeia Port.\textunderscore 
\section{Bella-feia}
\begin{itemize}
\item {Grp. gram.:f.}
\end{itemize}
Variedade de bôa pêra beirôa, que amadurece em Outubro.
\section{Bella-luísa}
\begin{itemize}
\item {Grp. gram.:f.}
\end{itemize}
\begin{itemize}
\item {Utilização:Prov.}
\end{itemize}
\begin{itemize}
\item {Utilização:alg.}
\end{itemize}
Planta, o mesmo que \textunderscore lúcia-lima\textunderscore .
\section{Bella-luz}
\begin{itemize}
\item {Grp. gram.:f.}
\end{itemize}
\begin{itemize}
\item {Utilização:Prov.}
\end{itemize}
Planta, (\textunderscore thymus masticina\textunderscore , Lin.).
\section{Bellamente}
\begin{itemize}
\item {Grp. gram.:adv.}
\end{itemize}
De modo \textunderscore bello\textunderscore .
Excellentemente.
\section{Bellasiano}
\begin{itemize}
\item {Grp. gram.:adj.}
\end{itemize}
\begin{itemize}
\item {Utilização:Geol.}
\end{itemize}
Diz-se de uma das espécies de terreno cretáceo.
\section{Bella-sombra}
\begin{itemize}
\item {Grp. gram.:f.}
\end{itemize}
Árvore de grande porte e lenho muito molle,
(\textunderscore pativeria dioica\textunderscore )
\section{Bellatrice}
\begin{itemize}
\item {Grp. gram.:f.}
\end{itemize}
O mesmo que \textunderscore bellatriz\textunderscore .
\section{Bellatriz}
\begin{itemize}
\item {Grp. gram.:f.}
\end{itemize}
\begin{itemize}
\item {Proveniência:(Lat. \textunderscore bellatrix\textunderscore )}
\end{itemize}
Mulher guerreira.
Grande estrêlla da constellação Oríon.
\section{Belletrista}
\begin{itemize}
\item {Grp. gram.:m.  e  f.}
\end{itemize}
\begin{itemize}
\item {Utilização:Neol.}
\end{itemize}
\begin{itemize}
\item {Proveniência:(Al. \textunderscore belletrist\textunderscore , do fr. \textunderscore belles-letres\textunderscore )}
\end{itemize}
Pessôa, que cultiva a belletrística.
\section{Belletrística}
\begin{itemize}
\item {Grp. gram.:f.}
\end{itemize}
\begin{itemize}
\item {Proveniência:(Al. \textunderscore belletristika\textunderscore , de \textunderscore belletrist\textunderscore )}
\end{itemize}
Designação das obras que constituem a chamada \textunderscore literatura amena\textunderscore , (romances, poesia, etc.).
\section{Belletrístico}
\begin{itemize}
\item {Grp. gram.:adj.}
\end{itemize}
Relativo á \textunderscore belletrística\textunderscore . Cf. C. Michaëlis, \textunderscore Estatinga\textunderscore , 12.
\section{Bellevália}
\begin{itemize}
\item {Grp. gram.:f.}
\end{itemize}
\begin{itemize}
\item {Proveniência:(De \textunderscore Belleval\textunderscore , n. p. de um bot. fr.)}
\end{itemize}
Gênero de plantas liliáceas.
\section{Belleza}
\begin{itemize}
\item {fónica:lê}
\end{itemize}
\begin{itemize}
\item {Grp. gram.:f.}
\end{itemize}
\begin{itemize}
\item {Utilização:Gír.}
\end{itemize}
Qualidade do que é bello, agradável ou que desperta admiração: \textunderscore a belleza do Sol\textunderscore .
Mulher bella: \textunderscore a Maria é uma belleza\textunderscore .
Coisa bella ou muito agradável: \textunderscore êste manjar está uma belleza\textunderscore .
Melena, cabelleira: \textunderscore o gajo cortou as bellezas\textunderscore .
\section{Béllico}
\begin{itemize}
\item {Grp. gram.:adj.}
\end{itemize}
\begin{itemize}
\item {Proveniência:(Lat. \textunderscore bellicus\textunderscore )}
\end{itemize}
Relativo á guerra, próprio da guerra: \textunderscore apparato béllico\textunderscore .
\section{Bellicosidade}
\begin{itemize}
\item {Grp. gram.:f.}
\end{itemize}
Qualidade de bellicoso.
\section{Bellicoso}
\begin{itemize}
\item {Grp. gram.:adj.}
\end{itemize}
\begin{itemize}
\item {Utilização:Prov.}
\end{itemize}
\begin{itemize}
\item {Utilização:trasm.}
\end{itemize}
\begin{itemize}
\item {Proveniência:(Lat. \textunderscore bellicosus\textunderscore )}
\end{itemize}
Que tem ânimo aguerrido.
Habituado á guerra.
Que incita á guerra.
Disposto para a guerra.
Rabugento, (falando-se de crianças).
\section{Belligerância}
\begin{itemize}
\item {Grp. gram.:f.}
\end{itemize}
\begin{itemize}
\item {Utilização:Neol.}
\end{itemize}
Qualidade do que é \textunderscore belligerante\textunderscore .
\section{Belligerante}
\begin{itemize}
\item {Grp. gram.:adj.}
\end{itemize}
\begin{itemize}
\item {Proveniência:(Lat. \textunderscore belligerans\textunderscore )}
\end{itemize}
Que faz guerra; que está em guerra, em luta: \textunderscore os dois países belligerantes\textunderscore .
\section{Bellígero}
\begin{itemize}
\item {Grp. gram.:adj.}
\end{itemize}
\begin{itemize}
\item {Utilização:Poét.}
\end{itemize}
\begin{itemize}
\item {Proveniência:(Lat. \textunderscore belliger\textunderscore )}
\end{itemize}
Bellicoso.
Que serve na guerra.
\section{Bellino}
\begin{itemize}
\item {Grp. gram.:m.}
\end{itemize}
\begin{itemize}
\item {Utilização:Bras}
\end{itemize}
Espécie de videira.
\section{Bellipotente}
\begin{itemize}
\item {Grp. gram.:adj.}
\end{itemize}
\begin{itemize}
\item {Proveniência:(Lat. \textunderscore bellipotens\textunderscore )}
\end{itemize}
Poderoso na guerra: \textunderscore a bellipotente Inglaterra\textunderscore .
\section{Bellisono}
\begin{itemize}
\item {fónica:so}
\end{itemize}
\begin{itemize}
\item {Grp. gram.:adj}
\end{itemize}
\begin{itemize}
\item {Proveniência:(Lat. \textunderscore bellisonus\textunderscore )}
\end{itemize}
Que tem som guerreiro: \textunderscore a trombeta bellisona\textunderscore .
\section{Bello}
\begin{itemize}
\item {Grp. gram.:m.}
\end{itemize}
\begin{itemize}
\item {Grp. gram.:Adj.}
\end{itemize}
\begin{itemize}
\item {Grp. gram.:Interj.}
\end{itemize}
\begin{itemize}
\item {Proveniência:(Lat. \textunderscore bellus\textunderscore )}
\end{itemize}
Perfeição.
Conjunto de qualidades, que despertam um sentimento elevado e especial de prazer e admiração: \textunderscore amar o bom e o bello\textunderscore .
Que tem fórma agradável.
Que tem proporções harmónicas: \textunderscore mulher bella\textunderscore .
Que agrada ao ouvido: \textunderscore bella música\textunderscore .
Bom: \textunderscore bello clima\textunderscore .
Elevado: \textunderscore bello talento\textunderscore .
Ameno.
Grande; vantajoso: \textunderscore bella herança\textunderscore .
Generoso, nobre: \textunderscore bello coração\textunderscore .
Que apraz ao coração e á intelligencia, como obra de arte.
Muito bem; excellentemente.
\section{Bello}
\begin{itemize}
\item {Grp. gram.:m.}
\end{itemize}
Árvore de Damão, (\textunderscore ogle marmelus\textunderscore ).
\section{Bellota}
\begin{itemize}
\item {Grp. gram.:f.}
\end{itemize}
O mesmo ou melhor que \textunderscore bolota\textunderscore .
\section{Belmandil}
\begin{itemize}
\item {Grp. gram.:f.}
\end{itemize}
Variedade de figueira algarvia.
\section{Belmaz}
\begin{itemize}
\item {Grp. gram.:adj.}
\end{itemize}
O mesmo que \textunderscore balmaz\textunderscore .
\section{Belo}
\begin{itemize}
\item {Grp. gram.:m.}
\end{itemize}
\begin{itemize}
\item {Grp. gram.:Adj.}
\end{itemize}
\begin{itemize}
\item {Grp. gram.:Interj.}
\end{itemize}
\begin{itemize}
\item {Proveniência:(Lat. \textunderscore bellus\textunderscore )}
\end{itemize}
Perfeição.
Conjunto de qualidades, que despertam um sentimento elevado e especial de prazer e admiração: \textunderscore amar o bom e o belo\textunderscore .
Que tem fórma agradável.
Que tem proporções harmónicas: \textunderscore mulher bela\textunderscore .
Que agrada ao ouvido: \textunderscore bela música\textunderscore .
Bom: \textunderscore belo clima\textunderscore .
Elevado: \textunderscore belo talento\textunderscore .
Ameno.
Grande; vantajoso: \textunderscore bela herança\textunderscore .
Generoso, nobre: \textunderscore belo coração\textunderscore .
Que apraz ao coração e á inteligencia, como obra de arte.
Muito bem; excelentemente.
\section{Belo}
\begin{itemize}
\item {Grp. gram.:m.}
\end{itemize}
Árvore de Damão, (\textunderscore ogle marmelus\textunderscore ).
\section{Belota}
\begin{itemize}
\item {Grp. gram.:f.}
\end{itemize}
O mesmo ou melhor que \textunderscore bolota\textunderscore .
\section{Bel-prazer}
\begin{itemize}
\item {Grp. gram.:m.}
\end{itemize}
Talante; vontade própria; arbítrio: \textunderscore procede a seu bel-prazer\textunderscore .
\section{Beltrano}
\begin{itemize}
\item {Grp. gram.:m.}
\end{itemize}
O mesmo que \textunderscore beltrão\textunderscore .
\section{Beltrão}
\begin{itemize}
\item {Grp. gram.:m.}
\end{itemize}
Fulano; certa pessôa; um \textunderscore quidam\textunderscore .
\section{Beluário}
\begin{itemize}
\item {Grp. gram.:m.}
\end{itemize}
\begin{itemize}
\item {Proveniência:(Do lat. \textunderscore belua\textunderscore )}
\end{itemize}
Antigo domador de feras.
Escravo, que tratava dos animaes destinados ao circo.
Homem, que nos circos combatia com as feras.
\section{Beluca}
\begin{itemize}
\item {Grp. gram.:f.}
\end{itemize}
Espécie de golfinho.
\section{Beluchi}
\begin{itemize}
\item {Grp. gram.:m.}
\end{itemize}
Aquelle que nasceu no Beluchistão.
Língua desta região. Cp. \textunderscore Baloches\textunderscore .
\section{Beluíno}
\begin{itemize}
\item {Grp. gram.:adj.}
\end{itemize}
\begin{itemize}
\item {Proveniência:(Lat. \textunderscore beluinus\textunderscore )}
\end{itemize}
Relativo a feras.
Selvagem; grosseiro.
\section{Beluoso}
\begin{itemize}
\item {Grp. gram.:adj.}
\end{itemize}
\begin{itemize}
\item {Proveniência:(Lat. \textunderscore beluosus\textunderscore )}
\end{itemize}
Abundante em feras: \textunderscore selvas beluosas\textunderscore .
\section{Belvedér}
\begin{itemize}
\item {Grp. gram.:m.}
\end{itemize}
O mesmo que \textunderscore belver\textunderscore .
\section{Belver}
\begin{itemize}
\item {Grp. gram.:m.}
\end{itemize}
\begin{itemize}
\item {Proveniência:(It. \textunderscore belvedere\textunderscore )}
\end{itemize}
Terraço, mirante, em parte elevada.
\section{Belverde}
\begin{itemize}
\item {Grp. gram.:m.}
\end{itemize}
\begin{itemize}
\item {Proveniência:(De \textunderscore bel\textunderscore  + \textunderscore verde\textunderscore )}
\end{itemize}
Planta ornamental, também conhecida por \textunderscore valverde\textunderscore .
\section{Belzebu}
\begin{itemize}
\item {Grp. gram.:m.}
\end{itemize}
\begin{itemize}
\item {Proveniência:(Do hebr. \textunderscore baal-zebub\textunderscore )}
\end{itemize}
Um dos demónios.
Demónio.
\section{Bem}
\begin{itemize}
\item {Grp. gram.:m.}
\end{itemize}
\begin{itemize}
\item {Grp. gram.:Pl.}
\end{itemize}
\begin{itemize}
\item {Grp. gram.:Adv.}
\end{itemize}
\begin{itemize}
\item {Grp. gram.:Interj.}
\end{itemize}
\begin{itemize}
\item {Proveniência:(Lat. \textunderscore bene\textunderscore )}
\end{itemize}
Aquillo que é bom ou conforme á moral.
Virtude.
Felicidade.
Utilidade; beneficio.
Pessôa namorada; derriço: \textunderscore és o meu bem\textunderscore .
Propriedade, domínio: \textunderscore herdar os bens de um parente\textunderscore .
Muito.
Convenientemente, com affeição: \textunderscore tratar bem\textunderscore .
Com saúde: \textunderscore tem passado bem\textunderscore .
Sim; apoiado; excellentemente.
\section{Bema}
\begin{itemize}
\item {Grp. gram.:f.}
\end{itemize}
Tribuna dos oradores gregos. Cf. Latino, \textunderscore Or. da Corôa\textunderscore , XI.
\section{Bem-andança}
\begin{itemize}
\item {Grp. gram.:f.}
\end{itemize}
\begin{itemize}
\item {Utilização:Ant.}
\end{itemize}
Felicidade, fortuna.
\section{Bem-aventuradamente}
\begin{itemize}
\item {Grp. gram.:adv.}
\end{itemize}
\begin{itemize}
\item {Proveniência:(De \textunderscore bem-aventurado\textunderscore )}
\end{itemize}
Com muita ventura.
\section{Bem-aventurado}
\begin{itemize}
\item {Grp. gram.:adj.}
\end{itemize}
\begin{itemize}
\item {Grp. gram.:M.}
\end{itemize}
Muito feliz: \textunderscore bem-aventurada criatura\textunderscore .
Aquelle que tem a felicidade celeste.
Santo.
\section{Bem-aventurança}
\begin{itemize}
\item {Grp. gram.:f.}
\end{itemize}
\begin{itemize}
\item {Grp. gram.:Pl.}
\end{itemize}
\begin{itemize}
\item {Proveniência:(De \textunderscore bem\textunderscore  + \textunderscore aventurar\textunderscore )}
\end{itemize}
Grande felicidade.
Felicidade celeste; o céu.
As oito virtudes evangélicas, preconizadas por Christo, para se conseguir a felicidade celeste.
\section{Bem-aventurar}
\begin{itemize}
\item {Grp. gram.:v. t.}
\end{itemize}
\begin{itemize}
\item {Proveniência:(De \textunderscore bem\textunderscore  + \textunderscore aventura\textunderscore )}
\end{itemize}
Tornar feliz.
Dar felicidade celeste a.
\section{Bem-avindo}
\begin{itemize}
\item {Grp. gram.:adj.}
\end{itemize}
Amigável.
Conciliado; que está em bôas relações.
\section{Bemba}
\begin{itemize}
\item {Grp. gram.:f.}
\end{itemize}
Tulha, em que os da Guiné guardam milho e arroz para engorda de animaes.
\section{Bembe}
\begin{itemize}
\item {Grp. gram.:m.}
\end{itemize}
\begin{itemize}
\item {Utilização:T. de Angola}
\end{itemize}
O mesmo que \textunderscore beldroéga\textunderscore .
\section{Bembom}
\begin{itemize}
\item {Grp. gram.:m.}
\end{itemize}
\begin{itemize}
\item {Utilização:Bras. de Minas}
\end{itemize}
Commodidade, bel-prazer: \textunderscore está no seu bembom\textunderscore .
\section{Bem-criado}
\begin{itemize}
\item {Grp. gram.:adj.}
\end{itemize}
\begin{itemize}
\item {Proveniência:(De \textunderscore bem\textunderscore  + \textunderscore criado\textunderscore )}
\end{itemize}
Polido, cortês; que tem bôa educação: \textunderscore um rapaz bem-criado\textunderscore .
\section{Bem-de-fala}
\begin{itemize}
\item {Grp. gram.:m.}
\end{itemize}
\begin{itemize}
\item {Utilização:Bras. do Rio}
\end{itemize}
Linguagem singela, desataviada, e sem malícia.
\section{Bem-dito}
\begin{itemize}
\item {Grp. gram.:m.}
\end{itemize}
\begin{itemize}
\item {Grp. gram.:Adj.}
\end{itemize}
Canto religioso, que principía por esta palavra.
Abençoado; feliz: \textunderscore bem-dito sejas\textunderscore .
(Por \textunderscore bendito\textunderscore , do lat. \textunderscore benedictus\textunderscore )
\section{Bem-dizente}
\begin{itemize}
\item {Grp. gram.:adj.}
\end{itemize}
\begin{itemize}
\item {Proveniência:(De \textunderscore bem-dizer\textunderscore )}
\end{itemize}
Que louva, bemdiz.
\section{Bem-dizer}
\begin{itemize}
\item {Grp. gram.:v. t.}
\end{itemize}
\begin{itemize}
\item {Proveniência:(De \textunderscore bem\textunderscore  + \textunderscore dizer\textunderscore )}
\end{itemize}
Dizer bem de.
Glorificar.
Abençoar; louvar.
\section{Bem-estar}
\begin{itemize}
\item {Grp. gram.:v. t.}
\end{itemize}
\begin{itemize}
\item {Proveniência:(De \textunderscore bem\textunderscore  + \textunderscore estar\textunderscore )}
\end{itemize}
Confôrto; estado em que nos sentimos bem, do corpo ou do espírito.
Haveres sufficientes para a commodidade da vida.
É gallicismo para alguns puristas.
\section{Bem-fadado}
\begin{itemize}
\item {Grp. gram.:adj.}
\end{itemize}
\begin{itemize}
\item {Proveniência:(De \textunderscore bem-fadar\textunderscore )}
\end{itemize}
Afortunado, feliz.
\section{Bem-fadar}
\begin{itemize}
\item {Grp. gram.:v. t.}
\end{itemize}
Fadar bem; predizer a bôa-fortuna de. Cf. Arn. Gama, \textunderscore Segr. do Abbade\textunderscore , 267.
\section{Bem-falante}
\begin{itemize}
\item {Grp. gram.:adj.}
\end{itemize}
Que fala bem.
\section{Bem-fazejo}
\begin{itemize}
\item {Grp. gram.:adj.}
\end{itemize}
\begin{itemize}
\item {Proveniência:(De \textunderscore bem-fazer\textunderscore )}
\end{itemize}
Caridoso; que faz bem.
Inclinado á piedade.
\section{Bem-fazente}
\begin{itemize}
\item {Grp. gram.:adj.}
\end{itemize}
O mesmo que \textunderscore bem-fazejo\textunderscore . Cf. D. Bernardes, \textunderscore Lima\textunderscore , ég. 1.^a
\section{Bem-fazer}
\begin{itemize}
\item {Grp. gram.:v. i.}
\end{itemize}
\begin{itemize}
\item {Grp. gram.:M.}
\end{itemize}
\begin{itemize}
\item {Proveniência:(De \textunderscore bem\textunderscore  + \textunderscore fazer\textunderscore )}
\end{itemize}
Fazer bem.
Benefício; caridade.
\section{Bem-feitor}
\begin{itemize}
\item {Grp. gram.:m.}
\end{itemize}
\begin{itemize}
\item {Proveniência:(Lat. \textunderscore benefactor\textunderscore )}
\end{itemize}
Aquelle que pratíca o bem; aquelle que beneficía.
Aquelle que bemfeitoriza propriedades.
\section{Bem-feitoria}
\begin{itemize}
\item {Grp. gram.:f.}
\end{itemize}
\begin{itemize}
\item {Utilização:Ant.}
\end{itemize}
Acto de melhorar um prédio.
Melhoramento, com que se torna mais rendoso um prédio.
Benefício.
(B. lat. \textunderscore benefactoria\textunderscore )
\section{Bem-feitorizar}
\begin{itemize}
\item {Grp. gram.:v. t.}
\end{itemize}
Fazer bemfeitorias em.
\section{Bem-me-quer}
\begin{itemize}
\item {Grp. gram.:m.}
\end{itemize}
Planta, da fam. das compostas, espécie de bonina.
Margarida dos prados.
\section{Bem-merecer}
\begin{itemize}
\item {Grp. gram.:v. i.}
\end{itemize}
Sêr digno de recompensas, de honras.
\section{Bem-nado}
\begin{itemize}
\item {Grp. gram.:adj.}
\end{itemize}
O mesmo que \textunderscore bem-nascido\textunderscore .
\section{Bem-nascido}
\begin{itemize}
\item {Grp. gram.:adj.}
\end{itemize}
Nascido para bem; bem fadado.
Nobre.
\section{Bemol}
\begin{itemize}
\item {Grp. gram.:m.}
\end{itemize}
Sinal musical, indicativo de que uma nota deve baixar meio tom.
(Do it.)
\section{Bem-parado}
\begin{itemize}
\item {Grp. gram.:adj.}
\end{itemize}
\begin{itemize}
\item {Utilização:Des.}
\end{itemize}
Que tem bôa sorte.
Afortunado.
Bem sabido.
Que tem grande e bom saber.
\section{Bem-parecido}
\begin{itemize}
\item {Grp. gram.:adj.}
\end{itemize}
Que parece bem.
Formoso.
Bem pôsto.
\section{Bemposta}
\begin{itemize}
\item {Grp. gram.:f.}
\end{itemize}
Variedade de maçan.
\section{Bem-que}
\begin{itemize}
\item {Grp. gram.:loc. conj.}
\end{itemize}
Pôsto que.
\section{Bem-querença}
\begin{itemize}
\item {Grp. gram.:f.}
\end{itemize}
\begin{itemize}
\item {Proveniência:(De \textunderscore bem-querer\textunderscore )}
\end{itemize}
Benevolência; sentimento de affeição.
\section{Bem-querente}
\begin{itemize}
\item {Grp. gram.:adj.}
\end{itemize}
\begin{itemize}
\item {Proveniência:(De \textunderscore bem-querer\textunderscore )}
\end{itemize}
Que quere bem; benévolo.
\section{Bem-querer}
\begin{itemize}
\item {Grp. gram.:v. t.}
\end{itemize}
\begin{itemize}
\item {Proveniência:(De \textunderscore bem\textunderscore  + \textunderscore querer\textunderscore )}
\end{itemize}
Querer bem; estimar muito; amar.
\section{Bem-quistar}
\begin{itemize}
\item {Grp. gram.:v. t.}
\end{itemize}
\begin{itemize}
\item {Proveniência:(De \textunderscore bem-quisto\textunderscore )}
\end{itemize}
Tornar bem-quisto; conciliar.
\section{Bem-quisto}
\begin{itemize}
\item {Grp. gram.:adj.}
\end{itemize}
Estimado; prezado; bem acceito.
\section{Bem-soante}
\begin{itemize}
\item {Grp. gram.:adj.}
\end{itemize}
\begin{itemize}
\item {Proveniência:(De \textunderscore bem\textunderscore  + \textunderscore soante\textunderscore )}
\end{itemize}
Que sôa bem.
\section{Bem-tere}
\begin{itemize}
\item {Grp. gram.:m.}
\end{itemize}
Ave brasileira.
\section{Bem-te-vi}
\begin{itemize}
\item {Grp. gram.:m.}
\end{itemize}
\begin{itemize}
\item {Utilização:Bras}
\end{itemize}
\begin{itemize}
\item {Proveniência:(T. onom.)}
\end{itemize}
Gênero de pássaros insectívoros do Brasil.
Parcialidade política do Maranhão.
\section{Bem-vindo}
\begin{itemize}
\item {Grp. gram.:adj.}
\end{itemize}
\begin{itemize}
\item {Proveniência:(De \textunderscore bem\textunderscore  + \textunderscore vindo\textunderscore )}
\end{itemize}
Que chegou bem, felizmente.
Bem recebido á sua chegada.
\section{Bem-visto}
\begin{itemize}
\item {Grp. gram.:adj.}
\end{itemize}
\begin{itemize}
\item {Proveniência:(De \textunderscore bem\textunderscore  + \textunderscore visto\textunderscore )}
\end{itemize}
Bem considerado; estimado.
\section{Bemzinho-amor}
\begin{itemize}
\item {Grp. gram.:m.}
\end{itemize}
\begin{itemize}
\item {Utilização:Bras}
\end{itemize}
Bailarico, espécie de fandango.
\section{Benairo}
\begin{itemize}
\item {Grp. gram.:m.}
\end{itemize}
\begin{itemize}
\item {Utilização:Prov.}
\end{itemize}
\begin{itemize}
\item {Utilização:trasm.}
\end{itemize}
Trapo.
Pedaço de qualquer coisa.
\section{Bênção}
\begin{itemize}
\item {Grp. gram.:f.}
\end{itemize}
Acção de benzer, de abençoar.
Favor divino.
Palavras e sentimentos de gratidão.
(Contr. de \textunderscore bendição\textunderscore , com deslocação de accento)
\section{Bênção-de-Deus}
\begin{itemize}
\item {Grp. gram.:f.}
\end{itemize}
\begin{itemize}
\item {Utilização:Bras}
\end{itemize}
Planta malvácea do Brasil, (\textunderscore abutilon esculentum\textunderscore , St. Hil.).
Bailado popular.
\section{Bênçoa}
\begin{itemize}
\item {Grp. gram.:f.}
\end{itemize}
(Fórma pop. de \textunderscore bênção\textunderscore )
\section{Bençoairo}
\begin{itemize}
\item {Grp. gram.:m.}
\end{itemize}
\begin{itemize}
\item {Utilização:Ant.}
\end{itemize}
Relação de bens, adquiridos por qualquer titulo.
(Por \textunderscore bensoairo\textunderscore , de \textunderscore bens\textunderscore )
\section{Bençoar}
\textunderscore v. t.\textunderscore  (e der.)
O mesmo que \textunderscore abençoar\textunderscore , etc. Cf. Filinto, IX, 26 e 217.
\section{Bendé}
\begin{itemize}
\item {Grp. gram.:m.}
\end{itemize}
Planta hortense da Índia, (\textunderscore hibiscus esculentus\textunderscore ).
\section{Bendenguê}
\textunderscore m. Bras.\textunderscore  de \textunderscore Cabofrio\textunderscore .
Dança de negros, ao som de cantigas africanas.
(Talvez do quimb.)
\section{Bendi}
\begin{itemize}
\item {Grp. gram.:m.}
\end{itemize}
Árvore indiana, (\textunderscore thespesia populnea\textunderscore ), o mesmo que \textunderscore pau-rosa\textunderscore .
\section{Bendição}
\begin{itemize}
\item {Grp. gram.:f.}
\end{itemize}
\begin{itemize}
\item {Utilização:Ant.}
\end{itemize}
O mesmo que \textunderscore bênção\textunderscore .
\section{Bendito}
\begin{itemize}
\item {Grp. gram.:adj.}
\end{itemize}
(Fórma exacta, em vez de \textunderscore bem-dito\textunderscore . Cf. \textunderscore Peregrinação\textunderscore , 299 e 335)
\section{Benedicta}
\begin{itemize}
\item {Grp. gram.:f.}
\end{itemize}
\begin{itemize}
\item {Proveniência:(Lat. \textunderscore benedicta\textunderscore )}
\end{itemize}
Antigo medicamento purgativo.
\section{Benedictinas}
\begin{itemize}
\item {Grp. gram.:f. pl.}
\end{itemize}
Freiras da regra de San-Bento.
\section{Benedictino}
\begin{itemize}
\item {Grp. gram.:m.}
\end{itemize}
\begin{itemize}
\item {Utilização:Ext.}
\end{itemize}
\begin{itemize}
\item {Grp. gram.:Adj.}
\end{itemize}
\begin{itemize}
\item {Proveniência:(De \textunderscore Benedictus\textunderscore , n. p.)}
\end{itemize}
Frade da Ordem de San-Bento.
Homem erudito e incansável no estudo.
Relativo aos benedictinos ou próprio delles: \textunderscore paciência benedictina\textunderscore .
\section{Benedita}
\begin{itemize}
\item {Grp. gram.:f.}
\end{itemize}
\begin{itemize}
\item {Grp. gram.:f.}
\end{itemize}
\begin{itemize}
\item {Utilização:Ant.}
\end{itemize}
\begin{itemize}
\item {Proveniência:(Lat. \textunderscore benedicta\textunderscore )}
\end{itemize}
Antigo medicamento purgativo.
Lábia, palavras convincentes:«\textunderscore e por aqui lhe disse minhas beneditas\textunderscore ». \textunderscore Eufrosina\textunderscore , act. I, sc. 3.
\section{Beneditinas}
\begin{itemize}
\item {Grp. gram.:f. pl.}
\end{itemize}
Freiras da regra de San-Bento.
\section{Beneditino}
\begin{itemize}
\item {Grp. gram.:m.}
\end{itemize}
\begin{itemize}
\item {Utilização:Ext.}
\end{itemize}
\begin{itemize}
\item {Grp. gram.:Adj.}
\end{itemize}
\begin{itemize}
\item {Proveniência:(De \textunderscore Benedictus\textunderscore , n. p.)}
\end{itemize}
Frade da Ordem de San-Bento.
Homem erudito e incansável no estudo.
Relativo aos benedictinos ou próprio delles: \textunderscore paciência benedictina\textunderscore .
\section{Benefactoria}
\begin{itemize}
\item {Grp. gram.:f.}
\end{itemize}
\begin{itemize}
\item {Utilização:Ant.}
\end{itemize}
O mesmo que \textunderscore bem-feitoria\textunderscore .
\section{Benefe}
\begin{itemize}
\item {Grp. gram.:f.}
\end{itemize}
Violeta brava.
\section{Beneficência}
\begin{itemize}
\item {Grp. gram.:f.}
\end{itemize}
\begin{itemize}
\item {Proveniência:(Lat. \textunderscore beneficentia\textunderscore )}
\end{itemize}
Acto de beneficiar.
Hábito de fazer bem.
Auxílio.
Prática de obras de caridade ou philantropia.
\section{Beneficente}
\begin{itemize}
\item {Grp. gram.:adj.}
\end{itemize}
Que beneficía.
(Cp. \textunderscore beneficência\textunderscore )
\section{Beneficiação}
\begin{itemize}
\item {Grp. gram.:f.}
\end{itemize}
Acção de \textunderscore beneficiar\textunderscore .
\section{Beneficiado}
\begin{itemize}
\item {Grp. gram.:m.}
\end{itemize}
\begin{itemize}
\item {Proveniência:(De \textunderscore beneficiar\textunderscore )}
\end{itemize}
Aquelle que tem benefício ecclesiástico: \textunderscore os beneficiados da sé\textunderscore .
Pessôa, para quem reverte o producto de um espectáculo de benefício.
\section{Beneficiador}
\begin{itemize}
\item {Grp. gram.:m.  e  adj.}
\end{itemize}
O que beneficía.
\section{Beneficial}
\begin{itemize}
\item {Grp. gram.:adj.}
\end{itemize}
\begin{itemize}
\item {Proveniência:(Lat. \textunderscore beneficialis\textunderscore )}
\end{itemize}
Relativo a benefícios ecclesiásticos.
\section{Beneficiar}
\begin{itemize}
\item {Grp. gram.:v. t.}
\end{itemize}
Fazer benefício a.
Melhorar: \textunderscore beneficiar um prédio\textunderscore .
Concertar.
\section{Beneficiário}
\begin{itemize}
\item {Grp. gram.:adj.}
\end{itemize}
\begin{itemize}
\item {Utilização:Jur.}
\end{itemize}
\begin{itemize}
\item {Proveniência:(Lat. \textunderscore beneficiarius\textunderscore )}
\end{itemize}
Diz-se do herdeiro, que acceita a herança a benefício de inventário.
\section{Beneficiável}
\begin{itemize}
\item {Grp. gram.:adj.}
\end{itemize}
\begin{itemize}
\item {Proveniência:(De \textunderscore beneficiar\textunderscore )}
\end{itemize}
Que póde ou merece sêr beneficiado.
\section{Benefício}
\begin{itemize}
\item {Grp. gram.:m.}
\end{itemize}
\begin{itemize}
\item {Proveniência:(Lat. \textunderscore beneficium\textunderscore )}
\end{itemize}
Bem, serviço, que se faz gratuitamente.
Favor, mercê.
Vantagem.
Cargo ecclesiástico, nas sés.
Ganho.
Espectáculo público, cuja receita reverte a favor de alguém, que não é a Empresa: \textunderscore é hoje o benefício do actor Melo\textunderscore .
Melhoramento.
Bem-feitoria.
\section{Beneficioso}
\begin{itemize}
\item {Grp. gram.:adj.}
\end{itemize}
O mesmo que \textunderscore benéfico\textunderscore .
\section{Benéfico}
\begin{itemize}
\item {Grp. gram.:adj.}
\end{itemize}
\begin{itemize}
\item {Proveniência:(Lat. \textunderscore beneficus\textunderscore )}
\end{itemize}
Que faz bem.
Bondoso.
Salutar.
\section{Benemerência}
\begin{itemize}
\item {Grp. gram.:f.}
\end{itemize}
\begin{itemize}
\item {Proveniência:(De \textunderscore benemerente\textunderscore )}
\end{itemize}
Qualidade de quem é benemérito.
\section{Benemerente}
\begin{itemize}
\item {Grp. gram.:adj.}
\end{itemize}
\begin{itemize}
\item {Proveniência:(Lat. \textunderscore bene\textunderscore  + \textunderscore merens\textunderscore )}
\end{itemize}
Que bem-merece ou é digno de recompensa ou applauso.
\section{Benemérito}
\begin{itemize}
\item {Grp. gram.:adj.}
\end{itemize}
\begin{itemize}
\item {Grp. gram.:M.}
\end{itemize}
\begin{itemize}
\item {Proveniência:(Lat. \textunderscore bene\textunderscore  + \textunderscore meritus\textunderscore )}
\end{itemize}
Que é digno de honras, louvores e recompensas, por serviços importantes.
Aquelle que é digno de honras ou recompensas.
\section{Beneplácito}
\begin{itemize}
\item {Grp. gram.:m.}
\end{itemize}
\begin{itemize}
\item {Proveniência:(Lat. \textunderscore beneplacitum\textunderscore )}
\end{itemize}
Approvação.
Consentimento; licença: \textunderscore a execução das bullas pontifícias depende do beneplácito do Estado\textunderscore .
\section{Benesse}
\begin{itemize}
\item {Grp. gram.:m.}
\end{itemize}
\begin{itemize}
\item {Proveniência:(Do lat. \textunderscore bene\textunderscore  + \textunderscore esse\textunderscore )}
\end{itemize}
Rendimento do pé-de-altar.
Sinecura; lucro, que não depende de trabalho.
\section{Benevolamente}
\begin{itemize}
\item {Grp. gram.:adv.}
\end{itemize}
De modo \textunderscore benévolo\textunderscore .
Com benevolência.
\section{Benevolência}
\begin{itemize}
\item {Grp. gram.:f.}
\end{itemize}
\begin{itemize}
\item {Proveniência:(Lat. \textunderscore benevolentia\textunderscore )}
\end{itemize}
Qualidade daquelle que é benévolo.
\section{Benevolente}
\begin{itemize}
\item {Grp. gram.:adj.}
\end{itemize}
\begin{itemize}
\item {Proveniência:(Lat. \textunderscore benevolens\textunderscore )}
\end{itemize}
O mesmo que \textunderscore benévolo\textunderscore .
\section{Benevolentemente}
\begin{itemize}
\item {Grp. gram.:adv.}
\end{itemize}
\begin{itemize}
\item {Proveniência:(De \textunderscore benevolente\textunderscore )}
\end{itemize}
Com benevolência.
\section{Benévolo}
\begin{itemize}
\item {Grp. gram.:adj.}
\end{itemize}
\begin{itemize}
\item {Proveniência:(Lat. \textunderscore benevolus\textunderscore )}
\end{itemize}
Que quere bem.
Bondoso.
Que tem disposições favoráveis; bem intencionado.
\section{Bengala}
\begin{itemize}
\item {Grp. gram.:f.}
\end{itemize}
\begin{itemize}
\item {Proveniência:(De \textunderscore Bengala\textunderscore , n. p.)}
\end{itemize}
Pequeno bastão, feito de cana de Bengala ou da Índia.
Qualquer outro pequeno bastão, em que apoiamos a mão, quando andamos.
Árvore brasileira.
\section{Bengalada}
\begin{itemize}
\item {Grp. gram.:f.}
\end{itemize}
Pancada com bengala.
\section{Bengalão}
\begin{itemize}
\item {Grp. gram.:m.}
\end{itemize}
Bengala grande e pesada:«\textunderscore o bengalão do quadrilheiro\textunderscore ». Herculano, \textunderscore Quest. Púb.\textunderscore , I, 280.
\section{Bengalas}
\begin{itemize}
\item {Grp. gram.:m. pl.}
\end{itemize}
Habitantes de Bengala.
\section{Bengalé}
\begin{itemize}
\item {Grp. gram.:f.}
\end{itemize}
Motim, chinfrim?:«\textunderscore uma bengalé de scelerados\textunderscore ». Camillo, \textunderscore Corja\textunderscore , 266.
\section{Bengaleira}
\begin{itemize}
\item {Grp. gram.:f.}
\end{itemize}
\begin{itemize}
\item {Proveniência:(De \textunderscore Bengala\textunderscore , n. p.)}
\end{itemize}
Cana da Índia.
\section{Bengaleiro}
\begin{itemize}
\item {Grp. gram.:m.}
\end{itemize}
Fabricante ou vendedor de bengalas.
Empregado que, á entrada dos theatros, guarda as bengalas dos espectadores.
Lugar, onde se guardam bengalas.
\section{Bengali}
\begin{itemize}
\item {Grp. gram.:m.}
\end{itemize}
Dialecto de Bengala.
Espécie de tentilhão de Bengala.
\section{Bengalinha}
\begin{itemize}
\item {Grp. gram.:f.}
\end{itemize}
Ave da Ásia e da África, notável por sua plumagem e pela meia-lua escarlate, que tem de cada lado da cabeça.
\section{Bengalório}
\begin{itemize}
\item {Grp. gram.:m.}
\end{itemize}
O mesmo que \textunderscore bengalão\textunderscore .
\section{Bengue}
\begin{itemize}
\item {Grp. gram.:m.}
\end{itemize}
Fôlha de cânhamo, que na Sibéria se fuma como tabaco.
\section{Bengue-de-obó}
\begin{itemize}
\item {Grp. gram.:m.}
\end{itemize}
Árvore medicinal da ilha de San-Thomé.
\section{Benguelas}
\begin{itemize}
\item {Grp. gram.:m. pl.}
\end{itemize}
Habitantes de Benguela, em Angola.
\section{Benguelinha}
\begin{itemize}
\item {Grp. gram.:f.}
\end{itemize}
(V.bengalinha)
\section{Benignamente}
\begin{itemize}
\item {Grp. gram.:adv.}
\end{itemize}
\begin{itemize}
\item {Proveniência:(De \textunderscore benigno\textunderscore )}
\end{itemize}
Com benignidade.
\section{Benignidade}
\begin{itemize}
\item {Grp. gram.:f.}
\end{itemize}
\begin{itemize}
\item {Proveniência:(Lat. \textunderscore benignitas\textunderscore )}
\end{itemize}
Qualidade do que é benigno.
\section{Benigno}
\begin{itemize}
\item {Grp. gram.:adj.}
\end{itemize}
\begin{itemize}
\item {Proveniência:(Lat. \textunderscore benignus\textunderscore )}
\end{itemize}
Benévolo; bondoso; affectuoso.
Favorável.
Suave.
Que não é perigoso: \textunderscore clima benigno\textunderscore .
\section{Benincasa}
\begin{itemize}
\item {Grp. gram.:f.}
\end{itemize}
Planta indiana, herbácea, da fam. das cucurbitáceas.
\section{Benino}
\begin{itemize}
\item {Grp. gram.:adj.}
\end{itemize}
(Fórma ant. de \textunderscore benigno\textunderscore . Cf. D. Bernardes, \textunderscore Lima\textunderscore , 68; Usque, \textunderscore Tribulações\textunderscore , 22, v.^o)
\section{Benjamim}
\begin{itemize}
\item {Grp. gram.:m.}
\end{itemize}
\begin{itemize}
\item {Utilização:Pop.}
\end{itemize}
\begin{itemize}
\item {Proveniência:(De \textunderscore Benjamim\textunderscore , n. p.)}
\end{itemize}
O filho preferido.
Criança amimada.
\section{Benjoeiro}
\begin{itemize}
\item {Grp. gram.:m.}
\end{itemize}
Árvore, que produz o benjoim.
\section{Benjoim}
\begin{itemize}
\item {Grp. gram.:m.}
\end{itemize}
O mesmo ou melhor que \textunderscore beijoim\textunderscore .
\section{Benodáctilo}
\begin{itemize}
\item {Grp. gram.:adj.}
\end{itemize}
\begin{itemize}
\item {Proveniência:(Do gr. \textunderscore baino\textunderscore  + \textunderscore daktulos\textunderscore )}
\end{itemize}
Diz-se dos animaes, que caminham sôbre os dedos.
\section{Benodáctylo}
\begin{itemize}
\item {Grp. gram.:adj.}
\end{itemize}
\begin{itemize}
\item {Proveniência:(Do gr. \textunderscore baino\textunderscore  + \textunderscore daktulos\textunderscore )}
\end{itemize}
Diz-se dos animaes, que caminham sôbre os dedos.
\section{Benta}
\begin{itemize}
\item {Grp. gram.:f.}
\end{itemize}
\begin{itemize}
\item {Utilização:Prov.}
\end{itemize}
\begin{itemize}
\item {Utilização:beir.}
\end{itemize}
\begin{itemize}
\item {Utilização:Prov.}
\end{itemize}
Furúnculo de mau carácter.
Bruxa; feiticeira.
\section{Bentâmia}
\begin{itemize}
\item {Grp. gram.:f.}
\end{itemize}
\begin{itemize}
\item {Proveniência:(De \textunderscore Bentham\textunderscore , n. p. de um bot. ingl.)}
\end{itemize}
Gênero de plantas araliáceas.
\section{Benteca}
\begin{itemize}
\item {Grp. gram.:f.}
\end{itemize}
Árvore da Guiné portuguesa.
\section{Benthâmia}
\begin{itemize}
\item {Grp. gram.:f.}
\end{itemize}
\begin{itemize}
\item {Proveniência:(De \textunderscore Bentham\textunderscore , n. p. de um bot. ingl.)}
\end{itemize}
Gênero de plantas araliáceas.
\section{Bentinho}
\begin{itemize}
\item {Grp. gram.:m.}
\end{itemize}
\begin{itemize}
\item {Utilização:Bras}
\end{itemize}
O mesmo que \textunderscore bentinhos\textunderscore .
\section{Bentinhos}
\begin{itemize}
\item {Grp. gram.:m. pl.}
\end{itemize}
\begin{itemize}
\item {Utilização:Prov.}
\end{itemize}
\begin{itemize}
\item {Utilização:alent.}
\end{itemize}
\begin{itemize}
\item {Proveniência:(De \textunderscore bento\textunderscore ^1)}
\end{itemize}
Escapulário, formado de dois pequenos quadrados de pano bento, unidos por fitas, e que as pessôas devotas trazem ao pescoço.
Alforges pequenos.
\section{Bento}
\begin{itemize}
\item {Grp. gram.:adj.}
\end{itemize}
\begin{itemize}
\item {Grp. gram.:M.}
\end{itemize}
\begin{itemize}
\item {Utilização:Prov.}
\end{itemize}
\begin{itemize}
\item {Utilização:beir.}
\end{itemize}
Consagrado pela bênção ecclesiástica: \textunderscore pão bento\textunderscore .
O mesmo que \textunderscore benzedeiro\textunderscore .
(Part. irr. de \textunderscore benzer\textunderscore )
\section{Bento}
\begin{itemize}
\item {Grp. gram.:m.}
\end{itemize}
Frade benedictino.
\section{Benzedeira}
\textunderscore fem.\textunderscore  de \textunderscore benzedeiro\textunderscore .
\section{Benzedeiro}
\begin{itemize}
\item {Grp. gram.:m.}
\end{itemize}
\begin{itemize}
\item {Proveniência:(De \textunderscore benzer\textunderscore )}
\end{itemize}
Aquelle que procura livrar de doenças e feitiços as pessôas que benze.
Feiticeiro, bruxo.
\section{Benzedela}
\begin{itemize}
\item {Grp. gram.:f.}
\end{itemize}
O mesmo que \textunderscore benzedura\textunderscore .
\section{Benzedor}
\begin{itemize}
\item {Grp. gram.:m.}
\end{itemize}
Aquelle que benze; benzedeiro.
\section{Benzedura}
\begin{itemize}
\item {Grp. gram.:f.}
\end{itemize}
Acção de benzer, acompanhada de rezas supersticiosas.
\section{Benzênico}
\begin{itemize}
\item {Grp. gram.:adj.}
\end{itemize}
\begin{itemize}
\item {Utilização:Chím.}
\end{itemize}
Diz-se de um grupo de carbonetos.
\section{Benzeno}
\begin{itemize}
\item {Grp. gram.:m.}
\end{itemize}
\begin{itemize}
\item {Utilização:Chím.}
\end{itemize}
Um dos carbonetos do grupo benzênico.
\section{Benzer}
\begin{itemize}
\item {Grp. gram.:v. t.}
\end{itemize}
\begin{itemize}
\item {Grp. gram.:V. p.}
\end{itemize}
\begin{itemize}
\item {Utilização:Pop.}
\end{itemize}
\begin{itemize}
\item {Grp. gram.:V. i.}
\end{itemize}
Deitar a bênção sobre: \textunderscore benzer uma bandeira\textunderscore .
Consagrar ao culto, por meio do sinal da cruz ou de outras ceremónias: \textunderscore benzer uma igreja\textunderscore .
Abençoar, tornar feliz.
Fazer uma cruz com a mão direita aberta, da testa ao peito e do ombro esquerdo ao direito.
Admirar-se: \textunderscore quando tal lhe contei, benzeu-se\textunderscore .
Estrear-se.
Fazer benzeduras.
(Contr. de \textunderscore bem-dizer\textunderscore , do lat. \textunderscore benedicere\textunderscore )
\section{Barbarisco}
\begin{itemize}
\item {Grp. gram.:m.}
\end{itemize}
\begin{itemize}
\item {Utilização:T. de Caminha}
\end{itemize}
O mesmo que \textunderscore berbigão\textunderscore .
\section{Benzido}
\begin{itemize}
\item {Grp. gram.:adj.}
\end{itemize}
O mesmo que \textunderscore bento\textunderscore ^1.
\section{Benzilhão}
\begin{itemize}
\item {Grp. gram.:m.}
\end{itemize}
O mesmo que \textunderscore benzedeiro\textunderscore . Cf. Castilho, \textunderscore Fastos\textunderscore , III, 359.
\section{Benzimento}
\begin{itemize}
\item {Grp. gram.:m.}
\end{itemize}
\begin{itemize}
\item {Utilização:Des.}
\end{itemize}
Acto de \textunderscore benzer\textunderscore : \textunderscore o benzimento do adro\textunderscore .
\section{Benzina}
\begin{itemize}
\item {Grp. gram.:f.}
\end{itemize}
\begin{itemize}
\item {Proveniência:(Do lat. bot. \textunderscore benzoe\textunderscore , nome do \textunderscore benjoim\textunderscore )}
\end{itemize}
Líquido volátil, que se extrai do alcatrão da hulha e serve para tirar nódoas.
\section{Benzoato}
\begin{itemize}
\item {Grp. gram.:m.}
\end{itemize}
\begin{itemize}
\item {Proveniência:(Do lat. bot. \textunderscore benzoe\textunderscore )}
\end{itemize}
Sal, resultante da combinação do ácido benzoico com uma base.
\section{Benzoico}
\begin{itemize}
\item {Grp. gram.:m.}
\end{itemize}
\begin{itemize}
\item {Proveniência:(Do lat. bot. \textunderscore benzoe\textunderscore )}
\end{itemize}
Diz-se do ácido extrahido do benjoim.
\section{Benzoína}
\begin{itemize}
\item {Grp. gram.:f.}
\end{itemize}
\begin{itemize}
\item {Proveniência:(Lat. bot. \textunderscore benzoine\textunderscore )}
\end{itemize}
Espécie de cânfora, formada no óleo de amêndoas amargas, sujeitas á acção da potassa e de outros córpos.
\section{Benzola}
\begin{itemize}
\item {Grp. gram.:f.}
\end{itemize}
Líquido oleoso, incolor, obtido pela destillação sêca de uma parte de ácido benzoico, crystallizado, e de três partes de hydrato de cal.
\section{Benzolina}
\begin{itemize}
\item {Grp. gram.:f.}
\end{itemize}
(V.benzola)
\section{Benzona}
\begin{itemize}
\item {Grp. gram.:f.}
\end{itemize}
\begin{itemize}
\item {Proveniência:(Do lat. bot. \textunderscore benzoe\textunderscore )}
\end{itemize}
Óleo, que se obtém pela destillação do benzoato de cal.
\section{Beócio}
\begin{itemize}
\item {Grp. gram.:adj.}
\end{itemize}
\begin{itemize}
\item {Grp. gram.:M.}
\end{itemize}
\begin{itemize}
\item {Utilização:Fam.}
\end{itemize}
\begin{itemize}
\item {Proveniência:(Lat. \textunderscore boeotius\textunderscore )}
\end{itemize}
Relativo á Beócia.
Dialecto da Beócia.
Habitante desta região.
Indivíduo ignorante.
\section{Bequadro}
\begin{itemize}
\item {Grp. gram.:m.}
\end{itemize}
Accidente musical, que desfaz a alteração produzida por sustenido ou bemol, repondo a nota no tom natural.
(B. lat. \textunderscore B. quadratus\textunderscore )
\section{Beque}
\begin{itemize}
\item {Grp. gram.:m.}
\end{itemize}
\begin{itemize}
\item {Utilização:Náut.}
\end{itemize}
\begin{itemize}
\item {Utilização:Pop.}
\end{itemize}
\begin{itemize}
\item {Utilização:ant.}
\end{itemize}
\begin{itemize}
\item {Utilização:Gír.}
\end{itemize}
\begin{itemize}
\item {Proveniência:(Fr. \textunderscore bec\textunderscore . Cp. \textunderscore bico\textunderscore ^1)}
\end{itemize}
Extremidade superior da prôa.
Nariz; grande nariz.
Parte posterior do vestido das mulheres.
Bôca.
\section{Béquico}
\begin{itemize}
\item {Grp. gram.:m.  e  adj.}
\end{itemize}
\begin{itemize}
\item {Proveniência:(Gr. \textunderscore bekhikos\textunderscore )}
\end{itemize}
Aquillo que é bom contra a tosse.
\section{Bér}
\begin{itemize}
\item {Grp. gram.:m.}
\end{itemize}
Árvore da Índia portuguesa.
\section{Berbequim}
\begin{itemize}
\item {Grp. gram.:m.}
\end{itemize}
\begin{itemize}
\item {Proveniência:(Fr. \textunderscore vilebrequin\textunderscore )}
\end{itemize}
Broca, formada de uma haste com ponta de aço, a que se dá movimento por meio de uma manivela, e que serve para furar madeira, metal, pedra ou loiça.
\section{Berbér}
\begin{itemize}
\item {Grp. gram.:m.}
\end{itemize}
\begin{itemize}
\item {Grp. gram.:Pl.}
\end{itemize}
Língua dos Berbéres.
Habitantes da Berberia, na África do Norte.
(Ár. \textunderscore berber\textunderscore , talvez do lat. \textunderscore barbarus\textunderscore )
\section{Berbére}
\begin{itemize}
\item {Grp. gram.:m.}
\end{itemize}
\begin{itemize}
\item {Grp. gram.:Pl.}
\end{itemize}
Língua dos Berbéres.
Habitantes da Berberia, na África do Norte.
(Ár. \textunderscore berber\textunderscore , talvez do lat. \textunderscore barbarus\textunderscore )
\section{Berberesco}
\begin{itemize}
\item {Grp. gram.:adj.}
\end{itemize}
Relativo aos Berbéres.
\section{Berberídeas}
\begin{itemize}
\item {Grp. gram.:f. pl.}
\end{itemize}
Família de plantas, a que serve de typo a bérberis.
\section{Berberina}
\begin{itemize}
\item {Grp. gram.:f.}
\end{itemize}
Alcaloide febrífugo da bérberis.
\section{Bérberis}
\begin{itemize}
\item {Grp. gram.:f.}
\end{itemize}
\begin{itemize}
\item {Proveniência:(Gr. \textunderscore berberi\textunderscore )}
\end{itemize}
Planta ornamental, (\textunderscore berberis vulgaris\textunderscore , Lin.).
\section{Berberisco}
\begin{itemize}
\item {Grp. gram.:adj.}
\end{itemize}
O mesmo que \textunderscore berberesco\textunderscore . Cf. F. Manuel, \textunderscore Apólogos\textunderscore .
\section{Berberisco}
\begin{itemize}
\item {Grp. gram.:m.}
\end{itemize}
\begin{itemize}
\item {Utilização:T. de Caminha}
\end{itemize}
O mesmo que \textunderscore berbigão\textunderscore .
\section{Berberixo}
\begin{itemize}
\item {Grp. gram.:m.}
\end{itemize}
O mesmo que \textunderscore berberisco\textunderscore ^2.
\section{Berbigão}
\begin{itemize}
\item {Grp. gram.:m.}
\end{itemize}
Mollusco acéphalo.
\section{Berbigoeira}
\begin{itemize}
\item {Grp. gram.:f.}
\end{itemize}
\begin{itemize}
\item {Proveniência:(De \textunderscore berbigão\textunderscore )}
\end{itemize}
Rede da ria de Aveiro.
\section{Berbim}
\begin{itemize}
\item {Grp. gram.:m.}
\end{itemize}
Antiga marca do pano dozeno de lan.
\section{Berça}
\begin{itemize}
\item {fónica:bêr}
\end{itemize}
\begin{itemize}
\item {Grp. gram.:f.}
\end{itemize}
\begin{itemize}
\item {Grp. gram.:Pl.}
\end{itemize}
\begin{itemize}
\item {Utilização:Fig.}
\end{itemize}
Espécie de couve; couve gallega.
Fôlhas de couves, ou de outra planta hortense, preparadas para a mesa.
Caldo verde.
Palavreado ôco.
(Cast. \textunderscore berza\textunderscore )
\section{Berceira}
\begin{itemize}
\item {Grp. gram.:f.}
\end{itemize}
\begin{itemize}
\item {Proveniência:(De \textunderscore berça\textunderscore )}
\end{itemize}
Vendedeira de hortaliça.
\section{Berceiro}
\begin{itemize}
\item {Grp. gram.:adj.}
\end{itemize}
\begin{itemize}
\item {Utilização:Prov.}
\end{itemize}
\begin{itemize}
\item {Utilização:trasm.}
\end{itemize}
\begin{itemize}
\item {Proveniência:(De \textunderscore berça\textunderscore ? Ou de \textunderscore berço\textunderscore , por allusão ao movimento descuidado do mau trabalhador?)}
\end{itemize}
Calaceiro, molle no trabalho.
\section{Berciano}
\begin{itemize}
\item {Grp. gram.:m.}
\end{itemize}
Um dos dialectos da Galliza.
\section{Berço}
\begin{itemize}
\item {fónica:bêr}
\end{itemize}
\begin{itemize}
\item {Grp. gram.:m.}
\end{itemize}
\begin{itemize}
\item {Utilização:Náut.}
\end{itemize}
Leito de crianças, ao qual geralmente se póde dar o movimento de balanço.
Primeira infância: \textunderscore recordações do berço\textunderscore .
Comêço.
Lugar, onde uma pessôa ou coisa teve nascimento, princípio: \textunderscore Portugal, berço de heróis\textunderscore .
Origem.
Antiga bôca de fogo, curta.
Fórma de abóbada.
Peça de madeira, sôbre que o navio corre, do estaleiro para o mar.
Abertura, na fêmea do leme.
Gradeamento de ferro, em volta de um coval.
\section{Berçudo}
\begin{itemize}
\item {Grp. gram.:adj.}
\end{itemize}
\begin{itemize}
\item {Utilização:Fig.}
\end{itemize}
\begin{itemize}
\item {Proveniência:(De \textunderscore berça\textunderscore )}
\end{itemize}
Que tem muitas fôlhas.
Peludo, cabelludo.
Carrancudo. Cf. \textunderscore Eufrosina\textunderscore , 109.
\section{Bere-bere!}
\begin{itemize}
\item {Grp. gram.:interj.}
\end{itemize}
(Designativa de hospitalidade cordial, cortezia e paz, entre os negros da Guiné. Cf. Barros, \textunderscore Déc.\textunderscore , I, l. III, c. 1)
\section{Berebere}
\begin{itemize}
\item {Grp. gram.:m.  e  adj.}
\end{itemize}
O mesmo que \textunderscore berbére\textunderscore . Cf. Herculano, \textunderscore Eurico\textunderscore , 226.
\section{Berebere}
\begin{itemize}
\item {Grp. gram.:m.}
\end{itemize}
Doença peculiar a algumas regiões tropicaes.
(Do cingalês \textunderscore beri-beri\textunderscore )
\section{Berenice}
\begin{itemize}
\item {Grp. gram.:f.}
\end{itemize}
\begin{itemize}
\item {Proveniência:(De \textunderscore Berenice\textunderscore , n. p.)}
\end{itemize}
Constellação boreal, abaixo da Ursa-Maior.
\section{Bereré}
\begin{itemize}
\item {Grp. gram.:m.}
\end{itemize}
\begin{itemize}
\item {Utilização:Bras}
\end{itemize}
Barulho, motim.
\section{Bergamota}
\begin{itemize}
\item {Grp. gram.:f.}
\end{itemize}
\begin{itemize}
\item {Utilização:T. da Guarda}
\end{itemize}
\begin{itemize}
\item {Proveniência:(De \textunderscore Bérgamo\textunderscore , n. p.)}
\end{itemize}
Planta odorífera, da fam. das labiadas.
Espécie de pêra sumarenta.
Espécie de limoeiro, de cujo fruto se extrai essência aromática para cosméticos.
Espécie de pêra.
\section{Berganhar}
\textunderscore v. t.\textunderscore  (e der.)
O mesmo que \textunderscore barganhar\textunderscore .
\section{Bergantim}
\begin{itemize}
\item {Grp. gram.:m.}
\end{itemize}
Embarcação de dois mastros, que arma como um brigue.
(Alter. de \textunderscore fragatim\textunderscore )
\section{Bergantine}
\begin{itemize}
\item {Grp. gram.:m.}
\end{itemize}
\begin{itemize}
\item {Utilização:Ant.}
\end{itemize}
O mesmo que \textunderscore bergantim\textunderscore .
\section{Bèribéri}
\begin{itemize}
\item {Grp. gram.:m.}
\end{itemize}
Doença peculiar a algumas regiões tropicaes.
(Do cingalês \textunderscore beri-beri\textunderscore )
\section{Bèribérico}
\begin{itemize}
\item {Grp. gram.:adj.}
\end{itemize}
\begin{itemize}
\item {Grp. gram.:M.}
\end{itemize}
Relativo ao bèribéri.
Que soffre bèribéri.
O doente de bèribéri.
\section{Bèriberígeno}
\begin{itemize}
\item {Grp. gram.:adj.}
\end{itemize}
\begin{itemize}
\item {Proveniência:(T. hýbr.)}
\end{itemize}
Que produz a doença do bèribéri.
\section{Bèriberizar}
\begin{itemize}
\item {Grp. gram.:v. t.}
\end{itemize}
\begin{itemize}
\item {Utilização:Neol.}
\end{itemize}
Produzir bèribéri em.
\section{Berifão}
\begin{itemize}
\item {Grp. gram.:m.}
\end{itemize}
Árvore da Guiné portuguesa.
\section{Berilo}
\begin{itemize}
\item {Grp. gram.:m.}
\end{itemize}
\begin{itemize}
\item {Proveniência:(Gr. \textunderscore berullos\textunderscore )}
\end{itemize}
Pedra preciosa, de côr verde-mar.
\section{Berimbau}
\begin{itemize}
\item {Grp. gram.:m.}
\end{itemize}
(V.brimbau)
\section{Beringela}
\begin{itemize}
\item {Grp. gram.:f.}
\end{itemize}
\begin{itemize}
\item {Grp. gram.:M.}
\end{itemize}
\begin{itemize}
\item {Utilização:des.}
\end{itemize}
\begin{itemize}
\item {Utilização:Pop.}
\end{itemize}
\begin{itemize}
\item {Proveniência:(Do ár. \textunderscore bedinjen\textunderscore )}
\end{itemize}
Planta solânea.
Fruto desta planta, ovoide, esbranquiçado ou roxo.
Seminarista, que trazia tiras verdes sôbre opa roxa.
\section{Berjaçote}
\begin{itemize}
\item {Grp. gram.:m.  e  adj.}
\end{itemize}
Espécie de figos, com polpa vermelha.
(Talvez de \textunderscore Burjasot\textunderscore , n. p., devendo, neste caso, escrever-se \textunderscore berjassote\textunderscore )
\section{Berlengas}
\begin{itemize}
\item {Grp. gram.:f. pl.}
\end{itemize}
\begin{itemize}
\item {Utilização:Prov.}
\end{itemize}
\begin{itemize}
\item {Utilização:beir.}
\end{itemize}
Intrujices, lábia.
(Por \textunderscore perlengas\textunderscore  = \textunderscore parlengas\textunderscore )
\section{Berliana}
\begin{itemize}
\item {Grp. gram.:f.}
\end{itemize}
Designação pop. e ant. da \textunderscore valeriana\textunderscore , planta. Cf. \textunderscore Desengano da Med.\textunderscore , 249.
\section{Berlina}
\begin{itemize}
\item {Grp. gram.:f.}
\end{itemize}
O mesmo ou melhor que \textunderscore berlinda\textunderscore .
\section{Berlinda}
\begin{itemize}
\item {Grp. gram.:f.}
\end{itemize}
\begin{itemize}
\item {Proveniência:(Fr. \textunderscore berline\textunderscore , de \textunderscore Berlim\textunderscore , n. p.)}
\end{itemize}
Pequeno coche de quatro rodas, suspenso entre dois varaes.
\textunderscore Estar na berlinda\textunderscore , condemnação em jôgo de prendas.
Sêr alvo de motejos.
Sêr objecto de largas apreciações, estar na ordem do dia.
\section{Berlinde}
\begin{itemize}
\item {Grp. gram.:m.}
\end{itemize}
(V.belindre)
\section{Berlinense}
\begin{itemize}
\item {Grp. gram.:m.  e  adj.}
\end{itemize}
O mesmo que \textunderscore berlinês\textunderscore .
\section{Berlinês}
\begin{itemize}
\item {Grp. gram.:adj.}
\end{itemize}
\begin{itemize}
\item {Grp. gram.:M.}
\end{itemize}
Relativo a Berlim.
Habitante de Berlim.
\section{Berlínia}
\begin{itemize}
\item {Grp. gram.:f.}
\end{itemize}
Gênero de plantas cesalpíneas.
\section{Berliques-e-berloques}
\begin{itemize}
\item {Grp. gram.:m. pl.}
\end{itemize}
\begin{itemize}
\item {Utilização:Pop.}
\end{itemize}
Escamoteação.
Arte, habilidade mysteriosa.
Intrujice.
\section{Berloque}
\begin{itemize}
\item {Grp. gram.:m.}
\end{itemize}
\begin{itemize}
\item {Proveniência:(Fr. \textunderscore breloque\textunderscore )}
\end{itemize}
Pequeno enfeite, que se traz pendente na cadeia do relógio ou nas pulseiras.
Curiosidade de pouco valor.
\section{Berlota}
\begin{itemize}
\item {Grp. gram.:f.}
\end{itemize}
\begin{itemize}
\item {Utilização:Prov.}
\end{itemize}
\begin{itemize}
\item {Utilização:alent.}
\end{itemize}
Pequeno disco de ferro, entre as tornejas e a roda.
\section{Berma}
\begin{itemize}
\item {Grp. gram.:f.}
\end{itemize}
\begin{itemize}
\item {Proveniência:(Al. \textunderscore berme\textunderscore )}
\end{itemize}
Espaço, entre a linha inferior da muralha e o fôsso; sapata.
Caminho, entre um molhe e a borda de um canal ou fôsso.
Faixa de estrada, entre a valleta e a parte empedrada.
\section{Bermá}
\begin{itemize}
\item {Grp. gram.:m.}
\end{itemize}
O mesmo que \textunderscore bermano\textunderscore .
\section{Bermano}
\begin{itemize}
\item {Grp. gram.:m.}
\end{itemize}
Língua culta de Bermá, na Ásia.
\section{Bermudo}
\begin{itemize}
\item {Grp. gram.:m.}
\end{itemize}
\begin{itemize}
\item {Utilização:T. de Moçambique}
\end{itemize}
O mesmo que \textunderscore viúva\textunderscore , (pássaro).
\section{Bernaca}
\begin{itemize}
\item {Grp. gram.:f.}
\end{itemize}
Espécie de ganso montesinho.
Álem dos mares do Norte.
(B. lat. \textunderscore bernaca\textunderscore )
\section{Bernarda}
\begin{itemize}
\item {Grp. gram.:f.}
\end{itemize}
\begin{itemize}
\item {Utilização:Fam.}
\end{itemize}
Revolta, motim.
\section{Bernarda}
\begin{itemize}
\item {Grp. gram.:f.}
\end{itemize}
Variedade de pêra.
\section{Bernardesco}
\begin{itemize}
\item {Grp. gram.:adj.}
\end{itemize}
Relativo a frades bernardos; relativo a bernardice. Cf. Garrett, \textunderscore D. Branca\textunderscore , 22.
\section{Bernardice}
\begin{itemize}
\item {Grp. gram.:f.}
\end{itemize}
\begin{itemize}
\item {Proveniência:(De \textunderscore bernardo\textunderscore )}
\end{itemize}
Locução tola ou disparatada; dislate.
Acto ou dito, próprio de frade bernardo.
\section{Bernardo}
\begin{itemize}
\item {Grp. gram.:m.  e  adj.}
\end{itemize}
\begin{itemize}
\item {Utilização:Fig.}
\end{itemize}
Frade da ordem de San-Bernardo.
Estúpido e glutão.
\section{Bernardo-ermita}
\begin{itemize}
\item {Grp. gram.:m.}
\end{itemize}
Espécie de caranguejo, comestível, matizado de vermelho, roxo e pardo.
\section{Bernari}
\begin{itemize}
\item {Grp. gram.:m.}
\end{itemize}
Planta americana.
\section{Bernás}
\begin{itemize}
\item {Grp. gram.:m.}
\end{itemize}
\begin{itemize}
\item {Utilização:Prov.}
\end{itemize}
\begin{itemize}
\item {Utilização:trasm.}
\end{itemize}
Granito miúdo, para obras de alvenaria.
\section{Berne}
\begin{itemize}
\item {Grp. gram.:m.}
\end{itemize}
Larva de certo insecto, que penetra na pelle dos animaes e até do homem, podendo occasionar-lhes a morte, se o não extrahem a tempo.
\section{Berne}
\begin{itemize}
\item {Grp. gram.:adj.}
\end{itemize}
Dizia-se de certo pano vermelho, usado em reposteiros e balandraus. Cf. Filinto,
XIII, 285; Camillo, \textunderscore Caveira\textunderscore , 453; Corvo, \textunderscore Anno na Côrte\textunderscore , I, 34.
(Cp. \textunderscore bérneo\textunderscore )
\section{Bérneo}
\begin{itemize}
\item {Grp. gram.:m.}
\end{itemize}
Antigo pano, que vinha da Irlanda.
Antiga capa, grosseira e comprida.
(Por \textunderscore hibérnio\textunderscore , de \textunderscore Hibérnia\textunderscore , n. p.)
\section{Bernicha}
\begin{itemize}
\item {Grp. gram.:f.}
\end{itemize}
(V.bernaca)
\section{Bérnio}
\begin{itemize}
\item {Grp. gram.:m.}
\end{itemize}
Antigo pano, que vinha da Irlanda.
Antiga capa, grosseira e comprida.
(Por \textunderscore hibérnio\textunderscore , de \textunderscore Hibérnia\textunderscore , n. p.)
\section{Bero}
\begin{itemize}
\item {fónica:bê}
\end{itemize}
\begin{itemize}
\item {Grp. gram.:m.}
\end{itemize}
O mesmo que \textunderscore beiro\textunderscore .
\section{Beró}
\begin{itemize}
\item {Grp. gram.:m.}
\end{itemize}
O mesmo que \textunderscore beroba\textunderscore .
\section{Beroba}
\begin{itemize}
\item {Grp. gram.:f.}
\end{itemize}
\begin{itemize}
\item {Utilização:Bras}
\end{itemize}
O mesmo que \textunderscore égua\textunderscore .
\section{Berol}
\begin{itemize}
\item {Grp. gram.:m.}
\end{itemize}
Planta submarina, também conhecida por \textunderscore pepino-do-mar\textunderscore .
\section{Berós}
\begin{itemize}
\item {Grp. gram.:m.}
\end{itemize}
Árvore da Índia portuguesa.
\section{Berra}
\begin{itemize}
\item {Grp. gram.:f.}
\end{itemize}
\begin{itemize}
\item {Proveniência:(De \textunderscore berrar\textunderscore )}
\end{itemize}
Cio de veados.
\textunderscore Andar na berra\textunderscore , estar em voga, sêr falado.
\section{Berraceira}
\begin{itemize}
\item {Grp. gram.:f.}
\end{itemize}
\begin{itemize}
\item {Utilização:Açor}
\end{itemize}
O mesmo que \textunderscore berraria\textunderscore .
\section{Berrador}
\begin{itemize}
\item {Grp. gram.:m.}
\end{itemize}
\begin{itemize}
\item {Grp. gram.:Adj.}
\end{itemize}
Aquelle que berra.
Que berra.
\section{Berradura}
\begin{itemize}
\item {Grp. gram.:f.}
\end{itemize}
\begin{itemize}
\item {Utilização:Des.}
\end{itemize}
O mesmo que \textunderscore berraria\textunderscore .
\section{Berrante}
\begin{itemize}
\item {Grp. gram.:adj.}
\end{itemize}
\begin{itemize}
\item {Utilização:Neol.}
\end{itemize}
Diz-se das côres muito vivas ou que dão muito na vista: \textunderscore um vestido de côr berrante\textunderscore .
\section{Berrão}
\begin{itemize}
\item {Grp. gram.:m.}
\end{itemize}
\begin{itemize}
\item {Utilização:Prov.}
\end{itemize}
\begin{itemize}
\item {Utilização:trasm.}
\end{itemize}
Porco, não castrado.
(Alter. de \textunderscore varrão\textunderscore )
\section{Berrão}
\begin{itemize}
\item {Grp. gram.:m.}
\end{itemize}
\begin{itemize}
\item {Utilização:Pop.}
\end{itemize}
Criança, que berra muito, chorando.
\section{Berrar}
\begin{itemize}
\item {Grp. gram.:v. i.}
\end{itemize}
\begin{itemize}
\item {Utilização:Pop.}
\end{itemize}
\begin{itemize}
\item {Utilização:Gal}
\end{itemize}
\begin{itemize}
\item {Proveniência:(Do lat. \textunderscore barrire\textunderscore )}
\end{itemize}
Dar berros.
Gritar.
Roncar.
Solicitar, instar.
Diz-se das côres muito vivas, espalhafatosas, e daquillo que tem essas côres: \textunderscore um vestido berrante\textunderscore .
\section{Berraria}
\begin{itemize}
\item {Grp. gram.:f.}
\end{itemize}
O mesmo que \textunderscore berreiro\textunderscore .
\section{Berrata}
\begin{itemize}
\item {Grp. gram.:f.}
\end{itemize}
\begin{itemize}
\item {Utilização:Fam.}
\end{itemize}
O mesmo que \textunderscore berreiro\textunderscore .
\section{Berrega}
\begin{itemize}
\item {Grp. gram.:m.  e  f.}
\end{itemize}
\begin{itemize}
\item {Utilização:T. da Bairrada}
\end{itemize}
\begin{itemize}
\item {Proveniência:(De \textunderscore berrar\textunderscore )}
\end{itemize}
Criança, que chora muito ou muitas vezes.
\section{Berregar}
\begin{itemize}
\item {Grp. gram.:v. i.}
\end{itemize}
\begin{itemize}
\item {Proveniência:(De \textunderscore berrar\textunderscore , ou corr. de \textunderscore borregar\textunderscore )}
\end{itemize}
Berrar muito, com frequencia.
O mesmo que \textunderscore balar\textunderscore .
\section{Berrêgo}
\begin{itemize}
\item {Grp. gram.:m.}
\end{itemize}
\begin{itemize}
\item {Utilização:Prov.}
\end{itemize}
\begin{itemize}
\item {Utilização:minh.}
\end{itemize}
Acto de \textunderscore berregar\textunderscore .
Berro.
Grito.
\section{Berreiro}
\begin{itemize}
\item {Grp. gram.:m.}
\end{itemize}
Berros frequentes e altos.
Gritaria.
Chôro ruidoso.
\section{Berrelas}
\begin{itemize}
\item {Grp. gram.:m.  e  f.}
\end{itemize}
\begin{itemize}
\item {Utilização:Prov.}
\end{itemize}
\begin{itemize}
\item {Utilização:trasm.}
\end{itemize}
Pessôa, que berra muito.
\section{Bérria}
\begin{itemize}
\item {Grp. gram.:f.}
\end{itemize}
\begin{itemize}
\item {Proveniência:(De \textunderscore Berry\textunderscore , n. p.)}
\end{itemize}
Gênero de plantas liliáceas.
\section{Berrincha}
\begin{itemize}
\item {Grp. gram.:f.}
\end{itemize}
\begin{itemize}
\item {Utilização:Prov.}
\end{itemize}
\begin{itemize}
\item {Utilização:beir.}
\end{itemize}
\begin{itemize}
\item {Utilização:alg.}
\end{itemize}
Teima.
Acto de serrazinar; altercação.
(Cast. \textunderscore berrinche\textunderscore )
\section{Berro}
\begin{itemize}
\item {Grp. gram.:m.}
\end{itemize}
\begin{itemize}
\item {Proveniência:(De \textunderscore berrar\textunderscore )}
\end{itemize}
Voz ou grito de certos animaes.
Grito alto e áspero de uma pessôa.
Rugido.
\section{Berro}
\begin{itemize}
\item {Grp. gram.:m.}
\end{itemize}
Insecto, (\textunderscore hypoderma bovis\textunderscore ), que entre os pêlos dos bovídeos põe ovos, de que saem larvas que furam a pelle, e vivem debaixo della, produzindo tumores dolorosos.
\section{Berroíça}
\begin{itemize}
\item {Grp. gram.:adj. f.}
\end{itemize}
\begin{itemize}
\item {Utilização:Prov.}
\end{itemize}
\begin{itemize}
\item {Utilização:trasm.}
\end{itemize}
\begin{itemize}
\item {Proveniência:(De \textunderscore berrão\textunderscore ^1)}
\end{itemize}
Diz-se da porca que está na sazão de ir ao macho.
\section{Bérrya}
\begin{itemize}
\item {Grp. gram.:f.}
\end{itemize}
\begin{itemize}
\item {Proveniência:(De \textunderscore Berry\textunderscore , n. p.)}
\end{itemize}
Gênero de plantas liliáceas.
\section{Bertalha}
\begin{itemize}
\item {Grp. gram.:f.}
\end{itemize}
Planta herbácea, (\textunderscore basella rubra\textunderscore ).
\section{Bertangil}
\begin{itemize}
\item {Grp. gram.:m.}
\end{itemize}
Tecido antigo de Cambaia.
\section{Bertanha}
\begin{itemize}
\item {Grp. gram.:f.}
\end{itemize}
\begin{itemize}
\item {Utilização:bras}
\end{itemize}
\begin{itemize}
\item {Utilização:Pop.}
\end{itemize}
O mesmo que \textunderscore bretanha\textunderscore .
\section{Bertéroa}
\begin{itemize}
\item {Grp. gram.:f.}
\end{itemize}
\begin{itemize}
\item {Proveniência:(De \textunderscore Bertero\textunderscore , n. p.)}
\end{itemize}
Planta crucífera.
\section{Bertoeja}
\begin{itemize}
\item {Grp. gram.:f.}
\end{itemize}
(Corr. de \textunderscore brotoeja\textunderscore )
\section{Bertoldo}
\begin{itemize}
\item {Grp. gram.:m.}
\end{itemize}
\begin{itemize}
\item {Utilização:Prov.}
\end{itemize}
\begin{itemize}
\item {Utilização:trasm.}
\end{itemize}
\begin{itemize}
\item {Proveniência:(De \textunderscore Bertholdo\textunderscore , n. p. de personagem romântica)}
\end{itemize}
Palerma.
Brutamontes.
\section{Bertolécia}
\begin{itemize}
\item {Grp. gram.:f.}
\end{itemize}
\begin{itemize}
\item {Proveniência:(De \textunderscore Bertollet\textunderscore , n. p.)}
\end{itemize}
Gênero de plantas myrtáceas.
\section{Bertollécia}
\begin{itemize}
\item {Grp. gram.:f.}
\end{itemize}
\begin{itemize}
\item {Proveniência:(De \textunderscore Bertollet\textunderscore , n. p.)}
\end{itemize}
Gênero de plantas myrtáceas.
\section{Bertolónia}
\begin{itemize}
\item {Grp. gram.:f.}
\end{itemize}
Planta ornamental.
\section{Beryllo}
\begin{itemize}
\item {Grp. gram.:m.}
\end{itemize}
\begin{itemize}
\item {Proveniência:(Gr. \textunderscore berullos\textunderscore )}
\end{itemize}
Pedra preciosa, de côr verde-mar.
\section{Berzabu}
\begin{itemize}
\item {Grp. gram.:m.}
\end{itemize}
O mesmo que \textunderscore belzebu\textunderscore .
Cf. Camillo, \textunderscore Eus. Macário\textunderscore , 179.
\section{Berzabum}
\begin{itemize}
\item {Grp. gram.:m.}
\end{itemize}
O mesmo que \textunderscore belzebu\textunderscore . Cf. Camillo, \textunderscore Bruxa\textunderscore , 2.^a p.^e, c. V; \textunderscore Idem\textunderscore , \textunderscore Scenas da Foz\textunderscore .
\section{Berzebu}
\begin{itemize}
\item {Grp. gram.:m.}
\end{itemize}
O mesmo que \textunderscore belzebu\textunderscore . Cf. G. Vicente, \textunderscore Auto da Fé\textunderscore .
\section{Berzunda}
\begin{itemize}
\item {Grp. gram.:f.}
\end{itemize}
O mesmo que \textunderscore berzundela\textunderscore .
\section{Berzundela}
\begin{itemize}
\item {Grp. gram.:f.}
\end{itemize}
\begin{itemize}
\item {Utilização:Prov.}
\end{itemize}
\begin{itemize}
\item {Utilização:alg.}
\end{itemize}
Bebedeira.
\section{Besantar}
\begin{itemize}
\item {Grp. gram.:v.}
\end{itemize}
\begin{itemize}
\item {Utilização:t. Heráld.}
\end{itemize}
Cobrir de besantes (um escudo de armas).
\section{Besante}
\begin{itemize}
\item {Grp. gram.:m.}
\end{itemize}
Antiga moéda byzantina.
Moéda sem marca, figurada nos brasões.
(Por \textunderscore byzante\textunderscore , de \textunderscore Byzantius\textunderscore , n. p. lat. de Constantinopla)
\section{Bescocinho}
\begin{itemize}
\item {Grp. gram.:m.}
\end{itemize}
\begin{itemize}
\item {Utilização:Prov.}
\end{itemize}
\begin{itemize}
\item {Utilização:alg.}
\end{itemize}
\begin{itemize}
\item {Proveniência:(De \textunderscore bescoço\textunderscore )}
\end{itemize}
Cabeção ecclesiástico.
\section{Bescoco}
\begin{itemize}
\item {fónica:cô}
\end{itemize}
\textunderscore m.\textunderscore  (e der) \textunderscore Prov. alg.\textunderscore 
O mesmo que \textunderscore pescoço\textunderscore , etc.
\section{Besoiral}
\begin{itemize}
\item {Grp. gram.:adj.}
\end{itemize}
Próprio de besoiro ou semelhante a besoiro. Cf. Cortesão, \textunderscore Subs.\textunderscore 
\section{Besoiro}
\begin{itemize}
\item {Grp. gram.:m.}
\end{itemize}
\begin{itemize}
\item {Proveniência:(T. onom., do som estrídulo das asas?)}
\end{itemize}
Insecto coleóptero, amarelo ou preto.
Peixe de Portugal.
\section{Besonha}
\begin{itemize}
\item {Grp. gram.:f.}
\end{itemize}
\begin{itemize}
\item {Utilização:Ant.}
\end{itemize}
\begin{itemize}
\item {Proveniência:(Fr. \textunderscore besogne\textunderscore )}
\end{itemize}
Necessidade.
\section{Besouro}
\begin{itemize}
\item {Grp. gram.:m.}
\end{itemize}
\begin{itemize}
\item {Proveniência:(T. onom., do som estrídulo das asas?)}
\end{itemize}
Insecto coleóptero, amarelo ou preto.
Peixe de Portugal.
\section{Bespa}
\begin{itemize}
\item {fónica:bês}
\end{itemize}
\begin{itemize}
\item {Grp. gram.:f.}
\end{itemize}
O mesmo que \textunderscore vespa\textunderscore .
(Cf. \textunderscore abespinhar-se\textunderscore )
\section{Bespe}
\begin{itemize}
\item {Grp. gram.:m.}
\end{itemize}
\begin{itemize}
\item {Utilização:Ant.}
\end{itemize}
Talvez o mesmo que \textunderscore vespícia\textunderscore . Cf. \textunderscore Lembrança das Cousas da India\textunderscore , nos \textunderscore Subsídios\textunderscore  de Felner, 47.
\section{Bessi}
\begin{itemize}
\item {Grp. gram.:m.}
\end{itemize}
Grande árvore leguminosa das Molucas, que dá bôa madeira para construcções.
\section{Bêsta}
\begin{itemize}
\item {Grp. gram.:f.}
\end{itemize}
\begin{itemize}
\item {Utilização:Bras. do N}
\end{itemize}
\begin{itemize}
\item {Utilização:Fig.}
\end{itemize}
\begin{itemize}
\item {Utilização:Ant.}
\end{itemize}
\begin{itemize}
\item {Proveniência:(Lat. \textunderscore bestia\textunderscore )}
\end{itemize}
Quadrúpede.
Animal de carga.
O mesmo que \textunderscore égua\textunderscore .
Pessôa muito estúpida.
\textunderscore Bêsta maiór\textunderscore , bêsta muar, ou cavallar.
\textunderscore Bêsta menór\textunderscore , besta asinina.
\section{Bésta}
\begin{itemize}
\item {Grp. gram.:f.}
\end{itemize}
\begin{itemize}
\item {Proveniência:(Do lat. \textunderscore ballista\textunderscore )}
\end{itemize}
Antiga arma, que disparava peloiros ou setas.
\section{Beia}
\begin{itemize}
\item {Grp. gram.:f.}
\end{itemize}
Nome que os alquimistas davam á água mercurial.
\section{Beiapuca}
\begin{itemize}
\item {Grp. gram.:f.}
\end{itemize}
Peixe das costas do Brasil.
\section{Bêsta-fera}
\begin{itemize}
\item {Grp. gram.:m.  e  f.}
\end{itemize}
Pessôa selvagem ou cruel. Cf. Camillo, \textunderscore Noites de Insómn.\textunderscore , VI, 96.
\section{Bestar}
\begin{itemize}
\item {Grp. gram.:v. i.}
\end{itemize}
\begin{itemize}
\item {Utilização:Bras. do N}
\end{itemize}
\begin{itemize}
\item {Proveniência:(De \textunderscore bêsta\textunderscore )}
\end{itemize}
Dizer asneiras.
Praticar inconveniências.
\section{Bestaraz}
\begin{itemize}
\item {Grp. gram.:m.}
\end{itemize}
\begin{itemize}
\item {Utilização:ant.}
\end{itemize}
\begin{itemize}
\item {Utilização:Burl.}
\end{itemize}
\begin{itemize}
\item {Proveniência:(De \textunderscore bêsta\textunderscore )}
\end{itemize}
Brutamontes. Cf. Sim. Machado.
\section{Bèstaria}
\begin{itemize}
\item {Grp. gram.:f.}
\end{itemize}
\begin{itemize}
\item {Utilização:Ant.}
\end{itemize}
\begin{itemize}
\item {Proveniência:(De \textunderscore bésta\textunderscore )}
\end{itemize}
Grande porção de béstas ou de bèsteiros.
\section{Bestearia}
\begin{itemize}
\item {Grp. gram.:f.}
\end{itemize}
\begin{itemize}
\item {Proveniência:(De \textunderscore bêsta\textunderscore )}
\end{itemize}
Brutalidade; procedimento incivil.
\section{Besteira}
\begin{itemize}
\item {Grp. gram.:adj. f.}
\end{itemize}
\begin{itemize}
\item {Grp. gram.:M.}
\end{itemize}
\begin{itemize}
\item {Utilização:Bras. do N}
\end{itemize}
\begin{itemize}
\item {Proveniência:(De \textunderscore bêsta\textunderscore )}
\end{itemize}
Diz-se da erva, conhecida scientificamente por \textunderscore helléboro\textunderscore .
Asneira, tolice.
\section{Bèsteiras}
\begin{itemize}
\item {Grp. gram.:f. pl.}
\end{itemize}
\begin{itemize}
\item {Proveniência:(De \textunderscore bésta\textunderscore )}
\end{itemize}
Aberturas nas galerias das fortificações antigas.
O mesmo que \textunderscore balestreiro\textunderscore .
\section{Bèsteiro}
\begin{itemize}
\item {Grp. gram.:m.}
\end{itemize}
\begin{itemize}
\item {Proveniência:(Do lat. \textunderscore ballistarius\textunderscore )}
\end{itemize}
Soldado, armado de bésta.
Fabricante de béstas.
\section{Bestiães}
\begin{itemize}
\item {Grp. gram.:m. pl.}
\end{itemize}
O mesmo que \textunderscore bastiães\textunderscore .
\section{Bestiaga}
\begin{itemize}
\item {Grp. gram.:f.}
\end{itemize}
\begin{itemize}
\item {Proveniência:(Do lat. \textunderscore bestia\textunderscore )}
\end{itemize}
Bêsta reles.
Pessôa muito estúpida.
\section{Bestiagem}
\begin{itemize}
\item {Grp. gram.:f.}
\end{itemize}
\begin{itemize}
\item {Proveniência:(Do lat. \textunderscore bestia\textunderscore )}
\end{itemize}
Reunião de bêstas.
\section{Bestial}
\begin{itemize}
\item {Grp. gram.:adj.}
\end{itemize}
\begin{itemize}
\item {Proveniência:(Lat. \textunderscore bestialis\textunderscore )}
\end{itemize}
Próprio de bêsta.
Brutal.
Estúpido.
Grosseiro.
\section{Bestialidade}
\begin{itemize}
\item {Grp. gram.:f.}
\end{itemize}
Qualidade do que é bestial.
Acção brutal.
Estupidez.
\section{Bestialização}
\begin{itemize}
\item {Grp. gram.:f.}
\end{itemize}
Acto de \textunderscore bestializar\textunderscore .
\section{Bestializar}
\begin{itemize}
\item {Grp. gram.:v. t.}
\end{itemize}
O mesmo que \textunderscore bestificar\textunderscore . Cf. Camillo, \textunderscore Crit. do Canc. Alegre\textunderscore , 23.
\section{Bestialmente}
\begin{itemize}
\item {Grp. gram.:adv.}
\end{itemize}
De modo \textunderscore bestial\textunderscore .
\section{Bestiário}
\begin{itemize}
\item {Grp. gram.:adj.}
\end{itemize}
\begin{itemize}
\item {Grp. gram.:M.}
\end{itemize}
\begin{itemize}
\item {Proveniência:(Lat. \textunderscore bestiarius\textunderscore )}
\end{itemize}
Relativo a bêstas.
Antigo gladiador, criminoso ou mercenário, que combatia com as feras no circo.
\section{Bestidade}
\begin{itemize}
\item {Grp. gram.:f.}
\end{itemize}
(V.bestialidade)
\section{Bestificação}
\begin{itemize}
\item {Grp. gram.:f.}
\end{itemize}
Acto de \textunderscore bestificar\textunderscore .
\section{Bestificante}
\begin{itemize}
\item {Grp. gram.:adj.}
\end{itemize}
Que bestifica.
\section{Bestificar}
\begin{itemize}
\item {Grp. gram.:v. t.}
\end{itemize}
\begin{itemize}
\item {Proveniência:(Do lat. \textunderscore bestia\textunderscore  + \textunderscore facere\textunderscore )}
\end{itemize}
Fazer estúpido.
\section{Bestigo}
\begin{itemize}
\item {Grp. gram.:m.}
\end{itemize}
\begin{itemize}
\item {Utilização:Prov.}
\end{itemize}
\begin{itemize}
\item {Utilização:Bras}
\end{itemize}
\begin{itemize}
\item {Utilização:trasm.}
\end{itemize}
\begin{itemize}
\item {Utilização:ant.}
\end{itemize}
Homem alto.
Animal grande:«\textunderscore senhor tartarago, digo que mentís como um bestigo\textunderscore ». G. Vicente, I, 262.
(Cp. lat. \textunderscore bestia\textunderscore )
\section{Bestiões}
\begin{itemize}
\item {Grp. gram.:m. pl.}
\end{itemize}
O mesmo que \textunderscore bastiões\textunderscore .
\section{Bestiola}
\begin{itemize}
\item {Grp. gram.:f.}
\end{itemize}
\begin{itemize}
\item {Utilização:Chul.}
\end{itemize}
O mesmo que \textunderscore bestiaga\textunderscore .
\section{Bestoiro}
\begin{itemize}
\item {Grp. gram.:m.}
\end{itemize}
\begin{itemize}
\item {Utilização:Prov.}
\end{itemize}
\begin{itemize}
\item {Utilização:trasm.}
\end{itemize}
Homem gordo e forte.
Cacete grosso.
\section{Bestoiro}
\begin{itemize}
\item {Grp. gram.:m.}
\end{itemize}
\begin{itemize}
\item {Utilização:Prov.}
\end{itemize}
\begin{itemize}
\item {Utilização:trasm.}
\end{itemize}
Porção sólida de excremento humano.
(Por \textunderscore bostoiro\textunderscore , de \textunderscore bosta\textunderscore )
\section{Bestouro}
\begin{itemize}
\item {Grp. gram.:m.}
\end{itemize}
\begin{itemize}
\item {Utilização:Prov.}
\end{itemize}
\begin{itemize}
\item {Utilização:trasm.}
\end{itemize}
Homem gordo e forte.
Cacete grosso.
\section{Bestouro}
\begin{itemize}
\item {Grp. gram.:m.}
\end{itemize}
\begin{itemize}
\item {Utilização:Prov.}
\end{itemize}
\begin{itemize}
\item {Utilização:trasm.}
\end{itemize}
Porção sólida de excremento humano.
(Por \textunderscore bostoiro\textunderscore , de \textunderscore bosta\textunderscore )
\section{Bestruço}
\begin{itemize}
\item {Grp. gram.:m.}
\end{itemize}
\begin{itemize}
\item {Utilização:Prov.}
\end{itemize}
\begin{itemize}
\item {Utilização:trasm.}
\end{itemize}
Coisa ou animal muito grande.
(Cp. \textunderscore mestrunço\textunderscore )
\section{Bestunto}
\begin{itemize}
\item {Grp. gram.:m.}
\end{itemize}
\begin{itemize}
\item {Utilização:Fam.}
\end{itemize}
\begin{itemize}
\item {Proveniência:(De \textunderscore bêsta\textunderscore )}
\end{itemize}
Cachimónia; cabêça estúpida ou de pouco alcance.
Limitada capacidade.
\section{Besugo}
\begin{itemize}
\item {Grp. gram.:m.}
\end{itemize}
\begin{itemize}
\item {Proveniência:(T. cast.)}
\end{itemize}
Peixe vulgar, da ordem dos acanthopterýgios.
Pessôa gorda, encorpada.
\section{Besuntadela}
\begin{itemize}
\item {Grp. gram.:f.}
\end{itemize}
Acto ou effeito de \textunderscore besuntar\textunderscore .
\section{Besuntão}
\begin{itemize}
\item {Grp. gram.:m.}
\end{itemize}
\begin{itemize}
\item {Utilização:Fam.}
\end{itemize}
\begin{itemize}
\item {Proveniência:(De \textunderscore besuntar\textunderscore )}
\end{itemize}
Homem ou menino, que traz o fato muito sujo, besuntado.
\section{Besuntar}
\begin{itemize}
\item {Grp. gram.:v. t.}
\end{itemize}
Untar muito.
Sujar com substância untuosa.
(Talvez de \textunderscore bis...\textunderscore  + \textunderscore untar\textunderscore )
\section{Besunto}
\begin{itemize}
\item {Grp. gram.:m.}
\end{itemize}
O mesmo que \textunderscore besuntadela\textunderscore . Cf. Filinto, XIX, 278.
\section{Besuntôna}
(fem. de \textunderscore besuntão\textunderscore )
\section{Bêta}
\begin{itemize}
\item {Grp. gram.:f.}
\end{itemize}
\begin{itemize}
\item {Utilização:Náut.}
\end{itemize}
\begin{itemize}
\item {Grp. gram.:Loc.}
\end{itemize}
\begin{itemize}
\item {Utilização:fam.}
\end{itemize}
\begin{itemize}
\item {Proveniência:(Do lat. \textunderscore vitta\textunderscore )}
\end{itemize}
Lista.
Pequeno filão mineral.
Mancha comprida.
Pequenino feixe de quaesquer fios.
Qualquer corda que, em navios, não tem nome especial.
Talha, collocada na extremidade dos guardins.
Malha, branca, entre as ventas do cavallo.
\textunderscore É de estrêlla e bêta\textunderscore , é de mau carácter; é pessôa, com quem deve haver cautela.
\section{Béta}
\begin{itemize}
\item {Grp. gram.:m.}
\end{itemize}
Designação da segunda letra do alphabeto grego.
\section{Béta}
\begin{itemize}
\item {Grp. gram.:f.}
\end{itemize}
Árvore africana, de fôlhas inteiras, lanceoladas, sem estípulas, e de flôres hermaphroditas, inodoras.
\section{Betânio}
\begin{itemize}
\item {Grp. gram.:m.}
\end{itemize}
\begin{itemize}
\item {Utilização:Ant.}
\end{itemize}
O mesmo que \textunderscore pisão\textunderscore ^1. Cf. \textunderscore Mss.\textunderscore  da Chancellaria de D. João II, na Tôrre do Tombo.
\section{Betão}
\begin{itemize}
\item {Grp. gram.:m.}
\end{itemize}
\begin{itemize}
\item {Proveniência:(Fr. \textunderscore beton\textunderscore )}
\end{itemize}
Espécie de cimento, composto de cal, areia e saibro.
\section{Betar}
\begin{itemize}
\item {Grp. gram.:v. t.}
\end{itemize}
\begin{itemize}
\item {Proveniência:(De \textunderscore bêta\textunderscore )}
\end{itemize}
Listar de côres variegadas; matizar.
\section{Betarda}
\begin{itemize}
\item {Grp. gram.:f.}
\end{itemize}
O mesmo que \textunderscore abetarda\textunderscore .
\section{Bétel}
\begin{itemize}
\item {Grp. gram.:m.}
\end{itemize}
(V.betle)
\section{Bétele}
\begin{itemize}
\item {Grp. gram.:m.}
\end{itemize}
(V.betle)
\section{Bétere}
\begin{itemize}
\item {Grp. gram.:m.}
\end{itemize}
O mesmo que \textunderscore betle\textunderscore .
\section{Beterraba}
\begin{itemize}
\item {Grp. gram.:f.}
\end{itemize}
\begin{itemize}
\item {Proveniência:(Fr. \textunderscore betterave\textunderscore )}
\end{itemize}
Raiz carnuda e grossa, de que se extrai açúcar idêntico ao da cana saccharina.
\section{Betesga}
\begin{itemize}
\item {Grp. gram.:f.}
\end{itemize}
O mesmo ou melhor que \textunderscore bitesga\textunderscore .
\section{Bethlemita}
\begin{itemize}
\item {Grp. gram.:adj.}
\end{itemize}
\begin{itemize}
\item {Grp. gram.:M.}
\end{itemize}
\begin{itemize}
\item {Proveniência:(Lat. \textunderscore bethlemites\textunderscore )}
\end{itemize}
Relativo a Bethlém, cidade da Judeia.
Habitante de Bethlém.
\section{Bethlemítico}
\begin{itemize}
\item {Grp. gram.:adj.}
\end{itemize}
Relativo aos bethlemitas.
\section{Bèticano}
\begin{itemize}
\item {Grp. gram.:adj.}
\end{itemize}
O mesmo que \textunderscore bético\textunderscore . Cf. \textunderscore Viriato Trág.\textunderscore , VIII, 46.
\section{Bético}
\begin{itemize}
\item {Grp. gram.:adj.}
\end{itemize}
\begin{itemize}
\item {Proveniência:(Lat. \textunderscore baeticus\textunderscore )}
\end{itemize}
Relativo á Bética.
\section{Betilho}
\begin{itemize}
\item {Grp. gram.:m.}
\end{itemize}
\begin{itemize}
\item {Proveniência:(De \textunderscore bêta\textunderscore )}
\end{itemize}
Cabresto para o boi.
\section{Bétilo}
\begin{itemize}
\item {Grp. gram.:m.}
\end{itemize}
\begin{itemize}
\item {Proveniência:(Gr. \textunderscore betulos\textunderscore )}
\end{itemize}
Pedra, que tinha certos sinaes e que entre os antigos era adorada como um ídolo.
\section{Betle}
\begin{itemize}
\item {Grp. gram.:m.}
\end{itemize}
\begin{itemize}
\item {Proveniência:(Do malab. \textunderscore vettila\textunderscore )}
\end{itemize}
Planta sarmentosa e aromática.
Mistura de substâncias, que se mastiga por hábito em algumas regiões tropicaes, e em que entram fôlhas de betle.
\section{Betlemita}
\begin{itemize}
\item {Grp. gram.:adj.}
\end{itemize}
\begin{itemize}
\item {Grp. gram.:M.}
\end{itemize}
\begin{itemize}
\item {Proveniência:(Lat. \textunderscore bethlemites\textunderscore )}
\end{itemize}
Relativo a Bethlém, cidade da Judeia.
Habitante de Bethlém.
\section{Betlemítico}
\begin{itemize}
\item {Grp. gram.:adj.}
\end{itemize}
Relativo aos bethlemitas.
\section{Beto}
\begin{itemize}
\item {Grp. gram.:m.}
\end{itemize}
\begin{itemize}
\item {Utilização:Prov.}
\end{itemize}
\begin{itemize}
\item {Utilização:trasm.}
\end{itemize}
Espécie de pá de madeira, com que se joga o \textunderscore toque-emboque\textunderscore .
Nome de um jôgo, semelhante ao \textunderscore crickets\textunderscore  inglês.
\section{Betoiro}
\begin{itemize}
\item {Grp. gram.:m.}
\end{itemize}
O mesmo que \textunderscore abetoiro\textunderscore .
\section{Betol}
\begin{itemize}
\item {Grp. gram.:m.}
\end{itemize}
\begin{itemize}
\item {Utilização:Pharm.}
\end{itemize}
Salicilato de naphtol, que se applica no catarro vesical, no reumatismo, etc.
\section{Betonar}
\begin{itemize}
\item {Grp. gram.:v. t.}
\end{itemize}
Revestir com betão.
Cimentar com betão.
\section{Betoneira}
\begin{itemize}
\item {Grp. gram.:f.}
\end{itemize}
Apparelho para fazer betão.
\section{Betónica}
\begin{itemize}
\item {Grp. gram.:f.}
\end{itemize}
\begin{itemize}
\item {Proveniência:(Lat. \textunderscore vettonica\textunderscore )}
\end{itemize}
Gênero de plantas labiadas, de raíz purgativa.
\section{Betonilha}
\begin{itemize}
\item {Grp. gram.:f.}
\end{itemize}
Substância, composta de areia e cimento de Portland, para revestimento de pavimentos.
(Cp. \textunderscore betão\textunderscore )
\section{Betouro}
\begin{itemize}
\item {Grp. gram.:m.}
\end{itemize}
O mesmo que \textunderscore abetoiro\textunderscore .
\section{Betral}
\begin{itemize}
\item {Grp. gram.:m.}
\end{itemize}
\begin{itemize}
\item {Utilização:Bras}
\end{itemize}
Terreno, plantado de betre.
\section{Betre}
\begin{itemize}
\item {Grp. gram.:m.}
\end{itemize}
\begin{itemize}
\item {Utilização:Bras}
\end{itemize}
O mesmo que \textunderscore betle\textunderscore .
\section{Bétula}
\begin{itemize}
\item {Grp. gram.:f.}
\end{itemize}
\begin{itemize}
\item {Proveniência:(Do lat. \textunderscore betula\textunderscore )}
\end{itemize}
Planta, também chamada vidoeiro.
\section{Betuláceas}
\begin{itemize}
\item {Grp. gram.:f. pl.}
\end{itemize}
Família de plantas, a que serve de typo a bétula.
\section{Betulineas}
\begin{itemize}
\item {Grp. gram.:f. pl.}
\end{itemize}
\begin{itemize}
\item {Proveniência:(De \textunderscore betulíneo\textunderscore )}
\end{itemize}
O mesmo que \textunderscore betuláceas\textunderscore .
\section{Betulíneo}
\begin{itemize}
\item {Grp. gram.:adj.}
\end{itemize}
Relativo á bétula.
\section{Bétulo}
\begin{itemize}
\item {Grp. gram.:m.}
\end{itemize}
\begin{itemize}
\item {Proveniência:(Lat. \textunderscore betulus\textunderscore )}
\end{itemize}
O mesmo ou melhor que \textunderscore bétylo\textunderscore .
\section{Betumar}
\begin{itemize}
\item {Grp. gram.:v. t.}
\end{itemize}
Cobrir, ligar, com betume.
\section{Betume}
\begin{itemize}
\item {Grp. gram.:m.}
\end{itemize}
\begin{itemize}
\item {Utilização:Mad}
\end{itemize}
\begin{itemize}
\item {Utilização:Prov.}
\end{itemize}
\begin{itemize}
\item {Utilização:minh.}
\end{itemize}
\begin{itemize}
\item {Proveniência:(Lat. \textunderscore bitumen\textunderscore )}
\end{itemize}
Espécie de mineral combustível.
Substância, preparada com cal, azeite e outros ingredientes, e que se empréga em vedações de água, etc.
Massa de arenito com óleo de linhaça, com que se pegam os vidros nos caixilhos e se tapam as fendas ou buracos da madeira.
O mesmo que \textunderscore graxa\textunderscore .
Caldo grosso.
\section{Betumeiro}
\begin{itemize}
\item {Grp. gram.:m.}
\end{itemize}
Fabricante ou vendedor de betume.
\section{Betuminoso}
\begin{itemize}
\item {Grp. gram.:adj.}
\end{itemize}
\begin{itemize}
\item {Proveniência:(Lat. \textunderscore bituminosus\textunderscore )}
\end{itemize}
Que tem betume.
Que é da natureza do betume.
\section{Betunes}
\begin{itemize}
\item {Grp. gram.:m. pl.}
\end{itemize}
Plantadores indianos.
\section{Bétylo}
\begin{itemize}
\item {Grp. gram.:m.}
\end{itemize}
\begin{itemize}
\item {Proveniência:(Gr. \textunderscore betulos\textunderscore )}
\end{itemize}
Pedra, que tinha certos sinaes e que entre os antigos era adorada como um ídolo.
\section{Béu}
\begin{itemize}
\item {Grp. gram.:m.}
\end{itemize}
Peixe marítimo, ordinário, do Brasil.
\section{Beudantina}
\begin{itemize}
\item {Grp. gram.:f.}
\end{itemize}
\begin{itemize}
\item {Proveniência:(De \textunderscore Beudant\textunderscore , n. p.)}
\end{itemize}
Variedade de nephelina, que se encontra nas vizinhanças do Vesúvio.
\section{Bêvera}
\begin{itemize}
\item {Grp. gram.:f.}
\end{itemize}
O mesmo ou melhor que \textunderscore bêbera\textunderscore .
\section{Bexiga}
\begin{itemize}
\item {Grp. gram.:f.}
\end{itemize}
\begin{itemize}
\item {Utilização:Chul.}
\end{itemize}
\begin{itemize}
\item {Utilização:Prov.}
\end{itemize}
\begin{itemize}
\item {Utilização:beir.}
\end{itemize}
\begin{itemize}
\item {Grp. gram.:Pl.}
\end{itemize}
\begin{itemize}
\item {Proveniência:(Lat. \textunderscore vesica\textunderscore )}
\end{itemize}
Reservatório músculo-membranoso, situado na parte inferior do abdome, e destinado a conter a urina, que deve sair pela urethra.
Chalaça.
Terreno, apparentemente afundam os pés dos transeuntes, as rodas dos carros, etc.
Varíola.
Vestígios, deixados no rosto pela varíola.
Fragmentos, que se despegam do casco, nas marinhas do Sado.
\section{Bexigal}
\begin{itemize}
\item {Grp. gram.:adj.}
\end{itemize}
Relativo ás bexigas, (\textunderscore variola\textunderscore ). Cf. Macedo, \textunderscore Burros\textunderscore , 315.
\section{Bexigar}
\begin{itemize}
\item {Grp. gram.:v. i.}
\end{itemize}
\begin{itemize}
\item {Utilização:Fam.}
\end{itemize}
\begin{itemize}
\item {Proveniência:(De \textunderscore bexiga\textunderscore )}
\end{itemize}
Chalacear.
Caçoar, motejar.
\section{Bexigas-de-carneiro}
\begin{itemize}
\item {Grp. gram.:f. pl.}
\end{itemize}
Gafeira, morrinha. Cf. \textunderscore Regul. de Saúde Pecuária\textunderscore , de 7-11-89, c. XXIII.
\section{Bexigoso}
\begin{itemize}
\item {Grp. gram.:adj.}
\end{itemize}
\begin{itemize}
\item {Proveniência:(De \textunderscore bexiga\textunderscore )}
\end{itemize}
Que tem os vestígios da varíola.
\section{Bexigueiro}
\begin{itemize}
\item {Grp. gram.:adj.}
\end{itemize}
\begin{itemize}
\item {Utilização:Chul.}
\end{itemize}
Que faz bexiga, que faz troça.
\section{Bexiguento}
\begin{itemize}
\item {Grp. gram.:adj.}
\end{itemize}
\begin{itemize}
\item {Utilização:Chul.}
\end{itemize}
Bexigoso.
O mesmo que \textunderscore bexigueiro\textunderscore .
\section{Bexuanas}
\begin{itemize}
\item {Grp. gram.:m. pl.}
\end{itemize}
Povos da África meridional.
\section{Bexuco}
\begin{itemize}
\item {Grp. gram.:m.}
\end{itemize}
Planta rasteira da América.
(Cast. \textunderscore bejuco\textunderscore )
\section{Bey}
\begin{itemize}
\item {Grp. gram.:m.}
\end{itemize}
(V.bei)
\section{Beya}
\begin{itemize}
\item {Grp. gram.:f.}
\end{itemize}
Nome que os alquimistas davam á água mercurial.
\section{Beyapuca}
\begin{itemize}
\item {Grp. gram.:f.}
\end{itemize}
Peixe das costas do Brasil.
\section{Bezerra}
\begin{itemize}
\item {fónica:zê}
\end{itemize}
\begin{itemize}
\item {Grp. gram.:f.}
\end{itemize}
Vitella, novilha.
(Fem. de \textunderscore bezerro\textunderscore )
\section{Bezerro}
\begin{itemize}
\item {Grp. gram.:m.}
\end{itemize}
Vitello, novilho.
Pelle curtida de vitello.
Planta escrofularínea.
Designação de várias espécies de phocas.
(B. lat. \textunderscore becerrus\textunderscore )
\section{Bezerro}
\begin{itemize}
\item {fónica:zê}
\end{itemize}
\begin{itemize}
\item {Grp. gram.:m.}
\end{itemize}
\begin{itemize}
\item {Utilização:T. de Leiria}
\end{itemize}
Buraco feito no fato por uma fagulha.
\section{Bezerrum}
\begin{itemize}
\item {Grp. gram.:adj.}
\end{itemize}
\begin{itemize}
\item {Utilização:Ant.}
\end{itemize}
Relativo a \textunderscore bezerro\textunderscore ^1: \textunderscore coiro bezerrum\textunderscore .
\section{Bezestan}
\begin{itemize}
\item {Grp. gram.:m.}
\end{itemize}
Designação turca dos mercados, nas cidades da Sýria.
\section{Bezigue}
\begin{itemize}
\item {Grp. gram.:m.}
\end{itemize}
\begin{itemize}
\item {Proveniência:(Fr. \textunderscore bézigue\textunderscore )}
\end{itemize}
Jôgo de cartas, entre dois parceiros, cada um dos quaes se serve de dois baralhos.
\section{Bezoar}
\begin{itemize}
\item {Grp. gram.:m.}
\end{itemize}
\begin{itemize}
\item {Proveniência:(Do ár. \textunderscore bahzar\textunderscore )}
\end{itemize}
Concreção calcária, que se fórma nos intestinos e vias urinárias dos quadrúpedes, e era considerada como antídoto.
\section{Bezoar}
\begin{itemize}
\item {Grp. gram.:v. i.}
\end{itemize}
\begin{itemize}
\item {Utilização:Prov.}
\end{itemize}
Diz-se da cabra, quando berra.
(Corr. de \textunderscore vozear\textunderscore )
\section{Bezoarticar}
\begin{itemize}
\item {Grp. gram.:v. t.}
\end{itemize}
Preparar com bezoártico.
\section{Bezoártico}
\begin{itemize}
\item {Grp. gram.:m.}
\end{itemize}
\begin{itemize}
\item {Proveniência:(De \textunderscore bezoar\textunderscore ^1)}
\end{itemize}
Designação desusada de um contra-veneno, em que entra o bezoar.
\section{Bfami}
\begin{itemize}
\item {Grp. gram.:m.}
\end{itemize}
\begin{itemize}
\item {Utilização:Mús.}
\end{itemize}
\begin{itemize}
\item {Utilização:ant.}
\end{itemize}
Nome, que se dava á nota \textunderscore si\textunderscore .
\section{Bi}
\begin{itemize}
\item {Grp. gram.:m.}
\end{itemize}
\begin{itemize}
\item {Utilização:Mús.}
\end{itemize}
\begin{itemize}
\item {Utilização:ant.}
\end{itemize}
Sýllaba, que se adoptou nos fins do século XVI, para designar no solfejo a última nota, que ainda não tinha nome.
\section{Bi...}
\begin{itemize}
\item {Grp. gram.:pref.}
\end{itemize}
\begin{itemize}
\item {Proveniência:(Do lat. \textunderscore bis\textunderscore )}
\end{itemize}
Duas vezes; duplicadamente.
\section{Biá}
\begin{itemize}
\item {Grp. gram.:m.}
\end{itemize}
Grande árvore indiana, (\textunderscore pterocarpus marsupium\textunderscore ).
\section{Biacuminado}
\begin{itemize}
\item {Grp. gram.:adj.}
\end{itemize}
\begin{itemize}
\item {Utilização:Bot.}
\end{itemize}
\begin{itemize}
\item {Proveniência:(Do lat. \textunderscore bis\textunderscore  + \textunderscore acuminatus\textunderscore )}
\end{itemize}
Diz-se de certos pêlos vegetaes, oppostos pela base.
\section{Biafada}
\begin{itemize}
\item {Grp. gram.:m.}
\end{itemize}
Língua do grupo felupo, falada pelos Biafadas.
\section{Biafadas}
\begin{itemize}
\item {Grp. gram.:m. pl.}
\end{itemize}
Uma das tríbos principaes da Guiné.
\section{Biagulhas}
\begin{itemize}
\item {Grp. gram.:f.}
\end{itemize}
\begin{itemize}
\item {Utilização:Prov.}
\end{itemize}
\begin{itemize}
\item {Utilização:trasm.}
\end{itemize}
\begin{itemize}
\item {Proveniência:(De \textunderscore bi...\textunderscore  + \textunderscore agulha\textunderscore )}
\end{itemize}
Erva, que se cria nos lameiros, e cuja fôlha é composta de dois filamentos.
Caruma dupla.
\section{Bialado}
\begin{itemize}
\item {Grp. gram.:adj.}
\end{itemize}
\begin{itemize}
\item {Proveniência:(De \textunderscore bi...\textunderscore  + \textunderscore alado\textunderscore )}
\end{itemize}
Que tem duas asas.
\section{Biangular}
\begin{itemize}
\item {Grp. gram.:adj.}
\end{itemize}
\begin{itemize}
\item {Proveniência:(De \textunderscore bi...\textunderscore  + \textunderscore angular\textunderscore )}
\end{itemize}
Que tem dois ângulos.
\section{Biaribu}
\begin{itemize}
\item {Grp. gram.:m.}
\end{itemize}
\begin{itemize}
\item {Utilização:Bras}
\end{itemize}
Maneira, que os selvagens têm, de assar a carne em covas abertas no chão.
\section{Biaristado}
\begin{itemize}
\item {Grp. gram.:adj.}
\end{itemize}
\begin{itemize}
\item {Proveniência:(Do lat. \textunderscore bis\textunderscore  + \textunderscore arista\textunderscore )}
\end{itemize}
Que tem duas arestas ou praganas.
\section{Biaro}
\begin{itemize}
\item {Grp. gram.:m.}
\end{itemize}
\begin{itemize}
\item {Proveniência:(Do lat. \textunderscore bis\textunderscore  + \textunderscore arum\textunderscore )}
\end{itemize}
Gênero de plantas aroídeas.
\section{Biatómico}
\begin{itemize}
\item {Grp. gram.:adj.}
\end{itemize}
\begin{itemize}
\item {Utilização:Chím.}
\end{itemize}
\begin{itemize}
\item {Proveniência:(De \textunderscore bi...\textunderscore  + \textunderscore atómico\textunderscore )}
\end{itemize}
Diz-se de um corpo que, tendo o mesmo volume e composição que outro, tem um número duplo de átomos simples.
\section{Biaxífero}
\begin{itemize}
\item {Grp. gram.:adj.}
\end{itemize}
\begin{itemize}
\item {Proveniência:(Do lat. \textunderscore bis\textunderscore  + \textunderscore axis\textunderscore  + \textunderscore ferre\textunderscore )}
\end{itemize}
Que tem dois eixos, (falando-se da inflorescência de certos vegetaes).
\section{Biba}
\begin{itemize}
\item {Grp. gram.:f.}
\end{itemize}
\begin{itemize}
\item {Utilização:T. de Macau}
\end{itemize}
Espécie de nêspera, (\textunderscore enobotiga japonica\textunderscore ).
(Talvez do chin. \textunderscore pi-po\textunderscore , nêspera)
\section{Bibásico}
\begin{itemize}
\item {Grp. gram.:adj.}
\end{itemize}
\begin{itemize}
\item {Utilização:Chím.}
\end{itemize}
\begin{itemize}
\item {Proveniência:(De \textunderscore bi...\textunderscore  + \textunderscore básico\textunderscore )}
\end{itemize}
Diz-se de um sal, que contém uma quantidade de base, dupla da do sal neutro que lhe corresponde.
\section{Bibe}
\begin{itemize}
\item {Grp. gram.:m.}
\end{itemize}
\begin{itemize}
\item {Proveniência:(Do lat. \textunderscore bibere\textunderscore ?)}
\end{itemize}
Espécie de avental para crianças, que lhes chega ao pescoço e é abotoado ou atado atrás, e destinado a evitar que os vestidos se sujem, com a comida ou a bebida.
\section{Bibe}
\begin{itemize}
\item {Grp. gram.:m.}
\end{itemize}
\begin{itemize}
\item {Utilização:Prov.}
\end{itemize}
\begin{itemize}
\item {Utilização:alent.}
\end{itemize}
Ave de arribação, o mesmo que \textunderscore abibe\textunderscore .
\section{Bibe}
\begin{itemize}
\item {Grp. gram.:m.}
\end{itemize}
\begin{itemize}
\item {Utilização:Prov.}
\end{itemize}
\begin{itemize}
\item {Utilização:alent.}
\end{itemize}
Vallador.
\section{Biberão}
\begin{itemize}
\item {Grp. gram.:m.}
\end{itemize}
\begin{itemize}
\item {Proveniência:(Do fr. \textunderscore biberon\textunderscore )}
\end{itemize}
Pequeno vaso, que se emprega na lactação artificial das crianças.
\section{Bibes}
\begin{itemize}
\item {Grp. gram.:m.}
\end{itemize}
\begin{itemize}
\item {Utilização:Prov.}
\end{itemize}
O mesmo que \textunderscore abibe\textunderscore .
\section{Bibi}
\begin{itemize}
\item {Grp. gram.:f.}
\end{itemize}
Palmeira americana, cuja madeira é negra.
\section{Bibió}
\begin{itemize}
\item {Grp. gram.:m.}
\end{itemize}
Feroz animal indiano, talvez espécie de tigre. Cf. Th. Ribeiro, \textunderscore Jornadas\textunderscore , II, 277; Lopes Mendes, \textunderscore Índia Port.\textunderscore 
\section{Biblá}
\begin{itemize}
\item {Grp. gram.:f.}
\end{itemize}
(V.biá)
\section{Bíblia}
\begin{itemize}
\item {Grp. gram.:f.}
\end{itemize}
\begin{itemize}
\item {Proveniência:(Gr. \textunderscore biblia\textunderscore )}
\end{itemize}
Sagrada escritura; livros sagrados do \textunderscore Antigo\textunderscore  e \textunderscore Novo Testamento\textunderscore .
\section{Bibliátrica}
\begin{itemize}
\item {Grp. gram.:f.}
\end{itemize}
\begin{itemize}
\item {Proveniência:(Do gr. \textunderscore biblion\textunderscore  + \textunderscore iatrike\textunderscore )}
\end{itemize}
Arte de restaurar os livros.
\section{Bíblico}
\begin{itemize}
\item {Grp. gram.:adj.}
\end{itemize}
Pertencente, relativo, á Bíblia: \textunderscore tradições bíblicas\textunderscore .
\section{Bibliografia}
\begin{itemize}
\item {Grp. gram.:f.}
\end{itemize}
Conhecimento dos livros, quanto aos seus característicos exteriores.
Relação das obras de um autor, ou das obras sôbre determinado assumpto.
Secção de uma publicação periódica, destinada ao registo das publicações recentes.
(Cp. \textunderscore bibliógrapho\textunderscore )
\section{Bibliográfico}
\begin{itemize}
\item {Grp. gram.:adj.}
\end{itemize}
Pertencente ou relativo á bibliografia.
\section{Bibliógrafo}
\begin{itemize}
\item {Grp. gram.:m.}
\end{itemize}
\begin{itemize}
\item {Proveniência:(Do gr. \textunderscore biblion\textunderscore  + \textunderscore graphein\textunderscore )}
\end{itemize}
Aquelle que escreve á cêrca de livros.
Aquelle que é versado em bibliografia.
\section{Bibliographia}
\begin{itemize}
\item {Grp. gram.:f.}
\end{itemize}
Conhecimento dos livros, quanto aos seus característicos exteriores.
Relação das obras de um autor, ou das obras sôbre determinado assumpto.
Secção de uma publicação periódica, destinada ao registo das publicações recentes.
(Cp. \textunderscore bibliógrapho\textunderscore )
\section{Bibliográphico}
\begin{itemize}
\item {Grp. gram.:adj.}
\end{itemize}
Pertencente ou relativo á bibliographia.
\section{Bibliógrapho}
\begin{itemize}
\item {Grp. gram.:m.}
\end{itemize}
\begin{itemize}
\item {Proveniência:(Do gr. \textunderscore biblion\textunderscore  + \textunderscore graphein\textunderscore )}
\end{itemize}
Aquelle que escreve á cêrca de livros.
Aquelle que é versado em bibliographia.
\section{Bibliólithos}
\begin{itemize}
\item {Grp. gram.:m. pl.}
\end{itemize}
\begin{itemize}
\item {Proveniência:(Do gr. \textunderscore biblion\textunderscore  + \textunderscore lithos\textunderscore )}
\end{itemize}
Pedras calcárias e xistosas, cujas lâminas parecem fôlhas de livros.
\section{Bibliólitos}
\begin{itemize}
\item {Grp. gram.:m. pl.}
\end{itemize}
\begin{itemize}
\item {Proveniência:(Do gr. \textunderscore biblion\textunderscore  + \textunderscore lithos\textunderscore )}
\end{itemize}
Pedras calcárias e xistosas, cujas lâminas parecem fôlhas de livros.
\section{Bibi}
\begin{itemize}
\item {Grp. gram.:m.}
\end{itemize}
Árvore da Índia portuguesa, (\textunderscore semecarpus anacardium\textunderscore , Roxb).
\section{Bibliófilo}
\begin{itemize}
\item {Grp. gram.:m.}
\end{itemize}
\begin{itemize}
\item {Proveniência:(Do gr. \textunderscore biblion\textunderscore  + \textunderscore philos\textunderscore )}
\end{itemize}
Aquelle que tem amor aos livros.
Colleccionador de livros.
\section{Bibliologia}
\begin{itemize}
\item {Grp. gram.:f.}
\end{itemize}
Parte theórica da bibliographia, que trata das regras desta sciência e lhe serve de preliminar.
(Cp. \textunderscore bibliologo\textunderscore )
\section{Bibliologo}
\begin{itemize}
\item {Grp. gram.:m.}
\end{itemize}
\begin{itemize}
\item {Proveniência:(Do gr. \textunderscore biblion\textunderscore  + \textunderscore logos\textunderscore )}
\end{itemize}
Aquelle que é versado em bibliologia.
\section{Bibliomancia}
\begin{itemize}
\item {Grp. gram.:f.}
\end{itemize}
\begin{itemize}
\item {Proveniência:(Do gr. \textunderscore biblion\textunderscore  + \textunderscore manteia\textunderscore )}
\end{itemize}
Supposta arte de adivinhar por meio de um livro, que se abre ao acaso.
\section{Bibliomania}
\begin{itemize}
\item {Grp. gram.:f.}
\end{itemize}
\begin{itemize}
\item {Proveniência:(Do gr. \textunderscore biblion\textunderscore  e \textunderscore mania\textunderscore )}
\end{itemize}
Paixão pelos livros, mormente pelos que são raros.
\section{Bibliomaníaco}
\begin{itemize}
\item {Grp. gram.:adj.}
\end{itemize}
Que tem bibliomania.
\section{Bibliómano}
\begin{itemize}
\item {Grp. gram.:m.}
\end{itemize}
Aquelle que é bibliomaníaco.
\section{Biblióphilo}
\begin{itemize}
\item {Grp. gram.:m.}
\end{itemize}
\begin{itemize}
\item {Proveniência:(Do gr. \textunderscore biblion\textunderscore  + \textunderscore philos\textunderscore )}
\end{itemize}
Aquelle que tem amor aos livros.
Colleccionador de livros.
\section{Bibliopola}
\begin{itemize}
\item {Grp. gram.:f.}
\end{itemize}
\begin{itemize}
\item {Utilização:Des.}
\end{itemize}
\begin{itemize}
\item {Proveniência:(Lat. \textunderscore bibliopola\textunderscore )}
\end{itemize}
Aquelle que vende livros; livreiro.
\section{Biblioteca}
\begin{itemize}
\item {Grp. gram.:f.}
\end{itemize}
\begin{itemize}
\item {Proveniência:(Lat. \textunderscore bibliotheca\textunderscore )}
\end{itemize}
Reunião de livros, ordenadamente dispostos.
Estantes, occupadas por livros.
Casa ou lugar, onde se depositam livros, para uso público ou particular.
\section{Bibliotecário}
\begin{itemize}
\item {Grp. gram.:m.}
\end{itemize}
\begin{itemize}
\item {Proveniência:(Lat. \textunderscore bibliothecarius\textunderscore )}
\end{itemize}
Aquelle que administra uma biblioteca.
\section{Biblioteconomia}
\begin{itemize}
\item {Grp. gram.:f.}
\end{itemize}
\begin{itemize}
\item {Proveniência:(Do gr. \textunderscore bibliotheke\textunderscore  + \textunderscore nomos\textunderscore )}
\end{itemize}
Arte de organizar bibliotecas.
\section{Bibliotheca}
\begin{itemize}
\item {Grp. gram.:f.}
\end{itemize}
\begin{itemize}
\item {Proveniência:(Lat. \textunderscore bibliotheca\textunderscore )}
\end{itemize}
Reunião de livros, ordenadamente dispostos.
Estantes, occupadas por livros.
Casa ou lugar, onde se depositam livros, para uso público ou particular.
\section{Bibliothecário}
\begin{itemize}
\item {Grp. gram.:m.}
\end{itemize}
\begin{itemize}
\item {Proveniência:(Lat. \textunderscore bibliothecarius\textunderscore )}
\end{itemize}
Aquelle que administra uma bibliotheca.
\section{Bibliotheconomia}
\begin{itemize}
\item {Grp. gram.:f.}
\end{itemize}
\begin{itemize}
\item {Proveniência:(Do gr. \textunderscore bibliotheke\textunderscore  + \textunderscore nomos\textunderscore )}
\end{itemize}
Arte de organizar bibliothecas.
\section{Biblista}
\begin{itemize}
\item {Grp. gram.:m.}
\end{itemize}
Aquelle que é versado na \textunderscore Bíblia\textunderscore .
\section{Biblística}
\begin{itemize}
\item {Grp. gram.:f.}
\end{itemize}
\begin{itemize}
\item {Proveniência:(De \textunderscore biblista\textunderscore )}
\end{itemize}
Conhecimento ou notícia bibliográphica dos livros da \textunderscore Biblia\textunderscore .
\section{Bibo}
\begin{itemize}
\item {Grp. gram.:m.}
\end{itemize}
O mesmo que \textunderscore anacardo\textunderscore .
\section{Bibó}
\begin{itemize}
\item {Grp. gram.:m.}
\end{itemize}
Árvore da Índia portuguesa, (\textunderscore semecarpus anacardium\textunderscore , Roxb).
\section{Biboca}
\begin{itemize}
\item {Grp. gram.:f.}
\end{itemize}
\begin{itemize}
\item {Utilização:Bras}
\end{itemize}
Barranco, feito por enxurradas, que torna diffícil e até perigoso o trânsito.
Baiuca, bodega.
(Do tupi \textunderscore ibiboca\textunderscore )
\section{Bibói}
\begin{itemize}
\item {Grp. gram.:m.}
\end{itemize}
Árvore da Índia portuguesa, (\textunderscore semecarpus anacardium\textunderscore , Roxb).
\section{Bíbulo}
\begin{itemize}
\item {Grp. gram.:adj.}
\end{itemize}
\begin{itemize}
\item {Proveniência:(Lat. \textunderscore bibulus\textunderscore )}
\end{itemize}
Que bebe, que absorve os líquidos: \textunderscore as bíbulas areias\textunderscore .
\section{Bica}
\begin{itemize}
\item {Grp. gram.:f.}
\end{itemize}
\begin{itemize}
\item {Utilização:Prov.}
\end{itemize}
\begin{itemize}
\item {Utilização:beir.}
\end{itemize}
\begin{itemize}
\item {Utilização:Prov.}
\end{itemize}
\begin{itemize}
\item {Utilização:minh.}
\end{itemize}
\begin{itemize}
\item {Utilização:Mad}
\end{itemize}
\begin{itemize}
\item {Utilização:Bras}
\end{itemize}
\begin{itemize}
\item {Utilização:escol.}
\end{itemize}
\begin{itemize}
\item {Utilização:T. de Caminha}
\end{itemize}
\begin{itemize}
\item {Grp. gram.:Adj.}
\end{itemize}
\begin{itemize}
\item {Utilização:T. de Barcelos}
\end{itemize}
\begin{itemize}
\item {Proveniência:(De \textunderscore bico\textunderscore )}
\end{itemize}
Tubo, pequeno canal, meia cana ou telha, por onde corre água, caíndo della de certa altura.
Líquido, que cái em veia ou fio.
Nome de um peixe das costas de Portugal.
Pão de trigo, comprido e chato.
Pão ázymo, cozido na lareira.
Espécie de rato.
O mesmo que \textunderscore carreiro\textunderscore .
Grande número de approvações em exames.
Facilidade de ficar approvado sem saber nada.
Sêmea fina.
Nome de uma planta madeirense, (\textunderscore anthus trivialis\textunderscore ).
(Cp. \textunderscore bico\textunderscore ^1)
Diz-se da estôpa de melhór qualidade, cujas estrigas terminam em bico ou ponta aguda.
\section{Biça}
\begin{itemize}
\item {Grp. gram.:f.}
\end{itemize}
Antigo pêso de oiro, na Índia:«\textunderscore eu lhe daria trinta mil biças de prata\textunderscore ». \textunderscore Peregrinação\textunderscore , XLVIII.
\section{Bicada}
\begin{itemize}
\item {Grp. gram.:f.}
\end{itemize}
\begin{itemize}
\item {Grp. gram.:Pl.}
\end{itemize}
Pancada, ou golpe com o bico.
Aquillo que uma ave leva no bico, de uma vez.
Sopé, princípio, entrada de um bosque:«\textunderscore com tudo se agasalhou em uma bicada de um mato\textunderscore ». Bernardim, \textunderscore Menina e Moça\textunderscore .
Extremidade longitudinal de uma serra.
Ramas de árvores.
\section{Bicada}
\begin{itemize}
\item {Grp. gram.:f.}
\end{itemize}
\begin{itemize}
\item {Utilização:Bras}
\end{itemize}
Grande bica; calha.
\section{Bicado}
\begin{itemize}
\item {Grp. gram.:adj.}
\end{itemize}
Diz-se da ave que, nos brasões, tem no bico esmalte differente do do corpo.
\section{Bical}
\begin{itemize}
\item {Grp. gram.:adj.}
\end{itemize}
Que tem bico.
Diz-se de uma espécie de cereja vermelha, rija e cordiforme, com uma pequena saliência em bico, na parte opposta ao pé.
Diz-se de uma variedade de azeitona, também chamada \textunderscore cornalhuda\textunderscore .
Diz-se de uma casta de uva da Bairrada.
\section{Bical}
\begin{itemize}
\item {Grp. gram.:m.}
\end{itemize}
Um dos dialectos das Filippinas.
\section{Bicalado}
\begin{itemize}
\item {Grp. gram.:m.}
\end{itemize}
Ave palmípede aquática.
\section{Bicanço}
\begin{itemize}
\item {Grp. gram.:m.}
\end{itemize}
\begin{itemize}
\item {Utilização:Pop.}
\end{itemize}
Bico grande.
\section{Bicancra}
\begin{itemize}
\item {Grp. gram.:m.  e  adj.}
\end{itemize}
\begin{itemize}
\item {Utilização:Prov.}
\end{itemize}
\begin{itemize}
\item {Utilização:beir.}
\end{itemize}
Indivíduo narigudo.
(Cp. \textunderscore bicanço\textunderscore )
\section{Bicançudo}
\begin{itemize}
\item {Grp. gram.:m.}
\end{itemize}
\begin{itemize}
\item {Proveniência:(De \textunderscore bicanço\textunderscore )}
\end{itemize}
Gênero de peixes cartilaginosos.
\section{Bicapsular}
\begin{itemize}
\item {Grp. gram.:adj.}
\end{itemize}
\begin{itemize}
\item {Utilização:Bot.}
\end{itemize}
\begin{itemize}
\item {Proveniência:(De \textunderscore bi...\textunderscore  + \textunderscore cápsula\textunderscore )}
\end{itemize}
Diz-se do órgão vegetal que tem duas cápsulas.
\section{Bicar}
\begin{itemize}
\item {Grp. gram.:v. t.}
\end{itemize}
Picar com o bico.
Exprimir com bicadas.
\section{Bicarada}
\begin{itemize}
\item {Grp. gram.:f.}
\end{itemize}
\begin{itemize}
\item {Utilização:Prov.}
\end{itemize}
\begin{itemize}
\item {Utilização:dur.}
\end{itemize}
\begin{itemize}
\item {Proveniência:(De \textunderscore bico\textunderscore )}
\end{itemize}
Porção de pequenas dívidas.
\section{Bicarbonado}
\begin{itemize}
\item {Grp. gram.:adj.}
\end{itemize}
\begin{itemize}
\item {Proveniência:(De \textunderscore bi...\textunderscore  + \textunderscore carbóne\textunderscore )}
\end{itemize}
Que contém duas proporções de carbóne.
\section{Bicarbonato}
\begin{itemize}
\item {Grp. gram.:m.}
\end{itemize}
\begin{itemize}
\item {Proveniência:(De \textunderscore bi...\textunderscore  + \textunderscore carbonato\textunderscore )}
\end{itemize}
Sal, em que o ácido carbónico contém duas vezes tanto oxygênio como a base.
\section{Bicarboneto}
\begin{itemize}
\item {fónica:nê}
\end{itemize}
\begin{itemize}
\item {Grp. gram.:m.}
\end{itemize}
\begin{itemize}
\item {Proveniência:(De \textunderscore bi...\textunderscore  + \textunderscore carboneto\textunderscore )}
\end{itemize}
Combinação, em que o carbóne é em quantidade dupla da que há no carboneto.
\section{Bicarbureto}
\begin{itemize}
\item {Grp. gram.:m.}
\end{itemize}
(V.bicarboneto)
\section{Bicarenado}
\begin{itemize}
\item {Grp. gram.:adj.}
\end{itemize}
\begin{itemize}
\item {Proveniência:(De \textunderscore bi...\textunderscore  + \textunderscore carena\textunderscore )}
\end{itemize}
Que tem duas carenas ou saliências longitudinaes.
\section{Bicarrada}
\begin{itemize}
\item {Grp. gram.:f.}
\end{itemize}
\begin{itemize}
\item {Utilização:T. da Bairrada}
\end{itemize}
\begin{itemize}
\item {Proveniência:(De \textunderscore bico\textunderscore )}
\end{itemize}
Quaesquer miudezas, que as aves levam no bico, para construir o ninho.
\section{Bicas}
\begin{itemize}
\item {Grp. gram.:f. pl.}
\end{itemize}
\begin{itemize}
\item {Utilização:Prov.}
\end{itemize}
\begin{itemize}
\item {Utilização:beir.}
\end{itemize}
Refeição festiva, com que os noivos e suas famílias celebram os proclamas do casamento.
(Talvez de \textunderscore bica\textunderscore , bolo chato e comprido, usado na Beira-Baixa)
\section{Bicaudado}
\begin{itemize}
\item {Grp. gram.:adj.}
\end{itemize}
\begin{itemize}
\item {Proveniência:(De \textunderscore bi...\textunderscore  + \textunderscore cauda\textunderscore )}
\end{itemize}
Que tem duas caudas ou dois appêndices em fórma de cauda.
\section{Bicéfalo}
\begin{itemize}
\item {Grp. gram.:adj.}
\end{itemize}
\begin{itemize}
\item {Proveniência:(De \textunderscore bi...\textunderscore  + gr. \textunderscore kephale\textunderscore )}
\end{itemize}
Que tem duas cabeças.
\section{Bicellular}
\begin{itemize}
\item {Grp. gram.:adj.}
\end{itemize}
\begin{itemize}
\item {Proveniência:(De \textunderscore bi...\textunderscore  + \textunderscore cellular\textunderscore )}
\end{itemize}
Que tem duas céllulas.
\section{Bicelular}
\begin{itemize}
\item {Grp. gram.:adj.}
\end{itemize}
\begin{itemize}
\item {Proveniência:(De \textunderscore bi...\textunderscore  + \textunderscore cellular\textunderscore )}
\end{itemize}
Que tem duas células.
\section{Bicéphalo}
\begin{itemize}
\item {Grp. gram.:adj.}
\end{itemize}
\begin{itemize}
\item {Proveniência:(De \textunderscore bi...\textunderscore  + gr. \textunderscore kephale\textunderscore )}
\end{itemize}
Que tem duas cabeças.
\section{Bíceps}
\begin{itemize}
\item {Grp. gram.:m.}
\end{itemize}
\begin{itemize}
\item {Proveniência:(Lat. \textunderscore biceps\textunderscore .)}
\end{itemize}
Nome de alguns músculos, cada um dos quaes tem dois ligamentos ou cabeças na parte superior.
\section{Bicha}
\begin{itemize}
\item {Grp. gram.:f.}
\end{itemize}
\begin{itemize}
\item {Utilização:Fam.}
\end{itemize}
\begin{itemize}
\item {Utilização:Prov.}
\end{itemize}
\begin{itemize}
\item {Utilização:dur.}
\end{itemize}
\begin{itemize}
\item {Utilização:Náut.}
\end{itemize}
\begin{itemize}
\item {Utilização:Mad}
\end{itemize}
\begin{itemize}
\item {Utilização:bras}
\end{itemize}
\begin{itemize}
\item {Utilização:Gír.}
\end{itemize}
\begin{itemize}
\item {Utilização:Pop.}
\end{itemize}
\begin{itemize}
\item {Utilização:Bras. do Rio}
\end{itemize}
\begin{itemize}
\item {Utilização:Prov.}
\end{itemize}
\begin{itemize}
\item {Utilização:trasm.}
\end{itemize}
\begin{itemize}
\item {Grp. gram.:Loc.}
\end{itemize}
\begin{itemize}
\item {Utilização:fam.}
\end{itemize}
\begin{itemize}
\item {Grp. gram.:Pl.}
\end{itemize}
\begin{itemize}
\item {Utilização:Bras}
\end{itemize}
\begin{itemize}
\item {Grp. gram.:Loc.}
\end{itemize}
\begin{itemize}
\item {Utilização:bras}
\end{itemize}
\begin{itemize}
\item {Proveniência:(Do it. \textunderscore biscia\textunderscore )}
\end{itemize}
Designação commum aos animaes, que têm corpo comprido, sem pernas.
Sanguesuga: \textunderscore deitar bichas nas pernas\textunderscore .
Figura de dança em que todos os pares dão as mãos uns aos outros, em fileira.
Mulher muito irritada.
Fileira de pessôas, umas atrás das outras.
Tumor.
Antigo corpo de tropa.
O mesmo que \textunderscore cobra\textunderscore .
Arrecada, em feitio de cobra.
Qualquer objecto que, pelo seu feitio ou movimento sinuoso, dá ideia de um reptil.
Tira de gachéta, que tem sapatilho numa extremidade e na outra uma mão com um pequeno cabo.
O mesmo que \textunderscore milhano\textunderscore .
O mesmo que \textunderscore aguardente\textunderscore .
Galão ou divisa na manga de um uniforme.
Serpentina do alambique, nos engenhos de açúcar.
O mesmo que \textunderscore víbora\textunderscore ^1.
\textunderscore Bicha de sete cabeças\textunderscore , grande dificuldade.
Arrecadas.
\textunderscore Fazer bichas\textunderscore , fazer diabruras.
\section{Bicha-cadela}
\begin{itemize}
\item {Grp. gram.:f.}
\end{itemize}
Insecto orthóptero, de corpo alongado, com seis pernas, (\textunderscore forficula auricularis\textunderscore , Lin.).
\section{Bichaço}
\begin{itemize}
\item {Grp. gram.:m.}
\end{itemize}
\begin{itemize}
\item {Utilização:Pop.}
\end{itemize}
\begin{itemize}
\item {Proveniência:(De \textunderscore bicho\textunderscore )}
\end{itemize}
Homem importante, rico.
\section{Bichaco}
\begin{itemize}
\item {Grp. gram.:m.}
\end{itemize}
\begin{itemize}
\item {Utilização:Marn.}
\end{itemize}
\textunderscore Virões de bichaco\textunderscore , buracos, por onde a água sai dos corredores para as cabeceiras.
\section{Bichado}
\begin{itemize}
\item {Grp. gram.:adj.}
\end{itemize}
\begin{itemize}
\item {Utilização:Bras}
\end{itemize}
\begin{itemize}
\item {Proveniência:(De \textunderscore bichar\textunderscore )}
\end{itemize}
O mesmo que \textunderscore bichoso\textunderscore .
\section{Bichana}
\begin{itemize}
\item {Utilização:T. da Bairrada}
\end{itemize}
\begin{itemize}
\item {Utilização:Fam.}
\end{itemize}
Partes pudendas da mulher.
O mesmo que \textunderscore gata\textunderscore .
(Fem. de \textunderscore bichano\textunderscore )
\section{Bichanado}
\begin{itemize}
\item {Grp. gram.:adj.}
\end{itemize}
\begin{itemize}
\item {Proveniência:(De \textunderscore bichanar\textunderscore )}
\end{itemize}
Pronunciado em voz baixa.
\section{Bichanar}
\begin{itemize}
\item {Grp. gram.:v. i.}
\end{itemize}
\begin{itemize}
\item {Utilização:Fam.}
\end{itemize}
\begin{itemize}
\item {Proveniência:(T. onom.)}
\end{itemize}
Falar baixo, ciciando as palavras.
\section{Bichancrice}
\begin{itemize}
\item {Grp. gram.:f.}
\end{itemize}
Acto de fazer bichancros.
\section{Bichancros}
\begin{itemize}
\item {Grp. gram.:m. pl.}
\end{itemize}
Gestos ridículos de namorado.
\section{Bichaneira}
\begin{itemize}
\item {Grp. gram.:m. pl.}
\end{itemize}
Abertura ou registo, por meio do qual os padeiros regularizam o calor do forno.
(Colhido em Turquel)
\section{Bichano}
\begin{itemize}
\item {Grp. gram.:m.}
\end{itemize}
\begin{itemize}
\item {Utilização:Fam.}
\end{itemize}
\begin{itemize}
\item {Proveniência:(De \textunderscore bicho\textunderscore )}
\end{itemize}
Gato, especialmente gato novo.
\section{Bichar}
\begin{itemize}
\item {Grp. gram.:v. i.}
\end{itemize}
\begin{itemize}
\item {Utilização:Bras}
\end{itemize}
Encher-se de bichos (a fruta e outras coisas).
\section{Bichará}
\begin{itemize}
\item {Grp. gram.:m.}
\end{itemize}
Tecido grosseiro de lan preta e branca, no sul do Brasil.
\section{Bicharengo}
\begin{itemize}
\item {Grp. gram.:m.}
\end{itemize}
\begin{itemize}
\item {Utilização:T. da Certã}
\end{itemize}
O mesmo que \textunderscore texugo\textunderscore .
\section{Bicharia}
\begin{itemize}
\item {Grp. gram.:f.}
\end{itemize}
\begin{itemize}
\item {Utilização:Pop.}
\end{itemize}
Reunião de bichos.
Ajuntamento de pessôas.
\section{Bicharoco}
\begin{itemize}
\item {fónica:charô}
\end{itemize}
\begin{itemize}
\item {Grp. gram.:m.}
\end{itemize}
\begin{itemize}
\item {Utilização:Pop.}
\end{itemize}
Grande bicho; bicho repugnante.
(Cp. cast. \textunderscore bicharroco\textunderscore )
\section{Bicharrão}
\begin{itemize}
\item {Grp. gram.:m.}
\end{itemize}
Bicho grande. Cf. Garrett, \textunderscore Fábulas\textunderscore , 61.
\section{Bicheira}
\begin{itemize}
\item {Grp. gram.:f.}
\end{itemize}
\begin{itemize}
\item {Utilização:Bras}
\end{itemize}
\begin{itemize}
\item {Utilização:Prov.}
\end{itemize}
\begin{itemize}
\item {Utilização:dur.}
\end{itemize}
\begin{itemize}
\item {Proveniência:(De \textunderscore bicho\textunderscore )}
\end{itemize}
Ferida nos animaes, com bichos que nella depositam os seus ovos.
O mesmo que \textunderscore bicheiro\textunderscore ^1.
Porção de piolhos na cabeça.
\section{Bicheiro}
\begin{itemize}
\item {Grp. gram.:m.}
\end{itemize}
\begin{itemize}
\item {Utilização:Bras}
\end{itemize}
\begin{itemize}
\item {Grp. gram.:Adj.}
\end{itemize}
\begin{itemize}
\item {Utilização:Fig.}
\end{itemize}
\begin{itemize}
\item {Utilização:Prov.}
\end{itemize}
\begin{itemize}
\item {Utilização:minh.}
\end{itemize}
\begin{itemize}
\item {Proveniência:(De \textunderscore bicho\textunderscore )}
\end{itemize}
Frasco, depósito, de sanguesugas.
Croque.
Utensílio piscatório, composto de uma vara com anzol.
Vendedor de bilhetes do jôgo dos bichos.
Que se sustenta de bichos.
Que procura muito, que é minucioso.
Diz-se do arado que lavra fundo.
\section{Bicheiro}
\begin{itemize}
\item {Grp. gram.:m.}
\end{itemize}
\begin{itemize}
\item {Utilização:Prov.}
\end{itemize}
\begin{itemize}
\item {Utilização:alent.}
\end{itemize}
Tubozinho de lata, por onde sái a extremidade superior da torcida das lanternas.
(Talvez do cast. \textunderscore mechero\textunderscore )
\section{Bicheiro}
\begin{itemize}
\item {Grp. gram.:m.}
\end{itemize}
\begin{itemize}
\item {Utilização:Prov.}
\end{itemize}
\begin{itemize}
\item {Utilização:trasm.}
\end{itemize}
\begin{itemize}
\item {Proveniência:(Do lat. hyp. \textunderscore aversiarius\textunderscore )}
\end{itemize}
Lugar, onde não dá o sol.
\section{Bichento}
\begin{itemize}
\item {Grp. gram.:adj.}
\end{itemize}
\begin{itemize}
\item {Utilização:Bras}
\end{itemize}
Que tem bichos nos pés.
Cambaio.
\section{Bichinha}
\begin{itemize}
\item {Grp. gram.:f.}
\end{itemize}
Pequeno bolo de farinha, açúcar e ovos.
Peça de fogo de artifício, o mesmo que \textunderscore valverde\textunderscore .
\section{Bichinha-gata}
\begin{itemize}
\item {Grp. gram.:f.}
\end{itemize}
\begin{itemize}
\item {Utilização:Fam.}
\end{itemize}
Afagos, caricias.
\section{Bichinina}
\begin{itemize}
\item {Grp. gram.:f.}
\end{itemize}
\begin{itemize}
\item {Utilização:Prov.}
\end{itemize}
\begin{itemize}
\item {Utilização:alent.}
\end{itemize}
\begin{itemize}
\item {Proveniência:(De \textunderscore bicha\textunderscore )}
\end{itemize}
Peça de fogo de artifício, conhecida geralmente por \textunderscore bicha de rabear\textunderscore .
\section{Bicho}
\begin{itemize}
\item {Grp. gram.:m.}
\end{itemize}
\begin{itemize}
\item {Utilização:Fam.}
\end{itemize}
\begin{itemize}
\item {Utilização:Pop.}
\end{itemize}
Nome commum aos animaes terrestres, especialmente aos vermes e insectos.
Piolho.
Pessôa feia; pessôa intratável, solitária.
Espécie de jôgo popular.
Cancro.
\textunderscore Matar o bicho\textunderscore , beber aguardente ou outra bebida alcoólica, antes de almôço.
(Cp. \textunderscore bicha\textunderscore )
\section{Bichoca}
\begin{itemize}
\item {Grp. gram.:f.}
\end{itemize}
\begin{itemize}
\item {Utilização:Pop.}
\end{itemize}
\begin{itemize}
\item {Utilização:Fam.}
\end{itemize}
\begin{itemize}
\item {Proveniência:(De \textunderscore bicho\textunderscore )}
\end{itemize}
Minhoca.
Pequeno leicenço.
O pênis de criancinhas.
\section{Bicho-carpinteiro}
\begin{itemize}
\item {Grp. gram.:m.}
\end{itemize}
\begin{itemize}
\item {Utilização:Pop.}
\end{itemize}
\begin{itemize}
\item {Grp. gram.:Loc.}
\end{itemize}
\begin{itemize}
\item {Utilização:pop.}
\end{itemize}
O mesmo que \textunderscore escaravêlho\textunderscore .
\textunderscore Têr bicho-carpinteiro\textunderscore , sêr inquieto ou traquinas.
\section{Bichoco}
\begin{itemize}
\item {fónica:chô}
\end{itemize}
\begin{itemize}
\item {Grp. gram.:m.}
\end{itemize}
\begin{itemize}
\item {Utilização:Pop.}
\end{itemize}
\begin{itemize}
\item {Utilização:Prov.}
\end{itemize}
\begin{itemize}
\item {Grp. gram.:Adj.}
\end{itemize}
\begin{itemize}
\item {Utilização:Bras. do S}
\end{itemize}
\begin{itemize}
\item {Proveniência:(De \textunderscore bicho\textunderscore )}
\end{itemize}
Leicenço.
Excreto, de côr verde, que os recém-nascidos expellem, depois do ferrado.
Diz-se do cavallo, a que incham os pés por falta de exercício.
\section{Bicho-da-toca}
\begin{itemize}
\item {Grp. gram.:m.}
\end{itemize}
\begin{itemize}
\item {Utilização:Fam.}
\end{itemize}
Pessôa bisonha, acanhada, que gosta de viver só.
\section{Bicho-de-conta}
\begin{itemize}
\item {Grp. gram.:m.}
\end{itemize}
Pequeno crustáceo, que vive entre pedras ou em lugares sombrios e húmidos.
\section{Bicho-do-areeiro}
\begin{itemize}
\item {Grp. gram.:m.}
\end{itemize}
\begin{itemize}
\item {Utilização:Mad}
\end{itemize}
Ave, o mesmo que \textunderscore patagarro\textunderscore .
\section{Bichoiro}
\begin{itemize}
\item {Grp. gram.:m.}
\end{itemize}
\begin{itemize}
\item {Utilização:T. da Bairrada}
\end{itemize}
Seixo miúdo, pedrinha.
\section{Bichoso}
\begin{itemize}
\item {Grp. gram.:adj.}
\end{itemize}
Que tem bichos: \textunderscore maçans bichosas\textunderscore .
\section{Bicínio}
\begin{itemize}
\item {Grp. gram.:m.}
\end{itemize}
\begin{itemize}
\item {Utilização:Des.}
\end{itemize}
\begin{itemize}
\item {Proveniência:(Lat. \textunderscore bicinium\textunderscore )}
\end{itemize}
O mesmo que \textunderscore dueto\textunderscore .
\section{Bicipital}
\begin{itemize}
\item {Grp. gram.:adj.}
\end{itemize}
Relativo ao bíceps.
\section{Bicípite}
\begin{itemize}
\item {Grp. gram.:adj.}
\end{itemize}
\begin{itemize}
\item {Proveniência:(Do lat. \textunderscore biceps\textunderscore , \textunderscore bicipitis\textunderscore )}
\end{itemize}
Que tem duas cabeças ou dois cumes.
\section{Biclíneo}
\begin{itemize}
\item {Grp. gram.:m.}
\end{itemize}
\begin{itemize}
\item {Utilização:Ant.}
\end{itemize}
\begin{itemize}
\item {Proveniência:(T. hyb., do lat. \textunderscore bis\textunderscore  + gr. \textunderscore kline\textunderscore )}
\end{itemize}
Leito de madeira para duas pessôas.
\section{Bico}
\begin{itemize}
\item {Grp. gram.:m.}
\end{itemize}
\begin{itemize}
\item {Utilização:Prov.}
\end{itemize}
\begin{itemize}
\item {Utilização:Fam.}
\end{itemize}
\begin{itemize}
\item {Utilização:Prov.}
\end{itemize}
\begin{itemize}
\item {Utilização:Açor}
\end{itemize}
\begin{itemize}
\item {Utilização:pop.}
\end{itemize}
\begin{itemize}
\item {Utilização:Ext.}
\end{itemize}
\begin{itemize}
\item {Utilização:Prov.}
\end{itemize}
\begin{itemize}
\item {Utilização:minh.}
\end{itemize}
\begin{itemize}
\item {Utilização:Fam.}
\end{itemize}
\begin{itemize}
\item {Utilização:Fam.}
\end{itemize}
\begin{itemize}
\item {Utilização:Fig.}
\end{itemize}
\begin{itemize}
\item {Grp. gram.:Pl.}
\end{itemize}
\begin{itemize}
\item {Utilização:Bras}
\end{itemize}
Saliência córnea, que constitue as membranas que cobrem os ossos maxillares das aves.
Bôca de alguns peixes.
Renda estreita, que termina lateralmente em bicos.
Variedade de bolo, semelhante ás tortas de Coímbra.
Princípio de bebedeira, embriaguez incompleta.
Bebedor de vinho.
O mesmo que \textunderscore tostão\textunderscore ^1.
Ave doméstica: \textunderscore sustentar trinta bicos\textunderscore .
A bôca humana: \textunderscore cale o bico\textunderscore !
O mesmo que \textunderscore beijo\textunderscore :«\textunderscore dás-me um bico\textunderscore ?--\textunderscore Não, que me pico\textunderscore ». (De uma canção popular)
Pessôa:«\textunderscore para o banquete, o preço é de 2$500 cada bico\textunderscore ». \textunderscore Luta\textunderscore , de 24-VIII-907.
Ponta, extremidade.
O mesmo que \textunderscore aparo\textunderscore  (de escrever).
Embaraço, difficuldade: \textunderscore êsse negócio tem bico\textunderscore .
\textunderscore Bico de obra\textunderscore , (a mesma sign.).
\textunderscore Bico do peito\textunderscore , o mamilo.
\textunderscore Pau de dois bicos\textunderscore , argumento, acto ou coisa, com que se podem satisfazer duas opiniões differentes.
\textunderscore Melro de bico amarelo\textunderscore , sujeito finório, astucioso.
\textunderscore Bico de grou\textunderscore , espécie de erva.
\textunderscore Água no bico\textunderscore , intenção reservada.
Restos de alguma coisa.
Dinheiro miúdo, quantia insignificante.
\section{Bico}
\begin{itemize}
\item {Grp. gram.:m.}
\end{itemize}
Antiguo dignitário chinês. Cf. \textunderscore Peregrinação\textunderscore , CLX.
\section{Bicó}
\begin{itemize}
\item {Grp. gram.:adj.}
\end{itemize}
\begin{itemize}
\item {Utilização:Bras. do N}
\end{itemize}
Que não tem rabo.
\section{Bicocco}
\begin{itemize}
\item {Grp. gram.:adj.}
\end{itemize}
\begin{itemize}
\item {Utilização:Bot.}
\end{itemize}
\begin{itemize}
\item {Proveniência:(Do lat. \textunderscore bis\textunderscore  + \textunderscore coccum\textunderscore )}
\end{itemize}
Diz-se do fruto que tem duas cóccas.
\section{Bicoco}
\begin{itemize}
\item {Grp. gram.:adj.}
\end{itemize}
\begin{itemize}
\item {Utilização:Bot.}
\end{itemize}
\begin{itemize}
\item {Proveniência:(Do lat. \textunderscore bis\textunderscore  + \textunderscore coccum\textunderscore )}
\end{itemize}
Diz-se do fruto que tem duas cóccas.
\section{Bico-cruzado}
\begin{itemize}
\item {Grp. gram.:m.}
\end{itemize}
Formoso pássaro que, como o papagaio, sobe, auxiliando-se com o bico, e cujas mandíbulas, ao fechar da bôca, ficam cruzadas.
\section{Bico-de-cegonha}
\begin{itemize}
\item {Grp. gram.:m.}
\end{itemize}
Planta medicinal, espécie de gerânio.
\section{Bico-de-corvo}
\begin{itemize}
\item {Grp. gram.:m.}
\end{itemize}
Azeitona, também chamada \textunderscore cordovesa\textunderscore .
\section{Bico-de-lacre}
\begin{itemize}
\item {Grp. gram.:m.}
\end{itemize}
Passarinho cinzento, de bico vermelho, e originário da África.
\section{Bico-de-mocho}
\begin{itemize}
\item {Grp. gram.:m.}
\end{itemize}
Pequeno filete, que fórma a borda de uma cornija.
(Cp. \textunderscore mocheta\textunderscore )
\section{Bico-de-prata}
\begin{itemize}
\item {Grp. gram.:m.}
\end{itemize}
O mesmo que \textunderscore jacapa\textunderscore .
\section{Bico-gordo}
\begin{itemize}
\item {Grp. gram.:m.}
\end{itemize}
O mesmo que \textunderscore bico-grossudo\textunderscore .
\section{Bico-grossudo}
\begin{itemize}
\item {Grp. gram.:m.}
\end{itemize}
Espécie de pardal, conhecido também por \textunderscore pardal do norte\textunderscore  e \textunderscore chincalhão do norte\textunderscore .
\section{Bicolor}
\begin{itemize}
\item {Grp. gram.:adj.}
\end{itemize}
\begin{itemize}
\item {Proveniência:(Lat. \textunderscore bicolor\textunderscore )}
\end{itemize}
Que tem duas côres.
\section{Bicôncavo}
\begin{itemize}
\item {Grp. gram.:adj.}
\end{itemize}
\begin{itemize}
\item {Proveniência:(De \textunderscore bi...\textunderscore  + \textunderscore côncavo\textunderscore )}
\end{itemize}
Que é côncavo dos dois lados.
\section{Bicónico}
\begin{itemize}
\item {Grp. gram.:adj.}
\end{itemize}
\begin{itemize}
\item {Proveniência:(De \textunderscore bi...\textunderscore  + \textunderscore cónico\textunderscore )}
\end{itemize}
Que tem dois cónes oppostos.
\section{Biconjugado}
\begin{itemize}
\item {Grp. gram.:adj.}
\end{itemize}
\begin{itemize}
\item {Proveniência:(De \textunderscore bi...\textunderscore  + \textunderscore conjugado\textunderscore )}
\end{itemize}
Que se divide em dois ramos.
\section{Biconvexo}
\begin{itemize}
\item {Grp. gram.:adj.}
\end{itemize}
\begin{itemize}
\item {Proveniência:(De \textunderscore bi...\textunderscore  + \textunderscore convexo\textunderscore )}
\end{itemize}
Que é convexo dos dois lados.
\section{Bico-rasteiro}
\begin{itemize}
\item {Grp. gram.:m.}
\end{itemize}
Ave do Brasil.
\section{Bicorne}
\begin{itemize}
\item {Grp. gram.:adj.}
\end{itemize}
\begin{itemize}
\item {Proveniência:(Lat. \textunderscore bicornis\textunderscore )}
\end{itemize}
Que tem dois cornos ou duas pontas.
\section{Bicornela}
\begin{itemize}
\item {Grp. gram.:f.}
\end{itemize}
\begin{itemize}
\item {Proveniência:(De \textunderscore bicorne\textunderscore )}
\end{itemize}
Planta de Madagáscar, da fam. das orchídeas.
\section{Bicórneo}
\begin{itemize}
\item {Grp. gram.:adj.}
\end{itemize}
O mesmo que \textunderscore bicorne\textunderscore .
\section{Bicornígero}
\begin{itemize}
\item {Grp. gram.:adj.}
\end{itemize}
\begin{itemize}
\item {Proveniência:(Do lat. \textunderscore bis\textunderscore  + \textunderscore cornu\textunderscore  + \textunderscore gerere\textunderscore )}
\end{itemize}
O mesmo que \textunderscore bicorne\textunderscore .
\section{Bicuda}
\begin{itemize}
\item {Grp. gram.:f.}
\end{itemize}
\begin{itemize}
\item {Utilização:Bras}
\end{itemize}
\begin{itemize}
\item {Utilização:Gír.}
\end{itemize}
Peixe do Brasil e dos Açores, de bico comprido e agudo.
O mesmo que \textunderscore gallinhola\textunderscore .
Azeitona, o mesmo que \textunderscore bical\textunderscore  ou \textunderscore cornalhuda\textunderscore .
Faca de ponta; punhal.
(Fem. de \textunderscore bicudo\textunderscore )
\section{Bicicleta}
\begin{itemize}
\item {fónica:clê}
\end{itemize}
\begin{itemize}
\item {Grp. gram.:f.}
\end{itemize}
\begin{itemize}
\item {Proveniência:(Fr. \textunderscore bicyclette\textunderscore )}
\end{itemize}
Conhecido e usado velocípede de duas rodas, iguaes e pequenas.
\section{Bicicletista}
\begin{itemize}
\item {Grp. gram.:m.  e  f.}
\end{itemize}
Pessôa, que anda em \textunderscore bicicleta\textunderscore .
\section{Biciclista}
\begin{itemize}
\item {Grp. gram.:m.  e  f.}
\end{itemize}
Pessôa, que anda em \textunderscore biciclo\textunderscore .
\section{Biciclizar}
\begin{itemize}
\item {Grp. gram.:v. i.}
\end{itemize}
\begin{itemize}
\item {Utilização:Neol.}
\end{itemize}
Andar em biciclo ou em velocipede.
\section{Biciclo}
\begin{itemize}
\item {Grp. gram.:m.}
\end{itemize}
\begin{itemize}
\item {Proveniência:(De \textunderscore bi...\textunderscore  + \textunderscore cyclo\textunderscore )}
\end{itemize}
Espécie desusada de velocipede de duas rodas.
\section{Bicudez}
\begin{itemize}
\item {Grp. gram.:f.}
\end{itemize}
\begin{itemize}
\item {Utilização:Neol.}
\end{itemize}
Qualidade daquillo que é bicudo ou diffícil: \textunderscore a bicudez dos tempos\textunderscore .
\section{Bicudo}
\begin{itemize}
\item {Grp. gram.:adj.}
\end{itemize}
\begin{itemize}
\item {Utilização:Fam.}
\end{itemize}
\begin{itemize}
\item {Utilização:Bras}
\end{itemize}
\begin{itemize}
\item {Utilização:Fam.}
\end{itemize}
\begin{itemize}
\item {Grp. gram.:M.}
\end{itemize}
\begin{itemize}
\item {Utilização:Gír.}
\end{itemize}
\begin{itemize}
\item {Utilização:Bras. de Minas}
\end{itemize}
Que tem bico.
Aguçado, ponteagudo: \textunderscore nariz bicudo\textunderscore .
Complicado, diffícil: \textunderscore questão bicuda\textunderscore .
Amuado, zangado, diffícil de aturar.
Que bebeu de mais e está em princípios de bebedeira.
Ave brasileira de bico grosso.
Peixe de Portugal.
Alfinete de peito.
Indivíduo, que se embriaga.
\section{Bicuíba}
\begin{itemize}
\item {Grp. gram.:f.}
\end{itemize}
Árvore myristicácea do Brasil.
Fruto da mesma árvore.
Óleo, extrahido dêsse fruto.
(Do tupi)
\section{Bicuibeira}
\begin{itemize}
\item {fónica:cu-i}
\end{itemize}
\begin{itemize}
\item {Grp. gram.:f.}
\end{itemize}
(V. \textunderscore bicuíba\textunderscore , árvore)
\section{Bicuibuçu}
\begin{itemize}
\item {fónica:cu-i}
\end{itemize}
\begin{itemize}
\item {Grp. gram.:m.}
\end{itemize}
Árvore brasileira.
\section{Bicúspide}
\begin{itemize}
\item {Grp. gram.:adj.}
\end{itemize}
\begin{itemize}
\item {Proveniência:(Do lat. \textunderscore bis\textunderscore  + \textunderscore cuspis\textunderscore )}
\end{itemize}
Que tem duas pontas, ou que termina em duas partes divergentes.
\section{Bicycleta}
\begin{itemize}
\item {fónica:clê}
\end{itemize}
\begin{itemize}
\item {Grp. gram.:f.}
\end{itemize}
\begin{itemize}
\item {Proveniência:(Fr. \textunderscore bicyclette\textunderscore )}
\end{itemize}
Conhecido e usado velocípede de duas rodas, iguaes e pequenas.
\section{Bicycletista}
\begin{itemize}
\item {Grp. gram.:m.  e  f.}
\end{itemize}
Pessôa, que anda em \textunderscore bicycleta\textunderscore .
\section{Bicyclista}
\begin{itemize}
\item {Grp. gram.:m.  e  f.}
\end{itemize}
Pessôa, que anda em \textunderscore bicyclo\textunderscore .
\section{Bicyclizar}
\begin{itemize}
\item {Grp. gram.:v. i.}
\end{itemize}
\begin{itemize}
\item {Utilização:Neol.}
\end{itemize}
Andar em bicyclo ou em velocipede.
\section{Bicyclo}
\begin{itemize}
\item {Grp. gram.:m.}
\end{itemize}
\begin{itemize}
\item {Proveniência:(De \textunderscore bi...\textunderscore  + \textunderscore cyclo\textunderscore )}
\end{itemize}
Espécie desusada de velocipede de duas rodas.
\section{Bidé}
\begin{itemize}
\item {Grp. gram.:m.}
\end{itemize}
\begin{itemize}
\item {Utilização:Bras}
\end{itemize}
\begin{itemize}
\item {Proveniência:(Fr. \textunderscore bídet\textunderscore )}
\end{itemize}
Pequeno móvel, em que há uma bacia, para lavagem das partes inferiores do tronco.
Mesa de cabeceira; criado-mudo.
\section{Bidentado}
\begin{itemize}
\item {Grp. gram.:adj.}
\end{itemize}
\begin{itemize}
\item {Proveniência:(De \textunderscore bi...\textunderscore  + \textunderscore dentado\textunderscore )}
\end{itemize}
Que tem dois dentes.
\section{Bidente}
\begin{itemize}
\item {Grp. gram.:m.}
\end{itemize}
\begin{itemize}
\item {Proveniência:(Lat. \textunderscore bidens\textunderscore )}
\end{itemize}
Alvião.
Gadanho com dois dentes.
\section{Bidenteado}
\begin{itemize}
\item {Grp. gram.:adj.}
\end{itemize}
(V.bidentado)
\section{Bidênteo}
\begin{itemize}
\item {Grp. gram.:adj.}
\end{itemize}
\begin{itemize}
\item {Proveniência:(De \textunderscore bi...\textunderscore  + \textunderscore dente\textunderscore )}
\end{itemize}
Que tem dois dentes.
\section{Bidete}
\begin{itemize}
\item {fónica:dê}
\end{itemize}
\begin{itemize}
\item {Grp. gram.:m.}
\end{itemize}
(V.bidé)
\section{Bidigitado}
\begin{itemize}
\item {Grp. gram.:adj.}
\end{itemize}
\begin{itemize}
\item {Proveniência:(Do lat. \textunderscore bis\textunderscore  + \textunderscore digitatus\textunderscore )}
\end{itemize}
Que tem dois dedos ou que se divide em duas digitações.
\section{Bíduo}
\begin{itemize}
\item {Grp. gram.:m.}
\end{itemize}
\begin{itemize}
\item {Utilização:Des.}
\end{itemize}
\begin{itemize}
\item {Proveniência:(Lat. \textunderscore biduum\textunderscore )}
\end{itemize}
O espaço de dois dias.
\section{Biela}
\begin{itemize}
\item {Grp. gram.:f.}
\end{itemize}
\begin{itemize}
\item {Utilização:Mecan.}
\end{itemize}
\begin{itemize}
\item {Proveniência:(Fr. \textunderscore bielle\textunderscore )}
\end{itemize}
Haste rígida, que serve para communicar movimento, entre duas peças afastadas, em certos maquinismos.
\section{Biella}
\begin{itemize}
\item {Grp. gram.:f.}
\end{itemize}
\begin{itemize}
\item {Utilização:Mecan.}
\end{itemize}
\begin{itemize}
\item {Proveniência:(Fr. \textunderscore bielle\textunderscore )}
\end{itemize}
Haste rígida, que serve para communicar movimento, entre duas peças afastadas, em certos maquinismos.
\section{Bienal}
\begin{itemize}
\item {Grp. gram.:adj.}
\end{itemize}
\begin{itemize}
\item {Proveniência:(Lat. \textunderscore biennalis\textunderscore )}
\end{itemize}
Relativo ao espaço de dois annos.
Que dura dois annos.
\section{Biênio}
\begin{itemize}
\item {Grp. gram.:m.}
\end{itemize}
\begin{itemize}
\item {Proveniência:(Lat. \textunderscore biennium\textunderscore )}
\end{itemize}
O espaço de dois annos successivos.
\section{Biennal}
\begin{itemize}
\item {Grp. gram.:adj.}
\end{itemize}
\begin{itemize}
\item {Proveniência:(Lat. \textunderscore biennalis\textunderscore )}
\end{itemize}
Relativo ao espaço de dois annos.
Que dura dois annos.
\section{Biênnio}
\begin{itemize}
\item {Grp. gram.:m.}
\end{itemize}
\begin{itemize}
\item {Proveniência:(Lat. \textunderscore biennium\textunderscore )}
\end{itemize}
O espaço de dois annos successivos.
\section{Bieno}
\begin{itemize}
\item {Grp. gram.:adj.}
\end{itemize}
\begin{itemize}
\item {Grp. gram.:M.}
\end{itemize}
Relativo ao Bihé.
Habitante do Bihé, na África.
\section{Bifa}
\begin{itemize}
\item {Grp. gram.:f.}
\end{itemize}
\begin{itemize}
\item {Utilização:Ant.}
\end{itemize}
\begin{itemize}
\item {Proveniência:(Do b. lat. \textunderscore bifax\textunderscore )}
\end{itemize}
Pano ou fazenda, a que Viterbo attribue \textunderscore duas caras\textunderscore . Cf. Herculano, \textunderscore Lendas e Narr.\textunderscore , I, 96.
\section{Bifa}
\begin{itemize}
\item {Grp. gram.:f.}
\end{itemize}
\begin{itemize}
\item {Utilização:Prov.}
\end{itemize}
\begin{itemize}
\item {Utilização:trasm.}
\end{itemize}
O mesmo que \textunderscore bélfa\textunderscore .
\section{Bifada}
\begin{itemize}
\item {Grp. gram.:f.}
\end{itemize}
\begin{itemize}
\item {Utilização:Fam.}
\end{itemize}
Porção de bifes.
\section{Bifada}
\begin{itemize}
\item {Grp. gram.:f.}
\end{itemize}
\begin{itemize}
\item {Utilização:Bras}
\end{itemize}
Fartum; mau hálito.
(Corr. de \textunderscore bafada\textunderscore , de \textunderscore bafo\textunderscore ?)
\section{Bifalhada}
\begin{itemize}
\item {Grp. gram.:f.}
\end{itemize}
\begin{itemize}
\item {Utilização:Fam.}
\end{itemize}
O mesmo que \textunderscore bifada\textunderscore ^1.
\section{Bifar}
\begin{itemize}
\item {Grp. gram.:v. t.}
\end{itemize}
\begin{itemize}
\item {Utilização:Fam.}
\end{itemize}
Furtar, tirar disfarçadamente.
(Cp. fr. \textunderscore biffer\textunderscore )
\section{Bifário}
\begin{itemize}
\item {Grp. gram.:adj.}
\end{itemize}
\begin{itemize}
\item {Proveniência:(Lat. \textunderscore bifarius\textunderscore )}
\end{itemize}
Desdobrado em duas partes.
\section{Bife}
\begin{itemize}
\item {Grp. gram.:m.}
\end{itemize}
\begin{itemize}
\item {Utilização:Gír.}
\end{itemize}
\begin{itemize}
\item {Utilização:Deprec.}
\end{itemize}
\begin{itemize}
\item {Grp. gram.:Adj.}
\end{itemize}
\begin{itemize}
\item {Utilização:Pop.}
\end{itemize}
\begin{itemize}
\item {Proveniência:(Do ingl. \textunderscore beef\textunderscore )}
\end{itemize}
Fatia de carne, batida ou picada, e assada em grelha, ou frita.
\textunderscore Bife sombrio\textunderscore , isca de fígado.
Qualquer indivíduo inglês.
Inglesado. Cf. Eça, \textunderscore P. Basilio\textunderscore , 17.
Indivíduo inglês. Cf. Macedo, \textunderscore Burros\textunderscore , (\textunderscore passim\textunderscore ).
\section{Bifeira}
\begin{itemize}
\item {Grp. gram.:f.}
\end{itemize}
Utensílio culinário, de fôlha de ferro, para fazer bifes, com presteza.
\section{Bifendido}
\begin{itemize}
\item {Grp. gram.:adj.}
\end{itemize}
\begin{itemize}
\item {Proveniência:(De \textunderscore bi...\textunderscore  + \textunderscore fendido\textunderscore )}
\end{itemize}
Dividido em duas partes por uma fenda.
Aberto ao meio.
\section{Bífero}
\begin{itemize}
\item {Grp. gram.:adj.}
\end{itemize}
\begin{itemize}
\item {Proveniência:(Lat. \textunderscore bifer\textunderscore )}
\end{itemize}
Que dá fruto duas vezes no anno.
\section{Bifesteque}
\begin{itemize}
\item {Grp. gram.:m.}
\end{itemize}
\begin{itemize}
\item {Proveniência:(Ingl. \textunderscore beefsteak\textunderscore )}
\end{itemize}
Posta de vaca, mal assada, com môlho da mesma carne.
\section{Bífido}
\begin{itemize}
\item {Grp. gram.:adj.}
\end{itemize}
\begin{itemize}
\item {Proveniência:(Lat. \textunderscore bifidus\textunderscore )}
\end{itemize}
O mesmo que \textunderscore bifendido\textunderscore .
\section{Biflexo}
\begin{itemize}
\item {Grp. gram.:adj.}
\end{itemize}
\begin{itemize}
\item {Proveniência:(Do lat. \textunderscore bis\textunderscore  + \textunderscore flexus\textunderscore )}
\end{itemize}
Dobrado para dois lados.
\section{Biflôr}
\begin{itemize}
\item {Grp. gram.:adj.}
\end{itemize}
O mesmo que \textunderscore bifloro\textunderscore .
\section{Bifloro}
\begin{itemize}
\item {Grp. gram.:adj.}
\end{itemize}
\begin{itemize}
\item {Proveniência:(Do lat. \textunderscore bis\textunderscore  + \textunderscore flos\textunderscore )}
\end{itemize}
Que tem duas flôres ou grupos de duas flôres.
\section{Bifólia}
\begin{itemize}
\item {Grp. gram.:f.  e  adj.}
\end{itemize}
\begin{itemize}
\item {Proveniência:(Do lat. \textunderscore bis\textunderscore  + \textunderscore folium\textunderscore )}
\end{itemize}
Diz-se das charruas, que têm duas aivecas.
\section{Bifoliado}
\begin{itemize}
\item {Grp. gram.:adj.}
\end{itemize}
\begin{itemize}
\item {Proveniência:(Do lat. \textunderscore bis\textunderscore  + \textunderscore folium\textunderscore )}
\end{itemize}
Que tem duas fôlhas.
\section{Bifólio}
\begin{itemize}
\item {Grp. gram.:adj.}
\end{itemize}
O mesmo que \textunderscore bifoliado\textunderscore .
\section{Bífore}
\begin{itemize}
\item {Grp. gram.:adj.}
\end{itemize}
\begin{itemize}
\item {Proveniência:(Lat. \textunderscore biforis\textunderscore )}
\end{itemize}
Diz-se do portal que tem duas portas ou dois batentes.
\section{Biforme}
\begin{itemize}
\item {Grp. gram.:adj.}
\end{itemize}
\begin{itemize}
\item {Utilização:Gram.}
\end{itemize}
\begin{itemize}
\item {Proveniência:(Lat. \textunderscore biformis\textunderscore )}
\end{itemize}
Que tem duas fórmas.
Que tem duas qualidades de flôres, (falando-se de plantas).
Que pensa de duas maneiras differentes.
Diz-se do adj. port. que tem uma fórma para cada gênero.
\section{Bifronte}
\begin{itemize}
\item {Grp. gram.:adj.}
\end{itemize}
\begin{itemize}
\item {Proveniência:(Lat. \textunderscore bifrons\textunderscore )}
\end{itemize}
Que tem duas caras.
Volúvel, traiçoeiro.
\section{Bifu}
\begin{itemize}
\item {Grp. gram.:m.}
\end{itemize}
\begin{itemize}
\item {Utilização:Fam.}
\end{itemize}
Pancada que, dobrando-se a perna pelo joêlho e movendo-se com fôrça para fóra, se dá de banda com o pé nas nádegas de outrem.
O mesmo que \textunderscore chulipa\textunderscore ^2.
\section{Bifurcação}
\begin{itemize}
\item {Grp. gram.:f.}
\end{itemize}
Acção de \textunderscore bifurcar\textunderscore .
\section{Bifurcar}
\begin{itemize}
\item {Grp. gram.:v. t.}
\end{itemize}
\begin{itemize}
\item {Grp. gram.:V. p.}
\end{itemize}
\begin{itemize}
\item {Proveniência:(Do lat. \textunderscore bis\textunderscore  + \textunderscore furca\textunderscore )}
\end{itemize}
Separar em dois ramos.
Dividir-se em dois ramos.
Montar: \textunderscore bifurcar-se numa égua\textunderscore .
\section{Biga}
\begin{itemize}
\item {Grp. gram.:f.}
\end{itemize}
\begin{itemize}
\item {Proveniência:(Lat. \textunderscore biga\textunderscore )}
\end{itemize}
Carro romano, puxado por dois cavallos.
\section{Bigamia}
\begin{itemize}
\item {Grp. gram.:f.}
\end{itemize}
\begin{itemize}
\item {Proveniência:(Lat. \textunderscore bigamia\textunderscore )}
\end{itemize}
Estado de bígamo.
\section{Bígamo}
\begin{itemize}
\item {Grp. gram.:m.  e  adj.}
\end{itemize}
\begin{itemize}
\item {Proveniência:(Lat. \textunderscore bigamus\textunderscore )}
\end{itemize}
O que tem dois cônjuges ao mesmo tempo.
\section{Biganau}
\begin{itemize}
\item {Grp. gram.:m.}
\end{itemize}
\begin{itemize}
\item {Utilização:Prov.}
\end{itemize}
\begin{itemize}
\item {Utilização:trasm.}
\end{itemize}
Indivíduo forte, agigantado.
\section{Bigarim}
\begin{itemize}
\item {Grp. gram.:m.}
\end{itemize}
\begin{itemize}
\item {Proveniência:(Do conc. \textunderscore bigari\textunderscore )}
\end{itemize}
Homem de trabalho braçal, na antiga Índia portuguesa.
\section{Bigêmeo}
\begin{itemize}
\item {Grp. gram.:adj.}
\end{itemize}
\begin{itemize}
\item {Utilização:Bot.}
\end{itemize}
\begin{itemize}
\item {Proveniência:(Do lat. \textunderscore bis\textunderscore  + \textunderscore geminus\textunderscore )}
\end{itemize}
Diz-se da fôlha ou da flôr, que cresce com outra em pedúnculo ou pecíolo commum.
\section{Bigeminado}
\begin{itemize}
\item {Grp. gram.:adj.}
\end{itemize}
O mesmo que \textunderscore bigêmeo\textunderscore .
\section{Bigemíneo}
\begin{itemize}
\item {Grp. gram.:adj.}
\end{itemize}
O mesmo que \textunderscore bigêmeo\textunderscore .
\section{Bigêmino}
\begin{itemize}
\item {Grp. gram.:adj.}
\end{itemize}
O mesmo que \textunderscore bigénero\textunderscore .
\section{Bigénero}
\begin{itemize}
\item {Grp. gram.:adj.}
\end{itemize}
\begin{itemize}
\item {Proveniência:(Do lat. \textunderscore bigener\textunderscore )}
\end{itemize}
Que pertence a dois gêneros.
\section{Bigênito}
\begin{itemize}
\item {Grp. gram.:adj.}
\end{itemize}
\begin{itemize}
\item {Proveniência:(De \textunderscore bi...\textunderscore  + \textunderscore génito\textunderscore )}
\end{itemize}
Gerado duas vezes, (falando-se de Baccho).
\section{Biglanduloso}
\begin{itemize}
\item {Grp. gram.:adj.}
\end{itemize}
\begin{itemize}
\item {Proveniência:(De \textunderscore bi...\textunderscore  + \textunderscore glanduloso\textunderscore )}
\end{itemize}
Que tem duas glândulas.
\section{Bigle}
\begin{itemize}
\item {Grp. gram.:m.}
\end{itemize}
\begin{itemize}
\item {Proveniência:(Ingl. \textunderscore beagle\textunderscore )}
\end{itemize}
Pequeno galgo.
\section{Bignónia}
\begin{itemize}
\item {Grp. gram.:f.}
\end{itemize}
\begin{itemize}
\item {Proveniência:(De \textunderscore Bignon\textunderscore , n. p.)}
\end{itemize}
Planta dicotyledónea das regiões equinociáes.
\section{Bignoniáceas}
\begin{itemize}
\item {Grp. gram.:f. pl.}
\end{itemize}
Família de plantas, que têm por typo a bignónia.
\section{Bigode}
\begin{itemize}
\item {Grp. gram.:m.}
\end{itemize}
\begin{itemize}
\item {Utilização:Fam.}
\end{itemize}
Parte da barba, que cresce por cima do lábio superior.
Um jôgo de cartas.
Quinau; descompostura.
\textunderscore Dar um bigode\textunderscore , matar a caça que outrem errou.
(Cp. cast. \textunderscore bigote\textunderscore )
\section{Bigode}
\begin{itemize}
\item {Grp. gram.:m.}
\end{itemize}
Espécie de canário de Angola.
\section{Bigodear}
\begin{itemize}
\item {Grp. gram.:v. t.}
\end{itemize}
\begin{itemize}
\item {Proveniência:(De \textunderscore bigode\textunderscore ^1)}
\end{itemize}
Escarnecer.
Enganar.
\section{Bigodeira}
\begin{itemize}
\item {Grp. gram.:f.}
\end{itemize}
\begin{itemize}
\item {Utilização:Ant.}
\end{itemize}
\begin{itemize}
\item {Utilização:Pop.}
\end{itemize}
\begin{itemize}
\item {Proveniência:(De \textunderscore bigode\textunderscore )}
\end{itemize}
Bôlsa, em que se metiam as barbas, para se não desconcertarem.
Escôva de limpar bêstas.
Bigode farto.
\section{Bigodelha}
\begin{itemize}
\item {Grp. gram.:f.}
\end{itemize}
\begin{itemize}
\item {Utilização:Prov.}
\end{itemize}
\begin{itemize}
\item {Utilização:alg.}
\end{itemize}
O mesmo que \textunderscore bigodeira\textunderscore , (grande bigode).
\section{Bigorna}
\begin{itemize}
\item {Grp. gram.:f.}
\end{itemize}
\begin{itemize}
\item {Proveniência:(Do lat. \textunderscore bicornis\textunderscore )}
\end{itemize}
Utensílio de ferro, com duas pontas, sôbre o qual se batem metaes.
Íncude.
Pequeno osso do ouvido.
\section{Bigorne}
\begin{itemize}
\item {Grp. gram.:adj.}
\end{itemize}
O mesmo que \textunderscore bicorne\textunderscore .
\section{Bigorrilha}
\begin{itemize}
\item {Grp. gram.:m.}
\end{itemize}
Homem vil, desprezível.
\section{Bigorrilhas}
\begin{itemize}
\item {Grp. gram.:m.}
\end{itemize}
Homem vil, desprezível.
\section{Bigota}
\begin{itemize}
\item {Grp. gram.:f.}
\end{itemize}
\begin{itemize}
\item {Utilização:Náut.}
\end{itemize}
\begin{itemize}
\item {Proveniência:(De \textunderscore biga\textunderscore , por \textunderscore viga\textunderscore ?)}
\end{itemize}
Moitão sem roldana, com um furo, por onde passa o colhedor da vela.
\section{Bigotão}
\begin{itemize}
\item {Grp. gram.:m.}
\end{itemize}
\begin{itemize}
\item {Utilização:Prov.}
\end{itemize}
\begin{itemize}
\item {Utilização:alg.}
\end{itemize}
O mesmo que \textunderscore bigodelha\textunderscore .
\section{Bigote}
\begin{itemize}
\item {Grp. gram.:m.}
\end{itemize}
\begin{itemize}
\item {Utilização:Prov.}
\end{itemize}
O mesmo que \textunderscore bigode\textunderscore ^1.
\section{Bigotismo}
\begin{itemize}
\item {Grp. gram.:m.}
\end{itemize}
Velhacaria. Cf. Camillo, \textunderscore Pombal\textunderscore , 95.
\section{Biguá}
\begin{itemize}
\item {Grp. gram.:f.}
\end{itemize}
\begin{itemize}
\item {Proveniência:(T. Tupi)}
\end{itemize}
Ave palmípede do Brasil.
\section{Biguane}
\begin{itemize}
\item {Grp. gram.:adj.}
\end{itemize}
\begin{itemize}
\item {Utilização:Bras. do N}
\end{itemize}
Muito grande, desmedido.
\section{Biguarim}
\begin{itemize}
\item {Grp. gram.:m.}
\end{itemize}
\begin{itemize}
\item {Utilização:Ant.}
\end{itemize}
O mesmo que \textunderscore bigorrilha\textunderscore .
\section{Bigúmeo}
\begin{itemize}
\item {Grp. gram.:adj.}
\end{itemize}
\begin{itemize}
\item {Proveniência:(De \textunderscore bi...\textunderscore  + \textunderscore gume\textunderscore )}
\end{itemize}
Que tem dois gumes.
\section{Biheno}
\begin{itemize}
\item {Grp. gram.:adj.}
\end{itemize}
\begin{itemize}
\item {Grp. gram.:M.}
\end{itemize}
Relativo ao Bihé.
Habitante do Bihé, na África.
\section{Bijagó}
\begin{itemize}
\item {Grp. gram.:m.}
\end{itemize}
\begin{itemize}
\item {Grp. gram.:Pl.}
\end{itemize}
Idioma africano do archipélago de Bijagós.
Indígenas do archipélago do mesmo nome.
\section{Biju}
\begin{itemize}
\item {Grp. gram.:m.}
\end{itemize}
\begin{itemize}
\item {Utilização:Bras. do Rio}
\end{itemize}
O mesmo que \textunderscore beiju\textunderscore .
\section{Bijugado}
\begin{itemize}
\item {Grp. gram.:adj.}
\end{itemize}
\begin{itemize}
\item {Utilização:Bot.}
\end{itemize}
\begin{itemize}
\item {Proveniência:(Do lat. \textunderscore bis\textunderscore  + \textunderscore jugum\textunderscore )}
\end{itemize}
Diz-se das fôlhas, que têm dois pares de folíolos em pecíolo commum.
\section{Bíjugo}
\begin{itemize}
\item {Grp. gram.:adj.}
\end{itemize}
\begin{itemize}
\item {Proveniência:(Lat. \textunderscore bijugus\textunderscore )}
\end{itemize}
Que é puxado por dois cavallos.
\section{Bilabiado}
\begin{itemize}
\item {Grp. gram.:adj.}
\end{itemize}
\begin{itemize}
\item {Proveniência:(De \textunderscore bi...\textunderscore  + \textunderscore lábio\textunderscore )}
\end{itemize}
Que tem dois lábios.
\section{Bilabial}
\begin{itemize}
\item {Grp. gram.:adj.}
\end{itemize}
\begin{itemize}
\item {Utilização:Gram.}
\end{itemize}
\begin{itemize}
\item {Proveniência:(De \textunderscore bi...\textunderscore  + \textunderscore labial\textunderscore )}
\end{itemize}
Diz-se da consoante, que se pronuncía com o lábio superior e inferior.
\section{Bilaminado}
\begin{itemize}
\item {Grp. gram.:adj.}
\end{itemize}
\begin{itemize}
\item {Proveniência:(De \textunderscore bi...\textunderscore  + \textunderscore laminado\textunderscore )}
\end{itemize}
Que tem duas lâminas.
\section{Bilaminoso}
\begin{itemize}
\item {Grp. gram.:adj.}
\end{itemize}
O mesmo que \textunderscore bilaminado\textunderscore .
\section{Bilare}
\begin{itemize}
\item {Grp. gram.:m.}
\end{itemize}
Variedade de arroz indiano.
\section{Bilaterado}
\begin{itemize}
\item {Grp. gram.:adj.}
\end{itemize}
\begin{itemize}
\item {Utilização:Bot.}
\end{itemize}
\begin{itemize}
\item {Proveniência:(Do lat. \textunderscore bis\textunderscore  + \textunderscore latus\textunderscore , \textunderscore lateris\textunderscore )}
\end{itemize}
Diz-se das fôlhas, collocadas em lados oppostos.
\section{Bilateral}
\begin{itemize}
\item {Grp. gram.:adj.}
\end{itemize}
\begin{itemize}
\item {Utilização:Jur.}
\end{itemize}
\begin{itemize}
\item {Proveniência:(De \textunderscore bi...\textunderscore  + \textunderscore lateral\textunderscore )}
\end{itemize}
Que tem dois lados.
Que se refere a lados oppostos.
Diz-se dos contratos, em que as partes tomam sôbre si obrigações recíprocas.
\section{Bilbérgia}
\begin{itemize}
\item {Grp. gram.:f.}
\end{itemize}
\begin{itemize}
\item {Proveniência:(De \textunderscore Bilberg\textunderscore , n. p. de um bot. sueco)}
\end{itemize}
Gênero de plantas bromeliáceas.
\section{Bilbode}
\begin{itemize}
\item {Grp. gram.:m.}
\end{itemize}
\begin{itemize}
\item {Proveniência:(Fr. \textunderscore billebaude\textunderscore )}
\end{itemize}
\textunderscore Fogo de bilbode\textunderscore , o disparar de muitas espingardas, umas após outras, sem intervallo sensível.
\section{Bilboqué}
\begin{itemize}
\item {Grp. gram.:m.}
\end{itemize}
\begin{itemize}
\item {Proveniência:(Fr. \textunderscore bilboquet\textunderscore )}
\end{itemize}
Utensílio de doirador, que consta de um pedaço de madeira, forrado de pano, e serve para levantar os pedaços cortados de fôlha de oiro.
\section{Bile}
\begin{itemize}
\item {Grp. gram.:f.}
\end{itemize}
\begin{itemize}
\item {Proveniência:(Lat. \textunderscore bilis\textunderscore )}
\end{itemize}
O mesmo que \textunderscore bílis\textunderscore . Cf. Castilho, \textunderscore Misanthropo\textunderscore , 69.
Líquido amargo e esverdeado, que o fígado segrega.
Mau humor.
Irascibilidade.
Hypocondria.
\section{Bilha}
\begin{itemize}
\item {Grp. gram.:f.}
\end{itemize}
\begin{itemize}
\item {Utilização:Açor}
\end{itemize}
Vaso bojudo e de gargalo estreito, ordinariamente de barro.
O mesmo que \textunderscore botija\textunderscore .
(Do germ.)
\section{Bilhafre}
\begin{itemize}
\item {Grp. gram.:m.}
\end{itemize}
O mesmo que \textunderscore milhafre\textunderscore .
\section{Bilhão}
\begin{itemize}
\item {Grp. gram.:m.}
\end{itemize}
\begin{itemize}
\item {Utilização:Ant.}
\end{itemize}
Qualquer moéda inferior.
(Cast. \textunderscore vellon\textunderscore )
\section{Bilhão}
\begin{itemize}
\item {Grp. gram.:m.}
\end{itemize}
\begin{itemize}
\item {Proveniência:(Fr. \textunderscore billion\textunderscore . Cp. \textunderscore milhã\textunderscore )}
\end{itemize}
O mesmo ou melhor que \textunderscore billião\textunderscore .
Mil milhões, (1.000.000.000) segundo o systema francês; ou um milhão de milhões ou um conto de contos, (1.000.000.000.000 ), segundo o systema inglês.
\section{Bilhar}
\begin{itemize}
\item {Grp. gram.:m.}
\end{itemize}
Jôgo de bólas de marfim, que são impellidas por um taco sôbre mesa forrada de vêrde.
A mesa, onde se joga o bilhar.
Casa, onde se joga o bilhar.
(Fr. \textunderscore billard\textunderscore ).
\section{Bilharda}
\begin{itemize}
\item {Grp. gram.:f.}
\end{itemize}
Jôgo de rapazes, que consiste em fazer saltar com um pau comprido outro mais pequeno, procurando-se que êste não caia dentro de um círculo que se traçou no chão.
O pau mais pequeno, que entra nesse jôgo.
(Da mesma or. que \textunderscore bilhar\textunderscore )
\section{Bilhardão}
\begin{itemize}
\item {Grp. gram.:m.}
\end{itemize}
O mesmo que \textunderscore bilhardeiro\textunderscore .
\section{Bilhardar}
\begin{itemize}
\item {Grp. gram.:v. i.}
\end{itemize}
\begin{itemize}
\item {Proveniência:(Fr. \textunderscore billarder\textunderscore )}
\end{itemize}
Dar duas vezes na bola com o taco ou tocar duas bolas ao mesmo tempo, no jôgo do bilhar.
\section{Bilhardar}
\begin{itemize}
\item {Grp. gram.:v. i.}
\end{itemize}
\begin{itemize}
\item {Utilização:Pop.}
\end{itemize}
Jogar a bilharda.
Vadiar.
\section{Bilhardeiro}
\begin{itemize}
\item {Grp. gram.:m.}
\end{itemize}
Jogador de bilharda.
Vadio, garoto.
\section{Bilhardona}
\begin{itemize}
\item {Grp. gram.:f.}
\end{itemize}
\begin{itemize}
\item {Proveniência:(De \textunderscore bilhardão\textunderscore )}
\end{itemize}
Mulher vadia.
\section{Bilharista}
\begin{itemize}
\item {Grp. gram.:m.}
\end{itemize}
Jogador de bilhar.
\section{Bilhestres}
\begin{itemize}
\item {Grp. gram.:m. pl.}
\end{itemize}
\begin{itemize}
\item {Utilização:trasm}
\end{itemize}
\begin{itemize}
\item {Utilização:Gír.}
\end{itemize}
Dinheiro.
(Cp. \textunderscore balhastros\textunderscore )
\section{Bilhetada}
\begin{itemize}
\item {Grp. gram.:f.}
\end{itemize}
Grande porção de bilhetes.
\section{Bilhete}
\begin{itemize}
\item {fónica:lhê}
\end{itemize}
\begin{itemize}
\item {Grp. gram.:m.}
\end{itemize}
\begin{itemize}
\item {Proveniência:(Fr. \textunderscore billet\textunderscore )}
\end{itemize}
Carta simples e breve, sem as fórmulas das cartas ordinárias.
Aviso.
Pedaço de cartão, com um nome impresso ou mais de um e, ás vezes, com indicação da profissão e morada respectivas; cartão de visita.
Senha, que autoriza a entrada nos espectáculos.
Documento impresso ou manuscrito, que torna o possuidor interessado numa lotaria ou rifa.
Nota promissória, usada no commércio.
\textunderscore Bilhete postal\textunderscore , cartão sellado, para correspondência postal, sôbre assumptos que não exigem segrêdo.--Os bilhetes postaes illustrados adquirem-se sem sêllo, mas têm de o levar, no curso postal.
\section{Bilheteira}
\begin{itemize}
\item {Grp. gram.:f.}
\end{itemize}
\begin{itemize}
\item {Proveniência:(De \textunderscore bilhete\textunderscore )}
\end{itemize}
Espécie de prato ou salva, em que se guardam bilhetes de visita.
Carteira.
Compartimento ou lugar, onde se vendem bilhetes de theatro, de caminhos de ferro, etc.
Pequeno móvel, suspenso na parede e destinado a guardar bilhetes de visita ou outros papéis.
\section{Bilheteiro}
\begin{itemize}
\item {Grp. gram.:m.}
\end{itemize}
Vendedor de bilhetes, que autorizam a entrada em espectáculos, combóios, etc.
\section{Bilhó}
\begin{itemize}
\item {Grp. gram.:f.}
\end{itemize}
\begin{itemize}
\item {Utilização:Prov.}
\end{itemize}
\begin{itemize}
\item {Utilização:trasm.}
\end{itemize}
\begin{itemize}
\item {Utilização:Ant.}
\end{itemize}
\begin{itemize}
\item {Utilização:Prov.}
\end{itemize}
\begin{itemize}
\item {Utilização:trasm.}
\end{itemize}
Castanha assada e descascada.
Noz sem casca.
Criança, não de peito, gorducha e mais baixa que o vulgar naquella idade.
\textunderscore Bilhó da serra\textunderscore , castanha pilada.
(Cp. \textunderscore beilhó\textunderscore )
\section{Bilhoreta}
\begin{itemize}
\item {fónica:lhorê}
\end{itemize}
\begin{itemize}
\item {Grp. gram.:f.}
\end{itemize}
\begin{itemize}
\item {Utilização:Prov.}
\end{itemize}
\begin{itemize}
\item {Utilização:alent.}
\end{itemize}
Tratantada, velhacaria.
\section{Bilhós}
\begin{itemize}
\item {Grp. gram.:f.}
\end{itemize}
O mesmo que \textunderscore beilhós\textunderscore .
\section{Bilhostre}
\begin{itemize}
\item {Grp. gram.:m.}
\end{itemize}
Designação depreciativa do estrangeiro.
Patife, biltre.
\section{Bilhostreira}
\begin{itemize}
\item {Grp. gram.:f.}
\end{itemize}
\begin{itemize}
\item {Utilização:Prov.}
\end{itemize}
\begin{itemize}
\item {Utilização:beir.}
\end{itemize}
\begin{itemize}
\item {Proveniência:(De \textunderscore bilhostre\textunderscore , se não é refl. de \textunderscore bisbilhoteira\textunderscore )}
\end{itemize}
Mulher mexeriqueira.
Mulher desajeitada e immunda.
\section{Biliário}
\begin{itemize}
\item {Grp. gram.:adj.}
\end{itemize}
Relativo á bílis.
\section{Bilifuscina}
\begin{itemize}
\item {Grp. gram.:f.}
\end{itemize}
Uma das matérias còrantes da bílis.
\section{Biligulado}
\begin{itemize}
\item {Grp. gram.:adj.}
\end{itemize}
\begin{itemize}
\item {Proveniência:(De \textunderscore bi...\textunderscore  + \textunderscore ligula\textunderscore )}
\end{itemize}
Dividido em duas lígulas.
\section{Bilimbi}
\begin{itemize}
\item {Grp. gram.:m.}
\end{itemize}
Arbusto, da fam. das oxalídeas, (\textunderscore averrhoa bilimbi\textunderscore , Lin.).
\section{Bilina}
\begin{itemize}
\item {Grp. gram.:f.}
\end{itemize}
Princípio, extrahido da bílis.
\section{Bilingue}
\begin{itemize}
\item {Grp. gram.:adj.}
\end{itemize}
\begin{itemize}
\item {Proveniência:(Lat. \textunderscore bilinguis\textunderscore )}
\end{itemize}
Que tem duas línguas.
Que fala duas línguas ou está escrito em duas línguas.
Que fala com fingimento, com doblez.
\section{Bilioso}
\begin{itemize}
\item {Grp. gram.:adj.}
\end{itemize}
\begin{itemize}
\item {Proveniência:(Lat. \textunderscore biliosus\textunderscore )}
\end{itemize}
Que tem muita bílis.
Relativo a bílis ou causada por ella.
Irascível.
\section{Bílis}
\begin{itemize}
\item {Grp. gram.:f.}
\end{itemize}
\begin{itemize}
\item {Proveniência:(Lat. \textunderscore bilis\textunderscore )}
\end{itemize}
Líquido amargo e esverdeado, que o fígado segrega.
Mau humor.
Irascibilidade.
Hypocondria.
\section{Bilião}
\begin{itemize}
\item {Grp. gram.:m.}
\end{itemize}
\begin{itemize}
\item {Proveniência:(Fr. \textunderscore billion\textunderscore . Cp. \textunderscore milhão\textunderscore ^1)}
\end{itemize}
Mil milhões, (1.000.000.000) segundo o systema francês; ou um milhão de milhões ou um conto de contos, (1.000.000.000.000), segundo o systema inglês.
\section{Bilionário}
\begin{itemize}
\item {Grp. gram.:m.  e  adj.}
\end{itemize}
\begin{itemize}
\item {Utilização:bras}
\end{itemize}
\begin{itemize}
\item {Utilização:Neol.}
\end{itemize}
Duas vezes milionário.
\section{Biliteral}
\begin{itemize}
\item {Grp. gram.:adj.}
\end{itemize}
\begin{itemize}
\item {Proveniência:(Do lat. \textunderscore bis\textunderscore  + \textunderscore litera\textunderscore )}
\end{itemize}
Que tem duas letras.
\section{Biliteralismo}
\begin{itemize}
\item {Grp. gram.:m.}
\end{itemize}
O mesmo que \textunderscore biliterismo\textunderscore .
\section{Biliterismo}
\begin{itemize}
\item {Grp. gram.:m.}
\end{itemize}
\begin{itemize}
\item {Utilização:Philol.}
\end{itemize}
\begin{itemize}
\item {Proveniência:(Do lat. \textunderscore bis\textunderscore  + \textunderscore litera\textunderscore )}
\end{itemize}
Qualidade, que as raízes semíticas teriam, quando formadas de duas letras. Cf. Gregório, \textunderscore Sc. da Linguagem\textunderscore , II, 15.
\section{Bilítero}
\begin{itemize}
\item {Grp. gram.:adj.}
\end{itemize}
O mesmo que \textunderscore biliteral\textunderscore .
\section{Biliverdina}
\begin{itemize}
\item {Grp. gram.:f.}
\end{itemize}
Uma das matérias còrantes da bílis.
\section{Billião}
\begin{itemize}
\item {Grp. gram.:m.}
\end{itemize}
\begin{itemize}
\item {Proveniência:(Fr. \textunderscore billion\textunderscore . Cp. \textunderscore milhão\textunderscore ^1)}
\end{itemize}
Mil milhões, (1.000.000.000) segundo o systema francês; ou um milhão de milhões ou um conto de contos, (1.000.000.000.000), segundo o systema inglês.
\section{Billionário}
\begin{itemize}
\item {Grp. gram.:m.  e  adj.}
\end{itemize}
\begin{itemize}
\item {Utilização:bras}
\end{itemize}
\begin{itemize}
\item {Utilização:Neol.}
\end{itemize}
Duas vezes millionário.
\section{Bilobado}
\begin{itemize}
\item {Grp. gram.:adj.}
\end{itemize}
\begin{itemize}
\item {Proveniência:(De \textunderscore bi...\textunderscore  + \textunderscore lóbo\textunderscore )}
\end{itemize}
Que tem dois lóbulos.
\section{Bilobite}
\begin{itemize}
\item {Grp. gram.:f.}
\end{itemize}
\begin{itemize}
\item {Utilização:Geol.}
\end{itemize}
\begin{itemize}
\item {Proveniência:(De \textunderscore bi...\textunderscore  + \textunderscore lóbo\textunderscore )}
\end{itemize}
Moldagem de meio relêvo com o aspecto de caule deprimido, a qual tem ao meio um sulco, que a divide em dois lóbos parallelos. Cf. G. Guimarães, \textunderscore Geol.\textunderscore , 207.
\section{Bilocação}
\begin{itemize}
\item {Grp. gram.:f.}
\end{itemize}
\begin{itemize}
\item {Utilização:T. eccles}
\end{itemize}
\begin{itemize}
\item {Proveniência:(Do lat. \textunderscore bis\textunderscore  + \textunderscore locare\textunderscore )}
\end{itemize}
Acto de uma pessôa estar, por milagre, em duas localidades ao mesmo tempo.
\section{Bilocular}
\begin{itemize}
\item {Grp. gram.:adj.}
\end{itemize}
\begin{itemize}
\item {Proveniência:(Do lat. \textunderscore bis\textunderscore  + \textunderscore loculus\textunderscore )}
\end{itemize}
Que tem duas cavidades.
\section{Bilontra}
\begin{itemize}
\item {Grp. gram.:m.}
\end{itemize}
\begin{itemize}
\item {Utilização:Bras}
\end{itemize}
Velhaco.
Espertalhão.
Homem desprezível, que frequenta lupanares e más companhias.
\section{Bilontragem}
\begin{itemize}
\item {Grp. gram.:f.}
\end{itemize}
Procedimento de bilontra.
Súcia de bilontras.
\section{Bilontrar}
\begin{itemize}
\item {Grp. gram.:v. i.}
\end{itemize}
Proceder como bilontra.
\section{Bilosca}
\begin{itemize}
\item {Grp. gram.:f.}
\end{itemize}
\begin{itemize}
\item {Utilização:Bras. de Minas}
\end{itemize}
Jôgo infantil.
\section{Bilrar}
\begin{itemize}
\item {Grp. gram.:v. i.}
\end{itemize}
Trabalhar com bilros.
\section{Bilreiro}
\begin{itemize}
\item {Grp. gram.:m.}
\end{itemize}
\begin{itemize}
\item {Proveniência:(De \textunderscore bilro\textunderscore )}
\end{itemize}
Árvore meliácea do Brasil.
\section{Bilro}
\begin{itemize}
\item {Grp. gram.:m.}
\end{itemize}
\begin{itemize}
\item {Utilização:Prov.}
\end{itemize}
\begin{itemize}
\item {Utilização:beir.}
\end{itemize}
\begin{itemize}
\item {Grp. gram.:Pl.}
\end{itemize}
\begin{itemize}
\item {Proveniência:(Do lat. \textunderscore pilula\textunderscore )}
\end{itemize}
Utensílio, semelhante a um pequeno fuso ou a uma pêra, e com o qual se fazem rendas ou obras de cabello.
Pau, com que se joga a bola.
O mesmo que \textunderscore pilrito\textunderscore .
As baquetas, com que se tocam timbales.
\section{Bilró!}
\begin{itemize}
\item {Grp. gram.:interj.}
\end{itemize}
\begin{itemize}
\item {Utilização:Açor}
\end{itemize}
Viva!
Bravo!
\section{Biltraço}
\begin{itemize}
\item {Grp. gram.:m.}
\end{itemize}
\begin{itemize}
\item {Utilização:Pop.}
\end{itemize}
Grande biltre.
\section{Biltragem}
\begin{itemize}
\item {Grp. gram.:f.}
\end{itemize}
Procedimento de biltre.
Os biltres.
\section{Biltraria}
\begin{itemize}
\item {Grp. gram.:f.}
\end{itemize}
O mesmo que \textunderscore biltragem\textunderscore . Cf. Camillo, \textunderscore Cancion. Alegre\textunderscore , 409.
\section{Biltre}
\begin{itemize}
\item {Grp. gram.:m.}
\end{itemize}
Patife, homem vil.
(Cast. \textunderscore belitre\textunderscore )
\section{Bimaculado}
\begin{itemize}
\item {Grp. gram.:adj.}
\end{itemize}
\begin{itemize}
\item {Proveniência:(De \textunderscore bi...\textunderscore  + \textunderscore maculado\textunderscore )}
\end{itemize}
Que tem duas malhas ou manchas.
\section{Bímano}
\begin{itemize}
\item {Grp. gram.:adj.}
\end{itemize}
\begin{itemize}
\item {Grp. gram.:Pl.}
\end{itemize}
\begin{itemize}
\item {Proveniência:(Do lat. \textunderscore bis\textunderscore  + \textunderscore manus\textunderscore )}
\end{itemize}
Que tem duas mãos.
A espécie humana.
\section{Bimar}
\begin{itemize}
\item {Grp. gram.:adj.}
\end{itemize}
\begin{itemize}
\item {Proveniência:(Lat. \textunderscore bimaris\textunderscore )}
\end{itemize}
Que está entre dois mares; banhado por dois mares.
\section{Bimarginado}
\begin{itemize}
\item {Grp. gram.:adj.}
\end{itemize}
\begin{itemize}
\item {Proveniência:(Do lat. \textunderscore bis\textunderscore  + \textunderscore marginatus\textunderscore )}
\end{itemize}
Que tem duas margens.
\section{Bimba}
\begin{itemize}
\item {Grp. gram.:f.}
\end{itemize}
\begin{itemize}
\item {Utilização:Chul.}
\end{itemize}
Coxa.
\section{Bimba}
\begin{itemize}
\item {Grp. gram.:f.}
\end{itemize}
\begin{itemize}
\item {Utilização:Bras. do N}
\end{itemize}
Pequeno pássaro africano, granívoro, (\textunderscore pentheiria hartlaubi\textunderscore ).
Pênis de criança; pênis pouco desenvolvido.
\section{Bimba}
\begin{itemize}
\item {Grp. gram.:f.}
\end{itemize}
Arvoreta leguminosa e aquática de Angola, (\textunderscore herminiera elaphroxylon\textunderscore , Guill.).
\section{Bimbadura}
\begin{itemize}
\item {Grp. gram.:f.}
\end{itemize}
\begin{itemize}
\item {Proveniência:(De \textunderscore bimbar\textunderscore )}
\end{itemize}
Fragmento de lodo, adherente aos travessões das salinas.
\section{Bimbalhada}
\begin{itemize}
\item {Grp. gram.:f.}
\end{itemize}
\begin{itemize}
\item {Proveniência:(De \textunderscore bimbalhar\textunderscore )}
\end{itemize}
Toque simultâneo de vários sinos.
\section{Bimbalhar}
\begin{itemize}
\item {Grp. gram.:v. i.}
\end{itemize}
\begin{itemize}
\item {Proveniência:(Fr. \textunderscore brimbaler\textunderscore )}
\end{itemize}
Agitar sinos, repicá-los.
\section{Bimbar}
\begin{itemize}
\item {Grp. gram.:v. t.}
\end{itemize}
Cortar (as bimbaduras) com ugalho ou rapão.
\section{Bimbarra}
\begin{itemize}
\item {Grp. gram.:f.}
\end{itemize}
Grande alavanca de madeira.
(Cp. fr. \textunderscore brimbale\textunderscore )
\section{Bimbarreta}
\begin{itemize}
\item {fónica:rê}
\end{itemize}
\begin{itemize}
\item {Grp. gram.:f.}
\end{itemize}
\begin{itemize}
\item {Proveniência:(De \textunderscore bimbarra\textunderscore )}
\end{itemize}
Pequena bimbarra, com que se movem as grandes bombas a bôrdo.
\section{Bimembre}
\begin{itemize}
\item {Grp. gram.:adj.}
\end{itemize}
\begin{itemize}
\item {Proveniência:(Lat. \textunderscore bimembris\textunderscore )}
\end{itemize}
Que tem dois membros.
\section{Bimensal}
\begin{itemize}
\item {Grp. gram.:adj.}
\end{itemize}
\begin{itemize}
\item {Proveniência:(De \textunderscore bi...\textunderscore  + \textunderscore mensal\textunderscore )}
\end{itemize}
Relativo ao espaço de dois meses.
Que se publica ou se realiza de dois em dois meses.
\section{Bimestral}
\begin{itemize}
\item {Grp. gram.:adj.}
\end{itemize}
\begin{itemize}
\item {Proveniência:(De \textunderscore bimestre\textunderscore )}
\end{itemize}
O mesmo que \textunderscore bimensal\textunderscore .
\section{Bimestre}
\begin{itemize}
\item {Grp. gram.:adj.}
\end{itemize}
\begin{itemize}
\item {Grp. gram.:M.}
\end{itemize}
\begin{itemize}
\item {Proveniência:(Lat. \textunderscore bimestris\textunderscore )}
\end{itemize}
Que dura dois meses.
O espaço de dois meses.
\section{Bimetalismo}
\begin{itemize}
\item {Grp. gram.:m.}
\end{itemize}
\begin{itemize}
\item {Proveniência:(De \textunderscore bi...\textunderscore  + \textunderscore metal\textunderscore )}
\end{itemize}
Systema dos economistas, que sustentam que os dois metaes preciosos, o oiro e a prata, devem simultaneamente têr valor legal e sêr cunhados em moéda.
\section{Bimetalista}
\begin{itemize}
\item {Grp. gram.:m.  e  adj.}
\end{itemize}
Sectário do bimetalismo.
\section{Bimetallismo}
\begin{itemize}
\item {Grp. gram.:m.}
\end{itemize}
\begin{itemize}
\item {Proveniência:(De \textunderscore bi...\textunderscore  + \textunderscore metal\textunderscore )}
\end{itemize}
Systema dos economistas, que sustentam que os dois metaes preciosos, o oiro e a prata, devem simultaneamente têr valor legal e sêr cunhados em moéda.
\section{Bimetallista}
\begin{itemize}
\item {Grp. gram.:m.  e  adj.}
\end{itemize}
Sectário do bimetallismo.
\section{Bimo}
\begin{itemize}
\item {Grp. gram.:adj.}
\end{itemize}
\begin{itemize}
\item {Proveniência:(Lat. \textunderscore bimus\textunderscore )}
\end{itemize}
Que tem dois annos.
\section{Bina}
\begin{itemize}
\item {Grp. gram.:f.}
\end{itemize}
\begin{itemize}
\item {Utilização:Prov.}
\end{itemize}
\begin{itemize}
\item {Utilização:trasm.}
\end{itemize}
\begin{itemize}
\item {Proveniência:(De \textunderscore binar\textunderscore )}
\end{itemize}
Segunda lavra de um terreno, no outono.
\section{Binação}
\begin{itemize}
\item {Grp. gram.:f.}
\end{itemize}
O mesmo que \textunderscore binágio\textunderscore .
\section{Binado}
\begin{itemize}
\item {Grp. gram.:adj.}
\end{itemize}
\begin{itemize}
\item {Utilização:Bot.}
\end{itemize}
\begin{itemize}
\item {Proveniência:(Do lat. \textunderscore bini\textunderscore )}
\end{itemize}
Diz-se das fôlhas, que nos ramos estão dispostas, duas a duas.
\section{Binagem}
\begin{itemize}
\item {Grp. gram.:f.}
\end{itemize}
\begin{itemize}
\item {Proveniência:(De \textunderscore binar\textunderscore )}
\end{itemize}
Operação de sericicultura, que consiste em juntar dois fios ao fio já torcido do casulo.
\section{Binágio}
\begin{itemize}
\item {Grp. gram.:m.}
\end{itemize}
\begin{itemize}
\item {Proveniência:(De \textunderscore binar\textunderscore )}
\end{itemize}
Celebração de duas Missas por um padre, no mesmo dia.
\section{Binar}
\begin{itemize}
\item {Grp. gram.:v. i.}
\end{itemize}
\begin{itemize}
\item {Proveniência:(Do lat. \textunderscore bini\textunderscore )}
\end{itemize}
Praticar a operação da binagem.
Dar segundo amanho a (um terreno).
Dizer duas Missas no mesmo dia, com permissão superior.
\section{Binário}
\begin{itemize}
\item {Grp. gram.:adj.}
\end{itemize}
\begin{itemize}
\item {Proveniência:(Lat. \textunderscore binarius\textunderscore )}
\end{itemize}
Que tem duas unidades, dois elementos.
Que tem dois tempos, (em música).
\section{Binascido}
\begin{itemize}
\item {Grp. gram.:adj.}
\end{itemize}
Nascido duas vezes (falando-se de Baccho). Cf. Castilho, \textunderscore Metam.\textunderscore , 138.
\section{Binda}
\begin{itemize}
\item {Grp. gram.:f.}
\end{itemize}
Vasilha pequena para líquidos, entre os indígenas da África occidental.
Espécie de cabaça. Cf. Capello, \textunderscore Benguella\textunderscore , I, 37 e 77.
\section{Bindongas}
\begin{itemize}
\item {Grp. gram.:m. pl.}
\end{itemize}
Tríbo da África oriental.
\section{Binérvio}
\begin{itemize}
\item {Grp. gram.:adj.}
\end{itemize}
\begin{itemize}
\item {Proveniência:(De \textunderscore bi...\textunderscore  + \textunderscore nérveo\textunderscore )}
\end{itemize}
Que tem duas nervuras.
\section{Binga}
\begin{itemize}
\item {Grp. gram.:f.}
\end{itemize}
\begin{itemize}
\item {Utilização:Bras}
\end{itemize}
\begin{itemize}
\item {Utilização:Bras}
\end{itemize}
\begin{itemize}
\item {Utilização:Bras}
\end{itemize}
\begin{itemize}
\item {Utilização:Bras}
\end{itemize}
Chifre.
Isqueiro de fuzil.
O mesmo que \textunderscore colibri\textunderscore .
Espécie de cascalho.
(Do \textunderscore quimb.\textunderscore )
\section{Bingundo}
\begin{itemize}
\item {Grp. gram.:m.}
\end{itemize}
Bebida africana, fermentada, feita de água, mel e farinha de lúpulo. Cf. Serpa Pinto, I, 245.
\section{Binoculado}
\begin{itemize}
\item {Grp. gram.:adj.}
\end{itemize}
\begin{itemize}
\item {Proveniência:(Do lat. \textunderscore bini\textunderscore  + \textunderscore oculus\textunderscore )}
\end{itemize}
Que tem dois olhos.
\section{Binocular}
\begin{itemize}
\item {Grp. gram.:adj.}
\end{itemize}
\begin{itemize}
\item {Proveniência:(Do lat. \textunderscore bini\textunderscore  + \textunderscore ocularis\textunderscore )}
\end{itemize}
Que serve para os dois olhos.
\section{Binoculizar}
\begin{itemize}
\item {Grp. gram.:v. t.}
\end{itemize}
\begin{itemize}
\item {Utilização:Neol.}
\end{itemize}
Vêr pelo binóculo. Cf. Ortigão, \textunderscore Hollanda\textunderscore .
\section{Binóculo}
\begin{itemize}
\item {Grp. gram.:m.}
\end{itemize}
\begin{itemize}
\item {Proveniência:(Do lat. \textunderscore bini\textunderscore  + \textunderscore oculus\textunderscore )}
\end{itemize}
Óculo duplo, usado principalmente em espectáculos públicos.
\section{Binominal}
\begin{itemize}
\item {Grp. gram.:adj.}
\end{itemize}
\begin{itemize}
\item {Utilização:Hist. Nat.}
\end{itemize}
Diz-se da lei, segundo a qual nenhum phenómeno se apresenta insulado, mas é expressão de uma série de factos análogos.
\section{Binómino}
\begin{itemize}
\item {Grp. gram.:adj.}
\end{itemize}
\begin{itemize}
\item {Proveniência:(Lat. \textunderscore binominis\textunderscore )}
\end{itemize}
Que tem dois nomes.
\section{Binómio}
\begin{itemize}
\item {Grp. gram.:m.}
\end{itemize}
\begin{itemize}
\item {Utilização:Mathem.}
\end{itemize}
\begin{itemize}
\item {Proveniência:(De \textunderscore bi...\textunderscore  + gr. \textunderscore nomos\textunderscore )}
\end{itemize}
Quantidade, composta de dois termos, ligados pelo sinal + ou -.
\section{Bínubo}
\begin{itemize}
\item {Grp. gram.:adj.}
\end{itemize}
\begin{itemize}
\item {Proveniência:(Lat. \textunderscore binubus\textunderscore )}
\end{itemize}
Que casou duas vezes. Cf. \textunderscore Cod. Civil\textunderscore , art. 224.
\section{Bio}
\begin{itemize}
\item {Grp. gram.:m.}
\end{itemize}
\begin{itemize}
\item {Utilização:Prov.}
\end{itemize}
\begin{itemize}
\item {Utilização:trasm.}
\end{itemize}
Prego de pau, com que se prega o fundo dos cortiços e ás vezes a sua costura lateral.
\section{Bio...}
\begin{itemize}
\item {Grp. gram.:pref.}
\end{itemize}
\begin{itemize}
\item {Proveniência:(Do gr. \textunderscore bios\textunderscore )}
\end{itemize}
(designativo de \textunderscore vida\textunderscore )
\section{Bioba}
\begin{itemize}
\item {Grp. gram.:f.}
\end{itemize}
\begin{itemize}
\item {Utilização:Prov.}
\end{itemize}
\begin{itemize}
\item {Utilização:alg.}
\end{itemize}
Pisão, com dois bicos, um de cada lado.
\section{Biobio}
\begin{itemize}
\item {Grp. gram.:m.}
\end{itemize}
\begin{itemize}
\item {Utilização:Prov.}
\end{itemize}
\begin{itemize}
\item {Utilização:beir.}
\end{itemize}
Chapéu de senhora.
\section{Biochímica}
\begin{itemize}
\item {fónica:qui}
\end{itemize}
\begin{itemize}
\item {Grp. gram.:f.}
\end{itemize}
\begin{itemize}
\item {Proveniência:(De \textunderscore bio...\textunderscore  + \textunderscore chímica\textunderscore )}
\end{itemize}
Parte da Biologia, que trata da constituição chímica das substâncias produzidas pela acção da vida.
\section{Biochímico}
\begin{itemize}
\item {fónica:qui}
\end{itemize}
\begin{itemize}
\item {Grp. gram.:adj.}
\end{itemize}
\begin{itemize}
\item {Proveniência:(De \textunderscore bio...\textunderscore  + \textunderscore chímico\textunderscore )}
\end{itemize}
Diz-se de um processo, com que o microbiólogo Koch pretende poder aniquilar o bacillo tuberculoso.
\section{Bioco}
\begin{itemize}
\item {fónica:ô}
\end{itemize}
\begin{itemize}
\item {Grp. gram.:m.}
\end{itemize}
Véu ou mantilha, com que as mulheres do Algarve cobrem o rosto, para denotar austeridade.
Qualquer envoltório da cabeça e parte do rosto.
Affectação.
Hypocrisia: \textunderscore falar sem biocos\textunderscore .
Gesto para assustar, ameaça.
(Por \textunderscore veoco\textunderscore , de \textunderscore véu\textunderscore ?)
\section{Biocrático}
\begin{itemize}
\item {Grp. gram.:adj.}
\end{itemize}
Diz-se do medicamento, que modifica as funcções.
\section{Biodinâmica}
\begin{itemize}
\item {Grp. gram.:f.}
\end{itemize}
\begin{itemize}
\item {Proveniência:(De \textunderscore bio...\textunderscore  + \textunderscore dynâmica\textunderscore )}
\end{itemize}
Theoria das forças vitaes.
\section{Biodynâmica}
\begin{itemize}
\item {Grp. gram.:f.}
\end{itemize}
\begin{itemize}
\item {Proveniência:(De \textunderscore bio...\textunderscore  + \textunderscore dynâmica\textunderscore )}
\end{itemize}
Theoria das forças vitaes.
\section{Biofobia}
\begin{itemize}
\item {Grp. gram.:f.}
\end{itemize}
\begin{itemize}
\item {Proveniência:(Do gr. \textunderscore bios\textunderscore  + \textunderscore phobos\textunderscore )}
\end{itemize}
Horror mórbido á existência.
\section{Bioforina}
\begin{itemize}
\item {Grp. gram.:f.}
\end{itemize}
\begin{itemize}
\item {Proveniência:(Do gr. \textunderscore bios\textunderscore  + \textunderscore phoros\textunderscore )}
\end{itemize}
Preparação pharmacêutica, applicada como tónico e medicamento reconstituinte.
\section{Biogênese}
\begin{itemize}
\item {Grp. gram.:f.}
\end{itemize}
\begin{itemize}
\item {Proveniência:(Do gr. \textunderscore bios\textunderscore  + \textunderscore genesis\textunderscore )}
\end{itemize}
Desenvolvimento da vida.
\section{Biogenésico}
\begin{itemize}
\item {Grp. gram.:adj.}
\end{itemize}
\begin{itemize}
\item {Proveniência:(De \textunderscore biogênese\textunderscore )}
\end{itemize}
Diz-se especialmente da lei de Haeckel, que estabelece a correlação entre os desenvolvimentos embryológico, taxonómico e philogênico.
\section{Biogenético}
\begin{itemize}
\item {Grp. gram.:adj.}
\end{itemize}
O mesmo que \textunderscore biogenésico\textunderscore .
\section{Biogenia}
\textunderscore f.\textunderscore  (e der)
O mesmo que \textunderscore biogênese\textunderscore , etc.
\section{Biografar}
\begin{itemize}
\item {Grp. gram.:v. t.}
\end{itemize}
\begin{itemize}
\item {Proveniência:(De \textunderscore biógrapho\textunderscore )}
\end{itemize}
Fazer a biographia de.
\section{Biografia}
\begin{itemize}
\item {Grp. gram.:f.}
\end{itemize}
\begin{itemize}
\item {Proveniência:(De \textunderscore biógrapho\textunderscore )}
\end{itemize}
Descripção da vida de alguém.
\section{Biográfico}
\begin{itemize}
\item {Grp. gram.:adj.}
\end{itemize}
Relativo a \textunderscore biografia\textunderscore .
\section{Biógrafo}
\begin{itemize}
\item {Grp. gram.:m.}
\end{itemize}
\begin{itemize}
\item {Proveniência:(Do gr. \textunderscore bios\textunderscore  + \textunderscore graphein\textunderscore )}
\end{itemize}
Aquelle que escreve uma ou mais biografias.
\section{Biographar}
\begin{itemize}
\item {Grp. gram.:v. t.}
\end{itemize}
\begin{itemize}
\item {Proveniência:(De \textunderscore biógrapho\textunderscore )}
\end{itemize}
Fazer a biographia de.
\section{Biographia}
\begin{itemize}
\item {Grp. gram.:f.}
\end{itemize}
\begin{itemize}
\item {Proveniência:(De \textunderscore biógrapho\textunderscore )}
\end{itemize}
Descripção da vida de alguém.
\section{Biográphico}
\begin{itemize}
\item {Grp. gram.:adj.}
\end{itemize}
Relativo a \textunderscore biographia\textunderscore .
\section{Biógrapho}
\begin{itemize}
\item {Grp. gram.:m.}
\end{itemize}
\begin{itemize}
\item {Proveniência:(Do gr. \textunderscore bios\textunderscore  + \textunderscore graphein\textunderscore )}
\end{itemize}
Aquelle que escreve uma ou mais biographias.
\section{Biologia}
\begin{itemize}
\item {Grp. gram.:f.}
\end{itemize}
\begin{itemize}
\item {Proveniência:(De \textunderscore biólogo\textunderscore )}
\end{itemize}
Sciência das leis orgânicas, dos seres vivos.
\section{Biológico}
\begin{itemize}
\item {Grp. gram.:adj.}
\end{itemize}
Relativo á Biologia.
\section{Biologista}
\begin{itemize}
\item {Grp. gram.:m.}
\end{itemize}
O mesmo que \textunderscore biólogo\textunderscore .
\section{Biólogo}
\begin{itemize}
\item {Grp. gram.:m.}
\end{itemize}
\begin{itemize}
\item {Proveniência:(Do gr. \textunderscore bios\textunderscore  + \textunderscore logos\textunderscore )}
\end{itemize}
Aquelle que é versado em Biologia.
\section{Biombo}
\begin{itemize}
\item {Grp. gram.:m.}
\end{itemize}
\begin{itemize}
\item {Utilização:Bras}
\end{itemize}
Tabique móvel, formado de caixilhos, ligados por dobradiças.
Compartimento, feito de peças de madeira ou de pano, próprio para armar e desarmar.
(Talvez do japon. \textunderscore biobo\textunderscore , se êste não é importação do port.)
\section{Biometria}
\begin{itemize}
\item {Grp. gram.:f.}
\end{itemize}
\begin{itemize}
\item {Utilização:Espir.}
\end{itemize}
Medida das apparições espiritistas.
\section{Biómetro}
\begin{itemize}
\item {Grp. gram.:m.}
\end{itemize}
\begin{itemize}
\item {Proveniência:(Do gr. \textunderscore bios\textunderscore  + \textunderscore metron\textunderscore )}
\end{itemize}
O mesmo que \textunderscore agenda\textunderscore .
\section{Bionomia}
\begin{itemize}
\item {Grp. gram.:f.}
\end{itemize}
\begin{itemize}
\item {Proveniência:(Do gr. \textunderscore bios\textunderscore  + \textunderscore nomos\textunderscore )}
\end{itemize}
O mesmo que \textunderscore demographia\textunderscore .
\section{Bionómico}
\begin{itemize}
\item {Grp. gram.:adj.}
\end{itemize}
Relativo á \textunderscore bionomia\textunderscore .
\section{Biophobia}
\begin{itemize}
\item {Grp. gram.:f.}
\end{itemize}
\begin{itemize}
\item {Proveniência:(Do gr. \textunderscore bios\textunderscore  + \textunderscore phobos\textunderscore )}
\end{itemize}
Horror mórbido á existência.
\section{Biophorina}
\begin{itemize}
\item {Grp. gram.:f.}
\end{itemize}
\begin{itemize}
\item {Proveniência:(Do gr. \textunderscore bios\textunderscore  + \textunderscore phoros\textunderscore )}
\end{itemize}
Preparação pharmacêutica, applicada como tónico e medicamento reconstituinte.
\section{Bioplasma}
\begin{itemize}
\item {Grp. gram.:m.  ou  f.}
\end{itemize}
\begin{itemize}
\item {Proveniência:(Do gr. \textunderscore bios\textunderscore  + \textunderscore plassein\textunderscore )}
\end{itemize}
O mesmo que \textunderscore protoplasma\textunderscore .
\section{Biopsia}
\begin{itemize}
\item {Grp. gram.:f.}
\end{itemize}
\begin{itemize}
\item {Utilização:Med.}
\end{itemize}
\begin{itemize}
\item {Proveniência:(Do gr. \textunderscore bios\textunderscore  + \textunderscore opsis\textunderscore )}
\end{itemize}
Méthodo de investigação, que consiste em retirar de um corpo vivo um fragmento de tecido, para o sujeitar a exame microscópico.
\section{Bioquice}
\begin{itemize}
\item {Grp. gram.:f.}
\end{itemize}
\begin{itemize}
\item {Proveniência:(De \textunderscore bioco\textunderscore )}
\end{itemize}
Pudor exaggerado.
Hypocrisia.
\section{Bioquímica}
\begin{itemize}
\item {Grp. gram.:f.}
\end{itemize}
\begin{itemize}
\item {Proveniência:(De \textunderscore bio...\textunderscore  + \textunderscore chímica\textunderscore )}
\end{itemize}
Parte da Biologia, que trata da constituição química das substâncias produzidas pela acção da vida.
\section{Bioquímico}
\begin{itemize}
\item {Grp. gram.:adj.}
\end{itemize}
\begin{itemize}
\item {Proveniência:(De \textunderscore bio...\textunderscore  + \textunderscore chímico\textunderscore )}
\end{itemize}
Diz-se de um processo, com que o microbiólogo Koch pretende poder aniquilar o bacillo tuberculoso.
\section{Bioscópico}
\begin{itemize}
\item {Grp. gram.:adj.}
\end{itemize}
Relativo ao bioscópio.
\section{Bioscópio}
\begin{itemize}
\item {Grp. gram.:m.}
\end{itemize}
\begin{itemize}
\item {Proveniência:(Do gr. \textunderscore bios\textunderscore  + \textunderscore skopein\textunderscore )}
\end{itemize}
Espécie de microscópio.
\section{Biota}
\begin{itemize}
\item {Grp. gram.:f.}
\end{itemize}
\begin{itemize}
\item {Proveniência:(Do gr. \textunderscore bios\textunderscore )}
\end{itemize}
Arbusto resinoso, da fam. das coníferas.
\section{Biotaxia}
\begin{itemize}
\item {Grp. gram.:f.}
\end{itemize}
\begin{itemize}
\item {Proveniência:(Do gr. \textunderscore bios\textunderscore  + \textunderscore taxis\textunderscore )}
\end{itemize}
Tratado da classificação dos seres organizados.
\section{Biotáxico}
\begin{itemize}
\item {Grp. gram.:adj.}
\end{itemize}
Relativo á biotaxia.
\section{Biotério}
\begin{itemize}
\item {Grp. gram.:m.}
\end{itemize}
\begin{itemize}
\item {Utilização:Bras}
\end{itemize}
\begin{itemize}
\item {Utilização:Neol.}
\end{itemize}
\begin{itemize}
\item {Proveniência:(Do gr. \textunderscore bios\textunderscore , vida, + \textunderscore terein\textunderscore , guardar)}
\end{itemize}
Depósito de animaes vivos, para estudos bacteriológicos.
\section{Biotina}
\begin{itemize}
\item {Grp. gram.:f.}
\end{itemize}
\begin{itemize}
\item {Proveniência:(De \textunderscore Biot\textunderscore , n. p.)}
\end{itemize}
Silicato duplo de alumina e cal.
\section{Bioxalato}
\begin{itemize}
\item {Grp. gram.:m.}
\end{itemize}
\begin{itemize}
\item {Proveniência:(De \textunderscore bi...\textunderscore  + \textunderscore oxalato\textunderscore )}
\end{itemize}
Sal, resultante da combinação do ácido oxálico com uma base, sendo o ácido em proporção dupla da do oxalato.
\section{Bióxido}
\begin{itemize}
\item {Grp. gram.:m.}
\end{itemize}
\begin{itemize}
\item {Proveniência:(De \textunderscore bi...\textunderscore  + \textunderscore óxydo\textunderscore )}
\end{itemize}
Óxydo, que contém duas proporções de oxygênio por uma de outro corpo simples.
\section{Bióxydo}
\begin{itemize}
\item {Grp. gram.:m.}
\end{itemize}
\begin{itemize}
\item {Proveniência:(De \textunderscore bi...\textunderscore  + \textunderscore óxydo\textunderscore )}
\end{itemize}
Óxydo, que contém duas proporções de oxygênio por uma de outro corpo simples.
\section{Biparietal}
\begin{itemize}
\item {Grp. gram.:adj.}
\end{itemize}
\begin{itemize}
\item {Utilização:Anat.}
\end{itemize}
\begin{itemize}
\item {Proveniência:(De \textunderscore bi...\textunderscore  + \textunderscore parietal\textunderscore )}
\end{itemize}
Relativo aos dois parietaes.
\section{Bíparo}
\begin{itemize}
\item {Grp. gram.:adj.}
\end{itemize}
\begin{itemize}
\item {Utilização:Bot.}
\end{itemize}
\begin{itemize}
\item {Proveniência:(Do lat. \textunderscore bis\textunderscore  + \textunderscore parere\textunderscore )}
\end{itemize}
Que se produz e reproduz aos pares, dois a dois.
\section{Bipartição}
\begin{itemize}
\item {Grp. gram.:f.}
\end{itemize}
\begin{itemize}
\item {Proveniência:(De \textunderscore bi...\textunderscore  + \textunderscore partição\textunderscore )}
\end{itemize}
Divisão em duas partes.
\section{Bipartido}
\begin{itemize}
\item {Grp. gram.:adj.}
\end{itemize}
\begin{itemize}
\item {Proveniência:(Lat. \textunderscore bipartitus\textunderscore )}
\end{itemize}
Dividido ao meio; dividido em duas partes.
\section{Bipartível}
\begin{itemize}
\item {Grp. gram.:adj.}
\end{itemize}
\begin{itemize}
\item {Proveniência:(De \textunderscore bi...\textunderscore  + \textunderscore partível\textunderscore )}
\end{itemize}
Que se póde dividir em duas partes.
\section{Bipatente}
\begin{itemize}
\item {Grp. gram.:adj.}
\end{itemize}
\begin{itemize}
\item {Proveniência:(Lat. \textunderscore bipatens\textunderscore )}
\end{itemize}
Aberto de dois lados ou para dois lados.
\section{Bipedal}
\begin{itemize}
\item {Grp. gram.:adj.}
\end{itemize}
Relativo aos bípedes.
\section{Bípede}
\begin{itemize}
\item {Grp. gram.:adj.}
\end{itemize}
\begin{itemize}
\item {Grp. gram.:M.}
\end{itemize}
\begin{itemize}
\item {Proveniência:(Lat. \textunderscore bipes\textunderscore )}
\end{itemize}
Que anda em dois pés.
Animal que anda sôbre dois pés.
\section{Bipenado}
\begin{itemize}
\item {Grp. gram.:adj.}
\end{itemize}
O mesmo que \textunderscore bipenne\textunderscore ^1.
\section{Bipene}
\begin{itemize}
\item {Grp. gram.:adj.}
\end{itemize}
\begin{itemize}
\item {Proveniência:(Lat. \textunderscore bipennis\textunderscore )}
\end{itemize}
Que tem duas asas.
\section{Bipene}
\begin{itemize}
\item {Grp. gram.:f.}
\end{itemize}
\begin{itemize}
\item {Proveniência:(Lat. \textunderscore bipennis\textunderscore )}
\end{itemize}
Machadinha com dois gumes.
\section{Bipenífero}
\begin{itemize}
\item {Grp. gram.:adj.}
\end{itemize}
\begin{itemize}
\item {Proveniência:(Do lat. \textunderscore bis\textunderscore  + \textunderscore penna\textunderscore  + \textunderscore ferre\textunderscore )}
\end{itemize}
Que tem duas asas.
\section{Bipennado}
\begin{itemize}
\item {Grp. gram.:adj.}
\end{itemize}
O mesmo que \textunderscore bipenne\textunderscore ^1.
\section{Bipenne}
\begin{itemize}
\item {Grp. gram.:adj.}
\end{itemize}
\begin{itemize}
\item {Proveniência:(Lat. \textunderscore bipennis\textunderscore )}
\end{itemize}
Que tem duas asas.
\section{Bipenne}
\begin{itemize}
\item {Grp. gram.:f.}
\end{itemize}
\begin{itemize}
\item {Proveniência:(Lat. \textunderscore bipennis\textunderscore )}
\end{itemize}
Machadinha com dois gumes.
\section{Bipennífero}
\begin{itemize}
\item {Grp. gram.:adj.}
\end{itemize}
\begin{itemize}
\item {Proveniência:(Do lat. \textunderscore bis\textunderscore  + \textunderscore penna\textunderscore  + \textunderscore ferre\textunderscore )}
\end{itemize}
Que tem duas asas.
\section{Bipétalo}
\begin{itemize}
\item {Grp. gram.:adj.}
\end{itemize}
\begin{itemize}
\item {Proveniência:(De \textunderscore bi...\textunderscore  + \textunderscore pétala\textunderscore )}
\end{itemize}
Que tem duas pétalas.
\section{Bipinnulado}
\begin{itemize}
\item {Grp. gram.:adj.}
\end{itemize}
\begin{itemize}
\item {Proveniência:(De \textunderscore bi...\textunderscore  + \textunderscore pinnulado\textunderscore )}
\end{itemize}
Diz-se das fôlhas, cujo pecíolo se divíde em outros pecíolos menores com muitos folíolos.
\section{Bipinulado}
\begin{itemize}
\item {Grp. gram.:adj.}
\end{itemize}
\begin{itemize}
\item {Proveniência:(De \textunderscore bi...\textunderscore  + \textunderscore pinnulado\textunderscore )}
\end{itemize}
Diz-se das fôlhas, cujo pecíolo se divíde em outros pecíolos menores com muitos folíolos.
\section{Biplume}
\begin{itemize}
\item {Grp. gram.:adj.}
\end{itemize}
(V. \textunderscore bipenne\textunderscore ^1)
\section{Bipolar}
\begin{itemize}
\item {Grp. gram.:adj.}
\end{itemize}
\begin{itemize}
\item {Proveniência:(De \textunderscore bi...\textunderscore  + \textunderscore polar\textunderscore )}
\end{itemize}
Que tem dois polos.
\section{Bipolaridade}
\begin{itemize}
\item {Grp. gram.:f.}
\end{itemize}
\begin{itemize}
\item {Proveniência:(De \textunderscore bipolar\textunderscore )}
\end{itemize}
Estado daquillo que tem dois polos contrários.
\section{Biquadrado}
\begin{itemize}
\item {Grp. gram.:adj.}
\end{itemize}
\begin{itemize}
\item {Utilização:Mathem.}
\end{itemize}
\begin{itemize}
\item {Proveniência:(De \textunderscore bi...\textunderscore  + \textunderscore quadrado\textunderscore )}
\end{itemize}
Diz-se do quadrado multiplicado por quadrado.
Diz-se de certas equações do segundo grau.
\section{Bique-bique}
\begin{itemize}
\item {Grp. gram.:m.}
\end{itemize}
\begin{itemize}
\item {Utilização:Prov.}
\end{itemize}
Ave ribeirinha, (\textunderscore totanus ochropus\textunderscore , Lin.).
\section{Biqueira}
\begin{itemize}
\item {Grp. gram.:f.}
\end{itemize}
\begin{itemize}
\item {Utilização:Bras. do S}
\end{itemize}
\begin{itemize}
\item {Proveniência:(De \textunderscore bico\textunderscore  e \textunderscore bica\textunderscore )}
\end{itemize}
Bica, ponta, extremidade.
Tubo ou telha, em que se reunem as águas que cáem nos telhados e que lhes dá saída com um jôrro, que sobresai á fachada do edifício.
Peça de metal, na ponta do calçado.
Concêrto na ponta da meia.
Beiral, veia de água, que cai dos telhados em bica.
Burnal ou saco de coiro no focinho do cavallo, para êste não pastar.
\section{Biqueirão}
\begin{itemize}
\item {Grp. gram.:m.}
\end{itemize}
\begin{itemize}
\item {Proveniência:(De \textunderscore bico\textunderscore )}
\end{itemize}
Peixe, o mesmo que \textunderscore anchova\textunderscore .
\section{Biqueiro}
\begin{itemize}
\item {Grp. gram.:adj.}
\end{itemize}
\begin{itemize}
\item {Utilização:Fam.}
\end{itemize}
\begin{itemize}
\item {Proveniência:(De \textunderscore bico\textunderscore )}
\end{itemize}
Que come pouco, que tem ma bôca.
\section{Biquinha}
\begin{itemize}
\item {Grp. gram.:f.}
\end{itemize}
\begin{itemize}
\item {Utilização:Mad}
\end{itemize}
Ave, o mesmo que \textunderscore carreiró\textunderscore .
\section{Birba}
\begin{itemize}
\item {Grp. gram.:m.}
\end{itemize}
Birbante?:«\textunderscore soletrou francês com um birba caldeireiro\textunderscore ». Cf. Filinto, IX, 96.
\section{Birbante}
\begin{itemize}
\item {Grp. gram.:m.}
\end{itemize}
\begin{itemize}
\item {Proveniência:(It. \textunderscore birbante\textunderscore )}
\end{itemize}
Bigorrilhas; patife; biltre.
\section{Birbantão}
\begin{itemize}
\item {Grp. gram.:m.  e  adj.}
\end{itemize}
\begin{itemize}
\item {Utilização:Prov.}
\end{itemize}
\begin{itemize}
\item {Utilização:minh.}
\end{itemize}
\begin{itemize}
\item {Proveniência:(De \textunderscore birbante\textunderscore )}
\end{itemize}
Respondão.
Malcriado.
Insultador.
\section{Birbigão}
\begin{itemize}
\item {Grp. gram.:m.}
\end{itemize}
O mesmo que \textunderscore berbigão\textunderscore .
\section{Birefringente}
\begin{itemize}
\item {fónica:re}
\end{itemize}
\begin{itemize}
\item {Grp. gram.:adj.}
\end{itemize}
\begin{itemize}
\item {Utilização:Phýs.}
\end{itemize}
Diz-se dos corpos ou substâncias, em que a luz se refrange, formando duas imagens, como no crystal de rocha.
\section{Bireme}
\begin{itemize}
\item {fónica:rê}
\end{itemize}
\begin{itemize}
\item {Grp. gram.:f.}
\end{itemize}
\begin{itemize}
\item {Utilização:Ant.}
\end{itemize}
\begin{itemize}
\item {Proveniência:(Lat. \textunderscore biremis\textunderscore )}
\end{itemize}
Galera com duas ordens de remos; embarcação de dois remos.
\section{Biriba}
\begin{itemize}
\item {Grp. gram.:f.}
\end{itemize}
\begin{itemize}
\item {Utilização:Bras}
\end{itemize}
Cacete.
\section{Biriba}
\begin{itemize}
\item {Grp. gram.:f.}
\end{itemize}
\begin{itemize}
\item {Utilização:Bras}
\end{itemize}
Égua nova, mas já apta para o trabalho.
(Do tupi \textunderscore subiriba\textunderscore )
\section{Biribá}
\begin{itemize}
\item {Grp. gram.:f.}
\end{itemize}
Árvore anonácea do Brasil.
Biribazeiro.
Fruta do biribazeiro.
\section{Biribazeiro}
\begin{itemize}
\item {Grp. gram.:m.}
\end{itemize}
Árvore anonácea da América.
\section{Biri-biri}
\begin{itemize}
\item {Grp. gram.:m.}
\end{itemize}
Tambor de guerra, entre os negros.
\section{Birica}
\begin{itemize}
\item {Grp. gram.:m.}
\end{itemize}
\begin{itemize}
\item {Utilização:Bras}
\end{itemize}
Aquelle que é natural de San-Paulo; paulistano.
\section{Birimbau}
\begin{itemize}
\item {Grp. gram.:m.}
\end{itemize}
(V.brimbau)
\section{Birimbote}
\begin{itemize}
\item {Grp. gram.:m.}
\end{itemize}
(?)«\textunderscore Os camarotes são taboletas de carões pintados, de ocos trunfos, de aéreos birimbotes.\textunderscore »Filinto, IV, 5.
\section{Birman}
\begin{itemize}
\item {Grp. gram.:m.}
\end{itemize}
Língua da Birmânia; bermá.
\section{Biró}
\begin{itemize}
\item {Grp. gram.:m.}
\end{itemize}
\begin{itemize}
\item {Utilização:Bras}
\end{itemize}
Bocado, porção que de uma vez se mete na boca.
\section{Birola}
\begin{itemize}
\item {Grp. gram.:f.}
\end{itemize}
\begin{itemize}
\item {Utilização:Bras}
\end{itemize}
Fazenda de algodão, fabricada em Inglaterra.
\section{Biroró}
\begin{itemize}
\item {Grp. gram.:m.}
\end{itemize}
\begin{itemize}
\item {Utilização:Bras}
\end{itemize}
Espécie de beiju.
\section{Birostrado}
\begin{itemize}
\item {fónica:ros}
\end{itemize}
\begin{itemize}
\item {Grp. gram.:adj.}
\end{itemize}
\begin{itemize}
\item {Proveniência:(De \textunderscore bi\textunderscore  + \textunderscore rostrado\textunderscore )}
\end{itemize}
Que tem dois esporões.
\section{Birota}
\begin{itemize}
\item {fónica:ro}
\end{itemize}
\begin{itemize}
\item {Grp. gram.:f.}
\end{itemize}
\begin{itemize}
\item {Proveniência:(Lat. \textunderscore birota\textunderscore , de \textunderscore bis\textunderscore  + \textunderscore rota\textunderscore )}
\end{itemize}
Carrêta de guerra com duas rodas, usada pelos Romanos.
\section{Birota}
\begin{itemize}
\item {Grp. gram.:f.}
\end{itemize}
\begin{itemize}
\item {Utilização:Bras}
\end{itemize}
Espécie de pano de algodão.--Êrro typogr., por \textunderscore birola\textunderscore ? ou viceversa?
\section{Birra}
\begin{itemize}
\item {Grp. gram.:f.}
\end{itemize}
Teima, obstinação.
Zanga.
Vício de algumas cavalgaduras, que ferram os dentes em qualquer objecto, mormente na mangedoira.
\section{Birra}
\begin{itemize}
\item {Grp. gram.:f.}
\end{itemize}
\begin{itemize}
\item {Utilização:Des.}
\end{itemize}
\begin{itemize}
\item {Proveniência:(It. \textunderscore birra\textunderscore )}
\end{itemize}
O mesmo que \textunderscore cerveja\textunderscore . Cf. Filinto, VIII, 253.
\section{Birrar}
\begin{itemize}
\item {Grp. gram.:v. i.}
\end{itemize}
Têr birra, embirrar.
\section{Birre}
\begin{itemize}
\item {Grp. gram.:m.}
\end{itemize}
\begin{itemize}
\item {Utilização:Prov.}
\end{itemize}
\begin{itemize}
\item {Utilização:alg.}
\end{itemize}
Porco para padreação.
\section{Birrefringente}
\begin{itemize}
\item {Grp. gram.:adj.}
\end{itemize}
\begin{itemize}
\item {Utilização:Phýs.}
\end{itemize}
Diz-se dos corpos ou substâncias, em que a luz se refrange, formando duas imagens, como no crystal de rocha.
\section{Birreme}
\begin{itemize}
\item {Grp. gram.:f.}
\end{itemize}
\begin{itemize}
\item {Utilização:Ant.}
\end{itemize}
\begin{itemize}
\item {Proveniência:(Lat. \textunderscore biremis\textunderscore )}
\end{itemize}
Galera com duas ordens de remos; embarcação de dois remos.
\section{Birrentamente}
\begin{itemize}
\item {Grp. gram.:adv.}
\end{itemize}
\begin{itemize}
\item {Proveniência:(De \textunderscore birrento\textunderscore )}
\end{itemize}
Com embirração.
Com antipathia.
\section{Birrento}
\begin{itemize}
\item {Grp. gram.:adj.}
\end{itemize}
Que tem birra, embirrento.
Antipáthico.
\section{Birreto}
\begin{itemize}
\item {fónica:rê}
\end{itemize}
\begin{itemize}
\item {Grp. gram.:m.}
\end{itemize}
Fórma alatinada e erudita de \textunderscore barrete\textunderscore .
\section{Birrhos}
\begin{itemize}
\item {Grp. gram.:m. pl.}
\end{itemize}
Insectos clavicórneos, da ordem dos coleópteros.
\section{Birro}
\begin{itemize}
\item {Grp. gram.:m.}
\end{itemize}
\begin{itemize}
\item {Utilização:Ant.}
\end{itemize}
\begin{itemize}
\item {Proveniência:(Lat. \textunderscore birrus\textunderscore )}
\end{itemize}
Chapéu ou barrete vermelho.
\section{Birro}
\begin{itemize}
\item {Grp. gram.:m.}
\end{itemize}
\begin{itemize}
\item {Utilização:Bras. do N}
\end{itemize}
Bengala grossa; cacete.
\section{Birros}
\begin{itemize}
\item {Grp. gram.:m. pl.}
\end{itemize}
Insectos clavicórneos, da ordem dos coleópteros.
\section{Birrostrado}
\begin{itemize}
\item {Grp. gram.:adj.}
\end{itemize}
\begin{itemize}
\item {Proveniência:(De \textunderscore bi\textunderscore  + \textunderscore rostrado\textunderscore )}
\end{itemize}
Que tem dois esporões.
\section{Birrota}
\begin{itemize}
\item {Grp. gram.:f.}
\end{itemize}
\begin{itemize}
\item {Proveniência:(Lat. \textunderscore birota\textunderscore , de \textunderscore bis\textunderscore  + \textunderscore rota\textunderscore )}
\end{itemize}
Carrêta de guerra com duas rodas, usada pelos Romanos.
\section{Bis}
\begin{itemize}
\item {Grp. gram.:adv.}
\end{itemize}
\begin{itemize}
\item {Proveniência:(T. lat.)}
\end{itemize}
Duas vezes.
\section{Bis...}
\begin{itemize}
\item {Grp. gram.:pref.}
\end{itemize}
\begin{itemize}
\item {Utilização:lat}
\end{itemize}
(que entra na composição de várias palavras portuguesas, com a significação de \textunderscore duas vezes\textunderscore , \textunderscore duplicadamente\textunderscore )
\section{Bisa}
\begin{itemize}
\item {Grp. gram.:f.}
\end{itemize}
\begin{itemize}
\item {Utilização:Ant.}
\end{itemize}
Enxergão; almofada.
\section{Biságio}
\begin{itemize}
\item {Grp. gram.:m.}
\end{itemize}
O mesmo que \textunderscore binágio\textunderscore .
\section{Bisagra}
\begin{itemize}
\item {Grp. gram.:f.}
\end{itemize}
Dobradiça, gonzo.
Leme.
(Cp. cast. \textunderscore bisagra\textunderscore )
\section{Bisalho}
\begin{itemize}
\item {Grp. gram.:m.}
\end{itemize}
\begin{itemize}
\item {Utilização:Ant.}
\end{itemize}
\begin{itemize}
\item {Proveniência:(Do lat. \textunderscore bis\textunderscore  + \textunderscore sacculus\textunderscore )}
\end{itemize}
Saquinho, para jóias ou relíquias.
Adornos femininos de pouco valor.
\section{Bisannual}
\begin{itemize}
\item {Grp. gram.:adj.}
\end{itemize}
\begin{itemize}
\item {Proveniência:(De \textunderscore bis...\textunderscore  + \textunderscore annual\textunderscore )}
\end{itemize}
Que dura dois annos.
Que succede de dois em dois annos.
\section{Bisanual}
\begin{itemize}
\item {Grp. gram.:adj.}
\end{itemize}
\begin{itemize}
\item {Proveniência:(De \textunderscore bis...\textunderscore  + \textunderscore annual\textunderscore )}
\end{itemize}
Que dura dois annos.
Que succede de dois em dois annos.
\section{Bisão}
\begin{itemize}
\item {Grp. gram.:m.}
\end{itemize}
\begin{itemize}
\item {Proveniência:(Lat. \textunderscore bison\textunderscore )}
\end{itemize}
Boi selvagem da América.
\section{Bisar}
\begin{itemize}
\item {Grp. gram.:v. t.}
\end{itemize}
\begin{itemize}
\item {Utilização:Neol.}
\end{itemize}
\begin{itemize}
\item {Proveniência:(De \textunderscore bis\textunderscore )}
\end{itemize}
Pedir que se repita (um trecho de música, uma recitação, etc.).
Repetir.
\section{Bisarma}
\begin{itemize}
\item {Grp. gram.:f.}
\end{itemize}
Antiga arma, espécie de alabarda.
Pessôa muito corpulenta.
(B. lat. \textunderscore gisarma\textunderscore )
\section{Bísaro}
\begin{itemize}
\item {Grp. gram.:m.  e  adj.}
\end{itemize}
\begin{itemize}
\item {Utilização:Prov.}
\end{itemize}
Diz-se de um porco, corpulento e pernalto.
\section{Bisavô}
\begin{itemize}
\item {Grp. gram.:m.}
\end{itemize}
\begin{itemize}
\item {Proveniência:(De \textunderscore bis\textunderscore  + \textunderscore avô\textunderscore )}
\end{itemize}
Pai do avô ou da avó.
\section{Bisavó}
\begin{itemize}
\item {Grp. gram.:f.}
\end{itemize}
\begin{itemize}
\item {Proveniência:(De \textunderscore bis\textunderscore  + \textunderscore avó\textunderscore )}
\end{itemize}
Mãe do avô ou da avó.
\section{Bisbi}
\begin{itemize}
\item {Grp. gram.:m.}
\end{itemize}
\begin{itemize}
\item {Utilização:Mad}
\end{itemize}
O mesmo que \textunderscore abibe\textunderscore .
\section{Bisbilhotar}
\begin{itemize}
\item {Grp. gram.:v. i.}
\end{itemize}
Fazer mexericos, intrigas.
Falar em segrêdo.
(Cp. \textunderscore bisbilhoteiro\textunderscore )
\section{Bisbilhoteiro}
\begin{itemize}
\item {Grp. gram.:m.}
\end{itemize}
\begin{itemize}
\item {Proveniência:(Do it. \textunderscore bisbigliatore\textunderscore )}
\end{itemize}
Intriguista.
\section{Bisbilhotice}
\begin{itemize}
\item {Grp. gram.:f.}
\end{itemize}
Acção de \textunderscore bisbilhotar\textunderscore .
Qualidade de bisbilhoteiro.
\section{Bis-bis}
\begin{itemize}
\item {Grp. gram.:m.}
\end{itemize}
\begin{itemize}
\item {Proveniência:(T. onom.)}
\end{itemize}
Acto de rezar em voz baixa, produzindo um leve ruído sibilante:«\textunderscore ave-marias rezadas de bis-bis\textunderscore ». \textunderscore Anat. Joc.\textunderscore , I, 10.
\section{Bisbis}
\begin{itemize}
\item {Grp. gram.:m.}
\end{itemize}
O mesmo que \textunderscore bisbi\textunderscore .
\section{Bisbórria}
\begin{itemize}
\item {Grp. gram.:m.}
\end{itemize}
\begin{itemize}
\item {Proveniência:(De \textunderscore bis\textunderscore  + \textunderscore bôrra\textunderscore ?)}
\end{itemize}
Homem desprezível; troca-tintas; trapalhão; safardana; homem de bôrra.
\section{Bisbórrias}
\begin{itemize}
\item {Grp. gram.:m.}
\end{itemize}
O mesmo que \textunderscore bisbórrio\textunderscore .
\section{Bisbórrio}
\begin{itemize}
\item {Grp. gram.:m.}
\end{itemize}
(V.bisbórria)
\section{Bisca}
\begin{itemize}
\item {Grp. gram.:f.}
\end{itemize}
\begin{itemize}
\item {Utilização:Fam.}
\end{itemize}
\begin{itemize}
\item {Proveniência:(It. \textunderscore bisca\textunderscore )}
\end{itemize}
Designação de vários jogos com um baralho de quarenta cartas.
A carta que tem oito pintas.
Remoque; allusão mordaz.
Pessôa de mau carácter, com dissimulação: \textunderscore o Fagundes sempre me saiu uma bisca\textunderscore !
\section{Biscaia}
\begin{itemize}
\item {Grp. gram.:f.}
\end{itemize}
\begin{itemize}
\item {Utilização:Bras}
\end{itemize}
O mesmo que \textunderscore égua\textunderscore .
\section{Biscaínho}
\begin{itemize}
\item {Grp. gram.:adj.}
\end{itemize}
\begin{itemize}
\item {Grp. gram.:M.}
\end{itemize}
Relativo á Biscaia.
Habitante da Biscaia.
Dialecto da Biscaia; vasconço.
Casta de uva preta do Minho.
\section{Biscalheira}
\begin{itemize}
\item {Grp. gram.:f.}
\end{itemize}
\begin{itemize}
\item {Utilização:Prov.}
\end{itemize}
\begin{itemize}
\item {Utilização:minh.}
\end{itemize}
\begin{itemize}
\item {Proveniência:(De \textunderscore biscalho\textunderscore )}
\end{itemize}
Vara, rachada na extremidade e destinada a colher a fruta, pendente da árvore. Cf. \textunderscore ladra\textunderscore .
\section{Biscalho}
\begin{itemize}
\item {Grp. gram.:m.}
\end{itemize}
\begin{itemize}
\item {Utilização:Prov.}
\end{itemize}
\begin{itemize}
\item {Utilização:minh.}
\end{itemize}
Fruta, que se colhe com a biscalheira.
(Cp. lat. \textunderscore vescus\textunderscore )
\section{Biscalongo}
\begin{itemize}
\item {Grp. gram.:m.}
\end{itemize}
Espécie de minhoca, que vive enterrada na areia das praias, e que serve para isca.
O mesmo que \textunderscore arenícola\textunderscore .
(Por \textunderscore isca-longa\textunderscore ?)
\section{Biscantar}
\begin{itemize}
\item {Grp. gram.:v. t.}
\end{itemize}
\begin{itemize}
\item {Proveniência:(De \textunderscore bis...\textunderscore  + \textunderscore cantar\textunderscore )}
\end{itemize}
Celebrar no mesmo dia (duas missas).
\section{Biscar}
\begin{itemize}
\item {Grp. gram.:v. i.}
\end{itemize}
\begin{itemize}
\item {Utilização:Fam.}
\end{itemize}
Jogar bisca.
\section{Biscate}
\begin{itemize}
\item {Grp. gram.:m.}
\end{itemize}
\begin{itemize}
\item {Utilização:Fam.}
\end{itemize}
\begin{itemize}
\item {Proveniência:(De \textunderscore bisca\textunderscore )}
\end{itemize}
Picuínha.
Motejo que offende.
\section{Biscate}
\begin{itemize}
\item {Grp. gram.:m.}
\end{itemize}
Obra ou trabalho de pouca monta.
(Cp. \textunderscore biscato\textunderscore )
\section{Biscato}
\begin{itemize}
\item {Grp. gram.:m.}
\end{itemize}
Alimento, que as aves levam de cada vez no bico, para os filhos, quando estão em o ninho.
Pequena porção.
Restos de qualquer coisa.
(Relaciona-se com o lat. \textunderscore vescus\textunderscore ?)
\section{Bisco}
\begin{itemize}
\item {Grp. gram.:adj.}
\end{itemize}
Diz-se do toiro, que tem uma haste mais baixa que outra.
\section{Biscoita}
\begin{itemize}
\item {Grp. gram.:f.}
\end{itemize}
\begin{itemize}
\item {Utilização:Prov.}
\end{itemize}
\begin{itemize}
\item {Utilização:alg.}
\end{itemize}
O mesmo que \textunderscore biscoito\textunderscore .
\section{Biscoitar}
\begin{itemize}
\item {Grp. gram.:v. t.}
\end{itemize}
(V.abiscoitar)
\section{Biscoitaria}
\begin{itemize}
\item {Grp. gram.:f.}
\end{itemize}
Fábrica de biscoitos.
Estabelecimento, onde se vendem biscoitos e bolachas.
\section{Biscoiteira}
\begin{itemize}
\item {Grp. gram.:f.}
\end{itemize}
Redoma, com tampa volante, para arrecadar biscoitos, bolachas, etc.
\section{Biscoiteiro}
\begin{itemize}
\item {Grp. gram.:m.}
\end{itemize}
Fabricante ou vendedor de biscoitos.
\section{Biscoito}
\begin{itemize}
\item {Grp. gram.:m.}
\end{itemize}
\begin{itemize}
\item {Utilização:Fam.}
\end{itemize}
\begin{itemize}
\item {Utilização:Açor}
\end{itemize}
\begin{itemize}
\item {Proveniência:(Do lat. \textunderscore bis\textunderscore  + \textunderscore coctus\textunderscore )}
\end{itemize}
Bolo sêco, e mais ou menos duro, de farinha de trigo.
Bolo de farinha e açúcar, ás vezes com ovos, e cozido no forno.
Bofetão.
Solo pedregoso, com lavas a descoberto.
Obra de porcelana, duas vezes cozida:«\textunderscore os biscoitos de Sèvres\textunderscore ». Garrett, \textunderscore Viagens\textunderscore .
\section{Biscouto}
\begin{itemize}
\item {Grp. gram.:m.}
\end{itemize}
\begin{itemize}
\item {Utilização:Fam.}
\end{itemize}
\begin{itemize}
\item {Utilização:Açor}
\end{itemize}
\begin{itemize}
\item {Proveniência:(Do lat. \textunderscore bis\textunderscore  + \textunderscore coctus\textunderscore )}
\end{itemize}
Bolo sêco, e mais ou menos duro, de farinha de trigo.
Bolo de farinha e açúcar, ás vezes com ovos, e cozido no forno.
Bofetão.
Solo pedregoso, com lavas a descoberto.
Obra de porcelana, duas vezes cozida:«\textunderscore os biscoitos de Sèvres\textunderscore ». Garrett, \textunderscore Viagens\textunderscore .
\section{Biscutela}
\begin{itemize}
\item {Grp. gram.:f.}
\end{itemize}
\begin{itemize}
\item {Proveniência:(Do lat. \textunderscore bis\textunderscore  + \textunderscore scutella\textunderscore )}
\end{itemize}
Gênero de plantas crucíferas.
\section{Biscutella}
\begin{itemize}
\item {Grp. gram.:f.}
\end{itemize}
\begin{itemize}
\item {Proveniência:(Do lat. \textunderscore bis\textunderscore  + \textunderscore scutella\textunderscore )}
\end{itemize}
Gênero de plantas crucíferas.
\section{Bisegmentação}
\begin{itemize}
\item {fónica:sé}
\end{itemize}
\begin{itemize}
\item {Grp. gram.:f.}
\end{itemize}
\begin{itemize}
\item {Proveniência:(De \textunderscore bi...\textunderscore  + \textunderscore segmentação\textunderscore )}
\end{itemize}
Acção de dividir em dois segmentos.
\section{Bisegmentar}
\begin{itemize}
\item {fónica:sé}
\end{itemize}
\begin{itemize}
\item {Grp. gram.:v. t.}
\end{itemize}
\begin{itemize}
\item {Proveniência:(De \textunderscore bi...\textunderscore  + \textunderscore segmento\textunderscore )}
\end{itemize}
Dividir em dois segmentos.
\section{Bisegre}
\begin{itemize}
\item {Grp. gram.:m.}
\end{itemize}
\begin{itemize}
\item {Proveniência:(Fr. \textunderscore bisaigle\textunderscore )}
\end{itemize}
Utensílio de buxo, com que os sapateiros brunem os saltos e bordas das solas do calçado.
\section{Bisel}
\begin{itemize}
\item {Grp. gram.:m.}
\end{itemize}
Borda do vidro de um espelho, cortada obliquamente, não terminando em aresta viva.
Córte de uma aresta, formando dois ângulos oblíquos.
Engaste de pedra de anel.
Chanfradura.
(Cast. \textunderscore bisel\textunderscore )
\section{Biselar}
\begin{itemize}
\item {Grp. gram.:v. t.}
\end{itemize}
\begin{itemize}
\item {Utilização:Neol.}
\end{itemize}
\begin{itemize}
\item {Proveniência:(De \textunderscore bisel\textunderscore )}
\end{itemize}
Cortar a aresta de, formando dois ângulos oblíquos.
Chanfrar.
\section{Biselho}
\begin{itemize}
\item {fónica:zê}
\end{itemize}
\begin{itemize}
\item {Grp. gram.:m.}
\end{itemize}
\begin{itemize}
\item {Utilização:Prov.}
\end{itemize}
O mesmo que \textunderscore atilho\textunderscore .
\section{Bisélia}
\begin{itemize}
\item {fónica:sé}
\end{itemize}
\begin{itemize}
\item {Grp. gram.:f.}
\end{itemize}
\begin{itemize}
\item {Utilização:Des.}
\end{itemize}
\begin{itemize}
\item {Proveniência:(Lat. \textunderscore biselium\textunderscore )}
\end{itemize}
Cadeira com dois assentos.
\section{Bisemanal}
\begin{itemize}
\item {fónica:se}
\end{itemize}
\begin{itemize}
\item {Grp. gram.:adj.}
\end{itemize}
Que se publica duas vezes por semana.
Que se realiza duas vezes por semana.
\section{Biseriado}
\begin{itemize}
\item {fónica:se}
\end{itemize}
\begin{itemize}
\item {Grp. gram.:adj.}
\end{itemize}
\begin{itemize}
\item {Proveniência:(Do lat. \textunderscore bis\textunderscore  + \textunderscore series\textunderscore )}
\end{itemize}
Disposto em duas séries.
\section{Bis-esdrúxulo}
\begin{itemize}
\item {Grp. gram.:adj.}
\end{itemize}
\begin{itemize}
\item {Utilização:Gram.}
\end{itemize}
\begin{itemize}
\item {Proveniência:(Do it. \textunderscore bisdrucciolo\textunderscore )}
\end{itemize}
Diz-se da fórma vocabular, que, por adjuncção de pronomes pessoáes átonos, tem o acento dominante antes da ante-penúltima sýllaba, como \textunderscore louvávamos-te\textunderscore , \textunderscore louvávamo-vo-lo\textunderscore .
\section{Bisgamau}
\begin{itemize}
\item {Grp. gram.:m.}
\end{itemize}
\begin{itemize}
\item {Utilização:T. de Moncorvo}
\end{itemize}
Homem alto, escanifrado.
Pessôa astuta.
\section{Bisilicato}
\begin{itemize}
\item {fónica:si}
\end{itemize}
\begin{itemize}
\item {Grp. gram.:m.}
\end{itemize}
\begin{itemize}
\item {Proveniência:(De \textunderscore bi...\textunderscore  + \textunderscore silicato\textunderscore )}
\end{itemize}
Silicato, que contém uma proporção dupla de ácido silícico.
\section{Bisinuado}
\begin{itemize}
\item {fónica:si}
\end{itemize}
\begin{itemize}
\item {Grp. gram.:adj.}
\end{itemize}
\begin{itemize}
\item {Proveniência:(De \textunderscore bi...\textunderscore  + \textunderscore sinuado\textunderscore )}
\end{itemize}
Que tem duas sinuosidades.
\section{Bislíngua}
\begin{itemize}
\item {Grp. gram.:f.}
\end{itemize}
Nome de uma planta.
(Cp. \textunderscore bilingue\textunderscore )
\section{Bismutal}
\begin{itemize}
\item {Grp. gram.:m.}
\end{itemize}
\begin{itemize}
\item {Utilização:Pharm.}
\end{itemize}
Leite de bismuto com pepsina, contra a diarreia.
\section{Bismutanto}
\begin{itemize}
\item {Grp. gram.:m.}
\end{itemize}
\begin{itemize}
\item {Utilização:Pharm.}
\end{itemize}
Medicamento, constituído por um pó amarelo, em que entra o bismuto.
\section{Bismuthal}
\begin{itemize}
\item {Grp. gram.:m.}
\end{itemize}
\begin{itemize}
\item {Utilização:Pharm.}
\end{itemize}
Leite de bismutho com pepsina, contra a diarreia.
\section{Bismuthantho}
\begin{itemize}
\item {Grp. gram.:m.}
\end{itemize}
\begin{itemize}
\item {Utilização:Pharm.}
\end{itemize}
Medicamento, constituído por um pó amarelo, em que entra o bismutho.
\section{Bismutho}
\begin{itemize}
\item {Grp. gram.:m.}
\end{itemize}
\begin{itemize}
\item {Proveniência:(Fr. \textunderscore bismuth\textunderscore )}
\end{itemize}
Metal branco-avermelhado, formado de lâminas brilhantes e quebradiço.
\section{Bismuto}
\begin{itemize}
\item {Grp. gram.:m.}
\end{itemize}
\begin{itemize}
\item {Proveniência:(Fr. \textunderscore bismuth\textunderscore )}
\end{itemize}
Metal branco-avermelhado, formado de lâminas brilhantes e quebradiço.
\section{Bisnaca}
\begin{itemize}
\item {Grp. gram.:f.}
\end{itemize}
\begin{itemize}
\item {Utilização:Ant.}
\end{itemize}
O mesmo que \textunderscore bisnaga\textunderscore ^1.
\section{Bisnaga}
\begin{itemize}
\item {Grp. gram.:f.}
\end{itemize}
\begin{itemize}
\item {Proveniência:(Do lat. \textunderscore pastinaca\textunderscore )}
\end{itemize}
Planta umbellífera.
\section{Bisnaga}
\begin{itemize}
\item {Grp. gram.:f.}
\end{itemize}
Tubo de fôlha de chumbo, cheio de água aromática, e que, comprimido, borrifa a gente, em folganças de Carnaval.
Tubo, com fórmula medicamentosa, especialmente para limpeza dos dentes.
\section{Bisnagar}
\begin{itemize}
\item {Grp. gram.:v. t.}
\end{itemize}
\begin{itemize}
\item {Proveniência:(De \textunderscore bisnaga\textunderscore ^2)}
\end{itemize}
Borrifar, molhar, com bisnaga.
\section{Bisnau}
\begin{itemize}
\item {Grp. gram.:adj.}
\end{itemize}
\textunderscore Pássaro bisnau\textunderscore , velhaco; homem finório, astucioso.
\section{Bisneta}
\begin{itemize}
\item {Grp. gram.:f.}
\end{itemize}
\begin{itemize}
\item {Proveniência:(De \textunderscore bis\textunderscore  + \textunderscore neta\textunderscore )}
\end{itemize}
Filha de neto ou neta.
\section{Bisneto}
\begin{itemize}
\item {Grp. gram.:m.}
\end{itemize}
\begin{itemize}
\item {Proveniência:(De \textunderscore bis\textunderscore  + \textunderscore neto\textunderscore )}
\end{itemize}
Filho de neto ou neta.
\section{Bisonde}
\begin{itemize}
\item {Grp. gram.:m.}
\end{itemize}
Espécie de formiga africana, de grande cabeça.
(Relaciona-se com \textunderscore bisonte\textunderscore ?)
\section{Bisonharia}
\begin{itemize}
\item {Grp. gram.:f.}
\end{itemize}
O mesmo que \textunderscore bisonhice\textunderscore .
\section{Bisonhice}
\begin{itemize}
\item {Grp. gram.:f.}
\end{itemize}
Acanhamento; qualidade do que é bisonho.
\section{Bisonho}
\begin{itemize}
\item {Grp. gram.:adj.}
\end{itemize}
Inexperiente em coisas da guerra.
Principiante.
Acanhado.
(Cast. \textunderscore bisoño\textunderscore )
\section{Bisonte}
\begin{itemize}
\item {Grp. gram.:m.}
\end{itemize}
O mesmo que \textunderscore bisão\textunderscore .
\section{Bispado}
\begin{itemize}
\item {Grp. gram.:m.}
\end{itemize}
\begin{itemize}
\item {Proveniência:(De \textunderscore bispo\textunderscore )}
\end{itemize}
Território, comprehendido na jurisdicção espiritual de um bispo.
Diocese.
Dignidade episcopal.
\section{Bispal}
\begin{itemize}
\item {Grp. gram.:adj.}
\end{itemize}
\begin{itemize}
\item {Proveniência:(De \textunderscore bispo\textunderscore )}
\end{itemize}
O mesmo que \textunderscore episcopal\textunderscore .
\section{Bispar}
\begin{itemize}
\item {Grp. gram.:v. t.}
\end{itemize}
\begin{itemize}
\item {Utilização:Fam.}
\end{itemize}
Entrever; lobrigar; avistar ao longe.
(Or. de gír.)
\section{Bispar}
\begin{itemize}
\item {Grp. gram.:v. i.}
\end{itemize}
Exercer as funcções de bispo.
\section{Bispar-se}
\begin{itemize}
\item {Grp. gram.:v. p.}
\end{itemize}
(V.víspar-se)
\section{Bispo}
\begin{itemize}
\item {Grp. gram.:m.}
\end{itemize}
\begin{itemize}
\item {Utilização:Zool.}
\end{itemize}
\begin{itemize}
\item {Utilização:Fam.}
\end{itemize}
\begin{itemize}
\item {Proveniência:(Do lat. \textunderscore episcopus\textunderscore )}
\end{itemize}
Prelado, que governa espiritualmente determinado território, em que se comprehendem muitas paróquias.
Uropígio de algumas aves.
Esturro (na comida).
Uma das peças do xadrez.
Peixe de Portugal.
\section{Bispontar}
\textunderscore v. t.\textunderscore  (e der.)
O mesmo que \textunderscore pespontar\textunderscore , etc. Cf. Camillo, \textunderscore Senhora Ratazzi\textunderscore .
\section{Bispotada}
\begin{itemize}
\item {Grp. gram.:f.}
\end{itemize}
O conteúdo de um bispote; penicada. Cf. Macedo, \textunderscore Burros\textunderscore , 221.
\section{Bispote}
\begin{itemize}
\item {Grp. gram.:m.}
\end{itemize}
\begin{itemize}
\item {Utilização:Chul.}
\end{itemize}
\begin{itemize}
\item {Proveniência:(Do ingl. \textunderscore pisspot\textunderscore )}
\end{itemize}
Bacio; vaso de quarto de dormir, próprio para receber a urina ou as dejecções
\textunderscore ou\textunderscore  uma e outra coisa.
\section{Bispoteira}
\begin{itemize}
\item {Grp. gram.:f.}
\end{itemize}
\begin{itemize}
\item {Utilização:Chul.}
\end{itemize}
Mesa de quarto, na qual se guarda o bispote.
\section{Bisquite}
\begin{itemize}
\item {Grp. gram.:m.}
\end{itemize}
Árvore da África meridional, de cujo fruto, reduzido a farinha, se fazem biscoitos.
\section{Bissagós}
\begin{itemize}
\item {Grp. gram.:m. pl.}
\end{itemize}
(V.bijagós)
\section{Bissecção}
\begin{itemize}
\item {Grp. gram.:f.}
\end{itemize}
\begin{itemize}
\item {Proveniência:(De \textunderscore bis\textunderscore  + \textunderscore secção\textunderscore )}
\end{itemize}
Divisão em duas partes iguaes.
\section{Bissector}
\begin{itemize}
\item {Grp. gram.:adj.}
\end{itemize}
\begin{itemize}
\item {Utilização:Mathem.}
\end{itemize}
\begin{itemize}
\item {Proveniência:(De \textunderscore bis\textunderscore  + \textunderscore sector\textunderscore )}
\end{itemize}
\textunderscore Plano bissector\textunderscore , o que divide em duas partes iguaes.
\section{Bissectriz}
\begin{itemize}
\item {Grp. gram.:f.}
\end{itemize}
\begin{itemize}
\item {Utilização:Mathem.}
\end{itemize}
\begin{itemize}
\item {Proveniência:(Do lat. \textunderscore bis\textunderscore  + \textunderscore sectrix\textunderscore )}
\end{itemize}
Linha perpendicular, baixada do vértice de um ângulo sôbre a corda de um arco, que tem por centro aquelle vértice.
\section{Bissegmentação}
\begin{itemize}
\item {Grp. gram.:f.}
\end{itemize}
\begin{itemize}
\item {Proveniência:(De \textunderscore bi...\textunderscore  + \textunderscore segmentação\textunderscore )}
\end{itemize}
Acção de dividir em dois segmentos.
\section{Bissegmentar}
\begin{itemize}
\item {Grp. gram.:v. t.}
\end{itemize}
\begin{itemize}
\item {Proveniência:(De \textunderscore bi...\textunderscore  + \textunderscore segmento\textunderscore )}
\end{itemize}
Dividir em dois segmentos.
\section{Bissélia}
\begin{itemize}
\item {Grp. gram.:f.}
\end{itemize}
\begin{itemize}
\item {Utilização:Des.}
\end{itemize}
\begin{itemize}
\item {Proveniência:(Lat. \textunderscore biselium\textunderscore )}
\end{itemize}
Cadeira com dois assentos.
\section{Bissemanal}
\begin{itemize}
\item {Grp. gram.:adj.}
\end{itemize}
Que se publica duas vezes por semana.
Que se realiza duas vezes por semana.
\section{Bisseriado}
\begin{itemize}
\item {Grp. gram.:adj.}
\end{itemize}
\begin{itemize}
\item {Proveniência:(Do lat. \textunderscore bis\textunderscore  + \textunderscore series\textunderscore )}
\end{itemize}
Disposto em duas séries.
\section{Bissexo}
\begin{itemize}
\item {Grp. gram.:adj.}
\end{itemize}
O mesmo que \textunderscore bissexual\textunderscore .
\section{Bissextil}
\begin{itemize}
\item {Grp. gram.:adj.}
\end{itemize}
(V.bissexto)
\section{Bissexto}
\begin{itemize}
\item {Grp. gram.:m.}
\end{itemize}
\begin{itemize}
\item {Grp. gram.:Adj.}
\end{itemize}
\begin{itemize}
\item {Proveniência:(Lat. \textunderscore bissextus\textunderscore )}
\end{itemize}
O dia que, de quatro em quatro annos, se ajunta ao mês de Fevereiro.
Diz-se do anno, em que o mês de Fevereiro tem aquelle accrescentamento.
\section{Bissexual}
\begin{itemize}
\item {Grp. gram.:adj.}
\end{itemize}
\begin{itemize}
\item {Proveniência:(De \textunderscore bis\textunderscore  + \textunderscore sexual\textunderscore )}
\end{itemize}
Hermaphrodita; que participa dos órgãos masculinos e femininos, (falando-se de plantas).
\section{Bissexualidade}
\begin{itemize}
\item {Grp. gram.:f.}
\end{itemize}
Qualidade de bissexual.
\section{Bissexualmente}
\begin{itemize}
\item {Grp. gram.:adv.}
\end{itemize}
De modo bissexual. Cf. Camillo, \textunderscore Narcót.\textunderscore 
\section{Bissilicato}
\begin{itemize}
\item {Grp. gram.:m.}
\end{itemize}
\begin{itemize}
\item {Proveniência:(De \textunderscore bi...\textunderscore  + \textunderscore silicato\textunderscore )}
\end{itemize}
Silicato, que contém uma proporção dupla de ácido silícico.
\section{Bissinuado}
\begin{itemize}
\item {Grp. gram.:adj.}
\end{itemize}
\begin{itemize}
\item {Proveniência:(De \textunderscore bi...\textunderscore  + \textunderscore sinuado\textunderscore )}
\end{itemize}
Que tem duas sinuosidades.
\section{Bissonde}
\begin{itemize}
\item {Grp. gram.:m.}
\end{itemize}
(V.bisonde)
\section{Bissulcado}
\begin{itemize}
\item {Grp. gram.:adj.}
\end{itemize}
\begin{itemize}
\item {Proveniência:(De \textunderscore bi...\textunderscore  + \textunderscore sulcado\textunderscore )}
\end{itemize}
Que tem dois sulcos.
\section{Bissulco}
\begin{itemize}
\item {Grp. gram.:adj.}
\end{itemize}
O mesmo que \textunderscore bissulcado\textunderscore .
\section{Bistáculo}
\begin{itemize}
\item {Grp. gram.:m.}
\end{itemize}
\begin{itemize}
\item {Utilização:Prov.}
\end{itemize}
\begin{itemize}
\item {Utilização:trasm.}
\end{itemize}
Parte mínima de qualquer coisa; último resto.
(Certamente por \textunderscore bistaco\textunderscore , metáth. de \textunderscore biscato\textunderscore )
\section{Bístones}
\begin{itemize}
\item {Grp. gram.:m. pl.}
\end{itemize}
\begin{itemize}
\item {Proveniência:(Lat. \textunderscore bistones\textunderscore )}
\end{itemize}
Povo da Thrácia; o mesmo que \textunderscore thrácios\textunderscore .
\section{Bistónio}
\begin{itemize}
\item {Grp. gram.:adj.}
\end{itemize}
\begin{itemize}
\item {Proveniência:(Lat. \textunderscore bistonius\textunderscore )}
\end{itemize}
Relativo aos \textunderscore bistones\textunderscore .
\section{Bistori}
\begin{itemize}
\item {Grp. gram.:m.}
\end{itemize}
\begin{itemize}
\item {Proveniência:(Fr. \textunderscore bistouri\textunderscore )}
\end{itemize}
Escalpêllo; pequeno instrumento cirúrgico, para incisão das carnes.
\section{Bistorta}
\begin{itemize}
\item {Grp. gram.:f.}
\end{itemize}
\begin{itemize}
\item {Proveniência:(De \textunderscore bis\textunderscore  + \textunderscore torta\textunderscore )}
\end{itemize}
Planta polygónea, de raiz torcida sôbre si mesma.
\section{Bistre}
\begin{itemize}
\item {Grp. gram.:m.}
\end{itemize}
\begin{itemize}
\item {Proveniência:(Fr. \textunderscore bistre\textunderscore )}
\end{itemize}
Tinta, feita com fuligem, e de que se usa nas aguarelas.
\section{Bisulcado}
\begin{itemize}
\item {fónica:sul}
\end{itemize}
\begin{itemize}
\item {Grp. gram.:adj.}
\end{itemize}
\begin{itemize}
\item {Proveniência:(De \textunderscore bi...\textunderscore  + \textunderscore sulcado\textunderscore )}
\end{itemize}
Que tem dois sulcos.
\section{Bisulco}
\begin{itemize}
\item {fónica:sul}
\end{itemize}
\begin{itemize}
\item {Grp. gram.:adj.}
\end{itemize}
O mesmo que \textunderscore bisulcado\textunderscore .
\section{Bisultor}
\begin{itemize}
\item {Grp. gram.:m.}
\end{itemize}
\begin{itemize}
\item {Proveniência:(Lat. \textunderscore bisultor\textunderscore )}
\end{itemize}
Aquelle que é vingador duas vezes. Cf. Rui Barbosa, \textunderscore Répl.\textunderscore , 157.
\section{Bitácula}
\begin{itemize}
\item {Grp. gram.:f.}
\end{itemize}
\begin{itemize}
\item {Utilização:Gír.}
\end{itemize}
\begin{itemize}
\item {Utilização:Pop.}
\end{itemize}
\begin{itemize}
\item {Proveniência:(Do lat. \textunderscore habitaculum\textunderscore )}
\end{itemize}
Armário ou caixa, com cobertura de vidro, para encerrar a bússola.
O mesmo que \textunderscore nariz\textunderscore .
\textunderscore Levar nas bitáculas\textunderscore , levar bofetadas.
\section{Bitafe}
\begin{itemize}
\item {Grp. gram.:m.}
\end{itemize}
\begin{itemize}
\item {Utilização:Fam.}
\end{itemize}
\begin{itemize}
\item {Utilização:Ant.}
\end{itemize}
Pecha.
Mania, excentricidade.
Título, rótulo, inscripção.
(Corr. de \textunderscore epitáphio\textunderscore )
\section{Bitalha}
\begin{itemize}
\item {Grp. gram.:f.}
\end{itemize}
\begin{itemize}
\item {Utilização:Ant.}
\end{itemize}
(V.vitualha)
\section{Bitar}
\begin{itemize}
\item {Grp. gram.:v. t.}
\end{itemize}
\begin{itemize}
\item {Utilização:Prov.}
\end{itemize}
\begin{itemize}
\item {Utilização:trasm.}
\end{itemize}
O mesmo que \textunderscore entornar\textunderscore .
\section{Bite-bite}
\begin{itemize}
\item {Grp. gram.:m.}
\end{itemize}
\begin{itemize}
\item {Utilização:Prov.}
\end{itemize}
O mesmo que \textunderscore bique-bique\textunderscore .
\section{Biternado}
\begin{itemize}
\item {Grp. gram.:adj.}
\end{itemize}
\begin{itemize}
\item {Utilização:Bot.}
\end{itemize}
\begin{itemize}
\item {Proveniência:(De \textunderscore bi...\textunderscore  + \textunderscore ternado\textunderscore )}
\end{itemize}
Diz-se das fôlhas, quando o pecíolo commum se divide em três, sustentando cada um três fôlhas ternadas.
\section{Bitesga}
\begin{itemize}
\item {fónica:tês}
\end{itemize}
\begin{itemize}
\item {Grp. gram.:f.}
\end{itemize}
\begin{itemize}
\item {Utilização:Des.}
\end{itemize}
Pequena rua; viella.
Bêco sem saída.
Cubículo.
Pequena taberna.
\section{Bitocatoca}
\begin{itemize}
\item {Grp. gram.:f.}
\end{itemize}
Ave de Angola.
\section{Bitoiro}
\begin{itemize}
\item {Grp. gram.:m.}
\end{itemize}
\begin{itemize}
\item {Utilização:Prov.}
\end{itemize}
\begin{itemize}
\item {Utilização:beir.}
\end{itemize}
Variedade de urze, que floresce no inverno.
\section{Bitola}
\begin{itemize}
\item {Grp. gram.:f.}
\end{itemize}
\begin{itemize}
\item {Utilização:Náut.}
\end{itemize}
Medida, pela qual se faz qualquer trabalho.
Padrão; estalão.
Craveira.
Norma.
Largura de uma via férrea.
Grossura de um cabo.
(Do norreno)
\section{Bitoncó}
\begin{itemize}
\item {Grp. gram.:m.}
\end{itemize}
Árvore da Guiné, de aroma semelhante ao da cidreira.
\section{Bitongas}
\begin{itemize}
\item {Grp. gram.:m. pl.}
\end{itemize}
Tríbo cafreal da África oriental, no território da Inhambane.
\section{Bitú}
\begin{itemize}
\item {Grp. gram.:m.}
\end{itemize}
\begin{itemize}
\item {Utilização:Bras}
\end{itemize}
Cantiga popular.
Côca, papão.
\section{Bitume}
\textunderscore m.\textunderscore  (e der.)
O mesmo que \textunderscore betume\textunderscore , etc.
\section{Bivacar}
\begin{itemize}
\item {Grp. gram.:v. i.}
\end{itemize}
Estar em bivaque.
\section{Bivalve}
\begin{itemize}
\item {Grp. gram.:adj.}
\end{itemize}
\begin{itemize}
\item {Utilização:Bot.}
\end{itemize}
\begin{itemize}
\item {Proveniência:(Do lat. \textunderscore bis\textunderscore  + \textunderscore valva\textunderscore )}
\end{itemize}
Que tem duas valvas.
\section{Bivalvulado}
\begin{itemize}
\item {Grp. gram.:adj.}
\end{itemize}
\begin{itemize}
\item {Utilização:Bot.}
\end{itemize}
\begin{itemize}
\item {Proveniência:(De \textunderscore bi...\textunderscore  + \textunderscore válvula\textunderscore )}
\end{itemize}
Diz-se da anthera, quando a sua dehiscência se realiza por duas válvulas.
\section{Bivalvular}
\begin{itemize}
\item {Grp. gram.:adj.}
\end{itemize}
\begin{itemize}
\item {Utilização:Bot.}
\end{itemize}
\begin{itemize}
\item {Proveniência:(De \textunderscore bi...\textunderscore  + \textunderscore válvula\textunderscore )}
\end{itemize}
Diz-se da gluma, que tem duas válvulas.
\section{Bivaque}
\begin{itemize}
\item {Grp. gram.:m.}
\end{itemize}
\begin{itemize}
\item {Proveniência:(Do fr. \textunderscore bivouac\textunderscore )}
\end{itemize}
Estação provisória ou acampamento ao ar livre.
Tropa, que está em bivaque.
\section{Bívio}
\begin{itemize}
\item {Grp. gram.:m.}
\end{itemize}
\begin{itemize}
\item {Proveniência:(Lat. \textunderscore bivium\textunderscore )}
\end{itemize}
Lugar, onde se ajuntam dois caminhos.
Caminho que, dividindo-se, vai dar a pontos differentes.
\section{Bívora}
\begin{itemize}
\item {Grp. gram.:f.}
\end{itemize}
O mesmo que \textunderscore víbora\textunderscore ^1. Cf. Usque, \textunderscore Tribulações\textunderscore , 42 v.^o.
\section{Bixáceas}
\begin{itemize}
\item {Grp. gram.:f.}
\end{itemize}
Família de plantas dicotyledóneas, formada á custa das rosáceas e liliáceas de Jussieu.
\section{Bixa-corimbo}
\begin{itemize}
\item {Grp. gram.:m.}
\end{itemize}
Ave laniádea de Angola.
\section{Bixina}
\begin{itemize}
\item {Grp. gram.:f.}
\end{itemize}
Substância còrante da açafrôa.
\section{Bixíneas}
\begin{itemize}
\item {Grp. gram.:f. pl.}
\end{itemize}
O mesmo que \textunderscore bixáceas\textunderscore .
\section{Bixô}
\begin{itemize}
\item {Grp. gram.:m.}
\end{itemize}
\begin{itemize}
\item {Utilização:T. de San-Thomé}
\end{itemize}
Insecto, que ataca os pés da gente, (\textunderscore pulex penetrans\textunderscore ), e que em Angola se diz \textunderscore mahundo\textunderscore .
\section{Biza}
\begin{itemize}
\item {Grp. gram.:f.}
\end{itemize}
O mesmo que \textunderscore beja\textunderscore ^1.
\section{Bizâncio}
\begin{itemize}
\item {Grp. gram.:m.}
\end{itemize}
\begin{itemize}
\item {Grp. gram.:m.}
\end{itemize}
\begin{itemize}
\item {Utilização:Ant.}
\end{itemize}
\begin{itemize}
\item {Proveniência:(De \textunderscore Byzâncio\textunderscore , n. p.)}
\end{itemize}
Antiga moéda portuguesa.
Moéda de oiro, procedente do Império Romano do Oriente e que teve curso na Península hispânica, no século XI a XIII, pelo menos.
\section{Bizante}
\begin{itemize}
\item {Grp. gram.:m.}
\end{itemize}
O mesmo ou melhor que \textunderscore besante\textunderscore . Cf. Pant. de Aveiro, \textunderscore Itiner.\textunderscore , 2, (2.^a ed.)
\section{Bizarraço}
\begin{itemize}
\item {Grp. gram.:m.  e  adj.}
\end{itemize}
Muito bizarro, muito gentil.
\section{Bizarramente}
\begin{itemize}
\item {Grp. gram.:adv.}
\end{itemize}
De modo \textunderscore bizarro\textunderscore .
Com bizarria.
\section{Bizarrão}
\begin{itemize}
\item {Grp. gram.:m.  e  adj.}
\end{itemize}
O mesmo que \textunderscore bizarraço\textunderscore .
\section{Bizarrear}
\begin{itemize}
\item {Grp. gram.:v. i.}
\end{itemize}
\begin{itemize}
\item {Proveniência:(De \textunderscore bizarro\textunderscore )}
\end{itemize}
Proceder bizarramente.
Vangloriar-se.
\section{Bizarria}
\begin{itemize}
\item {Grp. gram.:f.}
\end{itemize}
Qualidade do que é bizarro, esquisito, excêntrico:«\textunderscore a bizarria dos trajes\textunderscore ». Camillo, \textunderscore Caveira\textunderscore , 176. (Neste sentido, é gallicismo).
Galhardia; brio.
Valentia.
Bazófia.
\section{Bizarrice}
\begin{itemize}
\item {Grp. gram.:f.}
\end{itemize}
\begin{itemize}
\item {Proveniência:(De \textunderscore bizarro\textunderscore )}
\end{itemize}
Bazófia, ostentação.
\section{Bizarro}
\begin{itemize}
\item {Grp. gram.:adj.}
\end{itemize}
Gentil; bem apessoado.
Bem vestido.
Generoso; nobre.
Jactancioso.--Na accepção de \textunderscore excêntrico\textunderscore , \textunderscore esquisito\textunderscore , \textunderscore novo\textunderscore , é francesismo.
(Cast. \textunderscore bizarro\textunderscore , do vasc.)
\section{Bizigomático}
\begin{itemize}
\item {Grp. gram.:adj.}
\end{itemize}
\begin{itemize}
\item {Utilização:Anat.}
\end{itemize}
Relativo aos dois malares conjuntamente.
\section{Bizygomático}
\begin{itemize}
\item {Grp. gram.:adj.}
\end{itemize}
\begin{itemize}
\item {Utilização:Anat.}
\end{itemize}
Relativo aos dois malares conjuntamente.
\section{Blabosa}
\begin{itemize}
\item {Grp. gram.:f.}
\end{itemize}
Planta medicinal da ilha de San-Thomé.
\section{Blanca}
\begin{itemize}
\item {Grp. gram.:f.}
\end{itemize}
\begin{itemize}
\item {Proveniência:(T. cast.)}
\end{itemize}
Antiga moéda castelhana, que teve curso em Portugal, e valía meio real branco.
\section{Blandícia}
\begin{itemize}
\item {Grp. gram.:f.}
\end{itemize}
\begin{itemize}
\item {Proveniência:(Lat. \textunderscore blanditia\textunderscore )}
\end{itemize}
Brandura.
Afago; carícia.
\section{Blandicioso}
\begin{itemize}
\item {Grp. gram.:adj.}
\end{itemize}
\begin{itemize}
\item {Proveniência:(De \textunderscore blandícia\textunderscore )}
\end{itemize}
Que afaga; que faz carinhos.
\section{Blandífluo}
\begin{itemize}
\item {Grp. gram.:adj.}
\end{itemize}
\begin{itemize}
\item {Proveniência:(Lat. \textunderscore blandifluus\textunderscore )}
\end{itemize}
Que corre brandamente.
\section{Blandíloquo}
\begin{itemize}
\item {Grp. gram.:adj.}
\end{itemize}
\begin{itemize}
\item {Proveniência:(Lat. \textunderscore blandiloquus\textunderscore )}
\end{itemize}
Que tem voz branda; que fala suavemente. Cf. Latino, \textunderscore Camões\textunderscore , 311.
\section{Blandimento}
\begin{itemize}
\item {Grp. gram.:m.}
\end{itemize}
\begin{itemize}
\item {Proveniência:(Lat. \textunderscore blandimentum\textunderscore )}
\end{itemize}
Blandícia, afago.
\section{Blandina}
\begin{itemize}
\item {Grp. gram.:f.}
\end{itemize}
\begin{itemize}
\item {Utilização:Prov.}
\end{itemize}
\begin{itemize}
\item {Utilização:minh.}
\end{itemize}
\begin{itemize}
\item {Grp. gram.:Pl.}
\end{itemize}
\begin{itemize}
\item {Utilização:Prov.}
\end{itemize}
\begin{itemize}
\item {Utilização:trasm.}
\end{itemize}
Roda-viva, azáfama.
Ralhos.
Enredos, mexericos.
(Por \textunderscore bolandinas\textunderscore , de \textunderscore bolandas\textunderscore ?)
\section{Blandineira}
\begin{itemize}
\item {Grp. gram.:f.}
\end{itemize}
\begin{itemize}
\item {Utilização:Prov.}
\end{itemize}
\begin{itemize}
\item {Utilização:trasm.}
\end{itemize}
Mulher, que se occupa em blandinas.
\section{Blandinice}
\begin{itemize}
\item {Grp. gram.:f.}
\end{itemize}
\begin{itemize}
\item {Utilização:Prov.}
\end{itemize}
\begin{itemize}
\item {Utilização:trasm.}
\end{itemize}
Assumpto de blandinas; causa de blandinas.
\section{Blando}
\begin{itemize}
\item {Grp. gram.:adj.}
\end{itemize}
\begin{itemize}
\item {Utilização:Ant.}
\end{itemize}
\begin{itemize}
\item {Proveniência:(Lat. \textunderscore blandus\textunderscore )}
\end{itemize}
Acariciador.
Agradável.
\section{Blaque}
\begin{itemize}
\item {Grp. gram.:m.}
\end{itemize}
\begin{itemize}
\item {Proveniência:(Do gr. \textunderscore blax\textunderscore )}
\end{itemize}
Espécie de milhafre africano.
\section{Blasão}
\begin{itemize}
\item {Grp. gram.:m.}
\end{itemize}
\begin{itemize}
\item {Utilização:Ant.}
\end{itemize}
O mesmo que \textunderscore brasão\textunderscore . Cf. \textunderscore Orden. do Reino\textunderscore , l. V, t. 92.
\section{Blasfemador}
\begin{itemize}
\item {Grp. gram.:m.}
\end{itemize}
Aquelle que blasfema.
\section{Blasfemamente}
\begin{itemize}
\item {Grp. gram.:adv.}
\end{itemize}
Com blasfêmia.
\section{Blasfemar}
\begin{itemize}
\item {Grp. gram.:v. t.}
\end{itemize}
\begin{itemize}
\item {Grp. gram.:V. i.}
\end{itemize}
\begin{itemize}
\item {Proveniência:(Lat. \textunderscore blasphemare\textunderscore )}
\end{itemize}
Ultrajar com blasfêmia.
Pronunciar palavras blasfemas, ultrajantes.
\section{Blasfematório}
\begin{itemize}
\item {Grp. gram.:adj.}
\end{itemize}
Que contém blasfêmia.
\section{Blasfêmia}
\begin{itemize}
\item {Grp. gram.:f.}
\end{itemize}
\begin{itemize}
\item {Proveniência:(Gr. \textunderscore blasphemia\textunderscore )}
\end{itemize}
Palavras, que ultrajam a Divindade, a religião.
Offensa, insulto dirigido contra pessôa ou coisa que se deve respeitar.
\section{Blasfemo}
\begin{itemize}
\item {Grp. gram.:adj.}
\end{itemize}
\begin{itemize}
\item {Grp. gram.:M.}
\end{itemize}
\begin{itemize}
\item {Proveniência:(Gr. \textunderscore blasphemos\textunderscore )}
\end{itemize}
Que blasfema.
Insultante.
Aquelle que blasfema.
\section{Blasonador}
\begin{itemize}
\item {Grp. gram.:m.}
\end{itemize}
Aquelle que blasona.
\section{Blasonar}
\begin{itemize}
\item {Grp. gram.:v. t.}
\end{itemize}
\begin{itemize}
\item {Utilização:Heráld.}
\end{itemize}
\begin{itemize}
\item {Grp. gram.:V. i.}
\end{itemize}
\begin{itemize}
\item {Proveniência:(Do cast. \textunderscore blasón\textunderscore )}
\end{itemize}
Mostrar com alarde; ostentar.
Descrever o escudo de (alguém).
Vangloriar-se: \textunderscore blasonar de rico\textunderscore .
\section{Blasonaria}
\begin{itemize}
\item {Grp. gram.:f.}
\end{itemize}
Acto ou qualidade de quem blasona.
\section{Blasónico}
\begin{itemize}
\item {Grp. gram.:adj.}
\end{itemize}
Relativo a brasão.
(Cp. \textunderscore blasonar\textunderscore )
\section{Blasphemador}
\begin{itemize}
\item {Grp. gram.:m.}
\end{itemize}
Aquelle que blasphema.
\section{Blasphemamente}
\begin{itemize}
\item {Grp. gram.:adv.}
\end{itemize}
Com blasphêmia.
\section{Blasphemar}
\begin{itemize}
\item {Grp. gram.:v. t.}
\end{itemize}
\begin{itemize}
\item {Grp. gram.:V. i.}
\end{itemize}
\begin{itemize}
\item {Proveniência:(Lat. \textunderscore blasphemare\textunderscore )}
\end{itemize}
Ultrajar com blasphêmia.
Pronunciar palavras blasphemas, ultrajantes.
\section{Blasphematório}
\begin{itemize}
\item {Grp. gram.:adj.}
\end{itemize}
Que contém blasphêmia.
\section{Blasphêmia}
\begin{itemize}
\item {Grp. gram.:f.}
\end{itemize}
\begin{itemize}
\item {Proveniência:(Gr. \textunderscore blasphemia\textunderscore )}
\end{itemize}
Palavras, que ultrajam a Divindade, a religião.
Offensa, insulto dirigido contra pessôa ou coisa que se deve respeitar.
\section{Blasphemo}
\begin{itemize}
\item {Grp. gram.:adj.}
\end{itemize}
\begin{itemize}
\item {Grp. gram.:M.}
\end{itemize}
\begin{itemize}
\item {Proveniência:(Gr. \textunderscore blasphemos\textunderscore )}
\end{itemize}
Que blasphema.
Insultante.
Aquelle que blasphema.
\section{Blastema}
\begin{itemize}
\item {Grp. gram.:m.}
\end{itemize}
\begin{itemize}
\item {Utilização:Hist. Nat.}
\end{itemize}
\begin{itemize}
\item {Proveniência:(Gr. \textunderscore blastema\textunderscore )}
\end{itemize}
Complexo de membranas, que cercam o embryão animal.
Eixo do desenvolvimento do embryão vegetal.
Substâncias amorphas, que se derramam dentro, ou á superficie, de um tecido.
\section{Blasto}
\begin{itemize}
\item {Grp. gram.:m.}
\end{itemize}
\begin{itemize}
\item {Proveniência:(Gr. \textunderscore blastos\textunderscore )}
\end{itemize}
Parte do embryão, de grossas radículas, que se desenvolve por effeito da germinação.
\section{Blastocarpo}
\begin{itemize}
\item {Grp. gram.:adj.}
\end{itemize}
\begin{itemize}
\item {Proveniência:(Do gr. \textunderscore blastos\textunderscore  + \textunderscore karpos\textunderscore )}
\end{itemize}
Diz-se dos frutos, cuja semente germina, antes de sair do pericarpo.
\section{Blastocele}
\begin{itemize}
\item {Grp. gram.:m.}
\end{itemize}
\begin{itemize}
\item {Proveniência:(Do gr. \textunderscore blastos\textunderscore  + \textunderscore kele\textunderscore )}
\end{itemize}
Cavidade no centro da mórula, depois da segmentação do óvulo.
\section{Blastoderme}
\begin{itemize}
\item {Grp. gram.:m.}
\end{itemize}
\begin{itemize}
\item {Proveniência:(Do gr. \textunderscore blastos\textunderscore  + \textunderscore derma\textunderscore )}
\end{itemize}
Pellícula, que se desenvolve sôbre um germe e é formada de duas láminas, a exterior das quaes há de constituir a pelle, e a interna o intestino.
\section{Blastodérmico}
\begin{itemize}
\item {Grp. gram.:adj.}
\end{itemize}
Relativo ao \textunderscore blastoderme\textunderscore .
\section{Blastóforo}
\begin{itemize}
\item {Grp. gram.:m.}
\end{itemize}
\begin{itemize}
\item {Proveniência:(Do gr. \textunderscore blastos\textunderscore  + \textunderscore phoros\textunderscore )}
\end{itemize}
Parte do embryão macrorrhizo, que serve de base ao blasto.
\section{Blastomérico}
\begin{itemize}
\item {Grp. gram.:adj.}
\end{itemize}
Relativo ao \textunderscore blastómero\textunderscore .
\section{Blastomério}
\begin{itemize}
\item {Grp. gram.:m.}
\end{itemize}
\begin{itemize}
\item {Proveniência:(Do gr. \textunderscore blastos\textunderscore  + \textunderscore meros\textunderscore )}
\end{itemize}
Cada um dos corpúsculos, que constituem a mórula.
\section{Blastómero}
\begin{itemize}
\item {Grp. gram.:m.}
\end{itemize}
\begin{itemize}
\item {Proveniência:(Do gr. \textunderscore blastos\textunderscore  + \textunderscore meros\textunderscore )}
\end{itemize}
Cada um dos corpúsculos, que constituem a mórula.
\section{Blastóphoro}
\begin{itemize}
\item {Grp. gram.:m.}
\end{itemize}
\begin{itemize}
\item {Proveniência:(Do gr. \textunderscore blastos\textunderscore  + \textunderscore phoros\textunderscore )}
\end{itemize}
Parte do embryão macrorrhizo, que serve de base ao blasto.
\section{Blastóporo}
\begin{itemize}
\item {Grp. gram.:m.}
\end{itemize}
\begin{itemize}
\item {Proveniência:(Do gr. \textunderscore blastos\textunderscore  + \textunderscore poros\textunderscore )}
\end{itemize}
Orifício da entrada da cavidade da gástrula.
\section{Blástula}
\begin{itemize}
\item {Grp. gram.:f.}
\end{itemize}
\begin{itemize}
\item {Proveniência:(Do gr. \textunderscore blastos\textunderscore )}
\end{itemize}
Vesícula blastodérmica.
\section{Blatária}
\begin{itemize}
\item {Grp. gram.:f.}
\end{itemize}
\begin{itemize}
\item {Proveniência:(Do lat. \textunderscore blatta\textunderscore )}
\end{itemize}
Planta solânea, de florescência amarela.
\section{Blaterar}
\begin{itemize}
\item {Grp. gram.:v. i.}
\end{itemize}
\begin{itemize}
\item {Proveniência:(Lat. \textunderscore blaterare\textunderscore )}
\end{itemize}
Soltar a voz (o camelo)
\section{Blau}
\begin{itemize}
\item {Grp. gram.:adj.}
\end{itemize}
\begin{itemize}
\item {Utilização:Heráld.}
\end{itemize}
Diz-se da côr azul nos brasões.
(Ant. alt. al. \textunderscore blao\textunderscore )
\section{Blefaradenite}
\begin{itemize}
\item {Grp. gram.:f.}
\end{itemize}
\begin{itemize}
\item {Proveniência:(Do gr. \textunderscore blepharon\textunderscore  + \textunderscore aden\textunderscore )}
\end{itemize}
Inflammação das glândulas palpebraes.
\section{Blefarite}
\begin{itemize}
\item {Grp. gram.:f.}
\end{itemize}
\begin{itemize}
\item {Proveniência:(Do gr. \textunderscore blepharon\textunderscore )}
\end{itemize}
Inflammação das pálpebras.
\section{Blefarofimose}
\begin{itemize}
\item {Grp. gram.:f.}
\end{itemize}
\begin{itemize}
\item {Proveniência:(Do gr. \textunderscore blepharon\textunderscore  + \textunderscore phimosis\textunderscore )}
\end{itemize}
Juncção natural, mais ou menos completa, das pálpebras de um ôlho.
\section{Blefaroplegia}
\begin{itemize}
\item {Grp. gram.:f.}
\end{itemize}
\begin{itemize}
\item {Proveniência:(Do gr. \textunderscore blepharon\textunderscore  + \textunderscore plessein\textunderscore )}
\end{itemize}
Paralysia das pálpebras.
\section{Blefaroplastia}
\begin{itemize}
\item {Grp. gram.:f.}
\end{itemize}
\begin{itemize}
\item {Proveniência:(Do gr. \textunderscore blepharon\textunderscore  + \textunderscore plassein\textunderscore )}
\end{itemize}
Operação cirúrgica, que consiste em reformar, com a pelle vizinha do ôlho, uma pálpebra destruída.
\section{Blefaróstato}
\begin{itemize}
\item {Grp. gram.:m.}
\end{itemize}
\begin{itemize}
\item {Proveniência:(Do gr. \textunderscore blepharon\textunderscore  + \textunderscore statos\textunderscore )}
\end{itemize}
Instrumento, para immobilizar as pálpebras.
\section{Blemómetro}
\begin{itemize}
\item {Grp. gram.:m.}
\end{itemize}
\begin{itemize}
\item {Proveniência:(Do gr. \textunderscore blema\textunderscore  + \textunderscore metron\textunderscore )}
\end{itemize}
Instrumento, para medir a intensidade da explosão, nas armas de fogo.
\section{Blenda}
\begin{itemize}
\item {Grp. gram.:f.}
\end{itemize}
\begin{itemize}
\item {Proveniência:(Do al. \textunderscore blende\textunderscore )}
\end{itemize}
Sulfureto de zinco natural.
\section{Blênia}
\begin{itemize}
\item {Grp. gram.:f.}
\end{itemize}
Peixe saltador das proximidades do Oceano Índico.
\section{Blênio}
\begin{itemize}
\item {Grp. gram.:m.}
\end{itemize}
O mesmo que \textunderscore blênnia\textunderscore .
\section{Blênnia}
\begin{itemize}
\item {Grp. gram.:f.}
\end{itemize}
Peixe saltador das proximidades do Oceano Índico.
\section{Blênnio}
\begin{itemize}
\item {Grp. gram.:m.}
\end{itemize}
O mesmo que \textunderscore blênnia\textunderscore .
\section{Blennophthalmia}
\begin{itemize}
\item {Grp. gram.:f.}
\end{itemize}
\begin{itemize}
\item {Proveniência:(Do gr. \textunderscore blenna\textunderscore  + \textunderscore ophthalmia\textunderscore )}
\end{itemize}
Inflammação dos olhos, caracterizada pela exsudação de muco abundante.
\section{Blennorrhagia}
\begin{itemize}
\item {Grp. gram.:f.}
\end{itemize}
\begin{itemize}
\item {Proveniência:(Do gr. \textunderscore blenna\textunderscore  + \textunderscore regnumi\textunderscore )}
\end{itemize}
Inflammação das membranas mucosas, especialmente da dos canaes urinários, acompanhada de abundante secreção com fluxo catarral.
Gonorrheia.
\section{Blennorrhágico}
\begin{itemize}
\item {Grp. gram.:adj.}
\end{itemize}
Relativo á \textunderscore blennorrhagia\textunderscore .
\section{Blennorrheia}
\begin{itemize}
\item {Grp. gram.:f.}
\end{itemize}
\begin{itemize}
\item {Proveniência:(Do gr. \textunderscore blenna\textunderscore  + \textunderscore rhein\textunderscore )}
\end{itemize}
Fluxo mucoso pela urethra, sem carácter inflammatório.
Corrimento purulento.
\section{Blennosperma}
\begin{itemize}
\item {Grp. gram.:f.}
\end{itemize}
\begin{itemize}
\item {Proveniência:(Do gr. \textunderscore blenna\textunderscore  + \textunderscore sperma\textunderscore )}
\end{itemize}
Gênero de plantas, da fam. das compostas.
\section{Blennuria}
\begin{itemize}
\item {Grp. gram.:f.}
\end{itemize}
\begin{itemize}
\item {Proveniência:(Do gr. \textunderscore blenna\textunderscore  + \textunderscore ouron\textunderscore )}
\end{itemize}
Catarro da bexiga.
\section{Blenoftalmia}
\begin{itemize}
\item {Grp. gram.:f.}
\end{itemize}
\begin{itemize}
\item {Proveniência:(Do gr. \textunderscore blenna\textunderscore  + \textunderscore ophthalmia\textunderscore )}
\end{itemize}
Inflammação dos olhos, caracterizada pela exsudação de muco abundante.
\section{Blenorragia}
\begin{itemize}
\item {Grp. gram.:f.}
\end{itemize}
\begin{itemize}
\item {Proveniência:(Do gr. \textunderscore blenna\textunderscore  + \textunderscore regnumi\textunderscore )}
\end{itemize}
Inflammação das membranas mucosas, especialmente da dos canaes urinários, acompanhada de abundante secreção com fluxo catarral.
Gonorreia.
\section{Blenorrágico}
\begin{itemize}
\item {Grp. gram.:adj.}
\end{itemize}
Relativo á \textunderscore blennorragia\textunderscore .
\section{Blenorreia}
\begin{itemize}
\item {Grp. gram.:f.}
\end{itemize}
\begin{itemize}
\item {Proveniência:(Do gr. \textunderscore blenna\textunderscore  + \textunderscore rhein\textunderscore )}
\end{itemize}
Fluxo mucoso pela uretra, sem carácter inflammatório.
Corrimento purulento.
\section{Blenosperma}
\begin{itemize}
\item {Grp. gram.:f.}
\end{itemize}
\begin{itemize}
\item {Proveniência:(Do gr. \textunderscore blenna\textunderscore  + \textunderscore sperma\textunderscore )}
\end{itemize}
Gênero de plantas, da fam. das compostas.
\section{Blenuria}
\begin{itemize}
\item {Grp. gram.:f.}
\end{itemize}
\begin{itemize}
\item {Proveniência:(Do gr. \textunderscore blenna\textunderscore  + \textunderscore ouron\textunderscore )}
\end{itemize}
Catarro da bexiga.
\section{Blepharadenite}
\begin{itemize}
\item {Grp. gram.:f.}
\end{itemize}
\begin{itemize}
\item {Proveniência:(Do gr. \textunderscore blepharon\textunderscore  + \textunderscore aden\textunderscore )}
\end{itemize}
Inflammação das glândulas palpebraes.
\section{Blepharite}
\begin{itemize}
\item {Grp. gram.:f.}
\end{itemize}
\begin{itemize}
\item {Proveniência:(Do gr. \textunderscore blepharon\textunderscore )}
\end{itemize}
Inflammação das pálpebras.
\section{Blepharophimose}
\begin{itemize}
\item {Grp. gram.:f.}
\end{itemize}
\begin{itemize}
\item {Proveniência:(Do gr. \textunderscore blepharon\textunderscore  + \textunderscore phimosis\textunderscore )}
\end{itemize}
Juncção natural, mais ou menos completa, das pálpebras de um ôlho.
\section{Blepharoplegia}
\begin{itemize}
\item {Grp. gram.:f.}
\end{itemize}
\begin{itemize}
\item {Proveniência:(Do gr. \textunderscore blepharon\textunderscore  + \textunderscore plessein\textunderscore )}
\end{itemize}
Paralysia das pálpebras.
\section{Blepharoplastia}
\begin{itemize}
\item {Grp. gram.:f.}
\end{itemize}
\begin{itemize}
\item {Proveniência:(Do gr. \textunderscore blepharon\textunderscore  + \textunderscore plassein\textunderscore )}
\end{itemize}
Operação cirúrgica, que consiste em reformar, com a pelle vizinha do ôlho, uma pálpebra destruída.
\section{Blepharóstato}
\begin{itemize}
\item {Grp. gram.:m.}
\end{itemize}
\begin{itemize}
\item {Proveniência:(Do gr. \textunderscore blepharon\textunderscore  + \textunderscore statos\textunderscore )}
\end{itemize}
Instrumento, para immobilizar as pálpebras.
\section{Bléricos}
\begin{itemize}
\item {Grp. gram.:adj. pl.}
\end{itemize}
Diz-se de uma espécie de mirabólanos. Cf. Dom. Vieira, \textunderscore Diccion.\textunderscore , ed. 1873.
\section{Blesidade}
\begin{itemize}
\item {Grp. gram.:f.}
\end{itemize}
\begin{itemize}
\item {Proveniência:(De \textunderscore bleso\textunderscore )}
\end{itemize}
Vício de pronúncia, que consiste em substituir uma consoante forte por outra fraca.
\section{Bleso}
\begin{itemize}
\item {Grp. gram.:adj.}
\end{itemize}
\begin{itemize}
\item {Proveniência:(Lat. \textunderscore blaesus\textunderscore )}
\end{itemize}
Que tem o vício da blesidade.
Que fala confusamente, que articula mal.
\section{Blindado}
\begin{itemize}
\item {Grp. gram.:adj.}
\end{itemize}
\begin{itemize}
\item {Proveniência:(De \textunderscore blindar\textunderscore )}
\end{itemize}
Revestido de chapas de aço.
Coiraçado.
\section{Blindagem}
\begin{itemize}
\item {Grp. gram.:f.}
\end{itemize}
Acção de \textunderscore blindar\textunderscore .
\section{Blindar}
\begin{itemize}
\item {Grp. gram.:v. t.}
\end{itemize}
\begin{itemize}
\item {Proveniência:(De \textunderscore blindas\textunderscore )}
\end{itemize}
Revestir, cobrir, de pranchões ou chapas de aço, para resistir ao choque das balas; coiraçar.
\section{Blindas}
\begin{itemize}
\item {Grp. gram.:f. pl.}
\end{itemize}
\begin{itemize}
\item {Proveniência:(Al. \textunderscore blende\textunderscore )}
\end{itemize}
Peças de madeira, que sustentam as faxinas de um fôsso, para resguardo dos que trabalham em fortificações.
\section{Bloco}
\begin{itemize}
\item {Utilização:Fig.}
\end{itemize}
\begin{itemize}
\item {Grp. gram.:Loc. adv.}
\end{itemize}
\begin{itemize}
\item {Proveniência:(Fr. \textunderscore bloc\textunderscore )}
\end{itemize}
\textunderscore m. Neol.\textunderscore  ou antes \textunderscore gal.\textunderscore 
Porção volumosa e sólida de uma substância pesada: \textunderscore um bloco de mármore\textunderscore .
Reunião de vários elementos políticos, para a consecução de um fim commum.
\textunderscore Em bloco\textunderscore , por junto, conjuntamente, sem exame minucioso, por grosso.
\section{Bloida}
\begin{itemize}
\item {Grp. gram.:f.}
\end{itemize}
\begin{itemize}
\item {Utilização:Ant.}
\end{itemize}
Excremento.
(Por \textunderscore bolada\textunderscore , de \textunderscore bôla\textunderscore )
\section{Bloito}
\textunderscore m. Prov. alg.\textunderscore 
Vaso de barro, de grande bojo e gargalo estreito; bilha.
\section{Blonde}
\begin{itemize}
\item {Grp. gram.:m.}
\end{itemize}
\begin{itemize}
\item {Utilização:Des.}
\end{itemize}
Espécie de tecido.
(Cast. \textunderscore blonde\textunderscore )
\section{Blongojubá}
\begin{itemize}
\item {Grp. gram.:m.}
\end{itemize}
Árvore da Guiné, cuja casca é medicinal.
\section{Bloqueante}
\begin{itemize}
\item {Grp. gram.:adj.}
\end{itemize}
Que bloqueia.
\section{Bloquear}
\begin{itemize}
\item {Grp. gram.:v. t.}
\end{itemize}
Pór bloqueio a.
(Provavelmente do ant. al. \textunderscore blokhus\textunderscore  = mod. al. \textunderscore blockhaus\textunderscore )
\section{Bloqueio}
\begin{itemize}
\item {Grp. gram.:m.}
\end{itemize}
\begin{itemize}
\item {Proveniência:(De \textunderscore bloquear\textunderscore )}
\end{itemize}
Cêrco ou operação militar, que corta a uma praça ou a um pôrto as communicações com o exterior.
\section{Bluco}
\begin{itemize}
\item {Grp. gram.:adj.}
\end{itemize}
\begin{itemize}
\item {Utilização:T. de San-Thomé}
\end{itemize}
Bravio, encapellado, (falando-se do mar).
\section{Blusa}
\begin{itemize}
\item {Grp. gram.:f.}
\end{itemize}
\begin{itemize}
\item {Proveniência:(Fr. \textunderscore blouse\textunderscore )}
\end{itemize}
Vestuário leve e largo, que os operários e as crianças usam sobre a camisa.
Espécie de casaco largo e leve para senhora.
\section{Bô}
\begin{itemize}
\item {Grp. gram.:adj.}
\end{itemize}
\begin{itemize}
\item {Utilização:ant.}
\end{itemize}
\begin{itemize}
\item {Utilização:Pop.}
\end{itemize}
O mesmo que \textunderscore bom\textunderscore :«\textunderscore pois que tinha bô lugar\textunderscore ». G. Vicente, \textunderscore J. da Beira\textunderscore .
\section{Bôa}
\begin{itemize}
\item {Grp. gram.:adj.}
\end{itemize}
(fem. de \textunderscore bom\textunderscore )
\section{Bôa}
\begin{itemize}
\item {Grp. gram.:f.}
\end{itemize}
\begin{itemize}
\item {Utilização:Bras}
\end{itemize}
\begin{itemize}
\item {Proveniência:(Lat. \textunderscore boa\textunderscore , de \textunderscore bos\textunderscore )}
\end{itemize}
Gibóia, gênero de serpentes da classe dos reptis.
Rôlo de pelles, pennas, etc., com que as senhoras agasalham o pescoço.
Espécie de pomba.
\section{Boabá}
\begin{itemize}
\item {Grp. gram.:m.}
\end{itemize}
Árvore da zona tórrida.
Madeira dessa árvore.
\section{Bôa-geira!}
\begin{itemize}
\item {Grp. gram.:interj.}
\end{itemize}
\begin{itemize}
\item {Utilização:Prov.}
\end{itemize}
\begin{itemize}
\item {Utilização:trasm.}
\end{itemize}
Deus nos defenda!
Deus nos livre!
\section{Boal}
\begin{itemize}
\item {Grp. gram.:m.  e  adj.}
\end{itemize}
Diz-se de uma variedade de uva branca e doce.
(Provavelmente do ár., segundo Dozy)
\section{Boal-bonifácio}
\begin{itemize}
\item {Grp. gram.:m.}
\end{itemize}
Casta de uva de Tôrres-Vedras.
\section{Boal-cachudo}
\begin{itemize}
\item {Grp. gram.:m.}
\end{itemize}
Casta de uva extremenha.
\section{Boal-calhariz}
\begin{itemize}
\item {Grp. gram.:m.}
\end{itemize}
Casta de uva extremenha.
\section{Boal-de-alicante}
\begin{itemize}
\item {Grp. gram.:m.}
\end{itemize}
Casta de uva de Azeitão.
\section{Boal-esfarrapado}
\begin{itemize}
\item {Grp. gram.:m.}
\end{itemize}
Casta de uva extremenha.
\section{Boal-natura}
\begin{itemize}
\item {Grp. gram.:m.}
\end{itemize}
Casta de uva de Azeitão.
\section{Boal-ratinho}
\begin{itemize}
\item {Grp. gram.:m.}
\end{itemize}
Casta de uva de Tôrres-Vedras.
\section{Boal-roxo}
\begin{itemize}
\item {Grp. gram.:m.}
\end{itemize}
Casta de uva de Azeitão.
\section{Boal-tinto}
\begin{itemize}
\item {Grp. gram.:m.}
\end{itemize}
Casta de uva preta algarvia.
\section{Bôa-mente}
\begin{itemize}
\item {Grp. gram.:loc. adv.}
\end{itemize}
\textunderscore De bôa-mente\textunderscore , de bom grado, de bôa vontade.
Também se usou sem a partícula \textunderscore de\textunderscore :«\textunderscore aquillo que boamente lhes querem dar\textunderscore ». \textunderscore Peregrinação\textunderscore .
\section{Boana}
\begin{itemize}
\item {Grp. gram.:f.}
\end{itemize}
Tábua delgada; casquinha.
\section{Boanaris}
\begin{itemize}
\item {Grp. gram.:m. pl.}
\end{itemize}
Indígenas do norte do Brasil.
\section{Bôa-noite}
\begin{itemize}
\item {Grp. gram.:f.}
\end{itemize}
\begin{itemize}
\item {Utilização:T. da Bairrada}
\end{itemize}
O mesmo que \textunderscore noitibó\textunderscore .
O mesmo que \textunderscore bôas-noites\textunderscore .
\section{Boanova}
\begin{itemize}
\item {Grp. gram.:f.}
\end{itemize}
\begin{itemize}
\item {Proveniência:(De \textunderscore bôa\textunderscore ^1 + \textunderscore nova\textunderscore )}
\end{itemize}
Nome vulgar de uma pequena borboleta branca.
\section{Bôas-noites}
\begin{itemize}
\item {Grp. gram.:f. pl.}
\end{itemize}
\begin{itemize}
\item {Grp. gram.:M.}
\end{itemize}
\begin{itemize}
\item {Utilização:Prov.}
\end{itemize}
Planta e flôr, da fam. das nyctagináceas.
O mesmo que \textunderscore noitibó\textunderscore .
\section{Bôas-noutes}
\begin{itemize}
\item {Grp. gram.:f. pl.}
\end{itemize}
\begin{itemize}
\item {Grp. gram.:M.}
\end{itemize}
\begin{itemize}
\item {Utilização:Prov.}
\end{itemize}
Planta e flôr, da fam. das nyctagináceas.
O mesmo que \textunderscore noitibó\textunderscore .
\section{Bôas-vindas}
\begin{itemize}
\item {Grp. gram.:f. pl.}
\end{itemize}
Felicitação, expressão de contentamento, pela chegada de alguém.
\section{Boataria}
\begin{itemize}
\item {Grp. gram.:f.}
\end{itemize}
\begin{itemize}
\item {Utilização:Neol.}
\end{itemize}
Muitos boatos.
\section{Boateiro}
\begin{itemize}
\item {Grp. gram.:m.}
\end{itemize}
\begin{itemize}
\item {Utilização:Neol.}
\end{itemize}
Aquelle que espalha boatos.
\section{Boato}
\begin{itemize}
\item {Grp. gram.:m.}
\end{itemize}
\begin{itemize}
\item {Proveniência:(Lat. \textunderscore boatus\textunderscore )}
\end{itemize}
Notícia, que corre publicamente, sem precedência conhecida.
Balela; atoarda; rumores.
\section{Boava}
\begin{itemize}
\item {Grp. gram.:m.  e  adj.}
\end{itemize}
\begin{itemize}
\item {Utilização:Bras}
\end{itemize}
Diz-se de qualquer indivíduo estrangeiro, especialmente português.
(Cp. \textunderscore emboaba\textunderscore )
\section{Bôa-venturança}
\begin{itemize}
\item {Grp. gram.:f.}
\end{itemize}
\begin{itemize}
\item {Utilização:Des.}
\end{itemize}
O mesmo que \textunderscore bem-aventurança\textunderscore .
\section{Boavinda}
\begin{itemize}
\item {Grp. gram.:f.}
\end{itemize}
O mesmo que \textunderscore bôas-vindas\textunderscore .
\section{Bôa-volta}
\begin{itemize}
\item {Grp. gram.:f.}
\end{itemize}
Em cynegética, diz-se \textunderscore cão de bôa volta\textunderscore  o que, á voz do dono, se volta logo.
\section{Boazinha}
\begin{itemize}
\item {Grp. gram.:f.}
\end{itemize}
Variedade de pêra alentejana.
\section{Boba}
\begin{itemize}
\item {fónica:bô}
\end{itemize}
\textunderscore f.\textunderscore  (e der.)
O mesmo que \textunderscore buba\textunderscore , etc.
\section{Boba}
\begin{itemize}
\item {fónica:bô}
\end{itemize}
\begin{itemize}
\item {Grp. gram.:f.}
\end{itemize}
\begin{itemize}
\item {Utilização:Bras}
\end{itemize}
\begin{itemize}
\item {Proveniência:(De \textunderscore bobo\textunderscore )}
\end{itemize}
Mulher idiota ou aparvalhada.
\section{Bobagem}
\begin{itemize}
\item {Grp. gram.:f.}
\end{itemize}
\begin{itemize}
\item {Utilização:Bras. do N}
\end{itemize}
(V.bobice)
Asneira.
\section{Bobagens}
\begin{itemize}
\item {Grp. gram.:f. pl.}
\end{itemize}
\begin{itemize}
\item {Utilização:Bras. de Minas}
\end{itemize}
Qualquer coisa comestível.
Presente ou brinde modesto a alguém que faz annos.
\section{Bobal}
\begin{itemize}
\item {Grp. gram.:m.}
\end{itemize}
Espécie de videira brasileira.
\section{Bobalhão}
\begin{itemize}
\item {Grp. gram.:m.}
\end{itemize}
\begin{itemize}
\item {Utilização:Bras}
\end{itemize}
\begin{itemize}
\item {Proveniência:(De \textunderscore bobo\textunderscore )}
\end{itemize}
Indivíduo ridículo ou palerma, que serve de divertimento aos outros.
\section{Bobamente}
\begin{itemize}
\item {Grp. gram.:adv.}
\end{itemize}
Com maneiras de bobo.
\section{Bobear}
\begin{itemize}
\item {Grp. gram.:v. i.}
\end{itemize}
\begin{itemize}
\item {Proveniência:(De \textunderscore bobo\textunderscore )}
\end{itemize}
Fazer ou dizer bobices.
\section{Bobeche}
\begin{itemize}
\item {Grp. gram.:m.}
\end{itemize}
\begin{itemize}
\item {Proveniência:(T. fr.)}
\end{itemize}
O mesmo que \textunderscore arandela\textunderscore .
\section{Bóbeda}
\begin{itemize}
\item {Grp. gram.:f.}
\end{itemize}
\begin{itemize}
\item {Utilização:Prov.}
\end{itemize}
\begin{itemize}
\item {Utilização:trasm.}
\end{itemize}
O mesmo que \textunderscore abóbora\textunderscore .
\section{Bóbeda}
\begin{itemize}
\item {Grp. gram.:f.}
\end{itemize}
O mesmo que \textunderscore abóbada\textunderscore . Cf. Castilho, \textunderscore Fastos\textunderscore , I, 126.
\section{Bobice}
\begin{itemize}
\item {Grp. gram.:f.}
\end{itemize}
Maneiras, acção, de bobo.
Truanice; palhaçada.
\section{Bobina}
\begin{itemize}
\item {Grp. gram.:f.}
\end{itemize}
\begin{itemize}
\item {Utilização:Neol.}
\end{itemize}
\begin{itemize}
\item {Proveniência:(Fr. \textunderscore bobine\textunderscore )}
\end{itemize}
Parte dos instrumentos de Phýsica, formada de um fio metállico, enrolado num carretel.
Carrinho, pequeno cylindro, de madeira ou de metal, com rebordos, em que se enrolam fios de seda, linho, etc., para serviço de costura ou bordados.
Grande rôlo de papel contínuo, para impressões typográphicas de grande tiragem.
\section{Bobinador}
\begin{itemize}
\item {Grp. gram.:m.}
\end{itemize}
Máquina para \textunderscore bobinar\textunderscore .
Apparelho, para encher as bobinas de pellículas, em cinematographia.
\section{Bobinagem}
\begin{itemize}
\item {Grp. gram.:f.}
\end{itemize}
Operação de \textunderscore bobinar\textunderscore .
\section{Bobinar}
\begin{itemize}
\item {Grp. gram.:v. t.}
\end{itemize}
Pôr (papel) em bobina.
Enrolar, formando bobina.
\section{Bobinete}
\begin{itemize}
\item {fónica:nê}
\end{itemize}
\begin{itemize}
\item {Grp. gram.:m.}
\end{itemize}
\begin{itemize}
\item {Utilização:Bras}
\end{itemize}
O mesmo que \textunderscore filó\textunderscore .
\section{Bobo}
\begin{itemize}
\item {fónica:bô}
\end{itemize}
\begin{itemize}
\item {Grp. gram.:m.}
\end{itemize}
\begin{itemize}
\item {Proveniência:(Do lat. \textunderscore balbus\textunderscore )}
\end{itemize}
Personagem, adjunta aos príncipes e nobres da Idade-Média, para os divertir com truanices e esgares.
Aquelle que diverte os outros ou procura diverti-los com phrases ou gestos burlescos.
Truão.
\section{Bobó}
\begin{itemize}
\item {fónica:bó-bó}
\end{itemize}
\begin{itemize}
\item {Grp. gram.:m.}
\end{itemize}
\begin{itemize}
\item {Utilização:Bras}
\end{itemize}
Comida africana, usada na Baía, e feita de uma espécie de feijão com um pouco de banana.
\section{Bobô-bobô}
\begin{itemize}
\item {Grp. gram.:m.}
\end{itemize}
Árvore da ilha de San-Thomé, applicável a construcções de cubatas.--Outros escrevem \textunderscore bôbo-bôbo\textunderscore , como se vê no museu colonial da \textunderscore Sociedade de Geogr. de Lisbôa\textunderscore .
\section{Boboca}
\begin{itemize}
\item {Grp. gram.:f.}
\end{itemize}
\begin{itemize}
\item {Utilização:Bras}
\end{itemize}
O mesmo que \textunderscore biboca\textunderscore .
\section{Bóbonax}
\begin{itemize}
\item {Grp. gram.:m.}
\end{itemize}
Espécie de palmeira americana, cujas fôlhas recurvas e amareladas se dividem em tiras, com que se fazem os chapéus chamados \textunderscore panamás\textunderscore .
\section{Bôca}
\begin{itemize}
\item {Grp. gram.:f.}
\end{itemize}
\begin{itemize}
\item {Utilização:Ext.}
\end{itemize}
\begin{itemize}
\item {Utilização:Fam.}
\end{itemize}
\begin{itemize}
\item {Proveniência:(Do lat. \textunderscore bucca\textunderscore )}
\end{itemize}
Cavidade no rosto, pela qual os alimentos se ingerem no corpo.
Abertura, na parte anterior da cabeça de alguns animaes, e pela qual êstes ingerem os alimentos.
Qualquer abertura ou córte, que dê ideia de bôca.
Pessôa que come: \textunderscore em minha casa há seis bôcas\textunderscore .
Lábios: \textunderscore bôca linda\textunderscore .
Órgão da fala: \textunderscore cala essa bôca\textunderscore .
Entrada.
Mossa.
Barra (de rio ou baía).
Princípio ou fim de rua.
\textunderscore Têr má bôca\textunderscore , não gostar de todas as comidas, sêr exigente.
\textunderscore Têr boa bôca\textunderscore , gostar de tudo.
\textunderscore Bôca de fogo\textunderscore , peça de artilharia.
\textunderscore Bôca do estômago\textunderscore , parte externa e anterior do corpo, correspondente á abertura, que communíca o estômago com os intestinos.
\textunderscore Com o credo na bôca\textunderscore , em grande perigo; com muito medo.
\textunderscore Abrir a bôca\textunderscore , bocejar.
O mesmo que falar ou discorrer: \textunderscore aquelle deputado nunca abriu a bôca\textunderscore .
\section{Bóca!}
\begin{itemize}
\item {Grp. gram.:interj.}
\end{itemize}
\begin{itemize}
\item {Utilização:Prov.}
\end{itemize}
\begin{itemize}
\item {Utilização:beir.}
\end{itemize}
Voz, com que se chamam os cães, especialmente para comerem ou para apanharem com a bôca qualquer objecto.
\section{Boça}
\begin{itemize}
\item {Grp. gram.:f.}
\end{itemize}
\begin{itemize}
\item {Utilização:Náut.}
\end{itemize}
Nome commum a muitos cabos.
\section{Bôca-aberta}
\begin{itemize}
\item {Grp. gram.:m.  e  f.}
\end{itemize}
\begin{itemize}
\item {Utilização:Fam.}
\end{itemize}
Indivíduo, que se espanta com tudo.
Pessôa indolente, sem cuidados.
\section{Bocaça}
\begin{itemize}
\item {Grp. gram.:f.}
\end{itemize}
Bôca muito grande.
\section{Bocada}
\begin{itemize}
\item {Grp. gram.:f.}
\end{itemize}
\begin{itemize}
\item {Utilização:Prov.}
\end{itemize}
Bôca de saco, nos apparelhos piscatórios de arrastar para terra.
O mesmo que \textunderscore bocado\textunderscore . (Colhido em Turquel)
\section{Bôca-de-barbo}
\begin{itemize}
\item {Grp. gram.:f.}
\end{itemize}
\begin{itemize}
\item {Utilização:Bras}
\end{itemize}
Espécie de abelha.
\section{Bôca-de-leão}
\begin{itemize}
\item {Grp. gram.:f.}
\end{itemize}
\begin{itemize}
\item {Utilização:Bras}
\end{itemize}
Planta, o mesmo que \textunderscore antirrhino\textunderscore .
\section{Bôca-de-lobo}
\begin{itemize}
\item {Grp. gram.:f.}
\end{itemize}
\begin{itemize}
\item {Utilização:Carp.}
\end{itemize}
Peça fêmea de uma endentação em triângulo.
\section{Bôca-de-mina}
\begin{itemize}
\item {Grp. gram.:f.}
\end{itemize}
Casta de uva beirôa.
\section{Bocado}
\begin{itemize}
\item {Grp. gram.:m.}
\end{itemize}
\begin{itemize}
\item {Utilização:Prov.}
\end{itemize}
\begin{itemize}
\item {Proveniência:(De \textunderscore bôca\textunderscore )}
\end{itemize}
Porção de alimento, que se póde meter na bôca de uma vez.
Pedaço.
Pequeno decurso de tempo: \textunderscore encontrei-o há bocado\textunderscore .
Parte do freio que está dentro da bôca da cavalgadura.
O mesmo que \textunderscore sustento\textunderscore : \textunderscore não têm quem lhe dê o bocado\textunderscore .
\section{Bôca-doce}
\begin{itemize}
\item {Grp. gram.:m.}
\end{itemize}
\begin{itemize}
\item {Utilização:Prov.}
\end{itemize}
\begin{itemize}
\item {Utilização:alg.}
\end{itemize}
Designação de um peixe de água salgada.
\section{Bocadura}
\begin{itemize}
\item {Grp. gram.:f.}
\end{itemize}
Bôca da peça de artilharia.
\section{Bocage-e-capello}
\begin{itemize}
\item {Grp. gram.:m.}
\end{itemize}
\begin{itemize}
\item {Proveniência:(De \textunderscore Bocage\textunderscore , n. p. + \textunderscore Capello\textunderscore , n. p.)}
\end{itemize}
Peixe plagióstomo, de corpo prismático triangular, cabeça larga, escamas pedunculádas e dorso espinhoso.
\section{Bocageano}
\begin{itemize}
\item {Grp. gram.:adj.}
\end{itemize}
Relativo a Bocage, poéta português.
Que tem semelhança com a feição literária de Bocage.
\section{Bocaina}
\begin{itemize}
\item {Grp. gram.:f.}
\end{itemize}
\begin{itemize}
\item {Utilização:Bras}
\end{itemize}
\begin{itemize}
\item {Grp. gram.:M.}
\end{itemize}
\begin{itemize}
\item {Utilização:Prov.}
\end{itemize}
\begin{itemize}
\item {Utilização:minh.}
\end{itemize}
\begin{itemize}
\item {Proveniência:(De \textunderscore bôca\textunderscore )}
\end{itemize}
Depressão numa serra ou cordilheira.
Um bôca-aberta.
\section{Bocaiúva}
\begin{itemize}
\item {Grp. gram.:f.}
\end{itemize}
Espécie de coqueiro do Brasil.
\section{Bocal}
\begin{itemize}
\item {Grp. gram.:m.}
\end{itemize}
\begin{itemize}
\item {Proveniência:(De \textunderscore bôca\textunderscore )}
\end{itemize}
Abertura de um vaso, de um candeeiro, de um frasco, etc.
Parte do castiçal, onde entra a vela.
Parapeito em roda de um poço.
Embocadura dos instrumentos de vento.
Betilho.
Parte do freio, que entra na bôca do cavallo.
Casta de uva.
Canhão de casaco ou casaca. Cf. Corvo, \textunderscore Anno na Côrte\textunderscore , c. V.
\section{Boçal}
\begin{itemize}
\item {Grp. gram.:adj.}
\end{itemize}
Estúpido.
Grosseiro.
\section{Boçal}
\begin{itemize}
\item {Grp. gram.:m.}
\end{itemize}
\begin{itemize}
\item {Utilização:Prov.}
\end{itemize}
\begin{itemize}
\item {Utilização:alent.}
\end{itemize}
\begin{itemize}
\item {Utilização:Bras. do S}
\end{itemize}
Rêde de corda, que se adapta ao focinho dos animaes, para que não comam nas seáras.
Espécie de cabresto com focinheira.
(Cast. \textunderscore bozal\textunderscore )
\section{Boçalete}
\begin{itemize}
\item {Grp. gram.:m.}
\end{itemize}
\begin{itemize}
\item {Utilização:Bras. do S}
\end{itemize}
Pequeno boçal.
Cabresto aperfeiçoado. Cf. Cezimbra, \textunderscore Ensaio\textunderscore .
\section{Boçalidade}
\begin{itemize}
\item {Grp. gram.:f.}
\end{itemize}
Qualidade de boçal.
\section{Bocalrão}
\begin{itemize}
\item {Grp. gram.:m.}
\end{itemize}
Casta de uva preta algarvia.
\section{Bocalvo}
\begin{itemize}
\item {Grp. gram.:adj.}
\end{itemize}
\begin{itemize}
\item {Proveniência:(De \textunderscore bôca\textunderscore  + \textunderscore alvo\textunderscore )}
\end{itemize}
Diz-se do toiro, que tem o focinho branco, sendo escura a cabeça.
\section{Bôca-molle}
\begin{itemize}
\item {Grp. gram.:m.}
\end{itemize}
Peixe do Brasil.
\section{Bocana}
\begin{itemize}
\item {Grp. gram.:m.  e  f.}
\end{itemize}
\begin{itemize}
\item {Utilização:T. da Bairrada}
\end{itemize}
O mesmo que \textunderscore bôca-aberta\textunderscore , bocaina.
\section{Bôca-negra}
\begin{itemize}
\item {Grp. gram.:f.}
\end{itemize}
Peixe dos Açores, o mesmo que \textunderscore bagre\textunderscore .
\section{Bocanha}
\begin{itemize}
\item {Grp. gram.:f.}
\end{itemize}
\begin{itemize}
\item {Proveniência:(De \textunderscore bôca\textunderscore )}
\end{itemize}
A parte ôca do marfim.
\section{Bocanhar}
\begin{itemize}
\item {Grp. gram.:v. i.}
\end{itemize}
\begin{itemize}
\item {Utilização:Prov.}
\end{itemize}
\begin{itemize}
\item {Utilização:trasm.}
\end{itemize}
Fazer bocanho; o mesmo que \textunderscore abocanhar\textunderscore ^2.
\section{Bocanhim}
\begin{itemize}
\item {Grp. gram.:m.}
\end{itemize}
\begin{itemize}
\item {Utilização:Gír.}
\end{itemize}
Clavina; trabuco.
\section{Bocanho}
\begin{itemize}
\item {Grp. gram.:m.}
\end{itemize}
\begin{itemize}
\item {Utilização:Prov.}
\end{itemize}
\begin{itemize}
\item {Utilização:Prov.}
\end{itemize}
O mesmo que \textunderscore aberta\textunderscore , em dias de chuva.
Momento, instante.
\section{Bocar}
\begin{itemize}
\item {Grp. gram.:v. t.}
\end{itemize}
(V.abocar)
\section{Boçar}
\begin{itemize}
\item {Grp. gram.:v. t.}
\end{itemize}
(V.aboçar)
\section{Boçardas}
\begin{itemize}
\item {Grp. gram.:f. pl.}
\end{itemize}
\begin{itemize}
\item {Utilização:Náut.}
\end{itemize}
Travessões curvos, na roda da prôa, para a reforçarem.
(Cp. fr. \textunderscore bossoir\textunderscore )
\section{Bocarela}
\begin{itemize}
\item {Grp. gram.:m.  e  f.}
\end{itemize}
\begin{itemize}
\item {Utilização:Prov.}
\end{itemize}
\begin{itemize}
\item {Utilização:trasm.}
\end{itemize}
Pessôa, que fala muito; tagarela. (Colhido em V. P. de Aguíar)
\section{Bócas}
\begin{itemize}
\item {Grp. gram.:m.}
\end{itemize}
Crustáceo, espécie de pequeno caranguejo.
\section{Bocaxi}
\begin{itemize}
\item {Grp. gram.:m.}
\end{itemize}
\begin{itemize}
\item {Utilização:Ant.}
\end{itemize}
O mesmo que \textunderscore bocaxim\textunderscore .
\section{Bocaxim}
\begin{itemize}
\item {Grp. gram.:m.}
\end{itemize}
Entretela; tarlatana.
\section{Boccónia}
\begin{itemize}
\item {Grp. gram.:f.}
\end{itemize}
\begin{itemize}
\item {Proveniência:(De \textunderscore Bocconi\textunderscore , n. p.)}
\end{itemize}
Gênero de plantas papaveráceas.
\section{Bocejador}
\begin{itemize}
\item {Grp. gram.:m.}
\end{itemize}
Aquelle que boceja.
\section{Bocejar}
\begin{itemize}
\item {Grp. gram.:v. i.}
\end{itemize}
Fazer bocejo.
Têr aborrecimento, enfastiar-se.
\section{Bocejo}
\begin{itemize}
\item {Grp. gram.:m.}
\end{itemize}
\begin{itemize}
\item {Proveniência:(De \textunderscore bôca\textunderscore )}
\end{itemize}
Abrimento involuntário de bôca, aspirando o ar e expirando-o depois prolongadamente.
\section{Bocel}
\begin{itemize}
\item {Grp. gram.:m.}
\end{itemize}
Tóro, moldura redonda na base das columnas.
(Cast. \textunderscore bocel\textunderscore )
\section{Bocelar}
\begin{itemize}
\item {Grp. gram.:v. t.}
\end{itemize}
Ornar com bocéis.
Dar fórma de bocel ou meia cana a.
\section{Bocelinho}
\begin{itemize}
\item {Grp. gram.:m.}
\end{itemize}
O mesmo que \textunderscore bocelino\textunderscore .
\section{Bocelino}
\begin{itemize}
\item {Grp. gram.:m.}
\end{itemize}
\begin{itemize}
\item {Proveniência:(De \textunderscore bocel\textunderscore )}
\end{itemize}
A parte mais delgada da columna, junto ao capitel.
\section{Boceta}
\begin{itemize}
\item {fónica:cê}
\end{itemize}
\begin{itemize}
\item {Grp. gram.:f.}
\end{itemize}
\begin{itemize}
\item {Utilização:Mad}
\end{itemize}
\begin{itemize}
\item {Utilização:Bras}
\end{itemize}
\begin{itemize}
\item {Utilização:chul.}
\end{itemize}
\begin{itemize}
\item {Utilização:Fig.}
\end{itemize}
\begin{itemize}
\item {Proveniência:(Do b. lat. \textunderscore buxetum\textunderscore )}
\end{itemize}
Pequena caixa, cylíndrica ou oval, de papelão ou madeira.
Caixa de rapé.
Partes pudendas da mulher.
\textunderscore Boceta de Pandora\textunderscore , origem de todos os males.
\section{Bocete}
\begin{itemize}
\item {fónica:cê}
\end{itemize}
\begin{itemize}
\item {Grp. gram.:m.}
\end{itemize}
\begin{itemize}
\item {Proveniência:(Do fr. \textunderscore bossete\textunderscore , como affirmam diccionaristas? Em tal caso, deveriamos têr \textunderscore bossete\textunderscore )}
\end{itemize}
Ornato, em fórma de cabeça de prego convexa, nas antigas saias de malha e coiraças.
Florão, ornato circular, na intersecção dos artezões.
\section{Boceteiro}
\begin{itemize}
\item {Grp. gram.:m.}
\end{itemize}
Fabricante de bocetas.
\section{Bocha}
\begin{itemize}
\item {fónica:bô}
\end{itemize}
\begin{itemize}
\item {Grp. gram.:f.}
\end{itemize}
\begin{itemize}
\item {Utilização:Prov.}
\end{itemize}
\begin{itemize}
\item {Utilização:alg.}
\end{itemize}
\begin{itemize}
\item {Utilização:T. de Chaves}
\end{itemize}
Grande barriga; obesidade.
Bôlha, empôla, bojega.
\section{Bochacrar}
\begin{itemize}
\item {Grp. gram.:v. t.}
\end{itemize}
\begin{itemize}
\item {Utilização:Prov.}
\end{itemize}
\begin{itemize}
\item {Utilização:trasm.}
\end{itemize}
O mesmo que \textunderscore bochechar\textunderscore ; enxaguar (a bôca).
\section{Bochacro}
\begin{itemize}
\item {Grp. gram.:m.}
\end{itemize}
\begin{itemize}
\item {Utilização:Prov.}
\end{itemize}
\begin{itemize}
\item {Utilização:trasm.}
\end{itemize}
Acto de \textunderscore bochacrar\textunderscore .
Porção de líquido, que se toma na boca, para a enxaguar.
\section{Bochecha}
\begin{itemize}
\item {fónica:chê}
\end{itemize}
\begin{itemize}
\item {Grp. gram.:f.}
\end{itemize}
\begin{itemize}
\item {Proveniência:(Do ant. alto al. \textunderscore bozan\textunderscore )}
\end{itemize}
Parte mais saliente de cada uma das faces.
Effeito de inflar as faces, assoprando.
Parte mais saliente do bojo do navio, na direcção da prôa.
\section{Bochechada}
\begin{itemize}
\item {Grp. gram.:f.}
\end{itemize}
Pancada nas bochechas.
Bochecho.
\section{Bochechão}
\begin{itemize}
\item {Grp. gram.:m.}
\end{itemize}
O mesmo que \textunderscore bochechada\textunderscore .
\section{Bochechar}
\begin{itemize}
\item {Grp. gram.:v. t.  e  i.}
\end{itemize}
\begin{itemize}
\item {Proveniência:(De \textunderscore bochecha\textunderscore )}
\end{itemize}
Agitar, com o movimento das faces, um líquido, que se toma na bôca.
\section{Bochecho}
\begin{itemize}
\item {fónica:chê}
\end{itemize}
\begin{itemize}
\item {Grp. gram.:m.}
\end{itemize}
Acto de \textunderscore bochechar\textunderscore .
Porção de líquido, que se póde tomar de uma vez na bôca.
Pequena quantidade de líquido.
\section{Bochechudo}
\begin{itemize}
\item {Grp. gram.:m.}
\end{itemize}
Que tem grandes bochechas.
\section{Boches}
\begin{itemize}
\item {Grp. gram.:m. pl.}
\end{itemize}
\begin{itemize}
\item {Utilização:Prov.}
\end{itemize}
Coração e fígado dos animaes; bofes.
\section{Bochinche}
\begin{itemize}
\item {Grp. gram.:m.}
\end{itemize}
\begin{itemize}
\item {Utilização:Bras. do S}
\end{itemize}
\begin{itemize}
\item {Proveniência:(T. cast.)}
\end{itemize}
Dança popular, espécie de batuque.
\section{Bòchinho}
\begin{itemize}
\item {Grp. gram.:adj.}
\end{itemize}
\begin{itemize}
\item {Utilização:Prov.}
\end{itemize}
\begin{itemize}
\item {Utilização:trasm.}
\end{itemize}
Diz-se do indivíduo que se zanga facilmente.
\section{Bochornal}
\begin{itemize}
\item {Grp. gram.:adj.}
\end{itemize}
\begin{itemize}
\item {Proveniência:(De \textunderscore bochorno\textunderscore )}
\end{itemize}
Quente, abafadiço.
\section{Bochorno}
\begin{itemize}
\item {Grp. gram.:m.}
\end{itemize}
\begin{itemize}
\item {Proveniência:(Do lat. \textunderscore vulturnus\textunderscore )}
\end{itemize}
Ar abafadiço.
Vento quente.
\section{Bocicódio}
\begin{itemize}
\item {Grp. gram.:m.}
\end{itemize}
\begin{itemize}
\item {Utilização:Ant.}
\end{itemize}
Homem acanhado, sorumbático:«\textunderscore quem diz ausente, diz cabisbaixo, bocicódio.\textunderscore »\textunderscore Anat. Joc.\textunderscore , I, 197. Cf. \textunderscore Aulegrafia\textunderscore , 171.
\section{Bocim}
\begin{itemize}
\item {Grp. gram.:m.}
\end{itemize}
Peça de ferro, que se aparafusa nas caldeiras das máquinas de vapor.
(Cast. \textunderscore bocin\textunderscore )
\section{Bócio}
\begin{itemize}
\item {Grp. gram.:m.}
\end{itemize}
Papeira.
(Da mesma or. que \textunderscore bochecha\textunderscore . Cp. fr. \textunderscore bosse\textunderscore )
\section{Bocó}
\begin{itemize}
\item {Grp. gram.:m.  e  adj.}
\end{itemize}
\begin{itemize}
\item {Utilização:Bras. do S}
\end{itemize}
Pateta.
Pascácio.
\section{Bocó}
\begin{itemize}
\item {Grp. gram.:m.}
\end{itemize}
\begin{itemize}
\item {Utilização:Bras}
\end{itemize}
Pequena mala ou alforge de coiro ainda não curtido e com pêlo.
\section{Bocónia}
\begin{itemize}
\item {Grp. gram.:f.}
\end{itemize}
\begin{itemize}
\item {Proveniência:(De \textunderscore Bocconi\textunderscore , n. p.)}
\end{itemize}
Gênero de plantas papaveráceas.
\section{Bocório}
\begin{itemize}
\item {Grp. gram.:adj.}
\end{itemize}
\begin{itemize}
\item {Utilização:Bras}
\end{itemize}
Reles; patife.
\section{Boçudo}
\begin{itemize}
\item {Grp. gram.:adj.}
\end{itemize}
Diz-se do pau ou moca, usada como arma de guerra pelos gentios da África occidental portuguesa.
\section{Boda}
\begin{itemize}
\item {fónica:bô}
\end{itemize}
\begin{itemize}
\item {Grp. gram.:f.}
\end{itemize}
\begin{itemize}
\item {Utilização:Ext.}
\end{itemize}
\begin{itemize}
\item {Grp. gram.:Pl.}
\end{itemize}
\begin{itemize}
\item {Proveniência:(Lat. \textunderscore vota\textunderscore )}
\end{itemize}
Celebração de casamento.
Banquete e festa, para celebrar casamento.
Banquete.
\textunderscore Bodas de prata\textunderscore , celebração festiva do 25.^o anniversário de casamento.
\textunderscore Bodas de oiro\textunderscore , celebração festiva do 50.^o anniversário de casamento.
\section{Bodalha}
\begin{itemize}
\item {Grp. gram.:f.}
\end{itemize}
\begin{itemize}
\item {Utilização:Ant.}
\end{itemize}
Pequena porca; leitôa.
\section{Bodalhão}
\begin{itemize}
\item {Grp. gram.:m.}
\end{itemize}
\begin{itemize}
\item {Utilização:Prov.}
\end{itemize}
\begin{itemize}
\item {Utilização:trasm.}
\end{itemize}
\begin{itemize}
\item {Proveniência:(De \textunderscore bodalho\textunderscore )}
\end{itemize}
Homem sujo, immundo.
\section{Bodalho}
\begin{itemize}
\item {Grp. gram.:adj.}
\end{itemize}
\begin{itemize}
\item {Utilização:Prov.}
\end{itemize}
\begin{itemize}
\item {Utilização:beir.}
\end{itemize}
Sujo, porco.
(Cp. bodalha)
\section{Bodana}
\begin{itemize}
\item {Grp. gram.:f.}
\end{itemize}
\begin{itemize}
\item {Utilização:Prov.}
\end{itemize}
\begin{itemize}
\item {Utilização:trasm.}
\end{itemize}
Planta trepadeira de bagos vermelhos.
\section{Bode}
\begin{itemize}
\item {Grp. gram.:m.}
\end{itemize}
\begin{itemize}
\item {Utilização:Bras}
\end{itemize}
Ruminante cavicórneo, macho da cabra.
Mestiço, mulato.
(Cast. \textunderscore bode\textunderscore )
\section{Bode}
\begin{itemize}
\item {Grp. gram.:m.}
\end{itemize}
Antiga moéda de Cambaia.
\section{Bodefe}
\begin{itemize}
\item {Grp. gram.:adj.}
\end{itemize}
\begin{itemize}
\item {Utilização:Des.}
\end{itemize}
Feio? que tem cara de bode? Cf. Camillo, \textunderscore Noites de Insóm.\textunderscore , VII, 75.
\section{Bodega}
\begin{itemize}
\item {Grp. gram.:f.}
\end{itemize}
\begin{itemize}
\item {Utilização:Fam.}
\end{itemize}
\begin{itemize}
\item {Proveniência:(Do lat. \textunderscore apotheca\textunderscore )}
\end{itemize}
Taberna, tasca, locanda.
Comida grosseira.
Casa suja.
Porcaria.
\section{Bodegão}
\begin{itemize}
\item {Grp. gram.:m.}
\end{itemize}
O mesmo que \textunderscore bodego\textunderscore .
\section{Bodego}
\begin{itemize}
\item {fónica:dê}
\end{itemize}
\begin{itemize}
\item {Grp. gram.:m.}
\end{itemize}
\begin{itemize}
\item {Utilização:T. de Lanhoso}
\end{itemize}
O mesmo que \textunderscore bodegueiro\textunderscore .
(Cp. \textunderscore bodega\textunderscore )
\section{Bodegonice}
\begin{itemize}
\item {Grp. gram.:f.}
\end{itemize}
\begin{itemize}
\item {Utilização:Prov.}
\end{itemize}
\begin{itemize}
\item {Utilização:trasm.}
\end{itemize}
\begin{itemize}
\item {Proveniência:(De \textunderscore bodegão\textunderscore )}
\end{itemize}
O mesmo que \textunderscore bodeguice\textunderscore .
\section{Bodegueiro}
\begin{itemize}
\item {Grp. gram.:m.}
\end{itemize}
\begin{itemize}
\item {Proveniência:(De \textunderscore bodega\textunderscore )}
\end{itemize}
Taberneiro.
Pessôa pouco asseada, que se emporcalha comendo.
\section{Bodeguice}
\begin{itemize}
\item {Grp. gram.:f.}
\end{itemize}
Porcaria.
Acto ou coisa própria de \textunderscore bodega\textunderscore .
\section{Bodeguim}
\begin{itemize}
\item {Grp. gram.:m.}
\end{itemize}
\begin{itemize}
\item {Utilização:Bras}
\end{itemize}
Bode bravo.
\section{Bodeiro}
\begin{itemize}
\item {Grp. gram.:adj.}
\end{itemize}
\begin{itemize}
\item {Utilização:Prov.}
\end{itemize}
\begin{itemize}
\item {Utilização:beir.}
\end{itemize}
Que dá bodo.
\section{Bodejar}
\begin{itemize}
\item {Grp. gram.:v. i.}
\end{itemize}
\begin{itemize}
\item {Utilização:Bras. do N}
\end{itemize}
Soltar a voz (o bode).
Gaguejar.
\section{Bodejo}
\begin{itemize}
\item {Grp. gram.:m.}
\end{itemize}
\begin{itemize}
\item {Utilização:Bras. do N}
\end{itemize}
Voz do bode.
Acto de \textunderscore bodejar\textunderscore .
\section{Bodelgo}
\begin{itemize}
\item {Grp. gram.:m.}
\end{itemize}
\begin{itemize}
\item {Utilização:Prov.}
\end{itemize}
\begin{itemize}
\item {Utilização:trasm.}
\end{itemize}
Rapaz gordo e bochechudo.
\section{Bodelha}
\begin{itemize}
\item {fónica:dê}
\end{itemize}
\begin{itemize}
\item {Grp. gram.:f.}
\end{itemize}
O mesmo que \textunderscore bodelho\textunderscore .
\section{Bodelhão}
\begin{itemize}
\item {Grp. gram.:m.}
\end{itemize}
O mesmo que \textunderscore bodalhão\textunderscore . Cf. P. Ivo, \textunderscore Sêllo da Roda\textunderscore .
\section{Bodelho}
\begin{itemize}
\item {fónica:dê}
\end{itemize}
\begin{itemize}
\item {Grp. gram.:m.}
\end{itemize}
Alga vesiculosa, ou carvalho do mar.
\section{Bodemeria}
\begin{itemize}
\item {Grp. gram.:f.}
\end{itemize}
\begin{itemize}
\item {Utilização:Jur.}
\end{itemize}
\begin{itemize}
\item {Proveniência:(Do ingl. \textunderscore bottomry\textunderscore )}
\end{itemize}
Câmbio marítimo, ou contrato de empréstimo a risco, sobre o casco, quilha e apparelhos de navio. Cf. F. Borges, \textunderscore Diccion. Jur.\textunderscore 
\section{Bodianos}
\begin{itemize}
\item {Grp. gram.:m. pl.}
\end{itemize}
O mesmo que \textunderscore bodiões\textunderscore .
\section{Bodigo}
\begin{itemize}
\item {Grp. gram.:m.}
\end{itemize}
\begin{itemize}
\item {Utilização:Prov.}
\end{itemize}
\begin{itemize}
\item {Utilização:trasm.}
\end{itemize}
O mesmo que \textunderscore bodelgo\textunderscore .
\section{Bòdinho}
\begin{itemize}
\item {fónica:bó}
\end{itemize}
\begin{itemize}
\item {Grp. gram.:m.}
\end{itemize}
\begin{itemize}
\item {Utilização:Prov.}
\end{itemize}
\begin{itemize}
\item {Utilização:minh.}
\end{itemize}
\begin{itemize}
\item {Proveniência:(De \textunderscore bode\textunderscore )}
\end{itemize}
O mesmo que \textunderscore peixe-cão\textunderscore .
\section{Bodiões}
\begin{itemize}
\item {Grp. gram.:m. pl.}
\end{itemize}
\begin{itemize}
\item {Proveniência:(Do lat. \textunderscore bodio\textunderscore )}
\end{itemize}
Gênero de peixes acanthopterýgios, da fam. das percas.
\section{Bodivo}
\begin{itemize}
\item {Grp. gram.:m.}
\end{itemize}
\begin{itemize}
\item {Utilização:Ant.}
\end{itemize}
\begin{itemize}
\item {Proveniência:(Do lat. \textunderscore votivus\textunderscore )}
\end{itemize}
Offerta, que se fazia aos párochos, na celebração de um entêrro.
\section{Bodo}
\begin{itemize}
\item {fónica:bô}
\end{itemize}
\begin{itemize}
\item {Grp. gram.:m.}
\end{itemize}
\begin{itemize}
\item {Utilização:Ant.}
\end{itemize}
\begin{itemize}
\item {Proveniência:(Lat. \textunderscore votum\textunderscore )}
\end{itemize}
Festa, em que se distribuem alimentos, ou alimentos e dinheiro, aos pobres.
Banquete, que se dava nas igrejas, em certas solennidades.
\section{Bodocada}
\begin{itemize}
\item {Grp. gram.:f.}
\end{itemize}
\begin{itemize}
\item {Utilização:Bras}
\end{itemize}
Acto ou effeito de atirar o bodoque.
\section{Bodoque}
\begin{itemize}
\item {Grp. gram.:m.}
\end{itemize}
\begin{itemize}
\item {Utilização:Ant.}
\end{itemize}
\begin{itemize}
\item {Utilização:Bras}
\end{itemize}
Bóla de barro, que se atirava com bésta.
Arco, para atirar frechas ou bólas de barro.
(Ár. \textunderscore bondoque\textunderscore )
\section{Bodoqueiro}
\begin{itemize}
\item {Grp. gram.:m.}
\end{itemize}
\begin{itemize}
\item {Utilização:Bras}
\end{itemize}
Atirador de bodoque.
\section{Bodoso}
\begin{itemize}
\item {Grp. gram.:adj.}
\end{itemize}
\begin{itemize}
\item {Utilização:Bras}
\end{itemize}
\begin{itemize}
\item {Proveniência:(De \textunderscore bode\textunderscore )}
\end{itemize}
Sujo, immundo.
\section{Bodrelho}
\begin{itemize}
\item {fónica:drê}
\end{itemize}
\begin{itemize}
\item {Grp. gram.:m.}
\end{itemize}
\begin{itemize}
\item {Utilização:Prov.}
\end{itemize}
\begin{itemize}
\item {Utilização:trasm.}
\end{itemize}
Calhau, rebo.
Pedra miúda.
(Por \textunderscore pedrelho\textunderscore , de \textunderscore pedra\textunderscore ?)
\section{Bodum}
\begin{itemize}
\item {Grp. gram.:m.}
\end{itemize}
\begin{itemize}
\item {Proveniência:(De \textunderscore bode\textunderscore )}
\end{itemize}
Cheiro característico dos bodes não castrados.
Transpiração de alguém, mal cheirosa.
Cheiro e sabor do sebo na carne do carneiro.
Máu cheiro.
\section{Boeira}
\begin{itemize}
\item {Grp. gram.:adj. f.}
\end{itemize}
\begin{itemize}
\item {Utilização:Des.}
\end{itemize}
\begin{itemize}
\item {Proveniência:(Do lat. \textunderscore boaria\textunderscore )}
\end{itemize}
Diz-se da estrêlla da manhan.
O mesmo que \textunderscore boieira\textunderscore .
\section{Boeiro}
\begin{itemize}
\item {Grp. gram.:m.}
\end{itemize}
\begin{itemize}
\item {Utilização:Mad}
\end{itemize}
Ave, o mesmo que \textunderscore patagarro\textunderscore .
\section{Boer}
\begin{itemize}
\item {fónica:bu}
\end{itemize}
\begin{itemize}
\item {Grp. gram.:m.}
\end{itemize}
Habitante do Transvaal.
\section{Bões}
\begin{itemize}
\item {Grp. gram.:m. pl.}
\end{itemize}
\begin{itemize}
\item {Utilização:Ant.}
\end{itemize}
\begin{itemize}
\item {Proveniência:(T. ind.)}
\end{itemize}
Balisas, marcos.
\section{Boêta}
\begin{itemize}
\item {Grp. gram.:f.}
\end{itemize}
\begin{itemize}
\item {Utilização:Ant.}
\end{itemize}
\begin{itemize}
\item {Proveniência:(Fr. \textunderscore boîte\textunderscore )}
\end{itemize}
O mesmo que \textunderscore boceta\textunderscore .
\section{Bofá}
\begin{itemize}
\item {Grp. gram.:adv.}
\end{itemize}
\begin{itemize}
\item {Utilização:Ant.}
\end{itemize}
O mesmo que \textunderscore bofé\textunderscore .
\section{Bofada}
\begin{itemize}
\item {Grp. gram.:f.}
\end{itemize}
\begin{itemize}
\item {Utilização:Bras. do N}
\end{itemize}
Bofetada.
\section{Bofar}
\begin{itemize}
\item {Grp. gram.:v. i.}
\end{itemize}
\begin{itemize}
\item {Grp. gram.:V. i.}
\end{itemize}
Lançar do bofe.
Golfar.
Sair ás golfadas.
\section{Bofás}
\begin{itemize}
\item {Grp. gram.:adv.}
\end{itemize}
\begin{itemize}
\item {Utilização:Ant.}
\end{itemize}
O mesmo que \textunderscore bofé\textunderscore .
\section{Bofe}
\begin{itemize}
\item {Grp. gram.:m.}
\end{itemize}
\begin{itemize}
\item {Utilização:Fig.}
\end{itemize}
\begin{itemize}
\item {Grp. gram.:Pl.}
\end{itemize}
\begin{itemize}
\item {Utilização:Fig.}
\end{itemize}
Designação vulgar do pulmão.
Índole, carácter.
Fressura dos animaes.
Renda ou pano franzido e tufado, em peças de vestuário.
(Cast. \textunderscore bofe\textunderscore )
\section{Bofé}
\begin{itemize}
\item {Grp. gram.:adv.}
\end{itemize}
\begin{itemize}
\item {Utilização:Ant.}
\end{itemize}
\begin{itemize}
\item {Proveniência:(De \textunderscore bôa\textunderscore ^1 + \textunderscore fé\textunderscore )}
\end{itemize}
Em verdade; francamente.
Á bôa fé.
\section{Bofetá}
\begin{itemize}
\item {Grp. gram.:m.}
\end{itemize}
\begin{itemize}
\item {Utilização:Ant.}
\end{itemize}
Tecido de algodão asiático.
\section{Bofetada}
\begin{itemize}
\item {Grp. gram.:f.}
\end{itemize}
\begin{itemize}
\item {Utilização:Fig.}
\end{itemize}
\begin{itemize}
\item {Proveniência:(De \textunderscore bofete\textunderscore )}
\end{itemize}
Pancada com a mão no rosto.
Insulto, injúria.
\section{Bofetão}
\begin{itemize}
\item {Grp. gram.:m.}
\end{itemize}
\begin{itemize}
\item {Proveniência:(De \textunderscore bofete\textunderscore )}
\end{itemize}
Grande bofetada.
\section{Bofete}
\begin{itemize}
\item {fónica:fê}
\end{itemize}
\begin{itemize}
\item {Grp. gram.:m.}
\end{itemize}
\begin{itemize}
\item {Proveniência:(Fr. ant. \textunderscore buffet\textunderscore )}
\end{itemize}
Pequena bofetada, tabefe.
\section{Bofetear}
\begin{itemize}
\item {Grp. gram.:v. t.}
\end{itemize}
(V.esbofetear)
\section{Bofordo}
\begin{itemize}
\item {Grp. gram.:m.}
\end{itemize}
Espécie de torneio antigo. Cf. Herculano, \textunderscore Lendas\textunderscore  II., 64.
\section{Boga}
\begin{itemize}
\item {Grp. gram.:f.}
\end{itemize}
Peixe esparoide, raiado longitudinalmente.
Peixe cyprinoide, de água doce.
\section{Bogalha}
\begin{itemize}
\item {Grp. gram.:f.}
\end{itemize}
(V. \textunderscore bugalha\textunderscore ^1)
\section{Bogalhal}
\begin{itemize}
\item {Grp. gram.:m.}
\end{itemize}
(V.bugalhal)
\section{Bogalhão}
\begin{itemize}
\item {Grp. gram.:m.}
\end{itemize}
(V.bugalhão)
\section{Bogalhinha}
\begin{itemize}
\item {Grp. gram.:f.}
\end{itemize}
(V.bugalhinha)
\section{Bogalho}
\begin{itemize}
\item {Grp. gram.:m.}
\end{itemize}
(V.bugalho)
\section{Bogalhó}
\begin{itemize}
\item {Grp. gram.:m.}
\end{itemize}
\begin{itemize}
\item {Utilização:Prov.}
\end{itemize}
\begin{itemize}
\item {Utilização:beir.}
\end{itemize}
O mesmo que \textunderscore cicuta\textunderscore .
\section{Boêmia}
\begin{itemize}
\item {Grp. gram.:f.}
\end{itemize}
\begin{itemize}
\item {Utilização:Fig.}
\end{itemize}
\begin{itemize}
\item {Proveniência:(De \textunderscore bohêmio\textunderscore , cigano)}
\end{itemize}
Vadiagem; vida airada.
\section{Boemiamente}
\begin{itemize}
\item {Grp. gram.:adv.}
\end{itemize}
\begin{itemize}
\item {Utilização:Neol.}
\end{itemize}
\begin{itemize}
\item {Proveniência:(De \textunderscore bohêmio\textunderscore )}
\end{itemize}
Á maneira de boêmio; á maneira dos vagabundos.
\section{Boêmico}
\begin{itemize}
\item {Grp. gram.:adj.}
\end{itemize}
Relativo á Boêmia.
\section{Boêmio}
\begin{itemize}
\item {Grp. gram.:adj.}
\end{itemize}
\begin{itemize}
\item {Grp. gram.:M.}
\end{itemize}
Relativo á Boêmia.
Habitante da Boêmia.
Dialecto dos boêmios.
Espécie de capa antiga.
\section{Boêmio}
\begin{itemize}
\item {Grp. gram.:m.}
\end{itemize}
\begin{itemize}
\item {Proveniência:(Do fr. \textunderscore bohémien\textunderscore )}
\end{itemize}
O mesmo que \textunderscore cigano\textunderscore ^1.
Valdevinos, estroina.
\section{Boganga}
\begin{itemize}
\item {Grp. gram.:f.}
\end{itemize}
Espécie de abóbora, (\textunderscore cucurbita melanosperma\textunderscore , Braun).
\section{Bogar}
\begin{itemize}
\item {Grp. gram.:v. i.}
\end{itemize}
\begin{itemize}
\item {Utilização:Prov.}
\end{itemize}
\begin{itemize}
\item {Utilização:trasm.}
\end{itemize}
Importar, valer: \textunderscore mas que boga isso\textunderscore ?
(Por \textunderscore vogar\textunderscore ?)
\section{Bogardo}
\begin{itemize}
\item {Grp. gram.:m.}
\end{itemize}
\begin{itemize}
\item {Proveniência:(De \textunderscore boga\textunderscore )}
\end{itemize}
Pequeno peixe de água doce, de barbatanas avermelhadas.
\section{Bogari}
\begin{itemize}
\item {Grp. gram.:m.}
\end{itemize}
O mesmo que \textunderscore mogorim\textunderscore .
\section{Bogarim}
\begin{itemize}
\item {Grp. gram.:m.}
\end{itemize}
O mesmo que \textunderscore mogorim\textunderscore .
\section{Bogaxo}
\begin{itemize}
\item {Grp. gram.:m.}
\end{itemize}
\begin{itemize}
\item {Utilização:Prov.}
\end{itemize}
\begin{itemize}
\item {Utilização:beir.}
\end{itemize}
Pequeno novelo.
\section{Bogó}
\begin{itemize}
\item {Grp. gram.:m.}
\end{itemize}
\begin{itemize}
\item {Utilização:Bras}
\end{itemize}
Vasilha, com que se tira água dos poços.
\section{Bogueira}
\begin{itemize}
\item {Grp. gram.:f.}
\end{itemize}
Cova, onde se recolhem as bogas.
\section{Bogueiro}
\begin{itemize}
\item {Grp. gram.:m.}
\end{itemize}
\begin{itemize}
\item {Proveniência:(De \textunderscore boga\textunderscore )}
\end{itemize}
Rede, para apanhar bogas e outros peixes miúdos.
\section{Bohêmia}
\begin{itemize}
\item {Grp. gram.:f.}
\end{itemize}
\begin{itemize}
\item {Utilização:Fig.}
\end{itemize}
\begin{itemize}
\item {Proveniência:(De \textunderscore bohêmio\textunderscore , cigano)}
\end{itemize}
Vadiagem; vida airada.
\section{Bohemiamente}
\begin{itemize}
\item {Grp. gram.:adv.}
\end{itemize}
\begin{itemize}
\item {Utilização:Neol.}
\end{itemize}
\begin{itemize}
\item {Proveniência:(De \textunderscore bohêmio\textunderscore )}
\end{itemize}
Á maneira de bohêmio; á maneira dos vagabundos.
\section{Bohêmico}
\begin{itemize}
\item {Grp. gram.:adj.}
\end{itemize}
Relativo á Bohêmia.
\section{Bohêmio}
\begin{itemize}
\item {Grp. gram.:adj.}
\end{itemize}
\begin{itemize}
\item {Grp. gram.:M.}
\end{itemize}
Relativo á Bohêmia.
Habitante da Bohêmia.
Dialecto dos bohêmios.
Espécie de capa antiga.
\section{Bohêmio}
\begin{itemize}
\item {Grp. gram.:m.}
\end{itemize}
\begin{itemize}
\item {Proveniência:(Do fr. \textunderscore bohémien\textunderscore )}
\end{itemize}
O mesmo que \textunderscore cigano\textunderscore ^1.
Valdevinos, estroina.
\section{Bói}
\begin{itemize}
\item {Grp. gram.:m.}
\end{itemize}
\begin{itemize}
\item {Utilização:T. da Índia Port}
\end{itemize}
Criado, serviçal.
(Conc. \textunderscore bhuy\textunderscore )
\section{Boi}
\begin{itemize}
\item {Grp. gram.:m.}
\end{itemize}
\begin{itemize}
\item {Utilização:Fig.}
\end{itemize}
\begin{itemize}
\item {Proveniência:(Do lat. \textunderscore bos\textunderscore , \textunderscore bovis\textunderscore )}
\end{itemize}
Espécie de ruminante, da fam. dos bovídeos, destinado principalmente a serviços de lavoira e carga, e á alimentação do homem.
\textunderscore Pé de boi.\textunderscore  Pessôa grave, aferrada a costumes antigos.
\textunderscore Ôlho de boi\textunderscore , janela redonda, clarabóia.
\section{Bóia}
\begin{itemize}
\item {Grp. gram.:f.}
\end{itemize}
\begin{itemize}
\item {Utilização:Bras}
\end{itemize}
\begin{itemize}
\item {Utilização:Bras. do N}
\end{itemize}
Corpo fluctuante, ligado por uma corrente, para indicar o lugar desta no ancoradoiro.
Pedaço de cortiça, adaptado ás redes, para que estas se não afundam.
Cortiça, ligada á corda em que se apoiam os que aprendem a nadar.
Etapa de soldados.
Qualquer refeição.
(Cp. lat. \textunderscore bojae\textunderscore )
\section{Bóia}
\begin{itemize}
\item {Grp. gram.:m.}
\end{itemize}
Portador de machila, na Índia; o mesmo que \textunderscore bói\textunderscore .
\section{Bóia}
\begin{itemize}
\item {Grp. gram.:f.}
\end{itemize}
\begin{itemize}
\item {Utilização:Gír.}
\end{itemize}
Toicinho.
\section{Bóia-caá}
\begin{itemize}
\item {Grp. gram.:f.}
\end{itemize}
O mesmo que \textunderscore meladinha\textunderscore .
\section{Boiada}
\begin{itemize}
\item {Grp. gram.:f.}
\end{itemize}
Manada de bois.
\section{Boiadeiro}
\begin{itemize}
\item {Grp. gram.:m.}
\end{itemize}
(V.boieiro)
\section{Boiadoiro}
\begin{itemize}
\item {Grp. gram.:m.}
\end{itemize}
\begin{itemize}
\item {Utilização:Bras}
\end{itemize}
\begin{itemize}
\item {Proveniência:(De \textunderscore boiar\textunderscore )}
\end{itemize}
Lugar, onde boiam ou fluctuam as tartarugas, os peixes-bois, os pirarucus, e em que os pescadores os arpôam ou frecham.
\section{Boiante}
\begin{itemize}
\item {Grp. gram.:adj.}
\end{itemize}
\begin{itemize}
\item {Proveniência:(De \textunderscore boiar\textunderscore )}
\end{itemize}
Fluctuante; que bóia.
Que até ao fim da lide conserva a natural braveza, (falando-se do toiro).
\section{Boião}
\begin{itemize}
\item {Grp. gram.:m.}
\end{itemize}
\begin{itemize}
\item {Proveniência:(De \textunderscore bôjo\textunderscore )}
\end{itemize}
Vaso cylíndrico de barro vidrado.
Vaso cylíndrico de lata.
\section{Boiar}
\begin{itemize}
\item {fónica:bôi}
\end{itemize}
\begin{itemize}
\item {Grp. gram.:v. t.}
\end{itemize}
\begin{itemize}
\item {Grp. gram.:V. i.}
\end{itemize}
\begin{itemize}
\item {Utilização:Fig.}
\end{itemize}
\begin{itemize}
\item {Utilização:Gír.}
\end{itemize}
\begin{itemize}
\item {Utilização:Bras}
\end{itemize}
Ligar á boia.
Fluctuar.
Oscillar.
Hesitar.
Estrangular, afogar.
Almoçar ou jantar.
\section{Boibi}
\begin{itemize}
\item {fónica:bói}
\end{itemize}
\begin{itemize}
\item {Grp. gram.:m.}
\end{itemize}
Serpente do Brasil.
\section{Boi-bumba}
\begin{itemize}
\item {Grp. gram.:m.}
\end{itemize}
\begin{itemize}
\item {Utilização:Bras. do N}
\end{itemize}
O mesmo que \textunderscore bumba-meu-boi\textunderscore .
\section{Boiça}
\begin{itemize}
\item {Grp. gram.:f.}
\end{itemize}
\begin{itemize}
\item {Utilização:Prov.}
\end{itemize}
\begin{itemize}
\item {Utilização:minh.}
\end{itemize}
Terreno inculto; terreno, que só cria mato.
Terreno murado, ou delimitado por pedras ou montes de terra, em que se cria mato para várias applicações e pinheiros ou carvalhos.
(Alter. de \textunderscore balça\textunderscore )
\section{Boiçar}
\begin{itemize}
\item {Grp. gram.:v. t.}
\end{itemize}
\begin{itemize}
\item {Proveniência:(De \textunderscore boiça\textunderscore )}
\end{itemize}
Roçar e queimar o mato em terreno para lavoira.
\section{Boiceira}
\begin{itemize}
\item {Grp. gram.:f.}
\end{itemize}
Primeira estôpa, que se tira do linho; tomento.
\section{Boiceiro}
\begin{itemize}
\item {Grp. gram.:m.}
\end{itemize}
\begin{itemize}
\item {Utilização:Açor}
\end{itemize}
Sedeiro, para tirar a baga ao linho.
\section{Boicelado}
\begin{itemize}
\item {Grp. gram.:adj.}
\end{itemize}
O mesmo que [[esboicelado|esboicelar]].
\section{Boicelo}
\begin{itemize}
\item {fónica:cê}
\end{itemize}
\begin{itemize}
\item {Grp. gram.:m.}
\end{itemize}
\begin{itemize}
\item {Utilização:Prov.}
\end{itemize}
\begin{itemize}
\item {Proveniência:(Do lat. \textunderscore bucella\textunderscore , de \textunderscore bucca\textunderscore , bôca)}
\end{itemize}
Falha na bôca de uma panela de barro ou de qualquer vaso da mesma substância.
\section{Boicininga}
\begin{itemize}
\item {Grp. gram.:f.}
\end{itemize}
Cobra venenosa do Brasil, o mesmo que \textunderscore cascavel\textunderscore .
\section{Boicotagem}
\begin{itemize}
\item {Grp. gram.:f.}
\end{itemize}
\begin{itemize}
\item {Utilização:Neol.}
\end{itemize}
Acto ou effeito de \textunderscore boicotar\textunderscore .
\section{Boicotar}
\begin{itemize}
\item {Grp. gram.:v. t.}
\end{itemize}
\begin{itemize}
\item {Utilização:Neol.}
\end{itemize}
O mesmo que \textunderscore boicotear\textunderscore .
\section{Boicote}
\begin{itemize}
\item {Grp. gram.:m.}
\end{itemize}
O mesmo que \textunderscore boicotagem\textunderscore .
\section{Boicotear}
\begin{itemize}
\item {Grp. gram.:v. t.}
\end{itemize}
\begin{itemize}
\item {Utilização:Neol.}
\end{itemize}
\begin{itemize}
\item {Proveniência:(De \textunderscore Boicott\textunderscore , n. p.)}
\end{itemize}
Fazer opposição ou obstáculo aos negócios de (alguém).
Recusar protecção industrial ou commercial a.
\section{Boicuaba}
\begin{itemize}
\item {Grp. gram.:m.}
\end{itemize}
Serpente comestível do Brasil.
\section{Boidana}
\begin{itemize}
\item {Grp. gram.:f.}
\end{itemize}
Erva trepadeira.
\section{Boieira}
\begin{itemize}
\item {Grp. gram.:f.}
\end{itemize}
\begin{itemize}
\item {Grp. gram.:Adj.}
\end{itemize}
\begin{itemize}
\item {Utilização:Prov.}
\end{itemize}
\begin{itemize}
\item {Utilização:minh.}
\end{itemize}
\begin{itemize}
\item {Proveniência:(De \textunderscore boi\textunderscore )}
\end{itemize}
Estrêlla de alva.
Espécie de alvéola.
Mulher, que guarda ou guia bois.
Diz-se da vaca que procura o boi; barroneira.
\section{Boieiro}
\begin{itemize}
\item {Grp. gram.:m.}
\end{itemize}
\begin{itemize}
\item {Utilização:Mad}
\end{itemize}
\begin{itemize}
\item {Grp. gram.:Adj.}
\end{itemize}
\begin{itemize}
\item {Utilização:Prov.}
\end{itemize}
\begin{itemize}
\item {Utilização:alent.}
\end{itemize}
Conductor ou guarda de bois.
Ave, o mesmo que \textunderscore patagarro\textunderscore .
Constellação boreal.
Diz-se do cajado que, em vez de têr arqueada a extremidade superior, a tem em ângulo recto.
\section{Boi-espaço}
\begin{itemize}
\item {Grp. gram.:m.}
\end{itemize}
Boi, com os chifres muito abertos.
\section{Boiga}
\begin{itemize}
\item {Grp. gram.:f.}
\end{itemize}
Cobra africana, também conhecida por \textunderscore baiapua\textunderscore .
\section{Boi-gordo}
\begin{itemize}
\item {Grp. gram.:m.}
\end{itemize}
Planta leguminosa do Brasil.
\section{Bóina}
\begin{itemize}
\item {Grp. gram.:f.}
\end{itemize}
\begin{itemize}
\item {Proveniência:(T. cast.)}
\end{itemize}
Espécie de carapuça chata ou boné sem costura, usado no norte da Espanha.
Boné análogo para crianças.
\section{Boiil}
\begin{itemize}
\item {Grp. gram.:m.}
\end{itemize}
\begin{itemize}
\item {Utilização:T. de Miranda}
\end{itemize}
\begin{itemize}
\item {Proveniência:(Do lat. \textunderscore bovile\textunderscore )}
\end{itemize}
Curral de bois.
\section{Boínha}
\begin{itemize}
\item {Grp. gram.:f.}
\end{itemize}
\begin{itemize}
\item {Utilização:Prov.}
\end{itemize}
\begin{itemize}
\item {Utilização:beir.}
\end{itemize}
Verruga.
(Por \textunderscore bolinha\textunderscore , de \textunderscore bóla\textunderscore )
\section{Boiombo}
\begin{itemize}
\item {Grp. gram.:m.}
\end{itemize}
\begin{itemize}
\item {Utilização:Des.}
\end{itemize}
O mesmo que \textunderscore biombo\textunderscore .
\section{Boiota}
\begin{itemize}
\item {Grp. gram.:m.}
\end{itemize}
\begin{itemize}
\item {Utilização:Bras. de Goiás}
\end{itemize}
\begin{itemize}
\item {Grp. gram.:F.}
\end{itemize}
\begin{itemize}
\item {Utilização:Bras}
\end{itemize}
Mentecapto.
Partes exteriores do apparelho genital, engrossadas por hydrocele; testículos muito desenvolvidos.
\section{Boiqueira}
\begin{itemize}
\item {Grp. gram.:f.}
\end{itemize}
O mesmo que \textunderscore boiquira\textunderscore .
\section{Boiquira}
\begin{itemize}
\item {Grp. gram.:f.}
\end{itemize}
Cobra venenosa da América do Sul.
\section{Boirel}
\begin{itemize}
\item {Grp. gram.:m.}
\end{itemize}
Pequena bóia de cortiça.
\section{Boita}
\begin{itemize}
\item {Grp. gram.:f.}
\end{itemize}
\begin{itemize}
\item {Utilização:Prov.}
\end{itemize}
Massa informe.
Acervo de porcaria.
\section{Boitatá}
\begin{itemize}
\item {Grp. gram.:m.}
\end{itemize}
\begin{itemize}
\item {Utilização:Bras}
\end{itemize}
Fogo fátuo.
Côca, papão.
(Do tupi \textunderscore mbaé\textunderscore  + \textunderscore tatá\textunderscore )
\section{Boiúna}
\begin{itemize}
\item {Grp. gram.:f.}
\end{itemize}
\begin{itemize}
\item {Utilização:Bras. do N}
\end{itemize}
\begin{itemize}
\item {Proveniência:(De \textunderscore boiúno\textunderscore )}
\end{itemize}
Grande cobra preta.
\section{Boiúno}
\begin{itemize}
\item {Grp. gram.:adj.}
\end{itemize}
Relativo a boi; bovino. Cf. \textunderscore Bibl. do G. do Campo\textunderscore , 505.
\section{Boixa}
\begin{itemize}
\item {Grp. gram.:f.}
\end{itemize}
\begin{itemize}
\item {Utilização:Prov.}
\end{itemize}
\begin{itemize}
\item {Utilização:beir.}
\end{itemize}
Mato; terreno inculto.
(Outra fórma de \textunderscore boiça\textunderscore )
\section{Boixeiro}
\begin{itemize}
\item {Grp. gram.:m.}
\end{itemize}
\begin{itemize}
\item {Utilização:Prov.}
\end{itemize}
\begin{itemize}
\item {Utilização:beir.}
\end{itemize}
Espécie de alvião para arrancar boixa ou mato.
\section{Boiz}
\begin{itemize}
\item {Grp. gram.:f.}
\end{itemize}
\begin{itemize}
\item {Utilização:Fig.}
\end{itemize}
Armadilha para pássaros.
Engano, cilada.
\section{Boizana}
\begin{itemize}
\item {Grp. gram.:m.}
\end{itemize}
\begin{itemize}
\item {Utilização:Prov.}
\end{itemize}
\begin{itemize}
\item {Utilização:trasm.}
\end{itemize}
\begin{itemize}
\item {Proveniência:(De \textunderscore boi\textunderscore )}
\end{itemize}
Homem muito nutrido.
Pessôa, que tem vozeirão atroador.
\section{Bojador}
\begin{itemize}
\item {Grp. gram.:m.}
\end{itemize}
\begin{itemize}
\item {Grp. gram.:Adj.}
\end{itemize}
Aquelle que boja.
Que boja.
\section{Bojamento}
\begin{itemize}
\item {Grp. gram.:m.}
\end{itemize}
Effeito de \textunderscore bojar\textunderscore .
\section{Bojante}
\begin{itemize}
\item {Grp. gram.:adj.}
\end{itemize}
\begin{itemize}
\item {Proveniência:(De \textunderscore bojar\textunderscore )}
\end{itemize}
Que faz bôjo.
\section{Bojar}
\begin{itemize}
\item {Grp. gram.:v. t.}
\end{itemize}
\begin{itemize}
\item {Grp. gram.:V. i.}
\end{itemize}
\begin{itemize}
\item {Proveniência:(De \textunderscore bôjo\textunderscore )}
\end{itemize}
Tornar bojudo.
Enfunar.
Fazer sobresair.
Apresentar saliência arredondada.
\section{Bojarda}
\begin{itemize}
\item {Grp. gram.:f.}
\end{itemize}
\begin{itemize}
\item {Proveniência:(De \textunderscore bôjo\textunderscore ? Cp. it. \textunderscore bujiarda\textunderscore )}
\end{itemize}
Espécie de pêra sumarenta e doce.
\section{Bojeço}
\begin{itemize}
\item {fónica:jê}
\end{itemize}
\begin{itemize}
\item {Grp. gram.:m.}
\end{itemize}
\begin{itemize}
\item {Utilização:Prov.}
\end{itemize}
\begin{itemize}
\item {Proveniência:(De \textunderscore bôjo\textunderscore )}
\end{itemize}
Homem gordo, baixo e desajeitado.
\section{Bojega}
\begin{itemize}
\item {Grp. gram.:f.}
\end{itemize}
\begin{itemize}
\item {Utilização:Prov.}
\end{itemize}
\begin{itemize}
\item {Utilização:trasm}
\end{itemize}
\begin{itemize}
\item {Utilização:minh}
\end{itemize}
\begin{itemize}
\item {Utilização:beir.}
\end{itemize}
\begin{itemize}
\item {Proveniência:(Do lat. \textunderscore vesicula\textunderscore )}
\end{itemize}
Empôla nos pés, o mesmo que \textunderscore bejoga\textunderscore .
\section{Bojego}
\begin{itemize}
\item {Grp. gram.:m.}
\end{itemize}
\begin{itemize}
\item {Utilização:Prov.}
\end{itemize}
\begin{itemize}
\item {Utilização:minh.}
\end{itemize}
O mesmo que \textunderscore bojega\textunderscore .
\section{Bôjo}
\begin{itemize}
\item {Grp. gram.:m.}
\end{itemize}
\begin{itemize}
\item {Proveniência:(Do germ. Cp. ingl. \textunderscore bulge\textunderscore )}
\end{itemize}
Saliência convexa.
Barriga grande.
Capacidade (em sentido própr. e fig.): \textunderscore aquelle patife tem bôjo para tudo\textunderscore .
\section{Bojobi}
\begin{itemize}
\item {Grp. gram.:m.}
\end{itemize}
Grande serpente da América.
\section{Bojudo}
\begin{itemize}
\item {Grp. gram.:adj.}
\end{itemize}
Que tem bôjo.
\section{Bóla}
\begin{itemize}
\item {Grp. gram.:f.}
\end{itemize}
\begin{itemize}
\item {Utilização:Fam.}
\end{itemize}
\begin{itemize}
\item {Utilização:Bras}
\end{itemize}
\begin{itemize}
\item {Utilização:Bras. do N}
\end{itemize}
\begin{itemize}
\item {Utilização:Gír.}
\end{itemize}
\begin{itemize}
\item {Grp. gram.:Pl.}
\end{itemize}
\begin{itemize}
\item {Utilização:Bras}
\end{itemize}
\begin{itemize}
\item {Proveniência:(Lat. \textunderscore bulla\textunderscore )}
\end{itemize}
Corpo redondo em todos os sentidos; esphera.
Objecto arredondado.
Cabeça; juízo.
Pessôa baixa e gorda.
O mesmo que \textunderscore rebuçado\textunderscore .
Pequeno tatu, fechado na sua casca, onde se esconde dos caçadores, e que pela sua fórma arredondada se confunde com os seixos do campo.
Melancia.
Rodelas, feitas de pó de carvão, amassado com barro ou bosta de boi, para conservar o calor nos fogareiros.
Arma de aprehensão, para apanhar bois ou cavallos na corrida, e formada de tiras de coiro, presas entre si por uma das extremidades e tendo na outra pedras esphéricas forradas de coiro.
\section{Bôla}
\begin{itemize}
\item {Grp. gram.:f.}
\end{itemize}
\begin{itemize}
\item {Utilização:Fam.}
\end{itemize}
\begin{itemize}
\item {Utilização:Prov.}
\end{itemize}
\begin{itemize}
\item {Utilização:Prov.}
\end{itemize}
\begin{itemize}
\item {Utilização:beir.}
\end{itemize}
\begin{itemize}
\item {Utilização:Prov.}
\end{itemize}
Palmatoada: \textunderscore o mestre deu-me meia dúzia de bôlas\textunderscore .
Pão chato e redondo de milho.
O mesmo que \textunderscore fogaça\textunderscore .
Queijo grande de leite de ovelha.
(Cp. \textunderscore bolo\textunderscore ^1)
\section{Bolacha}
\begin{itemize}
\item {Grp. gram.:f.}
\end{itemize}
\begin{itemize}
\item {Utilização:Fam.}
\end{itemize}
\begin{itemize}
\item {Proveniência:(De \textunderscore bôlo\textunderscore ^1)}
\end{itemize}
Bôlo chato, de farinha, ás vezes com açúcar.
Bofetada: \textunderscore olha que levas duas bolachas\textunderscore .
\section{Bolacheira}
\begin{itemize}
\item {Grp. gram.:f.}
\end{itemize}
Mulher, que vende ou fabrica bolachas.
\section{Bolacheiro}
\begin{itemize}
\item {Grp. gram.:m.}
\end{itemize}
\begin{itemize}
\item {Grp. gram.:Adj.}
\end{itemize}
\begin{itemize}
\item {Utilização:Pop.}
\end{itemize}
Fabricante ou vendedor de bolachas.
Que tem cara larga e gorda.
\section{Bolachudo}
\begin{itemize}
\item {Grp. gram.:adj.}
\end{itemize}
\begin{itemize}
\item {Utilização:Fam.}
\end{itemize}
\begin{itemize}
\item {Proveniência:(De \textunderscore bolacha\textunderscore )}
\end{itemize}
Que tem faces gordas ou rechonchudas.
\section{Bolada}
\begin{itemize}
\item {Grp. gram.:f.}
\end{itemize}
Pancada com bóla.
Parte do canhão, entre a bôca e os munhões.
\section{Bolada}
\begin{itemize}
\item {Grp. gram.:f.}
\end{itemize}
\begin{itemize}
\item {Proveniência:(De \textunderscore bôlo\textunderscore ^2)}
\end{itemize}
Monte de dinheiro, ao jôgo.
Grande somma de dinheiro.
Desfalque.
\section{Bolandas}
\begin{itemize}
\item {Grp. gram.:f. pl.}
\end{itemize}
Azáfama.
Baldões: \textunderscore andar em bolandas\textunderscore .
(Cast. \textunderscore volandas\textunderscore , de \textunderscore volar\textunderscore , voar)
\section{Bolandeira}
\begin{itemize}
\item {Grp. gram.:f.}
\end{itemize}
Grande roda dentada, nos engenhos de açúcar, que trabalha horizontalmente, por impulso do rodete.
(Cp. \textunderscore bolandas\textunderscore )
\section{Bolantim}
\begin{itemize}
\item {Grp. gram.:m.}
\end{itemize}
(V.volatim)
\section{Bolão}
\begin{itemize}
\item {Grp. gram.:m.}
\end{itemize}
Grande bóla.
\section{Bolapé}
\begin{itemize}
\item {fónica:bô}
\end{itemize}
\begin{itemize}
\item {Grp. gram.:m.}
\end{itemize}
\begin{itemize}
\item {Utilização:Bras}
\end{itemize}
Vau, que o cavallo mal póde atravessar sem nadar.
(Cast. \textunderscore volapié\textunderscore )
\section{Bolar}
\begin{itemize}
\item {Grp. gram.:adj.}
\end{itemize}
\begin{itemize}
\item {Proveniência:(De \textunderscore bôlo\textunderscore ^3)}
\end{itemize}
Diz-se da terra argilosa, também chamada bolo-armênio.
\section{Bolar}
\begin{itemize}
\item {Grp. gram.:v. t.  e  i.}
\end{itemize}
Tocar com a bóla; acertar.
\section{Bolarda}
\begin{itemize}
\item {Grp. gram.:f.}
\end{itemize}
\begin{itemize}
\item {Utilização:Prov.}
\end{itemize}
\begin{itemize}
\item {Utilização:trasm.}
\end{itemize}
Bôlha, produzida pela mordedura de trombeteiros ou de outros insectos.
(Provavelmente por \textunderscore bolharda\textunderscore , de \textunderscore bôlha\textunderscore )
\section{Bóla-reversa}
\begin{itemize}
\item {Grp. gram.:f.}
\end{itemize}
\begin{itemize}
\item {Utilização:Prov.}
\end{itemize}
\begin{itemize}
\item {Utilização:beir.}
\end{itemize}
Instrumento de carpinteiro, espécie de plaina.
\section{Bolarmênico}
\begin{itemize}
\item {Grp. gram.:m.}
\end{itemize}
\begin{itemize}
\item {Utilização:Des.}
\end{itemize}
O mesmo que \textunderscore bolo-arménio\textunderscore .
\section{Bólas}
\begin{itemize}
\item {Grp. gram.:m.}
\end{itemize}
\begin{itemize}
\item {Utilização:Pop.}
\end{itemize}
\begin{itemize}
\item {Grp. gram.:Interj.}
\end{itemize}
\begin{itemize}
\item {Proveniência:(De \textunderscore bóla\textunderscore )}
\end{itemize}
Homem inútil, sem valor, estúpido.
(designativa de \textunderscore enfado\textunderscore , \textunderscore desapprovação\textunderscore ): \textunderscore ora bólas\textunderscore !
\section{Bolata}
\begin{itemize}
\item {Grp. gram.:f.}
\end{itemize}
\begin{itemize}
\item {Utilização:Prov.}
\end{itemize}
\begin{itemize}
\item {Utilização:trasm.}
\end{itemize}
Bôla de pão.
Chapada de lama no fato.
\section{Bolatim}
\begin{itemize}
\item {Grp. gram.:m.}
\end{itemize}
(V.volatim)
\section{Bolbar}
\begin{itemize}
\item {Grp. gram.:adj.}
\end{itemize}
O mesmo que \textunderscore bulbar\textunderscore .
\section{Bolbífero}
\begin{itemize}
\item {Grp. gram.:adj.}
\end{itemize}
\begin{itemize}
\item {Proveniência:(Do lat. \textunderscore bulbus\textunderscore  + \textunderscore ferre\textunderscore )}
\end{itemize}
Que dá bolbos.
\section{Bolbiforme}
\begin{itemize}
\item {Grp. gram.:adj.}
\end{itemize}
\begin{itemize}
\item {Proveniência:(Do lat. \textunderscore bulbus\textunderscore  + \textunderscore forma\textunderscore )}
\end{itemize}
Que tem fórma de bolbo.
\section{Bolbilho}
\begin{itemize}
\item {Grp. gram.:m.}
\end{itemize}
Pequeno bolbo.
\section{Bolbíparo}
\begin{itemize}
\item {Grp. gram.:adj.}
\end{itemize}
\begin{itemize}
\item {Proveniência:(Do lat. \textunderscore bulbus\textunderscore  + \textunderscore parere\textunderscore )}
\end{itemize}
Que produz bolbos.
\section{Bolbo}
\begin{itemize}
\item {fónica:bôl}
\end{itemize}
\begin{itemize}
\item {Grp. gram.:m.}
\end{itemize}
\begin{itemize}
\item {Proveniência:(Lat. \textunderscore bulbus\textunderscore )}
\end{itemize}
Raíz tuberculosa de várias plantas, como o jacintho, o narciso, a cebola, a batateira, etc.
\section{Bolboprotuberancial}
\begin{itemize}
\item {Grp. gram.:adj.}
\end{itemize}
\begin{itemize}
\item {Utilização:Anat.}
\end{itemize}
Diz-se da região do systema nervoso, na qual se comprehende o bolbo rachidiano e a protuberância.
\section{Bolboso}
\begin{itemize}
\item {Grp. gram.:adj.}
\end{itemize}
\begin{itemize}
\item {Proveniência:(Lat. \textunderscore bulbosus\textunderscore )}
\end{itemize}
Que tem bolbo.
Que tem fórma de bolbo.
Relativo a bolbo.
\section{Bolçado}
\begin{itemize}
\item {Grp. gram.:m.}
\end{itemize}
\begin{itemize}
\item {Proveniência:(De \textunderscore bolçar\textunderscore )}
\end{itemize}
Leite coalhado, que as crianças de mama bolçam.
\section{Bolcar}
\begin{itemize}
\item {Grp. gram.:v. t.}
\end{itemize}
\begin{itemize}
\item {Utilização:Prov.}
\end{itemize}
\begin{itemize}
\item {Utilização:trasm.}
\end{itemize}
Fazer cair, voltando.
(Cast. \textunderscore volcar\textunderscore )
\section{Bolçar}
\begin{itemize}
\item {Grp. gram.:v. t.}
\end{itemize}
\begin{itemize}
\item {Proveniência:(Do lat. \textunderscore vomitiare\textunderscore )}
\end{itemize}
Lançar fóra, vomitar.
\section{Bôlco}
\begin{itemize}
\item {Grp. gram.:m.}
\end{itemize}
Acto de bolcar.
\section{Boldina}
\begin{itemize}
\item {Grp. gram.:f.}
\end{itemize}
Alcaloide, extrahido do boldo.
\section{Boldo}
\begin{itemize}
\item {Grp. gram.:m.}
\end{itemize}
Árvore monimiácea do Chile, (\textunderscore pneumus boldus\textunderscore ).
\section{Boldrego}
\begin{itemize}
\item {fónica:drê}
\end{itemize}
\begin{itemize}
\item {Grp. gram.:m.}
\end{itemize}
\begin{itemize}
\item {Utilização:Prov.}
\end{itemize}
O mesmo que [[emboldregado|emboldregar-se]].
\section{Boldreguice}
\begin{itemize}
\item {Grp. gram.:f.}
\end{itemize}
\begin{itemize}
\item {Utilização:Prov.}
\end{itemize}
\begin{itemize}
\item {Utilização:trasm.}
\end{itemize}
\begin{itemize}
\item {Proveniência:(De \textunderscore boldrego\textunderscore )}
\end{itemize}
Porcaria.
\section{Boldrié}
\begin{itemize}
\item {Grp. gram.:m.}
\end{itemize}
\begin{itemize}
\item {Proveniência:(Fr. \textunderscore baudrier\textunderscore )}
\end{itemize}
Cinturão.
Correia a tiracollo, em que se prende uma arma, ou em que se firma a haste da bandeira.
\section{Boléa}
\begin{itemize}
\item {Grp. gram.:f.}
\end{itemize}
Peça torneada de pau, fixa na lança da carruagem, e á qual se prendem os tirantes.
Maneira de conduzir carruagens, indo o guia montado na bêsta de sella.
Assento do cocheiro.
(Cast. \textunderscore bolea\textunderscore )
\section{Boleado}
\begin{itemize}
\item {Grp. gram.:adj.}
\end{itemize}
\begin{itemize}
\item {Proveniência:(De \textunderscore bolear\textunderscore )}
\end{itemize}
Que tem superfície arredondada.
\section{Boleador}
\begin{itemize}
\item {Grp. gram.:m.}
\end{itemize}
\begin{itemize}
\item {Utilização:Bras}
\end{itemize}
\begin{itemize}
\item {Proveniência:(De \textunderscore bolear\textunderscore )}
\end{itemize}
Homem destro no manejo das bólas, (arma de aprehensão).
\section{Boleamento}
\begin{itemize}
\item {Grp. gram.:m.}
\end{itemize}
Acto ou effeito de \textunderscore bolear\textunderscore ^1.
\section{Bolear}
\begin{itemize}
\item {Grp. gram.:v. t.}
\end{itemize}
\begin{itemize}
\item {Utilização:Bras. do S}
\end{itemize}
\begin{itemize}
\item {Grp. gram.:V. p.}
\end{itemize}
\begin{itemize}
\item {Utilização:Bras. do S}
\end{itemize}
Dar fórma de bóla a; arredondar.
Pear com as bólas (um cavallo).
Deixar-se o cavallo cair com o cavalleiro.
\section{Bolear}
\begin{itemize}
\item {Grp. gram.:v. t.}
\end{itemize}
Conduzir á boleia.
\section{Bole-bole}
\begin{itemize}
\item {Grp. gram.:m.}
\end{itemize}
\begin{itemize}
\item {Proveniência:(De \textunderscore bulir\textunderscore )}
\end{itemize}
Planta gramínea.
\section{Boleco}
\begin{itemize}
\item {fónica:lê}
\end{itemize}
\begin{itemize}
\item {Grp. gram.:adj.}
\end{itemize}
\begin{itemize}
\item {Utilização:T. da Bairrada}
\end{itemize}
Diz-se do fruto arejado ou que amadureceu anormalmente.
\section{Boleeiro}
\begin{itemize}
\item {Grp. gram.:m.}
\end{itemize}
\begin{itemize}
\item {Proveniência:(De \textunderscore boleia\textunderscore )}
\end{itemize}
Aquelle que monta a bêsta de sella, nas carruagens de boleia.
Cocheiro.
\section{Bolego}
\begin{itemize}
\item {fónica:lê}
\end{itemize}
\begin{itemize}
\item {Grp. gram.:m.}
\end{itemize}
\begin{itemize}
\item {Utilização:Prov.}
\end{itemize}
\begin{itemize}
\item {Utilização:alent.}
\end{itemize}
\begin{itemize}
\item {Proveniência:(De \textunderscore bóla\textunderscore )}
\end{itemize}
Calhau rolado.
\section{Boleia}
\begin{itemize}
\item {Grp. gram.:f.}
\end{itemize}
Peça torneada de pau, fixa na lança da carruagem, e á qual se prendem os tirantes.
Maneira de conduzir carruagens, indo o guia montado na bêsta de sella.
Assento do cocheiro.
(Cast. \textunderscore bolea\textunderscore )
\section{Boleima}
\begin{itemize}
\item {Grp. gram.:f.}
\end{itemize}
\begin{itemize}
\item {Grp. gram.:M.  e  f.}
\end{itemize}
\begin{itemize}
\item {Utilização:Pop.}
\end{itemize}
Bôlo grosseiro.
Pessôa idiota, palerma.
\section{Boleio}
\begin{itemize}
\item {Grp. gram.:m.}
\end{itemize}
Acto de \textunderscore bolear\textunderscore ^2.
\section{Boleiro}
\begin{itemize}
\item {Grp. gram.:m.}
\end{itemize}
\begin{itemize}
\item {Utilização:Prov.}
\end{itemize}
\begin{itemize}
\item {Utilização:alent.}
\end{itemize}
\begin{itemize}
\item {Proveniência:(De \textunderscore bóla\textunderscore )}
\end{itemize}
Aquelle que faz ou vende bólas.
\section{Boleno}
\begin{itemize}
\item {Grp. gram.:m.  e  adj.}
\end{itemize}
\begin{itemize}
\item {Utilização:Prov.}
\end{itemize}
\begin{itemize}
\item {Utilização:alent.}
\end{itemize}
Homem mentiroso.
\section{Bolero}
\begin{itemize}
\item {Grp. gram.:m.}
\end{itemize}
\begin{itemize}
\item {Proveniência:(T. cast.)}
\end{itemize}
Dança espanhola.
Música, que acompanha essa dança.
\section{Boleta}
\begin{itemize}
\item {fónica:lê}
\end{itemize}
\begin{itemize}
\item {Grp. gram.:f.}
\end{itemize}
(V.bolota):«\textunderscore os pombos torcazes attrahidos pela boleta\textunderscore ». B. Pato.
\section{Bolete}
\begin{itemize}
\item {fónica:lê}
\end{itemize}
\begin{itemize}
\item {Grp. gram.:m.}
\end{itemize}
Haste de ferro que, fixa na extremidade superior do rodízio da azenha, atravessa o pé do moinho.
\section{Boletim}
\begin{itemize}
\item {Grp. gram.:m.}
\end{itemize}
\begin{itemize}
\item {Proveniência:(Fr. \textunderscore bulletin\textunderscore )}
\end{itemize}
Pequeno escrito noticioso.
Resenha noticiosa de operações militares.
Communicação telegráphica.
Publicação periódica official.
Periódico: \textunderscore Boletim da Sociedade de Geographia\textunderscore .
\section{Boletineiro}
\begin{itemize}
\item {Grp. gram.:m.}
\end{itemize}
Portador ou distribuidor de boletins; distribuidor de telegrammas.
\section{Bolandistas}
\begin{itemize}
\item {Grp. gram.:m. pl.}
\end{itemize}
\begin{itemize}
\item {Proveniência:(De \textunderscore Bolland\textunderscore , n. p.)}
\end{itemize}
Jesuítas que, desde 1643 até 1794, dirigiram a célebre publicação \textunderscore Acta Sanctorum\textunderscore .
\section{Boléo}
\begin{itemize}
\item {Grp. gram.:m.}
\end{itemize}
Quéda, sem consequências graves; trambolhão.
(Cast. \textunderscore boleo\textunderscore )
\section{Boletinista}
\begin{itemize}
\item {Grp. gram.:m.}
\end{itemize}
O mesmo que \textunderscore boletineiro\textunderscore .
Aquelle que escreve boletim ou boletins.
\section{Boleto}
\begin{itemize}
\item {fónica:lê}
\end{itemize}
\begin{itemize}
\item {Grp. gram.:m.}
\end{itemize}
Ordem escrita para que alguém dê alojamento a um ou mais militares.
Êsse alojamento.
(Cp. \textunderscore bilhete\textunderscore )
\section{Boleto}
\begin{itemize}
\item {fónica:lê}
\end{itemize}
\begin{itemize}
\item {Grp. gram.:m.}
\end{itemize}
\begin{itemize}
\item {Proveniência:(Lat. \textunderscore boletus\textunderscore )}
\end{itemize}
Gênero de cogumelos.
\section{Boletra}
\begin{itemize}
\item {fónica:lê}
\end{itemize}
\begin{itemize}
\item {Grp. gram.:f.}
\end{itemize}
\begin{itemize}
\item {Utilização:Prov.}
\end{itemize}
\begin{itemize}
\item {Utilização:beir.}
\end{itemize}
O mesmo que \textunderscore boleta\textunderscore .
\section{Boléu}
\begin{itemize}
\item {Grp. gram.:m.}
\end{itemize}
Quéda, sem consequências graves; trambolhão.
(Cast. \textunderscore boleo\textunderscore )
\section{Bôlha}
\begin{itemize}
\item {Grp. gram.:f.}
\end{itemize}
\begin{itemize}
\item {Utilização:Fam.}
\end{itemize}
\begin{itemize}
\item {Proveniência:(Lat. \textunderscore bulla\textunderscore )}
\end{itemize}
Vesícula sôbre a pelle.
Glóbulo de ar, á superfície dos líquidos em ebulição ou fermentação.
Patetice, pancada.
\section{Bolhaca}
\begin{itemize}
\item {Grp. gram.:f.}
\end{itemize}
\begin{itemize}
\item {Utilização:Prov.}
\end{itemize}
\begin{itemize}
\item {Utilização:trasm.}
\end{itemize}
Galha do carvalho bravo, rodeada de saliências mammilares e terminada em bico.
(Cp. \textunderscore bolhaco\textunderscore )
\section{Bolhaco}
\begin{itemize}
\item {Grp. gram.:m.}
\end{itemize}
\begin{itemize}
\item {Utilização:Prov.}
\end{itemize}
\begin{itemize}
\item {Utilização:trasm.}
\end{itemize}
\begin{itemize}
\item {Utilização:Ext.}
\end{itemize}
\begin{itemize}
\item {Proveniência:(De \textunderscore bôlha\textunderscore ? ou por \textunderscore bolhago\textunderscore , methát. de \textunderscore bogalho\textunderscore ==\textunderscore bugalho\textunderscore ?)}
\end{itemize}
Galha do carvalho bravo, mais redonda que a bolhaca.
Globo (do ôlho)
\section{Bolhão}
\begin{itemize}
\item {Grp. gram.:m.}
\end{itemize}
\begin{itemize}
\item {Utilização:Ant.}
\end{itemize}
\begin{itemize}
\item {Proveniência:(De \textunderscore bôlha\textunderscore )}
\end{itemize}
Borbotão de água.
\section{Bolhar}
\begin{itemize}
\item {Grp. gram.:v. i.}
\end{itemize}
\begin{itemize}
\item {Grp. gram.:V. t.}
\end{itemize}
\begin{itemize}
\item {Proveniência:(Lat. \textunderscore bullare\textunderscore )}
\end{itemize}
Apresentar bôlhas; formar bôlhas.
Fazer sair em borbotões. Cf. Filinto, VI, 196.
\section{Bôlhara}
\begin{itemize}
\item {Grp. gram.:f.}
\end{itemize}
\begin{itemize}
\item {Utilização:Prov.}
\end{itemize}
\begin{itemize}
\item {Utilização:trasm.}
\end{itemize}
Alluvião de terra e pedras, que se desprendem de uma encosta.
\section{Bolhelho}
\begin{itemize}
\item {fónica:lhê}
\end{itemize}
\begin{itemize}
\item {Grp. gram.:m.}
\end{itemize}
Bolo, feito de açúcar, ovos, leite e outras substâncias.
(Por \textunderscore bolelho\textunderscore , de \textunderscore bolo\textunderscore )
\section{Bolhó}
\begin{itemize}
\item {fónica:bo-lho}
\end{itemize}
\begin{itemize}
\item {Grp. gram.:f.}
\end{itemize}
O mesmo que \textunderscore moiçó\textunderscore .
\section{Bolhoso}
\begin{itemize}
\item {Grp. gram.:adj.}
\end{itemize}
Que tem bolhas.
\section{Boliche}
\begin{itemize}
\item {Grp. gram.:m.}
\end{itemize}
\begin{itemize}
\item {Utilização:Bras}
\end{itemize}
\begin{itemize}
\item {Proveniência:(T. cast.)}
\end{itemize}
Pequena taberna; baiúca.
Casa, onde se joga a bóla.
\section{Boliço}
\begin{itemize}
\item {Grp. gram.:m.}
\end{itemize}
\begin{itemize}
\item {Utilização:Ant.}
\end{itemize}
Movimento, azáfama. Cf. G. Viana, \textunderscore Apostilas\textunderscore .
\section{Bólida}
\begin{itemize}
\item {Grp. gram.:f.}
\end{itemize}
\begin{itemize}
\item {Proveniência:(Gr. \textunderscore bolis\textunderscore )}
\end{itemize}
Aerólitho, espécie de meteóro ígneo, que atravessa o espaço.
\section{Bólide}
\begin{itemize}
\item {Grp. gram.:f.}
\end{itemize}
O mesmo ou melhor que \textunderscore bólida\textunderscore .
\section{Bólido}
\begin{itemize}
\item {Grp. gram.:m.}
\end{itemize}
(V.bólida)
\section{Bolina}
\begin{itemize}
\item {Grp. gram.:f.}
\end{itemize}
\begin{itemize}
\item {Utilização:Náut.}
\end{itemize}
\begin{itemize}
\item {Utilização:Bras. do Ceará}
\end{itemize}
\begin{itemize}
\item {Proveniência:(Ingl. \textunderscore bawline\textunderscore )}
\end{itemize}
Cabo, que ala para vante de barlavento de uma vela, a fim de que o vento nella incida melhor.
Tábua, que se colloca na parte média da jangada, para cortar as águas e evitar que a véla descaia para sotavento.
\section{Bolinar}
\begin{itemize}
\item {Grp. gram.:v. i.}
\end{itemize}
Navegar á véla, por fórma a ganhar barlavento.
\section{Bolindro}
\begin{itemize}
\item {Grp. gram.:m.}
\end{itemize}
\begin{itemize}
\item {Utilização:Prov.}
\end{itemize}
\begin{itemize}
\item {Utilização:alg.}
\end{itemize}
O mesmo que \textunderscore belindre\textunderscore .
\section{Bolineiro}
\begin{itemize}
\item {Grp. gram.:adj.}
\end{itemize}
Que navega bem á bolina.
\section{Bolinete}
\begin{itemize}
\item {fónica:nê}
\end{itemize}
\begin{itemize}
\item {Grp. gram.:m.}
\end{itemize}
\begin{itemize}
\item {Proveniência:(De \textunderscore bolina\textunderscore )}
\end{itemize}
Cylindro de madeira, na coberta do navio, e que serve de cabrestante para a manobra.
Bateia.
\section{Bolinhol}
\begin{itemize}
\item {Grp. gram.:m.}
\end{itemize}
\begin{itemize}
\item {Utilização:Prov.}
\end{itemize}
\begin{itemize}
\item {Utilização:minh.}
\end{itemize}
Espécie de pão de ló, coberto de açúcar e quási sempre rectangular.
\section{Bolinholo}
\begin{itemize}
\item {Grp. gram.:m.}
\end{itemize}
Pequeno bolo, frito.
\section{Bolisco}
\begin{itemize}
\item {Grp. gram.:m.}
\end{itemize}
\begin{itemize}
\item {Utilização:Prov.}
\end{itemize}
\begin{itemize}
\item {Proveniência:(De \textunderscore bóla\textunderscore )}
\end{itemize}
Excremento de burro.
\section{Bolívar}
\begin{itemize}
\item {Grp. gram.:m.}
\end{itemize}
Unidade monetária em Venezuela, correspondente a duzentos reis.
\section{Boliviano}
\begin{itemize}
\item {Grp. gram.:adj.}
\end{itemize}
\begin{itemize}
\item {Utilização:Bras}
\end{itemize}
\begin{itemize}
\item {Grp. gram.:M.}
\end{itemize}
Relativo á Bolívia.
Diz-se do cavallo, que anda sem dono.
Habitante da Bolívia.
Somma de dinheiro, correspondente a quinhentos mil reis.
\section{Bollandistas}
\begin{itemize}
\item {Grp. gram.:m. pl.}
\end{itemize}
\begin{itemize}
\item {Proveniência:(De \textunderscore Bolland\textunderscore , n. p.)}
\end{itemize}
Jesuítas que, desde 1643 até 1794, dirigiram a célebre publicação \textunderscore Acta Sanctorum\textunderscore .
\section{Bolo}
\begin{itemize}
\item {fónica:bô}
\end{itemize}
\begin{itemize}
\item {Grp. gram.:m.}
\end{itemize}
\begin{itemize}
\item {Utilização:Fam.}
\end{itemize}
\begin{itemize}
\item {Proveniência:(De \textunderscore bóla\textunderscore )}
\end{itemize}
Massa de farinha, açúcar e outros temperos, geralmente redonda e cozida ou frita.
Palmatoada.
Prestação annual, estabelecida por lei ou costume, e com que os donos de propriedades contribuem para os rendimentos do seu párocho. Cf. \textunderscore Lei de 20 de Júl.\textunderscore  1839, art. 7.^o
\section{Bolo}
\begin{itemize}
\item {fónica:bô}
\end{itemize}
\begin{itemize}
\item {Grp. gram.:m.}
\end{itemize}
\begin{itemize}
\item {Proveniência:(Lat. \textunderscore bolus\textunderscore )}
\end{itemize}
Reunião de dinheiro, formada por entradas, apostas, multas e perdas dos parceiros ao jôgo; bolada; lance.
\section{Bolo}
\begin{itemize}
\item {fónica:bô}
\end{itemize}
\begin{itemize}
\item {Grp. gram.:m.}
\end{itemize}
\begin{itemize}
\item {Proveniência:(Gr. \textunderscore bolos\textunderscore )}
\end{itemize}
Terra argillosa, também conhecida por \textunderscore bolo-armênio\textunderscore  ou \textunderscore bolarmênico\textunderscore , e empregada antigamente como medicamento tónico.
\section{Bolo-arménio}
\begin{itemize}
\item {Grp. gram.:m.}
\end{itemize}
Terra argillosa e untuosa, vermelha ou amarela, que se encontra aos pedaços e é applicada entre os materiaes de doirador; bolarmênico.
\section{Boloirar}
\begin{itemize}
\item {Grp. gram.:v. i.}
\end{itemize}
\begin{itemize}
\item {Utilização:Prov.}
\end{itemize}
\begin{itemize}
\item {Utilização:trasm.}
\end{itemize}
\begin{itemize}
\item {Utilização:minh.}
\end{itemize}
Rolar (uma bóla).
Rolar como bóla.
Rebolar. Cf. Camillo, \textunderscore Noites de Insómn.\textunderscore , II, 35.
\section{Bolonhês}
\begin{itemize}
\item {Grp. gram.:adj.}
\end{itemize}
\begin{itemize}
\item {Grp. gram.:M.}
\end{itemize}
Relativo a Bolonha.
Habitante dessa cidade.
Dialecto de Bolonha.
\section{Boloniense}
\begin{itemize}
\item {Grp. gram.:adj.}
\end{itemize}
O mesmo que \textunderscore bolonhês\textunderscore .
\section{Bolónio}
\begin{itemize}
\item {Grp. gram.:m.  e  adj.}
\end{itemize}
\begin{itemize}
\item {Utilização:Pop.}
\end{itemize}
\begin{itemize}
\item {Proveniência:(De \textunderscore bólas\textunderscore )}
\end{itemize}
Homem simplório, rústico.
\section{Bolôr}
\begin{itemize}
\item {Grp. gram.:m.}
\end{itemize}
\begin{itemize}
\item {Utilização:Fig.}
\end{itemize}
\begin{itemize}
\item {Proveniência:(Do lat. \textunderscore pallor\textunderscore )}
\end{itemize}
Vegetação cryptogâmica, que se fórma nas matérias orgânicas, quando estas entram em decomposição.
Decadência, velhice.
\section{Bolór}
\begin{itemize}
\item {Grp. gram.:m.}
\end{itemize}
\begin{itemize}
\item {Utilização:Fig.}
\end{itemize}
\begin{itemize}
\item {Proveniência:(Do lat. \textunderscore pallor\textunderscore )}
\end{itemize}
Vegetação cryptogâmica, que se fórma nas matérias orgânicas, quando estas entram em decomposição.
Decadência, velhice.
\section{Bolorecer}
\begin{itemize}
\item {Grp. gram.:v. t.}
\end{itemize}
\begin{itemize}
\item {Grp. gram.:V. i.}
\end{itemize}
\begin{itemize}
\item {Proveniência:(De \textunderscore bolor\textunderscore )}
\end{itemize}
Tornar bolorento.
Tornar-se bolorento; abolorecer: \textunderscore o pão de milho abolorece facilmente\textunderscore .
\section{Bolorento}
\begin{itemize}
\item {Grp. gram.:adj.}
\end{itemize}
\begin{itemize}
\item {Utilização:Fig.}
\end{itemize}
Que tem bolor.
Decadente, velho.
\section{Bolota}
\begin{itemize}
\item {Grp. gram.:f.}
\end{itemize}
\begin{itemize}
\item {Grp. gram.:Pl.}
\end{itemize}
\begin{itemize}
\item {Utilização:Pop.}
\end{itemize}
\begin{itemize}
\item {Utilização:Bras. do N}
\end{itemize}
\begin{itemize}
\item {Proveniência:(Do ár. \textunderscore balluta\textunderscore )}
\end{itemize}
Glande do carvalho e do azinheiro.
Casta do uva.
Lama empastada, na orla do vestuário.
O mesmo que \textunderscore borla\textunderscore ^1; qualquer penduricalho.
Flôr da faveira-de-bolota.
\section{Bolotada}
\begin{itemize}
\item {Grp. gram.:f.}
\end{itemize}
Grande porção de bolota.
\section{Bolotal}
\begin{itemize}
\item {Grp. gram.:m.}
\end{itemize}
Mata de árvores que dão bolota.
\section{Bôlsa}
\begin{itemize}
\item {Grp. gram.:f.}
\end{itemize}
\begin{itemize}
\item {Grp. gram.:M.}
\end{itemize}
\begin{itemize}
\item {Grp. gram.:F. pl.}
\end{itemize}
\begin{itemize}
\item {Proveniência:(Do lat. \textunderscore bursa\textunderscore )}
\end{itemize}
Pequeno saco, em que se traz dinheiro.
Qualquer saco pequeno, com cordões que saem da bainha da bôca.
Dinheiro para despesas ordinárias.
Praça de commércio, casa ou sala, em que se juntam commerciantes, corretores, etc., para tratar de negócios financeiros, jôgo de fundos públicos, etc.
Membrana externa de alguns cogumelos.
Thesoireiro, caixa.
Alforges.
Pelle, que envolve os testículos, escroto.
\section{Bolsada}
\begin{itemize}
\item {Grp. gram.:f.}
\end{itemize}
\begin{itemize}
\item {Proveniência:(De \textunderscore bôlsa\textunderscore )}
\end{itemize}
Acervo de minério, com várias fórmas, no lugar em que se produz.
\section{Bôlsa-do-pastor}
\begin{itemize}
\item {Grp. gram.:f.}
\end{itemize}
Planta oxaliácea, (\textunderscore oxalis corniculata\textunderscore , Lin.).
Outro nome da calceolária.
\section{Bolsão}
\begin{itemize}
\item {Grp. gram.:m.}
\end{itemize}
\begin{itemize}
\item {Utilização:Des.}
\end{itemize}
Grande bôlsa. Cf. G. Vicente, I, 222.
\section{Bolsar}
\textunderscore v. t.\textunderscore  (e der.)
(V. \textunderscore bolçar\textunderscore , etc.)
\section{Bolsar}
\begin{itemize}
\item {Grp. gram.:v. i.}
\end{itemize}
\begin{itemize}
\item {Proveniência:(De \textunderscore bôlsa\textunderscore )}
\end{itemize}
Fazer bolsos.
Entufar-se, enrugar-se, (diz-se do vestido que se ajusta mal ao corpo).
\section{Bolseiro}
\begin{itemize}
\item {Grp. gram.:m.}
\end{itemize}
Aquelle que faz bôlsas.
Aquelle que administra a bôlsa ou dinheiro de uma communidade.
\section{Bolselho}
\begin{itemize}
\item {fónica:sê}
\end{itemize}
\begin{itemize}
\item {Grp. gram.:m.}
\end{itemize}
\begin{itemize}
\item {Proveniência:(De \textunderscore bôlso\textunderscore )}
\end{itemize}
Pouco pano, com que se navega, quando há muito vento ou quando se quere andar pouco.
\section{Bolsim}
\begin{itemize}
\item {Grp. gram.:m.}
\end{itemize}
\begin{itemize}
\item {Proveniência:(De \textunderscore bôlsa\textunderscore )}
\end{itemize}
Pequena praça commercial, para compra e venda de fundos públicos.
\section{Bolsinho}
\begin{itemize}
\item {Grp. gram.:m.}
\end{itemize}
\begin{itemize}
\item {Proveniência:(De \textunderscore bôlso\textunderscore )}
\end{itemize}
Dinheiro, destinado a despesas miúdas e particulares de alguém: \textunderscore pagar do seu bolsinho\textunderscore .
\section{Bolsista}
\begin{itemize}
\item {Grp. gram.:adj.}
\end{itemize}
\begin{itemize}
\item {Grp. gram.:M.}
\end{itemize}
\begin{itemize}
\item {Proveniência:(De \textunderscore bôlsa\textunderscore )}
\end{itemize}
Relativo ao jôgo ou movimento de fundos públicos.
Jogador de fundos públicos.
\section{Bôlso}
\begin{itemize}
\item {Grp. gram.:m.}
\end{itemize}
\begin{itemize}
\item {Utilização:Náut.}
\end{itemize}
Algibeira, saquinho preso interiormente a qualquer parte do vestuário, e em que geralmente se metem os objectos por uma abertura exterior.
Bôjo, que a vela faz, quando carregada, mas não ferrada.
\section{Boltênia}
\begin{itemize}
\item {Grp. gram.:f.}
\end{itemize}
Animal tunicado, que se prende ao solo por órgãos análogos ás raizes das plantas.
\section{Boltónia}
\begin{itemize}
\item {Grp. gram.:f.}
\end{itemize}
\begin{itemize}
\item {Proveniência:(De \textunderscore Bolton\textunderscore , n. p.)}
\end{itemize}
Gênero de plantas compostas.
\section{Bom}
\begin{itemize}
\item {Grp. gram.:adj.}
\end{itemize}
\begin{itemize}
\item {Grp. gram.:M.}
\end{itemize}
\begin{itemize}
\item {Grp. gram.:Interj.}
\end{itemize}
\begin{itemize}
\item {Proveniência:(Lat. \textunderscore bonus\textunderscore )}
\end{itemize}
Que possue as qualidades próprias da sua espécie.
Favorável, benigno: \textunderscore bom tempo\textunderscore .
Misericordioso; caritativo.
Proveitoso.
Rigoroso no cumprimento dos seus deveres: \textunderscore bom magistrado\textunderscore .
Próprio.
Perfeito; completo.
Seguro.
Grande; amplo.
Lucrativo: \textunderscore a empresa teve bom resultado\textunderscore .
Homem bom.
Aquillo que é bom: \textunderscore amar o bom e o bello\textunderscore .
(designativa de \textunderscore approvação\textunderscore , \textunderscore surpresa\textunderscore , etc.)
\section{Boma}
\begin{itemize}
\item {Grp. gram.:f.}
\end{itemize}
\begin{itemize}
\item {Utilização:Ant.}
\end{itemize}
Enfeite de pelles, usado ao pescoço por mulheres e que se estendia até os joêlhos.--Modernamente, chamam-lhe \textunderscore bôa\textunderscore .
\section{Boma}
\begin{itemize}
\item {Grp. gram.:f.}
\end{itemize}
Espécie de forte, na África oriental portuguesa.
\section{Bomba}
\begin{itemize}
\item {Grp. gram.:f.}
\end{itemize}
\begin{itemize}
\item {Utilização:Fig.}
\end{itemize}
\begin{itemize}
\item {Utilização:Gír.}
\end{itemize}
\begin{itemize}
\item {Proveniência:(Do lat. \textunderscore bombus\textunderscore , t. onom)}
\end{itemize}
Projéctil, que, contendo matéria explosiva, rebenta com estrépito quando uma espoleta ou mecha communica fogo àquella matéria.
Acontecimento inesperado.
Desastre imprevisto.
Nádegas.
\section{Bomba}
\begin{itemize}
\item {Grp. gram.:f.}
\end{itemize}
\begin{itemize}
\item {Utilização:Constr.}
\end{itemize}
\begin{itemize}
\item {Utilização:T. de Moncorvo}
\end{itemize}
Máquina, que, por meio da compressão do ar, serve para elevar a água.
Tubo recurvado, com que se passam líquidos de uma vasilha para outra.
Apparelho, com que se esgota a água de um navio.
Revestimento de metal, que une as partes principaes de alguns instrumentos de vento.
Instrumento de vidro, com que se extrai o leite do seio das mulheres.
Disco ou êmbolo de metal, que, nos vehículos das linhas férreas, recebe e suaviza o choque recíproco dos vehículos de um combóio.
Espaço central, rodeado por differentes lanços de uma escada.
Batoque de pau.
\section{Bomba}
\begin{itemize}
\item {Grp. gram.:f.}
\end{itemize}
Alçapão num sobrado, por onde se deita palha em mangedoira.
\section{Bombaça}
\begin{itemize}
\item {Grp. gram.:f.}
\end{itemize}
Espécie de chaminé, no Minho e Doiro.
\section{Bombáceas}
\begin{itemize}
\item {Grp. gram.:f. pl.}
\end{itemize}
Família de plantas, que têm por typo o \textunderscore bômbax\textunderscore .
\section{Bombachas}
\begin{itemize}
\item {Grp. gram.:f. pl.}
\end{itemize}
\begin{itemize}
\item {Utilização:Ant.}
\end{itemize}
Calções largos, que se atavam por baixo dos joêlhos.--Ainda são usados por cavalleiros, ao sul do Brasil, os quaes abotôam as bombachas logo acima dos pés.
\section{Bombacho}
\begin{itemize}
\item {Grp. gram.:m.}
\end{itemize}
\begin{itemize}
\item {Proveniência:(De \textunderscore bomba\textunderscore )}
\end{itemize}
Pequena bomba, para tirar ou elevar água.
\section{Bombada}
\begin{itemize}
\item {Grp. gram.:f.}
\end{itemize}
\begin{itemize}
\item {Utilização:Prov.}
\end{itemize}
\begin{itemize}
\item {Utilização:trasm.}
\end{itemize}
\begin{itemize}
\item {Proveniência:(De \textunderscore bomba\textunderscore ^1)}
\end{itemize}
Desgraça súbita.
\section{Bombanaça}
\begin{itemize}
\item {Grp. gram.:f.}
\end{itemize}
O mesmo que \textunderscore bombonaça\textunderscore .
\section{Bombarato}
\begin{itemize}
\item {Grp. gram.:m.}
\end{itemize}
Facilidade.
Aquillo que destrói difficuldades:«\textunderscore de tudo lhe fez bombarato\textunderscore ». Castilho, \textunderscore Fastos\textunderscore , II, 400.
\section{Bombarda}
\begin{itemize}
\item {Grp. gram.:f.}
\end{itemize}
\begin{itemize}
\item {Grp. gram.:Adj.}
\end{itemize}
Antiga máquina de guerra, para arremessar pedras.
Antiga peça de artilharia, que arremessava grandes balas de pedra.
Antiga barcaça chata, que transportava obuzes e morteiros.
Charamela antiga.
Dizia-se da pedra arremessada por peça de artilharia:«\textunderscore ferido na cabeça de uma racha de pedra bombarda\textunderscore ». Barros, \textunderscore Déc.\textunderscore 
(B. lat. \textunderscore bombarda\textunderscore , do lat. \textunderscore bombus\textunderscore )
\section{Bombardada}
\begin{itemize}
\item {Grp. gram.:f.}
\end{itemize}
Tiro de bombarda.
\section{Bombardão}
\begin{itemize}
\item {Grp. gram.:m.}
\end{itemize}
\begin{itemize}
\item {Utilização:Bras}
\end{itemize}
Instrumento de sopro, o mesmo que \textunderscore baixo\textunderscore .
\section{Bombardar}
\textunderscore v. t.\textunderscore  (e der.)
(V. \textunderscore bombardear\textunderscore , etc.)
\section{Bombardaria}
\begin{itemize}
\item {Grp. gram.:f.}
\end{itemize}
Conjunto de bombardas.
\section{Bombardeamento}
\begin{itemize}
\item {Grp. gram.:m.}
\end{itemize}
Acto de \textunderscore bombardear\textunderscore .
\section{Bombardear}
\begin{itemize}
\item {Grp. gram.:v. t.}
\end{itemize}
Atacar com tiros de bombarda ou com projécteis de artilharia.
\section{Bombardeio}
\begin{itemize}
\item {Grp. gram.:m.}
\end{itemize}
O mesmo que \textunderscore bombardeamento\textunderscore .
\section{Bombardeira}
\begin{itemize}
\item {Grp. gram.:f.}
\end{itemize}
\begin{itemize}
\item {Utilização:Bot.}
\end{itemize}
Cada um dos intervallos que separam as ameias, e nos quaes se collocava a parte anterior da bombarda.
Canhoneira.
Barcaça para transporte de bombardas.
Navio armado de artilharia.
Planta cucurbitácea de Cabo-Verde.
\section{Bombardeiro}
\begin{itemize}
\item {Grp. gram.:adj.}
\end{itemize}
\begin{itemize}
\item {Grp. gram.:M.}
\end{itemize}
Relativo a bombarda.
Soldado, que assesta e faz disparar a bombarda.
Marinheiro, que guiava a bombarda (barcaça).
Pequeno quadrúpede africano que, ao sêr perseguido, expelle na fuga os excrementos contra os perseguidores.
\section{Bombardeta}
\begin{itemize}
\item {fónica:dê}
\end{itemize}
\begin{itemize}
\item {Grp. gram.:f.}
\end{itemize}
Bombarda pequena.
\section{Bombardino}
\begin{itemize}
\item {Grp. gram.:m.}
\end{itemize}
Instrumento, espécie de trompa; barýtono.
\section{Bombarqueiro}
\begin{itemize}
\item {Grp. gram.:m.}
\end{itemize}
O mesmo que \textunderscore dom-barqueiro\textunderscore .
\section{Bombástico}
\begin{itemize}
\item {Grp. gram.:adj.}
\end{itemize}
\begin{itemize}
\item {Proveniência:(De \textunderscore bomba\textunderscore ^1)}
\end{itemize}
Estrondoso.
Extravagante.
Empolado; altisonante; pretensioso: \textunderscore estilo bombástico\textunderscore .
Baroco.
\section{Bombatul}
\begin{itemize}
\item {Grp. gram.:m.}
\end{itemize}
Árvore da Guiné, de raízes medicinaes.
\section{Bômbax}
\begin{itemize}
\item {Grp. gram.:m.}
\end{itemize}
Gênero de plantas, que produzem filamentos leves, finos e curtos, como a sumaúma.
\section{Bombazina}
\begin{itemize}
\item {Grp. gram.:f.}
\end{itemize}
\begin{itemize}
\item {Proveniência:(Do b. lat. \textunderscore bombacinium\textunderscore )}
\end{itemize}
Antigo tecido de seda.
Tecido riscado, de algodão, a imitar velludo.
\section{Bombeado}
\begin{itemize}
\item {Grp. gram.:adj.}
\end{itemize}
\begin{itemize}
\item {Proveniência:(De \textunderscore bombear\textunderscore )}
\end{itemize}
O mesmo que \textunderscore boleado\textunderscore .
\section{Bombeador}
\begin{itemize}
\item {Grp. gram.:m.}
\end{itemize}
\begin{itemize}
\item {Utilização:Bras}
\end{itemize}
\begin{itemize}
\item {Proveniência:(De \textunderscore bombear\textunderscore ^2)}
\end{itemize}
Aquelle que bombeia ou espiona o campo inimigo.
\section{Bombeamento}
\begin{itemize}
\item {Grp. gram.:m.}
\end{itemize}
Acto de bombear^1.
\section{Bombear}
\begin{itemize}
\item {Grp. gram.:v. t.}
\end{itemize}
\begin{itemize}
\item {Proveniência:(De \textunderscore bomba\textunderscore ^1)}
\end{itemize}
Bombardear.
Bolear, dar fórma redonda a.
\section{Bombear}
\begin{itemize}
\item {Grp. gram.:v. t.}
\end{itemize}
\begin{itemize}
\item {Utilização:Bras}
\end{itemize}
\begin{itemize}
\item {Proveniência:(De \textunderscore bombeiro\textunderscore ^2)}
\end{itemize}
Espionar (o campo inimigo), para lhe conhecer a fôrça, os recursos ou os planos.
\section{Bom-bedro}
\begin{itemize}
\item {Grp. gram.:f.}
\end{itemize}
O mesmo que \textunderscore bom-vedro\textunderscore .
\section{Bombeiro}
\begin{itemize}
\item {Grp. gram.:m.}
\end{itemize}
\begin{itemize}
\item {Utilização:Marn.}
\end{itemize}
\begin{itemize}
\item {Proveniência:(De \textunderscore bomba\textunderscore )}
\end{itemize}
Artilheiro, que fazia os tiros de bomba.
Aquelle que trabalha com bombas de incêndio.
Espécie de tabuleiro comprido, com um cabo e um pau roliço, atravessado a meio, que passa por dois buracos lateraes.
\section{Bombeiro}
\begin{itemize}
\item {Grp. gram.:m.}
\end{itemize}
\begin{itemize}
\item {Utilização:Bras}
\end{itemize}
Espião ou explorador do campo inimigo; indagador.
(Alter. de \textunderscore pombeiro\textunderscore ^2)
\section{Bombicar}
\begin{itemize}
\item {Grp. gram.:v. i.}
\end{itemize}
\begin{itemize}
\item {Utilização:T. de Angola}
\end{itemize}
Trabalhar no amanho ou arranjo de alguma coisa. Cf. Capello e Ivens, I, 331.
\section{Bombilim}
\begin{itemize}
\item {Grp. gram.:m.}
\end{itemize}
Peixe da costa de Dio.
\section{Bombilho}
\begin{itemize}
\item {Grp. gram.:m.}
\end{itemize}
Pequeno bombo, usado por sociedades de bandurristas.
\section{Bombillo}
\begin{itemize}
\item {Grp. gram.:m.}
\end{itemize}
Espécie de môsca ou tavão.
\section{Bombilo}
\begin{itemize}
\item {Grp. gram.:m.}
\end{itemize}
Espécie de môsca ou tavão.
\section{Bombo}
\begin{itemize}
\item {Grp. gram.:m.}
\end{itemize}
\begin{itemize}
\item {Proveniência:(T. onom. Cp. lat. \textunderscore bombus\textunderscore )}
\end{itemize}
Grande tambor; zabumba.
\section{Bombó}
\begin{itemize}
\item {Grp. gram.:m.}
\end{itemize}
\begin{itemize}
\item {Utilização:T. de Angola}
\end{itemize}
Tubérculo da mandióca, fermentado e enxuto, que, pisado num pilão, produz a farinha de que se faz o infúndi.
\section{Bom-bocado}
\begin{itemize}
\item {Grp. gram.:m.}
\end{itemize}
Variedade de doce, feito de açúcar, amêndoas pisadas, gemmas de ovos e chila.
\section{Bombolim}
\begin{itemize}
\item {Grp. gram.:m.}
\end{itemize}
O mesmo que \textunderscore bombilim\textunderscore .
\section{Bombolo}
\begin{itemize}
\item {Grp. gram.:m.}
\end{itemize}
Árvore meliácea de Angola, (\textunderscore melia bombolo\textunderscore ).
\section{Bombonaça}
\begin{itemize}
\item {Grp. gram.:f.}
\end{itemize}
Espécie de palmeira americana.
Fibra têxtil dessa palmeira, de que se fazem os chamados chapéus do Chile.
\section{Bômbice}
\begin{itemize}
\item {Grp. gram.:m.}
\end{itemize}
\begin{itemize}
\item {Proveniência:(Gr. \textunderscore bombux\textunderscore )}
\end{itemize}
Nome scientífico do bicho da seda.
\section{Bombícico}
\begin{itemize}
\item {Grp. gram.:adj.}
\end{itemize}
\begin{itemize}
\item {Proveniência:(De \textunderscore bômbyce\textunderscore )}
\end{itemize}
Diz-se de um ácido, que se achou no líquido contido na crisálida do bicho da seda.
\section{Bombicíneos}
\begin{itemize}
\item {Grp. gram.:m. pl.}
\end{itemize}
\begin{itemize}
\item {Proveniência:(De \textunderscore bômbyx\textunderscore )}
\end{itemize}
Família de insectos lepidópteros nocturnos.
\section{Bombílios}
\begin{itemize}
\item {Grp. gram.:m. pl.}
\end{itemize}
\begin{itemize}
\item {Proveniência:(Gr. \textunderscore bombulios\textunderscore )}
\end{itemize}
Gênero de insectos dípteros.
\section{Bombordo}
\begin{itemize}
\item {fónica:bôr}
\end{itemize}
\begin{itemize}
\item {Grp. gram.:m.}
\end{itemize}
Lado esquerdo de um navio, olhando-se da popa á proa.
\section{Bomboteiro}
\begin{itemize}
\item {Grp. gram.:m.}
\end{itemize}
\begin{itemize}
\item {Utilização:Mad}
\end{itemize}
\begin{itemize}
\item {Proveniência:(Do ingl. \textunderscore bumboat\textunderscore )}
\end{itemize}
Homem, que a bordo se emprega na venda de productos da Ilha da Madeira.
\section{Bômbyce}
\begin{itemize}
\item {Grp. gram.:m.}
\end{itemize}
\begin{itemize}
\item {Proveniência:(Gr. \textunderscore bombux\textunderscore )}
\end{itemize}
Nome scientífico do bicho da seda.
\section{Bombýcico}
\begin{itemize}
\item {Grp. gram.:adj.}
\end{itemize}
\begin{itemize}
\item {Proveniência:(De \textunderscore bômbyce\textunderscore )}
\end{itemize}
Diz-se de um ácido, que se achou no líquido contido na chrysállida do bicho da seda.
\section{Bombycíneos}
\begin{itemize}
\item {Grp. gram.:m. pl.}
\end{itemize}
\begin{itemize}
\item {Proveniência:(De \textunderscore bômbyx\textunderscore )}
\end{itemize}
Família de insectos lepidópteros nocturnos.
\section{Bombýlios}
\begin{itemize}
\item {Grp. gram.:m. pl.}
\end{itemize}
\begin{itemize}
\item {Proveniência:(Gr. \textunderscore bombulios\textunderscore )}
\end{itemize}
Gênero de insectos dípteros.
\section{Bômbyx}
\begin{itemize}
\item {Grp. gram.:m.}
\end{itemize}
(V.bômbice)
\section{Bomeria}
\begin{itemize}
\item {Grp. gram.:f.}
\end{itemize}
\begin{itemize}
\item {Utilização:Jur.}
\end{itemize}
O mesmo que \textunderscore bodemeria\textunderscore .
\section{Bom-nome}
\begin{itemize}
\item {Grp. gram.:m.}
\end{itemize}
Árvore silvestre do Brasil.
\section{Bomóloco}
\begin{itemize}
\item {Grp. gram.:adj.}
\end{itemize}
\begin{itemize}
\item {Utilização:Ant.}
\end{itemize}
Rude; chocarreiro.
\section{Bomoro}
\begin{itemize}
\item {Grp. gram.:m.}
\end{itemize}
Insecto, que ataca as palmeiras, ferindo-as na espadice e no caule.
\section{Bom-pastor}
\begin{itemize}
\item {Grp. gram.:m.}
\end{itemize}
Planta da serra de Cintra.
\section{Bom-serás}
\begin{itemize}
\item {Grp. gram.:m.}
\end{itemize}
\begin{itemize}
\item {Utilização:Fam.}
\end{itemize}
Homem bom, simplório. Cf. J. Dinís, \textunderscore Fidalgos\textunderscore , II, 24.
\section{Bom-vedro}
\begin{itemize}
\item {Grp. gram.:m.}
\end{itemize}
Espécie de uva alentejana, extremenha e algarvia.
\section{Bona}
\begin{itemize}
\item {Grp. gram.:f.}
\end{itemize}
\begin{itemize}
\item {Utilização:Ant.}
\end{itemize}
\begin{itemize}
\item {Proveniência:(Lat. \textunderscore bona\textunderscore , pl. de \textunderscore bonus\textunderscore )}
\end{itemize}
Fazenda, bens.
\section{Bonabóia}
\begin{itemize}
\item {Grp. gram.:m.}
\end{itemize}
O mesmo ou melhor que \textunderscore banabóia\textunderscore .
\section{Bonachão}
\begin{itemize}
\item {Grp. gram.:adj.}
\end{itemize}
O mesmo que \textunderscore bonacheirão\textunderscore .
\section{Bonacheirão}
\begin{itemize}
\item {Grp. gram.:adj.}
\end{itemize}
\begin{itemize}
\item {Proveniência:(Do lat. \textunderscore bonus\textunderscore )}
\end{itemize}
Que tem bondade e que é simples, ingênuo, paciente.
\section{Bonacheirice}
\begin{itemize}
\item {Grp. gram.:f.}
\end{itemize}
Qualidade de bonacheiro.
\section{Bonacheiro}
\begin{itemize}
\item {Grp. gram.:adj.}
\end{itemize}
(V.bonacheirão)
\section{Bona-chira}
\begin{itemize}
\item {Grp. gram.:f.}
\end{itemize}
\begin{itemize}
\item {Proveniência:(Fr. \textunderscore bonne-chère\textunderscore )}
\end{itemize}
Bôa mesa, mesa farta, regalada. Cf. Castilho, \textunderscore Fausto\textunderscore , 12.
\section{Bonacho}
\begin{itemize}
\item {Grp. gram.:m.}
\end{itemize}
O mesmo que \textunderscore bisonte\textunderscore .
\section{Bonança}
\begin{itemize}
\item {Grp. gram.:f.}
\end{itemize}
Bom tempo no mar, depois de tempestade.
Serenidade; sossêgo.
\textunderscore Tempo bonança\textunderscore , tempo bonançoso. Cf. \textunderscore Peregrinação\textunderscore , (\textunderscore passim\textunderscore ).
\section{Bonançar}
\begin{itemize}
\item {Grp. gram.:v. t.  e  i.}
\end{itemize}
(V.abonançar)
\section{Bonançoso}
\begin{itemize}
\item {Grp. gram.:adj.}
\end{itemize}
\begin{itemize}
\item {Proveniência:(De \textunderscore bonança\textunderscore )}
\end{itemize}
Que abonançou.
Que está tranquillo; sossegado.
\section{Bonapártea}
\begin{itemize}
\item {Grp. gram.:f.}
\end{itemize}
\begin{itemize}
\item {Proveniência:(De \textunderscore Bonaparte\textunderscore , n. p.)}
\end{itemize}
Gênero de plantas bromeliáceas.
\section{Bonapartismo}
\begin{itemize}
\item {Grp. gram.:m.}
\end{itemize}
Systema político de Napoleão Bonaparte ou dos seus partidários.
\section{Bonapartista}
\begin{itemize}
\item {Grp. gram.:m.}
\end{itemize}
\begin{itemize}
\item {Grp. gram.:Adj.}
\end{itemize}
Partidário de Bonaparte.
Relativo a Bonaparte.
\section{Bonavéria}
\begin{itemize}
\item {Grp. gram.:f.}
\end{itemize}
Gênero de plantas leguminosas.
\section{Bond}
\begin{itemize}
\item {Grp. gram.:m.}
\end{itemize}
(V.bonde)
\section{Bonda!}
\begin{itemize}
\item {Grp. gram.:interj.}
\end{itemize}
\begin{itemize}
\item {Utilização:Prov.}
\end{itemize}
\begin{itemize}
\item {Proveniência:(De \textunderscore bondar\textunderscore )}
\end{itemize}
Basta!
\section{Bonda}
\begin{itemize}
\item {Grp. gram.:f.}
\end{itemize}
Árvore africana.
\section{Bondade}
\begin{itemize}
\item {Grp. gram.:f.}
\end{itemize}
\begin{itemize}
\item {Proveniência:(Do lat. \textunderscore bonitas\textunderscore )}
\end{itemize}
Qualidade do que é bom.
Bôa índole.
Brandura, benevolência.
\section{Bondadoso}
\begin{itemize}
\item {Grp. gram.:adj.}
\end{itemize}
\begin{itemize}
\item {Proveniência:(De \textunderscore bondade\textunderscore )}
\end{itemize}
O mesmo que \textunderscore bondoso\textunderscore :«\textunderscore sob império moderado e bondadoso\textunderscore ». Filinto, \textunderscore D. Man.\textunderscore , II, 5.«\textunderscore ...a tão bondadoso e terno amigo\textunderscore ». Garrett, \textunderscore Camões\textunderscore .
\section{Bondar}
\begin{itemize}
\item {Grp. gram.:v. i.}
\end{itemize}
\begin{itemize}
\item {Utilização:Prov.}
\end{itemize}
Sêr bastante, sufficiente: \textunderscore mas isso não bonda\textunderscore .
(Alter. de \textunderscore abundar\textunderscore )
\section{Bondara}
\begin{itemize}
\item {Grp. gram.:f.}
\end{itemize}
Árvore de Damão, (\textunderscore lagerstroemia microcarpa\textunderscore ), o mesmo que a \textunderscore benteca\textunderscore  da Guiné.
\section{Bondará}
\begin{itemize}
\item {Grp. gram.:m.}
\end{itemize}
Árvore indiana.
O mesmo que \textunderscore bondara\textunderscore ?
\section{Bonde}
\begin{itemize}
\item {Grp. gram.:m.}
\end{itemize}
\begin{itemize}
\item {Utilização:Bras}
\end{itemize}
\begin{itemize}
\item {Proveniência:(Ingl. \textunderscore bond\textunderscore )}
\end{itemize}
Título de dívida externa, do juro de 3 por cento, pagável ao portador.
Carro de systema americano.
\section{Bondinho}
\begin{itemize}
\item {Grp. gram.:m.}
\end{itemize}
\begin{itemize}
\item {Utilização:Bras. do Rio}
\end{itemize}
\begin{itemize}
\item {Proveniência:(De \textunderscore bonde\textunderscore )}
\end{itemize}
Carro de viação urbana, mais pequeno que os que fazem carreira para fóra da cidade.
\section{Bondoso}
\begin{itemize}
\item {Grp. gram.:adj.}
\end{itemize}
Que tem bondade.
Benévolo.
Humanitário.
(Contr. de \textunderscore bondadoso\textunderscore )
\section{Bonduque}
\begin{itemize}
\item {Grp. gram.:m.}
\end{itemize}
\begin{itemize}
\item {Proveniência:(Fr. \textunderscore bonduc\textunderscore )}
\end{itemize}
Planta leguminosa, também conhecida por \textunderscore ôlho-de-gato\textunderscore .
\section{Boné}
\begin{itemize}
\item {fónica:bó-né}
\end{itemize}
\begin{itemize}
\item {Grp. gram.:m.}
\end{itemize}
\begin{itemize}
\item {Proveniência:(Fr. \textunderscore bonnet\textunderscore )}
\end{itemize}
Cobertura da cabeça, sem abas, de copa redonda, ás vezes com pala.
\section{Boneca}
\begin{itemize}
\item {Grp. gram.:f.}
\end{itemize}
\begin{itemize}
\item {Utilização:Carp.}
\end{itemize}
\begin{itemize}
\item {Utilização:Bras}
\end{itemize}
\begin{itemize}
\item {Utilização:Bras}
\end{itemize}
\begin{itemize}
\item {Utilização:Bras}
\end{itemize}
\begin{itemize}
\item {Grp. gram.:Pl.}
\end{itemize}
\begin{itemize}
\item {Utilização:Náut.}
\end{itemize}
Pequena figura de trapo, cera, cartão, etc., representando senhora ou menina, e destinada a brinquedo de crianças.
Senhora muito enfeitada e pouco animada.
Embrulho de estôfo, que contém uma substância solúvel, para envernizar.
Conjunto de duas escoras oblíquas, ligadas por uma peça horizontal, para segurar o ângulo interno de duas superfícies verticaes, que se encontram com uma superfície horizontal.
Peça de ferro, vertical, na boleia dos carros, e á qual se prendem, posteriormente, os tirantes.
Bandeirola do milho, em flôr.
Pequeno embrulho de linho, que se embebe em substância alimentícia, para que a chupem as crianças de peito, se o leite lhes falta.
Embrulho de estôfo, que contém qualquer substância:«\textunderscore ferver as ervas com uma bonecazinha, contendo cinza.\textunderscore »Ed. Magalhães, \textunderscore Hyg. Alim.\textunderscore , I, 293.
\textunderscore Boneca de milho\textunderscore , a espiga do milho, ainda com os estames ligados aos grãos.
Peças de madeira, que no convés servem de apoio ás antennas sobrecelentes avante do mastro grande.
(Cp. \textunderscore boneco\textunderscore )
\section{Bonecada}
\begin{itemize}
\item {Grp. gram.:f.}
\end{itemize}
Porção de bonecos.
\section{Bonecar}
\begin{itemize}
\item {Grp. gram.:v. i.}
\end{itemize}
\begin{itemize}
\item {Utilização:Bras}
\end{itemize}
\begin{itemize}
\item {Proveniência:(De \textunderscore boneca\textunderscore )}
\end{itemize}
Produzir bandeirola, espigar (o milho).
\section{Boneco}
\begin{itemize}
\item {Grp. gram.:m.}
\end{itemize}
\begin{itemize}
\item {Utilização:Fig.}
\end{itemize}
Pequena figura de trapo, cartão, cera, etc., para representar homem ou rapaz, e destinada a brinquedo de crianças.
Na linguagem infantil, estampas ou desenhos que representam pessôas ou animaes.
Homem presumido, que só pensa em agradar pela sua figura.
(Cp. \textunderscore bonifrate\textunderscore )
\section{Bonecra}
\begin{itemize}
\item {Grp. gram.:f.}
\end{itemize}
(V.boneca)
\section{Bonecra}
\begin{itemize}
\item {Grp. gram.:f.}
\end{itemize}
\begin{itemize}
\item {Utilização:Prov.}
\end{itemize}
\begin{itemize}
\item {Utilização:trasm.}
\end{itemize}
Castanha chocha.
(Gal. \textunderscore bolerca\textunderscore )
\section{Bonecrage}
\begin{itemize}
\item {Grp. gram.:f.}
\end{itemize}
Conjunto de bonecas. Cf. Castilho, \textunderscore Collóq. Ald.\textunderscore , 262.
\section{Bonecro}
\begin{itemize}
\item {Grp. gram.:m.}
\end{itemize}
(V.boneco)
\section{Bonefre}
\begin{itemize}
\item {Grp. gram.:m.}
\end{itemize}
Planta escrofularínea.
\section{Boneja}
\begin{itemize}
\item {Grp. gram.:f.}
\end{itemize}
\begin{itemize}
\item {Utilização:Des.}
\end{itemize}
Amásia.
Mulher de má fama.
\section{Bonéte}
\begin{itemize}
\item {Grp. gram.:m.}
\end{itemize}
Fruto mexicano, semelhante ao melão.
\section{Bonéte}
\begin{itemize}
\item {Grp. gram.:m.}
\end{itemize}
(V.boné)
\section{Bonête}
\begin{itemize}
\item {Grp. gram.:m.}
\end{itemize}
\begin{itemize}
\item {Utilização:Náut.}
\end{itemize}
\begin{itemize}
\item {Proveniência:(Fr. \textunderscore bonnette\textunderscore )}
\end{itemize}
Pequena vela, que se junta á grande e desce até vibordo.
\section{Bongar}
\begin{itemize}
\item {Grp. gram.:v. t.}
\end{itemize}
\begin{itemize}
\item {Utilização:Bras}
\end{itemize}
\begin{itemize}
\item {Proveniência:(Do quimb. \textunderscore cu-bonga\textunderscore )}
\end{itemize}
Buscar, procurar.
\section{Bongaro}
\begin{itemize}
\item {Grp. gram.:m.}
\end{itemize}
Serpente venenosa de Bengala e Java.
\section{Bonho}
\begin{itemize}
\item {Grp. gram.:m.}
\end{itemize}
\begin{itemize}
\item {Utilização:T. da Bairrada}
\end{itemize}
O mesmo que \textunderscore bunho\textunderscore  ou tabúa.
\section{Bonhomia}
\begin{itemize}
\item {fónica:no}
\end{itemize}
\begin{itemize}
\item {Grp. gram.:f.}
\end{itemize}
(V.bonomia)
\section{Bonico}
\begin{itemize}
\item {Grp. gram.:m.}
\end{itemize}
\begin{itemize}
\item {Utilização:Prov.}
\end{itemize}
\begin{itemize}
\item {Utilização:beir.}
\end{itemize}
\begin{itemize}
\item {Grp. gram.:Pl.}
\end{itemize}
\begin{itemize}
\item {Utilização:Pop.}
\end{itemize}
Bosta de novilho, com que se barram os cortiços de abelhas.
Caganitas.
(Por \textunderscore bolico\textunderscore , de \textunderscore bóla\textunderscore ? Cp. \textunderscore bolisco\textunderscore )
\section{Bonideco}
\begin{itemize}
\item {Grp. gram.:adv.}
\end{itemize}
\begin{itemize}
\item {Utilização:Açor}
\end{itemize}
\begin{itemize}
\item {Proveniência:(Do lat. \textunderscore bono et aequo\textunderscore )}
\end{itemize}
De bôa vontade, espontaneamente.
\section{Bonifácia}
\begin{itemize}
\item {Grp. gram.:f.}
\end{itemize}
Bolo supplementar, no jôgo do voltarete, do qual se tira a quantia necessária, para completar as entradas que são precisas para haver jôgo segundo.
\section{Bonificação}
\begin{itemize}
\item {Grp. gram.:f.}
\end{itemize}
\begin{itemize}
\item {Utilização:Neol.}
\end{itemize}
Acção de \textunderscore bonificar\textunderscore .
Benefício ou vantagem, que se dá em títulos e acções de Companhias mercantis e de Bancos; bónus.
\section{Bonificar}
\begin{itemize}
\item {Grp. gram.:v. t.}
\end{itemize}
\begin{itemize}
\item {Utilização:Ant.}
\end{itemize}
\begin{itemize}
\item {Utilização:Neol.}
\end{itemize}
\begin{itemize}
\item {Proveniência:(Do lat. \textunderscore bonus\textunderscore  + \textunderscore facere\textunderscore )}
\end{itemize}
Beneficiar; melhorar.
Dar bonificação ou bónus a.
\section{Bonifrate}
\begin{itemize}
\item {Grp. gram.:m.}
\end{itemize}
\begin{itemize}
\item {Proveniência:(Do lat. \textunderscore bonus\textunderscore  + \textunderscore frater\textunderscore )}
\end{itemize}
Boneco, que se move por arames; títere; fantoche.
Pessôa casquilha, leviana, ridícula.
\section{Bonifrateiro}
\begin{itemize}
\item {Grp. gram.:m.}
\end{itemize}
Fabricante ou vendedor de bonifrates.
\section{Bonina}
\begin{itemize}
\item {Grp. gram.:f.}
\end{itemize}
\begin{itemize}
\item {Utilização:Bras}
\end{itemize}
\begin{itemize}
\item {Proveniência:(Do lat. \textunderscore bonus\textunderscore )}
\end{itemize}
Margarida dos prados.
Bôas-noites.
\section{Boninal}
\begin{itemize}
\item {Grp. gram.:m.}
\end{itemize}
Campo cheio de boninas.
\section{Bonissimo}
\begin{itemize}
\item {Grp. gram.:adj.}
\end{itemize}
\begin{itemize}
\item {Proveniência:(Do lat. \textunderscore bonus\textunderscore )}
\end{itemize}
Muito bom.
\section{Bonitete}
\begin{itemize}
\item {fónica:tête}
\end{itemize}
\begin{itemize}
\item {Grp. gram.:adj.}
\end{itemize}
Um tanto bonito.
\section{Boniteza}
\begin{itemize}
\item {Grp. gram.:f.}
\end{itemize}
Qualidade do que é bonito.
\section{Bonito}
\begin{itemize}
\item {Grp. gram.:adj.}
\end{itemize}
\begin{itemize}
\item {Utilização:Irón.}
\end{itemize}
\begin{itemize}
\item {Grp. gram.:M.}
\end{itemize}
\begin{itemize}
\item {Utilização:Infant.}
\end{itemize}
\begin{itemize}
\item {Proveniência:(Do lat. \textunderscore bonus\textunderscore )}
\end{itemize}
Que agrada á vista; formoso, gentil.
Bom, nobre: \textunderscore uma bonita acção\textunderscore .
Incorrecto, censurável: \textunderscore fê-la bonita\textunderscore !
Espécie de atum.
Qualquer brinquedo. Cf. Castilho, \textunderscore Fausto\textunderscore , 218 e 227.
\section{Bonitote}
\begin{itemize}
\item {Grp. gram.:adj.}
\end{itemize}
O mesmo que \textunderscore bonitete\textunderscore .
\section{Bonomia}
\begin{itemize}
\item {Grp. gram.:f.}
\end{itemize}
\begin{itemize}
\item {Utilização:Neol.}
\end{itemize}
\begin{itemize}
\item {Proveniência:(Fr. \textunderscore bonhomie\textunderscore , mal der. de \textunderscore bon\textunderscore  + \textunderscore homme\textunderscore )}
\end{itemize}
Qualidade do homem, que é bom, simples e crédulo.
\section{Bonora}
\begin{itemize}
\item {Grp. gram.:adv.}
\end{itemize}
\begin{itemize}
\item {Utilização:Ant.}
\end{itemize}
\begin{itemize}
\item {Proveniência:(Do lat. \textunderscore bonus\textunderscore  + \textunderscore hora\textunderscore )}
\end{itemize}
Em bôa hora. Cf. G. Vicente, I, 216.
\section{Bons-dias}
\begin{itemize}
\item {Grp. gram.:m. pl.}
\end{itemize}
Planta, de flôres campanuladas, azues, brancas e amarelas, que abrem de manhan e fecham á noite; convólvulo; azuraque; adelaidinha.
\section{Bonus}
\begin{itemize}
\item {fónica:bó}
\end{itemize}
\begin{itemize}
\item {Grp. gram.:m.}
\end{itemize}
\begin{itemize}
\item {Proveniência:(T. lat.)}
\end{itemize}
Prêmio, que algumas Companhias ou Empresas concedem aos seus associados ou subscritores, sob condições estipuladas.
Desconto ou abatimento, que a alguns passageiros em caminhos de ferro se faz, no preço da passagem.
\section{Bonvedro}
\begin{itemize}
\item {Grp. gram.:m.}
\end{itemize}
O mesmo ou melhor que \textunderscore bomvedro\textunderscore .
\section{Bonzo}
\begin{itemize}
\item {Grp. gram.:m.}
\end{itemize}
\begin{itemize}
\item {Utilização:Bras}
\end{itemize}
Sacerdote budista.
Hypócrita; jesuíta.
(Japon. \textunderscore bozu\textunderscore )
\section{Boótes}
\begin{itemize}
\item {Grp. gram.:m.}
\end{itemize}
\begin{itemize}
\item {Proveniência:(Gr. \textunderscore bootes\textunderscore )}
\end{itemize}
O mesmo que \textunderscore Boieiro\textunderscore , constellação.
\section{Bopiano}
\begin{itemize}
\item {Grp. gram.:adj.}
\end{itemize}
Relativo ao philólogo Bopp.
\section{Boppiano}
\begin{itemize}
\item {Grp. gram.:adj.}
\end{itemize}
Relativo ao philólogo Bopp.
\section{Boque!}
\begin{itemize}
\item {Grp. gram.:interj.}
\end{itemize}
\begin{itemize}
\item {Utilização:T. da Bairrada}
\end{itemize}
O mesmo que \textunderscore bóca!\textunderscore 
\section{Boqueada}
\begin{itemize}
\item {Grp. gram.:f.}
\end{itemize}
Acção de \textunderscore boquear\textunderscore .
\section{Boquear}
\begin{itemize}
\item {Grp. gram.:v. i.}
\end{itemize}
\begin{itemize}
\item {Proveniência:(De \textunderscore bôca\textunderscore )}
\end{itemize}
Abrir a bôca, respirando com difficuldade.
Agonizar.
Boquejar.
\section{Boqueira}
\begin{itemize}
\item {Grp. gram.:f.}
\end{itemize}
\begin{itemize}
\item {Proveniência:(De \textunderscore bôca\textunderscore )}
\end{itemize}
Pequena ferida ao canto da bôca.
\section{Boqueirão}
\begin{itemize}
\item {Grp. gram.:m.}
\end{itemize}
\begin{itemize}
\item {Proveniência:(De \textunderscore bôca\textunderscore )}
\end{itemize}
Grande bôca.
Abertura de um canal.
Covão.
Rua, que dá para o rio.
Peixe da costa do Algarve e dos Açores.
\section{Boqueiro}
\begin{itemize}
\item {Grp. gram.:m.}
\end{itemize}
\begin{itemize}
\item {Utilização:Prov.}
\end{itemize}
\begin{itemize}
\item {Utilização:trasm.}
\end{itemize}
Buraco numa presa de água.
Entrada num cerrado, para gente e para gado.
\section{Boquejadura}
\begin{itemize}
\item {Grp. gram.:f.}
\end{itemize}
(V.boquejo)
\section{Boquejar}
\begin{itemize}
\item {Grp. gram.:v. i.}
\end{itemize}
\begin{itemize}
\item {Grp. gram.:V. t.}
\end{itemize}
\begin{itemize}
\item {Proveniência:(De \textunderscore bôca\textunderscore )}
\end{itemize}
Bocejar.
Falar baixo.
Murmurar, falar mal.
Tocar com a bôca.
\section{Boquejo}
\begin{itemize}
\item {Grp. gram.:m.}
\end{itemize}
Acção de \textunderscore boquejar\textunderscore .
\section{Boquelho}
\begin{itemize}
\item {fónica:quê}
\end{itemize}
\begin{itemize}
\item {Grp. gram.:m.}
\end{itemize}
\begin{itemize}
\item {Proveniência:(De \textunderscore bôca\textunderscore )}
\end{itemize}
Pequena bôca ou buraco, ao pé da bôca do forno.
\section{Boquete}
\begin{itemize}
\item {fónica:quê}
\end{itemize}
\begin{itemize}
\item {Grp. gram.:m.}
\end{itemize}
\begin{itemize}
\item {Utilização:Prov.}
\end{itemize}
\begin{itemize}
\item {Utilização:alent.}
\end{itemize}
\begin{itemize}
\item {Proveniência:(De \textunderscore bôca\textunderscore )}
\end{itemize}
Buraco, pequena bôca.
\section{Boquiaberto}
\begin{itemize}
\item {fónica:bô}
\end{itemize}
\begin{itemize}
\item {Grp. gram.:adj.}
\end{itemize}
\begin{itemize}
\item {Proveniência:(De \textunderscore bôca\textunderscore  + \textunderscore aberto\textunderscore )}
\end{itemize}
Que tem a bôca aberta.
Admirado.
Imbecil.
\section{Boquiabrir}
\begin{itemize}
\item {fónica:bô}
\end{itemize}
\begin{itemize}
\item {Grp. gram.:v. t.}
\end{itemize}
\begin{itemize}
\item {Utilização:bras}
\end{itemize}
\begin{itemize}
\item {Utilização:Neol.}
\end{itemize}
\begin{itemize}
\item {Proveniência:(De \textunderscore bôca\textunderscore  + \textunderscore abrir\textunderscore )}
\end{itemize}
Causar grande admiração.
\section{Boquialvo}
\begin{itemize}
\item {fónica:bô}
\end{itemize}
\begin{itemize}
\item {Grp. gram.:adj.}
\end{itemize}
O mesmo que \textunderscore bocalvo\textunderscore .
\section{Boquiardente}
\begin{itemize}
\item {fónica:bô}
\end{itemize}
\begin{itemize}
\item {Grp. gram.:adj.}
\end{itemize}
\begin{itemize}
\item {Proveniência:(De \textunderscore bôca\textunderscore  + \textunderscore ardente\textunderscore )}
\end{itemize}
Diz-se do cavallo, cuja bôca se resente muito do freio.
\section{Boquicheio}
\begin{itemize}
\item {fónica:bô}
\end{itemize}
\begin{itemize}
\item {Grp. gram.:adj.}
\end{itemize}
\begin{itemize}
\item {Proveniência:(De \textunderscore bôca\textunderscore  + \textunderscore cheio\textunderscore )}
\end{itemize}
Que pronuncía claramente.
\section{Boquiduro}
\begin{itemize}
\item {fónica:bô}
\end{itemize}
\begin{itemize}
\item {Grp. gram.:adj.}
\end{itemize}
\begin{itemize}
\item {Proveniência:(De \textunderscore bôca\textunderscore  + \textunderscore duro\textunderscore )}
\end{itemize}
Diz-se do cavallo, cuja bôca se resente pouco do freio.
\section{Boquifendido}
\begin{itemize}
\item {fónica:bô}
\end{itemize}
\begin{itemize}
\item {Grp. gram.:adj.}
\end{itemize}
\begin{itemize}
\item {Proveniência:(De \textunderscore bôca\textunderscore  + \textunderscore fendido\textunderscore )}
\end{itemize}
Diz-se do cavallo, que tem a bôca grande, muito fendida.
\section{Boquifranzido}
\begin{itemize}
\item {fónica:bô}
\end{itemize}
\begin{itemize}
\item {Grp. gram.:adj.}
\end{itemize}
\begin{itemize}
\item {Proveniência:(De \textunderscore bôca\textunderscore  + \textunderscore franzir\textunderscore )}
\end{itemize}
Que franziu os beiços.
\section{Boquilha}
\begin{itemize}
\item {Grp. gram.:f.}
\end{itemize}
\begin{itemize}
\item {Proveniência:(De \textunderscore bôca\textunderscore )}
\end{itemize}
Tubo, por onde se fuma o cigarro ou o charuto.
\section{Boquilheiro}
\begin{itemize}
\item {Grp. gram.:m.}
\end{itemize}
Estojo para guardar boquilhas dos instrumentos músicos.
\section{Boquim}
\begin{itemize}
\item {Grp. gram.:m.}
\end{itemize}
\begin{itemize}
\item {Proveniência:(De \textunderscore bôca\textunderscore )}
\end{itemize}
Bocal de corneta.
\section{Boquimolle}
\begin{itemize}
\item {fónica:bô}
\end{itemize}
\begin{itemize}
\item {Grp. gram.:adj.}
\end{itemize}
\begin{itemize}
\item {Proveniência:(De \textunderscore bôca\textunderscore  + \textunderscore molle\textunderscore )}
\end{itemize}
Diz-se do cavallo, que tem a bôca branda, doce.
\section{Boquinegro}
\begin{itemize}
\item {fónica:bô}
\end{itemize}
\begin{itemize}
\item {Grp. gram.:adj.}
\end{itemize}
\begin{itemize}
\item {Proveniência:(De \textunderscore bôca\textunderscore  + \textunderscore negro\textunderscore )}
\end{itemize}
Que tem a bôca negra.
\section{Boquinha}
\begin{itemize}
\item {Grp. gram.:f.}
\end{itemize}
\begin{itemize}
\item {Utilização:Bras}
\end{itemize}
Bôca pequena.
\textunderscore Fazer boquinha\textunderscore , franzir os lábios em sinal de desgôsto ou agastamento.
Beijo.
\section{Boquirasgado}
\begin{itemize}
\item {fónica:bô,ras}
\end{itemize}
\begin{itemize}
\item {Grp. gram.:adj.}
\end{itemize}
O mesmo que \textunderscore boquifendido\textunderscore .
\section{Boquirrasgado}
\begin{itemize}
\item {fónica:bô}
\end{itemize}
\begin{itemize}
\item {Grp. gram.:adj.}
\end{itemize}
O mesmo que \textunderscore boquifendido\textunderscore .
\section{Boquirroto}
\begin{itemize}
\item {fónica:bô}
\end{itemize}
\begin{itemize}
\item {Grp. gram.:adj.}
\end{itemize}
\begin{itemize}
\item {Proveniência:(De \textunderscore bôca\textunderscore  + \textunderscore roto\textunderscore )}
\end{itemize}
Que fala muito.
Que não guarda segrêdo.
\section{Boquisseco}
\begin{itemize}
\item {fónica:bô}
\end{itemize}
\begin{itemize}
\item {Grp. gram.:adj.}
\end{itemize}
\begin{itemize}
\item {Proveniência:(De \textunderscore bôca\textunderscore  + \textunderscore sêco\textunderscore )}
\end{itemize}
Que tem a bôca sêca.
Que não diz nada.
\section{Boquitorto}
\begin{itemize}
\item {fónica:bô}
\end{itemize}
\begin{itemize}
\item {Grp. gram.:adj.}
\end{itemize}
\begin{itemize}
\item {Proveniência:(De \textunderscore bôca\textunderscore  + \textunderscore torto\textunderscore )}
\end{itemize}
Que tem a bôca torta.
\section{Bôr}
\begin{itemize}
\item {Grp. gram.:m.}
\end{itemize}
Árvore da Índia portuguesa, (\textunderscore zizyphus jujuba\textunderscore ).
(Do conc.)
\section{Borá}
\begin{itemize}
\item {Grp. gram.:m.}
\end{itemize}
\begin{itemize}
\item {Utilização:Bras}
\end{itemize}
Espécie de abelha americana.
Substância amarela e amargosa, que se encontra nos cortiços de abelhas e que estas comem.
(Do tupi?)
\section{Borácico}
\begin{itemize}
\item {Grp. gram.:adj.}
\end{itemize}
(V.bórico)
\section{Boratado}
\begin{itemize}
\item {Grp. gram.:adj.}
\end{itemize}
\begin{itemize}
\item {Proveniência:(De \textunderscore borato\textunderscore )}
\end{itemize}
Que tem ácido bórico.
\section{Borato}
\begin{itemize}
\item {Grp. gram.:m.}
\end{itemize}
\begin{itemize}
\item {Utilização:Chím.}
\end{itemize}
\begin{itemize}
\item {Proveniência:(De \textunderscore boro\textunderscore )}
\end{itemize}
Sal, resultante da combinação do ácido bórico com uma base.
\section{Bórax}
\begin{itemize}
\item {Grp. gram.:m.}
\end{itemize}
Sub-borato de soda.
Tincal.
(Ár. \textunderscore burag\textunderscore )
\section{Borbeto}
\begin{itemize}
\item {fónica:bê}
\end{itemize}
\begin{itemize}
\item {Grp. gram.:m.}
\end{itemize}
\begin{itemize}
\item {Utilização:Prov.}
\end{itemize}
\begin{itemize}
\item {Utilização:minh.}
\end{itemize}
Qualquer objecto, que tome a apparência de caroço.
(Por \textunderscore borboto\textunderscore ?)
\section{Borboínha}
\begin{itemize}
\item {Grp. gram.:f.}
\end{itemize}
\begin{itemize}
\item {Utilização:Ant.}
\end{itemize}
O mesmo que \textunderscore borborinho\textunderscore :«\textunderscore e começou a haver grandes borboinhas\textunderscore ». D. do Couto, \textunderscore D. Paulo de Lima\textunderscore , c. II.
\section{Borboleta}
\begin{itemize}
\item {fónica:lê}
\end{itemize}
\begin{itemize}
\item {Grp. gram.:f.}
\end{itemize}
\begin{itemize}
\item {Utilização:Bot.}
\end{itemize}
\begin{itemize}
\item {Utilização:Bras}
\end{itemize}
\begin{itemize}
\item {Utilização:Bras}
\end{itemize}
\begin{itemize}
\item {Utilização:Fig.}
\end{itemize}
\begin{itemize}
\item {Utilização:T. de Lisbôa}
\end{itemize}
\begin{itemize}
\item {Utilização:fam.}
\end{itemize}
Insecto alado, da ordem dos lepidópteros.
Ranúnculo dos jardins.
Espécie de papoila.
Trepadeira vivaz, (\textunderscore rhyncosia\textunderscore ).
Pessôa volúvel.
Mulher, que vagueia de noite, provocando deshonestamente os transeuntes.
(Cp. gall. \textunderscore borboreta\textunderscore )
\section{Borboleteador}
\begin{itemize}
\item {Grp. gram.:adj.}
\end{itemize}
Que borboleteia.
Que devaneia.
\section{Borboleteamento}
\begin{itemize}
\item {Grp. gram.:m.}
\end{itemize}
Acto de \textunderscore borboletear\textunderscore .
\section{Borboletear}
\begin{itemize}
\item {Grp. gram.:v. i.}
\end{itemize}
\begin{itemize}
\item {Proveniência:(De \textunderscore borboleta\textunderscore )}
\end{itemize}
Vaguear, divagar, como as borboletas.
Devanear.
\section{Borboletice}
\begin{itemize}
\item {Grp. gram.:f.}
\end{itemize}
Capricho, modos de borboleta. Cf. M. Assis, \textunderscore Brás Cubas\textunderscore , 100.
\section{Borbónia}
\begin{itemize}
\item {Grp. gram.:f.}
\end{itemize}
\begin{itemize}
\item {Proveniência:(De \textunderscore Borbon\textunderscore , n. p.)}
\end{itemize}
Gênero de plantas leguminosas, de que há doze espécies, cultivadas em jardins.
\section{Borboniano}
\begin{itemize}
\item {Grp. gram.:adj.}
\end{itemize}
Relativo á família dos Borbons. Cf. Latino, \textunderscore Humboldt\textunderscore , 374.
\section{Borborijar}
\begin{itemize}
\item {Grp. gram.:v. i.}
\end{itemize}
Rumorejar como água em cachão. Cf. Arn. Gama, \textunderscore Última Dona\textunderscore , 314.
\section{Borborinhar}
\begin{itemize}
\item {Grp. gram.:v. i.}
\end{itemize}
Soar como borborinho; fazer borborinho. Cf. Camillo, \textunderscore Carlos Rib.\textunderscore , 36; Arn. Gama, \textunderscore Motim\textunderscore , 382.
\section{Borborinho}
\begin{itemize}
\item {Grp. gram.:m.}
\end{itemize}
Som confuso de vozes.
Rumor.
Tumulto, desordem.
(Por \textunderscore murmurinho\textunderscore )
\section{Borborismo}
\begin{itemize}
\item {Grp. gram.:m.}
\end{itemize}
\begin{itemize}
\item {Proveniência:(Do gr. \textunderscore borborugmos\textunderscore )}
\end{itemize}
Ruído, produzido no ventre pela deslocação de gases.
\section{Borboró}
\begin{itemize}
\item {Grp. gram.:adj.}
\end{itemize}
\begin{itemize}
\item {Utilização:Bras. do N}
\end{itemize}
Que é gago ou tartamudo.
\section{Borborygmo}
\begin{itemize}
\item {Grp. gram.:m.}
\end{itemize}
(V.borborismo)
\section{Borbotão}
\begin{itemize}
\item {Grp. gram.:m.}
\end{itemize}
Lufada.
Golfada.
Jacto de líquido.
(Do mesmo rad. que \textunderscore borbulha\textunderscore )
\section{Borbotar}
\begin{itemize}
\item {Grp. gram.:v. i.}
\end{itemize}
\begin{itemize}
\item {Grp. gram.:V. t.}
\end{itemize}
\begin{itemize}
\item {Grp. gram.:V. t.}
\end{itemize}
Sair em borbotões, em jorros.
Lançar em borbotões.
Formar botões (a planta), abotoar-se.
Expellir em borbotões.
\section{Borbôto}
\begin{itemize}
\item {Grp. gram.:m.}
\end{itemize}
\begin{itemize}
\item {Utilização:Prov.}
\end{itemize}
\begin{itemize}
\item {Proveniência:(De \textunderscore borbotar\textunderscore )}
\end{itemize}
Botão da planta.
\section{Borbulha}
\begin{itemize}
\item {Grp. gram.:f.}
\end{itemize}
\begin{itemize}
\item {Proveniência:(Do lat. \textunderscore bulla\textunderscore , com reduplicação da 1.^a sýllaba, segundo Diez)}
\end{itemize}
Pequena empôla, sob a epiderme.
Bolha de ar, á superfície da água.
Botãozinho vermelho, pequeno ponto inflammado, sôbre a pelle.
Fervura da água.
Defeito; mancha.
Mania, bolha.
Excrescência vegetal, donde há de sair a flôr ou a fôlha ou novo ramo.
\section{Borbulhagem}
\begin{itemize}
\item {Grp. gram.:f.}
\end{itemize}
Grande porção de borbulhas.
\section{Borbulhante}
\begin{itemize}
\item {Grp. gram.:adj.}
\end{itemize}
Que borbulha.
\section{Borbulhão}
\begin{itemize}
\item {Grp. gram.:m.}
\end{itemize}
Grande borbulha.
\section{Borbulhar}
\begin{itemize}
\item {Grp. gram.:v. i.}
\end{itemize}
\begin{itemize}
\item {Grp. gram.:V. t.}
\end{itemize}
\begin{itemize}
\item {Utilização:Des.}
\end{itemize}
\begin{itemize}
\item {Proveniência:(De \textunderscore borbulha\textunderscore )}
\end{itemize}
Borbotar.
Sair em borbulhas, em bolhas.
Cobrir-se de borbulhas.
Formar cachão, fervendo.
Sair em magotes, com ímpeto.
Fazer que as árvores abotôem, que lancem borbulha.
\section{Borbulhento}
\begin{itemize}
\item {Grp. gram.:adj.}
\end{itemize}
Que tem borbulhas.
\section{Borbulho}
\begin{itemize}
\item {Grp. gram.:m.}
\end{itemize}
(V.borbulhão)
\section{Borbulhoso}
\begin{itemize}
\item {Grp. gram.:adj.}
\end{itemize}
\begin{itemize}
\item {Proveniência:(De \textunderscore borbulha\textunderscore )}
\end{itemize}
Que tem borbulhas.
Que sái em bolhas ou gotas.
Que fórma bolhas.
\section{Borcar}
\textunderscore v. t.\textunderscore  (e der.) \textunderscore Prov. beir.\textunderscore 
O mesmo que \textunderscore emborcar\textunderscore .
\section{Borcelar}
\begin{itemize}
\item {Grp. gram.:v. t.}
\end{itemize}
Tirar um borcelo ou pedaço da bôca de (uma vasilha de barro), o mesmo que \textunderscore esboicelar\textunderscore .
\section{Borcelo}
\begin{itemize}
\item {fónica:cê}
\end{itemize}
\begin{itemize}
\item {Grp. gram.:m.}
\end{itemize}
\begin{itemize}
\item {Utilização:Ant.}
\end{itemize}
\begin{itemize}
\item {Proveniência:(Do lat. \textunderscore buccella\textunderscore , de \textunderscore bucca\textunderscore ?)}
\end{itemize}
Pedaço; fragmento.
\section{Bôrco}
\begin{itemize}
\item {Grp. gram.:m.}
\end{itemize}
\textunderscore De bôrco\textunderscore , de bôca para baixo, (falando-se de vaso ou vasilha).
De face para baixo, de cama, (falando-se de pessôas): \textunderscore minha tia ficou hoje de bôrco\textunderscore .
(Cp. \textunderscore bôlco\textunderscore )
\section{Borda}
\begin{itemize}
\item {Grp. gram.:f.}
\end{itemize}
\begin{itemize}
\item {Utilização:Náut.}
\end{itemize}
\begin{itemize}
\item {Utilização:Ant.}
\end{itemize}
\begin{itemize}
\item {Proveniência:(De \textunderscore bôrdo\textunderscore )}
\end{itemize}
Extremidade.
Orla; beira.
Fímbria: \textunderscore borda de um vestido\textunderscore .
Margem; praia: \textunderscore á borda do Tejo\textunderscore .
\textunderscore Dar borda\textunderscore , inclinar-se muito a embarcação, com risco.
Espécie de clava, armada de puas.
\section{Bordada}
\begin{itemize}
\item {Grp. gram.:f.}
\end{itemize}
\begin{itemize}
\item {Utilização:Náut.}
\end{itemize}
\begin{itemize}
\item {Proveniência:(De \textunderscore borda\textunderscore )}
\end{itemize}
Banda.
Acção de bordejar.
Acto de marear, bordejando.
Descanso de canhões, de cada um dos lados do navio.
Espécie de vela.
\section{Bordadágua}
\begin{itemize}
\item {fónica:bór}
\end{itemize}
\begin{itemize}
\item {Grp. gram.:f.}
\end{itemize}
\begin{itemize}
\item {Grp. gram.:M.}
\end{itemize}
\begin{itemize}
\item {Utilização:Pop.}
\end{itemize}
O mesmo que \textunderscore beiramar\textunderscore .
Calendário. Cf. Castilho, \textunderscore Fastos\textunderscore , II, 496.
\section{Bordadeira}
\begin{itemize}
\item {Grp. gram.:f.}
\end{itemize}
Mulher que borda.
\section{Bordado}
\begin{itemize}
\item {Grp. gram.:m.}
\end{itemize}
Obra de bordadura.
\section{Bordador}
\begin{itemize}
\item {Grp. gram.:m.}
\end{itemize}
Aquelle que borda.
\section{Bordadura}
\begin{itemize}
\item {Grp. gram.:f.}
\end{itemize}
Effeito de \textunderscore bordar\textunderscore .
Orla; cercadura bordada.
Ornato, que limita a superfície de um objecto.
Cercadura vegetal nas divisões de um jardim.
\section{Bordage}
\begin{itemize}
\item {Grp. gram.:f.}
\end{itemize}
\begin{itemize}
\item {Proveniência:(De \textunderscore bôrdo\textunderscore )}
\end{itemize}
Madeira do costado dos navios.
\section{Bordagem}
\begin{itemize}
\item {Grp. gram.:f.}
\end{itemize}
\begin{itemize}
\item {Proveniência:(De \textunderscore bôrdo\textunderscore )}
\end{itemize}
Madeira do costado dos navios.
\section{Bordaleira}
\begin{itemize}
\item {Grp. gram.:f.}
\end{itemize}
\begin{itemize}
\item {Utilização:Prov.}
\end{itemize}
\begin{itemize}
\item {Utilização:trasm.}
\end{itemize}
Abertura, o mesmo que \textunderscore gateira\textunderscore .
\section{Bordaleiro}
\begin{itemize}
\item {Grp. gram.:m.}
\end{itemize}
Espécie de carneiro português, de lan crespa.
\section{Bordalengo}
\begin{itemize}
\item {Grp. gram.:adj.}
\end{itemize}
Grosseiro.
Ignorante; imbecil.
(Cp. \textunderscore bordelês\textunderscore )
\section{Bordalo}
\begin{itemize}
\item {Grp. gram.:m.}
\end{itemize}
Pequeno peixe do rio, espécie de mugem.
Grande peixe eléctrico do Nilo, o mesmo que \textunderscore siluro\textunderscore  eléctrico.
\section{Bordamento}
\begin{itemize}
\item {Grp. gram.:m.}
\end{itemize}
O mesmo que \textunderscore bordadura\textunderscore . Acto ou effeito de \textunderscore bordar\textunderscore .
\section{Bordão}
\begin{itemize}
\item {Grp. gram.:m.}
\end{itemize}
\begin{itemize}
\item {Utilização:Fig.}
\end{itemize}
\begin{itemize}
\item {Utilização:Fam.}
\end{itemize}
Bastão; cajado grosso.
Pau, que termina superiormente em volta.
Amparo, arrimo.
Palavra ou phrase, que se repete inconscientemente, na conversa ou na escrita.
(B. lat. \textunderscore burdo\textunderscore , \textunderscore burdonis\textunderscore )
\section{Bordão}
\begin{itemize}
\item {Grp. gram.:m.}
\end{itemize}
O tom mais baixo, que, em certos instrumentos, serve de acompanhamento.
A corda mais grossa de certos instrumentos.
(Cast. \textunderscore bordón\textunderscore )
\section{Bordão}
\begin{itemize}
\item {Grp. gram.:m.}
\end{itemize}
\begin{itemize}
\item {Proveniência:(De \textunderscore bôrdo\textunderscore ^1?)}
\end{itemize}
Espécie de palmeira, de abundante seiva açucarada que, depois da sua fermentação, constitue o marufo.
\section{Bordar}
\begin{itemize}
\item {Grp. gram.:v. t.}
\end{itemize}
\begin{itemize}
\item {Utilização:Ext.}
\end{itemize}
\begin{itemize}
\item {Utilização:Fig.}
\end{itemize}
Guarnecer a borda de.
Enfeitar á roda.
Enfeitar, ornar.
Fazer desenhos ou relevos com agulha em.
Variar, enfeitando, (um discurso, uma composição literária, etc.).
Fantasiar, agrupando, (ideias, factos).
\section{Bordate}
\begin{itemize}
\item {Grp. gram.:m.}
\end{itemize}
Espécie de tecido antigo, usado em tempo de João III.
\section{Borde}
\begin{itemize}
\item {Grp. gram.:m.}
\end{itemize}
O mesmo que \textunderscore bordado\textunderscore :«\textunderscore espelho moldurado em um borde de florida verdura\textunderscore ». Castilho, \textunderscore Metam.\textunderscore , 299.
(Cp. cast. \textunderscore borde\textunderscore )
\section{Bordear}
\begin{itemize}
\item {Grp. gram.:v. i.}
\end{itemize}
\begin{itemize}
\item {Grp. gram.:V. t.}
\end{itemize}
\begin{itemize}
\item {Utilização:Prov.}
\end{itemize}
\begin{itemize}
\item {Utilização:alent.}
\end{itemize}
\begin{itemize}
\item {Proveniência:(De \textunderscore borda\textunderscore )}
\end{itemize}
O mesmo que \textunderscore bordejar\textunderscore .
Voltar a aresta de (qualquer peça de latão), em officina de latoeiro.
\section{Bordegano}
\begin{itemize}
\item {Grp. gram.:m.  e  adj.}
\end{itemize}
\begin{itemize}
\item {Utilização:Ant.}
\end{itemize}
O mesmo que \textunderscore bordegão\textunderscore .
\section{Bordegão}
\begin{itemize}
\item {Grp. gram.:m.}
\end{itemize}
\begin{itemize}
\item {Grp. gram.:Adj.}
\end{itemize}
Homem rústico, bodegão.
Bordalengo.
(Cp. \textunderscore bordalengo\textunderscore )
\section{Bordejar}
\begin{itemize}
\item {Grp. gram.:v. i.}
\end{itemize}
Navegar, mudando de rumo frequentemente, quando o vento não favorece a direcção conveniente.
Andar aos bordos, cambalear, por effeito de bebedeira. Cf. Castilho, \textunderscore Fastos\textunderscore , II, 61 e 289.
\section{Bordel}
\begin{itemize}
\item {Grp. gram.:m.}
\end{itemize}
Prostíbulo, lupanar.
\section{Bordelense}
\begin{itemize}
\item {Grp. gram.:m.  e  adj.}
\end{itemize}
O mesmo que \textunderscore bordelês\textunderscore .
\section{Bordelês}
\begin{itemize}
\item {Grp. gram.:adj.}
\end{itemize}
\begin{itemize}
\item {Grp. gram.:M.}
\end{itemize}
\begin{itemize}
\item {Proveniência:(Lat. \textunderscore burdigalensis\textunderscore , de \textunderscore Burdigala\textunderscore , n. p.)}
\end{itemize}
Relativo a Bordéus.
Habitante de Bordéus.
\section{Bordelete}
\begin{itemize}
\item {fónica:lê}
\end{itemize}
\begin{itemize}
\item {Grp. gram.:m.}
\end{itemize}
\begin{itemize}
\item {Utilização:Agr.}
\end{itemize}
O mesmo que \textunderscore cutidura\textunderscore .
Rebordo ou nó, no sítio do enxêrto.
\section{Bordéus}
\begin{itemize}
\item {Grp. gram.:m.}
\end{itemize}
Vinho, produzido pelos vinhedos das cercanías de \textunderscore Bordéus\textunderscore ^2.
\section{Bordidura}
\begin{itemize}
\item {Grp. gram.:f.}
\end{itemize}
(V.bordadura)
\section{Bordo}
\begin{itemize}
\item {fónica:bôr,bór}
\end{itemize}
\begin{itemize}
\item {Grp. gram.:m.}
\end{itemize}
\begin{itemize}
\item {Utilização:Fig.}
\end{itemize}
\begin{itemize}
\item {Grp. gram.:Loc. adv.}
\end{itemize}
\begin{itemize}
\item {Proveniência:(Do germ. \textunderscore bord\textunderscore )}
\end{itemize}
Lado do navio.
Rumo do navio.
Acto de bordejar.
Borda, beira.
Disposição de espírito.
\textunderscore A bordo\textunderscore , em navio.
\textunderscore Andar aos bordos\textunderscore , cambalear de bêbedo.
\section{Bordo}
\begin{itemize}
\item {fónica:bôr}
\end{itemize}
\begin{itemize}
\item {Grp. gram.:m.}
\end{itemize}
Árvore, que serve de typo ás aceráceas, (\textunderscore acer campestris\textunderscore ).
Madeira dessa árvore.
(Cp. cast. \textunderscore borde\textunderscore )
\section{Bordo}
\begin{itemize}
\item {fónica:bôr}
\end{itemize}
\begin{itemize}
\item {Grp. gram.:m.}
\end{itemize}
\begin{itemize}
\item {Utilização:Prov.}
\end{itemize}
\begin{itemize}
\item {Utilização:trasm.}
\end{itemize}
\begin{itemize}
\item {Proveniência:(De \textunderscore bordar\textunderscore )}
\end{itemize}
O mesmo que \textunderscore bordado\textunderscore .
\section{Bordoada}
\begin{itemize}
\item {Grp. gram.:f.}
\end{itemize}
\begin{itemize}
\item {Utilização:pop.}
\end{itemize}
\begin{itemize}
\item {Utilização:Fig.}
\end{itemize}
\begin{itemize}
\item {Proveniência:(De \textunderscore bordão\textunderscore ^1)}
\end{itemize}
Pancada com bordão; paulada.
Grande porção.
\section{Bordoado}
\begin{itemize}
\item {Grp. gram.:adj.}
\end{itemize}
\begin{itemize}
\item {Proveniência:(De \textunderscore bordão\textunderscore ^1)}
\end{itemize}
Diz-se da cruz heráldica, cujos braços terminam em fórma de bordão de peregrino.
\section{Bordoeira}
\begin{itemize}
\item {Grp. gram.:f.}
\end{itemize}
\begin{itemize}
\item {Utilização:Bras}
\end{itemize}
\begin{itemize}
\item {Proveniência:(De \textunderscore bordão\textunderscore ^1)}
\end{itemize}
Pancadaria, sova.
\section{Boré}
\begin{itemize}
\item {Grp. gram.:m.}
\end{itemize}
\begin{itemize}
\item {Utilização:Bras}
\end{itemize}
Trombeta ordinária, usada pela plebe em batuques.
(Do tupi)
\section{Boreal}
\begin{itemize}
\item {Grp. gram.:adj.}
\end{itemize}
\begin{itemize}
\item {Proveniência:(Lat. \textunderscore boreatis\textunderscore )}
\end{itemize}
Setentrional; que vem do Norte; que está do lado do Norte: \textunderscore regiões boreaes\textunderscore .
\section{Bóreas}
\begin{itemize}
\item {Grp. gram.:m.}
\end{itemize}
\begin{itemize}
\item {Proveniência:(Lat. \textunderscore boreas\textunderscore )}
\end{itemize}
O vento do Norte.
\section{Boreste}
\begin{itemize}
\item {Grp. gram.:m.}
\end{itemize}
\begin{itemize}
\item {Utilização:Bras}
\end{itemize}
\begin{itemize}
\item {Proveniência:(De \textunderscore bordo\textunderscore  + \textunderscore éste\textunderscore )}
\end{itemize}
O mesmo que \textunderscore estibordo\textunderscore .
\section{Borga}
\begin{itemize}
\item {Grp. gram.:f.}
\end{itemize}
\begin{itemize}
\item {Utilização:Gír.}
\end{itemize}
Pandega, estroinice.
\section{Borgo}
\begin{itemize}
\item {Grp. gram.:m.}
\end{itemize}
\begin{itemize}
\item {Utilização:Ant.}
\end{itemize}
Véu, com que os muçulmanos apparecem em público.
\section{Borgonha}
\begin{itemize}
\item {Grp. gram.:m.}
\end{itemize}
Vinho de Borgonha.
\section{Borgonhão}
\begin{itemize}
\item {Grp. gram.:m.}
\end{itemize}
O mesmo que \textunderscore borgonha\textunderscore . Cf. Camillo, \textunderscore Caveira\textunderscore , 468.
\section{Borgonhês}
\begin{itemize}
\item {Grp. gram.:adj.}
\end{itemize}
\begin{itemize}
\item {Grp. gram.:M.}
\end{itemize}
Relativo a Borgonha.
Habitante de Borgonha.
\section{Borgonhona}
\begin{itemize}
\item {Grp. gram.:f.}
\end{itemize}
\begin{itemize}
\item {Proveniência:(De \textunderscore Borgonha\textunderscore , n. p.)}
\end{itemize}
Antiga arma defensiva.
\section{Borguinhota}
\begin{itemize}
\item {Grp. gram.:f.}
\end{itemize}
\begin{itemize}
\item {Proveniência:(Fr. \textunderscore bourguignotte\textunderscore )}
\end{itemize}
Antigo capacete sem viseira.
Carapuça antiga.
\section{Bori}
\begin{itemize}
\item {Grp. gram.:m.}
\end{itemize}
Planta silvestre do Brasil.
\section{Bórico}
\begin{itemize}
\item {Grp. gram.:adj.}
\end{itemize}
\begin{itemize}
\item {Proveniência:(De \textunderscore boro\textunderscore )}
\end{itemize}
Diz-se do ácido, formado de oxygênio e boro.
\section{Borjeço}
\begin{itemize}
\item {fónica:jê}
\end{itemize}
\begin{itemize}
\item {Grp. gram.:m.}
\end{itemize}
\begin{itemize}
\item {Utilização:Prov.}
\end{itemize}
O mesmo que \textunderscore bojeço\textunderscore .
\section{Borla}
\begin{itemize}
\item {Grp. gram.:f.}
\end{itemize}
\begin{itemize}
\item {Proveniência:(De \textunderscore burrula\textunderscore , dem. do lat. \textunderscore burra\textunderscore )}
\end{itemize}
Obra de passamanaria, composta de um botão, de que pendem fios em fórma de campânula.
Barrete de doutor.
Tufo redondo, composto de fios.
Rodela, no tôpo dos paus de bandeira e dos mastros.
\section{Borla}
\begin{itemize}
\item {Grp. gram.:f.}
\end{itemize}
\begin{itemize}
\item {Utilização:Chul.}
\end{itemize}
\begin{itemize}
\item {Grp. gram.:Loc. adv.}
\end{itemize}
Burla, deixando-se de pagar o que é devido.
\textunderscore De borla\textunderscore : Gratuitamente.
(Alter. de \textunderscore burla\textunderscore )
\section{Borlador}
\begin{itemize}
\item {Grp. gram.:m.}
\end{itemize}
\begin{itemize}
\item {Utilização:Ant.}
\end{itemize}
O mesmo que \textunderscore bordador\textunderscore .
\section{Borlar}
\begin{itemize}
\item {Grp. gram.:v. t.}
\end{itemize}
\begin{itemize}
\item {Utilização:Ant.}
\end{itemize}
O mesmo que \textunderscore bordar\textunderscore .
\section{Borleta}
\begin{itemize}
\item {fónica:lê}
\end{itemize}
\begin{itemize}
\item {Grp. gram.:f.}
\end{itemize}
\begin{itemize}
\item {Proveniência:(De \textunderscore borla\textunderscore ^1)}
\end{itemize}
Pequena borla.
\section{Borlista}
\begin{itemize}
\item {Proveniência:(De \textunderscore borla\textunderscore ^2)}
\end{itemize}
\textunderscore m.\textunderscore  e \textunderscore adj. T. de Lisbôa\textunderscore .
Aquelle que costuma ir aos theatros ou divertir-se, sem pagar o que essas diversões custam.
\section{Bornaceira}
\begin{itemize}
\item {Grp. gram.:f.}
\end{itemize}
\begin{itemize}
\item {Utilização:Prov.}
\end{itemize}
\begin{itemize}
\item {Utilização:minh.}
\end{itemize}
\begin{itemize}
\item {Proveniência:(De \textunderscore borno\textunderscore )}
\end{itemize}
Tempo quente e abafado.
\section{Bornal}
\begin{itemize}
\item {Grp. gram.:m.}
\end{itemize}
\begin{itemize}
\item {Utilização:Gír. de soldado.}
\end{itemize}
Saco, em que se levam mantimentos.
Farnel.
Saco, em que se mete a cabeça das cavalgaduras, para comerem nelle.
O ânus.
(Apher. de \textunderscore embornal\textunderscore )
\section{Borne}
\begin{itemize}
\item {Grp. gram.:m.}
\end{itemize}
\begin{itemize}
\item {Utilização:Pop.}
\end{itemize}
\begin{itemize}
\item {Utilização:Prov.}
\end{itemize}
\begin{itemize}
\item {Utilização:alent.}
\end{itemize}
\begin{itemize}
\item {Utilização:Phýs.}
\end{itemize}
\begin{itemize}
\item {Proveniência:(T. fr.)}
\end{itemize}
O mesmo que \textunderscore alburno\textunderscore .
Nádegas.
Peça metállica, que se fixa num quadro ou mesa de applicações eléctricas, e tem superiormente um parafuso, que fixa o fio eléctrico que a atravessa.
\section{Bornear}
\begin{itemize}
\item {Grp. gram.:v. t.}
\end{itemize}
Alinhar com a vista.
Pôr em linha de pontaria (um canhão).
(Cp. fr. \textunderscore borne\textunderscore )
\section{Borneco}
\begin{itemize}
\item {fónica:nê}
\end{itemize}
\begin{itemize}
\item {Grp. gram.:m.}
\end{itemize}
\begin{itemize}
\item {Utilização:Prov.}
\end{itemize}
\begin{itemize}
\item {Utilização:alent.}
\end{itemize}
\begin{itemize}
\item {Proveniência:(De \textunderscore borne\textunderscore )}
\end{itemize}
Ramo, que secou em a própria árvore.
\section{Borneio}
\begin{itemize}
\item {Grp. gram.:m.}
\end{itemize}
\begin{itemize}
\item {Proveniência:(De \textunderscore bornear\textunderscore )}
\end{itemize}
Movimento circular, em que se percorre o perímetro de um objecto.
Lança antiga.
\section{Borneira}
\begin{itemize}
\item {Grp. gram.:f.}
\end{itemize}
\begin{itemize}
\item {Proveniência:(De \textunderscore borneiro\textunderscore ^1)}
\end{itemize}
Mó de pedra negra.
A pedra, de que se fazem essas mós.
\section{Borneiro}
\begin{itemize}
\item {Grp. gram.:adj.}
\end{itemize}
Diz-se de certa pedra negra, ou da mó que é feita dessa pedra.
E diz-se do trigo moído com borneira.
(Por \textunderscore bruneiro\textunderscore , de \textunderscore bruno\textunderscore )
\section{Borneiro}
\begin{itemize}
\item {Grp. gram.:m.}
\end{itemize}
\begin{itemize}
\item {Utilização:Prov.}
\end{itemize}
Buraco em o tampo do tonel ou pipa, no qual se introduz a torneira.
\section{Bornéus}
\begin{itemize}
\item {Grp. gram.:m. pl.}
\end{itemize}
Habitantes de Bornéu. Cf. \textunderscore Peregrinação\textunderscore , c. XV.
\section{Borni}
\begin{itemize}
\item {Grp. gram.:m.}
\end{itemize}
Espécie de falcão azul.
\section{Bornil}
\begin{itemize}
\item {Grp. gram.:m.}
\end{itemize}
\begin{itemize}
\item {Utilização:Prov.}
\end{itemize}
\begin{itemize}
\item {Utilização:alent.}
\end{itemize}
O mesmo que \textunderscore molhelha\textunderscore ^1.
\section{Bornim}
\begin{itemize}
\item {Grp. gram.:m.}
\end{itemize}
Antiga medida de Cananor, correspondente a 22 litros e meio proximamente.
\section{Borno}
\begin{itemize}
\item {fónica:bôr}
\end{itemize}
\begin{itemize}
\item {Grp. gram.:adj.}
\end{itemize}
\begin{itemize}
\item {Utilização:Prov.}
\end{itemize}
O mesmo que \textunderscore morno\textunderscore .
\section{Bornudo}
\begin{itemize}
\item {Grp. gram.:adj.}
\end{itemize}
Formosa ave da África oriental portuguesa.
\section{Boro}
\begin{itemize}
\item {Grp. gram.:m.}
\end{itemize}
\begin{itemize}
\item {Proveniência:(De \textunderscore bórax\textunderscore )}
\end{itemize}
Metalloide, que, no estado amorpho, é um pó escuro-esverdeado e inodoro.
\section{Boró}
\begin{itemize}
\item {Grp. gram.:m.}
\end{itemize}
\begin{itemize}
\item {Utilização:Bras}
\end{itemize}
\begin{itemize}
\item {Utilização:pop.}
\end{itemize}
Dinheiro.
\section{Borôa}
\begin{itemize}
\item {Grp. gram.:f.}
\end{itemize}
O mesmo que \textunderscore brôa\textunderscore .
\section{Bóroa}
\begin{itemize}
\item {Grp. gram.:f.}
\end{itemize}
\begin{itemize}
\item {Utilização:Ant.}
\end{itemize}
\begin{itemize}
\item {Proveniência:(Do gr. \textunderscore poros\textunderscore )}
\end{itemize}
Canal, estreito.
\section{Borococo}
\begin{itemize}
\item {fónica:côco}
\end{itemize}
\begin{itemize}
\item {Grp. gram.:m.}
\end{itemize}
Pássaro de Angola.
\section{Borocotó}
\begin{itemize}
\item {Grp. gram.:m.}
\end{itemize}
\begin{itemize}
\item {Utilização:Bras}
\end{itemize}
Terreno escabroso, escavado ou obstruído de pedras.
(Do tupi)
\section{Boroeiro}
\begin{itemize}
\item {Grp. gram.:adj.}
\end{itemize}
O mesmo que \textunderscore broeiro\textunderscore .
\section{Boror}
\begin{itemize}
\item {Grp. gram.:m.}
\end{itemize}
\begin{itemize}
\item {Utilização:Ant.}
\end{itemize}
O mesmo que \textunderscore bolór\textunderscore . Cf. \textunderscore Eufrosina\textunderscore , 118.
\section{Bororé}
\begin{itemize}
\item {Grp. gram.:m.}
\end{itemize}
Veneno, com que os indígenas brasileiros ervam as frechas.
\section{Bororós}
\begin{itemize}
\item {Grp. gram.:m.}
\end{itemize}
Antiga e poderosa nação de Índios do Brasil, a Oéste das nascentes do Araguaia, e submettida no século XVII pelo paulistano António Pires de Campos.
\section{Borós}
\begin{itemize}
\item {Grp. gram.:m.}
\end{itemize}
\begin{itemize}
\item {Utilização:bras. do N}
\end{itemize}
\begin{itemize}
\item {Utilização:Gír.}
\end{itemize}
O mesmo que \textunderscore dinheiro\textunderscore .
\section{Borotuto}
\begin{itemize}
\item {Grp. gram.:m.}
\end{itemize}
Elegante árvore bixácea de Angola, (\textunderscore cochlos permum angolense\textunderscore , Welw.).
\section{Bôrra}
\begin{itemize}
\item {Grp. gram.:f.}
\end{itemize}
\begin{itemize}
\item {Proveniência:(Lat. \textunderscore burra\textunderscore )}
\end{itemize}
Anafaia, parte que se não fia do casulo da seda.
Resíduo da seda, que se desperdiça durante a fiação.
Fezes; lia: \textunderscore a bôrra do vinho\textunderscore .
Coisa de pouco preço, bagatela.
\section{Bôrra}
\begin{itemize}
\item {Grp. gram.:m.  e  adj.}
\end{itemize}
Designação vulgar e antiga de certos frades.
\section{Bôrra}
\begin{itemize}
\item {Grp. gram.:f.}
\end{itemize}
\begin{itemize}
\item {Utilização:Prov.}
\end{itemize}
\begin{itemize}
\item {Utilização:alent.}
\end{itemize}
A fêmea do bôrro.
Ovelha de um anno.
\section{Bórra}
\begin{itemize}
\item {Grp. gram.:f.}
\end{itemize}
\begin{itemize}
\item {Utilização:Prov.}
\end{itemize}
\begin{itemize}
\item {Utilização:beir.}
\end{itemize}
Pequeno pássaro, talvez o mesmo que \textunderscore borrejo\textunderscore .
\section{Bórra-botas}
\begin{itemize}
\item {Grp. gram.:m.}
\end{itemize}
Mau engraxador de botas.
Sarrafaçal, homem reles; bigorrilha.
\section{Borraça}
\begin{itemize}
\item {Grp. gram.:f.}
\end{itemize}
O mesmo que \textunderscore borraceiro\textunderscore . (Colhido em Turquel)
\section{Borraçal}
\begin{itemize}
\item {Grp. gram.:m.}
\end{itemize}
Terra pantanosa com pastagem.
Lameiro.
Espécie de uva preta minhota.
(Cp. \textunderscore borraçar\textunderscore )
\section{Borraçar}
\begin{itemize}
\item {Grp. gram.:v. i.}
\end{itemize}
\begin{itemize}
\item {Utilização:Prov.}
\end{itemize}
\begin{itemize}
\item {Proveniência:(De \textunderscore borraça\textunderscore )}
\end{itemize}
Chuviscar.
\section{Borraceira}
\begin{itemize}
\item {Grp. gram.:f.}
\end{itemize}
Variedade de azeitona, graúda e pouco apreciada.
\section{Borraceiro}
\begin{itemize}
\item {Grp. gram.:m.}
\end{itemize}
\begin{itemize}
\item {Grp. gram.:Adj.}
\end{itemize}
\begin{itemize}
\item {Proveniência:(De \textunderscore bôrra\textunderscore ^1)}
\end{itemize}
Chuvisco.
Ligeiramente chuvoso.
Que apanhou alguma chuva.
\section{Borracha}
\begin{itemize}
\item {Grp. gram.:f.}
\end{itemize}
\begin{itemize}
\item {Utilização:Prov.}
\end{itemize}
\begin{itemize}
\item {Utilização:trasm.}
\end{itemize}
\begin{itemize}
\item {Utilização:Prov.}
\end{itemize}
\begin{itemize}
\item {Utilização:minh.}
\end{itemize}
\begin{itemize}
\item {Proveniência:(De \textunderscore bôrro\textunderscore )}
\end{itemize}
Vaso de coiro, bojudo, com bocal estreito de madeira.
Goma elástica.
Pequeno vaso, em fórma de borracha, e que serve para injecções ou clisteres.
Pedacinho de goma elástica, apropriada a apagar uma escrita.
O mesmo que \textunderscore bojega\textunderscore .
Cabaça, para conter vinho.
\section{Borracha}
\begin{itemize}
\item {Grp. gram.:f.}
\end{itemize}
\begin{itemize}
\item {Utilização:Bras. do S}
\end{itemize}
O mesmo que \textunderscore borragem\textunderscore .
\section{Borrachada}
\begin{itemize}
\item {Grp. gram.:f.}
\end{itemize}
\begin{itemize}
\item {Utilização:Bras}
\end{itemize}
\begin{itemize}
\item {Proveniência:(De \textunderscore borracha\textunderscore ^1)}
\end{itemize}
Clister, com seringa de borracha.
\section{Borrachão}
\begin{itemize}
\item {Grp. gram.:m.}
\end{itemize}
\begin{itemize}
\item {Proveniência:(De \textunderscore borracho\textunderscore )}
\end{itemize}
Homem, que se embriaga muitas vezes.
\section{Borrachão}
\begin{itemize}
\item {Grp. gram.:m.}
\end{itemize}
\begin{itemize}
\item {Utilização:Bras}
\end{itemize}
Borracha grande.
Chifre, apparelhado para conter e conduzir água ou outro líquido, e que é fechado na parte mais larga, tendo a abertura e a rolha na parte mais estreita.
\section{Borracheira}
\begin{itemize}
\item {Grp. gram.:f.}
\end{itemize}
\begin{itemize}
\item {Proveniência:(De \textunderscore borracho\textunderscore ^1)}
\end{itemize}
Bebedeira.
Palavras ou maneiras de bêbedo.
Disparate.
Grossaria.
\section{Borracheiro}
\begin{itemize}
\item {Grp. gram.:m.}
\end{itemize}
\begin{itemize}
\item {Proveniência:(De \textunderscore borracha\textunderscore ^1)}
\end{itemize}
Fabricante ou vendedor de borrachas.
Aquelle que transporta vinho em odres ou borrachas.
\section{Borrachice}
\begin{itemize}
\item {Grp. gram.:f.}
\end{itemize}
(V.borracheira)
\section{Borracho}
\begin{itemize}
\item {Grp. gram.:m.}
\end{itemize}
\begin{itemize}
\item {Proveniência:(De \textunderscore borracha\textunderscore ^1)}
\end{itemize}
O mesmo que \textunderscore bêbedo\textunderscore .
\section{Borracho}
\begin{itemize}
\item {Grp. gram.:m.}
\end{itemize}
Pombo novo, que ainda não vôa.
(Cp. \textunderscore bôrro\textunderscore )
\section{Borracho}
\begin{itemize}
\item {Grp. gram.:m.}
\end{itemize}
\begin{itemize}
\item {Utilização:Prov.}
\end{itemize}
\begin{itemize}
\item {Utilização:alent.}
\end{itemize}
Bolo de farinha e ovos, amassados com vinho branco.
(Cp. \textunderscore borracha\textunderscore ^1)
\section{Borrachudo}
\begin{itemize}
\item {Grp. gram.:adj.}
\end{itemize}
\begin{itemize}
\item {Grp. gram.:M.}
\end{itemize}
\begin{itemize}
\item {Proveniência:(De \textunderscore borracha\textunderscore ^1)}
\end{itemize}
Rotundo como uma borracha.
Mosquito do Brasil.
\section{Borrada}
\begin{itemize}
\item {Grp. gram.:f.}
\end{itemize}
Derramamento de bôrra.
Borratão.
Porcaria.
Tolíce.
Acção indecorosa.
\section{Borradela}
\begin{itemize}
\item {Grp. gram.:f.}
\end{itemize}
\begin{itemize}
\item {Proveniência:(De \textunderscore borrar\textunderscore )}
\end{itemize}
Borrão.
Ligeira mão de tinta com brocha.
Evacuação fecal de um insecto em superfície limpa.
\section{Borrador}
\begin{itemize}
\item {Grp. gram.:m.}
\end{itemize}
\begin{itemize}
\item {Proveniência:(De \textunderscore borrar\textunderscore )}
\end{itemize}
Carteira, para apontamentos ligeiros, que hão de sêr passados a limpo.
Livro, em que os negociantes inscrevem as suas operações, dia a dia, as quaes servem de base á escrituração regular.
Cadernos de esboços ou das primeiras linhas de desenhos.
Aquelle que faz o debuxo de alguma coisa; aquelle que pinta com brocha.
\section{Borradura}
\begin{itemize}
\item {Grp. gram.:f.}
\end{itemize}
Acção de \textunderscore borrar\textunderscore .
\section{Borragem}
\begin{itemize}
\item {Grp. gram.:f.}
\end{itemize}
\begin{itemize}
\item {Proveniência:(Lat. \textunderscore borago\textunderscore )}
\end{itemize}
Planta herbácea medicinal.
\section{Borragináceas}
\begin{itemize}
\item {Grp. gram.:f. pl.}
\end{itemize}
O mesmo ou melhor que \textunderscore borragíneas\textunderscore .
\section{Borragíneas}
\begin{itemize}
\item {Grp. gram.:f. pl.}
\end{itemize}
Família de plantas herbáceas, a que serve de typo a borragem.
\section{Borraina}
\begin{itemize}
\item {Grp. gram.:f.}
\end{itemize}
\begin{itemize}
\item {Proveniência:(De \textunderscore bôrra\textunderscore ^2)}
\end{itemize}
Estôfo de tomentos, nos arções da sella.
Debrum, que se faz nas fôlhas de chumbo, quando se quere uni-las sem soldadura.
\section{Borrainha}
\begin{itemize}
\item {Grp. gram.:f.}
\end{itemize}
(V.borraina)
\section{Borralha}
\begin{itemize}
\item {Grp. gram.:f.}
\end{itemize}
Borralho.
Cinzas quentes; cinzas.
(Cp. \textunderscore borralho\textunderscore ^1)
\section{Borralheira}
\begin{itemize}
\item {Grp. gram.:f.}
\end{itemize}
\begin{itemize}
\item {Utilização:T. de Moncorvo}
\end{itemize}
\begin{itemize}
\item {Proveniência:(De \textunderscore borralha\textunderscore )}
\end{itemize}
Lugar, onde se junta a borralha da cozinha ou do forno.
Coirela estreita, entre duas paredes.
\section{Borralheiro}
\begin{itemize}
\item {Grp. gram.:adj.}
\end{itemize}
\begin{itemize}
\item {Grp. gram.:M.}
\end{itemize}
\begin{itemize}
\item {Proveniência:(De \textunderscore borralha\textunderscore )}
\end{itemize}
Que gosta de estar junto do borralho.
Que gosta pouco de sair de casa.
O mesmo que \textunderscore borralheira\textunderscore .
\section{Borralhento}
\begin{itemize}
\item {Grp. gram.:adj.}
\end{itemize}
Que é da côr da borralha; cinzento. Cf. J. Dinis, \textunderscore Morgadinha\textunderscore , 68.
\section{Borralho}
\begin{itemize}
\item {Grp. gram.:m.}
\end{itemize}
\begin{itemize}
\item {Proveniência:(De \textunderscore bôrra\textunderscore ^1)}
\end{itemize}
Brasido quási extinto; cinzas quentes.
Lar, lareira.
\section{Borralho}
\begin{itemize}
\item {Grp. gram.:adj.}
\end{itemize}
\begin{itemize}
\item {Proveniência:(De \textunderscore borralha\textunderscore )}
\end{itemize}
Diz-se do toiro, que tem côr de cinza.
\section{Borra-môsca}
\begin{itemize}
\item {Grp. gram.:f.}
\end{itemize}
Casta de uva do districto de Leiria.
\section{Borrão}
\begin{itemize}
\item {Grp. gram.:m.}
\end{itemize}
\begin{itemize}
\item {Proveniência:(De \textunderscore borrar\textunderscore )}
\end{itemize}
Mancha de tinta.
Rascunho, minuta.
Borrador.
Acção indecorosa.
Debuxo.
\section{Borra-portas}
\begin{itemize}
\item {Grp. gram.:m.}
\end{itemize}
\begin{itemize}
\item {Utilização:Prov.}
\end{itemize}
\begin{itemize}
\item {Utilização:trasm.}
\end{itemize}
\begin{itemize}
\item {Utilização:Pop.}
\end{itemize}
O mesmo que \textunderscore caiador\textunderscore .
\section{Borrar}
\begin{itemize}
\item {Grp. gram.:v. t.}
\end{itemize}
\begin{itemize}
\item {Utilização:Pleb.}
\end{itemize}
\begin{itemize}
\item {Utilização:Pleb.}
\end{itemize}
\begin{itemize}
\item {Proveniência:(De \textunderscore bôrra\textunderscore ^1)}
\end{itemize}
Deitar borrões em.
Sujar.
Apagar.
Rabiscar.
Sujar com matérias fecaes.
Defecar, descomer. Cf. Macedo, \textunderscore Burros\textunderscore , 213.
\section{Borras-botas}
\begin{itemize}
\item {Grp. gram.:m.}
\end{itemize}
\begin{itemize}
\item {Utilização:Bras}
\end{itemize}
O mesmo que \textunderscore bórra-botas\textunderscore .
\section{Borrasca}
\begin{itemize}
\item {Grp. gram.:f.}
\end{itemize}
\begin{itemize}
\item {Utilização:Fig.}
\end{itemize}
Tempestade marítima, com vento e chuva.
Temporal.
Furacão.
Contrariedades súbitas.
Accesso de cólera, com palavras e gestos desordenados.
(Cast. \textunderscore borrasca\textunderscore )
\section{Borrascoso}
\begin{itemize}
\item {Grp. gram.:adj.}
\end{itemize}
Em que há borrascas; que traz borrascas.
\section{Borratada}
\begin{itemize}
\item {Grp. gram.:f.}
\end{itemize}
O mesmo que \textunderscore borratão\textunderscore .
\section{Borratão}
\begin{itemize}
\item {Grp. gram.:m.}
\end{itemize}
Borrão de tinta.
Tinta alastrada.
\section{Borratar}
\textunderscore v. t.\textunderscore  (e der)
O mesmo que \textunderscore borretear\textunderscore .
\section{Borrefa}
\begin{itemize}
\item {Grp. gram.:f.}
\end{itemize}
\begin{itemize}
\item {Utilização:Prov.}
\end{itemize}
\begin{itemize}
\item {Utilização:alg.}
\end{itemize}
\begin{itemize}
\item {Utilização:alent.}
\end{itemize}
Doença no gado.
Espécie de tumor.
Pústula ou vesícula, resultante de queimadura.
O mesmo que \textunderscore bejoga\textunderscore .
\section{Borrefo}
\begin{itemize}
\item {Grp. gram.:m.}
\end{itemize}
\begin{itemize}
\item {Utilização:Prov.}
\end{itemize}
\begin{itemize}
\item {Utilização:alent.}
\end{itemize}
O mesmo que \textunderscore borrelfo\textunderscore .
\section{Borrêga}
\begin{itemize}
\item {Grp. gram.:f.}
\end{itemize}
\begin{itemize}
\item {Utilização:Prov.}
\end{itemize}
\begin{itemize}
\item {Utilização:alent.}
\end{itemize}
Ovelha nova.
O mesmo que \textunderscore chiba\textunderscore ^2.
(Fem. de \textunderscore borrego\textunderscore )
\section{Borregada}
\begin{itemize}
\item {Grp. gram.:f.}
\end{itemize}
\begin{itemize}
\item {Utilização:Ant.}
\end{itemize}
Rebanho de borregos.
O mesmo que \textunderscore reprehensão\textunderscore . Cf. G. Vicente, \textunderscore Auto da Barca\textunderscore .
\section{Borregar}
\begin{itemize}
\item {Grp. gram.:v. i.}
\end{itemize}
Gritar como borrêgo; berregar.
\section{Borregata}
\begin{itemize}
\item {Grp. gram.:f.}
\end{itemize}
\begin{itemize}
\item {Utilização:Prov.}
\end{itemize}
\begin{itemize}
\item {Utilização:alg.}
\end{itemize}
\begin{itemize}
\item {Proveniência:(De \textunderscore borrêgo\textunderscore ^1)}
\end{itemize}
Nome de um peixe.
\section{Borrêgo}
\begin{itemize}
\item {Grp. gram.:m.}
\end{itemize}
\begin{itemize}
\item {Utilização:Fam.}
\end{itemize}
\begin{itemize}
\item {Proveniência:(Do lat. \textunderscore burrus\textunderscore ?)}
\end{itemize}
Cordeiro, que não tem mais de um anno.
Pessôa sossegada; pacífica.
\section{Borrêgo}
\begin{itemize}
\item {Grp. gram.:m.}
\end{itemize}
\begin{itemize}
\item {Utilização:Prov.}
\end{itemize}
\begin{itemize}
\item {Utilização:alent.}
\end{itemize}
\begin{itemize}
\item {Proveniência:(De \textunderscore borra\textunderscore ^1?)}
\end{itemize}
Pequena porção de coalhada.
\section{Borrêgo}
\begin{itemize}
\item {Grp. gram.:m.}
\end{itemize}
\begin{itemize}
\item {Utilização:Prov.}
\end{itemize}
\begin{itemize}
\item {Utilização:alent.}
\end{itemize}
Acção, havida como grosseira, mas ás vezes opportuna e significativa.
(Talvez por \textunderscore burrego\textunderscore , de \textunderscore burro\textunderscore )
\section{Borregueiro}
\begin{itemize}
\item {Grp. gram.:m.}
\end{itemize}
\begin{itemize}
\item {Proveniência:(De \textunderscore borrêgo\textunderscore ^1)}
\end{itemize}
Pastor de borregos.
\section{Borreguice}
\begin{itemize}
\item {Grp. gram.:f.}
\end{itemize}
\begin{itemize}
\item {Utilização:P. us.}
\end{itemize}
\begin{itemize}
\item {Proveniência:(De \textunderscore borrêgo\textunderscore ^3?)}
\end{itemize}
Pacovice; estolidez; indolência.
\section{Borreira}
\begin{itemize}
\item {Grp. gram.:adj. f.}
\end{itemize}
\begin{itemize}
\item {Utilização:Prov.}
\end{itemize}
\begin{itemize}
\item {Utilização:trasm.}
\end{itemize}
\begin{itemize}
\item {Proveniência:(De \textunderscore borrar\textunderscore )}
\end{itemize}
Diz-se de uma variedade de azeitona.
\section{Borreiro}
\begin{itemize}
\item {Grp. gram.:m.}
\end{itemize}
\begin{itemize}
\item {Utilização:T. da Bairrada}
\end{itemize}
\begin{itemize}
\item {Proveniência:(De \textunderscore bôrra\textunderscore ^1)}
\end{itemize}
Lugar, onde se juntam bôrras.
Buraco, no fundo da caldeira do alambique, por onde êste se limpa das bôrras.
\section{Borrejo}
\begin{itemize}
\item {Grp. gram.:m.}
\end{itemize}
\begin{itemize}
\item {Utilização:Prov.}
\end{itemize}
\begin{itemize}
\item {Utilização:beir.}
\end{itemize}
\begin{itemize}
\item {Proveniência:(Do lat. \textunderscore burrus\textunderscore ?)}
\end{itemize}
O mesmo que \textunderscore cartaxo\textunderscore , pássaro.
\section{Borrelfo}
\begin{itemize}
\item {Grp. gram.:m.}
\end{itemize}
\begin{itemize}
\item {Utilização:Prov.}
\end{itemize}
\begin{itemize}
\item {Utilização:alent.}
\end{itemize}
Ave implume.
\section{Borrelho}
\begin{itemize}
\item {fónica:rê}
\end{itemize}
\begin{itemize}
\item {Grp. gram.:m.}
\end{itemize}
\begin{itemize}
\item {Proveniência:(De \textunderscore bôrra\textunderscore ^1?)}
\end{itemize}
Ave palmípede, aquática.
\section{Borrento}
\begin{itemize}
\item {Grp. gram.:adj.}
\end{itemize}
\begin{itemize}
\item {Proveniência:(De \textunderscore bôrra\textunderscore ^1)}
\end{itemize}
Que tem \textunderscore bôrras\textunderscore .
\section{Borréria}
\begin{itemize}
\item {Grp. gram.:f.}
\end{itemize}
\begin{itemize}
\item {Proveniência:(De \textunderscore Borrer\textunderscore , n. p.)}
\end{itemize}
Gênero de plantas rubiaceas.
\section{Borretear}
\begin{itemize}
\item {Grp. gram.:v. t.}
\end{itemize}
\begin{itemize}
\item {Utilização:P. us.}
\end{itemize}
Traçar muitas vezes (um desenho).
Amiudar (linhas), fazendo umas e apagando outras.
(Cp. \textunderscore borrar\textunderscore )
\section{Borriçar}
\begin{itemize}
\item {Grp. gram.:v. i.}
\end{itemize}
\begin{itemize}
\item {Proveniência:(De \textunderscore borriço\textunderscore )}
\end{itemize}
Chuviscar.
\section{Borriço}
\begin{itemize}
\item {Grp. gram.:m.}
\end{itemize}
O mesmo que \textunderscore borraceiro\textunderscore .
\section{Borrifador}
\begin{itemize}
\item {Grp. gram.:m.}
\end{itemize}
Aquelle que borrifa.
Utensílio de fôlha, para rega de jardins; regador.
\section{Borrifar}
\begin{itemize}
\item {Grp. gram.:v. t.}
\end{itemize}
\begin{itemize}
\item {Grp. gram.:V. i.}
\end{itemize}
Molhar com borrifos, salpicar com gotas.
Orvalhar.
Chuviscar.
\section{Borrifo}
\begin{itemize}
\item {Grp. gram.:m.}
\end{itemize}
\begin{itemize}
\item {Grp. gram.:Pl.}
\end{itemize}
Pequenas gotas de chuva.
Diffusão de gotas.
Conjunto de pequenos fios de águas, que passam pelo crivo do borrifador (utensílio).
Salpicos.
Pequenos pontos ou pequenas manchas, que imitam gotas.
(Cp. \textunderscore borriço\textunderscore )
\section{Borriscada}
\begin{itemize}
\item {Grp. gram.:f.}
\end{itemize}
\begin{itemize}
\item {Utilização:Ant.}
\end{itemize}
O mesmo que \textunderscore borrasca\textunderscore .
\section{Bôrro}
\begin{itemize}
\item {Grp. gram.:m.}
\end{itemize}
\begin{itemize}
\item {Proveniência:(Lat. \textunderscore burrus\textunderscore )}
\end{itemize}
Carneiro, entre um e dois annos de idade.
\section{Borroso}
\begin{itemize}
\item {Grp. gram.:adj.}
\end{itemize}
Diz-se do centeio mais alentado e limpo.
\section{Borsa}
\begin{itemize}
\item {fónica:bôr}
\end{itemize}
\begin{itemize}
\item {Grp. gram.:f.}
\end{itemize}
\begin{itemize}
\item {Utilização:Ant.}
\end{itemize}
O mesmo que \textunderscore bôlsa\textunderscore .
\section{Bortalá}
\begin{itemize}
\item {Grp. gram.:m.}
\end{itemize}
\begin{itemize}
\item {Utilização:Bras}
\end{itemize}
Biôco, para assustar crianças.
\section{Borússico}
\begin{itemize}
\item {Grp. gram.:m.}
\end{itemize}
O mesmo que \textunderscore borusso\textunderscore .
\section{Borusso}
\begin{itemize}
\item {Grp. gram.:m.}
\end{itemize}
Língua esclavónica, que se falava na Prússia.
\section{Borzegui}
\begin{itemize}
\item {Grp. gram.:m.}
\end{itemize}
O mesmo que \textunderscore borzeguim\textunderscore .
\section{Borzeguiada}
\begin{itemize}
\item {Grp. gram.:f.}
\end{itemize}
Pancada com \textunderscore borzegui\textunderscore .
\section{Borzeguieiro}
\begin{itemize}
\item {Grp. gram.:m.}
\end{itemize}
Fabricante de borzeguis.
\section{Borzeguim}
\begin{itemize}
\item {Grp. gram.:m.}
\end{itemize}
Antiga espécie de bota, com atacadores.
Meia grossa, com sola de coiro, usada pelos Moiros.
(Talvez do neerl. \textunderscore brosekin\textunderscore , dem. de \textunderscore broos\textunderscore , botim)
\section{Borzeguineiro}
\begin{itemize}
\item {Grp. gram.:m.}
\end{itemize}
O mesmo que \textunderscore borzeguieiro\textunderscore .
\section{Borzoleta}
\begin{itemize}
\item {Grp. gram.:f.}
\end{itemize}
(V.barjoleta)
\section{Bosboque}
\begin{itemize}
\item {Grp. gram.:m.}
\end{itemize}
(V.bisonte)
\section{Bosca}
\begin{itemize}
\item {Grp. gram.:f.}
\end{itemize}
Rêde, em fórma de cóne, para a pesca da lagosta e do lavagante.
\section{Boscagem}
\begin{itemize}
\item {Grp. gram.:f.}
\end{itemize}
\begin{itemize}
\item {Proveniência:(De \textunderscore bosque\textunderscore )}
\end{itemize}
Representação de bosques, em pintura.
Conjunto de árvores, bosque.
\section{Boscarejo}
\begin{itemize}
\item {Grp. gram.:adj.}
\end{itemize}
Relativo a bosques.
\section{Bosque}
\begin{itemize}
\item {Grp. gram.:m.}
\end{itemize}
\begin{itemize}
\item {Proveniência:(Do germ. \textunderscore bosk\textunderscore )}
\end{itemize}
Grande arvoredo.
Mata, floresta.
Reunião de muitas coisas, que dão ideia de árvores ou varas.
\section{Bosquejar}
\begin{itemize}
\item {Grp. gram.:v. t.}
\end{itemize}
\begin{itemize}
\item {Proveniência:(De \textunderscore bosque\textunderscore )}
\end{itemize}
Pintar, sem contornar com rigor.
Descrever a traços largos.
Resumir, synthetizar: \textunderscore bosquejar um episódio\textunderscore .
Planear; esboçar.
\section{Bosquejar}
\begin{itemize}
\item {Grp. gram.:v. i.}
\end{itemize}
\begin{itemize}
\item {Proveniência:(De \textunderscore bosca\textunderscore )}
\end{itemize}
Manobrar, suspendendo as boscas.
\section{Bosquejo}
\begin{itemize}
\item {Grp. gram.:m.}
\end{itemize}
Acção de \textunderscore bosquejar\textunderscore ^1.
\section{Bosquerejar}
\begin{itemize}
\item {Grp. gram.:v. i.}
\end{itemize}
Andar nos bosques. Cf. \textunderscore Viriato Trág.\textunderscore , I, 38.
(Por \textunderscore boscarejar\textunderscore , de \textunderscore boscarejo\textunderscore )
\section{Bosquete}
\begin{itemize}
\item {fónica:quê}
\end{itemize}
\begin{itemize}
\item {Grp. gram.:m.}
\end{itemize}
Pequeno bosque.
\section{Bossa}
\begin{itemize}
\item {Grp. gram.:f.}
\end{itemize}
Tumor, resultante de contusão.
Protuberância craniana, considerada como indício de certa tendência ou aptidão.
Aptidão.
Carcunda.
Protuberância boleada de alguns ossos.
Protuberância no dorso de alguns animaes.
Pequena elevação numa superfície.
(Do alto al. méd. \textunderscore butre\textunderscore )
\section{Bossada}
\begin{itemize}
\item {Grp. gram.:f.}
\end{itemize}
\begin{itemize}
\item {Utilização:Prov.}
\end{itemize}
\begin{itemize}
\item {Utilização:minh.}
\end{itemize}
O mesmo que \textunderscore vessada\textunderscore ^1.
\section{Bossagem}
\begin{itemize}
\item {Grp. gram.:f.}
\end{itemize}
\begin{itemize}
\item {Proveniência:(De \textunderscore bossa\textunderscore )}
\end{itemize}
Parte de um edifício, que resai do prumo ou da superfície.
\section{Bosta}
\begin{itemize}
\item {Grp. gram.:f.}
\end{itemize}
Excremento do gado vacum.
(Or. incerta)
\section{Bostal}
\begin{itemize}
\item {Grp. gram.:m.}
\end{itemize}
\begin{itemize}
\item {Proveniência:(De \textunderscore bosta\textunderscore )}
\end{itemize}
Curral de gado vacum.
\section{Bostar}
\begin{itemize}
\item {Grp. gram.:v. t.}
\end{itemize}
\begin{itemize}
\item {Grp. gram.:V. i.}
\end{itemize}
\begin{itemize}
\item {Proveniência:(De \textunderscore bosta\textunderscore )}
\end{itemize}
Embostar.
Evacuar bosta.
\section{Bostear}
\begin{itemize}
\item {Grp. gram.:v. t.}
\end{itemize}
\begin{itemize}
\item {Utilização:T. da Índia Port}
\end{itemize}
Revestir de bosta (as paredes).
\section{Bosteira}
\begin{itemize}
\item {Grp. gram.:f.}
\end{itemize}
Bosta; acervo de bosta.
\section{Bosteiro}
\begin{itemize}
\item {Grp. gram.:m.}
\end{itemize}
O mesmo que \textunderscore escaravêlho\textunderscore .
\section{Bostela}
\begin{itemize}
\item {Grp. gram.:f.}
\end{itemize}
\begin{itemize}
\item {Proveniência:(Do b. lat. \textunderscore pustella\textunderscore )}
\end{itemize}
Pequena ferida com crosta.
\section{Bostelento}
\begin{itemize}
\item {Grp. gram.:adj.}
\end{itemize}
Que tem bostelas.
\section{Bostelo}
\begin{itemize}
\item {Grp. gram.:m.}
\end{itemize}
\begin{itemize}
\item {Utilização:Des.}
\end{itemize}
Pequeno bosque, tapada.
\section{Bostoiro}
\begin{itemize}
\item {Grp. gram.:m.}
\end{itemize}
\begin{itemize}
\item {Utilização:Prov.}
\end{itemize}
\begin{itemize}
\item {Utilização:trasm.}
\end{itemize}
\begin{itemize}
\item {Proveniência:(De \textunderscore bosta\textunderscore )}
\end{itemize}
Indivíduo inútil, sem préstimo, que só serve para estercar terras.
\section{Bostouro}
\begin{itemize}
\item {Grp. gram.:m.}
\end{itemize}
\begin{itemize}
\item {Utilização:Prov.}
\end{itemize}
\begin{itemize}
\item {Utilização:trasm.}
\end{itemize}
\begin{itemize}
\item {Proveniência:(De \textunderscore bosta\textunderscore )}
\end{itemize}
Indivíduo inútil, sem préstimo, que só serve para estercar terras.
\section{Bóstrico}
\begin{itemize}
\item {Grp. gram.:m.}
\end{itemize}
\begin{itemize}
\item {Proveniência:(Gr. \textunderscore bostrukhos\textunderscore )}
\end{itemize}
Insecto coleóptero tetrâmero.
\section{Bóstrycho}
\begin{itemize}
\item {fónica:co}
\end{itemize}
\begin{itemize}
\item {Grp. gram.:m.}
\end{itemize}
\begin{itemize}
\item {Proveniência:(Gr. \textunderscore bostrukhos\textunderscore )}
\end{itemize}
Insecto coleóptero tetrâmero.
\section{Bota}
\begin{itemize}
\item {Grp. gram.:f.}
\end{itemize}
Calçado, que envolve o pé e parte da perna.
(Segundo Littré, do b. lat. \textunderscore butta\textunderscore , borracha ou vasilha de vinho, pela sua analogia com a \textunderscore bota\textunderscore , calçado)
\section{Bota}
\begin{itemize}
\item {Grp. gram.:f.}
\end{itemize}
\begin{itemize}
\item {Utilização:Des.}
\end{itemize}
\begin{itemize}
\item {Proveniência:(Do b. lat. \textunderscore butta\textunderscore )}
\end{itemize}
Borracha.
Saco de coiro.
Vasilha para vinho.
\section{Bota}
\begin{itemize}
\item {Grp. gram.:f.}
\end{itemize}
\begin{itemize}
\item {Utilização:Pop.}
\end{itemize}
Mentirola.
\section{Bota}
\begin{itemize}
\item {Grp. gram.:m.}
\end{itemize}
\begin{itemize}
\item {Utilização:Prov.}
\end{itemize}
\begin{itemize}
\item {Utilização:alent.}
\end{itemize}
\begin{itemize}
\item {Utilização:Bras}
\end{itemize}
\begin{itemize}
\item {Utilização:fam.}
\end{itemize}
Trabalhador do norte, que não vem contratado, e que se apresenta como jornaleiro.
Composição ruim de pintor, gravador, etc.
\section{Bota}
\begin{itemize}
\item {Grp. gram.:f.}
\end{itemize}
Árvore do Congo.
\section{Botadela}
\begin{itemize}
\item {Grp. gram.:f.}
\end{itemize}
\begin{itemize}
\item {Proveniência:(De \textunderscore botar\textunderscore )}
\end{itemize}
Última preparação da marinha, para a crystallização do chloreto de sódio.
\section{Botado}
\begin{itemize}
\item {Grp. gram.:adj.}
\end{itemize}
\begin{itemize}
\item {Utilização:Prov.}
\end{itemize}
\begin{itemize}
\item {Utilização:minh.}
\end{itemize}
O mesmo que \textunderscore turvo\textunderscore , \textunderscore corrompido\textunderscore , (falando-se do vinho).
(Talvez corr. de \textunderscore voltado\textunderscore )
\section{Bota-fogo}
\begin{itemize}
\item {Grp. gram.:m.}
\end{itemize}
\begin{itemize}
\item {Utilização:Fig.}
\end{itemize}
Pau, em que está o morrão, com que se chega fogo á peça ou aos candeeiros da illuminação pública.
Artilheiro, que chega o morrão á peça.
Pessôa, que provoca desordens.
Aquelle que se irrita facilmente.
\section{Bota-fóra}
\begin{itemize}
\item {Grp. gram.:m.}
\end{itemize}
\begin{itemize}
\item {Utilização:Pop.}
\end{itemize}
Lançamento de um navio á água.
Acto ou festa, com que se despede alguém, acompanhando-o até o momento da partida.
Actividade.
Desperdício.
\section{Botalós}
\begin{itemize}
\item {Grp. gram.:m. pl.}
\end{itemize}
\begin{itemize}
\item {Utilização:Náut.}
\end{itemize}
Paus, com ferros de três bicos, para vários serviços a bordo.
(Cp. cast. \textunderscore botalón\textunderscore )
\section{Botana}
\begin{itemize}
\item {Grp. gram.:f.}
\end{itemize}
\begin{itemize}
\item {Utilização:Prov.}
\end{itemize}
\begin{itemize}
\item {Utilização:alent.}
\end{itemize}
\begin{itemize}
\item {Proveniência:(T. cast.)}
\end{itemize}
Rodela ou espécie de botão, com que se tapa a rotura de um odre, collocando-a interiormente e franzindo por baixo della o coiro.
\section{Botânica}
\begin{itemize}
\item {Grp. gram.:f.}
\end{itemize}
\begin{itemize}
\item {Proveniência:(Gr. \textunderscore botanike\textunderscore )}
\end{itemize}
Sciência, que trata dos vegetaes, descrevendo-os, classificando-os, etc.
\section{Botânico}
\begin{itemize}
\item {Grp. gram.:m.}
\end{itemize}
\begin{itemize}
\item {Grp. gram.:Adj.}
\end{itemize}
Aquelle que se dedica á Botânica.
Relativo á Botânica.
\section{Botanófago}
\begin{itemize}
\item {Grp. gram.:m.  e  adj.}
\end{itemize}
\begin{itemize}
\item {Proveniência:(Do gr. \textunderscore botane\textunderscore  + \textunderscore phagein\textunderscore )}
\end{itemize}
O que se alimenta de vegetaes.
\section{Botanófilo}
\begin{itemize}
\item {Grp. gram.:m.}
\end{itemize}
\begin{itemize}
\item {Proveniência:(Do gr. \textunderscore botane\textunderscore  + \textunderscore philos\textunderscore )}
\end{itemize}
Aquelle que é apaixonado pela Botânica.
\section{Botanografia}
\begin{itemize}
\item {Grp. gram.:f.}
\end{itemize}
Descripção das plantas.
(Cp. \textunderscore botanógrapho\textunderscore )
\section{Botanógrafo}
\begin{itemize}
\item {Grp. gram.:m.}
\end{itemize}
\begin{itemize}
\item {Proveniência:(Do greg. \textunderscore botane\textunderscore  + \textunderscore graphein\textunderscore )}
\end{itemize}
Aquelle que descreve scientificamente as plantas.
\section{Botanographia}
\begin{itemize}
\item {Grp. gram.:f.}
\end{itemize}
Descripção das plantas.
(Cp. \textunderscore botanógrapho\textunderscore )
\section{Botanógrapho}
\begin{itemize}
\item {Grp. gram.:m.}
\end{itemize}
\begin{itemize}
\item {Proveniência:(Do greg. \textunderscore botane\textunderscore  + \textunderscore graphein\textunderscore )}
\end{itemize}
Aquelle que descreve scientificamente as plantas.
\section{Botanologia}
\begin{itemize}
\item {Grp. gram.:f.}
\end{itemize}
\begin{itemize}
\item {Proveniência:(Do gr. \textunderscore botane\textunderscore  + \textunderscore logos\textunderscore )}
\end{itemize}
O mesmo que \textunderscore Botânica\textunderscore .
\section{Botanomancia}
\begin{itemize}
\item {Grp. gram.:f.}
\end{itemize}
\begin{itemize}
\item {Proveniência:(Do gr. \textunderscore botane\textunderscore  + \textunderscore manteia\textunderscore )}
\end{itemize}
Supposto processo de adivinhação por meio de vegetaes.
\section{Botanometria}
\begin{itemize}
\item {Grp. gram.:f.}
\end{itemize}
O mesmo que \textunderscore phyllotaxia\textunderscore .
\section{Botanóphago}
\begin{itemize}
\item {Grp. gram.:m.  e  adj.}
\end{itemize}
\begin{itemize}
\item {Proveniência:(Do gr. \textunderscore botane\textunderscore  + \textunderscore phagein\textunderscore )}
\end{itemize}
O que se alimenta de vegetaes.
\section{Botanóphilo}
\begin{itemize}
\item {Grp. gram.:m.}
\end{itemize}
\begin{itemize}
\item {Proveniência:(Do gr. \textunderscore botane\textunderscore  + \textunderscore philos\textunderscore )}
\end{itemize}
Aquelle que é apaixonado pela Botânica.
\section{Botão}
\begin{itemize}
\item {Grp. gram.:m.}
\end{itemize}
\begin{itemize}
\item {Proveniência:(Do rad. de \textunderscore botar\textunderscore ^1)}
\end{itemize}
Estado da flôr antes de desabrochar.
Pequena protuberância, que contém os rudimentos das hastes, fôlhas ou órgãos de fructificação.
Pequena peça, quási sempre arredondada, que se usa como ornato no vestuário ou para o fechar, entrando numa abertura chamada casa ou em uma aselha.
Peça redonda, com que se abre janela, porta ou gaveta, a que está presa por um espigão.
Bóla, que se põe na ponta do florete, para que êste não fira.
Verruga, na pelle.
Bostela.
Pequena bóla de ferro que, depois de escandecida, se applica como cautério.
Brinco de orelha, em fórma de pequena bóla, sem pingente.
Placa ou pequeno objecto, que está em communicação com uma campainha, e que se impelle ou comprime, para que ella toque.
Aquillo que é viçoso, tenro, ou que está ainda por desenvolver: \textunderscore rosa em botão\textunderscore .
Objecto, que tenha semelhança com o botão vegetal ou os botões do vestuário.
Espécie de jôgo popular.
\section{Botão-de-oiro}
\begin{itemize}
\item {Grp. gram.:m.}
\end{itemize}
Espécie de ranúnculo.
\section{Botar}
\begin{itemize}
\item {Grp. gram.:v. t.}
\end{itemize}
Atirar, deitar.
Lançar fóra; repellir.
(B. lat. \textunderscore botare\textunderscore , do germ.)
\section{Botar}
\begin{itemize}
\item {Grp. gram.:v. t.}
\end{itemize}
\begin{itemize}
\item {Utilização:Fig.}
\end{itemize}
Tornar bôto: \textunderscore botar os dentes\textunderscore .
Engrossar o fio ou gume de; tirar o gume a: \textunderscore botar uma faca\textunderscore .
Tirar a fôrça a.
Tornar insensível.(V.embotar). Cf. \textunderscore Eufrosina\textunderscore , 336.
\section{Botar}
\begin{itemize}
\item {Grp. gram.:v. t.}
\end{itemize}
(V.desbotar)
\section{Botaréo}
\begin{itemize}
\item {Grp. gram.:m.}
\end{itemize}
\begin{itemize}
\item {Proveniência:(Do cast. \textunderscore botarel\textunderscore )}
\end{itemize}
Contraforte; pegão; pilastra de refôrço.
Muro que, nas propriedades rústicas, sustenta a pressão dos terrenos declives.
\section{Botaréu}
\begin{itemize}
\item {Grp. gram.:m.}
\end{itemize}
\begin{itemize}
\item {Proveniência:(Do cast. \textunderscore botarel\textunderscore )}
\end{itemize}
Contraforte; pegão; pilastra de refôrço.
Muro que, nas propriedades rústicas, sustenta a pressão dos terrenos declives.
\section{Bota-sella}
\begin{itemize}
\item {Grp. gram.:f.}
\end{itemize}
Sinal de cornetas, para que os soldados de cavallaria arreiem os cavallos.
\section{Bote}
\begin{itemize}
\item {Grp. gram.:m.}
\end{itemize}
\begin{itemize}
\item {Utilização:Gír.}
\end{itemize}
\begin{itemize}
\item {Proveniência:(Do ingl. \textunderscore boat\textunderscore )}
\end{itemize}
Pequena embarcação de remos, para navegação nos rios ou para serviço nos portos.
Escaler.
Nádegas.
\section{Bote}
\begin{itemize}
\item {Grp. gram.:m.}
\end{itemize}
\begin{itemize}
\item {Proveniência:(De \textunderscore botar\textunderscore ^1)}
\end{itemize}
Golpe com arma branca.
Cutilada.
Censura.
Successo desastroso.
\section{Botefa}
\begin{itemize}
\item {Grp. gram.:f.}
\end{itemize}
\begin{itemize}
\item {Utilização:Prov.}
\end{itemize}
\begin{itemize}
\item {Utilização:trasm.}
\end{itemize}
O mesmo que \textunderscore colondro\textunderscore .
\section{Boteifa}
\begin{itemize}
\item {Grp. gram.:f.}
\end{itemize}
\begin{itemize}
\item {Utilização:Prov.}
\end{itemize}
\begin{itemize}
\item {Utilização:minh.}
\end{itemize}
O mesmo que \textunderscore botefa\textunderscore .
\section{Boteina}
\begin{itemize}
\item {Grp. gram.:f.}
\end{itemize}
\begin{itemize}
\item {Utilização:Prov.}
\end{itemize}
\begin{itemize}
\item {Utilização:trasm.}
\end{itemize}
O mesmo que \textunderscore botana\textunderscore .
\section{Boteladura}
\textunderscore f.\textunderscore  (?)«\textunderscore correndo a nao á vontade do vento, com o trapear que fez abriu pela proa, pela boteladura, por onde, lançando fora a estopa...\textunderscore »Couto, \textunderscore Paulo de Lima\textunderscore .
\section{Botelha}
\begin{itemize}
\item {fónica:tê}
\end{itemize}
\begin{itemize}
\item {Grp. gram.:f.}
\end{itemize}
Garrafa.
Vinho, contido numa garrafa.
(B. lat. \textunderscore butticula\textunderscore , de \textunderscore butta\textunderscore )
\section{Botelha}
\begin{itemize}
\item {fónica:tê}
\end{itemize}
\begin{itemize}
\item {Grp. gram.:f.}
\end{itemize}
\begin{itemize}
\item {Utilização:Prov.}
\end{itemize}
\begin{itemize}
\item {Utilização:alent.}
\end{itemize}
\begin{itemize}
\item {Utilização:Prov.}
\end{itemize}
\begin{itemize}
\item {Utilização:minh.}
\end{itemize}
Massiço ou ilhota de mato na charneca.
Espécie de alga, que se apanha para alimento de animaes.
Variedade de pêra portuguesa, hoje desconhecida.
Espécie de abóbora ou cabaço; botefa.
\section{Botelharia}
\begin{itemize}
\item {Grp. gram.:f.}
\end{itemize}
\begin{itemize}
\item {Proveniência:(De \textunderscore botelha\textunderscore ^1)}
\end{itemize}
Frasqueira.
Antigo cargo de botelheiro.
\section{Botelheira}
\begin{itemize}
\item {Grp. gram.:f.}
\end{itemize}
Casta de uva branca.
\section{Botelheiro}
\begin{itemize}
\item {Grp. gram.:m.}
\end{itemize}
\begin{itemize}
\item {Proveniência:(De \textunderscore botelha\textunderscore ^1)}
\end{itemize}
Aquelle que, nas casas nobres, tinha a seu cargo a administração da frasqueira.
\section{Botelho}
\begin{itemize}
\item {fónica:tê}
\end{itemize}
\begin{itemize}
\item {Grp. gram.:m.}
\end{itemize}
\begin{itemize}
\item {Utilização:T. da Guarda}
\end{itemize}
\begin{itemize}
\item {Proveniência:(De \textunderscore botelha\textunderscore ^1)}
\end{itemize}
Pequena medida antiga, menor que o celamim.
O saco da maquia, nos moínhos de cereaes.
\section{Botelho}
\begin{itemize}
\item {fónica:tê}
\end{itemize}
\begin{itemize}
\item {Grp. gram.:m.}
\end{itemize}
\begin{itemize}
\item {Utilização:Ant.}
\end{itemize}
Planta aquática. Cf. Pero Vaz de Caminha, \textunderscore Carta a D. Man.\textunderscore 
\section{Botequim}
\begin{itemize}
\item {Grp. gram.:m.}
\end{itemize}
Casa pública, onde se vendem bebidas.
Loja de bebidas.
Café.
(Por \textunderscore botiquim\textunderscore , de \textunderscore botica\textunderscore )
\section{Botequineira}
\begin{itemize}
\item {Grp. gram.:f.}
\end{itemize}
Mulher, que vende em botequim.
Dona de botequim.
Mulher de botequineiro.
\section{Botequineiro}
\begin{itemize}
\item {Grp. gram.:m.}
\end{itemize}
Dono ou administrador de botequim.
\section{Bóthia}
\begin{itemize}
\item {Grp. gram.:f.}
\end{itemize}
Planta ornamental.
\section{Bothrião}
\begin{itemize}
\item {Grp. gram.:m.}
\end{itemize}
\begin{itemize}
\item {Utilização:Med.}
\end{itemize}
\begin{itemize}
\item {Proveniência:(Gr. \textunderscore bothrion\textunderscore )}
\end{itemize}
Úlcera na córnea.
\section{Bothriocéphalo}
\begin{itemize}
\item {Grp. gram.:m.}
\end{itemize}
\begin{itemize}
\item {Proveniência:(Do gr. \textunderscore bothrion\textunderscore  + \textunderscore kephale\textunderscore )}
\end{itemize}
Verme intestinal, espécie de tênia.
\section{Bótia}
\begin{itemize}
\item {Grp. gram.:f.}
\end{itemize}
Planta ornamental.
\section{Botica}
\begin{itemize}
\item {Grp. gram.:f.}
\end{itemize}
\begin{itemize}
\item {Utilização:Ant.}
\end{itemize}
\begin{itemize}
\item {Utilização:Prov.}
\end{itemize}
\begin{itemize}
\item {Utilização:alent.}
\end{itemize}
\begin{itemize}
\item {Utilização:Gír.}
\end{itemize}
\begin{itemize}
\item {Proveniência:(Do lat. \textunderscore apotheca\textunderscore )}
\end{itemize}
Estabelecimento, em que se preparam e vendem medicamentos; pharmácia.
Pequena casa.
Loja, em que se vendem gêneros a retalho; mercearia.
Medicamento.
Cara.
\section{Boticada}
\begin{itemize}
\item {Grp. gram.:f.}
\end{itemize}
\begin{itemize}
\item {Utilização:Fam.}
\end{itemize}
Qualquer droga de pharmácia, ou medicamento preparado em botica.
\section{Boticão}
\begin{itemize}
\item {Grp. gram.:m.}
\end{itemize}
Instrumento, com que se tiram dentes.
\section{Boticária}
\begin{itemize}
\item {Grp. gram.:f.}
\end{itemize}
Dona de botica.
Mulher, que prepara medicamentos em pharmácia.
Mulher de boticário.
(Fem. de \textunderscore boticário\textunderscore )
\section{Boticaría}
\begin{itemize}
\item {Grp. gram.:f.}
\end{itemize}
\begin{itemize}
\item {Utilização:Ant.}
\end{itemize}
\begin{itemize}
\item {Proveniência:(De \textunderscore botica\textunderscore )}
\end{itemize}
Profissão de boticário.
Depósito de medicamentos.
\section{Boticário}
\begin{itemize}
\item {Grp. gram.:m.}
\end{itemize}
Dono de botica.
Aquelle que prepara e vende medicamentos na botica.
Pharmacêutico.
\section{Botifarra}
\begin{itemize}
\item {Grp. gram.:f.}
\end{itemize}
\begin{itemize}
\item {Utilização:Pop.}
\end{itemize}
Bota grosseira e grande. Cf. Herculano, \textunderscore Lendas\textunderscore , II, 215.
\section{Botija}
\begin{itemize}
\item {Grp. gram.:f.}
\end{itemize}
\begin{itemize}
\item {Utilização:Fig.}
\end{itemize}
\begin{itemize}
\item {Utilização:Bras. do N}
\end{itemize}
\begin{itemize}
\item {Proveniência:(Do b. lat. \textunderscore butticula\textunderscore , de \textunderscore butta\textunderscore )}
\end{itemize}
Vasilha cylíndrica de grés, de bôca estreita, gargalo curto e uma pequena asa.
Pessôa gorda.
Remate do chicote dos cabos náuticos.
Thesoiro enterrado.
\section{Botilhão}
\begin{itemize}
\item {Grp. gram.:m.}
\end{itemize}
Designação vulgar do abutilão.
(Cp. \textunderscore abutilão\textunderscore )
\section{Botilho}
\begin{itemize}
\item {Grp. gram.:m.}
\end{itemize}
\begin{itemize}
\item {Utilização:Prov.}
\end{itemize}
\begin{itemize}
\item {Utilização:trasm.}
\end{itemize}
Pauzinho, com que se enfreiam os chibatos, para os desmamar, ou que se colloca, a pino, entre o céu da bôca e a língua dos burros, para que não comam.
\section{Botim}
\begin{itemize}
\item {Grp. gram.:m.}
\end{itemize}
\begin{itemize}
\item {Proveniência:(De \textunderscore bota\textunderscore ^1)}
\end{itemize}
Bota de cano baixo.
\section{Botina}
\begin{itemize}
\item {Grp. gram.:f.}
\end{itemize}
\begin{itemize}
\item {Utilização:Prov.}
\end{itemize}
\begin{itemize}
\item {Utilização:alent.}
\end{itemize}
\begin{itemize}
\item {Proveniência:(T. it.)}
\end{itemize}
Pequena bota para senhora ou criança.
Espécie de polaina de coiro grosseiro.
\section{Botineiro}
\begin{itemize}
\item {Grp. gram.:adj.}
\end{itemize}
\begin{itemize}
\item {Proveniência:(De \textunderscore botim\textunderscore )}
\end{itemize}
Diz-se do toiro, que tem as pernas de côr differente da do resto do corpo.
\section{Botinha}
\begin{itemize}
\item {Grp. gram.:f.}
\end{itemize}
(V.botina)
\section{Botinos}
\begin{itemize}
\item {Grp. gram.:m. pl.}
\end{itemize}
\begin{itemize}
\item {Utilização:Prov.}
\end{itemize}
\begin{itemize}
\item {Utilização:alent.}
\end{itemize}
Polainas grosseiras de coiro.
(Cp. \textunderscore botina\textunderscore )
\section{Botiqueiro}
\begin{itemize}
\item {Grp. gram.:m.}
\end{itemize}
\begin{itemize}
\item {Utilização:Des.}
\end{itemize}
\begin{itemize}
\item {Proveniência:(De \textunderscore botica\textunderscore )}
\end{itemize}
Proprietário ou vendedor, em botica, ou em loja de gêneros alimentícios.
\section{Botiquim}
\begin{itemize}
\item {Grp. gram.:m.}
\end{itemize}
O mesmo que \textunderscore botequim\textunderscore .
\section{Botirão}
\begin{itemize}
\item {Grp. gram.:m.}
\end{itemize}
\begin{itemize}
\item {Utilização:T. de Aveiro}
\end{itemize}
Nassa, para pescar lampreias.
Espécie de rede de arrastar.
\section{Bóto}
\begin{itemize}
\item {Grp. gram.:m.}
\end{itemize}
\begin{itemize}
\item {Utilização:Prov.}
\end{itemize}
\begin{itemize}
\item {Utilização:trasm.}
\end{itemize}
\begin{itemize}
\item {Proveniência:(De \textunderscore bota\textunderscore ^2)}
\end{itemize}
O mesmo que \textunderscore odre\textunderscore .
\section{Bôto}
\begin{itemize}
\item {Grp. gram.:adj.}
\end{itemize}
Rombo; que perdeu o gume, (falando-se de arma, que não póde furar ou cortar).
Diz-se dos dentes, quando impressionados incommodamente pela mastigação de substâncias muito ácidas.
Hebetado.
Bronco, de intelligência obtusa.
O mesmo que \textunderscore mouco\textunderscore :«\textunderscore e suas orelhas faz bôtas\textunderscore ». Usque, \textunderscore Tribulações\textunderscore , 18 V.^o
\section{Bôto}
\begin{itemize}
\item {Grp. gram.:m.}
\end{itemize}
Peixe do Purus, do Tocantins e dos Açores, semelhante ao atum.
\section{Botoaria}
\begin{itemize}
\item {Grp. gram.:f.}
\end{itemize}
Fábrica de botões.
Estabelecimento, onde se vendem botões.
Indústria dos botões.
\section{Botocar}
\begin{itemize}
\item {Grp. gram.:v. i.}
\end{itemize}
\begin{itemize}
\item {Proveniência:(De \textunderscore botoque\textunderscore )}
\end{itemize}
Saltar para fóra, sair.
\section{Botocudos}
\begin{itemize}
\item {Grp. gram.:m. pl.}
\end{itemize}
\begin{itemize}
\item {Proveniência:(De \textunderscore botoque\textunderscore )}
\end{itemize}
Indígenas americanos, que usavam botoque.
\section{Botoeira}
\begin{itemize}
\item {Grp. gram.:f.}
\end{itemize}
\begin{itemize}
\item {Proveniência:(De \textunderscore botão\textunderscore )}
\end{itemize}
Abertura, em que entra o botão, para apertar ou fechar peça de vestuário.
Mulher, que faz botões.
\section{Botoeiro}
\begin{itemize}
\item {Grp. gram.:m.}
\end{itemize}
Aquelle que faz botões.
\section{Botoque}
\begin{itemize}
\item {Grp. gram.:m.}
\end{itemize}
Pedaço de pedra ou madeira, que algumas tríbos americanas usam embebido no beiço inferior.
O mesmo que \textunderscore batoque\textunderscore .
\section{Botos}
\begin{itemize}
\item {fónica:bô}
\end{itemize}
\begin{itemize}
\item {Grp. gram.:m. pl.}
\end{itemize}
Servidores do culto entre os indígenas de Satari, na Índia Portuguesa.
\section{Botrião}
\begin{itemize}
\item {Grp. gram.:m.}
\end{itemize}
\begin{itemize}
\item {Utilização:Med.}
\end{itemize}
\begin{itemize}
\item {Proveniência:(Gr. \textunderscore bothrion\textunderscore )}
\end{itemize}
Úlcera na córnea.
\section{Botriocéfalo}
\begin{itemize}
\item {Grp. gram.:m.}
\end{itemize}
\begin{itemize}
\item {Proveniência:(Do gr. \textunderscore bothrion\textunderscore  + \textunderscore kephale\textunderscore )}
\end{itemize}
Verme intestinal, espécie de tênia.
\section{Botrioide}
\begin{itemize}
\item {Grp. gram.:adj.}
\end{itemize}
\begin{itemize}
\item {Utilização:Miner.}
\end{itemize}
\begin{itemize}
\item {Proveniência:(Do gr. \textunderscore botrus\textunderscore  + \textunderscore eidos\textunderscore )}
\end{itemize}
Diz-se da concreção pedregosa, que tem grosseiramente o aspecto de um cacho de uvas.
\section{Botryoide}
\begin{itemize}
\item {Grp. gram.:adj.}
\end{itemize}
\begin{itemize}
\item {Utilização:Miner.}
\end{itemize}
\begin{itemize}
\item {Proveniência:(Do gr. \textunderscore botrus\textunderscore  + \textunderscore eidos\textunderscore )}
\end{itemize}
Diz-se da concreção pedregosa, que tem grosseiramente o aspecto de um cacho de uvas.
\section{Bottos}
\begin{itemize}
\item {fónica:bô}
\end{itemize}
\begin{itemize}
\item {Grp. gram.:m. pl.}
\end{itemize}
Servidores do culto entre os indígenas de Satari, na Índia Portuguesa.
\section{Botulismo}
\begin{itemize}
\item {Grp. gram.:m.}
\end{itemize}
Envenenamento, por ingestão de elementos avariados, devido ao \textunderscore bacillus botulinus\textunderscore  de Van Ermenghem.
\section{Bouba}
\begin{itemize}
\item {Grp. gram.:f.}
\end{itemize}
\begin{itemize}
\item {Utilização:Des.}
\end{itemize}
\begin{itemize}
\item {Utilização:Prov.}
\end{itemize}
\begin{itemize}
\item {Utilização:trasm.}
\end{itemize}
O mesmo que \textunderscore buba\textunderscore . Cf. \textunderscore Peregrinação\textunderscore , XCIX.
O mesmo que \textunderscore ferida\textunderscore .
(Colhido em Chaves)
\section{Bouba-da-praia}
\begin{itemize}
\item {Grp. gram.:f.}
\end{itemize}
Planta santhomense, cujas fôlhas tem propriedades refrigerantes.
\section{Boubela}
\begin{itemize}
\item {Grp. gram.:f.}
\end{itemize}
\begin{itemize}
\item {Utilização:Prov.}
\end{itemize}
\begin{itemize}
\item {Utilização:trasm.}
\end{itemize}
O mesmo que \textunderscore poupa\textunderscore ^1.
\section{Bouça}
\begin{itemize}
\item {Grp. gram.:f.}
\end{itemize}
\begin{itemize}
\item {Utilização:Prov.}
\end{itemize}
\begin{itemize}
\item {Utilização:minh.}
\end{itemize}
Terreno inculto; terreno, que só cria mato.
Terreno murado, ou delimitado por pedras ou montes de terra, em que se cria mato para várias applicações e pinheiros ou carvalhos.
(Alter. de \textunderscore balça\textunderscore )
\section{Bouçar}
\begin{itemize}
\item {Grp. gram.:v. t.}
\end{itemize}
\begin{itemize}
\item {Proveniência:(De \textunderscore boiça\textunderscore )}
\end{itemize}
Roçar e queimar o mato em terreno para lavoura.
\section{Boucha}
\begin{itemize}
\item {Grp. gram.:f.}
\end{itemize}
\begin{itemize}
\item {Utilização:Prov.}
\end{itemize}
Mato, que se queima, para se cultivar a terra que êlle occupava.
(Alter. de \textunderscore bouça\textunderscore ?)
\section{Boucim}
\begin{itemize}
\item {Grp. gram.:m.}
\end{itemize}
\begin{itemize}
\item {Utilização:Prov.}
\end{itemize}
\begin{itemize}
\item {Utilização:trasm.}
\end{itemize}
Abertura provisória na parede de palheiro ou curral, pela qual se recolhe a palha ou o feno, que já não póde entrar pela porta respectiva.
\section{Bouço}
\begin{itemize}
\item {Grp. gram.:m.}
\end{itemize}
Associação forçada dos arrendatários das casanas, nas communidades indianas.
\section{Bouga}
\begin{itemize}
\item {Grp. gram.:adj.}
\end{itemize}
\begin{itemize}
\item {Utilização:T. da Bairrada}
\end{itemize}
Adoidado, maluco.
(Cp. \textunderscore abougar\textunderscore )
\section{Bouganvílea}
\begin{itemize}
\item {Grp. gram.:f.}
\end{itemize}
O mesmo que \textunderscore buganvilla\textunderscore . Cf. Garrett, \textunderscore Helena\textunderscore , 109 e 114.
\section{Boulé}
\begin{itemize}
\item {Grp. gram.:f.}
\end{itemize}
Lugar, onde funccionava o senado, em Athenas. Cf. Latino, \textunderscore Or. da Corôa\textunderscore , CCXV.
\section{Bourar}
\begin{itemize}
\item {Grp. gram.:v. i.}
\end{itemize}
\begin{itemize}
\item {Utilização:Prov.}
\end{itemize}
\begin{itemize}
\item {Utilização:minh.}
\end{itemize}
Bater, dar pancadas.
\section{Bóveda}
\begin{itemize}
\item {Grp. gram.:f.}
\end{itemize}
\begin{itemize}
\item {Utilização:Ant.}
\end{itemize}
O mesmo que \textunderscore abóbada\textunderscore .
\section{Bovicida}
\begin{itemize}
\item {Grp. gram.:m.}
\end{itemize}
\begin{itemize}
\item {Proveniência:(Do lat. \textunderscore bos\textunderscore  + \textunderscore caedere\textunderscore )}
\end{itemize}
Aquelle que mata bois.
\section{Bovicídio}
\begin{itemize}
\item {Grp. gram.:m.}
\end{itemize}
Matança de bois.
(Cp. \textunderscore bovicida\textunderscore )
\section{Bovídeos}
\begin{itemize}
\item {Grp. gram.:m. pl.}
\end{itemize}
\begin{itemize}
\item {Proveniência:(Do gr. \textunderscore bous\textunderscore  + \textunderscore eidos\textunderscore )}
\end{itemize}
Classe de ruminantes, que compreende o boi, o búfalo, o bisonte, etc.
\section{Bovino}
\begin{itemize}
\item {Grp. gram.:adj.}
\end{itemize}
\begin{itemize}
\item {Proveniência:(Lat. \textunderscore bovinus\textunderscore )}
\end{itemize}
Relativo a bois.
\section{Bovista}
\begin{itemize}
\item {Grp. gram.:m.}
\end{itemize}
Gênero de cogumelos.
\section{Box}
\begin{itemize}
\item {fónica:cse}
\end{itemize}
\begin{itemize}
\item {Grp. gram.:m.}
\end{itemize}
\begin{itemize}
\item {Proveniência:(Ingl. \textunderscore box\textunderscore )}
\end{itemize}
Jôgo de murro, á inglesa.
Armadura metállica, com que, depois de enfiada nos dedos, se dão murros.
\section{Boxa}
\begin{itemize}
\item {Grp. gram.:f.}
\end{itemize}
\textunderscore Pôr o barco á boxa\textunderscore , pôr o barco de modo, que ganhe preferência no lançamento da rede de pesca.
\section{Boxá}
\begin{itemize}
\item {Grp. gram.:f.}
\end{itemize}
Pequena mala, usada entre os Moiros, para guardar o fato.
\section{Boxador}
\begin{itemize}
\item {fónica:csa}
\end{itemize}
\begin{itemize}
\item {Grp. gram.:m.}
\end{itemize}
\begin{itemize}
\item {Utilização:Bras}
\end{itemize}
Jogador de box.
\section{Boxe! boxe!}
\begin{itemize}
\item {Grp. gram.:interj.}
\end{itemize}
\begin{itemize}
\item {Utilização:Prov.}
\end{itemize}
O mesmo que \textunderscore baxe! baxe!\textunderscore 
\section{Boximanes}
\begin{itemize}
\item {Grp. gram.:m. pl.}
\end{itemize}
Indígenas africanos que, para alguns anthropólogos, representam os seres intermédios ao homem e ao animal.
Raça inferior da humanidade.
(Cp.ingl. \textunderscore boxen\textunderscore  + \textunderscore man\textunderscore , homem dos bosques)
\section{Bozerra}
\begin{itemize}
\item {fónica:zê}
\end{itemize}
\begin{itemize}
\item {Grp. gram.:f.}
\end{itemize}
\begin{itemize}
\item {Utilização:Bras}
\end{itemize}
\begin{itemize}
\item {Utilização:Fig.}
\end{itemize}
Monte de excremento.
Indivíduo mollangueirão ou inútil.
\section{Bozó}
\begin{itemize}
\item {Grp. gram.:m.}
\end{itemize}
\begin{itemize}
\item {Utilização:Bras. do N}
\end{itemize}
Jôgo, que se faz com uma bóla.
\section{Brabacan}
\begin{itemize}
\item {Grp. gram.:f.}
\end{itemize}
O mesmo que \textunderscore barbacan\textunderscore . Cf. \textunderscore Peregrinação\textunderscore , XCII.
\section{Brabanção}
\begin{itemize}
\item {Grp. gram.:adj.}
\end{itemize}
\begin{itemize}
\item {Grp. gram.:M.}
\end{itemize}
Relativo ao Brabante.
Homem natural do Brabante.
\section{Brabantês}
\begin{itemize}
\item {Grp. gram.:m.  e  adj.}
\end{itemize}
O mesmo que \textunderscore brabanção\textunderscore .
\section{Brabares}
\begin{itemize}
\item {Grp. gram.:m. pl.}
\end{itemize}
Mercadores indianos.
\section{Brabo}
\begin{itemize}
\item {Grp. gram.:adj.}
\end{itemize}
\begin{itemize}
\item {Utilização:Bras}
\end{itemize}
\begin{itemize}
\item {Utilização:ant.}
\end{itemize}
O mesmo que \textunderscore bravo\textunderscore . Cf. M. Soares, \textunderscore Dicc. Bras.\textunderscore ; e \textunderscore Eufrosina\textunderscore , 147.
\section{Braça}
\begin{itemize}
\item {Grp. gram.:f.}
\end{itemize}
\begin{itemize}
\item {Proveniência:(De \textunderscore braço\textunderscore )}
\end{itemize}
Antiga medida de extensão, correspondente a pouco mais de dois metros.
\section{Braçada}
\begin{itemize}
\item {Grp. gram.:f.}
\end{itemize}
O mesmo que \textunderscore braçado\textunderscore .
Braço de árvore, pernada. Cf. Benalcanfor, \textunderscore Cartas de Viagem\textunderscore , XXVIII.
\section{Braçadeira}
\begin{itemize}
\item {Grp. gram.:f.}
\end{itemize}
Argola, no lado interior do escudo, para se enfiar o braço.
Argola, que aperta a espingarda, no ponto em que o cano se liga á coronha Argola, correia ou tira, que abraça o apanhado lateral de uma cortina, reposteiro, etc.
Abraçadeira.
Suspensório interior de uma carruagem, para descansar o braço.
Anilho aberto de metal, que, no clarinete, une a palheta á boquilha, apertando-a com dois parafusos.
Espécie de bainha de coiro, que, nos tambores e bombos, cinge, a duas e duas, as voltas da corda, para se comprimir o arquilho sôbre a pelle.
\section{Braçado}
\begin{itemize}
\item {Grp. gram.:m.}
\end{itemize}
\begin{itemize}
\item {Proveniência:(De \textunderscore braço\textunderscore )}
\end{itemize}
Porção de coisas, ou objecto, que póde cingir-se com os braços.
Grande quantidade.
\section{Braçagem}
\begin{itemize}
\item {Grp. gram.:f.}
\end{itemize}
\begin{itemize}
\item {Proveniência:(De \textunderscore braço\textunderscore )}
\end{itemize}
Trabalho braçal.
\section{Braçajá}
\begin{itemize}
\item {Grp. gram.:m.}
\end{itemize}
Espécie de cajado do Brasil.
\section{Braçajote}
\begin{itemize}
\item {Grp. gram.:m.}
\end{itemize}
\begin{itemize}
\item {Utilização:Prov.}
\end{itemize}
\begin{itemize}
\item {Utilização:alg.}
\end{itemize}
O mesmo que \textunderscore berçajote\textunderscore .
\section{Braçal}
\begin{itemize}
\item {Grp. gram.:m.}
\end{itemize}
\begin{itemize}
\item {Grp. gram.:Adj.}
\end{itemize}
\begin{itemize}
\item {Proveniência:(De \textunderscore braço\textunderscore )}
\end{itemize}
Parte da armadura, com que se defendia o braço.
Relativo a braço, que se faz com os braços: \textunderscore trabalho braçal\textunderscore .
Material, mecânico.
\section{Braçalmente}
\begin{itemize}
\item {Grp. gram.:adv.}
\end{itemize}
De modo \textunderscore braçal\textunderscore .
\section{Bracamarte}
\begin{itemize}
\item {Grp. gram.:m.}
\end{itemize}
Antigo espadão, que se brandia com as mãos ambas.
(B. lat. \textunderscore braquemardus\textunderscore )
\section{Bracaraugustano}
\begin{itemize}
\item {fónica:brá}
\end{itemize}
\begin{itemize}
\item {Grp. gram.:m.}
\end{itemize}
Habitante da antiga Brácara Augusta, (Braga).
\section{Bracarense}
\begin{itemize}
\item {Grp. gram.:adj.}
\end{itemize}
\begin{itemize}
\item {Grp. gram.:M.}
\end{itemize}
\begin{itemize}
\item {Proveniência:(Do lat. \textunderscore Bracara\textunderscore , n. p.)}
\end{itemize}
Relativo a Braga.
Habitante de Braga.
\section{Braçaria}
\begin{itemize}
\item {Grp. gram.:f.}
\end{itemize}
\begin{itemize}
\item {Proveniência:(De \textunderscore braço\textunderscore )}
\end{itemize}
Arte de atirar projécteis com o braço.
\section{Brácaro}
\begin{itemize}
\item {Grp. gram.:adj.}
\end{itemize}
O mesmo que \textunderscore bracarense\textunderscore . Cf. Garrett, \textunderscore Obras\textunderscore , vol. XVII.
\section{Braceagem}
\begin{itemize}
\item {Grp. gram.:f.}
\end{itemize}
Acto de \textunderscore bracear\textunderscore .
Trabalho a braços.
Fabricação de moéda.
Antiga retribuição de moedeiros.
\section{Bracear}
\begin{itemize}
\item {Grp. gram.:v. i.}
\end{itemize}
\begin{itemize}
\item {Grp. gram.:V.}
\end{itemize}
\begin{itemize}
\item {Utilização:t. Náut.}
\end{itemize}
O mesmo que \textunderscore bracejar\textunderscore .
\textunderscore Bracear as vêrgas\textunderscore , dar ás vêrgas movimento horizontal em tôrno dos mastros, por meio de cabos, que se chamam braços.
\section{Braceira}
\begin{itemize}
\item {Grp. gram.:f.}
\end{itemize}
\begin{itemize}
\item {Utilização:Heráld.}
\end{itemize}
Faixa de cal ou argamassa, com que se fixam as telhas para vedarem os canaes.
O mesmo que \textunderscore braçadeira\textunderscore .
\section{Braceiro}
\begin{itemize}
\item {Grp. gram.:m.}
\end{itemize}
\begin{itemize}
\item {Grp. gram.:Adj.}
\end{itemize}
\begin{itemize}
\item {Utilização:Bras. de Minas}
\end{itemize}
\begin{itemize}
\item {Proveniência:(De \textunderscore braço\textunderscore )}
\end{itemize}
Trabalhador mecânico.
Aquelle que dá o braço a alguém, servindo-lhe de apoio.
Que tem grande fôrça nos braços.
Que se atira com o braço: \textunderscore um pedregulho braceiro\textunderscore .
Diz-se do cavallo, que ergue muito as patas deanteiras.
\section{Bracejador}
\begin{itemize}
\item {Grp. gram.:adj.}
\end{itemize}
Que braceja.
\section{Bracejamento}
\begin{itemize}
\item {Grp. gram.:m.}
\end{itemize}
Acto ou effeito de \textunderscore bracejar\textunderscore .
\section{Bracejar}
\begin{itemize}
\item {Grp. gram.:v. t.}
\end{itemize}
\begin{itemize}
\item {Grp. gram.:V. i.}
\end{itemize}
\begin{itemize}
\item {Proveniência:(De \textunderscore braços\textunderscore )}
\end{itemize}
Estender para um e outro lado.
Diffundir.
Agitar os braços.
Lidar.
Mover-se, á semelhança de braços.
\section{Bracejo}
\begin{itemize}
\item {Grp. gram.:m.}
\end{itemize}
Acção de \textunderscore bracejar\textunderscore .
\section{Bracel}
\begin{itemize}
\item {Grp. gram.:m.}
\end{itemize}
Casta de uva.
\section{Braceleira}
\begin{itemize}
\item {Grp. gram.:f.}
\end{itemize}
(V. \textunderscore braçal\textunderscore , m.)
\section{Bracelete}
\begin{itemize}
\item {fónica:lê}
\end{itemize}
\begin{itemize}
\item {Grp. gram.:m.}
\end{itemize}
\begin{itemize}
\item {Proveniência:(De \textunderscore braço\textunderscore )}
\end{itemize}
Pulseira, argola de adôrno, que as mulheres usam no braço, junto ao pulso.
\section{Bracelões}
\begin{itemize}
\item {Grp. gram.:m. pl.}
\end{itemize}
\begin{itemize}
\item {Proveniência:(De \textunderscore braço\textunderscore )}
\end{itemize}
Antiga armadura, com que se guarneciam os braços.
\section{Bracelone}
\begin{itemize}
\item {Grp. gram.:m.}
\end{itemize}
\begin{itemize}
\item {Utilização:Ant.}
\end{itemize}
O mesmo que \textunderscore braceleira\textunderscore .
\section{Bracelote}
\begin{itemize}
\item {Grp. gram.:m.}
\end{itemize}
\begin{itemize}
\item {Proveniência:(De \textunderscore braço\textunderscore )}
\end{itemize}
Prolongamento da alça dos moitões dos braços, nos navios.
\section{Brachelytro}
\begin{itemize}
\item {fónica:que}
\end{itemize}
\begin{itemize}
\item {Grp. gram.:adj.}
\end{itemize}
\begin{itemize}
\item {Grp. gram.:M. pl.}
\end{itemize}
\begin{itemize}
\item {Proveniência:(Do gr. \textunderscore brakhus\textunderscore  + \textunderscore elutron\textunderscore )}
\end{itemize}
Que tem elytros curtos.
Fam. de insectos coleópteros, com elytros curtos.
\section{Brachial}
\begin{itemize}
\item {fónica:qui}
\end{itemize}
\begin{itemize}
\item {Grp. gram.:adj.}
\end{itemize}
\begin{itemize}
\item {Proveniência:(Lat. \textunderscore brachialis\textunderscore )}
\end{itemize}
Relativo ao braço.
\section{Brachídeo}
\begin{itemize}
\item {fónica:qui}
\end{itemize}
\begin{itemize}
\item {Grp. gram.:adj.}
\end{itemize}
\begin{itemize}
\item {Proveniência:(Do gr. \textunderscore brakhion\textunderscore  + \textunderscore eidos\textunderscore )}
\end{itemize}
Que tem fórma de braço.
\section{Brachiocephálico}
\begin{itemize}
\item {fónica:qui}
\end{itemize}
\begin{itemize}
\item {Grp. gram.:adj.}
\end{itemize}
\begin{itemize}
\item {Utilização:Anat.}
\end{itemize}
\begin{itemize}
\item {Proveniência:(De \textunderscore braquiocéphalo\textunderscore )}
\end{itemize}
Que fornece os vasos á cabeça e ao braço, (falando-se do tronco arterial).
\section{Brachiocéphalo}
\begin{itemize}
\item {fónica:qui}
\end{itemize}
\begin{itemize}
\item {Grp. gram.:m.}
\end{itemize}
\begin{itemize}
\item {Proveniência:(Do gr. \textunderscore brakhion\textunderscore  + \textunderscore kephale\textunderscore )}
\end{itemize}
Cephalópode que tem braços.
\section{Brachiópode}
\begin{itemize}
\item {fónica:qui}
\end{itemize}
\begin{itemize}
\item {Grp. gram.:adj.}
\end{itemize}
\begin{itemize}
\item {Utilização:Zool.}
\end{itemize}
\begin{itemize}
\item {Grp. gram.:M. pl.}
\end{itemize}
\begin{itemize}
\item {Proveniência:(Do gr. \textunderscore brakhion\textunderscore  + \textunderscore pous\textunderscore , \textunderscore podos\textunderscore )}
\end{itemize}
Cujos braços servem de pés.
Classe de molluscos, cujos pés são representados por dois braços, que servem para a respiração e para a locomoção.
\section{Brachióptero}
\begin{itemize}
\item {fónica:qui}
\end{itemize}
\begin{itemize}
\item {Grp. gram.:m.}
\end{itemize}
\begin{itemize}
\item {Proveniência:(Do gr. \textunderscore brackhion\textunderscore  + \textunderscore pteron\textunderscore )}
\end{itemize}
Peixe, que tem as barbatanas em fórma de asas.
\section{Brachióstomo}
\begin{itemize}
\item {fónica:qui}
\end{itemize}
\begin{itemize}
\item {Grp. gram.:m.}
\end{itemize}
\begin{itemize}
\item {Proveniência:(Do gr. \textunderscore brakhion\textunderscore  + \textunderscore stoma\textunderscore )}
\end{itemize}
Espécie de pólypo, que tem a bôca rodeada de membros aprehensores.
\section{Brachistocéphalo}
\begin{itemize}
\item {fónica:qui}
\end{itemize}
\begin{itemize}
\item {Grp. gram.:adj.}
\end{itemize}
\begin{itemize}
\item {Proveniência:(Do gr. \textunderscore brakhistos\textunderscore  + \textunderscore kephale\textunderscore )}
\end{itemize}
Que tem a cabeça muito curta.
\section{Bráchya}
\begin{itemize}
\item {fónica:qui}
\end{itemize}
\begin{itemize}
\item {Grp. gram.:f.}
\end{itemize}
\begin{itemize}
\item {Proveniência:(Do gr. \textunderscore brakhus\textunderscore )}
\end{itemize}
Sinal orthográphico que, collocado sobre uma vogal, indica que ella é breve, assim: ě.
\section{Brachybiota}
\begin{itemize}
\item {fónica:qui}
\end{itemize}
\begin{itemize}
\item {Grp. gram.:adj.}
\end{itemize}
\begin{itemize}
\item {Proveniência:(Do gr. \textunderscore brakhus\textunderscore  + \textunderscore biotes\textunderscore )}
\end{itemize}
Que tem vida curta.
\section{Brachycataléctico}
\begin{itemize}
\item {fónica:qui}
\end{itemize}
\begin{itemize}
\item {Grp. gram.:adj.}
\end{itemize}
\begin{itemize}
\item {Proveniência:(Do gr. \textunderscore brakhus\textunderscore  + \textunderscore katalektos\textunderscore )}
\end{itemize}
Dizia-se dos versos gregos ou latinos, a que faltava um pé.
\section{Brachycatalecto}
\begin{itemize}
\item {fónica:qui}
\end{itemize}
\begin{itemize}
\item {Grp. gram.:adj.}
\end{itemize}
\begin{itemize}
\item {Proveniência:(Do gr. \textunderscore brakhus\textunderscore  + \textunderscore katalektos\textunderscore )}
\end{itemize}
Dizia-se dos versos gregos ou latinos, a que faltava um pé.
\section{Brachycephalia}
\begin{itemize}
\item {fónica:qui}
\end{itemize}
\begin{itemize}
\item {Grp. gram.:f.}
\end{itemize}
Estado de \textunderscore brachycéphalo\textunderscore .
\section{Brachycéphalo}
\begin{itemize}
\item {fónica:qui}
\end{itemize}
\begin{itemize}
\item {Grp. gram.:m.  e  adj.}
\end{itemize}
\begin{itemize}
\item {Proveniência:(Do gr. \textunderscore brakhus\textunderscore  + \textunderscore kephale\textunderscore )}
\end{itemize}
Diz-se do indivíduo, cujo crânio, observado de cima, apresenta a fórma de um ovo, mas mais curta e arredondada posteriormente.
\section{Brachýcero}
\begin{itemize}
\item {fónica:qui}
\end{itemize}
\begin{itemize}
\item {Grp. gram.:adj.}
\end{itemize}
\begin{itemize}
\item {Grp. gram.:M. pl.}
\end{itemize}
\begin{itemize}
\item {Proveniência:(Do gr. \textunderscore brakhus\textunderscore  + \textunderscore keras\textunderscore )}
\end{itemize}
Que tem cornos curtos.
Insectos coleópteros, de antennas curtas.
\section{Brachychoreia}
\begin{itemize}
\item {fónica:qui-co}
\end{itemize}
\begin{itemize}
\item {Grp. gram.:f.}
\end{itemize}
\begin{itemize}
\item {Proveniência:(Do gr. \textunderscore brakhus\textunderscore  + \textunderscore khoreios\textunderscore )}
\end{itemize}
Pé de verso; grego ou latino, formado de uma sýllaba longa entre duas breves.
\section{Brachydáctylo}
\begin{itemize}
\item {fónica:qui}
\end{itemize}
\begin{itemize}
\item {Grp. gram.:adj.}
\end{itemize}
\begin{itemize}
\item {Proveniência:(Do gr. \textunderscore brakus\textunderscore  + \textunderscore daktulos\textunderscore )}
\end{itemize}
Que tem dedos curtos.
\section{Brachydiagonal}
\begin{itemize}
\item {fónica:qui}
\end{itemize}
\begin{itemize}
\item {Grp. gram.:adj.}
\end{itemize}
\begin{itemize}
\item {Utilização:Geol.}
\end{itemize}
\begin{itemize}
\item {Proveniência:(Do gr. \textunderscore brakhus\textunderscore  + \textunderscore diagonios\textunderscore )}
\end{itemize}
Diz-se do menor dos três eixos dos crystaes do systema orthorhômbico.
\section{Brachýdoma}
\begin{itemize}
\item {fónica:qui}
\end{itemize}
\begin{itemize}
\item {Grp. gram.:m.}
\end{itemize}
\begin{itemize}
\item {Utilização:Geol.}
\end{itemize}
\begin{itemize}
\item {Proveniência:(Do gr. \textunderscore brakhus\textunderscore  + \textunderscore doma\textunderscore )}
\end{itemize}
Prisma transversal, com eixo brachydiagonal.
\section{Brachygraphia}
\begin{itemize}
\item {fónica:qui}
\end{itemize}
\begin{itemize}
\item {Grp. gram.:f.}
\end{itemize}
Arte de \textunderscore brachýgrapho\textunderscore .
\section{Brachýgrapho}
\begin{itemize}
\item {fónica:qui}
\end{itemize}
\begin{itemize}
\item {Grp. gram.:m.}
\end{itemize}
\begin{itemize}
\item {Proveniência:(Do gr. \textunderscore brakhus\textunderscore  + \textunderscore graphein\textunderscore )}
\end{itemize}
Aquelle que escreve por abreviaturas.
\section{Brachyologia}
\begin{itemize}
\item {fónica:qui}
\end{itemize}
\begin{itemize}
\item {Grp. gram.:f.}
\end{itemize}
\begin{itemize}
\item {Proveniência:(Do gr. \textunderscore brakhus\textunderscore  + \textunderscore logos\textunderscore )}
\end{itemize}
Locução obscura, por sêr muito lacónica.
\section{Brachiológico}
\begin{itemize}
\item {fónica:qui}
\end{itemize}
\begin{itemize}
\item {Grp. gram.:adj.}
\end{itemize}
Relativo a brachyologia.
Em que há \textunderscore brachyologia\textunderscore .
\section{Brachypinacoide}
\begin{itemize}
\item {fónica:qui}
\end{itemize}
\begin{itemize}
\item {Grp. gram.:m.}
\end{itemize}
\begin{itemize}
\item {Utilização:Geol.}
\end{itemize}
Prisma mineral, limitado por dois planos parallelos entre si e equidistantes do plano de symetria, que passa pelo eixo principal e pelo brachydiagonal.
\section{Brachypneia}
\begin{itemize}
\item {fónica:qui}
\end{itemize}
\begin{itemize}
\item {Grp. gram.:f.}
\end{itemize}
\begin{itemize}
\item {Proveniência:(Do gr. \textunderscore brakhus\textunderscore  + \textunderscore pnein\textunderscore )}
\end{itemize}
Respiração curta, difficil.
\section{Brachýpodes}
\begin{itemize}
\item {fónica:qui}
\end{itemize}
\begin{itemize}
\item {Grp. gram.:m. pl.}
\end{itemize}
\begin{itemize}
\item {Proveniência:(Do gr. \textunderscore brakhus\textunderscore  + \textunderscore pous\textunderscore , \textunderscore podos\textunderscore )}
\end{itemize}
Família de aves, com pés curtos.
\section{Brachýpteros}
\begin{itemize}
\item {fónica:qui}
\end{itemize}
\begin{itemize}
\item {Grp. gram.:m. pl.}
\end{itemize}
\begin{itemize}
\item {Proveniência:(Do gr. \textunderscore brakhus\textunderscore  + \textunderscore pteron\textunderscore )}
\end{itemize}
Aves palmípedes, de asas muito curtas.
\section{Brachýscio}
\begin{itemize}
\item {fónica:qui}
\end{itemize}
\begin{itemize}
\item {Grp. gram.:adj.}
\end{itemize}
\begin{itemize}
\item {Proveniência:(Do gr. \textunderscore brakhus\textunderscore  + \textunderscore skia\textunderscore )}
\end{itemize}
Diz-se dos indivíduos que, habitando a zona tórrida, projectam, expostos ao sol, uma sombra muito curta.
\section{Brachystégia}
\begin{itemize}
\item {fónica:quis}
\end{itemize}
\begin{itemize}
\item {Grp. gram.:f.}
\end{itemize}
Gênero de plantas africanas.
\section{Brachysýllabo}
\begin{itemize}
\item {fónica:quissi}
\end{itemize}
\begin{itemize}
\item {Grp. gram.:m.}
\end{itemize}
\begin{itemize}
\item {Proveniência:(Do gr. \textunderscore brakhus\textunderscore  + \textunderscore sullabe\textunderscore )}
\end{itemize}
Pé de verso grego ou latino, composto de três sýllabas breves.
\section{Brachyúro}
\begin{itemize}
\item {fónica:qui}
\end{itemize}
\begin{itemize}
\item {Grp. gram.:adj.}
\end{itemize}
\begin{itemize}
\item {Proveniência:(Do gr. \textunderscore brakhus\textunderscore  + \textunderscore oura\textunderscore )}
\end{itemize}
Que tem cauda curta.
\section{Bracicândido}
\begin{itemize}
\item {Grp. gram.:adj.}
\end{itemize}
\begin{itemize}
\item {Proveniência:(De \textunderscore braço\textunderscore  + \textunderscore cândido\textunderscore )}
\end{itemize}
Que tem braços muito brancos.
\section{Braço}
\begin{itemize}
\item {Grp. gram.:m.}
\end{itemize}
\begin{itemize}
\item {Grp. gram.:Pl.}
\end{itemize}
\begin{itemize}
\item {Utilização:Prov.}
\end{itemize}
\begin{itemize}
\item {Proveniência:(Lat. \textunderscore brachium\textunderscore )}
\end{itemize}
Cada um dos membros superiores, ligados ao ombro, no corpo humano.
Parte do braço, entre o ombro e o cotovelo.
O homem, que trabalha mecanicamente: \textunderscore a lavoira tem falta de braços\textunderscore .
Cada um dos membros anteriores dos quadrúmanos.
Poder; coragem.
Cada um dos tentáculos do pólypo.
Ramo (de árvore).
Aquillo que tem fórma de braço.
Barra, fixada horizontalmente, a sustentar alguma coisa.
Ramificação de montanha.
\textunderscore Braço de rio, de mar\textunderscore , esteiro.
\textunderscore Braço da âncora\textunderscore , cada uma das partes curvas da âncora, entre a cruz e a unha.
Madeiros, sôbre que assentam as cavernas dos navios.
Nome de diversos cabos náuticos.
A parte de differentes objectos, pela qual êstes se seguram ou se fazem mover.
\textunderscore Braço de cebolas\textunderscore , cabo ou réstia de cebolas atadas.
\section{Bracobi}
\begin{itemize}
\item {Grp. gram.:m.}
\end{itemize}
Variedade de madeira do Brasil.
\section{Braço-de-armas}
\begin{itemize}
\item {Grp. gram.:m.}
\end{itemize}
Homem forte.
Homem activo, laborioso.
\section{Braçolas}
\begin{itemize}
\item {Grp. gram.:f. pl.}
\end{itemize}
\begin{itemize}
\item {Utilização:Náut.}
\end{itemize}
\begin{itemize}
\item {Proveniência:(De \textunderscore braço\textunderscore ?)}
\end{itemize}
Lados salientes das escotilhas, para evitar que a água entre no convés.
\section{Bráctea}
\begin{itemize}
\item {Grp. gram.:f.}
\end{itemize}
\begin{itemize}
\item {Utilização:Bot.}
\end{itemize}
\begin{itemize}
\item {Proveniência:(Lat. \textunderscore bractea\textunderscore )}
\end{itemize}
Cada uma das fôlhas differentes, que cobrem a flôr antes de aberta.
\section{Bracteado}
\begin{itemize}
\item {Grp. gram.:adj.}
\end{itemize}
Que tem brácteas.
\section{Bracteal}
\begin{itemize}
\item {Grp. gram.:adj.}
\end{itemize}
Relativo a bráctea.
\section{Bracteífero}
\begin{itemize}
\item {Grp. gram.:adj.}
\end{itemize}
(V.bracteado)
\section{Bracteiforme}
\begin{itemize}
\item {Grp. gram.:adj.}
\end{itemize}
\begin{itemize}
\item {Proveniência:(De \textunderscore bráctea\textunderscore  + \textunderscore fórma\textunderscore )}
\end{itemize}
Que tem fórma de bráctea.
\section{Bractéola}
\begin{itemize}
\item {Grp. gram.:f.}
\end{itemize}
Pequena bráctea.
\section{Bracteolado}
\begin{itemize}
\item {Grp. gram.:adj.}
\end{itemize}
Que tem bractéolas.
\section{Braçudo}
\begin{itemize}
\item {Grp. gram.:adj.}
\end{itemize}
Que tem braços robustos.
\section{Bradado}
\begin{itemize}
\item {Grp. gram.:m.}
\end{itemize}
\begin{itemize}
\item {Utilização:Mús.}
\end{itemize}
\begin{itemize}
\item {Proveniência:(De \textunderscore bradar\textunderscore )}
\end{itemize}
O mesmo que \textunderscore brado\textunderscore .
Canto ecclesiástico, que reproduz as falas de Pilato no texto evangélico, e que se realiza no Domingo de Ramos e em Sexta-feira Santa.
\section{Bradador}
\begin{itemize}
\item {Grp. gram.:adj.}
\end{itemize}
Que brada.
\section{Bradal}
\begin{itemize}
\item {Grp. gram.:m.}
\end{itemize}
Instrumento de carpinteiro, feito de aço, e que substitue a verruma, quando se receia que a madeira estale ou rache.
(Por \textunderscore veredal\textunderscore , de \textunderscore vereda\textunderscore ?)
\section{Bradante}
\begin{itemize}
\item {Grp. gram.:m.  e  adj.}
\end{itemize}
Aquelle que brada.
\section{Bradar}
\begin{itemize}
\item {Grp. gram.:v. t.}
\end{itemize}
\begin{itemize}
\item {Grp. gram.:V. i.}
\end{itemize}
Dizer em alta voz: \textunderscore bradar blasphêmias\textunderscore .
Soltar brados.
Gritar.
Pedir soccorro.
Rugir.
Chamar alguém com instância.
Reclamar alguma coisa.
(Por \textunderscore badrar\textunderscore , do lat. hyp. \textunderscore balatrare\textunderscore , por \textunderscore belaterare\textunderscore )
\section{Bradejar}
\begin{itemize}
\item {Grp. gram.:v. i.}
\end{itemize}
Soltar brados. Cf. Arn. Gama, \textunderscore Motim\textunderscore , 408.
\section{Bradicardia}
\begin{itemize}
\item {Grp. gram.:f.}
\end{itemize}
\begin{itemize}
\item {Proveniência:(Do gr. \textunderscore bradus\textunderscore  + \textunderscore kardia\textunderscore )}
\end{itemize}
Pulsação lenta do coração.
\section{Bradicardíaco}
\begin{itemize}
\item {Grp. gram.:adj.}
\end{itemize}
Relativo á bradicardia.
Que soffre bradicardia.
\section{Bradipepsia}
\begin{itemize}
\item {Grp. gram.:f.}
\end{itemize}
\begin{itemize}
\item {Proveniência:(Do gr. \textunderscore bradus\textunderscore  + \textunderscore pepsia\textunderscore )}
\end{itemize}
Digestão difficil, demorada.
\section{Bradípode}
\begin{itemize}
\item {Grp. gram.:m.}
\end{itemize}
\begin{itemize}
\item {Proveniência:(Do gr. \textunderscore bradus\textunderscore  + \textunderscore pous\textunderscore , \textunderscore podos\textunderscore )}
\end{itemize}
Animal que, pela conformação dos pés, tem marcha difficil e muito lenta.
\section{Bradispermatismo}
\begin{itemize}
\item {Grp. gram.:m.}
\end{itemize}
\begin{itemize}
\item {Proveniência:(Do gr. \textunderscore bradus\textunderscore  + \textunderscore sperma\textunderscore )}
\end{itemize}
Emissão lenta e difficil do esperma.
\section{Braditrofia}
\begin{itemize}
\item {Grp. gram.:f.}
\end{itemize}
\begin{itemize}
\item {Proveniência:(Do gr. \textunderscore bradus\textunderscore  + \textunderscore trophe\textunderscore )}
\end{itemize}
Retardamento da nutrição.
\section{Braditrófico}
\begin{itemize}
\item {Grp. gram.:adj.}
\end{itemize}
Relativo á braditrofia.
\section{Brado}
\begin{itemize}
\item {Grp. gram.:m.}
\end{itemize}
Acto de \textunderscore bradar\textunderscore .
Grito; clamor.
Fama: \textunderscore aquella victória deu brado\textunderscore .
Queixa, reclamação em voz alta.
\section{Bradycardia}
\begin{itemize}
\item {Grp. gram.:f.}
\end{itemize}
\begin{itemize}
\item {Proveniência:(Do gr. \textunderscore bradus\textunderscore  + \textunderscore kardia\textunderscore )}
\end{itemize}
Pulsação lenta do coração.
\section{Bradycardíaco}
\begin{itemize}
\item {Grp. gram.:adj.}
\end{itemize}
Relativo á bradycardia.
Que soffre bradycardia.
\section{Bradypepsia}
\begin{itemize}
\item {Grp. gram.:f.}
\end{itemize}
\begin{itemize}
\item {Proveniência:(Do gr. \textunderscore bradus\textunderscore  + \textunderscore pepsia\textunderscore )}
\end{itemize}
Digestão difficil, demorada.
\section{Bradýpode}
\begin{itemize}
\item {Grp. gram.:m.}
\end{itemize}
\begin{itemize}
\item {Proveniência:(Do gr. \textunderscore bradus\textunderscore  + \textunderscore pous\textunderscore , \textunderscore podos\textunderscore )}
\end{itemize}
Animal que, pela conformação dos pés, tem marcha difficil e muito lenta.
\section{Bradyspermatismo}
\begin{itemize}
\item {Grp. gram.:m.}
\end{itemize}
\begin{itemize}
\item {Proveniência:(Do gr. \textunderscore bradus\textunderscore  + \textunderscore sperma\textunderscore )}
\end{itemize}
Emissão lenta e difficil do esperma.
\section{Bradytrophia}
\begin{itemize}
\item {Grp. gram.:f.}
\end{itemize}
\begin{itemize}
\item {Proveniência:(Do gr. \textunderscore bradus\textunderscore  + \textunderscore trophe\textunderscore )}
\end{itemize}
Retardamento da nutrição.
\section{Bradytróphico}
\begin{itemize}
\item {Grp. gram.:adj.}
\end{itemize}
Relativo á bradytrophia.
\section{Brafoneira}
\begin{itemize}
\item {Grp. gram.:f.}
\end{itemize}
Parte das armaduras antigas, para proteger a parte superior do braço e os ombros. Cf. \textunderscore Port. Mon. Hist.\textunderscore , \textunderscore Script.\textunderscore , 270.
(Ant. cast. \textunderscore brahonera\textunderscore )
\section{Braga}
\begin{itemize}
\item {Grp. gram.:f.}
\end{itemize}
\begin{itemize}
\item {Utilização:Ant.}
\end{itemize}
\begin{itemize}
\item {Proveniência:(Lat. \textunderscore braca\textunderscore )}
\end{itemize}
Argola de ferro, que cingia a parte inferior da perna dos condemnados a trabalhos forçados, e que se ligava a uma corrente.
Qualidade, laia, estôfo: \textunderscore o tal sujeito é da braga do diabo\textunderscore .
(Colhido em Turquel)
\section{Braga}
\begin{itemize}
\item {Grp. gram.:f.}
\end{itemize}
Muro, que servia de tranqueira, nas antigas fortificações.
(B. lat. \textunderscore braca\textunderscore )
\section{Bragada}
\begin{itemize}
\item {Grp. gram.:f.}
\end{itemize}
\begin{itemize}
\item {Grp. gram.:Pl.}
\end{itemize}
\begin{itemize}
\item {Proveniência:(De \textunderscore bragas\textunderscore )}
\end{itemize}
Parte da perna, coberta pelas bragas, (calções).
Veias da perna dos cavallos, nas quaes êstes se sangram.
\section{Bragádiga}
\begin{itemize}
\item {Grp. gram.:f.}
\end{itemize}
\begin{itemize}
\item {Utilização:Ant.}
\end{itemize}
\begin{itemize}
\item {Proveniência:(De \textunderscore bragal\textunderscore )}
\end{itemize}
O valor de um bragal, considerado como moéda ou como unidade, em contratos de compra e venda.
\section{Bragado}
\begin{itemize}
\item {Grp. gram.:adj.}
\end{itemize}
\begin{itemize}
\item {Utilização:Bras. do S}
\end{itemize}
\begin{itemize}
\item {Proveniência:(De \textunderscore bragas\textunderscore )}
\end{itemize}
Diz-se do animal, que tem as pernas de côr differente da do resto do corpo.
Que tem manchas brancas a atravessar-lhe a barriga.
\section{Bragal}
\begin{itemize}
\item {Grp. gram.:m.}
\end{itemize}
\begin{itemize}
\item {Proveniência:(Do b. lat. \textunderscore bracale\textunderscore )}
\end{itemize}
Pano grosso, de que se faziam bragas.
Preço de uma porção de bragal, que era tomado como unidade em certas transacções.
A roupa branca de uma casa.
\section{Bragançano}
\begin{itemize}
\item {Grp. gram.:m. adj.}
\end{itemize}
O mesmo que \textunderscore braganção\textunderscore .
\section{Braganção}
\begin{itemize}
\item {Grp. gram.:m.  e  adj.}
\end{itemize}
O mesmo que \textunderscore bragantino\textunderscore .
\section{Bragani}
\begin{itemize}
\item {Grp. gram.:m.}
\end{itemize}
O mesmo que \textunderscore barga\textunderscore ^1.
\section{Bragantão}
\begin{itemize}
\item {Grp. gram.:m.}
\end{itemize}
\begin{itemize}
\item {Proveniência:(De \textunderscore bragante\textunderscore )}
\end{itemize}
Homem devasso, valdevinos.
\section{Bragante}
\begin{itemize}
\item {Grp. gram.:m.  e  adj.}
\end{itemize}
(V.bargante)
\section{Bragantim}
\begin{itemize}
\item {Grp. gram.:m.}
\end{itemize}
\begin{itemize}
\item {Utilização:Ant.}
\end{itemize}
O mesmo que \textunderscore bergantim\textunderscore .
(Cp. \textunderscore fragatim\textunderscore )
\section{Bragantino}
\begin{itemize}
\item {Grp. gram.:m.}
\end{itemize}
\begin{itemize}
\item {Grp. gram.:Adj.}
\end{itemize}
\begin{itemize}
\item {Proveniência:(Do lat. \textunderscore Bragantia\textunderscore , n. p.)}
\end{itemize}
Homem natural de Bragança.
Relativo a esta cidade.
\section{Bragas}
\begin{itemize}
\item {Grp. gram.:f. pl.}
\end{itemize}
\begin{itemize}
\item {Utilização:Ant.}
\end{itemize}
\begin{itemize}
\item {Proveniência:(Lat. \textunderscore bracae\textunderscore )}
\end{itemize}
Calças largas e curtas; calções.
\section{Bragueiro}
\begin{itemize}
\item {Grp. gram.:m.}
\end{itemize}
\begin{itemize}
\item {Utilização:Ant.}
\end{itemize}
\begin{itemize}
\item {Proveniência:(De \textunderscore bragas\textunderscore )}
\end{itemize}
Cinta, funda, para comprimir roturas ou segurar hérnias.
Cueiro.
Nome de alguns cabos de navio.
Peça de vestuário, que cobria o corpo, desde a cintura aos joêlhos.
\section{Braguês}
\begin{itemize}
\item {Grp. gram.:m.  e  adj.}
\end{itemize}
O mesmo que \textunderscore bracarense\textunderscore .
\section{Braguilha}
\begin{itemize}
\item {Grp. gram.:f.}
\end{itemize}
\begin{itemize}
\item {Proveniência:(De \textunderscore bragas\textunderscore )}
\end{itemize}
Parte deanteira das bragas, calças, calções ou ceroilas, em que se abotôam estas peças de vestuário.
\section{Braguinha}
\begin{itemize}
\item {Grp. gram.:f.}
\end{itemize}
Antigo instrumento, espécie de viola, ainda hoje usada nos Açores.
\section{Brahmane}
\begin{itemize}
\item {Grp. gram.:m.}
\end{itemize}
Sacerdote indiano da religião de Brahma, ou membro da primeira das quatro castas indianas.
(Sânscr. \textunderscore brahman\textunderscore )
\section{Brahmânico}
\begin{itemize}
\item {Grp. gram.:adj.}
\end{itemize}
Relativo aos brâhmanes, ou aos seus systemas.
\section{Brahmanismo}
\begin{itemize}
\item {Grp. gram.:m.}
\end{itemize}
Religião e systema dos brâhmanes.
\section{Brahme}
\begin{itemize}
\item {Grp. gram.:m.}
\end{itemize}
O mesmo que \textunderscore brahmane\textunderscore .
\section{Brahmismo}
\begin{itemize}
\item {Grp. gram.:m.}
\end{itemize}
O mesmo que \textunderscore brahmanismo\textunderscore . Cf. Herculano, \textunderscore Opúscul.\textunderscore , IV, 56.
\section{Brahmista}
\begin{itemize}
\item {Grp. gram.:m.}
\end{itemize}
\begin{itemize}
\item {Proveniência:(De \textunderscore Brahma\textunderscore , n. p.)}
\end{itemize}
Sectário do culto de Brahma, do deus supremo do brahmanismo.
\section{Brai}
\begin{itemize}
\item {Grp. gram.:m.}
\end{itemize}
Pequeno arbusto da Guiné.
\section{Bralda}
\begin{itemize}
\item {Grp. gram.:m.}
\end{itemize}
O mesmo que \textunderscore bralha\textunderscore ^1?:«\textunderscore ...como os escritores das braldas de ouro testemunhão\textunderscore ». \textunderscore Peregrinação\textunderscore , LXIV.
\section{Bralha}
\begin{itemize}
\item {Grp. gram.:f.}
\end{itemize}
\begin{itemize}
\item {Utilização:Ant.}
\end{itemize}
Templo gentílico.
\section{Bralha}
\begin{itemize}
\item {Grp. gram.:f.}
\end{itemize}
\begin{itemize}
\item {Utilização:Bras. do N}
\end{itemize}
Certo passo das cavalgaduras, espécie de trote.
\section{Bralhar}
\begin{itemize}
\item {Grp. gram.:v. i.}
\end{itemize}
\begin{itemize}
\item {Utilização:Bras. do N}
\end{itemize}
Andar a passo de bralha.
\section{Brama}
\begin{itemize}
\item {Grp. gram.:f.}
\end{itemize}
O mesmo que \textunderscore berra\textunderscore , cio.
\section{Bramá}
\begin{itemize}
\item {Grp. gram.:m.  e  adj.}
\end{itemize}
O mesmo que \textunderscore birman\textunderscore . Cf. \textunderscore Lusíadas\textunderscore , X, 129.
\section{Bramadeiro}
\begin{itemize}
\item {Grp. gram.:m.}
\end{itemize}
\begin{itemize}
\item {Proveniência:(De \textunderscore bramar\textunderscore )}
\end{itemize}
Lugar, em que se juntam veados, quando estão com o cio.
\section{Bramador}
\begin{itemize}
\item {Grp. gram.:m.  e  adj.}
\end{itemize}
O mesmo que \textunderscore bramante\textunderscore ^1.
\section{Brâmane}
\begin{itemize}
\item {Grp. gram.:m.}
\end{itemize}
Sacerdote indiano da religião de Brahma, ou membro da primeira das quatro castas indianas.
(Sânscr. \textunderscore brahman\textunderscore )
\section{Bramânico}
\begin{itemize}
\item {Grp. gram.:adj.}
\end{itemize}
Relativo aos brâmanes, ou aos seus systemas.
\section{Bramanismo}
\begin{itemize}
\item {Grp. gram.:m.}
\end{itemize}
Religião e systema dos brâmanes.
\section{Bramante}
\begin{itemize}
\item {Grp. gram.:m.  e  adj.}
\end{itemize}
Aquelle que brama.
\section{Bramante}
\begin{itemize}
\item {Grp. gram.:m.}
\end{itemize}
\begin{itemize}
\item {Utilização:Ant.}
\end{itemize}
O mesmo que \textunderscore barbante\textunderscore .
\section{Bramar}
\begin{itemize}
\item {Grp. gram.:v. i.}
\end{itemize}
\begin{itemize}
\item {Proveniência:(It. \textunderscore bramare\textunderscore , do germ.)}
\end{itemize}
Berrar, (falando-se de veados).
Gritar.
Supplicar, em alta voz.
Irritar-se.
Fazer accusações.
Rugir; bramir.
Retumbar.
Têr cio, (falando-se de veados e ainda de outros animaes).
\section{Brame}
\begin{itemize}
\item {Grp. gram.:m.}
\end{itemize}
O mesmo que \textunderscore brâmane\textunderscore .
\section{Bramido}
\begin{itemize}
\item {Grp. gram.:m.}
\end{itemize}
Acção de \textunderscore bramir\textunderscore .
\section{Bramidor}
\begin{itemize}
\item {Grp. gram.:m.  e  adj.}
\end{itemize}
O que brame.
\section{Brâmine}
\begin{itemize}
\item {Grp. gram.:m.}
\end{itemize}
(V.brahmane)
\section{Bramir}
\begin{itemize}
\item {Grp. gram.:v. i.}
\end{itemize}
Rugir.
Gritar, berrar muito.
Soltar gritos de cólera.
Estrondear, retumbar.
(Cp. \textunderscore bramar\textunderscore )
\section{Bramismo}
\begin{itemize}
\item {Grp. gram.:m.}
\end{itemize}
O mesmo que \textunderscore bramanismo\textunderscore . Cf. Herculano, \textunderscore Opúscul.\textunderscore , IV, 56.
\section{Bramista}
\begin{itemize}
\item {Grp. gram.:m.}
\end{itemize}
\begin{itemize}
\item {Proveniência:(De \textunderscore Brahma\textunderscore , n. p.)}
\end{itemize}
Sectário do culto de Brahma, do deus supremo do bramanismo.
\section{Bramoso}
\begin{itemize}
\item {Grp. gram.:adj.}
\end{itemize}
Que brama.
Raivoso.
Tempestuoso. Cf. Macedo, \textunderscore Oriente\textunderscore , I, 38.
\section{Branca}
\begin{itemize}
\item {Grp. gram.:f.}
\end{itemize}
\begin{itemize}
\item {Utilização:Bras}
\end{itemize}
Cabello branco, can.
Antiga moéda de prata.
Aguardente ou cachaça.
(Fem. de \textunderscore branco\textunderscore )
\section{Branca}
\begin{itemize}
\item {Grp. gram.:f.}
\end{itemize}
Grilheta, braga.
(Cp. \textunderscore braga\textunderscore ^1)
\section{Brancacento}
\begin{itemize}
\item {Grp. gram.:adj.}
\end{itemize}
\begin{itemize}
\item {Proveniência:(De \textunderscore branco\textunderscore )}
\end{itemize}
Alvacento, quási branco.
\section{Brancagem}
\begin{itemize}
\item {Grp. gram.:f.}
\end{itemize}
\begin{itemize}
\item {Proveniência:(De \textunderscore branca\textunderscore , moéda)}
\end{itemize}
Antigo imposto, sôbre a carne e o pão que se vendiam.
\section{Brançal}
\begin{itemize}
\item {Grp. gram.:m.}
\end{itemize}
Casta de uva preta de Monção.
\section{Brancarana}
\begin{itemize}
\item {Grp. gram.:f.}
\end{itemize}
\begin{itemize}
\item {Utilização:Bras}
\end{itemize}
\begin{itemize}
\item {Proveniência:(Do port. \textunderscore branco\textunderscore  + tupi \textunderscore rana\textunderscore )}
\end{itemize}
Mulata clara.
\section{Branca-ursina}
\begin{itemize}
\item {Grp. gram.:f.}
\end{itemize}
O mesmo que \textunderscore acantho\textunderscore .
\section{Brancelhe}
\begin{itemize}
\item {fónica:cê}
\end{itemize}
\begin{itemize}
\item {Grp. gram.:m.}
\end{itemize}
\begin{itemize}
\item {Proveniência:(De \textunderscore brançal\textunderscore )}
\end{itemize}
Casta de uva minhota.
\section{Branchiado}
\begin{itemize}
\item {fónica:qui}
\end{itemize}
\begin{itemize}
\item {Grp. gram.:adj.}
\end{itemize}
Que tem brânchias.
\section{Branchial}
\begin{itemize}
\item {fónica:qui}
\end{itemize}
\begin{itemize}
\item {Grp. gram.:adj.}
\end{itemize}
Relativo ás \textunderscore brânchias\textunderscore .
\section{Branchiápodes}
\begin{itemize}
\item {fónica:qui}
\end{itemize}
\begin{itemize}
\item {Grp. gram.:m. pl.}
\end{itemize}
\begin{itemize}
\item {Proveniência:(Do gr. \textunderscore brankhia\textunderscore  + \textunderscore pous\textunderscore , \textunderscore podos\textunderscore )}
\end{itemize}
Ordem de crustáceos.
\section{Brânchias}
\begin{itemize}
\item {fónica:qui}
\end{itemize}
\begin{itemize}
\item {Grp. gram.:f. pl.}
\end{itemize}
\begin{itemize}
\item {Proveniência:(Do gr. \textunderscore brankhia\textunderscore )}
\end{itemize}
Guelras, apparelho respiratório dos animaes que vivem debaixo da água.
\section{Branchióstega}
\begin{itemize}
\item {fónica:qui}
\end{itemize}
\begin{itemize}
\item {Grp. gram.:f.}
\end{itemize}
Membrana, que fica debaixo dos opérculos dos peixes e que serve para cobrir as guelras, dilatando-se ou contrahindo-se.
\section{Branco}
\begin{itemize}
\item {Grp. gram.:adj.}
\end{itemize}
\begin{itemize}
\item {Grp. gram.:M.}
\end{itemize}
\begin{itemize}
\item {Utilização:Bras}
\end{itemize}
\begin{itemize}
\item {Grp. gram.:Loc. adv.}
\end{itemize}
\begin{itemize}
\item {Utilização:Gír.}
\end{itemize}
\begin{itemize}
\item {Utilização:Gír. de Lisbôa.}
\end{itemize}
\begin{itemize}
\item {Grp. gram.:Pl.}
\end{itemize}
\begin{itemize}
\item {Proveniência:(Do germ. \textunderscore blank\textunderscore )}
\end{itemize}
Que tem a côr da neve ou do leite: \textunderscore papel branco\textunderscore .
Cuja côr se aproxima daquella.
Pállido: \textunderscore ficou branco, de susto\textunderscore .
Que tem cans.
Que não teve prêmio, (falando-se de bilhetes de lotaria).
A côr branca: \textunderscore pintado de branco\textunderscore .
Substância, com que se pinta de branco.
Homem de raça branca: \textunderscore em Mossâmedes, há mais pretos do que brancos\textunderscore .
Senhor, patrão.
Clara do ovo.
Esclerótica.
Espaço entre linhas escritas.
Antiga moéda de prata.
\textunderscore De ponto em branco\textunderscore , com apuro, com esmêro.
\textunderscore Em branco\textunderscore , em jejum.
Sem têr estudado nada: \textunderscore o alumno levava a lição em branco\textunderscore .
Indivíduo bronco, pacóvio, que os gatunos illudem facílmente, sobretudo com o chamado \textunderscore conto do vigário\textunderscore . Cf. \textunderscore Capital\textunderscore , de 22-VII-912.
O mesmo que \textunderscore lençol\textunderscore .
\textunderscore Padres brancos\textunderscore , congregação religiosa, destinada ás missões da Arábia.
\section{Brancura}
\begin{itemize}
\item {Grp. gram.:f.}
\end{itemize}
Qualidade do que é branco.
\section{Branda}
\begin{itemize}
\item {Grp. gram.:f.}
\end{itemize}
\begin{itemize}
\item {Utilização:Prov.}
\end{itemize}
\begin{itemize}
\item {Utilização:minh.}
\end{itemize}
\begin{itemize}
\item {Proveniência:(De \textunderscore brando\textunderscore ?)}
\end{itemize}
Tapada ou pastagem nas montanhas.
\section{Branda}
\begin{itemize}
\item {Grp. gram.:f.}
\end{itemize}
Dança de roda, alegre e de andamento vivo, que foi usado em França. Cf. \textunderscore Diccion. Musical\textunderscore .
(Cf. fr. \textunderscore branle\textunderscore )
\section{Brandal}
\begin{itemize}
\item {Grp. gram.:m.}
\end{itemize}
\begin{itemize}
\item {Utilização:Náut.}
\end{itemize}
\begin{itemize}
\item {Proveniência:(De \textunderscore brando\textunderscore )}
\end{itemize}
Cada um dos cabos, que aguentam os mastaréus para barlavento, ficando brandos os de sotavento.
\section{Brandalhão}
\begin{itemize}
\item {Grp. gram.:adj.}
\end{itemize}
Muito brando, indolente.
\section{Brandamente}
\begin{itemize}
\item {Grp. gram.:adv.}
\end{itemize}
De modo brando.
Suavemente.
\section{Brandão}
\begin{itemize}
\item {Grp. gram.:m.}
\end{itemize}
Tocha, vela grande de cera.
(B. lat. \textunderscore brando\textunderscore )
\section{Brandear}
\textunderscore v. t.\textunderscore  (e der.)
O mesmo que \textunderscore abrandar\textunderscore , etc.
\section{Brandeza}
\begin{itemize}
\item {Grp. gram.:f.}
\end{itemize}
\begin{itemize}
\item {Utilização:Des.}
\end{itemize}
O mesmo que \textunderscore brandura\textunderscore .
\section{Brandezém}
\begin{itemize}
\item {Grp. gram.:m.}
\end{itemize}
\begin{itemize}
\item {Utilização:ant.}
\end{itemize}
Véu de linho branco e fino, com que se tocava nos corpos ou sepulcros dos santos, e que alguns Pontífices mandavam como relíquias aos Príncipes.
(Relaciona-se com \textunderscore Brindísi\textunderscore , n. p.)
\section{Brandíloquo}
\begin{itemize}
\item {Grp. gram.:adj.}
\end{itemize}
\begin{itemize}
\item {Proveniência:(Do lat. \textunderscore blandus\textunderscore  + \textunderscore loqui\textunderscore )}
\end{itemize}
Que tem voz suave, que fala com brandura.
\section{Brandimento}
\begin{itemize}
\item {Grp. gram.:f.}
\end{itemize}
Acção de \textunderscore brandir\textunderscore .
\section{Brandir}
\begin{itemize}
\item {Grp. gram.:v. t.}
\end{itemize}
\begin{itemize}
\item {Grp. gram.:V. i.}
\end{itemize}
Agitar com a mão, antes de atirar ou descarregar (espada, lança, etc.).
Menear.
Oscillar.
Agitar-se, vibrando.
(Cp. cast. \textunderscore blandir\textunderscore )
\section{Brando}
\begin{itemize}
\item {Grp. gram.:adj.}
\end{itemize}
\begin{itemize}
\item {Proveniência:(Lat. \textunderscore blandus\textunderscore )}
\end{itemize}
Que cede facilmente á pressão; molle; tenro.
Macio.
Affável; agradável.
Sereno: \textunderscore tempo brando\textunderscore .
Moderado.
Froixo; vagaroso: \textunderscore carácter brando\textunderscore .
\section{Brandouro}
\begin{itemize}
\item {Grp. gram.:m.}
\end{itemize}
\begin{itemize}
\item {Utilização:Prov.}
\end{itemize}
\begin{itemize}
\item {Utilização:minh.}
\end{itemize}
Pescaria no mais interior do rio.
\section{Brandura}
\begin{itemize}
\item {Grp. gram.:f.}
\end{itemize}
\begin{itemize}
\item {Utilização:Prov.}
\end{itemize}
\begin{itemize}
\item {Utilização:alent.}
\end{itemize}
\begin{itemize}
\item {Utilização:Prov.}
\end{itemize}
Qualidade do que é brando.
Moderação, suavidade: \textunderscore a brandura dos nossos costumes\textunderscore .
Humidade matutina.
Chuva branda; molinha. (Colhido em Turquel)
\section{Brandúzio}
\begin{itemize}
\item {Grp. gram.:adj.}
\end{itemize}
\begin{itemize}
\item {Utilização:Des.}
\end{itemize}
O mesmo que \textunderscore brandalhão\textunderscore .
\section{Branil}
\begin{itemize}
\item {Grp. gram.:m.}
\end{itemize}
\begin{itemize}
\item {Utilização:Prov.}
\end{itemize}
\begin{itemize}
\item {Utilização:trasm.}
\end{itemize}
Sítio, onde se desenvolvem muito os frutos.
\section{Branjo}
\begin{itemize}
\item {Grp. gram.:m.}
\end{itemize}
Casta de uva de Amarante.
\section{Branqueação}
\begin{itemize}
\item {Grp. gram.:f.}
\end{itemize}
Acto ou effeito de branquear.
\section{Branqueador}
\begin{itemize}
\item {Grp. gram.:m.  e  adj.}
\end{itemize}
O que branqueia.
\section{Branqueadura}
\begin{itemize}
\item {Grp. gram.:f.}
\end{itemize}
O mesmo que \textunderscore branqueamento\textunderscore .
\section{Branqueamento}
\begin{itemize}
\item {Grp. gram.:f.}
\end{itemize}
Acção de \textunderscore branquear\textunderscore .
\section{Branquear}
\begin{itemize}
\item {Grp. gram.:v. t.}
\end{itemize}
\begin{itemize}
\item {Grp. gram.:V. i.}
\end{itemize}
Tornar branco.
Cobrir com substância branca: \textunderscore branquear a parede\textunderscore .
Limpar.
O mesmo que \textunderscore branquejar\textunderscore .
\section{Branquearia}
\begin{itemize}
\item {Grp. gram.:f.}
\end{itemize}
\begin{itemize}
\item {Proveniência:(De \textunderscore branquear\textunderscore )}
\end{itemize}
Lugar ou estabelecimento, em que se coram teias de pano ou cera.
\section{Branqueio}
\begin{itemize}
\item {Grp. gram.:m.}
\end{itemize}
Acto ou effeito de \textunderscore branquear\textunderscore .
\section{Branqueira}
\begin{itemize}
\item {Grp. gram.:f.}
\end{itemize}
\begin{itemize}
\item {Proveniência:(De \textunderscore branco\textunderscore ?)}
\end{itemize}
Rede de três panos de emmalhar.
\section{Branquejar}
\begin{itemize}
\item {Grp. gram.:v. i.}
\end{itemize}
\begin{itemize}
\item {Proveniência:(De \textunderscore branco\textunderscore )}
\end{itemize}
Alvejar; mostrar-se branco: \textunderscore branquejam casas ao longe\textunderscore .
\section{Branqueta}
\begin{itemize}
\item {fónica:quê}
\end{itemize}
\begin{itemize}
\item {Grp. gram.:f.}
\end{itemize}
\begin{itemize}
\item {Proveniência:(De \textunderscore branco\textunderscore )}
\end{itemize}
Pano, que se collocava entre o týmpano e o tympanilho do prelo.
Espécie de tecido branco, de que se faz o vestuário dos sargaceiros e dos pescadores poveiros.
Esse vestuário.
\section{Branquiado}
\begin{itemize}
\item {Grp. gram.:adj.}
\end{itemize}
Que tem brânquias.
\section{Branquial}
\begin{itemize}
\item {Grp. gram.:adj.}
\end{itemize}
Relativo ás \textunderscore brânquias\textunderscore .
\section{Branquiápodes}
\begin{itemize}
\item {Grp. gram.:m. pl.}
\end{itemize}
\begin{itemize}
\item {Proveniência:(Do gr. \textunderscore brankhia\textunderscore  + \textunderscore pous\textunderscore , \textunderscore podos\textunderscore )}
\end{itemize}
Ordem de crustáceos.
\section{Brânquias}
\begin{itemize}
\item {Grp. gram.:f. pl.}
\end{itemize}
\begin{itemize}
\item {Proveniência:(Do gr. \textunderscore brankhia\textunderscore )}
\end{itemize}
Guelras, apparelho respiratório dos animaes que vivem debaixo da água.
\section{Branquicento}
\begin{itemize}
\item {Grp. gram.:adj.}
\end{itemize}
O mesmo que \textunderscore brancacento\textunderscore .
\section{Branquidão}
\begin{itemize}
\item {Grp. gram.:f.}
\end{itemize}
O mesmo que \textunderscore brancura\textunderscore .
\section{Branquidor}
\begin{itemize}
\item {Grp. gram.:m.}
\end{itemize}
\begin{itemize}
\item {Proveniência:(De \textunderscore branquir\textunderscore )}
\end{itemize}
Aquelle que branqueia ou limpa oiro ou prata.
\section{Branquimento}
\begin{itemize}
\item {Grp. gram.:m.}
\end{itemize}
\begin{itemize}
\item {Proveniência:(De \textunderscore branquir\textunderscore )}
\end{itemize}
Acto de \textunderscore branquir\textunderscore .
Preparação de sarro fervido com sal, para branquear peças de prata.
\section{Branquinha}
\begin{itemize}
\item {Grp. gram.:f.}
\end{itemize}
\begin{itemize}
\item {Utilização:Bras. do N}
\end{itemize}
\begin{itemize}
\item {Utilização:Bras. de Minas}
\end{itemize}
Ardil; fraude.
Aguardente.
\section{Branquióstega}
\begin{itemize}
\item {Grp. gram.:f.}
\end{itemize}
Membrana, que fica debaixo dos opérculos dos peixes e que serve para cobrir as guelras, dilatando-se ou contrahindo-se.
\section{Branquir}
\begin{itemize}
\item {Grp. gram.:v. t.}
\end{itemize}
\begin{itemize}
\item {Proveniência:(De \textunderscore branco\textunderscore )}
\end{itemize}
Branquear (peças de oiro ou prata).
\section{Branza}
\begin{itemize}
\item {Grp. gram.:f.}
\end{itemize}
Rama de pinheiro, caruma.
(Cp. \textunderscore franças\textunderscore  e fr. \textunderscore branche\textunderscore )
\section{Braquear}
\begin{itemize}
\item {Grp. gram.:v. i.}
\end{itemize}
Mover o estribo, para esporear de chaquéu o cavallo.
\section{Braquelitro}
\begin{itemize}
\item {Grp. gram.:adj.}
\end{itemize}
\begin{itemize}
\item {Grp. gram.:M. pl.}
\end{itemize}
\begin{itemize}
\item {Proveniência:(Do gr. \textunderscore brakhus\textunderscore  + \textunderscore elutron\textunderscore )}
\end{itemize}
Que tem elitros curtos.
Fam. de insectos coleópteros, com elitros curtos.
\section{Bráquia}
\begin{itemize}
\item {Grp. gram.:f.}
\end{itemize}
\begin{itemize}
\item {Proveniência:(Do gr. \textunderscore brakhus\textunderscore )}
\end{itemize}
Sinal orthográphico que, collocado sobre uma vogal, indica que ella é breve, assim: ě.
\section{Braquial}
\begin{itemize}
\item {Grp. gram.:adj.}
\end{itemize}
\begin{itemize}
\item {Proveniência:(Lat. \textunderscore brachialis\textunderscore )}
\end{itemize}
Relativo ao braço.
\section{Braquibiota}
\begin{itemize}
\item {Grp. gram.:adj.}
\end{itemize}
\begin{itemize}
\item {Proveniência:(Do gr. \textunderscore brakhus\textunderscore  + \textunderscore biotes\textunderscore )}
\end{itemize}
Que tem vida curta.
\section{Braquicataléctico}
\begin{itemize}
\item {Grp. gram.:adj.}
\end{itemize}
\begin{itemize}
\item {Proveniência:(Do gr. \textunderscore brakhus\textunderscore  + \textunderscore katalektos\textunderscore )}
\end{itemize}
Dizia-se dos versos gregos ou latinos, a que faltava um pé.
\section{Braquicatalecto}
\begin{itemize}
\item {Grp. gram.:adj.}
\end{itemize}
\begin{itemize}
\item {Proveniência:(Do gr. \textunderscore brakhus\textunderscore  + \textunderscore katalektos\textunderscore )}
\end{itemize}
Dizia-se dos versos gregos ou latinos, a que faltava um pé.
\section{Braquicefalia}
\begin{itemize}
\item {Grp. gram.:f.}
\end{itemize}
Estado de \textunderscore braquicéfalo\textunderscore .
\section{Braquicéfalo}
\begin{itemize}
\item {Grp. gram.:m.  e  adj.}
\end{itemize}
\begin{itemize}
\item {Proveniência:(Do gr. \textunderscore brakhus\textunderscore  + \textunderscore kephale\textunderscore )}
\end{itemize}
Diz-se do indivíduo, cujo crânio, observado de cima, apresenta a fórma de um ovo, mas mais curta e arredondada posteriormente.
\section{Braquícero}
\begin{itemize}
\item {Grp. gram.:adj.}
\end{itemize}
\begin{itemize}
\item {Grp. gram.:M. pl.}
\end{itemize}
\begin{itemize}
\item {Proveniência:(Do gr. \textunderscore brakhus\textunderscore  + \textunderscore keras\textunderscore )}
\end{itemize}
Que tem cornos curtos.
Insectos coleópteros, de antennas curtas.
\section{Braquicoreia}
\begin{itemize}
\item {Grp. gram.:f.}
\end{itemize}
\begin{itemize}
\item {Proveniência:(Do gr. \textunderscore brakhus\textunderscore  + \textunderscore khoreios\textunderscore )}
\end{itemize}
Pé de verso, grego ou latino, formado de uma sýllaba longa entre duas breves.
\section{Braquidáctilo}
\begin{itemize}
\item {Grp. gram.:adj.}
\end{itemize}
\begin{itemize}
\item {Proveniência:(Do gr. \textunderscore brakus\textunderscore  + \textunderscore daktulos\textunderscore )}
\end{itemize}
Que tem dedos curtos.
\section{Braquídeo}
\begin{itemize}
\item {Grp. gram.:adj.}
\end{itemize}
\begin{itemize}
\item {Proveniência:(Do gr. \textunderscore brakhion\textunderscore  + \textunderscore eidos\textunderscore )}
\end{itemize}
Que tem fórma de braço.
\section{Braquidiagonal}
\begin{itemize}
\item {Grp. gram.:adj.}
\end{itemize}
\begin{itemize}
\item {Utilização:Geol.}
\end{itemize}
\begin{itemize}
\item {Proveniência:(Do gr. \textunderscore brakhus\textunderscore  + \textunderscore diagonios\textunderscore )}
\end{itemize}
Diz-se do menor dos três eixos dos crystaes do systema orthorhômbico.
\section{Braquídoma}
\begin{itemize}
\item {Grp. gram.:m.}
\end{itemize}
\begin{itemize}
\item {Utilização:Geol.}
\end{itemize}
\begin{itemize}
\item {Proveniência:(Do gr. \textunderscore brakhus\textunderscore  + \textunderscore doma\textunderscore )}
\end{itemize}
Prisma transversal, com eixo braquidiagonal.
\section{Braquigrafia}
\begin{itemize}
\item {Grp. gram.:f.}
\end{itemize}
Arte de \textunderscore braquígrafo\textunderscore .
\section{Braquígrafo}
\begin{itemize}
\item {Grp. gram.:m.}
\end{itemize}
\begin{itemize}
\item {Proveniência:(Do gr. \textunderscore brakhus\textunderscore  + \textunderscore graphein\textunderscore )}
\end{itemize}
Aquelle que escreve por abreviaturas.
\section{Braquiocefálico}
\begin{itemize}
\item {Grp. gram.:adj.}
\end{itemize}
\begin{itemize}
\item {Utilização:Anat.}
\end{itemize}
\begin{itemize}
\item {Proveniência:(De \textunderscore braquiocéphalo\textunderscore )}
\end{itemize}
Que fornece os vasos á cabeça e ao braço, (falando-se do tronco arterial).
\section{Braquiocéfalo}
\begin{itemize}
\item {Grp. gram.:m.}
\end{itemize}
\begin{itemize}
\item {Proveniência:(Do gr. \textunderscore brakhion\textunderscore  + \textunderscore kephale\textunderscore )}
\end{itemize}
Cefalópode que tem braços.
\section{Braquiologia}
\begin{itemize}
\item {Grp. gram.:f.}
\end{itemize}
\begin{itemize}
\item {Proveniência:(Do gr. \textunderscore brakhus\textunderscore  + \textunderscore logos\textunderscore )}
\end{itemize}
Locução obscura, por sêr muito lacónica.
\section{Braquiológico}
\begin{itemize}
\item {Grp. gram.:adj.}
\end{itemize}
Relativo a braquiologia.
Em que há \textunderscore braquiologia\textunderscore .
\section{Braquiópode}
\begin{itemize}
\item {Grp. gram.:adj.}
\end{itemize}
\begin{itemize}
\item {Utilização:Zool.}
\end{itemize}
\begin{itemize}
\item {Grp. gram.:M. pl.}
\end{itemize}
\begin{itemize}
\item {Proveniência:(Do gr. \textunderscore brakhion\textunderscore  + \textunderscore pous\textunderscore , \textunderscore podos\textunderscore )}
\end{itemize}
Cujos braços servem de pés.
Classe de molluscos, cujos pés são representados por dois braços, que servem para a respiração e para a locomoção.
\section{Braquióptero}
\begin{itemize}
\item {Grp. gram.:m.}
\end{itemize}
\begin{itemize}
\item {Proveniência:(Do gr. \textunderscore brackhion\textunderscore  + \textunderscore pteron\textunderscore )}
\end{itemize}
Peixe, que tem as barbatanas em fórma de asas.
\section{Braquióstomo}
\begin{itemize}
\item {Grp. gram.:m.}
\end{itemize}
\begin{itemize}
\item {Proveniência:(Do gr. \textunderscore brakhion\textunderscore  + \textunderscore stoma\textunderscore )}
\end{itemize}
Espécie de pólypo, que tem a bôca rodeada de membros aprehensores.
\section{Braquipinacoide}
\begin{itemize}
\item {Grp. gram.:m.}
\end{itemize}
\begin{itemize}
\item {Utilização:Geol.}
\end{itemize}
Prisma mineral, limitado por dois planos parallelos entre si e equidistantes do plano de symetria, que passa pelo eixo principal e pelo braquidiagonal.
\section{Braquipnéa}
\begin{itemize}
\item {Grp. gram.:f.}
\end{itemize}
\begin{itemize}
\item {Proveniência:(Do gr. \textunderscore brakhus\textunderscore  + \textunderscore pnein\textunderscore )}
\end{itemize}
Respiração curta, difficil.
\section{Braquipneia}
\begin{itemize}
\item {Grp. gram.:f.}
\end{itemize}
\begin{itemize}
\item {Proveniência:(Do gr. \textunderscore brakhus\textunderscore  + \textunderscore pnein\textunderscore )}
\end{itemize}
Respiração curta, difficil.
\section{Braquípodes}
\begin{itemize}
\item {Grp. gram.:m. pl.}
\end{itemize}
\begin{itemize}
\item {Proveniência:(Do gr. \textunderscore brakhus\textunderscore  + \textunderscore pous\textunderscore , \textunderscore podos\textunderscore )}
\end{itemize}
Família de aves, com pés curtos.
\section{Braquípteros}
\begin{itemize}
\item {Grp. gram.:m. pl.}
\end{itemize}
\begin{itemize}
\item {Proveniência:(Do gr. \textunderscore brakhus\textunderscore  + \textunderscore pteron\textunderscore )}
\end{itemize}
Aves palmípedes, de asas muito curtas.
\section{Braquíscio}
\begin{itemize}
\item {Grp. gram.:adj.}
\end{itemize}
\begin{itemize}
\item {Proveniência:(Do gr. \textunderscore brakhus\textunderscore  + \textunderscore skia\textunderscore )}
\end{itemize}
Diz-se dos indivíduos que, habitando a zona tórrida, projectam, expostos ao sol, uma sombra muito curta.
\section{Braquissílabo}
\begin{itemize}
\item {Grp. gram.:m.}
\end{itemize}
\begin{itemize}
\item {Proveniência:(Do gr. \textunderscore brakhus\textunderscore  + \textunderscore sullabe\textunderscore )}
\end{itemize}
Pé de verso grego ou latino, composto de três sýllabas breves.
\section{Braquistégia}
\begin{itemize}
\item {Grp. gram.:f.}
\end{itemize}
Gênero de plantas africanas.
\section{Braquistocéfalo}
\begin{itemize}
\item {Grp. gram.:adj.}
\end{itemize}
\begin{itemize}
\item {Proveniência:(Do gr. \textunderscore brakhistos\textunderscore  + \textunderscore kephale\textunderscore )}
\end{itemize}
Que tem a cabeça muito curta.
\section{Braquiúro}
\begin{itemize}
\item {Grp. gram.:adj.}
\end{itemize}
\begin{itemize}
\item {Proveniência:(Do gr. \textunderscore brakhus\textunderscore  + \textunderscore oura\textunderscore )}
\end{itemize}
Que tem cauda curta.
\section{Brasa}
\begin{itemize}
\item {Grp. gram.:f.}
\end{itemize}
\begin{itemize}
\item {Utilização:Fam.}
\end{itemize}
\begin{itemize}
\item {Utilização:Bras. de Minas}
\end{itemize}
Carvão encandescente.
Qualidade de aquillo que está encandescente.
Ardor.
Inflammação.
Ansiedade, ira: \textunderscore estou em brasa\textunderscore !
Extremidade accesa do morrão de artilharia.
Pessôa, que está ardendo em febre.
\textunderscore Bater a brasa\textunderscore , disparar arma de fogo.
(Do baixo al. \textunderscore bras\textunderscore )
\section{Brasalisco}
\begin{itemize}
\item {Grp. gram.:m.}
\end{itemize}
\begin{itemize}
\item {Utilização:Prov.}
\end{itemize}
\begin{itemize}
\item {Proveniência:(De \textunderscore brasa\textunderscore ? ou alter. de \textunderscore basilisco\textunderscore ?)}
\end{itemize}
Rapaz inquieto, turbulento.
\section{Brasão}
\begin{itemize}
\item {Grp. gram.:m.}
\end{itemize}
\begin{itemize}
\item {Proveniência:(Do alt. al. médio \textunderscore blas\textunderscore )}
\end{itemize}
Escudo de armas.
Insígnia de pessôas ou famílias nobres.
Honra, glória: \textunderscore a probidade é o seu brasão\textunderscore .
\section{Braseal}
\begin{itemize}
\item {Grp. gram.:m.}
\end{itemize}
\begin{itemize}
\item {Utilização:Prov.}
\end{itemize}
\begin{itemize}
\item {Utilização:beir.}
\end{itemize}
O mesmo que \textunderscore brasido\textunderscore .
\section{Braseira}
\begin{itemize}
\item {Grp. gram.:f.}
\end{itemize}
O mesmo que \textunderscore braseiro\textunderscore .
\section{Braseiro}
\begin{itemize}
\item {Grp. gram.:m.}
\end{itemize}
\begin{itemize}
\item {Proveniência:(De \textunderscore brasa\textunderscore )}
\end{itemize}
Vaso de metal ou de loiça, para brasas.
Fogareiro.
Brasido.
Conjuncto de brasas ou de objectos incendiados: \textunderscore o palácio ficou reduzido a um braseiro\textunderscore .
\section{Brasfemar}
\begin{itemize}
\item {Grp. gram.:v. i.}
\end{itemize}
\begin{itemize}
\item {Utilização:Ant.}
\end{itemize}
O mesmo que \textunderscore blasfemar\textunderscore .
\section{Brasfêmia}
\begin{itemize}
\item {Grp. gram.:f.}
\end{itemize}
\begin{itemize}
\item {Utilização:Ant.}
\end{itemize}
O mesmo que \textunderscore blasfêmia\textunderscore . Cf. \textunderscore Aulegrafia\textunderscore , 158.
\section{Brasido}
\begin{itemize}
\item {Grp. gram.:m.}
\end{itemize}
Porção de brasas.
\section{Brasil}
\begin{itemize}
\item {Grp. gram.:m.}
\end{itemize}
\begin{itemize}
\item {Utilização:Ant.}
\end{itemize}
\begin{itemize}
\item {Grp. gram.:Adj.}
\end{itemize}
\begin{itemize}
\item {Grp. gram.:M. pl.}
\end{itemize}
Planta leguminosa, de que se tira o pau-brasil.
Côr encarnada, com que as senhoras se enfeitavam.
Diz-se de um pau vermelho empregado em tinturaria.
As terras do Brasil; os Brasileiros:«\textunderscore vy com fome da côr dos Brasis vossa pele\textunderscore ». Usque, \textunderscore Tribulações\textunderscore , 50, V.^o
\section{Brasileiramente}
\begin{itemize}
\item {Grp. gram.:adv.}
\end{itemize}
Á maneira dos Brasileiros.
\section{Brasileirismo}
\begin{itemize}
\item {Grp. gram.:m.}
\end{itemize}
Expressão própria de Brasileiros.
\section{Brasileira}
\begin{itemize}
\item {Grp. gram.:f.}
\end{itemize}
\begin{itemize}
\item {Utilização:Bras}
\end{itemize}
Planta ornamental de fôlhas verdes, matizadas de branco.
\section{Brasileirada}
\begin{itemize}
\item {Grp. gram.:f.}
\end{itemize}
\begin{itemize}
\item {Utilização:Deprec.}
\end{itemize}
Porção de Brasileiros.
Os Brasileiros.
\section{Brasileirice}
\begin{itemize}
\item {Grp. gram.:f.}
\end{itemize}
Expressão abrasileirada.
Languidez, denguice:«\textunderscore brasileirices inflammatorias\textunderscore ». Camillo, \textunderscore Corja\textunderscore , 171.
\section{Brasileiro}
\begin{itemize}
\item {Grp. gram.:adj.}
\end{itemize}
\begin{itemize}
\item {Grp. gram.:M.}
\end{itemize}
\begin{itemize}
\item {Utilização:Pop.}
\end{itemize}
\begin{itemize}
\item {Proveniência:(De \textunderscore Brasil\textunderscore , n. p.)}
\end{itemize}
Relativo ao Brasil: \textunderscore o povo brasileiro\textunderscore .
Aquelle que é natural do Brasil.
Português, que residiu no Brasil e que regressou, trazendo mais ou menos haveres; homem ricaço.
\section{Brasilense}
\begin{itemize}
\item {Grp. gram.:adj.}
\end{itemize}
O mesmo ou melhor que \textunderscore brasiliense\textunderscore .
\section{Brasilete}
\begin{itemize}
\item {fónica:lê}
\end{itemize}
\begin{itemize}
\item {Grp. gram.:m.}
\end{itemize}
\begin{itemize}
\item {Proveniência:(De \textunderscore brasil\textunderscore )}
\end{itemize}
Espécie de pau-brasil.
Planta terebinthácea, que dá madeira encarnada.
\section{Brasileto}
\begin{itemize}
\item {fónica:lê}
\end{itemize}
\begin{itemize}
\item {Grp. gram.:m.}
\end{itemize}
\begin{itemize}
\item {Proveniência:(De \textunderscore brasil\textunderscore )}
\end{itemize}
Espécie de pau-brasil.
Planta terebinthácea, que dá madeira encarnada.
\section{Brasiliano}
\begin{itemize}
\item {Grp. gram.:m.  e  adj.}
\end{itemize}
O mesmo que \textunderscore brasileiro\textunderscore :«\textunderscore certa dama, gentil brasiliana\textunderscore ». \textunderscore Caramuru\textunderscore , II, 77.
\section{Brasílico}
\begin{itemize}
\item {Grp. gram.:adj.}
\end{itemize}
O mesmo que \textunderscore brasiliense\textunderscore .
\section{Brasiliense}
\begin{itemize}
\item {Grp. gram.:adj.}
\end{itemize}
Relativo ao \textunderscore Brasil\textunderscore , n. p.
\section{Brasilina}
\begin{itemize}
\item {Grp. gram.:f.}
\end{itemize}
Substância còrante do pau-brasil.
\section{Brasílio}
\begin{itemize}
\item {Grp. gram.:adj.}
\end{itemize}
Relativo ao Brasil: \textunderscore o povo brasílio\textunderscore .
\section{Brasilita}
\begin{itemize}
\item {Grp. gram.:f.}
\end{itemize}
Espécie de pòlvora, que recentemente se experimentou no Brasil.
\section{Brasino}
\begin{itemize}
\item {Grp. gram.:adj.}
\end{itemize}
\begin{itemize}
\item {Utilização:Bras}
\end{itemize}
\begin{itemize}
\item {Grp. gram.:M.}
\end{itemize}
Que tem côr de brasa.
Vermelho com listas pretas.
Peixe da ria de Aveiro, (\textunderscore anguila Bibroni\textunderscore , Kant.).
\section{Brasio}
\begin{itemize}
\item {Grp. gram.:m.}
\end{itemize}
(V.brasido)
\section{Brasonar}
\begin{itemize}
\item {Grp. gram.:v. t.}
\end{itemize}
\begin{itemize}
\item {Grp. gram.:V. i.}
\end{itemize}
\begin{itemize}
\item {Proveniência:(De \textunderscore brasão\textunderscore )}
\end{itemize}
Ornar com brasão.
O mesmo que \textunderscore blasonar\textunderscore , alardear.
\section{Brassadura}
\begin{itemize}
\item {Grp. gram.:f.}
\end{itemize}
O mesmo que \textunderscore brassagem\textunderscore .
\section{Brassagem}
\begin{itemize}
\item {Grp. gram.:f.}
\end{itemize}
\begin{itemize}
\item {Utilização:Neol.}
\end{itemize}
Acto de fazer as misturas precisas para a fabricação da cerveja.
(Cp. fr. \textunderscore brasserie\textunderscore )
\section{Brassávola}
\begin{itemize}
\item {Grp. gram.:f.}
\end{itemize}
\begin{itemize}
\item {Proveniência:(De \textunderscore Brassavolo\textunderscore , n. p.)}
\end{itemize}
Gênero de orchídeas.
\section{Brássica}
\begin{itemize}
\item {Grp. gram.:f.}
\end{itemize}
\begin{itemize}
\item {Proveniência:(Lat. \textunderscore brassica\textunderscore )}
\end{itemize}
Designação scientífica da couve.
\section{Brassicáceas}
\begin{itemize}
\item {Grp. gram.:f. pl.}
\end{itemize}
\begin{itemize}
\item {Proveniência:(De \textunderscore brássica\textunderscore )}
\end{itemize}
Família de plantas, que tem por typo a couve.
\section{Brasume}
\begin{itemize}
\item {Grp. gram.:m.}
\end{itemize}
\begin{itemize}
\item {Proveniência:(De \textunderscore brasa\textunderscore )}
\end{itemize}
Grande ardor:«\textunderscore em brasumes de ternura lúbrica\textunderscore ». Camillo, \textunderscore Annos de prosa\textunderscore , 113.
\section{Braúna}
\begin{itemize}
\item {Grp. gram.:f.}
\end{itemize}
Árvore leguminosa do Brasil.
\section{Braunita}
\begin{itemize}
\item {fónica:bra-u}
\end{itemize}
\begin{itemize}
\item {Grp. gram.:f.}
\end{itemize}
Um dos ácidos do manganés.
\section{Bravalho}
\begin{itemize}
\item {Grp. gram.:adj.}
\end{itemize}
\begin{itemize}
\item {Utilização:Des.}
\end{itemize}
Muito bravo.
\section{Bravamente}
\begin{itemize}
\item {Grp. gram.:adv.}
\end{itemize}
\begin{itemize}
\item {Proveniência:(De \textunderscore bravo\textunderscore )}
\end{itemize}
Com bravura, com furor.
\section{Bravaria}
\begin{itemize}
\item {Grp. gram.:f.}
\end{itemize}
\begin{itemize}
\item {Utilização:Ant.}
\end{itemize}
O mesmo que \textunderscore bravata\textunderscore .
\section{Bravata}
\begin{itemize}
\item {Grp. gram.:f.}
\end{itemize}
\begin{itemize}
\item {Proveniência:(It. \textunderscore bravata\textunderscore , do rad. de \textunderscore bravo\textunderscore )}
\end{itemize}
Ameaça arrogante; fanfarrice.
Vanglória.
\section{Bravatão}
\begin{itemize}
\item {Grp. gram.:m.}
\end{itemize}
O mesmo que \textunderscore bravateador\textunderscore . Cf. B. Pato, \textunderscore Ciprestes\textunderscore , 248.
\section{Bravateador}
\begin{itemize}
\item {Grp. gram.:m.}
\end{itemize}
Aquelle que bravateia.
\section{Bravatear}
\begin{itemize}
\item {Grp. gram.:v. i.}
\end{itemize}
\begin{itemize}
\item {Proveniência:(De \textunderscore bravata\textunderscore )}
\end{itemize}
Dirigir ameaças.
Fazer-se arrogante, jactancioso.
\section{Bravateiro}
\begin{itemize}
\item {Grp. gram.:m.}
\end{itemize}
O mesmo que \textunderscore bravateador\textunderscore .
\section{Bravear}
\begin{itemize}
\item {Grp. gram.:v. i.}
\end{itemize}
O mesmo que \textunderscore bravejar\textunderscore .
\section{Braveira}
\begin{itemize}
\item {Grp. gram.:f.}
\end{itemize}
\begin{itemize}
\item {Proveniência:(De \textunderscore bravo\textunderscore )}
\end{itemize}
Perrice ou rabugice teimosa de criança.
Acto de chorar, esbracejando.
\section{Bravejar}
\begin{itemize}
\item {Grp. gram.:v. i.}
\end{itemize}
O mesmo que \textunderscore esbravejar\textunderscore ^1.
\section{Braveza}
\begin{itemize}
\item {Grp. gram.:f.}
\end{itemize}
\begin{itemize}
\item {Proveniência:(De \textunderscore bravo\textunderscore )}
\end{itemize}
Bravura.
Ferocidade.
Impetuosidade.
\section{Bravia}
\begin{itemize}
\item {Grp. gram.:f.  e  adj.}
\end{itemize}
\begin{itemize}
\item {Proveniência:(De \textunderscore bravio\textunderscore )}
\end{itemize}
Variedade de pêra, também conhecida por \textunderscore santiago\textunderscore .
\section{Bravio}
\begin{itemize}
\item {Grp. gram.:adj.}
\end{itemize}
\begin{itemize}
\item {Grp. gram.:M.}
\end{itemize}
\begin{itemize}
\item {Proveniência:(De \textunderscore bravo\textunderscore )}
\end{itemize}
Bravo.
Selvagem; agreste.
Rude.
Áspero, diffícil de transitar: \textunderscore terrenos bravios\textunderscore .
Terreno inculto, coberto apenas de rasteira vegetação, espontânea.
\section{Bravisco}
\begin{itemize}
\item {Grp. gram.:adj.}
\end{itemize}
\begin{itemize}
\item {Utilização:Prov.}
\end{itemize}
\begin{itemize}
\item {Utilização:trasm.}
\end{itemize}
Um tanto bravo.
Arisco.
\section{Bravito}
\begin{itemize}
\item {Grp. gram.:adj.}
\end{itemize}
\begin{itemize}
\item {Proveniência:(De \textunderscore bravo\textunderscore )}
\end{itemize}
Diz-se do toiro um tanto medroso.
\section{Bravo}
\begin{itemize}
\item {Grp. gram.:m.}
\end{itemize}
\begin{itemize}
\item {Grp. gram.:Adj.}
\end{itemize}
\begin{itemize}
\item {Grp. gram.:Interj.}
\end{itemize}
Homem valente, corajoso.
Applauso, approvação: \textunderscore soaram bravos\textunderscore .
Salteador italiano.
Bravio, feroz.
Tempestuoso.
Valoroso, intrépido: \textunderscore bravo militar\textunderscore .
Admirável.
Exaltado, furioso.
Impetuoso.
Inculto.
(para applaudir ou approvar)
(B. lat. \textunderscore bravus\textunderscore , do lat. \textunderscore barbarus\textunderscore )
\section{Brávoa}
\begin{itemize}
\item {Grp. gram.:f.}
\end{itemize}
Genero de plantas amaryllídeas.
\section{Bravo-de-esmolfo}
\begin{itemize}
\item {Grp. gram.:m.}
\end{itemize}
Variedade de maçan muito apreciada.
\section{Bravo-de-mundão}
\begin{itemize}
\item {Grp. gram.:m.}
\end{itemize}
Variedade de pêra da Beira-Alta, inverniça, sumarenta, e com granulações exteriores.
\section{Bravos}
\begin{itemize}
\item {Grp. gram.:interj.}
\end{itemize}
\begin{itemize}
\item {Utilização:Bras}
\end{itemize}
O mesmo que \textunderscore bravo\textunderscore 
\section{Bravosidade}
\begin{itemize}
\item {Grp. gram.:f.}
\end{itemize}
\begin{itemize}
\item {Utilização:Des.}
\end{itemize}
(V.bravura)
\section{Bravoso}
\begin{itemize}
\item {Grp. gram.:adj.}
\end{itemize}
\begin{itemize}
\item {Utilização:Des.}
\end{itemize}
(V.bravo)
\section{Bravura}
\begin{itemize}
\item {Grp. gram.:f.}
\end{itemize}
Qualidade do que é bravo.
Coragem.
Emprêgo de todos os recursos, no cantar: \textunderscore cantou com bravura\textunderscore .
\section{Breadura}
\begin{itemize}
\item {Grp. gram.:f.}
\end{itemize}
Acção ou effeito de \textunderscore brear\textunderscore .
Acto de brear.
\section{Breagem}
\begin{itemize}
\item {Grp. gram.:f.}
\end{itemize}
O mesmo que \textunderscore breadura\textunderscore .
\section{Breal}
\begin{itemize}
\item {Grp. gram.:m.  e  adj.}
\end{itemize}
\begin{itemize}
\item {Proveniência:(De \textunderscore breu\textunderscore )}
\end{itemize}
Casta de uva preta algarvia.
\section{Breamante}
\begin{itemize}
\item {Grp. gram.:m.}
\end{itemize}
Peixe de Portugal.
\section{Brear}
\begin{itemize}
\item {Grp. gram.:v. t.}
\end{itemize}
O mesmo que \textunderscore embrear\textunderscore .
\section{Breba}
\begin{itemize}
\item {Grp. gram.:f.}
\end{itemize}
\begin{itemize}
\item {Utilização:Prov.}
\end{itemize}
\begin{itemize}
\item {Utilização:trasm.}
\end{itemize}
\begin{itemize}
\item {Utilização:Chul.}
\end{itemize}
O mesmo que \textunderscore bêbera\textunderscore .
A vagina da mulher.
\section{Breca}
\begin{itemize}
\item {Grp. gram.:f.}
\end{itemize}
\begin{itemize}
\item {Utilização:Ant.}
\end{itemize}
Contracção espasmódica e dolorosa do tecido muscular; câimbra.
Fúria; ira.
Maldade, traquinice.
\textunderscore Levado da breca\textunderscore , endiabrado, travesso.
\textunderscore Faz coisas da breca\textunderscore , faz coisas diabólicas.
\textunderscore Foi-se com a breca\textunderscore , foi-se sem querer saber de mais nada.
\textunderscore Vai-te com a breca!\textunderscore  sume-te, coisa má!
\section{Breca}
\begin{itemize}
\item {Grp. gram.:f.}
\end{itemize}
Peixe das costas de Portugal.
\section{Breca-bica}
\begin{itemize}
\item {Grp. gram.:f.}
\end{itemize}
\begin{itemize}
\item {Utilização:Prov.}
\end{itemize}
\begin{itemize}
\item {Utilização:alg.}
\end{itemize}
Peixe, semelhante ao besugo.
\section{Brecha}
\begin{itemize}
\item {Grp. gram.:f.}
\end{itemize}
\begin{itemize}
\item {Proveniência:(Do fr. \textunderscore brèche\textunderscore )}
\end{itemize}
Abertura em qualquer vedação, em muralha, em sebe, etc.
Lacuna.
Prejuizo: \textunderscore aquelle acto fez-lhe brecha na reputação\textunderscore .
Ferimento largo e profundo.
Quebrada, entre montanhas.
\textunderscore Estar na brecha\textunderscore , lutar encarniçadamente.
\section{Brecha}
\begin{itemize}
\item {Grp. gram.:f.}
\end{itemize}
Corpo mineral ou màrnore, formado pela aggregação de elementos variegados.
\section{Brechão}
\begin{itemize}
\item {Grp. gram.:m.}
\end{itemize}
\begin{itemize}
\item {Utilização:Bras}
\end{itemize}
\begin{itemize}
\item {Utilização:Prov.}
\end{itemize}
Grande brecha; rasgão.
Represa em trabalhos de vallagem.
\section{Brechar}
\begin{itemize}
\item {Grp. gram.:v. i.}
\end{itemize}
\begin{itemize}
\item {Utilização:Gír.}
\end{itemize}
Pagar a patente.
\section{Brechil}
\begin{itemize}
\item {Grp. gram.:m.}
\end{itemize}
Espécie de lança arábica.
\section{Bredo}
\begin{itemize}
\item {fónica:brê}
\end{itemize}
\begin{itemize}
\item {Grp. gram.:m.}
\end{itemize}
\begin{itemize}
\item {Utilização:Bras. do N}
\end{itemize}
\begin{itemize}
\item {Proveniência:(Do gr. \textunderscore bliton\textunderscore )}
\end{itemize}
Planta hortense, annual.
Qualquer planta hortense, de que se faz esparregado.
Mato, brenha.
\textunderscore Ganhar o bredo\textunderscore , fugir para o mato.
\section{Brefo}
\begin{itemize}
\item {Grp. gram.:m.}
\end{itemize}
\begin{itemize}
\item {Proveniência:(Gr. \textunderscore brephos\textunderscore )}
\end{itemize}
Gênero de insectos lepidópteros nocturnos.
\section{Brefotrófio}
\begin{itemize}
\item {Grp. gram.:m.}
\end{itemize}
O mesmo que \textunderscore brefótrofo\textunderscore .
\section{Bregado}
\begin{itemize}
\item {Grp. gram.:adj.}
\end{itemize}
\begin{itemize}
\item {Utilização:Ant.}
\end{itemize}
Dizia-se do pão, quando se torna duro.
\section{Bregma}
\begin{itemize}
\item {Grp. gram.:f.}
\end{itemize}
\begin{itemize}
\item {Utilização:Anat.}
\end{itemize}
\begin{itemize}
\item {Proveniência:(Gr. \textunderscore bregma\textunderscore )}
\end{itemize}
A fontanella anterior, conhecida vulgarmente por \textunderscore molleirinha\textunderscore .
\section{Bregmal}
\begin{itemize}
\item {Grp. gram.:adj.}
\end{itemize}
O mesmo que \textunderscore brégmico\textunderscore .
\section{Bregmático}
\begin{itemize}
\item {Grp. gram.:adj.}
\end{itemize}
O mesmo que \textunderscore brégmico\textunderscore .
\section{Brégmico}
\begin{itemize}
\item {Grp. gram.:adj.}
\end{itemize}
Relativo ao \textunderscore bregma\textunderscore .
\section{Breguegúe}
\begin{itemize}
\item {Grp. gram.:m.}
\end{itemize}
\begin{itemize}
\item {Utilização:Bras. do N}
\end{itemize}
Pássaro aquático, parecido com o gavião.
\section{Breia}
\begin{itemize}
\item {Grp. gram.:f.}
\end{itemize}
\begin{itemize}
\item {Utilização:Prov.}
\end{itemize}
\begin{itemize}
\item {Utilização:alent.}
\end{itemize}
\begin{itemize}
\item {Utilização:Prov.}
\end{itemize}
Pedaço da manta de toicinho.
O mesmo que \textunderscore vreia\textunderscore , chan, planalto.
\section{Breja}
\begin{itemize}
\item {Grp. gram.:f.}
\end{itemize}
\begin{itemize}
\item {Utilização:Prov.}
\end{itemize}
\begin{itemize}
\item {Utilização:extrem.}
\end{itemize}
O mesmo que \textunderscore brejo\textunderscore .
\section{Brejal}
\begin{itemize}
\item {Grp. gram.:m.}
\end{itemize}
Reunião de brejos.
Ave canora do Brasil, tambem chamada \textunderscore colleira\textunderscore .
\section{Brejeira}
\begin{itemize}
\item {Grp. gram.:f.}
\end{itemize}
\begin{itemize}
\item {Utilização:Bras. do N}
\end{itemize}
Porção de tabaco que se mastiga.
\section{Brejeirada}
\begin{itemize}
\item {fónica:bré}
\end{itemize}
\begin{itemize}
\item {Grp. gram.:f.}
\end{itemize}
Palavras ou acção de brejeiro.
Reunião de brejeiros.
\section{Brejeiral}
\begin{itemize}
\item {fónica:bré}
\end{itemize}
\begin{itemize}
\item {Grp. gram.:adj.}
\end{itemize}
Próprio de brejeiro.
\section{Brejeirar}
\begin{itemize}
\item {fónica:bré}
\end{itemize}
\begin{itemize}
\item {Grp. gram.:v. i.}
\end{itemize}
Vadiar, garotar, fazer acções de brejeiro.
\section{Brejeirice}
\begin{itemize}
\item {fónica:bré}
\end{itemize}
\begin{itemize}
\item {Grp. gram.:f.}
\end{itemize}
Palavras ou acção de brejeiro.
\section{Brejeiro}
\begin{itemize}
\item {fónica:bré}
\end{itemize}
\begin{itemize}
\item {Grp. gram.:m.}
\end{itemize}
\begin{itemize}
\item {Grp. gram.:Adj.}
\end{itemize}
\begin{itemize}
\item {Utilização:Ant.}
\end{itemize}
\begin{itemize}
\item {Proveniência:(De \textunderscore brejo\textunderscore )}
\end{itemize}
Vadio.
Patife; pandilha.
Grosseiro, reles.
Malicioso.
Tunante.
O mesmo que \textunderscore brejo\textunderscore .
\section{Brejento}
\begin{itemize}
\item {Grp. gram.:Adj.}
\end{itemize}
Diz-se do lugar ou terreno em que há brejos. Cf. \textunderscore Bibl. da Gente do Campo\textunderscore , 229 e 304.
\section{Brejina}
\begin{itemize}
\item {Grp. gram.:f.}
\end{itemize}
\begin{itemize}
\item {Utilização:Gír.}
\end{itemize}
Cereja.
\section{Brejo}
\begin{itemize}
\item {fónica:bréoubrei}
\end{itemize}
\begin{itemize}
\item {Grp. gram.:m.}
\end{itemize}
\begin{itemize}
\item {Utilização:Fig.}
\end{itemize}
Pantano; paul.
Matagal; terra, que só produz urzes.
Urze.
Lugar frio.
(Talvez do lat. hýp. \textunderscore bragius\textunderscore , do gr. \textunderscore bragos\textunderscore , paul)
\section{Brejoso}
\begin{itemize}
\item {Grp. gram.:adj.}
\end{itemize}
Que tem brejos.
Semelhante ao brejo.
\section{Brelho}
\begin{itemize}
\item {fónica:brê}
\end{itemize}
\begin{itemize}
\item {Grp. gram.:m.}
\end{itemize}
\begin{itemize}
\item {Utilização:Prov.}
\end{itemize}
\begin{itemize}
\item {Utilização:minh.}
\end{itemize}
\begin{itemize}
\item {Proveniência:(De \textunderscore imbriculum\textunderscore , dim. do lat. \textunderscore imbrex\textunderscore )}
\end{itemize}
Fragmento de tijolo.
Cada um dos tijolos do lar do forno.
\section{Brema}
\begin{itemize}
\item {Grp. gram.:f.}
\end{itemize}
Peixe cyprinoide de água doce.
(Cast. \textunderscore brema\textunderscore , do germ.)
\section{Brendo}
\begin{itemize}
\item {Grp. gram.:m.}
\end{itemize}
\begin{itemize}
\item {Utilização:Prov.}
\end{itemize}
\begin{itemize}
\item {Utilização:beir.}
\end{itemize}
Espécie de garfo de quatro a seis dentes, fabricado de madeira.
\section{Brenha}
\begin{itemize}
\item {Grp. gram.:f.}
\end{itemize}
Floresta espêssa; mata virgem.
Confusão.
Arcano.
(Cp. cast. \textunderscore breña\textunderscore )
\section{Brenhoso}
\begin{itemize}
\item {Grp. gram.:adj.}
\end{itemize}
Cheio de brenhas.
\section{Brenno}
\begin{itemize}
\item {Grp. gram.:m.}
\end{itemize}
Designação do chefe da grei, entre os Gállios.
\section{Breno}
\begin{itemize}
\item {Grp. gram.:m.}
\end{itemize}
Designação do chefe da grei, entre os Gállios.
\section{Brenseda}
\begin{itemize}
\item {fónica:sê}
\end{itemize}
\begin{itemize}
\item {Grp. gram.:f.}
\end{itemize}
\begin{itemize}
\item {Utilização:Des.}
\end{itemize}
Conjuncto de brenhas, brejos e matagaes.
Escuridão. Cf. Viterbo, \textunderscore Elucidário\textunderscore .
\section{Brepho}
\begin{itemize}
\item {Grp. gram.:m.}
\end{itemize}
\begin{itemize}
\item {Proveniência:(Gr. \textunderscore brephos\textunderscore )}
\end{itemize}
Gênero de insectos lepidópteros nocturnos.
\section{Brephotróphio}
\begin{itemize}
\item {Grp. gram.:m.}
\end{itemize}
O mesmo que \textunderscore brephótropho\textunderscore .
\section{Brebião}
\begin{itemize}
\item {Grp. gram.:m.}
\end{itemize}
\begin{itemize}
\item {Utilização:Prov.}
\end{itemize}
\begin{itemize}
\item {Utilização:trasm.}
\end{itemize}
Peça horizontal do carro, na qual se fixam os fueiros.
\section{Brefótrofo}
\begin{itemize}
\item {Grp. gram.:m.}
\end{itemize}
\begin{itemize}
\item {Proveniência:(Do gr. \textunderscore brephos\textunderscore  + \textunderscore trophos\textunderscore )}
\end{itemize}
Hospício de enjeitados, na Idade-Média.
\section{Brephótropho}
\begin{itemize}
\item {Grp. gram.:m.}
\end{itemize}
\begin{itemize}
\item {Proveniência:(Do gr. \textunderscore brephos\textunderscore  + \textunderscore trophos\textunderscore )}
\end{itemize}
Hospício de enjeitados, na Idade-Média.
\section{Breque}
\begin{itemize}
\item {Grp. gram.:m.}
\end{itemize}
\begin{itemize}
\item {Proveniência:(Do ingl. \textunderscore break\textunderscore )}
\end{itemize}
Carruagem de quatro rodas, com um assento alto adeante, e ordinariamente dois bancos atrás, longitudinaes e fronteiros um ao outro.
\section{Brequefeste}
\begin{itemize}
\item {Grp. gram.:m.}
\end{itemize}
\begin{itemize}
\item {Proveniência:(Do ingl. \textunderscore breakfast\textunderscore )}
\end{itemize}
Festa, em que se come muito; comezaina; pândega.
\section{Breta}
\begin{itemize}
\item {fónica:brê}
\end{itemize}
\begin{itemize}
\item {Grp. gram.:f.}
\end{itemize}
\begin{itemize}
\item {Utilização:Prov.}
\end{itemize}
\begin{itemize}
\item {Utilização:minh.}
\end{itemize}
Pequeno peixe amarelado de água salgada.
\section{Bretangil}
\begin{itemize}
\item {Grp. gram.:m.}
\end{itemize}
Certo tecido de algodão, fabricado na África oriental.
\section{Bretanha}
\begin{itemize}
\item {Grp. gram.:f.}
\end{itemize}
\begin{itemize}
\item {Proveniência:(De \textunderscore Bretanha\textunderscore , n. p.)}
\end{itemize}
Tecido fino de algodão ou linho.
\section{Bretão}
\begin{itemize}
\item {Grp. gram.:adj.}
\end{itemize}
\begin{itemize}
\item {Grp. gram.:M.}
\end{itemize}
Relativo á Bretanha.
Habitante da Bretanha.
Dialecto desta antiga provincia francesa, (\textunderscore Britannia Minor\textunderscore , \textunderscore Armorica\textunderscore ).
\section{Bretão}
\begin{itemize}
\item {Grp. gram.:adj.}
\end{itemize}
\begin{itemize}
\item {Grp. gram.:M.}
\end{itemize}
Relativo á Inglaterra ou Gran-Bretanha.
Inglês, habitante da Gran-Bretanha, (\textunderscore Britannia Major\textunderscore , \textunderscore Britannia Romana\textunderscore ).
\section{Brete}
\begin{itemize}
\item {Grp. gram.:m.}
\end{itemize}
Armadilha, para apanhar pássaros.
Lôgro, cilada.
(Cast. \textunderscore brete\textunderscore )
\section{Brete}
\begin{itemize}
\item {Grp. gram.:m.}
\end{itemize}
\begin{itemize}
\item {Utilização:mil.}
\end{itemize}
\begin{itemize}
\item {Utilização:Gír.}
\end{itemize}
Pão negro de munição.
\section{Brête}
\begin{itemize}
\item {Grp. gram.:m.}
\end{itemize}
\begin{itemize}
\item {Utilização:Bras. do Sul}
\end{itemize}
O mesmo que \textunderscore curral\textunderscore .
\section{Breu}
\begin{itemize}
\item {Grp. gram.:m.}
\end{itemize}
\begin{itemize}
\item {Proveniência:(Fr. \textunderscore brai\textunderscore )}
\end{itemize}
Pez negro.
Substância análoga ao pez negro, composta de pez, sebo, etc.
\textunderscore Escuro como breu\textunderscore , muito escuro.
\section{Breu}
\begin{itemize}
\item {Grp. gram.:m.}
\end{itemize}
\begin{itemize}
\item {Utilização:Bras. do Rio}
\end{itemize}
Espécie de bote, que atraca aos vapores mercantes para vender fruta.
Tripulante dêsse bote.
\section{Breva}
\begin{itemize}
\item {fónica:brê}
\end{itemize}
\begin{itemize}
\item {Grp. gram.:m.}
\end{itemize}
Variedade de charuto de bôa qualidade.
\section{Brevas}
\begin{itemize}
\item {fónica:brê}
\end{itemize}
\begin{itemize}
\item {Grp. gram.:m. pl.}
\end{itemize}
\begin{itemize}
\item {Utilização:Ant.}
\end{itemize}
Espécie de polainas altas.
\section{Breve}
\begin{itemize}
\item {Grp. gram.:adj.}
\end{itemize}
\begin{itemize}
\item {Grp. gram.:M.}
\end{itemize}
\begin{itemize}
\item {Grp. gram.:F.}
\end{itemize}
\begin{itemize}
\item {Utilização:Gram.}
\end{itemize}
\begin{itemize}
\item {Utilização:Mús.}
\end{itemize}
\begin{itemize}
\item {Grp. gram.:Adv.}
\end{itemize}
\begin{itemize}
\item {Proveniência:(Lat. \textunderscore brevis\textunderscore )}
\end{itemize}
Que dura pouco: \textunderscore existência breve\textunderscore .
Curto.
Pequeno, pouco extenso: \textunderscore breve espaço\textunderscore .
Ligeiro.
Lacónico, resumido: \textunderscore breve epístola\textunderscore .
Carta ou rescrito pontifício, que contém declaração ou resolução que não é de interesse geral da Igreja.
Sýllaba ou vogal, que se pronuncía rapidamente.
Nota musical, que vale duas semi-breves, ou oito semínimas, etc.
Brevemente, dentro em pouco tempo: \textunderscore breve te escreverei\textunderscore .
\section{Brevemente}
\begin{itemize}
\item {Grp. gram.:adv.}
\end{itemize}
\begin{itemize}
\item {Proveniência:(De \textunderscore breve\textunderscore )}
\end{itemize}
Com brevidade; rapidamente.
Daqui a pouco.
\section{Brévia}
\begin{itemize}
\item {Grp. gram.:f.}
\end{itemize}
\begin{itemize}
\item {Utilização:Des.}
\end{itemize}
\begin{itemize}
\item {Proveniência:(Lat. \textunderscore brevia\textunderscore , pl. de \textunderscore brevis\textunderscore )}
\end{itemize}
Descanso; férias.
Lugar ou campo, onde se vai descansar ou folgar:«\textunderscore é brévia de armas a lição das musas\textunderscore ». \textunderscore Viriato Trág.\textunderscore 
\section{Breviador}
\begin{itemize}
\item {Grp. gram.:m.}
\end{itemize}
\begin{itemize}
\item {Proveniência:(De \textunderscore breve\textunderscore )}
\end{itemize}
Funccionário que, na cúria romana, tem a seu cargo o expediente dos breves.
\section{Brevião}
\begin{itemize}
\item {Grp. gram.:m.}
\end{itemize}
\begin{itemize}
\item {Utilização:Prov.}
\end{itemize}
\begin{itemize}
\item {Utilização:trasm.}
\end{itemize}
Peça horizontal do carro, na qual se fixam os fueiros.
\section{Breviário}
\begin{itemize}
\item {Grp. gram.:m.}
\end{itemize}
\begin{itemize}
\item {Utilização:Fig.}
\end{itemize}
\begin{itemize}
\item {Proveniência:(Lat. \textunderscore breviarius\textunderscore )}
\end{itemize}
Livro das orações que os clerigos devem lêr todos os dias.
Livro predilecto.
Synopse, resumo.
\textunderscore Lêr pelo mesmo breviário\textunderscore , têr as mesmas ideias, a mesma inclinação. Cf. Castilho, \textunderscore Fausto\textunderscore , 226.
\section{Brevidade}
\begin{itemize}
\item {Grp. gram.:f.}
\end{itemize}
\begin{itemize}
\item {Proveniência:(Lat. \textunderscore brevitas\textunderscore )}
\end{itemize}
Qualidade do que é breve.
\section{Brevifloro}
\begin{itemize}
\item {Grp. gram.:adj.}
\end{itemize}
\begin{itemize}
\item {Proveniência:(Do lat. \textunderscore brevis\textunderscore  + \textunderscore flos\textunderscore )}
\end{itemize}
Que tem flôres curtas.
\section{Brevifoliado}
\begin{itemize}
\item {Grp. gram.:adj.}
\end{itemize}
\begin{itemize}
\item {Proveniência:(Do lat. \textunderscore brevis\textunderscore  + \textunderscore folium\textunderscore )}
\end{itemize}
Que tem fôlhas curtas.
\section{Brevípede}
\begin{itemize}
\item {Grp. gram.:adj.}
\end{itemize}
\begin{itemize}
\item {Proveniência:(Do lat. \textunderscore brevis\textunderscore  + \textunderscore pes\textunderscore , \textunderscore pedis\textunderscore )}
\end{itemize}
Que tem pés curtos.
\section{Brevipenado}
\begin{itemize}
\item {Grp. gram.:adj.}
\end{itemize}
O mesmo que \textunderscore brevipenne\textunderscore .
\section{Brevipene}
\begin{itemize}
\item {Grp. gram.:adj.}
\end{itemize}
\begin{itemize}
\item {Proveniência:(Do lat. \textunderscore brevis\textunderscore  + \textunderscore penna\textunderscore )}
\end{itemize}
Que tem asas curtas.
\section{Brevipennado}
\begin{itemize}
\item {Grp. gram.:adj.}
\end{itemize}
O mesmo que \textunderscore brevipenne\textunderscore .
\section{Brevipenne}
\begin{itemize}
\item {Grp. gram.:adj.}
\end{itemize}
\begin{itemize}
\item {Proveniência:(Do lat. \textunderscore brevis\textunderscore  + \textunderscore penna\textunderscore )}
\end{itemize}
Que tem asas curtas.
\section{Brevirostrado}
\begin{itemize}
\item {fónica:ros}
\end{itemize}
\begin{itemize}
\item {Grp. gram.:adj.}
\end{itemize}
\begin{itemize}
\item {Proveniência:(Do lat. \textunderscore brevis\textunderscore  + \textunderscore rostrum\textunderscore )}
\end{itemize}
Que tem bico curto.
\section{Brevista}
\begin{itemize}
\item {Grp. gram.:m.}
\end{itemize}
Aquelle que trata de breves pontifícios.
\section{Bréxia}
\begin{itemize}
\item {Grp. gram.:f.}
\end{itemize}
Gênero de plantas saxifrágeas.
\section{Breza}
\begin{itemize}
\item {Grp. gram.:f.  e  adj.}
\end{itemize}
\begin{itemize}
\item {Utilização:Prov.}
\end{itemize}
\begin{itemize}
\item {Utilização:beir.}
\end{itemize}
Diz-se de uma cesta larga e baixa, de vêrga miúda e asa redonda.
\section{Briada}
\begin{itemize}
\item {Grp. gram.:f.}
\end{itemize}
\begin{itemize}
\item {Utilização:Prov.}
\end{itemize}
\begin{itemize}
\item {Utilização:trasm.}
\end{itemize}
Caminhada.
(Cp. \textunderscore breia\textunderscore )
\section{Brial}
\begin{itemize}
\item {Grp. gram.:m.}
\end{itemize}
\begin{itemize}
\item {Utilização:Prov.}
\end{itemize}
\begin{itemize}
\item {Utilização:trasm.}
\end{itemize}
Espécie de camisola, que os cavalleiros armados vestiam sôbre as armas \textunderscore ou\textunderscore , quando desarmados, sôbre o fato interior.
Vestido feminino de pano precioso.
Qualquer peça de vestuário.
(Cast. \textunderscore brial\textunderscore )
\section{Briareu}
\begin{itemize}
\item {Grp. gram.:m.}
\end{itemize}
\begin{itemize}
\item {Proveniência:(De \textunderscore Briareu\textunderscore , n. p. myth.)}
\end{itemize}
Mollusco gasterópode, de corpo gelatinoso.
\section{Briba}
\begin{itemize}
\item {Grp. gram.:f.}
\end{itemize}
\begin{itemize}
\item {Utilização:Bras. do N}
\end{itemize}
O mesmo que \textunderscore osga\textunderscore ^1.
\section{Bribigão}
\begin{itemize}
\item {Grp. gram.:m.}
\end{itemize}
O mesmo que \textunderscore berbigão\textunderscore .
\section{Brica}
\begin{itemize}
\item {Grp. gram.:f.}
\end{itemize}
\begin{itemize}
\item {Utilização:Heráld.}
\end{itemize}
Pequeno espaço nos brasões, para distinguir a linhagem dos filhos segundos.
\section{Bricabraque}
\begin{itemize}
\item {Grp. gram.:m.}
\end{itemize}
\begin{itemize}
\item {Proveniência:(Do fr. \textunderscore bric-à-brac\textunderscore )}
\end{itemize}
Estabelecimento ou conjunto de diversos e antigos objectos de arte, mobílias, vestuário, etc.
Ferros-velhos.
\section{Briche}
\begin{itemize}
\item {Grp. gram.:m.}
\end{itemize}
Pano grosso de lan, empregado geralmente em roupa de homens.
\section{Brichote}
\begin{itemize}
\item {Grp. gram.:m.}
\end{itemize}
Nome que, por desprêzo, se dava aos estrangeiros:«\textunderscore como has de tu entender, brichote de farelo?\textunderscore »Camillo, \textunderscore Noites de Insóm.\textunderscore , I, 35. Cf. Corvo, \textunderscore Anno na Côrte\textunderscore , I, 91.
\section{Brida}
\begin{itemize}
\item {Grp. gram.:f.}
\end{itemize}
\begin{itemize}
\item {Proveniência:(Fr. \textunderscore bride\textunderscore , do germ.)}
\end{itemize}
Rédea.
\textunderscore A toda a brida\textunderscore , á rédea sôlta, á desfilada.
Parte da cabeça das aves, entre a base do bico e os olhos. Cf. P. Moraes, \textunderscore Zool. Elem.\textunderscore 
\section{Bridão}
\begin{itemize}
\item {Grp. gram.:m.}
\end{itemize}
\begin{itemize}
\item {Utilização:Bras}
\end{itemize}
Brida grande.
Cavalleiro, que montava com os estribos compridos, ao contrário da gineta.
Freio leve e singelo, preso á brida, em corridas de cavallos.
\section{Bridar}
\begin{itemize}
\item {Grp. gram.:v. t.}
\end{itemize}
Pôr a brida em.
Enfrear; refrear.
Embridar.
\section{Briga}
\begin{itemize}
\item {Grp. gram.:f.}
\end{itemize}
\begin{itemize}
\item {Utilização:T. da Bairrada}
\end{itemize}
Luta; combate.
Disputa.
Desavença.
Canudo atacado de pólvora, que os rapazes accendem e atiram, em occasião de festa.
(B. lat. \textunderscore briga\textunderscore , do got. \textunderscore brikan\textunderscore )
\section{Brigada}
\begin{itemize}
\item {Grp. gram.:f.}
\end{itemize}
Corpo militar, de dois ou mais regimentos.
Conjunto de duas ou três batarias de campanha.
(B. lat. \textunderscore brigata\textunderscore , de \textunderscore brigare\textunderscore )
\section{Brigadas}
\begin{itemize}
\item {Grp. gram.:m.}
\end{itemize}
\begin{itemize}
\item {Proveniência:(De \textunderscore brigada\textunderscore )}
\end{itemize}
Antigo official do exército, correspondente ao actual sargento ajudante.
\section{Brigadeiro}
\begin{itemize}
\item {Grp. gram.:m.}
\end{itemize}
Official, que commandava uma brigada.
\section{Brigador}
\begin{itemize}
\item {Grp. gram.:m.}
\end{itemize}
Aquelle que briga.
Espécie de ave gallinácea.
\section{Brigalhão}
\begin{itemize}
\item {Grp. gram.:m.}
\end{itemize}
\begin{itemize}
\item {Utilização:Bras}
\end{itemize}
Indivíduo brigão.
\section{Brigandina}
\begin{itemize}
\item {Grp. gram.:f.}
\end{itemize}
\begin{itemize}
\item {Utilização:Ant.}
\end{itemize}
Pequena coiraça de malha.
(B. lat. \textunderscore brigandina\textunderscore , de \textunderscore brigandi\textunderscore )
\section{Brigante}
\begin{itemize}
\item {Grp. gram.:adj.}
\end{itemize}
Que briga.
Sedicioso.
Revolucionário.
\section{Brigantes}
\begin{itemize}
\item {Grp. gram.:m. pl.}
\end{itemize}
\begin{itemize}
\item {Proveniência:(Lat. \textunderscore brigantes\textunderscore )}
\end{itemize}
Aguerridos habitantes do norte da Britânnia, conquistada por César.
\section{Brigantino}
\begin{itemize}
\item {Grp. gram.:m.  e  adj.}
\end{itemize}
\begin{itemize}
\item {Proveniência:(Do lat. \textunderscore Bragantia\textunderscore , n. p. )}
\end{itemize}
O mesmo que \textunderscore bragantino\textunderscore .
\section{Brigão}
\begin{itemize}
\item {Grp. gram.:adj.}
\end{itemize}
\begin{itemize}
\item {Proveniência:(De \textunderscore brigar\textunderscore )}
\end{itemize}
Rixoso; que promove brigas.
\section{Brigar}
\begin{itemize}
\item {Grp. gram.:v. i.}
\end{itemize}
Têr briga; lutar.
Dicutir.
Destoar: \textunderscore as duas opiniões brigam\textunderscore .
\section{Brigoso}
\begin{itemize}
\item {Grp. gram.:adj.}
\end{itemize}
(V.brigão)
\section{Brigue}
\begin{itemize}
\item {Grp. gram.:m.}
\end{itemize}
\begin{itemize}
\item {Proveniência:(Ingl. \textunderscore brig\textunderscore )}
\end{itemize}
Embarcação de dois mastros, o maior dos quaes se inclina para a popa.
\section{Briguento}
\begin{itemize}
\item {Grp. gram.:m.}
\end{itemize}
(V.brigão)
\section{Brilha}
\begin{itemize}
\item {Grp. gram.:f.}
\end{itemize}
(Corr. de \textunderscore virilha\textunderscore )
\section{Brilhador}
\begin{itemize}
\item {Grp. gram.:adj.}
\end{itemize}
O mesmo que \textunderscore brilhante\textunderscore .
\section{Brilhante}
\begin{itemize}
\item {Grp. gram.:adj.}
\end{itemize}
\begin{itemize}
\item {Grp. gram.:M.}
\end{itemize}
\begin{itemize}
\item {Proveniência:(De \textunderscore brilhar\textunderscore )}
\end{itemize}
Que brilha.
Diamante, que tem plana a parte superior e facetados os lados e a parte inferior.
\section{Brilhantemente}
\begin{itemize}
\item {Grp. gram.:adv.}
\end{itemize}
De modo \textunderscore brilhante\textunderscore .
Com brilho.
\section{Brilhantez}
\begin{itemize}
\item {Grp. gram.:f.}
\end{itemize}
Qualidade daquillo que é brilhante. Cf. Latino, \textunderscore Camões\textunderscore , 315.
\section{Brilhantina}
\begin{itemize}
\item {Grp. gram.:f.}
\end{itemize}
\begin{itemize}
\item {Proveniência:(De \textunderscore brilhante\textunderscore )}
\end{itemize}
Pó mineral, com que se dá brilho.
Cosmético, para tornar lustroso o cabello.
\section{Brilhantismo}
\begin{itemize}
\item {Grp. gram.:m.}
\end{itemize}
Qualidade do que é \textunderscore brilhante\textunderscore .
\section{Brilhantura}
\begin{itemize}
\item {Grp. gram.:f.}
\end{itemize}
\begin{itemize}
\item {Utilização:Fam.}
\end{itemize}
\begin{itemize}
\item {Proveniência:(De \textunderscore brilhante\textunderscore )}
\end{itemize}
Eloquência ostentosa.
Execução brilhante de um encargo.
\section{Brilhar}
\begin{itemize}
\item {Grp. gram.:v. i.}
\end{itemize}
\begin{itemize}
\item {Proveniência:(De \textunderscore brilho\textunderscore )}
\end{itemize}
Têr brilho.
Dar luz, reflecti-la.
Evidenciar-se; tornar-se notável, por esplendor ou elegância ou riqueza ou belleza ou qualidades moraes.
\section{Brilhatura}
\begin{itemize}
\item {Grp. gram.:f.}
\end{itemize}
\begin{itemize}
\item {Utilização:Fam.}
\end{itemize}
O mesmo que \textunderscore brilhantura\textunderscore .
\section{Brilho}
\begin{itemize}
\item {Grp. gram.:m.}
\end{itemize}
\begin{itemize}
\item {Proveniência:(Do lat. \textunderscore beryllus\textunderscore )}
\end{itemize}
Luz viva; esplendor.
Vivacidade (no estilo, nas côres, etc.).
Magnificência.
Gloria.
Luz reflectida.
\section{Brim}
\begin{itemize}
\item {Grp. gram.:m.}
\end{itemize}
Tecido de linho, forte.
(Cast. \textunderscore brin\textunderscore )
\section{Brimbáo}
\begin{itemize}
\item {Grp. gram.:m.}
\end{itemize}
\begin{itemize}
\item {Proveniência:(Fr. \textunderscore brimbale\textunderscore )}
\end{itemize}
Pequeno instrumento de ferro, com uma lingueta de aço entre dois ramos, e que se applica contra os dentes, fazendo-se vibrar a lingueta com os dedos.
\section{Brimbau}
\begin{itemize}
\item {Grp. gram.:m.}
\end{itemize}
\begin{itemize}
\item {Proveniência:(Fr. \textunderscore brimbale\textunderscore )}
\end{itemize}
Pequeno instrumento de ferro, com uma lingueta de aço entre dois ramos, e que se applica contra os dentes, fazendo-se vibrar a lingueta com os dedos.
\section{Brinca}
\begin{itemize}
\item {Grp. gram.:f.}
\end{itemize}
\begin{itemize}
\item {Utilização:T. da Bairrada}
\end{itemize}
O mesmo que \textunderscore brincadeira\textunderscore , especialmente de crianças.
\section{Brinça}
\begin{itemize}
\item {Grp. gram.:f.}
\end{itemize}
\begin{itemize}
\item {Utilização:Ant.}
\end{itemize}
Designação pop. do \textunderscore peucédano\textunderscore . Cf. \textunderscore Desengano da Med.\textunderscore , 197.
\section{Brincadeira}
\begin{itemize}
\item {Grp. gram.:f.}
\end{itemize}
\begin{itemize}
\item {Utilização:Prov.}
\end{itemize}
Acção de \textunderscore brincar\textunderscore .
Bailarico; casa, onde se dança.
\section{Brincado}
\begin{itemize}
\item {Grp. gram.:adj.}
\end{itemize}
\begin{itemize}
\item {Proveniência:(De \textunderscore brincar\textunderscore )}
\end{itemize}
Que tem ornatos caprichosos.
Arrendado.
\section{Brincador}
\begin{itemize}
\item {Grp. gram.:m.  e  adj.}
\end{itemize}
O que brinca, brincalhão.
\section{Brincalhão}
\begin{itemize}
\item {Grp. gram.:m.  e  adj.}
\end{itemize}
\begin{itemize}
\item {Proveniência:(De \textunderscore brincar\textunderscore )}
\end{itemize}
O que gosta de brincar; o que está sempre disposto para brincar, zombar, galhofar.
\section{Brincalhar}
\begin{itemize}
\item {Grp. gram.:v. i.}
\end{itemize}
O mesmo que \textunderscore brincar\textunderscore . Cf. Filinto, X, 13.
\section{Brincalhotice}
\begin{itemize}
\item {Grp. gram.:f.}
\end{itemize}
Acto de brincalhão.
Brincadeira.
\section{Brincão}
\begin{itemize}
\item {Grp. gram.:m.  e  adj.}
\end{itemize}
O mesmo que \textunderscore brincalhão\textunderscore .
\section{Brincar}
\begin{itemize}
\item {Grp. gram.:v. i.}
\end{itemize}
\begin{itemize}
\item {Utilização:Prov.}
\end{itemize}
\begin{itemize}
\item {Grp. gram.:V. t.}
\end{itemize}
\begin{itemize}
\item {Proveniência:(Do al. \textunderscore springan\textunderscore ? Cp. G. Viana, \textunderscore Apostilas\textunderscore )}
\end{itemize}
Divertir-se infantilmente: \textunderscore as crianças brincam\textunderscore .
Folgar.
Saltar alegremente.
Agitar-se em movimentos caprichosos.
Gracejar.
Zombar: \textunderscore estás brincando comigo\textunderscore ?
Proceder levianamente.
Dançar: \textunderscore fui ao bailarico, mas não brinquei\textunderscore .
Enfeitar caprichosamente; rendilhar.
\section{Brinca-tudo}
\begin{itemize}
\item {Grp. gram.:m.}
\end{itemize}
\begin{itemize}
\item {Utilização:Prov.}
\end{itemize}
\begin{itemize}
\item {Utilização:alent.}
\end{itemize}
Espécie de dança de roda.
\section{Brincazão}
\begin{itemize}
\item {Grp. gram.:m.  e  adj.}
\end{itemize}
\begin{itemize}
\item {Utilização:Prov.}
\end{itemize}
\begin{itemize}
\item {Utilização:minh.}
\end{itemize}
O mesmo que \textunderscore brincalhão\textunderscore .
\section{Brinco}
\begin{itemize}
\item {Grp. gram.:m.}
\end{itemize}
\begin{itemize}
\item {Proveniência:(Do lat. \textunderscore vinculum\textunderscore ?)}
\end{itemize}
Objecto de adôrno para as orelhas; arrecada.
\section{Brinco}
\begin{itemize}
\item {Grp. gram.:m.}
\end{itemize}
Acção de \textunderscore brincar\textunderscore .
Objecto, destinado a divertimento de crianças; brinquedo.
Gracejo; galhofa.
Coisa muito asseada, muito limpa: \textunderscore ficou um brinco\textunderscore .
Espécie de dança, na Índia portuguesa.
Galantaria, bugiganga:«\textunderscore o embaixador comprou muitas peças ricas e brincos da China\textunderscore ». \textunderscore Peregrinação\textunderscore , CLXVI.
\section{Brindão}
\begin{itemize}
\item {Grp. gram.:m.}
\end{itemize}
Fruto do brindoeiro. Cf. Orta, \textunderscore Col.\textunderscore  X.
\section{Brindar}
\begin{itemize}
\item {Grp. gram.:v. t.}
\end{itemize}
\begin{itemize}
\item {Grp. gram.:V. i.}
\end{itemize}
\begin{itemize}
\item {Proveniência:(De \textunderscore brinde\textunderscore )}
\end{itemize}
Offerecer um mimo a.
Dar um presente a; presentear: \textunderscore brindou-me com um livro\textunderscore .
Conceder.
Attribuir por favor a.
Beber á saúde de alguém.
\section{Brinde}
\begin{itemize}
\item {Grp. gram.:m.}
\end{itemize}
\begin{itemize}
\item {Proveniência:(Do al. \textunderscore bringen\textunderscore )}
\end{itemize}
Acto de beber á saúde de alguém.
Offerta, dádiva.
\section{Brindoeiro}
\begin{itemize}
\item {Grp. gram.:m.}
\end{itemize}
Árvore da Índia portuguesa, (\textunderscore garcinia indica\textunderscore , Chois.).
\section{Brínie}
\begin{itemize}
\item {Grp. gram.:m.}
\end{itemize}
\begin{itemize}
\item {Utilização:Ant.}
\end{itemize}
Carne cozida com arroz.
\section{Brinquedo}
\begin{itemize}
\item {fónica:quê}
\end{itemize}
\begin{itemize}
\item {Grp. gram.:m.}
\end{itemize}
\begin{itemize}
\item {Proveniência:(De \textunderscore brincar\textunderscore )}
\end{itemize}
Brinco.
Folguedo.
\section{Brinquete}
\begin{itemize}
\item {fónica:quê}
\end{itemize}
\begin{itemize}
\item {Grp. gram.:m.}
\end{itemize}
\begin{itemize}
\item {Utilização:Bras}
\end{itemize}
\begin{itemize}
\item {Proveniência:(De \textunderscore brinco\textunderscore ?)}
\end{itemize}
Peça da prensa, com que se espreme a massa da mandioca.
\section{Brinquinharia}
\begin{itemize}
\item {Grp. gram.:f.}
\end{itemize}
Officina, em que se fabricam objectos para brinquedos de crianças.
(Cp. \textunderscore brinquinheiro\textunderscore )
\section{Brinquinheiro}
\begin{itemize}
\item {Grp. gram.:m.}
\end{itemize}
\begin{itemize}
\item {Proveniência:(De \textunderscore brinquinho\textunderscore , dem. de \textunderscore brinco\textunderscore )}
\end{itemize}
Fabricante de objectos que servem para divertimento de crianças.
\section{Brinzão}
\begin{itemize}
\item {Grp. gram.:m.}
\end{itemize}
\begin{itemize}
\item {Utilização:Des.}
\end{itemize}
Espécie de lona.
\section{Brio}
\begin{itemize}
\item {Grp. gram.:m.}
\end{itemize}
Sentimento da dignidade própria; pundonor.
Coragem.
Generosidade.
Garbo.
(Do gallês \textunderscore bri\textunderscore , fôrça)
\section{Briol}
\begin{itemize}
\item {Grp. gram.:m.}
\end{itemize}
\begin{itemize}
\item {Utilização:Náut.}
\end{itemize}
\begin{itemize}
\item {Utilização:Gír.}
\end{itemize}
\begin{itemize}
\item {Grp. gram.:Adj.}
\end{itemize}
\begin{itemize}
\item {Utilização:Prov.}
\end{itemize}
\begin{itemize}
\item {Utilização:trasm.}
\end{itemize}
\begin{itemize}
\item {Proveniência:(T. cast.)}
\end{itemize}
Cabo de ferrar as velas.
Vinho ordinário.
Bêbedo.
\section{Briosamente}
\begin{itemize}
\item {Grp. gram.:adv.}
\end{itemize}
De modo \textunderscore brioso\textunderscore , com brio.
\section{Brioso}
\begin{itemize}
\item {Grp. gram.:adj.}
\end{itemize}
Que tem brio.
Pundonoroso.
Generoso.
Corajoso.
Orgulhoso.
Garboso, (falando-se do cavallo).
\section{Briquitar}
\begin{itemize}
\item {Grp. gram.:v. i.}
\end{itemize}
\begin{itemize}
\item {Utilização:Bras. de Minas}
\end{itemize}
Matar o tempo, entreter-se.
Brigar.
(Por \textunderscore brinquitar\textunderscore , de \textunderscore brincar\textunderscore ?)
\section{Brisa}
\begin{itemize}
\item {Grp. gram.:f.}
\end{itemize}
Vento brando e fresco.
Vento brando, á beira-mar.
Aragem.
(Cast. \textunderscore brisa\textunderscore )
\section{Brisaque}
\begin{itemize}
\item {Grp. gram.:m.}
\end{itemize}
Antiga moéda do Tirol. Cf. F. Manuel, \textunderscore Apólogos\textunderscore .
\section{Brísio}
\begin{itemize}
\item {Grp. gram.:adj.}
\end{itemize}
\begin{itemize}
\item {Utilização:Prov.}
\end{itemize}
Diz-se da madeira que, por natureza ou velhice, é muito porosa.
\section{Brístol}
\begin{itemize}
\item {Grp. gram.:m.}
\end{itemize}
\begin{itemize}
\item {Proveniência:(De \textunderscore Bristol\textunderscore , n. p.)}
\end{itemize}
Antigo pano grosso de lan.
\section{Brita}
\begin{itemize}
\item {Grp. gram.:f.}
\end{itemize}
Pedra britada.
\section{Britador}
\begin{itemize}
\item {Grp. gram.:m.}
\end{itemize}
Aquelle que brita.
\section{Britamento}
\begin{itemize}
\item {Grp. gram.:m.}
\end{itemize}
Acção de \textunderscore britar\textunderscore .
\section{Britânia}
\begin{itemize}
\item {Grp. gram.:m.}
\end{itemize}
\begin{itemize}
\item {Proveniência:(Lat. \textunderscore Britannia\textunderscore , n. p.)}
\end{itemize}
Metal, composto de estanho e antimónio; metal inglês.
\section{Britanicamente}
\begin{itemize}
\item {Grp. gram.:adv.}
\end{itemize}
\begin{itemize}
\item {Proveniência:(De \textunderscore britânnico\textunderscore )}
\end{itemize}
Á maneira dos Inglêses.
\section{Britânico}
\begin{itemize}
\item {Grp. gram.:adj.}
\end{itemize}
\begin{itemize}
\item {Proveniência:(Lat. \textunderscore britannicus\textunderscore )}
\end{itemize}
Relativo á Gran-Bretanha.
\section{Britanista}
\begin{itemize}
\item {Grp. gram.:m.}
\end{itemize}
\begin{itemize}
\item {Utilização:P. us.}
\end{itemize}
\begin{itemize}
\item {Proveniência:(De \textunderscore Britannia\textunderscore , n. p.)}
\end{itemize}
O mesmo que \textunderscore anglóphilo\textunderscore ; indivíduo, conhecedor da literatura inglesa.
\section{Britânnia}
\begin{itemize}
\item {Grp. gram.:m.}
\end{itemize}
\begin{itemize}
\item {Proveniência:(Lat. \textunderscore Britannia\textunderscore , n. p.)}
\end{itemize}
Metal, composto de estanho e antimónio; metal inglês.
\section{Britannicamente}
\begin{itemize}
\item {Grp. gram.:adv.}
\end{itemize}
\begin{itemize}
\item {Proveniência:(De \textunderscore britânnico\textunderscore )}
\end{itemize}
Á maneira dos Inglêses.
\section{Britânnico}
\begin{itemize}
\item {Grp. gram.:adj.}
\end{itemize}
\begin{itemize}
\item {Proveniência:(Lat. \textunderscore britannicus\textunderscore )}
\end{itemize}
Relativo á Gran-Bretanha.
\section{Britannista}
\begin{itemize}
\item {Grp. gram.:m.}
\end{itemize}
\begin{itemize}
\item {Utilização:P. us.}
\end{itemize}
\begin{itemize}
\item {Proveniência:(De \textunderscore Britannia\textunderscore , n. p.)}
\end{itemize}
O mesmo que \textunderscore anglóphilo\textunderscore ; indivíduo, conhecedor da literatura inglesa.
\section{Britanno}
\begin{itemize}
\item {Grp. gram.:m.}
\end{itemize}
(V. \textunderscore bretão\textunderscore ^2)
\section{Britano}
\begin{itemize}
\item {Grp. gram.:m.}
\end{itemize}
(V. \textunderscore bretão\textunderscore ^2)
\section{Brita-ossos}
\begin{itemize}
\item {Grp. gram.:m.}
\end{itemize}
(V.xofrango)
\section{Britar}
\begin{itemize}
\item {Grp. gram.:v. t.}
\end{itemize}
\begin{itemize}
\item {Utilização:Fig.}
\end{itemize}
\begin{itemize}
\item {Proveniência:(Do anglo-sax. \textunderscore brittian\textunderscore ?)}
\end{itemize}
Partir, quebrar em bocadinhos: \textunderscore britar pedra\textunderscore .
Contundir, moer: \textunderscore britou-lhe os ossos com pancadas\textunderscore .
Reduzir a nada.
Invalidar.
\section{Brivana}
\begin{itemize}
\item {Grp. gram.:f.}
\end{itemize}
\begin{itemize}
\item {Utilização:Bras. do N}
\end{itemize}
O mesmo que \textunderscore égua\textunderscore .
\section{Brives}
\begin{itemize}
\item {Grp. gram.:m. pl.}
\end{itemize}
\begin{itemize}
\item {Utilização:Náut.}
\end{itemize}
Cabos, com que se recolhem as velas.
\section{Briza}
\begin{itemize}
\item {Grp. gram.:f.}
\end{itemize}
\begin{itemize}
\item {Proveniência:(Do gr. \textunderscore briza\textunderscore )}
\end{itemize}
Gênero de plantas gramíneas.
\section{Brizomancia}
\begin{itemize}
\item {Grp. gram.:f.}
\end{itemize}
\begin{itemize}
\item {Proveniência:(Do gr. \textunderscore brizein\textunderscore  + \textunderscore manteia\textunderscore )}
\end{itemize}
Arte de adivinhar pelos sonhos.
\section{Brizomantico}
\begin{itemize}
\item {Grp. gram.:adj.}
\end{itemize}
Relativo á \textunderscore brizomancia\textunderscore .
\section{Brôa}
\begin{itemize}
\item {Grp. gram.:f.}
\end{itemize}
\begin{itemize}
\item {Utilização:T. da Bairrada}
\end{itemize}
\begin{itemize}
\item {Grp. gram.:Pl.}
\end{itemize}
Pão de milho.
Bolo frito de farinha de milho.
Bolo da mesma farinha, com mel, azeite, etc., usado principalmente pelo Natal.
O mesmo que \textunderscore carcunda\textunderscore .
Presente de festa pelo Natal.
(Cp. al. \textunderscore brod\textunderscore )
\section{Breloque}
\begin{itemize}
\item {Grp. gram.:m.}
\end{itemize}
\begin{itemize}
\item {Proveniência:(Fr. \textunderscore breloque\textunderscore )}
\end{itemize}
Pequeno enfeite, que se traz pendente na cadeia do relógio ou nas pulseiras.
Curiosidade de pouco valor.
\section{Broba}
\begin{itemize}
\item {Grp. gram.:f.}
\end{itemize}
\begin{itemize}
\item {Utilização:Prov.}
\end{itemize}
O mesmo que \textunderscore abóbora\textunderscore .
\section{Bróca}
\begin{itemize}
\item {Grp. gram.:f.}
\end{itemize}
\begin{itemize}
\item {Utilização:Bras}
\end{itemize}
\begin{itemize}
\item {Utilização:Bras}
\end{itemize}
\begin{itemize}
\item {Utilização:Bras}
\end{itemize}
\begin{itemize}
\item {Utilização:Bras}
\end{itemize}
\begin{itemize}
\item {Utilização:Bras. do N}
\end{itemize}
\begin{itemize}
\item {Proveniência:(Do lat. \textunderscore brochus\textunderscore )}
\end{itemize}
Pua, instrumento com que se abrem buracos circulares, fazendo-se girar por meio de um arco.
Instrumento, formado pela pua, eixo e arco respectivo.
Eixo de fechadura.
Barra de ferro, com que se abrem nas pedreiras os orificios que, cheios de matéria explosiva, determinam o córte das mesmas pedreiras.
Falha na bôca do canhão.
Espécie de joeira, com que se limpa o café em grão.
Verme, que ataca as raízes de algumas plantas. Cf. \textunderscore Bibl. da Gente do Campo\textunderscore , 303 e 382.
Cavidade no casco do cavallo.
Mato rasteiro, entre árvores corpulentas.
Buraco, feito pelo instrumento chamado bróca. Cf. M. Soares, \textunderscore Diccion. Bras.\textunderscore 
Espécie de lagarto.
Acto de derrubar arbustos ou mato, preparando terreno para a roça.
\section{Brôca}
\begin{itemize}
\item {Grp. gram.:f.}
\end{itemize}
\begin{itemize}
\item {Utilização:Prov.}
\end{itemize}
\begin{itemize}
\item {Utilização:trasm.}
\end{itemize}
Ferroada de um pião noutro ou no sobrado.
(Alter. phonét. de \textunderscore bróca\textunderscore ?)
\section{Broça}
\begin{itemize}
\item {Grp. gram.:f.}
\end{itemize}
\begin{itemize}
\item {Utilização:Prov.}
\end{itemize}
\begin{itemize}
\item {Utilização:trasm.}
\end{itemize}
\begin{itemize}
\item {Utilização:Ext.}
\end{itemize}
\begin{itemize}
\item {Utilização:Gír.}
\end{itemize}
Comida de porcos, feita de batatas, abóboras e farelo.
Porcaria espêssa.
Dinheiro.
(Cast. \textunderscore broza\textunderscore )
\section{Brocadilho}
\begin{itemize}
\item {Grp. gram.:m.}
\end{itemize}
\begin{itemize}
\item {Proveniência:(De \textunderscore brocado\textunderscore )}
\end{itemize}
Brocado inferior.
\section{Brocado}
\begin{itemize}
\item {Grp. gram.:m.}
\end{itemize}
\begin{itemize}
\item {Proveniência:(It. \textunderscore brocatto\textunderscore )}
\end{itemize}
Estôfo, entretecido de seda e fios de oiro ou prata, com figuras ou flôres em relêvo.
\section{Brocador}
\begin{itemize}
\item {Grp. gram.:m.}
\end{itemize}
\begin{itemize}
\item {Utilização:Bras. do N}
\end{itemize}
\begin{itemize}
\item {Proveniência:(De \textunderscore brocar\textunderscore )}
\end{itemize}
Aquelle que corta ou derruba mato.
\section{Brocal}
\begin{itemize}
\item {Grp. gram.:m.}
\end{itemize}
\begin{itemize}
\item {Utilização:Heráld.}
\end{itemize}
\begin{itemize}
\item {Proveniência:(De \textunderscore bróca\textunderscore )}
\end{itemize}
Guarnição de aço nos escudos.
\section{Brocão}
\begin{itemize}
\item {Grp. gram.:m.}
\end{itemize}
Espécie de palmeira, donde mana o bdéllio.
\section{Brocar}
\begin{itemize}
\item {Grp. gram.:v. t.}
\end{itemize}
\begin{itemize}
\item {Utilização:Bras}
\end{itemize}
\begin{itemize}
\item {Utilização:Bras}
\end{itemize}
Furar com brôca.
Joeirar (café)
Cortar (mato fino), com foice, broquear.
\section{Brocar}
\begin{itemize}
\item {Grp. gram.:v. t.}
\end{itemize}
\begin{itemize}
\item {Utilização:Prov.}
\end{itemize}
\begin{itemize}
\item {Utilização:trasm.}
\end{itemize}
\begin{itemize}
\item {Proveniência:(De \textunderscore brôca\textunderscore )}
\end{itemize}
Dar brôcas em.
\section{Brocárdico}
\begin{itemize}
\item {Grp. gram.:m.}
\end{itemize}
\begin{itemize}
\item {Utilização:Des.}
\end{itemize}
O mesmo que \textunderscore brocardo\textunderscore .
\section{Brocardo}
\begin{itemize}
\item {Grp. gram.:m.}
\end{itemize}
Axioma jurídico.
Aphorismo; anexim, máxima.
(B. lat. \textunderscore brocardum\textunderscore , de \textunderscore Burchard\textunderscore , n. p. de um Bispo de Worms)
\section{Brocatel}
\begin{itemize}
\item {Grp. gram.:m.}
\end{itemize}
\begin{itemize}
\item {Proveniência:(It. \textunderscore brocattello\textunderscore )}
\end{itemize}
Tecido, semelhante ao brocado.
Tecido adamascado.
\section{Brocatello}
\begin{itemize}
\item {Grp. gram.:m.}
\end{itemize}
Mármore italiano, de côres variegadas.
(Cp. \textunderscore brocatel\textunderscore )
\section{Brocatelo}
\begin{itemize}
\item {Grp. gram.:m.}
\end{itemize}
Mármore italiano, de côres variegadas.
(Cp. \textunderscore brocatel\textunderscore )
\section{Brocha}
\begin{itemize}
\item {Grp. gram.:m.}
\end{itemize}
Pincel grande, para caiar, ou para pintura ordinária.
(Cp. \textunderscore brossa\textunderscore )
\section{Brocha}
\begin{itemize}
\item {Grp. gram.:f.}
\end{itemize}
\begin{itemize}
\item {Utilização:Ant.}
\end{itemize}
Prego curto, com cabeça larga e chata.
O mesmo que \textunderscore broche\textunderscore .
Chaveta nas extremidades do eixo do carro.
Corda, que liga os fueiros para segurar a carga.
Correia, que liga á canga o pescoço do boi.
Cinta, com que se apertam alporcas.
\textunderscore Estar á brocha\textunderscore , \textunderscore vêr-se á brocha\textunderscore , estar em apuros, vêr-se em talas.
(Cp. \textunderscore bróca\textunderscore )
\section{Brôcha}
\begin{itemize}
\item {Grp. gram.:f.}
\end{itemize}
\begin{itemize}
\item {Utilização:Prov.}
\end{itemize}
\begin{itemize}
\item {Utilização:beir.}
\end{itemize}
\begin{itemize}
\item {Utilização:Chul.}
\end{itemize}
O mesmo que \textunderscore dinheiro\textunderscore .
\section{Brochadeira}
\begin{itemize}
\item {Grp. gram.:f.}
\end{itemize}
\begin{itemize}
\item {Proveniência:(De \textunderscore brochar\textunderscore )}
\end{itemize}
Mulher, que brocha livros.
\section{Brochado}
\begin{itemize}
\item {Grp. gram.:adj.}
\end{itemize}
\begin{itemize}
\item {Proveniência:(De \textunderscore brochar\textunderscore )}
\end{itemize}
Diz-se dos livros que, não estando encadernados, têm as folhas cosidas e capa de papel.
\section{Brochador}
\begin{itemize}
\item {Grp. gram.:m.}
\end{itemize}
Aquelle que brocha.
\section{Brochagem}
\begin{itemize}
\item {Grp. gram.:f.}
\end{itemize}
Acto de \textunderscore brochar\textunderscore .
\section{Brochante}
\begin{itemize}
\item {Grp. gram.:m.}
\end{itemize}
\begin{itemize}
\item {Proveniência:(De \textunderscore brocha\textunderscore ^1)}
\end{itemize}
Official de pintor, que prepara as tintas e executa o trabalho mais ordinário de pintura.
\section{Brochar}
\begin{itemize}
\item {Grp. gram.:v. t.}
\end{itemize}
\begin{itemize}
\item {Proveniência:(De \textunderscore brocha\textunderscore ^2)}
\end{itemize}
Pregar com brochas: \textunderscore brochar as solas dos sapatos\textunderscore .
\section{Brochar}
\begin{itemize}
\item {Grp. gram.:v. t.}
\end{itemize}
\begin{itemize}
\item {Utilização:Bras}
\end{itemize}
Coser as fôlhas de (livros), depois de dobradas e ordenadas, ligando-lhes em seguida uma capa de papel.
Vestir com roupa vulgar.
\section{Brochar}
\begin{itemize}
\item {Grp. gram.:v. i.}
\end{itemize}
\begin{itemize}
\item {Utilização:Prov.}
\end{itemize}
\begin{itemize}
\item {Utilização:minh.}
\end{itemize}
\begin{itemize}
\item {Proveniência:(De \textunderscore brôcho\textunderscore )}
\end{itemize}
Bater com fôrça ou estrondo.
\section{Brochasa}
\begin{itemize}
\item {Grp. gram.:f.}
\end{itemize}
\begin{itemize}
\item {Utilização:Ant.}
\end{itemize}
Espécie de coberta ou colcha preciosa.
\section{Broche}
\begin{itemize}
\item {Grp. gram.:m.}
\end{itemize}
Fecho de metal, ou jóia, com que as mulheres prendem o chale sôbre o peito, ou que usam como ornato na parte anterior da gola do vestido.
(Cast. \textunderscore broche\textunderscore )
\section{Brôcho}
\begin{itemize}
\item {Grp. gram.:m.}
\end{itemize}
\begin{itemize}
\item {Utilização:Prov.}
\end{itemize}
\begin{itemize}
\item {Utilização:beir.}
\end{itemize}
Correia estreita, para apertar qualquer objecto.
Tira de coiro, que aperta o encedoiro do mangual.
(Cp. \textunderscore brocha\textunderscore ^2)
\section{Brochote}
\begin{itemize}
\item {Grp. gram.:m.}
\end{itemize}
\begin{itemize}
\item {Utilização:Bras. do N}
\end{itemize}
Indivíduo insignificante; bisbórria; borra-botas.
\section{Brochura}
\begin{itemize}
\item {Grp. gram.:f.}
\end{itemize}
\begin{itemize}
\item {Proveniência:(Fr. \textunderscore brochure\textunderscore . Cp. \textunderscore brochar\textunderscore ^2)}
\end{itemize}
Arte de brochar livros.
Livro ou folheto brochado.
Folheto, opúsculo.
\section{Bróciga}
\begin{itemize}
\item {Grp. gram.:f.}
\end{itemize}
\begin{itemize}
\item {Utilização:Prov.}
\end{itemize}
\begin{itemize}
\item {Utilização:trasm.}
\end{itemize}
O mesmo que \textunderscore broça\textunderscore .
\section{Broco}
\begin{itemize}
\item {Grp. gram.:adj.}
\end{itemize}
\begin{itemize}
\item {Utilização:Bras. do N}
\end{itemize}
Diz-se das reses que têm um dos chifres pequeno e rugoso, ou ambos.
\section{Brócolos}
\begin{itemize}
\item {Grp. gram.:m. pl.}
\end{itemize}
\begin{itemize}
\item {Proveniência:(It. \textunderscore broccoli\textunderscore )}
\end{itemize}
Planta hortense, espécie de couve.
\section{Brocos}
\begin{itemize}
\item {Grp. gram.:m. pl.}
\end{itemize}
(Corr. de \textunderscore brócolos\textunderscore )
\section{Bródio}
\begin{itemize}
\item {Grp. gram.:m.}
\end{itemize}
\begin{itemize}
\item {Utilização:Ant.}
\end{itemize}
Comezaina.
Refeição alegre; patuscada.
Caldo, que se dava aos pobres, á porta do convento.
(B. lat. \textunderscore brodium\textunderscore , do all. \textunderscore brod\textunderscore )
\section{Brodista}
\begin{itemize}
\item {Grp. gram.:m.}
\end{itemize}
Frequentador de bródios.
\section{Broeira}
\begin{itemize}
\item {Grp. gram.:f.}
\end{itemize}
\begin{itemize}
\item {Utilização:Prov.}
\end{itemize}
\begin{itemize}
\item {Utilização:trasm.}
\end{itemize}
\begin{itemize}
\item {Utilização:Prov.}
\end{itemize}
\begin{itemize}
\item {Utilização:minh.}
\end{itemize}
\begin{itemize}
\item {Grp. gram.:M.  e  f.}
\end{itemize}
\begin{itemize}
\item {Utilização:T. da Bairrada}
\end{itemize}
\begin{itemize}
\item {Proveniência:(De \textunderscore brôa\textunderscore )}
\end{itemize}
Apparelho, suspenso do tecto, no qual se collocam as brôas.
Espécie de planta, (\textunderscore scolopendrium officinalis\textunderscore , Lin.), sôbre cujas frondes assentam a brôa que vai para o fôrno.
Pessôa, que tem carcunda.
\section{Broeiro}
\begin{itemize}
\item {Grp. gram.:adj.}
\end{itemize}
\begin{itemize}
\item {Grp. gram.:M.}
\end{itemize}
Que gosta de brôa, que se alimenta de brôa.
Rústico.
Vendedor de brôas.
\section{Brofal}
\begin{itemize}
\item {Grp. gram.:m.}
\end{itemize}
Árvore da Guiné, de fibras têxteis.
\section{Brogúncias}
\begin{itemize}
\item {Grp. gram.:m. pl.}
\end{itemize}
\begin{itemize}
\item {Utilização:Bras. do N}
\end{itemize}
Coisas ou negócios miúdos.
Bagagem ordinária e pobre de quem viaja a pé.
Mobília de casa pobre.
\section{Broi}
\begin{itemize}
\item {Grp. gram.:adj.}
\end{itemize}
\begin{itemize}
\item {Utilização:Gír.}
\end{itemize}
Bom.
\section{Broia}
\begin{itemize}
\item {Grp. gram.:adj.}
\end{itemize}
(Fem. de \textunderscore broi\textunderscore )
\section{Brolar}
\textunderscore v. t.\textunderscore  (e der.)
O mesmo que \textunderscore broslar\textunderscore , etc.
\section{Brolho}
\begin{itemize}
\item {fónica:brô}
\end{itemize}
\begin{itemize}
\item {Grp. gram.:m.}
\end{itemize}
\begin{itemize}
\item {Utilização:Prov.}
\end{itemize}
Cangaço.
\section{Brollar}
\textunderscore v. t.\textunderscore  (e der.)
O mesmo que \textunderscore broslar\textunderscore , etc.
\section{Broma}
\begin{itemize}
\item {Grp. gram.:f.}
\end{itemize}
\begin{itemize}
\item {Grp. gram.:Adj.}
\end{itemize}
\begin{itemize}
\item {Grp. gram.:M.}
\end{itemize}
Verme, que rói a madeira.
Parte da ferradura, em que assenta a parede circular do casco.
Grosseiro, ordinário.
Homem estúpido.
Planta escrofularínea do Brasil.
\section{Broma}
\begin{itemize}
\item {Grp. gram.:f.}
\end{itemize}
\begin{itemize}
\item {Utilização:Prov.}
\end{itemize}
\begin{itemize}
\item {Utilização:alent.}
\end{itemize}
\begin{itemize}
\item {Proveniência:(T. cast.)}
\end{itemize}
Facécia; brincadeira; chalaça.
\section{Bromado}
\begin{itemize}
\item {Grp. gram.:adj.}
\end{itemize}
Que contém brómo.
\section{Bromalina}
\begin{itemize}
\item {Grp. gram.:f.}
\end{itemize}
Medicamento sedativo.
\section{Bromar}
\begin{itemize}
\item {Grp. gram.:v. t.}
\end{itemize}
\begin{itemize}
\item {Utilização:Bras}
\end{itemize}
\begin{itemize}
\item {Grp. gram.:V. i.}
\end{itemize}
\begin{itemize}
\item {Utilização:Bras}
\end{itemize}
\begin{itemize}
\item {Proveniência:(De \textunderscore broma\textunderscore ^1)}
\end{itemize}
Corroer, como a broma.
Estragar (o açúcar nos engenhos), tornando-o broma, ordinário.
Sêr mal succedido.
Inutilizar-se.
Soffrer quebra no pêso, medida ou valor.
\section{Bromato}
\begin{itemize}
\item {Grp. gram.:m.}
\end{itemize}
\begin{itemize}
\item {Proveniência:(De \textunderscore bromo\textunderscore )}
\end{itemize}
Sal, produzido pela combinação do ácido brómico com uma base.
\section{Bromatologia}
\begin{itemize}
\item {Grp. gram.:f.}
\end{itemize}
\begin{itemize}
\item {Proveniência:(Do gr. \textunderscore broma\textunderscore  + \textunderscore logos\textunderscore )}
\end{itemize}
Descripção dos alimentos.
\section{Bromatológico}
\begin{itemize}
\item {Grp. gram.:adj.}
\end{itemize}
Relativo á bromatologia.
\section{Bromatologista}
\begin{itemize}
\item {Grp. gram.:m.}
\end{itemize}
O mesmo que \textunderscore bromatólogo\textunderscore .
\section{Bromatólogo}
\begin{itemize}
\item {Grp. gram.:m.}
\end{itemize}
Aquelle que é versado em bromatologia.
\section{Bromélia}
\begin{itemize}
\item {Grp. gram.:f.}
\end{itemize}
O mesmo que \textunderscore ananás\textunderscore .
\section{Bromeliáceas}
\begin{itemize}
\item {Grp. gram.:f. pl.}
\end{itemize}
\begin{itemize}
\item {Proveniência:(De \textunderscore bromelia\textunderscore )}
\end{itemize}
Família de plantas, a que serve de typo o ananás.
\section{Bromelina}
\begin{itemize}
\item {Grp. gram.:f.}
\end{itemize}
Princípio activo do suco do abacaxi.
\section{Brometo}
\begin{itemize}
\item {fónica:mê}
\end{itemize}
\begin{itemize}
\item {Grp. gram.:m.}
\end{itemize}
\begin{itemize}
\item {Proveniência:(De \textunderscore brómo\textunderscore )}
\end{itemize}
Combinação do brómo com outro corpo simples.
\section{Brómico}
\begin{itemize}
\item {Grp. gram.:adj.}
\end{itemize}
\begin{itemize}
\item {Proveniência:(De \textunderscore brómo\textunderscore )}
\end{itemize}
Diz-se do ácido, resultante da combinação do brómo com o oxygênio.
\section{Bromídia}
\begin{itemize}
\item {Grp. gram.:f.}
\end{itemize}
Medicamento hypnótico, que consiste numa solução de brometo de potássio, chloral, etc.
\section{Bromidrose}
\begin{itemize}
\item {Grp. gram.:f.}
\end{itemize}
\begin{itemize}
\item {Utilização:Med.}
\end{itemize}
\begin{itemize}
\item {Proveniência:(Do gr. \textunderscore bromos\textunderscore , mau cheiro, e \textunderscore hidros\textunderscore , suor)}
\end{itemize}
Suor fétido.
\section{Bromina}
\begin{itemize}
\item {Grp. gram.:f.}
\end{itemize}
\begin{itemize}
\item {Proveniência:(Do gr. \textunderscore broma\textunderscore )}
\end{itemize}
Substância elementar de algumas plantas marinhas.
\section{Brómio}
\begin{itemize}
\item {Grp. gram.:m.}
\end{itemize}
O mesmo que \textunderscore brómo\textunderscore ^1.
\section{Bromismo}
\begin{itemize}
\item {Grp. gram.:m.}
\end{itemize}
\begin{itemize}
\item {Proveniência:(De \textunderscore brómo\textunderscore )}
\end{itemize}
Effeitos pathológicos do abuso do brometo de sódio.
\section{Brómo}
\begin{itemize}
\item {Grp. gram.:m.}
\end{itemize}
\begin{itemize}
\item {Proveniência:(Gr. \textunderscore bromos\textunderscore )}
\end{itemize}
Metalloide, líquido, avermelhado e venenoso.
\section{Brómo}
\begin{itemize}
\item {Grp. gram.:m.}
\end{itemize}
Gênero de plantas gramíneas, de praganas longas e duras.
\section{Bromofórmio}
\begin{itemize}
\item {Grp. gram.:m.}
\end{itemize}
\begin{itemize}
\item {Proveniência:(De \textunderscore bromo\textunderscore  e rad. de \textunderscore fórmico\textunderscore )}
\end{itemize}
Substância anesthésica, que contém brómo, e é análoga ao chlorofórmio.
\section{Bromografia}
\begin{itemize}
\item {Grp. gram.:f.}
\end{itemize}
\begin{itemize}
\item {Proveniência:(Do gr. \textunderscore broma\textunderscore  + \textunderscore graphein\textunderscore )}
\end{itemize}
O mesmo que \textunderscore bromatologia\textunderscore .
\section{Bromographia}
\begin{itemize}
\item {Grp. gram.:f.}
\end{itemize}
\begin{itemize}
\item {Proveniência:(Do gr. \textunderscore broma\textunderscore  + \textunderscore graphein\textunderscore )}
\end{itemize}
O mesmo que \textunderscore bromatologia\textunderscore .
\section{Bromurado}
\begin{itemize}
\item {Grp. gram.:adj.}
\end{itemize}
Que contém brómo.
\section{Bromureto}
\begin{itemize}
\item {fónica:murê}
\end{itemize}
\begin{itemize}
\item {Grp. gram.:m.}
\end{itemize}
O mesmo que \textunderscore brometo\textunderscore .
\section{Bronchial}
\begin{itemize}
\item {fónica:qui}
\end{itemize}
\begin{itemize}
\item {Grp. gram.:adj.}
\end{itemize}
Relativo aos brônchios.
\section{Brônchico}
\begin{itemize}
\item {fónica:qui}
\end{itemize}
\begin{itemize}
\item {Grp. gram.:adj.}
\end{itemize}
O mesmo que \textunderscore bronchial\textunderscore .
\section{Brônchio}
\begin{itemize}
\item {fónica:qui}
\end{itemize}
\begin{itemize}
\item {Grp. gram.:m.}
\end{itemize}
\begin{itemize}
\item {Proveniência:(Do gr. \textunderscore bronkhos\textunderscore )}
\end{itemize}
Cada um dos dois canaes, que são o prolongamento da tracheia e que se ramificam nos pulmões, communicando-lhes o ar.
\section{Bronchite}
\begin{itemize}
\item {fónica:qui}
\end{itemize}
\begin{itemize}
\item {Grp. gram.:f.}
\end{itemize}
Inflammação dos brônchios.
\section{Bronchocele}
\begin{itemize}
\item {fónica:co}
\end{itemize}
\begin{itemize}
\item {Grp. gram.:m.}
\end{itemize}
\begin{itemize}
\item {Proveniência:(Do gr. \textunderscore bronkhos\textunderscore  + \textunderscore kele\textunderscore )}
\end{itemize}
Tumor no pescoço, papeira.
\section{Bronchophonia}
\begin{itemize}
\item {fónica:co}
\end{itemize}
\begin{itemize}
\item {Grp. gram.:f.}
\end{itemize}
\begin{itemize}
\item {Proveniência:(Do gr. \textunderscore bronkhos\textunderscore  + \textunderscore phone\textunderscore )}
\end{itemize}
Resonância da voz na ramificação dos brônchios.
\section{Bronchoplastia}
\begin{itemize}
\item {Grp. gram.:f.}
\end{itemize}
\begin{itemize}
\item {Proveniência:(Do gr. \textunderscore bronkhos\textunderscore  + \textunderscore plassein\textunderscore )}
\end{itemize}
Operação cirúrgica, que preenche as soluções de continuidade na tracheia com uma porção da pelle do pescoço.
\section{Bronchorrheia}
\begin{itemize}
\item {Grp. gram.:f.}
\end{itemize}
\begin{itemize}
\item {Utilização:Med.}
\end{itemize}
\begin{itemize}
\item {Proveniência:(De \textunderscore brônchio\textunderscore  + gr. \textunderscore rhein\textunderscore )}
\end{itemize}
Fluxo mucoso e abundante pelos brônchios.
\section{Bronchotomia}
\begin{itemize}
\item {fónica:co}
\end{itemize}
\begin{itemize}
\item {Grp. gram.:f.}
\end{itemize}
O mesmo que \textunderscore tracheotomia\textunderscore .
(Cp. \textunderscore bronchótomo\textunderscore )
\section{Bronchótomo}
\begin{itemize}
\item {fónica:có}
\end{itemize}
\begin{itemize}
\item {Grp. gram.:m.}
\end{itemize}
\begin{itemize}
\item {Proveniência:(Do gr. \textunderscore bronkhos\textunderscore  + \textunderscore tome\textunderscore )}
\end{itemize}
Instrumento, com que se pratíca a bronchotomia.
\section{Bronco}
\begin{itemize}
\item {Grp. gram.:adj.}
\end{itemize}
\begin{itemize}
\item {Proveniência:(Do lat. \textunderscore bronchus\textunderscore ?)}
\end{itemize}
Áspero.
Rude, grosseiro.
Obtuso; estúpido.
Desajeitado, malfeito.
\section{Bronço}
\begin{itemize}
\item {Grp. gram.:m.}
\end{itemize}
\begin{itemize}
\item {Utilização:Ant.}
\end{itemize}
\begin{itemize}
\item {Proveniência:(Do it. \textunderscore bronzo\textunderscore )}
\end{itemize}
O mesmo que \textunderscore bronze\textunderscore .
\section{Broncocele}
\begin{itemize}
\item {Grp. gram.:m.}
\end{itemize}
\begin{itemize}
\item {Proveniência:(Do gr. \textunderscore bronkhos\textunderscore  + \textunderscore kele\textunderscore )}
\end{itemize}
Tumor no pescoço, papeira.
\section{Broncofonia}
\begin{itemize}
\item {Grp. gram.:f.}
\end{itemize}
\begin{itemize}
\item {Proveniência:(Do gr. \textunderscore bronkhos\textunderscore  + \textunderscore phone\textunderscore )}
\end{itemize}
Resonância da voz na ramificação dos brônchios.
\section{Broncoplastia}
\begin{itemize}
\item {Grp. gram.:f.}
\end{itemize}
\begin{itemize}
\item {Proveniência:(Do gr. \textunderscore bronkhos\textunderscore  + \textunderscore plassein\textunderscore )}
\end{itemize}
Operação cirúrgica, que preenche as soluções de continuidade na tracheia com uma porção da pelle do pescoço.
\section{Broncorreia}
\begin{itemize}
\item {Grp. gram.:f.}
\end{itemize}
\begin{itemize}
\item {Utilização:Med.}
\end{itemize}
\begin{itemize}
\item {Proveniência:(De \textunderscore brônchio\textunderscore  + gr. \textunderscore rhein\textunderscore )}
\end{itemize}
Fluxo mucoso e abundante pelos brônchios.
\section{Brondúzio}
\begin{itemize}
\item {Grp. gram.:adj.}
\end{itemize}
\begin{itemize}
\item {Utilização:Ant.}
\end{itemize}
Macambúzio? Cf. \textunderscore Anat. Joc.\textunderscore , I, 197.
\section{Bronhenta}
\begin{itemize}
\item {Grp. gram.:adj. f.}
\end{itemize}
Diz-se de uma casta de azeitona, o mesmo que \textunderscore barrenta\textunderscore .
\section{Bronquial}
\begin{itemize}
\item {Grp. gram.:adj.}
\end{itemize}
Relativo aos brônquios.
\section{Bronquice}
\begin{itemize}
\item {Grp. gram.:f.}
\end{itemize}
Qualidade de quem é bronco. Cf. Camillo, \textunderscore Canc. Alegre\textunderscore , 306.
\section{Brônquico}
\begin{itemize}
\item {Grp. gram.:adj.}
\end{itemize}
O mesmo que \textunderscore bronquial\textunderscore .
\section{Brônquio}
\begin{itemize}
\item {Grp. gram.:m.}
\end{itemize}
\begin{itemize}
\item {Proveniência:(Do gr. \textunderscore bronkhos\textunderscore )}
\end{itemize}
Cada um dos dois canaes, que são o prolongamento da traqueia e que se ramificam nos pulmões, communicando-lhes o ar.
\section{Bronquite}
\begin{itemize}
\item {Grp. gram.:f.}
\end{itemize}
Inflammação dos brônquios.
\section{Bronteu}
\begin{itemize}
\item {Grp. gram.:m.}
\end{itemize}
\begin{itemize}
\item {Proveniência:(Gr. \textunderscore bronteion\textunderscore )}
\end{itemize}
Vaso que, nos theatros antigos, imitava as trovoadas.
\section{Brontia}
\begin{itemize}
\item {Grp. gram.:f.}
\end{itemize}
\begin{itemize}
\item {Proveniência:(Gr. \textunderscore bronte\textunderscore )}
\end{itemize}
Nome ant. de uma pedra preciosa.
\section{Brontólitho}
\begin{itemize}
\item {Grp. gram.:m.}
\end{itemize}
\begin{itemize}
\item {Proveniência:(Do gr. \textunderscore bronte\textunderscore  + \textunderscore lithos\textunderscore )}
\end{itemize}
Pedaço de ferro sulfurado, que as chuvas copiosas põem a descoberto em terrenos cretáceos.
\section{Brontólito}
\begin{itemize}
\item {Grp. gram.:m.}
\end{itemize}
\begin{itemize}
\item {Proveniência:(Do gr. \textunderscore bronte\textunderscore  + \textunderscore lithos\textunderscore )}
\end{itemize}
Pedaço de ferro sulfurado, que as chuvas copiosas põem a descoberto em terrenos cretáceos.
\section{Brontómetro}
\begin{itemize}
\item {Grp. gram.:m.}
\end{itemize}
\begin{itemize}
\item {Proveniência:(Do gr. \textunderscore bronte\textunderscore  + \textunderscore metron\textunderscore )}
\end{itemize}
Instrumento, com que se avalia a electricidade atmosphérica, em occasião de tempestade.
\section{Bronzagem}
\begin{itemize}
\item {Grp. gram.:f.}
\end{itemize}
\begin{itemize}
\item {Proveniência:(De \textunderscore bronze\textunderscore )}
\end{itemize}
Operação de bronzear.
\section{Bronze}
\begin{itemize}
\item {Grp. gram.:m.}
\end{itemize}
\begin{itemize}
\item {Utilização:Fig.}
\end{itemize}
\begin{itemize}
\item {Proveniência:(Fr. \textunderscore bronze\textunderscore , talvez do lat. hyp. \textunderscore brunitius\textunderscore , do germ. \textunderscore brun\textunderscore )}
\end{itemize}
Liga de cobre e estanho, em que entram ás vezes outros metaes.
Esculptura em bronze: \textunderscore os bronzes florentinos\textunderscore .
Artilharia.
Sinos.
Moéda antiga.
Dureza; insensibilidade: \textunderscore coração de bronze\textunderscore .
\section{Bronzeado}
\begin{itemize}
\item {Grp. gram.:adj.}
\end{itemize}
Que tem côr de bronze.
\section{Bronzeador}
\begin{itemize}
\item {Grp. gram.:m.}
\end{itemize}
Aquelle que bronzeia.
\section{Bronzeamento}
\begin{itemize}
\item {Grp. gram.:m.}
\end{itemize}
O mesmo que \textunderscore bronzagem\textunderscore .
\section{Bronzear}
\begin{itemize}
\item {Grp. gram.:v. t.}
\end{itemize}
Dar côr de bronze a.
\section{Brônzeo}
\begin{itemize}
\item {Grp. gram.:adj.}
\end{itemize}
Feito de bronze.
Que tem côr de bronze.
Relativo a bronze.
\section{Bronzípede}
\begin{itemize}
\item {Grp. gram.:adj.}
\end{itemize}
Que tem pés de bronze. Cf. Filinto, VI, 213 e 214.
\section{Bronzista}
\begin{itemize}
\item {Grp. gram.:m.}
\end{itemize}
Aquelle que executa trabalhos em bronze.
\section{Bronzo}
\begin{itemize}
\item {Grp. gram.:m.}
\end{itemize}
\begin{itemize}
\item {Utilização:Ant.}
\end{itemize}
O mesmo que \textunderscore bronze\textunderscore . Cf. Pant. de Aveiro, \textunderscore Itiner.\textunderscore , 28, (2.^a ed.)
\section{Broque}
\begin{itemize}
\item {Grp. gram.:m.}
\end{itemize}
Cano dos ventiladores, nos fornos em que se fundem os metaes.
\section{Broqueamento}
\begin{itemize}
\item {Grp. gram.:m.}
\end{itemize}
Acto ou effeito de \textunderscore broquear\textunderscore .
\section{Broquear}
\begin{itemize}
\item {Grp. gram.:v. t.}
\end{itemize}
O mesmo que \textunderscore brocar\textunderscore ^1.
\section{Broquel}
\begin{itemize}
\item {Grp. gram.:m.}
\end{itemize}
\begin{itemize}
\item {Utilização:Fig.}
\end{itemize}
\begin{itemize}
\item {Proveniência:(Do b. lat. \textunderscore buccularius\textunderscore , do lat. \textunderscore bucca\textunderscore )}
\end{itemize}
Antigo escudo pequeno.
Protecção.
Tábua, em que o trolha segura a cal que vai atirando á parede.
\section{Broquelar}
\textunderscore v. t.\textunderscore  (e der.)
O mesmo que \textunderscore abroquelar\textunderscore .
\section{Broqueleira}
\begin{itemize}
\item {Grp. gram.:f.}
\end{itemize}
\begin{itemize}
\item {Proveniência:(De \textunderscore broquel\textunderscore )}
\end{itemize}
Insecto coleóptero pentâmero, o mesmo que \textunderscore silpha\textunderscore .
\section{Broqueleiro}
\begin{itemize}
\item {Grp. gram.:m.}
\end{itemize}
Fabricante de broquéis.
Aquelle que se armava de broquel.
\section{Broquento}
\begin{itemize}
\item {Grp. gram.:adj.}
\end{itemize}
\begin{itemize}
\item {Proveniência:(De \textunderscore bróca\textunderscore )}
\end{itemize}
Chagado, fistuloso.
\section{Brosimo}
\begin{itemize}
\item {Grp. gram.:m.}
\end{itemize}
Gênero de plantas urticáceas, a que pertence a famosa árvore americana, chamada \textunderscore árvore-da-vaca\textunderscore .
\section{Broslador}
\begin{itemize}
\item {Grp. gram.:m.}
\end{itemize}
Official que brosla.
\section{Broslar}
\begin{itemize}
\item {Grp. gram.:v. t.}
\end{itemize}
\begin{itemize}
\item {Utilização:Ant.}
\end{itemize}
Bordar.
(Cast. \textunderscore broslar\textunderscore )
\section{Brossa}
\begin{itemize}
\item {Grp. gram.:f.}
\end{itemize}
\begin{itemize}
\item {Proveniência:(Fr. \textunderscore brosse\textunderscore )}
\end{itemize}
Escôva de impressor.
Escôva de limpar bêstas.
Máquina, formada de um ou mais tambores guarnecidos de escôvas, para limpar as fazendas, nas fábricas de lanifícios.
\section{Brossográfico}
\begin{itemize}
\item {Grp. gram.:adj.}
\end{itemize}
Relativo a brossógrafo.
\section{Brossógrafo}
\begin{itemize}
\item {Grp. gram.:m.}
\end{itemize}
Espécie de barómetro, que regista num papel as oscillações do instrumento, segundo as variações atmosféricas.
\section{Brossográphico}
\begin{itemize}
\item {Grp. gram.:adj.}
\end{itemize}
Relativo a brossógrapho.
\section{Brossógrapho}
\begin{itemize}
\item {Grp. gram.:m.}
\end{itemize}
Espécie de barómetro, que regista num papel as oscillações do instrumento, segundo as variações atmosphéricas.
\section{Brotadura}
\begin{itemize}
\item {Grp. gram.:f.}
\end{itemize}
O mesmo que \textunderscore rebento\textunderscore . Cf. Usque, \textunderscore Tribulações\textunderscore , 8 v.^o
\section{Brotamento}
\begin{itemize}
\item {Grp. gram.:m.}
\end{itemize}
Acção de \textunderscore brotar\textunderscore .
\section{Brotar}
\begin{itemize}
\item {Grp. gram.:v. t.}
\end{itemize}
\begin{itemize}
\item {Grp. gram.:V. i.}
\end{itemize}
\begin{itemize}
\item {Proveniência:(Do ant. alt. al. \textunderscore brozzen\textunderscore )}
\end{itemize}
Produzir.
Criar.
Expellir.
Pronunciar.
Desabrochar.
Nascer.
Irromper, borbotar.
Mostrar-se.
\section{Broto}
\begin{itemize}
\item {Grp. gram.:m.}
\end{itemize}
\begin{itemize}
\item {Utilização:Bras}
\end{itemize}
Acto de \textunderscore brotar\textunderscore .
Gomo, rebento.
\section{Brotoeja}
\begin{itemize}
\item {Grp. gram.:f.}
\end{itemize}
\begin{itemize}
\item {Proveniência:(De \textunderscore brotar\textunderscore )}
\end{itemize}
Espécie de erupção cutânea.
\section{Brózio}
\begin{itemize}
\item {Grp. gram.:m.}
\end{itemize}
\begin{itemize}
\item {Utilização:T. da Bairrada}
\end{itemize}
Espécie de vime, de que se fazem palitos para os dentes.
\section{Bru}
\begin{itemize}
\item {Grp. gram.:m.}
\end{itemize}
\begin{itemize}
\item {Utilização:Prov.}
\end{itemize}
\begin{itemize}
\item {Utilização:trasm.}
\end{itemize}
Lagarta, nociva ás fôlhas das árvores.
\section{Bruaca}
\begin{itemize}
\item {Grp. gram.:f.}
\end{itemize}
\begin{itemize}
\item {Utilização:Bras}
\end{itemize}
\begin{itemize}
\item {Utilização:Bras. do N}
\end{itemize}
\begin{itemize}
\item {Utilização:Bras. do N}
\end{itemize}
Mala de coiro cru, para transporte de vários objectos sôbre bêstas.
Casaca velha.
Mulher velha.
\section{Briáceas}
\begin{itemize}
\item {Grp. gram.:f. pl.}
\end{itemize}
\begin{itemize}
\item {Proveniência:(De \textunderscore bryáceo\textunderscore )}
\end{itemize}
Família de plantas cryptogâmicas, que comprehende quási todas as espécies de musgos.
\section{Briáceo}
\begin{itemize}
\item {Grp. gram.:adj.}
\end{itemize}
\begin{itemize}
\item {Proveniência:(De \textunderscore brýon\textunderscore )}
\end{itemize}
Relativo ou semelhante a musgos.
\section{Briobião}
\begin{itemize}
\item {Grp. gram.:m.}
\end{itemize}
\begin{itemize}
\item {Proveniência:(Do gr. \textunderscore bruon\textunderscore  + \textunderscore bios\textunderscore )}
\end{itemize}
Planta, da fam. das orchídeas, que cresce nas Antilhas.
\section{Briófilo}
\begin{itemize}
\item {Grp. gram.:adj.}
\end{itemize}
\begin{itemize}
\item {Utilização:Bot.}
\end{itemize}
\begin{itemize}
\item {Proveniência:(Do gr. \textunderscore bruon\textunderscore  + \textunderscore philos\textunderscore )}
\end{itemize}
Que se dá bem entre musgos ou debaixo delles.
\section{Briofílo}
\begin{itemize}
\item {Grp. gram.:m.}
\end{itemize}
\begin{itemize}
\item {Proveniência:(Do gr. \textunderscore bruein\textunderscore  + \textunderscore phullon\textunderscore )}
\end{itemize}
Gênero de plantas crassuláceas.
\section{Briologia}
\begin{itemize}
\item {Grp. gram.:f.}
\end{itemize}
\begin{itemize}
\item {Proveniência:(Do gr. \textunderscore bruon\textunderscore  + \textunderscore logos\textunderscore )}
\end{itemize}
Parte da Botânica, que trata dos musgos.
\section{Briológico}
\begin{itemize}
\item {Grp. gram.:adj.}
\end{itemize}
Relativo á Briologia.
\section{Briologista}
\begin{itemize}
\item {Grp. gram.:m.}
\end{itemize}
Aquelle que se applica á Briologia.
\section{Bríon}
\begin{itemize}
\item {Grp. gram.:m.}
\end{itemize}
\begin{itemize}
\item {Proveniência:(Gr. \textunderscore bruon\textunderscore )}
\end{itemize}
Gênero de musgos, que crescem na casca das árvores.
\section{Briónia}
\begin{itemize}
\item {Grp. gram.:f.}
\end{itemize}
\begin{itemize}
\item {Proveniência:(Do gr. \textunderscore bruone\textunderscore )}
\end{itemize}
Planta cucurbitácea medicinal.
\section{Brionina}
\begin{itemize}
\item {Grp. gram.:f.}
\end{itemize}
Substância venenosa, que se extrái da raiz da briónia.
\section{Brionopse}
\begin{itemize}
\item {Grp. gram.:f.}
\end{itemize}
\begin{itemize}
\item {Proveniência:(Do gr. \textunderscore bruon\textunderscore  + \textunderscore ops\textunderscore )}
\end{itemize}
Gênero de plantas cucurbitáceas.
\section{Briopogão}
\begin{itemize}
\item {Grp. gram.:m.}
\end{itemize}
\begin{itemize}
\item {Proveniência:(Do gr. \textunderscore bruon\textunderscore  + \textunderscore pugon\textunderscore )}
\end{itemize}
Planta cryptogâmica, da fam. das lichenáceas.
\section{Briozoários}
\begin{itemize}
\item {Grp. gram.:m. pl.}
\end{itemize}
\begin{itemize}
\item {Proveniência:(Do gr. \textunderscore bruon\textunderscore  + \textunderscore zoarion\textunderscore )}
\end{itemize}
Mollúscos pequenissimos, que vivem na água.
\section{Britóleo}
\begin{itemize}
\item {Grp. gram.:m.}
\end{itemize}
\begin{itemize}
\item {Utilização:Pharm.}
\end{itemize}
Medicamento, que tem por excipiente a cerveja.
\section{Bruaqueiro}
\begin{itemize}
\item {Grp. gram.:m.}
\end{itemize}
\begin{itemize}
\item {Utilização:Bras}
\end{itemize}
\begin{itemize}
\item {Proveniência:(De \textunderscore bruaca\textunderscore )}
\end{itemize}
Aquelle que conduz gêneros alimentícios, das fazendas para os mercados municipaes.
\section{Bruar}
\begin{itemize}
\item {Grp. gram.:v. i.}
\end{itemize}
\begin{itemize}
\item {Utilização:Prov.}
\end{itemize}
\begin{itemize}
\item {Utilização:minh.}
\end{itemize}
O mesmo que \textunderscore acontecer\textunderscore .
\section{Bruça}
\begin{itemize}
\item {Grp. gram.:f.}
\end{itemize}
\begin{itemize}
\item {Utilização:Des.}
\end{itemize}
O mesmo que \textunderscore brossa\textunderscore .
\section{Bruco}
\begin{itemize}
\item {Grp. gram.:m.}
\end{itemize}
\begin{itemize}
\item {Proveniência:(Lat. \textunderscore bruchus\textunderscore )}
\end{itemize}
Pulgão.
\section{Bruco}
\begin{itemize}
\item {Grp. gram.:m.}
\end{itemize}
Gênero de plantas crassuláceas, (\textunderscore pencedanoides\textunderscore , Lin.).
\section{Bruços}
\begin{itemize}
\item {Grp. gram.:m. pl.}
\end{itemize}
Us. na loc. adv. \textunderscore de bruços\textunderscore , com o tronco inclinado para deante ou para baixo; com o rôsto no chão.
(Cast. \textunderscore bruces\textunderscore , talvez do caló)
\section{Bruega}
\begin{itemize}
\item {Grp. gram.:f.}
\end{itemize}
\begin{itemize}
\item {Utilização:Des.}
\end{itemize}
\begin{itemize}
\item {Utilização:Pleb.}
\end{itemize}
\begin{itemize}
\item {Utilização:Bras}
\end{itemize}
Chuva passageira e miúda.
Bebedeira.
Desordem, barulho.
\section{Brúgia}
\begin{itemize}
\item {Grp. gram.:f.}
\end{itemize}
\begin{itemize}
\item {Proveniência:(De \textunderscore Bruges\textunderscore , n. p.)}
\end{itemize}
Espécie de estamenha antiga.
\section{Brugo}
\begin{itemize}
\item {Grp. gram.:m.}
\end{itemize}
\begin{itemize}
\item {Utilização:Ant.}
\end{itemize}
Qualquer insecto, ou lagarto, prejudicial ás searas.
(Cp. \textunderscore bru\textunderscore )
\section{Brugus}
\begin{itemize}
\item {Grp. gram.:m.}
\end{itemize}
Planta rasteira, medicinal, da Guiné.
\section{Brulha}
\begin{itemize}
\item {Grp. gram.:f.}
\end{itemize}
Fórma de enxêrto.
Enxêrto de borbulha ou de escudete.
(Alter. de \textunderscore borbulha\textunderscore )
\section{Brulho}
\begin{itemize}
\item {Grp. gram.:m.}
\end{itemize}
Bagaço de azeitona.
\section{Brulote}
\begin{itemize}
\item {Grp. gram.:m.}
\end{itemize}
\begin{itemize}
\item {Proveniência:(Fr. \textunderscore brûlôt\textunderscore , de \textunderscore brûler\textunderscore )}
\end{itemize}
Embarcação, carregada de matérias explosivas e destinada a communicar fogo aos navios inimigos.
\section{Bruma}
\begin{itemize}
\item {Grp. gram.:f.}
\end{itemize}
\begin{itemize}
\item {Proveniência:(Lat. \textunderscore bruma\textunderscore )}
\end{itemize}
Nevoeiro.
Atmosphera escura e chuvosa.
Sombra.
Incerteza; mystério.
\section{Brumaceiro}
\begin{itemize}
\item {Grp. gram.:adj.}
\end{itemize}
\begin{itemize}
\item {Utilização:P. us.}
\end{itemize}
\begin{itemize}
\item {Proveniência:(De \textunderscore bruma\textunderscore )}
\end{itemize}
Diz-se do tempo escuro e úmido.
\section{Brumal}
\begin{itemize}
\item {Grp. gram.:adj.}
\end{itemize}
\begin{itemize}
\item {Proveniência:(Lat. \textunderscore brumalis\textunderscore )}
\end{itemize}
Relativo á bruma.
\section{Brumário}
\begin{itemize}
\item {Grp. gram.:m.}
\end{itemize}
\begin{itemize}
\item {Proveniência:(Do lat. \textunderscore bruma\textunderscore )}
\end{itemize}
Segundo mês do calendário da primeira República francesa, (23 de Outubro a 21 de Novembro).
\section{Brumeiro}
\begin{itemize}
\item {Grp. gram.:m.}
\end{itemize}
\begin{itemize}
\item {Utilização:T. da Bairrada}
\end{itemize}
Furúnculo ou tumor purulento.
(Por \textunderscore vurmeiro\textunderscore , de \textunderscore vurmo\textunderscore )
\section{Brumo}
\begin{itemize}
\item {Grp. gram.:m.}
\end{itemize}
(Corr. pop. de \textunderscore vurmo\textunderscore )
\section{Brumoso}
\begin{itemize}
\item {Grp. gram.:adj.}
\end{itemize}
Que tem brumas, nevoento: \textunderscore tempo brumoso\textunderscore .
\section{Brunal}
\begin{itemize}
\item {Grp. gram.:adj.}
\end{itemize}
\begin{itemize}
\item {Utilização:Des.}
\end{itemize}
\begin{itemize}
\item {Proveniência:(De \textunderscore bruno\textunderscore )}
\end{itemize}
Sombrio; triste.
\section{Brundúsio}
\begin{itemize}
\item {Grp. gram.:adj.}
\end{itemize}
\begin{itemize}
\item {Utilização:Ant.}
\end{itemize}
Triste, melancólico.
(Por \textunderscore brunúsio\textunderscore , de \textunderscore bruno\textunderscore )
\section{Brunela}
\begin{itemize}
\item {Grp. gram.:f.}
\end{itemize}
\begin{itemize}
\item {Proveniência:(Do al. \textunderscore brune\textunderscore )}
\end{itemize}
Gênero de plantas medicinaes, labiadas.
\section{Brunete}
\begin{itemize}
\item {fónica:nê}
\end{itemize}
\begin{itemize}
\item {Grp. gram.:adj.}
\end{itemize}
\begin{itemize}
\item {Utilização:Des.}
\end{itemize}
\begin{itemize}
\item {Grp. gram.:M.}
\end{itemize}
Um tanto escuro; acastanhado.
Espécie de tecido escuro de lan.
(Cast. \textunderscore brunete\textunderscore )
\section{Brunheiro}
\begin{itemize}
\item {Grp. gram.:m.}
\end{itemize}
O mesmo que \textunderscore abrunheiro\textunderscore .
\section{Brunhenta}
\begin{itemize}
\item {Grp. gram.:adj. f.}
\end{itemize}
\begin{itemize}
\item {Utilização:Prov.}
\end{itemize}
\begin{itemize}
\item {Utilização:trasm.}
\end{itemize}
Diz-se de uma variedade de azeitona.
(Cp. \textunderscore abrunho\textunderscore )
\section{Brunhete}
\begin{itemize}
\item {fónica:nhê}
\end{itemize}
\begin{itemize}
\item {Grp. gram.:m.  e  adj.}
\end{itemize}
(V.brunete)
\section{Brunhir}
\begin{itemize}
\item {Grp. gram.:v. t.}
\end{itemize}
O mesmo que \textunderscore brunir\textunderscore . Cf. M. Bernardes, \textunderscore Luz e Calor\textunderscore , 556; Castilho, \textunderscore Fastos\textunderscore , I, 45.
\section{Brunho}
\begin{itemize}
\item {Grp. gram.:m.}
\end{itemize}
(V.abrunho)
\section{Brunho-rei}
\begin{itemize}
\item {Grp. gram.:m.}
\end{itemize}
\begin{itemize}
\item {Utilização:Prov.}
\end{itemize}
\begin{itemize}
\item {Utilização:alg.}
\end{itemize}
Variedade de ameixa.
\section{Brunideira}
\begin{itemize}
\item {Grp. gram.:f.}
\end{itemize}
\begin{itemize}
\item {Proveniência:(De \textunderscore brunir\textunderscore )}
\end{itemize}
Mulher que engoma roupa a polimento.
\section{Brunido}
\begin{itemize}
\item {Grp. gram.:adj.}
\end{itemize}
\begin{itemize}
\item {Proveniência:(De \textunderscore brunir\textunderscore )}
\end{itemize}
Engomado.
Luzidío.
\section{Brunidor}
\begin{itemize}
\item {Grp. gram.:m.}
\end{itemize}
\begin{itemize}
\item {Proveniência:(De \textunderscore brunir\textunderscore )}
\end{itemize}
Aquelle ou aquillo que brune.
\section{Brunidura}
\begin{itemize}
\item {Grp. gram.:f.}
\end{itemize}
Acção de \textunderscore brunir\textunderscore .
\section{Brunir}
\begin{itemize}
\item {Grp. gram.:v. t.}
\end{itemize}
\begin{itemize}
\item {Proveniência:(Fr. \textunderscore brunir\textunderscore )}
\end{itemize}
Polir.
Tornar luzidío, brilhante.
Dar lustre a (roupa engomada)
\section{Bruno}
\begin{itemize}
\item {Grp. gram.:adj.}
\end{itemize}
Escuro.
Sombrio.
Infeliz.
(Talvez antigo e escusado gallicismo; fr. \textunderscore brun\textunderscore )
\section{Brusca}
\begin{itemize}
\item {Grp. gram.:f.}
\end{itemize}
\begin{itemize}
\item {Utilização:Ant.}
\end{itemize}
Um dos modos de fazer graminhos, segundo o \textunderscore Livro da Fábrica das Náos\textunderscore , Ms. da Bibl. Nac. de Lisbôa, pág. 95.
\section{Brusca}
\begin{itemize}
\item {Grp. gram.:f.}
\end{itemize}
\begin{itemize}
\item {Proveniência:(Do lat. \textunderscore ruscus\textunderscore )}
\end{itemize}
Planta liliácea, silvestre.
\section{Bruscamente}
\begin{itemize}
\item {Grp. gram.:adv.}
\end{itemize}
De modo \textunderscore brusco\textunderscore .
\section{Brusco}
\begin{itemize}
\item {Grp. gram.:adj.}
\end{itemize}
\begin{itemize}
\item {Proveniência:(Do lat. \textunderscore bruscum\textunderscore )}
\end{itemize}
Áspero.
Arrebatado: \textunderscore gênio brusco\textunderscore .
Imprevisto, rápido: \textunderscore assalto brusco\textunderscore . (Nesta última accepção, é gallicismo escusado)
Escuro; nublado.
\section{Brusquidão}
\begin{itemize}
\item {Grp. gram.:f.}
\end{itemize}
Qualidade de brusco.
\section{Brussa}
\begin{itemize}
\item {Grp. gram.:f.}
\end{itemize}
(V.brossa)
\section{Brutal}
\begin{itemize}
\item {Grp. gram.:adj.}
\end{itemize}
\begin{itemize}
\item {Proveniência:(De \textunderscore bruto\textunderscore )}
\end{itemize}
Grosseiro, selvagem.
Violento.
Próprio de bruto.
\section{Brutalidade}
\begin{itemize}
\item {Grp. gram.:f.}
\end{itemize}
\begin{itemize}
\item {Proveniência:(De \textunderscore brutal\textunderscore )}
\end{itemize}
Qualidade do que é bruto.
Violência.
Grossaria.
Acção brutal.
\section{Brutalizar}
\begin{itemize}
\item {Grp. gram.:v. t.}
\end{itemize}
\begin{itemize}
\item {Proveniência:(De \textunderscore brutal\textunderscore )}
\end{itemize}
Tornar bruto, estúpido; bestificar.
\section{Brutalmente}
\begin{itemize}
\item {Grp. gram.:adv.}
\end{itemize}
De modo \textunderscore brutal\textunderscore .
\section{Brutamente}
\begin{itemize}
\item {Grp. gram.:adv.}
\end{itemize}
(V.brutalmente)
\section{Brutamonte}
\begin{itemize}
\item {Grp. gram.:m.}
\end{itemize}
\begin{itemize}
\item {Utilização:Bras}
\end{itemize}
O mesmo que \textunderscore brutamontes\textunderscore .
\section{Brutamontes}
\begin{itemize}
\item {Grp. gram.:m.}
\end{itemize}
\begin{itemize}
\item {Utilização:Pop.}
\end{itemize}
\begin{itemize}
\item {Proveniência:(De \textunderscore bruto\textunderscore )}
\end{itemize}
Homem muito estúpido.
Selvagem.
\section{Brutaz}
\begin{itemize}
\item {Grp. gram.:adj.}
\end{itemize}
O mesmo que \textunderscore brutal\textunderscore . Cf. Rui Barbosa, \textunderscore Répl.\textunderscore , 157.
\section{Bruteiro}
\begin{itemize}
\item {Grp. gram.:m.}
\end{itemize}
\begin{itemize}
\item {Utilização:Bras. do N}
\end{itemize}
Árvore silvestre.
\section{Brutesco}
\begin{itemize}
\item {fónica:tês}
\end{itemize}
\begin{itemize}
\item {Grp. gram.:adj.}
\end{itemize}
\begin{itemize}
\item {Grp. gram.:M.}
\end{itemize}
Tôsco; semelhante ao que sái da natureza.
Mal feito, ridículo.
Grutesco.
Representação artística de animaes ou scenas agrestes.
(Cast. \textunderscore brutesco\textunderscore )
\section{Bruteza}
\begin{itemize}
\item {Grp. gram.:f.}
\end{itemize}
\begin{itemize}
\item {Proveniência:(De \textunderscore bruto\textunderscore )}
\end{itemize}
Brutalidade.
\section{Brutidade}
\begin{itemize}
\item {Grp. gram.:f.}
\end{itemize}
O mesmo que \textunderscore brutalidade\textunderscore .
\section{Brutidão}
\begin{itemize}
\item {Grp. gram.:m.}
\end{itemize}
(V.brutalidade)
\section{Brutificador}
\begin{itemize}
\item {Grp. gram.:adj.}
\end{itemize}
Que brutifica. Cf. Camillo, \textunderscore Estrêl. Fun.\textunderscore , 66.
\section{Brutificar}
\begin{itemize}
\item {Grp. gram.:v. t.}
\end{itemize}
Tornar bruto; bestificar. Cf. Camillo, \textunderscore Myst. de Lisbôa\textunderscore , II, 215.
\section{Brutitates}
\begin{itemize}
\item {Grp. gram.:m.}
\end{itemize}
\begin{itemize}
\item {Utilização:Prov.}
\end{itemize}
\begin{itemize}
\item {Utilização:trasm.}
\end{itemize}
Brutamontes.
Ignorante; alarve.
\section{Bruto}
\begin{itemize}
\item {Grp. gram.:adj.}
\end{itemize}
\begin{itemize}
\item {Grp. gram.:M.}
\end{itemize}
\begin{itemize}
\item {Proveniência:(Lat. \textunderscore brutus\textunderscore )}
\end{itemize}
Rude, tôsco, grosseiro.
Sem educação.
Que está como saiu da natureza: \textunderscore pedra bruta\textunderscore .
Inerte.
Violento, feroz.
Descommunal.
Completo, sem desconto: \textunderscore rendimento bruto\textunderscore .
Animal irracional.
Homem rude, mal educado, brutal.
\section{Bruxa}
\begin{itemize}
\item {Grp. gram.:f.}
\end{itemize}
\begin{itemize}
\item {Utilização:Bras}
\end{itemize}
Mulher, que se diz, ou que o povo crê, têr pacto com o demónio, adivinhar o futuro e praticar outras artes mysteriosas.
Panela de ferro com orifícios, para servir de braseira.
Pequeno pavio, que faz parte de uma lamparina com azeite.
Nome, que os pescadores do Doíro dão a um peixe maritimo.
Borboleta crepuscular e nocturna.
(Cp. cast. \textunderscore bruja\textunderscore )
\section{Bruxaria}
\begin{itemize}
\item {Grp. gram.:f.}
\end{itemize}
\begin{itemize}
\item {Proveniência:(De \textunderscore bruxa\textunderscore )}
\end{itemize}
Sortilégio.
Acção, attribuída a bruxas.
Acontecimento que, na falta de explicação, se atribue por gracejo a artes diabólicas.
\section{Bruxear}
\begin{itemize}
\item {Grp. gram.:v. i.}
\end{itemize}
\begin{itemize}
\item {Proveniência:(De \textunderscore bruxa\textunderscore )}
\end{itemize}
Fazer bruxarias.
\section{Bruxedo}
\begin{itemize}
\item {fónica:xê}
\end{itemize}
\begin{itemize}
\item {Grp. gram.:m.}
\end{itemize}
O mesmo que \textunderscore bruxaria\textunderscore .
\section{Bruxelense}
\begin{itemize}
\item {Grp. gram.:adj.}
\end{itemize}
\begin{itemize}
\item {Grp. gram.:M.}
\end{itemize}
Relativo a Bruxelas.
Habitante de Bruxelas.
\section{Bruxellense}
\begin{itemize}
\item {Grp. gram.:adj.}
\end{itemize}
\begin{itemize}
\item {Grp. gram.:M.}
\end{itemize}
Relativo a Bruxellas.
Habitante de Bruxellas.
\section{Bruxo}
\begin{itemize}
\item {Grp. gram.:m.}
\end{itemize}
\begin{itemize}
\item {Utilização:Bras}
\end{itemize}
\begin{itemize}
\item {Utilização:Bras. de Goiás}
\end{itemize}
\begin{itemize}
\item {Proveniência:(Do lat. \textunderscore bruchus\textunderscore ?)}
\end{itemize}
Homem, de quem se suppõe exercer as artes de bruxa.
Insecto coleóptero que, no estado de larva, ataca as ervilhas e outros legumes.
Espécie de camarão da ria de Aveiro.
Boi, mestiço de curraleiro e junqueira.
\section{Bruxoleante}
\begin{itemize}
\item {Grp. gram.:adj.}
\end{itemize}
Que bruxoleia.
\section{Bruxolear}
\begin{itemize}
\item {Grp. gram.:v. i.}
\end{itemize}
Tremeluzir; brilhar froixamente.
(Cp. cast. \textunderscore grujulear\textunderscore )
\section{Bruxoleio}
\begin{itemize}
\item {Grp. gram.:m.}
\end{itemize}
Acto de \textunderscore bruxolear\textunderscore .
\section{Bruxulear}
\begin{itemize}
\item {Grp. gram.:v. i.}
\end{itemize}
Tremeluzir; brilhar froixamente.
(Cp. cast. \textunderscore grujulear\textunderscore )
\section{Bruxuleio}
\begin{itemize}
\item {Grp. gram.:m.}
\end{itemize}
Acto de \textunderscore bruxolear\textunderscore .
\section{Bruxulhear}
\textunderscore v. i.\textunderscore  (e der.)
O mesmo que \textunderscore bruxolear\textunderscore , etc. Cf. \textunderscore Filinto\textunderscore , XIV, 78.
\section{Bryáceas}
\begin{itemize}
\item {Grp. gram.:f. pl.}
\end{itemize}
\begin{itemize}
\item {Proveniência:(De \textunderscore bryáceo\textunderscore )}
\end{itemize}
Família de plantas cryptogâmicas, que comprehende quási todas as espécies de musgos.
\section{Bryáceo}
\begin{itemize}
\item {Grp. gram.:adj.}
\end{itemize}
\begin{itemize}
\item {Proveniência:(De \textunderscore brýon\textunderscore )}
\end{itemize}
Relativo ou semelhante a musgos.
\section{Bryobião}
\begin{itemize}
\item {Grp. gram.:m.}
\end{itemize}
\begin{itemize}
\item {Proveniência:(Do gr. \textunderscore bruon\textunderscore  + \textunderscore bios\textunderscore )}
\end{itemize}
Planta, da fam. das orchídeas, que cresce nas Antilhas.
\section{Bryologia}
\begin{itemize}
\item {Grp. gram.:f.}
\end{itemize}
\begin{itemize}
\item {Proveniência:(Do gr. \textunderscore bruon\textunderscore  + \textunderscore logos\textunderscore )}
\end{itemize}
Parte da Botânica, que trata dos musgos.
\section{Bryológico}
\begin{itemize}
\item {Grp. gram.:adj.}
\end{itemize}
Relativo á Bryologia.
\section{Bryologista}
\begin{itemize}
\item {Grp. gram.:m.}
\end{itemize}
Aquelle que se applica á Bryologia.
\section{Brýon}
\begin{itemize}
\item {Grp. gram.:m.}
\end{itemize}
\begin{itemize}
\item {Proveniência:(Gr. \textunderscore bruon\textunderscore )}
\end{itemize}
Gênero de musgos, que crescem na casca das árvores.
\section{Bryónia}
\begin{itemize}
\item {Grp. gram.:f.}
\end{itemize}
\begin{itemize}
\item {Proveniência:(Do gr. \textunderscore bruone\textunderscore )}
\end{itemize}
Planta cucurbitácea medicinal.
\section{Bryonina}
\begin{itemize}
\item {Grp. gram.:f.}
\end{itemize}
Substância venenosa, que se extrái da raiz da bryónia.
\section{Bryonopse}
\begin{itemize}
\item {Grp. gram.:f.}
\end{itemize}
\begin{itemize}
\item {Proveniência:(Do gr. \textunderscore bruon\textunderscore  + \textunderscore ops\textunderscore )}
\end{itemize}
Gênero de plantas cucurbitáceas.
\section{Bryóphilo}
\begin{itemize}
\item {Grp. gram.:adj.}
\end{itemize}
\begin{itemize}
\item {Utilização:Bot.}
\end{itemize}
\begin{itemize}
\item {Proveniência:(Do gr. \textunderscore bruon\textunderscore  + \textunderscore philos\textunderscore )}
\end{itemize}
Que se dá bem entre musgos ou debaixo delles.
\section{Bryophýllo}
\begin{itemize}
\item {Grp. gram.:m.}
\end{itemize}
\begin{itemize}
\item {Proveniência:(Do gr. \textunderscore bruein\textunderscore  + \textunderscore phullon\textunderscore )}
\end{itemize}
Gênero de plantas crassuláceas.
\section{Bryopogão}
\begin{itemize}
\item {Grp. gram.:m.}
\end{itemize}
\begin{itemize}
\item {Proveniência:(Do gr. \textunderscore bruon\textunderscore  + \textunderscore pugon\textunderscore )}
\end{itemize}
Planta cryptogâmica, da fam. das lichenáceas.
\section{Bryozoários}
\begin{itemize}
\item {Grp. gram.:m. pl.}
\end{itemize}
\begin{itemize}
\item {Proveniência:(Do gr. \textunderscore bruon\textunderscore  + \textunderscore zoarion\textunderscore )}
\end{itemize}
Mollúscos pequenissimos, que vivem na água.
\section{Brytóleo}
\begin{itemize}
\item {Grp. gram.:m.}
\end{itemize}
\begin{itemize}
\item {Utilização:Pharm.}
\end{itemize}
Medicamento, que tem por excipiente a cerveja.
\section{Bua}
\begin{itemize}
\item {Grp. gram.:f.}
\end{itemize}
\begin{itemize}
\item {Utilização:Infant.}
\end{itemize}
\begin{itemize}
\item {Proveniência:(Lat. \textunderscore bua\textunderscore )}
\end{itemize}
Agua.
\section{Buama}
\begin{itemize}
\item {Grp. gram.:f.}
\end{itemize}
Pequeno peixe de água salgada.
\section{Buase}
\begin{itemize}
\item {Grp. gram.:m.}
\end{itemize}
Arvoreta polygálea de Angola e Moçambique, (\textunderscore securidaca longipedunculata\textunderscore , Fres.).
\section{Buba}
\begin{itemize}
\item {Grp. gram.:f.}
\end{itemize}
Pequeno tumor na pelle.
(Cp. \textunderscore bubão\textunderscore )
\section{Búbalo}
\begin{itemize}
\item {Grp. gram.:m.}
\end{itemize}
\begin{itemize}
\item {Proveniência:(Gr. \textunderscore bubalos\textunderscore )}
\end{itemize}
Ruminante, do gênero antílope, e semelhante ao veado, mas de cabeça maior e mais comprida.
\section{Buban}
\begin{itemize}
\item {Grp. gram.:f.}
\end{itemize}
(V.bubão)
\section{Bubão}
\begin{itemize}
\item {Grp. gram.:m.}
\end{itemize}
\begin{itemize}
\item {Proveniência:(Gr. \textunderscore boubon\textunderscore )}
\end{itemize}
Tumor duro e inflammatório, que apparece principalmente nas glândulas das virilhas, dos sovacos e do pescoço.
\section{Bubático}
\begin{itemize}
\item {Grp. gram.:adj.}
\end{itemize}
\begin{itemize}
\item {Utilização:Des.}
\end{itemize}
Que tem bubas.
\section{Bubela}
\begin{itemize}
\item {Grp. gram.:f.}
\end{itemize}
\begin{itemize}
\item {Utilização:Prov.}
\end{itemize}
\begin{itemize}
\item {Utilização:trasm.}
\end{itemize}
O mesmo que \textunderscore boubela\textunderscore .
\section{Bubo}
\begin{itemize}
\item {Grp. gram.:m.}
\end{itemize}
(V.bubão)
\section{Bubo-bubo-preto}
\begin{itemize}
\item {Grp. gram.:m.}
\end{itemize}
Planta da ilha de San-Thomé, de propriedades anti-syphilíticas.
\section{Bubónia}
\begin{itemize}
\item {Grp. gram.:f.}
\end{itemize}
Planta herbácea e medicinal, (\textunderscore inula bubonium\textunderscore , Lin.).
\section{Bubónica}
\begin{itemize}
\item {Grp. gram.:adj. f.}
\end{itemize}
\begin{itemize}
\item {Proveniência:(De \textunderscore bubão\textunderscore )}
\end{itemize}
Diz-se de uma peste, muito conhecida na Ásia oriental.
\section{Bubonocele}
\begin{itemize}
\item {Grp. gram.:m.}
\end{itemize}
\begin{itemize}
\item {Proveniência:(Do gr. \textunderscore boubon\textunderscore  + \textunderscore kele\textunderscore )}
\end{itemize}
Hérnia inguinal.
\section{Bubu}
\begin{itemize}
\item {Grp. gram.:m.}
\end{itemize}
Espécie de camisa, de mangas compridas e fechada á frente, usada pelos negros da Senegâmbia e da Nigrícia.
\section{Bubúia}
\begin{itemize}
\item {Grp. gram.:f.}
\end{itemize}
\begin{itemize}
\item {Utilização:Bras}
\end{itemize}
Acto de \textunderscore bubuiar\textunderscore .
\section{Bubuiar}
\begin{itemize}
\item {Grp. gram.:v. i.}
\end{itemize}
\begin{itemize}
\item {Proveniência:(Do guar. \textunderscore bebui\textunderscore )}
\end{itemize}
Boiar, fluctuar.
\section{Búbula}
\begin{itemize}
\item {Grp. gram.:f.}
\end{itemize}
Espécie de pêga indiana, de bico e pés amarelos.
\section{Bucal}
\begin{itemize}
\item {Grp. gram.:adj.}
\end{itemize}
\begin{itemize}
\item {Proveniência:(Do lat. \textunderscore bucca\textunderscore )}
\end{itemize}
Relativo á bôca.
\section{Bucaneiro}
\begin{itemize}
\item {Grp. gram.:m.}
\end{itemize}
\begin{itemize}
\item {Utilização:Bras}
\end{itemize}
Caçador de bois selvagens.
Grande espingarda, usada nessa caça.
Pirata, dos que infestavam as Antilhas.
\section{Bucárdia}
\begin{itemize}
\item {Grp. gram.:f.}
\end{itemize}
\begin{itemize}
\item {Proveniência:(Do gr. \textunderscore bous\textunderscore  + \textunderscore kardia\textunderscore )}
\end{itemize}
Mollusco acéphalo, maritimo.
\section{Buccal}
\begin{itemize}
\item {Grp. gram.:adj.}
\end{itemize}
\begin{itemize}
\item {Proveniência:(Do lat. \textunderscore bucca\textunderscore )}
\end{itemize}
Relativo á bôca.
\section{Buccellário}
\begin{itemize}
\item {Grp. gram.:m.}
\end{itemize}
\begin{itemize}
\item {Utilização:Ant.}
\end{itemize}
Homem adjunto a uma família nobre, que o sustentava, a trôco de certos serviços.
Parasito.
Soldado forte e destemido, que tinha a seu cargo a guarda de algum príncipe.
\section{Buccellário}
\begin{itemize}
\item {Grp. gram.:adj.}
\end{itemize}
\begin{itemize}
\item {Proveniência:(Do lat. \textunderscore buccella\textunderscore , dem. de \textunderscore bucca\textunderscore )}
\end{itemize}
Que tem fórma de pequena bôca.
\section{Buccinador}
\begin{itemize}
\item {Grp. gram.:m.  e  adj.}
\end{itemize}
\begin{itemize}
\item {Proveniência:(Lat. \textunderscore buccinator\textunderscore )}
\end{itemize}
Diz-se de um músculo facial, que serve na mastigação e no sopro.
\section{Bucéfalo}
\begin{itemize}
\item {Grp. gram.:m.}
\end{itemize}
\begin{itemize}
\item {Utilização:Pop.}
\end{itemize}
\begin{itemize}
\item {Proveniência:(Lat. \textunderscore bucephalus\textunderscore )}
\end{itemize}
Cavallo de batalha.
Cavallo ordinário, sendeiro.
\section{Bucelário}
\begin{itemize}
\item {Grp. gram.:m.}
\end{itemize}
\begin{itemize}
\item {Utilização:Ant.}
\end{itemize}
Homem adjunto a uma família nobre, que o sustentava, a trôco de certos serviços.
Parasito.
Soldado forte e destemido, que tinha a seu cargo a guarda de algum príncipe.
\section{Bucelário}
\begin{itemize}
\item {Grp. gram.:adj.}
\end{itemize}
\begin{itemize}
\item {Proveniência:(Do lat. \textunderscore buccella\textunderscore , dem. de \textunderscore bucca\textunderscore )}
\end{itemize}
Que tem fórma de pequena bôca.
\section{Bucelas}
\begin{itemize}
\item {Grp. gram.:m.}
\end{itemize}
Vinho, fabricado em Bucelas.
\section{Bucellas}
\begin{itemize}
\item {Grp. gram.:m.}
\end{itemize}
Vinho, fabricado em Bucellas.
\section{Bucentauro}
\begin{itemize}
\item {Grp. gram.:m.}
\end{itemize}
\begin{itemize}
\item {Utilização:Ant.}
\end{itemize}
Galeão rico de Veneza.
(B. lat. \textunderscore bucentaurus\textunderscore )
\section{Bucentório}
\begin{itemize}
\item {Grp. gram.:m.}
\end{itemize}
\begin{itemize}
\item {Utilização:Ant.}
\end{itemize}
O mesmo que \textunderscore bucentauro\textunderscore . Cf. Pant. de Aveiro, \textunderscore Itiner.\textunderscore , 4 e 5, (2.^a ed.)
\section{Bucéphalo}
\begin{itemize}
\item {Grp. gram.:m.}
\end{itemize}
\begin{itemize}
\item {Utilização:Pop.}
\end{itemize}
\begin{itemize}
\item {Proveniência:(Lat. \textunderscore bucephalus\textunderscore )}
\end{itemize}
Cavallo de batalha.
Cavallo ordinário, sendeiro.
\section{Bucha}
\begin{itemize}
\item {Grp. gram.:f.}
\end{itemize}
\begin{itemize}
\item {Utilização:Fam.}
\end{itemize}
\begin{itemize}
\item {Utilização:Prov.}
\end{itemize}
\begin{itemize}
\item {Utilização:beir.}
\end{itemize}
\begin{itemize}
\item {Utilização:Bras}
\end{itemize}
Aquillo que, dentro das armas de fogo, se põe por cima da carga, para a segurar.
Pedaço de pão ou outro alimento, que se mete na bôca, de uma vez.
Pau, que se mete no pêso do lagar do vinho, para não deixar sair o veio.
Pequeno pau roliço, com que os sapateiros brunem as solas do calçado.
Pedaço de pau, com que nos navios se tapam os rombos, etc.
Espécie de rolha ou chumaço, com que se tapam orifícios ou fendas em objectos de madeira.
Peça cylíndrica, vazía, em que giram as extremidades dos eixos de algumas rodas.
Aquillo que importuna ou incommoda.
Tubo de madeira, por onde sai a água das presas e tanques.
Arbusto silvestre, de fruto medicinal.
\section{Bucha-da-perna}
\begin{itemize}
\item {Grp. gram.:f.}
\end{itemize}
Barriga da perna.
(Cp. \textunderscore bucho\textunderscore ^1)
\section{Buchada}
\begin{itemize}
\item {Grp. gram.:f.}
\end{itemize}
\begin{itemize}
\item {Utilização:Fam.}
\end{itemize}
\begin{itemize}
\item {Proveniência:(De \textunderscore bucho\textunderscore )}
\end{itemize}
Estômago e vísceras de animaes.
Fartadela.
Estopada.
\section{Bucha-dos-paulistas}
\begin{itemize}
\item {Grp. gram.:f.}
\end{itemize}
\begin{itemize}
\item {Utilização:Bras}
\end{itemize}
O mesmo que \textunderscore fruta-dos-paulistas\textunderscore .
\section{Buche}
\begin{itemize}
\item {Grp. gram.:m.}
\end{itemize}
\begin{itemize}
\item {Utilização:Ant.}
\end{itemize}
Embarcação, em que os Hollandeses pescam o harenque.
(Cp. \textunderscore buque\textunderscore )
\section{Bucheira}
\begin{itemize}
\item {Grp. gram.:f.}
\end{itemize}
\begin{itemize}
\item {Utilização:Prov.}
\end{itemize}
\begin{itemize}
\item {Utilização:minh.}
\end{itemize}
Gancho de ferro, para fisgar polvo.
(Por \textunderscore bicheira\textunderscore ? Cp. \textunderscore bicheiro\textunderscore ^1)
\section{Bucheiro}
\begin{itemize}
\item {Grp. gram.:m.}
\end{itemize}
\begin{itemize}
\item {Utilização:Pop.}
\end{itemize}
\begin{itemize}
\item {Proveniência:(De \textunderscore bucha\textunderscore )}
\end{itemize}
Aquelle que tem por costume comer qualquer coisa, como pretexto para beber.
\section{Bucheiro}
\begin{itemize}
\item {Grp. gram.:m.}
\end{itemize}
\begin{itemize}
\item {Utilização:Bras. do N}
\end{itemize}
\begin{itemize}
\item {Proveniência:(De \textunderscore bucho\textunderscore )}
\end{itemize}
Aquelle que vende vísceras de animaes.
Tripeiro.
\section{Buchela}
\begin{itemize}
\item {Grp. gram.:f.}
\end{itemize}
Alicate, ou pequena tenaz, de que usam os ourives, esmaltadores, etc.
(Talvez do fr. \textunderscore bouchelle\textunderscore )
\section{Buchim}
\begin{itemize}
\item {Grp. gram.:m.}
\end{itemize}
\begin{itemize}
\item {Utilização:Prov.}
\end{itemize}
\begin{itemize}
\item {Utilização:alent.}
\end{itemize}
\begin{itemize}
\item {Proveniência:(De \textunderscore bucha\textunderscore )}
\end{itemize}
Revestimento de ferro nos buracos das rodas, onde se embebem as extremidades do eixo fixo de um carro.
\section{Buchinha}
\begin{itemize}
\item {Grp. gram.:f.}
\end{itemize}
Planta medicinal do Brasil.
\section{Bucho}
\begin{itemize}
\item {Grp. gram.:m.}
\end{itemize}
\begin{itemize}
\item {Utilização:Pop.}
\end{itemize}
\begin{itemize}
\item {Utilização:Pesc.}
\end{itemize}
\begin{itemize}
\item {Proveniência:(Do lat. \textunderscore musculus\textunderscore )}
\end{itemize}
Estômago dos animaes.
O estômago do homem.
Bojo.
Parte da armação de pesca de atum e sardinha, para onde entra o peixe depois de lhe passar a boca.
Parte do braço, desde o cotovelo ao ombro. Cp. \textunderscore bucha\textunderscore .
\section{Bucho}
\begin{itemize}
\item {Grp. gram.:m.}
\end{itemize}
\begin{itemize}
\item {Utilização:Prov.}
\end{itemize}
\begin{itemize}
\item {Utilização:trasm.}
\end{itemize}
O mesmo que \textunderscore bochacro\textunderscore .
\section{Buchu}
\begin{itemize}
\item {Grp. gram.:m.}
\end{itemize}
\begin{itemize}
\item {Utilização:Bras}
\end{itemize}
Planta rutácea, medicinal, (\textunderscore barosma crenata\textunderscore ).
\section{Buchuda}
\begin{itemize}
\item {Grp. gram.:adj. f.}
\end{itemize}
\begin{itemize}
\item {Utilização:Bras. do N}
\end{itemize}
\begin{itemize}
\item {Proveniência:(De \textunderscore bucho\textunderscore )}
\end{itemize}
Prenhe, grávida.
\section{Bucinador}
\begin{itemize}
\item {Grp. gram.:m.  e  adj.}
\end{itemize}
\begin{itemize}
\item {Proveniência:(Lat. \textunderscore buccinator\textunderscore )}
\end{itemize}
Diz-se de um músculo facial, que serve na mastigação e no sopro.
\section{Bucle}
\begin{itemize}
\item {Grp. gram.:m.}
\end{itemize}
O mesmo que \textunderscore bucre\textunderscore . Cf. Camillo, \textunderscore Corja\textunderscore , 224.
\section{Buco}
\begin{itemize}
\item {Grp. gram.:m.}
\end{itemize}
\begin{itemize}
\item {Utilização:Náut.}
\end{itemize}
\begin{itemize}
\item {Proveniência:(Do germ. \textunderscore buk\textunderscore )}
\end{itemize}
Bojo, capacidade, largura (de navio).
\section{Buço}
\begin{itemize}
\item {Grp. gram.:m.}
\end{itemize}
\begin{itemize}
\item {Utilização:T. da Bairrada}
\end{itemize}
Primeiros cabellos finos e curtos, que nascem no lábio superior do homem.
Pêlos finos e curtos no lábio superior de algumas mulheres.
Cachorro, não muito novo.
(Cp. cast. \textunderscore bozo\textunderscore )
\section{Bucólica}
\begin{itemize}
\item {Grp. gram.:f.}
\end{itemize}
\begin{itemize}
\item {Proveniência:(De \textunderscore bucólico\textunderscore )}
\end{itemize}
Poesia campestre; écloga.
\section{Bucólico}
\begin{itemize}
\item {Grp. gram.:adj.}
\end{itemize}
\begin{itemize}
\item {Proveniência:(Lat. \textunderscore bucolicus\textunderscore )}
\end{itemize}
Pastoril.
Campestre.
Simples; innocente.
Gracioso.
\section{Bucolismo}
\begin{itemize}
\item {Grp. gram.:m.}
\end{itemize}
O gênero da poesia bucólica.
\section{Bucolista}
\begin{itemize}
\item {Grp. gram.:m.}
\end{itemize}
Poéta, que escreve bucólicas.
\section{Bucolizar}
\begin{itemize}
\item {Grp. gram.:v. i.}
\end{itemize}
\begin{itemize}
\item {Utilização:Neol.}
\end{itemize}
Fazer bucólicas.
\section{Bucrânio}
\begin{itemize}
\item {Grp. gram.:m.}
\end{itemize}
\begin{itemize}
\item {Proveniência:(Do gr. \textunderscore bous\textunderscore  + \textunderscore kranion\textunderscore )}
\end{itemize}
Cabeça descarnada de boi, que servia de ornato em construcções gregas e romanas.
\section{Bucre}
\begin{itemize}
\item {Grp. gram.:m.}
\end{itemize}
\begin{itemize}
\item {Utilização:Ant.}
\end{itemize}
Anel, que se faz no cabello ou na cabelleira.
(Cast. \textunderscore bucle\textunderscore )
\section{Buçu}
\begin{itemize}
\item {Grp. gram.:m.}
\end{itemize}
\begin{itemize}
\item {Utilização:Bras}
\end{itemize}
\begin{itemize}
\item {Proveniência:(T. tupi)}
\end{itemize}
Espécie de palmeira.
\section{Bucurau}
\begin{itemize}
\item {Grp. gram.:m.}
\end{itemize}
\begin{itemize}
\item {Utilização:Bras}
\end{itemize}
Ave nocturna, que pia como o mocho.
\section{Budaísmo}
\begin{itemize}
\item {Grp. gram.:m.}
\end{itemize}
O mesmo que \textunderscore budismo\textunderscore . Cf. Latino, \textunderscore Or. da Corôa\textunderscore , XLI.
\section{Buddhaísmo}
\begin{itemize}
\item {Grp. gram.:m.}
\end{itemize}
O mesmo que \textunderscore buddhismo\textunderscore . Cf. Latino, \textunderscore Or. da Corôa\textunderscore , XLI.
\section{Búddhico}
\begin{itemize}
\item {Grp. gram.:adj.}
\end{itemize}
Relativo á religião de Budha.
\section{Buddhismo}
\begin{itemize}
\item {Grp. gram.:m.}
\end{itemize}
Religião philosóphica de Budha.
\section{Buddhista}
\begin{itemize}
\item {Grp. gram.:m.}
\end{itemize}
Sectário do buddismo.
\section{Búdico}
\begin{itemize}
\item {Grp. gram.:adj.}
\end{itemize}
Relativo á religião de Buda.
\section{Budismo}
\begin{itemize}
\item {Grp. gram.:m.}
\end{itemize}
Religião filosófica de Buda.
\section{Budista}
\begin{itemize}
\item {Grp. gram.:m.}
\end{itemize}
Sectário do budismo.
\section{Budleia}
\begin{itemize}
\item {Grp. gram.:f.}
\end{itemize}
Gênero de plantas escrofularineas.
\section{Budoar}
\begin{itemize}
\item {Grp. gram.:m.}
\end{itemize}
\begin{itemize}
\item {Utilização:Gal}
\end{itemize}
\begin{itemize}
\item {Proveniência:(Fr. \textunderscore boudoir\textunderscore )}
\end{itemize}
Toucador.
Quarto de vestir.
\section{Budsó}
\begin{itemize}
\item {Grp. gram.:m.}
\end{itemize}
Nome, que o budismo tomou no Japão.
\section{Bueira}
\begin{itemize}
\item {Grp. gram.:f.}
\end{itemize}
\begin{itemize}
\item {Utilização:Náut.}
\end{itemize}
Buraco, na parte inferior de uma embarcação, para a esgotar quando içada.
(Cp. \textunderscore bueiro\textunderscore )
\section{Bueiro}
\begin{itemize}
\item {Grp. gram.:m.}
\end{itemize}
\begin{itemize}
\item {Proveniência:(De \textunderscore bua\textunderscore )}
\end{itemize}
Canal ou buraco, feito numa embarcação, numa parede, etc., para dar esgôto a águas.
\section{Buenairense}
\begin{itemize}
\item {Grp. gram.:adj.}
\end{itemize}
\begin{itemize}
\item {Grp. gram.:M.}
\end{itemize}
Relativo a Buenos-Aires.
Habitante de Buenos-Aires.
\section{Bufa}
\begin{itemize}
\item {Grp. gram.:f.}
\end{itemize}
\begin{itemize}
\item {Utilização:Pleb.}
\end{itemize}
\begin{itemize}
\item {Proveniência:(De \textunderscore bufar\textunderscore )}
\end{itemize}
Ventosidade, que sái pelo ânus, sem estrépito.
\section{Bufalinheiro}
\begin{itemize}
\item {Grp. gram.:m.}
\end{itemize}
\begin{itemize}
\item {Utilização:Ant.}
\end{itemize}
O mesmo que \textunderscore bufarinheiro\textunderscore . Cf. G. Vicente, I, 173.
\section{Bufalino}
\begin{itemize}
\item {Grp. gram.:adj.}
\end{itemize}
Relativo ao búfalo.
\section{Búfalo}
\begin{itemize}
\item {Grp. gram.:m.}
\end{itemize}
\begin{itemize}
\item {Proveniência:(Lat. \textunderscore bubalum\textunderscore )}
\end{itemize}
Ruminante bovídeo.
Coiro de búfalo, já curtido.
\section{Búfano}
\begin{itemize}
\item {Grp. gram.:m.}
\end{itemize}
\begin{itemize}
\item {Utilização:Ant.}
\end{itemize}
O mesmo que \textunderscore búfalo\textunderscore . Cf. \textunderscore Eufrosina\textunderscore , 260.
\section{Bufão}
\begin{itemize}
\item {Grp. gram.:m.}
\end{itemize}
\begin{itemize}
\item {Utilização:Ant.}
\end{itemize}
\begin{itemize}
\item {Proveniência:(De \textunderscore bufo\textunderscore ^3)}
\end{itemize}
Fanfarrão.
Truão, bobo.
\section{Bufar}
\begin{itemize}
\item {Grp. gram.:v. i.}
\end{itemize}
\begin{itemize}
\item {Utilização:Bras. do N}
\end{itemize}
\begin{itemize}
\item {Proveniência:(De \textunderscore bufo\textunderscore ^1)}
\end{itemize}
Soprar, expellindo o ar com fôrça.
Bazofiar; alardear.
Expellir \textunderscore bufa\textunderscore .
\section{Búfara}
\begin{itemize}
\item {Grp. gram.:f.}
\end{itemize}
\begin{itemize}
\item {Utilização:Ant.}
\end{itemize}
O mesmo que \textunderscore búfalo\textunderscore . Cf. \textunderscore Ásia Sínica\textunderscore , 59.
\section{Bufarinha}
\begin{itemize}
\item {Grp. gram.:f.}
\end{itemize}
Cosméticos de pouco valor.
Bugiganga; quinquilharias.
\section{Bufarinheiro}
\begin{itemize}
\item {Grp. gram.:m.}
\end{itemize}
Vendedor ambulante de bufarinhas.
\section{Búfaro}
\begin{itemize}
\item {Grp. gram.:m.}
\end{itemize}
\begin{itemize}
\item {Utilização:Ant.}
\end{itemize}
O mesmo que \textunderscore búfalo\textunderscore . Cf. \textunderscore Ethiópia Or.\textunderscore , l. I, c. I.
\section{Bufarra}
\begin{itemize}
\item {Grp. gram.:f.}
\end{itemize}
\begin{itemize}
\item {Utilização:Prov.}
\end{itemize}
\begin{itemize}
\item {Utilização:trasm.}
\end{itemize}
O mesmo que \textunderscore nevoeiro\textunderscore .
\section{Bufas}
\begin{itemize}
\item {Grp. gram.:f. pl.}
\end{itemize}
\begin{itemize}
\item {Utilização:Pleb.}
\end{itemize}
O mesmo que [[suíças|suíça:1]].
\section{Bufeira}
\begin{itemize}
\item {Grp. gram.:f.}
\end{itemize}
\begin{itemize}
\item {Utilização:Prov.}
\end{itemize}
\begin{itemize}
\item {Utilização:beir.}
\end{itemize}
Espécie de chouriça.
(Talvez por \textunderscore bofeira\textunderscore , de \textunderscore bofe\textunderscore )
\section{Bufeira}
\begin{itemize}
\item {Grp. gram.:f.}
\end{itemize}
\begin{itemize}
\item {Utilização:Prov.}
\end{itemize}
\begin{itemize}
\item {Utilização:trasm.}
\end{itemize}
Farronca, embófia.
(Talvez de \textunderscore bufar\textunderscore )
\section{Bufete}
\begin{itemize}
\item {fónica:fê}
\end{itemize}
\begin{itemize}
\item {Grp. gram.:m.}
\end{itemize}
Aparador.
Mesa, em que se dispõem os objectos necessários a uma refeição.
Mesa, em que se servem refrescos, licôres, etc., aos convidados de um baile ou festa.
Compartimento, nas estações de caminho de ferro, em que os viajantes tomam refeição.
Secretária antiga, papeleira, de madeira preciosa.
(Cast. \textunderscore bufete\textunderscore )
\section{Bufido}
\begin{itemize}
\item {Grp. gram.:m.}
\end{itemize}
\begin{itemize}
\item {Proveniência:(De \textunderscore bufo\textunderscore )}
\end{itemize}
Som, que se produz, bufando.
\section{Bufo}
\begin{itemize}
\item {Grp. gram.:m.}
\end{itemize}
\begin{itemize}
\item {Utilização:Pleb.}
\end{itemize}
\begin{itemize}
\item {Proveniência:(T. onom.?)}
\end{itemize}
Acção de bufar.
\section{Bufo}
\begin{itemize}
\item {Grp. gram.:m.}
\end{itemize}
\begin{itemize}
\item {Utilização:Gír. de Lisbôa.}
\end{itemize}
\begin{itemize}
\item {Proveniência:(Do lat. \textunderscore bubo\textunderscore )}
\end{itemize}
Ave nocturna, semelhante á coruja; ujo, corujão (\textunderscore bubo maximus\textunderscore , Blenn.).
Homem avarento.
Homem misanthropo.
Polícia secreta.
Denunciante.
\section{Bufo}
\begin{itemize}
\item {Grp. gram.:m.}
\end{itemize}
\begin{itemize}
\item {Grp. gram.:Adj.}
\end{itemize}
\begin{itemize}
\item {Utilização:Neol.}
\end{itemize}
\begin{itemize}
\item {Proveniência:(It. \textunderscore buffo\textunderscore )}
\end{itemize}
O mesmo que \textunderscore bufão\textunderscore .
Burlesco.
Jovial.
\section{Bufo}
\begin{itemize}
\item {Grp. gram.:m.}
\end{itemize}
\begin{itemize}
\item {Proveniência:(Lat. \textunderscore bufo\textunderscore )}
\end{itemize}
Designação vulgar do gênero de batrácios, a que pertence o sapo.
\section{Bufon}
\begin{itemize}
\item {Grp. gram.:m.}
\end{itemize}
\begin{itemize}
\item {Utilização:Ant.}
\end{itemize}
O mesmo que \textunderscore bufarinheiro\textunderscore .
\section{Bufonaria}
\begin{itemize}
\item {Grp. gram.:f.}
\end{itemize}
Acção ou dito de bufão; chocarrice.
\section{Bufonear}
\begin{itemize}
\item {Grp. gram.:v. i.}
\end{itemize}
\begin{itemize}
\item {Grp. gram.:V. t.}
\end{itemize}
Fazer o papel de bufão.
Representar burlescamente.
\section{Bufotalina}
\begin{itemize}
\item {Grp. gram.:f.}
\end{itemize}
Substância activa, extrahida do sapo, e que é, por assim dizer, uma digitalina animal.
(Cf. \textunderscore bufo\textunderscore ^4)
\section{Bufotenina}
\begin{itemize}
\item {Grp. gram.:f.}
\end{itemize}
Substância, extrahida do sapo, mas menos activa que a bufotalina.
\section{Bufúrdio}
\begin{itemize}
\item {Grp. gram.:m.}
\end{itemize}
\begin{itemize}
\item {Utilização:Ant.}
\end{itemize}
Cavalhadas.
Justas ou torneios, por divertimento.
(B. lat. \textunderscore bufurdium\textunderscore )
\section{Bugacho}
\begin{itemize}
\item {Grp. gram.:m.}
\end{itemize}
\begin{itemize}
\item {Utilização:Prov.}
\end{itemize}
\begin{itemize}
\item {Utilização:beir.}
\end{itemize}
Pequeno novelo.
\section{Bugainvília}
\begin{itemize}
\item {Grp. gram.:f.}
\end{itemize}
O mesmo que \textunderscore buganvila\textunderscore .
\section{Bugainvíllia}
\begin{itemize}
\item {Grp. gram.:f.}
\end{itemize}
O mesmo que \textunderscore buganvilla\textunderscore .
\section{Bugalha}
\begin{itemize}
\item {Grp. gram.:f.}
\end{itemize}
\begin{itemize}
\item {Utilização:Prov.}
\end{itemize}
\begin{itemize}
\item {Utilização:beir.}
\end{itemize}
O mesmo que \textunderscore igualha\textunderscore : \textunderscore meta-se lá com os da sua bugalha\textunderscore .
\section{Bugalha}
\begin{itemize}
\item {Grp. gram.:m.}
\end{itemize}
O mesmo que \textunderscore bugalho\textunderscore .
\section{Bugalhal}
\begin{itemize}
\item {Grp. gram.:m.}
\end{itemize}
\begin{itemize}
\item {Grp. gram.:Adj.}
\end{itemize}
\begin{itemize}
\item {Utilização:Prov.}
\end{itemize}
\begin{itemize}
\item {Utilização:beir.}
\end{itemize}
\begin{itemize}
\item {Proveniência:(De \textunderscore bugalho\textunderscore )}
\end{itemize}
Casta de uva minhota.
Diz-se de uma espécie de figo branco, pequeno e redondo.
\section{Bugalhão}
\begin{itemize}
\item {Grp. gram.:m.}
\end{itemize}
\begin{itemize}
\item {Utilização:Gír.}
\end{itemize}
Valentão.
\section{Bugalhinha}
\begin{itemize}
\item {Grp. gram.:f.}
\end{itemize}
\begin{itemize}
\item {Proveniência:(De \textunderscore bugalha\textunderscore )}
\end{itemize}
Jôgo de rapazes, em que se empregam bugalhas de carvalheira.
\section{Bugalho}
\begin{itemize}
\item {Grp. gram.:m.}
\end{itemize}
\begin{itemize}
\item {Utilização:Prov.}
\end{itemize}
\begin{itemize}
\item {Utilização:alent.}
\end{itemize}
Noz de galha.
Galha.
Conta grande de rosário.
Qualquer objecto globular, semelhante á galha ou bugalho dos carvalhos.
Producto vegetal, que envolve a semente de uma espécie de cardo e que, depois de sêco, é facilmente inflammável e serve de isca.
(Por \textunderscore bagalho\textunderscore , de \textunderscore baga\textunderscore ?)
\section{Bugalhó}
\begin{itemize}
\item {Grp. gram.:m.}
\end{itemize}
\begin{itemize}
\item {Utilização:Prov.}
\end{itemize}
\begin{itemize}
\item {Utilização:minh.}
\end{itemize}
Espécie de planta, (\textunderscore ranunculus muricatus\textunderscore , Lin.), venenosa para os carneiros.
\section{Bugalhudo}
\begin{itemize}
\item {Grp. gram.:adj.}
\end{itemize}
Que tem fórma de bugalho.
Esbugalhado, grande, (falando-se dos olhos).
\section{Buganvila}
\begin{itemize}
\item {Grp. gram.:f.}
\end{itemize}
\begin{itemize}
\item {Proveniência:(De \textunderscore Bougainville\textunderscore , n. p. de um navegador fr.)}
\end{itemize}
Gênero de plantas nyctagíneas, trepadeiras.
\section{Buganvilla}
\begin{itemize}
\item {Grp. gram.:f.}
\end{itemize}
\begin{itemize}
\item {Proveniência:(De \textunderscore Bougainville\textunderscore , n. p. de um navegador fr.)}
\end{itemize}
Gênero de plantas nyctagíneas, trepadeiras.
\section{Bugarrém}
\begin{itemize}
\item {Grp. gram.:m.}
\end{itemize}
Casta de uva branca algarvia.
\section{Bugdió}
\begin{itemize}
\item {Grp. gram.:m.}
\end{itemize}
Enfeite, com que as bailadeiras indianas circumdam as orelhas.
\section{Bugés}
\begin{itemize}
\item {Grp. gram.:m. pl.}
\end{itemize}
Indígenas de norte do Brasil, nas margens do Juruá.
\section{Bugia}
\begin{itemize}
\item {Grp. gram.:f.}
\end{itemize}
\begin{itemize}
\item {Proveniência:(De \textunderscore Bugia\textunderscore , n. p. de uma cidade argelina)}
\end{itemize}
Pequena vela de cera.
Castiçal pequeno.
\section{Bugia}
\begin{itemize}
\item {Grp. gram.:f.}
\end{itemize}
A fêmea do macaco.
Linguagem de macaco:«\textunderscore vinham radando bem mona e bugia\textunderscore ». Garrett, \textunderscore Fábulas\textunderscore .
\section{Bugiar}
\begin{itemize}
\item {Grp. gram.:v. i.}
\end{itemize}
\begin{itemize}
\item {Proveniência:(De \textunderscore bugio\textunderscore )}
\end{itemize}
Fazer bugiarias.
\section{Bugiaria}
\begin{itemize}
\item {Grp. gram.:f.}
\end{itemize}
\begin{itemize}
\item {Proveniência:(De \textunderscore bugio\textunderscore )}
\end{itemize}
Modos de bugio; momice.
Coisa de pouco valor; bugiganga, bagatela.
\section{Bugiganga}
\begin{itemize}
\item {Utilização:Des.}
\end{itemize}
\begin{itemize}
\item {Utilização:Pesc.}
\end{itemize}
\begin{itemize}
\item {Proveniência:(De \textunderscore bugio\textunderscore )}
\end{itemize}
\textunderscore f.\textunderscore  (mais us. no \textunderscore pl.\textunderscore )
Bugiaria.
Bagatela.
Quinquilharias.
Dança de bugios.
Rede de cêrco.
\section{Buginíco}
\begin{itemize}
\item {Grp. gram.:m.}
\end{itemize}
\begin{itemize}
\item {Utilização:Des.}
\end{itemize}
\begin{itemize}
\item {Utilização:Fig.}
\end{itemize}
Pequeno bugio.
Rapazinho traquinas.
\section{Bugio}
\begin{itemize}
\item {Grp. gram.:m.}
\end{itemize}
\begin{itemize}
\item {Proveniência:(De \textunderscore Bugia\textunderscore , n. p. ?)}
\end{itemize}
Mono, espécie de macaco.
Bate-estacas, também conhecido por \textunderscore macaco\textunderscore .
\section{Buglossa}
\begin{itemize}
\item {Grp. gram.:f.}
\end{itemize}
\begin{itemize}
\item {Proveniência:(Lat. \textunderscore buglossa\textunderscore )}
\end{itemize}
Planta bórragínea, também chamada \textunderscore língua-de-vaca\textunderscore .
\section{Bugra}
\begin{itemize}
\item {Grp. gram.:f.}
\end{itemize}
\begin{itemize}
\item {Utilização:Bras}
\end{itemize}
Mulher, da raça dos bugres.
\section{Bugrada}
\begin{itemize}
\item {Grp. gram.:f.}
\end{itemize}
\begin{itemize}
\item {Utilização:Bras}
\end{itemize}
Porção ou malta de bugres.
Acção de bugre.
\section{Bugre}
\begin{itemize}
\item {Grp. gram.:m.}
\end{itemize}
\begin{itemize}
\item {Utilização:Bras}
\end{itemize}
\begin{itemize}
\item {Grp. gram.:Pl.}
\end{itemize}
\begin{itemize}
\item {Proveniência:(Do fr. \textunderscore bougre\textunderscore )}
\end{itemize}
Nome depreciativo, que se applica aos selvagens do Brasil.
Numerosa e bárbara nação indígena do Brasil, entre o rio Tieté e o Uruguai.
\section{Bugre}
\begin{itemize}
\item {Grp. gram.:m.}
\end{itemize}
O mesmo que \textunderscore bucre\textunderscore . Cf. Tolentino, III, 29.
\section{Bugueixo}
\begin{itemize}
\item {Grp. gram.:m.}
\end{itemize}
\begin{itemize}
\item {Utilização:T. de Avis}
\end{itemize}
Pedra pequena.
\section{Búgula}
\begin{itemize}
\item {Grp. gram.:f.}
\end{itemize}
\begin{itemize}
\item {Proveniência:(Fr. \textunderscore bugle\textunderscore )}
\end{itemize}
Planta labiada, chamada vulgarmente \textunderscore erva-de-San-Lourenço\textunderscore .
\section{Buílo}
\begin{itemize}
\item {Grp. gram.:m.}
\end{itemize}
Árvore do Congo.
\section{Buingelas}
\begin{itemize}
\item {Grp. gram.:m. pl.}
\end{itemize}
Uma das tríbos dos Landins, na África oriental.
\section{Buir}
\begin{itemize}
\item {Grp. gram.:v. t.}
\end{itemize}
Polir; alisar.
Gastar, friccionando.
(Alt. de \textunderscore poir\textunderscore )
\section{Buitra}
\begin{itemize}
\item {Grp. gram.:f.}
\end{itemize}
Peça nos antigos prelos, para dar firmeza á árvore da prensa.
\section{Buitre}
\begin{itemize}
\item {Grp. gram.:m.}
\end{itemize}
(V.abutre)
\section{Buitreira}
\begin{itemize}
\item {Grp. gram.:f.}
\end{itemize}
\begin{itemize}
\item {Utilização:Ant.}
\end{itemize}
\begin{itemize}
\item {Proveniência:(De \textunderscore buitre\textunderscore ?)}
\end{itemize}
Espécie de seteira, que tem um buraco redondo numa das extremidades.
\section{Buíz}
\begin{itemize}
\item {Grp. gram.:f.}
\end{itemize}
(V.boiz)
\section{Bujamé}
\begin{itemize}
\item {Grp. gram.:m.}
\end{itemize}
\begin{itemize}
\item {Proveniência:(T. afr.)}
\end{itemize}
Instrumento de sopro, usado pelos indígenas de Angola.
Filho de mulato e negra; mestiço.
\section{Bujão}
\begin{itemize}
\item {Grp. gram.:m.}
\end{itemize}
Bucha, rolha, cunha, com que se tapam bueiros ou fendas, a bordo.
(Talvez por \textunderscore buchão\textunderscore , de \textunderscore bucha\textunderscore )
\section{Bujarrona}
\begin{itemize}
\item {Grp. gram.:f.}
\end{itemize}
\begin{itemize}
\item {Utilização:Náut.}
\end{itemize}
\begin{itemize}
\item {Utilização:Fig.}
\end{itemize}
Vela triangular, que se iça á prôa.
Insulto, afronta: \textunderscore atirou-lhe uma bujarrona\textunderscore .
\section{Bujinga}
\begin{itemize}
\item {Grp. gram.:f.}
\end{itemize}
\begin{itemize}
\item {Utilização:Prov.}
\end{itemize}
\begin{itemize}
\item {Utilização:beir.}
\end{itemize}
O mesmo que \textunderscore monturo\textunderscore .
\section{Bul}
\begin{itemize}
\item {Grp. gram.:m.}
\end{itemize}
\begin{itemize}
\item {Utilização:Gír.}
\end{itemize}
O mesmo que \textunderscore ânus\textunderscore .
\section{Bula}
\begin{itemize}
\item {Grp. gram.:f.}
\end{itemize}
Árvore de Cabinda, própria para construcções.
\section{Bula}
\begin{itemize}
\item {Grp. gram.:f.}
\end{itemize}
Grande peixe africano. Cf. Serpa Pinto, I, 299.
\section{Bula}
\begin{itemize}
\item {Grp. gram.:f.}
\end{itemize}
\begin{itemize}
\item {Grp. gram.:Pl.}
\end{itemize}
\begin{itemize}
\item {Utilização:Fam.}
\end{itemize}
\begin{itemize}
\item {Proveniência:(Lat. \textunderscore bulla\textunderscore )}
\end{itemize}
Sêlo antigo, que tem pendente uma bóla de metal.
Carta patente, que contem decreto pontifício.
Capacidade, habilitações: \textunderscore deputado com poucas bulas\textunderscore .
Impostura.
Fanfarrice.
Mentira.
\section{Bular}
\begin{itemize}
\item {Grp. gram.:v. t.}
\end{itemize}
\begin{itemize}
\item {Utilização:Ant.}
\end{itemize}
Selar com \textunderscore bula\textunderscore ^3.
\section{Bulário}
\begin{itemize}
\item {Grp. gram.:m.}
\end{itemize}
Oficial, que copiava as bulas.
Colecção de bulas pontifícias.
(B. lat. \textunderscore bullarius\textunderscore )
\section{Bulbar}
\begin{itemize}
\item {Grp. gram.:adj.}
\end{itemize}
\begin{itemize}
\item {Proveniência:(Do lat. \textunderscore bulbus\textunderscore )}
\end{itemize}
Relativo a bolbo.
Que tem fórma de bolbo.
\section{Bulbífero}
\begin{itemize}
\item {Grp. gram.:adj.}
\end{itemize}
\begin{itemize}
\item {Proveniência:(Do lat. \textunderscore bulbus\textunderscore  + \textunderscore ferre\textunderscore )}
\end{itemize}
Que dá bolbos.
\section{Bulbiforme}
\begin{itemize}
\item {Grp. gram.:adj.}
\end{itemize}
\begin{itemize}
\item {Proveniência:(Do lat. \textunderscore bulbus\textunderscore  + \textunderscore forma\textunderscore )}
\end{itemize}
Que tem fórma de bolbo.
\section{Bulbíparo}
\begin{itemize}
\item {Grp. gram.:adj.}
\end{itemize}
\begin{itemize}
\item {Proveniência:(Do lat. \textunderscore bulbus\textunderscore  + \textunderscore parere\textunderscore )}
\end{itemize}
Que produz bolbos.
\section{Bulbo}
\begin{itemize}
\item {Grp. gram.:m.}
\end{itemize}
(V.bolbo)
\section{Bulboso}
\begin{itemize}
\item {Grp. gram.:adj.}
\end{itemize}
O mesmo que \textunderscore bolboso\textunderscore .
\section{Bulbul}
\begin{itemize}
\item {Grp. gram.:m.}
\end{itemize}
Ave de Dio.
\section{Búlbulo}
\begin{itemize}
\item {Grp. gram.:m.}
\end{itemize}
(Dem. de \textunderscore bulbo\textunderscore )
\section{Bulcão}
\begin{itemize}
\item {Grp. gram.:m.}
\end{itemize}
Nevoeiro espêsso, que preannuncia tempestade.
Redemoínho.
Nuvem de fumo.
Trevas.
(Alter. de \textunderscore vulcão\textunderscore ?)
\section{Buldogue}
\begin{itemize}
\item {Grp. gram.:m.}
\end{itemize}
\begin{itemize}
\item {Proveniência:(Do ingl. \textunderscore bulldog\textunderscore )}
\end{itemize}
Cão de fila, de raça inglesa.
\section{Buldra}
\begin{itemize}
\item {Grp. gram.:f.}
\end{itemize}
\begin{itemize}
\item {Utilização:Gír.}
\end{itemize}
Nádegas de mulher.
\section{Bule}
\begin{itemize}
\item {Grp. gram.:m.}
\end{itemize}
\begin{itemize}
\item {Proveniência:(T. mal.)}
\end{itemize}
Vaso, para serviço de chá.
\section{Bulebule}
\begin{itemize}
\item {Grp. gram.:m.}
\end{itemize}
O mesmo que \textunderscore bole-bole\textunderscore , planta.
Objecto, que está sempre em movimento.
(Cp. \textunderscore bole-bole\textunderscore )
\section{Buleiro}
\begin{itemize}
\item {Grp. gram.:m.}
\end{itemize}
Antigo empregado ecclesiástico, que distribuía a Bula da Cruzada.
(Cp. \textunderscore bullário\textunderscore )
\section{Bulevar}
\begin{itemize}
\item {Grp. gram.:m.}
\end{itemize}
\begin{itemize}
\item {Utilização:Neol.}
\end{itemize}
\begin{itemize}
\item {Proveniência:(Fr. \textunderscore boulevard\textunderscore )}
\end{itemize}
Rua larga e arborizada.
\section{Búlgaro}
\begin{itemize}
\item {Grp. gram.:adj.}
\end{itemize}
\begin{itemize}
\item {Grp. gram.:M.}
\end{itemize}
Relativo á Bulgária.
Habitante da Bulgária.
Dialecto do ramo esclavónico.
\section{Bulha}
\begin{itemize}
\item {Grp. gram.:f.}
\end{itemize}
Gritaria confusa.
Estrondo; barulho.
Desordem, motim.
(Cast. \textunderscore bulla\textunderscore )
\section{Bulhão}
\begin{itemize}
\item {Grp. gram.:m.  e  adj.}
\end{itemize}
(V.bulhento)
\section{Bulhão}
\begin{itemize}
\item {Grp. gram.:m.}
\end{itemize}
\begin{itemize}
\item {Utilização:Ant.}
\end{itemize}
\begin{itemize}
\item {Proveniência:(De \textunderscore bulla\textunderscore )}
\end{itemize}
Medalhão de oiro ou de prata.
\section{Bulhão}
\begin{itemize}
\item {Grp. gram.:m.}
\end{itemize}
Espécie de punhal antigo.
\section{Bulhar}
\begin{itemize}
\item {Grp. gram.:v. i.}
\end{itemize}
Fazer bulha.
Brigar.
Fazer motim.
\section{Bulhento}
\begin{itemize}
\item {Grp. gram.:m.  e  adj.}
\end{itemize}
O que costuma meter-se em bulhas; desordeiro.
\section{Bulho}
\begin{itemize}
\item {Grp. gram.:m.}
\end{itemize}
\begin{itemize}
\item {Utilização:Prov.}
\end{itemize}
\begin{itemize}
\item {Utilização:trasm.}
\end{itemize}
Espécie de chouriço, em que entram cartílagens e ossos tenros de porco, especialmente das costelas e do rabo.
\section{Buliceira}
\begin{itemize}
\item {Grp. gram.:f.}
\end{itemize}
\begin{itemize}
\item {Utilização:Prov.}
\end{itemize}
\begin{itemize}
\item {Utilização:extrem.}
\end{itemize}
Chuva miúda, chuvisco.
\section{Bulibulião}
\begin{itemize}
\item {Grp. gram.:m.}
\end{itemize}
\begin{itemize}
\item {Utilização:Ant.}
\end{itemize}
Imposto commercial, que se pagava em Malaca.
\section{Bulicio}
\begin{itemize}
\item {Grp. gram.:m.}
\end{itemize}
\begin{itemize}
\item {Proveniência:(De \textunderscore bulir\textunderscore )}
\end{itemize}
Agitação de coisas ou pessôas.
Murmúrio.
Rumor prolongado.
Motim.
Inquietação.
\section{Buliço}
\begin{itemize}
\item {Grp. gram.:m.}
\end{itemize}
\begin{itemize}
\item {Utilização:Ant.}
\end{itemize}
O mesmo que \textunderscore bulicio\textunderscore . Cf. Rui de Pina, \textunderscore Chrón. de Afonso V\textunderscore .(V.boliço)
\section{Buliçoso}
\begin{itemize}
\item {Grp. gram.:adj.}
\end{itemize}
\begin{itemize}
\item {Proveniência:(De \textunderscore bulício\textunderscore )}
\end{itemize}
Que se agita.
Inquieto; desenvolto.
Activo.
\section{Bulideira}
\begin{itemize}
\item {Grp. gram.:f.}
\end{itemize}
\begin{itemize}
\item {Utilização:Prov.}
\end{itemize}
\begin{itemize}
\item {Utilização:minh.}
\end{itemize}
\begin{itemize}
\item {Proveniência:(De \textunderscore bulir\textunderscore )}
\end{itemize}
Peixe pequenino, quási transparente, que, á beira-mar, fica em poças ou entre pedras que a maré deixou descobertas.
\section{Bulideiro}
\begin{itemize}
\item {Grp. gram.:adj.}
\end{itemize}
Que bole.
Que se agita: \textunderscore ondas bulideiras\textunderscore .
\section{Bulimento}
\begin{itemize}
\item {Grp. gram.:m.}
\end{itemize}
\begin{itemize}
\item {Utilização:Ant.}
\end{itemize}
\begin{itemize}
\item {Proveniência:(De \textunderscore bulir\textunderscore )}
\end{itemize}
Movimento de gente de guerra.
\section{Bulímia}
\begin{itemize}
\item {Grp. gram.:f.}
\end{itemize}
\begin{itemize}
\item {Proveniência:(Gr. \textunderscore boulímia\textunderscore )}
\end{itemize}
Fome insaciável, chamada vulgarmente \textunderscore fome canina\textunderscore .
\section{Bulimo}
\begin{itemize}
\item {Grp. gram.:m.}
\end{itemize}
Mollúsco gasterópode, de concha univalve e oblonga, (\textunderscore bolinas\textunderscore , Cuv.).
\section{Bulir}
\begin{itemize}
\item {Grp. gram.:v. i.}
\end{itemize}
\begin{itemize}
\item {Grp. gram.:V. t.}
\end{itemize}
\begin{itemize}
\item {Grp. gram.:V. p.}
\end{itemize}
\begin{itemize}
\item {Proveniência:(Lat. \textunderscore bullire\textunderscore )}
\end{itemize}
Mexer-se com pouca fôrça, agitar-se levemente, palpitar.
Tocar, mover levemente.
Mexer-se:«\textunderscore não ousavam bulir-se, como El-rei lhes havia dito\textunderscore ». \textunderscore Jornada de Áfr.\textunderscore , c. VI.
\section{Bulista}
\begin{itemize}
\item {Grp. gram.:m.}
\end{itemize}
Antigo empregado da cúria romana, encarregado do registo de bullas.
\section{Bulla}
\begin{itemize}
\item {Grp. gram.:f.}
\end{itemize}
\begin{itemize}
\item {Grp. gram.:Pl.}
\end{itemize}
\begin{itemize}
\item {Utilização:Fam.}
\end{itemize}
\begin{itemize}
\item {Proveniência:(Lat. \textunderscore bulla\textunderscore )}
\end{itemize}
Sêllo antigo, que tem pendente uma bóla de metal.
Carta patente, que contem decreto pontifício.
Capacidade, habilitações: \textunderscore deputado com poucas bullas\textunderscore .
Impostura.
Fanfarrice.
Mentira.
\section{Bullar}
\begin{itemize}
\item {Grp. gram.:v. t.}
\end{itemize}
\begin{itemize}
\item {Utilização:Ant.}
\end{itemize}
Sellar com \textunderscore bulla\textunderscore .
\section{Bullário}
\begin{itemize}
\item {Grp. gram.:m.}
\end{itemize}
Official, que copiava as bullas.
Collecção de bullas pontifícias.
(B. lat. \textunderscore bullarius\textunderscore )
\section{Bulleiro}
\begin{itemize}
\item {Grp. gram.:m.}
\end{itemize}
Antigo empregado ecclesiástico, que distribuía a Bulla da Cruzada.
(Cp. \textunderscore bullário\textunderscore )
\section{Bullista}
\begin{itemize}
\item {Grp. gram.:m.}
\end{itemize}
Antigo empregado da cúria romana, encarregado do registo de bullas.
\section{Bulr...}
(V.burl...)
\section{Bum!}
\begin{itemize}
\item {Grp. gram.:interj.}
\end{itemize}
Voz imitativa do tiro de peça. Cf. Rebello, \textunderscore Mocidade\textunderscore , I, 74.
\section{Bumba!}
\begin{itemize}
\item {Grp. gram.:interj.}
\end{itemize}
\begin{itemize}
\item {Proveniência:(T. onom.)}
\end{itemize}
(indicativa do estrondo, com que uma coisa cái ou bate, ou do movimento com que uma coisa se faz)
\section{Bumba-meu-boi}
\begin{itemize}
\item {Grp. gram.:m.}
\end{itemize}
\begin{itemize}
\item {Utilização:Bras}
\end{itemize}
Divertimento popular, que varia muito, segundo as localidades.
\section{Bumbar}
\begin{itemize}
\item {Grp. gram.:v. t.}
\end{itemize}
\begin{itemize}
\item {Utilização:Bras. do N}
\end{itemize}
\begin{itemize}
\item {Proveniência:(De \textunderscore bumba!\textunderscore )}
\end{itemize}
Sovar; espancar.
\section{Bumba-riachole}
\begin{itemize}
\item {Grp. gram.:m.}
\end{itemize}
Planta herbácea de Angola, no Golungo-Alto.
\section{Bumba-riala}
\begin{itemize}
\item {Grp. gram.:f.}
\end{itemize}
Planta herbácea e comestível do litoral de Angola.
\section{Búfaga}
\begin{itemize}
\item {Grp. gram.:f.}
\end{itemize}
\begin{itemize}
\item {Proveniência:(Do gr. \textunderscore bous\textunderscore  + \textunderscore phagein\textunderscore )}
\end{itemize}
Animal, que se alimenta das carraças adherentes á pelle dos bois.
\section{Bufónias}
\begin{itemize}
\item {Grp. gram.:f. Pl.}
\end{itemize}
Festas gregas, em honra de Júpiter.
\section{Buftalmia}
\begin{itemize}
\item {Grp. gram.:f.}
\end{itemize}
(Termo, indevidamente consignado em diccionários, em vez de \textunderscore buphthalmo\textunderscore , doença)
\section{Buftalmo}
\begin{itemize}
\item {Grp. gram.:m.}
\end{itemize}
\begin{itemize}
\item {Proveniência:(Do gr. \textunderscore bous\textunderscore  + \textunderscore ophthalmos\textunderscore )}
\end{itemize}
Dilatação do ôlho, produzida por hydropisia local.
Planta corymbífera.
\section{Bumbo}
\begin{itemize}
\item {Grp. gram.:m.}
\end{itemize}
\begin{itemize}
\item {Utilização:T. de Lisbôa}
\end{itemize}
O mesmo que \textunderscore bombo\textunderscore .
Selha alta, em que se expõe á venda o peixe, no mercado da lota.
\section{Bumbum}
\begin{itemize}
\item {Grp. gram.:m.}
\end{itemize}
\begin{itemize}
\item {Proveniência:(T. onom.)}
\end{itemize}
Estrondo; pancada repetida.
Som de zabumba.
\section{Buncho}
\begin{itemize}
\item {Grp. gram.:m.}
\end{itemize}
\begin{itemize}
\item {Utilização:Prov.}
\end{itemize}
\begin{itemize}
\item {Utilização:trasm.}
\end{itemize}
O mesmo que \textunderscore bucho\textunderscore ^2.
\section{Bunda}
\begin{itemize}
\item {Grp. gram.:adj. f.}
\end{itemize}
\begin{itemize}
\item {Grp. gram.:F.}
\end{itemize}
\begin{itemize}
\item {Utilização:Bras}
\end{itemize}
\begin{itemize}
\item {Grp. gram.:M. pl.}
\end{itemize}
\begin{itemize}
\item {Utilização:Bras. de San-Paulo}
\end{itemize}
Diz-se de uma língua africana, falada pelos indígenas de Angola.
Nádegas grandes.
Uma das tribus dos bantos de Angola.
O mesmo que [[nádegas|nádega]].
\section{Bundo}
\begin{itemize}
\item {Grp. gram.:m.}
\end{itemize}
\begin{itemize}
\item {Utilização:Ext.}
\end{itemize}
\begin{itemize}
\item {Utilização:Fig.}
\end{itemize}
A língua bunda, ou antes o quimbundo.
Negro de Angola.
Qualquer língua de negros.
Modo incorrecto ou errado de falar ou escrever.
\section{Bunga}
\begin{itemize}
\item {Grp. gram.:f.}
\end{itemize}
Árvore laurínea de San-Thomé.
\section{Bungama}
\begin{itemize}
\item {Grp. gram.:f.}
\end{itemize}
Arbusto africano, vivaz, de caule subterrâneo, fôlhas simples, glabras, e flôres miúdas, hermaphroditas.
\section{Bungo}
\begin{itemize}
\item {Grp. gram.:m.}
\end{itemize}
O mesmo que \textunderscore jibungo\textunderscore .
\section{Bunhar}
\begin{itemize}
\item {Grp. gram.:v. i.}
\end{itemize}
\begin{itemize}
\item {Utilização:T. de Barcelos}
\end{itemize}
\begin{itemize}
\item {Proveniência:(De \textunderscore bunho\textunderscore )}
\end{itemize}
Fazer bunhedos.
\section{Bunhedo}
\begin{itemize}
\item {fónica:nhê}
\end{itemize}
\begin{itemize}
\item {Grp. gram.:m.}
\end{itemize}
\begin{itemize}
\item {Utilização:T. de Barcelos}
\end{itemize}
\begin{itemize}
\item {Proveniência:(De \textunderscore bunho\textunderscore )}
\end{itemize}
Espécie de engenhoca, feita por crianças.
\section{Bunheiro}
\begin{itemize}
\item {Grp. gram.:m.}
\end{itemize}
\begin{itemize}
\item {Proveniência:(De \textunderscore bunho\textunderscore )}
\end{itemize}
Official, que faz obras de bunho, como esteiras, cadeiras, etc.
\section{Bunho}
\begin{itemize}
\item {Grp. gram.:m.}
\end{itemize}
\begin{itemize}
\item {Utilização:Prov.}
\end{itemize}
\begin{itemize}
\item {Utilização:alent.}
\end{itemize}
\begin{itemize}
\item {Utilização:T. de Aveiro}
\end{itemize}
\begin{itemize}
\item {Utilização:T. de Barcelos}
\end{itemize}
O mesmo que \textunderscore tabúa\textunderscore .
Espécie de junco, com que se tapam as mêdas do sal nas marinhas, para as resguardar contra a chuva.
O mesmo que \textunderscore bunhedo\textunderscore .
\section{Bunídeas}
\begin{itemize}
\item {Grp. gram.:f. pl.}
\end{itemize}
\begin{itemize}
\item {Proveniência:(De \textunderscore búnio\textunderscore )}
\end{itemize}
Tríbo de crucíferas, na classificação de De-Candolle.
\section{Búnio}
\begin{itemize}
\item {Grp. gram.:m.}
\end{itemize}
\begin{itemize}
\item {Proveniência:(Do gr. \textunderscore bounos\textunderscore , collina)}
\end{itemize}
Gênero de plantas umbellíferas.
\section{Búphaga}
\begin{itemize}
\item {Grp. gram.:f.}
\end{itemize}
\begin{itemize}
\item {Proveniência:(Do gr. \textunderscore bous\textunderscore  + \textunderscore phagein\textunderscore )}
\end{itemize}
Animal, que se alimenta das carraças adherentes á pelle dos bois.
\section{Buphónias}
\begin{itemize}
\item {Grp. gram.:f. Pl.}
\end{itemize}
Festas gregas, em honra de Júpiter.
\section{Buphthalmia}
\begin{itemize}
\item {Grp. gram.:f.}
\end{itemize}
(Termo, indevidamente consignado em diccionários, em vez de \textunderscore buphthalmo\textunderscore , doença)
\section{Buphthalmo}
\begin{itemize}
\item {Grp. gram.:m.}
\end{itemize}
\begin{itemize}
\item {Proveniência:(Do gr. \textunderscore bous\textunderscore  + \textunderscore ophthalmos\textunderscore )}
\end{itemize}
Dilatação do ôlho, produzida por hydropisia local.
Planta corymbífera.
\section{Bupreste}
\begin{itemize}
\item {Grp. gram.:m.}
\end{itemize}
\begin{itemize}
\item {Proveniência:(Gr. \textunderscore bouprestis\textunderscore )}
\end{itemize}
Insecto coleóptero de côres cambiantes.
\section{Buque}
\begin{itemize}
\item {Grp. gram.:m.}
\end{itemize}
Embarcação pequena, empregada especialmente em coadjuvar os galeões de pesca.
(B. lat. \textunderscore buca\textunderscore , tronco)
\section{Buqueiro}
\begin{itemize}
\item {Grp. gram.:m.}
\end{itemize}
\begin{itemize}
\item {Utilização:Prov.}
\end{itemize}
\begin{itemize}
\item {Utilização:trasm.}
\end{itemize}
\begin{itemize}
\item {Proveniência:(De \textunderscore buque\textunderscore )}
\end{itemize}
Barco grande.
\section{Buquete}
\begin{itemize}
\item {fónica:quê}
\end{itemize}
\begin{itemize}
\item {Grp. gram.:m.}
\end{itemize}
\begin{itemize}
\item {Utilização:Agr.}
\end{itemize}
Parte das máquinas de debulhar? Cf. \textunderscore Gazeta dos Lavr.\textunderscore , I, 17.
\section{Buraca}
\begin{itemize}
\item {Grp. gram.:f.}
\end{itemize}
\begin{itemize}
\item {Utilização:Prov.}
\end{itemize}
Grande buraco.
Gruta, caverna.
(Colhido em Turquel)
\section{Buraçanga}
\begin{itemize}
\item {Grp. gram.:f.}
\end{itemize}
\begin{itemize}
\item {Utilização:Bras}
\end{itemize}
\begin{itemize}
\item {Proveniência:(T. tupi)}
\end{itemize}
Pedaço cylíndrico de madeira, para bater a roupa que se lava.
\section{Buraco}
\begin{itemize}
\item {Grp. gram.:m.}
\end{itemize}
\begin{itemize}
\item {Proveniência:(Do ant. alt. al. \textunderscore bora\textunderscore )}
\end{itemize}
Orifício, pequena abertura.
Cova, barranco.
Toca, pequena casa.
Lacuna, falta: \textunderscore com aquelles poucos ganhos sempre se taparam uns buracos\textunderscore .
\section{Buramo}
\begin{itemize}
\item {Grp. gram.:m.}
\end{itemize}
Uma das línguas da Guiné.
\section{Buranhém}
\begin{itemize}
\item {Grp. gram.:m.}
\end{itemize}
Alta árvore sapotácea do Brasil.
\section{Buraqueira}
\begin{itemize}
\item {Grp. gram.:f.}
\end{itemize}
\begin{itemize}
\item {Proveniência:(De \textunderscore buraco\textunderscore )}
\end{itemize}
Espécie de codorniz do Brasil.
\section{Burara}
\begin{itemize}
\item {Grp. gram.:f.}
\end{itemize}
\begin{itemize}
\item {Utilização:Bras}
\end{itemize}
Qualquer árvore derrubada que, atravessando-se no caminho, impede o trânsito.
\section{Burarema}
\begin{itemize}
\item {Grp. gram.:f.}
\end{itemize}
Árvore do Brasil.
\section{Burato}
\begin{itemize}
\item {Grp. gram.:m.}
\end{itemize}
\begin{itemize}
\item {Proveniência:(Fr. \textunderscore burat\textunderscore )}
\end{itemize}
Antigo estôfo transparente.
\section{Burço}
\begin{itemize}
\item {Grp. gram.:m.}
\end{itemize}
\begin{itemize}
\item {Utilização:T. do distrito de Portalegre}
\end{itemize}
O mesmo que \textunderscore calaburço\textunderscore .
\section{Burdo}
\begin{itemize}
\item {Grp. gram.:adj.}
\end{itemize}
Grosseiro; de má qualidade.
(Cast. \textunderscore burdo\textunderscore )
\section{Burdo}
\begin{itemize}
\item {Grp. gram.:m.}
\end{itemize}
\begin{itemize}
\item {Utilização:Prov.}
\end{itemize}
\begin{itemize}
\item {Utilização:alent.}
\end{itemize}
Quéda de água em barranco ou ribeira.
\section{Burel}
\begin{itemize}
\item {Grp. gram.:m.}
\end{itemize}
\begin{itemize}
\item {Utilização:Fig.}
\end{itemize}
\begin{itemize}
\item {Proveniência:(It. \textunderscore burello\textunderscore . Cp. lat. \textunderscore burrus\textunderscore )}
\end{itemize}
Tecido de lan simples.
Hábito de frade ou de freira.
Luto.
\section{Burela}
\begin{itemize}
\item {Grp. gram.:f.}
\end{itemize}
\begin{itemize}
\item {Utilização:Heráld.}
\end{itemize}
Faixa estreita e repetida, no campo do escudo.
\section{Burelado}
\begin{itemize}
\item {Grp. gram.:adj.}
\end{itemize}
\begin{itemize}
\item {Utilização:Heráld.}
\end{itemize}
Diz-se do escudo, quando as burelas são da mesma largura que o espaço que as separa.
\section{Burelina}
\begin{itemize}
\item {Grp. gram.:f.}
\end{itemize}
\begin{itemize}
\item {Proveniência:(De \textunderscore burel\textunderscore )}
\end{itemize}
Pano de lan, menos grosso que o burel.
\section{Burgalês}
\begin{itemize}
\item {Grp. gram.:m.}
\end{itemize}
\begin{itemize}
\item {Proveniência:(De \textunderscore Burgos\textunderscore , n. p.)}
\end{itemize}
Moéda antiga, equivalente a 4 mealhas.
\section{Burgalhão}
\begin{itemize}
\item {Grp. gram.:m.}
\end{itemize}
\begin{itemize}
\item {Proveniência:(De \textunderscore burgau\textunderscore )}
\end{itemize}
Monte de cascalho, conchas e areia, debaixo de água.
\section{Burgar}
\begin{itemize}
\item {Grp. gram.:v. t.  e  i.}
\end{itemize}
\begin{itemize}
\item {Utilização:Prov.}
\end{itemize}
\begin{itemize}
\item {Utilização:minh.}
\end{itemize}
Cavar terras.
\section{Burgau}
\begin{itemize}
\item {Grp. gram.:m.}
\end{itemize}
\begin{itemize}
\item {Proveniência:(Fr. \textunderscore burgau\textunderscore )}
\end{itemize}
Mollusco gasterópode, de concha univalve; burrié.
Conchas, que se diffundem pelas praias.
Cascalho.
Burgaudina.
\section{Burgaudina}
\begin{itemize}
\item {Grp. gram.:f.}
\end{itemize}
Nácar, que se extrái do burgau.
\section{Burgenheira}
\begin{itemize}
\item {Grp. gram.:f.}
\end{itemize}
Planta da serra de Sintra.
\section{Burgo}
\begin{itemize}
\item {Grp. gram.:m.}
\end{itemize}
\begin{itemize}
\item {Proveniência:(Lat. \textunderscore burgus\textunderscore )}
\end{itemize}
Arrabalde de cidade.
Aldeia.
Paço.
Villa.
Mosteiro.
Casa nobre.
\section{Burgo}
\begin{itemize}
\item {Grp. gram.:m.}
\end{itemize}
Cascalho.
Pequeno seixo.
(Cp. \textunderscore burgau\textunderscore )
\section{Burgo}
\begin{itemize}
\item {Grp. gram.:m.}
\end{itemize}
\begin{itemize}
\item {Utilização:Prov.}
\end{itemize}
\begin{itemize}
\item {Utilização:alent.}
\end{itemize}
Doença das azinheiras.
\section{Burgo}
\begin{itemize}
\item {Grp. gram.:m.}
\end{itemize}
Dá-se êste nome, nas vizinhanças de Portalegre, a uma erva que se applica contra a raiva canina.
\section{Burgó}
\begin{itemize}
\item {Grp. gram.:m.}
\end{itemize}
Espécie de caracol das Antilhas.
(Cp. \textunderscore burgau\textunderscore )
\section{Burgomestre}
\begin{itemize}
\item {Grp. gram.:m.}
\end{itemize}
\begin{itemize}
\item {Proveniência:(Al. \textunderscore burgmeister\textunderscore )}
\end{itemize}
Magistrado principal, em alguns municípios da Alemanha, Bélgica, etc.
\section{Burgravado}
\begin{itemize}
\item {Grp. gram.:m.}
\end{itemize}
Jurisdicção, dignidade de \textunderscore burgrave\textunderscore .
\section{Burgrave}
\begin{itemize}
\item {Grp. gram.:m.}
\end{itemize}
\begin{itemize}
\item {Proveniência:(Al. \textunderscore burggraf\textunderscore )}
\end{itemize}
Antigo dignitário, na Alemanha.
\section{Burguês}
\begin{itemize}
\item {Grp. gram.:m.}
\end{itemize}
\begin{itemize}
\item {Grp. gram.:Adj.}
\end{itemize}
Aquelle que habita num burgo.
Homem da classe média.
Homem pouco delicado.
Que diz respeito a burgo.
Ordinário, trivial.
(B. lat. \textunderscore burgensis\textunderscore )
\section{Burguesia}
\begin{itemize}
\item {Grp. gram.:f.}
\end{itemize}
Qualidade de burguês.
A classe média, na sociedade.
\section{Burguesismo}
\begin{itemize}
\item {Grp. gram.:m.}
\end{itemize}
O mesmo que \textunderscore burguesia\textunderscore .
\section{Burguêsmente}
\begin{itemize}
\item {Grp. gram.:adv.}
\end{itemize}
A maneira de burguês.
De modo burguês.
\section{Burguete}
\begin{itemize}
\item {fónica:guê}
\end{itemize}
\begin{itemize}
\item {Grp. gram.:m.}
\end{itemize}
\begin{itemize}
\item {Utilização:Prov.}
\end{itemize}
\begin{itemize}
\item {Utilização:trasm.}
\end{itemize}
\begin{itemize}
\item {Proveniência:(De \textunderscore burgo\textunderscore ?)}
\end{itemize}
Pequeno cerrado, nas arribas, entre fragas.
\section{Burguinhões}
\begin{itemize}
\item {Grp. gram.:m. pl.}
\end{itemize}
\begin{itemize}
\item {Proveniência:(Lat. \textunderscore burgundiones\textunderscore )}
\end{itemize}
Povo germânico, que se estabeleceu a léste da Gállia.
\section{Burgundos}
\begin{itemize}
\item {Grp. gram.:m. pl.}
\end{itemize}
Povo da Espanha gótica. Cf. Herculano, \textunderscore Eurico\textunderscore , c. IV.
\section{Buri}
\begin{itemize}
\item {Grp. gram.:m.}
\end{itemize}
\begin{itemize}
\item {Utilização:Bras}
\end{itemize}
Espécie de palmeira da Baía.
\section{Buriata}
\begin{itemize}
\item {Grp. gram.:m.}
\end{itemize}
Lingua do ramo mongol e do uralo-altaico.
\section{Buril}
\begin{itemize}
\item {Grp. gram.:m.}
\end{itemize}
\begin{itemize}
\item {Utilização:Fig.}
\end{itemize}
\begin{itemize}
\item {Proveniência:(Do germ. \textunderscore bora\textunderscore )}
\end{itemize}
Instrumento de aço, para uso de gravadores.
Instrumento semelhante, para lavrar pedra.
Arte, modo de gravar.
Estilo acerado.
Constellação austral.
\section{Burilada}
\begin{itemize}
\item {Grp. gram.:f.}
\end{itemize}
Golpe de buril.
\section{Burilador}
\begin{itemize}
\item {Grp. gram.:m.  e  adj.}
\end{itemize}
Aquelle que burila.
\section{Burilar}
\begin{itemize}
\item {Grp. gram.:v. t.}
\end{itemize}
Lavrar com buril.
\section{Buriqui}
\begin{itemize}
\item {Grp. gram.:m.}
\end{itemize}
Espécie de macaco do Brasil.
\section{Buriti}
\begin{itemize}
\item {Grp. gram.:m.}
\end{itemize}
Arvore brasileira, da fam. das palmáceas; o mesmo que \textunderscore buritizeiro\textunderscore .
\section{Buritirana}
\begin{itemize}
\item {Grp. gram.:f.}
\end{itemize}
Variedade de buriti.
\section{Buritiz}
\begin{itemize}
\item {Grp. gram.:m.}
\end{itemize}
(V.buriti)
\section{Buritizada}
\begin{itemize}
\item {Grp. gram.:f.}
\end{itemize}
\begin{itemize}
\item {Utilização:Bras}
\end{itemize}
\begin{itemize}
\item {Proveniência:(De \textunderscore buritiz\textunderscore )}
\end{itemize}
Doce, feito da fruta do buriti.
\section{Buritizal}
\begin{itemize}
\item {Grp. gram.:m.}
\end{itemize}
Mata de buritis.
\section{Buritizeiro}
\begin{itemize}
\item {Grp. gram.:m.}
\end{itemize}
Espécie de palmeira, (\textunderscore mauritia vivifera\textunderscore , Mart.), de cujas flôres e pedúnculos se extrai uma seiva, de que se faz uma bebida alcoólica.
\section{Burjaca}
\begin{itemize}
\item {Grp. gram.:f.}
\end{itemize}
\begin{itemize}
\item {Utilização:Pop.}
\end{itemize}
Saco de coiro, em que os caldeireiros ambulantes levam utensílios miúdos.
Jaquetão largo e comprido.
(Cast. \textunderscore burjaca\textunderscore )
\section{Burjassote}
\begin{itemize}
\item {Grp. gram.:m.}
\end{itemize}
O mesmo ou melhor que \textunderscore berjaçote\textunderscore .
\section{Burla}
\begin{itemize}
\item {Grp. gram.:f.}
\end{itemize}
Acto de burlar; fraude.
Motejo, zombaria.
(B. lat. \textunderscore burula\textunderscore )
\section{Burlador}
\begin{itemize}
\item {Grp. gram.:m.  e  adj.}
\end{itemize}
O que burla.
\section{Burlante}
\begin{itemize}
\item {Grp. gram.:adj.}
\end{itemize}
\begin{itemize}
\item {Utilização:Prov.}
\end{itemize}
\begin{itemize}
\item {Utilização:minh.}
\end{itemize}
Que queima.
Ardente, abrasador.
(Por \textunderscore brulante\textunderscore . Cp. [[brulotes|brulote]])
\section{Burlantim}
\begin{itemize}
\item {Grp. gram.:m.}
\end{itemize}
\begin{itemize}
\item {Utilização:Bras}
\end{itemize}
Funâmbulo; saltimbanco.
(Alter. de \textunderscore volatim\textunderscore )
\section{Burlão}
\begin{itemize}
\item {Grp. gram.:m.  e  adj.}
\end{itemize}
O mesmo que \textunderscore burlador\textunderscore .
\section{Burlar}
\begin{itemize}
\item {Grp. gram.:v. t.}
\end{itemize}
\begin{itemize}
\item {Proveniência:(De \textunderscore burla\textunderscore )}
\end{itemize}
Enganar; ludibriar.
Motejar de.
\section{Burlaria}
\begin{itemize}
\item {Grp. gram.:f.}
\end{itemize}
(V.burla)
\section{Burlequeador}
\begin{itemize}
\item {Grp. gram.:m.}
\end{itemize}
\begin{itemize}
\item {Utilização:Bras}
\end{itemize}
Aquelle que burlequeia; vadio.
\section{Burlequear}
\begin{itemize}
\item {Grp. gram.:v. i.}
\end{itemize}
\begin{itemize}
\item {Utilização:Bras}
\end{itemize}
Passear á toa.
Vadiar.
\section{Burlescamente}
\begin{itemize}
\item {Grp. gram.:adv.}
\end{itemize}
De modo burlesco.
\section{Burlesco}
\begin{itemize}
\item {fónica:lês}
\end{itemize}
\begin{itemize}
\item {Grp. gram.:adj.}
\end{itemize}
\begin{itemize}
\item {Proveniência:(It. \textunderscore burlesco\textunderscore )}
\end{itemize}
Ridículo; grotesco.
Zombeteiro.
\section{Burlesquear}
\begin{itemize}
\item {Grp. gram.:v. i.}
\end{itemize}
\begin{itemize}
\item {Proveniência:(De \textunderscore burlesco\textunderscore )}
\end{itemize}
Usar de modos ridículos; falar burlescamente.
\section{Burleta}
\begin{itemize}
\item {fónica:lê}
\end{itemize}
\begin{itemize}
\item {Grp. gram.:f.}
\end{itemize}
\begin{itemize}
\item {Proveniência:(It. \textunderscore burletta\textunderscore )}
\end{itemize}
Ligeira representação cómica.
\section{Burlina}
\begin{itemize}
\item {Grp. gram.:f.}
\end{itemize}
(Fórma pop. e contrahida de \textunderscore burelina\textunderscore )
\section{Burlingtónia}
\begin{itemize}
\item {Grp. gram.:f.}
\end{itemize}
\begin{itemize}
\item {Proveniência:(Do n. da Condessa de \textunderscore Burlington\textunderscore )}
\end{itemize}
Gênero de orchídeas.
\section{Burlista}
\begin{itemize}
\item {Grp. gram.:m.  e  adj.}
\end{itemize}
O mesmo que \textunderscore burlador\textunderscore .
\section{Burlosamente}
\begin{itemize}
\item {Grp. gram.:adv.}
\end{itemize}
De modo burloso.
\section{Burloso}
\begin{itemize}
\item {Grp. gram.:adj.}
\end{itemize}
\begin{itemize}
\item {Proveniência:(De \textunderscore burla\textunderscore )}
\end{itemize}
Que contém burla.
Que é burlão.
\section{Burnir}
\textunderscore v. t.\textunderscore  (e der.)
(Fórma pop. de \textunderscore brunir\textunderscore , etc.)
\section{Burnu}
\begin{itemize}
\item {Grp. gram.:m.}
\end{itemize}
O mesmo que [[albornós|albornó]].
\section{Burnus}
\begin{itemize}
\item {Grp. gram.:m.}
\end{itemize}
O mesmo que \textunderscore burnu\textunderscore . Cf. Camillo, \textunderscore Noites\textunderscore , V, 35.
\section{Burocracia}
\begin{itemize}
\item {Grp. gram.:f.}
\end{itemize}
\begin{itemize}
\item {Utilização:Neol.}
\end{itemize}
\begin{itemize}
\item {Proveniência:(Fr. \textunderscore bureaucratie\textunderscore )}
\end{itemize}
Influência dos empregados públicos.
A classe dos funccionários públicos, especialmente dos das secretarias de Estado.
\section{Burocrata}
\begin{itemize}
\item {Grp. gram.:m.}
\end{itemize}
\begin{itemize}
\item {Utilização:Neol.}
\end{itemize}
\begin{itemize}
\item {Proveniência:(Fr. \textunderscore bureaucrate\textunderscore )}
\end{itemize}
Empregado público.
Aquelle que tem influência nas repartições públicas, fazendo parte do pessoal dellas.
\section{Burocraticamente}
\begin{itemize}
\item {Grp. gram.:adv.}
\end{itemize}
\begin{itemize}
\item {Proveniência:(De \textunderscore burocrático\textunderscore )}
\end{itemize}
Á maneira dos burocratas.
\section{Burocrático}
\begin{itemize}
\item {Grp. gram.:adj.}
\end{itemize}
\begin{itemize}
\item {Proveniência:(De \textunderscore burocrata\textunderscore )}
\end{itemize}
Relativo á burocracia: \textunderscore regulamentos burocráticos\textunderscore .
Próprio de burocrata.
\section{Burocratizar}
\begin{itemize}
\item {Grp. gram.:v. t.}
\end{itemize}
Dar feição ou carácter burocrático a.
\section{Burpilheiro}
\begin{itemize}
\item {Grp. gram.:m.}
\end{itemize}
\begin{itemize}
\item {Utilização:Prov.}
\end{itemize}
\begin{itemize}
\item {Utilização:Minh.}
\end{itemize}
Aquelle que tem pouca roupa.
Maltrapilho.
\section{Burquel}
\begin{itemize}
\item {Grp. gram.:m.}
\end{itemize}
O mesmo que \textunderscore broquel\textunderscore ?«\textunderscore Bem vos days nos burqueis\textunderscore ». \textunderscore Aulegrafia\textunderscore , 92.
\section{Burra}
\begin{itemize}
\item {Grp. gram.:f.}
\end{itemize}
\begin{itemize}
\item {Utilização:Náut.}
\end{itemize}
\begin{itemize}
\item {Utilização:Prov.}
\end{itemize}
\begin{itemize}
\item {Utilização:Prov.}
\end{itemize}
\begin{itemize}
\item {Utilização:alent.}
\end{itemize}
\begin{itemize}
\item {Utilização:Prov.}
\end{itemize}
Fêmea do burro.
Cofre, para guardar dinheiro.
Cabo de mezena.
Engenho para tirar água dos poços ou rios.
Cavallete, formado por um tronco, que descansa no chão uma das extremidades, tem na outra dois pés, e serve aos serradores para suster a madeira que estão serrando.
Escadote, usado em estabelecimentos, adegas, etc.
Porção de terreno, que se desprende de uma ribanceira, por effeito da chuva.
\section{Burraca}
\begin{itemize}
\item {Grp. gram.:f.}
\end{itemize}
Espécie de jôgo popular.
\section{Burrada}
\begin{itemize}
\item {Grp. gram.:f.}
\end{itemize}
\begin{itemize}
\item {Utilização:Pleb.}
\end{itemize}
Ajuntamento de burros.
Asneira.
\section{Burral}
\begin{itemize}
\item {Grp. gram.:m.}
\end{itemize}
Variedade de uva.
\section{Burrana}
\begin{itemize}
\item {Grp. gram.:m.}
\end{itemize}
\begin{itemize}
\item {Utilização:T. da Bairrada}
\end{itemize}
Paspalhão.
Aquelle que consente quanto lhe queiram fazer.
\section{Burranca}
\begin{itemize}
\item {Grp. gram.:f.}
\end{itemize}
\begin{itemize}
\item {Utilização:Prov.}
\end{itemize}
\begin{itemize}
\item {Utilização:trasm.}
\end{itemize}
Burra fraca.
\section{Burrancas}
\begin{itemize}
\item {Grp. gram.:m.}
\end{itemize}
\begin{itemize}
\item {Utilização:Pop.}
\end{itemize}
O mesmo que \textunderscore burrana\textunderscore .
\section{Burrão}
\begin{itemize}
\item {Grp. gram.:m.}
\end{itemize}
\begin{itemize}
\item {Proveniência:(De \textunderscore burro\textunderscore ^1)}
\end{itemize}
Casmurrice, amúo.
\section{Burrar}
\begin{itemize}
\item {Grp. gram.:v. t.}
\end{itemize}
\begin{itemize}
\item {Utilização:Prov.}
\end{itemize}
Desprender-se de uma ribanceira (uma porção de terreno), por effeito da chuva.
(Cp. \textunderscore burra\textunderscore )
\section{Burreca}
\begin{itemize}
\item {Grp. gram.:f.}
\end{itemize}
\begin{itemize}
\item {Utilização:Ant.}
\end{itemize}
\begin{itemize}
\item {Proveniência:(De \textunderscore burro\textunderscore ^1?)}
\end{itemize}
Carcunda.
\section{Burrela}
\begin{itemize}
\item {Grp. gram.:m.}
\end{itemize}
\begin{itemize}
\item {Utilização:Des.}
\end{itemize}
Termo injurioso:«\textunderscore burrela pançudo sejas tu\textunderscore ». Arn. Gama, \textunderscore Última Dona\textunderscore , 10.
\section{Burrequeiro}
\begin{itemize}
\item {Grp. gram.:m.  e  adj.}
\end{itemize}
\begin{itemize}
\item {Utilização:Ant.}
\end{itemize}
Aquelle que tem burreca.
\section{Burrica}
\begin{itemize}
\item {Grp. gram.:f.}
\end{itemize}
(Dem. de \textunderscore burra\textunderscore )
\section{Burricada}
\begin{itemize}
\item {Grp. gram.:f.}
\end{itemize}
\begin{itemize}
\item {Proveniência:(De \textunderscore burrico\textunderscore )}
\end{itemize}
Ajuntamento de burros.
Multidão de pessôas que montam burros.
Asneira.
Disparate.
\section{Burrical}
\begin{itemize}
\item {Grp. gram.:adj.}
\end{itemize}
\begin{itemize}
\item {Proveniência:(De \textunderscore burrico\textunderscore )}
\end{itemize}
Que diz respeito a burros.
Asnático.
Estúpido.
\section{Burrice}
\begin{itemize}
\item {Grp. gram.:f.}
\end{itemize}
\begin{itemize}
\item {Proveniência:(De \textunderscore burro\textunderscore ^1)}
\end{itemize}
Asneira.
Casmurrice; amúo.
\section{Burricego}
\begin{itemize}
\item {Grp. gram.:adj.}
\end{itemize}
\begin{itemize}
\item {Proveniência:(De \textunderscore burro\textunderscore ^1 + \textunderscore cego\textunderscore )}
\end{itemize}
Diz-se do toiro que vê pouco.
\section{Burricídio}
\begin{itemize}
\item {Grp. gram.:m.}
\end{itemize}
Morte violenta de um burro. Cf. Arn. Gama, \textunderscore Última Dona\textunderscore , 22.
\section{Burrico}
\begin{itemize}
\item {Grp. gram.:m.}
\end{itemize}
\begin{itemize}
\item {Proveniência:(De \textunderscore burro\textunderscore ^1)}
\end{itemize}
Pequeno burro.
\section{Burrié}
\begin{itemize}
\item {Grp. gram.:m.}
\end{itemize}
Mollúsco gasterópode, de concha univalve.
Caramujo.
Burgau.
\section{Burriélia}
\begin{itemize}
\item {Grp. gram.:f.}
\end{itemize}
\begin{itemize}
\item {Proveniência:(De \textunderscore Burriel\textunderscore , n. p.)}
\end{itemize}
Gênero de plantas compostas.
\section{Burrificar}
\begin{itemize}
\item {Grp. gram.:v. t.}
\end{itemize}
O mesmo que \textunderscore bestificar\textunderscore . Cf. Camillo, \textunderscore Noites de Insóm.\textunderscore , II, 32.
\section{Burrinho}
\begin{itemize}
\item {Grp. gram.:m.}
\end{itemize}
\begin{itemize}
\item {Utilização:Prov.}
\end{itemize}
\begin{itemize}
\item {Utilização:Bras}
\end{itemize}
Frigideira de barro, com cabo.
Espécie de bomba, para puxar líquidos.
\section{Burriqueiro}
\begin{itemize}
\item {Grp. gram.:m.}
\end{itemize}
\begin{itemize}
\item {Proveniência:(De \textunderscore burrico\textunderscore )}
\end{itemize}
Alugador, guia de burros.
\section{Burriquete}
\begin{itemize}
\item {fónica:quê}
\end{itemize}
\begin{itemize}
\item {Grp. gram.:m.}
\end{itemize}
\begin{itemize}
\item {Utilização:Bras}
\end{itemize}
Vela triangular, que se iça á popa das garoupeiras e bângulas.
\section{Burriquice}
\begin{itemize}
\item {Grp. gram.:f.}
\end{itemize}
O mesmo que \textunderscore burrice\textunderscore .
\section{Burríssimo}
\begin{itemize}
\item {Grp. gram.:adj.}
\end{itemize}
Muito burro, muito tolo ou estúpido. Cf. Cortesão, \textunderscore Subs.\textunderscore 
\section{Burro}
\begin{itemize}
\item {Grp. gram.:m.}
\end{itemize}
\begin{itemize}
\item {Utilização:Fig.}
\end{itemize}
\begin{itemize}
\item {Utilização:Náut.}
\end{itemize}
\begin{itemize}
\item {Utilização:Prov.}
\end{itemize}
\begin{itemize}
\item {Utilização:Prov.}
\end{itemize}
\begin{itemize}
\item {Utilização:Prov.}
\end{itemize}
\begin{itemize}
\item {Utilização:dur.}
\end{itemize}
\begin{itemize}
\item {Utilização:Prov.}
\end{itemize}
\begin{itemize}
\item {Utilização:trasm.}
\end{itemize}
\begin{itemize}
\item {Utilização:beir.}
\end{itemize}
\begin{itemize}
\item {Utilização:Náut.}
\end{itemize}
\begin{itemize}
\item {Utilização:Prov.}
\end{itemize}
\begin{itemize}
\item {Utilização:trasm.}
\end{itemize}
\begin{itemize}
\item {Utilização:Prov.}
\end{itemize}
\begin{itemize}
\item {Grp. gram.:Adj.}
\end{itemize}
\begin{itemize}
\item {Utilização:T. de Lisbôa}
\end{itemize}
\begin{itemize}
\item {Utilização:Prov.}
\end{itemize}
\begin{itemize}
\item {Proveniência:(Lat. \textunderscore burrus\textunderscore , ruívo)}
\end{itemize}
Quadrúpede, do mesmo gênero que o cavallo, mas menos corpulento, e com as orelhas mais compridas e crina curta.
Indivíduo estúpido, ou teimoso.
Burrice.
Pontalete, que mantém a posição horizontal de um carro.
Triângulo de pau, ou cavallete, em que se prende a madeira que há de serrar-se.
Versão literal de autor clássico, para uso de estudantes.
Jogo de cartas para crianças.
Cabo da vêrga de mezena.
Divisória das leiras de um batalhão. (Colhido em Turquel)
Instrumento de corticeiro, com que se aparam as arestas dos quadrados de cortiça.
Engenho, para tirar água de poços ou rios, por meio de balde.
Vergôntea delgada de videira, saída de um tronco que se cortou rente da terra, e coberta de terra depois, deixando-se-lhe apenas de fóra um ou dois olhos, para criar raizes e transformar-se em \textunderscore barbado\textunderscore .
Trasfogueiro, pequeno tronco, a que se encostam os cavacos, que ardem na lareira.
Pequeno motor auxiliar.
Espécie de banco de cardador.
Banco rústico, de pernadas de azinheiro, junto da chaminé.
Asnático, estúpido.
Diz-se de uma espécie de tijolo, o mesmo que \textunderscore lambaz\textunderscore .
Diz-se de uma espécie de pero esbranquiçado.
\section{Burro}
\begin{itemize}
\item {Grp. gram.:m.}
\end{itemize}
Nome que, em alguns pontos do país, se dá a um grande crustáceo de fórma de caranguejo.
\section{Burro}
\begin{itemize}
\item {Grp. gram.:adj.}
\end{itemize}
\begin{itemize}
\item {Utilização:Prov.}
\end{itemize}
\begin{itemize}
\item {Utilização:alg.}
\end{itemize}
Diz-se de uma espécie de milho amarelo, muito desenvolvido em fôlha e grão.
(Cp. \textunderscore zaburro\textunderscore )
\section{Birolina}
\begin{itemize}
\item {Grp. gram.:f.}
\end{itemize}
Cosmético, formado de ácido bórico, glicerina, etc.
\section{Bironiano}
\begin{itemize}
\item {fónica:bai}
\end{itemize}
\begin{itemize}
\item {Grp. gram.:adj.}
\end{itemize}
Relativo a Biron.
Que procura imitar o gôsto ou a escola de Biron.
\section{Birónico}
\begin{itemize}
\item {fónica:bai}
\end{itemize}
\begin{itemize}
\item {Grp. gram.:adj.}
\end{itemize}
O mesmo que \textunderscore bironiano\textunderscore .
\section{Bissáceo}
\begin{itemize}
\item {Grp. gram.:adj.}
\end{itemize}
Relativo ou semelhante ao bisso.
\section{Bisso}
\begin{itemize}
\item {Grp. gram.:m.}
\end{itemize}
\begin{itemize}
\item {Proveniência:(Gr. \textunderscore bussos\textunderscore )}
\end{itemize}
Planta cryptogâmica, esbranquiçada, da fam. dos musgos.
Filamentos, que sáem de algumas conchas bivalves.
Espécie de linho amarelado, com que os antigos fabricavam os estofos mais ricos.
\section{Bitnéria}
\begin{itemize}
\item {Grp. gram.:f.}
\end{itemize}
\begin{itemize}
\item {Proveniência:(De \textunderscore Byttner\textunderscore , n. p.)}
\end{itemize}
Gênero de plantas da Ásia e da América.
\section{Bitneriáceas}
\begin{itemize}
\item {Grp. gram.:f. Pl.}
\end{itemize}
Familia de plantas, que tem por typo a bitnéria.
\section{Bizantina}
\begin{itemize}
\item {Grp. gram.:f.}
\end{itemize}
Anêmona, côr de rosa.
(Cp. \textunderscore byzantino\textunderscore )
\section{Bizantinizar}
\begin{itemize}
\item {Grp. gram.:v. t.}
\end{itemize}
\begin{itemize}
\item {Utilização:Neol.}
\end{itemize}
Tornar bizantino, fútil.
\section{Bizantino}
\begin{itemize}
\item {Grp. gram.:adj.}
\end{itemize}
\begin{itemize}
\item {Grp. gram.:M.}
\end{itemize}
\begin{itemize}
\item {Proveniência:(Lat. \textunderscore byzantinus\textunderscore )}
\end{itemize}
Relativo a Bizâncio ou ao Baixo-Imperio.
Súbtil e fútil, como as questões theológicas da corte de Bizâncio.
Estilo ou arte, que se cultivou no Baixo-Império.
Habitante de Bizâncio.
\section{Burro-alto}
\begin{itemize}
\item {Grp. gram.:m.}
\end{itemize}
\begin{itemize}
\item {Utilização:T. da Bairrada}
\end{itemize}
Jôgo de rapazes, em que um, apoiando a cabeça contra uma parede, aguenta os que lhe sobem para cima, até que algum, descambando, vai substituir aquelle.
\section{Burruíço}
\begin{itemize}
\item {Grp. gram.:m.}
\end{itemize}
\begin{itemize}
\item {Utilização:Prov.}
\end{itemize}
\begin{itemize}
\item {Utilização:minh.}
\end{itemize}
\begin{itemize}
\item {Proveniência:(De \textunderscore burro\textunderscore ^2)}
\end{itemize}
Ajuntamento de mariscos na linha da costa marítima.
\section{Bursária}
\begin{itemize}
\item {Grp. gram.:f.}
\end{itemize}
\begin{itemize}
\item {Proveniência:(Do lat. \textunderscore bursa\textunderscore , bôlsa)}
\end{itemize}
Gênero de plantas pittospóreas.
\section{Búrsera}
\begin{itemize}
\item {Grp. gram.:f.}
\end{itemize}
\begin{itemize}
\item {Proveniência:(De \textunderscore Burser\textunderscore , n. p.)}
\end{itemize}
Planta das Antilhas, que serve de typo ás burseráceas.
\section{Burseráceas}
\begin{itemize}
\item {Grp. gram.:f. pl.}
\end{itemize}
\begin{itemize}
\item {Proveniência:(De \textunderscore búrsera\textunderscore )}
\end{itemize}
Família de plantas dicotyledóneas.
\section{Burserina}
\begin{itemize}
\item {Grp. gram.:f.}
\end{itemize}
Resina branca, que se extrai do bálsamo da búrsera.
\section{Buruaca}
\begin{itemize}
\item {Grp. gram.:f.}
\end{itemize}
\begin{itemize}
\item {Utilização:Bras}
\end{itemize}
O mesmo que \textunderscore bruaca\textunderscore .
\section{Burundagem}
\begin{itemize}
\item {Grp. gram.:f.}
\end{itemize}
\begin{itemize}
\item {Utilização:Ant.}
\end{itemize}
O mesmo que \textunderscore burundanga\textunderscore ? Cf. \textunderscore Anat. Joc.\textunderscore , II, 446.
\section{Burundanga}
\begin{itemize}
\item {Grp. gram.:f.}
\end{itemize}
\begin{itemize}
\item {Grp. gram.:Pl.}
\end{itemize}
Algaravia; palavreado confuso.
Mixórdia.
Cozinhado mal feito ou pouco limpo e repugnante.
Ninharias.
(Cp. cast. \textunderscore morondanga\textunderscore )
\section{Bururé}
\begin{itemize}
\item {Grp. gram.:m.}
\end{itemize}
\begin{itemize}
\item {Utilização:Bras}
\end{itemize}
Planta, o mesmo que \textunderscore mururé\textunderscore .
\section{Burururus}
\begin{itemize}
\item {Grp. gram.:m. pl.}
\end{itemize}
Indígenas da Guiana brasileira.
\section{Buruso}
\begin{itemize}
\item {Grp. gram.:m.}
\end{itemize}
Bagaço, resíduo de frutos, depois de espremidos.
(Cast. \textunderscore burujo\textunderscore )
\section{Burzalaque}
\begin{itemize}
\item {Grp. gram.:m.}
\end{itemize}
\begin{itemize}
\item {Utilização:Prov.}
\end{itemize}
\begin{itemize}
\item {Utilização:trasm.}
\end{itemize}
O mesmo que \textunderscore bazulaque\textunderscore .
\section{Burzigada}
\begin{itemize}
\item {Grp. gram.:f.}
\end{itemize}
\begin{itemize}
\item {Utilização:Prov.}
\end{itemize}
Sarapatel.
Fartadela com miúdos de porco.
Acervo de coisas amassadas, espapadas.
\section{Burziguiada}
\begin{itemize}
\item {Grp. gram.:f.}
\end{itemize}
O mesmo que \textunderscore burzigada\textunderscore . Cf. Camillo, \textunderscore Mosaico\textunderscore , 160.
\section{Bus}
\begin{itemize}
\item {Grp. gram.:m. pl.}
\end{itemize}
\begin{itemize}
\item {Utilização:Bras}
\end{itemize}
Indígenas, que habitavam no Maranhão.
\section{Busano}
\begin{itemize}
\item {Grp. gram.:m.}
\end{itemize}
\begin{itemize}
\item {Utilização:Prov.}
\end{itemize}
\begin{itemize}
\item {Utilização:alent.}
\end{itemize}
Verme, que se cria no ânus de certos animaes.
O mesmo que \textunderscore gusano\textunderscore .
\section{Búsara}
\textunderscore f.\textunderscore  (e der.)
(V. \textunderscore búzara\textunderscore , etc.)
\section{Busardo}
\begin{itemize}
\item {Grp. gram.:m.}
\end{itemize}
\begin{itemize}
\item {Proveniência:(Fr. \textunderscore brusard\textunderscore )}
\end{itemize}
Ave de rapina, da fam. dos falcões.
\section{Busca}
\begin{itemize}
\item {Grp. gram.:f.}
\end{itemize}
Acção de \textunderscore buscar\textunderscore ; pesquisa, investigação.
Cão ou pessôa, que busca e levanta a caça.
\section{Busca-caixas}
\begin{itemize}
\item {Grp. gram.:m.}
\end{itemize}
Espécie de antigo servente de alfândega. Cf. F. Borges, \textunderscore Diccion. Jur.\textunderscore 
\section{Buscado}
\begin{itemize}
\item {Grp. gram.:adj.}
\end{itemize}
\begin{itemize}
\item {Proveniência:(De \textunderscore buscar\textunderscore )}
\end{itemize}
Feito com cuidado.
Preparado com affectação; amaneirado.
\section{Buscador}
\begin{itemize}
\item {Grp. gram.:m.}
\end{itemize}
Aquelle que busca.
\section{Buscante}
\begin{itemize}
\item {Grp. gram.:adj.}
\end{itemize}
Que busca.
\section{Buscapé}
\begin{itemize}
\item {Grp. gram.:m.}
\end{itemize}
\begin{itemize}
\item {Proveniência:(De \textunderscore buscar\textunderscore  + \textunderscore pé\textunderscore )}
\end{itemize}
Peça de fogo de artifício, que gira e serpeia pelo chão, e que vulgarmente se chama \textunderscore bicha-de-rabear\textunderscore .
\section{Buscar}
\begin{itemize}
\item {Grp. gram.:v. t.}
\end{itemize}
\begin{itemize}
\item {Proveniência:(De \textunderscore bosque\textunderscore , segundo Diez, o que daria \textunderscore boscar\textunderscore )}
\end{itemize}
Tratar de descobrir.
Procurar.
Examinar, investigar.
Ir a um lugar e trazer de lá (alguma coisa): \textunderscore vai buscar o meu chapéu\textunderscore .
Tratar de obter.
Recorrer a.
Dirigir-se para.
\section{Busca-três}
\begin{itemize}
\item {Grp. gram.:m.}
\end{itemize}
Espécie de jôgo popular.
\section{Busca-vida}
\begin{itemize}
\item {Grp. gram.:m.}
\end{itemize}
\begin{itemize}
\item {Utilização:Náut.}
\end{itemize}
Pessôa activa; fura-vidas.
Fateixa, sem unhas para buscar os ferros, quando rebenta a amarra.
(Vocábulo mal formado, no sentido náutico, sobre o fr. \textunderscore cherche-vides\textunderscore )
\section{Busca-vida}
\begin{itemize}
\item {Grp. gram.:m.}
\end{itemize}
Instrumento de ferro, para abrir o ouvido da peça de artilharia, antes de a escorvar.
(Cp. \textunderscore busca-vida\textunderscore ^1, no sentido náutico)
\section{Buscavidas}
\begin{itemize}
\item {Grp. gram.:m.}
\end{itemize}
Instrumento de ferro, para abrir o ouvido da peça de artilharia, antes de a escorvar.
(Cp. \textunderscore busca-vida\textunderscore ^1, no sentido náutico)
\section{Buseira}
\begin{itemize}
\item {Grp. gram.:f.}
\end{itemize}
\begin{itemize}
\item {Utilização:Prov.}
\end{itemize}
\begin{itemize}
\item {Utilização:trasm.}
\end{itemize}
Excremento molle de gallinhas ou de outras aves grandes.
(Cp. \textunderscore buseiro\textunderscore )
\section{Buseirada}
\begin{itemize}
\item {Grp. gram.:f.}
\end{itemize}
\begin{itemize}
\item {Utilização:Prov.}
\end{itemize}
\begin{itemize}
\item {Utilização:trasm.}
\end{itemize}
Grande porção de buseira.
\section{Buseiro}
\begin{itemize}
\item {Grp. gram.:m.}
\end{itemize}
\begin{itemize}
\item {Utilização:Pleb.}
\end{itemize}
Acervo de excrementos.
\section{Busil}
\begin{itemize}
\item {Grp. gram.:m.}
\end{itemize}
\begin{itemize}
\item {Utilização:Prov.}
\end{itemize}
\begin{itemize}
\item {Utilização:trasm.}
\end{itemize}
Criança muito comilona.
\section{Busilhão}
\begin{itemize}
\item {Grp. gram.:m.}
\end{itemize}
\begin{itemize}
\item {Utilização:Gír.}
\end{itemize}
Muito dinheiro; thesoiro.
\section{Busílis}
\begin{itemize}
\item {Grp. gram.:m.}
\end{itemize}
A principal difficuldade em resolver uma coisa.
(Das três últimas sýllabas da loc. lat. \textunderscore in diebus illis\textunderscore , segundo a tradição escolar)
\section{Busmelé}
\begin{itemize}
\item {Grp. gram.:adj.}
\end{itemize}
\begin{itemize}
\item {Utilização:Ant.}
\end{itemize}
\begin{itemize}
\item {Grp. gram.:Interj.}
\end{itemize}
Calado.
Silêncio! Cf. J. Ribeiro, \textunderscore Frases Feitas\textunderscore , II, 581.
\section{Bussíl}
\begin{itemize}
\item {Grp. gram.:m.}
\end{itemize}
\begin{itemize}
\item {Utilização:Prov.}
\end{itemize}
\begin{itemize}
\item {Utilização:alent.}
\end{itemize}
Peça de ferro, que forra interiormente o núcleo central das rodas dos carros, para deminuir o attrito do eixo.
\section{Bússola}
\begin{itemize}
\item {Grp. gram.:f.}
\end{itemize}
\begin{itemize}
\item {Utilização:Fig.}
\end{itemize}
\begin{itemize}
\item {Proveniência:(It. \textunderscore bossolo\textunderscore )}
\end{itemize}
Caixa, com uma abertura circular, dentro da qual se move uma agulha magnética, collocada horizontalmente na extremidade superior de uma pequena haste vertical, e cuja ponta, oscillando, procura e determina o lado do Norte.
Constellação austral.
Aquillo que serve de guia ou que norteia, em assumpto embaraçado.
\section{Bussolar}
\begin{itemize}
\item {Grp. gram.:v. i.}
\end{itemize}
\begin{itemize}
\item {Utilização:Fig.}
\end{itemize}
\begin{itemize}
\item {Proveniência:(De \textunderscore bússola\textunderscore )}
\end{itemize}
Guiar, nortear.
\section{Bussolco}
\begin{itemize}
\item {Grp. gram.:m.}
\end{itemize}
Instrumento de topographia.
\section{Busto}
\begin{itemize}
\item {Grp. gram.:m.}
\end{itemize}
\begin{itemize}
\item {Proveniência:(Do it. \textunderscore fusto\textunderscore , segundo Diez)}
\end{itemize}
Representação esculptural ou pictórica de uma cabeça humana, com o pescoço e parte do peito.
A parte do corpo humano da cintura para cima.
\section{Busto}
\begin{itemize}
\item {Grp. gram.:m.}
\end{itemize}
\begin{itemize}
\item {Utilização:Ant.}
\end{itemize}
Curral de bois.
Bostal.
Campo cerrado para pastagem.
(Cp. \textunderscore bosteira\textunderscore )
\section{Bustrofédon}
\begin{itemize}
\item {Grp. gram.:adj.}
\end{itemize}
\begin{itemize}
\item {Proveniência:(Gr. \textunderscore boustrophedon\textunderscore )}
\end{itemize}
Dizia-se de um modo de escrever, em que a primeira linha, em vez de terminar na borda do papel, dá uma volta semi-circular, para continuar por baixo, da direita para a esquerda, tornando depois a baixar e a voltar da esquerda para a direita, e assim por deante.
\section{Bustrophédon}
\begin{itemize}
\item {Grp. gram.:adj.}
\end{itemize}
\begin{itemize}
\item {Proveniência:(Gr. \textunderscore boustrophedon\textunderscore )}
\end{itemize}
Dizia-se de um modo de escrever, em que a primeira linha, em vez de terminar na borda do papel, dá uma volta semi-circular, para continuar por baixo, da direita para a esquerda, tornando depois a baixar e a voltar da esquerda para a direita, e assim por deante.
\section{Bustuário}
\begin{itemize}
\item {Grp. gram.:m.}
\end{itemize}
\begin{itemize}
\item {Utilização:Ant.}
\end{itemize}
\begin{itemize}
\item {Proveniência:(Lat. \textunderscore bustuarius\textunderscore )}
\end{itemize}
Gladiador, que combatia em volta das fogueiras, em que se queimavam os mortos.
\section{Bustuário}
\begin{itemize}
\item {Grp. gram.:m.}
\end{itemize}
\begin{itemize}
\item {Proveniência:(De \textunderscore busto\textunderscore )}
\end{itemize}
Artista que faz bustos.
\section{Buta}
\begin{itemize}
\item {Grp. gram.:f.}
\end{itemize}
Planta trepadeira da ilha de San-Thomé.
Serpente de Angola.
\section{Butaca}
\begin{itemize}
\item {Grp. gram.:f.}
\end{itemize}
\begin{itemize}
\item {Proveniência:(T. cast.)}
\end{itemize}
Cadeira, que serve de throno, entre os negros de Angola.
\section{Butan}
\begin{itemize}
\item {Grp. gram.:m.}
\end{itemize}
Espécie de ligadura, em apparelhos náuticos.
(Por \textunderscore botan\textunderscore , de \textunderscore botão\textunderscore ?)
\section{Butano}
\begin{itemize}
\item {Grp. gram.:m.}
\end{itemize}
\begin{itemize}
\item {Utilização:Chím.}
\end{itemize}
Variedade de carboneto do grupo formênico.
\section{Butara}
\begin{itemize}
\item {Grp. gram.:f.}
\end{itemize}
\begin{itemize}
\item {Utilização:Bras}
\end{itemize}
Armadilha para animaes bravios.
\section{Butargas}
\begin{itemize}
\item {Grp. gram.:f. pl.}
\end{itemize}
\begin{itemize}
\item {Proveniência:(It. \textunderscore bultagra\textunderscore )}
\end{itemize}
Ovas de peixe, de conserva.
\section{Bute}
\begin{itemize}
\item {Grp. gram.:m.}
\end{itemize}
\begin{itemize}
\item {Utilização:Prov.}
\end{itemize}
\begin{itemize}
\item {Utilização:beir.}
\end{itemize}
\begin{itemize}
\item {Proveniência:(Ingl. \textunderscore boot\textunderscore )}
\end{itemize}
O mesmo que \textunderscore botim\textunderscore .
Antigo calçado de munição dos soldados.
\section{Buteiro}
\begin{itemize}
\item {Grp. gram.:m.}
\end{itemize}
\begin{itemize}
\item {Utilização:Bras}
\end{itemize}
\begin{itemize}
\item {Proveniência:(De \textunderscore bute\textunderscore )}
\end{itemize}
Remendão.
Alfaiate de concertos.
\section{Butélo}
\begin{itemize}
\item {Grp. gram.:m.}
\end{itemize}
\begin{itemize}
\item {Utilização:Bras. de Goiás}
\end{itemize}
Homem de grande estatura.
Objecto grande.
\section{Butêlo}
\begin{itemize}
\item {Grp. gram.:m.}
\end{itemize}
\begin{itemize}
\item {Utilização:Prov.}
\end{itemize}
\begin{itemize}
\item {Utilização:trasm.}
\end{itemize}
Chouriço grosso, que leva osso moído de suan.
\section{Búteo}
\begin{itemize}
\item {Grp. gram.:m.}
\end{itemize}
\begin{itemize}
\item {Utilização:Fig.}
\end{itemize}
\begin{itemize}
\item {Proveniência:(Lat. \textunderscore buteo\textunderscore )}
\end{itemize}
Espécie de falcão.
Tubo, que communica o ar aos folles, nas minas.
Homem preguiçoso.
\section{Butes}
\begin{itemize}
\item {Grp. gram.:m. pl.}
\end{itemize}
\begin{itemize}
\item {Utilização:Gír. Lisb.}
\end{itemize}
Pés.
\section{Butiá}
\begin{itemize}
\item {Grp. gram.:m.}
\end{itemize}
\begin{itemize}
\item {Utilização:Bras}
\end{itemize}
\begin{itemize}
\item {Proveniência:(T. tupi?)}
\end{itemize}
Espécie de palmeira.
\section{Butigas}
\begin{itemize}
\item {Grp. gram.:f. pl.}
\end{itemize}
\begin{itemize}
\item {Utilização:Prov.}
\end{itemize}
\begin{itemize}
\item {Utilização:minh.}
\end{itemize}
Espécie de planta, (\textunderscore cytinus hypocistis\textunderscore , Lin.).
\section{Butileno}
\begin{itemize}
\item {Grp. gram.:m.}
\end{itemize}
\begin{itemize}
\item {Utilização:Chím.}
\end{itemize}
Variedade de carboneto do grupo ethylênico.
\section{Butílico}
\begin{itemize}
\item {Grp. gram.:adj.}
\end{itemize}
Diz-se de um dos álcooes dos vinhos, cuja fórmula chimica é C^{4}H^{10}O.
\section{Bútio}
\begin{itemize}
\item {Grp. gram.:m.}
\end{itemize}
\begin{itemize}
\item {Utilização:Fig.}
\end{itemize}
\begin{itemize}
\item {Proveniência:(Lat. \textunderscore buteo\textunderscore )}
\end{itemize}
Espécie de falcão.
Tubo, que communica o ar aos folles, nas minas.
Homem preguiçoso.
\section{Butir}
\begin{itemize}
\item {Grp. gram.:m.}
\end{itemize}
\begin{itemize}
\item {Utilização:Ant.}
\end{itemize}
Espécie de jôgo prohibido, em que se atirava ao alvo.
(Cp. fr. \textunderscore aboutir\textunderscore )
\section{Butiráceo}
\begin{itemize}
\item {Grp. gram.:adj.}
\end{itemize}
\begin{itemize}
\item {Proveniência:(Do gr. \textunderscore bouturon\textunderscore )}
\end{itemize}
Relativo á manteiga.
Que tem propriedades da manteiga.
\section{Butirada}
\begin{itemize}
\item {Grp. gram.:f.}
\end{itemize}
\begin{itemize}
\item {Proveniência:(Do gr. \textunderscore bouturon\textunderscore )}
\end{itemize}
Bolo de manteiga.
\section{Butírico}
\begin{itemize}
\item {Grp. gram.:adj.}
\end{itemize}
\begin{itemize}
\item {Proveniência:(Do gr. \textunderscore bouturon\textunderscore )}
\end{itemize}
Diz-se de um ácido, que se encontra no vinho novo e lhe dá sabor untuoso e o aroma do queijo. Cf. \textunderscore Techn. Rur.\textunderscore , I, 26.
\section{Butiróleo}
\begin{itemize}
\item {Grp. gram.:m.}
\end{itemize}
\begin{itemize}
\item {Utilização:Pharm.}
\end{itemize}
\begin{itemize}
\item {Proveniência:(Do gr. \textunderscore bouturon\textunderscore )}
\end{itemize}
Medicamento, que tem por base a manteiga.
\section{Butiroso}
\begin{itemize}
\item {Grp. gram.:adj.}
\end{itemize}
O mesmo que \textunderscore butyráceo\textunderscore .
\section{Butneriáceas}
\begin{itemize}
\item {Grp. gram.:f. pl.}
\end{itemize}
\begin{itemize}
\item {Proveniência:(De \textunderscore Buttner\textunderscore , n. p.)}
\end{itemize}
Fam. de plantas, de sementes oleosas, e a que pertence o cacau.
\section{Butomáceas}
\begin{itemize}
\item {Grp. gram.:f. pl.}
\end{itemize}
\begin{itemize}
\item {Proveniência:(De \textunderscore bútomo\textunderscore )}
\end{itemize}
Plantas paludosas e medicinaes.
\section{Bútomo}
\begin{itemize}
\item {Grp. gram.:m.}
\end{itemize}
\begin{itemize}
\item {Proveniência:(Do gr. \textunderscore bous\textunderscore  + \textunderscore tome\textunderscore )}
\end{itemize}
Gênero de plantas, que serve de typo ás butomáceas.
\section{Butre}
\begin{itemize}
\item {Grp. gram.:m.}
\end{itemize}
(V.abutre)
\section{Buttneriáceas}
\begin{itemize}
\item {Grp. gram.:f. pl.}
\end{itemize}
\begin{itemize}
\item {Proveniência:(De \textunderscore Buttner\textunderscore , n. p.)}
\end{itemize}
Fam. de plantas, de sementes oleosas, e a que pertence o cacau.
\section{Bútua}
\begin{itemize}
\item {Grp. gram.:f.}
\end{itemize}
Designação de várias plantas menispermáceas.
Planta trepadeira e medicinal de Angola, (\textunderscore tiliacora chrysobotrya\textunderscore , Welw.).
\section{Butucaris}
\begin{itemize}
\item {Grp. gram.:m. pl.}
\end{itemize}
Numerosa tríbo de indígenas do Brasil, a oéste das nascentes do Jacuí.
\section{Butyleno}
\begin{itemize}
\item {Grp. gram.:m.}
\end{itemize}
\begin{itemize}
\item {Utilização:Chím.}
\end{itemize}
Variedade de carboneto do grupo ethylênico.
\section{Butýlico}
\begin{itemize}
\item {Grp. gram.:adj.}
\end{itemize}
Diz-se de um dos álcooes dos vinhos, cuja fórmula chimica é C^{4}H^{10}O.
\section{Butyráceo}
\begin{itemize}
\item {Grp. gram.:adj.}
\end{itemize}
\begin{itemize}
\item {Proveniência:(Do gr. \textunderscore bouturon\textunderscore )}
\end{itemize}
Relativo á manteiga.
Que tem propriedades da manteiga.
\section{Butyrada}
\begin{itemize}
\item {Grp. gram.:f.}
\end{itemize}
\begin{itemize}
\item {Proveniência:(Do gr. \textunderscore bouturon\textunderscore )}
\end{itemize}
Bolo de manteiga.
\section{Butýrico}
\begin{itemize}
\item {Grp. gram.:adj.}
\end{itemize}
\begin{itemize}
\item {Proveniência:(Do gr. \textunderscore bouturon\textunderscore )}
\end{itemize}
Diz-se de um ácido, que se encontra no vinho novo e lhe dá sabor untuoso e o aroma do queijo. Cf. \textunderscore Techn. Rur.\textunderscore , I, 26.
\section{Butyróleo}
\begin{itemize}
\item {Grp. gram.:m.}
\end{itemize}
\begin{itemize}
\item {Utilização:Pharm.}
\end{itemize}
\begin{itemize}
\item {Proveniência:(Do gr. \textunderscore bouturon\textunderscore )}
\end{itemize}
Medicamento, que tem por base a manteiga.
\section{Butyroso}
\begin{itemize}
\item {Grp. gram.:adj.}
\end{itemize}
O mesmo que \textunderscore butyráceo\textunderscore .
\section{Buvar}
\begin{itemize}
\item {Grp. gram.:m.}
\end{itemize}
\begin{itemize}
\item {Utilização:bras}
\end{itemize}
\begin{itemize}
\item {Utilização:Neol.}
\end{itemize}
\begin{itemize}
\item {Proveniência:(Fr. \textunderscore buvard\textunderscore )}
\end{itemize}
Peça de fôlha ou madeira, em que se colloca o mata-borrão, que vai sêr utilizado.
\section{Buxáceas}
\begin{itemize}
\item {Grp. gram.:f. pl.}
\end{itemize}
Família de plantas, que tem por typo o buxo.
\section{Buxal}
\begin{itemize}
\item {Grp. gram.:m.}
\end{itemize}
Moita de buxo.
\section{Buxeira}
\begin{itemize}
\item {Grp. gram.:f.}
\end{itemize}
\begin{itemize}
\item {Utilização:Bras}
\end{itemize}
Árvore rubiácea, de fibras têxteis.
\section{Buxina}
\begin{itemize}
\item {Grp. gram.:f.}
\end{itemize}
Substância, extrahida da casca da raíz do buxo.
\section{Buxinha}
\begin{itemize}
\item {Grp. gram.:f.}
\end{itemize}
\begin{itemize}
\item {Utilização:Bras}
\end{itemize}
Planta cucurbitácea e medicinal.
\section{Buxo}
\begin{itemize}
\item {Grp. gram.:m.}
\end{itemize}
\begin{itemize}
\item {Proveniência:(Lat. \textunderscore buxus\textunderscore )}
\end{itemize}
Arbusto, semelhante á murta, e de que há duas variedades, sendo uma arborescente e empregada em certas obras, e outra rasteira, applicada a guarnições de jardins.
\section{Buxo-anão}
\begin{itemize}
\item {Grp. gram.:m.}
\end{itemize}
O mesmo que \textunderscore murta\textunderscore .
\section{Buz!}
\begin{itemize}
\item {Grp. gram.:interj.}
\end{itemize}
\begin{itemize}
\item {Grp. gram.:M.}
\end{itemize}
\begin{itemize}
\item {Utilização:Ant.}
\end{itemize}
\begin{itemize}
\item {Grp. gram.:Loc. adv.}
\end{itemize}
Silêncio!
Ósculo na mão, em sinal de reverência.
Estrondo de armas de fogo.
Ruído de aves de rapina.
\textunderscore Nem tuz nem buz\textunderscore , ou \textunderscore nem chuz nem buz\textunderscore , sem dizer uma palavra.
\section{Buza}
\begin{itemize}
\item {Grp. gram.:f.}
\end{itemize}
Bebida fermentada, usada pelos Egýpcios.
\section{Búzara}
\begin{itemize}
\item {Grp. gram.:f.}
\end{itemize}
\begin{itemize}
\item {Utilização:Prov.}
\end{itemize}
\begin{itemize}
\item {Grp. gram.:M.}
\end{itemize}
\begin{itemize}
\item {Utilização:Prov.}
\end{itemize}
Barriga; pança.
Grande comilão.
\section{Buzaranha}
\begin{itemize}
\item {Grp. gram.:f.}
\end{itemize}
\begin{itemize}
\item {Utilização:Prov.}
\end{itemize}
Grande ventanía.
\section{Buzaranho}
\begin{itemize}
\item {Grp. gram.:m.}
\end{itemize}
\begin{itemize}
\item {Utilização:Prov.}
\end{itemize}
\begin{itemize}
\item {Grp. gram.:Pl.}
\end{itemize}
\begin{itemize}
\item {Utilização:Prov.}
\end{itemize}
\begin{itemize}
\item {Utilização:trasm.}
\end{itemize}
\begin{itemize}
\item {Utilização:Ant.}
\end{itemize}
O mesmo que \textunderscore buzaranha\textunderscore .
Indivíduo corpulento.
Coisas fantásticas, que se apresentam aos olhos de quem tem muita febre. Cf. G. Vicente, I, 261.
\section{Buzarate}
\begin{itemize}
\item {Grp. gram.:adj.}
\end{itemize}
\begin{itemize}
\item {Utilização:pop.}
\end{itemize}
\begin{itemize}
\item {Utilização:Ant.}
\end{itemize}
\begin{itemize}
\item {Proveniência:(De \textunderscore búzara\textunderscore )}
\end{itemize}
Fátuo, fanfarrão.
Pessôa corpulenta, barriguda.
Bazulaque.
\section{Buzarato}
\begin{itemize}
\item {Grp. gram.:m.}
\end{itemize}
\begin{itemize}
\item {Utilização:Des.}
\end{itemize}
Coveiro dos pobres.
\section{Buzeno}
\begin{itemize}
\item {Grp. gram.:m.}
\end{itemize}
\begin{itemize}
\item {Proveniência:(De \textunderscore búzio\textunderscore )}
\end{itemize}
Antiga medida portuguesa, equivalente a quatro alqueires.
\section{Búzera}
\begin{itemize}
\item {Grp. gram.:f.}
\end{itemize}
\begin{itemize}
\item {Utilização:Prov.}
\end{itemize}
\begin{itemize}
\item {Grp. gram.:M.}
\end{itemize}
\begin{itemize}
\item {Utilização:Prov.}
\end{itemize}
Barriga; pança.
Grande comilão.
\section{Buzi}
\begin{itemize}
\item {Grp. gram.:m.}
\end{itemize}
O mesmo que \textunderscore quichobo\textunderscore .
\section{Buzia}
\begin{itemize}
\item {Grp. gram.:f.  e  adj.}
\end{itemize}
\begin{itemize}
\item {Utilização:Prov.}
\end{itemize}
Diz-se de uma vara comprida.
\section{Buziar}
\begin{itemize}
\item {Grp. gram.:v. i.}
\end{itemize}
\begin{itemize}
\item {Utilização:Prov.}
\end{itemize}
\begin{itemize}
\item {Utilização:trasm.}
\end{itemize}
Tocar búzio.
\section{Buzilhão}
\begin{itemize}
\item {Grp. gram.:m.}
\end{itemize}
\begin{itemize}
\item {Utilização:Prov.}
\end{itemize}
\begin{itemize}
\item {Utilização:dur.}
\end{itemize}
\begin{itemize}
\item {Utilização:T. de Barcelos}
\end{itemize}
Mealheiro; pé de meia, com dinheiro.
Inchaço; tumor.
\section{Buzina}
\begin{itemize}
\item {Grp. gram.:f.}
\end{itemize}
\begin{itemize}
\item {Utilização:Bras. do S}
\end{itemize}
\begin{itemize}
\item {Grp. gram.:Pl.}
\end{itemize}
\begin{itemize}
\item {Utilização:Náut.}
\end{itemize}
\begin{itemize}
\item {Proveniência:(Lat. \textunderscore buccina\textunderscore )}
\end{itemize}
Trombeta de corno ou metal retorcído.
Grande búzio, de que se tira um som, semelhante ao da buzina.
Porta-voz.
Designação vulgar da \textunderscore Ursa-Menor\textunderscore .
Buraco do centro da roda do carro, onde entra o eixo.
Aberturas, forradas de ferro, no painel da popa.
\section{Buzinar}
\begin{itemize}
\item {Grp. gram.:v. t.}
\end{itemize}
Tocar buzina.
Soprar fortemente, produzindo sons como os da buzina.
Falar com impertinência.
\section{Búzio}
\begin{itemize}
\item {Grp. gram.:m.}
\end{itemize}
\begin{itemize}
\item {Grp. gram.:Adj.}
\end{itemize}
\begin{itemize}
\item {Utilização:Prov.}
\end{itemize}
\begin{itemize}
\item {Proveniência:(Do lat. \textunderscore buccinum\textunderscore ?)}
\end{itemize}
Concha univalve, de fórma cónica ou espiral, pertencente a mollúsco gasterópode.
Buzina.
Trombeta.
Antiga medida, o mesmo que \textunderscore buzeno\textunderscore .
Antigo alqueire, na Galliza e no Minho.
Pouco transparente.
\section{Búzio}
\begin{itemize}
\item {Grp. gram.:m.}
\end{itemize}
Mergulhador, que debaixo de água apanha conchas, pérolas ou peixes, ou executa qualquer trabalho.
(Cast. \textunderscore buzo\textunderscore )
\section{Buzo}
\begin{itemize}
\item {Grp. gram.:m.}
\end{itemize}
\begin{itemize}
\item {Utilização:Bras}
\end{itemize}
Jôgo popular, com rodelas de casca de laranja.
\section{Byrolina}
\begin{itemize}
\item {Grp. gram.:f.}
\end{itemize}
Cosmético, formado de ácido bórico, glycerina, etc.
\section{Byroniano}
\begin{itemize}
\item {fónica:bai}
\end{itemize}
\begin{itemize}
\item {Grp. gram.:adj.}
\end{itemize}
Relativo a Byron.
Que procura imitar o gôsto ou a escola de Byron.
\section{Byrónico}
\begin{itemize}
\item {fónica:bai}
\end{itemize}
\begin{itemize}
\item {Grp. gram.:adj.}
\end{itemize}
O mesmo que \textunderscore byroniano\textunderscore .
\section{Byssáceo}
\begin{itemize}
\item {Grp. gram.:adj.}
\end{itemize}
Relativo ou semelhante ao bysso.
\section{Bysso}
\begin{itemize}
\item {Grp. gram.:m.}
\end{itemize}
\begin{itemize}
\item {Proveniência:(Gr. \textunderscore bussos\textunderscore )}
\end{itemize}
Planta cryptogâmica, esbranquiçada, da fam. dos musgos.
Filamentos, que sáem de algumas conchas bivalves.
Espécie de linho amarelado, com que os antigos fabricavam os estofos mais ricos.
\section{Byttnéria}
\begin{itemize}
\item {Grp. gram.:f.}
\end{itemize}
\begin{itemize}
\item {Proveniência:(De \textunderscore Byttner\textunderscore , n. p.)}
\end{itemize}
Gênero de plantas da Ásia e da América.
\section{Byttneriáceas}
\begin{itemize}
\item {Grp. gram.:f. Pl.}
\end{itemize}
Familia de plantas, que tem por typo a byttnéria.
\section{Byzâncio}
\begin{itemize}
\item {Grp. gram.:m.}
\end{itemize}
\begin{itemize}
\item {Utilização:Ant.}
\end{itemize}
\begin{itemize}
\item {Proveniência:(De \textunderscore Byzâncio\textunderscore , n. p.)}
\end{itemize}
Moéda de oiro, procedente do Império Romano do Oriente e que teve curso na Península hispânica, no século XI a XIII, pelo menos.
\section{Byzantina}
\begin{itemize}
\item {Grp. gram.:f.}
\end{itemize}
Anêmona, côr de rosa.
(Cp. \textunderscore byzantino\textunderscore )
\section{Byzantinizar}
\begin{itemize}
\item {Grp. gram.:v. t.}
\end{itemize}
\begin{itemize}
\item {Utilização:Neol.}
\end{itemize}
Tornar byzantino, fútil.
\section{Byzantino}
\begin{itemize}
\item {Grp. gram.:adj.}
\end{itemize}
\begin{itemize}
\item {Grp. gram.:M.}
\end{itemize}
\begin{itemize}
\item {Proveniência:(Lat. \textunderscore byzantinus\textunderscore )}
\end{itemize}
\end{document}